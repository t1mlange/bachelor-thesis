\documentclass[../draft.tex]{subfiles}

\begin{document}
    \chapter{Implementation}

    \section{Integration}
    Document changes needed in non problems/ to fit backwards.
    \begin{itemize}
        \item BackwardsSourceSinkManager
        \item BackwardsInfoflowResults
    \end{itemize}
    \section{InfoflowProblem}
    \section{Rules}
    \subsection{Backwards Source Propagation Rule}\label{s:sourcerule}
    \subsection{Backwards Clinit Rule}
    \code{<clinit>} is a special method in the JVM and stands for class loader init. The function is generated by the compiler and it can not be called explicitly. Examples of statements which get compiled into clinit can be seen in \autoref{lst:clinit_examples}. The invokation is implicit at the initialization phase of the class and is executed at most once for each class \footnote{\url{https://docs.oracle.com/javase/specs/jvms/se8/html/jvms-2.html\#jvms-2.9}}. 
    This behavior is modelled as an overapproximation in FlowDroid's default call graph algorithm SPARK. SPARK adds an edge to \code{<clinit>} at each statement containing a StaticFieldRef, StaticInvokeExpr or NewExpr \footnote{\url{https://github.com/soot-oss/soot/blob/59931576784b910a7d38f81910b7313aa2feafea/src/main/java/soot/jimple/toolkits/callgraph/OnFlyCallGraphBuilder.java\#L969}}.
   
    \begin{figure}[ht]
        \centering
        \begin{subfigure}[b]{0.45\textwidth}
            \centering
            \begin{lstlisting}[gobble=16]
                class ClinitClass1 {
                    public static string str = source();
                }
            \end{lstlisting}
            \caption{static variable initialization}
            \label{lst:clinit_examples_a}
        \end{subfigure}
        \hfill
        \begin{subfigure}[b]{0.45\textwidth}
            \centering
            \begin{lstlisting}[gobble=16]
                class ClinitClass2 {
                    static {
                        ClinitClass2.sink();
                    }
                }
            \end{lstlisting}
            \caption{static block}
            \label{lst:clinit_examples_b}
        \end{subfigure}
        \caption{Examples of statements being in \code{<clint>}}
        \label{lst:clinit_examples}
    \end{figure}


    The need for this rule is rooted in the IFDSSolver of FlowDroid. The solver decides whether to use normal flow or call flow by calling \code{isCallStmt(Unit u)} on the interprocedural control-flow graph generated by Soot. Internally, this method calls \code{containsInvokeExpr()} on the unit object. \code{containsInvokeExpr()} for AssignStmt only returns true if the right hand side is an instance of InvokeExpr. Resulting, we miss the call to \code{<clinit>} for AssignStmts with NewExpr or StaticFieldRef on the right side.

    The Backwards Clinit Rule manually injects an edge to the \code{<clinit>} method in the infoflow solver when appropriate during the analysis. Also, it lessens the overapproximation of SPARK by carefully choosing whether to inject the edge. The rule works as follows:
    \begin{itemize}
        \item If the tainted static variable is a field of the methods class: Do not inject because we will at least encounter a NewExpr of the same class further in the call graph.
        \item If the tainted static variable matches the StaticFieldRef on the right hand side: inject the edge.
        \item If the class of the tainted static variable matches the class of the NewExpr: inject the edge.
    \end{itemize}

    This rule has no equivalent in forwards analysis because in fowards analysis the problem is not as severe. As taints are introducted at sources, if the source statement is a static initialization as shown in \autoref{lst:clinit_examples_a}, the propagation starts inside the \code{<clinit>} method. The solver has a followReturnsPastSeeds option which propagates return flows for unbalanced problems, for example when the taint was introducted inside a method and therefore there was no incoming flow. This allows the forwards analysis to detect leaks originated from static variable initializations but misses leaks inside static blocks as shown in \autoref{lst:clinit_examples_b}.

    \section{Code Optimizer}
    Before starting the analysis, FlowDroid applies code optimization to the interprocedural call graph. By default, dead code elimination and within constant value propagation is performed. Those are also applied before backwards analysis but we needed another code optimizer to handle an edge case in backwards analysis.

    \subsection{AddNOPStmts}
    First, take a look at StatictTestCode\#static2Test in \autoref{lst:static2TestJava}. The method and entry point static2Test is static and does not has any parameters. Same is true for the source method TelephonyManager\#getDeviceId. Due to these conditions, static2Test in Jimple has neither identity statements nor assign statements before the source statement and therefore the source statement is the first statement in the graph. 
    Next, a detail of FlowDroid's IFDS solver is important. The Return and CallToReturn flow function is only applied if a return site is available. When searching backwards, the source statement is the last statement and thus has no return sites. Now recall \autoref{s:sourcerule}, taints flowing into sources are registered in the CallToReturn flow function. Altogether, leaks can not be found if the source statement is the first statement.

    Moving the detection of incoming taints flows into sources from the CallToReturn to the Call flow function was not an option because by default source methods are not visited. 
    Our solution is to just add a NOP statement in such cases. This saves us from introducing new edge cases inside the flow functions which are already complex enough. Due to the entry points being known beforehand, the overhead is negligible.

    \begin{figure}
        \centering
        \begin{lstlisting}[gobble=12]
            public static void static2Test() {
                String tainted = TelephonyManager.getDeviceId();
                ClassWithStatic static1 = new ClassWithStatic();
                static1.setTitle(tainted);
                ClassWithStatic static2 = new ClassWithStatic();
                String alsoTainted = static2.getTitle();
                
                ConnectionManager cm = new ConnectionManager();
                cm.publish(alsoTainted);
            }
        \end{lstlisting}
        \caption{static2Test Java Code}
        \label{lst:static2TestJava}
    \end{figure}

\end{document}