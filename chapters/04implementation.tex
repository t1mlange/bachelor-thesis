\documentclass[../draft.tex]{subfiles}

\begin{document}
    \chapter{Implementation}

    \section{Integration}
    \textsc{FlowDroid} is built to be extensible from to ground up. We wanted to reuse as much components of \textsc{FlowDroid} as possible. For the backwards analysis, we introduce unconditional taints at sinks and check for the matching access paths at sources. Facts are propagated through a reversed interprocedural control flow graph.

    The methods for retrieving sources and sinks from a SourceSinkManager have different signatures because only at one end the access paths must match and at the other the taints are unconditional. 
    We added the interface \code{IReversibleSourceSinkManager} extending the \code{ISourceSinkManager}. It enforces two additional methods:
    \begin{itemize}
        \item \code{SourceInfo getInverseSinkInfo(Stmt sCallSite, InfoflowManager manager)}
        \item \code{SinkInfo getInverseSourceInfo(Stmt sCallSite, InfoflowManager manager, AccessPath ap)}
    \end{itemize}
    \code{getInverseSinkInfo} returns the necessary information for introducing unconditional taints at sinks while \code{getInverseSourceInfo} also matches the access paths at sources.
    All three source sink managers \code{DefaultSourceSinkManager} for modelling Java, \code{AccessPathBasedSourceSinkManager} for modelling Android and \code{SummarySourceSinkManager} for summaries now implement the \code{IReversibleSourceSinkManager} interface.
    Reversible source sink manager currently do not support the one-source-at-a-time mode.

    Due to the flow-sensitive aliasing of \textsc{FlowDroid} using IFDS, \textsc{FlowDroid} already provides an implementation of a reversed interprocedural control flow graph called \code{BackwardsInfoflowCFG}.
    For the core - the flow functions - we created two new components implementing \code{IInfoflowProblem}: the backwards infoflow problem and an alias problem. More on that in \autoref{s:problems}.

    To hide the fact that we internally swapped the sources and sinks, we also created a \code{BackwardsInfoflowResults} extending \code{InfoflowResults}. The implementation is quite simple. It overwrites the \code{addResult} implementations and reverses the constructed paths.

    The modularity of \textsc{FlowDroid} allowed us to easily use the newly created components. We created another implementation of \code{IInfoflow} responsible for initialization of those closely to the already existing default implementation \code{Infoflow}.

    \section{Flow Function Implementation}\label{s:problems}

    \subsection{Flow-Sensitive Alias Analysis}\label{s:aliasing}
    \textsc{FlowDroid} offers multiple aliasing strategies.
    In this work, we focus on the flow-sensitive alias analysis which is implemented as another IFDS problem called \code{BackwardsAliasProblem}. Basically, this is a forwards IFDS search with flow functions using aliasing rules. 

    \paragraph{Handover to Alias Analysis}
    Whenever we visit a statement and notice a taint could have an alias, the taint is handed over to the alias analysis. 
    Normal flow rule 3 is such a case. The taint is on the right side and we notice that the left side also refers to the same value in memory due to being stored in the heap. The left side gets tainted and propagated forwards to find out if we missed a write to the alias.
    In normal flow rule 1 and 2, we also turn around. \autoref{lst:aliasex} shows two cases where the turnaround is necessary.

    \begin{figure}[ht]
        \centering
        \begin{subfigure}[b]{0.45\textwidth}
            \centering
            \begin{lstlisting}[gobble=16]
                void aliasRule1() {
                    A a = b;
                    b.str = source();
                    sink(a.str);
                }
            \end{lstlisting}
            \caption{Example for alias analysis initiated by rule 1}
        \end{subfigure}
        \hfill
        \begin{subfigure}[b]{0.45\textwidth}
            \centering
            \begin{lstlisting}[gobble=16]
                void aliasRule3() {
                    A a = b;
                    a.str = source();
                    sink(b.str);
                }
            \end{lstlisting}
            \caption{Example for alias analysis initiated by rule 3}
        \end{subfigure}
        \caption{Aliasing examples}
        \label{lst:aliasex}
    \end{figure}

    \paragraph{Handing back to Infoflow}

    \paragraph{TurnUnit} 
    We added another field to the \code{Abstraction} class called \code{turnUnit}. This is the equivalent to the \code{activationUnit} in forwards analysis. The \code{turnUnit} references the last statement for which the taint is relevant for the infoflow search. At start, it is the sink it originated from. Later on, it is set whenever we visit an assigment with a primitive or string on the left side. An example can be found in \autoref{lst:turnunit}. Line 5 introduces the taint, line 3 taints $b.str$ and sets the turnUnit to this statement. In line 2, \code{a} is found to be an alias of \code{b} and causes a handover to the alias problem. The \code{turnUnit} now stops the alias search at line 3 and prevents a false positive.

    \begin{figure}[ht]
        \centering
        \begin{lstlisting}
            void turnStmtNeeded() {
                A a = b;
                String str = b.str;
                a.str = source();
                sink(str);
            }
        \end{lstlisting}
        \caption{Aliasing example with turn unit}
        \label{lst:turnunit}
    \end{figure}


    Explain TurnUnit, SkipUnit
    What the core problem tackles

    \section{Rules}
    Flow functions can get quite large, complicated to understand and hard to maintain \cite{Lerch2015}. To counteract this, \textsc{FlowDroid} outsources certain features into rules. These rules also provide the four flow functions and are applied in the corresponding flow function.

    \subsection{Backwards Sink Propagation Rule}


    \subsection{Backwards Source Propagation Rule}\label{s:sourcerule}

    \subsection{Backwards Array Propagation Rule}

    \subsection{Backwards Exception Propagation Rule}

    \subsection{Backwards Wrapper Propagation Rule}
    \textsc{FlowDroid} already provided a \code{IReversibleTaintWrapper} interface. Implementing taint wrappers support \code{getInverseTaints()} which takes the outcoming taint as an input and computes the incoming taints.

    This rule is similiar to its forward equivalent but enforces a reversible taint wrapper. Consequently, tainted return values are also passed into the taint wrapper.

    % \subsection{Backwards Implicit Propagation Rule}
    % Not implemented.

    \subsection{Backwards Strong Update Rule}
    Strong updates are assignments where the content of a variable gets overwritten. In our normal flow rules, this is modelled in rule 1. When a statement is observed with its left side tainted, we know it got its tainted content at this statement. Thus we kill the taint because the content above this statement is of no interest and taint the right hand side. So we performed a strong update on the left side. 

    But with aliasing this gets quite more complicated. Now, we can observe a taint not matching the left side and propagate it over the statement by default but this taint is an alias of the left side and should have been killed. Linking aliasing taints to support such strong updates would lose the distributive property of the flow functions and is not an option. 
    
    In this case, \textsc{FlowDroid} fallbacks to Soot's must-aliasing analysis. However, this analysis is an approximation because it is only intraprocedural and relies on SPARK which is flow-insensitive.

    \begin{itemize}
        \item \textbf{Strong Update Rule}: If the incoming taint must-aliases the left side then apply the normal flow rules just as if the left side was tainted. 
    \end{itemize}


    \subsection{Backwards Clinit Rule}\label{s:clinitrule}
    \code{<clinit>} is a special method in the JVM and stands for class loader init. The function is generated by the compiler and can not be called explicitly. Examples of statements which get compiled into clinit can be seen in \autoref{lst:clinit_examples}. The invokation is implicit at the initialization phase of the class and is executed at most once for each class \footnote{\url{https://docs.oracle.com/javase/specs/jvms/se8/html/jvms-2.html\#jvms-2.9}}. 
    This behavior is modelled as an overapproximation in \textsc{FlowDroid}'s default call graph algorithm SPARK. SPARK adds an edge to \code{<clinit>} at each statement containing a \code{StaticFieldRef}, \code{StaticInvokeExpr} or \code{NewExpr} \footnote{\url{https://github.com/soot-oss/soot/blob/59931576784b910a7d38f81910b7313aa2feafea/src/main/java/soot/jimple/toolkits/callgraph/OnFlyCallGraphBuilder.java\#L969}}.
   
    \begin{figure}[ht]
        \centering
        \begin{subfigure}[b]{0.45\textwidth}
            \centering
            \begin{lstlisting}[gobble=16]
                class ClinitClass1 {
                    public static string str = source();
                }
            \end{lstlisting}
            \caption{static variable initialization}
            \label{lst:clinit_examples_a}
        \end{subfigure}
        \hfill
        \begin{subfigure}[b]{0.45\textwidth}
            \centering
            \begin{lstlisting}[gobble=16]
                class ClinitClass2 {
                    static {
                        ClinitClass2.sink();
                    }
                }
            \end{lstlisting}
            \caption{static block}
            \label{lst:clinit_examples_b}
        \end{subfigure}
        \caption{Examples of statements being in \code{<clinit>}}
        \label{lst:clinit_examples}
    \end{figure}


    The need for this rule is rooted in the IFDS solver of \textsc{FlowDroid}. The solver decides whether to use normal flow or call flow by calling \code{isCallStmt(Unit u)} on the interprocedural control flow graph generated by Soot. Internally, this method calls \code{containsInvokeExpr()} on the \code{Unit} object. \code{containsInvokeExpr()} for \code{AssignStmt} only returns true if the right hand side is an instance of \code{InvokeExpr}. Resulting, we miss the call to \code{<clinit>} for AssignStmts with NewExpr or StaticFieldRef on the right side.

    The Backwards Clinit Rule manually injects an edge to the \code{<clinit>} method in the infoflow solver when appropriate during the analysis. Also, it lessens the overapproximation of SPARK by carefully choosing whether to inject the edge. The rule works as follows:
    \begin{itemize}
        \item \textbf{Clinit Rule 1}: If the tainted static variable is a field of the methods class: Do not inject because we will at least encounter a \code{NewExpr} of the same class further in the call graph.
        \item \textbf{Clinit Rule 2}: Else if the tainted static variable matches the \code{StaticFieldRef} on the right hand side: Inject the edge because we can not be sure whether we see another edge to \code{<clinit>}.
        \item \textbf{Clinit Rule 3}: Else if the class of the tainted static variable matches the class of the \code{NewExpr}: Inject the edge because we can not be sure whether we see another edge to \code{<clinit>}.
    \end{itemize}
    This is still an overapproximation of course. A precise solution would require bookkeeping of the first occurence in the code of every class. 

    This rule has no equivalent in forwards analysis because in fowards analysis the problem is not as severe. As taints are introducted at sources, if the source statement is a static initialization as shown in \autoref{lst:clinit_examples_a}, the propagation starts inside the \code{<clinit>} method. The solver has a \code{followReturnsPastSeeds} option which propagates return flows for unbalanced problems, for example when the taint was introducted inside a method and therefore there was no incoming flow. This allows the forwards analysis to detect leaks originated from static variable initializations but misses leaks inside static blocks as shown in \autoref{lst:clinit_examples_b}.

    \subsection{Other Rules}
    Skip System Class Rule and Stop After First K Flows Rule are not direction-dependent. Both are shared with the forwards search and therefore use the existing implementation in \textsc{FlowDroid}.
    
    Typing Propagation Rule has no backwards equivalent. We decided to implement type checking in the infoflow problem instead.



    \section{Code Optimizer}
    Before starting the analysis, \textsc{FlowDroid} applies code optimization to the interprocedural call graph. By default, dead code elimination and within constant value propagation is performed. Those are also applied before backwards analysis but we needed another code optimizer to handle an edge case in backwards analysis.

    \subsection{AddNOPStmts}
    First, take a look at \code{StatictTestCode#static2Test} in \autoref{lst:static2TestJava}. The method and entry point \code{static2Test} is static and does not have any parameters. Same is true for the source method \code{TelephonyManager#getDeviceId}. Due to the first condition, \code{static2Test} has no identity statements and because of the second  condition there are also no assign statements before the source statement in Jimple. Therefore the source statement is the first statement in the graph. 
    Next, a detail of \textsc{FlowDroid}'s IFDS solver is important. The Return and CallToReturn flow function is only applied if a return site is available \cite{Arzt2017PhD}.
    When searching backwards, the source statement is the last statement and thus has no return sites. Now recall \autoref{s:sourcerule}, taints flowing into sources are registered in the CallToReturn flow function. Altogether, leaks can not be found if the source statement is the first statement.

    Moving the detection of incoming taints flows into sources from the CallToReturn to the Call flow function was not an option because by default source methods are not visited. 
    Our solution is to just add a NOP statement in such cases. This saves us from introducing new edge cases inside the flow functions which are already complex enough. Due to the entry points being known beforehand, the overhead is negligible.

    \begin{figure}
        \centering
        \begin{lstlisting}[gobble=12]
            public static void static2Test() {
                String tainted = TelephonyManager.getDeviceId();
                ClassWithStatic static1 = new ClassWithStatic();
                static1.setTitle(tainted);
                ClassWithStatic static2 = new ClassWithStatic();
                String alsoTainted = static2.getTitle();
                
                ConnectionManager cm = new ConnectionManager();
                cm.publish(alsoTainted);
            }
        \end{lstlisting}
        \caption{static2Test Java Code}
        \label{lst:static2TestJava}
    \end{figure}

\end{document}