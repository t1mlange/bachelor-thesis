\documentclass[../draft.tex]{subfiles}

\begin{document}
    \chapter{Performance Evaluation}
    In the last chapter, we have shown that our implementation has the necessary soundness to be viable and yields the expected results. 
    We now evaluate our implementation against the existing implementation in \textsc{FlowDroid}.
    
    \section{DroidBench}
    We already introduced \textsc{Droidbench} in \autoref{s:droidbenchvalidation} to validate the soundness of our backward-directed implementation. In this section, we focus on the performance in comparison to the existing forward-directed implementation in \textsc{FlowDroid}. 

    \textsc{DroidBench} has the advantage that all apps are crafted explicitly for benchmarking taint analysis. So, most tests only contain a single-figure number of sources and sinks. Also, the number of sources and sinks are often equal or differ by one to test whether the tool can differentiate something. These simplify the comparison between both analysis directions as neither one has an initial disadvantage.

    Most test cases are small enough to be analyzed in sub-two seconds on an average four-core desktop CPU from 2012. Our test environment is not isolated, so background tasks and the process scheduler can affect the runtime. The short runtime, together with the variance of the unisolated testing environment, render the runtime unusable as a comparison point. In contrast, edge propagations are deterministic\footnote{This is only true if there are enough resources. \textsc{FlowDroid} tries to gracefully terminate when running low on memory. Also, timeouts result in a non-reproducible number of edge propagations.} and correlate with the runtime. Thus, we only use the number of propagations to compare both implementations.

    The configuration is the same as described in \autoref{s:droidbenchconfig}.

    \subsection{Results}
    We compare all test cases where both implementations yield the same result. When rows only contain hyphens, either the result of the test case differed between the two analyses or the IFDS analysis did not start, e.g., because no sink is in the reachable code. \#I denotes the number of edge propagations inside the infoflow analysis and \#A the number of edge propagations inside the alias analysis. We calculated the absolute difference as $\mathit{Result}_{\mathit{B}} - \mathit{Result}_{\mathit{F}}$. The relative difference calculates as follows: $\frac{\mathit{TotalDifference}}{|\mathit{\#I_\mathit{F} + \#A_\mathit{F}}|}$. Hence, negative values signify the backward analysis performed better. The full results are in \autoref{t:droidbenchevaluation}.

    In general, both implementations have similar average edge propagation counts. There are not many test cases where both perform identically; instead, dependent on the specific test case, the relative difference is between $-1$ and $1$. So, the expected behavior from \autoref{s:complexity} occurred: it highly depends on the analyzed app.
    However, we did not expect cases that let the backward edge propagations explode up to a factor of $10 000\%$, as seen in \code{LifecycleTest#BroadcastReceiverLifecycle3} and others. In contrast, the existing forward implementation only at most a relative difference of $100\%$.

    \footnotesize
    \newcommand{\tsubEight}[1]{\multicolumn{9}{c}{#1}\\\hline}
    \begin{longtable}{l | r | r | r | r | r | r | r | r}
        \centering
        & \multicolumn{2}{c|}{\textbf{Forwards}} & \multicolumn{2}{c|}{\textbf{Backwards}} & \multicolumn{4}{c}{\textbf{Difference}}\\
        \textbf{App Name} & \textbf{\#I} & \textbf{\#A} & \textbf{\#I} & \textbf{\#A} & \textbf{\#I} & \textbf{\#A}& \textbf{Total} & \textbf{Relative}\\
        \hline\hline
        \endhead
        \hline
        \tsubEight{AliasingTest}
        FlowSensitivity1 & $175$ & $72$ & $39$ & $4$ & $-136$ & $-68$ & $-204$ & $-0.83$\\
        Merge1 & $94$ & $44$ & $61$ & $9$ & $-33$ & $-35$ & $-68$ & $-0.49$\\
        SimpleAliasing1 & $35$ & $13$ & $20$ & $3$ & $-15$ & $-10$ & $-25$ & $-0.52$\\
        StrongUpdate1 & $30$ & $13$ & $11$ & $3$ & $-19$ & $-10$ & $-29$ & $-0.67$\\
        \hline
        \tsubEight{AndroidSpecificTest}
        ApplicationModeling1 & $235$ & $103$ & $851$ & $1208$ & $616$ & $1105$ & $1721$ & $5.09$\\
        DirectLeak1 & $3$ & $0$ & $4$ & $0$ & $1$ & $0$ & $1$ & $0.33$\\
        InactiveActivity & $-$ & $-$ & $-$ & $-$ & $-$ & $-$ & $-$ & $-$\\
        Library2 & $5$ & $0$ & $6$ & $0$ & $1$ & $0$ & $1$ & $0.2$\\
        LogNoLeak & $-$ & $-$ & $-$ & $-$ & $-$ & $-$ & $-$ & $-$\\
        Obfuscation1 & $4$ & $0$ & $4$ & $0$ & $0$ & $0$ & $0$ & $0.0$\\
        Parcel1 & $144$ & $15$ & $66$ & $68$ & $-78$ & $53$ & $-25$ & $-0.16$\\
        PrivateDataLeak1 & $410$ & $110$ & $599$ & $835$ & $189$ & $725$ & $914$ & $1.76$\\
        PrivateDataLeak2 & $15$ & $0$ & $5$ & $6$ & $-10$ & $6$ & $-4$ & $-0.27$\\
        PrivateDataLeak3 & $17$ & $2$ & $212$ & $143$ & $195$ & $141$ & $336$ & $17.68$\\
        runPublicAPIField1 & $89$ & $1$ & $62$ & $31$ & $-27$ & $30$ & $3$ & $0.03$\\
        runPublicAPIField2 & $5$ & $0$ & $11$ & $1$ & $6$ & $1$ & $7$ & $1.4$\\
        runView1 & $71$ & $50$ & $69$ & $0$ & $-2$ & $-50$ & $-52$ & $-0.43$\\
        \hline
        \tsubEight{ArrayAndListTest}
        ArrayAccess1 & $77$ & $34$ & $51$ & $100$ & $-26$ & $66$ & $40$ & $0.36$\\
        ArrayAccess2 & $16$ & $4$ & $12$ & $0$ & $-4$ & $-4$ & $-8$ & $-0.4$\\
        ArrayAccess3 & $77$ & $34$ & $51$ & $100$ & $-26$ & $66$ & $40$ & $0.36$\\
        ArrayAccess4 & $164$ & $84$ & $42$ & $21$ & $-122$ & $-63$ & $-185$ & $-0.75$\\
        ArrayAccess5 & $75$ & $5$ & $67$ & $63$ & $-8$ & $58$ & $50$ & $0.62$\\
        ArrayCopy1 & $18$ & $2$ & $9$ & $2$ & $-9$ & $0$ & $-9$ & $-0.45$\\
        ArrayToString1 & $10$ & $1$ & $6$ & $1$ & $-4$ & $0$ & $-4$ & $-0.36$\\
        HashMapAccess1 & $22$ & $5$ & $15$ & $1$ & $-7$ & $-4$ & $-11$ & $-0.41$\\
        ListAccess1 & $85$ & $9$ & $60$ & $97$ & $-25$ & $88$ & $63$ & $0.67$\\
        MultidimensionalArray1 & $29$ & $3$ & $16$ & $23$ & $-13$ & $20$ & $7$ & $0.22$\\
        \hline
        \tsubEight{CallbackTest}
        AnonymousClass1 & $152$ & $0$ & $208$ & $1$ & $56$ & $1$ & $57$ & $0.38$\\
        Button1 & $58$ & $39$ & $43$ & $0$ & $-15$ & $-39$ & $-54$ & $-0.56$\\
        Button2 & $454$ & $66$ & $155$ & $257$ & $-299$ & $191$ & $-108$ & $-0.21$\\
        Button3 & $355$ & $89$ & $109$ & $408$ & $-246$ & $319$ & $73$ & $0.16$\\
        Button4 & $58$ & $39$ & $43$ & $0$ & $-15$ & $-39$ & $-54$ & $-0.56$\\
        Button5 & $80$ & $40$ & $6$ & $6$ & $-74$ & $-34$ & $-108$ & $-0.9$\\
        LocationLeak1 & $617$ & $222$ & $260$ & $300$ & $-357$ & $78$ & $-279$ & $-0.33$\\
        LocationLeak2 & $212$ & $121$ & $152$ & $2$ & $-60$ & $-119$ & $-179$ & $-0.54$\\
        LocationLeak3 & $259$ & $73$ & $104$ & $117$ & $-155$ & $44$ & $-111$ & $-0.33$\\
        MethodOverride1 & $3$ & $0$ & $2$ & $0$ & $-1$ & $0$ & $-1$ & $-0.33$\\
        MultiHandlers1 & $17$ & $0$ & $145$ & $151$ & $128$ & $151$ & $279$ & $16.41$\\
        Ordering1 & $456$ & $151$ & $44$ & $2$ & $-412$ & $-149$ & $-561$ & $-0.92$\\
        RegisterGlobal1 & $207$ & $103$ & $49$ & $0$ & $-158$ & $-103$ & $-261$ & $-0.84$\\
        RegisterGlobal2 & $52$ & $37$ & $43$ & $0$ & $-9$ & $-37$ & $-46$ & $-0.52$\\
        Unregister1 & $11$ & $0$ & $9$ & $1$ & $-2$ & $1$ & $-1$ & $-0.09$\\
        \hline
        \tsubEight{EmulatorDetectionTest}
        Battery1 & $7$ & $0$ & $43$ & $15$ & $36$ & $15$ & $51$ & $7.29$\\
        Bluetooth1 & $4$ & $0$ & $4$ & $0$ & $0$ & $0$ & $0$ & $0.0$\\
        Build1 & $4$ & $0$ & $4$ & $0$ & $0$ & $0$ & $0$ & $0.0$\\
        Contacts1 & $52$ & $0$ & $210$ & $19$ & $158$ & $19$ & $177$ & $3.4$\\
        ContentProvider1 & $13$ & $0$ & $8$ & $0$ & $-5$ & $0$ & $-5$ & $-0.38$\\
        DeviceId1 & $15$ & $0$ & $6$ & $0$ & $-9$ & $0$ & $-9$ & $-0.6$\\
        File1 & $4$ & $0$ & $4$ & $0$ & $0$ & $0$ & $0$ & $0.0$\\
        IMEI1 & $129$ & $0$ & $140$ & $34$ & $11$ & $34$ & $45$ & $0.35$\\
        IP1 & $4$ & $0$ & $29$ & $1$ & $25$ & $1$ & $26$ & $6.5$\\
        PI1 & $6$ & $0$ & $4$ & $0$ & $-2$ & $0$ & $-2$ & $-0.33$\\
        PlayStore1 & $158$ & $0$ & $8$ & $0$ & $-150$ & $0$ & $-150$ & $-0.95$\\
        PlayStore2 & $4$ & $0$ & $4$ & $0$ & $0$ & $0$ & $0$ & $0.0$\\
        Sensors1 & $5$ & $0$ & $4$ & $0$ & $-1$ & $0$ & $-1$ & $-0.2$\\
        SubscriberId1 & $29$ & $0$ & $4$ & $0$ & $-25$ & $0$ & $-25$ & $-0.86$\\
        VoiceMail1 & $4$ & $0$ & $4$ & $0$ & $0$ & $0$ & $0$ & $0.0$\\
        \hline
        \tsubEight{FieldAndObjectSensitivityTest}
        FieldSensitivity1 & $98$ & $50$ & $25$ & $3$ & $-73$ & $-47$ & $-120$ & $-0.81$\\
        FieldSensitivity2 & $35$ & $15$ & $19$ & $0$ & $-16$ & $-15$ & $-31$ & $-0.62$\\
        FieldSensitivity3 & $38$ & $15$ & $16$ & $0$ & $-22$ & $-15$ & $-37$ & $-0.7$\\
        FieldSensitivity4 & $14$ & $6$ & $8$ & $0$ & $-6$ & $-6$ & $-12$ & $-0.6$\\
        InheritedObjects1 & $4$ & $0$ & $6$ & $0$ & $2$ & $0$ & $2$ & $0.5$\\
        ObjectSensitivity1 & $19$ & $7$ & $14$ & $1$ & $-5$ & $-6$ & $-11$ & $-0.42$\\
        ObjectSensitivity2 & $15$ & $8$ & $10$ & $0$ & $-5$ & $-8$ & $-13$ & $-0.57$\\
        \hline
        \tsubEight{GeneralJavaTest}
        Clone1 & $23$ & $2$ & $12$ & $4$ & $-11$ & $2$ & $-9$ & $-0.36$\\
        Exceptions1 & $16$ & $0$ & $13$ & $0$ & $-3$ & $0$ & $-3$ & $-0.19$\\
        Exceptions2 & $22$ & $0$ & $13$ & $0$ & $-9$ & $0$ & $-9$ & $-0.41$\\
        Exceptions3 & $18$ & $0$ & $11$ & $0$ & $-7$ & $0$ & $-7$ & $-0.39$\\
        Exceptions4 & $20$ & $1$ & $22$ & $1$ & $2$ & $0$ & $2$ & $0.1$\\
        Exceptions5 & $13$ & $1$ & $16$ & $1$ & $3$ & $0$ & $3$ & $0.21$\\
        Exceptions6 & $77$ & $12$ & $23$ & $0$ & $-54$ & $-12$ & $-66$ & $-0.74$\\
        Exceptions7 & $71$ & $12$ & $6$ & $0$ & $-65$ & $-12$ & $-77$ & $-0.93$\\
        FactoryMethods1 & $40$ & $0$ & $14$ & $2$ & $-26$ & $2$ & $-24$ & $-0.6$\\
        Loop1 & $93$ & $2$ & $46$ & $7$ & $-47$ & $5$ & $-42$ & $-0.44$\\
        Loop2 & $123$ & $2$ & $74$ & $14$ & $-49$ & $12$ & $-37$ & $-0.3$\\
        Serialization1 & $50$ & $4$ & $22$ & $29$ & $-28$ & $25$ & $-3$ & $-0.06$\\
        SourceCodeSpecific1 & $16$ & $0$ & $45$ & $7$ & $29$ & $7$ & $36$ & $2.25$\\
        StartProcessWithSecret1 & $29$ & $8$ & $17$ & $3$ & $-12$ & $-5$ & $-17$ & $-0.46$\\
        StaticInitialization1 & $-$ & $-$ & $9$ & $0$ & $-$ & $-$ & $-$ & $-$\\
        StaticInitialization2 & $57$ & $29$ & $86$ & $0$ & $29$ & $-29$ & $0$ & $0.0$\\
        StaticInitialization3 & $35$ & $9$ & $5$ & $0$ & $-30$ & $-9$ & $-39$ & $-0.89$\\
        StringFormatter1 & $16$ & $1$ & $10$ & $1$ & $-6$ & $0$ & $-6$ & $-0.35$\\
        StringPatternMatching1 & $23$ & $1$ & $8$ & $6$ & $-15$ & $5$ & $-10$ & $-0.42$\\
        StringToCharArray1 & $91$ & $4$ & $42$ & $6$ & $-49$ & $2$ & $-47$ & $-0.49$\\
        StringToOutputStream1 & $26$ & $3$ & $30$ & $3$ & $4$ & $0$ & $4$ & $0.14$\\
        UnreachableCode & $-$ & $-$ & $-$ & $-$ & $-$ & $-$ & $-$ & $-$\\
        VirtualDispatch1 & $128$ & $31$ & $88$ & $28$ & $-40$ & $-3$ & $-43$ & $-0.27$\\
        VirtualDispatch2 & $7$ & $0$ & $12$ & $0$ & $5$ & $0$ & $5$ & $0.71$\\
        VirtualDispatch3 & $8$ & $0$ & $6$ & $0$ & $-2$ & $0$ & $-2$ & $-0.25$\\
        VirtualDispatch4 & $-$ & $-$ & $-$ & $-$ & $-$ & $-$ & $-$ & $-$\\
        \hline
        \tsubEight{LifecycleTest}
        ActivityEventSequence1 & $58$ & $35$ & $73$ & $0$ & $15$ & $-35$ & $-20$ & $-0.22$\\
        ActivityEventSequence2 & $32$ & $24$ & $77$ & $0$ & $45$ & $-24$ & $21$ & $0.38$\\
        ActivityEventSequence3 & $233$ & $116$ & $156$ & $1$ & $-77$ & $-115$ & $-192$ & $-0.55$\\
        ActivityLifecycle1 & $99$ & $72$ & $156$ & $7$ & $57$ & $-65$ & $-8$ & $-0.05$\\
        ActivityLifecycle2 & $47$ & $34$ & $33$ & $0$ & $-14$ & $-34$ & $-48$ & $-0.59$\\
        ActivityLifecycle3 & $65$ & $31$ & $28$ & $0$ & $-37$ & $-31$ & $-68$ & $-0.71$\\
        ActivityLifecycle4 & $49$ & $33$ & $14$ & $0$ & $-35$ & $-33$ & $-68$ & $-0.83$\\
        ActivitySavedState1 & $20$ & $0$ & $7$ & $1$ & $-13$ & $1$ & $-12$ & $-0.6$\\
        ApplicationLifecycle1 & $37$ & $10$ & $82$ & $0$ & $45$ & $-10$ & $35$ & $0.74$\\
        ApplicationLifecycle2 & $86$ & $17$ & $94$ & $155$ & $8$ & $138$ & $146$ & $1.42$\\
        ApplicationLifecycle3 & $32$ & $12$ & $21$ & $0$ & $-11$ & $-12$ & $-23$ & $-0.52$\\
        AsynchronousEventOrdering1 & $51$ & $31$ & $16$ & $0$ & $-35$ & $-31$ & $-66$ & $-0.8$\\
        BroadcastReceiverLifecycle1 & $4$ & $0$ & $4$ & $0$ & $0$ & $0$ & $0$ & $0.0$\\
        BroadcastReceiverLifecycle2 & $109$ & $44$ & $248$ & $114$ & $139$ & $70$ & $209$ & $1.37$\\
        BroadcastReceiverLifecycle3 & $3$ & $0$ & $195$ & $110$ & $192$ & $110$ & $302$ & $100.67$\\
        EventOrdering1 & $61$ & $29$ & $30$ & $0$ & $-31$ & $-29$ & $-60$ & $-0.67$\\
        FragmentLifecycle1 & $187$ & $127$ & $90$ & $0$ & $-97$ & $-127$ & $-224$ & $-0.71$\\
        FragmentLifecycle2 & $-$ & $-$ & $-$ & $-$ & $-$ & $-$ & $-$ & $-$\\
        ServiceEventSequence1 & $53$ & $20$ & $152$ & $34$ & $99$ & $14$ & $113$ & $1.55$\\
        ServiceEventSequence2 & $64$ & $21$ & $389$ & $220$ & $325$ & $199$ & $524$ & $6.16$\\
        ServiceEventSequence3 & $46$ & $12$ & $275$ & $151$ & $229$ & $139$ & $368$ & $6.34$\\
        ServiceLifecycle1 & $119$ & $44$ & $42$ & $0$ & $-77$ & $-44$ & $-121$ & $-0.74$\\
        ServiceLifecycle2 & $68$ & $20$ & $89$ & $21$ & $21$ & $1$ & $22$ & $0.25$\\
        SharedPreferenceChanged1 & $13$ & $0$ & $20$ & $1$ & $7$ & $1$ & $8$ & $0.62$\\
        \hline
        \tsubEight{ReflectionTest}
        Reflection1 & $15$ & $5$ & $8$ & $0$ & $-7$ & $-5$ & $-12$ & $-0.6$\\
        Reflection2 & $21$ & $5$ & $11$ & $0$ & $-10$ & $-5$ & $-15$ & $-0.58$\\
        Reflection3 & $42$ & $9$ & $62$ & $25$ & $20$ & $16$ & $36$ & $0.71$\\
        Reflection4 & $9$ & $0$ & $8$ & $0$ & $-1$ & $0$ & $-1$ & $-0.11$\\
        Reflection5 & $16$ & $1$ & $11$ & $0$ & $-5$ & $-1$ & $-6$ & $-0.35$\\
        Reflection6 & $7$ & $0$ & $134$ & $51$ & $127$ & $51$ & $178$ & $25.43$\\
        Reflection7 & $15$ & $5$ & $15$ & $11$ & $0$ & $6$ & $6$ & $0.3$\\
        Reflection8 & $35$ & $7$ & $14$ & $0$ & $-21$ & $-7$ & $-28$ & $-0.67$\\
        Reflection9 & $42$ & $7$ & $21$ & $0$ & $-21$ & $-7$ & $-28$ & $-0.57$\\
        \hline
        \tsubEight{ThreadingTest}
        AsyncTask1 & $22$ & $2$ & $11$ & $1$ & $-11$ & $-1$ & $-12$ & $-0.5$\\
        Executor1 & $34$ & $7$ & $17$ & $0$ & $-17$ & $-7$ & $-24$ & $-0.59$\\
        JavaThread1 & $34$ & $7$ & $17$ & $0$ & $-17$ & $-7$ & $-24$ & $-0.59$\\
        JavaThread2 & $62$ & $10$ & $31$ & $8$ & $-31$ & $-2$ & $-33$ & $-0.46$\\
        Looper1 & $49$ & $3$ & $20$ & $16$ & $-29$ & $13$ & $-16$ & $-0.31$\\
        TimerTask1 & $203$ & $28$ & $32$ & $33$ & $-171$ & $5$ & $-166$ & $-0.72$\\
        \hline\hline
        $\varnothing$ Propagations & $70.74$ & $21.42$ & $61.94$ & $41.54$ & $-8.8$ & $20.11$ & $11.32$ & $1.41$\\        
        \caption{DroidBench Performance Evaluation Results}
        \label{t:droidbenchevaluation}
    \end{longtable}
    \normalsize

    \subsection{Result Explanation}
    We define tests with a relative difference greater than $10$ as worth investigating. In the following, we explain why our implementation performed worse than expected.

    \paragraph{PrivateDataLeak3} 
    This test contains two sinks and one source. The tainted data is written to a file, later read from the file and then leaked. \textsc{FlowDroid} does not support tracking taints over files, so it only finds a leak from source to file write but misses the leak from file read to send SMS. 
    Due to EasyTaintWrapper's simplicity, overtainting happens in the backward direction. When \code{FileInputStream fis = openFileInput("out.txt");}
    is called with \code{fis} tainted, EasyTaintWrapper also taints the base object - the \code{MainActivity} in this case. As the \code{MainActivity} has a enourmous scope, the taint has a long lifetime and many other taints could derive from this taint. This taint explains the relative difference of $17.68$.
    Using the more precise SummaryTaintWrapper, the edges reduce to $(51, 16)$ and a relative difference of $2.53$, which is more reasonable. It is still higher because of the second sink. 

    \paragraph{MultiHandlers1}
    Two \code{LocationListener}s are registered in different activities. In both activities, an instance field is a parameter of a sink.
    So there are two possible paths where something could be leaked. The LocationListener does not call any source on the first path, while the second path has an empty setter method killing the taint.
    For the first path, the backward analysis has to propagate the taint into the \code{LocationListener} to notice that this is a dead-end while the forward's search does not even start there.
    For the second path, the backward analysis seems to suffer because it starts at an instance field taint with a larger scope than a local variable.

    \paragraph{BroadcastReceiverLifecycle3}
    The test contains five sinks but only one source. If we only consider the leak path, both implementations perform equally. 
    The four other sinks are responsible for the overhead on edge propagations.
    
    \paragraph{Reflection6}
    The reflective call site has multiple callees in the interprocedural control-flow graph. Backward all of these callees are visited, of which only one contains a source statement. Forwards, the taint is introduced in the callee at the source and just one return site needs to be processed.

    \section{Real World Apps}

    \subsection{Configuration}
    Our test machine is equipped with four Intel Xeon E5-4650 and 1 TB of RAM. We limited the JVM to 50 GB RAM and \textsc{FlowDroid} on 16 threads per instance. We ran at most four instances in parallel to ensure a one-to-one mapping between CPU threads and \textsc{FlowDroid} threads. Note that the test machine is a shared system, but we made sure there are always enough resources for our evaluation available. Still, background services might influence the performance of a single run. To stamp out this factor, we ran each app three times. If there were outliers, we repeated the runs.
    
    For this evaluation, we chose to use a non-default configuration of \textsc{FlowDroid}. First, we disabled static field tracking due to the global scope as described in \autoref{s:complexity}. Next, instead of the \code{EasyTaintWrapper}, we use the \code{SummaryTaintWrapper}, which utilizes \textsc{StubDroid}. \textsc{StubDroid}'s precomputed data flow summaries are more precise than \code{EasyTaintWrapper}'s simple rules. We set the timeout for the data flow analysis to 10 minutes. 
    The configuration summary is in \autoref{t:realworldconfig}.

    \begin{table}[ht]
        \centering
        \begin{tabular}{l | l}
            \textbf{Option} & \textbf{Value}\\
            \hline\hline
            Array Size Tainting & disabled\\
            Inspect Sources \& Sinks & disabled\\
            Static Field Tracking & disabled\\
            Ignore Flows in System Packages & enabled\\
            Exclude Soot Library Classes & enabled\\
            Timeout & 10 minutes\\
            Taint Wrapper & \code{SummaryTaintWrapper}\\
        \end{tabular}
        \caption{Real World Apps Configuration}
        \label{t:realworldconfig}
    \end{table}
    
    We did not use the full sources and sinks list included in \textsc{FlowDroid} because such would result in hundreds of sources and sinks per app and probably a long runtime. Instead, we chose to analyze which sensitive and possibly user-identifying data is sent out to the internet. As we want to compare the forwards and backward implementation, it is also essential to not put one at a disadvantage. We opted for a 2:1 ratio of sources to sinks. This decision is based on the results of \textsc{SuSi}, a tool to automatically find sources and sinks in the Android framework \cite{Rasthofer2014}. Their extracted list of sources and sinks contains roughly $2.17$ times more sources than sinks.
    The list of sources and sinks used in this evaluation is in \autoref{t:realworldsources} and \autoref{t:realworldsinks}.

    \begin{table}[ht]
        \centering
        \begin{tabular}{l | l}
            \textbf{Class} & \textbf{Method}\\
            \hline\hline
            android.location.Location & getLatitude()\\
            & getLongitude()\\
            \hline
            android.location.LocationManager & getLastKnownLocation()\\
            \hline
            android.telephony.TelephonyManager & getDeviceId()\\
            & getSubscriberId()\\
            & getSimSerialNumber()\\
            & getLine1Number()\\
            & getImei()\\
            & getMeid()\\
            \hline
            android.bluetooth.BluetoothAdapter & getAddress()\\
            android.net.wifi.WifiInfo & getMacAddress()\\
            & getSSID()\\
            & getIpAddress()\\
            \hline
            java.net.InetAddress & getHostAddress()\\
            \hline
            android.telephony.gsm.GsmCellLocation & getCid()\\
            & getLac()\\
            \hline
            android.content.pm.PackageManager & getInstalledApplications()\\
            & getInstalledPackages()\\
            & queryIntentActivities()\\
            & queryIntentServices()\\
            & queryBroadcastReceivers()\\
            \hline
            android.content.SharedPreferences & getDefaultSharedPreferences()\\
            android.provider.Browser & getAllBookmarks()\\
            & getAllVisitedUrls\\
        \end{tabular}
        \caption{Sources for Real World Apps Evaluation}
        \label{t:realworldsources}
    \end{table}

    \begin{table}[ht]
        \centering
        \begin{tabular}{l | l}
            \textbf{Class} & \textbf{Method}\\
            \hline\hline
            java.net.URL & set()\\
            & openConnection()\\
            \hline
            java.net.URLConnection & connect()\\
            & setRequestProperty()\\
            \hline
            android.net.http.HttpsConnection & openConnection()\\
            \hline
            android.net.http.Headers & setEtag()\\
            & setContentType()\\
            & setLastModified()\\
            & setLocation()\\
            \hline
            android.net.http.AndroidHttpClientConnection & sendRequestHeader()\\
            \hline
            android.net.http.RequestQueue & queueRequest()\\
        \end{tabular}
        \caption{Sinks for Real World Apps Evaluation}
        \label{t:realworldsinks}
    \end{table}

    We used FlowDroid's forwards implementation on the to that date latest upstream commit\footnote{The latest upstream commit was at that time b436733fc4a5130dfe4ce8ddb3f76fd374e9a487.} from the develop branch for the point of comparison. The backward implementation ran on our latest commit with all changes merged from the upstream.

    We chose 200 apps randomly out of a Google Playstore dump from 2021 containing over 6000 apps for our evaluation set. The full list is appended to this work in \autoref{?}.

    \subsection{Results}
    Out of 200 apps, 60 apps do not have any sources or sinks and thus, the analysis did not start. For 15 apps, the analysis aborted with errors. We are left with 125 apps for which both implementations completed the analysis.
    \todo{Preliminary! Without memwatcher}

    
    \begin{table}[ht]
        \centering
        \begin{tabular}{l | r | r}
            \textbf{Metric} & \textbf{Forwards} & \textbf{Backwards}\\
            \hline\hline
            Average Runtime & $504.7s$ & $409.5s$\\
            Median Runtime & $600.0s$ & $600.0s$\\
            \hline
            Average Abstractions Infoflow & $34651810$ & $13741637$\\
            Average Abstractions Alias & $12492775$ & $33686397$\\
            Average Total Abstractions & $47144585$ & $47428034$\\
            \hline
            % Average Max Memory Consumption & $24830.68 GB$ & $25243.53 GB$\\
            % \hline
            Memory Timeouts & $3.48\%$ & $4.35\%$\\
            Time Timeouts & $59.13\%$ & $53.91\%$\\                     
        \end{tabular}
        \caption{Results}
        \label{t:realworldresults}
    \end{table}

    \begin{table}[ht]
        \centering
        \begin{tabular}{l | r | r}
            & \textbf{Forwards} & \textbf{Backwards}\\
            \hline\hline
            Average Runtime & $331.6s$ & $133.96s$\\
            Median Runtime & $591.0s$ & $1.0s$\\
            \hline
            Average Abstractions Infoflow & $7468347$ & $2262268$\\
            Average Abstractions Alias & $2658090$ & $4884307$\\
            Average Total Abstractions & $10126437$ & $7146575$\\
            \hline
            % Average Max Memory Consumption & $17127.26 GB$ & $18689.29 GB$\\                       
        \end{tabular}
        \caption{Results without Timeouts}
        \label{t:realworldresultswithouttimeout}
    \end{table}

    \begin{figure}
        \centering
        \begin{subfigure}[]{0.45\textwidth}
            \centering
            \resizebox{\columnwidth}{!}{
                %% Creator: Matplotlib, PGF backend
%%
%% To include the figure in your LaTeX document, write
%%   \input{<filename>.pgf}
%%
%% Make sure the required packages are loaded in your preamble
%%   \usepackage{pgf}
%%
%% and, on pdftex
%%   \usepackage[utf8]{inputenc}\DeclareUnicodeCharacter{2212}{-}
%%
%% or, on luatex and xetex
%%   \usepackage{unicode-math}
%%
%% Figures using additional raster images can only be included by \input if
%% they are in the same directory as the main LaTeX file. For loading figures
%% from other directories you can use the `import` package
%%   \usepackage{import}
%%
%% and then include the figures with
%%   \import{<path to file>}{<filename>.pgf}
%%
%% Matplotlib used the following preamble
%%   \usepackage{fontspec}
%%
\begingroup%
\makeatletter%
\begin{pgfpicture}%
\pgfpathrectangle{\pgfpointorigin}{\pgfqpoint{6.000000in}{4.000000in}}%
\pgfusepath{use as bounding box, clip}%
\begin{pgfscope}%
\pgfsetbuttcap%
\pgfsetmiterjoin%
\definecolor{currentfill}{rgb}{1.000000,1.000000,1.000000}%
\pgfsetfillcolor{currentfill}%
\pgfsetlinewidth{0.000000pt}%
\definecolor{currentstroke}{rgb}{1.000000,1.000000,1.000000}%
\pgfsetstrokecolor{currentstroke}%
\pgfsetdash{}{0pt}%
\pgfpathmoveto{\pgfqpoint{0.000000in}{0.000000in}}%
\pgfpathlineto{\pgfqpoint{6.000000in}{0.000000in}}%
\pgfpathlineto{\pgfqpoint{6.000000in}{4.000000in}}%
\pgfpathlineto{\pgfqpoint{0.000000in}{4.000000in}}%
\pgfpathclose%
\pgfusepath{fill}%
\end{pgfscope}%
\begin{pgfscope}%
\pgfsetbuttcap%
\pgfsetmiterjoin%
\definecolor{currentfill}{rgb}{1.000000,1.000000,1.000000}%
\pgfsetfillcolor{currentfill}%
\pgfsetlinewidth{0.000000pt}%
\definecolor{currentstroke}{rgb}{0.000000,0.000000,0.000000}%
\pgfsetstrokecolor{currentstroke}%
\pgfsetstrokeopacity{0.000000}%
\pgfsetdash{}{0pt}%
\pgfpathmoveto{\pgfqpoint{0.750000in}{0.500000in}}%
\pgfpathlineto{\pgfqpoint{5.400000in}{0.500000in}}%
\pgfpathlineto{\pgfqpoint{5.400000in}{3.520000in}}%
\pgfpathlineto{\pgfqpoint{0.750000in}{3.520000in}}%
\pgfpathclose%
\pgfusepath{fill}%
\end{pgfscope}%
\begin{pgfscope}%
\pgfpathrectangle{\pgfqpoint{0.750000in}{0.500000in}}{\pgfqpoint{4.650000in}{3.020000in}}%
\pgfusepath{clip}%
\pgfsetbuttcap%
\pgfsetroundjoin%
\definecolor{currentfill}{rgb}{1.000000,0.498039,0.054902}%
\pgfsetfillcolor{currentfill}%
\pgfsetlinewidth{1.003750pt}%
\definecolor{currentstroke}{rgb}{1.000000,0.498039,0.054902}%
\pgfsetstrokecolor{currentstroke}%
\pgfsetdash{}{0pt}%
\pgfpathmoveto{\pgfqpoint{1.625649in}{2.939952in}}%
\pgfpathcurveto{\pgfqpoint{1.636699in}{2.939952in}}{\pgfqpoint{1.647299in}{2.944342in}}{\pgfqpoint{1.655112in}{2.952156in}}%
\pgfpathcurveto{\pgfqpoint{1.662926in}{2.959969in}}{\pgfqpoint{1.667316in}{2.970568in}}{\pgfqpoint{1.667316in}{2.981618in}}%
\pgfpathcurveto{\pgfqpoint{1.667316in}{2.992668in}}{\pgfqpoint{1.662926in}{3.003267in}}{\pgfqpoint{1.655112in}{3.011081in}}%
\pgfpathcurveto{\pgfqpoint{1.647299in}{3.018895in}}{\pgfqpoint{1.636699in}{3.023285in}}{\pgfqpoint{1.625649in}{3.023285in}}%
\pgfpathcurveto{\pgfqpoint{1.614599in}{3.023285in}}{\pgfqpoint{1.604000in}{3.018895in}}{\pgfqpoint{1.596187in}{3.011081in}}%
\pgfpathcurveto{\pgfqpoint{1.588373in}{3.003267in}}{\pgfqpoint{1.583983in}{2.992668in}}{\pgfqpoint{1.583983in}{2.981618in}}%
\pgfpathcurveto{\pgfqpoint{1.583983in}{2.970568in}}{\pgfqpoint{1.588373in}{2.959969in}}{\pgfqpoint{1.596187in}{2.952156in}}%
\pgfpathcurveto{\pgfqpoint{1.604000in}{2.944342in}}{\pgfqpoint{1.614599in}{2.939952in}}{\pgfqpoint{1.625649in}{2.939952in}}%
\pgfpathclose%
\pgfusepath{stroke,fill}%
\end{pgfscope}%
\begin{pgfscope}%
\pgfpathrectangle{\pgfqpoint{0.750000in}{0.500000in}}{\pgfqpoint{4.650000in}{3.020000in}}%
\pgfusepath{clip}%
\pgfsetbuttcap%
\pgfsetroundjoin%
\definecolor{currentfill}{rgb}{0.121569,0.466667,0.705882}%
\pgfsetfillcolor{currentfill}%
\pgfsetlinewidth{1.003750pt}%
\definecolor{currentstroke}{rgb}{0.121569,0.466667,0.705882}%
\pgfsetstrokecolor{currentstroke}%
\pgfsetdash{}{0pt}%
\pgfpathmoveto{\pgfqpoint{1.323701in}{0.595606in}}%
\pgfpathcurveto{\pgfqpoint{1.334751in}{0.595606in}}{\pgfqpoint{1.345350in}{0.599996in}}{\pgfqpoint{1.353164in}{0.607810in}}%
\pgfpathcurveto{\pgfqpoint{1.360978in}{0.615624in}}{\pgfqpoint{1.365368in}{0.626223in}}{\pgfqpoint{1.365368in}{0.637273in}}%
\pgfpathcurveto{\pgfqpoint{1.365368in}{0.648323in}}{\pgfqpoint{1.360978in}{0.658922in}}{\pgfqpoint{1.353164in}{0.666736in}}%
\pgfpathcurveto{\pgfqpoint{1.345350in}{0.674549in}}{\pgfqpoint{1.334751in}{0.678939in}}{\pgfqpoint{1.323701in}{0.678939in}}%
\pgfpathcurveto{\pgfqpoint{1.312651in}{0.678939in}}{\pgfqpoint{1.302052in}{0.674549in}}{\pgfqpoint{1.294239in}{0.666736in}}%
\pgfpathcurveto{\pgfqpoint{1.286425in}{0.658922in}}{\pgfqpoint{1.282035in}{0.648323in}}{\pgfqpoint{1.282035in}{0.637273in}}%
\pgfpathcurveto{\pgfqpoint{1.282035in}{0.626223in}}{\pgfqpoint{1.286425in}{0.615624in}}{\pgfqpoint{1.294239in}{0.607810in}}%
\pgfpathcurveto{\pgfqpoint{1.302052in}{0.599996in}}{\pgfqpoint{1.312651in}{0.595606in}}{\pgfqpoint{1.323701in}{0.595606in}}%
\pgfpathclose%
\pgfusepath{stroke,fill}%
\end{pgfscope}%
\begin{pgfscope}%
\pgfpathrectangle{\pgfqpoint{0.750000in}{0.500000in}}{\pgfqpoint{4.650000in}{3.020000in}}%
\pgfusepath{clip}%
\pgfsetbuttcap%
\pgfsetroundjoin%
\definecolor{currentfill}{rgb}{1.000000,0.498039,0.054902}%
\pgfsetfillcolor{currentfill}%
\pgfsetlinewidth{1.003750pt}%
\definecolor{currentstroke}{rgb}{1.000000,0.498039,0.054902}%
\pgfsetstrokecolor{currentstroke}%
\pgfsetdash{}{0pt}%
\pgfpathmoveto{\pgfqpoint{2.652273in}{3.099616in}}%
\pgfpathcurveto{\pgfqpoint{2.663323in}{3.099616in}}{\pgfqpoint{2.673922in}{3.104007in}}{\pgfqpoint{2.681736in}{3.111820in}}%
\pgfpathcurveto{\pgfqpoint{2.689549in}{3.119634in}}{\pgfqpoint{2.693939in}{3.130233in}}{\pgfqpoint{2.693939in}{3.141283in}}%
\pgfpathcurveto{\pgfqpoint{2.693939in}{3.152333in}}{\pgfqpoint{2.689549in}{3.162932in}}{\pgfqpoint{2.681736in}{3.170746in}}%
\pgfpathcurveto{\pgfqpoint{2.673922in}{3.178559in}}{\pgfqpoint{2.663323in}{3.182950in}}{\pgfqpoint{2.652273in}{3.182950in}}%
\pgfpathcurveto{\pgfqpoint{2.641223in}{3.182950in}}{\pgfqpoint{2.630624in}{3.178559in}}{\pgfqpoint{2.622810in}{3.170746in}}%
\pgfpathcurveto{\pgfqpoint{2.614996in}{3.162932in}}{\pgfqpoint{2.610606in}{3.152333in}}{\pgfqpoint{2.610606in}{3.141283in}}%
\pgfpathcurveto{\pgfqpoint{2.610606in}{3.130233in}}{\pgfqpoint{2.614996in}{3.119634in}}{\pgfqpoint{2.622810in}{3.111820in}}%
\pgfpathcurveto{\pgfqpoint{2.630624in}{3.104007in}}{\pgfqpoint{2.641223in}{3.099616in}}{\pgfqpoint{2.652273in}{3.099616in}}%
\pgfpathclose%
\pgfusepath{stroke,fill}%
\end{pgfscope}%
\begin{pgfscope}%
\pgfpathrectangle{\pgfqpoint{0.750000in}{0.500000in}}{\pgfqpoint{4.650000in}{3.020000in}}%
\pgfusepath{clip}%
\pgfsetbuttcap%
\pgfsetroundjoin%
\definecolor{currentfill}{rgb}{1.000000,0.498039,0.054902}%
\pgfsetfillcolor{currentfill}%
\pgfsetlinewidth{1.003750pt}%
\definecolor{currentstroke}{rgb}{1.000000,0.498039,0.054902}%
\pgfsetstrokecolor{currentstroke}%
\pgfsetdash{}{0pt}%
\pgfpathmoveto{\pgfqpoint{2.531494in}{2.932163in}}%
\pgfpathcurveto{\pgfqpoint{2.542544in}{2.932163in}}{\pgfqpoint{2.553143in}{2.936553in}}{\pgfqpoint{2.560956in}{2.944367in}}%
\pgfpathcurveto{\pgfqpoint{2.568770in}{2.952181in}}{\pgfqpoint{2.573160in}{2.962780in}}{\pgfqpoint{2.573160in}{2.973830in}}%
\pgfpathcurveto{\pgfqpoint{2.573160in}{2.984880in}}{\pgfqpoint{2.568770in}{2.995479in}}{\pgfqpoint{2.560956in}{3.003293in}}%
\pgfpathcurveto{\pgfqpoint{2.553143in}{3.011106in}}{\pgfqpoint{2.542544in}{3.015496in}}{\pgfqpoint{2.531494in}{3.015496in}}%
\pgfpathcurveto{\pgfqpoint{2.520443in}{3.015496in}}{\pgfqpoint{2.509844in}{3.011106in}}{\pgfqpoint{2.502031in}{3.003293in}}%
\pgfpathcurveto{\pgfqpoint{2.494217in}{2.995479in}}{\pgfqpoint{2.489827in}{2.984880in}}{\pgfqpoint{2.489827in}{2.973830in}}%
\pgfpathcurveto{\pgfqpoint{2.489827in}{2.962780in}}{\pgfqpoint{2.494217in}{2.952181in}}{\pgfqpoint{2.502031in}{2.944367in}}%
\pgfpathcurveto{\pgfqpoint{2.509844in}{2.936553in}}{\pgfqpoint{2.520443in}{2.932163in}}{\pgfqpoint{2.531494in}{2.932163in}}%
\pgfpathclose%
\pgfusepath{stroke,fill}%
\end{pgfscope}%
\begin{pgfscope}%
\pgfpathrectangle{\pgfqpoint{0.750000in}{0.500000in}}{\pgfqpoint{4.650000in}{3.020000in}}%
\pgfusepath{clip}%
\pgfsetbuttcap%
\pgfsetroundjoin%
\definecolor{currentfill}{rgb}{1.000000,0.498039,0.054902}%
\pgfsetfillcolor{currentfill}%
\pgfsetlinewidth{1.003750pt}%
\definecolor{currentstroke}{rgb}{1.000000,0.498039,0.054902}%
\pgfsetstrokecolor{currentstroke}%
\pgfsetdash{}{0pt}%
\pgfpathmoveto{\pgfqpoint{2.108766in}{3.084039in}}%
\pgfpathcurveto{\pgfqpoint{2.119816in}{3.084039in}}{\pgfqpoint{2.130415in}{3.088430in}}{\pgfqpoint{2.138229in}{3.096243in}}%
\pgfpathcurveto{\pgfqpoint{2.146043in}{3.104057in}}{\pgfqpoint{2.150433in}{3.114656in}}{\pgfqpoint{2.150433in}{3.125706in}}%
\pgfpathcurveto{\pgfqpoint{2.150433in}{3.136756in}}{\pgfqpoint{2.146043in}{3.147355in}}{\pgfqpoint{2.138229in}{3.155169in}}%
\pgfpathcurveto{\pgfqpoint{2.130415in}{3.162982in}}{\pgfqpoint{2.119816in}{3.167373in}}{\pgfqpoint{2.108766in}{3.167373in}}%
\pgfpathcurveto{\pgfqpoint{2.097716in}{3.167373in}}{\pgfqpoint{2.087117in}{3.162982in}}{\pgfqpoint{2.079303in}{3.155169in}}%
\pgfpathcurveto{\pgfqpoint{2.071490in}{3.147355in}}{\pgfqpoint{2.067100in}{3.136756in}}{\pgfqpoint{2.067100in}{3.125706in}}%
\pgfpathcurveto{\pgfqpoint{2.067100in}{3.114656in}}{\pgfqpoint{2.071490in}{3.104057in}}{\pgfqpoint{2.079303in}{3.096243in}}%
\pgfpathcurveto{\pgfqpoint{2.087117in}{3.088430in}}{\pgfqpoint{2.097716in}{3.084039in}}{\pgfqpoint{2.108766in}{3.084039in}}%
\pgfpathclose%
\pgfusepath{stroke,fill}%
\end{pgfscope}%
\begin{pgfscope}%
\pgfpathrectangle{\pgfqpoint{0.750000in}{0.500000in}}{\pgfqpoint{4.650000in}{3.020000in}}%
\pgfusepath{clip}%
\pgfsetbuttcap%
\pgfsetroundjoin%
\definecolor{currentfill}{rgb}{1.000000,0.498039,0.054902}%
\pgfsetfillcolor{currentfill}%
\pgfsetlinewidth{1.003750pt}%
\definecolor{currentstroke}{rgb}{1.000000,0.498039,0.054902}%
\pgfsetstrokecolor{currentstroke}%
\pgfsetdash{}{0pt}%
\pgfpathmoveto{\pgfqpoint{2.350325in}{2.936057in}}%
\pgfpathcurveto{\pgfqpoint{2.361375in}{2.936057in}}{\pgfqpoint{2.371974in}{2.940448in}}{\pgfqpoint{2.379787in}{2.948261in}}%
\pgfpathcurveto{\pgfqpoint{2.387601in}{2.956075in}}{\pgfqpoint{2.391991in}{2.966674in}}{\pgfqpoint{2.391991in}{2.977724in}}%
\pgfpathcurveto{\pgfqpoint{2.391991in}{2.988774in}}{\pgfqpoint{2.387601in}{2.999373in}}{\pgfqpoint{2.379787in}{3.007187in}}%
\pgfpathcurveto{\pgfqpoint{2.371974in}{3.015000in}}{\pgfqpoint{2.361375in}{3.019391in}}{\pgfqpoint{2.350325in}{3.019391in}}%
\pgfpathcurveto{\pgfqpoint{2.339275in}{3.019391in}}{\pgfqpoint{2.328676in}{3.015000in}}{\pgfqpoint{2.320862in}{3.007187in}}%
\pgfpathcurveto{\pgfqpoint{2.313048in}{2.999373in}}{\pgfqpoint{2.308658in}{2.988774in}}{\pgfqpoint{2.308658in}{2.977724in}}%
\pgfpathcurveto{\pgfqpoint{2.308658in}{2.966674in}}{\pgfqpoint{2.313048in}{2.956075in}}{\pgfqpoint{2.320862in}{2.948261in}}%
\pgfpathcurveto{\pgfqpoint{2.328676in}{2.940448in}}{\pgfqpoint{2.339275in}{2.936057in}}{\pgfqpoint{2.350325in}{2.936057in}}%
\pgfpathclose%
\pgfusepath{stroke,fill}%
\end{pgfscope}%
\begin{pgfscope}%
\pgfpathrectangle{\pgfqpoint{0.750000in}{0.500000in}}{\pgfqpoint{4.650000in}{3.020000in}}%
\pgfusepath{clip}%
\pgfsetbuttcap%
\pgfsetroundjoin%
\definecolor{currentfill}{rgb}{1.000000,0.498039,0.054902}%
\pgfsetfillcolor{currentfill}%
\pgfsetlinewidth{1.003750pt}%
\definecolor{currentstroke}{rgb}{1.000000,0.498039,0.054902}%
\pgfsetstrokecolor{currentstroke}%
\pgfsetdash{}{0pt}%
\pgfpathmoveto{\pgfqpoint{1.686039in}{3.341061in}}%
\pgfpathcurveto{\pgfqpoint{1.697089in}{3.341061in}}{\pgfqpoint{1.707688in}{3.345451in}}{\pgfqpoint{1.715502in}{3.353264in}}%
\pgfpathcurveto{\pgfqpoint{1.723315in}{3.361078in}}{\pgfqpoint{1.727706in}{3.371677in}}{\pgfqpoint{1.727706in}{3.382727in}}%
\pgfpathcurveto{\pgfqpoint{1.727706in}{3.393777in}}{\pgfqpoint{1.723315in}{3.404376in}}{\pgfqpoint{1.715502in}{3.412190in}}%
\pgfpathcurveto{\pgfqpoint{1.707688in}{3.420004in}}{\pgfqpoint{1.697089in}{3.424394in}}{\pgfqpoint{1.686039in}{3.424394in}}%
\pgfpathcurveto{\pgfqpoint{1.674989in}{3.424394in}}{\pgfqpoint{1.664390in}{3.420004in}}{\pgfqpoint{1.656576in}{3.412190in}}%
\pgfpathcurveto{\pgfqpoint{1.648763in}{3.404376in}}{\pgfqpoint{1.644372in}{3.393777in}}{\pgfqpoint{1.644372in}{3.382727in}}%
\pgfpathcurveto{\pgfqpoint{1.644372in}{3.371677in}}{\pgfqpoint{1.648763in}{3.361078in}}{\pgfqpoint{1.656576in}{3.353264in}}%
\pgfpathcurveto{\pgfqpoint{1.664390in}{3.345451in}}{\pgfqpoint{1.674989in}{3.341061in}}{\pgfqpoint{1.686039in}{3.341061in}}%
\pgfpathclose%
\pgfusepath{stroke,fill}%
\end{pgfscope}%
\begin{pgfscope}%
\pgfpathrectangle{\pgfqpoint{0.750000in}{0.500000in}}{\pgfqpoint{4.650000in}{3.020000in}}%
\pgfusepath{clip}%
\pgfsetbuttcap%
\pgfsetroundjoin%
\definecolor{currentfill}{rgb}{1.000000,0.498039,0.054902}%
\pgfsetfillcolor{currentfill}%
\pgfsetlinewidth{1.003750pt}%
\definecolor{currentstroke}{rgb}{1.000000,0.498039,0.054902}%
\pgfsetstrokecolor{currentstroke}%
\pgfsetdash{}{0pt}%
\pgfpathmoveto{\pgfqpoint{1.504870in}{3.169713in}}%
\pgfpathcurveto{\pgfqpoint{1.515920in}{3.169713in}}{\pgfqpoint{1.526519in}{3.174103in}}{\pgfqpoint{1.534333in}{3.181917in}}%
\pgfpathcurveto{\pgfqpoint{1.542147in}{3.189731in}}{\pgfqpoint{1.546537in}{3.200330in}}{\pgfqpoint{1.546537in}{3.211380in}}%
\pgfpathcurveto{\pgfqpoint{1.546537in}{3.222430in}}{\pgfqpoint{1.542147in}{3.233029in}}{\pgfqpoint{1.534333in}{3.240843in}}%
\pgfpathcurveto{\pgfqpoint{1.526519in}{3.248656in}}{\pgfqpoint{1.515920in}{3.253046in}}{\pgfqpoint{1.504870in}{3.253046in}}%
\pgfpathcurveto{\pgfqpoint{1.493820in}{3.253046in}}{\pgfqpoint{1.483221in}{3.248656in}}{\pgfqpoint{1.475407in}{3.240843in}}%
\pgfpathcurveto{\pgfqpoint{1.467594in}{3.233029in}}{\pgfqpoint{1.463203in}{3.222430in}}{\pgfqpoint{1.463203in}{3.211380in}}%
\pgfpathcurveto{\pgfqpoint{1.463203in}{3.200330in}}{\pgfqpoint{1.467594in}{3.189731in}}{\pgfqpoint{1.475407in}{3.181917in}}%
\pgfpathcurveto{\pgfqpoint{1.483221in}{3.174103in}}{\pgfqpoint{1.493820in}{3.169713in}}{\pgfqpoint{1.504870in}{3.169713in}}%
\pgfpathclose%
\pgfusepath{stroke,fill}%
\end{pgfscope}%
\begin{pgfscope}%
\pgfpathrectangle{\pgfqpoint{0.750000in}{0.500000in}}{\pgfqpoint{4.650000in}{3.020000in}}%
\pgfusepath{clip}%
\pgfsetbuttcap%
\pgfsetroundjoin%
\definecolor{currentfill}{rgb}{1.000000,0.498039,0.054902}%
\pgfsetfillcolor{currentfill}%
\pgfsetlinewidth{1.003750pt}%
\definecolor{currentstroke}{rgb}{1.000000,0.498039,0.054902}%
\pgfsetstrokecolor{currentstroke}%
\pgfsetdash{}{0pt}%
\pgfpathmoveto{\pgfqpoint{1.927597in}{2.932163in}}%
\pgfpathcurveto{\pgfqpoint{1.938648in}{2.932163in}}{\pgfqpoint{1.949247in}{2.936553in}}{\pgfqpoint{1.957060in}{2.944367in}}%
\pgfpathcurveto{\pgfqpoint{1.964874in}{2.952181in}}{\pgfqpoint{1.969264in}{2.962780in}}{\pgfqpoint{1.969264in}{2.973830in}}%
\pgfpathcurveto{\pgfqpoint{1.969264in}{2.984880in}}{\pgfqpoint{1.964874in}{2.995479in}}{\pgfqpoint{1.957060in}{3.003293in}}%
\pgfpathcurveto{\pgfqpoint{1.949247in}{3.011106in}}{\pgfqpoint{1.938648in}{3.015496in}}{\pgfqpoint{1.927597in}{3.015496in}}%
\pgfpathcurveto{\pgfqpoint{1.916547in}{3.015496in}}{\pgfqpoint{1.905948in}{3.011106in}}{\pgfqpoint{1.898135in}{3.003293in}}%
\pgfpathcurveto{\pgfqpoint{1.890321in}{2.995479in}}{\pgfqpoint{1.885931in}{2.984880in}}{\pgfqpoint{1.885931in}{2.973830in}}%
\pgfpathcurveto{\pgfqpoint{1.885931in}{2.962780in}}{\pgfqpoint{1.890321in}{2.952181in}}{\pgfqpoint{1.898135in}{2.944367in}}%
\pgfpathcurveto{\pgfqpoint{1.905948in}{2.936553in}}{\pgfqpoint{1.916547in}{2.932163in}}{\pgfqpoint{1.927597in}{2.932163in}}%
\pgfpathclose%
\pgfusepath{stroke,fill}%
\end{pgfscope}%
\begin{pgfscope}%
\pgfpathrectangle{\pgfqpoint{0.750000in}{0.500000in}}{\pgfqpoint{4.650000in}{3.020000in}}%
\pgfusepath{clip}%
\pgfsetbuttcap%
\pgfsetroundjoin%
\definecolor{currentfill}{rgb}{1.000000,0.498039,0.054902}%
\pgfsetfillcolor{currentfill}%
\pgfsetlinewidth{1.003750pt}%
\definecolor{currentstroke}{rgb}{1.000000,0.498039,0.054902}%
\pgfsetstrokecolor{currentstroke}%
\pgfsetdash{}{0pt}%
\pgfpathmoveto{\pgfqpoint{1.686039in}{2.932163in}}%
\pgfpathcurveto{\pgfqpoint{1.697089in}{2.932163in}}{\pgfqpoint{1.707688in}{2.936553in}}{\pgfqpoint{1.715502in}{2.944367in}}%
\pgfpathcurveto{\pgfqpoint{1.723315in}{2.952181in}}{\pgfqpoint{1.727706in}{2.962780in}}{\pgfqpoint{1.727706in}{2.973830in}}%
\pgfpathcurveto{\pgfqpoint{1.727706in}{2.984880in}}{\pgfqpoint{1.723315in}{2.995479in}}{\pgfqpoint{1.715502in}{3.003293in}}%
\pgfpathcurveto{\pgfqpoint{1.707688in}{3.011106in}}{\pgfqpoint{1.697089in}{3.015496in}}{\pgfqpoint{1.686039in}{3.015496in}}%
\pgfpathcurveto{\pgfqpoint{1.674989in}{3.015496in}}{\pgfqpoint{1.664390in}{3.011106in}}{\pgfqpoint{1.656576in}{3.003293in}}%
\pgfpathcurveto{\pgfqpoint{1.648763in}{2.995479in}}{\pgfqpoint{1.644372in}{2.984880in}}{\pgfqpoint{1.644372in}{2.973830in}}%
\pgfpathcurveto{\pgfqpoint{1.644372in}{2.962780in}}{\pgfqpoint{1.648763in}{2.952181in}}{\pgfqpoint{1.656576in}{2.944367in}}%
\pgfpathcurveto{\pgfqpoint{1.664390in}{2.936553in}}{\pgfqpoint{1.674989in}{2.932163in}}{\pgfqpoint{1.686039in}{2.932163in}}%
\pgfpathclose%
\pgfusepath{stroke,fill}%
\end{pgfscope}%
\begin{pgfscope}%
\pgfpathrectangle{\pgfqpoint{0.750000in}{0.500000in}}{\pgfqpoint{4.650000in}{3.020000in}}%
\pgfusepath{clip}%
\pgfsetbuttcap%
\pgfsetroundjoin%
\definecolor{currentfill}{rgb}{1.000000,0.498039,0.054902}%
\pgfsetfillcolor{currentfill}%
\pgfsetlinewidth{1.003750pt}%
\definecolor{currentstroke}{rgb}{1.000000,0.498039,0.054902}%
\pgfsetstrokecolor{currentstroke}%
\pgfsetdash{}{0pt}%
\pgfpathmoveto{\pgfqpoint{3.437338in}{2.928269in}}%
\pgfpathcurveto{\pgfqpoint{3.448388in}{2.928269in}}{\pgfqpoint{3.458987in}{2.932659in}}{\pgfqpoint{3.466800in}{2.940473in}}%
\pgfpathcurveto{\pgfqpoint{3.474614in}{2.948286in}}{\pgfqpoint{3.479004in}{2.958885in}}{\pgfqpoint{3.479004in}{2.969936in}}%
\pgfpathcurveto{\pgfqpoint{3.479004in}{2.980986in}}{\pgfqpoint{3.474614in}{2.991585in}}{\pgfqpoint{3.466800in}{2.999398in}}%
\pgfpathcurveto{\pgfqpoint{3.458987in}{3.007212in}}{\pgfqpoint{3.448388in}{3.011602in}}{\pgfqpoint{3.437338in}{3.011602in}}%
\pgfpathcurveto{\pgfqpoint{3.426288in}{3.011602in}}{\pgfqpoint{3.415689in}{3.007212in}}{\pgfqpoint{3.407875in}{2.999398in}}%
\pgfpathcurveto{\pgfqpoint{3.400061in}{2.991585in}}{\pgfqpoint{3.395671in}{2.980986in}}{\pgfqpoint{3.395671in}{2.969936in}}%
\pgfpathcurveto{\pgfqpoint{3.395671in}{2.958885in}}{\pgfqpoint{3.400061in}{2.948286in}}{\pgfqpoint{3.407875in}{2.940473in}}%
\pgfpathcurveto{\pgfqpoint{3.415689in}{2.932659in}}{\pgfqpoint{3.426288in}{2.928269in}}{\pgfqpoint{3.437338in}{2.928269in}}%
\pgfpathclose%
\pgfusepath{stroke,fill}%
\end{pgfscope}%
\begin{pgfscope}%
\pgfpathrectangle{\pgfqpoint{0.750000in}{0.500000in}}{\pgfqpoint{4.650000in}{3.020000in}}%
\pgfusepath{clip}%
\pgfsetbuttcap%
\pgfsetroundjoin%
\definecolor{currentfill}{rgb}{1.000000,0.498039,0.054902}%
\pgfsetfillcolor{currentfill}%
\pgfsetlinewidth{1.003750pt}%
\definecolor{currentstroke}{rgb}{1.000000,0.498039,0.054902}%
\pgfsetstrokecolor{currentstroke}%
\pgfsetdash{}{0pt}%
\pgfpathmoveto{\pgfqpoint{1.323701in}{1.876818in}}%
\pgfpathcurveto{\pgfqpoint{1.334751in}{1.876818in}}{\pgfqpoint{1.345350in}{1.881208in}}{\pgfqpoint{1.353164in}{1.889022in}}%
\pgfpathcurveto{\pgfqpoint{1.360978in}{1.896836in}}{\pgfqpoint{1.365368in}{1.907435in}}{\pgfqpoint{1.365368in}{1.918485in}}%
\pgfpathcurveto{\pgfqpoint{1.365368in}{1.929535in}}{\pgfqpoint{1.360978in}{1.940134in}}{\pgfqpoint{1.353164in}{1.947948in}}%
\pgfpathcurveto{\pgfqpoint{1.345350in}{1.955761in}}{\pgfqpoint{1.334751in}{1.960152in}}{\pgfqpoint{1.323701in}{1.960152in}}%
\pgfpathcurveto{\pgfqpoint{1.312651in}{1.960152in}}{\pgfqpoint{1.302052in}{1.955761in}}{\pgfqpoint{1.294239in}{1.947948in}}%
\pgfpathcurveto{\pgfqpoint{1.286425in}{1.940134in}}{\pgfqpoint{1.282035in}{1.929535in}}{\pgfqpoint{1.282035in}{1.918485in}}%
\pgfpathcurveto{\pgfqpoint{1.282035in}{1.907435in}}{\pgfqpoint{1.286425in}{1.896836in}}{\pgfqpoint{1.294239in}{1.889022in}}%
\pgfpathcurveto{\pgfqpoint{1.302052in}{1.881208in}}{\pgfqpoint{1.312651in}{1.876818in}}{\pgfqpoint{1.323701in}{1.876818in}}%
\pgfpathclose%
\pgfusepath{stroke,fill}%
\end{pgfscope}%
\begin{pgfscope}%
\pgfpathrectangle{\pgfqpoint{0.750000in}{0.500000in}}{\pgfqpoint{4.650000in}{3.020000in}}%
\pgfusepath{clip}%
\pgfsetbuttcap%
\pgfsetroundjoin%
\definecolor{currentfill}{rgb}{1.000000,0.498039,0.054902}%
\pgfsetfillcolor{currentfill}%
\pgfsetlinewidth{1.003750pt}%
\definecolor{currentstroke}{rgb}{1.000000,0.498039,0.054902}%
\pgfsetstrokecolor{currentstroke}%
\pgfsetdash{}{0pt}%
\pgfpathmoveto{\pgfqpoint{2.531494in}{2.928269in}}%
\pgfpathcurveto{\pgfqpoint{2.542544in}{2.928269in}}{\pgfqpoint{2.553143in}{2.932659in}}{\pgfqpoint{2.560956in}{2.940473in}}%
\pgfpathcurveto{\pgfqpoint{2.568770in}{2.948286in}}{\pgfqpoint{2.573160in}{2.958885in}}{\pgfqpoint{2.573160in}{2.969936in}}%
\pgfpathcurveto{\pgfqpoint{2.573160in}{2.980986in}}{\pgfqpoint{2.568770in}{2.991585in}}{\pgfqpoint{2.560956in}{2.999398in}}%
\pgfpathcurveto{\pgfqpoint{2.553143in}{3.007212in}}{\pgfqpoint{2.542544in}{3.011602in}}{\pgfqpoint{2.531494in}{3.011602in}}%
\pgfpathcurveto{\pgfqpoint{2.520443in}{3.011602in}}{\pgfqpoint{2.509844in}{3.007212in}}{\pgfqpoint{2.502031in}{2.999398in}}%
\pgfpathcurveto{\pgfqpoint{2.494217in}{2.991585in}}{\pgfqpoint{2.489827in}{2.980986in}}{\pgfqpoint{2.489827in}{2.969936in}}%
\pgfpathcurveto{\pgfqpoint{2.489827in}{2.958885in}}{\pgfqpoint{2.494217in}{2.948286in}}{\pgfqpoint{2.502031in}{2.940473in}}%
\pgfpathcurveto{\pgfqpoint{2.509844in}{2.932659in}}{\pgfqpoint{2.520443in}{2.928269in}}{\pgfqpoint{2.531494in}{2.928269in}}%
\pgfpathclose%
\pgfusepath{stroke,fill}%
\end{pgfscope}%
\begin{pgfscope}%
\pgfpathrectangle{\pgfqpoint{0.750000in}{0.500000in}}{\pgfqpoint{4.650000in}{3.020000in}}%
\pgfusepath{clip}%
\pgfsetbuttcap%
\pgfsetroundjoin%
\definecolor{currentfill}{rgb}{1.000000,0.498039,0.054902}%
\pgfsetfillcolor{currentfill}%
\pgfsetlinewidth{1.003750pt}%
\definecolor{currentstroke}{rgb}{1.000000,0.498039,0.054902}%
\pgfsetstrokecolor{currentstroke}%
\pgfsetdash{}{0pt}%
\pgfpathmoveto{\pgfqpoint{1.867208in}{2.947740in}}%
\pgfpathcurveto{\pgfqpoint{1.878258in}{2.947740in}}{\pgfqpoint{1.888857in}{2.952130in}}{\pgfqpoint{1.896671in}{2.959944in}}%
\pgfpathcurveto{\pgfqpoint{1.904484in}{2.967758in}}{\pgfqpoint{1.908874in}{2.978357in}}{\pgfqpoint{1.908874in}{2.989407in}}%
\pgfpathcurveto{\pgfqpoint{1.908874in}{3.000457in}}{\pgfqpoint{1.904484in}{3.011056in}}{\pgfqpoint{1.896671in}{3.018870in}}%
\pgfpathcurveto{\pgfqpoint{1.888857in}{3.026683in}}{\pgfqpoint{1.878258in}{3.031074in}}{\pgfqpoint{1.867208in}{3.031074in}}%
\pgfpathcurveto{\pgfqpoint{1.856158in}{3.031074in}}{\pgfqpoint{1.845559in}{3.026683in}}{\pgfqpoint{1.837745in}{3.018870in}}%
\pgfpathcurveto{\pgfqpoint{1.829931in}{3.011056in}}{\pgfqpoint{1.825541in}{3.000457in}}{\pgfqpoint{1.825541in}{2.989407in}}%
\pgfpathcurveto{\pgfqpoint{1.825541in}{2.978357in}}{\pgfqpoint{1.829931in}{2.967758in}}{\pgfqpoint{1.837745in}{2.959944in}}%
\pgfpathcurveto{\pgfqpoint{1.845559in}{2.952130in}}{\pgfqpoint{1.856158in}{2.947740in}}{\pgfqpoint{1.867208in}{2.947740in}}%
\pgfpathclose%
\pgfusepath{stroke,fill}%
\end{pgfscope}%
\begin{pgfscope}%
\pgfpathrectangle{\pgfqpoint{0.750000in}{0.500000in}}{\pgfqpoint{4.650000in}{3.020000in}}%
\pgfusepath{clip}%
\pgfsetbuttcap%
\pgfsetroundjoin%
\definecolor{currentfill}{rgb}{1.000000,0.498039,0.054902}%
\pgfsetfillcolor{currentfill}%
\pgfsetlinewidth{1.003750pt}%
\definecolor{currentstroke}{rgb}{1.000000,0.498039,0.054902}%
\pgfsetstrokecolor{currentstroke}%
\pgfsetdash{}{0pt}%
\pgfpathmoveto{\pgfqpoint{1.444481in}{2.939952in}}%
\pgfpathcurveto{\pgfqpoint{1.455531in}{2.939952in}}{\pgfqpoint{1.466130in}{2.944342in}}{\pgfqpoint{1.473943in}{2.952156in}}%
\pgfpathcurveto{\pgfqpoint{1.481757in}{2.959969in}}{\pgfqpoint{1.486147in}{2.970568in}}{\pgfqpoint{1.486147in}{2.981618in}}%
\pgfpathcurveto{\pgfqpoint{1.486147in}{2.992668in}}{\pgfqpoint{1.481757in}{3.003267in}}{\pgfqpoint{1.473943in}{3.011081in}}%
\pgfpathcurveto{\pgfqpoint{1.466130in}{3.018895in}}{\pgfqpoint{1.455531in}{3.023285in}}{\pgfqpoint{1.444481in}{3.023285in}}%
\pgfpathcurveto{\pgfqpoint{1.433430in}{3.023285in}}{\pgfqpoint{1.422831in}{3.018895in}}{\pgfqpoint{1.415018in}{3.011081in}}%
\pgfpathcurveto{\pgfqpoint{1.407204in}{3.003267in}}{\pgfqpoint{1.402814in}{2.992668in}}{\pgfqpoint{1.402814in}{2.981618in}}%
\pgfpathcurveto{\pgfqpoint{1.402814in}{2.970568in}}{\pgfqpoint{1.407204in}{2.959969in}}{\pgfqpoint{1.415018in}{2.952156in}}%
\pgfpathcurveto{\pgfqpoint{1.422831in}{2.944342in}}{\pgfqpoint{1.433430in}{2.939952in}}{\pgfqpoint{1.444481in}{2.939952in}}%
\pgfpathclose%
\pgfusepath{stroke,fill}%
\end{pgfscope}%
\begin{pgfscope}%
\pgfpathrectangle{\pgfqpoint{0.750000in}{0.500000in}}{\pgfqpoint{4.650000in}{3.020000in}}%
\pgfusepath{clip}%
\pgfsetbuttcap%
\pgfsetroundjoin%
\definecolor{currentfill}{rgb}{1.000000,0.498039,0.054902}%
\pgfsetfillcolor{currentfill}%
\pgfsetlinewidth{1.003750pt}%
\definecolor{currentstroke}{rgb}{1.000000,0.498039,0.054902}%
\pgfsetstrokecolor{currentstroke}%
\pgfsetdash{}{0pt}%
\pgfpathmoveto{\pgfqpoint{1.987987in}{2.939952in}}%
\pgfpathcurveto{\pgfqpoint{1.999037in}{2.939952in}}{\pgfqpoint{2.009636in}{2.944342in}}{\pgfqpoint{2.017450in}{2.952156in}}%
\pgfpathcurveto{\pgfqpoint{2.025263in}{2.959969in}}{\pgfqpoint{2.029654in}{2.970568in}}{\pgfqpoint{2.029654in}{2.981618in}}%
\pgfpathcurveto{\pgfqpoint{2.029654in}{2.992668in}}{\pgfqpoint{2.025263in}{3.003267in}}{\pgfqpoint{2.017450in}{3.011081in}}%
\pgfpathcurveto{\pgfqpoint{2.009636in}{3.018895in}}{\pgfqpoint{1.999037in}{3.023285in}}{\pgfqpoint{1.987987in}{3.023285in}}%
\pgfpathcurveto{\pgfqpoint{1.976937in}{3.023285in}}{\pgfqpoint{1.966338in}{3.018895in}}{\pgfqpoint{1.958524in}{3.011081in}}%
\pgfpathcurveto{\pgfqpoint{1.950711in}{3.003267in}}{\pgfqpoint{1.946320in}{2.992668in}}{\pgfqpoint{1.946320in}{2.981618in}}%
\pgfpathcurveto{\pgfqpoint{1.946320in}{2.970568in}}{\pgfqpoint{1.950711in}{2.959969in}}{\pgfqpoint{1.958524in}{2.952156in}}%
\pgfpathcurveto{\pgfqpoint{1.966338in}{2.944342in}}{\pgfqpoint{1.976937in}{2.939952in}}{\pgfqpoint{1.987987in}{2.939952in}}%
\pgfpathclose%
\pgfusepath{stroke,fill}%
\end{pgfscope}%
\begin{pgfscope}%
\pgfpathrectangle{\pgfqpoint{0.750000in}{0.500000in}}{\pgfqpoint{4.650000in}{3.020000in}}%
\pgfusepath{clip}%
\pgfsetbuttcap%
\pgfsetroundjoin%
\definecolor{currentfill}{rgb}{0.121569,0.466667,0.705882}%
\pgfsetfillcolor{currentfill}%
\pgfsetlinewidth{1.003750pt}%
\definecolor{currentstroke}{rgb}{0.121569,0.466667,0.705882}%
\pgfsetstrokecolor{currentstroke}%
\pgfsetdash{}{0pt}%
\pgfpathmoveto{\pgfqpoint{1.202922in}{0.595606in}}%
\pgfpathcurveto{\pgfqpoint{1.213972in}{0.595606in}}{\pgfqpoint{1.224571in}{0.599996in}}{\pgfqpoint{1.232385in}{0.607810in}}%
\pgfpathcurveto{\pgfqpoint{1.240198in}{0.615624in}}{\pgfqpoint{1.244589in}{0.626223in}}{\pgfqpoint{1.244589in}{0.637273in}}%
\pgfpathcurveto{\pgfqpoint{1.244589in}{0.648323in}}{\pgfqpoint{1.240198in}{0.658922in}}{\pgfqpoint{1.232385in}{0.666736in}}%
\pgfpathcurveto{\pgfqpoint{1.224571in}{0.674549in}}{\pgfqpoint{1.213972in}{0.678939in}}{\pgfqpoint{1.202922in}{0.678939in}}%
\pgfpathcurveto{\pgfqpoint{1.191872in}{0.678939in}}{\pgfqpoint{1.181273in}{0.674549in}}{\pgfqpoint{1.173459in}{0.666736in}}%
\pgfpathcurveto{\pgfqpoint{1.165646in}{0.658922in}}{\pgfqpoint{1.161255in}{0.648323in}}{\pgfqpoint{1.161255in}{0.637273in}}%
\pgfpathcurveto{\pgfqpoint{1.161255in}{0.626223in}}{\pgfqpoint{1.165646in}{0.615624in}}{\pgfqpoint{1.173459in}{0.607810in}}%
\pgfpathcurveto{\pgfqpoint{1.181273in}{0.599996in}}{\pgfqpoint{1.191872in}{0.595606in}}{\pgfqpoint{1.202922in}{0.595606in}}%
\pgfpathclose%
\pgfusepath{stroke,fill}%
\end{pgfscope}%
\begin{pgfscope}%
\pgfpathrectangle{\pgfqpoint{0.750000in}{0.500000in}}{\pgfqpoint{4.650000in}{3.020000in}}%
\pgfusepath{clip}%
\pgfsetbuttcap%
\pgfsetroundjoin%
\definecolor{currentfill}{rgb}{1.000000,0.498039,0.054902}%
\pgfsetfillcolor{currentfill}%
\pgfsetlinewidth{1.003750pt}%
\definecolor{currentstroke}{rgb}{1.000000,0.498039,0.054902}%
\pgfsetstrokecolor{currentstroke}%
\pgfsetdash{}{0pt}%
\pgfpathmoveto{\pgfqpoint{1.444481in}{2.932163in}}%
\pgfpathcurveto{\pgfqpoint{1.455531in}{2.932163in}}{\pgfqpoint{1.466130in}{2.936553in}}{\pgfqpoint{1.473943in}{2.944367in}}%
\pgfpathcurveto{\pgfqpoint{1.481757in}{2.952181in}}{\pgfqpoint{1.486147in}{2.962780in}}{\pgfqpoint{1.486147in}{2.973830in}}%
\pgfpathcurveto{\pgfqpoint{1.486147in}{2.984880in}}{\pgfqpoint{1.481757in}{2.995479in}}{\pgfqpoint{1.473943in}{3.003293in}}%
\pgfpathcurveto{\pgfqpoint{1.466130in}{3.011106in}}{\pgfqpoint{1.455531in}{3.015496in}}{\pgfqpoint{1.444481in}{3.015496in}}%
\pgfpathcurveto{\pgfqpoint{1.433430in}{3.015496in}}{\pgfqpoint{1.422831in}{3.011106in}}{\pgfqpoint{1.415018in}{3.003293in}}%
\pgfpathcurveto{\pgfqpoint{1.407204in}{2.995479in}}{\pgfqpoint{1.402814in}{2.984880in}}{\pgfqpoint{1.402814in}{2.973830in}}%
\pgfpathcurveto{\pgfqpoint{1.402814in}{2.962780in}}{\pgfqpoint{1.407204in}{2.952181in}}{\pgfqpoint{1.415018in}{2.944367in}}%
\pgfpathcurveto{\pgfqpoint{1.422831in}{2.936553in}}{\pgfqpoint{1.433430in}{2.932163in}}{\pgfqpoint{1.444481in}{2.932163in}}%
\pgfpathclose%
\pgfusepath{stroke,fill}%
\end{pgfscope}%
\begin{pgfscope}%
\pgfpathrectangle{\pgfqpoint{0.750000in}{0.500000in}}{\pgfqpoint{4.650000in}{3.020000in}}%
\pgfusepath{clip}%
\pgfsetbuttcap%
\pgfsetroundjoin%
\definecolor{currentfill}{rgb}{1.000000,0.498039,0.054902}%
\pgfsetfillcolor{currentfill}%
\pgfsetlinewidth{1.003750pt}%
\definecolor{currentstroke}{rgb}{1.000000,0.498039,0.054902}%
\pgfsetstrokecolor{currentstroke}%
\pgfsetdash{}{0pt}%
\pgfpathmoveto{\pgfqpoint{1.686039in}{2.920480in}}%
\pgfpathcurveto{\pgfqpoint{1.697089in}{2.920480in}}{\pgfqpoint{1.707688in}{2.924871in}}{\pgfqpoint{1.715502in}{2.932684in}}%
\pgfpathcurveto{\pgfqpoint{1.723315in}{2.940498in}}{\pgfqpoint{1.727706in}{2.951097in}}{\pgfqpoint{1.727706in}{2.962147in}}%
\pgfpathcurveto{\pgfqpoint{1.727706in}{2.973197in}}{\pgfqpoint{1.723315in}{2.983796in}}{\pgfqpoint{1.715502in}{2.991610in}}%
\pgfpathcurveto{\pgfqpoint{1.707688in}{2.999423in}}{\pgfqpoint{1.697089in}{3.003814in}}{\pgfqpoint{1.686039in}{3.003814in}}%
\pgfpathcurveto{\pgfqpoint{1.674989in}{3.003814in}}{\pgfqpoint{1.664390in}{2.999423in}}{\pgfqpoint{1.656576in}{2.991610in}}%
\pgfpathcurveto{\pgfqpoint{1.648763in}{2.983796in}}{\pgfqpoint{1.644372in}{2.973197in}}{\pgfqpoint{1.644372in}{2.962147in}}%
\pgfpathcurveto{\pgfqpoint{1.644372in}{2.951097in}}{\pgfqpoint{1.648763in}{2.940498in}}{\pgfqpoint{1.656576in}{2.932684in}}%
\pgfpathcurveto{\pgfqpoint{1.664390in}{2.924871in}}{\pgfqpoint{1.674989in}{2.920480in}}{\pgfqpoint{1.686039in}{2.920480in}}%
\pgfpathclose%
\pgfusepath{stroke,fill}%
\end{pgfscope}%
\begin{pgfscope}%
\pgfpathrectangle{\pgfqpoint{0.750000in}{0.500000in}}{\pgfqpoint{4.650000in}{3.020000in}}%
\pgfusepath{clip}%
\pgfsetbuttcap%
\pgfsetroundjoin%
\definecolor{currentfill}{rgb}{1.000000,0.498039,0.054902}%
\pgfsetfillcolor{currentfill}%
\pgfsetlinewidth{1.003750pt}%
\definecolor{currentstroke}{rgb}{1.000000,0.498039,0.054902}%
\pgfsetstrokecolor{currentstroke}%
\pgfsetdash{}{0pt}%
\pgfpathmoveto{\pgfqpoint{2.289935in}{3.247598in}}%
\pgfpathcurveto{\pgfqpoint{2.300985in}{3.247598in}}{\pgfqpoint{2.311584in}{3.251989in}}{\pgfqpoint{2.319398in}{3.259802in}}%
\pgfpathcurveto{\pgfqpoint{2.327211in}{3.267616in}}{\pgfqpoint{2.331602in}{3.278215in}}{\pgfqpoint{2.331602in}{3.289265in}}%
\pgfpathcurveto{\pgfqpoint{2.331602in}{3.300315in}}{\pgfqpoint{2.327211in}{3.310914in}}{\pgfqpoint{2.319398in}{3.318728in}}%
\pgfpathcurveto{\pgfqpoint{2.311584in}{3.326541in}}{\pgfqpoint{2.300985in}{3.330932in}}{\pgfqpoint{2.289935in}{3.330932in}}%
\pgfpathcurveto{\pgfqpoint{2.278885in}{3.330932in}}{\pgfqpoint{2.268286in}{3.326541in}}{\pgfqpoint{2.260472in}{3.318728in}}%
\pgfpathcurveto{\pgfqpoint{2.252659in}{3.310914in}}{\pgfqpoint{2.248268in}{3.300315in}}{\pgfqpoint{2.248268in}{3.289265in}}%
\pgfpathcurveto{\pgfqpoint{2.248268in}{3.278215in}}{\pgfqpoint{2.252659in}{3.267616in}}{\pgfqpoint{2.260472in}{3.259802in}}%
\pgfpathcurveto{\pgfqpoint{2.268286in}{3.251989in}}{\pgfqpoint{2.278885in}{3.247598in}}{\pgfqpoint{2.289935in}{3.247598in}}%
\pgfpathclose%
\pgfusepath{stroke,fill}%
\end{pgfscope}%
\begin{pgfscope}%
\pgfpathrectangle{\pgfqpoint{0.750000in}{0.500000in}}{\pgfqpoint{4.650000in}{3.020000in}}%
\pgfusepath{clip}%
\pgfsetbuttcap%
\pgfsetroundjoin%
\definecolor{currentfill}{rgb}{1.000000,0.498039,0.054902}%
\pgfsetfillcolor{currentfill}%
\pgfsetlinewidth{1.003750pt}%
\definecolor{currentstroke}{rgb}{1.000000,0.498039,0.054902}%
\pgfsetstrokecolor{currentstroke}%
\pgfsetdash{}{0pt}%
\pgfpathmoveto{\pgfqpoint{2.833442in}{2.936057in}}%
\pgfpathcurveto{\pgfqpoint{2.844492in}{2.936057in}}{\pgfqpoint{2.855091in}{2.940448in}}{\pgfqpoint{2.862904in}{2.948261in}}%
\pgfpathcurveto{\pgfqpoint{2.870718in}{2.956075in}}{\pgfqpoint{2.875108in}{2.966674in}}{\pgfqpoint{2.875108in}{2.977724in}}%
\pgfpathcurveto{\pgfqpoint{2.875108in}{2.988774in}}{\pgfqpoint{2.870718in}{2.999373in}}{\pgfqpoint{2.862904in}{3.007187in}}%
\pgfpathcurveto{\pgfqpoint{2.855091in}{3.015000in}}{\pgfqpoint{2.844492in}{3.019391in}}{\pgfqpoint{2.833442in}{3.019391in}}%
\pgfpathcurveto{\pgfqpoint{2.822391in}{3.019391in}}{\pgfqpoint{2.811792in}{3.015000in}}{\pgfqpoint{2.803979in}{3.007187in}}%
\pgfpathcurveto{\pgfqpoint{2.796165in}{2.999373in}}{\pgfqpoint{2.791775in}{2.988774in}}{\pgfqpoint{2.791775in}{2.977724in}}%
\pgfpathcurveto{\pgfqpoint{2.791775in}{2.966674in}}{\pgfqpoint{2.796165in}{2.956075in}}{\pgfqpoint{2.803979in}{2.948261in}}%
\pgfpathcurveto{\pgfqpoint{2.811792in}{2.940448in}}{\pgfqpoint{2.822391in}{2.936057in}}{\pgfqpoint{2.833442in}{2.936057in}}%
\pgfpathclose%
\pgfusepath{stroke,fill}%
\end{pgfscope}%
\begin{pgfscope}%
\pgfpathrectangle{\pgfqpoint{0.750000in}{0.500000in}}{\pgfqpoint{4.650000in}{3.020000in}}%
\pgfusepath{clip}%
\pgfsetbuttcap%
\pgfsetroundjoin%
\definecolor{currentfill}{rgb}{1.000000,0.498039,0.054902}%
\pgfsetfillcolor{currentfill}%
\pgfsetlinewidth{1.003750pt}%
\definecolor{currentstroke}{rgb}{1.000000,0.498039,0.054902}%
\pgfsetstrokecolor{currentstroke}%
\pgfsetdash{}{0pt}%
\pgfpathmoveto{\pgfqpoint{1.444481in}{2.328553in}}%
\pgfpathcurveto{\pgfqpoint{1.455531in}{2.328553in}}{\pgfqpoint{1.466130in}{2.332943in}}{\pgfqpoint{1.473943in}{2.340756in}}%
\pgfpathcurveto{\pgfqpoint{1.481757in}{2.348570in}}{\pgfqpoint{1.486147in}{2.359169in}}{\pgfqpoint{1.486147in}{2.370219in}}%
\pgfpathcurveto{\pgfqpoint{1.486147in}{2.381269in}}{\pgfqpoint{1.481757in}{2.391868in}}{\pgfqpoint{1.473943in}{2.399682in}}%
\pgfpathcurveto{\pgfqpoint{1.466130in}{2.407496in}}{\pgfqpoint{1.455531in}{2.411886in}}{\pgfqpoint{1.444481in}{2.411886in}}%
\pgfpathcurveto{\pgfqpoint{1.433430in}{2.411886in}}{\pgfqpoint{1.422831in}{2.407496in}}{\pgfqpoint{1.415018in}{2.399682in}}%
\pgfpathcurveto{\pgfqpoint{1.407204in}{2.391868in}}{\pgfqpoint{1.402814in}{2.381269in}}{\pgfqpoint{1.402814in}{2.370219in}}%
\pgfpathcurveto{\pgfqpoint{1.402814in}{2.359169in}}{\pgfqpoint{1.407204in}{2.348570in}}{\pgfqpoint{1.415018in}{2.340756in}}%
\pgfpathcurveto{\pgfqpoint{1.422831in}{2.332943in}}{\pgfqpoint{1.433430in}{2.328553in}}{\pgfqpoint{1.444481in}{2.328553in}}%
\pgfpathclose%
\pgfusepath{stroke,fill}%
\end{pgfscope}%
\begin{pgfscope}%
\pgfpathrectangle{\pgfqpoint{0.750000in}{0.500000in}}{\pgfqpoint{4.650000in}{3.020000in}}%
\pgfusepath{clip}%
\pgfsetbuttcap%
\pgfsetroundjoin%
\definecolor{currentfill}{rgb}{1.000000,0.498039,0.054902}%
\pgfsetfillcolor{currentfill}%
\pgfsetlinewidth{1.003750pt}%
\definecolor{currentstroke}{rgb}{1.000000,0.498039,0.054902}%
\pgfsetstrokecolor{currentstroke}%
\pgfsetdash{}{0pt}%
\pgfpathmoveto{\pgfqpoint{1.746429in}{3.181396in}}%
\pgfpathcurveto{\pgfqpoint{1.757479in}{3.181396in}}{\pgfqpoint{1.768078in}{3.185786in}}{\pgfqpoint{1.775891in}{3.193600in}}%
\pgfpathcurveto{\pgfqpoint{1.783705in}{3.201413in}}{\pgfqpoint{1.788095in}{3.212012in}}{\pgfqpoint{1.788095in}{3.223063in}}%
\pgfpathcurveto{\pgfqpoint{1.788095in}{3.234113in}}{\pgfqpoint{1.783705in}{3.244712in}}{\pgfqpoint{1.775891in}{3.252525in}}%
\pgfpathcurveto{\pgfqpoint{1.768078in}{3.260339in}}{\pgfqpoint{1.757479in}{3.264729in}}{\pgfqpoint{1.746429in}{3.264729in}}%
\pgfpathcurveto{\pgfqpoint{1.735378in}{3.264729in}}{\pgfqpoint{1.724779in}{3.260339in}}{\pgfqpoint{1.716966in}{3.252525in}}%
\pgfpathcurveto{\pgfqpoint{1.709152in}{3.244712in}}{\pgfqpoint{1.704762in}{3.234113in}}{\pgfqpoint{1.704762in}{3.223063in}}%
\pgfpathcurveto{\pgfqpoint{1.704762in}{3.212012in}}{\pgfqpoint{1.709152in}{3.201413in}}{\pgfqpoint{1.716966in}{3.193600in}}%
\pgfpathcurveto{\pgfqpoint{1.724779in}{3.185786in}}{\pgfqpoint{1.735378in}{3.181396in}}{\pgfqpoint{1.746429in}{3.181396in}}%
\pgfpathclose%
\pgfusepath{stroke,fill}%
\end{pgfscope}%
\begin{pgfscope}%
\pgfpathrectangle{\pgfqpoint{0.750000in}{0.500000in}}{\pgfqpoint{4.650000in}{3.020000in}}%
\pgfusepath{clip}%
\pgfsetbuttcap%
\pgfsetroundjoin%
\definecolor{currentfill}{rgb}{0.839216,0.152941,0.156863}%
\pgfsetfillcolor{currentfill}%
\pgfsetlinewidth{1.003750pt}%
\definecolor{currentstroke}{rgb}{0.839216,0.152941,0.156863}%
\pgfsetstrokecolor{currentstroke}%
\pgfsetdash{}{0pt}%
\pgfpathmoveto{\pgfqpoint{1.806818in}{2.936057in}}%
\pgfpathcurveto{\pgfqpoint{1.817868in}{2.936057in}}{\pgfqpoint{1.828467in}{2.940448in}}{\pgfqpoint{1.836281in}{2.948261in}}%
\pgfpathcurveto{\pgfqpoint{1.844095in}{2.956075in}}{\pgfqpoint{1.848485in}{2.966674in}}{\pgfqpoint{1.848485in}{2.977724in}}%
\pgfpathcurveto{\pgfqpoint{1.848485in}{2.988774in}}{\pgfqpoint{1.844095in}{2.999373in}}{\pgfqpoint{1.836281in}{3.007187in}}%
\pgfpathcurveto{\pgfqpoint{1.828467in}{3.015000in}}{\pgfqpoint{1.817868in}{3.019391in}}{\pgfqpoint{1.806818in}{3.019391in}}%
\pgfpathcurveto{\pgfqpoint{1.795768in}{3.019391in}}{\pgfqpoint{1.785169in}{3.015000in}}{\pgfqpoint{1.777355in}{3.007187in}}%
\pgfpathcurveto{\pgfqpoint{1.769542in}{2.999373in}}{\pgfqpoint{1.765152in}{2.988774in}}{\pgfqpoint{1.765152in}{2.977724in}}%
\pgfpathcurveto{\pgfqpoint{1.765152in}{2.966674in}}{\pgfqpoint{1.769542in}{2.956075in}}{\pgfqpoint{1.777355in}{2.948261in}}%
\pgfpathcurveto{\pgfqpoint{1.785169in}{2.940448in}}{\pgfqpoint{1.795768in}{2.936057in}}{\pgfqpoint{1.806818in}{2.936057in}}%
\pgfpathclose%
\pgfusepath{stroke,fill}%
\end{pgfscope}%
\begin{pgfscope}%
\pgfpathrectangle{\pgfqpoint{0.750000in}{0.500000in}}{\pgfqpoint{4.650000in}{3.020000in}}%
\pgfusepath{clip}%
\pgfsetbuttcap%
\pgfsetroundjoin%
\definecolor{currentfill}{rgb}{1.000000,0.498039,0.054902}%
\pgfsetfillcolor{currentfill}%
\pgfsetlinewidth{1.003750pt}%
\definecolor{currentstroke}{rgb}{1.000000,0.498039,0.054902}%
\pgfsetstrokecolor{currentstroke}%
\pgfsetdash{}{0pt}%
\pgfpathmoveto{\pgfqpoint{1.746429in}{2.936057in}}%
\pgfpathcurveto{\pgfqpoint{1.757479in}{2.936057in}}{\pgfqpoint{1.768078in}{2.940448in}}{\pgfqpoint{1.775891in}{2.948261in}}%
\pgfpathcurveto{\pgfqpoint{1.783705in}{2.956075in}}{\pgfqpoint{1.788095in}{2.966674in}}{\pgfqpoint{1.788095in}{2.977724in}}%
\pgfpathcurveto{\pgfqpoint{1.788095in}{2.988774in}}{\pgfqpoint{1.783705in}{2.999373in}}{\pgfqpoint{1.775891in}{3.007187in}}%
\pgfpathcurveto{\pgfqpoint{1.768078in}{3.015000in}}{\pgfqpoint{1.757479in}{3.019391in}}{\pgfqpoint{1.746429in}{3.019391in}}%
\pgfpathcurveto{\pgfqpoint{1.735378in}{3.019391in}}{\pgfqpoint{1.724779in}{3.015000in}}{\pgfqpoint{1.716966in}{3.007187in}}%
\pgfpathcurveto{\pgfqpoint{1.709152in}{2.999373in}}{\pgfqpoint{1.704762in}{2.988774in}}{\pgfqpoint{1.704762in}{2.977724in}}%
\pgfpathcurveto{\pgfqpoint{1.704762in}{2.966674in}}{\pgfqpoint{1.709152in}{2.956075in}}{\pgfqpoint{1.716966in}{2.948261in}}%
\pgfpathcurveto{\pgfqpoint{1.724779in}{2.940448in}}{\pgfqpoint{1.735378in}{2.936057in}}{\pgfqpoint{1.746429in}{2.936057in}}%
\pgfpathclose%
\pgfusepath{stroke,fill}%
\end{pgfscope}%
\begin{pgfscope}%
\pgfpathrectangle{\pgfqpoint{0.750000in}{0.500000in}}{\pgfqpoint{4.650000in}{3.020000in}}%
\pgfusepath{clip}%
\pgfsetbuttcap%
\pgfsetroundjoin%
\definecolor{currentfill}{rgb}{0.121569,0.466667,0.705882}%
\pgfsetfillcolor{currentfill}%
\pgfsetlinewidth{1.003750pt}%
\definecolor{currentstroke}{rgb}{0.121569,0.466667,0.705882}%
\pgfsetstrokecolor{currentstroke}%
\pgfsetdash{}{0pt}%
\pgfpathmoveto{\pgfqpoint{1.021753in}{0.595606in}}%
\pgfpathcurveto{\pgfqpoint{1.032803in}{0.595606in}}{\pgfqpoint{1.043402in}{0.599996in}}{\pgfqpoint{1.051216in}{0.607810in}}%
\pgfpathcurveto{\pgfqpoint{1.059030in}{0.615624in}}{\pgfqpoint{1.063420in}{0.626223in}}{\pgfqpoint{1.063420in}{0.637273in}}%
\pgfpathcurveto{\pgfqpoint{1.063420in}{0.648323in}}{\pgfqpoint{1.059030in}{0.658922in}}{\pgfqpoint{1.051216in}{0.666736in}}%
\pgfpathcurveto{\pgfqpoint{1.043402in}{0.674549in}}{\pgfqpoint{1.032803in}{0.678939in}}{\pgfqpoint{1.021753in}{0.678939in}}%
\pgfpathcurveto{\pgfqpoint{1.010703in}{0.678939in}}{\pgfqpoint{1.000104in}{0.674549in}}{\pgfqpoint{0.992290in}{0.666736in}}%
\pgfpathcurveto{\pgfqpoint{0.984477in}{0.658922in}}{\pgfqpoint{0.980087in}{0.648323in}}{\pgfqpoint{0.980087in}{0.637273in}}%
\pgfpathcurveto{\pgfqpoint{0.980087in}{0.626223in}}{\pgfqpoint{0.984477in}{0.615624in}}{\pgfqpoint{0.992290in}{0.607810in}}%
\pgfpathcurveto{\pgfqpoint{1.000104in}{0.599996in}}{\pgfqpoint{1.010703in}{0.595606in}}{\pgfqpoint{1.021753in}{0.595606in}}%
\pgfpathclose%
\pgfusepath{stroke,fill}%
\end{pgfscope}%
\begin{pgfscope}%
\pgfpathrectangle{\pgfqpoint{0.750000in}{0.500000in}}{\pgfqpoint{4.650000in}{3.020000in}}%
\pgfusepath{clip}%
\pgfsetbuttcap%
\pgfsetroundjoin%
\definecolor{currentfill}{rgb}{1.000000,0.498039,0.054902}%
\pgfsetfillcolor{currentfill}%
\pgfsetlinewidth{1.003750pt}%
\definecolor{currentstroke}{rgb}{1.000000,0.498039,0.054902}%
\pgfsetstrokecolor{currentstroke}%
\pgfsetdash{}{0pt}%
\pgfpathmoveto{\pgfqpoint{1.565260in}{2.932163in}}%
\pgfpathcurveto{\pgfqpoint{1.576310in}{2.932163in}}{\pgfqpoint{1.586909in}{2.936553in}}{\pgfqpoint{1.594723in}{2.944367in}}%
\pgfpathcurveto{\pgfqpoint{1.602536in}{2.952181in}}{\pgfqpoint{1.606926in}{2.962780in}}{\pgfqpoint{1.606926in}{2.973830in}}%
\pgfpathcurveto{\pgfqpoint{1.606926in}{2.984880in}}{\pgfqpoint{1.602536in}{2.995479in}}{\pgfqpoint{1.594723in}{3.003293in}}%
\pgfpathcurveto{\pgfqpoint{1.586909in}{3.011106in}}{\pgfqpoint{1.576310in}{3.015496in}}{\pgfqpoint{1.565260in}{3.015496in}}%
\pgfpathcurveto{\pgfqpoint{1.554210in}{3.015496in}}{\pgfqpoint{1.543611in}{3.011106in}}{\pgfqpoint{1.535797in}{3.003293in}}%
\pgfpathcurveto{\pgfqpoint{1.527983in}{2.995479in}}{\pgfqpoint{1.523593in}{2.984880in}}{\pgfqpoint{1.523593in}{2.973830in}}%
\pgfpathcurveto{\pgfqpoint{1.523593in}{2.962780in}}{\pgfqpoint{1.527983in}{2.952181in}}{\pgfqpoint{1.535797in}{2.944367in}}%
\pgfpathcurveto{\pgfqpoint{1.543611in}{2.936553in}}{\pgfqpoint{1.554210in}{2.932163in}}{\pgfqpoint{1.565260in}{2.932163in}}%
\pgfpathclose%
\pgfusepath{stroke,fill}%
\end{pgfscope}%
\begin{pgfscope}%
\pgfpathrectangle{\pgfqpoint{0.750000in}{0.500000in}}{\pgfqpoint{4.650000in}{3.020000in}}%
\pgfusepath{clip}%
\pgfsetbuttcap%
\pgfsetroundjoin%
\definecolor{currentfill}{rgb}{1.000000,0.498039,0.054902}%
\pgfsetfillcolor{currentfill}%
\pgfsetlinewidth{1.003750pt}%
\definecolor{currentstroke}{rgb}{1.000000,0.498039,0.054902}%
\pgfsetstrokecolor{currentstroke}%
\pgfsetdash{}{0pt}%
\pgfpathmoveto{\pgfqpoint{2.048377in}{2.044271in}}%
\pgfpathcurveto{\pgfqpoint{2.059427in}{2.044271in}}{\pgfqpoint{2.070026in}{2.048662in}}{\pgfqpoint{2.077839in}{2.056475in}}%
\pgfpathcurveto{\pgfqpoint{2.085653in}{2.064289in}}{\pgfqpoint{2.090043in}{2.074888in}}{\pgfqpoint{2.090043in}{2.085938in}}%
\pgfpathcurveto{\pgfqpoint{2.090043in}{2.096988in}}{\pgfqpoint{2.085653in}{2.107587in}}{\pgfqpoint{2.077839in}{2.115401in}}%
\pgfpathcurveto{\pgfqpoint{2.070026in}{2.123215in}}{\pgfqpoint{2.059427in}{2.127605in}}{\pgfqpoint{2.048377in}{2.127605in}}%
\pgfpathcurveto{\pgfqpoint{2.037326in}{2.127605in}}{\pgfqpoint{2.026727in}{2.123215in}}{\pgfqpoint{2.018914in}{2.115401in}}%
\pgfpathcurveto{\pgfqpoint{2.011100in}{2.107587in}}{\pgfqpoint{2.006710in}{2.096988in}}{\pgfqpoint{2.006710in}{2.085938in}}%
\pgfpathcurveto{\pgfqpoint{2.006710in}{2.074888in}}{\pgfqpoint{2.011100in}{2.064289in}}{\pgfqpoint{2.018914in}{2.056475in}}%
\pgfpathcurveto{\pgfqpoint{2.026727in}{2.048662in}}{\pgfqpoint{2.037326in}{2.044271in}}{\pgfqpoint{2.048377in}{2.044271in}}%
\pgfpathclose%
\pgfusepath{stroke,fill}%
\end{pgfscope}%
\begin{pgfscope}%
\pgfpathrectangle{\pgfqpoint{0.750000in}{0.500000in}}{\pgfqpoint{4.650000in}{3.020000in}}%
\pgfusepath{clip}%
\pgfsetbuttcap%
\pgfsetroundjoin%
\definecolor{currentfill}{rgb}{0.839216,0.152941,0.156863}%
\pgfsetfillcolor{currentfill}%
\pgfsetlinewidth{1.003750pt}%
\definecolor{currentstroke}{rgb}{0.839216,0.152941,0.156863}%
\pgfsetstrokecolor{currentstroke}%
\pgfsetdash{}{0pt}%
\pgfpathmoveto{\pgfqpoint{1.746429in}{2.936057in}}%
\pgfpathcurveto{\pgfqpoint{1.757479in}{2.936057in}}{\pgfqpoint{1.768078in}{2.940448in}}{\pgfqpoint{1.775891in}{2.948261in}}%
\pgfpathcurveto{\pgfqpoint{1.783705in}{2.956075in}}{\pgfqpoint{1.788095in}{2.966674in}}{\pgfqpoint{1.788095in}{2.977724in}}%
\pgfpathcurveto{\pgfqpoint{1.788095in}{2.988774in}}{\pgfqpoint{1.783705in}{2.999373in}}{\pgfqpoint{1.775891in}{3.007187in}}%
\pgfpathcurveto{\pgfqpoint{1.768078in}{3.015000in}}{\pgfqpoint{1.757479in}{3.019391in}}{\pgfqpoint{1.746429in}{3.019391in}}%
\pgfpathcurveto{\pgfqpoint{1.735378in}{3.019391in}}{\pgfqpoint{1.724779in}{3.015000in}}{\pgfqpoint{1.716966in}{3.007187in}}%
\pgfpathcurveto{\pgfqpoint{1.709152in}{2.999373in}}{\pgfqpoint{1.704762in}{2.988774in}}{\pgfqpoint{1.704762in}{2.977724in}}%
\pgfpathcurveto{\pgfqpoint{1.704762in}{2.966674in}}{\pgfqpoint{1.709152in}{2.956075in}}{\pgfqpoint{1.716966in}{2.948261in}}%
\pgfpathcurveto{\pgfqpoint{1.724779in}{2.940448in}}{\pgfqpoint{1.735378in}{2.936057in}}{\pgfqpoint{1.746429in}{2.936057in}}%
\pgfpathclose%
\pgfusepath{stroke,fill}%
\end{pgfscope}%
\begin{pgfscope}%
\pgfpathrectangle{\pgfqpoint{0.750000in}{0.500000in}}{\pgfqpoint{4.650000in}{3.020000in}}%
\pgfusepath{clip}%
\pgfsetbuttcap%
\pgfsetroundjoin%
\definecolor{currentfill}{rgb}{1.000000,0.498039,0.054902}%
\pgfsetfillcolor{currentfill}%
\pgfsetlinewidth{1.003750pt}%
\definecolor{currentstroke}{rgb}{1.000000,0.498039,0.054902}%
\pgfsetstrokecolor{currentstroke}%
\pgfsetdash{}{0pt}%
\pgfpathmoveto{\pgfqpoint{5.188636in}{2.877643in}}%
\pgfpathcurveto{\pgfqpoint{5.199686in}{2.877643in}}{\pgfqpoint{5.210286in}{2.882034in}}{\pgfqpoint{5.218099in}{2.889847in}}%
\pgfpathcurveto{\pgfqpoint{5.225913in}{2.897661in}}{\pgfqpoint{5.230303in}{2.908260in}}{\pgfqpoint{5.230303in}{2.919310in}}%
\pgfpathcurveto{\pgfqpoint{5.230303in}{2.930360in}}{\pgfqpoint{5.225913in}{2.940959in}}{\pgfqpoint{5.218099in}{2.948773in}}%
\pgfpathcurveto{\pgfqpoint{5.210286in}{2.956587in}}{\pgfqpoint{5.199686in}{2.960977in}}{\pgfqpoint{5.188636in}{2.960977in}}%
\pgfpathcurveto{\pgfqpoint{5.177586in}{2.960977in}}{\pgfqpoint{5.166987in}{2.956587in}}{\pgfqpoint{5.159174in}{2.948773in}}%
\pgfpathcurveto{\pgfqpoint{5.151360in}{2.940959in}}{\pgfqpoint{5.146970in}{2.930360in}}{\pgfqpoint{5.146970in}{2.919310in}}%
\pgfpathcurveto{\pgfqpoint{5.146970in}{2.908260in}}{\pgfqpoint{5.151360in}{2.897661in}}{\pgfqpoint{5.159174in}{2.889847in}}%
\pgfpathcurveto{\pgfqpoint{5.166987in}{2.882034in}}{\pgfqpoint{5.177586in}{2.877643in}}{\pgfqpoint{5.188636in}{2.877643in}}%
\pgfpathclose%
\pgfusepath{stroke,fill}%
\end{pgfscope}%
\begin{pgfscope}%
\pgfpathrectangle{\pgfqpoint{0.750000in}{0.500000in}}{\pgfqpoint{4.650000in}{3.020000in}}%
\pgfusepath{clip}%
\pgfsetbuttcap%
\pgfsetroundjoin%
\definecolor{currentfill}{rgb}{1.000000,0.498039,0.054902}%
\pgfsetfillcolor{currentfill}%
\pgfsetlinewidth{1.003750pt}%
\definecolor{currentstroke}{rgb}{1.000000,0.498039,0.054902}%
\pgfsetstrokecolor{currentstroke}%
\pgfsetdash{}{0pt}%
\pgfpathmoveto{\pgfqpoint{2.229545in}{2.928269in}}%
\pgfpathcurveto{\pgfqpoint{2.240596in}{2.928269in}}{\pgfqpoint{2.251195in}{2.932659in}}{\pgfqpoint{2.259008in}{2.940473in}}%
\pgfpathcurveto{\pgfqpoint{2.266822in}{2.948286in}}{\pgfqpoint{2.271212in}{2.958885in}}{\pgfqpoint{2.271212in}{2.969936in}}%
\pgfpathcurveto{\pgfqpoint{2.271212in}{2.980986in}}{\pgfqpoint{2.266822in}{2.991585in}}{\pgfqpoint{2.259008in}{2.999398in}}%
\pgfpathcurveto{\pgfqpoint{2.251195in}{3.007212in}}{\pgfqpoint{2.240596in}{3.011602in}}{\pgfqpoint{2.229545in}{3.011602in}}%
\pgfpathcurveto{\pgfqpoint{2.218495in}{3.011602in}}{\pgfqpoint{2.207896in}{3.007212in}}{\pgfqpoint{2.200083in}{2.999398in}}%
\pgfpathcurveto{\pgfqpoint{2.192269in}{2.991585in}}{\pgfqpoint{2.187879in}{2.980986in}}{\pgfqpoint{2.187879in}{2.969936in}}%
\pgfpathcurveto{\pgfqpoint{2.187879in}{2.958885in}}{\pgfqpoint{2.192269in}{2.948286in}}{\pgfqpoint{2.200083in}{2.940473in}}%
\pgfpathcurveto{\pgfqpoint{2.207896in}{2.932659in}}{\pgfqpoint{2.218495in}{2.928269in}}{\pgfqpoint{2.229545in}{2.928269in}}%
\pgfpathclose%
\pgfusepath{stroke,fill}%
\end{pgfscope}%
\begin{pgfscope}%
\pgfpathrectangle{\pgfqpoint{0.750000in}{0.500000in}}{\pgfqpoint{4.650000in}{3.020000in}}%
\pgfusepath{clip}%
\pgfsetbuttcap%
\pgfsetroundjoin%
\definecolor{currentfill}{rgb}{1.000000,0.498039,0.054902}%
\pgfsetfillcolor{currentfill}%
\pgfsetlinewidth{1.003750pt}%
\definecolor{currentstroke}{rgb}{1.000000,0.498039,0.054902}%
\pgfsetstrokecolor{currentstroke}%
\pgfsetdash{}{0pt}%
\pgfpathmoveto{\pgfqpoint{1.746429in}{2.936057in}}%
\pgfpathcurveto{\pgfqpoint{1.757479in}{2.936057in}}{\pgfqpoint{1.768078in}{2.940448in}}{\pgfqpoint{1.775891in}{2.948261in}}%
\pgfpathcurveto{\pgfqpoint{1.783705in}{2.956075in}}{\pgfqpoint{1.788095in}{2.966674in}}{\pgfqpoint{1.788095in}{2.977724in}}%
\pgfpathcurveto{\pgfqpoint{1.788095in}{2.988774in}}{\pgfqpoint{1.783705in}{2.999373in}}{\pgfqpoint{1.775891in}{3.007187in}}%
\pgfpathcurveto{\pgfqpoint{1.768078in}{3.015000in}}{\pgfqpoint{1.757479in}{3.019391in}}{\pgfqpoint{1.746429in}{3.019391in}}%
\pgfpathcurveto{\pgfqpoint{1.735378in}{3.019391in}}{\pgfqpoint{1.724779in}{3.015000in}}{\pgfqpoint{1.716966in}{3.007187in}}%
\pgfpathcurveto{\pgfqpoint{1.709152in}{2.999373in}}{\pgfqpoint{1.704762in}{2.988774in}}{\pgfqpoint{1.704762in}{2.977724in}}%
\pgfpathcurveto{\pgfqpoint{1.704762in}{2.966674in}}{\pgfqpoint{1.709152in}{2.956075in}}{\pgfqpoint{1.716966in}{2.948261in}}%
\pgfpathcurveto{\pgfqpoint{1.724779in}{2.940448in}}{\pgfqpoint{1.735378in}{2.936057in}}{\pgfqpoint{1.746429in}{2.936057in}}%
\pgfpathclose%
\pgfusepath{stroke,fill}%
\end{pgfscope}%
\begin{pgfscope}%
\pgfpathrectangle{\pgfqpoint{0.750000in}{0.500000in}}{\pgfqpoint{4.650000in}{3.020000in}}%
\pgfusepath{clip}%
\pgfsetbuttcap%
\pgfsetroundjoin%
\definecolor{currentfill}{rgb}{0.121569,0.466667,0.705882}%
\pgfsetfillcolor{currentfill}%
\pgfsetlinewidth{1.003750pt}%
\definecolor{currentstroke}{rgb}{0.121569,0.466667,0.705882}%
\pgfsetstrokecolor{currentstroke}%
\pgfsetdash{}{0pt}%
\pgfpathmoveto{\pgfqpoint{1.021753in}{0.595606in}}%
\pgfpathcurveto{\pgfqpoint{1.032803in}{0.595606in}}{\pgfqpoint{1.043402in}{0.599996in}}{\pgfqpoint{1.051216in}{0.607810in}}%
\pgfpathcurveto{\pgfqpoint{1.059030in}{0.615624in}}{\pgfqpoint{1.063420in}{0.626223in}}{\pgfqpoint{1.063420in}{0.637273in}}%
\pgfpathcurveto{\pgfqpoint{1.063420in}{0.648323in}}{\pgfqpoint{1.059030in}{0.658922in}}{\pgfqpoint{1.051216in}{0.666736in}}%
\pgfpathcurveto{\pgfqpoint{1.043402in}{0.674549in}}{\pgfqpoint{1.032803in}{0.678939in}}{\pgfqpoint{1.021753in}{0.678939in}}%
\pgfpathcurveto{\pgfqpoint{1.010703in}{0.678939in}}{\pgfqpoint{1.000104in}{0.674549in}}{\pgfqpoint{0.992290in}{0.666736in}}%
\pgfpathcurveto{\pgfqpoint{0.984477in}{0.658922in}}{\pgfqpoint{0.980087in}{0.648323in}}{\pgfqpoint{0.980087in}{0.637273in}}%
\pgfpathcurveto{\pgfqpoint{0.980087in}{0.626223in}}{\pgfqpoint{0.984477in}{0.615624in}}{\pgfqpoint{0.992290in}{0.607810in}}%
\pgfpathcurveto{\pgfqpoint{1.000104in}{0.599996in}}{\pgfqpoint{1.010703in}{0.595606in}}{\pgfqpoint{1.021753in}{0.595606in}}%
\pgfpathclose%
\pgfusepath{stroke,fill}%
\end{pgfscope}%
\begin{pgfscope}%
\pgfpathrectangle{\pgfqpoint{0.750000in}{0.500000in}}{\pgfqpoint{4.650000in}{3.020000in}}%
\pgfusepath{clip}%
\pgfsetbuttcap%
\pgfsetroundjoin%
\definecolor{currentfill}{rgb}{1.000000,0.498039,0.054902}%
\pgfsetfillcolor{currentfill}%
\pgfsetlinewidth{1.003750pt}%
\definecolor{currentstroke}{rgb}{1.000000,0.498039,0.054902}%
\pgfsetstrokecolor{currentstroke}%
\pgfsetdash{}{0pt}%
\pgfpathmoveto{\pgfqpoint{1.444481in}{2.932163in}}%
\pgfpathcurveto{\pgfqpoint{1.455531in}{2.932163in}}{\pgfqpoint{1.466130in}{2.936553in}}{\pgfqpoint{1.473943in}{2.944367in}}%
\pgfpathcurveto{\pgfqpoint{1.481757in}{2.952181in}}{\pgfqpoint{1.486147in}{2.962780in}}{\pgfqpoint{1.486147in}{2.973830in}}%
\pgfpathcurveto{\pgfqpoint{1.486147in}{2.984880in}}{\pgfqpoint{1.481757in}{2.995479in}}{\pgfqpoint{1.473943in}{3.003293in}}%
\pgfpathcurveto{\pgfqpoint{1.466130in}{3.011106in}}{\pgfqpoint{1.455531in}{3.015496in}}{\pgfqpoint{1.444481in}{3.015496in}}%
\pgfpathcurveto{\pgfqpoint{1.433430in}{3.015496in}}{\pgfqpoint{1.422831in}{3.011106in}}{\pgfqpoint{1.415018in}{3.003293in}}%
\pgfpathcurveto{\pgfqpoint{1.407204in}{2.995479in}}{\pgfqpoint{1.402814in}{2.984880in}}{\pgfqpoint{1.402814in}{2.973830in}}%
\pgfpathcurveto{\pgfqpoint{1.402814in}{2.962780in}}{\pgfqpoint{1.407204in}{2.952181in}}{\pgfqpoint{1.415018in}{2.944367in}}%
\pgfpathcurveto{\pgfqpoint{1.422831in}{2.936553in}}{\pgfqpoint{1.433430in}{2.932163in}}{\pgfqpoint{1.444481in}{2.932163in}}%
\pgfpathclose%
\pgfusepath{stroke,fill}%
\end{pgfscope}%
\begin{pgfscope}%
\pgfpathrectangle{\pgfqpoint{0.750000in}{0.500000in}}{\pgfqpoint{4.650000in}{3.020000in}}%
\pgfusepath{clip}%
\pgfsetbuttcap%
\pgfsetroundjoin%
\definecolor{currentfill}{rgb}{0.121569,0.466667,0.705882}%
\pgfsetfillcolor{currentfill}%
\pgfsetlinewidth{1.003750pt}%
\definecolor{currentstroke}{rgb}{0.121569,0.466667,0.705882}%
\pgfsetstrokecolor{currentstroke}%
\pgfsetdash{}{0pt}%
\pgfpathmoveto{\pgfqpoint{1.625649in}{1.129120in}}%
\pgfpathcurveto{\pgfqpoint{1.636699in}{1.129120in}}{\pgfqpoint{1.647299in}{1.133510in}}{\pgfqpoint{1.655112in}{1.141324in}}%
\pgfpathcurveto{\pgfqpoint{1.662926in}{1.149137in}}{\pgfqpoint{1.667316in}{1.159736in}}{\pgfqpoint{1.667316in}{1.170787in}}%
\pgfpathcurveto{\pgfqpoint{1.667316in}{1.181837in}}{\pgfqpoint{1.662926in}{1.192436in}}{\pgfqpoint{1.655112in}{1.200249in}}%
\pgfpathcurveto{\pgfqpoint{1.647299in}{1.208063in}}{\pgfqpoint{1.636699in}{1.212453in}}{\pgfqpoint{1.625649in}{1.212453in}}%
\pgfpathcurveto{\pgfqpoint{1.614599in}{1.212453in}}{\pgfqpoint{1.604000in}{1.208063in}}{\pgfqpoint{1.596187in}{1.200249in}}%
\pgfpathcurveto{\pgfqpoint{1.588373in}{1.192436in}}{\pgfqpoint{1.583983in}{1.181837in}}{\pgfqpoint{1.583983in}{1.170787in}}%
\pgfpathcurveto{\pgfqpoint{1.583983in}{1.159736in}}{\pgfqpoint{1.588373in}{1.149137in}}{\pgfqpoint{1.596187in}{1.141324in}}%
\pgfpathcurveto{\pgfqpoint{1.604000in}{1.133510in}}{\pgfqpoint{1.614599in}{1.129120in}}{\pgfqpoint{1.625649in}{1.129120in}}%
\pgfpathclose%
\pgfusepath{stroke,fill}%
\end{pgfscope}%
\begin{pgfscope}%
\pgfpathrectangle{\pgfqpoint{0.750000in}{0.500000in}}{\pgfqpoint{4.650000in}{3.020000in}}%
\pgfusepath{clip}%
\pgfsetbuttcap%
\pgfsetroundjoin%
\definecolor{currentfill}{rgb}{1.000000,0.498039,0.054902}%
\pgfsetfillcolor{currentfill}%
\pgfsetlinewidth{1.003750pt}%
\definecolor{currentstroke}{rgb}{1.000000,0.498039,0.054902}%
\pgfsetstrokecolor{currentstroke}%
\pgfsetdash{}{0pt}%
\pgfpathmoveto{\pgfqpoint{1.444481in}{2.932163in}}%
\pgfpathcurveto{\pgfqpoint{1.455531in}{2.932163in}}{\pgfqpoint{1.466130in}{2.936553in}}{\pgfqpoint{1.473943in}{2.944367in}}%
\pgfpathcurveto{\pgfqpoint{1.481757in}{2.952181in}}{\pgfqpoint{1.486147in}{2.962780in}}{\pgfqpoint{1.486147in}{2.973830in}}%
\pgfpathcurveto{\pgfqpoint{1.486147in}{2.984880in}}{\pgfqpoint{1.481757in}{2.995479in}}{\pgfqpoint{1.473943in}{3.003293in}}%
\pgfpathcurveto{\pgfqpoint{1.466130in}{3.011106in}}{\pgfqpoint{1.455531in}{3.015496in}}{\pgfqpoint{1.444481in}{3.015496in}}%
\pgfpathcurveto{\pgfqpoint{1.433430in}{3.015496in}}{\pgfqpoint{1.422831in}{3.011106in}}{\pgfqpoint{1.415018in}{3.003293in}}%
\pgfpathcurveto{\pgfqpoint{1.407204in}{2.995479in}}{\pgfqpoint{1.402814in}{2.984880in}}{\pgfqpoint{1.402814in}{2.973830in}}%
\pgfpathcurveto{\pgfqpoint{1.402814in}{2.962780in}}{\pgfqpoint{1.407204in}{2.952181in}}{\pgfqpoint{1.415018in}{2.944367in}}%
\pgfpathcurveto{\pgfqpoint{1.422831in}{2.936553in}}{\pgfqpoint{1.433430in}{2.932163in}}{\pgfqpoint{1.444481in}{2.932163in}}%
\pgfpathclose%
\pgfusepath{stroke,fill}%
\end{pgfscope}%
\begin{pgfscope}%
\pgfpathrectangle{\pgfqpoint{0.750000in}{0.500000in}}{\pgfqpoint{4.650000in}{3.020000in}}%
\pgfusepath{clip}%
\pgfsetbuttcap%
\pgfsetroundjoin%
\definecolor{currentfill}{rgb}{1.000000,0.498039,0.054902}%
\pgfsetfillcolor{currentfill}%
\pgfsetlinewidth{1.003750pt}%
\definecolor{currentstroke}{rgb}{1.000000,0.498039,0.054902}%
\pgfsetstrokecolor{currentstroke}%
\pgfsetdash{}{0pt}%
\pgfpathmoveto{\pgfqpoint{2.229545in}{2.936057in}}%
\pgfpathcurveto{\pgfqpoint{2.240596in}{2.936057in}}{\pgfqpoint{2.251195in}{2.940448in}}{\pgfqpoint{2.259008in}{2.948261in}}%
\pgfpathcurveto{\pgfqpoint{2.266822in}{2.956075in}}{\pgfqpoint{2.271212in}{2.966674in}}{\pgfqpoint{2.271212in}{2.977724in}}%
\pgfpathcurveto{\pgfqpoint{2.271212in}{2.988774in}}{\pgfqpoint{2.266822in}{2.999373in}}{\pgfqpoint{2.259008in}{3.007187in}}%
\pgfpathcurveto{\pgfqpoint{2.251195in}{3.015000in}}{\pgfqpoint{2.240596in}{3.019391in}}{\pgfqpoint{2.229545in}{3.019391in}}%
\pgfpathcurveto{\pgfqpoint{2.218495in}{3.019391in}}{\pgfqpoint{2.207896in}{3.015000in}}{\pgfqpoint{2.200083in}{3.007187in}}%
\pgfpathcurveto{\pgfqpoint{2.192269in}{2.999373in}}{\pgfqpoint{2.187879in}{2.988774in}}{\pgfqpoint{2.187879in}{2.977724in}}%
\pgfpathcurveto{\pgfqpoint{2.187879in}{2.966674in}}{\pgfqpoint{2.192269in}{2.956075in}}{\pgfqpoint{2.200083in}{2.948261in}}%
\pgfpathcurveto{\pgfqpoint{2.207896in}{2.940448in}}{\pgfqpoint{2.218495in}{2.936057in}}{\pgfqpoint{2.229545in}{2.936057in}}%
\pgfpathclose%
\pgfusepath{stroke,fill}%
\end{pgfscope}%
\begin{pgfscope}%
\pgfpathrectangle{\pgfqpoint{0.750000in}{0.500000in}}{\pgfqpoint{4.650000in}{3.020000in}}%
\pgfusepath{clip}%
\pgfsetbuttcap%
\pgfsetroundjoin%
\definecolor{currentfill}{rgb}{1.000000,0.498039,0.054902}%
\pgfsetfillcolor{currentfill}%
\pgfsetlinewidth{1.003750pt}%
\definecolor{currentstroke}{rgb}{1.000000,0.498039,0.054902}%
\pgfsetstrokecolor{currentstroke}%
\pgfsetdash{}{0pt}%
\pgfpathmoveto{\pgfqpoint{2.893831in}{2.924375in}}%
\pgfpathcurveto{\pgfqpoint{2.904881in}{2.924375in}}{\pgfqpoint{2.915480in}{2.928765in}}{\pgfqpoint{2.923294in}{2.936578in}}%
\pgfpathcurveto{\pgfqpoint{2.931108in}{2.944392in}}{\pgfqpoint{2.935498in}{2.954991in}}{\pgfqpoint{2.935498in}{2.966041in}}%
\pgfpathcurveto{\pgfqpoint{2.935498in}{2.977091in}}{\pgfqpoint{2.931108in}{2.987690in}}{\pgfqpoint{2.923294in}{2.995504in}}%
\pgfpathcurveto{\pgfqpoint{2.915480in}{3.003318in}}{\pgfqpoint{2.904881in}{3.007708in}}{\pgfqpoint{2.893831in}{3.007708in}}%
\pgfpathcurveto{\pgfqpoint{2.882781in}{3.007708in}}{\pgfqpoint{2.872182in}{3.003318in}}{\pgfqpoint{2.864368in}{2.995504in}}%
\pgfpathcurveto{\pgfqpoint{2.856555in}{2.987690in}}{\pgfqpoint{2.852165in}{2.977091in}}{\pgfqpoint{2.852165in}{2.966041in}}%
\pgfpathcurveto{\pgfqpoint{2.852165in}{2.954991in}}{\pgfqpoint{2.856555in}{2.944392in}}{\pgfqpoint{2.864368in}{2.936578in}}%
\pgfpathcurveto{\pgfqpoint{2.872182in}{2.928765in}}{\pgfqpoint{2.882781in}{2.924375in}}{\pgfqpoint{2.893831in}{2.924375in}}%
\pgfpathclose%
\pgfusepath{stroke,fill}%
\end{pgfscope}%
\begin{pgfscope}%
\pgfpathrectangle{\pgfqpoint{0.750000in}{0.500000in}}{\pgfqpoint{4.650000in}{3.020000in}}%
\pgfusepath{clip}%
\pgfsetbuttcap%
\pgfsetroundjoin%
\definecolor{currentfill}{rgb}{1.000000,0.498039,0.054902}%
\pgfsetfillcolor{currentfill}%
\pgfsetlinewidth{1.003750pt}%
\definecolor{currentstroke}{rgb}{1.000000,0.498039,0.054902}%
\pgfsetstrokecolor{currentstroke}%
\pgfsetdash{}{0pt}%
\pgfpathmoveto{\pgfqpoint{3.618506in}{2.928269in}}%
\pgfpathcurveto{\pgfqpoint{3.629557in}{2.928269in}}{\pgfqpoint{3.640156in}{2.932659in}}{\pgfqpoint{3.647969in}{2.940473in}}%
\pgfpathcurveto{\pgfqpoint{3.655783in}{2.948286in}}{\pgfqpoint{3.660173in}{2.958885in}}{\pgfqpoint{3.660173in}{2.969936in}}%
\pgfpathcurveto{\pgfqpoint{3.660173in}{2.980986in}}{\pgfqpoint{3.655783in}{2.991585in}}{\pgfqpoint{3.647969in}{2.999398in}}%
\pgfpathcurveto{\pgfqpoint{3.640156in}{3.007212in}}{\pgfqpoint{3.629557in}{3.011602in}}{\pgfqpoint{3.618506in}{3.011602in}}%
\pgfpathcurveto{\pgfqpoint{3.607456in}{3.011602in}}{\pgfqpoint{3.596857in}{3.007212in}}{\pgfqpoint{3.589044in}{2.999398in}}%
\pgfpathcurveto{\pgfqpoint{3.581230in}{2.991585in}}{\pgfqpoint{3.576840in}{2.980986in}}{\pgfqpoint{3.576840in}{2.969936in}}%
\pgfpathcurveto{\pgfqpoint{3.576840in}{2.958885in}}{\pgfqpoint{3.581230in}{2.948286in}}{\pgfqpoint{3.589044in}{2.940473in}}%
\pgfpathcurveto{\pgfqpoint{3.596857in}{2.932659in}}{\pgfqpoint{3.607456in}{2.928269in}}{\pgfqpoint{3.618506in}{2.928269in}}%
\pgfpathclose%
\pgfusepath{stroke,fill}%
\end{pgfscope}%
\begin{pgfscope}%
\pgfpathrectangle{\pgfqpoint{0.750000in}{0.500000in}}{\pgfqpoint{4.650000in}{3.020000in}}%
\pgfusepath{clip}%
\pgfsetbuttcap%
\pgfsetroundjoin%
\definecolor{currentfill}{rgb}{1.000000,0.498039,0.054902}%
\pgfsetfillcolor{currentfill}%
\pgfsetlinewidth{1.003750pt}%
\definecolor{currentstroke}{rgb}{1.000000,0.498039,0.054902}%
\pgfsetstrokecolor{currentstroke}%
\pgfsetdash{}{0pt}%
\pgfpathmoveto{\pgfqpoint{1.987987in}{2.939952in}}%
\pgfpathcurveto{\pgfqpoint{1.999037in}{2.939952in}}{\pgfqpoint{2.009636in}{2.944342in}}{\pgfqpoint{2.017450in}{2.952156in}}%
\pgfpathcurveto{\pgfqpoint{2.025263in}{2.959969in}}{\pgfqpoint{2.029654in}{2.970568in}}{\pgfqpoint{2.029654in}{2.981618in}}%
\pgfpathcurveto{\pgfqpoint{2.029654in}{2.992668in}}{\pgfqpoint{2.025263in}{3.003267in}}{\pgfqpoint{2.017450in}{3.011081in}}%
\pgfpathcurveto{\pgfqpoint{2.009636in}{3.018895in}}{\pgfqpoint{1.999037in}{3.023285in}}{\pgfqpoint{1.987987in}{3.023285in}}%
\pgfpathcurveto{\pgfqpoint{1.976937in}{3.023285in}}{\pgfqpoint{1.966338in}{3.018895in}}{\pgfqpoint{1.958524in}{3.011081in}}%
\pgfpathcurveto{\pgfqpoint{1.950711in}{3.003267in}}{\pgfqpoint{1.946320in}{2.992668in}}{\pgfqpoint{1.946320in}{2.981618in}}%
\pgfpathcurveto{\pgfqpoint{1.946320in}{2.970568in}}{\pgfqpoint{1.950711in}{2.959969in}}{\pgfqpoint{1.958524in}{2.952156in}}%
\pgfpathcurveto{\pgfqpoint{1.966338in}{2.944342in}}{\pgfqpoint{1.976937in}{2.939952in}}{\pgfqpoint{1.987987in}{2.939952in}}%
\pgfpathclose%
\pgfusepath{stroke,fill}%
\end{pgfscope}%
\begin{pgfscope}%
\pgfpathrectangle{\pgfqpoint{0.750000in}{0.500000in}}{\pgfqpoint{4.650000in}{3.020000in}}%
\pgfusepath{clip}%
\pgfsetbuttcap%
\pgfsetroundjoin%
\definecolor{currentfill}{rgb}{1.000000,0.498039,0.054902}%
\pgfsetfillcolor{currentfill}%
\pgfsetlinewidth{1.003750pt}%
\definecolor{currentstroke}{rgb}{1.000000,0.498039,0.054902}%
\pgfsetstrokecolor{currentstroke}%
\pgfsetdash{}{0pt}%
\pgfpathmoveto{\pgfqpoint{1.686039in}{2.928269in}}%
\pgfpathcurveto{\pgfqpoint{1.697089in}{2.928269in}}{\pgfqpoint{1.707688in}{2.932659in}}{\pgfqpoint{1.715502in}{2.940473in}}%
\pgfpathcurveto{\pgfqpoint{1.723315in}{2.948286in}}{\pgfqpoint{1.727706in}{2.958885in}}{\pgfqpoint{1.727706in}{2.969936in}}%
\pgfpathcurveto{\pgfqpoint{1.727706in}{2.980986in}}{\pgfqpoint{1.723315in}{2.991585in}}{\pgfqpoint{1.715502in}{2.999398in}}%
\pgfpathcurveto{\pgfqpoint{1.707688in}{3.007212in}}{\pgfqpoint{1.697089in}{3.011602in}}{\pgfqpoint{1.686039in}{3.011602in}}%
\pgfpathcurveto{\pgfqpoint{1.674989in}{3.011602in}}{\pgfqpoint{1.664390in}{3.007212in}}{\pgfqpoint{1.656576in}{2.999398in}}%
\pgfpathcurveto{\pgfqpoint{1.648763in}{2.991585in}}{\pgfqpoint{1.644372in}{2.980986in}}{\pgfqpoint{1.644372in}{2.969936in}}%
\pgfpathcurveto{\pgfqpoint{1.644372in}{2.958885in}}{\pgfqpoint{1.648763in}{2.948286in}}{\pgfqpoint{1.656576in}{2.940473in}}%
\pgfpathcurveto{\pgfqpoint{1.664390in}{2.932659in}}{\pgfqpoint{1.674989in}{2.928269in}}{\pgfqpoint{1.686039in}{2.928269in}}%
\pgfpathclose%
\pgfusepath{stroke,fill}%
\end{pgfscope}%
\begin{pgfscope}%
\pgfpathrectangle{\pgfqpoint{0.750000in}{0.500000in}}{\pgfqpoint{4.650000in}{3.020000in}}%
\pgfusepath{clip}%
\pgfsetbuttcap%
\pgfsetroundjoin%
\definecolor{currentfill}{rgb}{1.000000,0.498039,0.054902}%
\pgfsetfillcolor{currentfill}%
\pgfsetlinewidth{1.003750pt}%
\definecolor{currentstroke}{rgb}{1.000000,0.498039,0.054902}%
\pgfsetstrokecolor{currentstroke}%
\pgfsetdash{}{0pt}%
\pgfpathmoveto{\pgfqpoint{2.169156in}{2.959423in}}%
\pgfpathcurveto{\pgfqpoint{2.180206in}{2.959423in}}{\pgfqpoint{2.190805in}{2.963813in}}{\pgfqpoint{2.198619in}{2.971627in}}%
\pgfpathcurveto{\pgfqpoint{2.206432in}{2.979440in}}{\pgfqpoint{2.210823in}{2.990039in}}{\pgfqpoint{2.210823in}{3.001090in}}%
\pgfpathcurveto{\pgfqpoint{2.210823in}{3.012140in}}{\pgfqpoint{2.206432in}{3.022739in}}{\pgfqpoint{2.198619in}{3.030552in}}%
\pgfpathcurveto{\pgfqpoint{2.190805in}{3.038366in}}{\pgfqpoint{2.180206in}{3.042756in}}{\pgfqpoint{2.169156in}{3.042756in}}%
\pgfpathcurveto{\pgfqpoint{2.158106in}{3.042756in}}{\pgfqpoint{2.147507in}{3.038366in}}{\pgfqpoint{2.139693in}{3.030552in}}%
\pgfpathcurveto{\pgfqpoint{2.131879in}{3.022739in}}{\pgfqpoint{2.127489in}{3.012140in}}{\pgfqpoint{2.127489in}{3.001090in}}%
\pgfpathcurveto{\pgfqpoint{2.127489in}{2.990039in}}{\pgfqpoint{2.131879in}{2.979440in}}{\pgfqpoint{2.139693in}{2.971627in}}%
\pgfpathcurveto{\pgfqpoint{2.147507in}{2.963813in}}{\pgfqpoint{2.158106in}{2.959423in}}{\pgfqpoint{2.169156in}{2.959423in}}%
\pgfpathclose%
\pgfusepath{stroke,fill}%
\end{pgfscope}%
\begin{pgfscope}%
\pgfpathrectangle{\pgfqpoint{0.750000in}{0.500000in}}{\pgfqpoint{4.650000in}{3.020000in}}%
\pgfusepath{clip}%
\pgfsetbuttcap%
\pgfsetroundjoin%
\definecolor{currentfill}{rgb}{1.000000,0.498039,0.054902}%
\pgfsetfillcolor{currentfill}%
\pgfsetlinewidth{1.003750pt}%
\definecolor{currentstroke}{rgb}{1.000000,0.498039,0.054902}%
\pgfsetstrokecolor{currentstroke}%
\pgfsetdash{}{0pt}%
\pgfpathmoveto{\pgfqpoint{2.471104in}{2.939952in}}%
\pgfpathcurveto{\pgfqpoint{2.482154in}{2.939952in}}{\pgfqpoint{2.492753in}{2.944342in}}{\pgfqpoint{2.500567in}{2.952156in}}%
\pgfpathcurveto{\pgfqpoint{2.508380in}{2.959969in}}{\pgfqpoint{2.512771in}{2.970568in}}{\pgfqpoint{2.512771in}{2.981618in}}%
\pgfpathcurveto{\pgfqpoint{2.512771in}{2.992668in}}{\pgfqpoint{2.508380in}{3.003267in}}{\pgfqpoint{2.500567in}{3.011081in}}%
\pgfpathcurveto{\pgfqpoint{2.492753in}{3.018895in}}{\pgfqpoint{2.482154in}{3.023285in}}{\pgfqpoint{2.471104in}{3.023285in}}%
\pgfpathcurveto{\pgfqpoint{2.460054in}{3.023285in}}{\pgfqpoint{2.449455in}{3.018895in}}{\pgfqpoint{2.441641in}{3.011081in}}%
\pgfpathcurveto{\pgfqpoint{2.433827in}{3.003267in}}{\pgfqpoint{2.429437in}{2.992668in}}{\pgfqpoint{2.429437in}{2.981618in}}%
\pgfpathcurveto{\pgfqpoint{2.429437in}{2.970568in}}{\pgfqpoint{2.433827in}{2.959969in}}{\pgfqpoint{2.441641in}{2.952156in}}%
\pgfpathcurveto{\pgfqpoint{2.449455in}{2.944342in}}{\pgfqpoint{2.460054in}{2.939952in}}{\pgfqpoint{2.471104in}{2.939952in}}%
\pgfpathclose%
\pgfusepath{stroke,fill}%
\end{pgfscope}%
\begin{pgfscope}%
\pgfpathrectangle{\pgfqpoint{0.750000in}{0.500000in}}{\pgfqpoint{4.650000in}{3.020000in}}%
\pgfusepath{clip}%
\pgfsetbuttcap%
\pgfsetroundjoin%
\definecolor{currentfill}{rgb}{1.000000,0.498039,0.054902}%
\pgfsetfillcolor{currentfill}%
\pgfsetlinewidth{1.003750pt}%
\definecolor{currentstroke}{rgb}{1.000000,0.498039,0.054902}%
\pgfsetstrokecolor{currentstroke}%
\pgfsetdash{}{0pt}%
\pgfpathmoveto{\pgfqpoint{1.323701in}{2.932163in}}%
\pgfpathcurveto{\pgfqpoint{1.334751in}{2.932163in}}{\pgfqpoint{1.345350in}{2.936553in}}{\pgfqpoint{1.353164in}{2.944367in}}%
\pgfpathcurveto{\pgfqpoint{1.360978in}{2.952181in}}{\pgfqpoint{1.365368in}{2.962780in}}{\pgfqpoint{1.365368in}{2.973830in}}%
\pgfpathcurveto{\pgfqpoint{1.365368in}{2.984880in}}{\pgfqpoint{1.360978in}{2.995479in}}{\pgfqpoint{1.353164in}{3.003293in}}%
\pgfpathcurveto{\pgfqpoint{1.345350in}{3.011106in}}{\pgfqpoint{1.334751in}{3.015496in}}{\pgfqpoint{1.323701in}{3.015496in}}%
\pgfpathcurveto{\pgfqpoint{1.312651in}{3.015496in}}{\pgfqpoint{1.302052in}{3.011106in}}{\pgfqpoint{1.294239in}{3.003293in}}%
\pgfpathcurveto{\pgfqpoint{1.286425in}{2.995479in}}{\pgfqpoint{1.282035in}{2.984880in}}{\pgfqpoint{1.282035in}{2.973830in}}%
\pgfpathcurveto{\pgfqpoint{1.282035in}{2.962780in}}{\pgfqpoint{1.286425in}{2.952181in}}{\pgfqpoint{1.294239in}{2.944367in}}%
\pgfpathcurveto{\pgfqpoint{1.302052in}{2.936553in}}{\pgfqpoint{1.312651in}{2.932163in}}{\pgfqpoint{1.323701in}{2.932163in}}%
\pgfpathclose%
\pgfusepath{stroke,fill}%
\end{pgfscope}%
\begin{pgfscope}%
\pgfpathrectangle{\pgfqpoint{0.750000in}{0.500000in}}{\pgfqpoint{4.650000in}{3.020000in}}%
\pgfusepath{clip}%
\pgfsetbuttcap%
\pgfsetroundjoin%
\definecolor{currentfill}{rgb}{1.000000,0.498039,0.054902}%
\pgfsetfillcolor{currentfill}%
\pgfsetlinewidth{1.003750pt}%
\definecolor{currentstroke}{rgb}{1.000000,0.498039,0.054902}%
\pgfsetstrokecolor{currentstroke}%
\pgfsetdash{}{0pt}%
\pgfpathmoveto{\pgfqpoint{1.504870in}{2.936057in}}%
\pgfpathcurveto{\pgfqpoint{1.515920in}{2.936057in}}{\pgfqpoint{1.526519in}{2.940448in}}{\pgfqpoint{1.534333in}{2.948261in}}%
\pgfpathcurveto{\pgfqpoint{1.542147in}{2.956075in}}{\pgfqpoint{1.546537in}{2.966674in}}{\pgfqpoint{1.546537in}{2.977724in}}%
\pgfpathcurveto{\pgfqpoint{1.546537in}{2.988774in}}{\pgfqpoint{1.542147in}{2.999373in}}{\pgfqpoint{1.534333in}{3.007187in}}%
\pgfpathcurveto{\pgfqpoint{1.526519in}{3.015000in}}{\pgfqpoint{1.515920in}{3.019391in}}{\pgfqpoint{1.504870in}{3.019391in}}%
\pgfpathcurveto{\pgfqpoint{1.493820in}{3.019391in}}{\pgfqpoint{1.483221in}{3.015000in}}{\pgfqpoint{1.475407in}{3.007187in}}%
\pgfpathcurveto{\pgfqpoint{1.467594in}{2.999373in}}{\pgfqpoint{1.463203in}{2.988774in}}{\pgfqpoint{1.463203in}{2.977724in}}%
\pgfpathcurveto{\pgfqpoint{1.463203in}{2.966674in}}{\pgfqpoint{1.467594in}{2.956075in}}{\pgfqpoint{1.475407in}{2.948261in}}%
\pgfpathcurveto{\pgfqpoint{1.483221in}{2.940448in}}{\pgfqpoint{1.493820in}{2.936057in}}{\pgfqpoint{1.504870in}{2.936057in}}%
\pgfpathclose%
\pgfusepath{stroke,fill}%
\end{pgfscope}%
\begin{pgfscope}%
\pgfpathrectangle{\pgfqpoint{0.750000in}{0.500000in}}{\pgfqpoint{4.650000in}{3.020000in}}%
\pgfusepath{clip}%
\pgfsetbuttcap%
\pgfsetroundjoin%
\definecolor{currentfill}{rgb}{0.121569,0.466667,0.705882}%
\pgfsetfillcolor{currentfill}%
\pgfsetlinewidth{1.003750pt}%
\definecolor{currentstroke}{rgb}{0.121569,0.466667,0.705882}%
\pgfsetstrokecolor{currentstroke}%
\pgfsetdash{}{0pt}%
\pgfpathmoveto{\pgfqpoint{1.202922in}{0.603395in}}%
\pgfpathcurveto{\pgfqpoint{1.213972in}{0.603395in}}{\pgfqpoint{1.224571in}{0.607785in}}{\pgfqpoint{1.232385in}{0.615598in}}%
\pgfpathcurveto{\pgfqpoint{1.240198in}{0.623412in}}{\pgfqpoint{1.244589in}{0.634011in}}{\pgfqpoint{1.244589in}{0.645061in}}%
\pgfpathcurveto{\pgfqpoint{1.244589in}{0.656111in}}{\pgfqpoint{1.240198in}{0.666710in}}{\pgfqpoint{1.232385in}{0.674524in}}%
\pgfpathcurveto{\pgfqpoint{1.224571in}{0.682338in}}{\pgfqpoint{1.213972in}{0.686728in}}{\pgfqpoint{1.202922in}{0.686728in}}%
\pgfpathcurveto{\pgfqpoint{1.191872in}{0.686728in}}{\pgfqpoint{1.181273in}{0.682338in}}{\pgfqpoint{1.173459in}{0.674524in}}%
\pgfpathcurveto{\pgfqpoint{1.165646in}{0.666710in}}{\pgfqpoint{1.161255in}{0.656111in}}{\pgfqpoint{1.161255in}{0.645061in}}%
\pgfpathcurveto{\pgfqpoint{1.161255in}{0.634011in}}{\pgfqpoint{1.165646in}{0.623412in}}{\pgfqpoint{1.173459in}{0.615598in}}%
\pgfpathcurveto{\pgfqpoint{1.181273in}{0.607785in}}{\pgfqpoint{1.191872in}{0.603395in}}{\pgfqpoint{1.202922in}{0.603395in}}%
\pgfpathclose%
\pgfusepath{stroke,fill}%
\end{pgfscope}%
\begin{pgfscope}%
\pgfpathrectangle{\pgfqpoint{0.750000in}{0.500000in}}{\pgfqpoint{4.650000in}{3.020000in}}%
\pgfusepath{clip}%
\pgfsetbuttcap%
\pgfsetroundjoin%
\definecolor{currentfill}{rgb}{1.000000,0.498039,0.054902}%
\pgfsetfillcolor{currentfill}%
\pgfsetlinewidth{1.003750pt}%
\definecolor{currentstroke}{rgb}{1.000000,0.498039,0.054902}%
\pgfsetstrokecolor{currentstroke}%
\pgfsetdash{}{0pt}%
\pgfpathmoveto{\pgfqpoint{1.625649in}{2.924375in}}%
\pgfpathcurveto{\pgfqpoint{1.636699in}{2.924375in}}{\pgfqpoint{1.647299in}{2.928765in}}{\pgfqpoint{1.655112in}{2.936578in}}%
\pgfpathcurveto{\pgfqpoint{1.662926in}{2.944392in}}{\pgfqpoint{1.667316in}{2.954991in}}{\pgfqpoint{1.667316in}{2.966041in}}%
\pgfpathcurveto{\pgfqpoint{1.667316in}{2.977091in}}{\pgfqpoint{1.662926in}{2.987690in}}{\pgfqpoint{1.655112in}{2.995504in}}%
\pgfpathcurveto{\pgfqpoint{1.647299in}{3.003318in}}{\pgfqpoint{1.636699in}{3.007708in}}{\pgfqpoint{1.625649in}{3.007708in}}%
\pgfpathcurveto{\pgfqpoint{1.614599in}{3.007708in}}{\pgfqpoint{1.604000in}{3.003318in}}{\pgfqpoint{1.596187in}{2.995504in}}%
\pgfpathcurveto{\pgfqpoint{1.588373in}{2.987690in}}{\pgfqpoint{1.583983in}{2.977091in}}{\pgfqpoint{1.583983in}{2.966041in}}%
\pgfpathcurveto{\pgfqpoint{1.583983in}{2.954991in}}{\pgfqpoint{1.588373in}{2.944392in}}{\pgfqpoint{1.596187in}{2.936578in}}%
\pgfpathcurveto{\pgfqpoint{1.604000in}{2.928765in}}{\pgfqpoint{1.614599in}{2.924375in}}{\pgfqpoint{1.625649in}{2.924375in}}%
\pgfpathclose%
\pgfusepath{stroke,fill}%
\end{pgfscope}%
\begin{pgfscope}%
\pgfpathrectangle{\pgfqpoint{0.750000in}{0.500000in}}{\pgfqpoint{4.650000in}{3.020000in}}%
\pgfusepath{clip}%
\pgfsetbuttcap%
\pgfsetroundjoin%
\definecolor{currentfill}{rgb}{0.121569,0.466667,0.705882}%
\pgfsetfillcolor{currentfill}%
\pgfsetlinewidth{1.003750pt}%
\definecolor{currentstroke}{rgb}{0.121569,0.466667,0.705882}%
\pgfsetstrokecolor{currentstroke}%
\pgfsetdash{}{0pt}%
\pgfpathmoveto{\pgfqpoint{1.202922in}{0.595606in}}%
\pgfpathcurveto{\pgfqpoint{1.213972in}{0.595606in}}{\pgfqpoint{1.224571in}{0.599996in}}{\pgfqpoint{1.232385in}{0.607810in}}%
\pgfpathcurveto{\pgfqpoint{1.240198in}{0.615624in}}{\pgfqpoint{1.244589in}{0.626223in}}{\pgfqpoint{1.244589in}{0.637273in}}%
\pgfpathcurveto{\pgfqpoint{1.244589in}{0.648323in}}{\pgfqpoint{1.240198in}{0.658922in}}{\pgfqpoint{1.232385in}{0.666736in}}%
\pgfpathcurveto{\pgfqpoint{1.224571in}{0.674549in}}{\pgfqpoint{1.213972in}{0.678939in}}{\pgfqpoint{1.202922in}{0.678939in}}%
\pgfpathcurveto{\pgfqpoint{1.191872in}{0.678939in}}{\pgfqpoint{1.181273in}{0.674549in}}{\pgfqpoint{1.173459in}{0.666736in}}%
\pgfpathcurveto{\pgfqpoint{1.165646in}{0.658922in}}{\pgfqpoint{1.161255in}{0.648323in}}{\pgfqpoint{1.161255in}{0.637273in}}%
\pgfpathcurveto{\pgfqpoint{1.161255in}{0.626223in}}{\pgfqpoint{1.165646in}{0.615624in}}{\pgfqpoint{1.173459in}{0.607810in}}%
\pgfpathcurveto{\pgfqpoint{1.181273in}{0.599996in}}{\pgfqpoint{1.191872in}{0.595606in}}{\pgfqpoint{1.202922in}{0.595606in}}%
\pgfpathclose%
\pgfusepath{stroke,fill}%
\end{pgfscope}%
\begin{pgfscope}%
\pgfpathrectangle{\pgfqpoint{0.750000in}{0.500000in}}{\pgfqpoint{4.650000in}{3.020000in}}%
\pgfusepath{clip}%
\pgfsetbuttcap%
\pgfsetroundjoin%
\definecolor{currentfill}{rgb}{1.000000,0.498039,0.054902}%
\pgfsetfillcolor{currentfill}%
\pgfsetlinewidth{1.003750pt}%
\definecolor{currentstroke}{rgb}{1.000000,0.498039,0.054902}%
\pgfsetstrokecolor{currentstroke}%
\pgfsetdash{}{0pt}%
\pgfpathmoveto{\pgfqpoint{2.531494in}{2.410332in}}%
\pgfpathcurveto{\pgfqpoint{2.542544in}{2.410332in}}{\pgfqpoint{2.553143in}{2.414722in}}{\pgfqpoint{2.560956in}{2.422536in}}%
\pgfpathcurveto{\pgfqpoint{2.568770in}{2.430350in}}{\pgfqpoint{2.573160in}{2.440949in}}{\pgfqpoint{2.573160in}{2.451999in}}%
\pgfpathcurveto{\pgfqpoint{2.573160in}{2.463049in}}{\pgfqpoint{2.568770in}{2.473648in}}{\pgfqpoint{2.560956in}{2.481461in}}%
\pgfpathcurveto{\pgfqpoint{2.553143in}{2.489275in}}{\pgfqpoint{2.542544in}{2.493665in}}{\pgfqpoint{2.531494in}{2.493665in}}%
\pgfpathcurveto{\pgfqpoint{2.520443in}{2.493665in}}{\pgfqpoint{2.509844in}{2.489275in}}{\pgfqpoint{2.502031in}{2.481461in}}%
\pgfpathcurveto{\pgfqpoint{2.494217in}{2.473648in}}{\pgfqpoint{2.489827in}{2.463049in}}{\pgfqpoint{2.489827in}{2.451999in}}%
\pgfpathcurveto{\pgfqpoint{2.489827in}{2.440949in}}{\pgfqpoint{2.494217in}{2.430350in}}{\pgfqpoint{2.502031in}{2.422536in}}%
\pgfpathcurveto{\pgfqpoint{2.509844in}{2.414722in}}{\pgfqpoint{2.520443in}{2.410332in}}{\pgfqpoint{2.531494in}{2.410332in}}%
\pgfpathclose%
\pgfusepath{stroke,fill}%
\end{pgfscope}%
\begin{pgfscope}%
\pgfpathrectangle{\pgfqpoint{0.750000in}{0.500000in}}{\pgfqpoint{4.650000in}{3.020000in}}%
\pgfusepath{clip}%
\pgfsetbuttcap%
\pgfsetroundjoin%
\definecolor{currentfill}{rgb}{1.000000,0.498039,0.054902}%
\pgfsetfillcolor{currentfill}%
\pgfsetlinewidth{1.003750pt}%
\definecolor{currentstroke}{rgb}{1.000000,0.498039,0.054902}%
\pgfsetstrokecolor{currentstroke}%
\pgfsetdash{}{0pt}%
\pgfpathmoveto{\pgfqpoint{2.652273in}{2.390861in}}%
\pgfpathcurveto{\pgfqpoint{2.663323in}{2.390861in}}{\pgfqpoint{2.673922in}{2.395251in}}{\pgfqpoint{2.681736in}{2.403065in}}%
\pgfpathcurveto{\pgfqpoint{2.689549in}{2.410878in}}{\pgfqpoint{2.693939in}{2.421477in}}{\pgfqpoint{2.693939in}{2.432527in}}%
\pgfpathcurveto{\pgfqpoint{2.693939in}{2.443578in}}{\pgfqpoint{2.689549in}{2.454177in}}{\pgfqpoint{2.681736in}{2.461990in}}%
\pgfpathcurveto{\pgfqpoint{2.673922in}{2.469804in}}{\pgfqpoint{2.663323in}{2.474194in}}{\pgfqpoint{2.652273in}{2.474194in}}%
\pgfpathcurveto{\pgfqpoint{2.641223in}{2.474194in}}{\pgfqpoint{2.630624in}{2.469804in}}{\pgfqpoint{2.622810in}{2.461990in}}%
\pgfpathcurveto{\pgfqpoint{2.614996in}{2.454177in}}{\pgfqpoint{2.610606in}{2.443578in}}{\pgfqpoint{2.610606in}{2.432527in}}%
\pgfpathcurveto{\pgfqpoint{2.610606in}{2.421477in}}{\pgfqpoint{2.614996in}{2.410878in}}{\pgfqpoint{2.622810in}{2.403065in}}%
\pgfpathcurveto{\pgfqpoint{2.630624in}{2.395251in}}{\pgfqpoint{2.641223in}{2.390861in}}{\pgfqpoint{2.652273in}{2.390861in}}%
\pgfpathclose%
\pgfusepath{stroke,fill}%
\end{pgfscope}%
\begin{pgfscope}%
\pgfpathrectangle{\pgfqpoint{0.750000in}{0.500000in}}{\pgfqpoint{4.650000in}{3.020000in}}%
\pgfusepath{clip}%
\pgfsetbuttcap%
\pgfsetroundjoin%
\definecolor{currentfill}{rgb}{1.000000,0.498039,0.054902}%
\pgfsetfillcolor{currentfill}%
\pgfsetlinewidth{1.003750pt}%
\definecolor{currentstroke}{rgb}{1.000000,0.498039,0.054902}%
\pgfsetstrokecolor{currentstroke}%
\pgfsetdash{}{0pt}%
\pgfpathmoveto{\pgfqpoint{2.048377in}{2.932163in}}%
\pgfpathcurveto{\pgfqpoint{2.059427in}{2.932163in}}{\pgfqpoint{2.070026in}{2.936553in}}{\pgfqpoint{2.077839in}{2.944367in}}%
\pgfpathcurveto{\pgfqpoint{2.085653in}{2.952181in}}{\pgfqpoint{2.090043in}{2.962780in}}{\pgfqpoint{2.090043in}{2.973830in}}%
\pgfpathcurveto{\pgfqpoint{2.090043in}{2.984880in}}{\pgfqpoint{2.085653in}{2.995479in}}{\pgfqpoint{2.077839in}{3.003293in}}%
\pgfpathcurveto{\pgfqpoint{2.070026in}{3.011106in}}{\pgfqpoint{2.059427in}{3.015496in}}{\pgfqpoint{2.048377in}{3.015496in}}%
\pgfpathcurveto{\pgfqpoint{2.037326in}{3.015496in}}{\pgfqpoint{2.026727in}{3.011106in}}{\pgfqpoint{2.018914in}{3.003293in}}%
\pgfpathcurveto{\pgfqpoint{2.011100in}{2.995479in}}{\pgfqpoint{2.006710in}{2.984880in}}{\pgfqpoint{2.006710in}{2.973830in}}%
\pgfpathcurveto{\pgfqpoint{2.006710in}{2.962780in}}{\pgfqpoint{2.011100in}{2.952181in}}{\pgfqpoint{2.018914in}{2.944367in}}%
\pgfpathcurveto{\pgfqpoint{2.026727in}{2.936553in}}{\pgfqpoint{2.037326in}{2.932163in}}{\pgfqpoint{2.048377in}{2.932163in}}%
\pgfpathclose%
\pgfusepath{stroke,fill}%
\end{pgfscope}%
\begin{pgfscope}%
\pgfpathrectangle{\pgfqpoint{0.750000in}{0.500000in}}{\pgfqpoint{4.650000in}{3.020000in}}%
\pgfusepath{clip}%
\pgfsetbuttcap%
\pgfsetroundjoin%
\definecolor{currentfill}{rgb}{1.000000,0.498039,0.054902}%
\pgfsetfillcolor{currentfill}%
\pgfsetlinewidth{1.003750pt}%
\definecolor{currentstroke}{rgb}{1.000000,0.498039,0.054902}%
\pgfsetstrokecolor{currentstroke}%
\pgfsetdash{}{0pt}%
\pgfpathmoveto{\pgfqpoint{2.169156in}{2.951634in}}%
\pgfpathcurveto{\pgfqpoint{2.180206in}{2.951634in}}{\pgfqpoint{2.190805in}{2.956025in}}{\pgfqpoint{2.198619in}{2.963838in}}%
\pgfpathcurveto{\pgfqpoint{2.206432in}{2.971652in}}{\pgfqpoint{2.210823in}{2.982251in}}{\pgfqpoint{2.210823in}{2.993301in}}%
\pgfpathcurveto{\pgfqpoint{2.210823in}{3.004351in}}{\pgfqpoint{2.206432in}{3.014950in}}{\pgfqpoint{2.198619in}{3.022764in}}%
\pgfpathcurveto{\pgfqpoint{2.190805in}{3.030577in}}{\pgfqpoint{2.180206in}{3.034968in}}{\pgfqpoint{2.169156in}{3.034968in}}%
\pgfpathcurveto{\pgfqpoint{2.158106in}{3.034968in}}{\pgfqpoint{2.147507in}{3.030577in}}{\pgfqpoint{2.139693in}{3.022764in}}%
\pgfpathcurveto{\pgfqpoint{2.131879in}{3.014950in}}{\pgfqpoint{2.127489in}{3.004351in}}{\pgfqpoint{2.127489in}{2.993301in}}%
\pgfpathcurveto{\pgfqpoint{2.127489in}{2.982251in}}{\pgfqpoint{2.131879in}{2.971652in}}{\pgfqpoint{2.139693in}{2.963838in}}%
\pgfpathcurveto{\pgfqpoint{2.147507in}{2.956025in}}{\pgfqpoint{2.158106in}{2.951634in}}{\pgfqpoint{2.169156in}{2.951634in}}%
\pgfpathclose%
\pgfusepath{stroke,fill}%
\end{pgfscope}%
\begin{pgfscope}%
\pgfpathrectangle{\pgfqpoint{0.750000in}{0.500000in}}{\pgfqpoint{4.650000in}{3.020000in}}%
\pgfusepath{clip}%
\pgfsetbuttcap%
\pgfsetroundjoin%
\definecolor{currentfill}{rgb}{1.000000,0.498039,0.054902}%
\pgfsetfillcolor{currentfill}%
\pgfsetlinewidth{1.003750pt}%
\definecolor{currentstroke}{rgb}{1.000000,0.498039,0.054902}%
\pgfsetstrokecolor{currentstroke}%
\pgfsetdash{}{0pt}%
\pgfpathmoveto{\pgfqpoint{1.927597in}{2.102685in}}%
\pgfpathcurveto{\pgfqpoint{1.938648in}{2.102685in}}{\pgfqpoint{1.949247in}{2.107076in}}{\pgfqpoint{1.957060in}{2.114889in}}%
\pgfpathcurveto{\pgfqpoint{1.964874in}{2.122703in}}{\pgfqpoint{1.969264in}{2.133302in}}{\pgfqpoint{1.969264in}{2.144352in}}%
\pgfpathcurveto{\pgfqpoint{1.969264in}{2.155402in}}{\pgfqpoint{1.964874in}{2.166001in}}{\pgfqpoint{1.957060in}{2.173815in}}%
\pgfpathcurveto{\pgfqpoint{1.949247in}{2.181628in}}{\pgfqpoint{1.938648in}{2.186019in}}{\pgfqpoint{1.927597in}{2.186019in}}%
\pgfpathcurveto{\pgfqpoint{1.916547in}{2.186019in}}{\pgfqpoint{1.905948in}{2.181628in}}{\pgfqpoint{1.898135in}{2.173815in}}%
\pgfpathcurveto{\pgfqpoint{1.890321in}{2.166001in}}{\pgfqpoint{1.885931in}{2.155402in}}{\pgfqpoint{1.885931in}{2.144352in}}%
\pgfpathcurveto{\pgfqpoint{1.885931in}{2.133302in}}{\pgfqpoint{1.890321in}{2.122703in}}{\pgfqpoint{1.898135in}{2.114889in}}%
\pgfpathcurveto{\pgfqpoint{1.905948in}{2.107076in}}{\pgfqpoint{1.916547in}{2.102685in}}{\pgfqpoint{1.927597in}{2.102685in}}%
\pgfpathclose%
\pgfusepath{stroke,fill}%
\end{pgfscope}%
\begin{pgfscope}%
\pgfpathrectangle{\pgfqpoint{0.750000in}{0.500000in}}{\pgfqpoint{4.650000in}{3.020000in}}%
\pgfusepath{clip}%
\pgfsetbuttcap%
\pgfsetroundjoin%
\definecolor{currentfill}{rgb}{1.000000,0.498039,0.054902}%
\pgfsetfillcolor{currentfill}%
\pgfsetlinewidth{1.003750pt}%
\definecolor{currentstroke}{rgb}{1.000000,0.498039,0.054902}%
\pgfsetstrokecolor{currentstroke}%
\pgfsetdash{}{0pt}%
\pgfpathmoveto{\pgfqpoint{1.867208in}{2.932163in}}%
\pgfpathcurveto{\pgfqpoint{1.878258in}{2.932163in}}{\pgfqpoint{1.888857in}{2.936553in}}{\pgfqpoint{1.896671in}{2.944367in}}%
\pgfpathcurveto{\pgfqpoint{1.904484in}{2.952181in}}{\pgfqpoint{1.908874in}{2.962780in}}{\pgfqpoint{1.908874in}{2.973830in}}%
\pgfpathcurveto{\pgfqpoint{1.908874in}{2.984880in}}{\pgfqpoint{1.904484in}{2.995479in}}{\pgfqpoint{1.896671in}{3.003293in}}%
\pgfpathcurveto{\pgfqpoint{1.888857in}{3.011106in}}{\pgfqpoint{1.878258in}{3.015496in}}{\pgfqpoint{1.867208in}{3.015496in}}%
\pgfpathcurveto{\pgfqpoint{1.856158in}{3.015496in}}{\pgfqpoint{1.845559in}{3.011106in}}{\pgfqpoint{1.837745in}{3.003293in}}%
\pgfpathcurveto{\pgfqpoint{1.829931in}{2.995479in}}{\pgfqpoint{1.825541in}{2.984880in}}{\pgfqpoint{1.825541in}{2.973830in}}%
\pgfpathcurveto{\pgfqpoint{1.825541in}{2.962780in}}{\pgfqpoint{1.829931in}{2.952181in}}{\pgfqpoint{1.837745in}{2.944367in}}%
\pgfpathcurveto{\pgfqpoint{1.845559in}{2.936553in}}{\pgfqpoint{1.856158in}{2.932163in}}{\pgfqpoint{1.867208in}{2.932163in}}%
\pgfpathclose%
\pgfusepath{stroke,fill}%
\end{pgfscope}%
\begin{pgfscope}%
\pgfpathrectangle{\pgfqpoint{0.750000in}{0.500000in}}{\pgfqpoint{4.650000in}{3.020000in}}%
\pgfusepath{clip}%
\pgfsetbuttcap%
\pgfsetroundjoin%
\definecolor{currentfill}{rgb}{1.000000,0.498039,0.054902}%
\pgfsetfillcolor{currentfill}%
\pgfsetlinewidth{1.003750pt}%
\definecolor{currentstroke}{rgb}{1.000000,0.498039,0.054902}%
\pgfsetstrokecolor{currentstroke}%
\pgfsetdash{}{0pt}%
\pgfpathmoveto{\pgfqpoint{4.403571in}{2.932163in}}%
\pgfpathcurveto{\pgfqpoint{4.414622in}{2.932163in}}{\pgfqpoint{4.425221in}{2.936553in}}{\pgfqpoint{4.433034in}{2.944367in}}%
\pgfpathcurveto{\pgfqpoint{4.440848in}{2.952181in}}{\pgfqpoint{4.445238in}{2.962780in}}{\pgfqpoint{4.445238in}{2.973830in}}%
\pgfpathcurveto{\pgfqpoint{4.445238in}{2.984880in}}{\pgfqpoint{4.440848in}{2.995479in}}{\pgfqpoint{4.433034in}{3.003293in}}%
\pgfpathcurveto{\pgfqpoint{4.425221in}{3.011106in}}{\pgfqpoint{4.414622in}{3.015496in}}{\pgfqpoint{4.403571in}{3.015496in}}%
\pgfpathcurveto{\pgfqpoint{4.392521in}{3.015496in}}{\pgfqpoint{4.381922in}{3.011106in}}{\pgfqpoint{4.374109in}{3.003293in}}%
\pgfpathcurveto{\pgfqpoint{4.366295in}{2.995479in}}{\pgfqpoint{4.361905in}{2.984880in}}{\pgfqpoint{4.361905in}{2.973830in}}%
\pgfpathcurveto{\pgfqpoint{4.361905in}{2.962780in}}{\pgfqpoint{4.366295in}{2.952181in}}{\pgfqpoint{4.374109in}{2.944367in}}%
\pgfpathcurveto{\pgfqpoint{4.381922in}{2.936553in}}{\pgfqpoint{4.392521in}{2.932163in}}{\pgfqpoint{4.403571in}{2.932163in}}%
\pgfpathclose%
\pgfusepath{stroke,fill}%
\end{pgfscope}%
\begin{pgfscope}%
\pgfpathrectangle{\pgfqpoint{0.750000in}{0.500000in}}{\pgfqpoint{4.650000in}{3.020000in}}%
\pgfusepath{clip}%
\pgfsetbuttcap%
\pgfsetroundjoin%
\definecolor{currentfill}{rgb}{1.000000,0.498039,0.054902}%
\pgfsetfillcolor{currentfill}%
\pgfsetlinewidth{1.003750pt}%
\definecolor{currentstroke}{rgb}{1.000000,0.498039,0.054902}%
\pgfsetstrokecolor{currentstroke}%
\pgfsetdash{}{0pt}%
\pgfpathmoveto{\pgfqpoint{1.504870in}{2.932163in}}%
\pgfpathcurveto{\pgfqpoint{1.515920in}{2.932163in}}{\pgfqpoint{1.526519in}{2.936553in}}{\pgfqpoint{1.534333in}{2.944367in}}%
\pgfpathcurveto{\pgfqpoint{1.542147in}{2.952181in}}{\pgfqpoint{1.546537in}{2.962780in}}{\pgfqpoint{1.546537in}{2.973830in}}%
\pgfpathcurveto{\pgfqpoint{1.546537in}{2.984880in}}{\pgfqpoint{1.542147in}{2.995479in}}{\pgfqpoint{1.534333in}{3.003293in}}%
\pgfpathcurveto{\pgfqpoint{1.526519in}{3.011106in}}{\pgfqpoint{1.515920in}{3.015496in}}{\pgfqpoint{1.504870in}{3.015496in}}%
\pgfpathcurveto{\pgfqpoint{1.493820in}{3.015496in}}{\pgfqpoint{1.483221in}{3.011106in}}{\pgfqpoint{1.475407in}{3.003293in}}%
\pgfpathcurveto{\pgfqpoint{1.467594in}{2.995479in}}{\pgfqpoint{1.463203in}{2.984880in}}{\pgfqpoint{1.463203in}{2.973830in}}%
\pgfpathcurveto{\pgfqpoint{1.463203in}{2.962780in}}{\pgfqpoint{1.467594in}{2.952181in}}{\pgfqpoint{1.475407in}{2.944367in}}%
\pgfpathcurveto{\pgfqpoint{1.483221in}{2.936553in}}{\pgfqpoint{1.493820in}{2.932163in}}{\pgfqpoint{1.504870in}{2.932163in}}%
\pgfpathclose%
\pgfusepath{stroke,fill}%
\end{pgfscope}%
\begin{pgfscope}%
\pgfpathrectangle{\pgfqpoint{0.750000in}{0.500000in}}{\pgfqpoint{4.650000in}{3.020000in}}%
\pgfusepath{clip}%
\pgfsetbuttcap%
\pgfsetroundjoin%
\definecolor{currentfill}{rgb}{0.839216,0.152941,0.156863}%
\pgfsetfillcolor{currentfill}%
\pgfsetlinewidth{1.003750pt}%
\definecolor{currentstroke}{rgb}{0.839216,0.152941,0.156863}%
\pgfsetstrokecolor{currentstroke}%
\pgfsetdash{}{0pt}%
\pgfpathmoveto{\pgfqpoint{1.384091in}{2.932163in}}%
\pgfpathcurveto{\pgfqpoint{1.395141in}{2.932163in}}{\pgfqpoint{1.405740in}{2.936553in}}{\pgfqpoint{1.413554in}{2.944367in}}%
\pgfpathcurveto{\pgfqpoint{1.421367in}{2.952181in}}{\pgfqpoint{1.425758in}{2.962780in}}{\pgfqpoint{1.425758in}{2.973830in}}%
\pgfpathcurveto{\pgfqpoint{1.425758in}{2.984880in}}{\pgfqpoint{1.421367in}{2.995479in}}{\pgfqpoint{1.413554in}{3.003293in}}%
\pgfpathcurveto{\pgfqpoint{1.405740in}{3.011106in}}{\pgfqpoint{1.395141in}{3.015496in}}{\pgfqpoint{1.384091in}{3.015496in}}%
\pgfpathcurveto{\pgfqpoint{1.373041in}{3.015496in}}{\pgfqpoint{1.362442in}{3.011106in}}{\pgfqpoint{1.354628in}{3.003293in}}%
\pgfpathcurveto{\pgfqpoint{1.346815in}{2.995479in}}{\pgfqpoint{1.342424in}{2.984880in}}{\pgfqpoint{1.342424in}{2.973830in}}%
\pgfpathcurveto{\pgfqpoint{1.342424in}{2.962780in}}{\pgfqpoint{1.346815in}{2.952181in}}{\pgfqpoint{1.354628in}{2.944367in}}%
\pgfpathcurveto{\pgfqpoint{1.362442in}{2.936553in}}{\pgfqpoint{1.373041in}{2.932163in}}{\pgfqpoint{1.384091in}{2.932163in}}%
\pgfpathclose%
\pgfusepath{stroke,fill}%
\end{pgfscope}%
\begin{pgfscope}%
\pgfpathrectangle{\pgfqpoint{0.750000in}{0.500000in}}{\pgfqpoint{4.650000in}{3.020000in}}%
\pgfusepath{clip}%
\pgfsetbuttcap%
\pgfsetroundjoin%
\definecolor{currentfill}{rgb}{1.000000,0.498039,0.054902}%
\pgfsetfillcolor{currentfill}%
\pgfsetlinewidth{1.003750pt}%
\definecolor{currentstroke}{rgb}{1.000000,0.498039,0.054902}%
\pgfsetstrokecolor{currentstroke}%
\pgfsetdash{}{0pt}%
\pgfpathmoveto{\pgfqpoint{1.323701in}{2.932163in}}%
\pgfpathcurveto{\pgfqpoint{1.334751in}{2.932163in}}{\pgfqpoint{1.345350in}{2.936553in}}{\pgfqpoint{1.353164in}{2.944367in}}%
\pgfpathcurveto{\pgfqpoint{1.360978in}{2.952181in}}{\pgfqpoint{1.365368in}{2.962780in}}{\pgfqpoint{1.365368in}{2.973830in}}%
\pgfpathcurveto{\pgfqpoint{1.365368in}{2.984880in}}{\pgfqpoint{1.360978in}{2.995479in}}{\pgfqpoint{1.353164in}{3.003293in}}%
\pgfpathcurveto{\pgfqpoint{1.345350in}{3.011106in}}{\pgfqpoint{1.334751in}{3.015496in}}{\pgfqpoint{1.323701in}{3.015496in}}%
\pgfpathcurveto{\pgfqpoint{1.312651in}{3.015496in}}{\pgfqpoint{1.302052in}{3.011106in}}{\pgfqpoint{1.294239in}{3.003293in}}%
\pgfpathcurveto{\pgfqpoint{1.286425in}{2.995479in}}{\pgfqpoint{1.282035in}{2.984880in}}{\pgfqpoint{1.282035in}{2.973830in}}%
\pgfpathcurveto{\pgfqpoint{1.282035in}{2.962780in}}{\pgfqpoint{1.286425in}{2.952181in}}{\pgfqpoint{1.294239in}{2.944367in}}%
\pgfpathcurveto{\pgfqpoint{1.302052in}{2.936553in}}{\pgfqpoint{1.312651in}{2.932163in}}{\pgfqpoint{1.323701in}{2.932163in}}%
\pgfpathclose%
\pgfusepath{stroke,fill}%
\end{pgfscope}%
\begin{pgfscope}%
\pgfpathrectangle{\pgfqpoint{0.750000in}{0.500000in}}{\pgfqpoint{4.650000in}{3.020000in}}%
\pgfusepath{clip}%
\pgfsetbuttcap%
\pgfsetroundjoin%
\definecolor{currentfill}{rgb}{0.121569,0.466667,0.705882}%
\pgfsetfillcolor{currentfill}%
\pgfsetlinewidth{1.003750pt}%
\definecolor{currentstroke}{rgb}{0.121569,0.466667,0.705882}%
\pgfsetstrokecolor{currentstroke}%
\pgfsetdash{}{0pt}%
\pgfpathmoveto{\pgfqpoint{1.323701in}{1.452344in}}%
\pgfpathcurveto{\pgfqpoint{1.334751in}{1.452344in}}{\pgfqpoint{1.345350in}{1.456734in}}{\pgfqpoint{1.353164in}{1.464548in}}%
\pgfpathcurveto{\pgfqpoint{1.360978in}{1.472361in}}{\pgfqpoint{1.365368in}{1.482960in}}{\pgfqpoint{1.365368in}{1.494010in}}%
\pgfpathcurveto{\pgfqpoint{1.365368in}{1.505060in}}{\pgfqpoint{1.360978in}{1.515659in}}{\pgfqpoint{1.353164in}{1.523473in}}%
\pgfpathcurveto{\pgfqpoint{1.345350in}{1.531287in}}{\pgfqpoint{1.334751in}{1.535677in}}{\pgfqpoint{1.323701in}{1.535677in}}%
\pgfpathcurveto{\pgfqpoint{1.312651in}{1.535677in}}{\pgfqpoint{1.302052in}{1.531287in}}{\pgfqpoint{1.294239in}{1.523473in}}%
\pgfpathcurveto{\pgfqpoint{1.286425in}{1.515659in}}{\pgfqpoint{1.282035in}{1.505060in}}{\pgfqpoint{1.282035in}{1.494010in}}%
\pgfpathcurveto{\pgfqpoint{1.282035in}{1.482960in}}{\pgfqpoint{1.286425in}{1.472361in}}{\pgfqpoint{1.294239in}{1.464548in}}%
\pgfpathcurveto{\pgfqpoint{1.302052in}{1.456734in}}{\pgfqpoint{1.312651in}{1.452344in}}{\pgfqpoint{1.323701in}{1.452344in}}%
\pgfpathclose%
\pgfusepath{stroke,fill}%
\end{pgfscope}%
\begin{pgfscope}%
\pgfpathrectangle{\pgfqpoint{0.750000in}{0.500000in}}{\pgfqpoint{4.650000in}{3.020000in}}%
\pgfusepath{clip}%
\pgfsetbuttcap%
\pgfsetroundjoin%
\definecolor{currentfill}{rgb}{1.000000,0.498039,0.054902}%
\pgfsetfillcolor{currentfill}%
\pgfsetlinewidth{1.003750pt}%
\definecolor{currentstroke}{rgb}{1.000000,0.498039,0.054902}%
\pgfsetstrokecolor{currentstroke}%
\pgfsetdash{}{0pt}%
\pgfpathmoveto{\pgfqpoint{1.444481in}{2.936057in}}%
\pgfpathcurveto{\pgfqpoint{1.455531in}{2.936057in}}{\pgfqpoint{1.466130in}{2.940448in}}{\pgfqpoint{1.473943in}{2.948261in}}%
\pgfpathcurveto{\pgfqpoint{1.481757in}{2.956075in}}{\pgfqpoint{1.486147in}{2.966674in}}{\pgfqpoint{1.486147in}{2.977724in}}%
\pgfpathcurveto{\pgfqpoint{1.486147in}{2.988774in}}{\pgfqpoint{1.481757in}{2.999373in}}{\pgfqpoint{1.473943in}{3.007187in}}%
\pgfpathcurveto{\pgfqpoint{1.466130in}{3.015000in}}{\pgfqpoint{1.455531in}{3.019391in}}{\pgfqpoint{1.444481in}{3.019391in}}%
\pgfpathcurveto{\pgfqpoint{1.433430in}{3.019391in}}{\pgfqpoint{1.422831in}{3.015000in}}{\pgfqpoint{1.415018in}{3.007187in}}%
\pgfpathcurveto{\pgfqpoint{1.407204in}{2.999373in}}{\pgfqpoint{1.402814in}{2.988774in}}{\pgfqpoint{1.402814in}{2.977724in}}%
\pgfpathcurveto{\pgfqpoint{1.402814in}{2.966674in}}{\pgfqpoint{1.407204in}{2.956075in}}{\pgfqpoint{1.415018in}{2.948261in}}%
\pgfpathcurveto{\pgfqpoint{1.422831in}{2.940448in}}{\pgfqpoint{1.433430in}{2.936057in}}{\pgfqpoint{1.444481in}{2.936057in}}%
\pgfpathclose%
\pgfusepath{stroke,fill}%
\end{pgfscope}%
\begin{pgfscope}%
\pgfpathrectangle{\pgfqpoint{0.750000in}{0.500000in}}{\pgfqpoint{4.650000in}{3.020000in}}%
\pgfusepath{clip}%
\pgfsetbuttcap%
\pgfsetroundjoin%
\definecolor{currentfill}{rgb}{1.000000,0.498039,0.054902}%
\pgfsetfillcolor{currentfill}%
\pgfsetlinewidth{1.003750pt}%
\definecolor{currentstroke}{rgb}{1.000000,0.498039,0.054902}%
\pgfsetstrokecolor{currentstroke}%
\pgfsetdash{}{0pt}%
\pgfpathmoveto{\pgfqpoint{1.444481in}{2.928269in}}%
\pgfpathcurveto{\pgfqpoint{1.455531in}{2.928269in}}{\pgfqpoint{1.466130in}{2.932659in}}{\pgfqpoint{1.473943in}{2.940473in}}%
\pgfpathcurveto{\pgfqpoint{1.481757in}{2.948286in}}{\pgfqpoint{1.486147in}{2.958885in}}{\pgfqpoint{1.486147in}{2.969936in}}%
\pgfpathcurveto{\pgfqpoint{1.486147in}{2.980986in}}{\pgfqpoint{1.481757in}{2.991585in}}{\pgfqpoint{1.473943in}{2.999398in}}%
\pgfpathcurveto{\pgfqpoint{1.466130in}{3.007212in}}{\pgfqpoint{1.455531in}{3.011602in}}{\pgfqpoint{1.444481in}{3.011602in}}%
\pgfpathcurveto{\pgfqpoint{1.433430in}{3.011602in}}{\pgfqpoint{1.422831in}{3.007212in}}{\pgfqpoint{1.415018in}{2.999398in}}%
\pgfpathcurveto{\pgfqpoint{1.407204in}{2.991585in}}{\pgfqpoint{1.402814in}{2.980986in}}{\pgfqpoint{1.402814in}{2.969936in}}%
\pgfpathcurveto{\pgfqpoint{1.402814in}{2.958885in}}{\pgfqpoint{1.407204in}{2.948286in}}{\pgfqpoint{1.415018in}{2.940473in}}%
\pgfpathcurveto{\pgfqpoint{1.422831in}{2.932659in}}{\pgfqpoint{1.433430in}{2.928269in}}{\pgfqpoint{1.444481in}{2.928269in}}%
\pgfpathclose%
\pgfusepath{stroke,fill}%
\end{pgfscope}%
\begin{pgfscope}%
\pgfpathrectangle{\pgfqpoint{0.750000in}{0.500000in}}{\pgfqpoint{4.650000in}{3.020000in}}%
\pgfusepath{clip}%
\pgfsetbuttcap%
\pgfsetroundjoin%
\definecolor{currentfill}{rgb}{1.000000,0.498039,0.054902}%
\pgfsetfillcolor{currentfill}%
\pgfsetlinewidth{1.003750pt}%
\definecolor{currentstroke}{rgb}{1.000000,0.498039,0.054902}%
\pgfsetstrokecolor{currentstroke}%
\pgfsetdash{}{0pt}%
\pgfpathmoveto{\pgfqpoint{2.289935in}{2.932163in}}%
\pgfpathcurveto{\pgfqpoint{2.300985in}{2.932163in}}{\pgfqpoint{2.311584in}{2.936553in}}{\pgfqpoint{2.319398in}{2.944367in}}%
\pgfpathcurveto{\pgfqpoint{2.327211in}{2.952181in}}{\pgfqpoint{2.331602in}{2.962780in}}{\pgfqpoint{2.331602in}{2.973830in}}%
\pgfpathcurveto{\pgfqpoint{2.331602in}{2.984880in}}{\pgfqpoint{2.327211in}{2.995479in}}{\pgfqpoint{2.319398in}{3.003293in}}%
\pgfpathcurveto{\pgfqpoint{2.311584in}{3.011106in}}{\pgfqpoint{2.300985in}{3.015496in}}{\pgfqpoint{2.289935in}{3.015496in}}%
\pgfpathcurveto{\pgfqpoint{2.278885in}{3.015496in}}{\pgfqpoint{2.268286in}{3.011106in}}{\pgfqpoint{2.260472in}{3.003293in}}%
\pgfpathcurveto{\pgfqpoint{2.252659in}{2.995479in}}{\pgfqpoint{2.248268in}{2.984880in}}{\pgfqpoint{2.248268in}{2.973830in}}%
\pgfpathcurveto{\pgfqpoint{2.248268in}{2.962780in}}{\pgfqpoint{2.252659in}{2.952181in}}{\pgfqpoint{2.260472in}{2.944367in}}%
\pgfpathcurveto{\pgfqpoint{2.268286in}{2.936553in}}{\pgfqpoint{2.278885in}{2.932163in}}{\pgfqpoint{2.289935in}{2.932163in}}%
\pgfpathclose%
\pgfusepath{stroke,fill}%
\end{pgfscope}%
\begin{pgfscope}%
\pgfpathrectangle{\pgfqpoint{0.750000in}{0.500000in}}{\pgfqpoint{4.650000in}{3.020000in}}%
\pgfusepath{clip}%
\pgfsetbuttcap%
\pgfsetroundjoin%
\definecolor{currentfill}{rgb}{1.000000,0.498039,0.054902}%
\pgfsetfillcolor{currentfill}%
\pgfsetlinewidth{1.003750pt}%
\definecolor{currentstroke}{rgb}{1.000000,0.498039,0.054902}%
\pgfsetstrokecolor{currentstroke}%
\pgfsetdash{}{0pt}%
\pgfpathmoveto{\pgfqpoint{2.229545in}{2.939952in}}%
\pgfpathcurveto{\pgfqpoint{2.240596in}{2.939952in}}{\pgfqpoint{2.251195in}{2.944342in}}{\pgfqpoint{2.259008in}{2.952156in}}%
\pgfpathcurveto{\pgfqpoint{2.266822in}{2.959969in}}{\pgfqpoint{2.271212in}{2.970568in}}{\pgfqpoint{2.271212in}{2.981618in}}%
\pgfpathcurveto{\pgfqpoint{2.271212in}{2.992668in}}{\pgfqpoint{2.266822in}{3.003267in}}{\pgfqpoint{2.259008in}{3.011081in}}%
\pgfpathcurveto{\pgfqpoint{2.251195in}{3.018895in}}{\pgfqpoint{2.240596in}{3.023285in}}{\pgfqpoint{2.229545in}{3.023285in}}%
\pgfpathcurveto{\pgfqpoint{2.218495in}{3.023285in}}{\pgfqpoint{2.207896in}{3.018895in}}{\pgfqpoint{2.200083in}{3.011081in}}%
\pgfpathcurveto{\pgfqpoint{2.192269in}{3.003267in}}{\pgfqpoint{2.187879in}{2.992668in}}{\pgfqpoint{2.187879in}{2.981618in}}%
\pgfpathcurveto{\pgfqpoint{2.187879in}{2.970568in}}{\pgfqpoint{2.192269in}{2.959969in}}{\pgfqpoint{2.200083in}{2.952156in}}%
\pgfpathcurveto{\pgfqpoint{2.207896in}{2.944342in}}{\pgfqpoint{2.218495in}{2.939952in}}{\pgfqpoint{2.229545in}{2.939952in}}%
\pgfpathclose%
\pgfusepath{stroke,fill}%
\end{pgfscope}%
\begin{pgfscope}%
\pgfpathrectangle{\pgfqpoint{0.750000in}{0.500000in}}{\pgfqpoint{4.650000in}{3.020000in}}%
\pgfusepath{clip}%
\pgfsetbuttcap%
\pgfsetroundjoin%
\definecolor{currentfill}{rgb}{1.000000,0.498039,0.054902}%
\pgfsetfillcolor{currentfill}%
\pgfsetlinewidth{1.003750pt}%
\definecolor{currentstroke}{rgb}{1.000000,0.498039,0.054902}%
\pgfsetstrokecolor{currentstroke}%
\pgfsetdash{}{0pt}%
\pgfpathmoveto{\pgfqpoint{2.108766in}{2.928269in}}%
\pgfpathcurveto{\pgfqpoint{2.119816in}{2.928269in}}{\pgfqpoint{2.130415in}{2.932659in}}{\pgfqpoint{2.138229in}{2.940473in}}%
\pgfpathcurveto{\pgfqpoint{2.146043in}{2.948286in}}{\pgfqpoint{2.150433in}{2.958885in}}{\pgfqpoint{2.150433in}{2.969936in}}%
\pgfpathcurveto{\pgfqpoint{2.150433in}{2.980986in}}{\pgfqpoint{2.146043in}{2.991585in}}{\pgfqpoint{2.138229in}{2.999398in}}%
\pgfpathcurveto{\pgfqpoint{2.130415in}{3.007212in}}{\pgfqpoint{2.119816in}{3.011602in}}{\pgfqpoint{2.108766in}{3.011602in}}%
\pgfpathcurveto{\pgfqpoint{2.097716in}{3.011602in}}{\pgfqpoint{2.087117in}{3.007212in}}{\pgfqpoint{2.079303in}{2.999398in}}%
\pgfpathcurveto{\pgfqpoint{2.071490in}{2.991585in}}{\pgfqpoint{2.067100in}{2.980986in}}{\pgfqpoint{2.067100in}{2.969936in}}%
\pgfpathcurveto{\pgfqpoint{2.067100in}{2.958885in}}{\pgfqpoint{2.071490in}{2.948286in}}{\pgfqpoint{2.079303in}{2.940473in}}%
\pgfpathcurveto{\pgfqpoint{2.087117in}{2.932659in}}{\pgfqpoint{2.097716in}{2.928269in}}{\pgfqpoint{2.108766in}{2.928269in}}%
\pgfpathclose%
\pgfusepath{stroke,fill}%
\end{pgfscope}%
\begin{pgfscope}%
\pgfpathrectangle{\pgfqpoint{0.750000in}{0.500000in}}{\pgfqpoint{4.650000in}{3.020000in}}%
\pgfusepath{clip}%
\pgfsetbuttcap%
\pgfsetroundjoin%
\definecolor{currentfill}{rgb}{1.000000,0.498039,0.054902}%
\pgfsetfillcolor{currentfill}%
\pgfsetlinewidth{1.003750pt}%
\definecolor{currentstroke}{rgb}{1.000000,0.498039,0.054902}%
\pgfsetstrokecolor{currentstroke}%
\pgfsetdash{}{0pt}%
\pgfpathmoveto{\pgfqpoint{1.384091in}{2.932163in}}%
\pgfpathcurveto{\pgfqpoint{1.395141in}{2.932163in}}{\pgfqpoint{1.405740in}{2.936553in}}{\pgfqpoint{1.413554in}{2.944367in}}%
\pgfpathcurveto{\pgfqpoint{1.421367in}{2.952181in}}{\pgfqpoint{1.425758in}{2.962780in}}{\pgfqpoint{1.425758in}{2.973830in}}%
\pgfpathcurveto{\pgfqpoint{1.425758in}{2.984880in}}{\pgfqpoint{1.421367in}{2.995479in}}{\pgfqpoint{1.413554in}{3.003293in}}%
\pgfpathcurveto{\pgfqpoint{1.405740in}{3.011106in}}{\pgfqpoint{1.395141in}{3.015496in}}{\pgfqpoint{1.384091in}{3.015496in}}%
\pgfpathcurveto{\pgfqpoint{1.373041in}{3.015496in}}{\pgfqpoint{1.362442in}{3.011106in}}{\pgfqpoint{1.354628in}{3.003293in}}%
\pgfpathcurveto{\pgfqpoint{1.346815in}{2.995479in}}{\pgfqpoint{1.342424in}{2.984880in}}{\pgfqpoint{1.342424in}{2.973830in}}%
\pgfpathcurveto{\pgfqpoint{1.342424in}{2.962780in}}{\pgfqpoint{1.346815in}{2.952181in}}{\pgfqpoint{1.354628in}{2.944367in}}%
\pgfpathcurveto{\pgfqpoint{1.362442in}{2.936553in}}{\pgfqpoint{1.373041in}{2.932163in}}{\pgfqpoint{1.384091in}{2.932163in}}%
\pgfpathclose%
\pgfusepath{stroke,fill}%
\end{pgfscope}%
\begin{pgfscope}%
\pgfpathrectangle{\pgfqpoint{0.750000in}{0.500000in}}{\pgfqpoint{4.650000in}{3.020000in}}%
\pgfusepath{clip}%
\pgfsetbuttcap%
\pgfsetroundjoin%
\definecolor{currentfill}{rgb}{1.000000,0.498039,0.054902}%
\pgfsetfillcolor{currentfill}%
\pgfsetlinewidth{1.003750pt}%
\definecolor{currentstroke}{rgb}{1.000000,0.498039,0.054902}%
\pgfsetstrokecolor{currentstroke}%
\pgfsetdash{}{0pt}%
\pgfpathmoveto{\pgfqpoint{1.625649in}{2.608939in}}%
\pgfpathcurveto{\pgfqpoint{1.636699in}{2.608939in}}{\pgfqpoint{1.647299in}{2.613330in}}{\pgfqpoint{1.655112in}{2.621143in}}%
\pgfpathcurveto{\pgfqpoint{1.662926in}{2.628957in}}{\pgfqpoint{1.667316in}{2.639556in}}{\pgfqpoint{1.667316in}{2.650606in}}%
\pgfpathcurveto{\pgfqpoint{1.667316in}{2.661656in}}{\pgfqpoint{1.662926in}{2.672255in}}{\pgfqpoint{1.655112in}{2.680069in}}%
\pgfpathcurveto{\pgfqpoint{1.647299in}{2.687882in}}{\pgfqpoint{1.636699in}{2.692273in}}{\pgfqpoint{1.625649in}{2.692273in}}%
\pgfpathcurveto{\pgfqpoint{1.614599in}{2.692273in}}{\pgfqpoint{1.604000in}{2.687882in}}{\pgfqpoint{1.596187in}{2.680069in}}%
\pgfpathcurveto{\pgfqpoint{1.588373in}{2.672255in}}{\pgfqpoint{1.583983in}{2.661656in}}{\pgfqpoint{1.583983in}{2.650606in}}%
\pgfpathcurveto{\pgfqpoint{1.583983in}{2.639556in}}{\pgfqpoint{1.588373in}{2.628957in}}{\pgfqpoint{1.596187in}{2.621143in}}%
\pgfpathcurveto{\pgfqpoint{1.604000in}{2.613330in}}{\pgfqpoint{1.614599in}{2.608939in}}{\pgfqpoint{1.625649in}{2.608939in}}%
\pgfpathclose%
\pgfusepath{stroke,fill}%
\end{pgfscope}%
\begin{pgfscope}%
\pgfpathrectangle{\pgfqpoint{0.750000in}{0.500000in}}{\pgfqpoint{4.650000in}{3.020000in}}%
\pgfusepath{clip}%
\pgfsetbuttcap%
\pgfsetroundjoin%
\definecolor{currentfill}{rgb}{1.000000,0.498039,0.054902}%
\pgfsetfillcolor{currentfill}%
\pgfsetlinewidth{1.003750pt}%
\definecolor{currentstroke}{rgb}{1.000000,0.498039,0.054902}%
\pgfsetstrokecolor{currentstroke}%
\pgfsetdash{}{0pt}%
\pgfpathmoveto{\pgfqpoint{3.376948in}{2.912692in}}%
\pgfpathcurveto{\pgfqpoint{3.387998in}{2.912692in}}{\pgfqpoint{3.398597in}{2.917082in}}{\pgfqpoint{3.406411in}{2.924896in}}%
\pgfpathcurveto{\pgfqpoint{3.414224in}{2.932709in}}{\pgfqpoint{3.418615in}{2.943308in}}{\pgfqpoint{3.418615in}{2.954358in}}%
\pgfpathcurveto{\pgfqpoint{3.418615in}{2.965409in}}{\pgfqpoint{3.414224in}{2.976008in}}{\pgfqpoint{3.406411in}{2.983821in}}%
\pgfpathcurveto{\pgfqpoint{3.398597in}{2.991635in}}{\pgfqpoint{3.387998in}{2.996025in}}{\pgfqpoint{3.376948in}{2.996025in}}%
\pgfpathcurveto{\pgfqpoint{3.365898in}{2.996025in}}{\pgfqpoint{3.355299in}{2.991635in}}{\pgfqpoint{3.347485in}{2.983821in}}%
\pgfpathcurveto{\pgfqpoint{3.339672in}{2.976008in}}{\pgfqpoint{3.335281in}{2.965409in}}{\pgfqpoint{3.335281in}{2.954358in}}%
\pgfpathcurveto{\pgfqpoint{3.335281in}{2.943308in}}{\pgfqpoint{3.339672in}{2.932709in}}{\pgfqpoint{3.347485in}{2.924896in}}%
\pgfpathcurveto{\pgfqpoint{3.355299in}{2.917082in}}{\pgfqpoint{3.365898in}{2.912692in}}{\pgfqpoint{3.376948in}{2.912692in}}%
\pgfpathclose%
\pgfusepath{stroke,fill}%
\end{pgfscope}%
\begin{pgfscope}%
\pgfpathrectangle{\pgfqpoint{0.750000in}{0.500000in}}{\pgfqpoint{4.650000in}{3.020000in}}%
\pgfusepath{clip}%
\pgfsetbuttcap%
\pgfsetroundjoin%
\definecolor{currentfill}{rgb}{1.000000,0.498039,0.054902}%
\pgfsetfillcolor{currentfill}%
\pgfsetlinewidth{1.003750pt}%
\definecolor{currentstroke}{rgb}{1.000000,0.498039,0.054902}%
\pgfsetstrokecolor{currentstroke}%
\pgfsetdash{}{0pt}%
\pgfpathmoveto{\pgfqpoint{1.504870in}{2.815335in}}%
\pgfpathcurveto{\pgfqpoint{1.515920in}{2.815335in}}{\pgfqpoint{1.526519in}{2.819726in}}{\pgfqpoint{1.534333in}{2.827539in}}%
\pgfpathcurveto{\pgfqpoint{1.542147in}{2.835353in}}{\pgfqpoint{1.546537in}{2.845952in}}{\pgfqpoint{1.546537in}{2.857002in}}%
\pgfpathcurveto{\pgfqpoint{1.546537in}{2.868052in}}{\pgfqpoint{1.542147in}{2.878651in}}{\pgfqpoint{1.534333in}{2.886465in}}%
\pgfpathcurveto{\pgfqpoint{1.526519in}{2.894278in}}{\pgfqpoint{1.515920in}{2.898669in}}{\pgfqpoint{1.504870in}{2.898669in}}%
\pgfpathcurveto{\pgfqpoint{1.493820in}{2.898669in}}{\pgfqpoint{1.483221in}{2.894278in}}{\pgfqpoint{1.475407in}{2.886465in}}%
\pgfpathcurveto{\pgfqpoint{1.467594in}{2.878651in}}{\pgfqpoint{1.463203in}{2.868052in}}{\pgfqpoint{1.463203in}{2.857002in}}%
\pgfpathcurveto{\pgfqpoint{1.463203in}{2.845952in}}{\pgfqpoint{1.467594in}{2.835353in}}{\pgfqpoint{1.475407in}{2.827539in}}%
\pgfpathcurveto{\pgfqpoint{1.483221in}{2.819726in}}{\pgfqpoint{1.493820in}{2.815335in}}{\pgfqpoint{1.504870in}{2.815335in}}%
\pgfpathclose%
\pgfusepath{stroke,fill}%
\end{pgfscope}%
\begin{pgfscope}%
\pgfpathrectangle{\pgfqpoint{0.750000in}{0.500000in}}{\pgfqpoint{4.650000in}{3.020000in}}%
\pgfusepath{clip}%
\pgfsetbuttcap%
\pgfsetroundjoin%
\definecolor{currentfill}{rgb}{0.121569,0.466667,0.705882}%
\pgfsetfillcolor{currentfill}%
\pgfsetlinewidth{1.003750pt}%
\definecolor{currentstroke}{rgb}{0.121569,0.466667,0.705882}%
\pgfsetstrokecolor{currentstroke}%
\pgfsetdash{}{0pt}%
\pgfpathmoveto{\pgfqpoint{1.202922in}{0.599500in}}%
\pgfpathcurveto{\pgfqpoint{1.213972in}{0.599500in}}{\pgfqpoint{1.224571in}{0.603891in}}{\pgfqpoint{1.232385in}{0.611704in}}%
\pgfpathcurveto{\pgfqpoint{1.240198in}{0.619518in}}{\pgfqpoint{1.244589in}{0.630117in}}{\pgfqpoint{1.244589in}{0.641167in}}%
\pgfpathcurveto{\pgfqpoint{1.244589in}{0.652217in}}{\pgfqpoint{1.240198in}{0.662816in}}{\pgfqpoint{1.232385in}{0.670630in}}%
\pgfpathcurveto{\pgfqpoint{1.224571in}{0.678443in}}{\pgfqpoint{1.213972in}{0.682834in}}{\pgfqpoint{1.202922in}{0.682834in}}%
\pgfpathcurveto{\pgfqpoint{1.191872in}{0.682834in}}{\pgfqpoint{1.181273in}{0.678443in}}{\pgfqpoint{1.173459in}{0.670630in}}%
\pgfpathcurveto{\pgfqpoint{1.165646in}{0.662816in}}{\pgfqpoint{1.161255in}{0.652217in}}{\pgfqpoint{1.161255in}{0.641167in}}%
\pgfpathcurveto{\pgfqpoint{1.161255in}{0.630117in}}{\pgfqpoint{1.165646in}{0.619518in}}{\pgfqpoint{1.173459in}{0.611704in}}%
\pgfpathcurveto{\pgfqpoint{1.181273in}{0.603891in}}{\pgfqpoint{1.191872in}{0.599500in}}{\pgfqpoint{1.202922in}{0.599500in}}%
\pgfpathclose%
\pgfusepath{stroke,fill}%
\end{pgfscope}%
\begin{pgfscope}%
\pgfpathrectangle{\pgfqpoint{0.750000in}{0.500000in}}{\pgfqpoint{4.650000in}{3.020000in}}%
\pgfusepath{clip}%
\pgfsetbuttcap%
\pgfsetroundjoin%
\definecolor{currentfill}{rgb}{1.000000,0.498039,0.054902}%
\pgfsetfillcolor{currentfill}%
\pgfsetlinewidth{1.003750pt}%
\definecolor{currentstroke}{rgb}{1.000000,0.498039,0.054902}%
\pgfsetstrokecolor{currentstroke}%
\pgfsetdash{}{0pt}%
\pgfpathmoveto{\pgfqpoint{1.867208in}{2.936057in}}%
\pgfpathcurveto{\pgfqpoint{1.878258in}{2.936057in}}{\pgfqpoint{1.888857in}{2.940448in}}{\pgfqpoint{1.896671in}{2.948261in}}%
\pgfpathcurveto{\pgfqpoint{1.904484in}{2.956075in}}{\pgfqpoint{1.908874in}{2.966674in}}{\pgfqpoint{1.908874in}{2.977724in}}%
\pgfpathcurveto{\pgfqpoint{1.908874in}{2.988774in}}{\pgfqpoint{1.904484in}{2.999373in}}{\pgfqpoint{1.896671in}{3.007187in}}%
\pgfpathcurveto{\pgfqpoint{1.888857in}{3.015000in}}{\pgfqpoint{1.878258in}{3.019391in}}{\pgfqpoint{1.867208in}{3.019391in}}%
\pgfpathcurveto{\pgfqpoint{1.856158in}{3.019391in}}{\pgfqpoint{1.845559in}{3.015000in}}{\pgfqpoint{1.837745in}{3.007187in}}%
\pgfpathcurveto{\pgfqpoint{1.829931in}{2.999373in}}{\pgfqpoint{1.825541in}{2.988774in}}{\pgfqpoint{1.825541in}{2.977724in}}%
\pgfpathcurveto{\pgfqpoint{1.825541in}{2.966674in}}{\pgfqpoint{1.829931in}{2.956075in}}{\pgfqpoint{1.837745in}{2.948261in}}%
\pgfpathcurveto{\pgfqpoint{1.845559in}{2.940448in}}{\pgfqpoint{1.856158in}{2.936057in}}{\pgfqpoint{1.867208in}{2.936057in}}%
\pgfpathclose%
\pgfusepath{stroke,fill}%
\end{pgfscope}%
\begin{pgfscope}%
\pgfpathrectangle{\pgfqpoint{0.750000in}{0.500000in}}{\pgfqpoint{4.650000in}{3.020000in}}%
\pgfusepath{clip}%
\pgfsetbuttcap%
\pgfsetroundjoin%
\definecolor{currentfill}{rgb}{0.121569,0.466667,0.705882}%
\pgfsetfillcolor{currentfill}%
\pgfsetlinewidth{1.003750pt}%
\definecolor{currentstroke}{rgb}{0.121569,0.466667,0.705882}%
\pgfsetstrokecolor{currentstroke}%
\pgfsetdash{}{0pt}%
\pgfpathmoveto{\pgfqpoint{1.263312in}{0.595606in}}%
\pgfpathcurveto{\pgfqpoint{1.274362in}{0.595606in}}{\pgfqpoint{1.284961in}{0.599996in}}{\pgfqpoint{1.292774in}{0.607810in}}%
\pgfpathcurveto{\pgfqpoint{1.300588in}{0.615624in}}{\pgfqpoint{1.304978in}{0.626223in}}{\pgfqpoint{1.304978in}{0.637273in}}%
\pgfpathcurveto{\pgfqpoint{1.304978in}{0.648323in}}{\pgfqpoint{1.300588in}{0.658922in}}{\pgfqpoint{1.292774in}{0.666736in}}%
\pgfpathcurveto{\pgfqpoint{1.284961in}{0.674549in}}{\pgfqpoint{1.274362in}{0.678939in}}{\pgfqpoint{1.263312in}{0.678939in}}%
\pgfpathcurveto{\pgfqpoint{1.252262in}{0.678939in}}{\pgfqpoint{1.241663in}{0.674549in}}{\pgfqpoint{1.233849in}{0.666736in}}%
\pgfpathcurveto{\pgfqpoint{1.226035in}{0.658922in}}{\pgfqpoint{1.221645in}{0.648323in}}{\pgfqpoint{1.221645in}{0.637273in}}%
\pgfpathcurveto{\pgfqpoint{1.221645in}{0.626223in}}{\pgfqpoint{1.226035in}{0.615624in}}{\pgfqpoint{1.233849in}{0.607810in}}%
\pgfpathcurveto{\pgfqpoint{1.241663in}{0.599996in}}{\pgfqpoint{1.252262in}{0.595606in}}{\pgfqpoint{1.263312in}{0.595606in}}%
\pgfpathclose%
\pgfusepath{stroke,fill}%
\end{pgfscope}%
\begin{pgfscope}%
\pgfpathrectangle{\pgfqpoint{0.750000in}{0.500000in}}{\pgfqpoint{4.650000in}{3.020000in}}%
\pgfusepath{clip}%
\pgfsetbuttcap%
\pgfsetroundjoin%
\definecolor{currentfill}{rgb}{1.000000,0.498039,0.054902}%
\pgfsetfillcolor{currentfill}%
\pgfsetlinewidth{1.003750pt}%
\definecolor{currentstroke}{rgb}{1.000000,0.498039,0.054902}%
\pgfsetstrokecolor{currentstroke}%
\pgfsetdash{}{0pt}%
\pgfpathmoveto{\pgfqpoint{1.625649in}{2.928269in}}%
\pgfpathcurveto{\pgfqpoint{1.636699in}{2.928269in}}{\pgfqpoint{1.647299in}{2.932659in}}{\pgfqpoint{1.655112in}{2.940473in}}%
\pgfpathcurveto{\pgfqpoint{1.662926in}{2.948286in}}{\pgfqpoint{1.667316in}{2.958885in}}{\pgfqpoint{1.667316in}{2.969936in}}%
\pgfpathcurveto{\pgfqpoint{1.667316in}{2.980986in}}{\pgfqpoint{1.662926in}{2.991585in}}{\pgfqpoint{1.655112in}{2.999398in}}%
\pgfpathcurveto{\pgfqpoint{1.647299in}{3.007212in}}{\pgfqpoint{1.636699in}{3.011602in}}{\pgfqpoint{1.625649in}{3.011602in}}%
\pgfpathcurveto{\pgfqpoint{1.614599in}{3.011602in}}{\pgfqpoint{1.604000in}{3.007212in}}{\pgfqpoint{1.596187in}{2.999398in}}%
\pgfpathcurveto{\pgfqpoint{1.588373in}{2.991585in}}{\pgfqpoint{1.583983in}{2.980986in}}{\pgfqpoint{1.583983in}{2.969936in}}%
\pgfpathcurveto{\pgfqpoint{1.583983in}{2.958885in}}{\pgfqpoint{1.588373in}{2.948286in}}{\pgfqpoint{1.596187in}{2.940473in}}%
\pgfpathcurveto{\pgfqpoint{1.604000in}{2.932659in}}{\pgfqpoint{1.614599in}{2.928269in}}{\pgfqpoint{1.625649in}{2.928269in}}%
\pgfpathclose%
\pgfusepath{stroke,fill}%
\end{pgfscope}%
\begin{pgfscope}%
\pgfpathrectangle{\pgfqpoint{0.750000in}{0.500000in}}{\pgfqpoint{4.650000in}{3.020000in}}%
\pgfusepath{clip}%
\pgfsetbuttcap%
\pgfsetroundjoin%
\definecolor{currentfill}{rgb}{1.000000,0.498039,0.054902}%
\pgfsetfillcolor{currentfill}%
\pgfsetlinewidth{1.003750pt}%
\definecolor{currentstroke}{rgb}{1.000000,0.498039,0.054902}%
\pgfsetstrokecolor{currentstroke}%
\pgfsetdash{}{0pt}%
\pgfpathmoveto{\pgfqpoint{2.229545in}{2.939952in}}%
\pgfpathcurveto{\pgfqpoint{2.240596in}{2.939952in}}{\pgfqpoint{2.251195in}{2.944342in}}{\pgfqpoint{2.259008in}{2.952156in}}%
\pgfpathcurveto{\pgfqpoint{2.266822in}{2.959969in}}{\pgfqpoint{2.271212in}{2.970568in}}{\pgfqpoint{2.271212in}{2.981618in}}%
\pgfpathcurveto{\pgfqpoint{2.271212in}{2.992668in}}{\pgfqpoint{2.266822in}{3.003267in}}{\pgfqpoint{2.259008in}{3.011081in}}%
\pgfpathcurveto{\pgfqpoint{2.251195in}{3.018895in}}{\pgfqpoint{2.240596in}{3.023285in}}{\pgfqpoint{2.229545in}{3.023285in}}%
\pgfpathcurveto{\pgfqpoint{2.218495in}{3.023285in}}{\pgfqpoint{2.207896in}{3.018895in}}{\pgfqpoint{2.200083in}{3.011081in}}%
\pgfpathcurveto{\pgfqpoint{2.192269in}{3.003267in}}{\pgfqpoint{2.187879in}{2.992668in}}{\pgfqpoint{2.187879in}{2.981618in}}%
\pgfpathcurveto{\pgfqpoint{2.187879in}{2.970568in}}{\pgfqpoint{2.192269in}{2.959969in}}{\pgfqpoint{2.200083in}{2.952156in}}%
\pgfpathcurveto{\pgfqpoint{2.207896in}{2.944342in}}{\pgfqpoint{2.218495in}{2.939952in}}{\pgfqpoint{2.229545in}{2.939952in}}%
\pgfpathclose%
\pgfusepath{stroke,fill}%
\end{pgfscope}%
\begin{pgfscope}%
\pgfpathrectangle{\pgfqpoint{0.750000in}{0.500000in}}{\pgfqpoint{4.650000in}{3.020000in}}%
\pgfusepath{clip}%
\pgfsetbuttcap%
\pgfsetroundjoin%
\definecolor{currentfill}{rgb}{0.121569,0.466667,0.705882}%
\pgfsetfillcolor{currentfill}%
\pgfsetlinewidth{1.003750pt}%
\definecolor{currentstroke}{rgb}{0.121569,0.466667,0.705882}%
\pgfsetstrokecolor{currentstroke}%
\pgfsetdash{}{0pt}%
\pgfpathmoveto{\pgfqpoint{1.021753in}{0.595606in}}%
\pgfpathcurveto{\pgfqpoint{1.032803in}{0.595606in}}{\pgfqpoint{1.043402in}{0.599996in}}{\pgfqpoint{1.051216in}{0.607810in}}%
\pgfpathcurveto{\pgfqpoint{1.059030in}{0.615624in}}{\pgfqpoint{1.063420in}{0.626223in}}{\pgfqpoint{1.063420in}{0.637273in}}%
\pgfpathcurveto{\pgfqpoint{1.063420in}{0.648323in}}{\pgfqpoint{1.059030in}{0.658922in}}{\pgfqpoint{1.051216in}{0.666736in}}%
\pgfpathcurveto{\pgfqpoint{1.043402in}{0.674549in}}{\pgfqpoint{1.032803in}{0.678939in}}{\pgfqpoint{1.021753in}{0.678939in}}%
\pgfpathcurveto{\pgfqpoint{1.010703in}{0.678939in}}{\pgfqpoint{1.000104in}{0.674549in}}{\pgfqpoint{0.992290in}{0.666736in}}%
\pgfpathcurveto{\pgfqpoint{0.984477in}{0.658922in}}{\pgfqpoint{0.980087in}{0.648323in}}{\pgfqpoint{0.980087in}{0.637273in}}%
\pgfpathcurveto{\pgfqpoint{0.980087in}{0.626223in}}{\pgfqpoint{0.984477in}{0.615624in}}{\pgfqpoint{0.992290in}{0.607810in}}%
\pgfpathcurveto{\pgfqpoint{1.000104in}{0.599996in}}{\pgfqpoint{1.010703in}{0.595606in}}{\pgfqpoint{1.021753in}{0.595606in}}%
\pgfpathclose%
\pgfusepath{stroke,fill}%
\end{pgfscope}%
\begin{pgfscope}%
\pgfpathrectangle{\pgfqpoint{0.750000in}{0.500000in}}{\pgfqpoint{4.650000in}{3.020000in}}%
\pgfusepath{clip}%
\pgfsetbuttcap%
\pgfsetroundjoin%
\definecolor{currentfill}{rgb}{1.000000,0.498039,0.054902}%
\pgfsetfillcolor{currentfill}%
\pgfsetlinewidth{1.003750pt}%
\definecolor{currentstroke}{rgb}{1.000000,0.498039,0.054902}%
\pgfsetstrokecolor{currentstroke}%
\pgfsetdash{}{0pt}%
\pgfpathmoveto{\pgfqpoint{1.625649in}{2.990577in}}%
\pgfpathcurveto{\pgfqpoint{1.636699in}{2.990577in}}{\pgfqpoint{1.647299in}{2.994967in}}{\pgfqpoint{1.655112in}{3.002781in}}%
\pgfpathcurveto{\pgfqpoint{1.662926in}{3.010595in}}{\pgfqpoint{1.667316in}{3.021194in}}{\pgfqpoint{1.667316in}{3.032244in}}%
\pgfpathcurveto{\pgfqpoint{1.667316in}{3.043294in}}{\pgfqpoint{1.662926in}{3.053893in}}{\pgfqpoint{1.655112in}{3.061706in}}%
\pgfpathcurveto{\pgfqpoint{1.647299in}{3.069520in}}{\pgfqpoint{1.636699in}{3.073910in}}{\pgfqpoint{1.625649in}{3.073910in}}%
\pgfpathcurveto{\pgfqpoint{1.614599in}{3.073910in}}{\pgfqpoint{1.604000in}{3.069520in}}{\pgfqpoint{1.596187in}{3.061706in}}%
\pgfpathcurveto{\pgfqpoint{1.588373in}{3.053893in}}{\pgfqpoint{1.583983in}{3.043294in}}{\pgfqpoint{1.583983in}{3.032244in}}%
\pgfpathcurveto{\pgfqpoint{1.583983in}{3.021194in}}{\pgfqpoint{1.588373in}{3.010595in}}{\pgfqpoint{1.596187in}{3.002781in}}%
\pgfpathcurveto{\pgfqpoint{1.604000in}{2.994967in}}{\pgfqpoint{1.614599in}{2.990577in}}{\pgfqpoint{1.625649in}{2.990577in}}%
\pgfpathclose%
\pgfusepath{stroke,fill}%
\end{pgfscope}%
\begin{pgfscope}%
\pgfpathrectangle{\pgfqpoint{0.750000in}{0.500000in}}{\pgfqpoint{4.650000in}{3.020000in}}%
\pgfusepath{clip}%
\pgfsetbuttcap%
\pgfsetroundjoin%
\definecolor{currentfill}{rgb}{1.000000,0.498039,0.054902}%
\pgfsetfillcolor{currentfill}%
\pgfsetlinewidth{1.003750pt}%
\definecolor{currentstroke}{rgb}{1.000000,0.498039,0.054902}%
\pgfsetstrokecolor{currentstroke}%
\pgfsetdash{}{0pt}%
\pgfpathmoveto{\pgfqpoint{1.625649in}{2.998366in}}%
\pgfpathcurveto{\pgfqpoint{1.636699in}{2.998366in}}{\pgfqpoint{1.647299in}{3.002756in}}{\pgfqpoint{1.655112in}{3.010569in}}%
\pgfpathcurveto{\pgfqpoint{1.662926in}{3.018383in}}{\pgfqpoint{1.667316in}{3.028982in}}{\pgfqpoint{1.667316in}{3.040032in}}%
\pgfpathcurveto{\pgfqpoint{1.667316in}{3.051082in}}{\pgfqpoint{1.662926in}{3.061681in}}{\pgfqpoint{1.655112in}{3.069495in}}%
\pgfpathcurveto{\pgfqpoint{1.647299in}{3.077309in}}{\pgfqpoint{1.636699in}{3.081699in}}{\pgfqpoint{1.625649in}{3.081699in}}%
\pgfpathcurveto{\pgfqpoint{1.614599in}{3.081699in}}{\pgfqpoint{1.604000in}{3.077309in}}{\pgfqpoint{1.596187in}{3.069495in}}%
\pgfpathcurveto{\pgfqpoint{1.588373in}{3.061681in}}{\pgfqpoint{1.583983in}{3.051082in}}{\pgfqpoint{1.583983in}{3.040032in}}%
\pgfpathcurveto{\pgfqpoint{1.583983in}{3.028982in}}{\pgfqpoint{1.588373in}{3.018383in}}{\pgfqpoint{1.596187in}{3.010569in}}%
\pgfpathcurveto{\pgfqpoint{1.604000in}{3.002756in}}{\pgfqpoint{1.614599in}{2.998366in}}{\pgfqpoint{1.625649in}{2.998366in}}%
\pgfpathclose%
\pgfusepath{stroke,fill}%
\end{pgfscope}%
\begin{pgfscope}%
\pgfpathrectangle{\pgfqpoint{0.750000in}{0.500000in}}{\pgfqpoint{4.650000in}{3.020000in}}%
\pgfusepath{clip}%
\pgfsetbuttcap%
\pgfsetroundjoin%
\definecolor{currentfill}{rgb}{1.000000,0.498039,0.054902}%
\pgfsetfillcolor{currentfill}%
\pgfsetlinewidth{1.003750pt}%
\definecolor{currentstroke}{rgb}{1.000000,0.498039,0.054902}%
\pgfsetstrokecolor{currentstroke}%
\pgfsetdash{}{0pt}%
\pgfpathmoveto{\pgfqpoint{1.565260in}{2.932163in}}%
\pgfpathcurveto{\pgfqpoint{1.576310in}{2.932163in}}{\pgfqpoint{1.586909in}{2.936553in}}{\pgfqpoint{1.594723in}{2.944367in}}%
\pgfpathcurveto{\pgfqpoint{1.602536in}{2.952181in}}{\pgfqpoint{1.606926in}{2.962780in}}{\pgfqpoint{1.606926in}{2.973830in}}%
\pgfpathcurveto{\pgfqpoint{1.606926in}{2.984880in}}{\pgfqpoint{1.602536in}{2.995479in}}{\pgfqpoint{1.594723in}{3.003293in}}%
\pgfpathcurveto{\pgfqpoint{1.586909in}{3.011106in}}{\pgfqpoint{1.576310in}{3.015496in}}{\pgfqpoint{1.565260in}{3.015496in}}%
\pgfpathcurveto{\pgfqpoint{1.554210in}{3.015496in}}{\pgfqpoint{1.543611in}{3.011106in}}{\pgfqpoint{1.535797in}{3.003293in}}%
\pgfpathcurveto{\pgfqpoint{1.527983in}{2.995479in}}{\pgfqpoint{1.523593in}{2.984880in}}{\pgfqpoint{1.523593in}{2.973830in}}%
\pgfpathcurveto{\pgfqpoint{1.523593in}{2.962780in}}{\pgfqpoint{1.527983in}{2.952181in}}{\pgfqpoint{1.535797in}{2.944367in}}%
\pgfpathcurveto{\pgfqpoint{1.543611in}{2.936553in}}{\pgfqpoint{1.554210in}{2.932163in}}{\pgfqpoint{1.565260in}{2.932163in}}%
\pgfpathclose%
\pgfusepath{stroke,fill}%
\end{pgfscope}%
\begin{pgfscope}%
\pgfpathrectangle{\pgfqpoint{0.750000in}{0.500000in}}{\pgfqpoint{4.650000in}{3.020000in}}%
\pgfusepath{clip}%
\pgfsetbuttcap%
\pgfsetroundjoin%
\definecolor{currentfill}{rgb}{1.000000,0.498039,0.054902}%
\pgfsetfillcolor{currentfill}%
\pgfsetlinewidth{1.003750pt}%
\definecolor{currentstroke}{rgb}{1.000000,0.498039,0.054902}%
\pgfsetstrokecolor{currentstroke}%
\pgfsetdash{}{0pt}%
\pgfpathmoveto{\pgfqpoint{1.806818in}{2.936057in}}%
\pgfpathcurveto{\pgfqpoint{1.817868in}{2.936057in}}{\pgfqpoint{1.828467in}{2.940448in}}{\pgfqpoint{1.836281in}{2.948261in}}%
\pgfpathcurveto{\pgfqpoint{1.844095in}{2.956075in}}{\pgfqpoint{1.848485in}{2.966674in}}{\pgfqpoint{1.848485in}{2.977724in}}%
\pgfpathcurveto{\pgfqpoint{1.848485in}{2.988774in}}{\pgfqpoint{1.844095in}{2.999373in}}{\pgfqpoint{1.836281in}{3.007187in}}%
\pgfpathcurveto{\pgfqpoint{1.828467in}{3.015000in}}{\pgfqpoint{1.817868in}{3.019391in}}{\pgfqpoint{1.806818in}{3.019391in}}%
\pgfpathcurveto{\pgfqpoint{1.795768in}{3.019391in}}{\pgfqpoint{1.785169in}{3.015000in}}{\pgfqpoint{1.777355in}{3.007187in}}%
\pgfpathcurveto{\pgfqpoint{1.769542in}{2.999373in}}{\pgfqpoint{1.765152in}{2.988774in}}{\pgfqpoint{1.765152in}{2.977724in}}%
\pgfpathcurveto{\pgfqpoint{1.765152in}{2.966674in}}{\pgfqpoint{1.769542in}{2.956075in}}{\pgfqpoint{1.777355in}{2.948261in}}%
\pgfpathcurveto{\pgfqpoint{1.785169in}{2.940448in}}{\pgfqpoint{1.795768in}{2.936057in}}{\pgfqpoint{1.806818in}{2.936057in}}%
\pgfpathclose%
\pgfusepath{stroke,fill}%
\end{pgfscope}%
\begin{pgfscope}%
\pgfpathrectangle{\pgfqpoint{0.750000in}{0.500000in}}{\pgfqpoint{4.650000in}{3.020000in}}%
\pgfusepath{clip}%
\pgfsetbuttcap%
\pgfsetroundjoin%
\definecolor{currentfill}{rgb}{1.000000,0.498039,0.054902}%
\pgfsetfillcolor{currentfill}%
\pgfsetlinewidth{1.003750pt}%
\definecolor{currentstroke}{rgb}{1.000000,0.498039,0.054902}%
\pgfsetstrokecolor{currentstroke}%
\pgfsetdash{}{0pt}%
\pgfpathmoveto{\pgfqpoint{1.565260in}{2.943846in}}%
\pgfpathcurveto{\pgfqpoint{1.576310in}{2.943846in}}{\pgfqpoint{1.586909in}{2.948236in}}{\pgfqpoint{1.594723in}{2.956050in}}%
\pgfpathcurveto{\pgfqpoint{1.602536in}{2.963863in}}{\pgfqpoint{1.606926in}{2.974462in}}{\pgfqpoint{1.606926in}{2.985513in}}%
\pgfpathcurveto{\pgfqpoint{1.606926in}{2.996563in}}{\pgfqpoint{1.602536in}{3.007162in}}{\pgfqpoint{1.594723in}{3.014975in}}%
\pgfpathcurveto{\pgfqpoint{1.586909in}{3.022789in}}{\pgfqpoint{1.576310in}{3.027179in}}{\pgfqpoint{1.565260in}{3.027179in}}%
\pgfpathcurveto{\pgfqpoint{1.554210in}{3.027179in}}{\pgfqpoint{1.543611in}{3.022789in}}{\pgfqpoint{1.535797in}{3.014975in}}%
\pgfpathcurveto{\pgfqpoint{1.527983in}{3.007162in}}{\pgfqpoint{1.523593in}{2.996563in}}{\pgfqpoint{1.523593in}{2.985513in}}%
\pgfpathcurveto{\pgfqpoint{1.523593in}{2.974462in}}{\pgfqpoint{1.527983in}{2.963863in}}{\pgfqpoint{1.535797in}{2.956050in}}%
\pgfpathcurveto{\pgfqpoint{1.543611in}{2.948236in}}{\pgfqpoint{1.554210in}{2.943846in}}{\pgfqpoint{1.565260in}{2.943846in}}%
\pgfpathclose%
\pgfusepath{stroke,fill}%
\end{pgfscope}%
\begin{pgfscope}%
\pgfpathrectangle{\pgfqpoint{0.750000in}{0.500000in}}{\pgfqpoint{4.650000in}{3.020000in}}%
\pgfusepath{clip}%
\pgfsetbuttcap%
\pgfsetroundjoin%
\definecolor{currentfill}{rgb}{1.000000,0.498039,0.054902}%
\pgfsetfillcolor{currentfill}%
\pgfsetlinewidth{1.003750pt}%
\definecolor{currentstroke}{rgb}{1.000000,0.498039,0.054902}%
\pgfsetstrokecolor{currentstroke}%
\pgfsetdash{}{0pt}%
\pgfpathmoveto{\pgfqpoint{1.504870in}{2.932163in}}%
\pgfpathcurveto{\pgfqpoint{1.515920in}{2.932163in}}{\pgfqpoint{1.526519in}{2.936553in}}{\pgfqpoint{1.534333in}{2.944367in}}%
\pgfpathcurveto{\pgfqpoint{1.542147in}{2.952181in}}{\pgfqpoint{1.546537in}{2.962780in}}{\pgfqpoint{1.546537in}{2.973830in}}%
\pgfpathcurveto{\pgfqpoint{1.546537in}{2.984880in}}{\pgfqpoint{1.542147in}{2.995479in}}{\pgfqpoint{1.534333in}{3.003293in}}%
\pgfpathcurveto{\pgfqpoint{1.526519in}{3.011106in}}{\pgfqpoint{1.515920in}{3.015496in}}{\pgfqpoint{1.504870in}{3.015496in}}%
\pgfpathcurveto{\pgfqpoint{1.493820in}{3.015496in}}{\pgfqpoint{1.483221in}{3.011106in}}{\pgfqpoint{1.475407in}{3.003293in}}%
\pgfpathcurveto{\pgfqpoint{1.467594in}{2.995479in}}{\pgfqpoint{1.463203in}{2.984880in}}{\pgfqpoint{1.463203in}{2.973830in}}%
\pgfpathcurveto{\pgfqpoint{1.463203in}{2.962780in}}{\pgfqpoint{1.467594in}{2.952181in}}{\pgfqpoint{1.475407in}{2.944367in}}%
\pgfpathcurveto{\pgfqpoint{1.483221in}{2.936553in}}{\pgfqpoint{1.493820in}{2.932163in}}{\pgfqpoint{1.504870in}{2.932163in}}%
\pgfpathclose%
\pgfusepath{stroke,fill}%
\end{pgfscope}%
\begin{pgfscope}%
\pgfpathrectangle{\pgfqpoint{0.750000in}{0.500000in}}{\pgfqpoint{4.650000in}{3.020000in}}%
\pgfusepath{clip}%
\pgfsetbuttcap%
\pgfsetroundjoin%
\definecolor{currentfill}{rgb}{0.121569,0.466667,0.705882}%
\pgfsetfillcolor{currentfill}%
\pgfsetlinewidth{1.003750pt}%
\definecolor{currentstroke}{rgb}{0.121569,0.466667,0.705882}%
\pgfsetstrokecolor{currentstroke}%
\pgfsetdash{}{0pt}%
\pgfpathmoveto{\pgfqpoint{1.384091in}{0.611183in}}%
\pgfpathcurveto{\pgfqpoint{1.395141in}{0.611183in}}{\pgfqpoint{1.405740in}{0.615573in}}{\pgfqpoint{1.413554in}{0.623387in}}%
\pgfpathcurveto{\pgfqpoint{1.421367in}{0.631201in}}{\pgfqpoint{1.425758in}{0.641800in}}{\pgfqpoint{1.425758in}{0.652850in}}%
\pgfpathcurveto{\pgfqpoint{1.425758in}{0.663900in}}{\pgfqpoint{1.421367in}{0.674499in}}{\pgfqpoint{1.413554in}{0.682313in}}%
\pgfpathcurveto{\pgfqpoint{1.405740in}{0.690126in}}{\pgfqpoint{1.395141in}{0.694516in}}{\pgfqpoint{1.384091in}{0.694516in}}%
\pgfpathcurveto{\pgfqpoint{1.373041in}{0.694516in}}{\pgfqpoint{1.362442in}{0.690126in}}{\pgfqpoint{1.354628in}{0.682313in}}%
\pgfpathcurveto{\pgfqpoint{1.346815in}{0.674499in}}{\pgfqpoint{1.342424in}{0.663900in}}{\pgfqpoint{1.342424in}{0.652850in}}%
\pgfpathcurveto{\pgfqpoint{1.342424in}{0.641800in}}{\pgfqpoint{1.346815in}{0.631201in}}{\pgfqpoint{1.354628in}{0.623387in}}%
\pgfpathcurveto{\pgfqpoint{1.362442in}{0.615573in}}{\pgfqpoint{1.373041in}{0.611183in}}{\pgfqpoint{1.384091in}{0.611183in}}%
\pgfpathclose%
\pgfusepath{stroke,fill}%
\end{pgfscope}%
\begin{pgfscope}%
\pgfpathrectangle{\pgfqpoint{0.750000in}{0.500000in}}{\pgfqpoint{4.650000in}{3.020000in}}%
\pgfusepath{clip}%
\pgfsetbuttcap%
\pgfsetroundjoin%
\definecolor{currentfill}{rgb}{1.000000,0.498039,0.054902}%
\pgfsetfillcolor{currentfill}%
\pgfsetlinewidth{1.003750pt}%
\definecolor{currentstroke}{rgb}{1.000000,0.498039,0.054902}%
\pgfsetstrokecolor{currentstroke}%
\pgfsetdash{}{0pt}%
\pgfpathmoveto{\pgfqpoint{1.927597in}{2.955529in}}%
\pgfpathcurveto{\pgfqpoint{1.938648in}{2.955529in}}{\pgfqpoint{1.949247in}{2.959919in}}{\pgfqpoint{1.957060in}{2.967733in}}%
\pgfpathcurveto{\pgfqpoint{1.964874in}{2.975546in}}{\pgfqpoint{1.969264in}{2.986145in}}{\pgfqpoint{1.969264in}{2.997195in}}%
\pgfpathcurveto{\pgfqpoint{1.969264in}{3.008245in}}{\pgfqpoint{1.964874in}{3.018845in}}{\pgfqpoint{1.957060in}{3.026658in}}%
\pgfpathcurveto{\pgfqpoint{1.949247in}{3.034472in}}{\pgfqpoint{1.938648in}{3.038862in}}{\pgfqpoint{1.927597in}{3.038862in}}%
\pgfpathcurveto{\pgfqpoint{1.916547in}{3.038862in}}{\pgfqpoint{1.905948in}{3.034472in}}{\pgfqpoint{1.898135in}{3.026658in}}%
\pgfpathcurveto{\pgfqpoint{1.890321in}{3.018845in}}{\pgfqpoint{1.885931in}{3.008245in}}{\pgfqpoint{1.885931in}{2.997195in}}%
\pgfpathcurveto{\pgfqpoint{1.885931in}{2.986145in}}{\pgfqpoint{1.890321in}{2.975546in}}{\pgfqpoint{1.898135in}{2.967733in}}%
\pgfpathcurveto{\pgfqpoint{1.905948in}{2.959919in}}{\pgfqpoint{1.916547in}{2.955529in}}{\pgfqpoint{1.927597in}{2.955529in}}%
\pgfpathclose%
\pgfusepath{stroke,fill}%
\end{pgfscope}%
\begin{pgfscope}%
\pgfpathrectangle{\pgfqpoint{0.750000in}{0.500000in}}{\pgfqpoint{4.650000in}{3.020000in}}%
\pgfusepath{clip}%
\pgfsetbuttcap%
\pgfsetroundjoin%
\definecolor{currentfill}{rgb}{1.000000,0.498039,0.054902}%
\pgfsetfillcolor{currentfill}%
\pgfsetlinewidth{1.003750pt}%
\definecolor{currentstroke}{rgb}{1.000000,0.498039,0.054902}%
\pgfsetstrokecolor{currentstroke}%
\pgfsetdash{}{0pt}%
\pgfpathmoveto{\pgfqpoint{3.014610in}{2.928269in}}%
\pgfpathcurveto{\pgfqpoint{3.025661in}{2.928269in}}{\pgfqpoint{3.036260in}{2.932659in}}{\pgfqpoint{3.044073in}{2.940473in}}%
\pgfpathcurveto{\pgfqpoint{3.051887in}{2.948286in}}{\pgfqpoint{3.056277in}{2.958885in}}{\pgfqpoint{3.056277in}{2.969936in}}%
\pgfpathcurveto{\pgfqpoint{3.056277in}{2.980986in}}{\pgfqpoint{3.051887in}{2.991585in}}{\pgfqpoint{3.044073in}{2.999398in}}%
\pgfpathcurveto{\pgfqpoint{3.036260in}{3.007212in}}{\pgfqpoint{3.025661in}{3.011602in}}{\pgfqpoint{3.014610in}{3.011602in}}%
\pgfpathcurveto{\pgfqpoint{3.003560in}{3.011602in}}{\pgfqpoint{2.992961in}{3.007212in}}{\pgfqpoint{2.985148in}{2.999398in}}%
\pgfpathcurveto{\pgfqpoint{2.977334in}{2.991585in}}{\pgfqpoint{2.972944in}{2.980986in}}{\pgfqpoint{2.972944in}{2.969936in}}%
\pgfpathcurveto{\pgfqpoint{2.972944in}{2.958885in}}{\pgfqpoint{2.977334in}{2.948286in}}{\pgfqpoint{2.985148in}{2.940473in}}%
\pgfpathcurveto{\pgfqpoint{2.992961in}{2.932659in}}{\pgfqpoint{3.003560in}{2.928269in}}{\pgfqpoint{3.014610in}{2.928269in}}%
\pgfpathclose%
\pgfusepath{stroke,fill}%
\end{pgfscope}%
\begin{pgfscope}%
\pgfpathrectangle{\pgfqpoint{0.750000in}{0.500000in}}{\pgfqpoint{4.650000in}{3.020000in}}%
\pgfusepath{clip}%
\pgfsetbuttcap%
\pgfsetroundjoin%
\definecolor{currentfill}{rgb}{1.000000,0.498039,0.054902}%
\pgfsetfillcolor{currentfill}%
\pgfsetlinewidth{1.003750pt}%
\definecolor{currentstroke}{rgb}{1.000000,0.498039,0.054902}%
\pgfsetstrokecolor{currentstroke}%
\pgfsetdash{}{0pt}%
\pgfpathmoveto{\pgfqpoint{1.384091in}{2.936057in}}%
\pgfpathcurveto{\pgfqpoint{1.395141in}{2.936057in}}{\pgfqpoint{1.405740in}{2.940448in}}{\pgfqpoint{1.413554in}{2.948261in}}%
\pgfpathcurveto{\pgfqpoint{1.421367in}{2.956075in}}{\pgfqpoint{1.425758in}{2.966674in}}{\pgfqpoint{1.425758in}{2.977724in}}%
\pgfpathcurveto{\pgfqpoint{1.425758in}{2.988774in}}{\pgfqpoint{1.421367in}{2.999373in}}{\pgfqpoint{1.413554in}{3.007187in}}%
\pgfpathcurveto{\pgfqpoint{1.405740in}{3.015000in}}{\pgfqpoint{1.395141in}{3.019391in}}{\pgfqpoint{1.384091in}{3.019391in}}%
\pgfpathcurveto{\pgfqpoint{1.373041in}{3.019391in}}{\pgfqpoint{1.362442in}{3.015000in}}{\pgfqpoint{1.354628in}{3.007187in}}%
\pgfpathcurveto{\pgfqpoint{1.346815in}{2.999373in}}{\pgfqpoint{1.342424in}{2.988774in}}{\pgfqpoint{1.342424in}{2.977724in}}%
\pgfpathcurveto{\pgfqpoint{1.342424in}{2.966674in}}{\pgfqpoint{1.346815in}{2.956075in}}{\pgfqpoint{1.354628in}{2.948261in}}%
\pgfpathcurveto{\pgfqpoint{1.362442in}{2.940448in}}{\pgfqpoint{1.373041in}{2.936057in}}{\pgfqpoint{1.384091in}{2.936057in}}%
\pgfpathclose%
\pgfusepath{stroke,fill}%
\end{pgfscope}%
\begin{pgfscope}%
\pgfpathrectangle{\pgfqpoint{0.750000in}{0.500000in}}{\pgfqpoint{4.650000in}{3.020000in}}%
\pgfusepath{clip}%
\pgfsetbuttcap%
\pgfsetroundjoin%
\definecolor{currentfill}{rgb}{1.000000,0.498039,0.054902}%
\pgfsetfillcolor{currentfill}%
\pgfsetlinewidth{1.003750pt}%
\definecolor{currentstroke}{rgb}{1.000000,0.498039,0.054902}%
\pgfsetstrokecolor{currentstroke}%
\pgfsetdash{}{0pt}%
\pgfpathmoveto{\pgfqpoint{1.323701in}{3.239810in}}%
\pgfpathcurveto{\pgfqpoint{1.334751in}{3.239810in}}{\pgfqpoint{1.345350in}{3.244200in}}{\pgfqpoint{1.353164in}{3.252014in}}%
\pgfpathcurveto{\pgfqpoint{1.360978in}{3.259827in}}{\pgfqpoint{1.365368in}{3.270426in}}{\pgfqpoint{1.365368in}{3.281476in}}%
\pgfpathcurveto{\pgfqpoint{1.365368in}{3.292527in}}{\pgfqpoint{1.360978in}{3.303126in}}{\pgfqpoint{1.353164in}{3.310939in}}%
\pgfpathcurveto{\pgfqpoint{1.345350in}{3.318753in}}{\pgfqpoint{1.334751in}{3.323143in}}{\pgfqpoint{1.323701in}{3.323143in}}%
\pgfpathcurveto{\pgfqpoint{1.312651in}{3.323143in}}{\pgfqpoint{1.302052in}{3.318753in}}{\pgfqpoint{1.294239in}{3.310939in}}%
\pgfpathcurveto{\pgfqpoint{1.286425in}{3.303126in}}{\pgfqpoint{1.282035in}{3.292527in}}{\pgfqpoint{1.282035in}{3.281476in}}%
\pgfpathcurveto{\pgfqpoint{1.282035in}{3.270426in}}{\pgfqpoint{1.286425in}{3.259827in}}{\pgfqpoint{1.294239in}{3.252014in}}%
\pgfpathcurveto{\pgfqpoint{1.302052in}{3.244200in}}{\pgfqpoint{1.312651in}{3.239810in}}{\pgfqpoint{1.323701in}{3.239810in}}%
\pgfpathclose%
\pgfusepath{stroke,fill}%
\end{pgfscope}%
\begin{pgfscope}%
\pgfpathrectangle{\pgfqpoint{0.750000in}{0.500000in}}{\pgfqpoint{4.650000in}{3.020000in}}%
\pgfusepath{clip}%
\pgfsetbuttcap%
\pgfsetroundjoin%
\definecolor{currentfill}{rgb}{1.000000,0.498039,0.054902}%
\pgfsetfillcolor{currentfill}%
\pgfsetlinewidth{1.003750pt}%
\definecolor{currentstroke}{rgb}{1.000000,0.498039,0.054902}%
\pgfsetstrokecolor{currentstroke}%
\pgfsetdash{}{0pt}%
\pgfpathmoveto{\pgfqpoint{1.625649in}{2.936057in}}%
\pgfpathcurveto{\pgfqpoint{1.636699in}{2.936057in}}{\pgfqpoint{1.647299in}{2.940448in}}{\pgfqpoint{1.655112in}{2.948261in}}%
\pgfpathcurveto{\pgfqpoint{1.662926in}{2.956075in}}{\pgfqpoint{1.667316in}{2.966674in}}{\pgfqpoint{1.667316in}{2.977724in}}%
\pgfpathcurveto{\pgfqpoint{1.667316in}{2.988774in}}{\pgfqpoint{1.662926in}{2.999373in}}{\pgfqpoint{1.655112in}{3.007187in}}%
\pgfpathcurveto{\pgfqpoint{1.647299in}{3.015000in}}{\pgfqpoint{1.636699in}{3.019391in}}{\pgfqpoint{1.625649in}{3.019391in}}%
\pgfpathcurveto{\pgfqpoint{1.614599in}{3.019391in}}{\pgfqpoint{1.604000in}{3.015000in}}{\pgfqpoint{1.596187in}{3.007187in}}%
\pgfpathcurveto{\pgfqpoint{1.588373in}{2.999373in}}{\pgfqpoint{1.583983in}{2.988774in}}{\pgfqpoint{1.583983in}{2.977724in}}%
\pgfpathcurveto{\pgfqpoint{1.583983in}{2.966674in}}{\pgfqpoint{1.588373in}{2.956075in}}{\pgfqpoint{1.596187in}{2.948261in}}%
\pgfpathcurveto{\pgfqpoint{1.604000in}{2.940448in}}{\pgfqpoint{1.614599in}{2.936057in}}{\pgfqpoint{1.625649in}{2.936057in}}%
\pgfpathclose%
\pgfusepath{stroke,fill}%
\end{pgfscope}%
\begin{pgfscope}%
\pgfpathrectangle{\pgfqpoint{0.750000in}{0.500000in}}{\pgfqpoint{4.650000in}{3.020000in}}%
\pgfusepath{clip}%
\pgfsetbuttcap%
\pgfsetroundjoin%
\definecolor{currentfill}{rgb}{0.121569,0.466667,0.705882}%
\pgfsetfillcolor{currentfill}%
\pgfsetlinewidth{1.003750pt}%
\definecolor{currentstroke}{rgb}{0.121569,0.466667,0.705882}%
\pgfsetstrokecolor{currentstroke}%
\pgfsetdash{}{0pt}%
\pgfpathmoveto{\pgfqpoint{1.263312in}{0.595606in}}%
\pgfpathcurveto{\pgfqpoint{1.274362in}{0.595606in}}{\pgfqpoint{1.284961in}{0.599996in}}{\pgfqpoint{1.292774in}{0.607810in}}%
\pgfpathcurveto{\pgfqpoint{1.300588in}{0.615624in}}{\pgfqpoint{1.304978in}{0.626223in}}{\pgfqpoint{1.304978in}{0.637273in}}%
\pgfpathcurveto{\pgfqpoint{1.304978in}{0.648323in}}{\pgfqpoint{1.300588in}{0.658922in}}{\pgfqpoint{1.292774in}{0.666736in}}%
\pgfpathcurveto{\pgfqpoint{1.284961in}{0.674549in}}{\pgfqpoint{1.274362in}{0.678939in}}{\pgfqpoint{1.263312in}{0.678939in}}%
\pgfpathcurveto{\pgfqpoint{1.252262in}{0.678939in}}{\pgfqpoint{1.241663in}{0.674549in}}{\pgfqpoint{1.233849in}{0.666736in}}%
\pgfpathcurveto{\pgfqpoint{1.226035in}{0.658922in}}{\pgfqpoint{1.221645in}{0.648323in}}{\pgfqpoint{1.221645in}{0.637273in}}%
\pgfpathcurveto{\pgfqpoint{1.221645in}{0.626223in}}{\pgfqpoint{1.226035in}{0.615624in}}{\pgfqpoint{1.233849in}{0.607810in}}%
\pgfpathcurveto{\pgfqpoint{1.241663in}{0.599996in}}{\pgfqpoint{1.252262in}{0.595606in}}{\pgfqpoint{1.263312in}{0.595606in}}%
\pgfpathclose%
\pgfusepath{stroke,fill}%
\end{pgfscope}%
\begin{pgfscope}%
\pgfpathrectangle{\pgfqpoint{0.750000in}{0.500000in}}{\pgfqpoint{4.650000in}{3.020000in}}%
\pgfusepath{clip}%
\pgfsetbuttcap%
\pgfsetroundjoin%
\definecolor{currentfill}{rgb}{0.121569,0.466667,0.705882}%
\pgfsetfillcolor{currentfill}%
\pgfsetlinewidth{1.003750pt}%
\definecolor{currentstroke}{rgb}{0.121569,0.466667,0.705882}%
\pgfsetstrokecolor{currentstroke}%
\pgfsetdash{}{0pt}%
\pgfpathmoveto{\pgfqpoint{1.384091in}{0.595606in}}%
\pgfpathcurveto{\pgfqpoint{1.395141in}{0.595606in}}{\pgfqpoint{1.405740in}{0.599996in}}{\pgfqpoint{1.413554in}{0.607810in}}%
\pgfpathcurveto{\pgfqpoint{1.421367in}{0.615624in}}{\pgfqpoint{1.425758in}{0.626223in}}{\pgfqpoint{1.425758in}{0.637273in}}%
\pgfpathcurveto{\pgfqpoint{1.425758in}{0.648323in}}{\pgfqpoint{1.421367in}{0.658922in}}{\pgfqpoint{1.413554in}{0.666736in}}%
\pgfpathcurveto{\pgfqpoint{1.405740in}{0.674549in}}{\pgfqpoint{1.395141in}{0.678939in}}{\pgfqpoint{1.384091in}{0.678939in}}%
\pgfpathcurveto{\pgfqpoint{1.373041in}{0.678939in}}{\pgfqpoint{1.362442in}{0.674549in}}{\pgfqpoint{1.354628in}{0.666736in}}%
\pgfpathcurveto{\pgfqpoint{1.346815in}{0.658922in}}{\pgfqpoint{1.342424in}{0.648323in}}{\pgfqpoint{1.342424in}{0.637273in}}%
\pgfpathcurveto{\pgfqpoint{1.342424in}{0.626223in}}{\pgfqpoint{1.346815in}{0.615624in}}{\pgfqpoint{1.354628in}{0.607810in}}%
\pgfpathcurveto{\pgfqpoint{1.362442in}{0.599996in}}{\pgfqpoint{1.373041in}{0.595606in}}{\pgfqpoint{1.384091in}{0.595606in}}%
\pgfpathclose%
\pgfusepath{stroke,fill}%
\end{pgfscope}%
\begin{pgfscope}%
\pgfpathrectangle{\pgfqpoint{0.750000in}{0.500000in}}{\pgfqpoint{4.650000in}{3.020000in}}%
\pgfusepath{clip}%
\pgfsetbuttcap%
\pgfsetroundjoin%
\definecolor{currentfill}{rgb}{0.121569,0.466667,0.705882}%
\pgfsetfillcolor{currentfill}%
\pgfsetlinewidth{1.003750pt}%
\definecolor{currentstroke}{rgb}{0.121569,0.466667,0.705882}%
\pgfsetstrokecolor{currentstroke}%
\pgfsetdash{}{0pt}%
\pgfpathmoveto{\pgfqpoint{1.867208in}{0.665703in}}%
\pgfpathcurveto{\pgfqpoint{1.878258in}{0.665703in}}{\pgfqpoint{1.888857in}{0.670093in}}{\pgfqpoint{1.896671in}{0.677907in}}%
\pgfpathcurveto{\pgfqpoint{1.904484in}{0.685720in}}{\pgfqpoint{1.908874in}{0.696319in}}{\pgfqpoint{1.908874in}{0.707369in}}%
\pgfpathcurveto{\pgfqpoint{1.908874in}{0.718420in}}{\pgfqpoint{1.904484in}{0.729019in}}{\pgfqpoint{1.896671in}{0.736832in}}%
\pgfpathcurveto{\pgfqpoint{1.888857in}{0.744646in}}{\pgfqpoint{1.878258in}{0.749036in}}{\pgfqpoint{1.867208in}{0.749036in}}%
\pgfpathcurveto{\pgfqpoint{1.856158in}{0.749036in}}{\pgfqpoint{1.845559in}{0.744646in}}{\pgfqpoint{1.837745in}{0.736832in}}%
\pgfpathcurveto{\pgfqpoint{1.829931in}{0.729019in}}{\pgfqpoint{1.825541in}{0.718420in}}{\pgfqpoint{1.825541in}{0.707369in}}%
\pgfpathcurveto{\pgfqpoint{1.825541in}{0.696319in}}{\pgfqpoint{1.829931in}{0.685720in}}{\pgfqpoint{1.837745in}{0.677907in}}%
\pgfpathcurveto{\pgfqpoint{1.845559in}{0.670093in}}{\pgfqpoint{1.856158in}{0.665703in}}{\pgfqpoint{1.867208in}{0.665703in}}%
\pgfpathclose%
\pgfusepath{stroke,fill}%
\end{pgfscope}%
\begin{pgfscope}%
\pgfpathrectangle{\pgfqpoint{0.750000in}{0.500000in}}{\pgfqpoint{4.650000in}{3.020000in}}%
\pgfusepath{clip}%
\pgfsetbuttcap%
\pgfsetroundjoin%
\definecolor{currentfill}{rgb}{1.000000,0.498039,0.054902}%
\pgfsetfillcolor{currentfill}%
\pgfsetlinewidth{1.003750pt}%
\definecolor{currentstroke}{rgb}{1.000000,0.498039,0.054902}%
\pgfsetstrokecolor{currentstroke}%
\pgfsetdash{}{0pt}%
\pgfpathmoveto{\pgfqpoint{1.444481in}{2.932163in}}%
\pgfpathcurveto{\pgfqpoint{1.455531in}{2.932163in}}{\pgfqpoint{1.466130in}{2.936553in}}{\pgfqpoint{1.473943in}{2.944367in}}%
\pgfpathcurveto{\pgfqpoint{1.481757in}{2.952181in}}{\pgfqpoint{1.486147in}{2.962780in}}{\pgfqpoint{1.486147in}{2.973830in}}%
\pgfpathcurveto{\pgfqpoint{1.486147in}{2.984880in}}{\pgfqpoint{1.481757in}{2.995479in}}{\pgfqpoint{1.473943in}{3.003293in}}%
\pgfpathcurveto{\pgfqpoint{1.466130in}{3.011106in}}{\pgfqpoint{1.455531in}{3.015496in}}{\pgfqpoint{1.444481in}{3.015496in}}%
\pgfpathcurveto{\pgfqpoint{1.433430in}{3.015496in}}{\pgfqpoint{1.422831in}{3.011106in}}{\pgfqpoint{1.415018in}{3.003293in}}%
\pgfpathcurveto{\pgfqpoint{1.407204in}{2.995479in}}{\pgfqpoint{1.402814in}{2.984880in}}{\pgfqpoint{1.402814in}{2.973830in}}%
\pgfpathcurveto{\pgfqpoint{1.402814in}{2.962780in}}{\pgfqpoint{1.407204in}{2.952181in}}{\pgfqpoint{1.415018in}{2.944367in}}%
\pgfpathcurveto{\pgfqpoint{1.422831in}{2.936553in}}{\pgfqpoint{1.433430in}{2.932163in}}{\pgfqpoint{1.444481in}{2.932163in}}%
\pgfpathclose%
\pgfusepath{stroke,fill}%
\end{pgfscope}%
\begin{pgfscope}%
\pgfpathrectangle{\pgfqpoint{0.750000in}{0.500000in}}{\pgfqpoint{4.650000in}{3.020000in}}%
\pgfusepath{clip}%
\pgfsetbuttcap%
\pgfsetroundjoin%
\definecolor{currentfill}{rgb}{0.839216,0.152941,0.156863}%
\pgfsetfillcolor{currentfill}%
\pgfsetlinewidth{1.003750pt}%
\definecolor{currentstroke}{rgb}{0.839216,0.152941,0.156863}%
\pgfsetstrokecolor{currentstroke}%
\pgfsetdash{}{0pt}%
\pgfpathmoveto{\pgfqpoint{2.169156in}{1.086283in}}%
\pgfpathcurveto{\pgfqpoint{2.180206in}{1.086283in}}{\pgfqpoint{2.190805in}{1.090673in}}{\pgfqpoint{2.198619in}{1.098487in}}%
\pgfpathcurveto{\pgfqpoint{2.206432in}{1.106301in}}{\pgfqpoint{2.210823in}{1.116900in}}{\pgfqpoint{2.210823in}{1.127950in}}%
\pgfpathcurveto{\pgfqpoint{2.210823in}{1.139000in}}{\pgfqpoint{2.206432in}{1.149599in}}{\pgfqpoint{2.198619in}{1.157412in}}%
\pgfpathcurveto{\pgfqpoint{2.190805in}{1.165226in}}{\pgfqpoint{2.180206in}{1.169616in}}{\pgfqpoint{2.169156in}{1.169616in}}%
\pgfpathcurveto{\pgfqpoint{2.158106in}{1.169616in}}{\pgfqpoint{2.147507in}{1.165226in}}{\pgfqpoint{2.139693in}{1.157412in}}%
\pgfpathcurveto{\pgfqpoint{2.131879in}{1.149599in}}{\pgfqpoint{2.127489in}{1.139000in}}{\pgfqpoint{2.127489in}{1.127950in}}%
\pgfpathcurveto{\pgfqpoint{2.127489in}{1.116900in}}{\pgfqpoint{2.131879in}{1.106301in}}{\pgfqpoint{2.139693in}{1.098487in}}%
\pgfpathcurveto{\pgfqpoint{2.147507in}{1.090673in}}{\pgfqpoint{2.158106in}{1.086283in}}{\pgfqpoint{2.169156in}{1.086283in}}%
\pgfpathclose%
\pgfusepath{stroke,fill}%
\end{pgfscope}%
\begin{pgfscope}%
\pgfpathrectangle{\pgfqpoint{0.750000in}{0.500000in}}{\pgfqpoint{4.650000in}{3.020000in}}%
\pgfusepath{clip}%
\pgfsetbuttcap%
\pgfsetroundjoin%
\definecolor{currentfill}{rgb}{1.000000,0.498039,0.054902}%
\pgfsetfillcolor{currentfill}%
\pgfsetlinewidth{1.003750pt}%
\definecolor{currentstroke}{rgb}{1.000000,0.498039,0.054902}%
\pgfsetstrokecolor{currentstroke}%
\pgfsetdash{}{0pt}%
\pgfpathmoveto{\pgfqpoint{2.048377in}{2.932163in}}%
\pgfpathcurveto{\pgfqpoint{2.059427in}{2.932163in}}{\pgfqpoint{2.070026in}{2.936553in}}{\pgfqpoint{2.077839in}{2.944367in}}%
\pgfpathcurveto{\pgfqpoint{2.085653in}{2.952181in}}{\pgfqpoint{2.090043in}{2.962780in}}{\pgfqpoint{2.090043in}{2.973830in}}%
\pgfpathcurveto{\pgfqpoint{2.090043in}{2.984880in}}{\pgfqpoint{2.085653in}{2.995479in}}{\pgfqpoint{2.077839in}{3.003293in}}%
\pgfpathcurveto{\pgfqpoint{2.070026in}{3.011106in}}{\pgfqpoint{2.059427in}{3.015496in}}{\pgfqpoint{2.048377in}{3.015496in}}%
\pgfpathcurveto{\pgfqpoint{2.037326in}{3.015496in}}{\pgfqpoint{2.026727in}{3.011106in}}{\pgfqpoint{2.018914in}{3.003293in}}%
\pgfpathcurveto{\pgfqpoint{2.011100in}{2.995479in}}{\pgfqpoint{2.006710in}{2.984880in}}{\pgfqpoint{2.006710in}{2.973830in}}%
\pgfpathcurveto{\pgfqpoint{2.006710in}{2.962780in}}{\pgfqpoint{2.011100in}{2.952181in}}{\pgfqpoint{2.018914in}{2.944367in}}%
\pgfpathcurveto{\pgfqpoint{2.026727in}{2.936553in}}{\pgfqpoint{2.037326in}{2.932163in}}{\pgfqpoint{2.048377in}{2.932163in}}%
\pgfpathclose%
\pgfusepath{stroke,fill}%
\end{pgfscope}%
\begin{pgfscope}%
\pgfpathrectangle{\pgfqpoint{0.750000in}{0.500000in}}{\pgfqpoint{4.650000in}{3.020000in}}%
\pgfusepath{clip}%
\pgfsetbuttcap%
\pgfsetroundjoin%
\definecolor{currentfill}{rgb}{1.000000,0.498039,0.054902}%
\pgfsetfillcolor{currentfill}%
\pgfsetlinewidth{1.003750pt}%
\definecolor{currentstroke}{rgb}{1.000000,0.498039,0.054902}%
\pgfsetstrokecolor{currentstroke}%
\pgfsetdash{}{0pt}%
\pgfpathmoveto{\pgfqpoint{1.987987in}{2.932163in}}%
\pgfpathcurveto{\pgfqpoint{1.999037in}{2.932163in}}{\pgfqpoint{2.009636in}{2.936553in}}{\pgfqpoint{2.017450in}{2.944367in}}%
\pgfpathcurveto{\pgfqpoint{2.025263in}{2.952181in}}{\pgfqpoint{2.029654in}{2.962780in}}{\pgfqpoint{2.029654in}{2.973830in}}%
\pgfpathcurveto{\pgfqpoint{2.029654in}{2.984880in}}{\pgfqpoint{2.025263in}{2.995479in}}{\pgfqpoint{2.017450in}{3.003293in}}%
\pgfpathcurveto{\pgfqpoint{2.009636in}{3.011106in}}{\pgfqpoint{1.999037in}{3.015496in}}{\pgfqpoint{1.987987in}{3.015496in}}%
\pgfpathcurveto{\pgfqpoint{1.976937in}{3.015496in}}{\pgfqpoint{1.966338in}{3.011106in}}{\pgfqpoint{1.958524in}{3.003293in}}%
\pgfpathcurveto{\pgfqpoint{1.950711in}{2.995479in}}{\pgfqpoint{1.946320in}{2.984880in}}{\pgfqpoint{1.946320in}{2.973830in}}%
\pgfpathcurveto{\pgfqpoint{1.946320in}{2.962780in}}{\pgfqpoint{1.950711in}{2.952181in}}{\pgfqpoint{1.958524in}{2.944367in}}%
\pgfpathcurveto{\pgfqpoint{1.966338in}{2.936553in}}{\pgfqpoint{1.976937in}{2.932163in}}{\pgfqpoint{1.987987in}{2.932163in}}%
\pgfpathclose%
\pgfusepath{stroke,fill}%
\end{pgfscope}%
\begin{pgfscope}%
\pgfpathrectangle{\pgfqpoint{0.750000in}{0.500000in}}{\pgfqpoint{4.650000in}{3.020000in}}%
\pgfusepath{clip}%
\pgfsetbuttcap%
\pgfsetroundjoin%
\definecolor{currentfill}{rgb}{1.000000,0.498039,0.054902}%
\pgfsetfillcolor{currentfill}%
\pgfsetlinewidth{1.003750pt}%
\definecolor{currentstroke}{rgb}{1.000000,0.498039,0.054902}%
\pgfsetstrokecolor{currentstroke}%
\pgfsetdash{}{0pt}%
\pgfpathmoveto{\pgfqpoint{1.806818in}{2.928269in}}%
\pgfpathcurveto{\pgfqpoint{1.817868in}{2.928269in}}{\pgfqpoint{1.828467in}{2.932659in}}{\pgfqpoint{1.836281in}{2.940473in}}%
\pgfpathcurveto{\pgfqpoint{1.844095in}{2.948286in}}{\pgfqpoint{1.848485in}{2.958885in}}{\pgfqpoint{1.848485in}{2.969936in}}%
\pgfpathcurveto{\pgfqpoint{1.848485in}{2.980986in}}{\pgfqpoint{1.844095in}{2.991585in}}{\pgfqpoint{1.836281in}{2.999398in}}%
\pgfpathcurveto{\pgfqpoint{1.828467in}{3.007212in}}{\pgfqpoint{1.817868in}{3.011602in}}{\pgfqpoint{1.806818in}{3.011602in}}%
\pgfpathcurveto{\pgfqpoint{1.795768in}{3.011602in}}{\pgfqpoint{1.785169in}{3.007212in}}{\pgfqpoint{1.777355in}{2.999398in}}%
\pgfpathcurveto{\pgfqpoint{1.769542in}{2.991585in}}{\pgfqpoint{1.765152in}{2.980986in}}{\pgfqpoint{1.765152in}{2.969936in}}%
\pgfpathcurveto{\pgfqpoint{1.765152in}{2.958885in}}{\pgfqpoint{1.769542in}{2.948286in}}{\pgfqpoint{1.777355in}{2.940473in}}%
\pgfpathcurveto{\pgfqpoint{1.785169in}{2.932659in}}{\pgfqpoint{1.795768in}{2.928269in}}{\pgfqpoint{1.806818in}{2.928269in}}%
\pgfpathclose%
\pgfusepath{stroke,fill}%
\end{pgfscope}%
\begin{pgfscope}%
\pgfpathrectangle{\pgfqpoint{0.750000in}{0.500000in}}{\pgfqpoint{4.650000in}{3.020000in}}%
\pgfusepath{clip}%
\pgfsetbuttcap%
\pgfsetroundjoin%
\definecolor{currentfill}{rgb}{1.000000,0.498039,0.054902}%
\pgfsetfillcolor{currentfill}%
\pgfsetlinewidth{1.003750pt}%
\definecolor{currentstroke}{rgb}{1.000000,0.498039,0.054902}%
\pgfsetstrokecolor{currentstroke}%
\pgfsetdash{}{0pt}%
\pgfpathmoveto{\pgfqpoint{1.867208in}{2.936057in}}%
\pgfpathcurveto{\pgfqpoint{1.878258in}{2.936057in}}{\pgfqpoint{1.888857in}{2.940448in}}{\pgfqpoint{1.896671in}{2.948261in}}%
\pgfpathcurveto{\pgfqpoint{1.904484in}{2.956075in}}{\pgfqpoint{1.908874in}{2.966674in}}{\pgfqpoint{1.908874in}{2.977724in}}%
\pgfpathcurveto{\pgfqpoint{1.908874in}{2.988774in}}{\pgfqpoint{1.904484in}{2.999373in}}{\pgfqpoint{1.896671in}{3.007187in}}%
\pgfpathcurveto{\pgfqpoint{1.888857in}{3.015000in}}{\pgfqpoint{1.878258in}{3.019391in}}{\pgfqpoint{1.867208in}{3.019391in}}%
\pgfpathcurveto{\pgfqpoint{1.856158in}{3.019391in}}{\pgfqpoint{1.845559in}{3.015000in}}{\pgfqpoint{1.837745in}{3.007187in}}%
\pgfpathcurveto{\pgfqpoint{1.829931in}{2.999373in}}{\pgfqpoint{1.825541in}{2.988774in}}{\pgfqpoint{1.825541in}{2.977724in}}%
\pgfpathcurveto{\pgfqpoint{1.825541in}{2.966674in}}{\pgfqpoint{1.829931in}{2.956075in}}{\pgfqpoint{1.837745in}{2.948261in}}%
\pgfpathcurveto{\pgfqpoint{1.845559in}{2.940448in}}{\pgfqpoint{1.856158in}{2.936057in}}{\pgfqpoint{1.867208in}{2.936057in}}%
\pgfpathclose%
\pgfusepath{stroke,fill}%
\end{pgfscope}%
\begin{pgfscope}%
\pgfpathrectangle{\pgfqpoint{0.750000in}{0.500000in}}{\pgfqpoint{4.650000in}{3.020000in}}%
\pgfusepath{clip}%
\pgfsetbuttcap%
\pgfsetroundjoin%
\definecolor{currentfill}{rgb}{1.000000,0.498039,0.054902}%
\pgfsetfillcolor{currentfill}%
\pgfsetlinewidth{1.003750pt}%
\definecolor{currentstroke}{rgb}{1.000000,0.498039,0.054902}%
\pgfsetstrokecolor{currentstroke}%
\pgfsetdash{}{0pt}%
\pgfpathmoveto{\pgfqpoint{1.867208in}{2.939952in}}%
\pgfpathcurveto{\pgfqpoint{1.878258in}{2.939952in}}{\pgfqpoint{1.888857in}{2.944342in}}{\pgfqpoint{1.896671in}{2.952156in}}%
\pgfpathcurveto{\pgfqpoint{1.904484in}{2.959969in}}{\pgfqpoint{1.908874in}{2.970568in}}{\pgfqpoint{1.908874in}{2.981618in}}%
\pgfpathcurveto{\pgfqpoint{1.908874in}{2.992668in}}{\pgfqpoint{1.904484in}{3.003267in}}{\pgfqpoint{1.896671in}{3.011081in}}%
\pgfpathcurveto{\pgfqpoint{1.888857in}{3.018895in}}{\pgfqpoint{1.878258in}{3.023285in}}{\pgfqpoint{1.867208in}{3.023285in}}%
\pgfpathcurveto{\pgfqpoint{1.856158in}{3.023285in}}{\pgfqpoint{1.845559in}{3.018895in}}{\pgfqpoint{1.837745in}{3.011081in}}%
\pgfpathcurveto{\pgfqpoint{1.829931in}{3.003267in}}{\pgfqpoint{1.825541in}{2.992668in}}{\pgfqpoint{1.825541in}{2.981618in}}%
\pgfpathcurveto{\pgfqpoint{1.825541in}{2.970568in}}{\pgfqpoint{1.829931in}{2.959969in}}{\pgfqpoint{1.837745in}{2.952156in}}%
\pgfpathcurveto{\pgfqpoint{1.845559in}{2.944342in}}{\pgfqpoint{1.856158in}{2.939952in}}{\pgfqpoint{1.867208in}{2.939952in}}%
\pgfpathclose%
\pgfusepath{stroke,fill}%
\end{pgfscope}%
\begin{pgfscope}%
\pgfpathrectangle{\pgfqpoint{0.750000in}{0.500000in}}{\pgfqpoint{4.650000in}{3.020000in}}%
\pgfusepath{clip}%
\pgfsetbuttcap%
\pgfsetroundjoin%
\definecolor{currentfill}{rgb}{1.000000,0.498039,0.054902}%
\pgfsetfillcolor{currentfill}%
\pgfsetlinewidth{1.003750pt}%
\definecolor{currentstroke}{rgb}{1.000000,0.498039,0.054902}%
\pgfsetstrokecolor{currentstroke}%
\pgfsetdash{}{0pt}%
\pgfpathmoveto{\pgfqpoint{1.867208in}{2.936057in}}%
\pgfpathcurveto{\pgfqpoint{1.878258in}{2.936057in}}{\pgfqpoint{1.888857in}{2.940448in}}{\pgfqpoint{1.896671in}{2.948261in}}%
\pgfpathcurveto{\pgfqpoint{1.904484in}{2.956075in}}{\pgfqpoint{1.908874in}{2.966674in}}{\pgfqpoint{1.908874in}{2.977724in}}%
\pgfpathcurveto{\pgfqpoint{1.908874in}{2.988774in}}{\pgfqpoint{1.904484in}{2.999373in}}{\pgfqpoint{1.896671in}{3.007187in}}%
\pgfpathcurveto{\pgfqpoint{1.888857in}{3.015000in}}{\pgfqpoint{1.878258in}{3.019391in}}{\pgfqpoint{1.867208in}{3.019391in}}%
\pgfpathcurveto{\pgfqpoint{1.856158in}{3.019391in}}{\pgfqpoint{1.845559in}{3.015000in}}{\pgfqpoint{1.837745in}{3.007187in}}%
\pgfpathcurveto{\pgfqpoint{1.829931in}{2.999373in}}{\pgfqpoint{1.825541in}{2.988774in}}{\pgfqpoint{1.825541in}{2.977724in}}%
\pgfpathcurveto{\pgfqpoint{1.825541in}{2.966674in}}{\pgfqpoint{1.829931in}{2.956075in}}{\pgfqpoint{1.837745in}{2.948261in}}%
\pgfpathcurveto{\pgfqpoint{1.845559in}{2.940448in}}{\pgfqpoint{1.856158in}{2.936057in}}{\pgfqpoint{1.867208in}{2.936057in}}%
\pgfpathclose%
\pgfusepath{stroke,fill}%
\end{pgfscope}%
\begin{pgfscope}%
\pgfpathrectangle{\pgfqpoint{0.750000in}{0.500000in}}{\pgfqpoint{4.650000in}{3.020000in}}%
\pgfusepath{clip}%
\pgfsetbuttcap%
\pgfsetroundjoin%
\definecolor{currentfill}{rgb}{1.000000,0.498039,0.054902}%
\pgfsetfillcolor{currentfill}%
\pgfsetlinewidth{1.003750pt}%
\definecolor{currentstroke}{rgb}{1.000000,0.498039,0.054902}%
\pgfsetstrokecolor{currentstroke}%
\pgfsetdash{}{0pt}%
\pgfpathmoveto{\pgfqpoint{2.773052in}{2.936057in}}%
\pgfpathcurveto{\pgfqpoint{2.784102in}{2.936057in}}{\pgfqpoint{2.794701in}{2.940448in}}{\pgfqpoint{2.802515in}{2.948261in}}%
\pgfpathcurveto{\pgfqpoint{2.810328in}{2.956075in}}{\pgfqpoint{2.814719in}{2.966674in}}{\pgfqpoint{2.814719in}{2.977724in}}%
\pgfpathcurveto{\pgfqpoint{2.814719in}{2.988774in}}{\pgfqpoint{2.810328in}{2.999373in}}{\pgfqpoint{2.802515in}{3.007187in}}%
\pgfpathcurveto{\pgfqpoint{2.794701in}{3.015000in}}{\pgfqpoint{2.784102in}{3.019391in}}{\pgfqpoint{2.773052in}{3.019391in}}%
\pgfpathcurveto{\pgfqpoint{2.762002in}{3.019391in}}{\pgfqpoint{2.751403in}{3.015000in}}{\pgfqpoint{2.743589in}{3.007187in}}%
\pgfpathcurveto{\pgfqpoint{2.735776in}{2.999373in}}{\pgfqpoint{2.731385in}{2.988774in}}{\pgfqpoint{2.731385in}{2.977724in}}%
\pgfpathcurveto{\pgfqpoint{2.731385in}{2.966674in}}{\pgfqpoint{2.735776in}{2.956075in}}{\pgfqpoint{2.743589in}{2.948261in}}%
\pgfpathcurveto{\pgfqpoint{2.751403in}{2.940448in}}{\pgfqpoint{2.762002in}{2.936057in}}{\pgfqpoint{2.773052in}{2.936057in}}%
\pgfpathclose%
\pgfusepath{stroke,fill}%
\end{pgfscope}%
\begin{pgfscope}%
\pgfpathrectangle{\pgfqpoint{0.750000in}{0.500000in}}{\pgfqpoint{4.650000in}{3.020000in}}%
\pgfusepath{clip}%
\pgfsetbuttcap%
\pgfsetroundjoin%
\definecolor{currentfill}{rgb}{1.000000,0.498039,0.054902}%
\pgfsetfillcolor{currentfill}%
\pgfsetlinewidth{1.003750pt}%
\definecolor{currentstroke}{rgb}{1.000000,0.498039,0.054902}%
\pgfsetstrokecolor{currentstroke}%
\pgfsetdash{}{0pt}%
\pgfpathmoveto{\pgfqpoint{3.980844in}{2.924375in}}%
\pgfpathcurveto{\pgfqpoint{3.991894in}{2.924375in}}{\pgfqpoint{4.002493in}{2.928765in}}{\pgfqpoint{4.010307in}{2.936578in}}%
\pgfpathcurveto{\pgfqpoint{4.018121in}{2.944392in}}{\pgfqpoint{4.022511in}{2.954991in}}{\pgfqpoint{4.022511in}{2.966041in}}%
\pgfpathcurveto{\pgfqpoint{4.022511in}{2.977091in}}{\pgfqpoint{4.018121in}{2.987690in}}{\pgfqpoint{4.010307in}{2.995504in}}%
\pgfpathcurveto{\pgfqpoint{4.002493in}{3.003318in}}{\pgfqpoint{3.991894in}{3.007708in}}{\pgfqpoint{3.980844in}{3.007708in}}%
\pgfpathcurveto{\pgfqpoint{3.969794in}{3.007708in}}{\pgfqpoint{3.959195in}{3.003318in}}{\pgfqpoint{3.951381in}{2.995504in}}%
\pgfpathcurveto{\pgfqpoint{3.943568in}{2.987690in}}{\pgfqpoint{3.939177in}{2.977091in}}{\pgfqpoint{3.939177in}{2.966041in}}%
\pgfpathcurveto{\pgfqpoint{3.939177in}{2.954991in}}{\pgfqpoint{3.943568in}{2.944392in}}{\pgfqpoint{3.951381in}{2.936578in}}%
\pgfpathcurveto{\pgfqpoint{3.959195in}{2.928765in}}{\pgfqpoint{3.969794in}{2.924375in}}{\pgfqpoint{3.980844in}{2.924375in}}%
\pgfpathclose%
\pgfusepath{stroke,fill}%
\end{pgfscope}%
\begin{pgfscope}%
\pgfpathrectangle{\pgfqpoint{0.750000in}{0.500000in}}{\pgfqpoint{4.650000in}{3.020000in}}%
\pgfusepath{clip}%
\pgfsetbuttcap%
\pgfsetroundjoin%
\definecolor{currentfill}{rgb}{1.000000,0.498039,0.054902}%
\pgfsetfillcolor{currentfill}%
\pgfsetlinewidth{1.003750pt}%
\definecolor{currentstroke}{rgb}{1.000000,0.498039,0.054902}%
\pgfsetstrokecolor{currentstroke}%
\pgfsetdash{}{0pt}%
\pgfpathmoveto{\pgfqpoint{1.867208in}{2.932163in}}%
\pgfpathcurveto{\pgfqpoint{1.878258in}{2.932163in}}{\pgfqpoint{1.888857in}{2.936553in}}{\pgfqpoint{1.896671in}{2.944367in}}%
\pgfpathcurveto{\pgfqpoint{1.904484in}{2.952181in}}{\pgfqpoint{1.908874in}{2.962780in}}{\pgfqpoint{1.908874in}{2.973830in}}%
\pgfpathcurveto{\pgfqpoint{1.908874in}{2.984880in}}{\pgfqpoint{1.904484in}{2.995479in}}{\pgfqpoint{1.896671in}{3.003293in}}%
\pgfpathcurveto{\pgfqpoint{1.888857in}{3.011106in}}{\pgfqpoint{1.878258in}{3.015496in}}{\pgfqpoint{1.867208in}{3.015496in}}%
\pgfpathcurveto{\pgfqpoint{1.856158in}{3.015496in}}{\pgfqpoint{1.845559in}{3.011106in}}{\pgfqpoint{1.837745in}{3.003293in}}%
\pgfpathcurveto{\pgfqpoint{1.829931in}{2.995479in}}{\pgfqpoint{1.825541in}{2.984880in}}{\pgfqpoint{1.825541in}{2.973830in}}%
\pgfpathcurveto{\pgfqpoint{1.825541in}{2.962780in}}{\pgfqpoint{1.829931in}{2.952181in}}{\pgfqpoint{1.837745in}{2.944367in}}%
\pgfpathcurveto{\pgfqpoint{1.845559in}{2.936553in}}{\pgfqpoint{1.856158in}{2.932163in}}{\pgfqpoint{1.867208in}{2.932163in}}%
\pgfpathclose%
\pgfusepath{stroke,fill}%
\end{pgfscope}%
\begin{pgfscope}%
\pgfpathrectangle{\pgfqpoint{0.750000in}{0.500000in}}{\pgfqpoint{4.650000in}{3.020000in}}%
\pgfusepath{clip}%
\pgfsetbuttcap%
\pgfsetroundjoin%
\definecolor{currentfill}{rgb}{1.000000,0.498039,0.054902}%
\pgfsetfillcolor{currentfill}%
\pgfsetlinewidth{1.003750pt}%
\definecolor{currentstroke}{rgb}{1.000000,0.498039,0.054902}%
\pgfsetstrokecolor{currentstroke}%
\pgfsetdash{}{0pt}%
\pgfpathmoveto{\pgfqpoint{1.746429in}{2.936057in}}%
\pgfpathcurveto{\pgfqpoint{1.757479in}{2.936057in}}{\pgfqpoint{1.768078in}{2.940448in}}{\pgfqpoint{1.775891in}{2.948261in}}%
\pgfpathcurveto{\pgfqpoint{1.783705in}{2.956075in}}{\pgfqpoint{1.788095in}{2.966674in}}{\pgfqpoint{1.788095in}{2.977724in}}%
\pgfpathcurveto{\pgfqpoint{1.788095in}{2.988774in}}{\pgfqpoint{1.783705in}{2.999373in}}{\pgfqpoint{1.775891in}{3.007187in}}%
\pgfpathcurveto{\pgfqpoint{1.768078in}{3.015000in}}{\pgfqpoint{1.757479in}{3.019391in}}{\pgfqpoint{1.746429in}{3.019391in}}%
\pgfpathcurveto{\pgfqpoint{1.735378in}{3.019391in}}{\pgfqpoint{1.724779in}{3.015000in}}{\pgfqpoint{1.716966in}{3.007187in}}%
\pgfpathcurveto{\pgfqpoint{1.709152in}{2.999373in}}{\pgfqpoint{1.704762in}{2.988774in}}{\pgfqpoint{1.704762in}{2.977724in}}%
\pgfpathcurveto{\pgfqpoint{1.704762in}{2.966674in}}{\pgfqpoint{1.709152in}{2.956075in}}{\pgfqpoint{1.716966in}{2.948261in}}%
\pgfpathcurveto{\pgfqpoint{1.724779in}{2.940448in}}{\pgfqpoint{1.735378in}{2.936057in}}{\pgfqpoint{1.746429in}{2.936057in}}%
\pgfpathclose%
\pgfusepath{stroke,fill}%
\end{pgfscope}%
\begin{pgfscope}%
\pgfpathrectangle{\pgfqpoint{0.750000in}{0.500000in}}{\pgfqpoint{4.650000in}{3.020000in}}%
\pgfusepath{clip}%
\pgfsetbuttcap%
\pgfsetroundjoin%
\definecolor{currentfill}{rgb}{1.000000,0.498039,0.054902}%
\pgfsetfillcolor{currentfill}%
\pgfsetlinewidth{1.003750pt}%
\definecolor{currentstroke}{rgb}{1.000000,0.498039,0.054902}%
\pgfsetstrokecolor{currentstroke}%
\pgfsetdash{}{0pt}%
\pgfpathmoveto{\pgfqpoint{1.686039in}{2.928269in}}%
\pgfpathcurveto{\pgfqpoint{1.697089in}{2.928269in}}{\pgfqpoint{1.707688in}{2.932659in}}{\pgfqpoint{1.715502in}{2.940473in}}%
\pgfpathcurveto{\pgfqpoint{1.723315in}{2.948286in}}{\pgfqpoint{1.727706in}{2.958885in}}{\pgfqpoint{1.727706in}{2.969936in}}%
\pgfpathcurveto{\pgfqpoint{1.727706in}{2.980986in}}{\pgfqpoint{1.723315in}{2.991585in}}{\pgfqpoint{1.715502in}{2.999398in}}%
\pgfpathcurveto{\pgfqpoint{1.707688in}{3.007212in}}{\pgfqpoint{1.697089in}{3.011602in}}{\pgfqpoint{1.686039in}{3.011602in}}%
\pgfpathcurveto{\pgfqpoint{1.674989in}{3.011602in}}{\pgfqpoint{1.664390in}{3.007212in}}{\pgfqpoint{1.656576in}{2.999398in}}%
\pgfpathcurveto{\pgfqpoint{1.648763in}{2.991585in}}{\pgfqpoint{1.644372in}{2.980986in}}{\pgfqpoint{1.644372in}{2.969936in}}%
\pgfpathcurveto{\pgfqpoint{1.644372in}{2.958885in}}{\pgfqpoint{1.648763in}{2.948286in}}{\pgfqpoint{1.656576in}{2.940473in}}%
\pgfpathcurveto{\pgfqpoint{1.664390in}{2.932659in}}{\pgfqpoint{1.674989in}{2.928269in}}{\pgfqpoint{1.686039in}{2.928269in}}%
\pgfpathclose%
\pgfusepath{stroke,fill}%
\end{pgfscope}%
\begin{pgfscope}%
\pgfpathrectangle{\pgfqpoint{0.750000in}{0.500000in}}{\pgfqpoint{4.650000in}{3.020000in}}%
\pgfusepath{clip}%
\pgfsetbuttcap%
\pgfsetroundjoin%
\definecolor{currentfill}{rgb}{1.000000,0.498039,0.054902}%
\pgfsetfillcolor{currentfill}%
\pgfsetlinewidth{1.003750pt}%
\definecolor{currentstroke}{rgb}{1.000000,0.498039,0.054902}%
\pgfsetstrokecolor{currentstroke}%
\pgfsetdash{}{0pt}%
\pgfpathmoveto{\pgfqpoint{1.323701in}{2.943846in}}%
\pgfpathcurveto{\pgfqpoint{1.334751in}{2.943846in}}{\pgfqpoint{1.345350in}{2.948236in}}{\pgfqpoint{1.353164in}{2.956050in}}%
\pgfpathcurveto{\pgfqpoint{1.360978in}{2.963863in}}{\pgfqpoint{1.365368in}{2.974462in}}{\pgfqpoint{1.365368in}{2.985513in}}%
\pgfpathcurveto{\pgfqpoint{1.365368in}{2.996563in}}{\pgfqpoint{1.360978in}{3.007162in}}{\pgfqpoint{1.353164in}{3.014975in}}%
\pgfpathcurveto{\pgfqpoint{1.345350in}{3.022789in}}{\pgfqpoint{1.334751in}{3.027179in}}{\pgfqpoint{1.323701in}{3.027179in}}%
\pgfpathcurveto{\pgfqpoint{1.312651in}{3.027179in}}{\pgfqpoint{1.302052in}{3.022789in}}{\pgfqpoint{1.294239in}{3.014975in}}%
\pgfpathcurveto{\pgfqpoint{1.286425in}{3.007162in}}{\pgfqpoint{1.282035in}{2.996563in}}{\pgfqpoint{1.282035in}{2.985513in}}%
\pgfpathcurveto{\pgfqpoint{1.282035in}{2.974462in}}{\pgfqpoint{1.286425in}{2.963863in}}{\pgfqpoint{1.294239in}{2.956050in}}%
\pgfpathcurveto{\pgfqpoint{1.302052in}{2.948236in}}{\pgfqpoint{1.312651in}{2.943846in}}{\pgfqpoint{1.323701in}{2.943846in}}%
\pgfpathclose%
\pgfusepath{stroke,fill}%
\end{pgfscope}%
\begin{pgfscope}%
\pgfpathrectangle{\pgfqpoint{0.750000in}{0.500000in}}{\pgfqpoint{4.650000in}{3.020000in}}%
\pgfusepath{clip}%
\pgfsetbuttcap%
\pgfsetroundjoin%
\definecolor{currentfill}{rgb}{1.000000,0.498039,0.054902}%
\pgfsetfillcolor{currentfill}%
\pgfsetlinewidth{1.003750pt}%
\definecolor{currentstroke}{rgb}{1.000000,0.498039,0.054902}%
\pgfsetstrokecolor{currentstroke}%
\pgfsetdash{}{0pt}%
\pgfpathmoveto{\pgfqpoint{1.806818in}{2.928269in}}%
\pgfpathcurveto{\pgfqpoint{1.817868in}{2.928269in}}{\pgfqpoint{1.828467in}{2.932659in}}{\pgfqpoint{1.836281in}{2.940473in}}%
\pgfpathcurveto{\pgfqpoint{1.844095in}{2.948286in}}{\pgfqpoint{1.848485in}{2.958885in}}{\pgfqpoint{1.848485in}{2.969936in}}%
\pgfpathcurveto{\pgfqpoint{1.848485in}{2.980986in}}{\pgfqpoint{1.844095in}{2.991585in}}{\pgfqpoint{1.836281in}{2.999398in}}%
\pgfpathcurveto{\pgfqpoint{1.828467in}{3.007212in}}{\pgfqpoint{1.817868in}{3.011602in}}{\pgfqpoint{1.806818in}{3.011602in}}%
\pgfpathcurveto{\pgfqpoint{1.795768in}{3.011602in}}{\pgfqpoint{1.785169in}{3.007212in}}{\pgfqpoint{1.777355in}{2.999398in}}%
\pgfpathcurveto{\pgfqpoint{1.769542in}{2.991585in}}{\pgfqpoint{1.765152in}{2.980986in}}{\pgfqpoint{1.765152in}{2.969936in}}%
\pgfpathcurveto{\pgfqpoint{1.765152in}{2.958885in}}{\pgfqpoint{1.769542in}{2.948286in}}{\pgfqpoint{1.777355in}{2.940473in}}%
\pgfpathcurveto{\pgfqpoint{1.785169in}{2.932659in}}{\pgfqpoint{1.795768in}{2.928269in}}{\pgfqpoint{1.806818in}{2.928269in}}%
\pgfpathclose%
\pgfusepath{stroke,fill}%
\end{pgfscope}%
\begin{pgfscope}%
\pgfpathrectangle{\pgfqpoint{0.750000in}{0.500000in}}{\pgfqpoint{4.650000in}{3.020000in}}%
\pgfusepath{clip}%
\pgfsetbuttcap%
\pgfsetroundjoin%
\definecolor{currentfill}{rgb}{0.121569,0.466667,0.705882}%
\pgfsetfillcolor{currentfill}%
\pgfsetlinewidth{1.003750pt}%
\definecolor{currentstroke}{rgb}{0.121569,0.466667,0.705882}%
\pgfsetstrokecolor{currentstroke}%
\pgfsetdash{}{0pt}%
\pgfpathmoveto{\pgfqpoint{1.021753in}{0.595606in}}%
\pgfpathcurveto{\pgfqpoint{1.032803in}{0.595606in}}{\pgfqpoint{1.043402in}{0.599996in}}{\pgfqpoint{1.051216in}{0.607810in}}%
\pgfpathcurveto{\pgfqpoint{1.059030in}{0.615624in}}{\pgfqpoint{1.063420in}{0.626223in}}{\pgfqpoint{1.063420in}{0.637273in}}%
\pgfpathcurveto{\pgfqpoint{1.063420in}{0.648323in}}{\pgfqpoint{1.059030in}{0.658922in}}{\pgfqpoint{1.051216in}{0.666736in}}%
\pgfpathcurveto{\pgfqpoint{1.043402in}{0.674549in}}{\pgfqpoint{1.032803in}{0.678939in}}{\pgfqpoint{1.021753in}{0.678939in}}%
\pgfpathcurveto{\pgfqpoint{1.010703in}{0.678939in}}{\pgfqpoint{1.000104in}{0.674549in}}{\pgfqpoint{0.992290in}{0.666736in}}%
\pgfpathcurveto{\pgfqpoint{0.984477in}{0.658922in}}{\pgfqpoint{0.980087in}{0.648323in}}{\pgfqpoint{0.980087in}{0.637273in}}%
\pgfpathcurveto{\pgfqpoint{0.980087in}{0.626223in}}{\pgfqpoint{0.984477in}{0.615624in}}{\pgfqpoint{0.992290in}{0.607810in}}%
\pgfpathcurveto{\pgfqpoint{1.000104in}{0.599996in}}{\pgfqpoint{1.010703in}{0.595606in}}{\pgfqpoint{1.021753in}{0.595606in}}%
\pgfpathclose%
\pgfusepath{stroke,fill}%
\end{pgfscope}%
\begin{pgfscope}%
\pgfpathrectangle{\pgfqpoint{0.750000in}{0.500000in}}{\pgfqpoint{4.650000in}{3.020000in}}%
\pgfusepath{clip}%
\pgfsetbuttcap%
\pgfsetroundjoin%
\definecolor{currentfill}{rgb}{1.000000,0.498039,0.054902}%
\pgfsetfillcolor{currentfill}%
\pgfsetlinewidth{1.003750pt}%
\definecolor{currentstroke}{rgb}{1.000000,0.498039,0.054902}%
\pgfsetstrokecolor{currentstroke}%
\pgfsetdash{}{0pt}%
\pgfpathmoveto{\pgfqpoint{1.686039in}{2.928269in}}%
\pgfpathcurveto{\pgfqpoint{1.697089in}{2.928269in}}{\pgfqpoint{1.707688in}{2.932659in}}{\pgfqpoint{1.715502in}{2.940473in}}%
\pgfpathcurveto{\pgfqpoint{1.723315in}{2.948286in}}{\pgfqpoint{1.727706in}{2.958885in}}{\pgfqpoint{1.727706in}{2.969936in}}%
\pgfpathcurveto{\pgfqpoint{1.727706in}{2.980986in}}{\pgfqpoint{1.723315in}{2.991585in}}{\pgfqpoint{1.715502in}{2.999398in}}%
\pgfpathcurveto{\pgfqpoint{1.707688in}{3.007212in}}{\pgfqpoint{1.697089in}{3.011602in}}{\pgfqpoint{1.686039in}{3.011602in}}%
\pgfpathcurveto{\pgfqpoint{1.674989in}{3.011602in}}{\pgfqpoint{1.664390in}{3.007212in}}{\pgfqpoint{1.656576in}{2.999398in}}%
\pgfpathcurveto{\pgfqpoint{1.648763in}{2.991585in}}{\pgfqpoint{1.644372in}{2.980986in}}{\pgfqpoint{1.644372in}{2.969936in}}%
\pgfpathcurveto{\pgfqpoint{1.644372in}{2.958885in}}{\pgfqpoint{1.648763in}{2.948286in}}{\pgfqpoint{1.656576in}{2.940473in}}%
\pgfpathcurveto{\pgfqpoint{1.664390in}{2.932659in}}{\pgfqpoint{1.674989in}{2.928269in}}{\pgfqpoint{1.686039in}{2.928269in}}%
\pgfpathclose%
\pgfusepath{stroke,fill}%
\end{pgfscope}%
\begin{pgfscope}%
\pgfpathrectangle{\pgfqpoint{0.750000in}{0.500000in}}{\pgfqpoint{4.650000in}{3.020000in}}%
\pgfusepath{clip}%
\pgfsetbuttcap%
\pgfsetroundjoin%
\definecolor{currentfill}{rgb}{1.000000,0.498039,0.054902}%
\pgfsetfillcolor{currentfill}%
\pgfsetlinewidth{1.003750pt}%
\definecolor{currentstroke}{rgb}{1.000000,0.498039,0.054902}%
\pgfsetstrokecolor{currentstroke}%
\pgfsetdash{}{0pt}%
\pgfpathmoveto{\pgfqpoint{1.504870in}{2.928269in}}%
\pgfpathcurveto{\pgfqpoint{1.515920in}{2.928269in}}{\pgfqpoint{1.526519in}{2.932659in}}{\pgfqpoint{1.534333in}{2.940473in}}%
\pgfpathcurveto{\pgfqpoint{1.542147in}{2.948286in}}{\pgfqpoint{1.546537in}{2.958885in}}{\pgfqpoint{1.546537in}{2.969936in}}%
\pgfpathcurveto{\pgfqpoint{1.546537in}{2.980986in}}{\pgfqpoint{1.542147in}{2.991585in}}{\pgfqpoint{1.534333in}{2.999398in}}%
\pgfpathcurveto{\pgfqpoint{1.526519in}{3.007212in}}{\pgfqpoint{1.515920in}{3.011602in}}{\pgfqpoint{1.504870in}{3.011602in}}%
\pgfpathcurveto{\pgfqpoint{1.493820in}{3.011602in}}{\pgfqpoint{1.483221in}{3.007212in}}{\pgfqpoint{1.475407in}{2.999398in}}%
\pgfpathcurveto{\pgfqpoint{1.467594in}{2.991585in}}{\pgfqpoint{1.463203in}{2.980986in}}{\pgfqpoint{1.463203in}{2.969936in}}%
\pgfpathcurveto{\pgfqpoint{1.463203in}{2.958885in}}{\pgfqpoint{1.467594in}{2.948286in}}{\pgfqpoint{1.475407in}{2.940473in}}%
\pgfpathcurveto{\pgfqpoint{1.483221in}{2.932659in}}{\pgfqpoint{1.493820in}{2.928269in}}{\pgfqpoint{1.504870in}{2.928269in}}%
\pgfpathclose%
\pgfusepath{stroke,fill}%
\end{pgfscope}%
\begin{pgfscope}%
\pgfpathrectangle{\pgfqpoint{0.750000in}{0.500000in}}{\pgfqpoint{4.650000in}{3.020000in}}%
\pgfusepath{clip}%
\pgfsetbuttcap%
\pgfsetroundjoin%
\definecolor{currentfill}{rgb}{1.000000,0.498039,0.054902}%
\pgfsetfillcolor{currentfill}%
\pgfsetlinewidth{1.003750pt}%
\definecolor{currentstroke}{rgb}{1.000000,0.498039,0.054902}%
\pgfsetstrokecolor{currentstroke}%
\pgfsetdash{}{0pt}%
\pgfpathmoveto{\pgfqpoint{1.444481in}{2.936057in}}%
\pgfpathcurveto{\pgfqpoint{1.455531in}{2.936057in}}{\pgfqpoint{1.466130in}{2.940448in}}{\pgfqpoint{1.473943in}{2.948261in}}%
\pgfpathcurveto{\pgfqpoint{1.481757in}{2.956075in}}{\pgfqpoint{1.486147in}{2.966674in}}{\pgfqpoint{1.486147in}{2.977724in}}%
\pgfpathcurveto{\pgfqpoint{1.486147in}{2.988774in}}{\pgfqpoint{1.481757in}{2.999373in}}{\pgfqpoint{1.473943in}{3.007187in}}%
\pgfpathcurveto{\pgfqpoint{1.466130in}{3.015000in}}{\pgfqpoint{1.455531in}{3.019391in}}{\pgfqpoint{1.444481in}{3.019391in}}%
\pgfpathcurveto{\pgfqpoint{1.433430in}{3.019391in}}{\pgfqpoint{1.422831in}{3.015000in}}{\pgfqpoint{1.415018in}{3.007187in}}%
\pgfpathcurveto{\pgfqpoint{1.407204in}{2.999373in}}{\pgfqpoint{1.402814in}{2.988774in}}{\pgfqpoint{1.402814in}{2.977724in}}%
\pgfpathcurveto{\pgfqpoint{1.402814in}{2.966674in}}{\pgfqpoint{1.407204in}{2.956075in}}{\pgfqpoint{1.415018in}{2.948261in}}%
\pgfpathcurveto{\pgfqpoint{1.422831in}{2.940448in}}{\pgfqpoint{1.433430in}{2.936057in}}{\pgfqpoint{1.444481in}{2.936057in}}%
\pgfpathclose%
\pgfusepath{stroke,fill}%
\end{pgfscope}%
\begin{pgfscope}%
\pgfpathrectangle{\pgfqpoint{0.750000in}{0.500000in}}{\pgfqpoint{4.650000in}{3.020000in}}%
\pgfusepath{clip}%
\pgfsetbuttcap%
\pgfsetroundjoin%
\definecolor{currentfill}{rgb}{1.000000,0.498039,0.054902}%
\pgfsetfillcolor{currentfill}%
\pgfsetlinewidth{1.003750pt}%
\definecolor{currentstroke}{rgb}{1.000000,0.498039,0.054902}%
\pgfsetstrokecolor{currentstroke}%
\pgfsetdash{}{0pt}%
\pgfpathmoveto{\pgfqpoint{1.444481in}{3.045097in}}%
\pgfpathcurveto{\pgfqpoint{1.455531in}{3.045097in}}{\pgfqpoint{1.466130in}{3.049487in}}{\pgfqpoint{1.473943in}{3.057301in}}%
\pgfpathcurveto{\pgfqpoint{1.481757in}{3.065114in}}{\pgfqpoint{1.486147in}{3.075713in}}{\pgfqpoint{1.486147in}{3.086763in}}%
\pgfpathcurveto{\pgfqpoint{1.486147in}{3.097814in}}{\pgfqpoint{1.481757in}{3.108413in}}{\pgfqpoint{1.473943in}{3.116226in}}%
\pgfpathcurveto{\pgfqpoint{1.466130in}{3.124040in}}{\pgfqpoint{1.455531in}{3.128430in}}{\pgfqpoint{1.444481in}{3.128430in}}%
\pgfpathcurveto{\pgfqpoint{1.433430in}{3.128430in}}{\pgfqpoint{1.422831in}{3.124040in}}{\pgfqpoint{1.415018in}{3.116226in}}%
\pgfpathcurveto{\pgfqpoint{1.407204in}{3.108413in}}{\pgfqpoint{1.402814in}{3.097814in}}{\pgfqpoint{1.402814in}{3.086763in}}%
\pgfpathcurveto{\pgfqpoint{1.402814in}{3.075713in}}{\pgfqpoint{1.407204in}{3.065114in}}{\pgfqpoint{1.415018in}{3.057301in}}%
\pgfpathcurveto{\pgfqpoint{1.422831in}{3.049487in}}{\pgfqpoint{1.433430in}{3.045097in}}{\pgfqpoint{1.444481in}{3.045097in}}%
\pgfpathclose%
\pgfusepath{stroke,fill}%
\end{pgfscope}%
\begin{pgfscope}%
\pgfpathrectangle{\pgfqpoint{0.750000in}{0.500000in}}{\pgfqpoint{4.650000in}{3.020000in}}%
\pgfusepath{clip}%
\pgfsetbuttcap%
\pgfsetroundjoin%
\definecolor{currentfill}{rgb}{1.000000,0.498039,0.054902}%
\pgfsetfillcolor{currentfill}%
\pgfsetlinewidth{1.003750pt}%
\definecolor{currentstroke}{rgb}{1.000000,0.498039,0.054902}%
\pgfsetstrokecolor{currentstroke}%
\pgfsetdash{}{0pt}%
\pgfpathmoveto{\pgfqpoint{1.323701in}{2.936057in}}%
\pgfpathcurveto{\pgfqpoint{1.334751in}{2.936057in}}{\pgfqpoint{1.345350in}{2.940448in}}{\pgfqpoint{1.353164in}{2.948261in}}%
\pgfpathcurveto{\pgfqpoint{1.360978in}{2.956075in}}{\pgfqpoint{1.365368in}{2.966674in}}{\pgfqpoint{1.365368in}{2.977724in}}%
\pgfpathcurveto{\pgfqpoint{1.365368in}{2.988774in}}{\pgfqpoint{1.360978in}{2.999373in}}{\pgfqpoint{1.353164in}{3.007187in}}%
\pgfpathcurveto{\pgfqpoint{1.345350in}{3.015000in}}{\pgfqpoint{1.334751in}{3.019391in}}{\pgfqpoint{1.323701in}{3.019391in}}%
\pgfpathcurveto{\pgfqpoint{1.312651in}{3.019391in}}{\pgfqpoint{1.302052in}{3.015000in}}{\pgfqpoint{1.294239in}{3.007187in}}%
\pgfpathcurveto{\pgfqpoint{1.286425in}{2.999373in}}{\pgfqpoint{1.282035in}{2.988774in}}{\pgfqpoint{1.282035in}{2.977724in}}%
\pgfpathcurveto{\pgfqpoint{1.282035in}{2.966674in}}{\pgfqpoint{1.286425in}{2.956075in}}{\pgfqpoint{1.294239in}{2.948261in}}%
\pgfpathcurveto{\pgfqpoint{1.302052in}{2.940448in}}{\pgfqpoint{1.312651in}{2.936057in}}{\pgfqpoint{1.323701in}{2.936057in}}%
\pgfpathclose%
\pgfusepath{stroke,fill}%
\end{pgfscope}%
\begin{pgfscope}%
\pgfpathrectangle{\pgfqpoint{0.750000in}{0.500000in}}{\pgfqpoint{4.650000in}{3.020000in}}%
\pgfusepath{clip}%
\pgfsetbuttcap%
\pgfsetroundjoin%
\definecolor{currentfill}{rgb}{1.000000,0.498039,0.054902}%
\pgfsetfillcolor{currentfill}%
\pgfsetlinewidth{1.003750pt}%
\definecolor{currentstroke}{rgb}{1.000000,0.498039,0.054902}%
\pgfsetstrokecolor{currentstroke}%
\pgfsetdash{}{0pt}%
\pgfpathmoveto{\pgfqpoint{2.229545in}{2.936057in}}%
\pgfpathcurveto{\pgfqpoint{2.240596in}{2.936057in}}{\pgfqpoint{2.251195in}{2.940448in}}{\pgfqpoint{2.259008in}{2.948261in}}%
\pgfpathcurveto{\pgfqpoint{2.266822in}{2.956075in}}{\pgfqpoint{2.271212in}{2.966674in}}{\pgfqpoint{2.271212in}{2.977724in}}%
\pgfpathcurveto{\pgfqpoint{2.271212in}{2.988774in}}{\pgfqpoint{2.266822in}{2.999373in}}{\pgfqpoint{2.259008in}{3.007187in}}%
\pgfpathcurveto{\pgfqpoint{2.251195in}{3.015000in}}{\pgfqpoint{2.240596in}{3.019391in}}{\pgfqpoint{2.229545in}{3.019391in}}%
\pgfpathcurveto{\pgfqpoint{2.218495in}{3.019391in}}{\pgfqpoint{2.207896in}{3.015000in}}{\pgfqpoint{2.200083in}{3.007187in}}%
\pgfpathcurveto{\pgfqpoint{2.192269in}{2.999373in}}{\pgfqpoint{2.187879in}{2.988774in}}{\pgfqpoint{2.187879in}{2.977724in}}%
\pgfpathcurveto{\pgfqpoint{2.187879in}{2.966674in}}{\pgfqpoint{2.192269in}{2.956075in}}{\pgfqpoint{2.200083in}{2.948261in}}%
\pgfpathcurveto{\pgfqpoint{2.207896in}{2.940448in}}{\pgfqpoint{2.218495in}{2.936057in}}{\pgfqpoint{2.229545in}{2.936057in}}%
\pgfpathclose%
\pgfusepath{stroke,fill}%
\end{pgfscope}%
\begin{pgfscope}%
\pgfpathrectangle{\pgfqpoint{0.750000in}{0.500000in}}{\pgfqpoint{4.650000in}{3.020000in}}%
\pgfusepath{clip}%
\pgfsetbuttcap%
\pgfsetroundjoin%
\definecolor{currentfill}{rgb}{1.000000,0.498039,0.054902}%
\pgfsetfillcolor{currentfill}%
\pgfsetlinewidth{1.003750pt}%
\definecolor{currentstroke}{rgb}{1.000000,0.498039,0.054902}%
\pgfsetstrokecolor{currentstroke}%
\pgfsetdash{}{0pt}%
\pgfpathmoveto{\pgfqpoint{1.686039in}{3.185290in}}%
\pgfpathcurveto{\pgfqpoint{1.697089in}{3.185290in}}{\pgfqpoint{1.707688in}{3.189680in}}{\pgfqpoint{1.715502in}{3.197494in}}%
\pgfpathcurveto{\pgfqpoint{1.723315in}{3.205308in}}{\pgfqpoint{1.727706in}{3.215907in}}{\pgfqpoint{1.727706in}{3.226957in}}%
\pgfpathcurveto{\pgfqpoint{1.727706in}{3.238007in}}{\pgfqpoint{1.723315in}{3.248606in}}{\pgfqpoint{1.715502in}{3.256420in}}%
\pgfpathcurveto{\pgfqpoint{1.707688in}{3.264233in}}{\pgfqpoint{1.697089in}{3.268623in}}{\pgfqpoint{1.686039in}{3.268623in}}%
\pgfpathcurveto{\pgfqpoint{1.674989in}{3.268623in}}{\pgfqpoint{1.664390in}{3.264233in}}{\pgfqpoint{1.656576in}{3.256420in}}%
\pgfpathcurveto{\pgfqpoint{1.648763in}{3.248606in}}{\pgfqpoint{1.644372in}{3.238007in}}{\pgfqpoint{1.644372in}{3.226957in}}%
\pgfpathcurveto{\pgfqpoint{1.644372in}{3.215907in}}{\pgfqpoint{1.648763in}{3.205308in}}{\pgfqpoint{1.656576in}{3.197494in}}%
\pgfpathcurveto{\pgfqpoint{1.664390in}{3.189680in}}{\pgfqpoint{1.674989in}{3.185290in}}{\pgfqpoint{1.686039in}{3.185290in}}%
\pgfpathclose%
\pgfusepath{stroke,fill}%
\end{pgfscope}%
\begin{pgfscope}%
\pgfpathrectangle{\pgfqpoint{0.750000in}{0.500000in}}{\pgfqpoint{4.650000in}{3.020000in}}%
\pgfusepath{clip}%
\pgfsetbuttcap%
\pgfsetroundjoin%
\definecolor{currentfill}{rgb}{1.000000,0.498039,0.054902}%
\pgfsetfillcolor{currentfill}%
\pgfsetlinewidth{1.003750pt}%
\definecolor{currentstroke}{rgb}{1.000000,0.498039,0.054902}%
\pgfsetstrokecolor{currentstroke}%
\pgfsetdash{}{0pt}%
\pgfpathmoveto{\pgfqpoint{1.927597in}{2.838701in}}%
\pgfpathcurveto{\pgfqpoint{1.938648in}{2.838701in}}{\pgfqpoint{1.949247in}{2.843091in}}{\pgfqpoint{1.957060in}{2.850905in}}%
\pgfpathcurveto{\pgfqpoint{1.964874in}{2.858718in}}{\pgfqpoint{1.969264in}{2.869317in}}{\pgfqpoint{1.969264in}{2.880368in}}%
\pgfpathcurveto{\pgfqpoint{1.969264in}{2.891418in}}{\pgfqpoint{1.964874in}{2.902017in}}{\pgfqpoint{1.957060in}{2.909830in}}%
\pgfpathcurveto{\pgfqpoint{1.949247in}{2.917644in}}{\pgfqpoint{1.938648in}{2.922034in}}{\pgfqpoint{1.927597in}{2.922034in}}%
\pgfpathcurveto{\pgfqpoint{1.916547in}{2.922034in}}{\pgfqpoint{1.905948in}{2.917644in}}{\pgfqpoint{1.898135in}{2.909830in}}%
\pgfpathcurveto{\pgfqpoint{1.890321in}{2.902017in}}{\pgfqpoint{1.885931in}{2.891418in}}{\pgfqpoint{1.885931in}{2.880368in}}%
\pgfpathcurveto{\pgfqpoint{1.885931in}{2.869317in}}{\pgfqpoint{1.890321in}{2.858718in}}{\pgfqpoint{1.898135in}{2.850905in}}%
\pgfpathcurveto{\pgfqpoint{1.905948in}{2.843091in}}{\pgfqpoint{1.916547in}{2.838701in}}{\pgfqpoint{1.927597in}{2.838701in}}%
\pgfpathclose%
\pgfusepath{stroke,fill}%
\end{pgfscope}%
\begin{pgfscope}%
\pgfpathrectangle{\pgfqpoint{0.750000in}{0.500000in}}{\pgfqpoint{4.650000in}{3.020000in}}%
\pgfusepath{clip}%
\pgfsetbuttcap%
\pgfsetroundjoin%
\definecolor{currentfill}{rgb}{1.000000,0.498039,0.054902}%
\pgfsetfillcolor{currentfill}%
\pgfsetlinewidth{1.003750pt}%
\definecolor{currentstroke}{rgb}{1.000000,0.498039,0.054902}%
\pgfsetstrokecolor{currentstroke}%
\pgfsetdash{}{0pt}%
\pgfpathmoveto{\pgfqpoint{1.987987in}{2.978894in}}%
\pgfpathcurveto{\pgfqpoint{1.999037in}{2.978894in}}{\pgfqpoint{2.009636in}{2.983285in}}{\pgfqpoint{2.017450in}{2.991098in}}%
\pgfpathcurveto{\pgfqpoint{2.025263in}{2.998912in}}{\pgfqpoint{2.029654in}{3.009511in}}{\pgfqpoint{2.029654in}{3.020561in}}%
\pgfpathcurveto{\pgfqpoint{2.029654in}{3.031611in}}{\pgfqpoint{2.025263in}{3.042210in}}{\pgfqpoint{2.017450in}{3.050024in}}%
\pgfpathcurveto{\pgfqpoint{2.009636in}{3.057837in}}{\pgfqpoint{1.999037in}{3.062228in}}{\pgfqpoint{1.987987in}{3.062228in}}%
\pgfpathcurveto{\pgfqpoint{1.976937in}{3.062228in}}{\pgfqpoint{1.966338in}{3.057837in}}{\pgfqpoint{1.958524in}{3.050024in}}%
\pgfpathcurveto{\pgfqpoint{1.950711in}{3.042210in}}{\pgfqpoint{1.946320in}{3.031611in}}{\pgfqpoint{1.946320in}{3.020561in}}%
\pgfpathcurveto{\pgfqpoint{1.946320in}{3.009511in}}{\pgfqpoint{1.950711in}{2.998912in}}{\pgfqpoint{1.958524in}{2.991098in}}%
\pgfpathcurveto{\pgfqpoint{1.966338in}{2.983285in}}{\pgfqpoint{1.976937in}{2.978894in}}{\pgfqpoint{1.987987in}{2.978894in}}%
\pgfpathclose%
\pgfusepath{stroke,fill}%
\end{pgfscope}%
\begin{pgfscope}%
\pgfpathrectangle{\pgfqpoint{0.750000in}{0.500000in}}{\pgfqpoint{4.650000in}{3.020000in}}%
\pgfusepath{clip}%
\pgfsetbuttcap%
\pgfsetroundjoin%
\definecolor{currentfill}{rgb}{1.000000,0.498039,0.054902}%
\pgfsetfillcolor{currentfill}%
\pgfsetlinewidth{1.003750pt}%
\definecolor{currentstroke}{rgb}{1.000000,0.498039,0.054902}%
\pgfsetstrokecolor{currentstroke}%
\pgfsetdash{}{0pt}%
\pgfpathmoveto{\pgfqpoint{1.686039in}{2.932163in}}%
\pgfpathcurveto{\pgfqpoint{1.697089in}{2.932163in}}{\pgfqpoint{1.707688in}{2.936553in}}{\pgfqpoint{1.715502in}{2.944367in}}%
\pgfpathcurveto{\pgfqpoint{1.723315in}{2.952181in}}{\pgfqpoint{1.727706in}{2.962780in}}{\pgfqpoint{1.727706in}{2.973830in}}%
\pgfpathcurveto{\pgfqpoint{1.727706in}{2.984880in}}{\pgfqpoint{1.723315in}{2.995479in}}{\pgfqpoint{1.715502in}{3.003293in}}%
\pgfpathcurveto{\pgfqpoint{1.707688in}{3.011106in}}{\pgfqpoint{1.697089in}{3.015496in}}{\pgfqpoint{1.686039in}{3.015496in}}%
\pgfpathcurveto{\pgfqpoint{1.674989in}{3.015496in}}{\pgfqpoint{1.664390in}{3.011106in}}{\pgfqpoint{1.656576in}{3.003293in}}%
\pgfpathcurveto{\pgfqpoint{1.648763in}{2.995479in}}{\pgfqpoint{1.644372in}{2.984880in}}{\pgfqpoint{1.644372in}{2.973830in}}%
\pgfpathcurveto{\pgfqpoint{1.644372in}{2.962780in}}{\pgfqpoint{1.648763in}{2.952181in}}{\pgfqpoint{1.656576in}{2.944367in}}%
\pgfpathcurveto{\pgfqpoint{1.664390in}{2.936553in}}{\pgfqpoint{1.674989in}{2.932163in}}{\pgfqpoint{1.686039in}{2.932163in}}%
\pgfpathclose%
\pgfusepath{stroke,fill}%
\end{pgfscope}%
\begin{pgfscope}%
\pgfpathrectangle{\pgfqpoint{0.750000in}{0.500000in}}{\pgfqpoint{4.650000in}{3.020000in}}%
\pgfusepath{clip}%
\pgfsetbuttcap%
\pgfsetroundjoin%
\definecolor{currentfill}{rgb}{1.000000,0.498039,0.054902}%
\pgfsetfillcolor{currentfill}%
\pgfsetlinewidth{1.003750pt}%
\definecolor{currentstroke}{rgb}{1.000000,0.498039,0.054902}%
\pgfsetstrokecolor{currentstroke}%
\pgfsetdash{}{0pt}%
\pgfpathmoveto{\pgfqpoint{1.565260in}{2.932163in}}%
\pgfpathcurveto{\pgfqpoint{1.576310in}{2.932163in}}{\pgfqpoint{1.586909in}{2.936553in}}{\pgfqpoint{1.594723in}{2.944367in}}%
\pgfpathcurveto{\pgfqpoint{1.602536in}{2.952181in}}{\pgfqpoint{1.606926in}{2.962780in}}{\pgfqpoint{1.606926in}{2.973830in}}%
\pgfpathcurveto{\pgfqpoint{1.606926in}{2.984880in}}{\pgfqpoint{1.602536in}{2.995479in}}{\pgfqpoint{1.594723in}{3.003293in}}%
\pgfpathcurveto{\pgfqpoint{1.586909in}{3.011106in}}{\pgfqpoint{1.576310in}{3.015496in}}{\pgfqpoint{1.565260in}{3.015496in}}%
\pgfpathcurveto{\pgfqpoint{1.554210in}{3.015496in}}{\pgfqpoint{1.543611in}{3.011106in}}{\pgfqpoint{1.535797in}{3.003293in}}%
\pgfpathcurveto{\pgfqpoint{1.527983in}{2.995479in}}{\pgfqpoint{1.523593in}{2.984880in}}{\pgfqpoint{1.523593in}{2.973830in}}%
\pgfpathcurveto{\pgfqpoint{1.523593in}{2.962780in}}{\pgfqpoint{1.527983in}{2.952181in}}{\pgfqpoint{1.535797in}{2.944367in}}%
\pgfpathcurveto{\pgfqpoint{1.543611in}{2.936553in}}{\pgfqpoint{1.554210in}{2.932163in}}{\pgfqpoint{1.565260in}{2.932163in}}%
\pgfpathclose%
\pgfusepath{stroke,fill}%
\end{pgfscope}%
\begin{pgfscope}%
\pgfpathrectangle{\pgfqpoint{0.750000in}{0.500000in}}{\pgfqpoint{4.650000in}{3.020000in}}%
\pgfusepath{clip}%
\pgfsetbuttcap%
\pgfsetroundjoin%
\definecolor{currentfill}{rgb}{1.000000,0.498039,0.054902}%
\pgfsetfillcolor{currentfill}%
\pgfsetlinewidth{1.003750pt}%
\definecolor{currentstroke}{rgb}{1.000000,0.498039,0.054902}%
\pgfsetstrokecolor{currentstroke}%
\pgfsetdash{}{0pt}%
\pgfpathmoveto{\pgfqpoint{2.591883in}{2.928269in}}%
\pgfpathcurveto{\pgfqpoint{2.602933in}{2.928269in}}{\pgfqpoint{2.613532in}{2.932659in}}{\pgfqpoint{2.621346in}{2.940473in}}%
\pgfpathcurveto{\pgfqpoint{2.629160in}{2.948286in}}{\pgfqpoint{2.633550in}{2.958885in}}{\pgfqpoint{2.633550in}{2.969936in}}%
\pgfpathcurveto{\pgfqpoint{2.633550in}{2.980986in}}{\pgfqpoint{2.629160in}{2.991585in}}{\pgfqpoint{2.621346in}{2.999398in}}%
\pgfpathcurveto{\pgfqpoint{2.613532in}{3.007212in}}{\pgfqpoint{2.602933in}{3.011602in}}{\pgfqpoint{2.591883in}{3.011602in}}%
\pgfpathcurveto{\pgfqpoint{2.580833in}{3.011602in}}{\pgfqpoint{2.570234in}{3.007212in}}{\pgfqpoint{2.562420in}{2.999398in}}%
\pgfpathcurveto{\pgfqpoint{2.554607in}{2.991585in}}{\pgfqpoint{2.550216in}{2.980986in}}{\pgfqpoint{2.550216in}{2.969936in}}%
\pgfpathcurveto{\pgfqpoint{2.550216in}{2.958885in}}{\pgfqpoint{2.554607in}{2.948286in}}{\pgfqpoint{2.562420in}{2.940473in}}%
\pgfpathcurveto{\pgfqpoint{2.570234in}{2.932659in}}{\pgfqpoint{2.580833in}{2.928269in}}{\pgfqpoint{2.591883in}{2.928269in}}%
\pgfpathclose%
\pgfusepath{stroke,fill}%
\end{pgfscope}%
\begin{pgfscope}%
\pgfpathrectangle{\pgfqpoint{0.750000in}{0.500000in}}{\pgfqpoint{4.650000in}{3.020000in}}%
\pgfusepath{clip}%
\pgfsetbuttcap%
\pgfsetroundjoin%
\definecolor{currentfill}{rgb}{0.121569,0.466667,0.705882}%
\pgfsetfillcolor{currentfill}%
\pgfsetlinewidth{1.003750pt}%
\definecolor{currentstroke}{rgb}{0.121569,0.466667,0.705882}%
\pgfsetstrokecolor{currentstroke}%
\pgfsetdash{}{0pt}%
\pgfpathmoveto{\pgfqpoint{0.961364in}{0.595606in}}%
\pgfpathcurveto{\pgfqpoint{0.972414in}{0.595606in}}{\pgfqpoint{0.983013in}{0.599996in}}{\pgfqpoint{0.990826in}{0.607810in}}%
\pgfpathcurveto{\pgfqpoint{0.998640in}{0.615624in}}{\pgfqpoint{1.003030in}{0.626223in}}{\pgfqpoint{1.003030in}{0.637273in}}%
\pgfpathcurveto{\pgfqpoint{1.003030in}{0.648323in}}{\pgfqpoint{0.998640in}{0.658922in}}{\pgfqpoint{0.990826in}{0.666736in}}%
\pgfpathcurveto{\pgfqpoint{0.983013in}{0.674549in}}{\pgfqpoint{0.972414in}{0.678939in}}{\pgfqpoint{0.961364in}{0.678939in}}%
\pgfpathcurveto{\pgfqpoint{0.950314in}{0.678939in}}{\pgfqpoint{0.939714in}{0.674549in}}{\pgfqpoint{0.931901in}{0.666736in}}%
\pgfpathcurveto{\pgfqpoint{0.924087in}{0.658922in}}{\pgfqpoint{0.919697in}{0.648323in}}{\pgfqpoint{0.919697in}{0.637273in}}%
\pgfpathcurveto{\pgfqpoint{0.919697in}{0.626223in}}{\pgfqpoint{0.924087in}{0.615624in}}{\pgfqpoint{0.931901in}{0.607810in}}%
\pgfpathcurveto{\pgfqpoint{0.939714in}{0.599996in}}{\pgfqpoint{0.950314in}{0.595606in}}{\pgfqpoint{0.961364in}{0.595606in}}%
\pgfpathclose%
\pgfusepath{stroke,fill}%
\end{pgfscope}%
\begin{pgfscope}%
\pgfpathrectangle{\pgfqpoint{0.750000in}{0.500000in}}{\pgfqpoint{4.650000in}{3.020000in}}%
\pgfusepath{clip}%
\pgfsetbuttcap%
\pgfsetroundjoin%
\definecolor{currentfill}{rgb}{0.121569,0.466667,0.705882}%
\pgfsetfillcolor{currentfill}%
\pgfsetlinewidth{1.003750pt}%
\definecolor{currentstroke}{rgb}{0.121569,0.466667,0.705882}%
\pgfsetstrokecolor{currentstroke}%
\pgfsetdash{}{0pt}%
\pgfpathmoveto{\pgfqpoint{1.202922in}{0.599500in}}%
\pgfpathcurveto{\pgfqpoint{1.213972in}{0.599500in}}{\pgfqpoint{1.224571in}{0.603891in}}{\pgfqpoint{1.232385in}{0.611704in}}%
\pgfpathcurveto{\pgfqpoint{1.240198in}{0.619518in}}{\pgfqpoint{1.244589in}{0.630117in}}{\pgfqpoint{1.244589in}{0.641167in}}%
\pgfpathcurveto{\pgfqpoint{1.244589in}{0.652217in}}{\pgfqpoint{1.240198in}{0.662816in}}{\pgfqpoint{1.232385in}{0.670630in}}%
\pgfpathcurveto{\pgfqpoint{1.224571in}{0.678443in}}{\pgfqpoint{1.213972in}{0.682834in}}{\pgfqpoint{1.202922in}{0.682834in}}%
\pgfpathcurveto{\pgfqpoint{1.191872in}{0.682834in}}{\pgfqpoint{1.181273in}{0.678443in}}{\pgfqpoint{1.173459in}{0.670630in}}%
\pgfpathcurveto{\pgfqpoint{1.165646in}{0.662816in}}{\pgfqpoint{1.161255in}{0.652217in}}{\pgfqpoint{1.161255in}{0.641167in}}%
\pgfpathcurveto{\pgfqpoint{1.161255in}{0.630117in}}{\pgfqpoint{1.165646in}{0.619518in}}{\pgfqpoint{1.173459in}{0.611704in}}%
\pgfpathcurveto{\pgfqpoint{1.181273in}{0.603891in}}{\pgfqpoint{1.191872in}{0.599500in}}{\pgfqpoint{1.202922in}{0.599500in}}%
\pgfpathclose%
\pgfusepath{stroke,fill}%
\end{pgfscope}%
\begin{pgfscope}%
\pgfpathrectangle{\pgfqpoint{0.750000in}{0.500000in}}{\pgfqpoint{4.650000in}{3.020000in}}%
\pgfusepath{clip}%
\pgfsetbuttcap%
\pgfsetroundjoin%
\definecolor{currentfill}{rgb}{1.000000,0.498039,0.054902}%
\pgfsetfillcolor{currentfill}%
\pgfsetlinewidth{1.003750pt}%
\definecolor{currentstroke}{rgb}{1.000000,0.498039,0.054902}%
\pgfsetstrokecolor{currentstroke}%
\pgfsetdash{}{0pt}%
\pgfpathmoveto{\pgfqpoint{1.504870in}{2.904903in}}%
\pgfpathcurveto{\pgfqpoint{1.515920in}{2.904903in}}{\pgfqpoint{1.526519in}{2.909294in}}{\pgfqpoint{1.534333in}{2.917107in}}%
\pgfpathcurveto{\pgfqpoint{1.542147in}{2.924921in}}{\pgfqpoint{1.546537in}{2.935520in}}{\pgfqpoint{1.546537in}{2.946570in}}%
\pgfpathcurveto{\pgfqpoint{1.546537in}{2.957620in}}{\pgfqpoint{1.542147in}{2.968219in}}{\pgfqpoint{1.534333in}{2.976033in}}%
\pgfpathcurveto{\pgfqpoint{1.526519in}{2.983846in}}{\pgfqpoint{1.515920in}{2.988237in}}{\pgfqpoint{1.504870in}{2.988237in}}%
\pgfpathcurveto{\pgfqpoint{1.493820in}{2.988237in}}{\pgfqpoint{1.483221in}{2.983846in}}{\pgfqpoint{1.475407in}{2.976033in}}%
\pgfpathcurveto{\pgfqpoint{1.467594in}{2.968219in}}{\pgfqpoint{1.463203in}{2.957620in}}{\pgfqpoint{1.463203in}{2.946570in}}%
\pgfpathcurveto{\pgfqpoint{1.463203in}{2.935520in}}{\pgfqpoint{1.467594in}{2.924921in}}{\pgfqpoint{1.475407in}{2.917107in}}%
\pgfpathcurveto{\pgfqpoint{1.483221in}{2.909294in}}{\pgfqpoint{1.493820in}{2.904903in}}{\pgfqpoint{1.504870in}{2.904903in}}%
\pgfpathclose%
\pgfusepath{stroke,fill}%
\end{pgfscope}%
\begin{pgfscope}%
\pgfpathrectangle{\pgfqpoint{0.750000in}{0.500000in}}{\pgfqpoint{4.650000in}{3.020000in}}%
\pgfusepath{clip}%
\pgfsetbuttcap%
\pgfsetroundjoin%
\definecolor{currentfill}{rgb}{1.000000,0.498039,0.054902}%
\pgfsetfillcolor{currentfill}%
\pgfsetlinewidth{1.003750pt}%
\definecolor{currentstroke}{rgb}{1.000000,0.498039,0.054902}%
\pgfsetstrokecolor{currentstroke}%
\pgfsetdash{}{0pt}%
\pgfpathmoveto{\pgfqpoint{1.323701in}{2.994471in}}%
\pgfpathcurveto{\pgfqpoint{1.334751in}{2.994471in}}{\pgfqpoint{1.345350in}{2.998862in}}{\pgfqpoint{1.353164in}{3.006675in}}%
\pgfpathcurveto{\pgfqpoint{1.360978in}{3.014489in}}{\pgfqpoint{1.365368in}{3.025088in}}{\pgfqpoint{1.365368in}{3.036138in}}%
\pgfpathcurveto{\pgfqpoint{1.365368in}{3.047188in}}{\pgfqpoint{1.360978in}{3.057787in}}{\pgfqpoint{1.353164in}{3.065601in}}%
\pgfpathcurveto{\pgfqpoint{1.345350in}{3.073414in}}{\pgfqpoint{1.334751in}{3.077805in}}{\pgfqpoint{1.323701in}{3.077805in}}%
\pgfpathcurveto{\pgfqpoint{1.312651in}{3.077805in}}{\pgfqpoint{1.302052in}{3.073414in}}{\pgfqpoint{1.294239in}{3.065601in}}%
\pgfpathcurveto{\pgfqpoint{1.286425in}{3.057787in}}{\pgfqpoint{1.282035in}{3.047188in}}{\pgfqpoint{1.282035in}{3.036138in}}%
\pgfpathcurveto{\pgfqpoint{1.282035in}{3.025088in}}{\pgfqpoint{1.286425in}{3.014489in}}{\pgfqpoint{1.294239in}{3.006675in}}%
\pgfpathcurveto{\pgfqpoint{1.302052in}{2.998862in}}{\pgfqpoint{1.312651in}{2.994471in}}{\pgfqpoint{1.323701in}{2.994471in}}%
\pgfpathclose%
\pgfusepath{stroke,fill}%
\end{pgfscope}%
\begin{pgfscope}%
\pgfpathrectangle{\pgfqpoint{0.750000in}{0.500000in}}{\pgfqpoint{4.650000in}{3.020000in}}%
\pgfusepath{clip}%
\pgfsetbuttcap%
\pgfsetroundjoin%
\definecolor{currentfill}{rgb}{0.121569,0.466667,0.705882}%
\pgfsetfillcolor{currentfill}%
\pgfsetlinewidth{1.003750pt}%
\definecolor{currentstroke}{rgb}{0.121569,0.466667,0.705882}%
\pgfsetstrokecolor{currentstroke}%
\pgfsetdash{}{0pt}%
\pgfpathmoveto{\pgfqpoint{0.961364in}{0.595606in}}%
\pgfpathcurveto{\pgfqpoint{0.972414in}{0.595606in}}{\pgfqpoint{0.983013in}{0.599996in}}{\pgfqpoint{0.990826in}{0.607810in}}%
\pgfpathcurveto{\pgfqpoint{0.998640in}{0.615624in}}{\pgfqpoint{1.003030in}{0.626223in}}{\pgfqpoint{1.003030in}{0.637273in}}%
\pgfpathcurveto{\pgfqpoint{1.003030in}{0.648323in}}{\pgfqpoint{0.998640in}{0.658922in}}{\pgfqpoint{0.990826in}{0.666736in}}%
\pgfpathcurveto{\pgfqpoint{0.983013in}{0.674549in}}{\pgfqpoint{0.972414in}{0.678939in}}{\pgfqpoint{0.961364in}{0.678939in}}%
\pgfpathcurveto{\pgfqpoint{0.950314in}{0.678939in}}{\pgfqpoint{0.939714in}{0.674549in}}{\pgfqpoint{0.931901in}{0.666736in}}%
\pgfpathcurveto{\pgfqpoint{0.924087in}{0.658922in}}{\pgfqpoint{0.919697in}{0.648323in}}{\pgfqpoint{0.919697in}{0.637273in}}%
\pgfpathcurveto{\pgfqpoint{0.919697in}{0.626223in}}{\pgfqpoint{0.924087in}{0.615624in}}{\pgfqpoint{0.931901in}{0.607810in}}%
\pgfpathcurveto{\pgfqpoint{0.939714in}{0.599996in}}{\pgfqpoint{0.950314in}{0.595606in}}{\pgfqpoint{0.961364in}{0.595606in}}%
\pgfpathclose%
\pgfusepath{stroke,fill}%
\end{pgfscope}%
\begin{pgfscope}%
\pgfpathrectangle{\pgfqpoint{0.750000in}{0.500000in}}{\pgfqpoint{4.650000in}{3.020000in}}%
\pgfusepath{clip}%
\pgfsetbuttcap%
\pgfsetroundjoin%
\definecolor{currentfill}{rgb}{0.121569,0.466667,0.705882}%
\pgfsetfillcolor{currentfill}%
\pgfsetlinewidth{1.003750pt}%
\definecolor{currentstroke}{rgb}{0.121569,0.466667,0.705882}%
\pgfsetstrokecolor{currentstroke}%
\pgfsetdash{}{0pt}%
\pgfpathmoveto{\pgfqpoint{1.021753in}{0.595606in}}%
\pgfpathcurveto{\pgfqpoint{1.032803in}{0.595606in}}{\pgfqpoint{1.043402in}{0.599996in}}{\pgfqpoint{1.051216in}{0.607810in}}%
\pgfpathcurveto{\pgfqpoint{1.059030in}{0.615624in}}{\pgfqpoint{1.063420in}{0.626223in}}{\pgfqpoint{1.063420in}{0.637273in}}%
\pgfpathcurveto{\pgfqpoint{1.063420in}{0.648323in}}{\pgfqpoint{1.059030in}{0.658922in}}{\pgfqpoint{1.051216in}{0.666736in}}%
\pgfpathcurveto{\pgfqpoint{1.043402in}{0.674549in}}{\pgfqpoint{1.032803in}{0.678939in}}{\pgfqpoint{1.021753in}{0.678939in}}%
\pgfpathcurveto{\pgfqpoint{1.010703in}{0.678939in}}{\pgfqpoint{1.000104in}{0.674549in}}{\pgfqpoint{0.992290in}{0.666736in}}%
\pgfpathcurveto{\pgfqpoint{0.984477in}{0.658922in}}{\pgfqpoint{0.980087in}{0.648323in}}{\pgfqpoint{0.980087in}{0.637273in}}%
\pgfpathcurveto{\pgfqpoint{0.980087in}{0.626223in}}{\pgfqpoint{0.984477in}{0.615624in}}{\pgfqpoint{0.992290in}{0.607810in}}%
\pgfpathcurveto{\pgfqpoint{1.000104in}{0.599996in}}{\pgfqpoint{1.010703in}{0.595606in}}{\pgfqpoint{1.021753in}{0.595606in}}%
\pgfpathclose%
\pgfusepath{stroke,fill}%
\end{pgfscope}%
\begin{pgfscope}%
\pgfpathrectangle{\pgfqpoint{0.750000in}{0.500000in}}{\pgfqpoint{4.650000in}{3.020000in}}%
\pgfusepath{clip}%
\pgfsetbuttcap%
\pgfsetroundjoin%
\definecolor{currentfill}{rgb}{1.000000,0.498039,0.054902}%
\pgfsetfillcolor{currentfill}%
\pgfsetlinewidth{1.003750pt}%
\definecolor{currentstroke}{rgb}{1.000000,0.498039,0.054902}%
\pgfsetstrokecolor{currentstroke}%
\pgfsetdash{}{0pt}%
\pgfpathmoveto{\pgfqpoint{1.927597in}{2.936057in}}%
\pgfpathcurveto{\pgfqpoint{1.938648in}{2.936057in}}{\pgfqpoint{1.949247in}{2.940448in}}{\pgfqpoint{1.957060in}{2.948261in}}%
\pgfpathcurveto{\pgfqpoint{1.964874in}{2.956075in}}{\pgfqpoint{1.969264in}{2.966674in}}{\pgfqpoint{1.969264in}{2.977724in}}%
\pgfpathcurveto{\pgfqpoint{1.969264in}{2.988774in}}{\pgfqpoint{1.964874in}{2.999373in}}{\pgfqpoint{1.957060in}{3.007187in}}%
\pgfpathcurveto{\pgfqpoint{1.949247in}{3.015000in}}{\pgfqpoint{1.938648in}{3.019391in}}{\pgfqpoint{1.927597in}{3.019391in}}%
\pgfpathcurveto{\pgfqpoint{1.916547in}{3.019391in}}{\pgfqpoint{1.905948in}{3.015000in}}{\pgfqpoint{1.898135in}{3.007187in}}%
\pgfpathcurveto{\pgfqpoint{1.890321in}{2.999373in}}{\pgfqpoint{1.885931in}{2.988774in}}{\pgfqpoint{1.885931in}{2.977724in}}%
\pgfpathcurveto{\pgfqpoint{1.885931in}{2.966674in}}{\pgfqpoint{1.890321in}{2.956075in}}{\pgfqpoint{1.898135in}{2.948261in}}%
\pgfpathcurveto{\pgfqpoint{1.905948in}{2.940448in}}{\pgfqpoint{1.916547in}{2.936057in}}{\pgfqpoint{1.927597in}{2.936057in}}%
\pgfpathclose%
\pgfusepath{stroke,fill}%
\end{pgfscope}%
\begin{pgfscope}%
\pgfpathrectangle{\pgfqpoint{0.750000in}{0.500000in}}{\pgfqpoint{4.650000in}{3.020000in}}%
\pgfusepath{clip}%
\pgfsetbuttcap%
\pgfsetroundjoin%
\definecolor{currentfill}{rgb}{1.000000,0.498039,0.054902}%
\pgfsetfillcolor{currentfill}%
\pgfsetlinewidth{1.003750pt}%
\definecolor{currentstroke}{rgb}{1.000000,0.498039,0.054902}%
\pgfsetstrokecolor{currentstroke}%
\pgfsetdash{}{0pt}%
\pgfpathmoveto{\pgfqpoint{1.806818in}{2.936057in}}%
\pgfpathcurveto{\pgfqpoint{1.817868in}{2.936057in}}{\pgfqpoint{1.828467in}{2.940448in}}{\pgfqpoint{1.836281in}{2.948261in}}%
\pgfpathcurveto{\pgfqpoint{1.844095in}{2.956075in}}{\pgfqpoint{1.848485in}{2.966674in}}{\pgfqpoint{1.848485in}{2.977724in}}%
\pgfpathcurveto{\pgfqpoint{1.848485in}{2.988774in}}{\pgfqpoint{1.844095in}{2.999373in}}{\pgfqpoint{1.836281in}{3.007187in}}%
\pgfpathcurveto{\pgfqpoint{1.828467in}{3.015000in}}{\pgfqpoint{1.817868in}{3.019391in}}{\pgfqpoint{1.806818in}{3.019391in}}%
\pgfpathcurveto{\pgfqpoint{1.795768in}{3.019391in}}{\pgfqpoint{1.785169in}{3.015000in}}{\pgfqpoint{1.777355in}{3.007187in}}%
\pgfpathcurveto{\pgfqpoint{1.769542in}{2.999373in}}{\pgfqpoint{1.765152in}{2.988774in}}{\pgfqpoint{1.765152in}{2.977724in}}%
\pgfpathcurveto{\pgfqpoint{1.765152in}{2.966674in}}{\pgfqpoint{1.769542in}{2.956075in}}{\pgfqpoint{1.777355in}{2.948261in}}%
\pgfpathcurveto{\pgfqpoint{1.785169in}{2.940448in}}{\pgfqpoint{1.795768in}{2.936057in}}{\pgfqpoint{1.806818in}{2.936057in}}%
\pgfpathclose%
\pgfusepath{stroke,fill}%
\end{pgfscope}%
\begin{pgfscope}%
\pgfsetbuttcap%
\pgfsetroundjoin%
\definecolor{currentfill}{rgb}{0.000000,0.000000,0.000000}%
\pgfsetfillcolor{currentfill}%
\pgfsetlinewidth{0.803000pt}%
\definecolor{currentstroke}{rgb}{0.000000,0.000000,0.000000}%
\pgfsetstrokecolor{currentstroke}%
\pgfsetdash{}{0pt}%
\pgfsys@defobject{currentmarker}{\pgfqpoint{0.000000in}{-0.048611in}}{\pgfqpoint{0.000000in}{0.000000in}}{%
\pgfpathmoveto{\pgfqpoint{0.000000in}{0.000000in}}%
\pgfpathlineto{\pgfqpoint{0.000000in}{-0.048611in}}%
\pgfusepath{stroke,fill}%
}%
\begin{pgfscope}%
\pgfsys@transformshift{0.900974in}{0.500000in}%
\pgfsys@useobject{currentmarker}{}%
\end{pgfscope}%
\end{pgfscope}%
\begin{pgfscope}%
\definecolor{textcolor}{rgb}{0.000000,0.000000,0.000000}%
\pgfsetstrokecolor{textcolor}%
\pgfsetfillcolor{textcolor}%
\pgftext[x=0.900974in,y=0.402778in,,top]{\color{textcolor}\rmfamily\fontsize{10.000000}{12.000000}\selectfont \(\displaystyle {0}\)}%
\end{pgfscope}%
\begin{pgfscope}%
\pgfsetbuttcap%
\pgfsetroundjoin%
\definecolor{currentfill}{rgb}{0.000000,0.000000,0.000000}%
\pgfsetfillcolor{currentfill}%
\pgfsetlinewidth{0.803000pt}%
\definecolor{currentstroke}{rgb}{0.000000,0.000000,0.000000}%
\pgfsetstrokecolor{currentstroke}%
\pgfsetdash{}{0pt}%
\pgfsys@defobject{currentmarker}{\pgfqpoint{0.000000in}{-0.048611in}}{\pgfqpoint{0.000000in}{0.000000in}}{%
\pgfpathmoveto{\pgfqpoint{0.000000in}{0.000000in}}%
\pgfpathlineto{\pgfqpoint{0.000000in}{-0.048611in}}%
\pgfusepath{stroke,fill}%
}%
\begin{pgfscope}%
\pgfsys@transformshift{1.504870in}{0.500000in}%
\pgfsys@useobject{currentmarker}{}%
\end{pgfscope}%
\end{pgfscope}%
\begin{pgfscope}%
\definecolor{textcolor}{rgb}{0.000000,0.000000,0.000000}%
\pgfsetstrokecolor{textcolor}%
\pgfsetfillcolor{textcolor}%
\pgftext[x=1.504870in,y=0.402778in,,top]{\color{textcolor}\rmfamily\fontsize{10.000000}{12.000000}\selectfont \(\displaystyle {10}\)}%
\end{pgfscope}%
\begin{pgfscope}%
\pgfsetbuttcap%
\pgfsetroundjoin%
\definecolor{currentfill}{rgb}{0.000000,0.000000,0.000000}%
\pgfsetfillcolor{currentfill}%
\pgfsetlinewidth{0.803000pt}%
\definecolor{currentstroke}{rgb}{0.000000,0.000000,0.000000}%
\pgfsetstrokecolor{currentstroke}%
\pgfsetdash{}{0pt}%
\pgfsys@defobject{currentmarker}{\pgfqpoint{0.000000in}{-0.048611in}}{\pgfqpoint{0.000000in}{0.000000in}}{%
\pgfpathmoveto{\pgfqpoint{0.000000in}{0.000000in}}%
\pgfpathlineto{\pgfqpoint{0.000000in}{-0.048611in}}%
\pgfusepath{stroke,fill}%
}%
\begin{pgfscope}%
\pgfsys@transformshift{2.108766in}{0.500000in}%
\pgfsys@useobject{currentmarker}{}%
\end{pgfscope}%
\end{pgfscope}%
\begin{pgfscope}%
\definecolor{textcolor}{rgb}{0.000000,0.000000,0.000000}%
\pgfsetstrokecolor{textcolor}%
\pgfsetfillcolor{textcolor}%
\pgftext[x=2.108766in,y=0.402778in,,top]{\color{textcolor}\rmfamily\fontsize{10.000000}{12.000000}\selectfont \(\displaystyle {20}\)}%
\end{pgfscope}%
\begin{pgfscope}%
\pgfsetbuttcap%
\pgfsetroundjoin%
\definecolor{currentfill}{rgb}{0.000000,0.000000,0.000000}%
\pgfsetfillcolor{currentfill}%
\pgfsetlinewidth{0.803000pt}%
\definecolor{currentstroke}{rgb}{0.000000,0.000000,0.000000}%
\pgfsetstrokecolor{currentstroke}%
\pgfsetdash{}{0pt}%
\pgfsys@defobject{currentmarker}{\pgfqpoint{0.000000in}{-0.048611in}}{\pgfqpoint{0.000000in}{0.000000in}}{%
\pgfpathmoveto{\pgfqpoint{0.000000in}{0.000000in}}%
\pgfpathlineto{\pgfqpoint{0.000000in}{-0.048611in}}%
\pgfusepath{stroke,fill}%
}%
\begin{pgfscope}%
\pgfsys@transformshift{2.712662in}{0.500000in}%
\pgfsys@useobject{currentmarker}{}%
\end{pgfscope}%
\end{pgfscope}%
\begin{pgfscope}%
\definecolor{textcolor}{rgb}{0.000000,0.000000,0.000000}%
\pgfsetstrokecolor{textcolor}%
\pgfsetfillcolor{textcolor}%
\pgftext[x=2.712662in,y=0.402778in,,top]{\color{textcolor}\rmfamily\fontsize{10.000000}{12.000000}\selectfont \(\displaystyle {30}\)}%
\end{pgfscope}%
\begin{pgfscope}%
\pgfsetbuttcap%
\pgfsetroundjoin%
\definecolor{currentfill}{rgb}{0.000000,0.000000,0.000000}%
\pgfsetfillcolor{currentfill}%
\pgfsetlinewidth{0.803000pt}%
\definecolor{currentstroke}{rgb}{0.000000,0.000000,0.000000}%
\pgfsetstrokecolor{currentstroke}%
\pgfsetdash{}{0pt}%
\pgfsys@defobject{currentmarker}{\pgfqpoint{0.000000in}{-0.048611in}}{\pgfqpoint{0.000000in}{0.000000in}}{%
\pgfpathmoveto{\pgfqpoint{0.000000in}{0.000000in}}%
\pgfpathlineto{\pgfqpoint{0.000000in}{-0.048611in}}%
\pgfusepath{stroke,fill}%
}%
\begin{pgfscope}%
\pgfsys@transformshift{3.316558in}{0.500000in}%
\pgfsys@useobject{currentmarker}{}%
\end{pgfscope}%
\end{pgfscope}%
\begin{pgfscope}%
\definecolor{textcolor}{rgb}{0.000000,0.000000,0.000000}%
\pgfsetstrokecolor{textcolor}%
\pgfsetfillcolor{textcolor}%
\pgftext[x=3.316558in,y=0.402778in,,top]{\color{textcolor}\rmfamily\fontsize{10.000000}{12.000000}\selectfont \(\displaystyle {40}\)}%
\end{pgfscope}%
\begin{pgfscope}%
\pgfsetbuttcap%
\pgfsetroundjoin%
\definecolor{currentfill}{rgb}{0.000000,0.000000,0.000000}%
\pgfsetfillcolor{currentfill}%
\pgfsetlinewidth{0.803000pt}%
\definecolor{currentstroke}{rgb}{0.000000,0.000000,0.000000}%
\pgfsetstrokecolor{currentstroke}%
\pgfsetdash{}{0pt}%
\pgfsys@defobject{currentmarker}{\pgfqpoint{0.000000in}{-0.048611in}}{\pgfqpoint{0.000000in}{0.000000in}}{%
\pgfpathmoveto{\pgfqpoint{0.000000in}{0.000000in}}%
\pgfpathlineto{\pgfqpoint{0.000000in}{-0.048611in}}%
\pgfusepath{stroke,fill}%
}%
\begin{pgfscope}%
\pgfsys@transformshift{3.920455in}{0.500000in}%
\pgfsys@useobject{currentmarker}{}%
\end{pgfscope}%
\end{pgfscope}%
\begin{pgfscope}%
\definecolor{textcolor}{rgb}{0.000000,0.000000,0.000000}%
\pgfsetstrokecolor{textcolor}%
\pgfsetfillcolor{textcolor}%
\pgftext[x=3.920455in,y=0.402778in,,top]{\color{textcolor}\rmfamily\fontsize{10.000000}{12.000000}\selectfont \(\displaystyle {50}\)}%
\end{pgfscope}%
\begin{pgfscope}%
\pgfsetbuttcap%
\pgfsetroundjoin%
\definecolor{currentfill}{rgb}{0.000000,0.000000,0.000000}%
\pgfsetfillcolor{currentfill}%
\pgfsetlinewidth{0.803000pt}%
\definecolor{currentstroke}{rgb}{0.000000,0.000000,0.000000}%
\pgfsetstrokecolor{currentstroke}%
\pgfsetdash{}{0pt}%
\pgfsys@defobject{currentmarker}{\pgfqpoint{0.000000in}{-0.048611in}}{\pgfqpoint{0.000000in}{0.000000in}}{%
\pgfpathmoveto{\pgfqpoint{0.000000in}{0.000000in}}%
\pgfpathlineto{\pgfqpoint{0.000000in}{-0.048611in}}%
\pgfusepath{stroke,fill}%
}%
\begin{pgfscope}%
\pgfsys@transformshift{4.524351in}{0.500000in}%
\pgfsys@useobject{currentmarker}{}%
\end{pgfscope}%
\end{pgfscope}%
\begin{pgfscope}%
\definecolor{textcolor}{rgb}{0.000000,0.000000,0.000000}%
\pgfsetstrokecolor{textcolor}%
\pgfsetfillcolor{textcolor}%
\pgftext[x=4.524351in,y=0.402778in,,top]{\color{textcolor}\rmfamily\fontsize{10.000000}{12.000000}\selectfont \(\displaystyle {60}\)}%
\end{pgfscope}%
\begin{pgfscope}%
\pgfsetbuttcap%
\pgfsetroundjoin%
\definecolor{currentfill}{rgb}{0.000000,0.000000,0.000000}%
\pgfsetfillcolor{currentfill}%
\pgfsetlinewidth{0.803000pt}%
\definecolor{currentstroke}{rgb}{0.000000,0.000000,0.000000}%
\pgfsetstrokecolor{currentstroke}%
\pgfsetdash{}{0pt}%
\pgfsys@defobject{currentmarker}{\pgfqpoint{0.000000in}{-0.048611in}}{\pgfqpoint{0.000000in}{0.000000in}}{%
\pgfpathmoveto{\pgfqpoint{0.000000in}{0.000000in}}%
\pgfpathlineto{\pgfqpoint{0.000000in}{-0.048611in}}%
\pgfusepath{stroke,fill}%
}%
\begin{pgfscope}%
\pgfsys@transformshift{5.128247in}{0.500000in}%
\pgfsys@useobject{currentmarker}{}%
\end{pgfscope}%
\end{pgfscope}%
\begin{pgfscope}%
\definecolor{textcolor}{rgb}{0.000000,0.000000,0.000000}%
\pgfsetstrokecolor{textcolor}%
\pgfsetfillcolor{textcolor}%
\pgftext[x=5.128247in,y=0.402778in,,top]{\color{textcolor}\rmfamily\fontsize{10.000000}{12.000000}\selectfont \(\displaystyle {70}\)}%
\end{pgfscope}%
\begin{pgfscope}%
\definecolor{textcolor}{rgb}{0.000000,0.000000,0.000000}%
\pgfsetstrokecolor{textcolor}%
\pgfsetfillcolor{textcolor}%
\pgftext[x=3.075000in,y=0.223889in,,top]{\color{textcolor}\rmfamily\fontsize{10.000000}{12.000000}\selectfont Number of Sources}%
\end{pgfscope}%
\begin{pgfscope}%
\pgfsetbuttcap%
\pgfsetroundjoin%
\definecolor{currentfill}{rgb}{0.000000,0.000000,0.000000}%
\pgfsetfillcolor{currentfill}%
\pgfsetlinewidth{0.803000pt}%
\definecolor{currentstroke}{rgb}{0.000000,0.000000,0.000000}%
\pgfsetstrokecolor{currentstroke}%
\pgfsetdash{}{0pt}%
\pgfsys@defobject{currentmarker}{\pgfqpoint{-0.048611in}{0.000000in}}{\pgfqpoint{0.000000in}{0.000000in}}{%
\pgfpathmoveto{\pgfqpoint{0.000000in}{0.000000in}}%
\pgfpathlineto{\pgfqpoint{-0.048611in}{0.000000in}}%
\pgfusepath{stroke,fill}%
}%
\begin{pgfscope}%
\pgfsys@transformshift{0.750000in}{0.637273in}%
\pgfsys@useobject{currentmarker}{}%
\end{pgfscope}%
\end{pgfscope}%
\begin{pgfscope}%
\definecolor{textcolor}{rgb}{0.000000,0.000000,0.000000}%
\pgfsetstrokecolor{textcolor}%
\pgfsetfillcolor{textcolor}%
\pgftext[x=0.583333in, y=0.589078in, left, base]{\color{textcolor}\rmfamily\fontsize{10.000000}{12.000000}\selectfont \(\displaystyle {0}\)}%
\end{pgfscope}%
\begin{pgfscope}%
\pgfsetbuttcap%
\pgfsetroundjoin%
\definecolor{currentfill}{rgb}{0.000000,0.000000,0.000000}%
\pgfsetfillcolor{currentfill}%
\pgfsetlinewidth{0.803000pt}%
\definecolor{currentstroke}{rgb}{0.000000,0.000000,0.000000}%
\pgfsetstrokecolor{currentstroke}%
\pgfsetdash{}{0pt}%
\pgfsys@defobject{currentmarker}{\pgfqpoint{-0.048611in}{0.000000in}}{\pgfqpoint{0.000000in}{0.000000in}}{%
\pgfpathmoveto{\pgfqpoint{0.000000in}{0.000000in}}%
\pgfpathlineto{\pgfqpoint{-0.048611in}{0.000000in}}%
\pgfusepath{stroke,fill}%
}%
\begin{pgfscope}%
\pgfsys@transformshift{0.750000in}{1.026699in}%
\pgfsys@useobject{currentmarker}{}%
\end{pgfscope}%
\end{pgfscope}%
\begin{pgfscope}%
\definecolor{textcolor}{rgb}{0.000000,0.000000,0.000000}%
\pgfsetstrokecolor{textcolor}%
\pgfsetfillcolor{textcolor}%
\pgftext[x=0.444444in, y=0.978504in, left, base]{\color{textcolor}\rmfamily\fontsize{10.000000}{12.000000}\selectfont \(\displaystyle {100}\)}%
\end{pgfscope}%
\begin{pgfscope}%
\pgfsetbuttcap%
\pgfsetroundjoin%
\definecolor{currentfill}{rgb}{0.000000,0.000000,0.000000}%
\pgfsetfillcolor{currentfill}%
\pgfsetlinewidth{0.803000pt}%
\definecolor{currentstroke}{rgb}{0.000000,0.000000,0.000000}%
\pgfsetstrokecolor{currentstroke}%
\pgfsetdash{}{0pt}%
\pgfsys@defobject{currentmarker}{\pgfqpoint{-0.048611in}{0.000000in}}{\pgfqpoint{0.000000in}{0.000000in}}{%
\pgfpathmoveto{\pgfqpoint{0.000000in}{0.000000in}}%
\pgfpathlineto{\pgfqpoint{-0.048611in}{0.000000in}}%
\pgfusepath{stroke,fill}%
}%
\begin{pgfscope}%
\pgfsys@transformshift{0.750000in}{1.416125in}%
\pgfsys@useobject{currentmarker}{}%
\end{pgfscope}%
\end{pgfscope}%
\begin{pgfscope}%
\definecolor{textcolor}{rgb}{0.000000,0.000000,0.000000}%
\pgfsetstrokecolor{textcolor}%
\pgfsetfillcolor{textcolor}%
\pgftext[x=0.444444in, y=1.367931in, left, base]{\color{textcolor}\rmfamily\fontsize{10.000000}{12.000000}\selectfont \(\displaystyle {200}\)}%
\end{pgfscope}%
\begin{pgfscope}%
\pgfsetbuttcap%
\pgfsetroundjoin%
\definecolor{currentfill}{rgb}{0.000000,0.000000,0.000000}%
\pgfsetfillcolor{currentfill}%
\pgfsetlinewidth{0.803000pt}%
\definecolor{currentstroke}{rgb}{0.000000,0.000000,0.000000}%
\pgfsetstrokecolor{currentstroke}%
\pgfsetdash{}{0pt}%
\pgfsys@defobject{currentmarker}{\pgfqpoint{-0.048611in}{0.000000in}}{\pgfqpoint{0.000000in}{0.000000in}}{%
\pgfpathmoveto{\pgfqpoint{0.000000in}{0.000000in}}%
\pgfpathlineto{\pgfqpoint{-0.048611in}{0.000000in}}%
\pgfusepath{stroke,fill}%
}%
\begin{pgfscope}%
\pgfsys@transformshift{0.750000in}{1.805551in}%
\pgfsys@useobject{currentmarker}{}%
\end{pgfscope}%
\end{pgfscope}%
\begin{pgfscope}%
\definecolor{textcolor}{rgb}{0.000000,0.000000,0.000000}%
\pgfsetstrokecolor{textcolor}%
\pgfsetfillcolor{textcolor}%
\pgftext[x=0.444444in, y=1.757357in, left, base]{\color{textcolor}\rmfamily\fontsize{10.000000}{12.000000}\selectfont \(\displaystyle {300}\)}%
\end{pgfscope}%
\begin{pgfscope}%
\pgfsetbuttcap%
\pgfsetroundjoin%
\definecolor{currentfill}{rgb}{0.000000,0.000000,0.000000}%
\pgfsetfillcolor{currentfill}%
\pgfsetlinewidth{0.803000pt}%
\definecolor{currentstroke}{rgb}{0.000000,0.000000,0.000000}%
\pgfsetstrokecolor{currentstroke}%
\pgfsetdash{}{0pt}%
\pgfsys@defobject{currentmarker}{\pgfqpoint{-0.048611in}{0.000000in}}{\pgfqpoint{0.000000in}{0.000000in}}{%
\pgfpathmoveto{\pgfqpoint{0.000000in}{0.000000in}}%
\pgfpathlineto{\pgfqpoint{-0.048611in}{0.000000in}}%
\pgfusepath{stroke,fill}%
}%
\begin{pgfscope}%
\pgfsys@transformshift{0.750000in}{2.194977in}%
\pgfsys@useobject{currentmarker}{}%
\end{pgfscope}%
\end{pgfscope}%
\begin{pgfscope}%
\definecolor{textcolor}{rgb}{0.000000,0.000000,0.000000}%
\pgfsetstrokecolor{textcolor}%
\pgfsetfillcolor{textcolor}%
\pgftext[x=0.444444in, y=2.146783in, left, base]{\color{textcolor}\rmfamily\fontsize{10.000000}{12.000000}\selectfont \(\displaystyle {400}\)}%
\end{pgfscope}%
\begin{pgfscope}%
\pgfsetbuttcap%
\pgfsetroundjoin%
\definecolor{currentfill}{rgb}{0.000000,0.000000,0.000000}%
\pgfsetfillcolor{currentfill}%
\pgfsetlinewidth{0.803000pt}%
\definecolor{currentstroke}{rgb}{0.000000,0.000000,0.000000}%
\pgfsetstrokecolor{currentstroke}%
\pgfsetdash{}{0pt}%
\pgfsys@defobject{currentmarker}{\pgfqpoint{-0.048611in}{0.000000in}}{\pgfqpoint{0.000000in}{0.000000in}}{%
\pgfpathmoveto{\pgfqpoint{0.000000in}{0.000000in}}%
\pgfpathlineto{\pgfqpoint{-0.048611in}{0.000000in}}%
\pgfusepath{stroke,fill}%
}%
\begin{pgfscope}%
\pgfsys@transformshift{0.750000in}{2.584404in}%
\pgfsys@useobject{currentmarker}{}%
\end{pgfscope}%
\end{pgfscope}%
\begin{pgfscope}%
\definecolor{textcolor}{rgb}{0.000000,0.000000,0.000000}%
\pgfsetstrokecolor{textcolor}%
\pgfsetfillcolor{textcolor}%
\pgftext[x=0.444444in, y=2.536209in, left, base]{\color{textcolor}\rmfamily\fontsize{10.000000}{12.000000}\selectfont \(\displaystyle {500}\)}%
\end{pgfscope}%
\begin{pgfscope}%
\pgfsetbuttcap%
\pgfsetroundjoin%
\definecolor{currentfill}{rgb}{0.000000,0.000000,0.000000}%
\pgfsetfillcolor{currentfill}%
\pgfsetlinewidth{0.803000pt}%
\definecolor{currentstroke}{rgb}{0.000000,0.000000,0.000000}%
\pgfsetstrokecolor{currentstroke}%
\pgfsetdash{}{0pt}%
\pgfsys@defobject{currentmarker}{\pgfqpoint{-0.048611in}{0.000000in}}{\pgfqpoint{0.000000in}{0.000000in}}{%
\pgfpathmoveto{\pgfqpoint{0.000000in}{0.000000in}}%
\pgfpathlineto{\pgfqpoint{-0.048611in}{0.000000in}}%
\pgfusepath{stroke,fill}%
}%
\begin{pgfscope}%
\pgfsys@transformshift{0.750000in}{2.973830in}%
\pgfsys@useobject{currentmarker}{}%
\end{pgfscope}%
\end{pgfscope}%
\begin{pgfscope}%
\definecolor{textcolor}{rgb}{0.000000,0.000000,0.000000}%
\pgfsetstrokecolor{textcolor}%
\pgfsetfillcolor{textcolor}%
\pgftext[x=0.444444in, y=2.925635in, left, base]{\color{textcolor}\rmfamily\fontsize{10.000000}{12.000000}\selectfont \(\displaystyle {600}\)}%
\end{pgfscope}%
\begin{pgfscope}%
\pgfsetbuttcap%
\pgfsetroundjoin%
\definecolor{currentfill}{rgb}{0.000000,0.000000,0.000000}%
\pgfsetfillcolor{currentfill}%
\pgfsetlinewidth{0.803000pt}%
\definecolor{currentstroke}{rgb}{0.000000,0.000000,0.000000}%
\pgfsetstrokecolor{currentstroke}%
\pgfsetdash{}{0pt}%
\pgfsys@defobject{currentmarker}{\pgfqpoint{-0.048611in}{0.000000in}}{\pgfqpoint{0.000000in}{0.000000in}}{%
\pgfpathmoveto{\pgfqpoint{0.000000in}{0.000000in}}%
\pgfpathlineto{\pgfqpoint{-0.048611in}{0.000000in}}%
\pgfusepath{stroke,fill}%
}%
\begin{pgfscope}%
\pgfsys@transformshift{0.750000in}{3.363256in}%
\pgfsys@useobject{currentmarker}{}%
\end{pgfscope}%
\end{pgfscope}%
\begin{pgfscope}%
\definecolor{textcolor}{rgb}{0.000000,0.000000,0.000000}%
\pgfsetstrokecolor{textcolor}%
\pgfsetfillcolor{textcolor}%
\pgftext[x=0.444444in, y=3.315062in, left, base]{\color{textcolor}\rmfamily\fontsize{10.000000}{12.000000}\selectfont \(\displaystyle {700}\)}%
\end{pgfscope}%
\begin{pgfscope}%
\definecolor{textcolor}{rgb}{0.000000,0.000000,0.000000}%
\pgfsetstrokecolor{textcolor}%
\pgfsetfillcolor{textcolor}%
\pgftext[x=0.388888in,y=2.010000in,,bottom,rotate=90.000000]{\color{textcolor}\rmfamily\fontsize{10.000000}{12.000000}\selectfont Dataflow Time}%
\end{pgfscope}%
\begin{pgfscope}%
\pgfsetrectcap%
\pgfsetmiterjoin%
\pgfsetlinewidth{0.803000pt}%
\definecolor{currentstroke}{rgb}{0.000000,0.000000,0.000000}%
\pgfsetstrokecolor{currentstroke}%
\pgfsetdash{}{0pt}%
\pgfpathmoveto{\pgfqpoint{0.750000in}{0.500000in}}%
\pgfpathlineto{\pgfqpoint{0.750000in}{3.520000in}}%
\pgfusepath{stroke}%
\end{pgfscope}%
\begin{pgfscope}%
\pgfsetrectcap%
\pgfsetmiterjoin%
\pgfsetlinewidth{0.803000pt}%
\definecolor{currentstroke}{rgb}{0.000000,0.000000,0.000000}%
\pgfsetstrokecolor{currentstroke}%
\pgfsetdash{}{0pt}%
\pgfpathmoveto{\pgfqpoint{5.400000in}{0.500000in}}%
\pgfpathlineto{\pgfqpoint{5.400000in}{3.520000in}}%
\pgfusepath{stroke}%
\end{pgfscope}%
\begin{pgfscope}%
\pgfsetrectcap%
\pgfsetmiterjoin%
\pgfsetlinewidth{0.803000pt}%
\definecolor{currentstroke}{rgb}{0.000000,0.000000,0.000000}%
\pgfsetstrokecolor{currentstroke}%
\pgfsetdash{}{0pt}%
\pgfpathmoveto{\pgfqpoint{0.750000in}{0.500000in}}%
\pgfpathlineto{\pgfqpoint{5.400000in}{0.500000in}}%
\pgfusepath{stroke}%
\end{pgfscope}%
\begin{pgfscope}%
\pgfsetrectcap%
\pgfsetmiterjoin%
\pgfsetlinewidth{0.803000pt}%
\definecolor{currentstroke}{rgb}{0.000000,0.000000,0.000000}%
\pgfsetstrokecolor{currentstroke}%
\pgfsetdash{}{0pt}%
\pgfpathmoveto{\pgfqpoint{0.750000in}{3.520000in}}%
\pgfpathlineto{\pgfqpoint{5.400000in}{3.520000in}}%
\pgfusepath{stroke}%
\end{pgfscope}%
\begin{pgfscope}%
\definecolor{textcolor}{rgb}{0.000000,0.000000,0.000000}%
\pgfsetstrokecolor{textcolor}%
\pgfsetfillcolor{textcolor}%
\pgftext[x=3.075000in,y=3.603333in,,base]{\color{textcolor}\rmfamily\fontsize{12.000000}{14.400000}\selectfont Forwards}%
\end{pgfscope}%
\begin{pgfscope}%
\pgfsetbuttcap%
\pgfsetmiterjoin%
\definecolor{currentfill}{rgb}{1.000000,1.000000,1.000000}%
\pgfsetfillcolor{currentfill}%
\pgfsetfillopacity{0.800000}%
\pgfsetlinewidth{1.003750pt}%
\definecolor{currentstroke}{rgb}{0.800000,0.800000,0.800000}%
\pgfsetstrokecolor{currentstroke}%
\pgfsetstrokeopacity{0.800000}%
\pgfsetdash{}{0pt}%
\pgfpathmoveto{\pgfqpoint{3.793194in}{0.569444in}}%
\pgfpathlineto{\pgfqpoint{5.302778in}{0.569444in}}%
\pgfpathquadraticcurveto{\pgfqpoint{5.330556in}{0.569444in}}{\pgfqpoint{5.330556in}{0.597222in}}%
\pgfpathlineto{\pgfqpoint{5.330556in}{1.165694in}}%
\pgfpathquadraticcurveto{\pgfqpoint{5.330556in}{1.193472in}}{\pgfqpoint{5.302778in}{1.193472in}}%
\pgfpathlineto{\pgfqpoint{3.793194in}{1.193472in}}%
\pgfpathquadraticcurveto{\pgfqpoint{3.765417in}{1.193472in}}{\pgfqpoint{3.765417in}{1.165694in}}%
\pgfpathlineto{\pgfqpoint{3.765417in}{0.597222in}}%
\pgfpathquadraticcurveto{\pgfqpoint{3.765417in}{0.569444in}}{\pgfqpoint{3.793194in}{0.569444in}}%
\pgfpathclose%
\pgfusepath{stroke,fill}%
\end{pgfscope}%
\begin{pgfscope}%
\pgfsetbuttcap%
\pgfsetroundjoin%
\definecolor{currentfill}{rgb}{0.121569,0.466667,0.705882}%
\pgfsetfillcolor{currentfill}%
\pgfsetlinewidth{1.003750pt}%
\definecolor{currentstroke}{rgb}{0.121569,0.466667,0.705882}%
\pgfsetstrokecolor{currentstroke}%
\pgfsetdash{}{0pt}%
\pgfsys@defobject{currentmarker}{\pgfqpoint{-0.034722in}{-0.034722in}}{\pgfqpoint{0.034722in}{0.034722in}}{%
\pgfpathmoveto{\pgfqpoint{0.000000in}{-0.034722in}}%
\pgfpathcurveto{\pgfqpoint{0.009208in}{-0.034722in}}{\pgfqpoint{0.018041in}{-0.031064in}}{\pgfqpoint{0.024552in}{-0.024552in}}%
\pgfpathcurveto{\pgfqpoint{0.031064in}{-0.018041in}}{\pgfqpoint{0.034722in}{-0.009208in}}{\pgfqpoint{0.034722in}{0.000000in}}%
\pgfpathcurveto{\pgfqpoint{0.034722in}{0.009208in}}{\pgfqpoint{0.031064in}{0.018041in}}{\pgfqpoint{0.024552in}{0.024552in}}%
\pgfpathcurveto{\pgfqpoint{0.018041in}{0.031064in}}{\pgfqpoint{0.009208in}{0.034722in}}{\pgfqpoint{0.000000in}{0.034722in}}%
\pgfpathcurveto{\pgfqpoint{-0.009208in}{0.034722in}}{\pgfqpoint{-0.018041in}{0.031064in}}{\pgfqpoint{-0.024552in}{0.024552in}}%
\pgfpathcurveto{\pgfqpoint{-0.031064in}{0.018041in}}{\pgfqpoint{-0.034722in}{0.009208in}}{\pgfqpoint{-0.034722in}{0.000000in}}%
\pgfpathcurveto{\pgfqpoint{-0.034722in}{-0.009208in}}{\pgfqpoint{-0.031064in}{-0.018041in}}{\pgfqpoint{-0.024552in}{-0.024552in}}%
\pgfpathcurveto{\pgfqpoint{-0.018041in}{-0.031064in}}{\pgfqpoint{-0.009208in}{-0.034722in}}{\pgfqpoint{0.000000in}{-0.034722in}}%
\pgfpathclose%
\pgfusepath{stroke,fill}%
}%
\begin{pgfscope}%
\pgfsys@transformshift{3.959861in}{1.089306in}%
\pgfsys@useobject{currentmarker}{}%
\end{pgfscope}%
\end{pgfscope}%
\begin{pgfscope}%
\definecolor{textcolor}{rgb}{0.000000,0.000000,0.000000}%
\pgfsetstrokecolor{textcolor}%
\pgfsetfillcolor{textcolor}%
\pgftext[x=4.209861in,y=1.040694in,left,base]{\color{textcolor}\rmfamily\fontsize{10.000000}{12.000000}\selectfont No Timeout}%
\end{pgfscope}%
\begin{pgfscope}%
\pgfsetbuttcap%
\pgfsetroundjoin%
\definecolor{currentfill}{rgb}{1.000000,0.498039,0.054902}%
\pgfsetfillcolor{currentfill}%
\pgfsetlinewidth{1.003750pt}%
\definecolor{currentstroke}{rgb}{1.000000,0.498039,0.054902}%
\pgfsetstrokecolor{currentstroke}%
\pgfsetdash{}{0pt}%
\pgfsys@defobject{currentmarker}{\pgfqpoint{-0.034722in}{-0.034722in}}{\pgfqpoint{0.034722in}{0.034722in}}{%
\pgfpathmoveto{\pgfqpoint{0.000000in}{-0.034722in}}%
\pgfpathcurveto{\pgfqpoint{0.009208in}{-0.034722in}}{\pgfqpoint{0.018041in}{-0.031064in}}{\pgfqpoint{0.024552in}{-0.024552in}}%
\pgfpathcurveto{\pgfqpoint{0.031064in}{-0.018041in}}{\pgfqpoint{0.034722in}{-0.009208in}}{\pgfqpoint{0.034722in}{0.000000in}}%
\pgfpathcurveto{\pgfqpoint{0.034722in}{0.009208in}}{\pgfqpoint{0.031064in}{0.018041in}}{\pgfqpoint{0.024552in}{0.024552in}}%
\pgfpathcurveto{\pgfqpoint{0.018041in}{0.031064in}}{\pgfqpoint{0.009208in}{0.034722in}}{\pgfqpoint{0.000000in}{0.034722in}}%
\pgfpathcurveto{\pgfqpoint{-0.009208in}{0.034722in}}{\pgfqpoint{-0.018041in}{0.031064in}}{\pgfqpoint{-0.024552in}{0.024552in}}%
\pgfpathcurveto{\pgfqpoint{-0.031064in}{0.018041in}}{\pgfqpoint{-0.034722in}{0.009208in}}{\pgfqpoint{-0.034722in}{0.000000in}}%
\pgfpathcurveto{\pgfqpoint{-0.034722in}{-0.009208in}}{\pgfqpoint{-0.031064in}{-0.018041in}}{\pgfqpoint{-0.024552in}{-0.024552in}}%
\pgfpathcurveto{\pgfqpoint{-0.018041in}{-0.031064in}}{\pgfqpoint{-0.009208in}{-0.034722in}}{\pgfqpoint{0.000000in}{-0.034722in}}%
\pgfpathclose%
\pgfusepath{stroke,fill}%
}%
\begin{pgfscope}%
\pgfsys@transformshift{3.959861in}{0.895694in}%
\pgfsys@useobject{currentmarker}{}%
\end{pgfscope}%
\end{pgfscope}%
\begin{pgfscope}%
\definecolor{textcolor}{rgb}{0.000000,0.000000,0.000000}%
\pgfsetstrokecolor{textcolor}%
\pgfsetfillcolor{textcolor}%
\pgftext[x=4.209861in,y=0.847083in,left,base]{\color{textcolor}\rmfamily\fontsize{10.000000}{12.000000}\selectfont Time Timeout}%
\end{pgfscope}%
\begin{pgfscope}%
\pgfsetbuttcap%
\pgfsetroundjoin%
\definecolor{currentfill}{rgb}{0.839216,0.152941,0.156863}%
\pgfsetfillcolor{currentfill}%
\pgfsetlinewidth{1.003750pt}%
\definecolor{currentstroke}{rgb}{0.839216,0.152941,0.156863}%
\pgfsetstrokecolor{currentstroke}%
\pgfsetdash{}{0pt}%
\pgfsys@defobject{currentmarker}{\pgfqpoint{-0.034722in}{-0.034722in}}{\pgfqpoint{0.034722in}{0.034722in}}{%
\pgfpathmoveto{\pgfqpoint{0.000000in}{-0.034722in}}%
\pgfpathcurveto{\pgfqpoint{0.009208in}{-0.034722in}}{\pgfqpoint{0.018041in}{-0.031064in}}{\pgfqpoint{0.024552in}{-0.024552in}}%
\pgfpathcurveto{\pgfqpoint{0.031064in}{-0.018041in}}{\pgfqpoint{0.034722in}{-0.009208in}}{\pgfqpoint{0.034722in}{0.000000in}}%
\pgfpathcurveto{\pgfqpoint{0.034722in}{0.009208in}}{\pgfqpoint{0.031064in}{0.018041in}}{\pgfqpoint{0.024552in}{0.024552in}}%
\pgfpathcurveto{\pgfqpoint{0.018041in}{0.031064in}}{\pgfqpoint{0.009208in}{0.034722in}}{\pgfqpoint{0.000000in}{0.034722in}}%
\pgfpathcurveto{\pgfqpoint{-0.009208in}{0.034722in}}{\pgfqpoint{-0.018041in}{0.031064in}}{\pgfqpoint{-0.024552in}{0.024552in}}%
\pgfpathcurveto{\pgfqpoint{-0.031064in}{0.018041in}}{\pgfqpoint{-0.034722in}{0.009208in}}{\pgfqpoint{-0.034722in}{0.000000in}}%
\pgfpathcurveto{\pgfqpoint{-0.034722in}{-0.009208in}}{\pgfqpoint{-0.031064in}{-0.018041in}}{\pgfqpoint{-0.024552in}{-0.024552in}}%
\pgfpathcurveto{\pgfqpoint{-0.018041in}{-0.031064in}}{\pgfqpoint{-0.009208in}{-0.034722in}}{\pgfqpoint{0.000000in}{-0.034722in}}%
\pgfpathclose%
\pgfusepath{stroke,fill}%
}%
\begin{pgfscope}%
\pgfsys@transformshift{3.959861in}{0.702083in}%
\pgfsys@useobject{currentmarker}{}%
\end{pgfscope}%
\end{pgfscope}%
\begin{pgfscope}%
\definecolor{textcolor}{rgb}{0.000000,0.000000,0.000000}%
\pgfsetstrokecolor{textcolor}%
\pgfsetfillcolor{textcolor}%
\pgftext[x=4.209861in,y=0.653472in,left,base]{\color{textcolor}\rmfamily\fontsize{10.000000}{12.000000}\selectfont Memory Timeout}%
\end{pgfscope}%
\end{pgfpicture}%
\makeatother%
\endgroup%

            }
        \end{subfigure}
        \qquad
        \begin{subfigure}[]{0.45\textwidth}
            \centering
            \resizebox{\columnwidth}{!}{
                %% Creator: Matplotlib, PGF backend
%%
%% To include the figure in your LaTeX document, write
%%   \input{<filename>.pgf}
%%
%% Make sure the required packages are loaded in your preamble
%%   \usepackage{pgf}
%%
%% and, on pdftex
%%   \usepackage[utf8]{inputenc}\DeclareUnicodeCharacter{2212}{-}
%%
%% or, on luatex and xetex
%%   \usepackage{unicode-math}
%%
%% Figures using additional raster images can only be included by \input if
%% they are in the same directory as the main LaTeX file. For loading figures
%% from other directories you can use the `import` package
%%   \usepackage{import}
%%
%% and then include the figures with
%%   \import{<path to file>}{<filename>.pgf}
%%
%% Matplotlib used the following preamble
%%   \usepackage{amsmath}
%%   \usepackage{fontspec}
%%
\begingroup%
\makeatletter%
\begin{pgfpicture}%
\pgfpathrectangle{\pgfpointorigin}{\pgfqpoint{6.000000in}{4.000000in}}%
\pgfusepath{use as bounding box, clip}%
\begin{pgfscope}%
\pgfsetbuttcap%
\pgfsetmiterjoin%
\definecolor{currentfill}{rgb}{1.000000,1.000000,1.000000}%
\pgfsetfillcolor{currentfill}%
\pgfsetlinewidth{0.000000pt}%
\definecolor{currentstroke}{rgb}{1.000000,1.000000,1.000000}%
\pgfsetstrokecolor{currentstroke}%
\pgfsetdash{}{0pt}%
\pgfpathmoveto{\pgfqpoint{0.000000in}{0.000000in}}%
\pgfpathlineto{\pgfqpoint{6.000000in}{0.000000in}}%
\pgfpathlineto{\pgfqpoint{6.000000in}{4.000000in}}%
\pgfpathlineto{\pgfqpoint{0.000000in}{4.000000in}}%
\pgfpathclose%
\pgfusepath{fill}%
\end{pgfscope}%
\begin{pgfscope}%
\pgfsetbuttcap%
\pgfsetmiterjoin%
\definecolor{currentfill}{rgb}{1.000000,1.000000,1.000000}%
\pgfsetfillcolor{currentfill}%
\pgfsetlinewidth{0.000000pt}%
\definecolor{currentstroke}{rgb}{0.000000,0.000000,0.000000}%
\pgfsetstrokecolor{currentstroke}%
\pgfsetstrokeopacity{0.000000}%
\pgfsetdash{}{0pt}%
\pgfpathmoveto{\pgfqpoint{0.750000in}{0.500000in}}%
\pgfpathlineto{\pgfqpoint{5.400000in}{0.500000in}}%
\pgfpathlineto{\pgfqpoint{5.400000in}{3.520000in}}%
\pgfpathlineto{\pgfqpoint{0.750000in}{3.520000in}}%
\pgfpathclose%
\pgfusepath{fill}%
\end{pgfscope}%
\begin{pgfscope}%
\pgfpathrectangle{\pgfqpoint{0.750000in}{0.500000in}}{\pgfqpoint{4.650000in}{3.020000in}}%
\pgfusepath{clip}%
\pgfsetbuttcap%
\pgfsetroundjoin%
\definecolor{currentfill}{rgb}{0.121569,0.466667,0.705882}%
\pgfsetfillcolor{currentfill}%
\pgfsetlinewidth{1.003750pt}%
\definecolor{currentstroke}{rgb}{0.121569,0.466667,0.705882}%
\pgfsetstrokecolor{currentstroke}%
\pgfsetdash{}{0pt}%
\pgfpathmoveto{\pgfqpoint{1.030663in}{0.595606in}}%
\pgfpathcurveto{\pgfqpoint{1.041713in}{0.595606in}}{\pgfqpoint{1.052312in}{0.599996in}}{\pgfqpoint{1.060126in}{0.607810in}}%
\pgfpathcurveto{\pgfqpoint{1.067940in}{0.615624in}}{\pgfqpoint{1.072330in}{0.626223in}}{\pgfqpoint{1.072330in}{0.637273in}}%
\pgfpathcurveto{\pgfqpoint{1.072330in}{0.648323in}}{\pgfqpoint{1.067940in}{0.658922in}}{\pgfqpoint{1.060126in}{0.666736in}}%
\pgfpathcurveto{\pgfqpoint{1.052312in}{0.674549in}}{\pgfqpoint{1.041713in}{0.678939in}}{\pgfqpoint{1.030663in}{0.678939in}}%
\pgfpathcurveto{\pgfqpoint{1.019613in}{0.678939in}}{\pgfqpoint{1.009014in}{0.674549in}}{\pgfqpoint{1.001200in}{0.666736in}}%
\pgfpathcurveto{\pgfqpoint{0.993387in}{0.658922in}}{\pgfqpoint{0.988997in}{0.648323in}}{\pgfqpoint{0.988997in}{0.637273in}}%
\pgfpathcurveto{\pgfqpoint{0.988997in}{0.626223in}}{\pgfqpoint{0.993387in}{0.615624in}}{\pgfqpoint{1.001200in}{0.607810in}}%
\pgfpathcurveto{\pgfqpoint{1.009014in}{0.599996in}}{\pgfqpoint{1.019613in}{0.595606in}}{\pgfqpoint{1.030663in}{0.595606in}}%
\pgfpathclose%
\pgfusepath{stroke,fill}%
\end{pgfscope}%
\begin{pgfscope}%
\pgfpathrectangle{\pgfqpoint{0.750000in}{0.500000in}}{\pgfqpoint{4.650000in}{3.020000in}}%
\pgfusepath{clip}%
\pgfsetbuttcap%
\pgfsetroundjoin%
\definecolor{currentfill}{rgb}{0.121569,0.466667,0.705882}%
\pgfsetfillcolor{currentfill}%
\pgfsetlinewidth{1.003750pt}%
\definecolor{currentstroke}{rgb}{0.121569,0.466667,0.705882}%
\pgfsetstrokecolor{currentstroke}%
\pgfsetdash{}{0pt}%
\pgfpathmoveto{\pgfqpoint{2.624553in}{3.014029in}}%
\pgfpathcurveto{\pgfqpoint{2.635603in}{3.014029in}}{\pgfqpoint{2.646202in}{3.018419in}}{\pgfqpoint{2.654016in}{3.026232in}}%
\pgfpathcurveto{\pgfqpoint{2.661829in}{3.034046in}}{\pgfqpoint{2.666220in}{3.044645in}}{\pgfqpoint{2.666220in}{3.055695in}}%
\pgfpathcurveto{\pgfqpoint{2.666220in}{3.066745in}}{\pgfqpoint{2.661829in}{3.077344in}}{\pgfqpoint{2.654016in}{3.085158in}}%
\pgfpathcurveto{\pgfqpoint{2.646202in}{3.092972in}}{\pgfqpoint{2.635603in}{3.097362in}}{\pgfqpoint{2.624553in}{3.097362in}}%
\pgfpathcurveto{\pgfqpoint{2.613503in}{3.097362in}}{\pgfqpoint{2.602904in}{3.092972in}}{\pgfqpoint{2.595090in}{3.085158in}}%
\pgfpathcurveto{\pgfqpoint{2.587277in}{3.077344in}}{\pgfqpoint{2.582886in}{3.066745in}}{\pgfqpoint{2.582886in}{3.055695in}}%
\pgfpathcurveto{\pgfqpoint{2.582886in}{3.044645in}}{\pgfqpoint{2.587277in}{3.034046in}}{\pgfqpoint{2.595090in}{3.026232in}}%
\pgfpathcurveto{\pgfqpoint{2.602904in}{3.018419in}}{\pgfqpoint{2.613503in}{3.014029in}}{\pgfqpoint{2.624553in}{3.014029in}}%
\pgfpathclose%
\pgfusepath{stroke,fill}%
\end{pgfscope}%
\begin{pgfscope}%
\pgfpathrectangle{\pgfqpoint{0.750000in}{0.500000in}}{\pgfqpoint{4.650000in}{3.020000in}}%
\pgfusepath{clip}%
\pgfsetbuttcap%
\pgfsetroundjoin%
\definecolor{currentfill}{rgb}{1.000000,0.498039,0.054902}%
\pgfsetfillcolor{currentfill}%
\pgfsetlinewidth{1.003750pt}%
\definecolor{currentstroke}{rgb}{1.000000,0.498039,0.054902}%
\pgfsetstrokecolor{currentstroke}%
\pgfsetdash{}{0pt}%
\pgfpathmoveto{\pgfqpoint{1.446461in}{3.018066in}}%
\pgfpathcurveto{\pgfqpoint{1.457511in}{3.018066in}}{\pgfqpoint{1.468110in}{3.022456in}}{\pgfqpoint{1.475923in}{3.030270in}}%
\pgfpathcurveto{\pgfqpoint{1.483737in}{3.038083in}}{\pgfqpoint{1.488127in}{3.048682in}}{\pgfqpoint{1.488127in}{3.059733in}}%
\pgfpathcurveto{\pgfqpoint{1.488127in}{3.070783in}}{\pgfqpoint{1.483737in}{3.081382in}}{\pgfqpoint{1.475923in}{3.089195in}}%
\pgfpathcurveto{\pgfqpoint{1.468110in}{3.097009in}}{\pgfqpoint{1.457511in}{3.101399in}}{\pgfqpoint{1.446461in}{3.101399in}}%
\pgfpathcurveto{\pgfqpoint{1.435410in}{3.101399in}}{\pgfqpoint{1.424811in}{3.097009in}}{\pgfqpoint{1.416998in}{3.089195in}}%
\pgfpathcurveto{\pgfqpoint{1.409184in}{3.081382in}}{\pgfqpoint{1.404794in}{3.070783in}}{\pgfqpoint{1.404794in}{3.059733in}}%
\pgfpathcurveto{\pgfqpoint{1.404794in}{3.048682in}}{\pgfqpoint{1.409184in}{3.038083in}}{\pgfqpoint{1.416998in}{3.030270in}}%
\pgfpathcurveto{\pgfqpoint{1.424811in}{3.022456in}}{\pgfqpoint{1.435410in}{3.018066in}}{\pgfqpoint{1.446461in}{3.018066in}}%
\pgfpathclose%
\pgfusepath{stroke,fill}%
\end{pgfscope}%
\begin{pgfscope}%
\pgfpathrectangle{\pgfqpoint{0.750000in}{0.500000in}}{\pgfqpoint{4.650000in}{3.020000in}}%
\pgfusepath{clip}%
\pgfsetbuttcap%
\pgfsetroundjoin%
\definecolor{currentfill}{rgb}{0.121569,0.466667,0.705882}%
\pgfsetfillcolor{currentfill}%
\pgfsetlinewidth{1.003750pt}%
\definecolor{currentstroke}{rgb}{0.121569,0.466667,0.705882}%
\pgfsetstrokecolor{currentstroke}%
\pgfsetdash{}{0pt}%
\pgfpathmoveto{\pgfqpoint{4.426341in}{3.014029in}}%
\pgfpathcurveto{\pgfqpoint{4.437391in}{3.014029in}}{\pgfqpoint{4.447990in}{3.018419in}}{\pgfqpoint{4.455804in}{3.026232in}}%
\pgfpathcurveto{\pgfqpoint{4.463618in}{3.034046in}}{\pgfqpoint{4.468008in}{3.044645in}}{\pgfqpoint{4.468008in}{3.055695in}}%
\pgfpathcurveto{\pgfqpoint{4.468008in}{3.066745in}}{\pgfqpoint{4.463618in}{3.077344in}}{\pgfqpoint{4.455804in}{3.085158in}}%
\pgfpathcurveto{\pgfqpoint{4.447990in}{3.092972in}}{\pgfqpoint{4.437391in}{3.097362in}}{\pgfqpoint{4.426341in}{3.097362in}}%
\pgfpathcurveto{\pgfqpoint{4.415291in}{3.097362in}}{\pgfqpoint{4.404692in}{3.092972in}}{\pgfqpoint{4.396878in}{3.085158in}}%
\pgfpathcurveto{\pgfqpoint{4.389065in}{3.077344in}}{\pgfqpoint{4.384675in}{3.066745in}}{\pgfqpoint{4.384675in}{3.055695in}}%
\pgfpathcurveto{\pgfqpoint{4.384675in}{3.044645in}}{\pgfqpoint{4.389065in}{3.034046in}}{\pgfqpoint{4.396878in}{3.026232in}}%
\pgfpathcurveto{\pgfqpoint{4.404692in}{3.018419in}}{\pgfqpoint{4.415291in}{3.014029in}}{\pgfqpoint{4.426341in}{3.014029in}}%
\pgfpathclose%
\pgfusepath{stroke,fill}%
\end{pgfscope}%
\begin{pgfscope}%
\pgfpathrectangle{\pgfqpoint{0.750000in}{0.500000in}}{\pgfqpoint{4.650000in}{3.020000in}}%
\pgfusepath{clip}%
\pgfsetbuttcap%
\pgfsetroundjoin%
\definecolor{currentfill}{rgb}{0.121569,0.466667,0.705882}%
\pgfsetfillcolor{currentfill}%
\pgfsetlinewidth{1.003750pt}%
\definecolor{currentstroke}{rgb}{0.121569,0.466667,0.705882}%
\pgfsetstrokecolor{currentstroke}%
\pgfsetdash{}{0pt}%
\pgfpathmoveto{\pgfqpoint{3.178949in}{3.009991in}}%
\pgfpathcurveto{\pgfqpoint{3.189999in}{3.009991in}}{\pgfqpoint{3.200598in}{3.014381in}}{\pgfqpoint{3.208412in}{3.022195in}}%
\pgfpathcurveto{\pgfqpoint{3.216226in}{3.030009in}}{\pgfqpoint{3.220616in}{3.040608in}}{\pgfqpoint{3.220616in}{3.051658in}}%
\pgfpathcurveto{\pgfqpoint{3.220616in}{3.062708in}}{\pgfqpoint{3.216226in}{3.073307in}}{\pgfqpoint{3.208412in}{3.081121in}}%
\pgfpathcurveto{\pgfqpoint{3.200598in}{3.088934in}}{\pgfqpoint{3.189999in}{3.093324in}}{\pgfqpoint{3.178949in}{3.093324in}}%
\pgfpathcurveto{\pgfqpoint{3.167899in}{3.093324in}}{\pgfqpoint{3.157300in}{3.088934in}}{\pgfqpoint{3.149487in}{3.081121in}}%
\pgfpathcurveto{\pgfqpoint{3.141673in}{3.073307in}}{\pgfqpoint{3.137283in}{3.062708in}}{\pgfqpoint{3.137283in}{3.051658in}}%
\pgfpathcurveto{\pgfqpoint{3.137283in}{3.040608in}}{\pgfqpoint{3.141673in}{3.030009in}}{\pgfqpoint{3.149487in}{3.022195in}}%
\pgfpathcurveto{\pgfqpoint{3.157300in}{3.014381in}}{\pgfqpoint{3.167899in}{3.009991in}}{\pgfqpoint{3.178949in}{3.009991in}}%
\pgfpathclose%
\pgfusepath{stroke,fill}%
\end{pgfscope}%
\begin{pgfscope}%
\pgfpathrectangle{\pgfqpoint{0.750000in}{0.500000in}}{\pgfqpoint{4.650000in}{3.020000in}}%
\pgfusepath{clip}%
\pgfsetbuttcap%
\pgfsetroundjoin%
\definecolor{currentfill}{rgb}{1.000000,0.498039,0.054902}%
\pgfsetfillcolor{currentfill}%
\pgfsetlinewidth{1.003750pt}%
\definecolor{currentstroke}{rgb}{1.000000,0.498039,0.054902}%
\pgfsetstrokecolor{currentstroke}%
\pgfsetdash{}{0pt}%
\pgfpathmoveto{\pgfqpoint{1.862258in}{3.191676in}}%
\pgfpathcurveto{\pgfqpoint{1.873308in}{3.191676in}}{\pgfqpoint{1.883907in}{3.196066in}}{\pgfqpoint{1.891721in}{3.203879in}}%
\pgfpathcurveto{\pgfqpoint{1.899534in}{3.211693in}}{\pgfqpoint{1.903924in}{3.222292in}}{\pgfqpoint{1.903924in}{3.233342in}}%
\pgfpathcurveto{\pgfqpoint{1.903924in}{3.244392in}}{\pgfqpoint{1.899534in}{3.254991in}}{\pgfqpoint{1.891721in}{3.262805in}}%
\pgfpathcurveto{\pgfqpoint{1.883907in}{3.270619in}}{\pgfqpoint{1.873308in}{3.275009in}}{\pgfqpoint{1.862258in}{3.275009in}}%
\pgfpathcurveto{\pgfqpoint{1.851208in}{3.275009in}}{\pgfqpoint{1.840609in}{3.270619in}}{\pgfqpoint{1.832795in}{3.262805in}}%
\pgfpathcurveto{\pgfqpoint{1.824981in}{3.254991in}}{\pgfqpoint{1.820591in}{3.244392in}}{\pgfqpoint{1.820591in}{3.233342in}}%
\pgfpathcurveto{\pgfqpoint{1.820591in}{3.222292in}}{\pgfqpoint{1.824981in}{3.211693in}}{\pgfqpoint{1.832795in}{3.203879in}}%
\pgfpathcurveto{\pgfqpoint{1.840609in}{3.196066in}}{\pgfqpoint{1.851208in}{3.191676in}}{\pgfqpoint{1.862258in}{3.191676in}}%
\pgfpathclose%
\pgfusepath{stroke,fill}%
\end{pgfscope}%
\begin{pgfscope}%
\pgfpathrectangle{\pgfqpoint{0.750000in}{0.500000in}}{\pgfqpoint{4.650000in}{3.020000in}}%
\pgfusepath{clip}%
\pgfsetbuttcap%
\pgfsetroundjoin%
\definecolor{currentfill}{rgb}{0.121569,0.466667,0.705882}%
\pgfsetfillcolor{currentfill}%
\pgfsetlinewidth{1.003750pt}%
\definecolor{currentstroke}{rgb}{0.121569,0.466667,0.705882}%
\pgfsetstrokecolor{currentstroke}%
\pgfsetdash{}{0pt}%
\pgfpathmoveto{\pgfqpoint{2.000857in}{3.009991in}}%
\pgfpathcurveto{\pgfqpoint{2.011907in}{3.009991in}}{\pgfqpoint{2.022506in}{3.014381in}}{\pgfqpoint{2.030320in}{3.022195in}}%
\pgfpathcurveto{\pgfqpoint{2.038133in}{3.030009in}}{\pgfqpoint{2.042524in}{3.040608in}}{\pgfqpoint{2.042524in}{3.051658in}}%
\pgfpathcurveto{\pgfqpoint{2.042524in}{3.062708in}}{\pgfqpoint{2.038133in}{3.073307in}}{\pgfqpoint{2.030320in}{3.081121in}}%
\pgfpathcurveto{\pgfqpoint{2.022506in}{3.088934in}}{\pgfqpoint{2.011907in}{3.093324in}}{\pgfqpoint{2.000857in}{3.093324in}}%
\pgfpathcurveto{\pgfqpoint{1.989807in}{3.093324in}}{\pgfqpoint{1.979208in}{3.088934in}}{\pgfqpoint{1.971394in}{3.081121in}}%
\pgfpathcurveto{\pgfqpoint{1.963581in}{3.073307in}}{\pgfqpoint{1.959190in}{3.062708in}}{\pgfqpoint{1.959190in}{3.051658in}}%
\pgfpathcurveto{\pgfqpoint{1.959190in}{3.040608in}}{\pgfqpoint{1.963581in}{3.030009in}}{\pgfqpoint{1.971394in}{3.022195in}}%
\pgfpathcurveto{\pgfqpoint{1.979208in}{3.014381in}}{\pgfqpoint{1.989807in}{3.009991in}}{\pgfqpoint{2.000857in}{3.009991in}}%
\pgfpathclose%
\pgfusepath{stroke,fill}%
\end{pgfscope}%
\begin{pgfscope}%
\pgfpathrectangle{\pgfqpoint{0.750000in}{0.500000in}}{\pgfqpoint{4.650000in}{3.020000in}}%
\pgfusepath{clip}%
\pgfsetbuttcap%
\pgfsetroundjoin%
\definecolor{currentfill}{rgb}{1.000000,0.498039,0.054902}%
\pgfsetfillcolor{currentfill}%
\pgfsetlinewidth{1.003750pt}%
\definecolor{currentstroke}{rgb}{1.000000,0.498039,0.054902}%
\pgfsetstrokecolor{currentstroke}%
\pgfsetdash{}{0pt}%
\pgfpathmoveto{\pgfqpoint{1.307861in}{3.018066in}}%
\pgfpathcurveto{\pgfqpoint{1.318912in}{3.018066in}}{\pgfqpoint{1.329511in}{3.022456in}}{\pgfqpoint{1.337324in}{3.030270in}}%
\pgfpathcurveto{\pgfqpoint{1.345138in}{3.038083in}}{\pgfqpoint{1.349528in}{3.048682in}}{\pgfqpoint{1.349528in}{3.059733in}}%
\pgfpathcurveto{\pgfqpoint{1.349528in}{3.070783in}}{\pgfqpoint{1.345138in}{3.081382in}}{\pgfqpoint{1.337324in}{3.089195in}}%
\pgfpathcurveto{\pgfqpoint{1.329511in}{3.097009in}}{\pgfqpoint{1.318912in}{3.101399in}}{\pgfqpoint{1.307861in}{3.101399in}}%
\pgfpathcurveto{\pgfqpoint{1.296811in}{3.101399in}}{\pgfqpoint{1.286212in}{3.097009in}}{\pgfqpoint{1.278399in}{3.089195in}}%
\pgfpathcurveto{\pgfqpoint{1.270585in}{3.081382in}}{\pgfqpoint{1.266195in}{3.070783in}}{\pgfqpoint{1.266195in}{3.059733in}}%
\pgfpathcurveto{\pgfqpoint{1.266195in}{3.048682in}}{\pgfqpoint{1.270585in}{3.038083in}}{\pgfqpoint{1.278399in}{3.030270in}}%
\pgfpathcurveto{\pgfqpoint{1.286212in}{3.022456in}}{\pgfqpoint{1.296811in}{3.018066in}}{\pgfqpoint{1.307861in}{3.018066in}}%
\pgfpathclose%
\pgfusepath{stroke,fill}%
\end{pgfscope}%
\begin{pgfscope}%
\pgfpathrectangle{\pgfqpoint{0.750000in}{0.500000in}}{\pgfqpoint{4.650000in}{3.020000in}}%
\pgfusepath{clip}%
\pgfsetbuttcap%
\pgfsetroundjoin%
\definecolor{currentfill}{rgb}{1.000000,0.498039,0.054902}%
\pgfsetfillcolor{currentfill}%
\pgfsetlinewidth{1.003750pt}%
\definecolor{currentstroke}{rgb}{1.000000,0.498039,0.054902}%
\pgfsetstrokecolor{currentstroke}%
\pgfsetdash{}{0pt}%
\pgfpathmoveto{\pgfqpoint{4.149143in}{3.018066in}}%
\pgfpathcurveto{\pgfqpoint{4.160193in}{3.018066in}}{\pgfqpoint{4.170792in}{3.022456in}}{\pgfqpoint{4.178606in}{3.030270in}}%
\pgfpathcurveto{\pgfqpoint{4.186419in}{3.038083in}}{\pgfqpoint{4.190810in}{3.048682in}}{\pgfqpoint{4.190810in}{3.059733in}}%
\pgfpathcurveto{\pgfqpoint{4.190810in}{3.070783in}}{\pgfqpoint{4.186419in}{3.081382in}}{\pgfqpoint{4.178606in}{3.089195in}}%
\pgfpathcurveto{\pgfqpoint{4.170792in}{3.097009in}}{\pgfqpoint{4.160193in}{3.101399in}}{\pgfqpoint{4.149143in}{3.101399in}}%
\pgfpathcurveto{\pgfqpoint{4.138093in}{3.101399in}}{\pgfqpoint{4.127494in}{3.097009in}}{\pgfqpoint{4.119680in}{3.089195in}}%
\pgfpathcurveto{\pgfqpoint{4.111867in}{3.081382in}}{\pgfqpoint{4.107476in}{3.070783in}}{\pgfqpoint{4.107476in}{3.059733in}}%
\pgfpathcurveto{\pgfqpoint{4.107476in}{3.048682in}}{\pgfqpoint{4.111867in}{3.038083in}}{\pgfqpoint{4.119680in}{3.030270in}}%
\pgfpathcurveto{\pgfqpoint{4.127494in}{3.022456in}}{\pgfqpoint{4.138093in}{3.018066in}}{\pgfqpoint{4.149143in}{3.018066in}}%
\pgfpathclose%
\pgfusepath{stroke,fill}%
\end{pgfscope}%
\begin{pgfscope}%
\pgfpathrectangle{\pgfqpoint{0.750000in}{0.500000in}}{\pgfqpoint{4.650000in}{3.020000in}}%
\pgfusepath{clip}%
\pgfsetbuttcap%
\pgfsetroundjoin%
\definecolor{currentfill}{rgb}{1.000000,0.498039,0.054902}%
\pgfsetfillcolor{currentfill}%
\pgfsetlinewidth{1.003750pt}%
\definecolor{currentstroke}{rgb}{1.000000,0.498039,0.054902}%
\pgfsetstrokecolor{currentstroke}%
\pgfsetdash{}{0pt}%
\pgfpathmoveto{\pgfqpoint{1.723659in}{3.030178in}}%
\pgfpathcurveto{\pgfqpoint{1.734709in}{3.030178in}}{\pgfqpoint{1.745308in}{3.034569in}}{\pgfqpoint{1.753122in}{3.042382in}}%
\pgfpathcurveto{\pgfqpoint{1.760935in}{3.050196in}}{\pgfqpoint{1.765325in}{3.060795in}}{\pgfqpoint{1.765325in}{3.071845in}}%
\pgfpathcurveto{\pgfqpoint{1.765325in}{3.082895in}}{\pgfqpoint{1.760935in}{3.093494in}}{\pgfqpoint{1.753122in}{3.101308in}}%
\pgfpathcurveto{\pgfqpoint{1.745308in}{3.109121in}}{\pgfqpoint{1.734709in}{3.113512in}}{\pgfqpoint{1.723659in}{3.113512in}}%
\pgfpathcurveto{\pgfqpoint{1.712609in}{3.113512in}}{\pgfqpoint{1.702010in}{3.109121in}}{\pgfqpoint{1.694196in}{3.101308in}}%
\pgfpathcurveto{\pgfqpoint{1.686382in}{3.093494in}}{\pgfqpoint{1.681992in}{3.082895in}}{\pgfqpoint{1.681992in}{3.071845in}}%
\pgfpathcurveto{\pgfqpoint{1.681992in}{3.060795in}}{\pgfqpoint{1.686382in}{3.050196in}}{\pgfqpoint{1.694196in}{3.042382in}}%
\pgfpathcurveto{\pgfqpoint{1.702010in}{3.034569in}}{\pgfqpoint{1.712609in}{3.030178in}}{\pgfqpoint{1.723659in}{3.030178in}}%
\pgfpathclose%
\pgfusepath{stroke,fill}%
\end{pgfscope}%
\begin{pgfscope}%
\pgfpathrectangle{\pgfqpoint{0.750000in}{0.500000in}}{\pgfqpoint{4.650000in}{3.020000in}}%
\pgfusepath{clip}%
\pgfsetbuttcap%
\pgfsetroundjoin%
\definecolor{currentfill}{rgb}{1.000000,0.498039,0.054902}%
\pgfsetfillcolor{currentfill}%
\pgfsetlinewidth{1.003750pt}%
\definecolor{currentstroke}{rgb}{1.000000,0.498039,0.054902}%
\pgfsetstrokecolor{currentstroke}%
\pgfsetdash{}{0pt}%
\pgfpathmoveto{\pgfqpoint{4.079844in}{3.022103in}}%
\pgfpathcurveto{\pgfqpoint{4.090894in}{3.022103in}}{\pgfqpoint{4.101493in}{3.026494in}}{\pgfqpoint{4.109306in}{3.034307in}}%
\pgfpathcurveto{\pgfqpoint{4.117120in}{3.042121in}}{\pgfqpoint{4.121510in}{3.052720in}}{\pgfqpoint{4.121510in}{3.063770in}}%
\pgfpathcurveto{\pgfqpoint{4.121510in}{3.074820in}}{\pgfqpoint{4.117120in}{3.085419in}}{\pgfqpoint{4.109306in}{3.093233in}}%
\pgfpathcurveto{\pgfqpoint{4.101493in}{3.101046in}}{\pgfqpoint{4.090894in}{3.105437in}}{\pgfqpoint{4.079844in}{3.105437in}}%
\pgfpathcurveto{\pgfqpoint{4.068793in}{3.105437in}}{\pgfqpoint{4.058194in}{3.101046in}}{\pgfqpoint{4.050381in}{3.093233in}}%
\pgfpathcurveto{\pgfqpoint{4.042567in}{3.085419in}}{\pgfqpoint{4.038177in}{3.074820in}}{\pgfqpoint{4.038177in}{3.063770in}}%
\pgfpathcurveto{\pgfqpoint{4.038177in}{3.052720in}}{\pgfqpoint{4.042567in}{3.042121in}}{\pgfqpoint{4.050381in}{3.034307in}}%
\pgfpathcurveto{\pgfqpoint{4.058194in}{3.026494in}}{\pgfqpoint{4.068793in}{3.022103in}}{\pgfqpoint{4.079844in}{3.022103in}}%
\pgfpathclose%
\pgfusepath{stroke,fill}%
\end{pgfscope}%
\begin{pgfscope}%
\pgfpathrectangle{\pgfqpoint{0.750000in}{0.500000in}}{\pgfqpoint{4.650000in}{3.020000in}}%
\pgfusepath{clip}%
\pgfsetbuttcap%
\pgfsetroundjoin%
\definecolor{currentfill}{rgb}{1.000000,0.498039,0.054902}%
\pgfsetfillcolor{currentfill}%
\pgfsetlinewidth{1.003750pt}%
\definecolor{currentstroke}{rgb}{1.000000,0.498039,0.054902}%
\pgfsetstrokecolor{currentstroke}%
\pgfsetdash{}{0pt}%
\pgfpathmoveto{\pgfqpoint{1.515760in}{3.018066in}}%
\pgfpathcurveto{\pgfqpoint{1.526810in}{3.018066in}}{\pgfqpoint{1.537409in}{3.022456in}}{\pgfqpoint{1.545223in}{3.030270in}}%
\pgfpathcurveto{\pgfqpoint{1.553036in}{3.038083in}}{\pgfqpoint{1.557427in}{3.048682in}}{\pgfqpoint{1.557427in}{3.059733in}}%
\pgfpathcurveto{\pgfqpoint{1.557427in}{3.070783in}}{\pgfqpoint{1.553036in}{3.081382in}}{\pgfqpoint{1.545223in}{3.089195in}}%
\pgfpathcurveto{\pgfqpoint{1.537409in}{3.097009in}}{\pgfqpoint{1.526810in}{3.101399in}}{\pgfqpoint{1.515760in}{3.101399in}}%
\pgfpathcurveto{\pgfqpoint{1.504710in}{3.101399in}}{\pgfqpoint{1.494111in}{3.097009in}}{\pgfqpoint{1.486297in}{3.089195in}}%
\pgfpathcurveto{\pgfqpoint{1.478484in}{3.081382in}}{\pgfqpoint{1.474093in}{3.070783in}}{\pgfqpoint{1.474093in}{3.059733in}}%
\pgfpathcurveto{\pgfqpoint{1.474093in}{3.048682in}}{\pgfqpoint{1.478484in}{3.038083in}}{\pgfqpoint{1.486297in}{3.030270in}}%
\pgfpathcurveto{\pgfqpoint{1.494111in}{3.022456in}}{\pgfqpoint{1.504710in}{3.018066in}}{\pgfqpoint{1.515760in}{3.018066in}}%
\pgfpathclose%
\pgfusepath{stroke,fill}%
\end{pgfscope}%
\begin{pgfscope}%
\pgfpathrectangle{\pgfqpoint{0.750000in}{0.500000in}}{\pgfqpoint{4.650000in}{3.020000in}}%
\pgfusepath{clip}%
\pgfsetbuttcap%
\pgfsetroundjoin%
\definecolor{currentfill}{rgb}{1.000000,0.498039,0.054902}%
\pgfsetfillcolor{currentfill}%
\pgfsetlinewidth{1.003750pt}%
\definecolor{currentstroke}{rgb}{1.000000,0.498039,0.054902}%
\pgfsetstrokecolor{currentstroke}%
\pgfsetdash{}{0pt}%
\pgfpathmoveto{\pgfqpoint{1.377161in}{3.119002in}}%
\pgfpathcurveto{\pgfqpoint{1.388211in}{3.119002in}}{\pgfqpoint{1.398810in}{3.123392in}}{\pgfqpoint{1.406624in}{3.131206in}}%
\pgfpathcurveto{\pgfqpoint{1.414437in}{3.139019in}}{\pgfqpoint{1.418828in}{3.149618in}}{\pgfqpoint{1.418828in}{3.160668in}}%
\pgfpathcurveto{\pgfqpoint{1.418828in}{3.171719in}}{\pgfqpoint{1.414437in}{3.182318in}}{\pgfqpoint{1.406624in}{3.190131in}}%
\pgfpathcurveto{\pgfqpoint{1.398810in}{3.197945in}}{\pgfqpoint{1.388211in}{3.202335in}}{\pgfqpoint{1.377161in}{3.202335in}}%
\pgfpathcurveto{\pgfqpoint{1.366111in}{3.202335in}}{\pgfqpoint{1.355512in}{3.197945in}}{\pgfqpoint{1.347698in}{3.190131in}}%
\pgfpathcurveto{\pgfqpoint{1.339885in}{3.182318in}}{\pgfqpoint{1.335494in}{3.171719in}}{\pgfqpoint{1.335494in}{3.160668in}}%
\pgfpathcurveto{\pgfqpoint{1.335494in}{3.149618in}}{\pgfqpoint{1.339885in}{3.139019in}}{\pgfqpoint{1.347698in}{3.131206in}}%
\pgfpathcurveto{\pgfqpoint{1.355512in}{3.123392in}}{\pgfqpoint{1.366111in}{3.119002in}}{\pgfqpoint{1.377161in}{3.119002in}}%
\pgfpathclose%
\pgfusepath{stroke,fill}%
\end{pgfscope}%
\begin{pgfscope}%
\pgfpathrectangle{\pgfqpoint{0.750000in}{0.500000in}}{\pgfqpoint{4.650000in}{3.020000in}}%
\pgfusepath{clip}%
\pgfsetbuttcap%
\pgfsetroundjoin%
\definecolor{currentfill}{rgb}{1.000000,0.498039,0.054902}%
\pgfsetfillcolor{currentfill}%
\pgfsetlinewidth{1.003750pt}%
\definecolor{currentstroke}{rgb}{1.000000,0.498039,0.054902}%
\pgfsetstrokecolor{currentstroke}%
\pgfsetdash{}{0pt}%
\pgfpathmoveto{\pgfqpoint{2.832452in}{3.098815in}}%
\pgfpathcurveto{\pgfqpoint{2.843502in}{3.098815in}}{\pgfqpoint{2.854101in}{3.103205in}}{\pgfqpoint{2.861914in}{3.111019in}}%
\pgfpathcurveto{\pgfqpoint{2.869728in}{3.118832in}}{\pgfqpoint{2.874118in}{3.129431in}}{\pgfqpoint{2.874118in}{3.140481in}}%
\pgfpathcurveto{\pgfqpoint{2.874118in}{3.151531in}}{\pgfqpoint{2.869728in}{3.162130in}}{\pgfqpoint{2.861914in}{3.169944in}}%
\pgfpathcurveto{\pgfqpoint{2.854101in}{3.177758in}}{\pgfqpoint{2.843502in}{3.182148in}}{\pgfqpoint{2.832452in}{3.182148in}}%
\pgfpathcurveto{\pgfqpoint{2.821401in}{3.182148in}}{\pgfqpoint{2.810802in}{3.177758in}}{\pgfqpoint{2.802989in}{3.169944in}}%
\pgfpathcurveto{\pgfqpoint{2.795175in}{3.162130in}}{\pgfqpoint{2.790785in}{3.151531in}}{\pgfqpoint{2.790785in}{3.140481in}}%
\pgfpathcurveto{\pgfqpoint{2.790785in}{3.129431in}}{\pgfqpoint{2.795175in}{3.118832in}}{\pgfqpoint{2.802989in}{3.111019in}}%
\pgfpathcurveto{\pgfqpoint{2.810802in}{3.103205in}}{\pgfqpoint{2.821401in}{3.098815in}}{\pgfqpoint{2.832452in}{3.098815in}}%
\pgfpathclose%
\pgfusepath{stroke,fill}%
\end{pgfscope}%
\begin{pgfscope}%
\pgfpathrectangle{\pgfqpoint{0.750000in}{0.500000in}}{\pgfqpoint{4.650000in}{3.020000in}}%
\pgfusepath{clip}%
\pgfsetbuttcap%
\pgfsetroundjoin%
\definecolor{currentfill}{rgb}{0.121569,0.466667,0.705882}%
\pgfsetfillcolor{currentfill}%
\pgfsetlinewidth{1.003750pt}%
\definecolor{currentstroke}{rgb}{0.121569,0.466667,0.705882}%
\pgfsetstrokecolor{currentstroke}%
\pgfsetdash{}{0pt}%
\pgfpathmoveto{\pgfqpoint{1.377161in}{0.611756in}}%
\pgfpathcurveto{\pgfqpoint{1.388211in}{0.611756in}}{\pgfqpoint{1.398810in}{0.616146in}}{\pgfqpoint{1.406624in}{0.623960in}}%
\pgfpathcurveto{\pgfqpoint{1.414437in}{0.631773in}}{\pgfqpoint{1.418828in}{0.642372in}}{\pgfqpoint{1.418828in}{0.653422in}}%
\pgfpathcurveto{\pgfqpoint{1.418828in}{0.664473in}}{\pgfqpoint{1.414437in}{0.675072in}}{\pgfqpoint{1.406624in}{0.682885in}}%
\pgfpathcurveto{\pgfqpoint{1.398810in}{0.690699in}}{\pgfqpoint{1.388211in}{0.695089in}}{\pgfqpoint{1.377161in}{0.695089in}}%
\pgfpathcurveto{\pgfqpoint{1.366111in}{0.695089in}}{\pgfqpoint{1.355512in}{0.690699in}}{\pgfqpoint{1.347698in}{0.682885in}}%
\pgfpathcurveto{\pgfqpoint{1.339885in}{0.675072in}}{\pgfqpoint{1.335494in}{0.664473in}}{\pgfqpoint{1.335494in}{0.653422in}}%
\pgfpathcurveto{\pgfqpoint{1.335494in}{0.642372in}}{\pgfqpoint{1.339885in}{0.631773in}}{\pgfqpoint{1.347698in}{0.623960in}}%
\pgfpathcurveto{\pgfqpoint{1.355512in}{0.616146in}}{\pgfqpoint{1.366111in}{0.611756in}}{\pgfqpoint{1.377161in}{0.611756in}}%
\pgfpathclose%
\pgfusepath{stroke,fill}%
\end{pgfscope}%
\begin{pgfscope}%
\pgfpathrectangle{\pgfqpoint{0.750000in}{0.500000in}}{\pgfqpoint{4.650000in}{3.020000in}}%
\pgfusepath{clip}%
\pgfsetbuttcap%
\pgfsetroundjoin%
\definecolor{currentfill}{rgb}{1.000000,0.498039,0.054902}%
\pgfsetfillcolor{currentfill}%
\pgfsetlinewidth{1.003750pt}%
\definecolor{currentstroke}{rgb}{1.000000,0.498039,0.054902}%
\pgfsetstrokecolor{currentstroke}%
\pgfsetdash{}{0pt}%
\pgfpathmoveto{\pgfqpoint{1.030663in}{3.038253in}}%
\pgfpathcurveto{\pgfqpoint{1.041713in}{3.038253in}}{\pgfqpoint{1.052312in}{3.042643in}}{\pgfqpoint{1.060126in}{3.050457in}}%
\pgfpathcurveto{\pgfqpoint{1.067940in}{3.058271in}}{\pgfqpoint{1.072330in}{3.068870in}}{\pgfqpoint{1.072330in}{3.079920in}}%
\pgfpathcurveto{\pgfqpoint{1.072330in}{3.090970in}}{\pgfqpoint{1.067940in}{3.101569in}}{\pgfqpoint{1.060126in}{3.109383in}}%
\pgfpathcurveto{\pgfqpoint{1.052312in}{3.117196in}}{\pgfqpoint{1.041713in}{3.121586in}}{\pgfqpoint{1.030663in}{3.121586in}}%
\pgfpathcurveto{\pgfqpoint{1.019613in}{3.121586in}}{\pgfqpoint{1.009014in}{3.117196in}}{\pgfqpoint{1.001200in}{3.109383in}}%
\pgfpathcurveto{\pgfqpoint{0.993387in}{3.101569in}}{\pgfqpoint{0.988997in}{3.090970in}}{\pgfqpoint{0.988997in}{3.079920in}}%
\pgfpathcurveto{\pgfqpoint{0.988997in}{3.068870in}}{\pgfqpoint{0.993387in}{3.058271in}}{\pgfqpoint{1.001200in}{3.050457in}}%
\pgfpathcurveto{\pgfqpoint{1.009014in}{3.042643in}}{\pgfqpoint{1.019613in}{3.038253in}}{\pgfqpoint{1.030663in}{3.038253in}}%
\pgfpathclose%
\pgfusepath{stroke,fill}%
\end{pgfscope}%
\begin{pgfscope}%
\pgfpathrectangle{\pgfqpoint{0.750000in}{0.500000in}}{\pgfqpoint{4.650000in}{3.020000in}}%
\pgfusepath{clip}%
\pgfsetbuttcap%
\pgfsetroundjoin%
\definecolor{currentfill}{rgb}{0.121569,0.466667,0.705882}%
\pgfsetfillcolor{currentfill}%
\pgfsetlinewidth{1.003750pt}%
\definecolor{currentstroke}{rgb}{0.121569,0.466667,0.705882}%
\pgfsetstrokecolor{currentstroke}%
\pgfsetdash{}{0pt}%
\pgfpathmoveto{\pgfqpoint{1.099963in}{3.001916in}}%
\pgfpathcurveto{\pgfqpoint{1.111013in}{3.001916in}}{\pgfqpoint{1.121612in}{3.006306in}}{\pgfqpoint{1.129426in}{3.014120in}}%
\pgfpathcurveto{\pgfqpoint{1.137239in}{3.021934in}}{\pgfqpoint{1.141629in}{3.032533in}}{\pgfqpoint{1.141629in}{3.043583in}}%
\pgfpathcurveto{\pgfqpoint{1.141629in}{3.054633in}}{\pgfqpoint{1.137239in}{3.065232in}}{\pgfqpoint{1.129426in}{3.073046in}}%
\pgfpathcurveto{\pgfqpoint{1.121612in}{3.080859in}}{\pgfqpoint{1.111013in}{3.085250in}}{\pgfqpoint{1.099963in}{3.085250in}}%
\pgfpathcurveto{\pgfqpoint{1.088913in}{3.085250in}}{\pgfqpoint{1.078314in}{3.080859in}}{\pgfqpoint{1.070500in}{3.073046in}}%
\pgfpathcurveto{\pgfqpoint{1.062686in}{3.065232in}}{\pgfqpoint{1.058296in}{3.054633in}}{\pgfqpoint{1.058296in}{3.043583in}}%
\pgfpathcurveto{\pgfqpoint{1.058296in}{3.032533in}}{\pgfqpoint{1.062686in}{3.021934in}}{\pgfqpoint{1.070500in}{3.014120in}}%
\pgfpathcurveto{\pgfqpoint{1.078314in}{3.006306in}}{\pgfqpoint{1.088913in}{3.001916in}}{\pgfqpoint{1.099963in}{3.001916in}}%
\pgfpathclose%
\pgfusepath{stroke,fill}%
\end{pgfscope}%
\begin{pgfscope}%
\pgfpathrectangle{\pgfqpoint{0.750000in}{0.500000in}}{\pgfqpoint{4.650000in}{3.020000in}}%
\pgfusepath{clip}%
\pgfsetbuttcap%
\pgfsetroundjoin%
\definecolor{currentfill}{rgb}{1.000000,0.498039,0.054902}%
\pgfsetfillcolor{currentfill}%
\pgfsetlinewidth{1.003750pt}%
\definecolor{currentstroke}{rgb}{1.000000,0.498039,0.054902}%
\pgfsetstrokecolor{currentstroke}%
\pgfsetdash{}{0pt}%
\pgfpathmoveto{\pgfqpoint{1.515760in}{3.022103in}}%
\pgfpathcurveto{\pgfqpoint{1.526810in}{3.022103in}}{\pgfqpoint{1.537409in}{3.026494in}}{\pgfqpoint{1.545223in}{3.034307in}}%
\pgfpathcurveto{\pgfqpoint{1.553036in}{3.042121in}}{\pgfqpoint{1.557427in}{3.052720in}}{\pgfqpoint{1.557427in}{3.063770in}}%
\pgfpathcurveto{\pgfqpoint{1.557427in}{3.074820in}}{\pgfqpoint{1.553036in}{3.085419in}}{\pgfqpoint{1.545223in}{3.093233in}}%
\pgfpathcurveto{\pgfqpoint{1.537409in}{3.101046in}}{\pgfqpoint{1.526810in}{3.105437in}}{\pgfqpoint{1.515760in}{3.105437in}}%
\pgfpathcurveto{\pgfqpoint{1.504710in}{3.105437in}}{\pgfqpoint{1.494111in}{3.101046in}}{\pgfqpoint{1.486297in}{3.093233in}}%
\pgfpathcurveto{\pgfqpoint{1.478484in}{3.085419in}}{\pgfqpoint{1.474093in}{3.074820in}}{\pgfqpoint{1.474093in}{3.063770in}}%
\pgfpathcurveto{\pgfqpoint{1.474093in}{3.052720in}}{\pgfqpoint{1.478484in}{3.042121in}}{\pgfqpoint{1.486297in}{3.034307in}}%
\pgfpathcurveto{\pgfqpoint{1.494111in}{3.026494in}}{\pgfqpoint{1.504710in}{3.022103in}}{\pgfqpoint{1.515760in}{3.022103in}}%
\pgfpathclose%
\pgfusepath{stroke,fill}%
\end{pgfscope}%
\begin{pgfscope}%
\pgfpathrectangle{\pgfqpoint{0.750000in}{0.500000in}}{\pgfqpoint{4.650000in}{3.020000in}}%
\pgfusepath{clip}%
\pgfsetbuttcap%
\pgfsetroundjoin%
\definecolor{currentfill}{rgb}{1.000000,0.498039,0.054902}%
\pgfsetfillcolor{currentfill}%
\pgfsetlinewidth{1.003750pt}%
\definecolor{currentstroke}{rgb}{1.000000,0.498039,0.054902}%
\pgfsetstrokecolor{currentstroke}%
\pgfsetdash{}{0pt}%
\pgfpathmoveto{\pgfqpoint{1.307861in}{3.018066in}}%
\pgfpathcurveto{\pgfqpoint{1.318912in}{3.018066in}}{\pgfqpoint{1.329511in}{3.022456in}}{\pgfqpoint{1.337324in}{3.030270in}}%
\pgfpathcurveto{\pgfqpoint{1.345138in}{3.038083in}}{\pgfqpoint{1.349528in}{3.048682in}}{\pgfqpoint{1.349528in}{3.059733in}}%
\pgfpathcurveto{\pgfqpoint{1.349528in}{3.070783in}}{\pgfqpoint{1.345138in}{3.081382in}}{\pgfqpoint{1.337324in}{3.089195in}}%
\pgfpathcurveto{\pgfqpoint{1.329511in}{3.097009in}}{\pgfqpoint{1.318912in}{3.101399in}}{\pgfqpoint{1.307861in}{3.101399in}}%
\pgfpathcurveto{\pgfqpoint{1.296811in}{3.101399in}}{\pgfqpoint{1.286212in}{3.097009in}}{\pgfqpoint{1.278399in}{3.089195in}}%
\pgfpathcurveto{\pgfqpoint{1.270585in}{3.081382in}}{\pgfqpoint{1.266195in}{3.070783in}}{\pgfqpoint{1.266195in}{3.059733in}}%
\pgfpathcurveto{\pgfqpoint{1.266195in}{3.048682in}}{\pgfqpoint{1.270585in}{3.038083in}}{\pgfqpoint{1.278399in}{3.030270in}}%
\pgfpathcurveto{\pgfqpoint{1.286212in}{3.022456in}}{\pgfqpoint{1.296811in}{3.018066in}}{\pgfqpoint{1.307861in}{3.018066in}}%
\pgfpathclose%
\pgfusepath{stroke,fill}%
\end{pgfscope}%
\begin{pgfscope}%
\pgfpathrectangle{\pgfqpoint{0.750000in}{0.500000in}}{\pgfqpoint{4.650000in}{3.020000in}}%
\pgfusepath{clip}%
\pgfsetbuttcap%
\pgfsetroundjoin%
\definecolor{currentfill}{rgb}{1.000000,0.498039,0.054902}%
\pgfsetfillcolor{currentfill}%
\pgfsetlinewidth{1.003750pt}%
\definecolor{currentstroke}{rgb}{1.000000,0.498039,0.054902}%
\pgfsetstrokecolor{currentstroke}%
\pgfsetdash{}{0pt}%
\pgfpathmoveto{\pgfqpoint{2.624553in}{3.022103in}}%
\pgfpathcurveto{\pgfqpoint{2.635603in}{3.022103in}}{\pgfqpoint{2.646202in}{3.026494in}}{\pgfqpoint{2.654016in}{3.034307in}}%
\pgfpathcurveto{\pgfqpoint{2.661829in}{3.042121in}}{\pgfqpoint{2.666220in}{3.052720in}}{\pgfqpoint{2.666220in}{3.063770in}}%
\pgfpathcurveto{\pgfqpoint{2.666220in}{3.074820in}}{\pgfqpoint{2.661829in}{3.085419in}}{\pgfqpoint{2.654016in}{3.093233in}}%
\pgfpathcurveto{\pgfqpoint{2.646202in}{3.101046in}}{\pgfqpoint{2.635603in}{3.105437in}}{\pgfqpoint{2.624553in}{3.105437in}}%
\pgfpathcurveto{\pgfqpoint{2.613503in}{3.105437in}}{\pgfqpoint{2.602904in}{3.101046in}}{\pgfqpoint{2.595090in}{3.093233in}}%
\pgfpathcurveto{\pgfqpoint{2.587277in}{3.085419in}}{\pgfqpoint{2.582886in}{3.074820in}}{\pgfqpoint{2.582886in}{3.063770in}}%
\pgfpathcurveto{\pgfqpoint{2.582886in}{3.052720in}}{\pgfqpoint{2.587277in}{3.042121in}}{\pgfqpoint{2.595090in}{3.034307in}}%
\pgfpathcurveto{\pgfqpoint{2.602904in}{3.026494in}}{\pgfqpoint{2.613503in}{3.022103in}}{\pgfqpoint{2.624553in}{3.022103in}}%
\pgfpathclose%
\pgfusepath{stroke,fill}%
\end{pgfscope}%
\begin{pgfscope}%
\pgfpathrectangle{\pgfqpoint{0.750000in}{0.500000in}}{\pgfqpoint{4.650000in}{3.020000in}}%
\pgfusepath{clip}%
\pgfsetbuttcap%
\pgfsetroundjoin%
\definecolor{currentfill}{rgb}{1.000000,0.498039,0.054902}%
\pgfsetfillcolor{currentfill}%
\pgfsetlinewidth{1.003750pt}%
\definecolor{currentstroke}{rgb}{1.000000,0.498039,0.054902}%
\pgfsetstrokecolor{currentstroke}%
\pgfsetdash{}{0pt}%
\pgfpathmoveto{\pgfqpoint{1.515760in}{3.018066in}}%
\pgfpathcurveto{\pgfqpoint{1.526810in}{3.018066in}}{\pgfqpoint{1.537409in}{3.022456in}}{\pgfqpoint{1.545223in}{3.030270in}}%
\pgfpathcurveto{\pgfqpoint{1.553036in}{3.038083in}}{\pgfqpoint{1.557427in}{3.048682in}}{\pgfqpoint{1.557427in}{3.059733in}}%
\pgfpathcurveto{\pgfqpoint{1.557427in}{3.070783in}}{\pgfqpoint{1.553036in}{3.081382in}}{\pgfqpoint{1.545223in}{3.089195in}}%
\pgfpathcurveto{\pgfqpoint{1.537409in}{3.097009in}}{\pgfqpoint{1.526810in}{3.101399in}}{\pgfqpoint{1.515760in}{3.101399in}}%
\pgfpathcurveto{\pgfqpoint{1.504710in}{3.101399in}}{\pgfqpoint{1.494111in}{3.097009in}}{\pgfqpoint{1.486297in}{3.089195in}}%
\pgfpathcurveto{\pgfqpoint{1.478484in}{3.081382in}}{\pgfqpoint{1.474093in}{3.070783in}}{\pgfqpoint{1.474093in}{3.059733in}}%
\pgfpathcurveto{\pgfqpoint{1.474093in}{3.048682in}}{\pgfqpoint{1.478484in}{3.038083in}}{\pgfqpoint{1.486297in}{3.030270in}}%
\pgfpathcurveto{\pgfqpoint{1.494111in}{3.022456in}}{\pgfqpoint{1.504710in}{3.018066in}}{\pgfqpoint{1.515760in}{3.018066in}}%
\pgfpathclose%
\pgfusepath{stroke,fill}%
\end{pgfscope}%
\begin{pgfscope}%
\pgfpathrectangle{\pgfqpoint{0.750000in}{0.500000in}}{\pgfqpoint{4.650000in}{3.020000in}}%
\pgfusepath{clip}%
\pgfsetbuttcap%
\pgfsetroundjoin%
\definecolor{currentfill}{rgb}{1.000000,0.498039,0.054902}%
\pgfsetfillcolor{currentfill}%
\pgfsetlinewidth{1.003750pt}%
\definecolor{currentstroke}{rgb}{1.000000,0.498039,0.054902}%
\pgfsetstrokecolor{currentstroke}%
\pgfsetdash{}{0pt}%
\pgfpathmoveto{\pgfqpoint{2.555253in}{3.018066in}}%
\pgfpathcurveto{\pgfqpoint{2.566303in}{3.018066in}}{\pgfqpoint{2.576903in}{3.022456in}}{\pgfqpoint{2.584716in}{3.030270in}}%
\pgfpathcurveto{\pgfqpoint{2.592530in}{3.038083in}}{\pgfqpoint{2.596920in}{3.048682in}}{\pgfqpoint{2.596920in}{3.059733in}}%
\pgfpathcurveto{\pgfqpoint{2.596920in}{3.070783in}}{\pgfqpoint{2.592530in}{3.081382in}}{\pgfqpoint{2.584716in}{3.089195in}}%
\pgfpathcurveto{\pgfqpoint{2.576903in}{3.097009in}}{\pgfqpoint{2.566303in}{3.101399in}}{\pgfqpoint{2.555253in}{3.101399in}}%
\pgfpathcurveto{\pgfqpoint{2.544203in}{3.101399in}}{\pgfqpoint{2.533604in}{3.097009in}}{\pgfqpoint{2.525791in}{3.089195in}}%
\pgfpathcurveto{\pgfqpoint{2.517977in}{3.081382in}}{\pgfqpoint{2.513587in}{3.070783in}}{\pgfqpoint{2.513587in}{3.059733in}}%
\pgfpathcurveto{\pgfqpoint{2.513587in}{3.048682in}}{\pgfqpoint{2.517977in}{3.038083in}}{\pgfqpoint{2.525791in}{3.030270in}}%
\pgfpathcurveto{\pgfqpoint{2.533604in}{3.022456in}}{\pgfqpoint{2.544203in}{3.018066in}}{\pgfqpoint{2.555253in}{3.018066in}}%
\pgfpathclose%
\pgfusepath{stroke,fill}%
\end{pgfscope}%
\begin{pgfscope}%
\pgfpathrectangle{\pgfqpoint{0.750000in}{0.500000in}}{\pgfqpoint{4.650000in}{3.020000in}}%
\pgfusepath{clip}%
\pgfsetbuttcap%
\pgfsetroundjoin%
\definecolor{currentfill}{rgb}{0.121569,0.466667,0.705882}%
\pgfsetfillcolor{currentfill}%
\pgfsetlinewidth{1.003750pt}%
\definecolor{currentstroke}{rgb}{0.121569,0.466667,0.705882}%
\pgfsetstrokecolor{currentstroke}%
\pgfsetdash{}{0pt}%
\pgfpathmoveto{\pgfqpoint{1.792958in}{0.595606in}}%
\pgfpathcurveto{\pgfqpoint{1.804008in}{0.595606in}}{\pgfqpoint{1.814607in}{0.599996in}}{\pgfqpoint{1.822421in}{0.607810in}}%
\pgfpathcurveto{\pgfqpoint{1.830235in}{0.615624in}}{\pgfqpoint{1.834625in}{0.626223in}}{\pgfqpoint{1.834625in}{0.637273in}}%
\pgfpathcurveto{\pgfqpoint{1.834625in}{0.648323in}}{\pgfqpoint{1.830235in}{0.658922in}}{\pgfqpoint{1.822421in}{0.666736in}}%
\pgfpathcurveto{\pgfqpoint{1.814607in}{0.674549in}}{\pgfqpoint{1.804008in}{0.678939in}}{\pgfqpoint{1.792958in}{0.678939in}}%
\pgfpathcurveto{\pgfqpoint{1.781908in}{0.678939in}}{\pgfqpoint{1.771309in}{0.674549in}}{\pgfqpoint{1.763495in}{0.666736in}}%
\pgfpathcurveto{\pgfqpoint{1.755682in}{0.658922in}}{\pgfqpoint{1.751292in}{0.648323in}}{\pgfqpoint{1.751292in}{0.637273in}}%
\pgfpathcurveto{\pgfqpoint{1.751292in}{0.626223in}}{\pgfqpoint{1.755682in}{0.615624in}}{\pgfqpoint{1.763495in}{0.607810in}}%
\pgfpathcurveto{\pgfqpoint{1.771309in}{0.599996in}}{\pgfqpoint{1.781908in}{0.595606in}}{\pgfqpoint{1.792958in}{0.595606in}}%
\pgfpathclose%
\pgfusepath{stroke,fill}%
\end{pgfscope}%
\begin{pgfscope}%
\pgfpathrectangle{\pgfqpoint{0.750000in}{0.500000in}}{\pgfqpoint{4.650000in}{3.020000in}}%
\pgfusepath{clip}%
\pgfsetbuttcap%
\pgfsetroundjoin%
\definecolor{currentfill}{rgb}{1.000000,0.498039,0.054902}%
\pgfsetfillcolor{currentfill}%
\pgfsetlinewidth{1.003750pt}%
\definecolor{currentstroke}{rgb}{1.000000,0.498039,0.054902}%
\pgfsetstrokecolor{currentstroke}%
\pgfsetdash{}{0pt}%
\pgfpathmoveto{\pgfqpoint{1.585060in}{3.018066in}}%
\pgfpathcurveto{\pgfqpoint{1.596110in}{3.018066in}}{\pgfqpoint{1.606709in}{3.022456in}}{\pgfqpoint{1.614522in}{3.030270in}}%
\pgfpathcurveto{\pgfqpoint{1.622336in}{3.038083in}}{\pgfqpoint{1.626726in}{3.048682in}}{\pgfqpoint{1.626726in}{3.059733in}}%
\pgfpathcurveto{\pgfqpoint{1.626726in}{3.070783in}}{\pgfqpoint{1.622336in}{3.081382in}}{\pgfqpoint{1.614522in}{3.089195in}}%
\pgfpathcurveto{\pgfqpoint{1.606709in}{3.097009in}}{\pgfqpoint{1.596110in}{3.101399in}}{\pgfqpoint{1.585060in}{3.101399in}}%
\pgfpathcurveto{\pgfqpoint{1.574009in}{3.101399in}}{\pgfqpoint{1.563410in}{3.097009in}}{\pgfqpoint{1.555597in}{3.089195in}}%
\pgfpathcurveto{\pgfqpoint{1.547783in}{3.081382in}}{\pgfqpoint{1.543393in}{3.070783in}}{\pgfqpoint{1.543393in}{3.059733in}}%
\pgfpathcurveto{\pgfqpoint{1.543393in}{3.048682in}}{\pgfqpoint{1.547783in}{3.038083in}}{\pgfqpoint{1.555597in}{3.030270in}}%
\pgfpathcurveto{\pgfqpoint{1.563410in}{3.022456in}}{\pgfqpoint{1.574009in}{3.018066in}}{\pgfqpoint{1.585060in}{3.018066in}}%
\pgfpathclose%
\pgfusepath{stroke,fill}%
\end{pgfscope}%
\begin{pgfscope}%
\pgfpathrectangle{\pgfqpoint{0.750000in}{0.500000in}}{\pgfqpoint{4.650000in}{3.020000in}}%
\pgfusepath{clip}%
\pgfsetbuttcap%
\pgfsetroundjoin%
\definecolor{currentfill}{rgb}{1.000000,0.498039,0.054902}%
\pgfsetfillcolor{currentfill}%
\pgfsetlinewidth{1.003750pt}%
\definecolor{currentstroke}{rgb}{1.000000,0.498039,0.054902}%
\pgfsetstrokecolor{currentstroke}%
\pgfsetdash{}{0pt}%
\pgfpathmoveto{\pgfqpoint{1.585060in}{3.018066in}}%
\pgfpathcurveto{\pgfqpoint{1.596110in}{3.018066in}}{\pgfqpoint{1.606709in}{3.022456in}}{\pgfqpoint{1.614522in}{3.030270in}}%
\pgfpathcurveto{\pgfqpoint{1.622336in}{3.038083in}}{\pgfqpoint{1.626726in}{3.048682in}}{\pgfqpoint{1.626726in}{3.059733in}}%
\pgfpathcurveto{\pgfqpoint{1.626726in}{3.070783in}}{\pgfqpoint{1.622336in}{3.081382in}}{\pgfqpoint{1.614522in}{3.089195in}}%
\pgfpathcurveto{\pgfqpoint{1.606709in}{3.097009in}}{\pgfqpoint{1.596110in}{3.101399in}}{\pgfqpoint{1.585060in}{3.101399in}}%
\pgfpathcurveto{\pgfqpoint{1.574009in}{3.101399in}}{\pgfqpoint{1.563410in}{3.097009in}}{\pgfqpoint{1.555597in}{3.089195in}}%
\pgfpathcurveto{\pgfqpoint{1.547783in}{3.081382in}}{\pgfqpoint{1.543393in}{3.070783in}}{\pgfqpoint{1.543393in}{3.059733in}}%
\pgfpathcurveto{\pgfqpoint{1.543393in}{3.048682in}}{\pgfqpoint{1.547783in}{3.038083in}}{\pgfqpoint{1.555597in}{3.030270in}}%
\pgfpathcurveto{\pgfqpoint{1.563410in}{3.022456in}}{\pgfqpoint{1.574009in}{3.018066in}}{\pgfqpoint{1.585060in}{3.018066in}}%
\pgfpathclose%
\pgfusepath{stroke,fill}%
\end{pgfscope}%
\begin{pgfscope}%
\pgfpathrectangle{\pgfqpoint{0.750000in}{0.500000in}}{\pgfqpoint{4.650000in}{3.020000in}}%
\pgfusepath{clip}%
\pgfsetbuttcap%
\pgfsetroundjoin%
\definecolor{currentfill}{rgb}{0.121569,0.466667,0.705882}%
\pgfsetfillcolor{currentfill}%
\pgfsetlinewidth{1.003750pt}%
\definecolor{currentstroke}{rgb}{0.121569,0.466667,0.705882}%
\pgfsetstrokecolor{currentstroke}%
\pgfsetdash{}{0pt}%
\pgfpathmoveto{\pgfqpoint{1.654359in}{2.735446in}}%
\pgfpathcurveto{\pgfqpoint{1.665409in}{2.735446in}}{\pgfqpoint{1.676008in}{2.739836in}}{\pgfqpoint{1.683822in}{2.747650in}}%
\pgfpathcurveto{\pgfqpoint{1.691636in}{2.755463in}}{\pgfqpoint{1.696026in}{2.766062in}}{\pgfqpoint{1.696026in}{2.777112in}}%
\pgfpathcurveto{\pgfqpoint{1.696026in}{2.788162in}}{\pgfqpoint{1.691636in}{2.798761in}}{\pgfqpoint{1.683822in}{2.806575in}}%
\pgfpathcurveto{\pgfqpoint{1.676008in}{2.814389in}}{\pgfqpoint{1.665409in}{2.818779in}}{\pgfqpoint{1.654359in}{2.818779in}}%
\pgfpathcurveto{\pgfqpoint{1.643309in}{2.818779in}}{\pgfqpoint{1.632710in}{2.814389in}}{\pgfqpoint{1.624896in}{2.806575in}}%
\pgfpathcurveto{\pgfqpoint{1.617083in}{2.798761in}}{\pgfqpoint{1.612692in}{2.788162in}}{\pgfqpoint{1.612692in}{2.777112in}}%
\pgfpathcurveto{\pgfqpoint{1.612692in}{2.766062in}}{\pgfqpoint{1.617083in}{2.755463in}}{\pgfqpoint{1.624896in}{2.747650in}}%
\pgfpathcurveto{\pgfqpoint{1.632710in}{2.739836in}}{\pgfqpoint{1.643309in}{2.735446in}}{\pgfqpoint{1.654359in}{2.735446in}}%
\pgfpathclose%
\pgfusepath{stroke,fill}%
\end{pgfscope}%
\begin{pgfscope}%
\pgfpathrectangle{\pgfqpoint{0.750000in}{0.500000in}}{\pgfqpoint{4.650000in}{3.020000in}}%
\pgfusepath{clip}%
\pgfsetbuttcap%
\pgfsetroundjoin%
\definecolor{currentfill}{rgb}{1.000000,0.498039,0.054902}%
\pgfsetfillcolor{currentfill}%
\pgfsetlinewidth{1.003750pt}%
\definecolor{currentstroke}{rgb}{1.000000,0.498039,0.054902}%
\pgfsetstrokecolor{currentstroke}%
\pgfsetdash{}{0pt}%
\pgfpathmoveto{\pgfqpoint{2.139456in}{3.026141in}}%
\pgfpathcurveto{\pgfqpoint{2.150506in}{3.026141in}}{\pgfqpoint{2.161105in}{3.030531in}}{\pgfqpoint{2.168919in}{3.038345in}}%
\pgfpathcurveto{\pgfqpoint{2.176732in}{3.046158in}}{\pgfqpoint{2.181123in}{3.056757in}}{\pgfqpoint{2.181123in}{3.067807in}}%
\pgfpathcurveto{\pgfqpoint{2.181123in}{3.078858in}}{\pgfqpoint{2.176732in}{3.089457in}}{\pgfqpoint{2.168919in}{3.097270in}}%
\pgfpathcurveto{\pgfqpoint{2.161105in}{3.105084in}}{\pgfqpoint{2.150506in}{3.109474in}}{\pgfqpoint{2.139456in}{3.109474in}}%
\pgfpathcurveto{\pgfqpoint{2.128406in}{3.109474in}}{\pgfqpoint{2.117807in}{3.105084in}}{\pgfqpoint{2.109993in}{3.097270in}}%
\pgfpathcurveto{\pgfqpoint{2.102180in}{3.089457in}}{\pgfqpoint{2.097789in}{3.078858in}}{\pgfqpoint{2.097789in}{3.067807in}}%
\pgfpathcurveto{\pgfqpoint{2.097789in}{3.056757in}}{\pgfqpoint{2.102180in}{3.046158in}}{\pgfqpoint{2.109993in}{3.038345in}}%
\pgfpathcurveto{\pgfqpoint{2.117807in}{3.030531in}}{\pgfqpoint{2.128406in}{3.026141in}}{\pgfqpoint{2.139456in}{3.026141in}}%
\pgfpathclose%
\pgfusepath{stroke,fill}%
\end{pgfscope}%
\begin{pgfscope}%
\pgfpathrectangle{\pgfqpoint{0.750000in}{0.500000in}}{\pgfqpoint{4.650000in}{3.020000in}}%
\pgfusepath{clip}%
\pgfsetbuttcap%
\pgfsetroundjoin%
\definecolor{currentfill}{rgb}{1.000000,0.498039,0.054902}%
\pgfsetfillcolor{currentfill}%
\pgfsetlinewidth{1.003750pt}%
\definecolor{currentstroke}{rgb}{1.000000,0.498039,0.054902}%
\pgfsetstrokecolor{currentstroke}%
\pgfsetdash{}{0pt}%
\pgfpathmoveto{\pgfqpoint{2.555253in}{3.191676in}}%
\pgfpathcurveto{\pgfqpoint{2.566303in}{3.191676in}}{\pgfqpoint{2.576903in}{3.196066in}}{\pgfqpoint{2.584716in}{3.203879in}}%
\pgfpathcurveto{\pgfqpoint{2.592530in}{3.211693in}}{\pgfqpoint{2.596920in}{3.222292in}}{\pgfqpoint{2.596920in}{3.233342in}}%
\pgfpathcurveto{\pgfqpoint{2.596920in}{3.244392in}}{\pgfqpoint{2.592530in}{3.254991in}}{\pgfqpoint{2.584716in}{3.262805in}}%
\pgfpathcurveto{\pgfqpoint{2.576903in}{3.270619in}}{\pgfqpoint{2.566303in}{3.275009in}}{\pgfqpoint{2.555253in}{3.275009in}}%
\pgfpathcurveto{\pgfqpoint{2.544203in}{3.275009in}}{\pgfqpoint{2.533604in}{3.270619in}}{\pgfqpoint{2.525791in}{3.262805in}}%
\pgfpathcurveto{\pgfqpoint{2.517977in}{3.254991in}}{\pgfqpoint{2.513587in}{3.244392in}}{\pgfqpoint{2.513587in}{3.233342in}}%
\pgfpathcurveto{\pgfqpoint{2.513587in}{3.222292in}}{\pgfqpoint{2.517977in}{3.211693in}}{\pgfqpoint{2.525791in}{3.203879in}}%
\pgfpathcurveto{\pgfqpoint{2.533604in}{3.196066in}}{\pgfqpoint{2.544203in}{3.191676in}}{\pgfqpoint{2.555253in}{3.191676in}}%
\pgfpathclose%
\pgfusepath{stroke,fill}%
\end{pgfscope}%
\begin{pgfscope}%
\pgfpathrectangle{\pgfqpoint{0.750000in}{0.500000in}}{\pgfqpoint{4.650000in}{3.020000in}}%
\pgfusepath{clip}%
\pgfsetbuttcap%
\pgfsetroundjoin%
\definecolor{currentfill}{rgb}{0.121569,0.466667,0.705882}%
\pgfsetfillcolor{currentfill}%
\pgfsetlinewidth{1.003750pt}%
\definecolor{currentstroke}{rgb}{0.121569,0.466667,0.705882}%
\pgfsetstrokecolor{currentstroke}%
\pgfsetdash{}{0pt}%
\pgfpathmoveto{\pgfqpoint{1.307861in}{0.595606in}}%
\pgfpathcurveto{\pgfqpoint{1.318912in}{0.595606in}}{\pgfqpoint{1.329511in}{0.599996in}}{\pgfqpoint{1.337324in}{0.607810in}}%
\pgfpathcurveto{\pgfqpoint{1.345138in}{0.615624in}}{\pgfqpoint{1.349528in}{0.626223in}}{\pgfqpoint{1.349528in}{0.637273in}}%
\pgfpathcurveto{\pgfqpoint{1.349528in}{0.648323in}}{\pgfqpoint{1.345138in}{0.658922in}}{\pgfqpoint{1.337324in}{0.666736in}}%
\pgfpathcurveto{\pgfqpoint{1.329511in}{0.674549in}}{\pgfqpoint{1.318912in}{0.678939in}}{\pgfqpoint{1.307861in}{0.678939in}}%
\pgfpathcurveto{\pgfqpoint{1.296811in}{0.678939in}}{\pgfqpoint{1.286212in}{0.674549in}}{\pgfqpoint{1.278399in}{0.666736in}}%
\pgfpathcurveto{\pgfqpoint{1.270585in}{0.658922in}}{\pgfqpoint{1.266195in}{0.648323in}}{\pgfqpoint{1.266195in}{0.637273in}}%
\pgfpathcurveto{\pgfqpoint{1.266195in}{0.626223in}}{\pgfqpoint{1.270585in}{0.615624in}}{\pgfqpoint{1.278399in}{0.607810in}}%
\pgfpathcurveto{\pgfqpoint{1.286212in}{0.599996in}}{\pgfqpoint{1.296811in}{0.595606in}}{\pgfqpoint{1.307861in}{0.595606in}}%
\pgfpathclose%
\pgfusepath{stroke,fill}%
\end{pgfscope}%
\begin{pgfscope}%
\pgfpathrectangle{\pgfqpoint{0.750000in}{0.500000in}}{\pgfqpoint{4.650000in}{3.020000in}}%
\pgfusepath{clip}%
\pgfsetbuttcap%
\pgfsetroundjoin%
\definecolor{currentfill}{rgb}{1.000000,0.498039,0.054902}%
\pgfsetfillcolor{currentfill}%
\pgfsetlinewidth{1.003750pt}%
\definecolor{currentstroke}{rgb}{1.000000,0.498039,0.054902}%
\pgfsetstrokecolor{currentstroke}%
\pgfsetdash{}{0pt}%
\pgfpathmoveto{\pgfqpoint{0.961364in}{3.022103in}}%
\pgfpathcurveto{\pgfqpoint{0.972414in}{3.022103in}}{\pgfqpoint{0.983013in}{3.026494in}}{\pgfqpoint{0.990826in}{3.034307in}}%
\pgfpathcurveto{\pgfqpoint{0.998640in}{3.042121in}}{\pgfqpoint{1.003030in}{3.052720in}}{\pgfqpoint{1.003030in}{3.063770in}}%
\pgfpathcurveto{\pgfqpoint{1.003030in}{3.074820in}}{\pgfqpoint{0.998640in}{3.085419in}}{\pgfqpoint{0.990826in}{3.093233in}}%
\pgfpathcurveto{\pgfqpoint{0.983013in}{3.101046in}}{\pgfqpoint{0.972414in}{3.105437in}}{\pgfqpoint{0.961364in}{3.105437in}}%
\pgfpathcurveto{\pgfqpoint{0.950314in}{3.105437in}}{\pgfqpoint{0.939714in}{3.101046in}}{\pgfqpoint{0.931901in}{3.093233in}}%
\pgfpathcurveto{\pgfqpoint{0.924087in}{3.085419in}}{\pgfqpoint{0.919697in}{3.074820in}}{\pgfqpoint{0.919697in}{3.063770in}}%
\pgfpathcurveto{\pgfqpoint{0.919697in}{3.052720in}}{\pgfqpoint{0.924087in}{3.042121in}}{\pgfqpoint{0.931901in}{3.034307in}}%
\pgfpathcurveto{\pgfqpoint{0.939714in}{3.026494in}}{\pgfqpoint{0.950314in}{3.022103in}}{\pgfqpoint{0.961364in}{3.022103in}}%
\pgfpathclose%
\pgfusepath{stroke,fill}%
\end{pgfscope}%
\begin{pgfscope}%
\pgfpathrectangle{\pgfqpoint{0.750000in}{0.500000in}}{\pgfqpoint{4.650000in}{3.020000in}}%
\pgfusepath{clip}%
\pgfsetbuttcap%
\pgfsetroundjoin%
\definecolor{currentfill}{rgb}{0.121569,0.466667,0.705882}%
\pgfsetfillcolor{currentfill}%
\pgfsetlinewidth{1.003750pt}%
\definecolor{currentstroke}{rgb}{0.121569,0.466667,0.705882}%
\pgfsetstrokecolor{currentstroke}%
\pgfsetdash{}{0pt}%
\pgfpathmoveto{\pgfqpoint{0.961364in}{1.100285in}}%
\pgfpathcurveto{\pgfqpoint{0.972414in}{1.100285in}}{\pgfqpoint{0.983013in}{1.104675in}}{\pgfqpoint{0.990826in}{1.112489in}}%
\pgfpathcurveto{\pgfqpoint{0.998640in}{1.120303in}}{\pgfqpoint{1.003030in}{1.130902in}}{\pgfqpoint{1.003030in}{1.141952in}}%
\pgfpathcurveto{\pgfqpoint{1.003030in}{1.153002in}}{\pgfqpoint{0.998640in}{1.163601in}}{\pgfqpoint{0.990826in}{1.171415in}}%
\pgfpathcurveto{\pgfqpoint{0.983013in}{1.179228in}}{\pgfqpoint{0.972414in}{1.183619in}}{\pgfqpoint{0.961364in}{1.183619in}}%
\pgfpathcurveto{\pgfqpoint{0.950314in}{1.183619in}}{\pgfqpoint{0.939714in}{1.179228in}}{\pgfqpoint{0.931901in}{1.171415in}}%
\pgfpathcurveto{\pgfqpoint{0.924087in}{1.163601in}}{\pgfqpoint{0.919697in}{1.153002in}}{\pgfqpoint{0.919697in}{1.141952in}}%
\pgfpathcurveto{\pgfqpoint{0.919697in}{1.130902in}}{\pgfqpoint{0.924087in}{1.120303in}}{\pgfqpoint{0.931901in}{1.112489in}}%
\pgfpathcurveto{\pgfqpoint{0.939714in}{1.104675in}}{\pgfqpoint{0.950314in}{1.100285in}}{\pgfqpoint{0.961364in}{1.100285in}}%
\pgfpathclose%
\pgfusepath{stroke,fill}%
\end{pgfscope}%
\begin{pgfscope}%
\pgfpathrectangle{\pgfqpoint{0.750000in}{0.500000in}}{\pgfqpoint{4.650000in}{3.020000in}}%
\pgfusepath{clip}%
\pgfsetbuttcap%
\pgfsetroundjoin%
\definecolor{currentfill}{rgb}{1.000000,0.498039,0.054902}%
\pgfsetfillcolor{currentfill}%
\pgfsetlinewidth{1.003750pt}%
\definecolor{currentstroke}{rgb}{1.000000,0.498039,0.054902}%
\pgfsetstrokecolor{currentstroke}%
\pgfsetdash{}{0pt}%
\pgfpathmoveto{\pgfqpoint{2.070156in}{3.018066in}}%
\pgfpathcurveto{\pgfqpoint{2.081207in}{3.018066in}}{\pgfqpoint{2.091806in}{3.022456in}}{\pgfqpoint{2.099619in}{3.030270in}}%
\pgfpathcurveto{\pgfqpoint{2.107433in}{3.038083in}}{\pgfqpoint{2.111823in}{3.048682in}}{\pgfqpoint{2.111823in}{3.059733in}}%
\pgfpathcurveto{\pgfqpoint{2.111823in}{3.070783in}}{\pgfqpoint{2.107433in}{3.081382in}}{\pgfqpoint{2.099619in}{3.089195in}}%
\pgfpathcurveto{\pgfqpoint{2.091806in}{3.097009in}}{\pgfqpoint{2.081207in}{3.101399in}}{\pgfqpoint{2.070156in}{3.101399in}}%
\pgfpathcurveto{\pgfqpoint{2.059106in}{3.101399in}}{\pgfqpoint{2.048507in}{3.097009in}}{\pgfqpoint{2.040694in}{3.089195in}}%
\pgfpathcurveto{\pgfqpoint{2.032880in}{3.081382in}}{\pgfqpoint{2.028490in}{3.070783in}}{\pgfqpoint{2.028490in}{3.059733in}}%
\pgfpathcurveto{\pgfqpoint{2.028490in}{3.048682in}}{\pgfqpoint{2.032880in}{3.038083in}}{\pgfqpoint{2.040694in}{3.030270in}}%
\pgfpathcurveto{\pgfqpoint{2.048507in}{3.022456in}}{\pgfqpoint{2.059106in}{3.018066in}}{\pgfqpoint{2.070156in}{3.018066in}}%
\pgfpathclose%
\pgfusepath{stroke,fill}%
\end{pgfscope}%
\begin{pgfscope}%
\pgfpathrectangle{\pgfqpoint{0.750000in}{0.500000in}}{\pgfqpoint{4.650000in}{3.020000in}}%
\pgfusepath{clip}%
\pgfsetbuttcap%
\pgfsetroundjoin%
\definecolor{currentfill}{rgb}{0.121569,0.466667,0.705882}%
\pgfsetfillcolor{currentfill}%
\pgfsetlinewidth{1.003750pt}%
\definecolor{currentstroke}{rgb}{0.121569,0.466667,0.705882}%
\pgfsetstrokecolor{currentstroke}%
\pgfsetdash{}{0pt}%
\pgfpathmoveto{\pgfqpoint{2.070156in}{3.014029in}}%
\pgfpathcurveto{\pgfqpoint{2.081207in}{3.014029in}}{\pgfqpoint{2.091806in}{3.018419in}}{\pgfqpoint{2.099619in}{3.026232in}}%
\pgfpathcurveto{\pgfqpoint{2.107433in}{3.034046in}}{\pgfqpoint{2.111823in}{3.044645in}}{\pgfqpoint{2.111823in}{3.055695in}}%
\pgfpathcurveto{\pgfqpoint{2.111823in}{3.066745in}}{\pgfqpoint{2.107433in}{3.077344in}}{\pgfqpoint{2.099619in}{3.085158in}}%
\pgfpathcurveto{\pgfqpoint{2.091806in}{3.092972in}}{\pgfqpoint{2.081207in}{3.097362in}}{\pgfqpoint{2.070156in}{3.097362in}}%
\pgfpathcurveto{\pgfqpoint{2.059106in}{3.097362in}}{\pgfqpoint{2.048507in}{3.092972in}}{\pgfqpoint{2.040694in}{3.085158in}}%
\pgfpathcurveto{\pgfqpoint{2.032880in}{3.077344in}}{\pgfqpoint{2.028490in}{3.066745in}}{\pgfqpoint{2.028490in}{3.055695in}}%
\pgfpathcurveto{\pgfqpoint{2.028490in}{3.044645in}}{\pgfqpoint{2.032880in}{3.034046in}}{\pgfqpoint{2.040694in}{3.026232in}}%
\pgfpathcurveto{\pgfqpoint{2.048507in}{3.018419in}}{\pgfqpoint{2.059106in}{3.014029in}}{\pgfqpoint{2.070156in}{3.014029in}}%
\pgfpathclose%
\pgfusepath{stroke,fill}%
\end{pgfscope}%
\begin{pgfscope}%
\pgfpathrectangle{\pgfqpoint{0.750000in}{0.500000in}}{\pgfqpoint{4.650000in}{3.020000in}}%
\pgfusepath{clip}%
\pgfsetbuttcap%
\pgfsetroundjoin%
\definecolor{currentfill}{rgb}{0.121569,0.466667,0.705882}%
\pgfsetfillcolor{currentfill}%
\pgfsetlinewidth{1.003750pt}%
\definecolor{currentstroke}{rgb}{0.121569,0.466667,0.705882}%
\pgfsetstrokecolor{currentstroke}%
\pgfsetdash{}{0pt}%
\pgfpathmoveto{\pgfqpoint{4.287742in}{3.014029in}}%
\pgfpathcurveto{\pgfqpoint{4.298792in}{3.014029in}}{\pgfqpoint{4.309391in}{3.018419in}}{\pgfqpoint{4.317205in}{3.026232in}}%
\pgfpathcurveto{\pgfqpoint{4.325019in}{3.034046in}}{\pgfqpoint{4.329409in}{3.044645in}}{\pgfqpoint{4.329409in}{3.055695in}}%
\pgfpathcurveto{\pgfqpoint{4.329409in}{3.066745in}}{\pgfqpoint{4.325019in}{3.077344in}}{\pgfqpoint{4.317205in}{3.085158in}}%
\pgfpathcurveto{\pgfqpoint{4.309391in}{3.092972in}}{\pgfqpoint{4.298792in}{3.097362in}}{\pgfqpoint{4.287742in}{3.097362in}}%
\pgfpathcurveto{\pgfqpoint{4.276692in}{3.097362in}}{\pgfqpoint{4.266093in}{3.092972in}}{\pgfqpoint{4.258279in}{3.085158in}}%
\pgfpathcurveto{\pgfqpoint{4.250466in}{3.077344in}}{\pgfqpoint{4.246076in}{3.066745in}}{\pgfqpoint{4.246076in}{3.055695in}}%
\pgfpathcurveto{\pgfqpoint{4.246076in}{3.044645in}}{\pgfqpoint{4.250466in}{3.034046in}}{\pgfqpoint{4.258279in}{3.026232in}}%
\pgfpathcurveto{\pgfqpoint{4.266093in}{3.018419in}}{\pgfqpoint{4.276692in}{3.014029in}}{\pgfqpoint{4.287742in}{3.014029in}}%
\pgfpathclose%
\pgfusepath{stroke,fill}%
\end{pgfscope}%
\begin{pgfscope}%
\pgfpathrectangle{\pgfqpoint{0.750000in}{0.500000in}}{\pgfqpoint{4.650000in}{3.020000in}}%
\pgfusepath{clip}%
\pgfsetbuttcap%
\pgfsetroundjoin%
\definecolor{currentfill}{rgb}{0.839216,0.152941,0.156863}%
\pgfsetfillcolor{currentfill}%
\pgfsetlinewidth{1.003750pt}%
\definecolor{currentstroke}{rgb}{0.839216,0.152941,0.156863}%
\pgfsetstrokecolor{currentstroke}%
\pgfsetdash{}{0pt}%
\pgfpathmoveto{\pgfqpoint{1.238562in}{3.030178in}}%
\pgfpathcurveto{\pgfqpoint{1.249612in}{3.030178in}}{\pgfqpoint{1.260211in}{3.034569in}}{\pgfqpoint{1.268025in}{3.042382in}}%
\pgfpathcurveto{\pgfqpoint{1.275838in}{3.050196in}}{\pgfqpoint{1.280229in}{3.060795in}}{\pgfqpoint{1.280229in}{3.071845in}}%
\pgfpathcurveto{\pgfqpoint{1.280229in}{3.082895in}}{\pgfqpoint{1.275838in}{3.093494in}}{\pgfqpoint{1.268025in}{3.101308in}}%
\pgfpathcurveto{\pgfqpoint{1.260211in}{3.109121in}}{\pgfqpoint{1.249612in}{3.113512in}}{\pgfqpoint{1.238562in}{3.113512in}}%
\pgfpathcurveto{\pgfqpoint{1.227512in}{3.113512in}}{\pgfqpoint{1.216913in}{3.109121in}}{\pgfqpoint{1.209099in}{3.101308in}}%
\pgfpathcurveto{\pgfqpoint{1.201285in}{3.093494in}}{\pgfqpoint{1.196895in}{3.082895in}}{\pgfqpoint{1.196895in}{3.071845in}}%
\pgfpathcurveto{\pgfqpoint{1.196895in}{3.060795in}}{\pgfqpoint{1.201285in}{3.050196in}}{\pgfqpoint{1.209099in}{3.042382in}}%
\pgfpathcurveto{\pgfqpoint{1.216913in}{3.034569in}}{\pgfqpoint{1.227512in}{3.030178in}}{\pgfqpoint{1.238562in}{3.030178in}}%
\pgfpathclose%
\pgfusepath{stroke,fill}%
\end{pgfscope}%
\begin{pgfscope}%
\pgfpathrectangle{\pgfqpoint{0.750000in}{0.500000in}}{\pgfqpoint{4.650000in}{3.020000in}}%
\pgfusepath{clip}%
\pgfsetbuttcap%
\pgfsetroundjoin%
\definecolor{currentfill}{rgb}{0.121569,0.466667,0.705882}%
\pgfsetfillcolor{currentfill}%
\pgfsetlinewidth{1.003750pt}%
\definecolor{currentstroke}{rgb}{0.121569,0.466667,0.705882}%
\pgfsetstrokecolor{currentstroke}%
\pgfsetdash{}{0pt}%
\pgfpathmoveto{\pgfqpoint{1.654359in}{3.014029in}}%
\pgfpathcurveto{\pgfqpoint{1.665409in}{3.014029in}}{\pgfqpoint{1.676008in}{3.018419in}}{\pgfqpoint{1.683822in}{3.026232in}}%
\pgfpathcurveto{\pgfqpoint{1.691636in}{3.034046in}}{\pgfqpoint{1.696026in}{3.044645in}}{\pgfqpoint{1.696026in}{3.055695in}}%
\pgfpathcurveto{\pgfqpoint{1.696026in}{3.066745in}}{\pgfqpoint{1.691636in}{3.077344in}}{\pgfqpoint{1.683822in}{3.085158in}}%
\pgfpathcurveto{\pgfqpoint{1.676008in}{3.092972in}}{\pgfqpoint{1.665409in}{3.097362in}}{\pgfqpoint{1.654359in}{3.097362in}}%
\pgfpathcurveto{\pgfqpoint{1.643309in}{3.097362in}}{\pgfqpoint{1.632710in}{3.092972in}}{\pgfqpoint{1.624896in}{3.085158in}}%
\pgfpathcurveto{\pgfqpoint{1.617083in}{3.077344in}}{\pgfqpoint{1.612692in}{3.066745in}}{\pgfqpoint{1.612692in}{3.055695in}}%
\pgfpathcurveto{\pgfqpoint{1.612692in}{3.044645in}}{\pgfqpoint{1.617083in}{3.034046in}}{\pgfqpoint{1.624896in}{3.026232in}}%
\pgfpathcurveto{\pgfqpoint{1.632710in}{3.018419in}}{\pgfqpoint{1.643309in}{3.014029in}}{\pgfqpoint{1.654359in}{3.014029in}}%
\pgfpathclose%
\pgfusepath{stroke,fill}%
\end{pgfscope}%
\begin{pgfscope}%
\pgfpathrectangle{\pgfqpoint{0.750000in}{0.500000in}}{\pgfqpoint{4.650000in}{3.020000in}}%
\pgfusepath{clip}%
\pgfsetbuttcap%
\pgfsetroundjoin%
\definecolor{currentfill}{rgb}{0.839216,0.152941,0.156863}%
\pgfsetfillcolor{currentfill}%
\pgfsetlinewidth{1.003750pt}%
\definecolor{currentstroke}{rgb}{0.839216,0.152941,0.156863}%
\pgfsetstrokecolor{currentstroke}%
\pgfsetdash{}{0pt}%
\pgfpathmoveto{\pgfqpoint{2.763152in}{3.001916in}}%
\pgfpathcurveto{\pgfqpoint{2.774202in}{3.001916in}}{\pgfqpoint{2.784801in}{3.006306in}}{\pgfqpoint{2.792615in}{3.014120in}}%
\pgfpathcurveto{\pgfqpoint{2.800428in}{3.021934in}}{\pgfqpoint{2.804819in}{3.032533in}}{\pgfqpoint{2.804819in}{3.043583in}}%
\pgfpathcurveto{\pgfqpoint{2.804819in}{3.054633in}}{\pgfqpoint{2.800428in}{3.065232in}}{\pgfqpoint{2.792615in}{3.073046in}}%
\pgfpathcurveto{\pgfqpoint{2.784801in}{3.080859in}}{\pgfqpoint{2.774202in}{3.085250in}}{\pgfqpoint{2.763152in}{3.085250in}}%
\pgfpathcurveto{\pgfqpoint{2.752102in}{3.085250in}}{\pgfqpoint{2.741503in}{3.080859in}}{\pgfqpoint{2.733689in}{3.073046in}}%
\pgfpathcurveto{\pgfqpoint{2.725876in}{3.065232in}}{\pgfqpoint{2.721485in}{3.054633in}}{\pgfqpoint{2.721485in}{3.043583in}}%
\pgfpathcurveto{\pgfqpoint{2.721485in}{3.032533in}}{\pgfqpoint{2.725876in}{3.021934in}}{\pgfqpoint{2.733689in}{3.014120in}}%
\pgfpathcurveto{\pgfqpoint{2.741503in}{3.006306in}}{\pgfqpoint{2.752102in}{3.001916in}}{\pgfqpoint{2.763152in}{3.001916in}}%
\pgfpathclose%
\pgfusepath{stroke,fill}%
\end{pgfscope}%
\begin{pgfscope}%
\pgfpathrectangle{\pgfqpoint{0.750000in}{0.500000in}}{\pgfqpoint{4.650000in}{3.020000in}}%
\pgfusepath{clip}%
\pgfsetbuttcap%
\pgfsetroundjoin%
\definecolor{currentfill}{rgb}{1.000000,0.498039,0.054902}%
\pgfsetfillcolor{currentfill}%
\pgfsetlinewidth{1.003750pt}%
\definecolor{currentstroke}{rgb}{1.000000,0.498039,0.054902}%
\pgfsetstrokecolor{currentstroke}%
\pgfsetdash{}{0pt}%
\pgfpathmoveto{\pgfqpoint{1.931557in}{3.123039in}}%
\pgfpathcurveto{\pgfqpoint{1.942608in}{3.123039in}}{\pgfqpoint{1.953207in}{3.127429in}}{\pgfqpoint{1.961020in}{3.135243in}}%
\pgfpathcurveto{\pgfqpoint{1.968834in}{3.143057in}}{\pgfqpoint{1.973224in}{3.153656in}}{\pgfqpoint{1.973224in}{3.164706in}}%
\pgfpathcurveto{\pgfqpoint{1.973224in}{3.175756in}}{\pgfqpoint{1.968834in}{3.186355in}}{\pgfqpoint{1.961020in}{3.194169in}}%
\pgfpathcurveto{\pgfqpoint{1.953207in}{3.201982in}}{\pgfqpoint{1.942608in}{3.206373in}}{\pgfqpoint{1.931557in}{3.206373in}}%
\pgfpathcurveto{\pgfqpoint{1.920507in}{3.206373in}}{\pgfqpoint{1.909908in}{3.201982in}}{\pgfqpoint{1.902095in}{3.194169in}}%
\pgfpathcurveto{\pgfqpoint{1.894281in}{3.186355in}}{\pgfqpoint{1.889891in}{3.175756in}}{\pgfqpoint{1.889891in}{3.164706in}}%
\pgfpathcurveto{\pgfqpoint{1.889891in}{3.153656in}}{\pgfqpoint{1.894281in}{3.143057in}}{\pgfqpoint{1.902095in}{3.135243in}}%
\pgfpathcurveto{\pgfqpoint{1.909908in}{3.127429in}}{\pgfqpoint{1.920507in}{3.123039in}}{\pgfqpoint{1.931557in}{3.123039in}}%
\pgfpathclose%
\pgfusepath{stroke,fill}%
\end{pgfscope}%
\begin{pgfscope}%
\pgfpathrectangle{\pgfqpoint{0.750000in}{0.500000in}}{\pgfqpoint{4.650000in}{3.020000in}}%
\pgfusepath{clip}%
\pgfsetbuttcap%
\pgfsetroundjoin%
\definecolor{currentfill}{rgb}{0.121569,0.466667,0.705882}%
\pgfsetfillcolor{currentfill}%
\pgfsetlinewidth{1.003750pt}%
\definecolor{currentstroke}{rgb}{0.121569,0.466667,0.705882}%
\pgfsetstrokecolor{currentstroke}%
\pgfsetdash{}{0pt}%
\pgfpathmoveto{\pgfqpoint{1.377161in}{0.603681in}}%
\pgfpathcurveto{\pgfqpoint{1.388211in}{0.603681in}}{\pgfqpoint{1.398810in}{0.608071in}}{\pgfqpoint{1.406624in}{0.615885in}}%
\pgfpathcurveto{\pgfqpoint{1.414437in}{0.623698in}}{\pgfqpoint{1.418828in}{0.634297in}}{\pgfqpoint{1.418828in}{0.645348in}}%
\pgfpathcurveto{\pgfqpoint{1.418828in}{0.656398in}}{\pgfqpoint{1.414437in}{0.666997in}}{\pgfqpoint{1.406624in}{0.674810in}}%
\pgfpathcurveto{\pgfqpoint{1.398810in}{0.682624in}}{\pgfqpoint{1.388211in}{0.687014in}}{\pgfqpoint{1.377161in}{0.687014in}}%
\pgfpathcurveto{\pgfqpoint{1.366111in}{0.687014in}}{\pgfqpoint{1.355512in}{0.682624in}}{\pgfqpoint{1.347698in}{0.674810in}}%
\pgfpathcurveto{\pgfqpoint{1.339885in}{0.666997in}}{\pgfqpoint{1.335494in}{0.656398in}}{\pgfqpoint{1.335494in}{0.645348in}}%
\pgfpathcurveto{\pgfqpoint{1.335494in}{0.634297in}}{\pgfqpoint{1.339885in}{0.623698in}}{\pgfqpoint{1.347698in}{0.615885in}}%
\pgfpathcurveto{\pgfqpoint{1.355512in}{0.608071in}}{\pgfqpoint{1.366111in}{0.603681in}}{\pgfqpoint{1.377161in}{0.603681in}}%
\pgfpathclose%
\pgfusepath{stroke,fill}%
\end{pgfscope}%
\begin{pgfscope}%
\pgfpathrectangle{\pgfqpoint{0.750000in}{0.500000in}}{\pgfqpoint{4.650000in}{3.020000in}}%
\pgfusepath{clip}%
\pgfsetbuttcap%
\pgfsetroundjoin%
\definecolor{currentfill}{rgb}{0.121569,0.466667,0.705882}%
\pgfsetfillcolor{currentfill}%
\pgfsetlinewidth{1.003750pt}%
\definecolor{currentstroke}{rgb}{0.121569,0.466667,0.705882}%
\pgfsetstrokecolor{currentstroke}%
\pgfsetdash{}{0pt}%
\pgfpathmoveto{\pgfqpoint{2.000857in}{3.005954in}}%
\pgfpathcurveto{\pgfqpoint{2.011907in}{3.005954in}}{\pgfqpoint{2.022506in}{3.010344in}}{\pgfqpoint{2.030320in}{3.018158in}}%
\pgfpathcurveto{\pgfqpoint{2.038133in}{3.025971in}}{\pgfqpoint{2.042524in}{3.036570in}}{\pgfqpoint{2.042524in}{3.047620in}}%
\pgfpathcurveto{\pgfqpoint{2.042524in}{3.058670in}}{\pgfqpoint{2.038133in}{3.069269in}}{\pgfqpoint{2.030320in}{3.077083in}}%
\pgfpathcurveto{\pgfqpoint{2.022506in}{3.084897in}}{\pgfqpoint{2.011907in}{3.089287in}}{\pgfqpoint{2.000857in}{3.089287in}}%
\pgfpathcurveto{\pgfqpoint{1.989807in}{3.089287in}}{\pgfqpoint{1.979208in}{3.084897in}}{\pgfqpoint{1.971394in}{3.077083in}}%
\pgfpathcurveto{\pgfqpoint{1.963581in}{3.069269in}}{\pgfqpoint{1.959190in}{3.058670in}}{\pgfqpoint{1.959190in}{3.047620in}}%
\pgfpathcurveto{\pgfqpoint{1.959190in}{3.036570in}}{\pgfqpoint{1.963581in}{3.025971in}}{\pgfqpoint{1.971394in}{3.018158in}}%
\pgfpathcurveto{\pgfqpoint{1.979208in}{3.010344in}}{\pgfqpoint{1.989807in}{3.005954in}}{\pgfqpoint{2.000857in}{3.005954in}}%
\pgfpathclose%
\pgfusepath{stroke,fill}%
\end{pgfscope}%
\begin{pgfscope}%
\pgfpathrectangle{\pgfqpoint{0.750000in}{0.500000in}}{\pgfqpoint{4.650000in}{3.020000in}}%
\pgfusepath{clip}%
\pgfsetbuttcap%
\pgfsetroundjoin%
\definecolor{currentfill}{rgb}{0.121569,0.466667,0.705882}%
\pgfsetfillcolor{currentfill}%
\pgfsetlinewidth{1.003750pt}%
\definecolor{currentstroke}{rgb}{0.121569,0.466667,0.705882}%
\pgfsetstrokecolor{currentstroke}%
\pgfsetdash{}{0pt}%
\pgfpathmoveto{\pgfqpoint{0.961364in}{0.595606in}}%
\pgfpathcurveto{\pgfqpoint{0.972414in}{0.595606in}}{\pgfqpoint{0.983013in}{0.599996in}}{\pgfqpoint{0.990826in}{0.607810in}}%
\pgfpathcurveto{\pgfqpoint{0.998640in}{0.615624in}}{\pgfqpoint{1.003030in}{0.626223in}}{\pgfqpoint{1.003030in}{0.637273in}}%
\pgfpathcurveto{\pgfqpoint{1.003030in}{0.648323in}}{\pgfqpoint{0.998640in}{0.658922in}}{\pgfqpoint{0.990826in}{0.666736in}}%
\pgfpathcurveto{\pgfqpoint{0.983013in}{0.674549in}}{\pgfqpoint{0.972414in}{0.678939in}}{\pgfqpoint{0.961364in}{0.678939in}}%
\pgfpathcurveto{\pgfqpoint{0.950314in}{0.678939in}}{\pgfqpoint{0.939714in}{0.674549in}}{\pgfqpoint{0.931901in}{0.666736in}}%
\pgfpathcurveto{\pgfqpoint{0.924087in}{0.658922in}}{\pgfqpoint{0.919697in}{0.648323in}}{\pgfqpoint{0.919697in}{0.637273in}}%
\pgfpathcurveto{\pgfqpoint{0.919697in}{0.626223in}}{\pgfqpoint{0.924087in}{0.615624in}}{\pgfqpoint{0.931901in}{0.607810in}}%
\pgfpathcurveto{\pgfqpoint{0.939714in}{0.599996in}}{\pgfqpoint{0.950314in}{0.595606in}}{\pgfqpoint{0.961364in}{0.595606in}}%
\pgfpathclose%
\pgfusepath{stroke,fill}%
\end{pgfscope}%
\begin{pgfscope}%
\pgfpathrectangle{\pgfqpoint{0.750000in}{0.500000in}}{\pgfqpoint{4.650000in}{3.020000in}}%
\pgfusepath{clip}%
\pgfsetbuttcap%
\pgfsetroundjoin%
\definecolor{currentfill}{rgb}{0.121569,0.466667,0.705882}%
\pgfsetfillcolor{currentfill}%
\pgfsetlinewidth{1.003750pt}%
\definecolor{currentstroke}{rgb}{0.121569,0.466667,0.705882}%
\pgfsetstrokecolor{currentstroke}%
\pgfsetdash{}{0pt}%
\pgfpathmoveto{\pgfqpoint{2.555253in}{3.009991in}}%
\pgfpathcurveto{\pgfqpoint{2.566303in}{3.009991in}}{\pgfqpoint{2.576903in}{3.014381in}}{\pgfqpoint{2.584716in}{3.022195in}}%
\pgfpathcurveto{\pgfqpoint{2.592530in}{3.030009in}}{\pgfqpoint{2.596920in}{3.040608in}}{\pgfqpoint{2.596920in}{3.051658in}}%
\pgfpathcurveto{\pgfqpoint{2.596920in}{3.062708in}}{\pgfqpoint{2.592530in}{3.073307in}}{\pgfqpoint{2.584716in}{3.081121in}}%
\pgfpathcurveto{\pgfqpoint{2.576903in}{3.088934in}}{\pgfqpoint{2.566303in}{3.093324in}}{\pgfqpoint{2.555253in}{3.093324in}}%
\pgfpathcurveto{\pgfqpoint{2.544203in}{3.093324in}}{\pgfqpoint{2.533604in}{3.088934in}}{\pgfqpoint{2.525791in}{3.081121in}}%
\pgfpathcurveto{\pgfqpoint{2.517977in}{3.073307in}}{\pgfqpoint{2.513587in}{3.062708in}}{\pgfqpoint{2.513587in}{3.051658in}}%
\pgfpathcurveto{\pgfqpoint{2.513587in}{3.040608in}}{\pgfqpoint{2.517977in}{3.030009in}}{\pgfqpoint{2.525791in}{3.022195in}}%
\pgfpathcurveto{\pgfqpoint{2.533604in}{3.014381in}}{\pgfqpoint{2.544203in}{3.009991in}}{\pgfqpoint{2.555253in}{3.009991in}}%
\pgfpathclose%
\pgfusepath{stroke,fill}%
\end{pgfscope}%
\begin{pgfscope}%
\pgfpathrectangle{\pgfqpoint{0.750000in}{0.500000in}}{\pgfqpoint{4.650000in}{3.020000in}}%
\pgfusepath{clip}%
\pgfsetbuttcap%
\pgfsetroundjoin%
\definecolor{currentfill}{rgb}{1.000000,0.498039,0.054902}%
\pgfsetfillcolor{currentfill}%
\pgfsetlinewidth{1.003750pt}%
\definecolor{currentstroke}{rgb}{1.000000,0.498039,0.054902}%
\pgfsetstrokecolor{currentstroke}%
\pgfsetdash{}{0pt}%
\pgfpathmoveto{\pgfqpoint{2.347355in}{3.018066in}}%
\pgfpathcurveto{\pgfqpoint{2.358405in}{3.018066in}}{\pgfqpoint{2.369004in}{3.022456in}}{\pgfqpoint{2.376817in}{3.030270in}}%
\pgfpathcurveto{\pgfqpoint{2.384631in}{3.038083in}}{\pgfqpoint{2.389021in}{3.048682in}}{\pgfqpoint{2.389021in}{3.059733in}}%
\pgfpathcurveto{\pgfqpoint{2.389021in}{3.070783in}}{\pgfqpoint{2.384631in}{3.081382in}}{\pgfqpoint{2.376817in}{3.089195in}}%
\pgfpathcurveto{\pgfqpoint{2.369004in}{3.097009in}}{\pgfqpoint{2.358405in}{3.101399in}}{\pgfqpoint{2.347355in}{3.101399in}}%
\pgfpathcurveto{\pgfqpoint{2.336305in}{3.101399in}}{\pgfqpoint{2.325706in}{3.097009in}}{\pgfqpoint{2.317892in}{3.089195in}}%
\pgfpathcurveto{\pgfqpoint{2.310078in}{3.081382in}}{\pgfqpoint{2.305688in}{3.070783in}}{\pgfqpoint{2.305688in}{3.059733in}}%
\pgfpathcurveto{\pgfqpoint{2.305688in}{3.048682in}}{\pgfqpoint{2.310078in}{3.038083in}}{\pgfqpoint{2.317892in}{3.030270in}}%
\pgfpathcurveto{\pgfqpoint{2.325706in}{3.022456in}}{\pgfqpoint{2.336305in}{3.018066in}}{\pgfqpoint{2.347355in}{3.018066in}}%
\pgfpathclose%
\pgfusepath{stroke,fill}%
\end{pgfscope}%
\begin{pgfscope}%
\pgfpathrectangle{\pgfqpoint{0.750000in}{0.500000in}}{\pgfqpoint{4.650000in}{3.020000in}}%
\pgfusepath{clip}%
\pgfsetbuttcap%
\pgfsetroundjoin%
\definecolor{currentfill}{rgb}{1.000000,0.498039,0.054902}%
\pgfsetfillcolor{currentfill}%
\pgfsetlinewidth{1.003750pt}%
\definecolor{currentstroke}{rgb}{1.000000,0.498039,0.054902}%
\pgfsetstrokecolor{currentstroke}%
\pgfsetdash{}{0pt}%
\pgfpathmoveto{\pgfqpoint{2.000857in}{3.018066in}}%
\pgfpathcurveto{\pgfqpoint{2.011907in}{3.018066in}}{\pgfqpoint{2.022506in}{3.022456in}}{\pgfqpoint{2.030320in}{3.030270in}}%
\pgfpathcurveto{\pgfqpoint{2.038133in}{3.038083in}}{\pgfqpoint{2.042524in}{3.048682in}}{\pgfqpoint{2.042524in}{3.059733in}}%
\pgfpathcurveto{\pgfqpoint{2.042524in}{3.070783in}}{\pgfqpoint{2.038133in}{3.081382in}}{\pgfqpoint{2.030320in}{3.089195in}}%
\pgfpathcurveto{\pgfqpoint{2.022506in}{3.097009in}}{\pgfqpoint{2.011907in}{3.101399in}}{\pgfqpoint{2.000857in}{3.101399in}}%
\pgfpathcurveto{\pgfqpoint{1.989807in}{3.101399in}}{\pgfqpoint{1.979208in}{3.097009in}}{\pgfqpoint{1.971394in}{3.089195in}}%
\pgfpathcurveto{\pgfqpoint{1.963581in}{3.081382in}}{\pgfqpoint{1.959190in}{3.070783in}}{\pgfqpoint{1.959190in}{3.059733in}}%
\pgfpathcurveto{\pgfqpoint{1.959190in}{3.048682in}}{\pgfqpoint{1.963581in}{3.038083in}}{\pgfqpoint{1.971394in}{3.030270in}}%
\pgfpathcurveto{\pgfqpoint{1.979208in}{3.022456in}}{\pgfqpoint{1.989807in}{3.018066in}}{\pgfqpoint{2.000857in}{3.018066in}}%
\pgfpathclose%
\pgfusepath{stroke,fill}%
\end{pgfscope}%
\begin{pgfscope}%
\pgfpathrectangle{\pgfqpoint{0.750000in}{0.500000in}}{\pgfqpoint{4.650000in}{3.020000in}}%
\pgfusepath{clip}%
\pgfsetbuttcap%
\pgfsetroundjoin%
\definecolor{currentfill}{rgb}{1.000000,0.498039,0.054902}%
\pgfsetfillcolor{currentfill}%
\pgfsetlinewidth{1.003750pt}%
\definecolor{currentstroke}{rgb}{1.000000,0.498039,0.054902}%
\pgfsetstrokecolor{currentstroke}%
\pgfsetdash{}{0pt}%
\pgfpathmoveto{\pgfqpoint{2.416654in}{3.018066in}}%
\pgfpathcurveto{\pgfqpoint{2.427704in}{3.018066in}}{\pgfqpoint{2.438303in}{3.022456in}}{\pgfqpoint{2.446117in}{3.030270in}}%
\pgfpathcurveto{\pgfqpoint{2.453931in}{3.038083in}}{\pgfqpoint{2.458321in}{3.048682in}}{\pgfqpoint{2.458321in}{3.059733in}}%
\pgfpathcurveto{\pgfqpoint{2.458321in}{3.070783in}}{\pgfqpoint{2.453931in}{3.081382in}}{\pgfqpoint{2.446117in}{3.089195in}}%
\pgfpathcurveto{\pgfqpoint{2.438303in}{3.097009in}}{\pgfqpoint{2.427704in}{3.101399in}}{\pgfqpoint{2.416654in}{3.101399in}}%
\pgfpathcurveto{\pgfqpoint{2.405604in}{3.101399in}}{\pgfqpoint{2.395005in}{3.097009in}}{\pgfqpoint{2.387191in}{3.089195in}}%
\pgfpathcurveto{\pgfqpoint{2.379378in}{3.081382in}}{\pgfqpoint{2.374988in}{3.070783in}}{\pgfqpoint{2.374988in}{3.059733in}}%
\pgfpathcurveto{\pgfqpoint{2.374988in}{3.048682in}}{\pgfqpoint{2.379378in}{3.038083in}}{\pgfqpoint{2.387191in}{3.030270in}}%
\pgfpathcurveto{\pgfqpoint{2.395005in}{3.022456in}}{\pgfqpoint{2.405604in}{3.018066in}}{\pgfqpoint{2.416654in}{3.018066in}}%
\pgfpathclose%
\pgfusepath{stroke,fill}%
\end{pgfscope}%
\begin{pgfscope}%
\pgfpathrectangle{\pgfqpoint{0.750000in}{0.500000in}}{\pgfqpoint{4.650000in}{3.020000in}}%
\pgfusepath{clip}%
\pgfsetbuttcap%
\pgfsetroundjoin%
\definecolor{currentfill}{rgb}{1.000000,0.498039,0.054902}%
\pgfsetfillcolor{currentfill}%
\pgfsetlinewidth{1.003750pt}%
\definecolor{currentstroke}{rgb}{1.000000,0.498039,0.054902}%
\pgfsetstrokecolor{currentstroke}%
\pgfsetdash{}{0pt}%
\pgfpathmoveto{\pgfqpoint{1.862258in}{3.022103in}}%
\pgfpathcurveto{\pgfqpoint{1.873308in}{3.022103in}}{\pgfqpoint{1.883907in}{3.026494in}}{\pgfqpoint{1.891721in}{3.034307in}}%
\pgfpathcurveto{\pgfqpoint{1.899534in}{3.042121in}}{\pgfqpoint{1.903924in}{3.052720in}}{\pgfqpoint{1.903924in}{3.063770in}}%
\pgfpathcurveto{\pgfqpoint{1.903924in}{3.074820in}}{\pgfqpoint{1.899534in}{3.085419in}}{\pgfqpoint{1.891721in}{3.093233in}}%
\pgfpathcurveto{\pgfqpoint{1.883907in}{3.101046in}}{\pgfqpoint{1.873308in}{3.105437in}}{\pgfqpoint{1.862258in}{3.105437in}}%
\pgfpathcurveto{\pgfqpoint{1.851208in}{3.105437in}}{\pgfqpoint{1.840609in}{3.101046in}}{\pgfqpoint{1.832795in}{3.093233in}}%
\pgfpathcurveto{\pgfqpoint{1.824981in}{3.085419in}}{\pgfqpoint{1.820591in}{3.074820in}}{\pgfqpoint{1.820591in}{3.063770in}}%
\pgfpathcurveto{\pgfqpoint{1.820591in}{3.052720in}}{\pgfqpoint{1.824981in}{3.042121in}}{\pgfqpoint{1.832795in}{3.034307in}}%
\pgfpathcurveto{\pgfqpoint{1.840609in}{3.026494in}}{\pgfqpoint{1.851208in}{3.022103in}}{\pgfqpoint{1.862258in}{3.022103in}}%
\pgfpathclose%
\pgfusepath{stroke,fill}%
\end{pgfscope}%
\begin{pgfscope}%
\pgfpathrectangle{\pgfqpoint{0.750000in}{0.500000in}}{\pgfqpoint{4.650000in}{3.020000in}}%
\pgfusepath{clip}%
\pgfsetbuttcap%
\pgfsetroundjoin%
\definecolor{currentfill}{rgb}{1.000000,0.498039,0.054902}%
\pgfsetfillcolor{currentfill}%
\pgfsetlinewidth{1.003750pt}%
\definecolor{currentstroke}{rgb}{1.000000,0.498039,0.054902}%
\pgfsetstrokecolor{currentstroke}%
\pgfsetdash{}{0pt}%
\pgfpathmoveto{\pgfqpoint{2.070156in}{3.018066in}}%
\pgfpathcurveto{\pgfqpoint{2.081207in}{3.018066in}}{\pgfqpoint{2.091806in}{3.022456in}}{\pgfqpoint{2.099619in}{3.030270in}}%
\pgfpathcurveto{\pgfqpoint{2.107433in}{3.038083in}}{\pgfqpoint{2.111823in}{3.048682in}}{\pgfqpoint{2.111823in}{3.059733in}}%
\pgfpathcurveto{\pgfqpoint{2.111823in}{3.070783in}}{\pgfqpoint{2.107433in}{3.081382in}}{\pgfqpoint{2.099619in}{3.089195in}}%
\pgfpathcurveto{\pgfqpoint{2.091806in}{3.097009in}}{\pgfqpoint{2.081207in}{3.101399in}}{\pgfqpoint{2.070156in}{3.101399in}}%
\pgfpathcurveto{\pgfqpoint{2.059106in}{3.101399in}}{\pgfqpoint{2.048507in}{3.097009in}}{\pgfqpoint{2.040694in}{3.089195in}}%
\pgfpathcurveto{\pgfqpoint{2.032880in}{3.081382in}}{\pgfqpoint{2.028490in}{3.070783in}}{\pgfqpoint{2.028490in}{3.059733in}}%
\pgfpathcurveto{\pgfqpoint{2.028490in}{3.048682in}}{\pgfqpoint{2.032880in}{3.038083in}}{\pgfqpoint{2.040694in}{3.030270in}}%
\pgfpathcurveto{\pgfqpoint{2.048507in}{3.022456in}}{\pgfqpoint{2.059106in}{3.018066in}}{\pgfqpoint{2.070156in}{3.018066in}}%
\pgfpathclose%
\pgfusepath{stroke,fill}%
\end{pgfscope}%
\begin{pgfscope}%
\pgfpathrectangle{\pgfqpoint{0.750000in}{0.500000in}}{\pgfqpoint{4.650000in}{3.020000in}}%
\pgfusepath{clip}%
\pgfsetbuttcap%
\pgfsetroundjoin%
\definecolor{currentfill}{rgb}{1.000000,0.498039,0.054902}%
\pgfsetfillcolor{currentfill}%
\pgfsetlinewidth{1.003750pt}%
\definecolor{currentstroke}{rgb}{1.000000,0.498039,0.054902}%
\pgfsetstrokecolor{currentstroke}%
\pgfsetdash{}{0pt}%
\pgfpathmoveto{\pgfqpoint{2.139456in}{3.046328in}}%
\pgfpathcurveto{\pgfqpoint{2.150506in}{3.046328in}}{\pgfqpoint{2.161105in}{3.050718in}}{\pgfqpoint{2.168919in}{3.058532in}}%
\pgfpathcurveto{\pgfqpoint{2.176732in}{3.066345in}}{\pgfqpoint{2.181123in}{3.076945in}}{\pgfqpoint{2.181123in}{3.087995in}}%
\pgfpathcurveto{\pgfqpoint{2.181123in}{3.099045in}}{\pgfqpoint{2.176732in}{3.109644in}}{\pgfqpoint{2.168919in}{3.117457in}}%
\pgfpathcurveto{\pgfqpoint{2.161105in}{3.125271in}}{\pgfqpoint{2.150506in}{3.129661in}}{\pgfqpoint{2.139456in}{3.129661in}}%
\pgfpathcurveto{\pgfqpoint{2.128406in}{3.129661in}}{\pgfqpoint{2.117807in}{3.125271in}}{\pgfqpoint{2.109993in}{3.117457in}}%
\pgfpathcurveto{\pgfqpoint{2.102180in}{3.109644in}}{\pgfqpoint{2.097789in}{3.099045in}}{\pgfqpoint{2.097789in}{3.087995in}}%
\pgfpathcurveto{\pgfqpoint{2.097789in}{3.076945in}}{\pgfqpoint{2.102180in}{3.066345in}}{\pgfqpoint{2.109993in}{3.058532in}}%
\pgfpathcurveto{\pgfqpoint{2.117807in}{3.050718in}}{\pgfqpoint{2.128406in}{3.046328in}}{\pgfqpoint{2.139456in}{3.046328in}}%
\pgfpathclose%
\pgfusepath{stroke,fill}%
\end{pgfscope}%
\begin{pgfscope}%
\pgfpathrectangle{\pgfqpoint{0.750000in}{0.500000in}}{\pgfqpoint{4.650000in}{3.020000in}}%
\pgfusepath{clip}%
\pgfsetbuttcap%
\pgfsetroundjoin%
\definecolor{currentfill}{rgb}{1.000000,0.498039,0.054902}%
\pgfsetfillcolor{currentfill}%
\pgfsetlinewidth{1.003750pt}%
\definecolor{currentstroke}{rgb}{1.000000,0.498039,0.054902}%
\pgfsetstrokecolor{currentstroke}%
\pgfsetdash{}{0pt}%
\pgfpathmoveto{\pgfqpoint{1.169262in}{3.026141in}}%
\pgfpathcurveto{\pgfqpoint{1.180312in}{3.026141in}}{\pgfqpoint{1.190911in}{3.030531in}}{\pgfqpoint{1.198725in}{3.038345in}}%
\pgfpathcurveto{\pgfqpoint{1.206539in}{3.046158in}}{\pgfqpoint{1.210929in}{3.056757in}}{\pgfqpoint{1.210929in}{3.067807in}}%
\pgfpathcurveto{\pgfqpoint{1.210929in}{3.078858in}}{\pgfqpoint{1.206539in}{3.089457in}}{\pgfqpoint{1.198725in}{3.097270in}}%
\pgfpathcurveto{\pgfqpoint{1.190911in}{3.105084in}}{\pgfqpoint{1.180312in}{3.109474in}}{\pgfqpoint{1.169262in}{3.109474in}}%
\pgfpathcurveto{\pgfqpoint{1.158212in}{3.109474in}}{\pgfqpoint{1.147613in}{3.105084in}}{\pgfqpoint{1.139800in}{3.097270in}}%
\pgfpathcurveto{\pgfqpoint{1.131986in}{3.089457in}}{\pgfqpoint{1.127596in}{3.078858in}}{\pgfqpoint{1.127596in}{3.067807in}}%
\pgfpathcurveto{\pgfqpoint{1.127596in}{3.056757in}}{\pgfqpoint{1.131986in}{3.046158in}}{\pgfqpoint{1.139800in}{3.038345in}}%
\pgfpathcurveto{\pgfqpoint{1.147613in}{3.030531in}}{\pgfqpoint{1.158212in}{3.026141in}}{\pgfqpoint{1.169262in}{3.026141in}}%
\pgfpathclose%
\pgfusepath{stroke,fill}%
\end{pgfscope}%
\begin{pgfscope}%
\pgfpathrectangle{\pgfqpoint{0.750000in}{0.500000in}}{\pgfqpoint{4.650000in}{3.020000in}}%
\pgfusepath{clip}%
\pgfsetbuttcap%
\pgfsetroundjoin%
\definecolor{currentfill}{rgb}{1.000000,0.498039,0.054902}%
\pgfsetfillcolor{currentfill}%
\pgfsetlinewidth{1.003750pt}%
\definecolor{currentstroke}{rgb}{1.000000,0.498039,0.054902}%
\pgfsetstrokecolor{currentstroke}%
\pgfsetdash{}{0pt}%
\pgfpathmoveto{\pgfqpoint{1.169262in}{3.038253in}}%
\pgfpathcurveto{\pgfqpoint{1.180312in}{3.038253in}}{\pgfqpoint{1.190911in}{3.042643in}}{\pgfqpoint{1.198725in}{3.050457in}}%
\pgfpathcurveto{\pgfqpoint{1.206539in}{3.058271in}}{\pgfqpoint{1.210929in}{3.068870in}}{\pgfqpoint{1.210929in}{3.079920in}}%
\pgfpathcurveto{\pgfqpoint{1.210929in}{3.090970in}}{\pgfqpoint{1.206539in}{3.101569in}}{\pgfqpoint{1.198725in}{3.109383in}}%
\pgfpathcurveto{\pgfqpoint{1.190911in}{3.117196in}}{\pgfqpoint{1.180312in}{3.121586in}}{\pgfqpoint{1.169262in}{3.121586in}}%
\pgfpathcurveto{\pgfqpoint{1.158212in}{3.121586in}}{\pgfqpoint{1.147613in}{3.117196in}}{\pgfqpoint{1.139800in}{3.109383in}}%
\pgfpathcurveto{\pgfqpoint{1.131986in}{3.101569in}}{\pgfqpoint{1.127596in}{3.090970in}}{\pgfqpoint{1.127596in}{3.079920in}}%
\pgfpathcurveto{\pgfqpoint{1.127596in}{3.068870in}}{\pgfqpoint{1.131986in}{3.058271in}}{\pgfqpoint{1.139800in}{3.050457in}}%
\pgfpathcurveto{\pgfqpoint{1.147613in}{3.042643in}}{\pgfqpoint{1.158212in}{3.038253in}}{\pgfqpoint{1.169262in}{3.038253in}}%
\pgfpathclose%
\pgfusepath{stroke,fill}%
\end{pgfscope}%
\begin{pgfscope}%
\pgfpathrectangle{\pgfqpoint{0.750000in}{0.500000in}}{\pgfqpoint{4.650000in}{3.020000in}}%
\pgfusepath{clip}%
\pgfsetbuttcap%
\pgfsetroundjoin%
\definecolor{currentfill}{rgb}{1.000000,0.498039,0.054902}%
\pgfsetfillcolor{currentfill}%
\pgfsetlinewidth{1.003750pt}%
\definecolor{currentstroke}{rgb}{1.000000,0.498039,0.054902}%
\pgfsetstrokecolor{currentstroke}%
\pgfsetdash{}{0pt}%
\pgfpathmoveto{\pgfqpoint{1.654359in}{3.199750in}}%
\pgfpathcurveto{\pgfqpoint{1.665409in}{3.199750in}}{\pgfqpoint{1.676008in}{3.204141in}}{\pgfqpoint{1.683822in}{3.211954in}}%
\pgfpathcurveto{\pgfqpoint{1.691636in}{3.219768in}}{\pgfqpoint{1.696026in}{3.230367in}}{\pgfqpoint{1.696026in}{3.241417in}}%
\pgfpathcurveto{\pgfqpoint{1.696026in}{3.252467in}}{\pgfqpoint{1.691636in}{3.263066in}}{\pgfqpoint{1.683822in}{3.270880in}}%
\pgfpathcurveto{\pgfqpoint{1.676008in}{3.278694in}}{\pgfqpoint{1.665409in}{3.283084in}}{\pgfqpoint{1.654359in}{3.283084in}}%
\pgfpathcurveto{\pgfqpoint{1.643309in}{3.283084in}}{\pgfqpoint{1.632710in}{3.278694in}}{\pgfqpoint{1.624896in}{3.270880in}}%
\pgfpathcurveto{\pgfqpoint{1.617083in}{3.263066in}}{\pgfqpoint{1.612692in}{3.252467in}}{\pgfqpoint{1.612692in}{3.241417in}}%
\pgfpathcurveto{\pgfqpoint{1.612692in}{3.230367in}}{\pgfqpoint{1.617083in}{3.219768in}}{\pgfqpoint{1.624896in}{3.211954in}}%
\pgfpathcurveto{\pgfqpoint{1.632710in}{3.204141in}}{\pgfqpoint{1.643309in}{3.199750in}}{\pgfqpoint{1.654359in}{3.199750in}}%
\pgfpathclose%
\pgfusepath{stroke,fill}%
\end{pgfscope}%
\begin{pgfscope}%
\pgfpathrectangle{\pgfqpoint{0.750000in}{0.500000in}}{\pgfqpoint{4.650000in}{3.020000in}}%
\pgfusepath{clip}%
\pgfsetbuttcap%
\pgfsetroundjoin%
\definecolor{currentfill}{rgb}{0.121569,0.466667,0.705882}%
\pgfsetfillcolor{currentfill}%
\pgfsetlinewidth{1.003750pt}%
\definecolor{currentstroke}{rgb}{0.121569,0.466667,0.705882}%
\pgfsetstrokecolor{currentstroke}%
\pgfsetdash{}{0pt}%
\pgfpathmoveto{\pgfqpoint{1.030663in}{2.311515in}}%
\pgfpathcurveto{\pgfqpoint{1.041713in}{2.311515in}}{\pgfqpoint{1.052312in}{2.315905in}}{\pgfqpoint{1.060126in}{2.323719in}}%
\pgfpathcurveto{\pgfqpoint{1.067940in}{2.331533in}}{\pgfqpoint{1.072330in}{2.342132in}}{\pgfqpoint{1.072330in}{2.353182in}}%
\pgfpathcurveto{\pgfqpoint{1.072330in}{2.364232in}}{\pgfqpoint{1.067940in}{2.374831in}}{\pgfqpoint{1.060126in}{2.382645in}}%
\pgfpathcurveto{\pgfqpoint{1.052312in}{2.390458in}}{\pgfqpoint{1.041713in}{2.394848in}}{\pgfqpoint{1.030663in}{2.394848in}}%
\pgfpathcurveto{\pgfqpoint{1.019613in}{2.394848in}}{\pgfqpoint{1.009014in}{2.390458in}}{\pgfqpoint{1.001200in}{2.382645in}}%
\pgfpathcurveto{\pgfqpoint{0.993387in}{2.374831in}}{\pgfqpoint{0.988997in}{2.364232in}}{\pgfqpoint{0.988997in}{2.353182in}}%
\pgfpathcurveto{\pgfqpoint{0.988997in}{2.342132in}}{\pgfqpoint{0.993387in}{2.331533in}}{\pgfqpoint{1.001200in}{2.323719in}}%
\pgfpathcurveto{\pgfqpoint{1.009014in}{2.315905in}}{\pgfqpoint{1.019613in}{2.311515in}}{\pgfqpoint{1.030663in}{2.311515in}}%
\pgfpathclose%
\pgfusepath{stroke,fill}%
\end{pgfscope}%
\begin{pgfscope}%
\pgfpathrectangle{\pgfqpoint{0.750000in}{0.500000in}}{\pgfqpoint{4.650000in}{3.020000in}}%
\pgfusepath{clip}%
\pgfsetbuttcap%
\pgfsetroundjoin%
\definecolor{currentfill}{rgb}{1.000000,0.498039,0.054902}%
\pgfsetfillcolor{currentfill}%
\pgfsetlinewidth{1.003750pt}%
\definecolor{currentstroke}{rgb}{1.000000,0.498039,0.054902}%
\pgfsetstrokecolor{currentstroke}%
\pgfsetdash{}{0pt}%
\pgfpathmoveto{\pgfqpoint{1.307861in}{3.078627in}}%
\pgfpathcurveto{\pgfqpoint{1.318912in}{3.078627in}}{\pgfqpoint{1.329511in}{3.083018in}}{\pgfqpoint{1.337324in}{3.090831in}}%
\pgfpathcurveto{\pgfqpoint{1.345138in}{3.098645in}}{\pgfqpoint{1.349528in}{3.109244in}}{\pgfqpoint{1.349528in}{3.120294in}}%
\pgfpathcurveto{\pgfqpoint{1.349528in}{3.131344in}}{\pgfqpoint{1.345138in}{3.141943in}}{\pgfqpoint{1.337324in}{3.149757in}}%
\pgfpathcurveto{\pgfqpoint{1.329511in}{3.157571in}}{\pgfqpoint{1.318912in}{3.161961in}}{\pgfqpoint{1.307861in}{3.161961in}}%
\pgfpathcurveto{\pgfqpoint{1.296811in}{3.161961in}}{\pgfqpoint{1.286212in}{3.157571in}}{\pgfqpoint{1.278399in}{3.149757in}}%
\pgfpathcurveto{\pgfqpoint{1.270585in}{3.141943in}}{\pgfqpoint{1.266195in}{3.131344in}}{\pgfqpoint{1.266195in}{3.120294in}}%
\pgfpathcurveto{\pgfqpoint{1.266195in}{3.109244in}}{\pgfqpoint{1.270585in}{3.098645in}}{\pgfqpoint{1.278399in}{3.090831in}}%
\pgfpathcurveto{\pgfqpoint{1.286212in}{3.083018in}}{\pgfqpoint{1.296811in}{3.078627in}}{\pgfqpoint{1.307861in}{3.078627in}}%
\pgfpathclose%
\pgfusepath{stroke,fill}%
\end{pgfscope}%
\begin{pgfscope}%
\pgfpathrectangle{\pgfqpoint{0.750000in}{0.500000in}}{\pgfqpoint{4.650000in}{3.020000in}}%
\pgfusepath{clip}%
\pgfsetbuttcap%
\pgfsetroundjoin%
\definecolor{currentfill}{rgb}{0.121569,0.466667,0.705882}%
\pgfsetfillcolor{currentfill}%
\pgfsetlinewidth{1.003750pt}%
\definecolor{currentstroke}{rgb}{0.121569,0.466667,0.705882}%
\pgfsetstrokecolor{currentstroke}%
\pgfsetdash{}{0pt}%
\pgfpathmoveto{\pgfqpoint{2.416654in}{3.014029in}}%
\pgfpathcurveto{\pgfqpoint{2.427704in}{3.014029in}}{\pgfqpoint{2.438303in}{3.018419in}}{\pgfqpoint{2.446117in}{3.026232in}}%
\pgfpathcurveto{\pgfqpoint{2.453931in}{3.034046in}}{\pgfqpoint{2.458321in}{3.044645in}}{\pgfqpoint{2.458321in}{3.055695in}}%
\pgfpathcurveto{\pgfqpoint{2.458321in}{3.066745in}}{\pgfqpoint{2.453931in}{3.077344in}}{\pgfqpoint{2.446117in}{3.085158in}}%
\pgfpathcurveto{\pgfqpoint{2.438303in}{3.092972in}}{\pgfqpoint{2.427704in}{3.097362in}}{\pgfqpoint{2.416654in}{3.097362in}}%
\pgfpathcurveto{\pgfqpoint{2.405604in}{3.097362in}}{\pgfqpoint{2.395005in}{3.092972in}}{\pgfqpoint{2.387191in}{3.085158in}}%
\pgfpathcurveto{\pgfqpoint{2.379378in}{3.077344in}}{\pgfqpoint{2.374988in}{3.066745in}}{\pgfqpoint{2.374988in}{3.055695in}}%
\pgfpathcurveto{\pgfqpoint{2.374988in}{3.044645in}}{\pgfqpoint{2.379378in}{3.034046in}}{\pgfqpoint{2.387191in}{3.026232in}}%
\pgfpathcurveto{\pgfqpoint{2.395005in}{3.018419in}}{\pgfqpoint{2.405604in}{3.014029in}}{\pgfqpoint{2.416654in}{3.014029in}}%
\pgfpathclose%
\pgfusepath{stroke,fill}%
\end{pgfscope}%
\begin{pgfscope}%
\pgfpathrectangle{\pgfqpoint{0.750000in}{0.500000in}}{\pgfqpoint{4.650000in}{3.020000in}}%
\pgfusepath{clip}%
\pgfsetbuttcap%
\pgfsetroundjoin%
\definecolor{currentfill}{rgb}{0.121569,0.466667,0.705882}%
\pgfsetfillcolor{currentfill}%
\pgfsetlinewidth{1.003750pt}%
\definecolor{currentstroke}{rgb}{0.121569,0.466667,0.705882}%
\pgfsetstrokecolor{currentstroke}%
\pgfsetdash{}{0pt}%
\pgfpathmoveto{\pgfqpoint{2.624553in}{3.005954in}}%
\pgfpathcurveto{\pgfqpoint{2.635603in}{3.005954in}}{\pgfqpoint{2.646202in}{3.010344in}}{\pgfqpoint{2.654016in}{3.018158in}}%
\pgfpathcurveto{\pgfqpoint{2.661829in}{3.025971in}}{\pgfqpoint{2.666220in}{3.036570in}}{\pgfqpoint{2.666220in}{3.047620in}}%
\pgfpathcurveto{\pgfqpoint{2.666220in}{3.058670in}}{\pgfqpoint{2.661829in}{3.069269in}}{\pgfqpoint{2.654016in}{3.077083in}}%
\pgfpathcurveto{\pgfqpoint{2.646202in}{3.084897in}}{\pgfqpoint{2.635603in}{3.089287in}}{\pgfqpoint{2.624553in}{3.089287in}}%
\pgfpathcurveto{\pgfqpoint{2.613503in}{3.089287in}}{\pgfqpoint{2.602904in}{3.084897in}}{\pgfqpoint{2.595090in}{3.077083in}}%
\pgfpathcurveto{\pgfqpoint{2.587277in}{3.069269in}}{\pgfqpoint{2.582886in}{3.058670in}}{\pgfqpoint{2.582886in}{3.047620in}}%
\pgfpathcurveto{\pgfqpoint{2.582886in}{3.036570in}}{\pgfqpoint{2.587277in}{3.025971in}}{\pgfqpoint{2.595090in}{3.018158in}}%
\pgfpathcurveto{\pgfqpoint{2.602904in}{3.010344in}}{\pgfqpoint{2.613503in}{3.005954in}}{\pgfqpoint{2.624553in}{3.005954in}}%
\pgfpathclose%
\pgfusepath{stroke,fill}%
\end{pgfscope}%
\begin{pgfscope}%
\pgfpathrectangle{\pgfqpoint{0.750000in}{0.500000in}}{\pgfqpoint{4.650000in}{3.020000in}}%
\pgfusepath{clip}%
\pgfsetbuttcap%
\pgfsetroundjoin%
\definecolor{currentfill}{rgb}{0.121569,0.466667,0.705882}%
\pgfsetfillcolor{currentfill}%
\pgfsetlinewidth{1.003750pt}%
\definecolor{currentstroke}{rgb}{0.121569,0.466667,0.705882}%
\pgfsetstrokecolor{currentstroke}%
\pgfsetdash{}{0pt}%
\pgfpathmoveto{\pgfqpoint{2.485954in}{3.014029in}}%
\pgfpathcurveto{\pgfqpoint{2.497004in}{3.014029in}}{\pgfqpoint{2.507603in}{3.018419in}}{\pgfqpoint{2.515417in}{3.026232in}}%
\pgfpathcurveto{\pgfqpoint{2.523230in}{3.034046in}}{\pgfqpoint{2.527620in}{3.044645in}}{\pgfqpoint{2.527620in}{3.055695in}}%
\pgfpathcurveto{\pgfqpoint{2.527620in}{3.066745in}}{\pgfqpoint{2.523230in}{3.077344in}}{\pgfqpoint{2.515417in}{3.085158in}}%
\pgfpathcurveto{\pgfqpoint{2.507603in}{3.092972in}}{\pgfqpoint{2.497004in}{3.097362in}}{\pgfqpoint{2.485954in}{3.097362in}}%
\pgfpathcurveto{\pgfqpoint{2.474904in}{3.097362in}}{\pgfqpoint{2.464305in}{3.092972in}}{\pgfqpoint{2.456491in}{3.085158in}}%
\pgfpathcurveto{\pgfqpoint{2.448677in}{3.077344in}}{\pgfqpoint{2.444287in}{3.066745in}}{\pgfqpoint{2.444287in}{3.055695in}}%
\pgfpathcurveto{\pgfqpoint{2.444287in}{3.044645in}}{\pgfqpoint{2.448677in}{3.034046in}}{\pgfqpoint{2.456491in}{3.026232in}}%
\pgfpathcurveto{\pgfqpoint{2.464305in}{3.018419in}}{\pgfqpoint{2.474904in}{3.014029in}}{\pgfqpoint{2.485954in}{3.014029in}}%
\pgfpathclose%
\pgfusepath{stroke,fill}%
\end{pgfscope}%
\begin{pgfscope}%
\pgfpathrectangle{\pgfqpoint{0.750000in}{0.500000in}}{\pgfqpoint{4.650000in}{3.020000in}}%
\pgfusepath{clip}%
\pgfsetbuttcap%
\pgfsetroundjoin%
\definecolor{currentfill}{rgb}{0.121569,0.466667,0.705882}%
\pgfsetfillcolor{currentfill}%
\pgfsetlinewidth{1.003750pt}%
\definecolor{currentstroke}{rgb}{0.121569,0.466667,0.705882}%
\pgfsetstrokecolor{currentstroke}%
\pgfsetdash{}{0pt}%
\pgfpathmoveto{\pgfqpoint{2.763152in}{3.014029in}}%
\pgfpathcurveto{\pgfqpoint{2.774202in}{3.014029in}}{\pgfqpoint{2.784801in}{3.018419in}}{\pgfqpoint{2.792615in}{3.026232in}}%
\pgfpathcurveto{\pgfqpoint{2.800428in}{3.034046in}}{\pgfqpoint{2.804819in}{3.044645in}}{\pgfqpoint{2.804819in}{3.055695in}}%
\pgfpathcurveto{\pgfqpoint{2.804819in}{3.066745in}}{\pgfqpoint{2.800428in}{3.077344in}}{\pgfqpoint{2.792615in}{3.085158in}}%
\pgfpathcurveto{\pgfqpoint{2.784801in}{3.092972in}}{\pgfqpoint{2.774202in}{3.097362in}}{\pgfqpoint{2.763152in}{3.097362in}}%
\pgfpathcurveto{\pgfqpoint{2.752102in}{3.097362in}}{\pgfqpoint{2.741503in}{3.092972in}}{\pgfqpoint{2.733689in}{3.085158in}}%
\pgfpathcurveto{\pgfqpoint{2.725876in}{3.077344in}}{\pgfqpoint{2.721485in}{3.066745in}}{\pgfqpoint{2.721485in}{3.055695in}}%
\pgfpathcurveto{\pgfqpoint{2.721485in}{3.044645in}}{\pgfqpoint{2.725876in}{3.034046in}}{\pgfqpoint{2.733689in}{3.026232in}}%
\pgfpathcurveto{\pgfqpoint{2.741503in}{3.018419in}}{\pgfqpoint{2.752102in}{3.014029in}}{\pgfqpoint{2.763152in}{3.014029in}}%
\pgfpathclose%
\pgfusepath{stroke,fill}%
\end{pgfscope}%
\begin{pgfscope}%
\pgfpathrectangle{\pgfqpoint{0.750000in}{0.500000in}}{\pgfqpoint{4.650000in}{3.020000in}}%
\pgfusepath{clip}%
\pgfsetbuttcap%
\pgfsetroundjoin%
\definecolor{currentfill}{rgb}{1.000000,0.498039,0.054902}%
\pgfsetfillcolor{currentfill}%
\pgfsetlinewidth{1.003750pt}%
\definecolor{currentstroke}{rgb}{1.000000,0.498039,0.054902}%
\pgfsetstrokecolor{currentstroke}%
\pgfsetdash{}{0pt}%
\pgfpathmoveto{\pgfqpoint{1.238562in}{3.030178in}}%
\pgfpathcurveto{\pgfqpoint{1.249612in}{3.030178in}}{\pgfqpoint{1.260211in}{3.034569in}}{\pgfqpoint{1.268025in}{3.042382in}}%
\pgfpathcurveto{\pgfqpoint{1.275838in}{3.050196in}}{\pgfqpoint{1.280229in}{3.060795in}}{\pgfqpoint{1.280229in}{3.071845in}}%
\pgfpathcurveto{\pgfqpoint{1.280229in}{3.082895in}}{\pgfqpoint{1.275838in}{3.093494in}}{\pgfqpoint{1.268025in}{3.101308in}}%
\pgfpathcurveto{\pgfqpoint{1.260211in}{3.109121in}}{\pgfqpoint{1.249612in}{3.113512in}}{\pgfqpoint{1.238562in}{3.113512in}}%
\pgfpathcurveto{\pgfqpoint{1.227512in}{3.113512in}}{\pgfqpoint{1.216913in}{3.109121in}}{\pgfqpoint{1.209099in}{3.101308in}}%
\pgfpathcurveto{\pgfqpoint{1.201285in}{3.093494in}}{\pgfqpoint{1.196895in}{3.082895in}}{\pgfqpoint{1.196895in}{3.071845in}}%
\pgfpathcurveto{\pgfqpoint{1.196895in}{3.060795in}}{\pgfqpoint{1.201285in}{3.050196in}}{\pgfqpoint{1.209099in}{3.042382in}}%
\pgfpathcurveto{\pgfqpoint{1.216913in}{3.034569in}}{\pgfqpoint{1.227512in}{3.030178in}}{\pgfqpoint{1.238562in}{3.030178in}}%
\pgfpathclose%
\pgfusepath{stroke,fill}%
\end{pgfscope}%
\begin{pgfscope}%
\pgfpathrectangle{\pgfqpoint{0.750000in}{0.500000in}}{\pgfqpoint{4.650000in}{3.020000in}}%
\pgfusepath{clip}%
\pgfsetbuttcap%
\pgfsetroundjoin%
\definecolor{currentfill}{rgb}{1.000000,0.498039,0.054902}%
\pgfsetfillcolor{currentfill}%
\pgfsetlinewidth{1.003750pt}%
\definecolor{currentstroke}{rgb}{1.000000,0.498039,0.054902}%
\pgfsetstrokecolor{currentstroke}%
\pgfsetdash{}{0pt}%
\pgfpathmoveto{\pgfqpoint{1.030663in}{3.018066in}}%
\pgfpathcurveto{\pgfqpoint{1.041713in}{3.018066in}}{\pgfqpoint{1.052312in}{3.022456in}}{\pgfqpoint{1.060126in}{3.030270in}}%
\pgfpathcurveto{\pgfqpoint{1.067940in}{3.038083in}}{\pgfqpoint{1.072330in}{3.048682in}}{\pgfqpoint{1.072330in}{3.059733in}}%
\pgfpathcurveto{\pgfqpoint{1.072330in}{3.070783in}}{\pgfqpoint{1.067940in}{3.081382in}}{\pgfqpoint{1.060126in}{3.089195in}}%
\pgfpathcurveto{\pgfqpoint{1.052312in}{3.097009in}}{\pgfqpoint{1.041713in}{3.101399in}}{\pgfqpoint{1.030663in}{3.101399in}}%
\pgfpathcurveto{\pgfqpoint{1.019613in}{3.101399in}}{\pgfqpoint{1.009014in}{3.097009in}}{\pgfqpoint{1.001200in}{3.089195in}}%
\pgfpathcurveto{\pgfqpoint{0.993387in}{3.081382in}}{\pgfqpoint{0.988997in}{3.070783in}}{\pgfqpoint{0.988997in}{3.059733in}}%
\pgfpathcurveto{\pgfqpoint{0.988997in}{3.048682in}}{\pgfqpoint{0.993387in}{3.038083in}}{\pgfqpoint{1.001200in}{3.030270in}}%
\pgfpathcurveto{\pgfqpoint{1.009014in}{3.022456in}}{\pgfqpoint{1.019613in}{3.018066in}}{\pgfqpoint{1.030663in}{3.018066in}}%
\pgfpathclose%
\pgfusepath{stroke,fill}%
\end{pgfscope}%
\begin{pgfscope}%
\pgfpathrectangle{\pgfqpoint{0.750000in}{0.500000in}}{\pgfqpoint{4.650000in}{3.020000in}}%
\pgfusepath{clip}%
\pgfsetbuttcap%
\pgfsetroundjoin%
\definecolor{currentfill}{rgb}{0.121569,0.466667,0.705882}%
\pgfsetfillcolor{currentfill}%
\pgfsetlinewidth{1.003750pt}%
\definecolor{currentstroke}{rgb}{0.121569,0.466667,0.705882}%
\pgfsetstrokecolor{currentstroke}%
\pgfsetdash{}{0pt}%
\pgfpathmoveto{\pgfqpoint{5.188636in}{2.989804in}}%
\pgfpathcurveto{\pgfqpoint{5.199686in}{2.989804in}}{\pgfqpoint{5.210286in}{2.994194in}}{\pgfqpoint{5.218099in}{3.002008in}}%
\pgfpathcurveto{\pgfqpoint{5.225913in}{3.009821in}}{\pgfqpoint{5.230303in}{3.020420in}}{\pgfqpoint{5.230303in}{3.031471in}}%
\pgfpathcurveto{\pgfqpoint{5.230303in}{3.042521in}}{\pgfqpoint{5.225913in}{3.053120in}}{\pgfqpoint{5.218099in}{3.060933in}}%
\pgfpathcurveto{\pgfqpoint{5.210286in}{3.068747in}}{\pgfqpoint{5.199686in}{3.073137in}}{\pgfqpoint{5.188636in}{3.073137in}}%
\pgfpathcurveto{\pgfqpoint{5.177586in}{3.073137in}}{\pgfqpoint{5.166987in}{3.068747in}}{\pgfqpoint{5.159174in}{3.060933in}}%
\pgfpathcurveto{\pgfqpoint{5.151360in}{3.053120in}}{\pgfqpoint{5.146970in}{3.042521in}}{\pgfqpoint{5.146970in}{3.031471in}}%
\pgfpathcurveto{\pgfqpoint{5.146970in}{3.020420in}}{\pgfqpoint{5.151360in}{3.009821in}}{\pgfqpoint{5.159174in}{3.002008in}}%
\pgfpathcurveto{\pgfqpoint{5.166987in}{2.994194in}}{\pgfqpoint{5.177586in}{2.989804in}}{\pgfqpoint{5.188636in}{2.989804in}}%
\pgfpathclose%
\pgfusepath{stroke,fill}%
\end{pgfscope}%
\begin{pgfscope}%
\pgfpathrectangle{\pgfqpoint{0.750000in}{0.500000in}}{\pgfqpoint{4.650000in}{3.020000in}}%
\pgfusepath{clip}%
\pgfsetbuttcap%
\pgfsetroundjoin%
\definecolor{currentfill}{rgb}{1.000000,0.498039,0.054902}%
\pgfsetfillcolor{currentfill}%
\pgfsetlinewidth{1.003750pt}%
\definecolor{currentstroke}{rgb}{1.000000,0.498039,0.054902}%
\pgfsetstrokecolor{currentstroke}%
\pgfsetdash{}{0pt}%
\pgfpathmoveto{\pgfqpoint{2.555253in}{3.070553in}}%
\pgfpathcurveto{\pgfqpoint{2.566303in}{3.070553in}}{\pgfqpoint{2.576903in}{3.074943in}}{\pgfqpoint{2.584716in}{3.082756in}}%
\pgfpathcurveto{\pgfqpoint{2.592530in}{3.090570in}}{\pgfqpoint{2.596920in}{3.101169in}}{\pgfqpoint{2.596920in}{3.112219in}}%
\pgfpathcurveto{\pgfqpoint{2.596920in}{3.123269in}}{\pgfqpoint{2.592530in}{3.133868in}}{\pgfqpoint{2.584716in}{3.141682in}}%
\pgfpathcurveto{\pgfqpoint{2.576903in}{3.149496in}}{\pgfqpoint{2.566303in}{3.153886in}}{\pgfqpoint{2.555253in}{3.153886in}}%
\pgfpathcurveto{\pgfqpoint{2.544203in}{3.153886in}}{\pgfqpoint{2.533604in}{3.149496in}}{\pgfqpoint{2.525791in}{3.141682in}}%
\pgfpathcurveto{\pgfqpoint{2.517977in}{3.133868in}}{\pgfqpoint{2.513587in}{3.123269in}}{\pgfqpoint{2.513587in}{3.112219in}}%
\pgfpathcurveto{\pgfqpoint{2.513587in}{3.101169in}}{\pgfqpoint{2.517977in}{3.090570in}}{\pgfqpoint{2.525791in}{3.082756in}}%
\pgfpathcurveto{\pgfqpoint{2.533604in}{3.074943in}}{\pgfqpoint{2.544203in}{3.070553in}}{\pgfqpoint{2.555253in}{3.070553in}}%
\pgfpathclose%
\pgfusepath{stroke,fill}%
\end{pgfscope}%
\begin{pgfscope}%
\pgfpathrectangle{\pgfqpoint{0.750000in}{0.500000in}}{\pgfqpoint{4.650000in}{3.020000in}}%
\pgfusepath{clip}%
\pgfsetbuttcap%
\pgfsetroundjoin%
\definecolor{currentfill}{rgb}{0.121569,0.466667,0.705882}%
\pgfsetfillcolor{currentfill}%
\pgfsetlinewidth{1.003750pt}%
\definecolor{currentstroke}{rgb}{0.121569,0.466667,0.705882}%
\pgfsetstrokecolor{currentstroke}%
\pgfsetdash{}{0pt}%
\pgfpathmoveto{\pgfqpoint{1.377161in}{0.611756in}}%
\pgfpathcurveto{\pgfqpoint{1.388211in}{0.611756in}}{\pgfqpoint{1.398810in}{0.616146in}}{\pgfqpoint{1.406624in}{0.623960in}}%
\pgfpathcurveto{\pgfqpoint{1.414437in}{0.631773in}}{\pgfqpoint{1.418828in}{0.642372in}}{\pgfqpoint{1.418828in}{0.653422in}}%
\pgfpathcurveto{\pgfqpoint{1.418828in}{0.664473in}}{\pgfqpoint{1.414437in}{0.675072in}}{\pgfqpoint{1.406624in}{0.682885in}}%
\pgfpathcurveto{\pgfqpoint{1.398810in}{0.690699in}}{\pgfqpoint{1.388211in}{0.695089in}}{\pgfqpoint{1.377161in}{0.695089in}}%
\pgfpathcurveto{\pgfqpoint{1.366111in}{0.695089in}}{\pgfqpoint{1.355512in}{0.690699in}}{\pgfqpoint{1.347698in}{0.682885in}}%
\pgfpathcurveto{\pgfqpoint{1.339885in}{0.675072in}}{\pgfqpoint{1.335494in}{0.664473in}}{\pgfqpoint{1.335494in}{0.653422in}}%
\pgfpathcurveto{\pgfqpoint{1.335494in}{0.642372in}}{\pgfqpoint{1.339885in}{0.631773in}}{\pgfqpoint{1.347698in}{0.623960in}}%
\pgfpathcurveto{\pgfqpoint{1.355512in}{0.616146in}}{\pgfqpoint{1.366111in}{0.611756in}}{\pgfqpoint{1.377161in}{0.611756in}}%
\pgfpathclose%
\pgfusepath{stroke,fill}%
\end{pgfscope}%
\begin{pgfscope}%
\pgfpathrectangle{\pgfqpoint{0.750000in}{0.500000in}}{\pgfqpoint{4.650000in}{3.020000in}}%
\pgfusepath{clip}%
\pgfsetbuttcap%
\pgfsetroundjoin%
\definecolor{currentfill}{rgb}{1.000000,0.498039,0.054902}%
\pgfsetfillcolor{currentfill}%
\pgfsetlinewidth{1.003750pt}%
\definecolor{currentstroke}{rgb}{1.000000,0.498039,0.054902}%
\pgfsetstrokecolor{currentstroke}%
\pgfsetdash{}{0pt}%
\pgfpathmoveto{\pgfqpoint{1.862258in}{3.026141in}}%
\pgfpathcurveto{\pgfqpoint{1.873308in}{3.026141in}}{\pgfqpoint{1.883907in}{3.030531in}}{\pgfqpoint{1.891721in}{3.038345in}}%
\pgfpathcurveto{\pgfqpoint{1.899534in}{3.046158in}}{\pgfqpoint{1.903924in}{3.056757in}}{\pgfqpoint{1.903924in}{3.067807in}}%
\pgfpathcurveto{\pgfqpoint{1.903924in}{3.078858in}}{\pgfqpoint{1.899534in}{3.089457in}}{\pgfqpoint{1.891721in}{3.097270in}}%
\pgfpathcurveto{\pgfqpoint{1.883907in}{3.105084in}}{\pgfqpoint{1.873308in}{3.109474in}}{\pgfqpoint{1.862258in}{3.109474in}}%
\pgfpathcurveto{\pgfqpoint{1.851208in}{3.109474in}}{\pgfqpoint{1.840609in}{3.105084in}}{\pgfqpoint{1.832795in}{3.097270in}}%
\pgfpathcurveto{\pgfqpoint{1.824981in}{3.089457in}}{\pgfqpoint{1.820591in}{3.078858in}}{\pgfqpoint{1.820591in}{3.067807in}}%
\pgfpathcurveto{\pgfqpoint{1.820591in}{3.056757in}}{\pgfqpoint{1.824981in}{3.046158in}}{\pgfqpoint{1.832795in}{3.038345in}}%
\pgfpathcurveto{\pgfqpoint{1.840609in}{3.030531in}}{\pgfqpoint{1.851208in}{3.026141in}}{\pgfqpoint{1.862258in}{3.026141in}}%
\pgfpathclose%
\pgfusepath{stroke,fill}%
\end{pgfscope}%
\begin{pgfscope}%
\pgfpathrectangle{\pgfqpoint{0.750000in}{0.500000in}}{\pgfqpoint{4.650000in}{3.020000in}}%
\pgfusepath{clip}%
\pgfsetbuttcap%
\pgfsetroundjoin%
\definecolor{currentfill}{rgb}{0.121569,0.466667,0.705882}%
\pgfsetfillcolor{currentfill}%
\pgfsetlinewidth{1.003750pt}%
\definecolor{currentstroke}{rgb}{0.121569,0.466667,0.705882}%
\pgfsetstrokecolor{currentstroke}%
\pgfsetdash{}{0pt}%
\pgfpathmoveto{\pgfqpoint{1.238562in}{0.595606in}}%
\pgfpathcurveto{\pgfqpoint{1.249612in}{0.595606in}}{\pgfqpoint{1.260211in}{0.599996in}}{\pgfqpoint{1.268025in}{0.607810in}}%
\pgfpathcurveto{\pgfqpoint{1.275838in}{0.615624in}}{\pgfqpoint{1.280229in}{0.626223in}}{\pgfqpoint{1.280229in}{0.637273in}}%
\pgfpathcurveto{\pgfqpoint{1.280229in}{0.648323in}}{\pgfqpoint{1.275838in}{0.658922in}}{\pgfqpoint{1.268025in}{0.666736in}}%
\pgfpathcurveto{\pgfqpoint{1.260211in}{0.674549in}}{\pgfqpoint{1.249612in}{0.678939in}}{\pgfqpoint{1.238562in}{0.678939in}}%
\pgfpathcurveto{\pgfqpoint{1.227512in}{0.678939in}}{\pgfqpoint{1.216913in}{0.674549in}}{\pgfqpoint{1.209099in}{0.666736in}}%
\pgfpathcurveto{\pgfqpoint{1.201285in}{0.658922in}}{\pgfqpoint{1.196895in}{0.648323in}}{\pgfqpoint{1.196895in}{0.637273in}}%
\pgfpathcurveto{\pgfqpoint{1.196895in}{0.626223in}}{\pgfqpoint{1.201285in}{0.615624in}}{\pgfqpoint{1.209099in}{0.607810in}}%
\pgfpathcurveto{\pgfqpoint{1.216913in}{0.599996in}}{\pgfqpoint{1.227512in}{0.595606in}}{\pgfqpoint{1.238562in}{0.595606in}}%
\pgfpathclose%
\pgfusepath{stroke,fill}%
\end{pgfscope}%
\begin{pgfscope}%
\pgfpathrectangle{\pgfqpoint{0.750000in}{0.500000in}}{\pgfqpoint{4.650000in}{3.020000in}}%
\pgfusepath{clip}%
\pgfsetbuttcap%
\pgfsetroundjoin%
\definecolor{currentfill}{rgb}{1.000000,0.498039,0.054902}%
\pgfsetfillcolor{currentfill}%
\pgfsetlinewidth{1.003750pt}%
\definecolor{currentstroke}{rgb}{1.000000,0.498039,0.054902}%
\pgfsetstrokecolor{currentstroke}%
\pgfsetdash{}{0pt}%
\pgfpathmoveto{\pgfqpoint{3.109650in}{3.030178in}}%
\pgfpathcurveto{\pgfqpoint{3.120700in}{3.030178in}}{\pgfqpoint{3.131299in}{3.034569in}}{\pgfqpoint{3.139113in}{3.042382in}}%
\pgfpathcurveto{\pgfqpoint{3.146926in}{3.050196in}}{\pgfqpoint{3.151316in}{3.060795in}}{\pgfqpoint{3.151316in}{3.071845in}}%
\pgfpathcurveto{\pgfqpoint{3.151316in}{3.082895in}}{\pgfqpoint{3.146926in}{3.093494in}}{\pgfqpoint{3.139113in}{3.101308in}}%
\pgfpathcurveto{\pgfqpoint{3.131299in}{3.109121in}}{\pgfqpoint{3.120700in}{3.113512in}}{\pgfqpoint{3.109650in}{3.113512in}}%
\pgfpathcurveto{\pgfqpoint{3.098600in}{3.113512in}}{\pgfqpoint{3.088001in}{3.109121in}}{\pgfqpoint{3.080187in}{3.101308in}}%
\pgfpathcurveto{\pgfqpoint{3.072373in}{3.093494in}}{\pgfqpoint{3.067983in}{3.082895in}}{\pgfqpoint{3.067983in}{3.071845in}}%
\pgfpathcurveto{\pgfqpoint{3.067983in}{3.060795in}}{\pgfqpoint{3.072373in}{3.050196in}}{\pgfqpoint{3.080187in}{3.042382in}}%
\pgfpathcurveto{\pgfqpoint{3.088001in}{3.034569in}}{\pgfqpoint{3.098600in}{3.030178in}}{\pgfqpoint{3.109650in}{3.030178in}}%
\pgfpathclose%
\pgfusepath{stroke,fill}%
\end{pgfscope}%
\begin{pgfscope}%
\pgfpathrectangle{\pgfqpoint{0.750000in}{0.500000in}}{\pgfqpoint{4.650000in}{3.020000in}}%
\pgfusepath{clip}%
\pgfsetbuttcap%
\pgfsetroundjoin%
\definecolor{currentfill}{rgb}{0.121569,0.466667,0.705882}%
\pgfsetfillcolor{currentfill}%
\pgfsetlinewidth{1.003750pt}%
\definecolor{currentstroke}{rgb}{0.121569,0.466667,0.705882}%
\pgfsetstrokecolor{currentstroke}%
\pgfsetdash{}{0pt}%
\pgfpathmoveto{\pgfqpoint{1.238562in}{0.595606in}}%
\pgfpathcurveto{\pgfqpoint{1.249612in}{0.595606in}}{\pgfqpoint{1.260211in}{0.599996in}}{\pgfqpoint{1.268025in}{0.607810in}}%
\pgfpathcurveto{\pgfqpoint{1.275838in}{0.615624in}}{\pgfqpoint{1.280229in}{0.626223in}}{\pgfqpoint{1.280229in}{0.637273in}}%
\pgfpathcurveto{\pgfqpoint{1.280229in}{0.648323in}}{\pgfqpoint{1.275838in}{0.658922in}}{\pgfqpoint{1.268025in}{0.666736in}}%
\pgfpathcurveto{\pgfqpoint{1.260211in}{0.674549in}}{\pgfqpoint{1.249612in}{0.678939in}}{\pgfqpoint{1.238562in}{0.678939in}}%
\pgfpathcurveto{\pgfqpoint{1.227512in}{0.678939in}}{\pgfqpoint{1.216913in}{0.674549in}}{\pgfqpoint{1.209099in}{0.666736in}}%
\pgfpathcurveto{\pgfqpoint{1.201285in}{0.658922in}}{\pgfqpoint{1.196895in}{0.648323in}}{\pgfqpoint{1.196895in}{0.637273in}}%
\pgfpathcurveto{\pgfqpoint{1.196895in}{0.626223in}}{\pgfqpoint{1.201285in}{0.615624in}}{\pgfqpoint{1.209099in}{0.607810in}}%
\pgfpathcurveto{\pgfqpoint{1.216913in}{0.599996in}}{\pgfqpoint{1.227512in}{0.595606in}}{\pgfqpoint{1.238562in}{0.595606in}}%
\pgfpathclose%
\pgfusepath{stroke,fill}%
\end{pgfscope}%
\begin{pgfscope}%
\pgfpathrectangle{\pgfqpoint{0.750000in}{0.500000in}}{\pgfqpoint{4.650000in}{3.020000in}}%
\pgfusepath{clip}%
\pgfsetbuttcap%
\pgfsetroundjoin%
\definecolor{currentfill}{rgb}{0.121569,0.466667,0.705882}%
\pgfsetfillcolor{currentfill}%
\pgfsetlinewidth{1.003750pt}%
\definecolor{currentstroke}{rgb}{0.121569,0.466667,0.705882}%
\pgfsetstrokecolor{currentstroke}%
\pgfsetdash{}{0pt}%
\pgfpathmoveto{\pgfqpoint{1.030663in}{3.005954in}}%
\pgfpathcurveto{\pgfqpoint{1.041713in}{3.005954in}}{\pgfqpoint{1.052312in}{3.010344in}}{\pgfqpoint{1.060126in}{3.018158in}}%
\pgfpathcurveto{\pgfqpoint{1.067940in}{3.025971in}}{\pgfqpoint{1.072330in}{3.036570in}}{\pgfqpoint{1.072330in}{3.047620in}}%
\pgfpathcurveto{\pgfqpoint{1.072330in}{3.058670in}}{\pgfqpoint{1.067940in}{3.069269in}}{\pgfqpoint{1.060126in}{3.077083in}}%
\pgfpathcurveto{\pgfqpoint{1.052312in}{3.084897in}}{\pgfqpoint{1.041713in}{3.089287in}}{\pgfqpoint{1.030663in}{3.089287in}}%
\pgfpathcurveto{\pgfqpoint{1.019613in}{3.089287in}}{\pgfqpoint{1.009014in}{3.084897in}}{\pgfqpoint{1.001200in}{3.077083in}}%
\pgfpathcurveto{\pgfqpoint{0.993387in}{3.069269in}}{\pgfqpoint{0.988997in}{3.058670in}}{\pgfqpoint{0.988997in}{3.047620in}}%
\pgfpathcurveto{\pgfqpoint{0.988997in}{3.036570in}}{\pgfqpoint{0.993387in}{3.025971in}}{\pgfqpoint{1.001200in}{3.018158in}}%
\pgfpathcurveto{\pgfqpoint{1.009014in}{3.010344in}}{\pgfqpoint{1.019613in}{3.005954in}}{\pgfqpoint{1.030663in}{3.005954in}}%
\pgfpathclose%
\pgfusepath{stroke,fill}%
\end{pgfscope}%
\begin{pgfscope}%
\pgfpathrectangle{\pgfqpoint{0.750000in}{0.500000in}}{\pgfqpoint{4.650000in}{3.020000in}}%
\pgfusepath{clip}%
\pgfsetbuttcap%
\pgfsetroundjoin%
\definecolor{currentfill}{rgb}{1.000000,0.498039,0.054902}%
\pgfsetfillcolor{currentfill}%
\pgfsetlinewidth{1.003750pt}%
\definecolor{currentstroke}{rgb}{1.000000,0.498039,0.054902}%
\pgfsetstrokecolor{currentstroke}%
\pgfsetdash{}{0pt}%
\pgfpathmoveto{\pgfqpoint{1.238562in}{3.018066in}}%
\pgfpathcurveto{\pgfqpoint{1.249612in}{3.018066in}}{\pgfqpoint{1.260211in}{3.022456in}}{\pgfqpoint{1.268025in}{3.030270in}}%
\pgfpathcurveto{\pgfqpoint{1.275838in}{3.038083in}}{\pgfqpoint{1.280229in}{3.048682in}}{\pgfqpoint{1.280229in}{3.059733in}}%
\pgfpathcurveto{\pgfqpoint{1.280229in}{3.070783in}}{\pgfqpoint{1.275838in}{3.081382in}}{\pgfqpoint{1.268025in}{3.089195in}}%
\pgfpathcurveto{\pgfqpoint{1.260211in}{3.097009in}}{\pgfqpoint{1.249612in}{3.101399in}}{\pgfqpoint{1.238562in}{3.101399in}}%
\pgfpathcurveto{\pgfqpoint{1.227512in}{3.101399in}}{\pgfqpoint{1.216913in}{3.097009in}}{\pgfqpoint{1.209099in}{3.089195in}}%
\pgfpathcurveto{\pgfqpoint{1.201285in}{3.081382in}}{\pgfqpoint{1.196895in}{3.070783in}}{\pgfqpoint{1.196895in}{3.059733in}}%
\pgfpathcurveto{\pgfqpoint{1.196895in}{3.048682in}}{\pgfqpoint{1.201285in}{3.038083in}}{\pgfqpoint{1.209099in}{3.030270in}}%
\pgfpathcurveto{\pgfqpoint{1.216913in}{3.022456in}}{\pgfqpoint{1.227512in}{3.018066in}}{\pgfqpoint{1.238562in}{3.018066in}}%
\pgfpathclose%
\pgfusepath{stroke,fill}%
\end{pgfscope}%
\begin{pgfscope}%
\pgfpathrectangle{\pgfqpoint{0.750000in}{0.500000in}}{\pgfqpoint{4.650000in}{3.020000in}}%
\pgfusepath{clip}%
\pgfsetbuttcap%
\pgfsetroundjoin%
\definecolor{currentfill}{rgb}{1.000000,0.498039,0.054902}%
\pgfsetfillcolor{currentfill}%
\pgfsetlinewidth{1.003750pt}%
\definecolor{currentstroke}{rgb}{1.000000,0.498039,0.054902}%
\pgfsetstrokecolor{currentstroke}%
\pgfsetdash{}{0pt}%
\pgfpathmoveto{\pgfqpoint{1.030663in}{3.022103in}}%
\pgfpathcurveto{\pgfqpoint{1.041713in}{3.022103in}}{\pgfqpoint{1.052312in}{3.026494in}}{\pgfqpoint{1.060126in}{3.034307in}}%
\pgfpathcurveto{\pgfqpoint{1.067940in}{3.042121in}}{\pgfqpoint{1.072330in}{3.052720in}}{\pgfqpoint{1.072330in}{3.063770in}}%
\pgfpathcurveto{\pgfqpoint{1.072330in}{3.074820in}}{\pgfqpoint{1.067940in}{3.085419in}}{\pgfqpoint{1.060126in}{3.093233in}}%
\pgfpathcurveto{\pgfqpoint{1.052312in}{3.101046in}}{\pgfqpoint{1.041713in}{3.105437in}}{\pgfqpoint{1.030663in}{3.105437in}}%
\pgfpathcurveto{\pgfqpoint{1.019613in}{3.105437in}}{\pgfqpoint{1.009014in}{3.101046in}}{\pgfqpoint{1.001200in}{3.093233in}}%
\pgfpathcurveto{\pgfqpoint{0.993387in}{3.085419in}}{\pgfqpoint{0.988997in}{3.074820in}}{\pgfqpoint{0.988997in}{3.063770in}}%
\pgfpathcurveto{\pgfqpoint{0.988997in}{3.052720in}}{\pgfqpoint{0.993387in}{3.042121in}}{\pgfqpoint{1.001200in}{3.034307in}}%
\pgfpathcurveto{\pgfqpoint{1.009014in}{3.026494in}}{\pgfqpoint{1.019613in}{3.022103in}}{\pgfqpoint{1.030663in}{3.022103in}}%
\pgfpathclose%
\pgfusepath{stroke,fill}%
\end{pgfscope}%
\begin{pgfscope}%
\pgfpathrectangle{\pgfqpoint{0.750000in}{0.500000in}}{\pgfqpoint{4.650000in}{3.020000in}}%
\pgfusepath{clip}%
\pgfsetbuttcap%
\pgfsetroundjoin%
\definecolor{currentfill}{rgb}{1.000000,0.498039,0.054902}%
\pgfsetfillcolor{currentfill}%
\pgfsetlinewidth{1.003750pt}%
\definecolor{currentstroke}{rgb}{1.000000,0.498039,0.054902}%
\pgfsetstrokecolor{currentstroke}%
\pgfsetdash{}{0pt}%
\pgfpathmoveto{\pgfqpoint{1.307861in}{3.026141in}}%
\pgfpathcurveto{\pgfqpoint{1.318912in}{3.026141in}}{\pgfqpoint{1.329511in}{3.030531in}}{\pgfqpoint{1.337324in}{3.038345in}}%
\pgfpathcurveto{\pgfqpoint{1.345138in}{3.046158in}}{\pgfqpoint{1.349528in}{3.056757in}}{\pgfqpoint{1.349528in}{3.067807in}}%
\pgfpathcurveto{\pgfqpoint{1.349528in}{3.078858in}}{\pgfqpoint{1.345138in}{3.089457in}}{\pgfqpoint{1.337324in}{3.097270in}}%
\pgfpathcurveto{\pgfqpoint{1.329511in}{3.105084in}}{\pgfqpoint{1.318912in}{3.109474in}}{\pgfqpoint{1.307861in}{3.109474in}}%
\pgfpathcurveto{\pgfqpoint{1.296811in}{3.109474in}}{\pgfqpoint{1.286212in}{3.105084in}}{\pgfqpoint{1.278399in}{3.097270in}}%
\pgfpathcurveto{\pgfqpoint{1.270585in}{3.089457in}}{\pgfqpoint{1.266195in}{3.078858in}}{\pgfqpoint{1.266195in}{3.067807in}}%
\pgfpathcurveto{\pgfqpoint{1.266195in}{3.056757in}}{\pgfqpoint{1.270585in}{3.046158in}}{\pgfqpoint{1.278399in}{3.038345in}}%
\pgfpathcurveto{\pgfqpoint{1.286212in}{3.030531in}}{\pgfqpoint{1.296811in}{3.026141in}}{\pgfqpoint{1.307861in}{3.026141in}}%
\pgfpathclose%
\pgfusepath{stroke,fill}%
\end{pgfscope}%
\begin{pgfscope}%
\pgfpathrectangle{\pgfqpoint{0.750000in}{0.500000in}}{\pgfqpoint{4.650000in}{3.020000in}}%
\pgfusepath{clip}%
\pgfsetbuttcap%
\pgfsetroundjoin%
\definecolor{currentfill}{rgb}{1.000000,0.498039,0.054902}%
\pgfsetfillcolor{currentfill}%
\pgfsetlinewidth{1.003750pt}%
\definecolor{currentstroke}{rgb}{1.000000,0.498039,0.054902}%
\pgfsetstrokecolor{currentstroke}%
\pgfsetdash{}{0pt}%
\pgfpathmoveto{\pgfqpoint{1.862258in}{3.026141in}}%
\pgfpathcurveto{\pgfqpoint{1.873308in}{3.026141in}}{\pgfqpoint{1.883907in}{3.030531in}}{\pgfqpoint{1.891721in}{3.038345in}}%
\pgfpathcurveto{\pgfqpoint{1.899534in}{3.046158in}}{\pgfqpoint{1.903924in}{3.056757in}}{\pgfqpoint{1.903924in}{3.067807in}}%
\pgfpathcurveto{\pgfqpoint{1.903924in}{3.078858in}}{\pgfqpoint{1.899534in}{3.089457in}}{\pgfqpoint{1.891721in}{3.097270in}}%
\pgfpathcurveto{\pgfqpoint{1.883907in}{3.105084in}}{\pgfqpoint{1.873308in}{3.109474in}}{\pgfqpoint{1.862258in}{3.109474in}}%
\pgfpathcurveto{\pgfqpoint{1.851208in}{3.109474in}}{\pgfqpoint{1.840609in}{3.105084in}}{\pgfqpoint{1.832795in}{3.097270in}}%
\pgfpathcurveto{\pgfqpoint{1.824981in}{3.089457in}}{\pgfqpoint{1.820591in}{3.078858in}}{\pgfqpoint{1.820591in}{3.067807in}}%
\pgfpathcurveto{\pgfqpoint{1.820591in}{3.056757in}}{\pgfqpoint{1.824981in}{3.046158in}}{\pgfqpoint{1.832795in}{3.038345in}}%
\pgfpathcurveto{\pgfqpoint{1.840609in}{3.030531in}}{\pgfqpoint{1.851208in}{3.026141in}}{\pgfqpoint{1.862258in}{3.026141in}}%
\pgfpathclose%
\pgfusepath{stroke,fill}%
\end{pgfscope}%
\begin{pgfscope}%
\pgfpathrectangle{\pgfqpoint{0.750000in}{0.500000in}}{\pgfqpoint{4.650000in}{3.020000in}}%
\pgfusepath{clip}%
\pgfsetbuttcap%
\pgfsetroundjoin%
\definecolor{currentfill}{rgb}{0.121569,0.466667,0.705882}%
\pgfsetfillcolor{currentfill}%
\pgfsetlinewidth{1.003750pt}%
\definecolor{currentstroke}{rgb}{0.121569,0.466667,0.705882}%
\pgfsetstrokecolor{currentstroke}%
\pgfsetdash{}{0pt}%
\pgfpathmoveto{\pgfqpoint{1.030663in}{3.014029in}}%
\pgfpathcurveto{\pgfqpoint{1.041713in}{3.014029in}}{\pgfqpoint{1.052312in}{3.018419in}}{\pgfqpoint{1.060126in}{3.026232in}}%
\pgfpathcurveto{\pgfqpoint{1.067940in}{3.034046in}}{\pgfqpoint{1.072330in}{3.044645in}}{\pgfqpoint{1.072330in}{3.055695in}}%
\pgfpathcurveto{\pgfqpoint{1.072330in}{3.066745in}}{\pgfqpoint{1.067940in}{3.077344in}}{\pgfqpoint{1.060126in}{3.085158in}}%
\pgfpathcurveto{\pgfqpoint{1.052312in}{3.092972in}}{\pgfqpoint{1.041713in}{3.097362in}}{\pgfqpoint{1.030663in}{3.097362in}}%
\pgfpathcurveto{\pgfqpoint{1.019613in}{3.097362in}}{\pgfqpoint{1.009014in}{3.092972in}}{\pgfqpoint{1.001200in}{3.085158in}}%
\pgfpathcurveto{\pgfqpoint{0.993387in}{3.077344in}}{\pgfqpoint{0.988997in}{3.066745in}}{\pgfqpoint{0.988997in}{3.055695in}}%
\pgfpathcurveto{\pgfqpoint{0.988997in}{3.044645in}}{\pgfqpoint{0.993387in}{3.034046in}}{\pgfqpoint{1.001200in}{3.026232in}}%
\pgfpathcurveto{\pgfqpoint{1.009014in}{3.018419in}}{\pgfqpoint{1.019613in}{3.014029in}}{\pgfqpoint{1.030663in}{3.014029in}}%
\pgfpathclose%
\pgfusepath{stroke,fill}%
\end{pgfscope}%
\begin{pgfscope}%
\pgfpathrectangle{\pgfqpoint{0.750000in}{0.500000in}}{\pgfqpoint{4.650000in}{3.020000in}}%
\pgfusepath{clip}%
\pgfsetbuttcap%
\pgfsetroundjoin%
\definecolor{currentfill}{rgb}{0.121569,0.466667,0.705882}%
\pgfsetfillcolor{currentfill}%
\pgfsetlinewidth{1.003750pt}%
\definecolor{currentstroke}{rgb}{0.121569,0.466667,0.705882}%
\pgfsetstrokecolor{currentstroke}%
\pgfsetdash{}{0pt}%
\pgfpathmoveto{\pgfqpoint{1.515760in}{0.607718in}}%
\pgfpathcurveto{\pgfqpoint{1.526810in}{0.607718in}}{\pgfqpoint{1.537409in}{0.612109in}}{\pgfqpoint{1.545223in}{0.619922in}}%
\pgfpathcurveto{\pgfqpoint{1.553036in}{0.627736in}}{\pgfqpoint{1.557427in}{0.638335in}}{\pgfqpoint{1.557427in}{0.649385in}}%
\pgfpathcurveto{\pgfqpoint{1.557427in}{0.660435in}}{\pgfqpoint{1.553036in}{0.671034in}}{\pgfqpoint{1.545223in}{0.678848in}}%
\pgfpathcurveto{\pgfqpoint{1.537409in}{0.686661in}}{\pgfqpoint{1.526810in}{0.691052in}}{\pgfqpoint{1.515760in}{0.691052in}}%
\pgfpathcurveto{\pgfqpoint{1.504710in}{0.691052in}}{\pgfqpoint{1.494111in}{0.686661in}}{\pgfqpoint{1.486297in}{0.678848in}}%
\pgfpathcurveto{\pgfqpoint{1.478484in}{0.671034in}}{\pgfqpoint{1.474093in}{0.660435in}}{\pgfqpoint{1.474093in}{0.649385in}}%
\pgfpathcurveto{\pgfqpoint{1.474093in}{0.638335in}}{\pgfqpoint{1.478484in}{0.627736in}}{\pgfqpoint{1.486297in}{0.619922in}}%
\pgfpathcurveto{\pgfqpoint{1.494111in}{0.612109in}}{\pgfqpoint{1.504710in}{0.607718in}}{\pgfqpoint{1.515760in}{0.607718in}}%
\pgfpathclose%
\pgfusepath{stroke,fill}%
\end{pgfscope}%
\begin{pgfscope}%
\pgfpathrectangle{\pgfqpoint{0.750000in}{0.500000in}}{\pgfqpoint{4.650000in}{3.020000in}}%
\pgfusepath{clip}%
\pgfsetbuttcap%
\pgfsetroundjoin%
\definecolor{currentfill}{rgb}{1.000000,0.498039,0.054902}%
\pgfsetfillcolor{currentfill}%
\pgfsetlinewidth{1.003750pt}%
\definecolor{currentstroke}{rgb}{1.000000,0.498039,0.054902}%
\pgfsetstrokecolor{currentstroke}%
\pgfsetdash{}{0pt}%
\pgfpathmoveto{\pgfqpoint{2.208756in}{3.026141in}}%
\pgfpathcurveto{\pgfqpoint{2.219806in}{3.026141in}}{\pgfqpoint{2.230405in}{3.030531in}}{\pgfqpoint{2.238218in}{3.038345in}}%
\pgfpathcurveto{\pgfqpoint{2.246032in}{3.046158in}}{\pgfqpoint{2.250422in}{3.056757in}}{\pgfqpoint{2.250422in}{3.067807in}}%
\pgfpathcurveto{\pgfqpoint{2.250422in}{3.078858in}}{\pgfqpoint{2.246032in}{3.089457in}}{\pgfqpoint{2.238218in}{3.097270in}}%
\pgfpathcurveto{\pgfqpoint{2.230405in}{3.105084in}}{\pgfqpoint{2.219806in}{3.109474in}}{\pgfqpoint{2.208756in}{3.109474in}}%
\pgfpathcurveto{\pgfqpoint{2.197705in}{3.109474in}}{\pgfqpoint{2.187106in}{3.105084in}}{\pgfqpoint{2.179293in}{3.097270in}}%
\pgfpathcurveto{\pgfqpoint{2.171479in}{3.089457in}}{\pgfqpoint{2.167089in}{3.078858in}}{\pgfqpoint{2.167089in}{3.067807in}}%
\pgfpathcurveto{\pgfqpoint{2.167089in}{3.056757in}}{\pgfqpoint{2.171479in}{3.046158in}}{\pgfqpoint{2.179293in}{3.038345in}}%
\pgfpathcurveto{\pgfqpoint{2.187106in}{3.030531in}}{\pgfqpoint{2.197705in}{3.026141in}}{\pgfqpoint{2.208756in}{3.026141in}}%
\pgfpathclose%
\pgfusepath{stroke,fill}%
\end{pgfscope}%
\begin{pgfscope}%
\pgfpathrectangle{\pgfqpoint{0.750000in}{0.500000in}}{\pgfqpoint{4.650000in}{3.020000in}}%
\pgfusepath{clip}%
\pgfsetbuttcap%
\pgfsetroundjoin%
\definecolor{currentfill}{rgb}{0.121569,0.466667,0.705882}%
\pgfsetfillcolor{currentfill}%
\pgfsetlinewidth{1.003750pt}%
\definecolor{currentstroke}{rgb}{0.121569,0.466667,0.705882}%
\pgfsetstrokecolor{currentstroke}%
\pgfsetdash{}{0pt}%
\pgfpathmoveto{\pgfqpoint{3.248249in}{3.014029in}}%
\pgfpathcurveto{\pgfqpoint{3.259299in}{3.014029in}}{\pgfqpoint{3.269898in}{3.018419in}}{\pgfqpoint{3.277712in}{3.026232in}}%
\pgfpathcurveto{\pgfqpoint{3.285525in}{3.034046in}}{\pgfqpoint{3.289916in}{3.044645in}}{\pgfqpoint{3.289916in}{3.055695in}}%
\pgfpathcurveto{\pgfqpoint{3.289916in}{3.066745in}}{\pgfqpoint{3.285525in}{3.077344in}}{\pgfqpoint{3.277712in}{3.085158in}}%
\pgfpathcurveto{\pgfqpoint{3.269898in}{3.092972in}}{\pgfqpoint{3.259299in}{3.097362in}}{\pgfqpoint{3.248249in}{3.097362in}}%
\pgfpathcurveto{\pgfqpoint{3.237199in}{3.097362in}}{\pgfqpoint{3.226600in}{3.092972in}}{\pgfqpoint{3.218786in}{3.085158in}}%
\pgfpathcurveto{\pgfqpoint{3.210972in}{3.077344in}}{\pgfqpoint{3.206582in}{3.066745in}}{\pgfqpoint{3.206582in}{3.055695in}}%
\pgfpathcurveto{\pgfqpoint{3.206582in}{3.044645in}}{\pgfqpoint{3.210972in}{3.034046in}}{\pgfqpoint{3.218786in}{3.026232in}}%
\pgfpathcurveto{\pgfqpoint{3.226600in}{3.018419in}}{\pgfqpoint{3.237199in}{3.014029in}}{\pgfqpoint{3.248249in}{3.014029in}}%
\pgfpathclose%
\pgfusepath{stroke,fill}%
\end{pgfscope}%
\begin{pgfscope}%
\pgfpathrectangle{\pgfqpoint{0.750000in}{0.500000in}}{\pgfqpoint{4.650000in}{3.020000in}}%
\pgfusepath{clip}%
\pgfsetbuttcap%
\pgfsetroundjoin%
\definecolor{currentfill}{rgb}{1.000000,0.498039,0.054902}%
\pgfsetfillcolor{currentfill}%
\pgfsetlinewidth{1.003750pt}%
\definecolor{currentstroke}{rgb}{1.000000,0.498039,0.054902}%
\pgfsetstrokecolor{currentstroke}%
\pgfsetdash{}{0pt}%
\pgfpathmoveto{\pgfqpoint{1.099963in}{3.018066in}}%
\pgfpathcurveto{\pgfqpoint{1.111013in}{3.018066in}}{\pgfqpoint{1.121612in}{3.022456in}}{\pgfqpoint{1.129426in}{3.030270in}}%
\pgfpathcurveto{\pgfqpoint{1.137239in}{3.038083in}}{\pgfqpoint{1.141629in}{3.048682in}}{\pgfqpoint{1.141629in}{3.059733in}}%
\pgfpathcurveto{\pgfqpoint{1.141629in}{3.070783in}}{\pgfqpoint{1.137239in}{3.081382in}}{\pgfqpoint{1.129426in}{3.089195in}}%
\pgfpathcurveto{\pgfqpoint{1.121612in}{3.097009in}}{\pgfqpoint{1.111013in}{3.101399in}}{\pgfqpoint{1.099963in}{3.101399in}}%
\pgfpathcurveto{\pgfqpoint{1.088913in}{3.101399in}}{\pgfqpoint{1.078314in}{3.097009in}}{\pgfqpoint{1.070500in}{3.089195in}}%
\pgfpathcurveto{\pgfqpoint{1.062686in}{3.081382in}}{\pgfqpoint{1.058296in}{3.070783in}}{\pgfqpoint{1.058296in}{3.059733in}}%
\pgfpathcurveto{\pgfqpoint{1.058296in}{3.048682in}}{\pgfqpoint{1.062686in}{3.038083in}}{\pgfqpoint{1.070500in}{3.030270in}}%
\pgfpathcurveto{\pgfqpoint{1.078314in}{3.022456in}}{\pgfqpoint{1.088913in}{3.018066in}}{\pgfqpoint{1.099963in}{3.018066in}}%
\pgfpathclose%
\pgfusepath{stroke,fill}%
\end{pgfscope}%
\begin{pgfscope}%
\pgfpathrectangle{\pgfqpoint{0.750000in}{0.500000in}}{\pgfqpoint{4.650000in}{3.020000in}}%
\pgfusepath{clip}%
\pgfsetbuttcap%
\pgfsetroundjoin%
\definecolor{currentfill}{rgb}{1.000000,0.498039,0.054902}%
\pgfsetfillcolor{currentfill}%
\pgfsetlinewidth{1.003750pt}%
\definecolor{currentstroke}{rgb}{1.000000,0.498039,0.054902}%
\pgfsetstrokecolor{currentstroke}%
\pgfsetdash{}{0pt}%
\pgfpathmoveto{\pgfqpoint{1.307861in}{3.038253in}}%
\pgfpathcurveto{\pgfqpoint{1.318912in}{3.038253in}}{\pgfqpoint{1.329511in}{3.042643in}}{\pgfqpoint{1.337324in}{3.050457in}}%
\pgfpathcurveto{\pgfqpoint{1.345138in}{3.058271in}}{\pgfqpoint{1.349528in}{3.068870in}}{\pgfqpoint{1.349528in}{3.079920in}}%
\pgfpathcurveto{\pgfqpoint{1.349528in}{3.090970in}}{\pgfqpoint{1.345138in}{3.101569in}}{\pgfqpoint{1.337324in}{3.109383in}}%
\pgfpathcurveto{\pgfqpoint{1.329511in}{3.117196in}}{\pgfqpoint{1.318912in}{3.121586in}}{\pgfqpoint{1.307861in}{3.121586in}}%
\pgfpathcurveto{\pgfqpoint{1.296811in}{3.121586in}}{\pgfqpoint{1.286212in}{3.117196in}}{\pgfqpoint{1.278399in}{3.109383in}}%
\pgfpathcurveto{\pgfqpoint{1.270585in}{3.101569in}}{\pgfqpoint{1.266195in}{3.090970in}}{\pgfqpoint{1.266195in}{3.079920in}}%
\pgfpathcurveto{\pgfqpoint{1.266195in}{3.068870in}}{\pgfqpoint{1.270585in}{3.058271in}}{\pgfqpoint{1.278399in}{3.050457in}}%
\pgfpathcurveto{\pgfqpoint{1.286212in}{3.042643in}}{\pgfqpoint{1.296811in}{3.038253in}}{\pgfqpoint{1.307861in}{3.038253in}}%
\pgfpathclose%
\pgfusepath{stroke,fill}%
\end{pgfscope}%
\begin{pgfscope}%
\pgfpathrectangle{\pgfqpoint{0.750000in}{0.500000in}}{\pgfqpoint{4.650000in}{3.020000in}}%
\pgfusepath{clip}%
\pgfsetbuttcap%
\pgfsetroundjoin%
\definecolor{currentfill}{rgb}{1.000000,0.498039,0.054902}%
\pgfsetfillcolor{currentfill}%
\pgfsetlinewidth{1.003750pt}%
\definecolor{currentstroke}{rgb}{1.000000,0.498039,0.054902}%
\pgfsetstrokecolor{currentstroke}%
\pgfsetdash{}{0pt}%
\pgfpathmoveto{\pgfqpoint{2.000857in}{3.022103in}}%
\pgfpathcurveto{\pgfqpoint{2.011907in}{3.022103in}}{\pgfqpoint{2.022506in}{3.026494in}}{\pgfqpoint{2.030320in}{3.034307in}}%
\pgfpathcurveto{\pgfqpoint{2.038133in}{3.042121in}}{\pgfqpoint{2.042524in}{3.052720in}}{\pgfqpoint{2.042524in}{3.063770in}}%
\pgfpathcurveto{\pgfqpoint{2.042524in}{3.074820in}}{\pgfqpoint{2.038133in}{3.085419in}}{\pgfqpoint{2.030320in}{3.093233in}}%
\pgfpathcurveto{\pgfqpoint{2.022506in}{3.101046in}}{\pgfqpoint{2.011907in}{3.105437in}}{\pgfqpoint{2.000857in}{3.105437in}}%
\pgfpathcurveto{\pgfqpoint{1.989807in}{3.105437in}}{\pgfqpoint{1.979208in}{3.101046in}}{\pgfqpoint{1.971394in}{3.093233in}}%
\pgfpathcurveto{\pgfqpoint{1.963581in}{3.085419in}}{\pgfqpoint{1.959190in}{3.074820in}}{\pgfqpoint{1.959190in}{3.063770in}}%
\pgfpathcurveto{\pgfqpoint{1.959190in}{3.052720in}}{\pgfqpoint{1.963581in}{3.042121in}}{\pgfqpoint{1.971394in}{3.034307in}}%
\pgfpathcurveto{\pgfqpoint{1.979208in}{3.026494in}}{\pgfqpoint{1.989807in}{3.022103in}}{\pgfqpoint{2.000857in}{3.022103in}}%
\pgfpathclose%
\pgfusepath{stroke,fill}%
\end{pgfscope}%
\begin{pgfscope}%
\pgfpathrectangle{\pgfqpoint{0.750000in}{0.500000in}}{\pgfqpoint{4.650000in}{3.020000in}}%
\pgfusepath{clip}%
\pgfsetbuttcap%
\pgfsetroundjoin%
\definecolor{currentfill}{rgb}{0.121569,0.466667,0.705882}%
\pgfsetfillcolor{currentfill}%
\pgfsetlinewidth{1.003750pt}%
\definecolor{currentstroke}{rgb}{0.121569,0.466667,0.705882}%
\pgfsetstrokecolor{currentstroke}%
\pgfsetdash{}{0pt}%
\pgfpathmoveto{\pgfqpoint{2.208756in}{0.595606in}}%
\pgfpathcurveto{\pgfqpoint{2.219806in}{0.595606in}}{\pgfqpoint{2.230405in}{0.599996in}}{\pgfqpoint{2.238218in}{0.607810in}}%
\pgfpathcurveto{\pgfqpoint{2.246032in}{0.615624in}}{\pgfqpoint{2.250422in}{0.626223in}}{\pgfqpoint{2.250422in}{0.637273in}}%
\pgfpathcurveto{\pgfqpoint{2.250422in}{0.648323in}}{\pgfqpoint{2.246032in}{0.658922in}}{\pgfqpoint{2.238218in}{0.666736in}}%
\pgfpathcurveto{\pgfqpoint{2.230405in}{0.674549in}}{\pgfqpoint{2.219806in}{0.678939in}}{\pgfqpoint{2.208756in}{0.678939in}}%
\pgfpathcurveto{\pgfqpoint{2.197705in}{0.678939in}}{\pgfqpoint{2.187106in}{0.674549in}}{\pgfqpoint{2.179293in}{0.666736in}}%
\pgfpathcurveto{\pgfqpoint{2.171479in}{0.658922in}}{\pgfqpoint{2.167089in}{0.648323in}}{\pgfqpoint{2.167089in}{0.637273in}}%
\pgfpathcurveto{\pgfqpoint{2.167089in}{0.626223in}}{\pgfqpoint{2.171479in}{0.615624in}}{\pgfqpoint{2.179293in}{0.607810in}}%
\pgfpathcurveto{\pgfqpoint{2.187106in}{0.599996in}}{\pgfqpoint{2.197705in}{0.595606in}}{\pgfqpoint{2.208756in}{0.595606in}}%
\pgfpathclose%
\pgfusepath{stroke,fill}%
\end{pgfscope}%
\begin{pgfscope}%
\pgfpathrectangle{\pgfqpoint{0.750000in}{0.500000in}}{\pgfqpoint{4.650000in}{3.020000in}}%
\pgfusepath{clip}%
\pgfsetbuttcap%
\pgfsetroundjoin%
\definecolor{currentfill}{rgb}{0.121569,0.466667,0.705882}%
\pgfsetfillcolor{currentfill}%
\pgfsetlinewidth{1.003750pt}%
\definecolor{currentstroke}{rgb}{0.121569,0.466667,0.705882}%
\pgfsetstrokecolor{currentstroke}%
\pgfsetdash{}{0pt}%
\pgfpathmoveto{\pgfqpoint{2.070156in}{0.595606in}}%
\pgfpathcurveto{\pgfqpoint{2.081207in}{0.595606in}}{\pgfqpoint{2.091806in}{0.599996in}}{\pgfqpoint{2.099619in}{0.607810in}}%
\pgfpathcurveto{\pgfqpoint{2.107433in}{0.615624in}}{\pgfqpoint{2.111823in}{0.626223in}}{\pgfqpoint{2.111823in}{0.637273in}}%
\pgfpathcurveto{\pgfqpoint{2.111823in}{0.648323in}}{\pgfqpoint{2.107433in}{0.658922in}}{\pgfqpoint{2.099619in}{0.666736in}}%
\pgfpathcurveto{\pgfqpoint{2.091806in}{0.674549in}}{\pgfqpoint{2.081207in}{0.678939in}}{\pgfqpoint{2.070156in}{0.678939in}}%
\pgfpathcurveto{\pgfqpoint{2.059106in}{0.678939in}}{\pgfqpoint{2.048507in}{0.674549in}}{\pgfqpoint{2.040694in}{0.666736in}}%
\pgfpathcurveto{\pgfqpoint{2.032880in}{0.658922in}}{\pgfqpoint{2.028490in}{0.648323in}}{\pgfqpoint{2.028490in}{0.637273in}}%
\pgfpathcurveto{\pgfqpoint{2.028490in}{0.626223in}}{\pgfqpoint{2.032880in}{0.615624in}}{\pgfqpoint{2.040694in}{0.607810in}}%
\pgfpathcurveto{\pgfqpoint{2.048507in}{0.599996in}}{\pgfqpoint{2.059106in}{0.595606in}}{\pgfqpoint{2.070156in}{0.595606in}}%
\pgfpathclose%
\pgfusepath{stroke,fill}%
\end{pgfscope}%
\begin{pgfscope}%
\pgfpathrectangle{\pgfqpoint{0.750000in}{0.500000in}}{\pgfqpoint{4.650000in}{3.020000in}}%
\pgfusepath{clip}%
\pgfsetbuttcap%
\pgfsetroundjoin%
\definecolor{currentfill}{rgb}{0.121569,0.466667,0.705882}%
\pgfsetfillcolor{currentfill}%
\pgfsetlinewidth{1.003750pt}%
\definecolor{currentstroke}{rgb}{0.121569,0.466667,0.705882}%
\pgfsetstrokecolor{currentstroke}%
\pgfsetdash{}{0pt}%
\pgfpathmoveto{\pgfqpoint{1.446461in}{0.644055in}}%
\pgfpathcurveto{\pgfqpoint{1.457511in}{0.644055in}}{\pgfqpoint{1.468110in}{0.648446in}}{\pgfqpoint{1.475923in}{0.656259in}}%
\pgfpathcurveto{\pgfqpoint{1.483737in}{0.664073in}}{\pgfqpoint{1.488127in}{0.674672in}}{\pgfqpoint{1.488127in}{0.685722in}}%
\pgfpathcurveto{\pgfqpoint{1.488127in}{0.696772in}}{\pgfqpoint{1.483737in}{0.707371in}}{\pgfqpoint{1.475923in}{0.715185in}}%
\pgfpathcurveto{\pgfqpoint{1.468110in}{0.722998in}}{\pgfqpoint{1.457511in}{0.727389in}}{\pgfqpoint{1.446461in}{0.727389in}}%
\pgfpathcurveto{\pgfqpoint{1.435410in}{0.727389in}}{\pgfqpoint{1.424811in}{0.722998in}}{\pgfqpoint{1.416998in}{0.715185in}}%
\pgfpathcurveto{\pgfqpoint{1.409184in}{0.707371in}}{\pgfqpoint{1.404794in}{0.696772in}}{\pgfqpoint{1.404794in}{0.685722in}}%
\pgfpathcurveto{\pgfqpoint{1.404794in}{0.674672in}}{\pgfqpoint{1.409184in}{0.664073in}}{\pgfqpoint{1.416998in}{0.656259in}}%
\pgfpathcurveto{\pgfqpoint{1.424811in}{0.648446in}}{\pgfqpoint{1.435410in}{0.644055in}}{\pgfqpoint{1.446461in}{0.644055in}}%
\pgfpathclose%
\pgfusepath{stroke,fill}%
\end{pgfscope}%
\begin{pgfscope}%
\pgfpathrectangle{\pgfqpoint{0.750000in}{0.500000in}}{\pgfqpoint{4.650000in}{3.020000in}}%
\pgfusepath{clip}%
\pgfsetbuttcap%
\pgfsetroundjoin%
\definecolor{currentfill}{rgb}{1.000000,0.498039,0.054902}%
\pgfsetfillcolor{currentfill}%
\pgfsetlinewidth{1.003750pt}%
\definecolor{currentstroke}{rgb}{1.000000,0.498039,0.054902}%
\pgfsetstrokecolor{currentstroke}%
\pgfsetdash{}{0pt}%
\pgfpathmoveto{\pgfqpoint{2.832452in}{3.022103in}}%
\pgfpathcurveto{\pgfqpoint{2.843502in}{3.022103in}}{\pgfqpoint{2.854101in}{3.026494in}}{\pgfqpoint{2.861914in}{3.034307in}}%
\pgfpathcurveto{\pgfqpoint{2.869728in}{3.042121in}}{\pgfqpoint{2.874118in}{3.052720in}}{\pgfqpoint{2.874118in}{3.063770in}}%
\pgfpathcurveto{\pgfqpoint{2.874118in}{3.074820in}}{\pgfqpoint{2.869728in}{3.085419in}}{\pgfqpoint{2.861914in}{3.093233in}}%
\pgfpathcurveto{\pgfqpoint{2.854101in}{3.101046in}}{\pgfqpoint{2.843502in}{3.105437in}}{\pgfqpoint{2.832452in}{3.105437in}}%
\pgfpathcurveto{\pgfqpoint{2.821401in}{3.105437in}}{\pgfqpoint{2.810802in}{3.101046in}}{\pgfqpoint{2.802989in}{3.093233in}}%
\pgfpathcurveto{\pgfqpoint{2.795175in}{3.085419in}}{\pgfqpoint{2.790785in}{3.074820in}}{\pgfqpoint{2.790785in}{3.063770in}}%
\pgfpathcurveto{\pgfqpoint{2.790785in}{3.052720in}}{\pgfqpoint{2.795175in}{3.042121in}}{\pgfqpoint{2.802989in}{3.034307in}}%
\pgfpathcurveto{\pgfqpoint{2.810802in}{3.026494in}}{\pgfqpoint{2.821401in}{3.022103in}}{\pgfqpoint{2.832452in}{3.022103in}}%
\pgfpathclose%
\pgfusepath{stroke,fill}%
\end{pgfscope}%
\begin{pgfscope}%
\pgfpathrectangle{\pgfqpoint{0.750000in}{0.500000in}}{\pgfqpoint{4.650000in}{3.020000in}}%
\pgfusepath{clip}%
\pgfsetbuttcap%
\pgfsetroundjoin%
\definecolor{currentfill}{rgb}{1.000000,0.498039,0.054902}%
\pgfsetfillcolor{currentfill}%
\pgfsetlinewidth{1.003750pt}%
\definecolor{currentstroke}{rgb}{1.000000,0.498039,0.054902}%
\pgfsetstrokecolor{currentstroke}%
\pgfsetdash{}{0pt}%
\pgfpathmoveto{\pgfqpoint{3.317548in}{3.022103in}}%
\pgfpathcurveto{\pgfqpoint{3.328599in}{3.022103in}}{\pgfqpoint{3.339198in}{3.026494in}}{\pgfqpoint{3.347011in}{3.034307in}}%
\pgfpathcurveto{\pgfqpoint{3.354825in}{3.042121in}}{\pgfqpoint{3.359215in}{3.052720in}}{\pgfqpoint{3.359215in}{3.063770in}}%
\pgfpathcurveto{\pgfqpoint{3.359215in}{3.074820in}}{\pgfqpoint{3.354825in}{3.085419in}}{\pgfqpoint{3.347011in}{3.093233in}}%
\pgfpathcurveto{\pgfqpoint{3.339198in}{3.101046in}}{\pgfqpoint{3.328599in}{3.105437in}}{\pgfqpoint{3.317548in}{3.105437in}}%
\pgfpathcurveto{\pgfqpoint{3.306498in}{3.105437in}}{\pgfqpoint{3.295899in}{3.101046in}}{\pgfqpoint{3.288086in}{3.093233in}}%
\pgfpathcurveto{\pgfqpoint{3.280272in}{3.085419in}}{\pgfqpoint{3.275882in}{3.074820in}}{\pgfqpoint{3.275882in}{3.063770in}}%
\pgfpathcurveto{\pgfqpoint{3.275882in}{3.052720in}}{\pgfqpoint{3.280272in}{3.042121in}}{\pgfqpoint{3.288086in}{3.034307in}}%
\pgfpathcurveto{\pgfqpoint{3.295899in}{3.026494in}}{\pgfqpoint{3.306498in}{3.022103in}}{\pgfqpoint{3.317548in}{3.022103in}}%
\pgfpathclose%
\pgfusepath{stroke,fill}%
\end{pgfscope}%
\begin{pgfscope}%
\pgfpathrectangle{\pgfqpoint{0.750000in}{0.500000in}}{\pgfqpoint{4.650000in}{3.020000in}}%
\pgfusepath{clip}%
\pgfsetbuttcap%
\pgfsetroundjoin%
\definecolor{currentfill}{rgb}{0.121569,0.466667,0.705882}%
\pgfsetfillcolor{currentfill}%
\pgfsetlinewidth{1.003750pt}%
\definecolor{currentstroke}{rgb}{0.121569,0.466667,0.705882}%
\pgfsetstrokecolor{currentstroke}%
\pgfsetdash{}{0pt}%
\pgfpathmoveto{\pgfqpoint{1.862258in}{3.014029in}}%
\pgfpathcurveto{\pgfqpoint{1.873308in}{3.014029in}}{\pgfqpoint{1.883907in}{3.018419in}}{\pgfqpoint{1.891721in}{3.026232in}}%
\pgfpathcurveto{\pgfqpoint{1.899534in}{3.034046in}}{\pgfqpoint{1.903924in}{3.044645in}}{\pgfqpoint{1.903924in}{3.055695in}}%
\pgfpathcurveto{\pgfqpoint{1.903924in}{3.066745in}}{\pgfqpoint{1.899534in}{3.077344in}}{\pgfqpoint{1.891721in}{3.085158in}}%
\pgfpathcurveto{\pgfqpoint{1.883907in}{3.092972in}}{\pgfqpoint{1.873308in}{3.097362in}}{\pgfqpoint{1.862258in}{3.097362in}}%
\pgfpathcurveto{\pgfqpoint{1.851208in}{3.097362in}}{\pgfqpoint{1.840609in}{3.092972in}}{\pgfqpoint{1.832795in}{3.085158in}}%
\pgfpathcurveto{\pgfqpoint{1.824981in}{3.077344in}}{\pgfqpoint{1.820591in}{3.066745in}}{\pgfqpoint{1.820591in}{3.055695in}}%
\pgfpathcurveto{\pgfqpoint{1.820591in}{3.044645in}}{\pgfqpoint{1.824981in}{3.034046in}}{\pgfqpoint{1.832795in}{3.026232in}}%
\pgfpathcurveto{\pgfqpoint{1.840609in}{3.018419in}}{\pgfqpoint{1.851208in}{3.014029in}}{\pgfqpoint{1.862258in}{3.014029in}}%
\pgfpathclose%
\pgfusepath{stroke,fill}%
\end{pgfscope}%
\begin{pgfscope}%
\pgfpathrectangle{\pgfqpoint{0.750000in}{0.500000in}}{\pgfqpoint{4.650000in}{3.020000in}}%
\pgfusepath{clip}%
\pgfsetbuttcap%
\pgfsetroundjoin%
\definecolor{currentfill}{rgb}{1.000000,0.498039,0.054902}%
\pgfsetfillcolor{currentfill}%
\pgfsetlinewidth{1.003750pt}%
\definecolor{currentstroke}{rgb}{1.000000,0.498039,0.054902}%
\pgfsetstrokecolor{currentstroke}%
\pgfsetdash{}{0pt}%
\pgfpathmoveto{\pgfqpoint{2.347355in}{3.022103in}}%
\pgfpathcurveto{\pgfqpoint{2.358405in}{3.022103in}}{\pgfqpoint{2.369004in}{3.026494in}}{\pgfqpoint{2.376817in}{3.034307in}}%
\pgfpathcurveto{\pgfqpoint{2.384631in}{3.042121in}}{\pgfqpoint{2.389021in}{3.052720in}}{\pgfqpoint{2.389021in}{3.063770in}}%
\pgfpathcurveto{\pgfqpoint{2.389021in}{3.074820in}}{\pgfqpoint{2.384631in}{3.085419in}}{\pgfqpoint{2.376817in}{3.093233in}}%
\pgfpathcurveto{\pgfqpoint{2.369004in}{3.101046in}}{\pgfqpoint{2.358405in}{3.105437in}}{\pgfqpoint{2.347355in}{3.105437in}}%
\pgfpathcurveto{\pgfqpoint{2.336305in}{3.105437in}}{\pgfqpoint{2.325706in}{3.101046in}}{\pgfqpoint{2.317892in}{3.093233in}}%
\pgfpathcurveto{\pgfqpoint{2.310078in}{3.085419in}}{\pgfqpoint{2.305688in}{3.074820in}}{\pgfqpoint{2.305688in}{3.063770in}}%
\pgfpathcurveto{\pgfqpoint{2.305688in}{3.052720in}}{\pgfqpoint{2.310078in}{3.042121in}}{\pgfqpoint{2.317892in}{3.034307in}}%
\pgfpathcurveto{\pgfqpoint{2.325706in}{3.026494in}}{\pgfqpoint{2.336305in}{3.022103in}}{\pgfqpoint{2.347355in}{3.022103in}}%
\pgfpathclose%
\pgfusepath{stroke,fill}%
\end{pgfscope}%
\begin{pgfscope}%
\pgfpathrectangle{\pgfqpoint{0.750000in}{0.500000in}}{\pgfqpoint{4.650000in}{3.020000in}}%
\pgfusepath{clip}%
\pgfsetbuttcap%
\pgfsetroundjoin%
\definecolor{currentfill}{rgb}{0.121569,0.466667,0.705882}%
\pgfsetfillcolor{currentfill}%
\pgfsetlinewidth{1.003750pt}%
\definecolor{currentstroke}{rgb}{0.121569,0.466667,0.705882}%
\pgfsetstrokecolor{currentstroke}%
\pgfsetdash{}{0pt}%
\pgfpathmoveto{\pgfqpoint{2.000857in}{3.014029in}}%
\pgfpathcurveto{\pgfqpoint{2.011907in}{3.014029in}}{\pgfqpoint{2.022506in}{3.018419in}}{\pgfqpoint{2.030320in}{3.026232in}}%
\pgfpathcurveto{\pgfqpoint{2.038133in}{3.034046in}}{\pgfqpoint{2.042524in}{3.044645in}}{\pgfqpoint{2.042524in}{3.055695in}}%
\pgfpathcurveto{\pgfqpoint{2.042524in}{3.066745in}}{\pgfqpoint{2.038133in}{3.077344in}}{\pgfqpoint{2.030320in}{3.085158in}}%
\pgfpathcurveto{\pgfqpoint{2.022506in}{3.092972in}}{\pgfqpoint{2.011907in}{3.097362in}}{\pgfqpoint{2.000857in}{3.097362in}}%
\pgfpathcurveto{\pgfqpoint{1.989807in}{3.097362in}}{\pgfqpoint{1.979208in}{3.092972in}}{\pgfqpoint{1.971394in}{3.085158in}}%
\pgfpathcurveto{\pgfqpoint{1.963581in}{3.077344in}}{\pgfqpoint{1.959190in}{3.066745in}}{\pgfqpoint{1.959190in}{3.055695in}}%
\pgfpathcurveto{\pgfqpoint{1.959190in}{3.044645in}}{\pgfqpoint{1.963581in}{3.034046in}}{\pgfqpoint{1.971394in}{3.026232in}}%
\pgfpathcurveto{\pgfqpoint{1.979208in}{3.018419in}}{\pgfqpoint{1.989807in}{3.014029in}}{\pgfqpoint{2.000857in}{3.014029in}}%
\pgfpathclose%
\pgfusepath{stroke,fill}%
\end{pgfscope}%
\begin{pgfscope}%
\pgfpathrectangle{\pgfqpoint{0.750000in}{0.500000in}}{\pgfqpoint{4.650000in}{3.020000in}}%
\pgfusepath{clip}%
\pgfsetbuttcap%
\pgfsetroundjoin%
\definecolor{currentfill}{rgb}{1.000000,0.498039,0.054902}%
\pgfsetfillcolor{currentfill}%
\pgfsetlinewidth{1.003750pt}%
\definecolor{currentstroke}{rgb}{1.000000,0.498039,0.054902}%
\pgfsetstrokecolor{currentstroke}%
\pgfsetdash{}{0pt}%
\pgfpathmoveto{\pgfqpoint{1.446461in}{3.026141in}}%
\pgfpathcurveto{\pgfqpoint{1.457511in}{3.026141in}}{\pgfqpoint{1.468110in}{3.030531in}}{\pgfqpoint{1.475923in}{3.038345in}}%
\pgfpathcurveto{\pgfqpoint{1.483737in}{3.046158in}}{\pgfqpoint{1.488127in}{3.056757in}}{\pgfqpoint{1.488127in}{3.067807in}}%
\pgfpathcurveto{\pgfqpoint{1.488127in}{3.078858in}}{\pgfqpoint{1.483737in}{3.089457in}}{\pgfqpoint{1.475923in}{3.097270in}}%
\pgfpathcurveto{\pgfqpoint{1.468110in}{3.105084in}}{\pgfqpoint{1.457511in}{3.109474in}}{\pgfqpoint{1.446461in}{3.109474in}}%
\pgfpathcurveto{\pgfqpoint{1.435410in}{3.109474in}}{\pgfqpoint{1.424811in}{3.105084in}}{\pgfqpoint{1.416998in}{3.097270in}}%
\pgfpathcurveto{\pgfqpoint{1.409184in}{3.089457in}}{\pgfqpoint{1.404794in}{3.078858in}}{\pgfqpoint{1.404794in}{3.067807in}}%
\pgfpathcurveto{\pgfqpoint{1.404794in}{3.056757in}}{\pgfqpoint{1.409184in}{3.046158in}}{\pgfqpoint{1.416998in}{3.038345in}}%
\pgfpathcurveto{\pgfqpoint{1.424811in}{3.030531in}}{\pgfqpoint{1.435410in}{3.026141in}}{\pgfqpoint{1.446461in}{3.026141in}}%
\pgfpathclose%
\pgfusepath{stroke,fill}%
\end{pgfscope}%
\begin{pgfscope}%
\pgfpathrectangle{\pgfqpoint{0.750000in}{0.500000in}}{\pgfqpoint{4.650000in}{3.020000in}}%
\pgfusepath{clip}%
\pgfsetbuttcap%
\pgfsetroundjoin%
\definecolor{currentfill}{rgb}{1.000000,0.498039,0.054902}%
\pgfsetfillcolor{currentfill}%
\pgfsetlinewidth{1.003750pt}%
\definecolor{currentstroke}{rgb}{1.000000,0.498039,0.054902}%
\pgfsetstrokecolor{currentstroke}%
\pgfsetdash{}{0pt}%
\pgfpathmoveto{\pgfqpoint{2.485954in}{3.098815in}}%
\pgfpathcurveto{\pgfqpoint{2.497004in}{3.098815in}}{\pgfqpoint{2.507603in}{3.103205in}}{\pgfqpoint{2.515417in}{3.111019in}}%
\pgfpathcurveto{\pgfqpoint{2.523230in}{3.118832in}}{\pgfqpoint{2.527620in}{3.129431in}}{\pgfqpoint{2.527620in}{3.140481in}}%
\pgfpathcurveto{\pgfqpoint{2.527620in}{3.151531in}}{\pgfqpoint{2.523230in}{3.162130in}}{\pgfqpoint{2.515417in}{3.169944in}}%
\pgfpathcurveto{\pgfqpoint{2.507603in}{3.177758in}}{\pgfqpoint{2.497004in}{3.182148in}}{\pgfqpoint{2.485954in}{3.182148in}}%
\pgfpathcurveto{\pgfqpoint{2.474904in}{3.182148in}}{\pgfqpoint{2.464305in}{3.177758in}}{\pgfqpoint{2.456491in}{3.169944in}}%
\pgfpathcurveto{\pgfqpoint{2.448677in}{3.162130in}}{\pgfqpoint{2.444287in}{3.151531in}}{\pgfqpoint{2.444287in}{3.140481in}}%
\pgfpathcurveto{\pgfqpoint{2.444287in}{3.129431in}}{\pgfqpoint{2.448677in}{3.118832in}}{\pgfqpoint{2.456491in}{3.111019in}}%
\pgfpathcurveto{\pgfqpoint{2.464305in}{3.103205in}}{\pgfqpoint{2.474904in}{3.098815in}}{\pgfqpoint{2.485954in}{3.098815in}}%
\pgfpathclose%
\pgfusepath{stroke,fill}%
\end{pgfscope}%
\begin{pgfscope}%
\pgfpathrectangle{\pgfqpoint{0.750000in}{0.500000in}}{\pgfqpoint{4.650000in}{3.020000in}}%
\pgfusepath{clip}%
\pgfsetbuttcap%
\pgfsetroundjoin%
\definecolor{currentfill}{rgb}{1.000000,0.498039,0.054902}%
\pgfsetfillcolor{currentfill}%
\pgfsetlinewidth{1.003750pt}%
\definecolor{currentstroke}{rgb}{1.000000,0.498039,0.054902}%
\pgfsetstrokecolor{currentstroke}%
\pgfsetdash{}{0pt}%
\pgfpathmoveto{\pgfqpoint{1.931557in}{3.022103in}}%
\pgfpathcurveto{\pgfqpoint{1.942608in}{3.022103in}}{\pgfqpoint{1.953207in}{3.026494in}}{\pgfqpoint{1.961020in}{3.034307in}}%
\pgfpathcurveto{\pgfqpoint{1.968834in}{3.042121in}}{\pgfqpoint{1.973224in}{3.052720in}}{\pgfqpoint{1.973224in}{3.063770in}}%
\pgfpathcurveto{\pgfqpoint{1.973224in}{3.074820in}}{\pgfqpoint{1.968834in}{3.085419in}}{\pgfqpoint{1.961020in}{3.093233in}}%
\pgfpathcurveto{\pgfqpoint{1.953207in}{3.101046in}}{\pgfqpoint{1.942608in}{3.105437in}}{\pgfqpoint{1.931557in}{3.105437in}}%
\pgfpathcurveto{\pgfqpoint{1.920507in}{3.105437in}}{\pgfqpoint{1.909908in}{3.101046in}}{\pgfqpoint{1.902095in}{3.093233in}}%
\pgfpathcurveto{\pgfqpoint{1.894281in}{3.085419in}}{\pgfqpoint{1.889891in}{3.074820in}}{\pgfqpoint{1.889891in}{3.063770in}}%
\pgfpathcurveto{\pgfqpoint{1.889891in}{3.052720in}}{\pgfqpoint{1.894281in}{3.042121in}}{\pgfqpoint{1.902095in}{3.034307in}}%
\pgfpathcurveto{\pgfqpoint{1.909908in}{3.026494in}}{\pgfqpoint{1.920507in}{3.022103in}}{\pgfqpoint{1.931557in}{3.022103in}}%
\pgfpathclose%
\pgfusepath{stroke,fill}%
\end{pgfscope}%
\begin{pgfscope}%
\pgfpathrectangle{\pgfqpoint{0.750000in}{0.500000in}}{\pgfqpoint{4.650000in}{3.020000in}}%
\pgfusepath{clip}%
\pgfsetbuttcap%
\pgfsetroundjoin%
\definecolor{currentfill}{rgb}{1.000000,0.498039,0.054902}%
\pgfsetfillcolor{currentfill}%
\pgfsetlinewidth{1.003750pt}%
\definecolor{currentstroke}{rgb}{1.000000,0.498039,0.054902}%
\pgfsetstrokecolor{currentstroke}%
\pgfsetdash{}{0pt}%
\pgfpathmoveto{\pgfqpoint{2.485954in}{3.026141in}}%
\pgfpathcurveto{\pgfqpoint{2.497004in}{3.026141in}}{\pgfqpoint{2.507603in}{3.030531in}}{\pgfqpoint{2.515417in}{3.038345in}}%
\pgfpathcurveto{\pgfqpoint{2.523230in}{3.046158in}}{\pgfqpoint{2.527620in}{3.056757in}}{\pgfqpoint{2.527620in}{3.067807in}}%
\pgfpathcurveto{\pgfqpoint{2.527620in}{3.078858in}}{\pgfqpoint{2.523230in}{3.089457in}}{\pgfqpoint{2.515417in}{3.097270in}}%
\pgfpathcurveto{\pgfqpoint{2.507603in}{3.105084in}}{\pgfqpoint{2.497004in}{3.109474in}}{\pgfqpoint{2.485954in}{3.109474in}}%
\pgfpathcurveto{\pgfqpoint{2.474904in}{3.109474in}}{\pgfqpoint{2.464305in}{3.105084in}}{\pgfqpoint{2.456491in}{3.097270in}}%
\pgfpathcurveto{\pgfqpoint{2.448677in}{3.089457in}}{\pgfqpoint{2.444287in}{3.078858in}}{\pgfqpoint{2.444287in}{3.067807in}}%
\pgfpathcurveto{\pgfqpoint{2.444287in}{3.056757in}}{\pgfqpoint{2.448677in}{3.046158in}}{\pgfqpoint{2.456491in}{3.038345in}}%
\pgfpathcurveto{\pgfqpoint{2.464305in}{3.030531in}}{\pgfqpoint{2.474904in}{3.026141in}}{\pgfqpoint{2.485954in}{3.026141in}}%
\pgfpathclose%
\pgfusepath{stroke,fill}%
\end{pgfscope}%
\begin{pgfscope}%
\pgfpathrectangle{\pgfqpoint{0.750000in}{0.500000in}}{\pgfqpoint{4.650000in}{3.020000in}}%
\pgfusepath{clip}%
\pgfsetbuttcap%
\pgfsetroundjoin%
\definecolor{currentfill}{rgb}{1.000000,0.498039,0.054902}%
\pgfsetfillcolor{currentfill}%
\pgfsetlinewidth{1.003750pt}%
\definecolor{currentstroke}{rgb}{1.000000,0.498039,0.054902}%
\pgfsetstrokecolor{currentstroke}%
\pgfsetdash{}{0pt}%
\pgfpathmoveto{\pgfqpoint{2.832452in}{3.018066in}}%
\pgfpathcurveto{\pgfqpoint{2.843502in}{3.018066in}}{\pgfqpoint{2.854101in}{3.022456in}}{\pgfqpoint{2.861914in}{3.030270in}}%
\pgfpathcurveto{\pgfqpoint{2.869728in}{3.038083in}}{\pgfqpoint{2.874118in}{3.048682in}}{\pgfqpoint{2.874118in}{3.059733in}}%
\pgfpathcurveto{\pgfqpoint{2.874118in}{3.070783in}}{\pgfqpoint{2.869728in}{3.081382in}}{\pgfqpoint{2.861914in}{3.089195in}}%
\pgfpathcurveto{\pgfqpoint{2.854101in}{3.097009in}}{\pgfqpoint{2.843502in}{3.101399in}}{\pgfqpoint{2.832452in}{3.101399in}}%
\pgfpathcurveto{\pgfqpoint{2.821401in}{3.101399in}}{\pgfqpoint{2.810802in}{3.097009in}}{\pgfqpoint{2.802989in}{3.089195in}}%
\pgfpathcurveto{\pgfqpoint{2.795175in}{3.081382in}}{\pgfqpoint{2.790785in}{3.070783in}}{\pgfqpoint{2.790785in}{3.059733in}}%
\pgfpathcurveto{\pgfqpoint{2.790785in}{3.048682in}}{\pgfqpoint{2.795175in}{3.038083in}}{\pgfqpoint{2.802989in}{3.030270in}}%
\pgfpathcurveto{\pgfqpoint{2.810802in}{3.022456in}}{\pgfqpoint{2.821401in}{3.018066in}}{\pgfqpoint{2.832452in}{3.018066in}}%
\pgfpathclose%
\pgfusepath{stroke,fill}%
\end{pgfscope}%
\begin{pgfscope}%
\pgfpathrectangle{\pgfqpoint{0.750000in}{0.500000in}}{\pgfqpoint{4.650000in}{3.020000in}}%
\pgfusepath{clip}%
\pgfsetbuttcap%
\pgfsetroundjoin%
\definecolor{currentfill}{rgb}{1.000000,0.498039,0.054902}%
\pgfsetfillcolor{currentfill}%
\pgfsetlinewidth{1.003750pt}%
\definecolor{currentstroke}{rgb}{1.000000,0.498039,0.054902}%
\pgfsetstrokecolor{currentstroke}%
\pgfsetdash{}{0pt}%
\pgfpathmoveto{\pgfqpoint{1.792958in}{3.026141in}}%
\pgfpathcurveto{\pgfqpoint{1.804008in}{3.026141in}}{\pgfqpoint{1.814607in}{3.030531in}}{\pgfqpoint{1.822421in}{3.038345in}}%
\pgfpathcurveto{\pgfqpoint{1.830235in}{3.046158in}}{\pgfqpoint{1.834625in}{3.056757in}}{\pgfqpoint{1.834625in}{3.067807in}}%
\pgfpathcurveto{\pgfqpoint{1.834625in}{3.078858in}}{\pgfqpoint{1.830235in}{3.089457in}}{\pgfqpoint{1.822421in}{3.097270in}}%
\pgfpathcurveto{\pgfqpoint{1.814607in}{3.105084in}}{\pgfqpoint{1.804008in}{3.109474in}}{\pgfqpoint{1.792958in}{3.109474in}}%
\pgfpathcurveto{\pgfqpoint{1.781908in}{3.109474in}}{\pgfqpoint{1.771309in}{3.105084in}}{\pgfqpoint{1.763495in}{3.097270in}}%
\pgfpathcurveto{\pgfqpoint{1.755682in}{3.089457in}}{\pgfqpoint{1.751292in}{3.078858in}}{\pgfqpoint{1.751292in}{3.067807in}}%
\pgfpathcurveto{\pgfqpoint{1.751292in}{3.056757in}}{\pgfqpoint{1.755682in}{3.046158in}}{\pgfqpoint{1.763495in}{3.038345in}}%
\pgfpathcurveto{\pgfqpoint{1.771309in}{3.030531in}}{\pgfqpoint{1.781908in}{3.026141in}}{\pgfqpoint{1.792958in}{3.026141in}}%
\pgfpathclose%
\pgfusepath{stroke,fill}%
\end{pgfscope}%
\begin{pgfscope}%
\pgfpathrectangle{\pgfqpoint{0.750000in}{0.500000in}}{\pgfqpoint{4.650000in}{3.020000in}}%
\pgfusepath{clip}%
\pgfsetbuttcap%
\pgfsetroundjoin%
\definecolor{currentfill}{rgb}{1.000000,0.498039,0.054902}%
\pgfsetfillcolor{currentfill}%
\pgfsetlinewidth{1.003750pt}%
\definecolor{currentstroke}{rgb}{1.000000,0.498039,0.054902}%
\pgfsetstrokecolor{currentstroke}%
\pgfsetdash{}{0pt}%
\pgfpathmoveto{\pgfqpoint{2.347355in}{3.341061in}}%
\pgfpathcurveto{\pgfqpoint{2.358405in}{3.341061in}}{\pgfqpoint{2.369004in}{3.345451in}}{\pgfqpoint{2.376817in}{3.353264in}}%
\pgfpathcurveto{\pgfqpoint{2.384631in}{3.361078in}}{\pgfqpoint{2.389021in}{3.371677in}}{\pgfqpoint{2.389021in}{3.382727in}}%
\pgfpathcurveto{\pgfqpoint{2.389021in}{3.393777in}}{\pgfqpoint{2.384631in}{3.404376in}}{\pgfqpoint{2.376817in}{3.412190in}}%
\pgfpathcurveto{\pgfqpoint{2.369004in}{3.420004in}}{\pgfqpoint{2.358405in}{3.424394in}}{\pgfqpoint{2.347355in}{3.424394in}}%
\pgfpathcurveto{\pgfqpoint{2.336305in}{3.424394in}}{\pgfqpoint{2.325706in}{3.420004in}}{\pgfqpoint{2.317892in}{3.412190in}}%
\pgfpathcurveto{\pgfqpoint{2.310078in}{3.404376in}}{\pgfqpoint{2.305688in}{3.393777in}}{\pgfqpoint{2.305688in}{3.382727in}}%
\pgfpathcurveto{\pgfqpoint{2.305688in}{3.371677in}}{\pgfqpoint{2.310078in}{3.361078in}}{\pgfqpoint{2.317892in}{3.353264in}}%
\pgfpathcurveto{\pgfqpoint{2.325706in}{3.345451in}}{\pgfqpoint{2.336305in}{3.341061in}}{\pgfqpoint{2.347355in}{3.341061in}}%
\pgfpathclose%
\pgfusepath{stroke,fill}%
\end{pgfscope}%
\begin{pgfscope}%
\pgfpathrectangle{\pgfqpoint{0.750000in}{0.500000in}}{\pgfqpoint{4.650000in}{3.020000in}}%
\pgfusepath{clip}%
\pgfsetbuttcap%
\pgfsetroundjoin%
\definecolor{currentfill}{rgb}{1.000000,0.498039,0.054902}%
\pgfsetfillcolor{currentfill}%
\pgfsetlinewidth{1.003750pt}%
\definecolor{currentstroke}{rgb}{1.000000,0.498039,0.054902}%
\pgfsetstrokecolor{currentstroke}%
\pgfsetdash{}{0pt}%
\pgfpathmoveto{\pgfqpoint{1.377161in}{3.223975in}}%
\pgfpathcurveto{\pgfqpoint{1.388211in}{3.223975in}}{\pgfqpoint{1.398810in}{3.228365in}}{\pgfqpoint{1.406624in}{3.236179in}}%
\pgfpathcurveto{\pgfqpoint{1.414437in}{3.243993in}}{\pgfqpoint{1.418828in}{3.254592in}}{\pgfqpoint{1.418828in}{3.265642in}}%
\pgfpathcurveto{\pgfqpoint{1.418828in}{3.276692in}}{\pgfqpoint{1.414437in}{3.287291in}}{\pgfqpoint{1.406624in}{3.295104in}}%
\pgfpathcurveto{\pgfqpoint{1.398810in}{3.302918in}}{\pgfqpoint{1.388211in}{3.307308in}}{\pgfqpoint{1.377161in}{3.307308in}}%
\pgfpathcurveto{\pgfqpoint{1.366111in}{3.307308in}}{\pgfqpoint{1.355512in}{3.302918in}}{\pgfqpoint{1.347698in}{3.295104in}}%
\pgfpathcurveto{\pgfqpoint{1.339885in}{3.287291in}}{\pgfqpoint{1.335494in}{3.276692in}}{\pgfqpoint{1.335494in}{3.265642in}}%
\pgfpathcurveto{\pgfqpoint{1.335494in}{3.254592in}}{\pgfqpoint{1.339885in}{3.243993in}}{\pgfqpoint{1.347698in}{3.236179in}}%
\pgfpathcurveto{\pgfqpoint{1.355512in}{3.228365in}}{\pgfqpoint{1.366111in}{3.223975in}}{\pgfqpoint{1.377161in}{3.223975in}}%
\pgfpathclose%
\pgfusepath{stroke,fill}%
\end{pgfscope}%
\begin{pgfscope}%
\pgfpathrectangle{\pgfqpoint{0.750000in}{0.500000in}}{\pgfqpoint{4.650000in}{3.020000in}}%
\pgfusepath{clip}%
\pgfsetbuttcap%
\pgfsetroundjoin%
\definecolor{currentfill}{rgb}{1.000000,0.498039,0.054902}%
\pgfsetfillcolor{currentfill}%
\pgfsetlinewidth{1.003750pt}%
\definecolor{currentstroke}{rgb}{1.000000,0.498039,0.054902}%
\pgfsetstrokecolor{currentstroke}%
\pgfsetdash{}{0pt}%
\pgfpathmoveto{\pgfqpoint{1.723659in}{3.018066in}}%
\pgfpathcurveto{\pgfqpoint{1.734709in}{3.018066in}}{\pgfqpoint{1.745308in}{3.022456in}}{\pgfqpoint{1.753122in}{3.030270in}}%
\pgfpathcurveto{\pgfqpoint{1.760935in}{3.038083in}}{\pgfqpoint{1.765325in}{3.048682in}}{\pgfqpoint{1.765325in}{3.059733in}}%
\pgfpathcurveto{\pgfqpoint{1.765325in}{3.070783in}}{\pgfqpoint{1.760935in}{3.081382in}}{\pgfqpoint{1.753122in}{3.089195in}}%
\pgfpathcurveto{\pgfqpoint{1.745308in}{3.097009in}}{\pgfqpoint{1.734709in}{3.101399in}}{\pgfqpoint{1.723659in}{3.101399in}}%
\pgfpathcurveto{\pgfqpoint{1.712609in}{3.101399in}}{\pgfqpoint{1.702010in}{3.097009in}}{\pgfqpoint{1.694196in}{3.089195in}}%
\pgfpathcurveto{\pgfqpoint{1.686382in}{3.081382in}}{\pgfqpoint{1.681992in}{3.070783in}}{\pgfqpoint{1.681992in}{3.059733in}}%
\pgfpathcurveto{\pgfqpoint{1.681992in}{3.048682in}}{\pgfqpoint{1.686382in}{3.038083in}}{\pgfqpoint{1.694196in}{3.030270in}}%
\pgfpathcurveto{\pgfqpoint{1.702010in}{3.022456in}}{\pgfqpoint{1.712609in}{3.018066in}}{\pgfqpoint{1.723659in}{3.018066in}}%
\pgfpathclose%
\pgfusepath{stroke,fill}%
\end{pgfscope}%
\begin{pgfscope}%
\pgfpathrectangle{\pgfqpoint{0.750000in}{0.500000in}}{\pgfqpoint{4.650000in}{3.020000in}}%
\pgfusepath{clip}%
\pgfsetbuttcap%
\pgfsetroundjoin%
\definecolor{currentfill}{rgb}{0.121569,0.466667,0.705882}%
\pgfsetfillcolor{currentfill}%
\pgfsetlinewidth{1.003750pt}%
\definecolor{currentstroke}{rgb}{0.121569,0.466667,0.705882}%
\pgfsetstrokecolor{currentstroke}%
\pgfsetdash{}{0pt}%
\pgfpathmoveto{\pgfqpoint{1.654359in}{0.595606in}}%
\pgfpathcurveto{\pgfqpoint{1.665409in}{0.595606in}}{\pgfqpoint{1.676008in}{0.599996in}}{\pgfqpoint{1.683822in}{0.607810in}}%
\pgfpathcurveto{\pgfqpoint{1.691636in}{0.615624in}}{\pgfqpoint{1.696026in}{0.626223in}}{\pgfqpoint{1.696026in}{0.637273in}}%
\pgfpathcurveto{\pgfqpoint{1.696026in}{0.648323in}}{\pgfqpoint{1.691636in}{0.658922in}}{\pgfqpoint{1.683822in}{0.666736in}}%
\pgfpathcurveto{\pgfqpoint{1.676008in}{0.674549in}}{\pgfqpoint{1.665409in}{0.678939in}}{\pgfqpoint{1.654359in}{0.678939in}}%
\pgfpathcurveto{\pgfqpoint{1.643309in}{0.678939in}}{\pgfqpoint{1.632710in}{0.674549in}}{\pgfqpoint{1.624896in}{0.666736in}}%
\pgfpathcurveto{\pgfqpoint{1.617083in}{0.658922in}}{\pgfqpoint{1.612692in}{0.648323in}}{\pgfqpoint{1.612692in}{0.637273in}}%
\pgfpathcurveto{\pgfqpoint{1.612692in}{0.626223in}}{\pgfqpoint{1.617083in}{0.615624in}}{\pgfqpoint{1.624896in}{0.607810in}}%
\pgfpathcurveto{\pgfqpoint{1.632710in}{0.599996in}}{\pgfqpoint{1.643309in}{0.595606in}}{\pgfqpoint{1.654359in}{0.595606in}}%
\pgfpathclose%
\pgfusepath{stroke,fill}%
\end{pgfscope}%
\begin{pgfscope}%
\pgfpathrectangle{\pgfqpoint{0.750000in}{0.500000in}}{\pgfqpoint{4.650000in}{3.020000in}}%
\pgfusepath{clip}%
\pgfsetbuttcap%
\pgfsetroundjoin%
\definecolor{currentfill}{rgb}{0.121569,0.466667,0.705882}%
\pgfsetfillcolor{currentfill}%
\pgfsetlinewidth{1.003750pt}%
\definecolor{currentstroke}{rgb}{0.121569,0.466667,0.705882}%
\pgfsetstrokecolor{currentstroke}%
\pgfsetdash{}{0pt}%
\pgfpathmoveto{\pgfqpoint{1.723659in}{3.014029in}}%
\pgfpathcurveto{\pgfqpoint{1.734709in}{3.014029in}}{\pgfqpoint{1.745308in}{3.018419in}}{\pgfqpoint{1.753122in}{3.026232in}}%
\pgfpathcurveto{\pgfqpoint{1.760935in}{3.034046in}}{\pgfqpoint{1.765325in}{3.044645in}}{\pgfqpoint{1.765325in}{3.055695in}}%
\pgfpathcurveto{\pgfqpoint{1.765325in}{3.066745in}}{\pgfqpoint{1.760935in}{3.077344in}}{\pgfqpoint{1.753122in}{3.085158in}}%
\pgfpathcurveto{\pgfqpoint{1.745308in}{3.092972in}}{\pgfqpoint{1.734709in}{3.097362in}}{\pgfqpoint{1.723659in}{3.097362in}}%
\pgfpathcurveto{\pgfqpoint{1.712609in}{3.097362in}}{\pgfqpoint{1.702010in}{3.092972in}}{\pgfqpoint{1.694196in}{3.085158in}}%
\pgfpathcurveto{\pgfqpoint{1.686382in}{3.077344in}}{\pgfqpoint{1.681992in}{3.066745in}}{\pgfqpoint{1.681992in}{3.055695in}}%
\pgfpathcurveto{\pgfqpoint{1.681992in}{3.044645in}}{\pgfqpoint{1.686382in}{3.034046in}}{\pgfqpoint{1.694196in}{3.026232in}}%
\pgfpathcurveto{\pgfqpoint{1.702010in}{3.018419in}}{\pgfqpoint{1.712609in}{3.014029in}}{\pgfqpoint{1.723659in}{3.014029in}}%
\pgfpathclose%
\pgfusepath{stroke,fill}%
\end{pgfscope}%
\begin{pgfscope}%
\pgfpathrectangle{\pgfqpoint{0.750000in}{0.500000in}}{\pgfqpoint{4.650000in}{3.020000in}}%
\pgfusepath{clip}%
\pgfsetbuttcap%
\pgfsetroundjoin%
\definecolor{currentfill}{rgb}{1.000000,0.498039,0.054902}%
\pgfsetfillcolor{currentfill}%
\pgfsetlinewidth{1.003750pt}%
\definecolor{currentstroke}{rgb}{1.000000,0.498039,0.054902}%
\pgfsetstrokecolor{currentstroke}%
\pgfsetdash{}{0pt}%
\pgfpathmoveto{\pgfqpoint{3.178949in}{3.018066in}}%
\pgfpathcurveto{\pgfqpoint{3.189999in}{3.018066in}}{\pgfqpoint{3.200598in}{3.022456in}}{\pgfqpoint{3.208412in}{3.030270in}}%
\pgfpathcurveto{\pgfqpoint{3.216226in}{3.038083in}}{\pgfqpoint{3.220616in}{3.048682in}}{\pgfqpoint{3.220616in}{3.059733in}}%
\pgfpathcurveto{\pgfqpoint{3.220616in}{3.070783in}}{\pgfqpoint{3.216226in}{3.081382in}}{\pgfqpoint{3.208412in}{3.089195in}}%
\pgfpathcurveto{\pgfqpoint{3.200598in}{3.097009in}}{\pgfqpoint{3.189999in}{3.101399in}}{\pgfqpoint{3.178949in}{3.101399in}}%
\pgfpathcurveto{\pgfqpoint{3.167899in}{3.101399in}}{\pgfqpoint{3.157300in}{3.097009in}}{\pgfqpoint{3.149487in}{3.089195in}}%
\pgfpathcurveto{\pgfqpoint{3.141673in}{3.081382in}}{\pgfqpoint{3.137283in}{3.070783in}}{\pgfqpoint{3.137283in}{3.059733in}}%
\pgfpathcurveto{\pgfqpoint{3.137283in}{3.048682in}}{\pgfqpoint{3.141673in}{3.038083in}}{\pgfqpoint{3.149487in}{3.030270in}}%
\pgfpathcurveto{\pgfqpoint{3.157300in}{3.022456in}}{\pgfqpoint{3.167899in}{3.018066in}}{\pgfqpoint{3.178949in}{3.018066in}}%
\pgfpathclose%
\pgfusepath{stroke,fill}%
\end{pgfscope}%
\begin{pgfscope}%
\pgfpathrectangle{\pgfqpoint{0.750000in}{0.500000in}}{\pgfqpoint{4.650000in}{3.020000in}}%
\pgfusepath{clip}%
\pgfsetbuttcap%
\pgfsetroundjoin%
\definecolor{currentfill}{rgb}{1.000000,0.498039,0.054902}%
\pgfsetfillcolor{currentfill}%
\pgfsetlinewidth{1.003750pt}%
\definecolor{currentstroke}{rgb}{1.000000,0.498039,0.054902}%
\pgfsetstrokecolor{currentstroke}%
\pgfsetdash{}{0pt}%
\pgfpathmoveto{\pgfqpoint{1.307861in}{3.026141in}}%
\pgfpathcurveto{\pgfqpoint{1.318912in}{3.026141in}}{\pgfqpoint{1.329511in}{3.030531in}}{\pgfqpoint{1.337324in}{3.038345in}}%
\pgfpathcurveto{\pgfqpoint{1.345138in}{3.046158in}}{\pgfqpoint{1.349528in}{3.056757in}}{\pgfqpoint{1.349528in}{3.067807in}}%
\pgfpathcurveto{\pgfqpoint{1.349528in}{3.078858in}}{\pgfqpoint{1.345138in}{3.089457in}}{\pgfqpoint{1.337324in}{3.097270in}}%
\pgfpathcurveto{\pgfqpoint{1.329511in}{3.105084in}}{\pgfqpoint{1.318912in}{3.109474in}}{\pgfqpoint{1.307861in}{3.109474in}}%
\pgfpathcurveto{\pgfqpoint{1.296811in}{3.109474in}}{\pgfqpoint{1.286212in}{3.105084in}}{\pgfqpoint{1.278399in}{3.097270in}}%
\pgfpathcurveto{\pgfqpoint{1.270585in}{3.089457in}}{\pgfqpoint{1.266195in}{3.078858in}}{\pgfqpoint{1.266195in}{3.067807in}}%
\pgfpathcurveto{\pgfqpoint{1.266195in}{3.056757in}}{\pgfqpoint{1.270585in}{3.046158in}}{\pgfqpoint{1.278399in}{3.038345in}}%
\pgfpathcurveto{\pgfqpoint{1.286212in}{3.030531in}}{\pgfqpoint{1.296811in}{3.026141in}}{\pgfqpoint{1.307861in}{3.026141in}}%
\pgfpathclose%
\pgfusepath{stroke,fill}%
\end{pgfscope}%
\begin{pgfscope}%
\pgfpathrectangle{\pgfqpoint{0.750000in}{0.500000in}}{\pgfqpoint{4.650000in}{3.020000in}}%
\pgfusepath{clip}%
\pgfsetbuttcap%
\pgfsetroundjoin%
\definecolor{currentfill}{rgb}{1.000000,0.498039,0.054902}%
\pgfsetfillcolor{currentfill}%
\pgfsetlinewidth{1.003750pt}%
\definecolor{currentstroke}{rgb}{1.000000,0.498039,0.054902}%
\pgfsetstrokecolor{currentstroke}%
\pgfsetdash{}{0pt}%
\pgfpathmoveto{\pgfqpoint{1.446461in}{3.038253in}}%
\pgfpathcurveto{\pgfqpoint{1.457511in}{3.038253in}}{\pgfqpoint{1.468110in}{3.042643in}}{\pgfqpoint{1.475923in}{3.050457in}}%
\pgfpathcurveto{\pgfqpoint{1.483737in}{3.058271in}}{\pgfqpoint{1.488127in}{3.068870in}}{\pgfqpoint{1.488127in}{3.079920in}}%
\pgfpathcurveto{\pgfqpoint{1.488127in}{3.090970in}}{\pgfqpoint{1.483737in}{3.101569in}}{\pgfqpoint{1.475923in}{3.109383in}}%
\pgfpathcurveto{\pgfqpoint{1.468110in}{3.117196in}}{\pgfqpoint{1.457511in}{3.121586in}}{\pgfqpoint{1.446461in}{3.121586in}}%
\pgfpathcurveto{\pgfqpoint{1.435410in}{3.121586in}}{\pgfqpoint{1.424811in}{3.117196in}}{\pgfqpoint{1.416998in}{3.109383in}}%
\pgfpathcurveto{\pgfqpoint{1.409184in}{3.101569in}}{\pgfqpoint{1.404794in}{3.090970in}}{\pgfqpoint{1.404794in}{3.079920in}}%
\pgfpathcurveto{\pgfqpoint{1.404794in}{3.068870in}}{\pgfqpoint{1.409184in}{3.058271in}}{\pgfqpoint{1.416998in}{3.050457in}}%
\pgfpathcurveto{\pgfqpoint{1.424811in}{3.042643in}}{\pgfqpoint{1.435410in}{3.038253in}}{\pgfqpoint{1.446461in}{3.038253in}}%
\pgfpathclose%
\pgfusepath{stroke,fill}%
\end{pgfscope}%
\begin{pgfscope}%
\pgfpathrectangle{\pgfqpoint{0.750000in}{0.500000in}}{\pgfqpoint{4.650000in}{3.020000in}}%
\pgfusepath{clip}%
\pgfsetbuttcap%
\pgfsetroundjoin%
\definecolor{currentfill}{rgb}{1.000000,0.498039,0.054902}%
\pgfsetfillcolor{currentfill}%
\pgfsetlinewidth{1.003750pt}%
\definecolor{currentstroke}{rgb}{1.000000,0.498039,0.054902}%
\pgfsetstrokecolor{currentstroke}%
\pgfsetdash{}{0pt}%
\pgfpathmoveto{\pgfqpoint{1.515760in}{3.018066in}}%
\pgfpathcurveto{\pgfqpoint{1.526810in}{3.018066in}}{\pgfqpoint{1.537409in}{3.022456in}}{\pgfqpoint{1.545223in}{3.030270in}}%
\pgfpathcurveto{\pgfqpoint{1.553036in}{3.038083in}}{\pgfqpoint{1.557427in}{3.048682in}}{\pgfqpoint{1.557427in}{3.059733in}}%
\pgfpathcurveto{\pgfqpoint{1.557427in}{3.070783in}}{\pgfqpoint{1.553036in}{3.081382in}}{\pgfqpoint{1.545223in}{3.089195in}}%
\pgfpathcurveto{\pgfqpoint{1.537409in}{3.097009in}}{\pgfqpoint{1.526810in}{3.101399in}}{\pgfqpoint{1.515760in}{3.101399in}}%
\pgfpathcurveto{\pgfqpoint{1.504710in}{3.101399in}}{\pgfqpoint{1.494111in}{3.097009in}}{\pgfqpoint{1.486297in}{3.089195in}}%
\pgfpathcurveto{\pgfqpoint{1.478484in}{3.081382in}}{\pgfqpoint{1.474093in}{3.070783in}}{\pgfqpoint{1.474093in}{3.059733in}}%
\pgfpathcurveto{\pgfqpoint{1.474093in}{3.048682in}}{\pgfqpoint{1.478484in}{3.038083in}}{\pgfqpoint{1.486297in}{3.030270in}}%
\pgfpathcurveto{\pgfqpoint{1.494111in}{3.022456in}}{\pgfqpoint{1.504710in}{3.018066in}}{\pgfqpoint{1.515760in}{3.018066in}}%
\pgfpathclose%
\pgfusepath{stroke,fill}%
\end{pgfscope}%
\begin{pgfscope}%
\pgfpathrectangle{\pgfqpoint{0.750000in}{0.500000in}}{\pgfqpoint{4.650000in}{3.020000in}}%
\pgfusepath{clip}%
\pgfsetbuttcap%
\pgfsetroundjoin%
\definecolor{currentfill}{rgb}{1.000000,0.498039,0.054902}%
\pgfsetfillcolor{currentfill}%
\pgfsetlinewidth{1.003750pt}%
\definecolor{currentstroke}{rgb}{1.000000,0.498039,0.054902}%
\pgfsetstrokecolor{currentstroke}%
\pgfsetdash{}{0pt}%
\pgfpathmoveto{\pgfqpoint{1.238562in}{3.026141in}}%
\pgfpathcurveto{\pgfqpoint{1.249612in}{3.026141in}}{\pgfqpoint{1.260211in}{3.030531in}}{\pgfqpoint{1.268025in}{3.038345in}}%
\pgfpathcurveto{\pgfqpoint{1.275838in}{3.046158in}}{\pgfqpoint{1.280229in}{3.056757in}}{\pgfqpoint{1.280229in}{3.067807in}}%
\pgfpathcurveto{\pgfqpoint{1.280229in}{3.078858in}}{\pgfqpoint{1.275838in}{3.089457in}}{\pgfqpoint{1.268025in}{3.097270in}}%
\pgfpathcurveto{\pgfqpoint{1.260211in}{3.105084in}}{\pgfqpoint{1.249612in}{3.109474in}}{\pgfqpoint{1.238562in}{3.109474in}}%
\pgfpathcurveto{\pgfqpoint{1.227512in}{3.109474in}}{\pgfqpoint{1.216913in}{3.105084in}}{\pgfqpoint{1.209099in}{3.097270in}}%
\pgfpathcurveto{\pgfqpoint{1.201285in}{3.089457in}}{\pgfqpoint{1.196895in}{3.078858in}}{\pgfqpoint{1.196895in}{3.067807in}}%
\pgfpathcurveto{\pgfqpoint{1.196895in}{3.056757in}}{\pgfqpoint{1.201285in}{3.046158in}}{\pgfqpoint{1.209099in}{3.038345in}}%
\pgfpathcurveto{\pgfqpoint{1.216913in}{3.030531in}}{\pgfqpoint{1.227512in}{3.026141in}}{\pgfqpoint{1.238562in}{3.026141in}}%
\pgfpathclose%
\pgfusepath{stroke,fill}%
\end{pgfscope}%
\begin{pgfscope}%
\pgfpathrectangle{\pgfqpoint{0.750000in}{0.500000in}}{\pgfqpoint{4.650000in}{3.020000in}}%
\pgfusepath{clip}%
\pgfsetbuttcap%
\pgfsetroundjoin%
\definecolor{currentfill}{rgb}{1.000000,0.498039,0.054902}%
\pgfsetfillcolor{currentfill}%
\pgfsetlinewidth{1.003750pt}%
\definecolor{currentstroke}{rgb}{1.000000,0.498039,0.054902}%
\pgfsetstrokecolor{currentstroke}%
\pgfsetdash{}{0pt}%
\pgfpathmoveto{\pgfqpoint{2.000857in}{3.070553in}}%
\pgfpathcurveto{\pgfqpoint{2.011907in}{3.070553in}}{\pgfqpoint{2.022506in}{3.074943in}}{\pgfqpoint{2.030320in}{3.082756in}}%
\pgfpathcurveto{\pgfqpoint{2.038133in}{3.090570in}}{\pgfqpoint{2.042524in}{3.101169in}}{\pgfqpoint{2.042524in}{3.112219in}}%
\pgfpathcurveto{\pgfqpoint{2.042524in}{3.123269in}}{\pgfqpoint{2.038133in}{3.133868in}}{\pgfqpoint{2.030320in}{3.141682in}}%
\pgfpathcurveto{\pgfqpoint{2.022506in}{3.149496in}}{\pgfqpoint{2.011907in}{3.153886in}}{\pgfqpoint{2.000857in}{3.153886in}}%
\pgfpathcurveto{\pgfqpoint{1.989807in}{3.153886in}}{\pgfqpoint{1.979208in}{3.149496in}}{\pgfqpoint{1.971394in}{3.141682in}}%
\pgfpathcurveto{\pgfqpoint{1.963581in}{3.133868in}}{\pgfqpoint{1.959190in}{3.123269in}}{\pgfqpoint{1.959190in}{3.112219in}}%
\pgfpathcurveto{\pgfqpoint{1.959190in}{3.101169in}}{\pgfqpoint{1.963581in}{3.090570in}}{\pgfqpoint{1.971394in}{3.082756in}}%
\pgfpathcurveto{\pgfqpoint{1.979208in}{3.074943in}}{\pgfqpoint{1.989807in}{3.070553in}}{\pgfqpoint{2.000857in}{3.070553in}}%
\pgfpathclose%
\pgfusepath{stroke,fill}%
\end{pgfscope}%
\begin{pgfscope}%
\pgfpathrectangle{\pgfqpoint{0.750000in}{0.500000in}}{\pgfqpoint{4.650000in}{3.020000in}}%
\pgfusepath{clip}%
\pgfsetbuttcap%
\pgfsetroundjoin%
\definecolor{currentfill}{rgb}{0.839216,0.152941,0.156863}%
\pgfsetfillcolor{currentfill}%
\pgfsetlinewidth{1.003750pt}%
\definecolor{currentstroke}{rgb}{0.839216,0.152941,0.156863}%
\pgfsetstrokecolor{currentstroke}%
\pgfsetdash{}{0pt}%
\pgfpathmoveto{\pgfqpoint{2.624553in}{3.018066in}}%
\pgfpathcurveto{\pgfqpoint{2.635603in}{3.018066in}}{\pgfqpoint{2.646202in}{3.022456in}}{\pgfqpoint{2.654016in}{3.030270in}}%
\pgfpathcurveto{\pgfqpoint{2.661829in}{3.038083in}}{\pgfqpoint{2.666220in}{3.048682in}}{\pgfqpoint{2.666220in}{3.059733in}}%
\pgfpathcurveto{\pgfqpoint{2.666220in}{3.070783in}}{\pgfqpoint{2.661829in}{3.081382in}}{\pgfqpoint{2.654016in}{3.089195in}}%
\pgfpathcurveto{\pgfqpoint{2.646202in}{3.097009in}}{\pgfqpoint{2.635603in}{3.101399in}}{\pgfqpoint{2.624553in}{3.101399in}}%
\pgfpathcurveto{\pgfqpoint{2.613503in}{3.101399in}}{\pgfqpoint{2.602904in}{3.097009in}}{\pgfqpoint{2.595090in}{3.089195in}}%
\pgfpathcurveto{\pgfqpoint{2.587277in}{3.081382in}}{\pgfqpoint{2.582886in}{3.070783in}}{\pgfqpoint{2.582886in}{3.059733in}}%
\pgfpathcurveto{\pgfqpoint{2.582886in}{3.048682in}}{\pgfqpoint{2.587277in}{3.038083in}}{\pgfqpoint{2.595090in}{3.030270in}}%
\pgfpathcurveto{\pgfqpoint{2.602904in}{3.022456in}}{\pgfqpoint{2.613503in}{3.018066in}}{\pgfqpoint{2.624553in}{3.018066in}}%
\pgfpathclose%
\pgfusepath{stroke,fill}%
\end{pgfscope}%
\begin{pgfscope}%
\pgfpathrectangle{\pgfqpoint{0.750000in}{0.500000in}}{\pgfqpoint{4.650000in}{3.020000in}}%
\pgfusepath{clip}%
\pgfsetbuttcap%
\pgfsetroundjoin%
\definecolor{currentfill}{rgb}{1.000000,0.498039,0.054902}%
\pgfsetfillcolor{currentfill}%
\pgfsetlinewidth{1.003750pt}%
\definecolor{currentstroke}{rgb}{1.000000,0.498039,0.054902}%
\pgfsetstrokecolor{currentstroke}%
\pgfsetdash{}{0pt}%
\pgfpathmoveto{\pgfqpoint{1.377161in}{3.086702in}}%
\pgfpathcurveto{\pgfqpoint{1.388211in}{3.086702in}}{\pgfqpoint{1.398810in}{3.091093in}}{\pgfqpoint{1.406624in}{3.098906in}}%
\pgfpathcurveto{\pgfqpoint{1.414437in}{3.106720in}}{\pgfqpoint{1.418828in}{3.117319in}}{\pgfqpoint{1.418828in}{3.128369in}}%
\pgfpathcurveto{\pgfqpoint{1.418828in}{3.139419in}}{\pgfqpoint{1.414437in}{3.150018in}}{\pgfqpoint{1.406624in}{3.157832in}}%
\pgfpathcurveto{\pgfqpoint{1.398810in}{3.165645in}}{\pgfqpoint{1.388211in}{3.170036in}}{\pgfqpoint{1.377161in}{3.170036in}}%
\pgfpathcurveto{\pgfqpoint{1.366111in}{3.170036in}}{\pgfqpoint{1.355512in}{3.165645in}}{\pgfqpoint{1.347698in}{3.157832in}}%
\pgfpathcurveto{\pgfqpoint{1.339885in}{3.150018in}}{\pgfqpoint{1.335494in}{3.139419in}}{\pgfqpoint{1.335494in}{3.128369in}}%
\pgfpathcurveto{\pgfqpoint{1.335494in}{3.117319in}}{\pgfqpoint{1.339885in}{3.106720in}}{\pgfqpoint{1.347698in}{3.098906in}}%
\pgfpathcurveto{\pgfqpoint{1.355512in}{3.091093in}}{\pgfqpoint{1.366111in}{3.086702in}}{\pgfqpoint{1.377161in}{3.086702in}}%
\pgfpathclose%
\pgfusepath{stroke,fill}%
\end{pgfscope}%
\begin{pgfscope}%
\pgfpathrectangle{\pgfqpoint{0.750000in}{0.500000in}}{\pgfqpoint{4.650000in}{3.020000in}}%
\pgfusepath{clip}%
\pgfsetbuttcap%
\pgfsetroundjoin%
\definecolor{currentfill}{rgb}{1.000000,0.498039,0.054902}%
\pgfsetfillcolor{currentfill}%
\pgfsetlinewidth{1.003750pt}%
\definecolor{currentstroke}{rgb}{1.000000,0.498039,0.054902}%
\pgfsetstrokecolor{currentstroke}%
\pgfsetdash{}{0pt}%
\pgfpathmoveto{\pgfqpoint{1.099963in}{3.022103in}}%
\pgfpathcurveto{\pgfqpoint{1.111013in}{3.022103in}}{\pgfqpoint{1.121612in}{3.026494in}}{\pgfqpoint{1.129426in}{3.034307in}}%
\pgfpathcurveto{\pgfqpoint{1.137239in}{3.042121in}}{\pgfqpoint{1.141629in}{3.052720in}}{\pgfqpoint{1.141629in}{3.063770in}}%
\pgfpathcurveto{\pgfqpoint{1.141629in}{3.074820in}}{\pgfqpoint{1.137239in}{3.085419in}}{\pgfqpoint{1.129426in}{3.093233in}}%
\pgfpathcurveto{\pgfqpoint{1.121612in}{3.101046in}}{\pgfqpoint{1.111013in}{3.105437in}}{\pgfqpoint{1.099963in}{3.105437in}}%
\pgfpathcurveto{\pgfqpoint{1.088913in}{3.105437in}}{\pgfqpoint{1.078314in}{3.101046in}}{\pgfqpoint{1.070500in}{3.093233in}}%
\pgfpathcurveto{\pgfqpoint{1.062686in}{3.085419in}}{\pgfqpoint{1.058296in}{3.074820in}}{\pgfqpoint{1.058296in}{3.063770in}}%
\pgfpathcurveto{\pgfqpoint{1.058296in}{3.052720in}}{\pgfqpoint{1.062686in}{3.042121in}}{\pgfqpoint{1.070500in}{3.034307in}}%
\pgfpathcurveto{\pgfqpoint{1.078314in}{3.026494in}}{\pgfqpoint{1.088913in}{3.022103in}}{\pgfqpoint{1.099963in}{3.022103in}}%
\pgfpathclose%
\pgfusepath{stroke,fill}%
\end{pgfscope}%
\begin{pgfscope}%
\pgfpathrectangle{\pgfqpoint{0.750000in}{0.500000in}}{\pgfqpoint{4.650000in}{3.020000in}}%
\pgfusepath{clip}%
\pgfsetbuttcap%
\pgfsetroundjoin%
\definecolor{currentfill}{rgb}{1.000000,0.498039,0.054902}%
\pgfsetfillcolor{currentfill}%
\pgfsetlinewidth{1.003750pt}%
\definecolor{currentstroke}{rgb}{1.000000,0.498039,0.054902}%
\pgfsetstrokecolor{currentstroke}%
\pgfsetdash{}{0pt}%
\pgfpathmoveto{\pgfqpoint{2.832452in}{3.163414in}}%
\pgfpathcurveto{\pgfqpoint{2.843502in}{3.163414in}}{\pgfqpoint{2.854101in}{3.167804in}}{\pgfqpoint{2.861914in}{3.175617in}}%
\pgfpathcurveto{\pgfqpoint{2.869728in}{3.183431in}}{\pgfqpoint{2.874118in}{3.194030in}}{\pgfqpoint{2.874118in}{3.205080in}}%
\pgfpathcurveto{\pgfqpoint{2.874118in}{3.216130in}}{\pgfqpoint{2.869728in}{3.226729in}}{\pgfqpoint{2.861914in}{3.234543in}}%
\pgfpathcurveto{\pgfqpoint{2.854101in}{3.242357in}}{\pgfqpoint{2.843502in}{3.246747in}}{\pgfqpoint{2.832452in}{3.246747in}}%
\pgfpathcurveto{\pgfqpoint{2.821401in}{3.246747in}}{\pgfqpoint{2.810802in}{3.242357in}}{\pgfqpoint{2.802989in}{3.234543in}}%
\pgfpathcurveto{\pgfqpoint{2.795175in}{3.226729in}}{\pgfqpoint{2.790785in}{3.216130in}}{\pgfqpoint{2.790785in}{3.205080in}}%
\pgfpathcurveto{\pgfqpoint{2.790785in}{3.194030in}}{\pgfqpoint{2.795175in}{3.183431in}}{\pgfqpoint{2.802989in}{3.175617in}}%
\pgfpathcurveto{\pgfqpoint{2.810802in}{3.167804in}}{\pgfqpoint{2.821401in}{3.163414in}}{\pgfqpoint{2.832452in}{3.163414in}}%
\pgfpathclose%
\pgfusepath{stroke,fill}%
\end{pgfscope}%
\begin{pgfscope}%
\pgfpathrectangle{\pgfqpoint{0.750000in}{0.500000in}}{\pgfqpoint{4.650000in}{3.020000in}}%
\pgfusepath{clip}%
\pgfsetbuttcap%
\pgfsetroundjoin%
\definecolor{currentfill}{rgb}{0.121569,0.466667,0.705882}%
\pgfsetfillcolor{currentfill}%
\pgfsetlinewidth{1.003750pt}%
\definecolor{currentstroke}{rgb}{0.121569,0.466667,0.705882}%
\pgfsetstrokecolor{currentstroke}%
\pgfsetdash{}{0pt}%
\pgfpathmoveto{\pgfqpoint{1.030663in}{0.595606in}}%
\pgfpathcurveto{\pgfqpoint{1.041713in}{0.595606in}}{\pgfqpoint{1.052312in}{0.599996in}}{\pgfqpoint{1.060126in}{0.607810in}}%
\pgfpathcurveto{\pgfqpoint{1.067940in}{0.615624in}}{\pgfqpoint{1.072330in}{0.626223in}}{\pgfqpoint{1.072330in}{0.637273in}}%
\pgfpathcurveto{\pgfqpoint{1.072330in}{0.648323in}}{\pgfqpoint{1.067940in}{0.658922in}}{\pgfqpoint{1.060126in}{0.666736in}}%
\pgfpathcurveto{\pgfqpoint{1.052312in}{0.674549in}}{\pgfqpoint{1.041713in}{0.678939in}}{\pgfqpoint{1.030663in}{0.678939in}}%
\pgfpathcurveto{\pgfqpoint{1.019613in}{0.678939in}}{\pgfqpoint{1.009014in}{0.674549in}}{\pgfqpoint{1.001200in}{0.666736in}}%
\pgfpathcurveto{\pgfqpoint{0.993387in}{0.658922in}}{\pgfqpoint{0.988997in}{0.648323in}}{\pgfqpoint{0.988997in}{0.637273in}}%
\pgfpathcurveto{\pgfqpoint{0.988997in}{0.626223in}}{\pgfqpoint{0.993387in}{0.615624in}}{\pgfqpoint{1.001200in}{0.607810in}}%
\pgfpathcurveto{\pgfqpoint{1.009014in}{0.599996in}}{\pgfqpoint{1.019613in}{0.595606in}}{\pgfqpoint{1.030663in}{0.595606in}}%
\pgfpathclose%
\pgfusepath{stroke,fill}%
\end{pgfscope}%
\begin{pgfscope}%
\pgfpathrectangle{\pgfqpoint{0.750000in}{0.500000in}}{\pgfqpoint{4.650000in}{3.020000in}}%
\pgfusepath{clip}%
\pgfsetbuttcap%
\pgfsetroundjoin%
\definecolor{currentfill}{rgb}{0.121569,0.466667,0.705882}%
\pgfsetfillcolor{currentfill}%
\pgfsetlinewidth{1.003750pt}%
\definecolor{currentstroke}{rgb}{0.121569,0.466667,0.705882}%
\pgfsetstrokecolor{currentstroke}%
\pgfsetdash{}{0pt}%
\pgfpathmoveto{\pgfqpoint{1.377161in}{0.603681in}}%
\pgfpathcurveto{\pgfqpoint{1.388211in}{0.603681in}}{\pgfqpoint{1.398810in}{0.608071in}}{\pgfqpoint{1.406624in}{0.615885in}}%
\pgfpathcurveto{\pgfqpoint{1.414437in}{0.623698in}}{\pgfqpoint{1.418828in}{0.634297in}}{\pgfqpoint{1.418828in}{0.645348in}}%
\pgfpathcurveto{\pgfqpoint{1.418828in}{0.656398in}}{\pgfqpoint{1.414437in}{0.666997in}}{\pgfqpoint{1.406624in}{0.674810in}}%
\pgfpathcurveto{\pgfqpoint{1.398810in}{0.682624in}}{\pgfqpoint{1.388211in}{0.687014in}}{\pgfqpoint{1.377161in}{0.687014in}}%
\pgfpathcurveto{\pgfqpoint{1.366111in}{0.687014in}}{\pgfqpoint{1.355512in}{0.682624in}}{\pgfqpoint{1.347698in}{0.674810in}}%
\pgfpathcurveto{\pgfqpoint{1.339885in}{0.666997in}}{\pgfqpoint{1.335494in}{0.656398in}}{\pgfqpoint{1.335494in}{0.645348in}}%
\pgfpathcurveto{\pgfqpoint{1.335494in}{0.634297in}}{\pgfqpoint{1.339885in}{0.623698in}}{\pgfqpoint{1.347698in}{0.615885in}}%
\pgfpathcurveto{\pgfqpoint{1.355512in}{0.608071in}}{\pgfqpoint{1.366111in}{0.603681in}}{\pgfqpoint{1.377161in}{0.603681in}}%
\pgfpathclose%
\pgfusepath{stroke,fill}%
\end{pgfscope}%
\begin{pgfscope}%
\pgfpathrectangle{\pgfqpoint{0.750000in}{0.500000in}}{\pgfqpoint{4.650000in}{3.020000in}}%
\pgfusepath{clip}%
\pgfsetbuttcap%
\pgfsetroundjoin%
\definecolor{currentfill}{rgb}{0.121569,0.466667,0.705882}%
\pgfsetfillcolor{currentfill}%
\pgfsetlinewidth{1.003750pt}%
\definecolor{currentstroke}{rgb}{0.121569,0.466667,0.705882}%
\pgfsetstrokecolor{currentstroke}%
\pgfsetdash{}{0pt}%
\pgfpathmoveto{\pgfqpoint{1.238562in}{2.981729in}}%
\pgfpathcurveto{\pgfqpoint{1.249612in}{2.981729in}}{\pgfqpoint{1.260211in}{2.986119in}}{\pgfqpoint{1.268025in}{2.993933in}}%
\pgfpathcurveto{\pgfqpoint{1.275838in}{3.001747in}}{\pgfqpoint{1.280229in}{3.012346in}}{\pgfqpoint{1.280229in}{3.023396in}}%
\pgfpathcurveto{\pgfqpoint{1.280229in}{3.034446in}}{\pgfqpoint{1.275838in}{3.045045in}}{\pgfqpoint{1.268025in}{3.052859in}}%
\pgfpathcurveto{\pgfqpoint{1.260211in}{3.060672in}}{\pgfqpoint{1.249612in}{3.065062in}}{\pgfqpoint{1.238562in}{3.065062in}}%
\pgfpathcurveto{\pgfqpoint{1.227512in}{3.065062in}}{\pgfqpoint{1.216913in}{3.060672in}}{\pgfqpoint{1.209099in}{3.052859in}}%
\pgfpathcurveto{\pgfqpoint{1.201285in}{3.045045in}}{\pgfqpoint{1.196895in}{3.034446in}}{\pgfqpoint{1.196895in}{3.023396in}}%
\pgfpathcurveto{\pgfqpoint{1.196895in}{3.012346in}}{\pgfqpoint{1.201285in}{3.001747in}}{\pgfqpoint{1.209099in}{2.993933in}}%
\pgfpathcurveto{\pgfqpoint{1.216913in}{2.986119in}}{\pgfqpoint{1.227512in}{2.981729in}}{\pgfqpoint{1.238562in}{2.981729in}}%
\pgfpathclose%
\pgfusepath{stroke,fill}%
\end{pgfscope}%
\begin{pgfscope}%
\pgfpathrectangle{\pgfqpoint{0.750000in}{0.500000in}}{\pgfqpoint{4.650000in}{3.020000in}}%
\pgfusepath{clip}%
\pgfsetbuttcap%
\pgfsetroundjoin%
\definecolor{currentfill}{rgb}{1.000000,0.498039,0.054902}%
\pgfsetfillcolor{currentfill}%
\pgfsetlinewidth{1.003750pt}%
\definecolor{currentstroke}{rgb}{1.000000,0.498039,0.054902}%
\pgfsetstrokecolor{currentstroke}%
\pgfsetdash{}{0pt}%
\pgfpathmoveto{\pgfqpoint{1.307861in}{3.018066in}}%
\pgfpathcurveto{\pgfqpoint{1.318912in}{3.018066in}}{\pgfqpoint{1.329511in}{3.022456in}}{\pgfqpoint{1.337324in}{3.030270in}}%
\pgfpathcurveto{\pgfqpoint{1.345138in}{3.038083in}}{\pgfqpoint{1.349528in}{3.048682in}}{\pgfqpoint{1.349528in}{3.059733in}}%
\pgfpathcurveto{\pgfqpoint{1.349528in}{3.070783in}}{\pgfqpoint{1.345138in}{3.081382in}}{\pgfqpoint{1.337324in}{3.089195in}}%
\pgfpathcurveto{\pgfqpoint{1.329511in}{3.097009in}}{\pgfqpoint{1.318912in}{3.101399in}}{\pgfqpoint{1.307861in}{3.101399in}}%
\pgfpathcurveto{\pgfqpoint{1.296811in}{3.101399in}}{\pgfqpoint{1.286212in}{3.097009in}}{\pgfqpoint{1.278399in}{3.089195in}}%
\pgfpathcurveto{\pgfqpoint{1.270585in}{3.081382in}}{\pgfqpoint{1.266195in}{3.070783in}}{\pgfqpoint{1.266195in}{3.059733in}}%
\pgfpathcurveto{\pgfqpoint{1.266195in}{3.048682in}}{\pgfqpoint{1.270585in}{3.038083in}}{\pgfqpoint{1.278399in}{3.030270in}}%
\pgfpathcurveto{\pgfqpoint{1.286212in}{3.022456in}}{\pgfqpoint{1.296811in}{3.018066in}}{\pgfqpoint{1.307861in}{3.018066in}}%
\pgfpathclose%
\pgfusepath{stroke,fill}%
\end{pgfscope}%
\begin{pgfscope}%
\pgfpathrectangle{\pgfqpoint{0.750000in}{0.500000in}}{\pgfqpoint{4.650000in}{3.020000in}}%
\pgfusepath{clip}%
\pgfsetbuttcap%
\pgfsetroundjoin%
\definecolor{currentfill}{rgb}{0.121569,0.466667,0.705882}%
\pgfsetfillcolor{currentfill}%
\pgfsetlinewidth{1.003750pt}%
\definecolor{currentstroke}{rgb}{0.121569,0.466667,0.705882}%
\pgfsetstrokecolor{currentstroke}%
\pgfsetdash{}{0pt}%
\pgfpathmoveto{\pgfqpoint{1.307861in}{0.595606in}}%
\pgfpathcurveto{\pgfqpoint{1.318912in}{0.595606in}}{\pgfqpoint{1.329511in}{0.599996in}}{\pgfqpoint{1.337324in}{0.607810in}}%
\pgfpathcurveto{\pgfqpoint{1.345138in}{0.615624in}}{\pgfqpoint{1.349528in}{0.626223in}}{\pgfqpoint{1.349528in}{0.637273in}}%
\pgfpathcurveto{\pgfqpoint{1.349528in}{0.648323in}}{\pgfqpoint{1.345138in}{0.658922in}}{\pgfqpoint{1.337324in}{0.666736in}}%
\pgfpathcurveto{\pgfqpoint{1.329511in}{0.674549in}}{\pgfqpoint{1.318912in}{0.678939in}}{\pgfqpoint{1.307861in}{0.678939in}}%
\pgfpathcurveto{\pgfqpoint{1.296811in}{0.678939in}}{\pgfqpoint{1.286212in}{0.674549in}}{\pgfqpoint{1.278399in}{0.666736in}}%
\pgfpathcurveto{\pgfqpoint{1.270585in}{0.658922in}}{\pgfqpoint{1.266195in}{0.648323in}}{\pgfqpoint{1.266195in}{0.637273in}}%
\pgfpathcurveto{\pgfqpoint{1.266195in}{0.626223in}}{\pgfqpoint{1.270585in}{0.615624in}}{\pgfqpoint{1.278399in}{0.607810in}}%
\pgfpathcurveto{\pgfqpoint{1.286212in}{0.599996in}}{\pgfqpoint{1.296811in}{0.595606in}}{\pgfqpoint{1.307861in}{0.595606in}}%
\pgfpathclose%
\pgfusepath{stroke,fill}%
\end{pgfscope}%
\begin{pgfscope}%
\pgfpathrectangle{\pgfqpoint{0.750000in}{0.500000in}}{\pgfqpoint{4.650000in}{3.020000in}}%
\pgfusepath{clip}%
\pgfsetbuttcap%
\pgfsetroundjoin%
\definecolor{currentfill}{rgb}{0.121569,0.466667,0.705882}%
\pgfsetfillcolor{currentfill}%
\pgfsetlinewidth{1.003750pt}%
\definecolor{currentstroke}{rgb}{0.121569,0.466667,0.705882}%
\pgfsetstrokecolor{currentstroke}%
\pgfsetdash{}{0pt}%
\pgfpathmoveto{\pgfqpoint{1.030663in}{0.595606in}}%
\pgfpathcurveto{\pgfqpoint{1.041713in}{0.595606in}}{\pgfqpoint{1.052312in}{0.599996in}}{\pgfqpoint{1.060126in}{0.607810in}}%
\pgfpathcurveto{\pgfqpoint{1.067940in}{0.615624in}}{\pgfqpoint{1.072330in}{0.626223in}}{\pgfqpoint{1.072330in}{0.637273in}}%
\pgfpathcurveto{\pgfqpoint{1.072330in}{0.648323in}}{\pgfqpoint{1.067940in}{0.658922in}}{\pgfqpoint{1.060126in}{0.666736in}}%
\pgfpathcurveto{\pgfqpoint{1.052312in}{0.674549in}}{\pgfqpoint{1.041713in}{0.678939in}}{\pgfqpoint{1.030663in}{0.678939in}}%
\pgfpathcurveto{\pgfqpoint{1.019613in}{0.678939in}}{\pgfqpoint{1.009014in}{0.674549in}}{\pgfqpoint{1.001200in}{0.666736in}}%
\pgfpathcurveto{\pgfqpoint{0.993387in}{0.658922in}}{\pgfqpoint{0.988997in}{0.648323in}}{\pgfqpoint{0.988997in}{0.637273in}}%
\pgfpathcurveto{\pgfqpoint{0.988997in}{0.626223in}}{\pgfqpoint{0.993387in}{0.615624in}}{\pgfqpoint{1.001200in}{0.607810in}}%
\pgfpathcurveto{\pgfqpoint{1.009014in}{0.599996in}}{\pgfqpoint{1.019613in}{0.595606in}}{\pgfqpoint{1.030663in}{0.595606in}}%
\pgfpathclose%
\pgfusepath{stroke,fill}%
\end{pgfscope}%
\begin{pgfscope}%
\pgfpathrectangle{\pgfqpoint{0.750000in}{0.500000in}}{\pgfqpoint{4.650000in}{3.020000in}}%
\pgfusepath{clip}%
\pgfsetbuttcap%
\pgfsetroundjoin%
\definecolor{currentfill}{rgb}{1.000000,0.498039,0.054902}%
\pgfsetfillcolor{currentfill}%
\pgfsetlinewidth{1.003750pt}%
\definecolor{currentstroke}{rgb}{1.000000,0.498039,0.054902}%
\pgfsetstrokecolor{currentstroke}%
\pgfsetdash{}{0pt}%
\pgfpathmoveto{\pgfqpoint{2.070156in}{3.018066in}}%
\pgfpathcurveto{\pgfqpoint{2.081207in}{3.018066in}}{\pgfqpoint{2.091806in}{3.022456in}}{\pgfqpoint{2.099619in}{3.030270in}}%
\pgfpathcurveto{\pgfqpoint{2.107433in}{3.038083in}}{\pgfqpoint{2.111823in}{3.048682in}}{\pgfqpoint{2.111823in}{3.059733in}}%
\pgfpathcurveto{\pgfqpoint{2.111823in}{3.070783in}}{\pgfqpoint{2.107433in}{3.081382in}}{\pgfqpoint{2.099619in}{3.089195in}}%
\pgfpathcurveto{\pgfqpoint{2.091806in}{3.097009in}}{\pgfqpoint{2.081207in}{3.101399in}}{\pgfqpoint{2.070156in}{3.101399in}}%
\pgfpathcurveto{\pgfqpoint{2.059106in}{3.101399in}}{\pgfqpoint{2.048507in}{3.097009in}}{\pgfqpoint{2.040694in}{3.089195in}}%
\pgfpathcurveto{\pgfqpoint{2.032880in}{3.081382in}}{\pgfqpoint{2.028490in}{3.070783in}}{\pgfqpoint{2.028490in}{3.059733in}}%
\pgfpathcurveto{\pgfqpoint{2.028490in}{3.048682in}}{\pgfqpoint{2.032880in}{3.038083in}}{\pgfqpoint{2.040694in}{3.030270in}}%
\pgfpathcurveto{\pgfqpoint{2.048507in}{3.022456in}}{\pgfqpoint{2.059106in}{3.018066in}}{\pgfqpoint{2.070156in}{3.018066in}}%
\pgfpathclose%
\pgfusepath{stroke,fill}%
\end{pgfscope}%
\begin{pgfscope}%
\pgfpathrectangle{\pgfqpoint{0.750000in}{0.500000in}}{\pgfqpoint{4.650000in}{3.020000in}}%
\pgfusepath{clip}%
\pgfsetbuttcap%
\pgfsetroundjoin%
\definecolor{currentfill}{rgb}{0.839216,0.152941,0.156863}%
\pgfsetfillcolor{currentfill}%
\pgfsetlinewidth{1.003750pt}%
\definecolor{currentstroke}{rgb}{0.839216,0.152941,0.156863}%
\pgfsetstrokecolor{currentstroke}%
\pgfsetdash{}{0pt}%
\pgfpathmoveto{\pgfqpoint{0.961364in}{3.284537in}}%
\pgfpathcurveto{\pgfqpoint{0.972414in}{3.284537in}}{\pgfqpoint{0.983013in}{3.288927in}}{\pgfqpoint{0.990826in}{3.296740in}}%
\pgfpathcurveto{\pgfqpoint{0.998640in}{3.304554in}}{\pgfqpoint{1.003030in}{3.315153in}}{\pgfqpoint{1.003030in}{3.326203in}}%
\pgfpathcurveto{\pgfqpoint{1.003030in}{3.337253in}}{\pgfqpoint{0.998640in}{3.347852in}}{\pgfqpoint{0.990826in}{3.355666in}}%
\pgfpathcurveto{\pgfqpoint{0.983013in}{3.363480in}}{\pgfqpoint{0.972414in}{3.367870in}}{\pgfqpoint{0.961364in}{3.367870in}}%
\pgfpathcurveto{\pgfqpoint{0.950314in}{3.367870in}}{\pgfqpoint{0.939714in}{3.363480in}}{\pgfqpoint{0.931901in}{3.355666in}}%
\pgfpathcurveto{\pgfqpoint{0.924087in}{3.347852in}}{\pgfqpoint{0.919697in}{3.337253in}}{\pgfqpoint{0.919697in}{3.326203in}}%
\pgfpathcurveto{\pgfqpoint{0.919697in}{3.315153in}}{\pgfqpoint{0.924087in}{3.304554in}}{\pgfqpoint{0.931901in}{3.296740in}}%
\pgfpathcurveto{\pgfqpoint{0.939714in}{3.288927in}}{\pgfqpoint{0.950314in}{3.284537in}}{\pgfqpoint{0.961364in}{3.284537in}}%
\pgfpathclose%
\pgfusepath{stroke,fill}%
\end{pgfscope}%
\begin{pgfscope}%
\pgfsetbuttcap%
\pgfsetroundjoin%
\definecolor{currentfill}{rgb}{0.000000,0.000000,0.000000}%
\pgfsetfillcolor{currentfill}%
\pgfsetlinewidth{0.803000pt}%
\definecolor{currentstroke}{rgb}{0.000000,0.000000,0.000000}%
\pgfsetstrokecolor{currentstroke}%
\pgfsetdash{}{0pt}%
\pgfsys@defobject{currentmarker}{\pgfqpoint{0.000000in}{-0.048611in}}{\pgfqpoint{0.000000in}{0.000000in}}{%
\pgfpathmoveto{\pgfqpoint{0.000000in}{0.000000in}}%
\pgfpathlineto{\pgfqpoint{0.000000in}{-0.048611in}}%
\pgfusepath{stroke,fill}%
}%
\begin{pgfscope}%
\pgfsys@transformshift{0.892064in}{0.500000in}%
\pgfsys@useobject{currentmarker}{}%
\end{pgfscope}%
\end{pgfscope}%
\begin{pgfscope}%
\definecolor{textcolor}{rgb}{0.000000,0.000000,0.000000}%
\pgfsetstrokecolor{textcolor}%
\pgfsetfillcolor{textcolor}%
\pgftext[x=0.892064in,y=0.402778in,,top]{\color{textcolor}\sffamily\fontsize{10.000000}{12.000000}\selectfont \(\displaystyle {0}\)}%
\end{pgfscope}%
\begin{pgfscope}%
\pgfsetbuttcap%
\pgfsetroundjoin%
\definecolor{currentfill}{rgb}{0.000000,0.000000,0.000000}%
\pgfsetfillcolor{currentfill}%
\pgfsetlinewidth{0.803000pt}%
\definecolor{currentstroke}{rgb}{0.000000,0.000000,0.000000}%
\pgfsetstrokecolor{currentstroke}%
\pgfsetdash{}{0pt}%
\pgfsys@defobject{currentmarker}{\pgfqpoint{0.000000in}{-0.048611in}}{\pgfqpoint{0.000000in}{0.000000in}}{%
\pgfpathmoveto{\pgfqpoint{0.000000in}{0.000000in}}%
\pgfpathlineto{\pgfqpoint{0.000000in}{-0.048611in}}%
\pgfusepath{stroke,fill}%
}%
\begin{pgfscope}%
\pgfsys@transformshift{1.585060in}{0.500000in}%
\pgfsys@useobject{currentmarker}{}%
\end{pgfscope}%
\end{pgfscope}%
\begin{pgfscope}%
\definecolor{textcolor}{rgb}{0.000000,0.000000,0.000000}%
\pgfsetstrokecolor{textcolor}%
\pgfsetfillcolor{textcolor}%
\pgftext[x=1.585060in,y=0.402778in,,top]{\color{textcolor}\sffamily\fontsize{10.000000}{12.000000}\selectfont \(\displaystyle {10}\)}%
\end{pgfscope}%
\begin{pgfscope}%
\pgfsetbuttcap%
\pgfsetroundjoin%
\definecolor{currentfill}{rgb}{0.000000,0.000000,0.000000}%
\pgfsetfillcolor{currentfill}%
\pgfsetlinewidth{0.803000pt}%
\definecolor{currentstroke}{rgb}{0.000000,0.000000,0.000000}%
\pgfsetstrokecolor{currentstroke}%
\pgfsetdash{}{0pt}%
\pgfsys@defobject{currentmarker}{\pgfqpoint{0.000000in}{-0.048611in}}{\pgfqpoint{0.000000in}{0.000000in}}{%
\pgfpathmoveto{\pgfqpoint{0.000000in}{0.000000in}}%
\pgfpathlineto{\pgfqpoint{0.000000in}{-0.048611in}}%
\pgfusepath{stroke,fill}%
}%
\begin{pgfscope}%
\pgfsys@transformshift{2.278055in}{0.500000in}%
\pgfsys@useobject{currentmarker}{}%
\end{pgfscope}%
\end{pgfscope}%
\begin{pgfscope}%
\definecolor{textcolor}{rgb}{0.000000,0.000000,0.000000}%
\pgfsetstrokecolor{textcolor}%
\pgfsetfillcolor{textcolor}%
\pgftext[x=2.278055in,y=0.402778in,,top]{\color{textcolor}\sffamily\fontsize{10.000000}{12.000000}\selectfont \(\displaystyle {20}\)}%
\end{pgfscope}%
\begin{pgfscope}%
\pgfsetbuttcap%
\pgfsetroundjoin%
\definecolor{currentfill}{rgb}{0.000000,0.000000,0.000000}%
\pgfsetfillcolor{currentfill}%
\pgfsetlinewidth{0.803000pt}%
\definecolor{currentstroke}{rgb}{0.000000,0.000000,0.000000}%
\pgfsetstrokecolor{currentstroke}%
\pgfsetdash{}{0pt}%
\pgfsys@defobject{currentmarker}{\pgfqpoint{0.000000in}{-0.048611in}}{\pgfqpoint{0.000000in}{0.000000in}}{%
\pgfpathmoveto{\pgfqpoint{0.000000in}{0.000000in}}%
\pgfpathlineto{\pgfqpoint{0.000000in}{-0.048611in}}%
\pgfusepath{stroke,fill}%
}%
\begin{pgfscope}%
\pgfsys@transformshift{2.971051in}{0.500000in}%
\pgfsys@useobject{currentmarker}{}%
\end{pgfscope}%
\end{pgfscope}%
\begin{pgfscope}%
\definecolor{textcolor}{rgb}{0.000000,0.000000,0.000000}%
\pgfsetstrokecolor{textcolor}%
\pgfsetfillcolor{textcolor}%
\pgftext[x=2.971051in,y=0.402778in,,top]{\color{textcolor}\sffamily\fontsize{10.000000}{12.000000}\selectfont \(\displaystyle {30}\)}%
\end{pgfscope}%
\begin{pgfscope}%
\pgfsetbuttcap%
\pgfsetroundjoin%
\definecolor{currentfill}{rgb}{0.000000,0.000000,0.000000}%
\pgfsetfillcolor{currentfill}%
\pgfsetlinewidth{0.803000pt}%
\definecolor{currentstroke}{rgb}{0.000000,0.000000,0.000000}%
\pgfsetstrokecolor{currentstroke}%
\pgfsetdash{}{0pt}%
\pgfsys@defobject{currentmarker}{\pgfqpoint{0.000000in}{-0.048611in}}{\pgfqpoint{0.000000in}{0.000000in}}{%
\pgfpathmoveto{\pgfqpoint{0.000000in}{0.000000in}}%
\pgfpathlineto{\pgfqpoint{0.000000in}{-0.048611in}}%
\pgfusepath{stroke,fill}%
}%
\begin{pgfscope}%
\pgfsys@transformshift{3.664046in}{0.500000in}%
\pgfsys@useobject{currentmarker}{}%
\end{pgfscope}%
\end{pgfscope}%
\begin{pgfscope}%
\definecolor{textcolor}{rgb}{0.000000,0.000000,0.000000}%
\pgfsetstrokecolor{textcolor}%
\pgfsetfillcolor{textcolor}%
\pgftext[x=3.664046in,y=0.402778in,,top]{\color{textcolor}\sffamily\fontsize{10.000000}{12.000000}\selectfont \(\displaystyle {40}\)}%
\end{pgfscope}%
\begin{pgfscope}%
\pgfsetbuttcap%
\pgfsetroundjoin%
\definecolor{currentfill}{rgb}{0.000000,0.000000,0.000000}%
\pgfsetfillcolor{currentfill}%
\pgfsetlinewidth{0.803000pt}%
\definecolor{currentstroke}{rgb}{0.000000,0.000000,0.000000}%
\pgfsetstrokecolor{currentstroke}%
\pgfsetdash{}{0pt}%
\pgfsys@defobject{currentmarker}{\pgfqpoint{0.000000in}{-0.048611in}}{\pgfqpoint{0.000000in}{0.000000in}}{%
\pgfpathmoveto{\pgfqpoint{0.000000in}{0.000000in}}%
\pgfpathlineto{\pgfqpoint{0.000000in}{-0.048611in}}%
\pgfusepath{stroke,fill}%
}%
\begin{pgfscope}%
\pgfsys@transformshift{4.357042in}{0.500000in}%
\pgfsys@useobject{currentmarker}{}%
\end{pgfscope}%
\end{pgfscope}%
\begin{pgfscope}%
\definecolor{textcolor}{rgb}{0.000000,0.000000,0.000000}%
\pgfsetstrokecolor{textcolor}%
\pgfsetfillcolor{textcolor}%
\pgftext[x=4.357042in,y=0.402778in,,top]{\color{textcolor}\sffamily\fontsize{10.000000}{12.000000}\selectfont \(\displaystyle {50}\)}%
\end{pgfscope}%
\begin{pgfscope}%
\pgfsetbuttcap%
\pgfsetroundjoin%
\definecolor{currentfill}{rgb}{0.000000,0.000000,0.000000}%
\pgfsetfillcolor{currentfill}%
\pgfsetlinewidth{0.803000pt}%
\definecolor{currentstroke}{rgb}{0.000000,0.000000,0.000000}%
\pgfsetstrokecolor{currentstroke}%
\pgfsetdash{}{0pt}%
\pgfsys@defobject{currentmarker}{\pgfqpoint{0.000000in}{-0.048611in}}{\pgfqpoint{0.000000in}{0.000000in}}{%
\pgfpathmoveto{\pgfqpoint{0.000000in}{0.000000in}}%
\pgfpathlineto{\pgfqpoint{0.000000in}{-0.048611in}}%
\pgfusepath{stroke,fill}%
}%
\begin{pgfscope}%
\pgfsys@transformshift{5.050037in}{0.500000in}%
\pgfsys@useobject{currentmarker}{}%
\end{pgfscope}%
\end{pgfscope}%
\begin{pgfscope}%
\definecolor{textcolor}{rgb}{0.000000,0.000000,0.000000}%
\pgfsetstrokecolor{textcolor}%
\pgfsetfillcolor{textcolor}%
\pgftext[x=5.050037in,y=0.402778in,,top]{\color{textcolor}\sffamily\fontsize{10.000000}{12.000000}\selectfont \(\displaystyle {60}\)}%
\end{pgfscope}%
\begin{pgfscope}%
\definecolor{textcolor}{rgb}{0.000000,0.000000,0.000000}%
\pgfsetstrokecolor{textcolor}%
\pgfsetfillcolor{textcolor}%
\pgftext[x=3.075000in,y=0.223889in,,top]{\color{textcolor}\sffamily\fontsize{10.000000}{12.000000}\selectfont Number of Sinks}%
\end{pgfscope}%
\begin{pgfscope}%
\pgfsetbuttcap%
\pgfsetroundjoin%
\definecolor{currentfill}{rgb}{0.000000,0.000000,0.000000}%
\pgfsetfillcolor{currentfill}%
\pgfsetlinewidth{0.803000pt}%
\definecolor{currentstroke}{rgb}{0.000000,0.000000,0.000000}%
\pgfsetstrokecolor{currentstroke}%
\pgfsetdash{}{0pt}%
\pgfsys@defobject{currentmarker}{\pgfqpoint{-0.048611in}{0.000000in}}{\pgfqpoint{0.000000in}{0.000000in}}{%
\pgfpathmoveto{\pgfqpoint{0.000000in}{0.000000in}}%
\pgfpathlineto{\pgfqpoint{-0.048611in}{0.000000in}}%
\pgfusepath{stroke,fill}%
}%
\begin{pgfscope}%
\pgfsys@transformshift{0.750000in}{0.637273in}%
\pgfsys@useobject{currentmarker}{}%
\end{pgfscope}%
\end{pgfscope}%
\begin{pgfscope}%
\definecolor{textcolor}{rgb}{0.000000,0.000000,0.000000}%
\pgfsetstrokecolor{textcolor}%
\pgfsetfillcolor{textcolor}%
\pgftext[x=0.583333in, y=0.589078in, left, base]{\color{textcolor}\sffamily\fontsize{10.000000}{12.000000}\selectfont \(\displaystyle {0}\)}%
\end{pgfscope}%
\begin{pgfscope}%
\pgfsetbuttcap%
\pgfsetroundjoin%
\definecolor{currentfill}{rgb}{0.000000,0.000000,0.000000}%
\pgfsetfillcolor{currentfill}%
\pgfsetlinewidth{0.803000pt}%
\definecolor{currentstroke}{rgb}{0.000000,0.000000,0.000000}%
\pgfsetstrokecolor{currentstroke}%
\pgfsetdash{}{0pt}%
\pgfsys@defobject{currentmarker}{\pgfqpoint{-0.048611in}{0.000000in}}{\pgfqpoint{0.000000in}{0.000000in}}{%
\pgfpathmoveto{\pgfqpoint{0.000000in}{0.000000in}}%
\pgfpathlineto{\pgfqpoint{-0.048611in}{0.000000in}}%
\pgfusepath{stroke,fill}%
}%
\begin{pgfscope}%
\pgfsys@transformshift{0.750000in}{1.041016in}%
\pgfsys@useobject{currentmarker}{}%
\end{pgfscope}%
\end{pgfscope}%
\begin{pgfscope}%
\definecolor{textcolor}{rgb}{0.000000,0.000000,0.000000}%
\pgfsetstrokecolor{textcolor}%
\pgfsetfillcolor{textcolor}%
\pgftext[x=0.444444in, y=0.992822in, left, base]{\color{textcolor}\sffamily\fontsize{10.000000}{12.000000}\selectfont \(\displaystyle {100}\)}%
\end{pgfscope}%
\begin{pgfscope}%
\pgfsetbuttcap%
\pgfsetroundjoin%
\definecolor{currentfill}{rgb}{0.000000,0.000000,0.000000}%
\pgfsetfillcolor{currentfill}%
\pgfsetlinewidth{0.803000pt}%
\definecolor{currentstroke}{rgb}{0.000000,0.000000,0.000000}%
\pgfsetstrokecolor{currentstroke}%
\pgfsetdash{}{0pt}%
\pgfsys@defobject{currentmarker}{\pgfqpoint{-0.048611in}{0.000000in}}{\pgfqpoint{0.000000in}{0.000000in}}{%
\pgfpathmoveto{\pgfqpoint{0.000000in}{0.000000in}}%
\pgfpathlineto{\pgfqpoint{-0.048611in}{0.000000in}}%
\pgfusepath{stroke,fill}%
}%
\begin{pgfscope}%
\pgfsys@transformshift{0.750000in}{1.444759in}%
\pgfsys@useobject{currentmarker}{}%
\end{pgfscope}%
\end{pgfscope}%
\begin{pgfscope}%
\definecolor{textcolor}{rgb}{0.000000,0.000000,0.000000}%
\pgfsetstrokecolor{textcolor}%
\pgfsetfillcolor{textcolor}%
\pgftext[x=0.444444in, y=1.396565in, left, base]{\color{textcolor}\sffamily\fontsize{10.000000}{12.000000}\selectfont \(\displaystyle {200}\)}%
\end{pgfscope}%
\begin{pgfscope}%
\pgfsetbuttcap%
\pgfsetroundjoin%
\definecolor{currentfill}{rgb}{0.000000,0.000000,0.000000}%
\pgfsetfillcolor{currentfill}%
\pgfsetlinewidth{0.803000pt}%
\definecolor{currentstroke}{rgb}{0.000000,0.000000,0.000000}%
\pgfsetstrokecolor{currentstroke}%
\pgfsetdash{}{0pt}%
\pgfsys@defobject{currentmarker}{\pgfqpoint{-0.048611in}{0.000000in}}{\pgfqpoint{0.000000in}{0.000000in}}{%
\pgfpathmoveto{\pgfqpoint{0.000000in}{0.000000in}}%
\pgfpathlineto{\pgfqpoint{-0.048611in}{0.000000in}}%
\pgfusepath{stroke,fill}%
}%
\begin{pgfscope}%
\pgfsys@transformshift{0.750000in}{1.848503in}%
\pgfsys@useobject{currentmarker}{}%
\end{pgfscope}%
\end{pgfscope}%
\begin{pgfscope}%
\definecolor{textcolor}{rgb}{0.000000,0.000000,0.000000}%
\pgfsetstrokecolor{textcolor}%
\pgfsetfillcolor{textcolor}%
\pgftext[x=0.444444in, y=1.800308in, left, base]{\color{textcolor}\sffamily\fontsize{10.000000}{12.000000}\selectfont \(\displaystyle {300}\)}%
\end{pgfscope}%
\begin{pgfscope}%
\pgfsetbuttcap%
\pgfsetroundjoin%
\definecolor{currentfill}{rgb}{0.000000,0.000000,0.000000}%
\pgfsetfillcolor{currentfill}%
\pgfsetlinewidth{0.803000pt}%
\definecolor{currentstroke}{rgb}{0.000000,0.000000,0.000000}%
\pgfsetstrokecolor{currentstroke}%
\pgfsetdash{}{0pt}%
\pgfsys@defobject{currentmarker}{\pgfqpoint{-0.048611in}{0.000000in}}{\pgfqpoint{0.000000in}{0.000000in}}{%
\pgfpathmoveto{\pgfqpoint{0.000000in}{0.000000in}}%
\pgfpathlineto{\pgfqpoint{-0.048611in}{0.000000in}}%
\pgfusepath{stroke,fill}%
}%
\begin{pgfscope}%
\pgfsys@transformshift{0.750000in}{2.252246in}%
\pgfsys@useobject{currentmarker}{}%
\end{pgfscope}%
\end{pgfscope}%
\begin{pgfscope}%
\definecolor{textcolor}{rgb}{0.000000,0.000000,0.000000}%
\pgfsetstrokecolor{textcolor}%
\pgfsetfillcolor{textcolor}%
\pgftext[x=0.444444in, y=2.204052in, left, base]{\color{textcolor}\sffamily\fontsize{10.000000}{12.000000}\selectfont \(\displaystyle {400}\)}%
\end{pgfscope}%
\begin{pgfscope}%
\pgfsetbuttcap%
\pgfsetroundjoin%
\definecolor{currentfill}{rgb}{0.000000,0.000000,0.000000}%
\pgfsetfillcolor{currentfill}%
\pgfsetlinewidth{0.803000pt}%
\definecolor{currentstroke}{rgb}{0.000000,0.000000,0.000000}%
\pgfsetstrokecolor{currentstroke}%
\pgfsetdash{}{0pt}%
\pgfsys@defobject{currentmarker}{\pgfqpoint{-0.048611in}{0.000000in}}{\pgfqpoint{0.000000in}{0.000000in}}{%
\pgfpathmoveto{\pgfqpoint{0.000000in}{0.000000in}}%
\pgfpathlineto{\pgfqpoint{-0.048611in}{0.000000in}}%
\pgfusepath{stroke,fill}%
}%
\begin{pgfscope}%
\pgfsys@transformshift{0.750000in}{2.655989in}%
\pgfsys@useobject{currentmarker}{}%
\end{pgfscope}%
\end{pgfscope}%
\begin{pgfscope}%
\definecolor{textcolor}{rgb}{0.000000,0.000000,0.000000}%
\pgfsetstrokecolor{textcolor}%
\pgfsetfillcolor{textcolor}%
\pgftext[x=0.444444in, y=2.607795in, left, base]{\color{textcolor}\sffamily\fontsize{10.000000}{12.000000}\selectfont \(\displaystyle {500}\)}%
\end{pgfscope}%
\begin{pgfscope}%
\pgfsetbuttcap%
\pgfsetroundjoin%
\definecolor{currentfill}{rgb}{0.000000,0.000000,0.000000}%
\pgfsetfillcolor{currentfill}%
\pgfsetlinewidth{0.803000pt}%
\definecolor{currentstroke}{rgb}{0.000000,0.000000,0.000000}%
\pgfsetstrokecolor{currentstroke}%
\pgfsetdash{}{0pt}%
\pgfsys@defobject{currentmarker}{\pgfqpoint{-0.048611in}{0.000000in}}{\pgfqpoint{0.000000in}{0.000000in}}{%
\pgfpathmoveto{\pgfqpoint{0.000000in}{0.000000in}}%
\pgfpathlineto{\pgfqpoint{-0.048611in}{0.000000in}}%
\pgfusepath{stroke,fill}%
}%
\begin{pgfscope}%
\pgfsys@transformshift{0.750000in}{3.059733in}%
\pgfsys@useobject{currentmarker}{}%
\end{pgfscope}%
\end{pgfscope}%
\begin{pgfscope}%
\definecolor{textcolor}{rgb}{0.000000,0.000000,0.000000}%
\pgfsetstrokecolor{textcolor}%
\pgfsetfillcolor{textcolor}%
\pgftext[x=0.444444in, y=3.011538in, left, base]{\color{textcolor}\sffamily\fontsize{10.000000}{12.000000}\selectfont \(\displaystyle {600}\)}%
\end{pgfscope}%
\begin{pgfscope}%
\pgfsetbuttcap%
\pgfsetroundjoin%
\definecolor{currentfill}{rgb}{0.000000,0.000000,0.000000}%
\pgfsetfillcolor{currentfill}%
\pgfsetlinewidth{0.803000pt}%
\definecolor{currentstroke}{rgb}{0.000000,0.000000,0.000000}%
\pgfsetstrokecolor{currentstroke}%
\pgfsetdash{}{0pt}%
\pgfsys@defobject{currentmarker}{\pgfqpoint{-0.048611in}{0.000000in}}{\pgfqpoint{0.000000in}{0.000000in}}{%
\pgfpathmoveto{\pgfqpoint{0.000000in}{0.000000in}}%
\pgfpathlineto{\pgfqpoint{-0.048611in}{0.000000in}}%
\pgfusepath{stroke,fill}%
}%
\begin{pgfscope}%
\pgfsys@transformshift{0.750000in}{3.463476in}%
\pgfsys@useobject{currentmarker}{}%
\end{pgfscope}%
\end{pgfscope}%
\begin{pgfscope}%
\definecolor{textcolor}{rgb}{0.000000,0.000000,0.000000}%
\pgfsetstrokecolor{textcolor}%
\pgfsetfillcolor{textcolor}%
\pgftext[x=0.444444in, y=3.415281in, left, base]{\color{textcolor}\sffamily\fontsize{10.000000}{12.000000}\selectfont \(\displaystyle {700}\)}%
\end{pgfscope}%
\begin{pgfscope}%
\definecolor{textcolor}{rgb}{0.000000,0.000000,0.000000}%
\pgfsetstrokecolor{textcolor}%
\pgfsetfillcolor{textcolor}%
\pgftext[x=0.388888in,y=2.010000in,,bottom,rotate=90.000000]{\color{textcolor}\sffamily\fontsize{10.000000}{12.000000}\selectfont Dataflow Time}%
\end{pgfscope}%
\begin{pgfscope}%
\pgfsetrectcap%
\pgfsetmiterjoin%
\pgfsetlinewidth{0.803000pt}%
\definecolor{currentstroke}{rgb}{0.000000,0.000000,0.000000}%
\pgfsetstrokecolor{currentstroke}%
\pgfsetdash{}{0pt}%
\pgfpathmoveto{\pgfqpoint{0.750000in}{0.500000in}}%
\pgfpathlineto{\pgfqpoint{0.750000in}{3.520000in}}%
\pgfusepath{stroke}%
\end{pgfscope}%
\begin{pgfscope}%
\pgfsetrectcap%
\pgfsetmiterjoin%
\pgfsetlinewidth{0.803000pt}%
\definecolor{currentstroke}{rgb}{0.000000,0.000000,0.000000}%
\pgfsetstrokecolor{currentstroke}%
\pgfsetdash{}{0pt}%
\pgfpathmoveto{\pgfqpoint{5.400000in}{0.500000in}}%
\pgfpathlineto{\pgfqpoint{5.400000in}{3.520000in}}%
\pgfusepath{stroke}%
\end{pgfscope}%
\begin{pgfscope}%
\pgfsetrectcap%
\pgfsetmiterjoin%
\pgfsetlinewidth{0.803000pt}%
\definecolor{currentstroke}{rgb}{0.000000,0.000000,0.000000}%
\pgfsetstrokecolor{currentstroke}%
\pgfsetdash{}{0pt}%
\pgfpathmoveto{\pgfqpoint{0.750000in}{0.500000in}}%
\pgfpathlineto{\pgfqpoint{5.400000in}{0.500000in}}%
\pgfusepath{stroke}%
\end{pgfscope}%
\begin{pgfscope}%
\pgfsetrectcap%
\pgfsetmiterjoin%
\pgfsetlinewidth{0.803000pt}%
\definecolor{currentstroke}{rgb}{0.000000,0.000000,0.000000}%
\pgfsetstrokecolor{currentstroke}%
\pgfsetdash{}{0pt}%
\pgfpathmoveto{\pgfqpoint{0.750000in}{3.520000in}}%
\pgfpathlineto{\pgfqpoint{5.400000in}{3.520000in}}%
\pgfusepath{stroke}%
\end{pgfscope}%
\begin{pgfscope}%
\definecolor{textcolor}{rgb}{0.000000,0.000000,0.000000}%
\pgfsetstrokecolor{textcolor}%
\pgfsetfillcolor{textcolor}%
\pgftext[x=3.075000in,y=3.603333in,,base]{\color{textcolor}\sffamily\fontsize{12.000000}{14.400000}\selectfont Forwards}%
\end{pgfscope}%
\begin{pgfscope}%
\pgfsetbuttcap%
\pgfsetmiterjoin%
\definecolor{currentfill}{rgb}{1.000000,1.000000,1.000000}%
\pgfsetfillcolor{currentfill}%
\pgfsetfillopacity{0.800000}%
\pgfsetlinewidth{1.003750pt}%
\definecolor{currentstroke}{rgb}{0.800000,0.800000,0.800000}%
\pgfsetstrokecolor{currentstroke}%
\pgfsetstrokeopacity{0.800000}%
\pgfsetdash{}{0pt}%
\pgfpathmoveto{\pgfqpoint{3.850417in}{0.569444in}}%
\pgfpathlineto{\pgfqpoint{5.302778in}{0.569444in}}%
\pgfpathquadraticcurveto{\pgfqpoint{5.330556in}{0.569444in}}{\pgfqpoint{5.330556in}{0.597222in}}%
\pgfpathlineto{\pgfqpoint{5.330556in}{1.165694in}}%
\pgfpathquadraticcurveto{\pgfqpoint{5.330556in}{1.193472in}}{\pgfqpoint{5.302778in}{1.193472in}}%
\pgfpathlineto{\pgfqpoint{3.850417in}{1.193472in}}%
\pgfpathquadraticcurveto{\pgfqpoint{3.822639in}{1.193472in}}{\pgfqpoint{3.822639in}{1.165694in}}%
\pgfpathlineto{\pgfqpoint{3.822639in}{0.597222in}}%
\pgfpathquadraticcurveto{\pgfqpoint{3.822639in}{0.569444in}}{\pgfqpoint{3.850417in}{0.569444in}}%
\pgfpathclose%
\pgfusepath{stroke,fill}%
\end{pgfscope}%
\begin{pgfscope}%
\pgfsetbuttcap%
\pgfsetroundjoin%
\definecolor{currentfill}{rgb}{0.121569,0.466667,0.705882}%
\pgfsetfillcolor{currentfill}%
\pgfsetlinewidth{1.003750pt}%
\definecolor{currentstroke}{rgb}{0.121569,0.466667,0.705882}%
\pgfsetstrokecolor{currentstroke}%
\pgfsetdash{}{0pt}%
\pgfsys@defobject{currentmarker}{\pgfqpoint{-0.034722in}{-0.034722in}}{\pgfqpoint{0.034722in}{0.034722in}}{%
\pgfpathmoveto{\pgfqpoint{0.000000in}{-0.034722in}}%
\pgfpathcurveto{\pgfqpoint{0.009208in}{-0.034722in}}{\pgfqpoint{0.018041in}{-0.031064in}}{\pgfqpoint{0.024552in}{-0.024552in}}%
\pgfpathcurveto{\pgfqpoint{0.031064in}{-0.018041in}}{\pgfqpoint{0.034722in}{-0.009208in}}{\pgfqpoint{0.034722in}{0.000000in}}%
\pgfpathcurveto{\pgfqpoint{0.034722in}{0.009208in}}{\pgfqpoint{0.031064in}{0.018041in}}{\pgfqpoint{0.024552in}{0.024552in}}%
\pgfpathcurveto{\pgfqpoint{0.018041in}{0.031064in}}{\pgfqpoint{0.009208in}{0.034722in}}{\pgfqpoint{0.000000in}{0.034722in}}%
\pgfpathcurveto{\pgfqpoint{-0.009208in}{0.034722in}}{\pgfqpoint{-0.018041in}{0.031064in}}{\pgfqpoint{-0.024552in}{0.024552in}}%
\pgfpathcurveto{\pgfqpoint{-0.031064in}{0.018041in}}{\pgfqpoint{-0.034722in}{0.009208in}}{\pgfqpoint{-0.034722in}{0.000000in}}%
\pgfpathcurveto{\pgfqpoint{-0.034722in}{-0.009208in}}{\pgfqpoint{-0.031064in}{-0.018041in}}{\pgfqpoint{-0.024552in}{-0.024552in}}%
\pgfpathcurveto{\pgfqpoint{-0.018041in}{-0.031064in}}{\pgfqpoint{-0.009208in}{-0.034722in}}{\pgfqpoint{0.000000in}{-0.034722in}}%
\pgfpathclose%
\pgfusepath{stroke,fill}%
}%
\begin{pgfscope}%
\pgfsys@transformshift{4.017083in}{1.089306in}%
\pgfsys@useobject{currentmarker}{}%
\end{pgfscope}%
\end{pgfscope}%
\begin{pgfscope}%
\definecolor{textcolor}{rgb}{0.000000,0.000000,0.000000}%
\pgfsetstrokecolor{textcolor}%
\pgfsetfillcolor{textcolor}%
\pgftext[x=4.267083in,y=1.040694in,left,base]{\color{textcolor}\sffamily\fontsize{10.000000}{12.000000}\selectfont No Timeout}%
\end{pgfscope}%
\begin{pgfscope}%
\pgfsetbuttcap%
\pgfsetroundjoin%
\definecolor{currentfill}{rgb}{1.000000,0.498039,0.054902}%
\pgfsetfillcolor{currentfill}%
\pgfsetlinewidth{1.003750pt}%
\definecolor{currentstroke}{rgb}{1.000000,0.498039,0.054902}%
\pgfsetstrokecolor{currentstroke}%
\pgfsetdash{}{0pt}%
\pgfsys@defobject{currentmarker}{\pgfqpoint{-0.034722in}{-0.034722in}}{\pgfqpoint{0.034722in}{0.034722in}}{%
\pgfpathmoveto{\pgfqpoint{0.000000in}{-0.034722in}}%
\pgfpathcurveto{\pgfqpoint{0.009208in}{-0.034722in}}{\pgfqpoint{0.018041in}{-0.031064in}}{\pgfqpoint{0.024552in}{-0.024552in}}%
\pgfpathcurveto{\pgfqpoint{0.031064in}{-0.018041in}}{\pgfqpoint{0.034722in}{-0.009208in}}{\pgfqpoint{0.034722in}{0.000000in}}%
\pgfpathcurveto{\pgfqpoint{0.034722in}{0.009208in}}{\pgfqpoint{0.031064in}{0.018041in}}{\pgfqpoint{0.024552in}{0.024552in}}%
\pgfpathcurveto{\pgfqpoint{0.018041in}{0.031064in}}{\pgfqpoint{0.009208in}{0.034722in}}{\pgfqpoint{0.000000in}{0.034722in}}%
\pgfpathcurveto{\pgfqpoint{-0.009208in}{0.034722in}}{\pgfqpoint{-0.018041in}{0.031064in}}{\pgfqpoint{-0.024552in}{0.024552in}}%
\pgfpathcurveto{\pgfqpoint{-0.031064in}{0.018041in}}{\pgfqpoint{-0.034722in}{0.009208in}}{\pgfqpoint{-0.034722in}{0.000000in}}%
\pgfpathcurveto{\pgfqpoint{-0.034722in}{-0.009208in}}{\pgfqpoint{-0.031064in}{-0.018041in}}{\pgfqpoint{-0.024552in}{-0.024552in}}%
\pgfpathcurveto{\pgfqpoint{-0.018041in}{-0.031064in}}{\pgfqpoint{-0.009208in}{-0.034722in}}{\pgfqpoint{0.000000in}{-0.034722in}}%
\pgfpathclose%
\pgfusepath{stroke,fill}%
}%
\begin{pgfscope}%
\pgfsys@transformshift{4.017083in}{0.895694in}%
\pgfsys@useobject{currentmarker}{}%
\end{pgfscope}%
\end{pgfscope}%
\begin{pgfscope}%
\definecolor{textcolor}{rgb}{0.000000,0.000000,0.000000}%
\pgfsetstrokecolor{textcolor}%
\pgfsetfillcolor{textcolor}%
\pgftext[x=4.267083in,y=0.847083in,left,base]{\color{textcolor}\sffamily\fontsize{10.000000}{12.000000}\selectfont Time Timeout}%
\end{pgfscope}%
\begin{pgfscope}%
\pgfsetbuttcap%
\pgfsetroundjoin%
\definecolor{currentfill}{rgb}{0.839216,0.152941,0.156863}%
\pgfsetfillcolor{currentfill}%
\pgfsetlinewidth{1.003750pt}%
\definecolor{currentstroke}{rgb}{0.839216,0.152941,0.156863}%
\pgfsetstrokecolor{currentstroke}%
\pgfsetdash{}{0pt}%
\pgfsys@defobject{currentmarker}{\pgfqpoint{-0.034722in}{-0.034722in}}{\pgfqpoint{0.034722in}{0.034722in}}{%
\pgfpathmoveto{\pgfqpoint{0.000000in}{-0.034722in}}%
\pgfpathcurveto{\pgfqpoint{0.009208in}{-0.034722in}}{\pgfqpoint{0.018041in}{-0.031064in}}{\pgfqpoint{0.024552in}{-0.024552in}}%
\pgfpathcurveto{\pgfqpoint{0.031064in}{-0.018041in}}{\pgfqpoint{0.034722in}{-0.009208in}}{\pgfqpoint{0.034722in}{0.000000in}}%
\pgfpathcurveto{\pgfqpoint{0.034722in}{0.009208in}}{\pgfqpoint{0.031064in}{0.018041in}}{\pgfqpoint{0.024552in}{0.024552in}}%
\pgfpathcurveto{\pgfqpoint{0.018041in}{0.031064in}}{\pgfqpoint{0.009208in}{0.034722in}}{\pgfqpoint{0.000000in}{0.034722in}}%
\pgfpathcurveto{\pgfqpoint{-0.009208in}{0.034722in}}{\pgfqpoint{-0.018041in}{0.031064in}}{\pgfqpoint{-0.024552in}{0.024552in}}%
\pgfpathcurveto{\pgfqpoint{-0.031064in}{0.018041in}}{\pgfqpoint{-0.034722in}{0.009208in}}{\pgfqpoint{-0.034722in}{0.000000in}}%
\pgfpathcurveto{\pgfqpoint{-0.034722in}{-0.009208in}}{\pgfqpoint{-0.031064in}{-0.018041in}}{\pgfqpoint{-0.024552in}{-0.024552in}}%
\pgfpathcurveto{\pgfqpoint{-0.018041in}{-0.031064in}}{\pgfqpoint{-0.009208in}{-0.034722in}}{\pgfqpoint{0.000000in}{-0.034722in}}%
\pgfpathclose%
\pgfusepath{stroke,fill}%
}%
\begin{pgfscope}%
\pgfsys@transformshift{4.017083in}{0.702083in}%
\pgfsys@useobject{currentmarker}{}%
\end{pgfscope}%
\end{pgfscope}%
\begin{pgfscope}%
\definecolor{textcolor}{rgb}{0.000000,0.000000,0.000000}%
\pgfsetstrokecolor{textcolor}%
\pgfsetfillcolor{textcolor}%
\pgftext[x=4.267083in,y=0.653472in,left,base]{\color{textcolor}\sffamily\fontsize{10.000000}{12.000000}\selectfont Memory Timeout}%
\end{pgfscope}%
\end{pgfpicture}%
\makeatother%
\endgroup%

            }
        \end{subfigure}
        \qquad
        \begin{subfigure}[]{0.45\textwidth}
            \centering
            \resizebox{\columnwidth}{!}{
                %% Creator: Matplotlib, PGF backend
%%
%% To include the figure in your LaTeX document, write
%%   \input{<filename>.pgf}
%%
%% Make sure the required packages are loaded in your preamble
%%   \usepackage{pgf}
%%
%% and, on pdftex
%%   \usepackage[utf8]{inputenc}\DeclareUnicodeCharacter{2212}{-}
%%
%% or, on luatex and xetex
%%   \usepackage{unicode-math}
%%
%% Figures using additional raster images can only be included by \input if
%% they are in the same directory as the main LaTeX file. For loading figures
%% from other directories you can use the `import` package
%%   \usepackage{import}
%%
%% and then include the figures with
%%   \import{<path to file>}{<filename>.pgf}
%%
%% Matplotlib used the following preamble
%%   \usepackage{fontspec}
%%
\begingroup%
\makeatletter%
\begin{pgfpicture}%
\pgfpathrectangle{\pgfpointorigin}{\pgfqpoint{6.000000in}{4.000000in}}%
\pgfusepath{use as bounding box, clip}%
\begin{pgfscope}%
\pgfsetbuttcap%
\pgfsetmiterjoin%
\definecolor{currentfill}{rgb}{1.000000,1.000000,1.000000}%
\pgfsetfillcolor{currentfill}%
\pgfsetlinewidth{0.000000pt}%
\definecolor{currentstroke}{rgb}{1.000000,1.000000,1.000000}%
\pgfsetstrokecolor{currentstroke}%
\pgfsetdash{}{0pt}%
\pgfpathmoveto{\pgfqpoint{0.000000in}{0.000000in}}%
\pgfpathlineto{\pgfqpoint{6.000000in}{0.000000in}}%
\pgfpathlineto{\pgfqpoint{6.000000in}{4.000000in}}%
\pgfpathlineto{\pgfqpoint{0.000000in}{4.000000in}}%
\pgfpathclose%
\pgfusepath{fill}%
\end{pgfscope}%
\begin{pgfscope}%
\pgfsetbuttcap%
\pgfsetmiterjoin%
\definecolor{currentfill}{rgb}{1.000000,1.000000,1.000000}%
\pgfsetfillcolor{currentfill}%
\pgfsetlinewidth{0.000000pt}%
\definecolor{currentstroke}{rgb}{0.000000,0.000000,0.000000}%
\pgfsetstrokecolor{currentstroke}%
\pgfsetstrokeopacity{0.000000}%
\pgfsetdash{}{0pt}%
\pgfpathmoveto{\pgfqpoint{0.750000in}{0.500000in}}%
\pgfpathlineto{\pgfqpoint{5.400000in}{0.500000in}}%
\pgfpathlineto{\pgfqpoint{5.400000in}{3.520000in}}%
\pgfpathlineto{\pgfqpoint{0.750000in}{3.520000in}}%
\pgfpathclose%
\pgfusepath{fill}%
\end{pgfscope}%
\begin{pgfscope}%
\pgfpathrectangle{\pgfqpoint{0.750000in}{0.500000in}}{\pgfqpoint{4.650000in}{3.020000in}}%
\pgfusepath{clip}%
\pgfsetbuttcap%
\pgfsetroundjoin%
\definecolor{currentfill}{rgb}{1.000000,0.498039,0.054902}%
\pgfsetfillcolor{currentfill}%
\pgfsetlinewidth{1.003750pt}%
\definecolor{currentstroke}{rgb}{1.000000,0.498039,0.054902}%
\pgfsetstrokecolor{currentstroke}%
\pgfsetdash{}{0pt}%
\pgfpathmoveto{\pgfqpoint{1.625649in}{2.987305in}}%
\pgfpathcurveto{\pgfqpoint{1.636699in}{2.987305in}}{\pgfqpoint{1.647299in}{2.991695in}}{\pgfqpoint{1.655112in}{2.999508in}}%
\pgfpathcurveto{\pgfqpoint{1.662926in}{3.007322in}}{\pgfqpoint{1.667316in}{3.017921in}}{\pgfqpoint{1.667316in}{3.028971in}}%
\pgfpathcurveto{\pgfqpoint{1.667316in}{3.040021in}}{\pgfqpoint{1.662926in}{3.050620in}}{\pgfqpoint{1.655112in}{3.058434in}}%
\pgfpathcurveto{\pgfqpoint{1.647299in}{3.066248in}}{\pgfqpoint{1.636699in}{3.070638in}}{\pgfqpoint{1.625649in}{3.070638in}}%
\pgfpathcurveto{\pgfqpoint{1.614599in}{3.070638in}}{\pgfqpoint{1.604000in}{3.066248in}}{\pgfqpoint{1.596187in}{3.058434in}}%
\pgfpathcurveto{\pgfqpoint{1.588373in}{3.050620in}}{\pgfqpoint{1.583983in}{3.040021in}}{\pgfqpoint{1.583983in}{3.028971in}}%
\pgfpathcurveto{\pgfqpoint{1.583983in}{3.017921in}}{\pgfqpoint{1.588373in}{3.007322in}}{\pgfqpoint{1.596187in}{2.999508in}}%
\pgfpathcurveto{\pgfqpoint{1.604000in}{2.991695in}}{\pgfqpoint{1.614599in}{2.987305in}}{\pgfqpoint{1.625649in}{2.987305in}}%
\pgfpathclose%
\pgfusepath{stroke,fill}%
\end{pgfscope}%
\begin{pgfscope}%
\pgfpathrectangle{\pgfqpoint{0.750000in}{0.500000in}}{\pgfqpoint{4.650000in}{3.020000in}}%
\pgfusepath{clip}%
\pgfsetbuttcap%
\pgfsetroundjoin%
\definecolor{currentfill}{rgb}{0.121569,0.466667,0.705882}%
\pgfsetfillcolor{currentfill}%
\pgfsetlinewidth{1.003750pt}%
\definecolor{currentstroke}{rgb}{0.121569,0.466667,0.705882}%
\pgfsetstrokecolor{currentstroke}%
\pgfsetdash{}{0pt}%
\pgfpathmoveto{\pgfqpoint{1.323701in}{0.618677in}}%
\pgfpathcurveto{\pgfqpoint{1.334751in}{0.618677in}}{\pgfqpoint{1.345350in}{0.623067in}}{\pgfqpoint{1.353164in}{0.630881in}}%
\pgfpathcurveto{\pgfqpoint{1.360978in}{0.638695in}}{\pgfqpoint{1.365368in}{0.649294in}}{\pgfqpoint{1.365368in}{0.660344in}}%
\pgfpathcurveto{\pgfqpoint{1.365368in}{0.671394in}}{\pgfqpoint{1.360978in}{0.681993in}}{\pgfqpoint{1.353164in}{0.689807in}}%
\pgfpathcurveto{\pgfqpoint{1.345350in}{0.697620in}}{\pgfqpoint{1.334751in}{0.702010in}}{\pgfqpoint{1.323701in}{0.702010in}}%
\pgfpathcurveto{\pgfqpoint{1.312651in}{0.702010in}}{\pgfqpoint{1.302052in}{0.697620in}}{\pgfqpoint{1.294239in}{0.689807in}}%
\pgfpathcurveto{\pgfqpoint{1.286425in}{0.681993in}}{\pgfqpoint{1.282035in}{0.671394in}}{\pgfqpoint{1.282035in}{0.660344in}}%
\pgfpathcurveto{\pgfqpoint{1.282035in}{0.649294in}}{\pgfqpoint{1.286425in}{0.638695in}}{\pgfqpoint{1.294239in}{0.630881in}}%
\pgfpathcurveto{\pgfqpoint{1.302052in}{0.623067in}}{\pgfqpoint{1.312651in}{0.618677in}}{\pgfqpoint{1.323701in}{0.618677in}}%
\pgfpathclose%
\pgfusepath{stroke,fill}%
\end{pgfscope}%
\begin{pgfscope}%
\pgfpathrectangle{\pgfqpoint{0.750000in}{0.500000in}}{\pgfqpoint{4.650000in}{3.020000in}}%
\pgfusepath{clip}%
\pgfsetbuttcap%
\pgfsetroundjoin%
\definecolor{currentfill}{rgb}{1.000000,0.498039,0.054902}%
\pgfsetfillcolor{currentfill}%
\pgfsetlinewidth{1.003750pt}%
\definecolor{currentstroke}{rgb}{1.000000,0.498039,0.054902}%
\pgfsetstrokecolor{currentstroke}%
\pgfsetdash{}{0pt}%
\pgfpathmoveto{\pgfqpoint{2.652273in}{2.891175in}}%
\pgfpathcurveto{\pgfqpoint{2.663323in}{2.891175in}}{\pgfqpoint{2.673922in}{2.895565in}}{\pgfqpoint{2.681736in}{2.903379in}}%
\pgfpathcurveto{\pgfqpoint{2.689549in}{2.911193in}}{\pgfqpoint{2.693939in}{2.921792in}}{\pgfqpoint{2.693939in}{2.932842in}}%
\pgfpathcurveto{\pgfqpoint{2.693939in}{2.943892in}}{\pgfqpoint{2.689549in}{2.954491in}}{\pgfqpoint{2.681736in}{2.962305in}}%
\pgfpathcurveto{\pgfqpoint{2.673922in}{2.970118in}}{\pgfqpoint{2.663323in}{2.974509in}}{\pgfqpoint{2.652273in}{2.974509in}}%
\pgfpathcurveto{\pgfqpoint{2.641223in}{2.974509in}}{\pgfqpoint{2.630624in}{2.970118in}}{\pgfqpoint{2.622810in}{2.962305in}}%
\pgfpathcurveto{\pgfqpoint{2.614996in}{2.954491in}}{\pgfqpoint{2.610606in}{2.943892in}}{\pgfqpoint{2.610606in}{2.932842in}}%
\pgfpathcurveto{\pgfqpoint{2.610606in}{2.921792in}}{\pgfqpoint{2.614996in}{2.911193in}}{\pgfqpoint{2.622810in}{2.903379in}}%
\pgfpathcurveto{\pgfqpoint{2.630624in}{2.895565in}}{\pgfqpoint{2.641223in}{2.891175in}}{\pgfqpoint{2.652273in}{2.891175in}}%
\pgfpathclose%
\pgfusepath{stroke,fill}%
\end{pgfscope}%
\begin{pgfscope}%
\pgfpathrectangle{\pgfqpoint{0.750000in}{0.500000in}}{\pgfqpoint{4.650000in}{3.020000in}}%
\pgfusepath{clip}%
\pgfsetbuttcap%
\pgfsetroundjoin%
\definecolor{currentfill}{rgb}{1.000000,0.498039,0.054902}%
\pgfsetfillcolor{currentfill}%
\pgfsetlinewidth{1.003750pt}%
\definecolor{currentstroke}{rgb}{1.000000,0.498039,0.054902}%
\pgfsetstrokecolor{currentstroke}%
\pgfsetdash{}{0pt}%
\pgfpathmoveto{\pgfqpoint{2.531494in}{2.898866in}}%
\pgfpathcurveto{\pgfqpoint{2.542544in}{2.898866in}}{\pgfqpoint{2.553143in}{2.903256in}}{\pgfqpoint{2.560956in}{2.911069in}}%
\pgfpathcurveto{\pgfqpoint{2.568770in}{2.918883in}}{\pgfqpoint{2.573160in}{2.929482in}}{\pgfqpoint{2.573160in}{2.940532in}}%
\pgfpathcurveto{\pgfqpoint{2.573160in}{2.951582in}}{\pgfqpoint{2.568770in}{2.962181in}}{\pgfqpoint{2.560956in}{2.969995in}}%
\pgfpathcurveto{\pgfqpoint{2.553143in}{2.977809in}}{\pgfqpoint{2.542544in}{2.982199in}}{\pgfqpoint{2.531494in}{2.982199in}}%
\pgfpathcurveto{\pgfqpoint{2.520443in}{2.982199in}}{\pgfqpoint{2.509844in}{2.977809in}}{\pgfqpoint{2.502031in}{2.969995in}}%
\pgfpathcurveto{\pgfqpoint{2.494217in}{2.962181in}}{\pgfqpoint{2.489827in}{2.951582in}}{\pgfqpoint{2.489827in}{2.940532in}}%
\pgfpathcurveto{\pgfqpoint{2.489827in}{2.929482in}}{\pgfqpoint{2.494217in}{2.918883in}}{\pgfqpoint{2.502031in}{2.911069in}}%
\pgfpathcurveto{\pgfqpoint{2.509844in}{2.903256in}}{\pgfqpoint{2.520443in}{2.898866in}}{\pgfqpoint{2.531494in}{2.898866in}}%
\pgfpathclose%
\pgfusepath{stroke,fill}%
\end{pgfscope}%
\begin{pgfscope}%
\pgfpathrectangle{\pgfqpoint{0.750000in}{0.500000in}}{\pgfqpoint{4.650000in}{3.020000in}}%
\pgfusepath{clip}%
\pgfsetbuttcap%
\pgfsetroundjoin%
\definecolor{currentfill}{rgb}{0.121569,0.466667,0.705882}%
\pgfsetfillcolor{currentfill}%
\pgfsetlinewidth{1.003750pt}%
\definecolor{currentstroke}{rgb}{0.121569,0.466667,0.705882}%
\pgfsetstrokecolor{currentstroke}%
\pgfsetdash{}{0pt}%
\pgfpathmoveto{\pgfqpoint{2.108766in}{0.595606in}}%
\pgfpathcurveto{\pgfqpoint{2.119816in}{0.595606in}}{\pgfqpoint{2.130415in}{0.599996in}}{\pgfqpoint{2.138229in}{0.607810in}}%
\pgfpathcurveto{\pgfqpoint{2.146043in}{0.615624in}}{\pgfqpoint{2.150433in}{0.626223in}}{\pgfqpoint{2.150433in}{0.637273in}}%
\pgfpathcurveto{\pgfqpoint{2.150433in}{0.648323in}}{\pgfqpoint{2.146043in}{0.658922in}}{\pgfqpoint{2.138229in}{0.666736in}}%
\pgfpathcurveto{\pgfqpoint{2.130415in}{0.674549in}}{\pgfqpoint{2.119816in}{0.678939in}}{\pgfqpoint{2.108766in}{0.678939in}}%
\pgfpathcurveto{\pgfqpoint{2.097716in}{0.678939in}}{\pgfqpoint{2.087117in}{0.674549in}}{\pgfqpoint{2.079303in}{0.666736in}}%
\pgfpathcurveto{\pgfqpoint{2.071490in}{0.658922in}}{\pgfqpoint{2.067100in}{0.648323in}}{\pgfqpoint{2.067100in}{0.637273in}}%
\pgfpathcurveto{\pgfqpoint{2.067100in}{0.626223in}}{\pgfqpoint{2.071490in}{0.615624in}}{\pgfqpoint{2.079303in}{0.607810in}}%
\pgfpathcurveto{\pgfqpoint{2.087117in}{0.599996in}}{\pgfqpoint{2.097716in}{0.595606in}}{\pgfqpoint{2.108766in}{0.595606in}}%
\pgfpathclose%
\pgfusepath{stroke,fill}%
\end{pgfscope}%
\begin{pgfscope}%
\pgfpathrectangle{\pgfqpoint{0.750000in}{0.500000in}}{\pgfqpoint{4.650000in}{3.020000in}}%
\pgfusepath{clip}%
\pgfsetbuttcap%
\pgfsetroundjoin%
\definecolor{currentfill}{rgb}{1.000000,0.498039,0.054902}%
\pgfsetfillcolor{currentfill}%
\pgfsetlinewidth{1.003750pt}%
\definecolor{currentstroke}{rgb}{1.000000,0.498039,0.054902}%
\pgfsetstrokecolor{currentstroke}%
\pgfsetdash{}{0pt}%
\pgfpathmoveto{\pgfqpoint{2.350325in}{2.898866in}}%
\pgfpathcurveto{\pgfqpoint{2.361375in}{2.898866in}}{\pgfqpoint{2.371974in}{2.903256in}}{\pgfqpoint{2.379787in}{2.911069in}}%
\pgfpathcurveto{\pgfqpoint{2.387601in}{2.918883in}}{\pgfqpoint{2.391991in}{2.929482in}}{\pgfqpoint{2.391991in}{2.940532in}}%
\pgfpathcurveto{\pgfqpoint{2.391991in}{2.951582in}}{\pgfqpoint{2.387601in}{2.962181in}}{\pgfqpoint{2.379787in}{2.969995in}}%
\pgfpathcurveto{\pgfqpoint{2.371974in}{2.977809in}}{\pgfqpoint{2.361375in}{2.982199in}}{\pgfqpoint{2.350325in}{2.982199in}}%
\pgfpathcurveto{\pgfqpoint{2.339275in}{2.982199in}}{\pgfqpoint{2.328676in}{2.977809in}}{\pgfqpoint{2.320862in}{2.969995in}}%
\pgfpathcurveto{\pgfqpoint{2.313048in}{2.962181in}}{\pgfqpoint{2.308658in}{2.951582in}}{\pgfqpoint{2.308658in}{2.940532in}}%
\pgfpathcurveto{\pgfqpoint{2.308658in}{2.929482in}}{\pgfqpoint{2.313048in}{2.918883in}}{\pgfqpoint{2.320862in}{2.911069in}}%
\pgfpathcurveto{\pgfqpoint{2.328676in}{2.903256in}}{\pgfqpoint{2.339275in}{2.898866in}}{\pgfqpoint{2.350325in}{2.898866in}}%
\pgfpathclose%
\pgfusepath{stroke,fill}%
\end{pgfscope}%
\begin{pgfscope}%
\pgfpathrectangle{\pgfqpoint{0.750000in}{0.500000in}}{\pgfqpoint{4.650000in}{3.020000in}}%
\pgfusepath{clip}%
\pgfsetbuttcap%
\pgfsetroundjoin%
\definecolor{currentfill}{rgb}{1.000000,0.498039,0.054902}%
\pgfsetfillcolor{currentfill}%
\pgfsetlinewidth{1.003750pt}%
\definecolor{currentstroke}{rgb}{1.000000,0.498039,0.054902}%
\pgfsetstrokecolor{currentstroke}%
\pgfsetdash{}{0pt}%
\pgfpathmoveto{\pgfqpoint{1.686039in}{2.918091in}}%
\pgfpathcurveto{\pgfqpoint{1.697089in}{2.918091in}}{\pgfqpoint{1.707688in}{2.922482in}}{\pgfqpoint{1.715502in}{2.930295in}}%
\pgfpathcurveto{\pgfqpoint{1.723315in}{2.938109in}}{\pgfqpoint{1.727706in}{2.948708in}}{\pgfqpoint{1.727706in}{2.959758in}}%
\pgfpathcurveto{\pgfqpoint{1.727706in}{2.970808in}}{\pgfqpoint{1.723315in}{2.981407in}}{\pgfqpoint{1.715502in}{2.989221in}}%
\pgfpathcurveto{\pgfqpoint{1.707688in}{2.997034in}}{\pgfqpoint{1.697089in}{3.001425in}}{\pgfqpoint{1.686039in}{3.001425in}}%
\pgfpathcurveto{\pgfqpoint{1.674989in}{3.001425in}}{\pgfqpoint{1.664390in}{2.997034in}}{\pgfqpoint{1.656576in}{2.989221in}}%
\pgfpathcurveto{\pgfqpoint{1.648763in}{2.981407in}}{\pgfqpoint{1.644372in}{2.970808in}}{\pgfqpoint{1.644372in}{2.959758in}}%
\pgfpathcurveto{\pgfqpoint{1.644372in}{2.948708in}}{\pgfqpoint{1.648763in}{2.938109in}}{\pgfqpoint{1.656576in}{2.930295in}}%
\pgfpathcurveto{\pgfqpoint{1.664390in}{2.922482in}}{\pgfqpoint{1.674989in}{2.918091in}}{\pgfqpoint{1.686039in}{2.918091in}}%
\pgfpathclose%
\pgfusepath{stroke,fill}%
\end{pgfscope}%
\begin{pgfscope}%
\pgfpathrectangle{\pgfqpoint{0.750000in}{0.500000in}}{\pgfqpoint{4.650000in}{3.020000in}}%
\pgfusepath{clip}%
\pgfsetbuttcap%
\pgfsetroundjoin%
\definecolor{currentfill}{rgb}{1.000000,0.498039,0.054902}%
\pgfsetfillcolor{currentfill}%
\pgfsetlinewidth{1.003750pt}%
\definecolor{currentstroke}{rgb}{1.000000,0.498039,0.054902}%
\pgfsetstrokecolor{currentstroke}%
\pgfsetdash{}{0pt}%
\pgfpathmoveto{\pgfqpoint{1.504870in}{2.960388in}}%
\pgfpathcurveto{\pgfqpoint{1.515920in}{2.960388in}}{\pgfqpoint{1.526519in}{2.964779in}}{\pgfqpoint{1.534333in}{2.972592in}}%
\pgfpathcurveto{\pgfqpoint{1.542147in}{2.980406in}}{\pgfqpoint{1.546537in}{2.991005in}}{\pgfqpoint{1.546537in}{3.002055in}}%
\pgfpathcurveto{\pgfqpoint{1.546537in}{3.013105in}}{\pgfqpoint{1.542147in}{3.023704in}}{\pgfqpoint{1.534333in}{3.031518in}}%
\pgfpathcurveto{\pgfqpoint{1.526519in}{3.039331in}}{\pgfqpoint{1.515920in}{3.043722in}}{\pgfqpoint{1.504870in}{3.043722in}}%
\pgfpathcurveto{\pgfqpoint{1.493820in}{3.043722in}}{\pgfqpoint{1.483221in}{3.039331in}}{\pgfqpoint{1.475407in}{3.031518in}}%
\pgfpathcurveto{\pgfqpoint{1.467594in}{3.023704in}}{\pgfqpoint{1.463203in}{3.013105in}}{\pgfqpoint{1.463203in}{3.002055in}}%
\pgfpathcurveto{\pgfqpoint{1.463203in}{2.991005in}}{\pgfqpoint{1.467594in}{2.980406in}}{\pgfqpoint{1.475407in}{2.972592in}}%
\pgfpathcurveto{\pgfqpoint{1.483221in}{2.964779in}}{\pgfqpoint{1.493820in}{2.960388in}}{\pgfqpoint{1.504870in}{2.960388in}}%
\pgfpathclose%
\pgfusepath{stroke,fill}%
\end{pgfscope}%
\begin{pgfscope}%
\pgfpathrectangle{\pgfqpoint{0.750000in}{0.500000in}}{\pgfqpoint{4.650000in}{3.020000in}}%
\pgfusepath{clip}%
\pgfsetbuttcap%
\pgfsetroundjoin%
\definecolor{currentfill}{rgb}{1.000000,0.498039,0.054902}%
\pgfsetfillcolor{currentfill}%
\pgfsetlinewidth{1.003750pt}%
\definecolor{currentstroke}{rgb}{1.000000,0.498039,0.054902}%
\pgfsetstrokecolor{currentstroke}%
\pgfsetdash{}{0pt}%
\pgfpathmoveto{\pgfqpoint{1.927597in}{2.898866in}}%
\pgfpathcurveto{\pgfqpoint{1.938648in}{2.898866in}}{\pgfqpoint{1.949247in}{2.903256in}}{\pgfqpoint{1.957060in}{2.911069in}}%
\pgfpathcurveto{\pgfqpoint{1.964874in}{2.918883in}}{\pgfqpoint{1.969264in}{2.929482in}}{\pgfqpoint{1.969264in}{2.940532in}}%
\pgfpathcurveto{\pgfqpoint{1.969264in}{2.951582in}}{\pgfqpoint{1.964874in}{2.962181in}}{\pgfqpoint{1.957060in}{2.969995in}}%
\pgfpathcurveto{\pgfqpoint{1.949247in}{2.977809in}}{\pgfqpoint{1.938648in}{2.982199in}}{\pgfqpoint{1.927597in}{2.982199in}}%
\pgfpathcurveto{\pgfqpoint{1.916547in}{2.982199in}}{\pgfqpoint{1.905948in}{2.977809in}}{\pgfqpoint{1.898135in}{2.969995in}}%
\pgfpathcurveto{\pgfqpoint{1.890321in}{2.962181in}}{\pgfqpoint{1.885931in}{2.951582in}}{\pgfqpoint{1.885931in}{2.940532in}}%
\pgfpathcurveto{\pgfqpoint{1.885931in}{2.929482in}}{\pgfqpoint{1.890321in}{2.918883in}}{\pgfqpoint{1.898135in}{2.911069in}}%
\pgfpathcurveto{\pgfqpoint{1.905948in}{2.903256in}}{\pgfqpoint{1.916547in}{2.898866in}}{\pgfqpoint{1.927597in}{2.898866in}}%
\pgfpathclose%
\pgfusepath{stroke,fill}%
\end{pgfscope}%
\begin{pgfscope}%
\pgfpathrectangle{\pgfqpoint{0.750000in}{0.500000in}}{\pgfqpoint{4.650000in}{3.020000in}}%
\pgfusepath{clip}%
\pgfsetbuttcap%
\pgfsetroundjoin%
\definecolor{currentfill}{rgb}{1.000000,0.498039,0.054902}%
\pgfsetfillcolor{currentfill}%
\pgfsetlinewidth{1.003750pt}%
\definecolor{currentstroke}{rgb}{1.000000,0.498039,0.054902}%
\pgfsetstrokecolor{currentstroke}%
\pgfsetdash{}{0pt}%
\pgfpathmoveto{\pgfqpoint{1.686039in}{2.902711in}}%
\pgfpathcurveto{\pgfqpoint{1.697089in}{2.902711in}}{\pgfqpoint{1.707688in}{2.907101in}}{\pgfqpoint{1.715502in}{2.914915in}}%
\pgfpathcurveto{\pgfqpoint{1.723315in}{2.922728in}}{\pgfqpoint{1.727706in}{2.933327in}}{\pgfqpoint{1.727706in}{2.944377in}}%
\pgfpathcurveto{\pgfqpoint{1.727706in}{2.955428in}}{\pgfqpoint{1.723315in}{2.966027in}}{\pgfqpoint{1.715502in}{2.973840in}}%
\pgfpathcurveto{\pgfqpoint{1.707688in}{2.981654in}}{\pgfqpoint{1.697089in}{2.986044in}}{\pgfqpoint{1.686039in}{2.986044in}}%
\pgfpathcurveto{\pgfqpoint{1.674989in}{2.986044in}}{\pgfqpoint{1.664390in}{2.981654in}}{\pgfqpoint{1.656576in}{2.973840in}}%
\pgfpathcurveto{\pgfqpoint{1.648763in}{2.966027in}}{\pgfqpoint{1.644372in}{2.955428in}}{\pgfqpoint{1.644372in}{2.944377in}}%
\pgfpathcurveto{\pgfqpoint{1.644372in}{2.933327in}}{\pgfqpoint{1.648763in}{2.922728in}}{\pgfqpoint{1.656576in}{2.914915in}}%
\pgfpathcurveto{\pgfqpoint{1.664390in}{2.907101in}}{\pgfqpoint{1.674989in}{2.902711in}}{\pgfqpoint{1.686039in}{2.902711in}}%
\pgfpathclose%
\pgfusepath{stroke,fill}%
\end{pgfscope}%
\begin{pgfscope}%
\pgfpathrectangle{\pgfqpoint{0.750000in}{0.500000in}}{\pgfqpoint{4.650000in}{3.020000in}}%
\pgfusepath{clip}%
\pgfsetbuttcap%
\pgfsetroundjoin%
\definecolor{currentfill}{rgb}{1.000000,0.498039,0.054902}%
\pgfsetfillcolor{currentfill}%
\pgfsetlinewidth{1.003750pt}%
\definecolor{currentstroke}{rgb}{1.000000,0.498039,0.054902}%
\pgfsetstrokecolor{currentstroke}%
\pgfsetdash{}{0pt}%
\pgfpathmoveto{\pgfqpoint{3.437338in}{2.898866in}}%
\pgfpathcurveto{\pgfqpoint{3.448388in}{2.898866in}}{\pgfqpoint{3.458987in}{2.903256in}}{\pgfqpoint{3.466800in}{2.911069in}}%
\pgfpathcurveto{\pgfqpoint{3.474614in}{2.918883in}}{\pgfqpoint{3.479004in}{2.929482in}}{\pgfqpoint{3.479004in}{2.940532in}}%
\pgfpathcurveto{\pgfqpoint{3.479004in}{2.951582in}}{\pgfqpoint{3.474614in}{2.962181in}}{\pgfqpoint{3.466800in}{2.969995in}}%
\pgfpathcurveto{\pgfqpoint{3.458987in}{2.977809in}}{\pgfqpoint{3.448388in}{2.982199in}}{\pgfqpoint{3.437338in}{2.982199in}}%
\pgfpathcurveto{\pgfqpoint{3.426288in}{2.982199in}}{\pgfqpoint{3.415689in}{2.977809in}}{\pgfqpoint{3.407875in}{2.969995in}}%
\pgfpathcurveto{\pgfqpoint{3.400061in}{2.962181in}}{\pgfqpoint{3.395671in}{2.951582in}}{\pgfqpoint{3.395671in}{2.940532in}}%
\pgfpathcurveto{\pgfqpoint{3.395671in}{2.929482in}}{\pgfqpoint{3.400061in}{2.918883in}}{\pgfqpoint{3.407875in}{2.911069in}}%
\pgfpathcurveto{\pgfqpoint{3.415689in}{2.903256in}}{\pgfqpoint{3.426288in}{2.898866in}}{\pgfqpoint{3.437338in}{2.898866in}}%
\pgfpathclose%
\pgfusepath{stroke,fill}%
\end{pgfscope}%
\begin{pgfscope}%
\pgfpathrectangle{\pgfqpoint{0.750000in}{0.500000in}}{\pgfqpoint{4.650000in}{3.020000in}}%
\pgfusepath{clip}%
\pgfsetbuttcap%
\pgfsetroundjoin%
\definecolor{currentfill}{rgb}{0.121569,0.466667,0.705882}%
\pgfsetfillcolor{currentfill}%
\pgfsetlinewidth{1.003750pt}%
\definecolor{currentstroke}{rgb}{0.121569,0.466667,0.705882}%
\pgfsetstrokecolor{currentstroke}%
\pgfsetdash{}{0pt}%
\pgfpathmoveto{\pgfqpoint{1.323701in}{0.595606in}}%
\pgfpathcurveto{\pgfqpoint{1.334751in}{0.595606in}}{\pgfqpoint{1.345350in}{0.599996in}}{\pgfqpoint{1.353164in}{0.607810in}}%
\pgfpathcurveto{\pgfqpoint{1.360978in}{0.615624in}}{\pgfqpoint{1.365368in}{0.626223in}}{\pgfqpoint{1.365368in}{0.637273in}}%
\pgfpathcurveto{\pgfqpoint{1.365368in}{0.648323in}}{\pgfqpoint{1.360978in}{0.658922in}}{\pgfqpoint{1.353164in}{0.666736in}}%
\pgfpathcurveto{\pgfqpoint{1.345350in}{0.674549in}}{\pgfqpoint{1.334751in}{0.678939in}}{\pgfqpoint{1.323701in}{0.678939in}}%
\pgfpathcurveto{\pgfqpoint{1.312651in}{0.678939in}}{\pgfqpoint{1.302052in}{0.674549in}}{\pgfqpoint{1.294239in}{0.666736in}}%
\pgfpathcurveto{\pgfqpoint{1.286425in}{0.658922in}}{\pgfqpoint{1.282035in}{0.648323in}}{\pgfqpoint{1.282035in}{0.637273in}}%
\pgfpathcurveto{\pgfqpoint{1.282035in}{0.626223in}}{\pgfqpoint{1.286425in}{0.615624in}}{\pgfqpoint{1.294239in}{0.607810in}}%
\pgfpathcurveto{\pgfqpoint{1.302052in}{0.599996in}}{\pgfqpoint{1.312651in}{0.595606in}}{\pgfqpoint{1.323701in}{0.595606in}}%
\pgfpathclose%
\pgfusepath{stroke,fill}%
\end{pgfscope}%
\begin{pgfscope}%
\pgfpathrectangle{\pgfqpoint{0.750000in}{0.500000in}}{\pgfqpoint{4.650000in}{3.020000in}}%
\pgfusepath{clip}%
\pgfsetbuttcap%
\pgfsetroundjoin%
\definecolor{currentfill}{rgb}{1.000000,0.498039,0.054902}%
\pgfsetfillcolor{currentfill}%
\pgfsetlinewidth{1.003750pt}%
\definecolor{currentstroke}{rgb}{1.000000,0.498039,0.054902}%
\pgfsetstrokecolor{currentstroke}%
\pgfsetdash{}{0pt}%
\pgfpathmoveto{\pgfqpoint{2.531494in}{2.941162in}}%
\pgfpathcurveto{\pgfqpoint{2.542544in}{2.941162in}}{\pgfqpoint{2.553143in}{2.945553in}}{\pgfqpoint{2.560956in}{2.953366in}}%
\pgfpathcurveto{\pgfqpoint{2.568770in}{2.961180in}}{\pgfqpoint{2.573160in}{2.971779in}}{\pgfqpoint{2.573160in}{2.982829in}}%
\pgfpathcurveto{\pgfqpoint{2.573160in}{2.993879in}}{\pgfqpoint{2.568770in}{3.004478in}}{\pgfqpoint{2.560956in}{3.012292in}}%
\pgfpathcurveto{\pgfqpoint{2.553143in}{3.020106in}}{\pgfqpoint{2.542544in}{3.024496in}}{\pgfqpoint{2.531494in}{3.024496in}}%
\pgfpathcurveto{\pgfqpoint{2.520443in}{3.024496in}}{\pgfqpoint{2.509844in}{3.020106in}}{\pgfqpoint{2.502031in}{3.012292in}}%
\pgfpathcurveto{\pgfqpoint{2.494217in}{3.004478in}}{\pgfqpoint{2.489827in}{2.993879in}}{\pgfqpoint{2.489827in}{2.982829in}}%
\pgfpathcurveto{\pgfqpoint{2.489827in}{2.971779in}}{\pgfqpoint{2.494217in}{2.961180in}}{\pgfqpoint{2.502031in}{2.953366in}}%
\pgfpathcurveto{\pgfqpoint{2.509844in}{2.945553in}}{\pgfqpoint{2.520443in}{2.941162in}}{\pgfqpoint{2.531494in}{2.941162in}}%
\pgfpathclose%
\pgfusepath{stroke,fill}%
\end{pgfscope}%
\begin{pgfscope}%
\pgfpathrectangle{\pgfqpoint{0.750000in}{0.500000in}}{\pgfqpoint{4.650000in}{3.020000in}}%
\pgfusepath{clip}%
\pgfsetbuttcap%
\pgfsetroundjoin%
\definecolor{currentfill}{rgb}{1.000000,0.498039,0.054902}%
\pgfsetfillcolor{currentfill}%
\pgfsetlinewidth{1.003750pt}%
\definecolor{currentstroke}{rgb}{1.000000,0.498039,0.054902}%
\pgfsetstrokecolor{currentstroke}%
\pgfsetdash{}{0pt}%
\pgfpathmoveto{\pgfqpoint{1.867208in}{2.902711in}}%
\pgfpathcurveto{\pgfqpoint{1.878258in}{2.902711in}}{\pgfqpoint{1.888857in}{2.907101in}}{\pgfqpoint{1.896671in}{2.914915in}}%
\pgfpathcurveto{\pgfqpoint{1.904484in}{2.922728in}}{\pgfqpoint{1.908874in}{2.933327in}}{\pgfqpoint{1.908874in}{2.944377in}}%
\pgfpathcurveto{\pgfqpoint{1.908874in}{2.955428in}}{\pgfqpoint{1.904484in}{2.966027in}}{\pgfqpoint{1.896671in}{2.973840in}}%
\pgfpathcurveto{\pgfqpoint{1.888857in}{2.981654in}}{\pgfqpoint{1.878258in}{2.986044in}}{\pgfqpoint{1.867208in}{2.986044in}}%
\pgfpathcurveto{\pgfqpoint{1.856158in}{2.986044in}}{\pgfqpoint{1.845559in}{2.981654in}}{\pgfqpoint{1.837745in}{2.973840in}}%
\pgfpathcurveto{\pgfqpoint{1.829931in}{2.966027in}}{\pgfqpoint{1.825541in}{2.955428in}}{\pgfqpoint{1.825541in}{2.944377in}}%
\pgfpathcurveto{\pgfqpoint{1.825541in}{2.933327in}}{\pgfqpoint{1.829931in}{2.922728in}}{\pgfqpoint{1.837745in}{2.914915in}}%
\pgfpathcurveto{\pgfqpoint{1.845559in}{2.907101in}}{\pgfqpoint{1.856158in}{2.902711in}}{\pgfqpoint{1.867208in}{2.902711in}}%
\pgfpathclose%
\pgfusepath{stroke,fill}%
\end{pgfscope}%
\begin{pgfscope}%
\pgfpathrectangle{\pgfqpoint{0.750000in}{0.500000in}}{\pgfqpoint{4.650000in}{3.020000in}}%
\pgfusepath{clip}%
\pgfsetbuttcap%
\pgfsetroundjoin%
\definecolor{currentfill}{rgb}{0.121569,0.466667,0.705882}%
\pgfsetfillcolor{currentfill}%
\pgfsetlinewidth{1.003750pt}%
\definecolor{currentstroke}{rgb}{0.121569,0.466667,0.705882}%
\pgfsetstrokecolor{currentstroke}%
\pgfsetdash{}{0pt}%
\pgfpathmoveto{\pgfqpoint{1.444481in}{0.699426in}}%
\pgfpathcurveto{\pgfqpoint{1.455531in}{0.699426in}}{\pgfqpoint{1.466130in}{0.703816in}}{\pgfqpoint{1.473943in}{0.711630in}}%
\pgfpathcurveto{\pgfqpoint{1.481757in}{0.719443in}}{\pgfqpoint{1.486147in}{0.730042in}}{\pgfqpoint{1.486147in}{0.741092in}}%
\pgfpathcurveto{\pgfqpoint{1.486147in}{0.752143in}}{\pgfqpoint{1.481757in}{0.762742in}}{\pgfqpoint{1.473943in}{0.770555in}}%
\pgfpathcurveto{\pgfqpoint{1.466130in}{0.778369in}}{\pgfqpoint{1.455531in}{0.782759in}}{\pgfqpoint{1.444481in}{0.782759in}}%
\pgfpathcurveto{\pgfqpoint{1.433430in}{0.782759in}}{\pgfqpoint{1.422831in}{0.778369in}}{\pgfqpoint{1.415018in}{0.770555in}}%
\pgfpathcurveto{\pgfqpoint{1.407204in}{0.762742in}}{\pgfqpoint{1.402814in}{0.752143in}}{\pgfqpoint{1.402814in}{0.741092in}}%
\pgfpathcurveto{\pgfqpoint{1.402814in}{0.730042in}}{\pgfqpoint{1.407204in}{0.719443in}}{\pgfqpoint{1.415018in}{0.711630in}}%
\pgfpathcurveto{\pgfqpoint{1.422831in}{0.703816in}}{\pgfqpoint{1.433430in}{0.699426in}}{\pgfqpoint{1.444481in}{0.699426in}}%
\pgfpathclose%
\pgfusepath{stroke,fill}%
\end{pgfscope}%
\begin{pgfscope}%
\pgfpathrectangle{\pgfqpoint{0.750000in}{0.500000in}}{\pgfqpoint{4.650000in}{3.020000in}}%
\pgfusepath{clip}%
\pgfsetbuttcap%
\pgfsetroundjoin%
\definecolor{currentfill}{rgb}{1.000000,0.498039,0.054902}%
\pgfsetfillcolor{currentfill}%
\pgfsetlinewidth{1.003750pt}%
\definecolor{currentstroke}{rgb}{1.000000,0.498039,0.054902}%
\pgfsetstrokecolor{currentstroke}%
\pgfsetdash{}{0pt}%
\pgfpathmoveto{\pgfqpoint{1.987987in}{2.898866in}}%
\pgfpathcurveto{\pgfqpoint{1.999037in}{2.898866in}}{\pgfqpoint{2.009636in}{2.903256in}}{\pgfqpoint{2.017450in}{2.911069in}}%
\pgfpathcurveto{\pgfqpoint{2.025263in}{2.918883in}}{\pgfqpoint{2.029654in}{2.929482in}}{\pgfqpoint{2.029654in}{2.940532in}}%
\pgfpathcurveto{\pgfqpoint{2.029654in}{2.951582in}}{\pgfqpoint{2.025263in}{2.962181in}}{\pgfqpoint{2.017450in}{2.969995in}}%
\pgfpathcurveto{\pgfqpoint{2.009636in}{2.977809in}}{\pgfqpoint{1.999037in}{2.982199in}}{\pgfqpoint{1.987987in}{2.982199in}}%
\pgfpathcurveto{\pgfqpoint{1.976937in}{2.982199in}}{\pgfqpoint{1.966338in}{2.977809in}}{\pgfqpoint{1.958524in}{2.969995in}}%
\pgfpathcurveto{\pgfqpoint{1.950711in}{2.962181in}}{\pgfqpoint{1.946320in}{2.951582in}}{\pgfqpoint{1.946320in}{2.940532in}}%
\pgfpathcurveto{\pgfqpoint{1.946320in}{2.929482in}}{\pgfqpoint{1.950711in}{2.918883in}}{\pgfqpoint{1.958524in}{2.911069in}}%
\pgfpathcurveto{\pgfqpoint{1.966338in}{2.903256in}}{\pgfqpoint{1.976937in}{2.898866in}}{\pgfqpoint{1.987987in}{2.898866in}}%
\pgfpathclose%
\pgfusepath{stroke,fill}%
\end{pgfscope}%
\begin{pgfscope}%
\pgfpathrectangle{\pgfqpoint{0.750000in}{0.500000in}}{\pgfqpoint{4.650000in}{3.020000in}}%
\pgfusepath{clip}%
\pgfsetbuttcap%
\pgfsetroundjoin%
\definecolor{currentfill}{rgb}{0.121569,0.466667,0.705882}%
\pgfsetfillcolor{currentfill}%
\pgfsetlinewidth{1.003750pt}%
\definecolor{currentstroke}{rgb}{0.121569,0.466667,0.705882}%
\pgfsetstrokecolor{currentstroke}%
\pgfsetdash{}{0pt}%
\pgfpathmoveto{\pgfqpoint{1.202922in}{0.668664in}}%
\pgfpathcurveto{\pgfqpoint{1.213972in}{0.668664in}}{\pgfqpoint{1.224571in}{0.673055in}}{\pgfqpoint{1.232385in}{0.680868in}}%
\pgfpathcurveto{\pgfqpoint{1.240198in}{0.688682in}}{\pgfqpoint{1.244589in}{0.699281in}}{\pgfqpoint{1.244589in}{0.710331in}}%
\pgfpathcurveto{\pgfqpoint{1.244589in}{0.721381in}}{\pgfqpoint{1.240198in}{0.731980in}}{\pgfqpoint{1.232385in}{0.739794in}}%
\pgfpathcurveto{\pgfqpoint{1.224571in}{0.747607in}}{\pgfqpoint{1.213972in}{0.751998in}}{\pgfqpoint{1.202922in}{0.751998in}}%
\pgfpathcurveto{\pgfqpoint{1.191872in}{0.751998in}}{\pgfqpoint{1.181273in}{0.747607in}}{\pgfqpoint{1.173459in}{0.739794in}}%
\pgfpathcurveto{\pgfqpoint{1.165646in}{0.731980in}}{\pgfqpoint{1.161255in}{0.721381in}}{\pgfqpoint{1.161255in}{0.710331in}}%
\pgfpathcurveto{\pgfqpoint{1.161255in}{0.699281in}}{\pgfqpoint{1.165646in}{0.688682in}}{\pgfqpoint{1.173459in}{0.680868in}}%
\pgfpathcurveto{\pgfqpoint{1.181273in}{0.673055in}}{\pgfqpoint{1.191872in}{0.668664in}}{\pgfqpoint{1.202922in}{0.668664in}}%
\pgfpathclose%
\pgfusepath{stroke,fill}%
\end{pgfscope}%
\begin{pgfscope}%
\pgfpathrectangle{\pgfqpoint{0.750000in}{0.500000in}}{\pgfqpoint{4.650000in}{3.020000in}}%
\pgfusepath{clip}%
\pgfsetbuttcap%
\pgfsetroundjoin%
\definecolor{currentfill}{rgb}{0.121569,0.466667,0.705882}%
\pgfsetfillcolor{currentfill}%
\pgfsetlinewidth{1.003750pt}%
\definecolor{currentstroke}{rgb}{0.121569,0.466667,0.705882}%
\pgfsetstrokecolor{currentstroke}%
\pgfsetdash{}{0pt}%
\pgfpathmoveto{\pgfqpoint{1.444481in}{0.599451in}}%
\pgfpathcurveto{\pgfqpoint{1.455531in}{0.599451in}}{\pgfqpoint{1.466130in}{0.603841in}}{\pgfqpoint{1.473943in}{0.611655in}}%
\pgfpathcurveto{\pgfqpoint{1.481757in}{0.619469in}}{\pgfqpoint{1.486147in}{0.630068in}}{\pgfqpoint{1.486147in}{0.641118in}}%
\pgfpathcurveto{\pgfqpoint{1.486147in}{0.652168in}}{\pgfqpoint{1.481757in}{0.662767in}}{\pgfqpoint{1.473943in}{0.670581in}}%
\pgfpathcurveto{\pgfqpoint{1.466130in}{0.678394in}}{\pgfqpoint{1.455531in}{0.682785in}}{\pgfqpoint{1.444481in}{0.682785in}}%
\pgfpathcurveto{\pgfqpoint{1.433430in}{0.682785in}}{\pgfqpoint{1.422831in}{0.678394in}}{\pgfqpoint{1.415018in}{0.670581in}}%
\pgfpathcurveto{\pgfqpoint{1.407204in}{0.662767in}}{\pgfqpoint{1.402814in}{0.652168in}}{\pgfqpoint{1.402814in}{0.641118in}}%
\pgfpathcurveto{\pgfqpoint{1.402814in}{0.630068in}}{\pgfqpoint{1.407204in}{0.619469in}}{\pgfqpoint{1.415018in}{0.611655in}}%
\pgfpathcurveto{\pgfqpoint{1.422831in}{0.603841in}}{\pgfqpoint{1.433430in}{0.599451in}}{\pgfqpoint{1.444481in}{0.599451in}}%
\pgfpathclose%
\pgfusepath{stroke,fill}%
\end{pgfscope}%
\begin{pgfscope}%
\pgfpathrectangle{\pgfqpoint{0.750000in}{0.500000in}}{\pgfqpoint{4.650000in}{3.020000in}}%
\pgfusepath{clip}%
\pgfsetbuttcap%
\pgfsetroundjoin%
\definecolor{currentfill}{rgb}{1.000000,0.498039,0.054902}%
\pgfsetfillcolor{currentfill}%
\pgfsetlinewidth{1.003750pt}%
\definecolor{currentstroke}{rgb}{1.000000,0.498039,0.054902}%
\pgfsetstrokecolor{currentstroke}%
\pgfsetdash{}{0pt}%
\pgfpathmoveto{\pgfqpoint{1.686039in}{2.906556in}}%
\pgfpathcurveto{\pgfqpoint{1.697089in}{2.906556in}}{\pgfqpoint{1.707688in}{2.910946in}}{\pgfqpoint{1.715502in}{2.918760in}}%
\pgfpathcurveto{\pgfqpoint{1.723315in}{2.926573in}}{\pgfqpoint{1.727706in}{2.937172in}}{\pgfqpoint{1.727706in}{2.948223in}}%
\pgfpathcurveto{\pgfqpoint{1.727706in}{2.959273in}}{\pgfqpoint{1.723315in}{2.969872in}}{\pgfqpoint{1.715502in}{2.977685in}}%
\pgfpathcurveto{\pgfqpoint{1.707688in}{2.985499in}}{\pgfqpoint{1.697089in}{2.989889in}}{\pgfqpoint{1.686039in}{2.989889in}}%
\pgfpathcurveto{\pgfqpoint{1.674989in}{2.989889in}}{\pgfqpoint{1.664390in}{2.985499in}}{\pgfqpoint{1.656576in}{2.977685in}}%
\pgfpathcurveto{\pgfqpoint{1.648763in}{2.969872in}}{\pgfqpoint{1.644372in}{2.959273in}}{\pgfqpoint{1.644372in}{2.948223in}}%
\pgfpathcurveto{\pgfqpoint{1.644372in}{2.937172in}}{\pgfqpoint{1.648763in}{2.926573in}}{\pgfqpoint{1.656576in}{2.918760in}}%
\pgfpathcurveto{\pgfqpoint{1.664390in}{2.910946in}}{\pgfqpoint{1.674989in}{2.906556in}}{\pgfqpoint{1.686039in}{2.906556in}}%
\pgfpathclose%
\pgfusepath{stroke,fill}%
\end{pgfscope}%
\begin{pgfscope}%
\pgfpathrectangle{\pgfqpoint{0.750000in}{0.500000in}}{\pgfqpoint{4.650000in}{3.020000in}}%
\pgfusepath{clip}%
\pgfsetbuttcap%
\pgfsetroundjoin%
\definecolor{currentfill}{rgb}{0.839216,0.152941,0.156863}%
\pgfsetfillcolor{currentfill}%
\pgfsetlinewidth{1.003750pt}%
\definecolor{currentstroke}{rgb}{0.839216,0.152941,0.156863}%
\pgfsetstrokecolor{currentstroke}%
\pgfsetdash{}{0pt}%
\pgfpathmoveto{\pgfqpoint{2.289935in}{1.249286in}}%
\pgfpathcurveto{\pgfqpoint{2.300985in}{1.249286in}}{\pgfqpoint{2.311584in}{1.253676in}}{\pgfqpoint{2.319398in}{1.261490in}}%
\pgfpathcurveto{\pgfqpoint{2.327211in}{1.269303in}}{\pgfqpoint{2.331602in}{1.279902in}}{\pgfqpoint{2.331602in}{1.290952in}}%
\pgfpathcurveto{\pgfqpoint{2.331602in}{1.302003in}}{\pgfqpoint{2.327211in}{1.312602in}}{\pgfqpoint{2.319398in}{1.320415in}}%
\pgfpathcurveto{\pgfqpoint{2.311584in}{1.328229in}}{\pgfqpoint{2.300985in}{1.332619in}}{\pgfqpoint{2.289935in}{1.332619in}}%
\pgfpathcurveto{\pgfqpoint{2.278885in}{1.332619in}}{\pgfqpoint{2.268286in}{1.328229in}}{\pgfqpoint{2.260472in}{1.320415in}}%
\pgfpathcurveto{\pgfqpoint{2.252659in}{1.312602in}}{\pgfqpoint{2.248268in}{1.302003in}}{\pgfqpoint{2.248268in}{1.290952in}}%
\pgfpathcurveto{\pgfqpoint{2.248268in}{1.279902in}}{\pgfqpoint{2.252659in}{1.269303in}}{\pgfqpoint{2.260472in}{1.261490in}}%
\pgfpathcurveto{\pgfqpoint{2.268286in}{1.253676in}}{\pgfqpoint{2.278885in}{1.249286in}}{\pgfqpoint{2.289935in}{1.249286in}}%
\pgfpathclose%
\pgfusepath{stroke,fill}%
\end{pgfscope}%
\begin{pgfscope}%
\pgfpathrectangle{\pgfqpoint{0.750000in}{0.500000in}}{\pgfqpoint{4.650000in}{3.020000in}}%
\pgfusepath{clip}%
\pgfsetbuttcap%
\pgfsetroundjoin%
\definecolor{currentfill}{rgb}{0.121569,0.466667,0.705882}%
\pgfsetfillcolor{currentfill}%
\pgfsetlinewidth{1.003750pt}%
\definecolor{currentstroke}{rgb}{0.121569,0.466667,0.705882}%
\pgfsetstrokecolor{currentstroke}%
\pgfsetdash{}{0pt}%
\pgfpathmoveto{\pgfqpoint{2.833442in}{0.595606in}}%
\pgfpathcurveto{\pgfqpoint{2.844492in}{0.595606in}}{\pgfqpoint{2.855091in}{0.599996in}}{\pgfqpoint{2.862904in}{0.607810in}}%
\pgfpathcurveto{\pgfqpoint{2.870718in}{0.615624in}}{\pgfqpoint{2.875108in}{0.626223in}}{\pgfqpoint{2.875108in}{0.637273in}}%
\pgfpathcurveto{\pgfqpoint{2.875108in}{0.648323in}}{\pgfqpoint{2.870718in}{0.658922in}}{\pgfqpoint{2.862904in}{0.666736in}}%
\pgfpathcurveto{\pgfqpoint{2.855091in}{0.674549in}}{\pgfqpoint{2.844492in}{0.678939in}}{\pgfqpoint{2.833442in}{0.678939in}}%
\pgfpathcurveto{\pgfqpoint{2.822391in}{0.678939in}}{\pgfqpoint{2.811792in}{0.674549in}}{\pgfqpoint{2.803979in}{0.666736in}}%
\pgfpathcurveto{\pgfqpoint{2.796165in}{0.658922in}}{\pgfqpoint{2.791775in}{0.648323in}}{\pgfqpoint{2.791775in}{0.637273in}}%
\pgfpathcurveto{\pgfqpoint{2.791775in}{0.626223in}}{\pgfqpoint{2.796165in}{0.615624in}}{\pgfqpoint{2.803979in}{0.607810in}}%
\pgfpathcurveto{\pgfqpoint{2.811792in}{0.599996in}}{\pgfqpoint{2.822391in}{0.595606in}}{\pgfqpoint{2.833442in}{0.595606in}}%
\pgfpathclose%
\pgfusepath{stroke,fill}%
\end{pgfscope}%
\begin{pgfscope}%
\pgfpathrectangle{\pgfqpoint{0.750000in}{0.500000in}}{\pgfqpoint{4.650000in}{3.020000in}}%
\pgfusepath{clip}%
\pgfsetbuttcap%
\pgfsetroundjoin%
\definecolor{currentfill}{rgb}{0.121569,0.466667,0.705882}%
\pgfsetfillcolor{currentfill}%
\pgfsetlinewidth{1.003750pt}%
\definecolor{currentstroke}{rgb}{0.121569,0.466667,0.705882}%
\pgfsetstrokecolor{currentstroke}%
\pgfsetdash{}{0pt}%
\pgfpathmoveto{\pgfqpoint{1.444481in}{0.595606in}}%
\pgfpathcurveto{\pgfqpoint{1.455531in}{0.595606in}}{\pgfqpoint{1.466130in}{0.599996in}}{\pgfqpoint{1.473943in}{0.607810in}}%
\pgfpathcurveto{\pgfqpoint{1.481757in}{0.615624in}}{\pgfqpoint{1.486147in}{0.626223in}}{\pgfqpoint{1.486147in}{0.637273in}}%
\pgfpathcurveto{\pgfqpoint{1.486147in}{0.648323in}}{\pgfqpoint{1.481757in}{0.658922in}}{\pgfqpoint{1.473943in}{0.666736in}}%
\pgfpathcurveto{\pgfqpoint{1.466130in}{0.674549in}}{\pgfqpoint{1.455531in}{0.678939in}}{\pgfqpoint{1.444481in}{0.678939in}}%
\pgfpathcurveto{\pgfqpoint{1.433430in}{0.678939in}}{\pgfqpoint{1.422831in}{0.674549in}}{\pgfqpoint{1.415018in}{0.666736in}}%
\pgfpathcurveto{\pgfqpoint{1.407204in}{0.658922in}}{\pgfqpoint{1.402814in}{0.648323in}}{\pgfqpoint{1.402814in}{0.637273in}}%
\pgfpathcurveto{\pgfqpoint{1.402814in}{0.626223in}}{\pgfqpoint{1.407204in}{0.615624in}}{\pgfqpoint{1.415018in}{0.607810in}}%
\pgfpathcurveto{\pgfqpoint{1.422831in}{0.599996in}}{\pgfqpoint{1.433430in}{0.595606in}}{\pgfqpoint{1.444481in}{0.595606in}}%
\pgfpathclose%
\pgfusepath{stroke,fill}%
\end{pgfscope}%
\begin{pgfscope}%
\pgfpathrectangle{\pgfqpoint{0.750000in}{0.500000in}}{\pgfqpoint{4.650000in}{3.020000in}}%
\pgfusepath{clip}%
\pgfsetbuttcap%
\pgfsetroundjoin%
\definecolor{currentfill}{rgb}{1.000000,0.498039,0.054902}%
\pgfsetfillcolor{currentfill}%
\pgfsetlinewidth{1.003750pt}%
\definecolor{currentstroke}{rgb}{1.000000,0.498039,0.054902}%
\pgfsetstrokecolor{currentstroke}%
\pgfsetdash{}{0pt}%
\pgfpathmoveto{\pgfqpoint{1.746429in}{3.341061in}}%
\pgfpathcurveto{\pgfqpoint{1.757479in}{3.341061in}}{\pgfqpoint{1.768078in}{3.345451in}}{\pgfqpoint{1.775891in}{3.353264in}}%
\pgfpathcurveto{\pgfqpoint{1.783705in}{3.361078in}}{\pgfqpoint{1.788095in}{3.371677in}}{\pgfqpoint{1.788095in}{3.382727in}}%
\pgfpathcurveto{\pgfqpoint{1.788095in}{3.393777in}}{\pgfqpoint{1.783705in}{3.404376in}}{\pgfqpoint{1.775891in}{3.412190in}}%
\pgfpathcurveto{\pgfqpoint{1.768078in}{3.420004in}}{\pgfqpoint{1.757479in}{3.424394in}}{\pgfqpoint{1.746429in}{3.424394in}}%
\pgfpathcurveto{\pgfqpoint{1.735378in}{3.424394in}}{\pgfqpoint{1.724779in}{3.420004in}}{\pgfqpoint{1.716966in}{3.412190in}}%
\pgfpathcurveto{\pgfqpoint{1.709152in}{3.404376in}}{\pgfqpoint{1.704762in}{3.393777in}}{\pgfqpoint{1.704762in}{3.382727in}}%
\pgfpathcurveto{\pgfqpoint{1.704762in}{3.371677in}}{\pgfqpoint{1.709152in}{3.361078in}}{\pgfqpoint{1.716966in}{3.353264in}}%
\pgfpathcurveto{\pgfqpoint{1.724779in}{3.345451in}}{\pgfqpoint{1.735378in}{3.341061in}}{\pgfqpoint{1.746429in}{3.341061in}}%
\pgfpathclose%
\pgfusepath{stroke,fill}%
\end{pgfscope}%
\begin{pgfscope}%
\pgfpathrectangle{\pgfqpoint{0.750000in}{0.500000in}}{\pgfqpoint{4.650000in}{3.020000in}}%
\pgfusepath{clip}%
\pgfsetbuttcap%
\pgfsetroundjoin%
\definecolor{currentfill}{rgb}{1.000000,0.498039,0.054902}%
\pgfsetfillcolor{currentfill}%
\pgfsetlinewidth{1.003750pt}%
\definecolor{currentstroke}{rgb}{1.000000,0.498039,0.054902}%
\pgfsetstrokecolor{currentstroke}%
\pgfsetdash{}{0pt}%
\pgfpathmoveto{\pgfqpoint{1.806818in}{3.221860in}}%
\pgfpathcurveto{\pgfqpoint{1.817868in}{3.221860in}}{\pgfqpoint{1.828467in}{3.226250in}}{\pgfqpoint{1.836281in}{3.234064in}}%
\pgfpathcurveto{\pgfqpoint{1.844095in}{3.241878in}}{\pgfqpoint{1.848485in}{3.252477in}}{\pgfqpoint{1.848485in}{3.263527in}}%
\pgfpathcurveto{\pgfqpoint{1.848485in}{3.274577in}}{\pgfqpoint{1.844095in}{3.285176in}}{\pgfqpoint{1.836281in}{3.292990in}}%
\pgfpathcurveto{\pgfqpoint{1.828467in}{3.300803in}}{\pgfqpoint{1.817868in}{3.305194in}}{\pgfqpoint{1.806818in}{3.305194in}}%
\pgfpathcurveto{\pgfqpoint{1.795768in}{3.305194in}}{\pgfqpoint{1.785169in}{3.300803in}}{\pgfqpoint{1.777355in}{3.292990in}}%
\pgfpathcurveto{\pgfqpoint{1.769542in}{3.285176in}}{\pgfqpoint{1.765152in}{3.274577in}}{\pgfqpoint{1.765152in}{3.263527in}}%
\pgfpathcurveto{\pgfqpoint{1.765152in}{3.252477in}}{\pgfqpoint{1.769542in}{3.241878in}}{\pgfqpoint{1.777355in}{3.234064in}}%
\pgfpathcurveto{\pgfqpoint{1.785169in}{3.226250in}}{\pgfqpoint{1.795768in}{3.221860in}}{\pgfqpoint{1.806818in}{3.221860in}}%
\pgfpathclose%
\pgfusepath{stroke,fill}%
\end{pgfscope}%
\begin{pgfscope}%
\pgfpathrectangle{\pgfqpoint{0.750000in}{0.500000in}}{\pgfqpoint{4.650000in}{3.020000in}}%
\pgfusepath{clip}%
\pgfsetbuttcap%
\pgfsetroundjoin%
\definecolor{currentfill}{rgb}{1.000000,0.498039,0.054902}%
\pgfsetfillcolor{currentfill}%
\pgfsetlinewidth{1.003750pt}%
\definecolor{currentstroke}{rgb}{1.000000,0.498039,0.054902}%
\pgfsetstrokecolor{currentstroke}%
\pgfsetdash{}{0pt}%
\pgfpathmoveto{\pgfqpoint{1.746429in}{3.118040in}}%
\pgfpathcurveto{\pgfqpoint{1.757479in}{3.118040in}}{\pgfqpoint{1.768078in}{3.122431in}}{\pgfqpoint{1.775891in}{3.130244in}}%
\pgfpathcurveto{\pgfqpoint{1.783705in}{3.138058in}}{\pgfqpoint{1.788095in}{3.148657in}}{\pgfqpoint{1.788095in}{3.159707in}}%
\pgfpathcurveto{\pgfqpoint{1.788095in}{3.170757in}}{\pgfqpoint{1.783705in}{3.181356in}}{\pgfqpoint{1.775891in}{3.189170in}}%
\pgfpathcurveto{\pgfqpoint{1.768078in}{3.196984in}}{\pgfqpoint{1.757479in}{3.201374in}}{\pgfqpoint{1.746429in}{3.201374in}}%
\pgfpathcurveto{\pgfqpoint{1.735378in}{3.201374in}}{\pgfqpoint{1.724779in}{3.196984in}}{\pgfqpoint{1.716966in}{3.189170in}}%
\pgfpathcurveto{\pgfqpoint{1.709152in}{3.181356in}}{\pgfqpoint{1.704762in}{3.170757in}}{\pgfqpoint{1.704762in}{3.159707in}}%
\pgfpathcurveto{\pgfqpoint{1.704762in}{3.148657in}}{\pgfqpoint{1.709152in}{3.138058in}}{\pgfqpoint{1.716966in}{3.130244in}}%
\pgfpathcurveto{\pgfqpoint{1.724779in}{3.122431in}}{\pgfqpoint{1.735378in}{3.118040in}}{\pgfqpoint{1.746429in}{3.118040in}}%
\pgfpathclose%
\pgfusepath{stroke,fill}%
\end{pgfscope}%
\begin{pgfscope}%
\pgfpathrectangle{\pgfqpoint{0.750000in}{0.500000in}}{\pgfqpoint{4.650000in}{3.020000in}}%
\pgfusepath{clip}%
\pgfsetbuttcap%
\pgfsetroundjoin%
\definecolor{currentfill}{rgb}{0.121569,0.466667,0.705882}%
\pgfsetfillcolor{currentfill}%
\pgfsetlinewidth{1.003750pt}%
\definecolor{currentstroke}{rgb}{0.121569,0.466667,0.705882}%
\pgfsetstrokecolor{currentstroke}%
\pgfsetdash{}{0pt}%
\pgfpathmoveto{\pgfqpoint{1.021753in}{0.626367in}}%
\pgfpathcurveto{\pgfqpoint{1.032803in}{0.626367in}}{\pgfqpoint{1.043402in}{0.630758in}}{\pgfqpoint{1.051216in}{0.638571in}}%
\pgfpathcurveto{\pgfqpoint{1.059030in}{0.646385in}}{\pgfqpoint{1.063420in}{0.656984in}}{\pgfqpoint{1.063420in}{0.668034in}}%
\pgfpathcurveto{\pgfqpoint{1.063420in}{0.679084in}}{\pgfqpoint{1.059030in}{0.689683in}}{\pgfqpoint{1.051216in}{0.697497in}}%
\pgfpathcurveto{\pgfqpoint{1.043402in}{0.705311in}}{\pgfqpoint{1.032803in}{0.709701in}}{\pgfqpoint{1.021753in}{0.709701in}}%
\pgfpathcurveto{\pgfqpoint{1.010703in}{0.709701in}}{\pgfqpoint{1.000104in}{0.705311in}}{\pgfqpoint{0.992290in}{0.697497in}}%
\pgfpathcurveto{\pgfqpoint{0.984477in}{0.689683in}}{\pgfqpoint{0.980087in}{0.679084in}}{\pgfqpoint{0.980087in}{0.668034in}}%
\pgfpathcurveto{\pgfqpoint{0.980087in}{0.656984in}}{\pgfqpoint{0.984477in}{0.646385in}}{\pgfqpoint{0.992290in}{0.638571in}}%
\pgfpathcurveto{\pgfqpoint{1.000104in}{0.630758in}}{\pgfqpoint{1.010703in}{0.626367in}}{\pgfqpoint{1.021753in}{0.626367in}}%
\pgfpathclose%
\pgfusepath{stroke,fill}%
\end{pgfscope}%
\begin{pgfscope}%
\pgfpathrectangle{\pgfqpoint{0.750000in}{0.500000in}}{\pgfqpoint{4.650000in}{3.020000in}}%
\pgfusepath{clip}%
\pgfsetbuttcap%
\pgfsetroundjoin%
\definecolor{currentfill}{rgb}{0.121569,0.466667,0.705882}%
\pgfsetfillcolor{currentfill}%
\pgfsetlinewidth{1.003750pt}%
\definecolor{currentstroke}{rgb}{0.121569,0.466667,0.705882}%
\pgfsetstrokecolor{currentstroke}%
\pgfsetdash{}{0pt}%
\pgfpathmoveto{\pgfqpoint{1.565260in}{0.726342in}}%
\pgfpathcurveto{\pgfqpoint{1.576310in}{0.726342in}}{\pgfqpoint{1.586909in}{0.730732in}}{\pgfqpoint{1.594723in}{0.738546in}}%
\pgfpathcurveto{\pgfqpoint{1.602536in}{0.746359in}}{\pgfqpoint{1.606926in}{0.756959in}}{\pgfqpoint{1.606926in}{0.768009in}}%
\pgfpathcurveto{\pgfqpoint{1.606926in}{0.779059in}}{\pgfqpoint{1.602536in}{0.789658in}}{\pgfqpoint{1.594723in}{0.797471in}}%
\pgfpathcurveto{\pgfqpoint{1.586909in}{0.805285in}}{\pgfqpoint{1.576310in}{0.809675in}}{\pgfqpoint{1.565260in}{0.809675in}}%
\pgfpathcurveto{\pgfqpoint{1.554210in}{0.809675in}}{\pgfqpoint{1.543611in}{0.805285in}}{\pgfqpoint{1.535797in}{0.797471in}}%
\pgfpathcurveto{\pgfqpoint{1.527983in}{0.789658in}}{\pgfqpoint{1.523593in}{0.779059in}}{\pgfqpoint{1.523593in}{0.768009in}}%
\pgfpathcurveto{\pgfqpoint{1.523593in}{0.756959in}}{\pgfqpoint{1.527983in}{0.746359in}}{\pgfqpoint{1.535797in}{0.738546in}}%
\pgfpathcurveto{\pgfqpoint{1.543611in}{0.730732in}}{\pgfqpoint{1.554210in}{0.726342in}}{\pgfqpoint{1.565260in}{0.726342in}}%
\pgfpathclose%
\pgfusepath{stroke,fill}%
\end{pgfscope}%
\begin{pgfscope}%
\pgfpathrectangle{\pgfqpoint{0.750000in}{0.500000in}}{\pgfqpoint{4.650000in}{3.020000in}}%
\pgfusepath{clip}%
\pgfsetbuttcap%
\pgfsetroundjoin%
\definecolor{currentfill}{rgb}{1.000000,0.498039,0.054902}%
\pgfsetfillcolor{currentfill}%
\pgfsetlinewidth{1.003750pt}%
\definecolor{currentstroke}{rgb}{1.000000,0.498039,0.054902}%
\pgfsetstrokecolor{currentstroke}%
\pgfsetdash{}{0pt}%
\pgfpathmoveto{\pgfqpoint{2.048377in}{2.902711in}}%
\pgfpathcurveto{\pgfqpoint{2.059427in}{2.902711in}}{\pgfqpoint{2.070026in}{2.907101in}}{\pgfqpoint{2.077839in}{2.914915in}}%
\pgfpathcurveto{\pgfqpoint{2.085653in}{2.922728in}}{\pgfqpoint{2.090043in}{2.933327in}}{\pgfqpoint{2.090043in}{2.944377in}}%
\pgfpathcurveto{\pgfqpoint{2.090043in}{2.955428in}}{\pgfqpoint{2.085653in}{2.966027in}}{\pgfqpoint{2.077839in}{2.973840in}}%
\pgfpathcurveto{\pgfqpoint{2.070026in}{2.981654in}}{\pgfqpoint{2.059427in}{2.986044in}}{\pgfqpoint{2.048377in}{2.986044in}}%
\pgfpathcurveto{\pgfqpoint{2.037326in}{2.986044in}}{\pgfqpoint{2.026727in}{2.981654in}}{\pgfqpoint{2.018914in}{2.973840in}}%
\pgfpathcurveto{\pgfqpoint{2.011100in}{2.966027in}}{\pgfqpoint{2.006710in}{2.955428in}}{\pgfqpoint{2.006710in}{2.944377in}}%
\pgfpathcurveto{\pgfqpoint{2.006710in}{2.933327in}}{\pgfqpoint{2.011100in}{2.922728in}}{\pgfqpoint{2.018914in}{2.914915in}}%
\pgfpathcurveto{\pgfqpoint{2.026727in}{2.907101in}}{\pgfqpoint{2.037326in}{2.902711in}}{\pgfqpoint{2.048377in}{2.902711in}}%
\pgfpathclose%
\pgfusepath{stroke,fill}%
\end{pgfscope}%
\begin{pgfscope}%
\pgfpathrectangle{\pgfqpoint{0.750000in}{0.500000in}}{\pgfqpoint{4.650000in}{3.020000in}}%
\pgfusepath{clip}%
\pgfsetbuttcap%
\pgfsetroundjoin%
\definecolor{currentfill}{rgb}{1.000000,0.498039,0.054902}%
\pgfsetfillcolor{currentfill}%
\pgfsetlinewidth{1.003750pt}%
\definecolor{currentstroke}{rgb}{1.000000,0.498039,0.054902}%
\pgfsetstrokecolor{currentstroke}%
\pgfsetdash{}{0pt}%
\pgfpathmoveto{\pgfqpoint{1.746429in}{2.918091in}}%
\pgfpathcurveto{\pgfqpoint{1.757479in}{2.918091in}}{\pgfqpoint{1.768078in}{2.922482in}}{\pgfqpoint{1.775891in}{2.930295in}}%
\pgfpathcurveto{\pgfqpoint{1.783705in}{2.938109in}}{\pgfqpoint{1.788095in}{2.948708in}}{\pgfqpoint{1.788095in}{2.959758in}}%
\pgfpathcurveto{\pgfqpoint{1.788095in}{2.970808in}}{\pgfqpoint{1.783705in}{2.981407in}}{\pgfqpoint{1.775891in}{2.989221in}}%
\pgfpathcurveto{\pgfqpoint{1.768078in}{2.997034in}}{\pgfqpoint{1.757479in}{3.001425in}}{\pgfqpoint{1.746429in}{3.001425in}}%
\pgfpathcurveto{\pgfqpoint{1.735378in}{3.001425in}}{\pgfqpoint{1.724779in}{2.997034in}}{\pgfqpoint{1.716966in}{2.989221in}}%
\pgfpathcurveto{\pgfqpoint{1.709152in}{2.981407in}}{\pgfqpoint{1.704762in}{2.970808in}}{\pgfqpoint{1.704762in}{2.959758in}}%
\pgfpathcurveto{\pgfqpoint{1.704762in}{2.948708in}}{\pgfqpoint{1.709152in}{2.938109in}}{\pgfqpoint{1.716966in}{2.930295in}}%
\pgfpathcurveto{\pgfqpoint{1.724779in}{2.922482in}}{\pgfqpoint{1.735378in}{2.918091in}}{\pgfqpoint{1.746429in}{2.918091in}}%
\pgfpathclose%
\pgfusepath{stroke,fill}%
\end{pgfscope}%
\begin{pgfscope}%
\pgfpathrectangle{\pgfqpoint{0.750000in}{0.500000in}}{\pgfqpoint{4.650000in}{3.020000in}}%
\pgfusepath{clip}%
\pgfsetbuttcap%
\pgfsetroundjoin%
\definecolor{currentfill}{rgb}{0.121569,0.466667,0.705882}%
\pgfsetfillcolor{currentfill}%
\pgfsetlinewidth{1.003750pt}%
\definecolor{currentstroke}{rgb}{0.121569,0.466667,0.705882}%
\pgfsetstrokecolor{currentstroke}%
\pgfsetdash{}{0pt}%
\pgfpathmoveto{\pgfqpoint{5.188636in}{0.595606in}}%
\pgfpathcurveto{\pgfqpoint{5.199686in}{0.595606in}}{\pgfqpoint{5.210286in}{0.599996in}}{\pgfqpoint{5.218099in}{0.607810in}}%
\pgfpathcurveto{\pgfqpoint{5.225913in}{0.615624in}}{\pgfqpoint{5.230303in}{0.626223in}}{\pgfqpoint{5.230303in}{0.637273in}}%
\pgfpathcurveto{\pgfqpoint{5.230303in}{0.648323in}}{\pgfqpoint{5.225913in}{0.658922in}}{\pgfqpoint{5.218099in}{0.666736in}}%
\pgfpathcurveto{\pgfqpoint{5.210286in}{0.674549in}}{\pgfqpoint{5.199686in}{0.678939in}}{\pgfqpoint{5.188636in}{0.678939in}}%
\pgfpathcurveto{\pgfqpoint{5.177586in}{0.678939in}}{\pgfqpoint{5.166987in}{0.674549in}}{\pgfqpoint{5.159174in}{0.666736in}}%
\pgfpathcurveto{\pgfqpoint{5.151360in}{0.658922in}}{\pgfqpoint{5.146970in}{0.648323in}}{\pgfqpoint{5.146970in}{0.637273in}}%
\pgfpathcurveto{\pgfqpoint{5.146970in}{0.626223in}}{\pgfqpoint{5.151360in}{0.615624in}}{\pgfqpoint{5.159174in}{0.607810in}}%
\pgfpathcurveto{\pgfqpoint{5.166987in}{0.599996in}}{\pgfqpoint{5.177586in}{0.595606in}}{\pgfqpoint{5.188636in}{0.595606in}}%
\pgfpathclose%
\pgfusepath{stroke,fill}%
\end{pgfscope}%
\begin{pgfscope}%
\pgfpathrectangle{\pgfqpoint{0.750000in}{0.500000in}}{\pgfqpoint{4.650000in}{3.020000in}}%
\pgfusepath{clip}%
\pgfsetbuttcap%
\pgfsetroundjoin%
\definecolor{currentfill}{rgb}{1.000000,0.498039,0.054902}%
\pgfsetfillcolor{currentfill}%
\pgfsetlinewidth{1.003750pt}%
\definecolor{currentstroke}{rgb}{1.000000,0.498039,0.054902}%
\pgfsetstrokecolor{currentstroke}%
\pgfsetdash{}{0pt}%
\pgfpathmoveto{\pgfqpoint{2.229545in}{2.906556in}}%
\pgfpathcurveto{\pgfqpoint{2.240596in}{2.906556in}}{\pgfqpoint{2.251195in}{2.910946in}}{\pgfqpoint{2.259008in}{2.918760in}}%
\pgfpathcurveto{\pgfqpoint{2.266822in}{2.926573in}}{\pgfqpoint{2.271212in}{2.937172in}}{\pgfqpoint{2.271212in}{2.948223in}}%
\pgfpathcurveto{\pgfqpoint{2.271212in}{2.959273in}}{\pgfqpoint{2.266822in}{2.969872in}}{\pgfqpoint{2.259008in}{2.977685in}}%
\pgfpathcurveto{\pgfqpoint{2.251195in}{2.985499in}}{\pgfqpoint{2.240596in}{2.989889in}}{\pgfqpoint{2.229545in}{2.989889in}}%
\pgfpathcurveto{\pgfqpoint{2.218495in}{2.989889in}}{\pgfqpoint{2.207896in}{2.985499in}}{\pgfqpoint{2.200083in}{2.977685in}}%
\pgfpathcurveto{\pgfqpoint{2.192269in}{2.969872in}}{\pgfqpoint{2.187879in}{2.959273in}}{\pgfqpoint{2.187879in}{2.948223in}}%
\pgfpathcurveto{\pgfqpoint{2.187879in}{2.937172in}}{\pgfqpoint{2.192269in}{2.926573in}}{\pgfqpoint{2.200083in}{2.918760in}}%
\pgfpathcurveto{\pgfqpoint{2.207896in}{2.910946in}}{\pgfqpoint{2.218495in}{2.906556in}}{\pgfqpoint{2.229545in}{2.906556in}}%
\pgfpathclose%
\pgfusepath{stroke,fill}%
\end{pgfscope}%
\begin{pgfscope}%
\pgfpathrectangle{\pgfqpoint{0.750000in}{0.500000in}}{\pgfqpoint{4.650000in}{3.020000in}}%
\pgfusepath{clip}%
\pgfsetbuttcap%
\pgfsetroundjoin%
\definecolor{currentfill}{rgb}{1.000000,0.498039,0.054902}%
\pgfsetfillcolor{currentfill}%
\pgfsetlinewidth{1.003750pt}%
\definecolor{currentstroke}{rgb}{1.000000,0.498039,0.054902}%
\pgfsetstrokecolor{currentstroke}%
\pgfsetdash{}{0pt}%
\pgfpathmoveto{\pgfqpoint{1.746429in}{2.921937in}}%
\pgfpathcurveto{\pgfqpoint{1.757479in}{2.921937in}}{\pgfqpoint{1.768078in}{2.926327in}}{\pgfqpoint{1.775891in}{2.934140in}}%
\pgfpathcurveto{\pgfqpoint{1.783705in}{2.941954in}}{\pgfqpoint{1.788095in}{2.952553in}}{\pgfqpoint{1.788095in}{2.963603in}}%
\pgfpathcurveto{\pgfqpoint{1.788095in}{2.974653in}}{\pgfqpoint{1.783705in}{2.985252in}}{\pgfqpoint{1.775891in}{2.993066in}}%
\pgfpathcurveto{\pgfqpoint{1.768078in}{3.000880in}}{\pgfqpoint{1.757479in}{3.005270in}}{\pgfqpoint{1.746429in}{3.005270in}}%
\pgfpathcurveto{\pgfqpoint{1.735378in}{3.005270in}}{\pgfqpoint{1.724779in}{3.000880in}}{\pgfqpoint{1.716966in}{2.993066in}}%
\pgfpathcurveto{\pgfqpoint{1.709152in}{2.985252in}}{\pgfqpoint{1.704762in}{2.974653in}}{\pgfqpoint{1.704762in}{2.963603in}}%
\pgfpathcurveto{\pgfqpoint{1.704762in}{2.952553in}}{\pgfqpoint{1.709152in}{2.941954in}}{\pgfqpoint{1.716966in}{2.934140in}}%
\pgfpathcurveto{\pgfqpoint{1.724779in}{2.926327in}}{\pgfqpoint{1.735378in}{2.921937in}}{\pgfqpoint{1.746429in}{2.921937in}}%
\pgfpathclose%
\pgfusepath{stroke,fill}%
\end{pgfscope}%
\begin{pgfscope}%
\pgfpathrectangle{\pgfqpoint{0.750000in}{0.500000in}}{\pgfqpoint{4.650000in}{3.020000in}}%
\pgfusepath{clip}%
\pgfsetbuttcap%
\pgfsetroundjoin%
\definecolor{currentfill}{rgb}{0.121569,0.466667,0.705882}%
\pgfsetfillcolor{currentfill}%
\pgfsetlinewidth{1.003750pt}%
\definecolor{currentstroke}{rgb}{0.121569,0.466667,0.705882}%
\pgfsetstrokecolor{currentstroke}%
\pgfsetdash{}{0pt}%
\pgfpathmoveto{\pgfqpoint{1.021753in}{0.595606in}}%
\pgfpathcurveto{\pgfqpoint{1.032803in}{0.595606in}}{\pgfqpoint{1.043402in}{0.599996in}}{\pgfqpoint{1.051216in}{0.607810in}}%
\pgfpathcurveto{\pgfqpoint{1.059030in}{0.615624in}}{\pgfqpoint{1.063420in}{0.626223in}}{\pgfqpoint{1.063420in}{0.637273in}}%
\pgfpathcurveto{\pgfqpoint{1.063420in}{0.648323in}}{\pgfqpoint{1.059030in}{0.658922in}}{\pgfqpoint{1.051216in}{0.666736in}}%
\pgfpathcurveto{\pgfqpoint{1.043402in}{0.674549in}}{\pgfqpoint{1.032803in}{0.678939in}}{\pgfqpoint{1.021753in}{0.678939in}}%
\pgfpathcurveto{\pgfqpoint{1.010703in}{0.678939in}}{\pgfqpoint{1.000104in}{0.674549in}}{\pgfqpoint{0.992290in}{0.666736in}}%
\pgfpathcurveto{\pgfqpoint{0.984477in}{0.658922in}}{\pgfqpoint{0.980087in}{0.648323in}}{\pgfqpoint{0.980087in}{0.637273in}}%
\pgfpathcurveto{\pgfqpoint{0.980087in}{0.626223in}}{\pgfqpoint{0.984477in}{0.615624in}}{\pgfqpoint{0.992290in}{0.607810in}}%
\pgfpathcurveto{\pgfqpoint{1.000104in}{0.599996in}}{\pgfqpoint{1.010703in}{0.595606in}}{\pgfqpoint{1.021753in}{0.595606in}}%
\pgfpathclose%
\pgfusepath{stroke,fill}%
\end{pgfscope}%
\begin{pgfscope}%
\pgfpathrectangle{\pgfqpoint{0.750000in}{0.500000in}}{\pgfqpoint{4.650000in}{3.020000in}}%
\pgfusepath{clip}%
\pgfsetbuttcap%
\pgfsetroundjoin%
\definecolor{currentfill}{rgb}{0.121569,0.466667,0.705882}%
\pgfsetfillcolor{currentfill}%
\pgfsetlinewidth{1.003750pt}%
\definecolor{currentstroke}{rgb}{0.121569,0.466667,0.705882}%
\pgfsetstrokecolor{currentstroke}%
\pgfsetdash{}{0pt}%
\pgfpathmoveto{\pgfqpoint{1.444481in}{1.545364in}}%
\pgfpathcurveto{\pgfqpoint{1.455531in}{1.545364in}}{\pgfqpoint{1.466130in}{1.549754in}}{\pgfqpoint{1.473943in}{1.557568in}}%
\pgfpathcurveto{\pgfqpoint{1.481757in}{1.565382in}}{\pgfqpoint{1.486147in}{1.575981in}}{\pgfqpoint{1.486147in}{1.587031in}}%
\pgfpathcurveto{\pgfqpoint{1.486147in}{1.598081in}}{\pgfqpoint{1.481757in}{1.608680in}}{\pgfqpoint{1.473943in}{1.616494in}}%
\pgfpathcurveto{\pgfqpoint{1.466130in}{1.624307in}}{\pgfqpoint{1.455531in}{1.628697in}}{\pgfqpoint{1.444481in}{1.628697in}}%
\pgfpathcurveto{\pgfqpoint{1.433430in}{1.628697in}}{\pgfqpoint{1.422831in}{1.624307in}}{\pgfqpoint{1.415018in}{1.616494in}}%
\pgfpathcurveto{\pgfqpoint{1.407204in}{1.608680in}}{\pgfqpoint{1.402814in}{1.598081in}}{\pgfqpoint{1.402814in}{1.587031in}}%
\pgfpathcurveto{\pgfqpoint{1.402814in}{1.575981in}}{\pgfqpoint{1.407204in}{1.565382in}}{\pgfqpoint{1.415018in}{1.557568in}}%
\pgfpathcurveto{\pgfqpoint{1.422831in}{1.549754in}}{\pgfqpoint{1.433430in}{1.545364in}}{\pgfqpoint{1.444481in}{1.545364in}}%
\pgfpathclose%
\pgfusepath{stroke,fill}%
\end{pgfscope}%
\begin{pgfscope}%
\pgfpathrectangle{\pgfqpoint{0.750000in}{0.500000in}}{\pgfqpoint{4.650000in}{3.020000in}}%
\pgfusepath{clip}%
\pgfsetbuttcap%
\pgfsetroundjoin%
\definecolor{currentfill}{rgb}{0.121569,0.466667,0.705882}%
\pgfsetfillcolor{currentfill}%
\pgfsetlinewidth{1.003750pt}%
\definecolor{currentstroke}{rgb}{0.121569,0.466667,0.705882}%
\pgfsetstrokecolor{currentstroke}%
\pgfsetdash{}{0pt}%
\pgfpathmoveto{\pgfqpoint{1.625649in}{0.595606in}}%
\pgfpathcurveto{\pgfqpoint{1.636699in}{0.595606in}}{\pgfqpoint{1.647299in}{0.599996in}}{\pgfqpoint{1.655112in}{0.607810in}}%
\pgfpathcurveto{\pgfqpoint{1.662926in}{0.615624in}}{\pgfqpoint{1.667316in}{0.626223in}}{\pgfqpoint{1.667316in}{0.637273in}}%
\pgfpathcurveto{\pgfqpoint{1.667316in}{0.648323in}}{\pgfqpoint{1.662926in}{0.658922in}}{\pgfqpoint{1.655112in}{0.666736in}}%
\pgfpathcurveto{\pgfqpoint{1.647299in}{0.674549in}}{\pgfqpoint{1.636699in}{0.678939in}}{\pgfqpoint{1.625649in}{0.678939in}}%
\pgfpathcurveto{\pgfqpoint{1.614599in}{0.678939in}}{\pgfqpoint{1.604000in}{0.674549in}}{\pgfqpoint{1.596187in}{0.666736in}}%
\pgfpathcurveto{\pgfqpoint{1.588373in}{0.658922in}}{\pgfqpoint{1.583983in}{0.648323in}}{\pgfqpoint{1.583983in}{0.637273in}}%
\pgfpathcurveto{\pgfqpoint{1.583983in}{0.626223in}}{\pgfqpoint{1.588373in}{0.615624in}}{\pgfqpoint{1.596187in}{0.607810in}}%
\pgfpathcurveto{\pgfqpoint{1.604000in}{0.599996in}}{\pgfqpoint{1.614599in}{0.595606in}}{\pgfqpoint{1.625649in}{0.595606in}}%
\pgfpathclose%
\pgfusepath{stroke,fill}%
\end{pgfscope}%
\begin{pgfscope}%
\pgfpathrectangle{\pgfqpoint{0.750000in}{0.500000in}}{\pgfqpoint{4.650000in}{3.020000in}}%
\pgfusepath{clip}%
\pgfsetbuttcap%
\pgfsetroundjoin%
\definecolor{currentfill}{rgb}{1.000000,0.498039,0.054902}%
\pgfsetfillcolor{currentfill}%
\pgfsetlinewidth{1.003750pt}%
\definecolor{currentstroke}{rgb}{1.000000,0.498039,0.054902}%
\pgfsetstrokecolor{currentstroke}%
\pgfsetdash{}{0pt}%
\pgfpathmoveto{\pgfqpoint{2.229545in}{2.906556in}}%
\pgfpathcurveto{\pgfqpoint{2.240596in}{2.906556in}}{\pgfqpoint{2.251195in}{2.910946in}}{\pgfqpoint{2.259008in}{2.918760in}}%
\pgfpathcurveto{\pgfqpoint{2.266822in}{2.926573in}}{\pgfqpoint{2.271212in}{2.937172in}}{\pgfqpoint{2.271212in}{2.948223in}}%
\pgfpathcurveto{\pgfqpoint{2.271212in}{2.959273in}}{\pgfqpoint{2.266822in}{2.969872in}}{\pgfqpoint{2.259008in}{2.977685in}}%
\pgfpathcurveto{\pgfqpoint{2.251195in}{2.985499in}}{\pgfqpoint{2.240596in}{2.989889in}}{\pgfqpoint{2.229545in}{2.989889in}}%
\pgfpathcurveto{\pgfqpoint{2.218495in}{2.989889in}}{\pgfqpoint{2.207896in}{2.985499in}}{\pgfqpoint{2.200083in}{2.977685in}}%
\pgfpathcurveto{\pgfqpoint{2.192269in}{2.969872in}}{\pgfqpoint{2.187879in}{2.959273in}}{\pgfqpoint{2.187879in}{2.948223in}}%
\pgfpathcurveto{\pgfqpoint{2.187879in}{2.937172in}}{\pgfqpoint{2.192269in}{2.926573in}}{\pgfqpoint{2.200083in}{2.918760in}}%
\pgfpathcurveto{\pgfqpoint{2.207896in}{2.910946in}}{\pgfqpoint{2.218495in}{2.906556in}}{\pgfqpoint{2.229545in}{2.906556in}}%
\pgfpathclose%
\pgfusepath{stroke,fill}%
\end{pgfscope}%
\begin{pgfscope}%
\pgfpathrectangle{\pgfqpoint{0.750000in}{0.500000in}}{\pgfqpoint{4.650000in}{3.020000in}}%
\pgfusepath{clip}%
\pgfsetbuttcap%
\pgfsetroundjoin%
\definecolor{currentfill}{rgb}{1.000000,0.498039,0.054902}%
\pgfsetfillcolor{currentfill}%
\pgfsetlinewidth{1.003750pt}%
\definecolor{currentstroke}{rgb}{1.000000,0.498039,0.054902}%
\pgfsetstrokecolor{currentstroke}%
\pgfsetdash{}{0pt}%
\pgfpathmoveto{\pgfqpoint{2.893831in}{3.056518in}}%
\pgfpathcurveto{\pgfqpoint{2.904881in}{3.056518in}}{\pgfqpoint{2.915480in}{3.060908in}}{\pgfqpoint{2.923294in}{3.068722in}}%
\pgfpathcurveto{\pgfqpoint{2.931108in}{3.076535in}}{\pgfqpoint{2.935498in}{3.087134in}}{\pgfqpoint{2.935498in}{3.098184in}}%
\pgfpathcurveto{\pgfqpoint{2.935498in}{3.109234in}}{\pgfqpoint{2.931108in}{3.119834in}}{\pgfqpoint{2.923294in}{3.127647in}}%
\pgfpathcurveto{\pgfqpoint{2.915480in}{3.135461in}}{\pgfqpoint{2.904881in}{3.139851in}}{\pgfqpoint{2.893831in}{3.139851in}}%
\pgfpathcurveto{\pgfqpoint{2.882781in}{3.139851in}}{\pgfqpoint{2.872182in}{3.135461in}}{\pgfqpoint{2.864368in}{3.127647in}}%
\pgfpathcurveto{\pgfqpoint{2.856555in}{3.119834in}}{\pgfqpoint{2.852165in}{3.109234in}}{\pgfqpoint{2.852165in}{3.098184in}}%
\pgfpathcurveto{\pgfqpoint{2.852165in}{3.087134in}}{\pgfqpoint{2.856555in}{3.076535in}}{\pgfqpoint{2.864368in}{3.068722in}}%
\pgfpathcurveto{\pgfqpoint{2.872182in}{3.060908in}}{\pgfqpoint{2.882781in}{3.056518in}}{\pgfqpoint{2.893831in}{3.056518in}}%
\pgfpathclose%
\pgfusepath{stroke,fill}%
\end{pgfscope}%
\begin{pgfscope}%
\pgfpathrectangle{\pgfqpoint{0.750000in}{0.500000in}}{\pgfqpoint{4.650000in}{3.020000in}}%
\pgfusepath{clip}%
\pgfsetbuttcap%
\pgfsetroundjoin%
\definecolor{currentfill}{rgb}{1.000000,0.498039,0.054902}%
\pgfsetfillcolor{currentfill}%
\pgfsetlinewidth{1.003750pt}%
\definecolor{currentstroke}{rgb}{1.000000,0.498039,0.054902}%
\pgfsetstrokecolor{currentstroke}%
\pgfsetdash{}{0pt}%
\pgfpathmoveto{\pgfqpoint{3.618506in}{2.906556in}}%
\pgfpathcurveto{\pgfqpoint{3.629557in}{2.906556in}}{\pgfqpoint{3.640156in}{2.910946in}}{\pgfqpoint{3.647969in}{2.918760in}}%
\pgfpathcurveto{\pgfqpoint{3.655783in}{2.926573in}}{\pgfqpoint{3.660173in}{2.937172in}}{\pgfqpoint{3.660173in}{2.948223in}}%
\pgfpathcurveto{\pgfqpoint{3.660173in}{2.959273in}}{\pgfqpoint{3.655783in}{2.969872in}}{\pgfqpoint{3.647969in}{2.977685in}}%
\pgfpathcurveto{\pgfqpoint{3.640156in}{2.985499in}}{\pgfqpoint{3.629557in}{2.989889in}}{\pgfqpoint{3.618506in}{2.989889in}}%
\pgfpathcurveto{\pgfqpoint{3.607456in}{2.989889in}}{\pgfqpoint{3.596857in}{2.985499in}}{\pgfqpoint{3.589044in}{2.977685in}}%
\pgfpathcurveto{\pgfqpoint{3.581230in}{2.969872in}}{\pgfqpoint{3.576840in}{2.959273in}}{\pgfqpoint{3.576840in}{2.948223in}}%
\pgfpathcurveto{\pgfqpoint{3.576840in}{2.937172in}}{\pgfqpoint{3.581230in}{2.926573in}}{\pgfqpoint{3.589044in}{2.918760in}}%
\pgfpathcurveto{\pgfqpoint{3.596857in}{2.910946in}}{\pgfqpoint{3.607456in}{2.906556in}}{\pgfqpoint{3.618506in}{2.906556in}}%
\pgfpathclose%
\pgfusepath{stroke,fill}%
\end{pgfscope}%
\begin{pgfscope}%
\pgfpathrectangle{\pgfqpoint{0.750000in}{0.500000in}}{\pgfqpoint{4.650000in}{3.020000in}}%
\pgfusepath{clip}%
\pgfsetbuttcap%
\pgfsetroundjoin%
\definecolor{currentfill}{rgb}{1.000000,0.498039,0.054902}%
\pgfsetfillcolor{currentfill}%
\pgfsetlinewidth{1.003750pt}%
\definecolor{currentstroke}{rgb}{1.000000,0.498039,0.054902}%
\pgfsetstrokecolor{currentstroke}%
\pgfsetdash{}{0pt}%
\pgfpathmoveto{\pgfqpoint{1.987987in}{2.898866in}}%
\pgfpathcurveto{\pgfqpoint{1.999037in}{2.898866in}}{\pgfqpoint{2.009636in}{2.903256in}}{\pgfqpoint{2.017450in}{2.911069in}}%
\pgfpathcurveto{\pgfqpoint{2.025263in}{2.918883in}}{\pgfqpoint{2.029654in}{2.929482in}}{\pgfqpoint{2.029654in}{2.940532in}}%
\pgfpathcurveto{\pgfqpoint{2.029654in}{2.951582in}}{\pgfqpoint{2.025263in}{2.962181in}}{\pgfqpoint{2.017450in}{2.969995in}}%
\pgfpathcurveto{\pgfqpoint{2.009636in}{2.977809in}}{\pgfqpoint{1.999037in}{2.982199in}}{\pgfqpoint{1.987987in}{2.982199in}}%
\pgfpathcurveto{\pgfqpoint{1.976937in}{2.982199in}}{\pgfqpoint{1.966338in}{2.977809in}}{\pgfqpoint{1.958524in}{2.969995in}}%
\pgfpathcurveto{\pgfqpoint{1.950711in}{2.962181in}}{\pgfqpoint{1.946320in}{2.951582in}}{\pgfqpoint{1.946320in}{2.940532in}}%
\pgfpathcurveto{\pgfqpoint{1.946320in}{2.929482in}}{\pgfqpoint{1.950711in}{2.918883in}}{\pgfqpoint{1.958524in}{2.911069in}}%
\pgfpathcurveto{\pgfqpoint{1.966338in}{2.903256in}}{\pgfqpoint{1.976937in}{2.898866in}}{\pgfqpoint{1.987987in}{2.898866in}}%
\pgfpathclose%
\pgfusepath{stroke,fill}%
\end{pgfscope}%
\begin{pgfscope}%
\pgfpathrectangle{\pgfqpoint{0.750000in}{0.500000in}}{\pgfqpoint{4.650000in}{3.020000in}}%
\pgfusepath{clip}%
\pgfsetbuttcap%
\pgfsetroundjoin%
\definecolor{currentfill}{rgb}{1.000000,0.498039,0.054902}%
\pgfsetfillcolor{currentfill}%
\pgfsetlinewidth{1.003750pt}%
\definecolor{currentstroke}{rgb}{1.000000,0.498039,0.054902}%
\pgfsetstrokecolor{currentstroke}%
\pgfsetdash{}{0pt}%
\pgfpathmoveto{\pgfqpoint{1.686039in}{2.902711in}}%
\pgfpathcurveto{\pgfqpoint{1.697089in}{2.902711in}}{\pgfqpoint{1.707688in}{2.907101in}}{\pgfqpoint{1.715502in}{2.914915in}}%
\pgfpathcurveto{\pgfqpoint{1.723315in}{2.922728in}}{\pgfqpoint{1.727706in}{2.933327in}}{\pgfqpoint{1.727706in}{2.944377in}}%
\pgfpathcurveto{\pgfqpoint{1.727706in}{2.955428in}}{\pgfqpoint{1.723315in}{2.966027in}}{\pgfqpoint{1.715502in}{2.973840in}}%
\pgfpathcurveto{\pgfqpoint{1.707688in}{2.981654in}}{\pgfqpoint{1.697089in}{2.986044in}}{\pgfqpoint{1.686039in}{2.986044in}}%
\pgfpathcurveto{\pgfqpoint{1.674989in}{2.986044in}}{\pgfqpoint{1.664390in}{2.981654in}}{\pgfqpoint{1.656576in}{2.973840in}}%
\pgfpathcurveto{\pgfqpoint{1.648763in}{2.966027in}}{\pgfqpoint{1.644372in}{2.955428in}}{\pgfqpoint{1.644372in}{2.944377in}}%
\pgfpathcurveto{\pgfqpoint{1.644372in}{2.933327in}}{\pgfqpoint{1.648763in}{2.922728in}}{\pgfqpoint{1.656576in}{2.914915in}}%
\pgfpathcurveto{\pgfqpoint{1.664390in}{2.907101in}}{\pgfqpoint{1.674989in}{2.902711in}}{\pgfqpoint{1.686039in}{2.902711in}}%
\pgfpathclose%
\pgfusepath{stroke,fill}%
\end{pgfscope}%
\begin{pgfscope}%
\pgfpathrectangle{\pgfqpoint{0.750000in}{0.500000in}}{\pgfqpoint{4.650000in}{3.020000in}}%
\pgfusepath{clip}%
\pgfsetbuttcap%
\pgfsetroundjoin%
\definecolor{currentfill}{rgb}{1.000000,0.498039,0.054902}%
\pgfsetfillcolor{currentfill}%
\pgfsetlinewidth{1.003750pt}%
\definecolor{currentstroke}{rgb}{1.000000,0.498039,0.054902}%
\pgfsetstrokecolor{currentstroke}%
\pgfsetdash{}{0pt}%
\pgfpathmoveto{\pgfqpoint{2.169156in}{3.110350in}}%
\pgfpathcurveto{\pgfqpoint{2.180206in}{3.110350in}}{\pgfqpoint{2.190805in}{3.114740in}}{\pgfqpoint{2.198619in}{3.122554in}}%
\pgfpathcurveto{\pgfqpoint{2.206432in}{3.130368in}}{\pgfqpoint{2.210823in}{3.140967in}}{\pgfqpoint{2.210823in}{3.152017in}}%
\pgfpathcurveto{\pgfqpoint{2.210823in}{3.163067in}}{\pgfqpoint{2.206432in}{3.173666in}}{\pgfqpoint{2.198619in}{3.181480in}}%
\pgfpathcurveto{\pgfqpoint{2.190805in}{3.189293in}}{\pgfqpoint{2.180206in}{3.193683in}}{\pgfqpoint{2.169156in}{3.193683in}}%
\pgfpathcurveto{\pgfqpoint{2.158106in}{3.193683in}}{\pgfqpoint{2.147507in}{3.189293in}}{\pgfqpoint{2.139693in}{3.181480in}}%
\pgfpathcurveto{\pgfqpoint{2.131879in}{3.173666in}}{\pgfqpoint{2.127489in}{3.163067in}}{\pgfqpoint{2.127489in}{3.152017in}}%
\pgfpathcurveto{\pgfqpoint{2.127489in}{3.140967in}}{\pgfqpoint{2.131879in}{3.130368in}}{\pgfqpoint{2.139693in}{3.122554in}}%
\pgfpathcurveto{\pgfqpoint{2.147507in}{3.114740in}}{\pgfqpoint{2.158106in}{3.110350in}}{\pgfqpoint{2.169156in}{3.110350in}}%
\pgfpathclose%
\pgfusepath{stroke,fill}%
\end{pgfscope}%
\begin{pgfscope}%
\pgfpathrectangle{\pgfqpoint{0.750000in}{0.500000in}}{\pgfqpoint{4.650000in}{3.020000in}}%
\pgfusepath{clip}%
\pgfsetbuttcap%
\pgfsetroundjoin%
\definecolor{currentfill}{rgb}{0.839216,0.152941,0.156863}%
\pgfsetfillcolor{currentfill}%
\pgfsetlinewidth{1.003750pt}%
\definecolor{currentstroke}{rgb}{0.839216,0.152941,0.156863}%
\pgfsetstrokecolor{currentstroke}%
\pgfsetdash{}{0pt}%
\pgfpathmoveto{\pgfqpoint{2.471104in}{2.898866in}}%
\pgfpathcurveto{\pgfqpoint{2.482154in}{2.898866in}}{\pgfqpoint{2.492753in}{2.903256in}}{\pgfqpoint{2.500567in}{2.911069in}}%
\pgfpathcurveto{\pgfqpoint{2.508380in}{2.918883in}}{\pgfqpoint{2.512771in}{2.929482in}}{\pgfqpoint{2.512771in}{2.940532in}}%
\pgfpathcurveto{\pgfqpoint{2.512771in}{2.951582in}}{\pgfqpoint{2.508380in}{2.962181in}}{\pgfqpoint{2.500567in}{2.969995in}}%
\pgfpathcurveto{\pgfqpoint{2.492753in}{2.977809in}}{\pgfqpoint{2.482154in}{2.982199in}}{\pgfqpoint{2.471104in}{2.982199in}}%
\pgfpathcurveto{\pgfqpoint{2.460054in}{2.982199in}}{\pgfqpoint{2.449455in}{2.977809in}}{\pgfqpoint{2.441641in}{2.969995in}}%
\pgfpathcurveto{\pgfqpoint{2.433827in}{2.962181in}}{\pgfqpoint{2.429437in}{2.951582in}}{\pgfqpoint{2.429437in}{2.940532in}}%
\pgfpathcurveto{\pgfqpoint{2.429437in}{2.929482in}}{\pgfqpoint{2.433827in}{2.918883in}}{\pgfqpoint{2.441641in}{2.911069in}}%
\pgfpathcurveto{\pgfqpoint{2.449455in}{2.903256in}}{\pgfqpoint{2.460054in}{2.898866in}}{\pgfqpoint{2.471104in}{2.898866in}}%
\pgfpathclose%
\pgfusepath{stroke,fill}%
\end{pgfscope}%
\begin{pgfscope}%
\pgfpathrectangle{\pgfqpoint{0.750000in}{0.500000in}}{\pgfqpoint{4.650000in}{3.020000in}}%
\pgfusepath{clip}%
\pgfsetbuttcap%
\pgfsetroundjoin%
\definecolor{currentfill}{rgb}{0.121569,0.466667,0.705882}%
\pgfsetfillcolor{currentfill}%
\pgfsetlinewidth{1.003750pt}%
\definecolor{currentstroke}{rgb}{0.121569,0.466667,0.705882}%
\pgfsetstrokecolor{currentstroke}%
\pgfsetdash{}{0pt}%
\pgfpathmoveto{\pgfqpoint{1.323701in}{0.595606in}}%
\pgfpathcurveto{\pgfqpoint{1.334751in}{0.595606in}}{\pgfqpoint{1.345350in}{0.599996in}}{\pgfqpoint{1.353164in}{0.607810in}}%
\pgfpathcurveto{\pgfqpoint{1.360978in}{0.615624in}}{\pgfqpoint{1.365368in}{0.626223in}}{\pgfqpoint{1.365368in}{0.637273in}}%
\pgfpathcurveto{\pgfqpoint{1.365368in}{0.648323in}}{\pgfqpoint{1.360978in}{0.658922in}}{\pgfqpoint{1.353164in}{0.666736in}}%
\pgfpathcurveto{\pgfqpoint{1.345350in}{0.674549in}}{\pgfqpoint{1.334751in}{0.678939in}}{\pgfqpoint{1.323701in}{0.678939in}}%
\pgfpathcurveto{\pgfqpoint{1.312651in}{0.678939in}}{\pgfqpoint{1.302052in}{0.674549in}}{\pgfqpoint{1.294239in}{0.666736in}}%
\pgfpathcurveto{\pgfqpoint{1.286425in}{0.658922in}}{\pgfqpoint{1.282035in}{0.648323in}}{\pgfqpoint{1.282035in}{0.637273in}}%
\pgfpathcurveto{\pgfqpoint{1.282035in}{0.626223in}}{\pgfqpoint{1.286425in}{0.615624in}}{\pgfqpoint{1.294239in}{0.607810in}}%
\pgfpathcurveto{\pgfqpoint{1.302052in}{0.599996in}}{\pgfqpoint{1.312651in}{0.595606in}}{\pgfqpoint{1.323701in}{0.595606in}}%
\pgfpathclose%
\pgfusepath{stroke,fill}%
\end{pgfscope}%
\begin{pgfscope}%
\pgfpathrectangle{\pgfqpoint{0.750000in}{0.500000in}}{\pgfqpoint{4.650000in}{3.020000in}}%
\pgfusepath{clip}%
\pgfsetbuttcap%
\pgfsetroundjoin%
\definecolor{currentfill}{rgb}{0.839216,0.152941,0.156863}%
\pgfsetfillcolor{currentfill}%
\pgfsetlinewidth{1.003750pt}%
\definecolor{currentstroke}{rgb}{0.839216,0.152941,0.156863}%
\pgfsetstrokecolor{currentstroke}%
\pgfsetdash{}{0pt}%
\pgfpathmoveto{\pgfqpoint{1.504870in}{2.910401in}}%
\pgfpathcurveto{\pgfqpoint{1.515920in}{2.910401in}}{\pgfqpoint{1.526519in}{2.914791in}}{\pgfqpoint{1.534333in}{2.922605in}}%
\pgfpathcurveto{\pgfqpoint{1.542147in}{2.930419in}}{\pgfqpoint{1.546537in}{2.941018in}}{\pgfqpoint{1.546537in}{2.952068in}}%
\pgfpathcurveto{\pgfqpoint{1.546537in}{2.963118in}}{\pgfqpoint{1.542147in}{2.973717in}}{\pgfqpoint{1.534333in}{2.981531in}}%
\pgfpathcurveto{\pgfqpoint{1.526519in}{2.989344in}}{\pgfqpoint{1.515920in}{2.993734in}}{\pgfqpoint{1.504870in}{2.993734in}}%
\pgfpathcurveto{\pgfqpoint{1.493820in}{2.993734in}}{\pgfqpoint{1.483221in}{2.989344in}}{\pgfqpoint{1.475407in}{2.981531in}}%
\pgfpathcurveto{\pgfqpoint{1.467594in}{2.973717in}}{\pgfqpoint{1.463203in}{2.963118in}}{\pgfqpoint{1.463203in}{2.952068in}}%
\pgfpathcurveto{\pgfqpoint{1.463203in}{2.941018in}}{\pgfqpoint{1.467594in}{2.930419in}}{\pgfqpoint{1.475407in}{2.922605in}}%
\pgfpathcurveto{\pgfqpoint{1.483221in}{2.914791in}}{\pgfqpoint{1.493820in}{2.910401in}}{\pgfqpoint{1.504870in}{2.910401in}}%
\pgfpathclose%
\pgfusepath{stroke,fill}%
\end{pgfscope}%
\begin{pgfscope}%
\pgfpathrectangle{\pgfqpoint{0.750000in}{0.500000in}}{\pgfqpoint{4.650000in}{3.020000in}}%
\pgfusepath{clip}%
\pgfsetbuttcap%
\pgfsetroundjoin%
\definecolor{currentfill}{rgb}{0.121569,0.466667,0.705882}%
\pgfsetfillcolor{currentfill}%
\pgfsetlinewidth{1.003750pt}%
\definecolor{currentstroke}{rgb}{0.121569,0.466667,0.705882}%
\pgfsetstrokecolor{currentstroke}%
\pgfsetdash{}{0pt}%
\pgfpathmoveto{\pgfqpoint{1.202922in}{0.726342in}}%
\pgfpathcurveto{\pgfqpoint{1.213972in}{0.726342in}}{\pgfqpoint{1.224571in}{0.730732in}}{\pgfqpoint{1.232385in}{0.738546in}}%
\pgfpathcurveto{\pgfqpoint{1.240198in}{0.746359in}}{\pgfqpoint{1.244589in}{0.756959in}}{\pgfqpoint{1.244589in}{0.768009in}}%
\pgfpathcurveto{\pgfqpoint{1.244589in}{0.779059in}}{\pgfqpoint{1.240198in}{0.789658in}}{\pgfqpoint{1.232385in}{0.797471in}}%
\pgfpathcurveto{\pgfqpoint{1.224571in}{0.805285in}}{\pgfqpoint{1.213972in}{0.809675in}}{\pgfqpoint{1.202922in}{0.809675in}}%
\pgfpathcurveto{\pgfqpoint{1.191872in}{0.809675in}}{\pgfqpoint{1.181273in}{0.805285in}}{\pgfqpoint{1.173459in}{0.797471in}}%
\pgfpathcurveto{\pgfqpoint{1.165646in}{0.789658in}}{\pgfqpoint{1.161255in}{0.779059in}}{\pgfqpoint{1.161255in}{0.768009in}}%
\pgfpathcurveto{\pgfqpoint{1.161255in}{0.756959in}}{\pgfqpoint{1.165646in}{0.746359in}}{\pgfqpoint{1.173459in}{0.738546in}}%
\pgfpathcurveto{\pgfqpoint{1.181273in}{0.730732in}}{\pgfqpoint{1.191872in}{0.726342in}}{\pgfqpoint{1.202922in}{0.726342in}}%
\pgfpathclose%
\pgfusepath{stroke,fill}%
\end{pgfscope}%
\begin{pgfscope}%
\pgfpathrectangle{\pgfqpoint{0.750000in}{0.500000in}}{\pgfqpoint{4.650000in}{3.020000in}}%
\pgfusepath{clip}%
\pgfsetbuttcap%
\pgfsetroundjoin%
\definecolor{currentfill}{rgb}{1.000000,0.498039,0.054902}%
\pgfsetfillcolor{currentfill}%
\pgfsetlinewidth{1.003750pt}%
\definecolor{currentstroke}{rgb}{1.000000,0.498039,0.054902}%
\pgfsetstrokecolor{currentstroke}%
\pgfsetdash{}{0pt}%
\pgfpathmoveto{\pgfqpoint{1.625649in}{2.898866in}}%
\pgfpathcurveto{\pgfqpoint{1.636699in}{2.898866in}}{\pgfqpoint{1.647299in}{2.903256in}}{\pgfqpoint{1.655112in}{2.911069in}}%
\pgfpathcurveto{\pgfqpoint{1.662926in}{2.918883in}}{\pgfqpoint{1.667316in}{2.929482in}}{\pgfqpoint{1.667316in}{2.940532in}}%
\pgfpathcurveto{\pgfqpoint{1.667316in}{2.951582in}}{\pgfqpoint{1.662926in}{2.962181in}}{\pgfqpoint{1.655112in}{2.969995in}}%
\pgfpathcurveto{\pgfqpoint{1.647299in}{2.977809in}}{\pgfqpoint{1.636699in}{2.982199in}}{\pgfqpoint{1.625649in}{2.982199in}}%
\pgfpathcurveto{\pgfqpoint{1.614599in}{2.982199in}}{\pgfqpoint{1.604000in}{2.977809in}}{\pgfqpoint{1.596187in}{2.969995in}}%
\pgfpathcurveto{\pgfqpoint{1.588373in}{2.962181in}}{\pgfqpoint{1.583983in}{2.951582in}}{\pgfqpoint{1.583983in}{2.940532in}}%
\pgfpathcurveto{\pgfqpoint{1.583983in}{2.929482in}}{\pgfqpoint{1.588373in}{2.918883in}}{\pgfqpoint{1.596187in}{2.911069in}}%
\pgfpathcurveto{\pgfqpoint{1.604000in}{2.903256in}}{\pgfqpoint{1.614599in}{2.898866in}}{\pgfqpoint{1.625649in}{2.898866in}}%
\pgfpathclose%
\pgfusepath{stroke,fill}%
\end{pgfscope}%
\begin{pgfscope}%
\pgfpathrectangle{\pgfqpoint{0.750000in}{0.500000in}}{\pgfqpoint{4.650000in}{3.020000in}}%
\pgfusepath{clip}%
\pgfsetbuttcap%
\pgfsetroundjoin%
\definecolor{currentfill}{rgb}{0.121569,0.466667,0.705882}%
\pgfsetfillcolor{currentfill}%
\pgfsetlinewidth{1.003750pt}%
\definecolor{currentstroke}{rgb}{0.121569,0.466667,0.705882}%
\pgfsetstrokecolor{currentstroke}%
\pgfsetdash{}{0pt}%
\pgfpathmoveto{\pgfqpoint{1.202922in}{0.595606in}}%
\pgfpathcurveto{\pgfqpoint{1.213972in}{0.595606in}}{\pgfqpoint{1.224571in}{0.599996in}}{\pgfqpoint{1.232385in}{0.607810in}}%
\pgfpathcurveto{\pgfqpoint{1.240198in}{0.615624in}}{\pgfqpoint{1.244589in}{0.626223in}}{\pgfqpoint{1.244589in}{0.637273in}}%
\pgfpathcurveto{\pgfqpoint{1.244589in}{0.648323in}}{\pgfqpoint{1.240198in}{0.658922in}}{\pgfqpoint{1.232385in}{0.666736in}}%
\pgfpathcurveto{\pgfqpoint{1.224571in}{0.674549in}}{\pgfqpoint{1.213972in}{0.678939in}}{\pgfqpoint{1.202922in}{0.678939in}}%
\pgfpathcurveto{\pgfqpoint{1.191872in}{0.678939in}}{\pgfqpoint{1.181273in}{0.674549in}}{\pgfqpoint{1.173459in}{0.666736in}}%
\pgfpathcurveto{\pgfqpoint{1.165646in}{0.658922in}}{\pgfqpoint{1.161255in}{0.648323in}}{\pgfqpoint{1.161255in}{0.637273in}}%
\pgfpathcurveto{\pgfqpoint{1.161255in}{0.626223in}}{\pgfqpoint{1.165646in}{0.615624in}}{\pgfqpoint{1.173459in}{0.607810in}}%
\pgfpathcurveto{\pgfqpoint{1.181273in}{0.599996in}}{\pgfqpoint{1.191872in}{0.595606in}}{\pgfqpoint{1.202922in}{0.595606in}}%
\pgfpathclose%
\pgfusepath{stroke,fill}%
\end{pgfscope}%
\begin{pgfscope}%
\pgfpathrectangle{\pgfqpoint{0.750000in}{0.500000in}}{\pgfqpoint{4.650000in}{3.020000in}}%
\pgfusepath{clip}%
\pgfsetbuttcap%
\pgfsetroundjoin%
\definecolor{currentfill}{rgb}{1.000000,0.498039,0.054902}%
\pgfsetfillcolor{currentfill}%
\pgfsetlinewidth{1.003750pt}%
\definecolor{currentstroke}{rgb}{1.000000,0.498039,0.054902}%
\pgfsetstrokecolor{currentstroke}%
\pgfsetdash{}{0pt}%
\pgfpathmoveto{\pgfqpoint{2.531494in}{2.921937in}}%
\pgfpathcurveto{\pgfqpoint{2.542544in}{2.921937in}}{\pgfqpoint{2.553143in}{2.926327in}}{\pgfqpoint{2.560956in}{2.934140in}}%
\pgfpathcurveto{\pgfqpoint{2.568770in}{2.941954in}}{\pgfqpoint{2.573160in}{2.952553in}}{\pgfqpoint{2.573160in}{2.963603in}}%
\pgfpathcurveto{\pgfqpoint{2.573160in}{2.974653in}}{\pgfqpoint{2.568770in}{2.985252in}}{\pgfqpoint{2.560956in}{2.993066in}}%
\pgfpathcurveto{\pgfqpoint{2.553143in}{3.000880in}}{\pgfqpoint{2.542544in}{3.005270in}}{\pgfqpoint{2.531494in}{3.005270in}}%
\pgfpathcurveto{\pgfqpoint{2.520443in}{3.005270in}}{\pgfqpoint{2.509844in}{3.000880in}}{\pgfqpoint{2.502031in}{2.993066in}}%
\pgfpathcurveto{\pgfqpoint{2.494217in}{2.985252in}}{\pgfqpoint{2.489827in}{2.974653in}}{\pgfqpoint{2.489827in}{2.963603in}}%
\pgfpathcurveto{\pgfqpoint{2.489827in}{2.952553in}}{\pgfqpoint{2.494217in}{2.941954in}}{\pgfqpoint{2.502031in}{2.934140in}}%
\pgfpathcurveto{\pgfqpoint{2.509844in}{2.926327in}}{\pgfqpoint{2.520443in}{2.921937in}}{\pgfqpoint{2.531494in}{2.921937in}}%
\pgfpathclose%
\pgfusepath{stroke,fill}%
\end{pgfscope}%
\begin{pgfscope}%
\pgfpathrectangle{\pgfqpoint{0.750000in}{0.500000in}}{\pgfqpoint{4.650000in}{3.020000in}}%
\pgfusepath{clip}%
\pgfsetbuttcap%
\pgfsetroundjoin%
\definecolor{currentfill}{rgb}{1.000000,0.498039,0.054902}%
\pgfsetfillcolor{currentfill}%
\pgfsetlinewidth{1.003750pt}%
\definecolor{currentstroke}{rgb}{1.000000,0.498039,0.054902}%
\pgfsetstrokecolor{currentstroke}%
\pgfsetdash{}{0pt}%
\pgfpathmoveto{\pgfqpoint{2.652273in}{2.914246in}}%
\pgfpathcurveto{\pgfqpoint{2.663323in}{2.914246in}}{\pgfqpoint{2.673922in}{2.918637in}}{\pgfqpoint{2.681736in}{2.926450in}}%
\pgfpathcurveto{\pgfqpoint{2.689549in}{2.934264in}}{\pgfqpoint{2.693939in}{2.944863in}}{\pgfqpoint{2.693939in}{2.955913in}}%
\pgfpathcurveto{\pgfqpoint{2.693939in}{2.966963in}}{\pgfqpoint{2.689549in}{2.977562in}}{\pgfqpoint{2.681736in}{2.985376in}}%
\pgfpathcurveto{\pgfqpoint{2.673922in}{2.993189in}}{\pgfqpoint{2.663323in}{2.997580in}}{\pgfqpoint{2.652273in}{2.997580in}}%
\pgfpathcurveto{\pgfqpoint{2.641223in}{2.997580in}}{\pgfqpoint{2.630624in}{2.993189in}}{\pgfqpoint{2.622810in}{2.985376in}}%
\pgfpathcurveto{\pgfqpoint{2.614996in}{2.977562in}}{\pgfqpoint{2.610606in}{2.966963in}}{\pgfqpoint{2.610606in}{2.955913in}}%
\pgfpathcurveto{\pgfqpoint{2.610606in}{2.944863in}}{\pgfqpoint{2.614996in}{2.934264in}}{\pgfqpoint{2.622810in}{2.926450in}}%
\pgfpathcurveto{\pgfqpoint{2.630624in}{2.918637in}}{\pgfqpoint{2.641223in}{2.914246in}}{\pgfqpoint{2.652273in}{2.914246in}}%
\pgfpathclose%
\pgfusepath{stroke,fill}%
\end{pgfscope}%
\begin{pgfscope}%
\pgfpathrectangle{\pgfqpoint{0.750000in}{0.500000in}}{\pgfqpoint{4.650000in}{3.020000in}}%
\pgfusepath{clip}%
\pgfsetbuttcap%
\pgfsetroundjoin%
\definecolor{currentfill}{rgb}{1.000000,0.498039,0.054902}%
\pgfsetfillcolor{currentfill}%
\pgfsetlinewidth{1.003750pt}%
\definecolor{currentstroke}{rgb}{1.000000,0.498039,0.054902}%
\pgfsetstrokecolor{currentstroke}%
\pgfsetdash{}{0pt}%
\pgfpathmoveto{\pgfqpoint{2.048377in}{2.914246in}}%
\pgfpathcurveto{\pgfqpoint{2.059427in}{2.914246in}}{\pgfqpoint{2.070026in}{2.918637in}}{\pgfqpoint{2.077839in}{2.926450in}}%
\pgfpathcurveto{\pgfqpoint{2.085653in}{2.934264in}}{\pgfqpoint{2.090043in}{2.944863in}}{\pgfqpoint{2.090043in}{2.955913in}}%
\pgfpathcurveto{\pgfqpoint{2.090043in}{2.966963in}}{\pgfqpoint{2.085653in}{2.977562in}}{\pgfqpoint{2.077839in}{2.985376in}}%
\pgfpathcurveto{\pgfqpoint{2.070026in}{2.993189in}}{\pgfqpoint{2.059427in}{2.997580in}}{\pgfqpoint{2.048377in}{2.997580in}}%
\pgfpathcurveto{\pgfqpoint{2.037326in}{2.997580in}}{\pgfqpoint{2.026727in}{2.993189in}}{\pgfqpoint{2.018914in}{2.985376in}}%
\pgfpathcurveto{\pgfqpoint{2.011100in}{2.977562in}}{\pgfqpoint{2.006710in}{2.966963in}}{\pgfqpoint{2.006710in}{2.955913in}}%
\pgfpathcurveto{\pgfqpoint{2.006710in}{2.944863in}}{\pgfqpoint{2.011100in}{2.934264in}}{\pgfqpoint{2.018914in}{2.926450in}}%
\pgfpathcurveto{\pgfqpoint{2.026727in}{2.918637in}}{\pgfqpoint{2.037326in}{2.914246in}}{\pgfqpoint{2.048377in}{2.914246in}}%
\pgfpathclose%
\pgfusepath{stroke,fill}%
\end{pgfscope}%
\begin{pgfscope}%
\pgfpathrectangle{\pgfqpoint{0.750000in}{0.500000in}}{\pgfqpoint{4.650000in}{3.020000in}}%
\pgfusepath{clip}%
\pgfsetbuttcap%
\pgfsetroundjoin%
\definecolor{currentfill}{rgb}{1.000000,0.498039,0.054902}%
\pgfsetfillcolor{currentfill}%
\pgfsetlinewidth{1.003750pt}%
\definecolor{currentstroke}{rgb}{1.000000,0.498039,0.054902}%
\pgfsetstrokecolor{currentstroke}%
\pgfsetdash{}{0pt}%
\pgfpathmoveto{\pgfqpoint{2.169156in}{3.271847in}}%
\pgfpathcurveto{\pgfqpoint{2.180206in}{3.271847in}}{\pgfqpoint{2.190805in}{3.276238in}}{\pgfqpoint{2.198619in}{3.284051in}}%
\pgfpathcurveto{\pgfqpoint{2.206432in}{3.291865in}}{\pgfqpoint{2.210823in}{3.302464in}}{\pgfqpoint{2.210823in}{3.313514in}}%
\pgfpathcurveto{\pgfqpoint{2.210823in}{3.324564in}}{\pgfqpoint{2.206432in}{3.335163in}}{\pgfqpoint{2.198619in}{3.342977in}}%
\pgfpathcurveto{\pgfqpoint{2.190805in}{3.350791in}}{\pgfqpoint{2.180206in}{3.355181in}}{\pgfqpoint{2.169156in}{3.355181in}}%
\pgfpathcurveto{\pgfqpoint{2.158106in}{3.355181in}}{\pgfqpoint{2.147507in}{3.350791in}}{\pgfqpoint{2.139693in}{3.342977in}}%
\pgfpathcurveto{\pgfqpoint{2.131879in}{3.335163in}}{\pgfqpoint{2.127489in}{3.324564in}}{\pgfqpoint{2.127489in}{3.313514in}}%
\pgfpathcurveto{\pgfqpoint{2.127489in}{3.302464in}}{\pgfqpoint{2.131879in}{3.291865in}}{\pgfqpoint{2.139693in}{3.284051in}}%
\pgfpathcurveto{\pgfqpoint{2.147507in}{3.276238in}}{\pgfqpoint{2.158106in}{3.271847in}}{\pgfqpoint{2.169156in}{3.271847in}}%
\pgfpathclose%
\pgfusepath{stroke,fill}%
\end{pgfscope}%
\begin{pgfscope}%
\pgfpathrectangle{\pgfqpoint{0.750000in}{0.500000in}}{\pgfqpoint{4.650000in}{3.020000in}}%
\pgfusepath{clip}%
\pgfsetbuttcap%
\pgfsetroundjoin%
\definecolor{currentfill}{rgb}{1.000000,0.498039,0.054902}%
\pgfsetfillcolor{currentfill}%
\pgfsetlinewidth{1.003750pt}%
\definecolor{currentstroke}{rgb}{1.000000,0.498039,0.054902}%
\pgfsetstrokecolor{currentstroke}%
\pgfsetdash{}{0pt}%
\pgfpathmoveto{\pgfqpoint{1.927597in}{2.910401in}}%
\pgfpathcurveto{\pgfqpoint{1.938648in}{2.910401in}}{\pgfqpoint{1.949247in}{2.914791in}}{\pgfqpoint{1.957060in}{2.922605in}}%
\pgfpathcurveto{\pgfqpoint{1.964874in}{2.930419in}}{\pgfqpoint{1.969264in}{2.941018in}}{\pgfqpoint{1.969264in}{2.952068in}}%
\pgfpathcurveto{\pgfqpoint{1.969264in}{2.963118in}}{\pgfqpoint{1.964874in}{2.973717in}}{\pgfqpoint{1.957060in}{2.981531in}}%
\pgfpathcurveto{\pgfqpoint{1.949247in}{2.989344in}}{\pgfqpoint{1.938648in}{2.993734in}}{\pgfqpoint{1.927597in}{2.993734in}}%
\pgfpathcurveto{\pgfqpoint{1.916547in}{2.993734in}}{\pgfqpoint{1.905948in}{2.989344in}}{\pgfqpoint{1.898135in}{2.981531in}}%
\pgfpathcurveto{\pgfqpoint{1.890321in}{2.973717in}}{\pgfqpoint{1.885931in}{2.963118in}}{\pgfqpoint{1.885931in}{2.952068in}}%
\pgfpathcurveto{\pgfqpoint{1.885931in}{2.941018in}}{\pgfqpoint{1.890321in}{2.930419in}}{\pgfqpoint{1.898135in}{2.922605in}}%
\pgfpathcurveto{\pgfqpoint{1.905948in}{2.914791in}}{\pgfqpoint{1.916547in}{2.910401in}}{\pgfqpoint{1.927597in}{2.910401in}}%
\pgfpathclose%
\pgfusepath{stroke,fill}%
\end{pgfscope}%
\begin{pgfscope}%
\pgfpathrectangle{\pgfqpoint{0.750000in}{0.500000in}}{\pgfqpoint{4.650000in}{3.020000in}}%
\pgfusepath{clip}%
\pgfsetbuttcap%
\pgfsetroundjoin%
\definecolor{currentfill}{rgb}{1.000000,0.498039,0.054902}%
\pgfsetfillcolor{currentfill}%
\pgfsetlinewidth{1.003750pt}%
\definecolor{currentstroke}{rgb}{1.000000,0.498039,0.054902}%
\pgfsetstrokecolor{currentstroke}%
\pgfsetdash{}{0pt}%
\pgfpathmoveto{\pgfqpoint{1.867208in}{2.906556in}}%
\pgfpathcurveto{\pgfqpoint{1.878258in}{2.906556in}}{\pgfqpoint{1.888857in}{2.910946in}}{\pgfqpoint{1.896671in}{2.918760in}}%
\pgfpathcurveto{\pgfqpoint{1.904484in}{2.926573in}}{\pgfqpoint{1.908874in}{2.937172in}}{\pgfqpoint{1.908874in}{2.948223in}}%
\pgfpathcurveto{\pgfqpoint{1.908874in}{2.959273in}}{\pgfqpoint{1.904484in}{2.969872in}}{\pgfqpoint{1.896671in}{2.977685in}}%
\pgfpathcurveto{\pgfqpoint{1.888857in}{2.985499in}}{\pgfqpoint{1.878258in}{2.989889in}}{\pgfqpoint{1.867208in}{2.989889in}}%
\pgfpathcurveto{\pgfqpoint{1.856158in}{2.989889in}}{\pgfqpoint{1.845559in}{2.985499in}}{\pgfqpoint{1.837745in}{2.977685in}}%
\pgfpathcurveto{\pgfqpoint{1.829931in}{2.969872in}}{\pgfqpoint{1.825541in}{2.959273in}}{\pgfqpoint{1.825541in}{2.948223in}}%
\pgfpathcurveto{\pgfqpoint{1.825541in}{2.937172in}}{\pgfqpoint{1.829931in}{2.926573in}}{\pgfqpoint{1.837745in}{2.918760in}}%
\pgfpathcurveto{\pgfqpoint{1.845559in}{2.910946in}}{\pgfqpoint{1.856158in}{2.906556in}}{\pgfqpoint{1.867208in}{2.906556in}}%
\pgfpathclose%
\pgfusepath{stroke,fill}%
\end{pgfscope}%
\begin{pgfscope}%
\pgfpathrectangle{\pgfqpoint{0.750000in}{0.500000in}}{\pgfqpoint{4.650000in}{3.020000in}}%
\pgfusepath{clip}%
\pgfsetbuttcap%
\pgfsetroundjoin%
\definecolor{currentfill}{rgb}{0.121569,0.466667,0.705882}%
\pgfsetfillcolor{currentfill}%
\pgfsetlinewidth{1.003750pt}%
\definecolor{currentstroke}{rgb}{0.121569,0.466667,0.705882}%
\pgfsetstrokecolor{currentstroke}%
\pgfsetdash{}{0pt}%
\pgfpathmoveto{\pgfqpoint{4.403571in}{0.599451in}}%
\pgfpathcurveto{\pgfqpoint{4.414622in}{0.599451in}}{\pgfqpoint{4.425221in}{0.603841in}}{\pgfqpoint{4.433034in}{0.611655in}}%
\pgfpathcurveto{\pgfqpoint{4.440848in}{0.619469in}}{\pgfqpoint{4.445238in}{0.630068in}}{\pgfqpoint{4.445238in}{0.641118in}}%
\pgfpathcurveto{\pgfqpoint{4.445238in}{0.652168in}}{\pgfqpoint{4.440848in}{0.662767in}}{\pgfqpoint{4.433034in}{0.670581in}}%
\pgfpathcurveto{\pgfqpoint{4.425221in}{0.678394in}}{\pgfqpoint{4.414622in}{0.682785in}}{\pgfqpoint{4.403571in}{0.682785in}}%
\pgfpathcurveto{\pgfqpoint{4.392521in}{0.682785in}}{\pgfqpoint{4.381922in}{0.678394in}}{\pgfqpoint{4.374109in}{0.670581in}}%
\pgfpathcurveto{\pgfqpoint{4.366295in}{0.662767in}}{\pgfqpoint{4.361905in}{0.652168in}}{\pgfqpoint{4.361905in}{0.641118in}}%
\pgfpathcurveto{\pgfqpoint{4.361905in}{0.630068in}}{\pgfqpoint{4.366295in}{0.619469in}}{\pgfqpoint{4.374109in}{0.611655in}}%
\pgfpathcurveto{\pgfqpoint{4.381922in}{0.603841in}}{\pgfqpoint{4.392521in}{0.599451in}}{\pgfqpoint{4.403571in}{0.599451in}}%
\pgfpathclose%
\pgfusepath{stroke,fill}%
\end{pgfscope}%
\begin{pgfscope}%
\pgfpathrectangle{\pgfqpoint{0.750000in}{0.500000in}}{\pgfqpoint{4.650000in}{3.020000in}}%
\pgfusepath{clip}%
\pgfsetbuttcap%
\pgfsetroundjoin%
\definecolor{currentfill}{rgb}{0.121569,0.466667,0.705882}%
\pgfsetfillcolor{currentfill}%
\pgfsetlinewidth{1.003750pt}%
\definecolor{currentstroke}{rgb}{0.121569,0.466667,0.705882}%
\pgfsetstrokecolor{currentstroke}%
\pgfsetdash{}{0pt}%
\pgfpathmoveto{\pgfqpoint{1.504870in}{0.595606in}}%
\pgfpathcurveto{\pgfqpoint{1.515920in}{0.595606in}}{\pgfqpoint{1.526519in}{0.599996in}}{\pgfqpoint{1.534333in}{0.607810in}}%
\pgfpathcurveto{\pgfqpoint{1.542147in}{0.615624in}}{\pgfqpoint{1.546537in}{0.626223in}}{\pgfqpoint{1.546537in}{0.637273in}}%
\pgfpathcurveto{\pgfqpoint{1.546537in}{0.648323in}}{\pgfqpoint{1.542147in}{0.658922in}}{\pgfqpoint{1.534333in}{0.666736in}}%
\pgfpathcurveto{\pgfqpoint{1.526519in}{0.674549in}}{\pgfqpoint{1.515920in}{0.678939in}}{\pgfqpoint{1.504870in}{0.678939in}}%
\pgfpathcurveto{\pgfqpoint{1.493820in}{0.678939in}}{\pgfqpoint{1.483221in}{0.674549in}}{\pgfqpoint{1.475407in}{0.666736in}}%
\pgfpathcurveto{\pgfqpoint{1.467594in}{0.658922in}}{\pgfqpoint{1.463203in}{0.648323in}}{\pgfqpoint{1.463203in}{0.637273in}}%
\pgfpathcurveto{\pgfqpoint{1.463203in}{0.626223in}}{\pgfqpoint{1.467594in}{0.615624in}}{\pgfqpoint{1.475407in}{0.607810in}}%
\pgfpathcurveto{\pgfqpoint{1.483221in}{0.599996in}}{\pgfqpoint{1.493820in}{0.595606in}}{\pgfqpoint{1.504870in}{0.595606in}}%
\pgfpathclose%
\pgfusepath{stroke,fill}%
\end{pgfscope}%
\begin{pgfscope}%
\pgfpathrectangle{\pgfqpoint{0.750000in}{0.500000in}}{\pgfqpoint{4.650000in}{3.020000in}}%
\pgfusepath{clip}%
\pgfsetbuttcap%
\pgfsetroundjoin%
\definecolor{currentfill}{rgb}{0.121569,0.466667,0.705882}%
\pgfsetfillcolor{currentfill}%
\pgfsetlinewidth{1.003750pt}%
\definecolor{currentstroke}{rgb}{0.121569,0.466667,0.705882}%
\pgfsetstrokecolor{currentstroke}%
\pgfsetdash{}{0pt}%
\pgfpathmoveto{\pgfqpoint{1.384091in}{0.595606in}}%
\pgfpathcurveto{\pgfqpoint{1.395141in}{0.595606in}}{\pgfqpoint{1.405740in}{0.599996in}}{\pgfqpoint{1.413554in}{0.607810in}}%
\pgfpathcurveto{\pgfqpoint{1.421367in}{0.615624in}}{\pgfqpoint{1.425758in}{0.626223in}}{\pgfqpoint{1.425758in}{0.637273in}}%
\pgfpathcurveto{\pgfqpoint{1.425758in}{0.648323in}}{\pgfqpoint{1.421367in}{0.658922in}}{\pgfqpoint{1.413554in}{0.666736in}}%
\pgfpathcurveto{\pgfqpoint{1.405740in}{0.674549in}}{\pgfqpoint{1.395141in}{0.678939in}}{\pgfqpoint{1.384091in}{0.678939in}}%
\pgfpathcurveto{\pgfqpoint{1.373041in}{0.678939in}}{\pgfqpoint{1.362442in}{0.674549in}}{\pgfqpoint{1.354628in}{0.666736in}}%
\pgfpathcurveto{\pgfqpoint{1.346815in}{0.658922in}}{\pgfqpoint{1.342424in}{0.648323in}}{\pgfqpoint{1.342424in}{0.637273in}}%
\pgfpathcurveto{\pgfqpoint{1.342424in}{0.626223in}}{\pgfqpoint{1.346815in}{0.615624in}}{\pgfqpoint{1.354628in}{0.607810in}}%
\pgfpathcurveto{\pgfqpoint{1.362442in}{0.599996in}}{\pgfqpoint{1.373041in}{0.595606in}}{\pgfqpoint{1.384091in}{0.595606in}}%
\pgfpathclose%
\pgfusepath{stroke,fill}%
\end{pgfscope}%
\begin{pgfscope}%
\pgfpathrectangle{\pgfqpoint{0.750000in}{0.500000in}}{\pgfqpoint{4.650000in}{3.020000in}}%
\pgfusepath{clip}%
\pgfsetbuttcap%
\pgfsetroundjoin%
\definecolor{currentfill}{rgb}{0.121569,0.466667,0.705882}%
\pgfsetfillcolor{currentfill}%
\pgfsetlinewidth{1.003750pt}%
\definecolor{currentstroke}{rgb}{0.121569,0.466667,0.705882}%
\pgfsetstrokecolor{currentstroke}%
\pgfsetdash{}{0pt}%
\pgfpathmoveto{\pgfqpoint{1.323701in}{0.595606in}}%
\pgfpathcurveto{\pgfqpoint{1.334751in}{0.595606in}}{\pgfqpoint{1.345350in}{0.599996in}}{\pgfqpoint{1.353164in}{0.607810in}}%
\pgfpathcurveto{\pgfqpoint{1.360978in}{0.615624in}}{\pgfqpoint{1.365368in}{0.626223in}}{\pgfqpoint{1.365368in}{0.637273in}}%
\pgfpathcurveto{\pgfqpoint{1.365368in}{0.648323in}}{\pgfqpoint{1.360978in}{0.658922in}}{\pgfqpoint{1.353164in}{0.666736in}}%
\pgfpathcurveto{\pgfqpoint{1.345350in}{0.674549in}}{\pgfqpoint{1.334751in}{0.678939in}}{\pgfqpoint{1.323701in}{0.678939in}}%
\pgfpathcurveto{\pgfqpoint{1.312651in}{0.678939in}}{\pgfqpoint{1.302052in}{0.674549in}}{\pgfqpoint{1.294239in}{0.666736in}}%
\pgfpathcurveto{\pgfqpoint{1.286425in}{0.658922in}}{\pgfqpoint{1.282035in}{0.648323in}}{\pgfqpoint{1.282035in}{0.637273in}}%
\pgfpathcurveto{\pgfqpoint{1.282035in}{0.626223in}}{\pgfqpoint{1.286425in}{0.615624in}}{\pgfqpoint{1.294239in}{0.607810in}}%
\pgfpathcurveto{\pgfqpoint{1.302052in}{0.599996in}}{\pgfqpoint{1.312651in}{0.595606in}}{\pgfqpoint{1.323701in}{0.595606in}}%
\pgfpathclose%
\pgfusepath{stroke,fill}%
\end{pgfscope}%
\begin{pgfscope}%
\pgfpathrectangle{\pgfqpoint{0.750000in}{0.500000in}}{\pgfqpoint{4.650000in}{3.020000in}}%
\pgfusepath{clip}%
\pgfsetbuttcap%
\pgfsetroundjoin%
\definecolor{currentfill}{rgb}{0.121569,0.466667,0.705882}%
\pgfsetfillcolor{currentfill}%
\pgfsetlinewidth{1.003750pt}%
\definecolor{currentstroke}{rgb}{0.121569,0.466667,0.705882}%
\pgfsetstrokecolor{currentstroke}%
\pgfsetdash{}{0pt}%
\pgfpathmoveto{\pgfqpoint{1.323701in}{0.595606in}}%
\pgfpathcurveto{\pgfqpoint{1.334751in}{0.595606in}}{\pgfqpoint{1.345350in}{0.599996in}}{\pgfqpoint{1.353164in}{0.607810in}}%
\pgfpathcurveto{\pgfqpoint{1.360978in}{0.615624in}}{\pgfqpoint{1.365368in}{0.626223in}}{\pgfqpoint{1.365368in}{0.637273in}}%
\pgfpathcurveto{\pgfqpoint{1.365368in}{0.648323in}}{\pgfqpoint{1.360978in}{0.658922in}}{\pgfqpoint{1.353164in}{0.666736in}}%
\pgfpathcurveto{\pgfqpoint{1.345350in}{0.674549in}}{\pgfqpoint{1.334751in}{0.678939in}}{\pgfqpoint{1.323701in}{0.678939in}}%
\pgfpathcurveto{\pgfqpoint{1.312651in}{0.678939in}}{\pgfqpoint{1.302052in}{0.674549in}}{\pgfqpoint{1.294239in}{0.666736in}}%
\pgfpathcurveto{\pgfqpoint{1.286425in}{0.658922in}}{\pgfqpoint{1.282035in}{0.648323in}}{\pgfqpoint{1.282035in}{0.637273in}}%
\pgfpathcurveto{\pgfqpoint{1.282035in}{0.626223in}}{\pgfqpoint{1.286425in}{0.615624in}}{\pgfqpoint{1.294239in}{0.607810in}}%
\pgfpathcurveto{\pgfqpoint{1.302052in}{0.599996in}}{\pgfqpoint{1.312651in}{0.595606in}}{\pgfqpoint{1.323701in}{0.595606in}}%
\pgfpathclose%
\pgfusepath{stroke,fill}%
\end{pgfscope}%
\begin{pgfscope}%
\pgfpathrectangle{\pgfqpoint{0.750000in}{0.500000in}}{\pgfqpoint{4.650000in}{3.020000in}}%
\pgfusepath{clip}%
\pgfsetbuttcap%
\pgfsetroundjoin%
\definecolor{currentfill}{rgb}{0.121569,0.466667,0.705882}%
\pgfsetfillcolor{currentfill}%
\pgfsetlinewidth{1.003750pt}%
\definecolor{currentstroke}{rgb}{0.121569,0.466667,0.705882}%
\pgfsetstrokecolor{currentstroke}%
\pgfsetdash{}{0pt}%
\pgfpathmoveto{\pgfqpoint{1.444481in}{0.595606in}}%
\pgfpathcurveto{\pgfqpoint{1.455531in}{0.595606in}}{\pgfqpoint{1.466130in}{0.599996in}}{\pgfqpoint{1.473943in}{0.607810in}}%
\pgfpathcurveto{\pgfqpoint{1.481757in}{0.615624in}}{\pgfqpoint{1.486147in}{0.626223in}}{\pgfqpoint{1.486147in}{0.637273in}}%
\pgfpathcurveto{\pgfqpoint{1.486147in}{0.648323in}}{\pgfqpoint{1.481757in}{0.658922in}}{\pgfqpoint{1.473943in}{0.666736in}}%
\pgfpathcurveto{\pgfqpoint{1.466130in}{0.674549in}}{\pgfqpoint{1.455531in}{0.678939in}}{\pgfqpoint{1.444481in}{0.678939in}}%
\pgfpathcurveto{\pgfqpoint{1.433430in}{0.678939in}}{\pgfqpoint{1.422831in}{0.674549in}}{\pgfqpoint{1.415018in}{0.666736in}}%
\pgfpathcurveto{\pgfqpoint{1.407204in}{0.658922in}}{\pgfqpoint{1.402814in}{0.648323in}}{\pgfqpoint{1.402814in}{0.637273in}}%
\pgfpathcurveto{\pgfqpoint{1.402814in}{0.626223in}}{\pgfqpoint{1.407204in}{0.615624in}}{\pgfqpoint{1.415018in}{0.607810in}}%
\pgfpathcurveto{\pgfqpoint{1.422831in}{0.599996in}}{\pgfqpoint{1.433430in}{0.595606in}}{\pgfqpoint{1.444481in}{0.595606in}}%
\pgfpathclose%
\pgfusepath{stroke,fill}%
\end{pgfscope}%
\begin{pgfscope}%
\pgfpathrectangle{\pgfqpoint{0.750000in}{0.500000in}}{\pgfqpoint{4.650000in}{3.020000in}}%
\pgfusepath{clip}%
\pgfsetbuttcap%
\pgfsetroundjoin%
\definecolor{currentfill}{rgb}{0.839216,0.152941,0.156863}%
\pgfsetfillcolor{currentfill}%
\pgfsetlinewidth{1.003750pt}%
\definecolor{currentstroke}{rgb}{0.839216,0.152941,0.156863}%
\pgfsetstrokecolor{currentstroke}%
\pgfsetdash{}{0pt}%
\pgfpathmoveto{\pgfqpoint{1.444481in}{2.902711in}}%
\pgfpathcurveto{\pgfqpoint{1.455531in}{2.902711in}}{\pgfqpoint{1.466130in}{2.907101in}}{\pgfqpoint{1.473943in}{2.914915in}}%
\pgfpathcurveto{\pgfqpoint{1.481757in}{2.922728in}}{\pgfqpoint{1.486147in}{2.933327in}}{\pgfqpoint{1.486147in}{2.944377in}}%
\pgfpathcurveto{\pgfqpoint{1.486147in}{2.955428in}}{\pgfqpoint{1.481757in}{2.966027in}}{\pgfqpoint{1.473943in}{2.973840in}}%
\pgfpathcurveto{\pgfqpoint{1.466130in}{2.981654in}}{\pgfqpoint{1.455531in}{2.986044in}}{\pgfqpoint{1.444481in}{2.986044in}}%
\pgfpathcurveto{\pgfqpoint{1.433430in}{2.986044in}}{\pgfqpoint{1.422831in}{2.981654in}}{\pgfqpoint{1.415018in}{2.973840in}}%
\pgfpathcurveto{\pgfqpoint{1.407204in}{2.966027in}}{\pgfqpoint{1.402814in}{2.955428in}}{\pgfqpoint{1.402814in}{2.944377in}}%
\pgfpathcurveto{\pgfqpoint{1.402814in}{2.933327in}}{\pgfqpoint{1.407204in}{2.922728in}}{\pgfqpoint{1.415018in}{2.914915in}}%
\pgfpathcurveto{\pgfqpoint{1.422831in}{2.907101in}}{\pgfqpoint{1.433430in}{2.902711in}}{\pgfqpoint{1.444481in}{2.902711in}}%
\pgfpathclose%
\pgfusepath{stroke,fill}%
\end{pgfscope}%
\begin{pgfscope}%
\pgfpathrectangle{\pgfqpoint{0.750000in}{0.500000in}}{\pgfqpoint{4.650000in}{3.020000in}}%
\pgfusepath{clip}%
\pgfsetbuttcap%
\pgfsetroundjoin%
\definecolor{currentfill}{rgb}{1.000000,0.498039,0.054902}%
\pgfsetfillcolor{currentfill}%
\pgfsetlinewidth{1.003750pt}%
\definecolor{currentstroke}{rgb}{1.000000,0.498039,0.054902}%
\pgfsetstrokecolor{currentstroke}%
\pgfsetdash{}{0pt}%
\pgfpathmoveto{\pgfqpoint{2.289935in}{2.898866in}}%
\pgfpathcurveto{\pgfqpoint{2.300985in}{2.898866in}}{\pgfqpoint{2.311584in}{2.903256in}}{\pgfqpoint{2.319398in}{2.911069in}}%
\pgfpathcurveto{\pgfqpoint{2.327211in}{2.918883in}}{\pgfqpoint{2.331602in}{2.929482in}}{\pgfqpoint{2.331602in}{2.940532in}}%
\pgfpathcurveto{\pgfqpoint{2.331602in}{2.951582in}}{\pgfqpoint{2.327211in}{2.962181in}}{\pgfqpoint{2.319398in}{2.969995in}}%
\pgfpathcurveto{\pgfqpoint{2.311584in}{2.977809in}}{\pgfqpoint{2.300985in}{2.982199in}}{\pgfqpoint{2.289935in}{2.982199in}}%
\pgfpathcurveto{\pgfqpoint{2.278885in}{2.982199in}}{\pgfqpoint{2.268286in}{2.977809in}}{\pgfqpoint{2.260472in}{2.969995in}}%
\pgfpathcurveto{\pgfqpoint{2.252659in}{2.962181in}}{\pgfqpoint{2.248268in}{2.951582in}}{\pgfqpoint{2.248268in}{2.940532in}}%
\pgfpathcurveto{\pgfqpoint{2.248268in}{2.929482in}}{\pgfqpoint{2.252659in}{2.918883in}}{\pgfqpoint{2.260472in}{2.911069in}}%
\pgfpathcurveto{\pgfqpoint{2.268286in}{2.903256in}}{\pgfqpoint{2.278885in}{2.898866in}}{\pgfqpoint{2.289935in}{2.898866in}}%
\pgfpathclose%
\pgfusepath{stroke,fill}%
\end{pgfscope}%
\begin{pgfscope}%
\pgfpathrectangle{\pgfqpoint{0.750000in}{0.500000in}}{\pgfqpoint{4.650000in}{3.020000in}}%
\pgfusepath{clip}%
\pgfsetbuttcap%
\pgfsetroundjoin%
\definecolor{currentfill}{rgb}{1.000000,0.498039,0.054902}%
\pgfsetfillcolor{currentfill}%
\pgfsetlinewidth{1.003750pt}%
\definecolor{currentstroke}{rgb}{1.000000,0.498039,0.054902}%
\pgfsetstrokecolor{currentstroke}%
\pgfsetdash{}{0pt}%
\pgfpathmoveto{\pgfqpoint{2.229545in}{2.910401in}}%
\pgfpathcurveto{\pgfqpoint{2.240596in}{2.910401in}}{\pgfqpoint{2.251195in}{2.914791in}}{\pgfqpoint{2.259008in}{2.922605in}}%
\pgfpathcurveto{\pgfqpoint{2.266822in}{2.930419in}}{\pgfqpoint{2.271212in}{2.941018in}}{\pgfqpoint{2.271212in}{2.952068in}}%
\pgfpathcurveto{\pgfqpoint{2.271212in}{2.963118in}}{\pgfqpoint{2.266822in}{2.973717in}}{\pgfqpoint{2.259008in}{2.981531in}}%
\pgfpathcurveto{\pgfqpoint{2.251195in}{2.989344in}}{\pgfqpoint{2.240596in}{2.993734in}}{\pgfqpoint{2.229545in}{2.993734in}}%
\pgfpathcurveto{\pgfqpoint{2.218495in}{2.993734in}}{\pgfqpoint{2.207896in}{2.989344in}}{\pgfqpoint{2.200083in}{2.981531in}}%
\pgfpathcurveto{\pgfqpoint{2.192269in}{2.973717in}}{\pgfqpoint{2.187879in}{2.963118in}}{\pgfqpoint{2.187879in}{2.952068in}}%
\pgfpathcurveto{\pgfqpoint{2.187879in}{2.941018in}}{\pgfqpoint{2.192269in}{2.930419in}}{\pgfqpoint{2.200083in}{2.922605in}}%
\pgfpathcurveto{\pgfqpoint{2.207896in}{2.914791in}}{\pgfqpoint{2.218495in}{2.910401in}}{\pgfqpoint{2.229545in}{2.910401in}}%
\pgfpathclose%
\pgfusepath{stroke,fill}%
\end{pgfscope}%
\begin{pgfscope}%
\pgfpathrectangle{\pgfqpoint{0.750000in}{0.500000in}}{\pgfqpoint{4.650000in}{3.020000in}}%
\pgfusepath{clip}%
\pgfsetbuttcap%
\pgfsetroundjoin%
\definecolor{currentfill}{rgb}{1.000000,0.498039,0.054902}%
\pgfsetfillcolor{currentfill}%
\pgfsetlinewidth{1.003750pt}%
\definecolor{currentstroke}{rgb}{1.000000,0.498039,0.054902}%
\pgfsetstrokecolor{currentstroke}%
\pgfsetdash{}{0pt}%
\pgfpathmoveto{\pgfqpoint{2.108766in}{2.910401in}}%
\pgfpathcurveto{\pgfqpoint{2.119816in}{2.910401in}}{\pgfqpoint{2.130415in}{2.914791in}}{\pgfqpoint{2.138229in}{2.922605in}}%
\pgfpathcurveto{\pgfqpoint{2.146043in}{2.930419in}}{\pgfqpoint{2.150433in}{2.941018in}}{\pgfqpoint{2.150433in}{2.952068in}}%
\pgfpathcurveto{\pgfqpoint{2.150433in}{2.963118in}}{\pgfqpoint{2.146043in}{2.973717in}}{\pgfqpoint{2.138229in}{2.981531in}}%
\pgfpathcurveto{\pgfqpoint{2.130415in}{2.989344in}}{\pgfqpoint{2.119816in}{2.993734in}}{\pgfqpoint{2.108766in}{2.993734in}}%
\pgfpathcurveto{\pgfqpoint{2.097716in}{2.993734in}}{\pgfqpoint{2.087117in}{2.989344in}}{\pgfqpoint{2.079303in}{2.981531in}}%
\pgfpathcurveto{\pgfqpoint{2.071490in}{2.973717in}}{\pgfqpoint{2.067100in}{2.963118in}}{\pgfqpoint{2.067100in}{2.952068in}}%
\pgfpathcurveto{\pgfqpoint{2.067100in}{2.941018in}}{\pgfqpoint{2.071490in}{2.930419in}}{\pgfqpoint{2.079303in}{2.922605in}}%
\pgfpathcurveto{\pgfqpoint{2.087117in}{2.914791in}}{\pgfqpoint{2.097716in}{2.910401in}}{\pgfqpoint{2.108766in}{2.910401in}}%
\pgfpathclose%
\pgfusepath{stroke,fill}%
\end{pgfscope}%
\begin{pgfscope}%
\pgfpathrectangle{\pgfqpoint{0.750000in}{0.500000in}}{\pgfqpoint{4.650000in}{3.020000in}}%
\pgfusepath{clip}%
\pgfsetbuttcap%
\pgfsetroundjoin%
\definecolor{currentfill}{rgb}{0.121569,0.466667,0.705882}%
\pgfsetfillcolor{currentfill}%
\pgfsetlinewidth{1.003750pt}%
\definecolor{currentstroke}{rgb}{0.121569,0.466667,0.705882}%
\pgfsetstrokecolor{currentstroke}%
\pgfsetdash{}{0pt}%
\pgfpathmoveto{\pgfqpoint{1.384091in}{0.860923in}}%
\pgfpathcurveto{\pgfqpoint{1.395141in}{0.860923in}}{\pgfqpoint{1.405740in}{0.865313in}}{\pgfqpoint{1.413554in}{0.873127in}}%
\pgfpathcurveto{\pgfqpoint{1.421367in}{0.880941in}}{\pgfqpoint{1.425758in}{0.891540in}}{\pgfqpoint{1.425758in}{0.902590in}}%
\pgfpathcurveto{\pgfqpoint{1.425758in}{0.913640in}}{\pgfqpoint{1.421367in}{0.924239in}}{\pgfqpoint{1.413554in}{0.932053in}}%
\pgfpathcurveto{\pgfqpoint{1.405740in}{0.939866in}}{\pgfqpoint{1.395141in}{0.944256in}}{\pgfqpoint{1.384091in}{0.944256in}}%
\pgfpathcurveto{\pgfqpoint{1.373041in}{0.944256in}}{\pgfqpoint{1.362442in}{0.939866in}}{\pgfqpoint{1.354628in}{0.932053in}}%
\pgfpathcurveto{\pgfqpoint{1.346815in}{0.924239in}}{\pgfqpoint{1.342424in}{0.913640in}}{\pgfqpoint{1.342424in}{0.902590in}}%
\pgfpathcurveto{\pgfqpoint{1.342424in}{0.891540in}}{\pgfqpoint{1.346815in}{0.880941in}}{\pgfqpoint{1.354628in}{0.873127in}}%
\pgfpathcurveto{\pgfqpoint{1.362442in}{0.865313in}}{\pgfqpoint{1.373041in}{0.860923in}}{\pgfqpoint{1.384091in}{0.860923in}}%
\pgfpathclose%
\pgfusepath{stroke,fill}%
\end{pgfscope}%
\begin{pgfscope}%
\pgfpathrectangle{\pgfqpoint{0.750000in}{0.500000in}}{\pgfqpoint{4.650000in}{3.020000in}}%
\pgfusepath{clip}%
\pgfsetbuttcap%
\pgfsetroundjoin%
\definecolor{currentfill}{rgb}{0.121569,0.466667,0.705882}%
\pgfsetfillcolor{currentfill}%
\pgfsetlinewidth{1.003750pt}%
\definecolor{currentstroke}{rgb}{0.121569,0.466667,0.705882}%
\pgfsetstrokecolor{currentstroke}%
\pgfsetdash{}{0pt}%
\pgfpathmoveto{\pgfqpoint{1.625649in}{0.595606in}}%
\pgfpathcurveto{\pgfqpoint{1.636699in}{0.595606in}}{\pgfqpoint{1.647299in}{0.599996in}}{\pgfqpoint{1.655112in}{0.607810in}}%
\pgfpathcurveto{\pgfqpoint{1.662926in}{0.615624in}}{\pgfqpoint{1.667316in}{0.626223in}}{\pgfqpoint{1.667316in}{0.637273in}}%
\pgfpathcurveto{\pgfqpoint{1.667316in}{0.648323in}}{\pgfqpoint{1.662926in}{0.658922in}}{\pgfqpoint{1.655112in}{0.666736in}}%
\pgfpathcurveto{\pgfqpoint{1.647299in}{0.674549in}}{\pgfqpoint{1.636699in}{0.678939in}}{\pgfqpoint{1.625649in}{0.678939in}}%
\pgfpathcurveto{\pgfqpoint{1.614599in}{0.678939in}}{\pgfqpoint{1.604000in}{0.674549in}}{\pgfqpoint{1.596187in}{0.666736in}}%
\pgfpathcurveto{\pgfqpoint{1.588373in}{0.658922in}}{\pgfqpoint{1.583983in}{0.648323in}}{\pgfqpoint{1.583983in}{0.637273in}}%
\pgfpathcurveto{\pgfqpoint{1.583983in}{0.626223in}}{\pgfqpoint{1.588373in}{0.615624in}}{\pgfqpoint{1.596187in}{0.607810in}}%
\pgfpathcurveto{\pgfqpoint{1.604000in}{0.599996in}}{\pgfqpoint{1.614599in}{0.595606in}}{\pgfqpoint{1.625649in}{0.595606in}}%
\pgfpathclose%
\pgfusepath{stroke,fill}%
\end{pgfscope}%
\begin{pgfscope}%
\pgfpathrectangle{\pgfqpoint{0.750000in}{0.500000in}}{\pgfqpoint{4.650000in}{3.020000in}}%
\pgfusepath{clip}%
\pgfsetbuttcap%
\pgfsetroundjoin%
\definecolor{currentfill}{rgb}{1.000000,0.498039,0.054902}%
\pgfsetfillcolor{currentfill}%
\pgfsetlinewidth{1.003750pt}%
\definecolor{currentstroke}{rgb}{1.000000,0.498039,0.054902}%
\pgfsetstrokecolor{currentstroke}%
\pgfsetdash{}{0pt}%
\pgfpathmoveto{\pgfqpoint{3.376948in}{2.929627in}}%
\pgfpathcurveto{\pgfqpoint{3.387998in}{2.929627in}}{\pgfqpoint{3.398597in}{2.934017in}}{\pgfqpoint{3.406411in}{2.941831in}}%
\pgfpathcurveto{\pgfqpoint{3.414224in}{2.949644in}}{\pgfqpoint{3.418615in}{2.960243in}}{\pgfqpoint{3.418615in}{2.971294in}}%
\pgfpathcurveto{\pgfqpoint{3.418615in}{2.982344in}}{\pgfqpoint{3.414224in}{2.992943in}}{\pgfqpoint{3.406411in}{3.000756in}}%
\pgfpathcurveto{\pgfqpoint{3.398597in}{3.008570in}}{\pgfqpoint{3.387998in}{3.012960in}}{\pgfqpoint{3.376948in}{3.012960in}}%
\pgfpathcurveto{\pgfqpoint{3.365898in}{3.012960in}}{\pgfqpoint{3.355299in}{3.008570in}}{\pgfqpoint{3.347485in}{3.000756in}}%
\pgfpathcurveto{\pgfqpoint{3.339672in}{2.992943in}}{\pgfqpoint{3.335281in}{2.982344in}}{\pgfqpoint{3.335281in}{2.971294in}}%
\pgfpathcurveto{\pgfqpoint{3.335281in}{2.960243in}}{\pgfqpoint{3.339672in}{2.949644in}}{\pgfqpoint{3.347485in}{2.941831in}}%
\pgfpathcurveto{\pgfqpoint{3.355299in}{2.934017in}}{\pgfqpoint{3.365898in}{2.929627in}}{\pgfqpoint{3.376948in}{2.929627in}}%
\pgfpathclose%
\pgfusepath{stroke,fill}%
\end{pgfscope}%
\begin{pgfscope}%
\pgfpathrectangle{\pgfqpoint{0.750000in}{0.500000in}}{\pgfqpoint{4.650000in}{3.020000in}}%
\pgfusepath{clip}%
\pgfsetbuttcap%
\pgfsetroundjoin%
\definecolor{currentfill}{rgb}{1.000000,0.498039,0.054902}%
\pgfsetfillcolor{currentfill}%
\pgfsetlinewidth{1.003750pt}%
\definecolor{currentstroke}{rgb}{1.000000,0.498039,0.054902}%
\pgfsetstrokecolor{currentstroke}%
\pgfsetdash{}{0pt}%
\pgfpathmoveto{\pgfqpoint{1.504870in}{2.906556in}}%
\pgfpathcurveto{\pgfqpoint{1.515920in}{2.906556in}}{\pgfqpoint{1.526519in}{2.910946in}}{\pgfqpoint{1.534333in}{2.918760in}}%
\pgfpathcurveto{\pgfqpoint{1.542147in}{2.926573in}}{\pgfqpoint{1.546537in}{2.937172in}}{\pgfqpoint{1.546537in}{2.948223in}}%
\pgfpathcurveto{\pgfqpoint{1.546537in}{2.959273in}}{\pgfqpoint{1.542147in}{2.969872in}}{\pgfqpoint{1.534333in}{2.977685in}}%
\pgfpathcurveto{\pgfqpoint{1.526519in}{2.985499in}}{\pgfqpoint{1.515920in}{2.989889in}}{\pgfqpoint{1.504870in}{2.989889in}}%
\pgfpathcurveto{\pgfqpoint{1.493820in}{2.989889in}}{\pgfqpoint{1.483221in}{2.985499in}}{\pgfqpoint{1.475407in}{2.977685in}}%
\pgfpathcurveto{\pgfqpoint{1.467594in}{2.969872in}}{\pgfqpoint{1.463203in}{2.959273in}}{\pgfqpoint{1.463203in}{2.948223in}}%
\pgfpathcurveto{\pgfqpoint{1.463203in}{2.937172in}}{\pgfqpoint{1.467594in}{2.926573in}}{\pgfqpoint{1.475407in}{2.918760in}}%
\pgfpathcurveto{\pgfqpoint{1.483221in}{2.910946in}}{\pgfqpoint{1.493820in}{2.906556in}}{\pgfqpoint{1.504870in}{2.906556in}}%
\pgfpathclose%
\pgfusepath{stroke,fill}%
\end{pgfscope}%
\begin{pgfscope}%
\pgfpathrectangle{\pgfqpoint{0.750000in}{0.500000in}}{\pgfqpoint{4.650000in}{3.020000in}}%
\pgfusepath{clip}%
\pgfsetbuttcap%
\pgfsetroundjoin%
\definecolor{currentfill}{rgb}{0.121569,0.466667,0.705882}%
\pgfsetfillcolor{currentfill}%
\pgfsetlinewidth{1.003750pt}%
\definecolor{currentstroke}{rgb}{0.121569,0.466667,0.705882}%
\pgfsetstrokecolor{currentstroke}%
\pgfsetdash{}{0pt}%
\pgfpathmoveto{\pgfqpoint{1.202922in}{0.745568in}}%
\pgfpathcurveto{\pgfqpoint{1.213972in}{0.745568in}}{\pgfqpoint{1.224571in}{0.749958in}}{\pgfqpoint{1.232385in}{0.757772in}}%
\pgfpathcurveto{\pgfqpoint{1.240198in}{0.765585in}}{\pgfqpoint{1.244589in}{0.776184in}}{\pgfqpoint{1.244589in}{0.787235in}}%
\pgfpathcurveto{\pgfqpoint{1.244589in}{0.798285in}}{\pgfqpoint{1.240198in}{0.808884in}}{\pgfqpoint{1.232385in}{0.816697in}}%
\pgfpathcurveto{\pgfqpoint{1.224571in}{0.824511in}}{\pgfqpoint{1.213972in}{0.828901in}}{\pgfqpoint{1.202922in}{0.828901in}}%
\pgfpathcurveto{\pgfqpoint{1.191872in}{0.828901in}}{\pgfqpoint{1.181273in}{0.824511in}}{\pgfqpoint{1.173459in}{0.816697in}}%
\pgfpathcurveto{\pgfqpoint{1.165646in}{0.808884in}}{\pgfqpoint{1.161255in}{0.798285in}}{\pgfqpoint{1.161255in}{0.787235in}}%
\pgfpathcurveto{\pgfqpoint{1.161255in}{0.776184in}}{\pgfqpoint{1.165646in}{0.765585in}}{\pgfqpoint{1.173459in}{0.757772in}}%
\pgfpathcurveto{\pgfqpoint{1.181273in}{0.749958in}}{\pgfqpoint{1.191872in}{0.745568in}}{\pgfqpoint{1.202922in}{0.745568in}}%
\pgfpathclose%
\pgfusepath{stroke,fill}%
\end{pgfscope}%
\begin{pgfscope}%
\pgfpathrectangle{\pgfqpoint{0.750000in}{0.500000in}}{\pgfqpoint{4.650000in}{3.020000in}}%
\pgfusepath{clip}%
\pgfsetbuttcap%
\pgfsetroundjoin%
\definecolor{currentfill}{rgb}{0.839216,0.152941,0.156863}%
\pgfsetfillcolor{currentfill}%
\pgfsetlinewidth{1.003750pt}%
\definecolor{currentstroke}{rgb}{0.839216,0.152941,0.156863}%
\pgfsetstrokecolor{currentstroke}%
\pgfsetdash{}{0pt}%
\pgfpathmoveto{\pgfqpoint{1.867208in}{2.914246in}}%
\pgfpathcurveto{\pgfqpoint{1.878258in}{2.914246in}}{\pgfqpoint{1.888857in}{2.918637in}}{\pgfqpoint{1.896671in}{2.926450in}}%
\pgfpathcurveto{\pgfqpoint{1.904484in}{2.934264in}}{\pgfqpoint{1.908874in}{2.944863in}}{\pgfqpoint{1.908874in}{2.955913in}}%
\pgfpathcurveto{\pgfqpoint{1.908874in}{2.966963in}}{\pgfqpoint{1.904484in}{2.977562in}}{\pgfqpoint{1.896671in}{2.985376in}}%
\pgfpathcurveto{\pgfqpoint{1.888857in}{2.993189in}}{\pgfqpoint{1.878258in}{2.997580in}}{\pgfqpoint{1.867208in}{2.997580in}}%
\pgfpathcurveto{\pgfqpoint{1.856158in}{2.997580in}}{\pgfqpoint{1.845559in}{2.993189in}}{\pgfqpoint{1.837745in}{2.985376in}}%
\pgfpathcurveto{\pgfqpoint{1.829931in}{2.977562in}}{\pgfqpoint{1.825541in}{2.966963in}}{\pgfqpoint{1.825541in}{2.955913in}}%
\pgfpathcurveto{\pgfqpoint{1.825541in}{2.944863in}}{\pgfqpoint{1.829931in}{2.934264in}}{\pgfqpoint{1.837745in}{2.926450in}}%
\pgfpathcurveto{\pgfqpoint{1.845559in}{2.918637in}}{\pgfqpoint{1.856158in}{2.914246in}}{\pgfqpoint{1.867208in}{2.914246in}}%
\pgfpathclose%
\pgfusepath{stroke,fill}%
\end{pgfscope}%
\begin{pgfscope}%
\pgfpathrectangle{\pgfqpoint{0.750000in}{0.500000in}}{\pgfqpoint{4.650000in}{3.020000in}}%
\pgfusepath{clip}%
\pgfsetbuttcap%
\pgfsetroundjoin%
\definecolor{currentfill}{rgb}{0.121569,0.466667,0.705882}%
\pgfsetfillcolor{currentfill}%
\pgfsetlinewidth{1.003750pt}%
\definecolor{currentstroke}{rgb}{0.121569,0.466667,0.705882}%
\pgfsetstrokecolor{currentstroke}%
\pgfsetdash{}{0pt}%
\pgfpathmoveto{\pgfqpoint{1.263312in}{0.595606in}}%
\pgfpathcurveto{\pgfqpoint{1.274362in}{0.595606in}}{\pgfqpoint{1.284961in}{0.599996in}}{\pgfqpoint{1.292774in}{0.607810in}}%
\pgfpathcurveto{\pgfqpoint{1.300588in}{0.615624in}}{\pgfqpoint{1.304978in}{0.626223in}}{\pgfqpoint{1.304978in}{0.637273in}}%
\pgfpathcurveto{\pgfqpoint{1.304978in}{0.648323in}}{\pgfqpoint{1.300588in}{0.658922in}}{\pgfqpoint{1.292774in}{0.666736in}}%
\pgfpathcurveto{\pgfqpoint{1.284961in}{0.674549in}}{\pgfqpoint{1.274362in}{0.678939in}}{\pgfqpoint{1.263312in}{0.678939in}}%
\pgfpathcurveto{\pgfqpoint{1.252262in}{0.678939in}}{\pgfqpoint{1.241663in}{0.674549in}}{\pgfqpoint{1.233849in}{0.666736in}}%
\pgfpathcurveto{\pgfqpoint{1.226035in}{0.658922in}}{\pgfqpoint{1.221645in}{0.648323in}}{\pgfqpoint{1.221645in}{0.637273in}}%
\pgfpathcurveto{\pgfqpoint{1.221645in}{0.626223in}}{\pgfqpoint{1.226035in}{0.615624in}}{\pgfqpoint{1.233849in}{0.607810in}}%
\pgfpathcurveto{\pgfqpoint{1.241663in}{0.599996in}}{\pgfqpoint{1.252262in}{0.595606in}}{\pgfqpoint{1.263312in}{0.595606in}}%
\pgfpathclose%
\pgfusepath{stroke,fill}%
\end{pgfscope}%
\begin{pgfscope}%
\pgfpathrectangle{\pgfqpoint{0.750000in}{0.500000in}}{\pgfqpoint{4.650000in}{3.020000in}}%
\pgfusepath{clip}%
\pgfsetbuttcap%
\pgfsetroundjoin%
\definecolor{currentfill}{rgb}{1.000000,0.498039,0.054902}%
\pgfsetfillcolor{currentfill}%
\pgfsetlinewidth{1.003750pt}%
\definecolor{currentstroke}{rgb}{1.000000,0.498039,0.054902}%
\pgfsetstrokecolor{currentstroke}%
\pgfsetdash{}{0pt}%
\pgfpathmoveto{\pgfqpoint{1.625649in}{2.902711in}}%
\pgfpathcurveto{\pgfqpoint{1.636699in}{2.902711in}}{\pgfqpoint{1.647299in}{2.907101in}}{\pgfqpoint{1.655112in}{2.914915in}}%
\pgfpathcurveto{\pgfqpoint{1.662926in}{2.922728in}}{\pgfqpoint{1.667316in}{2.933327in}}{\pgfqpoint{1.667316in}{2.944377in}}%
\pgfpathcurveto{\pgfqpoint{1.667316in}{2.955428in}}{\pgfqpoint{1.662926in}{2.966027in}}{\pgfqpoint{1.655112in}{2.973840in}}%
\pgfpathcurveto{\pgfqpoint{1.647299in}{2.981654in}}{\pgfqpoint{1.636699in}{2.986044in}}{\pgfqpoint{1.625649in}{2.986044in}}%
\pgfpathcurveto{\pgfqpoint{1.614599in}{2.986044in}}{\pgfqpoint{1.604000in}{2.981654in}}{\pgfqpoint{1.596187in}{2.973840in}}%
\pgfpathcurveto{\pgfqpoint{1.588373in}{2.966027in}}{\pgfqpoint{1.583983in}{2.955428in}}{\pgfqpoint{1.583983in}{2.944377in}}%
\pgfpathcurveto{\pgfqpoint{1.583983in}{2.933327in}}{\pgfqpoint{1.588373in}{2.922728in}}{\pgfqpoint{1.596187in}{2.914915in}}%
\pgfpathcurveto{\pgfqpoint{1.604000in}{2.907101in}}{\pgfqpoint{1.614599in}{2.902711in}}{\pgfqpoint{1.625649in}{2.902711in}}%
\pgfpathclose%
\pgfusepath{stroke,fill}%
\end{pgfscope}%
\begin{pgfscope}%
\pgfpathrectangle{\pgfqpoint{0.750000in}{0.500000in}}{\pgfqpoint{4.650000in}{3.020000in}}%
\pgfusepath{clip}%
\pgfsetbuttcap%
\pgfsetroundjoin%
\definecolor{currentfill}{rgb}{0.121569,0.466667,0.705882}%
\pgfsetfillcolor{currentfill}%
\pgfsetlinewidth{1.003750pt}%
\definecolor{currentstroke}{rgb}{0.121569,0.466667,0.705882}%
\pgfsetstrokecolor{currentstroke}%
\pgfsetdash{}{0pt}%
\pgfpathmoveto{\pgfqpoint{2.229545in}{1.118550in}}%
\pgfpathcurveto{\pgfqpoint{2.240596in}{1.118550in}}{\pgfqpoint{2.251195in}{1.122940in}}{\pgfqpoint{2.259008in}{1.130754in}}%
\pgfpathcurveto{\pgfqpoint{2.266822in}{1.138567in}}{\pgfqpoint{2.271212in}{1.149166in}}{\pgfqpoint{2.271212in}{1.160216in}}%
\pgfpathcurveto{\pgfqpoint{2.271212in}{1.171267in}}{\pgfqpoint{2.266822in}{1.181866in}}{\pgfqpoint{2.259008in}{1.189679in}}%
\pgfpathcurveto{\pgfqpoint{2.251195in}{1.197493in}}{\pgfqpoint{2.240596in}{1.201883in}}{\pgfqpoint{2.229545in}{1.201883in}}%
\pgfpathcurveto{\pgfqpoint{2.218495in}{1.201883in}}{\pgfqpoint{2.207896in}{1.197493in}}{\pgfqpoint{2.200083in}{1.189679in}}%
\pgfpathcurveto{\pgfqpoint{2.192269in}{1.181866in}}{\pgfqpoint{2.187879in}{1.171267in}}{\pgfqpoint{2.187879in}{1.160216in}}%
\pgfpathcurveto{\pgfqpoint{2.187879in}{1.149166in}}{\pgfqpoint{2.192269in}{1.138567in}}{\pgfqpoint{2.200083in}{1.130754in}}%
\pgfpathcurveto{\pgfqpoint{2.207896in}{1.122940in}}{\pgfqpoint{2.218495in}{1.118550in}}{\pgfqpoint{2.229545in}{1.118550in}}%
\pgfpathclose%
\pgfusepath{stroke,fill}%
\end{pgfscope}%
\begin{pgfscope}%
\pgfpathrectangle{\pgfqpoint{0.750000in}{0.500000in}}{\pgfqpoint{4.650000in}{3.020000in}}%
\pgfusepath{clip}%
\pgfsetbuttcap%
\pgfsetroundjoin%
\definecolor{currentfill}{rgb}{0.121569,0.466667,0.705882}%
\pgfsetfillcolor{currentfill}%
\pgfsetlinewidth{1.003750pt}%
\definecolor{currentstroke}{rgb}{0.121569,0.466667,0.705882}%
\pgfsetstrokecolor{currentstroke}%
\pgfsetdash{}{0pt}%
\pgfpathmoveto{\pgfqpoint{1.021753in}{0.872459in}}%
\pgfpathcurveto{\pgfqpoint{1.032803in}{0.872459in}}{\pgfqpoint{1.043402in}{0.876849in}}{\pgfqpoint{1.051216in}{0.884663in}}%
\pgfpathcurveto{\pgfqpoint{1.059030in}{0.892476in}}{\pgfqpoint{1.063420in}{0.903075in}}{\pgfqpoint{1.063420in}{0.914125in}}%
\pgfpathcurveto{\pgfqpoint{1.063420in}{0.925175in}}{\pgfqpoint{1.059030in}{0.935774in}}{\pgfqpoint{1.051216in}{0.943588in}}%
\pgfpathcurveto{\pgfqpoint{1.043402in}{0.951402in}}{\pgfqpoint{1.032803in}{0.955792in}}{\pgfqpoint{1.021753in}{0.955792in}}%
\pgfpathcurveto{\pgfqpoint{1.010703in}{0.955792in}}{\pgfqpoint{1.000104in}{0.951402in}}{\pgfqpoint{0.992290in}{0.943588in}}%
\pgfpathcurveto{\pgfqpoint{0.984477in}{0.935774in}}{\pgfqpoint{0.980087in}{0.925175in}}{\pgfqpoint{0.980087in}{0.914125in}}%
\pgfpathcurveto{\pgfqpoint{0.980087in}{0.903075in}}{\pgfqpoint{0.984477in}{0.892476in}}{\pgfqpoint{0.992290in}{0.884663in}}%
\pgfpathcurveto{\pgfqpoint{1.000104in}{0.876849in}}{\pgfqpoint{1.010703in}{0.872459in}}{\pgfqpoint{1.021753in}{0.872459in}}%
\pgfpathclose%
\pgfusepath{stroke,fill}%
\end{pgfscope}%
\begin{pgfscope}%
\pgfpathrectangle{\pgfqpoint{0.750000in}{0.500000in}}{\pgfqpoint{4.650000in}{3.020000in}}%
\pgfusepath{clip}%
\pgfsetbuttcap%
\pgfsetroundjoin%
\definecolor{currentfill}{rgb}{0.121569,0.466667,0.705882}%
\pgfsetfillcolor{currentfill}%
\pgfsetlinewidth{1.003750pt}%
\definecolor{currentstroke}{rgb}{0.121569,0.466667,0.705882}%
\pgfsetstrokecolor{currentstroke}%
\pgfsetdash{}{0pt}%
\pgfpathmoveto{\pgfqpoint{1.625649in}{0.595606in}}%
\pgfpathcurveto{\pgfqpoint{1.636699in}{0.595606in}}{\pgfqpoint{1.647299in}{0.599996in}}{\pgfqpoint{1.655112in}{0.607810in}}%
\pgfpathcurveto{\pgfqpoint{1.662926in}{0.615624in}}{\pgfqpoint{1.667316in}{0.626223in}}{\pgfqpoint{1.667316in}{0.637273in}}%
\pgfpathcurveto{\pgfqpoint{1.667316in}{0.648323in}}{\pgfqpoint{1.662926in}{0.658922in}}{\pgfqpoint{1.655112in}{0.666736in}}%
\pgfpathcurveto{\pgfqpoint{1.647299in}{0.674549in}}{\pgfqpoint{1.636699in}{0.678939in}}{\pgfqpoint{1.625649in}{0.678939in}}%
\pgfpathcurveto{\pgfqpoint{1.614599in}{0.678939in}}{\pgfqpoint{1.604000in}{0.674549in}}{\pgfqpoint{1.596187in}{0.666736in}}%
\pgfpathcurveto{\pgfqpoint{1.588373in}{0.658922in}}{\pgfqpoint{1.583983in}{0.648323in}}{\pgfqpoint{1.583983in}{0.637273in}}%
\pgfpathcurveto{\pgfqpoint{1.583983in}{0.626223in}}{\pgfqpoint{1.588373in}{0.615624in}}{\pgfqpoint{1.596187in}{0.607810in}}%
\pgfpathcurveto{\pgfqpoint{1.604000in}{0.599996in}}{\pgfqpoint{1.614599in}{0.595606in}}{\pgfqpoint{1.625649in}{0.595606in}}%
\pgfpathclose%
\pgfusepath{stroke,fill}%
\end{pgfscope}%
\begin{pgfscope}%
\pgfpathrectangle{\pgfqpoint{0.750000in}{0.500000in}}{\pgfqpoint{4.650000in}{3.020000in}}%
\pgfusepath{clip}%
\pgfsetbuttcap%
\pgfsetroundjoin%
\definecolor{currentfill}{rgb}{1.000000,0.498039,0.054902}%
\pgfsetfillcolor{currentfill}%
\pgfsetlinewidth{1.003750pt}%
\definecolor{currentstroke}{rgb}{1.000000,0.498039,0.054902}%
\pgfsetstrokecolor{currentstroke}%
\pgfsetdash{}{0pt}%
\pgfpathmoveto{\pgfqpoint{1.625649in}{2.906556in}}%
\pgfpathcurveto{\pgfqpoint{1.636699in}{2.906556in}}{\pgfqpoint{1.647299in}{2.910946in}}{\pgfqpoint{1.655112in}{2.918760in}}%
\pgfpathcurveto{\pgfqpoint{1.662926in}{2.926573in}}{\pgfqpoint{1.667316in}{2.937172in}}{\pgfqpoint{1.667316in}{2.948223in}}%
\pgfpathcurveto{\pgfqpoint{1.667316in}{2.959273in}}{\pgfqpoint{1.662926in}{2.969872in}}{\pgfqpoint{1.655112in}{2.977685in}}%
\pgfpathcurveto{\pgfqpoint{1.647299in}{2.985499in}}{\pgfqpoint{1.636699in}{2.989889in}}{\pgfqpoint{1.625649in}{2.989889in}}%
\pgfpathcurveto{\pgfqpoint{1.614599in}{2.989889in}}{\pgfqpoint{1.604000in}{2.985499in}}{\pgfqpoint{1.596187in}{2.977685in}}%
\pgfpathcurveto{\pgfqpoint{1.588373in}{2.969872in}}{\pgfqpoint{1.583983in}{2.959273in}}{\pgfqpoint{1.583983in}{2.948223in}}%
\pgfpathcurveto{\pgfqpoint{1.583983in}{2.937172in}}{\pgfqpoint{1.588373in}{2.926573in}}{\pgfqpoint{1.596187in}{2.918760in}}%
\pgfpathcurveto{\pgfqpoint{1.604000in}{2.910946in}}{\pgfqpoint{1.614599in}{2.906556in}}{\pgfqpoint{1.625649in}{2.906556in}}%
\pgfpathclose%
\pgfusepath{stroke,fill}%
\end{pgfscope}%
\begin{pgfscope}%
\pgfpathrectangle{\pgfqpoint{0.750000in}{0.500000in}}{\pgfqpoint{4.650000in}{3.020000in}}%
\pgfusepath{clip}%
\pgfsetbuttcap%
\pgfsetroundjoin%
\definecolor{currentfill}{rgb}{0.121569,0.466667,0.705882}%
\pgfsetfillcolor{currentfill}%
\pgfsetlinewidth{1.003750pt}%
\definecolor{currentstroke}{rgb}{0.121569,0.466667,0.705882}%
\pgfsetstrokecolor{currentstroke}%
\pgfsetdash{}{0pt}%
\pgfpathmoveto{\pgfqpoint{1.565260in}{0.595606in}}%
\pgfpathcurveto{\pgfqpoint{1.576310in}{0.595606in}}{\pgfqpoint{1.586909in}{0.599996in}}{\pgfqpoint{1.594723in}{0.607810in}}%
\pgfpathcurveto{\pgfqpoint{1.602536in}{0.615624in}}{\pgfqpoint{1.606926in}{0.626223in}}{\pgfqpoint{1.606926in}{0.637273in}}%
\pgfpathcurveto{\pgfqpoint{1.606926in}{0.648323in}}{\pgfqpoint{1.602536in}{0.658922in}}{\pgfqpoint{1.594723in}{0.666736in}}%
\pgfpathcurveto{\pgfqpoint{1.586909in}{0.674549in}}{\pgfqpoint{1.576310in}{0.678939in}}{\pgfqpoint{1.565260in}{0.678939in}}%
\pgfpathcurveto{\pgfqpoint{1.554210in}{0.678939in}}{\pgfqpoint{1.543611in}{0.674549in}}{\pgfqpoint{1.535797in}{0.666736in}}%
\pgfpathcurveto{\pgfqpoint{1.527983in}{0.658922in}}{\pgfqpoint{1.523593in}{0.648323in}}{\pgfqpoint{1.523593in}{0.637273in}}%
\pgfpathcurveto{\pgfqpoint{1.523593in}{0.626223in}}{\pgfqpoint{1.527983in}{0.615624in}}{\pgfqpoint{1.535797in}{0.607810in}}%
\pgfpathcurveto{\pgfqpoint{1.543611in}{0.599996in}}{\pgfqpoint{1.554210in}{0.595606in}}{\pgfqpoint{1.565260in}{0.595606in}}%
\pgfpathclose%
\pgfusepath{stroke,fill}%
\end{pgfscope}%
\begin{pgfscope}%
\pgfpathrectangle{\pgfqpoint{0.750000in}{0.500000in}}{\pgfqpoint{4.650000in}{3.020000in}}%
\pgfusepath{clip}%
\pgfsetbuttcap%
\pgfsetroundjoin%
\definecolor{currentfill}{rgb}{1.000000,0.498039,0.054902}%
\pgfsetfillcolor{currentfill}%
\pgfsetlinewidth{1.003750pt}%
\definecolor{currentstroke}{rgb}{1.000000,0.498039,0.054902}%
\pgfsetstrokecolor{currentstroke}%
\pgfsetdash{}{0pt}%
\pgfpathmoveto{\pgfqpoint{1.806818in}{2.987305in}}%
\pgfpathcurveto{\pgfqpoint{1.817868in}{2.987305in}}{\pgfqpoint{1.828467in}{2.991695in}}{\pgfqpoint{1.836281in}{2.999508in}}%
\pgfpathcurveto{\pgfqpoint{1.844095in}{3.007322in}}{\pgfqpoint{1.848485in}{3.017921in}}{\pgfqpoint{1.848485in}{3.028971in}}%
\pgfpathcurveto{\pgfqpoint{1.848485in}{3.040021in}}{\pgfqpoint{1.844095in}{3.050620in}}{\pgfqpoint{1.836281in}{3.058434in}}%
\pgfpathcurveto{\pgfqpoint{1.828467in}{3.066248in}}{\pgfqpoint{1.817868in}{3.070638in}}{\pgfqpoint{1.806818in}{3.070638in}}%
\pgfpathcurveto{\pgfqpoint{1.795768in}{3.070638in}}{\pgfqpoint{1.785169in}{3.066248in}}{\pgfqpoint{1.777355in}{3.058434in}}%
\pgfpathcurveto{\pgfqpoint{1.769542in}{3.050620in}}{\pgfqpoint{1.765152in}{3.040021in}}{\pgfqpoint{1.765152in}{3.028971in}}%
\pgfpathcurveto{\pgfqpoint{1.765152in}{3.017921in}}{\pgfqpoint{1.769542in}{3.007322in}}{\pgfqpoint{1.777355in}{2.999508in}}%
\pgfpathcurveto{\pgfqpoint{1.785169in}{2.991695in}}{\pgfqpoint{1.795768in}{2.987305in}}{\pgfqpoint{1.806818in}{2.987305in}}%
\pgfpathclose%
\pgfusepath{stroke,fill}%
\end{pgfscope}%
\begin{pgfscope}%
\pgfpathrectangle{\pgfqpoint{0.750000in}{0.500000in}}{\pgfqpoint{4.650000in}{3.020000in}}%
\pgfusepath{clip}%
\pgfsetbuttcap%
\pgfsetroundjoin%
\definecolor{currentfill}{rgb}{1.000000,0.498039,0.054902}%
\pgfsetfillcolor{currentfill}%
\pgfsetlinewidth{1.003750pt}%
\definecolor{currentstroke}{rgb}{1.000000,0.498039,0.054902}%
\pgfsetstrokecolor{currentstroke}%
\pgfsetdash{}{0pt}%
\pgfpathmoveto{\pgfqpoint{1.565260in}{2.918091in}}%
\pgfpathcurveto{\pgfqpoint{1.576310in}{2.918091in}}{\pgfqpoint{1.586909in}{2.922482in}}{\pgfqpoint{1.594723in}{2.930295in}}%
\pgfpathcurveto{\pgfqpoint{1.602536in}{2.938109in}}{\pgfqpoint{1.606926in}{2.948708in}}{\pgfqpoint{1.606926in}{2.959758in}}%
\pgfpathcurveto{\pgfqpoint{1.606926in}{2.970808in}}{\pgfqpoint{1.602536in}{2.981407in}}{\pgfqpoint{1.594723in}{2.989221in}}%
\pgfpathcurveto{\pgfqpoint{1.586909in}{2.997034in}}{\pgfqpoint{1.576310in}{3.001425in}}{\pgfqpoint{1.565260in}{3.001425in}}%
\pgfpathcurveto{\pgfqpoint{1.554210in}{3.001425in}}{\pgfqpoint{1.543611in}{2.997034in}}{\pgfqpoint{1.535797in}{2.989221in}}%
\pgfpathcurveto{\pgfqpoint{1.527983in}{2.981407in}}{\pgfqpoint{1.523593in}{2.970808in}}{\pgfqpoint{1.523593in}{2.959758in}}%
\pgfpathcurveto{\pgfqpoint{1.523593in}{2.948708in}}{\pgfqpoint{1.527983in}{2.938109in}}{\pgfqpoint{1.535797in}{2.930295in}}%
\pgfpathcurveto{\pgfqpoint{1.543611in}{2.922482in}}{\pgfqpoint{1.554210in}{2.918091in}}{\pgfqpoint{1.565260in}{2.918091in}}%
\pgfpathclose%
\pgfusepath{stroke,fill}%
\end{pgfscope}%
\begin{pgfscope}%
\pgfpathrectangle{\pgfqpoint{0.750000in}{0.500000in}}{\pgfqpoint{4.650000in}{3.020000in}}%
\pgfusepath{clip}%
\pgfsetbuttcap%
\pgfsetroundjoin%
\definecolor{currentfill}{rgb}{0.121569,0.466667,0.705882}%
\pgfsetfillcolor{currentfill}%
\pgfsetlinewidth{1.003750pt}%
\definecolor{currentstroke}{rgb}{0.121569,0.466667,0.705882}%
\pgfsetstrokecolor{currentstroke}%
\pgfsetdash{}{0pt}%
\pgfpathmoveto{\pgfqpoint{1.504870in}{0.595606in}}%
\pgfpathcurveto{\pgfqpoint{1.515920in}{0.595606in}}{\pgfqpoint{1.526519in}{0.599996in}}{\pgfqpoint{1.534333in}{0.607810in}}%
\pgfpathcurveto{\pgfqpoint{1.542147in}{0.615624in}}{\pgfqpoint{1.546537in}{0.626223in}}{\pgfqpoint{1.546537in}{0.637273in}}%
\pgfpathcurveto{\pgfqpoint{1.546537in}{0.648323in}}{\pgfqpoint{1.542147in}{0.658922in}}{\pgfqpoint{1.534333in}{0.666736in}}%
\pgfpathcurveto{\pgfqpoint{1.526519in}{0.674549in}}{\pgfqpoint{1.515920in}{0.678939in}}{\pgfqpoint{1.504870in}{0.678939in}}%
\pgfpathcurveto{\pgfqpoint{1.493820in}{0.678939in}}{\pgfqpoint{1.483221in}{0.674549in}}{\pgfqpoint{1.475407in}{0.666736in}}%
\pgfpathcurveto{\pgfqpoint{1.467594in}{0.658922in}}{\pgfqpoint{1.463203in}{0.648323in}}{\pgfqpoint{1.463203in}{0.637273in}}%
\pgfpathcurveto{\pgfqpoint{1.463203in}{0.626223in}}{\pgfqpoint{1.467594in}{0.615624in}}{\pgfqpoint{1.475407in}{0.607810in}}%
\pgfpathcurveto{\pgfqpoint{1.483221in}{0.599996in}}{\pgfqpoint{1.493820in}{0.595606in}}{\pgfqpoint{1.504870in}{0.595606in}}%
\pgfpathclose%
\pgfusepath{stroke,fill}%
\end{pgfscope}%
\begin{pgfscope}%
\pgfpathrectangle{\pgfqpoint{0.750000in}{0.500000in}}{\pgfqpoint{4.650000in}{3.020000in}}%
\pgfusepath{clip}%
\pgfsetbuttcap%
\pgfsetroundjoin%
\definecolor{currentfill}{rgb}{1.000000,0.498039,0.054902}%
\pgfsetfillcolor{currentfill}%
\pgfsetlinewidth{1.003750pt}%
\definecolor{currentstroke}{rgb}{1.000000,0.498039,0.054902}%
\pgfsetstrokecolor{currentstroke}%
\pgfsetdash{}{0pt}%
\pgfpathmoveto{\pgfqpoint{1.384091in}{2.956543in}}%
\pgfpathcurveto{\pgfqpoint{1.395141in}{2.956543in}}{\pgfqpoint{1.405740in}{2.960933in}}{\pgfqpoint{1.413554in}{2.968747in}}%
\pgfpathcurveto{\pgfqpoint{1.421367in}{2.976561in}}{\pgfqpoint{1.425758in}{2.987160in}}{\pgfqpoint{1.425758in}{2.998210in}}%
\pgfpathcurveto{\pgfqpoint{1.425758in}{3.009260in}}{\pgfqpoint{1.421367in}{3.019859in}}{\pgfqpoint{1.413554in}{3.027673in}}%
\pgfpathcurveto{\pgfqpoint{1.405740in}{3.035486in}}{\pgfqpoint{1.395141in}{3.039876in}}{\pgfqpoint{1.384091in}{3.039876in}}%
\pgfpathcurveto{\pgfqpoint{1.373041in}{3.039876in}}{\pgfqpoint{1.362442in}{3.035486in}}{\pgfqpoint{1.354628in}{3.027673in}}%
\pgfpathcurveto{\pgfqpoint{1.346815in}{3.019859in}}{\pgfqpoint{1.342424in}{3.009260in}}{\pgfqpoint{1.342424in}{2.998210in}}%
\pgfpathcurveto{\pgfqpoint{1.342424in}{2.987160in}}{\pgfqpoint{1.346815in}{2.976561in}}{\pgfqpoint{1.354628in}{2.968747in}}%
\pgfpathcurveto{\pgfqpoint{1.362442in}{2.960933in}}{\pgfqpoint{1.373041in}{2.956543in}}{\pgfqpoint{1.384091in}{2.956543in}}%
\pgfpathclose%
\pgfusepath{stroke,fill}%
\end{pgfscope}%
\begin{pgfscope}%
\pgfpathrectangle{\pgfqpoint{0.750000in}{0.500000in}}{\pgfqpoint{4.650000in}{3.020000in}}%
\pgfusepath{clip}%
\pgfsetbuttcap%
\pgfsetroundjoin%
\definecolor{currentfill}{rgb}{0.121569,0.466667,0.705882}%
\pgfsetfillcolor{currentfill}%
\pgfsetlinewidth{1.003750pt}%
\definecolor{currentstroke}{rgb}{0.121569,0.466667,0.705882}%
\pgfsetstrokecolor{currentstroke}%
\pgfsetdash{}{0pt}%
\pgfpathmoveto{\pgfqpoint{1.927597in}{2.118295in}}%
\pgfpathcurveto{\pgfqpoint{1.938648in}{2.118295in}}{\pgfqpoint{1.949247in}{2.122685in}}{\pgfqpoint{1.957060in}{2.130499in}}%
\pgfpathcurveto{\pgfqpoint{1.964874in}{2.138313in}}{\pgfqpoint{1.969264in}{2.148912in}}{\pgfqpoint{1.969264in}{2.159962in}}%
\pgfpathcurveto{\pgfqpoint{1.969264in}{2.171012in}}{\pgfqpoint{1.964874in}{2.181611in}}{\pgfqpoint{1.957060in}{2.189425in}}%
\pgfpathcurveto{\pgfqpoint{1.949247in}{2.197238in}}{\pgfqpoint{1.938648in}{2.201628in}}{\pgfqpoint{1.927597in}{2.201628in}}%
\pgfpathcurveto{\pgfqpoint{1.916547in}{2.201628in}}{\pgfqpoint{1.905948in}{2.197238in}}{\pgfqpoint{1.898135in}{2.189425in}}%
\pgfpathcurveto{\pgfqpoint{1.890321in}{2.181611in}}{\pgfqpoint{1.885931in}{2.171012in}}{\pgfqpoint{1.885931in}{2.159962in}}%
\pgfpathcurveto{\pgfqpoint{1.885931in}{2.148912in}}{\pgfqpoint{1.890321in}{2.138313in}}{\pgfqpoint{1.898135in}{2.130499in}}%
\pgfpathcurveto{\pgfqpoint{1.905948in}{2.122685in}}{\pgfqpoint{1.916547in}{2.118295in}}{\pgfqpoint{1.927597in}{2.118295in}}%
\pgfpathclose%
\pgfusepath{stroke,fill}%
\end{pgfscope}%
\begin{pgfscope}%
\pgfpathrectangle{\pgfqpoint{0.750000in}{0.500000in}}{\pgfqpoint{4.650000in}{3.020000in}}%
\pgfusepath{clip}%
\pgfsetbuttcap%
\pgfsetroundjoin%
\definecolor{currentfill}{rgb}{0.121569,0.466667,0.705882}%
\pgfsetfillcolor{currentfill}%
\pgfsetlinewidth{1.003750pt}%
\definecolor{currentstroke}{rgb}{0.121569,0.466667,0.705882}%
\pgfsetstrokecolor{currentstroke}%
\pgfsetdash{}{0pt}%
\pgfpathmoveto{\pgfqpoint{3.014610in}{0.968588in}}%
\pgfpathcurveto{\pgfqpoint{3.025661in}{0.968588in}}{\pgfqpoint{3.036260in}{0.972978in}}{\pgfqpoint{3.044073in}{0.980792in}}%
\pgfpathcurveto{\pgfqpoint{3.051887in}{0.988605in}}{\pgfqpoint{3.056277in}{0.999205in}}{\pgfqpoint{3.056277in}{1.010255in}}%
\pgfpathcurveto{\pgfqpoint{3.056277in}{1.021305in}}{\pgfqpoint{3.051887in}{1.031904in}}{\pgfqpoint{3.044073in}{1.039717in}}%
\pgfpathcurveto{\pgfqpoint{3.036260in}{1.047531in}}{\pgfqpoint{3.025661in}{1.051921in}}{\pgfqpoint{3.014610in}{1.051921in}}%
\pgfpathcurveto{\pgfqpoint{3.003560in}{1.051921in}}{\pgfqpoint{2.992961in}{1.047531in}}{\pgfqpoint{2.985148in}{1.039717in}}%
\pgfpathcurveto{\pgfqpoint{2.977334in}{1.031904in}}{\pgfqpoint{2.972944in}{1.021305in}}{\pgfqpoint{2.972944in}{1.010255in}}%
\pgfpathcurveto{\pgfqpoint{2.972944in}{0.999205in}}{\pgfqpoint{2.977334in}{0.988605in}}{\pgfqpoint{2.985148in}{0.980792in}}%
\pgfpathcurveto{\pgfqpoint{2.992961in}{0.972978in}}{\pgfqpoint{3.003560in}{0.968588in}}{\pgfqpoint{3.014610in}{0.968588in}}%
\pgfpathclose%
\pgfusepath{stroke,fill}%
\end{pgfscope}%
\begin{pgfscope}%
\pgfpathrectangle{\pgfqpoint{0.750000in}{0.500000in}}{\pgfqpoint{4.650000in}{3.020000in}}%
\pgfusepath{clip}%
\pgfsetbuttcap%
\pgfsetroundjoin%
\definecolor{currentfill}{rgb}{1.000000,0.498039,0.054902}%
\pgfsetfillcolor{currentfill}%
\pgfsetlinewidth{1.003750pt}%
\definecolor{currentstroke}{rgb}{1.000000,0.498039,0.054902}%
\pgfsetstrokecolor{currentstroke}%
\pgfsetdash{}{0pt}%
\pgfpathmoveto{\pgfqpoint{1.384091in}{2.902711in}}%
\pgfpathcurveto{\pgfqpoint{1.395141in}{2.902711in}}{\pgfqpoint{1.405740in}{2.907101in}}{\pgfqpoint{1.413554in}{2.914915in}}%
\pgfpathcurveto{\pgfqpoint{1.421367in}{2.922728in}}{\pgfqpoint{1.425758in}{2.933327in}}{\pgfqpoint{1.425758in}{2.944377in}}%
\pgfpathcurveto{\pgfqpoint{1.425758in}{2.955428in}}{\pgfqpoint{1.421367in}{2.966027in}}{\pgfqpoint{1.413554in}{2.973840in}}%
\pgfpathcurveto{\pgfqpoint{1.405740in}{2.981654in}}{\pgfqpoint{1.395141in}{2.986044in}}{\pgfqpoint{1.384091in}{2.986044in}}%
\pgfpathcurveto{\pgfqpoint{1.373041in}{2.986044in}}{\pgfqpoint{1.362442in}{2.981654in}}{\pgfqpoint{1.354628in}{2.973840in}}%
\pgfpathcurveto{\pgfqpoint{1.346815in}{2.966027in}}{\pgfqpoint{1.342424in}{2.955428in}}{\pgfqpoint{1.342424in}{2.944377in}}%
\pgfpathcurveto{\pgfqpoint{1.342424in}{2.933327in}}{\pgfqpoint{1.346815in}{2.922728in}}{\pgfqpoint{1.354628in}{2.914915in}}%
\pgfpathcurveto{\pgfqpoint{1.362442in}{2.907101in}}{\pgfqpoint{1.373041in}{2.902711in}}{\pgfqpoint{1.384091in}{2.902711in}}%
\pgfpathclose%
\pgfusepath{stroke,fill}%
\end{pgfscope}%
\begin{pgfscope}%
\pgfpathrectangle{\pgfqpoint{0.750000in}{0.500000in}}{\pgfqpoint{4.650000in}{3.020000in}}%
\pgfusepath{clip}%
\pgfsetbuttcap%
\pgfsetroundjoin%
\definecolor{currentfill}{rgb}{0.121569,0.466667,0.705882}%
\pgfsetfillcolor{currentfill}%
\pgfsetlinewidth{1.003750pt}%
\definecolor{currentstroke}{rgb}{0.121569,0.466667,0.705882}%
\pgfsetstrokecolor{currentstroke}%
\pgfsetdash{}{0pt}%
\pgfpathmoveto{\pgfqpoint{1.323701in}{0.595606in}}%
\pgfpathcurveto{\pgfqpoint{1.334751in}{0.595606in}}{\pgfqpoint{1.345350in}{0.599996in}}{\pgfqpoint{1.353164in}{0.607810in}}%
\pgfpathcurveto{\pgfqpoint{1.360978in}{0.615624in}}{\pgfqpoint{1.365368in}{0.626223in}}{\pgfqpoint{1.365368in}{0.637273in}}%
\pgfpathcurveto{\pgfqpoint{1.365368in}{0.648323in}}{\pgfqpoint{1.360978in}{0.658922in}}{\pgfqpoint{1.353164in}{0.666736in}}%
\pgfpathcurveto{\pgfqpoint{1.345350in}{0.674549in}}{\pgfqpoint{1.334751in}{0.678939in}}{\pgfqpoint{1.323701in}{0.678939in}}%
\pgfpathcurveto{\pgfqpoint{1.312651in}{0.678939in}}{\pgfqpoint{1.302052in}{0.674549in}}{\pgfqpoint{1.294239in}{0.666736in}}%
\pgfpathcurveto{\pgfqpoint{1.286425in}{0.658922in}}{\pgfqpoint{1.282035in}{0.648323in}}{\pgfqpoint{1.282035in}{0.637273in}}%
\pgfpathcurveto{\pgfqpoint{1.282035in}{0.626223in}}{\pgfqpoint{1.286425in}{0.615624in}}{\pgfqpoint{1.294239in}{0.607810in}}%
\pgfpathcurveto{\pgfqpoint{1.302052in}{0.599996in}}{\pgfqpoint{1.312651in}{0.595606in}}{\pgfqpoint{1.323701in}{0.595606in}}%
\pgfpathclose%
\pgfusepath{stroke,fill}%
\end{pgfscope}%
\begin{pgfscope}%
\pgfpathrectangle{\pgfqpoint{0.750000in}{0.500000in}}{\pgfqpoint{4.650000in}{3.020000in}}%
\pgfusepath{clip}%
\pgfsetbuttcap%
\pgfsetroundjoin%
\definecolor{currentfill}{rgb}{1.000000,0.498039,0.054902}%
\pgfsetfillcolor{currentfill}%
\pgfsetlinewidth{1.003750pt}%
\definecolor{currentstroke}{rgb}{1.000000,0.498039,0.054902}%
\pgfsetstrokecolor{currentstroke}%
\pgfsetdash{}{0pt}%
\pgfpathmoveto{\pgfqpoint{1.625649in}{2.921937in}}%
\pgfpathcurveto{\pgfqpoint{1.636699in}{2.921937in}}{\pgfqpoint{1.647299in}{2.926327in}}{\pgfqpoint{1.655112in}{2.934140in}}%
\pgfpathcurveto{\pgfqpoint{1.662926in}{2.941954in}}{\pgfqpoint{1.667316in}{2.952553in}}{\pgfqpoint{1.667316in}{2.963603in}}%
\pgfpathcurveto{\pgfqpoint{1.667316in}{2.974653in}}{\pgfqpoint{1.662926in}{2.985252in}}{\pgfqpoint{1.655112in}{2.993066in}}%
\pgfpathcurveto{\pgfqpoint{1.647299in}{3.000880in}}{\pgfqpoint{1.636699in}{3.005270in}}{\pgfqpoint{1.625649in}{3.005270in}}%
\pgfpathcurveto{\pgfqpoint{1.614599in}{3.005270in}}{\pgfqpoint{1.604000in}{3.000880in}}{\pgfqpoint{1.596187in}{2.993066in}}%
\pgfpathcurveto{\pgfqpoint{1.588373in}{2.985252in}}{\pgfqpoint{1.583983in}{2.974653in}}{\pgfqpoint{1.583983in}{2.963603in}}%
\pgfpathcurveto{\pgfqpoint{1.583983in}{2.952553in}}{\pgfqpoint{1.588373in}{2.941954in}}{\pgfqpoint{1.596187in}{2.934140in}}%
\pgfpathcurveto{\pgfqpoint{1.604000in}{2.926327in}}{\pgfqpoint{1.614599in}{2.921937in}}{\pgfqpoint{1.625649in}{2.921937in}}%
\pgfpathclose%
\pgfusepath{stroke,fill}%
\end{pgfscope}%
\begin{pgfscope}%
\pgfpathrectangle{\pgfqpoint{0.750000in}{0.500000in}}{\pgfqpoint{4.650000in}{3.020000in}}%
\pgfusepath{clip}%
\pgfsetbuttcap%
\pgfsetroundjoin%
\definecolor{currentfill}{rgb}{1.000000,0.498039,0.054902}%
\pgfsetfillcolor{currentfill}%
\pgfsetlinewidth{1.003750pt}%
\definecolor{currentstroke}{rgb}{1.000000,0.498039,0.054902}%
\pgfsetstrokecolor{currentstroke}%
\pgfsetdash{}{0pt}%
\pgfpathmoveto{\pgfqpoint{1.263312in}{2.902711in}}%
\pgfpathcurveto{\pgfqpoint{1.274362in}{2.902711in}}{\pgfqpoint{1.284961in}{2.907101in}}{\pgfqpoint{1.292774in}{2.914915in}}%
\pgfpathcurveto{\pgfqpoint{1.300588in}{2.922728in}}{\pgfqpoint{1.304978in}{2.933327in}}{\pgfqpoint{1.304978in}{2.944377in}}%
\pgfpathcurveto{\pgfqpoint{1.304978in}{2.955428in}}{\pgfqpoint{1.300588in}{2.966027in}}{\pgfqpoint{1.292774in}{2.973840in}}%
\pgfpathcurveto{\pgfqpoint{1.284961in}{2.981654in}}{\pgfqpoint{1.274362in}{2.986044in}}{\pgfqpoint{1.263312in}{2.986044in}}%
\pgfpathcurveto{\pgfqpoint{1.252262in}{2.986044in}}{\pgfqpoint{1.241663in}{2.981654in}}{\pgfqpoint{1.233849in}{2.973840in}}%
\pgfpathcurveto{\pgfqpoint{1.226035in}{2.966027in}}{\pgfqpoint{1.221645in}{2.955428in}}{\pgfqpoint{1.221645in}{2.944377in}}%
\pgfpathcurveto{\pgfqpoint{1.221645in}{2.933327in}}{\pgfqpoint{1.226035in}{2.922728in}}{\pgfqpoint{1.233849in}{2.914915in}}%
\pgfpathcurveto{\pgfqpoint{1.241663in}{2.907101in}}{\pgfqpoint{1.252262in}{2.902711in}}{\pgfqpoint{1.263312in}{2.902711in}}%
\pgfpathclose%
\pgfusepath{stroke,fill}%
\end{pgfscope}%
\begin{pgfscope}%
\pgfpathrectangle{\pgfqpoint{0.750000in}{0.500000in}}{\pgfqpoint{4.650000in}{3.020000in}}%
\pgfusepath{clip}%
\pgfsetbuttcap%
\pgfsetroundjoin%
\definecolor{currentfill}{rgb}{1.000000,0.498039,0.054902}%
\pgfsetfillcolor{currentfill}%
\pgfsetlinewidth{1.003750pt}%
\definecolor{currentstroke}{rgb}{1.000000,0.498039,0.054902}%
\pgfsetstrokecolor{currentstroke}%
\pgfsetdash{}{0pt}%
\pgfpathmoveto{\pgfqpoint{1.384091in}{2.906556in}}%
\pgfpathcurveto{\pgfqpoint{1.395141in}{2.906556in}}{\pgfqpoint{1.405740in}{2.910946in}}{\pgfqpoint{1.413554in}{2.918760in}}%
\pgfpathcurveto{\pgfqpoint{1.421367in}{2.926573in}}{\pgfqpoint{1.425758in}{2.937172in}}{\pgfqpoint{1.425758in}{2.948223in}}%
\pgfpathcurveto{\pgfqpoint{1.425758in}{2.959273in}}{\pgfqpoint{1.421367in}{2.969872in}}{\pgfqpoint{1.413554in}{2.977685in}}%
\pgfpathcurveto{\pgfqpoint{1.405740in}{2.985499in}}{\pgfqpoint{1.395141in}{2.989889in}}{\pgfqpoint{1.384091in}{2.989889in}}%
\pgfpathcurveto{\pgfqpoint{1.373041in}{2.989889in}}{\pgfqpoint{1.362442in}{2.985499in}}{\pgfqpoint{1.354628in}{2.977685in}}%
\pgfpathcurveto{\pgfqpoint{1.346815in}{2.969872in}}{\pgfqpoint{1.342424in}{2.959273in}}{\pgfqpoint{1.342424in}{2.948223in}}%
\pgfpathcurveto{\pgfqpoint{1.342424in}{2.937172in}}{\pgfqpoint{1.346815in}{2.926573in}}{\pgfqpoint{1.354628in}{2.918760in}}%
\pgfpathcurveto{\pgfqpoint{1.362442in}{2.910946in}}{\pgfqpoint{1.373041in}{2.906556in}}{\pgfqpoint{1.384091in}{2.906556in}}%
\pgfpathclose%
\pgfusepath{stroke,fill}%
\end{pgfscope}%
\begin{pgfscope}%
\pgfpathrectangle{\pgfqpoint{0.750000in}{0.500000in}}{\pgfqpoint{4.650000in}{3.020000in}}%
\pgfusepath{clip}%
\pgfsetbuttcap%
\pgfsetroundjoin%
\definecolor{currentfill}{rgb}{0.839216,0.152941,0.156863}%
\pgfsetfillcolor{currentfill}%
\pgfsetlinewidth{1.003750pt}%
\definecolor{currentstroke}{rgb}{0.839216,0.152941,0.156863}%
\pgfsetstrokecolor{currentstroke}%
\pgfsetdash{}{0pt}%
\pgfpathmoveto{\pgfqpoint{1.867208in}{2.918091in}}%
\pgfpathcurveto{\pgfqpoint{1.878258in}{2.918091in}}{\pgfqpoint{1.888857in}{2.922482in}}{\pgfqpoint{1.896671in}{2.930295in}}%
\pgfpathcurveto{\pgfqpoint{1.904484in}{2.938109in}}{\pgfqpoint{1.908874in}{2.948708in}}{\pgfqpoint{1.908874in}{2.959758in}}%
\pgfpathcurveto{\pgfqpoint{1.908874in}{2.970808in}}{\pgfqpoint{1.904484in}{2.981407in}}{\pgfqpoint{1.896671in}{2.989221in}}%
\pgfpathcurveto{\pgfqpoint{1.888857in}{2.997034in}}{\pgfqpoint{1.878258in}{3.001425in}}{\pgfqpoint{1.867208in}{3.001425in}}%
\pgfpathcurveto{\pgfqpoint{1.856158in}{3.001425in}}{\pgfqpoint{1.845559in}{2.997034in}}{\pgfqpoint{1.837745in}{2.989221in}}%
\pgfpathcurveto{\pgfqpoint{1.829931in}{2.981407in}}{\pgfqpoint{1.825541in}{2.970808in}}{\pgfqpoint{1.825541in}{2.959758in}}%
\pgfpathcurveto{\pgfqpoint{1.825541in}{2.948708in}}{\pgfqpoint{1.829931in}{2.938109in}}{\pgfqpoint{1.837745in}{2.930295in}}%
\pgfpathcurveto{\pgfqpoint{1.845559in}{2.922482in}}{\pgfqpoint{1.856158in}{2.918091in}}{\pgfqpoint{1.867208in}{2.918091in}}%
\pgfpathclose%
\pgfusepath{stroke,fill}%
\end{pgfscope}%
\begin{pgfscope}%
\pgfpathrectangle{\pgfqpoint{0.750000in}{0.500000in}}{\pgfqpoint{4.650000in}{3.020000in}}%
\pgfusepath{clip}%
\pgfsetbuttcap%
\pgfsetroundjoin%
\definecolor{currentfill}{rgb}{1.000000,0.498039,0.054902}%
\pgfsetfillcolor{currentfill}%
\pgfsetlinewidth{1.003750pt}%
\definecolor{currentstroke}{rgb}{1.000000,0.498039,0.054902}%
\pgfsetstrokecolor{currentstroke}%
\pgfsetdash{}{0pt}%
\pgfpathmoveto{\pgfqpoint{1.444481in}{2.929627in}}%
\pgfpathcurveto{\pgfqpoint{1.455531in}{2.929627in}}{\pgfqpoint{1.466130in}{2.934017in}}{\pgfqpoint{1.473943in}{2.941831in}}%
\pgfpathcurveto{\pgfqpoint{1.481757in}{2.949644in}}{\pgfqpoint{1.486147in}{2.960243in}}{\pgfqpoint{1.486147in}{2.971294in}}%
\pgfpathcurveto{\pgfqpoint{1.486147in}{2.982344in}}{\pgfqpoint{1.481757in}{2.992943in}}{\pgfqpoint{1.473943in}{3.000756in}}%
\pgfpathcurveto{\pgfqpoint{1.466130in}{3.008570in}}{\pgfqpoint{1.455531in}{3.012960in}}{\pgfqpoint{1.444481in}{3.012960in}}%
\pgfpathcurveto{\pgfqpoint{1.433430in}{3.012960in}}{\pgfqpoint{1.422831in}{3.008570in}}{\pgfqpoint{1.415018in}{3.000756in}}%
\pgfpathcurveto{\pgfqpoint{1.407204in}{2.992943in}}{\pgfqpoint{1.402814in}{2.982344in}}{\pgfqpoint{1.402814in}{2.971294in}}%
\pgfpathcurveto{\pgfqpoint{1.402814in}{2.960243in}}{\pgfqpoint{1.407204in}{2.949644in}}{\pgfqpoint{1.415018in}{2.941831in}}%
\pgfpathcurveto{\pgfqpoint{1.422831in}{2.934017in}}{\pgfqpoint{1.433430in}{2.929627in}}{\pgfqpoint{1.444481in}{2.929627in}}%
\pgfpathclose%
\pgfusepath{stroke,fill}%
\end{pgfscope}%
\begin{pgfscope}%
\pgfpathrectangle{\pgfqpoint{0.750000in}{0.500000in}}{\pgfqpoint{4.650000in}{3.020000in}}%
\pgfusepath{clip}%
\pgfsetbuttcap%
\pgfsetroundjoin%
\definecolor{currentfill}{rgb}{1.000000,0.498039,0.054902}%
\pgfsetfillcolor{currentfill}%
\pgfsetlinewidth{1.003750pt}%
\definecolor{currentstroke}{rgb}{1.000000,0.498039,0.054902}%
\pgfsetstrokecolor{currentstroke}%
\pgfsetdash{}{0pt}%
\pgfpathmoveto{\pgfqpoint{2.169156in}{2.898866in}}%
\pgfpathcurveto{\pgfqpoint{2.180206in}{2.898866in}}{\pgfqpoint{2.190805in}{2.903256in}}{\pgfqpoint{2.198619in}{2.911069in}}%
\pgfpathcurveto{\pgfqpoint{2.206432in}{2.918883in}}{\pgfqpoint{2.210823in}{2.929482in}}{\pgfqpoint{2.210823in}{2.940532in}}%
\pgfpathcurveto{\pgfqpoint{2.210823in}{2.951582in}}{\pgfqpoint{2.206432in}{2.962181in}}{\pgfqpoint{2.198619in}{2.969995in}}%
\pgfpathcurveto{\pgfqpoint{2.190805in}{2.977809in}}{\pgfqpoint{2.180206in}{2.982199in}}{\pgfqpoint{2.169156in}{2.982199in}}%
\pgfpathcurveto{\pgfqpoint{2.158106in}{2.982199in}}{\pgfqpoint{2.147507in}{2.977809in}}{\pgfqpoint{2.139693in}{2.969995in}}%
\pgfpathcurveto{\pgfqpoint{2.131879in}{2.962181in}}{\pgfqpoint{2.127489in}{2.951582in}}{\pgfqpoint{2.127489in}{2.940532in}}%
\pgfpathcurveto{\pgfqpoint{2.127489in}{2.929482in}}{\pgfqpoint{2.131879in}{2.918883in}}{\pgfqpoint{2.139693in}{2.911069in}}%
\pgfpathcurveto{\pgfqpoint{2.147507in}{2.903256in}}{\pgfqpoint{2.158106in}{2.898866in}}{\pgfqpoint{2.169156in}{2.898866in}}%
\pgfpathclose%
\pgfusepath{stroke,fill}%
\end{pgfscope}%
\begin{pgfscope}%
\pgfpathrectangle{\pgfqpoint{0.750000in}{0.500000in}}{\pgfqpoint{4.650000in}{3.020000in}}%
\pgfusepath{clip}%
\pgfsetbuttcap%
\pgfsetroundjoin%
\definecolor{currentfill}{rgb}{1.000000,0.498039,0.054902}%
\pgfsetfillcolor{currentfill}%
\pgfsetlinewidth{1.003750pt}%
\definecolor{currentstroke}{rgb}{1.000000,0.498039,0.054902}%
\pgfsetstrokecolor{currentstroke}%
\pgfsetdash{}{0pt}%
\pgfpathmoveto{\pgfqpoint{2.048377in}{2.910401in}}%
\pgfpathcurveto{\pgfqpoint{2.059427in}{2.910401in}}{\pgfqpoint{2.070026in}{2.914791in}}{\pgfqpoint{2.077839in}{2.922605in}}%
\pgfpathcurveto{\pgfqpoint{2.085653in}{2.930419in}}{\pgfqpoint{2.090043in}{2.941018in}}{\pgfqpoint{2.090043in}{2.952068in}}%
\pgfpathcurveto{\pgfqpoint{2.090043in}{2.963118in}}{\pgfqpoint{2.085653in}{2.973717in}}{\pgfqpoint{2.077839in}{2.981531in}}%
\pgfpathcurveto{\pgfqpoint{2.070026in}{2.989344in}}{\pgfqpoint{2.059427in}{2.993734in}}{\pgfqpoint{2.048377in}{2.993734in}}%
\pgfpathcurveto{\pgfqpoint{2.037326in}{2.993734in}}{\pgfqpoint{2.026727in}{2.989344in}}{\pgfqpoint{2.018914in}{2.981531in}}%
\pgfpathcurveto{\pgfqpoint{2.011100in}{2.973717in}}{\pgfqpoint{2.006710in}{2.963118in}}{\pgfqpoint{2.006710in}{2.952068in}}%
\pgfpathcurveto{\pgfqpoint{2.006710in}{2.941018in}}{\pgfqpoint{2.011100in}{2.930419in}}{\pgfqpoint{2.018914in}{2.922605in}}%
\pgfpathcurveto{\pgfqpoint{2.026727in}{2.914791in}}{\pgfqpoint{2.037326in}{2.910401in}}{\pgfqpoint{2.048377in}{2.910401in}}%
\pgfpathclose%
\pgfusepath{stroke,fill}%
\end{pgfscope}%
\begin{pgfscope}%
\pgfpathrectangle{\pgfqpoint{0.750000in}{0.500000in}}{\pgfqpoint{4.650000in}{3.020000in}}%
\pgfusepath{clip}%
\pgfsetbuttcap%
\pgfsetroundjoin%
\definecolor{currentfill}{rgb}{1.000000,0.498039,0.054902}%
\pgfsetfillcolor{currentfill}%
\pgfsetlinewidth{1.003750pt}%
\definecolor{currentstroke}{rgb}{1.000000,0.498039,0.054902}%
\pgfsetstrokecolor{currentstroke}%
\pgfsetdash{}{0pt}%
\pgfpathmoveto{\pgfqpoint{1.987987in}{2.914246in}}%
\pgfpathcurveto{\pgfqpoint{1.999037in}{2.914246in}}{\pgfqpoint{2.009636in}{2.918637in}}{\pgfqpoint{2.017450in}{2.926450in}}%
\pgfpathcurveto{\pgfqpoint{2.025263in}{2.934264in}}{\pgfqpoint{2.029654in}{2.944863in}}{\pgfqpoint{2.029654in}{2.955913in}}%
\pgfpathcurveto{\pgfqpoint{2.029654in}{2.966963in}}{\pgfqpoint{2.025263in}{2.977562in}}{\pgfqpoint{2.017450in}{2.985376in}}%
\pgfpathcurveto{\pgfqpoint{2.009636in}{2.993189in}}{\pgfqpoint{1.999037in}{2.997580in}}{\pgfqpoint{1.987987in}{2.997580in}}%
\pgfpathcurveto{\pgfqpoint{1.976937in}{2.997580in}}{\pgfqpoint{1.966338in}{2.993189in}}{\pgfqpoint{1.958524in}{2.985376in}}%
\pgfpathcurveto{\pgfqpoint{1.950711in}{2.977562in}}{\pgfqpoint{1.946320in}{2.966963in}}{\pgfqpoint{1.946320in}{2.955913in}}%
\pgfpathcurveto{\pgfqpoint{1.946320in}{2.944863in}}{\pgfqpoint{1.950711in}{2.934264in}}{\pgfqpoint{1.958524in}{2.926450in}}%
\pgfpathcurveto{\pgfqpoint{1.966338in}{2.918637in}}{\pgfqpoint{1.976937in}{2.914246in}}{\pgfqpoint{1.987987in}{2.914246in}}%
\pgfpathclose%
\pgfusepath{stroke,fill}%
\end{pgfscope}%
\begin{pgfscope}%
\pgfpathrectangle{\pgfqpoint{0.750000in}{0.500000in}}{\pgfqpoint{4.650000in}{3.020000in}}%
\pgfusepath{clip}%
\pgfsetbuttcap%
\pgfsetroundjoin%
\definecolor{currentfill}{rgb}{0.121569,0.466667,0.705882}%
\pgfsetfillcolor{currentfill}%
\pgfsetlinewidth{1.003750pt}%
\definecolor{currentstroke}{rgb}{0.121569,0.466667,0.705882}%
\pgfsetstrokecolor{currentstroke}%
\pgfsetdash{}{0pt}%
\pgfpathmoveto{\pgfqpoint{1.806818in}{0.603296in}}%
\pgfpathcurveto{\pgfqpoint{1.817868in}{0.603296in}}{\pgfqpoint{1.828467in}{0.607687in}}{\pgfqpoint{1.836281in}{0.615500in}}%
\pgfpathcurveto{\pgfqpoint{1.844095in}{0.623314in}}{\pgfqpoint{1.848485in}{0.633913in}}{\pgfqpoint{1.848485in}{0.644963in}}%
\pgfpathcurveto{\pgfqpoint{1.848485in}{0.656013in}}{\pgfqpoint{1.844095in}{0.666612in}}{\pgfqpoint{1.836281in}{0.674426in}}%
\pgfpathcurveto{\pgfqpoint{1.828467in}{0.682239in}}{\pgfqpoint{1.817868in}{0.686630in}}{\pgfqpoint{1.806818in}{0.686630in}}%
\pgfpathcurveto{\pgfqpoint{1.795768in}{0.686630in}}{\pgfqpoint{1.785169in}{0.682239in}}{\pgfqpoint{1.777355in}{0.674426in}}%
\pgfpathcurveto{\pgfqpoint{1.769542in}{0.666612in}}{\pgfqpoint{1.765152in}{0.656013in}}{\pgfqpoint{1.765152in}{0.644963in}}%
\pgfpathcurveto{\pgfqpoint{1.765152in}{0.633913in}}{\pgfqpoint{1.769542in}{0.623314in}}{\pgfqpoint{1.777355in}{0.615500in}}%
\pgfpathcurveto{\pgfqpoint{1.785169in}{0.607687in}}{\pgfqpoint{1.795768in}{0.603296in}}{\pgfqpoint{1.806818in}{0.603296in}}%
\pgfpathclose%
\pgfusepath{stroke,fill}%
\end{pgfscope}%
\begin{pgfscope}%
\pgfpathrectangle{\pgfqpoint{0.750000in}{0.500000in}}{\pgfqpoint{4.650000in}{3.020000in}}%
\pgfusepath{clip}%
\pgfsetbuttcap%
\pgfsetroundjoin%
\definecolor{currentfill}{rgb}{1.000000,0.498039,0.054902}%
\pgfsetfillcolor{currentfill}%
\pgfsetlinewidth{1.003750pt}%
\definecolor{currentstroke}{rgb}{1.000000,0.498039,0.054902}%
\pgfsetstrokecolor{currentstroke}%
\pgfsetdash{}{0pt}%
\pgfpathmoveto{\pgfqpoint{1.867208in}{2.906556in}}%
\pgfpathcurveto{\pgfqpoint{1.878258in}{2.906556in}}{\pgfqpoint{1.888857in}{2.910946in}}{\pgfqpoint{1.896671in}{2.918760in}}%
\pgfpathcurveto{\pgfqpoint{1.904484in}{2.926573in}}{\pgfqpoint{1.908874in}{2.937172in}}{\pgfqpoint{1.908874in}{2.948223in}}%
\pgfpathcurveto{\pgfqpoint{1.908874in}{2.959273in}}{\pgfqpoint{1.904484in}{2.969872in}}{\pgfqpoint{1.896671in}{2.977685in}}%
\pgfpathcurveto{\pgfqpoint{1.888857in}{2.985499in}}{\pgfqpoint{1.878258in}{2.989889in}}{\pgfqpoint{1.867208in}{2.989889in}}%
\pgfpathcurveto{\pgfqpoint{1.856158in}{2.989889in}}{\pgfqpoint{1.845559in}{2.985499in}}{\pgfqpoint{1.837745in}{2.977685in}}%
\pgfpathcurveto{\pgfqpoint{1.829931in}{2.969872in}}{\pgfqpoint{1.825541in}{2.959273in}}{\pgfqpoint{1.825541in}{2.948223in}}%
\pgfpathcurveto{\pgfqpoint{1.825541in}{2.937172in}}{\pgfqpoint{1.829931in}{2.926573in}}{\pgfqpoint{1.837745in}{2.918760in}}%
\pgfpathcurveto{\pgfqpoint{1.845559in}{2.910946in}}{\pgfqpoint{1.856158in}{2.906556in}}{\pgfqpoint{1.867208in}{2.906556in}}%
\pgfpathclose%
\pgfusepath{stroke,fill}%
\end{pgfscope}%
\begin{pgfscope}%
\pgfpathrectangle{\pgfqpoint{0.750000in}{0.500000in}}{\pgfqpoint{4.650000in}{3.020000in}}%
\pgfusepath{clip}%
\pgfsetbuttcap%
\pgfsetroundjoin%
\definecolor{currentfill}{rgb}{1.000000,0.498039,0.054902}%
\pgfsetfillcolor{currentfill}%
\pgfsetlinewidth{1.003750pt}%
\definecolor{currentstroke}{rgb}{1.000000,0.498039,0.054902}%
\pgfsetstrokecolor{currentstroke}%
\pgfsetdash{}{0pt}%
\pgfpathmoveto{\pgfqpoint{1.867208in}{2.898866in}}%
\pgfpathcurveto{\pgfqpoint{1.878258in}{2.898866in}}{\pgfqpoint{1.888857in}{2.903256in}}{\pgfqpoint{1.896671in}{2.911069in}}%
\pgfpathcurveto{\pgfqpoint{1.904484in}{2.918883in}}{\pgfqpoint{1.908874in}{2.929482in}}{\pgfqpoint{1.908874in}{2.940532in}}%
\pgfpathcurveto{\pgfqpoint{1.908874in}{2.951582in}}{\pgfqpoint{1.904484in}{2.962181in}}{\pgfqpoint{1.896671in}{2.969995in}}%
\pgfpathcurveto{\pgfqpoint{1.888857in}{2.977809in}}{\pgfqpoint{1.878258in}{2.982199in}}{\pgfqpoint{1.867208in}{2.982199in}}%
\pgfpathcurveto{\pgfqpoint{1.856158in}{2.982199in}}{\pgfqpoint{1.845559in}{2.977809in}}{\pgfqpoint{1.837745in}{2.969995in}}%
\pgfpathcurveto{\pgfqpoint{1.829931in}{2.962181in}}{\pgfqpoint{1.825541in}{2.951582in}}{\pgfqpoint{1.825541in}{2.940532in}}%
\pgfpathcurveto{\pgfqpoint{1.825541in}{2.929482in}}{\pgfqpoint{1.829931in}{2.918883in}}{\pgfqpoint{1.837745in}{2.911069in}}%
\pgfpathcurveto{\pgfqpoint{1.845559in}{2.903256in}}{\pgfqpoint{1.856158in}{2.898866in}}{\pgfqpoint{1.867208in}{2.898866in}}%
\pgfpathclose%
\pgfusepath{stroke,fill}%
\end{pgfscope}%
\begin{pgfscope}%
\pgfpathrectangle{\pgfqpoint{0.750000in}{0.500000in}}{\pgfqpoint{4.650000in}{3.020000in}}%
\pgfusepath{clip}%
\pgfsetbuttcap%
\pgfsetroundjoin%
\definecolor{currentfill}{rgb}{1.000000,0.498039,0.054902}%
\pgfsetfillcolor{currentfill}%
\pgfsetlinewidth{1.003750pt}%
\definecolor{currentstroke}{rgb}{1.000000,0.498039,0.054902}%
\pgfsetstrokecolor{currentstroke}%
\pgfsetdash{}{0pt}%
\pgfpathmoveto{\pgfqpoint{1.867208in}{2.918091in}}%
\pgfpathcurveto{\pgfqpoint{1.878258in}{2.918091in}}{\pgfqpoint{1.888857in}{2.922482in}}{\pgfqpoint{1.896671in}{2.930295in}}%
\pgfpathcurveto{\pgfqpoint{1.904484in}{2.938109in}}{\pgfqpoint{1.908874in}{2.948708in}}{\pgfqpoint{1.908874in}{2.959758in}}%
\pgfpathcurveto{\pgfqpoint{1.908874in}{2.970808in}}{\pgfqpoint{1.904484in}{2.981407in}}{\pgfqpoint{1.896671in}{2.989221in}}%
\pgfpathcurveto{\pgfqpoint{1.888857in}{2.997034in}}{\pgfqpoint{1.878258in}{3.001425in}}{\pgfqpoint{1.867208in}{3.001425in}}%
\pgfpathcurveto{\pgfqpoint{1.856158in}{3.001425in}}{\pgfqpoint{1.845559in}{2.997034in}}{\pgfqpoint{1.837745in}{2.989221in}}%
\pgfpathcurveto{\pgfqpoint{1.829931in}{2.981407in}}{\pgfqpoint{1.825541in}{2.970808in}}{\pgfqpoint{1.825541in}{2.959758in}}%
\pgfpathcurveto{\pgfqpoint{1.825541in}{2.948708in}}{\pgfqpoint{1.829931in}{2.938109in}}{\pgfqpoint{1.837745in}{2.930295in}}%
\pgfpathcurveto{\pgfqpoint{1.845559in}{2.922482in}}{\pgfqpoint{1.856158in}{2.918091in}}{\pgfqpoint{1.867208in}{2.918091in}}%
\pgfpathclose%
\pgfusepath{stroke,fill}%
\end{pgfscope}%
\begin{pgfscope}%
\pgfpathrectangle{\pgfqpoint{0.750000in}{0.500000in}}{\pgfqpoint{4.650000in}{3.020000in}}%
\pgfusepath{clip}%
\pgfsetbuttcap%
\pgfsetroundjoin%
\definecolor{currentfill}{rgb}{1.000000,0.498039,0.054902}%
\pgfsetfillcolor{currentfill}%
\pgfsetlinewidth{1.003750pt}%
\definecolor{currentstroke}{rgb}{1.000000,0.498039,0.054902}%
\pgfsetstrokecolor{currentstroke}%
\pgfsetdash{}{0pt}%
\pgfpathmoveto{\pgfqpoint{2.773052in}{2.906556in}}%
\pgfpathcurveto{\pgfqpoint{2.784102in}{2.906556in}}{\pgfqpoint{2.794701in}{2.910946in}}{\pgfqpoint{2.802515in}{2.918760in}}%
\pgfpathcurveto{\pgfqpoint{2.810328in}{2.926573in}}{\pgfqpoint{2.814719in}{2.937172in}}{\pgfqpoint{2.814719in}{2.948223in}}%
\pgfpathcurveto{\pgfqpoint{2.814719in}{2.959273in}}{\pgfqpoint{2.810328in}{2.969872in}}{\pgfqpoint{2.802515in}{2.977685in}}%
\pgfpathcurveto{\pgfqpoint{2.794701in}{2.985499in}}{\pgfqpoint{2.784102in}{2.989889in}}{\pgfqpoint{2.773052in}{2.989889in}}%
\pgfpathcurveto{\pgfqpoint{2.762002in}{2.989889in}}{\pgfqpoint{2.751403in}{2.985499in}}{\pgfqpoint{2.743589in}{2.977685in}}%
\pgfpathcurveto{\pgfqpoint{2.735776in}{2.969872in}}{\pgfqpoint{2.731385in}{2.959273in}}{\pgfqpoint{2.731385in}{2.948223in}}%
\pgfpathcurveto{\pgfqpoint{2.731385in}{2.937172in}}{\pgfqpoint{2.735776in}{2.926573in}}{\pgfqpoint{2.743589in}{2.918760in}}%
\pgfpathcurveto{\pgfqpoint{2.751403in}{2.910946in}}{\pgfqpoint{2.762002in}{2.906556in}}{\pgfqpoint{2.773052in}{2.906556in}}%
\pgfpathclose%
\pgfusepath{stroke,fill}%
\end{pgfscope}%
\begin{pgfscope}%
\pgfpathrectangle{\pgfqpoint{0.750000in}{0.500000in}}{\pgfqpoint{4.650000in}{3.020000in}}%
\pgfusepath{clip}%
\pgfsetbuttcap%
\pgfsetroundjoin%
\definecolor{currentfill}{rgb}{1.000000,0.498039,0.054902}%
\pgfsetfillcolor{currentfill}%
\pgfsetlinewidth{1.003750pt}%
\definecolor{currentstroke}{rgb}{1.000000,0.498039,0.054902}%
\pgfsetstrokecolor{currentstroke}%
\pgfsetdash{}{0pt}%
\pgfpathmoveto{\pgfqpoint{3.980844in}{2.898866in}}%
\pgfpathcurveto{\pgfqpoint{3.991894in}{2.898866in}}{\pgfqpoint{4.002493in}{2.903256in}}{\pgfqpoint{4.010307in}{2.911069in}}%
\pgfpathcurveto{\pgfqpoint{4.018121in}{2.918883in}}{\pgfqpoint{4.022511in}{2.929482in}}{\pgfqpoint{4.022511in}{2.940532in}}%
\pgfpathcurveto{\pgfqpoint{4.022511in}{2.951582in}}{\pgfqpoint{4.018121in}{2.962181in}}{\pgfqpoint{4.010307in}{2.969995in}}%
\pgfpathcurveto{\pgfqpoint{4.002493in}{2.977809in}}{\pgfqpoint{3.991894in}{2.982199in}}{\pgfqpoint{3.980844in}{2.982199in}}%
\pgfpathcurveto{\pgfqpoint{3.969794in}{2.982199in}}{\pgfqpoint{3.959195in}{2.977809in}}{\pgfqpoint{3.951381in}{2.969995in}}%
\pgfpathcurveto{\pgfqpoint{3.943568in}{2.962181in}}{\pgfqpoint{3.939177in}{2.951582in}}{\pgfqpoint{3.939177in}{2.940532in}}%
\pgfpathcurveto{\pgfqpoint{3.939177in}{2.929482in}}{\pgfqpoint{3.943568in}{2.918883in}}{\pgfqpoint{3.951381in}{2.911069in}}%
\pgfpathcurveto{\pgfqpoint{3.959195in}{2.903256in}}{\pgfqpoint{3.969794in}{2.898866in}}{\pgfqpoint{3.980844in}{2.898866in}}%
\pgfpathclose%
\pgfusepath{stroke,fill}%
\end{pgfscope}%
\begin{pgfscope}%
\pgfpathrectangle{\pgfqpoint{0.750000in}{0.500000in}}{\pgfqpoint{4.650000in}{3.020000in}}%
\pgfusepath{clip}%
\pgfsetbuttcap%
\pgfsetroundjoin%
\definecolor{currentfill}{rgb}{1.000000,0.498039,0.054902}%
\pgfsetfillcolor{currentfill}%
\pgfsetlinewidth{1.003750pt}%
\definecolor{currentstroke}{rgb}{1.000000,0.498039,0.054902}%
\pgfsetstrokecolor{currentstroke}%
\pgfsetdash{}{0pt}%
\pgfpathmoveto{\pgfqpoint{1.867208in}{3.044982in}}%
\pgfpathcurveto{\pgfqpoint{1.878258in}{3.044982in}}{\pgfqpoint{1.888857in}{3.049372in}}{\pgfqpoint{1.896671in}{3.057186in}}%
\pgfpathcurveto{\pgfqpoint{1.904484in}{3.065000in}}{\pgfqpoint{1.908874in}{3.075599in}}{\pgfqpoint{1.908874in}{3.086649in}}%
\pgfpathcurveto{\pgfqpoint{1.908874in}{3.097699in}}{\pgfqpoint{1.904484in}{3.108298in}}{\pgfqpoint{1.896671in}{3.116112in}}%
\pgfpathcurveto{\pgfqpoint{1.888857in}{3.123925in}}{\pgfqpoint{1.878258in}{3.128316in}}{\pgfqpoint{1.867208in}{3.128316in}}%
\pgfpathcurveto{\pgfqpoint{1.856158in}{3.128316in}}{\pgfqpoint{1.845559in}{3.123925in}}{\pgfqpoint{1.837745in}{3.116112in}}%
\pgfpathcurveto{\pgfqpoint{1.829931in}{3.108298in}}{\pgfqpoint{1.825541in}{3.097699in}}{\pgfqpoint{1.825541in}{3.086649in}}%
\pgfpathcurveto{\pgfqpoint{1.825541in}{3.075599in}}{\pgfqpoint{1.829931in}{3.065000in}}{\pgfqpoint{1.837745in}{3.057186in}}%
\pgfpathcurveto{\pgfqpoint{1.845559in}{3.049372in}}{\pgfqpoint{1.856158in}{3.044982in}}{\pgfqpoint{1.867208in}{3.044982in}}%
\pgfpathclose%
\pgfusepath{stroke,fill}%
\end{pgfscope}%
\begin{pgfscope}%
\pgfpathrectangle{\pgfqpoint{0.750000in}{0.500000in}}{\pgfqpoint{4.650000in}{3.020000in}}%
\pgfusepath{clip}%
\pgfsetbuttcap%
\pgfsetroundjoin%
\definecolor{currentfill}{rgb}{1.000000,0.498039,0.054902}%
\pgfsetfillcolor{currentfill}%
\pgfsetlinewidth{1.003750pt}%
\definecolor{currentstroke}{rgb}{1.000000,0.498039,0.054902}%
\pgfsetstrokecolor{currentstroke}%
\pgfsetdash{}{0pt}%
\pgfpathmoveto{\pgfqpoint{1.746429in}{2.906556in}}%
\pgfpathcurveto{\pgfqpoint{1.757479in}{2.906556in}}{\pgfqpoint{1.768078in}{2.910946in}}{\pgfqpoint{1.775891in}{2.918760in}}%
\pgfpathcurveto{\pgfqpoint{1.783705in}{2.926573in}}{\pgfqpoint{1.788095in}{2.937172in}}{\pgfqpoint{1.788095in}{2.948223in}}%
\pgfpathcurveto{\pgfqpoint{1.788095in}{2.959273in}}{\pgfqpoint{1.783705in}{2.969872in}}{\pgfqpoint{1.775891in}{2.977685in}}%
\pgfpathcurveto{\pgfqpoint{1.768078in}{2.985499in}}{\pgfqpoint{1.757479in}{2.989889in}}{\pgfqpoint{1.746429in}{2.989889in}}%
\pgfpathcurveto{\pgfqpoint{1.735378in}{2.989889in}}{\pgfqpoint{1.724779in}{2.985499in}}{\pgfqpoint{1.716966in}{2.977685in}}%
\pgfpathcurveto{\pgfqpoint{1.709152in}{2.969872in}}{\pgfqpoint{1.704762in}{2.959273in}}{\pgfqpoint{1.704762in}{2.948223in}}%
\pgfpathcurveto{\pgfqpoint{1.704762in}{2.937172in}}{\pgfqpoint{1.709152in}{2.926573in}}{\pgfqpoint{1.716966in}{2.918760in}}%
\pgfpathcurveto{\pgfqpoint{1.724779in}{2.910946in}}{\pgfqpoint{1.735378in}{2.906556in}}{\pgfqpoint{1.746429in}{2.906556in}}%
\pgfpathclose%
\pgfusepath{stroke,fill}%
\end{pgfscope}%
\begin{pgfscope}%
\pgfpathrectangle{\pgfqpoint{0.750000in}{0.500000in}}{\pgfqpoint{4.650000in}{3.020000in}}%
\pgfusepath{clip}%
\pgfsetbuttcap%
\pgfsetroundjoin%
\definecolor{currentfill}{rgb}{0.839216,0.152941,0.156863}%
\pgfsetfillcolor{currentfill}%
\pgfsetlinewidth{1.003750pt}%
\definecolor{currentstroke}{rgb}{0.839216,0.152941,0.156863}%
\pgfsetstrokecolor{currentstroke}%
\pgfsetdash{}{0pt}%
\pgfpathmoveto{\pgfqpoint{1.686039in}{1.033956in}}%
\pgfpathcurveto{\pgfqpoint{1.697089in}{1.033956in}}{\pgfqpoint{1.707688in}{1.038346in}}{\pgfqpoint{1.715502in}{1.046160in}}%
\pgfpathcurveto{\pgfqpoint{1.723315in}{1.053973in}}{\pgfqpoint{1.727706in}{1.064572in}}{\pgfqpoint{1.727706in}{1.075623in}}%
\pgfpathcurveto{\pgfqpoint{1.727706in}{1.086673in}}{\pgfqpoint{1.723315in}{1.097272in}}{\pgfqpoint{1.715502in}{1.105085in}}%
\pgfpathcurveto{\pgfqpoint{1.707688in}{1.112899in}}{\pgfqpoint{1.697089in}{1.117289in}}{\pgfqpoint{1.686039in}{1.117289in}}%
\pgfpathcurveto{\pgfqpoint{1.674989in}{1.117289in}}{\pgfqpoint{1.664390in}{1.112899in}}{\pgfqpoint{1.656576in}{1.105085in}}%
\pgfpathcurveto{\pgfqpoint{1.648763in}{1.097272in}}{\pgfqpoint{1.644372in}{1.086673in}}{\pgfqpoint{1.644372in}{1.075623in}}%
\pgfpathcurveto{\pgfqpoint{1.644372in}{1.064572in}}{\pgfqpoint{1.648763in}{1.053973in}}{\pgfqpoint{1.656576in}{1.046160in}}%
\pgfpathcurveto{\pgfqpoint{1.664390in}{1.038346in}}{\pgfqpoint{1.674989in}{1.033956in}}{\pgfqpoint{1.686039in}{1.033956in}}%
\pgfpathclose%
\pgfusepath{stroke,fill}%
\end{pgfscope}%
\begin{pgfscope}%
\pgfpathrectangle{\pgfqpoint{0.750000in}{0.500000in}}{\pgfqpoint{4.650000in}{3.020000in}}%
\pgfusepath{clip}%
\pgfsetbuttcap%
\pgfsetroundjoin%
\definecolor{currentfill}{rgb}{1.000000,0.498039,0.054902}%
\pgfsetfillcolor{currentfill}%
\pgfsetlinewidth{1.003750pt}%
\definecolor{currentstroke}{rgb}{1.000000,0.498039,0.054902}%
\pgfsetstrokecolor{currentstroke}%
\pgfsetdash{}{0pt}%
\pgfpathmoveto{\pgfqpoint{1.323701in}{3.071898in}}%
\pgfpathcurveto{\pgfqpoint{1.334751in}{3.071898in}}{\pgfqpoint{1.345350in}{3.076289in}}{\pgfqpoint{1.353164in}{3.084102in}}%
\pgfpathcurveto{\pgfqpoint{1.360978in}{3.091916in}}{\pgfqpoint{1.365368in}{3.102515in}}{\pgfqpoint{1.365368in}{3.113565in}}%
\pgfpathcurveto{\pgfqpoint{1.365368in}{3.124615in}}{\pgfqpoint{1.360978in}{3.135214in}}{\pgfqpoint{1.353164in}{3.143028in}}%
\pgfpathcurveto{\pgfqpoint{1.345350in}{3.150841in}}{\pgfqpoint{1.334751in}{3.155232in}}{\pgfqpoint{1.323701in}{3.155232in}}%
\pgfpathcurveto{\pgfqpoint{1.312651in}{3.155232in}}{\pgfqpoint{1.302052in}{3.150841in}}{\pgfqpoint{1.294239in}{3.143028in}}%
\pgfpathcurveto{\pgfqpoint{1.286425in}{3.135214in}}{\pgfqpoint{1.282035in}{3.124615in}}{\pgfqpoint{1.282035in}{3.113565in}}%
\pgfpathcurveto{\pgfqpoint{1.282035in}{3.102515in}}{\pgfqpoint{1.286425in}{3.091916in}}{\pgfqpoint{1.294239in}{3.084102in}}%
\pgfpathcurveto{\pgfqpoint{1.302052in}{3.076289in}}{\pgfqpoint{1.312651in}{3.071898in}}{\pgfqpoint{1.323701in}{3.071898in}}%
\pgfpathclose%
\pgfusepath{stroke,fill}%
\end{pgfscope}%
\begin{pgfscope}%
\pgfpathrectangle{\pgfqpoint{0.750000in}{0.500000in}}{\pgfqpoint{4.650000in}{3.020000in}}%
\pgfusepath{clip}%
\pgfsetbuttcap%
\pgfsetroundjoin%
\definecolor{currentfill}{rgb}{1.000000,0.498039,0.054902}%
\pgfsetfillcolor{currentfill}%
\pgfsetlinewidth{1.003750pt}%
\definecolor{currentstroke}{rgb}{1.000000,0.498039,0.054902}%
\pgfsetstrokecolor{currentstroke}%
\pgfsetdash{}{0pt}%
\pgfpathmoveto{\pgfqpoint{1.806818in}{2.906556in}}%
\pgfpathcurveto{\pgfqpoint{1.817868in}{2.906556in}}{\pgfqpoint{1.828467in}{2.910946in}}{\pgfqpoint{1.836281in}{2.918760in}}%
\pgfpathcurveto{\pgfqpoint{1.844095in}{2.926573in}}{\pgfqpoint{1.848485in}{2.937172in}}{\pgfqpoint{1.848485in}{2.948223in}}%
\pgfpathcurveto{\pgfqpoint{1.848485in}{2.959273in}}{\pgfqpoint{1.844095in}{2.969872in}}{\pgfqpoint{1.836281in}{2.977685in}}%
\pgfpathcurveto{\pgfqpoint{1.828467in}{2.985499in}}{\pgfqpoint{1.817868in}{2.989889in}}{\pgfqpoint{1.806818in}{2.989889in}}%
\pgfpathcurveto{\pgfqpoint{1.795768in}{2.989889in}}{\pgfqpoint{1.785169in}{2.985499in}}{\pgfqpoint{1.777355in}{2.977685in}}%
\pgfpathcurveto{\pgfqpoint{1.769542in}{2.969872in}}{\pgfqpoint{1.765152in}{2.959273in}}{\pgfqpoint{1.765152in}{2.948223in}}%
\pgfpathcurveto{\pgfqpoint{1.765152in}{2.937172in}}{\pgfqpoint{1.769542in}{2.926573in}}{\pgfqpoint{1.777355in}{2.918760in}}%
\pgfpathcurveto{\pgfqpoint{1.785169in}{2.910946in}}{\pgfqpoint{1.795768in}{2.906556in}}{\pgfqpoint{1.806818in}{2.906556in}}%
\pgfpathclose%
\pgfusepath{stroke,fill}%
\end{pgfscope}%
\begin{pgfscope}%
\pgfpathrectangle{\pgfqpoint{0.750000in}{0.500000in}}{\pgfqpoint{4.650000in}{3.020000in}}%
\pgfusepath{clip}%
\pgfsetbuttcap%
\pgfsetroundjoin%
\definecolor{currentfill}{rgb}{1.000000,0.498039,0.054902}%
\pgfsetfillcolor{currentfill}%
\pgfsetlinewidth{1.003750pt}%
\definecolor{currentstroke}{rgb}{1.000000,0.498039,0.054902}%
\pgfsetstrokecolor{currentstroke}%
\pgfsetdash{}{0pt}%
\pgfpathmoveto{\pgfqpoint{1.021753in}{3.064208in}}%
\pgfpathcurveto{\pgfqpoint{1.032803in}{3.064208in}}{\pgfqpoint{1.043402in}{3.068598in}}{\pgfqpoint{1.051216in}{3.076412in}}%
\pgfpathcurveto{\pgfqpoint{1.059030in}{3.084226in}}{\pgfqpoint{1.063420in}{3.094825in}}{\pgfqpoint{1.063420in}{3.105875in}}%
\pgfpathcurveto{\pgfqpoint{1.063420in}{3.116925in}}{\pgfqpoint{1.059030in}{3.127524in}}{\pgfqpoint{1.051216in}{3.135337in}}%
\pgfpathcurveto{\pgfqpoint{1.043402in}{3.143151in}}{\pgfqpoint{1.032803in}{3.147541in}}{\pgfqpoint{1.021753in}{3.147541in}}%
\pgfpathcurveto{\pgfqpoint{1.010703in}{3.147541in}}{\pgfqpoint{1.000104in}{3.143151in}}{\pgfqpoint{0.992290in}{3.135337in}}%
\pgfpathcurveto{\pgfqpoint{0.984477in}{3.127524in}}{\pgfqpoint{0.980087in}{3.116925in}}{\pgfqpoint{0.980087in}{3.105875in}}%
\pgfpathcurveto{\pgfqpoint{0.980087in}{3.094825in}}{\pgfqpoint{0.984477in}{3.084226in}}{\pgfqpoint{0.992290in}{3.076412in}}%
\pgfpathcurveto{\pgfqpoint{1.000104in}{3.068598in}}{\pgfqpoint{1.010703in}{3.064208in}}{\pgfqpoint{1.021753in}{3.064208in}}%
\pgfpathclose%
\pgfusepath{stroke,fill}%
\end{pgfscope}%
\begin{pgfscope}%
\pgfpathrectangle{\pgfqpoint{0.750000in}{0.500000in}}{\pgfqpoint{4.650000in}{3.020000in}}%
\pgfusepath{clip}%
\pgfsetbuttcap%
\pgfsetroundjoin%
\definecolor{currentfill}{rgb}{1.000000,0.498039,0.054902}%
\pgfsetfillcolor{currentfill}%
\pgfsetlinewidth{1.003750pt}%
\definecolor{currentstroke}{rgb}{1.000000,0.498039,0.054902}%
\pgfsetstrokecolor{currentstroke}%
\pgfsetdash{}{0pt}%
\pgfpathmoveto{\pgfqpoint{1.686039in}{2.948853in}}%
\pgfpathcurveto{\pgfqpoint{1.697089in}{2.948853in}}{\pgfqpoint{1.707688in}{2.953243in}}{\pgfqpoint{1.715502in}{2.961057in}}%
\pgfpathcurveto{\pgfqpoint{1.723315in}{2.968870in}}{\pgfqpoint{1.727706in}{2.979469in}}{\pgfqpoint{1.727706in}{2.990519in}}%
\pgfpathcurveto{\pgfqpoint{1.727706in}{3.001570in}}{\pgfqpoint{1.723315in}{3.012169in}}{\pgfqpoint{1.715502in}{3.019982in}}%
\pgfpathcurveto{\pgfqpoint{1.707688in}{3.027796in}}{\pgfqpoint{1.697089in}{3.032186in}}{\pgfqpoint{1.686039in}{3.032186in}}%
\pgfpathcurveto{\pgfqpoint{1.674989in}{3.032186in}}{\pgfqpoint{1.664390in}{3.027796in}}{\pgfqpoint{1.656576in}{3.019982in}}%
\pgfpathcurveto{\pgfqpoint{1.648763in}{3.012169in}}{\pgfqpoint{1.644372in}{3.001570in}}{\pgfqpoint{1.644372in}{2.990519in}}%
\pgfpathcurveto{\pgfqpoint{1.644372in}{2.979469in}}{\pgfqpoint{1.648763in}{2.968870in}}{\pgfqpoint{1.656576in}{2.961057in}}%
\pgfpathcurveto{\pgfqpoint{1.664390in}{2.953243in}}{\pgfqpoint{1.674989in}{2.948853in}}{\pgfqpoint{1.686039in}{2.948853in}}%
\pgfpathclose%
\pgfusepath{stroke,fill}%
\end{pgfscope}%
\begin{pgfscope}%
\pgfpathrectangle{\pgfqpoint{0.750000in}{0.500000in}}{\pgfqpoint{4.650000in}{3.020000in}}%
\pgfusepath{clip}%
\pgfsetbuttcap%
\pgfsetroundjoin%
\definecolor{currentfill}{rgb}{1.000000,0.498039,0.054902}%
\pgfsetfillcolor{currentfill}%
\pgfsetlinewidth{1.003750pt}%
\definecolor{currentstroke}{rgb}{1.000000,0.498039,0.054902}%
\pgfsetstrokecolor{currentstroke}%
\pgfsetdash{}{0pt}%
\pgfpathmoveto{\pgfqpoint{1.504870in}{2.902711in}}%
\pgfpathcurveto{\pgfqpoint{1.515920in}{2.902711in}}{\pgfqpoint{1.526519in}{2.907101in}}{\pgfqpoint{1.534333in}{2.914915in}}%
\pgfpathcurveto{\pgfqpoint{1.542147in}{2.922728in}}{\pgfqpoint{1.546537in}{2.933327in}}{\pgfqpoint{1.546537in}{2.944377in}}%
\pgfpathcurveto{\pgfqpoint{1.546537in}{2.955428in}}{\pgfqpoint{1.542147in}{2.966027in}}{\pgfqpoint{1.534333in}{2.973840in}}%
\pgfpathcurveto{\pgfqpoint{1.526519in}{2.981654in}}{\pgfqpoint{1.515920in}{2.986044in}}{\pgfqpoint{1.504870in}{2.986044in}}%
\pgfpathcurveto{\pgfqpoint{1.493820in}{2.986044in}}{\pgfqpoint{1.483221in}{2.981654in}}{\pgfqpoint{1.475407in}{2.973840in}}%
\pgfpathcurveto{\pgfqpoint{1.467594in}{2.966027in}}{\pgfqpoint{1.463203in}{2.955428in}}{\pgfqpoint{1.463203in}{2.944377in}}%
\pgfpathcurveto{\pgfqpoint{1.463203in}{2.933327in}}{\pgfqpoint{1.467594in}{2.922728in}}{\pgfqpoint{1.475407in}{2.914915in}}%
\pgfpathcurveto{\pgfqpoint{1.483221in}{2.907101in}}{\pgfqpoint{1.493820in}{2.902711in}}{\pgfqpoint{1.504870in}{2.902711in}}%
\pgfpathclose%
\pgfusepath{stroke,fill}%
\end{pgfscope}%
\begin{pgfscope}%
\pgfpathrectangle{\pgfqpoint{0.750000in}{0.500000in}}{\pgfqpoint{4.650000in}{3.020000in}}%
\pgfusepath{clip}%
\pgfsetbuttcap%
\pgfsetroundjoin%
\definecolor{currentfill}{rgb}{0.121569,0.466667,0.705882}%
\pgfsetfillcolor{currentfill}%
\pgfsetlinewidth{1.003750pt}%
\definecolor{currentstroke}{rgb}{0.121569,0.466667,0.705882}%
\pgfsetstrokecolor{currentstroke}%
\pgfsetdash{}{0pt}%
\pgfpathmoveto{\pgfqpoint{1.444481in}{0.595606in}}%
\pgfpathcurveto{\pgfqpoint{1.455531in}{0.595606in}}{\pgfqpoint{1.466130in}{0.599996in}}{\pgfqpoint{1.473943in}{0.607810in}}%
\pgfpathcurveto{\pgfqpoint{1.481757in}{0.615624in}}{\pgfqpoint{1.486147in}{0.626223in}}{\pgfqpoint{1.486147in}{0.637273in}}%
\pgfpathcurveto{\pgfqpoint{1.486147in}{0.648323in}}{\pgfqpoint{1.481757in}{0.658922in}}{\pgfqpoint{1.473943in}{0.666736in}}%
\pgfpathcurveto{\pgfqpoint{1.466130in}{0.674549in}}{\pgfqpoint{1.455531in}{0.678939in}}{\pgfqpoint{1.444481in}{0.678939in}}%
\pgfpathcurveto{\pgfqpoint{1.433430in}{0.678939in}}{\pgfqpoint{1.422831in}{0.674549in}}{\pgfqpoint{1.415018in}{0.666736in}}%
\pgfpathcurveto{\pgfqpoint{1.407204in}{0.658922in}}{\pgfqpoint{1.402814in}{0.648323in}}{\pgfqpoint{1.402814in}{0.637273in}}%
\pgfpathcurveto{\pgfqpoint{1.402814in}{0.626223in}}{\pgfqpoint{1.407204in}{0.615624in}}{\pgfqpoint{1.415018in}{0.607810in}}%
\pgfpathcurveto{\pgfqpoint{1.422831in}{0.599996in}}{\pgfqpoint{1.433430in}{0.595606in}}{\pgfqpoint{1.444481in}{0.595606in}}%
\pgfpathclose%
\pgfusepath{stroke,fill}%
\end{pgfscope}%
\begin{pgfscope}%
\pgfpathrectangle{\pgfqpoint{0.750000in}{0.500000in}}{\pgfqpoint{4.650000in}{3.020000in}}%
\pgfusepath{clip}%
\pgfsetbuttcap%
\pgfsetroundjoin%
\definecolor{currentfill}{rgb}{0.121569,0.466667,0.705882}%
\pgfsetfillcolor{currentfill}%
\pgfsetlinewidth{1.003750pt}%
\definecolor{currentstroke}{rgb}{0.121569,0.466667,0.705882}%
\pgfsetstrokecolor{currentstroke}%
\pgfsetdash{}{0pt}%
\pgfpathmoveto{\pgfqpoint{1.323701in}{0.595606in}}%
\pgfpathcurveto{\pgfqpoint{1.334751in}{0.595606in}}{\pgfqpoint{1.345350in}{0.599996in}}{\pgfqpoint{1.353164in}{0.607810in}}%
\pgfpathcurveto{\pgfqpoint{1.360978in}{0.615624in}}{\pgfqpoint{1.365368in}{0.626223in}}{\pgfqpoint{1.365368in}{0.637273in}}%
\pgfpathcurveto{\pgfqpoint{1.365368in}{0.648323in}}{\pgfqpoint{1.360978in}{0.658922in}}{\pgfqpoint{1.353164in}{0.666736in}}%
\pgfpathcurveto{\pgfqpoint{1.345350in}{0.674549in}}{\pgfqpoint{1.334751in}{0.678939in}}{\pgfqpoint{1.323701in}{0.678939in}}%
\pgfpathcurveto{\pgfqpoint{1.312651in}{0.678939in}}{\pgfqpoint{1.302052in}{0.674549in}}{\pgfqpoint{1.294239in}{0.666736in}}%
\pgfpathcurveto{\pgfqpoint{1.286425in}{0.658922in}}{\pgfqpoint{1.282035in}{0.648323in}}{\pgfqpoint{1.282035in}{0.637273in}}%
\pgfpathcurveto{\pgfqpoint{1.282035in}{0.626223in}}{\pgfqpoint{1.286425in}{0.615624in}}{\pgfqpoint{1.294239in}{0.607810in}}%
\pgfpathcurveto{\pgfqpoint{1.302052in}{0.599996in}}{\pgfqpoint{1.312651in}{0.595606in}}{\pgfqpoint{1.323701in}{0.595606in}}%
\pgfpathclose%
\pgfusepath{stroke,fill}%
\end{pgfscope}%
\begin{pgfscope}%
\pgfpathrectangle{\pgfqpoint{0.750000in}{0.500000in}}{\pgfqpoint{4.650000in}{3.020000in}}%
\pgfusepath{clip}%
\pgfsetbuttcap%
\pgfsetroundjoin%
\definecolor{currentfill}{rgb}{1.000000,0.498039,0.054902}%
\pgfsetfillcolor{currentfill}%
\pgfsetlinewidth{1.003750pt}%
\definecolor{currentstroke}{rgb}{1.000000,0.498039,0.054902}%
\pgfsetstrokecolor{currentstroke}%
\pgfsetdash{}{0pt}%
\pgfpathmoveto{\pgfqpoint{2.229545in}{2.902711in}}%
\pgfpathcurveto{\pgfqpoint{2.240596in}{2.902711in}}{\pgfqpoint{2.251195in}{2.907101in}}{\pgfqpoint{2.259008in}{2.914915in}}%
\pgfpathcurveto{\pgfqpoint{2.266822in}{2.922728in}}{\pgfqpoint{2.271212in}{2.933327in}}{\pgfqpoint{2.271212in}{2.944377in}}%
\pgfpathcurveto{\pgfqpoint{2.271212in}{2.955428in}}{\pgfqpoint{2.266822in}{2.966027in}}{\pgfqpoint{2.259008in}{2.973840in}}%
\pgfpathcurveto{\pgfqpoint{2.251195in}{2.981654in}}{\pgfqpoint{2.240596in}{2.986044in}}{\pgfqpoint{2.229545in}{2.986044in}}%
\pgfpathcurveto{\pgfqpoint{2.218495in}{2.986044in}}{\pgfqpoint{2.207896in}{2.981654in}}{\pgfqpoint{2.200083in}{2.973840in}}%
\pgfpathcurveto{\pgfqpoint{2.192269in}{2.966027in}}{\pgfqpoint{2.187879in}{2.955428in}}{\pgfqpoint{2.187879in}{2.944377in}}%
\pgfpathcurveto{\pgfqpoint{2.187879in}{2.933327in}}{\pgfqpoint{2.192269in}{2.922728in}}{\pgfqpoint{2.200083in}{2.914915in}}%
\pgfpathcurveto{\pgfqpoint{2.207896in}{2.907101in}}{\pgfqpoint{2.218495in}{2.902711in}}{\pgfqpoint{2.229545in}{2.902711in}}%
\pgfpathclose%
\pgfusepath{stroke,fill}%
\end{pgfscope}%
\begin{pgfscope}%
\pgfpathrectangle{\pgfqpoint{0.750000in}{0.500000in}}{\pgfqpoint{4.650000in}{3.020000in}}%
\pgfusepath{clip}%
\pgfsetbuttcap%
\pgfsetroundjoin%
\definecolor{currentfill}{rgb}{1.000000,0.498039,0.054902}%
\pgfsetfillcolor{currentfill}%
\pgfsetlinewidth{1.003750pt}%
\definecolor{currentstroke}{rgb}{1.000000,0.498039,0.054902}%
\pgfsetstrokecolor{currentstroke}%
\pgfsetdash{}{0pt}%
\pgfpathmoveto{\pgfqpoint{1.686039in}{2.910401in}}%
\pgfpathcurveto{\pgfqpoint{1.697089in}{2.910401in}}{\pgfqpoint{1.707688in}{2.914791in}}{\pgfqpoint{1.715502in}{2.922605in}}%
\pgfpathcurveto{\pgfqpoint{1.723315in}{2.930419in}}{\pgfqpoint{1.727706in}{2.941018in}}{\pgfqpoint{1.727706in}{2.952068in}}%
\pgfpathcurveto{\pgfqpoint{1.727706in}{2.963118in}}{\pgfqpoint{1.723315in}{2.973717in}}{\pgfqpoint{1.715502in}{2.981531in}}%
\pgfpathcurveto{\pgfqpoint{1.707688in}{2.989344in}}{\pgfqpoint{1.697089in}{2.993734in}}{\pgfqpoint{1.686039in}{2.993734in}}%
\pgfpathcurveto{\pgfqpoint{1.674989in}{2.993734in}}{\pgfqpoint{1.664390in}{2.989344in}}{\pgfqpoint{1.656576in}{2.981531in}}%
\pgfpathcurveto{\pgfqpoint{1.648763in}{2.973717in}}{\pgfqpoint{1.644372in}{2.963118in}}{\pgfqpoint{1.644372in}{2.952068in}}%
\pgfpathcurveto{\pgfqpoint{1.644372in}{2.941018in}}{\pgfqpoint{1.648763in}{2.930419in}}{\pgfqpoint{1.656576in}{2.922605in}}%
\pgfpathcurveto{\pgfqpoint{1.664390in}{2.914791in}}{\pgfqpoint{1.674989in}{2.910401in}}{\pgfqpoint{1.686039in}{2.910401in}}%
\pgfpathclose%
\pgfusepath{stroke,fill}%
\end{pgfscope}%
\begin{pgfscope}%
\pgfpathrectangle{\pgfqpoint{0.750000in}{0.500000in}}{\pgfqpoint{4.650000in}{3.020000in}}%
\pgfusepath{clip}%
\pgfsetbuttcap%
\pgfsetroundjoin%
\definecolor{currentfill}{rgb}{1.000000,0.498039,0.054902}%
\pgfsetfillcolor{currentfill}%
\pgfsetlinewidth{1.003750pt}%
\definecolor{currentstroke}{rgb}{1.000000,0.498039,0.054902}%
\pgfsetstrokecolor{currentstroke}%
\pgfsetdash{}{0pt}%
\pgfpathmoveto{\pgfqpoint{1.927597in}{2.910401in}}%
\pgfpathcurveto{\pgfqpoint{1.938648in}{2.910401in}}{\pgfqpoint{1.949247in}{2.914791in}}{\pgfqpoint{1.957060in}{2.922605in}}%
\pgfpathcurveto{\pgfqpoint{1.964874in}{2.930419in}}{\pgfqpoint{1.969264in}{2.941018in}}{\pgfqpoint{1.969264in}{2.952068in}}%
\pgfpathcurveto{\pgfqpoint{1.969264in}{2.963118in}}{\pgfqpoint{1.964874in}{2.973717in}}{\pgfqpoint{1.957060in}{2.981531in}}%
\pgfpathcurveto{\pgfqpoint{1.949247in}{2.989344in}}{\pgfqpoint{1.938648in}{2.993734in}}{\pgfqpoint{1.927597in}{2.993734in}}%
\pgfpathcurveto{\pgfqpoint{1.916547in}{2.993734in}}{\pgfqpoint{1.905948in}{2.989344in}}{\pgfqpoint{1.898135in}{2.981531in}}%
\pgfpathcurveto{\pgfqpoint{1.890321in}{2.973717in}}{\pgfqpoint{1.885931in}{2.963118in}}{\pgfqpoint{1.885931in}{2.952068in}}%
\pgfpathcurveto{\pgfqpoint{1.885931in}{2.941018in}}{\pgfqpoint{1.890321in}{2.930419in}}{\pgfqpoint{1.898135in}{2.922605in}}%
\pgfpathcurveto{\pgfqpoint{1.905948in}{2.914791in}}{\pgfqpoint{1.916547in}{2.910401in}}{\pgfqpoint{1.927597in}{2.910401in}}%
\pgfpathclose%
\pgfusepath{stroke,fill}%
\end{pgfscope}%
\begin{pgfscope}%
\pgfpathrectangle{\pgfqpoint{0.750000in}{0.500000in}}{\pgfqpoint{4.650000in}{3.020000in}}%
\pgfusepath{clip}%
\pgfsetbuttcap%
\pgfsetroundjoin%
\definecolor{currentfill}{rgb}{1.000000,0.498039,0.054902}%
\pgfsetfillcolor{currentfill}%
\pgfsetlinewidth{1.003750pt}%
\definecolor{currentstroke}{rgb}{1.000000,0.498039,0.054902}%
\pgfsetstrokecolor{currentstroke}%
\pgfsetdash{}{0pt}%
\pgfpathmoveto{\pgfqpoint{1.987987in}{2.925782in}}%
\pgfpathcurveto{\pgfqpoint{1.999037in}{2.925782in}}{\pgfqpoint{2.009636in}{2.930172in}}{\pgfqpoint{2.017450in}{2.937986in}}%
\pgfpathcurveto{\pgfqpoint{2.025263in}{2.945799in}}{\pgfqpoint{2.029654in}{2.956398in}}{\pgfqpoint{2.029654in}{2.967448in}}%
\pgfpathcurveto{\pgfqpoint{2.029654in}{2.978499in}}{\pgfqpoint{2.025263in}{2.989098in}}{\pgfqpoint{2.017450in}{2.996911in}}%
\pgfpathcurveto{\pgfqpoint{2.009636in}{3.004725in}}{\pgfqpoint{1.999037in}{3.009115in}}{\pgfqpoint{1.987987in}{3.009115in}}%
\pgfpathcurveto{\pgfqpoint{1.976937in}{3.009115in}}{\pgfqpoint{1.966338in}{3.004725in}}{\pgfqpoint{1.958524in}{2.996911in}}%
\pgfpathcurveto{\pgfqpoint{1.950711in}{2.989098in}}{\pgfqpoint{1.946320in}{2.978499in}}{\pgfqpoint{1.946320in}{2.967448in}}%
\pgfpathcurveto{\pgfqpoint{1.946320in}{2.956398in}}{\pgfqpoint{1.950711in}{2.945799in}}{\pgfqpoint{1.958524in}{2.937986in}}%
\pgfpathcurveto{\pgfqpoint{1.966338in}{2.930172in}}{\pgfqpoint{1.976937in}{2.925782in}}{\pgfqpoint{1.987987in}{2.925782in}}%
\pgfpathclose%
\pgfusepath{stroke,fill}%
\end{pgfscope}%
\begin{pgfscope}%
\pgfpathrectangle{\pgfqpoint{0.750000in}{0.500000in}}{\pgfqpoint{4.650000in}{3.020000in}}%
\pgfusepath{clip}%
\pgfsetbuttcap%
\pgfsetroundjoin%
\definecolor{currentfill}{rgb}{0.839216,0.152941,0.156863}%
\pgfsetfillcolor{currentfill}%
\pgfsetlinewidth{1.003750pt}%
\definecolor{currentstroke}{rgb}{0.839216,0.152941,0.156863}%
\pgfsetstrokecolor{currentstroke}%
\pgfsetdash{}{0pt}%
\pgfpathmoveto{\pgfqpoint{1.686039in}{2.902711in}}%
\pgfpathcurveto{\pgfqpoint{1.697089in}{2.902711in}}{\pgfqpoint{1.707688in}{2.907101in}}{\pgfqpoint{1.715502in}{2.914915in}}%
\pgfpathcurveto{\pgfqpoint{1.723315in}{2.922728in}}{\pgfqpoint{1.727706in}{2.933327in}}{\pgfqpoint{1.727706in}{2.944377in}}%
\pgfpathcurveto{\pgfqpoint{1.727706in}{2.955428in}}{\pgfqpoint{1.723315in}{2.966027in}}{\pgfqpoint{1.715502in}{2.973840in}}%
\pgfpathcurveto{\pgfqpoint{1.707688in}{2.981654in}}{\pgfqpoint{1.697089in}{2.986044in}}{\pgfqpoint{1.686039in}{2.986044in}}%
\pgfpathcurveto{\pgfqpoint{1.674989in}{2.986044in}}{\pgfqpoint{1.664390in}{2.981654in}}{\pgfqpoint{1.656576in}{2.973840in}}%
\pgfpathcurveto{\pgfqpoint{1.648763in}{2.966027in}}{\pgfqpoint{1.644372in}{2.955428in}}{\pgfqpoint{1.644372in}{2.944377in}}%
\pgfpathcurveto{\pgfqpoint{1.644372in}{2.933327in}}{\pgfqpoint{1.648763in}{2.922728in}}{\pgfqpoint{1.656576in}{2.914915in}}%
\pgfpathcurveto{\pgfqpoint{1.664390in}{2.907101in}}{\pgfqpoint{1.674989in}{2.902711in}}{\pgfqpoint{1.686039in}{2.902711in}}%
\pgfpathclose%
\pgfusepath{stroke,fill}%
\end{pgfscope}%
\begin{pgfscope}%
\pgfpathrectangle{\pgfqpoint{0.750000in}{0.500000in}}{\pgfqpoint{4.650000in}{3.020000in}}%
\pgfusepath{clip}%
\pgfsetbuttcap%
\pgfsetroundjoin%
\definecolor{currentfill}{rgb}{1.000000,0.498039,0.054902}%
\pgfsetfillcolor{currentfill}%
\pgfsetlinewidth{1.003750pt}%
\definecolor{currentstroke}{rgb}{1.000000,0.498039,0.054902}%
\pgfsetstrokecolor{currentstroke}%
\pgfsetdash{}{0pt}%
\pgfpathmoveto{\pgfqpoint{1.565260in}{2.910401in}}%
\pgfpathcurveto{\pgfqpoint{1.576310in}{2.910401in}}{\pgfqpoint{1.586909in}{2.914791in}}{\pgfqpoint{1.594723in}{2.922605in}}%
\pgfpathcurveto{\pgfqpoint{1.602536in}{2.930419in}}{\pgfqpoint{1.606926in}{2.941018in}}{\pgfqpoint{1.606926in}{2.952068in}}%
\pgfpathcurveto{\pgfqpoint{1.606926in}{2.963118in}}{\pgfqpoint{1.602536in}{2.973717in}}{\pgfqpoint{1.594723in}{2.981531in}}%
\pgfpathcurveto{\pgfqpoint{1.586909in}{2.989344in}}{\pgfqpoint{1.576310in}{2.993734in}}{\pgfqpoint{1.565260in}{2.993734in}}%
\pgfpathcurveto{\pgfqpoint{1.554210in}{2.993734in}}{\pgfqpoint{1.543611in}{2.989344in}}{\pgfqpoint{1.535797in}{2.981531in}}%
\pgfpathcurveto{\pgfqpoint{1.527983in}{2.973717in}}{\pgfqpoint{1.523593in}{2.963118in}}{\pgfqpoint{1.523593in}{2.952068in}}%
\pgfpathcurveto{\pgfqpoint{1.523593in}{2.941018in}}{\pgfqpoint{1.527983in}{2.930419in}}{\pgfqpoint{1.535797in}{2.922605in}}%
\pgfpathcurveto{\pgfqpoint{1.543611in}{2.914791in}}{\pgfqpoint{1.554210in}{2.910401in}}{\pgfqpoint{1.565260in}{2.910401in}}%
\pgfpathclose%
\pgfusepath{stroke,fill}%
\end{pgfscope}%
\begin{pgfscope}%
\pgfpathrectangle{\pgfqpoint{0.750000in}{0.500000in}}{\pgfqpoint{4.650000in}{3.020000in}}%
\pgfusepath{clip}%
\pgfsetbuttcap%
\pgfsetroundjoin%
\definecolor{currentfill}{rgb}{1.000000,0.498039,0.054902}%
\pgfsetfillcolor{currentfill}%
\pgfsetlinewidth{1.003750pt}%
\definecolor{currentstroke}{rgb}{1.000000,0.498039,0.054902}%
\pgfsetstrokecolor{currentstroke}%
\pgfsetdash{}{0pt}%
\pgfpathmoveto{\pgfqpoint{2.591883in}{3.321835in}}%
\pgfpathcurveto{\pgfqpoint{2.602933in}{3.321835in}}{\pgfqpoint{2.613532in}{3.326225in}}{\pgfqpoint{2.621346in}{3.334039in}}%
\pgfpathcurveto{\pgfqpoint{2.629160in}{3.341852in}}{\pgfqpoint{2.633550in}{3.352451in}}{\pgfqpoint{2.633550in}{3.363501in}}%
\pgfpathcurveto{\pgfqpoint{2.633550in}{3.374552in}}{\pgfqpoint{2.629160in}{3.385151in}}{\pgfqpoint{2.621346in}{3.392964in}}%
\pgfpathcurveto{\pgfqpoint{2.613532in}{3.400778in}}{\pgfqpoint{2.602933in}{3.405168in}}{\pgfqpoint{2.591883in}{3.405168in}}%
\pgfpathcurveto{\pgfqpoint{2.580833in}{3.405168in}}{\pgfqpoint{2.570234in}{3.400778in}}{\pgfqpoint{2.562420in}{3.392964in}}%
\pgfpathcurveto{\pgfqpoint{2.554607in}{3.385151in}}{\pgfqpoint{2.550216in}{3.374552in}}{\pgfqpoint{2.550216in}{3.363501in}}%
\pgfpathcurveto{\pgfqpoint{2.550216in}{3.352451in}}{\pgfqpoint{2.554607in}{3.341852in}}{\pgfqpoint{2.562420in}{3.334039in}}%
\pgfpathcurveto{\pgfqpoint{2.570234in}{3.326225in}}{\pgfqpoint{2.580833in}{3.321835in}}{\pgfqpoint{2.591883in}{3.321835in}}%
\pgfpathclose%
\pgfusepath{stroke,fill}%
\end{pgfscope}%
\begin{pgfscope}%
\pgfpathrectangle{\pgfqpoint{0.750000in}{0.500000in}}{\pgfqpoint{4.650000in}{3.020000in}}%
\pgfusepath{clip}%
\pgfsetbuttcap%
\pgfsetroundjoin%
\definecolor{currentfill}{rgb}{0.121569,0.466667,0.705882}%
\pgfsetfillcolor{currentfill}%
\pgfsetlinewidth{1.003750pt}%
\definecolor{currentstroke}{rgb}{0.121569,0.466667,0.705882}%
\pgfsetstrokecolor{currentstroke}%
\pgfsetdash{}{0pt}%
\pgfpathmoveto{\pgfqpoint{0.961364in}{0.595606in}}%
\pgfpathcurveto{\pgfqpoint{0.972414in}{0.595606in}}{\pgfqpoint{0.983013in}{0.599996in}}{\pgfqpoint{0.990826in}{0.607810in}}%
\pgfpathcurveto{\pgfqpoint{0.998640in}{0.615624in}}{\pgfqpoint{1.003030in}{0.626223in}}{\pgfqpoint{1.003030in}{0.637273in}}%
\pgfpathcurveto{\pgfqpoint{1.003030in}{0.648323in}}{\pgfqpoint{0.998640in}{0.658922in}}{\pgfqpoint{0.990826in}{0.666736in}}%
\pgfpathcurveto{\pgfqpoint{0.983013in}{0.674549in}}{\pgfqpoint{0.972414in}{0.678939in}}{\pgfqpoint{0.961364in}{0.678939in}}%
\pgfpathcurveto{\pgfqpoint{0.950314in}{0.678939in}}{\pgfqpoint{0.939714in}{0.674549in}}{\pgfqpoint{0.931901in}{0.666736in}}%
\pgfpathcurveto{\pgfqpoint{0.924087in}{0.658922in}}{\pgfqpoint{0.919697in}{0.648323in}}{\pgfqpoint{0.919697in}{0.637273in}}%
\pgfpathcurveto{\pgfqpoint{0.919697in}{0.626223in}}{\pgfqpoint{0.924087in}{0.615624in}}{\pgfqpoint{0.931901in}{0.607810in}}%
\pgfpathcurveto{\pgfqpoint{0.939714in}{0.599996in}}{\pgfqpoint{0.950314in}{0.595606in}}{\pgfqpoint{0.961364in}{0.595606in}}%
\pgfpathclose%
\pgfusepath{stroke,fill}%
\end{pgfscope}%
\begin{pgfscope}%
\pgfpathrectangle{\pgfqpoint{0.750000in}{0.500000in}}{\pgfqpoint{4.650000in}{3.020000in}}%
\pgfusepath{clip}%
\pgfsetbuttcap%
\pgfsetroundjoin%
\definecolor{currentfill}{rgb}{0.121569,0.466667,0.705882}%
\pgfsetfillcolor{currentfill}%
\pgfsetlinewidth{1.003750pt}%
\definecolor{currentstroke}{rgb}{0.121569,0.466667,0.705882}%
\pgfsetstrokecolor{currentstroke}%
\pgfsetdash{}{0pt}%
\pgfpathmoveto{\pgfqpoint{1.202922in}{0.730187in}}%
\pgfpathcurveto{\pgfqpoint{1.213972in}{0.730187in}}{\pgfqpoint{1.224571in}{0.734577in}}{\pgfqpoint{1.232385in}{0.742391in}}%
\pgfpathcurveto{\pgfqpoint{1.240198in}{0.750205in}}{\pgfqpoint{1.244589in}{0.760804in}}{\pgfqpoint{1.244589in}{0.771854in}}%
\pgfpathcurveto{\pgfqpoint{1.244589in}{0.782904in}}{\pgfqpoint{1.240198in}{0.793503in}}{\pgfqpoint{1.232385in}{0.801317in}}%
\pgfpathcurveto{\pgfqpoint{1.224571in}{0.809130in}}{\pgfqpoint{1.213972in}{0.813520in}}{\pgfqpoint{1.202922in}{0.813520in}}%
\pgfpathcurveto{\pgfqpoint{1.191872in}{0.813520in}}{\pgfqpoint{1.181273in}{0.809130in}}{\pgfqpoint{1.173459in}{0.801317in}}%
\pgfpathcurveto{\pgfqpoint{1.165646in}{0.793503in}}{\pgfqpoint{1.161255in}{0.782904in}}{\pgfqpoint{1.161255in}{0.771854in}}%
\pgfpathcurveto{\pgfqpoint{1.161255in}{0.760804in}}{\pgfqpoint{1.165646in}{0.750205in}}{\pgfqpoint{1.173459in}{0.742391in}}%
\pgfpathcurveto{\pgfqpoint{1.181273in}{0.734577in}}{\pgfqpoint{1.191872in}{0.730187in}}{\pgfqpoint{1.202922in}{0.730187in}}%
\pgfpathclose%
\pgfusepath{stroke,fill}%
\end{pgfscope}%
\begin{pgfscope}%
\pgfpathrectangle{\pgfqpoint{0.750000in}{0.500000in}}{\pgfqpoint{4.650000in}{3.020000in}}%
\pgfusepath{clip}%
\pgfsetbuttcap%
\pgfsetroundjoin%
\definecolor{currentfill}{rgb}{1.000000,0.498039,0.054902}%
\pgfsetfillcolor{currentfill}%
\pgfsetlinewidth{1.003750pt}%
\definecolor{currentstroke}{rgb}{1.000000,0.498039,0.054902}%
\pgfsetstrokecolor{currentstroke}%
\pgfsetdash{}{0pt}%
\pgfpathmoveto{\pgfqpoint{1.504870in}{2.875794in}}%
\pgfpathcurveto{\pgfqpoint{1.515920in}{2.875794in}}{\pgfqpoint{1.526519in}{2.880185in}}{\pgfqpoint{1.534333in}{2.887998in}}%
\pgfpathcurveto{\pgfqpoint{1.542147in}{2.895812in}}{\pgfqpoint{1.546537in}{2.906411in}}{\pgfqpoint{1.546537in}{2.917461in}}%
\pgfpathcurveto{\pgfqpoint{1.546537in}{2.928511in}}{\pgfqpoint{1.542147in}{2.939110in}}{\pgfqpoint{1.534333in}{2.946924in}}%
\pgfpathcurveto{\pgfqpoint{1.526519in}{2.954738in}}{\pgfqpoint{1.515920in}{2.959128in}}{\pgfqpoint{1.504870in}{2.959128in}}%
\pgfpathcurveto{\pgfqpoint{1.493820in}{2.959128in}}{\pgfqpoint{1.483221in}{2.954738in}}{\pgfqpoint{1.475407in}{2.946924in}}%
\pgfpathcurveto{\pgfqpoint{1.467594in}{2.939110in}}{\pgfqpoint{1.463203in}{2.928511in}}{\pgfqpoint{1.463203in}{2.917461in}}%
\pgfpathcurveto{\pgfqpoint{1.463203in}{2.906411in}}{\pgfqpoint{1.467594in}{2.895812in}}{\pgfqpoint{1.475407in}{2.887998in}}%
\pgfpathcurveto{\pgfqpoint{1.483221in}{2.880185in}}{\pgfqpoint{1.493820in}{2.875794in}}{\pgfqpoint{1.504870in}{2.875794in}}%
\pgfpathclose%
\pgfusepath{stroke,fill}%
\end{pgfscope}%
\begin{pgfscope}%
\pgfpathrectangle{\pgfqpoint{0.750000in}{0.500000in}}{\pgfqpoint{4.650000in}{3.020000in}}%
\pgfusepath{clip}%
\pgfsetbuttcap%
\pgfsetroundjoin%
\definecolor{currentfill}{rgb}{0.121569,0.466667,0.705882}%
\pgfsetfillcolor{currentfill}%
\pgfsetlinewidth{1.003750pt}%
\definecolor{currentstroke}{rgb}{0.121569,0.466667,0.705882}%
\pgfsetstrokecolor{currentstroke}%
\pgfsetdash{}{0pt}%
\pgfpathmoveto{\pgfqpoint{1.323701in}{0.595606in}}%
\pgfpathcurveto{\pgfqpoint{1.334751in}{0.595606in}}{\pgfqpoint{1.345350in}{0.599996in}}{\pgfqpoint{1.353164in}{0.607810in}}%
\pgfpathcurveto{\pgfqpoint{1.360978in}{0.615624in}}{\pgfqpoint{1.365368in}{0.626223in}}{\pgfqpoint{1.365368in}{0.637273in}}%
\pgfpathcurveto{\pgfqpoint{1.365368in}{0.648323in}}{\pgfqpoint{1.360978in}{0.658922in}}{\pgfqpoint{1.353164in}{0.666736in}}%
\pgfpathcurveto{\pgfqpoint{1.345350in}{0.674549in}}{\pgfqpoint{1.334751in}{0.678939in}}{\pgfqpoint{1.323701in}{0.678939in}}%
\pgfpathcurveto{\pgfqpoint{1.312651in}{0.678939in}}{\pgfqpoint{1.302052in}{0.674549in}}{\pgfqpoint{1.294239in}{0.666736in}}%
\pgfpathcurveto{\pgfqpoint{1.286425in}{0.658922in}}{\pgfqpoint{1.282035in}{0.648323in}}{\pgfqpoint{1.282035in}{0.637273in}}%
\pgfpathcurveto{\pgfqpoint{1.282035in}{0.626223in}}{\pgfqpoint{1.286425in}{0.615624in}}{\pgfqpoint{1.294239in}{0.607810in}}%
\pgfpathcurveto{\pgfqpoint{1.302052in}{0.599996in}}{\pgfqpoint{1.312651in}{0.595606in}}{\pgfqpoint{1.323701in}{0.595606in}}%
\pgfpathclose%
\pgfusepath{stroke,fill}%
\end{pgfscope}%
\begin{pgfscope}%
\pgfpathrectangle{\pgfqpoint{0.750000in}{0.500000in}}{\pgfqpoint{4.650000in}{3.020000in}}%
\pgfusepath{clip}%
\pgfsetbuttcap%
\pgfsetroundjoin%
\definecolor{currentfill}{rgb}{0.121569,0.466667,0.705882}%
\pgfsetfillcolor{currentfill}%
\pgfsetlinewidth{1.003750pt}%
\definecolor{currentstroke}{rgb}{0.121569,0.466667,0.705882}%
\pgfsetstrokecolor{currentstroke}%
\pgfsetdash{}{0pt}%
\pgfpathmoveto{\pgfqpoint{0.961364in}{0.595606in}}%
\pgfpathcurveto{\pgfqpoint{0.972414in}{0.595606in}}{\pgfqpoint{0.983013in}{0.599996in}}{\pgfqpoint{0.990826in}{0.607810in}}%
\pgfpathcurveto{\pgfqpoint{0.998640in}{0.615624in}}{\pgfqpoint{1.003030in}{0.626223in}}{\pgfqpoint{1.003030in}{0.637273in}}%
\pgfpathcurveto{\pgfqpoint{1.003030in}{0.648323in}}{\pgfqpoint{0.998640in}{0.658922in}}{\pgfqpoint{0.990826in}{0.666736in}}%
\pgfpathcurveto{\pgfqpoint{0.983013in}{0.674549in}}{\pgfqpoint{0.972414in}{0.678939in}}{\pgfqpoint{0.961364in}{0.678939in}}%
\pgfpathcurveto{\pgfqpoint{0.950314in}{0.678939in}}{\pgfqpoint{0.939714in}{0.674549in}}{\pgfqpoint{0.931901in}{0.666736in}}%
\pgfpathcurveto{\pgfqpoint{0.924087in}{0.658922in}}{\pgfqpoint{0.919697in}{0.648323in}}{\pgfqpoint{0.919697in}{0.637273in}}%
\pgfpathcurveto{\pgfqpoint{0.919697in}{0.626223in}}{\pgfqpoint{0.924087in}{0.615624in}}{\pgfqpoint{0.931901in}{0.607810in}}%
\pgfpathcurveto{\pgfqpoint{0.939714in}{0.599996in}}{\pgfqpoint{0.950314in}{0.595606in}}{\pgfqpoint{0.961364in}{0.595606in}}%
\pgfpathclose%
\pgfusepath{stroke,fill}%
\end{pgfscope}%
\begin{pgfscope}%
\pgfpathrectangle{\pgfqpoint{0.750000in}{0.500000in}}{\pgfqpoint{4.650000in}{3.020000in}}%
\pgfusepath{clip}%
\pgfsetbuttcap%
\pgfsetroundjoin%
\definecolor{currentfill}{rgb}{0.121569,0.466667,0.705882}%
\pgfsetfillcolor{currentfill}%
\pgfsetlinewidth{1.003750pt}%
\definecolor{currentstroke}{rgb}{0.121569,0.466667,0.705882}%
\pgfsetstrokecolor{currentstroke}%
\pgfsetdash{}{0pt}%
\pgfpathmoveto{\pgfqpoint{1.021753in}{2.291328in}}%
\pgfpathcurveto{\pgfqpoint{1.032803in}{2.291328in}}{\pgfqpoint{1.043402in}{2.295718in}}{\pgfqpoint{1.051216in}{2.303532in}}%
\pgfpathcurveto{\pgfqpoint{1.059030in}{2.311345in}}{\pgfqpoint{1.063420in}{2.321945in}}{\pgfqpoint{1.063420in}{2.332995in}}%
\pgfpathcurveto{\pgfqpoint{1.063420in}{2.344045in}}{\pgfqpoint{1.059030in}{2.354644in}}{\pgfqpoint{1.051216in}{2.362457in}}%
\pgfpathcurveto{\pgfqpoint{1.043402in}{2.370271in}}{\pgfqpoint{1.032803in}{2.374661in}}{\pgfqpoint{1.021753in}{2.374661in}}%
\pgfpathcurveto{\pgfqpoint{1.010703in}{2.374661in}}{\pgfqpoint{1.000104in}{2.370271in}}{\pgfqpoint{0.992290in}{2.362457in}}%
\pgfpathcurveto{\pgfqpoint{0.984477in}{2.354644in}}{\pgfqpoint{0.980087in}{2.344045in}}{\pgfqpoint{0.980087in}{2.332995in}}%
\pgfpathcurveto{\pgfqpoint{0.980087in}{2.321945in}}{\pgfqpoint{0.984477in}{2.311345in}}{\pgfqpoint{0.992290in}{2.303532in}}%
\pgfpathcurveto{\pgfqpoint{1.000104in}{2.295718in}}{\pgfqpoint{1.010703in}{2.291328in}}{\pgfqpoint{1.021753in}{2.291328in}}%
\pgfpathclose%
\pgfusepath{stroke,fill}%
\end{pgfscope}%
\begin{pgfscope}%
\pgfpathrectangle{\pgfqpoint{0.750000in}{0.500000in}}{\pgfqpoint{4.650000in}{3.020000in}}%
\pgfusepath{clip}%
\pgfsetbuttcap%
\pgfsetroundjoin%
\definecolor{currentfill}{rgb}{1.000000,0.498039,0.054902}%
\pgfsetfillcolor{currentfill}%
\pgfsetlinewidth{1.003750pt}%
\definecolor{currentstroke}{rgb}{1.000000,0.498039,0.054902}%
\pgfsetstrokecolor{currentstroke}%
\pgfsetdash{}{0pt}%
\pgfpathmoveto{\pgfqpoint{1.927597in}{2.906556in}}%
\pgfpathcurveto{\pgfqpoint{1.938648in}{2.906556in}}{\pgfqpoint{1.949247in}{2.910946in}}{\pgfqpoint{1.957060in}{2.918760in}}%
\pgfpathcurveto{\pgfqpoint{1.964874in}{2.926573in}}{\pgfqpoint{1.969264in}{2.937172in}}{\pgfqpoint{1.969264in}{2.948223in}}%
\pgfpathcurveto{\pgfqpoint{1.969264in}{2.959273in}}{\pgfqpoint{1.964874in}{2.969872in}}{\pgfqpoint{1.957060in}{2.977685in}}%
\pgfpathcurveto{\pgfqpoint{1.949247in}{2.985499in}}{\pgfqpoint{1.938648in}{2.989889in}}{\pgfqpoint{1.927597in}{2.989889in}}%
\pgfpathcurveto{\pgfqpoint{1.916547in}{2.989889in}}{\pgfqpoint{1.905948in}{2.985499in}}{\pgfqpoint{1.898135in}{2.977685in}}%
\pgfpathcurveto{\pgfqpoint{1.890321in}{2.969872in}}{\pgfqpoint{1.885931in}{2.959273in}}{\pgfqpoint{1.885931in}{2.948223in}}%
\pgfpathcurveto{\pgfqpoint{1.885931in}{2.937172in}}{\pgfqpoint{1.890321in}{2.926573in}}{\pgfqpoint{1.898135in}{2.918760in}}%
\pgfpathcurveto{\pgfqpoint{1.905948in}{2.910946in}}{\pgfqpoint{1.916547in}{2.906556in}}{\pgfqpoint{1.927597in}{2.906556in}}%
\pgfpathclose%
\pgfusepath{stroke,fill}%
\end{pgfscope}%
\begin{pgfscope}%
\pgfpathrectangle{\pgfqpoint{0.750000in}{0.500000in}}{\pgfqpoint{4.650000in}{3.020000in}}%
\pgfusepath{clip}%
\pgfsetbuttcap%
\pgfsetroundjoin%
\definecolor{currentfill}{rgb}{1.000000,0.498039,0.054902}%
\pgfsetfillcolor{currentfill}%
\pgfsetlinewidth{1.003750pt}%
\definecolor{currentstroke}{rgb}{1.000000,0.498039,0.054902}%
\pgfsetstrokecolor{currentstroke}%
\pgfsetdash{}{0pt}%
\pgfpathmoveto{\pgfqpoint{1.806818in}{2.914246in}}%
\pgfpathcurveto{\pgfqpoint{1.817868in}{2.914246in}}{\pgfqpoint{1.828467in}{2.918637in}}{\pgfqpoint{1.836281in}{2.926450in}}%
\pgfpathcurveto{\pgfqpoint{1.844095in}{2.934264in}}{\pgfqpoint{1.848485in}{2.944863in}}{\pgfqpoint{1.848485in}{2.955913in}}%
\pgfpathcurveto{\pgfqpoint{1.848485in}{2.966963in}}{\pgfqpoint{1.844095in}{2.977562in}}{\pgfqpoint{1.836281in}{2.985376in}}%
\pgfpathcurveto{\pgfqpoint{1.828467in}{2.993189in}}{\pgfqpoint{1.817868in}{2.997580in}}{\pgfqpoint{1.806818in}{2.997580in}}%
\pgfpathcurveto{\pgfqpoint{1.795768in}{2.997580in}}{\pgfqpoint{1.785169in}{2.993189in}}{\pgfqpoint{1.777355in}{2.985376in}}%
\pgfpathcurveto{\pgfqpoint{1.769542in}{2.977562in}}{\pgfqpoint{1.765152in}{2.966963in}}{\pgfqpoint{1.765152in}{2.955913in}}%
\pgfpathcurveto{\pgfqpoint{1.765152in}{2.944863in}}{\pgfqpoint{1.769542in}{2.934264in}}{\pgfqpoint{1.777355in}{2.926450in}}%
\pgfpathcurveto{\pgfqpoint{1.785169in}{2.918637in}}{\pgfqpoint{1.795768in}{2.914246in}}{\pgfqpoint{1.806818in}{2.914246in}}%
\pgfpathclose%
\pgfusepath{stroke,fill}%
\end{pgfscope}%
\begin{pgfscope}%
\pgfsetbuttcap%
\pgfsetroundjoin%
\definecolor{currentfill}{rgb}{0.000000,0.000000,0.000000}%
\pgfsetfillcolor{currentfill}%
\pgfsetlinewidth{0.803000pt}%
\definecolor{currentstroke}{rgb}{0.000000,0.000000,0.000000}%
\pgfsetstrokecolor{currentstroke}%
\pgfsetdash{}{0pt}%
\pgfsys@defobject{currentmarker}{\pgfqpoint{0.000000in}{-0.048611in}}{\pgfqpoint{0.000000in}{0.000000in}}{%
\pgfpathmoveto{\pgfqpoint{0.000000in}{0.000000in}}%
\pgfpathlineto{\pgfqpoint{0.000000in}{-0.048611in}}%
\pgfusepath{stroke,fill}%
}%
\begin{pgfscope}%
\pgfsys@transformshift{0.900974in}{0.500000in}%
\pgfsys@useobject{currentmarker}{}%
\end{pgfscope}%
\end{pgfscope}%
\begin{pgfscope}%
\definecolor{textcolor}{rgb}{0.000000,0.000000,0.000000}%
\pgfsetstrokecolor{textcolor}%
\pgfsetfillcolor{textcolor}%
\pgftext[x=0.900974in,y=0.402778in,,top]{\color{textcolor}\rmfamily\fontsize{10.000000}{12.000000}\selectfont \(\displaystyle {0}\)}%
\end{pgfscope}%
\begin{pgfscope}%
\pgfsetbuttcap%
\pgfsetroundjoin%
\definecolor{currentfill}{rgb}{0.000000,0.000000,0.000000}%
\pgfsetfillcolor{currentfill}%
\pgfsetlinewidth{0.803000pt}%
\definecolor{currentstroke}{rgb}{0.000000,0.000000,0.000000}%
\pgfsetstrokecolor{currentstroke}%
\pgfsetdash{}{0pt}%
\pgfsys@defobject{currentmarker}{\pgfqpoint{0.000000in}{-0.048611in}}{\pgfqpoint{0.000000in}{0.000000in}}{%
\pgfpathmoveto{\pgfqpoint{0.000000in}{0.000000in}}%
\pgfpathlineto{\pgfqpoint{0.000000in}{-0.048611in}}%
\pgfusepath{stroke,fill}%
}%
\begin{pgfscope}%
\pgfsys@transformshift{1.504870in}{0.500000in}%
\pgfsys@useobject{currentmarker}{}%
\end{pgfscope}%
\end{pgfscope}%
\begin{pgfscope}%
\definecolor{textcolor}{rgb}{0.000000,0.000000,0.000000}%
\pgfsetstrokecolor{textcolor}%
\pgfsetfillcolor{textcolor}%
\pgftext[x=1.504870in,y=0.402778in,,top]{\color{textcolor}\rmfamily\fontsize{10.000000}{12.000000}\selectfont \(\displaystyle {10}\)}%
\end{pgfscope}%
\begin{pgfscope}%
\pgfsetbuttcap%
\pgfsetroundjoin%
\definecolor{currentfill}{rgb}{0.000000,0.000000,0.000000}%
\pgfsetfillcolor{currentfill}%
\pgfsetlinewidth{0.803000pt}%
\definecolor{currentstroke}{rgb}{0.000000,0.000000,0.000000}%
\pgfsetstrokecolor{currentstroke}%
\pgfsetdash{}{0pt}%
\pgfsys@defobject{currentmarker}{\pgfqpoint{0.000000in}{-0.048611in}}{\pgfqpoint{0.000000in}{0.000000in}}{%
\pgfpathmoveto{\pgfqpoint{0.000000in}{0.000000in}}%
\pgfpathlineto{\pgfqpoint{0.000000in}{-0.048611in}}%
\pgfusepath{stroke,fill}%
}%
\begin{pgfscope}%
\pgfsys@transformshift{2.108766in}{0.500000in}%
\pgfsys@useobject{currentmarker}{}%
\end{pgfscope}%
\end{pgfscope}%
\begin{pgfscope}%
\definecolor{textcolor}{rgb}{0.000000,0.000000,0.000000}%
\pgfsetstrokecolor{textcolor}%
\pgfsetfillcolor{textcolor}%
\pgftext[x=2.108766in,y=0.402778in,,top]{\color{textcolor}\rmfamily\fontsize{10.000000}{12.000000}\selectfont \(\displaystyle {20}\)}%
\end{pgfscope}%
\begin{pgfscope}%
\pgfsetbuttcap%
\pgfsetroundjoin%
\definecolor{currentfill}{rgb}{0.000000,0.000000,0.000000}%
\pgfsetfillcolor{currentfill}%
\pgfsetlinewidth{0.803000pt}%
\definecolor{currentstroke}{rgb}{0.000000,0.000000,0.000000}%
\pgfsetstrokecolor{currentstroke}%
\pgfsetdash{}{0pt}%
\pgfsys@defobject{currentmarker}{\pgfqpoint{0.000000in}{-0.048611in}}{\pgfqpoint{0.000000in}{0.000000in}}{%
\pgfpathmoveto{\pgfqpoint{0.000000in}{0.000000in}}%
\pgfpathlineto{\pgfqpoint{0.000000in}{-0.048611in}}%
\pgfusepath{stroke,fill}%
}%
\begin{pgfscope}%
\pgfsys@transformshift{2.712662in}{0.500000in}%
\pgfsys@useobject{currentmarker}{}%
\end{pgfscope}%
\end{pgfscope}%
\begin{pgfscope}%
\definecolor{textcolor}{rgb}{0.000000,0.000000,0.000000}%
\pgfsetstrokecolor{textcolor}%
\pgfsetfillcolor{textcolor}%
\pgftext[x=2.712662in,y=0.402778in,,top]{\color{textcolor}\rmfamily\fontsize{10.000000}{12.000000}\selectfont \(\displaystyle {30}\)}%
\end{pgfscope}%
\begin{pgfscope}%
\pgfsetbuttcap%
\pgfsetroundjoin%
\definecolor{currentfill}{rgb}{0.000000,0.000000,0.000000}%
\pgfsetfillcolor{currentfill}%
\pgfsetlinewidth{0.803000pt}%
\definecolor{currentstroke}{rgb}{0.000000,0.000000,0.000000}%
\pgfsetstrokecolor{currentstroke}%
\pgfsetdash{}{0pt}%
\pgfsys@defobject{currentmarker}{\pgfqpoint{0.000000in}{-0.048611in}}{\pgfqpoint{0.000000in}{0.000000in}}{%
\pgfpathmoveto{\pgfqpoint{0.000000in}{0.000000in}}%
\pgfpathlineto{\pgfqpoint{0.000000in}{-0.048611in}}%
\pgfusepath{stroke,fill}%
}%
\begin{pgfscope}%
\pgfsys@transformshift{3.316558in}{0.500000in}%
\pgfsys@useobject{currentmarker}{}%
\end{pgfscope}%
\end{pgfscope}%
\begin{pgfscope}%
\definecolor{textcolor}{rgb}{0.000000,0.000000,0.000000}%
\pgfsetstrokecolor{textcolor}%
\pgfsetfillcolor{textcolor}%
\pgftext[x=3.316558in,y=0.402778in,,top]{\color{textcolor}\rmfamily\fontsize{10.000000}{12.000000}\selectfont \(\displaystyle {40}\)}%
\end{pgfscope}%
\begin{pgfscope}%
\pgfsetbuttcap%
\pgfsetroundjoin%
\definecolor{currentfill}{rgb}{0.000000,0.000000,0.000000}%
\pgfsetfillcolor{currentfill}%
\pgfsetlinewidth{0.803000pt}%
\definecolor{currentstroke}{rgb}{0.000000,0.000000,0.000000}%
\pgfsetstrokecolor{currentstroke}%
\pgfsetdash{}{0pt}%
\pgfsys@defobject{currentmarker}{\pgfqpoint{0.000000in}{-0.048611in}}{\pgfqpoint{0.000000in}{0.000000in}}{%
\pgfpathmoveto{\pgfqpoint{0.000000in}{0.000000in}}%
\pgfpathlineto{\pgfqpoint{0.000000in}{-0.048611in}}%
\pgfusepath{stroke,fill}%
}%
\begin{pgfscope}%
\pgfsys@transformshift{3.920455in}{0.500000in}%
\pgfsys@useobject{currentmarker}{}%
\end{pgfscope}%
\end{pgfscope}%
\begin{pgfscope}%
\definecolor{textcolor}{rgb}{0.000000,0.000000,0.000000}%
\pgfsetstrokecolor{textcolor}%
\pgfsetfillcolor{textcolor}%
\pgftext[x=3.920455in,y=0.402778in,,top]{\color{textcolor}\rmfamily\fontsize{10.000000}{12.000000}\selectfont \(\displaystyle {50}\)}%
\end{pgfscope}%
\begin{pgfscope}%
\pgfsetbuttcap%
\pgfsetroundjoin%
\definecolor{currentfill}{rgb}{0.000000,0.000000,0.000000}%
\pgfsetfillcolor{currentfill}%
\pgfsetlinewidth{0.803000pt}%
\definecolor{currentstroke}{rgb}{0.000000,0.000000,0.000000}%
\pgfsetstrokecolor{currentstroke}%
\pgfsetdash{}{0pt}%
\pgfsys@defobject{currentmarker}{\pgfqpoint{0.000000in}{-0.048611in}}{\pgfqpoint{0.000000in}{0.000000in}}{%
\pgfpathmoveto{\pgfqpoint{0.000000in}{0.000000in}}%
\pgfpathlineto{\pgfqpoint{0.000000in}{-0.048611in}}%
\pgfusepath{stroke,fill}%
}%
\begin{pgfscope}%
\pgfsys@transformshift{4.524351in}{0.500000in}%
\pgfsys@useobject{currentmarker}{}%
\end{pgfscope}%
\end{pgfscope}%
\begin{pgfscope}%
\definecolor{textcolor}{rgb}{0.000000,0.000000,0.000000}%
\pgfsetstrokecolor{textcolor}%
\pgfsetfillcolor{textcolor}%
\pgftext[x=4.524351in,y=0.402778in,,top]{\color{textcolor}\rmfamily\fontsize{10.000000}{12.000000}\selectfont \(\displaystyle {60}\)}%
\end{pgfscope}%
\begin{pgfscope}%
\pgfsetbuttcap%
\pgfsetroundjoin%
\definecolor{currentfill}{rgb}{0.000000,0.000000,0.000000}%
\pgfsetfillcolor{currentfill}%
\pgfsetlinewidth{0.803000pt}%
\definecolor{currentstroke}{rgb}{0.000000,0.000000,0.000000}%
\pgfsetstrokecolor{currentstroke}%
\pgfsetdash{}{0pt}%
\pgfsys@defobject{currentmarker}{\pgfqpoint{0.000000in}{-0.048611in}}{\pgfqpoint{0.000000in}{0.000000in}}{%
\pgfpathmoveto{\pgfqpoint{0.000000in}{0.000000in}}%
\pgfpathlineto{\pgfqpoint{0.000000in}{-0.048611in}}%
\pgfusepath{stroke,fill}%
}%
\begin{pgfscope}%
\pgfsys@transformshift{5.128247in}{0.500000in}%
\pgfsys@useobject{currentmarker}{}%
\end{pgfscope}%
\end{pgfscope}%
\begin{pgfscope}%
\definecolor{textcolor}{rgb}{0.000000,0.000000,0.000000}%
\pgfsetstrokecolor{textcolor}%
\pgfsetfillcolor{textcolor}%
\pgftext[x=5.128247in,y=0.402778in,,top]{\color{textcolor}\rmfamily\fontsize{10.000000}{12.000000}\selectfont \(\displaystyle {70}\)}%
\end{pgfscope}%
\begin{pgfscope}%
\definecolor{textcolor}{rgb}{0.000000,0.000000,0.000000}%
\pgfsetstrokecolor{textcolor}%
\pgfsetfillcolor{textcolor}%
\pgftext[x=3.075000in,y=0.223889in,,top]{\color{textcolor}\rmfamily\fontsize{10.000000}{12.000000}\selectfont Number of Sources}%
\end{pgfscope}%
\begin{pgfscope}%
\pgfsetbuttcap%
\pgfsetroundjoin%
\definecolor{currentfill}{rgb}{0.000000,0.000000,0.000000}%
\pgfsetfillcolor{currentfill}%
\pgfsetlinewidth{0.803000pt}%
\definecolor{currentstroke}{rgb}{0.000000,0.000000,0.000000}%
\pgfsetstrokecolor{currentstroke}%
\pgfsetdash{}{0pt}%
\pgfsys@defobject{currentmarker}{\pgfqpoint{-0.048611in}{0.000000in}}{\pgfqpoint{0.000000in}{0.000000in}}{%
\pgfpathmoveto{\pgfqpoint{0.000000in}{0.000000in}}%
\pgfpathlineto{\pgfqpoint{-0.048611in}{0.000000in}}%
\pgfusepath{stroke,fill}%
}%
\begin{pgfscope}%
\pgfsys@transformshift{0.750000in}{0.637273in}%
\pgfsys@useobject{currentmarker}{}%
\end{pgfscope}%
\end{pgfscope}%
\begin{pgfscope}%
\definecolor{textcolor}{rgb}{0.000000,0.000000,0.000000}%
\pgfsetstrokecolor{textcolor}%
\pgfsetfillcolor{textcolor}%
\pgftext[x=0.583333in, y=0.589078in, left, base]{\color{textcolor}\rmfamily\fontsize{10.000000}{12.000000}\selectfont \(\displaystyle {0}\)}%
\end{pgfscope}%
\begin{pgfscope}%
\pgfsetbuttcap%
\pgfsetroundjoin%
\definecolor{currentfill}{rgb}{0.000000,0.000000,0.000000}%
\pgfsetfillcolor{currentfill}%
\pgfsetlinewidth{0.803000pt}%
\definecolor{currentstroke}{rgb}{0.000000,0.000000,0.000000}%
\pgfsetstrokecolor{currentstroke}%
\pgfsetdash{}{0pt}%
\pgfsys@defobject{currentmarker}{\pgfqpoint{-0.048611in}{0.000000in}}{\pgfqpoint{0.000000in}{0.000000in}}{%
\pgfpathmoveto{\pgfqpoint{0.000000in}{0.000000in}}%
\pgfpathlineto{\pgfqpoint{-0.048611in}{0.000000in}}%
\pgfusepath{stroke,fill}%
}%
\begin{pgfscope}%
\pgfsys@transformshift{0.750000in}{1.021790in}%
\pgfsys@useobject{currentmarker}{}%
\end{pgfscope}%
\end{pgfscope}%
\begin{pgfscope}%
\definecolor{textcolor}{rgb}{0.000000,0.000000,0.000000}%
\pgfsetstrokecolor{textcolor}%
\pgfsetfillcolor{textcolor}%
\pgftext[x=0.444444in, y=0.973596in, left, base]{\color{textcolor}\rmfamily\fontsize{10.000000}{12.000000}\selectfont \(\displaystyle {100}\)}%
\end{pgfscope}%
\begin{pgfscope}%
\pgfsetbuttcap%
\pgfsetroundjoin%
\definecolor{currentfill}{rgb}{0.000000,0.000000,0.000000}%
\pgfsetfillcolor{currentfill}%
\pgfsetlinewidth{0.803000pt}%
\definecolor{currentstroke}{rgb}{0.000000,0.000000,0.000000}%
\pgfsetstrokecolor{currentstroke}%
\pgfsetdash{}{0pt}%
\pgfsys@defobject{currentmarker}{\pgfqpoint{-0.048611in}{0.000000in}}{\pgfqpoint{0.000000in}{0.000000in}}{%
\pgfpathmoveto{\pgfqpoint{0.000000in}{0.000000in}}%
\pgfpathlineto{\pgfqpoint{-0.048611in}{0.000000in}}%
\pgfusepath{stroke,fill}%
}%
\begin{pgfscope}%
\pgfsys@transformshift{0.750000in}{1.406308in}%
\pgfsys@useobject{currentmarker}{}%
\end{pgfscope}%
\end{pgfscope}%
\begin{pgfscope}%
\definecolor{textcolor}{rgb}{0.000000,0.000000,0.000000}%
\pgfsetstrokecolor{textcolor}%
\pgfsetfillcolor{textcolor}%
\pgftext[x=0.444444in, y=1.358113in, left, base]{\color{textcolor}\rmfamily\fontsize{10.000000}{12.000000}\selectfont \(\displaystyle {200}\)}%
\end{pgfscope}%
\begin{pgfscope}%
\pgfsetbuttcap%
\pgfsetroundjoin%
\definecolor{currentfill}{rgb}{0.000000,0.000000,0.000000}%
\pgfsetfillcolor{currentfill}%
\pgfsetlinewidth{0.803000pt}%
\definecolor{currentstroke}{rgb}{0.000000,0.000000,0.000000}%
\pgfsetstrokecolor{currentstroke}%
\pgfsetdash{}{0pt}%
\pgfsys@defobject{currentmarker}{\pgfqpoint{-0.048611in}{0.000000in}}{\pgfqpoint{0.000000in}{0.000000in}}{%
\pgfpathmoveto{\pgfqpoint{0.000000in}{0.000000in}}%
\pgfpathlineto{\pgfqpoint{-0.048611in}{0.000000in}}%
\pgfusepath{stroke,fill}%
}%
\begin{pgfscope}%
\pgfsys@transformshift{0.750000in}{1.790825in}%
\pgfsys@useobject{currentmarker}{}%
\end{pgfscope}%
\end{pgfscope}%
\begin{pgfscope}%
\definecolor{textcolor}{rgb}{0.000000,0.000000,0.000000}%
\pgfsetstrokecolor{textcolor}%
\pgfsetfillcolor{textcolor}%
\pgftext[x=0.444444in, y=1.742631in, left, base]{\color{textcolor}\rmfamily\fontsize{10.000000}{12.000000}\selectfont \(\displaystyle {300}\)}%
\end{pgfscope}%
\begin{pgfscope}%
\pgfsetbuttcap%
\pgfsetroundjoin%
\definecolor{currentfill}{rgb}{0.000000,0.000000,0.000000}%
\pgfsetfillcolor{currentfill}%
\pgfsetlinewidth{0.803000pt}%
\definecolor{currentstroke}{rgb}{0.000000,0.000000,0.000000}%
\pgfsetstrokecolor{currentstroke}%
\pgfsetdash{}{0pt}%
\pgfsys@defobject{currentmarker}{\pgfqpoint{-0.048611in}{0.000000in}}{\pgfqpoint{0.000000in}{0.000000in}}{%
\pgfpathmoveto{\pgfqpoint{0.000000in}{0.000000in}}%
\pgfpathlineto{\pgfqpoint{-0.048611in}{0.000000in}}%
\pgfusepath{stroke,fill}%
}%
\begin{pgfscope}%
\pgfsys@transformshift{0.750000in}{2.175343in}%
\pgfsys@useobject{currentmarker}{}%
\end{pgfscope}%
\end{pgfscope}%
\begin{pgfscope}%
\definecolor{textcolor}{rgb}{0.000000,0.000000,0.000000}%
\pgfsetstrokecolor{textcolor}%
\pgfsetfillcolor{textcolor}%
\pgftext[x=0.444444in, y=2.127148in, left, base]{\color{textcolor}\rmfamily\fontsize{10.000000}{12.000000}\selectfont \(\displaystyle {400}\)}%
\end{pgfscope}%
\begin{pgfscope}%
\pgfsetbuttcap%
\pgfsetroundjoin%
\definecolor{currentfill}{rgb}{0.000000,0.000000,0.000000}%
\pgfsetfillcolor{currentfill}%
\pgfsetlinewidth{0.803000pt}%
\definecolor{currentstroke}{rgb}{0.000000,0.000000,0.000000}%
\pgfsetstrokecolor{currentstroke}%
\pgfsetdash{}{0pt}%
\pgfsys@defobject{currentmarker}{\pgfqpoint{-0.048611in}{0.000000in}}{\pgfqpoint{0.000000in}{0.000000in}}{%
\pgfpathmoveto{\pgfqpoint{0.000000in}{0.000000in}}%
\pgfpathlineto{\pgfqpoint{-0.048611in}{0.000000in}}%
\pgfusepath{stroke,fill}%
}%
\begin{pgfscope}%
\pgfsys@transformshift{0.750000in}{2.559860in}%
\pgfsys@useobject{currentmarker}{}%
\end{pgfscope}%
\end{pgfscope}%
\begin{pgfscope}%
\definecolor{textcolor}{rgb}{0.000000,0.000000,0.000000}%
\pgfsetstrokecolor{textcolor}%
\pgfsetfillcolor{textcolor}%
\pgftext[x=0.444444in, y=2.511665in, left, base]{\color{textcolor}\rmfamily\fontsize{10.000000}{12.000000}\selectfont \(\displaystyle {500}\)}%
\end{pgfscope}%
\begin{pgfscope}%
\pgfsetbuttcap%
\pgfsetroundjoin%
\definecolor{currentfill}{rgb}{0.000000,0.000000,0.000000}%
\pgfsetfillcolor{currentfill}%
\pgfsetlinewidth{0.803000pt}%
\definecolor{currentstroke}{rgb}{0.000000,0.000000,0.000000}%
\pgfsetstrokecolor{currentstroke}%
\pgfsetdash{}{0pt}%
\pgfsys@defobject{currentmarker}{\pgfqpoint{-0.048611in}{0.000000in}}{\pgfqpoint{0.000000in}{0.000000in}}{%
\pgfpathmoveto{\pgfqpoint{0.000000in}{0.000000in}}%
\pgfpathlineto{\pgfqpoint{-0.048611in}{0.000000in}}%
\pgfusepath{stroke,fill}%
}%
\begin{pgfscope}%
\pgfsys@transformshift{0.750000in}{2.944377in}%
\pgfsys@useobject{currentmarker}{}%
\end{pgfscope}%
\end{pgfscope}%
\begin{pgfscope}%
\definecolor{textcolor}{rgb}{0.000000,0.000000,0.000000}%
\pgfsetstrokecolor{textcolor}%
\pgfsetfillcolor{textcolor}%
\pgftext[x=0.444444in, y=2.896183in, left, base]{\color{textcolor}\rmfamily\fontsize{10.000000}{12.000000}\selectfont \(\displaystyle {600}\)}%
\end{pgfscope}%
\begin{pgfscope}%
\pgfsetbuttcap%
\pgfsetroundjoin%
\definecolor{currentfill}{rgb}{0.000000,0.000000,0.000000}%
\pgfsetfillcolor{currentfill}%
\pgfsetlinewidth{0.803000pt}%
\definecolor{currentstroke}{rgb}{0.000000,0.000000,0.000000}%
\pgfsetstrokecolor{currentstroke}%
\pgfsetdash{}{0pt}%
\pgfsys@defobject{currentmarker}{\pgfqpoint{-0.048611in}{0.000000in}}{\pgfqpoint{0.000000in}{0.000000in}}{%
\pgfpathmoveto{\pgfqpoint{0.000000in}{0.000000in}}%
\pgfpathlineto{\pgfqpoint{-0.048611in}{0.000000in}}%
\pgfusepath{stroke,fill}%
}%
\begin{pgfscope}%
\pgfsys@transformshift{0.750000in}{3.328895in}%
\pgfsys@useobject{currentmarker}{}%
\end{pgfscope}%
\end{pgfscope}%
\begin{pgfscope}%
\definecolor{textcolor}{rgb}{0.000000,0.000000,0.000000}%
\pgfsetstrokecolor{textcolor}%
\pgfsetfillcolor{textcolor}%
\pgftext[x=0.444444in, y=3.280700in, left, base]{\color{textcolor}\rmfamily\fontsize{10.000000}{12.000000}\selectfont \(\displaystyle {700}\)}%
\end{pgfscope}%
\begin{pgfscope}%
\definecolor{textcolor}{rgb}{0.000000,0.000000,0.000000}%
\pgfsetstrokecolor{textcolor}%
\pgfsetfillcolor{textcolor}%
\pgftext[x=0.388888in,y=2.010000in,,bottom,rotate=90.000000]{\color{textcolor}\rmfamily\fontsize{10.000000}{12.000000}\selectfont Dataflow Time}%
\end{pgfscope}%
\begin{pgfscope}%
\pgfsetrectcap%
\pgfsetmiterjoin%
\pgfsetlinewidth{0.803000pt}%
\definecolor{currentstroke}{rgb}{0.000000,0.000000,0.000000}%
\pgfsetstrokecolor{currentstroke}%
\pgfsetdash{}{0pt}%
\pgfpathmoveto{\pgfqpoint{0.750000in}{0.500000in}}%
\pgfpathlineto{\pgfqpoint{0.750000in}{3.520000in}}%
\pgfusepath{stroke}%
\end{pgfscope}%
\begin{pgfscope}%
\pgfsetrectcap%
\pgfsetmiterjoin%
\pgfsetlinewidth{0.803000pt}%
\definecolor{currentstroke}{rgb}{0.000000,0.000000,0.000000}%
\pgfsetstrokecolor{currentstroke}%
\pgfsetdash{}{0pt}%
\pgfpathmoveto{\pgfqpoint{5.400000in}{0.500000in}}%
\pgfpathlineto{\pgfqpoint{5.400000in}{3.520000in}}%
\pgfusepath{stroke}%
\end{pgfscope}%
\begin{pgfscope}%
\pgfsetrectcap%
\pgfsetmiterjoin%
\pgfsetlinewidth{0.803000pt}%
\definecolor{currentstroke}{rgb}{0.000000,0.000000,0.000000}%
\pgfsetstrokecolor{currentstroke}%
\pgfsetdash{}{0pt}%
\pgfpathmoveto{\pgfqpoint{0.750000in}{0.500000in}}%
\pgfpathlineto{\pgfqpoint{5.400000in}{0.500000in}}%
\pgfusepath{stroke}%
\end{pgfscope}%
\begin{pgfscope}%
\pgfsetrectcap%
\pgfsetmiterjoin%
\pgfsetlinewidth{0.803000pt}%
\definecolor{currentstroke}{rgb}{0.000000,0.000000,0.000000}%
\pgfsetstrokecolor{currentstroke}%
\pgfsetdash{}{0pt}%
\pgfpathmoveto{\pgfqpoint{0.750000in}{3.520000in}}%
\pgfpathlineto{\pgfqpoint{5.400000in}{3.520000in}}%
\pgfusepath{stroke}%
\end{pgfscope}%
\begin{pgfscope}%
\definecolor{textcolor}{rgb}{0.000000,0.000000,0.000000}%
\pgfsetstrokecolor{textcolor}%
\pgfsetfillcolor{textcolor}%
\pgftext[x=3.075000in,y=3.603333in,,base]{\color{textcolor}\rmfamily\fontsize{12.000000}{14.400000}\selectfont Backwards}%
\end{pgfscope}%
\begin{pgfscope}%
\pgfsetbuttcap%
\pgfsetmiterjoin%
\definecolor{currentfill}{rgb}{1.000000,1.000000,1.000000}%
\pgfsetfillcolor{currentfill}%
\pgfsetfillopacity{0.800000}%
\pgfsetlinewidth{1.003750pt}%
\definecolor{currentstroke}{rgb}{0.800000,0.800000,0.800000}%
\pgfsetstrokecolor{currentstroke}%
\pgfsetstrokeopacity{0.800000}%
\pgfsetdash{}{0pt}%
\pgfpathmoveto{\pgfqpoint{3.793194in}{1.697986in}}%
\pgfpathlineto{\pgfqpoint{5.302778in}{1.697986in}}%
\pgfpathquadraticcurveto{\pgfqpoint{5.330556in}{1.697986in}}{\pgfqpoint{5.330556in}{1.725764in}}%
\pgfpathlineto{\pgfqpoint{5.330556in}{2.294236in}}%
\pgfpathquadraticcurveto{\pgfqpoint{5.330556in}{2.322014in}}{\pgfqpoint{5.302778in}{2.322014in}}%
\pgfpathlineto{\pgfqpoint{3.793194in}{2.322014in}}%
\pgfpathquadraticcurveto{\pgfqpoint{3.765417in}{2.322014in}}{\pgfqpoint{3.765417in}{2.294236in}}%
\pgfpathlineto{\pgfqpoint{3.765417in}{1.725764in}}%
\pgfpathquadraticcurveto{\pgfqpoint{3.765417in}{1.697986in}}{\pgfqpoint{3.793194in}{1.697986in}}%
\pgfpathclose%
\pgfusepath{stroke,fill}%
\end{pgfscope}%
\begin{pgfscope}%
\pgfsetbuttcap%
\pgfsetroundjoin%
\definecolor{currentfill}{rgb}{0.121569,0.466667,0.705882}%
\pgfsetfillcolor{currentfill}%
\pgfsetlinewidth{1.003750pt}%
\definecolor{currentstroke}{rgb}{0.121569,0.466667,0.705882}%
\pgfsetstrokecolor{currentstroke}%
\pgfsetdash{}{0pt}%
\pgfsys@defobject{currentmarker}{\pgfqpoint{-0.034722in}{-0.034722in}}{\pgfqpoint{0.034722in}{0.034722in}}{%
\pgfpathmoveto{\pgfqpoint{0.000000in}{-0.034722in}}%
\pgfpathcurveto{\pgfqpoint{0.009208in}{-0.034722in}}{\pgfqpoint{0.018041in}{-0.031064in}}{\pgfqpoint{0.024552in}{-0.024552in}}%
\pgfpathcurveto{\pgfqpoint{0.031064in}{-0.018041in}}{\pgfqpoint{0.034722in}{-0.009208in}}{\pgfqpoint{0.034722in}{0.000000in}}%
\pgfpathcurveto{\pgfqpoint{0.034722in}{0.009208in}}{\pgfqpoint{0.031064in}{0.018041in}}{\pgfqpoint{0.024552in}{0.024552in}}%
\pgfpathcurveto{\pgfqpoint{0.018041in}{0.031064in}}{\pgfqpoint{0.009208in}{0.034722in}}{\pgfqpoint{0.000000in}{0.034722in}}%
\pgfpathcurveto{\pgfqpoint{-0.009208in}{0.034722in}}{\pgfqpoint{-0.018041in}{0.031064in}}{\pgfqpoint{-0.024552in}{0.024552in}}%
\pgfpathcurveto{\pgfqpoint{-0.031064in}{0.018041in}}{\pgfqpoint{-0.034722in}{0.009208in}}{\pgfqpoint{-0.034722in}{0.000000in}}%
\pgfpathcurveto{\pgfqpoint{-0.034722in}{-0.009208in}}{\pgfqpoint{-0.031064in}{-0.018041in}}{\pgfqpoint{-0.024552in}{-0.024552in}}%
\pgfpathcurveto{\pgfqpoint{-0.018041in}{-0.031064in}}{\pgfqpoint{-0.009208in}{-0.034722in}}{\pgfqpoint{0.000000in}{-0.034722in}}%
\pgfpathclose%
\pgfusepath{stroke,fill}%
}%
\begin{pgfscope}%
\pgfsys@transformshift{3.959861in}{2.217847in}%
\pgfsys@useobject{currentmarker}{}%
\end{pgfscope}%
\end{pgfscope}%
\begin{pgfscope}%
\definecolor{textcolor}{rgb}{0.000000,0.000000,0.000000}%
\pgfsetstrokecolor{textcolor}%
\pgfsetfillcolor{textcolor}%
\pgftext[x=4.209861in,y=2.169236in,left,base]{\color{textcolor}\rmfamily\fontsize{10.000000}{12.000000}\selectfont No Timeout}%
\end{pgfscope}%
\begin{pgfscope}%
\pgfsetbuttcap%
\pgfsetroundjoin%
\definecolor{currentfill}{rgb}{1.000000,0.498039,0.054902}%
\pgfsetfillcolor{currentfill}%
\pgfsetlinewidth{1.003750pt}%
\definecolor{currentstroke}{rgb}{1.000000,0.498039,0.054902}%
\pgfsetstrokecolor{currentstroke}%
\pgfsetdash{}{0pt}%
\pgfsys@defobject{currentmarker}{\pgfqpoint{-0.034722in}{-0.034722in}}{\pgfqpoint{0.034722in}{0.034722in}}{%
\pgfpathmoveto{\pgfqpoint{0.000000in}{-0.034722in}}%
\pgfpathcurveto{\pgfqpoint{0.009208in}{-0.034722in}}{\pgfqpoint{0.018041in}{-0.031064in}}{\pgfqpoint{0.024552in}{-0.024552in}}%
\pgfpathcurveto{\pgfqpoint{0.031064in}{-0.018041in}}{\pgfqpoint{0.034722in}{-0.009208in}}{\pgfqpoint{0.034722in}{0.000000in}}%
\pgfpathcurveto{\pgfqpoint{0.034722in}{0.009208in}}{\pgfqpoint{0.031064in}{0.018041in}}{\pgfqpoint{0.024552in}{0.024552in}}%
\pgfpathcurveto{\pgfqpoint{0.018041in}{0.031064in}}{\pgfqpoint{0.009208in}{0.034722in}}{\pgfqpoint{0.000000in}{0.034722in}}%
\pgfpathcurveto{\pgfqpoint{-0.009208in}{0.034722in}}{\pgfqpoint{-0.018041in}{0.031064in}}{\pgfqpoint{-0.024552in}{0.024552in}}%
\pgfpathcurveto{\pgfqpoint{-0.031064in}{0.018041in}}{\pgfqpoint{-0.034722in}{0.009208in}}{\pgfqpoint{-0.034722in}{0.000000in}}%
\pgfpathcurveto{\pgfqpoint{-0.034722in}{-0.009208in}}{\pgfqpoint{-0.031064in}{-0.018041in}}{\pgfqpoint{-0.024552in}{-0.024552in}}%
\pgfpathcurveto{\pgfqpoint{-0.018041in}{-0.031064in}}{\pgfqpoint{-0.009208in}{-0.034722in}}{\pgfqpoint{0.000000in}{-0.034722in}}%
\pgfpathclose%
\pgfusepath{stroke,fill}%
}%
\begin{pgfscope}%
\pgfsys@transformshift{3.959861in}{2.024236in}%
\pgfsys@useobject{currentmarker}{}%
\end{pgfscope}%
\end{pgfscope}%
\begin{pgfscope}%
\definecolor{textcolor}{rgb}{0.000000,0.000000,0.000000}%
\pgfsetstrokecolor{textcolor}%
\pgfsetfillcolor{textcolor}%
\pgftext[x=4.209861in,y=1.975625in,left,base]{\color{textcolor}\rmfamily\fontsize{10.000000}{12.000000}\selectfont Time Timeout}%
\end{pgfscope}%
\begin{pgfscope}%
\pgfsetbuttcap%
\pgfsetroundjoin%
\definecolor{currentfill}{rgb}{0.839216,0.152941,0.156863}%
\pgfsetfillcolor{currentfill}%
\pgfsetlinewidth{1.003750pt}%
\definecolor{currentstroke}{rgb}{0.839216,0.152941,0.156863}%
\pgfsetstrokecolor{currentstroke}%
\pgfsetdash{}{0pt}%
\pgfsys@defobject{currentmarker}{\pgfqpoint{-0.034722in}{-0.034722in}}{\pgfqpoint{0.034722in}{0.034722in}}{%
\pgfpathmoveto{\pgfqpoint{0.000000in}{-0.034722in}}%
\pgfpathcurveto{\pgfqpoint{0.009208in}{-0.034722in}}{\pgfqpoint{0.018041in}{-0.031064in}}{\pgfqpoint{0.024552in}{-0.024552in}}%
\pgfpathcurveto{\pgfqpoint{0.031064in}{-0.018041in}}{\pgfqpoint{0.034722in}{-0.009208in}}{\pgfqpoint{0.034722in}{0.000000in}}%
\pgfpathcurveto{\pgfqpoint{0.034722in}{0.009208in}}{\pgfqpoint{0.031064in}{0.018041in}}{\pgfqpoint{0.024552in}{0.024552in}}%
\pgfpathcurveto{\pgfqpoint{0.018041in}{0.031064in}}{\pgfqpoint{0.009208in}{0.034722in}}{\pgfqpoint{0.000000in}{0.034722in}}%
\pgfpathcurveto{\pgfqpoint{-0.009208in}{0.034722in}}{\pgfqpoint{-0.018041in}{0.031064in}}{\pgfqpoint{-0.024552in}{0.024552in}}%
\pgfpathcurveto{\pgfqpoint{-0.031064in}{0.018041in}}{\pgfqpoint{-0.034722in}{0.009208in}}{\pgfqpoint{-0.034722in}{0.000000in}}%
\pgfpathcurveto{\pgfqpoint{-0.034722in}{-0.009208in}}{\pgfqpoint{-0.031064in}{-0.018041in}}{\pgfqpoint{-0.024552in}{-0.024552in}}%
\pgfpathcurveto{\pgfqpoint{-0.018041in}{-0.031064in}}{\pgfqpoint{-0.009208in}{-0.034722in}}{\pgfqpoint{0.000000in}{-0.034722in}}%
\pgfpathclose%
\pgfusepath{stroke,fill}%
}%
\begin{pgfscope}%
\pgfsys@transformshift{3.959861in}{1.830625in}%
\pgfsys@useobject{currentmarker}{}%
\end{pgfscope}%
\end{pgfscope}%
\begin{pgfscope}%
\definecolor{textcolor}{rgb}{0.000000,0.000000,0.000000}%
\pgfsetstrokecolor{textcolor}%
\pgfsetfillcolor{textcolor}%
\pgftext[x=4.209861in,y=1.782014in,left,base]{\color{textcolor}\rmfamily\fontsize{10.000000}{12.000000}\selectfont Memory Timeout}%
\end{pgfscope}%
\end{pgfpicture}%
\makeatother%
\endgroup%

            }
        \end{subfigure}
        \qquad
        \begin{subfigure}[]{0.45\textwidth}
            \centering
            \resizebox{\columnwidth}{!}{
                %% Creator: Matplotlib, PGF backend
%%
%% To include the figure in your LaTeX document, write
%%   \input{<filename>.pgf}
%%
%% Make sure the required packages are loaded in your preamble
%%   \usepackage{pgf}
%%
%% and, on pdftex
%%   \usepackage[utf8]{inputenc}\DeclareUnicodeCharacter{2212}{-}
%%
%% or, on luatex and xetex
%%   \usepackage{unicode-math}
%%
%% Figures using additional raster images can only be included by \input if
%% they are in the same directory as the main LaTeX file. For loading figures
%% from other directories you can use the `import` package
%%   \usepackage{import}
%%
%% and then include the figures with
%%   \import{<path to file>}{<filename>.pgf}
%%
%% Matplotlib used the following preamble
%%   \usepackage{amsmath}
%%   \usepackage{fontspec}
%%
\begingroup%
\makeatletter%
\begin{pgfpicture}%
\pgfpathrectangle{\pgfpointorigin}{\pgfqpoint{6.000000in}{4.000000in}}%
\pgfusepath{use as bounding box, clip}%
\begin{pgfscope}%
\pgfsetbuttcap%
\pgfsetmiterjoin%
\definecolor{currentfill}{rgb}{1.000000,1.000000,1.000000}%
\pgfsetfillcolor{currentfill}%
\pgfsetlinewidth{0.000000pt}%
\definecolor{currentstroke}{rgb}{1.000000,1.000000,1.000000}%
\pgfsetstrokecolor{currentstroke}%
\pgfsetdash{}{0pt}%
\pgfpathmoveto{\pgfqpoint{0.000000in}{0.000000in}}%
\pgfpathlineto{\pgfqpoint{6.000000in}{0.000000in}}%
\pgfpathlineto{\pgfqpoint{6.000000in}{4.000000in}}%
\pgfpathlineto{\pgfqpoint{0.000000in}{4.000000in}}%
\pgfpathclose%
\pgfusepath{fill}%
\end{pgfscope}%
\begin{pgfscope}%
\pgfsetbuttcap%
\pgfsetmiterjoin%
\definecolor{currentfill}{rgb}{1.000000,1.000000,1.000000}%
\pgfsetfillcolor{currentfill}%
\pgfsetlinewidth{0.000000pt}%
\definecolor{currentstroke}{rgb}{0.000000,0.000000,0.000000}%
\pgfsetstrokecolor{currentstroke}%
\pgfsetstrokeopacity{0.000000}%
\pgfsetdash{}{0pt}%
\pgfpathmoveto{\pgfqpoint{0.648703in}{0.548769in}}%
\pgfpathlineto{\pgfqpoint{5.850000in}{0.548769in}}%
\pgfpathlineto{\pgfqpoint{5.850000in}{3.651359in}}%
\pgfpathlineto{\pgfqpoint{0.648703in}{3.651359in}}%
\pgfpathclose%
\pgfusepath{fill}%
\end{pgfscope}%
\begin{pgfscope}%
\pgfpathrectangle{\pgfqpoint{0.648703in}{0.548769in}}{\pgfqpoint{5.201297in}{3.102590in}}%
\pgfusepath{clip}%
\pgfsetbuttcap%
\pgfsetroundjoin%
\definecolor{currentfill}{rgb}{0.121569,0.466667,0.705882}%
\pgfsetfillcolor{currentfill}%
\pgfsetlinewidth{1.003750pt}%
\definecolor{currentstroke}{rgb}{0.121569,0.466667,0.705882}%
\pgfsetstrokecolor{currentstroke}%
\pgfsetdash{}{0pt}%
\pgfpathmoveto{\pgfqpoint{0.949899in}{0.673501in}}%
\pgfpathcurveto{\pgfqpoint{0.960949in}{0.673501in}}{\pgfqpoint{0.971548in}{0.677891in}}{\pgfqpoint{0.979362in}{0.685705in}}%
\pgfpathcurveto{\pgfqpoint{0.987176in}{0.693519in}}{\pgfqpoint{0.991566in}{0.704118in}}{\pgfqpoint{0.991566in}{0.715168in}}%
\pgfpathcurveto{\pgfqpoint{0.991566in}{0.726218in}}{\pgfqpoint{0.987176in}{0.736817in}}{\pgfqpoint{0.979362in}{0.744631in}}%
\pgfpathcurveto{\pgfqpoint{0.971548in}{0.752444in}}{\pgfqpoint{0.960949in}{0.756834in}}{\pgfqpoint{0.949899in}{0.756834in}}%
\pgfpathcurveto{\pgfqpoint{0.938849in}{0.756834in}}{\pgfqpoint{0.928250in}{0.752444in}}{\pgfqpoint{0.920437in}{0.744631in}}%
\pgfpathcurveto{\pgfqpoint{0.912623in}{0.736817in}}{\pgfqpoint{0.908233in}{0.726218in}}{\pgfqpoint{0.908233in}{0.715168in}}%
\pgfpathcurveto{\pgfqpoint{0.908233in}{0.704118in}}{\pgfqpoint{0.912623in}{0.693519in}}{\pgfqpoint{0.920437in}{0.685705in}}%
\pgfpathcurveto{\pgfqpoint{0.928250in}{0.677891in}}{\pgfqpoint{0.938849in}{0.673501in}}{\pgfqpoint{0.949899in}{0.673501in}}%
\pgfpathclose%
\pgfusepath{stroke,fill}%
\end{pgfscope}%
\begin{pgfscope}%
\pgfpathrectangle{\pgfqpoint{0.648703in}{0.548769in}}{\pgfqpoint{5.201297in}{3.102590in}}%
\pgfusepath{clip}%
\pgfsetbuttcap%
\pgfsetroundjoin%
\definecolor{currentfill}{rgb}{0.121569,0.466667,0.705882}%
\pgfsetfillcolor{currentfill}%
\pgfsetlinewidth{1.003750pt}%
\definecolor{currentstroke}{rgb}{0.121569,0.466667,0.705882}%
\pgfsetstrokecolor{currentstroke}%
\pgfsetdash{}{0pt}%
\pgfpathmoveto{\pgfqpoint{2.051046in}{3.176886in}}%
\pgfpathcurveto{\pgfqpoint{2.062096in}{3.176886in}}{\pgfqpoint{2.072695in}{3.181276in}}{\pgfqpoint{2.080508in}{3.189089in}}%
\pgfpathcurveto{\pgfqpoint{2.088322in}{3.196903in}}{\pgfqpoint{2.092712in}{3.207502in}}{\pgfqpoint{2.092712in}{3.218552in}}%
\pgfpathcurveto{\pgfqpoint{2.092712in}{3.229602in}}{\pgfqpoint{2.088322in}{3.240201in}}{\pgfqpoint{2.080508in}{3.248015in}}%
\pgfpathcurveto{\pgfqpoint{2.072695in}{3.255829in}}{\pgfqpoint{2.062096in}{3.260219in}}{\pgfqpoint{2.051046in}{3.260219in}}%
\pgfpathcurveto{\pgfqpoint{2.039995in}{3.260219in}}{\pgfqpoint{2.029396in}{3.255829in}}{\pgfqpoint{2.021583in}{3.248015in}}%
\pgfpathcurveto{\pgfqpoint{2.013769in}{3.240201in}}{\pgfqpoint{2.009379in}{3.229602in}}{\pgfqpoint{2.009379in}{3.218552in}}%
\pgfpathcurveto{\pgfqpoint{2.009379in}{3.207502in}}{\pgfqpoint{2.013769in}{3.196903in}}{\pgfqpoint{2.021583in}{3.189089in}}%
\pgfpathcurveto{\pgfqpoint{2.029396in}{3.181276in}}{\pgfqpoint{2.039995in}{3.176886in}}{\pgfqpoint{2.051046in}{3.176886in}}%
\pgfpathclose%
\pgfusepath{stroke,fill}%
\end{pgfscope}%
\begin{pgfscope}%
\pgfpathrectangle{\pgfqpoint{0.648703in}{0.548769in}}{\pgfqpoint{5.201297in}{3.102590in}}%
\pgfusepath{clip}%
\pgfsetbuttcap%
\pgfsetroundjoin%
\definecolor{currentfill}{rgb}{1.000000,0.498039,0.054902}%
\pgfsetfillcolor{currentfill}%
\pgfsetlinewidth{1.003750pt}%
\definecolor{currentstroke}{rgb}{1.000000,0.498039,0.054902}%
\pgfsetstrokecolor{currentstroke}%
\pgfsetdash{}{0pt}%
\pgfpathmoveto{\pgfqpoint{0.885126in}{3.198029in}}%
\pgfpathcurveto{\pgfqpoint{0.896176in}{3.198029in}}{\pgfqpoint{0.906775in}{3.202419in}}{\pgfqpoint{0.914589in}{3.210233in}}%
\pgfpathcurveto{\pgfqpoint{0.922402in}{3.218046in}}{\pgfqpoint{0.926793in}{3.228646in}}{\pgfqpoint{0.926793in}{3.239696in}}%
\pgfpathcurveto{\pgfqpoint{0.926793in}{3.250746in}}{\pgfqpoint{0.922402in}{3.261345in}}{\pgfqpoint{0.914589in}{3.269158in}}%
\pgfpathcurveto{\pgfqpoint{0.906775in}{3.276972in}}{\pgfqpoint{0.896176in}{3.281362in}}{\pgfqpoint{0.885126in}{3.281362in}}%
\pgfpathcurveto{\pgfqpoint{0.874076in}{3.281362in}}{\pgfqpoint{0.863477in}{3.276972in}}{\pgfqpoint{0.855663in}{3.269158in}}%
\pgfpathcurveto{\pgfqpoint{0.847850in}{3.261345in}}{\pgfqpoint{0.843459in}{3.250746in}}{\pgfqpoint{0.843459in}{3.239696in}}%
\pgfpathcurveto{\pgfqpoint{0.843459in}{3.228646in}}{\pgfqpoint{0.847850in}{3.218046in}}{\pgfqpoint{0.855663in}{3.210233in}}%
\pgfpathcurveto{\pgfqpoint{0.863477in}{3.202419in}}{\pgfqpoint{0.874076in}{3.198029in}}{\pgfqpoint{0.885126in}{3.198029in}}%
\pgfpathclose%
\pgfusepath{stroke,fill}%
\end{pgfscope}%
\begin{pgfscope}%
\pgfpathrectangle{\pgfqpoint{0.648703in}{0.548769in}}{\pgfqpoint{5.201297in}{3.102590in}}%
\pgfusepath{clip}%
\pgfsetbuttcap%
\pgfsetroundjoin%
\definecolor{currentfill}{rgb}{1.000000,0.498039,0.054902}%
\pgfsetfillcolor{currentfill}%
\pgfsetlinewidth{1.003750pt}%
\definecolor{currentstroke}{rgb}{1.000000,0.498039,0.054902}%
\pgfsetstrokecolor{currentstroke}%
\pgfsetdash{}{0pt}%
\pgfpathmoveto{\pgfqpoint{2.439685in}{3.185343in}}%
\pgfpathcurveto{\pgfqpoint{2.450735in}{3.185343in}}{\pgfqpoint{2.461335in}{3.189733in}}{\pgfqpoint{2.469148in}{3.197547in}}%
\pgfpathcurveto{\pgfqpoint{2.476962in}{3.205360in}}{\pgfqpoint{2.481352in}{3.215959in}}{\pgfqpoint{2.481352in}{3.227010in}}%
\pgfpathcurveto{\pgfqpoint{2.481352in}{3.238060in}}{\pgfqpoint{2.476962in}{3.248659in}}{\pgfqpoint{2.469148in}{3.256472in}}%
\pgfpathcurveto{\pgfqpoint{2.461335in}{3.264286in}}{\pgfqpoint{2.450735in}{3.268676in}}{\pgfqpoint{2.439685in}{3.268676in}}%
\pgfpathcurveto{\pgfqpoint{2.428635in}{3.268676in}}{\pgfqpoint{2.418036in}{3.264286in}}{\pgfqpoint{2.410223in}{3.256472in}}%
\pgfpathcurveto{\pgfqpoint{2.402409in}{3.248659in}}{\pgfqpoint{2.398019in}{3.238060in}}{\pgfqpoint{2.398019in}{3.227010in}}%
\pgfpathcurveto{\pgfqpoint{2.398019in}{3.215959in}}{\pgfqpoint{2.402409in}{3.205360in}}{\pgfqpoint{2.410223in}{3.197547in}}%
\pgfpathcurveto{\pgfqpoint{2.418036in}{3.189733in}}{\pgfqpoint{2.428635in}{3.185343in}}{\pgfqpoint{2.439685in}{3.185343in}}%
\pgfpathclose%
\pgfusepath{stroke,fill}%
\end{pgfscope}%
\begin{pgfscope}%
\pgfpathrectangle{\pgfqpoint{0.648703in}{0.548769in}}{\pgfqpoint{5.201297in}{3.102590in}}%
\pgfusepath{clip}%
\pgfsetbuttcap%
\pgfsetroundjoin%
\definecolor{currentfill}{rgb}{0.121569,0.466667,0.705882}%
\pgfsetfillcolor{currentfill}%
\pgfsetlinewidth{1.003750pt}%
\definecolor{currentstroke}{rgb}{0.121569,0.466667,0.705882}%
\pgfsetstrokecolor{currentstroke}%
\pgfsetdash{}{0pt}%
\pgfpathmoveto{\pgfqpoint{1.338539in}{0.652358in}}%
\pgfpathcurveto{\pgfqpoint{1.349589in}{0.652358in}}{\pgfqpoint{1.360188in}{0.656748in}}{\pgfqpoint{1.368002in}{0.664562in}}%
\pgfpathcurveto{\pgfqpoint{1.375816in}{0.672375in}}{\pgfqpoint{1.380206in}{0.682974in}}{\pgfqpoint{1.380206in}{0.694024in}}%
\pgfpathcurveto{\pgfqpoint{1.380206in}{0.705074in}}{\pgfqpoint{1.375816in}{0.715673in}}{\pgfqpoint{1.368002in}{0.723487in}}%
\pgfpathcurveto{\pgfqpoint{1.360188in}{0.731301in}}{\pgfqpoint{1.349589in}{0.735691in}}{\pgfqpoint{1.338539in}{0.735691in}}%
\pgfpathcurveto{\pgfqpoint{1.327489in}{0.735691in}}{\pgfqpoint{1.316890in}{0.731301in}}{\pgfqpoint{1.309076in}{0.723487in}}%
\pgfpathcurveto{\pgfqpoint{1.301263in}{0.715673in}}{\pgfqpoint{1.296872in}{0.705074in}}{\pgfqpoint{1.296872in}{0.694024in}}%
\pgfpathcurveto{\pgfqpoint{1.296872in}{0.682974in}}{\pgfqpoint{1.301263in}{0.672375in}}{\pgfqpoint{1.309076in}{0.664562in}}%
\pgfpathcurveto{\pgfqpoint{1.316890in}{0.656748in}}{\pgfqpoint{1.327489in}{0.652358in}}{\pgfqpoint{1.338539in}{0.652358in}}%
\pgfpathclose%
\pgfusepath{stroke,fill}%
\end{pgfscope}%
\begin{pgfscope}%
\pgfpathrectangle{\pgfqpoint{0.648703in}{0.548769in}}{\pgfqpoint{5.201297in}{3.102590in}}%
\pgfusepath{clip}%
\pgfsetbuttcap%
\pgfsetroundjoin%
\definecolor{currentfill}{rgb}{0.121569,0.466667,0.705882}%
\pgfsetfillcolor{currentfill}%
\pgfsetlinewidth{1.003750pt}%
\definecolor{currentstroke}{rgb}{0.121569,0.466667,0.705882}%
\pgfsetstrokecolor{currentstroke}%
\pgfsetdash{}{0pt}%
\pgfpathmoveto{\pgfqpoint{4.123791in}{3.181114in}}%
\pgfpathcurveto{\pgfqpoint{4.134841in}{3.181114in}}{\pgfqpoint{4.145441in}{3.185504in}}{\pgfqpoint{4.153254in}{3.193318in}}%
\pgfpathcurveto{\pgfqpoint{4.161068in}{3.201132in}}{\pgfqpoint{4.165458in}{3.211731in}}{\pgfqpoint{4.165458in}{3.222781in}}%
\pgfpathcurveto{\pgfqpoint{4.165458in}{3.233831in}}{\pgfqpoint{4.161068in}{3.244430in}}{\pgfqpoint{4.153254in}{3.252244in}}%
\pgfpathcurveto{\pgfqpoint{4.145441in}{3.260057in}}{\pgfqpoint{4.134841in}{3.264448in}}{\pgfqpoint{4.123791in}{3.264448in}}%
\pgfpathcurveto{\pgfqpoint{4.112741in}{3.264448in}}{\pgfqpoint{4.102142in}{3.260057in}}{\pgfqpoint{4.094329in}{3.252244in}}%
\pgfpathcurveto{\pgfqpoint{4.086515in}{3.244430in}}{\pgfqpoint{4.082125in}{3.233831in}}{\pgfqpoint{4.082125in}{3.222781in}}%
\pgfpathcurveto{\pgfqpoint{4.082125in}{3.211731in}}{\pgfqpoint{4.086515in}{3.201132in}}{\pgfqpoint{4.094329in}{3.193318in}}%
\pgfpathcurveto{\pgfqpoint{4.102142in}{3.185504in}}{\pgfqpoint{4.112741in}{3.181114in}}{\pgfqpoint{4.123791in}{3.181114in}}%
\pgfpathclose%
\pgfusepath{stroke,fill}%
\end{pgfscope}%
\begin{pgfscope}%
\pgfpathrectangle{\pgfqpoint{0.648703in}{0.548769in}}{\pgfqpoint{5.201297in}{3.102590in}}%
\pgfusepath{clip}%
\pgfsetbuttcap%
\pgfsetroundjoin%
\definecolor{currentfill}{rgb}{1.000000,0.498039,0.054902}%
\pgfsetfillcolor{currentfill}%
\pgfsetlinewidth{1.003750pt}%
\definecolor{currentstroke}{rgb}{1.000000,0.498039,0.054902}%
\pgfsetstrokecolor{currentstroke}%
\pgfsetdash{}{0pt}%
\pgfpathmoveto{\pgfqpoint{1.856726in}{3.189572in}}%
\pgfpathcurveto{\pgfqpoint{1.867776in}{3.189572in}}{\pgfqpoint{1.878375in}{3.193962in}}{\pgfqpoint{1.886188in}{3.201775in}}%
\pgfpathcurveto{\pgfqpoint{1.894002in}{3.209589in}}{\pgfqpoint{1.898392in}{3.220188in}}{\pgfqpoint{1.898392in}{3.231238in}}%
\pgfpathcurveto{\pgfqpoint{1.898392in}{3.242288in}}{\pgfqpoint{1.894002in}{3.252887in}}{\pgfqpoint{1.886188in}{3.260701in}}%
\pgfpathcurveto{\pgfqpoint{1.878375in}{3.268515in}}{\pgfqpoint{1.867776in}{3.272905in}}{\pgfqpoint{1.856726in}{3.272905in}}%
\pgfpathcurveto{\pgfqpoint{1.845675in}{3.272905in}}{\pgfqpoint{1.835076in}{3.268515in}}{\pgfqpoint{1.827263in}{3.260701in}}%
\pgfpathcurveto{\pgfqpoint{1.819449in}{3.252887in}}{\pgfqpoint{1.815059in}{3.242288in}}{\pgfqpoint{1.815059in}{3.231238in}}%
\pgfpathcurveto{\pgfqpoint{1.815059in}{3.220188in}}{\pgfqpoint{1.819449in}{3.209589in}}{\pgfqpoint{1.827263in}{3.201775in}}%
\pgfpathcurveto{\pgfqpoint{1.835076in}{3.193962in}}{\pgfqpoint{1.845675in}{3.189572in}}{\pgfqpoint{1.856726in}{3.189572in}}%
\pgfpathclose%
\pgfusepath{stroke,fill}%
\end{pgfscope}%
\begin{pgfscope}%
\pgfpathrectangle{\pgfqpoint{0.648703in}{0.548769in}}{\pgfqpoint{5.201297in}{3.102590in}}%
\pgfusepath{clip}%
\pgfsetbuttcap%
\pgfsetroundjoin%
\definecolor{currentfill}{rgb}{0.121569,0.466667,0.705882}%
\pgfsetfillcolor{currentfill}%
\pgfsetlinewidth{1.003750pt}%
\definecolor{currentstroke}{rgb}{0.121569,0.466667,0.705882}%
\pgfsetstrokecolor{currentstroke}%
\pgfsetdash{}{0pt}%
\pgfpathmoveto{\pgfqpoint{1.597632in}{0.648129in}}%
\pgfpathcurveto{\pgfqpoint{1.608682in}{0.648129in}}{\pgfqpoint{1.619282in}{0.652519in}}{\pgfqpoint{1.627095in}{0.660333in}}%
\pgfpathcurveto{\pgfqpoint{1.634909in}{0.668146in}}{\pgfqpoint{1.639299in}{0.678745in}}{\pgfqpoint{1.639299in}{0.689796in}}%
\pgfpathcurveto{\pgfqpoint{1.639299in}{0.700846in}}{\pgfqpoint{1.634909in}{0.711445in}}{\pgfqpoint{1.627095in}{0.719258in}}%
\pgfpathcurveto{\pgfqpoint{1.619282in}{0.727072in}}{\pgfqpoint{1.608682in}{0.731462in}}{\pgfqpoint{1.597632in}{0.731462in}}%
\pgfpathcurveto{\pgfqpoint{1.586582in}{0.731462in}}{\pgfqpoint{1.575983in}{0.727072in}}{\pgfqpoint{1.568170in}{0.719258in}}%
\pgfpathcurveto{\pgfqpoint{1.560356in}{0.711445in}}{\pgfqpoint{1.555966in}{0.700846in}}{\pgfqpoint{1.555966in}{0.689796in}}%
\pgfpathcurveto{\pgfqpoint{1.555966in}{0.678745in}}{\pgfqpoint{1.560356in}{0.668146in}}{\pgfqpoint{1.568170in}{0.660333in}}%
\pgfpathcurveto{\pgfqpoint{1.575983in}{0.652519in}}{\pgfqpoint{1.586582in}{0.648129in}}{\pgfqpoint{1.597632in}{0.648129in}}%
\pgfpathclose%
\pgfusepath{stroke,fill}%
\end{pgfscope}%
\begin{pgfscope}%
\pgfpathrectangle{\pgfqpoint{0.648703in}{0.548769in}}{\pgfqpoint{5.201297in}{3.102590in}}%
\pgfusepath{clip}%
\pgfsetbuttcap%
\pgfsetroundjoin%
\definecolor{currentfill}{rgb}{0.121569,0.466667,0.705882}%
\pgfsetfillcolor{currentfill}%
\pgfsetlinewidth{1.003750pt}%
\definecolor{currentstroke}{rgb}{0.121569,0.466667,0.705882}%
\pgfsetstrokecolor{currentstroke}%
\pgfsetdash{}{0pt}%
\pgfpathmoveto{\pgfqpoint{1.273766in}{0.774990in}}%
\pgfpathcurveto{\pgfqpoint{1.284816in}{0.774990in}}{\pgfqpoint{1.295415in}{0.779380in}}{\pgfqpoint{1.303229in}{0.787194in}}%
\pgfpathcurveto{\pgfqpoint{1.311042in}{0.795007in}}{\pgfqpoint{1.315432in}{0.805606in}}{\pgfqpoint{1.315432in}{0.816656in}}%
\pgfpathcurveto{\pgfqpoint{1.315432in}{0.827706in}}{\pgfqpoint{1.311042in}{0.838305in}}{\pgfqpoint{1.303229in}{0.846119in}}%
\pgfpathcurveto{\pgfqpoint{1.295415in}{0.853933in}}{\pgfqpoint{1.284816in}{0.858323in}}{\pgfqpoint{1.273766in}{0.858323in}}%
\pgfpathcurveto{\pgfqpoint{1.262716in}{0.858323in}}{\pgfqpoint{1.252117in}{0.853933in}}{\pgfqpoint{1.244303in}{0.846119in}}%
\pgfpathcurveto{\pgfqpoint{1.236489in}{0.838305in}}{\pgfqpoint{1.232099in}{0.827706in}}{\pgfqpoint{1.232099in}{0.816656in}}%
\pgfpathcurveto{\pgfqpoint{1.232099in}{0.805606in}}{\pgfqpoint{1.236489in}{0.795007in}}{\pgfqpoint{1.244303in}{0.787194in}}%
\pgfpathcurveto{\pgfqpoint{1.252117in}{0.779380in}}{\pgfqpoint{1.262716in}{0.774990in}}{\pgfqpoint{1.273766in}{0.774990in}}%
\pgfpathclose%
\pgfusepath{stroke,fill}%
\end{pgfscope}%
\begin{pgfscope}%
\pgfpathrectangle{\pgfqpoint{0.648703in}{0.548769in}}{\pgfqpoint{5.201297in}{3.102590in}}%
\pgfusepath{clip}%
\pgfsetbuttcap%
\pgfsetroundjoin%
\definecolor{currentfill}{rgb}{0.121569,0.466667,0.705882}%
\pgfsetfillcolor{currentfill}%
\pgfsetlinewidth{1.003750pt}%
\definecolor{currentstroke}{rgb}{0.121569,0.466667,0.705882}%
\pgfsetstrokecolor{currentstroke}%
\pgfsetdash{}{0pt}%
\pgfpathmoveto{\pgfqpoint{1.273766in}{0.783447in}}%
\pgfpathcurveto{\pgfqpoint{1.284816in}{0.783447in}}{\pgfqpoint{1.295415in}{0.787837in}}{\pgfqpoint{1.303229in}{0.795651in}}%
\pgfpathcurveto{\pgfqpoint{1.311042in}{0.803465in}}{\pgfqpoint{1.315432in}{0.814064in}}{\pgfqpoint{1.315432in}{0.825114in}}%
\pgfpathcurveto{\pgfqpoint{1.315432in}{0.836164in}}{\pgfqpoint{1.311042in}{0.846763in}}{\pgfqpoint{1.303229in}{0.854576in}}%
\pgfpathcurveto{\pgfqpoint{1.295415in}{0.862390in}}{\pgfqpoint{1.284816in}{0.866780in}}{\pgfqpoint{1.273766in}{0.866780in}}%
\pgfpathcurveto{\pgfqpoint{1.262716in}{0.866780in}}{\pgfqpoint{1.252117in}{0.862390in}}{\pgfqpoint{1.244303in}{0.854576in}}%
\pgfpathcurveto{\pgfqpoint{1.236489in}{0.846763in}}{\pgfqpoint{1.232099in}{0.836164in}}{\pgfqpoint{1.232099in}{0.825114in}}%
\pgfpathcurveto{\pgfqpoint{1.232099in}{0.814064in}}{\pgfqpoint{1.236489in}{0.803465in}}{\pgfqpoint{1.244303in}{0.795651in}}%
\pgfpathcurveto{\pgfqpoint{1.252117in}{0.787837in}}{\pgfqpoint{1.262716in}{0.783447in}}{\pgfqpoint{1.273766in}{0.783447in}}%
\pgfpathclose%
\pgfusepath{stroke,fill}%
\end{pgfscope}%
\begin{pgfscope}%
\pgfpathrectangle{\pgfqpoint{0.648703in}{0.548769in}}{\pgfqpoint{5.201297in}{3.102590in}}%
\pgfusepath{clip}%
\pgfsetbuttcap%
\pgfsetroundjoin%
\definecolor{currentfill}{rgb}{0.121569,0.466667,0.705882}%
\pgfsetfillcolor{currentfill}%
\pgfsetlinewidth{1.003750pt}%
\definecolor{currentstroke}{rgb}{0.121569,0.466667,0.705882}%
\pgfsetstrokecolor{currentstroke}%
\pgfsetdash{}{0pt}%
\pgfpathmoveto{\pgfqpoint{0.949899in}{0.652358in}}%
\pgfpathcurveto{\pgfqpoint{0.960949in}{0.652358in}}{\pgfqpoint{0.971548in}{0.656748in}}{\pgfqpoint{0.979362in}{0.664562in}}%
\pgfpathcurveto{\pgfqpoint{0.987176in}{0.672375in}}{\pgfqpoint{0.991566in}{0.682974in}}{\pgfqpoint{0.991566in}{0.694024in}}%
\pgfpathcurveto{\pgfqpoint{0.991566in}{0.705074in}}{\pgfqpoint{0.987176in}{0.715673in}}{\pgfqpoint{0.979362in}{0.723487in}}%
\pgfpathcurveto{\pgfqpoint{0.971548in}{0.731301in}}{\pgfqpoint{0.960949in}{0.735691in}}{\pgfqpoint{0.949899in}{0.735691in}}%
\pgfpathcurveto{\pgfqpoint{0.938849in}{0.735691in}}{\pgfqpoint{0.928250in}{0.731301in}}{\pgfqpoint{0.920437in}{0.723487in}}%
\pgfpathcurveto{\pgfqpoint{0.912623in}{0.715673in}}{\pgfqpoint{0.908233in}{0.705074in}}{\pgfqpoint{0.908233in}{0.694024in}}%
\pgfpathcurveto{\pgfqpoint{0.908233in}{0.682974in}}{\pgfqpoint{0.912623in}{0.672375in}}{\pgfqpoint{0.920437in}{0.664562in}}%
\pgfpathcurveto{\pgfqpoint{0.928250in}{0.656748in}}{\pgfqpoint{0.938849in}{0.652358in}}{\pgfqpoint{0.949899in}{0.652358in}}%
\pgfpathclose%
\pgfusepath{stroke,fill}%
\end{pgfscope}%
\begin{pgfscope}%
\pgfpathrectangle{\pgfqpoint{0.648703in}{0.548769in}}{\pgfqpoint{5.201297in}{3.102590in}}%
\pgfusepath{clip}%
\pgfsetbuttcap%
\pgfsetroundjoin%
\definecolor{currentfill}{rgb}{0.121569,0.466667,0.705882}%
\pgfsetfillcolor{currentfill}%
\pgfsetlinewidth{1.003750pt}%
\definecolor{currentstroke}{rgb}{0.121569,0.466667,0.705882}%
\pgfsetstrokecolor{currentstroke}%
\pgfsetdash{}{0pt}%
\pgfpathmoveto{\pgfqpoint{1.403312in}{0.648129in}}%
\pgfpathcurveto{\pgfqpoint{1.414363in}{0.648129in}}{\pgfqpoint{1.424962in}{0.652519in}}{\pgfqpoint{1.432775in}{0.660333in}}%
\pgfpathcurveto{\pgfqpoint{1.440589in}{0.668146in}}{\pgfqpoint{1.444979in}{0.678745in}}{\pgfqpoint{1.444979in}{0.689796in}}%
\pgfpathcurveto{\pgfqpoint{1.444979in}{0.700846in}}{\pgfqpoint{1.440589in}{0.711445in}}{\pgfqpoint{1.432775in}{0.719258in}}%
\pgfpathcurveto{\pgfqpoint{1.424962in}{0.727072in}}{\pgfqpoint{1.414363in}{0.731462in}}{\pgfqpoint{1.403312in}{0.731462in}}%
\pgfpathcurveto{\pgfqpoint{1.392262in}{0.731462in}}{\pgfqpoint{1.381663in}{0.727072in}}{\pgfqpoint{1.373850in}{0.719258in}}%
\pgfpathcurveto{\pgfqpoint{1.366036in}{0.711445in}}{\pgfqpoint{1.361646in}{0.700846in}}{\pgfqpoint{1.361646in}{0.689796in}}%
\pgfpathcurveto{\pgfqpoint{1.361646in}{0.678745in}}{\pgfqpoint{1.366036in}{0.668146in}}{\pgfqpoint{1.373850in}{0.660333in}}%
\pgfpathcurveto{\pgfqpoint{1.381663in}{0.652519in}}{\pgfqpoint{1.392262in}{0.648129in}}{\pgfqpoint{1.403312in}{0.648129in}}%
\pgfpathclose%
\pgfusepath{stroke,fill}%
\end{pgfscope}%
\begin{pgfscope}%
\pgfpathrectangle{\pgfqpoint{0.648703in}{0.548769in}}{\pgfqpoint{5.201297in}{3.102590in}}%
\pgfusepath{clip}%
\pgfsetbuttcap%
\pgfsetroundjoin%
\definecolor{currentfill}{rgb}{0.121569,0.466667,0.705882}%
\pgfsetfillcolor{currentfill}%
\pgfsetlinewidth{1.003750pt}%
\definecolor{currentstroke}{rgb}{0.121569,0.466667,0.705882}%
\pgfsetstrokecolor{currentstroke}%
\pgfsetdash{}{0pt}%
\pgfpathmoveto{\pgfqpoint{1.208993in}{0.648129in}}%
\pgfpathcurveto{\pgfqpoint{1.220043in}{0.648129in}}{\pgfqpoint{1.230642in}{0.652519in}}{\pgfqpoint{1.238455in}{0.660333in}}%
\pgfpathcurveto{\pgfqpoint{1.246269in}{0.668146in}}{\pgfqpoint{1.250659in}{0.678745in}}{\pgfqpoint{1.250659in}{0.689796in}}%
\pgfpathcurveto{\pgfqpoint{1.250659in}{0.700846in}}{\pgfqpoint{1.246269in}{0.711445in}}{\pgfqpoint{1.238455in}{0.719258in}}%
\pgfpathcurveto{\pgfqpoint{1.230642in}{0.727072in}}{\pgfqpoint{1.220043in}{0.731462in}}{\pgfqpoint{1.208993in}{0.731462in}}%
\pgfpathcurveto{\pgfqpoint{1.197942in}{0.731462in}}{\pgfqpoint{1.187343in}{0.727072in}}{\pgfqpoint{1.179530in}{0.719258in}}%
\pgfpathcurveto{\pgfqpoint{1.171716in}{0.711445in}}{\pgfqpoint{1.167326in}{0.700846in}}{\pgfqpoint{1.167326in}{0.689796in}}%
\pgfpathcurveto{\pgfqpoint{1.167326in}{0.678745in}}{\pgfqpoint{1.171716in}{0.668146in}}{\pgfqpoint{1.179530in}{0.660333in}}%
\pgfpathcurveto{\pgfqpoint{1.187343in}{0.652519in}}{\pgfqpoint{1.197942in}{0.648129in}}{\pgfqpoint{1.208993in}{0.648129in}}%
\pgfpathclose%
\pgfusepath{stroke,fill}%
\end{pgfscope}%
\begin{pgfscope}%
\pgfpathrectangle{\pgfqpoint{0.648703in}{0.548769in}}{\pgfqpoint{5.201297in}{3.102590in}}%
\pgfusepath{clip}%
\pgfsetbuttcap%
\pgfsetroundjoin%
\definecolor{currentfill}{rgb}{1.000000,0.498039,0.054902}%
\pgfsetfillcolor{currentfill}%
\pgfsetlinewidth{1.003750pt}%
\definecolor{currentstroke}{rgb}{1.000000,0.498039,0.054902}%
\pgfsetstrokecolor{currentstroke}%
\pgfsetdash{}{0pt}%
\pgfpathmoveto{\pgfqpoint{1.403312in}{3.193800in}}%
\pgfpathcurveto{\pgfqpoint{1.414363in}{3.193800in}}{\pgfqpoint{1.424962in}{3.198191in}}{\pgfqpoint{1.432775in}{3.206004in}}%
\pgfpathcurveto{\pgfqpoint{1.440589in}{3.213818in}}{\pgfqpoint{1.444979in}{3.224417in}}{\pgfqpoint{1.444979in}{3.235467in}}%
\pgfpathcurveto{\pgfqpoint{1.444979in}{3.246517in}}{\pgfqpoint{1.440589in}{3.257116in}}{\pgfqpoint{1.432775in}{3.264930in}}%
\pgfpathcurveto{\pgfqpoint{1.424962in}{3.272743in}}{\pgfqpoint{1.414363in}{3.277134in}}{\pgfqpoint{1.403312in}{3.277134in}}%
\pgfpathcurveto{\pgfqpoint{1.392262in}{3.277134in}}{\pgfqpoint{1.381663in}{3.272743in}}{\pgfqpoint{1.373850in}{3.264930in}}%
\pgfpathcurveto{\pgfqpoint{1.366036in}{3.257116in}}{\pgfqpoint{1.361646in}{3.246517in}}{\pgfqpoint{1.361646in}{3.235467in}}%
\pgfpathcurveto{\pgfqpoint{1.361646in}{3.224417in}}{\pgfqpoint{1.366036in}{3.213818in}}{\pgfqpoint{1.373850in}{3.206004in}}%
\pgfpathcurveto{\pgfqpoint{1.381663in}{3.198191in}}{\pgfqpoint{1.392262in}{3.193800in}}{\pgfqpoint{1.403312in}{3.193800in}}%
\pgfpathclose%
\pgfusepath{stroke,fill}%
\end{pgfscope}%
\begin{pgfscope}%
\pgfpathrectangle{\pgfqpoint{0.648703in}{0.548769in}}{\pgfqpoint{5.201297in}{3.102590in}}%
\pgfusepath{clip}%
\pgfsetbuttcap%
\pgfsetroundjoin%
\definecolor{currentfill}{rgb}{1.000000,0.498039,0.054902}%
\pgfsetfillcolor{currentfill}%
\pgfsetlinewidth{1.003750pt}%
\definecolor{currentstroke}{rgb}{1.000000,0.498039,0.054902}%
\pgfsetstrokecolor{currentstroke}%
\pgfsetdash{}{0pt}%
\pgfpathmoveto{\pgfqpoint{2.374912in}{3.231859in}}%
\pgfpathcurveto{\pgfqpoint{2.385962in}{3.231859in}}{\pgfqpoint{2.396561in}{3.236249in}}{\pgfqpoint{2.404375in}{3.244062in}}%
\pgfpathcurveto{\pgfqpoint{2.412188in}{3.251876in}}{\pgfqpoint{2.416579in}{3.262475in}}{\pgfqpoint{2.416579in}{3.273525in}}%
\pgfpathcurveto{\pgfqpoint{2.416579in}{3.284575in}}{\pgfqpoint{2.412188in}{3.295174in}}{\pgfqpoint{2.404375in}{3.302988in}}%
\pgfpathcurveto{\pgfqpoint{2.396561in}{3.310802in}}{\pgfqpoint{2.385962in}{3.315192in}}{\pgfqpoint{2.374912in}{3.315192in}}%
\pgfpathcurveto{\pgfqpoint{2.363862in}{3.315192in}}{\pgfqpoint{2.353263in}{3.310802in}}{\pgfqpoint{2.345449in}{3.302988in}}%
\pgfpathcurveto{\pgfqpoint{2.337636in}{3.295174in}}{\pgfqpoint{2.333245in}{3.284575in}}{\pgfqpoint{2.333245in}{3.273525in}}%
\pgfpathcurveto{\pgfqpoint{2.333245in}{3.262475in}}{\pgfqpoint{2.337636in}{3.251876in}}{\pgfqpoint{2.345449in}{3.244062in}}%
\pgfpathcurveto{\pgfqpoint{2.353263in}{3.236249in}}{\pgfqpoint{2.363862in}{3.231859in}}{\pgfqpoint{2.374912in}{3.231859in}}%
\pgfpathclose%
\pgfusepath{stroke,fill}%
\end{pgfscope}%
\begin{pgfscope}%
\pgfpathrectangle{\pgfqpoint{0.648703in}{0.548769in}}{\pgfqpoint{5.201297in}{3.102590in}}%
\pgfusepath{clip}%
\pgfsetbuttcap%
\pgfsetroundjoin%
\definecolor{currentfill}{rgb}{0.121569,0.466667,0.705882}%
\pgfsetfillcolor{currentfill}%
\pgfsetlinewidth{1.003750pt}%
\definecolor{currentstroke}{rgb}{0.121569,0.466667,0.705882}%
\pgfsetstrokecolor{currentstroke}%
\pgfsetdash{}{0pt}%
\pgfpathmoveto{\pgfqpoint{1.468086in}{0.758075in}}%
\pgfpathcurveto{\pgfqpoint{1.479136in}{0.758075in}}{\pgfqpoint{1.489735in}{0.762465in}}{\pgfqpoint{1.497549in}{0.770279in}}%
\pgfpathcurveto{\pgfqpoint{1.505362in}{0.778092in}}{\pgfqpoint{1.509752in}{0.788691in}}{\pgfqpoint{1.509752in}{0.799742in}}%
\pgfpathcurveto{\pgfqpoint{1.509752in}{0.810792in}}{\pgfqpoint{1.505362in}{0.821391in}}{\pgfqpoint{1.497549in}{0.829204in}}%
\pgfpathcurveto{\pgfqpoint{1.489735in}{0.837018in}}{\pgfqpoint{1.479136in}{0.841408in}}{\pgfqpoint{1.468086in}{0.841408in}}%
\pgfpathcurveto{\pgfqpoint{1.457036in}{0.841408in}}{\pgfqpoint{1.446437in}{0.837018in}}{\pgfqpoint{1.438623in}{0.829204in}}%
\pgfpathcurveto{\pgfqpoint{1.430809in}{0.821391in}}{\pgfqpoint{1.426419in}{0.810792in}}{\pgfqpoint{1.426419in}{0.799742in}}%
\pgfpathcurveto{\pgfqpoint{1.426419in}{0.788691in}}{\pgfqpoint{1.430809in}{0.778092in}}{\pgfqpoint{1.438623in}{0.770279in}}%
\pgfpathcurveto{\pgfqpoint{1.446437in}{0.762465in}}{\pgfqpoint{1.457036in}{0.758075in}}{\pgfqpoint{1.468086in}{0.758075in}}%
\pgfpathclose%
\pgfusepath{stroke,fill}%
\end{pgfscope}%
\begin{pgfscope}%
\pgfpathrectangle{\pgfqpoint{0.648703in}{0.548769in}}{\pgfqpoint{5.201297in}{3.102590in}}%
\pgfusepath{clip}%
\pgfsetbuttcap%
\pgfsetroundjoin%
\definecolor{currentfill}{rgb}{1.000000,0.498039,0.054902}%
\pgfsetfillcolor{currentfill}%
\pgfsetlinewidth{1.003750pt}%
\definecolor{currentstroke}{rgb}{1.000000,0.498039,0.054902}%
\pgfsetstrokecolor{currentstroke}%
\pgfsetdash{}{0pt}%
\pgfpathmoveto{\pgfqpoint{1.468086in}{3.185343in}}%
\pgfpathcurveto{\pgfqpoint{1.479136in}{3.185343in}}{\pgfqpoint{1.489735in}{3.189733in}}{\pgfqpoint{1.497549in}{3.197547in}}%
\pgfpathcurveto{\pgfqpoint{1.505362in}{3.205360in}}{\pgfqpoint{1.509752in}{3.215959in}}{\pgfqpoint{1.509752in}{3.227010in}}%
\pgfpathcurveto{\pgfqpoint{1.509752in}{3.238060in}}{\pgfqpoint{1.505362in}{3.248659in}}{\pgfqpoint{1.497549in}{3.256472in}}%
\pgfpathcurveto{\pgfqpoint{1.489735in}{3.264286in}}{\pgfqpoint{1.479136in}{3.268676in}}{\pgfqpoint{1.468086in}{3.268676in}}%
\pgfpathcurveto{\pgfqpoint{1.457036in}{3.268676in}}{\pgfqpoint{1.446437in}{3.264286in}}{\pgfqpoint{1.438623in}{3.256472in}}%
\pgfpathcurveto{\pgfqpoint{1.430809in}{3.248659in}}{\pgfqpoint{1.426419in}{3.238060in}}{\pgfqpoint{1.426419in}{3.227010in}}%
\pgfpathcurveto{\pgfqpoint{1.426419in}{3.215959in}}{\pgfqpoint{1.430809in}{3.205360in}}{\pgfqpoint{1.438623in}{3.197547in}}%
\pgfpathcurveto{\pgfqpoint{1.446437in}{3.189733in}}{\pgfqpoint{1.457036in}{3.185343in}}{\pgfqpoint{1.468086in}{3.185343in}}%
\pgfpathclose%
\pgfusepath{stroke,fill}%
\end{pgfscope}%
\begin{pgfscope}%
\pgfpathrectangle{\pgfqpoint{0.648703in}{0.548769in}}{\pgfqpoint{5.201297in}{3.102590in}}%
\pgfusepath{clip}%
\pgfsetbuttcap%
\pgfsetroundjoin%
\definecolor{currentfill}{rgb}{1.000000,0.498039,0.054902}%
\pgfsetfillcolor{currentfill}%
\pgfsetlinewidth{1.003750pt}%
\definecolor{currentstroke}{rgb}{1.000000,0.498039,0.054902}%
\pgfsetstrokecolor{currentstroke}%
\pgfsetdash{}{0pt}%
\pgfpathmoveto{\pgfqpoint{1.532859in}{3.202258in}}%
\pgfpathcurveto{\pgfqpoint{1.543909in}{3.202258in}}{\pgfqpoint{1.554508in}{3.206648in}}{\pgfqpoint{1.562322in}{3.214462in}}%
\pgfpathcurveto{\pgfqpoint{1.570135in}{3.222275in}}{\pgfqpoint{1.574526in}{3.232874in}}{\pgfqpoint{1.574526in}{3.243924in}}%
\pgfpathcurveto{\pgfqpoint{1.574526in}{3.254974in}}{\pgfqpoint{1.570135in}{3.265573in}}{\pgfqpoint{1.562322in}{3.273387in}}%
\pgfpathcurveto{\pgfqpoint{1.554508in}{3.281201in}}{\pgfqpoint{1.543909in}{3.285591in}}{\pgfqpoint{1.532859in}{3.285591in}}%
\pgfpathcurveto{\pgfqpoint{1.521809in}{3.285591in}}{\pgfqpoint{1.511210in}{3.281201in}}{\pgfqpoint{1.503396in}{3.273387in}}%
\pgfpathcurveto{\pgfqpoint{1.495583in}{3.265573in}}{\pgfqpoint{1.491192in}{3.254974in}}{\pgfqpoint{1.491192in}{3.243924in}}%
\pgfpathcurveto{\pgfqpoint{1.491192in}{3.232874in}}{\pgfqpoint{1.495583in}{3.222275in}}{\pgfqpoint{1.503396in}{3.214462in}}%
\pgfpathcurveto{\pgfqpoint{1.511210in}{3.206648in}}{\pgfqpoint{1.521809in}{3.202258in}}{\pgfqpoint{1.532859in}{3.202258in}}%
\pgfpathclose%
\pgfusepath{stroke,fill}%
\end{pgfscope}%
\begin{pgfscope}%
\pgfpathrectangle{\pgfqpoint{0.648703in}{0.548769in}}{\pgfqpoint{5.201297in}{3.102590in}}%
\pgfusepath{clip}%
\pgfsetbuttcap%
\pgfsetroundjoin%
\definecolor{currentfill}{rgb}{0.121569,0.466667,0.705882}%
\pgfsetfillcolor{currentfill}%
\pgfsetlinewidth{1.003750pt}%
\definecolor{currentstroke}{rgb}{0.121569,0.466667,0.705882}%
\pgfsetstrokecolor{currentstroke}%
\pgfsetdash{}{0pt}%
\pgfpathmoveto{\pgfqpoint{1.208993in}{0.648129in}}%
\pgfpathcurveto{\pgfqpoint{1.220043in}{0.648129in}}{\pgfqpoint{1.230642in}{0.652519in}}{\pgfqpoint{1.238455in}{0.660333in}}%
\pgfpathcurveto{\pgfqpoint{1.246269in}{0.668146in}}{\pgfqpoint{1.250659in}{0.678745in}}{\pgfqpoint{1.250659in}{0.689796in}}%
\pgfpathcurveto{\pgfqpoint{1.250659in}{0.700846in}}{\pgfqpoint{1.246269in}{0.711445in}}{\pgfqpoint{1.238455in}{0.719258in}}%
\pgfpathcurveto{\pgfqpoint{1.230642in}{0.727072in}}{\pgfqpoint{1.220043in}{0.731462in}}{\pgfqpoint{1.208993in}{0.731462in}}%
\pgfpathcurveto{\pgfqpoint{1.197942in}{0.731462in}}{\pgfqpoint{1.187343in}{0.727072in}}{\pgfqpoint{1.179530in}{0.719258in}}%
\pgfpathcurveto{\pgfqpoint{1.171716in}{0.711445in}}{\pgfqpoint{1.167326in}{0.700846in}}{\pgfqpoint{1.167326in}{0.689796in}}%
\pgfpathcurveto{\pgfqpoint{1.167326in}{0.678745in}}{\pgfqpoint{1.171716in}{0.668146in}}{\pgfqpoint{1.179530in}{0.660333in}}%
\pgfpathcurveto{\pgfqpoint{1.187343in}{0.652519in}}{\pgfqpoint{1.197942in}{0.648129in}}{\pgfqpoint{1.208993in}{0.648129in}}%
\pgfpathclose%
\pgfusepath{stroke,fill}%
\end{pgfscope}%
\begin{pgfscope}%
\pgfpathrectangle{\pgfqpoint{0.648703in}{0.548769in}}{\pgfqpoint{5.201297in}{3.102590in}}%
\pgfusepath{clip}%
\pgfsetbuttcap%
\pgfsetroundjoin%
\definecolor{currentfill}{rgb}{1.000000,0.498039,0.054902}%
\pgfsetfillcolor{currentfill}%
\pgfsetlinewidth{1.003750pt}%
\definecolor{currentstroke}{rgb}{1.000000,0.498039,0.054902}%
\pgfsetstrokecolor{currentstroke}%
\pgfsetdash{}{0pt}%
\pgfpathmoveto{\pgfqpoint{2.374912in}{3.206486in}}%
\pgfpathcurveto{\pgfqpoint{2.385962in}{3.206486in}}{\pgfqpoint{2.396561in}{3.210877in}}{\pgfqpoint{2.404375in}{3.218690in}}%
\pgfpathcurveto{\pgfqpoint{2.412188in}{3.226504in}}{\pgfqpoint{2.416579in}{3.237103in}}{\pgfqpoint{2.416579in}{3.248153in}}%
\pgfpathcurveto{\pgfqpoint{2.416579in}{3.259203in}}{\pgfqpoint{2.412188in}{3.269802in}}{\pgfqpoint{2.404375in}{3.277616in}}%
\pgfpathcurveto{\pgfqpoint{2.396561in}{3.285429in}}{\pgfqpoint{2.385962in}{3.289820in}}{\pgfqpoint{2.374912in}{3.289820in}}%
\pgfpathcurveto{\pgfqpoint{2.363862in}{3.289820in}}{\pgfqpoint{2.353263in}{3.285429in}}{\pgfqpoint{2.345449in}{3.277616in}}%
\pgfpathcurveto{\pgfqpoint{2.337636in}{3.269802in}}{\pgfqpoint{2.333245in}{3.259203in}}{\pgfqpoint{2.333245in}{3.248153in}}%
\pgfpathcurveto{\pgfqpoint{2.333245in}{3.237103in}}{\pgfqpoint{2.337636in}{3.226504in}}{\pgfqpoint{2.345449in}{3.218690in}}%
\pgfpathcurveto{\pgfqpoint{2.353263in}{3.210877in}}{\pgfqpoint{2.363862in}{3.206486in}}{\pgfqpoint{2.374912in}{3.206486in}}%
\pgfpathclose%
\pgfusepath{stroke,fill}%
\end{pgfscope}%
\begin{pgfscope}%
\pgfpathrectangle{\pgfqpoint{0.648703in}{0.548769in}}{\pgfqpoint{5.201297in}{3.102590in}}%
\pgfusepath{clip}%
\pgfsetbuttcap%
\pgfsetroundjoin%
\definecolor{currentfill}{rgb}{0.121569,0.466667,0.705882}%
\pgfsetfillcolor{currentfill}%
\pgfsetlinewidth{1.003750pt}%
\definecolor{currentstroke}{rgb}{0.121569,0.466667,0.705882}%
\pgfsetstrokecolor{currentstroke}%
\pgfsetdash{}{0pt}%
\pgfpathmoveto{\pgfqpoint{1.208993in}{0.648129in}}%
\pgfpathcurveto{\pgfqpoint{1.220043in}{0.648129in}}{\pgfqpoint{1.230642in}{0.652519in}}{\pgfqpoint{1.238455in}{0.660333in}}%
\pgfpathcurveto{\pgfqpoint{1.246269in}{0.668146in}}{\pgfqpoint{1.250659in}{0.678745in}}{\pgfqpoint{1.250659in}{0.689796in}}%
\pgfpathcurveto{\pgfqpoint{1.250659in}{0.700846in}}{\pgfqpoint{1.246269in}{0.711445in}}{\pgfqpoint{1.238455in}{0.719258in}}%
\pgfpathcurveto{\pgfqpoint{1.230642in}{0.727072in}}{\pgfqpoint{1.220043in}{0.731462in}}{\pgfqpoint{1.208993in}{0.731462in}}%
\pgfpathcurveto{\pgfqpoint{1.197942in}{0.731462in}}{\pgfqpoint{1.187343in}{0.727072in}}{\pgfqpoint{1.179530in}{0.719258in}}%
\pgfpathcurveto{\pgfqpoint{1.171716in}{0.711445in}}{\pgfqpoint{1.167326in}{0.700846in}}{\pgfqpoint{1.167326in}{0.689796in}}%
\pgfpathcurveto{\pgfqpoint{1.167326in}{0.678745in}}{\pgfqpoint{1.171716in}{0.668146in}}{\pgfqpoint{1.179530in}{0.660333in}}%
\pgfpathcurveto{\pgfqpoint{1.187343in}{0.652519in}}{\pgfqpoint{1.197942in}{0.648129in}}{\pgfqpoint{1.208993in}{0.648129in}}%
\pgfpathclose%
\pgfusepath{stroke,fill}%
\end{pgfscope}%
\begin{pgfscope}%
\pgfpathrectangle{\pgfqpoint{0.648703in}{0.548769in}}{\pgfqpoint{5.201297in}{3.102590in}}%
\pgfusepath{clip}%
\pgfsetbuttcap%
\pgfsetroundjoin%
\definecolor{currentfill}{rgb}{0.121569,0.466667,0.705882}%
\pgfsetfillcolor{currentfill}%
\pgfsetlinewidth{1.003750pt}%
\definecolor{currentstroke}{rgb}{0.121569,0.466667,0.705882}%
\pgfsetstrokecolor{currentstroke}%
\pgfsetdash{}{0pt}%
\pgfpathmoveto{\pgfqpoint{0.885126in}{0.648129in}}%
\pgfpathcurveto{\pgfqpoint{0.896176in}{0.648129in}}{\pgfqpoint{0.906775in}{0.652519in}}{\pgfqpoint{0.914589in}{0.660333in}}%
\pgfpathcurveto{\pgfqpoint{0.922402in}{0.668146in}}{\pgfqpoint{0.926793in}{0.678745in}}{\pgfqpoint{0.926793in}{0.689796in}}%
\pgfpathcurveto{\pgfqpoint{0.926793in}{0.700846in}}{\pgfqpoint{0.922402in}{0.711445in}}{\pgfqpoint{0.914589in}{0.719258in}}%
\pgfpathcurveto{\pgfqpoint{0.906775in}{0.727072in}}{\pgfqpoint{0.896176in}{0.731462in}}{\pgfqpoint{0.885126in}{0.731462in}}%
\pgfpathcurveto{\pgfqpoint{0.874076in}{0.731462in}}{\pgfqpoint{0.863477in}{0.727072in}}{\pgfqpoint{0.855663in}{0.719258in}}%
\pgfpathcurveto{\pgfqpoint{0.847850in}{0.711445in}}{\pgfqpoint{0.843459in}{0.700846in}}{\pgfqpoint{0.843459in}{0.689796in}}%
\pgfpathcurveto{\pgfqpoint{0.843459in}{0.678745in}}{\pgfqpoint{0.847850in}{0.668146in}}{\pgfqpoint{0.855663in}{0.660333in}}%
\pgfpathcurveto{\pgfqpoint{0.863477in}{0.652519in}}{\pgfqpoint{0.874076in}{0.648129in}}{\pgfqpoint{0.885126in}{0.648129in}}%
\pgfpathclose%
\pgfusepath{stroke,fill}%
\end{pgfscope}%
\begin{pgfscope}%
\pgfpathrectangle{\pgfqpoint{0.648703in}{0.548769in}}{\pgfqpoint{5.201297in}{3.102590in}}%
\pgfusepath{clip}%
\pgfsetbuttcap%
\pgfsetroundjoin%
\definecolor{currentfill}{rgb}{0.121569,0.466667,0.705882}%
\pgfsetfillcolor{currentfill}%
\pgfsetlinewidth{1.003750pt}%
\definecolor{currentstroke}{rgb}{0.121569,0.466667,0.705882}%
\pgfsetstrokecolor{currentstroke}%
\pgfsetdash{}{0pt}%
\pgfpathmoveto{\pgfqpoint{1.532859in}{0.817277in}}%
\pgfpathcurveto{\pgfqpoint{1.543909in}{0.817277in}}{\pgfqpoint{1.554508in}{0.821667in}}{\pgfqpoint{1.562322in}{0.829480in}}%
\pgfpathcurveto{\pgfqpoint{1.570135in}{0.837294in}}{\pgfqpoint{1.574526in}{0.847893in}}{\pgfqpoint{1.574526in}{0.858943in}}%
\pgfpathcurveto{\pgfqpoint{1.574526in}{0.869993in}}{\pgfqpoint{1.570135in}{0.880592in}}{\pgfqpoint{1.562322in}{0.888406in}}%
\pgfpathcurveto{\pgfqpoint{1.554508in}{0.896220in}}{\pgfqpoint{1.543909in}{0.900610in}}{\pgfqpoint{1.532859in}{0.900610in}}%
\pgfpathcurveto{\pgfqpoint{1.521809in}{0.900610in}}{\pgfqpoint{1.511210in}{0.896220in}}{\pgfqpoint{1.503396in}{0.888406in}}%
\pgfpathcurveto{\pgfqpoint{1.495583in}{0.880592in}}{\pgfqpoint{1.491192in}{0.869993in}}{\pgfqpoint{1.491192in}{0.858943in}}%
\pgfpathcurveto{\pgfqpoint{1.491192in}{0.847893in}}{\pgfqpoint{1.495583in}{0.837294in}}{\pgfqpoint{1.503396in}{0.829480in}}%
\pgfpathcurveto{\pgfqpoint{1.511210in}{0.821667in}}{\pgfqpoint{1.521809in}{0.817277in}}{\pgfqpoint{1.532859in}{0.817277in}}%
\pgfpathclose%
\pgfusepath{stroke,fill}%
\end{pgfscope}%
\begin{pgfscope}%
\pgfpathrectangle{\pgfqpoint{0.648703in}{0.548769in}}{\pgfqpoint{5.201297in}{3.102590in}}%
\pgfusepath{clip}%
\pgfsetbuttcap%
\pgfsetroundjoin%
\definecolor{currentfill}{rgb}{1.000000,0.498039,0.054902}%
\pgfsetfillcolor{currentfill}%
\pgfsetlinewidth{1.003750pt}%
\definecolor{currentstroke}{rgb}{1.000000,0.498039,0.054902}%
\pgfsetstrokecolor{currentstroke}%
\pgfsetdash{}{0pt}%
\pgfpathmoveto{\pgfqpoint{3.994245in}{3.189572in}}%
\pgfpathcurveto{\pgfqpoint{4.005295in}{3.189572in}}{\pgfqpoint{4.015894in}{3.193962in}}{\pgfqpoint{4.023708in}{3.201775in}}%
\pgfpathcurveto{\pgfqpoint{4.031521in}{3.209589in}}{\pgfqpoint{4.035911in}{3.220188in}}{\pgfqpoint{4.035911in}{3.231238in}}%
\pgfpathcurveto{\pgfqpoint{4.035911in}{3.242288in}}{\pgfqpoint{4.031521in}{3.252887in}}{\pgfqpoint{4.023708in}{3.260701in}}%
\pgfpathcurveto{\pgfqpoint{4.015894in}{3.268515in}}{\pgfqpoint{4.005295in}{3.272905in}}{\pgfqpoint{3.994245in}{3.272905in}}%
\pgfpathcurveto{\pgfqpoint{3.983195in}{3.272905in}}{\pgfqpoint{3.972596in}{3.268515in}}{\pgfqpoint{3.964782in}{3.260701in}}%
\pgfpathcurveto{\pgfqpoint{3.956968in}{3.252887in}}{\pgfqpoint{3.952578in}{3.242288in}}{\pgfqpoint{3.952578in}{3.231238in}}%
\pgfpathcurveto{\pgfqpoint{3.952578in}{3.220188in}}{\pgfqpoint{3.956968in}{3.209589in}}{\pgfqpoint{3.964782in}{3.201775in}}%
\pgfpathcurveto{\pgfqpoint{3.972596in}{3.193962in}}{\pgfqpoint{3.983195in}{3.189572in}}{\pgfqpoint{3.994245in}{3.189572in}}%
\pgfpathclose%
\pgfusepath{stroke,fill}%
\end{pgfscope}%
\begin{pgfscope}%
\pgfpathrectangle{\pgfqpoint{0.648703in}{0.548769in}}{\pgfqpoint{5.201297in}{3.102590in}}%
\pgfusepath{clip}%
\pgfsetbuttcap%
\pgfsetroundjoin%
\definecolor{currentfill}{rgb}{1.000000,0.498039,0.054902}%
\pgfsetfillcolor{currentfill}%
\pgfsetlinewidth{1.003750pt}%
\definecolor{currentstroke}{rgb}{1.000000,0.498039,0.054902}%
\pgfsetstrokecolor{currentstroke}%
\pgfsetdash{}{0pt}%
\pgfpathmoveto{\pgfqpoint{1.144219in}{3.214944in}}%
\pgfpathcurveto{\pgfqpoint{1.155269in}{3.214944in}}{\pgfqpoint{1.165868in}{3.219334in}}{\pgfqpoint{1.173682in}{3.227148in}}%
\pgfpathcurveto{\pgfqpoint{1.181496in}{3.234961in}}{\pgfqpoint{1.185886in}{3.245560in}}{\pgfqpoint{1.185886in}{3.256610in}}%
\pgfpathcurveto{\pgfqpoint{1.185886in}{3.267661in}}{\pgfqpoint{1.181496in}{3.278260in}}{\pgfqpoint{1.173682in}{3.286073in}}%
\pgfpathcurveto{\pgfqpoint{1.165868in}{3.293887in}}{\pgfqpoint{1.155269in}{3.298277in}}{\pgfqpoint{1.144219in}{3.298277in}}%
\pgfpathcurveto{\pgfqpoint{1.133169in}{3.298277in}}{\pgfqpoint{1.122570in}{3.293887in}}{\pgfqpoint{1.114756in}{3.286073in}}%
\pgfpathcurveto{\pgfqpoint{1.106943in}{3.278260in}}{\pgfqpoint{1.102553in}{3.267661in}}{\pgfqpoint{1.102553in}{3.256610in}}%
\pgfpathcurveto{\pgfqpoint{1.102553in}{3.245560in}}{\pgfqpoint{1.106943in}{3.234961in}}{\pgfqpoint{1.114756in}{3.227148in}}%
\pgfpathcurveto{\pgfqpoint{1.122570in}{3.219334in}}{\pgfqpoint{1.133169in}{3.214944in}}{\pgfqpoint{1.144219in}{3.214944in}}%
\pgfpathclose%
\pgfusepath{stroke,fill}%
\end{pgfscope}%
\begin{pgfscope}%
\pgfpathrectangle{\pgfqpoint{0.648703in}{0.548769in}}{\pgfqpoint{5.201297in}{3.102590in}}%
\pgfusepath{clip}%
\pgfsetbuttcap%
\pgfsetroundjoin%
\definecolor{currentfill}{rgb}{0.121569,0.466667,0.705882}%
\pgfsetfillcolor{currentfill}%
\pgfsetlinewidth{1.003750pt}%
\definecolor{currentstroke}{rgb}{0.121569,0.466667,0.705882}%
\pgfsetstrokecolor{currentstroke}%
\pgfsetdash{}{0pt}%
\pgfpathmoveto{\pgfqpoint{0.949899in}{0.648129in}}%
\pgfpathcurveto{\pgfqpoint{0.960949in}{0.648129in}}{\pgfqpoint{0.971548in}{0.652519in}}{\pgfqpoint{0.979362in}{0.660333in}}%
\pgfpathcurveto{\pgfqpoint{0.987176in}{0.668146in}}{\pgfqpoint{0.991566in}{0.678745in}}{\pgfqpoint{0.991566in}{0.689796in}}%
\pgfpathcurveto{\pgfqpoint{0.991566in}{0.700846in}}{\pgfqpoint{0.987176in}{0.711445in}}{\pgfqpoint{0.979362in}{0.719258in}}%
\pgfpathcurveto{\pgfqpoint{0.971548in}{0.727072in}}{\pgfqpoint{0.960949in}{0.731462in}}{\pgfqpoint{0.949899in}{0.731462in}}%
\pgfpathcurveto{\pgfqpoint{0.938849in}{0.731462in}}{\pgfqpoint{0.928250in}{0.727072in}}{\pgfqpoint{0.920437in}{0.719258in}}%
\pgfpathcurveto{\pgfqpoint{0.912623in}{0.711445in}}{\pgfqpoint{0.908233in}{0.700846in}}{\pgfqpoint{0.908233in}{0.689796in}}%
\pgfpathcurveto{\pgfqpoint{0.908233in}{0.678745in}}{\pgfqpoint{0.912623in}{0.668146in}}{\pgfqpoint{0.920437in}{0.660333in}}%
\pgfpathcurveto{\pgfqpoint{0.928250in}{0.652519in}}{\pgfqpoint{0.938849in}{0.648129in}}{\pgfqpoint{0.949899in}{0.648129in}}%
\pgfpathclose%
\pgfusepath{stroke,fill}%
\end{pgfscope}%
\begin{pgfscope}%
\pgfpathrectangle{\pgfqpoint{0.648703in}{0.548769in}}{\pgfqpoint{5.201297in}{3.102590in}}%
\pgfusepath{clip}%
\pgfsetbuttcap%
\pgfsetroundjoin%
\definecolor{currentfill}{rgb}{0.121569,0.466667,0.705882}%
\pgfsetfillcolor{currentfill}%
\pgfsetlinewidth{1.003750pt}%
\definecolor{currentstroke}{rgb}{0.121569,0.466667,0.705882}%
\pgfsetstrokecolor{currentstroke}%
\pgfsetdash{}{0pt}%
\pgfpathmoveto{\pgfqpoint{1.273766in}{0.758075in}}%
\pgfpathcurveto{\pgfqpoint{1.284816in}{0.758075in}}{\pgfqpoint{1.295415in}{0.762465in}}{\pgfqpoint{1.303229in}{0.770279in}}%
\pgfpathcurveto{\pgfqpoint{1.311042in}{0.778092in}}{\pgfqpoint{1.315432in}{0.788691in}}{\pgfqpoint{1.315432in}{0.799742in}}%
\pgfpathcurveto{\pgfqpoint{1.315432in}{0.810792in}}{\pgfqpoint{1.311042in}{0.821391in}}{\pgfqpoint{1.303229in}{0.829204in}}%
\pgfpathcurveto{\pgfqpoint{1.295415in}{0.837018in}}{\pgfqpoint{1.284816in}{0.841408in}}{\pgfqpoint{1.273766in}{0.841408in}}%
\pgfpathcurveto{\pgfqpoint{1.262716in}{0.841408in}}{\pgfqpoint{1.252117in}{0.837018in}}{\pgfqpoint{1.244303in}{0.829204in}}%
\pgfpathcurveto{\pgfqpoint{1.236489in}{0.821391in}}{\pgfqpoint{1.232099in}{0.810792in}}{\pgfqpoint{1.232099in}{0.799742in}}%
\pgfpathcurveto{\pgfqpoint{1.232099in}{0.788691in}}{\pgfqpoint{1.236489in}{0.778092in}}{\pgfqpoint{1.244303in}{0.770279in}}%
\pgfpathcurveto{\pgfqpoint{1.252117in}{0.762465in}}{\pgfqpoint{1.262716in}{0.758075in}}{\pgfqpoint{1.273766in}{0.758075in}}%
\pgfpathclose%
\pgfusepath{stroke,fill}%
\end{pgfscope}%
\begin{pgfscope}%
\pgfpathrectangle{\pgfqpoint{0.648703in}{0.548769in}}{\pgfqpoint{5.201297in}{3.102590in}}%
\pgfusepath{clip}%
\pgfsetbuttcap%
\pgfsetroundjoin%
\definecolor{currentfill}{rgb}{1.000000,0.498039,0.054902}%
\pgfsetfillcolor{currentfill}%
\pgfsetlinewidth{1.003750pt}%
\definecolor{currentstroke}{rgb}{1.000000,0.498039,0.054902}%
\pgfsetstrokecolor{currentstroke}%
\pgfsetdash{}{0pt}%
\pgfpathmoveto{\pgfqpoint{1.856726in}{3.193800in}}%
\pgfpathcurveto{\pgfqpoint{1.867776in}{3.193800in}}{\pgfqpoint{1.878375in}{3.198191in}}{\pgfqpoint{1.886188in}{3.206004in}}%
\pgfpathcurveto{\pgfqpoint{1.894002in}{3.213818in}}{\pgfqpoint{1.898392in}{3.224417in}}{\pgfqpoint{1.898392in}{3.235467in}}%
\pgfpathcurveto{\pgfqpoint{1.898392in}{3.246517in}}{\pgfqpoint{1.894002in}{3.257116in}}{\pgfqpoint{1.886188in}{3.264930in}}%
\pgfpathcurveto{\pgfqpoint{1.878375in}{3.272743in}}{\pgfqpoint{1.867776in}{3.277134in}}{\pgfqpoint{1.856726in}{3.277134in}}%
\pgfpathcurveto{\pgfqpoint{1.845675in}{3.277134in}}{\pgfqpoint{1.835076in}{3.272743in}}{\pgfqpoint{1.827263in}{3.264930in}}%
\pgfpathcurveto{\pgfqpoint{1.819449in}{3.257116in}}{\pgfqpoint{1.815059in}{3.246517in}}{\pgfqpoint{1.815059in}{3.235467in}}%
\pgfpathcurveto{\pgfqpoint{1.815059in}{3.224417in}}{\pgfqpoint{1.819449in}{3.213818in}}{\pgfqpoint{1.827263in}{3.206004in}}%
\pgfpathcurveto{\pgfqpoint{1.835076in}{3.198191in}}{\pgfqpoint{1.845675in}{3.193800in}}{\pgfqpoint{1.856726in}{3.193800in}}%
\pgfpathclose%
\pgfusepath{stroke,fill}%
\end{pgfscope}%
\begin{pgfscope}%
\pgfpathrectangle{\pgfqpoint{0.648703in}{0.548769in}}{\pgfqpoint{5.201297in}{3.102590in}}%
\pgfusepath{clip}%
\pgfsetbuttcap%
\pgfsetroundjoin%
\definecolor{currentfill}{rgb}{0.121569,0.466667,0.705882}%
\pgfsetfillcolor{currentfill}%
\pgfsetlinewidth{1.003750pt}%
\definecolor{currentstroke}{rgb}{0.121569,0.466667,0.705882}%
\pgfsetstrokecolor{currentstroke}%
\pgfsetdash{}{0pt}%
\pgfpathmoveto{\pgfqpoint{0.885126in}{0.648129in}}%
\pgfpathcurveto{\pgfqpoint{0.896176in}{0.648129in}}{\pgfqpoint{0.906775in}{0.652519in}}{\pgfqpoint{0.914589in}{0.660333in}}%
\pgfpathcurveto{\pgfqpoint{0.922402in}{0.668146in}}{\pgfqpoint{0.926793in}{0.678745in}}{\pgfqpoint{0.926793in}{0.689796in}}%
\pgfpathcurveto{\pgfqpoint{0.926793in}{0.700846in}}{\pgfqpoint{0.922402in}{0.711445in}}{\pgfqpoint{0.914589in}{0.719258in}}%
\pgfpathcurveto{\pgfqpoint{0.906775in}{0.727072in}}{\pgfqpoint{0.896176in}{0.731462in}}{\pgfqpoint{0.885126in}{0.731462in}}%
\pgfpathcurveto{\pgfqpoint{0.874076in}{0.731462in}}{\pgfqpoint{0.863477in}{0.727072in}}{\pgfqpoint{0.855663in}{0.719258in}}%
\pgfpathcurveto{\pgfqpoint{0.847850in}{0.711445in}}{\pgfqpoint{0.843459in}{0.700846in}}{\pgfqpoint{0.843459in}{0.689796in}}%
\pgfpathcurveto{\pgfqpoint{0.843459in}{0.678745in}}{\pgfqpoint{0.847850in}{0.668146in}}{\pgfqpoint{0.855663in}{0.660333in}}%
\pgfpathcurveto{\pgfqpoint{0.863477in}{0.652519in}}{\pgfqpoint{0.874076in}{0.648129in}}{\pgfqpoint{0.885126in}{0.648129in}}%
\pgfpathclose%
\pgfusepath{stroke,fill}%
\end{pgfscope}%
\begin{pgfscope}%
\pgfpathrectangle{\pgfqpoint{0.648703in}{0.548769in}}{\pgfqpoint{5.201297in}{3.102590in}}%
\pgfusepath{clip}%
\pgfsetbuttcap%
\pgfsetroundjoin%
\definecolor{currentfill}{rgb}{1.000000,0.498039,0.054902}%
\pgfsetfillcolor{currentfill}%
\pgfsetlinewidth{1.003750pt}%
\definecolor{currentstroke}{rgb}{1.000000,0.498039,0.054902}%
\pgfsetstrokecolor{currentstroke}%
\pgfsetdash{}{0pt}%
\pgfpathmoveto{\pgfqpoint{1.856726in}{3.198029in}}%
\pgfpathcurveto{\pgfqpoint{1.867776in}{3.198029in}}{\pgfqpoint{1.878375in}{3.202419in}}{\pgfqpoint{1.886188in}{3.210233in}}%
\pgfpathcurveto{\pgfqpoint{1.894002in}{3.218046in}}{\pgfqpoint{1.898392in}{3.228646in}}{\pgfqpoint{1.898392in}{3.239696in}}%
\pgfpathcurveto{\pgfqpoint{1.898392in}{3.250746in}}{\pgfqpoint{1.894002in}{3.261345in}}{\pgfqpoint{1.886188in}{3.269158in}}%
\pgfpathcurveto{\pgfqpoint{1.878375in}{3.276972in}}{\pgfqpoint{1.867776in}{3.281362in}}{\pgfqpoint{1.856726in}{3.281362in}}%
\pgfpathcurveto{\pgfqpoint{1.845675in}{3.281362in}}{\pgfqpoint{1.835076in}{3.276972in}}{\pgfqpoint{1.827263in}{3.269158in}}%
\pgfpathcurveto{\pgfqpoint{1.819449in}{3.261345in}}{\pgfqpoint{1.815059in}{3.250746in}}{\pgfqpoint{1.815059in}{3.239696in}}%
\pgfpathcurveto{\pgfqpoint{1.815059in}{3.228646in}}{\pgfqpoint{1.819449in}{3.218046in}}{\pgfqpoint{1.827263in}{3.210233in}}%
\pgfpathcurveto{\pgfqpoint{1.835076in}{3.202419in}}{\pgfqpoint{1.845675in}{3.198029in}}{\pgfqpoint{1.856726in}{3.198029in}}%
\pgfpathclose%
\pgfusepath{stroke,fill}%
\end{pgfscope}%
\begin{pgfscope}%
\pgfpathrectangle{\pgfqpoint{0.648703in}{0.548769in}}{\pgfqpoint{5.201297in}{3.102590in}}%
\pgfusepath{clip}%
\pgfsetbuttcap%
\pgfsetroundjoin%
\definecolor{currentfill}{rgb}{1.000000,0.498039,0.054902}%
\pgfsetfillcolor{currentfill}%
\pgfsetlinewidth{1.003750pt}%
\definecolor{currentstroke}{rgb}{1.000000,0.498039,0.054902}%
\pgfsetstrokecolor{currentstroke}%
\pgfsetdash{}{0pt}%
\pgfpathmoveto{\pgfqpoint{1.727179in}{3.193800in}}%
\pgfpathcurveto{\pgfqpoint{1.738229in}{3.193800in}}{\pgfqpoint{1.748828in}{3.198191in}}{\pgfqpoint{1.756642in}{3.206004in}}%
\pgfpathcurveto{\pgfqpoint{1.764455in}{3.213818in}}{\pgfqpoint{1.768846in}{3.224417in}}{\pgfqpoint{1.768846in}{3.235467in}}%
\pgfpathcurveto{\pgfqpoint{1.768846in}{3.246517in}}{\pgfqpoint{1.764455in}{3.257116in}}{\pgfqpoint{1.756642in}{3.264930in}}%
\pgfpathcurveto{\pgfqpoint{1.748828in}{3.272743in}}{\pgfqpoint{1.738229in}{3.277134in}}{\pgfqpoint{1.727179in}{3.277134in}}%
\pgfpathcurveto{\pgfqpoint{1.716129in}{3.277134in}}{\pgfqpoint{1.705530in}{3.272743in}}{\pgfqpoint{1.697716in}{3.264930in}}%
\pgfpathcurveto{\pgfqpoint{1.689903in}{3.257116in}}{\pgfqpoint{1.685512in}{3.246517in}}{\pgfqpoint{1.685512in}{3.235467in}}%
\pgfpathcurveto{\pgfqpoint{1.685512in}{3.224417in}}{\pgfqpoint{1.689903in}{3.213818in}}{\pgfqpoint{1.697716in}{3.206004in}}%
\pgfpathcurveto{\pgfqpoint{1.705530in}{3.198191in}}{\pgfqpoint{1.716129in}{3.193800in}}{\pgfqpoint{1.727179in}{3.193800in}}%
\pgfpathclose%
\pgfusepath{stroke,fill}%
\end{pgfscope}%
\begin{pgfscope}%
\pgfpathrectangle{\pgfqpoint{0.648703in}{0.548769in}}{\pgfqpoint{5.201297in}{3.102590in}}%
\pgfusepath{clip}%
\pgfsetbuttcap%
\pgfsetroundjoin%
\definecolor{currentfill}{rgb}{0.121569,0.466667,0.705882}%
\pgfsetfillcolor{currentfill}%
\pgfsetlinewidth{1.003750pt}%
\definecolor{currentstroke}{rgb}{0.121569,0.466667,0.705882}%
\pgfsetstrokecolor{currentstroke}%
\pgfsetdash{}{0pt}%
\pgfpathmoveto{\pgfqpoint{1.986272in}{0.652358in}}%
\pgfpathcurveto{\pgfqpoint{1.997322in}{0.652358in}}{\pgfqpoint{2.007921in}{0.656748in}}{\pgfqpoint{2.015735in}{0.664562in}}%
\pgfpathcurveto{\pgfqpoint{2.023549in}{0.672375in}}{\pgfqpoint{2.027939in}{0.682974in}}{\pgfqpoint{2.027939in}{0.694024in}}%
\pgfpathcurveto{\pgfqpoint{2.027939in}{0.705074in}}{\pgfqpoint{2.023549in}{0.715673in}}{\pgfqpoint{2.015735in}{0.723487in}}%
\pgfpathcurveto{\pgfqpoint{2.007921in}{0.731301in}}{\pgfqpoint{1.997322in}{0.735691in}}{\pgfqpoint{1.986272in}{0.735691in}}%
\pgfpathcurveto{\pgfqpoint{1.975222in}{0.735691in}}{\pgfqpoint{1.964623in}{0.731301in}}{\pgfqpoint{1.956809in}{0.723487in}}%
\pgfpathcurveto{\pgfqpoint{1.948996in}{0.715673in}}{\pgfqpoint{1.944606in}{0.705074in}}{\pgfqpoint{1.944606in}{0.694024in}}%
\pgfpathcurveto{\pgfqpoint{1.944606in}{0.682974in}}{\pgfqpoint{1.948996in}{0.672375in}}{\pgfqpoint{1.956809in}{0.664562in}}%
\pgfpathcurveto{\pgfqpoint{1.964623in}{0.656748in}}{\pgfqpoint{1.975222in}{0.652358in}}{\pgfqpoint{1.986272in}{0.652358in}}%
\pgfpathclose%
\pgfusepath{stroke,fill}%
\end{pgfscope}%
\begin{pgfscope}%
\pgfpathrectangle{\pgfqpoint{0.648703in}{0.548769in}}{\pgfqpoint{5.201297in}{3.102590in}}%
\pgfusepath{clip}%
\pgfsetbuttcap%
\pgfsetroundjoin%
\definecolor{currentfill}{rgb}{0.121569,0.466667,0.705882}%
\pgfsetfillcolor{currentfill}%
\pgfsetlinewidth{1.003750pt}%
\definecolor{currentstroke}{rgb}{0.121569,0.466667,0.705882}%
\pgfsetstrokecolor{currentstroke}%
\pgfsetdash{}{0pt}%
\pgfpathmoveto{\pgfqpoint{1.079446in}{0.648129in}}%
\pgfpathcurveto{\pgfqpoint{1.090496in}{0.648129in}}{\pgfqpoint{1.101095in}{0.652519in}}{\pgfqpoint{1.108909in}{0.660333in}}%
\pgfpathcurveto{\pgfqpoint{1.116722in}{0.668146in}}{\pgfqpoint{1.121113in}{0.678745in}}{\pgfqpoint{1.121113in}{0.689796in}}%
\pgfpathcurveto{\pgfqpoint{1.121113in}{0.700846in}}{\pgfqpoint{1.116722in}{0.711445in}}{\pgfqpoint{1.108909in}{0.719258in}}%
\pgfpathcurveto{\pgfqpoint{1.101095in}{0.727072in}}{\pgfqpoint{1.090496in}{0.731462in}}{\pgfqpoint{1.079446in}{0.731462in}}%
\pgfpathcurveto{\pgfqpoint{1.068396in}{0.731462in}}{\pgfqpoint{1.057797in}{0.727072in}}{\pgfqpoint{1.049983in}{0.719258in}}%
\pgfpathcurveto{\pgfqpoint{1.042170in}{0.711445in}}{\pgfqpoint{1.037779in}{0.700846in}}{\pgfqpoint{1.037779in}{0.689796in}}%
\pgfpathcurveto{\pgfqpoint{1.037779in}{0.678745in}}{\pgfqpoint{1.042170in}{0.668146in}}{\pgfqpoint{1.049983in}{0.660333in}}%
\pgfpathcurveto{\pgfqpoint{1.057797in}{0.652519in}}{\pgfqpoint{1.068396in}{0.648129in}}{\pgfqpoint{1.079446in}{0.648129in}}%
\pgfpathclose%
\pgfusepath{stroke,fill}%
\end{pgfscope}%
\begin{pgfscope}%
\pgfpathrectangle{\pgfqpoint{0.648703in}{0.548769in}}{\pgfqpoint{5.201297in}{3.102590in}}%
\pgfusepath{clip}%
\pgfsetbuttcap%
\pgfsetroundjoin%
\definecolor{currentfill}{rgb}{0.121569,0.466667,0.705882}%
\pgfsetfillcolor{currentfill}%
\pgfsetlinewidth{1.003750pt}%
\definecolor{currentstroke}{rgb}{0.121569,0.466667,0.705882}%
\pgfsetstrokecolor{currentstroke}%
\pgfsetdash{}{0pt}%
\pgfpathmoveto{\pgfqpoint{1.079446in}{0.648129in}}%
\pgfpathcurveto{\pgfqpoint{1.090496in}{0.648129in}}{\pgfqpoint{1.101095in}{0.652519in}}{\pgfqpoint{1.108909in}{0.660333in}}%
\pgfpathcurveto{\pgfqpoint{1.116722in}{0.668146in}}{\pgfqpoint{1.121113in}{0.678745in}}{\pgfqpoint{1.121113in}{0.689796in}}%
\pgfpathcurveto{\pgfqpoint{1.121113in}{0.700846in}}{\pgfqpoint{1.116722in}{0.711445in}}{\pgfqpoint{1.108909in}{0.719258in}}%
\pgfpathcurveto{\pgfqpoint{1.101095in}{0.727072in}}{\pgfqpoint{1.090496in}{0.731462in}}{\pgfqpoint{1.079446in}{0.731462in}}%
\pgfpathcurveto{\pgfqpoint{1.068396in}{0.731462in}}{\pgfqpoint{1.057797in}{0.727072in}}{\pgfqpoint{1.049983in}{0.719258in}}%
\pgfpathcurveto{\pgfqpoint{1.042170in}{0.711445in}}{\pgfqpoint{1.037779in}{0.700846in}}{\pgfqpoint{1.037779in}{0.689796in}}%
\pgfpathcurveto{\pgfqpoint{1.037779in}{0.678745in}}{\pgfqpoint{1.042170in}{0.668146in}}{\pgfqpoint{1.049983in}{0.660333in}}%
\pgfpathcurveto{\pgfqpoint{1.057797in}{0.652519in}}{\pgfqpoint{1.068396in}{0.648129in}}{\pgfqpoint{1.079446in}{0.648129in}}%
\pgfpathclose%
\pgfusepath{stroke,fill}%
\end{pgfscope}%
\begin{pgfscope}%
\pgfpathrectangle{\pgfqpoint{0.648703in}{0.548769in}}{\pgfqpoint{5.201297in}{3.102590in}}%
\pgfusepath{clip}%
\pgfsetbuttcap%
\pgfsetroundjoin%
\definecolor{currentfill}{rgb}{0.121569,0.466667,0.705882}%
\pgfsetfillcolor{currentfill}%
\pgfsetlinewidth{1.003750pt}%
\definecolor{currentstroke}{rgb}{0.121569,0.466667,0.705882}%
\pgfsetstrokecolor{currentstroke}%
\pgfsetdash{}{0pt}%
\pgfpathmoveto{\pgfqpoint{1.532859in}{0.648129in}}%
\pgfpathcurveto{\pgfqpoint{1.543909in}{0.648129in}}{\pgfqpoint{1.554508in}{0.652519in}}{\pgfqpoint{1.562322in}{0.660333in}}%
\pgfpathcurveto{\pgfqpoint{1.570135in}{0.668146in}}{\pgfqpoint{1.574526in}{0.678745in}}{\pgfqpoint{1.574526in}{0.689796in}}%
\pgfpathcurveto{\pgfqpoint{1.574526in}{0.700846in}}{\pgfqpoint{1.570135in}{0.711445in}}{\pgfqpoint{1.562322in}{0.719258in}}%
\pgfpathcurveto{\pgfqpoint{1.554508in}{0.727072in}}{\pgfqpoint{1.543909in}{0.731462in}}{\pgfqpoint{1.532859in}{0.731462in}}%
\pgfpathcurveto{\pgfqpoint{1.521809in}{0.731462in}}{\pgfqpoint{1.511210in}{0.727072in}}{\pgfqpoint{1.503396in}{0.719258in}}%
\pgfpathcurveto{\pgfqpoint{1.495583in}{0.711445in}}{\pgfqpoint{1.491192in}{0.700846in}}{\pgfqpoint{1.491192in}{0.689796in}}%
\pgfpathcurveto{\pgfqpoint{1.491192in}{0.678745in}}{\pgfqpoint{1.495583in}{0.668146in}}{\pgfqpoint{1.503396in}{0.660333in}}%
\pgfpathcurveto{\pgfqpoint{1.511210in}{0.652519in}}{\pgfqpoint{1.521809in}{0.648129in}}{\pgfqpoint{1.532859in}{0.648129in}}%
\pgfpathclose%
\pgfusepath{stroke,fill}%
\end{pgfscope}%
\begin{pgfscope}%
\pgfpathrectangle{\pgfqpoint{0.648703in}{0.548769in}}{\pgfqpoint{5.201297in}{3.102590in}}%
\pgfusepath{clip}%
\pgfsetbuttcap%
\pgfsetroundjoin%
\definecolor{currentfill}{rgb}{0.121569,0.466667,0.705882}%
\pgfsetfillcolor{currentfill}%
\pgfsetlinewidth{1.003750pt}%
\definecolor{currentstroke}{rgb}{0.121569,0.466667,0.705882}%
\pgfsetstrokecolor{currentstroke}%
\pgfsetdash{}{0pt}%
\pgfpathmoveto{\pgfqpoint{0.949899in}{0.648129in}}%
\pgfpathcurveto{\pgfqpoint{0.960949in}{0.648129in}}{\pgfqpoint{0.971548in}{0.652519in}}{\pgfqpoint{0.979362in}{0.660333in}}%
\pgfpathcurveto{\pgfqpoint{0.987176in}{0.668146in}}{\pgfqpoint{0.991566in}{0.678745in}}{\pgfqpoint{0.991566in}{0.689796in}}%
\pgfpathcurveto{\pgfqpoint{0.991566in}{0.700846in}}{\pgfqpoint{0.987176in}{0.711445in}}{\pgfqpoint{0.979362in}{0.719258in}}%
\pgfpathcurveto{\pgfqpoint{0.971548in}{0.727072in}}{\pgfqpoint{0.960949in}{0.731462in}}{\pgfqpoint{0.949899in}{0.731462in}}%
\pgfpathcurveto{\pgfqpoint{0.938849in}{0.731462in}}{\pgfqpoint{0.928250in}{0.727072in}}{\pgfqpoint{0.920437in}{0.719258in}}%
\pgfpathcurveto{\pgfqpoint{0.912623in}{0.711445in}}{\pgfqpoint{0.908233in}{0.700846in}}{\pgfqpoint{0.908233in}{0.689796in}}%
\pgfpathcurveto{\pgfqpoint{0.908233in}{0.678745in}}{\pgfqpoint{0.912623in}{0.668146in}}{\pgfqpoint{0.920437in}{0.660333in}}%
\pgfpathcurveto{\pgfqpoint{0.928250in}{0.652519in}}{\pgfqpoint{0.938849in}{0.648129in}}{\pgfqpoint{0.949899in}{0.648129in}}%
\pgfpathclose%
\pgfusepath{stroke,fill}%
\end{pgfscope}%
\begin{pgfscope}%
\pgfpathrectangle{\pgfqpoint{0.648703in}{0.548769in}}{\pgfqpoint{5.201297in}{3.102590in}}%
\pgfusepath{clip}%
\pgfsetbuttcap%
\pgfsetroundjoin%
\definecolor{currentfill}{rgb}{0.121569,0.466667,0.705882}%
\pgfsetfillcolor{currentfill}%
\pgfsetlinewidth{1.003750pt}%
\definecolor{currentstroke}{rgb}{0.121569,0.466667,0.705882}%
\pgfsetstrokecolor{currentstroke}%
\pgfsetdash{}{0pt}%
\pgfpathmoveto{\pgfqpoint{1.208993in}{0.648129in}}%
\pgfpathcurveto{\pgfqpoint{1.220043in}{0.648129in}}{\pgfqpoint{1.230642in}{0.652519in}}{\pgfqpoint{1.238455in}{0.660333in}}%
\pgfpathcurveto{\pgfqpoint{1.246269in}{0.668146in}}{\pgfqpoint{1.250659in}{0.678745in}}{\pgfqpoint{1.250659in}{0.689796in}}%
\pgfpathcurveto{\pgfqpoint{1.250659in}{0.700846in}}{\pgfqpoint{1.246269in}{0.711445in}}{\pgfqpoint{1.238455in}{0.719258in}}%
\pgfpathcurveto{\pgfqpoint{1.230642in}{0.727072in}}{\pgfqpoint{1.220043in}{0.731462in}}{\pgfqpoint{1.208993in}{0.731462in}}%
\pgfpathcurveto{\pgfqpoint{1.197942in}{0.731462in}}{\pgfqpoint{1.187343in}{0.727072in}}{\pgfqpoint{1.179530in}{0.719258in}}%
\pgfpathcurveto{\pgfqpoint{1.171716in}{0.711445in}}{\pgfqpoint{1.167326in}{0.700846in}}{\pgfqpoint{1.167326in}{0.689796in}}%
\pgfpathcurveto{\pgfqpoint{1.167326in}{0.678745in}}{\pgfqpoint{1.171716in}{0.668146in}}{\pgfqpoint{1.179530in}{0.660333in}}%
\pgfpathcurveto{\pgfqpoint{1.187343in}{0.652519in}}{\pgfqpoint{1.197942in}{0.648129in}}{\pgfqpoint{1.208993in}{0.648129in}}%
\pgfpathclose%
\pgfusepath{stroke,fill}%
\end{pgfscope}%
\begin{pgfscope}%
\pgfpathrectangle{\pgfqpoint{0.648703in}{0.548769in}}{\pgfqpoint{5.201297in}{3.102590in}}%
\pgfusepath{clip}%
\pgfsetbuttcap%
\pgfsetroundjoin%
\definecolor{currentfill}{rgb}{0.121569,0.466667,0.705882}%
\pgfsetfillcolor{currentfill}%
\pgfsetlinewidth{1.003750pt}%
\definecolor{currentstroke}{rgb}{0.121569,0.466667,0.705882}%
\pgfsetstrokecolor{currentstroke}%
\pgfsetdash{}{0pt}%
\pgfpathmoveto{\pgfqpoint{2.439685in}{3.181114in}}%
\pgfpathcurveto{\pgfqpoint{2.450735in}{3.181114in}}{\pgfqpoint{2.461335in}{3.185504in}}{\pgfqpoint{2.469148in}{3.193318in}}%
\pgfpathcurveto{\pgfqpoint{2.476962in}{3.201132in}}{\pgfqpoint{2.481352in}{3.211731in}}{\pgfqpoint{2.481352in}{3.222781in}}%
\pgfpathcurveto{\pgfqpoint{2.481352in}{3.233831in}}{\pgfqpoint{2.476962in}{3.244430in}}{\pgfqpoint{2.469148in}{3.252244in}}%
\pgfpathcurveto{\pgfqpoint{2.461335in}{3.260057in}}{\pgfqpoint{2.450735in}{3.264448in}}{\pgfqpoint{2.439685in}{3.264448in}}%
\pgfpathcurveto{\pgfqpoint{2.428635in}{3.264448in}}{\pgfqpoint{2.418036in}{3.260057in}}{\pgfqpoint{2.410223in}{3.252244in}}%
\pgfpathcurveto{\pgfqpoint{2.402409in}{3.244430in}}{\pgfqpoint{2.398019in}{3.233831in}}{\pgfqpoint{2.398019in}{3.222781in}}%
\pgfpathcurveto{\pgfqpoint{2.398019in}{3.211731in}}{\pgfqpoint{2.402409in}{3.201132in}}{\pgfqpoint{2.410223in}{3.193318in}}%
\pgfpathcurveto{\pgfqpoint{2.418036in}{3.185504in}}{\pgfqpoint{2.428635in}{3.181114in}}{\pgfqpoint{2.439685in}{3.181114in}}%
\pgfpathclose%
\pgfusepath{stroke,fill}%
\end{pgfscope}%
\begin{pgfscope}%
\pgfpathrectangle{\pgfqpoint{0.648703in}{0.548769in}}{\pgfqpoint{5.201297in}{3.102590in}}%
\pgfusepath{clip}%
\pgfsetbuttcap%
\pgfsetroundjoin%
\definecolor{currentfill}{rgb}{1.000000,0.498039,0.054902}%
\pgfsetfillcolor{currentfill}%
\pgfsetlinewidth{1.003750pt}%
\definecolor{currentstroke}{rgb}{1.000000,0.498039,0.054902}%
\pgfsetstrokecolor{currentstroke}%
\pgfsetdash{}{0pt}%
\pgfpathmoveto{\pgfqpoint{2.310139in}{3.193800in}}%
\pgfpathcurveto{\pgfqpoint{2.321189in}{3.193800in}}{\pgfqpoint{2.331788in}{3.198191in}}{\pgfqpoint{2.339602in}{3.206004in}}%
\pgfpathcurveto{\pgfqpoint{2.347415in}{3.213818in}}{\pgfqpoint{2.351805in}{3.224417in}}{\pgfqpoint{2.351805in}{3.235467in}}%
\pgfpathcurveto{\pgfqpoint{2.351805in}{3.246517in}}{\pgfqpoint{2.347415in}{3.257116in}}{\pgfqpoint{2.339602in}{3.264930in}}%
\pgfpathcurveto{\pgfqpoint{2.331788in}{3.272743in}}{\pgfqpoint{2.321189in}{3.277134in}}{\pgfqpoint{2.310139in}{3.277134in}}%
\pgfpathcurveto{\pgfqpoint{2.299089in}{3.277134in}}{\pgfqpoint{2.288490in}{3.272743in}}{\pgfqpoint{2.280676in}{3.264930in}}%
\pgfpathcurveto{\pgfqpoint{2.272862in}{3.257116in}}{\pgfqpoint{2.268472in}{3.246517in}}{\pgfqpoint{2.268472in}{3.235467in}}%
\pgfpathcurveto{\pgfqpoint{2.268472in}{3.224417in}}{\pgfqpoint{2.272862in}{3.213818in}}{\pgfqpoint{2.280676in}{3.206004in}}%
\pgfpathcurveto{\pgfqpoint{2.288490in}{3.198191in}}{\pgfqpoint{2.299089in}{3.193800in}}{\pgfqpoint{2.310139in}{3.193800in}}%
\pgfpathclose%
\pgfusepath{stroke,fill}%
\end{pgfscope}%
\begin{pgfscope}%
\pgfpathrectangle{\pgfqpoint{0.648703in}{0.548769in}}{\pgfqpoint{5.201297in}{3.102590in}}%
\pgfusepath{clip}%
\pgfsetbuttcap%
\pgfsetroundjoin%
\definecolor{currentfill}{rgb}{1.000000,0.498039,0.054902}%
\pgfsetfillcolor{currentfill}%
\pgfsetlinewidth{1.003750pt}%
\definecolor{currentstroke}{rgb}{1.000000,0.498039,0.054902}%
\pgfsetstrokecolor{currentstroke}%
\pgfsetdash{}{0pt}%
\pgfpathmoveto{\pgfqpoint{2.569232in}{3.193800in}}%
\pgfpathcurveto{\pgfqpoint{2.580282in}{3.193800in}}{\pgfqpoint{2.590881in}{3.198191in}}{\pgfqpoint{2.598695in}{3.206004in}}%
\pgfpathcurveto{\pgfqpoint{2.606508in}{3.213818in}}{\pgfqpoint{2.610899in}{3.224417in}}{\pgfqpoint{2.610899in}{3.235467in}}%
\pgfpathcurveto{\pgfqpoint{2.610899in}{3.246517in}}{\pgfqpoint{2.606508in}{3.257116in}}{\pgfqpoint{2.598695in}{3.264930in}}%
\pgfpathcurveto{\pgfqpoint{2.590881in}{3.272743in}}{\pgfqpoint{2.580282in}{3.277134in}}{\pgfqpoint{2.569232in}{3.277134in}}%
\pgfpathcurveto{\pgfqpoint{2.558182in}{3.277134in}}{\pgfqpoint{2.547583in}{3.272743in}}{\pgfqpoint{2.539769in}{3.264930in}}%
\pgfpathcurveto{\pgfqpoint{2.531956in}{3.257116in}}{\pgfqpoint{2.527565in}{3.246517in}}{\pgfqpoint{2.527565in}{3.235467in}}%
\pgfpathcurveto{\pgfqpoint{2.527565in}{3.224417in}}{\pgfqpoint{2.531956in}{3.213818in}}{\pgfqpoint{2.539769in}{3.206004in}}%
\pgfpathcurveto{\pgfqpoint{2.547583in}{3.198191in}}{\pgfqpoint{2.558182in}{3.193800in}}{\pgfqpoint{2.569232in}{3.193800in}}%
\pgfpathclose%
\pgfusepath{stroke,fill}%
\end{pgfscope}%
\begin{pgfscope}%
\pgfpathrectangle{\pgfqpoint{0.648703in}{0.548769in}}{\pgfqpoint{5.201297in}{3.102590in}}%
\pgfusepath{clip}%
\pgfsetbuttcap%
\pgfsetroundjoin%
\definecolor{currentfill}{rgb}{0.121569,0.466667,0.705882}%
\pgfsetfillcolor{currentfill}%
\pgfsetlinewidth{1.003750pt}%
\definecolor{currentstroke}{rgb}{0.121569,0.466667,0.705882}%
\pgfsetstrokecolor{currentstroke}%
\pgfsetdash{}{0pt}%
\pgfpathmoveto{\pgfqpoint{0.949899in}{0.648129in}}%
\pgfpathcurveto{\pgfqpoint{0.960949in}{0.648129in}}{\pgfqpoint{0.971548in}{0.652519in}}{\pgfqpoint{0.979362in}{0.660333in}}%
\pgfpathcurveto{\pgfqpoint{0.987176in}{0.668146in}}{\pgfqpoint{0.991566in}{0.678745in}}{\pgfqpoint{0.991566in}{0.689796in}}%
\pgfpathcurveto{\pgfqpoint{0.991566in}{0.700846in}}{\pgfqpoint{0.987176in}{0.711445in}}{\pgfqpoint{0.979362in}{0.719258in}}%
\pgfpathcurveto{\pgfqpoint{0.971548in}{0.727072in}}{\pgfqpoint{0.960949in}{0.731462in}}{\pgfqpoint{0.949899in}{0.731462in}}%
\pgfpathcurveto{\pgfqpoint{0.938849in}{0.731462in}}{\pgfqpoint{0.928250in}{0.727072in}}{\pgfqpoint{0.920437in}{0.719258in}}%
\pgfpathcurveto{\pgfqpoint{0.912623in}{0.711445in}}{\pgfqpoint{0.908233in}{0.700846in}}{\pgfqpoint{0.908233in}{0.689796in}}%
\pgfpathcurveto{\pgfqpoint{0.908233in}{0.678745in}}{\pgfqpoint{0.912623in}{0.668146in}}{\pgfqpoint{0.920437in}{0.660333in}}%
\pgfpathcurveto{\pgfqpoint{0.928250in}{0.652519in}}{\pgfqpoint{0.938849in}{0.648129in}}{\pgfqpoint{0.949899in}{0.648129in}}%
\pgfpathclose%
\pgfusepath{stroke,fill}%
\end{pgfscope}%
\begin{pgfscope}%
\pgfpathrectangle{\pgfqpoint{0.648703in}{0.548769in}}{\pgfqpoint{5.201297in}{3.102590in}}%
\pgfusepath{clip}%
\pgfsetbuttcap%
\pgfsetroundjoin%
\definecolor{currentfill}{rgb}{0.121569,0.466667,0.705882}%
\pgfsetfillcolor{currentfill}%
\pgfsetlinewidth{1.003750pt}%
\definecolor{currentstroke}{rgb}{0.121569,0.466667,0.705882}%
\pgfsetstrokecolor{currentstroke}%
\pgfsetdash{}{0pt}%
\pgfpathmoveto{\pgfqpoint{1.014673in}{0.665044in}}%
\pgfpathcurveto{\pgfqpoint{1.025723in}{0.665044in}}{\pgfqpoint{1.036322in}{0.669434in}}{\pgfqpoint{1.044135in}{0.677248in}}%
\pgfpathcurveto{\pgfqpoint{1.051949in}{0.685061in}}{\pgfqpoint{1.056339in}{0.695660in}}{\pgfqpoint{1.056339in}{0.706710in}}%
\pgfpathcurveto{\pgfqpoint{1.056339in}{0.717760in}}{\pgfqpoint{1.051949in}{0.728360in}}{\pgfqpoint{1.044135in}{0.736173in}}%
\pgfpathcurveto{\pgfqpoint{1.036322in}{0.743987in}}{\pgfqpoint{1.025723in}{0.748377in}}{\pgfqpoint{1.014673in}{0.748377in}}%
\pgfpathcurveto{\pgfqpoint{1.003622in}{0.748377in}}{\pgfqpoint{0.993023in}{0.743987in}}{\pgfqpoint{0.985210in}{0.736173in}}%
\pgfpathcurveto{\pgfqpoint{0.977396in}{0.728360in}}{\pgfqpoint{0.973006in}{0.717760in}}{\pgfqpoint{0.973006in}{0.706710in}}%
\pgfpathcurveto{\pgfqpoint{0.973006in}{0.695660in}}{\pgfqpoint{0.977396in}{0.685061in}}{\pgfqpoint{0.985210in}{0.677248in}}%
\pgfpathcurveto{\pgfqpoint{0.993023in}{0.669434in}}{\pgfqpoint{1.003622in}{0.665044in}}{\pgfqpoint{1.014673in}{0.665044in}}%
\pgfpathclose%
\pgfusepath{stroke,fill}%
\end{pgfscope}%
\begin{pgfscope}%
\pgfpathrectangle{\pgfqpoint{0.648703in}{0.548769in}}{\pgfqpoint{5.201297in}{3.102590in}}%
\pgfusepath{clip}%
\pgfsetbuttcap%
\pgfsetroundjoin%
\definecolor{currentfill}{rgb}{1.000000,0.498039,0.054902}%
\pgfsetfillcolor{currentfill}%
\pgfsetlinewidth{1.003750pt}%
\definecolor{currentstroke}{rgb}{1.000000,0.498039,0.054902}%
\pgfsetstrokecolor{currentstroke}%
\pgfsetdash{}{0pt}%
\pgfpathmoveto{\pgfqpoint{2.374912in}{3.189572in}}%
\pgfpathcurveto{\pgfqpoint{2.385962in}{3.189572in}}{\pgfqpoint{2.396561in}{3.193962in}}{\pgfqpoint{2.404375in}{3.201775in}}%
\pgfpathcurveto{\pgfqpoint{2.412188in}{3.209589in}}{\pgfqpoint{2.416579in}{3.220188in}}{\pgfqpoint{2.416579in}{3.231238in}}%
\pgfpathcurveto{\pgfqpoint{2.416579in}{3.242288in}}{\pgfqpoint{2.412188in}{3.252887in}}{\pgfqpoint{2.404375in}{3.260701in}}%
\pgfpathcurveto{\pgfqpoint{2.396561in}{3.268515in}}{\pgfqpoint{2.385962in}{3.272905in}}{\pgfqpoint{2.374912in}{3.272905in}}%
\pgfpathcurveto{\pgfqpoint{2.363862in}{3.272905in}}{\pgfqpoint{2.353263in}{3.268515in}}{\pgfqpoint{2.345449in}{3.260701in}}%
\pgfpathcurveto{\pgfqpoint{2.337636in}{3.252887in}}{\pgfqpoint{2.333245in}{3.242288in}}{\pgfqpoint{2.333245in}{3.231238in}}%
\pgfpathcurveto{\pgfqpoint{2.333245in}{3.220188in}}{\pgfqpoint{2.337636in}{3.209589in}}{\pgfqpoint{2.345449in}{3.201775in}}%
\pgfpathcurveto{\pgfqpoint{2.353263in}{3.193962in}}{\pgfqpoint{2.363862in}{3.189572in}}{\pgfqpoint{2.374912in}{3.189572in}}%
\pgfpathclose%
\pgfusepath{stroke,fill}%
\end{pgfscope}%
\begin{pgfscope}%
\pgfpathrectangle{\pgfqpoint{0.648703in}{0.548769in}}{\pgfqpoint{5.201297in}{3.102590in}}%
\pgfusepath{clip}%
\pgfsetbuttcap%
\pgfsetroundjoin%
\definecolor{currentfill}{rgb}{0.121569,0.466667,0.705882}%
\pgfsetfillcolor{currentfill}%
\pgfsetlinewidth{1.003750pt}%
\definecolor{currentstroke}{rgb}{0.121569,0.466667,0.705882}%
\pgfsetstrokecolor{currentstroke}%
\pgfsetdash{}{0pt}%
\pgfpathmoveto{\pgfqpoint{1.273766in}{0.808819in}}%
\pgfpathcurveto{\pgfqpoint{1.284816in}{0.808819in}}{\pgfqpoint{1.295415in}{0.813209in}}{\pgfqpoint{1.303229in}{0.821023in}}%
\pgfpathcurveto{\pgfqpoint{1.311042in}{0.828837in}}{\pgfqpoint{1.315432in}{0.839436in}}{\pgfqpoint{1.315432in}{0.850486in}}%
\pgfpathcurveto{\pgfqpoint{1.315432in}{0.861536in}}{\pgfqpoint{1.311042in}{0.872135in}}{\pgfqpoint{1.303229in}{0.879949in}}%
\pgfpathcurveto{\pgfqpoint{1.295415in}{0.887762in}}{\pgfqpoint{1.284816in}{0.892152in}}{\pgfqpoint{1.273766in}{0.892152in}}%
\pgfpathcurveto{\pgfqpoint{1.262716in}{0.892152in}}{\pgfqpoint{1.252117in}{0.887762in}}{\pgfqpoint{1.244303in}{0.879949in}}%
\pgfpathcurveto{\pgfqpoint{1.236489in}{0.872135in}}{\pgfqpoint{1.232099in}{0.861536in}}{\pgfqpoint{1.232099in}{0.850486in}}%
\pgfpathcurveto{\pgfqpoint{1.232099in}{0.839436in}}{\pgfqpoint{1.236489in}{0.828837in}}{\pgfqpoint{1.244303in}{0.821023in}}%
\pgfpathcurveto{\pgfqpoint{1.252117in}{0.813209in}}{\pgfqpoint{1.262716in}{0.808819in}}{\pgfqpoint{1.273766in}{0.808819in}}%
\pgfpathclose%
\pgfusepath{stroke,fill}%
\end{pgfscope}%
\begin{pgfscope}%
\pgfpathrectangle{\pgfqpoint{0.648703in}{0.548769in}}{\pgfqpoint{5.201297in}{3.102590in}}%
\pgfusepath{clip}%
\pgfsetbuttcap%
\pgfsetroundjoin%
\definecolor{currentfill}{rgb}{0.839216,0.152941,0.156863}%
\pgfsetfillcolor{currentfill}%
\pgfsetlinewidth{1.003750pt}%
\definecolor{currentstroke}{rgb}{0.839216,0.152941,0.156863}%
\pgfsetstrokecolor{currentstroke}%
\pgfsetdash{}{0pt}%
\pgfpathmoveto{\pgfqpoint{1.727179in}{3.198029in}}%
\pgfpathcurveto{\pgfqpoint{1.738229in}{3.198029in}}{\pgfqpoint{1.748828in}{3.202419in}}{\pgfqpoint{1.756642in}{3.210233in}}%
\pgfpathcurveto{\pgfqpoint{1.764455in}{3.218046in}}{\pgfqpoint{1.768846in}{3.228646in}}{\pgfqpoint{1.768846in}{3.239696in}}%
\pgfpathcurveto{\pgfqpoint{1.768846in}{3.250746in}}{\pgfqpoint{1.764455in}{3.261345in}}{\pgfqpoint{1.756642in}{3.269158in}}%
\pgfpathcurveto{\pgfqpoint{1.748828in}{3.276972in}}{\pgfqpoint{1.738229in}{3.281362in}}{\pgfqpoint{1.727179in}{3.281362in}}%
\pgfpathcurveto{\pgfqpoint{1.716129in}{3.281362in}}{\pgfqpoint{1.705530in}{3.276972in}}{\pgfqpoint{1.697716in}{3.269158in}}%
\pgfpathcurveto{\pgfqpoint{1.689903in}{3.261345in}}{\pgfqpoint{1.685512in}{3.250746in}}{\pgfqpoint{1.685512in}{3.239696in}}%
\pgfpathcurveto{\pgfqpoint{1.685512in}{3.228646in}}{\pgfqpoint{1.689903in}{3.218046in}}{\pgfqpoint{1.697716in}{3.210233in}}%
\pgfpathcurveto{\pgfqpoint{1.705530in}{3.202419in}}{\pgfqpoint{1.716129in}{3.198029in}}{\pgfqpoint{1.727179in}{3.198029in}}%
\pgfpathclose%
\pgfusepath{stroke,fill}%
\end{pgfscope}%
\begin{pgfscope}%
\pgfpathrectangle{\pgfqpoint{0.648703in}{0.548769in}}{\pgfqpoint{5.201297in}{3.102590in}}%
\pgfusepath{clip}%
\pgfsetbuttcap%
\pgfsetroundjoin%
\definecolor{currentfill}{rgb}{0.121569,0.466667,0.705882}%
\pgfsetfillcolor{currentfill}%
\pgfsetlinewidth{1.003750pt}%
\definecolor{currentstroke}{rgb}{0.121569,0.466667,0.705882}%
\pgfsetstrokecolor{currentstroke}%
\pgfsetdash{}{0pt}%
\pgfpathmoveto{\pgfqpoint{1.144219in}{0.648129in}}%
\pgfpathcurveto{\pgfqpoint{1.155269in}{0.648129in}}{\pgfqpoint{1.165868in}{0.652519in}}{\pgfqpoint{1.173682in}{0.660333in}}%
\pgfpathcurveto{\pgfqpoint{1.181496in}{0.668146in}}{\pgfqpoint{1.185886in}{0.678745in}}{\pgfqpoint{1.185886in}{0.689796in}}%
\pgfpathcurveto{\pgfqpoint{1.185886in}{0.700846in}}{\pgfqpoint{1.181496in}{0.711445in}}{\pgfqpoint{1.173682in}{0.719258in}}%
\pgfpathcurveto{\pgfqpoint{1.165868in}{0.727072in}}{\pgfqpoint{1.155269in}{0.731462in}}{\pgfqpoint{1.144219in}{0.731462in}}%
\pgfpathcurveto{\pgfqpoint{1.133169in}{0.731462in}}{\pgfqpoint{1.122570in}{0.727072in}}{\pgfqpoint{1.114756in}{0.719258in}}%
\pgfpathcurveto{\pgfqpoint{1.106943in}{0.711445in}}{\pgfqpoint{1.102553in}{0.700846in}}{\pgfqpoint{1.102553in}{0.689796in}}%
\pgfpathcurveto{\pgfqpoint{1.102553in}{0.678745in}}{\pgfqpoint{1.106943in}{0.668146in}}{\pgfqpoint{1.114756in}{0.660333in}}%
\pgfpathcurveto{\pgfqpoint{1.122570in}{0.652519in}}{\pgfqpoint{1.133169in}{0.648129in}}{\pgfqpoint{1.144219in}{0.648129in}}%
\pgfpathclose%
\pgfusepath{stroke,fill}%
\end{pgfscope}%
\begin{pgfscope}%
\pgfpathrectangle{\pgfqpoint{0.648703in}{0.548769in}}{\pgfqpoint{5.201297in}{3.102590in}}%
\pgfusepath{clip}%
\pgfsetbuttcap%
\pgfsetroundjoin%
\definecolor{currentfill}{rgb}{1.000000,0.498039,0.054902}%
\pgfsetfillcolor{currentfill}%
\pgfsetlinewidth{1.003750pt}%
\definecolor{currentstroke}{rgb}{1.000000,0.498039,0.054902}%
\pgfsetstrokecolor{currentstroke}%
\pgfsetdash{}{0pt}%
\pgfpathmoveto{\pgfqpoint{2.893099in}{3.185343in}}%
\pgfpathcurveto{\pgfqpoint{2.904149in}{3.185343in}}{\pgfqpoint{2.914748in}{3.189733in}}{\pgfqpoint{2.922561in}{3.197547in}}%
\pgfpathcurveto{\pgfqpoint{2.930375in}{3.205360in}}{\pgfqpoint{2.934765in}{3.215959in}}{\pgfqpoint{2.934765in}{3.227010in}}%
\pgfpathcurveto{\pgfqpoint{2.934765in}{3.238060in}}{\pgfqpoint{2.930375in}{3.248659in}}{\pgfqpoint{2.922561in}{3.256472in}}%
\pgfpathcurveto{\pgfqpoint{2.914748in}{3.264286in}}{\pgfqpoint{2.904149in}{3.268676in}}{\pgfqpoint{2.893099in}{3.268676in}}%
\pgfpathcurveto{\pgfqpoint{2.882048in}{3.268676in}}{\pgfqpoint{2.871449in}{3.264286in}}{\pgfqpoint{2.863636in}{3.256472in}}%
\pgfpathcurveto{\pgfqpoint{2.855822in}{3.248659in}}{\pgfqpoint{2.851432in}{3.238060in}}{\pgfqpoint{2.851432in}{3.227010in}}%
\pgfpathcurveto{\pgfqpoint{2.851432in}{3.215959in}}{\pgfqpoint{2.855822in}{3.205360in}}{\pgfqpoint{2.863636in}{3.197547in}}%
\pgfpathcurveto{\pgfqpoint{2.871449in}{3.189733in}}{\pgfqpoint{2.882048in}{3.185343in}}{\pgfqpoint{2.893099in}{3.185343in}}%
\pgfpathclose%
\pgfusepath{stroke,fill}%
\end{pgfscope}%
\begin{pgfscope}%
\pgfpathrectangle{\pgfqpoint{0.648703in}{0.548769in}}{\pgfqpoint{5.201297in}{3.102590in}}%
\pgfusepath{clip}%
\pgfsetbuttcap%
\pgfsetroundjoin%
\definecolor{currentfill}{rgb}{0.121569,0.466667,0.705882}%
\pgfsetfillcolor{currentfill}%
\pgfsetlinewidth{1.003750pt}%
\definecolor{currentstroke}{rgb}{0.121569,0.466667,0.705882}%
\pgfsetstrokecolor{currentstroke}%
\pgfsetdash{}{0pt}%
\pgfpathmoveto{\pgfqpoint{0.949899in}{0.648129in}}%
\pgfpathcurveto{\pgfqpoint{0.960949in}{0.648129in}}{\pgfqpoint{0.971548in}{0.652519in}}{\pgfqpoint{0.979362in}{0.660333in}}%
\pgfpathcurveto{\pgfqpoint{0.987176in}{0.668146in}}{\pgfqpoint{0.991566in}{0.678745in}}{\pgfqpoint{0.991566in}{0.689796in}}%
\pgfpathcurveto{\pgfqpoint{0.991566in}{0.700846in}}{\pgfqpoint{0.987176in}{0.711445in}}{\pgfqpoint{0.979362in}{0.719258in}}%
\pgfpathcurveto{\pgfqpoint{0.971548in}{0.727072in}}{\pgfqpoint{0.960949in}{0.731462in}}{\pgfqpoint{0.949899in}{0.731462in}}%
\pgfpathcurveto{\pgfqpoint{0.938849in}{0.731462in}}{\pgfqpoint{0.928250in}{0.727072in}}{\pgfqpoint{0.920437in}{0.719258in}}%
\pgfpathcurveto{\pgfqpoint{0.912623in}{0.711445in}}{\pgfqpoint{0.908233in}{0.700846in}}{\pgfqpoint{0.908233in}{0.689796in}}%
\pgfpathcurveto{\pgfqpoint{0.908233in}{0.678745in}}{\pgfqpoint{0.912623in}{0.668146in}}{\pgfqpoint{0.920437in}{0.660333in}}%
\pgfpathcurveto{\pgfqpoint{0.928250in}{0.652519in}}{\pgfqpoint{0.938849in}{0.648129in}}{\pgfqpoint{0.949899in}{0.648129in}}%
\pgfpathclose%
\pgfusepath{stroke,fill}%
\end{pgfscope}%
\begin{pgfscope}%
\pgfpathrectangle{\pgfqpoint{0.648703in}{0.548769in}}{\pgfqpoint{5.201297in}{3.102590in}}%
\pgfusepath{clip}%
\pgfsetbuttcap%
\pgfsetroundjoin%
\definecolor{currentfill}{rgb}{1.000000,0.498039,0.054902}%
\pgfsetfillcolor{currentfill}%
\pgfsetlinewidth{1.003750pt}%
\definecolor{currentstroke}{rgb}{1.000000,0.498039,0.054902}%
\pgfsetstrokecolor{currentstroke}%
\pgfsetdash{}{0pt}%
\pgfpathmoveto{\pgfqpoint{1.144219in}{3.185343in}}%
\pgfpathcurveto{\pgfqpoint{1.155269in}{3.185343in}}{\pgfqpoint{1.165868in}{3.189733in}}{\pgfqpoint{1.173682in}{3.197547in}}%
\pgfpathcurveto{\pgfqpoint{1.181496in}{3.205360in}}{\pgfqpoint{1.185886in}{3.215959in}}{\pgfqpoint{1.185886in}{3.227010in}}%
\pgfpathcurveto{\pgfqpoint{1.185886in}{3.238060in}}{\pgfqpoint{1.181496in}{3.248659in}}{\pgfqpoint{1.173682in}{3.256472in}}%
\pgfpathcurveto{\pgfqpoint{1.165868in}{3.264286in}}{\pgfqpoint{1.155269in}{3.268676in}}{\pgfqpoint{1.144219in}{3.268676in}}%
\pgfpathcurveto{\pgfqpoint{1.133169in}{3.268676in}}{\pgfqpoint{1.122570in}{3.264286in}}{\pgfqpoint{1.114756in}{3.256472in}}%
\pgfpathcurveto{\pgfqpoint{1.106943in}{3.248659in}}{\pgfqpoint{1.102553in}{3.238060in}}{\pgfqpoint{1.102553in}{3.227010in}}%
\pgfpathcurveto{\pgfqpoint{1.102553in}{3.215959in}}{\pgfqpoint{1.106943in}{3.205360in}}{\pgfqpoint{1.114756in}{3.197547in}}%
\pgfpathcurveto{\pgfqpoint{1.122570in}{3.189733in}}{\pgfqpoint{1.133169in}{3.185343in}}{\pgfqpoint{1.144219in}{3.185343in}}%
\pgfpathclose%
\pgfusepath{stroke,fill}%
\end{pgfscope}%
\begin{pgfscope}%
\pgfpathrectangle{\pgfqpoint{0.648703in}{0.548769in}}{\pgfqpoint{5.201297in}{3.102590in}}%
\pgfusepath{clip}%
\pgfsetbuttcap%
\pgfsetroundjoin%
\definecolor{currentfill}{rgb}{0.121569,0.466667,0.705882}%
\pgfsetfillcolor{currentfill}%
\pgfsetlinewidth{1.003750pt}%
\definecolor{currentstroke}{rgb}{0.121569,0.466667,0.705882}%
\pgfsetstrokecolor{currentstroke}%
\pgfsetdash{}{0pt}%
\pgfpathmoveto{\pgfqpoint{0.949899in}{0.648129in}}%
\pgfpathcurveto{\pgfqpoint{0.960949in}{0.648129in}}{\pgfqpoint{0.971548in}{0.652519in}}{\pgfqpoint{0.979362in}{0.660333in}}%
\pgfpathcurveto{\pgfqpoint{0.987176in}{0.668146in}}{\pgfqpoint{0.991566in}{0.678745in}}{\pgfqpoint{0.991566in}{0.689796in}}%
\pgfpathcurveto{\pgfqpoint{0.991566in}{0.700846in}}{\pgfqpoint{0.987176in}{0.711445in}}{\pgfqpoint{0.979362in}{0.719258in}}%
\pgfpathcurveto{\pgfqpoint{0.971548in}{0.727072in}}{\pgfqpoint{0.960949in}{0.731462in}}{\pgfqpoint{0.949899in}{0.731462in}}%
\pgfpathcurveto{\pgfqpoint{0.938849in}{0.731462in}}{\pgfqpoint{0.928250in}{0.727072in}}{\pgfqpoint{0.920437in}{0.719258in}}%
\pgfpathcurveto{\pgfqpoint{0.912623in}{0.711445in}}{\pgfqpoint{0.908233in}{0.700846in}}{\pgfqpoint{0.908233in}{0.689796in}}%
\pgfpathcurveto{\pgfqpoint{0.908233in}{0.678745in}}{\pgfqpoint{0.912623in}{0.668146in}}{\pgfqpoint{0.920437in}{0.660333in}}%
\pgfpathcurveto{\pgfqpoint{0.928250in}{0.652519in}}{\pgfqpoint{0.938849in}{0.648129in}}{\pgfqpoint{0.949899in}{0.648129in}}%
\pgfpathclose%
\pgfusepath{stroke,fill}%
\end{pgfscope}%
\begin{pgfscope}%
\pgfpathrectangle{\pgfqpoint{0.648703in}{0.548769in}}{\pgfqpoint{5.201297in}{3.102590in}}%
\pgfusepath{clip}%
\pgfsetbuttcap%
\pgfsetroundjoin%
\definecolor{currentfill}{rgb}{1.000000,0.498039,0.054902}%
\pgfsetfillcolor{currentfill}%
\pgfsetlinewidth{1.003750pt}%
\definecolor{currentstroke}{rgb}{1.000000,0.498039,0.054902}%
\pgfsetstrokecolor{currentstroke}%
\pgfsetdash{}{0pt}%
\pgfpathmoveto{\pgfqpoint{1.208993in}{3.278374in}}%
\pgfpathcurveto{\pgfqpoint{1.220043in}{3.278374in}}{\pgfqpoint{1.230642in}{3.282764in}}{\pgfqpoint{1.238455in}{3.290578in}}%
\pgfpathcurveto{\pgfqpoint{1.246269in}{3.298392in}}{\pgfqpoint{1.250659in}{3.308991in}}{\pgfqpoint{1.250659in}{3.320041in}}%
\pgfpathcurveto{\pgfqpoint{1.250659in}{3.331091in}}{\pgfqpoint{1.246269in}{3.341690in}}{\pgfqpoint{1.238455in}{3.349504in}}%
\pgfpathcurveto{\pgfqpoint{1.230642in}{3.357317in}}{\pgfqpoint{1.220043in}{3.361707in}}{\pgfqpoint{1.208993in}{3.361707in}}%
\pgfpathcurveto{\pgfqpoint{1.197942in}{3.361707in}}{\pgfqpoint{1.187343in}{3.357317in}}{\pgfqpoint{1.179530in}{3.349504in}}%
\pgfpathcurveto{\pgfqpoint{1.171716in}{3.341690in}}{\pgfqpoint{1.167326in}{3.331091in}}{\pgfqpoint{1.167326in}{3.320041in}}%
\pgfpathcurveto{\pgfqpoint{1.167326in}{3.308991in}}{\pgfqpoint{1.171716in}{3.298392in}}{\pgfqpoint{1.179530in}{3.290578in}}%
\pgfpathcurveto{\pgfqpoint{1.187343in}{3.282764in}}{\pgfqpoint{1.197942in}{3.278374in}}{\pgfqpoint{1.208993in}{3.278374in}}%
\pgfpathclose%
\pgfusepath{stroke,fill}%
\end{pgfscope}%
\begin{pgfscope}%
\pgfpathrectangle{\pgfqpoint{0.648703in}{0.548769in}}{\pgfqpoint{5.201297in}{3.102590in}}%
\pgfusepath{clip}%
\pgfsetbuttcap%
\pgfsetroundjoin%
\definecolor{currentfill}{rgb}{0.121569,0.466667,0.705882}%
\pgfsetfillcolor{currentfill}%
\pgfsetlinewidth{1.003750pt}%
\definecolor{currentstroke}{rgb}{0.121569,0.466667,0.705882}%
\pgfsetstrokecolor{currentstroke}%
\pgfsetdash{}{0pt}%
\pgfpathmoveto{\pgfqpoint{0.949899in}{0.648129in}}%
\pgfpathcurveto{\pgfqpoint{0.960949in}{0.648129in}}{\pgfqpoint{0.971548in}{0.652519in}}{\pgfqpoint{0.979362in}{0.660333in}}%
\pgfpathcurveto{\pgfqpoint{0.987176in}{0.668146in}}{\pgfqpoint{0.991566in}{0.678745in}}{\pgfqpoint{0.991566in}{0.689796in}}%
\pgfpathcurveto{\pgfqpoint{0.991566in}{0.700846in}}{\pgfqpoint{0.987176in}{0.711445in}}{\pgfqpoint{0.979362in}{0.719258in}}%
\pgfpathcurveto{\pgfqpoint{0.971548in}{0.727072in}}{\pgfqpoint{0.960949in}{0.731462in}}{\pgfqpoint{0.949899in}{0.731462in}}%
\pgfpathcurveto{\pgfqpoint{0.938849in}{0.731462in}}{\pgfqpoint{0.928250in}{0.727072in}}{\pgfqpoint{0.920437in}{0.719258in}}%
\pgfpathcurveto{\pgfqpoint{0.912623in}{0.711445in}}{\pgfqpoint{0.908233in}{0.700846in}}{\pgfqpoint{0.908233in}{0.689796in}}%
\pgfpathcurveto{\pgfqpoint{0.908233in}{0.678745in}}{\pgfqpoint{0.912623in}{0.668146in}}{\pgfqpoint{0.920437in}{0.660333in}}%
\pgfpathcurveto{\pgfqpoint{0.928250in}{0.652519in}}{\pgfqpoint{0.938849in}{0.648129in}}{\pgfqpoint{0.949899in}{0.648129in}}%
\pgfpathclose%
\pgfusepath{stroke,fill}%
\end{pgfscope}%
\begin{pgfscope}%
\pgfpathrectangle{\pgfqpoint{0.648703in}{0.548769in}}{\pgfqpoint{5.201297in}{3.102590in}}%
\pgfusepath{clip}%
\pgfsetbuttcap%
\pgfsetroundjoin%
\definecolor{currentfill}{rgb}{1.000000,0.498039,0.054902}%
\pgfsetfillcolor{currentfill}%
\pgfsetlinewidth{1.003750pt}%
\definecolor{currentstroke}{rgb}{1.000000,0.498039,0.054902}%
\pgfsetstrokecolor{currentstroke}%
\pgfsetdash{}{0pt}%
\pgfpathmoveto{\pgfqpoint{1.403312in}{3.202258in}}%
\pgfpathcurveto{\pgfqpoint{1.414363in}{3.202258in}}{\pgfqpoint{1.424962in}{3.206648in}}{\pgfqpoint{1.432775in}{3.214462in}}%
\pgfpathcurveto{\pgfqpoint{1.440589in}{3.222275in}}{\pgfqpoint{1.444979in}{3.232874in}}{\pgfqpoint{1.444979in}{3.243924in}}%
\pgfpathcurveto{\pgfqpoint{1.444979in}{3.254974in}}{\pgfqpoint{1.440589in}{3.265573in}}{\pgfqpoint{1.432775in}{3.273387in}}%
\pgfpathcurveto{\pgfqpoint{1.424962in}{3.281201in}}{\pgfqpoint{1.414363in}{3.285591in}}{\pgfqpoint{1.403312in}{3.285591in}}%
\pgfpathcurveto{\pgfqpoint{1.392262in}{3.285591in}}{\pgfqpoint{1.381663in}{3.281201in}}{\pgfqpoint{1.373850in}{3.273387in}}%
\pgfpathcurveto{\pgfqpoint{1.366036in}{3.265573in}}{\pgfqpoint{1.361646in}{3.254974in}}{\pgfqpoint{1.361646in}{3.243924in}}%
\pgfpathcurveto{\pgfqpoint{1.361646in}{3.232874in}}{\pgfqpoint{1.366036in}{3.222275in}}{\pgfqpoint{1.373850in}{3.214462in}}%
\pgfpathcurveto{\pgfqpoint{1.381663in}{3.206648in}}{\pgfqpoint{1.392262in}{3.202258in}}{\pgfqpoint{1.403312in}{3.202258in}}%
\pgfpathclose%
\pgfusepath{stroke,fill}%
\end{pgfscope}%
\begin{pgfscope}%
\pgfpathrectangle{\pgfqpoint{0.648703in}{0.548769in}}{\pgfqpoint{5.201297in}{3.102590in}}%
\pgfusepath{clip}%
\pgfsetbuttcap%
\pgfsetroundjoin%
\definecolor{currentfill}{rgb}{1.000000,0.498039,0.054902}%
\pgfsetfillcolor{currentfill}%
\pgfsetlinewidth{1.003750pt}%
\definecolor{currentstroke}{rgb}{1.000000,0.498039,0.054902}%
\pgfsetstrokecolor{currentstroke}%
\pgfsetdash{}{0pt}%
\pgfpathmoveto{\pgfqpoint{1.014673in}{3.185343in}}%
\pgfpathcurveto{\pgfqpoint{1.025723in}{3.185343in}}{\pgfqpoint{1.036322in}{3.189733in}}{\pgfqpoint{1.044135in}{3.197547in}}%
\pgfpathcurveto{\pgfqpoint{1.051949in}{3.205360in}}{\pgfqpoint{1.056339in}{3.215959in}}{\pgfqpoint{1.056339in}{3.227010in}}%
\pgfpathcurveto{\pgfqpoint{1.056339in}{3.238060in}}{\pgfqpoint{1.051949in}{3.248659in}}{\pgfqpoint{1.044135in}{3.256472in}}%
\pgfpathcurveto{\pgfqpoint{1.036322in}{3.264286in}}{\pgfqpoint{1.025723in}{3.268676in}}{\pgfqpoint{1.014673in}{3.268676in}}%
\pgfpathcurveto{\pgfqpoint{1.003622in}{3.268676in}}{\pgfqpoint{0.993023in}{3.264286in}}{\pgfqpoint{0.985210in}{3.256472in}}%
\pgfpathcurveto{\pgfqpoint{0.977396in}{3.248659in}}{\pgfqpoint{0.973006in}{3.238060in}}{\pgfqpoint{0.973006in}{3.227010in}}%
\pgfpathcurveto{\pgfqpoint{0.973006in}{3.215959in}}{\pgfqpoint{0.977396in}{3.205360in}}{\pgfqpoint{0.985210in}{3.197547in}}%
\pgfpathcurveto{\pgfqpoint{0.993023in}{3.189733in}}{\pgfqpoint{1.003622in}{3.185343in}}{\pgfqpoint{1.014673in}{3.185343in}}%
\pgfpathclose%
\pgfusepath{stroke,fill}%
\end{pgfscope}%
\begin{pgfscope}%
\pgfpathrectangle{\pgfqpoint{0.648703in}{0.548769in}}{\pgfqpoint{5.201297in}{3.102590in}}%
\pgfusepath{clip}%
\pgfsetbuttcap%
\pgfsetroundjoin%
\definecolor{currentfill}{rgb}{0.121569,0.466667,0.705882}%
\pgfsetfillcolor{currentfill}%
\pgfsetlinewidth{1.003750pt}%
\definecolor{currentstroke}{rgb}{0.121569,0.466667,0.705882}%
\pgfsetstrokecolor{currentstroke}%
\pgfsetdash{}{0pt}%
\pgfpathmoveto{\pgfqpoint{1.208993in}{0.648129in}}%
\pgfpathcurveto{\pgfqpoint{1.220043in}{0.648129in}}{\pgfqpoint{1.230642in}{0.652519in}}{\pgfqpoint{1.238455in}{0.660333in}}%
\pgfpathcurveto{\pgfqpoint{1.246269in}{0.668146in}}{\pgfqpoint{1.250659in}{0.678745in}}{\pgfqpoint{1.250659in}{0.689796in}}%
\pgfpathcurveto{\pgfqpoint{1.250659in}{0.700846in}}{\pgfqpoint{1.246269in}{0.711445in}}{\pgfqpoint{1.238455in}{0.719258in}}%
\pgfpathcurveto{\pgfqpoint{1.230642in}{0.727072in}}{\pgfqpoint{1.220043in}{0.731462in}}{\pgfqpoint{1.208993in}{0.731462in}}%
\pgfpathcurveto{\pgfqpoint{1.197942in}{0.731462in}}{\pgfqpoint{1.187343in}{0.727072in}}{\pgfqpoint{1.179530in}{0.719258in}}%
\pgfpathcurveto{\pgfqpoint{1.171716in}{0.711445in}}{\pgfqpoint{1.167326in}{0.700846in}}{\pgfqpoint{1.167326in}{0.689796in}}%
\pgfpathcurveto{\pgfqpoint{1.167326in}{0.678745in}}{\pgfqpoint{1.171716in}{0.668146in}}{\pgfqpoint{1.179530in}{0.660333in}}%
\pgfpathcurveto{\pgfqpoint{1.187343in}{0.652519in}}{\pgfqpoint{1.197942in}{0.648129in}}{\pgfqpoint{1.208993in}{0.648129in}}%
\pgfpathclose%
\pgfusepath{stroke,fill}%
\end{pgfscope}%
\begin{pgfscope}%
\pgfpathrectangle{\pgfqpoint{0.648703in}{0.548769in}}{\pgfqpoint{5.201297in}{3.102590in}}%
\pgfusepath{clip}%
\pgfsetbuttcap%
\pgfsetroundjoin%
\definecolor{currentfill}{rgb}{1.000000,0.498039,0.054902}%
\pgfsetfillcolor{currentfill}%
\pgfsetlinewidth{1.003750pt}%
\definecolor{currentstroke}{rgb}{1.000000,0.498039,0.054902}%
\pgfsetstrokecolor{currentstroke}%
\pgfsetdash{}{0pt}%
\pgfpathmoveto{\pgfqpoint{2.051046in}{3.185343in}}%
\pgfpathcurveto{\pgfqpoint{2.062096in}{3.185343in}}{\pgfqpoint{2.072695in}{3.189733in}}{\pgfqpoint{2.080508in}{3.197547in}}%
\pgfpathcurveto{\pgfqpoint{2.088322in}{3.205360in}}{\pgfqpoint{2.092712in}{3.215959in}}{\pgfqpoint{2.092712in}{3.227010in}}%
\pgfpathcurveto{\pgfqpoint{2.092712in}{3.238060in}}{\pgfqpoint{2.088322in}{3.248659in}}{\pgfqpoint{2.080508in}{3.256472in}}%
\pgfpathcurveto{\pgfqpoint{2.072695in}{3.264286in}}{\pgfqpoint{2.062096in}{3.268676in}}{\pgfqpoint{2.051046in}{3.268676in}}%
\pgfpathcurveto{\pgfqpoint{2.039995in}{3.268676in}}{\pgfqpoint{2.029396in}{3.264286in}}{\pgfqpoint{2.021583in}{3.256472in}}%
\pgfpathcurveto{\pgfqpoint{2.013769in}{3.248659in}}{\pgfqpoint{2.009379in}{3.238060in}}{\pgfqpoint{2.009379in}{3.227010in}}%
\pgfpathcurveto{\pgfqpoint{2.009379in}{3.215959in}}{\pgfqpoint{2.013769in}{3.205360in}}{\pgfqpoint{2.021583in}{3.197547in}}%
\pgfpathcurveto{\pgfqpoint{2.029396in}{3.189733in}}{\pgfqpoint{2.039995in}{3.185343in}}{\pgfqpoint{2.051046in}{3.185343in}}%
\pgfpathclose%
\pgfusepath{stroke,fill}%
\end{pgfscope}%
\begin{pgfscope}%
\pgfpathrectangle{\pgfqpoint{0.648703in}{0.548769in}}{\pgfqpoint{5.201297in}{3.102590in}}%
\pgfusepath{clip}%
\pgfsetbuttcap%
\pgfsetroundjoin%
\definecolor{currentfill}{rgb}{1.000000,0.498039,0.054902}%
\pgfsetfillcolor{currentfill}%
\pgfsetlinewidth{1.003750pt}%
\definecolor{currentstroke}{rgb}{1.000000,0.498039,0.054902}%
\pgfsetstrokecolor{currentstroke}%
\pgfsetdash{}{0pt}%
\pgfpathmoveto{\pgfqpoint{1.921499in}{3.193800in}}%
\pgfpathcurveto{\pgfqpoint{1.932549in}{3.193800in}}{\pgfqpoint{1.943148in}{3.198191in}}{\pgfqpoint{1.950962in}{3.206004in}}%
\pgfpathcurveto{\pgfqpoint{1.958775in}{3.213818in}}{\pgfqpoint{1.963166in}{3.224417in}}{\pgfqpoint{1.963166in}{3.235467in}}%
\pgfpathcurveto{\pgfqpoint{1.963166in}{3.246517in}}{\pgfqpoint{1.958775in}{3.257116in}}{\pgfqpoint{1.950962in}{3.264930in}}%
\pgfpathcurveto{\pgfqpoint{1.943148in}{3.272743in}}{\pgfqpoint{1.932549in}{3.277134in}}{\pgfqpoint{1.921499in}{3.277134in}}%
\pgfpathcurveto{\pgfqpoint{1.910449in}{3.277134in}}{\pgfqpoint{1.899850in}{3.272743in}}{\pgfqpoint{1.892036in}{3.264930in}}%
\pgfpathcurveto{\pgfqpoint{1.884223in}{3.257116in}}{\pgfqpoint{1.879832in}{3.246517in}}{\pgfqpoint{1.879832in}{3.235467in}}%
\pgfpathcurveto{\pgfqpoint{1.879832in}{3.224417in}}{\pgfqpoint{1.884223in}{3.213818in}}{\pgfqpoint{1.892036in}{3.206004in}}%
\pgfpathcurveto{\pgfqpoint{1.899850in}{3.198191in}}{\pgfqpoint{1.910449in}{3.193800in}}{\pgfqpoint{1.921499in}{3.193800in}}%
\pgfpathclose%
\pgfusepath{stroke,fill}%
\end{pgfscope}%
\begin{pgfscope}%
\pgfpathrectangle{\pgfqpoint{0.648703in}{0.548769in}}{\pgfqpoint{5.201297in}{3.102590in}}%
\pgfusepath{clip}%
\pgfsetbuttcap%
\pgfsetroundjoin%
\definecolor{currentfill}{rgb}{0.121569,0.466667,0.705882}%
\pgfsetfillcolor{currentfill}%
\pgfsetlinewidth{1.003750pt}%
\definecolor{currentstroke}{rgb}{0.121569,0.466667,0.705882}%
\pgfsetstrokecolor{currentstroke}%
\pgfsetdash{}{0pt}%
\pgfpathmoveto{\pgfqpoint{1.856726in}{0.656586in}}%
\pgfpathcurveto{\pgfqpoint{1.867776in}{0.656586in}}{\pgfqpoint{1.878375in}{0.660977in}}{\pgfqpoint{1.886188in}{0.668790in}}%
\pgfpathcurveto{\pgfqpoint{1.894002in}{0.676604in}}{\pgfqpoint{1.898392in}{0.687203in}}{\pgfqpoint{1.898392in}{0.698253in}}%
\pgfpathcurveto{\pgfqpoint{1.898392in}{0.709303in}}{\pgfqpoint{1.894002in}{0.719902in}}{\pgfqpoint{1.886188in}{0.727716in}}%
\pgfpathcurveto{\pgfqpoint{1.878375in}{0.735529in}}{\pgfqpoint{1.867776in}{0.739920in}}{\pgfqpoint{1.856726in}{0.739920in}}%
\pgfpathcurveto{\pgfqpoint{1.845675in}{0.739920in}}{\pgfqpoint{1.835076in}{0.735529in}}{\pgfqpoint{1.827263in}{0.727716in}}%
\pgfpathcurveto{\pgfqpoint{1.819449in}{0.719902in}}{\pgfqpoint{1.815059in}{0.709303in}}{\pgfqpoint{1.815059in}{0.698253in}}%
\pgfpathcurveto{\pgfqpoint{1.815059in}{0.687203in}}{\pgfqpoint{1.819449in}{0.676604in}}{\pgfqpoint{1.827263in}{0.668790in}}%
\pgfpathcurveto{\pgfqpoint{1.835076in}{0.660977in}}{\pgfqpoint{1.845675in}{0.656586in}}{\pgfqpoint{1.856726in}{0.656586in}}%
\pgfpathclose%
\pgfusepath{stroke,fill}%
\end{pgfscope}%
\begin{pgfscope}%
\pgfpathrectangle{\pgfqpoint{0.648703in}{0.548769in}}{\pgfqpoint{5.201297in}{3.102590in}}%
\pgfusepath{clip}%
\pgfsetbuttcap%
\pgfsetroundjoin%
\definecolor{currentfill}{rgb}{1.000000,0.498039,0.054902}%
\pgfsetfillcolor{currentfill}%
\pgfsetlinewidth{1.003750pt}%
\definecolor{currentstroke}{rgb}{1.000000,0.498039,0.054902}%
\pgfsetstrokecolor{currentstroke}%
\pgfsetdash{}{0pt}%
\pgfpathmoveto{\pgfqpoint{2.310139in}{3.214944in}}%
\pgfpathcurveto{\pgfqpoint{2.321189in}{3.214944in}}{\pgfqpoint{2.331788in}{3.219334in}}{\pgfqpoint{2.339602in}{3.227148in}}%
\pgfpathcurveto{\pgfqpoint{2.347415in}{3.234961in}}{\pgfqpoint{2.351805in}{3.245560in}}{\pgfqpoint{2.351805in}{3.256610in}}%
\pgfpathcurveto{\pgfqpoint{2.351805in}{3.267661in}}{\pgfqpoint{2.347415in}{3.278260in}}{\pgfqpoint{2.339602in}{3.286073in}}%
\pgfpathcurveto{\pgfqpoint{2.331788in}{3.293887in}}{\pgfqpoint{2.321189in}{3.298277in}}{\pgfqpoint{2.310139in}{3.298277in}}%
\pgfpathcurveto{\pgfqpoint{2.299089in}{3.298277in}}{\pgfqpoint{2.288490in}{3.293887in}}{\pgfqpoint{2.280676in}{3.286073in}}%
\pgfpathcurveto{\pgfqpoint{2.272862in}{3.278260in}}{\pgfqpoint{2.268472in}{3.267661in}}{\pgfqpoint{2.268472in}{3.256610in}}%
\pgfpathcurveto{\pgfqpoint{2.268472in}{3.245560in}}{\pgfqpoint{2.272862in}{3.234961in}}{\pgfqpoint{2.280676in}{3.227148in}}%
\pgfpathcurveto{\pgfqpoint{2.288490in}{3.219334in}}{\pgfqpoint{2.299089in}{3.214944in}}{\pgfqpoint{2.310139in}{3.214944in}}%
\pgfpathclose%
\pgfusepath{stroke,fill}%
\end{pgfscope}%
\begin{pgfscope}%
\pgfpathrectangle{\pgfqpoint{0.648703in}{0.548769in}}{\pgfqpoint{5.201297in}{3.102590in}}%
\pgfusepath{clip}%
\pgfsetbuttcap%
\pgfsetroundjoin%
\definecolor{currentfill}{rgb}{1.000000,0.498039,0.054902}%
\pgfsetfillcolor{currentfill}%
\pgfsetlinewidth{1.003750pt}%
\definecolor{currentstroke}{rgb}{1.000000,0.498039,0.054902}%
\pgfsetstrokecolor{currentstroke}%
\pgfsetdash{}{0pt}%
\pgfpathmoveto{\pgfqpoint{1.791952in}{3.202258in}}%
\pgfpathcurveto{\pgfqpoint{1.803002in}{3.202258in}}{\pgfqpoint{1.813601in}{3.206648in}}{\pgfqpoint{1.821415in}{3.214462in}}%
\pgfpathcurveto{\pgfqpoint{1.829229in}{3.222275in}}{\pgfqpoint{1.833619in}{3.232874in}}{\pgfqpoint{1.833619in}{3.243924in}}%
\pgfpathcurveto{\pgfqpoint{1.833619in}{3.254974in}}{\pgfqpoint{1.829229in}{3.265573in}}{\pgfqpoint{1.821415in}{3.273387in}}%
\pgfpathcurveto{\pgfqpoint{1.813601in}{3.281201in}}{\pgfqpoint{1.803002in}{3.285591in}}{\pgfqpoint{1.791952in}{3.285591in}}%
\pgfpathcurveto{\pgfqpoint{1.780902in}{3.285591in}}{\pgfqpoint{1.770303in}{3.281201in}}{\pgfqpoint{1.762490in}{3.273387in}}%
\pgfpathcurveto{\pgfqpoint{1.754676in}{3.265573in}}{\pgfqpoint{1.750286in}{3.254974in}}{\pgfqpoint{1.750286in}{3.243924in}}%
\pgfpathcurveto{\pgfqpoint{1.750286in}{3.232874in}}{\pgfqpoint{1.754676in}{3.222275in}}{\pgfqpoint{1.762490in}{3.214462in}}%
\pgfpathcurveto{\pgfqpoint{1.770303in}{3.206648in}}{\pgfqpoint{1.780902in}{3.202258in}}{\pgfqpoint{1.791952in}{3.202258in}}%
\pgfpathclose%
\pgfusepath{stroke,fill}%
\end{pgfscope}%
\begin{pgfscope}%
\pgfpathrectangle{\pgfqpoint{0.648703in}{0.548769in}}{\pgfqpoint{5.201297in}{3.102590in}}%
\pgfusepath{clip}%
\pgfsetbuttcap%
\pgfsetroundjoin%
\definecolor{currentfill}{rgb}{1.000000,0.498039,0.054902}%
\pgfsetfillcolor{currentfill}%
\pgfsetlinewidth{1.003750pt}%
\definecolor{currentstroke}{rgb}{1.000000,0.498039,0.054902}%
\pgfsetstrokecolor{currentstroke}%
\pgfsetdash{}{0pt}%
\pgfpathmoveto{\pgfqpoint{1.662406in}{3.244545in}}%
\pgfpathcurveto{\pgfqpoint{1.673456in}{3.244545in}}{\pgfqpoint{1.684055in}{3.248935in}}{\pgfqpoint{1.691868in}{3.256748in}}%
\pgfpathcurveto{\pgfqpoint{1.699682in}{3.264562in}}{\pgfqpoint{1.704072in}{3.275161in}}{\pgfqpoint{1.704072in}{3.286211in}}%
\pgfpathcurveto{\pgfqpoint{1.704072in}{3.297261in}}{\pgfqpoint{1.699682in}{3.307860in}}{\pgfqpoint{1.691868in}{3.315674in}}%
\pgfpathcurveto{\pgfqpoint{1.684055in}{3.323488in}}{\pgfqpoint{1.673456in}{3.327878in}}{\pgfqpoint{1.662406in}{3.327878in}}%
\pgfpathcurveto{\pgfqpoint{1.651356in}{3.327878in}}{\pgfqpoint{1.640757in}{3.323488in}}{\pgfqpoint{1.632943in}{3.315674in}}%
\pgfpathcurveto{\pgfqpoint{1.625129in}{3.307860in}}{\pgfqpoint{1.620739in}{3.297261in}}{\pgfqpoint{1.620739in}{3.286211in}}%
\pgfpathcurveto{\pgfqpoint{1.620739in}{3.275161in}}{\pgfqpoint{1.625129in}{3.264562in}}{\pgfqpoint{1.632943in}{3.256748in}}%
\pgfpathcurveto{\pgfqpoint{1.640757in}{3.248935in}}{\pgfqpoint{1.651356in}{3.244545in}}{\pgfqpoint{1.662406in}{3.244545in}}%
\pgfpathclose%
\pgfusepath{stroke,fill}%
\end{pgfscope}%
\begin{pgfscope}%
\pgfpathrectangle{\pgfqpoint{0.648703in}{0.548769in}}{\pgfqpoint{5.201297in}{3.102590in}}%
\pgfusepath{clip}%
\pgfsetbuttcap%
\pgfsetroundjoin%
\definecolor{currentfill}{rgb}{1.000000,0.498039,0.054902}%
\pgfsetfillcolor{currentfill}%
\pgfsetlinewidth{1.003750pt}%
\definecolor{currentstroke}{rgb}{1.000000,0.498039,0.054902}%
\pgfsetstrokecolor{currentstroke}%
\pgfsetdash{}{0pt}%
\pgfpathmoveto{\pgfqpoint{1.597632in}{3.312204in}}%
\pgfpathcurveto{\pgfqpoint{1.608682in}{3.312204in}}{\pgfqpoint{1.619282in}{3.316594in}}{\pgfqpoint{1.627095in}{3.324407in}}%
\pgfpathcurveto{\pgfqpoint{1.634909in}{3.332221in}}{\pgfqpoint{1.639299in}{3.342820in}}{\pgfqpoint{1.639299in}{3.353870in}}%
\pgfpathcurveto{\pgfqpoint{1.639299in}{3.364920in}}{\pgfqpoint{1.634909in}{3.375519in}}{\pgfqpoint{1.627095in}{3.383333in}}%
\pgfpathcurveto{\pgfqpoint{1.619282in}{3.391147in}}{\pgfqpoint{1.608682in}{3.395537in}}{\pgfqpoint{1.597632in}{3.395537in}}%
\pgfpathcurveto{\pgfqpoint{1.586582in}{3.395537in}}{\pgfqpoint{1.575983in}{3.391147in}}{\pgfqpoint{1.568170in}{3.383333in}}%
\pgfpathcurveto{\pgfqpoint{1.560356in}{3.375519in}}{\pgfqpoint{1.555966in}{3.364920in}}{\pgfqpoint{1.555966in}{3.353870in}}%
\pgfpathcurveto{\pgfqpoint{1.555966in}{3.342820in}}{\pgfqpoint{1.560356in}{3.332221in}}{\pgfqpoint{1.568170in}{3.324407in}}%
\pgfpathcurveto{\pgfqpoint{1.575983in}{3.316594in}}{\pgfqpoint{1.586582in}{3.312204in}}{\pgfqpoint{1.597632in}{3.312204in}}%
\pgfpathclose%
\pgfusepath{stroke,fill}%
\end{pgfscope}%
\begin{pgfscope}%
\pgfpathrectangle{\pgfqpoint{0.648703in}{0.548769in}}{\pgfqpoint{5.201297in}{3.102590in}}%
\pgfusepath{clip}%
\pgfsetbuttcap%
\pgfsetroundjoin%
\definecolor{currentfill}{rgb}{1.000000,0.498039,0.054902}%
\pgfsetfillcolor{currentfill}%
\pgfsetlinewidth{1.003750pt}%
\definecolor{currentstroke}{rgb}{1.000000,0.498039,0.054902}%
\pgfsetstrokecolor{currentstroke}%
\pgfsetdash{}{0pt}%
\pgfpathmoveto{\pgfqpoint{1.532859in}{3.210715in}}%
\pgfpathcurveto{\pgfqpoint{1.543909in}{3.210715in}}{\pgfqpoint{1.554508in}{3.215105in}}{\pgfqpoint{1.562322in}{3.222919in}}%
\pgfpathcurveto{\pgfqpoint{1.570135in}{3.230733in}}{\pgfqpoint{1.574526in}{3.241332in}}{\pgfqpoint{1.574526in}{3.252382in}}%
\pgfpathcurveto{\pgfqpoint{1.574526in}{3.263432in}}{\pgfqpoint{1.570135in}{3.274031in}}{\pgfqpoint{1.562322in}{3.281844in}}%
\pgfpathcurveto{\pgfqpoint{1.554508in}{3.289658in}}{\pgfqpoint{1.543909in}{3.294048in}}{\pgfqpoint{1.532859in}{3.294048in}}%
\pgfpathcurveto{\pgfqpoint{1.521809in}{3.294048in}}{\pgfqpoint{1.511210in}{3.289658in}}{\pgfqpoint{1.503396in}{3.281844in}}%
\pgfpathcurveto{\pgfqpoint{1.495583in}{3.274031in}}{\pgfqpoint{1.491192in}{3.263432in}}{\pgfqpoint{1.491192in}{3.252382in}}%
\pgfpathcurveto{\pgfqpoint{1.491192in}{3.241332in}}{\pgfqpoint{1.495583in}{3.230733in}}{\pgfqpoint{1.503396in}{3.222919in}}%
\pgfpathcurveto{\pgfqpoint{1.511210in}{3.215105in}}{\pgfqpoint{1.521809in}{3.210715in}}{\pgfqpoint{1.532859in}{3.210715in}}%
\pgfpathclose%
\pgfusepath{stroke,fill}%
\end{pgfscope}%
\begin{pgfscope}%
\pgfpathrectangle{\pgfqpoint{0.648703in}{0.548769in}}{\pgfqpoint{5.201297in}{3.102590in}}%
\pgfusepath{clip}%
\pgfsetbuttcap%
\pgfsetroundjoin%
\definecolor{currentfill}{rgb}{1.000000,0.498039,0.054902}%
\pgfsetfillcolor{currentfill}%
\pgfsetlinewidth{1.003750pt}%
\definecolor{currentstroke}{rgb}{1.000000,0.498039,0.054902}%
\pgfsetstrokecolor{currentstroke}%
\pgfsetdash{}{0pt}%
\pgfpathmoveto{\pgfqpoint{1.597632in}{3.236087in}}%
\pgfpathcurveto{\pgfqpoint{1.608682in}{3.236087in}}{\pgfqpoint{1.619282in}{3.240477in}}{\pgfqpoint{1.627095in}{3.248291in}}%
\pgfpathcurveto{\pgfqpoint{1.634909in}{3.256105in}}{\pgfqpoint{1.639299in}{3.266704in}}{\pgfqpoint{1.639299in}{3.277754in}}%
\pgfpathcurveto{\pgfqpoint{1.639299in}{3.288804in}}{\pgfqpoint{1.634909in}{3.299403in}}{\pgfqpoint{1.627095in}{3.307217in}}%
\pgfpathcurveto{\pgfqpoint{1.619282in}{3.315030in}}{\pgfqpoint{1.608682in}{3.319421in}}{\pgfqpoint{1.597632in}{3.319421in}}%
\pgfpathcurveto{\pgfqpoint{1.586582in}{3.319421in}}{\pgfqpoint{1.575983in}{3.315030in}}{\pgfqpoint{1.568170in}{3.307217in}}%
\pgfpathcurveto{\pgfqpoint{1.560356in}{3.299403in}}{\pgfqpoint{1.555966in}{3.288804in}}{\pgfqpoint{1.555966in}{3.277754in}}%
\pgfpathcurveto{\pgfqpoint{1.555966in}{3.266704in}}{\pgfqpoint{1.560356in}{3.256105in}}{\pgfqpoint{1.568170in}{3.248291in}}%
\pgfpathcurveto{\pgfqpoint{1.575983in}{3.240477in}}{\pgfqpoint{1.586582in}{3.236087in}}{\pgfqpoint{1.597632in}{3.236087in}}%
\pgfpathclose%
\pgfusepath{stroke,fill}%
\end{pgfscope}%
\begin{pgfscope}%
\pgfpathrectangle{\pgfqpoint{0.648703in}{0.548769in}}{\pgfqpoint{5.201297in}{3.102590in}}%
\pgfusepath{clip}%
\pgfsetbuttcap%
\pgfsetroundjoin%
\definecolor{currentfill}{rgb}{0.121569,0.466667,0.705882}%
\pgfsetfillcolor{currentfill}%
\pgfsetlinewidth{1.003750pt}%
\definecolor{currentstroke}{rgb}{0.121569,0.466667,0.705882}%
\pgfsetstrokecolor{currentstroke}%
\pgfsetdash{}{0pt}%
\pgfpathmoveto{\pgfqpoint{1.208993in}{0.648129in}}%
\pgfpathcurveto{\pgfqpoint{1.220043in}{0.648129in}}{\pgfqpoint{1.230642in}{0.652519in}}{\pgfqpoint{1.238455in}{0.660333in}}%
\pgfpathcurveto{\pgfqpoint{1.246269in}{0.668146in}}{\pgfqpoint{1.250659in}{0.678745in}}{\pgfqpoint{1.250659in}{0.689796in}}%
\pgfpathcurveto{\pgfqpoint{1.250659in}{0.700846in}}{\pgfqpoint{1.246269in}{0.711445in}}{\pgfqpoint{1.238455in}{0.719258in}}%
\pgfpathcurveto{\pgfqpoint{1.230642in}{0.727072in}}{\pgfqpoint{1.220043in}{0.731462in}}{\pgfqpoint{1.208993in}{0.731462in}}%
\pgfpathcurveto{\pgfqpoint{1.197942in}{0.731462in}}{\pgfqpoint{1.187343in}{0.727072in}}{\pgfqpoint{1.179530in}{0.719258in}}%
\pgfpathcurveto{\pgfqpoint{1.171716in}{0.711445in}}{\pgfqpoint{1.167326in}{0.700846in}}{\pgfqpoint{1.167326in}{0.689796in}}%
\pgfpathcurveto{\pgfqpoint{1.167326in}{0.678745in}}{\pgfqpoint{1.171716in}{0.668146in}}{\pgfqpoint{1.179530in}{0.660333in}}%
\pgfpathcurveto{\pgfqpoint{1.187343in}{0.652519in}}{\pgfqpoint{1.197942in}{0.648129in}}{\pgfqpoint{1.208993in}{0.648129in}}%
\pgfpathclose%
\pgfusepath{stroke,fill}%
\end{pgfscope}%
\begin{pgfscope}%
\pgfpathrectangle{\pgfqpoint{0.648703in}{0.548769in}}{\pgfqpoint{5.201297in}{3.102590in}}%
\pgfusepath{clip}%
\pgfsetbuttcap%
\pgfsetroundjoin%
\definecolor{currentfill}{rgb}{0.121569,0.466667,0.705882}%
\pgfsetfillcolor{currentfill}%
\pgfsetlinewidth{1.003750pt}%
\definecolor{currentstroke}{rgb}{0.121569,0.466667,0.705882}%
\pgfsetstrokecolor{currentstroke}%
\pgfsetdash{}{0pt}%
\pgfpathmoveto{\pgfqpoint{0.949899in}{0.648129in}}%
\pgfpathcurveto{\pgfqpoint{0.960949in}{0.648129in}}{\pgfqpoint{0.971548in}{0.652519in}}{\pgfqpoint{0.979362in}{0.660333in}}%
\pgfpathcurveto{\pgfqpoint{0.987176in}{0.668146in}}{\pgfqpoint{0.991566in}{0.678745in}}{\pgfqpoint{0.991566in}{0.689796in}}%
\pgfpathcurveto{\pgfqpoint{0.991566in}{0.700846in}}{\pgfqpoint{0.987176in}{0.711445in}}{\pgfqpoint{0.979362in}{0.719258in}}%
\pgfpathcurveto{\pgfqpoint{0.971548in}{0.727072in}}{\pgfqpoint{0.960949in}{0.731462in}}{\pgfqpoint{0.949899in}{0.731462in}}%
\pgfpathcurveto{\pgfqpoint{0.938849in}{0.731462in}}{\pgfqpoint{0.928250in}{0.727072in}}{\pgfqpoint{0.920437in}{0.719258in}}%
\pgfpathcurveto{\pgfqpoint{0.912623in}{0.711445in}}{\pgfqpoint{0.908233in}{0.700846in}}{\pgfqpoint{0.908233in}{0.689796in}}%
\pgfpathcurveto{\pgfqpoint{0.908233in}{0.678745in}}{\pgfqpoint{0.912623in}{0.668146in}}{\pgfqpoint{0.920437in}{0.660333in}}%
\pgfpathcurveto{\pgfqpoint{0.928250in}{0.652519in}}{\pgfqpoint{0.938849in}{0.648129in}}{\pgfqpoint{0.949899in}{0.648129in}}%
\pgfpathclose%
\pgfusepath{stroke,fill}%
\end{pgfscope}%
\begin{pgfscope}%
\pgfpathrectangle{\pgfqpoint{0.648703in}{0.548769in}}{\pgfqpoint{5.201297in}{3.102590in}}%
\pgfusepath{clip}%
\pgfsetbuttcap%
\pgfsetroundjoin%
\definecolor{currentfill}{rgb}{0.121569,0.466667,0.705882}%
\pgfsetfillcolor{currentfill}%
\pgfsetlinewidth{1.003750pt}%
\definecolor{currentstroke}{rgb}{0.121569,0.466667,0.705882}%
\pgfsetstrokecolor{currentstroke}%
\pgfsetdash{}{0pt}%
\pgfpathmoveto{\pgfqpoint{1.338539in}{0.648129in}}%
\pgfpathcurveto{\pgfqpoint{1.349589in}{0.648129in}}{\pgfqpoint{1.360188in}{0.652519in}}{\pgfqpoint{1.368002in}{0.660333in}}%
\pgfpathcurveto{\pgfqpoint{1.375816in}{0.668146in}}{\pgfqpoint{1.380206in}{0.678745in}}{\pgfqpoint{1.380206in}{0.689796in}}%
\pgfpathcurveto{\pgfqpoint{1.380206in}{0.700846in}}{\pgfqpoint{1.375816in}{0.711445in}}{\pgfqpoint{1.368002in}{0.719258in}}%
\pgfpathcurveto{\pgfqpoint{1.360188in}{0.727072in}}{\pgfqpoint{1.349589in}{0.731462in}}{\pgfqpoint{1.338539in}{0.731462in}}%
\pgfpathcurveto{\pgfqpoint{1.327489in}{0.731462in}}{\pgfqpoint{1.316890in}{0.727072in}}{\pgfqpoint{1.309076in}{0.719258in}}%
\pgfpathcurveto{\pgfqpoint{1.301263in}{0.711445in}}{\pgfqpoint{1.296872in}{0.700846in}}{\pgfqpoint{1.296872in}{0.689796in}}%
\pgfpathcurveto{\pgfqpoint{1.296872in}{0.678745in}}{\pgfqpoint{1.301263in}{0.668146in}}{\pgfqpoint{1.309076in}{0.660333in}}%
\pgfpathcurveto{\pgfqpoint{1.316890in}{0.652519in}}{\pgfqpoint{1.327489in}{0.648129in}}{\pgfqpoint{1.338539in}{0.648129in}}%
\pgfpathclose%
\pgfusepath{stroke,fill}%
\end{pgfscope}%
\begin{pgfscope}%
\pgfpathrectangle{\pgfqpoint{0.648703in}{0.548769in}}{\pgfqpoint{5.201297in}{3.102590in}}%
\pgfusepath{clip}%
\pgfsetbuttcap%
\pgfsetroundjoin%
\definecolor{currentfill}{rgb}{1.000000,0.498039,0.054902}%
\pgfsetfillcolor{currentfill}%
\pgfsetlinewidth{1.003750pt}%
\definecolor{currentstroke}{rgb}{1.000000,0.498039,0.054902}%
\pgfsetstrokecolor{currentstroke}%
\pgfsetdash{}{0pt}%
\pgfpathmoveto{\pgfqpoint{1.144219in}{3.405235in}}%
\pgfpathcurveto{\pgfqpoint{1.155269in}{3.405235in}}{\pgfqpoint{1.165868in}{3.409625in}}{\pgfqpoint{1.173682in}{3.417439in}}%
\pgfpathcurveto{\pgfqpoint{1.181496in}{3.425252in}}{\pgfqpoint{1.185886in}{3.435851in}}{\pgfqpoint{1.185886in}{3.446901in}}%
\pgfpathcurveto{\pgfqpoint{1.185886in}{3.457952in}}{\pgfqpoint{1.181496in}{3.468551in}}{\pgfqpoint{1.173682in}{3.476364in}}%
\pgfpathcurveto{\pgfqpoint{1.165868in}{3.484178in}}{\pgfqpoint{1.155269in}{3.488568in}}{\pgfqpoint{1.144219in}{3.488568in}}%
\pgfpathcurveto{\pgfqpoint{1.133169in}{3.488568in}}{\pgfqpoint{1.122570in}{3.484178in}}{\pgfqpoint{1.114756in}{3.476364in}}%
\pgfpathcurveto{\pgfqpoint{1.106943in}{3.468551in}}{\pgfqpoint{1.102553in}{3.457952in}}{\pgfqpoint{1.102553in}{3.446901in}}%
\pgfpathcurveto{\pgfqpoint{1.102553in}{3.435851in}}{\pgfqpoint{1.106943in}{3.425252in}}{\pgfqpoint{1.114756in}{3.417439in}}%
\pgfpathcurveto{\pgfqpoint{1.122570in}{3.409625in}}{\pgfqpoint{1.133169in}{3.405235in}}{\pgfqpoint{1.144219in}{3.405235in}}%
\pgfpathclose%
\pgfusepath{stroke,fill}%
\end{pgfscope}%
\begin{pgfscope}%
\pgfpathrectangle{\pgfqpoint{0.648703in}{0.548769in}}{\pgfqpoint{5.201297in}{3.102590in}}%
\pgfusepath{clip}%
\pgfsetbuttcap%
\pgfsetroundjoin%
\definecolor{currentfill}{rgb}{1.000000,0.498039,0.054902}%
\pgfsetfillcolor{currentfill}%
\pgfsetlinewidth{1.003750pt}%
\definecolor{currentstroke}{rgb}{1.000000,0.498039,0.054902}%
\pgfsetstrokecolor{currentstroke}%
\pgfsetdash{}{0pt}%
\pgfpathmoveto{\pgfqpoint{1.856726in}{3.193800in}}%
\pgfpathcurveto{\pgfqpoint{1.867776in}{3.193800in}}{\pgfqpoint{1.878375in}{3.198191in}}{\pgfqpoint{1.886188in}{3.206004in}}%
\pgfpathcurveto{\pgfqpoint{1.894002in}{3.213818in}}{\pgfqpoint{1.898392in}{3.224417in}}{\pgfqpoint{1.898392in}{3.235467in}}%
\pgfpathcurveto{\pgfqpoint{1.898392in}{3.246517in}}{\pgfqpoint{1.894002in}{3.257116in}}{\pgfqpoint{1.886188in}{3.264930in}}%
\pgfpathcurveto{\pgfqpoint{1.878375in}{3.272743in}}{\pgfqpoint{1.867776in}{3.277134in}}{\pgfqpoint{1.856726in}{3.277134in}}%
\pgfpathcurveto{\pgfqpoint{1.845675in}{3.277134in}}{\pgfqpoint{1.835076in}{3.272743in}}{\pgfqpoint{1.827263in}{3.264930in}}%
\pgfpathcurveto{\pgfqpoint{1.819449in}{3.257116in}}{\pgfqpoint{1.815059in}{3.246517in}}{\pgfqpoint{1.815059in}{3.235467in}}%
\pgfpathcurveto{\pgfqpoint{1.815059in}{3.224417in}}{\pgfqpoint{1.819449in}{3.213818in}}{\pgfqpoint{1.827263in}{3.206004in}}%
\pgfpathcurveto{\pgfqpoint{1.835076in}{3.198191in}}{\pgfqpoint{1.845675in}{3.193800in}}{\pgfqpoint{1.856726in}{3.193800in}}%
\pgfpathclose%
\pgfusepath{stroke,fill}%
\end{pgfscope}%
\begin{pgfscope}%
\pgfpathrectangle{\pgfqpoint{0.648703in}{0.548769in}}{\pgfqpoint{5.201297in}{3.102590in}}%
\pgfusepath{clip}%
\pgfsetbuttcap%
\pgfsetroundjoin%
\definecolor{currentfill}{rgb}{1.000000,0.498039,0.054902}%
\pgfsetfillcolor{currentfill}%
\pgfsetlinewidth{1.003750pt}%
\definecolor{currentstroke}{rgb}{1.000000,0.498039,0.054902}%
\pgfsetstrokecolor{currentstroke}%
\pgfsetdash{}{0pt}%
\pgfpathmoveto{\pgfqpoint{1.273766in}{3.358719in}}%
\pgfpathcurveto{\pgfqpoint{1.284816in}{3.358719in}}{\pgfqpoint{1.295415in}{3.363109in}}{\pgfqpoint{1.303229in}{3.370923in}}%
\pgfpathcurveto{\pgfqpoint{1.311042in}{3.378737in}}{\pgfqpoint{1.315432in}{3.389336in}}{\pgfqpoint{1.315432in}{3.400386in}}%
\pgfpathcurveto{\pgfqpoint{1.315432in}{3.411436in}}{\pgfqpoint{1.311042in}{3.422035in}}{\pgfqpoint{1.303229in}{3.429849in}}%
\pgfpathcurveto{\pgfqpoint{1.295415in}{3.437662in}}{\pgfqpoint{1.284816in}{3.442053in}}{\pgfqpoint{1.273766in}{3.442053in}}%
\pgfpathcurveto{\pgfqpoint{1.262716in}{3.442053in}}{\pgfqpoint{1.252117in}{3.437662in}}{\pgfqpoint{1.244303in}{3.429849in}}%
\pgfpathcurveto{\pgfqpoint{1.236489in}{3.422035in}}{\pgfqpoint{1.232099in}{3.411436in}}{\pgfqpoint{1.232099in}{3.400386in}}%
\pgfpathcurveto{\pgfqpoint{1.232099in}{3.389336in}}{\pgfqpoint{1.236489in}{3.378737in}}{\pgfqpoint{1.244303in}{3.370923in}}%
\pgfpathcurveto{\pgfqpoint{1.252117in}{3.363109in}}{\pgfqpoint{1.262716in}{3.358719in}}{\pgfqpoint{1.273766in}{3.358719in}}%
\pgfpathclose%
\pgfusepath{stroke,fill}%
\end{pgfscope}%
\begin{pgfscope}%
\pgfpathrectangle{\pgfqpoint{0.648703in}{0.548769in}}{\pgfqpoint{5.201297in}{3.102590in}}%
\pgfusepath{clip}%
\pgfsetbuttcap%
\pgfsetroundjoin%
\definecolor{currentfill}{rgb}{0.121569,0.466667,0.705882}%
\pgfsetfillcolor{currentfill}%
\pgfsetlinewidth{1.003750pt}%
\definecolor{currentstroke}{rgb}{0.121569,0.466667,0.705882}%
\pgfsetstrokecolor{currentstroke}%
\pgfsetdash{}{0pt}%
\pgfpathmoveto{\pgfqpoint{0.949899in}{0.648129in}}%
\pgfpathcurveto{\pgfqpoint{0.960949in}{0.648129in}}{\pgfqpoint{0.971548in}{0.652519in}}{\pgfqpoint{0.979362in}{0.660333in}}%
\pgfpathcurveto{\pgfqpoint{0.987176in}{0.668146in}}{\pgfqpoint{0.991566in}{0.678745in}}{\pgfqpoint{0.991566in}{0.689796in}}%
\pgfpathcurveto{\pgfqpoint{0.991566in}{0.700846in}}{\pgfqpoint{0.987176in}{0.711445in}}{\pgfqpoint{0.979362in}{0.719258in}}%
\pgfpathcurveto{\pgfqpoint{0.971548in}{0.727072in}}{\pgfqpoint{0.960949in}{0.731462in}}{\pgfqpoint{0.949899in}{0.731462in}}%
\pgfpathcurveto{\pgfqpoint{0.938849in}{0.731462in}}{\pgfqpoint{0.928250in}{0.727072in}}{\pgfqpoint{0.920437in}{0.719258in}}%
\pgfpathcurveto{\pgfqpoint{0.912623in}{0.711445in}}{\pgfqpoint{0.908233in}{0.700846in}}{\pgfqpoint{0.908233in}{0.689796in}}%
\pgfpathcurveto{\pgfqpoint{0.908233in}{0.678745in}}{\pgfqpoint{0.912623in}{0.668146in}}{\pgfqpoint{0.920437in}{0.660333in}}%
\pgfpathcurveto{\pgfqpoint{0.928250in}{0.652519in}}{\pgfqpoint{0.938849in}{0.648129in}}{\pgfqpoint{0.949899in}{0.648129in}}%
\pgfpathclose%
\pgfusepath{stroke,fill}%
\end{pgfscope}%
\begin{pgfscope}%
\pgfpathrectangle{\pgfqpoint{0.648703in}{0.548769in}}{\pgfqpoint{5.201297in}{3.102590in}}%
\pgfusepath{clip}%
\pgfsetbuttcap%
\pgfsetroundjoin%
\definecolor{currentfill}{rgb}{0.121569,0.466667,0.705882}%
\pgfsetfillcolor{currentfill}%
\pgfsetlinewidth{1.003750pt}%
\definecolor{currentstroke}{rgb}{0.121569,0.466667,0.705882}%
\pgfsetstrokecolor{currentstroke}%
\pgfsetdash{}{0pt}%
\pgfpathmoveto{\pgfqpoint{1.273766in}{0.796133in}}%
\pgfpathcurveto{\pgfqpoint{1.284816in}{0.796133in}}{\pgfqpoint{1.295415in}{0.800523in}}{\pgfqpoint{1.303229in}{0.808337in}}%
\pgfpathcurveto{\pgfqpoint{1.311042in}{0.816151in}}{\pgfqpoint{1.315432in}{0.826750in}}{\pgfqpoint{1.315432in}{0.837800in}}%
\pgfpathcurveto{\pgfqpoint{1.315432in}{0.848850in}}{\pgfqpoint{1.311042in}{0.859449in}}{\pgfqpoint{1.303229in}{0.867263in}}%
\pgfpathcurveto{\pgfqpoint{1.295415in}{0.875076in}}{\pgfqpoint{1.284816in}{0.879466in}}{\pgfqpoint{1.273766in}{0.879466in}}%
\pgfpathcurveto{\pgfqpoint{1.262716in}{0.879466in}}{\pgfqpoint{1.252117in}{0.875076in}}{\pgfqpoint{1.244303in}{0.867263in}}%
\pgfpathcurveto{\pgfqpoint{1.236489in}{0.859449in}}{\pgfqpoint{1.232099in}{0.848850in}}{\pgfqpoint{1.232099in}{0.837800in}}%
\pgfpathcurveto{\pgfqpoint{1.232099in}{0.826750in}}{\pgfqpoint{1.236489in}{0.816151in}}{\pgfqpoint{1.244303in}{0.808337in}}%
\pgfpathcurveto{\pgfqpoint{1.252117in}{0.800523in}}{\pgfqpoint{1.262716in}{0.796133in}}{\pgfqpoint{1.273766in}{0.796133in}}%
\pgfpathclose%
\pgfusepath{stroke,fill}%
\end{pgfscope}%
\begin{pgfscope}%
\pgfpathrectangle{\pgfqpoint{0.648703in}{0.548769in}}{\pgfqpoint{5.201297in}{3.102590in}}%
\pgfusepath{clip}%
\pgfsetbuttcap%
\pgfsetroundjoin%
\definecolor{currentfill}{rgb}{0.121569,0.466667,0.705882}%
\pgfsetfillcolor{currentfill}%
\pgfsetlinewidth{1.003750pt}%
\definecolor{currentstroke}{rgb}{0.121569,0.466667,0.705882}%
\pgfsetstrokecolor{currentstroke}%
\pgfsetdash{}{0pt}%
\pgfpathmoveto{\pgfqpoint{1.144219in}{3.155742in}}%
\pgfpathcurveto{\pgfqpoint{1.155269in}{3.155742in}}{\pgfqpoint{1.165868in}{3.160132in}}{\pgfqpoint{1.173682in}{3.167946in}}%
\pgfpathcurveto{\pgfqpoint{1.181496in}{3.175760in}}{\pgfqpoint{1.185886in}{3.186359in}}{\pgfqpoint{1.185886in}{3.197409in}}%
\pgfpathcurveto{\pgfqpoint{1.185886in}{3.208459in}}{\pgfqpoint{1.181496in}{3.219058in}}{\pgfqpoint{1.173682in}{3.226872in}}%
\pgfpathcurveto{\pgfqpoint{1.165868in}{3.234685in}}{\pgfqpoint{1.155269in}{3.239075in}}{\pgfqpoint{1.144219in}{3.239075in}}%
\pgfpathcurveto{\pgfqpoint{1.133169in}{3.239075in}}{\pgfqpoint{1.122570in}{3.234685in}}{\pgfqpoint{1.114756in}{3.226872in}}%
\pgfpathcurveto{\pgfqpoint{1.106943in}{3.219058in}}{\pgfqpoint{1.102553in}{3.208459in}}{\pgfqpoint{1.102553in}{3.197409in}}%
\pgfpathcurveto{\pgfqpoint{1.102553in}{3.186359in}}{\pgfqpoint{1.106943in}{3.175760in}}{\pgfqpoint{1.114756in}{3.167946in}}%
\pgfpathcurveto{\pgfqpoint{1.122570in}{3.160132in}}{\pgfqpoint{1.133169in}{3.155742in}}{\pgfqpoint{1.144219in}{3.155742in}}%
\pgfpathclose%
\pgfusepath{stroke,fill}%
\end{pgfscope}%
\begin{pgfscope}%
\pgfpathrectangle{\pgfqpoint{0.648703in}{0.548769in}}{\pgfqpoint{5.201297in}{3.102590in}}%
\pgfusepath{clip}%
\pgfsetbuttcap%
\pgfsetroundjoin%
\definecolor{currentfill}{rgb}{0.121569,0.466667,0.705882}%
\pgfsetfillcolor{currentfill}%
\pgfsetlinewidth{1.003750pt}%
\definecolor{currentstroke}{rgb}{0.121569,0.466667,0.705882}%
\pgfsetstrokecolor{currentstroke}%
\pgfsetdash{}{0pt}%
\pgfpathmoveto{\pgfqpoint{1.208993in}{0.648129in}}%
\pgfpathcurveto{\pgfqpoint{1.220043in}{0.648129in}}{\pgfqpoint{1.230642in}{0.652519in}}{\pgfqpoint{1.238455in}{0.660333in}}%
\pgfpathcurveto{\pgfqpoint{1.246269in}{0.668146in}}{\pgfqpoint{1.250659in}{0.678745in}}{\pgfqpoint{1.250659in}{0.689796in}}%
\pgfpathcurveto{\pgfqpoint{1.250659in}{0.700846in}}{\pgfqpoint{1.246269in}{0.711445in}}{\pgfqpoint{1.238455in}{0.719258in}}%
\pgfpathcurveto{\pgfqpoint{1.230642in}{0.727072in}}{\pgfqpoint{1.220043in}{0.731462in}}{\pgfqpoint{1.208993in}{0.731462in}}%
\pgfpathcurveto{\pgfqpoint{1.197942in}{0.731462in}}{\pgfqpoint{1.187343in}{0.727072in}}{\pgfqpoint{1.179530in}{0.719258in}}%
\pgfpathcurveto{\pgfqpoint{1.171716in}{0.711445in}}{\pgfqpoint{1.167326in}{0.700846in}}{\pgfqpoint{1.167326in}{0.689796in}}%
\pgfpathcurveto{\pgfqpoint{1.167326in}{0.678745in}}{\pgfqpoint{1.171716in}{0.668146in}}{\pgfqpoint{1.179530in}{0.660333in}}%
\pgfpathcurveto{\pgfqpoint{1.187343in}{0.652519in}}{\pgfqpoint{1.197942in}{0.648129in}}{\pgfqpoint{1.208993in}{0.648129in}}%
\pgfpathclose%
\pgfusepath{stroke,fill}%
\end{pgfscope}%
\begin{pgfscope}%
\pgfpathrectangle{\pgfqpoint{0.648703in}{0.548769in}}{\pgfqpoint{5.201297in}{3.102590in}}%
\pgfusepath{clip}%
\pgfsetbuttcap%
\pgfsetroundjoin%
\definecolor{currentfill}{rgb}{0.121569,0.466667,0.705882}%
\pgfsetfillcolor{currentfill}%
\pgfsetlinewidth{1.003750pt}%
\definecolor{currentstroke}{rgb}{0.121569,0.466667,0.705882}%
\pgfsetstrokecolor{currentstroke}%
\pgfsetdash{}{0pt}%
\pgfpathmoveto{\pgfqpoint{1.208993in}{0.648129in}}%
\pgfpathcurveto{\pgfqpoint{1.220043in}{0.648129in}}{\pgfqpoint{1.230642in}{0.652519in}}{\pgfqpoint{1.238455in}{0.660333in}}%
\pgfpathcurveto{\pgfqpoint{1.246269in}{0.668146in}}{\pgfqpoint{1.250659in}{0.678745in}}{\pgfqpoint{1.250659in}{0.689796in}}%
\pgfpathcurveto{\pgfqpoint{1.250659in}{0.700846in}}{\pgfqpoint{1.246269in}{0.711445in}}{\pgfqpoint{1.238455in}{0.719258in}}%
\pgfpathcurveto{\pgfqpoint{1.230642in}{0.727072in}}{\pgfqpoint{1.220043in}{0.731462in}}{\pgfqpoint{1.208993in}{0.731462in}}%
\pgfpathcurveto{\pgfqpoint{1.197942in}{0.731462in}}{\pgfqpoint{1.187343in}{0.727072in}}{\pgfqpoint{1.179530in}{0.719258in}}%
\pgfpathcurveto{\pgfqpoint{1.171716in}{0.711445in}}{\pgfqpoint{1.167326in}{0.700846in}}{\pgfqpoint{1.167326in}{0.689796in}}%
\pgfpathcurveto{\pgfqpoint{1.167326in}{0.678745in}}{\pgfqpoint{1.171716in}{0.668146in}}{\pgfqpoint{1.179530in}{0.660333in}}%
\pgfpathcurveto{\pgfqpoint{1.187343in}{0.652519in}}{\pgfqpoint{1.197942in}{0.648129in}}{\pgfqpoint{1.208993in}{0.648129in}}%
\pgfpathclose%
\pgfusepath{stroke,fill}%
\end{pgfscope}%
\begin{pgfscope}%
\pgfpathrectangle{\pgfqpoint{0.648703in}{0.548769in}}{\pgfqpoint{5.201297in}{3.102590in}}%
\pgfusepath{clip}%
\pgfsetbuttcap%
\pgfsetroundjoin%
\definecolor{currentfill}{rgb}{0.121569,0.466667,0.705882}%
\pgfsetfillcolor{currentfill}%
\pgfsetlinewidth{1.003750pt}%
\definecolor{currentstroke}{rgb}{0.121569,0.466667,0.705882}%
\pgfsetstrokecolor{currentstroke}%
\pgfsetdash{}{0pt}%
\pgfpathmoveto{\pgfqpoint{0.949899in}{2.512981in}}%
\pgfpathcurveto{\pgfqpoint{0.960949in}{2.512981in}}{\pgfqpoint{0.971548in}{2.517371in}}{\pgfqpoint{0.979362in}{2.525185in}}%
\pgfpathcurveto{\pgfqpoint{0.987176in}{2.532999in}}{\pgfqpoint{0.991566in}{2.543598in}}{\pgfqpoint{0.991566in}{2.554648in}}%
\pgfpathcurveto{\pgfqpoint{0.991566in}{2.565698in}}{\pgfqpoint{0.987176in}{2.576297in}}{\pgfqpoint{0.979362in}{2.584111in}}%
\pgfpathcurveto{\pgfqpoint{0.971548in}{2.591924in}}{\pgfqpoint{0.960949in}{2.596315in}}{\pgfqpoint{0.949899in}{2.596315in}}%
\pgfpathcurveto{\pgfqpoint{0.938849in}{2.596315in}}{\pgfqpoint{0.928250in}{2.591924in}}{\pgfqpoint{0.920437in}{2.584111in}}%
\pgfpathcurveto{\pgfqpoint{0.912623in}{2.576297in}}{\pgfqpoint{0.908233in}{2.565698in}}{\pgfqpoint{0.908233in}{2.554648in}}%
\pgfpathcurveto{\pgfqpoint{0.908233in}{2.543598in}}{\pgfqpoint{0.912623in}{2.532999in}}{\pgfqpoint{0.920437in}{2.525185in}}%
\pgfpathcurveto{\pgfqpoint{0.928250in}{2.517371in}}{\pgfqpoint{0.938849in}{2.512981in}}{\pgfqpoint{0.949899in}{2.512981in}}%
\pgfpathclose%
\pgfusepath{stroke,fill}%
\end{pgfscope}%
\begin{pgfscope}%
\pgfpathrectangle{\pgfqpoint{0.648703in}{0.548769in}}{\pgfqpoint{5.201297in}{3.102590in}}%
\pgfusepath{clip}%
\pgfsetbuttcap%
\pgfsetroundjoin%
\definecolor{currentfill}{rgb}{1.000000,0.498039,0.054902}%
\pgfsetfillcolor{currentfill}%
\pgfsetlinewidth{1.003750pt}%
\definecolor{currentstroke}{rgb}{1.000000,0.498039,0.054902}%
\pgfsetstrokecolor{currentstroke}%
\pgfsetdash{}{0pt}%
\pgfpathmoveto{\pgfqpoint{0.885126in}{3.206486in}}%
\pgfpathcurveto{\pgfqpoint{0.896176in}{3.206486in}}{\pgfqpoint{0.906775in}{3.210877in}}{\pgfqpoint{0.914589in}{3.218690in}}%
\pgfpathcurveto{\pgfqpoint{0.922402in}{3.226504in}}{\pgfqpoint{0.926793in}{3.237103in}}{\pgfqpoint{0.926793in}{3.248153in}}%
\pgfpathcurveto{\pgfqpoint{0.926793in}{3.259203in}}{\pgfqpoint{0.922402in}{3.269802in}}{\pgfqpoint{0.914589in}{3.277616in}}%
\pgfpathcurveto{\pgfqpoint{0.906775in}{3.285429in}}{\pgfqpoint{0.896176in}{3.289820in}}{\pgfqpoint{0.885126in}{3.289820in}}%
\pgfpathcurveto{\pgfqpoint{0.874076in}{3.289820in}}{\pgfqpoint{0.863477in}{3.285429in}}{\pgfqpoint{0.855663in}{3.277616in}}%
\pgfpathcurveto{\pgfqpoint{0.847850in}{3.269802in}}{\pgfqpoint{0.843459in}{3.259203in}}{\pgfqpoint{0.843459in}{3.248153in}}%
\pgfpathcurveto{\pgfqpoint{0.843459in}{3.237103in}}{\pgfqpoint{0.847850in}{3.226504in}}{\pgfqpoint{0.855663in}{3.218690in}}%
\pgfpathcurveto{\pgfqpoint{0.863477in}{3.210877in}}{\pgfqpoint{0.874076in}{3.206486in}}{\pgfqpoint{0.885126in}{3.206486in}}%
\pgfpathclose%
\pgfusepath{stroke,fill}%
\end{pgfscope}%
\begin{pgfscope}%
\pgfpathrectangle{\pgfqpoint{0.648703in}{0.548769in}}{\pgfqpoint{5.201297in}{3.102590in}}%
\pgfusepath{clip}%
\pgfsetbuttcap%
\pgfsetroundjoin%
\definecolor{currentfill}{rgb}{1.000000,0.498039,0.054902}%
\pgfsetfillcolor{currentfill}%
\pgfsetlinewidth{1.003750pt}%
\definecolor{currentstroke}{rgb}{1.000000,0.498039,0.054902}%
\pgfsetstrokecolor{currentstroke}%
\pgfsetdash{}{0pt}%
\pgfpathmoveto{\pgfqpoint{2.957872in}{3.198029in}}%
\pgfpathcurveto{\pgfqpoint{2.968922in}{3.198029in}}{\pgfqpoint{2.979521in}{3.202419in}}{\pgfqpoint{2.987335in}{3.210233in}}%
\pgfpathcurveto{\pgfqpoint{2.995148in}{3.218046in}}{\pgfqpoint{2.999538in}{3.228646in}}{\pgfqpoint{2.999538in}{3.239696in}}%
\pgfpathcurveto{\pgfqpoint{2.999538in}{3.250746in}}{\pgfqpoint{2.995148in}{3.261345in}}{\pgfqpoint{2.987335in}{3.269158in}}%
\pgfpathcurveto{\pgfqpoint{2.979521in}{3.276972in}}{\pgfqpoint{2.968922in}{3.281362in}}{\pgfqpoint{2.957872in}{3.281362in}}%
\pgfpathcurveto{\pgfqpoint{2.946822in}{3.281362in}}{\pgfqpoint{2.936223in}{3.276972in}}{\pgfqpoint{2.928409in}{3.269158in}}%
\pgfpathcurveto{\pgfqpoint{2.920595in}{3.261345in}}{\pgfqpoint{2.916205in}{3.250746in}}{\pgfqpoint{2.916205in}{3.239696in}}%
\pgfpathcurveto{\pgfqpoint{2.916205in}{3.228646in}}{\pgfqpoint{2.920595in}{3.218046in}}{\pgfqpoint{2.928409in}{3.210233in}}%
\pgfpathcurveto{\pgfqpoint{2.936223in}{3.202419in}}{\pgfqpoint{2.946822in}{3.198029in}}{\pgfqpoint{2.957872in}{3.198029in}}%
\pgfpathclose%
\pgfusepath{stroke,fill}%
\end{pgfscope}%
\begin{pgfscope}%
\pgfpathrectangle{\pgfqpoint{0.648703in}{0.548769in}}{\pgfqpoint{5.201297in}{3.102590in}}%
\pgfusepath{clip}%
\pgfsetbuttcap%
\pgfsetroundjoin%
\definecolor{currentfill}{rgb}{1.000000,0.498039,0.054902}%
\pgfsetfillcolor{currentfill}%
\pgfsetlinewidth{1.003750pt}%
\definecolor{currentstroke}{rgb}{1.000000,0.498039,0.054902}%
\pgfsetstrokecolor{currentstroke}%
\pgfsetdash{}{0pt}%
\pgfpathmoveto{\pgfqpoint{1.727179in}{3.248773in}}%
\pgfpathcurveto{\pgfqpoint{1.738229in}{3.248773in}}{\pgfqpoint{1.748828in}{3.253164in}}{\pgfqpoint{1.756642in}{3.260977in}}%
\pgfpathcurveto{\pgfqpoint{1.764455in}{3.268791in}}{\pgfqpoint{1.768846in}{3.279390in}}{\pgfqpoint{1.768846in}{3.290440in}}%
\pgfpathcurveto{\pgfqpoint{1.768846in}{3.301490in}}{\pgfqpoint{1.764455in}{3.312089in}}{\pgfqpoint{1.756642in}{3.319903in}}%
\pgfpathcurveto{\pgfqpoint{1.748828in}{3.327716in}}{\pgfqpoint{1.738229in}{3.332107in}}{\pgfqpoint{1.727179in}{3.332107in}}%
\pgfpathcurveto{\pgfqpoint{1.716129in}{3.332107in}}{\pgfqpoint{1.705530in}{3.327716in}}{\pgfqpoint{1.697716in}{3.319903in}}%
\pgfpathcurveto{\pgfqpoint{1.689903in}{3.312089in}}{\pgfqpoint{1.685512in}{3.301490in}}{\pgfqpoint{1.685512in}{3.290440in}}%
\pgfpathcurveto{\pgfqpoint{1.685512in}{3.279390in}}{\pgfqpoint{1.689903in}{3.268791in}}{\pgfqpoint{1.697716in}{3.260977in}}%
\pgfpathcurveto{\pgfqpoint{1.705530in}{3.253164in}}{\pgfqpoint{1.716129in}{3.248773in}}{\pgfqpoint{1.727179in}{3.248773in}}%
\pgfpathclose%
\pgfusepath{stroke,fill}%
\end{pgfscope}%
\begin{pgfscope}%
\pgfpathrectangle{\pgfqpoint{0.648703in}{0.548769in}}{\pgfqpoint{5.201297in}{3.102590in}}%
\pgfusepath{clip}%
\pgfsetbuttcap%
\pgfsetroundjoin%
\definecolor{currentfill}{rgb}{1.000000,0.498039,0.054902}%
\pgfsetfillcolor{currentfill}%
\pgfsetlinewidth{1.003750pt}%
\definecolor{currentstroke}{rgb}{1.000000,0.498039,0.054902}%
\pgfsetstrokecolor{currentstroke}%
\pgfsetdash{}{0pt}%
\pgfpathmoveto{\pgfqpoint{1.208993in}{3.189572in}}%
\pgfpathcurveto{\pgfqpoint{1.220043in}{3.189572in}}{\pgfqpoint{1.230642in}{3.193962in}}{\pgfqpoint{1.238455in}{3.201775in}}%
\pgfpathcurveto{\pgfqpoint{1.246269in}{3.209589in}}{\pgfqpoint{1.250659in}{3.220188in}}{\pgfqpoint{1.250659in}{3.231238in}}%
\pgfpathcurveto{\pgfqpoint{1.250659in}{3.242288in}}{\pgfqpoint{1.246269in}{3.252887in}}{\pgfqpoint{1.238455in}{3.260701in}}%
\pgfpathcurveto{\pgfqpoint{1.230642in}{3.268515in}}{\pgfqpoint{1.220043in}{3.272905in}}{\pgfqpoint{1.208993in}{3.272905in}}%
\pgfpathcurveto{\pgfqpoint{1.197942in}{3.272905in}}{\pgfqpoint{1.187343in}{3.268515in}}{\pgfqpoint{1.179530in}{3.260701in}}%
\pgfpathcurveto{\pgfqpoint{1.171716in}{3.252887in}}{\pgfqpoint{1.167326in}{3.242288in}}{\pgfqpoint{1.167326in}{3.231238in}}%
\pgfpathcurveto{\pgfqpoint{1.167326in}{3.220188in}}{\pgfqpoint{1.171716in}{3.209589in}}{\pgfqpoint{1.179530in}{3.201775in}}%
\pgfpathcurveto{\pgfqpoint{1.187343in}{3.193962in}}{\pgfqpoint{1.197942in}{3.189572in}}{\pgfqpoint{1.208993in}{3.189572in}}%
\pgfpathclose%
\pgfusepath{stroke,fill}%
\end{pgfscope}%
\begin{pgfscope}%
\pgfpathrectangle{\pgfqpoint{0.648703in}{0.548769in}}{\pgfqpoint{5.201297in}{3.102590in}}%
\pgfusepath{clip}%
\pgfsetbuttcap%
\pgfsetroundjoin%
\definecolor{currentfill}{rgb}{0.121569,0.466667,0.705882}%
\pgfsetfillcolor{currentfill}%
\pgfsetlinewidth{1.003750pt}%
\definecolor{currentstroke}{rgb}{0.121569,0.466667,0.705882}%
\pgfsetstrokecolor{currentstroke}%
\pgfsetdash{}{0pt}%
\pgfpathmoveto{\pgfqpoint{3.864698in}{3.181114in}}%
\pgfpathcurveto{\pgfqpoint{3.875748in}{3.181114in}}{\pgfqpoint{3.886347in}{3.185504in}}{\pgfqpoint{3.894161in}{3.193318in}}%
\pgfpathcurveto{\pgfqpoint{3.901975in}{3.201132in}}{\pgfqpoint{3.906365in}{3.211731in}}{\pgfqpoint{3.906365in}{3.222781in}}%
\pgfpathcurveto{\pgfqpoint{3.906365in}{3.233831in}}{\pgfqpoint{3.901975in}{3.244430in}}{\pgfqpoint{3.894161in}{3.252244in}}%
\pgfpathcurveto{\pgfqpoint{3.886347in}{3.260057in}}{\pgfqpoint{3.875748in}{3.264448in}}{\pgfqpoint{3.864698in}{3.264448in}}%
\pgfpathcurveto{\pgfqpoint{3.853648in}{3.264448in}}{\pgfqpoint{3.843049in}{3.260057in}}{\pgfqpoint{3.835235in}{3.252244in}}%
\pgfpathcurveto{\pgfqpoint{3.827422in}{3.244430in}}{\pgfqpoint{3.823031in}{3.233831in}}{\pgfqpoint{3.823031in}{3.222781in}}%
\pgfpathcurveto{\pgfqpoint{3.823031in}{3.211731in}}{\pgfqpoint{3.827422in}{3.201132in}}{\pgfqpoint{3.835235in}{3.193318in}}%
\pgfpathcurveto{\pgfqpoint{3.843049in}{3.185504in}}{\pgfqpoint{3.853648in}{3.181114in}}{\pgfqpoint{3.864698in}{3.181114in}}%
\pgfpathclose%
\pgfusepath{stroke,fill}%
\end{pgfscope}%
\begin{pgfscope}%
\pgfpathrectangle{\pgfqpoint{0.648703in}{0.548769in}}{\pgfqpoint{5.201297in}{3.102590in}}%
\pgfusepath{clip}%
\pgfsetbuttcap%
\pgfsetroundjoin%
\definecolor{currentfill}{rgb}{0.839216,0.152941,0.156863}%
\pgfsetfillcolor{currentfill}%
\pgfsetlinewidth{1.003750pt}%
\definecolor{currentstroke}{rgb}{0.839216,0.152941,0.156863}%
\pgfsetstrokecolor{currentstroke}%
\pgfsetdash{}{0pt}%
\pgfpathmoveto{\pgfqpoint{3.799925in}{3.202258in}}%
\pgfpathcurveto{\pgfqpoint{3.810975in}{3.202258in}}{\pgfqpoint{3.821574in}{3.206648in}}{\pgfqpoint{3.829388in}{3.214462in}}%
\pgfpathcurveto{\pgfqpoint{3.837201in}{3.222275in}}{\pgfqpoint{3.841591in}{3.232874in}}{\pgfqpoint{3.841591in}{3.243924in}}%
\pgfpathcurveto{\pgfqpoint{3.841591in}{3.254974in}}{\pgfqpoint{3.837201in}{3.265573in}}{\pgfqpoint{3.829388in}{3.273387in}}%
\pgfpathcurveto{\pgfqpoint{3.821574in}{3.281201in}}{\pgfqpoint{3.810975in}{3.285591in}}{\pgfqpoint{3.799925in}{3.285591in}}%
\pgfpathcurveto{\pgfqpoint{3.788875in}{3.285591in}}{\pgfqpoint{3.778276in}{3.281201in}}{\pgfqpoint{3.770462in}{3.273387in}}%
\pgfpathcurveto{\pgfqpoint{3.762648in}{3.265573in}}{\pgfqpoint{3.758258in}{3.254974in}}{\pgfqpoint{3.758258in}{3.243924in}}%
\pgfpathcurveto{\pgfqpoint{3.758258in}{3.232874in}}{\pgfqpoint{3.762648in}{3.222275in}}{\pgfqpoint{3.770462in}{3.214462in}}%
\pgfpathcurveto{\pgfqpoint{3.778276in}{3.206648in}}{\pgfqpoint{3.788875in}{3.202258in}}{\pgfqpoint{3.799925in}{3.202258in}}%
\pgfpathclose%
\pgfusepath{stroke,fill}%
\end{pgfscope}%
\begin{pgfscope}%
\pgfpathrectangle{\pgfqpoint{0.648703in}{0.548769in}}{\pgfqpoint{5.201297in}{3.102590in}}%
\pgfusepath{clip}%
\pgfsetbuttcap%
\pgfsetroundjoin%
\definecolor{currentfill}{rgb}{1.000000,0.498039,0.054902}%
\pgfsetfillcolor{currentfill}%
\pgfsetlinewidth{1.003750pt}%
\definecolor{currentstroke}{rgb}{1.000000,0.498039,0.054902}%
\pgfsetstrokecolor{currentstroke}%
\pgfsetdash{}{0pt}%
\pgfpathmoveto{\pgfqpoint{1.403312in}{3.185343in}}%
\pgfpathcurveto{\pgfqpoint{1.414363in}{3.185343in}}{\pgfqpoint{1.424962in}{3.189733in}}{\pgfqpoint{1.432775in}{3.197547in}}%
\pgfpathcurveto{\pgfqpoint{1.440589in}{3.205360in}}{\pgfqpoint{1.444979in}{3.215959in}}{\pgfqpoint{1.444979in}{3.227010in}}%
\pgfpathcurveto{\pgfqpoint{1.444979in}{3.238060in}}{\pgfqpoint{1.440589in}{3.248659in}}{\pgfqpoint{1.432775in}{3.256472in}}%
\pgfpathcurveto{\pgfqpoint{1.424962in}{3.264286in}}{\pgfqpoint{1.414363in}{3.268676in}}{\pgfqpoint{1.403312in}{3.268676in}}%
\pgfpathcurveto{\pgfqpoint{1.392262in}{3.268676in}}{\pgfqpoint{1.381663in}{3.264286in}}{\pgfqpoint{1.373850in}{3.256472in}}%
\pgfpathcurveto{\pgfqpoint{1.366036in}{3.248659in}}{\pgfqpoint{1.361646in}{3.238060in}}{\pgfqpoint{1.361646in}{3.227010in}}%
\pgfpathcurveto{\pgfqpoint{1.361646in}{3.215959in}}{\pgfqpoint{1.366036in}{3.205360in}}{\pgfqpoint{1.373850in}{3.197547in}}%
\pgfpathcurveto{\pgfqpoint{1.381663in}{3.189733in}}{\pgfqpoint{1.392262in}{3.185343in}}{\pgfqpoint{1.403312in}{3.185343in}}%
\pgfpathclose%
\pgfusepath{stroke,fill}%
\end{pgfscope}%
\begin{pgfscope}%
\pgfpathrectangle{\pgfqpoint{0.648703in}{0.548769in}}{\pgfqpoint{5.201297in}{3.102590in}}%
\pgfusepath{clip}%
\pgfsetbuttcap%
\pgfsetroundjoin%
\definecolor{currentfill}{rgb}{1.000000,0.498039,0.054902}%
\pgfsetfillcolor{currentfill}%
\pgfsetlinewidth{1.003750pt}%
\definecolor{currentstroke}{rgb}{1.000000,0.498039,0.054902}%
\pgfsetstrokecolor{currentstroke}%
\pgfsetdash{}{0pt}%
\pgfpathmoveto{\pgfqpoint{3.022645in}{3.278374in}}%
\pgfpathcurveto{\pgfqpoint{3.033695in}{3.278374in}}{\pgfqpoint{3.044294in}{3.282764in}}{\pgfqpoint{3.052108in}{3.290578in}}%
\pgfpathcurveto{\pgfqpoint{3.059922in}{3.298392in}}{\pgfqpoint{3.064312in}{3.308991in}}{\pgfqpoint{3.064312in}{3.320041in}}%
\pgfpathcurveto{\pgfqpoint{3.064312in}{3.331091in}}{\pgfqpoint{3.059922in}{3.341690in}}{\pgfqpoint{3.052108in}{3.349504in}}%
\pgfpathcurveto{\pgfqpoint{3.044294in}{3.357317in}}{\pgfqpoint{3.033695in}{3.361707in}}{\pgfqpoint{3.022645in}{3.361707in}}%
\pgfpathcurveto{\pgfqpoint{3.011595in}{3.361707in}}{\pgfqpoint{3.000996in}{3.357317in}}{\pgfqpoint{2.993182in}{3.349504in}}%
\pgfpathcurveto{\pgfqpoint{2.985369in}{3.341690in}}{\pgfqpoint{2.980978in}{3.331091in}}{\pgfqpoint{2.980978in}{3.320041in}}%
\pgfpathcurveto{\pgfqpoint{2.980978in}{3.308991in}}{\pgfqpoint{2.985369in}{3.298392in}}{\pgfqpoint{2.993182in}{3.290578in}}%
\pgfpathcurveto{\pgfqpoint{3.000996in}{3.282764in}}{\pgfqpoint{3.011595in}{3.278374in}}{\pgfqpoint{3.022645in}{3.278374in}}%
\pgfpathclose%
\pgfusepath{stroke,fill}%
\end{pgfscope}%
\begin{pgfscope}%
\pgfpathrectangle{\pgfqpoint{0.648703in}{0.548769in}}{\pgfqpoint{5.201297in}{3.102590in}}%
\pgfusepath{clip}%
\pgfsetbuttcap%
\pgfsetroundjoin%
\definecolor{currentfill}{rgb}{0.121569,0.466667,0.705882}%
\pgfsetfillcolor{currentfill}%
\pgfsetlinewidth{1.003750pt}%
\definecolor{currentstroke}{rgb}{0.121569,0.466667,0.705882}%
\pgfsetstrokecolor{currentstroke}%
\pgfsetdash{}{0pt}%
\pgfpathmoveto{\pgfqpoint{2.634005in}{3.181114in}}%
\pgfpathcurveto{\pgfqpoint{2.645055in}{3.181114in}}{\pgfqpoint{2.655654in}{3.185504in}}{\pgfqpoint{2.663468in}{3.193318in}}%
\pgfpathcurveto{\pgfqpoint{2.671282in}{3.201132in}}{\pgfqpoint{2.675672in}{3.211731in}}{\pgfqpoint{2.675672in}{3.222781in}}%
\pgfpathcurveto{\pgfqpoint{2.675672in}{3.233831in}}{\pgfqpoint{2.671282in}{3.244430in}}{\pgfqpoint{2.663468in}{3.252244in}}%
\pgfpathcurveto{\pgfqpoint{2.655654in}{3.260057in}}{\pgfqpoint{2.645055in}{3.264448in}}{\pgfqpoint{2.634005in}{3.264448in}}%
\pgfpathcurveto{\pgfqpoint{2.622955in}{3.264448in}}{\pgfqpoint{2.612356in}{3.260057in}}{\pgfqpoint{2.604543in}{3.252244in}}%
\pgfpathcurveto{\pgfqpoint{2.596729in}{3.244430in}}{\pgfqpoint{2.592339in}{3.233831in}}{\pgfqpoint{2.592339in}{3.222781in}}%
\pgfpathcurveto{\pgfqpoint{2.592339in}{3.211731in}}{\pgfqpoint{2.596729in}{3.201132in}}{\pgfqpoint{2.604543in}{3.193318in}}%
\pgfpathcurveto{\pgfqpoint{2.612356in}{3.185504in}}{\pgfqpoint{2.622955in}{3.181114in}}{\pgfqpoint{2.634005in}{3.181114in}}%
\pgfpathclose%
\pgfusepath{stroke,fill}%
\end{pgfscope}%
\begin{pgfscope}%
\pgfpathrectangle{\pgfqpoint{0.648703in}{0.548769in}}{\pgfqpoint{5.201297in}{3.102590in}}%
\pgfusepath{clip}%
\pgfsetbuttcap%
\pgfsetroundjoin%
\definecolor{currentfill}{rgb}{1.000000,0.498039,0.054902}%
\pgfsetfillcolor{currentfill}%
\pgfsetlinewidth{1.003750pt}%
\definecolor{currentstroke}{rgb}{1.000000,0.498039,0.054902}%
\pgfsetstrokecolor{currentstroke}%
\pgfsetdash{}{0pt}%
\pgfpathmoveto{\pgfqpoint{1.014673in}{3.219172in}}%
\pgfpathcurveto{\pgfqpoint{1.025723in}{3.219172in}}{\pgfqpoint{1.036322in}{3.223563in}}{\pgfqpoint{1.044135in}{3.231376in}}%
\pgfpathcurveto{\pgfqpoint{1.051949in}{3.239190in}}{\pgfqpoint{1.056339in}{3.249789in}}{\pgfqpoint{1.056339in}{3.260839in}}%
\pgfpathcurveto{\pgfqpoint{1.056339in}{3.271889in}}{\pgfqpoint{1.051949in}{3.282488in}}{\pgfqpoint{1.044135in}{3.290302in}}%
\pgfpathcurveto{\pgfqpoint{1.036322in}{3.298116in}}{\pgfqpoint{1.025723in}{3.302506in}}{\pgfqpoint{1.014673in}{3.302506in}}%
\pgfpathcurveto{\pgfqpoint{1.003622in}{3.302506in}}{\pgfqpoint{0.993023in}{3.298116in}}{\pgfqpoint{0.985210in}{3.290302in}}%
\pgfpathcurveto{\pgfqpoint{0.977396in}{3.282488in}}{\pgfqpoint{0.973006in}{3.271889in}}{\pgfqpoint{0.973006in}{3.260839in}}%
\pgfpathcurveto{\pgfqpoint{0.973006in}{3.249789in}}{\pgfqpoint{0.977396in}{3.239190in}}{\pgfqpoint{0.985210in}{3.231376in}}%
\pgfpathcurveto{\pgfqpoint{0.993023in}{3.223563in}}{\pgfqpoint{1.003622in}{3.219172in}}{\pgfqpoint{1.014673in}{3.219172in}}%
\pgfpathclose%
\pgfusepath{stroke,fill}%
\end{pgfscope}%
\begin{pgfscope}%
\pgfpathrectangle{\pgfqpoint{0.648703in}{0.548769in}}{\pgfqpoint{5.201297in}{3.102590in}}%
\pgfusepath{clip}%
\pgfsetbuttcap%
\pgfsetroundjoin%
\definecolor{currentfill}{rgb}{1.000000,0.498039,0.054902}%
\pgfsetfillcolor{currentfill}%
\pgfsetlinewidth{1.003750pt}%
\definecolor{currentstroke}{rgb}{1.000000,0.498039,0.054902}%
\pgfsetstrokecolor{currentstroke}%
\pgfsetdash{}{0pt}%
\pgfpathmoveto{\pgfqpoint{2.439685in}{3.468665in}}%
\pgfpathcurveto{\pgfqpoint{2.450735in}{3.468665in}}{\pgfqpoint{2.461335in}{3.473055in}}{\pgfqpoint{2.469148in}{3.480869in}}%
\pgfpathcurveto{\pgfqpoint{2.476962in}{3.488683in}}{\pgfqpoint{2.481352in}{3.499282in}}{\pgfqpoint{2.481352in}{3.510332in}}%
\pgfpathcurveto{\pgfqpoint{2.481352in}{3.521382in}}{\pgfqpoint{2.476962in}{3.531981in}}{\pgfqpoint{2.469148in}{3.539795in}}%
\pgfpathcurveto{\pgfqpoint{2.461335in}{3.547608in}}{\pgfqpoint{2.450735in}{3.551998in}}{\pgfqpoint{2.439685in}{3.551998in}}%
\pgfpathcurveto{\pgfqpoint{2.428635in}{3.551998in}}{\pgfqpoint{2.418036in}{3.547608in}}{\pgfqpoint{2.410223in}{3.539795in}}%
\pgfpathcurveto{\pgfqpoint{2.402409in}{3.531981in}}{\pgfqpoint{2.398019in}{3.521382in}}{\pgfqpoint{2.398019in}{3.510332in}}%
\pgfpathcurveto{\pgfqpoint{2.398019in}{3.499282in}}{\pgfqpoint{2.402409in}{3.488683in}}{\pgfqpoint{2.410223in}{3.480869in}}%
\pgfpathcurveto{\pgfqpoint{2.418036in}{3.473055in}}{\pgfqpoint{2.428635in}{3.468665in}}{\pgfqpoint{2.439685in}{3.468665in}}%
\pgfpathclose%
\pgfusepath{stroke,fill}%
\end{pgfscope}%
\begin{pgfscope}%
\pgfpathrectangle{\pgfqpoint{0.648703in}{0.548769in}}{\pgfqpoint{5.201297in}{3.102590in}}%
\pgfusepath{clip}%
\pgfsetbuttcap%
\pgfsetroundjoin%
\definecolor{currentfill}{rgb}{1.000000,0.498039,0.054902}%
\pgfsetfillcolor{currentfill}%
\pgfsetlinewidth{1.003750pt}%
\definecolor{currentstroke}{rgb}{1.000000,0.498039,0.054902}%
\pgfsetstrokecolor{currentstroke}%
\pgfsetdash{}{0pt}%
\pgfpathmoveto{\pgfqpoint{1.986272in}{3.189572in}}%
\pgfpathcurveto{\pgfqpoint{1.997322in}{3.189572in}}{\pgfqpoint{2.007921in}{3.193962in}}{\pgfqpoint{2.015735in}{3.201775in}}%
\pgfpathcurveto{\pgfqpoint{2.023549in}{3.209589in}}{\pgfqpoint{2.027939in}{3.220188in}}{\pgfqpoint{2.027939in}{3.231238in}}%
\pgfpathcurveto{\pgfqpoint{2.027939in}{3.242288in}}{\pgfqpoint{2.023549in}{3.252887in}}{\pgfqpoint{2.015735in}{3.260701in}}%
\pgfpathcurveto{\pgfqpoint{2.007921in}{3.268515in}}{\pgfqpoint{1.997322in}{3.272905in}}{\pgfqpoint{1.986272in}{3.272905in}}%
\pgfpathcurveto{\pgfqpoint{1.975222in}{3.272905in}}{\pgfqpoint{1.964623in}{3.268515in}}{\pgfqpoint{1.956809in}{3.260701in}}%
\pgfpathcurveto{\pgfqpoint{1.948996in}{3.252887in}}{\pgfqpoint{1.944606in}{3.242288in}}{\pgfqpoint{1.944606in}{3.231238in}}%
\pgfpathcurveto{\pgfqpoint{1.944606in}{3.220188in}}{\pgfqpoint{1.948996in}{3.209589in}}{\pgfqpoint{1.956809in}{3.201775in}}%
\pgfpathcurveto{\pgfqpoint{1.964623in}{3.193962in}}{\pgfqpoint{1.975222in}{3.189572in}}{\pgfqpoint{1.986272in}{3.189572in}}%
\pgfpathclose%
\pgfusepath{stroke,fill}%
\end{pgfscope}%
\begin{pgfscope}%
\pgfpathrectangle{\pgfqpoint{0.648703in}{0.548769in}}{\pgfqpoint{5.201297in}{3.102590in}}%
\pgfusepath{clip}%
\pgfsetbuttcap%
\pgfsetroundjoin%
\definecolor{currentfill}{rgb}{1.000000,0.498039,0.054902}%
\pgfsetfillcolor{currentfill}%
\pgfsetlinewidth{1.003750pt}%
\definecolor{currentstroke}{rgb}{1.000000,0.498039,0.054902}%
\pgfsetstrokecolor{currentstroke}%
\pgfsetdash{}{0pt}%
\pgfpathmoveto{\pgfqpoint{1.921499in}{3.189572in}}%
\pgfpathcurveto{\pgfqpoint{1.932549in}{3.189572in}}{\pgfqpoint{1.943148in}{3.193962in}}{\pgfqpoint{1.950962in}{3.201775in}}%
\pgfpathcurveto{\pgfqpoint{1.958775in}{3.209589in}}{\pgfqpoint{1.963166in}{3.220188in}}{\pgfqpoint{1.963166in}{3.231238in}}%
\pgfpathcurveto{\pgfqpoint{1.963166in}{3.242288in}}{\pgfqpoint{1.958775in}{3.252887in}}{\pgfqpoint{1.950962in}{3.260701in}}%
\pgfpathcurveto{\pgfqpoint{1.943148in}{3.268515in}}{\pgfqpoint{1.932549in}{3.272905in}}{\pgfqpoint{1.921499in}{3.272905in}}%
\pgfpathcurveto{\pgfqpoint{1.910449in}{3.272905in}}{\pgfqpoint{1.899850in}{3.268515in}}{\pgfqpoint{1.892036in}{3.260701in}}%
\pgfpathcurveto{\pgfqpoint{1.884223in}{3.252887in}}{\pgfqpoint{1.879832in}{3.242288in}}{\pgfqpoint{1.879832in}{3.231238in}}%
\pgfpathcurveto{\pgfqpoint{1.879832in}{3.220188in}}{\pgfqpoint{1.884223in}{3.209589in}}{\pgfqpoint{1.892036in}{3.201775in}}%
\pgfpathcurveto{\pgfqpoint{1.899850in}{3.193962in}}{\pgfqpoint{1.910449in}{3.189572in}}{\pgfqpoint{1.921499in}{3.189572in}}%
\pgfpathclose%
\pgfusepath{stroke,fill}%
\end{pgfscope}%
\begin{pgfscope}%
\pgfpathrectangle{\pgfqpoint{0.648703in}{0.548769in}}{\pgfqpoint{5.201297in}{3.102590in}}%
\pgfusepath{clip}%
\pgfsetbuttcap%
\pgfsetroundjoin%
\definecolor{currentfill}{rgb}{0.121569,0.466667,0.705882}%
\pgfsetfillcolor{currentfill}%
\pgfsetlinewidth{1.003750pt}%
\definecolor{currentstroke}{rgb}{0.121569,0.466667,0.705882}%
\pgfsetstrokecolor{currentstroke}%
\pgfsetdash{}{0pt}%
\pgfpathmoveto{\pgfqpoint{1.921499in}{3.181114in}}%
\pgfpathcurveto{\pgfqpoint{1.932549in}{3.181114in}}{\pgfqpoint{1.943148in}{3.185504in}}{\pgfqpoint{1.950962in}{3.193318in}}%
\pgfpathcurveto{\pgfqpoint{1.958775in}{3.201132in}}{\pgfqpoint{1.963166in}{3.211731in}}{\pgfqpoint{1.963166in}{3.222781in}}%
\pgfpathcurveto{\pgfqpoint{1.963166in}{3.233831in}}{\pgfqpoint{1.958775in}{3.244430in}}{\pgfqpoint{1.950962in}{3.252244in}}%
\pgfpathcurveto{\pgfqpoint{1.943148in}{3.260057in}}{\pgfqpoint{1.932549in}{3.264448in}}{\pgfqpoint{1.921499in}{3.264448in}}%
\pgfpathcurveto{\pgfqpoint{1.910449in}{3.264448in}}{\pgfqpoint{1.899850in}{3.260057in}}{\pgfqpoint{1.892036in}{3.252244in}}%
\pgfpathcurveto{\pgfqpoint{1.884223in}{3.244430in}}{\pgfqpoint{1.879832in}{3.233831in}}{\pgfqpoint{1.879832in}{3.222781in}}%
\pgfpathcurveto{\pgfqpoint{1.879832in}{3.211731in}}{\pgfqpoint{1.884223in}{3.201132in}}{\pgfqpoint{1.892036in}{3.193318in}}%
\pgfpathcurveto{\pgfqpoint{1.899850in}{3.185504in}}{\pgfqpoint{1.910449in}{3.181114in}}{\pgfqpoint{1.921499in}{3.181114in}}%
\pgfpathclose%
\pgfusepath{stroke,fill}%
\end{pgfscope}%
\begin{pgfscope}%
\pgfpathrectangle{\pgfqpoint{0.648703in}{0.548769in}}{\pgfqpoint{5.201297in}{3.102590in}}%
\pgfusepath{clip}%
\pgfsetbuttcap%
\pgfsetroundjoin%
\definecolor{currentfill}{rgb}{1.000000,0.498039,0.054902}%
\pgfsetfillcolor{currentfill}%
\pgfsetlinewidth{1.003750pt}%
\definecolor{currentstroke}{rgb}{1.000000,0.498039,0.054902}%
\pgfsetstrokecolor{currentstroke}%
\pgfsetdash{}{0pt}%
\pgfpathmoveto{\pgfqpoint{1.532859in}{3.185343in}}%
\pgfpathcurveto{\pgfqpoint{1.543909in}{3.185343in}}{\pgfqpoint{1.554508in}{3.189733in}}{\pgfqpoint{1.562322in}{3.197547in}}%
\pgfpathcurveto{\pgfqpoint{1.570135in}{3.205360in}}{\pgfqpoint{1.574526in}{3.215959in}}{\pgfqpoint{1.574526in}{3.227010in}}%
\pgfpathcurveto{\pgfqpoint{1.574526in}{3.238060in}}{\pgfqpoint{1.570135in}{3.248659in}}{\pgfqpoint{1.562322in}{3.256472in}}%
\pgfpathcurveto{\pgfqpoint{1.554508in}{3.264286in}}{\pgfqpoint{1.543909in}{3.268676in}}{\pgfqpoint{1.532859in}{3.268676in}}%
\pgfpathcurveto{\pgfqpoint{1.521809in}{3.268676in}}{\pgfqpoint{1.511210in}{3.264286in}}{\pgfqpoint{1.503396in}{3.256472in}}%
\pgfpathcurveto{\pgfqpoint{1.495583in}{3.248659in}}{\pgfqpoint{1.491192in}{3.238060in}}{\pgfqpoint{1.491192in}{3.227010in}}%
\pgfpathcurveto{\pgfqpoint{1.491192in}{3.215959in}}{\pgfqpoint{1.495583in}{3.205360in}}{\pgfqpoint{1.503396in}{3.197547in}}%
\pgfpathcurveto{\pgfqpoint{1.511210in}{3.189733in}}{\pgfqpoint{1.521809in}{3.185343in}}{\pgfqpoint{1.532859in}{3.185343in}}%
\pgfpathclose%
\pgfusepath{stroke,fill}%
\end{pgfscope}%
\begin{pgfscope}%
\pgfpathrectangle{\pgfqpoint{0.648703in}{0.548769in}}{\pgfqpoint{5.201297in}{3.102590in}}%
\pgfusepath{clip}%
\pgfsetbuttcap%
\pgfsetroundjoin%
\definecolor{currentfill}{rgb}{1.000000,0.498039,0.054902}%
\pgfsetfillcolor{currentfill}%
\pgfsetlinewidth{1.003750pt}%
\definecolor{currentstroke}{rgb}{1.000000,0.498039,0.054902}%
\pgfsetstrokecolor{currentstroke}%
\pgfsetdash{}{0pt}%
\pgfpathmoveto{\pgfqpoint{5.613577in}{3.358719in}}%
\pgfpathcurveto{\pgfqpoint{5.624628in}{3.358719in}}{\pgfqpoint{5.635227in}{3.363109in}}{\pgfqpoint{5.643040in}{3.370923in}}%
\pgfpathcurveto{\pgfqpoint{5.650854in}{3.378737in}}{\pgfqpoint{5.655244in}{3.389336in}}{\pgfqpoint{5.655244in}{3.400386in}}%
\pgfpathcurveto{\pgfqpoint{5.655244in}{3.411436in}}{\pgfqpoint{5.650854in}{3.422035in}}{\pgfqpoint{5.643040in}{3.429849in}}%
\pgfpathcurveto{\pgfqpoint{5.635227in}{3.437662in}}{\pgfqpoint{5.624628in}{3.442053in}}{\pgfqpoint{5.613577in}{3.442053in}}%
\pgfpathcurveto{\pgfqpoint{5.602527in}{3.442053in}}{\pgfqpoint{5.591928in}{3.437662in}}{\pgfqpoint{5.584115in}{3.429849in}}%
\pgfpathcurveto{\pgfqpoint{5.576301in}{3.422035in}}{\pgfqpoint{5.571911in}{3.411436in}}{\pgfqpoint{5.571911in}{3.400386in}}%
\pgfpathcurveto{\pgfqpoint{5.571911in}{3.389336in}}{\pgfqpoint{5.576301in}{3.378737in}}{\pgfqpoint{5.584115in}{3.370923in}}%
\pgfpathcurveto{\pgfqpoint{5.591928in}{3.363109in}}{\pgfqpoint{5.602527in}{3.358719in}}{\pgfqpoint{5.613577in}{3.358719in}}%
\pgfpathclose%
\pgfusepath{stroke,fill}%
\end{pgfscope}%
\begin{pgfscope}%
\pgfpathrectangle{\pgfqpoint{0.648703in}{0.548769in}}{\pgfqpoint{5.201297in}{3.102590in}}%
\pgfusepath{clip}%
\pgfsetbuttcap%
\pgfsetroundjoin%
\definecolor{currentfill}{rgb}{1.000000,0.498039,0.054902}%
\pgfsetfillcolor{currentfill}%
\pgfsetlinewidth{1.003750pt}%
\definecolor{currentstroke}{rgb}{1.000000,0.498039,0.054902}%
\pgfsetstrokecolor{currentstroke}%
\pgfsetdash{}{0pt}%
\pgfpathmoveto{\pgfqpoint{2.569232in}{3.198029in}}%
\pgfpathcurveto{\pgfqpoint{2.580282in}{3.198029in}}{\pgfqpoint{2.590881in}{3.202419in}}{\pgfqpoint{2.598695in}{3.210233in}}%
\pgfpathcurveto{\pgfqpoint{2.606508in}{3.218046in}}{\pgfqpoint{2.610899in}{3.228646in}}{\pgfqpoint{2.610899in}{3.239696in}}%
\pgfpathcurveto{\pgfqpoint{2.610899in}{3.250746in}}{\pgfqpoint{2.606508in}{3.261345in}}{\pgfqpoint{2.598695in}{3.269158in}}%
\pgfpathcurveto{\pgfqpoint{2.590881in}{3.276972in}}{\pgfqpoint{2.580282in}{3.281362in}}{\pgfqpoint{2.569232in}{3.281362in}}%
\pgfpathcurveto{\pgfqpoint{2.558182in}{3.281362in}}{\pgfqpoint{2.547583in}{3.276972in}}{\pgfqpoint{2.539769in}{3.269158in}}%
\pgfpathcurveto{\pgfqpoint{2.531956in}{3.261345in}}{\pgfqpoint{2.527565in}{3.250746in}}{\pgfqpoint{2.527565in}{3.239696in}}%
\pgfpathcurveto{\pgfqpoint{2.527565in}{3.228646in}}{\pgfqpoint{2.531956in}{3.218046in}}{\pgfqpoint{2.539769in}{3.210233in}}%
\pgfpathcurveto{\pgfqpoint{2.547583in}{3.202419in}}{\pgfqpoint{2.558182in}{3.198029in}}{\pgfqpoint{2.569232in}{3.198029in}}%
\pgfpathclose%
\pgfusepath{stroke,fill}%
\end{pgfscope}%
\begin{pgfscope}%
\pgfpathrectangle{\pgfqpoint{0.648703in}{0.548769in}}{\pgfqpoint{5.201297in}{3.102590in}}%
\pgfusepath{clip}%
\pgfsetbuttcap%
\pgfsetroundjoin%
\definecolor{currentfill}{rgb}{0.839216,0.152941,0.156863}%
\pgfsetfillcolor{currentfill}%
\pgfsetlinewidth{1.003750pt}%
\definecolor{currentstroke}{rgb}{0.839216,0.152941,0.156863}%
\pgfsetstrokecolor{currentstroke}%
\pgfsetdash{}{0pt}%
\pgfpathmoveto{\pgfqpoint{1.791952in}{3.193800in}}%
\pgfpathcurveto{\pgfqpoint{1.803002in}{3.193800in}}{\pgfqpoint{1.813601in}{3.198191in}}{\pgfqpoint{1.821415in}{3.206004in}}%
\pgfpathcurveto{\pgfqpoint{1.829229in}{3.213818in}}{\pgfqpoint{1.833619in}{3.224417in}}{\pgfqpoint{1.833619in}{3.235467in}}%
\pgfpathcurveto{\pgfqpoint{1.833619in}{3.246517in}}{\pgfqpoint{1.829229in}{3.257116in}}{\pgfqpoint{1.821415in}{3.264930in}}%
\pgfpathcurveto{\pgfqpoint{1.813601in}{3.272743in}}{\pgfqpoint{1.803002in}{3.277134in}}{\pgfqpoint{1.791952in}{3.277134in}}%
\pgfpathcurveto{\pgfqpoint{1.780902in}{3.277134in}}{\pgfqpoint{1.770303in}{3.272743in}}{\pgfqpoint{1.762490in}{3.264930in}}%
\pgfpathcurveto{\pgfqpoint{1.754676in}{3.257116in}}{\pgfqpoint{1.750286in}{3.246517in}}{\pgfqpoint{1.750286in}{3.235467in}}%
\pgfpathcurveto{\pgfqpoint{1.750286in}{3.224417in}}{\pgfqpoint{1.754676in}{3.213818in}}{\pgfqpoint{1.762490in}{3.206004in}}%
\pgfpathcurveto{\pgfqpoint{1.770303in}{3.198191in}}{\pgfqpoint{1.780902in}{3.193800in}}{\pgfqpoint{1.791952in}{3.193800in}}%
\pgfpathclose%
\pgfusepath{stroke,fill}%
\end{pgfscope}%
\begin{pgfscope}%
\pgfpathrectangle{\pgfqpoint{0.648703in}{0.548769in}}{\pgfqpoint{5.201297in}{3.102590in}}%
\pgfusepath{clip}%
\pgfsetbuttcap%
\pgfsetroundjoin%
\definecolor{currentfill}{rgb}{1.000000,0.498039,0.054902}%
\pgfsetfillcolor{currentfill}%
\pgfsetlinewidth{1.003750pt}%
\definecolor{currentstroke}{rgb}{1.000000,0.498039,0.054902}%
\pgfsetstrokecolor{currentstroke}%
\pgfsetdash{}{0pt}%
\pgfpathmoveto{\pgfqpoint{2.374912in}{3.193800in}}%
\pgfpathcurveto{\pgfqpoint{2.385962in}{3.193800in}}{\pgfqpoint{2.396561in}{3.198191in}}{\pgfqpoint{2.404375in}{3.206004in}}%
\pgfpathcurveto{\pgfqpoint{2.412188in}{3.213818in}}{\pgfqpoint{2.416579in}{3.224417in}}{\pgfqpoint{2.416579in}{3.235467in}}%
\pgfpathcurveto{\pgfqpoint{2.416579in}{3.246517in}}{\pgfqpoint{2.412188in}{3.257116in}}{\pgfqpoint{2.404375in}{3.264930in}}%
\pgfpathcurveto{\pgfqpoint{2.396561in}{3.272743in}}{\pgfqpoint{2.385962in}{3.277134in}}{\pgfqpoint{2.374912in}{3.277134in}}%
\pgfpathcurveto{\pgfqpoint{2.363862in}{3.277134in}}{\pgfqpoint{2.353263in}{3.272743in}}{\pgfqpoint{2.345449in}{3.264930in}}%
\pgfpathcurveto{\pgfqpoint{2.337636in}{3.257116in}}{\pgfqpoint{2.333245in}{3.246517in}}{\pgfqpoint{2.333245in}{3.235467in}}%
\pgfpathcurveto{\pgfqpoint{2.333245in}{3.224417in}}{\pgfqpoint{2.337636in}{3.213818in}}{\pgfqpoint{2.345449in}{3.206004in}}%
\pgfpathcurveto{\pgfqpoint{2.353263in}{3.198191in}}{\pgfqpoint{2.363862in}{3.193800in}}{\pgfqpoint{2.374912in}{3.193800in}}%
\pgfpathclose%
\pgfusepath{stroke,fill}%
\end{pgfscope}%
\begin{pgfscope}%
\pgfpathrectangle{\pgfqpoint{0.648703in}{0.548769in}}{\pgfqpoint{5.201297in}{3.102590in}}%
\pgfusepath{clip}%
\pgfsetbuttcap%
\pgfsetroundjoin%
\definecolor{currentfill}{rgb}{1.000000,0.498039,0.054902}%
\pgfsetfillcolor{currentfill}%
\pgfsetlinewidth{1.003750pt}%
\definecolor{currentstroke}{rgb}{1.000000,0.498039,0.054902}%
\pgfsetstrokecolor{currentstroke}%
\pgfsetdash{}{0pt}%
\pgfpathmoveto{\pgfqpoint{2.180592in}{3.185343in}}%
\pgfpathcurveto{\pgfqpoint{2.191642in}{3.185343in}}{\pgfqpoint{2.202241in}{3.189733in}}{\pgfqpoint{2.210055in}{3.197547in}}%
\pgfpathcurveto{\pgfqpoint{2.217869in}{3.205360in}}{\pgfqpoint{2.222259in}{3.215959in}}{\pgfqpoint{2.222259in}{3.227010in}}%
\pgfpathcurveto{\pgfqpoint{2.222259in}{3.238060in}}{\pgfqpoint{2.217869in}{3.248659in}}{\pgfqpoint{2.210055in}{3.256472in}}%
\pgfpathcurveto{\pgfqpoint{2.202241in}{3.264286in}}{\pgfqpoint{2.191642in}{3.268676in}}{\pgfqpoint{2.180592in}{3.268676in}}%
\pgfpathcurveto{\pgfqpoint{2.169542in}{3.268676in}}{\pgfqpoint{2.158943in}{3.264286in}}{\pgfqpoint{2.151129in}{3.256472in}}%
\pgfpathcurveto{\pgfqpoint{2.143316in}{3.248659in}}{\pgfqpoint{2.138925in}{3.238060in}}{\pgfqpoint{2.138925in}{3.227010in}}%
\pgfpathcurveto{\pgfqpoint{2.138925in}{3.215959in}}{\pgfqpoint{2.143316in}{3.205360in}}{\pgfqpoint{2.151129in}{3.197547in}}%
\pgfpathcurveto{\pgfqpoint{2.158943in}{3.189733in}}{\pgfqpoint{2.169542in}{3.185343in}}{\pgfqpoint{2.180592in}{3.185343in}}%
\pgfpathclose%
\pgfusepath{stroke,fill}%
\end{pgfscope}%
\begin{pgfscope}%
\pgfpathrectangle{\pgfqpoint{0.648703in}{0.548769in}}{\pgfqpoint{5.201297in}{3.102590in}}%
\pgfusepath{clip}%
\pgfsetbuttcap%
\pgfsetroundjoin%
\definecolor{currentfill}{rgb}{1.000000,0.498039,0.054902}%
\pgfsetfillcolor{currentfill}%
\pgfsetlinewidth{1.003750pt}%
\definecolor{currentstroke}{rgb}{1.000000,0.498039,0.054902}%
\pgfsetstrokecolor{currentstroke}%
\pgfsetdash{}{0pt}%
\pgfpathmoveto{\pgfqpoint{2.245365in}{3.185343in}}%
\pgfpathcurveto{\pgfqpoint{2.256416in}{3.185343in}}{\pgfqpoint{2.267015in}{3.189733in}}{\pgfqpoint{2.274828in}{3.197547in}}%
\pgfpathcurveto{\pgfqpoint{2.282642in}{3.205360in}}{\pgfqpoint{2.287032in}{3.215959in}}{\pgfqpoint{2.287032in}{3.227010in}}%
\pgfpathcurveto{\pgfqpoint{2.287032in}{3.238060in}}{\pgfqpoint{2.282642in}{3.248659in}}{\pgfqpoint{2.274828in}{3.256472in}}%
\pgfpathcurveto{\pgfqpoint{2.267015in}{3.264286in}}{\pgfqpoint{2.256416in}{3.268676in}}{\pgfqpoint{2.245365in}{3.268676in}}%
\pgfpathcurveto{\pgfqpoint{2.234315in}{3.268676in}}{\pgfqpoint{2.223716in}{3.264286in}}{\pgfqpoint{2.215903in}{3.256472in}}%
\pgfpathcurveto{\pgfqpoint{2.208089in}{3.248659in}}{\pgfqpoint{2.203699in}{3.238060in}}{\pgfqpoint{2.203699in}{3.227010in}}%
\pgfpathcurveto{\pgfqpoint{2.203699in}{3.215959in}}{\pgfqpoint{2.208089in}{3.205360in}}{\pgfqpoint{2.215903in}{3.197547in}}%
\pgfpathcurveto{\pgfqpoint{2.223716in}{3.189733in}}{\pgfqpoint{2.234315in}{3.185343in}}{\pgfqpoint{2.245365in}{3.185343in}}%
\pgfpathclose%
\pgfusepath{stroke,fill}%
\end{pgfscope}%
\begin{pgfscope}%
\pgfpathrectangle{\pgfqpoint{0.648703in}{0.548769in}}{\pgfqpoint{5.201297in}{3.102590in}}%
\pgfusepath{clip}%
\pgfsetbuttcap%
\pgfsetroundjoin%
\definecolor{currentfill}{rgb}{1.000000,0.498039,0.054902}%
\pgfsetfillcolor{currentfill}%
\pgfsetlinewidth{1.003750pt}%
\definecolor{currentstroke}{rgb}{1.000000,0.498039,0.054902}%
\pgfsetstrokecolor{currentstroke}%
\pgfsetdash{}{0pt}%
\pgfpathmoveto{\pgfqpoint{1.921499in}{3.193800in}}%
\pgfpathcurveto{\pgfqpoint{1.932549in}{3.193800in}}{\pgfqpoint{1.943148in}{3.198191in}}{\pgfqpoint{1.950962in}{3.206004in}}%
\pgfpathcurveto{\pgfqpoint{1.958775in}{3.213818in}}{\pgfqpoint{1.963166in}{3.224417in}}{\pgfqpoint{1.963166in}{3.235467in}}%
\pgfpathcurveto{\pgfqpoint{1.963166in}{3.246517in}}{\pgfqpoint{1.958775in}{3.257116in}}{\pgfqpoint{1.950962in}{3.264930in}}%
\pgfpathcurveto{\pgfqpoint{1.943148in}{3.272743in}}{\pgfqpoint{1.932549in}{3.277134in}}{\pgfqpoint{1.921499in}{3.277134in}}%
\pgfpathcurveto{\pgfqpoint{1.910449in}{3.277134in}}{\pgfqpoint{1.899850in}{3.272743in}}{\pgfqpoint{1.892036in}{3.264930in}}%
\pgfpathcurveto{\pgfqpoint{1.884223in}{3.257116in}}{\pgfqpoint{1.879832in}{3.246517in}}{\pgfqpoint{1.879832in}{3.235467in}}%
\pgfpathcurveto{\pgfqpoint{1.879832in}{3.224417in}}{\pgfqpoint{1.884223in}{3.213818in}}{\pgfqpoint{1.892036in}{3.206004in}}%
\pgfpathcurveto{\pgfqpoint{1.899850in}{3.198191in}}{\pgfqpoint{1.910449in}{3.193800in}}{\pgfqpoint{1.921499in}{3.193800in}}%
\pgfpathclose%
\pgfusepath{stroke,fill}%
\end{pgfscope}%
\begin{pgfscope}%
\pgfpathrectangle{\pgfqpoint{0.648703in}{0.548769in}}{\pgfqpoint{5.201297in}{3.102590in}}%
\pgfusepath{clip}%
\pgfsetbuttcap%
\pgfsetroundjoin%
\definecolor{currentfill}{rgb}{0.839216,0.152941,0.156863}%
\pgfsetfillcolor{currentfill}%
\pgfsetlinewidth{1.003750pt}%
\definecolor{currentstroke}{rgb}{0.839216,0.152941,0.156863}%
\pgfsetstrokecolor{currentstroke}%
\pgfsetdash{}{0pt}%
\pgfpathmoveto{\pgfqpoint{2.245365in}{3.189572in}}%
\pgfpathcurveto{\pgfqpoint{2.256416in}{3.189572in}}{\pgfqpoint{2.267015in}{3.193962in}}{\pgfqpoint{2.274828in}{3.201775in}}%
\pgfpathcurveto{\pgfqpoint{2.282642in}{3.209589in}}{\pgfqpoint{2.287032in}{3.220188in}}{\pgfqpoint{2.287032in}{3.231238in}}%
\pgfpathcurveto{\pgfqpoint{2.287032in}{3.242288in}}{\pgfqpoint{2.282642in}{3.252887in}}{\pgfqpoint{2.274828in}{3.260701in}}%
\pgfpathcurveto{\pgfqpoint{2.267015in}{3.268515in}}{\pgfqpoint{2.256416in}{3.272905in}}{\pgfqpoint{2.245365in}{3.272905in}}%
\pgfpathcurveto{\pgfqpoint{2.234315in}{3.272905in}}{\pgfqpoint{2.223716in}{3.268515in}}{\pgfqpoint{2.215903in}{3.260701in}}%
\pgfpathcurveto{\pgfqpoint{2.208089in}{3.252887in}}{\pgfqpoint{2.203699in}{3.242288in}}{\pgfqpoint{2.203699in}{3.231238in}}%
\pgfpathcurveto{\pgfqpoint{2.203699in}{3.220188in}}{\pgfqpoint{2.208089in}{3.209589in}}{\pgfqpoint{2.215903in}{3.201775in}}%
\pgfpathcurveto{\pgfqpoint{2.223716in}{3.193962in}}{\pgfqpoint{2.234315in}{3.189572in}}{\pgfqpoint{2.245365in}{3.189572in}}%
\pgfpathclose%
\pgfusepath{stroke,fill}%
\end{pgfscope}%
\begin{pgfscope}%
\pgfpathrectangle{\pgfqpoint{0.648703in}{0.548769in}}{\pgfqpoint{5.201297in}{3.102590in}}%
\pgfusepath{clip}%
\pgfsetbuttcap%
\pgfsetroundjoin%
\definecolor{currentfill}{rgb}{1.000000,0.498039,0.054902}%
\pgfsetfillcolor{currentfill}%
\pgfsetlinewidth{1.003750pt}%
\definecolor{currentstroke}{rgb}{1.000000,0.498039,0.054902}%
\pgfsetstrokecolor{currentstroke}%
\pgfsetdash{}{0pt}%
\pgfpathmoveto{\pgfqpoint{4.836298in}{3.198029in}}%
\pgfpathcurveto{\pgfqpoint{4.847348in}{3.198029in}}{\pgfqpoint{4.857947in}{3.202419in}}{\pgfqpoint{4.865761in}{3.210233in}}%
\pgfpathcurveto{\pgfqpoint{4.873574in}{3.218046in}}{\pgfqpoint{4.877964in}{3.228646in}}{\pgfqpoint{4.877964in}{3.239696in}}%
\pgfpathcurveto{\pgfqpoint{4.877964in}{3.250746in}}{\pgfqpoint{4.873574in}{3.261345in}}{\pgfqpoint{4.865761in}{3.269158in}}%
\pgfpathcurveto{\pgfqpoint{4.857947in}{3.276972in}}{\pgfqpoint{4.847348in}{3.281362in}}{\pgfqpoint{4.836298in}{3.281362in}}%
\pgfpathcurveto{\pgfqpoint{4.825248in}{3.281362in}}{\pgfqpoint{4.814649in}{3.276972in}}{\pgfqpoint{4.806835in}{3.269158in}}%
\pgfpathcurveto{\pgfqpoint{4.799021in}{3.261345in}}{\pgfqpoint{4.794631in}{3.250746in}}{\pgfqpoint{4.794631in}{3.239696in}}%
\pgfpathcurveto{\pgfqpoint{4.794631in}{3.228646in}}{\pgfqpoint{4.799021in}{3.218046in}}{\pgfqpoint{4.806835in}{3.210233in}}%
\pgfpathcurveto{\pgfqpoint{4.814649in}{3.202419in}}{\pgfqpoint{4.825248in}{3.198029in}}{\pgfqpoint{4.836298in}{3.198029in}}%
\pgfpathclose%
\pgfusepath{stroke,fill}%
\end{pgfscope}%
\begin{pgfscope}%
\pgfpathrectangle{\pgfqpoint{0.648703in}{0.548769in}}{\pgfqpoint{5.201297in}{3.102590in}}%
\pgfusepath{clip}%
\pgfsetbuttcap%
\pgfsetroundjoin%
\definecolor{currentfill}{rgb}{0.121569,0.466667,0.705882}%
\pgfsetfillcolor{currentfill}%
\pgfsetlinewidth{1.003750pt}%
\definecolor{currentstroke}{rgb}{0.121569,0.466667,0.705882}%
\pgfsetstrokecolor{currentstroke}%
\pgfsetdash{}{0pt}%
\pgfpathmoveto{\pgfqpoint{2.634005in}{3.181114in}}%
\pgfpathcurveto{\pgfqpoint{2.645055in}{3.181114in}}{\pgfqpoint{2.655654in}{3.185504in}}{\pgfqpoint{2.663468in}{3.193318in}}%
\pgfpathcurveto{\pgfqpoint{2.671282in}{3.201132in}}{\pgfqpoint{2.675672in}{3.211731in}}{\pgfqpoint{2.675672in}{3.222781in}}%
\pgfpathcurveto{\pgfqpoint{2.675672in}{3.233831in}}{\pgfqpoint{2.671282in}{3.244430in}}{\pgfqpoint{2.663468in}{3.252244in}}%
\pgfpathcurveto{\pgfqpoint{2.655654in}{3.260057in}}{\pgfqpoint{2.645055in}{3.264448in}}{\pgfqpoint{2.634005in}{3.264448in}}%
\pgfpathcurveto{\pgfqpoint{2.622955in}{3.264448in}}{\pgfqpoint{2.612356in}{3.260057in}}{\pgfqpoint{2.604543in}{3.252244in}}%
\pgfpathcurveto{\pgfqpoint{2.596729in}{3.244430in}}{\pgfqpoint{2.592339in}{3.233831in}}{\pgfqpoint{2.592339in}{3.222781in}}%
\pgfpathcurveto{\pgfqpoint{2.592339in}{3.211731in}}{\pgfqpoint{2.596729in}{3.201132in}}{\pgfqpoint{2.604543in}{3.193318in}}%
\pgfpathcurveto{\pgfqpoint{2.612356in}{3.185504in}}{\pgfqpoint{2.622955in}{3.181114in}}{\pgfqpoint{2.634005in}{3.181114in}}%
\pgfpathclose%
\pgfusepath{stroke,fill}%
\end{pgfscope}%
\begin{pgfscope}%
\pgfpathrectangle{\pgfqpoint{0.648703in}{0.548769in}}{\pgfqpoint{5.201297in}{3.102590in}}%
\pgfusepath{clip}%
\pgfsetbuttcap%
\pgfsetroundjoin%
\definecolor{currentfill}{rgb}{0.121569,0.466667,0.705882}%
\pgfsetfillcolor{currentfill}%
\pgfsetlinewidth{1.003750pt}%
\definecolor{currentstroke}{rgb}{0.121569,0.466667,0.705882}%
\pgfsetstrokecolor{currentstroke}%
\pgfsetdash{}{0pt}%
\pgfpathmoveto{\pgfqpoint{1.144219in}{0.939909in}}%
\pgfpathcurveto{\pgfqpoint{1.155269in}{0.939909in}}{\pgfqpoint{1.165868in}{0.944299in}}{\pgfqpoint{1.173682in}{0.952112in}}%
\pgfpathcurveto{\pgfqpoint{1.181496in}{0.959926in}}{\pgfqpoint{1.185886in}{0.970525in}}{\pgfqpoint{1.185886in}{0.981575in}}%
\pgfpathcurveto{\pgfqpoint{1.185886in}{0.992625in}}{\pgfqpoint{1.181496in}{1.003224in}}{\pgfqpoint{1.173682in}{1.011038in}}%
\pgfpathcurveto{\pgfqpoint{1.165868in}{1.018852in}}{\pgfqpoint{1.155269in}{1.023242in}}{\pgfqpoint{1.144219in}{1.023242in}}%
\pgfpathcurveto{\pgfqpoint{1.133169in}{1.023242in}}{\pgfqpoint{1.122570in}{1.018852in}}{\pgfqpoint{1.114756in}{1.011038in}}%
\pgfpathcurveto{\pgfqpoint{1.106943in}{1.003224in}}{\pgfqpoint{1.102553in}{0.992625in}}{\pgfqpoint{1.102553in}{0.981575in}}%
\pgfpathcurveto{\pgfqpoint{1.102553in}{0.970525in}}{\pgfqpoint{1.106943in}{0.959926in}}{\pgfqpoint{1.114756in}{0.952112in}}%
\pgfpathcurveto{\pgfqpoint{1.122570in}{0.944299in}}{\pgfqpoint{1.133169in}{0.939909in}}{\pgfqpoint{1.144219in}{0.939909in}}%
\pgfpathclose%
\pgfusepath{stroke,fill}%
\end{pgfscope}%
\begin{pgfscope}%
\pgfpathrectangle{\pgfqpoint{0.648703in}{0.548769in}}{\pgfqpoint{5.201297in}{3.102590in}}%
\pgfusepath{clip}%
\pgfsetbuttcap%
\pgfsetroundjoin%
\definecolor{currentfill}{rgb}{0.839216,0.152941,0.156863}%
\pgfsetfillcolor{currentfill}%
\pgfsetlinewidth{1.003750pt}%
\definecolor{currentstroke}{rgb}{0.839216,0.152941,0.156863}%
\pgfsetstrokecolor{currentstroke}%
\pgfsetdash{}{0pt}%
\pgfpathmoveto{\pgfqpoint{2.698779in}{3.210715in}}%
\pgfpathcurveto{\pgfqpoint{2.709829in}{3.210715in}}{\pgfqpoint{2.720428in}{3.215105in}}{\pgfqpoint{2.728241in}{3.222919in}}%
\pgfpathcurveto{\pgfqpoint{2.736055in}{3.230733in}}{\pgfqpoint{2.740445in}{3.241332in}}{\pgfqpoint{2.740445in}{3.252382in}}%
\pgfpathcurveto{\pgfqpoint{2.740445in}{3.263432in}}{\pgfqpoint{2.736055in}{3.274031in}}{\pgfqpoint{2.728241in}{3.281844in}}%
\pgfpathcurveto{\pgfqpoint{2.720428in}{3.289658in}}{\pgfqpoint{2.709829in}{3.294048in}}{\pgfqpoint{2.698779in}{3.294048in}}%
\pgfpathcurveto{\pgfqpoint{2.687728in}{3.294048in}}{\pgfqpoint{2.677129in}{3.289658in}}{\pgfqpoint{2.669316in}{3.281844in}}%
\pgfpathcurveto{\pgfqpoint{2.661502in}{3.274031in}}{\pgfqpoint{2.657112in}{3.263432in}}{\pgfqpoint{2.657112in}{3.252382in}}%
\pgfpathcurveto{\pgfqpoint{2.657112in}{3.241332in}}{\pgfqpoint{2.661502in}{3.230733in}}{\pgfqpoint{2.669316in}{3.222919in}}%
\pgfpathcurveto{\pgfqpoint{2.677129in}{3.215105in}}{\pgfqpoint{2.687728in}{3.210715in}}{\pgfqpoint{2.698779in}{3.210715in}}%
\pgfpathclose%
\pgfusepath{stroke,fill}%
\end{pgfscope}%
\begin{pgfscope}%
\pgfpathrectangle{\pgfqpoint{0.648703in}{0.548769in}}{\pgfqpoint{5.201297in}{3.102590in}}%
\pgfusepath{clip}%
\pgfsetbuttcap%
\pgfsetroundjoin%
\definecolor{currentfill}{rgb}{1.000000,0.498039,0.054902}%
\pgfsetfillcolor{currentfill}%
\pgfsetlinewidth{1.003750pt}%
\definecolor{currentstroke}{rgb}{1.000000,0.498039,0.054902}%
\pgfsetstrokecolor{currentstroke}%
\pgfsetdash{}{0pt}%
\pgfpathmoveto{\pgfqpoint{1.727179in}{3.202258in}}%
\pgfpathcurveto{\pgfqpoint{1.738229in}{3.202258in}}{\pgfqpoint{1.748828in}{3.206648in}}{\pgfqpoint{1.756642in}{3.214462in}}%
\pgfpathcurveto{\pgfqpoint{1.764455in}{3.222275in}}{\pgfqpoint{1.768846in}{3.232874in}}{\pgfqpoint{1.768846in}{3.243924in}}%
\pgfpathcurveto{\pgfqpoint{1.768846in}{3.254974in}}{\pgfqpoint{1.764455in}{3.265573in}}{\pgfqpoint{1.756642in}{3.273387in}}%
\pgfpathcurveto{\pgfqpoint{1.748828in}{3.281201in}}{\pgfqpoint{1.738229in}{3.285591in}}{\pgfqpoint{1.727179in}{3.285591in}}%
\pgfpathcurveto{\pgfqpoint{1.716129in}{3.285591in}}{\pgfqpoint{1.705530in}{3.281201in}}{\pgfqpoint{1.697716in}{3.273387in}}%
\pgfpathcurveto{\pgfqpoint{1.689903in}{3.265573in}}{\pgfqpoint{1.685512in}{3.254974in}}{\pgfqpoint{1.685512in}{3.243924in}}%
\pgfpathcurveto{\pgfqpoint{1.685512in}{3.232874in}}{\pgfqpoint{1.689903in}{3.222275in}}{\pgfqpoint{1.697716in}{3.214462in}}%
\pgfpathcurveto{\pgfqpoint{1.705530in}{3.206648in}}{\pgfqpoint{1.716129in}{3.202258in}}{\pgfqpoint{1.727179in}{3.202258in}}%
\pgfpathclose%
\pgfusepath{stroke,fill}%
\end{pgfscope}%
\begin{pgfscope}%
\pgfpathrectangle{\pgfqpoint{0.648703in}{0.548769in}}{\pgfqpoint{5.201297in}{3.102590in}}%
\pgfusepath{clip}%
\pgfsetbuttcap%
\pgfsetroundjoin%
\definecolor{currentfill}{rgb}{1.000000,0.498039,0.054902}%
\pgfsetfillcolor{currentfill}%
\pgfsetlinewidth{1.003750pt}%
\definecolor{currentstroke}{rgb}{1.000000,0.498039,0.054902}%
\pgfsetstrokecolor{currentstroke}%
\pgfsetdash{}{0pt}%
\pgfpathmoveto{\pgfqpoint{1.856726in}{3.206486in}}%
\pgfpathcurveto{\pgfqpoint{1.867776in}{3.206486in}}{\pgfqpoint{1.878375in}{3.210877in}}{\pgfqpoint{1.886188in}{3.218690in}}%
\pgfpathcurveto{\pgfqpoint{1.894002in}{3.226504in}}{\pgfqpoint{1.898392in}{3.237103in}}{\pgfqpoint{1.898392in}{3.248153in}}%
\pgfpathcurveto{\pgfqpoint{1.898392in}{3.259203in}}{\pgfqpoint{1.894002in}{3.269802in}}{\pgfqpoint{1.886188in}{3.277616in}}%
\pgfpathcurveto{\pgfqpoint{1.878375in}{3.285429in}}{\pgfqpoint{1.867776in}{3.289820in}}{\pgfqpoint{1.856726in}{3.289820in}}%
\pgfpathcurveto{\pgfqpoint{1.845675in}{3.289820in}}{\pgfqpoint{1.835076in}{3.285429in}}{\pgfqpoint{1.827263in}{3.277616in}}%
\pgfpathcurveto{\pgfqpoint{1.819449in}{3.269802in}}{\pgfqpoint{1.815059in}{3.259203in}}{\pgfqpoint{1.815059in}{3.248153in}}%
\pgfpathcurveto{\pgfqpoint{1.815059in}{3.237103in}}{\pgfqpoint{1.819449in}{3.226504in}}{\pgfqpoint{1.827263in}{3.218690in}}%
\pgfpathcurveto{\pgfqpoint{1.835076in}{3.210877in}}{\pgfqpoint{1.845675in}{3.206486in}}{\pgfqpoint{1.856726in}{3.206486in}}%
\pgfpathclose%
\pgfusepath{stroke,fill}%
\end{pgfscope}%
\begin{pgfscope}%
\pgfpathrectangle{\pgfqpoint{0.648703in}{0.548769in}}{\pgfqpoint{5.201297in}{3.102590in}}%
\pgfusepath{clip}%
\pgfsetbuttcap%
\pgfsetroundjoin%
\definecolor{currentfill}{rgb}{1.000000,0.498039,0.054902}%
\pgfsetfillcolor{currentfill}%
\pgfsetlinewidth{1.003750pt}%
\definecolor{currentstroke}{rgb}{1.000000,0.498039,0.054902}%
\pgfsetstrokecolor{currentstroke}%
\pgfsetdash{}{0pt}%
\pgfpathmoveto{\pgfqpoint{1.338539in}{3.198029in}}%
\pgfpathcurveto{\pgfqpoint{1.349589in}{3.198029in}}{\pgfqpoint{1.360188in}{3.202419in}}{\pgfqpoint{1.368002in}{3.210233in}}%
\pgfpathcurveto{\pgfqpoint{1.375816in}{3.218046in}}{\pgfqpoint{1.380206in}{3.228646in}}{\pgfqpoint{1.380206in}{3.239696in}}%
\pgfpathcurveto{\pgfqpoint{1.380206in}{3.250746in}}{\pgfqpoint{1.375816in}{3.261345in}}{\pgfqpoint{1.368002in}{3.269158in}}%
\pgfpathcurveto{\pgfqpoint{1.360188in}{3.276972in}}{\pgfqpoint{1.349589in}{3.281362in}}{\pgfqpoint{1.338539in}{3.281362in}}%
\pgfpathcurveto{\pgfqpoint{1.327489in}{3.281362in}}{\pgfqpoint{1.316890in}{3.276972in}}{\pgfqpoint{1.309076in}{3.269158in}}%
\pgfpathcurveto{\pgfqpoint{1.301263in}{3.261345in}}{\pgfqpoint{1.296872in}{3.250746in}}{\pgfqpoint{1.296872in}{3.239696in}}%
\pgfpathcurveto{\pgfqpoint{1.296872in}{3.228646in}}{\pgfqpoint{1.301263in}{3.218046in}}{\pgfqpoint{1.309076in}{3.210233in}}%
\pgfpathcurveto{\pgfqpoint{1.316890in}{3.202419in}}{\pgfqpoint{1.327489in}{3.198029in}}{\pgfqpoint{1.338539in}{3.198029in}}%
\pgfpathclose%
\pgfusepath{stroke,fill}%
\end{pgfscope}%
\begin{pgfscope}%
\pgfpathrectangle{\pgfqpoint{0.648703in}{0.548769in}}{\pgfqpoint{5.201297in}{3.102590in}}%
\pgfusepath{clip}%
\pgfsetbuttcap%
\pgfsetroundjoin%
\definecolor{currentfill}{rgb}{1.000000,0.498039,0.054902}%
\pgfsetfillcolor{currentfill}%
\pgfsetlinewidth{1.003750pt}%
\definecolor{currentstroke}{rgb}{1.000000,0.498039,0.054902}%
\pgfsetstrokecolor{currentstroke}%
\pgfsetdash{}{0pt}%
\pgfpathmoveto{\pgfqpoint{2.634005in}{3.202258in}}%
\pgfpathcurveto{\pgfqpoint{2.645055in}{3.202258in}}{\pgfqpoint{2.655654in}{3.206648in}}{\pgfqpoint{2.663468in}{3.214462in}}%
\pgfpathcurveto{\pgfqpoint{2.671282in}{3.222275in}}{\pgfqpoint{2.675672in}{3.232874in}}{\pgfqpoint{2.675672in}{3.243924in}}%
\pgfpathcurveto{\pgfqpoint{2.675672in}{3.254974in}}{\pgfqpoint{2.671282in}{3.265573in}}{\pgfqpoint{2.663468in}{3.273387in}}%
\pgfpathcurveto{\pgfqpoint{2.655654in}{3.281201in}}{\pgfqpoint{2.645055in}{3.285591in}}{\pgfqpoint{2.634005in}{3.285591in}}%
\pgfpathcurveto{\pgfqpoint{2.622955in}{3.285591in}}{\pgfqpoint{2.612356in}{3.281201in}}{\pgfqpoint{2.604543in}{3.273387in}}%
\pgfpathcurveto{\pgfqpoint{2.596729in}{3.265573in}}{\pgfqpoint{2.592339in}{3.254974in}}{\pgfqpoint{2.592339in}{3.243924in}}%
\pgfpathcurveto{\pgfqpoint{2.592339in}{3.232874in}}{\pgfqpoint{2.596729in}{3.222275in}}{\pgfqpoint{2.604543in}{3.214462in}}%
\pgfpathcurveto{\pgfqpoint{2.612356in}{3.206648in}}{\pgfqpoint{2.622955in}{3.202258in}}{\pgfqpoint{2.634005in}{3.202258in}}%
\pgfpathclose%
\pgfusepath{stroke,fill}%
\end{pgfscope}%
\begin{pgfscope}%
\pgfpathrectangle{\pgfqpoint{0.648703in}{0.548769in}}{\pgfqpoint{5.201297in}{3.102590in}}%
\pgfusepath{clip}%
\pgfsetbuttcap%
\pgfsetroundjoin%
\definecolor{currentfill}{rgb}{1.000000,0.498039,0.054902}%
\pgfsetfillcolor{currentfill}%
\pgfsetlinewidth{1.003750pt}%
\definecolor{currentstroke}{rgb}{1.000000,0.498039,0.054902}%
\pgfsetstrokecolor{currentstroke}%
\pgfsetdash{}{0pt}%
\pgfpathmoveto{\pgfqpoint{2.439685in}{3.189572in}}%
\pgfpathcurveto{\pgfqpoint{2.450735in}{3.189572in}}{\pgfqpoint{2.461335in}{3.193962in}}{\pgfqpoint{2.469148in}{3.201775in}}%
\pgfpathcurveto{\pgfqpoint{2.476962in}{3.209589in}}{\pgfqpoint{2.481352in}{3.220188in}}{\pgfqpoint{2.481352in}{3.231238in}}%
\pgfpathcurveto{\pgfqpoint{2.481352in}{3.242288in}}{\pgfqpoint{2.476962in}{3.252887in}}{\pgfqpoint{2.469148in}{3.260701in}}%
\pgfpathcurveto{\pgfqpoint{2.461335in}{3.268515in}}{\pgfqpoint{2.450735in}{3.272905in}}{\pgfqpoint{2.439685in}{3.272905in}}%
\pgfpathcurveto{\pgfqpoint{2.428635in}{3.272905in}}{\pgfqpoint{2.418036in}{3.268515in}}{\pgfqpoint{2.410223in}{3.260701in}}%
\pgfpathcurveto{\pgfqpoint{2.402409in}{3.252887in}}{\pgfqpoint{2.398019in}{3.242288in}}{\pgfqpoint{2.398019in}{3.231238in}}%
\pgfpathcurveto{\pgfqpoint{2.398019in}{3.220188in}}{\pgfqpoint{2.402409in}{3.209589in}}{\pgfqpoint{2.410223in}{3.201775in}}%
\pgfpathcurveto{\pgfqpoint{2.418036in}{3.193962in}}{\pgfqpoint{2.428635in}{3.189572in}}{\pgfqpoint{2.439685in}{3.189572in}}%
\pgfpathclose%
\pgfusepath{stroke,fill}%
\end{pgfscope}%
\begin{pgfscope}%
\pgfpathrectangle{\pgfqpoint{0.648703in}{0.548769in}}{\pgfqpoint{5.201297in}{3.102590in}}%
\pgfusepath{clip}%
\pgfsetbuttcap%
\pgfsetroundjoin%
\definecolor{currentfill}{rgb}{0.121569,0.466667,0.705882}%
\pgfsetfillcolor{currentfill}%
\pgfsetlinewidth{1.003750pt}%
\definecolor{currentstroke}{rgb}{0.121569,0.466667,0.705882}%
\pgfsetstrokecolor{currentstroke}%
\pgfsetdash{}{0pt}%
\pgfpathmoveto{\pgfqpoint{3.087418in}{3.181114in}}%
\pgfpathcurveto{\pgfqpoint{3.098469in}{3.181114in}}{\pgfqpoint{3.109068in}{3.185504in}}{\pgfqpoint{3.116881in}{3.193318in}}%
\pgfpathcurveto{\pgfqpoint{3.124695in}{3.201132in}}{\pgfqpoint{3.129085in}{3.211731in}}{\pgfqpoint{3.129085in}{3.222781in}}%
\pgfpathcurveto{\pgfqpoint{3.129085in}{3.233831in}}{\pgfqpoint{3.124695in}{3.244430in}}{\pgfqpoint{3.116881in}{3.252244in}}%
\pgfpathcurveto{\pgfqpoint{3.109068in}{3.260057in}}{\pgfqpoint{3.098469in}{3.264448in}}{\pgfqpoint{3.087418in}{3.264448in}}%
\pgfpathcurveto{\pgfqpoint{3.076368in}{3.264448in}}{\pgfqpoint{3.065769in}{3.260057in}}{\pgfqpoint{3.057956in}{3.252244in}}%
\pgfpathcurveto{\pgfqpoint{3.050142in}{3.244430in}}{\pgfqpoint{3.045752in}{3.233831in}}{\pgfqpoint{3.045752in}{3.222781in}}%
\pgfpathcurveto{\pgfqpoint{3.045752in}{3.211731in}}{\pgfqpoint{3.050142in}{3.201132in}}{\pgfqpoint{3.057956in}{3.193318in}}%
\pgfpathcurveto{\pgfqpoint{3.065769in}{3.185504in}}{\pgfqpoint{3.076368in}{3.181114in}}{\pgfqpoint{3.087418in}{3.181114in}}%
\pgfpathclose%
\pgfusepath{stroke,fill}%
\end{pgfscope}%
\begin{pgfscope}%
\pgfpathrectangle{\pgfqpoint{0.648703in}{0.548769in}}{\pgfqpoint{5.201297in}{3.102590in}}%
\pgfusepath{clip}%
\pgfsetbuttcap%
\pgfsetroundjoin%
\definecolor{currentfill}{rgb}{1.000000,0.498039,0.054902}%
\pgfsetfillcolor{currentfill}%
\pgfsetlinewidth{1.003750pt}%
\definecolor{currentstroke}{rgb}{1.000000,0.498039,0.054902}%
\pgfsetstrokecolor{currentstroke}%
\pgfsetdash{}{0pt}%
\pgfpathmoveto{\pgfqpoint{1.727179in}{3.244545in}}%
\pgfpathcurveto{\pgfqpoint{1.738229in}{3.244545in}}{\pgfqpoint{1.748828in}{3.248935in}}{\pgfqpoint{1.756642in}{3.256748in}}%
\pgfpathcurveto{\pgfqpoint{1.764455in}{3.264562in}}{\pgfqpoint{1.768846in}{3.275161in}}{\pgfqpoint{1.768846in}{3.286211in}}%
\pgfpathcurveto{\pgfqpoint{1.768846in}{3.297261in}}{\pgfqpoint{1.764455in}{3.307860in}}{\pgfqpoint{1.756642in}{3.315674in}}%
\pgfpathcurveto{\pgfqpoint{1.748828in}{3.323488in}}{\pgfqpoint{1.738229in}{3.327878in}}{\pgfqpoint{1.727179in}{3.327878in}}%
\pgfpathcurveto{\pgfqpoint{1.716129in}{3.327878in}}{\pgfqpoint{1.705530in}{3.323488in}}{\pgfqpoint{1.697716in}{3.315674in}}%
\pgfpathcurveto{\pgfqpoint{1.689903in}{3.307860in}}{\pgfqpoint{1.685512in}{3.297261in}}{\pgfqpoint{1.685512in}{3.286211in}}%
\pgfpathcurveto{\pgfqpoint{1.685512in}{3.275161in}}{\pgfqpoint{1.689903in}{3.264562in}}{\pgfqpoint{1.697716in}{3.256748in}}%
\pgfpathcurveto{\pgfqpoint{1.705530in}{3.248935in}}{\pgfqpoint{1.716129in}{3.244545in}}{\pgfqpoint{1.727179in}{3.244545in}}%
\pgfpathclose%
\pgfusepath{stroke,fill}%
\end{pgfscope}%
\begin{pgfscope}%
\pgfpathrectangle{\pgfqpoint{0.648703in}{0.548769in}}{\pgfqpoint{5.201297in}{3.102590in}}%
\pgfusepath{clip}%
\pgfsetbuttcap%
\pgfsetroundjoin%
\definecolor{currentfill}{rgb}{1.000000,0.498039,0.054902}%
\pgfsetfillcolor{currentfill}%
\pgfsetlinewidth{1.003750pt}%
\definecolor{currentstroke}{rgb}{1.000000,0.498039,0.054902}%
\pgfsetstrokecolor{currentstroke}%
\pgfsetdash{}{0pt}%
\pgfpathmoveto{\pgfqpoint{2.180592in}{3.189572in}}%
\pgfpathcurveto{\pgfqpoint{2.191642in}{3.189572in}}{\pgfqpoint{2.202241in}{3.193962in}}{\pgfqpoint{2.210055in}{3.201775in}}%
\pgfpathcurveto{\pgfqpoint{2.217869in}{3.209589in}}{\pgfqpoint{2.222259in}{3.220188in}}{\pgfqpoint{2.222259in}{3.231238in}}%
\pgfpathcurveto{\pgfqpoint{2.222259in}{3.242288in}}{\pgfqpoint{2.217869in}{3.252887in}}{\pgfqpoint{2.210055in}{3.260701in}}%
\pgfpathcurveto{\pgfqpoint{2.202241in}{3.268515in}}{\pgfqpoint{2.191642in}{3.272905in}}{\pgfqpoint{2.180592in}{3.272905in}}%
\pgfpathcurveto{\pgfqpoint{2.169542in}{3.272905in}}{\pgfqpoint{2.158943in}{3.268515in}}{\pgfqpoint{2.151129in}{3.260701in}}%
\pgfpathcurveto{\pgfqpoint{2.143316in}{3.252887in}}{\pgfqpoint{2.138925in}{3.242288in}}{\pgfqpoint{2.138925in}{3.231238in}}%
\pgfpathcurveto{\pgfqpoint{2.138925in}{3.220188in}}{\pgfqpoint{2.143316in}{3.209589in}}{\pgfqpoint{2.151129in}{3.201775in}}%
\pgfpathcurveto{\pgfqpoint{2.158943in}{3.193962in}}{\pgfqpoint{2.169542in}{3.189572in}}{\pgfqpoint{2.180592in}{3.189572in}}%
\pgfpathclose%
\pgfusepath{stroke,fill}%
\end{pgfscope}%
\begin{pgfscope}%
\pgfpathrectangle{\pgfqpoint{0.648703in}{0.548769in}}{\pgfqpoint{5.201297in}{3.102590in}}%
\pgfusepath{clip}%
\pgfsetbuttcap%
\pgfsetroundjoin%
\definecolor{currentfill}{rgb}{1.000000,0.498039,0.054902}%
\pgfsetfillcolor{currentfill}%
\pgfsetlinewidth{1.003750pt}%
\definecolor{currentstroke}{rgb}{1.000000,0.498039,0.054902}%
\pgfsetstrokecolor{currentstroke}%
\pgfsetdash{}{0pt}%
\pgfpathmoveto{\pgfqpoint{1.338539in}{3.210715in}}%
\pgfpathcurveto{\pgfqpoint{1.349589in}{3.210715in}}{\pgfqpoint{1.360188in}{3.215105in}}{\pgfqpoint{1.368002in}{3.222919in}}%
\pgfpathcurveto{\pgfqpoint{1.375816in}{3.230733in}}{\pgfqpoint{1.380206in}{3.241332in}}{\pgfqpoint{1.380206in}{3.252382in}}%
\pgfpathcurveto{\pgfqpoint{1.380206in}{3.263432in}}{\pgfqpoint{1.375816in}{3.274031in}}{\pgfqpoint{1.368002in}{3.281844in}}%
\pgfpathcurveto{\pgfqpoint{1.360188in}{3.289658in}}{\pgfqpoint{1.349589in}{3.294048in}}{\pgfqpoint{1.338539in}{3.294048in}}%
\pgfpathcurveto{\pgfqpoint{1.327489in}{3.294048in}}{\pgfqpoint{1.316890in}{3.289658in}}{\pgfqpoint{1.309076in}{3.281844in}}%
\pgfpathcurveto{\pgfqpoint{1.301263in}{3.274031in}}{\pgfqpoint{1.296872in}{3.263432in}}{\pgfqpoint{1.296872in}{3.252382in}}%
\pgfpathcurveto{\pgfqpoint{1.296872in}{3.241332in}}{\pgfqpoint{1.301263in}{3.230733in}}{\pgfqpoint{1.309076in}{3.222919in}}%
\pgfpathcurveto{\pgfqpoint{1.316890in}{3.215105in}}{\pgfqpoint{1.327489in}{3.210715in}}{\pgfqpoint{1.338539in}{3.210715in}}%
\pgfpathclose%
\pgfusepath{stroke,fill}%
\end{pgfscope}%
\begin{pgfscope}%
\pgfpathrectangle{\pgfqpoint{0.648703in}{0.548769in}}{\pgfqpoint{5.201297in}{3.102590in}}%
\pgfusepath{clip}%
\pgfsetbuttcap%
\pgfsetroundjoin%
\definecolor{currentfill}{rgb}{1.000000,0.498039,0.054902}%
\pgfsetfillcolor{currentfill}%
\pgfsetlinewidth{1.003750pt}%
\definecolor{currentstroke}{rgb}{1.000000,0.498039,0.054902}%
\pgfsetstrokecolor{currentstroke}%
\pgfsetdash{}{0pt}%
\pgfpathmoveto{\pgfqpoint{2.310139in}{3.189572in}}%
\pgfpathcurveto{\pgfqpoint{2.321189in}{3.189572in}}{\pgfqpoint{2.331788in}{3.193962in}}{\pgfqpoint{2.339602in}{3.201775in}}%
\pgfpathcurveto{\pgfqpoint{2.347415in}{3.209589in}}{\pgfqpoint{2.351805in}{3.220188in}}{\pgfqpoint{2.351805in}{3.231238in}}%
\pgfpathcurveto{\pgfqpoint{2.351805in}{3.242288in}}{\pgfqpoint{2.347415in}{3.252887in}}{\pgfqpoint{2.339602in}{3.260701in}}%
\pgfpathcurveto{\pgfqpoint{2.331788in}{3.268515in}}{\pgfqpoint{2.321189in}{3.272905in}}{\pgfqpoint{2.310139in}{3.272905in}}%
\pgfpathcurveto{\pgfqpoint{2.299089in}{3.272905in}}{\pgfqpoint{2.288490in}{3.268515in}}{\pgfqpoint{2.280676in}{3.260701in}}%
\pgfpathcurveto{\pgfqpoint{2.272862in}{3.252887in}}{\pgfqpoint{2.268472in}{3.242288in}}{\pgfqpoint{2.268472in}{3.231238in}}%
\pgfpathcurveto{\pgfqpoint{2.268472in}{3.220188in}}{\pgfqpoint{2.272862in}{3.209589in}}{\pgfqpoint{2.280676in}{3.201775in}}%
\pgfpathcurveto{\pgfqpoint{2.288490in}{3.193962in}}{\pgfqpoint{2.299089in}{3.189572in}}{\pgfqpoint{2.310139in}{3.189572in}}%
\pgfpathclose%
\pgfusepath{stroke,fill}%
\end{pgfscope}%
\begin{pgfscope}%
\pgfpathrectangle{\pgfqpoint{0.648703in}{0.548769in}}{\pgfqpoint{5.201297in}{3.102590in}}%
\pgfusepath{clip}%
\pgfsetbuttcap%
\pgfsetroundjoin%
\definecolor{currentfill}{rgb}{0.839216,0.152941,0.156863}%
\pgfsetfillcolor{currentfill}%
\pgfsetlinewidth{1.003750pt}%
\definecolor{currentstroke}{rgb}{0.839216,0.152941,0.156863}%
\pgfsetstrokecolor{currentstroke}%
\pgfsetdash{}{0pt}%
\pgfpathmoveto{\pgfqpoint{2.634005in}{3.181114in}}%
\pgfpathcurveto{\pgfqpoint{2.645055in}{3.181114in}}{\pgfqpoint{2.655654in}{3.185504in}}{\pgfqpoint{2.663468in}{3.193318in}}%
\pgfpathcurveto{\pgfqpoint{2.671282in}{3.201132in}}{\pgfqpoint{2.675672in}{3.211731in}}{\pgfqpoint{2.675672in}{3.222781in}}%
\pgfpathcurveto{\pgfqpoint{2.675672in}{3.233831in}}{\pgfqpoint{2.671282in}{3.244430in}}{\pgfqpoint{2.663468in}{3.252244in}}%
\pgfpathcurveto{\pgfqpoint{2.655654in}{3.260057in}}{\pgfqpoint{2.645055in}{3.264448in}}{\pgfqpoint{2.634005in}{3.264448in}}%
\pgfpathcurveto{\pgfqpoint{2.622955in}{3.264448in}}{\pgfqpoint{2.612356in}{3.260057in}}{\pgfqpoint{2.604543in}{3.252244in}}%
\pgfpathcurveto{\pgfqpoint{2.596729in}{3.244430in}}{\pgfqpoint{2.592339in}{3.233831in}}{\pgfqpoint{2.592339in}{3.222781in}}%
\pgfpathcurveto{\pgfqpoint{2.592339in}{3.211731in}}{\pgfqpoint{2.596729in}{3.201132in}}{\pgfqpoint{2.604543in}{3.193318in}}%
\pgfpathcurveto{\pgfqpoint{2.612356in}{3.185504in}}{\pgfqpoint{2.622955in}{3.181114in}}{\pgfqpoint{2.634005in}{3.181114in}}%
\pgfpathclose%
\pgfusepath{stroke,fill}%
\end{pgfscope}%
\begin{pgfscope}%
\pgfpathrectangle{\pgfqpoint{0.648703in}{0.548769in}}{\pgfqpoint{5.201297in}{3.102590in}}%
\pgfusepath{clip}%
\pgfsetbuttcap%
\pgfsetroundjoin%
\definecolor{currentfill}{rgb}{1.000000,0.498039,0.054902}%
\pgfsetfillcolor{currentfill}%
\pgfsetlinewidth{1.003750pt}%
\definecolor{currentstroke}{rgb}{1.000000,0.498039,0.054902}%
\pgfsetstrokecolor{currentstroke}%
\pgfsetdash{}{0pt}%
\pgfpathmoveto{\pgfqpoint{2.180592in}{3.202258in}}%
\pgfpathcurveto{\pgfqpoint{2.191642in}{3.202258in}}{\pgfqpoint{2.202241in}{3.206648in}}{\pgfqpoint{2.210055in}{3.214462in}}%
\pgfpathcurveto{\pgfqpoint{2.217869in}{3.222275in}}{\pgfqpoint{2.222259in}{3.232874in}}{\pgfqpoint{2.222259in}{3.243924in}}%
\pgfpathcurveto{\pgfqpoint{2.222259in}{3.254974in}}{\pgfqpoint{2.217869in}{3.265573in}}{\pgfqpoint{2.210055in}{3.273387in}}%
\pgfpathcurveto{\pgfqpoint{2.202241in}{3.281201in}}{\pgfqpoint{2.191642in}{3.285591in}}{\pgfqpoint{2.180592in}{3.285591in}}%
\pgfpathcurveto{\pgfqpoint{2.169542in}{3.285591in}}{\pgfqpoint{2.158943in}{3.281201in}}{\pgfqpoint{2.151129in}{3.273387in}}%
\pgfpathcurveto{\pgfqpoint{2.143316in}{3.265573in}}{\pgfqpoint{2.138925in}{3.254974in}}{\pgfqpoint{2.138925in}{3.243924in}}%
\pgfpathcurveto{\pgfqpoint{2.138925in}{3.232874in}}{\pgfqpoint{2.143316in}{3.222275in}}{\pgfqpoint{2.151129in}{3.214462in}}%
\pgfpathcurveto{\pgfqpoint{2.158943in}{3.206648in}}{\pgfqpoint{2.169542in}{3.202258in}}{\pgfqpoint{2.180592in}{3.202258in}}%
\pgfpathclose%
\pgfusepath{stroke,fill}%
\end{pgfscope}%
\begin{pgfscope}%
\pgfpathrectangle{\pgfqpoint{0.648703in}{0.548769in}}{\pgfqpoint{5.201297in}{3.102590in}}%
\pgfusepath{clip}%
\pgfsetbuttcap%
\pgfsetroundjoin%
\definecolor{currentfill}{rgb}{0.839216,0.152941,0.156863}%
\pgfsetfillcolor{currentfill}%
\pgfsetlinewidth{1.003750pt}%
\definecolor{currentstroke}{rgb}{0.839216,0.152941,0.156863}%
\pgfsetstrokecolor{currentstroke}%
\pgfsetdash{}{0pt}%
\pgfpathmoveto{\pgfqpoint{1.532859in}{3.265688in}}%
\pgfpathcurveto{\pgfqpoint{1.543909in}{3.265688in}}{\pgfqpoint{1.554508in}{3.270078in}}{\pgfqpoint{1.562322in}{3.277892in}}%
\pgfpathcurveto{\pgfqpoint{1.570135in}{3.285706in}}{\pgfqpoint{1.574526in}{3.296305in}}{\pgfqpoint{1.574526in}{3.307355in}}%
\pgfpathcurveto{\pgfqpoint{1.574526in}{3.318405in}}{\pgfqpoint{1.570135in}{3.329004in}}{\pgfqpoint{1.562322in}{3.336817in}}%
\pgfpathcurveto{\pgfqpoint{1.554508in}{3.344631in}}{\pgfqpoint{1.543909in}{3.349021in}}{\pgfqpoint{1.532859in}{3.349021in}}%
\pgfpathcurveto{\pgfqpoint{1.521809in}{3.349021in}}{\pgfqpoint{1.511210in}{3.344631in}}{\pgfqpoint{1.503396in}{3.336817in}}%
\pgfpathcurveto{\pgfqpoint{1.495583in}{3.329004in}}{\pgfqpoint{1.491192in}{3.318405in}}{\pgfqpoint{1.491192in}{3.307355in}}%
\pgfpathcurveto{\pgfqpoint{1.491192in}{3.296305in}}{\pgfqpoint{1.495583in}{3.285706in}}{\pgfqpoint{1.503396in}{3.277892in}}%
\pgfpathcurveto{\pgfqpoint{1.511210in}{3.270078in}}{\pgfqpoint{1.521809in}{3.265688in}}{\pgfqpoint{1.532859in}{3.265688in}}%
\pgfpathclose%
\pgfusepath{stroke,fill}%
\end{pgfscope}%
\begin{pgfscope}%
\pgfpathrectangle{\pgfqpoint{0.648703in}{0.548769in}}{\pgfqpoint{5.201297in}{3.102590in}}%
\pgfusepath{clip}%
\pgfsetbuttcap%
\pgfsetroundjoin%
\definecolor{currentfill}{rgb}{0.121569,0.466667,0.705882}%
\pgfsetfillcolor{currentfill}%
\pgfsetlinewidth{1.003750pt}%
\definecolor{currentstroke}{rgb}{0.121569,0.466667,0.705882}%
\pgfsetstrokecolor{currentstroke}%
\pgfsetdash{}{0pt}%
\pgfpathmoveto{\pgfqpoint{2.957872in}{2.808990in}}%
\pgfpathcurveto{\pgfqpoint{2.968922in}{2.808990in}}{\pgfqpoint{2.979521in}{2.813380in}}{\pgfqpoint{2.987335in}{2.821193in}}%
\pgfpathcurveto{\pgfqpoint{2.995148in}{2.829007in}}{\pgfqpoint{2.999538in}{2.839606in}}{\pgfqpoint{2.999538in}{2.850656in}}%
\pgfpathcurveto{\pgfqpoint{2.999538in}{2.861706in}}{\pgfqpoint{2.995148in}{2.872305in}}{\pgfqpoint{2.987335in}{2.880119in}}%
\pgfpathcurveto{\pgfqpoint{2.979521in}{2.887933in}}{\pgfqpoint{2.968922in}{2.892323in}}{\pgfqpoint{2.957872in}{2.892323in}}%
\pgfpathcurveto{\pgfqpoint{2.946822in}{2.892323in}}{\pgfqpoint{2.936223in}{2.887933in}}{\pgfqpoint{2.928409in}{2.880119in}}%
\pgfpathcurveto{\pgfqpoint{2.920595in}{2.872305in}}{\pgfqpoint{2.916205in}{2.861706in}}{\pgfqpoint{2.916205in}{2.850656in}}%
\pgfpathcurveto{\pgfqpoint{2.916205in}{2.839606in}}{\pgfqpoint{2.920595in}{2.829007in}}{\pgfqpoint{2.928409in}{2.821193in}}%
\pgfpathcurveto{\pgfqpoint{2.936223in}{2.813380in}}{\pgfqpoint{2.946822in}{2.808990in}}{\pgfqpoint{2.957872in}{2.808990in}}%
\pgfpathclose%
\pgfusepath{stroke,fill}%
\end{pgfscope}%
\begin{pgfscope}%
\pgfpathrectangle{\pgfqpoint{0.648703in}{0.548769in}}{\pgfqpoint{5.201297in}{3.102590in}}%
\pgfusepath{clip}%
\pgfsetbuttcap%
\pgfsetroundjoin%
\definecolor{currentfill}{rgb}{1.000000,0.498039,0.054902}%
\pgfsetfillcolor{currentfill}%
\pgfsetlinewidth{1.003750pt}%
\definecolor{currentstroke}{rgb}{1.000000,0.498039,0.054902}%
\pgfsetstrokecolor{currentstroke}%
\pgfsetdash{}{0pt}%
\pgfpathmoveto{\pgfqpoint{1.403312in}{3.185343in}}%
\pgfpathcurveto{\pgfqpoint{1.414363in}{3.185343in}}{\pgfqpoint{1.424962in}{3.189733in}}{\pgfqpoint{1.432775in}{3.197547in}}%
\pgfpathcurveto{\pgfqpoint{1.440589in}{3.205360in}}{\pgfqpoint{1.444979in}{3.215959in}}{\pgfqpoint{1.444979in}{3.227010in}}%
\pgfpathcurveto{\pgfqpoint{1.444979in}{3.238060in}}{\pgfqpoint{1.440589in}{3.248659in}}{\pgfqpoint{1.432775in}{3.256472in}}%
\pgfpathcurveto{\pgfqpoint{1.424962in}{3.264286in}}{\pgfqpoint{1.414363in}{3.268676in}}{\pgfqpoint{1.403312in}{3.268676in}}%
\pgfpathcurveto{\pgfqpoint{1.392262in}{3.268676in}}{\pgfqpoint{1.381663in}{3.264286in}}{\pgfqpoint{1.373850in}{3.256472in}}%
\pgfpathcurveto{\pgfqpoint{1.366036in}{3.248659in}}{\pgfqpoint{1.361646in}{3.238060in}}{\pgfqpoint{1.361646in}{3.227010in}}%
\pgfpathcurveto{\pgfqpoint{1.361646in}{3.215959in}}{\pgfqpoint{1.366036in}{3.205360in}}{\pgfqpoint{1.373850in}{3.197547in}}%
\pgfpathcurveto{\pgfqpoint{1.381663in}{3.189733in}}{\pgfqpoint{1.392262in}{3.185343in}}{\pgfqpoint{1.403312in}{3.185343in}}%
\pgfpathclose%
\pgfusepath{stroke,fill}%
\end{pgfscope}%
\begin{pgfscope}%
\pgfpathrectangle{\pgfqpoint{0.648703in}{0.548769in}}{\pgfqpoint{5.201297in}{3.102590in}}%
\pgfusepath{clip}%
\pgfsetbuttcap%
\pgfsetroundjoin%
\definecolor{currentfill}{rgb}{1.000000,0.498039,0.054902}%
\pgfsetfillcolor{currentfill}%
\pgfsetlinewidth{1.003750pt}%
\definecolor{currentstroke}{rgb}{1.000000,0.498039,0.054902}%
\pgfsetstrokecolor{currentstroke}%
\pgfsetdash{}{0pt}%
\pgfpathmoveto{\pgfqpoint{2.439685in}{3.193800in}}%
\pgfpathcurveto{\pgfqpoint{2.450735in}{3.193800in}}{\pgfqpoint{2.461335in}{3.198191in}}{\pgfqpoint{2.469148in}{3.206004in}}%
\pgfpathcurveto{\pgfqpoint{2.476962in}{3.213818in}}{\pgfqpoint{2.481352in}{3.224417in}}{\pgfqpoint{2.481352in}{3.235467in}}%
\pgfpathcurveto{\pgfqpoint{2.481352in}{3.246517in}}{\pgfqpoint{2.476962in}{3.257116in}}{\pgfqpoint{2.469148in}{3.264930in}}%
\pgfpathcurveto{\pgfqpoint{2.461335in}{3.272743in}}{\pgfqpoint{2.450735in}{3.277134in}}{\pgfqpoint{2.439685in}{3.277134in}}%
\pgfpathcurveto{\pgfqpoint{2.428635in}{3.277134in}}{\pgfqpoint{2.418036in}{3.272743in}}{\pgfqpoint{2.410223in}{3.264930in}}%
\pgfpathcurveto{\pgfqpoint{2.402409in}{3.257116in}}{\pgfqpoint{2.398019in}{3.246517in}}{\pgfqpoint{2.398019in}{3.235467in}}%
\pgfpathcurveto{\pgfqpoint{2.398019in}{3.224417in}}{\pgfqpoint{2.402409in}{3.213818in}}{\pgfqpoint{2.410223in}{3.206004in}}%
\pgfpathcurveto{\pgfqpoint{2.418036in}{3.198191in}}{\pgfqpoint{2.428635in}{3.193800in}}{\pgfqpoint{2.439685in}{3.193800in}}%
\pgfpathclose%
\pgfusepath{stroke,fill}%
\end{pgfscope}%
\begin{pgfscope}%
\pgfpathrectangle{\pgfqpoint{0.648703in}{0.548769in}}{\pgfqpoint{5.201297in}{3.102590in}}%
\pgfusepath{clip}%
\pgfsetbuttcap%
\pgfsetroundjoin%
\definecolor{currentfill}{rgb}{1.000000,0.498039,0.054902}%
\pgfsetfillcolor{currentfill}%
\pgfsetlinewidth{1.003750pt}%
\definecolor{currentstroke}{rgb}{1.000000,0.498039,0.054902}%
\pgfsetstrokecolor{currentstroke}%
\pgfsetdash{}{0pt}%
\pgfpathmoveto{\pgfqpoint{1.014673in}{3.362948in}}%
\pgfpathcurveto{\pgfqpoint{1.025723in}{3.362948in}}{\pgfqpoint{1.036322in}{3.367338in}}{\pgfqpoint{1.044135in}{3.375152in}}%
\pgfpathcurveto{\pgfqpoint{1.051949in}{3.382965in}}{\pgfqpoint{1.056339in}{3.393564in}}{\pgfqpoint{1.056339in}{3.404615in}}%
\pgfpathcurveto{\pgfqpoint{1.056339in}{3.415665in}}{\pgfqpoint{1.051949in}{3.426264in}}{\pgfqpoint{1.044135in}{3.434077in}}%
\pgfpathcurveto{\pgfqpoint{1.036322in}{3.441891in}}{\pgfqpoint{1.025723in}{3.446281in}}{\pgfqpoint{1.014673in}{3.446281in}}%
\pgfpathcurveto{\pgfqpoint{1.003622in}{3.446281in}}{\pgfqpoint{0.993023in}{3.441891in}}{\pgfqpoint{0.985210in}{3.434077in}}%
\pgfpathcurveto{\pgfqpoint{0.977396in}{3.426264in}}{\pgfqpoint{0.973006in}{3.415665in}}{\pgfqpoint{0.973006in}{3.404615in}}%
\pgfpathcurveto{\pgfqpoint{0.973006in}{3.393564in}}{\pgfqpoint{0.977396in}{3.382965in}}{\pgfqpoint{0.985210in}{3.375152in}}%
\pgfpathcurveto{\pgfqpoint{0.993023in}{3.367338in}}{\pgfqpoint{1.003622in}{3.362948in}}{\pgfqpoint{1.014673in}{3.362948in}}%
\pgfpathclose%
\pgfusepath{stroke,fill}%
\end{pgfscope}%
\begin{pgfscope}%
\pgfpathrectangle{\pgfqpoint{0.648703in}{0.548769in}}{\pgfqpoint{5.201297in}{3.102590in}}%
\pgfusepath{clip}%
\pgfsetbuttcap%
\pgfsetroundjoin%
\definecolor{currentfill}{rgb}{1.000000,0.498039,0.054902}%
\pgfsetfillcolor{currentfill}%
\pgfsetlinewidth{1.003750pt}%
\definecolor{currentstroke}{rgb}{1.000000,0.498039,0.054902}%
\pgfsetstrokecolor{currentstroke}%
\pgfsetdash{}{0pt}%
\pgfpathmoveto{\pgfqpoint{2.634005in}{3.257231in}}%
\pgfpathcurveto{\pgfqpoint{2.645055in}{3.257231in}}{\pgfqpoint{2.655654in}{3.261621in}}{\pgfqpoint{2.663468in}{3.269435in}}%
\pgfpathcurveto{\pgfqpoint{2.671282in}{3.277248in}}{\pgfqpoint{2.675672in}{3.287847in}}{\pgfqpoint{2.675672in}{3.298897in}}%
\pgfpathcurveto{\pgfqpoint{2.675672in}{3.309947in}}{\pgfqpoint{2.671282in}{3.320546in}}{\pgfqpoint{2.663468in}{3.328360in}}%
\pgfpathcurveto{\pgfqpoint{2.655654in}{3.336174in}}{\pgfqpoint{2.645055in}{3.340564in}}{\pgfqpoint{2.634005in}{3.340564in}}%
\pgfpathcurveto{\pgfqpoint{2.622955in}{3.340564in}}{\pgfqpoint{2.612356in}{3.336174in}}{\pgfqpoint{2.604543in}{3.328360in}}%
\pgfpathcurveto{\pgfqpoint{2.596729in}{3.320546in}}{\pgfqpoint{2.592339in}{3.309947in}}{\pgfqpoint{2.592339in}{3.298897in}}%
\pgfpathcurveto{\pgfqpoint{2.592339in}{3.287847in}}{\pgfqpoint{2.596729in}{3.277248in}}{\pgfqpoint{2.604543in}{3.269435in}}%
\pgfpathcurveto{\pgfqpoint{2.612356in}{3.261621in}}{\pgfqpoint{2.622955in}{3.257231in}}{\pgfqpoint{2.634005in}{3.257231in}}%
\pgfpathclose%
\pgfusepath{stroke,fill}%
\end{pgfscope}%
\begin{pgfscope}%
\pgfpathrectangle{\pgfqpoint{0.648703in}{0.548769in}}{\pgfqpoint{5.201297in}{3.102590in}}%
\pgfusepath{clip}%
\pgfsetbuttcap%
\pgfsetroundjoin%
\definecolor{currentfill}{rgb}{1.000000,0.498039,0.054902}%
\pgfsetfillcolor{currentfill}%
\pgfsetlinewidth{1.003750pt}%
\definecolor{currentstroke}{rgb}{1.000000,0.498039,0.054902}%
\pgfsetstrokecolor{currentstroke}%
\pgfsetdash{}{0pt}%
\pgfpathmoveto{\pgfqpoint{1.921499in}{3.189572in}}%
\pgfpathcurveto{\pgfqpoint{1.932549in}{3.189572in}}{\pgfqpoint{1.943148in}{3.193962in}}{\pgfqpoint{1.950962in}{3.201775in}}%
\pgfpathcurveto{\pgfqpoint{1.958775in}{3.209589in}}{\pgfqpoint{1.963166in}{3.220188in}}{\pgfqpoint{1.963166in}{3.231238in}}%
\pgfpathcurveto{\pgfqpoint{1.963166in}{3.242288in}}{\pgfqpoint{1.958775in}{3.252887in}}{\pgfqpoint{1.950962in}{3.260701in}}%
\pgfpathcurveto{\pgfqpoint{1.943148in}{3.268515in}}{\pgfqpoint{1.932549in}{3.272905in}}{\pgfqpoint{1.921499in}{3.272905in}}%
\pgfpathcurveto{\pgfqpoint{1.910449in}{3.272905in}}{\pgfqpoint{1.899850in}{3.268515in}}{\pgfqpoint{1.892036in}{3.260701in}}%
\pgfpathcurveto{\pgfqpoint{1.884223in}{3.252887in}}{\pgfqpoint{1.879832in}{3.242288in}}{\pgfqpoint{1.879832in}{3.231238in}}%
\pgfpathcurveto{\pgfqpoint{1.879832in}{3.220188in}}{\pgfqpoint{1.884223in}{3.209589in}}{\pgfqpoint{1.892036in}{3.201775in}}%
\pgfpathcurveto{\pgfqpoint{1.899850in}{3.193962in}}{\pgfqpoint{1.910449in}{3.189572in}}{\pgfqpoint{1.921499in}{3.189572in}}%
\pgfpathclose%
\pgfusepath{stroke,fill}%
\end{pgfscope}%
\begin{pgfscope}%
\pgfpathrectangle{\pgfqpoint{0.648703in}{0.548769in}}{\pgfqpoint{5.201297in}{3.102590in}}%
\pgfusepath{clip}%
\pgfsetbuttcap%
\pgfsetroundjoin%
\definecolor{currentfill}{rgb}{1.000000,0.498039,0.054902}%
\pgfsetfillcolor{currentfill}%
\pgfsetlinewidth{1.003750pt}%
\definecolor{currentstroke}{rgb}{1.000000,0.498039,0.054902}%
\pgfsetstrokecolor{currentstroke}%
\pgfsetdash{}{0pt}%
\pgfpathmoveto{\pgfqpoint{2.828325in}{3.185343in}}%
\pgfpathcurveto{\pgfqpoint{2.839375in}{3.185343in}}{\pgfqpoint{2.849974in}{3.189733in}}{\pgfqpoint{2.857788in}{3.197547in}}%
\pgfpathcurveto{\pgfqpoint{2.865602in}{3.205360in}}{\pgfqpoint{2.869992in}{3.215959in}}{\pgfqpoint{2.869992in}{3.227010in}}%
\pgfpathcurveto{\pgfqpoint{2.869992in}{3.238060in}}{\pgfqpoint{2.865602in}{3.248659in}}{\pgfqpoint{2.857788in}{3.256472in}}%
\pgfpathcurveto{\pgfqpoint{2.849974in}{3.264286in}}{\pgfqpoint{2.839375in}{3.268676in}}{\pgfqpoint{2.828325in}{3.268676in}}%
\pgfpathcurveto{\pgfqpoint{2.817275in}{3.268676in}}{\pgfqpoint{2.806676in}{3.264286in}}{\pgfqpoint{2.798862in}{3.256472in}}%
\pgfpathcurveto{\pgfqpoint{2.791049in}{3.248659in}}{\pgfqpoint{2.786659in}{3.238060in}}{\pgfqpoint{2.786659in}{3.227010in}}%
\pgfpathcurveto{\pgfqpoint{2.786659in}{3.215959in}}{\pgfqpoint{2.791049in}{3.205360in}}{\pgfqpoint{2.798862in}{3.197547in}}%
\pgfpathcurveto{\pgfqpoint{2.806676in}{3.189733in}}{\pgfqpoint{2.817275in}{3.185343in}}{\pgfqpoint{2.828325in}{3.185343in}}%
\pgfpathclose%
\pgfusepath{stroke,fill}%
\end{pgfscope}%
\begin{pgfscope}%
\pgfpathrectangle{\pgfqpoint{0.648703in}{0.548769in}}{\pgfqpoint{5.201297in}{3.102590in}}%
\pgfusepath{clip}%
\pgfsetbuttcap%
\pgfsetroundjoin%
\definecolor{currentfill}{rgb}{0.121569,0.466667,0.705882}%
\pgfsetfillcolor{currentfill}%
\pgfsetlinewidth{1.003750pt}%
\definecolor{currentstroke}{rgb}{0.121569,0.466667,0.705882}%
\pgfsetstrokecolor{currentstroke}%
\pgfsetdash{}{0pt}%
\pgfpathmoveto{\pgfqpoint{1.662406in}{0.681958in}}%
\pgfpathcurveto{\pgfqpoint{1.673456in}{0.681958in}}{\pgfqpoint{1.684055in}{0.686349in}}{\pgfqpoint{1.691868in}{0.694162in}}%
\pgfpathcurveto{\pgfqpoint{1.699682in}{0.701976in}}{\pgfqpoint{1.704072in}{0.712575in}}{\pgfqpoint{1.704072in}{0.723625in}}%
\pgfpathcurveto{\pgfqpoint{1.704072in}{0.734675in}}{\pgfqpoint{1.699682in}{0.745274in}}{\pgfqpoint{1.691868in}{0.753088in}}%
\pgfpathcurveto{\pgfqpoint{1.684055in}{0.760902in}}{\pgfqpoint{1.673456in}{0.765292in}}{\pgfqpoint{1.662406in}{0.765292in}}%
\pgfpathcurveto{\pgfqpoint{1.651356in}{0.765292in}}{\pgfqpoint{1.640757in}{0.760902in}}{\pgfqpoint{1.632943in}{0.753088in}}%
\pgfpathcurveto{\pgfqpoint{1.625129in}{0.745274in}}{\pgfqpoint{1.620739in}{0.734675in}}{\pgfqpoint{1.620739in}{0.723625in}}%
\pgfpathcurveto{\pgfqpoint{1.620739in}{0.712575in}}{\pgfqpoint{1.625129in}{0.701976in}}{\pgfqpoint{1.632943in}{0.694162in}}%
\pgfpathcurveto{\pgfqpoint{1.640757in}{0.686349in}}{\pgfqpoint{1.651356in}{0.681958in}}{\pgfqpoint{1.662406in}{0.681958in}}%
\pgfpathclose%
\pgfusepath{stroke,fill}%
\end{pgfscope}%
\begin{pgfscope}%
\pgfpathrectangle{\pgfqpoint{0.648703in}{0.548769in}}{\pgfqpoint{5.201297in}{3.102590in}}%
\pgfusepath{clip}%
\pgfsetbuttcap%
\pgfsetroundjoin%
\definecolor{currentfill}{rgb}{0.839216,0.152941,0.156863}%
\pgfsetfillcolor{currentfill}%
\pgfsetlinewidth{1.003750pt}%
\definecolor{currentstroke}{rgb}{0.839216,0.152941,0.156863}%
\pgfsetstrokecolor{currentstroke}%
\pgfsetdash{}{0pt}%
\pgfpathmoveto{\pgfqpoint{1.727179in}{3.214944in}}%
\pgfpathcurveto{\pgfqpoint{1.738229in}{3.214944in}}{\pgfqpoint{1.748828in}{3.219334in}}{\pgfqpoint{1.756642in}{3.227148in}}%
\pgfpathcurveto{\pgfqpoint{1.764455in}{3.234961in}}{\pgfqpoint{1.768846in}{3.245560in}}{\pgfqpoint{1.768846in}{3.256610in}}%
\pgfpathcurveto{\pgfqpoint{1.768846in}{3.267661in}}{\pgfqpoint{1.764455in}{3.278260in}}{\pgfqpoint{1.756642in}{3.286073in}}%
\pgfpathcurveto{\pgfqpoint{1.748828in}{3.293887in}}{\pgfqpoint{1.738229in}{3.298277in}}{\pgfqpoint{1.727179in}{3.298277in}}%
\pgfpathcurveto{\pgfqpoint{1.716129in}{3.298277in}}{\pgfqpoint{1.705530in}{3.293887in}}{\pgfqpoint{1.697716in}{3.286073in}}%
\pgfpathcurveto{\pgfqpoint{1.689903in}{3.278260in}}{\pgfqpoint{1.685512in}{3.267661in}}{\pgfqpoint{1.685512in}{3.256610in}}%
\pgfpathcurveto{\pgfqpoint{1.685512in}{3.245560in}}{\pgfqpoint{1.689903in}{3.234961in}}{\pgfqpoint{1.697716in}{3.227148in}}%
\pgfpathcurveto{\pgfqpoint{1.705530in}{3.219334in}}{\pgfqpoint{1.716129in}{3.214944in}}{\pgfqpoint{1.727179in}{3.214944in}}%
\pgfpathclose%
\pgfusepath{stroke,fill}%
\end{pgfscope}%
\begin{pgfscope}%
\pgfpathrectangle{\pgfqpoint{0.648703in}{0.548769in}}{\pgfqpoint{5.201297in}{3.102590in}}%
\pgfusepath{clip}%
\pgfsetbuttcap%
\pgfsetroundjoin%
\definecolor{currentfill}{rgb}{1.000000,0.498039,0.054902}%
\pgfsetfillcolor{currentfill}%
\pgfsetlinewidth{1.003750pt}%
\definecolor{currentstroke}{rgb}{1.000000,0.498039,0.054902}%
\pgfsetstrokecolor{currentstroke}%
\pgfsetdash{}{0pt}%
\pgfpathmoveto{\pgfqpoint{0.885126in}{3.185343in}}%
\pgfpathcurveto{\pgfqpoint{0.896176in}{3.185343in}}{\pgfqpoint{0.906775in}{3.189733in}}{\pgfqpoint{0.914589in}{3.197547in}}%
\pgfpathcurveto{\pgfqpoint{0.922402in}{3.205360in}}{\pgfqpoint{0.926793in}{3.215959in}}{\pgfqpoint{0.926793in}{3.227010in}}%
\pgfpathcurveto{\pgfqpoint{0.926793in}{3.238060in}}{\pgfqpoint{0.922402in}{3.248659in}}{\pgfqpoint{0.914589in}{3.256472in}}%
\pgfpathcurveto{\pgfqpoint{0.906775in}{3.264286in}}{\pgfqpoint{0.896176in}{3.268676in}}{\pgfqpoint{0.885126in}{3.268676in}}%
\pgfpathcurveto{\pgfqpoint{0.874076in}{3.268676in}}{\pgfqpoint{0.863477in}{3.264286in}}{\pgfqpoint{0.855663in}{3.256472in}}%
\pgfpathcurveto{\pgfqpoint{0.847850in}{3.248659in}}{\pgfqpoint{0.843459in}{3.238060in}}{\pgfqpoint{0.843459in}{3.227010in}}%
\pgfpathcurveto{\pgfqpoint{0.843459in}{3.215959in}}{\pgfqpoint{0.847850in}{3.205360in}}{\pgfqpoint{0.855663in}{3.197547in}}%
\pgfpathcurveto{\pgfqpoint{0.863477in}{3.189733in}}{\pgfqpoint{0.874076in}{3.185343in}}{\pgfqpoint{0.885126in}{3.185343in}}%
\pgfpathclose%
\pgfusepath{stroke,fill}%
\end{pgfscope}%
\begin{pgfscope}%
\pgfpathrectangle{\pgfqpoint{0.648703in}{0.548769in}}{\pgfqpoint{5.201297in}{3.102590in}}%
\pgfusepath{clip}%
\pgfsetbuttcap%
\pgfsetroundjoin%
\definecolor{currentfill}{rgb}{0.121569,0.466667,0.705882}%
\pgfsetfillcolor{currentfill}%
\pgfsetlinewidth{1.003750pt}%
\definecolor{currentstroke}{rgb}{0.121569,0.466667,0.705882}%
\pgfsetstrokecolor{currentstroke}%
\pgfsetdash{}{0pt}%
\pgfpathmoveto{\pgfqpoint{1.144219in}{1.138657in}}%
\pgfpathcurveto{\pgfqpoint{1.155269in}{1.138657in}}{\pgfqpoint{1.165868in}{1.143047in}}{\pgfqpoint{1.173682in}{1.150861in}}%
\pgfpathcurveto{\pgfqpoint{1.181496in}{1.158674in}}{\pgfqpoint{1.185886in}{1.169274in}}{\pgfqpoint{1.185886in}{1.180324in}}%
\pgfpathcurveto{\pgfqpoint{1.185886in}{1.191374in}}{\pgfqpoint{1.181496in}{1.201973in}}{\pgfqpoint{1.173682in}{1.209786in}}%
\pgfpathcurveto{\pgfqpoint{1.165868in}{1.217600in}}{\pgfqpoint{1.155269in}{1.221990in}}{\pgfqpoint{1.144219in}{1.221990in}}%
\pgfpathcurveto{\pgfqpoint{1.133169in}{1.221990in}}{\pgfqpoint{1.122570in}{1.217600in}}{\pgfqpoint{1.114756in}{1.209786in}}%
\pgfpathcurveto{\pgfqpoint{1.106943in}{1.201973in}}{\pgfqpoint{1.102553in}{1.191374in}}{\pgfqpoint{1.102553in}{1.180324in}}%
\pgfpathcurveto{\pgfqpoint{1.102553in}{1.169274in}}{\pgfqpoint{1.106943in}{1.158674in}}{\pgfqpoint{1.114756in}{1.150861in}}%
\pgfpathcurveto{\pgfqpoint{1.122570in}{1.143047in}}{\pgfqpoint{1.133169in}{1.138657in}}{\pgfqpoint{1.144219in}{1.138657in}}%
\pgfpathclose%
\pgfusepath{stroke,fill}%
\end{pgfscope}%
\begin{pgfscope}%
\pgfpathrectangle{\pgfqpoint{0.648703in}{0.548769in}}{\pgfqpoint{5.201297in}{3.102590in}}%
\pgfusepath{clip}%
\pgfsetbuttcap%
\pgfsetroundjoin%
\definecolor{currentfill}{rgb}{0.121569,0.466667,0.705882}%
\pgfsetfillcolor{currentfill}%
\pgfsetlinewidth{1.003750pt}%
\definecolor{currentstroke}{rgb}{0.121569,0.466667,0.705882}%
\pgfsetstrokecolor{currentstroke}%
\pgfsetdash{}{0pt}%
\pgfpathmoveto{\pgfqpoint{2.115819in}{1.087913in}}%
\pgfpathcurveto{\pgfqpoint{2.126869in}{1.087913in}}{\pgfqpoint{2.137468in}{1.092303in}}{\pgfqpoint{2.145282in}{1.100117in}}%
\pgfpathcurveto{\pgfqpoint{2.153095in}{1.107930in}}{\pgfqpoint{2.157485in}{1.118529in}}{\pgfqpoint{2.157485in}{1.129579in}}%
\pgfpathcurveto{\pgfqpoint{2.157485in}{1.140629in}}{\pgfqpoint{2.153095in}{1.151229in}}{\pgfqpoint{2.145282in}{1.159042in}}%
\pgfpathcurveto{\pgfqpoint{2.137468in}{1.166856in}}{\pgfqpoint{2.126869in}{1.171246in}}{\pgfqpoint{2.115819in}{1.171246in}}%
\pgfpathcurveto{\pgfqpoint{2.104769in}{1.171246in}}{\pgfqpoint{2.094170in}{1.166856in}}{\pgfqpoint{2.086356in}{1.159042in}}%
\pgfpathcurveto{\pgfqpoint{2.078542in}{1.151229in}}{\pgfqpoint{2.074152in}{1.140629in}}{\pgfqpoint{2.074152in}{1.129579in}}%
\pgfpathcurveto{\pgfqpoint{2.074152in}{1.118529in}}{\pgfqpoint{2.078542in}{1.107930in}}{\pgfqpoint{2.086356in}{1.100117in}}%
\pgfpathcurveto{\pgfqpoint{2.094170in}{1.092303in}}{\pgfqpoint{2.104769in}{1.087913in}}{\pgfqpoint{2.115819in}{1.087913in}}%
\pgfpathclose%
\pgfusepath{stroke,fill}%
\end{pgfscope}%
\begin{pgfscope}%
\pgfpathrectangle{\pgfqpoint{0.648703in}{0.548769in}}{\pgfqpoint{5.201297in}{3.102590in}}%
\pgfusepath{clip}%
\pgfsetbuttcap%
\pgfsetroundjoin%
\definecolor{currentfill}{rgb}{0.121569,0.466667,0.705882}%
\pgfsetfillcolor{currentfill}%
\pgfsetlinewidth{1.003750pt}%
\definecolor{currentstroke}{rgb}{0.121569,0.466667,0.705882}%
\pgfsetstrokecolor{currentstroke}%
\pgfsetdash{}{0pt}%
\pgfpathmoveto{\pgfqpoint{2.051046in}{1.024482in}}%
\pgfpathcurveto{\pgfqpoint{2.062096in}{1.024482in}}{\pgfqpoint{2.072695in}{1.028873in}}{\pgfqpoint{2.080508in}{1.036686in}}%
\pgfpathcurveto{\pgfqpoint{2.088322in}{1.044500in}}{\pgfqpoint{2.092712in}{1.055099in}}{\pgfqpoint{2.092712in}{1.066149in}}%
\pgfpathcurveto{\pgfqpoint{2.092712in}{1.077199in}}{\pgfqpoint{2.088322in}{1.087798in}}{\pgfqpoint{2.080508in}{1.095612in}}%
\pgfpathcurveto{\pgfqpoint{2.072695in}{1.103425in}}{\pgfqpoint{2.062096in}{1.107816in}}{\pgfqpoint{2.051046in}{1.107816in}}%
\pgfpathcurveto{\pgfqpoint{2.039995in}{1.107816in}}{\pgfqpoint{2.029396in}{1.103425in}}{\pgfqpoint{2.021583in}{1.095612in}}%
\pgfpathcurveto{\pgfqpoint{2.013769in}{1.087798in}}{\pgfqpoint{2.009379in}{1.077199in}}{\pgfqpoint{2.009379in}{1.066149in}}%
\pgfpathcurveto{\pgfqpoint{2.009379in}{1.055099in}}{\pgfqpoint{2.013769in}{1.044500in}}{\pgfqpoint{2.021583in}{1.036686in}}%
\pgfpathcurveto{\pgfqpoint{2.029396in}{1.028873in}}{\pgfqpoint{2.039995in}{1.024482in}}{\pgfqpoint{2.051046in}{1.024482in}}%
\pgfpathclose%
\pgfusepath{stroke,fill}%
\end{pgfscope}%
\begin{pgfscope}%
\pgfpathrectangle{\pgfqpoint{0.648703in}{0.548769in}}{\pgfqpoint{5.201297in}{3.102590in}}%
\pgfusepath{clip}%
\pgfsetbuttcap%
\pgfsetroundjoin%
\definecolor{currentfill}{rgb}{0.121569,0.466667,0.705882}%
\pgfsetfillcolor{currentfill}%
\pgfsetlinewidth{1.003750pt}%
\definecolor{currentstroke}{rgb}{0.121569,0.466667,0.705882}%
\pgfsetstrokecolor{currentstroke}%
\pgfsetdash{}{0pt}%
\pgfpathmoveto{\pgfqpoint{3.022645in}{0.813048in}}%
\pgfpathcurveto{\pgfqpoint{3.033695in}{0.813048in}}{\pgfqpoint{3.044294in}{0.817438in}}{\pgfqpoint{3.052108in}{0.825252in}}%
\pgfpathcurveto{\pgfqpoint{3.059922in}{0.833065in}}{\pgfqpoint{3.064312in}{0.843664in}}{\pgfqpoint{3.064312in}{0.854715in}}%
\pgfpathcurveto{\pgfqpoint{3.064312in}{0.865765in}}{\pgfqpoint{3.059922in}{0.876364in}}{\pgfqpoint{3.052108in}{0.884177in}}%
\pgfpathcurveto{\pgfqpoint{3.044294in}{0.891991in}}{\pgfqpoint{3.033695in}{0.896381in}}{\pgfqpoint{3.022645in}{0.896381in}}%
\pgfpathcurveto{\pgfqpoint{3.011595in}{0.896381in}}{\pgfqpoint{3.000996in}{0.891991in}}{\pgfqpoint{2.993182in}{0.884177in}}%
\pgfpathcurveto{\pgfqpoint{2.985369in}{0.876364in}}{\pgfqpoint{2.980978in}{0.865765in}}{\pgfqpoint{2.980978in}{0.854715in}}%
\pgfpathcurveto{\pgfqpoint{2.980978in}{0.843664in}}{\pgfqpoint{2.985369in}{0.833065in}}{\pgfqpoint{2.993182in}{0.825252in}}%
\pgfpathcurveto{\pgfqpoint{3.000996in}{0.817438in}}{\pgfqpoint{3.011595in}{0.813048in}}{\pgfqpoint{3.022645in}{0.813048in}}%
\pgfpathclose%
\pgfusepath{stroke,fill}%
\end{pgfscope}%
\begin{pgfscope}%
\pgfpathrectangle{\pgfqpoint{0.648703in}{0.548769in}}{\pgfqpoint{5.201297in}{3.102590in}}%
\pgfusepath{clip}%
\pgfsetbuttcap%
\pgfsetroundjoin%
\definecolor{currentfill}{rgb}{0.839216,0.152941,0.156863}%
\pgfsetfillcolor{currentfill}%
\pgfsetlinewidth{1.003750pt}%
\definecolor{currentstroke}{rgb}{0.839216,0.152941,0.156863}%
\pgfsetstrokecolor{currentstroke}%
\pgfsetdash{}{0pt}%
\pgfpathmoveto{\pgfqpoint{1.273766in}{3.202258in}}%
\pgfpathcurveto{\pgfqpoint{1.284816in}{3.202258in}}{\pgfqpoint{1.295415in}{3.206648in}}{\pgfqpoint{1.303229in}{3.214462in}}%
\pgfpathcurveto{\pgfqpoint{1.311042in}{3.222275in}}{\pgfqpoint{1.315432in}{3.232874in}}{\pgfqpoint{1.315432in}{3.243924in}}%
\pgfpathcurveto{\pgfqpoint{1.315432in}{3.254974in}}{\pgfqpoint{1.311042in}{3.265573in}}{\pgfqpoint{1.303229in}{3.273387in}}%
\pgfpathcurveto{\pgfqpoint{1.295415in}{3.281201in}}{\pgfqpoint{1.284816in}{3.285591in}}{\pgfqpoint{1.273766in}{3.285591in}}%
\pgfpathcurveto{\pgfqpoint{1.262716in}{3.285591in}}{\pgfqpoint{1.252117in}{3.281201in}}{\pgfqpoint{1.244303in}{3.273387in}}%
\pgfpathcurveto{\pgfqpoint{1.236489in}{3.265573in}}{\pgfqpoint{1.232099in}{3.254974in}}{\pgfqpoint{1.232099in}{3.243924in}}%
\pgfpathcurveto{\pgfqpoint{1.232099in}{3.232874in}}{\pgfqpoint{1.236489in}{3.222275in}}{\pgfqpoint{1.244303in}{3.214462in}}%
\pgfpathcurveto{\pgfqpoint{1.252117in}{3.206648in}}{\pgfqpoint{1.262716in}{3.202258in}}{\pgfqpoint{1.273766in}{3.202258in}}%
\pgfpathclose%
\pgfusepath{stroke,fill}%
\end{pgfscope}%
\begin{pgfscope}%
\pgfsetbuttcap%
\pgfsetroundjoin%
\definecolor{currentfill}{rgb}{0.000000,0.000000,0.000000}%
\pgfsetfillcolor{currentfill}%
\pgfsetlinewidth{0.803000pt}%
\definecolor{currentstroke}{rgb}{0.000000,0.000000,0.000000}%
\pgfsetstrokecolor{currentstroke}%
\pgfsetdash{}{0pt}%
\pgfsys@defobject{currentmarker}{\pgfqpoint{0.000000in}{-0.048611in}}{\pgfqpoint{0.000000in}{0.000000in}}{%
\pgfpathmoveto{\pgfqpoint{0.000000in}{0.000000in}}%
\pgfpathlineto{\pgfqpoint{0.000000in}{-0.048611in}}%
\pgfusepath{stroke,fill}%
}%
\begin{pgfscope}%
\pgfsys@transformshift{0.820353in}{0.548769in}%
\pgfsys@useobject{currentmarker}{}%
\end{pgfscope}%
\end{pgfscope}%
\begin{pgfscope}%
\definecolor{textcolor}{rgb}{0.000000,0.000000,0.000000}%
\pgfsetstrokecolor{textcolor}%
\pgfsetfillcolor{textcolor}%
\pgftext[x=0.820353in,y=0.451547in,,top]{\color{textcolor}\sffamily\fontsize{10.000000}{12.000000}\selectfont \(\displaystyle {0}\)}%
\end{pgfscope}%
\begin{pgfscope}%
\pgfsetbuttcap%
\pgfsetroundjoin%
\definecolor{currentfill}{rgb}{0.000000,0.000000,0.000000}%
\pgfsetfillcolor{currentfill}%
\pgfsetlinewidth{0.803000pt}%
\definecolor{currentstroke}{rgb}{0.000000,0.000000,0.000000}%
\pgfsetstrokecolor{currentstroke}%
\pgfsetdash{}{0pt}%
\pgfsys@defobject{currentmarker}{\pgfqpoint{0.000000in}{-0.048611in}}{\pgfqpoint{0.000000in}{0.000000in}}{%
\pgfpathmoveto{\pgfqpoint{0.000000in}{0.000000in}}%
\pgfpathlineto{\pgfqpoint{0.000000in}{-0.048611in}}%
\pgfusepath{stroke,fill}%
}%
\begin{pgfscope}%
\pgfsys@transformshift{1.468086in}{0.548769in}%
\pgfsys@useobject{currentmarker}{}%
\end{pgfscope}%
\end{pgfscope}%
\begin{pgfscope}%
\definecolor{textcolor}{rgb}{0.000000,0.000000,0.000000}%
\pgfsetstrokecolor{textcolor}%
\pgfsetfillcolor{textcolor}%
\pgftext[x=1.468086in,y=0.451547in,,top]{\color{textcolor}\sffamily\fontsize{10.000000}{12.000000}\selectfont \(\displaystyle {10}\)}%
\end{pgfscope}%
\begin{pgfscope}%
\pgfsetbuttcap%
\pgfsetroundjoin%
\definecolor{currentfill}{rgb}{0.000000,0.000000,0.000000}%
\pgfsetfillcolor{currentfill}%
\pgfsetlinewidth{0.803000pt}%
\definecolor{currentstroke}{rgb}{0.000000,0.000000,0.000000}%
\pgfsetstrokecolor{currentstroke}%
\pgfsetdash{}{0pt}%
\pgfsys@defobject{currentmarker}{\pgfqpoint{0.000000in}{-0.048611in}}{\pgfqpoint{0.000000in}{0.000000in}}{%
\pgfpathmoveto{\pgfqpoint{0.000000in}{0.000000in}}%
\pgfpathlineto{\pgfqpoint{0.000000in}{-0.048611in}}%
\pgfusepath{stroke,fill}%
}%
\begin{pgfscope}%
\pgfsys@transformshift{2.115819in}{0.548769in}%
\pgfsys@useobject{currentmarker}{}%
\end{pgfscope}%
\end{pgfscope}%
\begin{pgfscope}%
\definecolor{textcolor}{rgb}{0.000000,0.000000,0.000000}%
\pgfsetstrokecolor{textcolor}%
\pgfsetfillcolor{textcolor}%
\pgftext[x=2.115819in,y=0.451547in,,top]{\color{textcolor}\sffamily\fontsize{10.000000}{12.000000}\selectfont \(\displaystyle {20}\)}%
\end{pgfscope}%
\begin{pgfscope}%
\pgfsetbuttcap%
\pgfsetroundjoin%
\definecolor{currentfill}{rgb}{0.000000,0.000000,0.000000}%
\pgfsetfillcolor{currentfill}%
\pgfsetlinewidth{0.803000pt}%
\definecolor{currentstroke}{rgb}{0.000000,0.000000,0.000000}%
\pgfsetstrokecolor{currentstroke}%
\pgfsetdash{}{0pt}%
\pgfsys@defobject{currentmarker}{\pgfqpoint{0.000000in}{-0.048611in}}{\pgfqpoint{0.000000in}{0.000000in}}{%
\pgfpathmoveto{\pgfqpoint{0.000000in}{0.000000in}}%
\pgfpathlineto{\pgfqpoint{0.000000in}{-0.048611in}}%
\pgfusepath{stroke,fill}%
}%
\begin{pgfscope}%
\pgfsys@transformshift{2.763552in}{0.548769in}%
\pgfsys@useobject{currentmarker}{}%
\end{pgfscope}%
\end{pgfscope}%
\begin{pgfscope}%
\definecolor{textcolor}{rgb}{0.000000,0.000000,0.000000}%
\pgfsetstrokecolor{textcolor}%
\pgfsetfillcolor{textcolor}%
\pgftext[x=2.763552in,y=0.451547in,,top]{\color{textcolor}\sffamily\fontsize{10.000000}{12.000000}\selectfont \(\displaystyle {30}\)}%
\end{pgfscope}%
\begin{pgfscope}%
\pgfsetbuttcap%
\pgfsetroundjoin%
\definecolor{currentfill}{rgb}{0.000000,0.000000,0.000000}%
\pgfsetfillcolor{currentfill}%
\pgfsetlinewidth{0.803000pt}%
\definecolor{currentstroke}{rgb}{0.000000,0.000000,0.000000}%
\pgfsetstrokecolor{currentstroke}%
\pgfsetdash{}{0pt}%
\pgfsys@defobject{currentmarker}{\pgfqpoint{0.000000in}{-0.048611in}}{\pgfqpoint{0.000000in}{0.000000in}}{%
\pgfpathmoveto{\pgfqpoint{0.000000in}{0.000000in}}%
\pgfpathlineto{\pgfqpoint{0.000000in}{-0.048611in}}%
\pgfusepath{stroke,fill}%
}%
\begin{pgfscope}%
\pgfsys@transformshift{3.411285in}{0.548769in}%
\pgfsys@useobject{currentmarker}{}%
\end{pgfscope}%
\end{pgfscope}%
\begin{pgfscope}%
\definecolor{textcolor}{rgb}{0.000000,0.000000,0.000000}%
\pgfsetstrokecolor{textcolor}%
\pgfsetfillcolor{textcolor}%
\pgftext[x=3.411285in,y=0.451547in,,top]{\color{textcolor}\sffamily\fontsize{10.000000}{12.000000}\selectfont \(\displaystyle {40}\)}%
\end{pgfscope}%
\begin{pgfscope}%
\pgfsetbuttcap%
\pgfsetroundjoin%
\definecolor{currentfill}{rgb}{0.000000,0.000000,0.000000}%
\pgfsetfillcolor{currentfill}%
\pgfsetlinewidth{0.803000pt}%
\definecolor{currentstroke}{rgb}{0.000000,0.000000,0.000000}%
\pgfsetstrokecolor{currentstroke}%
\pgfsetdash{}{0pt}%
\pgfsys@defobject{currentmarker}{\pgfqpoint{0.000000in}{-0.048611in}}{\pgfqpoint{0.000000in}{0.000000in}}{%
\pgfpathmoveto{\pgfqpoint{0.000000in}{0.000000in}}%
\pgfpathlineto{\pgfqpoint{0.000000in}{-0.048611in}}%
\pgfusepath{stroke,fill}%
}%
\begin{pgfscope}%
\pgfsys@transformshift{4.059018in}{0.548769in}%
\pgfsys@useobject{currentmarker}{}%
\end{pgfscope}%
\end{pgfscope}%
\begin{pgfscope}%
\definecolor{textcolor}{rgb}{0.000000,0.000000,0.000000}%
\pgfsetstrokecolor{textcolor}%
\pgfsetfillcolor{textcolor}%
\pgftext[x=4.059018in,y=0.451547in,,top]{\color{textcolor}\sffamily\fontsize{10.000000}{12.000000}\selectfont \(\displaystyle {50}\)}%
\end{pgfscope}%
\begin{pgfscope}%
\pgfsetbuttcap%
\pgfsetroundjoin%
\definecolor{currentfill}{rgb}{0.000000,0.000000,0.000000}%
\pgfsetfillcolor{currentfill}%
\pgfsetlinewidth{0.803000pt}%
\definecolor{currentstroke}{rgb}{0.000000,0.000000,0.000000}%
\pgfsetstrokecolor{currentstroke}%
\pgfsetdash{}{0pt}%
\pgfsys@defobject{currentmarker}{\pgfqpoint{0.000000in}{-0.048611in}}{\pgfqpoint{0.000000in}{0.000000in}}{%
\pgfpathmoveto{\pgfqpoint{0.000000in}{0.000000in}}%
\pgfpathlineto{\pgfqpoint{0.000000in}{-0.048611in}}%
\pgfusepath{stroke,fill}%
}%
\begin{pgfscope}%
\pgfsys@transformshift{4.706751in}{0.548769in}%
\pgfsys@useobject{currentmarker}{}%
\end{pgfscope}%
\end{pgfscope}%
\begin{pgfscope}%
\definecolor{textcolor}{rgb}{0.000000,0.000000,0.000000}%
\pgfsetstrokecolor{textcolor}%
\pgfsetfillcolor{textcolor}%
\pgftext[x=4.706751in,y=0.451547in,,top]{\color{textcolor}\sffamily\fontsize{10.000000}{12.000000}\selectfont \(\displaystyle {60}\)}%
\end{pgfscope}%
\begin{pgfscope}%
\pgfsetbuttcap%
\pgfsetroundjoin%
\definecolor{currentfill}{rgb}{0.000000,0.000000,0.000000}%
\pgfsetfillcolor{currentfill}%
\pgfsetlinewidth{0.803000pt}%
\definecolor{currentstroke}{rgb}{0.000000,0.000000,0.000000}%
\pgfsetstrokecolor{currentstroke}%
\pgfsetdash{}{0pt}%
\pgfsys@defobject{currentmarker}{\pgfqpoint{0.000000in}{-0.048611in}}{\pgfqpoint{0.000000in}{0.000000in}}{%
\pgfpathmoveto{\pgfqpoint{0.000000in}{0.000000in}}%
\pgfpathlineto{\pgfqpoint{0.000000in}{-0.048611in}}%
\pgfusepath{stroke,fill}%
}%
\begin{pgfscope}%
\pgfsys@transformshift{5.354484in}{0.548769in}%
\pgfsys@useobject{currentmarker}{}%
\end{pgfscope}%
\end{pgfscope}%
\begin{pgfscope}%
\definecolor{textcolor}{rgb}{0.000000,0.000000,0.000000}%
\pgfsetstrokecolor{textcolor}%
\pgfsetfillcolor{textcolor}%
\pgftext[x=5.354484in,y=0.451547in,,top]{\color{textcolor}\sffamily\fontsize{10.000000}{12.000000}\selectfont \(\displaystyle {70}\)}%
\end{pgfscope}%
\begin{pgfscope}%
\definecolor{textcolor}{rgb}{0.000000,0.000000,0.000000}%
\pgfsetstrokecolor{textcolor}%
\pgfsetfillcolor{textcolor}%
\pgftext[x=3.249352in,y=0.272658in,,top]{\color{textcolor}\sffamily\fontsize{10.000000}{12.000000}\selectfont Number of Sinks}%
\end{pgfscope}%
\begin{pgfscope}%
\pgfsetbuttcap%
\pgfsetroundjoin%
\definecolor{currentfill}{rgb}{0.000000,0.000000,0.000000}%
\pgfsetfillcolor{currentfill}%
\pgfsetlinewidth{0.803000pt}%
\definecolor{currentstroke}{rgb}{0.000000,0.000000,0.000000}%
\pgfsetstrokecolor{currentstroke}%
\pgfsetdash{}{0pt}%
\pgfsys@defobject{currentmarker}{\pgfqpoint{-0.048611in}{0.000000in}}{\pgfqpoint{0.000000in}{0.000000in}}{%
\pgfpathmoveto{\pgfqpoint{0.000000in}{0.000000in}}%
\pgfpathlineto{\pgfqpoint{-0.048611in}{0.000000in}}%
\pgfusepath{stroke,fill}%
}%
\begin{pgfscope}%
\pgfsys@transformshift{0.648703in}{0.689796in}%
\pgfsys@useobject{currentmarker}{}%
\end{pgfscope}%
\end{pgfscope}%
\begin{pgfscope}%
\definecolor{textcolor}{rgb}{0.000000,0.000000,0.000000}%
\pgfsetstrokecolor{textcolor}%
\pgfsetfillcolor{textcolor}%
\pgftext[x=0.482036in, y=0.641601in, left, base]{\color{textcolor}\sffamily\fontsize{10.000000}{12.000000}\selectfont \(\displaystyle {0}\)}%
\end{pgfscope}%
\begin{pgfscope}%
\pgfsetbuttcap%
\pgfsetroundjoin%
\definecolor{currentfill}{rgb}{0.000000,0.000000,0.000000}%
\pgfsetfillcolor{currentfill}%
\pgfsetlinewidth{0.803000pt}%
\definecolor{currentstroke}{rgb}{0.000000,0.000000,0.000000}%
\pgfsetstrokecolor{currentstroke}%
\pgfsetdash{}{0pt}%
\pgfsys@defobject{currentmarker}{\pgfqpoint{-0.048611in}{0.000000in}}{\pgfqpoint{0.000000in}{0.000000in}}{%
\pgfpathmoveto{\pgfqpoint{0.000000in}{0.000000in}}%
\pgfpathlineto{\pgfqpoint{-0.048611in}{0.000000in}}%
\pgfusepath{stroke,fill}%
}%
\begin{pgfscope}%
\pgfsys@transformshift{0.648703in}{1.112665in}%
\pgfsys@useobject{currentmarker}{}%
\end{pgfscope}%
\end{pgfscope}%
\begin{pgfscope}%
\definecolor{textcolor}{rgb}{0.000000,0.000000,0.000000}%
\pgfsetstrokecolor{textcolor}%
\pgfsetfillcolor{textcolor}%
\pgftext[x=0.343147in, y=1.064470in, left, base]{\color{textcolor}\sffamily\fontsize{10.000000}{12.000000}\selectfont \(\displaystyle {100}\)}%
\end{pgfscope}%
\begin{pgfscope}%
\pgfsetbuttcap%
\pgfsetroundjoin%
\definecolor{currentfill}{rgb}{0.000000,0.000000,0.000000}%
\pgfsetfillcolor{currentfill}%
\pgfsetlinewidth{0.803000pt}%
\definecolor{currentstroke}{rgb}{0.000000,0.000000,0.000000}%
\pgfsetstrokecolor{currentstroke}%
\pgfsetdash{}{0pt}%
\pgfsys@defobject{currentmarker}{\pgfqpoint{-0.048611in}{0.000000in}}{\pgfqpoint{0.000000in}{0.000000in}}{%
\pgfpathmoveto{\pgfqpoint{0.000000in}{0.000000in}}%
\pgfpathlineto{\pgfqpoint{-0.048611in}{0.000000in}}%
\pgfusepath{stroke,fill}%
}%
\begin{pgfscope}%
\pgfsys@transformshift{0.648703in}{1.535534in}%
\pgfsys@useobject{currentmarker}{}%
\end{pgfscope}%
\end{pgfscope}%
\begin{pgfscope}%
\definecolor{textcolor}{rgb}{0.000000,0.000000,0.000000}%
\pgfsetstrokecolor{textcolor}%
\pgfsetfillcolor{textcolor}%
\pgftext[x=0.343147in, y=1.487339in, left, base]{\color{textcolor}\sffamily\fontsize{10.000000}{12.000000}\selectfont \(\displaystyle {200}\)}%
\end{pgfscope}%
\begin{pgfscope}%
\pgfsetbuttcap%
\pgfsetroundjoin%
\definecolor{currentfill}{rgb}{0.000000,0.000000,0.000000}%
\pgfsetfillcolor{currentfill}%
\pgfsetlinewidth{0.803000pt}%
\definecolor{currentstroke}{rgb}{0.000000,0.000000,0.000000}%
\pgfsetstrokecolor{currentstroke}%
\pgfsetdash{}{0pt}%
\pgfsys@defobject{currentmarker}{\pgfqpoint{-0.048611in}{0.000000in}}{\pgfqpoint{0.000000in}{0.000000in}}{%
\pgfpathmoveto{\pgfqpoint{0.000000in}{0.000000in}}%
\pgfpathlineto{\pgfqpoint{-0.048611in}{0.000000in}}%
\pgfusepath{stroke,fill}%
}%
\begin{pgfscope}%
\pgfsys@transformshift{0.648703in}{1.958403in}%
\pgfsys@useobject{currentmarker}{}%
\end{pgfscope}%
\end{pgfscope}%
\begin{pgfscope}%
\definecolor{textcolor}{rgb}{0.000000,0.000000,0.000000}%
\pgfsetstrokecolor{textcolor}%
\pgfsetfillcolor{textcolor}%
\pgftext[x=0.343147in, y=1.910208in, left, base]{\color{textcolor}\sffamily\fontsize{10.000000}{12.000000}\selectfont \(\displaystyle {300}\)}%
\end{pgfscope}%
\begin{pgfscope}%
\pgfsetbuttcap%
\pgfsetroundjoin%
\definecolor{currentfill}{rgb}{0.000000,0.000000,0.000000}%
\pgfsetfillcolor{currentfill}%
\pgfsetlinewidth{0.803000pt}%
\definecolor{currentstroke}{rgb}{0.000000,0.000000,0.000000}%
\pgfsetstrokecolor{currentstroke}%
\pgfsetdash{}{0pt}%
\pgfsys@defobject{currentmarker}{\pgfqpoint{-0.048611in}{0.000000in}}{\pgfqpoint{0.000000in}{0.000000in}}{%
\pgfpathmoveto{\pgfqpoint{0.000000in}{0.000000in}}%
\pgfpathlineto{\pgfqpoint{-0.048611in}{0.000000in}}%
\pgfusepath{stroke,fill}%
}%
\begin{pgfscope}%
\pgfsys@transformshift{0.648703in}{2.381272in}%
\pgfsys@useobject{currentmarker}{}%
\end{pgfscope}%
\end{pgfscope}%
\begin{pgfscope}%
\definecolor{textcolor}{rgb}{0.000000,0.000000,0.000000}%
\pgfsetstrokecolor{textcolor}%
\pgfsetfillcolor{textcolor}%
\pgftext[x=0.343147in, y=2.333077in, left, base]{\color{textcolor}\sffamily\fontsize{10.000000}{12.000000}\selectfont \(\displaystyle {400}\)}%
\end{pgfscope}%
\begin{pgfscope}%
\pgfsetbuttcap%
\pgfsetroundjoin%
\definecolor{currentfill}{rgb}{0.000000,0.000000,0.000000}%
\pgfsetfillcolor{currentfill}%
\pgfsetlinewidth{0.803000pt}%
\definecolor{currentstroke}{rgb}{0.000000,0.000000,0.000000}%
\pgfsetstrokecolor{currentstroke}%
\pgfsetdash{}{0pt}%
\pgfsys@defobject{currentmarker}{\pgfqpoint{-0.048611in}{0.000000in}}{\pgfqpoint{0.000000in}{0.000000in}}{%
\pgfpathmoveto{\pgfqpoint{0.000000in}{0.000000in}}%
\pgfpathlineto{\pgfqpoint{-0.048611in}{0.000000in}}%
\pgfusepath{stroke,fill}%
}%
\begin{pgfscope}%
\pgfsys@transformshift{0.648703in}{2.804141in}%
\pgfsys@useobject{currentmarker}{}%
\end{pgfscope}%
\end{pgfscope}%
\begin{pgfscope}%
\definecolor{textcolor}{rgb}{0.000000,0.000000,0.000000}%
\pgfsetstrokecolor{textcolor}%
\pgfsetfillcolor{textcolor}%
\pgftext[x=0.343147in, y=2.755946in, left, base]{\color{textcolor}\sffamily\fontsize{10.000000}{12.000000}\selectfont \(\displaystyle {500}\)}%
\end{pgfscope}%
\begin{pgfscope}%
\pgfsetbuttcap%
\pgfsetroundjoin%
\definecolor{currentfill}{rgb}{0.000000,0.000000,0.000000}%
\pgfsetfillcolor{currentfill}%
\pgfsetlinewidth{0.803000pt}%
\definecolor{currentstroke}{rgb}{0.000000,0.000000,0.000000}%
\pgfsetstrokecolor{currentstroke}%
\pgfsetdash{}{0pt}%
\pgfsys@defobject{currentmarker}{\pgfqpoint{-0.048611in}{0.000000in}}{\pgfqpoint{0.000000in}{0.000000in}}{%
\pgfpathmoveto{\pgfqpoint{0.000000in}{0.000000in}}%
\pgfpathlineto{\pgfqpoint{-0.048611in}{0.000000in}}%
\pgfusepath{stroke,fill}%
}%
\begin{pgfscope}%
\pgfsys@transformshift{0.648703in}{3.227010in}%
\pgfsys@useobject{currentmarker}{}%
\end{pgfscope}%
\end{pgfscope}%
\begin{pgfscope}%
\definecolor{textcolor}{rgb}{0.000000,0.000000,0.000000}%
\pgfsetstrokecolor{textcolor}%
\pgfsetfillcolor{textcolor}%
\pgftext[x=0.343147in, y=3.178815in, left, base]{\color{textcolor}\sffamily\fontsize{10.000000}{12.000000}\selectfont \(\displaystyle {600}\)}%
\end{pgfscope}%
\begin{pgfscope}%
\pgfsetbuttcap%
\pgfsetroundjoin%
\definecolor{currentfill}{rgb}{0.000000,0.000000,0.000000}%
\pgfsetfillcolor{currentfill}%
\pgfsetlinewidth{0.803000pt}%
\definecolor{currentstroke}{rgb}{0.000000,0.000000,0.000000}%
\pgfsetstrokecolor{currentstroke}%
\pgfsetdash{}{0pt}%
\pgfsys@defobject{currentmarker}{\pgfqpoint{-0.048611in}{0.000000in}}{\pgfqpoint{0.000000in}{0.000000in}}{%
\pgfpathmoveto{\pgfqpoint{0.000000in}{0.000000in}}%
\pgfpathlineto{\pgfqpoint{-0.048611in}{0.000000in}}%
\pgfusepath{stroke,fill}%
}%
\begin{pgfscope}%
\pgfsys@transformshift{0.648703in}{3.649879in}%
\pgfsys@useobject{currentmarker}{}%
\end{pgfscope}%
\end{pgfscope}%
\begin{pgfscope}%
\definecolor{textcolor}{rgb}{0.000000,0.000000,0.000000}%
\pgfsetstrokecolor{textcolor}%
\pgfsetfillcolor{textcolor}%
\pgftext[x=0.343147in, y=3.601684in, left, base]{\color{textcolor}\sffamily\fontsize{10.000000}{12.000000}\selectfont \(\displaystyle {700}\)}%
\end{pgfscope}%
\begin{pgfscope}%
\definecolor{textcolor}{rgb}{0.000000,0.000000,0.000000}%
\pgfsetstrokecolor{textcolor}%
\pgfsetfillcolor{textcolor}%
\pgftext[x=0.287592in,y=2.100064in,,bottom,rotate=90.000000]{\color{textcolor}\sffamily\fontsize{10.000000}{12.000000}\selectfont Data Flow Time (s)}%
\end{pgfscope}%
\begin{pgfscope}%
\pgfsetrectcap%
\pgfsetmiterjoin%
\pgfsetlinewidth{0.803000pt}%
\definecolor{currentstroke}{rgb}{0.000000,0.000000,0.000000}%
\pgfsetstrokecolor{currentstroke}%
\pgfsetdash{}{0pt}%
\pgfpathmoveto{\pgfqpoint{0.648703in}{0.548769in}}%
\pgfpathlineto{\pgfqpoint{0.648703in}{3.651359in}}%
\pgfusepath{stroke}%
\end{pgfscope}%
\begin{pgfscope}%
\pgfsetrectcap%
\pgfsetmiterjoin%
\pgfsetlinewidth{0.803000pt}%
\definecolor{currentstroke}{rgb}{0.000000,0.000000,0.000000}%
\pgfsetstrokecolor{currentstroke}%
\pgfsetdash{}{0pt}%
\pgfpathmoveto{\pgfqpoint{5.850000in}{0.548769in}}%
\pgfpathlineto{\pgfqpoint{5.850000in}{3.651359in}}%
\pgfusepath{stroke}%
\end{pgfscope}%
\begin{pgfscope}%
\pgfsetrectcap%
\pgfsetmiterjoin%
\pgfsetlinewidth{0.803000pt}%
\definecolor{currentstroke}{rgb}{0.000000,0.000000,0.000000}%
\pgfsetstrokecolor{currentstroke}%
\pgfsetdash{}{0pt}%
\pgfpathmoveto{\pgfqpoint{0.648703in}{0.548769in}}%
\pgfpathlineto{\pgfqpoint{5.850000in}{0.548769in}}%
\pgfusepath{stroke}%
\end{pgfscope}%
\begin{pgfscope}%
\pgfsetrectcap%
\pgfsetmiterjoin%
\pgfsetlinewidth{0.803000pt}%
\definecolor{currentstroke}{rgb}{0.000000,0.000000,0.000000}%
\pgfsetstrokecolor{currentstroke}%
\pgfsetdash{}{0pt}%
\pgfpathmoveto{\pgfqpoint{0.648703in}{3.651359in}}%
\pgfpathlineto{\pgfqpoint{5.850000in}{3.651359in}}%
\pgfusepath{stroke}%
\end{pgfscope}%
\begin{pgfscope}%
\definecolor{textcolor}{rgb}{0.000000,0.000000,0.000000}%
\pgfsetstrokecolor{textcolor}%
\pgfsetfillcolor{textcolor}%
\pgftext[x=3.249352in,y=3.734692in,,base]{\color{textcolor}\sffamily\fontsize{12.000000}{14.400000}\selectfont Backwards}%
\end{pgfscope}%
\begin{pgfscope}%
\pgfsetbuttcap%
\pgfsetmiterjoin%
\definecolor{currentfill}{rgb}{1.000000,1.000000,1.000000}%
\pgfsetfillcolor{currentfill}%
\pgfsetfillopacity{0.800000}%
\pgfsetlinewidth{1.003750pt}%
\definecolor{currentstroke}{rgb}{0.800000,0.800000,0.800000}%
\pgfsetstrokecolor{currentstroke}%
\pgfsetstrokeopacity{0.800000}%
\pgfsetdash{}{0pt}%
\pgfpathmoveto{\pgfqpoint{4.300417in}{0.618213in}}%
\pgfpathlineto{\pgfqpoint{5.752778in}{0.618213in}}%
\pgfpathquadraticcurveto{\pgfqpoint{5.780556in}{0.618213in}}{\pgfqpoint{5.780556in}{0.645991in}}%
\pgfpathlineto{\pgfqpoint{5.780556in}{1.214463in}}%
\pgfpathquadraticcurveto{\pgfqpoint{5.780556in}{1.242241in}}{\pgfqpoint{5.752778in}{1.242241in}}%
\pgfpathlineto{\pgfqpoint{4.300417in}{1.242241in}}%
\pgfpathquadraticcurveto{\pgfqpoint{4.272639in}{1.242241in}}{\pgfqpoint{4.272639in}{1.214463in}}%
\pgfpathlineto{\pgfqpoint{4.272639in}{0.645991in}}%
\pgfpathquadraticcurveto{\pgfqpoint{4.272639in}{0.618213in}}{\pgfqpoint{4.300417in}{0.618213in}}%
\pgfpathclose%
\pgfusepath{stroke,fill}%
\end{pgfscope}%
\begin{pgfscope}%
\pgfsetbuttcap%
\pgfsetroundjoin%
\definecolor{currentfill}{rgb}{0.121569,0.466667,0.705882}%
\pgfsetfillcolor{currentfill}%
\pgfsetlinewidth{1.003750pt}%
\definecolor{currentstroke}{rgb}{0.121569,0.466667,0.705882}%
\pgfsetstrokecolor{currentstroke}%
\pgfsetdash{}{0pt}%
\pgfsys@defobject{currentmarker}{\pgfqpoint{-0.034722in}{-0.034722in}}{\pgfqpoint{0.034722in}{0.034722in}}{%
\pgfpathmoveto{\pgfqpoint{0.000000in}{-0.034722in}}%
\pgfpathcurveto{\pgfqpoint{0.009208in}{-0.034722in}}{\pgfqpoint{0.018041in}{-0.031064in}}{\pgfqpoint{0.024552in}{-0.024552in}}%
\pgfpathcurveto{\pgfqpoint{0.031064in}{-0.018041in}}{\pgfqpoint{0.034722in}{-0.009208in}}{\pgfqpoint{0.034722in}{0.000000in}}%
\pgfpathcurveto{\pgfqpoint{0.034722in}{0.009208in}}{\pgfqpoint{0.031064in}{0.018041in}}{\pgfqpoint{0.024552in}{0.024552in}}%
\pgfpathcurveto{\pgfqpoint{0.018041in}{0.031064in}}{\pgfqpoint{0.009208in}{0.034722in}}{\pgfqpoint{0.000000in}{0.034722in}}%
\pgfpathcurveto{\pgfqpoint{-0.009208in}{0.034722in}}{\pgfqpoint{-0.018041in}{0.031064in}}{\pgfqpoint{-0.024552in}{0.024552in}}%
\pgfpathcurveto{\pgfqpoint{-0.031064in}{0.018041in}}{\pgfqpoint{-0.034722in}{0.009208in}}{\pgfqpoint{-0.034722in}{0.000000in}}%
\pgfpathcurveto{\pgfqpoint{-0.034722in}{-0.009208in}}{\pgfqpoint{-0.031064in}{-0.018041in}}{\pgfqpoint{-0.024552in}{-0.024552in}}%
\pgfpathcurveto{\pgfqpoint{-0.018041in}{-0.031064in}}{\pgfqpoint{-0.009208in}{-0.034722in}}{\pgfqpoint{0.000000in}{-0.034722in}}%
\pgfpathclose%
\pgfusepath{stroke,fill}%
}%
\begin{pgfscope}%
\pgfsys@transformshift{4.467083in}{1.138074in}%
\pgfsys@useobject{currentmarker}{}%
\end{pgfscope}%
\end{pgfscope}%
\begin{pgfscope}%
\definecolor{textcolor}{rgb}{0.000000,0.000000,0.000000}%
\pgfsetstrokecolor{textcolor}%
\pgfsetfillcolor{textcolor}%
\pgftext[x=4.717083in,y=1.089463in,left,base]{\color{textcolor}\sffamily\fontsize{10.000000}{12.000000}\selectfont No Timeout}%
\end{pgfscope}%
\begin{pgfscope}%
\pgfsetbuttcap%
\pgfsetroundjoin%
\definecolor{currentfill}{rgb}{1.000000,0.498039,0.054902}%
\pgfsetfillcolor{currentfill}%
\pgfsetlinewidth{1.003750pt}%
\definecolor{currentstroke}{rgb}{1.000000,0.498039,0.054902}%
\pgfsetstrokecolor{currentstroke}%
\pgfsetdash{}{0pt}%
\pgfsys@defobject{currentmarker}{\pgfqpoint{-0.034722in}{-0.034722in}}{\pgfqpoint{0.034722in}{0.034722in}}{%
\pgfpathmoveto{\pgfqpoint{0.000000in}{-0.034722in}}%
\pgfpathcurveto{\pgfqpoint{0.009208in}{-0.034722in}}{\pgfqpoint{0.018041in}{-0.031064in}}{\pgfqpoint{0.024552in}{-0.024552in}}%
\pgfpathcurveto{\pgfqpoint{0.031064in}{-0.018041in}}{\pgfqpoint{0.034722in}{-0.009208in}}{\pgfqpoint{0.034722in}{0.000000in}}%
\pgfpathcurveto{\pgfqpoint{0.034722in}{0.009208in}}{\pgfqpoint{0.031064in}{0.018041in}}{\pgfqpoint{0.024552in}{0.024552in}}%
\pgfpathcurveto{\pgfqpoint{0.018041in}{0.031064in}}{\pgfqpoint{0.009208in}{0.034722in}}{\pgfqpoint{0.000000in}{0.034722in}}%
\pgfpathcurveto{\pgfqpoint{-0.009208in}{0.034722in}}{\pgfqpoint{-0.018041in}{0.031064in}}{\pgfqpoint{-0.024552in}{0.024552in}}%
\pgfpathcurveto{\pgfqpoint{-0.031064in}{0.018041in}}{\pgfqpoint{-0.034722in}{0.009208in}}{\pgfqpoint{-0.034722in}{0.000000in}}%
\pgfpathcurveto{\pgfqpoint{-0.034722in}{-0.009208in}}{\pgfqpoint{-0.031064in}{-0.018041in}}{\pgfqpoint{-0.024552in}{-0.024552in}}%
\pgfpathcurveto{\pgfqpoint{-0.018041in}{-0.031064in}}{\pgfqpoint{-0.009208in}{-0.034722in}}{\pgfqpoint{0.000000in}{-0.034722in}}%
\pgfpathclose%
\pgfusepath{stroke,fill}%
}%
\begin{pgfscope}%
\pgfsys@transformshift{4.467083in}{0.944463in}%
\pgfsys@useobject{currentmarker}{}%
\end{pgfscope}%
\end{pgfscope}%
\begin{pgfscope}%
\definecolor{textcolor}{rgb}{0.000000,0.000000,0.000000}%
\pgfsetstrokecolor{textcolor}%
\pgfsetfillcolor{textcolor}%
\pgftext[x=4.717083in,y=0.895852in,left,base]{\color{textcolor}\sffamily\fontsize{10.000000}{12.000000}\selectfont Time Timeout}%
\end{pgfscope}%
\begin{pgfscope}%
\pgfsetbuttcap%
\pgfsetroundjoin%
\definecolor{currentfill}{rgb}{0.839216,0.152941,0.156863}%
\pgfsetfillcolor{currentfill}%
\pgfsetlinewidth{1.003750pt}%
\definecolor{currentstroke}{rgb}{0.839216,0.152941,0.156863}%
\pgfsetstrokecolor{currentstroke}%
\pgfsetdash{}{0pt}%
\pgfsys@defobject{currentmarker}{\pgfqpoint{-0.034722in}{-0.034722in}}{\pgfqpoint{0.034722in}{0.034722in}}{%
\pgfpathmoveto{\pgfqpoint{0.000000in}{-0.034722in}}%
\pgfpathcurveto{\pgfqpoint{0.009208in}{-0.034722in}}{\pgfqpoint{0.018041in}{-0.031064in}}{\pgfqpoint{0.024552in}{-0.024552in}}%
\pgfpathcurveto{\pgfqpoint{0.031064in}{-0.018041in}}{\pgfqpoint{0.034722in}{-0.009208in}}{\pgfqpoint{0.034722in}{0.000000in}}%
\pgfpathcurveto{\pgfqpoint{0.034722in}{0.009208in}}{\pgfqpoint{0.031064in}{0.018041in}}{\pgfqpoint{0.024552in}{0.024552in}}%
\pgfpathcurveto{\pgfqpoint{0.018041in}{0.031064in}}{\pgfqpoint{0.009208in}{0.034722in}}{\pgfqpoint{0.000000in}{0.034722in}}%
\pgfpathcurveto{\pgfqpoint{-0.009208in}{0.034722in}}{\pgfqpoint{-0.018041in}{0.031064in}}{\pgfqpoint{-0.024552in}{0.024552in}}%
\pgfpathcurveto{\pgfqpoint{-0.031064in}{0.018041in}}{\pgfqpoint{-0.034722in}{0.009208in}}{\pgfqpoint{-0.034722in}{0.000000in}}%
\pgfpathcurveto{\pgfqpoint{-0.034722in}{-0.009208in}}{\pgfqpoint{-0.031064in}{-0.018041in}}{\pgfqpoint{-0.024552in}{-0.024552in}}%
\pgfpathcurveto{\pgfqpoint{-0.018041in}{-0.031064in}}{\pgfqpoint{-0.009208in}{-0.034722in}}{\pgfqpoint{0.000000in}{-0.034722in}}%
\pgfpathclose%
\pgfusepath{stroke,fill}%
}%
\begin{pgfscope}%
\pgfsys@transformshift{4.467083in}{0.750852in}%
\pgfsys@useobject{currentmarker}{}%
\end{pgfscope}%
\end{pgfscope}%
\begin{pgfscope}%
\definecolor{textcolor}{rgb}{0.000000,0.000000,0.000000}%
\pgfsetstrokecolor{textcolor}%
\pgfsetfillcolor{textcolor}%
\pgftext[x=4.717083in,y=0.702241in,left,base]{\color{textcolor}\sffamily\fontsize{10.000000}{12.000000}\selectfont Memory Timeout}%
\end{pgfscope}%
\end{pgfpicture}%
\makeatother%
\endgroup%

            }
        \end{subfigure}
        \caption{Data flow time in comparison to source and sink count}
    \end{figure}

    \begin{figure}
        \centering
        \resizebox{0.75\columnwidth}{!}{
            %% Creator: Matplotlib, PGF backend
%%
%% To include the figure in your LaTeX document, write
%%   \input{<filename>.pgf}
%%
%% Make sure the required packages are loaded in your preamble
%%   \usepackage{pgf}
%%
%% and, on pdftex
%%   \usepackage[utf8]{inputenc}\DeclareUnicodeCharacter{2212}{-}
%%
%% or, on luatex and xetex
%%   \usepackage{unicode-math}
%%
%% Figures using additional raster images can only be included by \input if
%% they are in the same directory as the main LaTeX file. For loading figures
%% from other directories you can use the `import` package
%%   \usepackage{import}
%%
%% and then include the figures with
%%   \import{<path to file>}{<filename>.pgf}
%%
%% Matplotlib used the following preamble
%%   \usepackage{amsmath}
%%   \usepackage{fontspec}
%%
\begingroup%
\makeatletter%
\begin{pgfpicture}%
\pgfpathrectangle{\pgfpointorigin}{\pgfqpoint{6.000000in}{4.000000in}}%
\pgfusepath{use as bounding box, clip}%
\begin{pgfscope}%
\pgfsetbuttcap%
\pgfsetmiterjoin%
\definecolor{currentfill}{rgb}{1.000000,1.000000,1.000000}%
\pgfsetfillcolor{currentfill}%
\pgfsetlinewidth{0.000000pt}%
\definecolor{currentstroke}{rgb}{1.000000,1.000000,1.000000}%
\pgfsetstrokecolor{currentstroke}%
\pgfsetdash{}{0pt}%
\pgfpathmoveto{\pgfqpoint{0.000000in}{0.000000in}}%
\pgfpathlineto{\pgfqpoint{6.000000in}{0.000000in}}%
\pgfpathlineto{\pgfqpoint{6.000000in}{4.000000in}}%
\pgfpathlineto{\pgfqpoint{0.000000in}{4.000000in}}%
\pgfpathclose%
\pgfusepath{fill}%
\end{pgfscope}%
\begin{pgfscope}%
\pgfsetbuttcap%
\pgfsetmiterjoin%
\definecolor{currentfill}{rgb}{1.000000,1.000000,1.000000}%
\pgfsetfillcolor{currentfill}%
\pgfsetlinewidth{0.000000pt}%
\definecolor{currentstroke}{rgb}{0.000000,0.000000,0.000000}%
\pgfsetstrokecolor{currentstroke}%
\pgfsetstrokeopacity{0.000000}%
\pgfsetdash{}{0pt}%
\pgfpathmoveto{\pgfqpoint{0.750000in}{0.500000in}}%
\pgfpathlineto{\pgfqpoint{5.400000in}{0.500000in}}%
\pgfpathlineto{\pgfqpoint{5.400000in}{3.520000in}}%
\pgfpathlineto{\pgfqpoint{0.750000in}{3.520000in}}%
\pgfpathclose%
\pgfusepath{fill}%
\end{pgfscope}%
\begin{pgfscope}%
\pgfpathrectangle{\pgfqpoint{0.750000in}{0.500000in}}{\pgfqpoint{4.650000in}{3.020000in}}%
\pgfusepath{clip}%
\pgfsetbuttcap%
\pgfsetmiterjoin%
\definecolor{currentfill}{rgb}{0.121569,0.466667,0.705882}%
\pgfsetfillcolor{currentfill}%
\pgfsetlinewidth{1.003750pt}%
\definecolor{currentstroke}{rgb}{0.000000,0.000000,0.000000}%
\pgfsetstrokecolor{currentstroke}%
\pgfsetdash{}{0pt}%
\pgfpathmoveto{\pgfqpoint{0.961364in}{0.500000in}}%
\pgfpathlineto{\pgfqpoint{1.130455in}{0.500000in}}%
\pgfpathlineto{\pgfqpoint{1.130455in}{1.092157in}}%
\pgfpathlineto{\pgfqpoint{0.961364in}{1.092157in}}%
\pgfpathclose%
\pgfusepath{stroke,fill}%
\end{pgfscope}%
\begin{pgfscope}%
\pgfpathrectangle{\pgfqpoint{0.750000in}{0.500000in}}{\pgfqpoint{4.650000in}{3.020000in}}%
\pgfusepath{clip}%
\pgfsetbuttcap%
\pgfsetmiterjoin%
\definecolor{currentfill}{rgb}{0.121569,0.466667,0.705882}%
\pgfsetfillcolor{currentfill}%
\pgfsetlinewidth{1.003750pt}%
\definecolor{currentstroke}{rgb}{0.000000,0.000000,0.000000}%
\pgfsetstrokecolor{currentstroke}%
\pgfsetdash{}{0pt}%
\pgfpathmoveto{\pgfqpoint{1.130455in}{0.500000in}}%
\pgfpathlineto{\pgfqpoint{1.299545in}{0.500000in}}%
\pgfpathlineto{\pgfqpoint{1.299545in}{1.049860in}}%
\pgfpathlineto{\pgfqpoint{1.130455in}{1.049860in}}%
\pgfpathclose%
\pgfusepath{stroke,fill}%
\end{pgfscope}%
\begin{pgfscope}%
\pgfpathrectangle{\pgfqpoint{0.750000in}{0.500000in}}{\pgfqpoint{4.650000in}{3.020000in}}%
\pgfusepath{clip}%
\pgfsetbuttcap%
\pgfsetmiterjoin%
\definecolor{currentfill}{rgb}{0.121569,0.466667,0.705882}%
\pgfsetfillcolor{currentfill}%
\pgfsetlinewidth{1.003750pt}%
\definecolor{currentstroke}{rgb}{0.000000,0.000000,0.000000}%
\pgfsetstrokecolor{currentstroke}%
\pgfsetdash{}{0pt}%
\pgfpathmoveto{\pgfqpoint{1.299545in}{0.500000in}}%
\pgfpathlineto{\pgfqpoint{1.468636in}{0.500000in}}%
\pgfpathlineto{\pgfqpoint{1.468636in}{0.542297in}}%
\pgfpathlineto{\pgfqpoint{1.299545in}{0.542297in}}%
\pgfpathclose%
\pgfusepath{stroke,fill}%
\end{pgfscope}%
\begin{pgfscope}%
\pgfpathrectangle{\pgfqpoint{0.750000in}{0.500000in}}{\pgfqpoint{4.650000in}{3.020000in}}%
\pgfusepath{clip}%
\pgfsetbuttcap%
\pgfsetmiterjoin%
\definecolor{currentfill}{rgb}{0.121569,0.466667,0.705882}%
\pgfsetfillcolor{currentfill}%
\pgfsetlinewidth{1.003750pt}%
\definecolor{currentstroke}{rgb}{0.000000,0.000000,0.000000}%
\pgfsetstrokecolor{currentstroke}%
\pgfsetdash{}{0pt}%
\pgfpathmoveto{\pgfqpoint{1.468636in}{0.500000in}}%
\pgfpathlineto{\pgfqpoint{1.637727in}{0.500000in}}%
\pgfpathlineto{\pgfqpoint{1.637727in}{0.584594in}}%
\pgfpathlineto{\pgfqpoint{1.468636in}{0.584594in}}%
\pgfpathclose%
\pgfusepath{stroke,fill}%
\end{pgfscope}%
\begin{pgfscope}%
\pgfpathrectangle{\pgfqpoint{0.750000in}{0.500000in}}{\pgfqpoint{4.650000in}{3.020000in}}%
\pgfusepath{clip}%
\pgfsetbuttcap%
\pgfsetmiterjoin%
\definecolor{currentfill}{rgb}{0.121569,0.466667,0.705882}%
\pgfsetfillcolor{currentfill}%
\pgfsetlinewidth{1.003750pt}%
\definecolor{currentstroke}{rgb}{0.000000,0.000000,0.000000}%
\pgfsetstrokecolor{currentstroke}%
\pgfsetdash{}{0pt}%
\pgfpathmoveto{\pgfqpoint{1.637727in}{0.500000in}}%
\pgfpathlineto{\pgfqpoint{1.806818in}{0.500000in}}%
\pgfpathlineto{\pgfqpoint{1.806818in}{0.542297in}}%
\pgfpathlineto{\pgfqpoint{1.637727in}{0.542297in}}%
\pgfpathclose%
\pgfusepath{stroke,fill}%
\end{pgfscope}%
\begin{pgfscope}%
\pgfpathrectangle{\pgfqpoint{0.750000in}{0.500000in}}{\pgfqpoint{4.650000in}{3.020000in}}%
\pgfusepath{clip}%
\pgfsetbuttcap%
\pgfsetmiterjoin%
\definecolor{currentfill}{rgb}{0.121569,0.466667,0.705882}%
\pgfsetfillcolor{currentfill}%
\pgfsetlinewidth{1.003750pt}%
\definecolor{currentstroke}{rgb}{0.000000,0.000000,0.000000}%
\pgfsetstrokecolor{currentstroke}%
\pgfsetdash{}{0pt}%
\pgfpathmoveto{\pgfqpoint{1.806818in}{0.500000in}}%
\pgfpathlineto{\pgfqpoint{1.975909in}{0.500000in}}%
\pgfpathlineto{\pgfqpoint{1.975909in}{0.500000in}}%
\pgfpathlineto{\pgfqpoint{1.806818in}{0.500000in}}%
\pgfpathclose%
\pgfusepath{stroke,fill}%
\end{pgfscope}%
\begin{pgfscope}%
\pgfpathrectangle{\pgfqpoint{0.750000in}{0.500000in}}{\pgfqpoint{4.650000in}{3.020000in}}%
\pgfusepath{clip}%
\pgfsetbuttcap%
\pgfsetmiterjoin%
\definecolor{currentfill}{rgb}{0.121569,0.466667,0.705882}%
\pgfsetfillcolor{currentfill}%
\pgfsetlinewidth{1.003750pt}%
\definecolor{currentstroke}{rgb}{0.000000,0.000000,0.000000}%
\pgfsetstrokecolor{currentstroke}%
\pgfsetdash{}{0pt}%
\pgfpathmoveto{\pgfqpoint{1.975909in}{0.500000in}}%
\pgfpathlineto{\pgfqpoint{2.145000in}{0.500000in}}%
\pgfpathlineto{\pgfqpoint{2.145000in}{0.500000in}}%
\pgfpathlineto{\pgfqpoint{1.975909in}{0.500000in}}%
\pgfpathclose%
\pgfusepath{stroke,fill}%
\end{pgfscope}%
\begin{pgfscope}%
\pgfpathrectangle{\pgfqpoint{0.750000in}{0.500000in}}{\pgfqpoint{4.650000in}{3.020000in}}%
\pgfusepath{clip}%
\pgfsetbuttcap%
\pgfsetmiterjoin%
\definecolor{currentfill}{rgb}{0.121569,0.466667,0.705882}%
\pgfsetfillcolor{currentfill}%
\pgfsetlinewidth{1.003750pt}%
\definecolor{currentstroke}{rgb}{0.000000,0.000000,0.000000}%
\pgfsetstrokecolor{currentstroke}%
\pgfsetdash{}{0pt}%
\pgfpathmoveto{\pgfqpoint{2.145000in}{0.500000in}}%
\pgfpathlineto{\pgfqpoint{2.314091in}{0.500000in}}%
\pgfpathlineto{\pgfqpoint{2.314091in}{0.500000in}}%
\pgfpathlineto{\pgfqpoint{2.145000in}{0.500000in}}%
\pgfpathclose%
\pgfusepath{stroke,fill}%
\end{pgfscope}%
\begin{pgfscope}%
\pgfpathrectangle{\pgfqpoint{0.750000in}{0.500000in}}{\pgfqpoint{4.650000in}{3.020000in}}%
\pgfusepath{clip}%
\pgfsetbuttcap%
\pgfsetmiterjoin%
\definecolor{currentfill}{rgb}{0.121569,0.466667,0.705882}%
\pgfsetfillcolor{currentfill}%
\pgfsetlinewidth{1.003750pt}%
\definecolor{currentstroke}{rgb}{0.000000,0.000000,0.000000}%
\pgfsetstrokecolor{currentstroke}%
\pgfsetdash{}{0pt}%
\pgfpathmoveto{\pgfqpoint{2.314091in}{0.500000in}}%
\pgfpathlineto{\pgfqpoint{2.483182in}{0.500000in}}%
\pgfpathlineto{\pgfqpoint{2.483182in}{0.500000in}}%
\pgfpathlineto{\pgfqpoint{2.314091in}{0.500000in}}%
\pgfpathclose%
\pgfusepath{stroke,fill}%
\end{pgfscope}%
\begin{pgfscope}%
\pgfpathrectangle{\pgfqpoint{0.750000in}{0.500000in}}{\pgfqpoint{4.650000in}{3.020000in}}%
\pgfusepath{clip}%
\pgfsetbuttcap%
\pgfsetmiterjoin%
\definecolor{currentfill}{rgb}{0.121569,0.466667,0.705882}%
\pgfsetfillcolor{currentfill}%
\pgfsetlinewidth{1.003750pt}%
\definecolor{currentstroke}{rgb}{0.000000,0.000000,0.000000}%
\pgfsetstrokecolor{currentstroke}%
\pgfsetdash{}{0pt}%
\pgfpathmoveto{\pgfqpoint{2.483182in}{0.500000in}}%
\pgfpathlineto{\pgfqpoint{2.652273in}{0.500000in}}%
\pgfpathlineto{\pgfqpoint{2.652273in}{0.500000in}}%
\pgfpathlineto{\pgfqpoint{2.483182in}{0.500000in}}%
\pgfpathclose%
\pgfusepath{stroke,fill}%
\end{pgfscope}%
\begin{pgfscope}%
\pgfpathrectangle{\pgfqpoint{0.750000in}{0.500000in}}{\pgfqpoint{4.650000in}{3.020000in}}%
\pgfusepath{clip}%
\pgfsetbuttcap%
\pgfsetmiterjoin%
\definecolor{currentfill}{rgb}{0.121569,0.466667,0.705882}%
\pgfsetfillcolor{currentfill}%
\pgfsetlinewidth{1.003750pt}%
\definecolor{currentstroke}{rgb}{0.000000,0.000000,0.000000}%
\pgfsetstrokecolor{currentstroke}%
\pgfsetdash{}{0pt}%
\pgfpathmoveto{\pgfqpoint{2.652273in}{0.500000in}}%
\pgfpathlineto{\pgfqpoint{2.821364in}{0.500000in}}%
\pgfpathlineto{\pgfqpoint{2.821364in}{0.542297in}}%
\pgfpathlineto{\pgfqpoint{2.652273in}{0.542297in}}%
\pgfpathclose%
\pgfusepath{stroke,fill}%
\end{pgfscope}%
\begin{pgfscope}%
\pgfpathrectangle{\pgfqpoint{0.750000in}{0.500000in}}{\pgfqpoint{4.650000in}{3.020000in}}%
\pgfusepath{clip}%
\pgfsetbuttcap%
\pgfsetmiterjoin%
\definecolor{currentfill}{rgb}{0.121569,0.466667,0.705882}%
\pgfsetfillcolor{currentfill}%
\pgfsetlinewidth{1.003750pt}%
\definecolor{currentstroke}{rgb}{0.000000,0.000000,0.000000}%
\pgfsetstrokecolor{currentstroke}%
\pgfsetdash{}{0pt}%
\pgfpathmoveto{\pgfqpoint{2.821364in}{0.500000in}}%
\pgfpathlineto{\pgfqpoint{2.990455in}{0.500000in}}%
\pgfpathlineto{\pgfqpoint{2.990455in}{0.626891in}}%
\pgfpathlineto{\pgfqpoint{2.821364in}{0.626891in}}%
\pgfpathclose%
\pgfusepath{stroke,fill}%
\end{pgfscope}%
\begin{pgfscope}%
\pgfpathrectangle{\pgfqpoint{0.750000in}{0.500000in}}{\pgfqpoint{4.650000in}{3.020000in}}%
\pgfusepath{clip}%
\pgfsetbuttcap%
\pgfsetmiterjoin%
\definecolor{currentfill}{rgb}{0.121569,0.466667,0.705882}%
\pgfsetfillcolor{currentfill}%
\pgfsetlinewidth{1.003750pt}%
\definecolor{currentstroke}{rgb}{0.000000,0.000000,0.000000}%
\pgfsetstrokecolor{currentstroke}%
\pgfsetdash{}{0pt}%
\pgfpathmoveto{\pgfqpoint{2.990455in}{0.500000in}}%
\pgfpathlineto{\pgfqpoint{3.159545in}{0.500000in}}%
\pgfpathlineto{\pgfqpoint{3.159545in}{1.261345in}}%
\pgfpathlineto{\pgfqpoint{2.990455in}{1.261345in}}%
\pgfpathclose%
\pgfusepath{stroke,fill}%
\end{pgfscope}%
\begin{pgfscope}%
\pgfpathrectangle{\pgfqpoint{0.750000in}{0.500000in}}{\pgfqpoint{4.650000in}{3.020000in}}%
\pgfusepath{clip}%
\pgfsetbuttcap%
\pgfsetmiterjoin%
\definecolor{currentfill}{rgb}{0.121569,0.466667,0.705882}%
\pgfsetfillcolor{currentfill}%
\pgfsetlinewidth{1.003750pt}%
\definecolor{currentstroke}{rgb}{0.000000,0.000000,0.000000}%
\pgfsetstrokecolor{currentstroke}%
\pgfsetdash{}{0pt}%
\pgfpathmoveto{\pgfqpoint{3.159545in}{0.500000in}}%
\pgfpathlineto{\pgfqpoint{3.328636in}{0.500000in}}%
\pgfpathlineto{\pgfqpoint{3.328636in}{3.376190in}}%
\pgfpathlineto{\pgfqpoint{3.159545in}{3.376190in}}%
\pgfpathclose%
\pgfusepath{stroke,fill}%
\end{pgfscope}%
\begin{pgfscope}%
\pgfpathrectangle{\pgfqpoint{0.750000in}{0.500000in}}{\pgfqpoint{4.650000in}{3.020000in}}%
\pgfusepath{clip}%
\pgfsetbuttcap%
\pgfsetmiterjoin%
\definecolor{currentfill}{rgb}{0.121569,0.466667,0.705882}%
\pgfsetfillcolor{currentfill}%
\pgfsetlinewidth{1.003750pt}%
\definecolor{currentstroke}{rgb}{0.000000,0.000000,0.000000}%
\pgfsetstrokecolor{currentstroke}%
\pgfsetdash{}{0pt}%
\pgfpathmoveto{\pgfqpoint{3.328636in}{0.500000in}}%
\pgfpathlineto{\pgfqpoint{3.497727in}{0.500000in}}%
\pgfpathlineto{\pgfqpoint{3.497727in}{0.669188in}}%
\pgfpathlineto{\pgfqpoint{3.328636in}{0.669188in}}%
\pgfpathclose%
\pgfusepath{stroke,fill}%
\end{pgfscope}%
\begin{pgfscope}%
\pgfpathrectangle{\pgfqpoint{0.750000in}{0.500000in}}{\pgfqpoint{4.650000in}{3.020000in}}%
\pgfusepath{clip}%
\pgfsetbuttcap%
\pgfsetmiterjoin%
\definecolor{currentfill}{rgb}{0.121569,0.466667,0.705882}%
\pgfsetfillcolor{currentfill}%
\pgfsetlinewidth{1.003750pt}%
\definecolor{currentstroke}{rgb}{0.000000,0.000000,0.000000}%
\pgfsetstrokecolor{currentstroke}%
\pgfsetdash{}{0pt}%
\pgfpathmoveto{\pgfqpoint{3.497727in}{0.500000in}}%
\pgfpathlineto{\pgfqpoint{3.666818in}{0.500000in}}%
\pgfpathlineto{\pgfqpoint{3.666818in}{0.500000in}}%
\pgfpathlineto{\pgfqpoint{3.497727in}{0.500000in}}%
\pgfpathclose%
\pgfusepath{stroke,fill}%
\end{pgfscope}%
\begin{pgfscope}%
\pgfpathrectangle{\pgfqpoint{0.750000in}{0.500000in}}{\pgfqpoint{4.650000in}{3.020000in}}%
\pgfusepath{clip}%
\pgfsetbuttcap%
\pgfsetmiterjoin%
\definecolor{currentfill}{rgb}{0.121569,0.466667,0.705882}%
\pgfsetfillcolor{currentfill}%
\pgfsetlinewidth{1.003750pt}%
\definecolor{currentstroke}{rgb}{0.000000,0.000000,0.000000}%
\pgfsetstrokecolor{currentstroke}%
\pgfsetdash{}{0pt}%
\pgfpathmoveto{\pgfqpoint{3.666818in}{0.500000in}}%
\pgfpathlineto{\pgfqpoint{3.835909in}{0.500000in}}%
\pgfpathlineto{\pgfqpoint{3.835909in}{0.500000in}}%
\pgfpathlineto{\pgfqpoint{3.666818in}{0.500000in}}%
\pgfpathclose%
\pgfusepath{stroke,fill}%
\end{pgfscope}%
\begin{pgfscope}%
\pgfpathrectangle{\pgfqpoint{0.750000in}{0.500000in}}{\pgfqpoint{4.650000in}{3.020000in}}%
\pgfusepath{clip}%
\pgfsetbuttcap%
\pgfsetmiterjoin%
\definecolor{currentfill}{rgb}{0.121569,0.466667,0.705882}%
\pgfsetfillcolor{currentfill}%
\pgfsetlinewidth{1.003750pt}%
\definecolor{currentstroke}{rgb}{0.000000,0.000000,0.000000}%
\pgfsetstrokecolor{currentstroke}%
\pgfsetdash{}{0pt}%
\pgfpathmoveto{\pgfqpoint{3.835909in}{0.500000in}}%
\pgfpathlineto{\pgfqpoint{4.005000in}{0.500000in}}%
\pgfpathlineto{\pgfqpoint{4.005000in}{0.500000in}}%
\pgfpathlineto{\pgfqpoint{3.835909in}{0.500000in}}%
\pgfpathclose%
\pgfusepath{stroke,fill}%
\end{pgfscope}%
\begin{pgfscope}%
\pgfpathrectangle{\pgfqpoint{0.750000in}{0.500000in}}{\pgfqpoint{4.650000in}{3.020000in}}%
\pgfusepath{clip}%
\pgfsetbuttcap%
\pgfsetmiterjoin%
\definecolor{currentfill}{rgb}{0.121569,0.466667,0.705882}%
\pgfsetfillcolor{currentfill}%
\pgfsetlinewidth{1.003750pt}%
\definecolor{currentstroke}{rgb}{0.000000,0.000000,0.000000}%
\pgfsetstrokecolor{currentstroke}%
\pgfsetdash{}{0pt}%
\pgfpathmoveto{\pgfqpoint{4.005000in}{0.500000in}}%
\pgfpathlineto{\pgfqpoint{4.174091in}{0.500000in}}%
\pgfpathlineto{\pgfqpoint{4.174091in}{0.500000in}}%
\pgfpathlineto{\pgfqpoint{4.005000in}{0.500000in}}%
\pgfpathclose%
\pgfusepath{stroke,fill}%
\end{pgfscope}%
\begin{pgfscope}%
\pgfpathrectangle{\pgfqpoint{0.750000in}{0.500000in}}{\pgfqpoint{4.650000in}{3.020000in}}%
\pgfusepath{clip}%
\pgfsetbuttcap%
\pgfsetmiterjoin%
\definecolor{currentfill}{rgb}{0.121569,0.466667,0.705882}%
\pgfsetfillcolor{currentfill}%
\pgfsetlinewidth{1.003750pt}%
\definecolor{currentstroke}{rgb}{0.000000,0.000000,0.000000}%
\pgfsetstrokecolor{currentstroke}%
\pgfsetdash{}{0pt}%
\pgfpathmoveto{\pgfqpoint{4.174091in}{0.500000in}}%
\pgfpathlineto{\pgfqpoint{4.343182in}{0.500000in}}%
\pgfpathlineto{\pgfqpoint{4.343182in}{0.500000in}}%
\pgfpathlineto{\pgfqpoint{4.174091in}{0.500000in}}%
\pgfpathclose%
\pgfusepath{stroke,fill}%
\end{pgfscope}%
\begin{pgfscope}%
\pgfpathrectangle{\pgfqpoint{0.750000in}{0.500000in}}{\pgfqpoint{4.650000in}{3.020000in}}%
\pgfusepath{clip}%
\pgfsetbuttcap%
\pgfsetmiterjoin%
\definecolor{currentfill}{rgb}{0.121569,0.466667,0.705882}%
\pgfsetfillcolor{currentfill}%
\pgfsetlinewidth{1.003750pt}%
\definecolor{currentstroke}{rgb}{0.000000,0.000000,0.000000}%
\pgfsetstrokecolor{currentstroke}%
\pgfsetdash{}{0pt}%
\pgfpathmoveto{\pgfqpoint{4.343182in}{0.500000in}}%
\pgfpathlineto{\pgfqpoint{4.512273in}{0.500000in}}%
\pgfpathlineto{\pgfqpoint{4.512273in}{0.500000in}}%
\pgfpathlineto{\pgfqpoint{4.343182in}{0.500000in}}%
\pgfpathclose%
\pgfusepath{stroke,fill}%
\end{pgfscope}%
\begin{pgfscope}%
\pgfpathrectangle{\pgfqpoint{0.750000in}{0.500000in}}{\pgfqpoint{4.650000in}{3.020000in}}%
\pgfusepath{clip}%
\pgfsetbuttcap%
\pgfsetmiterjoin%
\definecolor{currentfill}{rgb}{0.121569,0.466667,0.705882}%
\pgfsetfillcolor{currentfill}%
\pgfsetlinewidth{1.003750pt}%
\definecolor{currentstroke}{rgb}{0.000000,0.000000,0.000000}%
\pgfsetstrokecolor{currentstroke}%
\pgfsetdash{}{0pt}%
\pgfpathmoveto{\pgfqpoint{4.512273in}{0.500000in}}%
\pgfpathlineto{\pgfqpoint{4.681364in}{0.500000in}}%
\pgfpathlineto{\pgfqpoint{4.681364in}{0.542297in}}%
\pgfpathlineto{\pgfqpoint{4.512273in}{0.542297in}}%
\pgfpathclose%
\pgfusepath{stroke,fill}%
\end{pgfscope}%
\begin{pgfscope}%
\pgfpathrectangle{\pgfqpoint{0.750000in}{0.500000in}}{\pgfqpoint{4.650000in}{3.020000in}}%
\pgfusepath{clip}%
\pgfsetbuttcap%
\pgfsetmiterjoin%
\definecolor{currentfill}{rgb}{0.121569,0.466667,0.705882}%
\pgfsetfillcolor{currentfill}%
\pgfsetlinewidth{1.003750pt}%
\definecolor{currentstroke}{rgb}{0.000000,0.000000,0.000000}%
\pgfsetstrokecolor{currentstroke}%
\pgfsetdash{}{0pt}%
\pgfpathmoveto{\pgfqpoint{4.681364in}{0.500000in}}%
\pgfpathlineto{\pgfqpoint{4.850455in}{0.500000in}}%
\pgfpathlineto{\pgfqpoint{4.850455in}{0.500000in}}%
\pgfpathlineto{\pgfqpoint{4.681364in}{0.500000in}}%
\pgfpathclose%
\pgfusepath{stroke,fill}%
\end{pgfscope}%
\begin{pgfscope}%
\pgfpathrectangle{\pgfqpoint{0.750000in}{0.500000in}}{\pgfqpoint{4.650000in}{3.020000in}}%
\pgfusepath{clip}%
\pgfsetbuttcap%
\pgfsetmiterjoin%
\definecolor{currentfill}{rgb}{0.121569,0.466667,0.705882}%
\pgfsetfillcolor{currentfill}%
\pgfsetlinewidth{1.003750pt}%
\definecolor{currentstroke}{rgb}{0.000000,0.000000,0.000000}%
\pgfsetstrokecolor{currentstroke}%
\pgfsetdash{}{0pt}%
\pgfpathmoveto{\pgfqpoint{4.850455in}{0.500000in}}%
\pgfpathlineto{\pgfqpoint{5.019545in}{0.500000in}}%
\pgfpathlineto{\pgfqpoint{5.019545in}{0.500000in}}%
\pgfpathlineto{\pgfqpoint{4.850455in}{0.500000in}}%
\pgfpathclose%
\pgfusepath{stroke,fill}%
\end{pgfscope}%
\begin{pgfscope}%
\pgfpathrectangle{\pgfqpoint{0.750000in}{0.500000in}}{\pgfqpoint{4.650000in}{3.020000in}}%
\pgfusepath{clip}%
\pgfsetbuttcap%
\pgfsetmiterjoin%
\definecolor{currentfill}{rgb}{0.121569,0.466667,0.705882}%
\pgfsetfillcolor{currentfill}%
\pgfsetlinewidth{1.003750pt}%
\definecolor{currentstroke}{rgb}{0.000000,0.000000,0.000000}%
\pgfsetstrokecolor{currentstroke}%
\pgfsetdash{}{0pt}%
\pgfpathmoveto{\pgfqpoint{5.019545in}{0.500000in}}%
\pgfpathlineto{\pgfqpoint{5.188636in}{0.500000in}}%
\pgfpathlineto{\pgfqpoint{5.188636in}{0.584594in}}%
\pgfpathlineto{\pgfqpoint{5.019545in}{0.584594in}}%
\pgfpathclose%
\pgfusepath{stroke,fill}%
\end{pgfscope}%
\begin{pgfscope}%
\pgfsetbuttcap%
\pgfsetroundjoin%
\definecolor{currentfill}{rgb}{0.000000,0.000000,0.000000}%
\pgfsetfillcolor{currentfill}%
\pgfsetlinewidth{0.803000pt}%
\definecolor{currentstroke}{rgb}{0.000000,0.000000,0.000000}%
\pgfsetstrokecolor{currentstroke}%
\pgfsetdash{}{0pt}%
\pgfsys@defobject{currentmarker}{\pgfqpoint{0.000000in}{-0.048611in}}{\pgfqpoint{0.000000in}{0.000000in}}{%
\pgfpathmoveto{\pgfqpoint{0.000000in}{0.000000in}}%
\pgfpathlineto{\pgfqpoint{0.000000in}{-0.048611in}}%
\pgfusepath{stroke,fill}%
}%
\begin{pgfscope}%
\pgfsys@transformshift{1.130455in}{0.500000in}%
\pgfsys@useobject{currentmarker}{}%
\end{pgfscope}%
\end{pgfscope}%
\begin{pgfscope}%
\definecolor{textcolor}{rgb}{0.000000,0.000000,0.000000}%
\pgfsetstrokecolor{textcolor}%
\pgfsetfillcolor{textcolor}%
\pgftext[x=1.130455in,y=0.402778in,,top]{\color{textcolor}\sffamily\fontsize{10.000000}{12.000000}\selectfont \(\displaystyle {-600}\)}%
\end{pgfscope}%
\begin{pgfscope}%
\pgfsetbuttcap%
\pgfsetroundjoin%
\definecolor{currentfill}{rgb}{0.000000,0.000000,0.000000}%
\pgfsetfillcolor{currentfill}%
\pgfsetlinewidth{0.803000pt}%
\definecolor{currentstroke}{rgb}{0.000000,0.000000,0.000000}%
\pgfsetstrokecolor{currentstroke}%
\pgfsetdash{}{0pt}%
\pgfsys@defobject{currentmarker}{\pgfqpoint{0.000000in}{-0.048611in}}{\pgfqpoint{0.000000in}{0.000000in}}{%
\pgfpathmoveto{\pgfqpoint{0.000000in}{0.000000in}}%
\pgfpathlineto{\pgfqpoint{0.000000in}{-0.048611in}}%
\pgfusepath{stroke,fill}%
}%
\begin{pgfscope}%
\pgfsys@transformshift{1.806818in}{0.500000in}%
\pgfsys@useobject{currentmarker}{}%
\end{pgfscope}%
\end{pgfscope}%
\begin{pgfscope}%
\definecolor{textcolor}{rgb}{0.000000,0.000000,0.000000}%
\pgfsetstrokecolor{textcolor}%
\pgfsetfillcolor{textcolor}%
\pgftext[x=1.806818in,y=0.402778in,,top]{\color{textcolor}\sffamily\fontsize{10.000000}{12.000000}\selectfont \(\displaystyle {-400}\)}%
\end{pgfscope}%
\begin{pgfscope}%
\pgfsetbuttcap%
\pgfsetroundjoin%
\definecolor{currentfill}{rgb}{0.000000,0.000000,0.000000}%
\pgfsetfillcolor{currentfill}%
\pgfsetlinewidth{0.803000pt}%
\definecolor{currentstroke}{rgb}{0.000000,0.000000,0.000000}%
\pgfsetstrokecolor{currentstroke}%
\pgfsetdash{}{0pt}%
\pgfsys@defobject{currentmarker}{\pgfqpoint{0.000000in}{-0.048611in}}{\pgfqpoint{0.000000in}{0.000000in}}{%
\pgfpathmoveto{\pgfqpoint{0.000000in}{0.000000in}}%
\pgfpathlineto{\pgfqpoint{0.000000in}{-0.048611in}}%
\pgfusepath{stroke,fill}%
}%
\begin{pgfscope}%
\pgfsys@transformshift{2.483182in}{0.500000in}%
\pgfsys@useobject{currentmarker}{}%
\end{pgfscope}%
\end{pgfscope}%
\begin{pgfscope}%
\definecolor{textcolor}{rgb}{0.000000,0.000000,0.000000}%
\pgfsetstrokecolor{textcolor}%
\pgfsetfillcolor{textcolor}%
\pgftext[x=2.483182in,y=0.402778in,,top]{\color{textcolor}\sffamily\fontsize{10.000000}{12.000000}\selectfont \(\displaystyle {-200}\)}%
\end{pgfscope}%
\begin{pgfscope}%
\pgfsetbuttcap%
\pgfsetroundjoin%
\definecolor{currentfill}{rgb}{0.000000,0.000000,0.000000}%
\pgfsetfillcolor{currentfill}%
\pgfsetlinewidth{0.803000pt}%
\definecolor{currentstroke}{rgb}{0.000000,0.000000,0.000000}%
\pgfsetstrokecolor{currentstroke}%
\pgfsetdash{}{0pt}%
\pgfsys@defobject{currentmarker}{\pgfqpoint{0.000000in}{-0.048611in}}{\pgfqpoint{0.000000in}{0.000000in}}{%
\pgfpathmoveto{\pgfqpoint{0.000000in}{0.000000in}}%
\pgfpathlineto{\pgfqpoint{0.000000in}{-0.048611in}}%
\pgfusepath{stroke,fill}%
}%
\begin{pgfscope}%
\pgfsys@transformshift{3.159545in}{0.500000in}%
\pgfsys@useobject{currentmarker}{}%
\end{pgfscope}%
\end{pgfscope}%
\begin{pgfscope}%
\definecolor{textcolor}{rgb}{0.000000,0.000000,0.000000}%
\pgfsetstrokecolor{textcolor}%
\pgfsetfillcolor{textcolor}%
\pgftext[x=3.159545in,y=0.402778in,,top]{\color{textcolor}\sffamily\fontsize{10.000000}{12.000000}\selectfont \(\displaystyle {0}\)}%
\end{pgfscope}%
\begin{pgfscope}%
\pgfsetbuttcap%
\pgfsetroundjoin%
\definecolor{currentfill}{rgb}{0.000000,0.000000,0.000000}%
\pgfsetfillcolor{currentfill}%
\pgfsetlinewidth{0.803000pt}%
\definecolor{currentstroke}{rgb}{0.000000,0.000000,0.000000}%
\pgfsetstrokecolor{currentstroke}%
\pgfsetdash{}{0pt}%
\pgfsys@defobject{currentmarker}{\pgfqpoint{0.000000in}{-0.048611in}}{\pgfqpoint{0.000000in}{0.000000in}}{%
\pgfpathmoveto{\pgfqpoint{0.000000in}{0.000000in}}%
\pgfpathlineto{\pgfqpoint{0.000000in}{-0.048611in}}%
\pgfusepath{stroke,fill}%
}%
\begin{pgfscope}%
\pgfsys@transformshift{3.835909in}{0.500000in}%
\pgfsys@useobject{currentmarker}{}%
\end{pgfscope}%
\end{pgfscope}%
\begin{pgfscope}%
\definecolor{textcolor}{rgb}{0.000000,0.000000,0.000000}%
\pgfsetstrokecolor{textcolor}%
\pgfsetfillcolor{textcolor}%
\pgftext[x=3.835909in,y=0.402778in,,top]{\color{textcolor}\sffamily\fontsize{10.000000}{12.000000}\selectfont \(\displaystyle {200}\)}%
\end{pgfscope}%
\begin{pgfscope}%
\pgfsetbuttcap%
\pgfsetroundjoin%
\definecolor{currentfill}{rgb}{0.000000,0.000000,0.000000}%
\pgfsetfillcolor{currentfill}%
\pgfsetlinewidth{0.803000pt}%
\definecolor{currentstroke}{rgb}{0.000000,0.000000,0.000000}%
\pgfsetstrokecolor{currentstroke}%
\pgfsetdash{}{0pt}%
\pgfsys@defobject{currentmarker}{\pgfqpoint{0.000000in}{-0.048611in}}{\pgfqpoint{0.000000in}{0.000000in}}{%
\pgfpathmoveto{\pgfqpoint{0.000000in}{0.000000in}}%
\pgfpathlineto{\pgfqpoint{0.000000in}{-0.048611in}}%
\pgfusepath{stroke,fill}%
}%
\begin{pgfscope}%
\pgfsys@transformshift{4.512273in}{0.500000in}%
\pgfsys@useobject{currentmarker}{}%
\end{pgfscope}%
\end{pgfscope}%
\begin{pgfscope}%
\definecolor{textcolor}{rgb}{0.000000,0.000000,0.000000}%
\pgfsetstrokecolor{textcolor}%
\pgfsetfillcolor{textcolor}%
\pgftext[x=4.512273in,y=0.402778in,,top]{\color{textcolor}\sffamily\fontsize{10.000000}{12.000000}\selectfont \(\displaystyle {400}\)}%
\end{pgfscope}%
\begin{pgfscope}%
\pgfsetbuttcap%
\pgfsetroundjoin%
\definecolor{currentfill}{rgb}{0.000000,0.000000,0.000000}%
\pgfsetfillcolor{currentfill}%
\pgfsetlinewidth{0.803000pt}%
\definecolor{currentstroke}{rgb}{0.000000,0.000000,0.000000}%
\pgfsetstrokecolor{currentstroke}%
\pgfsetdash{}{0pt}%
\pgfsys@defobject{currentmarker}{\pgfqpoint{0.000000in}{-0.048611in}}{\pgfqpoint{0.000000in}{0.000000in}}{%
\pgfpathmoveto{\pgfqpoint{0.000000in}{0.000000in}}%
\pgfpathlineto{\pgfqpoint{0.000000in}{-0.048611in}}%
\pgfusepath{stroke,fill}%
}%
\begin{pgfscope}%
\pgfsys@transformshift{5.188636in}{0.500000in}%
\pgfsys@useobject{currentmarker}{}%
\end{pgfscope}%
\end{pgfscope}%
\begin{pgfscope}%
\definecolor{textcolor}{rgb}{0.000000,0.000000,0.000000}%
\pgfsetstrokecolor{textcolor}%
\pgfsetfillcolor{textcolor}%
\pgftext[x=5.188636in,y=0.402778in,,top]{\color{textcolor}\sffamily\fontsize{10.000000}{12.000000}\selectfont \(\displaystyle {600}\)}%
\end{pgfscope}%
\begin{pgfscope}%
\definecolor{textcolor}{rgb}{0.000000,0.000000,0.000000}%
\pgfsetstrokecolor{textcolor}%
\pgfsetfillcolor{textcolor}%
\pgftext[x=3.075000in,y=0.223889in,,top]{\color{textcolor}\sffamily\fontsize{10.000000}{12.000000}\selectfont \(\displaystyle \Delta\) Data Flow Time}%
\end{pgfscope}%
\begin{pgfscope}%
\pgfsetbuttcap%
\pgfsetroundjoin%
\definecolor{currentfill}{rgb}{0.000000,0.000000,0.000000}%
\pgfsetfillcolor{currentfill}%
\pgfsetlinewidth{0.803000pt}%
\definecolor{currentstroke}{rgb}{0.000000,0.000000,0.000000}%
\pgfsetstrokecolor{currentstroke}%
\pgfsetdash{}{0pt}%
\pgfsys@defobject{currentmarker}{\pgfqpoint{-0.048611in}{0.000000in}}{\pgfqpoint{0.000000in}{0.000000in}}{%
\pgfpathmoveto{\pgfqpoint{0.000000in}{0.000000in}}%
\pgfpathlineto{\pgfqpoint{-0.048611in}{0.000000in}}%
\pgfusepath{stroke,fill}%
}%
\begin{pgfscope}%
\pgfsys@transformshift{0.750000in}{0.500000in}%
\pgfsys@useobject{currentmarker}{}%
\end{pgfscope}%
\end{pgfscope}%
\begin{pgfscope}%
\definecolor{textcolor}{rgb}{0.000000,0.000000,0.000000}%
\pgfsetstrokecolor{textcolor}%
\pgfsetfillcolor{textcolor}%
\pgftext[x=0.583333in, y=0.451806in, left, base]{\color{textcolor}\sffamily\fontsize{10.000000}{12.000000}\selectfont \(\displaystyle {0}\)}%
\end{pgfscope}%
\begin{pgfscope}%
\pgfsetbuttcap%
\pgfsetroundjoin%
\definecolor{currentfill}{rgb}{0.000000,0.000000,0.000000}%
\pgfsetfillcolor{currentfill}%
\pgfsetlinewidth{0.803000pt}%
\definecolor{currentstroke}{rgb}{0.000000,0.000000,0.000000}%
\pgfsetstrokecolor{currentstroke}%
\pgfsetdash{}{0pt}%
\pgfsys@defobject{currentmarker}{\pgfqpoint{-0.048611in}{0.000000in}}{\pgfqpoint{0.000000in}{0.000000in}}{%
\pgfpathmoveto{\pgfqpoint{0.000000in}{0.000000in}}%
\pgfpathlineto{\pgfqpoint{-0.048611in}{0.000000in}}%
\pgfusepath{stroke,fill}%
}%
\begin{pgfscope}%
\pgfsys@transformshift{0.750000in}{0.922969in}%
\pgfsys@useobject{currentmarker}{}%
\end{pgfscope}%
\end{pgfscope}%
\begin{pgfscope}%
\definecolor{textcolor}{rgb}{0.000000,0.000000,0.000000}%
\pgfsetstrokecolor{textcolor}%
\pgfsetfillcolor{textcolor}%
\pgftext[x=0.513888in, y=0.874775in, left, base]{\color{textcolor}\sffamily\fontsize{10.000000}{12.000000}\selectfont \(\displaystyle {10}\)}%
\end{pgfscope}%
\begin{pgfscope}%
\pgfsetbuttcap%
\pgfsetroundjoin%
\definecolor{currentfill}{rgb}{0.000000,0.000000,0.000000}%
\pgfsetfillcolor{currentfill}%
\pgfsetlinewidth{0.803000pt}%
\definecolor{currentstroke}{rgb}{0.000000,0.000000,0.000000}%
\pgfsetstrokecolor{currentstroke}%
\pgfsetdash{}{0pt}%
\pgfsys@defobject{currentmarker}{\pgfqpoint{-0.048611in}{0.000000in}}{\pgfqpoint{0.000000in}{0.000000in}}{%
\pgfpathmoveto{\pgfqpoint{0.000000in}{0.000000in}}%
\pgfpathlineto{\pgfqpoint{-0.048611in}{0.000000in}}%
\pgfusepath{stroke,fill}%
}%
\begin{pgfscope}%
\pgfsys@transformshift{0.750000in}{1.345938in}%
\pgfsys@useobject{currentmarker}{}%
\end{pgfscope}%
\end{pgfscope}%
\begin{pgfscope}%
\definecolor{textcolor}{rgb}{0.000000,0.000000,0.000000}%
\pgfsetstrokecolor{textcolor}%
\pgfsetfillcolor{textcolor}%
\pgftext[x=0.513888in, y=1.297744in, left, base]{\color{textcolor}\sffamily\fontsize{10.000000}{12.000000}\selectfont \(\displaystyle {20}\)}%
\end{pgfscope}%
\begin{pgfscope}%
\pgfsetbuttcap%
\pgfsetroundjoin%
\definecolor{currentfill}{rgb}{0.000000,0.000000,0.000000}%
\pgfsetfillcolor{currentfill}%
\pgfsetlinewidth{0.803000pt}%
\definecolor{currentstroke}{rgb}{0.000000,0.000000,0.000000}%
\pgfsetstrokecolor{currentstroke}%
\pgfsetdash{}{0pt}%
\pgfsys@defobject{currentmarker}{\pgfqpoint{-0.048611in}{0.000000in}}{\pgfqpoint{0.000000in}{0.000000in}}{%
\pgfpathmoveto{\pgfqpoint{0.000000in}{0.000000in}}%
\pgfpathlineto{\pgfqpoint{-0.048611in}{0.000000in}}%
\pgfusepath{stroke,fill}%
}%
\begin{pgfscope}%
\pgfsys@transformshift{0.750000in}{1.768908in}%
\pgfsys@useobject{currentmarker}{}%
\end{pgfscope}%
\end{pgfscope}%
\begin{pgfscope}%
\definecolor{textcolor}{rgb}{0.000000,0.000000,0.000000}%
\pgfsetstrokecolor{textcolor}%
\pgfsetfillcolor{textcolor}%
\pgftext[x=0.513888in, y=1.720713in, left, base]{\color{textcolor}\sffamily\fontsize{10.000000}{12.000000}\selectfont \(\displaystyle {30}\)}%
\end{pgfscope}%
\begin{pgfscope}%
\pgfsetbuttcap%
\pgfsetroundjoin%
\definecolor{currentfill}{rgb}{0.000000,0.000000,0.000000}%
\pgfsetfillcolor{currentfill}%
\pgfsetlinewidth{0.803000pt}%
\definecolor{currentstroke}{rgb}{0.000000,0.000000,0.000000}%
\pgfsetstrokecolor{currentstroke}%
\pgfsetdash{}{0pt}%
\pgfsys@defobject{currentmarker}{\pgfqpoint{-0.048611in}{0.000000in}}{\pgfqpoint{0.000000in}{0.000000in}}{%
\pgfpathmoveto{\pgfqpoint{0.000000in}{0.000000in}}%
\pgfpathlineto{\pgfqpoint{-0.048611in}{0.000000in}}%
\pgfusepath{stroke,fill}%
}%
\begin{pgfscope}%
\pgfsys@transformshift{0.750000in}{2.191877in}%
\pgfsys@useobject{currentmarker}{}%
\end{pgfscope}%
\end{pgfscope}%
\begin{pgfscope}%
\definecolor{textcolor}{rgb}{0.000000,0.000000,0.000000}%
\pgfsetstrokecolor{textcolor}%
\pgfsetfillcolor{textcolor}%
\pgftext[x=0.513888in, y=2.143682in, left, base]{\color{textcolor}\sffamily\fontsize{10.000000}{12.000000}\selectfont \(\displaystyle {40}\)}%
\end{pgfscope}%
\begin{pgfscope}%
\pgfsetbuttcap%
\pgfsetroundjoin%
\definecolor{currentfill}{rgb}{0.000000,0.000000,0.000000}%
\pgfsetfillcolor{currentfill}%
\pgfsetlinewidth{0.803000pt}%
\definecolor{currentstroke}{rgb}{0.000000,0.000000,0.000000}%
\pgfsetstrokecolor{currentstroke}%
\pgfsetdash{}{0pt}%
\pgfsys@defobject{currentmarker}{\pgfqpoint{-0.048611in}{0.000000in}}{\pgfqpoint{0.000000in}{0.000000in}}{%
\pgfpathmoveto{\pgfqpoint{0.000000in}{0.000000in}}%
\pgfpathlineto{\pgfqpoint{-0.048611in}{0.000000in}}%
\pgfusepath{stroke,fill}%
}%
\begin{pgfscope}%
\pgfsys@transformshift{0.750000in}{2.614846in}%
\pgfsys@useobject{currentmarker}{}%
\end{pgfscope}%
\end{pgfscope}%
\begin{pgfscope}%
\definecolor{textcolor}{rgb}{0.000000,0.000000,0.000000}%
\pgfsetstrokecolor{textcolor}%
\pgfsetfillcolor{textcolor}%
\pgftext[x=0.513888in, y=2.566651in, left, base]{\color{textcolor}\sffamily\fontsize{10.000000}{12.000000}\selectfont \(\displaystyle {50}\)}%
\end{pgfscope}%
\begin{pgfscope}%
\pgfsetbuttcap%
\pgfsetroundjoin%
\definecolor{currentfill}{rgb}{0.000000,0.000000,0.000000}%
\pgfsetfillcolor{currentfill}%
\pgfsetlinewidth{0.803000pt}%
\definecolor{currentstroke}{rgb}{0.000000,0.000000,0.000000}%
\pgfsetstrokecolor{currentstroke}%
\pgfsetdash{}{0pt}%
\pgfsys@defobject{currentmarker}{\pgfqpoint{-0.048611in}{0.000000in}}{\pgfqpoint{0.000000in}{0.000000in}}{%
\pgfpathmoveto{\pgfqpoint{0.000000in}{0.000000in}}%
\pgfpathlineto{\pgfqpoint{-0.048611in}{0.000000in}}%
\pgfusepath{stroke,fill}%
}%
\begin{pgfscope}%
\pgfsys@transformshift{0.750000in}{3.037815in}%
\pgfsys@useobject{currentmarker}{}%
\end{pgfscope}%
\end{pgfscope}%
\begin{pgfscope}%
\definecolor{textcolor}{rgb}{0.000000,0.000000,0.000000}%
\pgfsetstrokecolor{textcolor}%
\pgfsetfillcolor{textcolor}%
\pgftext[x=0.513888in, y=2.989621in, left, base]{\color{textcolor}\sffamily\fontsize{10.000000}{12.000000}\selectfont \(\displaystyle {60}\)}%
\end{pgfscope}%
\begin{pgfscope}%
\pgfsetbuttcap%
\pgfsetroundjoin%
\definecolor{currentfill}{rgb}{0.000000,0.000000,0.000000}%
\pgfsetfillcolor{currentfill}%
\pgfsetlinewidth{0.803000pt}%
\definecolor{currentstroke}{rgb}{0.000000,0.000000,0.000000}%
\pgfsetstrokecolor{currentstroke}%
\pgfsetdash{}{0pt}%
\pgfsys@defobject{currentmarker}{\pgfqpoint{-0.048611in}{0.000000in}}{\pgfqpoint{0.000000in}{0.000000in}}{%
\pgfpathmoveto{\pgfqpoint{0.000000in}{0.000000in}}%
\pgfpathlineto{\pgfqpoint{-0.048611in}{0.000000in}}%
\pgfusepath{stroke,fill}%
}%
\begin{pgfscope}%
\pgfsys@transformshift{0.750000in}{3.460784in}%
\pgfsys@useobject{currentmarker}{}%
\end{pgfscope}%
\end{pgfscope}%
\begin{pgfscope}%
\definecolor{textcolor}{rgb}{0.000000,0.000000,0.000000}%
\pgfsetstrokecolor{textcolor}%
\pgfsetfillcolor{textcolor}%
\pgftext[x=0.513888in, y=3.412590in, left, base]{\color{textcolor}\sffamily\fontsize{10.000000}{12.000000}\selectfont \(\displaystyle {70}\)}%
\end{pgfscope}%
\begin{pgfscope}%
\definecolor{textcolor}{rgb}{0.000000,0.000000,0.000000}%
\pgfsetstrokecolor{textcolor}%
\pgfsetfillcolor{textcolor}%
\pgftext[x=0.458333in,y=2.010000in,,bottom,rotate=90.000000]{\color{textcolor}\sffamily\fontsize{10.000000}{12.000000}\selectfont Frequency}%
\end{pgfscope}%
\begin{pgfscope}%
\pgfsetrectcap%
\pgfsetmiterjoin%
\pgfsetlinewidth{0.803000pt}%
\definecolor{currentstroke}{rgb}{0.000000,0.000000,0.000000}%
\pgfsetstrokecolor{currentstroke}%
\pgfsetdash{}{0pt}%
\pgfpathmoveto{\pgfqpoint{0.750000in}{0.500000in}}%
\pgfpathlineto{\pgfqpoint{0.750000in}{3.520000in}}%
\pgfusepath{stroke}%
\end{pgfscope}%
\begin{pgfscope}%
\pgfsetrectcap%
\pgfsetmiterjoin%
\pgfsetlinewidth{0.803000pt}%
\definecolor{currentstroke}{rgb}{0.000000,0.000000,0.000000}%
\pgfsetstrokecolor{currentstroke}%
\pgfsetdash{}{0pt}%
\pgfpathmoveto{\pgfqpoint{5.400000in}{0.500000in}}%
\pgfpathlineto{\pgfqpoint{5.400000in}{3.520000in}}%
\pgfusepath{stroke}%
\end{pgfscope}%
\begin{pgfscope}%
\pgfsetrectcap%
\pgfsetmiterjoin%
\pgfsetlinewidth{0.803000pt}%
\definecolor{currentstroke}{rgb}{0.000000,0.000000,0.000000}%
\pgfsetstrokecolor{currentstroke}%
\pgfsetdash{}{0pt}%
\pgfpathmoveto{\pgfqpoint{0.750000in}{0.500000in}}%
\pgfpathlineto{\pgfqpoint{5.400000in}{0.500000in}}%
\pgfusepath{stroke}%
\end{pgfscope}%
\begin{pgfscope}%
\pgfsetrectcap%
\pgfsetmiterjoin%
\pgfsetlinewidth{0.803000pt}%
\definecolor{currentstroke}{rgb}{0.000000,0.000000,0.000000}%
\pgfsetstrokecolor{currentstroke}%
\pgfsetdash{}{0pt}%
\pgfpathmoveto{\pgfqpoint{0.750000in}{3.520000in}}%
\pgfpathlineto{\pgfqpoint{5.400000in}{3.520000in}}%
\pgfusepath{stroke}%
\end{pgfscope}%
\end{pgfpicture}%
\makeatother%
\endgroup%

        }
        \caption{Histogram of the Delta Data Flow Time}
    \end{figure}


    \subsection{Comparison to forwards analysis}
    Basically the answer to RQ1: Is the backwards search efficient enough to perform analysis on real world apps?


    Basically the answer to RQ2: Can we find a pre-analysis known parameter to decide which analysis is more efficient?


\end{document}