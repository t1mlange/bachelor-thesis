\documentclass[../draft.tex]{subfiles}

\newcommand{\tsubEight}[1]{\multicolumn{9}{c}{#1}\\\hline}

\begin{document}
    \chapter{Performance Evaluation}
    In the last chapter, we have shown that our implementation has the necessary soundness to be viable and yields the expected results.
    We now evaluate our implementation against the existing implementation in \textsc{FlowDroid}.

    \section{DroidBench}
    We already introduced \textsc{Droidbench} in \autoref{s:droidbenchvalidation} to validate the soundness of our backward-directed implementation.
    In this section, we focus on the performance in comparison to the existing forward-directed implementation in \textsc{FlowDroid}.

    DroidBench has the advantage that all apps are crafted explicitly for benchmarking taint analysis.
    So, most tests only contain a single-figure number of sources and sinks.
    Also, the number of sources and sinks are often equal or differ by one to test whether the tool can differentiate something.
    These simplify the comparison between both analysis directions as neither one has an initial disadvantage.

    Most test cases are small enough to be analyzed in sub-two seconds on an average four-core desktop CPU from 2012.
    Our test environment is not isolated, so background tasks and the process scheduler can affect the runtime.
    The short runtime, together with the variance of the unisolated testing environment, render the runtime unusable as a comparison point.
    In contrast, edge propagations are deterministic\footnotemark{} and correlate with the runtime.
    \footnotetext{This is only true if there are enough resources. \textsc{FlowDroid} tries to gracefully terminate when running low on memory.
    Also, timeouts result in a non-reproducible number of edge propagations.}
    Thus, we only use the number of propagations to compare both implementations.

    The configuration is the same as described in \autoref{s:droidbenchconfig}.

    \subsection{Results}
    The full results are listed in \autoref{t:droidbenchevaluation}.
    When rows only contain hyphens, the IFDS analysis did not start, e.g., because no sink is in the reachable code.
    \#I denotes the number of edge propagations inside the infoflow analysis and \#A the number of edge propagations inside the alias analysis.
    We calculated the absolute difference with the existing implementation as the reference: $\mathit{Result}_{\mathit{B}} - \mathit{Result}_{\mathit{F}}$.
    The relative difference is calculated similar: $\frac{\mathit{TotalDifference}}{|\mathit{\#I_\mathit{F} + \#A_\mathit{F}}|}$.
    Hence, negative values signify the backward analysis performed better.

    On average, our implementation needs more edge propagations to finish the analysis.
    Even though for explicit flows the backward analysis needs less propagations in the infoflow analysis, it then suffers from more encountered aliases.
    If we look at it on a per test basis, there are not many test cases where both perform identically.
    Instead, dependent on the specific test case, the relative difference is between $-1$ and $1$.
    However, we did not expect cases that let the edge propagations of our implementation explode up to a factor of $100$, as seen in \code{LifecycleTest#BroadcastReceiverLifecycle3}.
    In contrast, the existing forward implementation only at most a relative difference of $-0.95$.

    \footnotesize
    \begin{longtable}{l | r | r | r | r | r | r | r | r}
        & \multicolumn{2}{c|}{\textbf{Forwards}} & \multicolumn{2}{c|}{\textbf{Backwards}} & \multicolumn{4}{c}{\textbf{Difference}}\\
        \multirow{-2}{*}{\textbf{Test Case}} & \textbf{\#I} & \textbf{\#A} & \textbf{\#I} & \textbf{\#A} & \textbf{\#I} & \textbf{\#A}& \textbf{Total} & \textbf{Relative}\\
        \hhline
        \endhead
        \tsubEight{AliasingTest}
        FlowSensitivity1 & $175$ & $72$ & $39$ & $4$ & $-136$ & $-68$ & $-204$ & $-0.83$\\
        Merge1 & $137$ & $65$ & $89$ & $47$ & $-48$ & $-18$ & $-66$ & $-0.33$\\
        SimpleAliasing1 & $35$ & $13$ & $20$ & $3$ & $-15$ & $-10$ & $-25$ & $-0.52$\\
        StrongUpdate1 & $30$ & $13$ & $11$ & $3$ & $-19$ & $-10$ & $-29$ & $-0.67$\\
        \hline
        \tsubEight{AndroidSpecificTest}
        ApplicationModeling1 & $212$ & $96$ & $851$ & $1208$ & $639$ & $1112$ & $1751$ & $5.69$\\
        DirectLeak1 & $3$ & $0$ & $4$ & $0$ & $1$ & $0$ & $1$ & $0.33$\\
        InactiveActivity & $-$ & $-$ & $-$ & $-$ & $-$ & $-$ & $-$ & $-$\\
        Library2 & $5$ & $0$ & $6$ & $0$ & $1$ & $0$ & $1$ & $0.2$\\
        LogNoLeak & $-$ & $-$ & $-$ & $-$ & $-$ & $-$ & $-$ & $-$\\
        Obfuscation1 & $4$ & $0$ & $4$ & $0$ & $0$ & $0$ & $0$ & $0.0$\\
        Parcel1 & $144$ & $15$ & $86$ & $93$ & $-58$ & $78$ & $20$ & $0.13$\\
        PrivateDataLeak1 & $410$ & $110$ & $608$ & $766$ & $198$ & $656$ & $854$ & $1.64$\\
        PrivateDataLeak2 & $15$ & $0$ & $5$ & $3$ & $-10$ & $3$ & $-7$ & $-0.47$\\
        PrivateDataLeak3 & $17$ & $2$ & $210$ & $140$ & $193$ & $138$ & $331$ & $17.42$\\
        runPublicAPIField1 & $89$ & $1$ & $43$ & $0$ & $-46$ & $-1$ & $-47$ & $-0.52$\\
        runPublicAPIField2 & $5$ & $0$ & $11$ & $0$ & $6$ & $0$ & $6$ & $1.2$\\
        runView1 & $71$ & $50$ & $69$ & $0$ & $-2$ & $-50$ & $-52$ & $-0.43$\\
        \hline
        \tsubEight{ArrayAndListTest}
        ArrayAccess1 & $77$ & $34$ & $51$ & $100$ & $-26$ & $66$ & $40$ & $0.36$\\
        ArrayAccess2 & $16$ & $4$ & $12$ & $0$ & $-4$ & $-4$ & $-8$ & $-0.4$\\
        ArrayAccess3 & $77$ & $34$ & $51$ & $100$ & $-26$ & $66$ & $40$ & $0.36$\\
        ArrayAccess4 & $164$ & $84$ & $42$ & $21$ & $-122$ & $-63$ & $-185$ & $-0.75$\\
        ArrayAccess5 & $75$ & $5$ & $34$ & $23$ & $-41$ & $18$ & $-23$ & $-0.29$\\
        ArrayCopy1 & $18$ & $2$ & $9$ & $2$ & $-9$ & $0$ & $-9$ & $-0.45$\\
        ArrayToString1 & $10$ & $1$ & $6$ & $0$ & $-4$ & $-1$ & $-5$ & $-0.45$\\
        HashMapAccess1 & $22$ & $5$ & $15$ & $1$ & $-7$ & $-4$ & $-11$ & $-0.41$\\
        ListAccess1 & $85$ & $9$ & $60$ & $97$ & $-25$ & $88$ & $63$ & $0.67$\\
        MultidimensionalArray1 & $29$ & $3$ & $16$ & $23$ & $-13$ & $20$ & $7$ & $0.22$\\
        \hline
        \tsubEight{CallbackTest}
        AnonymousClass1 & $152$ & $0$ & $208$ & $0$ & $56$ & $0$ & $56$ & $0.37$\\
        Button1 & $58$ & $39$ & $43$ & $0$ & $-15$ & $-39$ & $-54$ & $-0.56$\\
        Button2 & $444$ & $66$ & $155$ & $254$ & $-289$ & $188$ & $-101$ & $-0.2$\\
        Button3 & $360$ & $89$ & $109$ & $408$ & $-251$ & $319$ & $68$ & $0.15$\\
        Button4 & $58$ & $39$ & $43$ & $0$ & $-15$ & $-39$ & $-54$ & $-0.56$\\
        Button5 & $80$ & $40$ & $6$ & $3$ & $-74$ & $-37$ & $-111$ & $-0.93$\\
        LocationLeak1 & $617$ & $222$ & $260$ & $298$ & $-357$ & $76$ & $-281$ & $-0.33$\\
        LocationLeak2 & $212$ & $121$ & $152$ & $0$ & $-60$ & $-121$ & $-181$ & $-0.54$\\
        LocationLeak3 & $220$ & $73$ & $104$ & $115$ & $-116$ & $42$ & $-74$ & $-0.25$\\
        MethodOverride1 & $3$ & $0$ & $2$ & $0$ & $-1$ & $0$ & $-1$ & $-0.33$\\
        MultiHandlers1 & $17$ & $0$ & $145$ & $149$ & $128$ & $149$ & $277$ & $16.29$\\
        Ordering1 & $456$ & $151$ & $44$ & $0$ & $-412$ & $-151$ & $-563$ & $-0.93$\\
        RegisterGlobal1 & $291$ & $162$ & $49$ & $0$ & $-242$ & $-162$ & $-404$ & $-0.89$\\
        RegisterGlobal2 & $52$ & $37$ & $43$ & $0$ & $-9$ & $-37$ & $-46$ & $-0.52$\\
        Unregister1 & $11$ & $0$ & $9$ & $0$ & $-2$ & $0$ & $-2$ & $-0.18$\\
        \hline
        \tsubEight{EmulatorDetectionTest}
        Battery1 & $7$ & $0$ & $39$ & $15$ & $32$ & $15$ & $47$ & $6.71$\\
        Bluetooth1 & $4$ & $0$ & $4$ & $0$ & $0$ & $0$ & $0$ & $0.0$\\
        Build1 & $4$ & $0$ & $4$ & $0$ & $0$ & $0$ & $0$ & $0.0$\\
        Contacts1 & $53$ & $0$ & $200$ & $19$ & $147$ & $19$ & $166$ & $3.13$\\
        ContentProvider1 & $13$ & $0$ & $8$ & $0$ & $-5$ & $0$ & $-5$ & $-0.38$\\
        DeviceId1 & $15$ & $0$ & $6$ & $0$ & $-9$ & $0$ & $-9$ & $-0.6$\\
        File1 & $4$ & $0$ & $4$ & $0$ & $0$ & $0$ & $0$ & $0.0$\\
        IMEI1 & $137$ & $0$ & $422$ & $1$ & $285$ & $1$ & $286$ & $2.09$\\
        IP1 & $4$ & $0$ & $29$ & $0$ & $25$ & $0$ & $25$ & $6.25$\\
        PI1 & $6$ & $0$ & $4$ & $0$ & $-2$ & $0$ & $-2$ & $-0.33$\\
        PlayStore1 & $158$ & $0$ & $8$ & $0$ & $-150$ & $0$ & $-150$ & $-0.95$\\
        PlayStore2 & $4$ & $0$ & $4$ & $0$ & $0$ & $0$ & $0$ & $0.0$\\
        Sensors1 & $5$ & $0$ & $4$ & $0$ & $-1$ & $0$ & $-1$ & $-0.2$\\
        SubscriberId1 & $29$ & $0$ & $4$ & $0$ & $-25$ & $0$ & $-25$ & $-0.86$\\
        VoiceMail1 & $4$ & $0$ & $4$ & $0$ & $0$ & $0$ & $0$ & $0.0$\\
        \hline
        \tsubEight{FieldAndObjectSensitivityTest}
        FieldSensitivity1 & $98$ & $50$ & $25$ & $3$ & $-73$ & $-47$ & $-120$ & $-0.81$\\
        FieldSensitivity2 & $35$ & $15$ & $19$ & $0$ & $-16$ & $-15$ & $-31$ & $-0.62$\\
        FieldSensitivity3 & $38$ & $15$ & $16$ & $0$ & $-22$ & $-15$ & $-37$ & $-0.7$\\
        FieldSensitivity4 & $14$ & $6$ & $8$ & $0$ & $-6$ & $-6$ & $-12$ & $-0.6$\\
        InheritedObjects1 & $4$ & $0$ & $6$ & $0$ & $2$ & $0$ & $2$ & $0.5$\\
        ObjectSensitivity1 & $19$ & $7$ & $14$ & $1$ & $-5$ & $-6$ & $-11$ & $-0.42$\\
        ObjectSensitivity2 & $15$ & $8$ & $10$ & $0$ & $-5$ & $-8$ & $-13$ & $-0.57$\\
        \hline
        \tsubEight{GeneralJavaTest}
        Clone1 & $23$ & $2$ & $12$ & $4$ & $-11$ & $2$ & $-9$ & $-0.36$\\
        Exceptions1 & $16$ & $0$ & $13$ & $0$ & $-3$ & $0$ & $-3$ & $-0.19$\\
        Exceptions2 & $22$ & $0$ & $13$ & $0$ & $-9$ & $0$ & $-9$ & $-0.41$\\
        Exceptions3 & $18$ & $0$ & $11$ & $0$ & $-7$ & $0$ & $-7$ & $-0.39$\\
        Exceptions4 & $21$ & $1$ & $22$ & $0$ & $1$ & $-1$ & $0$ & $0.0$\\
        Exceptions5 & $13$ & $1$ & $16$ & $0$ & $3$ & $-1$ & $2$ & $0.14$\\
        Exceptions6 & $78$ & $12$ & $23$ & $0$ & $-55$ & $-12$ & $-67$ & $-0.74$\\
        Exceptions7 & $71$ & $12$ & $6$ & $0$ & $-65$ & $-12$ & $-77$ & $-0.93$\\
        FactoryMethods1 & $40$ & $0$ & $14$ & $0$ & $-26$ & $0$ & $-26$ & $-0.65$\\
        Loop1 & $93$ & $2$ & $51$ & $0$ & $-42$ & $-2$ & $-44$ & $-0.46$\\
        Loop2 & $123$ & $2$ & $79$ & $0$ & $-44$ & $-2$ & $-46$ & $-0.37$\\
        Serialization1 & $50$ & $4$ & $22$ & $29$ & $-28$ & $25$ & $-3$ & $-0.06$\\
        SourceCodeSpecific1 & $16$ & $0$ & $45$ & $7$ & $29$ & $7$ & $36$ & $2.25$\\
        StartProcessWithSecret1 & $29$ & $8$ & $17$ & $3$ & $-12$ & $-5$ & $-17$ & $-0.46$\\
        StaticInitialization1 & $26$ & $27$ & $9$ & $0$ & $-17$ & $-27$ & $-44$ & $-0.83$\\
        StaticInitialization2 & $57$ & $29$ & $86$ & $0$ & $29$ & $-29$ & $0$ & $0.0$\\
        StaticInitialization3 & $35$ & $9$ & $5$ & $0$ & $-30$ & $-9$ & $-39$ & $-0.89$\\
        StringFormatter1 & $16$ & $1$ & $10$ & $0$ & $-6$ & $-1$ & $-7$ & $-0.41$\\
        StringPatternMatching1 & $23$ & $1$ & $8$ & $4$ & $-15$ & $3$ & $-12$ & $-0.5$\\
        StringToCharArray1 & $91$ & $4$ & $47$ & $0$ & $-44$ & $-4$ & $-48$ & $-0.51$\\
        StringToOutputStream1 & $26$ & $3$ & $25$ & $1$ & $-1$ & $-2$ & $-3$ & $-0.1$\\
        UnreachableCode & $-$ & $-$ & $-$ & $-$ & $-$ & $-$ & $-$ & $-$\\
        VirtualDispatch1 & $128$ & $31$ & $88$ & $28$ & $-40$ & $-3$ & $-43$ & $-0.27$\\
        VirtualDispatch2 & $7$ & $0$ & $12$ & $0$ & $5$ & $0$ & $5$ & $0.71$\\
        VirtualDispatch3 & $8$ & $0$ & $6$ & $0$ & $-2$ & $0$ & $-2$ & $-0.25$\\
        VirtualDispatch4 & $-$ & $-$ & $-$ & $-$ & $-$ & $-$ & $-$ & $-$\\
        \hline
        \tsubEight{ImplicitFlowTest}
        ImplicitFlow1 & $1823$ & $144$ & $3315$ & $11$ & $1492$ & $-133$ & $1359$ & $0.69$\\
        ImplicitFlow2 & $146$ & $63$ & $991$ & $3$ & $845$ & $-60$ & $785$ & $3.76$\\
        ImplicitFlow3 & $148$ & $50$ & $1023$ & $20$ & $875$ & $-30$ & $845$ & $4.27$\\
        ImplicitFlow4 & $67$ & $0$ & $1864$ & $12$ & $1797$ & $12$ & $1809$ & $27.0$\\
        ImplicitFlow6 & $18$ & $0$ & $112$ & $0$ & $94$ & $0$ & $94$ & $5.22$\\
        \hline
        \tsubEight{LifecycleTest}
        ActivityEventSequence1 & $58$ & $35$ & $72$ & $0$ & $14$ & $-35$ & $-21$ & $-0.23$\\
        ActivityEventSequence2 & $32$ & $24$ & $77$ & $0$ & $45$ & $-24$ & $21$ & $0.38$\\
        ActivityEventSequence3 & $209$ & $116$ & $156$ & $0$ & $-53$ & $-116$ & $-169$ & $-0.52$\\
        ActivityLifecycle1 & $99$ & $72$ & $156$ & $7$ & $57$ & $-65$ & $-8$ & $-0.05$\\
        ActivityLifecycle2 & $47$ & $34$ & $33$ & $0$ & $-14$ & $-34$ & $-48$ & $-0.59$\\
        ActivityLifecycle3 & $65$ & $31$ & $28$ & $0$ & $-37$ & $-31$ & $-68$ & $-0.71$\\
        ActivityLifecycle4 & $49$ & $33$ & $14$ & $0$ & $-35$ & $-33$ & $-68$ & $-0.83$\\
        ActivitySavedState1 & $20$ & $0$ & $7$ & $0$ & $-13$ & $0$ & $-13$ & $-0.65$\\
        ApplicationLifecycle1 & $37$ & $10$ & $82$ & $0$ & $45$ & $-10$ & $35$ & $0.74$\\
        ApplicationLifecycle2 & $86$ & $17$ & $94$ & $155$ & $8$ & $138$ & $146$ & $1.42$\\
        ApplicationLifecycle3 & $32$ & $12$ & $21$ & $0$ & $-11$ & $-12$ & $-23$ & $-0.52$\\
        AsynchronousEventOrdering1 & $58$ & $31$ & $16$ & $0$ & $-42$ & $-31$ & $-73$ & $-0.82$\\
        BroadcastReceiverLifecycle1 & $4$ & $0$ & $4$ & $0$ & $0$ & $0$ & $0$ & $0.0$\\
        BroadcastReceiverLifecycle2 & $109$ & $44$ & $248$ & $114$ & $139$ & $70$ & $209$ & $1.37$\\
        BroadcastReceiverLifecycle3 & $3$ & $0$ & $195$ & $110$ & $192$ & $110$ & $302$ & $100.67$\\
        EventOrdering1 & $61$ & $29$ & $30$ & $0$ & $-31$ & $-29$ & $-60$ & $-0.67$\\
        FragmentLifecycle1 & $187$ & $127$ & $90$ & $0$ & $-97$ & $-127$ & $-224$ & $-0.71$\\
        FragmentLifecycle2 & $-$ & $-$ & $-$ & $-$ & $-$ & $-$ & $-$ & $-$\\
        ServiceEventSequence1 & $53$ & $20$ & $124$ & $34$ & $71$ & $14$ & $85$ & $1.16$\\
        ServiceEventSequence2 & $105$ & $49$ & $389$ & $220$ & $284$ & $171$ & $455$ & $2.95$\\
        ServiceEventSequence3 & $46$ & $12$ & $275$ & $151$ & $229$ & $139$ & $368$ & $6.34$\\
        ServiceLifecycle1 & $119$ & $44$ & $42$ & $0$ & $-77$ & $-44$ & $-121$ & $-0.74$\\
        ServiceLifecycle2 & $68$ & $20$ & $89$ & $21$ & $21$ & $1$ & $22$ & $0.25$\\
        SharedPreferenceChanged1 & $13$ & $0$ & $11$ & $0$ & $-2$ & $0$ & $-2$ & $-0.15$\\
        \hline
        \tsubEight{ReflectionTest}
        Reflection1 & $15$ & $5$ & $8$ & $0$ & $-7$ & $-5$ & $-12$ & $-0.6$\\
        Reflection2 & $21$ & $5$ & $11$ & $0$ & $-10$ & $-5$ & $-15$ & $-0.58$\\
        Reflection3 & $42$ & $9$ & $62$ & $25$ & $20$ & $16$ & $36$ & $0.71$\\
        Reflection4 & $9$ & $0$ & $8$ & $0$ & $-1$ & $0$ & $-1$ & $-0.11$\\
        Reflection5 & $16$ & $1$ & $11$ & $0$ & $-5$ & $-1$ & $-6$ & $-0.35$\\
        Reflection6 & $7$ & $0$ & $134$ & $51$ & $127$ & $51$ & $178$ & $25.43$\\
        Reflection7 & $15$ & $5$ & $15$ & $11$ & $0$ & $6$ & $6$ & $0.3$\\
        Reflection8 & $35$ & $7$ & $14$ & $0$ & $-21$ & $-7$ & $-28$ & $-0.67$\\
        Reflection9 & $42$ & $7$ & $21$ & $0$ & $-21$ & $-7$ & $-28$ & $-0.57$\\
        \hline
        \tsubEight{ThreadingTest}
        AsyncTask1 & $22$ & $2$ & $11$ & $1$ & $-11$ & $-1$ & $-12$ & $-0.5$\\
        Executor1 & $34$ & $7$ & $17$ & $0$ & $-17$ & $-7$ & $-24$ & $-0.59$\\
        JavaThread1 & $34$ & $7$ & $17$ & $0$ & $-17$ & $-7$ & $-24$ & $-0.59$\\
        JavaThread2 & $62$ & $10$ & $31$ & $8$ & $-31$ & $-2$ & $-33$ & $-0.46$\\
        Looper1 & $49$ & $3$ & $20$ & $16$ & $-29$ & $13$ & $-16$ & $-0.31$\\
        TimerTask1 & $203$ & $28$ & $32$ & $33$ & $-171$ & $5$ & $-166$ & $-0.72$\\
        \hhline
        \hiderowcolors
        $\varnothing$ Propagations & $85.46$ & $23.41$ & $117.64$ & $38.6$ & $32.19$ & $15.19$ & $47.37$ & $1.61$\\
        $\varnothing$ without Implicit Flow & $70.61$ & $22.46$ & $60.56$ & $40.1$ & $-10.05$ & $17.63$ & $7.59$ & $1.34$\\
        \caption{DroidBench Performance Evaluation Results}
        \label{t:droidbenchevaluation}
    \end{longtable}
    \normalsize

    \subsection{Result Explanation}
    We define tests with a relative difference greater than $10$ as worth investigating.
    In the following, we explain why our implementation performed worse than expected.

    \paragraph{PrivateDataLeak3}
    This test contains two sinks and one source.
    The tainted data is written to a file, later read from the file and then leaked.
    \textsc{FlowDroid} does not support tracking taints over files, so it only finds a leak from source to file write but misses the leak from file read to send SMS.
    Due to EasyTaintWrapper's simplicity, overtainting happens in the backward direction. When \code{FileInputStream fis = openFileInput("out.txt");} is called with \code{fis} tainted, EasyTaintWrapper also taints the base object - the \code{MainActivity} in this case.
    As the \code{MainActivity} has an enormous scope, the taint has a long lifetime and many other taints could derive from this taint.
    This taint explains the relative difference of $17.68$.
    Using the more precise SummaryTaintWrapper, the edges reduce to $(51, 16)$ and a relative difference of $2.53$, which is more reasonable.
    It is still higher because of the second sink.

    \paragraph{MultiHandlers1}
    Two \code{LocationListener}s are registered in different activities.
    In both activities, an instance field is a parameter of a sink.
    So there are two possible paths where something could be leaked.
    The LocationListener does not call any source on the first path, while the second path has an empty setter method killing the taint.
    For the first path, the backward analysis has to propagate the taint into the \code{LocationListener} to notice that this is a dead-end while the forward's search does not even start there.
    For the second path, the backward analysis seems to suffer because it starts at an instance field taint with a larger scope than a local variable.

    \paragraph{BroadcastReceiverLifecycle3}
    The test contains five sinks but only one source.
    If we only consider the leak path, both implementations perform equally.
    The four other sinks are responsible for the overhead on edge propagations.

    \paragraph{Reflection6}
    The reflective call site has multiple possible callees in the interprocedural control-flow graph.
    Backward all of these callees are visited, of which only one contains a source statement.
    Forward, the taint is introduced in the callee at the source and just one return site needs to be processed.

    \paragraph{A Note On Implicit Flows}
    All implicit flow tests and the IMEI1 test need the implicit flow rule to find the leaks.
    In those test cases our implementation does not stand a chance.
    We especially want to highlight the "every sink call influenced by conditional" semantics here.
    This semantic forces us to derive an empty taint for every conditional that is theoretically reachable from a sink.
    Beyond, we also taint the base object without any fields at every sink to detect a possible conditional object instantiation.
    Even in simple test cases such as ImplicitFlow4 this results in 10 additional taints per sink.
    Important to note is also that the prior computation of reachable conditionals is not represented in the edge propagations.
    We thus conclude that it is probably better to live without a backward-directed implicit data flow analysis.

    \subsection{Using A More Precise Taint Wrapper}
    We noticed the overtainting in \code{PrivateDataLeak3} is caused by the \code{EasyTaintWrapper}.
    Thus, we now look how using the \code{SummaryTaintWrapper} influences the edge propagations.
    The full results are in \autoref{t:droidbenchevaluation_sum}.
    In the table, we take the \code{EasyTaintWrapper} as the reference and compare it against the \code{SummaryTaintWrapper} on our implementation.
    The structure of the table is as in the last subsection.

    As we already described, \code{PrivateDataLeak3} benefits from the more precise taint wrapper.
    Similarily, many other test cases also benefit.
    Others, especially Serialization1 have more edge propagations because the \code{SummaryTaintWrapper} has a summary for a method which the \code{EasyTaintWrapper} does not handle\footnotemark{} resulting in a premature kill of a taint.
    \footnotetext{The \footnotecode{EasyTaintWrapper} contains a list of supported classes. Every method from those classes is excluded from the analysis, regardless of the method being in the list of the handled methods.}
    Even with Serialization1 included in the average, the \code{SummaryTaintWrapper} needs less total edge propagations.
    Excluding it also equals out the relative difference.
    Altogether, the \code{SummaryTaintWrapper} should be the default choice for real-world applications because it is more precise without compromising on the edge propagations.

    \footnotesize
    \begin{longtable}{l | r | r | r | r | r | r | r | r}
        & \multicolumn{2}{c|}{\textbf{EasyTW}} & \multicolumn{2}{c|}{\textbf{SummaryTW}} & \multicolumn{4}{c}{\textbf{Difference}}\\
        \multirow{-2}{*}{\textbf{Test Case}} & \textbf{\#I} & \textbf{\#A} & \textbf{\#I} & \textbf{\#A} & \textbf{\#I} & \textbf{\#A}& \textbf{Total} & \textbf{Relative}\\
        \hhline
        \endhead
        \hline
        \tsubEight{AliasingTest}
        FlowSensitivity1 & $39$ & $4$ & $71$ & $13$ & $32$ & $9$ & $41$ & $0.95$\\
        Merge1 & $89$ & $47$ & $109$ & $91$ & $20$ & $44$ & $64$ & $0.47$\\
        SimpleAliasing1 & $20$ & $3$ & $20$ & $3$ & $0$ & $0$ & $0$ & $0.0$\\
        StrongUpdate1 & $11$ & $3$ & $11$ & $3$ & $0$ & $0$ & $0$ & $0.0$\\
        \hline
        \tsubEight{AndroidSpecificTest}
        ApplicationModeling1 & $851$ & $1208$ & $427$ & $792$ & $-424$ & $-416$ & $-840$ & $-0.41$\\
        DirectLeak1 & $4$ & $0$ & $4$ & $0$ & $0$ & $0$ & $0$ & $0.0$\\
        InactiveActivity & $-$ & $-$ & $-$ & $-$ & $-$ & $-$ & $-$ & $-$\\
        Library2 & $6$ & $0$ & $6$ & $0$ & $0$ & $0$ & $0$ & $0.0$\\
        LogNoLeak & $-$ & $-$ & $-$ & $-$ & $-$ & $-$ & $-$ & $-$\\
        Obfuscation1 & $4$ & $0$ & $4$ & $0$ & $0$ & $0$ & $0$ & $0.0$\\
        Parcel1 & $86$ & $93$ & $87$ & $76$ & $1$ & $-17$ & $-16$ & $-0.09$\\
        PrivateDataLeak1 & $608$ & $766$ & $585$ & $766$ & $-23$ & $0$ & $-23$ & $-0.02$\\
        PrivateDataLeak2 & $5$ & $3$ & $5$ & $3$ & $0$ & $0$ & $0$ & $0.0$\\
        PrivateDataLeak3 & $210$ & $140$ & $38$ & $12$ & $-172$ & $-128$ & $-300$ & $-0.86$\\
        runPublicAPIField1 & $43$ & $0$ & $36$ & $0$ & $-7$ & $0$ & $-7$ & $-0.16$\\
        runPublicAPIField2 & $11$ & $0$ & $14$ & $0$ & $3$ & $0$ & $3$ & $0.27$\\
        runView1 & $69$ & $0$ & $69$ & $0$ & $0$ & $0$ & $0$ & $0.0$\\
        \hline
        \tsubEight{ArrayAndListTest}
        ArrayAccess1 & $51$ & $100$ & $51$ & $100$ & $0$ & $0$ & $0$ & $0.0$\\
        ArrayAccess2 & $12$ & $0$ & $12$ & $0$ & $0$ & $0$ & $0$ & $0.0$\\
        ArrayAccess3 & $51$ & $100$ & $51$ & $100$ & $0$ & $0$ & $0$ & $0.0$\\
        ArrayAccess4 & $42$ & $21$ & $42$ & $21$ & $0$ & $0$ & $0$ & $0.0$\\
        ArrayAccess5 & $34$ & $23$ & $34$ & $23$ & $0$ & $0$ & $0$ & $0.0$\\
        ArrayCopy1 & $9$ & $2$ & $9$ & $2$ & $0$ & $0$ & $0$ & $0.0$\\
        ArrayToString1 & $6$ & $0$ & $6$ & $0$ & $0$ & $0$ & $0$ & $0.0$\\
        HashMapAccess1 & $15$ & $1$ & $15$ & $1$ & $0$ & $0$ & $0$ & $0.0$\\
        ListAccess1 & $60$ & $97$ & $77$ & $118$ & $17$ & $21$ & $38$ & $0.24$\\
        MultidimensionalArray1 & $16$ & $23$ & $16$ & $23$ & $0$ & $0$ & $0$ & $0.0$\\
        \hline
        \tsubEight{CallbackTest}
        AnonymousClass1 & $208$ & $0$ & $208$ & $0$ & $0$ & $0$ & $0$ & $0.0$\\
        Button1 & $43$ & $0$ & $43$ & $0$ & $0$ & $0$ & $0$ & $0.0$\\
        Button2 & $155$ & $254$ & $184$ & $272$ & $29$ & $18$ & $47$ & $0.11$\\
        Button3 & $109$ & $408$ & $120$ & $357$ & $11$ & $-51$ & $-40$ & $-0.08$\\
        Button4 & $43$ & $0$ & $43$ & $0$ & $0$ & $0$ & $0$ & $0.0$\\
        Button5 & $6$ & $3$ & $7$ & $3$ & $1$ & $0$ & $1$ & $0.11$\\
        LocationLeak1 & $260$ & $298$ & $286$ & $314$ & $26$ & $16$ & $42$ & $0.08$\\
        LocationLeak2 & $152$ & $0$ & $152$ & $0$ & $0$ & $0$ & $0$ & $0.0$\\
        LocationLeak3 & $104$ & $115$ & $107$ & $115$ & $3$ & $0$ & $3$ & $0.01$\\
        MethodOverride1 & $2$ & $0$ & $2$ & $0$ & $0$ & $0$ & $0$ & $0.0$\\
        MultiHandlers1 & $145$ & $149$ & $148$ & $149$ & $3$ & $0$ & $3$ & $0.01$\\
        Ordering1 & $44$ & $0$ & $44$ & $0$ & $0$ & $0$ & $0$ & $0.0$\\
        RegisterGlobal1 & $49$ & $0$ & $49$ & $0$ & $0$ & $0$ & $0$ & $0.0$\\
        RegisterGlobal2 & $43$ & $0$ & $43$ & $0$ & $0$ & $0$ & $0$ & $0.0$\\
        Unregister1 & $9$ & $0$ & $9$ & $0$ & $0$ & $0$ & $0$ & $0.0$\\
        \hline
        \tsubEight{EmulatorDetectionTest}
        Battery1 & $39$ & $15$ & $39$ & $15$ & $0$ & $0$ & $0$ & $0.0$\\
        Bluetooth1 & $4$ & $0$ & $4$ & $0$ & $0$ & $0$ & $0$ & $0.0$\\
        Build1 & $4$ & $0$ & $4$ & $0$ & $0$ & $0$ & $0$ & $0.0$\\
        Contacts1 & $200$ & $19$ & $167$ & $4$ & $-33$ & $-15$ & $-48$ & $-0.22$\\
        ContentProvider1 & $8$ & $0$ & $8$ & $0$ & $0$ & $0$ & $0$ & $0.0$\\
        DeviceId1 & $6$ & $0$ & $6$ & $0$ & $0$ & $0$ & $0$ & $0.0$\\
        File1 & $4$ & $0$ & $4$ & $0$ & $0$ & $0$ & $0$ & $0.0$\\
        IP1 & $29$ & $0$ & $52$ & $0$ & $23$ & $0$ & $23$ & $0.79$\\
        PI1 & $4$ & $0$ & $4$ & $0$ & $0$ & $0$ & $0$ & $0.0$\\
        PlayStore1 & $8$ & $0$ & $8$ & $0$ & $0$ & $0$ & $0$ & $0.0$\\
        PlayStore2 & $4$ & $0$ & $4$ & $0$ & $0$ & $0$ & $0$ & $0.0$\\
        Sensors1 & $4$ & $0$ & $4$ & $0$ & $0$ & $0$ & $0$ & $0.0$\\
        SubscriberId1 & $4$ & $0$ & $4$ & $0$ & $0$ & $0$ & $0$ & $0.0$\\
        VoiceMail1 & $4$ & $0$ & $4$ & $0$ & $0$ & $0$ & $0$ & $0.0$\\
        \hline
        \tsubEight{FieldAndObjectSensitivityTest}
        FieldSensitivity1 & $25$ & $3$ & $25$ & $3$ & $0$ & $0$ & $0$ & $0.0$\\
        FieldSensitivity2 & $19$ & $0$ & $19$ & $0$ & $0$ & $0$ & $0$ & $0.0$\\
        FieldSensitivity3 & $16$ & $0$ & $16$ & $0$ & $0$ & $0$ & $0$ & $0.0$\\
        FieldSensitivity4 & $8$ & $0$ & $8$ & $0$ & $0$ & $0$ & $0$ & $0.0$\\
        InheritedObjects1 & $6$ & $0$ & $6$ & $0$ & $0$ & $0$ & $0$ & $0.0$\\
        ObjectSensitivity1 & $14$ & $1$ & $14$ & $1$ & $0$ & $0$ & $0$ & $0.0$\\
        ObjectSensitivity2 & $10$ & $0$ & $10$ & $0$ & $0$ & $0$ & $0$ & $0.0$\\
        \hline
        \tsubEight{GeneralJavaTest}
        Clone1 & $12$ & $4$ & $19$ & $10$ & $7$ & $6$ & $13$ & $0.81$\\
        Exceptions1 & $13$ & $0$ & $13$ & $0$ & $0$ & $0$ & $0$ & $0.0$\\
        Exceptions2 & $13$ & $0$ & $13$ & $0$ & $0$ & $0$ & $0$ & $0.0$\\
        Exceptions3 & $11$ & $0$ & $11$ & $0$ & $0$ & $0$ & $0$ & $0.0$\\
        Exceptions4 & $22$ & $0$ & $22$ & $0$ & $0$ & $0$ & $0$ & $0.0$\\
        Exceptions5 & $16$ & $0$ & $16$ & $0$ & $0$ & $0$ & $0$ & $0.0$\\
        Exceptions6 & $23$ & $0$ & $23$ & $0$ & $0$ & $0$ & $0$ & $0.0$\\
        Exceptions7 & $6$ & $0$ & $6$ & $0$ & $0$ & $0$ & $0$ & $0.0$\\
        FactoryMethods1 & $14$ & $0$ & $14$ & $0$ & $0$ & $0$ & $0$ & $0.0$\\
        Loop1 & $51$ & $0$ & $47$ & $0$ & $-4$ & $0$ & $-4$ & $-0.08$\\
        Loop2 & $79$ & $0$ & $75$ & $0$ & $-4$ & $0$ & $-4$ & $-0.05$\\
        Serialization1 & $22$ & $29$ & $332$ & $547$ & $310$ & $518$ & $828$ & $16.24$\\
        SourceCodeSpecific1 & $45$ & $7$ & $45$ & $7$ & $0$ & $0$ & $0$ & $0.0$\\
        StartProcessWithSecret1 & $17$ & $3$ & $18$ & $4$ & $1$ & $1$ & $2$ & $0.1$\\
        StaticInitialization1 & $9$ & $0$ & $9$ & $0$ & $0$ & $0$ & $0$ & $0.0$\\
        StaticInitialization2 & $86$ & $0$ & $86$ & $0$ & $0$ & $0$ & $0$ & $0.0$\\
        StaticInitialization3 & $5$ & $0$ & $5$ & $0$ & $0$ & $0$ & $0$ & $0.0$\\
        StringFormatter1 & $10$ & $0$ & $10$ & $0$ & $0$ & $0$ & $0$ & $0.0$\\
        StringPatternMatching1 & $8$ & $4$ & $7$ & $0$ & $-1$ & $-4$ & $-5$ & $-0.42$\\
        StringToCharArray1 & $47$ & $0$ & $43$ & $0$ & $-4$ & $0$ & $-4$ & $-0.09$\\
        StringToOutputStream1 & $25$ & $1$ & $24$ & $1$ & $-1$ & $0$ & $-1$ & $-0.04$\\
        UnreachableCode & $-$ & $-$ & $-$ & $-$ & $-$ & $-$ & $-$ & $-$\\
        VirtualDispatch1 & $88$ & $28$ & $110$ & $88$ & $22$ & $60$ & $82$ & $0.71$\\
        VirtualDispatch2 & $12$ & $0$ & $12$ & $0$ & $0$ & $0$ & $0$ & $0.0$\\
        VirtualDispatch3 & $6$ & $0$ & $6$ & $0$ & $0$ & $0$ & $0$ & $0.0$\\
        VirtualDispatch4 & $-$ & $-$ & $-$ & $-$ & $-$ & $-$ & $-$ & $-$\\
        \hline
        \tsubEight{LifecycleTest}
        ActivityEventSequence1 & $72$ & $0$ & $73$ & $0$ & $1$ & $0$ & $1$ & $0.01$\\
        ActivityEventSequence2 & $77$ & $0$ & $77$ & $0$ & $0$ & $0$ & $0$ & $0.0$\\
        ActivityEventSequence3 & $156$ & $0$ & $156$ & $0$ & $0$ & $0$ & $0$ & $0.0$\\
        ActivityLifecycle1 & $156$ & $7$ & $156$ & $7$ & $0$ & $0$ & $0$ & $0.0$\\
        ActivityLifecycle2 & $33$ & $0$ & $33$ & $0$ & $0$ & $0$ & $0$ & $0.0$\\
        ActivityLifecycle3 & $28$ & $0$ & $28$ & $0$ & $0$ & $0$ & $0$ & $0.0$\\
        ActivityLifecycle4 & $14$ & $0$ & $14$ & $0$ & $0$ & $0$ & $0$ & $0.0$\\
        ActivitySavedState1 & $7$ & $0$ & $7$ & $0$ & $0$ & $0$ & $0$ & $0.0$\\
        ApplicationLifecycle1 & $82$ & $0$ & $82$ & $0$ & $0$ & $0$ & $0$ & $0.0$\\
        ApplicationLifecycle2 & $94$ & $155$ & $94$ & $155$ & $0$ & $0$ & $0$ & $0.0$\\
        ApplicationLifecycle3 & $21$ & $0$ & $21$ & $0$ & $0$ & $0$ & $0$ & $0.0$\\
        AsynchronousEventOrdering1 & $16$ & $0$ & $16$ & $0$ & $0$ & $0$ & $0$ & $0.0$\\
        BroadcastReceiverLifecycle1 & $4$ & $0$ & $4$ & $0$ & $0$ & $0$ & $0$ & $0.0$\\
        BroadcastReceiverLifecycle2 & $248$ & $114$ & $208$ & $98$ & $-40$ & $-16$ & $-56$ & $-0.15$\\
        BroadcastReceiverLifecycle3 & $195$ & $110$ & $144$ & $82$ & $-51$ & $-28$ & $-79$ & $-0.26$\\
        EventOrdering1 & $30$ & $0$ & $30$ & $0$ & $0$ & $0$ & $0$ & $0.0$\\
        FragmentLifecycle1 & $90$ & $0$ & $90$ & $0$ & $0$ & $0$ & $0$ & $0.0$\\
        FragmentLifecycle2 & $-$ & $-$ & $-$ & $-$ & $-$ & $-$ & $-$ & $-$\\
        ServiceEventSequence1 & $124$ & $34$ & $122$ & $38$ & $-2$ & $4$ & $2$ & $0.01$\\
        ServiceEventSequence2 & $389$ & $220$ & $315$ & $176$ & $-74$ & $-44$ & $-118$ & $-0.19$\\
        ServiceEventSequence3 & $275$ & $151$ & $232$ & $110$ & $-43$ & $-41$ & $-84$ & $-0.2$\\
        ServiceLifecycle1 & $42$ & $0$ & $42$ & $0$ & $0$ & $0$ & $0$ & $0.0$\\
        ServiceLifecycle2 & $89$ & $21$ & $89$ & $21$ & $0$ & $0$ & $0$ & $0.0$\\
        SharedPreferenceChanged1 & $11$ & $0$ & $8$ & $0$ & $-3$ & $0$ & $-3$ & $-0.27$\\
        \hline
        \tsubEight{ReflectionTest}
        Reflection1 & $8$ & $0$ & $8$ & $0$ & $0$ & $0$ & $0$ & $0.0$\\
        Reflection2 & $11$ & $0$ & $11$ & $0$ & $0$ & $0$ & $0$ & $0.0$\\
        Reflection3 & $62$ & $25$ & $50$ & $0$ & $-12$ & $-25$ & $-37$ & $-0.43$\\
        Reflection4 & $8$ & $0$ & $8$ & $0$ & $0$ & $0$ & $0$ & $0.0$\\
        Reflection5 & $11$ & $0$ & $11$ & $0$ & $0$ & $0$ & $0$ & $0.0$\\
        Reflection6 & $134$ & $51$ & $122$ & $31$ & $-12$ & $-20$ & $-32$ & $-0.17$\\
        Reflection7 & $15$ & $11$ & $3$ & $0$ & $-12$ & $-11$ & $-23$ & $-0.88$\\
        Reflection8 & $14$ & $0$ & $14$ & $0$ & $0$ & $0$ & $0$ & $0.0$\\
        Reflection9 & $21$ & $0$ & $21$ & $0$ & $0$ & $0$ & $0$ & $0.0$\\
        \hline
        \tsubEight{ThreadingTest}
        AsyncTask1 & $11$ & $1$ & $11$ & $1$ & $0$ & $0$ & $0$ & $0.0$\\
        Executor1 & $17$ & $0$ & $17$ & $0$ & $0$ & $0$ & $0$ & $0.0$\\
        JavaThread1 & $17$ & $0$ & $17$ & $0$ & $0$ & $0$ & $0$ & $0.0$\\
        JavaThread2 & $31$ & $8$ & $28$ & $8$ & $-3$ & $0$ & $-3$ & $-0.08$\\
        Looper1 & $20$ & $16$ & $20$ & $16$ & $0$ & $0$ & $0$ & $0.0$\\
        TimerTask1 & $32$ & $33$ & $45$ & $37$ & $13$ & $4$ & $17$ & $0.26$\\
        \hhline
        $\varnothing$ Propagations & $60.56$ & $40.1$ & $57.29$ & $39.16$ & $-3.27$ & $-0.93$ & $-4.2$ & $0.13$\\
        $\varnothing$ without Serialization1& $60.88$ & $40.19$ & $55.04$ & $35.0$ & $-5.84$ & $-5.19$ & $-11.02$ & $0.0$\\
        \caption{\textsc{DroidBench} Evaluation with Summary Taint Wrapper}
        \label{t:droidbenchevaluation_sum}
    \end{longtable}
    \normalsize

    \section{Real World Apps}\label{s:realworld}

    \subsection{Configuration}
    Our test machine is equipped with four Intel Xeon E5-4650 and 1 TB of RAM.
    We limited the JVM to 50 GB RAM and \textsc{FlowDroid} on 16 threads per instance.
    We ran at most four instances in parallel to ensure a one-to-one mapping between CPU threads and \textsc{FlowDroid} threads.
    Note that the test machine is a shared system, but we made sure there are always enough resources for our evaluation available.
    Still, background services might influence the performance of a single run. To stamp out this factor, we ran each app three times with a distance of time\footnotemark.
    \footnotetext{The time distance between each run is at least the elapsed time from the analysis of the remaining 199 apps.}
    If there were outliers\footnotemark{}, we repeated the run.
    \footnote{Outliers are runs with at least $25\%$ difference to the median run and a minimum of $5$ seconds absolute difference.}
    Some runs did not comply to our outlier norm even after we ran them multiple times, but this only concers 9 out of 600 runs.

    We also measured the memory usage of both implementations.
    Using the memory amount reported by the JVM is not precise because the JVM prefers to take up free memory before running the garbage collector \cite{Arzt2017PhD}.
    We borrowed the memory evaluation tool from CleanDroid, which internally depends on a memory calculation tool from Twitter\footnote{\url{https://mvnrepository.com/artifact/com.twitter.common/objectsize} (visited on 18.04.2021)}.
    The memory evaluation tool measures the size of the exploded supergraph in 15 seconds intervals \cite{Arzt2021}.
    Because we do not want to pollute the measured data flow time with the memory evaluation tool's latency, the memory measuring runs were run independently of the time measuring runs.
    The memory sampling also takes up memory and because our test system has enough memory available, we bumped the maximum heap size up to 100GB, effectively eliminating memory timeouts.

    For this evaluation, we chose to use a non-default configuration of \textsc{FlowDroid}.
    First, we disabled static field tracking due to the global scope as described in \autoref{s:complexity}.
    Next, instead of the \code{EasyTaintWrapper}, we use the \code{SummaryTaintWrapper}, which utilizes StubDroid.
    We set the timeout for the data flow analysis to 10 minutes\footnotemark{}.
    \footnotetext{A timeout in FlowDroid prevents processing new edges but lets the solver finish the current edge propagation. Thus, some apps may have a data flow time of above 600 seconds.}
    The call graph generation was limited to 180 seconds and the call-graphs were serialized before, so every run was on the same call-graph.
    The configuration summary is in \autoref{t:realworldconfig}.

    \begin{table}[tbp]
        \centering
        \begin{tabular}{l | l}
            \textbf{Option} & \textbf{Value}\\
            \hline\hline
            Array Size Tainting & disabled\\
            Inspect Sources \& Sinks & disabled\\
            Static Field Tracking & disabled\\
            Ignore Flows in System Packages & enabled\\
            Exclude Soot Library Classes & enabled\\
            Timeout & 10 minutes\\
            Taint Wrapper & \code{SummaryTaintWrapper}\\
        \end{tabular}
        \caption{Real World Apps Configuration}
        \label{t:realworldconfig}
    \end{table}

    We did not use the full sources and sinks list included in \textsc{FlowDroid} because such would result in hundreds of sources and sinks per app and probably a long runtime.
    Instead, we chose to analyze which sensitive and possibly user-identifying data is sent out to the internet.
    As we want to compare the forwards and backward implementation, it is also essential to not put one at a disadvantage.
    We opted for a 2:1 ratio of sources to sinks.
    This decision is based on the results of SuSi, to find sources and sinks in the Android framework automatically \cite{Rasthofer2014}.
    Their extracted list of sources and sinks contains roughly $2.17$ times more sources than sinks.
    The list of sources and sinks used in this evaluation is in \autoref{t:realworldsources} and \autoref{t:realworldsinks}.

    \begin{table}[tbp]
        \centering
        \begin{tabular}{l | l}
            \textbf{Class} & \textbf{Method}\\
            \hline\hline
            android.location.Location & getLatitude()\\
            & getLongitude()\\
            \hline
            android.location.LocationManager & getLastKnownLocation()\\
            \hline
            android.telephony.TelephonyManager & getDeviceId()\\
            & getSubscriberId()\\
            & getSimSerialNumber()\\
            & getLine1Number()\\
            & getImei()\\
            & getMeid()\\
            \hline
            android.bluetooth.BluetoothAdapter & getAddress()\\
            android.net.wifi.WifiInfo & getMacAddress()\\
            & getSSID()\\
            & getIpAddress()\\
            \hline
            java.net.InetAddress & getHostAddress()\\
            \hline
            android.telephony.gsm.GsmCellLocation & getCid()\\
            & getLac()\\
            \hline
            android.content.pm.PackageManager & getInstalledApplications()\\
            & getInstalledPackages()\\
            & queryIntentActivities()\\
            & queryIntentServices()\\
            & queryBroadcastReceivers()\\
            \hline
            android.content.SharedPreferences & getDefaultSharedPreferences()\\
            android.provider.Browser & getAllBookmarks()\\
            & getAllVisitedUrls\\
        \end{tabular}
        \caption{Sources for Real World Apps Evaluation}
        \label{t:realworldsources}
    \end{table}

    \begin{table}[tbp]
        \centering
        \begin{tabular}{l | l}
            \textbf{Class} & \textbf{Method}\\
            \hline\hline
            java.net.URL & set()\\
            & openConnection()\\
            \hline
            java.net.URLConnection & connect()\\
            & setRequestProperty()\\
            \hline
            android.net.http.HttpsConnection & openConnection()\\
            \hline
            android.net.http.Headers & setEtag()\\
            & setContentType()\\
            & setLastModified()\\
            & setLocation()\\
            \hline
            android.net.http.AndroidHttpClientConnection & sendRequestHeader()\\
            \hline
            android.net.http.RequestQueue & queueRequest()\\
        \end{tabular}
        \caption{Sinks for Real World Apps Evaluation}
        \label{t:realworldsinks}
    \end{table}

    We used \textsc{FlowDroid}'s forward implementation on the to that date latest upstream commit\footnote{The latest upstream commit was at that time \footnotecode{b436733fc4a5130dfe4ce8ddb3f76fd374e9a487}.} from the develop branch for the point of comparison.
    The backward implementation ran on our latest commit\footnotemark{} at that time with all changes from the upstream merged into.
    \footnotetext{Our latest commit was \footnotecode{87bf33ba40ef8b4fb25f33439d887ebc98c2c184}. Note that during the real-world evaluation we found some bugs and also later on fixed some edge cases in the analysis. All fixes should not influence the runtime in a bad way.}

    We chose 200 apps randomly out of a Google Playstore dump from 2021 containing over 6000 apps for our evaluation set.
    Out of 200 apps, 60 apps do not have any sources or sinks and thus, the analysis did not start.
    For six apps, the analysis aborted with errors on at least one run. All thrown exceptions happened outside of \textsc{FlowDroid}.
    We are left with 131 apps for which both implementations completed all runs without errors.
    The full list is appended to this work in \autoref{a:appset}.

    \FloatBarrier
    \subsection{Time Evaluation}
    In general, the individual apps' runtimes were far apart from each other.
    We had many apps with a single-digit analysis time and on the other side, we also found many apps that triggered a timeout or were close to triggering one.
    In between those extrema are only a few apps.
    Recall that we set a soft limit on the runtime at 600 seconds.
    The reference forward runs have a standard deviation of $209$ seconds and the runs of our implementation has $277$ seconds standard deviation.
    It is important to keep this in mind when interpreting the results.

    We first begin with an overview of the results.
    \autoref{t:realworldresults} shows the results, including timeouts.
    Notably, the backward analysis had $20\%$ less time timeouts than the forward analysis.
    In return, it seems a bit more memory-hungry with $3.63\%$ more memory timeouts.
    We conducted a t-test to check the significance of those differences with the null hypothesis of equal average expected values.
    The p-value for the memory timeouts is $0.156$, thus being insignificant.
    A t-test over the runtime yielded a p-value of $0.00036$, meaning the advantage for our implementation is significant.
    We cover the memory consumption extensively in the \hyperref[s:memex]{next subsection} and focus on the time for now.
    Interestingly, the propagated edges along the same interprocedural call-graph are of the same order of magnitude.
    Also, the 85\textsuperscript{th} percentile runtime is nearly equal and the median is equal.
    However, claims based on the runtime and edges with timeouts are only possible to a limited extent because the timeout highly influences both values.

    \begin{table}[tbp]
        \centering
        \begin{tabular}{l | r | r | r}
            & \multicolumn{3}{c}{\textbf{Forward}}\\
            \textbf{Metric} & \textbf{Avg} & \textbf{Median} & $\mathbf{P_{85}}$\\
            \hline\hline
            Data Flow Time & $518.93s$ & $600.00s$ & $605.10s$\\
            \hline
            Edge Propagations Infoflow & $34555326.97$ & $41743088.00$ & $52163969.60$\\
            Edge Propagations Alias & $12562479.07$ & $14598571.50$ & $18638900.10$\\
            Total Edge Propagations & $47117806.04$ & $57697027.00$ & $70602469.30$\\
            \hline
            Memory Timeouts & \multicolumn{3}{r}{$2.99\%$}\\
            Time Timeouts & \multicolumn{3}{r}{$80.60\%$}\\
            \multicolumn{4}{c}{}\\
            & \multicolumn{3}{c}{\textbf{Backward}}\\
            \textbf{Metric} & \textbf{Avg} & \textbf{Median} & $\mathbf{P_{85}}$\\
            \hline\hline
            Data Flow Time & $413.19s$ & $600.00s$ & $606.00s$\\
            \hline
            Edge Propagations Infoflow & $13826566.90$ & $14981108.50$ & $23712802.00$\\
            Edge Propagations Alias & $33567561.46$ & $43444060.00$ & $56773141.00$\\
            Total Edge Propagations & $47394128.36$ & $60855935.50$ & $79405729.00$\\
            \hline
            Memory Timeouts & \multicolumn{3}{r}{$6.62\%$}\\
            Time Timeouts & \multicolumn{3}{r}{$59.56\%$}\\
        \end{tabular}
        \caption{Results With Timeouts}
        \label{t:realworldresults}
    \end{table}

    Next, we only consider the runs without any timeouts in \autoref{t:realworldresultswithouttimeout}.
    This time we can still observe a relation between backward infoflow edges and forward alias edges even though to a lesser extent.
    More significant, backward needed way less forward propagations either because fewer aliases were on the path or the alias analysis could be stopped earlier due to a near turn unit.
    The runtimes also represent this fact. In the 85\textsuperscript{th} percentile, both analyses are more close than the average suggest, with the backward analysis needing $2.25$ seconds less.
    The median here renders useless as a comparison point because of the huge variance in the data set.

    \begin{table}[tbp]
        \centering
        \begin{tabular}{l | r | r | r}
            & \multicolumn{3}{c}{\textbf{Forward}}\\
            \textbf{Metric} & \textbf{Avg} & \textbf{Median} & $\mathbf{P_{85}}$\\
            \hline\hline
            Data Flow Time & $76.91s$ & $0.00s$ & $108.05s$\\
            \hline
            Edge Propagations Infoflow & $5651572.23$ & $28692.00$ & $7164554.30$\\
            Edge Propagations Alias & $1864306.77$ & $8689.50$ & $4212872.40$\\
            Total Edge Propagations & $7515879.00$ & $37310.00$ & $11377426.70$\\
            \multicolumn{4}{c}{}\\
            & \multicolumn{3}{c}{\textbf{Backward}}\\
            \textbf{Metric} & \textbf{Avg} & \textbf{Median} & $\mathbf{P_{85}}$\\
            \hline\hline
            Data Flow Time & $35.20s$ & $0.00s$ & $39.25s$\\
            \hline
            Edge Propagations Infoflow & $2353723.61$ & $19940.50$ & $1733519.75$\\
            Edge Propagations Alias & $3795159.13$ & $20013.50$ & $6883734.25$\\
            Total Edge Propagations & $6148882.74$ & $39954.00$ & $8692988.50$\\
        \end{tabular}
        \caption{Results without Timeouts}
        \label{t:realworldresultswithouttimeout}
    \end{table}

    A knowledgeable reader might have noticed the results in \autoref{t:realworldresults} are worse than in previous publications where \textsc{FlowDroid} was evaluated \cite{Arzt2017PhD, Arzt2021}.
    We want to emphasize that none of our changes did influence the reference runs in a bad way as we used the upstream version without a single line changed to conduct the first run\footnotemark{}.
    \footnotetext{We found some exceptions in the upstream project while evaluating, so after the first run we switched to our version with the fixes included. This did not change the results we got.}
    The existing implementation suffers as ours, so we suspect it partly depends on an unfortunatly drawn app set and further development in the call-graph generation leading to more possible edges.

    With that out of the way, let us look at the results in greater detail. We now compare the analysis on a per-app basis.
    The histogram is in \autoref{f:deltaHist}.
    We compiled the delta data flow time of the analyses per app, calculated as in the last section with the forward implementation being the reference: $t_{\mathit{Backward}} - t_{\mathit{Forward}}$.
    Hence, negative values represent that our implementation performed better.
    The delta on the x-axis is given in seconds and the frequency on the y-axis in number of apps.
    The bins always span over $50$ seconds.
    The graph shows a large number of apps around $0$ with a slight bias towards the forward implementation.
    Equivalent to the distribution of the data flow times, there are only few deltas in the range from $\pm100$ to $\pm500$.
    More interestingly, there are significantly more apps around $-600$ than around $600$.
    Recall, the timeout is set to $600s$.
    So, our implementation terminates nearly instantaneous in some cases on which the forward analysis times out.
    As expected, there is no general advantage for a direction.
    Instead, we observe a per-app advantage in around $60\%$ of the test set, while for the rest, the performance is similar.

    \begin{figure}[tbp]
        \centering
        \resizebox{0.75\columnwidth}{!}{
            %% Creator: Matplotlib, PGF backend
%%
%% To include the figure in your LaTeX document, write
%%   \input{<filename>.pgf}
%%
%% Make sure the required packages are loaded in your preamble
%%   \usepackage{pgf}
%%
%% and, on pdftex
%%   \usepackage[utf8]{inputenc}\DeclareUnicodeCharacter{2212}{-}
%%
%% or, on luatex and xetex
%%   \usepackage{unicode-math}
%%
%% Figures using additional raster images can only be included by \input if
%% they are in the same directory as the main LaTeX file. For loading figures
%% from other directories you can use the `import` package
%%   \usepackage{import}
%%
%% and then include the figures with
%%   \import{<path to file>}{<filename>.pgf}
%%
%% Matplotlib used the following preamble
%%   \usepackage{amsmath}
%%   \usepackage{fontspec}
%%
\begingroup%
\makeatletter%
\begin{pgfpicture}%
\pgfpathrectangle{\pgfpointorigin}{\pgfqpoint{6.000000in}{4.000000in}}%
\pgfusepath{use as bounding box, clip}%
\begin{pgfscope}%
\pgfsetbuttcap%
\pgfsetmiterjoin%
\definecolor{currentfill}{rgb}{1.000000,1.000000,1.000000}%
\pgfsetfillcolor{currentfill}%
\pgfsetlinewidth{0.000000pt}%
\definecolor{currentstroke}{rgb}{1.000000,1.000000,1.000000}%
\pgfsetstrokecolor{currentstroke}%
\pgfsetdash{}{0pt}%
\pgfpathmoveto{\pgfqpoint{0.000000in}{0.000000in}}%
\pgfpathlineto{\pgfqpoint{6.000000in}{0.000000in}}%
\pgfpathlineto{\pgfqpoint{6.000000in}{4.000000in}}%
\pgfpathlineto{\pgfqpoint{0.000000in}{4.000000in}}%
\pgfpathclose%
\pgfusepath{fill}%
\end{pgfscope}%
\begin{pgfscope}%
\pgfsetbuttcap%
\pgfsetmiterjoin%
\definecolor{currentfill}{rgb}{1.000000,1.000000,1.000000}%
\pgfsetfillcolor{currentfill}%
\pgfsetlinewidth{0.000000pt}%
\definecolor{currentstroke}{rgb}{0.000000,0.000000,0.000000}%
\pgfsetstrokecolor{currentstroke}%
\pgfsetstrokeopacity{0.000000}%
\pgfsetdash{}{0pt}%
\pgfpathmoveto{\pgfqpoint{0.750000in}{0.500000in}}%
\pgfpathlineto{\pgfqpoint{5.400000in}{0.500000in}}%
\pgfpathlineto{\pgfqpoint{5.400000in}{3.520000in}}%
\pgfpathlineto{\pgfqpoint{0.750000in}{3.520000in}}%
\pgfpathclose%
\pgfusepath{fill}%
\end{pgfscope}%
\begin{pgfscope}%
\pgfpathrectangle{\pgfqpoint{0.750000in}{0.500000in}}{\pgfqpoint{4.650000in}{3.020000in}}%
\pgfusepath{clip}%
\pgfsetbuttcap%
\pgfsetmiterjoin%
\definecolor{currentfill}{rgb}{0.121569,0.466667,0.705882}%
\pgfsetfillcolor{currentfill}%
\pgfsetlinewidth{1.003750pt}%
\definecolor{currentstroke}{rgb}{0.000000,0.000000,0.000000}%
\pgfsetstrokecolor{currentstroke}%
\pgfsetdash{}{0pt}%
\pgfpathmoveto{\pgfqpoint{0.961364in}{0.500000in}}%
\pgfpathlineto{\pgfqpoint{1.130455in}{0.500000in}}%
\pgfpathlineto{\pgfqpoint{1.130455in}{1.092157in}}%
\pgfpathlineto{\pgfqpoint{0.961364in}{1.092157in}}%
\pgfpathclose%
\pgfusepath{stroke,fill}%
\end{pgfscope}%
\begin{pgfscope}%
\pgfpathrectangle{\pgfqpoint{0.750000in}{0.500000in}}{\pgfqpoint{4.650000in}{3.020000in}}%
\pgfusepath{clip}%
\pgfsetbuttcap%
\pgfsetmiterjoin%
\definecolor{currentfill}{rgb}{0.121569,0.466667,0.705882}%
\pgfsetfillcolor{currentfill}%
\pgfsetlinewidth{1.003750pt}%
\definecolor{currentstroke}{rgb}{0.000000,0.000000,0.000000}%
\pgfsetstrokecolor{currentstroke}%
\pgfsetdash{}{0pt}%
\pgfpathmoveto{\pgfqpoint{1.130455in}{0.500000in}}%
\pgfpathlineto{\pgfqpoint{1.299545in}{0.500000in}}%
\pgfpathlineto{\pgfqpoint{1.299545in}{1.049860in}}%
\pgfpathlineto{\pgfqpoint{1.130455in}{1.049860in}}%
\pgfpathclose%
\pgfusepath{stroke,fill}%
\end{pgfscope}%
\begin{pgfscope}%
\pgfpathrectangle{\pgfqpoint{0.750000in}{0.500000in}}{\pgfqpoint{4.650000in}{3.020000in}}%
\pgfusepath{clip}%
\pgfsetbuttcap%
\pgfsetmiterjoin%
\definecolor{currentfill}{rgb}{0.121569,0.466667,0.705882}%
\pgfsetfillcolor{currentfill}%
\pgfsetlinewidth{1.003750pt}%
\definecolor{currentstroke}{rgb}{0.000000,0.000000,0.000000}%
\pgfsetstrokecolor{currentstroke}%
\pgfsetdash{}{0pt}%
\pgfpathmoveto{\pgfqpoint{1.299545in}{0.500000in}}%
\pgfpathlineto{\pgfqpoint{1.468636in}{0.500000in}}%
\pgfpathlineto{\pgfqpoint{1.468636in}{0.542297in}}%
\pgfpathlineto{\pgfqpoint{1.299545in}{0.542297in}}%
\pgfpathclose%
\pgfusepath{stroke,fill}%
\end{pgfscope}%
\begin{pgfscope}%
\pgfpathrectangle{\pgfqpoint{0.750000in}{0.500000in}}{\pgfqpoint{4.650000in}{3.020000in}}%
\pgfusepath{clip}%
\pgfsetbuttcap%
\pgfsetmiterjoin%
\definecolor{currentfill}{rgb}{0.121569,0.466667,0.705882}%
\pgfsetfillcolor{currentfill}%
\pgfsetlinewidth{1.003750pt}%
\definecolor{currentstroke}{rgb}{0.000000,0.000000,0.000000}%
\pgfsetstrokecolor{currentstroke}%
\pgfsetdash{}{0pt}%
\pgfpathmoveto{\pgfqpoint{1.468636in}{0.500000in}}%
\pgfpathlineto{\pgfqpoint{1.637727in}{0.500000in}}%
\pgfpathlineto{\pgfqpoint{1.637727in}{0.584594in}}%
\pgfpathlineto{\pgfqpoint{1.468636in}{0.584594in}}%
\pgfpathclose%
\pgfusepath{stroke,fill}%
\end{pgfscope}%
\begin{pgfscope}%
\pgfpathrectangle{\pgfqpoint{0.750000in}{0.500000in}}{\pgfqpoint{4.650000in}{3.020000in}}%
\pgfusepath{clip}%
\pgfsetbuttcap%
\pgfsetmiterjoin%
\definecolor{currentfill}{rgb}{0.121569,0.466667,0.705882}%
\pgfsetfillcolor{currentfill}%
\pgfsetlinewidth{1.003750pt}%
\definecolor{currentstroke}{rgb}{0.000000,0.000000,0.000000}%
\pgfsetstrokecolor{currentstroke}%
\pgfsetdash{}{0pt}%
\pgfpathmoveto{\pgfqpoint{1.637727in}{0.500000in}}%
\pgfpathlineto{\pgfqpoint{1.806818in}{0.500000in}}%
\pgfpathlineto{\pgfqpoint{1.806818in}{0.542297in}}%
\pgfpathlineto{\pgfqpoint{1.637727in}{0.542297in}}%
\pgfpathclose%
\pgfusepath{stroke,fill}%
\end{pgfscope}%
\begin{pgfscope}%
\pgfpathrectangle{\pgfqpoint{0.750000in}{0.500000in}}{\pgfqpoint{4.650000in}{3.020000in}}%
\pgfusepath{clip}%
\pgfsetbuttcap%
\pgfsetmiterjoin%
\definecolor{currentfill}{rgb}{0.121569,0.466667,0.705882}%
\pgfsetfillcolor{currentfill}%
\pgfsetlinewidth{1.003750pt}%
\definecolor{currentstroke}{rgb}{0.000000,0.000000,0.000000}%
\pgfsetstrokecolor{currentstroke}%
\pgfsetdash{}{0pt}%
\pgfpathmoveto{\pgfqpoint{1.806818in}{0.500000in}}%
\pgfpathlineto{\pgfqpoint{1.975909in}{0.500000in}}%
\pgfpathlineto{\pgfqpoint{1.975909in}{0.500000in}}%
\pgfpathlineto{\pgfqpoint{1.806818in}{0.500000in}}%
\pgfpathclose%
\pgfusepath{stroke,fill}%
\end{pgfscope}%
\begin{pgfscope}%
\pgfpathrectangle{\pgfqpoint{0.750000in}{0.500000in}}{\pgfqpoint{4.650000in}{3.020000in}}%
\pgfusepath{clip}%
\pgfsetbuttcap%
\pgfsetmiterjoin%
\definecolor{currentfill}{rgb}{0.121569,0.466667,0.705882}%
\pgfsetfillcolor{currentfill}%
\pgfsetlinewidth{1.003750pt}%
\definecolor{currentstroke}{rgb}{0.000000,0.000000,0.000000}%
\pgfsetstrokecolor{currentstroke}%
\pgfsetdash{}{0pt}%
\pgfpathmoveto{\pgfqpoint{1.975909in}{0.500000in}}%
\pgfpathlineto{\pgfqpoint{2.145000in}{0.500000in}}%
\pgfpathlineto{\pgfqpoint{2.145000in}{0.500000in}}%
\pgfpathlineto{\pgfqpoint{1.975909in}{0.500000in}}%
\pgfpathclose%
\pgfusepath{stroke,fill}%
\end{pgfscope}%
\begin{pgfscope}%
\pgfpathrectangle{\pgfqpoint{0.750000in}{0.500000in}}{\pgfqpoint{4.650000in}{3.020000in}}%
\pgfusepath{clip}%
\pgfsetbuttcap%
\pgfsetmiterjoin%
\definecolor{currentfill}{rgb}{0.121569,0.466667,0.705882}%
\pgfsetfillcolor{currentfill}%
\pgfsetlinewidth{1.003750pt}%
\definecolor{currentstroke}{rgb}{0.000000,0.000000,0.000000}%
\pgfsetstrokecolor{currentstroke}%
\pgfsetdash{}{0pt}%
\pgfpathmoveto{\pgfqpoint{2.145000in}{0.500000in}}%
\pgfpathlineto{\pgfqpoint{2.314091in}{0.500000in}}%
\pgfpathlineto{\pgfqpoint{2.314091in}{0.500000in}}%
\pgfpathlineto{\pgfqpoint{2.145000in}{0.500000in}}%
\pgfpathclose%
\pgfusepath{stroke,fill}%
\end{pgfscope}%
\begin{pgfscope}%
\pgfpathrectangle{\pgfqpoint{0.750000in}{0.500000in}}{\pgfqpoint{4.650000in}{3.020000in}}%
\pgfusepath{clip}%
\pgfsetbuttcap%
\pgfsetmiterjoin%
\definecolor{currentfill}{rgb}{0.121569,0.466667,0.705882}%
\pgfsetfillcolor{currentfill}%
\pgfsetlinewidth{1.003750pt}%
\definecolor{currentstroke}{rgb}{0.000000,0.000000,0.000000}%
\pgfsetstrokecolor{currentstroke}%
\pgfsetdash{}{0pt}%
\pgfpathmoveto{\pgfqpoint{2.314091in}{0.500000in}}%
\pgfpathlineto{\pgfqpoint{2.483182in}{0.500000in}}%
\pgfpathlineto{\pgfqpoint{2.483182in}{0.500000in}}%
\pgfpathlineto{\pgfqpoint{2.314091in}{0.500000in}}%
\pgfpathclose%
\pgfusepath{stroke,fill}%
\end{pgfscope}%
\begin{pgfscope}%
\pgfpathrectangle{\pgfqpoint{0.750000in}{0.500000in}}{\pgfqpoint{4.650000in}{3.020000in}}%
\pgfusepath{clip}%
\pgfsetbuttcap%
\pgfsetmiterjoin%
\definecolor{currentfill}{rgb}{0.121569,0.466667,0.705882}%
\pgfsetfillcolor{currentfill}%
\pgfsetlinewidth{1.003750pt}%
\definecolor{currentstroke}{rgb}{0.000000,0.000000,0.000000}%
\pgfsetstrokecolor{currentstroke}%
\pgfsetdash{}{0pt}%
\pgfpathmoveto{\pgfqpoint{2.483182in}{0.500000in}}%
\pgfpathlineto{\pgfqpoint{2.652273in}{0.500000in}}%
\pgfpathlineto{\pgfqpoint{2.652273in}{0.500000in}}%
\pgfpathlineto{\pgfqpoint{2.483182in}{0.500000in}}%
\pgfpathclose%
\pgfusepath{stroke,fill}%
\end{pgfscope}%
\begin{pgfscope}%
\pgfpathrectangle{\pgfqpoint{0.750000in}{0.500000in}}{\pgfqpoint{4.650000in}{3.020000in}}%
\pgfusepath{clip}%
\pgfsetbuttcap%
\pgfsetmiterjoin%
\definecolor{currentfill}{rgb}{0.121569,0.466667,0.705882}%
\pgfsetfillcolor{currentfill}%
\pgfsetlinewidth{1.003750pt}%
\definecolor{currentstroke}{rgb}{0.000000,0.000000,0.000000}%
\pgfsetstrokecolor{currentstroke}%
\pgfsetdash{}{0pt}%
\pgfpathmoveto{\pgfqpoint{2.652273in}{0.500000in}}%
\pgfpathlineto{\pgfqpoint{2.821364in}{0.500000in}}%
\pgfpathlineto{\pgfqpoint{2.821364in}{0.542297in}}%
\pgfpathlineto{\pgfqpoint{2.652273in}{0.542297in}}%
\pgfpathclose%
\pgfusepath{stroke,fill}%
\end{pgfscope}%
\begin{pgfscope}%
\pgfpathrectangle{\pgfqpoint{0.750000in}{0.500000in}}{\pgfqpoint{4.650000in}{3.020000in}}%
\pgfusepath{clip}%
\pgfsetbuttcap%
\pgfsetmiterjoin%
\definecolor{currentfill}{rgb}{0.121569,0.466667,0.705882}%
\pgfsetfillcolor{currentfill}%
\pgfsetlinewidth{1.003750pt}%
\definecolor{currentstroke}{rgb}{0.000000,0.000000,0.000000}%
\pgfsetstrokecolor{currentstroke}%
\pgfsetdash{}{0pt}%
\pgfpathmoveto{\pgfqpoint{2.821364in}{0.500000in}}%
\pgfpathlineto{\pgfqpoint{2.990455in}{0.500000in}}%
\pgfpathlineto{\pgfqpoint{2.990455in}{0.626891in}}%
\pgfpathlineto{\pgfqpoint{2.821364in}{0.626891in}}%
\pgfpathclose%
\pgfusepath{stroke,fill}%
\end{pgfscope}%
\begin{pgfscope}%
\pgfpathrectangle{\pgfqpoint{0.750000in}{0.500000in}}{\pgfqpoint{4.650000in}{3.020000in}}%
\pgfusepath{clip}%
\pgfsetbuttcap%
\pgfsetmiterjoin%
\definecolor{currentfill}{rgb}{0.121569,0.466667,0.705882}%
\pgfsetfillcolor{currentfill}%
\pgfsetlinewidth{1.003750pt}%
\definecolor{currentstroke}{rgb}{0.000000,0.000000,0.000000}%
\pgfsetstrokecolor{currentstroke}%
\pgfsetdash{}{0pt}%
\pgfpathmoveto{\pgfqpoint{2.990455in}{0.500000in}}%
\pgfpathlineto{\pgfqpoint{3.159545in}{0.500000in}}%
\pgfpathlineto{\pgfqpoint{3.159545in}{1.261345in}}%
\pgfpathlineto{\pgfqpoint{2.990455in}{1.261345in}}%
\pgfpathclose%
\pgfusepath{stroke,fill}%
\end{pgfscope}%
\begin{pgfscope}%
\pgfpathrectangle{\pgfqpoint{0.750000in}{0.500000in}}{\pgfqpoint{4.650000in}{3.020000in}}%
\pgfusepath{clip}%
\pgfsetbuttcap%
\pgfsetmiterjoin%
\definecolor{currentfill}{rgb}{0.121569,0.466667,0.705882}%
\pgfsetfillcolor{currentfill}%
\pgfsetlinewidth{1.003750pt}%
\definecolor{currentstroke}{rgb}{0.000000,0.000000,0.000000}%
\pgfsetstrokecolor{currentstroke}%
\pgfsetdash{}{0pt}%
\pgfpathmoveto{\pgfqpoint{3.159545in}{0.500000in}}%
\pgfpathlineto{\pgfqpoint{3.328636in}{0.500000in}}%
\pgfpathlineto{\pgfqpoint{3.328636in}{3.376190in}}%
\pgfpathlineto{\pgfqpoint{3.159545in}{3.376190in}}%
\pgfpathclose%
\pgfusepath{stroke,fill}%
\end{pgfscope}%
\begin{pgfscope}%
\pgfpathrectangle{\pgfqpoint{0.750000in}{0.500000in}}{\pgfqpoint{4.650000in}{3.020000in}}%
\pgfusepath{clip}%
\pgfsetbuttcap%
\pgfsetmiterjoin%
\definecolor{currentfill}{rgb}{0.121569,0.466667,0.705882}%
\pgfsetfillcolor{currentfill}%
\pgfsetlinewidth{1.003750pt}%
\definecolor{currentstroke}{rgb}{0.000000,0.000000,0.000000}%
\pgfsetstrokecolor{currentstroke}%
\pgfsetdash{}{0pt}%
\pgfpathmoveto{\pgfqpoint{3.328636in}{0.500000in}}%
\pgfpathlineto{\pgfqpoint{3.497727in}{0.500000in}}%
\pgfpathlineto{\pgfqpoint{3.497727in}{0.669188in}}%
\pgfpathlineto{\pgfqpoint{3.328636in}{0.669188in}}%
\pgfpathclose%
\pgfusepath{stroke,fill}%
\end{pgfscope}%
\begin{pgfscope}%
\pgfpathrectangle{\pgfqpoint{0.750000in}{0.500000in}}{\pgfqpoint{4.650000in}{3.020000in}}%
\pgfusepath{clip}%
\pgfsetbuttcap%
\pgfsetmiterjoin%
\definecolor{currentfill}{rgb}{0.121569,0.466667,0.705882}%
\pgfsetfillcolor{currentfill}%
\pgfsetlinewidth{1.003750pt}%
\definecolor{currentstroke}{rgb}{0.000000,0.000000,0.000000}%
\pgfsetstrokecolor{currentstroke}%
\pgfsetdash{}{0pt}%
\pgfpathmoveto{\pgfqpoint{3.497727in}{0.500000in}}%
\pgfpathlineto{\pgfqpoint{3.666818in}{0.500000in}}%
\pgfpathlineto{\pgfqpoint{3.666818in}{0.500000in}}%
\pgfpathlineto{\pgfqpoint{3.497727in}{0.500000in}}%
\pgfpathclose%
\pgfusepath{stroke,fill}%
\end{pgfscope}%
\begin{pgfscope}%
\pgfpathrectangle{\pgfqpoint{0.750000in}{0.500000in}}{\pgfqpoint{4.650000in}{3.020000in}}%
\pgfusepath{clip}%
\pgfsetbuttcap%
\pgfsetmiterjoin%
\definecolor{currentfill}{rgb}{0.121569,0.466667,0.705882}%
\pgfsetfillcolor{currentfill}%
\pgfsetlinewidth{1.003750pt}%
\definecolor{currentstroke}{rgb}{0.000000,0.000000,0.000000}%
\pgfsetstrokecolor{currentstroke}%
\pgfsetdash{}{0pt}%
\pgfpathmoveto{\pgfqpoint{3.666818in}{0.500000in}}%
\pgfpathlineto{\pgfqpoint{3.835909in}{0.500000in}}%
\pgfpathlineto{\pgfqpoint{3.835909in}{0.500000in}}%
\pgfpathlineto{\pgfqpoint{3.666818in}{0.500000in}}%
\pgfpathclose%
\pgfusepath{stroke,fill}%
\end{pgfscope}%
\begin{pgfscope}%
\pgfpathrectangle{\pgfqpoint{0.750000in}{0.500000in}}{\pgfqpoint{4.650000in}{3.020000in}}%
\pgfusepath{clip}%
\pgfsetbuttcap%
\pgfsetmiterjoin%
\definecolor{currentfill}{rgb}{0.121569,0.466667,0.705882}%
\pgfsetfillcolor{currentfill}%
\pgfsetlinewidth{1.003750pt}%
\definecolor{currentstroke}{rgb}{0.000000,0.000000,0.000000}%
\pgfsetstrokecolor{currentstroke}%
\pgfsetdash{}{0pt}%
\pgfpathmoveto{\pgfqpoint{3.835909in}{0.500000in}}%
\pgfpathlineto{\pgfqpoint{4.005000in}{0.500000in}}%
\pgfpathlineto{\pgfqpoint{4.005000in}{0.500000in}}%
\pgfpathlineto{\pgfqpoint{3.835909in}{0.500000in}}%
\pgfpathclose%
\pgfusepath{stroke,fill}%
\end{pgfscope}%
\begin{pgfscope}%
\pgfpathrectangle{\pgfqpoint{0.750000in}{0.500000in}}{\pgfqpoint{4.650000in}{3.020000in}}%
\pgfusepath{clip}%
\pgfsetbuttcap%
\pgfsetmiterjoin%
\definecolor{currentfill}{rgb}{0.121569,0.466667,0.705882}%
\pgfsetfillcolor{currentfill}%
\pgfsetlinewidth{1.003750pt}%
\definecolor{currentstroke}{rgb}{0.000000,0.000000,0.000000}%
\pgfsetstrokecolor{currentstroke}%
\pgfsetdash{}{0pt}%
\pgfpathmoveto{\pgfqpoint{4.005000in}{0.500000in}}%
\pgfpathlineto{\pgfqpoint{4.174091in}{0.500000in}}%
\pgfpathlineto{\pgfqpoint{4.174091in}{0.500000in}}%
\pgfpathlineto{\pgfqpoint{4.005000in}{0.500000in}}%
\pgfpathclose%
\pgfusepath{stroke,fill}%
\end{pgfscope}%
\begin{pgfscope}%
\pgfpathrectangle{\pgfqpoint{0.750000in}{0.500000in}}{\pgfqpoint{4.650000in}{3.020000in}}%
\pgfusepath{clip}%
\pgfsetbuttcap%
\pgfsetmiterjoin%
\definecolor{currentfill}{rgb}{0.121569,0.466667,0.705882}%
\pgfsetfillcolor{currentfill}%
\pgfsetlinewidth{1.003750pt}%
\definecolor{currentstroke}{rgb}{0.000000,0.000000,0.000000}%
\pgfsetstrokecolor{currentstroke}%
\pgfsetdash{}{0pt}%
\pgfpathmoveto{\pgfqpoint{4.174091in}{0.500000in}}%
\pgfpathlineto{\pgfqpoint{4.343182in}{0.500000in}}%
\pgfpathlineto{\pgfqpoint{4.343182in}{0.500000in}}%
\pgfpathlineto{\pgfqpoint{4.174091in}{0.500000in}}%
\pgfpathclose%
\pgfusepath{stroke,fill}%
\end{pgfscope}%
\begin{pgfscope}%
\pgfpathrectangle{\pgfqpoint{0.750000in}{0.500000in}}{\pgfqpoint{4.650000in}{3.020000in}}%
\pgfusepath{clip}%
\pgfsetbuttcap%
\pgfsetmiterjoin%
\definecolor{currentfill}{rgb}{0.121569,0.466667,0.705882}%
\pgfsetfillcolor{currentfill}%
\pgfsetlinewidth{1.003750pt}%
\definecolor{currentstroke}{rgb}{0.000000,0.000000,0.000000}%
\pgfsetstrokecolor{currentstroke}%
\pgfsetdash{}{0pt}%
\pgfpathmoveto{\pgfqpoint{4.343182in}{0.500000in}}%
\pgfpathlineto{\pgfqpoint{4.512273in}{0.500000in}}%
\pgfpathlineto{\pgfqpoint{4.512273in}{0.500000in}}%
\pgfpathlineto{\pgfqpoint{4.343182in}{0.500000in}}%
\pgfpathclose%
\pgfusepath{stroke,fill}%
\end{pgfscope}%
\begin{pgfscope}%
\pgfpathrectangle{\pgfqpoint{0.750000in}{0.500000in}}{\pgfqpoint{4.650000in}{3.020000in}}%
\pgfusepath{clip}%
\pgfsetbuttcap%
\pgfsetmiterjoin%
\definecolor{currentfill}{rgb}{0.121569,0.466667,0.705882}%
\pgfsetfillcolor{currentfill}%
\pgfsetlinewidth{1.003750pt}%
\definecolor{currentstroke}{rgb}{0.000000,0.000000,0.000000}%
\pgfsetstrokecolor{currentstroke}%
\pgfsetdash{}{0pt}%
\pgfpathmoveto{\pgfqpoint{4.512273in}{0.500000in}}%
\pgfpathlineto{\pgfqpoint{4.681364in}{0.500000in}}%
\pgfpathlineto{\pgfqpoint{4.681364in}{0.542297in}}%
\pgfpathlineto{\pgfqpoint{4.512273in}{0.542297in}}%
\pgfpathclose%
\pgfusepath{stroke,fill}%
\end{pgfscope}%
\begin{pgfscope}%
\pgfpathrectangle{\pgfqpoint{0.750000in}{0.500000in}}{\pgfqpoint{4.650000in}{3.020000in}}%
\pgfusepath{clip}%
\pgfsetbuttcap%
\pgfsetmiterjoin%
\definecolor{currentfill}{rgb}{0.121569,0.466667,0.705882}%
\pgfsetfillcolor{currentfill}%
\pgfsetlinewidth{1.003750pt}%
\definecolor{currentstroke}{rgb}{0.000000,0.000000,0.000000}%
\pgfsetstrokecolor{currentstroke}%
\pgfsetdash{}{0pt}%
\pgfpathmoveto{\pgfqpoint{4.681364in}{0.500000in}}%
\pgfpathlineto{\pgfqpoint{4.850455in}{0.500000in}}%
\pgfpathlineto{\pgfqpoint{4.850455in}{0.500000in}}%
\pgfpathlineto{\pgfqpoint{4.681364in}{0.500000in}}%
\pgfpathclose%
\pgfusepath{stroke,fill}%
\end{pgfscope}%
\begin{pgfscope}%
\pgfpathrectangle{\pgfqpoint{0.750000in}{0.500000in}}{\pgfqpoint{4.650000in}{3.020000in}}%
\pgfusepath{clip}%
\pgfsetbuttcap%
\pgfsetmiterjoin%
\definecolor{currentfill}{rgb}{0.121569,0.466667,0.705882}%
\pgfsetfillcolor{currentfill}%
\pgfsetlinewidth{1.003750pt}%
\definecolor{currentstroke}{rgb}{0.000000,0.000000,0.000000}%
\pgfsetstrokecolor{currentstroke}%
\pgfsetdash{}{0pt}%
\pgfpathmoveto{\pgfqpoint{4.850455in}{0.500000in}}%
\pgfpathlineto{\pgfqpoint{5.019545in}{0.500000in}}%
\pgfpathlineto{\pgfqpoint{5.019545in}{0.500000in}}%
\pgfpathlineto{\pgfqpoint{4.850455in}{0.500000in}}%
\pgfpathclose%
\pgfusepath{stroke,fill}%
\end{pgfscope}%
\begin{pgfscope}%
\pgfpathrectangle{\pgfqpoint{0.750000in}{0.500000in}}{\pgfqpoint{4.650000in}{3.020000in}}%
\pgfusepath{clip}%
\pgfsetbuttcap%
\pgfsetmiterjoin%
\definecolor{currentfill}{rgb}{0.121569,0.466667,0.705882}%
\pgfsetfillcolor{currentfill}%
\pgfsetlinewidth{1.003750pt}%
\definecolor{currentstroke}{rgb}{0.000000,0.000000,0.000000}%
\pgfsetstrokecolor{currentstroke}%
\pgfsetdash{}{0pt}%
\pgfpathmoveto{\pgfqpoint{5.019545in}{0.500000in}}%
\pgfpathlineto{\pgfqpoint{5.188636in}{0.500000in}}%
\pgfpathlineto{\pgfqpoint{5.188636in}{0.584594in}}%
\pgfpathlineto{\pgfqpoint{5.019545in}{0.584594in}}%
\pgfpathclose%
\pgfusepath{stroke,fill}%
\end{pgfscope}%
\begin{pgfscope}%
\pgfsetbuttcap%
\pgfsetroundjoin%
\definecolor{currentfill}{rgb}{0.000000,0.000000,0.000000}%
\pgfsetfillcolor{currentfill}%
\pgfsetlinewidth{0.803000pt}%
\definecolor{currentstroke}{rgb}{0.000000,0.000000,0.000000}%
\pgfsetstrokecolor{currentstroke}%
\pgfsetdash{}{0pt}%
\pgfsys@defobject{currentmarker}{\pgfqpoint{0.000000in}{-0.048611in}}{\pgfqpoint{0.000000in}{0.000000in}}{%
\pgfpathmoveto{\pgfqpoint{0.000000in}{0.000000in}}%
\pgfpathlineto{\pgfqpoint{0.000000in}{-0.048611in}}%
\pgfusepath{stroke,fill}%
}%
\begin{pgfscope}%
\pgfsys@transformshift{1.130455in}{0.500000in}%
\pgfsys@useobject{currentmarker}{}%
\end{pgfscope}%
\end{pgfscope}%
\begin{pgfscope}%
\definecolor{textcolor}{rgb}{0.000000,0.000000,0.000000}%
\pgfsetstrokecolor{textcolor}%
\pgfsetfillcolor{textcolor}%
\pgftext[x=1.130455in,y=0.402778in,,top]{\color{textcolor}\sffamily\fontsize{10.000000}{12.000000}\selectfont \(\displaystyle {-600}\)}%
\end{pgfscope}%
\begin{pgfscope}%
\pgfsetbuttcap%
\pgfsetroundjoin%
\definecolor{currentfill}{rgb}{0.000000,0.000000,0.000000}%
\pgfsetfillcolor{currentfill}%
\pgfsetlinewidth{0.803000pt}%
\definecolor{currentstroke}{rgb}{0.000000,0.000000,0.000000}%
\pgfsetstrokecolor{currentstroke}%
\pgfsetdash{}{0pt}%
\pgfsys@defobject{currentmarker}{\pgfqpoint{0.000000in}{-0.048611in}}{\pgfqpoint{0.000000in}{0.000000in}}{%
\pgfpathmoveto{\pgfqpoint{0.000000in}{0.000000in}}%
\pgfpathlineto{\pgfqpoint{0.000000in}{-0.048611in}}%
\pgfusepath{stroke,fill}%
}%
\begin{pgfscope}%
\pgfsys@transformshift{1.806818in}{0.500000in}%
\pgfsys@useobject{currentmarker}{}%
\end{pgfscope}%
\end{pgfscope}%
\begin{pgfscope}%
\definecolor{textcolor}{rgb}{0.000000,0.000000,0.000000}%
\pgfsetstrokecolor{textcolor}%
\pgfsetfillcolor{textcolor}%
\pgftext[x=1.806818in,y=0.402778in,,top]{\color{textcolor}\sffamily\fontsize{10.000000}{12.000000}\selectfont \(\displaystyle {-400}\)}%
\end{pgfscope}%
\begin{pgfscope}%
\pgfsetbuttcap%
\pgfsetroundjoin%
\definecolor{currentfill}{rgb}{0.000000,0.000000,0.000000}%
\pgfsetfillcolor{currentfill}%
\pgfsetlinewidth{0.803000pt}%
\definecolor{currentstroke}{rgb}{0.000000,0.000000,0.000000}%
\pgfsetstrokecolor{currentstroke}%
\pgfsetdash{}{0pt}%
\pgfsys@defobject{currentmarker}{\pgfqpoint{0.000000in}{-0.048611in}}{\pgfqpoint{0.000000in}{0.000000in}}{%
\pgfpathmoveto{\pgfqpoint{0.000000in}{0.000000in}}%
\pgfpathlineto{\pgfqpoint{0.000000in}{-0.048611in}}%
\pgfusepath{stroke,fill}%
}%
\begin{pgfscope}%
\pgfsys@transformshift{2.483182in}{0.500000in}%
\pgfsys@useobject{currentmarker}{}%
\end{pgfscope}%
\end{pgfscope}%
\begin{pgfscope}%
\definecolor{textcolor}{rgb}{0.000000,0.000000,0.000000}%
\pgfsetstrokecolor{textcolor}%
\pgfsetfillcolor{textcolor}%
\pgftext[x=2.483182in,y=0.402778in,,top]{\color{textcolor}\sffamily\fontsize{10.000000}{12.000000}\selectfont \(\displaystyle {-200}\)}%
\end{pgfscope}%
\begin{pgfscope}%
\pgfsetbuttcap%
\pgfsetroundjoin%
\definecolor{currentfill}{rgb}{0.000000,0.000000,0.000000}%
\pgfsetfillcolor{currentfill}%
\pgfsetlinewidth{0.803000pt}%
\definecolor{currentstroke}{rgb}{0.000000,0.000000,0.000000}%
\pgfsetstrokecolor{currentstroke}%
\pgfsetdash{}{0pt}%
\pgfsys@defobject{currentmarker}{\pgfqpoint{0.000000in}{-0.048611in}}{\pgfqpoint{0.000000in}{0.000000in}}{%
\pgfpathmoveto{\pgfqpoint{0.000000in}{0.000000in}}%
\pgfpathlineto{\pgfqpoint{0.000000in}{-0.048611in}}%
\pgfusepath{stroke,fill}%
}%
\begin{pgfscope}%
\pgfsys@transformshift{3.159545in}{0.500000in}%
\pgfsys@useobject{currentmarker}{}%
\end{pgfscope}%
\end{pgfscope}%
\begin{pgfscope}%
\definecolor{textcolor}{rgb}{0.000000,0.000000,0.000000}%
\pgfsetstrokecolor{textcolor}%
\pgfsetfillcolor{textcolor}%
\pgftext[x=3.159545in,y=0.402778in,,top]{\color{textcolor}\sffamily\fontsize{10.000000}{12.000000}\selectfont \(\displaystyle {0}\)}%
\end{pgfscope}%
\begin{pgfscope}%
\pgfsetbuttcap%
\pgfsetroundjoin%
\definecolor{currentfill}{rgb}{0.000000,0.000000,0.000000}%
\pgfsetfillcolor{currentfill}%
\pgfsetlinewidth{0.803000pt}%
\definecolor{currentstroke}{rgb}{0.000000,0.000000,0.000000}%
\pgfsetstrokecolor{currentstroke}%
\pgfsetdash{}{0pt}%
\pgfsys@defobject{currentmarker}{\pgfqpoint{0.000000in}{-0.048611in}}{\pgfqpoint{0.000000in}{0.000000in}}{%
\pgfpathmoveto{\pgfqpoint{0.000000in}{0.000000in}}%
\pgfpathlineto{\pgfqpoint{0.000000in}{-0.048611in}}%
\pgfusepath{stroke,fill}%
}%
\begin{pgfscope}%
\pgfsys@transformshift{3.835909in}{0.500000in}%
\pgfsys@useobject{currentmarker}{}%
\end{pgfscope}%
\end{pgfscope}%
\begin{pgfscope}%
\definecolor{textcolor}{rgb}{0.000000,0.000000,0.000000}%
\pgfsetstrokecolor{textcolor}%
\pgfsetfillcolor{textcolor}%
\pgftext[x=3.835909in,y=0.402778in,,top]{\color{textcolor}\sffamily\fontsize{10.000000}{12.000000}\selectfont \(\displaystyle {200}\)}%
\end{pgfscope}%
\begin{pgfscope}%
\pgfsetbuttcap%
\pgfsetroundjoin%
\definecolor{currentfill}{rgb}{0.000000,0.000000,0.000000}%
\pgfsetfillcolor{currentfill}%
\pgfsetlinewidth{0.803000pt}%
\definecolor{currentstroke}{rgb}{0.000000,0.000000,0.000000}%
\pgfsetstrokecolor{currentstroke}%
\pgfsetdash{}{0pt}%
\pgfsys@defobject{currentmarker}{\pgfqpoint{0.000000in}{-0.048611in}}{\pgfqpoint{0.000000in}{0.000000in}}{%
\pgfpathmoveto{\pgfqpoint{0.000000in}{0.000000in}}%
\pgfpathlineto{\pgfqpoint{0.000000in}{-0.048611in}}%
\pgfusepath{stroke,fill}%
}%
\begin{pgfscope}%
\pgfsys@transformshift{4.512273in}{0.500000in}%
\pgfsys@useobject{currentmarker}{}%
\end{pgfscope}%
\end{pgfscope}%
\begin{pgfscope}%
\definecolor{textcolor}{rgb}{0.000000,0.000000,0.000000}%
\pgfsetstrokecolor{textcolor}%
\pgfsetfillcolor{textcolor}%
\pgftext[x=4.512273in,y=0.402778in,,top]{\color{textcolor}\sffamily\fontsize{10.000000}{12.000000}\selectfont \(\displaystyle {400}\)}%
\end{pgfscope}%
\begin{pgfscope}%
\pgfsetbuttcap%
\pgfsetroundjoin%
\definecolor{currentfill}{rgb}{0.000000,0.000000,0.000000}%
\pgfsetfillcolor{currentfill}%
\pgfsetlinewidth{0.803000pt}%
\definecolor{currentstroke}{rgb}{0.000000,0.000000,0.000000}%
\pgfsetstrokecolor{currentstroke}%
\pgfsetdash{}{0pt}%
\pgfsys@defobject{currentmarker}{\pgfqpoint{0.000000in}{-0.048611in}}{\pgfqpoint{0.000000in}{0.000000in}}{%
\pgfpathmoveto{\pgfqpoint{0.000000in}{0.000000in}}%
\pgfpathlineto{\pgfqpoint{0.000000in}{-0.048611in}}%
\pgfusepath{stroke,fill}%
}%
\begin{pgfscope}%
\pgfsys@transformshift{5.188636in}{0.500000in}%
\pgfsys@useobject{currentmarker}{}%
\end{pgfscope}%
\end{pgfscope}%
\begin{pgfscope}%
\definecolor{textcolor}{rgb}{0.000000,0.000000,0.000000}%
\pgfsetstrokecolor{textcolor}%
\pgfsetfillcolor{textcolor}%
\pgftext[x=5.188636in,y=0.402778in,,top]{\color{textcolor}\sffamily\fontsize{10.000000}{12.000000}\selectfont \(\displaystyle {600}\)}%
\end{pgfscope}%
\begin{pgfscope}%
\definecolor{textcolor}{rgb}{0.000000,0.000000,0.000000}%
\pgfsetstrokecolor{textcolor}%
\pgfsetfillcolor{textcolor}%
\pgftext[x=3.075000in,y=0.223889in,,top]{\color{textcolor}\sffamily\fontsize{10.000000}{12.000000}\selectfont \(\displaystyle \Delta\) Data Flow Time}%
\end{pgfscope}%
\begin{pgfscope}%
\pgfsetbuttcap%
\pgfsetroundjoin%
\definecolor{currentfill}{rgb}{0.000000,0.000000,0.000000}%
\pgfsetfillcolor{currentfill}%
\pgfsetlinewidth{0.803000pt}%
\definecolor{currentstroke}{rgb}{0.000000,0.000000,0.000000}%
\pgfsetstrokecolor{currentstroke}%
\pgfsetdash{}{0pt}%
\pgfsys@defobject{currentmarker}{\pgfqpoint{-0.048611in}{0.000000in}}{\pgfqpoint{0.000000in}{0.000000in}}{%
\pgfpathmoveto{\pgfqpoint{0.000000in}{0.000000in}}%
\pgfpathlineto{\pgfqpoint{-0.048611in}{0.000000in}}%
\pgfusepath{stroke,fill}%
}%
\begin{pgfscope}%
\pgfsys@transformshift{0.750000in}{0.500000in}%
\pgfsys@useobject{currentmarker}{}%
\end{pgfscope}%
\end{pgfscope}%
\begin{pgfscope}%
\definecolor{textcolor}{rgb}{0.000000,0.000000,0.000000}%
\pgfsetstrokecolor{textcolor}%
\pgfsetfillcolor{textcolor}%
\pgftext[x=0.583333in, y=0.451806in, left, base]{\color{textcolor}\sffamily\fontsize{10.000000}{12.000000}\selectfont \(\displaystyle {0}\)}%
\end{pgfscope}%
\begin{pgfscope}%
\pgfsetbuttcap%
\pgfsetroundjoin%
\definecolor{currentfill}{rgb}{0.000000,0.000000,0.000000}%
\pgfsetfillcolor{currentfill}%
\pgfsetlinewidth{0.803000pt}%
\definecolor{currentstroke}{rgb}{0.000000,0.000000,0.000000}%
\pgfsetstrokecolor{currentstroke}%
\pgfsetdash{}{0pt}%
\pgfsys@defobject{currentmarker}{\pgfqpoint{-0.048611in}{0.000000in}}{\pgfqpoint{0.000000in}{0.000000in}}{%
\pgfpathmoveto{\pgfqpoint{0.000000in}{0.000000in}}%
\pgfpathlineto{\pgfqpoint{-0.048611in}{0.000000in}}%
\pgfusepath{stroke,fill}%
}%
\begin{pgfscope}%
\pgfsys@transformshift{0.750000in}{0.922969in}%
\pgfsys@useobject{currentmarker}{}%
\end{pgfscope}%
\end{pgfscope}%
\begin{pgfscope}%
\definecolor{textcolor}{rgb}{0.000000,0.000000,0.000000}%
\pgfsetstrokecolor{textcolor}%
\pgfsetfillcolor{textcolor}%
\pgftext[x=0.513888in, y=0.874775in, left, base]{\color{textcolor}\sffamily\fontsize{10.000000}{12.000000}\selectfont \(\displaystyle {10}\)}%
\end{pgfscope}%
\begin{pgfscope}%
\pgfsetbuttcap%
\pgfsetroundjoin%
\definecolor{currentfill}{rgb}{0.000000,0.000000,0.000000}%
\pgfsetfillcolor{currentfill}%
\pgfsetlinewidth{0.803000pt}%
\definecolor{currentstroke}{rgb}{0.000000,0.000000,0.000000}%
\pgfsetstrokecolor{currentstroke}%
\pgfsetdash{}{0pt}%
\pgfsys@defobject{currentmarker}{\pgfqpoint{-0.048611in}{0.000000in}}{\pgfqpoint{0.000000in}{0.000000in}}{%
\pgfpathmoveto{\pgfqpoint{0.000000in}{0.000000in}}%
\pgfpathlineto{\pgfqpoint{-0.048611in}{0.000000in}}%
\pgfusepath{stroke,fill}%
}%
\begin{pgfscope}%
\pgfsys@transformshift{0.750000in}{1.345938in}%
\pgfsys@useobject{currentmarker}{}%
\end{pgfscope}%
\end{pgfscope}%
\begin{pgfscope}%
\definecolor{textcolor}{rgb}{0.000000,0.000000,0.000000}%
\pgfsetstrokecolor{textcolor}%
\pgfsetfillcolor{textcolor}%
\pgftext[x=0.513888in, y=1.297744in, left, base]{\color{textcolor}\sffamily\fontsize{10.000000}{12.000000}\selectfont \(\displaystyle {20}\)}%
\end{pgfscope}%
\begin{pgfscope}%
\pgfsetbuttcap%
\pgfsetroundjoin%
\definecolor{currentfill}{rgb}{0.000000,0.000000,0.000000}%
\pgfsetfillcolor{currentfill}%
\pgfsetlinewidth{0.803000pt}%
\definecolor{currentstroke}{rgb}{0.000000,0.000000,0.000000}%
\pgfsetstrokecolor{currentstroke}%
\pgfsetdash{}{0pt}%
\pgfsys@defobject{currentmarker}{\pgfqpoint{-0.048611in}{0.000000in}}{\pgfqpoint{0.000000in}{0.000000in}}{%
\pgfpathmoveto{\pgfqpoint{0.000000in}{0.000000in}}%
\pgfpathlineto{\pgfqpoint{-0.048611in}{0.000000in}}%
\pgfusepath{stroke,fill}%
}%
\begin{pgfscope}%
\pgfsys@transformshift{0.750000in}{1.768908in}%
\pgfsys@useobject{currentmarker}{}%
\end{pgfscope}%
\end{pgfscope}%
\begin{pgfscope}%
\definecolor{textcolor}{rgb}{0.000000,0.000000,0.000000}%
\pgfsetstrokecolor{textcolor}%
\pgfsetfillcolor{textcolor}%
\pgftext[x=0.513888in, y=1.720713in, left, base]{\color{textcolor}\sffamily\fontsize{10.000000}{12.000000}\selectfont \(\displaystyle {30}\)}%
\end{pgfscope}%
\begin{pgfscope}%
\pgfsetbuttcap%
\pgfsetroundjoin%
\definecolor{currentfill}{rgb}{0.000000,0.000000,0.000000}%
\pgfsetfillcolor{currentfill}%
\pgfsetlinewidth{0.803000pt}%
\definecolor{currentstroke}{rgb}{0.000000,0.000000,0.000000}%
\pgfsetstrokecolor{currentstroke}%
\pgfsetdash{}{0pt}%
\pgfsys@defobject{currentmarker}{\pgfqpoint{-0.048611in}{0.000000in}}{\pgfqpoint{0.000000in}{0.000000in}}{%
\pgfpathmoveto{\pgfqpoint{0.000000in}{0.000000in}}%
\pgfpathlineto{\pgfqpoint{-0.048611in}{0.000000in}}%
\pgfusepath{stroke,fill}%
}%
\begin{pgfscope}%
\pgfsys@transformshift{0.750000in}{2.191877in}%
\pgfsys@useobject{currentmarker}{}%
\end{pgfscope}%
\end{pgfscope}%
\begin{pgfscope}%
\definecolor{textcolor}{rgb}{0.000000,0.000000,0.000000}%
\pgfsetstrokecolor{textcolor}%
\pgfsetfillcolor{textcolor}%
\pgftext[x=0.513888in, y=2.143682in, left, base]{\color{textcolor}\sffamily\fontsize{10.000000}{12.000000}\selectfont \(\displaystyle {40}\)}%
\end{pgfscope}%
\begin{pgfscope}%
\pgfsetbuttcap%
\pgfsetroundjoin%
\definecolor{currentfill}{rgb}{0.000000,0.000000,0.000000}%
\pgfsetfillcolor{currentfill}%
\pgfsetlinewidth{0.803000pt}%
\definecolor{currentstroke}{rgb}{0.000000,0.000000,0.000000}%
\pgfsetstrokecolor{currentstroke}%
\pgfsetdash{}{0pt}%
\pgfsys@defobject{currentmarker}{\pgfqpoint{-0.048611in}{0.000000in}}{\pgfqpoint{0.000000in}{0.000000in}}{%
\pgfpathmoveto{\pgfqpoint{0.000000in}{0.000000in}}%
\pgfpathlineto{\pgfqpoint{-0.048611in}{0.000000in}}%
\pgfusepath{stroke,fill}%
}%
\begin{pgfscope}%
\pgfsys@transformshift{0.750000in}{2.614846in}%
\pgfsys@useobject{currentmarker}{}%
\end{pgfscope}%
\end{pgfscope}%
\begin{pgfscope}%
\definecolor{textcolor}{rgb}{0.000000,0.000000,0.000000}%
\pgfsetstrokecolor{textcolor}%
\pgfsetfillcolor{textcolor}%
\pgftext[x=0.513888in, y=2.566651in, left, base]{\color{textcolor}\sffamily\fontsize{10.000000}{12.000000}\selectfont \(\displaystyle {50}\)}%
\end{pgfscope}%
\begin{pgfscope}%
\pgfsetbuttcap%
\pgfsetroundjoin%
\definecolor{currentfill}{rgb}{0.000000,0.000000,0.000000}%
\pgfsetfillcolor{currentfill}%
\pgfsetlinewidth{0.803000pt}%
\definecolor{currentstroke}{rgb}{0.000000,0.000000,0.000000}%
\pgfsetstrokecolor{currentstroke}%
\pgfsetdash{}{0pt}%
\pgfsys@defobject{currentmarker}{\pgfqpoint{-0.048611in}{0.000000in}}{\pgfqpoint{0.000000in}{0.000000in}}{%
\pgfpathmoveto{\pgfqpoint{0.000000in}{0.000000in}}%
\pgfpathlineto{\pgfqpoint{-0.048611in}{0.000000in}}%
\pgfusepath{stroke,fill}%
}%
\begin{pgfscope}%
\pgfsys@transformshift{0.750000in}{3.037815in}%
\pgfsys@useobject{currentmarker}{}%
\end{pgfscope}%
\end{pgfscope}%
\begin{pgfscope}%
\definecolor{textcolor}{rgb}{0.000000,0.000000,0.000000}%
\pgfsetstrokecolor{textcolor}%
\pgfsetfillcolor{textcolor}%
\pgftext[x=0.513888in, y=2.989621in, left, base]{\color{textcolor}\sffamily\fontsize{10.000000}{12.000000}\selectfont \(\displaystyle {60}\)}%
\end{pgfscope}%
\begin{pgfscope}%
\pgfsetbuttcap%
\pgfsetroundjoin%
\definecolor{currentfill}{rgb}{0.000000,0.000000,0.000000}%
\pgfsetfillcolor{currentfill}%
\pgfsetlinewidth{0.803000pt}%
\definecolor{currentstroke}{rgb}{0.000000,0.000000,0.000000}%
\pgfsetstrokecolor{currentstroke}%
\pgfsetdash{}{0pt}%
\pgfsys@defobject{currentmarker}{\pgfqpoint{-0.048611in}{0.000000in}}{\pgfqpoint{0.000000in}{0.000000in}}{%
\pgfpathmoveto{\pgfqpoint{0.000000in}{0.000000in}}%
\pgfpathlineto{\pgfqpoint{-0.048611in}{0.000000in}}%
\pgfusepath{stroke,fill}%
}%
\begin{pgfscope}%
\pgfsys@transformshift{0.750000in}{3.460784in}%
\pgfsys@useobject{currentmarker}{}%
\end{pgfscope}%
\end{pgfscope}%
\begin{pgfscope}%
\definecolor{textcolor}{rgb}{0.000000,0.000000,0.000000}%
\pgfsetstrokecolor{textcolor}%
\pgfsetfillcolor{textcolor}%
\pgftext[x=0.513888in, y=3.412590in, left, base]{\color{textcolor}\sffamily\fontsize{10.000000}{12.000000}\selectfont \(\displaystyle {70}\)}%
\end{pgfscope}%
\begin{pgfscope}%
\definecolor{textcolor}{rgb}{0.000000,0.000000,0.000000}%
\pgfsetstrokecolor{textcolor}%
\pgfsetfillcolor{textcolor}%
\pgftext[x=0.458333in,y=2.010000in,,bottom,rotate=90.000000]{\color{textcolor}\sffamily\fontsize{10.000000}{12.000000}\selectfont Frequency}%
\end{pgfscope}%
\begin{pgfscope}%
\pgfsetrectcap%
\pgfsetmiterjoin%
\pgfsetlinewidth{0.803000pt}%
\definecolor{currentstroke}{rgb}{0.000000,0.000000,0.000000}%
\pgfsetstrokecolor{currentstroke}%
\pgfsetdash{}{0pt}%
\pgfpathmoveto{\pgfqpoint{0.750000in}{0.500000in}}%
\pgfpathlineto{\pgfqpoint{0.750000in}{3.520000in}}%
\pgfusepath{stroke}%
\end{pgfscope}%
\begin{pgfscope}%
\pgfsetrectcap%
\pgfsetmiterjoin%
\pgfsetlinewidth{0.803000pt}%
\definecolor{currentstroke}{rgb}{0.000000,0.000000,0.000000}%
\pgfsetstrokecolor{currentstroke}%
\pgfsetdash{}{0pt}%
\pgfpathmoveto{\pgfqpoint{5.400000in}{0.500000in}}%
\pgfpathlineto{\pgfqpoint{5.400000in}{3.520000in}}%
\pgfusepath{stroke}%
\end{pgfscope}%
\begin{pgfscope}%
\pgfsetrectcap%
\pgfsetmiterjoin%
\pgfsetlinewidth{0.803000pt}%
\definecolor{currentstroke}{rgb}{0.000000,0.000000,0.000000}%
\pgfsetstrokecolor{currentstroke}%
\pgfsetdash{}{0pt}%
\pgfpathmoveto{\pgfqpoint{0.750000in}{0.500000in}}%
\pgfpathlineto{\pgfqpoint{5.400000in}{0.500000in}}%
\pgfusepath{stroke}%
\end{pgfscope}%
\begin{pgfscope}%
\pgfsetrectcap%
\pgfsetmiterjoin%
\pgfsetlinewidth{0.803000pt}%
\definecolor{currentstroke}{rgb}{0.000000,0.000000,0.000000}%
\pgfsetstrokecolor{currentstroke}%
\pgfsetdash{}{0pt}%
\pgfpathmoveto{\pgfqpoint{0.750000in}{3.520000in}}%
\pgfpathlineto{\pgfqpoint{5.400000in}{3.520000in}}%
\pgfusepath{stroke}%
\end{pgfscope}%
\end{pgfpicture}%
\makeatother%
\endgroup%

        }
        \caption{Histogram of the Delta Data Flow Time}
        \label{f:deltaHist}
    \end{figure}

    We confirmed that the right direction choice can speed up the analysis by a magnificent amount. To take advantage of the favorable direction, we now investigate the correlating conditions for the advantageous direction.
    Most straightforward would be a correlation between the difference of source and sink count and the data flow time.
    In \autoref{f:dfratio} are two graphs with the ratio of sources and sinks ($\frac{\mathit{Sinks} - \mathit{Sources}}{\mathit{Sources}}$) on the x-axis and the data flow time in seconds on the y-axis.
    The left graph is always the forward implementation and the right graph is our implementation.
    Blue dots represent apps without a timeout, orange a time timeout and red a memory timeout.
    Intuitively, a negative ratio should put our implementation at an advantage. The graphs show no correlation between the ratio and the runtime, neither forward nor backward.
    We also included the forward data flow time by sources and the backward data flow time by sinks in \autoref{f:dfsources} and \autoref{f:dfsinks}.
    The number of sinks backward and the number of sources forward do not influence the runtime.
    So we can confirm Arzt's evaluation\cite{Arzt2017PhD} as there is no correlation between sources and the forward runtime in our app set.
    Parallel to this observation, the sink count does not influence the backward runtime.
    The sink count for forward and the source count for backward can not influence the runtime they have no influence on the edge propagations.


    \begin{figure}[tbp]
        \centering
        \begin{subfigure}[b]{0.45\textwidth}
            \centering
            \begin{subfigure}[]{\textwidth}
                \centering
                \resizebox{\columnwidth}{!}{
                    %% Creator: Matplotlib, PGF backend
%%
%% To include the figure in your LaTeX document, write
%%   \input{<filename>.pgf}
%%
%% Make sure the required packages are loaded in your preamble
%%   \usepackage{pgf}
%%
%% and, on pdftex
%%   \usepackage[utf8]{inputenc}\DeclareUnicodeCharacter{2212}{-}
%%
%% or, on luatex and xetex
%%   \usepackage{unicode-math}
%%
%% Figures using additional raster images can only be included by \input if
%% they are in the same directory as the main LaTeX file. For loading figures
%% from other directories you can use the `import` package
%%   \usepackage{import}
%%
%% and then include the figures with
%%   \import{<path to file>}{<filename>.pgf}
%%
%% Matplotlib used the following preamble
%%   \usepackage{fontspec}
%%
\begingroup%
\makeatletter%
\begin{pgfpicture}%
\pgfpathrectangle{\pgfpointorigin}{\pgfqpoint{6.000000in}{4.000000in}}%
\pgfusepath{use as bounding box, clip}%
\begin{pgfscope}%
\pgfsetbuttcap%
\pgfsetmiterjoin%
\definecolor{currentfill}{rgb}{1.000000,1.000000,1.000000}%
\pgfsetfillcolor{currentfill}%
\pgfsetlinewidth{0.000000pt}%
\definecolor{currentstroke}{rgb}{1.000000,1.000000,1.000000}%
\pgfsetstrokecolor{currentstroke}%
\pgfsetdash{}{0pt}%
\pgfpathmoveto{\pgfqpoint{0.000000in}{0.000000in}}%
\pgfpathlineto{\pgfqpoint{6.000000in}{0.000000in}}%
\pgfpathlineto{\pgfqpoint{6.000000in}{4.000000in}}%
\pgfpathlineto{\pgfqpoint{0.000000in}{4.000000in}}%
\pgfpathclose%
\pgfusepath{fill}%
\end{pgfscope}%
\begin{pgfscope}%
\pgfsetbuttcap%
\pgfsetmiterjoin%
\definecolor{currentfill}{rgb}{1.000000,1.000000,1.000000}%
\pgfsetfillcolor{currentfill}%
\pgfsetlinewidth{0.000000pt}%
\definecolor{currentstroke}{rgb}{0.000000,0.000000,0.000000}%
\pgfsetstrokecolor{currentstroke}%
\pgfsetstrokeopacity{0.000000}%
\pgfsetdash{}{0pt}%
\pgfpathmoveto{\pgfqpoint{0.750000in}{0.500000in}}%
\pgfpathlineto{\pgfqpoint{5.400000in}{0.500000in}}%
\pgfpathlineto{\pgfqpoint{5.400000in}{3.520000in}}%
\pgfpathlineto{\pgfqpoint{0.750000in}{3.520000in}}%
\pgfpathclose%
\pgfusepath{fill}%
\end{pgfscope}%
\begin{pgfscope}%
\pgfpathrectangle{\pgfqpoint{0.750000in}{0.500000in}}{\pgfqpoint{4.650000in}{3.020000in}}%
\pgfusepath{clip}%
\pgfsetbuttcap%
\pgfsetroundjoin%
\definecolor{currentfill}{rgb}{1.000000,0.498039,0.054902}%
\pgfsetfillcolor{currentfill}%
\pgfsetlinewidth{1.003750pt}%
\definecolor{currentstroke}{rgb}{1.000000,0.498039,0.054902}%
\pgfsetstrokecolor{currentstroke}%
\pgfsetdash{}{0pt}%
\pgfpathmoveto{\pgfqpoint{1.625649in}{2.939952in}}%
\pgfpathcurveto{\pgfqpoint{1.636699in}{2.939952in}}{\pgfqpoint{1.647299in}{2.944342in}}{\pgfqpoint{1.655112in}{2.952156in}}%
\pgfpathcurveto{\pgfqpoint{1.662926in}{2.959969in}}{\pgfqpoint{1.667316in}{2.970568in}}{\pgfqpoint{1.667316in}{2.981618in}}%
\pgfpathcurveto{\pgfqpoint{1.667316in}{2.992668in}}{\pgfqpoint{1.662926in}{3.003267in}}{\pgfqpoint{1.655112in}{3.011081in}}%
\pgfpathcurveto{\pgfqpoint{1.647299in}{3.018895in}}{\pgfqpoint{1.636699in}{3.023285in}}{\pgfqpoint{1.625649in}{3.023285in}}%
\pgfpathcurveto{\pgfqpoint{1.614599in}{3.023285in}}{\pgfqpoint{1.604000in}{3.018895in}}{\pgfqpoint{1.596187in}{3.011081in}}%
\pgfpathcurveto{\pgfqpoint{1.588373in}{3.003267in}}{\pgfqpoint{1.583983in}{2.992668in}}{\pgfqpoint{1.583983in}{2.981618in}}%
\pgfpathcurveto{\pgfqpoint{1.583983in}{2.970568in}}{\pgfqpoint{1.588373in}{2.959969in}}{\pgfqpoint{1.596187in}{2.952156in}}%
\pgfpathcurveto{\pgfqpoint{1.604000in}{2.944342in}}{\pgfqpoint{1.614599in}{2.939952in}}{\pgfqpoint{1.625649in}{2.939952in}}%
\pgfpathclose%
\pgfusepath{stroke,fill}%
\end{pgfscope}%
\begin{pgfscope}%
\pgfpathrectangle{\pgfqpoint{0.750000in}{0.500000in}}{\pgfqpoint{4.650000in}{3.020000in}}%
\pgfusepath{clip}%
\pgfsetbuttcap%
\pgfsetroundjoin%
\definecolor{currentfill}{rgb}{0.121569,0.466667,0.705882}%
\pgfsetfillcolor{currentfill}%
\pgfsetlinewidth{1.003750pt}%
\definecolor{currentstroke}{rgb}{0.121569,0.466667,0.705882}%
\pgfsetstrokecolor{currentstroke}%
\pgfsetdash{}{0pt}%
\pgfpathmoveto{\pgfqpoint{1.323701in}{0.595606in}}%
\pgfpathcurveto{\pgfqpoint{1.334751in}{0.595606in}}{\pgfqpoint{1.345350in}{0.599996in}}{\pgfqpoint{1.353164in}{0.607810in}}%
\pgfpathcurveto{\pgfqpoint{1.360978in}{0.615624in}}{\pgfqpoint{1.365368in}{0.626223in}}{\pgfqpoint{1.365368in}{0.637273in}}%
\pgfpathcurveto{\pgfqpoint{1.365368in}{0.648323in}}{\pgfqpoint{1.360978in}{0.658922in}}{\pgfqpoint{1.353164in}{0.666736in}}%
\pgfpathcurveto{\pgfqpoint{1.345350in}{0.674549in}}{\pgfqpoint{1.334751in}{0.678939in}}{\pgfqpoint{1.323701in}{0.678939in}}%
\pgfpathcurveto{\pgfqpoint{1.312651in}{0.678939in}}{\pgfqpoint{1.302052in}{0.674549in}}{\pgfqpoint{1.294239in}{0.666736in}}%
\pgfpathcurveto{\pgfqpoint{1.286425in}{0.658922in}}{\pgfqpoint{1.282035in}{0.648323in}}{\pgfqpoint{1.282035in}{0.637273in}}%
\pgfpathcurveto{\pgfqpoint{1.282035in}{0.626223in}}{\pgfqpoint{1.286425in}{0.615624in}}{\pgfqpoint{1.294239in}{0.607810in}}%
\pgfpathcurveto{\pgfqpoint{1.302052in}{0.599996in}}{\pgfqpoint{1.312651in}{0.595606in}}{\pgfqpoint{1.323701in}{0.595606in}}%
\pgfpathclose%
\pgfusepath{stroke,fill}%
\end{pgfscope}%
\begin{pgfscope}%
\pgfpathrectangle{\pgfqpoint{0.750000in}{0.500000in}}{\pgfqpoint{4.650000in}{3.020000in}}%
\pgfusepath{clip}%
\pgfsetbuttcap%
\pgfsetroundjoin%
\definecolor{currentfill}{rgb}{1.000000,0.498039,0.054902}%
\pgfsetfillcolor{currentfill}%
\pgfsetlinewidth{1.003750pt}%
\definecolor{currentstroke}{rgb}{1.000000,0.498039,0.054902}%
\pgfsetstrokecolor{currentstroke}%
\pgfsetdash{}{0pt}%
\pgfpathmoveto{\pgfqpoint{2.652273in}{3.099616in}}%
\pgfpathcurveto{\pgfqpoint{2.663323in}{3.099616in}}{\pgfqpoint{2.673922in}{3.104007in}}{\pgfqpoint{2.681736in}{3.111820in}}%
\pgfpathcurveto{\pgfqpoint{2.689549in}{3.119634in}}{\pgfqpoint{2.693939in}{3.130233in}}{\pgfqpoint{2.693939in}{3.141283in}}%
\pgfpathcurveto{\pgfqpoint{2.693939in}{3.152333in}}{\pgfqpoint{2.689549in}{3.162932in}}{\pgfqpoint{2.681736in}{3.170746in}}%
\pgfpathcurveto{\pgfqpoint{2.673922in}{3.178559in}}{\pgfqpoint{2.663323in}{3.182950in}}{\pgfqpoint{2.652273in}{3.182950in}}%
\pgfpathcurveto{\pgfqpoint{2.641223in}{3.182950in}}{\pgfqpoint{2.630624in}{3.178559in}}{\pgfqpoint{2.622810in}{3.170746in}}%
\pgfpathcurveto{\pgfqpoint{2.614996in}{3.162932in}}{\pgfqpoint{2.610606in}{3.152333in}}{\pgfqpoint{2.610606in}{3.141283in}}%
\pgfpathcurveto{\pgfqpoint{2.610606in}{3.130233in}}{\pgfqpoint{2.614996in}{3.119634in}}{\pgfqpoint{2.622810in}{3.111820in}}%
\pgfpathcurveto{\pgfqpoint{2.630624in}{3.104007in}}{\pgfqpoint{2.641223in}{3.099616in}}{\pgfqpoint{2.652273in}{3.099616in}}%
\pgfpathclose%
\pgfusepath{stroke,fill}%
\end{pgfscope}%
\begin{pgfscope}%
\pgfpathrectangle{\pgfqpoint{0.750000in}{0.500000in}}{\pgfqpoint{4.650000in}{3.020000in}}%
\pgfusepath{clip}%
\pgfsetbuttcap%
\pgfsetroundjoin%
\definecolor{currentfill}{rgb}{1.000000,0.498039,0.054902}%
\pgfsetfillcolor{currentfill}%
\pgfsetlinewidth{1.003750pt}%
\definecolor{currentstroke}{rgb}{1.000000,0.498039,0.054902}%
\pgfsetstrokecolor{currentstroke}%
\pgfsetdash{}{0pt}%
\pgfpathmoveto{\pgfqpoint{2.531494in}{2.932163in}}%
\pgfpathcurveto{\pgfqpoint{2.542544in}{2.932163in}}{\pgfqpoint{2.553143in}{2.936553in}}{\pgfqpoint{2.560956in}{2.944367in}}%
\pgfpathcurveto{\pgfqpoint{2.568770in}{2.952181in}}{\pgfqpoint{2.573160in}{2.962780in}}{\pgfqpoint{2.573160in}{2.973830in}}%
\pgfpathcurveto{\pgfqpoint{2.573160in}{2.984880in}}{\pgfqpoint{2.568770in}{2.995479in}}{\pgfqpoint{2.560956in}{3.003293in}}%
\pgfpathcurveto{\pgfqpoint{2.553143in}{3.011106in}}{\pgfqpoint{2.542544in}{3.015496in}}{\pgfqpoint{2.531494in}{3.015496in}}%
\pgfpathcurveto{\pgfqpoint{2.520443in}{3.015496in}}{\pgfqpoint{2.509844in}{3.011106in}}{\pgfqpoint{2.502031in}{3.003293in}}%
\pgfpathcurveto{\pgfqpoint{2.494217in}{2.995479in}}{\pgfqpoint{2.489827in}{2.984880in}}{\pgfqpoint{2.489827in}{2.973830in}}%
\pgfpathcurveto{\pgfqpoint{2.489827in}{2.962780in}}{\pgfqpoint{2.494217in}{2.952181in}}{\pgfqpoint{2.502031in}{2.944367in}}%
\pgfpathcurveto{\pgfqpoint{2.509844in}{2.936553in}}{\pgfqpoint{2.520443in}{2.932163in}}{\pgfqpoint{2.531494in}{2.932163in}}%
\pgfpathclose%
\pgfusepath{stroke,fill}%
\end{pgfscope}%
\begin{pgfscope}%
\pgfpathrectangle{\pgfqpoint{0.750000in}{0.500000in}}{\pgfqpoint{4.650000in}{3.020000in}}%
\pgfusepath{clip}%
\pgfsetbuttcap%
\pgfsetroundjoin%
\definecolor{currentfill}{rgb}{1.000000,0.498039,0.054902}%
\pgfsetfillcolor{currentfill}%
\pgfsetlinewidth{1.003750pt}%
\definecolor{currentstroke}{rgb}{1.000000,0.498039,0.054902}%
\pgfsetstrokecolor{currentstroke}%
\pgfsetdash{}{0pt}%
\pgfpathmoveto{\pgfqpoint{2.108766in}{3.084039in}}%
\pgfpathcurveto{\pgfqpoint{2.119816in}{3.084039in}}{\pgfqpoint{2.130415in}{3.088430in}}{\pgfqpoint{2.138229in}{3.096243in}}%
\pgfpathcurveto{\pgfqpoint{2.146043in}{3.104057in}}{\pgfqpoint{2.150433in}{3.114656in}}{\pgfqpoint{2.150433in}{3.125706in}}%
\pgfpathcurveto{\pgfqpoint{2.150433in}{3.136756in}}{\pgfqpoint{2.146043in}{3.147355in}}{\pgfqpoint{2.138229in}{3.155169in}}%
\pgfpathcurveto{\pgfqpoint{2.130415in}{3.162982in}}{\pgfqpoint{2.119816in}{3.167373in}}{\pgfqpoint{2.108766in}{3.167373in}}%
\pgfpathcurveto{\pgfqpoint{2.097716in}{3.167373in}}{\pgfqpoint{2.087117in}{3.162982in}}{\pgfqpoint{2.079303in}{3.155169in}}%
\pgfpathcurveto{\pgfqpoint{2.071490in}{3.147355in}}{\pgfqpoint{2.067100in}{3.136756in}}{\pgfqpoint{2.067100in}{3.125706in}}%
\pgfpathcurveto{\pgfqpoint{2.067100in}{3.114656in}}{\pgfqpoint{2.071490in}{3.104057in}}{\pgfqpoint{2.079303in}{3.096243in}}%
\pgfpathcurveto{\pgfqpoint{2.087117in}{3.088430in}}{\pgfqpoint{2.097716in}{3.084039in}}{\pgfqpoint{2.108766in}{3.084039in}}%
\pgfpathclose%
\pgfusepath{stroke,fill}%
\end{pgfscope}%
\begin{pgfscope}%
\pgfpathrectangle{\pgfqpoint{0.750000in}{0.500000in}}{\pgfqpoint{4.650000in}{3.020000in}}%
\pgfusepath{clip}%
\pgfsetbuttcap%
\pgfsetroundjoin%
\definecolor{currentfill}{rgb}{1.000000,0.498039,0.054902}%
\pgfsetfillcolor{currentfill}%
\pgfsetlinewidth{1.003750pt}%
\definecolor{currentstroke}{rgb}{1.000000,0.498039,0.054902}%
\pgfsetstrokecolor{currentstroke}%
\pgfsetdash{}{0pt}%
\pgfpathmoveto{\pgfqpoint{2.350325in}{2.936057in}}%
\pgfpathcurveto{\pgfqpoint{2.361375in}{2.936057in}}{\pgfqpoint{2.371974in}{2.940448in}}{\pgfqpoint{2.379787in}{2.948261in}}%
\pgfpathcurveto{\pgfqpoint{2.387601in}{2.956075in}}{\pgfqpoint{2.391991in}{2.966674in}}{\pgfqpoint{2.391991in}{2.977724in}}%
\pgfpathcurveto{\pgfqpoint{2.391991in}{2.988774in}}{\pgfqpoint{2.387601in}{2.999373in}}{\pgfqpoint{2.379787in}{3.007187in}}%
\pgfpathcurveto{\pgfqpoint{2.371974in}{3.015000in}}{\pgfqpoint{2.361375in}{3.019391in}}{\pgfqpoint{2.350325in}{3.019391in}}%
\pgfpathcurveto{\pgfqpoint{2.339275in}{3.019391in}}{\pgfqpoint{2.328676in}{3.015000in}}{\pgfqpoint{2.320862in}{3.007187in}}%
\pgfpathcurveto{\pgfqpoint{2.313048in}{2.999373in}}{\pgfqpoint{2.308658in}{2.988774in}}{\pgfqpoint{2.308658in}{2.977724in}}%
\pgfpathcurveto{\pgfqpoint{2.308658in}{2.966674in}}{\pgfqpoint{2.313048in}{2.956075in}}{\pgfqpoint{2.320862in}{2.948261in}}%
\pgfpathcurveto{\pgfqpoint{2.328676in}{2.940448in}}{\pgfqpoint{2.339275in}{2.936057in}}{\pgfqpoint{2.350325in}{2.936057in}}%
\pgfpathclose%
\pgfusepath{stroke,fill}%
\end{pgfscope}%
\begin{pgfscope}%
\pgfpathrectangle{\pgfqpoint{0.750000in}{0.500000in}}{\pgfqpoint{4.650000in}{3.020000in}}%
\pgfusepath{clip}%
\pgfsetbuttcap%
\pgfsetroundjoin%
\definecolor{currentfill}{rgb}{1.000000,0.498039,0.054902}%
\pgfsetfillcolor{currentfill}%
\pgfsetlinewidth{1.003750pt}%
\definecolor{currentstroke}{rgb}{1.000000,0.498039,0.054902}%
\pgfsetstrokecolor{currentstroke}%
\pgfsetdash{}{0pt}%
\pgfpathmoveto{\pgfqpoint{1.686039in}{3.341061in}}%
\pgfpathcurveto{\pgfqpoint{1.697089in}{3.341061in}}{\pgfqpoint{1.707688in}{3.345451in}}{\pgfqpoint{1.715502in}{3.353264in}}%
\pgfpathcurveto{\pgfqpoint{1.723315in}{3.361078in}}{\pgfqpoint{1.727706in}{3.371677in}}{\pgfqpoint{1.727706in}{3.382727in}}%
\pgfpathcurveto{\pgfqpoint{1.727706in}{3.393777in}}{\pgfqpoint{1.723315in}{3.404376in}}{\pgfqpoint{1.715502in}{3.412190in}}%
\pgfpathcurveto{\pgfqpoint{1.707688in}{3.420004in}}{\pgfqpoint{1.697089in}{3.424394in}}{\pgfqpoint{1.686039in}{3.424394in}}%
\pgfpathcurveto{\pgfqpoint{1.674989in}{3.424394in}}{\pgfqpoint{1.664390in}{3.420004in}}{\pgfqpoint{1.656576in}{3.412190in}}%
\pgfpathcurveto{\pgfqpoint{1.648763in}{3.404376in}}{\pgfqpoint{1.644372in}{3.393777in}}{\pgfqpoint{1.644372in}{3.382727in}}%
\pgfpathcurveto{\pgfqpoint{1.644372in}{3.371677in}}{\pgfqpoint{1.648763in}{3.361078in}}{\pgfqpoint{1.656576in}{3.353264in}}%
\pgfpathcurveto{\pgfqpoint{1.664390in}{3.345451in}}{\pgfqpoint{1.674989in}{3.341061in}}{\pgfqpoint{1.686039in}{3.341061in}}%
\pgfpathclose%
\pgfusepath{stroke,fill}%
\end{pgfscope}%
\begin{pgfscope}%
\pgfpathrectangle{\pgfqpoint{0.750000in}{0.500000in}}{\pgfqpoint{4.650000in}{3.020000in}}%
\pgfusepath{clip}%
\pgfsetbuttcap%
\pgfsetroundjoin%
\definecolor{currentfill}{rgb}{1.000000,0.498039,0.054902}%
\pgfsetfillcolor{currentfill}%
\pgfsetlinewidth{1.003750pt}%
\definecolor{currentstroke}{rgb}{1.000000,0.498039,0.054902}%
\pgfsetstrokecolor{currentstroke}%
\pgfsetdash{}{0pt}%
\pgfpathmoveto{\pgfqpoint{1.504870in}{3.169713in}}%
\pgfpathcurveto{\pgfqpoint{1.515920in}{3.169713in}}{\pgfqpoint{1.526519in}{3.174103in}}{\pgfqpoint{1.534333in}{3.181917in}}%
\pgfpathcurveto{\pgfqpoint{1.542147in}{3.189731in}}{\pgfqpoint{1.546537in}{3.200330in}}{\pgfqpoint{1.546537in}{3.211380in}}%
\pgfpathcurveto{\pgfqpoint{1.546537in}{3.222430in}}{\pgfqpoint{1.542147in}{3.233029in}}{\pgfqpoint{1.534333in}{3.240843in}}%
\pgfpathcurveto{\pgfqpoint{1.526519in}{3.248656in}}{\pgfqpoint{1.515920in}{3.253046in}}{\pgfqpoint{1.504870in}{3.253046in}}%
\pgfpathcurveto{\pgfqpoint{1.493820in}{3.253046in}}{\pgfqpoint{1.483221in}{3.248656in}}{\pgfqpoint{1.475407in}{3.240843in}}%
\pgfpathcurveto{\pgfqpoint{1.467594in}{3.233029in}}{\pgfqpoint{1.463203in}{3.222430in}}{\pgfqpoint{1.463203in}{3.211380in}}%
\pgfpathcurveto{\pgfqpoint{1.463203in}{3.200330in}}{\pgfqpoint{1.467594in}{3.189731in}}{\pgfqpoint{1.475407in}{3.181917in}}%
\pgfpathcurveto{\pgfqpoint{1.483221in}{3.174103in}}{\pgfqpoint{1.493820in}{3.169713in}}{\pgfqpoint{1.504870in}{3.169713in}}%
\pgfpathclose%
\pgfusepath{stroke,fill}%
\end{pgfscope}%
\begin{pgfscope}%
\pgfpathrectangle{\pgfqpoint{0.750000in}{0.500000in}}{\pgfqpoint{4.650000in}{3.020000in}}%
\pgfusepath{clip}%
\pgfsetbuttcap%
\pgfsetroundjoin%
\definecolor{currentfill}{rgb}{1.000000,0.498039,0.054902}%
\pgfsetfillcolor{currentfill}%
\pgfsetlinewidth{1.003750pt}%
\definecolor{currentstroke}{rgb}{1.000000,0.498039,0.054902}%
\pgfsetstrokecolor{currentstroke}%
\pgfsetdash{}{0pt}%
\pgfpathmoveto{\pgfqpoint{1.927597in}{2.932163in}}%
\pgfpathcurveto{\pgfqpoint{1.938648in}{2.932163in}}{\pgfqpoint{1.949247in}{2.936553in}}{\pgfqpoint{1.957060in}{2.944367in}}%
\pgfpathcurveto{\pgfqpoint{1.964874in}{2.952181in}}{\pgfqpoint{1.969264in}{2.962780in}}{\pgfqpoint{1.969264in}{2.973830in}}%
\pgfpathcurveto{\pgfqpoint{1.969264in}{2.984880in}}{\pgfqpoint{1.964874in}{2.995479in}}{\pgfqpoint{1.957060in}{3.003293in}}%
\pgfpathcurveto{\pgfqpoint{1.949247in}{3.011106in}}{\pgfqpoint{1.938648in}{3.015496in}}{\pgfqpoint{1.927597in}{3.015496in}}%
\pgfpathcurveto{\pgfqpoint{1.916547in}{3.015496in}}{\pgfqpoint{1.905948in}{3.011106in}}{\pgfqpoint{1.898135in}{3.003293in}}%
\pgfpathcurveto{\pgfqpoint{1.890321in}{2.995479in}}{\pgfqpoint{1.885931in}{2.984880in}}{\pgfqpoint{1.885931in}{2.973830in}}%
\pgfpathcurveto{\pgfqpoint{1.885931in}{2.962780in}}{\pgfqpoint{1.890321in}{2.952181in}}{\pgfqpoint{1.898135in}{2.944367in}}%
\pgfpathcurveto{\pgfqpoint{1.905948in}{2.936553in}}{\pgfqpoint{1.916547in}{2.932163in}}{\pgfqpoint{1.927597in}{2.932163in}}%
\pgfpathclose%
\pgfusepath{stroke,fill}%
\end{pgfscope}%
\begin{pgfscope}%
\pgfpathrectangle{\pgfqpoint{0.750000in}{0.500000in}}{\pgfqpoint{4.650000in}{3.020000in}}%
\pgfusepath{clip}%
\pgfsetbuttcap%
\pgfsetroundjoin%
\definecolor{currentfill}{rgb}{1.000000,0.498039,0.054902}%
\pgfsetfillcolor{currentfill}%
\pgfsetlinewidth{1.003750pt}%
\definecolor{currentstroke}{rgb}{1.000000,0.498039,0.054902}%
\pgfsetstrokecolor{currentstroke}%
\pgfsetdash{}{0pt}%
\pgfpathmoveto{\pgfqpoint{1.686039in}{2.932163in}}%
\pgfpathcurveto{\pgfqpoint{1.697089in}{2.932163in}}{\pgfqpoint{1.707688in}{2.936553in}}{\pgfqpoint{1.715502in}{2.944367in}}%
\pgfpathcurveto{\pgfqpoint{1.723315in}{2.952181in}}{\pgfqpoint{1.727706in}{2.962780in}}{\pgfqpoint{1.727706in}{2.973830in}}%
\pgfpathcurveto{\pgfqpoint{1.727706in}{2.984880in}}{\pgfqpoint{1.723315in}{2.995479in}}{\pgfqpoint{1.715502in}{3.003293in}}%
\pgfpathcurveto{\pgfqpoint{1.707688in}{3.011106in}}{\pgfqpoint{1.697089in}{3.015496in}}{\pgfqpoint{1.686039in}{3.015496in}}%
\pgfpathcurveto{\pgfqpoint{1.674989in}{3.015496in}}{\pgfqpoint{1.664390in}{3.011106in}}{\pgfqpoint{1.656576in}{3.003293in}}%
\pgfpathcurveto{\pgfqpoint{1.648763in}{2.995479in}}{\pgfqpoint{1.644372in}{2.984880in}}{\pgfqpoint{1.644372in}{2.973830in}}%
\pgfpathcurveto{\pgfqpoint{1.644372in}{2.962780in}}{\pgfqpoint{1.648763in}{2.952181in}}{\pgfqpoint{1.656576in}{2.944367in}}%
\pgfpathcurveto{\pgfqpoint{1.664390in}{2.936553in}}{\pgfqpoint{1.674989in}{2.932163in}}{\pgfqpoint{1.686039in}{2.932163in}}%
\pgfpathclose%
\pgfusepath{stroke,fill}%
\end{pgfscope}%
\begin{pgfscope}%
\pgfpathrectangle{\pgfqpoint{0.750000in}{0.500000in}}{\pgfqpoint{4.650000in}{3.020000in}}%
\pgfusepath{clip}%
\pgfsetbuttcap%
\pgfsetroundjoin%
\definecolor{currentfill}{rgb}{1.000000,0.498039,0.054902}%
\pgfsetfillcolor{currentfill}%
\pgfsetlinewidth{1.003750pt}%
\definecolor{currentstroke}{rgb}{1.000000,0.498039,0.054902}%
\pgfsetstrokecolor{currentstroke}%
\pgfsetdash{}{0pt}%
\pgfpathmoveto{\pgfqpoint{3.437338in}{2.928269in}}%
\pgfpathcurveto{\pgfqpoint{3.448388in}{2.928269in}}{\pgfqpoint{3.458987in}{2.932659in}}{\pgfqpoint{3.466800in}{2.940473in}}%
\pgfpathcurveto{\pgfqpoint{3.474614in}{2.948286in}}{\pgfqpoint{3.479004in}{2.958885in}}{\pgfqpoint{3.479004in}{2.969936in}}%
\pgfpathcurveto{\pgfqpoint{3.479004in}{2.980986in}}{\pgfqpoint{3.474614in}{2.991585in}}{\pgfqpoint{3.466800in}{2.999398in}}%
\pgfpathcurveto{\pgfqpoint{3.458987in}{3.007212in}}{\pgfqpoint{3.448388in}{3.011602in}}{\pgfqpoint{3.437338in}{3.011602in}}%
\pgfpathcurveto{\pgfqpoint{3.426288in}{3.011602in}}{\pgfqpoint{3.415689in}{3.007212in}}{\pgfqpoint{3.407875in}{2.999398in}}%
\pgfpathcurveto{\pgfqpoint{3.400061in}{2.991585in}}{\pgfqpoint{3.395671in}{2.980986in}}{\pgfqpoint{3.395671in}{2.969936in}}%
\pgfpathcurveto{\pgfqpoint{3.395671in}{2.958885in}}{\pgfqpoint{3.400061in}{2.948286in}}{\pgfqpoint{3.407875in}{2.940473in}}%
\pgfpathcurveto{\pgfqpoint{3.415689in}{2.932659in}}{\pgfqpoint{3.426288in}{2.928269in}}{\pgfqpoint{3.437338in}{2.928269in}}%
\pgfpathclose%
\pgfusepath{stroke,fill}%
\end{pgfscope}%
\begin{pgfscope}%
\pgfpathrectangle{\pgfqpoint{0.750000in}{0.500000in}}{\pgfqpoint{4.650000in}{3.020000in}}%
\pgfusepath{clip}%
\pgfsetbuttcap%
\pgfsetroundjoin%
\definecolor{currentfill}{rgb}{1.000000,0.498039,0.054902}%
\pgfsetfillcolor{currentfill}%
\pgfsetlinewidth{1.003750pt}%
\definecolor{currentstroke}{rgb}{1.000000,0.498039,0.054902}%
\pgfsetstrokecolor{currentstroke}%
\pgfsetdash{}{0pt}%
\pgfpathmoveto{\pgfqpoint{1.323701in}{1.876818in}}%
\pgfpathcurveto{\pgfqpoint{1.334751in}{1.876818in}}{\pgfqpoint{1.345350in}{1.881208in}}{\pgfqpoint{1.353164in}{1.889022in}}%
\pgfpathcurveto{\pgfqpoint{1.360978in}{1.896836in}}{\pgfqpoint{1.365368in}{1.907435in}}{\pgfqpoint{1.365368in}{1.918485in}}%
\pgfpathcurveto{\pgfqpoint{1.365368in}{1.929535in}}{\pgfqpoint{1.360978in}{1.940134in}}{\pgfqpoint{1.353164in}{1.947948in}}%
\pgfpathcurveto{\pgfqpoint{1.345350in}{1.955761in}}{\pgfqpoint{1.334751in}{1.960152in}}{\pgfqpoint{1.323701in}{1.960152in}}%
\pgfpathcurveto{\pgfqpoint{1.312651in}{1.960152in}}{\pgfqpoint{1.302052in}{1.955761in}}{\pgfqpoint{1.294239in}{1.947948in}}%
\pgfpathcurveto{\pgfqpoint{1.286425in}{1.940134in}}{\pgfqpoint{1.282035in}{1.929535in}}{\pgfqpoint{1.282035in}{1.918485in}}%
\pgfpathcurveto{\pgfqpoint{1.282035in}{1.907435in}}{\pgfqpoint{1.286425in}{1.896836in}}{\pgfqpoint{1.294239in}{1.889022in}}%
\pgfpathcurveto{\pgfqpoint{1.302052in}{1.881208in}}{\pgfqpoint{1.312651in}{1.876818in}}{\pgfqpoint{1.323701in}{1.876818in}}%
\pgfpathclose%
\pgfusepath{stroke,fill}%
\end{pgfscope}%
\begin{pgfscope}%
\pgfpathrectangle{\pgfqpoint{0.750000in}{0.500000in}}{\pgfqpoint{4.650000in}{3.020000in}}%
\pgfusepath{clip}%
\pgfsetbuttcap%
\pgfsetroundjoin%
\definecolor{currentfill}{rgb}{1.000000,0.498039,0.054902}%
\pgfsetfillcolor{currentfill}%
\pgfsetlinewidth{1.003750pt}%
\definecolor{currentstroke}{rgb}{1.000000,0.498039,0.054902}%
\pgfsetstrokecolor{currentstroke}%
\pgfsetdash{}{0pt}%
\pgfpathmoveto{\pgfqpoint{2.531494in}{2.928269in}}%
\pgfpathcurveto{\pgfqpoint{2.542544in}{2.928269in}}{\pgfqpoint{2.553143in}{2.932659in}}{\pgfqpoint{2.560956in}{2.940473in}}%
\pgfpathcurveto{\pgfqpoint{2.568770in}{2.948286in}}{\pgfqpoint{2.573160in}{2.958885in}}{\pgfqpoint{2.573160in}{2.969936in}}%
\pgfpathcurveto{\pgfqpoint{2.573160in}{2.980986in}}{\pgfqpoint{2.568770in}{2.991585in}}{\pgfqpoint{2.560956in}{2.999398in}}%
\pgfpathcurveto{\pgfqpoint{2.553143in}{3.007212in}}{\pgfqpoint{2.542544in}{3.011602in}}{\pgfqpoint{2.531494in}{3.011602in}}%
\pgfpathcurveto{\pgfqpoint{2.520443in}{3.011602in}}{\pgfqpoint{2.509844in}{3.007212in}}{\pgfqpoint{2.502031in}{2.999398in}}%
\pgfpathcurveto{\pgfqpoint{2.494217in}{2.991585in}}{\pgfqpoint{2.489827in}{2.980986in}}{\pgfqpoint{2.489827in}{2.969936in}}%
\pgfpathcurveto{\pgfqpoint{2.489827in}{2.958885in}}{\pgfqpoint{2.494217in}{2.948286in}}{\pgfqpoint{2.502031in}{2.940473in}}%
\pgfpathcurveto{\pgfqpoint{2.509844in}{2.932659in}}{\pgfqpoint{2.520443in}{2.928269in}}{\pgfqpoint{2.531494in}{2.928269in}}%
\pgfpathclose%
\pgfusepath{stroke,fill}%
\end{pgfscope}%
\begin{pgfscope}%
\pgfpathrectangle{\pgfqpoint{0.750000in}{0.500000in}}{\pgfqpoint{4.650000in}{3.020000in}}%
\pgfusepath{clip}%
\pgfsetbuttcap%
\pgfsetroundjoin%
\definecolor{currentfill}{rgb}{1.000000,0.498039,0.054902}%
\pgfsetfillcolor{currentfill}%
\pgfsetlinewidth{1.003750pt}%
\definecolor{currentstroke}{rgb}{1.000000,0.498039,0.054902}%
\pgfsetstrokecolor{currentstroke}%
\pgfsetdash{}{0pt}%
\pgfpathmoveto{\pgfqpoint{1.867208in}{2.947740in}}%
\pgfpathcurveto{\pgfqpoint{1.878258in}{2.947740in}}{\pgfqpoint{1.888857in}{2.952130in}}{\pgfqpoint{1.896671in}{2.959944in}}%
\pgfpathcurveto{\pgfqpoint{1.904484in}{2.967758in}}{\pgfqpoint{1.908874in}{2.978357in}}{\pgfqpoint{1.908874in}{2.989407in}}%
\pgfpathcurveto{\pgfqpoint{1.908874in}{3.000457in}}{\pgfqpoint{1.904484in}{3.011056in}}{\pgfqpoint{1.896671in}{3.018870in}}%
\pgfpathcurveto{\pgfqpoint{1.888857in}{3.026683in}}{\pgfqpoint{1.878258in}{3.031074in}}{\pgfqpoint{1.867208in}{3.031074in}}%
\pgfpathcurveto{\pgfqpoint{1.856158in}{3.031074in}}{\pgfqpoint{1.845559in}{3.026683in}}{\pgfqpoint{1.837745in}{3.018870in}}%
\pgfpathcurveto{\pgfqpoint{1.829931in}{3.011056in}}{\pgfqpoint{1.825541in}{3.000457in}}{\pgfqpoint{1.825541in}{2.989407in}}%
\pgfpathcurveto{\pgfqpoint{1.825541in}{2.978357in}}{\pgfqpoint{1.829931in}{2.967758in}}{\pgfqpoint{1.837745in}{2.959944in}}%
\pgfpathcurveto{\pgfqpoint{1.845559in}{2.952130in}}{\pgfqpoint{1.856158in}{2.947740in}}{\pgfqpoint{1.867208in}{2.947740in}}%
\pgfpathclose%
\pgfusepath{stroke,fill}%
\end{pgfscope}%
\begin{pgfscope}%
\pgfpathrectangle{\pgfqpoint{0.750000in}{0.500000in}}{\pgfqpoint{4.650000in}{3.020000in}}%
\pgfusepath{clip}%
\pgfsetbuttcap%
\pgfsetroundjoin%
\definecolor{currentfill}{rgb}{1.000000,0.498039,0.054902}%
\pgfsetfillcolor{currentfill}%
\pgfsetlinewidth{1.003750pt}%
\definecolor{currentstroke}{rgb}{1.000000,0.498039,0.054902}%
\pgfsetstrokecolor{currentstroke}%
\pgfsetdash{}{0pt}%
\pgfpathmoveto{\pgfqpoint{1.444481in}{2.939952in}}%
\pgfpathcurveto{\pgfqpoint{1.455531in}{2.939952in}}{\pgfqpoint{1.466130in}{2.944342in}}{\pgfqpoint{1.473943in}{2.952156in}}%
\pgfpathcurveto{\pgfqpoint{1.481757in}{2.959969in}}{\pgfqpoint{1.486147in}{2.970568in}}{\pgfqpoint{1.486147in}{2.981618in}}%
\pgfpathcurveto{\pgfqpoint{1.486147in}{2.992668in}}{\pgfqpoint{1.481757in}{3.003267in}}{\pgfqpoint{1.473943in}{3.011081in}}%
\pgfpathcurveto{\pgfqpoint{1.466130in}{3.018895in}}{\pgfqpoint{1.455531in}{3.023285in}}{\pgfqpoint{1.444481in}{3.023285in}}%
\pgfpathcurveto{\pgfqpoint{1.433430in}{3.023285in}}{\pgfqpoint{1.422831in}{3.018895in}}{\pgfqpoint{1.415018in}{3.011081in}}%
\pgfpathcurveto{\pgfqpoint{1.407204in}{3.003267in}}{\pgfqpoint{1.402814in}{2.992668in}}{\pgfqpoint{1.402814in}{2.981618in}}%
\pgfpathcurveto{\pgfqpoint{1.402814in}{2.970568in}}{\pgfqpoint{1.407204in}{2.959969in}}{\pgfqpoint{1.415018in}{2.952156in}}%
\pgfpathcurveto{\pgfqpoint{1.422831in}{2.944342in}}{\pgfqpoint{1.433430in}{2.939952in}}{\pgfqpoint{1.444481in}{2.939952in}}%
\pgfpathclose%
\pgfusepath{stroke,fill}%
\end{pgfscope}%
\begin{pgfscope}%
\pgfpathrectangle{\pgfqpoint{0.750000in}{0.500000in}}{\pgfqpoint{4.650000in}{3.020000in}}%
\pgfusepath{clip}%
\pgfsetbuttcap%
\pgfsetroundjoin%
\definecolor{currentfill}{rgb}{1.000000,0.498039,0.054902}%
\pgfsetfillcolor{currentfill}%
\pgfsetlinewidth{1.003750pt}%
\definecolor{currentstroke}{rgb}{1.000000,0.498039,0.054902}%
\pgfsetstrokecolor{currentstroke}%
\pgfsetdash{}{0pt}%
\pgfpathmoveto{\pgfqpoint{1.987987in}{2.939952in}}%
\pgfpathcurveto{\pgfqpoint{1.999037in}{2.939952in}}{\pgfqpoint{2.009636in}{2.944342in}}{\pgfqpoint{2.017450in}{2.952156in}}%
\pgfpathcurveto{\pgfqpoint{2.025263in}{2.959969in}}{\pgfqpoint{2.029654in}{2.970568in}}{\pgfqpoint{2.029654in}{2.981618in}}%
\pgfpathcurveto{\pgfqpoint{2.029654in}{2.992668in}}{\pgfqpoint{2.025263in}{3.003267in}}{\pgfqpoint{2.017450in}{3.011081in}}%
\pgfpathcurveto{\pgfqpoint{2.009636in}{3.018895in}}{\pgfqpoint{1.999037in}{3.023285in}}{\pgfqpoint{1.987987in}{3.023285in}}%
\pgfpathcurveto{\pgfqpoint{1.976937in}{3.023285in}}{\pgfqpoint{1.966338in}{3.018895in}}{\pgfqpoint{1.958524in}{3.011081in}}%
\pgfpathcurveto{\pgfqpoint{1.950711in}{3.003267in}}{\pgfqpoint{1.946320in}{2.992668in}}{\pgfqpoint{1.946320in}{2.981618in}}%
\pgfpathcurveto{\pgfqpoint{1.946320in}{2.970568in}}{\pgfqpoint{1.950711in}{2.959969in}}{\pgfqpoint{1.958524in}{2.952156in}}%
\pgfpathcurveto{\pgfqpoint{1.966338in}{2.944342in}}{\pgfqpoint{1.976937in}{2.939952in}}{\pgfqpoint{1.987987in}{2.939952in}}%
\pgfpathclose%
\pgfusepath{stroke,fill}%
\end{pgfscope}%
\begin{pgfscope}%
\pgfpathrectangle{\pgfqpoint{0.750000in}{0.500000in}}{\pgfqpoint{4.650000in}{3.020000in}}%
\pgfusepath{clip}%
\pgfsetbuttcap%
\pgfsetroundjoin%
\definecolor{currentfill}{rgb}{0.121569,0.466667,0.705882}%
\pgfsetfillcolor{currentfill}%
\pgfsetlinewidth{1.003750pt}%
\definecolor{currentstroke}{rgb}{0.121569,0.466667,0.705882}%
\pgfsetstrokecolor{currentstroke}%
\pgfsetdash{}{0pt}%
\pgfpathmoveto{\pgfqpoint{1.202922in}{0.595606in}}%
\pgfpathcurveto{\pgfqpoint{1.213972in}{0.595606in}}{\pgfqpoint{1.224571in}{0.599996in}}{\pgfqpoint{1.232385in}{0.607810in}}%
\pgfpathcurveto{\pgfqpoint{1.240198in}{0.615624in}}{\pgfqpoint{1.244589in}{0.626223in}}{\pgfqpoint{1.244589in}{0.637273in}}%
\pgfpathcurveto{\pgfqpoint{1.244589in}{0.648323in}}{\pgfqpoint{1.240198in}{0.658922in}}{\pgfqpoint{1.232385in}{0.666736in}}%
\pgfpathcurveto{\pgfqpoint{1.224571in}{0.674549in}}{\pgfqpoint{1.213972in}{0.678939in}}{\pgfqpoint{1.202922in}{0.678939in}}%
\pgfpathcurveto{\pgfqpoint{1.191872in}{0.678939in}}{\pgfqpoint{1.181273in}{0.674549in}}{\pgfqpoint{1.173459in}{0.666736in}}%
\pgfpathcurveto{\pgfqpoint{1.165646in}{0.658922in}}{\pgfqpoint{1.161255in}{0.648323in}}{\pgfqpoint{1.161255in}{0.637273in}}%
\pgfpathcurveto{\pgfqpoint{1.161255in}{0.626223in}}{\pgfqpoint{1.165646in}{0.615624in}}{\pgfqpoint{1.173459in}{0.607810in}}%
\pgfpathcurveto{\pgfqpoint{1.181273in}{0.599996in}}{\pgfqpoint{1.191872in}{0.595606in}}{\pgfqpoint{1.202922in}{0.595606in}}%
\pgfpathclose%
\pgfusepath{stroke,fill}%
\end{pgfscope}%
\begin{pgfscope}%
\pgfpathrectangle{\pgfqpoint{0.750000in}{0.500000in}}{\pgfqpoint{4.650000in}{3.020000in}}%
\pgfusepath{clip}%
\pgfsetbuttcap%
\pgfsetroundjoin%
\definecolor{currentfill}{rgb}{1.000000,0.498039,0.054902}%
\pgfsetfillcolor{currentfill}%
\pgfsetlinewidth{1.003750pt}%
\definecolor{currentstroke}{rgb}{1.000000,0.498039,0.054902}%
\pgfsetstrokecolor{currentstroke}%
\pgfsetdash{}{0pt}%
\pgfpathmoveto{\pgfqpoint{1.444481in}{2.932163in}}%
\pgfpathcurveto{\pgfqpoint{1.455531in}{2.932163in}}{\pgfqpoint{1.466130in}{2.936553in}}{\pgfqpoint{1.473943in}{2.944367in}}%
\pgfpathcurveto{\pgfqpoint{1.481757in}{2.952181in}}{\pgfqpoint{1.486147in}{2.962780in}}{\pgfqpoint{1.486147in}{2.973830in}}%
\pgfpathcurveto{\pgfqpoint{1.486147in}{2.984880in}}{\pgfqpoint{1.481757in}{2.995479in}}{\pgfqpoint{1.473943in}{3.003293in}}%
\pgfpathcurveto{\pgfqpoint{1.466130in}{3.011106in}}{\pgfqpoint{1.455531in}{3.015496in}}{\pgfqpoint{1.444481in}{3.015496in}}%
\pgfpathcurveto{\pgfqpoint{1.433430in}{3.015496in}}{\pgfqpoint{1.422831in}{3.011106in}}{\pgfqpoint{1.415018in}{3.003293in}}%
\pgfpathcurveto{\pgfqpoint{1.407204in}{2.995479in}}{\pgfqpoint{1.402814in}{2.984880in}}{\pgfqpoint{1.402814in}{2.973830in}}%
\pgfpathcurveto{\pgfqpoint{1.402814in}{2.962780in}}{\pgfqpoint{1.407204in}{2.952181in}}{\pgfqpoint{1.415018in}{2.944367in}}%
\pgfpathcurveto{\pgfqpoint{1.422831in}{2.936553in}}{\pgfqpoint{1.433430in}{2.932163in}}{\pgfqpoint{1.444481in}{2.932163in}}%
\pgfpathclose%
\pgfusepath{stroke,fill}%
\end{pgfscope}%
\begin{pgfscope}%
\pgfpathrectangle{\pgfqpoint{0.750000in}{0.500000in}}{\pgfqpoint{4.650000in}{3.020000in}}%
\pgfusepath{clip}%
\pgfsetbuttcap%
\pgfsetroundjoin%
\definecolor{currentfill}{rgb}{1.000000,0.498039,0.054902}%
\pgfsetfillcolor{currentfill}%
\pgfsetlinewidth{1.003750pt}%
\definecolor{currentstroke}{rgb}{1.000000,0.498039,0.054902}%
\pgfsetstrokecolor{currentstroke}%
\pgfsetdash{}{0pt}%
\pgfpathmoveto{\pgfqpoint{1.686039in}{2.920480in}}%
\pgfpathcurveto{\pgfqpoint{1.697089in}{2.920480in}}{\pgfqpoint{1.707688in}{2.924871in}}{\pgfqpoint{1.715502in}{2.932684in}}%
\pgfpathcurveto{\pgfqpoint{1.723315in}{2.940498in}}{\pgfqpoint{1.727706in}{2.951097in}}{\pgfqpoint{1.727706in}{2.962147in}}%
\pgfpathcurveto{\pgfqpoint{1.727706in}{2.973197in}}{\pgfqpoint{1.723315in}{2.983796in}}{\pgfqpoint{1.715502in}{2.991610in}}%
\pgfpathcurveto{\pgfqpoint{1.707688in}{2.999423in}}{\pgfqpoint{1.697089in}{3.003814in}}{\pgfqpoint{1.686039in}{3.003814in}}%
\pgfpathcurveto{\pgfqpoint{1.674989in}{3.003814in}}{\pgfqpoint{1.664390in}{2.999423in}}{\pgfqpoint{1.656576in}{2.991610in}}%
\pgfpathcurveto{\pgfqpoint{1.648763in}{2.983796in}}{\pgfqpoint{1.644372in}{2.973197in}}{\pgfqpoint{1.644372in}{2.962147in}}%
\pgfpathcurveto{\pgfqpoint{1.644372in}{2.951097in}}{\pgfqpoint{1.648763in}{2.940498in}}{\pgfqpoint{1.656576in}{2.932684in}}%
\pgfpathcurveto{\pgfqpoint{1.664390in}{2.924871in}}{\pgfqpoint{1.674989in}{2.920480in}}{\pgfqpoint{1.686039in}{2.920480in}}%
\pgfpathclose%
\pgfusepath{stroke,fill}%
\end{pgfscope}%
\begin{pgfscope}%
\pgfpathrectangle{\pgfqpoint{0.750000in}{0.500000in}}{\pgfqpoint{4.650000in}{3.020000in}}%
\pgfusepath{clip}%
\pgfsetbuttcap%
\pgfsetroundjoin%
\definecolor{currentfill}{rgb}{1.000000,0.498039,0.054902}%
\pgfsetfillcolor{currentfill}%
\pgfsetlinewidth{1.003750pt}%
\definecolor{currentstroke}{rgb}{1.000000,0.498039,0.054902}%
\pgfsetstrokecolor{currentstroke}%
\pgfsetdash{}{0pt}%
\pgfpathmoveto{\pgfqpoint{2.289935in}{3.247598in}}%
\pgfpathcurveto{\pgfqpoint{2.300985in}{3.247598in}}{\pgfqpoint{2.311584in}{3.251989in}}{\pgfqpoint{2.319398in}{3.259802in}}%
\pgfpathcurveto{\pgfqpoint{2.327211in}{3.267616in}}{\pgfqpoint{2.331602in}{3.278215in}}{\pgfqpoint{2.331602in}{3.289265in}}%
\pgfpathcurveto{\pgfqpoint{2.331602in}{3.300315in}}{\pgfqpoint{2.327211in}{3.310914in}}{\pgfqpoint{2.319398in}{3.318728in}}%
\pgfpathcurveto{\pgfqpoint{2.311584in}{3.326541in}}{\pgfqpoint{2.300985in}{3.330932in}}{\pgfqpoint{2.289935in}{3.330932in}}%
\pgfpathcurveto{\pgfqpoint{2.278885in}{3.330932in}}{\pgfqpoint{2.268286in}{3.326541in}}{\pgfqpoint{2.260472in}{3.318728in}}%
\pgfpathcurveto{\pgfqpoint{2.252659in}{3.310914in}}{\pgfqpoint{2.248268in}{3.300315in}}{\pgfqpoint{2.248268in}{3.289265in}}%
\pgfpathcurveto{\pgfqpoint{2.248268in}{3.278215in}}{\pgfqpoint{2.252659in}{3.267616in}}{\pgfqpoint{2.260472in}{3.259802in}}%
\pgfpathcurveto{\pgfqpoint{2.268286in}{3.251989in}}{\pgfqpoint{2.278885in}{3.247598in}}{\pgfqpoint{2.289935in}{3.247598in}}%
\pgfpathclose%
\pgfusepath{stroke,fill}%
\end{pgfscope}%
\begin{pgfscope}%
\pgfpathrectangle{\pgfqpoint{0.750000in}{0.500000in}}{\pgfqpoint{4.650000in}{3.020000in}}%
\pgfusepath{clip}%
\pgfsetbuttcap%
\pgfsetroundjoin%
\definecolor{currentfill}{rgb}{1.000000,0.498039,0.054902}%
\pgfsetfillcolor{currentfill}%
\pgfsetlinewidth{1.003750pt}%
\definecolor{currentstroke}{rgb}{1.000000,0.498039,0.054902}%
\pgfsetstrokecolor{currentstroke}%
\pgfsetdash{}{0pt}%
\pgfpathmoveto{\pgfqpoint{2.833442in}{2.936057in}}%
\pgfpathcurveto{\pgfqpoint{2.844492in}{2.936057in}}{\pgfqpoint{2.855091in}{2.940448in}}{\pgfqpoint{2.862904in}{2.948261in}}%
\pgfpathcurveto{\pgfqpoint{2.870718in}{2.956075in}}{\pgfqpoint{2.875108in}{2.966674in}}{\pgfqpoint{2.875108in}{2.977724in}}%
\pgfpathcurveto{\pgfqpoint{2.875108in}{2.988774in}}{\pgfqpoint{2.870718in}{2.999373in}}{\pgfqpoint{2.862904in}{3.007187in}}%
\pgfpathcurveto{\pgfqpoint{2.855091in}{3.015000in}}{\pgfqpoint{2.844492in}{3.019391in}}{\pgfqpoint{2.833442in}{3.019391in}}%
\pgfpathcurveto{\pgfqpoint{2.822391in}{3.019391in}}{\pgfqpoint{2.811792in}{3.015000in}}{\pgfqpoint{2.803979in}{3.007187in}}%
\pgfpathcurveto{\pgfqpoint{2.796165in}{2.999373in}}{\pgfqpoint{2.791775in}{2.988774in}}{\pgfqpoint{2.791775in}{2.977724in}}%
\pgfpathcurveto{\pgfqpoint{2.791775in}{2.966674in}}{\pgfqpoint{2.796165in}{2.956075in}}{\pgfqpoint{2.803979in}{2.948261in}}%
\pgfpathcurveto{\pgfqpoint{2.811792in}{2.940448in}}{\pgfqpoint{2.822391in}{2.936057in}}{\pgfqpoint{2.833442in}{2.936057in}}%
\pgfpathclose%
\pgfusepath{stroke,fill}%
\end{pgfscope}%
\begin{pgfscope}%
\pgfpathrectangle{\pgfqpoint{0.750000in}{0.500000in}}{\pgfqpoint{4.650000in}{3.020000in}}%
\pgfusepath{clip}%
\pgfsetbuttcap%
\pgfsetroundjoin%
\definecolor{currentfill}{rgb}{1.000000,0.498039,0.054902}%
\pgfsetfillcolor{currentfill}%
\pgfsetlinewidth{1.003750pt}%
\definecolor{currentstroke}{rgb}{1.000000,0.498039,0.054902}%
\pgfsetstrokecolor{currentstroke}%
\pgfsetdash{}{0pt}%
\pgfpathmoveto{\pgfqpoint{1.444481in}{2.328553in}}%
\pgfpathcurveto{\pgfqpoint{1.455531in}{2.328553in}}{\pgfqpoint{1.466130in}{2.332943in}}{\pgfqpoint{1.473943in}{2.340756in}}%
\pgfpathcurveto{\pgfqpoint{1.481757in}{2.348570in}}{\pgfqpoint{1.486147in}{2.359169in}}{\pgfqpoint{1.486147in}{2.370219in}}%
\pgfpathcurveto{\pgfqpoint{1.486147in}{2.381269in}}{\pgfqpoint{1.481757in}{2.391868in}}{\pgfqpoint{1.473943in}{2.399682in}}%
\pgfpathcurveto{\pgfqpoint{1.466130in}{2.407496in}}{\pgfqpoint{1.455531in}{2.411886in}}{\pgfqpoint{1.444481in}{2.411886in}}%
\pgfpathcurveto{\pgfqpoint{1.433430in}{2.411886in}}{\pgfqpoint{1.422831in}{2.407496in}}{\pgfqpoint{1.415018in}{2.399682in}}%
\pgfpathcurveto{\pgfqpoint{1.407204in}{2.391868in}}{\pgfqpoint{1.402814in}{2.381269in}}{\pgfqpoint{1.402814in}{2.370219in}}%
\pgfpathcurveto{\pgfqpoint{1.402814in}{2.359169in}}{\pgfqpoint{1.407204in}{2.348570in}}{\pgfqpoint{1.415018in}{2.340756in}}%
\pgfpathcurveto{\pgfqpoint{1.422831in}{2.332943in}}{\pgfqpoint{1.433430in}{2.328553in}}{\pgfqpoint{1.444481in}{2.328553in}}%
\pgfpathclose%
\pgfusepath{stroke,fill}%
\end{pgfscope}%
\begin{pgfscope}%
\pgfpathrectangle{\pgfqpoint{0.750000in}{0.500000in}}{\pgfqpoint{4.650000in}{3.020000in}}%
\pgfusepath{clip}%
\pgfsetbuttcap%
\pgfsetroundjoin%
\definecolor{currentfill}{rgb}{1.000000,0.498039,0.054902}%
\pgfsetfillcolor{currentfill}%
\pgfsetlinewidth{1.003750pt}%
\definecolor{currentstroke}{rgb}{1.000000,0.498039,0.054902}%
\pgfsetstrokecolor{currentstroke}%
\pgfsetdash{}{0pt}%
\pgfpathmoveto{\pgfqpoint{1.746429in}{3.181396in}}%
\pgfpathcurveto{\pgfqpoint{1.757479in}{3.181396in}}{\pgfqpoint{1.768078in}{3.185786in}}{\pgfqpoint{1.775891in}{3.193600in}}%
\pgfpathcurveto{\pgfqpoint{1.783705in}{3.201413in}}{\pgfqpoint{1.788095in}{3.212012in}}{\pgfqpoint{1.788095in}{3.223063in}}%
\pgfpathcurveto{\pgfqpoint{1.788095in}{3.234113in}}{\pgfqpoint{1.783705in}{3.244712in}}{\pgfqpoint{1.775891in}{3.252525in}}%
\pgfpathcurveto{\pgfqpoint{1.768078in}{3.260339in}}{\pgfqpoint{1.757479in}{3.264729in}}{\pgfqpoint{1.746429in}{3.264729in}}%
\pgfpathcurveto{\pgfqpoint{1.735378in}{3.264729in}}{\pgfqpoint{1.724779in}{3.260339in}}{\pgfqpoint{1.716966in}{3.252525in}}%
\pgfpathcurveto{\pgfqpoint{1.709152in}{3.244712in}}{\pgfqpoint{1.704762in}{3.234113in}}{\pgfqpoint{1.704762in}{3.223063in}}%
\pgfpathcurveto{\pgfqpoint{1.704762in}{3.212012in}}{\pgfqpoint{1.709152in}{3.201413in}}{\pgfqpoint{1.716966in}{3.193600in}}%
\pgfpathcurveto{\pgfqpoint{1.724779in}{3.185786in}}{\pgfqpoint{1.735378in}{3.181396in}}{\pgfqpoint{1.746429in}{3.181396in}}%
\pgfpathclose%
\pgfusepath{stroke,fill}%
\end{pgfscope}%
\begin{pgfscope}%
\pgfpathrectangle{\pgfqpoint{0.750000in}{0.500000in}}{\pgfqpoint{4.650000in}{3.020000in}}%
\pgfusepath{clip}%
\pgfsetbuttcap%
\pgfsetroundjoin%
\definecolor{currentfill}{rgb}{0.839216,0.152941,0.156863}%
\pgfsetfillcolor{currentfill}%
\pgfsetlinewidth{1.003750pt}%
\definecolor{currentstroke}{rgb}{0.839216,0.152941,0.156863}%
\pgfsetstrokecolor{currentstroke}%
\pgfsetdash{}{0pt}%
\pgfpathmoveto{\pgfqpoint{1.806818in}{2.936057in}}%
\pgfpathcurveto{\pgfqpoint{1.817868in}{2.936057in}}{\pgfqpoint{1.828467in}{2.940448in}}{\pgfqpoint{1.836281in}{2.948261in}}%
\pgfpathcurveto{\pgfqpoint{1.844095in}{2.956075in}}{\pgfqpoint{1.848485in}{2.966674in}}{\pgfqpoint{1.848485in}{2.977724in}}%
\pgfpathcurveto{\pgfqpoint{1.848485in}{2.988774in}}{\pgfqpoint{1.844095in}{2.999373in}}{\pgfqpoint{1.836281in}{3.007187in}}%
\pgfpathcurveto{\pgfqpoint{1.828467in}{3.015000in}}{\pgfqpoint{1.817868in}{3.019391in}}{\pgfqpoint{1.806818in}{3.019391in}}%
\pgfpathcurveto{\pgfqpoint{1.795768in}{3.019391in}}{\pgfqpoint{1.785169in}{3.015000in}}{\pgfqpoint{1.777355in}{3.007187in}}%
\pgfpathcurveto{\pgfqpoint{1.769542in}{2.999373in}}{\pgfqpoint{1.765152in}{2.988774in}}{\pgfqpoint{1.765152in}{2.977724in}}%
\pgfpathcurveto{\pgfqpoint{1.765152in}{2.966674in}}{\pgfqpoint{1.769542in}{2.956075in}}{\pgfqpoint{1.777355in}{2.948261in}}%
\pgfpathcurveto{\pgfqpoint{1.785169in}{2.940448in}}{\pgfqpoint{1.795768in}{2.936057in}}{\pgfqpoint{1.806818in}{2.936057in}}%
\pgfpathclose%
\pgfusepath{stroke,fill}%
\end{pgfscope}%
\begin{pgfscope}%
\pgfpathrectangle{\pgfqpoint{0.750000in}{0.500000in}}{\pgfqpoint{4.650000in}{3.020000in}}%
\pgfusepath{clip}%
\pgfsetbuttcap%
\pgfsetroundjoin%
\definecolor{currentfill}{rgb}{1.000000,0.498039,0.054902}%
\pgfsetfillcolor{currentfill}%
\pgfsetlinewidth{1.003750pt}%
\definecolor{currentstroke}{rgb}{1.000000,0.498039,0.054902}%
\pgfsetstrokecolor{currentstroke}%
\pgfsetdash{}{0pt}%
\pgfpathmoveto{\pgfqpoint{1.746429in}{2.936057in}}%
\pgfpathcurveto{\pgfqpoint{1.757479in}{2.936057in}}{\pgfqpoint{1.768078in}{2.940448in}}{\pgfqpoint{1.775891in}{2.948261in}}%
\pgfpathcurveto{\pgfqpoint{1.783705in}{2.956075in}}{\pgfqpoint{1.788095in}{2.966674in}}{\pgfqpoint{1.788095in}{2.977724in}}%
\pgfpathcurveto{\pgfqpoint{1.788095in}{2.988774in}}{\pgfqpoint{1.783705in}{2.999373in}}{\pgfqpoint{1.775891in}{3.007187in}}%
\pgfpathcurveto{\pgfqpoint{1.768078in}{3.015000in}}{\pgfqpoint{1.757479in}{3.019391in}}{\pgfqpoint{1.746429in}{3.019391in}}%
\pgfpathcurveto{\pgfqpoint{1.735378in}{3.019391in}}{\pgfqpoint{1.724779in}{3.015000in}}{\pgfqpoint{1.716966in}{3.007187in}}%
\pgfpathcurveto{\pgfqpoint{1.709152in}{2.999373in}}{\pgfqpoint{1.704762in}{2.988774in}}{\pgfqpoint{1.704762in}{2.977724in}}%
\pgfpathcurveto{\pgfqpoint{1.704762in}{2.966674in}}{\pgfqpoint{1.709152in}{2.956075in}}{\pgfqpoint{1.716966in}{2.948261in}}%
\pgfpathcurveto{\pgfqpoint{1.724779in}{2.940448in}}{\pgfqpoint{1.735378in}{2.936057in}}{\pgfqpoint{1.746429in}{2.936057in}}%
\pgfpathclose%
\pgfusepath{stroke,fill}%
\end{pgfscope}%
\begin{pgfscope}%
\pgfpathrectangle{\pgfqpoint{0.750000in}{0.500000in}}{\pgfqpoint{4.650000in}{3.020000in}}%
\pgfusepath{clip}%
\pgfsetbuttcap%
\pgfsetroundjoin%
\definecolor{currentfill}{rgb}{0.121569,0.466667,0.705882}%
\pgfsetfillcolor{currentfill}%
\pgfsetlinewidth{1.003750pt}%
\definecolor{currentstroke}{rgb}{0.121569,0.466667,0.705882}%
\pgfsetstrokecolor{currentstroke}%
\pgfsetdash{}{0pt}%
\pgfpathmoveto{\pgfqpoint{1.021753in}{0.595606in}}%
\pgfpathcurveto{\pgfqpoint{1.032803in}{0.595606in}}{\pgfqpoint{1.043402in}{0.599996in}}{\pgfqpoint{1.051216in}{0.607810in}}%
\pgfpathcurveto{\pgfqpoint{1.059030in}{0.615624in}}{\pgfqpoint{1.063420in}{0.626223in}}{\pgfqpoint{1.063420in}{0.637273in}}%
\pgfpathcurveto{\pgfqpoint{1.063420in}{0.648323in}}{\pgfqpoint{1.059030in}{0.658922in}}{\pgfqpoint{1.051216in}{0.666736in}}%
\pgfpathcurveto{\pgfqpoint{1.043402in}{0.674549in}}{\pgfqpoint{1.032803in}{0.678939in}}{\pgfqpoint{1.021753in}{0.678939in}}%
\pgfpathcurveto{\pgfqpoint{1.010703in}{0.678939in}}{\pgfqpoint{1.000104in}{0.674549in}}{\pgfqpoint{0.992290in}{0.666736in}}%
\pgfpathcurveto{\pgfqpoint{0.984477in}{0.658922in}}{\pgfqpoint{0.980087in}{0.648323in}}{\pgfqpoint{0.980087in}{0.637273in}}%
\pgfpathcurveto{\pgfqpoint{0.980087in}{0.626223in}}{\pgfqpoint{0.984477in}{0.615624in}}{\pgfqpoint{0.992290in}{0.607810in}}%
\pgfpathcurveto{\pgfqpoint{1.000104in}{0.599996in}}{\pgfqpoint{1.010703in}{0.595606in}}{\pgfqpoint{1.021753in}{0.595606in}}%
\pgfpathclose%
\pgfusepath{stroke,fill}%
\end{pgfscope}%
\begin{pgfscope}%
\pgfpathrectangle{\pgfqpoint{0.750000in}{0.500000in}}{\pgfqpoint{4.650000in}{3.020000in}}%
\pgfusepath{clip}%
\pgfsetbuttcap%
\pgfsetroundjoin%
\definecolor{currentfill}{rgb}{1.000000,0.498039,0.054902}%
\pgfsetfillcolor{currentfill}%
\pgfsetlinewidth{1.003750pt}%
\definecolor{currentstroke}{rgb}{1.000000,0.498039,0.054902}%
\pgfsetstrokecolor{currentstroke}%
\pgfsetdash{}{0pt}%
\pgfpathmoveto{\pgfqpoint{1.565260in}{2.932163in}}%
\pgfpathcurveto{\pgfqpoint{1.576310in}{2.932163in}}{\pgfqpoint{1.586909in}{2.936553in}}{\pgfqpoint{1.594723in}{2.944367in}}%
\pgfpathcurveto{\pgfqpoint{1.602536in}{2.952181in}}{\pgfqpoint{1.606926in}{2.962780in}}{\pgfqpoint{1.606926in}{2.973830in}}%
\pgfpathcurveto{\pgfqpoint{1.606926in}{2.984880in}}{\pgfqpoint{1.602536in}{2.995479in}}{\pgfqpoint{1.594723in}{3.003293in}}%
\pgfpathcurveto{\pgfqpoint{1.586909in}{3.011106in}}{\pgfqpoint{1.576310in}{3.015496in}}{\pgfqpoint{1.565260in}{3.015496in}}%
\pgfpathcurveto{\pgfqpoint{1.554210in}{3.015496in}}{\pgfqpoint{1.543611in}{3.011106in}}{\pgfqpoint{1.535797in}{3.003293in}}%
\pgfpathcurveto{\pgfqpoint{1.527983in}{2.995479in}}{\pgfqpoint{1.523593in}{2.984880in}}{\pgfqpoint{1.523593in}{2.973830in}}%
\pgfpathcurveto{\pgfqpoint{1.523593in}{2.962780in}}{\pgfqpoint{1.527983in}{2.952181in}}{\pgfqpoint{1.535797in}{2.944367in}}%
\pgfpathcurveto{\pgfqpoint{1.543611in}{2.936553in}}{\pgfqpoint{1.554210in}{2.932163in}}{\pgfqpoint{1.565260in}{2.932163in}}%
\pgfpathclose%
\pgfusepath{stroke,fill}%
\end{pgfscope}%
\begin{pgfscope}%
\pgfpathrectangle{\pgfqpoint{0.750000in}{0.500000in}}{\pgfqpoint{4.650000in}{3.020000in}}%
\pgfusepath{clip}%
\pgfsetbuttcap%
\pgfsetroundjoin%
\definecolor{currentfill}{rgb}{1.000000,0.498039,0.054902}%
\pgfsetfillcolor{currentfill}%
\pgfsetlinewidth{1.003750pt}%
\definecolor{currentstroke}{rgb}{1.000000,0.498039,0.054902}%
\pgfsetstrokecolor{currentstroke}%
\pgfsetdash{}{0pt}%
\pgfpathmoveto{\pgfqpoint{2.048377in}{2.044271in}}%
\pgfpathcurveto{\pgfqpoint{2.059427in}{2.044271in}}{\pgfqpoint{2.070026in}{2.048662in}}{\pgfqpoint{2.077839in}{2.056475in}}%
\pgfpathcurveto{\pgfqpoint{2.085653in}{2.064289in}}{\pgfqpoint{2.090043in}{2.074888in}}{\pgfqpoint{2.090043in}{2.085938in}}%
\pgfpathcurveto{\pgfqpoint{2.090043in}{2.096988in}}{\pgfqpoint{2.085653in}{2.107587in}}{\pgfqpoint{2.077839in}{2.115401in}}%
\pgfpathcurveto{\pgfqpoint{2.070026in}{2.123215in}}{\pgfqpoint{2.059427in}{2.127605in}}{\pgfqpoint{2.048377in}{2.127605in}}%
\pgfpathcurveto{\pgfqpoint{2.037326in}{2.127605in}}{\pgfqpoint{2.026727in}{2.123215in}}{\pgfqpoint{2.018914in}{2.115401in}}%
\pgfpathcurveto{\pgfqpoint{2.011100in}{2.107587in}}{\pgfqpoint{2.006710in}{2.096988in}}{\pgfqpoint{2.006710in}{2.085938in}}%
\pgfpathcurveto{\pgfqpoint{2.006710in}{2.074888in}}{\pgfqpoint{2.011100in}{2.064289in}}{\pgfqpoint{2.018914in}{2.056475in}}%
\pgfpathcurveto{\pgfqpoint{2.026727in}{2.048662in}}{\pgfqpoint{2.037326in}{2.044271in}}{\pgfqpoint{2.048377in}{2.044271in}}%
\pgfpathclose%
\pgfusepath{stroke,fill}%
\end{pgfscope}%
\begin{pgfscope}%
\pgfpathrectangle{\pgfqpoint{0.750000in}{0.500000in}}{\pgfqpoint{4.650000in}{3.020000in}}%
\pgfusepath{clip}%
\pgfsetbuttcap%
\pgfsetroundjoin%
\definecolor{currentfill}{rgb}{0.839216,0.152941,0.156863}%
\pgfsetfillcolor{currentfill}%
\pgfsetlinewidth{1.003750pt}%
\definecolor{currentstroke}{rgb}{0.839216,0.152941,0.156863}%
\pgfsetstrokecolor{currentstroke}%
\pgfsetdash{}{0pt}%
\pgfpathmoveto{\pgfqpoint{1.746429in}{2.936057in}}%
\pgfpathcurveto{\pgfqpoint{1.757479in}{2.936057in}}{\pgfqpoint{1.768078in}{2.940448in}}{\pgfqpoint{1.775891in}{2.948261in}}%
\pgfpathcurveto{\pgfqpoint{1.783705in}{2.956075in}}{\pgfqpoint{1.788095in}{2.966674in}}{\pgfqpoint{1.788095in}{2.977724in}}%
\pgfpathcurveto{\pgfqpoint{1.788095in}{2.988774in}}{\pgfqpoint{1.783705in}{2.999373in}}{\pgfqpoint{1.775891in}{3.007187in}}%
\pgfpathcurveto{\pgfqpoint{1.768078in}{3.015000in}}{\pgfqpoint{1.757479in}{3.019391in}}{\pgfqpoint{1.746429in}{3.019391in}}%
\pgfpathcurveto{\pgfqpoint{1.735378in}{3.019391in}}{\pgfqpoint{1.724779in}{3.015000in}}{\pgfqpoint{1.716966in}{3.007187in}}%
\pgfpathcurveto{\pgfqpoint{1.709152in}{2.999373in}}{\pgfqpoint{1.704762in}{2.988774in}}{\pgfqpoint{1.704762in}{2.977724in}}%
\pgfpathcurveto{\pgfqpoint{1.704762in}{2.966674in}}{\pgfqpoint{1.709152in}{2.956075in}}{\pgfqpoint{1.716966in}{2.948261in}}%
\pgfpathcurveto{\pgfqpoint{1.724779in}{2.940448in}}{\pgfqpoint{1.735378in}{2.936057in}}{\pgfqpoint{1.746429in}{2.936057in}}%
\pgfpathclose%
\pgfusepath{stroke,fill}%
\end{pgfscope}%
\begin{pgfscope}%
\pgfpathrectangle{\pgfqpoint{0.750000in}{0.500000in}}{\pgfqpoint{4.650000in}{3.020000in}}%
\pgfusepath{clip}%
\pgfsetbuttcap%
\pgfsetroundjoin%
\definecolor{currentfill}{rgb}{1.000000,0.498039,0.054902}%
\pgfsetfillcolor{currentfill}%
\pgfsetlinewidth{1.003750pt}%
\definecolor{currentstroke}{rgb}{1.000000,0.498039,0.054902}%
\pgfsetstrokecolor{currentstroke}%
\pgfsetdash{}{0pt}%
\pgfpathmoveto{\pgfqpoint{5.188636in}{2.877643in}}%
\pgfpathcurveto{\pgfqpoint{5.199686in}{2.877643in}}{\pgfqpoint{5.210286in}{2.882034in}}{\pgfqpoint{5.218099in}{2.889847in}}%
\pgfpathcurveto{\pgfqpoint{5.225913in}{2.897661in}}{\pgfqpoint{5.230303in}{2.908260in}}{\pgfqpoint{5.230303in}{2.919310in}}%
\pgfpathcurveto{\pgfqpoint{5.230303in}{2.930360in}}{\pgfqpoint{5.225913in}{2.940959in}}{\pgfqpoint{5.218099in}{2.948773in}}%
\pgfpathcurveto{\pgfqpoint{5.210286in}{2.956587in}}{\pgfqpoint{5.199686in}{2.960977in}}{\pgfqpoint{5.188636in}{2.960977in}}%
\pgfpathcurveto{\pgfqpoint{5.177586in}{2.960977in}}{\pgfqpoint{5.166987in}{2.956587in}}{\pgfqpoint{5.159174in}{2.948773in}}%
\pgfpathcurveto{\pgfqpoint{5.151360in}{2.940959in}}{\pgfqpoint{5.146970in}{2.930360in}}{\pgfqpoint{5.146970in}{2.919310in}}%
\pgfpathcurveto{\pgfqpoint{5.146970in}{2.908260in}}{\pgfqpoint{5.151360in}{2.897661in}}{\pgfqpoint{5.159174in}{2.889847in}}%
\pgfpathcurveto{\pgfqpoint{5.166987in}{2.882034in}}{\pgfqpoint{5.177586in}{2.877643in}}{\pgfqpoint{5.188636in}{2.877643in}}%
\pgfpathclose%
\pgfusepath{stroke,fill}%
\end{pgfscope}%
\begin{pgfscope}%
\pgfpathrectangle{\pgfqpoint{0.750000in}{0.500000in}}{\pgfqpoint{4.650000in}{3.020000in}}%
\pgfusepath{clip}%
\pgfsetbuttcap%
\pgfsetroundjoin%
\definecolor{currentfill}{rgb}{1.000000,0.498039,0.054902}%
\pgfsetfillcolor{currentfill}%
\pgfsetlinewidth{1.003750pt}%
\definecolor{currentstroke}{rgb}{1.000000,0.498039,0.054902}%
\pgfsetstrokecolor{currentstroke}%
\pgfsetdash{}{0pt}%
\pgfpathmoveto{\pgfqpoint{2.229545in}{2.928269in}}%
\pgfpathcurveto{\pgfqpoint{2.240596in}{2.928269in}}{\pgfqpoint{2.251195in}{2.932659in}}{\pgfqpoint{2.259008in}{2.940473in}}%
\pgfpathcurveto{\pgfqpoint{2.266822in}{2.948286in}}{\pgfqpoint{2.271212in}{2.958885in}}{\pgfqpoint{2.271212in}{2.969936in}}%
\pgfpathcurveto{\pgfqpoint{2.271212in}{2.980986in}}{\pgfqpoint{2.266822in}{2.991585in}}{\pgfqpoint{2.259008in}{2.999398in}}%
\pgfpathcurveto{\pgfqpoint{2.251195in}{3.007212in}}{\pgfqpoint{2.240596in}{3.011602in}}{\pgfqpoint{2.229545in}{3.011602in}}%
\pgfpathcurveto{\pgfqpoint{2.218495in}{3.011602in}}{\pgfqpoint{2.207896in}{3.007212in}}{\pgfqpoint{2.200083in}{2.999398in}}%
\pgfpathcurveto{\pgfqpoint{2.192269in}{2.991585in}}{\pgfqpoint{2.187879in}{2.980986in}}{\pgfqpoint{2.187879in}{2.969936in}}%
\pgfpathcurveto{\pgfqpoint{2.187879in}{2.958885in}}{\pgfqpoint{2.192269in}{2.948286in}}{\pgfqpoint{2.200083in}{2.940473in}}%
\pgfpathcurveto{\pgfqpoint{2.207896in}{2.932659in}}{\pgfqpoint{2.218495in}{2.928269in}}{\pgfqpoint{2.229545in}{2.928269in}}%
\pgfpathclose%
\pgfusepath{stroke,fill}%
\end{pgfscope}%
\begin{pgfscope}%
\pgfpathrectangle{\pgfqpoint{0.750000in}{0.500000in}}{\pgfqpoint{4.650000in}{3.020000in}}%
\pgfusepath{clip}%
\pgfsetbuttcap%
\pgfsetroundjoin%
\definecolor{currentfill}{rgb}{1.000000,0.498039,0.054902}%
\pgfsetfillcolor{currentfill}%
\pgfsetlinewidth{1.003750pt}%
\definecolor{currentstroke}{rgb}{1.000000,0.498039,0.054902}%
\pgfsetstrokecolor{currentstroke}%
\pgfsetdash{}{0pt}%
\pgfpathmoveto{\pgfqpoint{1.746429in}{2.936057in}}%
\pgfpathcurveto{\pgfqpoint{1.757479in}{2.936057in}}{\pgfqpoint{1.768078in}{2.940448in}}{\pgfqpoint{1.775891in}{2.948261in}}%
\pgfpathcurveto{\pgfqpoint{1.783705in}{2.956075in}}{\pgfqpoint{1.788095in}{2.966674in}}{\pgfqpoint{1.788095in}{2.977724in}}%
\pgfpathcurveto{\pgfqpoint{1.788095in}{2.988774in}}{\pgfqpoint{1.783705in}{2.999373in}}{\pgfqpoint{1.775891in}{3.007187in}}%
\pgfpathcurveto{\pgfqpoint{1.768078in}{3.015000in}}{\pgfqpoint{1.757479in}{3.019391in}}{\pgfqpoint{1.746429in}{3.019391in}}%
\pgfpathcurveto{\pgfqpoint{1.735378in}{3.019391in}}{\pgfqpoint{1.724779in}{3.015000in}}{\pgfqpoint{1.716966in}{3.007187in}}%
\pgfpathcurveto{\pgfqpoint{1.709152in}{2.999373in}}{\pgfqpoint{1.704762in}{2.988774in}}{\pgfqpoint{1.704762in}{2.977724in}}%
\pgfpathcurveto{\pgfqpoint{1.704762in}{2.966674in}}{\pgfqpoint{1.709152in}{2.956075in}}{\pgfqpoint{1.716966in}{2.948261in}}%
\pgfpathcurveto{\pgfqpoint{1.724779in}{2.940448in}}{\pgfqpoint{1.735378in}{2.936057in}}{\pgfqpoint{1.746429in}{2.936057in}}%
\pgfpathclose%
\pgfusepath{stroke,fill}%
\end{pgfscope}%
\begin{pgfscope}%
\pgfpathrectangle{\pgfqpoint{0.750000in}{0.500000in}}{\pgfqpoint{4.650000in}{3.020000in}}%
\pgfusepath{clip}%
\pgfsetbuttcap%
\pgfsetroundjoin%
\definecolor{currentfill}{rgb}{0.121569,0.466667,0.705882}%
\pgfsetfillcolor{currentfill}%
\pgfsetlinewidth{1.003750pt}%
\definecolor{currentstroke}{rgb}{0.121569,0.466667,0.705882}%
\pgfsetstrokecolor{currentstroke}%
\pgfsetdash{}{0pt}%
\pgfpathmoveto{\pgfqpoint{1.021753in}{0.595606in}}%
\pgfpathcurveto{\pgfqpoint{1.032803in}{0.595606in}}{\pgfqpoint{1.043402in}{0.599996in}}{\pgfqpoint{1.051216in}{0.607810in}}%
\pgfpathcurveto{\pgfqpoint{1.059030in}{0.615624in}}{\pgfqpoint{1.063420in}{0.626223in}}{\pgfqpoint{1.063420in}{0.637273in}}%
\pgfpathcurveto{\pgfqpoint{1.063420in}{0.648323in}}{\pgfqpoint{1.059030in}{0.658922in}}{\pgfqpoint{1.051216in}{0.666736in}}%
\pgfpathcurveto{\pgfqpoint{1.043402in}{0.674549in}}{\pgfqpoint{1.032803in}{0.678939in}}{\pgfqpoint{1.021753in}{0.678939in}}%
\pgfpathcurveto{\pgfqpoint{1.010703in}{0.678939in}}{\pgfqpoint{1.000104in}{0.674549in}}{\pgfqpoint{0.992290in}{0.666736in}}%
\pgfpathcurveto{\pgfqpoint{0.984477in}{0.658922in}}{\pgfqpoint{0.980087in}{0.648323in}}{\pgfqpoint{0.980087in}{0.637273in}}%
\pgfpathcurveto{\pgfqpoint{0.980087in}{0.626223in}}{\pgfqpoint{0.984477in}{0.615624in}}{\pgfqpoint{0.992290in}{0.607810in}}%
\pgfpathcurveto{\pgfqpoint{1.000104in}{0.599996in}}{\pgfqpoint{1.010703in}{0.595606in}}{\pgfqpoint{1.021753in}{0.595606in}}%
\pgfpathclose%
\pgfusepath{stroke,fill}%
\end{pgfscope}%
\begin{pgfscope}%
\pgfpathrectangle{\pgfqpoint{0.750000in}{0.500000in}}{\pgfqpoint{4.650000in}{3.020000in}}%
\pgfusepath{clip}%
\pgfsetbuttcap%
\pgfsetroundjoin%
\definecolor{currentfill}{rgb}{1.000000,0.498039,0.054902}%
\pgfsetfillcolor{currentfill}%
\pgfsetlinewidth{1.003750pt}%
\definecolor{currentstroke}{rgb}{1.000000,0.498039,0.054902}%
\pgfsetstrokecolor{currentstroke}%
\pgfsetdash{}{0pt}%
\pgfpathmoveto{\pgfqpoint{1.444481in}{2.932163in}}%
\pgfpathcurveto{\pgfqpoint{1.455531in}{2.932163in}}{\pgfqpoint{1.466130in}{2.936553in}}{\pgfqpoint{1.473943in}{2.944367in}}%
\pgfpathcurveto{\pgfqpoint{1.481757in}{2.952181in}}{\pgfqpoint{1.486147in}{2.962780in}}{\pgfqpoint{1.486147in}{2.973830in}}%
\pgfpathcurveto{\pgfqpoint{1.486147in}{2.984880in}}{\pgfqpoint{1.481757in}{2.995479in}}{\pgfqpoint{1.473943in}{3.003293in}}%
\pgfpathcurveto{\pgfqpoint{1.466130in}{3.011106in}}{\pgfqpoint{1.455531in}{3.015496in}}{\pgfqpoint{1.444481in}{3.015496in}}%
\pgfpathcurveto{\pgfqpoint{1.433430in}{3.015496in}}{\pgfqpoint{1.422831in}{3.011106in}}{\pgfqpoint{1.415018in}{3.003293in}}%
\pgfpathcurveto{\pgfqpoint{1.407204in}{2.995479in}}{\pgfqpoint{1.402814in}{2.984880in}}{\pgfqpoint{1.402814in}{2.973830in}}%
\pgfpathcurveto{\pgfqpoint{1.402814in}{2.962780in}}{\pgfqpoint{1.407204in}{2.952181in}}{\pgfqpoint{1.415018in}{2.944367in}}%
\pgfpathcurveto{\pgfqpoint{1.422831in}{2.936553in}}{\pgfqpoint{1.433430in}{2.932163in}}{\pgfqpoint{1.444481in}{2.932163in}}%
\pgfpathclose%
\pgfusepath{stroke,fill}%
\end{pgfscope}%
\begin{pgfscope}%
\pgfpathrectangle{\pgfqpoint{0.750000in}{0.500000in}}{\pgfqpoint{4.650000in}{3.020000in}}%
\pgfusepath{clip}%
\pgfsetbuttcap%
\pgfsetroundjoin%
\definecolor{currentfill}{rgb}{0.121569,0.466667,0.705882}%
\pgfsetfillcolor{currentfill}%
\pgfsetlinewidth{1.003750pt}%
\definecolor{currentstroke}{rgb}{0.121569,0.466667,0.705882}%
\pgfsetstrokecolor{currentstroke}%
\pgfsetdash{}{0pt}%
\pgfpathmoveto{\pgfqpoint{1.625649in}{1.129120in}}%
\pgfpathcurveto{\pgfqpoint{1.636699in}{1.129120in}}{\pgfqpoint{1.647299in}{1.133510in}}{\pgfqpoint{1.655112in}{1.141324in}}%
\pgfpathcurveto{\pgfqpoint{1.662926in}{1.149137in}}{\pgfqpoint{1.667316in}{1.159736in}}{\pgfqpoint{1.667316in}{1.170787in}}%
\pgfpathcurveto{\pgfqpoint{1.667316in}{1.181837in}}{\pgfqpoint{1.662926in}{1.192436in}}{\pgfqpoint{1.655112in}{1.200249in}}%
\pgfpathcurveto{\pgfqpoint{1.647299in}{1.208063in}}{\pgfqpoint{1.636699in}{1.212453in}}{\pgfqpoint{1.625649in}{1.212453in}}%
\pgfpathcurveto{\pgfqpoint{1.614599in}{1.212453in}}{\pgfqpoint{1.604000in}{1.208063in}}{\pgfqpoint{1.596187in}{1.200249in}}%
\pgfpathcurveto{\pgfqpoint{1.588373in}{1.192436in}}{\pgfqpoint{1.583983in}{1.181837in}}{\pgfqpoint{1.583983in}{1.170787in}}%
\pgfpathcurveto{\pgfqpoint{1.583983in}{1.159736in}}{\pgfqpoint{1.588373in}{1.149137in}}{\pgfqpoint{1.596187in}{1.141324in}}%
\pgfpathcurveto{\pgfqpoint{1.604000in}{1.133510in}}{\pgfqpoint{1.614599in}{1.129120in}}{\pgfqpoint{1.625649in}{1.129120in}}%
\pgfpathclose%
\pgfusepath{stroke,fill}%
\end{pgfscope}%
\begin{pgfscope}%
\pgfpathrectangle{\pgfqpoint{0.750000in}{0.500000in}}{\pgfqpoint{4.650000in}{3.020000in}}%
\pgfusepath{clip}%
\pgfsetbuttcap%
\pgfsetroundjoin%
\definecolor{currentfill}{rgb}{1.000000,0.498039,0.054902}%
\pgfsetfillcolor{currentfill}%
\pgfsetlinewidth{1.003750pt}%
\definecolor{currentstroke}{rgb}{1.000000,0.498039,0.054902}%
\pgfsetstrokecolor{currentstroke}%
\pgfsetdash{}{0pt}%
\pgfpathmoveto{\pgfqpoint{1.444481in}{2.932163in}}%
\pgfpathcurveto{\pgfqpoint{1.455531in}{2.932163in}}{\pgfqpoint{1.466130in}{2.936553in}}{\pgfqpoint{1.473943in}{2.944367in}}%
\pgfpathcurveto{\pgfqpoint{1.481757in}{2.952181in}}{\pgfqpoint{1.486147in}{2.962780in}}{\pgfqpoint{1.486147in}{2.973830in}}%
\pgfpathcurveto{\pgfqpoint{1.486147in}{2.984880in}}{\pgfqpoint{1.481757in}{2.995479in}}{\pgfqpoint{1.473943in}{3.003293in}}%
\pgfpathcurveto{\pgfqpoint{1.466130in}{3.011106in}}{\pgfqpoint{1.455531in}{3.015496in}}{\pgfqpoint{1.444481in}{3.015496in}}%
\pgfpathcurveto{\pgfqpoint{1.433430in}{3.015496in}}{\pgfqpoint{1.422831in}{3.011106in}}{\pgfqpoint{1.415018in}{3.003293in}}%
\pgfpathcurveto{\pgfqpoint{1.407204in}{2.995479in}}{\pgfqpoint{1.402814in}{2.984880in}}{\pgfqpoint{1.402814in}{2.973830in}}%
\pgfpathcurveto{\pgfqpoint{1.402814in}{2.962780in}}{\pgfqpoint{1.407204in}{2.952181in}}{\pgfqpoint{1.415018in}{2.944367in}}%
\pgfpathcurveto{\pgfqpoint{1.422831in}{2.936553in}}{\pgfqpoint{1.433430in}{2.932163in}}{\pgfqpoint{1.444481in}{2.932163in}}%
\pgfpathclose%
\pgfusepath{stroke,fill}%
\end{pgfscope}%
\begin{pgfscope}%
\pgfpathrectangle{\pgfqpoint{0.750000in}{0.500000in}}{\pgfqpoint{4.650000in}{3.020000in}}%
\pgfusepath{clip}%
\pgfsetbuttcap%
\pgfsetroundjoin%
\definecolor{currentfill}{rgb}{1.000000,0.498039,0.054902}%
\pgfsetfillcolor{currentfill}%
\pgfsetlinewidth{1.003750pt}%
\definecolor{currentstroke}{rgb}{1.000000,0.498039,0.054902}%
\pgfsetstrokecolor{currentstroke}%
\pgfsetdash{}{0pt}%
\pgfpathmoveto{\pgfqpoint{2.229545in}{2.936057in}}%
\pgfpathcurveto{\pgfqpoint{2.240596in}{2.936057in}}{\pgfqpoint{2.251195in}{2.940448in}}{\pgfqpoint{2.259008in}{2.948261in}}%
\pgfpathcurveto{\pgfqpoint{2.266822in}{2.956075in}}{\pgfqpoint{2.271212in}{2.966674in}}{\pgfqpoint{2.271212in}{2.977724in}}%
\pgfpathcurveto{\pgfqpoint{2.271212in}{2.988774in}}{\pgfqpoint{2.266822in}{2.999373in}}{\pgfqpoint{2.259008in}{3.007187in}}%
\pgfpathcurveto{\pgfqpoint{2.251195in}{3.015000in}}{\pgfqpoint{2.240596in}{3.019391in}}{\pgfqpoint{2.229545in}{3.019391in}}%
\pgfpathcurveto{\pgfqpoint{2.218495in}{3.019391in}}{\pgfqpoint{2.207896in}{3.015000in}}{\pgfqpoint{2.200083in}{3.007187in}}%
\pgfpathcurveto{\pgfqpoint{2.192269in}{2.999373in}}{\pgfqpoint{2.187879in}{2.988774in}}{\pgfqpoint{2.187879in}{2.977724in}}%
\pgfpathcurveto{\pgfqpoint{2.187879in}{2.966674in}}{\pgfqpoint{2.192269in}{2.956075in}}{\pgfqpoint{2.200083in}{2.948261in}}%
\pgfpathcurveto{\pgfqpoint{2.207896in}{2.940448in}}{\pgfqpoint{2.218495in}{2.936057in}}{\pgfqpoint{2.229545in}{2.936057in}}%
\pgfpathclose%
\pgfusepath{stroke,fill}%
\end{pgfscope}%
\begin{pgfscope}%
\pgfpathrectangle{\pgfqpoint{0.750000in}{0.500000in}}{\pgfqpoint{4.650000in}{3.020000in}}%
\pgfusepath{clip}%
\pgfsetbuttcap%
\pgfsetroundjoin%
\definecolor{currentfill}{rgb}{1.000000,0.498039,0.054902}%
\pgfsetfillcolor{currentfill}%
\pgfsetlinewidth{1.003750pt}%
\definecolor{currentstroke}{rgb}{1.000000,0.498039,0.054902}%
\pgfsetstrokecolor{currentstroke}%
\pgfsetdash{}{0pt}%
\pgfpathmoveto{\pgfqpoint{2.893831in}{2.924375in}}%
\pgfpathcurveto{\pgfqpoint{2.904881in}{2.924375in}}{\pgfqpoint{2.915480in}{2.928765in}}{\pgfqpoint{2.923294in}{2.936578in}}%
\pgfpathcurveto{\pgfqpoint{2.931108in}{2.944392in}}{\pgfqpoint{2.935498in}{2.954991in}}{\pgfqpoint{2.935498in}{2.966041in}}%
\pgfpathcurveto{\pgfqpoint{2.935498in}{2.977091in}}{\pgfqpoint{2.931108in}{2.987690in}}{\pgfqpoint{2.923294in}{2.995504in}}%
\pgfpathcurveto{\pgfqpoint{2.915480in}{3.003318in}}{\pgfqpoint{2.904881in}{3.007708in}}{\pgfqpoint{2.893831in}{3.007708in}}%
\pgfpathcurveto{\pgfqpoint{2.882781in}{3.007708in}}{\pgfqpoint{2.872182in}{3.003318in}}{\pgfqpoint{2.864368in}{2.995504in}}%
\pgfpathcurveto{\pgfqpoint{2.856555in}{2.987690in}}{\pgfqpoint{2.852165in}{2.977091in}}{\pgfqpoint{2.852165in}{2.966041in}}%
\pgfpathcurveto{\pgfqpoint{2.852165in}{2.954991in}}{\pgfqpoint{2.856555in}{2.944392in}}{\pgfqpoint{2.864368in}{2.936578in}}%
\pgfpathcurveto{\pgfqpoint{2.872182in}{2.928765in}}{\pgfqpoint{2.882781in}{2.924375in}}{\pgfqpoint{2.893831in}{2.924375in}}%
\pgfpathclose%
\pgfusepath{stroke,fill}%
\end{pgfscope}%
\begin{pgfscope}%
\pgfpathrectangle{\pgfqpoint{0.750000in}{0.500000in}}{\pgfqpoint{4.650000in}{3.020000in}}%
\pgfusepath{clip}%
\pgfsetbuttcap%
\pgfsetroundjoin%
\definecolor{currentfill}{rgb}{1.000000,0.498039,0.054902}%
\pgfsetfillcolor{currentfill}%
\pgfsetlinewidth{1.003750pt}%
\definecolor{currentstroke}{rgb}{1.000000,0.498039,0.054902}%
\pgfsetstrokecolor{currentstroke}%
\pgfsetdash{}{0pt}%
\pgfpathmoveto{\pgfqpoint{3.618506in}{2.928269in}}%
\pgfpathcurveto{\pgfqpoint{3.629557in}{2.928269in}}{\pgfqpoint{3.640156in}{2.932659in}}{\pgfqpoint{3.647969in}{2.940473in}}%
\pgfpathcurveto{\pgfqpoint{3.655783in}{2.948286in}}{\pgfqpoint{3.660173in}{2.958885in}}{\pgfqpoint{3.660173in}{2.969936in}}%
\pgfpathcurveto{\pgfqpoint{3.660173in}{2.980986in}}{\pgfqpoint{3.655783in}{2.991585in}}{\pgfqpoint{3.647969in}{2.999398in}}%
\pgfpathcurveto{\pgfqpoint{3.640156in}{3.007212in}}{\pgfqpoint{3.629557in}{3.011602in}}{\pgfqpoint{3.618506in}{3.011602in}}%
\pgfpathcurveto{\pgfqpoint{3.607456in}{3.011602in}}{\pgfqpoint{3.596857in}{3.007212in}}{\pgfqpoint{3.589044in}{2.999398in}}%
\pgfpathcurveto{\pgfqpoint{3.581230in}{2.991585in}}{\pgfqpoint{3.576840in}{2.980986in}}{\pgfqpoint{3.576840in}{2.969936in}}%
\pgfpathcurveto{\pgfqpoint{3.576840in}{2.958885in}}{\pgfqpoint{3.581230in}{2.948286in}}{\pgfqpoint{3.589044in}{2.940473in}}%
\pgfpathcurveto{\pgfqpoint{3.596857in}{2.932659in}}{\pgfqpoint{3.607456in}{2.928269in}}{\pgfqpoint{3.618506in}{2.928269in}}%
\pgfpathclose%
\pgfusepath{stroke,fill}%
\end{pgfscope}%
\begin{pgfscope}%
\pgfpathrectangle{\pgfqpoint{0.750000in}{0.500000in}}{\pgfqpoint{4.650000in}{3.020000in}}%
\pgfusepath{clip}%
\pgfsetbuttcap%
\pgfsetroundjoin%
\definecolor{currentfill}{rgb}{1.000000,0.498039,0.054902}%
\pgfsetfillcolor{currentfill}%
\pgfsetlinewidth{1.003750pt}%
\definecolor{currentstroke}{rgb}{1.000000,0.498039,0.054902}%
\pgfsetstrokecolor{currentstroke}%
\pgfsetdash{}{0pt}%
\pgfpathmoveto{\pgfqpoint{1.987987in}{2.939952in}}%
\pgfpathcurveto{\pgfqpoint{1.999037in}{2.939952in}}{\pgfqpoint{2.009636in}{2.944342in}}{\pgfqpoint{2.017450in}{2.952156in}}%
\pgfpathcurveto{\pgfqpoint{2.025263in}{2.959969in}}{\pgfqpoint{2.029654in}{2.970568in}}{\pgfqpoint{2.029654in}{2.981618in}}%
\pgfpathcurveto{\pgfqpoint{2.029654in}{2.992668in}}{\pgfqpoint{2.025263in}{3.003267in}}{\pgfqpoint{2.017450in}{3.011081in}}%
\pgfpathcurveto{\pgfqpoint{2.009636in}{3.018895in}}{\pgfqpoint{1.999037in}{3.023285in}}{\pgfqpoint{1.987987in}{3.023285in}}%
\pgfpathcurveto{\pgfqpoint{1.976937in}{3.023285in}}{\pgfqpoint{1.966338in}{3.018895in}}{\pgfqpoint{1.958524in}{3.011081in}}%
\pgfpathcurveto{\pgfqpoint{1.950711in}{3.003267in}}{\pgfqpoint{1.946320in}{2.992668in}}{\pgfqpoint{1.946320in}{2.981618in}}%
\pgfpathcurveto{\pgfqpoint{1.946320in}{2.970568in}}{\pgfqpoint{1.950711in}{2.959969in}}{\pgfqpoint{1.958524in}{2.952156in}}%
\pgfpathcurveto{\pgfqpoint{1.966338in}{2.944342in}}{\pgfqpoint{1.976937in}{2.939952in}}{\pgfqpoint{1.987987in}{2.939952in}}%
\pgfpathclose%
\pgfusepath{stroke,fill}%
\end{pgfscope}%
\begin{pgfscope}%
\pgfpathrectangle{\pgfqpoint{0.750000in}{0.500000in}}{\pgfqpoint{4.650000in}{3.020000in}}%
\pgfusepath{clip}%
\pgfsetbuttcap%
\pgfsetroundjoin%
\definecolor{currentfill}{rgb}{1.000000,0.498039,0.054902}%
\pgfsetfillcolor{currentfill}%
\pgfsetlinewidth{1.003750pt}%
\definecolor{currentstroke}{rgb}{1.000000,0.498039,0.054902}%
\pgfsetstrokecolor{currentstroke}%
\pgfsetdash{}{0pt}%
\pgfpathmoveto{\pgfqpoint{1.686039in}{2.928269in}}%
\pgfpathcurveto{\pgfqpoint{1.697089in}{2.928269in}}{\pgfqpoint{1.707688in}{2.932659in}}{\pgfqpoint{1.715502in}{2.940473in}}%
\pgfpathcurveto{\pgfqpoint{1.723315in}{2.948286in}}{\pgfqpoint{1.727706in}{2.958885in}}{\pgfqpoint{1.727706in}{2.969936in}}%
\pgfpathcurveto{\pgfqpoint{1.727706in}{2.980986in}}{\pgfqpoint{1.723315in}{2.991585in}}{\pgfqpoint{1.715502in}{2.999398in}}%
\pgfpathcurveto{\pgfqpoint{1.707688in}{3.007212in}}{\pgfqpoint{1.697089in}{3.011602in}}{\pgfqpoint{1.686039in}{3.011602in}}%
\pgfpathcurveto{\pgfqpoint{1.674989in}{3.011602in}}{\pgfqpoint{1.664390in}{3.007212in}}{\pgfqpoint{1.656576in}{2.999398in}}%
\pgfpathcurveto{\pgfqpoint{1.648763in}{2.991585in}}{\pgfqpoint{1.644372in}{2.980986in}}{\pgfqpoint{1.644372in}{2.969936in}}%
\pgfpathcurveto{\pgfqpoint{1.644372in}{2.958885in}}{\pgfqpoint{1.648763in}{2.948286in}}{\pgfqpoint{1.656576in}{2.940473in}}%
\pgfpathcurveto{\pgfqpoint{1.664390in}{2.932659in}}{\pgfqpoint{1.674989in}{2.928269in}}{\pgfqpoint{1.686039in}{2.928269in}}%
\pgfpathclose%
\pgfusepath{stroke,fill}%
\end{pgfscope}%
\begin{pgfscope}%
\pgfpathrectangle{\pgfqpoint{0.750000in}{0.500000in}}{\pgfqpoint{4.650000in}{3.020000in}}%
\pgfusepath{clip}%
\pgfsetbuttcap%
\pgfsetroundjoin%
\definecolor{currentfill}{rgb}{1.000000,0.498039,0.054902}%
\pgfsetfillcolor{currentfill}%
\pgfsetlinewidth{1.003750pt}%
\definecolor{currentstroke}{rgb}{1.000000,0.498039,0.054902}%
\pgfsetstrokecolor{currentstroke}%
\pgfsetdash{}{0pt}%
\pgfpathmoveto{\pgfqpoint{2.169156in}{2.959423in}}%
\pgfpathcurveto{\pgfqpoint{2.180206in}{2.959423in}}{\pgfqpoint{2.190805in}{2.963813in}}{\pgfqpoint{2.198619in}{2.971627in}}%
\pgfpathcurveto{\pgfqpoint{2.206432in}{2.979440in}}{\pgfqpoint{2.210823in}{2.990039in}}{\pgfqpoint{2.210823in}{3.001090in}}%
\pgfpathcurveto{\pgfqpoint{2.210823in}{3.012140in}}{\pgfqpoint{2.206432in}{3.022739in}}{\pgfqpoint{2.198619in}{3.030552in}}%
\pgfpathcurveto{\pgfqpoint{2.190805in}{3.038366in}}{\pgfqpoint{2.180206in}{3.042756in}}{\pgfqpoint{2.169156in}{3.042756in}}%
\pgfpathcurveto{\pgfqpoint{2.158106in}{3.042756in}}{\pgfqpoint{2.147507in}{3.038366in}}{\pgfqpoint{2.139693in}{3.030552in}}%
\pgfpathcurveto{\pgfqpoint{2.131879in}{3.022739in}}{\pgfqpoint{2.127489in}{3.012140in}}{\pgfqpoint{2.127489in}{3.001090in}}%
\pgfpathcurveto{\pgfqpoint{2.127489in}{2.990039in}}{\pgfqpoint{2.131879in}{2.979440in}}{\pgfqpoint{2.139693in}{2.971627in}}%
\pgfpathcurveto{\pgfqpoint{2.147507in}{2.963813in}}{\pgfqpoint{2.158106in}{2.959423in}}{\pgfqpoint{2.169156in}{2.959423in}}%
\pgfpathclose%
\pgfusepath{stroke,fill}%
\end{pgfscope}%
\begin{pgfscope}%
\pgfpathrectangle{\pgfqpoint{0.750000in}{0.500000in}}{\pgfqpoint{4.650000in}{3.020000in}}%
\pgfusepath{clip}%
\pgfsetbuttcap%
\pgfsetroundjoin%
\definecolor{currentfill}{rgb}{1.000000,0.498039,0.054902}%
\pgfsetfillcolor{currentfill}%
\pgfsetlinewidth{1.003750pt}%
\definecolor{currentstroke}{rgb}{1.000000,0.498039,0.054902}%
\pgfsetstrokecolor{currentstroke}%
\pgfsetdash{}{0pt}%
\pgfpathmoveto{\pgfqpoint{2.471104in}{2.939952in}}%
\pgfpathcurveto{\pgfqpoint{2.482154in}{2.939952in}}{\pgfqpoint{2.492753in}{2.944342in}}{\pgfqpoint{2.500567in}{2.952156in}}%
\pgfpathcurveto{\pgfqpoint{2.508380in}{2.959969in}}{\pgfqpoint{2.512771in}{2.970568in}}{\pgfqpoint{2.512771in}{2.981618in}}%
\pgfpathcurveto{\pgfqpoint{2.512771in}{2.992668in}}{\pgfqpoint{2.508380in}{3.003267in}}{\pgfqpoint{2.500567in}{3.011081in}}%
\pgfpathcurveto{\pgfqpoint{2.492753in}{3.018895in}}{\pgfqpoint{2.482154in}{3.023285in}}{\pgfqpoint{2.471104in}{3.023285in}}%
\pgfpathcurveto{\pgfqpoint{2.460054in}{3.023285in}}{\pgfqpoint{2.449455in}{3.018895in}}{\pgfqpoint{2.441641in}{3.011081in}}%
\pgfpathcurveto{\pgfqpoint{2.433827in}{3.003267in}}{\pgfqpoint{2.429437in}{2.992668in}}{\pgfqpoint{2.429437in}{2.981618in}}%
\pgfpathcurveto{\pgfqpoint{2.429437in}{2.970568in}}{\pgfqpoint{2.433827in}{2.959969in}}{\pgfqpoint{2.441641in}{2.952156in}}%
\pgfpathcurveto{\pgfqpoint{2.449455in}{2.944342in}}{\pgfqpoint{2.460054in}{2.939952in}}{\pgfqpoint{2.471104in}{2.939952in}}%
\pgfpathclose%
\pgfusepath{stroke,fill}%
\end{pgfscope}%
\begin{pgfscope}%
\pgfpathrectangle{\pgfqpoint{0.750000in}{0.500000in}}{\pgfqpoint{4.650000in}{3.020000in}}%
\pgfusepath{clip}%
\pgfsetbuttcap%
\pgfsetroundjoin%
\definecolor{currentfill}{rgb}{1.000000,0.498039,0.054902}%
\pgfsetfillcolor{currentfill}%
\pgfsetlinewidth{1.003750pt}%
\definecolor{currentstroke}{rgb}{1.000000,0.498039,0.054902}%
\pgfsetstrokecolor{currentstroke}%
\pgfsetdash{}{0pt}%
\pgfpathmoveto{\pgfqpoint{1.323701in}{2.932163in}}%
\pgfpathcurveto{\pgfqpoint{1.334751in}{2.932163in}}{\pgfqpoint{1.345350in}{2.936553in}}{\pgfqpoint{1.353164in}{2.944367in}}%
\pgfpathcurveto{\pgfqpoint{1.360978in}{2.952181in}}{\pgfqpoint{1.365368in}{2.962780in}}{\pgfqpoint{1.365368in}{2.973830in}}%
\pgfpathcurveto{\pgfqpoint{1.365368in}{2.984880in}}{\pgfqpoint{1.360978in}{2.995479in}}{\pgfqpoint{1.353164in}{3.003293in}}%
\pgfpathcurveto{\pgfqpoint{1.345350in}{3.011106in}}{\pgfqpoint{1.334751in}{3.015496in}}{\pgfqpoint{1.323701in}{3.015496in}}%
\pgfpathcurveto{\pgfqpoint{1.312651in}{3.015496in}}{\pgfqpoint{1.302052in}{3.011106in}}{\pgfqpoint{1.294239in}{3.003293in}}%
\pgfpathcurveto{\pgfqpoint{1.286425in}{2.995479in}}{\pgfqpoint{1.282035in}{2.984880in}}{\pgfqpoint{1.282035in}{2.973830in}}%
\pgfpathcurveto{\pgfqpoint{1.282035in}{2.962780in}}{\pgfqpoint{1.286425in}{2.952181in}}{\pgfqpoint{1.294239in}{2.944367in}}%
\pgfpathcurveto{\pgfqpoint{1.302052in}{2.936553in}}{\pgfqpoint{1.312651in}{2.932163in}}{\pgfqpoint{1.323701in}{2.932163in}}%
\pgfpathclose%
\pgfusepath{stroke,fill}%
\end{pgfscope}%
\begin{pgfscope}%
\pgfpathrectangle{\pgfqpoint{0.750000in}{0.500000in}}{\pgfqpoint{4.650000in}{3.020000in}}%
\pgfusepath{clip}%
\pgfsetbuttcap%
\pgfsetroundjoin%
\definecolor{currentfill}{rgb}{1.000000,0.498039,0.054902}%
\pgfsetfillcolor{currentfill}%
\pgfsetlinewidth{1.003750pt}%
\definecolor{currentstroke}{rgb}{1.000000,0.498039,0.054902}%
\pgfsetstrokecolor{currentstroke}%
\pgfsetdash{}{0pt}%
\pgfpathmoveto{\pgfqpoint{1.504870in}{2.936057in}}%
\pgfpathcurveto{\pgfqpoint{1.515920in}{2.936057in}}{\pgfqpoint{1.526519in}{2.940448in}}{\pgfqpoint{1.534333in}{2.948261in}}%
\pgfpathcurveto{\pgfqpoint{1.542147in}{2.956075in}}{\pgfqpoint{1.546537in}{2.966674in}}{\pgfqpoint{1.546537in}{2.977724in}}%
\pgfpathcurveto{\pgfqpoint{1.546537in}{2.988774in}}{\pgfqpoint{1.542147in}{2.999373in}}{\pgfqpoint{1.534333in}{3.007187in}}%
\pgfpathcurveto{\pgfqpoint{1.526519in}{3.015000in}}{\pgfqpoint{1.515920in}{3.019391in}}{\pgfqpoint{1.504870in}{3.019391in}}%
\pgfpathcurveto{\pgfqpoint{1.493820in}{3.019391in}}{\pgfqpoint{1.483221in}{3.015000in}}{\pgfqpoint{1.475407in}{3.007187in}}%
\pgfpathcurveto{\pgfqpoint{1.467594in}{2.999373in}}{\pgfqpoint{1.463203in}{2.988774in}}{\pgfqpoint{1.463203in}{2.977724in}}%
\pgfpathcurveto{\pgfqpoint{1.463203in}{2.966674in}}{\pgfqpoint{1.467594in}{2.956075in}}{\pgfqpoint{1.475407in}{2.948261in}}%
\pgfpathcurveto{\pgfqpoint{1.483221in}{2.940448in}}{\pgfqpoint{1.493820in}{2.936057in}}{\pgfqpoint{1.504870in}{2.936057in}}%
\pgfpathclose%
\pgfusepath{stroke,fill}%
\end{pgfscope}%
\begin{pgfscope}%
\pgfpathrectangle{\pgfqpoint{0.750000in}{0.500000in}}{\pgfqpoint{4.650000in}{3.020000in}}%
\pgfusepath{clip}%
\pgfsetbuttcap%
\pgfsetroundjoin%
\definecolor{currentfill}{rgb}{0.121569,0.466667,0.705882}%
\pgfsetfillcolor{currentfill}%
\pgfsetlinewidth{1.003750pt}%
\definecolor{currentstroke}{rgb}{0.121569,0.466667,0.705882}%
\pgfsetstrokecolor{currentstroke}%
\pgfsetdash{}{0pt}%
\pgfpathmoveto{\pgfqpoint{1.202922in}{0.603395in}}%
\pgfpathcurveto{\pgfqpoint{1.213972in}{0.603395in}}{\pgfqpoint{1.224571in}{0.607785in}}{\pgfqpoint{1.232385in}{0.615598in}}%
\pgfpathcurveto{\pgfqpoint{1.240198in}{0.623412in}}{\pgfqpoint{1.244589in}{0.634011in}}{\pgfqpoint{1.244589in}{0.645061in}}%
\pgfpathcurveto{\pgfqpoint{1.244589in}{0.656111in}}{\pgfqpoint{1.240198in}{0.666710in}}{\pgfqpoint{1.232385in}{0.674524in}}%
\pgfpathcurveto{\pgfqpoint{1.224571in}{0.682338in}}{\pgfqpoint{1.213972in}{0.686728in}}{\pgfqpoint{1.202922in}{0.686728in}}%
\pgfpathcurveto{\pgfqpoint{1.191872in}{0.686728in}}{\pgfqpoint{1.181273in}{0.682338in}}{\pgfqpoint{1.173459in}{0.674524in}}%
\pgfpathcurveto{\pgfqpoint{1.165646in}{0.666710in}}{\pgfqpoint{1.161255in}{0.656111in}}{\pgfqpoint{1.161255in}{0.645061in}}%
\pgfpathcurveto{\pgfqpoint{1.161255in}{0.634011in}}{\pgfqpoint{1.165646in}{0.623412in}}{\pgfqpoint{1.173459in}{0.615598in}}%
\pgfpathcurveto{\pgfqpoint{1.181273in}{0.607785in}}{\pgfqpoint{1.191872in}{0.603395in}}{\pgfqpoint{1.202922in}{0.603395in}}%
\pgfpathclose%
\pgfusepath{stroke,fill}%
\end{pgfscope}%
\begin{pgfscope}%
\pgfpathrectangle{\pgfqpoint{0.750000in}{0.500000in}}{\pgfqpoint{4.650000in}{3.020000in}}%
\pgfusepath{clip}%
\pgfsetbuttcap%
\pgfsetroundjoin%
\definecolor{currentfill}{rgb}{1.000000,0.498039,0.054902}%
\pgfsetfillcolor{currentfill}%
\pgfsetlinewidth{1.003750pt}%
\definecolor{currentstroke}{rgb}{1.000000,0.498039,0.054902}%
\pgfsetstrokecolor{currentstroke}%
\pgfsetdash{}{0pt}%
\pgfpathmoveto{\pgfqpoint{1.625649in}{2.924375in}}%
\pgfpathcurveto{\pgfqpoint{1.636699in}{2.924375in}}{\pgfqpoint{1.647299in}{2.928765in}}{\pgfqpoint{1.655112in}{2.936578in}}%
\pgfpathcurveto{\pgfqpoint{1.662926in}{2.944392in}}{\pgfqpoint{1.667316in}{2.954991in}}{\pgfqpoint{1.667316in}{2.966041in}}%
\pgfpathcurveto{\pgfqpoint{1.667316in}{2.977091in}}{\pgfqpoint{1.662926in}{2.987690in}}{\pgfqpoint{1.655112in}{2.995504in}}%
\pgfpathcurveto{\pgfqpoint{1.647299in}{3.003318in}}{\pgfqpoint{1.636699in}{3.007708in}}{\pgfqpoint{1.625649in}{3.007708in}}%
\pgfpathcurveto{\pgfqpoint{1.614599in}{3.007708in}}{\pgfqpoint{1.604000in}{3.003318in}}{\pgfqpoint{1.596187in}{2.995504in}}%
\pgfpathcurveto{\pgfqpoint{1.588373in}{2.987690in}}{\pgfqpoint{1.583983in}{2.977091in}}{\pgfqpoint{1.583983in}{2.966041in}}%
\pgfpathcurveto{\pgfqpoint{1.583983in}{2.954991in}}{\pgfqpoint{1.588373in}{2.944392in}}{\pgfqpoint{1.596187in}{2.936578in}}%
\pgfpathcurveto{\pgfqpoint{1.604000in}{2.928765in}}{\pgfqpoint{1.614599in}{2.924375in}}{\pgfqpoint{1.625649in}{2.924375in}}%
\pgfpathclose%
\pgfusepath{stroke,fill}%
\end{pgfscope}%
\begin{pgfscope}%
\pgfpathrectangle{\pgfqpoint{0.750000in}{0.500000in}}{\pgfqpoint{4.650000in}{3.020000in}}%
\pgfusepath{clip}%
\pgfsetbuttcap%
\pgfsetroundjoin%
\definecolor{currentfill}{rgb}{0.121569,0.466667,0.705882}%
\pgfsetfillcolor{currentfill}%
\pgfsetlinewidth{1.003750pt}%
\definecolor{currentstroke}{rgb}{0.121569,0.466667,0.705882}%
\pgfsetstrokecolor{currentstroke}%
\pgfsetdash{}{0pt}%
\pgfpathmoveto{\pgfqpoint{1.202922in}{0.595606in}}%
\pgfpathcurveto{\pgfqpoint{1.213972in}{0.595606in}}{\pgfqpoint{1.224571in}{0.599996in}}{\pgfqpoint{1.232385in}{0.607810in}}%
\pgfpathcurveto{\pgfqpoint{1.240198in}{0.615624in}}{\pgfqpoint{1.244589in}{0.626223in}}{\pgfqpoint{1.244589in}{0.637273in}}%
\pgfpathcurveto{\pgfqpoint{1.244589in}{0.648323in}}{\pgfqpoint{1.240198in}{0.658922in}}{\pgfqpoint{1.232385in}{0.666736in}}%
\pgfpathcurveto{\pgfqpoint{1.224571in}{0.674549in}}{\pgfqpoint{1.213972in}{0.678939in}}{\pgfqpoint{1.202922in}{0.678939in}}%
\pgfpathcurveto{\pgfqpoint{1.191872in}{0.678939in}}{\pgfqpoint{1.181273in}{0.674549in}}{\pgfqpoint{1.173459in}{0.666736in}}%
\pgfpathcurveto{\pgfqpoint{1.165646in}{0.658922in}}{\pgfqpoint{1.161255in}{0.648323in}}{\pgfqpoint{1.161255in}{0.637273in}}%
\pgfpathcurveto{\pgfqpoint{1.161255in}{0.626223in}}{\pgfqpoint{1.165646in}{0.615624in}}{\pgfqpoint{1.173459in}{0.607810in}}%
\pgfpathcurveto{\pgfqpoint{1.181273in}{0.599996in}}{\pgfqpoint{1.191872in}{0.595606in}}{\pgfqpoint{1.202922in}{0.595606in}}%
\pgfpathclose%
\pgfusepath{stroke,fill}%
\end{pgfscope}%
\begin{pgfscope}%
\pgfpathrectangle{\pgfqpoint{0.750000in}{0.500000in}}{\pgfqpoint{4.650000in}{3.020000in}}%
\pgfusepath{clip}%
\pgfsetbuttcap%
\pgfsetroundjoin%
\definecolor{currentfill}{rgb}{1.000000,0.498039,0.054902}%
\pgfsetfillcolor{currentfill}%
\pgfsetlinewidth{1.003750pt}%
\definecolor{currentstroke}{rgb}{1.000000,0.498039,0.054902}%
\pgfsetstrokecolor{currentstroke}%
\pgfsetdash{}{0pt}%
\pgfpathmoveto{\pgfqpoint{2.531494in}{2.410332in}}%
\pgfpathcurveto{\pgfqpoint{2.542544in}{2.410332in}}{\pgfqpoint{2.553143in}{2.414722in}}{\pgfqpoint{2.560956in}{2.422536in}}%
\pgfpathcurveto{\pgfqpoint{2.568770in}{2.430350in}}{\pgfqpoint{2.573160in}{2.440949in}}{\pgfqpoint{2.573160in}{2.451999in}}%
\pgfpathcurveto{\pgfqpoint{2.573160in}{2.463049in}}{\pgfqpoint{2.568770in}{2.473648in}}{\pgfqpoint{2.560956in}{2.481461in}}%
\pgfpathcurveto{\pgfqpoint{2.553143in}{2.489275in}}{\pgfqpoint{2.542544in}{2.493665in}}{\pgfqpoint{2.531494in}{2.493665in}}%
\pgfpathcurveto{\pgfqpoint{2.520443in}{2.493665in}}{\pgfqpoint{2.509844in}{2.489275in}}{\pgfqpoint{2.502031in}{2.481461in}}%
\pgfpathcurveto{\pgfqpoint{2.494217in}{2.473648in}}{\pgfqpoint{2.489827in}{2.463049in}}{\pgfqpoint{2.489827in}{2.451999in}}%
\pgfpathcurveto{\pgfqpoint{2.489827in}{2.440949in}}{\pgfqpoint{2.494217in}{2.430350in}}{\pgfqpoint{2.502031in}{2.422536in}}%
\pgfpathcurveto{\pgfqpoint{2.509844in}{2.414722in}}{\pgfqpoint{2.520443in}{2.410332in}}{\pgfqpoint{2.531494in}{2.410332in}}%
\pgfpathclose%
\pgfusepath{stroke,fill}%
\end{pgfscope}%
\begin{pgfscope}%
\pgfpathrectangle{\pgfqpoint{0.750000in}{0.500000in}}{\pgfqpoint{4.650000in}{3.020000in}}%
\pgfusepath{clip}%
\pgfsetbuttcap%
\pgfsetroundjoin%
\definecolor{currentfill}{rgb}{1.000000,0.498039,0.054902}%
\pgfsetfillcolor{currentfill}%
\pgfsetlinewidth{1.003750pt}%
\definecolor{currentstroke}{rgb}{1.000000,0.498039,0.054902}%
\pgfsetstrokecolor{currentstroke}%
\pgfsetdash{}{0pt}%
\pgfpathmoveto{\pgfqpoint{2.652273in}{2.390861in}}%
\pgfpathcurveto{\pgfqpoint{2.663323in}{2.390861in}}{\pgfqpoint{2.673922in}{2.395251in}}{\pgfqpoint{2.681736in}{2.403065in}}%
\pgfpathcurveto{\pgfqpoint{2.689549in}{2.410878in}}{\pgfqpoint{2.693939in}{2.421477in}}{\pgfqpoint{2.693939in}{2.432527in}}%
\pgfpathcurveto{\pgfqpoint{2.693939in}{2.443578in}}{\pgfqpoint{2.689549in}{2.454177in}}{\pgfqpoint{2.681736in}{2.461990in}}%
\pgfpathcurveto{\pgfqpoint{2.673922in}{2.469804in}}{\pgfqpoint{2.663323in}{2.474194in}}{\pgfqpoint{2.652273in}{2.474194in}}%
\pgfpathcurveto{\pgfqpoint{2.641223in}{2.474194in}}{\pgfqpoint{2.630624in}{2.469804in}}{\pgfqpoint{2.622810in}{2.461990in}}%
\pgfpathcurveto{\pgfqpoint{2.614996in}{2.454177in}}{\pgfqpoint{2.610606in}{2.443578in}}{\pgfqpoint{2.610606in}{2.432527in}}%
\pgfpathcurveto{\pgfqpoint{2.610606in}{2.421477in}}{\pgfqpoint{2.614996in}{2.410878in}}{\pgfqpoint{2.622810in}{2.403065in}}%
\pgfpathcurveto{\pgfqpoint{2.630624in}{2.395251in}}{\pgfqpoint{2.641223in}{2.390861in}}{\pgfqpoint{2.652273in}{2.390861in}}%
\pgfpathclose%
\pgfusepath{stroke,fill}%
\end{pgfscope}%
\begin{pgfscope}%
\pgfpathrectangle{\pgfqpoint{0.750000in}{0.500000in}}{\pgfqpoint{4.650000in}{3.020000in}}%
\pgfusepath{clip}%
\pgfsetbuttcap%
\pgfsetroundjoin%
\definecolor{currentfill}{rgb}{1.000000,0.498039,0.054902}%
\pgfsetfillcolor{currentfill}%
\pgfsetlinewidth{1.003750pt}%
\definecolor{currentstroke}{rgb}{1.000000,0.498039,0.054902}%
\pgfsetstrokecolor{currentstroke}%
\pgfsetdash{}{0pt}%
\pgfpathmoveto{\pgfqpoint{2.048377in}{2.932163in}}%
\pgfpathcurveto{\pgfqpoint{2.059427in}{2.932163in}}{\pgfqpoint{2.070026in}{2.936553in}}{\pgfqpoint{2.077839in}{2.944367in}}%
\pgfpathcurveto{\pgfqpoint{2.085653in}{2.952181in}}{\pgfqpoint{2.090043in}{2.962780in}}{\pgfqpoint{2.090043in}{2.973830in}}%
\pgfpathcurveto{\pgfqpoint{2.090043in}{2.984880in}}{\pgfqpoint{2.085653in}{2.995479in}}{\pgfqpoint{2.077839in}{3.003293in}}%
\pgfpathcurveto{\pgfqpoint{2.070026in}{3.011106in}}{\pgfqpoint{2.059427in}{3.015496in}}{\pgfqpoint{2.048377in}{3.015496in}}%
\pgfpathcurveto{\pgfqpoint{2.037326in}{3.015496in}}{\pgfqpoint{2.026727in}{3.011106in}}{\pgfqpoint{2.018914in}{3.003293in}}%
\pgfpathcurveto{\pgfqpoint{2.011100in}{2.995479in}}{\pgfqpoint{2.006710in}{2.984880in}}{\pgfqpoint{2.006710in}{2.973830in}}%
\pgfpathcurveto{\pgfqpoint{2.006710in}{2.962780in}}{\pgfqpoint{2.011100in}{2.952181in}}{\pgfqpoint{2.018914in}{2.944367in}}%
\pgfpathcurveto{\pgfqpoint{2.026727in}{2.936553in}}{\pgfqpoint{2.037326in}{2.932163in}}{\pgfqpoint{2.048377in}{2.932163in}}%
\pgfpathclose%
\pgfusepath{stroke,fill}%
\end{pgfscope}%
\begin{pgfscope}%
\pgfpathrectangle{\pgfqpoint{0.750000in}{0.500000in}}{\pgfqpoint{4.650000in}{3.020000in}}%
\pgfusepath{clip}%
\pgfsetbuttcap%
\pgfsetroundjoin%
\definecolor{currentfill}{rgb}{1.000000,0.498039,0.054902}%
\pgfsetfillcolor{currentfill}%
\pgfsetlinewidth{1.003750pt}%
\definecolor{currentstroke}{rgb}{1.000000,0.498039,0.054902}%
\pgfsetstrokecolor{currentstroke}%
\pgfsetdash{}{0pt}%
\pgfpathmoveto{\pgfqpoint{2.169156in}{2.951634in}}%
\pgfpathcurveto{\pgfqpoint{2.180206in}{2.951634in}}{\pgfqpoint{2.190805in}{2.956025in}}{\pgfqpoint{2.198619in}{2.963838in}}%
\pgfpathcurveto{\pgfqpoint{2.206432in}{2.971652in}}{\pgfqpoint{2.210823in}{2.982251in}}{\pgfqpoint{2.210823in}{2.993301in}}%
\pgfpathcurveto{\pgfqpoint{2.210823in}{3.004351in}}{\pgfqpoint{2.206432in}{3.014950in}}{\pgfqpoint{2.198619in}{3.022764in}}%
\pgfpathcurveto{\pgfqpoint{2.190805in}{3.030577in}}{\pgfqpoint{2.180206in}{3.034968in}}{\pgfqpoint{2.169156in}{3.034968in}}%
\pgfpathcurveto{\pgfqpoint{2.158106in}{3.034968in}}{\pgfqpoint{2.147507in}{3.030577in}}{\pgfqpoint{2.139693in}{3.022764in}}%
\pgfpathcurveto{\pgfqpoint{2.131879in}{3.014950in}}{\pgfqpoint{2.127489in}{3.004351in}}{\pgfqpoint{2.127489in}{2.993301in}}%
\pgfpathcurveto{\pgfqpoint{2.127489in}{2.982251in}}{\pgfqpoint{2.131879in}{2.971652in}}{\pgfqpoint{2.139693in}{2.963838in}}%
\pgfpathcurveto{\pgfqpoint{2.147507in}{2.956025in}}{\pgfqpoint{2.158106in}{2.951634in}}{\pgfqpoint{2.169156in}{2.951634in}}%
\pgfpathclose%
\pgfusepath{stroke,fill}%
\end{pgfscope}%
\begin{pgfscope}%
\pgfpathrectangle{\pgfqpoint{0.750000in}{0.500000in}}{\pgfqpoint{4.650000in}{3.020000in}}%
\pgfusepath{clip}%
\pgfsetbuttcap%
\pgfsetroundjoin%
\definecolor{currentfill}{rgb}{1.000000,0.498039,0.054902}%
\pgfsetfillcolor{currentfill}%
\pgfsetlinewidth{1.003750pt}%
\definecolor{currentstroke}{rgb}{1.000000,0.498039,0.054902}%
\pgfsetstrokecolor{currentstroke}%
\pgfsetdash{}{0pt}%
\pgfpathmoveto{\pgfqpoint{1.927597in}{2.102685in}}%
\pgfpathcurveto{\pgfqpoint{1.938648in}{2.102685in}}{\pgfqpoint{1.949247in}{2.107076in}}{\pgfqpoint{1.957060in}{2.114889in}}%
\pgfpathcurveto{\pgfqpoint{1.964874in}{2.122703in}}{\pgfqpoint{1.969264in}{2.133302in}}{\pgfqpoint{1.969264in}{2.144352in}}%
\pgfpathcurveto{\pgfqpoint{1.969264in}{2.155402in}}{\pgfqpoint{1.964874in}{2.166001in}}{\pgfqpoint{1.957060in}{2.173815in}}%
\pgfpathcurveto{\pgfqpoint{1.949247in}{2.181628in}}{\pgfqpoint{1.938648in}{2.186019in}}{\pgfqpoint{1.927597in}{2.186019in}}%
\pgfpathcurveto{\pgfqpoint{1.916547in}{2.186019in}}{\pgfqpoint{1.905948in}{2.181628in}}{\pgfqpoint{1.898135in}{2.173815in}}%
\pgfpathcurveto{\pgfqpoint{1.890321in}{2.166001in}}{\pgfqpoint{1.885931in}{2.155402in}}{\pgfqpoint{1.885931in}{2.144352in}}%
\pgfpathcurveto{\pgfqpoint{1.885931in}{2.133302in}}{\pgfqpoint{1.890321in}{2.122703in}}{\pgfqpoint{1.898135in}{2.114889in}}%
\pgfpathcurveto{\pgfqpoint{1.905948in}{2.107076in}}{\pgfqpoint{1.916547in}{2.102685in}}{\pgfqpoint{1.927597in}{2.102685in}}%
\pgfpathclose%
\pgfusepath{stroke,fill}%
\end{pgfscope}%
\begin{pgfscope}%
\pgfpathrectangle{\pgfqpoint{0.750000in}{0.500000in}}{\pgfqpoint{4.650000in}{3.020000in}}%
\pgfusepath{clip}%
\pgfsetbuttcap%
\pgfsetroundjoin%
\definecolor{currentfill}{rgb}{1.000000,0.498039,0.054902}%
\pgfsetfillcolor{currentfill}%
\pgfsetlinewidth{1.003750pt}%
\definecolor{currentstroke}{rgb}{1.000000,0.498039,0.054902}%
\pgfsetstrokecolor{currentstroke}%
\pgfsetdash{}{0pt}%
\pgfpathmoveto{\pgfqpoint{1.867208in}{2.932163in}}%
\pgfpathcurveto{\pgfqpoint{1.878258in}{2.932163in}}{\pgfqpoint{1.888857in}{2.936553in}}{\pgfqpoint{1.896671in}{2.944367in}}%
\pgfpathcurveto{\pgfqpoint{1.904484in}{2.952181in}}{\pgfqpoint{1.908874in}{2.962780in}}{\pgfqpoint{1.908874in}{2.973830in}}%
\pgfpathcurveto{\pgfqpoint{1.908874in}{2.984880in}}{\pgfqpoint{1.904484in}{2.995479in}}{\pgfqpoint{1.896671in}{3.003293in}}%
\pgfpathcurveto{\pgfqpoint{1.888857in}{3.011106in}}{\pgfqpoint{1.878258in}{3.015496in}}{\pgfqpoint{1.867208in}{3.015496in}}%
\pgfpathcurveto{\pgfqpoint{1.856158in}{3.015496in}}{\pgfqpoint{1.845559in}{3.011106in}}{\pgfqpoint{1.837745in}{3.003293in}}%
\pgfpathcurveto{\pgfqpoint{1.829931in}{2.995479in}}{\pgfqpoint{1.825541in}{2.984880in}}{\pgfqpoint{1.825541in}{2.973830in}}%
\pgfpathcurveto{\pgfqpoint{1.825541in}{2.962780in}}{\pgfqpoint{1.829931in}{2.952181in}}{\pgfqpoint{1.837745in}{2.944367in}}%
\pgfpathcurveto{\pgfqpoint{1.845559in}{2.936553in}}{\pgfqpoint{1.856158in}{2.932163in}}{\pgfqpoint{1.867208in}{2.932163in}}%
\pgfpathclose%
\pgfusepath{stroke,fill}%
\end{pgfscope}%
\begin{pgfscope}%
\pgfpathrectangle{\pgfqpoint{0.750000in}{0.500000in}}{\pgfqpoint{4.650000in}{3.020000in}}%
\pgfusepath{clip}%
\pgfsetbuttcap%
\pgfsetroundjoin%
\definecolor{currentfill}{rgb}{1.000000,0.498039,0.054902}%
\pgfsetfillcolor{currentfill}%
\pgfsetlinewidth{1.003750pt}%
\definecolor{currentstroke}{rgb}{1.000000,0.498039,0.054902}%
\pgfsetstrokecolor{currentstroke}%
\pgfsetdash{}{0pt}%
\pgfpathmoveto{\pgfqpoint{4.403571in}{2.932163in}}%
\pgfpathcurveto{\pgfqpoint{4.414622in}{2.932163in}}{\pgfqpoint{4.425221in}{2.936553in}}{\pgfqpoint{4.433034in}{2.944367in}}%
\pgfpathcurveto{\pgfqpoint{4.440848in}{2.952181in}}{\pgfqpoint{4.445238in}{2.962780in}}{\pgfqpoint{4.445238in}{2.973830in}}%
\pgfpathcurveto{\pgfqpoint{4.445238in}{2.984880in}}{\pgfqpoint{4.440848in}{2.995479in}}{\pgfqpoint{4.433034in}{3.003293in}}%
\pgfpathcurveto{\pgfqpoint{4.425221in}{3.011106in}}{\pgfqpoint{4.414622in}{3.015496in}}{\pgfqpoint{4.403571in}{3.015496in}}%
\pgfpathcurveto{\pgfqpoint{4.392521in}{3.015496in}}{\pgfqpoint{4.381922in}{3.011106in}}{\pgfqpoint{4.374109in}{3.003293in}}%
\pgfpathcurveto{\pgfqpoint{4.366295in}{2.995479in}}{\pgfqpoint{4.361905in}{2.984880in}}{\pgfqpoint{4.361905in}{2.973830in}}%
\pgfpathcurveto{\pgfqpoint{4.361905in}{2.962780in}}{\pgfqpoint{4.366295in}{2.952181in}}{\pgfqpoint{4.374109in}{2.944367in}}%
\pgfpathcurveto{\pgfqpoint{4.381922in}{2.936553in}}{\pgfqpoint{4.392521in}{2.932163in}}{\pgfqpoint{4.403571in}{2.932163in}}%
\pgfpathclose%
\pgfusepath{stroke,fill}%
\end{pgfscope}%
\begin{pgfscope}%
\pgfpathrectangle{\pgfqpoint{0.750000in}{0.500000in}}{\pgfqpoint{4.650000in}{3.020000in}}%
\pgfusepath{clip}%
\pgfsetbuttcap%
\pgfsetroundjoin%
\definecolor{currentfill}{rgb}{1.000000,0.498039,0.054902}%
\pgfsetfillcolor{currentfill}%
\pgfsetlinewidth{1.003750pt}%
\definecolor{currentstroke}{rgb}{1.000000,0.498039,0.054902}%
\pgfsetstrokecolor{currentstroke}%
\pgfsetdash{}{0pt}%
\pgfpathmoveto{\pgfqpoint{1.504870in}{2.932163in}}%
\pgfpathcurveto{\pgfqpoint{1.515920in}{2.932163in}}{\pgfqpoint{1.526519in}{2.936553in}}{\pgfqpoint{1.534333in}{2.944367in}}%
\pgfpathcurveto{\pgfqpoint{1.542147in}{2.952181in}}{\pgfqpoint{1.546537in}{2.962780in}}{\pgfqpoint{1.546537in}{2.973830in}}%
\pgfpathcurveto{\pgfqpoint{1.546537in}{2.984880in}}{\pgfqpoint{1.542147in}{2.995479in}}{\pgfqpoint{1.534333in}{3.003293in}}%
\pgfpathcurveto{\pgfqpoint{1.526519in}{3.011106in}}{\pgfqpoint{1.515920in}{3.015496in}}{\pgfqpoint{1.504870in}{3.015496in}}%
\pgfpathcurveto{\pgfqpoint{1.493820in}{3.015496in}}{\pgfqpoint{1.483221in}{3.011106in}}{\pgfqpoint{1.475407in}{3.003293in}}%
\pgfpathcurveto{\pgfqpoint{1.467594in}{2.995479in}}{\pgfqpoint{1.463203in}{2.984880in}}{\pgfqpoint{1.463203in}{2.973830in}}%
\pgfpathcurveto{\pgfqpoint{1.463203in}{2.962780in}}{\pgfqpoint{1.467594in}{2.952181in}}{\pgfqpoint{1.475407in}{2.944367in}}%
\pgfpathcurveto{\pgfqpoint{1.483221in}{2.936553in}}{\pgfqpoint{1.493820in}{2.932163in}}{\pgfqpoint{1.504870in}{2.932163in}}%
\pgfpathclose%
\pgfusepath{stroke,fill}%
\end{pgfscope}%
\begin{pgfscope}%
\pgfpathrectangle{\pgfqpoint{0.750000in}{0.500000in}}{\pgfqpoint{4.650000in}{3.020000in}}%
\pgfusepath{clip}%
\pgfsetbuttcap%
\pgfsetroundjoin%
\definecolor{currentfill}{rgb}{0.839216,0.152941,0.156863}%
\pgfsetfillcolor{currentfill}%
\pgfsetlinewidth{1.003750pt}%
\definecolor{currentstroke}{rgb}{0.839216,0.152941,0.156863}%
\pgfsetstrokecolor{currentstroke}%
\pgfsetdash{}{0pt}%
\pgfpathmoveto{\pgfqpoint{1.384091in}{2.932163in}}%
\pgfpathcurveto{\pgfqpoint{1.395141in}{2.932163in}}{\pgfqpoint{1.405740in}{2.936553in}}{\pgfqpoint{1.413554in}{2.944367in}}%
\pgfpathcurveto{\pgfqpoint{1.421367in}{2.952181in}}{\pgfqpoint{1.425758in}{2.962780in}}{\pgfqpoint{1.425758in}{2.973830in}}%
\pgfpathcurveto{\pgfqpoint{1.425758in}{2.984880in}}{\pgfqpoint{1.421367in}{2.995479in}}{\pgfqpoint{1.413554in}{3.003293in}}%
\pgfpathcurveto{\pgfqpoint{1.405740in}{3.011106in}}{\pgfqpoint{1.395141in}{3.015496in}}{\pgfqpoint{1.384091in}{3.015496in}}%
\pgfpathcurveto{\pgfqpoint{1.373041in}{3.015496in}}{\pgfqpoint{1.362442in}{3.011106in}}{\pgfqpoint{1.354628in}{3.003293in}}%
\pgfpathcurveto{\pgfqpoint{1.346815in}{2.995479in}}{\pgfqpoint{1.342424in}{2.984880in}}{\pgfqpoint{1.342424in}{2.973830in}}%
\pgfpathcurveto{\pgfqpoint{1.342424in}{2.962780in}}{\pgfqpoint{1.346815in}{2.952181in}}{\pgfqpoint{1.354628in}{2.944367in}}%
\pgfpathcurveto{\pgfqpoint{1.362442in}{2.936553in}}{\pgfqpoint{1.373041in}{2.932163in}}{\pgfqpoint{1.384091in}{2.932163in}}%
\pgfpathclose%
\pgfusepath{stroke,fill}%
\end{pgfscope}%
\begin{pgfscope}%
\pgfpathrectangle{\pgfqpoint{0.750000in}{0.500000in}}{\pgfqpoint{4.650000in}{3.020000in}}%
\pgfusepath{clip}%
\pgfsetbuttcap%
\pgfsetroundjoin%
\definecolor{currentfill}{rgb}{1.000000,0.498039,0.054902}%
\pgfsetfillcolor{currentfill}%
\pgfsetlinewidth{1.003750pt}%
\definecolor{currentstroke}{rgb}{1.000000,0.498039,0.054902}%
\pgfsetstrokecolor{currentstroke}%
\pgfsetdash{}{0pt}%
\pgfpathmoveto{\pgfqpoint{1.323701in}{2.932163in}}%
\pgfpathcurveto{\pgfqpoint{1.334751in}{2.932163in}}{\pgfqpoint{1.345350in}{2.936553in}}{\pgfqpoint{1.353164in}{2.944367in}}%
\pgfpathcurveto{\pgfqpoint{1.360978in}{2.952181in}}{\pgfqpoint{1.365368in}{2.962780in}}{\pgfqpoint{1.365368in}{2.973830in}}%
\pgfpathcurveto{\pgfqpoint{1.365368in}{2.984880in}}{\pgfqpoint{1.360978in}{2.995479in}}{\pgfqpoint{1.353164in}{3.003293in}}%
\pgfpathcurveto{\pgfqpoint{1.345350in}{3.011106in}}{\pgfqpoint{1.334751in}{3.015496in}}{\pgfqpoint{1.323701in}{3.015496in}}%
\pgfpathcurveto{\pgfqpoint{1.312651in}{3.015496in}}{\pgfqpoint{1.302052in}{3.011106in}}{\pgfqpoint{1.294239in}{3.003293in}}%
\pgfpathcurveto{\pgfqpoint{1.286425in}{2.995479in}}{\pgfqpoint{1.282035in}{2.984880in}}{\pgfqpoint{1.282035in}{2.973830in}}%
\pgfpathcurveto{\pgfqpoint{1.282035in}{2.962780in}}{\pgfqpoint{1.286425in}{2.952181in}}{\pgfqpoint{1.294239in}{2.944367in}}%
\pgfpathcurveto{\pgfqpoint{1.302052in}{2.936553in}}{\pgfqpoint{1.312651in}{2.932163in}}{\pgfqpoint{1.323701in}{2.932163in}}%
\pgfpathclose%
\pgfusepath{stroke,fill}%
\end{pgfscope}%
\begin{pgfscope}%
\pgfpathrectangle{\pgfqpoint{0.750000in}{0.500000in}}{\pgfqpoint{4.650000in}{3.020000in}}%
\pgfusepath{clip}%
\pgfsetbuttcap%
\pgfsetroundjoin%
\definecolor{currentfill}{rgb}{0.121569,0.466667,0.705882}%
\pgfsetfillcolor{currentfill}%
\pgfsetlinewidth{1.003750pt}%
\definecolor{currentstroke}{rgb}{0.121569,0.466667,0.705882}%
\pgfsetstrokecolor{currentstroke}%
\pgfsetdash{}{0pt}%
\pgfpathmoveto{\pgfqpoint{1.323701in}{1.452344in}}%
\pgfpathcurveto{\pgfqpoint{1.334751in}{1.452344in}}{\pgfqpoint{1.345350in}{1.456734in}}{\pgfqpoint{1.353164in}{1.464548in}}%
\pgfpathcurveto{\pgfqpoint{1.360978in}{1.472361in}}{\pgfqpoint{1.365368in}{1.482960in}}{\pgfqpoint{1.365368in}{1.494010in}}%
\pgfpathcurveto{\pgfqpoint{1.365368in}{1.505060in}}{\pgfqpoint{1.360978in}{1.515659in}}{\pgfqpoint{1.353164in}{1.523473in}}%
\pgfpathcurveto{\pgfqpoint{1.345350in}{1.531287in}}{\pgfqpoint{1.334751in}{1.535677in}}{\pgfqpoint{1.323701in}{1.535677in}}%
\pgfpathcurveto{\pgfqpoint{1.312651in}{1.535677in}}{\pgfqpoint{1.302052in}{1.531287in}}{\pgfqpoint{1.294239in}{1.523473in}}%
\pgfpathcurveto{\pgfqpoint{1.286425in}{1.515659in}}{\pgfqpoint{1.282035in}{1.505060in}}{\pgfqpoint{1.282035in}{1.494010in}}%
\pgfpathcurveto{\pgfqpoint{1.282035in}{1.482960in}}{\pgfqpoint{1.286425in}{1.472361in}}{\pgfqpoint{1.294239in}{1.464548in}}%
\pgfpathcurveto{\pgfqpoint{1.302052in}{1.456734in}}{\pgfqpoint{1.312651in}{1.452344in}}{\pgfqpoint{1.323701in}{1.452344in}}%
\pgfpathclose%
\pgfusepath{stroke,fill}%
\end{pgfscope}%
\begin{pgfscope}%
\pgfpathrectangle{\pgfqpoint{0.750000in}{0.500000in}}{\pgfqpoint{4.650000in}{3.020000in}}%
\pgfusepath{clip}%
\pgfsetbuttcap%
\pgfsetroundjoin%
\definecolor{currentfill}{rgb}{1.000000,0.498039,0.054902}%
\pgfsetfillcolor{currentfill}%
\pgfsetlinewidth{1.003750pt}%
\definecolor{currentstroke}{rgb}{1.000000,0.498039,0.054902}%
\pgfsetstrokecolor{currentstroke}%
\pgfsetdash{}{0pt}%
\pgfpathmoveto{\pgfqpoint{1.444481in}{2.936057in}}%
\pgfpathcurveto{\pgfqpoint{1.455531in}{2.936057in}}{\pgfqpoint{1.466130in}{2.940448in}}{\pgfqpoint{1.473943in}{2.948261in}}%
\pgfpathcurveto{\pgfqpoint{1.481757in}{2.956075in}}{\pgfqpoint{1.486147in}{2.966674in}}{\pgfqpoint{1.486147in}{2.977724in}}%
\pgfpathcurveto{\pgfqpoint{1.486147in}{2.988774in}}{\pgfqpoint{1.481757in}{2.999373in}}{\pgfqpoint{1.473943in}{3.007187in}}%
\pgfpathcurveto{\pgfqpoint{1.466130in}{3.015000in}}{\pgfqpoint{1.455531in}{3.019391in}}{\pgfqpoint{1.444481in}{3.019391in}}%
\pgfpathcurveto{\pgfqpoint{1.433430in}{3.019391in}}{\pgfqpoint{1.422831in}{3.015000in}}{\pgfqpoint{1.415018in}{3.007187in}}%
\pgfpathcurveto{\pgfqpoint{1.407204in}{2.999373in}}{\pgfqpoint{1.402814in}{2.988774in}}{\pgfqpoint{1.402814in}{2.977724in}}%
\pgfpathcurveto{\pgfqpoint{1.402814in}{2.966674in}}{\pgfqpoint{1.407204in}{2.956075in}}{\pgfqpoint{1.415018in}{2.948261in}}%
\pgfpathcurveto{\pgfqpoint{1.422831in}{2.940448in}}{\pgfqpoint{1.433430in}{2.936057in}}{\pgfqpoint{1.444481in}{2.936057in}}%
\pgfpathclose%
\pgfusepath{stroke,fill}%
\end{pgfscope}%
\begin{pgfscope}%
\pgfpathrectangle{\pgfqpoint{0.750000in}{0.500000in}}{\pgfqpoint{4.650000in}{3.020000in}}%
\pgfusepath{clip}%
\pgfsetbuttcap%
\pgfsetroundjoin%
\definecolor{currentfill}{rgb}{1.000000,0.498039,0.054902}%
\pgfsetfillcolor{currentfill}%
\pgfsetlinewidth{1.003750pt}%
\definecolor{currentstroke}{rgb}{1.000000,0.498039,0.054902}%
\pgfsetstrokecolor{currentstroke}%
\pgfsetdash{}{0pt}%
\pgfpathmoveto{\pgfqpoint{1.444481in}{2.928269in}}%
\pgfpathcurveto{\pgfqpoint{1.455531in}{2.928269in}}{\pgfqpoint{1.466130in}{2.932659in}}{\pgfqpoint{1.473943in}{2.940473in}}%
\pgfpathcurveto{\pgfqpoint{1.481757in}{2.948286in}}{\pgfqpoint{1.486147in}{2.958885in}}{\pgfqpoint{1.486147in}{2.969936in}}%
\pgfpathcurveto{\pgfqpoint{1.486147in}{2.980986in}}{\pgfqpoint{1.481757in}{2.991585in}}{\pgfqpoint{1.473943in}{2.999398in}}%
\pgfpathcurveto{\pgfqpoint{1.466130in}{3.007212in}}{\pgfqpoint{1.455531in}{3.011602in}}{\pgfqpoint{1.444481in}{3.011602in}}%
\pgfpathcurveto{\pgfqpoint{1.433430in}{3.011602in}}{\pgfqpoint{1.422831in}{3.007212in}}{\pgfqpoint{1.415018in}{2.999398in}}%
\pgfpathcurveto{\pgfqpoint{1.407204in}{2.991585in}}{\pgfqpoint{1.402814in}{2.980986in}}{\pgfqpoint{1.402814in}{2.969936in}}%
\pgfpathcurveto{\pgfqpoint{1.402814in}{2.958885in}}{\pgfqpoint{1.407204in}{2.948286in}}{\pgfqpoint{1.415018in}{2.940473in}}%
\pgfpathcurveto{\pgfqpoint{1.422831in}{2.932659in}}{\pgfqpoint{1.433430in}{2.928269in}}{\pgfqpoint{1.444481in}{2.928269in}}%
\pgfpathclose%
\pgfusepath{stroke,fill}%
\end{pgfscope}%
\begin{pgfscope}%
\pgfpathrectangle{\pgfqpoint{0.750000in}{0.500000in}}{\pgfqpoint{4.650000in}{3.020000in}}%
\pgfusepath{clip}%
\pgfsetbuttcap%
\pgfsetroundjoin%
\definecolor{currentfill}{rgb}{1.000000,0.498039,0.054902}%
\pgfsetfillcolor{currentfill}%
\pgfsetlinewidth{1.003750pt}%
\definecolor{currentstroke}{rgb}{1.000000,0.498039,0.054902}%
\pgfsetstrokecolor{currentstroke}%
\pgfsetdash{}{0pt}%
\pgfpathmoveto{\pgfqpoint{2.289935in}{2.932163in}}%
\pgfpathcurveto{\pgfqpoint{2.300985in}{2.932163in}}{\pgfqpoint{2.311584in}{2.936553in}}{\pgfqpoint{2.319398in}{2.944367in}}%
\pgfpathcurveto{\pgfqpoint{2.327211in}{2.952181in}}{\pgfqpoint{2.331602in}{2.962780in}}{\pgfqpoint{2.331602in}{2.973830in}}%
\pgfpathcurveto{\pgfqpoint{2.331602in}{2.984880in}}{\pgfqpoint{2.327211in}{2.995479in}}{\pgfqpoint{2.319398in}{3.003293in}}%
\pgfpathcurveto{\pgfqpoint{2.311584in}{3.011106in}}{\pgfqpoint{2.300985in}{3.015496in}}{\pgfqpoint{2.289935in}{3.015496in}}%
\pgfpathcurveto{\pgfqpoint{2.278885in}{3.015496in}}{\pgfqpoint{2.268286in}{3.011106in}}{\pgfqpoint{2.260472in}{3.003293in}}%
\pgfpathcurveto{\pgfqpoint{2.252659in}{2.995479in}}{\pgfqpoint{2.248268in}{2.984880in}}{\pgfqpoint{2.248268in}{2.973830in}}%
\pgfpathcurveto{\pgfqpoint{2.248268in}{2.962780in}}{\pgfqpoint{2.252659in}{2.952181in}}{\pgfqpoint{2.260472in}{2.944367in}}%
\pgfpathcurveto{\pgfqpoint{2.268286in}{2.936553in}}{\pgfqpoint{2.278885in}{2.932163in}}{\pgfqpoint{2.289935in}{2.932163in}}%
\pgfpathclose%
\pgfusepath{stroke,fill}%
\end{pgfscope}%
\begin{pgfscope}%
\pgfpathrectangle{\pgfqpoint{0.750000in}{0.500000in}}{\pgfqpoint{4.650000in}{3.020000in}}%
\pgfusepath{clip}%
\pgfsetbuttcap%
\pgfsetroundjoin%
\definecolor{currentfill}{rgb}{1.000000,0.498039,0.054902}%
\pgfsetfillcolor{currentfill}%
\pgfsetlinewidth{1.003750pt}%
\definecolor{currentstroke}{rgb}{1.000000,0.498039,0.054902}%
\pgfsetstrokecolor{currentstroke}%
\pgfsetdash{}{0pt}%
\pgfpathmoveto{\pgfqpoint{2.229545in}{2.939952in}}%
\pgfpathcurveto{\pgfqpoint{2.240596in}{2.939952in}}{\pgfqpoint{2.251195in}{2.944342in}}{\pgfqpoint{2.259008in}{2.952156in}}%
\pgfpathcurveto{\pgfqpoint{2.266822in}{2.959969in}}{\pgfqpoint{2.271212in}{2.970568in}}{\pgfqpoint{2.271212in}{2.981618in}}%
\pgfpathcurveto{\pgfqpoint{2.271212in}{2.992668in}}{\pgfqpoint{2.266822in}{3.003267in}}{\pgfqpoint{2.259008in}{3.011081in}}%
\pgfpathcurveto{\pgfqpoint{2.251195in}{3.018895in}}{\pgfqpoint{2.240596in}{3.023285in}}{\pgfqpoint{2.229545in}{3.023285in}}%
\pgfpathcurveto{\pgfqpoint{2.218495in}{3.023285in}}{\pgfqpoint{2.207896in}{3.018895in}}{\pgfqpoint{2.200083in}{3.011081in}}%
\pgfpathcurveto{\pgfqpoint{2.192269in}{3.003267in}}{\pgfqpoint{2.187879in}{2.992668in}}{\pgfqpoint{2.187879in}{2.981618in}}%
\pgfpathcurveto{\pgfqpoint{2.187879in}{2.970568in}}{\pgfqpoint{2.192269in}{2.959969in}}{\pgfqpoint{2.200083in}{2.952156in}}%
\pgfpathcurveto{\pgfqpoint{2.207896in}{2.944342in}}{\pgfqpoint{2.218495in}{2.939952in}}{\pgfqpoint{2.229545in}{2.939952in}}%
\pgfpathclose%
\pgfusepath{stroke,fill}%
\end{pgfscope}%
\begin{pgfscope}%
\pgfpathrectangle{\pgfqpoint{0.750000in}{0.500000in}}{\pgfqpoint{4.650000in}{3.020000in}}%
\pgfusepath{clip}%
\pgfsetbuttcap%
\pgfsetroundjoin%
\definecolor{currentfill}{rgb}{1.000000,0.498039,0.054902}%
\pgfsetfillcolor{currentfill}%
\pgfsetlinewidth{1.003750pt}%
\definecolor{currentstroke}{rgb}{1.000000,0.498039,0.054902}%
\pgfsetstrokecolor{currentstroke}%
\pgfsetdash{}{0pt}%
\pgfpathmoveto{\pgfqpoint{2.108766in}{2.928269in}}%
\pgfpathcurveto{\pgfqpoint{2.119816in}{2.928269in}}{\pgfqpoint{2.130415in}{2.932659in}}{\pgfqpoint{2.138229in}{2.940473in}}%
\pgfpathcurveto{\pgfqpoint{2.146043in}{2.948286in}}{\pgfqpoint{2.150433in}{2.958885in}}{\pgfqpoint{2.150433in}{2.969936in}}%
\pgfpathcurveto{\pgfqpoint{2.150433in}{2.980986in}}{\pgfqpoint{2.146043in}{2.991585in}}{\pgfqpoint{2.138229in}{2.999398in}}%
\pgfpathcurveto{\pgfqpoint{2.130415in}{3.007212in}}{\pgfqpoint{2.119816in}{3.011602in}}{\pgfqpoint{2.108766in}{3.011602in}}%
\pgfpathcurveto{\pgfqpoint{2.097716in}{3.011602in}}{\pgfqpoint{2.087117in}{3.007212in}}{\pgfqpoint{2.079303in}{2.999398in}}%
\pgfpathcurveto{\pgfqpoint{2.071490in}{2.991585in}}{\pgfqpoint{2.067100in}{2.980986in}}{\pgfqpoint{2.067100in}{2.969936in}}%
\pgfpathcurveto{\pgfqpoint{2.067100in}{2.958885in}}{\pgfqpoint{2.071490in}{2.948286in}}{\pgfqpoint{2.079303in}{2.940473in}}%
\pgfpathcurveto{\pgfqpoint{2.087117in}{2.932659in}}{\pgfqpoint{2.097716in}{2.928269in}}{\pgfqpoint{2.108766in}{2.928269in}}%
\pgfpathclose%
\pgfusepath{stroke,fill}%
\end{pgfscope}%
\begin{pgfscope}%
\pgfpathrectangle{\pgfqpoint{0.750000in}{0.500000in}}{\pgfqpoint{4.650000in}{3.020000in}}%
\pgfusepath{clip}%
\pgfsetbuttcap%
\pgfsetroundjoin%
\definecolor{currentfill}{rgb}{1.000000,0.498039,0.054902}%
\pgfsetfillcolor{currentfill}%
\pgfsetlinewidth{1.003750pt}%
\definecolor{currentstroke}{rgb}{1.000000,0.498039,0.054902}%
\pgfsetstrokecolor{currentstroke}%
\pgfsetdash{}{0pt}%
\pgfpathmoveto{\pgfqpoint{1.384091in}{2.932163in}}%
\pgfpathcurveto{\pgfqpoint{1.395141in}{2.932163in}}{\pgfqpoint{1.405740in}{2.936553in}}{\pgfqpoint{1.413554in}{2.944367in}}%
\pgfpathcurveto{\pgfqpoint{1.421367in}{2.952181in}}{\pgfqpoint{1.425758in}{2.962780in}}{\pgfqpoint{1.425758in}{2.973830in}}%
\pgfpathcurveto{\pgfqpoint{1.425758in}{2.984880in}}{\pgfqpoint{1.421367in}{2.995479in}}{\pgfqpoint{1.413554in}{3.003293in}}%
\pgfpathcurveto{\pgfqpoint{1.405740in}{3.011106in}}{\pgfqpoint{1.395141in}{3.015496in}}{\pgfqpoint{1.384091in}{3.015496in}}%
\pgfpathcurveto{\pgfqpoint{1.373041in}{3.015496in}}{\pgfqpoint{1.362442in}{3.011106in}}{\pgfqpoint{1.354628in}{3.003293in}}%
\pgfpathcurveto{\pgfqpoint{1.346815in}{2.995479in}}{\pgfqpoint{1.342424in}{2.984880in}}{\pgfqpoint{1.342424in}{2.973830in}}%
\pgfpathcurveto{\pgfqpoint{1.342424in}{2.962780in}}{\pgfqpoint{1.346815in}{2.952181in}}{\pgfqpoint{1.354628in}{2.944367in}}%
\pgfpathcurveto{\pgfqpoint{1.362442in}{2.936553in}}{\pgfqpoint{1.373041in}{2.932163in}}{\pgfqpoint{1.384091in}{2.932163in}}%
\pgfpathclose%
\pgfusepath{stroke,fill}%
\end{pgfscope}%
\begin{pgfscope}%
\pgfpathrectangle{\pgfqpoint{0.750000in}{0.500000in}}{\pgfqpoint{4.650000in}{3.020000in}}%
\pgfusepath{clip}%
\pgfsetbuttcap%
\pgfsetroundjoin%
\definecolor{currentfill}{rgb}{1.000000,0.498039,0.054902}%
\pgfsetfillcolor{currentfill}%
\pgfsetlinewidth{1.003750pt}%
\definecolor{currentstroke}{rgb}{1.000000,0.498039,0.054902}%
\pgfsetstrokecolor{currentstroke}%
\pgfsetdash{}{0pt}%
\pgfpathmoveto{\pgfqpoint{1.625649in}{2.608939in}}%
\pgfpathcurveto{\pgfqpoint{1.636699in}{2.608939in}}{\pgfqpoint{1.647299in}{2.613330in}}{\pgfqpoint{1.655112in}{2.621143in}}%
\pgfpathcurveto{\pgfqpoint{1.662926in}{2.628957in}}{\pgfqpoint{1.667316in}{2.639556in}}{\pgfqpoint{1.667316in}{2.650606in}}%
\pgfpathcurveto{\pgfqpoint{1.667316in}{2.661656in}}{\pgfqpoint{1.662926in}{2.672255in}}{\pgfqpoint{1.655112in}{2.680069in}}%
\pgfpathcurveto{\pgfqpoint{1.647299in}{2.687882in}}{\pgfqpoint{1.636699in}{2.692273in}}{\pgfqpoint{1.625649in}{2.692273in}}%
\pgfpathcurveto{\pgfqpoint{1.614599in}{2.692273in}}{\pgfqpoint{1.604000in}{2.687882in}}{\pgfqpoint{1.596187in}{2.680069in}}%
\pgfpathcurveto{\pgfqpoint{1.588373in}{2.672255in}}{\pgfqpoint{1.583983in}{2.661656in}}{\pgfqpoint{1.583983in}{2.650606in}}%
\pgfpathcurveto{\pgfqpoint{1.583983in}{2.639556in}}{\pgfqpoint{1.588373in}{2.628957in}}{\pgfqpoint{1.596187in}{2.621143in}}%
\pgfpathcurveto{\pgfqpoint{1.604000in}{2.613330in}}{\pgfqpoint{1.614599in}{2.608939in}}{\pgfqpoint{1.625649in}{2.608939in}}%
\pgfpathclose%
\pgfusepath{stroke,fill}%
\end{pgfscope}%
\begin{pgfscope}%
\pgfpathrectangle{\pgfqpoint{0.750000in}{0.500000in}}{\pgfqpoint{4.650000in}{3.020000in}}%
\pgfusepath{clip}%
\pgfsetbuttcap%
\pgfsetroundjoin%
\definecolor{currentfill}{rgb}{1.000000,0.498039,0.054902}%
\pgfsetfillcolor{currentfill}%
\pgfsetlinewidth{1.003750pt}%
\definecolor{currentstroke}{rgb}{1.000000,0.498039,0.054902}%
\pgfsetstrokecolor{currentstroke}%
\pgfsetdash{}{0pt}%
\pgfpathmoveto{\pgfqpoint{3.376948in}{2.912692in}}%
\pgfpathcurveto{\pgfqpoint{3.387998in}{2.912692in}}{\pgfqpoint{3.398597in}{2.917082in}}{\pgfqpoint{3.406411in}{2.924896in}}%
\pgfpathcurveto{\pgfqpoint{3.414224in}{2.932709in}}{\pgfqpoint{3.418615in}{2.943308in}}{\pgfqpoint{3.418615in}{2.954358in}}%
\pgfpathcurveto{\pgfqpoint{3.418615in}{2.965409in}}{\pgfqpoint{3.414224in}{2.976008in}}{\pgfqpoint{3.406411in}{2.983821in}}%
\pgfpathcurveto{\pgfqpoint{3.398597in}{2.991635in}}{\pgfqpoint{3.387998in}{2.996025in}}{\pgfqpoint{3.376948in}{2.996025in}}%
\pgfpathcurveto{\pgfqpoint{3.365898in}{2.996025in}}{\pgfqpoint{3.355299in}{2.991635in}}{\pgfqpoint{3.347485in}{2.983821in}}%
\pgfpathcurveto{\pgfqpoint{3.339672in}{2.976008in}}{\pgfqpoint{3.335281in}{2.965409in}}{\pgfqpoint{3.335281in}{2.954358in}}%
\pgfpathcurveto{\pgfqpoint{3.335281in}{2.943308in}}{\pgfqpoint{3.339672in}{2.932709in}}{\pgfqpoint{3.347485in}{2.924896in}}%
\pgfpathcurveto{\pgfqpoint{3.355299in}{2.917082in}}{\pgfqpoint{3.365898in}{2.912692in}}{\pgfqpoint{3.376948in}{2.912692in}}%
\pgfpathclose%
\pgfusepath{stroke,fill}%
\end{pgfscope}%
\begin{pgfscope}%
\pgfpathrectangle{\pgfqpoint{0.750000in}{0.500000in}}{\pgfqpoint{4.650000in}{3.020000in}}%
\pgfusepath{clip}%
\pgfsetbuttcap%
\pgfsetroundjoin%
\definecolor{currentfill}{rgb}{1.000000,0.498039,0.054902}%
\pgfsetfillcolor{currentfill}%
\pgfsetlinewidth{1.003750pt}%
\definecolor{currentstroke}{rgb}{1.000000,0.498039,0.054902}%
\pgfsetstrokecolor{currentstroke}%
\pgfsetdash{}{0pt}%
\pgfpathmoveto{\pgfqpoint{1.504870in}{2.815335in}}%
\pgfpathcurveto{\pgfqpoint{1.515920in}{2.815335in}}{\pgfqpoint{1.526519in}{2.819726in}}{\pgfqpoint{1.534333in}{2.827539in}}%
\pgfpathcurveto{\pgfqpoint{1.542147in}{2.835353in}}{\pgfqpoint{1.546537in}{2.845952in}}{\pgfqpoint{1.546537in}{2.857002in}}%
\pgfpathcurveto{\pgfqpoint{1.546537in}{2.868052in}}{\pgfqpoint{1.542147in}{2.878651in}}{\pgfqpoint{1.534333in}{2.886465in}}%
\pgfpathcurveto{\pgfqpoint{1.526519in}{2.894278in}}{\pgfqpoint{1.515920in}{2.898669in}}{\pgfqpoint{1.504870in}{2.898669in}}%
\pgfpathcurveto{\pgfqpoint{1.493820in}{2.898669in}}{\pgfqpoint{1.483221in}{2.894278in}}{\pgfqpoint{1.475407in}{2.886465in}}%
\pgfpathcurveto{\pgfqpoint{1.467594in}{2.878651in}}{\pgfqpoint{1.463203in}{2.868052in}}{\pgfqpoint{1.463203in}{2.857002in}}%
\pgfpathcurveto{\pgfqpoint{1.463203in}{2.845952in}}{\pgfqpoint{1.467594in}{2.835353in}}{\pgfqpoint{1.475407in}{2.827539in}}%
\pgfpathcurveto{\pgfqpoint{1.483221in}{2.819726in}}{\pgfqpoint{1.493820in}{2.815335in}}{\pgfqpoint{1.504870in}{2.815335in}}%
\pgfpathclose%
\pgfusepath{stroke,fill}%
\end{pgfscope}%
\begin{pgfscope}%
\pgfpathrectangle{\pgfqpoint{0.750000in}{0.500000in}}{\pgfqpoint{4.650000in}{3.020000in}}%
\pgfusepath{clip}%
\pgfsetbuttcap%
\pgfsetroundjoin%
\definecolor{currentfill}{rgb}{0.121569,0.466667,0.705882}%
\pgfsetfillcolor{currentfill}%
\pgfsetlinewidth{1.003750pt}%
\definecolor{currentstroke}{rgb}{0.121569,0.466667,0.705882}%
\pgfsetstrokecolor{currentstroke}%
\pgfsetdash{}{0pt}%
\pgfpathmoveto{\pgfqpoint{1.202922in}{0.599500in}}%
\pgfpathcurveto{\pgfqpoint{1.213972in}{0.599500in}}{\pgfqpoint{1.224571in}{0.603891in}}{\pgfqpoint{1.232385in}{0.611704in}}%
\pgfpathcurveto{\pgfqpoint{1.240198in}{0.619518in}}{\pgfqpoint{1.244589in}{0.630117in}}{\pgfqpoint{1.244589in}{0.641167in}}%
\pgfpathcurveto{\pgfqpoint{1.244589in}{0.652217in}}{\pgfqpoint{1.240198in}{0.662816in}}{\pgfqpoint{1.232385in}{0.670630in}}%
\pgfpathcurveto{\pgfqpoint{1.224571in}{0.678443in}}{\pgfqpoint{1.213972in}{0.682834in}}{\pgfqpoint{1.202922in}{0.682834in}}%
\pgfpathcurveto{\pgfqpoint{1.191872in}{0.682834in}}{\pgfqpoint{1.181273in}{0.678443in}}{\pgfqpoint{1.173459in}{0.670630in}}%
\pgfpathcurveto{\pgfqpoint{1.165646in}{0.662816in}}{\pgfqpoint{1.161255in}{0.652217in}}{\pgfqpoint{1.161255in}{0.641167in}}%
\pgfpathcurveto{\pgfqpoint{1.161255in}{0.630117in}}{\pgfqpoint{1.165646in}{0.619518in}}{\pgfqpoint{1.173459in}{0.611704in}}%
\pgfpathcurveto{\pgfqpoint{1.181273in}{0.603891in}}{\pgfqpoint{1.191872in}{0.599500in}}{\pgfqpoint{1.202922in}{0.599500in}}%
\pgfpathclose%
\pgfusepath{stroke,fill}%
\end{pgfscope}%
\begin{pgfscope}%
\pgfpathrectangle{\pgfqpoint{0.750000in}{0.500000in}}{\pgfqpoint{4.650000in}{3.020000in}}%
\pgfusepath{clip}%
\pgfsetbuttcap%
\pgfsetroundjoin%
\definecolor{currentfill}{rgb}{1.000000,0.498039,0.054902}%
\pgfsetfillcolor{currentfill}%
\pgfsetlinewidth{1.003750pt}%
\definecolor{currentstroke}{rgb}{1.000000,0.498039,0.054902}%
\pgfsetstrokecolor{currentstroke}%
\pgfsetdash{}{0pt}%
\pgfpathmoveto{\pgfqpoint{1.867208in}{2.936057in}}%
\pgfpathcurveto{\pgfqpoint{1.878258in}{2.936057in}}{\pgfqpoint{1.888857in}{2.940448in}}{\pgfqpoint{1.896671in}{2.948261in}}%
\pgfpathcurveto{\pgfqpoint{1.904484in}{2.956075in}}{\pgfqpoint{1.908874in}{2.966674in}}{\pgfqpoint{1.908874in}{2.977724in}}%
\pgfpathcurveto{\pgfqpoint{1.908874in}{2.988774in}}{\pgfqpoint{1.904484in}{2.999373in}}{\pgfqpoint{1.896671in}{3.007187in}}%
\pgfpathcurveto{\pgfqpoint{1.888857in}{3.015000in}}{\pgfqpoint{1.878258in}{3.019391in}}{\pgfqpoint{1.867208in}{3.019391in}}%
\pgfpathcurveto{\pgfqpoint{1.856158in}{3.019391in}}{\pgfqpoint{1.845559in}{3.015000in}}{\pgfqpoint{1.837745in}{3.007187in}}%
\pgfpathcurveto{\pgfqpoint{1.829931in}{2.999373in}}{\pgfqpoint{1.825541in}{2.988774in}}{\pgfqpoint{1.825541in}{2.977724in}}%
\pgfpathcurveto{\pgfqpoint{1.825541in}{2.966674in}}{\pgfqpoint{1.829931in}{2.956075in}}{\pgfqpoint{1.837745in}{2.948261in}}%
\pgfpathcurveto{\pgfqpoint{1.845559in}{2.940448in}}{\pgfqpoint{1.856158in}{2.936057in}}{\pgfqpoint{1.867208in}{2.936057in}}%
\pgfpathclose%
\pgfusepath{stroke,fill}%
\end{pgfscope}%
\begin{pgfscope}%
\pgfpathrectangle{\pgfqpoint{0.750000in}{0.500000in}}{\pgfqpoint{4.650000in}{3.020000in}}%
\pgfusepath{clip}%
\pgfsetbuttcap%
\pgfsetroundjoin%
\definecolor{currentfill}{rgb}{0.121569,0.466667,0.705882}%
\pgfsetfillcolor{currentfill}%
\pgfsetlinewidth{1.003750pt}%
\definecolor{currentstroke}{rgb}{0.121569,0.466667,0.705882}%
\pgfsetstrokecolor{currentstroke}%
\pgfsetdash{}{0pt}%
\pgfpathmoveto{\pgfqpoint{1.263312in}{0.595606in}}%
\pgfpathcurveto{\pgfqpoint{1.274362in}{0.595606in}}{\pgfqpoint{1.284961in}{0.599996in}}{\pgfqpoint{1.292774in}{0.607810in}}%
\pgfpathcurveto{\pgfqpoint{1.300588in}{0.615624in}}{\pgfqpoint{1.304978in}{0.626223in}}{\pgfqpoint{1.304978in}{0.637273in}}%
\pgfpathcurveto{\pgfqpoint{1.304978in}{0.648323in}}{\pgfqpoint{1.300588in}{0.658922in}}{\pgfqpoint{1.292774in}{0.666736in}}%
\pgfpathcurveto{\pgfqpoint{1.284961in}{0.674549in}}{\pgfqpoint{1.274362in}{0.678939in}}{\pgfqpoint{1.263312in}{0.678939in}}%
\pgfpathcurveto{\pgfqpoint{1.252262in}{0.678939in}}{\pgfqpoint{1.241663in}{0.674549in}}{\pgfqpoint{1.233849in}{0.666736in}}%
\pgfpathcurveto{\pgfqpoint{1.226035in}{0.658922in}}{\pgfqpoint{1.221645in}{0.648323in}}{\pgfqpoint{1.221645in}{0.637273in}}%
\pgfpathcurveto{\pgfqpoint{1.221645in}{0.626223in}}{\pgfqpoint{1.226035in}{0.615624in}}{\pgfqpoint{1.233849in}{0.607810in}}%
\pgfpathcurveto{\pgfqpoint{1.241663in}{0.599996in}}{\pgfqpoint{1.252262in}{0.595606in}}{\pgfqpoint{1.263312in}{0.595606in}}%
\pgfpathclose%
\pgfusepath{stroke,fill}%
\end{pgfscope}%
\begin{pgfscope}%
\pgfpathrectangle{\pgfqpoint{0.750000in}{0.500000in}}{\pgfqpoint{4.650000in}{3.020000in}}%
\pgfusepath{clip}%
\pgfsetbuttcap%
\pgfsetroundjoin%
\definecolor{currentfill}{rgb}{1.000000,0.498039,0.054902}%
\pgfsetfillcolor{currentfill}%
\pgfsetlinewidth{1.003750pt}%
\definecolor{currentstroke}{rgb}{1.000000,0.498039,0.054902}%
\pgfsetstrokecolor{currentstroke}%
\pgfsetdash{}{0pt}%
\pgfpathmoveto{\pgfqpoint{1.625649in}{2.928269in}}%
\pgfpathcurveto{\pgfqpoint{1.636699in}{2.928269in}}{\pgfqpoint{1.647299in}{2.932659in}}{\pgfqpoint{1.655112in}{2.940473in}}%
\pgfpathcurveto{\pgfqpoint{1.662926in}{2.948286in}}{\pgfqpoint{1.667316in}{2.958885in}}{\pgfqpoint{1.667316in}{2.969936in}}%
\pgfpathcurveto{\pgfqpoint{1.667316in}{2.980986in}}{\pgfqpoint{1.662926in}{2.991585in}}{\pgfqpoint{1.655112in}{2.999398in}}%
\pgfpathcurveto{\pgfqpoint{1.647299in}{3.007212in}}{\pgfqpoint{1.636699in}{3.011602in}}{\pgfqpoint{1.625649in}{3.011602in}}%
\pgfpathcurveto{\pgfqpoint{1.614599in}{3.011602in}}{\pgfqpoint{1.604000in}{3.007212in}}{\pgfqpoint{1.596187in}{2.999398in}}%
\pgfpathcurveto{\pgfqpoint{1.588373in}{2.991585in}}{\pgfqpoint{1.583983in}{2.980986in}}{\pgfqpoint{1.583983in}{2.969936in}}%
\pgfpathcurveto{\pgfqpoint{1.583983in}{2.958885in}}{\pgfqpoint{1.588373in}{2.948286in}}{\pgfqpoint{1.596187in}{2.940473in}}%
\pgfpathcurveto{\pgfqpoint{1.604000in}{2.932659in}}{\pgfqpoint{1.614599in}{2.928269in}}{\pgfqpoint{1.625649in}{2.928269in}}%
\pgfpathclose%
\pgfusepath{stroke,fill}%
\end{pgfscope}%
\begin{pgfscope}%
\pgfpathrectangle{\pgfqpoint{0.750000in}{0.500000in}}{\pgfqpoint{4.650000in}{3.020000in}}%
\pgfusepath{clip}%
\pgfsetbuttcap%
\pgfsetroundjoin%
\definecolor{currentfill}{rgb}{1.000000,0.498039,0.054902}%
\pgfsetfillcolor{currentfill}%
\pgfsetlinewidth{1.003750pt}%
\definecolor{currentstroke}{rgb}{1.000000,0.498039,0.054902}%
\pgfsetstrokecolor{currentstroke}%
\pgfsetdash{}{0pt}%
\pgfpathmoveto{\pgfqpoint{2.229545in}{2.939952in}}%
\pgfpathcurveto{\pgfqpoint{2.240596in}{2.939952in}}{\pgfqpoint{2.251195in}{2.944342in}}{\pgfqpoint{2.259008in}{2.952156in}}%
\pgfpathcurveto{\pgfqpoint{2.266822in}{2.959969in}}{\pgfqpoint{2.271212in}{2.970568in}}{\pgfqpoint{2.271212in}{2.981618in}}%
\pgfpathcurveto{\pgfqpoint{2.271212in}{2.992668in}}{\pgfqpoint{2.266822in}{3.003267in}}{\pgfqpoint{2.259008in}{3.011081in}}%
\pgfpathcurveto{\pgfqpoint{2.251195in}{3.018895in}}{\pgfqpoint{2.240596in}{3.023285in}}{\pgfqpoint{2.229545in}{3.023285in}}%
\pgfpathcurveto{\pgfqpoint{2.218495in}{3.023285in}}{\pgfqpoint{2.207896in}{3.018895in}}{\pgfqpoint{2.200083in}{3.011081in}}%
\pgfpathcurveto{\pgfqpoint{2.192269in}{3.003267in}}{\pgfqpoint{2.187879in}{2.992668in}}{\pgfqpoint{2.187879in}{2.981618in}}%
\pgfpathcurveto{\pgfqpoint{2.187879in}{2.970568in}}{\pgfqpoint{2.192269in}{2.959969in}}{\pgfqpoint{2.200083in}{2.952156in}}%
\pgfpathcurveto{\pgfqpoint{2.207896in}{2.944342in}}{\pgfqpoint{2.218495in}{2.939952in}}{\pgfqpoint{2.229545in}{2.939952in}}%
\pgfpathclose%
\pgfusepath{stroke,fill}%
\end{pgfscope}%
\begin{pgfscope}%
\pgfpathrectangle{\pgfqpoint{0.750000in}{0.500000in}}{\pgfqpoint{4.650000in}{3.020000in}}%
\pgfusepath{clip}%
\pgfsetbuttcap%
\pgfsetroundjoin%
\definecolor{currentfill}{rgb}{0.121569,0.466667,0.705882}%
\pgfsetfillcolor{currentfill}%
\pgfsetlinewidth{1.003750pt}%
\definecolor{currentstroke}{rgb}{0.121569,0.466667,0.705882}%
\pgfsetstrokecolor{currentstroke}%
\pgfsetdash{}{0pt}%
\pgfpathmoveto{\pgfqpoint{1.021753in}{0.595606in}}%
\pgfpathcurveto{\pgfqpoint{1.032803in}{0.595606in}}{\pgfqpoint{1.043402in}{0.599996in}}{\pgfqpoint{1.051216in}{0.607810in}}%
\pgfpathcurveto{\pgfqpoint{1.059030in}{0.615624in}}{\pgfqpoint{1.063420in}{0.626223in}}{\pgfqpoint{1.063420in}{0.637273in}}%
\pgfpathcurveto{\pgfqpoint{1.063420in}{0.648323in}}{\pgfqpoint{1.059030in}{0.658922in}}{\pgfqpoint{1.051216in}{0.666736in}}%
\pgfpathcurveto{\pgfqpoint{1.043402in}{0.674549in}}{\pgfqpoint{1.032803in}{0.678939in}}{\pgfqpoint{1.021753in}{0.678939in}}%
\pgfpathcurveto{\pgfqpoint{1.010703in}{0.678939in}}{\pgfqpoint{1.000104in}{0.674549in}}{\pgfqpoint{0.992290in}{0.666736in}}%
\pgfpathcurveto{\pgfqpoint{0.984477in}{0.658922in}}{\pgfqpoint{0.980087in}{0.648323in}}{\pgfqpoint{0.980087in}{0.637273in}}%
\pgfpathcurveto{\pgfqpoint{0.980087in}{0.626223in}}{\pgfqpoint{0.984477in}{0.615624in}}{\pgfqpoint{0.992290in}{0.607810in}}%
\pgfpathcurveto{\pgfqpoint{1.000104in}{0.599996in}}{\pgfqpoint{1.010703in}{0.595606in}}{\pgfqpoint{1.021753in}{0.595606in}}%
\pgfpathclose%
\pgfusepath{stroke,fill}%
\end{pgfscope}%
\begin{pgfscope}%
\pgfpathrectangle{\pgfqpoint{0.750000in}{0.500000in}}{\pgfqpoint{4.650000in}{3.020000in}}%
\pgfusepath{clip}%
\pgfsetbuttcap%
\pgfsetroundjoin%
\definecolor{currentfill}{rgb}{1.000000,0.498039,0.054902}%
\pgfsetfillcolor{currentfill}%
\pgfsetlinewidth{1.003750pt}%
\definecolor{currentstroke}{rgb}{1.000000,0.498039,0.054902}%
\pgfsetstrokecolor{currentstroke}%
\pgfsetdash{}{0pt}%
\pgfpathmoveto{\pgfqpoint{1.625649in}{2.990577in}}%
\pgfpathcurveto{\pgfqpoint{1.636699in}{2.990577in}}{\pgfqpoint{1.647299in}{2.994967in}}{\pgfqpoint{1.655112in}{3.002781in}}%
\pgfpathcurveto{\pgfqpoint{1.662926in}{3.010595in}}{\pgfqpoint{1.667316in}{3.021194in}}{\pgfqpoint{1.667316in}{3.032244in}}%
\pgfpathcurveto{\pgfqpoint{1.667316in}{3.043294in}}{\pgfqpoint{1.662926in}{3.053893in}}{\pgfqpoint{1.655112in}{3.061706in}}%
\pgfpathcurveto{\pgfqpoint{1.647299in}{3.069520in}}{\pgfqpoint{1.636699in}{3.073910in}}{\pgfqpoint{1.625649in}{3.073910in}}%
\pgfpathcurveto{\pgfqpoint{1.614599in}{3.073910in}}{\pgfqpoint{1.604000in}{3.069520in}}{\pgfqpoint{1.596187in}{3.061706in}}%
\pgfpathcurveto{\pgfqpoint{1.588373in}{3.053893in}}{\pgfqpoint{1.583983in}{3.043294in}}{\pgfqpoint{1.583983in}{3.032244in}}%
\pgfpathcurveto{\pgfqpoint{1.583983in}{3.021194in}}{\pgfqpoint{1.588373in}{3.010595in}}{\pgfqpoint{1.596187in}{3.002781in}}%
\pgfpathcurveto{\pgfqpoint{1.604000in}{2.994967in}}{\pgfqpoint{1.614599in}{2.990577in}}{\pgfqpoint{1.625649in}{2.990577in}}%
\pgfpathclose%
\pgfusepath{stroke,fill}%
\end{pgfscope}%
\begin{pgfscope}%
\pgfpathrectangle{\pgfqpoint{0.750000in}{0.500000in}}{\pgfqpoint{4.650000in}{3.020000in}}%
\pgfusepath{clip}%
\pgfsetbuttcap%
\pgfsetroundjoin%
\definecolor{currentfill}{rgb}{1.000000,0.498039,0.054902}%
\pgfsetfillcolor{currentfill}%
\pgfsetlinewidth{1.003750pt}%
\definecolor{currentstroke}{rgb}{1.000000,0.498039,0.054902}%
\pgfsetstrokecolor{currentstroke}%
\pgfsetdash{}{0pt}%
\pgfpathmoveto{\pgfqpoint{1.625649in}{2.998366in}}%
\pgfpathcurveto{\pgfqpoint{1.636699in}{2.998366in}}{\pgfqpoint{1.647299in}{3.002756in}}{\pgfqpoint{1.655112in}{3.010569in}}%
\pgfpathcurveto{\pgfqpoint{1.662926in}{3.018383in}}{\pgfqpoint{1.667316in}{3.028982in}}{\pgfqpoint{1.667316in}{3.040032in}}%
\pgfpathcurveto{\pgfqpoint{1.667316in}{3.051082in}}{\pgfqpoint{1.662926in}{3.061681in}}{\pgfqpoint{1.655112in}{3.069495in}}%
\pgfpathcurveto{\pgfqpoint{1.647299in}{3.077309in}}{\pgfqpoint{1.636699in}{3.081699in}}{\pgfqpoint{1.625649in}{3.081699in}}%
\pgfpathcurveto{\pgfqpoint{1.614599in}{3.081699in}}{\pgfqpoint{1.604000in}{3.077309in}}{\pgfqpoint{1.596187in}{3.069495in}}%
\pgfpathcurveto{\pgfqpoint{1.588373in}{3.061681in}}{\pgfqpoint{1.583983in}{3.051082in}}{\pgfqpoint{1.583983in}{3.040032in}}%
\pgfpathcurveto{\pgfqpoint{1.583983in}{3.028982in}}{\pgfqpoint{1.588373in}{3.018383in}}{\pgfqpoint{1.596187in}{3.010569in}}%
\pgfpathcurveto{\pgfqpoint{1.604000in}{3.002756in}}{\pgfqpoint{1.614599in}{2.998366in}}{\pgfqpoint{1.625649in}{2.998366in}}%
\pgfpathclose%
\pgfusepath{stroke,fill}%
\end{pgfscope}%
\begin{pgfscope}%
\pgfpathrectangle{\pgfqpoint{0.750000in}{0.500000in}}{\pgfqpoint{4.650000in}{3.020000in}}%
\pgfusepath{clip}%
\pgfsetbuttcap%
\pgfsetroundjoin%
\definecolor{currentfill}{rgb}{1.000000,0.498039,0.054902}%
\pgfsetfillcolor{currentfill}%
\pgfsetlinewidth{1.003750pt}%
\definecolor{currentstroke}{rgb}{1.000000,0.498039,0.054902}%
\pgfsetstrokecolor{currentstroke}%
\pgfsetdash{}{0pt}%
\pgfpathmoveto{\pgfqpoint{1.565260in}{2.932163in}}%
\pgfpathcurveto{\pgfqpoint{1.576310in}{2.932163in}}{\pgfqpoint{1.586909in}{2.936553in}}{\pgfqpoint{1.594723in}{2.944367in}}%
\pgfpathcurveto{\pgfqpoint{1.602536in}{2.952181in}}{\pgfqpoint{1.606926in}{2.962780in}}{\pgfqpoint{1.606926in}{2.973830in}}%
\pgfpathcurveto{\pgfqpoint{1.606926in}{2.984880in}}{\pgfqpoint{1.602536in}{2.995479in}}{\pgfqpoint{1.594723in}{3.003293in}}%
\pgfpathcurveto{\pgfqpoint{1.586909in}{3.011106in}}{\pgfqpoint{1.576310in}{3.015496in}}{\pgfqpoint{1.565260in}{3.015496in}}%
\pgfpathcurveto{\pgfqpoint{1.554210in}{3.015496in}}{\pgfqpoint{1.543611in}{3.011106in}}{\pgfqpoint{1.535797in}{3.003293in}}%
\pgfpathcurveto{\pgfqpoint{1.527983in}{2.995479in}}{\pgfqpoint{1.523593in}{2.984880in}}{\pgfqpoint{1.523593in}{2.973830in}}%
\pgfpathcurveto{\pgfqpoint{1.523593in}{2.962780in}}{\pgfqpoint{1.527983in}{2.952181in}}{\pgfqpoint{1.535797in}{2.944367in}}%
\pgfpathcurveto{\pgfqpoint{1.543611in}{2.936553in}}{\pgfqpoint{1.554210in}{2.932163in}}{\pgfqpoint{1.565260in}{2.932163in}}%
\pgfpathclose%
\pgfusepath{stroke,fill}%
\end{pgfscope}%
\begin{pgfscope}%
\pgfpathrectangle{\pgfqpoint{0.750000in}{0.500000in}}{\pgfqpoint{4.650000in}{3.020000in}}%
\pgfusepath{clip}%
\pgfsetbuttcap%
\pgfsetroundjoin%
\definecolor{currentfill}{rgb}{1.000000,0.498039,0.054902}%
\pgfsetfillcolor{currentfill}%
\pgfsetlinewidth{1.003750pt}%
\definecolor{currentstroke}{rgb}{1.000000,0.498039,0.054902}%
\pgfsetstrokecolor{currentstroke}%
\pgfsetdash{}{0pt}%
\pgfpathmoveto{\pgfqpoint{1.806818in}{2.936057in}}%
\pgfpathcurveto{\pgfqpoint{1.817868in}{2.936057in}}{\pgfqpoint{1.828467in}{2.940448in}}{\pgfqpoint{1.836281in}{2.948261in}}%
\pgfpathcurveto{\pgfqpoint{1.844095in}{2.956075in}}{\pgfqpoint{1.848485in}{2.966674in}}{\pgfqpoint{1.848485in}{2.977724in}}%
\pgfpathcurveto{\pgfqpoint{1.848485in}{2.988774in}}{\pgfqpoint{1.844095in}{2.999373in}}{\pgfqpoint{1.836281in}{3.007187in}}%
\pgfpathcurveto{\pgfqpoint{1.828467in}{3.015000in}}{\pgfqpoint{1.817868in}{3.019391in}}{\pgfqpoint{1.806818in}{3.019391in}}%
\pgfpathcurveto{\pgfqpoint{1.795768in}{3.019391in}}{\pgfqpoint{1.785169in}{3.015000in}}{\pgfqpoint{1.777355in}{3.007187in}}%
\pgfpathcurveto{\pgfqpoint{1.769542in}{2.999373in}}{\pgfqpoint{1.765152in}{2.988774in}}{\pgfqpoint{1.765152in}{2.977724in}}%
\pgfpathcurveto{\pgfqpoint{1.765152in}{2.966674in}}{\pgfqpoint{1.769542in}{2.956075in}}{\pgfqpoint{1.777355in}{2.948261in}}%
\pgfpathcurveto{\pgfqpoint{1.785169in}{2.940448in}}{\pgfqpoint{1.795768in}{2.936057in}}{\pgfqpoint{1.806818in}{2.936057in}}%
\pgfpathclose%
\pgfusepath{stroke,fill}%
\end{pgfscope}%
\begin{pgfscope}%
\pgfpathrectangle{\pgfqpoint{0.750000in}{0.500000in}}{\pgfqpoint{4.650000in}{3.020000in}}%
\pgfusepath{clip}%
\pgfsetbuttcap%
\pgfsetroundjoin%
\definecolor{currentfill}{rgb}{1.000000,0.498039,0.054902}%
\pgfsetfillcolor{currentfill}%
\pgfsetlinewidth{1.003750pt}%
\definecolor{currentstroke}{rgb}{1.000000,0.498039,0.054902}%
\pgfsetstrokecolor{currentstroke}%
\pgfsetdash{}{0pt}%
\pgfpathmoveto{\pgfqpoint{1.565260in}{2.943846in}}%
\pgfpathcurveto{\pgfqpoint{1.576310in}{2.943846in}}{\pgfqpoint{1.586909in}{2.948236in}}{\pgfqpoint{1.594723in}{2.956050in}}%
\pgfpathcurveto{\pgfqpoint{1.602536in}{2.963863in}}{\pgfqpoint{1.606926in}{2.974462in}}{\pgfqpoint{1.606926in}{2.985513in}}%
\pgfpathcurveto{\pgfqpoint{1.606926in}{2.996563in}}{\pgfqpoint{1.602536in}{3.007162in}}{\pgfqpoint{1.594723in}{3.014975in}}%
\pgfpathcurveto{\pgfqpoint{1.586909in}{3.022789in}}{\pgfqpoint{1.576310in}{3.027179in}}{\pgfqpoint{1.565260in}{3.027179in}}%
\pgfpathcurveto{\pgfqpoint{1.554210in}{3.027179in}}{\pgfqpoint{1.543611in}{3.022789in}}{\pgfqpoint{1.535797in}{3.014975in}}%
\pgfpathcurveto{\pgfqpoint{1.527983in}{3.007162in}}{\pgfqpoint{1.523593in}{2.996563in}}{\pgfqpoint{1.523593in}{2.985513in}}%
\pgfpathcurveto{\pgfqpoint{1.523593in}{2.974462in}}{\pgfqpoint{1.527983in}{2.963863in}}{\pgfqpoint{1.535797in}{2.956050in}}%
\pgfpathcurveto{\pgfqpoint{1.543611in}{2.948236in}}{\pgfqpoint{1.554210in}{2.943846in}}{\pgfqpoint{1.565260in}{2.943846in}}%
\pgfpathclose%
\pgfusepath{stroke,fill}%
\end{pgfscope}%
\begin{pgfscope}%
\pgfpathrectangle{\pgfqpoint{0.750000in}{0.500000in}}{\pgfqpoint{4.650000in}{3.020000in}}%
\pgfusepath{clip}%
\pgfsetbuttcap%
\pgfsetroundjoin%
\definecolor{currentfill}{rgb}{1.000000,0.498039,0.054902}%
\pgfsetfillcolor{currentfill}%
\pgfsetlinewidth{1.003750pt}%
\definecolor{currentstroke}{rgb}{1.000000,0.498039,0.054902}%
\pgfsetstrokecolor{currentstroke}%
\pgfsetdash{}{0pt}%
\pgfpathmoveto{\pgfqpoint{1.504870in}{2.932163in}}%
\pgfpathcurveto{\pgfqpoint{1.515920in}{2.932163in}}{\pgfqpoint{1.526519in}{2.936553in}}{\pgfqpoint{1.534333in}{2.944367in}}%
\pgfpathcurveto{\pgfqpoint{1.542147in}{2.952181in}}{\pgfqpoint{1.546537in}{2.962780in}}{\pgfqpoint{1.546537in}{2.973830in}}%
\pgfpathcurveto{\pgfqpoint{1.546537in}{2.984880in}}{\pgfqpoint{1.542147in}{2.995479in}}{\pgfqpoint{1.534333in}{3.003293in}}%
\pgfpathcurveto{\pgfqpoint{1.526519in}{3.011106in}}{\pgfqpoint{1.515920in}{3.015496in}}{\pgfqpoint{1.504870in}{3.015496in}}%
\pgfpathcurveto{\pgfqpoint{1.493820in}{3.015496in}}{\pgfqpoint{1.483221in}{3.011106in}}{\pgfqpoint{1.475407in}{3.003293in}}%
\pgfpathcurveto{\pgfqpoint{1.467594in}{2.995479in}}{\pgfqpoint{1.463203in}{2.984880in}}{\pgfqpoint{1.463203in}{2.973830in}}%
\pgfpathcurveto{\pgfqpoint{1.463203in}{2.962780in}}{\pgfqpoint{1.467594in}{2.952181in}}{\pgfqpoint{1.475407in}{2.944367in}}%
\pgfpathcurveto{\pgfqpoint{1.483221in}{2.936553in}}{\pgfqpoint{1.493820in}{2.932163in}}{\pgfqpoint{1.504870in}{2.932163in}}%
\pgfpathclose%
\pgfusepath{stroke,fill}%
\end{pgfscope}%
\begin{pgfscope}%
\pgfpathrectangle{\pgfqpoint{0.750000in}{0.500000in}}{\pgfqpoint{4.650000in}{3.020000in}}%
\pgfusepath{clip}%
\pgfsetbuttcap%
\pgfsetroundjoin%
\definecolor{currentfill}{rgb}{0.121569,0.466667,0.705882}%
\pgfsetfillcolor{currentfill}%
\pgfsetlinewidth{1.003750pt}%
\definecolor{currentstroke}{rgb}{0.121569,0.466667,0.705882}%
\pgfsetstrokecolor{currentstroke}%
\pgfsetdash{}{0pt}%
\pgfpathmoveto{\pgfqpoint{1.384091in}{0.611183in}}%
\pgfpathcurveto{\pgfqpoint{1.395141in}{0.611183in}}{\pgfqpoint{1.405740in}{0.615573in}}{\pgfqpoint{1.413554in}{0.623387in}}%
\pgfpathcurveto{\pgfqpoint{1.421367in}{0.631201in}}{\pgfqpoint{1.425758in}{0.641800in}}{\pgfqpoint{1.425758in}{0.652850in}}%
\pgfpathcurveto{\pgfqpoint{1.425758in}{0.663900in}}{\pgfqpoint{1.421367in}{0.674499in}}{\pgfqpoint{1.413554in}{0.682313in}}%
\pgfpathcurveto{\pgfqpoint{1.405740in}{0.690126in}}{\pgfqpoint{1.395141in}{0.694516in}}{\pgfqpoint{1.384091in}{0.694516in}}%
\pgfpathcurveto{\pgfqpoint{1.373041in}{0.694516in}}{\pgfqpoint{1.362442in}{0.690126in}}{\pgfqpoint{1.354628in}{0.682313in}}%
\pgfpathcurveto{\pgfqpoint{1.346815in}{0.674499in}}{\pgfqpoint{1.342424in}{0.663900in}}{\pgfqpoint{1.342424in}{0.652850in}}%
\pgfpathcurveto{\pgfqpoint{1.342424in}{0.641800in}}{\pgfqpoint{1.346815in}{0.631201in}}{\pgfqpoint{1.354628in}{0.623387in}}%
\pgfpathcurveto{\pgfqpoint{1.362442in}{0.615573in}}{\pgfqpoint{1.373041in}{0.611183in}}{\pgfqpoint{1.384091in}{0.611183in}}%
\pgfpathclose%
\pgfusepath{stroke,fill}%
\end{pgfscope}%
\begin{pgfscope}%
\pgfpathrectangle{\pgfqpoint{0.750000in}{0.500000in}}{\pgfqpoint{4.650000in}{3.020000in}}%
\pgfusepath{clip}%
\pgfsetbuttcap%
\pgfsetroundjoin%
\definecolor{currentfill}{rgb}{1.000000,0.498039,0.054902}%
\pgfsetfillcolor{currentfill}%
\pgfsetlinewidth{1.003750pt}%
\definecolor{currentstroke}{rgb}{1.000000,0.498039,0.054902}%
\pgfsetstrokecolor{currentstroke}%
\pgfsetdash{}{0pt}%
\pgfpathmoveto{\pgfqpoint{1.927597in}{2.955529in}}%
\pgfpathcurveto{\pgfqpoint{1.938648in}{2.955529in}}{\pgfqpoint{1.949247in}{2.959919in}}{\pgfqpoint{1.957060in}{2.967733in}}%
\pgfpathcurveto{\pgfqpoint{1.964874in}{2.975546in}}{\pgfqpoint{1.969264in}{2.986145in}}{\pgfqpoint{1.969264in}{2.997195in}}%
\pgfpathcurveto{\pgfqpoint{1.969264in}{3.008245in}}{\pgfqpoint{1.964874in}{3.018845in}}{\pgfqpoint{1.957060in}{3.026658in}}%
\pgfpathcurveto{\pgfqpoint{1.949247in}{3.034472in}}{\pgfqpoint{1.938648in}{3.038862in}}{\pgfqpoint{1.927597in}{3.038862in}}%
\pgfpathcurveto{\pgfqpoint{1.916547in}{3.038862in}}{\pgfqpoint{1.905948in}{3.034472in}}{\pgfqpoint{1.898135in}{3.026658in}}%
\pgfpathcurveto{\pgfqpoint{1.890321in}{3.018845in}}{\pgfqpoint{1.885931in}{3.008245in}}{\pgfqpoint{1.885931in}{2.997195in}}%
\pgfpathcurveto{\pgfqpoint{1.885931in}{2.986145in}}{\pgfqpoint{1.890321in}{2.975546in}}{\pgfqpoint{1.898135in}{2.967733in}}%
\pgfpathcurveto{\pgfqpoint{1.905948in}{2.959919in}}{\pgfqpoint{1.916547in}{2.955529in}}{\pgfqpoint{1.927597in}{2.955529in}}%
\pgfpathclose%
\pgfusepath{stroke,fill}%
\end{pgfscope}%
\begin{pgfscope}%
\pgfpathrectangle{\pgfqpoint{0.750000in}{0.500000in}}{\pgfqpoint{4.650000in}{3.020000in}}%
\pgfusepath{clip}%
\pgfsetbuttcap%
\pgfsetroundjoin%
\definecolor{currentfill}{rgb}{1.000000,0.498039,0.054902}%
\pgfsetfillcolor{currentfill}%
\pgfsetlinewidth{1.003750pt}%
\definecolor{currentstroke}{rgb}{1.000000,0.498039,0.054902}%
\pgfsetstrokecolor{currentstroke}%
\pgfsetdash{}{0pt}%
\pgfpathmoveto{\pgfqpoint{3.014610in}{2.928269in}}%
\pgfpathcurveto{\pgfqpoint{3.025661in}{2.928269in}}{\pgfqpoint{3.036260in}{2.932659in}}{\pgfqpoint{3.044073in}{2.940473in}}%
\pgfpathcurveto{\pgfqpoint{3.051887in}{2.948286in}}{\pgfqpoint{3.056277in}{2.958885in}}{\pgfqpoint{3.056277in}{2.969936in}}%
\pgfpathcurveto{\pgfqpoint{3.056277in}{2.980986in}}{\pgfqpoint{3.051887in}{2.991585in}}{\pgfqpoint{3.044073in}{2.999398in}}%
\pgfpathcurveto{\pgfqpoint{3.036260in}{3.007212in}}{\pgfqpoint{3.025661in}{3.011602in}}{\pgfqpoint{3.014610in}{3.011602in}}%
\pgfpathcurveto{\pgfqpoint{3.003560in}{3.011602in}}{\pgfqpoint{2.992961in}{3.007212in}}{\pgfqpoint{2.985148in}{2.999398in}}%
\pgfpathcurveto{\pgfqpoint{2.977334in}{2.991585in}}{\pgfqpoint{2.972944in}{2.980986in}}{\pgfqpoint{2.972944in}{2.969936in}}%
\pgfpathcurveto{\pgfqpoint{2.972944in}{2.958885in}}{\pgfqpoint{2.977334in}{2.948286in}}{\pgfqpoint{2.985148in}{2.940473in}}%
\pgfpathcurveto{\pgfqpoint{2.992961in}{2.932659in}}{\pgfqpoint{3.003560in}{2.928269in}}{\pgfqpoint{3.014610in}{2.928269in}}%
\pgfpathclose%
\pgfusepath{stroke,fill}%
\end{pgfscope}%
\begin{pgfscope}%
\pgfpathrectangle{\pgfqpoint{0.750000in}{0.500000in}}{\pgfqpoint{4.650000in}{3.020000in}}%
\pgfusepath{clip}%
\pgfsetbuttcap%
\pgfsetroundjoin%
\definecolor{currentfill}{rgb}{1.000000,0.498039,0.054902}%
\pgfsetfillcolor{currentfill}%
\pgfsetlinewidth{1.003750pt}%
\definecolor{currentstroke}{rgb}{1.000000,0.498039,0.054902}%
\pgfsetstrokecolor{currentstroke}%
\pgfsetdash{}{0pt}%
\pgfpathmoveto{\pgfqpoint{1.384091in}{2.936057in}}%
\pgfpathcurveto{\pgfqpoint{1.395141in}{2.936057in}}{\pgfqpoint{1.405740in}{2.940448in}}{\pgfqpoint{1.413554in}{2.948261in}}%
\pgfpathcurveto{\pgfqpoint{1.421367in}{2.956075in}}{\pgfqpoint{1.425758in}{2.966674in}}{\pgfqpoint{1.425758in}{2.977724in}}%
\pgfpathcurveto{\pgfqpoint{1.425758in}{2.988774in}}{\pgfqpoint{1.421367in}{2.999373in}}{\pgfqpoint{1.413554in}{3.007187in}}%
\pgfpathcurveto{\pgfqpoint{1.405740in}{3.015000in}}{\pgfqpoint{1.395141in}{3.019391in}}{\pgfqpoint{1.384091in}{3.019391in}}%
\pgfpathcurveto{\pgfqpoint{1.373041in}{3.019391in}}{\pgfqpoint{1.362442in}{3.015000in}}{\pgfqpoint{1.354628in}{3.007187in}}%
\pgfpathcurveto{\pgfqpoint{1.346815in}{2.999373in}}{\pgfqpoint{1.342424in}{2.988774in}}{\pgfqpoint{1.342424in}{2.977724in}}%
\pgfpathcurveto{\pgfqpoint{1.342424in}{2.966674in}}{\pgfqpoint{1.346815in}{2.956075in}}{\pgfqpoint{1.354628in}{2.948261in}}%
\pgfpathcurveto{\pgfqpoint{1.362442in}{2.940448in}}{\pgfqpoint{1.373041in}{2.936057in}}{\pgfqpoint{1.384091in}{2.936057in}}%
\pgfpathclose%
\pgfusepath{stroke,fill}%
\end{pgfscope}%
\begin{pgfscope}%
\pgfpathrectangle{\pgfqpoint{0.750000in}{0.500000in}}{\pgfqpoint{4.650000in}{3.020000in}}%
\pgfusepath{clip}%
\pgfsetbuttcap%
\pgfsetroundjoin%
\definecolor{currentfill}{rgb}{1.000000,0.498039,0.054902}%
\pgfsetfillcolor{currentfill}%
\pgfsetlinewidth{1.003750pt}%
\definecolor{currentstroke}{rgb}{1.000000,0.498039,0.054902}%
\pgfsetstrokecolor{currentstroke}%
\pgfsetdash{}{0pt}%
\pgfpathmoveto{\pgfqpoint{1.323701in}{3.239810in}}%
\pgfpathcurveto{\pgfqpoint{1.334751in}{3.239810in}}{\pgfqpoint{1.345350in}{3.244200in}}{\pgfqpoint{1.353164in}{3.252014in}}%
\pgfpathcurveto{\pgfqpoint{1.360978in}{3.259827in}}{\pgfqpoint{1.365368in}{3.270426in}}{\pgfqpoint{1.365368in}{3.281476in}}%
\pgfpathcurveto{\pgfqpoint{1.365368in}{3.292527in}}{\pgfqpoint{1.360978in}{3.303126in}}{\pgfqpoint{1.353164in}{3.310939in}}%
\pgfpathcurveto{\pgfqpoint{1.345350in}{3.318753in}}{\pgfqpoint{1.334751in}{3.323143in}}{\pgfqpoint{1.323701in}{3.323143in}}%
\pgfpathcurveto{\pgfqpoint{1.312651in}{3.323143in}}{\pgfqpoint{1.302052in}{3.318753in}}{\pgfqpoint{1.294239in}{3.310939in}}%
\pgfpathcurveto{\pgfqpoint{1.286425in}{3.303126in}}{\pgfqpoint{1.282035in}{3.292527in}}{\pgfqpoint{1.282035in}{3.281476in}}%
\pgfpathcurveto{\pgfqpoint{1.282035in}{3.270426in}}{\pgfqpoint{1.286425in}{3.259827in}}{\pgfqpoint{1.294239in}{3.252014in}}%
\pgfpathcurveto{\pgfqpoint{1.302052in}{3.244200in}}{\pgfqpoint{1.312651in}{3.239810in}}{\pgfqpoint{1.323701in}{3.239810in}}%
\pgfpathclose%
\pgfusepath{stroke,fill}%
\end{pgfscope}%
\begin{pgfscope}%
\pgfpathrectangle{\pgfqpoint{0.750000in}{0.500000in}}{\pgfqpoint{4.650000in}{3.020000in}}%
\pgfusepath{clip}%
\pgfsetbuttcap%
\pgfsetroundjoin%
\definecolor{currentfill}{rgb}{1.000000,0.498039,0.054902}%
\pgfsetfillcolor{currentfill}%
\pgfsetlinewidth{1.003750pt}%
\definecolor{currentstroke}{rgb}{1.000000,0.498039,0.054902}%
\pgfsetstrokecolor{currentstroke}%
\pgfsetdash{}{0pt}%
\pgfpathmoveto{\pgfqpoint{1.625649in}{2.936057in}}%
\pgfpathcurveto{\pgfqpoint{1.636699in}{2.936057in}}{\pgfqpoint{1.647299in}{2.940448in}}{\pgfqpoint{1.655112in}{2.948261in}}%
\pgfpathcurveto{\pgfqpoint{1.662926in}{2.956075in}}{\pgfqpoint{1.667316in}{2.966674in}}{\pgfqpoint{1.667316in}{2.977724in}}%
\pgfpathcurveto{\pgfqpoint{1.667316in}{2.988774in}}{\pgfqpoint{1.662926in}{2.999373in}}{\pgfqpoint{1.655112in}{3.007187in}}%
\pgfpathcurveto{\pgfqpoint{1.647299in}{3.015000in}}{\pgfqpoint{1.636699in}{3.019391in}}{\pgfqpoint{1.625649in}{3.019391in}}%
\pgfpathcurveto{\pgfqpoint{1.614599in}{3.019391in}}{\pgfqpoint{1.604000in}{3.015000in}}{\pgfqpoint{1.596187in}{3.007187in}}%
\pgfpathcurveto{\pgfqpoint{1.588373in}{2.999373in}}{\pgfqpoint{1.583983in}{2.988774in}}{\pgfqpoint{1.583983in}{2.977724in}}%
\pgfpathcurveto{\pgfqpoint{1.583983in}{2.966674in}}{\pgfqpoint{1.588373in}{2.956075in}}{\pgfqpoint{1.596187in}{2.948261in}}%
\pgfpathcurveto{\pgfqpoint{1.604000in}{2.940448in}}{\pgfqpoint{1.614599in}{2.936057in}}{\pgfqpoint{1.625649in}{2.936057in}}%
\pgfpathclose%
\pgfusepath{stroke,fill}%
\end{pgfscope}%
\begin{pgfscope}%
\pgfpathrectangle{\pgfqpoint{0.750000in}{0.500000in}}{\pgfqpoint{4.650000in}{3.020000in}}%
\pgfusepath{clip}%
\pgfsetbuttcap%
\pgfsetroundjoin%
\definecolor{currentfill}{rgb}{0.121569,0.466667,0.705882}%
\pgfsetfillcolor{currentfill}%
\pgfsetlinewidth{1.003750pt}%
\definecolor{currentstroke}{rgb}{0.121569,0.466667,0.705882}%
\pgfsetstrokecolor{currentstroke}%
\pgfsetdash{}{0pt}%
\pgfpathmoveto{\pgfqpoint{1.263312in}{0.595606in}}%
\pgfpathcurveto{\pgfqpoint{1.274362in}{0.595606in}}{\pgfqpoint{1.284961in}{0.599996in}}{\pgfqpoint{1.292774in}{0.607810in}}%
\pgfpathcurveto{\pgfqpoint{1.300588in}{0.615624in}}{\pgfqpoint{1.304978in}{0.626223in}}{\pgfqpoint{1.304978in}{0.637273in}}%
\pgfpathcurveto{\pgfqpoint{1.304978in}{0.648323in}}{\pgfqpoint{1.300588in}{0.658922in}}{\pgfqpoint{1.292774in}{0.666736in}}%
\pgfpathcurveto{\pgfqpoint{1.284961in}{0.674549in}}{\pgfqpoint{1.274362in}{0.678939in}}{\pgfqpoint{1.263312in}{0.678939in}}%
\pgfpathcurveto{\pgfqpoint{1.252262in}{0.678939in}}{\pgfqpoint{1.241663in}{0.674549in}}{\pgfqpoint{1.233849in}{0.666736in}}%
\pgfpathcurveto{\pgfqpoint{1.226035in}{0.658922in}}{\pgfqpoint{1.221645in}{0.648323in}}{\pgfqpoint{1.221645in}{0.637273in}}%
\pgfpathcurveto{\pgfqpoint{1.221645in}{0.626223in}}{\pgfqpoint{1.226035in}{0.615624in}}{\pgfqpoint{1.233849in}{0.607810in}}%
\pgfpathcurveto{\pgfqpoint{1.241663in}{0.599996in}}{\pgfqpoint{1.252262in}{0.595606in}}{\pgfqpoint{1.263312in}{0.595606in}}%
\pgfpathclose%
\pgfusepath{stroke,fill}%
\end{pgfscope}%
\begin{pgfscope}%
\pgfpathrectangle{\pgfqpoint{0.750000in}{0.500000in}}{\pgfqpoint{4.650000in}{3.020000in}}%
\pgfusepath{clip}%
\pgfsetbuttcap%
\pgfsetroundjoin%
\definecolor{currentfill}{rgb}{0.121569,0.466667,0.705882}%
\pgfsetfillcolor{currentfill}%
\pgfsetlinewidth{1.003750pt}%
\definecolor{currentstroke}{rgb}{0.121569,0.466667,0.705882}%
\pgfsetstrokecolor{currentstroke}%
\pgfsetdash{}{0pt}%
\pgfpathmoveto{\pgfqpoint{1.384091in}{0.595606in}}%
\pgfpathcurveto{\pgfqpoint{1.395141in}{0.595606in}}{\pgfqpoint{1.405740in}{0.599996in}}{\pgfqpoint{1.413554in}{0.607810in}}%
\pgfpathcurveto{\pgfqpoint{1.421367in}{0.615624in}}{\pgfqpoint{1.425758in}{0.626223in}}{\pgfqpoint{1.425758in}{0.637273in}}%
\pgfpathcurveto{\pgfqpoint{1.425758in}{0.648323in}}{\pgfqpoint{1.421367in}{0.658922in}}{\pgfqpoint{1.413554in}{0.666736in}}%
\pgfpathcurveto{\pgfqpoint{1.405740in}{0.674549in}}{\pgfqpoint{1.395141in}{0.678939in}}{\pgfqpoint{1.384091in}{0.678939in}}%
\pgfpathcurveto{\pgfqpoint{1.373041in}{0.678939in}}{\pgfqpoint{1.362442in}{0.674549in}}{\pgfqpoint{1.354628in}{0.666736in}}%
\pgfpathcurveto{\pgfqpoint{1.346815in}{0.658922in}}{\pgfqpoint{1.342424in}{0.648323in}}{\pgfqpoint{1.342424in}{0.637273in}}%
\pgfpathcurveto{\pgfqpoint{1.342424in}{0.626223in}}{\pgfqpoint{1.346815in}{0.615624in}}{\pgfqpoint{1.354628in}{0.607810in}}%
\pgfpathcurveto{\pgfqpoint{1.362442in}{0.599996in}}{\pgfqpoint{1.373041in}{0.595606in}}{\pgfqpoint{1.384091in}{0.595606in}}%
\pgfpathclose%
\pgfusepath{stroke,fill}%
\end{pgfscope}%
\begin{pgfscope}%
\pgfpathrectangle{\pgfqpoint{0.750000in}{0.500000in}}{\pgfqpoint{4.650000in}{3.020000in}}%
\pgfusepath{clip}%
\pgfsetbuttcap%
\pgfsetroundjoin%
\definecolor{currentfill}{rgb}{0.121569,0.466667,0.705882}%
\pgfsetfillcolor{currentfill}%
\pgfsetlinewidth{1.003750pt}%
\definecolor{currentstroke}{rgb}{0.121569,0.466667,0.705882}%
\pgfsetstrokecolor{currentstroke}%
\pgfsetdash{}{0pt}%
\pgfpathmoveto{\pgfqpoint{1.867208in}{0.665703in}}%
\pgfpathcurveto{\pgfqpoint{1.878258in}{0.665703in}}{\pgfqpoint{1.888857in}{0.670093in}}{\pgfqpoint{1.896671in}{0.677907in}}%
\pgfpathcurveto{\pgfqpoint{1.904484in}{0.685720in}}{\pgfqpoint{1.908874in}{0.696319in}}{\pgfqpoint{1.908874in}{0.707369in}}%
\pgfpathcurveto{\pgfqpoint{1.908874in}{0.718420in}}{\pgfqpoint{1.904484in}{0.729019in}}{\pgfqpoint{1.896671in}{0.736832in}}%
\pgfpathcurveto{\pgfqpoint{1.888857in}{0.744646in}}{\pgfqpoint{1.878258in}{0.749036in}}{\pgfqpoint{1.867208in}{0.749036in}}%
\pgfpathcurveto{\pgfqpoint{1.856158in}{0.749036in}}{\pgfqpoint{1.845559in}{0.744646in}}{\pgfqpoint{1.837745in}{0.736832in}}%
\pgfpathcurveto{\pgfqpoint{1.829931in}{0.729019in}}{\pgfqpoint{1.825541in}{0.718420in}}{\pgfqpoint{1.825541in}{0.707369in}}%
\pgfpathcurveto{\pgfqpoint{1.825541in}{0.696319in}}{\pgfqpoint{1.829931in}{0.685720in}}{\pgfqpoint{1.837745in}{0.677907in}}%
\pgfpathcurveto{\pgfqpoint{1.845559in}{0.670093in}}{\pgfqpoint{1.856158in}{0.665703in}}{\pgfqpoint{1.867208in}{0.665703in}}%
\pgfpathclose%
\pgfusepath{stroke,fill}%
\end{pgfscope}%
\begin{pgfscope}%
\pgfpathrectangle{\pgfqpoint{0.750000in}{0.500000in}}{\pgfqpoint{4.650000in}{3.020000in}}%
\pgfusepath{clip}%
\pgfsetbuttcap%
\pgfsetroundjoin%
\definecolor{currentfill}{rgb}{1.000000,0.498039,0.054902}%
\pgfsetfillcolor{currentfill}%
\pgfsetlinewidth{1.003750pt}%
\definecolor{currentstroke}{rgb}{1.000000,0.498039,0.054902}%
\pgfsetstrokecolor{currentstroke}%
\pgfsetdash{}{0pt}%
\pgfpathmoveto{\pgfqpoint{1.444481in}{2.932163in}}%
\pgfpathcurveto{\pgfqpoint{1.455531in}{2.932163in}}{\pgfqpoint{1.466130in}{2.936553in}}{\pgfqpoint{1.473943in}{2.944367in}}%
\pgfpathcurveto{\pgfqpoint{1.481757in}{2.952181in}}{\pgfqpoint{1.486147in}{2.962780in}}{\pgfqpoint{1.486147in}{2.973830in}}%
\pgfpathcurveto{\pgfqpoint{1.486147in}{2.984880in}}{\pgfqpoint{1.481757in}{2.995479in}}{\pgfqpoint{1.473943in}{3.003293in}}%
\pgfpathcurveto{\pgfqpoint{1.466130in}{3.011106in}}{\pgfqpoint{1.455531in}{3.015496in}}{\pgfqpoint{1.444481in}{3.015496in}}%
\pgfpathcurveto{\pgfqpoint{1.433430in}{3.015496in}}{\pgfqpoint{1.422831in}{3.011106in}}{\pgfqpoint{1.415018in}{3.003293in}}%
\pgfpathcurveto{\pgfqpoint{1.407204in}{2.995479in}}{\pgfqpoint{1.402814in}{2.984880in}}{\pgfqpoint{1.402814in}{2.973830in}}%
\pgfpathcurveto{\pgfqpoint{1.402814in}{2.962780in}}{\pgfqpoint{1.407204in}{2.952181in}}{\pgfqpoint{1.415018in}{2.944367in}}%
\pgfpathcurveto{\pgfqpoint{1.422831in}{2.936553in}}{\pgfqpoint{1.433430in}{2.932163in}}{\pgfqpoint{1.444481in}{2.932163in}}%
\pgfpathclose%
\pgfusepath{stroke,fill}%
\end{pgfscope}%
\begin{pgfscope}%
\pgfpathrectangle{\pgfqpoint{0.750000in}{0.500000in}}{\pgfqpoint{4.650000in}{3.020000in}}%
\pgfusepath{clip}%
\pgfsetbuttcap%
\pgfsetroundjoin%
\definecolor{currentfill}{rgb}{0.839216,0.152941,0.156863}%
\pgfsetfillcolor{currentfill}%
\pgfsetlinewidth{1.003750pt}%
\definecolor{currentstroke}{rgb}{0.839216,0.152941,0.156863}%
\pgfsetstrokecolor{currentstroke}%
\pgfsetdash{}{0pt}%
\pgfpathmoveto{\pgfqpoint{2.169156in}{1.086283in}}%
\pgfpathcurveto{\pgfqpoint{2.180206in}{1.086283in}}{\pgfqpoint{2.190805in}{1.090673in}}{\pgfqpoint{2.198619in}{1.098487in}}%
\pgfpathcurveto{\pgfqpoint{2.206432in}{1.106301in}}{\pgfqpoint{2.210823in}{1.116900in}}{\pgfqpoint{2.210823in}{1.127950in}}%
\pgfpathcurveto{\pgfqpoint{2.210823in}{1.139000in}}{\pgfqpoint{2.206432in}{1.149599in}}{\pgfqpoint{2.198619in}{1.157412in}}%
\pgfpathcurveto{\pgfqpoint{2.190805in}{1.165226in}}{\pgfqpoint{2.180206in}{1.169616in}}{\pgfqpoint{2.169156in}{1.169616in}}%
\pgfpathcurveto{\pgfqpoint{2.158106in}{1.169616in}}{\pgfqpoint{2.147507in}{1.165226in}}{\pgfqpoint{2.139693in}{1.157412in}}%
\pgfpathcurveto{\pgfqpoint{2.131879in}{1.149599in}}{\pgfqpoint{2.127489in}{1.139000in}}{\pgfqpoint{2.127489in}{1.127950in}}%
\pgfpathcurveto{\pgfqpoint{2.127489in}{1.116900in}}{\pgfqpoint{2.131879in}{1.106301in}}{\pgfqpoint{2.139693in}{1.098487in}}%
\pgfpathcurveto{\pgfqpoint{2.147507in}{1.090673in}}{\pgfqpoint{2.158106in}{1.086283in}}{\pgfqpoint{2.169156in}{1.086283in}}%
\pgfpathclose%
\pgfusepath{stroke,fill}%
\end{pgfscope}%
\begin{pgfscope}%
\pgfpathrectangle{\pgfqpoint{0.750000in}{0.500000in}}{\pgfqpoint{4.650000in}{3.020000in}}%
\pgfusepath{clip}%
\pgfsetbuttcap%
\pgfsetroundjoin%
\definecolor{currentfill}{rgb}{1.000000,0.498039,0.054902}%
\pgfsetfillcolor{currentfill}%
\pgfsetlinewidth{1.003750pt}%
\definecolor{currentstroke}{rgb}{1.000000,0.498039,0.054902}%
\pgfsetstrokecolor{currentstroke}%
\pgfsetdash{}{0pt}%
\pgfpathmoveto{\pgfqpoint{2.048377in}{2.932163in}}%
\pgfpathcurveto{\pgfqpoint{2.059427in}{2.932163in}}{\pgfqpoint{2.070026in}{2.936553in}}{\pgfqpoint{2.077839in}{2.944367in}}%
\pgfpathcurveto{\pgfqpoint{2.085653in}{2.952181in}}{\pgfqpoint{2.090043in}{2.962780in}}{\pgfqpoint{2.090043in}{2.973830in}}%
\pgfpathcurveto{\pgfqpoint{2.090043in}{2.984880in}}{\pgfqpoint{2.085653in}{2.995479in}}{\pgfqpoint{2.077839in}{3.003293in}}%
\pgfpathcurveto{\pgfqpoint{2.070026in}{3.011106in}}{\pgfqpoint{2.059427in}{3.015496in}}{\pgfqpoint{2.048377in}{3.015496in}}%
\pgfpathcurveto{\pgfqpoint{2.037326in}{3.015496in}}{\pgfqpoint{2.026727in}{3.011106in}}{\pgfqpoint{2.018914in}{3.003293in}}%
\pgfpathcurveto{\pgfqpoint{2.011100in}{2.995479in}}{\pgfqpoint{2.006710in}{2.984880in}}{\pgfqpoint{2.006710in}{2.973830in}}%
\pgfpathcurveto{\pgfqpoint{2.006710in}{2.962780in}}{\pgfqpoint{2.011100in}{2.952181in}}{\pgfqpoint{2.018914in}{2.944367in}}%
\pgfpathcurveto{\pgfqpoint{2.026727in}{2.936553in}}{\pgfqpoint{2.037326in}{2.932163in}}{\pgfqpoint{2.048377in}{2.932163in}}%
\pgfpathclose%
\pgfusepath{stroke,fill}%
\end{pgfscope}%
\begin{pgfscope}%
\pgfpathrectangle{\pgfqpoint{0.750000in}{0.500000in}}{\pgfqpoint{4.650000in}{3.020000in}}%
\pgfusepath{clip}%
\pgfsetbuttcap%
\pgfsetroundjoin%
\definecolor{currentfill}{rgb}{1.000000,0.498039,0.054902}%
\pgfsetfillcolor{currentfill}%
\pgfsetlinewidth{1.003750pt}%
\definecolor{currentstroke}{rgb}{1.000000,0.498039,0.054902}%
\pgfsetstrokecolor{currentstroke}%
\pgfsetdash{}{0pt}%
\pgfpathmoveto{\pgfqpoint{1.987987in}{2.932163in}}%
\pgfpathcurveto{\pgfqpoint{1.999037in}{2.932163in}}{\pgfqpoint{2.009636in}{2.936553in}}{\pgfqpoint{2.017450in}{2.944367in}}%
\pgfpathcurveto{\pgfqpoint{2.025263in}{2.952181in}}{\pgfqpoint{2.029654in}{2.962780in}}{\pgfqpoint{2.029654in}{2.973830in}}%
\pgfpathcurveto{\pgfqpoint{2.029654in}{2.984880in}}{\pgfqpoint{2.025263in}{2.995479in}}{\pgfqpoint{2.017450in}{3.003293in}}%
\pgfpathcurveto{\pgfqpoint{2.009636in}{3.011106in}}{\pgfqpoint{1.999037in}{3.015496in}}{\pgfqpoint{1.987987in}{3.015496in}}%
\pgfpathcurveto{\pgfqpoint{1.976937in}{3.015496in}}{\pgfqpoint{1.966338in}{3.011106in}}{\pgfqpoint{1.958524in}{3.003293in}}%
\pgfpathcurveto{\pgfqpoint{1.950711in}{2.995479in}}{\pgfqpoint{1.946320in}{2.984880in}}{\pgfqpoint{1.946320in}{2.973830in}}%
\pgfpathcurveto{\pgfqpoint{1.946320in}{2.962780in}}{\pgfqpoint{1.950711in}{2.952181in}}{\pgfqpoint{1.958524in}{2.944367in}}%
\pgfpathcurveto{\pgfqpoint{1.966338in}{2.936553in}}{\pgfqpoint{1.976937in}{2.932163in}}{\pgfqpoint{1.987987in}{2.932163in}}%
\pgfpathclose%
\pgfusepath{stroke,fill}%
\end{pgfscope}%
\begin{pgfscope}%
\pgfpathrectangle{\pgfqpoint{0.750000in}{0.500000in}}{\pgfqpoint{4.650000in}{3.020000in}}%
\pgfusepath{clip}%
\pgfsetbuttcap%
\pgfsetroundjoin%
\definecolor{currentfill}{rgb}{1.000000,0.498039,0.054902}%
\pgfsetfillcolor{currentfill}%
\pgfsetlinewidth{1.003750pt}%
\definecolor{currentstroke}{rgb}{1.000000,0.498039,0.054902}%
\pgfsetstrokecolor{currentstroke}%
\pgfsetdash{}{0pt}%
\pgfpathmoveto{\pgfqpoint{1.806818in}{2.928269in}}%
\pgfpathcurveto{\pgfqpoint{1.817868in}{2.928269in}}{\pgfqpoint{1.828467in}{2.932659in}}{\pgfqpoint{1.836281in}{2.940473in}}%
\pgfpathcurveto{\pgfqpoint{1.844095in}{2.948286in}}{\pgfqpoint{1.848485in}{2.958885in}}{\pgfqpoint{1.848485in}{2.969936in}}%
\pgfpathcurveto{\pgfqpoint{1.848485in}{2.980986in}}{\pgfqpoint{1.844095in}{2.991585in}}{\pgfqpoint{1.836281in}{2.999398in}}%
\pgfpathcurveto{\pgfqpoint{1.828467in}{3.007212in}}{\pgfqpoint{1.817868in}{3.011602in}}{\pgfqpoint{1.806818in}{3.011602in}}%
\pgfpathcurveto{\pgfqpoint{1.795768in}{3.011602in}}{\pgfqpoint{1.785169in}{3.007212in}}{\pgfqpoint{1.777355in}{2.999398in}}%
\pgfpathcurveto{\pgfqpoint{1.769542in}{2.991585in}}{\pgfqpoint{1.765152in}{2.980986in}}{\pgfqpoint{1.765152in}{2.969936in}}%
\pgfpathcurveto{\pgfqpoint{1.765152in}{2.958885in}}{\pgfqpoint{1.769542in}{2.948286in}}{\pgfqpoint{1.777355in}{2.940473in}}%
\pgfpathcurveto{\pgfqpoint{1.785169in}{2.932659in}}{\pgfqpoint{1.795768in}{2.928269in}}{\pgfqpoint{1.806818in}{2.928269in}}%
\pgfpathclose%
\pgfusepath{stroke,fill}%
\end{pgfscope}%
\begin{pgfscope}%
\pgfpathrectangle{\pgfqpoint{0.750000in}{0.500000in}}{\pgfqpoint{4.650000in}{3.020000in}}%
\pgfusepath{clip}%
\pgfsetbuttcap%
\pgfsetroundjoin%
\definecolor{currentfill}{rgb}{1.000000,0.498039,0.054902}%
\pgfsetfillcolor{currentfill}%
\pgfsetlinewidth{1.003750pt}%
\definecolor{currentstroke}{rgb}{1.000000,0.498039,0.054902}%
\pgfsetstrokecolor{currentstroke}%
\pgfsetdash{}{0pt}%
\pgfpathmoveto{\pgfqpoint{1.867208in}{2.936057in}}%
\pgfpathcurveto{\pgfqpoint{1.878258in}{2.936057in}}{\pgfqpoint{1.888857in}{2.940448in}}{\pgfqpoint{1.896671in}{2.948261in}}%
\pgfpathcurveto{\pgfqpoint{1.904484in}{2.956075in}}{\pgfqpoint{1.908874in}{2.966674in}}{\pgfqpoint{1.908874in}{2.977724in}}%
\pgfpathcurveto{\pgfqpoint{1.908874in}{2.988774in}}{\pgfqpoint{1.904484in}{2.999373in}}{\pgfqpoint{1.896671in}{3.007187in}}%
\pgfpathcurveto{\pgfqpoint{1.888857in}{3.015000in}}{\pgfqpoint{1.878258in}{3.019391in}}{\pgfqpoint{1.867208in}{3.019391in}}%
\pgfpathcurveto{\pgfqpoint{1.856158in}{3.019391in}}{\pgfqpoint{1.845559in}{3.015000in}}{\pgfqpoint{1.837745in}{3.007187in}}%
\pgfpathcurveto{\pgfqpoint{1.829931in}{2.999373in}}{\pgfqpoint{1.825541in}{2.988774in}}{\pgfqpoint{1.825541in}{2.977724in}}%
\pgfpathcurveto{\pgfqpoint{1.825541in}{2.966674in}}{\pgfqpoint{1.829931in}{2.956075in}}{\pgfqpoint{1.837745in}{2.948261in}}%
\pgfpathcurveto{\pgfqpoint{1.845559in}{2.940448in}}{\pgfqpoint{1.856158in}{2.936057in}}{\pgfqpoint{1.867208in}{2.936057in}}%
\pgfpathclose%
\pgfusepath{stroke,fill}%
\end{pgfscope}%
\begin{pgfscope}%
\pgfpathrectangle{\pgfqpoint{0.750000in}{0.500000in}}{\pgfqpoint{4.650000in}{3.020000in}}%
\pgfusepath{clip}%
\pgfsetbuttcap%
\pgfsetroundjoin%
\definecolor{currentfill}{rgb}{1.000000,0.498039,0.054902}%
\pgfsetfillcolor{currentfill}%
\pgfsetlinewidth{1.003750pt}%
\definecolor{currentstroke}{rgb}{1.000000,0.498039,0.054902}%
\pgfsetstrokecolor{currentstroke}%
\pgfsetdash{}{0pt}%
\pgfpathmoveto{\pgfqpoint{1.867208in}{2.939952in}}%
\pgfpathcurveto{\pgfqpoint{1.878258in}{2.939952in}}{\pgfqpoint{1.888857in}{2.944342in}}{\pgfqpoint{1.896671in}{2.952156in}}%
\pgfpathcurveto{\pgfqpoint{1.904484in}{2.959969in}}{\pgfqpoint{1.908874in}{2.970568in}}{\pgfqpoint{1.908874in}{2.981618in}}%
\pgfpathcurveto{\pgfqpoint{1.908874in}{2.992668in}}{\pgfqpoint{1.904484in}{3.003267in}}{\pgfqpoint{1.896671in}{3.011081in}}%
\pgfpathcurveto{\pgfqpoint{1.888857in}{3.018895in}}{\pgfqpoint{1.878258in}{3.023285in}}{\pgfqpoint{1.867208in}{3.023285in}}%
\pgfpathcurveto{\pgfqpoint{1.856158in}{3.023285in}}{\pgfqpoint{1.845559in}{3.018895in}}{\pgfqpoint{1.837745in}{3.011081in}}%
\pgfpathcurveto{\pgfqpoint{1.829931in}{3.003267in}}{\pgfqpoint{1.825541in}{2.992668in}}{\pgfqpoint{1.825541in}{2.981618in}}%
\pgfpathcurveto{\pgfqpoint{1.825541in}{2.970568in}}{\pgfqpoint{1.829931in}{2.959969in}}{\pgfqpoint{1.837745in}{2.952156in}}%
\pgfpathcurveto{\pgfqpoint{1.845559in}{2.944342in}}{\pgfqpoint{1.856158in}{2.939952in}}{\pgfqpoint{1.867208in}{2.939952in}}%
\pgfpathclose%
\pgfusepath{stroke,fill}%
\end{pgfscope}%
\begin{pgfscope}%
\pgfpathrectangle{\pgfqpoint{0.750000in}{0.500000in}}{\pgfqpoint{4.650000in}{3.020000in}}%
\pgfusepath{clip}%
\pgfsetbuttcap%
\pgfsetroundjoin%
\definecolor{currentfill}{rgb}{1.000000,0.498039,0.054902}%
\pgfsetfillcolor{currentfill}%
\pgfsetlinewidth{1.003750pt}%
\definecolor{currentstroke}{rgb}{1.000000,0.498039,0.054902}%
\pgfsetstrokecolor{currentstroke}%
\pgfsetdash{}{0pt}%
\pgfpathmoveto{\pgfqpoint{1.867208in}{2.936057in}}%
\pgfpathcurveto{\pgfqpoint{1.878258in}{2.936057in}}{\pgfqpoint{1.888857in}{2.940448in}}{\pgfqpoint{1.896671in}{2.948261in}}%
\pgfpathcurveto{\pgfqpoint{1.904484in}{2.956075in}}{\pgfqpoint{1.908874in}{2.966674in}}{\pgfqpoint{1.908874in}{2.977724in}}%
\pgfpathcurveto{\pgfqpoint{1.908874in}{2.988774in}}{\pgfqpoint{1.904484in}{2.999373in}}{\pgfqpoint{1.896671in}{3.007187in}}%
\pgfpathcurveto{\pgfqpoint{1.888857in}{3.015000in}}{\pgfqpoint{1.878258in}{3.019391in}}{\pgfqpoint{1.867208in}{3.019391in}}%
\pgfpathcurveto{\pgfqpoint{1.856158in}{3.019391in}}{\pgfqpoint{1.845559in}{3.015000in}}{\pgfqpoint{1.837745in}{3.007187in}}%
\pgfpathcurveto{\pgfqpoint{1.829931in}{2.999373in}}{\pgfqpoint{1.825541in}{2.988774in}}{\pgfqpoint{1.825541in}{2.977724in}}%
\pgfpathcurveto{\pgfqpoint{1.825541in}{2.966674in}}{\pgfqpoint{1.829931in}{2.956075in}}{\pgfqpoint{1.837745in}{2.948261in}}%
\pgfpathcurveto{\pgfqpoint{1.845559in}{2.940448in}}{\pgfqpoint{1.856158in}{2.936057in}}{\pgfqpoint{1.867208in}{2.936057in}}%
\pgfpathclose%
\pgfusepath{stroke,fill}%
\end{pgfscope}%
\begin{pgfscope}%
\pgfpathrectangle{\pgfqpoint{0.750000in}{0.500000in}}{\pgfqpoint{4.650000in}{3.020000in}}%
\pgfusepath{clip}%
\pgfsetbuttcap%
\pgfsetroundjoin%
\definecolor{currentfill}{rgb}{1.000000,0.498039,0.054902}%
\pgfsetfillcolor{currentfill}%
\pgfsetlinewidth{1.003750pt}%
\definecolor{currentstroke}{rgb}{1.000000,0.498039,0.054902}%
\pgfsetstrokecolor{currentstroke}%
\pgfsetdash{}{0pt}%
\pgfpathmoveto{\pgfqpoint{2.773052in}{2.936057in}}%
\pgfpathcurveto{\pgfqpoint{2.784102in}{2.936057in}}{\pgfqpoint{2.794701in}{2.940448in}}{\pgfqpoint{2.802515in}{2.948261in}}%
\pgfpathcurveto{\pgfqpoint{2.810328in}{2.956075in}}{\pgfqpoint{2.814719in}{2.966674in}}{\pgfqpoint{2.814719in}{2.977724in}}%
\pgfpathcurveto{\pgfqpoint{2.814719in}{2.988774in}}{\pgfqpoint{2.810328in}{2.999373in}}{\pgfqpoint{2.802515in}{3.007187in}}%
\pgfpathcurveto{\pgfqpoint{2.794701in}{3.015000in}}{\pgfqpoint{2.784102in}{3.019391in}}{\pgfqpoint{2.773052in}{3.019391in}}%
\pgfpathcurveto{\pgfqpoint{2.762002in}{3.019391in}}{\pgfqpoint{2.751403in}{3.015000in}}{\pgfqpoint{2.743589in}{3.007187in}}%
\pgfpathcurveto{\pgfqpoint{2.735776in}{2.999373in}}{\pgfqpoint{2.731385in}{2.988774in}}{\pgfqpoint{2.731385in}{2.977724in}}%
\pgfpathcurveto{\pgfqpoint{2.731385in}{2.966674in}}{\pgfqpoint{2.735776in}{2.956075in}}{\pgfqpoint{2.743589in}{2.948261in}}%
\pgfpathcurveto{\pgfqpoint{2.751403in}{2.940448in}}{\pgfqpoint{2.762002in}{2.936057in}}{\pgfqpoint{2.773052in}{2.936057in}}%
\pgfpathclose%
\pgfusepath{stroke,fill}%
\end{pgfscope}%
\begin{pgfscope}%
\pgfpathrectangle{\pgfqpoint{0.750000in}{0.500000in}}{\pgfqpoint{4.650000in}{3.020000in}}%
\pgfusepath{clip}%
\pgfsetbuttcap%
\pgfsetroundjoin%
\definecolor{currentfill}{rgb}{1.000000,0.498039,0.054902}%
\pgfsetfillcolor{currentfill}%
\pgfsetlinewidth{1.003750pt}%
\definecolor{currentstroke}{rgb}{1.000000,0.498039,0.054902}%
\pgfsetstrokecolor{currentstroke}%
\pgfsetdash{}{0pt}%
\pgfpathmoveto{\pgfqpoint{3.980844in}{2.924375in}}%
\pgfpathcurveto{\pgfqpoint{3.991894in}{2.924375in}}{\pgfqpoint{4.002493in}{2.928765in}}{\pgfqpoint{4.010307in}{2.936578in}}%
\pgfpathcurveto{\pgfqpoint{4.018121in}{2.944392in}}{\pgfqpoint{4.022511in}{2.954991in}}{\pgfqpoint{4.022511in}{2.966041in}}%
\pgfpathcurveto{\pgfqpoint{4.022511in}{2.977091in}}{\pgfqpoint{4.018121in}{2.987690in}}{\pgfqpoint{4.010307in}{2.995504in}}%
\pgfpathcurveto{\pgfqpoint{4.002493in}{3.003318in}}{\pgfqpoint{3.991894in}{3.007708in}}{\pgfqpoint{3.980844in}{3.007708in}}%
\pgfpathcurveto{\pgfqpoint{3.969794in}{3.007708in}}{\pgfqpoint{3.959195in}{3.003318in}}{\pgfqpoint{3.951381in}{2.995504in}}%
\pgfpathcurveto{\pgfqpoint{3.943568in}{2.987690in}}{\pgfqpoint{3.939177in}{2.977091in}}{\pgfqpoint{3.939177in}{2.966041in}}%
\pgfpathcurveto{\pgfqpoint{3.939177in}{2.954991in}}{\pgfqpoint{3.943568in}{2.944392in}}{\pgfqpoint{3.951381in}{2.936578in}}%
\pgfpathcurveto{\pgfqpoint{3.959195in}{2.928765in}}{\pgfqpoint{3.969794in}{2.924375in}}{\pgfqpoint{3.980844in}{2.924375in}}%
\pgfpathclose%
\pgfusepath{stroke,fill}%
\end{pgfscope}%
\begin{pgfscope}%
\pgfpathrectangle{\pgfqpoint{0.750000in}{0.500000in}}{\pgfqpoint{4.650000in}{3.020000in}}%
\pgfusepath{clip}%
\pgfsetbuttcap%
\pgfsetroundjoin%
\definecolor{currentfill}{rgb}{1.000000,0.498039,0.054902}%
\pgfsetfillcolor{currentfill}%
\pgfsetlinewidth{1.003750pt}%
\definecolor{currentstroke}{rgb}{1.000000,0.498039,0.054902}%
\pgfsetstrokecolor{currentstroke}%
\pgfsetdash{}{0pt}%
\pgfpathmoveto{\pgfqpoint{1.867208in}{2.932163in}}%
\pgfpathcurveto{\pgfqpoint{1.878258in}{2.932163in}}{\pgfqpoint{1.888857in}{2.936553in}}{\pgfqpoint{1.896671in}{2.944367in}}%
\pgfpathcurveto{\pgfqpoint{1.904484in}{2.952181in}}{\pgfqpoint{1.908874in}{2.962780in}}{\pgfqpoint{1.908874in}{2.973830in}}%
\pgfpathcurveto{\pgfqpoint{1.908874in}{2.984880in}}{\pgfqpoint{1.904484in}{2.995479in}}{\pgfqpoint{1.896671in}{3.003293in}}%
\pgfpathcurveto{\pgfqpoint{1.888857in}{3.011106in}}{\pgfqpoint{1.878258in}{3.015496in}}{\pgfqpoint{1.867208in}{3.015496in}}%
\pgfpathcurveto{\pgfqpoint{1.856158in}{3.015496in}}{\pgfqpoint{1.845559in}{3.011106in}}{\pgfqpoint{1.837745in}{3.003293in}}%
\pgfpathcurveto{\pgfqpoint{1.829931in}{2.995479in}}{\pgfqpoint{1.825541in}{2.984880in}}{\pgfqpoint{1.825541in}{2.973830in}}%
\pgfpathcurveto{\pgfqpoint{1.825541in}{2.962780in}}{\pgfqpoint{1.829931in}{2.952181in}}{\pgfqpoint{1.837745in}{2.944367in}}%
\pgfpathcurveto{\pgfqpoint{1.845559in}{2.936553in}}{\pgfqpoint{1.856158in}{2.932163in}}{\pgfqpoint{1.867208in}{2.932163in}}%
\pgfpathclose%
\pgfusepath{stroke,fill}%
\end{pgfscope}%
\begin{pgfscope}%
\pgfpathrectangle{\pgfqpoint{0.750000in}{0.500000in}}{\pgfqpoint{4.650000in}{3.020000in}}%
\pgfusepath{clip}%
\pgfsetbuttcap%
\pgfsetroundjoin%
\definecolor{currentfill}{rgb}{1.000000,0.498039,0.054902}%
\pgfsetfillcolor{currentfill}%
\pgfsetlinewidth{1.003750pt}%
\definecolor{currentstroke}{rgb}{1.000000,0.498039,0.054902}%
\pgfsetstrokecolor{currentstroke}%
\pgfsetdash{}{0pt}%
\pgfpathmoveto{\pgfqpoint{1.746429in}{2.936057in}}%
\pgfpathcurveto{\pgfqpoint{1.757479in}{2.936057in}}{\pgfqpoint{1.768078in}{2.940448in}}{\pgfqpoint{1.775891in}{2.948261in}}%
\pgfpathcurveto{\pgfqpoint{1.783705in}{2.956075in}}{\pgfqpoint{1.788095in}{2.966674in}}{\pgfqpoint{1.788095in}{2.977724in}}%
\pgfpathcurveto{\pgfqpoint{1.788095in}{2.988774in}}{\pgfqpoint{1.783705in}{2.999373in}}{\pgfqpoint{1.775891in}{3.007187in}}%
\pgfpathcurveto{\pgfqpoint{1.768078in}{3.015000in}}{\pgfqpoint{1.757479in}{3.019391in}}{\pgfqpoint{1.746429in}{3.019391in}}%
\pgfpathcurveto{\pgfqpoint{1.735378in}{3.019391in}}{\pgfqpoint{1.724779in}{3.015000in}}{\pgfqpoint{1.716966in}{3.007187in}}%
\pgfpathcurveto{\pgfqpoint{1.709152in}{2.999373in}}{\pgfqpoint{1.704762in}{2.988774in}}{\pgfqpoint{1.704762in}{2.977724in}}%
\pgfpathcurveto{\pgfqpoint{1.704762in}{2.966674in}}{\pgfqpoint{1.709152in}{2.956075in}}{\pgfqpoint{1.716966in}{2.948261in}}%
\pgfpathcurveto{\pgfqpoint{1.724779in}{2.940448in}}{\pgfqpoint{1.735378in}{2.936057in}}{\pgfqpoint{1.746429in}{2.936057in}}%
\pgfpathclose%
\pgfusepath{stroke,fill}%
\end{pgfscope}%
\begin{pgfscope}%
\pgfpathrectangle{\pgfqpoint{0.750000in}{0.500000in}}{\pgfqpoint{4.650000in}{3.020000in}}%
\pgfusepath{clip}%
\pgfsetbuttcap%
\pgfsetroundjoin%
\definecolor{currentfill}{rgb}{1.000000,0.498039,0.054902}%
\pgfsetfillcolor{currentfill}%
\pgfsetlinewidth{1.003750pt}%
\definecolor{currentstroke}{rgb}{1.000000,0.498039,0.054902}%
\pgfsetstrokecolor{currentstroke}%
\pgfsetdash{}{0pt}%
\pgfpathmoveto{\pgfqpoint{1.686039in}{2.928269in}}%
\pgfpathcurveto{\pgfqpoint{1.697089in}{2.928269in}}{\pgfqpoint{1.707688in}{2.932659in}}{\pgfqpoint{1.715502in}{2.940473in}}%
\pgfpathcurveto{\pgfqpoint{1.723315in}{2.948286in}}{\pgfqpoint{1.727706in}{2.958885in}}{\pgfqpoint{1.727706in}{2.969936in}}%
\pgfpathcurveto{\pgfqpoint{1.727706in}{2.980986in}}{\pgfqpoint{1.723315in}{2.991585in}}{\pgfqpoint{1.715502in}{2.999398in}}%
\pgfpathcurveto{\pgfqpoint{1.707688in}{3.007212in}}{\pgfqpoint{1.697089in}{3.011602in}}{\pgfqpoint{1.686039in}{3.011602in}}%
\pgfpathcurveto{\pgfqpoint{1.674989in}{3.011602in}}{\pgfqpoint{1.664390in}{3.007212in}}{\pgfqpoint{1.656576in}{2.999398in}}%
\pgfpathcurveto{\pgfqpoint{1.648763in}{2.991585in}}{\pgfqpoint{1.644372in}{2.980986in}}{\pgfqpoint{1.644372in}{2.969936in}}%
\pgfpathcurveto{\pgfqpoint{1.644372in}{2.958885in}}{\pgfqpoint{1.648763in}{2.948286in}}{\pgfqpoint{1.656576in}{2.940473in}}%
\pgfpathcurveto{\pgfqpoint{1.664390in}{2.932659in}}{\pgfqpoint{1.674989in}{2.928269in}}{\pgfqpoint{1.686039in}{2.928269in}}%
\pgfpathclose%
\pgfusepath{stroke,fill}%
\end{pgfscope}%
\begin{pgfscope}%
\pgfpathrectangle{\pgfqpoint{0.750000in}{0.500000in}}{\pgfqpoint{4.650000in}{3.020000in}}%
\pgfusepath{clip}%
\pgfsetbuttcap%
\pgfsetroundjoin%
\definecolor{currentfill}{rgb}{1.000000,0.498039,0.054902}%
\pgfsetfillcolor{currentfill}%
\pgfsetlinewidth{1.003750pt}%
\definecolor{currentstroke}{rgb}{1.000000,0.498039,0.054902}%
\pgfsetstrokecolor{currentstroke}%
\pgfsetdash{}{0pt}%
\pgfpathmoveto{\pgfqpoint{1.323701in}{2.943846in}}%
\pgfpathcurveto{\pgfqpoint{1.334751in}{2.943846in}}{\pgfqpoint{1.345350in}{2.948236in}}{\pgfqpoint{1.353164in}{2.956050in}}%
\pgfpathcurveto{\pgfqpoint{1.360978in}{2.963863in}}{\pgfqpoint{1.365368in}{2.974462in}}{\pgfqpoint{1.365368in}{2.985513in}}%
\pgfpathcurveto{\pgfqpoint{1.365368in}{2.996563in}}{\pgfqpoint{1.360978in}{3.007162in}}{\pgfqpoint{1.353164in}{3.014975in}}%
\pgfpathcurveto{\pgfqpoint{1.345350in}{3.022789in}}{\pgfqpoint{1.334751in}{3.027179in}}{\pgfqpoint{1.323701in}{3.027179in}}%
\pgfpathcurveto{\pgfqpoint{1.312651in}{3.027179in}}{\pgfqpoint{1.302052in}{3.022789in}}{\pgfqpoint{1.294239in}{3.014975in}}%
\pgfpathcurveto{\pgfqpoint{1.286425in}{3.007162in}}{\pgfqpoint{1.282035in}{2.996563in}}{\pgfqpoint{1.282035in}{2.985513in}}%
\pgfpathcurveto{\pgfqpoint{1.282035in}{2.974462in}}{\pgfqpoint{1.286425in}{2.963863in}}{\pgfqpoint{1.294239in}{2.956050in}}%
\pgfpathcurveto{\pgfqpoint{1.302052in}{2.948236in}}{\pgfqpoint{1.312651in}{2.943846in}}{\pgfqpoint{1.323701in}{2.943846in}}%
\pgfpathclose%
\pgfusepath{stroke,fill}%
\end{pgfscope}%
\begin{pgfscope}%
\pgfpathrectangle{\pgfqpoint{0.750000in}{0.500000in}}{\pgfqpoint{4.650000in}{3.020000in}}%
\pgfusepath{clip}%
\pgfsetbuttcap%
\pgfsetroundjoin%
\definecolor{currentfill}{rgb}{1.000000,0.498039,0.054902}%
\pgfsetfillcolor{currentfill}%
\pgfsetlinewidth{1.003750pt}%
\definecolor{currentstroke}{rgb}{1.000000,0.498039,0.054902}%
\pgfsetstrokecolor{currentstroke}%
\pgfsetdash{}{0pt}%
\pgfpathmoveto{\pgfqpoint{1.806818in}{2.928269in}}%
\pgfpathcurveto{\pgfqpoint{1.817868in}{2.928269in}}{\pgfqpoint{1.828467in}{2.932659in}}{\pgfqpoint{1.836281in}{2.940473in}}%
\pgfpathcurveto{\pgfqpoint{1.844095in}{2.948286in}}{\pgfqpoint{1.848485in}{2.958885in}}{\pgfqpoint{1.848485in}{2.969936in}}%
\pgfpathcurveto{\pgfqpoint{1.848485in}{2.980986in}}{\pgfqpoint{1.844095in}{2.991585in}}{\pgfqpoint{1.836281in}{2.999398in}}%
\pgfpathcurveto{\pgfqpoint{1.828467in}{3.007212in}}{\pgfqpoint{1.817868in}{3.011602in}}{\pgfqpoint{1.806818in}{3.011602in}}%
\pgfpathcurveto{\pgfqpoint{1.795768in}{3.011602in}}{\pgfqpoint{1.785169in}{3.007212in}}{\pgfqpoint{1.777355in}{2.999398in}}%
\pgfpathcurveto{\pgfqpoint{1.769542in}{2.991585in}}{\pgfqpoint{1.765152in}{2.980986in}}{\pgfqpoint{1.765152in}{2.969936in}}%
\pgfpathcurveto{\pgfqpoint{1.765152in}{2.958885in}}{\pgfqpoint{1.769542in}{2.948286in}}{\pgfqpoint{1.777355in}{2.940473in}}%
\pgfpathcurveto{\pgfqpoint{1.785169in}{2.932659in}}{\pgfqpoint{1.795768in}{2.928269in}}{\pgfqpoint{1.806818in}{2.928269in}}%
\pgfpathclose%
\pgfusepath{stroke,fill}%
\end{pgfscope}%
\begin{pgfscope}%
\pgfpathrectangle{\pgfqpoint{0.750000in}{0.500000in}}{\pgfqpoint{4.650000in}{3.020000in}}%
\pgfusepath{clip}%
\pgfsetbuttcap%
\pgfsetroundjoin%
\definecolor{currentfill}{rgb}{0.121569,0.466667,0.705882}%
\pgfsetfillcolor{currentfill}%
\pgfsetlinewidth{1.003750pt}%
\definecolor{currentstroke}{rgb}{0.121569,0.466667,0.705882}%
\pgfsetstrokecolor{currentstroke}%
\pgfsetdash{}{0pt}%
\pgfpathmoveto{\pgfqpoint{1.021753in}{0.595606in}}%
\pgfpathcurveto{\pgfqpoint{1.032803in}{0.595606in}}{\pgfqpoint{1.043402in}{0.599996in}}{\pgfqpoint{1.051216in}{0.607810in}}%
\pgfpathcurveto{\pgfqpoint{1.059030in}{0.615624in}}{\pgfqpoint{1.063420in}{0.626223in}}{\pgfqpoint{1.063420in}{0.637273in}}%
\pgfpathcurveto{\pgfqpoint{1.063420in}{0.648323in}}{\pgfqpoint{1.059030in}{0.658922in}}{\pgfqpoint{1.051216in}{0.666736in}}%
\pgfpathcurveto{\pgfqpoint{1.043402in}{0.674549in}}{\pgfqpoint{1.032803in}{0.678939in}}{\pgfqpoint{1.021753in}{0.678939in}}%
\pgfpathcurveto{\pgfqpoint{1.010703in}{0.678939in}}{\pgfqpoint{1.000104in}{0.674549in}}{\pgfqpoint{0.992290in}{0.666736in}}%
\pgfpathcurveto{\pgfqpoint{0.984477in}{0.658922in}}{\pgfqpoint{0.980087in}{0.648323in}}{\pgfqpoint{0.980087in}{0.637273in}}%
\pgfpathcurveto{\pgfqpoint{0.980087in}{0.626223in}}{\pgfqpoint{0.984477in}{0.615624in}}{\pgfqpoint{0.992290in}{0.607810in}}%
\pgfpathcurveto{\pgfqpoint{1.000104in}{0.599996in}}{\pgfqpoint{1.010703in}{0.595606in}}{\pgfqpoint{1.021753in}{0.595606in}}%
\pgfpathclose%
\pgfusepath{stroke,fill}%
\end{pgfscope}%
\begin{pgfscope}%
\pgfpathrectangle{\pgfqpoint{0.750000in}{0.500000in}}{\pgfqpoint{4.650000in}{3.020000in}}%
\pgfusepath{clip}%
\pgfsetbuttcap%
\pgfsetroundjoin%
\definecolor{currentfill}{rgb}{1.000000,0.498039,0.054902}%
\pgfsetfillcolor{currentfill}%
\pgfsetlinewidth{1.003750pt}%
\definecolor{currentstroke}{rgb}{1.000000,0.498039,0.054902}%
\pgfsetstrokecolor{currentstroke}%
\pgfsetdash{}{0pt}%
\pgfpathmoveto{\pgfqpoint{1.686039in}{2.928269in}}%
\pgfpathcurveto{\pgfqpoint{1.697089in}{2.928269in}}{\pgfqpoint{1.707688in}{2.932659in}}{\pgfqpoint{1.715502in}{2.940473in}}%
\pgfpathcurveto{\pgfqpoint{1.723315in}{2.948286in}}{\pgfqpoint{1.727706in}{2.958885in}}{\pgfqpoint{1.727706in}{2.969936in}}%
\pgfpathcurveto{\pgfqpoint{1.727706in}{2.980986in}}{\pgfqpoint{1.723315in}{2.991585in}}{\pgfqpoint{1.715502in}{2.999398in}}%
\pgfpathcurveto{\pgfqpoint{1.707688in}{3.007212in}}{\pgfqpoint{1.697089in}{3.011602in}}{\pgfqpoint{1.686039in}{3.011602in}}%
\pgfpathcurveto{\pgfqpoint{1.674989in}{3.011602in}}{\pgfqpoint{1.664390in}{3.007212in}}{\pgfqpoint{1.656576in}{2.999398in}}%
\pgfpathcurveto{\pgfqpoint{1.648763in}{2.991585in}}{\pgfqpoint{1.644372in}{2.980986in}}{\pgfqpoint{1.644372in}{2.969936in}}%
\pgfpathcurveto{\pgfqpoint{1.644372in}{2.958885in}}{\pgfqpoint{1.648763in}{2.948286in}}{\pgfqpoint{1.656576in}{2.940473in}}%
\pgfpathcurveto{\pgfqpoint{1.664390in}{2.932659in}}{\pgfqpoint{1.674989in}{2.928269in}}{\pgfqpoint{1.686039in}{2.928269in}}%
\pgfpathclose%
\pgfusepath{stroke,fill}%
\end{pgfscope}%
\begin{pgfscope}%
\pgfpathrectangle{\pgfqpoint{0.750000in}{0.500000in}}{\pgfqpoint{4.650000in}{3.020000in}}%
\pgfusepath{clip}%
\pgfsetbuttcap%
\pgfsetroundjoin%
\definecolor{currentfill}{rgb}{1.000000,0.498039,0.054902}%
\pgfsetfillcolor{currentfill}%
\pgfsetlinewidth{1.003750pt}%
\definecolor{currentstroke}{rgb}{1.000000,0.498039,0.054902}%
\pgfsetstrokecolor{currentstroke}%
\pgfsetdash{}{0pt}%
\pgfpathmoveto{\pgfqpoint{1.504870in}{2.928269in}}%
\pgfpathcurveto{\pgfqpoint{1.515920in}{2.928269in}}{\pgfqpoint{1.526519in}{2.932659in}}{\pgfqpoint{1.534333in}{2.940473in}}%
\pgfpathcurveto{\pgfqpoint{1.542147in}{2.948286in}}{\pgfqpoint{1.546537in}{2.958885in}}{\pgfqpoint{1.546537in}{2.969936in}}%
\pgfpathcurveto{\pgfqpoint{1.546537in}{2.980986in}}{\pgfqpoint{1.542147in}{2.991585in}}{\pgfqpoint{1.534333in}{2.999398in}}%
\pgfpathcurveto{\pgfqpoint{1.526519in}{3.007212in}}{\pgfqpoint{1.515920in}{3.011602in}}{\pgfqpoint{1.504870in}{3.011602in}}%
\pgfpathcurveto{\pgfqpoint{1.493820in}{3.011602in}}{\pgfqpoint{1.483221in}{3.007212in}}{\pgfqpoint{1.475407in}{2.999398in}}%
\pgfpathcurveto{\pgfqpoint{1.467594in}{2.991585in}}{\pgfqpoint{1.463203in}{2.980986in}}{\pgfqpoint{1.463203in}{2.969936in}}%
\pgfpathcurveto{\pgfqpoint{1.463203in}{2.958885in}}{\pgfqpoint{1.467594in}{2.948286in}}{\pgfqpoint{1.475407in}{2.940473in}}%
\pgfpathcurveto{\pgfqpoint{1.483221in}{2.932659in}}{\pgfqpoint{1.493820in}{2.928269in}}{\pgfqpoint{1.504870in}{2.928269in}}%
\pgfpathclose%
\pgfusepath{stroke,fill}%
\end{pgfscope}%
\begin{pgfscope}%
\pgfpathrectangle{\pgfqpoint{0.750000in}{0.500000in}}{\pgfqpoint{4.650000in}{3.020000in}}%
\pgfusepath{clip}%
\pgfsetbuttcap%
\pgfsetroundjoin%
\definecolor{currentfill}{rgb}{1.000000,0.498039,0.054902}%
\pgfsetfillcolor{currentfill}%
\pgfsetlinewidth{1.003750pt}%
\definecolor{currentstroke}{rgb}{1.000000,0.498039,0.054902}%
\pgfsetstrokecolor{currentstroke}%
\pgfsetdash{}{0pt}%
\pgfpathmoveto{\pgfqpoint{1.444481in}{2.936057in}}%
\pgfpathcurveto{\pgfqpoint{1.455531in}{2.936057in}}{\pgfqpoint{1.466130in}{2.940448in}}{\pgfqpoint{1.473943in}{2.948261in}}%
\pgfpathcurveto{\pgfqpoint{1.481757in}{2.956075in}}{\pgfqpoint{1.486147in}{2.966674in}}{\pgfqpoint{1.486147in}{2.977724in}}%
\pgfpathcurveto{\pgfqpoint{1.486147in}{2.988774in}}{\pgfqpoint{1.481757in}{2.999373in}}{\pgfqpoint{1.473943in}{3.007187in}}%
\pgfpathcurveto{\pgfqpoint{1.466130in}{3.015000in}}{\pgfqpoint{1.455531in}{3.019391in}}{\pgfqpoint{1.444481in}{3.019391in}}%
\pgfpathcurveto{\pgfqpoint{1.433430in}{3.019391in}}{\pgfqpoint{1.422831in}{3.015000in}}{\pgfqpoint{1.415018in}{3.007187in}}%
\pgfpathcurveto{\pgfqpoint{1.407204in}{2.999373in}}{\pgfqpoint{1.402814in}{2.988774in}}{\pgfqpoint{1.402814in}{2.977724in}}%
\pgfpathcurveto{\pgfqpoint{1.402814in}{2.966674in}}{\pgfqpoint{1.407204in}{2.956075in}}{\pgfqpoint{1.415018in}{2.948261in}}%
\pgfpathcurveto{\pgfqpoint{1.422831in}{2.940448in}}{\pgfqpoint{1.433430in}{2.936057in}}{\pgfqpoint{1.444481in}{2.936057in}}%
\pgfpathclose%
\pgfusepath{stroke,fill}%
\end{pgfscope}%
\begin{pgfscope}%
\pgfpathrectangle{\pgfqpoint{0.750000in}{0.500000in}}{\pgfqpoint{4.650000in}{3.020000in}}%
\pgfusepath{clip}%
\pgfsetbuttcap%
\pgfsetroundjoin%
\definecolor{currentfill}{rgb}{1.000000,0.498039,0.054902}%
\pgfsetfillcolor{currentfill}%
\pgfsetlinewidth{1.003750pt}%
\definecolor{currentstroke}{rgb}{1.000000,0.498039,0.054902}%
\pgfsetstrokecolor{currentstroke}%
\pgfsetdash{}{0pt}%
\pgfpathmoveto{\pgfqpoint{1.444481in}{3.045097in}}%
\pgfpathcurveto{\pgfqpoint{1.455531in}{3.045097in}}{\pgfqpoint{1.466130in}{3.049487in}}{\pgfqpoint{1.473943in}{3.057301in}}%
\pgfpathcurveto{\pgfqpoint{1.481757in}{3.065114in}}{\pgfqpoint{1.486147in}{3.075713in}}{\pgfqpoint{1.486147in}{3.086763in}}%
\pgfpathcurveto{\pgfqpoint{1.486147in}{3.097814in}}{\pgfqpoint{1.481757in}{3.108413in}}{\pgfqpoint{1.473943in}{3.116226in}}%
\pgfpathcurveto{\pgfqpoint{1.466130in}{3.124040in}}{\pgfqpoint{1.455531in}{3.128430in}}{\pgfqpoint{1.444481in}{3.128430in}}%
\pgfpathcurveto{\pgfqpoint{1.433430in}{3.128430in}}{\pgfqpoint{1.422831in}{3.124040in}}{\pgfqpoint{1.415018in}{3.116226in}}%
\pgfpathcurveto{\pgfqpoint{1.407204in}{3.108413in}}{\pgfqpoint{1.402814in}{3.097814in}}{\pgfqpoint{1.402814in}{3.086763in}}%
\pgfpathcurveto{\pgfqpoint{1.402814in}{3.075713in}}{\pgfqpoint{1.407204in}{3.065114in}}{\pgfqpoint{1.415018in}{3.057301in}}%
\pgfpathcurveto{\pgfqpoint{1.422831in}{3.049487in}}{\pgfqpoint{1.433430in}{3.045097in}}{\pgfqpoint{1.444481in}{3.045097in}}%
\pgfpathclose%
\pgfusepath{stroke,fill}%
\end{pgfscope}%
\begin{pgfscope}%
\pgfpathrectangle{\pgfqpoint{0.750000in}{0.500000in}}{\pgfqpoint{4.650000in}{3.020000in}}%
\pgfusepath{clip}%
\pgfsetbuttcap%
\pgfsetroundjoin%
\definecolor{currentfill}{rgb}{1.000000,0.498039,0.054902}%
\pgfsetfillcolor{currentfill}%
\pgfsetlinewidth{1.003750pt}%
\definecolor{currentstroke}{rgb}{1.000000,0.498039,0.054902}%
\pgfsetstrokecolor{currentstroke}%
\pgfsetdash{}{0pt}%
\pgfpathmoveto{\pgfqpoint{1.323701in}{2.936057in}}%
\pgfpathcurveto{\pgfqpoint{1.334751in}{2.936057in}}{\pgfqpoint{1.345350in}{2.940448in}}{\pgfqpoint{1.353164in}{2.948261in}}%
\pgfpathcurveto{\pgfqpoint{1.360978in}{2.956075in}}{\pgfqpoint{1.365368in}{2.966674in}}{\pgfqpoint{1.365368in}{2.977724in}}%
\pgfpathcurveto{\pgfqpoint{1.365368in}{2.988774in}}{\pgfqpoint{1.360978in}{2.999373in}}{\pgfqpoint{1.353164in}{3.007187in}}%
\pgfpathcurveto{\pgfqpoint{1.345350in}{3.015000in}}{\pgfqpoint{1.334751in}{3.019391in}}{\pgfqpoint{1.323701in}{3.019391in}}%
\pgfpathcurveto{\pgfqpoint{1.312651in}{3.019391in}}{\pgfqpoint{1.302052in}{3.015000in}}{\pgfqpoint{1.294239in}{3.007187in}}%
\pgfpathcurveto{\pgfqpoint{1.286425in}{2.999373in}}{\pgfqpoint{1.282035in}{2.988774in}}{\pgfqpoint{1.282035in}{2.977724in}}%
\pgfpathcurveto{\pgfqpoint{1.282035in}{2.966674in}}{\pgfqpoint{1.286425in}{2.956075in}}{\pgfqpoint{1.294239in}{2.948261in}}%
\pgfpathcurveto{\pgfqpoint{1.302052in}{2.940448in}}{\pgfqpoint{1.312651in}{2.936057in}}{\pgfqpoint{1.323701in}{2.936057in}}%
\pgfpathclose%
\pgfusepath{stroke,fill}%
\end{pgfscope}%
\begin{pgfscope}%
\pgfpathrectangle{\pgfqpoint{0.750000in}{0.500000in}}{\pgfqpoint{4.650000in}{3.020000in}}%
\pgfusepath{clip}%
\pgfsetbuttcap%
\pgfsetroundjoin%
\definecolor{currentfill}{rgb}{1.000000,0.498039,0.054902}%
\pgfsetfillcolor{currentfill}%
\pgfsetlinewidth{1.003750pt}%
\definecolor{currentstroke}{rgb}{1.000000,0.498039,0.054902}%
\pgfsetstrokecolor{currentstroke}%
\pgfsetdash{}{0pt}%
\pgfpathmoveto{\pgfqpoint{2.229545in}{2.936057in}}%
\pgfpathcurveto{\pgfqpoint{2.240596in}{2.936057in}}{\pgfqpoint{2.251195in}{2.940448in}}{\pgfqpoint{2.259008in}{2.948261in}}%
\pgfpathcurveto{\pgfqpoint{2.266822in}{2.956075in}}{\pgfqpoint{2.271212in}{2.966674in}}{\pgfqpoint{2.271212in}{2.977724in}}%
\pgfpathcurveto{\pgfqpoint{2.271212in}{2.988774in}}{\pgfqpoint{2.266822in}{2.999373in}}{\pgfqpoint{2.259008in}{3.007187in}}%
\pgfpathcurveto{\pgfqpoint{2.251195in}{3.015000in}}{\pgfqpoint{2.240596in}{3.019391in}}{\pgfqpoint{2.229545in}{3.019391in}}%
\pgfpathcurveto{\pgfqpoint{2.218495in}{3.019391in}}{\pgfqpoint{2.207896in}{3.015000in}}{\pgfqpoint{2.200083in}{3.007187in}}%
\pgfpathcurveto{\pgfqpoint{2.192269in}{2.999373in}}{\pgfqpoint{2.187879in}{2.988774in}}{\pgfqpoint{2.187879in}{2.977724in}}%
\pgfpathcurveto{\pgfqpoint{2.187879in}{2.966674in}}{\pgfqpoint{2.192269in}{2.956075in}}{\pgfqpoint{2.200083in}{2.948261in}}%
\pgfpathcurveto{\pgfqpoint{2.207896in}{2.940448in}}{\pgfqpoint{2.218495in}{2.936057in}}{\pgfqpoint{2.229545in}{2.936057in}}%
\pgfpathclose%
\pgfusepath{stroke,fill}%
\end{pgfscope}%
\begin{pgfscope}%
\pgfpathrectangle{\pgfqpoint{0.750000in}{0.500000in}}{\pgfqpoint{4.650000in}{3.020000in}}%
\pgfusepath{clip}%
\pgfsetbuttcap%
\pgfsetroundjoin%
\definecolor{currentfill}{rgb}{1.000000,0.498039,0.054902}%
\pgfsetfillcolor{currentfill}%
\pgfsetlinewidth{1.003750pt}%
\definecolor{currentstroke}{rgb}{1.000000,0.498039,0.054902}%
\pgfsetstrokecolor{currentstroke}%
\pgfsetdash{}{0pt}%
\pgfpathmoveto{\pgfqpoint{1.686039in}{3.185290in}}%
\pgfpathcurveto{\pgfqpoint{1.697089in}{3.185290in}}{\pgfqpoint{1.707688in}{3.189680in}}{\pgfqpoint{1.715502in}{3.197494in}}%
\pgfpathcurveto{\pgfqpoint{1.723315in}{3.205308in}}{\pgfqpoint{1.727706in}{3.215907in}}{\pgfqpoint{1.727706in}{3.226957in}}%
\pgfpathcurveto{\pgfqpoint{1.727706in}{3.238007in}}{\pgfqpoint{1.723315in}{3.248606in}}{\pgfqpoint{1.715502in}{3.256420in}}%
\pgfpathcurveto{\pgfqpoint{1.707688in}{3.264233in}}{\pgfqpoint{1.697089in}{3.268623in}}{\pgfqpoint{1.686039in}{3.268623in}}%
\pgfpathcurveto{\pgfqpoint{1.674989in}{3.268623in}}{\pgfqpoint{1.664390in}{3.264233in}}{\pgfqpoint{1.656576in}{3.256420in}}%
\pgfpathcurveto{\pgfqpoint{1.648763in}{3.248606in}}{\pgfqpoint{1.644372in}{3.238007in}}{\pgfqpoint{1.644372in}{3.226957in}}%
\pgfpathcurveto{\pgfqpoint{1.644372in}{3.215907in}}{\pgfqpoint{1.648763in}{3.205308in}}{\pgfqpoint{1.656576in}{3.197494in}}%
\pgfpathcurveto{\pgfqpoint{1.664390in}{3.189680in}}{\pgfqpoint{1.674989in}{3.185290in}}{\pgfqpoint{1.686039in}{3.185290in}}%
\pgfpathclose%
\pgfusepath{stroke,fill}%
\end{pgfscope}%
\begin{pgfscope}%
\pgfpathrectangle{\pgfqpoint{0.750000in}{0.500000in}}{\pgfqpoint{4.650000in}{3.020000in}}%
\pgfusepath{clip}%
\pgfsetbuttcap%
\pgfsetroundjoin%
\definecolor{currentfill}{rgb}{1.000000,0.498039,0.054902}%
\pgfsetfillcolor{currentfill}%
\pgfsetlinewidth{1.003750pt}%
\definecolor{currentstroke}{rgb}{1.000000,0.498039,0.054902}%
\pgfsetstrokecolor{currentstroke}%
\pgfsetdash{}{0pt}%
\pgfpathmoveto{\pgfqpoint{1.927597in}{2.838701in}}%
\pgfpathcurveto{\pgfqpoint{1.938648in}{2.838701in}}{\pgfqpoint{1.949247in}{2.843091in}}{\pgfqpoint{1.957060in}{2.850905in}}%
\pgfpathcurveto{\pgfqpoint{1.964874in}{2.858718in}}{\pgfqpoint{1.969264in}{2.869317in}}{\pgfqpoint{1.969264in}{2.880368in}}%
\pgfpathcurveto{\pgfqpoint{1.969264in}{2.891418in}}{\pgfqpoint{1.964874in}{2.902017in}}{\pgfqpoint{1.957060in}{2.909830in}}%
\pgfpathcurveto{\pgfqpoint{1.949247in}{2.917644in}}{\pgfqpoint{1.938648in}{2.922034in}}{\pgfqpoint{1.927597in}{2.922034in}}%
\pgfpathcurveto{\pgfqpoint{1.916547in}{2.922034in}}{\pgfqpoint{1.905948in}{2.917644in}}{\pgfqpoint{1.898135in}{2.909830in}}%
\pgfpathcurveto{\pgfqpoint{1.890321in}{2.902017in}}{\pgfqpoint{1.885931in}{2.891418in}}{\pgfqpoint{1.885931in}{2.880368in}}%
\pgfpathcurveto{\pgfqpoint{1.885931in}{2.869317in}}{\pgfqpoint{1.890321in}{2.858718in}}{\pgfqpoint{1.898135in}{2.850905in}}%
\pgfpathcurveto{\pgfqpoint{1.905948in}{2.843091in}}{\pgfqpoint{1.916547in}{2.838701in}}{\pgfqpoint{1.927597in}{2.838701in}}%
\pgfpathclose%
\pgfusepath{stroke,fill}%
\end{pgfscope}%
\begin{pgfscope}%
\pgfpathrectangle{\pgfqpoint{0.750000in}{0.500000in}}{\pgfqpoint{4.650000in}{3.020000in}}%
\pgfusepath{clip}%
\pgfsetbuttcap%
\pgfsetroundjoin%
\definecolor{currentfill}{rgb}{1.000000,0.498039,0.054902}%
\pgfsetfillcolor{currentfill}%
\pgfsetlinewidth{1.003750pt}%
\definecolor{currentstroke}{rgb}{1.000000,0.498039,0.054902}%
\pgfsetstrokecolor{currentstroke}%
\pgfsetdash{}{0pt}%
\pgfpathmoveto{\pgfqpoint{1.987987in}{2.978894in}}%
\pgfpathcurveto{\pgfqpoint{1.999037in}{2.978894in}}{\pgfqpoint{2.009636in}{2.983285in}}{\pgfqpoint{2.017450in}{2.991098in}}%
\pgfpathcurveto{\pgfqpoint{2.025263in}{2.998912in}}{\pgfqpoint{2.029654in}{3.009511in}}{\pgfqpoint{2.029654in}{3.020561in}}%
\pgfpathcurveto{\pgfqpoint{2.029654in}{3.031611in}}{\pgfqpoint{2.025263in}{3.042210in}}{\pgfqpoint{2.017450in}{3.050024in}}%
\pgfpathcurveto{\pgfqpoint{2.009636in}{3.057837in}}{\pgfqpoint{1.999037in}{3.062228in}}{\pgfqpoint{1.987987in}{3.062228in}}%
\pgfpathcurveto{\pgfqpoint{1.976937in}{3.062228in}}{\pgfqpoint{1.966338in}{3.057837in}}{\pgfqpoint{1.958524in}{3.050024in}}%
\pgfpathcurveto{\pgfqpoint{1.950711in}{3.042210in}}{\pgfqpoint{1.946320in}{3.031611in}}{\pgfqpoint{1.946320in}{3.020561in}}%
\pgfpathcurveto{\pgfqpoint{1.946320in}{3.009511in}}{\pgfqpoint{1.950711in}{2.998912in}}{\pgfqpoint{1.958524in}{2.991098in}}%
\pgfpathcurveto{\pgfqpoint{1.966338in}{2.983285in}}{\pgfqpoint{1.976937in}{2.978894in}}{\pgfqpoint{1.987987in}{2.978894in}}%
\pgfpathclose%
\pgfusepath{stroke,fill}%
\end{pgfscope}%
\begin{pgfscope}%
\pgfpathrectangle{\pgfqpoint{0.750000in}{0.500000in}}{\pgfqpoint{4.650000in}{3.020000in}}%
\pgfusepath{clip}%
\pgfsetbuttcap%
\pgfsetroundjoin%
\definecolor{currentfill}{rgb}{1.000000,0.498039,0.054902}%
\pgfsetfillcolor{currentfill}%
\pgfsetlinewidth{1.003750pt}%
\definecolor{currentstroke}{rgb}{1.000000,0.498039,0.054902}%
\pgfsetstrokecolor{currentstroke}%
\pgfsetdash{}{0pt}%
\pgfpathmoveto{\pgfqpoint{1.686039in}{2.932163in}}%
\pgfpathcurveto{\pgfqpoint{1.697089in}{2.932163in}}{\pgfqpoint{1.707688in}{2.936553in}}{\pgfqpoint{1.715502in}{2.944367in}}%
\pgfpathcurveto{\pgfqpoint{1.723315in}{2.952181in}}{\pgfqpoint{1.727706in}{2.962780in}}{\pgfqpoint{1.727706in}{2.973830in}}%
\pgfpathcurveto{\pgfqpoint{1.727706in}{2.984880in}}{\pgfqpoint{1.723315in}{2.995479in}}{\pgfqpoint{1.715502in}{3.003293in}}%
\pgfpathcurveto{\pgfqpoint{1.707688in}{3.011106in}}{\pgfqpoint{1.697089in}{3.015496in}}{\pgfqpoint{1.686039in}{3.015496in}}%
\pgfpathcurveto{\pgfqpoint{1.674989in}{3.015496in}}{\pgfqpoint{1.664390in}{3.011106in}}{\pgfqpoint{1.656576in}{3.003293in}}%
\pgfpathcurveto{\pgfqpoint{1.648763in}{2.995479in}}{\pgfqpoint{1.644372in}{2.984880in}}{\pgfqpoint{1.644372in}{2.973830in}}%
\pgfpathcurveto{\pgfqpoint{1.644372in}{2.962780in}}{\pgfqpoint{1.648763in}{2.952181in}}{\pgfqpoint{1.656576in}{2.944367in}}%
\pgfpathcurveto{\pgfqpoint{1.664390in}{2.936553in}}{\pgfqpoint{1.674989in}{2.932163in}}{\pgfqpoint{1.686039in}{2.932163in}}%
\pgfpathclose%
\pgfusepath{stroke,fill}%
\end{pgfscope}%
\begin{pgfscope}%
\pgfpathrectangle{\pgfqpoint{0.750000in}{0.500000in}}{\pgfqpoint{4.650000in}{3.020000in}}%
\pgfusepath{clip}%
\pgfsetbuttcap%
\pgfsetroundjoin%
\definecolor{currentfill}{rgb}{1.000000,0.498039,0.054902}%
\pgfsetfillcolor{currentfill}%
\pgfsetlinewidth{1.003750pt}%
\definecolor{currentstroke}{rgb}{1.000000,0.498039,0.054902}%
\pgfsetstrokecolor{currentstroke}%
\pgfsetdash{}{0pt}%
\pgfpathmoveto{\pgfqpoint{1.565260in}{2.932163in}}%
\pgfpathcurveto{\pgfqpoint{1.576310in}{2.932163in}}{\pgfqpoint{1.586909in}{2.936553in}}{\pgfqpoint{1.594723in}{2.944367in}}%
\pgfpathcurveto{\pgfqpoint{1.602536in}{2.952181in}}{\pgfqpoint{1.606926in}{2.962780in}}{\pgfqpoint{1.606926in}{2.973830in}}%
\pgfpathcurveto{\pgfqpoint{1.606926in}{2.984880in}}{\pgfqpoint{1.602536in}{2.995479in}}{\pgfqpoint{1.594723in}{3.003293in}}%
\pgfpathcurveto{\pgfqpoint{1.586909in}{3.011106in}}{\pgfqpoint{1.576310in}{3.015496in}}{\pgfqpoint{1.565260in}{3.015496in}}%
\pgfpathcurveto{\pgfqpoint{1.554210in}{3.015496in}}{\pgfqpoint{1.543611in}{3.011106in}}{\pgfqpoint{1.535797in}{3.003293in}}%
\pgfpathcurveto{\pgfqpoint{1.527983in}{2.995479in}}{\pgfqpoint{1.523593in}{2.984880in}}{\pgfqpoint{1.523593in}{2.973830in}}%
\pgfpathcurveto{\pgfqpoint{1.523593in}{2.962780in}}{\pgfqpoint{1.527983in}{2.952181in}}{\pgfqpoint{1.535797in}{2.944367in}}%
\pgfpathcurveto{\pgfqpoint{1.543611in}{2.936553in}}{\pgfqpoint{1.554210in}{2.932163in}}{\pgfqpoint{1.565260in}{2.932163in}}%
\pgfpathclose%
\pgfusepath{stroke,fill}%
\end{pgfscope}%
\begin{pgfscope}%
\pgfpathrectangle{\pgfqpoint{0.750000in}{0.500000in}}{\pgfqpoint{4.650000in}{3.020000in}}%
\pgfusepath{clip}%
\pgfsetbuttcap%
\pgfsetroundjoin%
\definecolor{currentfill}{rgb}{1.000000,0.498039,0.054902}%
\pgfsetfillcolor{currentfill}%
\pgfsetlinewidth{1.003750pt}%
\definecolor{currentstroke}{rgb}{1.000000,0.498039,0.054902}%
\pgfsetstrokecolor{currentstroke}%
\pgfsetdash{}{0pt}%
\pgfpathmoveto{\pgfqpoint{2.591883in}{2.928269in}}%
\pgfpathcurveto{\pgfqpoint{2.602933in}{2.928269in}}{\pgfqpoint{2.613532in}{2.932659in}}{\pgfqpoint{2.621346in}{2.940473in}}%
\pgfpathcurveto{\pgfqpoint{2.629160in}{2.948286in}}{\pgfqpoint{2.633550in}{2.958885in}}{\pgfqpoint{2.633550in}{2.969936in}}%
\pgfpathcurveto{\pgfqpoint{2.633550in}{2.980986in}}{\pgfqpoint{2.629160in}{2.991585in}}{\pgfqpoint{2.621346in}{2.999398in}}%
\pgfpathcurveto{\pgfqpoint{2.613532in}{3.007212in}}{\pgfqpoint{2.602933in}{3.011602in}}{\pgfqpoint{2.591883in}{3.011602in}}%
\pgfpathcurveto{\pgfqpoint{2.580833in}{3.011602in}}{\pgfqpoint{2.570234in}{3.007212in}}{\pgfqpoint{2.562420in}{2.999398in}}%
\pgfpathcurveto{\pgfqpoint{2.554607in}{2.991585in}}{\pgfqpoint{2.550216in}{2.980986in}}{\pgfqpoint{2.550216in}{2.969936in}}%
\pgfpathcurveto{\pgfqpoint{2.550216in}{2.958885in}}{\pgfqpoint{2.554607in}{2.948286in}}{\pgfqpoint{2.562420in}{2.940473in}}%
\pgfpathcurveto{\pgfqpoint{2.570234in}{2.932659in}}{\pgfqpoint{2.580833in}{2.928269in}}{\pgfqpoint{2.591883in}{2.928269in}}%
\pgfpathclose%
\pgfusepath{stroke,fill}%
\end{pgfscope}%
\begin{pgfscope}%
\pgfpathrectangle{\pgfqpoint{0.750000in}{0.500000in}}{\pgfqpoint{4.650000in}{3.020000in}}%
\pgfusepath{clip}%
\pgfsetbuttcap%
\pgfsetroundjoin%
\definecolor{currentfill}{rgb}{0.121569,0.466667,0.705882}%
\pgfsetfillcolor{currentfill}%
\pgfsetlinewidth{1.003750pt}%
\definecolor{currentstroke}{rgb}{0.121569,0.466667,0.705882}%
\pgfsetstrokecolor{currentstroke}%
\pgfsetdash{}{0pt}%
\pgfpathmoveto{\pgfqpoint{0.961364in}{0.595606in}}%
\pgfpathcurveto{\pgfqpoint{0.972414in}{0.595606in}}{\pgfqpoint{0.983013in}{0.599996in}}{\pgfqpoint{0.990826in}{0.607810in}}%
\pgfpathcurveto{\pgfqpoint{0.998640in}{0.615624in}}{\pgfqpoint{1.003030in}{0.626223in}}{\pgfqpoint{1.003030in}{0.637273in}}%
\pgfpathcurveto{\pgfqpoint{1.003030in}{0.648323in}}{\pgfqpoint{0.998640in}{0.658922in}}{\pgfqpoint{0.990826in}{0.666736in}}%
\pgfpathcurveto{\pgfqpoint{0.983013in}{0.674549in}}{\pgfqpoint{0.972414in}{0.678939in}}{\pgfqpoint{0.961364in}{0.678939in}}%
\pgfpathcurveto{\pgfqpoint{0.950314in}{0.678939in}}{\pgfqpoint{0.939714in}{0.674549in}}{\pgfqpoint{0.931901in}{0.666736in}}%
\pgfpathcurveto{\pgfqpoint{0.924087in}{0.658922in}}{\pgfqpoint{0.919697in}{0.648323in}}{\pgfqpoint{0.919697in}{0.637273in}}%
\pgfpathcurveto{\pgfqpoint{0.919697in}{0.626223in}}{\pgfqpoint{0.924087in}{0.615624in}}{\pgfqpoint{0.931901in}{0.607810in}}%
\pgfpathcurveto{\pgfqpoint{0.939714in}{0.599996in}}{\pgfqpoint{0.950314in}{0.595606in}}{\pgfqpoint{0.961364in}{0.595606in}}%
\pgfpathclose%
\pgfusepath{stroke,fill}%
\end{pgfscope}%
\begin{pgfscope}%
\pgfpathrectangle{\pgfqpoint{0.750000in}{0.500000in}}{\pgfqpoint{4.650000in}{3.020000in}}%
\pgfusepath{clip}%
\pgfsetbuttcap%
\pgfsetroundjoin%
\definecolor{currentfill}{rgb}{0.121569,0.466667,0.705882}%
\pgfsetfillcolor{currentfill}%
\pgfsetlinewidth{1.003750pt}%
\definecolor{currentstroke}{rgb}{0.121569,0.466667,0.705882}%
\pgfsetstrokecolor{currentstroke}%
\pgfsetdash{}{0pt}%
\pgfpathmoveto{\pgfqpoint{1.202922in}{0.599500in}}%
\pgfpathcurveto{\pgfqpoint{1.213972in}{0.599500in}}{\pgfqpoint{1.224571in}{0.603891in}}{\pgfqpoint{1.232385in}{0.611704in}}%
\pgfpathcurveto{\pgfqpoint{1.240198in}{0.619518in}}{\pgfqpoint{1.244589in}{0.630117in}}{\pgfqpoint{1.244589in}{0.641167in}}%
\pgfpathcurveto{\pgfqpoint{1.244589in}{0.652217in}}{\pgfqpoint{1.240198in}{0.662816in}}{\pgfqpoint{1.232385in}{0.670630in}}%
\pgfpathcurveto{\pgfqpoint{1.224571in}{0.678443in}}{\pgfqpoint{1.213972in}{0.682834in}}{\pgfqpoint{1.202922in}{0.682834in}}%
\pgfpathcurveto{\pgfqpoint{1.191872in}{0.682834in}}{\pgfqpoint{1.181273in}{0.678443in}}{\pgfqpoint{1.173459in}{0.670630in}}%
\pgfpathcurveto{\pgfqpoint{1.165646in}{0.662816in}}{\pgfqpoint{1.161255in}{0.652217in}}{\pgfqpoint{1.161255in}{0.641167in}}%
\pgfpathcurveto{\pgfqpoint{1.161255in}{0.630117in}}{\pgfqpoint{1.165646in}{0.619518in}}{\pgfqpoint{1.173459in}{0.611704in}}%
\pgfpathcurveto{\pgfqpoint{1.181273in}{0.603891in}}{\pgfqpoint{1.191872in}{0.599500in}}{\pgfqpoint{1.202922in}{0.599500in}}%
\pgfpathclose%
\pgfusepath{stroke,fill}%
\end{pgfscope}%
\begin{pgfscope}%
\pgfpathrectangle{\pgfqpoint{0.750000in}{0.500000in}}{\pgfqpoint{4.650000in}{3.020000in}}%
\pgfusepath{clip}%
\pgfsetbuttcap%
\pgfsetroundjoin%
\definecolor{currentfill}{rgb}{1.000000,0.498039,0.054902}%
\pgfsetfillcolor{currentfill}%
\pgfsetlinewidth{1.003750pt}%
\definecolor{currentstroke}{rgb}{1.000000,0.498039,0.054902}%
\pgfsetstrokecolor{currentstroke}%
\pgfsetdash{}{0pt}%
\pgfpathmoveto{\pgfqpoint{1.504870in}{2.904903in}}%
\pgfpathcurveto{\pgfqpoint{1.515920in}{2.904903in}}{\pgfqpoint{1.526519in}{2.909294in}}{\pgfqpoint{1.534333in}{2.917107in}}%
\pgfpathcurveto{\pgfqpoint{1.542147in}{2.924921in}}{\pgfqpoint{1.546537in}{2.935520in}}{\pgfqpoint{1.546537in}{2.946570in}}%
\pgfpathcurveto{\pgfqpoint{1.546537in}{2.957620in}}{\pgfqpoint{1.542147in}{2.968219in}}{\pgfqpoint{1.534333in}{2.976033in}}%
\pgfpathcurveto{\pgfqpoint{1.526519in}{2.983846in}}{\pgfqpoint{1.515920in}{2.988237in}}{\pgfqpoint{1.504870in}{2.988237in}}%
\pgfpathcurveto{\pgfqpoint{1.493820in}{2.988237in}}{\pgfqpoint{1.483221in}{2.983846in}}{\pgfqpoint{1.475407in}{2.976033in}}%
\pgfpathcurveto{\pgfqpoint{1.467594in}{2.968219in}}{\pgfqpoint{1.463203in}{2.957620in}}{\pgfqpoint{1.463203in}{2.946570in}}%
\pgfpathcurveto{\pgfqpoint{1.463203in}{2.935520in}}{\pgfqpoint{1.467594in}{2.924921in}}{\pgfqpoint{1.475407in}{2.917107in}}%
\pgfpathcurveto{\pgfqpoint{1.483221in}{2.909294in}}{\pgfqpoint{1.493820in}{2.904903in}}{\pgfqpoint{1.504870in}{2.904903in}}%
\pgfpathclose%
\pgfusepath{stroke,fill}%
\end{pgfscope}%
\begin{pgfscope}%
\pgfpathrectangle{\pgfqpoint{0.750000in}{0.500000in}}{\pgfqpoint{4.650000in}{3.020000in}}%
\pgfusepath{clip}%
\pgfsetbuttcap%
\pgfsetroundjoin%
\definecolor{currentfill}{rgb}{1.000000,0.498039,0.054902}%
\pgfsetfillcolor{currentfill}%
\pgfsetlinewidth{1.003750pt}%
\definecolor{currentstroke}{rgb}{1.000000,0.498039,0.054902}%
\pgfsetstrokecolor{currentstroke}%
\pgfsetdash{}{0pt}%
\pgfpathmoveto{\pgfqpoint{1.323701in}{2.994471in}}%
\pgfpathcurveto{\pgfqpoint{1.334751in}{2.994471in}}{\pgfqpoint{1.345350in}{2.998862in}}{\pgfqpoint{1.353164in}{3.006675in}}%
\pgfpathcurveto{\pgfqpoint{1.360978in}{3.014489in}}{\pgfqpoint{1.365368in}{3.025088in}}{\pgfqpoint{1.365368in}{3.036138in}}%
\pgfpathcurveto{\pgfqpoint{1.365368in}{3.047188in}}{\pgfqpoint{1.360978in}{3.057787in}}{\pgfqpoint{1.353164in}{3.065601in}}%
\pgfpathcurveto{\pgfqpoint{1.345350in}{3.073414in}}{\pgfqpoint{1.334751in}{3.077805in}}{\pgfqpoint{1.323701in}{3.077805in}}%
\pgfpathcurveto{\pgfqpoint{1.312651in}{3.077805in}}{\pgfqpoint{1.302052in}{3.073414in}}{\pgfqpoint{1.294239in}{3.065601in}}%
\pgfpathcurveto{\pgfqpoint{1.286425in}{3.057787in}}{\pgfqpoint{1.282035in}{3.047188in}}{\pgfqpoint{1.282035in}{3.036138in}}%
\pgfpathcurveto{\pgfqpoint{1.282035in}{3.025088in}}{\pgfqpoint{1.286425in}{3.014489in}}{\pgfqpoint{1.294239in}{3.006675in}}%
\pgfpathcurveto{\pgfqpoint{1.302052in}{2.998862in}}{\pgfqpoint{1.312651in}{2.994471in}}{\pgfqpoint{1.323701in}{2.994471in}}%
\pgfpathclose%
\pgfusepath{stroke,fill}%
\end{pgfscope}%
\begin{pgfscope}%
\pgfpathrectangle{\pgfqpoint{0.750000in}{0.500000in}}{\pgfqpoint{4.650000in}{3.020000in}}%
\pgfusepath{clip}%
\pgfsetbuttcap%
\pgfsetroundjoin%
\definecolor{currentfill}{rgb}{0.121569,0.466667,0.705882}%
\pgfsetfillcolor{currentfill}%
\pgfsetlinewidth{1.003750pt}%
\definecolor{currentstroke}{rgb}{0.121569,0.466667,0.705882}%
\pgfsetstrokecolor{currentstroke}%
\pgfsetdash{}{0pt}%
\pgfpathmoveto{\pgfqpoint{0.961364in}{0.595606in}}%
\pgfpathcurveto{\pgfqpoint{0.972414in}{0.595606in}}{\pgfqpoint{0.983013in}{0.599996in}}{\pgfqpoint{0.990826in}{0.607810in}}%
\pgfpathcurveto{\pgfqpoint{0.998640in}{0.615624in}}{\pgfqpoint{1.003030in}{0.626223in}}{\pgfqpoint{1.003030in}{0.637273in}}%
\pgfpathcurveto{\pgfqpoint{1.003030in}{0.648323in}}{\pgfqpoint{0.998640in}{0.658922in}}{\pgfqpoint{0.990826in}{0.666736in}}%
\pgfpathcurveto{\pgfqpoint{0.983013in}{0.674549in}}{\pgfqpoint{0.972414in}{0.678939in}}{\pgfqpoint{0.961364in}{0.678939in}}%
\pgfpathcurveto{\pgfqpoint{0.950314in}{0.678939in}}{\pgfqpoint{0.939714in}{0.674549in}}{\pgfqpoint{0.931901in}{0.666736in}}%
\pgfpathcurveto{\pgfqpoint{0.924087in}{0.658922in}}{\pgfqpoint{0.919697in}{0.648323in}}{\pgfqpoint{0.919697in}{0.637273in}}%
\pgfpathcurveto{\pgfqpoint{0.919697in}{0.626223in}}{\pgfqpoint{0.924087in}{0.615624in}}{\pgfqpoint{0.931901in}{0.607810in}}%
\pgfpathcurveto{\pgfqpoint{0.939714in}{0.599996in}}{\pgfqpoint{0.950314in}{0.595606in}}{\pgfqpoint{0.961364in}{0.595606in}}%
\pgfpathclose%
\pgfusepath{stroke,fill}%
\end{pgfscope}%
\begin{pgfscope}%
\pgfpathrectangle{\pgfqpoint{0.750000in}{0.500000in}}{\pgfqpoint{4.650000in}{3.020000in}}%
\pgfusepath{clip}%
\pgfsetbuttcap%
\pgfsetroundjoin%
\definecolor{currentfill}{rgb}{0.121569,0.466667,0.705882}%
\pgfsetfillcolor{currentfill}%
\pgfsetlinewidth{1.003750pt}%
\definecolor{currentstroke}{rgb}{0.121569,0.466667,0.705882}%
\pgfsetstrokecolor{currentstroke}%
\pgfsetdash{}{0pt}%
\pgfpathmoveto{\pgfqpoint{1.021753in}{0.595606in}}%
\pgfpathcurveto{\pgfqpoint{1.032803in}{0.595606in}}{\pgfqpoint{1.043402in}{0.599996in}}{\pgfqpoint{1.051216in}{0.607810in}}%
\pgfpathcurveto{\pgfqpoint{1.059030in}{0.615624in}}{\pgfqpoint{1.063420in}{0.626223in}}{\pgfqpoint{1.063420in}{0.637273in}}%
\pgfpathcurveto{\pgfqpoint{1.063420in}{0.648323in}}{\pgfqpoint{1.059030in}{0.658922in}}{\pgfqpoint{1.051216in}{0.666736in}}%
\pgfpathcurveto{\pgfqpoint{1.043402in}{0.674549in}}{\pgfqpoint{1.032803in}{0.678939in}}{\pgfqpoint{1.021753in}{0.678939in}}%
\pgfpathcurveto{\pgfqpoint{1.010703in}{0.678939in}}{\pgfqpoint{1.000104in}{0.674549in}}{\pgfqpoint{0.992290in}{0.666736in}}%
\pgfpathcurveto{\pgfqpoint{0.984477in}{0.658922in}}{\pgfqpoint{0.980087in}{0.648323in}}{\pgfqpoint{0.980087in}{0.637273in}}%
\pgfpathcurveto{\pgfqpoint{0.980087in}{0.626223in}}{\pgfqpoint{0.984477in}{0.615624in}}{\pgfqpoint{0.992290in}{0.607810in}}%
\pgfpathcurveto{\pgfqpoint{1.000104in}{0.599996in}}{\pgfqpoint{1.010703in}{0.595606in}}{\pgfqpoint{1.021753in}{0.595606in}}%
\pgfpathclose%
\pgfusepath{stroke,fill}%
\end{pgfscope}%
\begin{pgfscope}%
\pgfpathrectangle{\pgfqpoint{0.750000in}{0.500000in}}{\pgfqpoint{4.650000in}{3.020000in}}%
\pgfusepath{clip}%
\pgfsetbuttcap%
\pgfsetroundjoin%
\definecolor{currentfill}{rgb}{1.000000,0.498039,0.054902}%
\pgfsetfillcolor{currentfill}%
\pgfsetlinewidth{1.003750pt}%
\definecolor{currentstroke}{rgb}{1.000000,0.498039,0.054902}%
\pgfsetstrokecolor{currentstroke}%
\pgfsetdash{}{0pt}%
\pgfpathmoveto{\pgfqpoint{1.927597in}{2.936057in}}%
\pgfpathcurveto{\pgfqpoint{1.938648in}{2.936057in}}{\pgfqpoint{1.949247in}{2.940448in}}{\pgfqpoint{1.957060in}{2.948261in}}%
\pgfpathcurveto{\pgfqpoint{1.964874in}{2.956075in}}{\pgfqpoint{1.969264in}{2.966674in}}{\pgfqpoint{1.969264in}{2.977724in}}%
\pgfpathcurveto{\pgfqpoint{1.969264in}{2.988774in}}{\pgfqpoint{1.964874in}{2.999373in}}{\pgfqpoint{1.957060in}{3.007187in}}%
\pgfpathcurveto{\pgfqpoint{1.949247in}{3.015000in}}{\pgfqpoint{1.938648in}{3.019391in}}{\pgfqpoint{1.927597in}{3.019391in}}%
\pgfpathcurveto{\pgfqpoint{1.916547in}{3.019391in}}{\pgfqpoint{1.905948in}{3.015000in}}{\pgfqpoint{1.898135in}{3.007187in}}%
\pgfpathcurveto{\pgfqpoint{1.890321in}{2.999373in}}{\pgfqpoint{1.885931in}{2.988774in}}{\pgfqpoint{1.885931in}{2.977724in}}%
\pgfpathcurveto{\pgfqpoint{1.885931in}{2.966674in}}{\pgfqpoint{1.890321in}{2.956075in}}{\pgfqpoint{1.898135in}{2.948261in}}%
\pgfpathcurveto{\pgfqpoint{1.905948in}{2.940448in}}{\pgfqpoint{1.916547in}{2.936057in}}{\pgfqpoint{1.927597in}{2.936057in}}%
\pgfpathclose%
\pgfusepath{stroke,fill}%
\end{pgfscope}%
\begin{pgfscope}%
\pgfpathrectangle{\pgfqpoint{0.750000in}{0.500000in}}{\pgfqpoint{4.650000in}{3.020000in}}%
\pgfusepath{clip}%
\pgfsetbuttcap%
\pgfsetroundjoin%
\definecolor{currentfill}{rgb}{1.000000,0.498039,0.054902}%
\pgfsetfillcolor{currentfill}%
\pgfsetlinewidth{1.003750pt}%
\definecolor{currentstroke}{rgb}{1.000000,0.498039,0.054902}%
\pgfsetstrokecolor{currentstroke}%
\pgfsetdash{}{0pt}%
\pgfpathmoveto{\pgfqpoint{1.806818in}{2.936057in}}%
\pgfpathcurveto{\pgfqpoint{1.817868in}{2.936057in}}{\pgfqpoint{1.828467in}{2.940448in}}{\pgfqpoint{1.836281in}{2.948261in}}%
\pgfpathcurveto{\pgfqpoint{1.844095in}{2.956075in}}{\pgfqpoint{1.848485in}{2.966674in}}{\pgfqpoint{1.848485in}{2.977724in}}%
\pgfpathcurveto{\pgfqpoint{1.848485in}{2.988774in}}{\pgfqpoint{1.844095in}{2.999373in}}{\pgfqpoint{1.836281in}{3.007187in}}%
\pgfpathcurveto{\pgfqpoint{1.828467in}{3.015000in}}{\pgfqpoint{1.817868in}{3.019391in}}{\pgfqpoint{1.806818in}{3.019391in}}%
\pgfpathcurveto{\pgfqpoint{1.795768in}{3.019391in}}{\pgfqpoint{1.785169in}{3.015000in}}{\pgfqpoint{1.777355in}{3.007187in}}%
\pgfpathcurveto{\pgfqpoint{1.769542in}{2.999373in}}{\pgfqpoint{1.765152in}{2.988774in}}{\pgfqpoint{1.765152in}{2.977724in}}%
\pgfpathcurveto{\pgfqpoint{1.765152in}{2.966674in}}{\pgfqpoint{1.769542in}{2.956075in}}{\pgfqpoint{1.777355in}{2.948261in}}%
\pgfpathcurveto{\pgfqpoint{1.785169in}{2.940448in}}{\pgfqpoint{1.795768in}{2.936057in}}{\pgfqpoint{1.806818in}{2.936057in}}%
\pgfpathclose%
\pgfusepath{stroke,fill}%
\end{pgfscope}%
\begin{pgfscope}%
\pgfsetbuttcap%
\pgfsetroundjoin%
\definecolor{currentfill}{rgb}{0.000000,0.000000,0.000000}%
\pgfsetfillcolor{currentfill}%
\pgfsetlinewidth{0.803000pt}%
\definecolor{currentstroke}{rgb}{0.000000,0.000000,0.000000}%
\pgfsetstrokecolor{currentstroke}%
\pgfsetdash{}{0pt}%
\pgfsys@defobject{currentmarker}{\pgfqpoint{0.000000in}{-0.048611in}}{\pgfqpoint{0.000000in}{0.000000in}}{%
\pgfpathmoveto{\pgfqpoint{0.000000in}{0.000000in}}%
\pgfpathlineto{\pgfqpoint{0.000000in}{-0.048611in}}%
\pgfusepath{stroke,fill}%
}%
\begin{pgfscope}%
\pgfsys@transformshift{0.900974in}{0.500000in}%
\pgfsys@useobject{currentmarker}{}%
\end{pgfscope}%
\end{pgfscope}%
\begin{pgfscope}%
\definecolor{textcolor}{rgb}{0.000000,0.000000,0.000000}%
\pgfsetstrokecolor{textcolor}%
\pgfsetfillcolor{textcolor}%
\pgftext[x=0.900974in,y=0.402778in,,top]{\color{textcolor}\rmfamily\fontsize{10.000000}{12.000000}\selectfont \(\displaystyle {0}\)}%
\end{pgfscope}%
\begin{pgfscope}%
\pgfsetbuttcap%
\pgfsetroundjoin%
\definecolor{currentfill}{rgb}{0.000000,0.000000,0.000000}%
\pgfsetfillcolor{currentfill}%
\pgfsetlinewidth{0.803000pt}%
\definecolor{currentstroke}{rgb}{0.000000,0.000000,0.000000}%
\pgfsetstrokecolor{currentstroke}%
\pgfsetdash{}{0pt}%
\pgfsys@defobject{currentmarker}{\pgfqpoint{0.000000in}{-0.048611in}}{\pgfqpoint{0.000000in}{0.000000in}}{%
\pgfpathmoveto{\pgfqpoint{0.000000in}{0.000000in}}%
\pgfpathlineto{\pgfqpoint{0.000000in}{-0.048611in}}%
\pgfusepath{stroke,fill}%
}%
\begin{pgfscope}%
\pgfsys@transformshift{1.504870in}{0.500000in}%
\pgfsys@useobject{currentmarker}{}%
\end{pgfscope}%
\end{pgfscope}%
\begin{pgfscope}%
\definecolor{textcolor}{rgb}{0.000000,0.000000,0.000000}%
\pgfsetstrokecolor{textcolor}%
\pgfsetfillcolor{textcolor}%
\pgftext[x=1.504870in,y=0.402778in,,top]{\color{textcolor}\rmfamily\fontsize{10.000000}{12.000000}\selectfont \(\displaystyle {10}\)}%
\end{pgfscope}%
\begin{pgfscope}%
\pgfsetbuttcap%
\pgfsetroundjoin%
\definecolor{currentfill}{rgb}{0.000000,0.000000,0.000000}%
\pgfsetfillcolor{currentfill}%
\pgfsetlinewidth{0.803000pt}%
\definecolor{currentstroke}{rgb}{0.000000,0.000000,0.000000}%
\pgfsetstrokecolor{currentstroke}%
\pgfsetdash{}{0pt}%
\pgfsys@defobject{currentmarker}{\pgfqpoint{0.000000in}{-0.048611in}}{\pgfqpoint{0.000000in}{0.000000in}}{%
\pgfpathmoveto{\pgfqpoint{0.000000in}{0.000000in}}%
\pgfpathlineto{\pgfqpoint{0.000000in}{-0.048611in}}%
\pgfusepath{stroke,fill}%
}%
\begin{pgfscope}%
\pgfsys@transformshift{2.108766in}{0.500000in}%
\pgfsys@useobject{currentmarker}{}%
\end{pgfscope}%
\end{pgfscope}%
\begin{pgfscope}%
\definecolor{textcolor}{rgb}{0.000000,0.000000,0.000000}%
\pgfsetstrokecolor{textcolor}%
\pgfsetfillcolor{textcolor}%
\pgftext[x=2.108766in,y=0.402778in,,top]{\color{textcolor}\rmfamily\fontsize{10.000000}{12.000000}\selectfont \(\displaystyle {20}\)}%
\end{pgfscope}%
\begin{pgfscope}%
\pgfsetbuttcap%
\pgfsetroundjoin%
\definecolor{currentfill}{rgb}{0.000000,0.000000,0.000000}%
\pgfsetfillcolor{currentfill}%
\pgfsetlinewidth{0.803000pt}%
\definecolor{currentstroke}{rgb}{0.000000,0.000000,0.000000}%
\pgfsetstrokecolor{currentstroke}%
\pgfsetdash{}{0pt}%
\pgfsys@defobject{currentmarker}{\pgfqpoint{0.000000in}{-0.048611in}}{\pgfqpoint{0.000000in}{0.000000in}}{%
\pgfpathmoveto{\pgfqpoint{0.000000in}{0.000000in}}%
\pgfpathlineto{\pgfqpoint{0.000000in}{-0.048611in}}%
\pgfusepath{stroke,fill}%
}%
\begin{pgfscope}%
\pgfsys@transformshift{2.712662in}{0.500000in}%
\pgfsys@useobject{currentmarker}{}%
\end{pgfscope}%
\end{pgfscope}%
\begin{pgfscope}%
\definecolor{textcolor}{rgb}{0.000000,0.000000,0.000000}%
\pgfsetstrokecolor{textcolor}%
\pgfsetfillcolor{textcolor}%
\pgftext[x=2.712662in,y=0.402778in,,top]{\color{textcolor}\rmfamily\fontsize{10.000000}{12.000000}\selectfont \(\displaystyle {30}\)}%
\end{pgfscope}%
\begin{pgfscope}%
\pgfsetbuttcap%
\pgfsetroundjoin%
\definecolor{currentfill}{rgb}{0.000000,0.000000,0.000000}%
\pgfsetfillcolor{currentfill}%
\pgfsetlinewidth{0.803000pt}%
\definecolor{currentstroke}{rgb}{0.000000,0.000000,0.000000}%
\pgfsetstrokecolor{currentstroke}%
\pgfsetdash{}{0pt}%
\pgfsys@defobject{currentmarker}{\pgfqpoint{0.000000in}{-0.048611in}}{\pgfqpoint{0.000000in}{0.000000in}}{%
\pgfpathmoveto{\pgfqpoint{0.000000in}{0.000000in}}%
\pgfpathlineto{\pgfqpoint{0.000000in}{-0.048611in}}%
\pgfusepath{stroke,fill}%
}%
\begin{pgfscope}%
\pgfsys@transformshift{3.316558in}{0.500000in}%
\pgfsys@useobject{currentmarker}{}%
\end{pgfscope}%
\end{pgfscope}%
\begin{pgfscope}%
\definecolor{textcolor}{rgb}{0.000000,0.000000,0.000000}%
\pgfsetstrokecolor{textcolor}%
\pgfsetfillcolor{textcolor}%
\pgftext[x=3.316558in,y=0.402778in,,top]{\color{textcolor}\rmfamily\fontsize{10.000000}{12.000000}\selectfont \(\displaystyle {40}\)}%
\end{pgfscope}%
\begin{pgfscope}%
\pgfsetbuttcap%
\pgfsetroundjoin%
\definecolor{currentfill}{rgb}{0.000000,0.000000,0.000000}%
\pgfsetfillcolor{currentfill}%
\pgfsetlinewidth{0.803000pt}%
\definecolor{currentstroke}{rgb}{0.000000,0.000000,0.000000}%
\pgfsetstrokecolor{currentstroke}%
\pgfsetdash{}{0pt}%
\pgfsys@defobject{currentmarker}{\pgfqpoint{0.000000in}{-0.048611in}}{\pgfqpoint{0.000000in}{0.000000in}}{%
\pgfpathmoveto{\pgfqpoint{0.000000in}{0.000000in}}%
\pgfpathlineto{\pgfqpoint{0.000000in}{-0.048611in}}%
\pgfusepath{stroke,fill}%
}%
\begin{pgfscope}%
\pgfsys@transformshift{3.920455in}{0.500000in}%
\pgfsys@useobject{currentmarker}{}%
\end{pgfscope}%
\end{pgfscope}%
\begin{pgfscope}%
\definecolor{textcolor}{rgb}{0.000000,0.000000,0.000000}%
\pgfsetstrokecolor{textcolor}%
\pgfsetfillcolor{textcolor}%
\pgftext[x=3.920455in,y=0.402778in,,top]{\color{textcolor}\rmfamily\fontsize{10.000000}{12.000000}\selectfont \(\displaystyle {50}\)}%
\end{pgfscope}%
\begin{pgfscope}%
\pgfsetbuttcap%
\pgfsetroundjoin%
\definecolor{currentfill}{rgb}{0.000000,0.000000,0.000000}%
\pgfsetfillcolor{currentfill}%
\pgfsetlinewidth{0.803000pt}%
\definecolor{currentstroke}{rgb}{0.000000,0.000000,0.000000}%
\pgfsetstrokecolor{currentstroke}%
\pgfsetdash{}{0pt}%
\pgfsys@defobject{currentmarker}{\pgfqpoint{0.000000in}{-0.048611in}}{\pgfqpoint{0.000000in}{0.000000in}}{%
\pgfpathmoveto{\pgfqpoint{0.000000in}{0.000000in}}%
\pgfpathlineto{\pgfqpoint{0.000000in}{-0.048611in}}%
\pgfusepath{stroke,fill}%
}%
\begin{pgfscope}%
\pgfsys@transformshift{4.524351in}{0.500000in}%
\pgfsys@useobject{currentmarker}{}%
\end{pgfscope}%
\end{pgfscope}%
\begin{pgfscope}%
\definecolor{textcolor}{rgb}{0.000000,0.000000,0.000000}%
\pgfsetstrokecolor{textcolor}%
\pgfsetfillcolor{textcolor}%
\pgftext[x=4.524351in,y=0.402778in,,top]{\color{textcolor}\rmfamily\fontsize{10.000000}{12.000000}\selectfont \(\displaystyle {60}\)}%
\end{pgfscope}%
\begin{pgfscope}%
\pgfsetbuttcap%
\pgfsetroundjoin%
\definecolor{currentfill}{rgb}{0.000000,0.000000,0.000000}%
\pgfsetfillcolor{currentfill}%
\pgfsetlinewidth{0.803000pt}%
\definecolor{currentstroke}{rgb}{0.000000,0.000000,0.000000}%
\pgfsetstrokecolor{currentstroke}%
\pgfsetdash{}{0pt}%
\pgfsys@defobject{currentmarker}{\pgfqpoint{0.000000in}{-0.048611in}}{\pgfqpoint{0.000000in}{0.000000in}}{%
\pgfpathmoveto{\pgfqpoint{0.000000in}{0.000000in}}%
\pgfpathlineto{\pgfqpoint{0.000000in}{-0.048611in}}%
\pgfusepath{stroke,fill}%
}%
\begin{pgfscope}%
\pgfsys@transformshift{5.128247in}{0.500000in}%
\pgfsys@useobject{currentmarker}{}%
\end{pgfscope}%
\end{pgfscope}%
\begin{pgfscope}%
\definecolor{textcolor}{rgb}{0.000000,0.000000,0.000000}%
\pgfsetstrokecolor{textcolor}%
\pgfsetfillcolor{textcolor}%
\pgftext[x=5.128247in,y=0.402778in,,top]{\color{textcolor}\rmfamily\fontsize{10.000000}{12.000000}\selectfont \(\displaystyle {70}\)}%
\end{pgfscope}%
\begin{pgfscope}%
\definecolor{textcolor}{rgb}{0.000000,0.000000,0.000000}%
\pgfsetstrokecolor{textcolor}%
\pgfsetfillcolor{textcolor}%
\pgftext[x=3.075000in,y=0.223889in,,top]{\color{textcolor}\rmfamily\fontsize{10.000000}{12.000000}\selectfont Number of Sources}%
\end{pgfscope}%
\begin{pgfscope}%
\pgfsetbuttcap%
\pgfsetroundjoin%
\definecolor{currentfill}{rgb}{0.000000,0.000000,0.000000}%
\pgfsetfillcolor{currentfill}%
\pgfsetlinewidth{0.803000pt}%
\definecolor{currentstroke}{rgb}{0.000000,0.000000,0.000000}%
\pgfsetstrokecolor{currentstroke}%
\pgfsetdash{}{0pt}%
\pgfsys@defobject{currentmarker}{\pgfqpoint{-0.048611in}{0.000000in}}{\pgfqpoint{0.000000in}{0.000000in}}{%
\pgfpathmoveto{\pgfqpoint{0.000000in}{0.000000in}}%
\pgfpathlineto{\pgfqpoint{-0.048611in}{0.000000in}}%
\pgfusepath{stroke,fill}%
}%
\begin{pgfscope}%
\pgfsys@transformshift{0.750000in}{0.637273in}%
\pgfsys@useobject{currentmarker}{}%
\end{pgfscope}%
\end{pgfscope}%
\begin{pgfscope}%
\definecolor{textcolor}{rgb}{0.000000,0.000000,0.000000}%
\pgfsetstrokecolor{textcolor}%
\pgfsetfillcolor{textcolor}%
\pgftext[x=0.583333in, y=0.589078in, left, base]{\color{textcolor}\rmfamily\fontsize{10.000000}{12.000000}\selectfont \(\displaystyle {0}\)}%
\end{pgfscope}%
\begin{pgfscope}%
\pgfsetbuttcap%
\pgfsetroundjoin%
\definecolor{currentfill}{rgb}{0.000000,0.000000,0.000000}%
\pgfsetfillcolor{currentfill}%
\pgfsetlinewidth{0.803000pt}%
\definecolor{currentstroke}{rgb}{0.000000,0.000000,0.000000}%
\pgfsetstrokecolor{currentstroke}%
\pgfsetdash{}{0pt}%
\pgfsys@defobject{currentmarker}{\pgfqpoint{-0.048611in}{0.000000in}}{\pgfqpoint{0.000000in}{0.000000in}}{%
\pgfpathmoveto{\pgfqpoint{0.000000in}{0.000000in}}%
\pgfpathlineto{\pgfqpoint{-0.048611in}{0.000000in}}%
\pgfusepath{stroke,fill}%
}%
\begin{pgfscope}%
\pgfsys@transformshift{0.750000in}{1.026699in}%
\pgfsys@useobject{currentmarker}{}%
\end{pgfscope}%
\end{pgfscope}%
\begin{pgfscope}%
\definecolor{textcolor}{rgb}{0.000000,0.000000,0.000000}%
\pgfsetstrokecolor{textcolor}%
\pgfsetfillcolor{textcolor}%
\pgftext[x=0.444444in, y=0.978504in, left, base]{\color{textcolor}\rmfamily\fontsize{10.000000}{12.000000}\selectfont \(\displaystyle {100}\)}%
\end{pgfscope}%
\begin{pgfscope}%
\pgfsetbuttcap%
\pgfsetroundjoin%
\definecolor{currentfill}{rgb}{0.000000,0.000000,0.000000}%
\pgfsetfillcolor{currentfill}%
\pgfsetlinewidth{0.803000pt}%
\definecolor{currentstroke}{rgb}{0.000000,0.000000,0.000000}%
\pgfsetstrokecolor{currentstroke}%
\pgfsetdash{}{0pt}%
\pgfsys@defobject{currentmarker}{\pgfqpoint{-0.048611in}{0.000000in}}{\pgfqpoint{0.000000in}{0.000000in}}{%
\pgfpathmoveto{\pgfqpoint{0.000000in}{0.000000in}}%
\pgfpathlineto{\pgfqpoint{-0.048611in}{0.000000in}}%
\pgfusepath{stroke,fill}%
}%
\begin{pgfscope}%
\pgfsys@transformshift{0.750000in}{1.416125in}%
\pgfsys@useobject{currentmarker}{}%
\end{pgfscope}%
\end{pgfscope}%
\begin{pgfscope}%
\definecolor{textcolor}{rgb}{0.000000,0.000000,0.000000}%
\pgfsetstrokecolor{textcolor}%
\pgfsetfillcolor{textcolor}%
\pgftext[x=0.444444in, y=1.367931in, left, base]{\color{textcolor}\rmfamily\fontsize{10.000000}{12.000000}\selectfont \(\displaystyle {200}\)}%
\end{pgfscope}%
\begin{pgfscope}%
\pgfsetbuttcap%
\pgfsetroundjoin%
\definecolor{currentfill}{rgb}{0.000000,0.000000,0.000000}%
\pgfsetfillcolor{currentfill}%
\pgfsetlinewidth{0.803000pt}%
\definecolor{currentstroke}{rgb}{0.000000,0.000000,0.000000}%
\pgfsetstrokecolor{currentstroke}%
\pgfsetdash{}{0pt}%
\pgfsys@defobject{currentmarker}{\pgfqpoint{-0.048611in}{0.000000in}}{\pgfqpoint{0.000000in}{0.000000in}}{%
\pgfpathmoveto{\pgfqpoint{0.000000in}{0.000000in}}%
\pgfpathlineto{\pgfqpoint{-0.048611in}{0.000000in}}%
\pgfusepath{stroke,fill}%
}%
\begin{pgfscope}%
\pgfsys@transformshift{0.750000in}{1.805551in}%
\pgfsys@useobject{currentmarker}{}%
\end{pgfscope}%
\end{pgfscope}%
\begin{pgfscope}%
\definecolor{textcolor}{rgb}{0.000000,0.000000,0.000000}%
\pgfsetstrokecolor{textcolor}%
\pgfsetfillcolor{textcolor}%
\pgftext[x=0.444444in, y=1.757357in, left, base]{\color{textcolor}\rmfamily\fontsize{10.000000}{12.000000}\selectfont \(\displaystyle {300}\)}%
\end{pgfscope}%
\begin{pgfscope}%
\pgfsetbuttcap%
\pgfsetroundjoin%
\definecolor{currentfill}{rgb}{0.000000,0.000000,0.000000}%
\pgfsetfillcolor{currentfill}%
\pgfsetlinewidth{0.803000pt}%
\definecolor{currentstroke}{rgb}{0.000000,0.000000,0.000000}%
\pgfsetstrokecolor{currentstroke}%
\pgfsetdash{}{0pt}%
\pgfsys@defobject{currentmarker}{\pgfqpoint{-0.048611in}{0.000000in}}{\pgfqpoint{0.000000in}{0.000000in}}{%
\pgfpathmoveto{\pgfqpoint{0.000000in}{0.000000in}}%
\pgfpathlineto{\pgfqpoint{-0.048611in}{0.000000in}}%
\pgfusepath{stroke,fill}%
}%
\begin{pgfscope}%
\pgfsys@transformshift{0.750000in}{2.194977in}%
\pgfsys@useobject{currentmarker}{}%
\end{pgfscope}%
\end{pgfscope}%
\begin{pgfscope}%
\definecolor{textcolor}{rgb}{0.000000,0.000000,0.000000}%
\pgfsetstrokecolor{textcolor}%
\pgfsetfillcolor{textcolor}%
\pgftext[x=0.444444in, y=2.146783in, left, base]{\color{textcolor}\rmfamily\fontsize{10.000000}{12.000000}\selectfont \(\displaystyle {400}\)}%
\end{pgfscope}%
\begin{pgfscope}%
\pgfsetbuttcap%
\pgfsetroundjoin%
\definecolor{currentfill}{rgb}{0.000000,0.000000,0.000000}%
\pgfsetfillcolor{currentfill}%
\pgfsetlinewidth{0.803000pt}%
\definecolor{currentstroke}{rgb}{0.000000,0.000000,0.000000}%
\pgfsetstrokecolor{currentstroke}%
\pgfsetdash{}{0pt}%
\pgfsys@defobject{currentmarker}{\pgfqpoint{-0.048611in}{0.000000in}}{\pgfqpoint{0.000000in}{0.000000in}}{%
\pgfpathmoveto{\pgfqpoint{0.000000in}{0.000000in}}%
\pgfpathlineto{\pgfqpoint{-0.048611in}{0.000000in}}%
\pgfusepath{stroke,fill}%
}%
\begin{pgfscope}%
\pgfsys@transformshift{0.750000in}{2.584404in}%
\pgfsys@useobject{currentmarker}{}%
\end{pgfscope}%
\end{pgfscope}%
\begin{pgfscope}%
\definecolor{textcolor}{rgb}{0.000000,0.000000,0.000000}%
\pgfsetstrokecolor{textcolor}%
\pgfsetfillcolor{textcolor}%
\pgftext[x=0.444444in, y=2.536209in, left, base]{\color{textcolor}\rmfamily\fontsize{10.000000}{12.000000}\selectfont \(\displaystyle {500}\)}%
\end{pgfscope}%
\begin{pgfscope}%
\pgfsetbuttcap%
\pgfsetroundjoin%
\definecolor{currentfill}{rgb}{0.000000,0.000000,0.000000}%
\pgfsetfillcolor{currentfill}%
\pgfsetlinewidth{0.803000pt}%
\definecolor{currentstroke}{rgb}{0.000000,0.000000,0.000000}%
\pgfsetstrokecolor{currentstroke}%
\pgfsetdash{}{0pt}%
\pgfsys@defobject{currentmarker}{\pgfqpoint{-0.048611in}{0.000000in}}{\pgfqpoint{0.000000in}{0.000000in}}{%
\pgfpathmoveto{\pgfqpoint{0.000000in}{0.000000in}}%
\pgfpathlineto{\pgfqpoint{-0.048611in}{0.000000in}}%
\pgfusepath{stroke,fill}%
}%
\begin{pgfscope}%
\pgfsys@transformshift{0.750000in}{2.973830in}%
\pgfsys@useobject{currentmarker}{}%
\end{pgfscope}%
\end{pgfscope}%
\begin{pgfscope}%
\definecolor{textcolor}{rgb}{0.000000,0.000000,0.000000}%
\pgfsetstrokecolor{textcolor}%
\pgfsetfillcolor{textcolor}%
\pgftext[x=0.444444in, y=2.925635in, left, base]{\color{textcolor}\rmfamily\fontsize{10.000000}{12.000000}\selectfont \(\displaystyle {600}\)}%
\end{pgfscope}%
\begin{pgfscope}%
\pgfsetbuttcap%
\pgfsetroundjoin%
\definecolor{currentfill}{rgb}{0.000000,0.000000,0.000000}%
\pgfsetfillcolor{currentfill}%
\pgfsetlinewidth{0.803000pt}%
\definecolor{currentstroke}{rgb}{0.000000,0.000000,0.000000}%
\pgfsetstrokecolor{currentstroke}%
\pgfsetdash{}{0pt}%
\pgfsys@defobject{currentmarker}{\pgfqpoint{-0.048611in}{0.000000in}}{\pgfqpoint{0.000000in}{0.000000in}}{%
\pgfpathmoveto{\pgfqpoint{0.000000in}{0.000000in}}%
\pgfpathlineto{\pgfqpoint{-0.048611in}{0.000000in}}%
\pgfusepath{stroke,fill}%
}%
\begin{pgfscope}%
\pgfsys@transformshift{0.750000in}{3.363256in}%
\pgfsys@useobject{currentmarker}{}%
\end{pgfscope}%
\end{pgfscope}%
\begin{pgfscope}%
\definecolor{textcolor}{rgb}{0.000000,0.000000,0.000000}%
\pgfsetstrokecolor{textcolor}%
\pgfsetfillcolor{textcolor}%
\pgftext[x=0.444444in, y=3.315062in, left, base]{\color{textcolor}\rmfamily\fontsize{10.000000}{12.000000}\selectfont \(\displaystyle {700}\)}%
\end{pgfscope}%
\begin{pgfscope}%
\definecolor{textcolor}{rgb}{0.000000,0.000000,0.000000}%
\pgfsetstrokecolor{textcolor}%
\pgfsetfillcolor{textcolor}%
\pgftext[x=0.388888in,y=2.010000in,,bottom,rotate=90.000000]{\color{textcolor}\rmfamily\fontsize{10.000000}{12.000000}\selectfont Dataflow Time}%
\end{pgfscope}%
\begin{pgfscope}%
\pgfsetrectcap%
\pgfsetmiterjoin%
\pgfsetlinewidth{0.803000pt}%
\definecolor{currentstroke}{rgb}{0.000000,0.000000,0.000000}%
\pgfsetstrokecolor{currentstroke}%
\pgfsetdash{}{0pt}%
\pgfpathmoveto{\pgfqpoint{0.750000in}{0.500000in}}%
\pgfpathlineto{\pgfqpoint{0.750000in}{3.520000in}}%
\pgfusepath{stroke}%
\end{pgfscope}%
\begin{pgfscope}%
\pgfsetrectcap%
\pgfsetmiterjoin%
\pgfsetlinewidth{0.803000pt}%
\definecolor{currentstroke}{rgb}{0.000000,0.000000,0.000000}%
\pgfsetstrokecolor{currentstroke}%
\pgfsetdash{}{0pt}%
\pgfpathmoveto{\pgfqpoint{5.400000in}{0.500000in}}%
\pgfpathlineto{\pgfqpoint{5.400000in}{3.520000in}}%
\pgfusepath{stroke}%
\end{pgfscope}%
\begin{pgfscope}%
\pgfsetrectcap%
\pgfsetmiterjoin%
\pgfsetlinewidth{0.803000pt}%
\definecolor{currentstroke}{rgb}{0.000000,0.000000,0.000000}%
\pgfsetstrokecolor{currentstroke}%
\pgfsetdash{}{0pt}%
\pgfpathmoveto{\pgfqpoint{0.750000in}{0.500000in}}%
\pgfpathlineto{\pgfqpoint{5.400000in}{0.500000in}}%
\pgfusepath{stroke}%
\end{pgfscope}%
\begin{pgfscope}%
\pgfsetrectcap%
\pgfsetmiterjoin%
\pgfsetlinewidth{0.803000pt}%
\definecolor{currentstroke}{rgb}{0.000000,0.000000,0.000000}%
\pgfsetstrokecolor{currentstroke}%
\pgfsetdash{}{0pt}%
\pgfpathmoveto{\pgfqpoint{0.750000in}{3.520000in}}%
\pgfpathlineto{\pgfqpoint{5.400000in}{3.520000in}}%
\pgfusepath{stroke}%
\end{pgfscope}%
\begin{pgfscope}%
\definecolor{textcolor}{rgb}{0.000000,0.000000,0.000000}%
\pgfsetstrokecolor{textcolor}%
\pgfsetfillcolor{textcolor}%
\pgftext[x=3.075000in,y=3.603333in,,base]{\color{textcolor}\rmfamily\fontsize{12.000000}{14.400000}\selectfont Forwards}%
\end{pgfscope}%
\begin{pgfscope}%
\pgfsetbuttcap%
\pgfsetmiterjoin%
\definecolor{currentfill}{rgb}{1.000000,1.000000,1.000000}%
\pgfsetfillcolor{currentfill}%
\pgfsetfillopacity{0.800000}%
\pgfsetlinewidth{1.003750pt}%
\definecolor{currentstroke}{rgb}{0.800000,0.800000,0.800000}%
\pgfsetstrokecolor{currentstroke}%
\pgfsetstrokeopacity{0.800000}%
\pgfsetdash{}{0pt}%
\pgfpathmoveto{\pgfqpoint{3.793194in}{0.569444in}}%
\pgfpathlineto{\pgfqpoint{5.302778in}{0.569444in}}%
\pgfpathquadraticcurveto{\pgfqpoint{5.330556in}{0.569444in}}{\pgfqpoint{5.330556in}{0.597222in}}%
\pgfpathlineto{\pgfqpoint{5.330556in}{1.165694in}}%
\pgfpathquadraticcurveto{\pgfqpoint{5.330556in}{1.193472in}}{\pgfqpoint{5.302778in}{1.193472in}}%
\pgfpathlineto{\pgfqpoint{3.793194in}{1.193472in}}%
\pgfpathquadraticcurveto{\pgfqpoint{3.765417in}{1.193472in}}{\pgfqpoint{3.765417in}{1.165694in}}%
\pgfpathlineto{\pgfqpoint{3.765417in}{0.597222in}}%
\pgfpathquadraticcurveto{\pgfqpoint{3.765417in}{0.569444in}}{\pgfqpoint{3.793194in}{0.569444in}}%
\pgfpathclose%
\pgfusepath{stroke,fill}%
\end{pgfscope}%
\begin{pgfscope}%
\pgfsetbuttcap%
\pgfsetroundjoin%
\definecolor{currentfill}{rgb}{0.121569,0.466667,0.705882}%
\pgfsetfillcolor{currentfill}%
\pgfsetlinewidth{1.003750pt}%
\definecolor{currentstroke}{rgb}{0.121569,0.466667,0.705882}%
\pgfsetstrokecolor{currentstroke}%
\pgfsetdash{}{0pt}%
\pgfsys@defobject{currentmarker}{\pgfqpoint{-0.034722in}{-0.034722in}}{\pgfqpoint{0.034722in}{0.034722in}}{%
\pgfpathmoveto{\pgfqpoint{0.000000in}{-0.034722in}}%
\pgfpathcurveto{\pgfqpoint{0.009208in}{-0.034722in}}{\pgfqpoint{0.018041in}{-0.031064in}}{\pgfqpoint{0.024552in}{-0.024552in}}%
\pgfpathcurveto{\pgfqpoint{0.031064in}{-0.018041in}}{\pgfqpoint{0.034722in}{-0.009208in}}{\pgfqpoint{0.034722in}{0.000000in}}%
\pgfpathcurveto{\pgfqpoint{0.034722in}{0.009208in}}{\pgfqpoint{0.031064in}{0.018041in}}{\pgfqpoint{0.024552in}{0.024552in}}%
\pgfpathcurveto{\pgfqpoint{0.018041in}{0.031064in}}{\pgfqpoint{0.009208in}{0.034722in}}{\pgfqpoint{0.000000in}{0.034722in}}%
\pgfpathcurveto{\pgfqpoint{-0.009208in}{0.034722in}}{\pgfqpoint{-0.018041in}{0.031064in}}{\pgfqpoint{-0.024552in}{0.024552in}}%
\pgfpathcurveto{\pgfqpoint{-0.031064in}{0.018041in}}{\pgfqpoint{-0.034722in}{0.009208in}}{\pgfqpoint{-0.034722in}{0.000000in}}%
\pgfpathcurveto{\pgfqpoint{-0.034722in}{-0.009208in}}{\pgfqpoint{-0.031064in}{-0.018041in}}{\pgfqpoint{-0.024552in}{-0.024552in}}%
\pgfpathcurveto{\pgfqpoint{-0.018041in}{-0.031064in}}{\pgfqpoint{-0.009208in}{-0.034722in}}{\pgfqpoint{0.000000in}{-0.034722in}}%
\pgfpathclose%
\pgfusepath{stroke,fill}%
}%
\begin{pgfscope}%
\pgfsys@transformshift{3.959861in}{1.089306in}%
\pgfsys@useobject{currentmarker}{}%
\end{pgfscope}%
\end{pgfscope}%
\begin{pgfscope}%
\definecolor{textcolor}{rgb}{0.000000,0.000000,0.000000}%
\pgfsetstrokecolor{textcolor}%
\pgfsetfillcolor{textcolor}%
\pgftext[x=4.209861in,y=1.040694in,left,base]{\color{textcolor}\rmfamily\fontsize{10.000000}{12.000000}\selectfont No Timeout}%
\end{pgfscope}%
\begin{pgfscope}%
\pgfsetbuttcap%
\pgfsetroundjoin%
\definecolor{currentfill}{rgb}{1.000000,0.498039,0.054902}%
\pgfsetfillcolor{currentfill}%
\pgfsetlinewidth{1.003750pt}%
\definecolor{currentstroke}{rgb}{1.000000,0.498039,0.054902}%
\pgfsetstrokecolor{currentstroke}%
\pgfsetdash{}{0pt}%
\pgfsys@defobject{currentmarker}{\pgfqpoint{-0.034722in}{-0.034722in}}{\pgfqpoint{0.034722in}{0.034722in}}{%
\pgfpathmoveto{\pgfqpoint{0.000000in}{-0.034722in}}%
\pgfpathcurveto{\pgfqpoint{0.009208in}{-0.034722in}}{\pgfqpoint{0.018041in}{-0.031064in}}{\pgfqpoint{0.024552in}{-0.024552in}}%
\pgfpathcurveto{\pgfqpoint{0.031064in}{-0.018041in}}{\pgfqpoint{0.034722in}{-0.009208in}}{\pgfqpoint{0.034722in}{0.000000in}}%
\pgfpathcurveto{\pgfqpoint{0.034722in}{0.009208in}}{\pgfqpoint{0.031064in}{0.018041in}}{\pgfqpoint{0.024552in}{0.024552in}}%
\pgfpathcurveto{\pgfqpoint{0.018041in}{0.031064in}}{\pgfqpoint{0.009208in}{0.034722in}}{\pgfqpoint{0.000000in}{0.034722in}}%
\pgfpathcurveto{\pgfqpoint{-0.009208in}{0.034722in}}{\pgfqpoint{-0.018041in}{0.031064in}}{\pgfqpoint{-0.024552in}{0.024552in}}%
\pgfpathcurveto{\pgfqpoint{-0.031064in}{0.018041in}}{\pgfqpoint{-0.034722in}{0.009208in}}{\pgfqpoint{-0.034722in}{0.000000in}}%
\pgfpathcurveto{\pgfqpoint{-0.034722in}{-0.009208in}}{\pgfqpoint{-0.031064in}{-0.018041in}}{\pgfqpoint{-0.024552in}{-0.024552in}}%
\pgfpathcurveto{\pgfqpoint{-0.018041in}{-0.031064in}}{\pgfqpoint{-0.009208in}{-0.034722in}}{\pgfqpoint{0.000000in}{-0.034722in}}%
\pgfpathclose%
\pgfusepath{stroke,fill}%
}%
\begin{pgfscope}%
\pgfsys@transformshift{3.959861in}{0.895694in}%
\pgfsys@useobject{currentmarker}{}%
\end{pgfscope}%
\end{pgfscope}%
\begin{pgfscope}%
\definecolor{textcolor}{rgb}{0.000000,0.000000,0.000000}%
\pgfsetstrokecolor{textcolor}%
\pgfsetfillcolor{textcolor}%
\pgftext[x=4.209861in,y=0.847083in,left,base]{\color{textcolor}\rmfamily\fontsize{10.000000}{12.000000}\selectfont Time Timeout}%
\end{pgfscope}%
\begin{pgfscope}%
\pgfsetbuttcap%
\pgfsetroundjoin%
\definecolor{currentfill}{rgb}{0.839216,0.152941,0.156863}%
\pgfsetfillcolor{currentfill}%
\pgfsetlinewidth{1.003750pt}%
\definecolor{currentstroke}{rgb}{0.839216,0.152941,0.156863}%
\pgfsetstrokecolor{currentstroke}%
\pgfsetdash{}{0pt}%
\pgfsys@defobject{currentmarker}{\pgfqpoint{-0.034722in}{-0.034722in}}{\pgfqpoint{0.034722in}{0.034722in}}{%
\pgfpathmoveto{\pgfqpoint{0.000000in}{-0.034722in}}%
\pgfpathcurveto{\pgfqpoint{0.009208in}{-0.034722in}}{\pgfqpoint{0.018041in}{-0.031064in}}{\pgfqpoint{0.024552in}{-0.024552in}}%
\pgfpathcurveto{\pgfqpoint{0.031064in}{-0.018041in}}{\pgfqpoint{0.034722in}{-0.009208in}}{\pgfqpoint{0.034722in}{0.000000in}}%
\pgfpathcurveto{\pgfqpoint{0.034722in}{0.009208in}}{\pgfqpoint{0.031064in}{0.018041in}}{\pgfqpoint{0.024552in}{0.024552in}}%
\pgfpathcurveto{\pgfqpoint{0.018041in}{0.031064in}}{\pgfqpoint{0.009208in}{0.034722in}}{\pgfqpoint{0.000000in}{0.034722in}}%
\pgfpathcurveto{\pgfqpoint{-0.009208in}{0.034722in}}{\pgfqpoint{-0.018041in}{0.031064in}}{\pgfqpoint{-0.024552in}{0.024552in}}%
\pgfpathcurveto{\pgfqpoint{-0.031064in}{0.018041in}}{\pgfqpoint{-0.034722in}{0.009208in}}{\pgfqpoint{-0.034722in}{0.000000in}}%
\pgfpathcurveto{\pgfqpoint{-0.034722in}{-0.009208in}}{\pgfqpoint{-0.031064in}{-0.018041in}}{\pgfqpoint{-0.024552in}{-0.024552in}}%
\pgfpathcurveto{\pgfqpoint{-0.018041in}{-0.031064in}}{\pgfqpoint{-0.009208in}{-0.034722in}}{\pgfqpoint{0.000000in}{-0.034722in}}%
\pgfpathclose%
\pgfusepath{stroke,fill}%
}%
\begin{pgfscope}%
\pgfsys@transformshift{3.959861in}{0.702083in}%
\pgfsys@useobject{currentmarker}{}%
\end{pgfscope}%
\end{pgfscope}%
\begin{pgfscope}%
\definecolor{textcolor}{rgb}{0.000000,0.000000,0.000000}%
\pgfsetstrokecolor{textcolor}%
\pgfsetfillcolor{textcolor}%
\pgftext[x=4.209861in,y=0.653472in,left,base]{\color{textcolor}\rmfamily\fontsize{10.000000}{12.000000}\selectfont Memory Timeout}%
\end{pgfscope}%
\end{pgfpicture}%
\makeatother%
\endgroup%

                }
            \end{subfigure}
            \caption{Source Count}
            \label{f:dfsources}
        \end{subfigure}
        \qquad
        \begin{subfigure}[b]{0.45\textwidth}
            \centering
            \begin{subfigure}[]{\textwidth}
                \centering
                \resizebox{\columnwidth}{!}{
                    %% Creator: Matplotlib, PGF backend
%%
%% To include the figure in your LaTeX document, write
%%   \input{<filename>.pgf}
%%
%% Make sure the required packages are loaded in your preamble
%%   \usepackage{pgf}
%%
%% and, on pdftex
%%   \usepackage[utf8]{inputenc}\DeclareUnicodeCharacter{2212}{-}
%%
%% or, on luatex and xetex
%%   \usepackage{unicode-math}
%%
%% Figures using additional raster images can only be included by \input if
%% they are in the same directory as the main LaTeX file. For loading figures
%% from other directories you can use the `import` package
%%   \usepackage{import}
%%
%% and then include the figures with
%%   \import{<path to file>}{<filename>.pgf}
%%
%% Matplotlib used the following preamble
%%   \usepackage{amsmath}
%%   \usepackage{fontspec}
%%
\begingroup%
\makeatletter%
\begin{pgfpicture}%
\pgfpathrectangle{\pgfpointorigin}{\pgfqpoint{6.000000in}{4.000000in}}%
\pgfusepath{use as bounding box, clip}%
\begin{pgfscope}%
\pgfsetbuttcap%
\pgfsetmiterjoin%
\definecolor{currentfill}{rgb}{1.000000,1.000000,1.000000}%
\pgfsetfillcolor{currentfill}%
\pgfsetlinewidth{0.000000pt}%
\definecolor{currentstroke}{rgb}{1.000000,1.000000,1.000000}%
\pgfsetstrokecolor{currentstroke}%
\pgfsetdash{}{0pt}%
\pgfpathmoveto{\pgfqpoint{0.000000in}{0.000000in}}%
\pgfpathlineto{\pgfqpoint{6.000000in}{0.000000in}}%
\pgfpathlineto{\pgfqpoint{6.000000in}{4.000000in}}%
\pgfpathlineto{\pgfqpoint{0.000000in}{4.000000in}}%
\pgfpathclose%
\pgfusepath{fill}%
\end{pgfscope}%
\begin{pgfscope}%
\pgfsetbuttcap%
\pgfsetmiterjoin%
\definecolor{currentfill}{rgb}{1.000000,1.000000,1.000000}%
\pgfsetfillcolor{currentfill}%
\pgfsetlinewidth{0.000000pt}%
\definecolor{currentstroke}{rgb}{0.000000,0.000000,0.000000}%
\pgfsetstrokecolor{currentstroke}%
\pgfsetstrokeopacity{0.000000}%
\pgfsetdash{}{0pt}%
\pgfpathmoveto{\pgfqpoint{0.648703in}{0.548769in}}%
\pgfpathlineto{\pgfqpoint{5.850000in}{0.548769in}}%
\pgfpathlineto{\pgfqpoint{5.850000in}{3.651359in}}%
\pgfpathlineto{\pgfqpoint{0.648703in}{3.651359in}}%
\pgfpathclose%
\pgfusepath{fill}%
\end{pgfscope}%
\begin{pgfscope}%
\pgfpathrectangle{\pgfqpoint{0.648703in}{0.548769in}}{\pgfqpoint{5.201297in}{3.102590in}}%
\pgfusepath{clip}%
\pgfsetbuttcap%
\pgfsetroundjoin%
\definecolor{currentfill}{rgb}{0.121569,0.466667,0.705882}%
\pgfsetfillcolor{currentfill}%
\pgfsetlinewidth{1.003750pt}%
\definecolor{currentstroke}{rgb}{0.121569,0.466667,0.705882}%
\pgfsetstrokecolor{currentstroke}%
\pgfsetdash{}{0pt}%
\pgfpathmoveto{\pgfqpoint{0.962642in}{0.673501in}}%
\pgfpathcurveto{\pgfqpoint{0.973692in}{0.673501in}}{\pgfqpoint{0.984291in}{0.677891in}}{\pgfqpoint{0.992104in}{0.685705in}}%
\pgfpathcurveto{\pgfqpoint{0.999918in}{0.693519in}}{\pgfqpoint{1.004308in}{0.704118in}}{\pgfqpoint{1.004308in}{0.715168in}}%
\pgfpathcurveto{\pgfqpoint{1.004308in}{0.726218in}}{\pgfqpoint{0.999918in}{0.736817in}}{\pgfqpoint{0.992104in}{0.744631in}}%
\pgfpathcurveto{\pgfqpoint{0.984291in}{0.752444in}}{\pgfqpoint{0.973692in}{0.756834in}}{\pgfqpoint{0.962642in}{0.756834in}}%
\pgfpathcurveto{\pgfqpoint{0.951591in}{0.756834in}}{\pgfqpoint{0.940992in}{0.752444in}}{\pgfqpoint{0.933179in}{0.744631in}}%
\pgfpathcurveto{\pgfqpoint{0.925365in}{0.736817in}}{\pgfqpoint{0.920975in}{0.726218in}}{\pgfqpoint{0.920975in}{0.715168in}}%
\pgfpathcurveto{\pgfqpoint{0.920975in}{0.704118in}}{\pgfqpoint{0.925365in}{0.693519in}}{\pgfqpoint{0.933179in}{0.685705in}}%
\pgfpathcurveto{\pgfqpoint{0.940992in}{0.677891in}}{\pgfqpoint{0.951591in}{0.673501in}}{\pgfqpoint{0.962642in}{0.673501in}}%
\pgfpathclose%
\pgfusepath{stroke,fill}%
\end{pgfscope}%
\begin{pgfscope}%
\pgfpathrectangle{\pgfqpoint{0.648703in}{0.548769in}}{\pgfqpoint{5.201297in}{3.102590in}}%
\pgfusepath{clip}%
\pgfsetbuttcap%
\pgfsetroundjoin%
\definecolor{currentfill}{rgb}{1.000000,0.498039,0.054902}%
\pgfsetfillcolor{currentfill}%
\pgfsetlinewidth{1.003750pt}%
\definecolor{currentstroke}{rgb}{1.000000,0.498039,0.054902}%
\pgfsetstrokecolor{currentstroke}%
\pgfsetdash{}{0pt}%
\pgfpathmoveto{\pgfqpoint{2.745500in}{3.185343in}}%
\pgfpathcurveto{\pgfqpoint{2.756550in}{3.185343in}}{\pgfqpoint{2.767149in}{3.189733in}}{\pgfqpoint{2.774963in}{3.197547in}}%
\pgfpathcurveto{\pgfqpoint{2.782777in}{3.205360in}}{\pgfqpoint{2.787167in}{3.215959in}}{\pgfqpoint{2.787167in}{3.227010in}}%
\pgfpathcurveto{\pgfqpoint{2.787167in}{3.238060in}}{\pgfqpoint{2.782777in}{3.248659in}}{\pgfqpoint{2.774963in}{3.256472in}}%
\pgfpathcurveto{\pgfqpoint{2.767149in}{3.264286in}}{\pgfqpoint{2.756550in}{3.268676in}}{\pgfqpoint{2.745500in}{3.268676in}}%
\pgfpathcurveto{\pgfqpoint{2.734450in}{3.268676in}}{\pgfqpoint{2.723851in}{3.264286in}}{\pgfqpoint{2.716038in}{3.256472in}}%
\pgfpathcurveto{\pgfqpoint{2.708224in}{3.248659in}}{\pgfqpoint{2.703834in}{3.238060in}}{\pgfqpoint{2.703834in}{3.227010in}}%
\pgfpathcurveto{\pgfqpoint{2.703834in}{3.215959in}}{\pgfqpoint{2.708224in}{3.205360in}}{\pgfqpoint{2.716038in}{3.197547in}}%
\pgfpathcurveto{\pgfqpoint{2.723851in}{3.189733in}}{\pgfqpoint{2.734450in}{3.185343in}}{\pgfqpoint{2.745500in}{3.185343in}}%
\pgfpathclose%
\pgfusepath{stroke,fill}%
\end{pgfscope}%
\begin{pgfscope}%
\pgfpathrectangle{\pgfqpoint{0.648703in}{0.548769in}}{\pgfqpoint{5.201297in}{3.102590in}}%
\pgfusepath{clip}%
\pgfsetbuttcap%
\pgfsetroundjoin%
\definecolor{currentfill}{rgb}{0.121569,0.466667,0.705882}%
\pgfsetfillcolor{currentfill}%
\pgfsetlinewidth{1.003750pt}%
\definecolor{currentstroke}{rgb}{0.121569,0.466667,0.705882}%
\pgfsetstrokecolor{currentstroke}%
\pgfsetdash{}{0pt}%
\pgfpathmoveto{\pgfqpoint{1.427735in}{0.652358in}}%
\pgfpathcurveto{\pgfqpoint{1.438785in}{0.652358in}}{\pgfqpoint{1.449384in}{0.656748in}}{\pgfqpoint{1.457198in}{0.664562in}}%
\pgfpathcurveto{\pgfqpoint{1.465012in}{0.672375in}}{\pgfqpoint{1.469402in}{0.682974in}}{\pgfqpoint{1.469402in}{0.694024in}}%
\pgfpathcurveto{\pgfqpoint{1.469402in}{0.705074in}}{\pgfqpoint{1.465012in}{0.715673in}}{\pgfqpoint{1.457198in}{0.723487in}}%
\pgfpathcurveto{\pgfqpoint{1.449384in}{0.731301in}}{\pgfqpoint{1.438785in}{0.735691in}}{\pgfqpoint{1.427735in}{0.735691in}}%
\pgfpathcurveto{\pgfqpoint{1.416685in}{0.735691in}}{\pgfqpoint{1.406086in}{0.731301in}}{\pgfqpoint{1.398272in}{0.723487in}}%
\pgfpathcurveto{\pgfqpoint{1.390459in}{0.715673in}}{\pgfqpoint{1.386069in}{0.705074in}}{\pgfqpoint{1.386069in}{0.694024in}}%
\pgfpathcurveto{\pgfqpoint{1.386069in}{0.682974in}}{\pgfqpoint{1.390459in}{0.672375in}}{\pgfqpoint{1.398272in}{0.664562in}}%
\pgfpathcurveto{\pgfqpoint{1.406086in}{0.656748in}}{\pgfqpoint{1.416685in}{0.652358in}}{\pgfqpoint{1.427735in}{0.652358in}}%
\pgfpathclose%
\pgfusepath{stroke,fill}%
\end{pgfscope}%
\begin{pgfscope}%
\pgfpathrectangle{\pgfqpoint{0.648703in}{0.548769in}}{\pgfqpoint{5.201297in}{3.102590in}}%
\pgfusepath{clip}%
\pgfsetbuttcap%
\pgfsetroundjoin%
\definecolor{currentfill}{rgb}{0.121569,0.466667,0.705882}%
\pgfsetfillcolor{currentfill}%
\pgfsetlinewidth{1.003750pt}%
\definecolor{currentstroke}{rgb}{0.121569,0.466667,0.705882}%
\pgfsetstrokecolor{currentstroke}%
\pgfsetdash{}{0pt}%
\pgfpathmoveto{\pgfqpoint{4.760906in}{3.181114in}}%
\pgfpathcurveto{\pgfqpoint{4.771956in}{3.181114in}}{\pgfqpoint{4.782555in}{3.185504in}}{\pgfqpoint{4.790369in}{3.193318in}}%
\pgfpathcurveto{\pgfqpoint{4.798182in}{3.201132in}}{\pgfqpoint{4.802573in}{3.211731in}}{\pgfqpoint{4.802573in}{3.222781in}}%
\pgfpathcurveto{\pgfqpoint{4.802573in}{3.233831in}}{\pgfqpoint{4.798182in}{3.244430in}}{\pgfqpoint{4.790369in}{3.252244in}}%
\pgfpathcurveto{\pgfqpoint{4.782555in}{3.260057in}}{\pgfqpoint{4.771956in}{3.264448in}}{\pgfqpoint{4.760906in}{3.264448in}}%
\pgfpathcurveto{\pgfqpoint{4.749856in}{3.264448in}}{\pgfqpoint{4.739257in}{3.260057in}}{\pgfqpoint{4.731443in}{3.252244in}}%
\pgfpathcurveto{\pgfqpoint{4.723629in}{3.244430in}}{\pgfqpoint{4.719239in}{3.233831in}}{\pgfqpoint{4.719239in}{3.222781in}}%
\pgfpathcurveto{\pgfqpoint{4.719239in}{3.211731in}}{\pgfqpoint{4.723629in}{3.201132in}}{\pgfqpoint{4.731443in}{3.193318in}}%
\pgfpathcurveto{\pgfqpoint{4.739257in}{3.185504in}}{\pgfqpoint{4.749856in}{3.181114in}}{\pgfqpoint{4.760906in}{3.181114in}}%
\pgfpathclose%
\pgfusepath{stroke,fill}%
\end{pgfscope}%
\begin{pgfscope}%
\pgfpathrectangle{\pgfqpoint{0.648703in}{0.548769in}}{\pgfqpoint{5.201297in}{3.102590in}}%
\pgfusepath{clip}%
\pgfsetbuttcap%
\pgfsetroundjoin%
\definecolor{currentfill}{rgb}{1.000000,0.498039,0.054902}%
\pgfsetfillcolor{currentfill}%
\pgfsetlinewidth{1.003750pt}%
\definecolor{currentstroke}{rgb}{1.000000,0.498039,0.054902}%
\pgfsetstrokecolor{currentstroke}%
\pgfsetdash{}{0pt}%
\pgfpathmoveto{\pgfqpoint{3.365625in}{3.198029in}}%
\pgfpathcurveto{\pgfqpoint{3.376675in}{3.198029in}}{\pgfqpoint{3.387274in}{3.202419in}}{\pgfqpoint{3.395088in}{3.210233in}}%
\pgfpathcurveto{\pgfqpoint{3.402902in}{3.218046in}}{\pgfqpoint{3.407292in}{3.228646in}}{\pgfqpoint{3.407292in}{3.239696in}}%
\pgfpathcurveto{\pgfqpoint{3.407292in}{3.250746in}}{\pgfqpoint{3.402902in}{3.261345in}}{\pgfqpoint{3.395088in}{3.269158in}}%
\pgfpathcurveto{\pgfqpoint{3.387274in}{3.276972in}}{\pgfqpoint{3.376675in}{3.281362in}}{\pgfqpoint{3.365625in}{3.281362in}}%
\pgfpathcurveto{\pgfqpoint{3.354575in}{3.281362in}}{\pgfqpoint{3.343976in}{3.276972in}}{\pgfqpoint{3.336162in}{3.269158in}}%
\pgfpathcurveto{\pgfqpoint{3.328349in}{3.261345in}}{\pgfqpoint{3.323958in}{3.250746in}}{\pgfqpoint{3.323958in}{3.239696in}}%
\pgfpathcurveto{\pgfqpoint{3.323958in}{3.228646in}}{\pgfqpoint{3.328349in}{3.218046in}}{\pgfqpoint{3.336162in}{3.210233in}}%
\pgfpathcurveto{\pgfqpoint{3.343976in}{3.202419in}}{\pgfqpoint{3.354575in}{3.198029in}}{\pgfqpoint{3.365625in}{3.198029in}}%
\pgfpathclose%
\pgfusepath{stroke,fill}%
\end{pgfscope}%
\begin{pgfscope}%
\pgfpathrectangle{\pgfqpoint{0.648703in}{0.548769in}}{\pgfqpoint{5.201297in}{3.102590in}}%
\pgfusepath{clip}%
\pgfsetbuttcap%
\pgfsetroundjoin%
\definecolor{currentfill}{rgb}{1.000000,0.498039,0.054902}%
\pgfsetfillcolor{currentfill}%
\pgfsetlinewidth{1.003750pt}%
\definecolor{currentstroke}{rgb}{1.000000,0.498039,0.054902}%
\pgfsetstrokecolor{currentstroke}%
\pgfsetdash{}{0pt}%
\pgfpathmoveto{\pgfqpoint{1.892829in}{3.248773in}}%
\pgfpathcurveto{\pgfqpoint{1.903879in}{3.248773in}}{\pgfqpoint{1.914478in}{3.253164in}}{\pgfqpoint{1.922292in}{3.260977in}}%
\pgfpathcurveto{\pgfqpoint{1.930105in}{3.268791in}}{\pgfqpoint{1.934495in}{3.279390in}}{\pgfqpoint{1.934495in}{3.290440in}}%
\pgfpathcurveto{\pgfqpoint{1.934495in}{3.301490in}}{\pgfqpoint{1.930105in}{3.312089in}}{\pgfqpoint{1.922292in}{3.319903in}}%
\pgfpathcurveto{\pgfqpoint{1.914478in}{3.327716in}}{\pgfqpoint{1.903879in}{3.332107in}}{\pgfqpoint{1.892829in}{3.332107in}}%
\pgfpathcurveto{\pgfqpoint{1.881779in}{3.332107in}}{\pgfqpoint{1.871180in}{3.327716in}}{\pgfqpoint{1.863366in}{3.319903in}}%
\pgfpathcurveto{\pgfqpoint{1.855552in}{3.312089in}}{\pgfqpoint{1.851162in}{3.301490in}}{\pgfqpoint{1.851162in}{3.290440in}}%
\pgfpathcurveto{\pgfqpoint{1.851162in}{3.279390in}}{\pgfqpoint{1.855552in}{3.268791in}}{\pgfqpoint{1.863366in}{3.260977in}}%
\pgfpathcurveto{\pgfqpoint{1.871180in}{3.253164in}}{\pgfqpoint{1.881779in}{3.248773in}}{\pgfqpoint{1.892829in}{3.248773in}}%
\pgfpathclose%
\pgfusepath{stroke,fill}%
\end{pgfscope}%
\begin{pgfscope}%
\pgfpathrectangle{\pgfqpoint{0.648703in}{0.548769in}}{\pgfqpoint{5.201297in}{3.102590in}}%
\pgfusepath{clip}%
\pgfsetbuttcap%
\pgfsetroundjoin%
\definecolor{currentfill}{rgb}{1.000000,0.498039,0.054902}%
\pgfsetfillcolor{currentfill}%
\pgfsetlinewidth{1.003750pt}%
\definecolor{currentstroke}{rgb}{1.000000,0.498039,0.054902}%
\pgfsetstrokecolor{currentstroke}%
\pgfsetdash{}{0pt}%
\pgfpathmoveto{\pgfqpoint{2.047860in}{3.189572in}}%
\pgfpathcurveto{\pgfqpoint{2.058910in}{3.189572in}}{\pgfqpoint{2.069509in}{3.193962in}}{\pgfqpoint{2.077323in}{3.201775in}}%
\pgfpathcurveto{\pgfqpoint{2.085136in}{3.209589in}}{\pgfqpoint{2.089527in}{3.220188in}}{\pgfqpoint{2.089527in}{3.231238in}}%
\pgfpathcurveto{\pgfqpoint{2.089527in}{3.242288in}}{\pgfqpoint{2.085136in}{3.252887in}}{\pgfqpoint{2.077323in}{3.260701in}}%
\pgfpathcurveto{\pgfqpoint{2.069509in}{3.268515in}}{\pgfqpoint{2.058910in}{3.272905in}}{\pgfqpoint{2.047860in}{3.272905in}}%
\pgfpathcurveto{\pgfqpoint{2.036810in}{3.272905in}}{\pgfqpoint{2.026211in}{3.268515in}}{\pgfqpoint{2.018397in}{3.260701in}}%
\pgfpathcurveto{\pgfqpoint{2.010584in}{3.252887in}}{\pgfqpoint{2.006193in}{3.242288in}}{\pgfqpoint{2.006193in}{3.231238in}}%
\pgfpathcurveto{\pgfqpoint{2.006193in}{3.220188in}}{\pgfqpoint{2.010584in}{3.209589in}}{\pgfqpoint{2.018397in}{3.201775in}}%
\pgfpathcurveto{\pgfqpoint{2.026211in}{3.193962in}}{\pgfqpoint{2.036810in}{3.189572in}}{\pgfqpoint{2.047860in}{3.189572in}}%
\pgfpathclose%
\pgfusepath{stroke,fill}%
\end{pgfscope}%
\begin{pgfscope}%
\pgfpathrectangle{\pgfqpoint{0.648703in}{0.548769in}}{\pgfqpoint{5.201297in}{3.102590in}}%
\pgfusepath{clip}%
\pgfsetbuttcap%
\pgfsetroundjoin%
\definecolor{currentfill}{rgb}{1.000000,0.498039,0.054902}%
\pgfsetfillcolor{currentfill}%
\pgfsetlinewidth{1.003750pt}%
\definecolor{currentstroke}{rgb}{1.000000,0.498039,0.054902}%
\pgfsetstrokecolor{currentstroke}%
\pgfsetdash{}{0pt}%
\pgfpathmoveto{\pgfqpoint{1.272704in}{3.189572in}}%
\pgfpathcurveto{\pgfqpoint{1.283754in}{3.189572in}}{\pgfqpoint{1.294353in}{3.193962in}}{\pgfqpoint{1.302167in}{3.201775in}}%
\pgfpathcurveto{\pgfqpoint{1.309980in}{3.209589in}}{\pgfqpoint{1.314371in}{3.220188in}}{\pgfqpoint{1.314371in}{3.231238in}}%
\pgfpathcurveto{\pgfqpoint{1.314371in}{3.242288in}}{\pgfqpoint{1.309980in}{3.252887in}}{\pgfqpoint{1.302167in}{3.260701in}}%
\pgfpathcurveto{\pgfqpoint{1.294353in}{3.268515in}}{\pgfqpoint{1.283754in}{3.272905in}}{\pgfqpoint{1.272704in}{3.272905in}}%
\pgfpathcurveto{\pgfqpoint{1.261654in}{3.272905in}}{\pgfqpoint{1.251055in}{3.268515in}}{\pgfqpoint{1.243241in}{3.260701in}}%
\pgfpathcurveto{\pgfqpoint{1.235428in}{3.252887in}}{\pgfqpoint{1.231037in}{3.242288in}}{\pgfqpoint{1.231037in}{3.231238in}}%
\pgfpathcurveto{\pgfqpoint{1.231037in}{3.220188in}}{\pgfqpoint{1.235428in}{3.209589in}}{\pgfqpoint{1.243241in}{3.201775in}}%
\pgfpathcurveto{\pgfqpoint{1.251055in}{3.193962in}}{\pgfqpoint{1.261654in}{3.189572in}}{\pgfqpoint{1.272704in}{3.189572in}}%
\pgfpathclose%
\pgfusepath{stroke,fill}%
\end{pgfscope}%
\begin{pgfscope}%
\pgfpathrectangle{\pgfqpoint{0.648703in}{0.548769in}}{\pgfqpoint{5.201297in}{3.102590in}}%
\pgfusepath{clip}%
\pgfsetbuttcap%
\pgfsetroundjoin%
\definecolor{currentfill}{rgb}{0.121569,0.466667,0.705882}%
\pgfsetfillcolor{currentfill}%
\pgfsetlinewidth{1.003750pt}%
\definecolor{currentstroke}{rgb}{0.121569,0.466667,0.705882}%
\pgfsetstrokecolor{currentstroke}%
\pgfsetdash{}{0pt}%
\pgfpathmoveto{\pgfqpoint{4.450843in}{3.181114in}}%
\pgfpathcurveto{\pgfqpoint{4.461894in}{3.181114in}}{\pgfqpoint{4.472493in}{3.185504in}}{\pgfqpoint{4.480306in}{3.193318in}}%
\pgfpathcurveto{\pgfqpoint{4.488120in}{3.201132in}}{\pgfqpoint{4.492510in}{3.211731in}}{\pgfqpoint{4.492510in}{3.222781in}}%
\pgfpathcurveto{\pgfqpoint{4.492510in}{3.233831in}}{\pgfqpoint{4.488120in}{3.244430in}}{\pgfqpoint{4.480306in}{3.252244in}}%
\pgfpathcurveto{\pgfqpoint{4.472493in}{3.260057in}}{\pgfqpoint{4.461894in}{3.264448in}}{\pgfqpoint{4.450843in}{3.264448in}}%
\pgfpathcurveto{\pgfqpoint{4.439793in}{3.264448in}}{\pgfqpoint{4.429194in}{3.260057in}}{\pgfqpoint{4.421381in}{3.252244in}}%
\pgfpathcurveto{\pgfqpoint{4.413567in}{3.244430in}}{\pgfqpoint{4.409177in}{3.233831in}}{\pgfqpoint{4.409177in}{3.222781in}}%
\pgfpathcurveto{\pgfqpoint{4.409177in}{3.211731in}}{\pgfqpoint{4.413567in}{3.201132in}}{\pgfqpoint{4.421381in}{3.193318in}}%
\pgfpathcurveto{\pgfqpoint{4.429194in}{3.185504in}}{\pgfqpoint{4.439793in}{3.181114in}}{\pgfqpoint{4.450843in}{3.181114in}}%
\pgfpathclose%
\pgfusepath{stroke,fill}%
\end{pgfscope}%
\begin{pgfscope}%
\pgfpathrectangle{\pgfqpoint{0.648703in}{0.548769in}}{\pgfqpoint{5.201297in}{3.102590in}}%
\pgfusepath{clip}%
\pgfsetbuttcap%
\pgfsetroundjoin%
\definecolor{currentfill}{rgb}{0.121569,0.466667,0.705882}%
\pgfsetfillcolor{currentfill}%
\pgfsetlinewidth{1.003750pt}%
\definecolor{currentstroke}{rgb}{0.121569,0.466667,0.705882}%
\pgfsetstrokecolor{currentstroke}%
\pgfsetdash{}{0pt}%
\pgfpathmoveto{\pgfqpoint{1.737798in}{0.648129in}}%
\pgfpathcurveto{\pgfqpoint{1.748848in}{0.648129in}}{\pgfqpoint{1.759447in}{0.652519in}}{\pgfqpoint{1.767260in}{0.660333in}}%
\pgfpathcurveto{\pgfqpoint{1.775074in}{0.668146in}}{\pgfqpoint{1.779464in}{0.678745in}}{\pgfqpoint{1.779464in}{0.689796in}}%
\pgfpathcurveto{\pgfqpoint{1.779464in}{0.700846in}}{\pgfqpoint{1.775074in}{0.711445in}}{\pgfqpoint{1.767260in}{0.719258in}}%
\pgfpathcurveto{\pgfqpoint{1.759447in}{0.727072in}}{\pgfqpoint{1.748848in}{0.731462in}}{\pgfqpoint{1.737798in}{0.731462in}}%
\pgfpathcurveto{\pgfqpoint{1.726747in}{0.731462in}}{\pgfqpoint{1.716148in}{0.727072in}}{\pgfqpoint{1.708335in}{0.719258in}}%
\pgfpathcurveto{\pgfqpoint{1.700521in}{0.711445in}}{\pgfqpoint{1.696131in}{0.700846in}}{\pgfqpoint{1.696131in}{0.689796in}}%
\pgfpathcurveto{\pgfqpoint{1.696131in}{0.678745in}}{\pgfqpoint{1.700521in}{0.668146in}}{\pgfqpoint{1.708335in}{0.660333in}}%
\pgfpathcurveto{\pgfqpoint{1.716148in}{0.652519in}}{\pgfqpoint{1.726747in}{0.648129in}}{\pgfqpoint{1.737798in}{0.648129in}}%
\pgfpathclose%
\pgfusepath{stroke,fill}%
\end{pgfscope}%
\begin{pgfscope}%
\pgfpathrectangle{\pgfqpoint{0.648703in}{0.548769in}}{\pgfqpoint{5.201297in}{3.102590in}}%
\pgfusepath{clip}%
\pgfsetbuttcap%
\pgfsetroundjoin%
\definecolor{currentfill}{rgb}{0.839216,0.152941,0.156863}%
\pgfsetfillcolor{currentfill}%
\pgfsetlinewidth{1.003750pt}%
\definecolor{currentstroke}{rgb}{0.839216,0.152941,0.156863}%
\pgfsetstrokecolor{currentstroke}%
\pgfsetdash{}{0pt}%
\pgfpathmoveto{\pgfqpoint{4.373328in}{3.202258in}}%
\pgfpathcurveto{\pgfqpoint{4.384378in}{3.202258in}}{\pgfqpoint{4.394977in}{3.206648in}}{\pgfqpoint{4.402791in}{3.214462in}}%
\pgfpathcurveto{\pgfqpoint{4.410604in}{3.222275in}}{\pgfqpoint{4.414995in}{3.232874in}}{\pgfqpoint{4.414995in}{3.243924in}}%
\pgfpathcurveto{\pgfqpoint{4.414995in}{3.254974in}}{\pgfqpoint{4.410604in}{3.265573in}}{\pgfqpoint{4.402791in}{3.273387in}}%
\pgfpathcurveto{\pgfqpoint{4.394977in}{3.281201in}}{\pgfqpoint{4.384378in}{3.285591in}}{\pgfqpoint{4.373328in}{3.285591in}}%
\pgfpathcurveto{\pgfqpoint{4.362278in}{3.285591in}}{\pgfqpoint{4.351679in}{3.281201in}}{\pgfqpoint{4.343865in}{3.273387in}}%
\pgfpathcurveto{\pgfqpoint{4.336051in}{3.265573in}}{\pgfqpoint{4.331661in}{3.254974in}}{\pgfqpoint{4.331661in}{3.243924in}}%
\pgfpathcurveto{\pgfqpoint{4.331661in}{3.232874in}}{\pgfqpoint{4.336051in}{3.222275in}}{\pgfqpoint{4.343865in}{3.214462in}}%
\pgfpathcurveto{\pgfqpoint{4.351679in}{3.206648in}}{\pgfqpoint{4.362278in}{3.202258in}}{\pgfqpoint{4.373328in}{3.202258in}}%
\pgfpathclose%
\pgfusepath{stroke,fill}%
\end{pgfscope}%
\begin{pgfscope}%
\pgfpathrectangle{\pgfqpoint{0.648703in}{0.548769in}}{\pgfqpoint{5.201297in}{3.102590in}}%
\pgfusepath{clip}%
\pgfsetbuttcap%
\pgfsetroundjoin%
\definecolor{currentfill}{rgb}{1.000000,0.498039,0.054902}%
\pgfsetfillcolor{currentfill}%
\pgfsetlinewidth{1.003750pt}%
\definecolor{currentstroke}{rgb}{1.000000,0.498039,0.054902}%
\pgfsetstrokecolor{currentstroke}%
\pgfsetdash{}{0pt}%
\pgfpathmoveto{\pgfqpoint{1.505251in}{3.185343in}}%
\pgfpathcurveto{\pgfqpoint{1.516301in}{3.185343in}}{\pgfqpoint{1.526900in}{3.189733in}}{\pgfqpoint{1.534714in}{3.197547in}}%
\pgfpathcurveto{\pgfqpoint{1.542527in}{3.205360in}}{\pgfqpoint{1.546917in}{3.215959in}}{\pgfqpoint{1.546917in}{3.227010in}}%
\pgfpathcurveto{\pgfqpoint{1.546917in}{3.238060in}}{\pgfqpoint{1.542527in}{3.248659in}}{\pgfqpoint{1.534714in}{3.256472in}}%
\pgfpathcurveto{\pgfqpoint{1.526900in}{3.264286in}}{\pgfqpoint{1.516301in}{3.268676in}}{\pgfqpoint{1.505251in}{3.268676in}}%
\pgfpathcurveto{\pgfqpoint{1.494201in}{3.268676in}}{\pgfqpoint{1.483602in}{3.264286in}}{\pgfqpoint{1.475788in}{3.256472in}}%
\pgfpathcurveto{\pgfqpoint{1.467974in}{3.248659in}}{\pgfqpoint{1.463584in}{3.238060in}}{\pgfqpoint{1.463584in}{3.227010in}}%
\pgfpathcurveto{\pgfqpoint{1.463584in}{3.215959in}}{\pgfqpoint{1.467974in}{3.205360in}}{\pgfqpoint{1.475788in}{3.197547in}}%
\pgfpathcurveto{\pgfqpoint{1.483602in}{3.189733in}}{\pgfqpoint{1.494201in}{3.185343in}}{\pgfqpoint{1.505251in}{3.185343in}}%
\pgfpathclose%
\pgfusepath{stroke,fill}%
\end{pgfscope}%
\begin{pgfscope}%
\pgfpathrectangle{\pgfqpoint{0.648703in}{0.548769in}}{\pgfqpoint{5.201297in}{3.102590in}}%
\pgfusepath{clip}%
\pgfsetbuttcap%
\pgfsetroundjoin%
\definecolor{currentfill}{rgb}{0.121569,0.466667,0.705882}%
\pgfsetfillcolor{currentfill}%
\pgfsetlinewidth{1.003750pt}%
\definecolor{currentstroke}{rgb}{0.121569,0.466667,0.705882}%
\pgfsetstrokecolor{currentstroke}%
\pgfsetdash{}{0pt}%
\pgfpathmoveto{\pgfqpoint{1.350220in}{0.774990in}}%
\pgfpathcurveto{\pgfqpoint{1.361270in}{0.774990in}}{\pgfqpoint{1.371869in}{0.779380in}}{\pgfqpoint{1.379682in}{0.787194in}}%
\pgfpathcurveto{\pgfqpoint{1.387496in}{0.795007in}}{\pgfqpoint{1.391886in}{0.805606in}}{\pgfqpoint{1.391886in}{0.816656in}}%
\pgfpathcurveto{\pgfqpoint{1.391886in}{0.827706in}}{\pgfqpoint{1.387496in}{0.838305in}}{\pgfqpoint{1.379682in}{0.846119in}}%
\pgfpathcurveto{\pgfqpoint{1.371869in}{0.853933in}}{\pgfqpoint{1.361270in}{0.858323in}}{\pgfqpoint{1.350220in}{0.858323in}}%
\pgfpathcurveto{\pgfqpoint{1.339169in}{0.858323in}}{\pgfqpoint{1.328570in}{0.853933in}}{\pgfqpoint{1.320757in}{0.846119in}}%
\pgfpathcurveto{\pgfqpoint{1.312943in}{0.838305in}}{\pgfqpoint{1.308553in}{0.827706in}}{\pgfqpoint{1.308553in}{0.816656in}}%
\pgfpathcurveto{\pgfqpoint{1.308553in}{0.805606in}}{\pgfqpoint{1.312943in}{0.795007in}}{\pgfqpoint{1.320757in}{0.787194in}}%
\pgfpathcurveto{\pgfqpoint{1.328570in}{0.779380in}}{\pgfqpoint{1.339169in}{0.774990in}}{\pgfqpoint{1.350220in}{0.774990in}}%
\pgfpathclose%
\pgfusepath{stroke,fill}%
\end{pgfscope}%
\begin{pgfscope}%
\pgfpathrectangle{\pgfqpoint{0.648703in}{0.548769in}}{\pgfqpoint{5.201297in}{3.102590in}}%
\pgfusepath{clip}%
\pgfsetbuttcap%
\pgfsetroundjoin%
\definecolor{currentfill}{rgb}{0.121569,0.466667,0.705882}%
\pgfsetfillcolor{currentfill}%
\pgfsetlinewidth{1.003750pt}%
\definecolor{currentstroke}{rgb}{0.121569,0.466667,0.705882}%
\pgfsetstrokecolor{currentstroke}%
\pgfsetdash{}{0pt}%
\pgfpathmoveto{\pgfqpoint{2.978047in}{3.181114in}}%
\pgfpathcurveto{\pgfqpoint{2.989097in}{3.181114in}}{\pgfqpoint{2.999696in}{3.185504in}}{\pgfqpoint{3.007510in}{3.193318in}}%
\pgfpathcurveto{\pgfqpoint{3.015324in}{3.201132in}}{\pgfqpoint{3.019714in}{3.211731in}}{\pgfqpoint{3.019714in}{3.222781in}}%
\pgfpathcurveto{\pgfqpoint{3.019714in}{3.233831in}}{\pgfqpoint{3.015324in}{3.244430in}}{\pgfqpoint{3.007510in}{3.252244in}}%
\pgfpathcurveto{\pgfqpoint{2.999696in}{3.260057in}}{\pgfqpoint{2.989097in}{3.264448in}}{\pgfqpoint{2.978047in}{3.264448in}}%
\pgfpathcurveto{\pgfqpoint{2.966997in}{3.264448in}}{\pgfqpoint{2.956398in}{3.260057in}}{\pgfqpoint{2.948584in}{3.252244in}}%
\pgfpathcurveto{\pgfqpoint{2.940771in}{3.244430in}}{\pgfqpoint{2.936380in}{3.233831in}}{\pgfqpoint{2.936380in}{3.222781in}}%
\pgfpathcurveto{\pgfqpoint{2.936380in}{3.211731in}}{\pgfqpoint{2.940771in}{3.201132in}}{\pgfqpoint{2.948584in}{3.193318in}}%
\pgfpathcurveto{\pgfqpoint{2.956398in}{3.185504in}}{\pgfqpoint{2.966997in}{3.181114in}}{\pgfqpoint{2.978047in}{3.181114in}}%
\pgfpathclose%
\pgfusepath{stroke,fill}%
\end{pgfscope}%
\begin{pgfscope}%
\pgfpathrectangle{\pgfqpoint{0.648703in}{0.548769in}}{\pgfqpoint{5.201297in}{3.102590in}}%
\pgfusepath{clip}%
\pgfsetbuttcap%
\pgfsetroundjoin%
\definecolor{currentfill}{rgb}{0.121569,0.466667,0.705882}%
\pgfsetfillcolor{currentfill}%
\pgfsetlinewidth{1.003750pt}%
\definecolor{currentstroke}{rgb}{0.121569,0.466667,0.705882}%
\pgfsetstrokecolor{currentstroke}%
\pgfsetdash{}{0pt}%
\pgfpathmoveto{\pgfqpoint{1.350220in}{0.783447in}}%
\pgfpathcurveto{\pgfqpoint{1.361270in}{0.783447in}}{\pgfqpoint{1.371869in}{0.787837in}}{\pgfqpoint{1.379682in}{0.795651in}}%
\pgfpathcurveto{\pgfqpoint{1.387496in}{0.803465in}}{\pgfqpoint{1.391886in}{0.814064in}}{\pgfqpoint{1.391886in}{0.825114in}}%
\pgfpathcurveto{\pgfqpoint{1.391886in}{0.836164in}}{\pgfqpoint{1.387496in}{0.846763in}}{\pgfqpoint{1.379682in}{0.854576in}}%
\pgfpathcurveto{\pgfqpoint{1.371869in}{0.862390in}}{\pgfqpoint{1.361270in}{0.866780in}}{\pgfqpoint{1.350220in}{0.866780in}}%
\pgfpathcurveto{\pgfqpoint{1.339169in}{0.866780in}}{\pgfqpoint{1.328570in}{0.862390in}}{\pgfqpoint{1.320757in}{0.854576in}}%
\pgfpathcurveto{\pgfqpoint{1.312943in}{0.846763in}}{\pgfqpoint{1.308553in}{0.836164in}}{\pgfqpoint{1.308553in}{0.825114in}}%
\pgfpathcurveto{\pgfqpoint{1.308553in}{0.814064in}}{\pgfqpoint{1.312943in}{0.803465in}}{\pgfqpoint{1.320757in}{0.795651in}}%
\pgfpathcurveto{\pgfqpoint{1.328570in}{0.787837in}}{\pgfqpoint{1.339169in}{0.783447in}}{\pgfqpoint{1.350220in}{0.783447in}}%
\pgfpathclose%
\pgfusepath{stroke,fill}%
\end{pgfscope}%
\begin{pgfscope}%
\pgfpathrectangle{\pgfqpoint{0.648703in}{0.548769in}}{\pgfqpoint{5.201297in}{3.102590in}}%
\pgfusepath{clip}%
\pgfsetbuttcap%
\pgfsetroundjoin%
\definecolor{currentfill}{rgb}{0.121569,0.466667,0.705882}%
\pgfsetfillcolor{currentfill}%
\pgfsetlinewidth{1.003750pt}%
\definecolor{currentstroke}{rgb}{0.121569,0.466667,0.705882}%
\pgfsetstrokecolor{currentstroke}%
\pgfsetdash{}{0pt}%
\pgfpathmoveto{\pgfqpoint{0.962642in}{0.652358in}}%
\pgfpathcurveto{\pgfqpoint{0.973692in}{0.652358in}}{\pgfqpoint{0.984291in}{0.656748in}}{\pgfqpoint{0.992104in}{0.664562in}}%
\pgfpathcurveto{\pgfqpoint{0.999918in}{0.672375in}}{\pgfqpoint{1.004308in}{0.682974in}}{\pgfqpoint{1.004308in}{0.694024in}}%
\pgfpathcurveto{\pgfqpoint{1.004308in}{0.705074in}}{\pgfqpoint{0.999918in}{0.715673in}}{\pgfqpoint{0.992104in}{0.723487in}}%
\pgfpathcurveto{\pgfqpoint{0.984291in}{0.731301in}}{\pgfqpoint{0.973692in}{0.735691in}}{\pgfqpoint{0.962642in}{0.735691in}}%
\pgfpathcurveto{\pgfqpoint{0.951591in}{0.735691in}}{\pgfqpoint{0.940992in}{0.731301in}}{\pgfqpoint{0.933179in}{0.723487in}}%
\pgfpathcurveto{\pgfqpoint{0.925365in}{0.715673in}}{\pgfqpoint{0.920975in}{0.705074in}}{\pgfqpoint{0.920975in}{0.694024in}}%
\pgfpathcurveto{\pgfqpoint{0.920975in}{0.682974in}}{\pgfqpoint{0.925365in}{0.672375in}}{\pgfqpoint{0.933179in}{0.664562in}}%
\pgfpathcurveto{\pgfqpoint{0.940992in}{0.656748in}}{\pgfqpoint{0.951591in}{0.652358in}}{\pgfqpoint{0.962642in}{0.652358in}}%
\pgfpathclose%
\pgfusepath{stroke,fill}%
\end{pgfscope}%
\begin{pgfscope}%
\pgfpathrectangle{\pgfqpoint{0.648703in}{0.548769in}}{\pgfqpoint{5.201297in}{3.102590in}}%
\pgfusepath{clip}%
\pgfsetbuttcap%
\pgfsetroundjoin%
\definecolor{currentfill}{rgb}{1.000000,0.498039,0.054902}%
\pgfsetfillcolor{currentfill}%
\pgfsetlinewidth{1.003750pt}%
\definecolor{currentstroke}{rgb}{1.000000,0.498039,0.054902}%
\pgfsetstrokecolor{currentstroke}%
\pgfsetdash{}{0pt}%
\pgfpathmoveto{\pgfqpoint{1.040157in}{3.219172in}}%
\pgfpathcurveto{\pgfqpoint{1.051207in}{3.219172in}}{\pgfqpoint{1.061806in}{3.223563in}}{\pgfqpoint{1.069620in}{3.231376in}}%
\pgfpathcurveto{\pgfqpoint{1.077434in}{3.239190in}}{\pgfqpoint{1.081824in}{3.249789in}}{\pgfqpoint{1.081824in}{3.260839in}}%
\pgfpathcurveto{\pgfqpoint{1.081824in}{3.271889in}}{\pgfqpoint{1.077434in}{3.282488in}}{\pgfqpoint{1.069620in}{3.290302in}}%
\pgfpathcurveto{\pgfqpoint{1.061806in}{3.298116in}}{\pgfqpoint{1.051207in}{3.302506in}}{\pgfqpoint{1.040157in}{3.302506in}}%
\pgfpathcurveto{\pgfqpoint{1.029107in}{3.302506in}}{\pgfqpoint{1.018508in}{3.298116in}}{\pgfqpoint{1.010694in}{3.290302in}}%
\pgfpathcurveto{\pgfqpoint{1.002881in}{3.282488in}}{\pgfqpoint{0.998491in}{3.271889in}}{\pgfqpoint{0.998491in}{3.260839in}}%
\pgfpathcurveto{\pgfqpoint{0.998491in}{3.249789in}}{\pgfqpoint{1.002881in}{3.239190in}}{\pgfqpoint{1.010694in}{3.231376in}}%
\pgfpathcurveto{\pgfqpoint{1.018508in}{3.223563in}}{\pgfqpoint{1.029107in}{3.219172in}}{\pgfqpoint{1.040157in}{3.219172in}}%
\pgfpathclose%
\pgfusepath{stroke,fill}%
\end{pgfscope}%
\begin{pgfscope}%
\pgfpathrectangle{\pgfqpoint{0.648703in}{0.548769in}}{\pgfqpoint{5.201297in}{3.102590in}}%
\pgfusepath{clip}%
\pgfsetbuttcap%
\pgfsetroundjoin%
\definecolor{currentfill}{rgb}{0.121569,0.466667,0.705882}%
\pgfsetfillcolor{currentfill}%
\pgfsetlinewidth{1.003750pt}%
\definecolor{currentstroke}{rgb}{0.121569,0.466667,0.705882}%
\pgfsetstrokecolor{currentstroke}%
\pgfsetdash{}{0pt}%
\pgfpathmoveto{\pgfqpoint{1.505251in}{0.648129in}}%
\pgfpathcurveto{\pgfqpoint{1.516301in}{0.648129in}}{\pgfqpoint{1.526900in}{0.652519in}}{\pgfqpoint{1.534714in}{0.660333in}}%
\pgfpathcurveto{\pgfqpoint{1.542527in}{0.668146in}}{\pgfqpoint{1.546917in}{0.678745in}}{\pgfqpoint{1.546917in}{0.689796in}}%
\pgfpathcurveto{\pgfqpoint{1.546917in}{0.700846in}}{\pgfqpoint{1.542527in}{0.711445in}}{\pgfqpoint{1.534714in}{0.719258in}}%
\pgfpathcurveto{\pgfqpoint{1.526900in}{0.727072in}}{\pgfqpoint{1.516301in}{0.731462in}}{\pgfqpoint{1.505251in}{0.731462in}}%
\pgfpathcurveto{\pgfqpoint{1.494201in}{0.731462in}}{\pgfqpoint{1.483602in}{0.727072in}}{\pgfqpoint{1.475788in}{0.719258in}}%
\pgfpathcurveto{\pgfqpoint{1.467974in}{0.711445in}}{\pgfqpoint{1.463584in}{0.700846in}}{\pgfqpoint{1.463584in}{0.689796in}}%
\pgfpathcurveto{\pgfqpoint{1.463584in}{0.678745in}}{\pgfqpoint{1.467974in}{0.668146in}}{\pgfqpoint{1.475788in}{0.660333in}}%
\pgfpathcurveto{\pgfqpoint{1.483602in}{0.652519in}}{\pgfqpoint{1.494201in}{0.648129in}}{\pgfqpoint{1.505251in}{0.648129in}}%
\pgfpathclose%
\pgfusepath{stroke,fill}%
\end{pgfscope}%
\begin{pgfscope}%
\pgfpathrectangle{\pgfqpoint{0.648703in}{0.548769in}}{\pgfqpoint{5.201297in}{3.102590in}}%
\pgfusepath{clip}%
\pgfsetbuttcap%
\pgfsetroundjoin%
\definecolor{currentfill}{rgb}{0.121569,0.466667,0.705882}%
\pgfsetfillcolor{currentfill}%
\pgfsetlinewidth{1.003750pt}%
\definecolor{currentstroke}{rgb}{0.121569,0.466667,0.705882}%
\pgfsetstrokecolor{currentstroke}%
\pgfsetdash{}{0pt}%
\pgfpathmoveto{\pgfqpoint{1.272704in}{0.648129in}}%
\pgfpathcurveto{\pgfqpoint{1.283754in}{0.648129in}}{\pgfqpoint{1.294353in}{0.652519in}}{\pgfqpoint{1.302167in}{0.660333in}}%
\pgfpathcurveto{\pgfqpoint{1.309980in}{0.668146in}}{\pgfqpoint{1.314371in}{0.678745in}}{\pgfqpoint{1.314371in}{0.689796in}}%
\pgfpathcurveto{\pgfqpoint{1.314371in}{0.700846in}}{\pgfqpoint{1.309980in}{0.711445in}}{\pgfqpoint{1.302167in}{0.719258in}}%
\pgfpathcurveto{\pgfqpoint{1.294353in}{0.727072in}}{\pgfqpoint{1.283754in}{0.731462in}}{\pgfqpoint{1.272704in}{0.731462in}}%
\pgfpathcurveto{\pgfqpoint{1.261654in}{0.731462in}}{\pgfqpoint{1.251055in}{0.727072in}}{\pgfqpoint{1.243241in}{0.719258in}}%
\pgfpathcurveto{\pgfqpoint{1.235428in}{0.711445in}}{\pgfqpoint{1.231037in}{0.700846in}}{\pgfqpoint{1.231037in}{0.689796in}}%
\pgfpathcurveto{\pgfqpoint{1.231037in}{0.678745in}}{\pgfqpoint{1.235428in}{0.668146in}}{\pgfqpoint{1.243241in}{0.660333in}}%
\pgfpathcurveto{\pgfqpoint{1.251055in}{0.652519in}}{\pgfqpoint{1.261654in}{0.648129in}}{\pgfqpoint{1.272704in}{0.648129in}}%
\pgfpathclose%
\pgfusepath{stroke,fill}%
\end{pgfscope}%
\begin{pgfscope}%
\pgfpathrectangle{\pgfqpoint{0.648703in}{0.548769in}}{\pgfqpoint{5.201297in}{3.102590in}}%
\pgfusepath{clip}%
\pgfsetbuttcap%
\pgfsetroundjoin%
\definecolor{currentfill}{rgb}{1.000000,0.498039,0.054902}%
\pgfsetfillcolor{currentfill}%
\pgfsetlinewidth{1.003750pt}%
\definecolor{currentstroke}{rgb}{1.000000,0.498039,0.054902}%
\pgfsetstrokecolor{currentstroke}%
\pgfsetdash{}{0pt}%
\pgfpathmoveto{\pgfqpoint{2.745500in}{3.468665in}}%
\pgfpathcurveto{\pgfqpoint{2.756550in}{3.468665in}}{\pgfqpoint{2.767149in}{3.473055in}}{\pgfqpoint{2.774963in}{3.480869in}}%
\pgfpathcurveto{\pgfqpoint{2.782777in}{3.488683in}}{\pgfqpoint{2.787167in}{3.499282in}}{\pgfqpoint{2.787167in}{3.510332in}}%
\pgfpathcurveto{\pgfqpoint{2.787167in}{3.521382in}}{\pgfqpoint{2.782777in}{3.531981in}}{\pgfqpoint{2.774963in}{3.539795in}}%
\pgfpathcurveto{\pgfqpoint{2.767149in}{3.547608in}}{\pgfqpoint{2.756550in}{3.551998in}}{\pgfqpoint{2.745500in}{3.551998in}}%
\pgfpathcurveto{\pgfqpoint{2.734450in}{3.551998in}}{\pgfqpoint{2.723851in}{3.547608in}}{\pgfqpoint{2.716038in}{3.539795in}}%
\pgfpathcurveto{\pgfqpoint{2.708224in}{3.531981in}}{\pgfqpoint{2.703834in}{3.521382in}}{\pgfqpoint{2.703834in}{3.510332in}}%
\pgfpathcurveto{\pgfqpoint{2.703834in}{3.499282in}}{\pgfqpoint{2.708224in}{3.488683in}}{\pgfqpoint{2.716038in}{3.480869in}}%
\pgfpathcurveto{\pgfqpoint{2.723851in}{3.473055in}}{\pgfqpoint{2.734450in}{3.468665in}}{\pgfqpoint{2.745500in}{3.468665in}}%
\pgfpathclose%
\pgfusepath{stroke,fill}%
\end{pgfscope}%
\begin{pgfscope}%
\pgfpathrectangle{\pgfqpoint{0.648703in}{0.548769in}}{\pgfqpoint{5.201297in}{3.102590in}}%
\pgfusepath{clip}%
\pgfsetbuttcap%
\pgfsetroundjoin%
\definecolor{currentfill}{rgb}{1.000000,0.498039,0.054902}%
\pgfsetfillcolor{currentfill}%
\pgfsetlinewidth{1.003750pt}%
\definecolor{currentstroke}{rgb}{1.000000,0.498039,0.054902}%
\pgfsetstrokecolor{currentstroke}%
\pgfsetdash{}{0pt}%
\pgfpathmoveto{\pgfqpoint{1.505251in}{3.193800in}}%
\pgfpathcurveto{\pgfqpoint{1.516301in}{3.193800in}}{\pgfqpoint{1.526900in}{3.198191in}}{\pgfqpoint{1.534714in}{3.206004in}}%
\pgfpathcurveto{\pgfqpoint{1.542527in}{3.213818in}}{\pgfqpoint{1.546917in}{3.224417in}}{\pgfqpoint{1.546917in}{3.235467in}}%
\pgfpathcurveto{\pgfqpoint{1.546917in}{3.246517in}}{\pgfqpoint{1.542527in}{3.257116in}}{\pgfqpoint{1.534714in}{3.264930in}}%
\pgfpathcurveto{\pgfqpoint{1.526900in}{3.272743in}}{\pgfqpoint{1.516301in}{3.277134in}}{\pgfqpoint{1.505251in}{3.277134in}}%
\pgfpathcurveto{\pgfqpoint{1.494201in}{3.277134in}}{\pgfqpoint{1.483602in}{3.272743in}}{\pgfqpoint{1.475788in}{3.264930in}}%
\pgfpathcurveto{\pgfqpoint{1.467974in}{3.257116in}}{\pgfqpoint{1.463584in}{3.246517in}}{\pgfqpoint{1.463584in}{3.235467in}}%
\pgfpathcurveto{\pgfqpoint{1.463584in}{3.224417in}}{\pgfqpoint{1.467974in}{3.213818in}}{\pgfqpoint{1.475788in}{3.206004in}}%
\pgfpathcurveto{\pgfqpoint{1.483602in}{3.198191in}}{\pgfqpoint{1.494201in}{3.193800in}}{\pgfqpoint{1.505251in}{3.193800in}}%
\pgfpathclose%
\pgfusepath{stroke,fill}%
\end{pgfscope}%
\begin{pgfscope}%
\pgfpathrectangle{\pgfqpoint{0.648703in}{0.548769in}}{\pgfqpoint{5.201297in}{3.102590in}}%
\pgfusepath{clip}%
\pgfsetbuttcap%
\pgfsetroundjoin%
\definecolor{currentfill}{rgb}{1.000000,0.498039,0.054902}%
\pgfsetfillcolor{currentfill}%
\pgfsetlinewidth{1.003750pt}%
\definecolor{currentstroke}{rgb}{1.000000,0.498039,0.054902}%
\pgfsetstrokecolor{currentstroke}%
\pgfsetdash{}{0pt}%
\pgfpathmoveto{\pgfqpoint{2.667985in}{3.231859in}}%
\pgfpathcurveto{\pgfqpoint{2.679035in}{3.231859in}}{\pgfqpoint{2.689634in}{3.236249in}}{\pgfqpoint{2.697448in}{3.244062in}}%
\pgfpathcurveto{\pgfqpoint{2.705261in}{3.251876in}}{\pgfqpoint{2.709651in}{3.262475in}}{\pgfqpoint{2.709651in}{3.273525in}}%
\pgfpathcurveto{\pgfqpoint{2.709651in}{3.284575in}}{\pgfqpoint{2.705261in}{3.295174in}}{\pgfqpoint{2.697448in}{3.302988in}}%
\pgfpathcurveto{\pgfqpoint{2.689634in}{3.310802in}}{\pgfqpoint{2.679035in}{3.315192in}}{\pgfqpoint{2.667985in}{3.315192in}}%
\pgfpathcurveto{\pgfqpoint{2.656935in}{3.315192in}}{\pgfqpoint{2.646336in}{3.310802in}}{\pgfqpoint{2.638522in}{3.302988in}}%
\pgfpathcurveto{\pgfqpoint{2.630708in}{3.295174in}}{\pgfqpoint{2.626318in}{3.284575in}}{\pgfqpoint{2.626318in}{3.273525in}}%
\pgfpathcurveto{\pgfqpoint{2.626318in}{3.262475in}}{\pgfqpoint{2.630708in}{3.251876in}}{\pgfqpoint{2.638522in}{3.244062in}}%
\pgfpathcurveto{\pgfqpoint{2.646336in}{3.236249in}}{\pgfqpoint{2.656935in}{3.231859in}}{\pgfqpoint{2.667985in}{3.231859in}}%
\pgfpathclose%
\pgfusepath{stroke,fill}%
\end{pgfscope}%
\begin{pgfscope}%
\pgfpathrectangle{\pgfqpoint{0.648703in}{0.548769in}}{\pgfqpoint{5.201297in}{3.102590in}}%
\pgfusepath{clip}%
\pgfsetbuttcap%
\pgfsetroundjoin%
\definecolor{currentfill}{rgb}{0.121569,0.466667,0.705882}%
\pgfsetfillcolor{currentfill}%
\pgfsetlinewidth{1.003750pt}%
\definecolor{currentstroke}{rgb}{0.121569,0.466667,0.705882}%
\pgfsetstrokecolor{currentstroke}%
\pgfsetdash{}{0pt}%
\pgfpathmoveto{\pgfqpoint{1.815313in}{0.681958in}}%
\pgfpathcurveto{\pgfqpoint{1.826363in}{0.681958in}}{\pgfqpoint{1.836962in}{0.686349in}}{\pgfqpoint{1.844776in}{0.694162in}}%
\pgfpathcurveto{\pgfqpoint{1.852590in}{0.701976in}}{\pgfqpoint{1.856980in}{0.712575in}}{\pgfqpoint{1.856980in}{0.723625in}}%
\pgfpathcurveto{\pgfqpoint{1.856980in}{0.734675in}}{\pgfqpoint{1.852590in}{0.745274in}}{\pgfqpoint{1.844776in}{0.753088in}}%
\pgfpathcurveto{\pgfqpoint{1.836962in}{0.760902in}}{\pgfqpoint{1.826363in}{0.765292in}}{\pgfqpoint{1.815313in}{0.765292in}}%
\pgfpathcurveto{\pgfqpoint{1.804263in}{0.765292in}}{\pgfqpoint{1.793664in}{0.760902in}}{\pgfqpoint{1.785850in}{0.753088in}}%
\pgfpathcurveto{\pgfqpoint{1.778037in}{0.745274in}}{\pgfqpoint{1.773646in}{0.734675in}}{\pgfqpoint{1.773646in}{0.723625in}}%
\pgfpathcurveto{\pgfqpoint{1.773646in}{0.712575in}}{\pgfqpoint{1.778037in}{0.701976in}}{\pgfqpoint{1.785850in}{0.694162in}}%
\pgfpathcurveto{\pgfqpoint{1.793664in}{0.686349in}}{\pgfqpoint{1.804263in}{0.681958in}}{\pgfqpoint{1.815313in}{0.681958in}}%
\pgfpathclose%
\pgfusepath{stroke,fill}%
\end{pgfscope}%
\begin{pgfscope}%
\pgfpathrectangle{\pgfqpoint{0.648703in}{0.548769in}}{\pgfqpoint{5.201297in}{3.102590in}}%
\pgfusepath{clip}%
\pgfsetbuttcap%
\pgfsetroundjoin%
\definecolor{currentfill}{rgb}{0.121569,0.466667,0.705882}%
\pgfsetfillcolor{currentfill}%
\pgfsetlinewidth{1.003750pt}%
\definecolor{currentstroke}{rgb}{0.121569,0.466667,0.705882}%
\pgfsetstrokecolor{currentstroke}%
\pgfsetdash{}{0pt}%
\pgfpathmoveto{\pgfqpoint{1.582766in}{0.758075in}}%
\pgfpathcurveto{\pgfqpoint{1.593816in}{0.758075in}}{\pgfqpoint{1.604416in}{0.762465in}}{\pgfqpoint{1.612229in}{0.770279in}}%
\pgfpathcurveto{\pgfqpoint{1.620043in}{0.778092in}}{\pgfqpoint{1.624433in}{0.788691in}}{\pgfqpoint{1.624433in}{0.799742in}}%
\pgfpathcurveto{\pgfqpoint{1.624433in}{0.810792in}}{\pgfqpoint{1.620043in}{0.821391in}}{\pgfqpoint{1.612229in}{0.829204in}}%
\pgfpathcurveto{\pgfqpoint{1.604416in}{0.837018in}}{\pgfqpoint{1.593816in}{0.841408in}}{\pgfqpoint{1.582766in}{0.841408in}}%
\pgfpathcurveto{\pgfqpoint{1.571716in}{0.841408in}}{\pgfqpoint{1.561117in}{0.837018in}}{\pgfqpoint{1.553304in}{0.829204in}}%
\pgfpathcurveto{\pgfqpoint{1.545490in}{0.821391in}}{\pgfqpoint{1.541100in}{0.810792in}}{\pgfqpoint{1.541100in}{0.799742in}}%
\pgfpathcurveto{\pgfqpoint{1.541100in}{0.788691in}}{\pgfqpoint{1.545490in}{0.778092in}}{\pgfqpoint{1.553304in}{0.770279in}}%
\pgfpathcurveto{\pgfqpoint{1.561117in}{0.762465in}}{\pgfqpoint{1.571716in}{0.758075in}}{\pgfqpoint{1.582766in}{0.758075in}}%
\pgfpathclose%
\pgfusepath{stroke,fill}%
\end{pgfscope}%
\begin{pgfscope}%
\pgfpathrectangle{\pgfqpoint{0.648703in}{0.548769in}}{\pgfqpoint{5.201297in}{3.102590in}}%
\pgfusepath{clip}%
\pgfsetbuttcap%
\pgfsetroundjoin%
\definecolor{currentfill}{rgb}{1.000000,0.498039,0.054902}%
\pgfsetfillcolor{currentfill}%
\pgfsetlinewidth{1.003750pt}%
\definecolor{currentstroke}{rgb}{1.000000,0.498039,0.054902}%
\pgfsetstrokecolor{currentstroke}%
\pgfsetdash{}{0pt}%
\pgfpathmoveto{\pgfqpoint{1.582766in}{3.185343in}}%
\pgfpathcurveto{\pgfqpoint{1.593816in}{3.185343in}}{\pgfqpoint{1.604416in}{3.189733in}}{\pgfqpoint{1.612229in}{3.197547in}}%
\pgfpathcurveto{\pgfqpoint{1.620043in}{3.205360in}}{\pgfqpoint{1.624433in}{3.215959in}}{\pgfqpoint{1.624433in}{3.227010in}}%
\pgfpathcurveto{\pgfqpoint{1.624433in}{3.238060in}}{\pgfqpoint{1.620043in}{3.248659in}}{\pgfqpoint{1.612229in}{3.256472in}}%
\pgfpathcurveto{\pgfqpoint{1.604416in}{3.264286in}}{\pgfqpoint{1.593816in}{3.268676in}}{\pgfqpoint{1.582766in}{3.268676in}}%
\pgfpathcurveto{\pgfqpoint{1.571716in}{3.268676in}}{\pgfqpoint{1.561117in}{3.264286in}}{\pgfqpoint{1.553304in}{3.256472in}}%
\pgfpathcurveto{\pgfqpoint{1.545490in}{3.248659in}}{\pgfqpoint{1.541100in}{3.238060in}}{\pgfqpoint{1.541100in}{3.227010in}}%
\pgfpathcurveto{\pgfqpoint{1.541100in}{3.215959in}}{\pgfqpoint{1.545490in}{3.205360in}}{\pgfqpoint{1.553304in}{3.197547in}}%
\pgfpathcurveto{\pgfqpoint{1.561117in}{3.189733in}}{\pgfqpoint{1.571716in}{3.185343in}}{\pgfqpoint{1.582766in}{3.185343in}}%
\pgfpathclose%
\pgfusepath{stroke,fill}%
\end{pgfscope}%
\begin{pgfscope}%
\pgfpathrectangle{\pgfqpoint{0.648703in}{0.548769in}}{\pgfqpoint{5.201297in}{3.102590in}}%
\pgfusepath{clip}%
\pgfsetbuttcap%
\pgfsetroundjoin%
\definecolor{currentfill}{rgb}{1.000000,0.498039,0.054902}%
\pgfsetfillcolor{currentfill}%
\pgfsetlinewidth{1.003750pt}%
\definecolor{currentstroke}{rgb}{1.000000,0.498039,0.054902}%
\pgfsetstrokecolor{currentstroke}%
\pgfsetdash{}{0pt}%
\pgfpathmoveto{\pgfqpoint{1.660282in}{3.202258in}}%
\pgfpathcurveto{\pgfqpoint{1.671332in}{3.202258in}}{\pgfqpoint{1.681931in}{3.206648in}}{\pgfqpoint{1.689745in}{3.214462in}}%
\pgfpathcurveto{\pgfqpoint{1.697558in}{3.222275in}}{\pgfqpoint{1.701949in}{3.232874in}}{\pgfqpoint{1.701949in}{3.243924in}}%
\pgfpathcurveto{\pgfqpoint{1.701949in}{3.254974in}}{\pgfqpoint{1.697558in}{3.265573in}}{\pgfqpoint{1.689745in}{3.273387in}}%
\pgfpathcurveto{\pgfqpoint{1.681931in}{3.281201in}}{\pgfqpoint{1.671332in}{3.285591in}}{\pgfqpoint{1.660282in}{3.285591in}}%
\pgfpathcurveto{\pgfqpoint{1.649232in}{3.285591in}}{\pgfqpoint{1.638633in}{3.281201in}}{\pgfqpoint{1.630819in}{3.273387in}}%
\pgfpathcurveto{\pgfqpoint{1.623006in}{3.265573in}}{\pgfqpoint{1.618615in}{3.254974in}}{\pgfqpoint{1.618615in}{3.243924in}}%
\pgfpathcurveto{\pgfqpoint{1.618615in}{3.232874in}}{\pgfqpoint{1.623006in}{3.222275in}}{\pgfqpoint{1.630819in}{3.214462in}}%
\pgfpathcurveto{\pgfqpoint{1.638633in}{3.206648in}}{\pgfqpoint{1.649232in}{3.202258in}}{\pgfqpoint{1.660282in}{3.202258in}}%
\pgfpathclose%
\pgfusepath{stroke,fill}%
\end{pgfscope}%
\begin{pgfscope}%
\pgfpathrectangle{\pgfqpoint{0.648703in}{0.548769in}}{\pgfqpoint{5.201297in}{3.102590in}}%
\pgfusepath{clip}%
\pgfsetbuttcap%
\pgfsetroundjoin%
\definecolor{currentfill}{rgb}{1.000000,0.498039,0.054902}%
\pgfsetfillcolor{currentfill}%
\pgfsetlinewidth{1.003750pt}%
\definecolor{currentstroke}{rgb}{1.000000,0.498039,0.054902}%
\pgfsetstrokecolor{currentstroke}%
\pgfsetdash{}{0pt}%
\pgfpathmoveto{\pgfqpoint{2.202891in}{3.189572in}}%
\pgfpathcurveto{\pgfqpoint{2.213941in}{3.189572in}}{\pgfqpoint{2.224540in}{3.193962in}}{\pgfqpoint{2.232354in}{3.201775in}}%
\pgfpathcurveto{\pgfqpoint{2.240168in}{3.209589in}}{\pgfqpoint{2.244558in}{3.220188in}}{\pgfqpoint{2.244558in}{3.231238in}}%
\pgfpathcurveto{\pgfqpoint{2.244558in}{3.242288in}}{\pgfqpoint{2.240168in}{3.252887in}}{\pgfqpoint{2.232354in}{3.260701in}}%
\pgfpathcurveto{\pgfqpoint{2.224540in}{3.268515in}}{\pgfqpoint{2.213941in}{3.272905in}}{\pgfqpoint{2.202891in}{3.272905in}}%
\pgfpathcurveto{\pgfqpoint{2.191841in}{3.272905in}}{\pgfqpoint{2.181242in}{3.268515in}}{\pgfqpoint{2.173428in}{3.260701in}}%
\pgfpathcurveto{\pgfqpoint{2.165615in}{3.252887in}}{\pgfqpoint{2.161224in}{3.242288in}}{\pgfqpoint{2.161224in}{3.231238in}}%
\pgfpathcurveto{\pgfqpoint{2.161224in}{3.220188in}}{\pgfqpoint{2.165615in}{3.209589in}}{\pgfqpoint{2.173428in}{3.201775in}}%
\pgfpathcurveto{\pgfqpoint{2.181242in}{3.193962in}}{\pgfqpoint{2.191841in}{3.189572in}}{\pgfqpoint{2.202891in}{3.189572in}}%
\pgfpathclose%
\pgfusepath{stroke,fill}%
\end{pgfscope}%
\begin{pgfscope}%
\pgfpathrectangle{\pgfqpoint{0.648703in}{0.548769in}}{\pgfqpoint{5.201297in}{3.102590in}}%
\pgfusepath{clip}%
\pgfsetbuttcap%
\pgfsetroundjoin%
\definecolor{currentfill}{rgb}{1.000000,0.498039,0.054902}%
\pgfsetfillcolor{currentfill}%
\pgfsetlinewidth{1.003750pt}%
\definecolor{currentstroke}{rgb}{1.000000,0.498039,0.054902}%
\pgfsetstrokecolor{currentstroke}%
\pgfsetdash{}{0pt}%
\pgfpathmoveto{\pgfqpoint{2.667985in}{3.206486in}}%
\pgfpathcurveto{\pgfqpoint{2.679035in}{3.206486in}}{\pgfqpoint{2.689634in}{3.210877in}}{\pgfqpoint{2.697448in}{3.218690in}}%
\pgfpathcurveto{\pgfqpoint{2.705261in}{3.226504in}}{\pgfqpoint{2.709651in}{3.237103in}}{\pgfqpoint{2.709651in}{3.248153in}}%
\pgfpathcurveto{\pgfqpoint{2.709651in}{3.259203in}}{\pgfqpoint{2.705261in}{3.269802in}}{\pgfqpoint{2.697448in}{3.277616in}}%
\pgfpathcurveto{\pgfqpoint{2.689634in}{3.285429in}}{\pgfqpoint{2.679035in}{3.289820in}}{\pgfqpoint{2.667985in}{3.289820in}}%
\pgfpathcurveto{\pgfqpoint{2.656935in}{3.289820in}}{\pgfqpoint{2.646336in}{3.285429in}}{\pgfqpoint{2.638522in}{3.277616in}}%
\pgfpathcurveto{\pgfqpoint{2.630708in}{3.269802in}}{\pgfqpoint{2.626318in}{3.259203in}}{\pgfqpoint{2.626318in}{3.248153in}}%
\pgfpathcurveto{\pgfqpoint{2.626318in}{3.237103in}}{\pgfqpoint{2.630708in}{3.226504in}}{\pgfqpoint{2.638522in}{3.218690in}}%
\pgfpathcurveto{\pgfqpoint{2.646336in}{3.210877in}}{\pgfqpoint{2.656935in}{3.206486in}}{\pgfqpoint{2.667985in}{3.206486in}}%
\pgfpathclose%
\pgfusepath{stroke,fill}%
\end{pgfscope}%
\begin{pgfscope}%
\pgfpathrectangle{\pgfqpoint{0.648703in}{0.548769in}}{\pgfqpoint{5.201297in}{3.102590in}}%
\pgfusepath{clip}%
\pgfsetbuttcap%
\pgfsetroundjoin%
\definecolor{currentfill}{rgb}{0.121569,0.466667,0.705882}%
\pgfsetfillcolor{currentfill}%
\pgfsetlinewidth{1.003750pt}%
\definecolor{currentstroke}{rgb}{0.121569,0.466667,0.705882}%
\pgfsetstrokecolor{currentstroke}%
\pgfsetdash{}{0pt}%
\pgfpathmoveto{\pgfqpoint{1.272704in}{0.648129in}}%
\pgfpathcurveto{\pgfqpoint{1.283754in}{0.648129in}}{\pgfqpoint{1.294353in}{0.652519in}}{\pgfqpoint{1.302167in}{0.660333in}}%
\pgfpathcurveto{\pgfqpoint{1.309980in}{0.668146in}}{\pgfqpoint{1.314371in}{0.678745in}}{\pgfqpoint{1.314371in}{0.689796in}}%
\pgfpathcurveto{\pgfqpoint{1.314371in}{0.700846in}}{\pgfqpoint{1.309980in}{0.711445in}}{\pgfqpoint{1.302167in}{0.719258in}}%
\pgfpathcurveto{\pgfqpoint{1.294353in}{0.727072in}}{\pgfqpoint{1.283754in}{0.731462in}}{\pgfqpoint{1.272704in}{0.731462in}}%
\pgfpathcurveto{\pgfqpoint{1.261654in}{0.731462in}}{\pgfqpoint{1.251055in}{0.727072in}}{\pgfqpoint{1.243241in}{0.719258in}}%
\pgfpathcurveto{\pgfqpoint{1.235428in}{0.711445in}}{\pgfqpoint{1.231037in}{0.700846in}}{\pgfqpoint{1.231037in}{0.689796in}}%
\pgfpathcurveto{\pgfqpoint{1.231037in}{0.678745in}}{\pgfqpoint{1.235428in}{0.668146in}}{\pgfqpoint{1.243241in}{0.660333in}}%
\pgfpathcurveto{\pgfqpoint{1.251055in}{0.652519in}}{\pgfqpoint{1.261654in}{0.648129in}}{\pgfqpoint{1.272704in}{0.648129in}}%
\pgfpathclose%
\pgfusepath{stroke,fill}%
\end{pgfscope}%
\begin{pgfscope}%
\pgfpathrectangle{\pgfqpoint{0.648703in}{0.548769in}}{\pgfqpoint{5.201297in}{3.102590in}}%
\pgfusepath{clip}%
\pgfsetbuttcap%
\pgfsetroundjoin%
\definecolor{currentfill}{rgb}{0.121569,0.466667,0.705882}%
\pgfsetfillcolor{currentfill}%
\pgfsetlinewidth{1.003750pt}%
\definecolor{currentstroke}{rgb}{0.121569,0.466667,0.705882}%
\pgfsetstrokecolor{currentstroke}%
\pgfsetdash{}{0pt}%
\pgfpathmoveto{\pgfqpoint{0.885126in}{1.692615in}}%
\pgfpathcurveto{\pgfqpoint{0.896176in}{1.692615in}}{\pgfqpoint{0.906775in}{1.697006in}}{\pgfqpoint{0.914589in}{1.704819in}}%
\pgfpathcurveto{\pgfqpoint{0.922402in}{1.712633in}}{\pgfqpoint{0.926793in}{1.723232in}}{\pgfqpoint{0.926793in}{1.734282in}}%
\pgfpathcurveto{\pgfqpoint{0.926793in}{1.745332in}}{\pgfqpoint{0.922402in}{1.755931in}}{\pgfqpoint{0.914589in}{1.763745in}}%
\pgfpathcurveto{\pgfqpoint{0.906775in}{1.771558in}}{\pgfqpoint{0.896176in}{1.775949in}}{\pgfqpoint{0.885126in}{1.775949in}}%
\pgfpathcurveto{\pgfqpoint{0.874076in}{1.775949in}}{\pgfqpoint{0.863477in}{1.771558in}}{\pgfqpoint{0.855663in}{1.763745in}}%
\pgfpathcurveto{\pgfqpoint{0.847850in}{1.755931in}}{\pgfqpoint{0.843459in}{1.745332in}}{\pgfqpoint{0.843459in}{1.734282in}}%
\pgfpathcurveto{\pgfqpoint{0.843459in}{1.723232in}}{\pgfqpoint{0.847850in}{1.712633in}}{\pgfqpoint{0.855663in}{1.704819in}}%
\pgfpathcurveto{\pgfqpoint{0.863477in}{1.697006in}}{\pgfqpoint{0.874076in}{1.692615in}}{\pgfqpoint{0.885126in}{1.692615in}}%
\pgfpathclose%
\pgfusepath{stroke,fill}%
\end{pgfscope}%
\begin{pgfscope}%
\pgfpathrectangle{\pgfqpoint{0.648703in}{0.548769in}}{\pgfqpoint{5.201297in}{3.102590in}}%
\pgfusepath{clip}%
\pgfsetbuttcap%
\pgfsetroundjoin%
\definecolor{currentfill}{rgb}{0.121569,0.466667,0.705882}%
\pgfsetfillcolor{currentfill}%
\pgfsetlinewidth{1.003750pt}%
\definecolor{currentstroke}{rgb}{0.121569,0.466667,0.705882}%
\pgfsetstrokecolor{currentstroke}%
\pgfsetdash{}{0pt}%
\pgfpathmoveto{\pgfqpoint{0.885126in}{0.648129in}}%
\pgfpathcurveto{\pgfqpoint{0.896176in}{0.648129in}}{\pgfqpoint{0.906775in}{0.652519in}}{\pgfqpoint{0.914589in}{0.660333in}}%
\pgfpathcurveto{\pgfqpoint{0.922402in}{0.668146in}}{\pgfqpoint{0.926793in}{0.678745in}}{\pgfqpoint{0.926793in}{0.689796in}}%
\pgfpathcurveto{\pgfqpoint{0.926793in}{0.700846in}}{\pgfqpoint{0.922402in}{0.711445in}}{\pgfqpoint{0.914589in}{0.719258in}}%
\pgfpathcurveto{\pgfqpoint{0.906775in}{0.727072in}}{\pgfqpoint{0.896176in}{0.731462in}}{\pgfqpoint{0.885126in}{0.731462in}}%
\pgfpathcurveto{\pgfqpoint{0.874076in}{0.731462in}}{\pgfqpoint{0.863477in}{0.727072in}}{\pgfqpoint{0.855663in}{0.719258in}}%
\pgfpathcurveto{\pgfqpoint{0.847850in}{0.711445in}}{\pgfqpoint{0.843459in}{0.700846in}}{\pgfqpoint{0.843459in}{0.689796in}}%
\pgfpathcurveto{\pgfqpoint{0.843459in}{0.678745in}}{\pgfqpoint{0.847850in}{0.668146in}}{\pgfqpoint{0.855663in}{0.660333in}}%
\pgfpathcurveto{\pgfqpoint{0.863477in}{0.652519in}}{\pgfqpoint{0.874076in}{0.648129in}}{\pgfqpoint{0.885126in}{0.648129in}}%
\pgfpathclose%
\pgfusepath{stroke,fill}%
\end{pgfscope}%
\begin{pgfscope}%
\pgfpathrectangle{\pgfqpoint{0.648703in}{0.548769in}}{\pgfqpoint{5.201297in}{3.102590in}}%
\pgfusepath{clip}%
\pgfsetbuttcap%
\pgfsetroundjoin%
\definecolor{currentfill}{rgb}{1.000000,0.498039,0.054902}%
\pgfsetfillcolor{currentfill}%
\pgfsetlinewidth{1.003750pt}%
\definecolor{currentstroke}{rgb}{1.000000,0.498039,0.054902}%
\pgfsetstrokecolor{currentstroke}%
\pgfsetdash{}{0pt}%
\pgfpathmoveto{\pgfqpoint{2.125376in}{3.189572in}}%
\pgfpathcurveto{\pgfqpoint{2.136426in}{3.189572in}}{\pgfqpoint{2.147025in}{3.193962in}}{\pgfqpoint{2.154838in}{3.201775in}}%
\pgfpathcurveto{\pgfqpoint{2.162652in}{3.209589in}}{\pgfqpoint{2.167042in}{3.220188in}}{\pgfqpoint{2.167042in}{3.231238in}}%
\pgfpathcurveto{\pgfqpoint{2.167042in}{3.242288in}}{\pgfqpoint{2.162652in}{3.252887in}}{\pgfqpoint{2.154838in}{3.260701in}}%
\pgfpathcurveto{\pgfqpoint{2.147025in}{3.268515in}}{\pgfqpoint{2.136426in}{3.272905in}}{\pgfqpoint{2.125376in}{3.272905in}}%
\pgfpathcurveto{\pgfqpoint{2.114325in}{3.272905in}}{\pgfqpoint{2.103726in}{3.268515in}}{\pgfqpoint{2.095913in}{3.260701in}}%
\pgfpathcurveto{\pgfqpoint{2.088099in}{3.252887in}}{\pgfqpoint{2.083709in}{3.242288in}}{\pgfqpoint{2.083709in}{3.231238in}}%
\pgfpathcurveto{\pgfqpoint{2.083709in}{3.220188in}}{\pgfqpoint{2.088099in}{3.209589in}}{\pgfqpoint{2.095913in}{3.201775in}}%
\pgfpathcurveto{\pgfqpoint{2.103726in}{3.193962in}}{\pgfqpoint{2.114325in}{3.189572in}}{\pgfqpoint{2.125376in}{3.189572in}}%
\pgfpathclose%
\pgfusepath{stroke,fill}%
\end{pgfscope}%
\begin{pgfscope}%
\pgfpathrectangle{\pgfqpoint{0.648703in}{0.548769in}}{\pgfqpoint{5.201297in}{3.102590in}}%
\pgfusepath{clip}%
\pgfsetbuttcap%
\pgfsetroundjoin%
\definecolor{currentfill}{rgb}{0.121569,0.466667,0.705882}%
\pgfsetfillcolor{currentfill}%
\pgfsetlinewidth{1.003750pt}%
\definecolor{currentstroke}{rgb}{0.121569,0.466667,0.705882}%
\pgfsetstrokecolor{currentstroke}%
\pgfsetdash{}{0pt}%
\pgfpathmoveto{\pgfqpoint{2.125376in}{3.181114in}}%
\pgfpathcurveto{\pgfqpoint{2.136426in}{3.181114in}}{\pgfqpoint{2.147025in}{3.185504in}}{\pgfqpoint{2.154838in}{3.193318in}}%
\pgfpathcurveto{\pgfqpoint{2.162652in}{3.201132in}}{\pgfqpoint{2.167042in}{3.211731in}}{\pgfqpoint{2.167042in}{3.222781in}}%
\pgfpathcurveto{\pgfqpoint{2.167042in}{3.233831in}}{\pgfqpoint{2.162652in}{3.244430in}}{\pgfqpoint{2.154838in}{3.252244in}}%
\pgfpathcurveto{\pgfqpoint{2.147025in}{3.260057in}}{\pgfqpoint{2.136426in}{3.264448in}}{\pgfqpoint{2.125376in}{3.264448in}}%
\pgfpathcurveto{\pgfqpoint{2.114325in}{3.264448in}}{\pgfqpoint{2.103726in}{3.260057in}}{\pgfqpoint{2.095913in}{3.252244in}}%
\pgfpathcurveto{\pgfqpoint{2.088099in}{3.244430in}}{\pgfqpoint{2.083709in}{3.233831in}}{\pgfqpoint{2.083709in}{3.222781in}}%
\pgfpathcurveto{\pgfqpoint{2.083709in}{3.211731in}}{\pgfqpoint{2.088099in}{3.201132in}}{\pgfqpoint{2.095913in}{3.193318in}}%
\pgfpathcurveto{\pgfqpoint{2.103726in}{3.185504in}}{\pgfqpoint{2.114325in}{3.181114in}}{\pgfqpoint{2.125376in}{3.181114in}}%
\pgfpathclose%
\pgfusepath{stroke,fill}%
\end{pgfscope}%
\begin{pgfscope}%
\pgfpathrectangle{\pgfqpoint{0.648703in}{0.548769in}}{\pgfqpoint{5.201297in}{3.102590in}}%
\pgfusepath{clip}%
\pgfsetbuttcap%
\pgfsetroundjoin%
\definecolor{currentfill}{rgb}{1.000000,0.498039,0.054902}%
\pgfsetfillcolor{currentfill}%
\pgfsetlinewidth{1.003750pt}%
\definecolor{currentstroke}{rgb}{1.000000,0.498039,0.054902}%
\pgfsetstrokecolor{currentstroke}%
\pgfsetdash{}{0pt}%
\pgfpathmoveto{\pgfqpoint{4.605875in}{3.189572in}}%
\pgfpathcurveto{\pgfqpoint{4.616925in}{3.189572in}}{\pgfqpoint{4.627524in}{3.193962in}}{\pgfqpoint{4.635337in}{3.201775in}}%
\pgfpathcurveto{\pgfqpoint{4.643151in}{3.209589in}}{\pgfqpoint{4.647541in}{3.220188in}}{\pgfqpoint{4.647541in}{3.231238in}}%
\pgfpathcurveto{\pgfqpoint{4.647541in}{3.242288in}}{\pgfqpoint{4.643151in}{3.252887in}}{\pgfqpoint{4.635337in}{3.260701in}}%
\pgfpathcurveto{\pgfqpoint{4.627524in}{3.268515in}}{\pgfqpoint{4.616925in}{3.272905in}}{\pgfqpoint{4.605875in}{3.272905in}}%
\pgfpathcurveto{\pgfqpoint{4.594825in}{3.272905in}}{\pgfqpoint{4.584226in}{3.268515in}}{\pgfqpoint{4.576412in}{3.260701in}}%
\pgfpathcurveto{\pgfqpoint{4.568598in}{3.252887in}}{\pgfqpoint{4.564208in}{3.242288in}}{\pgfqpoint{4.564208in}{3.231238in}}%
\pgfpathcurveto{\pgfqpoint{4.564208in}{3.220188in}}{\pgfqpoint{4.568598in}{3.209589in}}{\pgfqpoint{4.576412in}{3.201775in}}%
\pgfpathcurveto{\pgfqpoint{4.584226in}{3.193962in}}{\pgfqpoint{4.594825in}{3.189572in}}{\pgfqpoint{4.605875in}{3.189572in}}%
\pgfpathclose%
\pgfusepath{stroke,fill}%
\end{pgfscope}%
\begin{pgfscope}%
\pgfpathrectangle{\pgfqpoint{0.648703in}{0.548769in}}{\pgfqpoint{5.201297in}{3.102590in}}%
\pgfusepath{clip}%
\pgfsetbuttcap%
\pgfsetroundjoin%
\definecolor{currentfill}{rgb}{1.000000,0.498039,0.054902}%
\pgfsetfillcolor{currentfill}%
\pgfsetlinewidth{1.003750pt}%
\definecolor{currentstroke}{rgb}{1.000000,0.498039,0.054902}%
\pgfsetstrokecolor{currentstroke}%
\pgfsetdash{}{0pt}%
\pgfpathmoveto{\pgfqpoint{1.195188in}{3.214944in}}%
\pgfpathcurveto{\pgfqpoint{1.206239in}{3.214944in}}{\pgfqpoint{1.216838in}{3.219334in}}{\pgfqpoint{1.224651in}{3.227148in}}%
\pgfpathcurveto{\pgfqpoint{1.232465in}{3.234961in}}{\pgfqpoint{1.236855in}{3.245560in}}{\pgfqpoint{1.236855in}{3.256610in}}%
\pgfpathcurveto{\pgfqpoint{1.236855in}{3.267661in}}{\pgfqpoint{1.232465in}{3.278260in}}{\pgfqpoint{1.224651in}{3.286073in}}%
\pgfpathcurveto{\pgfqpoint{1.216838in}{3.293887in}}{\pgfqpoint{1.206239in}{3.298277in}}{\pgfqpoint{1.195188in}{3.298277in}}%
\pgfpathcurveto{\pgfqpoint{1.184138in}{3.298277in}}{\pgfqpoint{1.173539in}{3.293887in}}{\pgfqpoint{1.165726in}{3.286073in}}%
\pgfpathcurveto{\pgfqpoint{1.157912in}{3.278260in}}{\pgfqpoint{1.153522in}{3.267661in}}{\pgfqpoint{1.153522in}{3.256610in}}%
\pgfpathcurveto{\pgfqpoint{1.153522in}{3.245560in}}{\pgfqpoint{1.157912in}{3.234961in}}{\pgfqpoint{1.165726in}{3.227148in}}%
\pgfpathcurveto{\pgfqpoint{1.173539in}{3.219334in}}{\pgfqpoint{1.184138in}{3.214944in}}{\pgfqpoint{1.195188in}{3.214944in}}%
\pgfpathclose%
\pgfusepath{stroke,fill}%
\end{pgfscope}%
\begin{pgfscope}%
\pgfpathrectangle{\pgfqpoint{0.648703in}{0.548769in}}{\pgfqpoint{5.201297in}{3.102590in}}%
\pgfusepath{clip}%
\pgfsetbuttcap%
\pgfsetroundjoin%
\definecolor{currentfill}{rgb}{1.000000,0.498039,0.054902}%
\pgfsetfillcolor{currentfill}%
\pgfsetlinewidth{1.003750pt}%
\definecolor{currentstroke}{rgb}{1.000000,0.498039,0.054902}%
\pgfsetstrokecolor{currentstroke}%
\pgfsetdash{}{0pt}%
\pgfpathmoveto{\pgfqpoint{1.660282in}{3.185343in}}%
\pgfpathcurveto{\pgfqpoint{1.671332in}{3.185343in}}{\pgfqpoint{1.681931in}{3.189733in}}{\pgfqpoint{1.689745in}{3.197547in}}%
\pgfpathcurveto{\pgfqpoint{1.697558in}{3.205360in}}{\pgfqpoint{1.701949in}{3.215959in}}{\pgfqpoint{1.701949in}{3.227010in}}%
\pgfpathcurveto{\pgfqpoint{1.701949in}{3.238060in}}{\pgfqpoint{1.697558in}{3.248659in}}{\pgfqpoint{1.689745in}{3.256472in}}%
\pgfpathcurveto{\pgfqpoint{1.681931in}{3.264286in}}{\pgfqpoint{1.671332in}{3.268676in}}{\pgfqpoint{1.660282in}{3.268676in}}%
\pgfpathcurveto{\pgfqpoint{1.649232in}{3.268676in}}{\pgfqpoint{1.638633in}{3.264286in}}{\pgfqpoint{1.630819in}{3.256472in}}%
\pgfpathcurveto{\pgfqpoint{1.623006in}{3.248659in}}{\pgfqpoint{1.618615in}{3.238060in}}{\pgfqpoint{1.618615in}{3.227010in}}%
\pgfpathcurveto{\pgfqpoint{1.618615in}{3.215959in}}{\pgfqpoint{1.623006in}{3.205360in}}{\pgfqpoint{1.630819in}{3.197547in}}%
\pgfpathcurveto{\pgfqpoint{1.638633in}{3.189733in}}{\pgfqpoint{1.649232in}{3.185343in}}{\pgfqpoint{1.660282in}{3.185343in}}%
\pgfpathclose%
\pgfusepath{stroke,fill}%
\end{pgfscope}%
\begin{pgfscope}%
\pgfpathrectangle{\pgfqpoint{0.648703in}{0.548769in}}{\pgfqpoint{5.201297in}{3.102590in}}%
\pgfusepath{clip}%
\pgfsetbuttcap%
\pgfsetroundjoin%
\definecolor{currentfill}{rgb}{1.000000,0.498039,0.054902}%
\pgfsetfillcolor{currentfill}%
\pgfsetlinewidth{1.003750pt}%
\definecolor{currentstroke}{rgb}{1.000000,0.498039,0.054902}%
\pgfsetstrokecolor{currentstroke}%
\pgfsetdash{}{0pt}%
\pgfpathmoveto{\pgfqpoint{2.900532in}{3.198029in}}%
\pgfpathcurveto{\pgfqpoint{2.911582in}{3.198029in}}{\pgfqpoint{2.922181in}{3.202419in}}{\pgfqpoint{2.929994in}{3.210233in}}%
\pgfpathcurveto{\pgfqpoint{2.937808in}{3.218046in}}{\pgfqpoint{2.942198in}{3.228646in}}{\pgfqpoint{2.942198in}{3.239696in}}%
\pgfpathcurveto{\pgfqpoint{2.942198in}{3.250746in}}{\pgfqpoint{2.937808in}{3.261345in}}{\pgfqpoint{2.929994in}{3.269158in}}%
\pgfpathcurveto{\pgfqpoint{2.922181in}{3.276972in}}{\pgfqpoint{2.911582in}{3.281362in}}{\pgfqpoint{2.900532in}{3.281362in}}%
\pgfpathcurveto{\pgfqpoint{2.889481in}{3.281362in}}{\pgfqpoint{2.878882in}{3.276972in}}{\pgfqpoint{2.871069in}{3.269158in}}%
\pgfpathcurveto{\pgfqpoint{2.863255in}{3.261345in}}{\pgfqpoint{2.858865in}{3.250746in}}{\pgfqpoint{2.858865in}{3.239696in}}%
\pgfpathcurveto{\pgfqpoint{2.858865in}{3.228646in}}{\pgfqpoint{2.863255in}{3.218046in}}{\pgfqpoint{2.871069in}{3.210233in}}%
\pgfpathcurveto{\pgfqpoint{2.878882in}{3.202419in}}{\pgfqpoint{2.889481in}{3.198029in}}{\pgfqpoint{2.900532in}{3.198029in}}%
\pgfpathclose%
\pgfusepath{stroke,fill}%
\end{pgfscope}%
\begin{pgfscope}%
\pgfpathrectangle{\pgfqpoint{0.648703in}{0.548769in}}{\pgfqpoint{5.201297in}{3.102590in}}%
\pgfusepath{clip}%
\pgfsetbuttcap%
\pgfsetroundjoin%
\definecolor{currentfill}{rgb}{0.839216,0.152941,0.156863}%
\pgfsetfillcolor{currentfill}%
\pgfsetlinewidth{1.003750pt}%
\definecolor{currentstroke}{rgb}{0.839216,0.152941,0.156863}%
\pgfsetstrokecolor{currentstroke}%
\pgfsetdash{}{0pt}%
\pgfpathmoveto{\pgfqpoint{1.970344in}{3.193800in}}%
\pgfpathcurveto{\pgfqpoint{1.981394in}{3.193800in}}{\pgfqpoint{1.991994in}{3.198191in}}{\pgfqpoint{1.999807in}{3.206004in}}%
\pgfpathcurveto{\pgfqpoint{2.007621in}{3.213818in}}{\pgfqpoint{2.012011in}{3.224417in}}{\pgfqpoint{2.012011in}{3.235467in}}%
\pgfpathcurveto{\pgfqpoint{2.012011in}{3.246517in}}{\pgfqpoint{2.007621in}{3.257116in}}{\pgfqpoint{1.999807in}{3.264930in}}%
\pgfpathcurveto{\pgfqpoint{1.991994in}{3.272743in}}{\pgfqpoint{1.981394in}{3.277134in}}{\pgfqpoint{1.970344in}{3.277134in}}%
\pgfpathcurveto{\pgfqpoint{1.959294in}{3.277134in}}{\pgfqpoint{1.948695in}{3.272743in}}{\pgfqpoint{1.940882in}{3.264930in}}%
\pgfpathcurveto{\pgfqpoint{1.933068in}{3.257116in}}{\pgfqpoint{1.928678in}{3.246517in}}{\pgfqpoint{1.928678in}{3.235467in}}%
\pgfpathcurveto{\pgfqpoint{1.928678in}{3.224417in}}{\pgfqpoint{1.933068in}{3.213818in}}{\pgfqpoint{1.940882in}{3.206004in}}%
\pgfpathcurveto{\pgfqpoint{1.948695in}{3.198191in}}{\pgfqpoint{1.959294in}{3.193800in}}{\pgfqpoint{1.970344in}{3.193800in}}%
\pgfpathclose%
\pgfusepath{stroke,fill}%
\end{pgfscope}%
\begin{pgfscope}%
\pgfpathrectangle{\pgfqpoint{0.648703in}{0.548769in}}{\pgfqpoint{5.201297in}{3.102590in}}%
\pgfusepath{clip}%
\pgfsetbuttcap%
\pgfsetroundjoin%
\definecolor{currentfill}{rgb}{0.121569,0.466667,0.705882}%
\pgfsetfillcolor{currentfill}%
\pgfsetlinewidth{1.003750pt}%
\definecolor{currentstroke}{rgb}{0.121569,0.466667,0.705882}%
\pgfsetstrokecolor{currentstroke}%
\pgfsetdash{}{0pt}%
\pgfpathmoveto{\pgfqpoint{1.350220in}{0.758075in}}%
\pgfpathcurveto{\pgfqpoint{1.361270in}{0.758075in}}{\pgfqpoint{1.371869in}{0.762465in}}{\pgfqpoint{1.379682in}{0.770279in}}%
\pgfpathcurveto{\pgfqpoint{1.387496in}{0.778092in}}{\pgfqpoint{1.391886in}{0.788691in}}{\pgfqpoint{1.391886in}{0.799742in}}%
\pgfpathcurveto{\pgfqpoint{1.391886in}{0.810792in}}{\pgfqpoint{1.387496in}{0.821391in}}{\pgfqpoint{1.379682in}{0.829204in}}%
\pgfpathcurveto{\pgfqpoint{1.371869in}{0.837018in}}{\pgfqpoint{1.361270in}{0.841408in}}{\pgfqpoint{1.350220in}{0.841408in}}%
\pgfpathcurveto{\pgfqpoint{1.339169in}{0.841408in}}{\pgfqpoint{1.328570in}{0.837018in}}{\pgfqpoint{1.320757in}{0.829204in}}%
\pgfpathcurveto{\pgfqpoint{1.312943in}{0.821391in}}{\pgfqpoint{1.308553in}{0.810792in}}{\pgfqpoint{1.308553in}{0.799742in}}%
\pgfpathcurveto{\pgfqpoint{1.308553in}{0.788691in}}{\pgfqpoint{1.312943in}{0.778092in}}{\pgfqpoint{1.320757in}{0.770279in}}%
\pgfpathcurveto{\pgfqpoint{1.328570in}{0.762465in}}{\pgfqpoint{1.339169in}{0.758075in}}{\pgfqpoint{1.350220in}{0.758075in}}%
\pgfpathclose%
\pgfusepath{stroke,fill}%
\end{pgfscope}%
\begin{pgfscope}%
\pgfpathrectangle{\pgfqpoint{0.648703in}{0.548769in}}{\pgfqpoint{5.201297in}{3.102590in}}%
\pgfusepath{clip}%
\pgfsetbuttcap%
\pgfsetroundjoin%
\definecolor{currentfill}{rgb}{1.000000,0.498039,0.054902}%
\pgfsetfillcolor{currentfill}%
\pgfsetlinewidth{1.003750pt}%
\definecolor{currentstroke}{rgb}{1.000000,0.498039,0.054902}%
\pgfsetstrokecolor{currentstroke}%
\pgfsetdash{}{0pt}%
\pgfpathmoveto{\pgfqpoint{2.047860in}{3.193800in}}%
\pgfpathcurveto{\pgfqpoint{2.058910in}{3.193800in}}{\pgfqpoint{2.069509in}{3.198191in}}{\pgfqpoint{2.077323in}{3.206004in}}%
\pgfpathcurveto{\pgfqpoint{2.085136in}{3.213818in}}{\pgfqpoint{2.089527in}{3.224417in}}{\pgfqpoint{2.089527in}{3.235467in}}%
\pgfpathcurveto{\pgfqpoint{2.089527in}{3.246517in}}{\pgfqpoint{2.085136in}{3.257116in}}{\pgfqpoint{2.077323in}{3.264930in}}%
\pgfpathcurveto{\pgfqpoint{2.069509in}{3.272743in}}{\pgfqpoint{2.058910in}{3.277134in}}{\pgfqpoint{2.047860in}{3.277134in}}%
\pgfpathcurveto{\pgfqpoint{2.036810in}{3.277134in}}{\pgfqpoint{2.026211in}{3.272743in}}{\pgfqpoint{2.018397in}{3.264930in}}%
\pgfpathcurveto{\pgfqpoint{2.010584in}{3.257116in}}{\pgfqpoint{2.006193in}{3.246517in}}{\pgfqpoint{2.006193in}{3.235467in}}%
\pgfpathcurveto{\pgfqpoint{2.006193in}{3.224417in}}{\pgfqpoint{2.010584in}{3.213818in}}{\pgfqpoint{2.018397in}{3.206004in}}%
\pgfpathcurveto{\pgfqpoint{2.026211in}{3.198191in}}{\pgfqpoint{2.036810in}{3.193800in}}{\pgfqpoint{2.047860in}{3.193800in}}%
\pgfpathclose%
\pgfusepath{stroke,fill}%
\end{pgfscope}%
\begin{pgfscope}%
\pgfpathrectangle{\pgfqpoint{0.648703in}{0.548769in}}{\pgfqpoint{5.201297in}{3.102590in}}%
\pgfusepath{clip}%
\pgfsetbuttcap%
\pgfsetroundjoin%
\definecolor{currentfill}{rgb}{0.121569,0.466667,0.705882}%
\pgfsetfillcolor{currentfill}%
\pgfsetlinewidth{1.003750pt}%
\definecolor{currentstroke}{rgb}{0.121569,0.466667,0.705882}%
\pgfsetstrokecolor{currentstroke}%
\pgfsetdash{}{0pt}%
\pgfpathmoveto{\pgfqpoint{0.885126in}{0.648129in}}%
\pgfpathcurveto{\pgfqpoint{0.896176in}{0.648129in}}{\pgfqpoint{0.906775in}{0.652519in}}{\pgfqpoint{0.914589in}{0.660333in}}%
\pgfpathcurveto{\pgfqpoint{0.922402in}{0.668146in}}{\pgfqpoint{0.926793in}{0.678745in}}{\pgfqpoint{0.926793in}{0.689796in}}%
\pgfpathcurveto{\pgfqpoint{0.926793in}{0.700846in}}{\pgfqpoint{0.922402in}{0.711445in}}{\pgfqpoint{0.914589in}{0.719258in}}%
\pgfpathcurveto{\pgfqpoint{0.906775in}{0.727072in}}{\pgfqpoint{0.896176in}{0.731462in}}{\pgfqpoint{0.885126in}{0.731462in}}%
\pgfpathcurveto{\pgfqpoint{0.874076in}{0.731462in}}{\pgfqpoint{0.863477in}{0.727072in}}{\pgfqpoint{0.855663in}{0.719258in}}%
\pgfpathcurveto{\pgfqpoint{0.847850in}{0.711445in}}{\pgfqpoint{0.843459in}{0.700846in}}{\pgfqpoint{0.843459in}{0.689796in}}%
\pgfpathcurveto{\pgfqpoint{0.843459in}{0.678745in}}{\pgfqpoint{0.847850in}{0.668146in}}{\pgfqpoint{0.855663in}{0.660333in}}%
\pgfpathcurveto{\pgfqpoint{0.863477in}{0.652519in}}{\pgfqpoint{0.874076in}{0.648129in}}{\pgfqpoint{0.885126in}{0.648129in}}%
\pgfpathclose%
\pgfusepath{stroke,fill}%
\end{pgfscope}%
\begin{pgfscope}%
\pgfpathrectangle{\pgfqpoint{0.648703in}{0.548769in}}{\pgfqpoint{5.201297in}{3.102590in}}%
\pgfusepath{clip}%
\pgfsetbuttcap%
\pgfsetroundjoin%
\definecolor{currentfill}{rgb}{1.000000,0.498039,0.054902}%
\pgfsetfillcolor{currentfill}%
\pgfsetlinewidth{1.003750pt}%
\definecolor{currentstroke}{rgb}{1.000000,0.498039,0.054902}%
\pgfsetstrokecolor{currentstroke}%
\pgfsetdash{}{0pt}%
\pgfpathmoveto{\pgfqpoint{2.667985in}{3.193800in}}%
\pgfpathcurveto{\pgfqpoint{2.679035in}{3.193800in}}{\pgfqpoint{2.689634in}{3.198191in}}{\pgfqpoint{2.697448in}{3.206004in}}%
\pgfpathcurveto{\pgfqpoint{2.705261in}{3.213818in}}{\pgfqpoint{2.709651in}{3.224417in}}{\pgfqpoint{2.709651in}{3.235467in}}%
\pgfpathcurveto{\pgfqpoint{2.709651in}{3.246517in}}{\pgfqpoint{2.705261in}{3.257116in}}{\pgfqpoint{2.697448in}{3.264930in}}%
\pgfpathcurveto{\pgfqpoint{2.689634in}{3.272743in}}{\pgfqpoint{2.679035in}{3.277134in}}{\pgfqpoint{2.667985in}{3.277134in}}%
\pgfpathcurveto{\pgfqpoint{2.656935in}{3.277134in}}{\pgfqpoint{2.646336in}{3.272743in}}{\pgfqpoint{2.638522in}{3.264930in}}%
\pgfpathcurveto{\pgfqpoint{2.630708in}{3.257116in}}{\pgfqpoint{2.626318in}{3.246517in}}{\pgfqpoint{2.626318in}{3.235467in}}%
\pgfpathcurveto{\pgfqpoint{2.626318in}{3.224417in}}{\pgfqpoint{2.630708in}{3.213818in}}{\pgfqpoint{2.638522in}{3.206004in}}%
\pgfpathcurveto{\pgfqpoint{2.646336in}{3.198191in}}{\pgfqpoint{2.656935in}{3.193800in}}{\pgfqpoint{2.667985in}{3.193800in}}%
\pgfpathclose%
\pgfusepath{stroke,fill}%
\end{pgfscope}%
\begin{pgfscope}%
\pgfpathrectangle{\pgfqpoint{0.648703in}{0.548769in}}{\pgfqpoint{5.201297in}{3.102590in}}%
\pgfusepath{clip}%
\pgfsetbuttcap%
\pgfsetroundjoin%
\definecolor{currentfill}{rgb}{1.000000,0.498039,0.054902}%
\pgfsetfillcolor{currentfill}%
\pgfsetlinewidth{1.003750pt}%
\definecolor{currentstroke}{rgb}{1.000000,0.498039,0.054902}%
\pgfsetstrokecolor{currentstroke}%
\pgfsetdash{}{0pt}%
\pgfpathmoveto{\pgfqpoint{2.435438in}{3.185343in}}%
\pgfpathcurveto{\pgfqpoint{2.446488in}{3.185343in}}{\pgfqpoint{2.457087in}{3.189733in}}{\pgfqpoint{2.464901in}{3.197547in}}%
\pgfpathcurveto{\pgfqpoint{2.472714in}{3.205360in}}{\pgfqpoint{2.477105in}{3.215959in}}{\pgfqpoint{2.477105in}{3.227010in}}%
\pgfpathcurveto{\pgfqpoint{2.477105in}{3.238060in}}{\pgfqpoint{2.472714in}{3.248659in}}{\pgfqpoint{2.464901in}{3.256472in}}%
\pgfpathcurveto{\pgfqpoint{2.457087in}{3.264286in}}{\pgfqpoint{2.446488in}{3.268676in}}{\pgfqpoint{2.435438in}{3.268676in}}%
\pgfpathcurveto{\pgfqpoint{2.424388in}{3.268676in}}{\pgfqpoint{2.413789in}{3.264286in}}{\pgfqpoint{2.405975in}{3.256472in}}%
\pgfpathcurveto{\pgfqpoint{2.398162in}{3.248659in}}{\pgfqpoint{2.393771in}{3.238060in}}{\pgfqpoint{2.393771in}{3.227010in}}%
\pgfpathcurveto{\pgfqpoint{2.393771in}{3.215959in}}{\pgfqpoint{2.398162in}{3.205360in}}{\pgfqpoint{2.405975in}{3.197547in}}%
\pgfpathcurveto{\pgfqpoint{2.413789in}{3.189733in}}{\pgfqpoint{2.424388in}{3.185343in}}{\pgfqpoint{2.435438in}{3.185343in}}%
\pgfpathclose%
\pgfusepath{stroke,fill}%
\end{pgfscope}%
\begin{pgfscope}%
\pgfpathrectangle{\pgfqpoint{0.648703in}{0.548769in}}{\pgfqpoint{5.201297in}{3.102590in}}%
\pgfusepath{clip}%
\pgfsetbuttcap%
\pgfsetroundjoin%
\definecolor{currentfill}{rgb}{1.000000,0.498039,0.054902}%
\pgfsetfillcolor{currentfill}%
\pgfsetlinewidth{1.003750pt}%
\definecolor{currentstroke}{rgb}{1.000000,0.498039,0.054902}%
\pgfsetstrokecolor{currentstroke}%
\pgfsetdash{}{0pt}%
\pgfpathmoveto{\pgfqpoint{2.047860in}{3.198029in}}%
\pgfpathcurveto{\pgfqpoint{2.058910in}{3.198029in}}{\pgfqpoint{2.069509in}{3.202419in}}{\pgfqpoint{2.077323in}{3.210233in}}%
\pgfpathcurveto{\pgfqpoint{2.085136in}{3.218046in}}{\pgfqpoint{2.089527in}{3.228646in}}{\pgfqpoint{2.089527in}{3.239696in}}%
\pgfpathcurveto{\pgfqpoint{2.089527in}{3.250746in}}{\pgfqpoint{2.085136in}{3.261345in}}{\pgfqpoint{2.077323in}{3.269158in}}%
\pgfpathcurveto{\pgfqpoint{2.069509in}{3.276972in}}{\pgfqpoint{2.058910in}{3.281362in}}{\pgfqpoint{2.047860in}{3.281362in}}%
\pgfpathcurveto{\pgfqpoint{2.036810in}{3.281362in}}{\pgfqpoint{2.026211in}{3.276972in}}{\pgfqpoint{2.018397in}{3.269158in}}%
\pgfpathcurveto{\pgfqpoint{2.010584in}{3.261345in}}{\pgfqpoint{2.006193in}{3.250746in}}{\pgfqpoint{2.006193in}{3.239696in}}%
\pgfpathcurveto{\pgfqpoint{2.006193in}{3.228646in}}{\pgfqpoint{2.010584in}{3.218046in}}{\pgfqpoint{2.018397in}{3.210233in}}%
\pgfpathcurveto{\pgfqpoint{2.026211in}{3.202419in}}{\pgfqpoint{2.036810in}{3.198029in}}{\pgfqpoint{2.047860in}{3.198029in}}%
\pgfpathclose%
\pgfusepath{stroke,fill}%
\end{pgfscope}%
\begin{pgfscope}%
\pgfpathrectangle{\pgfqpoint{0.648703in}{0.548769in}}{\pgfqpoint{5.201297in}{3.102590in}}%
\pgfusepath{clip}%
\pgfsetbuttcap%
\pgfsetroundjoin%
\definecolor{currentfill}{rgb}{1.000000,0.498039,0.054902}%
\pgfsetfillcolor{currentfill}%
\pgfsetlinewidth{1.003750pt}%
\definecolor{currentstroke}{rgb}{1.000000,0.498039,0.054902}%
\pgfsetstrokecolor{currentstroke}%
\pgfsetdash{}{0pt}%
\pgfpathmoveto{\pgfqpoint{2.512954in}{3.185343in}}%
\pgfpathcurveto{\pgfqpoint{2.524004in}{3.185343in}}{\pgfqpoint{2.534603in}{3.189733in}}{\pgfqpoint{2.542416in}{3.197547in}}%
\pgfpathcurveto{\pgfqpoint{2.550230in}{3.205360in}}{\pgfqpoint{2.554620in}{3.215959in}}{\pgfqpoint{2.554620in}{3.227010in}}%
\pgfpathcurveto{\pgfqpoint{2.554620in}{3.238060in}}{\pgfqpoint{2.550230in}{3.248659in}}{\pgfqpoint{2.542416in}{3.256472in}}%
\pgfpathcurveto{\pgfqpoint{2.534603in}{3.264286in}}{\pgfqpoint{2.524004in}{3.268676in}}{\pgfqpoint{2.512954in}{3.268676in}}%
\pgfpathcurveto{\pgfqpoint{2.501903in}{3.268676in}}{\pgfqpoint{2.491304in}{3.264286in}}{\pgfqpoint{2.483491in}{3.256472in}}%
\pgfpathcurveto{\pgfqpoint{2.475677in}{3.248659in}}{\pgfqpoint{2.471287in}{3.238060in}}{\pgfqpoint{2.471287in}{3.227010in}}%
\pgfpathcurveto{\pgfqpoint{2.471287in}{3.215959in}}{\pgfqpoint{2.475677in}{3.205360in}}{\pgfqpoint{2.483491in}{3.197547in}}%
\pgfpathcurveto{\pgfqpoint{2.491304in}{3.189733in}}{\pgfqpoint{2.501903in}{3.185343in}}{\pgfqpoint{2.512954in}{3.185343in}}%
\pgfpathclose%
\pgfusepath{stroke,fill}%
\end{pgfscope}%
\begin{pgfscope}%
\pgfpathrectangle{\pgfqpoint{0.648703in}{0.548769in}}{\pgfqpoint{5.201297in}{3.102590in}}%
\pgfusepath{clip}%
\pgfsetbuttcap%
\pgfsetroundjoin%
\definecolor{currentfill}{rgb}{1.000000,0.498039,0.054902}%
\pgfsetfillcolor{currentfill}%
\pgfsetlinewidth{1.003750pt}%
\definecolor{currentstroke}{rgb}{1.000000,0.498039,0.054902}%
\pgfsetstrokecolor{currentstroke}%
\pgfsetdash{}{0pt}%
\pgfpathmoveto{\pgfqpoint{1.892829in}{3.193800in}}%
\pgfpathcurveto{\pgfqpoint{1.903879in}{3.193800in}}{\pgfqpoint{1.914478in}{3.198191in}}{\pgfqpoint{1.922292in}{3.206004in}}%
\pgfpathcurveto{\pgfqpoint{1.930105in}{3.213818in}}{\pgfqpoint{1.934495in}{3.224417in}}{\pgfqpoint{1.934495in}{3.235467in}}%
\pgfpathcurveto{\pgfqpoint{1.934495in}{3.246517in}}{\pgfqpoint{1.930105in}{3.257116in}}{\pgfqpoint{1.922292in}{3.264930in}}%
\pgfpathcurveto{\pgfqpoint{1.914478in}{3.272743in}}{\pgfqpoint{1.903879in}{3.277134in}}{\pgfqpoint{1.892829in}{3.277134in}}%
\pgfpathcurveto{\pgfqpoint{1.881779in}{3.277134in}}{\pgfqpoint{1.871180in}{3.272743in}}{\pgfqpoint{1.863366in}{3.264930in}}%
\pgfpathcurveto{\pgfqpoint{1.855552in}{3.257116in}}{\pgfqpoint{1.851162in}{3.246517in}}{\pgfqpoint{1.851162in}{3.235467in}}%
\pgfpathcurveto{\pgfqpoint{1.851162in}{3.224417in}}{\pgfqpoint{1.855552in}{3.213818in}}{\pgfqpoint{1.863366in}{3.206004in}}%
\pgfpathcurveto{\pgfqpoint{1.871180in}{3.198191in}}{\pgfqpoint{1.881779in}{3.193800in}}{\pgfqpoint{1.892829in}{3.193800in}}%
\pgfpathclose%
\pgfusepath{stroke,fill}%
\end{pgfscope}%
\begin{pgfscope}%
\pgfpathrectangle{\pgfqpoint{0.648703in}{0.548769in}}{\pgfqpoint{5.201297in}{3.102590in}}%
\pgfusepath{clip}%
\pgfsetbuttcap%
\pgfsetroundjoin%
\definecolor{currentfill}{rgb}{1.000000,0.498039,0.054902}%
\pgfsetfillcolor{currentfill}%
\pgfsetlinewidth{1.003750pt}%
\definecolor{currentstroke}{rgb}{1.000000,0.498039,0.054902}%
\pgfsetstrokecolor{currentstroke}%
\pgfsetdash{}{0pt}%
\pgfpathmoveto{\pgfqpoint{2.125376in}{3.193800in}}%
\pgfpathcurveto{\pgfqpoint{2.136426in}{3.193800in}}{\pgfqpoint{2.147025in}{3.198191in}}{\pgfqpoint{2.154838in}{3.206004in}}%
\pgfpathcurveto{\pgfqpoint{2.162652in}{3.213818in}}{\pgfqpoint{2.167042in}{3.224417in}}{\pgfqpoint{2.167042in}{3.235467in}}%
\pgfpathcurveto{\pgfqpoint{2.167042in}{3.246517in}}{\pgfqpoint{2.162652in}{3.257116in}}{\pgfqpoint{2.154838in}{3.264930in}}%
\pgfpathcurveto{\pgfqpoint{2.147025in}{3.272743in}}{\pgfqpoint{2.136426in}{3.277134in}}{\pgfqpoint{2.125376in}{3.277134in}}%
\pgfpathcurveto{\pgfqpoint{2.114325in}{3.277134in}}{\pgfqpoint{2.103726in}{3.272743in}}{\pgfqpoint{2.095913in}{3.264930in}}%
\pgfpathcurveto{\pgfqpoint{2.088099in}{3.257116in}}{\pgfqpoint{2.083709in}{3.246517in}}{\pgfqpoint{2.083709in}{3.235467in}}%
\pgfpathcurveto{\pgfqpoint{2.083709in}{3.224417in}}{\pgfqpoint{2.088099in}{3.213818in}}{\pgfqpoint{2.095913in}{3.206004in}}%
\pgfpathcurveto{\pgfqpoint{2.103726in}{3.198191in}}{\pgfqpoint{2.114325in}{3.193800in}}{\pgfqpoint{2.125376in}{3.193800in}}%
\pgfpathclose%
\pgfusepath{stroke,fill}%
\end{pgfscope}%
\begin{pgfscope}%
\pgfpathrectangle{\pgfqpoint{0.648703in}{0.548769in}}{\pgfqpoint{5.201297in}{3.102590in}}%
\pgfusepath{clip}%
\pgfsetbuttcap%
\pgfsetroundjoin%
\definecolor{currentfill}{rgb}{0.121569,0.466667,0.705882}%
\pgfsetfillcolor{currentfill}%
\pgfsetlinewidth{1.003750pt}%
\definecolor{currentstroke}{rgb}{0.121569,0.466667,0.705882}%
\pgfsetstrokecolor{currentstroke}%
\pgfsetdash{}{0pt}%
\pgfpathmoveto{\pgfqpoint{2.202891in}{0.652358in}}%
\pgfpathcurveto{\pgfqpoint{2.213941in}{0.652358in}}{\pgfqpoint{2.224540in}{0.656748in}}{\pgfqpoint{2.232354in}{0.664562in}}%
\pgfpathcurveto{\pgfqpoint{2.240168in}{0.672375in}}{\pgfqpoint{2.244558in}{0.682974in}}{\pgfqpoint{2.244558in}{0.694024in}}%
\pgfpathcurveto{\pgfqpoint{2.244558in}{0.705074in}}{\pgfqpoint{2.240168in}{0.715673in}}{\pgfqpoint{2.232354in}{0.723487in}}%
\pgfpathcurveto{\pgfqpoint{2.224540in}{0.731301in}}{\pgfqpoint{2.213941in}{0.735691in}}{\pgfqpoint{2.202891in}{0.735691in}}%
\pgfpathcurveto{\pgfqpoint{2.191841in}{0.735691in}}{\pgfqpoint{2.181242in}{0.731301in}}{\pgfqpoint{2.173428in}{0.723487in}}%
\pgfpathcurveto{\pgfqpoint{2.165615in}{0.715673in}}{\pgfqpoint{2.161224in}{0.705074in}}{\pgfqpoint{2.161224in}{0.694024in}}%
\pgfpathcurveto{\pgfqpoint{2.161224in}{0.682974in}}{\pgfqpoint{2.165615in}{0.672375in}}{\pgfqpoint{2.173428in}{0.664562in}}%
\pgfpathcurveto{\pgfqpoint{2.181242in}{0.656748in}}{\pgfqpoint{2.191841in}{0.652358in}}{\pgfqpoint{2.202891in}{0.652358in}}%
\pgfpathclose%
\pgfusepath{stroke,fill}%
\end{pgfscope}%
\begin{pgfscope}%
\pgfpathrectangle{\pgfqpoint{0.648703in}{0.548769in}}{\pgfqpoint{5.201297in}{3.102590in}}%
\pgfusepath{clip}%
\pgfsetbuttcap%
\pgfsetroundjoin%
\definecolor{currentfill}{rgb}{0.121569,0.466667,0.705882}%
\pgfsetfillcolor{currentfill}%
\pgfsetlinewidth{1.003750pt}%
\definecolor{currentstroke}{rgb}{0.121569,0.466667,0.705882}%
\pgfsetstrokecolor{currentstroke}%
\pgfsetdash{}{0pt}%
\pgfpathmoveto{\pgfqpoint{1.117673in}{0.648129in}}%
\pgfpathcurveto{\pgfqpoint{1.128723in}{0.648129in}}{\pgfqpoint{1.139322in}{0.652519in}}{\pgfqpoint{1.147136in}{0.660333in}}%
\pgfpathcurveto{\pgfqpoint{1.154949in}{0.668146in}}{\pgfqpoint{1.159339in}{0.678745in}}{\pgfqpoint{1.159339in}{0.689796in}}%
\pgfpathcurveto{\pgfqpoint{1.159339in}{0.700846in}}{\pgfqpoint{1.154949in}{0.711445in}}{\pgfqpoint{1.147136in}{0.719258in}}%
\pgfpathcurveto{\pgfqpoint{1.139322in}{0.727072in}}{\pgfqpoint{1.128723in}{0.731462in}}{\pgfqpoint{1.117673in}{0.731462in}}%
\pgfpathcurveto{\pgfqpoint{1.106623in}{0.731462in}}{\pgfqpoint{1.096024in}{0.727072in}}{\pgfqpoint{1.088210in}{0.719258in}}%
\pgfpathcurveto{\pgfqpoint{1.080396in}{0.711445in}}{\pgfqpoint{1.076006in}{0.700846in}}{\pgfqpoint{1.076006in}{0.689796in}}%
\pgfpathcurveto{\pgfqpoint{1.076006in}{0.678745in}}{\pgfqpoint{1.080396in}{0.668146in}}{\pgfqpoint{1.088210in}{0.660333in}}%
\pgfpathcurveto{\pgfqpoint{1.096024in}{0.652519in}}{\pgfqpoint{1.106623in}{0.648129in}}{\pgfqpoint{1.117673in}{0.648129in}}%
\pgfpathclose%
\pgfusepath{stroke,fill}%
\end{pgfscope}%
\begin{pgfscope}%
\pgfpathrectangle{\pgfqpoint{0.648703in}{0.548769in}}{\pgfqpoint{5.201297in}{3.102590in}}%
\pgfusepath{clip}%
\pgfsetbuttcap%
\pgfsetroundjoin%
\definecolor{currentfill}{rgb}{0.121569,0.466667,0.705882}%
\pgfsetfillcolor{currentfill}%
\pgfsetlinewidth{1.003750pt}%
\definecolor{currentstroke}{rgb}{0.121569,0.466667,0.705882}%
\pgfsetstrokecolor{currentstroke}%
\pgfsetdash{}{0pt}%
\pgfpathmoveto{\pgfqpoint{1.117673in}{0.648129in}}%
\pgfpathcurveto{\pgfqpoint{1.128723in}{0.648129in}}{\pgfqpoint{1.139322in}{0.652519in}}{\pgfqpoint{1.147136in}{0.660333in}}%
\pgfpathcurveto{\pgfqpoint{1.154949in}{0.668146in}}{\pgfqpoint{1.159339in}{0.678745in}}{\pgfqpoint{1.159339in}{0.689796in}}%
\pgfpathcurveto{\pgfqpoint{1.159339in}{0.700846in}}{\pgfqpoint{1.154949in}{0.711445in}}{\pgfqpoint{1.147136in}{0.719258in}}%
\pgfpathcurveto{\pgfqpoint{1.139322in}{0.727072in}}{\pgfqpoint{1.128723in}{0.731462in}}{\pgfqpoint{1.117673in}{0.731462in}}%
\pgfpathcurveto{\pgfqpoint{1.106623in}{0.731462in}}{\pgfqpoint{1.096024in}{0.727072in}}{\pgfqpoint{1.088210in}{0.719258in}}%
\pgfpathcurveto{\pgfqpoint{1.080396in}{0.711445in}}{\pgfqpoint{1.076006in}{0.700846in}}{\pgfqpoint{1.076006in}{0.689796in}}%
\pgfpathcurveto{\pgfqpoint{1.076006in}{0.678745in}}{\pgfqpoint{1.080396in}{0.668146in}}{\pgfqpoint{1.088210in}{0.660333in}}%
\pgfpathcurveto{\pgfqpoint{1.096024in}{0.652519in}}{\pgfqpoint{1.106623in}{0.648129in}}{\pgfqpoint{1.117673in}{0.648129in}}%
\pgfpathclose%
\pgfusepath{stroke,fill}%
\end{pgfscope}%
\begin{pgfscope}%
\pgfpathrectangle{\pgfqpoint{0.648703in}{0.548769in}}{\pgfqpoint{5.201297in}{3.102590in}}%
\pgfusepath{clip}%
\pgfsetbuttcap%
\pgfsetroundjoin%
\definecolor{currentfill}{rgb}{0.121569,0.466667,0.705882}%
\pgfsetfillcolor{currentfill}%
\pgfsetlinewidth{1.003750pt}%
\definecolor{currentstroke}{rgb}{0.121569,0.466667,0.705882}%
\pgfsetstrokecolor{currentstroke}%
\pgfsetdash{}{0pt}%
\pgfpathmoveto{\pgfqpoint{1.660282in}{0.648129in}}%
\pgfpathcurveto{\pgfqpoint{1.671332in}{0.648129in}}{\pgfqpoint{1.681931in}{0.652519in}}{\pgfqpoint{1.689745in}{0.660333in}}%
\pgfpathcurveto{\pgfqpoint{1.697558in}{0.668146in}}{\pgfqpoint{1.701949in}{0.678745in}}{\pgfqpoint{1.701949in}{0.689796in}}%
\pgfpathcurveto{\pgfqpoint{1.701949in}{0.700846in}}{\pgfqpoint{1.697558in}{0.711445in}}{\pgfqpoint{1.689745in}{0.719258in}}%
\pgfpathcurveto{\pgfqpoint{1.681931in}{0.727072in}}{\pgfqpoint{1.671332in}{0.731462in}}{\pgfqpoint{1.660282in}{0.731462in}}%
\pgfpathcurveto{\pgfqpoint{1.649232in}{0.731462in}}{\pgfqpoint{1.638633in}{0.727072in}}{\pgfqpoint{1.630819in}{0.719258in}}%
\pgfpathcurveto{\pgfqpoint{1.623006in}{0.711445in}}{\pgfqpoint{1.618615in}{0.700846in}}{\pgfqpoint{1.618615in}{0.689796in}}%
\pgfpathcurveto{\pgfqpoint{1.618615in}{0.678745in}}{\pgfqpoint{1.623006in}{0.668146in}}{\pgfqpoint{1.630819in}{0.660333in}}%
\pgfpathcurveto{\pgfqpoint{1.638633in}{0.652519in}}{\pgfqpoint{1.649232in}{0.648129in}}{\pgfqpoint{1.660282in}{0.648129in}}%
\pgfpathclose%
\pgfusepath{stroke,fill}%
\end{pgfscope}%
\begin{pgfscope}%
\pgfpathrectangle{\pgfqpoint{0.648703in}{0.548769in}}{\pgfqpoint{5.201297in}{3.102590in}}%
\pgfusepath{clip}%
\pgfsetbuttcap%
\pgfsetroundjoin%
\definecolor{currentfill}{rgb}{0.121569,0.466667,0.705882}%
\pgfsetfillcolor{currentfill}%
\pgfsetlinewidth{1.003750pt}%
\definecolor{currentstroke}{rgb}{0.121569,0.466667,0.705882}%
\pgfsetstrokecolor{currentstroke}%
\pgfsetdash{}{0pt}%
\pgfpathmoveto{\pgfqpoint{0.962642in}{0.648129in}}%
\pgfpathcurveto{\pgfqpoint{0.973692in}{0.648129in}}{\pgfqpoint{0.984291in}{0.652519in}}{\pgfqpoint{0.992104in}{0.660333in}}%
\pgfpathcurveto{\pgfqpoint{0.999918in}{0.668146in}}{\pgfqpoint{1.004308in}{0.678745in}}{\pgfqpoint{1.004308in}{0.689796in}}%
\pgfpathcurveto{\pgfqpoint{1.004308in}{0.700846in}}{\pgfqpoint{0.999918in}{0.711445in}}{\pgfqpoint{0.992104in}{0.719258in}}%
\pgfpathcurveto{\pgfqpoint{0.984291in}{0.727072in}}{\pgfqpoint{0.973692in}{0.731462in}}{\pgfqpoint{0.962642in}{0.731462in}}%
\pgfpathcurveto{\pgfqpoint{0.951591in}{0.731462in}}{\pgfqpoint{0.940992in}{0.727072in}}{\pgfqpoint{0.933179in}{0.719258in}}%
\pgfpathcurveto{\pgfqpoint{0.925365in}{0.711445in}}{\pgfqpoint{0.920975in}{0.700846in}}{\pgfqpoint{0.920975in}{0.689796in}}%
\pgfpathcurveto{\pgfqpoint{0.920975in}{0.678745in}}{\pgfqpoint{0.925365in}{0.668146in}}{\pgfqpoint{0.933179in}{0.660333in}}%
\pgfpathcurveto{\pgfqpoint{0.940992in}{0.652519in}}{\pgfqpoint{0.951591in}{0.648129in}}{\pgfqpoint{0.962642in}{0.648129in}}%
\pgfpathclose%
\pgfusepath{stroke,fill}%
\end{pgfscope}%
\begin{pgfscope}%
\pgfpathrectangle{\pgfqpoint{0.648703in}{0.548769in}}{\pgfqpoint{5.201297in}{3.102590in}}%
\pgfusepath{clip}%
\pgfsetbuttcap%
\pgfsetroundjoin%
\definecolor{currentfill}{rgb}{0.121569,0.466667,0.705882}%
\pgfsetfillcolor{currentfill}%
\pgfsetlinewidth{1.003750pt}%
\definecolor{currentstroke}{rgb}{0.121569,0.466667,0.705882}%
\pgfsetstrokecolor{currentstroke}%
\pgfsetdash{}{0pt}%
\pgfpathmoveto{\pgfqpoint{1.272704in}{0.648129in}}%
\pgfpathcurveto{\pgfqpoint{1.283754in}{0.648129in}}{\pgfqpoint{1.294353in}{0.652519in}}{\pgfqpoint{1.302167in}{0.660333in}}%
\pgfpathcurveto{\pgfqpoint{1.309980in}{0.668146in}}{\pgfqpoint{1.314371in}{0.678745in}}{\pgfqpoint{1.314371in}{0.689796in}}%
\pgfpathcurveto{\pgfqpoint{1.314371in}{0.700846in}}{\pgfqpoint{1.309980in}{0.711445in}}{\pgfqpoint{1.302167in}{0.719258in}}%
\pgfpathcurveto{\pgfqpoint{1.294353in}{0.727072in}}{\pgfqpoint{1.283754in}{0.731462in}}{\pgfqpoint{1.272704in}{0.731462in}}%
\pgfpathcurveto{\pgfqpoint{1.261654in}{0.731462in}}{\pgfqpoint{1.251055in}{0.727072in}}{\pgfqpoint{1.243241in}{0.719258in}}%
\pgfpathcurveto{\pgfqpoint{1.235428in}{0.711445in}}{\pgfqpoint{1.231037in}{0.700846in}}{\pgfqpoint{1.231037in}{0.689796in}}%
\pgfpathcurveto{\pgfqpoint{1.231037in}{0.678745in}}{\pgfqpoint{1.235428in}{0.668146in}}{\pgfqpoint{1.243241in}{0.660333in}}%
\pgfpathcurveto{\pgfqpoint{1.251055in}{0.652519in}}{\pgfqpoint{1.261654in}{0.648129in}}{\pgfqpoint{1.272704in}{0.648129in}}%
\pgfpathclose%
\pgfusepath{stroke,fill}%
\end{pgfscope}%
\begin{pgfscope}%
\pgfpathrectangle{\pgfqpoint{0.648703in}{0.548769in}}{\pgfqpoint{5.201297in}{3.102590in}}%
\pgfusepath{clip}%
\pgfsetbuttcap%
\pgfsetroundjoin%
\definecolor{currentfill}{rgb}{0.839216,0.152941,0.156863}%
\pgfsetfillcolor{currentfill}%
\pgfsetlinewidth{1.003750pt}%
\definecolor{currentstroke}{rgb}{0.839216,0.152941,0.156863}%
\pgfsetstrokecolor{currentstroke}%
\pgfsetdash{}{0pt}%
\pgfpathmoveto{\pgfqpoint{2.512954in}{3.189572in}}%
\pgfpathcurveto{\pgfqpoint{2.524004in}{3.189572in}}{\pgfqpoint{2.534603in}{3.193962in}}{\pgfqpoint{2.542416in}{3.201775in}}%
\pgfpathcurveto{\pgfqpoint{2.550230in}{3.209589in}}{\pgfqpoint{2.554620in}{3.220188in}}{\pgfqpoint{2.554620in}{3.231238in}}%
\pgfpathcurveto{\pgfqpoint{2.554620in}{3.242288in}}{\pgfqpoint{2.550230in}{3.252887in}}{\pgfqpoint{2.542416in}{3.260701in}}%
\pgfpathcurveto{\pgfqpoint{2.534603in}{3.268515in}}{\pgfqpoint{2.524004in}{3.272905in}}{\pgfqpoint{2.512954in}{3.272905in}}%
\pgfpathcurveto{\pgfqpoint{2.501903in}{3.272905in}}{\pgfqpoint{2.491304in}{3.268515in}}{\pgfqpoint{2.483491in}{3.260701in}}%
\pgfpathcurveto{\pgfqpoint{2.475677in}{3.252887in}}{\pgfqpoint{2.471287in}{3.242288in}}{\pgfqpoint{2.471287in}{3.231238in}}%
\pgfpathcurveto{\pgfqpoint{2.471287in}{3.220188in}}{\pgfqpoint{2.475677in}{3.209589in}}{\pgfqpoint{2.483491in}{3.201775in}}%
\pgfpathcurveto{\pgfqpoint{2.491304in}{3.193962in}}{\pgfqpoint{2.501903in}{3.189572in}}{\pgfqpoint{2.512954in}{3.189572in}}%
\pgfpathclose%
\pgfusepath{stroke,fill}%
\end{pgfscope}%
\begin{pgfscope}%
\pgfpathrectangle{\pgfqpoint{0.648703in}{0.548769in}}{\pgfqpoint{5.201297in}{3.102590in}}%
\pgfusepath{clip}%
\pgfsetbuttcap%
\pgfsetroundjoin%
\definecolor{currentfill}{rgb}{0.121569,0.466667,0.705882}%
\pgfsetfillcolor{currentfill}%
\pgfsetlinewidth{1.003750pt}%
\definecolor{currentstroke}{rgb}{0.121569,0.466667,0.705882}%
\pgfsetstrokecolor{currentstroke}%
\pgfsetdash{}{0pt}%
\pgfpathmoveto{\pgfqpoint{2.745500in}{3.181114in}}%
\pgfpathcurveto{\pgfqpoint{2.756550in}{3.181114in}}{\pgfqpoint{2.767149in}{3.185504in}}{\pgfqpoint{2.774963in}{3.193318in}}%
\pgfpathcurveto{\pgfqpoint{2.782777in}{3.201132in}}{\pgfqpoint{2.787167in}{3.211731in}}{\pgfqpoint{2.787167in}{3.222781in}}%
\pgfpathcurveto{\pgfqpoint{2.787167in}{3.233831in}}{\pgfqpoint{2.782777in}{3.244430in}}{\pgfqpoint{2.774963in}{3.252244in}}%
\pgfpathcurveto{\pgfqpoint{2.767149in}{3.260057in}}{\pgfqpoint{2.756550in}{3.264448in}}{\pgfqpoint{2.745500in}{3.264448in}}%
\pgfpathcurveto{\pgfqpoint{2.734450in}{3.264448in}}{\pgfqpoint{2.723851in}{3.260057in}}{\pgfqpoint{2.716038in}{3.252244in}}%
\pgfpathcurveto{\pgfqpoint{2.708224in}{3.244430in}}{\pgfqpoint{2.703834in}{3.233831in}}{\pgfqpoint{2.703834in}{3.222781in}}%
\pgfpathcurveto{\pgfqpoint{2.703834in}{3.211731in}}{\pgfqpoint{2.708224in}{3.201132in}}{\pgfqpoint{2.716038in}{3.193318in}}%
\pgfpathcurveto{\pgfqpoint{2.723851in}{3.185504in}}{\pgfqpoint{2.734450in}{3.181114in}}{\pgfqpoint{2.745500in}{3.181114in}}%
\pgfpathclose%
\pgfusepath{stroke,fill}%
\end{pgfscope}%
\begin{pgfscope}%
\pgfpathrectangle{\pgfqpoint{0.648703in}{0.548769in}}{\pgfqpoint{5.201297in}{3.102590in}}%
\pgfusepath{clip}%
\pgfsetbuttcap%
\pgfsetroundjoin%
\definecolor{currentfill}{rgb}{1.000000,0.498039,0.054902}%
\pgfsetfillcolor{currentfill}%
\pgfsetlinewidth{1.003750pt}%
\definecolor{currentstroke}{rgb}{1.000000,0.498039,0.054902}%
\pgfsetstrokecolor{currentstroke}%
\pgfsetdash{}{0pt}%
\pgfpathmoveto{\pgfqpoint{2.590469in}{3.193800in}}%
\pgfpathcurveto{\pgfqpoint{2.601519in}{3.193800in}}{\pgfqpoint{2.612118in}{3.198191in}}{\pgfqpoint{2.619932in}{3.206004in}}%
\pgfpathcurveto{\pgfqpoint{2.627746in}{3.213818in}}{\pgfqpoint{2.632136in}{3.224417in}}{\pgfqpoint{2.632136in}{3.235467in}}%
\pgfpathcurveto{\pgfqpoint{2.632136in}{3.246517in}}{\pgfqpoint{2.627746in}{3.257116in}}{\pgfqpoint{2.619932in}{3.264930in}}%
\pgfpathcurveto{\pgfqpoint{2.612118in}{3.272743in}}{\pgfqpoint{2.601519in}{3.277134in}}{\pgfqpoint{2.590469in}{3.277134in}}%
\pgfpathcurveto{\pgfqpoint{2.579419in}{3.277134in}}{\pgfqpoint{2.568820in}{3.272743in}}{\pgfqpoint{2.561006in}{3.264930in}}%
\pgfpathcurveto{\pgfqpoint{2.553193in}{3.257116in}}{\pgfqpoint{2.548802in}{3.246517in}}{\pgfqpoint{2.548802in}{3.235467in}}%
\pgfpathcurveto{\pgfqpoint{2.548802in}{3.224417in}}{\pgfqpoint{2.553193in}{3.213818in}}{\pgfqpoint{2.561006in}{3.206004in}}%
\pgfpathcurveto{\pgfqpoint{2.568820in}{3.198191in}}{\pgfqpoint{2.579419in}{3.193800in}}{\pgfqpoint{2.590469in}{3.193800in}}%
\pgfpathclose%
\pgfusepath{stroke,fill}%
\end{pgfscope}%
\begin{pgfscope}%
\pgfpathrectangle{\pgfqpoint{0.648703in}{0.548769in}}{\pgfqpoint{5.201297in}{3.102590in}}%
\pgfusepath{clip}%
\pgfsetbuttcap%
\pgfsetroundjoin%
\definecolor{currentfill}{rgb}{1.000000,0.498039,0.054902}%
\pgfsetfillcolor{currentfill}%
\pgfsetlinewidth{1.003750pt}%
\definecolor{currentstroke}{rgb}{1.000000,0.498039,0.054902}%
\pgfsetstrokecolor{currentstroke}%
\pgfsetdash{}{0pt}%
\pgfpathmoveto{\pgfqpoint{2.900532in}{3.193800in}}%
\pgfpathcurveto{\pgfqpoint{2.911582in}{3.193800in}}{\pgfqpoint{2.922181in}{3.198191in}}{\pgfqpoint{2.929994in}{3.206004in}}%
\pgfpathcurveto{\pgfqpoint{2.937808in}{3.213818in}}{\pgfqpoint{2.942198in}{3.224417in}}{\pgfqpoint{2.942198in}{3.235467in}}%
\pgfpathcurveto{\pgfqpoint{2.942198in}{3.246517in}}{\pgfqpoint{2.937808in}{3.257116in}}{\pgfqpoint{2.929994in}{3.264930in}}%
\pgfpathcurveto{\pgfqpoint{2.922181in}{3.272743in}}{\pgfqpoint{2.911582in}{3.277134in}}{\pgfqpoint{2.900532in}{3.277134in}}%
\pgfpathcurveto{\pgfqpoint{2.889481in}{3.277134in}}{\pgfqpoint{2.878882in}{3.272743in}}{\pgfqpoint{2.871069in}{3.264930in}}%
\pgfpathcurveto{\pgfqpoint{2.863255in}{3.257116in}}{\pgfqpoint{2.858865in}{3.246517in}}{\pgfqpoint{2.858865in}{3.235467in}}%
\pgfpathcurveto{\pgfqpoint{2.858865in}{3.224417in}}{\pgfqpoint{2.863255in}{3.213818in}}{\pgfqpoint{2.871069in}{3.206004in}}%
\pgfpathcurveto{\pgfqpoint{2.878882in}{3.198191in}}{\pgfqpoint{2.889481in}{3.193800in}}{\pgfqpoint{2.900532in}{3.193800in}}%
\pgfpathclose%
\pgfusepath{stroke,fill}%
\end{pgfscope}%
\begin{pgfscope}%
\pgfpathrectangle{\pgfqpoint{0.648703in}{0.548769in}}{\pgfqpoint{5.201297in}{3.102590in}}%
\pgfusepath{clip}%
\pgfsetbuttcap%
\pgfsetroundjoin%
\definecolor{currentfill}{rgb}{0.121569,0.466667,0.705882}%
\pgfsetfillcolor{currentfill}%
\pgfsetlinewidth{1.003750pt}%
\definecolor{currentstroke}{rgb}{0.121569,0.466667,0.705882}%
\pgfsetstrokecolor{currentstroke}%
\pgfsetdash{}{0pt}%
\pgfpathmoveto{\pgfqpoint{1.195188in}{2.551039in}}%
\pgfpathcurveto{\pgfqpoint{1.206239in}{2.551039in}}{\pgfqpoint{1.216838in}{2.555430in}}{\pgfqpoint{1.224651in}{2.563243in}}%
\pgfpathcurveto{\pgfqpoint{1.232465in}{2.571057in}}{\pgfqpoint{1.236855in}{2.581656in}}{\pgfqpoint{1.236855in}{2.592706in}}%
\pgfpathcurveto{\pgfqpoint{1.236855in}{2.603756in}}{\pgfqpoint{1.232465in}{2.614355in}}{\pgfqpoint{1.224651in}{2.622169in}}%
\pgfpathcurveto{\pgfqpoint{1.216838in}{2.629982in}}{\pgfqpoint{1.206239in}{2.634373in}}{\pgfqpoint{1.195188in}{2.634373in}}%
\pgfpathcurveto{\pgfqpoint{1.184138in}{2.634373in}}{\pgfqpoint{1.173539in}{2.629982in}}{\pgfqpoint{1.165726in}{2.622169in}}%
\pgfpathcurveto{\pgfqpoint{1.157912in}{2.614355in}}{\pgfqpoint{1.153522in}{2.603756in}}{\pgfqpoint{1.153522in}{2.592706in}}%
\pgfpathcurveto{\pgfqpoint{1.153522in}{2.581656in}}{\pgfqpoint{1.157912in}{2.571057in}}{\pgfqpoint{1.165726in}{2.563243in}}%
\pgfpathcurveto{\pgfqpoint{1.173539in}{2.555430in}}{\pgfqpoint{1.184138in}{2.551039in}}{\pgfqpoint{1.195188in}{2.551039in}}%
\pgfpathclose%
\pgfusepath{stroke,fill}%
\end{pgfscope}%
\begin{pgfscope}%
\pgfpathrectangle{\pgfqpoint{0.648703in}{0.548769in}}{\pgfqpoint{5.201297in}{3.102590in}}%
\pgfusepath{clip}%
\pgfsetbuttcap%
\pgfsetroundjoin%
\definecolor{currentfill}{rgb}{0.121569,0.466667,0.705882}%
\pgfsetfillcolor{currentfill}%
\pgfsetlinewidth{1.003750pt}%
\definecolor{currentstroke}{rgb}{0.121569,0.466667,0.705882}%
\pgfsetstrokecolor{currentstroke}%
\pgfsetdash{}{0pt}%
\pgfpathmoveto{\pgfqpoint{0.962642in}{0.648129in}}%
\pgfpathcurveto{\pgfqpoint{0.973692in}{0.648129in}}{\pgfqpoint{0.984291in}{0.652519in}}{\pgfqpoint{0.992104in}{0.660333in}}%
\pgfpathcurveto{\pgfqpoint{0.999918in}{0.668146in}}{\pgfqpoint{1.004308in}{0.678745in}}{\pgfqpoint{1.004308in}{0.689796in}}%
\pgfpathcurveto{\pgfqpoint{1.004308in}{0.700846in}}{\pgfqpoint{0.999918in}{0.711445in}}{\pgfqpoint{0.992104in}{0.719258in}}%
\pgfpathcurveto{\pgfqpoint{0.984291in}{0.727072in}}{\pgfqpoint{0.973692in}{0.731462in}}{\pgfqpoint{0.962642in}{0.731462in}}%
\pgfpathcurveto{\pgfqpoint{0.951591in}{0.731462in}}{\pgfqpoint{0.940992in}{0.727072in}}{\pgfqpoint{0.933179in}{0.719258in}}%
\pgfpathcurveto{\pgfqpoint{0.925365in}{0.711445in}}{\pgfqpoint{0.920975in}{0.700846in}}{\pgfqpoint{0.920975in}{0.689796in}}%
\pgfpathcurveto{\pgfqpoint{0.920975in}{0.678745in}}{\pgfqpoint{0.925365in}{0.668146in}}{\pgfqpoint{0.933179in}{0.660333in}}%
\pgfpathcurveto{\pgfqpoint{0.940992in}{0.652519in}}{\pgfqpoint{0.951591in}{0.648129in}}{\pgfqpoint{0.962642in}{0.648129in}}%
\pgfpathclose%
\pgfusepath{stroke,fill}%
\end{pgfscope}%
\begin{pgfscope}%
\pgfpathrectangle{\pgfqpoint{0.648703in}{0.548769in}}{\pgfqpoint{5.201297in}{3.102590in}}%
\pgfusepath{clip}%
\pgfsetbuttcap%
\pgfsetroundjoin%
\definecolor{currentfill}{rgb}{1.000000,0.498039,0.054902}%
\pgfsetfillcolor{currentfill}%
\pgfsetlinewidth{1.003750pt}%
\definecolor{currentstroke}{rgb}{1.000000,0.498039,0.054902}%
\pgfsetstrokecolor{currentstroke}%
\pgfsetdash{}{0pt}%
\pgfpathmoveto{\pgfqpoint{5.613577in}{3.198029in}}%
\pgfpathcurveto{\pgfqpoint{5.624628in}{3.198029in}}{\pgfqpoint{5.635227in}{3.202419in}}{\pgfqpoint{5.643040in}{3.210233in}}%
\pgfpathcurveto{\pgfqpoint{5.650854in}{3.218046in}}{\pgfqpoint{5.655244in}{3.228646in}}{\pgfqpoint{5.655244in}{3.239696in}}%
\pgfpathcurveto{\pgfqpoint{5.655244in}{3.250746in}}{\pgfqpoint{5.650854in}{3.261345in}}{\pgfqpoint{5.643040in}{3.269158in}}%
\pgfpathcurveto{\pgfqpoint{5.635227in}{3.276972in}}{\pgfqpoint{5.624628in}{3.281362in}}{\pgfqpoint{5.613577in}{3.281362in}}%
\pgfpathcurveto{\pgfqpoint{5.602527in}{3.281362in}}{\pgfqpoint{5.591928in}{3.276972in}}{\pgfqpoint{5.584115in}{3.269158in}}%
\pgfpathcurveto{\pgfqpoint{5.576301in}{3.261345in}}{\pgfqpoint{5.571911in}{3.250746in}}{\pgfqpoint{5.571911in}{3.239696in}}%
\pgfpathcurveto{\pgfqpoint{5.571911in}{3.228646in}}{\pgfqpoint{5.576301in}{3.218046in}}{\pgfqpoint{5.584115in}{3.210233in}}%
\pgfpathcurveto{\pgfqpoint{5.591928in}{3.202419in}}{\pgfqpoint{5.602527in}{3.198029in}}{\pgfqpoint{5.613577in}{3.198029in}}%
\pgfpathclose%
\pgfusepath{stroke,fill}%
\end{pgfscope}%
\begin{pgfscope}%
\pgfpathrectangle{\pgfqpoint{0.648703in}{0.548769in}}{\pgfqpoint{5.201297in}{3.102590in}}%
\pgfusepath{clip}%
\pgfsetbuttcap%
\pgfsetroundjoin%
\definecolor{currentfill}{rgb}{1.000000,0.498039,0.054902}%
\pgfsetfillcolor{currentfill}%
\pgfsetlinewidth{1.003750pt}%
\definecolor{currentstroke}{rgb}{1.000000,0.498039,0.054902}%
\pgfsetstrokecolor{currentstroke}%
\pgfsetdash{}{0pt}%
\pgfpathmoveto{\pgfqpoint{2.667985in}{3.189572in}}%
\pgfpathcurveto{\pgfqpoint{2.679035in}{3.189572in}}{\pgfqpoint{2.689634in}{3.193962in}}{\pgfqpoint{2.697448in}{3.201775in}}%
\pgfpathcurveto{\pgfqpoint{2.705261in}{3.209589in}}{\pgfqpoint{2.709651in}{3.220188in}}{\pgfqpoint{2.709651in}{3.231238in}}%
\pgfpathcurveto{\pgfqpoint{2.709651in}{3.242288in}}{\pgfqpoint{2.705261in}{3.252887in}}{\pgfqpoint{2.697448in}{3.260701in}}%
\pgfpathcurveto{\pgfqpoint{2.689634in}{3.268515in}}{\pgfqpoint{2.679035in}{3.272905in}}{\pgfqpoint{2.667985in}{3.272905in}}%
\pgfpathcurveto{\pgfqpoint{2.656935in}{3.272905in}}{\pgfqpoint{2.646336in}{3.268515in}}{\pgfqpoint{2.638522in}{3.260701in}}%
\pgfpathcurveto{\pgfqpoint{2.630708in}{3.252887in}}{\pgfqpoint{2.626318in}{3.242288in}}{\pgfqpoint{2.626318in}{3.231238in}}%
\pgfpathcurveto{\pgfqpoint{2.626318in}{3.220188in}}{\pgfqpoint{2.630708in}{3.209589in}}{\pgfqpoint{2.638522in}{3.201775in}}%
\pgfpathcurveto{\pgfqpoint{2.646336in}{3.193962in}}{\pgfqpoint{2.656935in}{3.189572in}}{\pgfqpoint{2.667985in}{3.189572in}}%
\pgfpathclose%
\pgfusepath{stroke,fill}%
\end{pgfscope}%
\begin{pgfscope}%
\pgfpathrectangle{\pgfqpoint{0.648703in}{0.548769in}}{\pgfqpoint{5.201297in}{3.102590in}}%
\pgfusepath{clip}%
\pgfsetbuttcap%
\pgfsetroundjoin%
\definecolor{currentfill}{rgb}{0.121569,0.466667,0.705882}%
\pgfsetfillcolor{currentfill}%
\pgfsetlinewidth{1.003750pt}%
\definecolor{currentstroke}{rgb}{0.121569,0.466667,0.705882}%
\pgfsetstrokecolor{currentstroke}%
\pgfsetdash{}{0pt}%
\pgfpathmoveto{\pgfqpoint{1.350220in}{0.808819in}}%
\pgfpathcurveto{\pgfqpoint{1.361270in}{0.808819in}}{\pgfqpoint{1.371869in}{0.813209in}}{\pgfqpoint{1.379682in}{0.821023in}}%
\pgfpathcurveto{\pgfqpoint{1.387496in}{0.828837in}}{\pgfqpoint{1.391886in}{0.839436in}}{\pgfqpoint{1.391886in}{0.850486in}}%
\pgfpathcurveto{\pgfqpoint{1.391886in}{0.861536in}}{\pgfqpoint{1.387496in}{0.872135in}}{\pgfqpoint{1.379682in}{0.879949in}}%
\pgfpathcurveto{\pgfqpoint{1.371869in}{0.887762in}}{\pgfqpoint{1.361270in}{0.892152in}}{\pgfqpoint{1.350220in}{0.892152in}}%
\pgfpathcurveto{\pgfqpoint{1.339169in}{0.892152in}}{\pgfqpoint{1.328570in}{0.887762in}}{\pgfqpoint{1.320757in}{0.879949in}}%
\pgfpathcurveto{\pgfqpoint{1.312943in}{0.872135in}}{\pgfqpoint{1.308553in}{0.861536in}}{\pgfqpoint{1.308553in}{0.850486in}}%
\pgfpathcurveto{\pgfqpoint{1.308553in}{0.839436in}}{\pgfqpoint{1.312943in}{0.828837in}}{\pgfqpoint{1.320757in}{0.821023in}}%
\pgfpathcurveto{\pgfqpoint{1.328570in}{0.813209in}}{\pgfqpoint{1.339169in}{0.808819in}}{\pgfqpoint{1.350220in}{0.808819in}}%
\pgfpathclose%
\pgfusepath{stroke,fill}%
\end{pgfscope}%
\begin{pgfscope}%
\pgfpathrectangle{\pgfqpoint{0.648703in}{0.548769in}}{\pgfqpoint{5.201297in}{3.102590in}}%
\pgfusepath{clip}%
\pgfsetbuttcap%
\pgfsetroundjoin%
\definecolor{currentfill}{rgb}{0.839216,0.152941,0.156863}%
\pgfsetfillcolor{currentfill}%
\pgfsetlinewidth{1.003750pt}%
\definecolor{currentstroke}{rgb}{0.839216,0.152941,0.156863}%
\pgfsetstrokecolor{currentstroke}%
\pgfsetdash{}{0pt}%
\pgfpathmoveto{\pgfqpoint{1.892829in}{3.198029in}}%
\pgfpathcurveto{\pgfqpoint{1.903879in}{3.198029in}}{\pgfqpoint{1.914478in}{3.202419in}}{\pgfqpoint{1.922292in}{3.210233in}}%
\pgfpathcurveto{\pgfqpoint{1.930105in}{3.218046in}}{\pgfqpoint{1.934495in}{3.228646in}}{\pgfqpoint{1.934495in}{3.239696in}}%
\pgfpathcurveto{\pgfqpoint{1.934495in}{3.250746in}}{\pgfqpoint{1.930105in}{3.261345in}}{\pgfqpoint{1.922292in}{3.269158in}}%
\pgfpathcurveto{\pgfqpoint{1.914478in}{3.276972in}}{\pgfqpoint{1.903879in}{3.281362in}}{\pgfqpoint{1.892829in}{3.281362in}}%
\pgfpathcurveto{\pgfqpoint{1.881779in}{3.281362in}}{\pgfqpoint{1.871180in}{3.276972in}}{\pgfqpoint{1.863366in}{3.269158in}}%
\pgfpathcurveto{\pgfqpoint{1.855552in}{3.261345in}}{\pgfqpoint{1.851162in}{3.250746in}}{\pgfqpoint{1.851162in}{3.239696in}}%
\pgfpathcurveto{\pgfqpoint{1.851162in}{3.228646in}}{\pgfqpoint{1.855552in}{3.218046in}}{\pgfqpoint{1.863366in}{3.210233in}}%
\pgfpathcurveto{\pgfqpoint{1.871180in}{3.202419in}}{\pgfqpoint{1.881779in}{3.198029in}}{\pgfqpoint{1.892829in}{3.198029in}}%
\pgfpathclose%
\pgfusepath{stroke,fill}%
\end{pgfscope}%
\begin{pgfscope}%
\pgfpathrectangle{\pgfqpoint{0.648703in}{0.548769in}}{\pgfqpoint{5.201297in}{3.102590in}}%
\pgfusepath{clip}%
\pgfsetbuttcap%
\pgfsetroundjoin%
\definecolor{currentfill}{rgb}{0.121569,0.466667,0.705882}%
\pgfsetfillcolor{currentfill}%
\pgfsetlinewidth{1.003750pt}%
\definecolor{currentstroke}{rgb}{0.121569,0.466667,0.705882}%
\pgfsetstrokecolor{currentstroke}%
\pgfsetdash{}{0pt}%
\pgfpathmoveto{\pgfqpoint{1.195188in}{0.648129in}}%
\pgfpathcurveto{\pgfqpoint{1.206239in}{0.648129in}}{\pgfqpoint{1.216838in}{0.652519in}}{\pgfqpoint{1.224651in}{0.660333in}}%
\pgfpathcurveto{\pgfqpoint{1.232465in}{0.668146in}}{\pgfqpoint{1.236855in}{0.678745in}}{\pgfqpoint{1.236855in}{0.689796in}}%
\pgfpathcurveto{\pgfqpoint{1.236855in}{0.700846in}}{\pgfqpoint{1.232465in}{0.711445in}}{\pgfqpoint{1.224651in}{0.719258in}}%
\pgfpathcurveto{\pgfqpoint{1.216838in}{0.727072in}}{\pgfqpoint{1.206239in}{0.731462in}}{\pgfqpoint{1.195188in}{0.731462in}}%
\pgfpathcurveto{\pgfqpoint{1.184138in}{0.731462in}}{\pgfqpoint{1.173539in}{0.727072in}}{\pgfqpoint{1.165726in}{0.719258in}}%
\pgfpathcurveto{\pgfqpoint{1.157912in}{0.711445in}}{\pgfqpoint{1.153522in}{0.700846in}}{\pgfqpoint{1.153522in}{0.689796in}}%
\pgfpathcurveto{\pgfqpoint{1.153522in}{0.678745in}}{\pgfqpoint{1.157912in}{0.668146in}}{\pgfqpoint{1.165726in}{0.660333in}}%
\pgfpathcurveto{\pgfqpoint{1.173539in}{0.652519in}}{\pgfqpoint{1.184138in}{0.648129in}}{\pgfqpoint{1.195188in}{0.648129in}}%
\pgfpathclose%
\pgfusepath{stroke,fill}%
\end{pgfscope}%
\begin{pgfscope}%
\pgfpathrectangle{\pgfqpoint{0.648703in}{0.548769in}}{\pgfqpoint{5.201297in}{3.102590in}}%
\pgfusepath{clip}%
\pgfsetbuttcap%
\pgfsetroundjoin%
\definecolor{currentfill}{rgb}{1.000000,0.498039,0.054902}%
\pgfsetfillcolor{currentfill}%
\pgfsetlinewidth{1.003750pt}%
\definecolor{currentstroke}{rgb}{1.000000,0.498039,0.054902}%
\pgfsetstrokecolor{currentstroke}%
\pgfsetdash{}{0pt}%
\pgfpathmoveto{\pgfqpoint{3.288110in}{3.185343in}}%
\pgfpathcurveto{\pgfqpoint{3.299160in}{3.185343in}}{\pgfqpoint{3.309759in}{3.189733in}}{\pgfqpoint{3.317572in}{3.197547in}}%
\pgfpathcurveto{\pgfqpoint{3.325386in}{3.205360in}}{\pgfqpoint{3.329776in}{3.215959in}}{\pgfqpoint{3.329776in}{3.227010in}}%
\pgfpathcurveto{\pgfqpoint{3.329776in}{3.238060in}}{\pgfqpoint{3.325386in}{3.248659in}}{\pgfqpoint{3.317572in}{3.256472in}}%
\pgfpathcurveto{\pgfqpoint{3.309759in}{3.264286in}}{\pgfqpoint{3.299160in}{3.268676in}}{\pgfqpoint{3.288110in}{3.268676in}}%
\pgfpathcurveto{\pgfqpoint{3.277059in}{3.268676in}}{\pgfqpoint{3.266460in}{3.264286in}}{\pgfqpoint{3.258647in}{3.256472in}}%
\pgfpathcurveto{\pgfqpoint{3.250833in}{3.248659in}}{\pgfqpoint{3.246443in}{3.238060in}}{\pgfqpoint{3.246443in}{3.227010in}}%
\pgfpathcurveto{\pgfqpoint{3.246443in}{3.215959in}}{\pgfqpoint{3.250833in}{3.205360in}}{\pgfqpoint{3.258647in}{3.197547in}}%
\pgfpathcurveto{\pgfqpoint{3.266460in}{3.189733in}}{\pgfqpoint{3.277059in}{3.185343in}}{\pgfqpoint{3.288110in}{3.185343in}}%
\pgfpathclose%
\pgfusepath{stroke,fill}%
\end{pgfscope}%
\begin{pgfscope}%
\pgfpathrectangle{\pgfqpoint{0.648703in}{0.548769in}}{\pgfqpoint{5.201297in}{3.102590in}}%
\pgfusepath{clip}%
\pgfsetbuttcap%
\pgfsetroundjoin%
\definecolor{currentfill}{rgb}{0.121569,0.466667,0.705882}%
\pgfsetfillcolor{currentfill}%
\pgfsetlinewidth{1.003750pt}%
\definecolor{currentstroke}{rgb}{0.121569,0.466667,0.705882}%
\pgfsetstrokecolor{currentstroke}%
\pgfsetdash{}{0pt}%
\pgfpathmoveto{\pgfqpoint{1.195188in}{0.939909in}}%
\pgfpathcurveto{\pgfqpoint{1.206239in}{0.939909in}}{\pgfqpoint{1.216838in}{0.944299in}}{\pgfqpoint{1.224651in}{0.952112in}}%
\pgfpathcurveto{\pgfqpoint{1.232465in}{0.959926in}}{\pgfqpoint{1.236855in}{0.970525in}}{\pgfqpoint{1.236855in}{0.981575in}}%
\pgfpathcurveto{\pgfqpoint{1.236855in}{0.992625in}}{\pgfqpoint{1.232465in}{1.003224in}}{\pgfqpoint{1.224651in}{1.011038in}}%
\pgfpathcurveto{\pgfqpoint{1.216838in}{1.018852in}}{\pgfqpoint{1.206239in}{1.023242in}}{\pgfqpoint{1.195188in}{1.023242in}}%
\pgfpathcurveto{\pgfqpoint{1.184138in}{1.023242in}}{\pgfqpoint{1.173539in}{1.018852in}}{\pgfqpoint{1.165726in}{1.011038in}}%
\pgfpathcurveto{\pgfqpoint{1.157912in}{1.003224in}}{\pgfqpoint{1.153522in}{0.992625in}}{\pgfqpoint{1.153522in}{0.981575in}}%
\pgfpathcurveto{\pgfqpoint{1.153522in}{0.970525in}}{\pgfqpoint{1.157912in}{0.959926in}}{\pgfqpoint{1.165726in}{0.952112in}}%
\pgfpathcurveto{\pgfqpoint{1.173539in}{0.944299in}}{\pgfqpoint{1.184138in}{0.939909in}}{\pgfqpoint{1.195188in}{0.939909in}}%
\pgfpathclose%
\pgfusepath{stroke,fill}%
\end{pgfscope}%
\begin{pgfscope}%
\pgfpathrectangle{\pgfqpoint{0.648703in}{0.548769in}}{\pgfqpoint{5.201297in}{3.102590in}}%
\pgfusepath{clip}%
\pgfsetbuttcap%
\pgfsetroundjoin%
\definecolor{currentfill}{rgb}{0.121569,0.466667,0.705882}%
\pgfsetfillcolor{currentfill}%
\pgfsetlinewidth{1.003750pt}%
\definecolor{currentstroke}{rgb}{0.121569,0.466667,0.705882}%
\pgfsetstrokecolor{currentstroke}%
\pgfsetdash{}{0pt}%
\pgfpathmoveto{\pgfqpoint{0.962642in}{0.648129in}}%
\pgfpathcurveto{\pgfqpoint{0.973692in}{0.648129in}}{\pgfqpoint{0.984291in}{0.652519in}}{\pgfqpoint{0.992104in}{0.660333in}}%
\pgfpathcurveto{\pgfqpoint{0.999918in}{0.668146in}}{\pgfqpoint{1.004308in}{0.678745in}}{\pgfqpoint{1.004308in}{0.689796in}}%
\pgfpathcurveto{\pgfqpoint{1.004308in}{0.700846in}}{\pgfqpoint{0.999918in}{0.711445in}}{\pgfqpoint{0.992104in}{0.719258in}}%
\pgfpathcurveto{\pgfqpoint{0.984291in}{0.727072in}}{\pgfqpoint{0.973692in}{0.731462in}}{\pgfqpoint{0.962642in}{0.731462in}}%
\pgfpathcurveto{\pgfqpoint{0.951591in}{0.731462in}}{\pgfqpoint{0.940992in}{0.727072in}}{\pgfqpoint{0.933179in}{0.719258in}}%
\pgfpathcurveto{\pgfqpoint{0.925365in}{0.711445in}}{\pgfqpoint{0.920975in}{0.700846in}}{\pgfqpoint{0.920975in}{0.689796in}}%
\pgfpathcurveto{\pgfqpoint{0.920975in}{0.678745in}}{\pgfqpoint{0.925365in}{0.668146in}}{\pgfqpoint{0.933179in}{0.660333in}}%
\pgfpathcurveto{\pgfqpoint{0.940992in}{0.652519in}}{\pgfqpoint{0.951591in}{0.648129in}}{\pgfqpoint{0.962642in}{0.648129in}}%
\pgfpathclose%
\pgfusepath{stroke,fill}%
\end{pgfscope}%
\begin{pgfscope}%
\pgfpathrectangle{\pgfqpoint{0.648703in}{0.548769in}}{\pgfqpoint{5.201297in}{3.102590in}}%
\pgfusepath{clip}%
\pgfsetbuttcap%
\pgfsetroundjoin%
\definecolor{currentfill}{rgb}{1.000000,0.498039,0.054902}%
\pgfsetfillcolor{currentfill}%
\pgfsetlinewidth{1.003750pt}%
\definecolor{currentstroke}{rgb}{1.000000,0.498039,0.054902}%
\pgfsetstrokecolor{currentstroke}%
\pgfsetdash{}{0pt}%
\pgfpathmoveto{\pgfqpoint{1.195188in}{3.185343in}}%
\pgfpathcurveto{\pgfqpoint{1.206239in}{3.185343in}}{\pgfqpoint{1.216838in}{3.189733in}}{\pgfqpoint{1.224651in}{3.197547in}}%
\pgfpathcurveto{\pgfqpoint{1.232465in}{3.205360in}}{\pgfqpoint{1.236855in}{3.215959in}}{\pgfqpoint{1.236855in}{3.227010in}}%
\pgfpathcurveto{\pgfqpoint{1.236855in}{3.238060in}}{\pgfqpoint{1.232465in}{3.248659in}}{\pgfqpoint{1.224651in}{3.256472in}}%
\pgfpathcurveto{\pgfqpoint{1.216838in}{3.264286in}}{\pgfqpoint{1.206239in}{3.268676in}}{\pgfqpoint{1.195188in}{3.268676in}}%
\pgfpathcurveto{\pgfqpoint{1.184138in}{3.268676in}}{\pgfqpoint{1.173539in}{3.264286in}}{\pgfqpoint{1.165726in}{3.256472in}}%
\pgfpathcurveto{\pgfqpoint{1.157912in}{3.248659in}}{\pgfqpoint{1.153522in}{3.238060in}}{\pgfqpoint{1.153522in}{3.227010in}}%
\pgfpathcurveto{\pgfqpoint{1.153522in}{3.215959in}}{\pgfqpoint{1.157912in}{3.205360in}}{\pgfqpoint{1.165726in}{3.197547in}}%
\pgfpathcurveto{\pgfqpoint{1.173539in}{3.189733in}}{\pgfqpoint{1.184138in}{3.185343in}}{\pgfqpoint{1.195188in}{3.185343in}}%
\pgfpathclose%
\pgfusepath{stroke,fill}%
\end{pgfscope}%
\begin{pgfscope}%
\pgfpathrectangle{\pgfqpoint{0.648703in}{0.548769in}}{\pgfqpoint{5.201297in}{3.102590in}}%
\pgfusepath{clip}%
\pgfsetbuttcap%
\pgfsetroundjoin%
\definecolor{currentfill}{rgb}{0.121569,0.466667,0.705882}%
\pgfsetfillcolor{currentfill}%
\pgfsetlinewidth{1.003750pt}%
\definecolor{currentstroke}{rgb}{0.121569,0.466667,0.705882}%
\pgfsetstrokecolor{currentstroke}%
\pgfsetdash{}{0pt}%
\pgfpathmoveto{\pgfqpoint{0.962642in}{0.648129in}}%
\pgfpathcurveto{\pgfqpoint{0.973692in}{0.648129in}}{\pgfqpoint{0.984291in}{0.652519in}}{\pgfqpoint{0.992104in}{0.660333in}}%
\pgfpathcurveto{\pgfqpoint{0.999918in}{0.668146in}}{\pgfqpoint{1.004308in}{0.678745in}}{\pgfqpoint{1.004308in}{0.689796in}}%
\pgfpathcurveto{\pgfqpoint{1.004308in}{0.700846in}}{\pgfqpoint{0.999918in}{0.711445in}}{\pgfqpoint{0.992104in}{0.719258in}}%
\pgfpathcurveto{\pgfqpoint{0.984291in}{0.727072in}}{\pgfqpoint{0.973692in}{0.731462in}}{\pgfqpoint{0.962642in}{0.731462in}}%
\pgfpathcurveto{\pgfqpoint{0.951591in}{0.731462in}}{\pgfqpoint{0.940992in}{0.727072in}}{\pgfqpoint{0.933179in}{0.719258in}}%
\pgfpathcurveto{\pgfqpoint{0.925365in}{0.711445in}}{\pgfqpoint{0.920975in}{0.700846in}}{\pgfqpoint{0.920975in}{0.689796in}}%
\pgfpathcurveto{\pgfqpoint{0.920975in}{0.678745in}}{\pgfqpoint{0.925365in}{0.668146in}}{\pgfqpoint{0.933179in}{0.660333in}}%
\pgfpathcurveto{\pgfqpoint{0.940992in}{0.652519in}}{\pgfqpoint{0.951591in}{0.648129in}}{\pgfqpoint{0.962642in}{0.648129in}}%
\pgfpathclose%
\pgfusepath{stroke,fill}%
\end{pgfscope}%
\begin{pgfscope}%
\pgfpathrectangle{\pgfqpoint{0.648703in}{0.548769in}}{\pgfqpoint{5.201297in}{3.102590in}}%
\pgfusepath{clip}%
\pgfsetbuttcap%
\pgfsetroundjoin%
\definecolor{currentfill}{rgb}{1.000000,0.498039,0.054902}%
\pgfsetfillcolor{currentfill}%
\pgfsetlinewidth{1.003750pt}%
\definecolor{currentstroke}{rgb}{1.000000,0.498039,0.054902}%
\pgfsetstrokecolor{currentstroke}%
\pgfsetdash{}{0pt}%
\pgfpathmoveto{\pgfqpoint{1.272704in}{3.278374in}}%
\pgfpathcurveto{\pgfqpoint{1.283754in}{3.278374in}}{\pgfqpoint{1.294353in}{3.282764in}}{\pgfqpoint{1.302167in}{3.290578in}}%
\pgfpathcurveto{\pgfqpoint{1.309980in}{3.298392in}}{\pgfqpoint{1.314371in}{3.308991in}}{\pgfqpoint{1.314371in}{3.320041in}}%
\pgfpathcurveto{\pgfqpoint{1.314371in}{3.331091in}}{\pgfqpoint{1.309980in}{3.341690in}}{\pgfqpoint{1.302167in}{3.349504in}}%
\pgfpathcurveto{\pgfqpoint{1.294353in}{3.357317in}}{\pgfqpoint{1.283754in}{3.361707in}}{\pgfqpoint{1.272704in}{3.361707in}}%
\pgfpathcurveto{\pgfqpoint{1.261654in}{3.361707in}}{\pgfqpoint{1.251055in}{3.357317in}}{\pgfqpoint{1.243241in}{3.349504in}}%
\pgfpathcurveto{\pgfqpoint{1.235428in}{3.341690in}}{\pgfqpoint{1.231037in}{3.331091in}}{\pgfqpoint{1.231037in}{3.320041in}}%
\pgfpathcurveto{\pgfqpoint{1.231037in}{3.308991in}}{\pgfqpoint{1.235428in}{3.298392in}}{\pgfqpoint{1.243241in}{3.290578in}}%
\pgfpathcurveto{\pgfqpoint{1.251055in}{3.282764in}}{\pgfqpoint{1.261654in}{3.278374in}}{\pgfqpoint{1.272704in}{3.278374in}}%
\pgfpathclose%
\pgfusepath{stroke,fill}%
\end{pgfscope}%
\begin{pgfscope}%
\pgfpathrectangle{\pgfqpoint{0.648703in}{0.548769in}}{\pgfqpoint{5.201297in}{3.102590in}}%
\pgfusepath{clip}%
\pgfsetbuttcap%
\pgfsetroundjoin%
\definecolor{currentfill}{rgb}{1.000000,0.498039,0.054902}%
\pgfsetfillcolor{currentfill}%
\pgfsetlinewidth{1.003750pt}%
\definecolor{currentstroke}{rgb}{1.000000,0.498039,0.054902}%
\pgfsetstrokecolor{currentstroke}%
\pgfsetdash{}{0pt}%
\pgfpathmoveto{\pgfqpoint{1.892829in}{3.202258in}}%
\pgfpathcurveto{\pgfqpoint{1.903879in}{3.202258in}}{\pgfqpoint{1.914478in}{3.206648in}}{\pgfqpoint{1.922292in}{3.214462in}}%
\pgfpathcurveto{\pgfqpoint{1.930105in}{3.222275in}}{\pgfqpoint{1.934495in}{3.232874in}}{\pgfqpoint{1.934495in}{3.243924in}}%
\pgfpathcurveto{\pgfqpoint{1.934495in}{3.254974in}}{\pgfqpoint{1.930105in}{3.265573in}}{\pgfqpoint{1.922292in}{3.273387in}}%
\pgfpathcurveto{\pgfqpoint{1.914478in}{3.281201in}}{\pgfqpoint{1.903879in}{3.285591in}}{\pgfqpoint{1.892829in}{3.285591in}}%
\pgfpathcurveto{\pgfqpoint{1.881779in}{3.285591in}}{\pgfqpoint{1.871180in}{3.281201in}}{\pgfqpoint{1.863366in}{3.273387in}}%
\pgfpathcurveto{\pgfqpoint{1.855552in}{3.265573in}}{\pgfqpoint{1.851162in}{3.254974in}}{\pgfqpoint{1.851162in}{3.243924in}}%
\pgfpathcurveto{\pgfqpoint{1.851162in}{3.232874in}}{\pgfqpoint{1.855552in}{3.222275in}}{\pgfqpoint{1.863366in}{3.214462in}}%
\pgfpathcurveto{\pgfqpoint{1.871180in}{3.206648in}}{\pgfqpoint{1.881779in}{3.202258in}}{\pgfqpoint{1.892829in}{3.202258in}}%
\pgfpathclose%
\pgfusepath{stroke,fill}%
\end{pgfscope}%
\begin{pgfscope}%
\pgfpathrectangle{\pgfqpoint{0.648703in}{0.548769in}}{\pgfqpoint{5.201297in}{3.102590in}}%
\pgfusepath{clip}%
\pgfsetbuttcap%
\pgfsetroundjoin%
\definecolor{currentfill}{rgb}{0.121569,0.466667,0.705882}%
\pgfsetfillcolor{currentfill}%
\pgfsetlinewidth{1.003750pt}%
\definecolor{currentstroke}{rgb}{0.121569,0.466667,0.705882}%
\pgfsetstrokecolor{currentstroke}%
\pgfsetdash{}{0pt}%
\pgfpathmoveto{\pgfqpoint{0.962642in}{0.648129in}}%
\pgfpathcurveto{\pgfqpoint{0.973692in}{0.648129in}}{\pgfqpoint{0.984291in}{0.652519in}}{\pgfqpoint{0.992104in}{0.660333in}}%
\pgfpathcurveto{\pgfqpoint{0.999918in}{0.668146in}}{\pgfqpoint{1.004308in}{0.678745in}}{\pgfqpoint{1.004308in}{0.689796in}}%
\pgfpathcurveto{\pgfqpoint{1.004308in}{0.700846in}}{\pgfqpoint{0.999918in}{0.711445in}}{\pgfqpoint{0.992104in}{0.719258in}}%
\pgfpathcurveto{\pgfqpoint{0.984291in}{0.727072in}}{\pgfqpoint{0.973692in}{0.731462in}}{\pgfqpoint{0.962642in}{0.731462in}}%
\pgfpathcurveto{\pgfqpoint{0.951591in}{0.731462in}}{\pgfqpoint{0.940992in}{0.727072in}}{\pgfqpoint{0.933179in}{0.719258in}}%
\pgfpathcurveto{\pgfqpoint{0.925365in}{0.711445in}}{\pgfqpoint{0.920975in}{0.700846in}}{\pgfqpoint{0.920975in}{0.689796in}}%
\pgfpathcurveto{\pgfqpoint{0.920975in}{0.678745in}}{\pgfqpoint{0.925365in}{0.668146in}}{\pgfqpoint{0.933179in}{0.660333in}}%
\pgfpathcurveto{\pgfqpoint{0.940992in}{0.652519in}}{\pgfqpoint{0.951591in}{0.648129in}}{\pgfqpoint{0.962642in}{0.648129in}}%
\pgfpathclose%
\pgfusepath{stroke,fill}%
\end{pgfscope}%
\begin{pgfscope}%
\pgfpathrectangle{\pgfqpoint{0.648703in}{0.548769in}}{\pgfqpoint{5.201297in}{3.102590in}}%
\pgfusepath{clip}%
\pgfsetbuttcap%
\pgfsetroundjoin%
\definecolor{currentfill}{rgb}{1.000000,0.498039,0.054902}%
\pgfsetfillcolor{currentfill}%
\pgfsetlinewidth{1.003750pt}%
\definecolor{currentstroke}{rgb}{1.000000,0.498039,0.054902}%
\pgfsetstrokecolor{currentstroke}%
\pgfsetdash{}{0pt}%
\pgfpathmoveto{\pgfqpoint{1.505251in}{3.202258in}}%
\pgfpathcurveto{\pgfqpoint{1.516301in}{3.202258in}}{\pgfqpoint{1.526900in}{3.206648in}}{\pgfqpoint{1.534714in}{3.214462in}}%
\pgfpathcurveto{\pgfqpoint{1.542527in}{3.222275in}}{\pgfqpoint{1.546917in}{3.232874in}}{\pgfqpoint{1.546917in}{3.243924in}}%
\pgfpathcurveto{\pgfqpoint{1.546917in}{3.254974in}}{\pgfqpoint{1.542527in}{3.265573in}}{\pgfqpoint{1.534714in}{3.273387in}}%
\pgfpathcurveto{\pgfqpoint{1.526900in}{3.281201in}}{\pgfqpoint{1.516301in}{3.285591in}}{\pgfqpoint{1.505251in}{3.285591in}}%
\pgfpathcurveto{\pgfqpoint{1.494201in}{3.285591in}}{\pgfqpoint{1.483602in}{3.281201in}}{\pgfqpoint{1.475788in}{3.273387in}}%
\pgfpathcurveto{\pgfqpoint{1.467974in}{3.265573in}}{\pgfqpoint{1.463584in}{3.254974in}}{\pgfqpoint{1.463584in}{3.243924in}}%
\pgfpathcurveto{\pgfqpoint{1.463584in}{3.232874in}}{\pgfqpoint{1.467974in}{3.222275in}}{\pgfqpoint{1.475788in}{3.214462in}}%
\pgfpathcurveto{\pgfqpoint{1.483602in}{3.206648in}}{\pgfqpoint{1.494201in}{3.202258in}}{\pgfqpoint{1.505251in}{3.202258in}}%
\pgfpathclose%
\pgfusepath{stroke,fill}%
\end{pgfscope}%
\begin{pgfscope}%
\pgfpathrectangle{\pgfqpoint{0.648703in}{0.548769in}}{\pgfqpoint{5.201297in}{3.102590in}}%
\pgfusepath{clip}%
\pgfsetbuttcap%
\pgfsetroundjoin%
\definecolor{currentfill}{rgb}{0.121569,0.466667,0.705882}%
\pgfsetfillcolor{currentfill}%
\pgfsetlinewidth{1.003750pt}%
\definecolor{currentstroke}{rgb}{0.121569,0.466667,0.705882}%
\pgfsetstrokecolor{currentstroke}%
\pgfsetdash{}{0pt}%
\pgfpathmoveto{\pgfqpoint{2.280407in}{1.726445in}}%
\pgfpathcurveto{\pgfqpoint{2.291457in}{1.726445in}}{\pgfqpoint{2.302056in}{1.730835in}}{\pgfqpoint{2.309870in}{1.738649in}}%
\pgfpathcurveto{\pgfqpoint{2.317683in}{1.746462in}}{\pgfqpoint{2.322073in}{1.757061in}}{\pgfqpoint{2.322073in}{1.768112in}}%
\pgfpathcurveto{\pgfqpoint{2.322073in}{1.779162in}}{\pgfqpoint{2.317683in}{1.789761in}}{\pgfqpoint{2.309870in}{1.797574in}}%
\pgfpathcurveto{\pgfqpoint{2.302056in}{1.805388in}}{\pgfqpoint{2.291457in}{1.809778in}}{\pgfqpoint{2.280407in}{1.809778in}}%
\pgfpathcurveto{\pgfqpoint{2.269357in}{1.809778in}}{\pgfqpoint{2.258758in}{1.805388in}}{\pgfqpoint{2.250944in}{1.797574in}}%
\pgfpathcurveto{\pgfqpoint{2.243130in}{1.789761in}}{\pgfqpoint{2.238740in}{1.779162in}}{\pgfqpoint{2.238740in}{1.768112in}}%
\pgfpathcurveto{\pgfqpoint{2.238740in}{1.757061in}}{\pgfqpoint{2.243130in}{1.746462in}}{\pgfqpoint{2.250944in}{1.738649in}}%
\pgfpathcurveto{\pgfqpoint{2.258758in}{1.730835in}}{\pgfqpoint{2.269357in}{1.726445in}}{\pgfqpoint{2.280407in}{1.726445in}}%
\pgfpathclose%
\pgfusepath{stroke,fill}%
\end{pgfscope}%
\begin{pgfscope}%
\pgfpathrectangle{\pgfqpoint{0.648703in}{0.548769in}}{\pgfqpoint{5.201297in}{3.102590in}}%
\pgfusepath{clip}%
\pgfsetbuttcap%
\pgfsetroundjoin%
\definecolor{currentfill}{rgb}{0.121569,0.466667,0.705882}%
\pgfsetfillcolor{currentfill}%
\pgfsetlinewidth{1.003750pt}%
\definecolor{currentstroke}{rgb}{0.121569,0.466667,0.705882}%
\pgfsetstrokecolor{currentstroke}%
\pgfsetdash{}{0pt}%
\pgfpathmoveto{\pgfqpoint{3.443141in}{0.918765in}}%
\pgfpathcurveto{\pgfqpoint{3.454191in}{0.918765in}}{\pgfqpoint{3.464790in}{0.923155in}}{\pgfqpoint{3.472603in}{0.930969in}}%
\pgfpathcurveto{\pgfqpoint{3.480417in}{0.938783in}}{\pgfqpoint{3.484807in}{0.949382in}}{\pgfqpoint{3.484807in}{0.960432in}}%
\pgfpathcurveto{\pgfqpoint{3.484807in}{0.971482in}}{\pgfqpoint{3.480417in}{0.982081in}}{\pgfqpoint{3.472603in}{0.989895in}}%
\pgfpathcurveto{\pgfqpoint{3.464790in}{0.997708in}}{\pgfqpoint{3.454191in}{1.002098in}}{\pgfqpoint{3.443141in}{1.002098in}}%
\pgfpathcurveto{\pgfqpoint{3.432091in}{1.002098in}}{\pgfqpoint{3.421492in}{0.997708in}}{\pgfqpoint{3.413678in}{0.989895in}}%
\pgfpathcurveto{\pgfqpoint{3.405864in}{0.982081in}}{\pgfqpoint{3.401474in}{0.971482in}}{\pgfqpoint{3.401474in}{0.960432in}}%
\pgfpathcurveto{\pgfqpoint{3.401474in}{0.949382in}}{\pgfqpoint{3.405864in}{0.938783in}}{\pgfqpoint{3.413678in}{0.930969in}}%
\pgfpathcurveto{\pgfqpoint{3.421492in}{0.923155in}}{\pgfqpoint{3.432091in}{0.918765in}}{\pgfqpoint{3.443141in}{0.918765in}}%
\pgfpathclose%
\pgfusepath{stroke,fill}%
\end{pgfscope}%
\begin{pgfscope}%
\pgfpathrectangle{\pgfqpoint{0.648703in}{0.548769in}}{\pgfqpoint{5.201297in}{3.102590in}}%
\pgfusepath{clip}%
\pgfsetbuttcap%
\pgfsetroundjoin%
\definecolor{currentfill}{rgb}{1.000000,0.498039,0.054902}%
\pgfsetfillcolor{currentfill}%
\pgfsetlinewidth{1.003750pt}%
\definecolor{currentstroke}{rgb}{1.000000,0.498039,0.054902}%
\pgfsetstrokecolor{currentstroke}%
\pgfsetdash{}{0pt}%
\pgfpathmoveto{\pgfqpoint{1.040157in}{3.185343in}}%
\pgfpathcurveto{\pgfqpoint{1.051207in}{3.185343in}}{\pgfqpoint{1.061806in}{3.189733in}}{\pgfqpoint{1.069620in}{3.197547in}}%
\pgfpathcurveto{\pgfqpoint{1.077434in}{3.205360in}}{\pgfqpoint{1.081824in}{3.215959in}}{\pgfqpoint{1.081824in}{3.227010in}}%
\pgfpathcurveto{\pgfqpoint{1.081824in}{3.238060in}}{\pgfqpoint{1.077434in}{3.248659in}}{\pgfqpoint{1.069620in}{3.256472in}}%
\pgfpathcurveto{\pgfqpoint{1.061806in}{3.264286in}}{\pgfqpoint{1.051207in}{3.268676in}}{\pgfqpoint{1.040157in}{3.268676in}}%
\pgfpathcurveto{\pgfqpoint{1.029107in}{3.268676in}}{\pgfqpoint{1.018508in}{3.264286in}}{\pgfqpoint{1.010694in}{3.256472in}}%
\pgfpathcurveto{\pgfqpoint{1.002881in}{3.248659in}}{\pgfqpoint{0.998491in}{3.238060in}}{\pgfqpoint{0.998491in}{3.227010in}}%
\pgfpathcurveto{\pgfqpoint{0.998491in}{3.215959in}}{\pgfqpoint{1.002881in}{3.205360in}}{\pgfqpoint{1.010694in}{3.197547in}}%
\pgfpathcurveto{\pgfqpoint{1.018508in}{3.189733in}}{\pgfqpoint{1.029107in}{3.185343in}}{\pgfqpoint{1.040157in}{3.185343in}}%
\pgfpathclose%
\pgfusepath{stroke,fill}%
\end{pgfscope}%
\begin{pgfscope}%
\pgfpathrectangle{\pgfqpoint{0.648703in}{0.548769in}}{\pgfqpoint{5.201297in}{3.102590in}}%
\pgfusepath{clip}%
\pgfsetbuttcap%
\pgfsetroundjoin%
\definecolor{currentfill}{rgb}{0.121569,0.466667,0.705882}%
\pgfsetfillcolor{currentfill}%
\pgfsetlinewidth{1.003750pt}%
\definecolor{currentstroke}{rgb}{0.121569,0.466667,0.705882}%
\pgfsetstrokecolor{currentstroke}%
\pgfsetdash{}{0pt}%
\pgfpathmoveto{\pgfqpoint{1.272704in}{0.648129in}}%
\pgfpathcurveto{\pgfqpoint{1.283754in}{0.648129in}}{\pgfqpoint{1.294353in}{0.652519in}}{\pgfqpoint{1.302167in}{0.660333in}}%
\pgfpathcurveto{\pgfqpoint{1.309980in}{0.668146in}}{\pgfqpoint{1.314371in}{0.678745in}}{\pgfqpoint{1.314371in}{0.689796in}}%
\pgfpathcurveto{\pgfqpoint{1.314371in}{0.700846in}}{\pgfqpoint{1.309980in}{0.711445in}}{\pgfqpoint{1.302167in}{0.719258in}}%
\pgfpathcurveto{\pgfqpoint{1.294353in}{0.727072in}}{\pgfqpoint{1.283754in}{0.731462in}}{\pgfqpoint{1.272704in}{0.731462in}}%
\pgfpathcurveto{\pgfqpoint{1.261654in}{0.731462in}}{\pgfqpoint{1.251055in}{0.727072in}}{\pgfqpoint{1.243241in}{0.719258in}}%
\pgfpathcurveto{\pgfqpoint{1.235428in}{0.711445in}}{\pgfqpoint{1.231037in}{0.700846in}}{\pgfqpoint{1.231037in}{0.689796in}}%
\pgfpathcurveto{\pgfqpoint{1.231037in}{0.678745in}}{\pgfqpoint{1.235428in}{0.668146in}}{\pgfqpoint{1.243241in}{0.660333in}}%
\pgfpathcurveto{\pgfqpoint{1.251055in}{0.652519in}}{\pgfqpoint{1.261654in}{0.648129in}}{\pgfqpoint{1.272704in}{0.648129in}}%
\pgfpathclose%
\pgfusepath{stroke,fill}%
\end{pgfscope}%
\begin{pgfscope}%
\pgfpathrectangle{\pgfqpoint{0.648703in}{0.548769in}}{\pgfqpoint{5.201297in}{3.102590in}}%
\pgfusepath{clip}%
\pgfsetbuttcap%
\pgfsetroundjoin%
\definecolor{currentfill}{rgb}{1.000000,0.498039,0.054902}%
\pgfsetfillcolor{currentfill}%
\pgfsetlinewidth{1.003750pt}%
\definecolor{currentstroke}{rgb}{1.000000,0.498039,0.054902}%
\pgfsetstrokecolor{currentstroke}%
\pgfsetdash{}{0pt}%
\pgfpathmoveto{\pgfqpoint{2.047860in}{3.206486in}}%
\pgfpathcurveto{\pgfqpoint{2.058910in}{3.206486in}}{\pgfqpoint{2.069509in}{3.210877in}}{\pgfqpoint{2.077323in}{3.218690in}}%
\pgfpathcurveto{\pgfqpoint{2.085136in}{3.226504in}}{\pgfqpoint{2.089527in}{3.237103in}}{\pgfqpoint{2.089527in}{3.248153in}}%
\pgfpathcurveto{\pgfqpoint{2.089527in}{3.259203in}}{\pgfqpoint{2.085136in}{3.269802in}}{\pgfqpoint{2.077323in}{3.277616in}}%
\pgfpathcurveto{\pgfqpoint{2.069509in}{3.285429in}}{\pgfqpoint{2.058910in}{3.289820in}}{\pgfqpoint{2.047860in}{3.289820in}}%
\pgfpathcurveto{\pgfqpoint{2.036810in}{3.289820in}}{\pgfqpoint{2.026211in}{3.285429in}}{\pgfqpoint{2.018397in}{3.277616in}}%
\pgfpathcurveto{\pgfqpoint{2.010584in}{3.269802in}}{\pgfqpoint{2.006193in}{3.259203in}}{\pgfqpoint{2.006193in}{3.248153in}}%
\pgfpathcurveto{\pgfqpoint{2.006193in}{3.237103in}}{\pgfqpoint{2.010584in}{3.226504in}}{\pgfqpoint{2.018397in}{3.218690in}}%
\pgfpathcurveto{\pgfqpoint{2.026211in}{3.210877in}}{\pgfqpoint{2.036810in}{3.206486in}}{\pgfqpoint{2.047860in}{3.206486in}}%
\pgfpathclose%
\pgfusepath{stroke,fill}%
\end{pgfscope}%
\begin{pgfscope}%
\pgfpathrectangle{\pgfqpoint{0.648703in}{0.548769in}}{\pgfqpoint{5.201297in}{3.102590in}}%
\pgfusepath{clip}%
\pgfsetbuttcap%
\pgfsetroundjoin%
\definecolor{currentfill}{rgb}{1.000000,0.498039,0.054902}%
\pgfsetfillcolor{currentfill}%
\pgfsetlinewidth{1.003750pt}%
\definecolor{currentstroke}{rgb}{1.000000,0.498039,0.054902}%
\pgfsetstrokecolor{currentstroke}%
\pgfsetdash{}{0pt}%
\pgfpathmoveto{\pgfqpoint{2.280407in}{3.185343in}}%
\pgfpathcurveto{\pgfqpoint{2.291457in}{3.185343in}}{\pgfqpoint{2.302056in}{3.189733in}}{\pgfqpoint{2.309870in}{3.197547in}}%
\pgfpathcurveto{\pgfqpoint{2.317683in}{3.205360in}}{\pgfqpoint{2.322073in}{3.215959in}}{\pgfqpoint{2.322073in}{3.227010in}}%
\pgfpathcurveto{\pgfqpoint{2.322073in}{3.238060in}}{\pgfqpoint{2.317683in}{3.248659in}}{\pgfqpoint{2.309870in}{3.256472in}}%
\pgfpathcurveto{\pgfqpoint{2.302056in}{3.264286in}}{\pgfqpoint{2.291457in}{3.268676in}}{\pgfqpoint{2.280407in}{3.268676in}}%
\pgfpathcurveto{\pgfqpoint{2.269357in}{3.268676in}}{\pgfqpoint{2.258758in}{3.264286in}}{\pgfqpoint{2.250944in}{3.256472in}}%
\pgfpathcurveto{\pgfqpoint{2.243130in}{3.248659in}}{\pgfqpoint{2.238740in}{3.238060in}}{\pgfqpoint{2.238740in}{3.227010in}}%
\pgfpathcurveto{\pgfqpoint{2.238740in}{3.215959in}}{\pgfqpoint{2.243130in}{3.205360in}}{\pgfqpoint{2.250944in}{3.197547in}}%
\pgfpathcurveto{\pgfqpoint{2.258758in}{3.189733in}}{\pgfqpoint{2.269357in}{3.185343in}}{\pgfqpoint{2.280407in}{3.185343in}}%
\pgfpathclose%
\pgfusepath{stroke,fill}%
\end{pgfscope}%
\begin{pgfscope}%
\pgfpathrectangle{\pgfqpoint{0.648703in}{0.548769in}}{\pgfqpoint{5.201297in}{3.102590in}}%
\pgfusepath{clip}%
\pgfsetbuttcap%
\pgfsetroundjoin%
\definecolor{currentfill}{rgb}{1.000000,0.498039,0.054902}%
\pgfsetfillcolor{currentfill}%
\pgfsetlinewidth{1.003750pt}%
\definecolor{currentstroke}{rgb}{1.000000,0.498039,0.054902}%
\pgfsetstrokecolor{currentstroke}%
\pgfsetdash{}{0pt}%
\pgfpathmoveto{\pgfqpoint{2.125376in}{3.193800in}}%
\pgfpathcurveto{\pgfqpoint{2.136426in}{3.193800in}}{\pgfqpoint{2.147025in}{3.198191in}}{\pgfqpoint{2.154838in}{3.206004in}}%
\pgfpathcurveto{\pgfqpoint{2.162652in}{3.213818in}}{\pgfqpoint{2.167042in}{3.224417in}}{\pgfqpoint{2.167042in}{3.235467in}}%
\pgfpathcurveto{\pgfqpoint{2.167042in}{3.246517in}}{\pgfqpoint{2.162652in}{3.257116in}}{\pgfqpoint{2.154838in}{3.264930in}}%
\pgfpathcurveto{\pgfqpoint{2.147025in}{3.272743in}}{\pgfqpoint{2.136426in}{3.277134in}}{\pgfqpoint{2.125376in}{3.277134in}}%
\pgfpathcurveto{\pgfqpoint{2.114325in}{3.277134in}}{\pgfqpoint{2.103726in}{3.272743in}}{\pgfqpoint{2.095913in}{3.264930in}}%
\pgfpathcurveto{\pgfqpoint{2.088099in}{3.257116in}}{\pgfqpoint{2.083709in}{3.246517in}}{\pgfqpoint{2.083709in}{3.235467in}}%
\pgfpathcurveto{\pgfqpoint{2.083709in}{3.224417in}}{\pgfqpoint{2.088099in}{3.213818in}}{\pgfqpoint{2.095913in}{3.206004in}}%
\pgfpathcurveto{\pgfqpoint{2.103726in}{3.198191in}}{\pgfqpoint{2.114325in}{3.193800in}}{\pgfqpoint{2.125376in}{3.193800in}}%
\pgfpathclose%
\pgfusepath{stroke,fill}%
\end{pgfscope}%
\begin{pgfscope}%
\pgfpathrectangle{\pgfqpoint{0.648703in}{0.548769in}}{\pgfqpoint{5.201297in}{3.102590in}}%
\pgfusepath{clip}%
\pgfsetbuttcap%
\pgfsetroundjoin%
\definecolor{currentfill}{rgb}{1.000000,0.498039,0.054902}%
\pgfsetfillcolor{currentfill}%
\pgfsetlinewidth{1.003750pt}%
\definecolor{currentstroke}{rgb}{1.000000,0.498039,0.054902}%
\pgfsetstrokecolor{currentstroke}%
\pgfsetdash{}{0pt}%
\pgfpathmoveto{\pgfqpoint{1.427735in}{3.198029in}}%
\pgfpathcurveto{\pgfqpoint{1.438785in}{3.198029in}}{\pgfqpoint{1.449384in}{3.202419in}}{\pgfqpoint{1.457198in}{3.210233in}}%
\pgfpathcurveto{\pgfqpoint{1.465012in}{3.218046in}}{\pgfqpoint{1.469402in}{3.228646in}}{\pgfqpoint{1.469402in}{3.239696in}}%
\pgfpathcurveto{\pgfqpoint{1.469402in}{3.250746in}}{\pgfqpoint{1.465012in}{3.261345in}}{\pgfqpoint{1.457198in}{3.269158in}}%
\pgfpathcurveto{\pgfqpoint{1.449384in}{3.276972in}}{\pgfqpoint{1.438785in}{3.281362in}}{\pgfqpoint{1.427735in}{3.281362in}}%
\pgfpathcurveto{\pgfqpoint{1.416685in}{3.281362in}}{\pgfqpoint{1.406086in}{3.276972in}}{\pgfqpoint{1.398272in}{3.269158in}}%
\pgfpathcurveto{\pgfqpoint{1.390459in}{3.261345in}}{\pgfqpoint{1.386069in}{3.250746in}}{\pgfqpoint{1.386069in}{3.239696in}}%
\pgfpathcurveto{\pgfqpoint{1.386069in}{3.228646in}}{\pgfqpoint{1.390459in}{3.218046in}}{\pgfqpoint{1.398272in}{3.210233in}}%
\pgfpathcurveto{\pgfqpoint{1.406086in}{3.202419in}}{\pgfqpoint{1.416685in}{3.198029in}}{\pgfqpoint{1.427735in}{3.198029in}}%
\pgfpathclose%
\pgfusepath{stroke,fill}%
\end{pgfscope}%
\begin{pgfscope}%
\pgfpathrectangle{\pgfqpoint{0.648703in}{0.548769in}}{\pgfqpoint{5.201297in}{3.102590in}}%
\pgfusepath{clip}%
\pgfsetbuttcap%
\pgfsetroundjoin%
\definecolor{currentfill}{rgb}{1.000000,0.498039,0.054902}%
\pgfsetfillcolor{currentfill}%
\pgfsetlinewidth{1.003750pt}%
\definecolor{currentstroke}{rgb}{1.000000,0.498039,0.054902}%
\pgfsetstrokecolor{currentstroke}%
\pgfsetdash{}{0pt}%
\pgfpathmoveto{\pgfqpoint{2.978047in}{3.202258in}}%
\pgfpathcurveto{\pgfqpoint{2.989097in}{3.202258in}}{\pgfqpoint{2.999696in}{3.206648in}}{\pgfqpoint{3.007510in}{3.214462in}}%
\pgfpathcurveto{\pgfqpoint{3.015324in}{3.222275in}}{\pgfqpoint{3.019714in}{3.232874in}}{\pgfqpoint{3.019714in}{3.243924in}}%
\pgfpathcurveto{\pgfqpoint{3.019714in}{3.254974in}}{\pgfqpoint{3.015324in}{3.265573in}}{\pgfqpoint{3.007510in}{3.273387in}}%
\pgfpathcurveto{\pgfqpoint{2.999696in}{3.281201in}}{\pgfqpoint{2.989097in}{3.285591in}}{\pgfqpoint{2.978047in}{3.285591in}}%
\pgfpathcurveto{\pgfqpoint{2.966997in}{3.285591in}}{\pgfqpoint{2.956398in}{3.281201in}}{\pgfqpoint{2.948584in}{3.273387in}}%
\pgfpathcurveto{\pgfqpoint{2.940771in}{3.265573in}}{\pgfqpoint{2.936380in}{3.254974in}}{\pgfqpoint{2.936380in}{3.243924in}}%
\pgfpathcurveto{\pgfqpoint{2.936380in}{3.232874in}}{\pgfqpoint{2.940771in}{3.222275in}}{\pgfqpoint{2.948584in}{3.214462in}}%
\pgfpathcurveto{\pgfqpoint{2.956398in}{3.206648in}}{\pgfqpoint{2.966997in}{3.202258in}}{\pgfqpoint{2.978047in}{3.202258in}}%
\pgfpathclose%
\pgfusepath{stroke,fill}%
\end{pgfscope}%
\begin{pgfscope}%
\pgfpathrectangle{\pgfqpoint{0.648703in}{0.548769in}}{\pgfqpoint{5.201297in}{3.102590in}}%
\pgfusepath{clip}%
\pgfsetbuttcap%
\pgfsetroundjoin%
\definecolor{currentfill}{rgb}{0.121569,0.466667,0.705882}%
\pgfsetfillcolor{currentfill}%
\pgfsetlinewidth{1.003750pt}%
\definecolor{currentstroke}{rgb}{0.121569,0.466667,0.705882}%
\pgfsetstrokecolor{currentstroke}%
\pgfsetdash{}{0pt}%
\pgfpathmoveto{\pgfqpoint{3.520656in}{3.181114in}}%
\pgfpathcurveto{\pgfqpoint{3.531706in}{3.181114in}}{\pgfqpoint{3.542305in}{3.185504in}}{\pgfqpoint{3.550119in}{3.193318in}}%
\pgfpathcurveto{\pgfqpoint{3.557933in}{3.201132in}}{\pgfqpoint{3.562323in}{3.211731in}}{\pgfqpoint{3.562323in}{3.222781in}}%
\pgfpathcurveto{\pgfqpoint{3.562323in}{3.233831in}}{\pgfqpoint{3.557933in}{3.244430in}}{\pgfqpoint{3.550119in}{3.252244in}}%
\pgfpathcurveto{\pgfqpoint{3.542305in}{3.260057in}}{\pgfqpoint{3.531706in}{3.264448in}}{\pgfqpoint{3.520656in}{3.264448in}}%
\pgfpathcurveto{\pgfqpoint{3.509606in}{3.264448in}}{\pgfqpoint{3.499007in}{3.260057in}}{\pgfqpoint{3.491194in}{3.252244in}}%
\pgfpathcurveto{\pgfqpoint{3.483380in}{3.244430in}}{\pgfqpoint{3.478990in}{3.233831in}}{\pgfqpoint{3.478990in}{3.222781in}}%
\pgfpathcurveto{\pgfqpoint{3.478990in}{3.211731in}}{\pgfqpoint{3.483380in}{3.201132in}}{\pgfqpoint{3.491194in}{3.193318in}}%
\pgfpathcurveto{\pgfqpoint{3.499007in}{3.185504in}}{\pgfqpoint{3.509606in}{3.181114in}}{\pgfqpoint{3.520656in}{3.181114in}}%
\pgfpathclose%
\pgfusepath{stroke,fill}%
\end{pgfscope}%
\begin{pgfscope}%
\pgfpathrectangle{\pgfqpoint{0.648703in}{0.548769in}}{\pgfqpoint{5.201297in}{3.102590in}}%
\pgfusepath{clip}%
\pgfsetbuttcap%
\pgfsetroundjoin%
\definecolor{currentfill}{rgb}{1.000000,0.498039,0.054902}%
\pgfsetfillcolor{currentfill}%
\pgfsetlinewidth{1.003750pt}%
\definecolor{currentstroke}{rgb}{1.000000,0.498039,0.054902}%
\pgfsetstrokecolor{currentstroke}%
\pgfsetdash{}{0pt}%
\pgfpathmoveto{\pgfqpoint{1.892829in}{3.244545in}}%
\pgfpathcurveto{\pgfqpoint{1.903879in}{3.244545in}}{\pgfqpoint{1.914478in}{3.248935in}}{\pgfqpoint{1.922292in}{3.256748in}}%
\pgfpathcurveto{\pgfqpoint{1.930105in}{3.264562in}}{\pgfqpoint{1.934495in}{3.275161in}}{\pgfqpoint{1.934495in}{3.286211in}}%
\pgfpathcurveto{\pgfqpoint{1.934495in}{3.297261in}}{\pgfqpoint{1.930105in}{3.307860in}}{\pgfqpoint{1.922292in}{3.315674in}}%
\pgfpathcurveto{\pgfqpoint{1.914478in}{3.323488in}}{\pgfqpoint{1.903879in}{3.327878in}}{\pgfqpoint{1.892829in}{3.327878in}}%
\pgfpathcurveto{\pgfqpoint{1.881779in}{3.327878in}}{\pgfqpoint{1.871180in}{3.323488in}}{\pgfqpoint{1.863366in}{3.315674in}}%
\pgfpathcurveto{\pgfqpoint{1.855552in}{3.307860in}}{\pgfqpoint{1.851162in}{3.297261in}}{\pgfqpoint{1.851162in}{3.286211in}}%
\pgfpathcurveto{\pgfqpoint{1.851162in}{3.275161in}}{\pgfqpoint{1.855552in}{3.264562in}}{\pgfqpoint{1.863366in}{3.256748in}}%
\pgfpathcurveto{\pgfqpoint{1.871180in}{3.248935in}}{\pgfqpoint{1.881779in}{3.244545in}}{\pgfqpoint{1.892829in}{3.244545in}}%
\pgfpathclose%
\pgfusepath{stroke,fill}%
\end{pgfscope}%
\begin{pgfscope}%
\pgfpathrectangle{\pgfqpoint{0.648703in}{0.548769in}}{\pgfqpoint{5.201297in}{3.102590in}}%
\pgfusepath{clip}%
\pgfsetbuttcap%
\pgfsetroundjoin%
\definecolor{currentfill}{rgb}{1.000000,0.498039,0.054902}%
\pgfsetfillcolor{currentfill}%
\pgfsetlinewidth{1.003750pt}%
\definecolor{currentstroke}{rgb}{1.000000,0.498039,0.054902}%
\pgfsetstrokecolor{currentstroke}%
\pgfsetdash{}{0pt}%
\pgfpathmoveto{\pgfqpoint{2.435438in}{3.189572in}}%
\pgfpathcurveto{\pgfqpoint{2.446488in}{3.189572in}}{\pgfqpoint{2.457087in}{3.193962in}}{\pgfqpoint{2.464901in}{3.201775in}}%
\pgfpathcurveto{\pgfqpoint{2.472714in}{3.209589in}}{\pgfqpoint{2.477105in}{3.220188in}}{\pgfqpoint{2.477105in}{3.231238in}}%
\pgfpathcurveto{\pgfqpoint{2.477105in}{3.242288in}}{\pgfqpoint{2.472714in}{3.252887in}}{\pgfqpoint{2.464901in}{3.260701in}}%
\pgfpathcurveto{\pgfqpoint{2.457087in}{3.268515in}}{\pgfqpoint{2.446488in}{3.272905in}}{\pgfqpoint{2.435438in}{3.272905in}}%
\pgfpathcurveto{\pgfqpoint{2.424388in}{3.272905in}}{\pgfqpoint{2.413789in}{3.268515in}}{\pgfqpoint{2.405975in}{3.260701in}}%
\pgfpathcurveto{\pgfqpoint{2.398162in}{3.252887in}}{\pgfqpoint{2.393771in}{3.242288in}}{\pgfqpoint{2.393771in}{3.231238in}}%
\pgfpathcurveto{\pgfqpoint{2.393771in}{3.220188in}}{\pgfqpoint{2.398162in}{3.209589in}}{\pgfqpoint{2.405975in}{3.201775in}}%
\pgfpathcurveto{\pgfqpoint{2.413789in}{3.193962in}}{\pgfqpoint{2.424388in}{3.189572in}}{\pgfqpoint{2.435438in}{3.189572in}}%
\pgfpathclose%
\pgfusepath{stroke,fill}%
\end{pgfscope}%
\begin{pgfscope}%
\pgfpathrectangle{\pgfqpoint{0.648703in}{0.548769in}}{\pgfqpoint{5.201297in}{3.102590in}}%
\pgfusepath{clip}%
\pgfsetbuttcap%
\pgfsetroundjoin%
\definecolor{currentfill}{rgb}{0.121569,0.466667,0.705882}%
\pgfsetfillcolor{currentfill}%
\pgfsetlinewidth{1.003750pt}%
\definecolor{currentstroke}{rgb}{0.121569,0.466667,0.705882}%
\pgfsetstrokecolor{currentstroke}%
\pgfsetdash{}{0pt}%
\pgfpathmoveto{\pgfqpoint{2.047860in}{0.656586in}}%
\pgfpathcurveto{\pgfqpoint{2.058910in}{0.656586in}}{\pgfqpoint{2.069509in}{0.660977in}}{\pgfqpoint{2.077323in}{0.668790in}}%
\pgfpathcurveto{\pgfqpoint{2.085136in}{0.676604in}}{\pgfqpoint{2.089527in}{0.687203in}}{\pgfqpoint{2.089527in}{0.698253in}}%
\pgfpathcurveto{\pgfqpoint{2.089527in}{0.709303in}}{\pgfqpoint{2.085136in}{0.719902in}}{\pgfqpoint{2.077323in}{0.727716in}}%
\pgfpathcurveto{\pgfqpoint{2.069509in}{0.735529in}}{\pgfqpoint{2.058910in}{0.739920in}}{\pgfqpoint{2.047860in}{0.739920in}}%
\pgfpathcurveto{\pgfqpoint{2.036810in}{0.739920in}}{\pgfqpoint{2.026211in}{0.735529in}}{\pgfqpoint{2.018397in}{0.727716in}}%
\pgfpathcurveto{\pgfqpoint{2.010584in}{0.719902in}}{\pgfqpoint{2.006193in}{0.709303in}}{\pgfqpoint{2.006193in}{0.698253in}}%
\pgfpathcurveto{\pgfqpoint{2.006193in}{0.687203in}}{\pgfqpoint{2.010584in}{0.676604in}}{\pgfqpoint{2.018397in}{0.668790in}}%
\pgfpathcurveto{\pgfqpoint{2.026211in}{0.660977in}}{\pgfqpoint{2.036810in}{0.656586in}}{\pgfqpoint{2.047860in}{0.656586in}}%
\pgfpathclose%
\pgfusepath{stroke,fill}%
\end{pgfscope}%
\begin{pgfscope}%
\pgfpathrectangle{\pgfqpoint{0.648703in}{0.548769in}}{\pgfqpoint{5.201297in}{3.102590in}}%
\pgfusepath{clip}%
\pgfsetbuttcap%
\pgfsetroundjoin%
\definecolor{currentfill}{rgb}{1.000000,0.498039,0.054902}%
\pgfsetfillcolor{currentfill}%
\pgfsetlinewidth{1.003750pt}%
\definecolor{currentstroke}{rgb}{1.000000,0.498039,0.054902}%
\pgfsetstrokecolor{currentstroke}%
\pgfsetdash{}{0pt}%
\pgfpathmoveto{\pgfqpoint{1.427735in}{3.210715in}}%
\pgfpathcurveto{\pgfqpoint{1.438785in}{3.210715in}}{\pgfqpoint{1.449384in}{3.215105in}}{\pgfqpoint{1.457198in}{3.222919in}}%
\pgfpathcurveto{\pgfqpoint{1.465012in}{3.230733in}}{\pgfqpoint{1.469402in}{3.241332in}}{\pgfqpoint{1.469402in}{3.252382in}}%
\pgfpathcurveto{\pgfqpoint{1.469402in}{3.263432in}}{\pgfqpoint{1.465012in}{3.274031in}}{\pgfqpoint{1.457198in}{3.281844in}}%
\pgfpathcurveto{\pgfqpoint{1.449384in}{3.289658in}}{\pgfqpoint{1.438785in}{3.294048in}}{\pgfqpoint{1.427735in}{3.294048in}}%
\pgfpathcurveto{\pgfqpoint{1.416685in}{3.294048in}}{\pgfqpoint{1.406086in}{3.289658in}}{\pgfqpoint{1.398272in}{3.281844in}}%
\pgfpathcurveto{\pgfqpoint{1.390459in}{3.274031in}}{\pgfqpoint{1.386069in}{3.263432in}}{\pgfqpoint{1.386069in}{3.252382in}}%
\pgfpathcurveto{\pgfqpoint{1.386069in}{3.241332in}}{\pgfqpoint{1.390459in}{3.230733in}}{\pgfqpoint{1.398272in}{3.222919in}}%
\pgfpathcurveto{\pgfqpoint{1.406086in}{3.215105in}}{\pgfqpoint{1.416685in}{3.210715in}}{\pgfqpoint{1.427735in}{3.210715in}}%
\pgfpathclose%
\pgfusepath{stroke,fill}%
\end{pgfscope}%
\begin{pgfscope}%
\pgfpathrectangle{\pgfqpoint{0.648703in}{0.548769in}}{\pgfqpoint{5.201297in}{3.102590in}}%
\pgfusepath{clip}%
\pgfsetbuttcap%
\pgfsetroundjoin%
\definecolor{currentfill}{rgb}{1.000000,0.498039,0.054902}%
\pgfsetfillcolor{currentfill}%
\pgfsetlinewidth{1.003750pt}%
\definecolor{currentstroke}{rgb}{1.000000,0.498039,0.054902}%
\pgfsetstrokecolor{currentstroke}%
\pgfsetdash{}{0pt}%
\pgfpathmoveto{\pgfqpoint{2.590469in}{3.214944in}}%
\pgfpathcurveto{\pgfqpoint{2.601519in}{3.214944in}}{\pgfqpoint{2.612118in}{3.219334in}}{\pgfqpoint{2.619932in}{3.227148in}}%
\pgfpathcurveto{\pgfqpoint{2.627746in}{3.234961in}}{\pgfqpoint{2.632136in}{3.245560in}}{\pgfqpoint{2.632136in}{3.256610in}}%
\pgfpathcurveto{\pgfqpoint{2.632136in}{3.267661in}}{\pgfqpoint{2.627746in}{3.278260in}}{\pgfqpoint{2.619932in}{3.286073in}}%
\pgfpathcurveto{\pgfqpoint{2.612118in}{3.293887in}}{\pgfqpoint{2.601519in}{3.298277in}}{\pgfqpoint{2.590469in}{3.298277in}}%
\pgfpathcurveto{\pgfqpoint{2.579419in}{3.298277in}}{\pgfqpoint{2.568820in}{3.293887in}}{\pgfqpoint{2.561006in}{3.286073in}}%
\pgfpathcurveto{\pgfqpoint{2.553193in}{3.278260in}}{\pgfqpoint{2.548802in}{3.267661in}}{\pgfqpoint{2.548802in}{3.256610in}}%
\pgfpathcurveto{\pgfqpoint{2.548802in}{3.245560in}}{\pgfqpoint{2.553193in}{3.234961in}}{\pgfqpoint{2.561006in}{3.227148in}}%
\pgfpathcurveto{\pgfqpoint{2.568820in}{3.219334in}}{\pgfqpoint{2.579419in}{3.214944in}}{\pgfqpoint{2.590469in}{3.214944in}}%
\pgfpathclose%
\pgfusepath{stroke,fill}%
\end{pgfscope}%
\begin{pgfscope}%
\pgfpathrectangle{\pgfqpoint{0.648703in}{0.548769in}}{\pgfqpoint{5.201297in}{3.102590in}}%
\pgfusepath{clip}%
\pgfsetbuttcap%
\pgfsetroundjoin%
\definecolor{currentfill}{rgb}{1.000000,0.498039,0.054902}%
\pgfsetfillcolor{currentfill}%
\pgfsetlinewidth{1.003750pt}%
\definecolor{currentstroke}{rgb}{1.000000,0.498039,0.054902}%
\pgfsetstrokecolor{currentstroke}%
\pgfsetdash{}{0pt}%
\pgfpathmoveto{\pgfqpoint{1.970344in}{3.202258in}}%
\pgfpathcurveto{\pgfqpoint{1.981394in}{3.202258in}}{\pgfqpoint{1.991994in}{3.206648in}}{\pgfqpoint{1.999807in}{3.214462in}}%
\pgfpathcurveto{\pgfqpoint{2.007621in}{3.222275in}}{\pgfqpoint{2.012011in}{3.232874in}}{\pgfqpoint{2.012011in}{3.243924in}}%
\pgfpathcurveto{\pgfqpoint{2.012011in}{3.254974in}}{\pgfqpoint{2.007621in}{3.265573in}}{\pgfqpoint{1.999807in}{3.273387in}}%
\pgfpathcurveto{\pgfqpoint{1.991994in}{3.281201in}}{\pgfqpoint{1.981394in}{3.285591in}}{\pgfqpoint{1.970344in}{3.285591in}}%
\pgfpathcurveto{\pgfqpoint{1.959294in}{3.285591in}}{\pgfqpoint{1.948695in}{3.281201in}}{\pgfqpoint{1.940882in}{3.273387in}}%
\pgfpathcurveto{\pgfqpoint{1.933068in}{3.265573in}}{\pgfqpoint{1.928678in}{3.254974in}}{\pgfqpoint{1.928678in}{3.243924in}}%
\pgfpathcurveto{\pgfqpoint{1.928678in}{3.232874in}}{\pgfqpoint{1.933068in}{3.222275in}}{\pgfqpoint{1.940882in}{3.214462in}}%
\pgfpathcurveto{\pgfqpoint{1.948695in}{3.206648in}}{\pgfqpoint{1.959294in}{3.202258in}}{\pgfqpoint{1.970344in}{3.202258in}}%
\pgfpathclose%
\pgfusepath{stroke,fill}%
\end{pgfscope}%
\begin{pgfscope}%
\pgfpathrectangle{\pgfqpoint{0.648703in}{0.548769in}}{\pgfqpoint{5.201297in}{3.102590in}}%
\pgfusepath{clip}%
\pgfsetbuttcap%
\pgfsetroundjoin%
\definecolor{currentfill}{rgb}{1.000000,0.498039,0.054902}%
\pgfsetfillcolor{currentfill}%
\pgfsetlinewidth{1.003750pt}%
\definecolor{currentstroke}{rgb}{1.000000,0.498039,0.054902}%
\pgfsetstrokecolor{currentstroke}%
\pgfsetdash{}{0pt}%
\pgfpathmoveto{\pgfqpoint{2.590469in}{3.189572in}}%
\pgfpathcurveto{\pgfqpoint{2.601519in}{3.189572in}}{\pgfqpoint{2.612118in}{3.193962in}}{\pgfqpoint{2.619932in}{3.201775in}}%
\pgfpathcurveto{\pgfqpoint{2.627746in}{3.209589in}}{\pgfqpoint{2.632136in}{3.220188in}}{\pgfqpoint{2.632136in}{3.231238in}}%
\pgfpathcurveto{\pgfqpoint{2.632136in}{3.242288in}}{\pgfqpoint{2.627746in}{3.252887in}}{\pgfqpoint{2.619932in}{3.260701in}}%
\pgfpathcurveto{\pgfqpoint{2.612118in}{3.268515in}}{\pgfqpoint{2.601519in}{3.272905in}}{\pgfqpoint{2.590469in}{3.272905in}}%
\pgfpathcurveto{\pgfqpoint{2.579419in}{3.272905in}}{\pgfqpoint{2.568820in}{3.268515in}}{\pgfqpoint{2.561006in}{3.260701in}}%
\pgfpathcurveto{\pgfqpoint{2.553193in}{3.252887in}}{\pgfqpoint{2.548802in}{3.242288in}}{\pgfqpoint{2.548802in}{3.231238in}}%
\pgfpathcurveto{\pgfqpoint{2.548802in}{3.220188in}}{\pgfqpoint{2.553193in}{3.209589in}}{\pgfqpoint{2.561006in}{3.201775in}}%
\pgfpathcurveto{\pgfqpoint{2.568820in}{3.193962in}}{\pgfqpoint{2.579419in}{3.189572in}}{\pgfqpoint{2.590469in}{3.189572in}}%
\pgfpathclose%
\pgfusepath{stroke,fill}%
\end{pgfscope}%
\begin{pgfscope}%
\pgfpathrectangle{\pgfqpoint{0.648703in}{0.548769in}}{\pgfqpoint{5.201297in}{3.102590in}}%
\pgfusepath{clip}%
\pgfsetbuttcap%
\pgfsetroundjoin%
\definecolor{currentfill}{rgb}{0.839216,0.152941,0.156863}%
\pgfsetfillcolor{currentfill}%
\pgfsetlinewidth{1.003750pt}%
\definecolor{currentstroke}{rgb}{0.839216,0.152941,0.156863}%
\pgfsetstrokecolor{currentstroke}%
\pgfsetdash{}{0pt}%
\pgfpathmoveto{\pgfqpoint{2.978047in}{3.181114in}}%
\pgfpathcurveto{\pgfqpoint{2.989097in}{3.181114in}}{\pgfqpoint{2.999696in}{3.185504in}}{\pgfqpoint{3.007510in}{3.193318in}}%
\pgfpathcurveto{\pgfqpoint{3.015324in}{3.201132in}}{\pgfqpoint{3.019714in}{3.211731in}}{\pgfqpoint{3.019714in}{3.222781in}}%
\pgfpathcurveto{\pgfqpoint{3.019714in}{3.233831in}}{\pgfqpoint{3.015324in}{3.244430in}}{\pgfqpoint{3.007510in}{3.252244in}}%
\pgfpathcurveto{\pgfqpoint{2.999696in}{3.260057in}}{\pgfqpoint{2.989097in}{3.264448in}}{\pgfqpoint{2.978047in}{3.264448in}}%
\pgfpathcurveto{\pgfqpoint{2.966997in}{3.264448in}}{\pgfqpoint{2.956398in}{3.260057in}}{\pgfqpoint{2.948584in}{3.252244in}}%
\pgfpathcurveto{\pgfqpoint{2.940771in}{3.244430in}}{\pgfqpoint{2.936380in}{3.233831in}}{\pgfqpoint{2.936380in}{3.222781in}}%
\pgfpathcurveto{\pgfqpoint{2.936380in}{3.211731in}}{\pgfqpoint{2.940771in}{3.201132in}}{\pgfqpoint{2.948584in}{3.193318in}}%
\pgfpathcurveto{\pgfqpoint{2.956398in}{3.185504in}}{\pgfqpoint{2.966997in}{3.181114in}}{\pgfqpoint{2.978047in}{3.181114in}}%
\pgfpathclose%
\pgfusepath{stroke,fill}%
\end{pgfscope}%
\begin{pgfscope}%
\pgfpathrectangle{\pgfqpoint{0.648703in}{0.548769in}}{\pgfqpoint{5.201297in}{3.102590in}}%
\pgfusepath{clip}%
\pgfsetbuttcap%
\pgfsetroundjoin%
\definecolor{currentfill}{rgb}{1.000000,0.498039,0.054902}%
\pgfsetfillcolor{currentfill}%
\pgfsetlinewidth{1.003750pt}%
\definecolor{currentstroke}{rgb}{1.000000,0.498039,0.054902}%
\pgfsetstrokecolor{currentstroke}%
\pgfsetdash{}{0pt}%
\pgfpathmoveto{\pgfqpoint{1.815313in}{3.244545in}}%
\pgfpathcurveto{\pgfqpoint{1.826363in}{3.244545in}}{\pgfqpoint{1.836962in}{3.248935in}}{\pgfqpoint{1.844776in}{3.256748in}}%
\pgfpathcurveto{\pgfqpoint{1.852590in}{3.264562in}}{\pgfqpoint{1.856980in}{3.275161in}}{\pgfqpoint{1.856980in}{3.286211in}}%
\pgfpathcurveto{\pgfqpoint{1.856980in}{3.297261in}}{\pgfqpoint{1.852590in}{3.307860in}}{\pgfqpoint{1.844776in}{3.315674in}}%
\pgfpathcurveto{\pgfqpoint{1.836962in}{3.323488in}}{\pgfqpoint{1.826363in}{3.327878in}}{\pgfqpoint{1.815313in}{3.327878in}}%
\pgfpathcurveto{\pgfqpoint{1.804263in}{3.327878in}}{\pgfqpoint{1.793664in}{3.323488in}}{\pgfqpoint{1.785850in}{3.315674in}}%
\pgfpathcurveto{\pgfqpoint{1.778037in}{3.307860in}}{\pgfqpoint{1.773646in}{3.297261in}}{\pgfqpoint{1.773646in}{3.286211in}}%
\pgfpathcurveto{\pgfqpoint{1.773646in}{3.275161in}}{\pgfqpoint{1.778037in}{3.264562in}}{\pgfqpoint{1.785850in}{3.256748in}}%
\pgfpathcurveto{\pgfqpoint{1.793664in}{3.248935in}}{\pgfqpoint{1.804263in}{3.244545in}}{\pgfqpoint{1.815313in}{3.244545in}}%
\pgfpathclose%
\pgfusepath{stroke,fill}%
\end{pgfscope}%
\begin{pgfscope}%
\pgfpathrectangle{\pgfqpoint{0.648703in}{0.548769in}}{\pgfqpoint{5.201297in}{3.102590in}}%
\pgfusepath{clip}%
\pgfsetbuttcap%
\pgfsetroundjoin%
\definecolor{currentfill}{rgb}{1.000000,0.498039,0.054902}%
\pgfsetfillcolor{currentfill}%
\pgfsetlinewidth{1.003750pt}%
\definecolor{currentstroke}{rgb}{1.000000,0.498039,0.054902}%
\pgfsetstrokecolor{currentstroke}%
\pgfsetdash{}{0pt}%
\pgfpathmoveto{\pgfqpoint{2.435438in}{3.202258in}}%
\pgfpathcurveto{\pgfqpoint{2.446488in}{3.202258in}}{\pgfqpoint{2.457087in}{3.206648in}}{\pgfqpoint{2.464901in}{3.214462in}}%
\pgfpathcurveto{\pgfqpoint{2.472714in}{3.222275in}}{\pgfqpoint{2.477105in}{3.232874in}}{\pgfqpoint{2.477105in}{3.243924in}}%
\pgfpathcurveto{\pgfqpoint{2.477105in}{3.254974in}}{\pgfqpoint{2.472714in}{3.265573in}}{\pgfqpoint{2.464901in}{3.273387in}}%
\pgfpathcurveto{\pgfqpoint{2.457087in}{3.281201in}}{\pgfqpoint{2.446488in}{3.285591in}}{\pgfqpoint{2.435438in}{3.285591in}}%
\pgfpathcurveto{\pgfqpoint{2.424388in}{3.285591in}}{\pgfqpoint{2.413789in}{3.281201in}}{\pgfqpoint{2.405975in}{3.273387in}}%
\pgfpathcurveto{\pgfqpoint{2.398162in}{3.265573in}}{\pgfqpoint{2.393771in}{3.254974in}}{\pgfqpoint{2.393771in}{3.243924in}}%
\pgfpathcurveto{\pgfqpoint{2.393771in}{3.232874in}}{\pgfqpoint{2.398162in}{3.222275in}}{\pgfqpoint{2.405975in}{3.214462in}}%
\pgfpathcurveto{\pgfqpoint{2.413789in}{3.206648in}}{\pgfqpoint{2.424388in}{3.202258in}}{\pgfqpoint{2.435438in}{3.202258in}}%
\pgfpathclose%
\pgfusepath{stroke,fill}%
\end{pgfscope}%
\begin{pgfscope}%
\pgfpathrectangle{\pgfqpoint{0.648703in}{0.548769in}}{\pgfqpoint{5.201297in}{3.102590in}}%
\pgfusepath{clip}%
\pgfsetbuttcap%
\pgfsetroundjoin%
\definecolor{currentfill}{rgb}{1.000000,0.498039,0.054902}%
\pgfsetfillcolor{currentfill}%
\pgfsetlinewidth{1.003750pt}%
\definecolor{currentstroke}{rgb}{1.000000,0.498039,0.054902}%
\pgfsetstrokecolor{currentstroke}%
\pgfsetdash{}{0pt}%
\pgfpathmoveto{\pgfqpoint{1.350220in}{3.185343in}}%
\pgfpathcurveto{\pgfqpoint{1.361270in}{3.185343in}}{\pgfqpoint{1.371869in}{3.189733in}}{\pgfqpoint{1.379682in}{3.197547in}}%
\pgfpathcurveto{\pgfqpoint{1.387496in}{3.205360in}}{\pgfqpoint{1.391886in}{3.215959in}}{\pgfqpoint{1.391886in}{3.227010in}}%
\pgfpathcurveto{\pgfqpoint{1.391886in}{3.238060in}}{\pgfqpoint{1.387496in}{3.248659in}}{\pgfqpoint{1.379682in}{3.256472in}}%
\pgfpathcurveto{\pgfqpoint{1.371869in}{3.264286in}}{\pgfqpoint{1.361270in}{3.268676in}}{\pgfqpoint{1.350220in}{3.268676in}}%
\pgfpathcurveto{\pgfqpoint{1.339169in}{3.268676in}}{\pgfqpoint{1.328570in}{3.264286in}}{\pgfqpoint{1.320757in}{3.256472in}}%
\pgfpathcurveto{\pgfqpoint{1.312943in}{3.248659in}}{\pgfqpoint{1.308553in}{3.238060in}}{\pgfqpoint{1.308553in}{3.227010in}}%
\pgfpathcurveto{\pgfqpoint{1.308553in}{3.215959in}}{\pgfqpoint{1.312943in}{3.205360in}}{\pgfqpoint{1.320757in}{3.197547in}}%
\pgfpathcurveto{\pgfqpoint{1.328570in}{3.189733in}}{\pgfqpoint{1.339169in}{3.185343in}}{\pgfqpoint{1.350220in}{3.185343in}}%
\pgfpathclose%
\pgfusepath{stroke,fill}%
\end{pgfscope}%
\begin{pgfscope}%
\pgfpathrectangle{\pgfqpoint{0.648703in}{0.548769in}}{\pgfqpoint{5.201297in}{3.102590in}}%
\pgfusepath{clip}%
\pgfsetbuttcap%
\pgfsetroundjoin%
\definecolor{currentfill}{rgb}{1.000000,0.498039,0.054902}%
\pgfsetfillcolor{currentfill}%
\pgfsetlinewidth{1.003750pt}%
\definecolor{currentstroke}{rgb}{1.000000,0.498039,0.054902}%
\pgfsetstrokecolor{currentstroke}%
\pgfsetdash{}{0pt}%
\pgfpathmoveto{\pgfqpoint{1.737798in}{3.312204in}}%
\pgfpathcurveto{\pgfqpoint{1.748848in}{3.312204in}}{\pgfqpoint{1.759447in}{3.316594in}}{\pgfqpoint{1.767260in}{3.324407in}}%
\pgfpathcurveto{\pgfqpoint{1.775074in}{3.332221in}}{\pgfqpoint{1.779464in}{3.342820in}}{\pgfqpoint{1.779464in}{3.353870in}}%
\pgfpathcurveto{\pgfqpoint{1.779464in}{3.364920in}}{\pgfqpoint{1.775074in}{3.375519in}}{\pgfqpoint{1.767260in}{3.383333in}}%
\pgfpathcurveto{\pgfqpoint{1.759447in}{3.391147in}}{\pgfqpoint{1.748848in}{3.395537in}}{\pgfqpoint{1.737798in}{3.395537in}}%
\pgfpathcurveto{\pgfqpoint{1.726747in}{3.395537in}}{\pgfqpoint{1.716148in}{3.391147in}}{\pgfqpoint{1.708335in}{3.383333in}}%
\pgfpathcurveto{\pgfqpoint{1.700521in}{3.375519in}}{\pgfqpoint{1.696131in}{3.364920in}}{\pgfqpoint{1.696131in}{3.353870in}}%
\pgfpathcurveto{\pgfqpoint{1.696131in}{3.342820in}}{\pgfqpoint{1.700521in}{3.332221in}}{\pgfqpoint{1.708335in}{3.324407in}}%
\pgfpathcurveto{\pgfqpoint{1.716148in}{3.316594in}}{\pgfqpoint{1.726747in}{3.312204in}}{\pgfqpoint{1.737798in}{3.312204in}}%
\pgfpathclose%
\pgfusepath{stroke,fill}%
\end{pgfscope}%
\begin{pgfscope}%
\pgfpathrectangle{\pgfqpoint{0.648703in}{0.548769in}}{\pgfqpoint{5.201297in}{3.102590in}}%
\pgfusepath{clip}%
\pgfsetbuttcap%
\pgfsetroundjoin%
\definecolor{currentfill}{rgb}{1.000000,0.498039,0.054902}%
\pgfsetfillcolor{currentfill}%
\pgfsetlinewidth{1.003750pt}%
\definecolor{currentstroke}{rgb}{1.000000,0.498039,0.054902}%
\pgfsetstrokecolor{currentstroke}%
\pgfsetdash{}{0pt}%
\pgfpathmoveto{\pgfqpoint{1.660282in}{3.210715in}}%
\pgfpathcurveto{\pgfqpoint{1.671332in}{3.210715in}}{\pgfqpoint{1.681931in}{3.215105in}}{\pgfqpoint{1.689745in}{3.222919in}}%
\pgfpathcurveto{\pgfqpoint{1.697558in}{3.230733in}}{\pgfqpoint{1.701949in}{3.241332in}}{\pgfqpoint{1.701949in}{3.252382in}}%
\pgfpathcurveto{\pgfqpoint{1.701949in}{3.263432in}}{\pgfqpoint{1.697558in}{3.274031in}}{\pgfqpoint{1.689745in}{3.281844in}}%
\pgfpathcurveto{\pgfqpoint{1.681931in}{3.289658in}}{\pgfqpoint{1.671332in}{3.294048in}}{\pgfqpoint{1.660282in}{3.294048in}}%
\pgfpathcurveto{\pgfqpoint{1.649232in}{3.294048in}}{\pgfqpoint{1.638633in}{3.289658in}}{\pgfqpoint{1.630819in}{3.281844in}}%
\pgfpathcurveto{\pgfqpoint{1.623006in}{3.274031in}}{\pgfqpoint{1.618615in}{3.263432in}}{\pgfqpoint{1.618615in}{3.252382in}}%
\pgfpathcurveto{\pgfqpoint{1.618615in}{3.241332in}}{\pgfqpoint{1.623006in}{3.230733in}}{\pgfqpoint{1.630819in}{3.222919in}}%
\pgfpathcurveto{\pgfqpoint{1.638633in}{3.215105in}}{\pgfqpoint{1.649232in}{3.210715in}}{\pgfqpoint{1.660282in}{3.210715in}}%
\pgfpathclose%
\pgfusepath{stroke,fill}%
\end{pgfscope}%
\begin{pgfscope}%
\pgfpathrectangle{\pgfqpoint{0.648703in}{0.548769in}}{\pgfqpoint{5.201297in}{3.102590in}}%
\pgfusepath{clip}%
\pgfsetbuttcap%
\pgfsetroundjoin%
\definecolor{currentfill}{rgb}{1.000000,0.498039,0.054902}%
\pgfsetfillcolor{currentfill}%
\pgfsetlinewidth{1.003750pt}%
\definecolor{currentstroke}{rgb}{1.000000,0.498039,0.054902}%
\pgfsetstrokecolor{currentstroke}%
\pgfsetdash{}{0pt}%
\pgfpathmoveto{\pgfqpoint{1.737798in}{3.236087in}}%
\pgfpathcurveto{\pgfqpoint{1.748848in}{3.236087in}}{\pgfqpoint{1.759447in}{3.240477in}}{\pgfqpoint{1.767260in}{3.248291in}}%
\pgfpathcurveto{\pgfqpoint{1.775074in}{3.256105in}}{\pgfqpoint{1.779464in}{3.266704in}}{\pgfqpoint{1.779464in}{3.277754in}}%
\pgfpathcurveto{\pgfqpoint{1.779464in}{3.288804in}}{\pgfqpoint{1.775074in}{3.299403in}}{\pgfqpoint{1.767260in}{3.307217in}}%
\pgfpathcurveto{\pgfqpoint{1.759447in}{3.315030in}}{\pgfqpoint{1.748848in}{3.319421in}}{\pgfqpoint{1.737798in}{3.319421in}}%
\pgfpathcurveto{\pgfqpoint{1.726747in}{3.319421in}}{\pgfqpoint{1.716148in}{3.315030in}}{\pgfqpoint{1.708335in}{3.307217in}}%
\pgfpathcurveto{\pgfqpoint{1.700521in}{3.299403in}}{\pgfqpoint{1.696131in}{3.288804in}}{\pgfqpoint{1.696131in}{3.277754in}}%
\pgfpathcurveto{\pgfqpoint{1.696131in}{3.266704in}}{\pgfqpoint{1.700521in}{3.256105in}}{\pgfqpoint{1.708335in}{3.248291in}}%
\pgfpathcurveto{\pgfqpoint{1.716148in}{3.240477in}}{\pgfqpoint{1.726747in}{3.236087in}}{\pgfqpoint{1.737798in}{3.236087in}}%
\pgfpathclose%
\pgfusepath{stroke,fill}%
\end{pgfscope}%
\begin{pgfscope}%
\pgfpathrectangle{\pgfqpoint{0.648703in}{0.548769in}}{\pgfqpoint{5.201297in}{3.102590in}}%
\pgfusepath{clip}%
\pgfsetbuttcap%
\pgfsetroundjoin%
\definecolor{currentfill}{rgb}{0.121569,0.466667,0.705882}%
\pgfsetfillcolor{currentfill}%
\pgfsetlinewidth{1.003750pt}%
\definecolor{currentstroke}{rgb}{0.121569,0.466667,0.705882}%
\pgfsetstrokecolor{currentstroke}%
\pgfsetdash{}{0pt}%
\pgfpathmoveto{\pgfqpoint{3.365625in}{2.808990in}}%
\pgfpathcurveto{\pgfqpoint{3.376675in}{2.808990in}}{\pgfqpoint{3.387274in}{2.813380in}}{\pgfqpoint{3.395088in}{2.821193in}}%
\pgfpathcurveto{\pgfqpoint{3.402902in}{2.829007in}}{\pgfqpoint{3.407292in}{2.839606in}}{\pgfqpoint{3.407292in}{2.850656in}}%
\pgfpathcurveto{\pgfqpoint{3.407292in}{2.861706in}}{\pgfqpoint{3.402902in}{2.872305in}}{\pgfqpoint{3.395088in}{2.880119in}}%
\pgfpathcurveto{\pgfqpoint{3.387274in}{2.887933in}}{\pgfqpoint{3.376675in}{2.892323in}}{\pgfqpoint{3.365625in}{2.892323in}}%
\pgfpathcurveto{\pgfqpoint{3.354575in}{2.892323in}}{\pgfqpoint{3.343976in}{2.887933in}}{\pgfqpoint{3.336162in}{2.880119in}}%
\pgfpathcurveto{\pgfqpoint{3.328349in}{2.872305in}}{\pgfqpoint{3.323958in}{2.861706in}}{\pgfqpoint{3.323958in}{2.850656in}}%
\pgfpathcurveto{\pgfqpoint{3.323958in}{2.839606in}}{\pgfqpoint{3.328349in}{2.829007in}}{\pgfqpoint{3.336162in}{2.821193in}}%
\pgfpathcurveto{\pgfqpoint{3.343976in}{2.813380in}}{\pgfqpoint{3.354575in}{2.808990in}}{\pgfqpoint{3.365625in}{2.808990in}}%
\pgfpathclose%
\pgfusepath{stroke,fill}%
\end{pgfscope}%
\begin{pgfscope}%
\pgfpathrectangle{\pgfqpoint{0.648703in}{0.548769in}}{\pgfqpoint{5.201297in}{3.102590in}}%
\pgfusepath{clip}%
\pgfsetbuttcap%
\pgfsetroundjoin%
\definecolor{currentfill}{rgb}{0.121569,0.466667,0.705882}%
\pgfsetfillcolor{currentfill}%
\pgfsetlinewidth{1.003750pt}%
\definecolor{currentstroke}{rgb}{0.121569,0.466667,0.705882}%
\pgfsetstrokecolor{currentstroke}%
\pgfsetdash{}{0pt}%
\pgfpathmoveto{\pgfqpoint{1.272704in}{0.648129in}}%
\pgfpathcurveto{\pgfqpoint{1.283754in}{0.648129in}}{\pgfqpoint{1.294353in}{0.652519in}}{\pgfqpoint{1.302167in}{0.660333in}}%
\pgfpathcurveto{\pgfqpoint{1.309980in}{0.668146in}}{\pgfqpoint{1.314371in}{0.678745in}}{\pgfqpoint{1.314371in}{0.689796in}}%
\pgfpathcurveto{\pgfqpoint{1.314371in}{0.700846in}}{\pgfqpoint{1.309980in}{0.711445in}}{\pgfqpoint{1.302167in}{0.719258in}}%
\pgfpathcurveto{\pgfqpoint{1.294353in}{0.727072in}}{\pgfqpoint{1.283754in}{0.731462in}}{\pgfqpoint{1.272704in}{0.731462in}}%
\pgfpathcurveto{\pgfqpoint{1.261654in}{0.731462in}}{\pgfqpoint{1.251055in}{0.727072in}}{\pgfqpoint{1.243241in}{0.719258in}}%
\pgfpathcurveto{\pgfqpoint{1.235428in}{0.711445in}}{\pgfqpoint{1.231037in}{0.700846in}}{\pgfqpoint{1.231037in}{0.689796in}}%
\pgfpathcurveto{\pgfqpoint{1.231037in}{0.678745in}}{\pgfqpoint{1.235428in}{0.668146in}}{\pgfqpoint{1.243241in}{0.660333in}}%
\pgfpathcurveto{\pgfqpoint{1.251055in}{0.652519in}}{\pgfqpoint{1.261654in}{0.648129in}}{\pgfqpoint{1.272704in}{0.648129in}}%
\pgfpathclose%
\pgfusepath{stroke,fill}%
\end{pgfscope}%
\begin{pgfscope}%
\pgfpathrectangle{\pgfqpoint{0.648703in}{0.548769in}}{\pgfqpoint{5.201297in}{3.102590in}}%
\pgfusepath{clip}%
\pgfsetbuttcap%
\pgfsetroundjoin%
\definecolor{currentfill}{rgb}{0.121569,0.466667,0.705882}%
\pgfsetfillcolor{currentfill}%
\pgfsetlinewidth{1.003750pt}%
\definecolor{currentstroke}{rgb}{0.121569,0.466667,0.705882}%
\pgfsetstrokecolor{currentstroke}%
\pgfsetdash{}{0pt}%
\pgfpathmoveto{\pgfqpoint{1.427735in}{0.648129in}}%
\pgfpathcurveto{\pgfqpoint{1.438785in}{0.648129in}}{\pgfqpoint{1.449384in}{0.652519in}}{\pgfqpoint{1.457198in}{0.660333in}}%
\pgfpathcurveto{\pgfqpoint{1.465012in}{0.668146in}}{\pgfqpoint{1.469402in}{0.678745in}}{\pgfqpoint{1.469402in}{0.689796in}}%
\pgfpathcurveto{\pgfqpoint{1.469402in}{0.700846in}}{\pgfqpoint{1.465012in}{0.711445in}}{\pgfqpoint{1.457198in}{0.719258in}}%
\pgfpathcurveto{\pgfqpoint{1.449384in}{0.727072in}}{\pgfqpoint{1.438785in}{0.731462in}}{\pgfqpoint{1.427735in}{0.731462in}}%
\pgfpathcurveto{\pgfqpoint{1.416685in}{0.731462in}}{\pgfqpoint{1.406086in}{0.727072in}}{\pgfqpoint{1.398272in}{0.719258in}}%
\pgfpathcurveto{\pgfqpoint{1.390459in}{0.711445in}}{\pgfqpoint{1.386069in}{0.700846in}}{\pgfqpoint{1.386069in}{0.689796in}}%
\pgfpathcurveto{\pgfqpoint{1.386069in}{0.678745in}}{\pgfqpoint{1.390459in}{0.668146in}}{\pgfqpoint{1.398272in}{0.660333in}}%
\pgfpathcurveto{\pgfqpoint{1.406086in}{0.652519in}}{\pgfqpoint{1.416685in}{0.648129in}}{\pgfqpoint{1.427735in}{0.648129in}}%
\pgfpathclose%
\pgfusepath{stroke,fill}%
\end{pgfscope}%
\begin{pgfscope}%
\pgfpathrectangle{\pgfqpoint{0.648703in}{0.548769in}}{\pgfqpoint{5.201297in}{3.102590in}}%
\pgfusepath{clip}%
\pgfsetbuttcap%
\pgfsetroundjoin%
\definecolor{currentfill}{rgb}{1.000000,0.498039,0.054902}%
\pgfsetfillcolor{currentfill}%
\pgfsetlinewidth{1.003750pt}%
\definecolor{currentstroke}{rgb}{1.000000,0.498039,0.054902}%
\pgfsetstrokecolor{currentstroke}%
\pgfsetdash{}{0pt}%
\pgfpathmoveto{\pgfqpoint{1.505251in}{3.185343in}}%
\pgfpathcurveto{\pgfqpoint{1.516301in}{3.185343in}}{\pgfqpoint{1.526900in}{3.189733in}}{\pgfqpoint{1.534714in}{3.197547in}}%
\pgfpathcurveto{\pgfqpoint{1.542527in}{3.205360in}}{\pgfqpoint{1.546917in}{3.215959in}}{\pgfqpoint{1.546917in}{3.227010in}}%
\pgfpathcurveto{\pgfqpoint{1.546917in}{3.238060in}}{\pgfqpoint{1.542527in}{3.248659in}}{\pgfqpoint{1.534714in}{3.256472in}}%
\pgfpathcurveto{\pgfqpoint{1.526900in}{3.264286in}}{\pgfqpoint{1.516301in}{3.268676in}}{\pgfqpoint{1.505251in}{3.268676in}}%
\pgfpathcurveto{\pgfqpoint{1.494201in}{3.268676in}}{\pgfqpoint{1.483602in}{3.264286in}}{\pgfqpoint{1.475788in}{3.256472in}}%
\pgfpathcurveto{\pgfqpoint{1.467974in}{3.248659in}}{\pgfqpoint{1.463584in}{3.238060in}}{\pgfqpoint{1.463584in}{3.227010in}}%
\pgfpathcurveto{\pgfqpoint{1.463584in}{3.215959in}}{\pgfqpoint{1.467974in}{3.205360in}}{\pgfqpoint{1.475788in}{3.197547in}}%
\pgfpathcurveto{\pgfqpoint{1.483602in}{3.189733in}}{\pgfqpoint{1.494201in}{3.185343in}}{\pgfqpoint{1.505251in}{3.185343in}}%
\pgfpathclose%
\pgfusepath{stroke,fill}%
\end{pgfscope}%
\begin{pgfscope}%
\pgfpathrectangle{\pgfqpoint{0.648703in}{0.548769in}}{\pgfqpoint{5.201297in}{3.102590in}}%
\pgfusepath{clip}%
\pgfsetbuttcap%
\pgfsetroundjoin%
\definecolor{currentfill}{rgb}{1.000000,0.498039,0.054902}%
\pgfsetfillcolor{currentfill}%
\pgfsetlinewidth{1.003750pt}%
\definecolor{currentstroke}{rgb}{1.000000,0.498039,0.054902}%
\pgfsetstrokecolor{currentstroke}%
\pgfsetdash{}{0pt}%
\pgfpathmoveto{\pgfqpoint{1.195188in}{3.405235in}}%
\pgfpathcurveto{\pgfqpoint{1.206239in}{3.405235in}}{\pgfqpoint{1.216838in}{3.409625in}}{\pgfqpoint{1.224651in}{3.417439in}}%
\pgfpathcurveto{\pgfqpoint{1.232465in}{3.425252in}}{\pgfqpoint{1.236855in}{3.435851in}}{\pgfqpoint{1.236855in}{3.446901in}}%
\pgfpathcurveto{\pgfqpoint{1.236855in}{3.457952in}}{\pgfqpoint{1.232465in}{3.468551in}}{\pgfqpoint{1.224651in}{3.476364in}}%
\pgfpathcurveto{\pgfqpoint{1.216838in}{3.484178in}}{\pgfqpoint{1.206239in}{3.488568in}}{\pgfqpoint{1.195188in}{3.488568in}}%
\pgfpathcurveto{\pgfqpoint{1.184138in}{3.488568in}}{\pgfqpoint{1.173539in}{3.484178in}}{\pgfqpoint{1.165726in}{3.476364in}}%
\pgfpathcurveto{\pgfqpoint{1.157912in}{3.468551in}}{\pgfqpoint{1.153522in}{3.457952in}}{\pgfqpoint{1.153522in}{3.446901in}}%
\pgfpathcurveto{\pgfqpoint{1.153522in}{3.435851in}}{\pgfqpoint{1.157912in}{3.425252in}}{\pgfqpoint{1.165726in}{3.417439in}}%
\pgfpathcurveto{\pgfqpoint{1.173539in}{3.409625in}}{\pgfqpoint{1.184138in}{3.405235in}}{\pgfqpoint{1.195188in}{3.405235in}}%
\pgfpathclose%
\pgfusepath{stroke,fill}%
\end{pgfscope}%
\begin{pgfscope}%
\pgfpathrectangle{\pgfqpoint{0.648703in}{0.548769in}}{\pgfqpoint{5.201297in}{3.102590in}}%
\pgfusepath{clip}%
\pgfsetbuttcap%
\pgfsetroundjoin%
\definecolor{currentfill}{rgb}{1.000000,0.498039,0.054902}%
\pgfsetfillcolor{currentfill}%
\pgfsetlinewidth{1.003750pt}%
\definecolor{currentstroke}{rgb}{1.000000,0.498039,0.054902}%
\pgfsetstrokecolor{currentstroke}%
\pgfsetdash{}{0pt}%
\pgfpathmoveto{\pgfqpoint{2.047860in}{3.193800in}}%
\pgfpathcurveto{\pgfqpoint{2.058910in}{3.193800in}}{\pgfqpoint{2.069509in}{3.198191in}}{\pgfqpoint{2.077323in}{3.206004in}}%
\pgfpathcurveto{\pgfqpoint{2.085136in}{3.213818in}}{\pgfqpoint{2.089527in}{3.224417in}}{\pgfqpoint{2.089527in}{3.235467in}}%
\pgfpathcurveto{\pgfqpoint{2.089527in}{3.246517in}}{\pgfqpoint{2.085136in}{3.257116in}}{\pgfqpoint{2.077323in}{3.264930in}}%
\pgfpathcurveto{\pgfqpoint{2.069509in}{3.272743in}}{\pgfqpoint{2.058910in}{3.277134in}}{\pgfqpoint{2.047860in}{3.277134in}}%
\pgfpathcurveto{\pgfqpoint{2.036810in}{3.277134in}}{\pgfqpoint{2.026211in}{3.272743in}}{\pgfqpoint{2.018397in}{3.264930in}}%
\pgfpathcurveto{\pgfqpoint{2.010584in}{3.257116in}}{\pgfqpoint{2.006193in}{3.246517in}}{\pgfqpoint{2.006193in}{3.235467in}}%
\pgfpathcurveto{\pgfqpoint{2.006193in}{3.224417in}}{\pgfqpoint{2.010584in}{3.213818in}}{\pgfqpoint{2.018397in}{3.206004in}}%
\pgfpathcurveto{\pgfqpoint{2.026211in}{3.198191in}}{\pgfqpoint{2.036810in}{3.193800in}}{\pgfqpoint{2.047860in}{3.193800in}}%
\pgfpathclose%
\pgfusepath{stroke,fill}%
\end{pgfscope}%
\begin{pgfscope}%
\pgfpathrectangle{\pgfqpoint{0.648703in}{0.548769in}}{\pgfqpoint{5.201297in}{3.102590in}}%
\pgfusepath{clip}%
\pgfsetbuttcap%
\pgfsetroundjoin%
\definecolor{currentfill}{rgb}{1.000000,0.498039,0.054902}%
\pgfsetfillcolor{currentfill}%
\pgfsetlinewidth{1.003750pt}%
\definecolor{currentstroke}{rgb}{1.000000,0.498039,0.054902}%
\pgfsetstrokecolor{currentstroke}%
\pgfsetdash{}{0pt}%
\pgfpathmoveto{\pgfqpoint{2.745500in}{3.193800in}}%
\pgfpathcurveto{\pgfqpoint{2.756550in}{3.193800in}}{\pgfqpoint{2.767149in}{3.198191in}}{\pgfqpoint{2.774963in}{3.206004in}}%
\pgfpathcurveto{\pgfqpoint{2.782777in}{3.213818in}}{\pgfqpoint{2.787167in}{3.224417in}}{\pgfqpoint{2.787167in}{3.235467in}}%
\pgfpathcurveto{\pgfqpoint{2.787167in}{3.246517in}}{\pgfqpoint{2.782777in}{3.257116in}}{\pgfqpoint{2.774963in}{3.264930in}}%
\pgfpathcurveto{\pgfqpoint{2.767149in}{3.272743in}}{\pgfqpoint{2.756550in}{3.277134in}}{\pgfqpoint{2.745500in}{3.277134in}}%
\pgfpathcurveto{\pgfqpoint{2.734450in}{3.277134in}}{\pgfqpoint{2.723851in}{3.272743in}}{\pgfqpoint{2.716038in}{3.264930in}}%
\pgfpathcurveto{\pgfqpoint{2.708224in}{3.257116in}}{\pgfqpoint{2.703834in}{3.246517in}}{\pgfqpoint{2.703834in}{3.235467in}}%
\pgfpathcurveto{\pgfqpoint{2.703834in}{3.224417in}}{\pgfqpoint{2.708224in}{3.213818in}}{\pgfqpoint{2.716038in}{3.206004in}}%
\pgfpathcurveto{\pgfqpoint{2.723851in}{3.198191in}}{\pgfqpoint{2.734450in}{3.193800in}}{\pgfqpoint{2.745500in}{3.193800in}}%
\pgfpathclose%
\pgfusepath{stroke,fill}%
\end{pgfscope}%
\begin{pgfscope}%
\pgfpathrectangle{\pgfqpoint{0.648703in}{0.548769in}}{\pgfqpoint{5.201297in}{3.102590in}}%
\pgfusepath{clip}%
\pgfsetbuttcap%
\pgfsetroundjoin%
\definecolor{currentfill}{rgb}{1.000000,0.498039,0.054902}%
\pgfsetfillcolor{currentfill}%
\pgfsetlinewidth{1.003750pt}%
\definecolor{currentstroke}{rgb}{1.000000,0.498039,0.054902}%
\pgfsetstrokecolor{currentstroke}%
\pgfsetdash{}{0pt}%
\pgfpathmoveto{\pgfqpoint{1.350220in}{3.358719in}}%
\pgfpathcurveto{\pgfqpoint{1.361270in}{3.358719in}}{\pgfqpoint{1.371869in}{3.363109in}}{\pgfqpoint{1.379682in}{3.370923in}}%
\pgfpathcurveto{\pgfqpoint{1.387496in}{3.378737in}}{\pgfqpoint{1.391886in}{3.389336in}}{\pgfqpoint{1.391886in}{3.400386in}}%
\pgfpathcurveto{\pgfqpoint{1.391886in}{3.411436in}}{\pgfqpoint{1.387496in}{3.422035in}}{\pgfqpoint{1.379682in}{3.429849in}}%
\pgfpathcurveto{\pgfqpoint{1.371869in}{3.437662in}}{\pgfqpoint{1.361270in}{3.442053in}}{\pgfqpoint{1.350220in}{3.442053in}}%
\pgfpathcurveto{\pgfqpoint{1.339169in}{3.442053in}}{\pgfqpoint{1.328570in}{3.437662in}}{\pgfqpoint{1.320757in}{3.429849in}}%
\pgfpathcurveto{\pgfqpoint{1.312943in}{3.422035in}}{\pgfqpoint{1.308553in}{3.411436in}}{\pgfqpoint{1.308553in}{3.400386in}}%
\pgfpathcurveto{\pgfqpoint{1.308553in}{3.389336in}}{\pgfqpoint{1.312943in}{3.378737in}}{\pgfqpoint{1.320757in}{3.370923in}}%
\pgfpathcurveto{\pgfqpoint{1.328570in}{3.363109in}}{\pgfqpoint{1.339169in}{3.358719in}}{\pgfqpoint{1.350220in}{3.358719in}}%
\pgfpathclose%
\pgfusepath{stroke,fill}%
\end{pgfscope}%
\begin{pgfscope}%
\pgfpathrectangle{\pgfqpoint{0.648703in}{0.548769in}}{\pgfqpoint{5.201297in}{3.102590in}}%
\pgfusepath{clip}%
\pgfsetbuttcap%
\pgfsetroundjoin%
\definecolor{currentfill}{rgb}{1.000000,0.498039,0.054902}%
\pgfsetfillcolor{currentfill}%
\pgfsetlinewidth{1.003750pt}%
\definecolor{currentstroke}{rgb}{1.000000,0.498039,0.054902}%
\pgfsetstrokecolor{currentstroke}%
\pgfsetdash{}{0pt}%
\pgfpathmoveto{\pgfqpoint{1.040157in}{3.362948in}}%
\pgfpathcurveto{\pgfqpoint{1.051207in}{3.362948in}}{\pgfqpoint{1.061806in}{3.367338in}}{\pgfqpoint{1.069620in}{3.375152in}}%
\pgfpathcurveto{\pgfqpoint{1.077434in}{3.382965in}}{\pgfqpoint{1.081824in}{3.393564in}}{\pgfqpoint{1.081824in}{3.404615in}}%
\pgfpathcurveto{\pgfqpoint{1.081824in}{3.415665in}}{\pgfqpoint{1.077434in}{3.426264in}}{\pgfqpoint{1.069620in}{3.434077in}}%
\pgfpathcurveto{\pgfqpoint{1.061806in}{3.441891in}}{\pgfqpoint{1.051207in}{3.446281in}}{\pgfqpoint{1.040157in}{3.446281in}}%
\pgfpathcurveto{\pgfqpoint{1.029107in}{3.446281in}}{\pgfqpoint{1.018508in}{3.441891in}}{\pgfqpoint{1.010694in}{3.434077in}}%
\pgfpathcurveto{\pgfqpoint{1.002881in}{3.426264in}}{\pgfqpoint{0.998491in}{3.415665in}}{\pgfqpoint{0.998491in}{3.404615in}}%
\pgfpathcurveto{\pgfqpoint{0.998491in}{3.393564in}}{\pgfqpoint{1.002881in}{3.382965in}}{\pgfqpoint{1.010694in}{3.375152in}}%
\pgfpathcurveto{\pgfqpoint{1.018508in}{3.367338in}}{\pgfqpoint{1.029107in}{3.362948in}}{\pgfqpoint{1.040157in}{3.362948in}}%
\pgfpathclose%
\pgfusepath{stroke,fill}%
\end{pgfscope}%
\begin{pgfscope}%
\pgfpathrectangle{\pgfqpoint{0.648703in}{0.548769in}}{\pgfqpoint{5.201297in}{3.102590in}}%
\pgfusepath{clip}%
\pgfsetbuttcap%
\pgfsetroundjoin%
\definecolor{currentfill}{rgb}{1.000000,0.498039,0.054902}%
\pgfsetfillcolor{currentfill}%
\pgfsetlinewidth{1.003750pt}%
\definecolor{currentstroke}{rgb}{1.000000,0.498039,0.054902}%
\pgfsetstrokecolor{currentstroke}%
\pgfsetdash{}{0pt}%
\pgfpathmoveto{\pgfqpoint{2.978047in}{3.257231in}}%
\pgfpathcurveto{\pgfqpoint{2.989097in}{3.257231in}}{\pgfqpoint{2.999696in}{3.261621in}}{\pgfqpoint{3.007510in}{3.269435in}}%
\pgfpathcurveto{\pgfqpoint{3.015324in}{3.277248in}}{\pgfqpoint{3.019714in}{3.287847in}}{\pgfqpoint{3.019714in}{3.298897in}}%
\pgfpathcurveto{\pgfqpoint{3.019714in}{3.309947in}}{\pgfqpoint{3.015324in}{3.320546in}}{\pgfqpoint{3.007510in}{3.328360in}}%
\pgfpathcurveto{\pgfqpoint{2.999696in}{3.336174in}}{\pgfqpoint{2.989097in}{3.340564in}}{\pgfqpoint{2.978047in}{3.340564in}}%
\pgfpathcurveto{\pgfqpoint{2.966997in}{3.340564in}}{\pgfqpoint{2.956398in}{3.336174in}}{\pgfqpoint{2.948584in}{3.328360in}}%
\pgfpathcurveto{\pgfqpoint{2.940771in}{3.320546in}}{\pgfqpoint{2.936380in}{3.309947in}}{\pgfqpoint{2.936380in}{3.298897in}}%
\pgfpathcurveto{\pgfqpoint{2.936380in}{3.287847in}}{\pgfqpoint{2.940771in}{3.277248in}}{\pgfqpoint{2.948584in}{3.269435in}}%
\pgfpathcurveto{\pgfqpoint{2.956398in}{3.261621in}}{\pgfqpoint{2.966997in}{3.257231in}}{\pgfqpoint{2.978047in}{3.257231in}}%
\pgfpathclose%
\pgfusepath{stroke,fill}%
\end{pgfscope}%
\begin{pgfscope}%
\pgfpathrectangle{\pgfqpoint{0.648703in}{0.548769in}}{\pgfqpoint{5.201297in}{3.102590in}}%
\pgfusepath{clip}%
\pgfsetbuttcap%
\pgfsetroundjoin%
\definecolor{currentfill}{rgb}{0.121569,0.466667,0.705882}%
\pgfsetfillcolor{currentfill}%
\pgfsetlinewidth{1.003750pt}%
\definecolor{currentstroke}{rgb}{0.121569,0.466667,0.705882}%
\pgfsetstrokecolor{currentstroke}%
\pgfsetdash{}{0pt}%
\pgfpathmoveto{\pgfqpoint{0.962642in}{0.648129in}}%
\pgfpathcurveto{\pgfqpoint{0.973692in}{0.648129in}}{\pgfqpoint{0.984291in}{0.652519in}}{\pgfqpoint{0.992104in}{0.660333in}}%
\pgfpathcurveto{\pgfqpoint{0.999918in}{0.668146in}}{\pgfqpoint{1.004308in}{0.678745in}}{\pgfqpoint{1.004308in}{0.689796in}}%
\pgfpathcurveto{\pgfqpoint{1.004308in}{0.700846in}}{\pgfqpoint{0.999918in}{0.711445in}}{\pgfqpoint{0.992104in}{0.719258in}}%
\pgfpathcurveto{\pgfqpoint{0.984291in}{0.727072in}}{\pgfqpoint{0.973692in}{0.731462in}}{\pgfqpoint{0.962642in}{0.731462in}}%
\pgfpathcurveto{\pgfqpoint{0.951591in}{0.731462in}}{\pgfqpoint{0.940992in}{0.727072in}}{\pgfqpoint{0.933179in}{0.719258in}}%
\pgfpathcurveto{\pgfqpoint{0.925365in}{0.711445in}}{\pgfqpoint{0.920975in}{0.700846in}}{\pgfqpoint{0.920975in}{0.689796in}}%
\pgfpathcurveto{\pgfqpoint{0.920975in}{0.678745in}}{\pgfqpoint{0.925365in}{0.668146in}}{\pgfqpoint{0.933179in}{0.660333in}}%
\pgfpathcurveto{\pgfqpoint{0.940992in}{0.652519in}}{\pgfqpoint{0.951591in}{0.648129in}}{\pgfqpoint{0.962642in}{0.648129in}}%
\pgfpathclose%
\pgfusepath{stroke,fill}%
\end{pgfscope}%
\begin{pgfscope}%
\pgfpathrectangle{\pgfqpoint{0.648703in}{0.548769in}}{\pgfqpoint{5.201297in}{3.102590in}}%
\pgfusepath{clip}%
\pgfsetbuttcap%
\pgfsetroundjoin%
\definecolor{currentfill}{rgb}{0.121569,0.466667,0.705882}%
\pgfsetfillcolor{currentfill}%
\pgfsetlinewidth{1.003750pt}%
\definecolor{currentstroke}{rgb}{0.121569,0.466667,0.705882}%
\pgfsetstrokecolor{currentstroke}%
\pgfsetdash{}{0pt}%
\pgfpathmoveto{\pgfqpoint{1.350220in}{0.796133in}}%
\pgfpathcurveto{\pgfqpoint{1.361270in}{0.796133in}}{\pgfqpoint{1.371869in}{0.800523in}}{\pgfqpoint{1.379682in}{0.808337in}}%
\pgfpathcurveto{\pgfqpoint{1.387496in}{0.816151in}}{\pgfqpoint{1.391886in}{0.826750in}}{\pgfqpoint{1.391886in}{0.837800in}}%
\pgfpathcurveto{\pgfqpoint{1.391886in}{0.848850in}}{\pgfqpoint{1.387496in}{0.859449in}}{\pgfqpoint{1.379682in}{0.867263in}}%
\pgfpathcurveto{\pgfqpoint{1.371869in}{0.875076in}}{\pgfqpoint{1.361270in}{0.879466in}}{\pgfqpoint{1.350220in}{0.879466in}}%
\pgfpathcurveto{\pgfqpoint{1.339169in}{0.879466in}}{\pgfqpoint{1.328570in}{0.875076in}}{\pgfqpoint{1.320757in}{0.867263in}}%
\pgfpathcurveto{\pgfqpoint{1.312943in}{0.859449in}}{\pgfqpoint{1.308553in}{0.848850in}}{\pgfqpoint{1.308553in}{0.837800in}}%
\pgfpathcurveto{\pgfqpoint{1.308553in}{0.826750in}}{\pgfqpoint{1.312943in}{0.816151in}}{\pgfqpoint{1.320757in}{0.808337in}}%
\pgfpathcurveto{\pgfqpoint{1.328570in}{0.800523in}}{\pgfqpoint{1.339169in}{0.796133in}}{\pgfqpoint{1.350220in}{0.796133in}}%
\pgfpathclose%
\pgfusepath{stroke,fill}%
\end{pgfscope}%
\begin{pgfscope}%
\pgfpathrectangle{\pgfqpoint{0.648703in}{0.548769in}}{\pgfqpoint{5.201297in}{3.102590in}}%
\pgfusepath{clip}%
\pgfsetbuttcap%
\pgfsetroundjoin%
\definecolor{currentfill}{rgb}{0.121569,0.466667,0.705882}%
\pgfsetfillcolor{currentfill}%
\pgfsetlinewidth{1.003750pt}%
\definecolor{currentstroke}{rgb}{0.121569,0.466667,0.705882}%
\pgfsetstrokecolor{currentstroke}%
\pgfsetdash{}{0pt}%
\pgfpathmoveto{\pgfqpoint{1.195188in}{3.155742in}}%
\pgfpathcurveto{\pgfqpoint{1.206239in}{3.155742in}}{\pgfqpoint{1.216838in}{3.160132in}}{\pgfqpoint{1.224651in}{3.167946in}}%
\pgfpathcurveto{\pgfqpoint{1.232465in}{3.175760in}}{\pgfqpoint{1.236855in}{3.186359in}}{\pgfqpoint{1.236855in}{3.197409in}}%
\pgfpathcurveto{\pgfqpoint{1.236855in}{3.208459in}}{\pgfqpoint{1.232465in}{3.219058in}}{\pgfqpoint{1.224651in}{3.226872in}}%
\pgfpathcurveto{\pgfqpoint{1.216838in}{3.234685in}}{\pgfqpoint{1.206239in}{3.239075in}}{\pgfqpoint{1.195188in}{3.239075in}}%
\pgfpathcurveto{\pgfqpoint{1.184138in}{3.239075in}}{\pgfqpoint{1.173539in}{3.234685in}}{\pgfqpoint{1.165726in}{3.226872in}}%
\pgfpathcurveto{\pgfqpoint{1.157912in}{3.219058in}}{\pgfqpoint{1.153522in}{3.208459in}}{\pgfqpoint{1.153522in}{3.197409in}}%
\pgfpathcurveto{\pgfqpoint{1.153522in}{3.186359in}}{\pgfqpoint{1.157912in}{3.175760in}}{\pgfqpoint{1.165726in}{3.167946in}}%
\pgfpathcurveto{\pgfqpoint{1.173539in}{3.160132in}}{\pgfqpoint{1.184138in}{3.155742in}}{\pgfqpoint{1.195188in}{3.155742in}}%
\pgfpathclose%
\pgfusepath{stroke,fill}%
\end{pgfscope}%
\begin{pgfscope}%
\pgfpathrectangle{\pgfqpoint{0.648703in}{0.548769in}}{\pgfqpoint{5.201297in}{3.102590in}}%
\pgfusepath{clip}%
\pgfsetbuttcap%
\pgfsetroundjoin%
\definecolor{currentfill}{rgb}{0.121569,0.466667,0.705882}%
\pgfsetfillcolor{currentfill}%
\pgfsetlinewidth{1.003750pt}%
\definecolor{currentstroke}{rgb}{0.121569,0.466667,0.705882}%
\pgfsetstrokecolor{currentstroke}%
\pgfsetdash{}{0pt}%
\pgfpathmoveto{\pgfqpoint{1.272704in}{0.648129in}}%
\pgfpathcurveto{\pgfqpoint{1.283754in}{0.648129in}}{\pgfqpoint{1.294353in}{0.652519in}}{\pgfqpoint{1.302167in}{0.660333in}}%
\pgfpathcurveto{\pgfqpoint{1.309980in}{0.668146in}}{\pgfqpoint{1.314371in}{0.678745in}}{\pgfqpoint{1.314371in}{0.689796in}}%
\pgfpathcurveto{\pgfqpoint{1.314371in}{0.700846in}}{\pgfqpoint{1.309980in}{0.711445in}}{\pgfqpoint{1.302167in}{0.719258in}}%
\pgfpathcurveto{\pgfqpoint{1.294353in}{0.727072in}}{\pgfqpoint{1.283754in}{0.731462in}}{\pgfqpoint{1.272704in}{0.731462in}}%
\pgfpathcurveto{\pgfqpoint{1.261654in}{0.731462in}}{\pgfqpoint{1.251055in}{0.727072in}}{\pgfqpoint{1.243241in}{0.719258in}}%
\pgfpathcurveto{\pgfqpoint{1.235428in}{0.711445in}}{\pgfqpoint{1.231037in}{0.700846in}}{\pgfqpoint{1.231037in}{0.689796in}}%
\pgfpathcurveto{\pgfqpoint{1.231037in}{0.678745in}}{\pgfqpoint{1.235428in}{0.668146in}}{\pgfqpoint{1.243241in}{0.660333in}}%
\pgfpathcurveto{\pgfqpoint{1.251055in}{0.652519in}}{\pgfqpoint{1.261654in}{0.648129in}}{\pgfqpoint{1.272704in}{0.648129in}}%
\pgfpathclose%
\pgfusepath{stroke,fill}%
\end{pgfscope}%
\begin{pgfscope}%
\pgfpathrectangle{\pgfqpoint{0.648703in}{0.548769in}}{\pgfqpoint{5.201297in}{3.102590in}}%
\pgfusepath{clip}%
\pgfsetbuttcap%
\pgfsetroundjoin%
\definecolor{currentfill}{rgb}{0.121569,0.466667,0.705882}%
\pgfsetfillcolor{currentfill}%
\pgfsetlinewidth{1.003750pt}%
\definecolor{currentstroke}{rgb}{0.121569,0.466667,0.705882}%
\pgfsetstrokecolor{currentstroke}%
\pgfsetdash{}{0pt}%
\pgfpathmoveto{\pgfqpoint{1.272704in}{0.648129in}}%
\pgfpathcurveto{\pgfqpoint{1.283754in}{0.648129in}}{\pgfqpoint{1.294353in}{0.652519in}}{\pgfqpoint{1.302167in}{0.660333in}}%
\pgfpathcurveto{\pgfqpoint{1.309980in}{0.668146in}}{\pgfqpoint{1.314371in}{0.678745in}}{\pgfqpoint{1.314371in}{0.689796in}}%
\pgfpathcurveto{\pgfqpoint{1.314371in}{0.700846in}}{\pgfqpoint{1.309980in}{0.711445in}}{\pgfqpoint{1.302167in}{0.719258in}}%
\pgfpathcurveto{\pgfqpoint{1.294353in}{0.727072in}}{\pgfqpoint{1.283754in}{0.731462in}}{\pgfqpoint{1.272704in}{0.731462in}}%
\pgfpathcurveto{\pgfqpoint{1.261654in}{0.731462in}}{\pgfqpoint{1.251055in}{0.727072in}}{\pgfqpoint{1.243241in}{0.719258in}}%
\pgfpathcurveto{\pgfqpoint{1.235428in}{0.711445in}}{\pgfqpoint{1.231037in}{0.700846in}}{\pgfqpoint{1.231037in}{0.689796in}}%
\pgfpathcurveto{\pgfqpoint{1.231037in}{0.678745in}}{\pgfqpoint{1.235428in}{0.668146in}}{\pgfqpoint{1.243241in}{0.660333in}}%
\pgfpathcurveto{\pgfqpoint{1.251055in}{0.652519in}}{\pgfqpoint{1.261654in}{0.648129in}}{\pgfqpoint{1.272704in}{0.648129in}}%
\pgfpathclose%
\pgfusepath{stroke,fill}%
\end{pgfscope}%
\begin{pgfscope}%
\pgfpathrectangle{\pgfqpoint{0.648703in}{0.548769in}}{\pgfqpoint{5.201297in}{3.102590in}}%
\pgfusepath{clip}%
\pgfsetbuttcap%
\pgfsetroundjoin%
\definecolor{currentfill}{rgb}{0.121569,0.466667,0.705882}%
\pgfsetfillcolor{currentfill}%
\pgfsetlinewidth{1.003750pt}%
\definecolor{currentstroke}{rgb}{0.121569,0.466667,0.705882}%
\pgfsetstrokecolor{currentstroke}%
\pgfsetdash{}{0pt}%
\pgfpathmoveto{\pgfqpoint{0.962642in}{2.512981in}}%
\pgfpathcurveto{\pgfqpoint{0.973692in}{2.512981in}}{\pgfqpoint{0.984291in}{2.517371in}}{\pgfqpoint{0.992104in}{2.525185in}}%
\pgfpathcurveto{\pgfqpoint{0.999918in}{2.532999in}}{\pgfqpoint{1.004308in}{2.543598in}}{\pgfqpoint{1.004308in}{2.554648in}}%
\pgfpathcurveto{\pgfqpoint{1.004308in}{2.565698in}}{\pgfqpoint{0.999918in}{2.576297in}}{\pgfqpoint{0.992104in}{2.584111in}}%
\pgfpathcurveto{\pgfqpoint{0.984291in}{2.591924in}}{\pgfqpoint{0.973692in}{2.596315in}}{\pgfqpoint{0.962642in}{2.596315in}}%
\pgfpathcurveto{\pgfqpoint{0.951591in}{2.596315in}}{\pgfqpoint{0.940992in}{2.591924in}}{\pgfqpoint{0.933179in}{2.584111in}}%
\pgfpathcurveto{\pgfqpoint{0.925365in}{2.576297in}}{\pgfqpoint{0.920975in}{2.565698in}}{\pgfqpoint{0.920975in}{2.554648in}}%
\pgfpathcurveto{\pgfqpoint{0.920975in}{2.543598in}}{\pgfqpoint{0.925365in}{2.532999in}}{\pgfqpoint{0.933179in}{2.525185in}}%
\pgfpathcurveto{\pgfqpoint{0.940992in}{2.517371in}}{\pgfqpoint{0.951591in}{2.512981in}}{\pgfqpoint{0.962642in}{2.512981in}}%
\pgfpathclose%
\pgfusepath{stroke,fill}%
\end{pgfscope}%
\begin{pgfscope}%
\pgfpathrectangle{\pgfqpoint{0.648703in}{0.548769in}}{\pgfqpoint{5.201297in}{3.102590in}}%
\pgfusepath{clip}%
\pgfsetbuttcap%
\pgfsetroundjoin%
\definecolor{currentfill}{rgb}{1.000000,0.498039,0.054902}%
\pgfsetfillcolor{currentfill}%
\pgfsetlinewidth{1.003750pt}%
\definecolor{currentstroke}{rgb}{1.000000,0.498039,0.054902}%
\pgfsetstrokecolor{currentstroke}%
\pgfsetdash{}{0pt}%
\pgfpathmoveto{\pgfqpoint{2.125376in}{3.189572in}}%
\pgfpathcurveto{\pgfqpoint{2.136426in}{3.189572in}}{\pgfqpoint{2.147025in}{3.193962in}}{\pgfqpoint{2.154838in}{3.201775in}}%
\pgfpathcurveto{\pgfqpoint{2.162652in}{3.209589in}}{\pgfqpoint{2.167042in}{3.220188in}}{\pgfqpoint{2.167042in}{3.231238in}}%
\pgfpathcurveto{\pgfqpoint{2.167042in}{3.242288in}}{\pgfqpoint{2.162652in}{3.252887in}}{\pgfqpoint{2.154838in}{3.260701in}}%
\pgfpathcurveto{\pgfqpoint{2.147025in}{3.268515in}}{\pgfqpoint{2.136426in}{3.272905in}}{\pgfqpoint{2.125376in}{3.272905in}}%
\pgfpathcurveto{\pgfqpoint{2.114325in}{3.272905in}}{\pgfqpoint{2.103726in}{3.268515in}}{\pgfqpoint{2.095913in}{3.260701in}}%
\pgfpathcurveto{\pgfqpoint{2.088099in}{3.252887in}}{\pgfqpoint{2.083709in}{3.242288in}}{\pgfqpoint{2.083709in}{3.231238in}}%
\pgfpathcurveto{\pgfqpoint{2.083709in}{3.220188in}}{\pgfqpoint{2.088099in}{3.209589in}}{\pgfqpoint{2.095913in}{3.201775in}}%
\pgfpathcurveto{\pgfqpoint{2.103726in}{3.193962in}}{\pgfqpoint{2.114325in}{3.189572in}}{\pgfqpoint{2.125376in}{3.189572in}}%
\pgfpathclose%
\pgfusepath{stroke,fill}%
\end{pgfscope}%
\begin{pgfscope}%
\pgfpathrectangle{\pgfqpoint{0.648703in}{0.548769in}}{\pgfqpoint{5.201297in}{3.102590in}}%
\pgfusepath{clip}%
\pgfsetbuttcap%
\pgfsetroundjoin%
\definecolor{currentfill}{rgb}{1.000000,0.498039,0.054902}%
\pgfsetfillcolor{currentfill}%
\pgfsetlinewidth{1.003750pt}%
\definecolor{currentstroke}{rgb}{1.000000,0.498039,0.054902}%
\pgfsetstrokecolor{currentstroke}%
\pgfsetdash{}{0pt}%
\pgfpathmoveto{\pgfqpoint{0.885126in}{3.206486in}}%
\pgfpathcurveto{\pgfqpoint{0.896176in}{3.206486in}}{\pgfqpoint{0.906775in}{3.210877in}}{\pgfqpoint{0.914589in}{3.218690in}}%
\pgfpathcurveto{\pgfqpoint{0.922402in}{3.226504in}}{\pgfqpoint{0.926793in}{3.237103in}}{\pgfqpoint{0.926793in}{3.248153in}}%
\pgfpathcurveto{\pgfqpoint{0.926793in}{3.259203in}}{\pgfqpoint{0.922402in}{3.269802in}}{\pgfqpoint{0.914589in}{3.277616in}}%
\pgfpathcurveto{\pgfqpoint{0.906775in}{3.285429in}}{\pgfqpoint{0.896176in}{3.289820in}}{\pgfqpoint{0.885126in}{3.289820in}}%
\pgfpathcurveto{\pgfqpoint{0.874076in}{3.289820in}}{\pgfqpoint{0.863477in}{3.285429in}}{\pgfqpoint{0.855663in}{3.277616in}}%
\pgfpathcurveto{\pgfqpoint{0.847850in}{3.269802in}}{\pgfqpoint{0.843459in}{3.259203in}}{\pgfqpoint{0.843459in}{3.248153in}}%
\pgfpathcurveto{\pgfqpoint{0.843459in}{3.237103in}}{\pgfqpoint{0.847850in}{3.226504in}}{\pgfqpoint{0.855663in}{3.218690in}}%
\pgfpathcurveto{\pgfqpoint{0.863477in}{3.210877in}}{\pgfqpoint{0.874076in}{3.206486in}}{\pgfqpoint{0.885126in}{3.206486in}}%
\pgfpathclose%
\pgfusepath{stroke,fill}%
\end{pgfscope}%
\begin{pgfscope}%
\pgfsetbuttcap%
\pgfsetroundjoin%
\definecolor{currentfill}{rgb}{0.000000,0.000000,0.000000}%
\pgfsetfillcolor{currentfill}%
\pgfsetlinewidth{0.803000pt}%
\definecolor{currentstroke}{rgb}{0.000000,0.000000,0.000000}%
\pgfsetstrokecolor{currentstroke}%
\pgfsetdash{}{0pt}%
\pgfsys@defobject{currentmarker}{\pgfqpoint{0.000000in}{-0.048611in}}{\pgfqpoint{0.000000in}{0.000000in}}{%
\pgfpathmoveto{\pgfqpoint{0.000000in}{0.000000in}}%
\pgfpathlineto{\pgfqpoint{0.000000in}{-0.048611in}}%
\pgfusepath{stroke,fill}%
}%
\begin{pgfscope}%
\pgfsys@transformshift{0.807610in}{0.548769in}%
\pgfsys@useobject{currentmarker}{}%
\end{pgfscope}%
\end{pgfscope}%
\begin{pgfscope}%
\definecolor{textcolor}{rgb}{0.000000,0.000000,0.000000}%
\pgfsetstrokecolor{textcolor}%
\pgfsetfillcolor{textcolor}%
\pgftext[x=0.807610in,y=0.451547in,,top]{\color{textcolor}\sffamily\fontsize{10.000000}{12.000000}\selectfont \(\displaystyle {0}\)}%
\end{pgfscope}%
\begin{pgfscope}%
\pgfsetbuttcap%
\pgfsetroundjoin%
\definecolor{currentfill}{rgb}{0.000000,0.000000,0.000000}%
\pgfsetfillcolor{currentfill}%
\pgfsetlinewidth{0.803000pt}%
\definecolor{currentstroke}{rgb}{0.000000,0.000000,0.000000}%
\pgfsetstrokecolor{currentstroke}%
\pgfsetdash{}{0pt}%
\pgfsys@defobject{currentmarker}{\pgfqpoint{0.000000in}{-0.048611in}}{\pgfqpoint{0.000000in}{0.000000in}}{%
\pgfpathmoveto{\pgfqpoint{0.000000in}{0.000000in}}%
\pgfpathlineto{\pgfqpoint{0.000000in}{-0.048611in}}%
\pgfusepath{stroke,fill}%
}%
\begin{pgfscope}%
\pgfsys@transformshift{1.582766in}{0.548769in}%
\pgfsys@useobject{currentmarker}{}%
\end{pgfscope}%
\end{pgfscope}%
\begin{pgfscope}%
\definecolor{textcolor}{rgb}{0.000000,0.000000,0.000000}%
\pgfsetstrokecolor{textcolor}%
\pgfsetfillcolor{textcolor}%
\pgftext[x=1.582766in,y=0.451547in,,top]{\color{textcolor}\sffamily\fontsize{10.000000}{12.000000}\selectfont \(\displaystyle {10}\)}%
\end{pgfscope}%
\begin{pgfscope}%
\pgfsetbuttcap%
\pgfsetroundjoin%
\definecolor{currentfill}{rgb}{0.000000,0.000000,0.000000}%
\pgfsetfillcolor{currentfill}%
\pgfsetlinewidth{0.803000pt}%
\definecolor{currentstroke}{rgb}{0.000000,0.000000,0.000000}%
\pgfsetstrokecolor{currentstroke}%
\pgfsetdash{}{0pt}%
\pgfsys@defobject{currentmarker}{\pgfqpoint{0.000000in}{-0.048611in}}{\pgfqpoint{0.000000in}{0.000000in}}{%
\pgfpathmoveto{\pgfqpoint{0.000000in}{0.000000in}}%
\pgfpathlineto{\pgfqpoint{0.000000in}{-0.048611in}}%
\pgfusepath{stroke,fill}%
}%
\begin{pgfscope}%
\pgfsys@transformshift{2.357922in}{0.548769in}%
\pgfsys@useobject{currentmarker}{}%
\end{pgfscope}%
\end{pgfscope}%
\begin{pgfscope}%
\definecolor{textcolor}{rgb}{0.000000,0.000000,0.000000}%
\pgfsetstrokecolor{textcolor}%
\pgfsetfillcolor{textcolor}%
\pgftext[x=2.357922in,y=0.451547in,,top]{\color{textcolor}\sffamily\fontsize{10.000000}{12.000000}\selectfont \(\displaystyle {20}\)}%
\end{pgfscope}%
\begin{pgfscope}%
\pgfsetbuttcap%
\pgfsetroundjoin%
\definecolor{currentfill}{rgb}{0.000000,0.000000,0.000000}%
\pgfsetfillcolor{currentfill}%
\pgfsetlinewidth{0.803000pt}%
\definecolor{currentstroke}{rgb}{0.000000,0.000000,0.000000}%
\pgfsetstrokecolor{currentstroke}%
\pgfsetdash{}{0pt}%
\pgfsys@defobject{currentmarker}{\pgfqpoint{0.000000in}{-0.048611in}}{\pgfqpoint{0.000000in}{0.000000in}}{%
\pgfpathmoveto{\pgfqpoint{0.000000in}{0.000000in}}%
\pgfpathlineto{\pgfqpoint{0.000000in}{-0.048611in}}%
\pgfusepath{stroke,fill}%
}%
\begin{pgfscope}%
\pgfsys@transformshift{3.133078in}{0.548769in}%
\pgfsys@useobject{currentmarker}{}%
\end{pgfscope}%
\end{pgfscope}%
\begin{pgfscope}%
\definecolor{textcolor}{rgb}{0.000000,0.000000,0.000000}%
\pgfsetstrokecolor{textcolor}%
\pgfsetfillcolor{textcolor}%
\pgftext[x=3.133078in,y=0.451547in,,top]{\color{textcolor}\sffamily\fontsize{10.000000}{12.000000}\selectfont \(\displaystyle {30}\)}%
\end{pgfscope}%
\begin{pgfscope}%
\pgfsetbuttcap%
\pgfsetroundjoin%
\definecolor{currentfill}{rgb}{0.000000,0.000000,0.000000}%
\pgfsetfillcolor{currentfill}%
\pgfsetlinewidth{0.803000pt}%
\definecolor{currentstroke}{rgb}{0.000000,0.000000,0.000000}%
\pgfsetstrokecolor{currentstroke}%
\pgfsetdash{}{0pt}%
\pgfsys@defobject{currentmarker}{\pgfqpoint{0.000000in}{-0.048611in}}{\pgfqpoint{0.000000in}{0.000000in}}{%
\pgfpathmoveto{\pgfqpoint{0.000000in}{0.000000in}}%
\pgfpathlineto{\pgfqpoint{0.000000in}{-0.048611in}}%
\pgfusepath{stroke,fill}%
}%
\begin{pgfscope}%
\pgfsys@transformshift{3.908234in}{0.548769in}%
\pgfsys@useobject{currentmarker}{}%
\end{pgfscope}%
\end{pgfscope}%
\begin{pgfscope}%
\definecolor{textcolor}{rgb}{0.000000,0.000000,0.000000}%
\pgfsetstrokecolor{textcolor}%
\pgfsetfillcolor{textcolor}%
\pgftext[x=3.908234in,y=0.451547in,,top]{\color{textcolor}\sffamily\fontsize{10.000000}{12.000000}\selectfont \(\displaystyle {40}\)}%
\end{pgfscope}%
\begin{pgfscope}%
\pgfsetbuttcap%
\pgfsetroundjoin%
\definecolor{currentfill}{rgb}{0.000000,0.000000,0.000000}%
\pgfsetfillcolor{currentfill}%
\pgfsetlinewidth{0.803000pt}%
\definecolor{currentstroke}{rgb}{0.000000,0.000000,0.000000}%
\pgfsetstrokecolor{currentstroke}%
\pgfsetdash{}{0pt}%
\pgfsys@defobject{currentmarker}{\pgfqpoint{0.000000in}{-0.048611in}}{\pgfqpoint{0.000000in}{0.000000in}}{%
\pgfpathmoveto{\pgfqpoint{0.000000in}{0.000000in}}%
\pgfpathlineto{\pgfqpoint{0.000000in}{-0.048611in}}%
\pgfusepath{stroke,fill}%
}%
\begin{pgfscope}%
\pgfsys@transformshift{4.683390in}{0.548769in}%
\pgfsys@useobject{currentmarker}{}%
\end{pgfscope}%
\end{pgfscope}%
\begin{pgfscope}%
\definecolor{textcolor}{rgb}{0.000000,0.000000,0.000000}%
\pgfsetstrokecolor{textcolor}%
\pgfsetfillcolor{textcolor}%
\pgftext[x=4.683390in,y=0.451547in,,top]{\color{textcolor}\sffamily\fontsize{10.000000}{12.000000}\selectfont \(\displaystyle {50}\)}%
\end{pgfscope}%
\begin{pgfscope}%
\pgfsetbuttcap%
\pgfsetroundjoin%
\definecolor{currentfill}{rgb}{0.000000,0.000000,0.000000}%
\pgfsetfillcolor{currentfill}%
\pgfsetlinewidth{0.803000pt}%
\definecolor{currentstroke}{rgb}{0.000000,0.000000,0.000000}%
\pgfsetstrokecolor{currentstroke}%
\pgfsetdash{}{0pt}%
\pgfsys@defobject{currentmarker}{\pgfqpoint{0.000000in}{-0.048611in}}{\pgfqpoint{0.000000in}{0.000000in}}{%
\pgfpathmoveto{\pgfqpoint{0.000000in}{0.000000in}}%
\pgfpathlineto{\pgfqpoint{0.000000in}{-0.048611in}}%
\pgfusepath{stroke,fill}%
}%
\begin{pgfscope}%
\pgfsys@transformshift{5.458546in}{0.548769in}%
\pgfsys@useobject{currentmarker}{}%
\end{pgfscope}%
\end{pgfscope}%
\begin{pgfscope}%
\definecolor{textcolor}{rgb}{0.000000,0.000000,0.000000}%
\pgfsetstrokecolor{textcolor}%
\pgfsetfillcolor{textcolor}%
\pgftext[x=5.458546in,y=0.451547in,,top]{\color{textcolor}\sffamily\fontsize{10.000000}{12.000000}\selectfont \(\displaystyle {60}\)}%
\end{pgfscope}%
\begin{pgfscope}%
\definecolor{textcolor}{rgb}{0.000000,0.000000,0.000000}%
\pgfsetstrokecolor{textcolor}%
\pgfsetfillcolor{textcolor}%
\pgftext[x=3.249352in,y=0.272658in,,top]{\color{textcolor}\sffamily\fontsize{10.000000}{12.000000}\selectfont Number of Sinks}%
\end{pgfscope}%
\begin{pgfscope}%
\pgfsetbuttcap%
\pgfsetroundjoin%
\definecolor{currentfill}{rgb}{0.000000,0.000000,0.000000}%
\pgfsetfillcolor{currentfill}%
\pgfsetlinewidth{0.803000pt}%
\definecolor{currentstroke}{rgb}{0.000000,0.000000,0.000000}%
\pgfsetstrokecolor{currentstroke}%
\pgfsetdash{}{0pt}%
\pgfsys@defobject{currentmarker}{\pgfqpoint{-0.048611in}{0.000000in}}{\pgfqpoint{0.000000in}{0.000000in}}{%
\pgfpathmoveto{\pgfqpoint{0.000000in}{0.000000in}}%
\pgfpathlineto{\pgfqpoint{-0.048611in}{0.000000in}}%
\pgfusepath{stroke,fill}%
}%
\begin{pgfscope}%
\pgfsys@transformshift{0.648703in}{0.689796in}%
\pgfsys@useobject{currentmarker}{}%
\end{pgfscope}%
\end{pgfscope}%
\begin{pgfscope}%
\definecolor{textcolor}{rgb}{0.000000,0.000000,0.000000}%
\pgfsetstrokecolor{textcolor}%
\pgfsetfillcolor{textcolor}%
\pgftext[x=0.482036in, y=0.641601in, left, base]{\color{textcolor}\sffamily\fontsize{10.000000}{12.000000}\selectfont \(\displaystyle {0}\)}%
\end{pgfscope}%
\begin{pgfscope}%
\pgfsetbuttcap%
\pgfsetroundjoin%
\definecolor{currentfill}{rgb}{0.000000,0.000000,0.000000}%
\pgfsetfillcolor{currentfill}%
\pgfsetlinewidth{0.803000pt}%
\definecolor{currentstroke}{rgb}{0.000000,0.000000,0.000000}%
\pgfsetstrokecolor{currentstroke}%
\pgfsetdash{}{0pt}%
\pgfsys@defobject{currentmarker}{\pgfqpoint{-0.048611in}{0.000000in}}{\pgfqpoint{0.000000in}{0.000000in}}{%
\pgfpathmoveto{\pgfqpoint{0.000000in}{0.000000in}}%
\pgfpathlineto{\pgfqpoint{-0.048611in}{0.000000in}}%
\pgfusepath{stroke,fill}%
}%
\begin{pgfscope}%
\pgfsys@transformshift{0.648703in}{1.112665in}%
\pgfsys@useobject{currentmarker}{}%
\end{pgfscope}%
\end{pgfscope}%
\begin{pgfscope}%
\definecolor{textcolor}{rgb}{0.000000,0.000000,0.000000}%
\pgfsetstrokecolor{textcolor}%
\pgfsetfillcolor{textcolor}%
\pgftext[x=0.343147in, y=1.064470in, left, base]{\color{textcolor}\sffamily\fontsize{10.000000}{12.000000}\selectfont \(\displaystyle {100}\)}%
\end{pgfscope}%
\begin{pgfscope}%
\pgfsetbuttcap%
\pgfsetroundjoin%
\definecolor{currentfill}{rgb}{0.000000,0.000000,0.000000}%
\pgfsetfillcolor{currentfill}%
\pgfsetlinewidth{0.803000pt}%
\definecolor{currentstroke}{rgb}{0.000000,0.000000,0.000000}%
\pgfsetstrokecolor{currentstroke}%
\pgfsetdash{}{0pt}%
\pgfsys@defobject{currentmarker}{\pgfqpoint{-0.048611in}{0.000000in}}{\pgfqpoint{0.000000in}{0.000000in}}{%
\pgfpathmoveto{\pgfqpoint{0.000000in}{0.000000in}}%
\pgfpathlineto{\pgfqpoint{-0.048611in}{0.000000in}}%
\pgfusepath{stroke,fill}%
}%
\begin{pgfscope}%
\pgfsys@transformshift{0.648703in}{1.535534in}%
\pgfsys@useobject{currentmarker}{}%
\end{pgfscope}%
\end{pgfscope}%
\begin{pgfscope}%
\definecolor{textcolor}{rgb}{0.000000,0.000000,0.000000}%
\pgfsetstrokecolor{textcolor}%
\pgfsetfillcolor{textcolor}%
\pgftext[x=0.343147in, y=1.487339in, left, base]{\color{textcolor}\sffamily\fontsize{10.000000}{12.000000}\selectfont \(\displaystyle {200}\)}%
\end{pgfscope}%
\begin{pgfscope}%
\pgfsetbuttcap%
\pgfsetroundjoin%
\definecolor{currentfill}{rgb}{0.000000,0.000000,0.000000}%
\pgfsetfillcolor{currentfill}%
\pgfsetlinewidth{0.803000pt}%
\definecolor{currentstroke}{rgb}{0.000000,0.000000,0.000000}%
\pgfsetstrokecolor{currentstroke}%
\pgfsetdash{}{0pt}%
\pgfsys@defobject{currentmarker}{\pgfqpoint{-0.048611in}{0.000000in}}{\pgfqpoint{0.000000in}{0.000000in}}{%
\pgfpathmoveto{\pgfqpoint{0.000000in}{0.000000in}}%
\pgfpathlineto{\pgfqpoint{-0.048611in}{0.000000in}}%
\pgfusepath{stroke,fill}%
}%
\begin{pgfscope}%
\pgfsys@transformshift{0.648703in}{1.958403in}%
\pgfsys@useobject{currentmarker}{}%
\end{pgfscope}%
\end{pgfscope}%
\begin{pgfscope}%
\definecolor{textcolor}{rgb}{0.000000,0.000000,0.000000}%
\pgfsetstrokecolor{textcolor}%
\pgfsetfillcolor{textcolor}%
\pgftext[x=0.343147in, y=1.910208in, left, base]{\color{textcolor}\sffamily\fontsize{10.000000}{12.000000}\selectfont \(\displaystyle {300}\)}%
\end{pgfscope}%
\begin{pgfscope}%
\pgfsetbuttcap%
\pgfsetroundjoin%
\definecolor{currentfill}{rgb}{0.000000,0.000000,0.000000}%
\pgfsetfillcolor{currentfill}%
\pgfsetlinewidth{0.803000pt}%
\definecolor{currentstroke}{rgb}{0.000000,0.000000,0.000000}%
\pgfsetstrokecolor{currentstroke}%
\pgfsetdash{}{0pt}%
\pgfsys@defobject{currentmarker}{\pgfqpoint{-0.048611in}{0.000000in}}{\pgfqpoint{0.000000in}{0.000000in}}{%
\pgfpathmoveto{\pgfqpoint{0.000000in}{0.000000in}}%
\pgfpathlineto{\pgfqpoint{-0.048611in}{0.000000in}}%
\pgfusepath{stroke,fill}%
}%
\begin{pgfscope}%
\pgfsys@transformshift{0.648703in}{2.381272in}%
\pgfsys@useobject{currentmarker}{}%
\end{pgfscope}%
\end{pgfscope}%
\begin{pgfscope}%
\definecolor{textcolor}{rgb}{0.000000,0.000000,0.000000}%
\pgfsetstrokecolor{textcolor}%
\pgfsetfillcolor{textcolor}%
\pgftext[x=0.343147in, y=2.333077in, left, base]{\color{textcolor}\sffamily\fontsize{10.000000}{12.000000}\selectfont \(\displaystyle {400}\)}%
\end{pgfscope}%
\begin{pgfscope}%
\pgfsetbuttcap%
\pgfsetroundjoin%
\definecolor{currentfill}{rgb}{0.000000,0.000000,0.000000}%
\pgfsetfillcolor{currentfill}%
\pgfsetlinewidth{0.803000pt}%
\definecolor{currentstroke}{rgb}{0.000000,0.000000,0.000000}%
\pgfsetstrokecolor{currentstroke}%
\pgfsetdash{}{0pt}%
\pgfsys@defobject{currentmarker}{\pgfqpoint{-0.048611in}{0.000000in}}{\pgfqpoint{0.000000in}{0.000000in}}{%
\pgfpathmoveto{\pgfqpoint{0.000000in}{0.000000in}}%
\pgfpathlineto{\pgfqpoint{-0.048611in}{0.000000in}}%
\pgfusepath{stroke,fill}%
}%
\begin{pgfscope}%
\pgfsys@transformshift{0.648703in}{2.804141in}%
\pgfsys@useobject{currentmarker}{}%
\end{pgfscope}%
\end{pgfscope}%
\begin{pgfscope}%
\definecolor{textcolor}{rgb}{0.000000,0.000000,0.000000}%
\pgfsetstrokecolor{textcolor}%
\pgfsetfillcolor{textcolor}%
\pgftext[x=0.343147in, y=2.755946in, left, base]{\color{textcolor}\sffamily\fontsize{10.000000}{12.000000}\selectfont \(\displaystyle {500}\)}%
\end{pgfscope}%
\begin{pgfscope}%
\pgfsetbuttcap%
\pgfsetroundjoin%
\definecolor{currentfill}{rgb}{0.000000,0.000000,0.000000}%
\pgfsetfillcolor{currentfill}%
\pgfsetlinewidth{0.803000pt}%
\definecolor{currentstroke}{rgb}{0.000000,0.000000,0.000000}%
\pgfsetstrokecolor{currentstroke}%
\pgfsetdash{}{0pt}%
\pgfsys@defobject{currentmarker}{\pgfqpoint{-0.048611in}{0.000000in}}{\pgfqpoint{0.000000in}{0.000000in}}{%
\pgfpathmoveto{\pgfqpoint{0.000000in}{0.000000in}}%
\pgfpathlineto{\pgfqpoint{-0.048611in}{0.000000in}}%
\pgfusepath{stroke,fill}%
}%
\begin{pgfscope}%
\pgfsys@transformshift{0.648703in}{3.227010in}%
\pgfsys@useobject{currentmarker}{}%
\end{pgfscope}%
\end{pgfscope}%
\begin{pgfscope}%
\definecolor{textcolor}{rgb}{0.000000,0.000000,0.000000}%
\pgfsetstrokecolor{textcolor}%
\pgfsetfillcolor{textcolor}%
\pgftext[x=0.343147in, y=3.178815in, left, base]{\color{textcolor}\sffamily\fontsize{10.000000}{12.000000}\selectfont \(\displaystyle {600}\)}%
\end{pgfscope}%
\begin{pgfscope}%
\pgfsetbuttcap%
\pgfsetroundjoin%
\definecolor{currentfill}{rgb}{0.000000,0.000000,0.000000}%
\pgfsetfillcolor{currentfill}%
\pgfsetlinewidth{0.803000pt}%
\definecolor{currentstroke}{rgb}{0.000000,0.000000,0.000000}%
\pgfsetstrokecolor{currentstroke}%
\pgfsetdash{}{0pt}%
\pgfsys@defobject{currentmarker}{\pgfqpoint{-0.048611in}{0.000000in}}{\pgfqpoint{0.000000in}{0.000000in}}{%
\pgfpathmoveto{\pgfqpoint{0.000000in}{0.000000in}}%
\pgfpathlineto{\pgfqpoint{-0.048611in}{0.000000in}}%
\pgfusepath{stroke,fill}%
}%
\begin{pgfscope}%
\pgfsys@transformshift{0.648703in}{3.649879in}%
\pgfsys@useobject{currentmarker}{}%
\end{pgfscope}%
\end{pgfscope}%
\begin{pgfscope}%
\definecolor{textcolor}{rgb}{0.000000,0.000000,0.000000}%
\pgfsetstrokecolor{textcolor}%
\pgfsetfillcolor{textcolor}%
\pgftext[x=0.343147in, y=3.601684in, left, base]{\color{textcolor}\sffamily\fontsize{10.000000}{12.000000}\selectfont \(\displaystyle {700}\)}%
\end{pgfscope}%
\begin{pgfscope}%
\definecolor{textcolor}{rgb}{0.000000,0.000000,0.000000}%
\pgfsetstrokecolor{textcolor}%
\pgfsetfillcolor{textcolor}%
\pgftext[x=0.287592in,y=2.100064in,,bottom,rotate=90.000000]{\color{textcolor}\sffamily\fontsize{10.000000}{12.000000}\selectfont Data Flow Time (s)}%
\end{pgfscope}%
\begin{pgfscope}%
\pgfsetrectcap%
\pgfsetmiterjoin%
\pgfsetlinewidth{0.803000pt}%
\definecolor{currentstroke}{rgb}{0.000000,0.000000,0.000000}%
\pgfsetstrokecolor{currentstroke}%
\pgfsetdash{}{0pt}%
\pgfpathmoveto{\pgfqpoint{0.648703in}{0.548769in}}%
\pgfpathlineto{\pgfqpoint{0.648703in}{3.651359in}}%
\pgfusepath{stroke}%
\end{pgfscope}%
\begin{pgfscope}%
\pgfsetrectcap%
\pgfsetmiterjoin%
\pgfsetlinewidth{0.803000pt}%
\definecolor{currentstroke}{rgb}{0.000000,0.000000,0.000000}%
\pgfsetstrokecolor{currentstroke}%
\pgfsetdash{}{0pt}%
\pgfpathmoveto{\pgfqpoint{5.850000in}{0.548769in}}%
\pgfpathlineto{\pgfqpoint{5.850000in}{3.651359in}}%
\pgfusepath{stroke}%
\end{pgfscope}%
\begin{pgfscope}%
\pgfsetrectcap%
\pgfsetmiterjoin%
\pgfsetlinewidth{0.803000pt}%
\definecolor{currentstroke}{rgb}{0.000000,0.000000,0.000000}%
\pgfsetstrokecolor{currentstroke}%
\pgfsetdash{}{0pt}%
\pgfpathmoveto{\pgfqpoint{0.648703in}{0.548769in}}%
\pgfpathlineto{\pgfqpoint{5.850000in}{0.548769in}}%
\pgfusepath{stroke}%
\end{pgfscope}%
\begin{pgfscope}%
\pgfsetrectcap%
\pgfsetmiterjoin%
\pgfsetlinewidth{0.803000pt}%
\definecolor{currentstroke}{rgb}{0.000000,0.000000,0.000000}%
\pgfsetstrokecolor{currentstroke}%
\pgfsetdash{}{0pt}%
\pgfpathmoveto{\pgfqpoint{0.648703in}{3.651359in}}%
\pgfpathlineto{\pgfqpoint{5.850000in}{3.651359in}}%
\pgfusepath{stroke}%
\end{pgfscope}%
\begin{pgfscope}%
\definecolor{textcolor}{rgb}{0.000000,0.000000,0.000000}%
\pgfsetstrokecolor{textcolor}%
\pgfsetfillcolor{textcolor}%
\pgftext[x=3.249352in,y=3.734692in,,base]{\color{textcolor}\sffamily\fontsize{12.000000}{14.400000}\selectfont Backwards}%
\end{pgfscope}%
\begin{pgfscope}%
\pgfsetbuttcap%
\pgfsetmiterjoin%
\definecolor{currentfill}{rgb}{1.000000,1.000000,1.000000}%
\pgfsetfillcolor{currentfill}%
\pgfsetfillopacity{0.800000}%
\pgfsetlinewidth{1.003750pt}%
\definecolor{currentstroke}{rgb}{0.800000,0.800000,0.800000}%
\pgfsetstrokecolor{currentstroke}%
\pgfsetstrokeopacity{0.800000}%
\pgfsetdash{}{0pt}%
\pgfpathmoveto{\pgfqpoint{4.300417in}{0.618213in}}%
\pgfpathlineto{\pgfqpoint{5.752778in}{0.618213in}}%
\pgfpathquadraticcurveto{\pgfqpoint{5.780556in}{0.618213in}}{\pgfqpoint{5.780556in}{0.645991in}}%
\pgfpathlineto{\pgfqpoint{5.780556in}{1.214463in}}%
\pgfpathquadraticcurveto{\pgfqpoint{5.780556in}{1.242241in}}{\pgfqpoint{5.752778in}{1.242241in}}%
\pgfpathlineto{\pgfqpoint{4.300417in}{1.242241in}}%
\pgfpathquadraticcurveto{\pgfqpoint{4.272639in}{1.242241in}}{\pgfqpoint{4.272639in}{1.214463in}}%
\pgfpathlineto{\pgfqpoint{4.272639in}{0.645991in}}%
\pgfpathquadraticcurveto{\pgfqpoint{4.272639in}{0.618213in}}{\pgfqpoint{4.300417in}{0.618213in}}%
\pgfpathclose%
\pgfusepath{stroke,fill}%
\end{pgfscope}%
\begin{pgfscope}%
\pgfsetbuttcap%
\pgfsetroundjoin%
\definecolor{currentfill}{rgb}{0.121569,0.466667,0.705882}%
\pgfsetfillcolor{currentfill}%
\pgfsetlinewidth{1.003750pt}%
\definecolor{currentstroke}{rgb}{0.121569,0.466667,0.705882}%
\pgfsetstrokecolor{currentstroke}%
\pgfsetdash{}{0pt}%
\pgfsys@defobject{currentmarker}{\pgfqpoint{-0.034722in}{-0.034722in}}{\pgfqpoint{0.034722in}{0.034722in}}{%
\pgfpathmoveto{\pgfqpoint{0.000000in}{-0.034722in}}%
\pgfpathcurveto{\pgfqpoint{0.009208in}{-0.034722in}}{\pgfqpoint{0.018041in}{-0.031064in}}{\pgfqpoint{0.024552in}{-0.024552in}}%
\pgfpathcurveto{\pgfqpoint{0.031064in}{-0.018041in}}{\pgfqpoint{0.034722in}{-0.009208in}}{\pgfqpoint{0.034722in}{0.000000in}}%
\pgfpathcurveto{\pgfqpoint{0.034722in}{0.009208in}}{\pgfqpoint{0.031064in}{0.018041in}}{\pgfqpoint{0.024552in}{0.024552in}}%
\pgfpathcurveto{\pgfqpoint{0.018041in}{0.031064in}}{\pgfqpoint{0.009208in}{0.034722in}}{\pgfqpoint{0.000000in}{0.034722in}}%
\pgfpathcurveto{\pgfqpoint{-0.009208in}{0.034722in}}{\pgfqpoint{-0.018041in}{0.031064in}}{\pgfqpoint{-0.024552in}{0.024552in}}%
\pgfpathcurveto{\pgfqpoint{-0.031064in}{0.018041in}}{\pgfqpoint{-0.034722in}{0.009208in}}{\pgfqpoint{-0.034722in}{0.000000in}}%
\pgfpathcurveto{\pgfqpoint{-0.034722in}{-0.009208in}}{\pgfqpoint{-0.031064in}{-0.018041in}}{\pgfqpoint{-0.024552in}{-0.024552in}}%
\pgfpathcurveto{\pgfqpoint{-0.018041in}{-0.031064in}}{\pgfqpoint{-0.009208in}{-0.034722in}}{\pgfqpoint{0.000000in}{-0.034722in}}%
\pgfpathclose%
\pgfusepath{stroke,fill}%
}%
\begin{pgfscope}%
\pgfsys@transformshift{4.467083in}{1.138074in}%
\pgfsys@useobject{currentmarker}{}%
\end{pgfscope}%
\end{pgfscope}%
\begin{pgfscope}%
\definecolor{textcolor}{rgb}{0.000000,0.000000,0.000000}%
\pgfsetstrokecolor{textcolor}%
\pgfsetfillcolor{textcolor}%
\pgftext[x=4.717083in,y=1.089463in,left,base]{\color{textcolor}\sffamily\fontsize{10.000000}{12.000000}\selectfont No Timeout}%
\end{pgfscope}%
\begin{pgfscope}%
\pgfsetbuttcap%
\pgfsetroundjoin%
\definecolor{currentfill}{rgb}{1.000000,0.498039,0.054902}%
\pgfsetfillcolor{currentfill}%
\pgfsetlinewidth{1.003750pt}%
\definecolor{currentstroke}{rgb}{1.000000,0.498039,0.054902}%
\pgfsetstrokecolor{currentstroke}%
\pgfsetdash{}{0pt}%
\pgfsys@defobject{currentmarker}{\pgfqpoint{-0.034722in}{-0.034722in}}{\pgfqpoint{0.034722in}{0.034722in}}{%
\pgfpathmoveto{\pgfqpoint{0.000000in}{-0.034722in}}%
\pgfpathcurveto{\pgfqpoint{0.009208in}{-0.034722in}}{\pgfqpoint{0.018041in}{-0.031064in}}{\pgfqpoint{0.024552in}{-0.024552in}}%
\pgfpathcurveto{\pgfqpoint{0.031064in}{-0.018041in}}{\pgfqpoint{0.034722in}{-0.009208in}}{\pgfqpoint{0.034722in}{0.000000in}}%
\pgfpathcurveto{\pgfqpoint{0.034722in}{0.009208in}}{\pgfqpoint{0.031064in}{0.018041in}}{\pgfqpoint{0.024552in}{0.024552in}}%
\pgfpathcurveto{\pgfqpoint{0.018041in}{0.031064in}}{\pgfqpoint{0.009208in}{0.034722in}}{\pgfqpoint{0.000000in}{0.034722in}}%
\pgfpathcurveto{\pgfqpoint{-0.009208in}{0.034722in}}{\pgfqpoint{-0.018041in}{0.031064in}}{\pgfqpoint{-0.024552in}{0.024552in}}%
\pgfpathcurveto{\pgfqpoint{-0.031064in}{0.018041in}}{\pgfqpoint{-0.034722in}{0.009208in}}{\pgfqpoint{-0.034722in}{0.000000in}}%
\pgfpathcurveto{\pgfqpoint{-0.034722in}{-0.009208in}}{\pgfqpoint{-0.031064in}{-0.018041in}}{\pgfqpoint{-0.024552in}{-0.024552in}}%
\pgfpathcurveto{\pgfqpoint{-0.018041in}{-0.031064in}}{\pgfqpoint{-0.009208in}{-0.034722in}}{\pgfqpoint{0.000000in}{-0.034722in}}%
\pgfpathclose%
\pgfusepath{stroke,fill}%
}%
\begin{pgfscope}%
\pgfsys@transformshift{4.467083in}{0.944463in}%
\pgfsys@useobject{currentmarker}{}%
\end{pgfscope}%
\end{pgfscope}%
\begin{pgfscope}%
\definecolor{textcolor}{rgb}{0.000000,0.000000,0.000000}%
\pgfsetstrokecolor{textcolor}%
\pgfsetfillcolor{textcolor}%
\pgftext[x=4.717083in,y=0.895852in,left,base]{\color{textcolor}\sffamily\fontsize{10.000000}{12.000000}\selectfont Time Timeout}%
\end{pgfscope}%
\begin{pgfscope}%
\pgfsetbuttcap%
\pgfsetroundjoin%
\definecolor{currentfill}{rgb}{0.839216,0.152941,0.156863}%
\pgfsetfillcolor{currentfill}%
\pgfsetlinewidth{1.003750pt}%
\definecolor{currentstroke}{rgb}{0.839216,0.152941,0.156863}%
\pgfsetstrokecolor{currentstroke}%
\pgfsetdash{}{0pt}%
\pgfsys@defobject{currentmarker}{\pgfqpoint{-0.034722in}{-0.034722in}}{\pgfqpoint{0.034722in}{0.034722in}}{%
\pgfpathmoveto{\pgfqpoint{0.000000in}{-0.034722in}}%
\pgfpathcurveto{\pgfqpoint{0.009208in}{-0.034722in}}{\pgfqpoint{0.018041in}{-0.031064in}}{\pgfqpoint{0.024552in}{-0.024552in}}%
\pgfpathcurveto{\pgfqpoint{0.031064in}{-0.018041in}}{\pgfqpoint{0.034722in}{-0.009208in}}{\pgfqpoint{0.034722in}{0.000000in}}%
\pgfpathcurveto{\pgfqpoint{0.034722in}{0.009208in}}{\pgfqpoint{0.031064in}{0.018041in}}{\pgfqpoint{0.024552in}{0.024552in}}%
\pgfpathcurveto{\pgfqpoint{0.018041in}{0.031064in}}{\pgfqpoint{0.009208in}{0.034722in}}{\pgfqpoint{0.000000in}{0.034722in}}%
\pgfpathcurveto{\pgfqpoint{-0.009208in}{0.034722in}}{\pgfqpoint{-0.018041in}{0.031064in}}{\pgfqpoint{-0.024552in}{0.024552in}}%
\pgfpathcurveto{\pgfqpoint{-0.031064in}{0.018041in}}{\pgfqpoint{-0.034722in}{0.009208in}}{\pgfqpoint{-0.034722in}{0.000000in}}%
\pgfpathcurveto{\pgfqpoint{-0.034722in}{-0.009208in}}{\pgfqpoint{-0.031064in}{-0.018041in}}{\pgfqpoint{-0.024552in}{-0.024552in}}%
\pgfpathcurveto{\pgfqpoint{-0.018041in}{-0.031064in}}{\pgfqpoint{-0.009208in}{-0.034722in}}{\pgfqpoint{0.000000in}{-0.034722in}}%
\pgfpathclose%
\pgfusepath{stroke,fill}%
}%
\begin{pgfscope}%
\pgfsys@transformshift{4.467083in}{0.750852in}%
\pgfsys@useobject{currentmarker}{}%
\end{pgfscope}%
\end{pgfscope}%
\begin{pgfscope}%
\definecolor{textcolor}{rgb}{0.000000,0.000000,0.000000}%
\pgfsetstrokecolor{textcolor}%
\pgfsetfillcolor{textcolor}%
\pgftext[x=4.717083in,y=0.702241in,left,base]{\color{textcolor}\sffamily\fontsize{10.000000}{12.000000}\selectfont Memory Timeout}%
\end{pgfscope}%
\end{pgfpicture}%
\makeatother%
\endgroup%

                }
            \end{subfigure}
            \caption{Sink Count}
            \label{f:dfsinks}
        \end{subfigure}
        \bigbreak
        \begin{subfigure}[b]{\textwidth}
            \centering
            \begin{subfigure}[]{0.45\textwidth}
                \centering
                \resizebox{\columnwidth}{!}{
                    %% Creator: Matplotlib, PGF backend
%%
%% To include the figure in your LaTeX document, write
%%   \input{<filename>.pgf}
%%
%% Make sure the required packages are loaded in your preamble
%%   \usepackage{pgf}
%%
%% and, on pdftex
%%   \usepackage[utf8]{inputenc}\DeclareUnicodeCharacter{2212}{-}
%%
%% or, on luatex and xetex
%%   \usepackage{unicode-math}
%%
%% Figures using additional raster images can only be included by \input if
%% they are in the same directory as the main LaTeX file. For loading figures
%% from other directories you can use the `import` package
%%   \usepackage{import}
%%
%% and then include the figures with
%%   \import{<path to file>}{<filename>.pgf}
%%
%% Matplotlib used the following preamble
%%   \usepackage{amsmath}
%%   \usepackage{fontspec}
%%
\begingroup%
\makeatletter%
\begin{pgfpicture}%
\pgfpathrectangle{\pgfpointorigin}{\pgfqpoint{6.000000in}{4.000000in}}%
\pgfusepath{use as bounding box, clip}%
\begin{pgfscope}%
\pgfsetbuttcap%
\pgfsetmiterjoin%
\definecolor{currentfill}{rgb}{1.000000,1.000000,1.000000}%
\pgfsetfillcolor{currentfill}%
\pgfsetlinewidth{0.000000pt}%
\definecolor{currentstroke}{rgb}{1.000000,1.000000,1.000000}%
\pgfsetstrokecolor{currentstroke}%
\pgfsetdash{}{0pt}%
\pgfpathmoveto{\pgfqpoint{0.000000in}{0.000000in}}%
\pgfpathlineto{\pgfqpoint{6.000000in}{0.000000in}}%
\pgfpathlineto{\pgfqpoint{6.000000in}{4.000000in}}%
\pgfpathlineto{\pgfqpoint{0.000000in}{4.000000in}}%
\pgfpathclose%
\pgfusepath{fill}%
\end{pgfscope}%
\begin{pgfscope}%
\pgfsetbuttcap%
\pgfsetmiterjoin%
\definecolor{currentfill}{rgb}{1.000000,1.000000,1.000000}%
\pgfsetfillcolor{currentfill}%
\pgfsetlinewidth{0.000000pt}%
\definecolor{currentstroke}{rgb}{0.000000,0.000000,0.000000}%
\pgfsetstrokecolor{currentstroke}%
\pgfsetstrokeopacity{0.000000}%
\pgfsetdash{}{0pt}%
\pgfpathmoveto{\pgfqpoint{0.648703in}{0.548769in}}%
\pgfpathlineto{\pgfqpoint{5.850000in}{0.548769in}}%
\pgfpathlineto{\pgfqpoint{5.850000in}{3.651359in}}%
\pgfpathlineto{\pgfqpoint{0.648703in}{3.651359in}}%
\pgfpathclose%
\pgfusepath{fill}%
\end{pgfscope}%
\begin{pgfscope}%
\pgfpathrectangle{\pgfqpoint{0.648703in}{0.548769in}}{\pgfqpoint{5.201297in}{3.102590in}}%
\pgfusepath{clip}%
\pgfsetbuttcap%
\pgfsetroundjoin%
\definecolor{currentfill}{rgb}{0.121569,0.466667,0.705882}%
\pgfsetfillcolor{currentfill}%
\pgfsetlinewidth{1.003750pt}%
\definecolor{currentstroke}{rgb}{0.121569,0.466667,0.705882}%
\pgfsetstrokecolor{currentstroke}%
\pgfsetdash{}{0pt}%
\pgfpathmoveto{\pgfqpoint{3.289423in}{0.648129in}}%
\pgfpathcurveto{\pgfqpoint{3.300473in}{0.648129in}}{\pgfqpoint{3.311072in}{0.652519in}}{\pgfqpoint{3.318886in}{0.660333in}}%
\pgfpathcurveto{\pgfqpoint{3.326700in}{0.668146in}}{\pgfqpoint{3.331090in}{0.678745in}}{\pgfqpoint{3.331090in}{0.689796in}}%
\pgfpathcurveto{\pgfqpoint{3.331090in}{0.700846in}}{\pgfqpoint{3.326700in}{0.711445in}}{\pgfqpoint{3.318886in}{0.719258in}}%
\pgfpathcurveto{\pgfqpoint{3.311072in}{0.727072in}}{\pgfqpoint{3.300473in}{0.731462in}}{\pgfqpoint{3.289423in}{0.731462in}}%
\pgfpathcurveto{\pgfqpoint{3.278373in}{0.731462in}}{\pgfqpoint{3.267774in}{0.727072in}}{\pgfqpoint{3.259961in}{0.719258in}}%
\pgfpathcurveto{\pgfqpoint{3.252147in}{0.711445in}}{\pgfqpoint{3.247757in}{0.700846in}}{\pgfqpoint{3.247757in}{0.689796in}}%
\pgfpathcurveto{\pgfqpoint{3.247757in}{0.678745in}}{\pgfqpoint{3.252147in}{0.668146in}}{\pgfqpoint{3.259961in}{0.660333in}}%
\pgfpathcurveto{\pgfqpoint{3.267774in}{0.652519in}}{\pgfqpoint{3.278373in}{0.648129in}}{\pgfqpoint{3.289423in}{0.648129in}}%
\pgfpathclose%
\pgfusepath{stroke,fill}%
\end{pgfscope}%
\begin{pgfscope}%
\pgfpathrectangle{\pgfqpoint{0.648703in}{0.548769in}}{\pgfqpoint{5.201297in}{3.102590in}}%
\pgfusepath{clip}%
\pgfsetbuttcap%
\pgfsetroundjoin%
\definecolor{currentfill}{rgb}{0.121569,0.466667,0.705882}%
\pgfsetfillcolor{currentfill}%
\pgfsetlinewidth{1.003750pt}%
\definecolor{currentstroke}{rgb}{0.121569,0.466667,0.705882}%
\pgfsetstrokecolor{currentstroke}%
\pgfsetdash{}{0pt}%
\pgfpathmoveto{\pgfqpoint{3.089065in}{3.124394in}}%
\pgfpathcurveto{\pgfqpoint{3.100115in}{3.124394in}}{\pgfqpoint{3.110714in}{3.128784in}}{\pgfqpoint{3.118528in}{3.136598in}}%
\pgfpathcurveto{\pgfqpoint{3.126342in}{3.144411in}}{\pgfqpoint{3.130732in}{3.155010in}}{\pgfqpoint{3.130732in}{3.166060in}}%
\pgfpathcurveto{\pgfqpoint{3.130732in}{3.177111in}}{\pgfqpoint{3.126342in}{3.187710in}}{\pgfqpoint{3.118528in}{3.195523in}}%
\pgfpathcurveto{\pgfqpoint{3.110714in}{3.203337in}}{\pgfqpoint{3.100115in}{3.207727in}}{\pgfqpoint{3.089065in}{3.207727in}}%
\pgfpathcurveto{\pgfqpoint{3.078015in}{3.207727in}}{\pgfqpoint{3.067416in}{3.203337in}}{\pgfqpoint{3.059602in}{3.195523in}}%
\pgfpathcurveto{\pgfqpoint{3.051789in}{3.187710in}}{\pgfqpoint{3.047399in}{3.177111in}}{\pgfqpoint{3.047399in}{3.166060in}}%
\pgfpathcurveto{\pgfqpoint{3.047399in}{3.155010in}}{\pgfqpoint{3.051789in}{3.144411in}}{\pgfqpoint{3.059602in}{3.136598in}}%
\pgfpathcurveto{\pgfqpoint{3.067416in}{3.128784in}}{\pgfqpoint{3.078015in}{3.124394in}}{\pgfqpoint{3.089065in}{3.124394in}}%
\pgfpathclose%
\pgfusepath{stroke,fill}%
\end{pgfscope}%
\begin{pgfscope}%
\pgfpathrectangle{\pgfqpoint{0.648703in}{0.548769in}}{\pgfqpoint{5.201297in}{3.102590in}}%
\pgfusepath{clip}%
\pgfsetbuttcap%
\pgfsetroundjoin%
\definecolor{currentfill}{rgb}{1.000000,0.498039,0.054902}%
\pgfsetfillcolor{currentfill}%
\pgfsetlinewidth{1.003750pt}%
\definecolor{currentstroke}{rgb}{1.000000,0.498039,0.054902}%
\pgfsetstrokecolor{currentstroke}%
\pgfsetdash{}{0pt}%
\pgfpathmoveto{\pgfqpoint{3.129137in}{3.140985in}}%
\pgfpathcurveto{\pgfqpoint{3.140187in}{3.140985in}}{\pgfqpoint{3.150786in}{3.145375in}}{\pgfqpoint{3.158600in}{3.153189in}}%
\pgfpathcurveto{\pgfqpoint{3.166413in}{3.161003in}}{\pgfqpoint{3.170804in}{3.171602in}}{\pgfqpoint{3.170804in}{3.182652in}}%
\pgfpathcurveto{\pgfqpoint{3.170804in}{3.193702in}}{\pgfqpoint{3.166413in}{3.204301in}}{\pgfqpoint{3.158600in}{3.212115in}}%
\pgfpathcurveto{\pgfqpoint{3.150786in}{3.219928in}}{\pgfqpoint{3.140187in}{3.224319in}}{\pgfqpoint{3.129137in}{3.224319in}}%
\pgfpathcurveto{\pgfqpoint{3.118087in}{3.224319in}}{\pgfqpoint{3.107488in}{3.219928in}}{\pgfqpoint{3.099674in}{3.212115in}}%
\pgfpathcurveto{\pgfqpoint{3.091860in}{3.204301in}}{\pgfqpoint{3.087470in}{3.193702in}}{\pgfqpoint{3.087470in}{3.182652in}}%
\pgfpathcurveto{\pgfqpoint{3.087470in}{3.171602in}}{\pgfqpoint{3.091860in}{3.161003in}}{\pgfqpoint{3.099674in}{3.153189in}}%
\pgfpathcurveto{\pgfqpoint{3.107488in}{3.145375in}}{\pgfqpoint{3.118087in}{3.140985in}}{\pgfqpoint{3.129137in}{3.140985in}}%
\pgfpathclose%
\pgfusepath{stroke,fill}%
\end{pgfscope}%
\begin{pgfscope}%
\pgfpathrectangle{\pgfqpoint{0.648703in}{0.548769in}}{\pgfqpoint{5.201297in}{3.102590in}}%
\pgfusepath{clip}%
\pgfsetbuttcap%
\pgfsetroundjoin%
\definecolor{currentfill}{rgb}{0.121569,0.466667,0.705882}%
\pgfsetfillcolor{currentfill}%
\pgfsetlinewidth{1.003750pt}%
\definecolor{currentstroke}{rgb}{0.121569,0.466667,0.705882}%
\pgfsetstrokecolor{currentstroke}%
\pgfsetdash{}{0pt}%
\pgfpathmoveto{\pgfqpoint{3.409638in}{3.132690in}}%
\pgfpathcurveto{\pgfqpoint{3.420688in}{3.132690in}}{\pgfqpoint{3.431287in}{3.137080in}}{\pgfqpoint{3.439101in}{3.144893in}}%
\pgfpathcurveto{\pgfqpoint{3.446915in}{3.152707in}}{\pgfqpoint{3.451305in}{3.163306in}}{\pgfqpoint{3.451305in}{3.174356in}}%
\pgfpathcurveto{\pgfqpoint{3.451305in}{3.185406in}}{\pgfqpoint{3.446915in}{3.196005in}}{\pgfqpoint{3.439101in}{3.203819in}}%
\pgfpathcurveto{\pgfqpoint{3.431287in}{3.211633in}}{\pgfqpoint{3.420688in}{3.216023in}}{\pgfqpoint{3.409638in}{3.216023in}}%
\pgfpathcurveto{\pgfqpoint{3.398588in}{3.216023in}}{\pgfqpoint{3.387989in}{3.211633in}}{\pgfqpoint{3.380175in}{3.203819in}}%
\pgfpathcurveto{\pgfqpoint{3.372362in}{3.196005in}}{\pgfqpoint{3.367972in}{3.185406in}}{\pgfqpoint{3.367972in}{3.174356in}}%
\pgfpathcurveto{\pgfqpoint{3.367972in}{3.163306in}}{\pgfqpoint{3.372362in}{3.152707in}}{\pgfqpoint{3.380175in}{3.144893in}}%
\pgfpathcurveto{\pgfqpoint{3.387989in}{3.137080in}}{\pgfqpoint{3.398588in}{3.132690in}}{\pgfqpoint{3.409638in}{3.132690in}}%
\pgfpathclose%
\pgfusepath{stroke,fill}%
\end{pgfscope}%
\begin{pgfscope}%
\pgfpathrectangle{\pgfqpoint{0.648703in}{0.548769in}}{\pgfqpoint{5.201297in}{3.102590in}}%
\pgfusepath{clip}%
\pgfsetbuttcap%
\pgfsetroundjoin%
\definecolor{currentfill}{rgb}{1.000000,0.498039,0.054902}%
\pgfsetfillcolor{currentfill}%
\pgfsetlinewidth{1.003750pt}%
\definecolor{currentstroke}{rgb}{1.000000,0.498039,0.054902}%
\pgfsetstrokecolor{currentstroke}%
\pgfsetdash{}{0pt}%
\pgfpathmoveto{\pgfqpoint{3.008922in}{3.136837in}}%
\pgfpathcurveto{\pgfqpoint{3.019972in}{3.136837in}}{\pgfqpoint{3.030571in}{3.141228in}}{\pgfqpoint{3.038385in}{3.149041in}}%
\pgfpathcurveto{\pgfqpoint{3.046198in}{3.156855in}}{\pgfqpoint{3.050589in}{3.167454in}}{\pgfqpoint{3.050589in}{3.178504in}}%
\pgfpathcurveto{\pgfqpoint{3.050589in}{3.189554in}}{\pgfqpoint{3.046198in}{3.200153in}}{\pgfqpoint{3.038385in}{3.207967in}}%
\pgfpathcurveto{\pgfqpoint{3.030571in}{3.215780in}}{\pgfqpoint{3.019972in}{3.220171in}}{\pgfqpoint{3.008922in}{3.220171in}}%
\pgfpathcurveto{\pgfqpoint{2.997872in}{3.220171in}}{\pgfqpoint{2.987273in}{3.215780in}}{\pgfqpoint{2.979459in}{3.207967in}}%
\pgfpathcurveto{\pgfqpoint{2.971646in}{3.200153in}}{\pgfqpoint{2.967255in}{3.189554in}}{\pgfqpoint{2.967255in}{3.178504in}}%
\pgfpathcurveto{\pgfqpoint{2.967255in}{3.167454in}}{\pgfqpoint{2.971646in}{3.156855in}}{\pgfqpoint{2.979459in}{3.149041in}}%
\pgfpathcurveto{\pgfqpoint{2.987273in}{3.141228in}}{\pgfqpoint{2.997872in}{3.136837in}}{\pgfqpoint{3.008922in}{3.136837in}}%
\pgfpathclose%
\pgfusepath{stroke,fill}%
\end{pgfscope}%
\begin{pgfscope}%
\pgfpathrectangle{\pgfqpoint{0.648703in}{0.548769in}}{\pgfqpoint{5.201297in}{3.102590in}}%
\pgfusepath{clip}%
\pgfsetbuttcap%
\pgfsetroundjoin%
\definecolor{currentfill}{rgb}{0.121569,0.466667,0.705882}%
\pgfsetfillcolor{currentfill}%
\pgfsetlinewidth{1.003750pt}%
\definecolor{currentstroke}{rgb}{0.121569,0.466667,0.705882}%
\pgfsetstrokecolor{currentstroke}%
\pgfsetdash{}{0pt}%
\pgfpathmoveto{\pgfqpoint{4.571715in}{3.132690in}}%
\pgfpathcurveto{\pgfqpoint{4.582765in}{3.132690in}}{\pgfqpoint{4.593364in}{3.137080in}}{\pgfqpoint{4.601178in}{3.144893in}}%
\pgfpathcurveto{\pgfqpoint{4.608992in}{3.152707in}}{\pgfqpoint{4.613382in}{3.163306in}}{\pgfqpoint{4.613382in}{3.174356in}}%
\pgfpathcurveto{\pgfqpoint{4.613382in}{3.185406in}}{\pgfqpoint{4.608992in}{3.196005in}}{\pgfqpoint{4.601178in}{3.203819in}}%
\pgfpathcurveto{\pgfqpoint{4.593364in}{3.211633in}}{\pgfqpoint{4.582765in}{3.216023in}}{\pgfqpoint{4.571715in}{3.216023in}}%
\pgfpathcurveto{\pgfqpoint{4.560665in}{3.216023in}}{\pgfqpoint{4.550066in}{3.211633in}}{\pgfqpoint{4.542252in}{3.203819in}}%
\pgfpathcurveto{\pgfqpoint{4.534439in}{3.196005in}}{\pgfqpoint{4.530049in}{3.185406in}}{\pgfqpoint{4.530049in}{3.174356in}}%
\pgfpathcurveto{\pgfqpoint{4.530049in}{3.163306in}}{\pgfqpoint{4.534439in}{3.152707in}}{\pgfqpoint{4.542252in}{3.144893in}}%
\pgfpathcurveto{\pgfqpoint{4.550066in}{3.137080in}}{\pgfqpoint{4.560665in}{3.132690in}}{\pgfqpoint{4.571715in}{3.132690in}}%
\pgfpathclose%
\pgfusepath{stroke,fill}%
\end{pgfscope}%
\begin{pgfscope}%
\pgfpathrectangle{\pgfqpoint{0.648703in}{0.548769in}}{\pgfqpoint{5.201297in}{3.102590in}}%
\pgfusepath{clip}%
\pgfsetbuttcap%
\pgfsetroundjoin%
\definecolor{currentfill}{rgb}{0.121569,0.466667,0.705882}%
\pgfsetfillcolor{currentfill}%
\pgfsetlinewidth{1.003750pt}%
\definecolor{currentstroke}{rgb}{0.121569,0.466667,0.705882}%
\pgfsetstrokecolor{currentstroke}%
\pgfsetdash{}{0pt}%
\pgfpathmoveto{\pgfqpoint{3.449710in}{3.128542in}}%
\pgfpathcurveto{\pgfqpoint{3.460760in}{3.128542in}}{\pgfqpoint{3.471359in}{3.132932in}}{\pgfqpoint{3.479173in}{3.140746in}}%
\pgfpathcurveto{\pgfqpoint{3.486986in}{3.148559in}}{\pgfqpoint{3.491376in}{3.159158in}}{\pgfqpoint{3.491376in}{3.170208in}}%
\pgfpathcurveto{\pgfqpoint{3.491376in}{3.181258in}}{\pgfqpoint{3.486986in}{3.191857in}}{\pgfqpoint{3.479173in}{3.199671in}}%
\pgfpathcurveto{\pgfqpoint{3.471359in}{3.207485in}}{\pgfqpoint{3.460760in}{3.211875in}}{\pgfqpoint{3.449710in}{3.211875in}}%
\pgfpathcurveto{\pgfqpoint{3.438660in}{3.211875in}}{\pgfqpoint{3.428061in}{3.207485in}}{\pgfqpoint{3.420247in}{3.199671in}}%
\pgfpathcurveto{\pgfqpoint{3.412433in}{3.191857in}}{\pgfqpoint{3.408043in}{3.181258in}}{\pgfqpoint{3.408043in}{3.170208in}}%
\pgfpathcurveto{\pgfqpoint{3.408043in}{3.159158in}}{\pgfqpoint{3.412433in}{3.148559in}}{\pgfqpoint{3.420247in}{3.140746in}}%
\pgfpathcurveto{\pgfqpoint{3.428061in}{3.132932in}}{\pgfqpoint{3.438660in}{3.128542in}}{\pgfqpoint{3.449710in}{3.128542in}}%
\pgfpathclose%
\pgfusepath{stroke,fill}%
\end{pgfscope}%
\begin{pgfscope}%
\pgfpathrectangle{\pgfqpoint{0.648703in}{0.548769in}}{\pgfqpoint{5.201297in}{3.102590in}}%
\pgfusepath{clip}%
\pgfsetbuttcap%
\pgfsetroundjoin%
\definecolor{currentfill}{rgb}{1.000000,0.498039,0.054902}%
\pgfsetfillcolor{currentfill}%
\pgfsetlinewidth{1.003750pt}%
\definecolor{currentstroke}{rgb}{1.000000,0.498039,0.054902}%
\pgfsetstrokecolor{currentstroke}%
\pgfsetdash{}{0pt}%
\pgfpathmoveto{\pgfqpoint{3.690140in}{3.149281in}}%
\pgfpathcurveto{\pgfqpoint{3.701190in}{3.149281in}}{\pgfqpoint{3.711789in}{3.153671in}}{\pgfqpoint{3.719602in}{3.161485in}}%
\pgfpathcurveto{\pgfqpoint{3.727416in}{3.169298in}}{\pgfqpoint{3.731806in}{3.179897in}}{\pgfqpoint{3.731806in}{3.190948in}}%
\pgfpathcurveto{\pgfqpoint{3.731806in}{3.201998in}}{\pgfqpoint{3.727416in}{3.212597in}}{\pgfqpoint{3.719602in}{3.220410in}}%
\pgfpathcurveto{\pgfqpoint{3.711789in}{3.228224in}}{\pgfqpoint{3.701190in}{3.232614in}}{\pgfqpoint{3.690140in}{3.232614in}}%
\pgfpathcurveto{\pgfqpoint{3.679089in}{3.232614in}}{\pgfqpoint{3.668490in}{3.228224in}}{\pgfqpoint{3.660677in}{3.220410in}}%
\pgfpathcurveto{\pgfqpoint{3.652863in}{3.212597in}}{\pgfqpoint{3.648473in}{3.201998in}}{\pgfqpoint{3.648473in}{3.190948in}}%
\pgfpathcurveto{\pgfqpoint{3.648473in}{3.179897in}}{\pgfqpoint{3.652863in}{3.169298in}}{\pgfqpoint{3.660677in}{3.161485in}}%
\pgfpathcurveto{\pgfqpoint{3.668490in}{3.153671in}}{\pgfqpoint{3.679089in}{3.149281in}}{\pgfqpoint{3.690140in}{3.149281in}}%
\pgfpathclose%
\pgfusepath{stroke,fill}%
\end{pgfscope}%
\begin{pgfscope}%
\pgfpathrectangle{\pgfqpoint{0.648703in}{0.548769in}}{\pgfqpoint{5.201297in}{3.102590in}}%
\pgfusepath{clip}%
\pgfsetbuttcap%
\pgfsetroundjoin%
\definecolor{currentfill}{rgb}{1.000000,0.498039,0.054902}%
\pgfsetfillcolor{currentfill}%
\pgfsetlinewidth{1.003750pt}%
\definecolor{currentstroke}{rgb}{1.000000,0.498039,0.054902}%
\pgfsetstrokecolor{currentstroke}%
\pgfsetdash{}{0pt}%
\pgfpathmoveto{\pgfqpoint{3.409638in}{3.240534in}}%
\pgfpathcurveto{\pgfqpoint{3.420688in}{3.240534in}}{\pgfqpoint{3.431287in}{3.244924in}}{\pgfqpoint{3.439101in}{3.252737in}}%
\pgfpathcurveto{\pgfqpoint{3.446915in}{3.260551in}}{\pgfqpoint{3.451305in}{3.271150in}}{\pgfqpoint{3.451305in}{3.282200in}}%
\pgfpathcurveto{\pgfqpoint{3.451305in}{3.293250in}}{\pgfqpoint{3.446915in}{3.303849in}}{\pgfqpoint{3.439101in}{3.311663in}}%
\pgfpathcurveto{\pgfqpoint{3.431287in}{3.319477in}}{\pgfqpoint{3.420688in}{3.323867in}}{\pgfqpoint{3.409638in}{3.323867in}}%
\pgfpathcurveto{\pgfqpoint{3.398588in}{3.323867in}}{\pgfqpoint{3.387989in}{3.319477in}}{\pgfqpoint{3.380175in}{3.311663in}}%
\pgfpathcurveto{\pgfqpoint{3.372362in}{3.303849in}}{\pgfqpoint{3.367972in}{3.293250in}}{\pgfqpoint{3.367972in}{3.282200in}}%
\pgfpathcurveto{\pgfqpoint{3.367972in}{3.271150in}}{\pgfqpoint{3.372362in}{3.260551in}}{\pgfqpoint{3.380175in}{3.252737in}}%
\pgfpathcurveto{\pgfqpoint{3.387989in}{3.244924in}}{\pgfqpoint{3.398588in}{3.240534in}}{\pgfqpoint{3.409638in}{3.240534in}}%
\pgfpathclose%
\pgfusepath{stroke,fill}%
\end{pgfscope}%
\begin{pgfscope}%
\pgfpathrectangle{\pgfqpoint{0.648703in}{0.548769in}}{\pgfqpoint{5.201297in}{3.102590in}}%
\pgfusepath{clip}%
\pgfsetbuttcap%
\pgfsetroundjoin%
\definecolor{currentfill}{rgb}{0.121569,0.466667,0.705882}%
\pgfsetfillcolor{currentfill}%
\pgfsetlinewidth{1.003750pt}%
\definecolor{currentstroke}{rgb}{0.121569,0.466667,0.705882}%
\pgfsetstrokecolor{currentstroke}%
\pgfsetdash{}{0pt}%
\pgfpathmoveto{\pgfqpoint{3.569925in}{0.664720in}}%
\pgfpathcurveto{\pgfqpoint{3.580975in}{0.664720in}}{\pgfqpoint{3.591574in}{0.669111in}}{\pgfqpoint{3.599387in}{0.676924in}}%
\pgfpathcurveto{\pgfqpoint{3.607201in}{0.684738in}}{\pgfqpoint{3.611591in}{0.695337in}}{\pgfqpoint{3.611591in}{0.706387in}}%
\pgfpathcurveto{\pgfqpoint{3.611591in}{0.717437in}}{\pgfqpoint{3.607201in}{0.728036in}}{\pgfqpoint{3.599387in}{0.735850in}}%
\pgfpathcurveto{\pgfqpoint{3.591574in}{0.743663in}}{\pgfqpoint{3.580975in}{0.748054in}}{\pgfqpoint{3.569925in}{0.748054in}}%
\pgfpathcurveto{\pgfqpoint{3.558875in}{0.748054in}}{\pgfqpoint{3.548276in}{0.743663in}}{\pgfqpoint{3.540462in}{0.735850in}}%
\pgfpathcurveto{\pgfqpoint{3.532648in}{0.728036in}}{\pgfqpoint{3.528258in}{0.717437in}}{\pgfqpoint{3.528258in}{0.706387in}}%
\pgfpathcurveto{\pgfqpoint{3.528258in}{0.695337in}}{\pgfqpoint{3.532648in}{0.684738in}}{\pgfqpoint{3.540462in}{0.676924in}}%
\pgfpathcurveto{\pgfqpoint{3.548276in}{0.669111in}}{\pgfqpoint{3.558875in}{0.664720in}}{\pgfqpoint{3.569925in}{0.664720in}}%
\pgfpathclose%
\pgfusepath{stroke,fill}%
\end{pgfscope}%
\begin{pgfscope}%
\pgfpathrectangle{\pgfqpoint{0.648703in}{0.548769in}}{\pgfqpoint{5.201297in}{3.102590in}}%
\pgfusepath{clip}%
\pgfsetbuttcap%
\pgfsetroundjoin%
\definecolor{currentfill}{rgb}{1.000000,0.498039,0.054902}%
\pgfsetfillcolor{currentfill}%
\pgfsetlinewidth{1.003750pt}%
\definecolor{currentstroke}{rgb}{1.000000,0.498039,0.054902}%
\pgfsetstrokecolor{currentstroke}%
\pgfsetdash{}{0pt}%
\pgfpathmoveto{\pgfqpoint{3.209280in}{3.157577in}}%
\pgfpathcurveto{\pgfqpoint{3.220330in}{3.157577in}}{\pgfqpoint{3.230929in}{3.161967in}}{\pgfqpoint{3.238743in}{3.169780in}}%
\pgfpathcurveto{\pgfqpoint{3.246556in}{3.177594in}}{\pgfqpoint{3.250947in}{3.188193in}}{\pgfqpoint{3.250947in}{3.199243in}}%
\pgfpathcurveto{\pgfqpoint{3.250947in}{3.210293in}}{\pgfqpoint{3.246556in}{3.220892in}}{\pgfqpoint{3.238743in}{3.228706in}}%
\pgfpathcurveto{\pgfqpoint{3.230929in}{3.236520in}}{\pgfqpoint{3.220330in}{3.240910in}}{\pgfqpoint{3.209280in}{3.240910in}}%
\pgfpathcurveto{\pgfqpoint{3.198230in}{3.240910in}}{\pgfqpoint{3.187631in}{3.236520in}}{\pgfqpoint{3.179817in}{3.228706in}}%
\pgfpathcurveto{\pgfqpoint{3.172004in}{3.220892in}}{\pgfqpoint{3.167613in}{3.210293in}}{\pgfqpoint{3.167613in}{3.199243in}}%
\pgfpathcurveto{\pgfqpoint{3.167613in}{3.188193in}}{\pgfqpoint{3.172004in}{3.177594in}}{\pgfqpoint{3.179817in}{3.169780in}}%
\pgfpathcurveto{\pgfqpoint{3.187631in}{3.161967in}}{\pgfqpoint{3.198230in}{3.157577in}}{\pgfqpoint{3.209280in}{3.157577in}}%
\pgfpathclose%
\pgfusepath{stroke,fill}%
\end{pgfscope}%
\begin{pgfscope}%
\pgfpathrectangle{\pgfqpoint{0.648703in}{0.548769in}}{\pgfqpoint{5.201297in}{3.102590in}}%
\pgfusepath{clip}%
\pgfsetbuttcap%
\pgfsetroundjoin%
\definecolor{currentfill}{rgb}{1.000000,0.498039,0.054902}%
\pgfsetfillcolor{currentfill}%
\pgfsetlinewidth{1.003750pt}%
\definecolor{currentstroke}{rgb}{1.000000,0.498039,0.054902}%
\pgfsetstrokecolor{currentstroke}%
\pgfsetdash{}{0pt}%
\pgfpathmoveto{\pgfqpoint{2.568134in}{3.140985in}}%
\pgfpathcurveto{\pgfqpoint{2.579184in}{3.140985in}}{\pgfqpoint{2.589783in}{3.145375in}}{\pgfqpoint{2.597597in}{3.153189in}}%
\pgfpathcurveto{\pgfqpoint{2.605411in}{3.161003in}}{\pgfqpoint{2.609801in}{3.171602in}}{\pgfqpoint{2.609801in}{3.182652in}}%
\pgfpathcurveto{\pgfqpoint{2.609801in}{3.193702in}}{\pgfqpoint{2.605411in}{3.204301in}}{\pgfqpoint{2.597597in}{3.212115in}}%
\pgfpathcurveto{\pgfqpoint{2.589783in}{3.219928in}}{\pgfqpoint{2.579184in}{3.224319in}}{\pgfqpoint{2.568134in}{3.224319in}}%
\pgfpathcurveto{\pgfqpoint{2.557084in}{3.224319in}}{\pgfqpoint{2.546485in}{3.219928in}}{\pgfqpoint{2.538671in}{3.212115in}}%
\pgfpathcurveto{\pgfqpoint{2.530858in}{3.204301in}}{\pgfqpoint{2.526467in}{3.193702in}}{\pgfqpoint{2.526467in}{3.182652in}}%
\pgfpathcurveto{\pgfqpoint{2.526467in}{3.171602in}}{\pgfqpoint{2.530858in}{3.161003in}}{\pgfqpoint{2.538671in}{3.153189in}}%
\pgfpathcurveto{\pgfqpoint{2.546485in}{3.145375in}}{\pgfqpoint{2.557084in}{3.140985in}}{\pgfqpoint{2.568134in}{3.140985in}}%
\pgfpathclose%
\pgfusepath{stroke,fill}%
\end{pgfscope}%
\begin{pgfscope}%
\pgfpathrectangle{\pgfqpoint{0.648703in}{0.548769in}}{\pgfqpoint{5.201297in}{3.102590in}}%
\pgfusepath{clip}%
\pgfsetbuttcap%
\pgfsetroundjoin%
\definecolor{currentfill}{rgb}{1.000000,0.498039,0.054902}%
\pgfsetfillcolor{currentfill}%
\pgfsetlinewidth{1.003750pt}%
\definecolor{currentstroke}{rgb}{1.000000,0.498039,0.054902}%
\pgfsetstrokecolor{currentstroke}%
\pgfsetdash{}{0pt}%
\pgfpathmoveto{\pgfqpoint{3.369567in}{3.136837in}}%
\pgfpathcurveto{\pgfqpoint{3.380617in}{3.136837in}}{\pgfqpoint{3.391216in}{3.141228in}}{\pgfqpoint{3.399029in}{3.149041in}}%
\pgfpathcurveto{\pgfqpoint{3.406843in}{3.156855in}}{\pgfqpoint{3.411233in}{3.167454in}}{\pgfqpoint{3.411233in}{3.178504in}}%
\pgfpathcurveto{\pgfqpoint{3.411233in}{3.189554in}}{\pgfqpoint{3.406843in}{3.200153in}}{\pgfqpoint{3.399029in}{3.207967in}}%
\pgfpathcurveto{\pgfqpoint{3.391216in}{3.215780in}}{\pgfqpoint{3.380617in}{3.220171in}}{\pgfqpoint{3.369567in}{3.220171in}}%
\pgfpathcurveto{\pgfqpoint{3.358516in}{3.220171in}}{\pgfqpoint{3.347917in}{3.215780in}}{\pgfqpoint{3.340104in}{3.207967in}}%
\pgfpathcurveto{\pgfqpoint{3.332290in}{3.200153in}}{\pgfqpoint{3.327900in}{3.189554in}}{\pgfqpoint{3.327900in}{3.178504in}}%
\pgfpathcurveto{\pgfqpoint{3.327900in}{3.167454in}}{\pgfqpoint{3.332290in}{3.156855in}}{\pgfqpoint{3.340104in}{3.149041in}}%
\pgfpathcurveto{\pgfqpoint{3.347917in}{3.141228in}}{\pgfqpoint{3.358516in}{3.136837in}}{\pgfqpoint{3.369567in}{3.136837in}}%
\pgfpathclose%
\pgfusepath{stroke,fill}%
\end{pgfscope}%
\begin{pgfscope}%
\pgfpathrectangle{\pgfqpoint{0.648703in}{0.548769in}}{\pgfqpoint{5.201297in}{3.102590in}}%
\pgfusepath{clip}%
\pgfsetbuttcap%
\pgfsetroundjoin%
\definecolor{currentfill}{rgb}{1.000000,0.498039,0.054902}%
\pgfsetfillcolor{currentfill}%
\pgfsetlinewidth{1.003750pt}%
\definecolor{currentstroke}{rgb}{1.000000,0.498039,0.054902}%
\pgfsetstrokecolor{currentstroke}%
\pgfsetdash{}{0pt}%
\pgfpathmoveto{\pgfqpoint{3.249352in}{3.136837in}}%
\pgfpathcurveto{\pgfqpoint{3.260402in}{3.136837in}}{\pgfqpoint{3.271001in}{3.141228in}}{\pgfqpoint{3.278814in}{3.149041in}}%
\pgfpathcurveto{\pgfqpoint{3.286628in}{3.156855in}}{\pgfqpoint{3.291018in}{3.167454in}}{\pgfqpoint{3.291018in}{3.178504in}}%
\pgfpathcurveto{\pgfqpoint{3.291018in}{3.189554in}}{\pgfqpoint{3.286628in}{3.200153in}}{\pgfqpoint{3.278814in}{3.207967in}}%
\pgfpathcurveto{\pgfqpoint{3.271001in}{3.215780in}}{\pgfqpoint{3.260402in}{3.220171in}}{\pgfqpoint{3.249352in}{3.220171in}}%
\pgfpathcurveto{\pgfqpoint{3.238302in}{3.220171in}}{\pgfqpoint{3.227703in}{3.215780in}}{\pgfqpoint{3.219889in}{3.207967in}}%
\pgfpathcurveto{\pgfqpoint{3.212075in}{3.200153in}}{\pgfqpoint{3.207685in}{3.189554in}}{\pgfqpoint{3.207685in}{3.178504in}}%
\pgfpathcurveto{\pgfqpoint{3.207685in}{3.167454in}}{\pgfqpoint{3.212075in}{3.156855in}}{\pgfqpoint{3.219889in}{3.149041in}}%
\pgfpathcurveto{\pgfqpoint{3.227703in}{3.141228in}}{\pgfqpoint{3.238302in}{3.136837in}}{\pgfqpoint{3.249352in}{3.136837in}}%
\pgfpathclose%
\pgfusepath{stroke,fill}%
\end{pgfscope}%
\begin{pgfscope}%
\pgfpathrectangle{\pgfqpoint{0.648703in}{0.548769in}}{\pgfqpoint{5.201297in}{3.102590in}}%
\pgfusepath{clip}%
\pgfsetbuttcap%
\pgfsetroundjoin%
\definecolor{currentfill}{rgb}{1.000000,0.498039,0.054902}%
\pgfsetfillcolor{currentfill}%
\pgfsetlinewidth{1.003750pt}%
\definecolor{currentstroke}{rgb}{1.000000,0.498039,0.054902}%
\pgfsetstrokecolor{currentstroke}%
\pgfsetdash{}{0pt}%
\pgfpathmoveto{\pgfqpoint{3.890498in}{3.136837in}}%
\pgfpathcurveto{\pgfqpoint{3.901548in}{3.136837in}}{\pgfqpoint{3.912147in}{3.141228in}}{\pgfqpoint{3.919960in}{3.149041in}}%
\pgfpathcurveto{\pgfqpoint{3.927774in}{3.156855in}}{\pgfqpoint{3.932164in}{3.167454in}}{\pgfqpoint{3.932164in}{3.178504in}}%
\pgfpathcurveto{\pgfqpoint{3.932164in}{3.189554in}}{\pgfqpoint{3.927774in}{3.200153in}}{\pgfqpoint{3.919960in}{3.207967in}}%
\pgfpathcurveto{\pgfqpoint{3.912147in}{3.215780in}}{\pgfqpoint{3.901548in}{3.220171in}}{\pgfqpoint{3.890498in}{3.220171in}}%
\pgfpathcurveto{\pgfqpoint{3.879448in}{3.220171in}}{\pgfqpoint{3.868849in}{3.215780in}}{\pgfqpoint{3.861035in}{3.207967in}}%
\pgfpathcurveto{\pgfqpoint{3.853221in}{3.200153in}}{\pgfqpoint{3.848831in}{3.189554in}}{\pgfqpoint{3.848831in}{3.178504in}}%
\pgfpathcurveto{\pgfqpoint{3.848831in}{3.167454in}}{\pgfqpoint{3.853221in}{3.156855in}}{\pgfqpoint{3.861035in}{3.149041in}}%
\pgfpathcurveto{\pgfqpoint{3.868849in}{3.141228in}}{\pgfqpoint{3.879448in}{3.136837in}}{\pgfqpoint{3.890498in}{3.136837in}}%
\pgfpathclose%
\pgfusepath{stroke,fill}%
\end{pgfscope}%
\begin{pgfscope}%
\pgfpathrectangle{\pgfqpoint{0.648703in}{0.548769in}}{\pgfqpoint{5.201297in}{3.102590in}}%
\pgfusepath{clip}%
\pgfsetbuttcap%
\pgfsetroundjoin%
\definecolor{currentfill}{rgb}{1.000000,0.498039,0.054902}%
\pgfsetfillcolor{currentfill}%
\pgfsetlinewidth{1.003750pt}%
\definecolor{currentstroke}{rgb}{1.000000,0.498039,0.054902}%
\pgfsetstrokecolor{currentstroke}%
\pgfsetdash{}{0pt}%
\pgfpathmoveto{\pgfqpoint{3.449710in}{3.136837in}}%
\pgfpathcurveto{\pgfqpoint{3.460760in}{3.136837in}}{\pgfqpoint{3.471359in}{3.141228in}}{\pgfqpoint{3.479173in}{3.149041in}}%
\pgfpathcurveto{\pgfqpoint{3.486986in}{3.156855in}}{\pgfqpoint{3.491376in}{3.167454in}}{\pgfqpoint{3.491376in}{3.178504in}}%
\pgfpathcurveto{\pgfqpoint{3.491376in}{3.189554in}}{\pgfqpoint{3.486986in}{3.200153in}}{\pgfqpoint{3.479173in}{3.207967in}}%
\pgfpathcurveto{\pgfqpoint{3.471359in}{3.215780in}}{\pgfqpoint{3.460760in}{3.220171in}}{\pgfqpoint{3.449710in}{3.220171in}}%
\pgfpathcurveto{\pgfqpoint{3.438660in}{3.220171in}}{\pgfqpoint{3.428061in}{3.215780in}}{\pgfqpoint{3.420247in}{3.207967in}}%
\pgfpathcurveto{\pgfqpoint{3.412433in}{3.200153in}}{\pgfqpoint{3.408043in}{3.189554in}}{\pgfqpoint{3.408043in}{3.178504in}}%
\pgfpathcurveto{\pgfqpoint{3.408043in}{3.167454in}}{\pgfqpoint{3.412433in}{3.156855in}}{\pgfqpoint{3.420247in}{3.149041in}}%
\pgfpathcurveto{\pgfqpoint{3.428061in}{3.141228in}}{\pgfqpoint{3.438660in}{3.136837in}}{\pgfqpoint{3.449710in}{3.136837in}}%
\pgfpathclose%
\pgfusepath{stroke,fill}%
\end{pgfscope}%
\begin{pgfscope}%
\pgfpathrectangle{\pgfqpoint{0.648703in}{0.548769in}}{\pgfqpoint{5.201297in}{3.102590in}}%
\pgfusepath{clip}%
\pgfsetbuttcap%
\pgfsetroundjoin%
\definecolor{currentfill}{rgb}{1.000000,0.498039,0.054902}%
\pgfsetfillcolor{currentfill}%
\pgfsetlinewidth{1.003750pt}%
\definecolor{currentstroke}{rgb}{1.000000,0.498039,0.054902}%
\pgfsetstrokecolor{currentstroke}%
\pgfsetdash{}{0pt}%
\pgfpathmoveto{\pgfqpoint{3.129137in}{3.136837in}}%
\pgfpathcurveto{\pgfqpoint{3.140187in}{3.136837in}}{\pgfqpoint{3.150786in}{3.141228in}}{\pgfqpoint{3.158600in}{3.149041in}}%
\pgfpathcurveto{\pgfqpoint{3.166413in}{3.156855in}}{\pgfqpoint{3.170804in}{3.167454in}}{\pgfqpoint{3.170804in}{3.178504in}}%
\pgfpathcurveto{\pgfqpoint{3.170804in}{3.189554in}}{\pgfqpoint{3.166413in}{3.200153in}}{\pgfqpoint{3.158600in}{3.207967in}}%
\pgfpathcurveto{\pgfqpoint{3.150786in}{3.215780in}}{\pgfqpoint{3.140187in}{3.220171in}}{\pgfqpoint{3.129137in}{3.220171in}}%
\pgfpathcurveto{\pgfqpoint{3.118087in}{3.220171in}}{\pgfqpoint{3.107488in}{3.215780in}}{\pgfqpoint{3.099674in}{3.207967in}}%
\pgfpathcurveto{\pgfqpoint{3.091860in}{3.200153in}}{\pgfqpoint{3.087470in}{3.189554in}}{\pgfqpoint{3.087470in}{3.178504in}}%
\pgfpathcurveto{\pgfqpoint{3.087470in}{3.167454in}}{\pgfqpoint{3.091860in}{3.156855in}}{\pgfqpoint{3.099674in}{3.149041in}}%
\pgfpathcurveto{\pgfqpoint{3.107488in}{3.141228in}}{\pgfqpoint{3.118087in}{3.136837in}}{\pgfqpoint{3.129137in}{3.136837in}}%
\pgfpathclose%
\pgfusepath{stroke,fill}%
\end{pgfscope}%
\begin{pgfscope}%
\pgfpathrectangle{\pgfqpoint{0.648703in}{0.548769in}}{\pgfqpoint{5.201297in}{3.102590in}}%
\pgfusepath{clip}%
\pgfsetbuttcap%
\pgfsetroundjoin%
\definecolor{currentfill}{rgb}{0.121569,0.466667,0.705882}%
\pgfsetfillcolor{currentfill}%
\pgfsetlinewidth{1.003750pt}%
\definecolor{currentstroke}{rgb}{0.121569,0.466667,0.705882}%
\pgfsetstrokecolor{currentstroke}%
\pgfsetdash{}{0pt}%
\pgfpathmoveto{\pgfqpoint{3.369567in}{2.846488in}}%
\pgfpathcurveto{\pgfqpoint{3.380617in}{2.846488in}}{\pgfqpoint{3.391216in}{2.850878in}}{\pgfqpoint{3.399029in}{2.858692in}}%
\pgfpathcurveto{\pgfqpoint{3.406843in}{2.866506in}}{\pgfqpoint{3.411233in}{2.877105in}}{\pgfqpoint{3.411233in}{2.888155in}}%
\pgfpathcurveto{\pgfqpoint{3.411233in}{2.899205in}}{\pgfqpoint{3.406843in}{2.909804in}}{\pgfqpoint{3.399029in}{2.917617in}}%
\pgfpathcurveto{\pgfqpoint{3.391216in}{2.925431in}}{\pgfqpoint{3.380617in}{2.929821in}}{\pgfqpoint{3.369567in}{2.929821in}}%
\pgfpathcurveto{\pgfqpoint{3.358516in}{2.929821in}}{\pgfqpoint{3.347917in}{2.925431in}}{\pgfqpoint{3.340104in}{2.917617in}}%
\pgfpathcurveto{\pgfqpoint{3.332290in}{2.909804in}}{\pgfqpoint{3.327900in}{2.899205in}}{\pgfqpoint{3.327900in}{2.888155in}}%
\pgfpathcurveto{\pgfqpoint{3.327900in}{2.877105in}}{\pgfqpoint{3.332290in}{2.866506in}}{\pgfqpoint{3.340104in}{2.858692in}}%
\pgfpathcurveto{\pgfqpoint{3.347917in}{2.850878in}}{\pgfqpoint{3.358516in}{2.846488in}}{\pgfqpoint{3.369567in}{2.846488in}}%
\pgfpathclose%
\pgfusepath{stroke,fill}%
\end{pgfscope}%
\begin{pgfscope}%
\pgfpathrectangle{\pgfqpoint{0.648703in}{0.548769in}}{\pgfqpoint{5.201297in}{3.102590in}}%
\pgfusepath{clip}%
\pgfsetbuttcap%
\pgfsetroundjoin%
\definecolor{currentfill}{rgb}{0.121569,0.466667,0.705882}%
\pgfsetfillcolor{currentfill}%
\pgfsetlinewidth{1.003750pt}%
\definecolor{currentstroke}{rgb}{0.121569,0.466667,0.705882}%
\pgfsetstrokecolor{currentstroke}%
\pgfsetdash{}{0pt}%
\pgfpathmoveto{\pgfqpoint{0.885126in}{3.074620in}}%
\pgfpathcurveto{\pgfqpoint{0.896176in}{3.074620in}}{\pgfqpoint{0.906775in}{3.079010in}}{\pgfqpoint{0.914589in}{3.086824in}}%
\pgfpathcurveto{\pgfqpoint{0.922402in}{3.094637in}}{\pgfqpoint{0.926793in}{3.105236in}}{\pgfqpoint{0.926793in}{3.116286in}}%
\pgfpathcurveto{\pgfqpoint{0.926793in}{3.127336in}}{\pgfqpoint{0.922402in}{3.137935in}}{\pgfqpoint{0.914589in}{3.145749in}}%
\pgfpathcurveto{\pgfqpoint{0.906775in}{3.153563in}}{\pgfqpoint{0.896176in}{3.157953in}}{\pgfqpoint{0.885126in}{3.157953in}}%
\pgfpathcurveto{\pgfqpoint{0.874076in}{3.157953in}}{\pgfqpoint{0.863477in}{3.153563in}}{\pgfqpoint{0.855663in}{3.145749in}}%
\pgfpathcurveto{\pgfqpoint{0.847850in}{3.137935in}}{\pgfqpoint{0.843459in}{3.127336in}}{\pgfqpoint{0.843459in}{3.116286in}}%
\pgfpathcurveto{\pgfqpoint{0.843459in}{3.105236in}}{\pgfqpoint{0.847850in}{3.094637in}}{\pgfqpoint{0.855663in}{3.086824in}}%
\pgfpathcurveto{\pgfqpoint{0.863477in}{3.079010in}}{\pgfqpoint{0.874076in}{3.074620in}}{\pgfqpoint{0.885126in}{3.074620in}}%
\pgfpathclose%
\pgfusepath{stroke,fill}%
\end{pgfscope}%
\begin{pgfscope}%
\pgfpathrectangle{\pgfqpoint{0.648703in}{0.548769in}}{\pgfqpoint{5.201297in}{3.102590in}}%
\pgfusepath{clip}%
\pgfsetbuttcap%
\pgfsetroundjoin%
\definecolor{currentfill}{rgb}{1.000000,0.498039,0.054902}%
\pgfsetfillcolor{currentfill}%
\pgfsetlinewidth{1.003750pt}%
\definecolor{currentstroke}{rgb}{1.000000,0.498039,0.054902}%
\pgfsetstrokecolor{currentstroke}%
\pgfsetdash{}{0pt}%
\pgfpathmoveto{\pgfqpoint{3.890498in}{3.315195in}}%
\pgfpathcurveto{\pgfqpoint{3.901548in}{3.315195in}}{\pgfqpoint{3.912147in}{3.319585in}}{\pgfqpoint{3.919960in}{3.327399in}}%
\pgfpathcurveto{\pgfqpoint{3.927774in}{3.335212in}}{\pgfqpoint{3.932164in}{3.345811in}}{\pgfqpoint{3.932164in}{3.356861in}}%
\pgfpathcurveto{\pgfqpoint{3.932164in}{3.367912in}}{\pgfqpoint{3.927774in}{3.378511in}}{\pgfqpoint{3.919960in}{3.386324in}}%
\pgfpathcurveto{\pgfqpoint{3.912147in}{3.394138in}}{\pgfqpoint{3.901548in}{3.398528in}}{\pgfqpoint{3.890498in}{3.398528in}}%
\pgfpathcurveto{\pgfqpoint{3.879448in}{3.398528in}}{\pgfqpoint{3.868849in}{3.394138in}}{\pgfqpoint{3.861035in}{3.386324in}}%
\pgfpathcurveto{\pgfqpoint{3.853221in}{3.378511in}}{\pgfqpoint{3.848831in}{3.367912in}}{\pgfqpoint{3.848831in}{3.356861in}}%
\pgfpathcurveto{\pgfqpoint{3.848831in}{3.345811in}}{\pgfqpoint{3.853221in}{3.335212in}}{\pgfqpoint{3.861035in}{3.327399in}}%
\pgfpathcurveto{\pgfqpoint{3.868849in}{3.319585in}}{\pgfqpoint{3.879448in}{3.315195in}}{\pgfqpoint{3.890498in}{3.315195in}}%
\pgfpathclose%
\pgfusepath{stroke,fill}%
\end{pgfscope}%
\begin{pgfscope}%
\pgfpathrectangle{\pgfqpoint{0.648703in}{0.548769in}}{\pgfqpoint{5.201297in}{3.102590in}}%
\pgfusepath{clip}%
\pgfsetbuttcap%
\pgfsetroundjoin%
\definecolor{currentfill}{rgb}{0.121569,0.466667,0.705882}%
\pgfsetfillcolor{currentfill}%
\pgfsetlinewidth{1.003750pt}%
\definecolor{currentstroke}{rgb}{0.121569,0.466667,0.705882}%
\pgfsetstrokecolor{currentstroke}%
\pgfsetdash{}{0pt}%
\pgfpathmoveto{\pgfqpoint{3.650068in}{0.648129in}}%
\pgfpathcurveto{\pgfqpoint{3.661118in}{0.648129in}}{\pgfqpoint{3.671717in}{0.652519in}}{\pgfqpoint{3.679531in}{0.660333in}}%
\pgfpathcurveto{\pgfqpoint{3.687344in}{0.668146in}}{\pgfqpoint{3.691735in}{0.678745in}}{\pgfqpoint{3.691735in}{0.689796in}}%
\pgfpathcurveto{\pgfqpoint{3.691735in}{0.700846in}}{\pgfqpoint{3.687344in}{0.711445in}}{\pgfqpoint{3.679531in}{0.719258in}}%
\pgfpathcurveto{\pgfqpoint{3.671717in}{0.727072in}}{\pgfqpoint{3.661118in}{0.731462in}}{\pgfqpoint{3.650068in}{0.731462in}}%
\pgfpathcurveto{\pgfqpoint{3.639018in}{0.731462in}}{\pgfqpoint{3.628419in}{0.727072in}}{\pgfqpoint{3.620605in}{0.719258in}}%
\pgfpathcurveto{\pgfqpoint{3.612792in}{0.711445in}}{\pgfqpoint{3.608401in}{0.700846in}}{\pgfqpoint{3.608401in}{0.689796in}}%
\pgfpathcurveto{\pgfqpoint{3.608401in}{0.678745in}}{\pgfqpoint{3.612792in}{0.668146in}}{\pgfqpoint{3.620605in}{0.660333in}}%
\pgfpathcurveto{\pgfqpoint{3.628419in}{0.652519in}}{\pgfqpoint{3.639018in}{0.648129in}}{\pgfqpoint{3.650068in}{0.648129in}}%
\pgfpathclose%
\pgfusepath{stroke,fill}%
\end{pgfscope}%
\begin{pgfscope}%
\pgfpathrectangle{\pgfqpoint{0.648703in}{0.548769in}}{\pgfqpoint{5.201297in}{3.102590in}}%
\pgfusepath{clip}%
\pgfsetbuttcap%
\pgfsetroundjoin%
\definecolor{currentfill}{rgb}{0.121569,0.466667,0.705882}%
\pgfsetfillcolor{currentfill}%
\pgfsetlinewidth{1.003750pt}%
\definecolor{currentstroke}{rgb}{0.121569,0.466667,0.705882}%
\pgfsetstrokecolor{currentstroke}%
\pgfsetdash{}{0pt}%
\pgfpathmoveto{\pgfqpoint{3.048994in}{1.166610in}}%
\pgfpathcurveto{\pgfqpoint{3.060044in}{1.166610in}}{\pgfqpoint{3.070643in}{1.171000in}}{\pgfqpoint{3.078456in}{1.178814in}}%
\pgfpathcurveto{\pgfqpoint{3.086270in}{1.186627in}}{\pgfqpoint{3.090660in}{1.197226in}}{\pgfqpoint{3.090660in}{1.208277in}}%
\pgfpathcurveto{\pgfqpoint{3.090660in}{1.219327in}}{\pgfqpoint{3.086270in}{1.229926in}}{\pgfqpoint{3.078456in}{1.237739in}}%
\pgfpathcurveto{\pgfqpoint{3.070643in}{1.245553in}}{\pgfqpoint{3.060044in}{1.249943in}}{\pgfqpoint{3.048994in}{1.249943in}}%
\pgfpathcurveto{\pgfqpoint{3.037943in}{1.249943in}}{\pgfqpoint{3.027344in}{1.245553in}}{\pgfqpoint{3.019531in}{1.237739in}}%
\pgfpathcurveto{\pgfqpoint{3.011717in}{1.229926in}}{\pgfqpoint{3.007327in}{1.219327in}}{\pgfqpoint{3.007327in}{1.208277in}}%
\pgfpathcurveto{\pgfqpoint{3.007327in}{1.197226in}}{\pgfqpoint{3.011717in}{1.186627in}}{\pgfqpoint{3.019531in}{1.178814in}}%
\pgfpathcurveto{\pgfqpoint{3.027344in}{1.171000in}}{\pgfqpoint{3.037943in}{1.166610in}}{\pgfqpoint{3.048994in}{1.166610in}}%
\pgfpathclose%
\pgfusepath{stroke,fill}%
\end{pgfscope}%
\begin{pgfscope}%
\pgfpathrectangle{\pgfqpoint{0.648703in}{0.548769in}}{\pgfqpoint{5.201297in}{3.102590in}}%
\pgfusepath{clip}%
\pgfsetbuttcap%
\pgfsetroundjoin%
\definecolor{currentfill}{rgb}{1.000000,0.498039,0.054902}%
\pgfsetfillcolor{currentfill}%
\pgfsetlinewidth{1.003750pt}%
\definecolor{currentstroke}{rgb}{1.000000,0.498039,0.054902}%
\pgfsetstrokecolor{currentstroke}%
\pgfsetdash{}{0pt}%
\pgfpathmoveto{\pgfqpoint{3.569925in}{3.136837in}}%
\pgfpathcurveto{\pgfqpoint{3.580975in}{3.136837in}}{\pgfqpoint{3.591574in}{3.141228in}}{\pgfqpoint{3.599387in}{3.149041in}}%
\pgfpathcurveto{\pgfqpoint{3.607201in}{3.156855in}}{\pgfqpoint{3.611591in}{3.167454in}}{\pgfqpoint{3.611591in}{3.178504in}}%
\pgfpathcurveto{\pgfqpoint{3.611591in}{3.189554in}}{\pgfqpoint{3.607201in}{3.200153in}}{\pgfqpoint{3.599387in}{3.207967in}}%
\pgfpathcurveto{\pgfqpoint{3.591574in}{3.215780in}}{\pgfqpoint{3.580975in}{3.220171in}}{\pgfqpoint{3.569925in}{3.220171in}}%
\pgfpathcurveto{\pgfqpoint{3.558875in}{3.220171in}}{\pgfqpoint{3.548276in}{3.215780in}}{\pgfqpoint{3.540462in}{3.207967in}}%
\pgfpathcurveto{\pgfqpoint{3.532648in}{3.200153in}}{\pgfqpoint{3.528258in}{3.189554in}}{\pgfqpoint{3.528258in}{3.178504in}}%
\pgfpathcurveto{\pgfqpoint{3.528258in}{3.167454in}}{\pgfqpoint{3.532648in}{3.156855in}}{\pgfqpoint{3.540462in}{3.149041in}}%
\pgfpathcurveto{\pgfqpoint{3.548276in}{3.141228in}}{\pgfqpoint{3.558875in}{3.136837in}}{\pgfqpoint{3.569925in}{3.136837in}}%
\pgfpathclose%
\pgfusepath{stroke,fill}%
\end{pgfscope}%
\begin{pgfscope}%
\pgfpathrectangle{\pgfqpoint{0.648703in}{0.548769in}}{\pgfqpoint{5.201297in}{3.102590in}}%
\pgfusepath{clip}%
\pgfsetbuttcap%
\pgfsetroundjoin%
\definecolor{currentfill}{rgb}{0.121569,0.466667,0.705882}%
\pgfsetfillcolor{currentfill}%
\pgfsetlinewidth{1.003750pt}%
\definecolor{currentstroke}{rgb}{0.121569,0.466667,0.705882}%
\pgfsetstrokecolor{currentstroke}%
\pgfsetdash{}{0pt}%
\pgfpathmoveto{\pgfqpoint{3.650068in}{3.132690in}}%
\pgfpathcurveto{\pgfqpoint{3.661118in}{3.132690in}}{\pgfqpoint{3.671717in}{3.137080in}}{\pgfqpoint{3.679531in}{3.144893in}}%
\pgfpathcurveto{\pgfqpoint{3.687344in}{3.152707in}}{\pgfqpoint{3.691735in}{3.163306in}}{\pgfqpoint{3.691735in}{3.174356in}}%
\pgfpathcurveto{\pgfqpoint{3.691735in}{3.185406in}}{\pgfqpoint{3.687344in}{3.196005in}}{\pgfqpoint{3.679531in}{3.203819in}}%
\pgfpathcurveto{\pgfqpoint{3.671717in}{3.211633in}}{\pgfqpoint{3.661118in}{3.216023in}}{\pgfqpoint{3.650068in}{3.216023in}}%
\pgfpathcurveto{\pgfqpoint{3.639018in}{3.216023in}}{\pgfqpoint{3.628419in}{3.211633in}}{\pgfqpoint{3.620605in}{3.203819in}}%
\pgfpathcurveto{\pgfqpoint{3.612792in}{3.196005in}}{\pgfqpoint{3.608401in}{3.185406in}}{\pgfqpoint{3.608401in}{3.174356in}}%
\pgfpathcurveto{\pgfqpoint{3.608401in}{3.163306in}}{\pgfqpoint{3.612792in}{3.152707in}}{\pgfqpoint{3.620605in}{3.144893in}}%
\pgfpathcurveto{\pgfqpoint{3.628419in}{3.137080in}}{\pgfqpoint{3.639018in}{3.132690in}}{\pgfqpoint{3.650068in}{3.132690in}}%
\pgfpathclose%
\pgfusepath{stroke,fill}%
\end{pgfscope}%
\begin{pgfscope}%
\pgfpathrectangle{\pgfqpoint{0.648703in}{0.548769in}}{\pgfqpoint{5.201297in}{3.102590in}}%
\pgfusepath{clip}%
\pgfsetbuttcap%
\pgfsetroundjoin%
\definecolor{currentfill}{rgb}{0.839216,0.152941,0.156863}%
\pgfsetfillcolor{currentfill}%
\pgfsetlinewidth{1.003750pt}%
\definecolor{currentstroke}{rgb}{0.839216,0.152941,0.156863}%
\pgfsetstrokecolor{currentstroke}%
\pgfsetdash{}{0pt}%
\pgfpathmoveto{\pgfqpoint{2.968850in}{3.149281in}}%
\pgfpathcurveto{\pgfqpoint{2.979900in}{3.149281in}}{\pgfqpoint{2.990500in}{3.153671in}}{\pgfqpoint{2.998313in}{3.161485in}}%
\pgfpathcurveto{\pgfqpoint{3.006127in}{3.169298in}}{\pgfqpoint{3.010517in}{3.179897in}}{\pgfqpoint{3.010517in}{3.190948in}}%
\pgfpathcurveto{\pgfqpoint{3.010517in}{3.201998in}}{\pgfqpoint{3.006127in}{3.212597in}}{\pgfqpoint{2.998313in}{3.220410in}}%
\pgfpathcurveto{\pgfqpoint{2.990500in}{3.228224in}}{\pgfqpoint{2.979900in}{3.232614in}}{\pgfqpoint{2.968850in}{3.232614in}}%
\pgfpathcurveto{\pgfqpoint{2.957800in}{3.232614in}}{\pgfqpoint{2.947201in}{3.228224in}}{\pgfqpoint{2.939388in}{3.220410in}}%
\pgfpathcurveto{\pgfqpoint{2.931574in}{3.212597in}}{\pgfqpoint{2.927184in}{3.201998in}}{\pgfqpoint{2.927184in}{3.190948in}}%
\pgfpathcurveto{\pgfqpoint{2.927184in}{3.179897in}}{\pgfqpoint{2.931574in}{3.169298in}}{\pgfqpoint{2.939388in}{3.161485in}}%
\pgfpathcurveto{\pgfqpoint{2.947201in}{3.153671in}}{\pgfqpoint{2.957800in}{3.149281in}}{\pgfqpoint{2.968850in}{3.149281in}}%
\pgfpathclose%
\pgfusepath{stroke,fill}%
\end{pgfscope}%
\begin{pgfscope}%
\pgfpathrectangle{\pgfqpoint{0.648703in}{0.548769in}}{\pgfqpoint{5.201297in}{3.102590in}}%
\pgfusepath{clip}%
\pgfsetbuttcap%
\pgfsetroundjoin%
\definecolor{currentfill}{rgb}{1.000000,0.498039,0.054902}%
\pgfsetfillcolor{currentfill}%
\pgfsetlinewidth{1.003750pt}%
\definecolor{currentstroke}{rgb}{1.000000,0.498039,0.054902}%
\pgfsetstrokecolor{currentstroke}%
\pgfsetdash{}{0pt}%
\pgfpathmoveto{\pgfqpoint{3.289423in}{3.174168in}}%
\pgfpathcurveto{\pgfqpoint{3.300473in}{3.174168in}}{\pgfqpoint{3.311072in}{3.178558in}}{\pgfqpoint{3.318886in}{3.186372in}}%
\pgfpathcurveto{\pgfqpoint{3.326700in}{3.194185in}}{\pgfqpoint{3.331090in}{3.204785in}}{\pgfqpoint{3.331090in}{3.215835in}}%
\pgfpathcurveto{\pgfqpoint{3.331090in}{3.226885in}}{\pgfqpoint{3.326700in}{3.237484in}}{\pgfqpoint{3.318886in}{3.245297in}}%
\pgfpathcurveto{\pgfqpoint{3.311072in}{3.253111in}}{\pgfqpoint{3.300473in}{3.257501in}}{\pgfqpoint{3.289423in}{3.257501in}}%
\pgfpathcurveto{\pgfqpoint{3.278373in}{3.257501in}}{\pgfqpoint{3.267774in}{3.253111in}}{\pgfqpoint{3.259961in}{3.245297in}}%
\pgfpathcurveto{\pgfqpoint{3.252147in}{3.237484in}}{\pgfqpoint{3.247757in}{3.226885in}}{\pgfqpoint{3.247757in}{3.215835in}}%
\pgfpathcurveto{\pgfqpoint{3.247757in}{3.204785in}}{\pgfqpoint{3.252147in}{3.194185in}}{\pgfqpoint{3.259961in}{3.186372in}}%
\pgfpathcurveto{\pgfqpoint{3.267774in}{3.178558in}}{\pgfqpoint{3.278373in}{3.174168in}}{\pgfqpoint{3.289423in}{3.174168in}}%
\pgfpathclose%
\pgfusepath{stroke,fill}%
\end{pgfscope}%
\begin{pgfscope}%
\pgfpathrectangle{\pgfqpoint{0.648703in}{0.548769in}}{\pgfqpoint{5.201297in}{3.102590in}}%
\pgfusepath{clip}%
\pgfsetbuttcap%
\pgfsetroundjoin%
\definecolor{currentfill}{rgb}{0.121569,0.466667,0.705882}%
\pgfsetfillcolor{currentfill}%
\pgfsetlinewidth{1.003750pt}%
\definecolor{currentstroke}{rgb}{0.121569,0.466667,0.705882}%
\pgfsetstrokecolor{currentstroke}%
\pgfsetdash{}{0pt}%
\pgfpathmoveto{\pgfqpoint{3.569925in}{0.656425in}}%
\pgfpathcurveto{\pgfqpoint{3.580975in}{0.656425in}}{\pgfqpoint{3.591574in}{0.660815in}}{\pgfqpoint{3.599387in}{0.668629in}}%
\pgfpathcurveto{\pgfqpoint{3.607201in}{0.676442in}}{\pgfqpoint{3.611591in}{0.687041in}}{\pgfqpoint{3.611591in}{0.698091in}}%
\pgfpathcurveto{\pgfqpoint{3.611591in}{0.709141in}}{\pgfqpoint{3.607201in}{0.719740in}}{\pgfqpoint{3.599387in}{0.727554in}}%
\pgfpathcurveto{\pgfqpoint{3.591574in}{0.735368in}}{\pgfqpoint{3.580975in}{0.739758in}}{\pgfqpoint{3.569925in}{0.739758in}}%
\pgfpathcurveto{\pgfqpoint{3.558875in}{0.739758in}}{\pgfqpoint{3.548276in}{0.735368in}}{\pgfqpoint{3.540462in}{0.727554in}}%
\pgfpathcurveto{\pgfqpoint{3.532648in}{0.719740in}}{\pgfqpoint{3.528258in}{0.709141in}}{\pgfqpoint{3.528258in}{0.698091in}}%
\pgfpathcurveto{\pgfqpoint{3.528258in}{0.687041in}}{\pgfqpoint{3.532648in}{0.676442in}}{\pgfqpoint{3.540462in}{0.668629in}}%
\pgfpathcurveto{\pgfqpoint{3.548276in}{0.660815in}}{\pgfqpoint{3.558875in}{0.656425in}}{\pgfqpoint{3.569925in}{0.656425in}}%
\pgfpathclose%
\pgfusepath{stroke,fill}%
\end{pgfscope}%
\begin{pgfscope}%
\pgfpathrectangle{\pgfqpoint{0.648703in}{0.548769in}}{\pgfqpoint{5.201297in}{3.102590in}}%
\pgfusepath{clip}%
\pgfsetbuttcap%
\pgfsetroundjoin%
\definecolor{currentfill}{rgb}{0.121569,0.466667,0.705882}%
\pgfsetfillcolor{currentfill}%
\pgfsetlinewidth{1.003750pt}%
\definecolor{currentstroke}{rgb}{0.121569,0.466667,0.705882}%
\pgfsetstrokecolor{currentstroke}%
\pgfsetdash{}{0pt}%
\pgfpathmoveto{\pgfqpoint{3.650068in}{3.124394in}}%
\pgfpathcurveto{\pgfqpoint{3.661118in}{3.124394in}}{\pgfqpoint{3.671717in}{3.128784in}}{\pgfqpoint{3.679531in}{3.136598in}}%
\pgfpathcurveto{\pgfqpoint{3.687344in}{3.144411in}}{\pgfqpoint{3.691735in}{3.155010in}}{\pgfqpoint{3.691735in}{3.166060in}}%
\pgfpathcurveto{\pgfqpoint{3.691735in}{3.177111in}}{\pgfqpoint{3.687344in}{3.187710in}}{\pgfqpoint{3.679531in}{3.195523in}}%
\pgfpathcurveto{\pgfqpoint{3.671717in}{3.203337in}}{\pgfqpoint{3.661118in}{3.207727in}}{\pgfqpoint{3.650068in}{3.207727in}}%
\pgfpathcurveto{\pgfqpoint{3.639018in}{3.207727in}}{\pgfqpoint{3.628419in}{3.203337in}}{\pgfqpoint{3.620605in}{3.195523in}}%
\pgfpathcurveto{\pgfqpoint{3.612792in}{3.187710in}}{\pgfqpoint{3.608401in}{3.177111in}}{\pgfqpoint{3.608401in}{3.166060in}}%
\pgfpathcurveto{\pgfqpoint{3.608401in}{3.155010in}}{\pgfqpoint{3.612792in}{3.144411in}}{\pgfqpoint{3.620605in}{3.136598in}}%
\pgfpathcurveto{\pgfqpoint{3.628419in}{3.128784in}}{\pgfqpoint{3.639018in}{3.124394in}}{\pgfqpoint{3.650068in}{3.124394in}}%
\pgfpathclose%
\pgfusepath{stroke,fill}%
\end{pgfscope}%
\begin{pgfscope}%
\pgfpathrectangle{\pgfqpoint{0.648703in}{0.548769in}}{\pgfqpoint{5.201297in}{3.102590in}}%
\pgfusepath{clip}%
\pgfsetbuttcap%
\pgfsetroundjoin%
\definecolor{currentfill}{rgb}{0.121569,0.466667,0.705882}%
\pgfsetfillcolor{currentfill}%
\pgfsetlinewidth{1.003750pt}%
\definecolor{currentstroke}{rgb}{0.121569,0.466667,0.705882}%
\pgfsetstrokecolor{currentstroke}%
\pgfsetdash{}{0pt}%
\pgfpathmoveto{\pgfqpoint{3.329495in}{0.648129in}}%
\pgfpathcurveto{\pgfqpoint{3.340545in}{0.648129in}}{\pgfqpoint{3.351144in}{0.652519in}}{\pgfqpoint{3.358958in}{0.660333in}}%
\pgfpathcurveto{\pgfqpoint{3.366771in}{0.668146in}}{\pgfqpoint{3.371162in}{0.678745in}}{\pgfqpoint{3.371162in}{0.689796in}}%
\pgfpathcurveto{\pgfqpoint{3.371162in}{0.700846in}}{\pgfqpoint{3.366771in}{0.711445in}}{\pgfqpoint{3.358958in}{0.719258in}}%
\pgfpathcurveto{\pgfqpoint{3.351144in}{0.727072in}}{\pgfqpoint{3.340545in}{0.731462in}}{\pgfqpoint{3.329495in}{0.731462in}}%
\pgfpathcurveto{\pgfqpoint{3.318445in}{0.731462in}}{\pgfqpoint{3.307846in}{0.727072in}}{\pgfqpoint{3.300032in}{0.719258in}}%
\pgfpathcurveto{\pgfqpoint{3.292219in}{0.711445in}}{\pgfqpoint{3.287828in}{0.700846in}}{\pgfqpoint{3.287828in}{0.689796in}}%
\pgfpathcurveto{\pgfqpoint{3.287828in}{0.678745in}}{\pgfqpoint{3.292219in}{0.668146in}}{\pgfqpoint{3.300032in}{0.660333in}}%
\pgfpathcurveto{\pgfqpoint{3.307846in}{0.652519in}}{\pgfqpoint{3.318445in}{0.648129in}}{\pgfqpoint{3.329495in}{0.648129in}}%
\pgfpathclose%
\pgfusepath{stroke,fill}%
\end{pgfscope}%
\begin{pgfscope}%
\pgfpathrectangle{\pgfqpoint{0.648703in}{0.548769in}}{\pgfqpoint{5.201297in}{3.102590in}}%
\pgfusepath{clip}%
\pgfsetbuttcap%
\pgfsetroundjoin%
\definecolor{currentfill}{rgb}{1.000000,0.498039,0.054902}%
\pgfsetfillcolor{currentfill}%
\pgfsetlinewidth{1.003750pt}%
\definecolor{currentstroke}{rgb}{1.000000,0.498039,0.054902}%
\pgfsetstrokecolor{currentstroke}%
\pgfsetdash{}{0pt}%
\pgfpathmoveto{\pgfqpoint{3.369567in}{3.136837in}}%
\pgfpathcurveto{\pgfqpoint{3.380617in}{3.136837in}}{\pgfqpoint{3.391216in}{3.141228in}}{\pgfqpoint{3.399029in}{3.149041in}}%
\pgfpathcurveto{\pgfqpoint{3.406843in}{3.156855in}}{\pgfqpoint{3.411233in}{3.167454in}}{\pgfqpoint{3.411233in}{3.178504in}}%
\pgfpathcurveto{\pgfqpoint{3.411233in}{3.189554in}}{\pgfqpoint{3.406843in}{3.200153in}}{\pgfqpoint{3.399029in}{3.207967in}}%
\pgfpathcurveto{\pgfqpoint{3.391216in}{3.215780in}}{\pgfqpoint{3.380617in}{3.220171in}}{\pgfqpoint{3.369567in}{3.220171in}}%
\pgfpathcurveto{\pgfqpoint{3.358516in}{3.220171in}}{\pgfqpoint{3.347917in}{3.215780in}}{\pgfqpoint{3.340104in}{3.207967in}}%
\pgfpathcurveto{\pgfqpoint{3.332290in}{3.200153in}}{\pgfqpoint{3.327900in}{3.189554in}}{\pgfqpoint{3.327900in}{3.178504in}}%
\pgfpathcurveto{\pgfqpoint{3.327900in}{3.167454in}}{\pgfqpoint{3.332290in}{3.156855in}}{\pgfqpoint{3.340104in}{3.149041in}}%
\pgfpathcurveto{\pgfqpoint{3.347917in}{3.141228in}}{\pgfqpoint{3.358516in}{3.136837in}}{\pgfqpoint{3.369567in}{3.136837in}}%
\pgfpathclose%
\pgfusepath{stroke,fill}%
\end{pgfscope}%
\begin{pgfscope}%
\pgfpathrectangle{\pgfqpoint{0.648703in}{0.548769in}}{\pgfqpoint{5.201297in}{3.102590in}}%
\pgfusepath{clip}%
\pgfsetbuttcap%
\pgfsetroundjoin%
\definecolor{currentfill}{rgb}{1.000000,0.498039,0.054902}%
\pgfsetfillcolor{currentfill}%
\pgfsetlinewidth{1.003750pt}%
\definecolor{currentstroke}{rgb}{1.000000,0.498039,0.054902}%
\pgfsetstrokecolor{currentstroke}%
\pgfsetdash{}{0pt}%
\pgfpathmoveto{\pgfqpoint{3.369567in}{3.140985in}}%
\pgfpathcurveto{\pgfqpoint{3.380617in}{3.140985in}}{\pgfqpoint{3.391216in}{3.145375in}}{\pgfqpoint{3.399029in}{3.153189in}}%
\pgfpathcurveto{\pgfqpoint{3.406843in}{3.161003in}}{\pgfqpoint{3.411233in}{3.171602in}}{\pgfqpoint{3.411233in}{3.182652in}}%
\pgfpathcurveto{\pgfqpoint{3.411233in}{3.193702in}}{\pgfqpoint{3.406843in}{3.204301in}}{\pgfqpoint{3.399029in}{3.212115in}}%
\pgfpathcurveto{\pgfqpoint{3.391216in}{3.219928in}}{\pgfqpoint{3.380617in}{3.224319in}}{\pgfqpoint{3.369567in}{3.224319in}}%
\pgfpathcurveto{\pgfqpoint{3.358516in}{3.224319in}}{\pgfqpoint{3.347917in}{3.219928in}}{\pgfqpoint{3.340104in}{3.212115in}}%
\pgfpathcurveto{\pgfqpoint{3.332290in}{3.204301in}}{\pgfqpoint{3.327900in}{3.193702in}}{\pgfqpoint{3.327900in}{3.182652in}}%
\pgfpathcurveto{\pgfqpoint{3.327900in}{3.171602in}}{\pgfqpoint{3.332290in}{3.161003in}}{\pgfqpoint{3.340104in}{3.153189in}}%
\pgfpathcurveto{\pgfqpoint{3.347917in}{3.145375in}}{\pgfqpoint{3.358516in}{3.140985in}}{\pgfqpoint{3.369567in}{3.140985in}}%
\pgfpathclose%
\pgfusepath{stroke,fill}%
\end{pgfscope}%
\begin{pgfscope}%
\pgfpathrectangle{\pgfqpoint{0.648703in}{0.548769in}}{\pgfqpoint{5.201297in}{3.102590in}}%
\pgfusepath{clip}%
\pgfsetbuttcap%
\pgfsetroundjoin%
\definecolor{currentfill}{rgb}{1.000000,0.498039,0.054902}%
\pgfsetfillcolor{currentfill}%
\pgfsetlinewidth{1.003750pt}%
\definecolor{currentstroke}{rgb}{1.000000,0.498039,0.054902}%
\pgfsetstrokecolor{currentstroke}%
\pgfsetdash{}{0pt}%
\pgfpathmoveto{\pgfqpoint{1.886917in}{3.165872in}}%
\pgfpathcurveto{\pgfqpoint{1.897967in}{3.165872in}}{\pgfqpoint{1.908566in}{3.170263in}}{\pgfqpoint{1.916379in}{3.178076in}}%
\pgfpathcurveto{\pgfqpoint{1.924193in}{3.185890in}}{\pgfqpoint{1.928583in}{3.196489in}}{\pgfqpoint{1.928583in}{3.207539in}}%
\pgfpathcurveto{\pgfqpoint{1.928583in}{3.218589in}}{\pgfqpoint{1.924193in}{3.229188in}}{\pgfqpoint{1.916379in}{3.237002in}}%
\pgfpathcurveto{\pgfqpoint{1.908566in}{3.244815in}}{\pgfqpoint{1.897967in}{3.249206in}}{\pgfqpoint{1.886917in}{3.249206in}}%
\pgfpathcurveto{\pgfqpoint{1.875866in}{3.249206in}}{\pgfqpoint{1.865267in}{3.244815in}}{\pgfqpoint{1.857454in}{3.237002in}}%
\pgfpathcurveto{\pgfqpoint{1.849640in}{3.229188in}}{\pgfqpoint{1.845250in}{3.218589in}}{\pgfqpoint{1.845250in}{3.207539in}}%
\pgfpathcurveto{\pgfqpoint{1.845250in}{3.196489in}}{\pgfqpoint{1.849640in}{3.185890in}}{\pgfqpoint{1.857454in}{3.178076in}}%
\pgfpathcurveto{\pgfqpoint{1.865267in}{3.170263in}}{\pgfqpoint{1.875866in}{3.165872in}}{\pgfqpoint{1.886917in}{3.165872in}}%
\pgfpathclose%
\pgfusepath{stroke,fill}%
\end{pgfscope}%
\begin{pgfscope}%
\pgfpathrectangle{\pgfqpoint{0.648703in}{0.548769in}}{\pgfqpoint{5.201297in}{3.102590in}}%
\pgfusepath{clip}%
\pgfsetbuttcap%
\pgfsetroundjoin%
\definecolor{currentfill}{rgb}{1.000000,0.498039,0.054902}%
\pgfsetfillcolor{currentfill}%
\pgfsetlinewidth{1.003750pt}%
\definecolor{currentstroke}{rgb}{1.000000,0.498039,0.054902}%
\pgfsetstrokecolor{currentstroke}%
\pgfsetdash{}{0pt}%
\pgfpathmoveto{\pgfqpoint{3.249352in}{3.145133in}}%
\pgfpathcurveto{\pgfqpoint{3.260402in}{3.145133in}}{\pgfqpoint{3.271001in}{3.149523in}}{\pgfqpoint{3.278814in}{3.157337in}}%
\pgfpathcurveto{\pgfqpoint{3.286628in}{3.165151in}}{\pgfqpoint{3.291018in}{3.175750in}}{\pgfqpoint{3.291018in}{3.186800in}}%
\pgfpathcurveto{\pgfqpoint{3.291018in}{3.197850in}}{\pgfqpoint{3.286628in}{3.208449in}}{\pgfqpoint{3.278814in}{3.216262in}}%
\pgfpathcurveto{\pgfqpoint{3.271001in}{3.224076in}}{\pgfqpoint{3.260402in}{3.228466in}}{\pgfqpoint{3.249352in}{3.228466in}}%
\pgfpathcurveto{\pgfqpoint{3.238302in}{3.228466in}}{\pgfqpoint{3.227703in}{3.224076in}}{\pgfqpoint{3.219889in}{3.216262in}}%
\pgfpathcurveto{\pgfqpoint{3.212075in}{3.208449in}}{\pgfqpoint{3.207685in}{3.197850in}}{\pgfqpoint{3.207685in}{3.186800in}}%
\pgfpathcurveto{\pgfqpoint{3.207685in}{3.175750in}}{\pgfqpoint{3.212075in}{3.165151in}}{\pgfqpoint{3.219889in}{3.157337in}}%
\pgfpathcurveto{\pgfqpoint{3.227703in}{3.149523in}}{\pgfqpoint{3.238302in}{3.145133in}}{\pgfqpoint{3.249352in}{3.145133in}}%
\pgfpathclose%
\pgfusepath{stroke,fill}%
\end{pgfscope}%
\begin{pgfscope}%
\pgfpathrectangle{\pgfqpoint{0.648703in}{0.548769in}}{\pgfqpoint{5.201297in}{3.102590in}}%
\pgfusepath{clip}%
\pgfsetbuttcap%
\pgfsetroundjoin%
\definecolor{currentfill}{rgb}{1.000000,0.498039,0.054902}%
\pgfsetfillcolor{currentfill}%
\pgfsetlinewidth{1.003750pt}%
\definecolor{currentstroke}{rgb}{1.000000,0.498039,0.054902}%
\pgfsetstrokecolor{currentstroke}%
\pgfsetdash{}{0pt}%
\pgfpathmoveto{\pgfqpoint{3.329495in}{3.157577in}}%
\pgfpathcurveto{\pgfqpoint{3.340545in}{3.157577in}}{\pgfqpoint{3.351144in}{3.161967in}}{\pgfqpoint{3.358958in}{3.169780in}}%
\pgfpathcurveto{\pgfqpoint{3.366771in}{3.177594in}}{\pgfqpoint{3.371162in}{3.188193in}}{\pgfqpoint{3.371162in}{3.199243in}}%
\pgfpathcurveto{\pgfqpoint{3.371162in}{3.210293in}}{\pgfqpoint{3.366771in}{3.220892in}}{\pgfqpoint{3.358958in}{3.228706in}}%
\pgfpathcurveto{\pgfqpoint{3.351144in}{3.236520in}}{\pgfqpoint{3.340545in}{3.240910in}}{\pgfqpoint{3.329495in}{3.240910in}}%
\pgfpathcurveto{\pgfqpoint{3.318445in}{3.240910in}}{\pgfqpoint{3.307846in}{3.236520in}}{\pgfqpoint{3.300032in}{3.228706in}}%
\pgfpathcurveto{\pgfqpoint{3.292219in}{3.220892in}}{\pgfqpoint{3.287828in}{3.210293in}}{\pgfqpoint{3.287828in}{3.199243in}}%
\pgfpathcurveto{\pgfqpoint{3.287828in}{3.188193in}}{\pgfqpoint{3.292219in}{3.177594in}}{\pgfqpoint{3.300032in}{3.169780in}}%
\pgfpathcurveto{\pgfqpoint{3.307846in}{3.161967in}}{\pgfqpoint{3.318445in}{3.157577in}}{\pgfqpoint{3.329495in}{3.157577in}}%
\pgfpathclose%
\pgfusepath{stroke,fill}%
\end{pgfscope}%
\begin{pgfscope}%
\pgfpathrectangle{\pgfqpoint{0.648703in}{0.548769in}}{\pgfqpoint{5.201297in}{3.102590in}}%
\pgfusepath{clip}%
\pgfsetbuttcap%
\pgfsetroundjoin%
\definecolor{currentfill}{rgb}{1.000000,0.498039,0.054902}%
\pgfsetfillcolor{currentfill}%
\pgfsetlinewidth{1.003750pt}%
\definecolor{currentstroke}{rgb}{1.000000,0.498039,0.054902}%
\pgfsetstrokecolor{currentstroke}%
\pgfsetdash{}{0pt}%
\pgfpathmoveto{\pgfqpoint{3.650068in}{3.323490in}}%
\pgfpathcurveto{\pgfqpoint{3.661118in}{3.323490in}}{\pgfqpoint{3.671717in}{3.327881in}}{\pgfqpoint{3.679531in}{3.335694in}}%
\pgfpathcurveto{\pgfqpoint{3.687344in}{3.343508in}}{\pgfqpoint{3.691735in}{3.354107in}}{\pgfqpoint{3.691735in}{3.365157in}}%
\pgfpathcurveto{\pgfqpoint{3.691735in}{3.376207in}}{\pgfqpoint{3.687344in}{3.386806in}}{\pgfqpoint{3.679531in}{3.394620in}}%
\pgfpathcurveto{\pgfqpoint{3.671717in}{3.402434in}}{\pgfqpoint{3.661118in}{3.406824in}}{\pgfqpoint{3.650068in}{3.406824in}}%
\pgfpathcurveto{\pgfqpoint{3.639018in}{3.406824in}}{\pgfqpoint{3.628419in}{3.402434in}}{\pgfqpoint{3.620605in}{3.394620in}}%
\pgfpathcurveto{\pgfqpoint{3.612792in}{3.386806in}}{\pgfqpoint{3.608401in}{3.376207in}}{\pgfqpoint{3.608401in}{3.365157in}}%
\pgfpathcurveto{\pgfqpoint{3.608401in}{3.354107in}}{\pgfqpoint{3.612792in}{3.343508in}}{\pgfqpoint{3.620605in}{3.335694in}}%
\pgfpathcurveto{\pgfqpoint{3.628419in}{3.327881in}}{\pgfqpoint{3.639018in}{3.323490in}}{\pgfqpoint{3.650068in}{3.323490in}}%
\pgfpathclose%
\pgfusepath{stroke,fill}%
\end{pgfscope}%
\begin{pgfscope}%
\pgfpathrectangle{\pgfqpoint{0.648703in}{0.548769in}}{\pgfqpoint{5.201297in}{3.102590in}}%
\pgfusepath{clip}%
\pgfsetbuttcap%
\pgfsetroundjoin%
\definecolor{currentfill}{rgb}{0.121569,0.466667,0.705882}%
\pgfsetfillcolor{currentfill}%
\pgfsetlinewidth{1.003750pt}%
\definecolor{currentstroke}{rgb}{0.121569,0.466667,0.705882}%
\pgfsetstrokecolor{currentstroke}%
\pgfsetdash{}{0pt}%
\pgfpathmoveto{\pgfqpoint{3.289423in}{2.410964in}}%
\pgfpathcurveto{\pgfqpoint{3.300473in}{2.410964in}}{\pgfqpoint{3.311072in}{2.415354in}}{\pgfqpoint{3.318886in}{2.423168in}}%
\pgfpathcurveto{\pgfqpoint{3.326700in}{2.430982in}}{\pgfqpoint{3.331090in}{2.441581in}}{\pgfqpoint{3.331090in}{2.452631in}}%
\pgfpathcurveto{\pgfqpoint{3.331090in}{2.463681in}}{\pgfqpoint{3.326700in}{2.474280in}}{\pgfqpoint{3.318886in}{2.482094in}}%
\pgfpathcurveto{\pgfqpoint{3.311072in}{2.489907in}}{\pgfqpoint{3.300473in}{2.494297in}}{\pgfqpoint{3.289423in}{2.494297in}}%
\pgfpathcurveto{\pgfqpoint{3.278373in}{2.494297in}}{\pgfqpoint{3.267774in}{2.489907in}}{\pgfqpoint{3.259961in}{2.482094in}}%
\pgfpathcurveto{\pgfqpoint{3.252147in}{2.474280in}}{\pgfqpoint{3.247757in}{2.463681in}}{\pgfqpoint{3.247757in}{2.452631in}}%
\pgfpathcurveto{\pgfqpoint{3.247757in}{2.441581in}}{\pgfqpoint{3.252147in}{2.430982in}}{\pgfqpoint{3.259961in}{2.423168in}}%
\pgfpathcurveto{\pgfqpoint{3.267774in}{2.415354in}}{\pgfqpoint{3.278373in}{2.410964in}}{\pgfqpoint{3.289423in}{2.410964in}}%
\pgfpathclose%
\pgfusepath{stroke,fill}%
\end{pgfscope}%
\begin{pgfscope}%
\pgfpathrectangle{\pgfqpoint{0.648703in}{0.548769in}}{\pgfqpoint{5.201297in}{3.102590in}}%
\pgfusepath{clip}%
\pgfsetbuttcap%
\pgfsetroundjoin%
\definecolor{currentfill}{rgb}{1.000000,0.498039,0.054902}%
\pgfsetfillcolor{currentfill}%
\pgfsetlinewidth{1.003750pt}%
\definecolor{currentstroke}{rgb}{1.000000,0.498039,0.054902}%
\pgfsetstrokecolor{currentstroke}%
\pgfsetdash{}{0pt}%
\pgfpathmoveto{\pgfqpoint{3.369567in}{3.199055in}}%
\pgfpathcurveto{\pgfqpoint{3.380617in}{3.199055in}}{\pgfqpoint{3.391216in}{3.203445in}}{\pgfqpoint{3.399029in}{3.211259in}}%
\pgfpathcurveto{\pgfqpoint{3.406843in}{3.219073in}}{\pgfqpoint{3.411233in}{3.229672in}}{\pgfqpoint{3.411233in}{3.240722in}}%
\pgfpathcurveto{\pgfqpoint{3.411233in}{3.251772in}}{\pgfqpoint{3.406843in}{3.262371in}}{\pgfqpoint{3.399029in}{3.270185in}}%
\pgfpathcurveto{\pgfqpoint{3.391216in}{3.277998in}}{\pgfqpoint{3.380617in}{3.282388in}}{\pgfqpoint{3.369567in}{3.282388in}}%
\pgfpathcurveto{\pgfqpoint{3.358516in}{3.282388in}}{\pgfqpoint{3.347917in}{3.277998in}}{\pgfqpoint{3.340104in}{3.270185in}}%
\pgfpathcurveto{\pgfqpoint{3.332290in}{3.262371in}}{\pgfqpoint{3.327900in}{3.251772in}}{\pgfqpoint{3.327900in}{3.240722in}}%
\pgfpathcurveto{\pgfqpoint{3.327900in}{3.229672in}}{\pgfqpoint{3.332290in}{3.219073in}}{\pgfqpoint{3.340104in}{3.211259in}}%
\pgfpathcurveto{\pgfqpoint{3.347917in}{3.203445in}}{\pgfqpoint{3.358516in}{3.199055in}}{\pgfqpoint{3.369567in}{3.199055in}}%
\pgfpathclose%
\pgfusepath{stroke,fill}%
\end{pgfscope}%
\begin{pgfscope}%
\pgfpathrectangle{\pgfqpoint{0.648703in}{0.548769in}}{\pgfqpoint{5.201297in}{3.102590in}}%
\pgfusepath{clip}%
\pgfsetbuttcap%
\pgfsetroundjoin%
\definecolor{currentfill}{rgb}{0.121569,0.466667,0.705882}%
\pgfsetfillcolor{currentfill}%
\pgfsetlinewidth{1.003750pt}%
\definecolor{currentstroke}{rgb}{0.121569,0.466667,0.705882}%
\pgfsetstrokecolor{currentstroke}%
\pgfsetdash{}{0pt}%
\pgfpathmoveto{\pgfqpoint{3.569925in}{3.124394in}}%
\pgfpathcurveto{\pgfqpoint{3.580975in}{3.124394in}}{\pgfqpoint{3.591574in}{3.128784in}}{\pgfqpoint{3.599387in}{3.136598in}}%
\pgfpathcurveto{\pgfqpoint{3.607201in}{3.144411in}}{\pgfqpoint{3.611591in}{3.155010in}}{\pgfqpoint{3.611591in}{3.166060in}}%
\pgfpathcurveto{\pgfqpoint{3.611591in}{3.177111in}}{\pgfqpoint{3.607201in}{3.187710in}}{\pgfqpoint{3.599387in}{3.195523in}}%
\pgfpathcurveto{\pgfqpoint{3.591574in}{3.203337in}}{\pgfqpoint{3.580975in}{3.207727in}}{\pgfqpoint{3.569925in}{3.207727in}}%
\pgfpathcurveto{\pgfqpoint{3.558875in}{3.207727in}}{\pgfqpoint{3.548276in}{3.203337in}}{\pgfqpoint{3.540462in}{3.195523in}}%
\pgfpathcurveto{\pgfqpoint{3.532648in}{3.187710in}}{\pgfqpoint{3.528258in}{3.177111in}}{\pgfqpoint{3.528258in}{3.166060in}}%
\pgfpathcurveto{\pgfqpoint{3.528258in}{3.155010in}}{\pgfqpoint{3.532648in}{3.144411in}}{\pgfqpoint{3.540462in}{3.136598in}}%
\pgfpathcurveto{\pgfqpoint{3.548276in}{3.128784in}}{\pgfqpoint{3.558875in}{3.124394in}}{\pgfqpoint{3.569925in}{3.124394in}}%
\pgfpathclose%
\pgfusepath{stroke,fill}%
\end{pgfscope}%
\begin{pgfscope}%
\pgfpathrectangle{\pgfqpoint{0.648703in}{0.548769in}}{\pgfqpoint{5.201297in}{3.102590in}}%
\pgfusepath{clip}%
\pgfsetbuttcap%
\pgfsetroundjoin%
\definecolor{currentfill}{rgb}{0.121569,0.466667,0.705882}%
\pgfsetfillcolor{currentfill}%
\pgfsetlinewidth{1.003750pt}%
\definecolor{currentstroke}{rgb}{0.121569,0.466667,0.705882}%
\pgfsetstrokecolor{currentstroke}%
\pgfsetdash{}{0pt}%
\pgfpathmoveto{\pgfqpoint{3.529853in}{3.132690in}}%
\pgfpathcurveto{\pgfqpoint{3.540903in}{3.132690in}}{\pgfqpoint{3.551502in}{3.137080in}}{\pgfqpoint{3.559316in}{3.144893in}}%
\pgfpathcurveto{\pgfqpoint{3.567129in}{3.152707in}}{\pgfqpoint{3.571520in}{3.163306in}}{\pgfqpoint{3.571520in}{3.174356in}}%
\pgfpathcurveto{\pgfqpoint{3.571520in}{3.185406in}}{\pgfqpoint{3.567129in}{3.196005in}}{\pgfqpoint{3.559316in}{3.203819in}}%
\pgfpathcurveto{\pgfqpoint{3.551502in}{3.211633in}}{\pgfqpoint{3.540903in}{3.216023in}}{\pgfqpoint{3.529853in}{3.216023in}}%
\pgfpathcurveto{\pgfqpoint{3.518803in}{3.216023in}}{\pgfqpoint{3.508204in}{3.211633in}}{\pgfqpoint{3.500390in}{3.203819in}}%
\pgfpathcurveto{\pgfqpoint{3.492577in}{3.196005in}}{\pgfqpoint{3.488186in}{3.185406in}}{\pgfqpoint{3.488186in}{3.174356in}}%
\pgfpathcurveto{\pgfqpoint{3.488186in}{3.163306in}}{\pgfqpoint{3.492577in}{3.152707in}}{\pgfqpoint{3.500390in}{3.144893in}}%
\pgfpathcurveto{\pgfqpoint{3.508204in}{3.137080in}}{\pgfqpoint{3.518803in}{3.132690in}}{\pgfqpoint{3.529853in}{3.132690in}}%
\pgfpathclose%
\pgfusepath{stroke,fill}%
\end{pgfscope}%
\begin{pgfscope}%
\pgfpathrectangle{\pgfqpoint{0.648703in}{0.548769in}}{\pgfqpoint{5.201297in}{3.102590in}}%
\pgfusepath{clip}%
\pgfsetbuttcap%
\pgfsetroundjoin%
\definecolor{currentfill}{rgb}{0.121569,0.466667,0.705882}%
\pgfsetfillcolor{currentfill}%
\pgfsetlinewidth{1.003750pt}%
\definecolor{currentstroke}{rgb}{0.121569,0.466667,0.705882}%
\pgfsetstrokecolor{currentstroke}%
\pgfsetdash{}{0pt}%
\pgfpathmoveto{\pgfqpoint{3.770283in}{3.132690in}}%
\pgfpathcurveto{\pgfqpoint{3.781333in}{3.132690in}}{\pgfqpoint{3.791932in}{3.137080in}}{\pgfqpoint{3.799746in}{3.144893in}}%
\pgfpathcurveto{\pgfqpoint{3.807559in}{3.152707in}}{\pgfqpoint{3.811949in}{3.163306in}}{\pgfqpoint{3.811949in}{3.174356in}}%
\pgfpathcurveto{\pgfqpoint{3.811949in}{3.185406in}}{\pgfqpoint{3.807559in}{3.196005in}}{\pgfqpoint{3.799746in}{3.203819in}}%
\pgfpathcurveto{\pgfqpoint{3.791932in}{3.211633in}}{\pgfqpoint{3.781333in}{3.216023in}}{\pgfqpoint{3.770283in}{3.216023in}}%
\pgfpathcurveto{\pgfqpoint{3.759233in}{3.216023in}}{\pgfqpoint{3.748634in}{3.211633in}}{\pgfqpoint{3.740820in}{3.203819in}}%
\pgfpathcurveto{\pgfqpoint{3.733006in}{3.196005in}}{\pgfqpoint{3.728616in}{3.185406in}}{\pgfqpoint{3.728616in}{3.174356in}}%
\pgfpathcurveto{\pgfqpoint{3.728616in}{3.163306in}}{\pgfqpoint{3.733006in}{3.152707in}}{\pgfqpoint{3.740820in}{3.144893in}}%
\pgfpathcurveto{\pgfqpoint{3.748634in}{3.137080in}}{\pgfqpoint{3.759233in}{3.132690in}}{\pgfqpoint{3.770283in}{3.132690in}}%
\pgfpathclose%
\pgfusepath{stroke,fill}%
\end{pgfscope}%
\begin{pgfscope}%
\pgfpathrectangle{\pgfqpoint{0.648703in}{0.548769in}}{\pgfqpoint{5.201297in}{3.102590in}}%
\pgfusepath{clip}%
\pgfsetbuttcap%
\pgfsetroundjoin%
\definecolor{currentfill}{rgb}{1.000000,0.498039,0.054902}%
\pgfsetfillcolor{currentfill}%
\pgfsetlinewidth{1.003750pt}%
\definecolor{currentstroke}{rgb}{1.000000,0.498039,0.054902}%
\pgfsetstrokecolor{currentstroke}%
\pgfsetdash{}{0pt}%
\pgfpathmoveto{\pgfqpoint{3.089065in}{3.136837in}}%
\pgfpathcurveto{\pgfqpoint{3.100115in}{3.136837in}}{\pgfqpoint{3.110714in}{3.141228in}}{\pgfqpoint{3.118528in}{3.149041in}}%
\pgfpathcurveto{\pgfqpoint{3.126342in}{3.156855in}}{\pgfqpoint{3.130732in}{3.167454in}}{\pgfqpoint{3.130732in}{3.178504in}}%
\pgfpathcurveto{\pgfqpoint{3.130732in}{3.189554in}}{\pgfqpoint{3.126342in}{3.200153in}}{\pgfqpoint{3.118528in}{3.207967in}}%
\pgfpathcurveto{\pgfqpoint{3.110714in}{3.215780in}}{\pgfqpoint{3.100115in}{3.220171in}}{\pgfqpoint{3.089065in}{3.220171in}}%
\pgfpathcurveto{\pgfqpoint{3.078015in}{3.220171in}}{\pgfqpoint{3.067416in}{3.215780in}}{\pgfqpoint{3.059602in}{3.207967in}}%
\pgfpathcurveto{\pgfqpoint{3.051789in}{3.200153in}}{\pgfqpoint{3.047399in}{3.189554in}}{\pgfqpoint{3.047399in}{3.178504in}}%
\pgfpathcurveto{\pgfqpoint{3.047399in}{3.167454in}}{\pgfqpoint{3.051789in}{3.156855in}}{\pgfqpoint{3.059602in}{3.149041in}}%
\pgfpathcurveto{\pgfqpoint{3.067416in}{3.141228in}}{\pgfqpoint{3.078015in}{3.136837in}}{\pgfqpoint{3.089065in}{3.136837in}}%
\pgfpathclose%
\pgfusepath{stroke,fill}%
\end{pgfscope}%
\begin{pgfscope}%
\pgfpathrectangle{\pgfqpoint{0.648703in}{0.548769in}}{\pgfqpoint{5.201297in}{3.102590in}}%
\pgfusepath{clip}%
\pgfsetbuttcap%
\pgfsetroundjoin%
\definecolor{currentfill}{rgb}{1.000000,0.498039,0.054902}%
\pgfsetfillcolor{currentfill}%
\pgfsetlinewidth{1.003750pt}%
\definecolor{currentstroke}{rgb}{1.000000,0.498039,0.054902}%
\pgfsetstrokecolor{currentstroke}%
\pgfsetdash{}{0pt}%
\pgfpathmoveto{\pgfqpoint{3.249352in}{3.244681in}}%
\pgfpathcurveto{\pgfqpoint{3.260402in}{3.244681in}}{\pgfqpoint{3.271001in}{3.249072in}}{\pgfqpoint{3.278814in}{3.256885in}}%
\pgfpathcurveto{\pgfqpoint{3.286628in}{3.264699in}}{\pgfqpoint{3.291018in}{3.275298in}}{\pgfqpoint{3.291018in}{3.286348in}}%
\pgfpathcurveto{\pgfqpoint{3.291018in}{3.297398in}}{\pgfqpoint{3.286628in}{3.307997in}}{\pgfqpoint{3.278814in}{3.315811in}}%
\pgfpathcurveto{\pgfqpoint{3.271001in}{3.323624in}}{\pgfqpoint{3.260402in}{3.328015in}}{\pgfqpoint{3.249352in}{3.328015in}}%
\pgfpathcurveto{\pgfqpoint{3.238302in}{3.328015in}}{\pgfqpoint{3.227703in}{3.323624in}}{\pgfqpoint{3.219889in}{3.315811in}}%
\pgfpathcurveto{\pgfqpoint{3.212075in}{3.307997in}}{\pgfqpoint{3.207685in}{3.297398in}}{\pgfqpoint{3.207685in}{3.286348in}}%
\pgfpathcurveto{\pgfqpoint{3.207685in}{3.275298in}}{\pgfqpoint{3.212075in}{3.264699in}}{\pgfqpoint{3.219889in}{3.256885in}}%
\pgfpathcurveto{\pgfqpoint{3.227703in}{3.249072in}}{\pgfqpoint{3.238302in}{3.244681in}}{\pgfqpoint{3.249352in}{3.244681in}}%
\pgfpathclose%
\pgfusepath{stroke,fill}%
\end{pgfscope}%
\begin{pgfscope}%
\pgfpathrectangle{\pgfqpoint{0.648703in}{0.548769in}}{\pgfqpoint{5.201297in}{3.102590in}}%
\pgfusepath{clip}%
\pgfsetbuttcap%
\pgfsetroundjoin%
\definecolor{currentfill}{rgb}{1.000000,0.498039,0.054902}%
\pgfsetfillcolor{currentfill}%
\pgfsetlinewidth{1.003750pt}%
\definecolor{currentstroke}{rgb}{1.000000,0.498039,0.054902}%
\pgfsetstrokecolor{currentstroke}%
\pgfsetdash{}{0pt}%
\pgfpathmoveto{\pgfqpoint{4.050784in}{3.190759in}}%
\pgfpathcurveto{\pgfqpoint{4.061834in}{3.190759in}}{\pgfqpoint{4.072433in}{3.195150in}}{\pgfqpoint{4.080247in}{3.202963in}}%
\pgfpathcurveto{\pgfqpoint{4.088061in}{3.210777in}}{\pgfqpoint{4.092451in}{3.221376in}}{\pgfqpoint{4.092451in}{3.232426in}}%
\pgfpathcurveto{\pgfqpoint{4.092451in}{3.243476in}}{\pgfqpoint{4.088061in}{3.254075in}}{\pgfqpoint{4.080247in}{3.261889in}}%
\pgfpathcurveto{\pgfqpoint{4.072433in}{3.269702in}}{\pgfqpoint{4.061834in}{3.274093in}}{\pgfqpoint{4.050784in}{3.274093in}}%
\pgfpathcurveto{\pgfqpoint{4.039734in}{3.274093in}}{\pgfqpoint{4.029135in}{3.269702in}}{\pgfqpoint{4.021321in}{3.261889in}}%
\pgfpathcurveto{\pgfqpoint{4.013508in}{3.254075in}}{\pgfqpoint{4.009117in}{3.243476in}}{\pgfqpoint{4.009117in}{3.232426in}}%
\pgfpathcurveto{\pgfqpoint{4.009117in}{3.221376in}}{\pgfqpoint{4.013508in}{3.210777in}}{\pgfqpoint{4.021321in}{3.202963in}}%
\pgfpathcurveto{\pgfqpoint{4.029135in}{3.195150in}}{\pgfqpoint{4.039734in}{3.190759in}}{\pgfqpoint{4.050784in}{3.190759in}}%
\pgfpathclose%
\pgfusepath{stroke,fill}%
\end{pgfscope}%
\begin{pgfscope}%
\pgfpathrectangle{\pgfqpoint{0.648703in}{0.548769in}}{\pgfqpoint{5.201297in}{3.102590in}}%
\pgfusepath{clip}%
\pgfsetbuttcap%
\pgfsetroundjoin%
\definecolor{currentfill}{rgb}{0.121569,0.466667,0.705882}%
\pgfsetfillcolor{currentfill}%
\pgfsetlinewidth{1.003750pt}%
\definecolor{currentstroke}{rgb}{0.121569,0.466667,0.705882}%
\pgfsetstrokecolor{currentstroke}%
\pgfsetdash{}{0pt}%
\pgfpathmoveto{\pgfqpoint{3.569925in}{0.664720in}}%
\pgfpathcurveto{\pgfqpoint{3.580975in}{0.664720in}}{\pgfqpoint{3.591574in}{0.669111in}}{\pgfqpoint{3.599387in}{0.676924in}}%
\pgfpathcurveto{\pgfqpoint{3.607201in}{0.684738in}}{\pgfqpoint{3.611591in}{0.695337in}}{\pgfqpoint{3.611591in}{0.706387in}}%
\pgfpathcurveto{\pgfqpoint{3.611591in}{0.717437in}}{\pgfqpoint{3.607201in}{0.728036in}}{\pgfqpoint{3.599387in}{0.735850in}}%
\pgfpathcurveto{\pgfqpoint{3.591574in}{0.743663in}}{\pgfqpoint{3.580975in}{0.748054in}}{\pgfqpoint{3.569925in}{0.748054in}}%
\pgfpathcurveto{\pgfqpoint{3.558875in}{0.748054in}}{\pgfqpoint{3.548276in}{0.743663in}}{\pgfqpoint{3.540462in}{0.735850in}}%
\pgfpathcurveto{\pgfqpoint{3.532648in}{0.728036in}}{\pgfqpoint{3.528258in}{0.717437in}}{\pgfqpoint{3.528258in}{0.706387in}}%
\pgfpathcurveto{\pgfqpoint{3.528258in}{0.695337in}}{\pgfqpoint{3.532648in}{0.684738in}}{\pgfqpoint{3.540462in}{0.676924in}}%
\pgfpathcurveto{\pgfqpoint{3.548276in}{0.669111in}}{\pgfqpoint{3.558875in}{0.664720in}}{\pgfqpoint{3.569925in}{0.664720in}}%
\pgfpathclose%
\pgfusepath{stroke,fill}%
\end{pgfscope}%
\begin{pgfscope}%
\pgfpathrectangle{\pgfqpoint{0.648703in}{0.548769in}}{\pgfqpoint{5.201297in}{3.102590in}}%
\pgfusepath{clip}%
\pgfsetbuttcap%
\pgfsetroundjoin%
\definecolor{currentfill}{rgb}{1.000000,0.498039,0.054902}%
\pgfsetfillcolor{currentfill}%
\pgfsetlinewidth{1.003750pt}%
\definecolor{currentstroke}{rgb}{1.000000,0.498039,0.054902}%
\pgfsetstrokecolor{currentstroke}%
\pgfsetdash{}{0pt}%
\pgfpathmoveto{\pgfqpoint{3.409638in}{3.145133in}}%
\pgfpathcurveto{\pgfqpoint{3.420688in}{3.145133in}}{\pgfqpoint{3.431287in}{3.149523in}}{\pgfqpoint{3.439101in}{3.157337in}}%
\pgfpathcurveto{\pgfqpoint{3.446915in}{3.165151in}}{\pgfqpoint{3.451305in}{3.175750in}}{\pgfqpoint{3.451305in}{3.186800in}}%
\pgfpathcurveto{\pgfqpoint{3.451305in}{3.197850in}}{\pgfqpoint{3.446915in}{3.208449in}}{\pgfqpoint{3.439101in}{3.216262in}}%
\pgfpathcurveto{\pgfqpoint{3.431287in}{3.224076in}}{\pgfqpoint{3.420688in}{3.228466in}}{\pgfqpoint{3.409638in}{3.228466in}}%
\pgfpathcurveto{\pgfqpoint{3.398588in}{3.228466in}}{\pgfqpoint{3.387989in}{3.224076in}}{\pgfqpoint{3.380175in}{3.216262in}}%
\pgfpathcurveto{\pgfqpoint{3.372362in}{3.208449in}}{\pgfqpoint{3.367972in}{3.197850in}}{\pgfqpoint{3.367972in}{3.186800in}}%
\pgfpathcurveto{\pgfqpoint{3.367972in}{3.175750in}}{\pgfqpoint{3.372362in}{3.165151in}}{\pgfqpoint{3.380175in}{3.157337in}}%
\pgfpathcurveto{\pgfqpoint{3.387989in}{3.149523in}}{\pgfqpoint{3.398588in}{3.145133in}}{\pgfqpoint{3.409638in}{3.145133in}}%
\pgfpathclose%
\pgfusepath{stroke,fill}%
\end{pgfscope}%
\begin{pgfscope}%
\pgfpathrectangle{\pgfqpoint{0.648703in}{0.548769in}}{\pgfqpoint{5.201297in}{3.102590in}}%
\pgfusepath{clip}%
\pgfsetbuttcap%
\pgfsetroundjoin%
\definecolor{currentfill}{rgb}{0.121569,0.466667,0.705882}%
\pgfsetfillcolor{currentfill}%
\pgfsetlinewidth{1.003750pt}%
\definecolor{currentstroke}{rgb}{0.121569,0.466667,0.705882}%
\pgfsetstrokecolor{currentstroke}%
\pgfsetdash{}{0pt}%
\pgfpathmoveto{\pgfqpoint{3.449710in}{0.648129in}}%
\pgfpathcurveto{\pgfqpoint{3.460760in}{0.648129in}}{\pgfqpoint{3.471359in}{0.652519in}}{\pgfqpoint{3.479173in}{0.660333in}}%
\pgfpathcurveto{\pgfqpoint{3.486986in}{0.668146in}}{\pgfqpoint{3.491376in}{0.678745in}}{\pgfqpoint{3.491376in}{0.689796in}}%
\pgfpathcurveto{\pgfqpoint{3.491376in}{0.700846in}}{\pgfqpoint{3.486986in}{0.711445in}}{\pgfqpoint{3.479173in}{0.719258in}}%
\pgfpathcurveto{\pgfqpoint{3.471359in}{0.727072in}}{\pgfqpoint{3.460760in}{0.731462in}}{\pgfqpoint{3.449710in}{0.731462in}}%
\pgfpathcurveto{\pgfqpoint{3.438660in}{0.731462in}}{\pgfqpoint{3.428061in}{0.727072in}}{\pgfqpoint{3.420247in}{0.719258in}}%
\pgfpathcurveto{\pgfqpoint{3.412433in}{0.711445in}}{\pgfqpoint{3.408043in}{0.700846in}}{\pgfqpoint{3.408043in}{0.689796in}}%
\pgfpathcurveto{\pgfqpoint{3.408043in}{0.678745in}}{\pgfqpoint{3.412433in}{0.668146in}}{\pgfqpoint{3.420247in}{0.660333in}}%
\pgfpathcurveto{\pgfqpoint{3.428061in}{0.652519in}}{\pgfqpoint{3.438660in}{0.648129in}}{\pgfqpoint{3.449710in}{0.648129in}}%
\pgfpathclose%
\pgfusepath{stroke,fill}%
\end{pgfscope}%
\begin{pgfscope}%
\pgfpathrectangle{\pgfqpoint{0.648703in}{0.548769in}}{\pgfqpoint{5.201297in}{3.102590in}}%
\pgfusepath{clip}%
\pgfsetbuttcap%
\pgfsetroundjoin%
\definecolor{currentfill}{rgb}{1.000000,0.498039,0.054902}%
\pgfsetfillcolor{currentfill}%
\pgfsetlinewidth{1.003750pt}%
\definecolor{currentstroke}{rgb}{1.000000,0.498039,0.054902}%
\pgfsetstrokecolor{currentstroke}%
\pgfsetdash{}{0pt}%
\pgfpathmoveto{\pgfqpoint{4.291214in}{3.149281in}}%
\pgfpathcurveto{\pgfqpoint{4.302264in}{3.149281in}}{\pgfqpoint{4.312863in}{3.153671in}}{\pgfqpoint{4.320677in}{3.161485in}}%
\pgfpathcurveto{\pgfqpoint{4.328490in}{3.169298in}}{\pgfqpoint{4.332881in}{3.179897in}}{\pgfqpoint{4.332881in}{3.190948in}}%
\pgfpathcurveto{\pgfqpoint{4.332881in}{3.201998in}}{\pgfqpoint{4.328490in}{3.212597in}}{\pgfqpoint{4.320677in}{3.220410in}}%
\pgfpathcurveto{\pgfqpoint{4.312863in}{3.228224in}}{\pgfqpoint{4.302264in}{3.232614in}}{\pgfqpoint{4.291214in}{3.232614in}}%
\pgfpathcurveto{\pgfqpoint{4.280164in}{3.232614in}}{\pgfqpoint{4.269565in}{3.228224in}}{\pgfqpoint{4.261751in}{3.220410in}}%
\pgfpathcurveto{\pgfqpoint{4.253937in}{3.212597in}}{\pgfqpoint{4.249547in}{3.201998in}}{\pgfqpoint{4.249547in}{3.190948in}}%
\pgfpathcurveto{\pgfqpoint{4.249547in}{3.179897in}}{\pgfqpoint{4.253937in}{3.169298in}}{\pgfqpoint{4.261751in}{3.161485in}}%
\pgfpathcurveto{\pgfqpoint{4.269565in}{3.153671in}}{\pgfqpoint{4.280164in}{3.149281in}}{\pgfqpoint{4.291214in}{3.149281in}}%
\pgfpathclose%
\pgfusepath{stroke,fill}%
\end{pgfscope}%
\begin{pgfscope}%
\pgfpathrectangle{\pgfqpoint{0.648703in}{0.548769in}}{\pgfqpoint{5.201297in}{3.102590in}}%
\pgfusepath{clip}%
\pgfsetbuttcap%
\pgfsetroundjoin%
\definecolor{currentfill}{rgb}{0.121569,0.466667,0.705882}%
\pgfsetfillcolor{currentfill}%
\pgfsetlinewidth{1.003750pt}%
\definecolor{currentstroke}{rgb}{0.121569,0.466667,0.705882}%
\pgfsetstrokecolor{currentstroke}%
\pgfsetdash{}{0pt}%
\pgfpathmoveto{\pgfqpoint{3.089065in}{3.124394in}}%
\pgfpathcurveto{\pgfqpoint{3.100115in}{3.124394in}}{\pgfqpoint{3.110714in}{3.128784in}}{\pgfqpoint{3.118528in}{3.136598in}}%
\pgfpathcurveto{\pgfqpoint{3.126342in}{3.144411in}}{\pgfqpoint{3.130732in}{3.155010in}}{\pgfqpoint{3.130732in}{3.166060in}}%
\pgfpathcurveto{\pgfqpoint{3.130732in}{3.177111in}}{\pgfqpoint{3.126342in}{3.187710in}}{\pgfqpoint{3.118528in}{3.195523in}}%
\pgfpathcurveto{\pgfqpoint{3.110714in}{3.203337in}}{\pgfqpoint{3.100115in}{3.207727in}}{\pgfqpoint{3.089065in}{3.207727in}}%
\pgfpathcurveto{\pgfqpoint{3.078015in}{3.207727in}}{\pgfqpoint{3.067416in}{3.203337in}}{\pgfqpoint{3.059602in}{3.195523in}}%
\pgfpathcurveto{\pgfqpoint{3.051789in}{3.187710in}}{\pgfqpoint{3.047399in}{3.177111in}}{\pgfqpoint{3.047399in}{3.166060in}}%
\pgfpathcurveto{\pgfqpoint{3.047399in}{3.155010in}}{\pgfqpoint{3.051789in}{3.144411in}}{\pgfqpoint{3.059602in}{3.136598in}}%
\pgfpathcurveto{\pgfqpoint{3.067416in}{3.128784in}}{\pgfqpoint{3.078015in}{3.124394in}}{\pgfqpoint{3.089065in}{3.124394in}}%
\pgfpathclose%
\pgfusepath{stroke,fill}%
\end{pgfscope}%
\begin{pgfscope}%
\pgfpathrectangle{\pgfqpoint{0.648703in}{0.548769in}}{\pgfqpoint{5.201297in}{3.102590in}}%
\pgfusepath{clip}%
\pgfsetbuttcap%
\pgfsetroundjoin%
\definecolor{currentfill}{rgb}{1.000000,0.498039,0.054902}%
\pgfsetfillcolor{currentfill}%
\pgfsetlinewidth{1.003750pt}%
\definecolor{currentstroke}{rgb}{1.000000,0.498039,0.054902}%
\pgfsetstrokecolor{currentstroke}%
\pgfsetdash{}{0pt}%
\pgfpathmoveto{\pgfqpoint{3.209280in}{3.136837in}}%
\pgfpathcurveto{\pgfqpoint{3.220330in}{3.136837in}}{\pgfqpoint{3.230929in}{3.141228in}}{\pgfqpoint{3.238743in}{3.149041in}}%
\pgfpathcurveto{\pgfqpoint{3.246556in}{3.156855in}}{\pgfqpoint{3.250947in}{3.167454in}}{\pgfqpoint{3.250947in}{3.178504in}}%
\pgfpathcurveto{\pgfqpoint{3.250947in}{3.189554in}}{\pgfqpoint{3.246556in}{3.200153in}}{\pgfqpoint{3.238743in}{3.207967in}}%
\pgfpathcurveto{\pgfqpoint{3.230929in}{3.215780in}}{\pgfqpoint{3.220330in}{3.220171in}}{\pgfqpoint{3.209280in}{3.220171in}}%
\pgfpathcurveto{\pgfqpoint{3.198230in}{3.220171in}}{\pgfqpoint{3.187631in}{3.215780in}}{\pgfqpoint{3.179817in}{3.207967in}}%
\pgfpathcurveto{\pgfqpoint{3.172004in}{3.200153in}}{\pgfqpoint{3.167613in}{3.189554in}}{\pgfqpoint{3.167613in}{3.178504in}}%
\pgfpathcurveto{\pgfqpoint{3.167613in}{3.167454in}}{\pgfqpoint{3.172004in}{3.156855in}}{\pgfqpoint{3.179817in}{3.149041in}}%
\pgfpathcurveto{\pgfqpoint{3.187631in}{3.141228in}}{\pgfqpoint{3.198230in}{3.136837in}}{\pgfqpoint{3.209280in}{3.136837in}}%
\pgfpathclose%
\pgfusepath{stroke,fill}%
\end{pgfscope}%
\begin{pgfscope}%
\pgfpathrectangle{\pgfqpoint{0.648703in}{0.548769in}}{\pgfqpoint{5.201297in}{3.102590in}}%
\pgfusepath{clip}%
\pgfsetbuttcap%
\pgfsetroundjoin%
\definecolor{currentfill}{rgb}{1.000000,0.498039,0.054902}%
\pgfsetfillcolor{currentfill}%
\pgfsetlinewidth{1.003750pt}%
\definecolor{currentstroke}{rgb}{1.000000,0.498039,0.054902}%
\pgfsetstrokecolor{currentstroke}%
\pgfsetdash{}{0pt}%
\pgfpathmoveto{\pgfqpoint{3.129137in}{3.140985in}}%
\pgfpathcurveto{\pgfqpoint{3.140187in}{3.140985in}}{\pgfqpoint{3.150786in}{3.145375in}}{\pgfqpoint{3.158600in}{3.153189in}}%
\pgfpathcurveto{\pgfqpoint{3.166413in}{3.161003in}}{\pgfqpoint{3.170804in}{3.171602in}}{\pgfqpoint{3.170804in}{3.182652in}}%
\pgfpathcurveto{\pgfqpoint{3.170804in}{3.193702in}}{\pgfqpoint{3.166413in}{3.204301in}}{\pgfqpoint{3.158600in}{3.212115in}}%
\pgfpathcurveto{\pgfqpoint{3.150786in}{3.219928in}}{\pgfqpoint{3.140187in}{3.224319in}}{\pgfqpoint{3.129137in}{3.224319in}}%
\pgfpathcurveto{\pgfqpoint{3.118087in}{3.224319in}}{\pgfqpoint{3.107488in}{3.219928in}}{\pgfqpoint{3.099674in}{3.212115in}}%
\pgfpathcurveto{\pgfqpoint{3.091860in}{3.204301in}}{\pgfqpoint{3.087470in}{3.193702in}}{\pgfqpoint{3.087470in}{3.182652in}}%
\pgfpathcurveto{\pgfqpoint{3.087470in}{3.171602in}}{\pgfqpoint{3.091860in}{3.161003in}}{\pgfqpoint{3.099674in}{3.153189in}}%
\pgfpathcurveto{\pgfqpoint{3.107488in}{3.145375in}}{\pgfqpoint{3.118087in}{3.140985in}}{\pgfqpoint{3.129137in}{3.140985in}}%
\pgfpathclose%
\pgfusepath{stroke,fill}%
\end{pgfscope}%
\begin{pgfscope}%
\pgfpathrectangle{\pgfqpoint{0.648703in}{0.548769in}}{\pgfqpoint{5.201297in}{3.102590in}}%
\pgfusepath{clip}%
\pgfsetbuttcap%
\pgfsetroundjoin%
\definecolor{currentfill}{rgb}{1.000000,0.498039,0.054902}%
\pgfsetfillcolor{currentfill}%
\pgfsetlinewidth{1.003750pt}%
\definecolor{currentstroke}{rgb}{1.000000,0.498039,0.054902}%
\pgfsetstrokecolor{currentstroke}%
\pgfsetdash{}{0pt}%
\pgfpathmoveto{\pgfqpoint{3.129137in}{3.145133in}}%
\pgfpathcurveto{\pgfqpoint{3.140187in}{3.145133in}}{\pgfqpoint{3.150786in}{3.149523in}}{\pgfqpoint{3.158600in}{3.157337in}}%
\pgfpathcurveto{\pgfqpoint{3.166413in}{3.165151in}}{\pgfqpoint{3.170804in}{3.175750in}}{\pgfqpoint{3.170804in}{3.186800in}}%
\pgfpathcurveto{\pgfqpoint{3.170804in}{3.197850in}}{\pgfqpoint{3.166413in}{3.208449in}}{\pgfqpoint{3.158600in}{3.216262in}}%
\pgfpathcurveto{\pgfqpoint{3.150786in}{3.224076in}}{\pgfqpoint{3.140187in}{3.228466in}}{\pgfqpoint{3.129137in}{3.228466in}}%
\pgfpathcurveto{\pgfqpoint{3.118087in}{3.228466in}}{\pgfqpoint{3.107488in}{3.224076in}}{\pgfqpoint{3.099674in}{3.216262in}}%
\pgfpathcurveto{\pgfqpoint{3.091860in}{3.208449in}}{\pgfqpoint{3.087470in}{3.197850in}}{\pgfqpoint{3.087470in}{3.186800in}}%
\pgfpathcurveto{\pgfqpoint{3.087470in}{3.175750in}}{\pgfqpoint{3.091860in}{3.165151in}}{\pgfqpoint{3.099674in}{3.157337in}}%
\pgfpathcurveto{\pgfqpoint{3.107488in}{3.149523in}}{\pgfqpoint{3.118087in}{3.145133in}}{\pgfqpoint{3.129137in}{3.145133in}}%
\pgfpathclose%
\pgfusepath{stroke,fill}%
\end{pgfscope}%
\begin{pgfscope}%
\pgfpathrectangle{\pgfqpoint{0.648703in}{0.548769in}}{\pgfqpoint{5.201297in}{3.102590in}}%
\pgfusepath{clip}%
\pgfsetbuttcap%
\pgfsetroundjoin%
\definecolor{currentfill}{rgb}{0.121569,0.466667,0.705882}%
\pgfsetfillcolor{currentfill}%
\pgfsetlinewidth{1.003750pt}%
\definecolor{currentstroke}{rgb}{0.121569,0.466667,0.705882}%
\pgfsetstrokecolor{currentstroke}%
\pgfsetdash{}{0pt}%
\pgfpathmoveto{\pgfqpoint{3.169208in}{3.132690in}}%
\pgfpathcurveto{\pgfqpoint{3.180259in}{3.132690in}}{\pgfqpoint{3.190858in}{3.137080in}}{\pgfqpoint{3.198671in}{3.144893in}}%
\pgfpathcurveto{\pgfqpoint{3.206485in}{3.152707in}}{\pgfqpoint{3.210875in}{3.163306in}}{\pgfqpoint{3.210875in}{3.174356in}}%
\pgfpathcurveto{\pgfqpoint{3.210875in}{3.185406in}}{\pgfqpoint{3.206485in}{3.196005in}}{\pgfqpoint{3.198671in}{3.203819in}}%
\pgfpathcurveto{\pgfqpoint{3.190858in}{3.211633in}}{\pgfqpoint{3.180259in}{3.216023in}}{\pgfqpoint{3.169208in}{3.216023in}}%
\pgfpathcurveto{\pgfqpoint{3.158158in}{3.216023in}}{\pgfqpoint{3.147559in}{3.211633in}}{\pgfqpoint{3.139746in}{3.203819in}}%
\pgfpathcurveto{\pgfqpoint{3.131932in}{3.196005in}}{\pgfqpoint{3.127542in}{3.185406in}}{\pgfqpoint{3.127542in}{3.174356in}}%
\pgfpathcurveto{\pgfqpoint{3.127542in}{3.163306in}}{\pgfqpoint{3.131932in}{3.152707in}}{\pgfqpoint{3.139746in}{3.144893in}}%
\pgfpathcurveto{\pgfqpoint{3.147559in}{3.137080in}}{\pgfqpoint{3.158158in}{3.132690in}}{\pgfqpoint{3.169208in}{3.132690in}}%
\pgfpathclose%
\pgfusepath{stroke,fill}%
\end{pgfscope}%
\begin{pgfscope}%
\pgfpathrectangle{\pgfqpoint{0.648703in}{0.548769in}}{\pgfqpoint{5.201297in}{3.102590in}}%
\pgfusepath{clip}%
\pgfsetbuttcap%
\pgfsetroundjoin%
\definecolor{currentfill}{rgb}{0.121569,0.466667,0.705882}%
\pgfsetfillcolor{currentfill}%
\pgfsetlinewidth{1.003750pt}%
\definecolor{currentstroke}{rgb}{0.121569,0.466667,0.705882}%
\pgfsetstrokecolor{currentstroke}%
\pgfsetdash{}{0pt}%
\pgfpathmoveto{\pgfqpoint{3.529853in}{0.660572in}}%
\pgfpathcurveto{\pgfqpoint{3.540903in}{0.660572in}}{\pgfqpoint{3.551502in}{0.664963in}}{\pgfqpoint{3.559316in}{0.672776in}}%
\pgfpathcurveto{\pgfqpoint{3.567129in}{0.680590in}}{\pgfqpoint{3.571520in}{0.691189in}}{\pgfqpoint{3.571520in}{0.702239in}}%
\pgfpathcurveto{\pgfqpoint{3.571520in}{0.713289in}}{\pgfqpoint{3.567129in}{0.723888in}}{\pgfqpoint{3.559316in}{0.731702in}}%
\pgfpathcurveto{\pgfqpoint{3.551502in}{0.739516in}}{\pgfqpoint{3.540903in}{0.743906in}}{\pgfqpoint{3.529853in}{0.743906in}}%
\pgfpathcurveto{\pgfqpoint{3.518803in}{0.743906in}}{\pgfqpoint{3.508204in}{0.739516in}}{\pgfqpoint{3.500390in}{0.731702in}}%
\pgfpathcurveto{\pgfqpoint{3.492577in}{0.723888in}}{\pgfqpoint{3.488186in}{0.713289in}}{\pgfqpoint{3.488186in}{0.702239in}}%
\pgfpathcurveto{\pgfqpoint{3.488186in}{0.691189in}}{\pgfqpoint{3.492577in}{0.680590in}}{\pgfqpoint{3.500390in}{0.672776in}}%
\pgfpathcurveto{\pgfqpoint{3.508204in}{0.664963in}}{\pgfqpoint{3.518803in}{0.660572in}}{\pgfqpoint{3.529853in}{0.660572in}}%
\pgfpathclose%
\pgfusepath{stroke,fill}%
\end{pgfscope}%
\begin{pgfscope}%
\pgfpathrectangle{\pgfqpoint{0.648703in}{0.548769in}}{\pgfqpoint{5.201297in}{3.102590in}}%
\pgfusepath{clip}%
\pgfsetbuttcap%
\pgfsetroundjoin%
\definecolor{currentfill}{rgb}{1.000000,0.498039,0.054902}%
\pgfsetfillcolor{currentfill}%
\pgfsetlinewidth{1.003750pt}%
\definecolor{currentstroke}{rgb}{1.000000,0.498039,0.054902}%
\pgfsetstrokecolor{currentstroke}%
\pgfsetdash{}{0pt}%
\pgfpathmoveto{\pgfqpoint{3.289423in}{3.136837in}}%
\pgfpathcurveto{\pgfqpoint{3.300473in}{3.136837in}}{\pgfqpoint{3.311072in}{3.141228in}}{\pgfqpoint{3.318886in}{3.149041in}}%
\pgfpathcurveto{\pgfqpoint{3.326700in}{3.156855in}}{\pgfqpoint{3.331090in}{3.167454in}}{\pgfqpoint{3.331090in}{3.178504in}}%
\pgfpathcurveto{\pgfqpoint{3.331090in}{3.189554in}}{\pgfqpoint{3.326700in}{3.200153in}}{\pgfqpoint{3.318886in}{3.207967in}}%
\pgfpathcurveto{\pgfqpoint{3.311072in}{3.215780in}}{\pgfqpoint{3.300473in}{3.220171in}}{\pgfqpoint{3.289423in}{3.220171in}}%
\pgfpathcurveto{\pgfqpoint{3.278373in}{3.220171in}}{\pgfqpoint{3.267774in}{3.215780in}}{\pgfqpoint{3.259961in}{3.207967in}}%
\pgfpathcurveto{\pgfqpoint{3.252147in}{3.200153in}}{\pgfqpoint{3.247757in}{3.189554in}}{\pgfqpoint{3.247757in}{3.178504in}}%
\pgfpathcurveto{\pgfqpoint{3.247757in}{3.167454in}}{\pgfqpoint{3.252147in}{3.156855in}}{\pgfqpoint{3.259961in}{3.149041in}}%
\pgfpathcurveto{\pgfqpoint{3.267774in}{3.141228in}}{\pgfqpoint{3.278373in}{3.136837in}}{\pgfqpoint{3.289423in}{3.136837in}}%
\pgfpathclose%
\pgfusepath{stroke,fill}%
\end{pgfscope}%
\begin{pgfscope}%
\pgfpathrectangle{\pgfqpoint{0.648703in}{0.548769in}}{\pgfqpoint{5.201297in}{3.102590in}}%
\pgfusepath{clip}%
\pgfsetbuttcap%
\pgfsetroundjoin%
\definecolor{currentfill}{rgb}{1.000000,0.498039,0.054902}%
\pgfsetfillcolor{currentfill}%
\pgfsetlinewidth{1.003750pt}%
\definecolor{currentstroke}{rgb}{1.000000,0.498039,0.054902}%
\pgfsetstrokecolor{currentstroke}%
\pgfsetdash{}{0pt}%
\pgfpathmoveto{\pgfqpoint{3.449710in}{3.157577in}}%
\pgfpathcurveto{\pgfqpoint{3.460760in}{3.157577in}}{\pgfqpoint{3.471359in}{3.161967in}}{\pgfqpoint{3.479173in}{3.169780in}}%
\pgfpathcurveto{\pgfqpoint{3.486986in}{3.177594in}}{\pgfqpoint{3.491376in}{3.188193in}}{\pgfqpoint{3.491376in}{3.199243in}}%
\pgfpathcurveto{\pgfqpoint{3.491376in}{3.210293in}}{\pgfqpoint{3.486986in}{3.220892in}}{\pgfqpoint{3.479173in}{3.228706in}}%
\pgfpathcurveto{\pgfqpoint{3.471359in}{3.236520in}}{\pgfqpoint{3.460760in}{3.240910in}}{\pgfqpoint{3.449710in}{3.240910in}}%
\pgfpathcurveto{\pgfqpoint{3.438660in}{3.240910in}}{\pgfqpoint{3.428061in}{3.236520in}}{\pgfqpoint{3.420247in}{3.228706in}}%
\pgfpathcurveto{\pgfqpoint{3.412433in}{3.220892in}}{\pgfqpoint{3.408043in}{3.210293in}}{\pgfqpoint{3.408043in}{3.199243in}}%
\pgfpathcurveto{\pgfqpoint{3.408043in}{3.188193in}}{\pgfqpoint{3.412433in}{3.177594in}}{\pgfqpoint{3.420247in}{3.169780in}}%
\pgfpathcurveto{\pgfqpoint{3.428061in}{3.161967in}}{\pgfqpoint{3.438660in}{3.157577in}}{\pgfqpoint{3.449710in}{3.157577in}}%
\pgfpathclose%
\pgfusepath{stroke,fill}%
\end{pgfscope}%
\begin{pgfscope}%
\pgfpathrectangle{\pgfqpoint{0.648703in}{0.548769in}}{\pgfqpoint{5.201297in}{3.102590in}}%
\pgfusepath{clip}%
\pgfsetbuttcap%
\pgfsetroundjoin%
\definecolor{currentfill}{rgb}{0.121569,0.466667,0.705882}%
\pgfsetfillcolor{currentfill}%
\pgfsetlinewidth{1.003750pt}%
\definecolor{currentstroke}{rgb}{0.121569,0.466667,0.705882}%
\pgfsetstrokecolor{currentstroke}%
\pgfsetdash{}{0pt}%
\pgfpathmoveto{\pgfqpoint{4.010713in}{0.648129in}}%
\pgfpathcurveto{\pgfqpoint{4.021763in}{0.648129in}}{\pgfqpoint{4.032362in}{0.652519in}}{\pgfqpoint{4.040175in}{0.660333in}}%
\pgfpathcurveto{\pgfqpoint{4.047989in}{0.668146in}}{\pgfqpoint{4.052379in}{0.678745in}}{\pgfqpoint{4.052379in}{0.689796in}}%
\pgfpathcurveto{\pgfqpoint{4.052379in}{0.700846in}}{\pgfqpoint{4.047989in}{0.711445in}}{\pgfqpoint{4.040175in}{0.719258in}}%
\pgfpathcurveto{\pgfqpoint{4.032362in}{0.727072in}}{\pgfqpoint{4.021763in}{0.731462in}}{\pgfqpoint{4.010713in}{0.731462in}}%
\pgfpathcurveto{\pgfqpoint{3.999662in}{0.731462in}}{\pgfqpoint{3.989063in}{0.727072in}}{\pgfqpoint{3.981250in}{0.719258in}}%
\pgfpathcurveto{\pgfqpoint{3.973436in}{0.711445in}}{\pgfqpoint{3.969046in}{0.700846in}}{\pgfqpoint{3.969046in}{0.689796in}}%
\pgfpathcurveto{\pgfqpoint{3.969046in}{0.678745in}}{\pgfqpoint{3.973436in}{0.668146in}}{\pgfqpoint{3.981250in}{0.660333in}}%
\pgfpathcurveto{\pgfqpoint{3.989063in}{0.652519in}}{\pgfqpoint{3.999662in}{0.648129in}}{\pgfqpoint{4.010713in}{0.648129in}}%
\pgfpathclose%
\pgfusepath{stroke,fill}%
\end{pgfscope}%
\begin{pgfscope}%
\pgfpathrectangle{\pgfqpoint{0.648703in}{0.548769in}}{\pgfqpoint{5.201297in}{3.102590in}}%
\pgfusepath{clip}%
\pgfsetbuttcap%
\pgfsetroundjoin%
\definecolor{currentfill}{rgb}{0.121569,0.466667,0.705882}%
\pgfsetfillcolor{currentfill}%
\pgfsetlinewidth{1.003750pt}%
\definecolor{currentstroke}{rgb}{0.121569,0.466667,0.705882}%
\pgfsetstrokecolor{currentstroke}%
\pgfsetdash{}{0pt}%
\pgfpathmoveto{\pgfqpoint{3.850426in}{0.648129in}}%
\pgfpathcurveto{\pgfqpoint{3.861476in}{0.648129in}}{\pgfqpoint{3.872075in}{0.652519in}}{\pgfqpoint{3.879889in}{0.660333in}}%
\pgfpathcurveto{\pgfqpoint{3.887702in}{0.668146in}}{\pgfqpoint{3.892093in}{0.678745in}}{\pgfqpoint{3.892093in}{0.689796in}}%
\pgfpathcurveto{\pgfqpoint{3.892093in}{0.700846in}}{\pgfqpoint{3.887702in}{0.711445in}}{\pgfqpoint{3.879889in}{0.719258in}}%
\pgfpathcurveto{\pgfqpoint{3.872075in}{0.727072in}}{\pgfqpoint{3.861476in}{0.731462in}}{\pgfqpoint{3.850426in}{0.731462in}}%
\pgfpathcurveto{\pgfqpoint{3.839376in}{0.731462in}}{\pgfqpoint{3.828777in}{0.727072in}}{\pgfqpoint{3.820963in}{0.719258in}}%
\pgfpathcurveto{\pgfqpoint{3.813150in}{0.711445in}}{\pgfqpoint{3.808759in}{0.700846in}}{\pgfqpoint{3.808759in}{0.689796in}}%
\pgfpathcurveto{\pgfqpoint{3.808759in}{0.678745in}}{\pgfqpoint{3.813150in}{0.668146in}}{\pgfqpoint{3.820963in}{0.660333in}}%
\pgfpathcurveto{\pgfqpoint{3.828777in}{0.652519in}}{\pgfqpoint{3.839376in}{0.648129in}}{\pgfqpoint{3.850426in}{0.648129in}}%
\pgfpathclose%
\pgfusepath{stroke,fill}%
\end{pgfscope}%
\begin{pgfscope}%
\pgfpathrectangle{\pgfqpoint{0.648703in}{0.548769in}}{\pgfqpoint{5.201297in}{3.102590in}}%
\pgfusepath{clip}%
\pgfsetbuttcap%
\pgfsetroundjoin%
\definecolor{currentfill}{rgb}{0.121569,0.466667,0.705882}%
\pgfsetfillcolor{currentfill}%
\pgfsetlinewidth{1.003750pt}%
\definecolor{currentstroke}{rgb}{0.121569,0.466667,0.705882}%
\pgfsetstrokecolor{currentstroke}%
\pgfsetdash{}{0pt}%
\pgfpathmoveto{\pgfqpoint{3.529853in}{3.132690in}}%
\pgfpathcurveto{\pgfqpoint{3.540903in}{3.132690in}}{\pgfqpoint{3.551502in}{3.137080in}}{\pgfqpoint{3.559316in}{3.144893in}}%
\pgfpathcurveto{\pgfqpoint{3.567129in}{3.152707in}}{\pgfqpoint{3.571520in}{3.163306in}}{\pgfqpoint{3.571520in}{3.174356in}}%
\pgfpathcurveto{\pgfqpoint{3.571520in}{3.185406in}}{\pgfqpoint{3.567129in}{3.196005in}}{\pgfqpoint{3.559316in}{3.203819in}}%
\pgfpathcurveto{\pgfqpoint{3.551502in}{3.211633in}}{\pgfqpoint{3.540903in}{3.216023in}}{\pgfqpoint{3.529853in}{3.216023in}}%
\pgfpathcurveto{\pgfqpoint{3.518803in}{3.216023in}}{\pgfqpoint{3.508204in}{3.211633in}}{\pgfqpoint{3.500390in}{3.203819in}}%
\pgfpathcurveto{\pgfqpoint{3.492577in}{3.196005in}}{\pgfqpoint{3.488186in}{3.185406in}}{\pgfqpoint{3.488186in}{3.174356in}}%
\pgfpathcurveto{\pgfqpoint{3.488186in}{3.163306in}}{\pgfqpoint{3.492577in}{3.152707in}}{\pgfqpoint{3.500390in}{3.144893in}}%
\pgfpathcurveto{\pgfqpoint{3.508204in}{3.137080in}}{\pgfqpoint{3.518803in}{3.132690in}}{\pgfqpoint{3.529853in}{3.132690in}}%
\pgfpathclose%
\pgfusepath{stroke,fill}%
\end{pgfscope}%
\begin{pgfscope}%
\pgfpathrectangle{\pgfqpoint{0.648703in}{0.548769in}}{\pgfqpoint{5.201297in}{3.102590in}}%
\pgfusepath{clip}%
\pgfsetbuttcap%
\pgfsetroundjoin%
\definecolor{currentfill}{rgb}{1.000000,0.498039,0.054902}%
\pgfsetfillcolor{currentfill}%
\pgfsetlinewidth{1.003750pt}%
\definecolor{currentstroke}{rgb}{1.000000,0.498039,0.054902}%
\pgfsetstrokecolor{currentstroke}%
\pgfsetdash{}{0pt}%
\pgfpathmoveto{\pgfqpoint{3.770283in}{3.219794in}}%
\pgfpathcurveto{\pgfqpoint{3.781333in}{3.219794in}}{\pgfqpoint{3.791932in}{3.224185in}}{\pgfqpoint{3.799746in}{3.231998in}}%
\pgfpathcurveto{\pgfqpoint{3.807559in}{3.239812in}}{\pgfqpoint{3.811949in}{3.250411in}}{\pgfqpoint{3.811949in}{3.261461in}}%
\pgfpathcurveto{\pgfqpoint{3.811949in}{3.272511in}}{\pgfqpoint{3.807559in}{3.283110in}}{\pgfqpoint{3.799746in}{3.290924in}}%
\pgfpathcurveto{\pgfqpoint{3.791932in}{3.298737in}}{\pgfqpoint{3.781333in}{3.303128in}}{\pgfqpoint{3.770283in}{3.303128in}}%
\pgfpathcurveto{\pgfqpoint{3.759233in}{3.303128in}}{\pgfqpoint{3.748634in}{3.298737in}}{\pgfqpoint{3.740820in}{3.290924in}}%
\pgfpathcurveto{\pgfqpoint{3.733006in}{3.283110in}}{\pgfqpoint{3.728616in}{3.272511in}}{\pgfqpoint{3.728616in}{3.261461in}}%
\pgfpathcurveto{\pgfqpoint{3.728616in}{3.250411in}}{\pgfqpoint{3.733006in}{3.239812in}}{\pgfqpoint{3.740820in}{3.231998in}}%
\pgfpathcurveto{\pgfqpoint{3.748634in}{3.224185in}}{\pgfqpoint{3.759233in}{3.219794in}}{\pgfqpoint{3.770283in}{3.219794in}}%
\pgfpathclose%
\pgfusepath{stroke,fill}%
\end{pgfscope}%
\begin{pgfscope}%
\pgfpathrectangle{\pgfqpoint{0.648703in}{0.548769in}}{\pgfqpoint{5.201297in}{3.102590in}}%
\pgfusepath{clip}%
\pgfsetbuttcap%
\pgfsetroundjoin%
\definecolor{currentfill}{rgb}{1.000000,0.498039,0.054902}%
\pgfsetfillcolor{currentfill}%
\pgfsetlinewidth{1.003750pt}%
\definecolor{currentstroke}{rgb}{1.000000,0.498039,0.054902}%
\pgfsetstrokecolor{currentstroke}%
\pgfsetdash{}{0pt}%
\pgfpathmoveto{\pgfqpoint{3.449710in}{3.140985in}}%
\pgfpathcurveto{\pgfqpoint{3.460760in}{3.140985in}}{\pgfqpoint{3.471359in}{3.145375in}}{\pgfqpoint{3.479173in}{3.153189in}}%
\pgfpathcurveto{\pgfqpoint{3.486986in}{3.161003in}}{\pgfqpoint{3.491376in}{3.171602in}}{\pgfqpoint{3.491376in}{3.182652in}}%
\pgfpathcurveto{\pgfqpoint{3.491376in}{3.193702in}}{\pgfqpoint{3.486986in}{3.204301in}}{\pgfqpoint{3.479173in}{3.212115in}}%
\pgfpathcurveto{\pgfqpoint{3.471359in}{3.219928in}}{\pgfqpoint{3.460760in}{3.224319in}}{\pgfqpoint{3.449710in}{3.224319in}}%
\pgfpathcurveto{\pgfqpoint{3.438660in}{3.224319in}}{\pgfqpoint{3.428061in}{3.219928in}}{\pgfqpoint{3.420247in}{3.212115in}}%
\pgfpathcurveto{\pgfqpoint{3.412433in}{3.204301in}}{\pgfqpoint{3.408043in}{3.193702in}}{\pgfqpoint{3.408043in}{3.182652in}}%
\pgfpathcurveto{\pgfqpoint{3.408043in}{3.171602in}}{\pgfqpoint{3.412433in}{3.161003in}}{\pgfqpoint{3.420247in}{3.153189in}}%
\pgfpathcurveto{\pgfqpoint{3.428061in}{3.145375in}}{\pgfqpoint{3.438660in}{3.140985in}}{\pgfqpoint{3.449710in}{3.140985in}}%
\pgfpathclose%
\pgfusepath{stroke,fill}%
\end{pgfscope}%
\begin{pgfscope}%
\pgfpathrectangle{\pgfqpoint{0.648703in}{0.548769in}}{\pgfqpoint{5.201297in}{3.102590in}}%
\pgfusepath{clip}%
\pgfsetbuttcap%
\pgfsetroundjoin%
\definecolor{currentfill}{rgb}{1.000000,0.498039,0.054902}%
\pgfsetfillcolor{currentfill}%
\pgfsetlinewidth{1.003750pt}%
\definecolor{currentstroke}{rgb}{1.000000,0.498039,0.054902}%
\pgfsetstrokecolor{currentstroke}%
\pgfsetdash{}{0pt}%
\pgfpathmoveto{\pgfqpoint{3.369567in}{3.145133in}}%
\pgfpathcurveto{\pgfqpoint{3.380617in}{3.145133in}}{\pgfqpoint{3.391216in}{3.149523in}}{\pgfqpoint{3.399029in}{3.157337in}}%
\pgfpathcurveto{\pgfqpoint{3.406843in}{3.165151in}}{\pgfqpoint{3.411233in}{3.175750in}}{\pgfqpoint{3.411233in}{3.186800in}}%
\pgfpathcurveto{\pgfqpoint{3.411233in}{3.197850in}}{\pgfqpoint{3.406843in}{3.208449in}}{\pgfqpoint{3.399029in}{3.216262in}}%
\pgfpathcurveto{\pgfqpoint{3.391216in}{3.224076in}}{\pgfqpoint{3.380617in}{3.228466in}}{\pgfqpoint{3.369567in}{3.228466in}}%
\pgfpathcurveto{\pgfqpoint{3.358516in}{3.228466in}}{\pgfqpoint{3.347917in}{3.224076in}}{\pgfqpoint{3.340104in}{3.216262in}}%
\pgfpathcurveto{\pgfqpoint{3.332290in}{3.208449in}}{\pgfqpoint{3.327900in}{3.197850in}}{\pgfqpoint{3.327900in}{3.186800in}}%
\pgfpathcurveto{\pgfqpoint{3.327900in}{3.175750in}}{\pgfqpoint{3.332290in}{3.165151in}}{\pgfqpoint{3.340104in}{3.157337in}}%
\pgfpathcurveto{\pgfqpoint{3.347917in}{3.149523in}}{\pgfqpoint{3.358516in}{3.145133in}}{\pgfqpoint{3.369567in}{3.145133in}}%
\pgfpathclose%
\pgfusepath{stroke,fill}%
\end{pgfscope}%
\begin{pgfscope}%
\pgfpathrectangle{\pgfqpoint{0.648703in}{0.548769in}}{\pgfqpoint{5.201297in}{3.102590in}}%
\pgfusepath{clip}%
\pgfsetbuttcap%
\pgfsetroundjoin%
\definecolor{currentfill}{rgb}{1.000000,0.498039,0.054902}%
\pgfsetfillcolor{currentfill}%
\pgfsetlinewidth{1.003750pt}%
\definecolor{currentstroke}{rgb}{1.000000,0.498039,0.054902}%
\pgfsetstrokecolor{currentstroke}%
\pgfsetdash{}{0pt}%
\pgfpathmoveto{\pgfqpoint{3.690140in}{3.136837in}}%
\pgfpathcurveto{\pgfqpoint{3.701190in}{3.136837in}}{\pgfqpoint{3.711789in}{3.141228in}}{\pgfqpoint{3.719602in}{3.149041in}}%
\pgfpathcurveto{\pgfqpoint{3.727416in}{3.156855in}}{\pgfqpoint{3.731806in}{3.167454in}}{\pgfqpoint{3.731806in}{3.178504in}}%
\pgfpathcurveto{\pgfqpoint{3.731806in}{3.189554in}}{\pgfqpoint{3.727416in}{3.200153in}}{\pgfqpoint{3.719602in}{3.207967in}}%
\pgfpathcurveto{\pgfqpoint{3.711789in}{3.215780in}}{\pgfqpoint{3.701190in}{3.220171in}}{\pgfqpoint{3.690140in}{3.220171in}}%
\pgfpathcurveto{\pgfqpoint{3.679089in}{3.220171in}}{\pgfqpoint{3.668490in}{3.215780in}}{\pgfqpoint{3.660677in}{3.207967in}}%
\pgfpathcurveto{\pgfqpoint{3.652863in}{3.200153in}}{\pgfqpoint{3.648473in}{3.189554in}}{\pgfqpoint{3.648473in}{3.178504in}}%
\pgfpathcurveto{\pgfqpoint{3.648473in}{3.167454in}}{\pgfqpoint{3.652863in}{3.156855in}}{\pgfqpoint{3.660677in}{3.149041in}}%
\pgfpathcurveto{\pgfqpoint{3.668490in}{3.141228in}}{\pgfqpoint{3.679089in}{3.136837in}}{\pgfqpoint{3.690140in}{3.136837in}}%
\pgfpathclose%
\pgfusepath{stroke,fill}%
\end{pgfscope}%
\begin{pgfscope}%
\pgfpathrectangle{\pgfqpoint{0.648703in}{0.548769in}}{\pgfqpoint{5.201297in}{3.102590in}}%
\pgfusepath{clip}%
\pgfsetbuttcap%
\pgfsetroundjoin%
\definecolor{currentfill}{rgb}{0.121569,0.466667,0.705882}%
\pgfsetfillcolor{currentfill}%
\pgfsetlinewidth{1.003750pt}%
\definecolor{currentstroke}{rgb}{0.121569,0.466667,0.705882}%
\pgfsetstrokecolor{currentstroke}%
\pgfsetdash{}{0pt}%
\pgfpathmoveto{\pgfqpoint{3.850426in}{0.648129in}}%
\pgfpathcurveto{\pgfqpoint{3.861476in}{0.648129in}}{\pgfqpoint{3.872075in}{0.652519in}}{\pgfqpoint{3.879889in}{0.660333in}}%
\pgfpathcurveto{\pgfqpoint{3.887702in}{0.668146in}}{\pgfqpoint{3.892093in}{0.678745in}}{\pgfqpoint{3.892093in}{0.689796in}}%
\pgfpathcurveto{\pgfqpoint{3.892093in}{0.700846in}}{\pgfqpoint{3.887702in}{0.711445in}}{\pgfqpoint{3.879889in}{0.719258in}}%
\pgfpathcurveto{\pgfqpoint{3.872075in}{0.727072in}}{\pgfqpoint{3.861476in}{0.731462in}}{\pgfqpoint{3.850426in}{0.731462in}}%
\pgfpathcurveto{\pgfqpoint{3.839376in}{0.731462in}}{\pgfqpoint{3.828777in}{0.727072in}}{\pgfqpoint{3.820963in}{0.719258in}}%
\pgfpathcurveto{\pgfqpoint{3.813150in}{0.711445in}}{\pgfqpoint{3.808759in}{0.700846in}}{\pgfqpoint{3.808759in}{0.689796in}}%
\pgfpathcurveto{\pgfqpoint{3.808759in}{0.678745in}}{\pgfqpoint{3.813150in}{0.668146in}}{\pgfqpoint{3.820963in}{0.660333in}}%
\pgfpathcurveto{\pgfqpoint{3.828777in}{0.652519in}}{\pgfqpoint{3.839376in}{0.648129in}}{\pgfqpoint{3.850426in}{0.648129in}}%
\pgfpathclose%
\pgfusepath{stroke,fill}%
\end{pgfscope}%
\begin{pgfscope}%
\pgfpathrectangle{\pgfqpoint{0.648703in}{0.548769in}}{\pgfqpoint{5.201297in}{3.102590in}}%
\pgfusepath{clip}%
\pgfsetbuttcap%
\pgfsetroundjoin%
\definecolor{currentfill}{rgb}{0.121569,0.466667,0.705882}%
\pgfsetfillcolor{currentfill}%
\pgfsetlinewidth{1.003750pt}%
\definecolor{currentstroke}{rgb}{0.121569,0.466667,0.705882}%
\pgfsetstrokecolor{currentstroke}%
\pgfsetdash{}{0pt}%
\pgfpathmoveto{\pgfqpoint{3.449710in}{3.132690in}}%
\pgfpathcurveto{\pgfqpoint{3.460760in}{3.132690in}}{\pgfqpoint{3.471359in}{3.137080in}}{\pgfqpoint{3.479173in}{3.144893in}}%
\pgfpathcurveto{\pgfqpoint{3.486986in}{3.152707in}}{\pgfqpoint{3.491376in}{3.163306in}}{\pgfqpoint{3.491376in}{3.174356in}}%
\pgfpathcurveto{\pgfqpoint{3.491376in}{3.185406in}}{\pgfqpoint{3.486986in}{3.196005in}}{\pgfqpoint{3.479173in}{3.203819in}}%
\pgfpathcurveto{\pgfqpoint{3.471359in}{3.211633in}}{\pgfqpoint{3.460760in}{3.216023in}}{\pgfqpoint{3.449710in}{3.216023in}}%
\pgfpathcurveto{\pgfqpoint{3.438660in}{3.216023in}}{\pgfqpoint{3.428061in}{3.211633in}}{\pgfqpoint{3.420247in}{3.203819in}}%
\pgfpathcurveto{\pgfqpoint{3.412433in}{3.196005in}}{\pgfqpoint{3.408043in}{3.185406in}}{\pgfqpoint{3.408043in}{3.174356in}}%
\pgfpathcurveto{\pgfqpoint{3.408043in}{3.163306in}}{\pgfqpoint{3.412433in}{3.152707in}}{\pgfqpoint{3.420247in}{3.144893in}}%
\pgfpathcurveto{\pgfqpoint{3.428061in}{3.137080in}}{\pgfqpoint{3.438660in}{3.132690in}}{\pgfqpoint{3.449710in}{3.132690in}}%
\pgfpathclose%
\pgfusepath{stroke,fill}%
\end{pgfscope}%
\begin{pgfscope}%
\pgfpathrectangle{\pgfqpoint{0.648703in}{0.548769in}}{\pgfqpoint{5.201297in}{3.102590in}}%
\pgfusepath{clip}%
\pgfsetbuttcap%
\pgfsetroundjoin%
\definecolor{currentfill}{rgb}{1.000000,0.498039,0.054902}%
\pgfsetfillcolor{currentfill}%
\pgfsetlinewidth{1.003750pt}%
\definecolor{currentstroke}{rgb}{1.000000,0.498039,0.054902}%
\pgfsetstrokecolor{currentstroke}%
\pgfsetdash{}{0pt}%
\pgfpathmoveto{\pgfqpoint{3.369567in}{3.145133in}}%
\pgfpathcurveto{\pgfqpoint{3.380617in}{3.145133in}}{\pgfqpoint{3.391216in}{3.149523in}}{\pgfqpoint{3.399029in}{3.157337in}}%
\pgfpathcurveto{\pgfqpoint{3.406843in}{3.165151in}}{\pgfqpoint{3.411233in}{3.175750in}}{\pgfqpoint{3.411233in}{3.186800in}}%
\pgfpathcurveto{\pgfqpoint{3.411233in}{3.197850in}}{\pgfqpoint{3.406843in}{3.208449in}}{\pgfqpoint{3.399029in}{3.216262in}}%
\pgfpathcurveto{\pgfqpoint{3.391216in}{3.224076in}}{\pgfqpoint{3.380617in}{3.228466in}}{\pgfqpoint{3.369567in}{3.228466in}}%
\pgfpathcurveto{\pgfqpoint{3.358516in}{3.228466in}}{\pgfqpoint{3.347917in}{3.224076in}}{\pgfqpoint{3.340104in}{3.216262in}}%
\pgfpathcurveto{\pgfqpoint{3.332290in}{3.208449in}}{\pgfqpoint{3.327900in}{3.197850in}}{\pgfqpoint{3.327900in}{3.186800in}}%
\pgfpathcurveto{\pgfqpoint{3.327900in}{3.175750in}}{\pgfqpoint{3.332290in}{3.165151in}}{\pgfqpoint{3.340104in}{3.157337in}}%
\pgfpathcurveto{\pgfqpoint{3.347917in}{3.149523in}}{\pgfqpoint{3.358516in}{3.145133in}}{\pgfqpoint{3.369567in}{3.145133in}}%
\pgfpathclose%
\pgfusepath{stroke,fill}%
\end{pgfscope}%
\begin{pgfscope}%
\pgfpathrectangle{\pgfqpoint{0.648703in}{0.548769in}}{\pgfqpoint{5.201297in}{3.102590in}}%
\pgfusepath{clip}%
\pgfsetbuttcap%
\pgfsetroundjoin%
\definecolor{currentfill}{rgb}{0.121569,0.466667,0.705882}%
\pgfsetfillcolor{currentfill}%
\pgfsetlinewidth{1.003750pt}%
\definecolor{currentstroke}{rgb}{0.121569,0.466667,0.705882}%
\pgfsetstrokecolor{currentstroke}%
\pgfsetdash{}{0pt}%
\pgfpathmoveto{\pgfqpoint{3.209280in}{3.132690in}}%
\pgfpathcurveto{\pgfqpoint{3.220330in}{3.132690in}}{\pgfqpoint{3.230929in}{3.137080in}}{\pgfqpoint{3.238743in}{3.144893in}}%
\pgfpathcurveto{\pgfqpoint{3.246556in}{3.152707in}}{\pgfqpoint{3.250947in}{3.163306in}}{\pgfqpoint{3.250947in}{3.174356in}}%
\pgfpathcurveto{\pgfqpoint{3.250947in}{3.185406in}}{\pgfqpoint{3.246556in}{3.196005in}}{\pgfqpoint{3.238743in}{3.203819in}}%
\pgfpathcurveto{\pgfqpoint{3.230929in}{3.211633in}}{\pgfqpoint{3.220330in}{3.216023in}}{\pgfqpoint{3.209280in}{3.216023in}}%
\pgfpathcurveto{\pgfqpoint{3.198230in}{3.216023in}}{\pgfqpoint{3.187631in}{3.211633in}}{\pgfqpoint{3.179817in}{3.203819in}}%
\pgfpathcurveto{\pgfqpoint{3.172004in}{3.196005in}}{\pgfqpoint{3.167613in}{3.185406in}}{\pgfqpoint{3.167613in}{3.174356in}}%
\pgfpathcurveto{\pgfqpoint{3.167613in}{3.163306in}}{\pgfqpoint{3.172004in}{3.152707in}}{\pgfqpoint{3.179817in}{3.144893in}}%
\pgfpathcurveto{\pgfqpoint{3.187631in}{3.137080in}}{\pgfqpoint{3.198230in}{3.132690in}}{\pgfqpoint{3.209280in}{3.132690in}}%
\pgfpathclose%
\pgfusepath{stroke,fill}%
\end{pgfscope}%
\begin{pgfscope}%
\pgfpathrectangle{\pgfqpoint{0.648703in}{0.548769in}}{\pgfqpoint{5.201297in}{3.102590in}}%
\pgfusepath{clip}%
\pgfsetbuttcap%
\pgfsetroundjoin%
\definecolor{currentfill}{rgb}{1.000000,0.498039,0.054902}%
\pgfsetfillcolor{currentfill}%
\pgfsetlinewidth{1.003750pt}%
\definecolor{currentstroke}{rgb}{1.000000,0.498039,0.054902}%
\pgfsetstrokecolor{currentstroke}%
\pgfsetdash{}{0pt}%
\pgfpathmoveto{\pgfqpoint{3.529853in}{3.157577in}}%
\pgfpathcurveto{\pgfqpoint{3.540903in}{3.157577in}}{\pgfqpoint{3.551502in}{3.161967in}}{\pgfqpoint{3.559316in}{3.169780in}}%
\pgfpathcurveto{\pgfqpoint{3.567129in}{3.177594in}}{\pgfqpoint{3.571520in}{3.188193in}}{\pgfqpoint{3.571520in}{3.199243in}}%
\pgfpathcurveto{\pgfqpoint{3.571520in}{3.210293in}}{\pgfqpoint{3.567129in}{3.220892in}}{\pgfqpoint{3.559316in}{3.228706in}}%
\pgfpathcurveto{\pgfqpoint{3.551502in}{3.236520in}}{\pgfqpoint{3.540903in}{3.240910in}}{\pgfqpoint{3.529853in}{3.240910in}}%
\pgfpathcurveto{\pgfqpoint{3.518803in}{3.240910in}}{\pgfqpoint{3.508204in}{3.236520in}}{\pgfqpoint{3.500390in}{3.228706in}}%
\pgfpathcurveto{\pgfqpoint{3.492577in}{3.220892in}}{\pgfqpoint{3.488186in}{3.210293in}}{\pgfqpoint{3.488186in}{3.199243in}}%
\pgfpathcurveto{\pgfqpoint{3.488186in}{3.188193in}}{\pgfqpoint{3.492577in}{3.177594in}}{\pgfqpoint{3.500390in}{3.169780in}}%
\pgfpathcurveto{\pgfqpoint{3.508204in}{3.161967in}}{\pgfqpoint{3.518803in}{3.157577in}}{\pgfqpoint{3.529853in}{3.157577in}}%
\pgfpathclose%
\pgfusepath{stroke,fill}%
\end{pgfscope}%
\begin{pgfscope}%
\pgfpathrectangle{\pgfqpoint{0.648703in}{0.548769in}}{\pgfqpoint{5.201297in}{3.102590in}}%
\pgfusepath{clip}%
\pgfsetbuttcap%
\pgfsetroundjoin%
\definecolor{currentfill}{rgb}{1.000000,0.498039,0.054902}%
\pgfsetfillcolor{currentfill}%
\pgfsetlinewidth{1.003750pt}%
\definecolor{currentstroke}{rgb}{1.000000,0.498039,0.054902}%
\pgfsetstrokecolor{currentstroke}%
\pgfsetdash{}{0pt}%
\pgfpathmoveto{\pgfqpoint{3.169208in}{3.145133in}}%
\pgfpathcurveto{\pgfqpoint{3.180259in}{3.145133in}}{\pgfqpoint{3.190858in}{3.149523in}}{\pgfqpoint{3.198671in}{3.157337in}}%
\pgfpathcurveto{\pgfqpoint{3.206485in}{3.165151in}}{\pgfqpoint{3.210875in}{3.175750in}}{\pgfqpoint{3.210875in}{3.186800in}}%
\pgfpathcurveto{\pgfqpoint{3.210875in}{3.197850in}}{\pgfqpoint{3.206485in}{3.208449in}}{\pgfqpoint{3.198671in}{3.216262in}}%
\pgfpathcurveto{\pgfqpoint{3.190858in}{3.224076in}}{\pgfqpoint{3.180259in}{3.228466in}}{\pgfqpoint{3.169208in}{3.228466in}}%
\pgfpathcurveto{\pgfqpoint{3.158158in}{3.228466in}}{\pgfqpoint{3.147559in}{3.224076in}}{\pgfqpoint{3.139746in}{3.216262in}}%
\pgfpathcurveto{\pgfqpoint{3.131932in}{3.208449in}}{\pgfqpoint{3.127542in}{3.197850in}}{\pgfqpoint{3.127542in}{3.186800in}}%
\pgfpathcurveto{\pgfqpoint{3.127542in}{3.175750in}}{\pgfqpoint{3.131932in}{3.165151in}}{\pgfqpoint{3.139746in}{3.157337in}}%
\pgfpathcurveto{\pgfqpoint{3.147559in}{3.149523in}}{\pgfqpoint{3.158158in}{3.145133in}}{\pgfqpoint{3.169208in}{3.145133in}}%
\pgfpathclose%
\pgfusepath{stroke,fill}%
\end{pgfscope}%
\begin{pgfscope}%
\pgfpathrectangle{\pgfqpoint{0.648703in}{0.548769in}}{\pgfqpoint{5.201297in}{3.102590in}}%
\pgfusepath{clip}%
\pgfsetbuttcap%
\pgfsetroundjoin%
\definecolor{currentfill}{rgb}{1.000000,0.498039,0.054902}%
\pgfsetfillcolor{currentfill}%
\pgfsetlinewidth{1.003750pt}%
\definecolor{currentstroke}{rgb}{1.000000,0.498039,0.054902}%
\pgfsetstrokecolor{currentstroke}%
\pgfsetdash{}{0pt}%
\pgfpathmoveto{\pgfqpoint{3.449710in}{3.190759in}}%
\pgfpathcurveto{\pgfqpoint{3.460760in}{3.190759in}}{\pgfqpoint{3.471359in}{3.195150in}}{\pgfqpoint{3.479173in}{3.202963in}}%
\pgfpathcurveto{\pgfqpoint{3.486986in}{3.210777in}}{\pgfqpoint{3.491376in}{3.221376in}}{\pgfqpoint{3.491376in}{3.232426in}}%
\pgfpathcurveto{\pgfqpoint{3.491376in}{3.243476in}}{\pgfqpoint{3.486986in}{3.254075in}}{\pgfqpoint{3.479173in}{3.261889in}}%
\pgfpathcurveto{\pgfqpoint{3.471359in}{3.269702in}}{\pgfqpoint{3.460760in}{3.274093in}}{\pgfqpoint{3.449710in}{3.274093in}}%
\pgfpathcurveto{\pgfqpoint{3.438660in}{3.274093in}}{\pgfqpoint{3.428061in}{3.269702in}}{\pgfqpoint{3.420247in}{3.261889in}}%
\pgfpathcurveto{\pgfqpoint{3.412433in}{3.254075in}}{\pgfqpoint{3.408043in}{3.243476in}}{\pgfqpoint{3.408043in}{3.232426in}}%
\pgfpathcurveto{\pgfqpoint{3.408043in}{3.221376in}}{\pgfqpoint{3.412433in}{3.210777in}}{\pgfqpoint{3.420247in}{3.202963in}}%
\pgfpathcurveto{\pgfqpoint{3.428061in}{3.195150in}}{\pgfqpoint{3.438660in}{3.190759in}}{\pgfqpoint{3.449710in}{3.190759in}}%
\pgfpathclose%
\pgfusepath{stroke,fill}%
\end{pgfscope}%
\begin{pgfscope}%
\pgfpathrectangle{\pgfqpoint{0.648703in}{0.548769in}}{\pgfqpoint{5.201297in}{3.102590in}}%
\pgfusepath{clip}%
\pgfsetbuttcap%
\pgfsetroundjoin%
\definecolor{currentfill}{rgb}{1.000000,0.498039,0.054902}%
\pgfsetfillcolor{currentfill}%
\pgfsetlinewidth{1.003750pt}%
\definecolor{currentstroke}{rgb}{1.000000,0.498039,0.054902}%
\pgfsetstrokecolor{currentstroke}%
\pgfsetdash{}{0pt}%
\pgfpathmoveto{\pgfqpoint{3.249352in}{3.207351in}}%
\pgfpathcurveto{\pgfqpoint{3.260402in}{3.207351in}}{\pgfqpoint{3.271001in}{3.211741in}}{\pgfqpoint{3.278814in}{3.219555in}}%
\pgfpathcurveto{\pgfqpoint{3.286628in}{3.227368in}}{\pgfqpoint{3.291018in}{3.237967in}}{\pgfqpoint{3.291018in}{3.249017in}}%
\pgfpathcurveto{\pgfqpoint{3.291018in}{3.260068in}}{\pgfqpoint{3.286628in}{3.270667in}}{\pgfqpoint{3.278814in}{3.278480in}}%
\pgfpathcurveto{\pgfqpoint{3.271001in}{3.286294in}}{\pgfqpoint{3.260402in}{3.290684in}}{\pgfqpoint{3.249352in}{3.290684in}}%
\pgfpathcurveto{\pgfqpoint{3.238302in}{3.290684in}}{\pgfqpoint{3.227703in}{3.286294in}}{\pgfqpoint{3.219889in}{3.278480in}}%
\pgfpathcurveto{\pgfqpoint{3.212075in}{3.270667in}}{\pgfqpoint{3.207685in}{3.260068in}}{\pgfqpoint{3.207685in}{3.249017in}}%
\pgfpathcurveto{\pgfqpoint{3.207685in}{3.237967in}}{\pgfqpoint{3.212075in}{3.227368in}}{\pgfqpoint{3.219889in}{3.219555in}}%
\pgfpathcurveto{\pgfqpoint{3.227703in}{3.211741in}}{\pgfqpoint{3.238302in}{3.207351in}}{\pgfqpoint{3.249352in}{3.207351in}}%
\pgfpathclose%
\pgfusepath{stroke,fill}%
\end{pgfscope}%
\begin{pgfscope}%
\pgfpathrectangle{\pgfqpoint{0.648703in}{0.548769in}}{\pgfqpoint{5.201297in}{3.102590in}}%
\pgfusepath{clip}%
\pgfsetbuttcap%
\pgfsetroundjoin%
\definecolor{currentfill}{rgb}{0.121569,0.466667,0.705882}%
\pgfsetfillcolor{currentfill}%
\pgfsetlinewidth{1.003750pt}%
\definecolor{currentstroke}{rgb}{0.121569,0.466667,0.705882}%
\pgfsetstrokecolor{currentstroke}%
\pgfsetdash{}{0pt}%
\pgfpathmoveto{\pgfqpoint{3.529853in}{0.648129in}}%
\pgfpathcurveto{\pgfqpoint{3.540903in}{0.648129in}}{\pgfqpoint{3.551502in}{0.652519in}}{\pgfqpoint{3.559316in}{0.660333in}}%
\pgfpathcurveto{\pgfqpoint{3.567129in}{0.668146in}}{\pgfqpoint{3.571520in}{0.678745in}}{\pgfqpoint{3.571520in}{0.689796in}}%
\pgfpathcurveto{\pgfqpoint{3.571520in}{0.700846in}}{\pgfqpoint{3.567129in}{0.711445in}}{\pgfqpoint{3.559316in}{0.719258in}}%
\pgfpathcurveto{\pgfqpoint{3.551502in}{0.727072in}}{\pgfqpoint{3.540903in}{0.731462in}}{\pgfqpoint{3.529853in}{0.731462in}}%
\pgfpathcurveto{\pgfqpoint{3.518803in}{0.731462in}}{\pgfqpoint{3.508204in}{0.727072in}}{\pgfqpoint{3.500390in}{0.719258in}}%
\pgfpathcurveto{\pgfqpoint{3.492577in}{0.711445in}}{\pgfqpoint{3.488186in}{0.700846in}}{\pgfqpoint{3.488186in}{0.689796in}}%
\pgfpathcurveto{\pgfqpoint{3.488186in}{0.678745in}}{\pgfqpoint{3.492577in}{0.668146in}}{\pgfqpoint{3.500390in}{0.660333in}}%
\pgfpathcurveto{\pgfqpoint{3.508204in}{0.652519in}}{\pgfqpoint{3.518803in}{0.648129in}}{\pgfqpoint{3.529853in}{0.648129in}}%
\pgfpathclose%
\pgfusepath{stroke,fill}%
\end{pgfscope}%
\begin{pgfscope}%
\pgfpathrectangle{\pgfqpoint{0.648703in}{0.548769in}}{\pgfqpoint{5.201297in}{3.102590in}}%
\pgfusepath{clip}%
\pgfsetbuttcap%
\pgfsetroundjoin%
\definecolor{currentfill}{rgb}{0.121569,0.466667,0.705882}%
\pgfsetfillcolor{currentfill}%
\pgfsetlinewidth{1.003750pt}%
\definecolor{currentstroke}{rgb}{0.121569,0.466667,0.705882}%
\pgfsetstrokecolor{currentstroke}%
\pgfsetdash{}{0pt}%
\pgfpathmoveto{\pgfqpoint{3.569925in}{0.656425in}}%
\pgfpathcurveto{\pgfqpoint{3.580975in}{0.656425in}}{\pgfqpoint{3.591574in}{0.660815in}}{\pgfqpoint{3.599387in}{0.668629in}}%
\pgfpathcurveto{\pgfqpoint{3.607201in}{0.676442in}}{\pgfqpoint{3.611591in}{0.687041in}}{\pgfqpoint{3.611591in}{0.698091in}}%
\pgfpathcurveto{\pgfqpoint{3.611591in}{0.709141in}}{\pgfqpoint{3.607201in}{0.719740in}}{\pgfqpoint{3.599387in}{0.727554in}}%
\pgfpathcurveto{\pgfqpoint{3.591574in}{0.735368in}}{\pgfqpoint{3.580975in}{0.739758in}}{\pgfqpoint{3.569925in}{0.739758in}}%
\pgfpathcurveto{\pgfqpoint{3.558875in}{0.739758in}}{\pgfqpoint{3.548276in}{0.735368in}}{\pgfqpoint{3.540462in}{0.727554in}}%
\pgfpathcurveto{\pgfqpoint{3.532648in}{0.719740in}}{\pgfqpoint{3.528258in}{0.709141in}}{\pgfqpoint{3.528258in}{0.698091in}}%
\pgfpathcurveto{\pgfqpoint{3.528258in}{0.687041in}}{\pgfqpoint{3.532648in}{0.676442in}}{\pgfqpoint{3.540462in}{0.668629in}}%
\pgfpathcurveto{\pgfqpoint{3.548276in}{0.660815in}}{\pgfqpoint{3.558875in}{0.656425in}}{\pgfqpoint{3.569925in}{0.656425in}}%
\pgfpathclose%
\pgfusepath{stroke,fill}%
\end{pgfscope}%
\begin{pgfscope}%
\pgfpathrectangle{\pgfqpoint{0.648703in}{0.548769in}}{\pgfqpoint{5.201297in}{3.102590in}}%
\pgfusepath{clip}%
\pgfsetbuttcap%
\pgfsetroundjoin%
\definecolor{currentfill}{rgb}{0.121569,0.466667,0.705882}%
\pgfsetfillcolor{currentfill}%
\pgfsetlinewidth{1.003750pt}%
\definecolor{currentstroke}{rgb}{0.121569,0.466667,0.705882}%
\pgfsetstrokecolor{currentstroke}%
\pgfsetdash{}{0pt}%
\pgfpathmoveto{\pgfqpoint{3.289423in}{3.099507in}}%
\pgfpathcurveto{\pgfqpoint{3.300473in}{3.099507in}}{\pgfqpoint{3.311072in}{3.103897in}}{\pgfqpoint{3.318886in}{3.111711in}}%
\pgfpathcurveto{\pgfqpoint{3.326700in}{3.119524in}}{\pgfqpoint{3.331090in}{3.130123in}}{\pgfqpoint{3.331090in}{3.141173in}}%
\pgfpathcurveto{\pgfqpoint{3.331090in}{3.152224in}}{\pgfqpoint{3.326700in}{3.162823in}}{\pgfqpoint{3.318886in}{3.170636in}}%
\pgfpathcurveto{\pgfqpoint{3.311072in}{3.178450in}}{\pgfqpoint{3.300473in}{3.182840in}}{\pgfqpoint{3.289423in}{3.182840in}}%
\pgfpathcurveto{\pgfqpoint{3.278373in}{3.182840in}}{\pgfqpoint{3.267774in}{3.178450in}}{\pgfqpoint{3.259961in}{3.170636in}}%
\pgfpathcurveto{\pgfqpoint{3.252147in}{3.162823in}}{\pgfqpoint{3.247757in}{3.152224in}}{\pgfqpoint{3.247757in}{3.141173in}}%
\pgfpathcurveto{\pgfqpoint{3.247757in}{3.130123in}}{\pgfqpoint{3.252147in}{3.119524in}}{\pgfqpoint{3.259961in}{3.111711in}}%
\pgfpathcurveto{\pgfqpoint{3.267774in}{3.103897in}}{\pgfqpoint{3.278373in}{3.099507in}}{\pgfqpoint{3.289423in}{3.099507in}}%
\pgfpathclose%
\pgfusepath{stroke,fill}%
\end{pgfscope}%
\begin{pgfscope}%
\pgfpathrectangle{\pgfqpoint{0.648703in}{0.548769in}}{\pgfqpoint{5.201297in}{3.102590in}}%
\pgfusepath{clip}%
\pgfsetbuttcap%
\pgfsetroundjoin%
\definecolor{currentfill}{rgb}{1.000000,0.498039,0.054902}%
\pgfsetfillcolor{currentfill}%
\pgfsetlinewidth{1.003750pt}%
\definecolor{currentstroke}{rgb}{1.000000,0.498039,0.054902}%
\pgfsetstrokecolor{currentstroke}%
\pgfsetdash{}{0pt}%
\pgfpathmoveto{\pgfqpoint{3.449710in}{3.136837in}}%
\pgfpathcurveto{\pgfqpoint{3.460760in}{3.136837in}}{\pgfqpoint{3.471359in}{3.141228in}}{\pgfqpoint{3.479173in}{3.149041in}}%
\pgfpathcurveto{\pgfqpoint{3.486986in}{3.156855in}}{\pgfqpoint{3.491376in}{3.167454in}}{\pgfqpoint{3.491376in}{3.178504in}}%
\pgfpathcurveto{\pgfqpoint{3.491376in}{3.189554in}}{\pgfqpoint{3.486986in}{3.200153in}}{\pgfqpoint{3.479173in}{3.207967in}}%
\pgfpathcurveto{\pgfqpoint{3.471359in}{3.215780in}}{\pgfqpoint{3.460760in}{3.220171in}}{\pgfqpoint{3.449710in}{3.220171in}}%
\pgfpathcurveto{\pgfqpoint{3.438660in}{3.220171in}}{\pgfqpoint{3.428061in}{3.215780in}}{\pgfqpoint{3.420247in}{3.207967in}}%
\pgfpathcurveto{\pgfqpoint{3.412433in}{3.200153in}}{\pgfqpoint{3.408043in}{3.189554in}}{\pgfqpoint{3.408043in}{3.178504in}}%
\pgfpathcurveto{\pgfqpoint{3.408043in}{3.167454in}}{\pgfqpoint{3.412433in}{3.156855in}}{\pgfqpoint{3.420247in}{3.149041in}}%
\pgfpathcurveto{\pgfqpoint{3.428061in}{3.141228in}}{\pgfqpoint{3.438660in}{3.136837in}}{\pgfqpoint{3.449710in}{3.136837in}}%
\pgfpathclose%
\pgfusepath{stroke,fill}%
\end{pgfscope}%
\begin{pgfscope}%
\pgfpathrectangle{\pgfqpoint{0.648703in}{0.548769in}}{\pgfqpoint{5.201297in}{3.102590in}}%
\pgfusepath{clip}%
\pgfsetbuttcap%
\pgfsetroundjoin%
\definecolor{currentfill}{rgb}{0.121569,0.466667,0.705882}%
\pgfsetfillcolor{currentfill}%
\pgfsetlinewidth{1.003750pt}%
\definecolor{currentstroke}{rgb}{0.121569,0.466667,0.705882}%
\pgfsetstrokecolor{currentstroke}%
\pgfsetdash{}{0pt}%
\pgfpathmoveto{\pgfqpoint{3.690140in}{0.648129in}}%
\pgfpathcurveto{\pgfqpoint{3.701190in}{0.648129in}}{\pgfqpoint{3.711789in}{0.652519in}}{\pgfqpoint{3.719602in}{0.660333in}}%
\pgfpathcurveto{\pgfqpoint{3.727416in}{0.668146in}}{\pgfqpoint{3.731806in}{0.678745in}}{\pgfqpoint{3.731806in}{0.689796in}}%
\pgfpathcurveto{\pgfqpoint{3.731806in}{0.700846in}}{\pgfqpoint{3.727416in}{0.711445in}}{\pgfqpoint{3.719602in}{0.719258in}}%
\pgfpathcurveto{\pgfqpoint{3.711789in}{0.727072in}}{\pgfqpoint{3.701190in}{0.731462in}}{\pgfqpoint{3.690140in}{0.731462in}}%
\pgfpathcurveto{\pgfqpoint{3.679089in}{0.731462in}}{\pgfqpoint{3.668490in}{0.727072in}}{\pgfqpoint{3.660677in}{0.719258in}}%
\pgfpathcurveto{\pgfqpoint{3.652863in}{0.711445in}}{\pgfqpoint{3.648473in}{0.700846in}}{\pgfqpoint{3.648473in}{0.689796in}}%
\pgfpathcurveto{\pgfqpoint{3.648473in}{0.678745in}}{\pgfqpoint{3.652863in}{0.668146in}}{\pgfqpoint{3.660677in}{0.660333in}}%
\pgfpathcurveto{\pgfqpoint{3.668490in}{0.652519in}}{\pgfqpoint{3.679089in}{0.648129in}}{\pgfqpoint{3.690140in}{0.648129in}}%
\pgfpathclose%
\pgfusepath{stroke,fill}%
\end{pgfscope}%
\begin{pgfscope}%
\pgfpathrectangle{\pgfqpoint{0.648703in}{0.548769in}}{\pgfqpoint{5.201297in}{3.102590in}}%
\pgfusepath{clip}%
\pgfsetbuttcap%
\pgfsetroundjoin%
\definecolor{currentfill}{rgb}{0.121569,0.466667,0.705882}%
\pgfsetfillcolor{currentfill}%
\pgfsetlinewidth{1.003750pt}%
\definecolor{currentstroke}{rgb}{0.121569,0.466667,0.705882}%
\pgfsetstrokecolor{currentstroke}%
\pgfsetdash{}{0pt}%
\pgfpathmoveto{\pgfqpoint{3.489781in}{0.648129in}}%
\pgfpathcurveto{\pgfqpoint{3.500832in}{0.648129in}}{\pgfqpoint{3.511431in}{0.652519in}}{\pgfqpoint{3.519244in}{0.660333in}}%
\pgfpathcurveto{\pgfqpoint{3.527058in}{0.668146in}}{\pgfqpoint{3.531448in}{0.678745in}}{\pgfqpoint{3.531448in}{0.689796in}}%
\pgfpathcurveto{\pgfqpoint{3.531448in}{0.700846in}}{\pgfqpoint{3.527058in}{0.711445in}}{\pgfqpoint{3.519244in}{0.719258in}}%
\pgfpathcurveto{\pgfqpoint{3.511431in}{0.727072in}}{\pgfqpoint{3.500832in}{0.731462in}}{\pgfqpoint{3.489781in}{0.731462in}}%
\pgfpathcurveto{\pgfqpoint{3.478731in}{0.731462in}}{\pgfqpoint{3.468132in}{0.727072in}}{\pgfqpoint{3.460319in}{0.719258in}}%
\pgfpathcurveto{\pgfqpoint{3.452505in}{0.711445in}}{\pgfqpoint{3.448115in}{0.700846in}}{\pgfqpoint{3.448115in}{0.689796in}}%
\pgfpathcurveto{\pgfqpoint{3.448115in}{0.678745in}}{\pgfqpoint{3.452505in}{0.668146in}}{\pgfqpoint{3.460319in}{0.660333in}}%
\pgfpathcurveto{\pgfqpoint{3.468132in}{0.652519in}}{\pgfqpoint{3.478731in}{0.648129in}}{\pgfqpoint{3.489781in}{0.648129in}}%
\pgfpathclose%
\pgfusepath{stroke,fill}%
\end{pgfscope}%
\begin{pgfscope}%
\pgfpathrectangle{\pgfqpoint{0.648703in}{0.548769in}}{\pgfqpoint{5.201297in}{3.102590in}}%
\pgfusepath{clip}%
\pgfsetbuttcap%
\pgfsetroundjoin%
\definecolor{currentfill}{rgb}{0.839216,0.152941,0.156863}%
\pgfsetfillcolor{currentfill}%
\pgfsetlinewidth{1.003750pt}%
\definecolor{currentstroke}{rgb}{0.839216,0.152941,0.156863}%
\pgfsetstrokecolor{currentstroke}%
\pgfsetdash{}{0pt}%
\pgfpathmoveto{\pgfqpoint{2.928779in}{3.410595in}}%
\pgfpathcurveto{\pgfqpoint{2.939829in}{3.410595in}}{\pgfqpoint{2.950428in}{3.414986in}}{\pgfqpoint{2.958242in}{3.422799in}}%
\pgfpathcurveto{\pgfqpoint{2.966055in}{3.430613in}}{\pgfqpoint{2.970445in}{3.441212in}}{\pgfqpoint{2.970445in}{3.452262in}}%
\pgfpathcurveto{\pgfqpoint{2.970445in}{3.463312in}}{\pgfqpoint{2.966055in}{3.473911in}}{\pgfqpoint{2.958242in}{3.481725in}}%
\pgfpathcurveto{\pgfqpoint{2.950428in}{3.489538in}}{\pgfqpoint{2.939829in}{3.493929in}}{\pgfqpoint{2.928779in}{3.493929in}}%
\pgfpathcurveto{\pgfqpoint{2.917729in}{3.493929in}}{\pgfqpoint{2.907130in}{3.489538in}}{\pgfqpoint{2.899316in}{3.481725in}}%
\pgfpathcurveto{\pgfqpoint{2.891502in}{3.473911in}}{\pgfqpoint{2.887112in}{3.463312in}}{\pgfqpoint{2.887112in}{3.452262in}}%
\pgfpathcurveto{\pgfqpoint{2.887112in}{3.441212in}}{\pgfqpoint{2.891502in}{3.430613in}}{\pgfqpoint{2.899316in}{3.422799in}}%
\pgfpathcurveto{\pgfqpoint{2.907130in}{3.414986in}}{\pgfqpoint{2.917729in}{3.410595in}}{\pgfqpoint{2.928779in}{3.410595in}}%
\pgfpathclose%
\pgfusepath{stroke,fill}%
\end{pgfscope}%
\begin{pgfscope}%
\pgfpathrectangle{\pgfqpoint{0.648703in}{0.548769in}}{\pgfqpoint{5.201297in}{3.102590in}}%
\pgfusepath{clip}%
\pgfsetbuttcap%
\pgfsetroundjoin%
\definecolor{currentfill}{rgb}{0.121569,0.466667,0.705882}%
\pgfsetfillcolor{currentfill}%
\pgfsetlinewidth{1.003750pt}%
\definecolor{currentstroke}{rgb}{0.121569,0.466667,0.705882}%
\pgfsetstrokecolor{currentstroke}%
\pgfsetdash{}{0pt}%
\pgfpathmoveto{\pgfqpoint{4.291214in}{3.128542in}}%
\pgfpathcurveto{\pgfqpoint{4.302264in}{3.128542in}}{\pgfqpoint{4.312863in}{3.132932in}}{\pgfqpoint{4.320677in}{3.140746in}}%
\pgfpathcurveto{\pgfqpoint{4.328490in}{3.148559in}}{\pgfqpoint{4.332881in}{3.159158in}}{\pgfqpoint{4.332881in}{3.170208in}}%
\pgfpathcurveto{\pgfqpoint{4.332881in}{3.181258in}}{\pgfqpoint{4.328490in}{3.191857in}}{\pgfqpoint{4.320677in}{3.199671in}}%
\pgfpathcurveto{\pgfqpoint{4.312863in}{3.207485in}}{\pgfqpoint{4.302264in}{3.211875in}}{\pgfqpoint{4.291214in}{3.211875in}}%
\pgfpathcurveto{\pgfqpoint{4.280164in}{3.211875in}}{\pgfqpoint{4.269565in}{3.207485in}}{\pgfqpoint{4.261751in}{3.199671in}}%
\pgfpathcurveto{\pgfqpoint{4.253937in}{3.191857in}}{\pgfqpoint{4.249547in}{3.181258in}}{\pgfqpoint{4.249547in}{3.170208in}}%
\pgfpathcurveto{\pgfqpoint{4.249547in}{3.159158in}}{\pgfqpoint{4.253937in}{3.148559in}}{\pgfqpoint{4.261751in}{3.140746in}}%
\pgfpathcurveto{\pgfqpoint{4.269565in}{3.132932in}}{\pgfqpoint{4.280164in}{3.128542in}}{\pgfqpoint{4.291214in}{3.128542in}}%
\pgfpathclose%
\pgfusepath{stroke,fill}%
\end{pgfscope}%
\begin{pgfscope}%
\pgfpathrectangle{\pgfqpoint{0.648703in}{0.548769in}}{\pgfqpoint{5.201297in}{3.102590in}}%
\pgfusepath{clip}%
\pgfsetbuttcap%
\pgfsetroundjoin%
\definecolor{currentfill}{rgb}{1.000000,0.498039,0.054902}%
\pgfsetfillcolor{currentfill}%
\pgfsetlinewidth{1.003750pt}%
\definecolor{currentstroke}{rgb}{1.000000,0.498039,0.054902}%
\pgfsetstrokecolor{currentstroke}%
\pgfsetdash{}{0pt}%
\pgfpathmoveto{\pgfqpoint{3.650068in}{3.315195in}}%
\pgfpathcurveto{\pgfqpoint{3.661118in}{3.315195in}}{\pgfqpoint{3.671717in}{3.319585in}}{\pgfqpoint{3.679531in}{3.327399in}}%
\pgfpathcurveto{\pgfqpoint{3.687344in}{3.335212in}}{\pgfqpoint{3.691735in}{3.345811in}}{\pgfqpoint{3.691735in}{3.356861in}}%
\pgfpathcurveto{\pgfqpoint{3.691735in}{3.367912in}}{\pgfqpoint{3.687344in}{3.378511in}}{\pgfqpoint{3.679531in}{3.386324in}}%
\pgfpathcurveto{\pgfqpoint{3.671717in}{3.394138in}}{\pgfqpoint{3.661118in}{3.398528in}}{\pgfqpoint{3.650068in}{3.398528in}}%
\pgfpathcurveto{\pgfqpoint{3.639018in}{3.398528in}}{\pgfqpoint{3.628419in}{3.394138in}}{\pgfqpoint{3.620605in}{3.386324in}}%
\pgfpathcurveto{\pgfqpoint{3.612792in}{3.378511in}}{\pgfqpoint{3.608401in}{3.367912in}}{\pgfqpoint{3.608401in}{3.356861in}}%
\pgfpathcurveto{\pgfqpoint{3.608401in}{3.345811in}}{\pgfqpoint{3.612792in}{3.335212in}}{\pgfqpoint{3.620605in}{3.327399in}}%
\pgfpathcurveto{\pgfqpoint{3.628419in}{3.319585in}}{\pgfqpoint{3.639018in}{3.315195in}}{\pgfqpoint{3.650068in}{3.315195in}}%
\pgfpathclose%
\pgfusepath{stroke,fill}%
\end{pgfscope}%
\begin{pgfscope}%
\pgfpathrectangle{\pgfqpoint{0.648703in}{0.548769in}}{\pgfqpoint{5.201297in}{3.102590in}}%
\pgfusepath{clip}%
\pgfsetbuttcap%
\pgfsetroundjoin%
\definecolor{currentfill}{rgb}{1.000000,0.498039,0.054902}%
\pgfsetfillcolor{currentfill}%
\pgfsetlinewidth{1.003750pt}%
\definecolor{currentstroke}{rgb}{1.000000,0.498039,0.054902}%
\pgfsetstrokecolor{currentstroke}%
\pgfsetdash{}{0pt}%
\pgfpathmoveto{\pgfqpoint{3.209280in}{3.136837in}}%
\pgfpathcurveto{\pgfqpoint{3.220330in}{3.136837in}}{\pgfqpoint{3.230929in}{3.141228in}}{\pgfqpoint{3.238743in}{3.149041in}}%
\pgfpathcurveto{\pgfqpoint{3.246556in}{3.156855in}}{\pgfqpoint{3.250947in}{3.167454in}}{\pgfqpoint{3.250947in}{3.178504in}}%
\pgfpathcurveto{\pgfqpoint{3.250947in}{3.189554in}}{\pgfqpoint{3.246556in}{3.200153in}}{\pgfqpoint{3.238743in}{3.207967in}}%
\pgfpathcurveto{\pgfqpoint{3.230929in}{3.215780in}}{\pgfqpoint{3.220330in}{3.220171in}}{\pgfqpoint{3.209280in}{3.220171in}}%
\pgfpathcurveto{\pgfqpoint{3.198230in}{3.220171in}}{\pgfqpoint{3.187631in}{3.215780in}}{\pgfqpoint{3.179817in}{3.207967in}}%
\pgfpathcurveto{\pgfqpoint{3.172004in}{3.200153in}}{\pgfqpoint{3.167613in}{3.189554in}}{\pgfqpoint{3.167613in}{3.178504in}}%
\pgfpathcurveto{\pgfqpoint{3.167613in}{3.167454in}}{\pgfqpoint{3.172004in}{3.156855in}}{\pgfqpoint{3.179817in}{3.149041in}}%
\pgfpathcurveto{\pgfqpoint{3.187631in}{3.141228in}}{\pgfqpoint{3.198230in}{3.136837in}}{\pgfqpoint{3.209280in}{3.136837in}}%
\pgfpathclose%
\pgfusepath{stroke,fill}%
\end{pgfscope}%
\begin{pgfscope}%
\pgfpathrectangle{\pgfqpoint{0.648703in}{0.548769in}}{\pgfqpoint{5.201297in}{3.102590in}}%
\pgfusepath{clip}%
\pgfsetbuttcap%
\pgfsetroundjoin%
\definecolor{currentfill}{rgb}{1.000000,0.498039,0.054902}%
\pgfsetfillcolor{currentfill}%
\pgfsetlinewidth{1.003750pt}%
\definecolor{currentstroke}{rgb}{1.000000,0.498039,0.054902}%
\pgfsetstrokecolor{currentstroke}%
\pgfsetdash{}{0pt}%
\pgfpathmoveto{\pgfqpoint{3.690140in}{3.136837in}}%
\pgfpathcurveto{\pgfqpoint{3.701190in}{3.136837in}}{\pgfqpoint{3.711789in}{3.141228in}}{\pgfqpoint{3.719602in}{3.149041in}}%
\pgfpathcurveto{\pgfqpoint{3.727416in}{3.156855in}}{\pgfqpoint{3.731806in}{3.167454in}}{\pgfqpoint{3.731806in}{3.178504in}}%
\pgfpathcurveto{\pgfqpoint{3.731806in}{3.189554in}}{\pgfqpoint{3.727416in}{3.200153in}}{\pgfqpoint{3.719602in}{3.207967in}}%
\pgfpathcurveto{\pgfqpoint{3.711789in}{3.215780in}}{\pgfqpoint{3.701190in}{3.220171in}}{\pgfqpoint{3.690140in}{3.220171in}}%
\pgfpathcurveto{\pgfqpoint{3.679089in}{3.220171in}}{\pgfqpoint{3.668490in}{3.215780in}}{\pgfqpoint{3.660677in}{3.207967in}}%
\pgfpathcurveto{\pgfqpoint{3.652863in}{3.200153in}}{\pgfqpoint{3.648473in}{3.189554in}}{\pgfqpoint{3.648473in}{3.178504in}}%
\pgfpathcurveto{\pgfqpoint{3.648473in}{3.167454in}}{\pgfqpoint{3.652863in}{3.156855in}}{\pgfqpoint{3.660677in}{3.149041in}}%
\pgfpathcurveto{\pgfqpoint{3.668490in}{3.141228in}}{\pgfqpoint{3.679089in}{3.136837in}}{\pgfqpoint{3.690140in}{3.136837in}}%
\pgfpathclose%
\pgfusepath{stroke,fill}%
\end{pgfscope}%
\begin{pgfscope}%
\pgfpathrectangle{\pgfqpoint{0.648703in}{0.548769in}}{\pgfqpoint{5.201297in}{3.102590in}}%
\pgfusepath{clip}%
\pgfsetbuttcap%
\pgfsetroundjoin%
\definecolor{currentfill}{rgb}{1.000000,0.498039,0.054902}%
\pgfsetfillcolor{currentfill}%
\pgfsetlinewidth{1.003750pt}%
\definecolor{currentstroke}{rgb}{1.000000,0.498039,0.054902}%
\pgfsetstrokecolor{currentstroke}%
\pgfsetdash{}{0pt}%
\pgfpathmoveto{\pgfqpoint{4.251142in}{3.140985in}}%
\pgfpathcurveto{\pgfqpoint{4.262192in}{3.140985in}}{\pgfqpoint{4.272791in}{3.145375in}}{\pgfqpoint{4.280605in}{3.153189in}}%
\pgfpathcurveto{\pgfqpoint{4.288419in}{3.161003in}}{\pgfqpoint{4.292809in}{3.171602in}}{\pgfqpoint{4.292809in}{3.182652in}}%
\pgfpathcurveto{\pgfqpoint{4.292809in}{3.193702in}}{\pgfqpoint{4.288419in}{3.204301in}}{\pgfqpoint{4.280605in}{3.212115in}}%
\pgfpathcurveto{\pgfqpoint{4.272791in}{3.219928in}}{\pgfqpoint{4.262192in}{3.224319in}}{\pgfqpoint{4.251142in}{3.224319in}}%
\pgfpathcurveto{\pgfqpoint{4.240092in}{3.224319in}}{\pgfqpoint{4.229493in}{3.219928in}}{\pgfqpoint{4.221679in}{3.212115in}}%
\pgfpathcurveto{\pgfqpoint{4.213866in}{3.204301in}}{\pgfqpoint{4.209476in}{3.193702in}}{\pgfqpoint{4.209476in}{3.182652in}}%
\pgfpathcurveto{\pgfqpoint{4.209476in}{3.171602in}}{\pgfqpoint{4.213866in}{3.161003in}}{\pgfqpoint{4.221679in}{3.153189in}}%
\pgfpathcurveto{\pgfqpoint{4.229493in}{3.145375in}}{\pgfqpoint{4.240092in}{3.140985in}}{\pgfqpoint{4.251142in}{3.140985in}}%
\pgfpathclose%
\pgfusepath{stroke,fill}%
\end{pgfscope}%
\begin{pgfscope}%
\pgfpathrectangle{\pgfqpoint{0.648703in}{0.548769in}}{\pgfqpoint{5.201297in}{3.102590in}}%
\pgfusepath{clip}%
\pgfsetbuttcap%
\pgfsetroundjoin%
\definecolor{currentfill}{rgb}{1.000000,0.498039,0.054902}%
\pgfsetfillcolor{currentfill}%
\pgfsetlinewidth{1.003750pt}%
\definecolor{currentstroke}{rgb}{1.000000,0.498039,0.054902}%
\pgfsetstrokecolor{currentstroke}%
\pgfsetdash{}{0pt}%
\pgfpathmoveto{\pgfqpoint{3.209280in}{3.136837in}}%
\pgfpathcurveto{\pgfqpoint{3.220330in}{3.136837in}}{\pgfqpoint{3.230929in}{3.141228in}}{\pgfqpoint{3.238743in}{3.149041in}}%
\pgfpathcurveto{\pgfqpoint{3.246556in}{3.156855in}}{\pgfqpoint{3.250947in}{3.167454in}}{\pgfqpoint{3.250947in}{3.178504in}}%
\pgfpathcurveto{\pgfqpoint{3.250947in}{3.189554in}}{\pgfqpoint{3.246556in}{3.200153in}}{\pgfqpoint{3.238743in}{3.207967in}}%
\pgfpathcurveto{\pgfqpoint{3.230929in}{3.215780in}}{\pgfqpoint{3.220330in}{3.220171in}}{\pgfqpoint{3.209280in}{3.220171in}}%
\pgfpathcurveto{\pgfqpoint{3.198230in}{3.220171in}}{\pgfqpoint{3.187631in}{3.215780in}}{\pgfqpoint{3.179817in}{3.207967in}}%
\pgfpathcurveto{\pgfqpoint{3.172004in}{3.200153in}}{\pgfqpoint{3.167613in}{3.189554in}}{\pgfqpoint{3.167613in}{3.178504in}}%
\pgfpathcurveto{\pgfqpoint{3.167613in}{3.167454in}}{\pgfqpoint{3.172004in}{3.156855in}}{\pgfqpoint{3.179817in}{3.149041in}}%
\pgfpathcurveto{\pgfqpoint{3.187631in}{3.141228in}}{\pgfqpoint{3.198230in}{3.136837in}}{\pgfqpoint{3.209280in}{3.136837in}}%
\pgfpathclose%
\pgfusepath{stroke,fill}%
\end{pgfscope}%
\begin{pgfscope}%
\pgfpathrectangle{\pgfqpoint{0.648703in}{0.548769in}}{\pgfqpoint{5.201297in}{3.102590in}}%
\pgfusepath{clip}%
\pgfsetbuttcap%
\pgfsetroundjoin%
\definecolor{currentfill}{rgb}{1.000000,0.498039,0.054902}%
\pgfsetfillcolor{currentfill}%
\pgfsetlinewidth{1.003750pt}%
\definecolor{currentstroke}{rgb}{1.000000,0.498039,0.054902}%
\pgfsetstrokecolor{currentstroke}%
\pgfsetdash{}{0pt}%
\pgfpathmoveto{\pgfqpoint{4.371357in}{3.356673in}}%
\pgfpathcurveto{\pgfqpoint{4.382407in}{3.356673in}}{\pgfqpoint{4.393006in}{3.361064in}}{\pgfqpoint{4.400820in}{3.368877in}}%
\pgfpathcurveto{\pgfqpoint{4.408634in}{3.376691in}}{\pgfqpoint{4.413024in}{3.387290in}}{\pgfqpoint{4.413024in}{3.398340in}}%
\pgfpathcurveto{\pgfqpoint{4.413024in}{3.409390in}}{\pgfqpoint{4.408634in}{3.419989in}}{\pgfqpoint{4.400820in}{3.427803in}}%
\pgfpathcurveto{\pgfqpoint{4.393006in}{3.435616in}}{\pgfqpoint{4.382407in}{3.440007in}}{\pgfqpoint{4.371357in}{3.440007in}}%
\pgfpathcurveto{\pgfqpoint{4.360307in}{3.440007in}}{\pgfqpoint{4.349708in}{3.435616in}}{\pgfqpoint{4.341894in}{3.427803in}}%
\pgfpathcurveto{\pgfqpoint{4.334081in}{3.419989in}}{\pgfqpoint{4.329690in}{3.409390in}}{\pgfqpoint{4.329690in}{3.398340in}}%
\pgfpathcurveto{\pgfqpoint{4.329690in}{3.387290in}}{\pgfqpoint{4.334081in}{3.376691in}}{\pgfqpoint{4.341894in}{3.368877in}}%
\pgfpathcurveto{\pgfqpoint{4.349708in}{3.361064in}}{\pgfqpoint{4.360307in}{3.356673in}}{\pgfqpoint{4.371357in}{3.356673in}}%
\pgfpathclose%
\pgfusepath{stroke,fill}%
\end{pgfscope}%
\begin{pgfscope}%
\pgfpathrectangle{\pgfqpoint{0.648703in}{0.548769in}}{\pgfqpoint{5.201297in}{3.102590in}}%
\pgfusepath{clip}%
\pgfsetbuttcap%
\pgfsetroundjoin%
\definecolor{currentfill}{rgb}{1.000000,0.498039,0.054902}%
\pgfsetfillcolor{currentfill}%
\pgfsetlinewidth{1.003750pt}%
\definecolor{currentstroke}{rgb}{1.000000,0.498039,0.054902}%
\pgfsetstrokecolor{currentstroke}%
\pgfsetdash{}{0pt}%
\pgfpathmoveto{\pgfqpoint{3.890498in}{3.219794in}}%
\pgfpathcurveto{\pgfqpoint{3.901548in}{3.219794in}}{\pgfqpoint{3.912147in}{3.224185in}}{\pgfqpoint{3.919960in}{3.231998in}}%
\pgfpathcurveto{\pgfqpoint{3.927774in}{3.239812in}}{\pgfqpoint{3.932164in}{3.250411in}}{\pgfqpoint{3.932164in}{3.261461in}}%
\pgfpathcurveto{\pgfqpoint{3.932164in}{3.272511in}}{\pgfqpoint{3.927774in}{3.283110in}}{\pgfqpoint{3.919960in}{3.290924in}}%
\pgfpathcurveto{\pgfqpoint{3.912147in}{3.298737in}}{\pgfqpoint{3.901548in}{3.303128in}}{\pgfqpoint{3.890498in}{3.303128in}}%
\pgfpathcurveto{\pgfqpoint{3.879448in}{3.303128in}}{\pgfqpoint{3.868849in}{3.298737in}}{\pgfqpoint{3.861035in}{3.290924in}}%
\pgfpathcurveto{\pgfqpoint{3.853221in}{3.283110in}}{\pgfqpoint{3.848831in}{3.272511in}}{\pgfqpoint{3.848831in}{3.261461in}}%
\pgfpathcurveto{\pgfqpoint{3.848831in}{3.250411in}}{\pgfqpoint{3.853221in}{3.239812in}}{\pgfqpoint{3.861035in}{3.231998in}}%
\pgfpathcurveto{\pgfqpoint{3.868849in}{3.224185in}}{\pgfqpoint{3.879448in}{3.219794in}}{\pgfqpoint{3.890498in}{3.219794in}}%
\pgfpathclose%
\pgfusepath{stroke,fill}%
\end{pgfscope}%
\begin{pgfscope}%
\pgfpathrectangle{\pgfqpoint{0.648703in}{0.548769in}}{\pgfqpoint{5.201297in}{3.102590in}}%
\pgfusepath{clip}%
\pgfsetbuttcap%
\pgfsetroundjoin%
\definecolor{currentfill}{rgb}{0.121569,0.466667,0.705882}%
\pgfsetfillcolor{currentfill}%
\pgfsetlinewidth{1.003750pt}%
\definecolor{currentstroke}{rgb}{0.121569,0.466667,0.705882}%
\pgfsetstrokecolor{currentstroke}%
\pgfsetdash{}{0pt}%
\pgfpathmoveto{\pgfqpoint{3.089065in}{3.120246in}}%
\pgfpathcurveto{\pgfqpoint{3.100115in}{3.120246in}}{\pgfqpoint{3.110714in}{3.124636in}}{\pgfqpoint{3.118528in}{3.132450in}}%
\pgfpathcurveto{\pgfqpoint{3.126342in}{3.140263in}}{\pgfqpoint{3.130732in}{3.150862in}}{\pgfqpoint{3.130732in}{3.161913in}}%
\pgfpathcurveto{\pgfqpoint{3.130732in}{3.172963in}}{\pgfqpoint{3.126342in}{3.183562in}}{\pgfqpoint{3.118528in}{3.191375in}}%
\pgfpathcurveto{\pgfqpoint{3.110714in}{3.199189in}}{\pgfqpoint{3.100115in}{3.203579in}}{\pgfqpoint{3.089065in}{3.203579in}}%
\pgfpathcurveto{\pgfqpoint{3.078015in}{3.203579in}}{\pgfqpoint{3.067416in}{3.199189in}}{\pgfqpoint{3.059602in}{3.191375in}}%
\pgfpathcurveto{\pgfqpoint{3.051789in}{3.183562in}}{\pgfqpoint{3.047399in}{3.172963in}}{\pgfqpoint{3.047399in}{3.161913in}}%
\pgfpathcurveto{\pgfqpoint{3.047399in}{3.150862in}}{\pgfqpoint{3.051789in}{3.140263in}}{\pgfqpoint{3.059602in}{3.132450in}}%
\pgfpathcurveto{\pgfqpoint{3.067416in}{3.124636in}}{\pgfqpoint{3.078015in}{3.120246in}}{\pgfqpoint{3.089065in}{3.120246in}}%
\pgfpathclose%
\pgfusepath{stroke,fill}%
\end{pgfscope}%
\begin{pgfscope}%
\pgfpathrectangle{\pgfqpoint{0.648703in}{0.548769in}}{\pgfqpoint{5.201297in}{3.102590in}}%
\pgfusepath{clip}%
\pgfsetbuttcap%
\pgfsetroundjoin%
\definecolor{currentfill}{rgb}{1.000000,0.498039,0.054902}%
\pgfsetfillcolor{currentfill}%
\pgfsetlinewidth{1.003750pt}%
\definecolor{currentstroke}{rgb}{1.000000,0.498039,0.054902}%
\pgfsetstrokecolor{currentstroke}%
\pgfsetdash{}{0pt}%
\pgfpathmoveto{\pgfqpoint{3.930569in}{3.140985in}}%
\pgfpathcurveto{\pgfqpoint{3.941619in}{3.140985in}}{\pgfqpoint{3.952218in}{3.145375in}}{\pgfqpoint{3.960032in}{3.153189in}}%
\pgfpathcurveto{\pgfqpoint{3.967846in}{3.161003in}}{\pgfqpoint{3.972236in}{3.171602in}}{\pgfqpoint{3.972236in}{3.182652in}}%
\pgfpathcurveto{\pgfqpoint{3.972236in}{3.193702in}}{\pgfqpoint{3.967846in}{3.204301in}}{\pgfqpoint{3.960032in}{3.212115in}}%
\pgfpathcurveto{\pgfqpoint{3.952218in}{3.219928in}}{\pgfqpoint{3.941619in}{3.224319in}}{\pgfqpoint{3.930569in}{3.224319in}}%
\pgfpathcurveto{\pgfqpoint{3.919519in}{3.224319in}}{\pgfqpoint{3.908920in}{3.219928in}}{\pgfqpoint{3.901107in}{3.212115in}}%
\pgfpathcurveto{\pgfqpoint{3.893293in}{3.204301in}}{\pgfqpoint{3.888903in}{3.193702in}}{\pgfqpoint{3.888903in}{3.182652in}}%
\pgfpathcurveto{\pgfqpoint{3.888903in}{3.171602in}}{\pgfqpoint{3.893293in}{3.161003in}}{\pgfqpoint{3.901107in}{3.153189in}}%
\pgfpathcurveto{\pgfqpoint{3.908920in}{3.145375in}}{\pgfqpoint{3.919519in}{3.140985in}}{\pgfqpoint{3.930569in}{3.140985in}}%
\pgfpathclose%
\pgfusepath{stroke,fill}%
\end{pgfscope}%
\begin{pgfscope}%
\pgfpathrectangle{\pgfqpoint{0.648703in}{0.548769in}}{\pgfqpoint{5.201297in}{3.102590in}}%
\pgfusepath{clip}%
\pgfsetbuttcap%
\pgfsetroundjoin%
\definecolor{currentfill}{rgb}{1.000000,0.498039,0.054902}%
\pgfsetfillcolor{currentfill}%
\pgfsetlinewidth{1.003750pt}%
\definecolor{currentstroke}{rgb}{1.000000,0.498039,0.054902}%
\pgfsetstrokecolor{currentstroke}%
\pgfsetdash{}{0pt}%
\pgfpathmoveto{\pgfqpoint{3.329495in}{3.145133in}}%
\pgfpathcurveto{\pgfqpoint{3.340545in}{3.145133in}}{\pgfqpoint{3.351144in}{3.149523in}}{\pgfqpoint{3.358958in}{3.157337in}}%
\pgfpathcurveto{\pgfqpoint{3.366771in}{3.165151in}}{\pgfqpoint{3.371162in}{3.175750in}}{\pgfqpoint{3.371162in}{3.186800in}}%
\pgfpathcurveto{\pgfqpoint{3.371162in}{3.197850in}}{\pgfqpoint{3.366771in}{3.208449in}}{\pgfqpoint{3.358958in}{3.216262in}}%
\pgfpathcurveto{\pgfqpoint{3.351144in}{3.224076in}}{\pgfqpoint{3.340545in}{3.228466in}}{\pgfqpoint{3.329495in}{3.228466in}}%
\pgfpathcurveto{\pgfqpoint{3.318445in}{3.228466in}}{\pgfqpoint{3.307846in}{3.224076in}}{\pgfqpoint{3.300032in}{3.216262in}}%
\pgfpathcurveto{\pgfqpoint{3.292219in}{3.208449in}}{\pgfqpoint{3.287828in}{3.197850in}}{\pgfqpoint{3.287828in}{3.186800in}}%
\pgfpathcurveto{\pgfqpoint{3.287828in}{3.175750in}}{\pgfqpoint{3.292219in}{3.165151in}}{\pgfqpoint{3.300032in}{3.157337in}}%
\pgfpathcurveto{\pgfqpoint{3.307846in}{3.149523in}}{\pgfqpoint{3.318445in}{3.145133in}}{\pgfqpoint{3.329495in}{3.145133in}}%
\pgfpathclose%
\pgfusepath{stroke,fill}%
\end{pgfscope}%
\begin{pgfscope}%
\pgfpathrectangle{\pgfqpoint{0.648703in}{0.548769in}}{\pgfqpoint{5.201297in}{3.102590in}}%
\pgfusepath{clip}%
\pgfsetbuttcap%
\pgfsetroundjoin%
\definecolor{currentfill}{rgb}{1.000000,0.498039,0.054902}%
\pgfsetfillcolor{currentfill}%
\pgfsetlinewidth{1.003750pt}%
\definecolor{currentstroke}{rgb}{1.000000,0.498039,0.054902}%
\pgfsetstrokecolor{currentstroke}%
\pgfsetdash{}{0pt}%
\pgfpathmoveto{\pgfqpoint{3.289423in}{3.136837in}}%
\pgfpathcurveto{\pgfqpoint{3.300473in}{3.136837in}}{\pgfqpoint{3.311072in}{3.141228in}}{\pgfqpoint{3.318886in}{3.149041in}}%
\pgfpathcurveto{\pgfqpoint{3.326700in}{3.156855in}}{\pgfqpoint{3.331090in}{3.167454in}}{\pgfqpoint{3.331090in}{3.178504in}}%
\pgfpathcurveto{\pgfqpoint{3.331090in}{3.189554in}}{\pgfqpoint{3.326700in}{3.200153in}}{\pgfqpoint{3.318886in}{3.207967in}}%
\pgfpathcurveto{\pgfqpoint{3.311072in}{3.215780in}}{\pgfqpoint{3.300473in}{3.220171in}}{\pgfqpoint{3.289423in}{3.220171in}}%
\pgfpathcurveto{\pgfqpoint{3.278373in}{3.220171in}}{\pgfqpoint{3.267774in}{3.215780in}}{\pgfqpoint{3.259961in}{3.207967in}}%
\pgfpathcurveto{\pgfqpoint{3.252147in}{3.200153in}}{\pgfqpoint{3.247757in}{3.189554in}}{\pgfqpoint{3.247757in}{3.178504in}}%
\pgfpathcurveto{\pgfqpoint{3.247757in}{3.167454in}}{\pgfqpoint{3.252147in}{3.156855in}}{\pgfqpoint{3.259961in}{3.149041in}}%
\pgfpathcurveto{\pgfqpoint{3.267774in}{3.141228in}}{\pgfqpoint{3.278373in}{3.136837in}}{\pgfqpoint{3.289423in}{3.136837in}}%
\pgfpathclose%
\pgfusepath{stroke,fill}%
\end{pgfscope}%
\begin{pgfscope}%
\pgfpathrectangle{\pgfqpoint{0.648703in}{0.548769in}}{\pgfqpoint{5.201297in}{3.102590in}}%
\pgfusepath{clip}%
\pgfsetbuttcap%
\pgfsetroundjoin%
\definecolor{currentfill}{rgb}{0.121569,0.466667,0.705882}%
\pgfsetfillcolor{currentfill}%
\pgfsetlinewidth{1.003750pt}%
\definecolor{currentstroke}{rgb}{0.121569,0.466667,0.705882}%
\pgfsetstrokecolor{currentstroke}%
\pgfsetdash{}{0pt}%
\pgfpathmoveto{\pgfqpoint{2.848635in}{3.132690in}}%
\pgfpathcurveto{\pgfqpoint{2.859686in}{3.132690in}}{\pgfqpoint{2.870285in}{3.137080in}}{\pgfqpoint{2.878098in}{3.144893in}}%
\pgfpathcurveto{\pgfqpoint{2.885912in}{3.152707in}}{\pgfqpoint{2.890302in}{3.163306in}}{\pgfqpoint{2.890302in}{3.174356in}}%
\pgfpathcurveto{\pgfqpoint{2.890302in}{3.185406in}}{\pgfqpoint{2.885912in}{3.196005in}}{\pgfqpoint{2.878098in}{3.203819in}}%
\pgfpathcurveto{\pgfqpoint{2.870285in}{3.211633in}}{\pgfqpoint{2.859686in}{3.216023in}}{\pgfqpoint{2.848635in}{3.216023in}}%
\pgfpathcurveto{\pgfqpoint{2.837585in}{3.216023in}}{\pgfqpoint{2.826986in}{3.211633in}}{\pgfqpoint{2.819173in}{3.203819in}}%
\pgfpathcurveto{\pgfqpoint{2.811359in}{3.196005in}}{\pgfqpoint{2.806969in}{3.185406in}}{\pgfqpoint{2.806969in}{3.174356in}}%
\pgfpathcurveto{\pgfqpoint{2.806969in}{3.163306in}}{\pgfqpoint{2.811359in}{3.152707in}}{\pgfqpoint{2.819173in}{3.144893in}}%
\pgfpathcurveto{\pgfqpoint{2.826986in}{3.137080in}}{\pgfqpoint{2.837585in}{3.132690in}}{\pgfqpoint{2.848635in}{3.132690in}}%
\pgfpathclose%
\pgfusepath{stroke,fill}%
\end{pgfscope}%
\begin{pgfscope}%
\pgfpathrectangle{\pgfqpoint{0.648703in}{0.548769in}}{\pgfqpoint{5.201297in}{3.102590in}}%
\pgfusepath{clip}%
\pgfsetbuttcap%
\pgfsetroundjoin%
\definecolor{currentfill}{rgb}{0.121569,0.466667,0.705882}%
\pgfsetfillcolor{currentfill}%
\pgfsetlinewidth{1.003750pt}%
\definecolor{currentstroke}{rgb}{0.121569,0.466667,0.705882}%
\pgfsetstrokecolor{currentstroke}%
\pgfsetdash{}{0pt}%
\pgfpathmoveto{\pgfqpoint{3.409638in}{3.132690in}}%
\pgfpathcurveto{\pgfqpoint{3.420688in}{3.132690in}}{\pgfqpoint{3.431287in}{3.137080in}}{\pgfqpoint{3.439101in}{3.144893in}}%
\pgfpathcurveto{\pgfqpoint{3.446915in}{3.152707in}}{\pgfqpoint{3.451305in}{3.163306in}}{\pgfqpoint{3.451305in}{3.174356in}}%
\pgfpathcurveto{\pgfqpoint{3.451305in}{3.185406in}}{\pgfqpoint{3.446915in}{3.196005in}}{\pgfqpoint{3.439101in}{3.203819in}}%
\pgfpathcurveto{\pgfqpoint{3.431287in}{3.211633in}}{\pgfqpoint{3.420688in}{3.216023in}}{\pgfqpoint{3.409638in}{3.216023in}}%
\pgfpathcurveto{\pgfqpoint{3.398588in}{3.216023in}}{\pgfqpoint{3.387989in}{3.211633in}}{\pgfqpoint{3.380175in}{3.203819in}}%
\pgfpathcurveto{\pgfqpoint{3.372362in}{3.196005in}}{\pgfqpoint{3.367972in}{3.185406in}}{\pgfqpoint{3.367972in}{3.174356in}}%
\pgfpathcurveto{\pgfqpoint{3.367972in}{3.163306in}}{\pgfqpoint{3.372362in}{3.152707in}}{\pgfqpoint{3.380175in}{3.144893in}}%
\pgfpathcurveto{\pgfqpoint{3.387989in}{3.137080in}}{\pgfqpoint{3.398588in}{3.132690in}}{\pgfqpoint{3.409638in}{3.132690in}}%
\pgfpathclose%
\pgfusepath{stroke,fill}%
\end{pgfscope}%
\begin{pgfscope}%
\pgfpathrectangle{\pgfqpoint{0.648703in}{0.548769in}}{\pgfqpoint{5.201297in}{3.102590in}}%
\pgfusepath{clip}%
\pgfsetbuttcap%
\pgfsetroundjoin%
\definecolor{currentfill}{rgb}{1.000000,0.498039,0.054902}%
\pgfsetfillcolor{currentfill}%
\pgfsetlinewidth{1.003750pt}%
\definecolor{currentstroke}{rgb}{1.000000,0.498039,0.054902}%
\pgfsetstrokecolor{currentstroke}%
\pgfsetdash{}{0pt}%
\pgfpathmoveto{\pgfqpoint{5.613577in}{3.136837in}}%
\pgfpathcurveto{\pgfqpoint{5.624628in}{3.136837in}}{\pgfqpoint{5.635227in}{3.141228in}}{\pgfqpoint{5.643040in}{3.149041in}}%
\pgfpathcurveto{\pgfqpoint{5.650854in}{3.156855in}}{\pgfqpoint{5.655244in}{3.167454in}}{\pgfqpoint{5.655244in}{3.178504in}}%
\pgfpathcurveto{\pgfqpoint{5.655244in}{3.189554in}}{\pgfqpoint{5.650854in}{3.200153in}}{\pgfqpoint{5.643040in}{3.207967in}}%
\pgfpathcurveto{\pgfqpoint{5.635227in}{3.215780in}}{\pgfqpoint{5.624628in}{3.220171in}}{\pgfqpoint{5.613577in}{3.220171in}}%
\pgfpathcurveto{\pgfqpoint{5.602527in}{3.220171in}}{\pgfqpoint{5.591928in}{3.215780in}}{\pgfqpoint{5.584115in}{3.207967in}}%
\pgfpathcurveto{\pgfqpoint{5.576301in}{3.200153in}}{\pgfqpoint{5.571911in}{3.189554in}}{\pgfqpoint{5.571911in}{3.178504in}}%
\pgfpathcurveto{\pgfqpoint{5.571911in}{3.167454in}}{\pgfqpoint{5.576301in}{3.156855in}}{\pgfqpoint{5.584115in}{3.149041in}}%
\pgfpathcurveto{\pgfqpoint{5.591928in}{3.141228in}}{\pgfqpoint{5.602527in}{3.136837in}}{\pgfqpoint{5.613577in}{3.136837in}}%
\pgfpathclose%
\pgfusepath{stroke,fill}%
\end{pgfscope}%
\begin{pgfscope}%
\pgfpathrectangle{\pgfqpoint{0.648703in}{0.548769in}}{\pgfqpoint{5.201297in}{3.102590in}}%
\pgfusepath{clip}%
\pgfsetbuttcap%
\pgfsetroundjoin%
\definecolor{currentfill}{rgb}{0.839216,0.152941,0.156863}%
\pgfsetfillcolor{currentfill}%
\pgfsetlinewidth{1.003750pt}%
\definecolor{currentstroke}{rgb}{0.839216,0.152941,0.156863}%
\pgfsetstrokecolor{currentstroke}%
\pgfsetdash{}{0pt}%
\pgfpathmoveto{\pgfqpoint{3.529853in}{3.120246in}}%
\pgfpathcurveto{\pgfqpoint{3.540903in}{3.120246in}}{\pgfqpoint{3.551502in}{3.124636in}}{\pgfqpoint{3.559316in}{3.132450in}}%
\pgfpathcurveto{\pgfqpoint{3.567129in}{3.140263in}}{\pgfqpoint{3.571520in}{3.150862in}}{\pgfqpoint{3.571520in}{3.161913in}}%
\pgfpathcurveto{\pgfqpoint{3.571520in}{3.172963in}}{\pgfqpoint{3.567129in}{3.183562in}}{\pgfqpoint{3.559316in}{3.191375in}}%
\pgfpathcurveto{\pgfqpoint{3.551502in}{3.199189in}}{\pgfqpoint{3.540903in}{3.203579in}}{\pgfqpoint{3.529853in}{3.203579in}}%
\pgfpathcurveto{\pgfqpoint{3.518803in}{3.203579in}}{\pgfqpoint{3.508204in}{3.199189in}}{\pgfqpoint{3.500390in}{3.191375in}}%
\pgfpathcurveto{\pgfqpoint{3.492577in}{3.183562in}}{\pgfqpoint{3.488186in}{3.172963in}}{\pgfqpoint{3.488186in}{3.161913in}}%
\pgfpathcurveto{\pgfqpoint{3.488186in}{3.150862in}}{\pgfqpoint{3.492577in}{3.140263in}}{\pgfqpoint{3.500390in}{3.132450in}}%
\pgfpathcurveto{\pgfqpoint{3.508204in}{3.124636in}}{\pgfqpoint{3.518803in}{3.120246in}}{\pgfqpoint{3.529853in}{3.120246in}}%
\pgfpathclose%
\pgfusepath{stroke,fill}%
\end{pgfscope}%
\begin{pgfscope}%
\pgfpathrectangle{\pgfqpoint{0.648703in}{0.548769in}}{\pgfqpoint{5.201297in}{3.102590in}}%
\pgfusepath{clip}%
\pgfsetbuttcap%
\pgfsetroundjoin%
\definecolor{currentfill}{rgb}{1.000000,0.498039,0.054902}%
\pgfsetfillcolor{currentfill}%
\pgfsetlinewidth{1.003750pt}%
\definecolor{currentstroke}{rgb}{1.000000,0.498039,0.054902}%
\pgfsetstrokecolor{currentstroke}%
\pgfsetdash{}{0pt}%
\pgfpathmoveto{\pgfqpoint{3.690140in}{3.244681in}}%
\pgfpathcurveto{\pgfqpoint{3.701190in}{3.244681in}}{\pgfqpoint{3.711789in}{3.249072in}}{\pgfqpoint{3.719602in}{3.256885in}}%
\pgfpathcurveto{\pgfqpoint{3.727416in}{3.264699in}}{\pgfqpoint{3.731806in}{3.275298in}}{\pgfqpoint{3.731806in}{3.286348in}}%
\pgfpathcurveto{\pgfqpoint{3.731806in}{3.297398in}}{\pgfqpoint{3.727416in}{3.307997in}}{\pgfqpoint{3.719602in}{3.315811in}}%
\pgfpathcurveto{\pgfqpoint{3.711789in}{3.323624in}}{\pgfqpoint{3.701190in}{3.328015in}}{\pgfqpoint{3.690140in}{3.328015in}}%
\pgfpathcurveto{\pgfqpoint{3.679089in}{3.328015in}}{\pgfqpoint{3.668490in}{3.323624in}}{\pgfqpoint{3.660677in}{3.315811in}}%
\pgfpathcurveto{\pgfqpoint{3.652863in}{3.307997in}}{\pgfqpoint{3.648473in}{3.297398in}}{\pgfqpoint{3.648473in}{3.286348in}}%
\pgfpathcurveto{\pgfqpoint{3.648473in}{3.275298in}}{\pgfqpoint{3.652863in}{3.264699in}}{\pgfqpoint{3.660677in}{3.256885in}}%
\pgfpathcurveto{\pgfqpoint{3.668490in}{3.249072in}}{\pgfqpoint{3.679089in}{3.244681in}}{\pgfqpoint{3.690140in}{3.244681in}}%
\pgfpathclose%
\pgfusepath{stroke,fill}%
\end{pgfscope}%
\begin{pgfscope}%
\pgfpathrectangle{\pgfqpoint{0.648703in}{0.548769in}}{\pgfqpoint{5.201297in}{3.102590in}}%
\pgfusepath{clip}%
\pgfsetbuttcap%
\pgfsetroundjoin%
\definecolor{currentfill}{rgb}{0.121569,0.466667,0.705882}%
\pgfsetfillcolor{currentfill}%
\pgfsetlinewidth{1.003750pt}%
\definecolor{currentstroke}{rgb}{0.121569,0.466667,0.705882}%
\pgfsetstrokecolor{currentstroke}%
\pgfsetdash{}{0pt}%
\pgfpathmoveto{\pgfqpoint{3.369567in}{3.128542in}}%
\pgfpathcurveto{\pgfqpoint{3.380617in}{3.128542in}}{\pgfqpoint{3.391216in}{3.132932in}}{\pgfqpoint{3.399029in}{3.140746in}}%
\pgfpathcurveto{\pgfqpoint{3.406843in}{3.148559in}}{\pgfqpoint{3.411233in}{3.159158in}}{\pgfqpoint{3.411233in}{3.170208in}}%
\pgfpathcurveto{\pgfqpoint{3.411233in}{3.181258in}}{\pgfqpoint{3.406843in}{3.191857in}}{\pgfqpoint{3.399029in}{3.199671in}}%
\pgfpathcurveto{\pgfqpoint{3.391216in}{3.207485in}}{\pgfqpoint{3.380617in}{3.211875in}}{\pgfqpoint{3.369567in}{3.211875in}}%
\pgfpathcurveto{\pgfqpoint{3.358516in}{3.211875in}}{\pgfqpoint{3.347917in}{3.207485in}}{\pgfqpoint{3.340104in}{3.199671in}}%
\pgfpathcurveto{\pgfqpoint{3.332290in}{3.191857in}}{\pgfqpoint{3.327900in}{3.181258in}}{\pgfqpoint{3.327900in}{3.170208in}}%
\pgfpathcurveto{\pgfqpoint{3.327900in}{3.159158in}}{\pgfqpoint{3.332290in}{3.148559in}}{\pgfqpoint{3.340104in}{3.140746in}}%
\pgfpathcurveto{\pgfqpoint{3.347917in}{3.132932in}}{\pgfqpoint{3.358516in}{3.128542in}}{\pgfqpoint{3.369567in}{3.128542in}}%
\pgfpathclose%
\pgfusepath{stroke,fill}%
\end{pgfscope}%
\begin{pgfscope}%
\pgfpathrectangle{\pgfqpoint{0.648703in}{0.548769in}}{\pgfqpoint{5.201297in}{3.102590in}}%
\pgfusepath{clip}%
\pgfsetbuttcap%
\pgfsetroundjoin%
\definecolor{currentfill}{rgb}{1.000000,0.498039,0.054902}%
\pgfsetfillcolor{currentfill}%
\pgfsetlinewidth{1.003750pt}%
\definecolor{currentstroke}{rgb}{1.000000,0.498039,0.054902}%
\pgfsetstrokecolor{currentstroke}%
\pgfsetdash{}{0pt}%
\pgfpathmoveto{\pgfqpoint{3.169208in}{3.136837in}}%
\pgfpathcurveto{\pgfqpoint{3.180259in}{3.136837in}}{\pgfqpoint{3.190858in}{3.141228in}}{\pgfqpoint{3.198671in}{3.149041in}}%
\pgfpathcurveto{\pgfqpoint{3.206485in}{3.156855in}}{\pgfqpoint{3.210875in}{3.167454in}}{\pgfqpoint{3.210875in}{3.178504in}}%
\pgfpathcurveto{\pgfqpoint{3.210875in}{3.189554in}}{\pgfqpoint{3.206485in}{3.200153in}}{\pgfqpoint{3.198671in}{3.207967in}}%
\pgfpathcurveto{\pgfqpoint{3.190858in}{3.215780in}}{\pgfqpoint{3.180259in}{3.220171in}}{\pgfqpoint{3.169208in}{3.220171in}}%
\pgfpathcurveto{\pgfqpoint{3.158158in}{3.220171in}}{\pgfqpoint{3.147559in}{3.215780in}}{\pgfqpoint{3.139746in}{3.207967in}}%
\pgfpathcurveto{\pgfqpoint{3.131932in}{3.200153in}}{\pgfqpoint{3.127542in}{3.189554in}}{\pgfqpoint{3.127542in}{3.178504in}}%
\pgfpathcurveto{\pgfqpoint{3.127542in}{3.167454in}}{\pgfqpoint{3.131932in}{3.156855in}}{\pgfqpoint{3.139746in}{3.149041in}}%
\pgfpathcurveto{\pgfqpoint{3.147559in}{3.141228in}}{\pgfqpoint{3.158158in}{3.136837in}}{\pgfqpoint{3.169208in}{3.136837in}}%
\pgfpathclose%
\pgfusepath{stroke,fill}%
\end{pgfscope}%
\begin{pgfscope}%
\pgfpathrectangle{\pgfqpoint{0.648703in}{0.548769in}}{\pgfqpoint{5.201297in}{3.102590in}}%
\pgfusepath{clip}%
\pgfsetbuttcap%
\pgfsetroundjoin%
\definecolor{currentfill}{rgb}{1.000000,0.498039,0.054902}%
\pgfsetfillcolor{currentfill}%
\pgfsetlinewidth{1.003750pt}%
\definecolor{currentstroke}{rgb}{1.000000,0.498039,0.054902}%
\pgfsetstrokecolor{currentstroke}%
\pgfsetdash{}{0pt}%
\pgfpathmoveto{\pgfqpoint{3.529853in}{3.136837in}}%
\pgfpathcurveto{\pgfqpoint{3.540903in}{3.136837in}}{\pgfqpoint{3.551502in}{3.141228in}}{\pgfqpoint{3.559316in}{3.149041in}}%
\pgfpathcurveto{\pgfqpoint{3.567129in}{3.156855in}}{\pgfqpoint{3.571520in}{3.167454in}}{\pgfqpoint{3.571520in}{3.178504in}}%
\pgfpathcurveto{\pgfqpoint{3.571520in}{3.189554in}}{\pgfqpoint{3.567129in}{3.200153in}}{\pgfqpoint{3.559316in}{3.207967in}}%
\pgfpathcurveto{\pgfqpoint{3.551502in}{3.215780in}}{\pgfqpoint{3.540903in}{3.220171in}}{\pgfqpoint{3.529853in}{3.220171in}}%
\pgfpathcurveto{\pgfqpoint{3.518803in}{3.220171in}}{\pgfqpoint{3.508204in}{3.215780in}}{\pgfqpoint{3.500390in}{3.207967in}}%
\pgfpathcurveto{\pgfqpoint{3.492577in}{3.200153in}}{\pgfqpoint{3.488186in}{3.189554in}}{\pgfqpoint{3.488186in}{3.178504in}}%
\pgfpathcurveto{\pgfqpoint{3.488186in}{3.167454in}}{\pgfqpoint{3.492577in}{3.156855in}}{\pgfqpoint{3.500390in}{3.149041in}}%
\pgfpathcurveto{\pgfqpoint{3.508204in}{3.141228in}}{\pgfqpoint{3.518803in}{3.136837in}}{\pgfqpoint{3.529853in}{3.136837in}}%
\pgfpathclose%
\pgfusepath{stroke,fill}%
\end{pgfscope}%
\begin{pgfscope}%
\pgfpathrectangle{\pgfqpoint{0.648703in}{0.548769in}}{\pgfqpoint{5.201297in}{3.102590in}}%
\pgfusepath{clip}%
\pgfsetbuttcap%
\pgfsetroundjoin%
\definecolor{currentfill}{rgb}{1.000000,0.498039,0.054902}%
\pgfsetfillcolor{currentfill}%
\pgfsetlinewidth{1.003750pt}%
\definecolor{currentstroke}{rgb}{1.000000,0.498039,0.054902}%
\pgfsetstrokecolor{currentstroke}%
\pgfsetdash{}{0pt}%
\pgfpathmoveto{\pgfqpoint{3.529853in}{3.136837in}}%
\pgfpathcurveto{\pgfqpoint{3.540903in}{3.136837in}}{\pgfqpoint{3.551502in}{3.141228in}}{\pgfqpoint{3.559316in}{3.149041in}}%
\pgfpathcurveto{\pgfqpoint{3.567129in}{3.156855in}}{\pgfqpoint{3.571520in}{3.167454in}}{\pgfqpoint{3.571520in}{3.178504in}}%
\pgfpathcurveto{\pgfqpoint{3.571520in}{3.189554in}}{\pgfqpoint{3.567129in}{3.200153in}}{\pgfqpoint{3.559316in}{3.207967in}}%
\pgfpathcurveto{\pgfqpoint{3.551502in}{3.215780in}}{\pgfqpoint{3.540903in}{3.220171in}}{\pgfqpoint{3.529853in}{3.220171in}}%
\pgfpathcurveto{\pgfqpoint{3.518803in}{3.220171in}}{\pgfqpoint{3.508204in}{3.215780in}}{\pgfqpoint{3.500390in}{3.207967in}}%
\pgfpathcurveto{\pgfqpoint{3.492577in}{3.200153in}}{\pgfqpoint{3.488186in}{3.189554in}}{\pgfqpoint{3.488186in}{3.178504in}}%
\pgfpathcurveto{\pgfqpoint{3.488186in}{3.167454in}}{\pgfqpoint{3.492577in}{3.156855in}}{\pgfqpoint{3.500390in}{3.149041in}}%
\pgfpathcurveto{\pgfqpoint{3.508204in}{3.141228in}}{\pgfqpoint{3.518803in}{3.136837in}}{\pgfqpoint{3.529853in}{3.136837in}}%
\pgfpathclose%
\pgfusepath{stroke,fill}%
\end{pgfscope}%
\begin{pgfscope}%
\pgfpathrectangle{\pgfqpoint{0.648703in}{0.548769in}}{\pgfqpoint{5.201297in}{3.102590in}}%
\pgfusepath{clip}%
\pgfsetbuttcap%
\pgfsetroundjoin%
\definecolor{currentfill}{rgb}{0.121569,0.466667,0.705882}%
\pgfsetfillcolor{currentfill}%
\pgfsetlinewidth{1.003750pt}%
\definecolor{currentstroke}{rgb}{0.121569,0.466667,0.705882}%
\pgfsetstrokecolor{currentstroke}%
\pgfsetdash{}{0pt}%
\pgfpathmoveto{\pgfqpoint{4.010713in}{3.132690in}}%
\pgfpathcurveto{\pgfqpoint{4.021763in}{3.132690in}}{\pgfqpoint{4.032362in}{3.137080in}}{\pgfqpoint{4.040175in}{3.144893in}}%
\pgfpathcurveto{\pgfqpoint{4.047989in}{3.152707in}}{\pgfqpoint{4.052379in}{3.163306in}}{\pgfqpoint{4.052379in}{3.174356in}}%
\pgfpathcurveto{\pgfqpoint{4.052379in}{3.185406in}}{\pgfqpoint{4.047989in}{3.196005in}}{\pgfqpoint{4.040175in}{3.203819in}}%
\pgfpathcurveto{\pgfqpoint{4.032362in}{3.211633in}}{\pgfqpoint{4.021763in}{3.216023in}}{\pgfqpoint{4.010713in}{3.216023in}}%
\pgfpathcurveto{\pgfqpoint{3.999662in}{3.216023in}}{\pgfqpoint{3.989063in}{3.211633in}}{\pgfqpoint{3.981250in}{3.203819in}}%
\pgfpathcurveto{\pgfqpoint{3.973436in}{3.196005in}}{\pgfqpoint{3.969046in}{3.185406in}}{\pgfqpoint{3.969046in}{3.174356in}}%
\pgfpathcurveto{\pgfqpoint{3.969046in}{3.163306in}}{\pgfqpoint{3.973436in}{3.152707in}}{\pgfqpoint{3.981250in}{3.144893in}}%
\pgfpathcurveto{\pgfqpoint{3.989063in}{3.137080in}}{\pgfqpoint{3.999662in}{3.132690in}}{\pgfqpoint{4.010713in}{3.132690in}}%
\pgfpathclose%
\pgfusepath{stroke,fill}%
\end{pgfscope}%
\begin{pgfscope}%
\pgfpathrectangle{\pgfqpoint{0.648703in}{0.548769in}}{\pgfqpoint{5.201297in}{3.102590in}}%
\pgfusepath{clip}%
\pgfsetbuttcap%
\pgfsetroundjoin%
\definecolor{currentfill}{rgb}{0.121569,0.466667,0.705882}%
\pgfsetfillcolor{currentfill}%
\pgfsetlinewidth{1.003750pt}%
\definecolor{currentstroke}{rgb}{0.121569,0.466667,0.705882}%
\pgfsetstrokecolor{currentstroke}%
\pgfsetdash{}{0pt}%
\pgfpathmoveto{\pgfqpoint{4.331286in}{3.107802in}}%
\pgfpathcurveto{\pgfqpoint{4.342336in}{3.107802in}}{\pgfqpoint{4.352935in}{3.112193in}}{\pgfqpoint{4.360748in}{3.120006in}}%
\pgfpathcurveto{\pgfqpoint{4.368562in}{3.127820in}}{\pgfqpoint{4.372952in}{3.138419in}}{\pgfqpoint{4.372952in}{3.149469in}}%
\pgfpathcurveto{\pgfqpoint{4.372952in}{3.160519in}}{\pgfqpoint{4.368562in}{3.171118in}}{\pgfqpoint{4.360748in}{3.178932in}}%
\pgfpathcurveto{\pgfqpoint{4.352935in}{3.186745in}}{\pgfqpoint{4.342336in}{3.191136in}}{\pgfqpoint{4.331286in}{3.191136in}}%
\pgfpathcurveto{\pgfqpoint{4.320235in}{3.191136in}}{\pgfqpoint{4.309636in}{3.186745in}}{\pgfqpoint{4.301823in}{3.178932in}}%
\pgfpathcurveto{\pgfqpoint{4.294009in}{3.171118in}}{\pgfqpoint{4.289619in}{3.160519in}}{\pgfqpoint{4.289619in}{3.149469in}}%
\pgfpathcurveto{\pgfqpoint{4.289619in}{3.138419in}}{\pgfqpoint{4.294009in}{3.127820in}}{\pgfqpoint{4.301823in}{3.120006in}}%
\pgfpathcurveto{\pgfqpoint{4.309636in}{3.112193in}}{\pgfqpoint{4.320235in}{3.107802in}}{\pgfqpoint{4.331286in}{3.107802in}}%
\pgfpathclose%
\pgfusepath{stroke,fill}%
\end{pgfscope}%
\begin{pgfscope}%
\pgfpathrectangle{\pgfqpoint{0.648703in}{0.548769in}}{\pgfqpoint{5.201297in}{3.102590in}}%
\pgfusepath{clip}%
\pgfsetbuttcap%
\pgfsetroundjoin%
\definecolor{currentfill}{rgb}{0.121569,0.466667,0.705882}%
\pgfsetfillcolor{currentfill}%
\pgfsetlinewidth{1.003750pt}%
\definecolor{currentstroke}{rgb}{0.121569,0.466667,0.705882}%
\pgfsetstrokecolor{currentstroke}%
\pgfsetdash{}{0pt}%
\pgfpathmoveto{\pgfqpoint{3.209280in}{3.132690in}}%
\pgfpathcurveto{\pgfqpoint{3.220330in}{3.132690in}}{\pgfqpoint{3.230929in}{3.137080in}}{\pgfqpoint{3.238743in}{3.144893in}}%
\pgfpathcurveto{\pgfqpoint{3.246556in}{3.152707in}}{\pgfqpoint{3.250947in}{3.163306in}}{\pgfqpoint{3.250947in}{3.174356in}}%
\pgfpathcurveto{\pgfqpoint{3.250947in}{3.185406in}}{\pgfqpoint{3.246556in}{3.196005in}}{\pgfqpoint{3.238743in}{3.203819in}}%
\pgfpathcurveto{\pgfqpoint{3.230929in}{3.211633in}}{\pgfqpoint{3.220330in}{3.216023in}}{\pgfqpoint{3.209280in}{3.216023in}}%
\pgfpathcurveto{\pgfqpoint{3.198230in}{3.216023in}}{\pgfqpoint{3.187631in}{3.211633in}}{\pgfqpoint{3.179817in}{3.203819in}}%
\pgfpathcurveto{\pgfqpoint{3.172004in}{3.196005in}}{\pgfqpoint{3.167613in}{3.185406in}}{\pgfqpoint{3.167613in}{3.174356in}}%
\pgfpathcurveto{\pgfqpoint{3.167613in}{3.163306in}}{\pgfqpoint{3.172004in}{3.152707in}}{\pgfqpoint{3.179817in}{3.144893in}}%
\pgfpathcurveto{\pgfqpoint{3.187631in}{3.137080in}}{\pgfqpoint{3.198230in}{3.132690in}}{\pgfqpoint{3.209280in}{3.132690in}}%
\pgfpathclose%
\pgfusepath{stroke,fill}%
\end{pgfscope}%
\begin{pgfscope}%
\pgfpathrectangle{\pgfqpoint{0.648703in}{0.548769in}}{\pgfqpoint{5.201297in}{3.102590in}}%
\pgfusepath{clip}%
\pgfsetbuttcap%
\pgfsetroundjoin%
\definecolor{currentfill}{rgb}{0.121569,0.466667,0.705882}%
\pgfsetfillcolor{currentfill}%
\pgfsetlinewidth{1.003750pt}%
\definecolor{currentstroke}{rgb}{0.121569,0.466667,0.705882}%
\pgfsetstrokecolor{currentstroke}%
\pgfsetdash{}{0pt}%
\pgfpathmoveto{\pgfqpoint{3.609996in}{0.648129in}}%
\pgfpathcurveto{\pgfqpoint{3.621046in}{0.648129in}}{\pgfqpoint{3.631645in}{0.652519in}}{\pgfqpoint{3.639459in}{0.660333in}}%
\pgfpathcurveto{\pgfqpoint{3.647273in}{0.668146in}}{\pgfqpoint{3.651663in}{0.678745in}}{\pgfqpoint{3.651663in}{0.689796in}}%
\pgfpathcurveto{\pgfqpoint{3.651663in}{0.700846in}}{\pgfqpoint{3.647273in}{0.711445in}}{\pgfqpoint{3.639459in}{0.719258in}}%
\pgfpathcurveto{\pgfqpoint{3.631645in}{0.727072in}}{\pgfqpoint{3.621046in}{0.731462in}}{\pgfqpoint{3.609996in}{0.731462in}}%
\pgfpathcurveto{\pgfqpoint{3.598946in}{0.731462in}}{\pgfqpoint{3.588347in}{0.727072in}}{\pgfqpoint{3.580534in}{0.719258in}}%
\pgfpathcurveto{\pgfqpoint{3.572720in}{0.711445in}}{\pgfqpoint{3.568330in}{0.700846in}}{\pgfqpoint{3.568330in}{0.689796in}}%
\pgfpathcurveto{\pgfqpoint{3.568330in}{0.678745in}}{\pgfqpoint{3.572720in}{0.668146in}}{\pgfqpoint{3.580534in}{0.660333in}}%
\pgfpathcurveto{\pgfqpoint{3.588347in}{0.652519in}}{\pgfqpoint{3.598946in}{0.648129in}}{\pgfqpoint{3.609996in}{0.648129in}}%
\pgfpathclose%
\pgfusepath{stroke,fill}%
\end{pgfscope}%
\begin{pgfscope}%
\pgfpathrectangle{\pgfqpoint{0.648703in}{0.548769in}}{\pgfqpoint{5.201297in}{3.102590in}}%
\pgfusepath{clip}%
\pgfsetbuttcap%
\pgfsetroundjoin%
\definecolor{currentfill}{rgb}{1.000000,0.498039,0.054902}%
\pgfsetfillcolor{currentfill}%
\pgfsetlinewidth{1.003750pt}%
\definecolor{currentstroke}{rgb}{1.000000,0.498039,0.054902}%
\pgfsetstrokecolor{currentstroke}%
\pgfsetdash{}{0pt}%
\pgfpathmoveto{\pgfqpoint{3.609996in}{3.140985in}}%
\pgfpathcurveto{\pgfqpoint{3.621046in}{3.140985in}}{\pgfqpoint{3.631645in}{3.145375in}}{\pgfqpoint{3.639459in}{3.153189in}}%
\pgfpathcurveto{\pgfqpoint{3.647273in}{3.161003in}}{\pgfqpoint{3.651663in}{3.171602in}}{\pgfqpoint{3.651663in}{3.182652in}}%
\pgfpathcurveto{\pgfqpoint{3.651663in}{3.193702in}}{\pgfqpoint{3.647273in}{3.204301in}}{\pgfqpoint{3.639459in}{3.212115in}}%
\pgfpathcurveto{\pgfqpoint{3.631645in}{3.219928in}}{\pgfqpoint{3.621046in}{3.224319in}}{\pgfqpoint{3.609996in}{3.224319in}}%
\pgfpathcurveto{\pgfqpoint{3.598946in}{3.224319in}}{\pgfqpoint{3.588347in}{3.219928in}}{\pgfqpoint{3.580534in}{3.212115in}}%
\pgfpathcurveto{\pgfqpoint{3.572720in}{3.204301in}}{\pgfqpoint{3.568330in}{3.193702in}}{\pgfqpoint{3.568330in}{3.182652in}}%
\pgfpathcurveto{\pgfqpoint{3.568330in}{3.171602in}}{\pgfqpoint{3.572720in}{3.161003in}}{\pgfqpoint{3.580534in}{3.153189in}}%
\pgfpathcurveto{\pgfqpoint{3.588347in}{3.145375in}}{\pgfqpoint{3.598946in}{3.140985in}}{\pgfqpoint{3.609996in}{3.140985in}}%
\pgfpathclose%
\pgfusepath{stroke,fill}%
\end{pgfscope}%
\begin{pgfscope}%
\pgfpathrectangle{\pgfqpoint{0.648703in}{0.548769in}}{\pgfqpoint{5.201297in}{3.102590in}}%
\pgfusepath{clip}%
\pgfsetbuttcap%
\pgfsetroundjoin%
\definecolor{currentfill}{rgb}{1.000000,0.498039,0.054902}%
\pgfsetfillcolor{currentfill}%
\pgfsetlinewidth{1.003750pt}%
\definecolor{currentstroke}{rgb}{1.000000,0.498039,0.054902}%
\pgfsetstrokecolor{currentstroke}%
\pgfsetdash{}{0pt}%
\pgfpathmoveto{\pgfqpoint{3.609996in}{3.145133in}}%
\pgfpathcurveto{\pgfqpoint{3.621046in}{3.145133in}}{\pgfqpoint{3.631645in}{3.149523in}}{\pgfqpoint{3.639459in}{3.157337in}}%
\pgfpathcurveto{\pgfqpoint{3.647273in}{3.165151in}}{\pgfqpoint{3.651663in}{3.175750in}}{\pgfqpoint{3.651663in}{3.186800in}}%
\pgfpathcurveto{\pgfqpoint{3.651663in}{3.197850in}}{\pgfqpoint{3.647273in}{3.208449in}}{\pgfqpoint{3.639459in}{3.216262in}}%
\pgfpathcurveto{\pgfqpoint{3.631645in}{3.224076in}}{\pgfqpoint{3.621046in}{3.228466in}}{\pgfqpoint{3.609996in}{3.228466in}}%
\pgfpathcurveto{\pgfqpoint{3.598946in}{3.228466in}}{\pgfqpoint{3.588347in}{3.224076in}}{\pgfqpoint{3.580534in}{3.216262in}}%
\pgfpathcurveto{\pgfqpoint{3.572720in}{3.208449in}}{\pgfqpoint{3.568330in}{3.197850in}}{\pgfqpoint{3.568330in}{3.186800in}}%
\pgfpathcurveto{\pgfqpoint{3.568330in}{3.175750in}}{\pgfqpoint{3.572720in}{3.165151in}}{\pgfqpoint{3.580534in}{3.157337in}}%
\pgfpathcurveto{\pgfqpoint{3.588347in}{3.149523in}}{\pgfqpoint{3.598946in}{3.145133in}}{\pgfqpoint{3.609996in}{3.145133in}}%
\pgfpathclose%
\pgfusepath{stroke,fill}%
\end{pgfscope}%
\begin{pgfscope}%
\pgfpathrectangle{\pgfqpoint{0.648703in}{0.548769in}}{\pgfqpoint{5.201297in}{3.102590in}}%
\pgfusepath{clip}%
\pgfsetbuttcap%
\pgfsetroundjoin%
\definecolor{currentfill}{rgb}{1.000000,0.498039,0.054902}%
\pgfsetfillcolor{currentfill}%
\pgfsetlinewidth{1.003750pt}%
\definecolor{currentstroke}{rgb}{1.000000,0.498039,0.054902}%
\pgfsetstrokecolor{currentstroke}%
\pgfsetdash{}{0pt}%
\pgfpathmoveto{\pgfqpoint{3.650068in}{3.140985in}}%
\pgfpathcurveto{\pgfqpoint{3.661118in}{3.140985in}}{\pgfqpoint{3.671717in}{3.145375in}}{\pgfqpoint{3.679531in}{3.153189in}}%
\pgfpathcurveto{\pgfqpoint{3.687344in}{3.161003in}}{\pgfqpoint{3.691735in}{3.171602in}}{\pgfqpoint{3.691735in}{3.182652in}}%
\pgfpathcurveto{\pgfqpoint{3.691735in}{3.193702in}}{\pgfqpoint{3.687344in}{3.204301in}}{\pgfqpoint{3.679531in}{3.212115in}}%
\pgfpathcurveto{\pgfqpoint{3.671717in}{3.219928in}}{\pgfqpoint{3.661118in}{3.224319in}}{\pgfqpoint{3.650068in}{3.224319in}}%
\pgfpathcurveto{\pgfqpoint{3.639018in}{3.224319in}}{\pgfqpoint{3.628419in}{3.219928in}}{\pgfqpoint{3.620605in}{3.212115in}}%
\pgfpathcurveto{\pgfqpoint{3.612792in}{3.204301in}}{\pgfqpoint{3.608401in}{3.193702in}}{\pgfqpoint{3.608401in}{3.182652in}}%
\pgfpathcurveto{\pgfqpoint{3.608401in}{3.171602in}}{\pgfqpoint{3.612792in}{3.161003in}}{\pgfqpoint{3.620605in}{3.153189in}}%
\pgfpathcurveto{\pgfqpoint{3.628419in}{3.145375in}}{\pgfqpoint{3.639018in}{3.140985in}}{\pgfqpoint{3.650068in}{3.140985in}}%
\pgfpathclose%
\pgfusepath{stroke,fill}%
\end{pgfscope}%
\begin{pgfscope}%
\pgfpathrectangle{\pgfqpoint{0.648703in}{0.548769in}}{\pgfqpoint{5.201297in}{3.102590in}}%
\pgfusepath{clip}%
\pgfsetbuttcap%
\pgfsetroundjoin%
\definecolor{currentfill}{rgb}{0.121569,0.466667,0.705882}%
\pgfsetfillcolor{currentfill}%
\pgfsetlinewidth{1.003750pt}%
\definecolor{currentstroke}{rgb}{0.121569,0.466667,0.705882}%
\pgfsetstrokecolor{currentstroke}%
\pgfsetdash{}{0pt}%
\pgfpathmoveto{\pgfqpoint{3.169208in}{0.697903in}}%
\pgfpathcurveto{\pgfqpoint{3.180259in}{0.697903in}}{\pgfqpoint{3.190858in}{0.702293in}}{\pgfqpoint{3.198671in}{0.710107in}}%
\pgfpathcurveto{\pgfqpoint{3.206485in}{0.717921in}}{\pgfqpoint{3.210875in}{0.728520in}}{\pgfqpoint{3.210875in}{0.739570in}}%
\pgfpathcurveto{\pgfqpoint{3.210875in}{0.750620in}}{\pgfqpoint{3.206485in}{0.761219in}}{\pgfqpoint{3.198671in}{0.769033in}}%
\pgfpathcurveto{\pgfqpoint{3.190858in}{0.776846in}}{\pgfqpoint{3.180259in}{0.781236in}}{\pgfqpoint{3.169208in}{0.781236in}}%
\pgfpathcurveto{\pgfqpoint{3.158158in}{0.781236in}}{\pgfqpoint{3.147559in}{0.776846in}}{\pgfqpoint{3.139746in}{0.769033in}}%
\pgfpathcurveto{\pgfqpoint{3.131932in}{0.761219in}}{\pgfqpoint{3.127542in}{0.750620in}}{\pgfqpoint{3.127542in}{0.739570in}}%
\pgfpathcurveto{\pgfqpoint{3.127542in}{0.728520in}}{\pgfqpoint{3.131932in}{0.717921in}}{\pgfqpoint{3.139746in}{0.710107in}}%
\pgfpathcurveto{\pgfqpoint{3.147559in}{0.702293in}}{\pgfqpoint{3.158158in}{0.697903in}}{\pgfqpoint{3.169208in}{0.697903in}}%
\pgfpathclose%
\pgfusepath{stroke,fill}%
\end{pgfscope}%
\begin{pgfscope}%
\pgfpathrectangle{\pgfqpoint{0.648703in}{0.548769in}}{\pgfqpoint{5.201297in}{3.102590in}}%
\pgfusepath{clip}%
\pgfsetbuttcap%
\pgfsetroundjoin%
\definecolor{currentfill}{rgb}{1.000000,0.498039,0.054902}%
\pgfsetfillcolor{currentfill}%
\pgfsetlinewidth{1.003750pt}%
\definecolor{currentstroke}{rgb}{1.000000,0.498039,0.054902}%
\pgfsetstrokecolor{currentstroke}%
\pgfsetdash{}{0pt}%
\pgfpathmoveto{\pgfqpoint{4.251142in}{3.140985in}}%
\pgfpathcurveto{\pgfqpoint{4.262192in}{3.140985in}}{\pgfqpoint{4.272791in}{3.145375in}}{\pgfqpoint{4.280605in}{3.153189in}}%
\pgfpathcurveto{\pgfqpoint{4.288419in}{3.161003in}}{\pgfqpoint{4.292809in}{3.171602in}}{\pgfqpoint{4.292809in}{3.182652in}}%
\pgfpathcurveto{\pgfqpoint{4.292809in}{3.193702in}}{\pgfqpoint{4.288419in}{3.204301in}}{\pgfqpoint{4.280605in}{3.212115in}}%
\pgfpathcurveto{\pgfqpoint{4.272791in}{3.219928in}}{\pgfqpoint{4.262192in}{3.224319in}}{\pgfqpoint{4.251142in}{3.224319in}}%
\pgfpathcurveto{\pgfqpoint{4.240092in}{3.224319in}}{\pgfqpoint{4.229493in}{3.219928in}}{\pgfqpoint{4.221679in}{3.212115in}}%
\pgfpathcurveto{\pgfqpoint{4.213866in}{3.204301in}}{\pgfqpoint{4.209476in}{3.193702in}}{\pgfqpoint{4.209476in}{3.182652in}}%
\pgfpathcurveto{\pgfqpoint{4.209476in}{3.171602in}}{\pgfqpoint{4.213866in}{3.161003in}}{\pgfqpoint{4.221679in}{3.153189in}}%
\pgfpathcurveto{\pgfqpoint{4.229493in}{3.145375in}}{\pgfqpoint{4.240092in}{3.140985in}}{\pgfqpoint{4.251142in}{3.140985in}}%
\pgfpathclose%
\pgfusepath{stroke,fill}%
\end{pgfscope}%
\begin{pgfscope}%
\pgfpathrectangle{\pgfqpoint{0.648703in}{0.548769in}}{\pgfqpoint{5.201297in}{3.102590in}}%
\pgfusepath{clip}%
\pgfsetbuttcap%
\pgfsetroundjoin%
\definecolor{currentfill}{rgb}{0.121569,0.466667,0.705882}%
\pgfsetfillcolor{currentfill}%
\pgfsetlinewidth{1.003750pt}%
\definecolor{currentstroke}{rgb}{0.121569,0.466667,0.705882}%
\pgfsetstrokecolor{currentstroke}%
\pgfsetdash{}{0pt}%
\pgfpathmoveto{\pgfqpoint{3.730211in}{3.132690in}}%
\pgfpathcurveto{\pgfqpoint{3.741261in}{3.132690in}}{\pgfqpoint{3.751860in}{3.137080in}}{\pgfqpoint{3.759674in}{3.144893in}}%
\pgfpathcurveto{\pgfqpoint{3.767488in}{3.152707in}}{\pgfqpoint{3.771878in}{3.163306in}}{\pgfqpoint{3.771878in}{3.174356in}}%
\pgfpathcurveto{\pgfqpoint{3.771878in}{3.185406in}}{\pgfqpoint{3.767488in}{3.196005in}}{\pgfqpoint{3.759674in}{3.203819in}}%
\pgfpathcurveto{\pgfqpoint{3.751860in}{3.211633in}}{\pgfqpoint{3.741261in}{3.216023in}}{\pgfqpoint{3.730211in}{3.216023in}}%
\pgfpathcurveto{\pgfqpoint{3.719161in}{3.216023in}}{\pgfqpoint{3.708562in}{3.211633in}}{\pgfqpoint{3.700748in}{3.203819in}}%
\pgfpathcurveto{\pgfqpoint{3.692935in}{3.196005in}}{\pgfqpoint{3.688545in}{3.185406in}}{\pgfqpoint{3.688545in}{3.174356in}}%
\pgfpathcurveto{\pgfqpoint{3.688545in}{3.163306in}}{\pgfqpoint{3.692935in}{3.152707in}}{\pgfqpoint{3.700748in}{3.144893in}}%
\pgfpathcurveto{\pgfqpoint{3.708562in}{3.137080in}}{\pgfqpoint{3.719161in}{3.132690in}}{\pgfqpoint{3.730211in}{3.132690in}}%
\pgfpathclose%
\pgfusepath{stroke,fill}%
\end{pgfscope}%
\begin{pgfscope}%
\pgfpathrectangle{\pgfqpoint{0.648703in}{0.548769in}}{\pgfqpoint{5.201297in}{3.102590in}}%
\pgfusepath{clip}%
\pgfsetbuttcap%
\pgfsetroundjoin%
\definecolor{currentfill}{rgb}{1.000000,0.498039,0.054902}%
\pgfsetfillcolor{currentfill}%
\pgfsetlinewidth{1.003750pt}%
\definecolor{currentstroke}{rgb}{1.000000,0.498039,0.054902}%
\pgfsetstrokecolor{currentstroke}%
\pgfsetdash{}{0pt}%
\pgfpathmoveto{\pgfqpoint{4.050784in}{3.140985in}}%
\pgfpathcurveto{\pgfqpoint{4.061834in}{3.140985in}}{\pgfqpoint{4.072433in}{3.145375in}}{\pgfqpoint{4.080247in}{3.153189in}}%
\pgfpathcurveto{\pgfqpoint{4.088061in}{3.161003in}}{\pgfqpoint{4.092451in}{3.171602in}}{\pgfqpoint{4.092451in}{3.182652in}}%
\pgfpathcurveto{\pgfqpoint{4.092451in}{3.193702in}}{\pgfqpoint{4.088061in}{3.204301in}}{\pgfqpoint{4.080247in}{3.212115in}}%
\pgfpathcurveto{\pgfqpoint{4.072433in}{3.219928in}}{\pgfqpoint{4.061834in}{3.224319in}}{\pgfqpoint{4.050784in}{3.224319in}}%
\pgfpathcurveto{\pgfqpoint{4.039734in}{3.224319in}}{\pgfqpoint{4.029135in}{3.219928in}}{\pgfqpoint{4.021321in}{3.212115in}}%
\pgfpathcurveto{\pgfqpoint{4.013508in}{3.204301in}}{\pgfqpoint{4.009117in}{3.193702in}}{\pgfqpoint{4.009117in}{3.182652in}}%
\pgfpathcurveto{\pgfqpoint{4.009117in}{3.171602in}}{\pgfqpoint{4.013508in}{3.161003in}}{\pgfqpoint{4.021321in}{3.153189in}}%
\pgfpathcurveto{\pgfqpoint{4.029135in}{3.145375in}}{\pgfqpoint{4.039734in}{3.140985in}}{\pgfqpoint{4.050784in}{3.140985in}}%
\pgfpathclose%
\pgfusepath{stroke,fill}%
\end{pgfscope}%
\begin{pgfscope}%
\pgfpathrectangle{\pgfqpoint{0.648703in}{0.548769in}}{\pgfqpoint{5.201297in}{3.102590in}}%
\pgfusepath{clip}%
\pgfsetbuttcap%
\pgfsetroundjoin%
\definecolor{currentfill}{rgb}{0.121569,0.466667,0.705882}%
\pgfsetfillcolor{currentfill}%
\pgfsetlinewidth{1.003750pt}%
\definecolor{currentstroke}{rgb}{0.121569,0.466667,0.705882}%
\pgfsetstrokecolor{currentstroke}%
\pgfsetdash{}{0pt}%
\pgfpathmoveto{\pgfqpoint{3.289423in}{3.132690in}}%
\pgfpathcurveto{\pgfqpoint{3.300473in}{3.132690in}}{\pgfqpoint{3.311072in}{3.137080in}}{\pgfqpoint{3.318886in}{3.144893in}}%
\pgfpathcurveto{\pgfqpoint{3.326700in}{3.152707in}}{\pgfqpoint{3.331090in}{3.163306in}}{\pgfqpoint{3.331090in}{3.174356in}}%
\pgfpathcurveto{\pgfqpoint{3.331090in}{3.185406in}}{\pgfqpoint{3.326700in}{3.196005in}}{\pgfqpoint{3.318886in}{3.203819in}}%
\pgfpathcurveto{\pgfqpoint{3.311072in}{3.211633in}}{\pgfqpoint{3.300473in}{3.216023in}}{\pgfqpoint{3.289423in}{3.216023in}}%
\pgfpathcurveto{\pgfqpoint{3.278373in}{3.216023in}}{\pgfqpoint{3.267774in}{3.211633in}}{\pgfqpoint{3.259961in}{3.203819in}}%
\pgfpathcurveto{\pgfqpoint{3.252147in}{3.196005in}}{\pgfqpoint{3.247757in}{3.185406in}}{\pgfqpoint{3.247757in}{3.174356in}}%
\pgfpathcurveto{\pgfqpoint{3.247757in}{3.163306in}}{\pgfqpoint{3.252147in}{3.152707in}}{\pgfqpoint{3.259961in}{3.144893in}}%
\pgfpathcurveto{\pgfqpoint{3.267774in}{3.137080in}}{\pgfqpoint{3.278373in}{3.132690in}}{\pgfqpoint{3.289423in}{3.132690in}}%
\pgfpathclose%
\pgfusepath{stroke,fill}%
\end{pgfscope}%
\begin{pgfscope}%
\pgfpathrectangle{\pgfqpoint{0.648703in}{0.548769in}}{\pgfqpoint{5.201297in}{3.102590in}}%
\pgfusepath{clip}%
\pgfsetbuttcap%
\pgfsetroundjoin%
\definecolor{currentfill}{rgb}{1.000000,0.498039,0.054902}%
\pgfsetfillcolor{currentfill}%
\pgfsetlinewidth{1.003750pt}%
\definecolor{currentstroke}{rgb}{1.000000,0.498039,0.054902}%
\pgfsetstrokecolor{currentstroke}%
\pgfsetdash{}{0pt}%
\pgfpathmoveto{\pgfqpoint{3.609996in}{3.140985in}}%
\pgfpathcurveto{\pgfqpoint{3.621046in}{3.140985in}}{\pgfqpoint{3.631645in}{3.145375in}}{\pgfqpoint{3.639459in}{3.153189in}}%
\pgfpathcurveto{\pgfqpoint{3.647273in}{3.161003in}}{\pgfqpoint{3.651663in}{3.171602in}}{\pgfqpoint{3.651663in}{3.182652in}}%
\pgfpathcurveto{\pgfqpoint{3.651663in}{3.193702in}}{\pgfqpoint{3.647273in}{3.204301in}}{\pgfqpoint{3.639459in}{3.212115in}}%
\pgfpathcurveto{\pgfqpoint{3.631645in}{3.219928in}}{\pgfqpoint{3.621046in}{3.224319in}}{\pgfqpoint{3.609996in}{3.224319in}}%
\pgfpathcurveto{\pgfqpoint{3.598946in}{3.224319in}}{\pgfqpoint{3.588347in}{3.219928in}}{\pgfqpoint{3.580534in}{3.212115in}}%
\pgfpathcurveto{\pgfqpoint{3.572720in}{3.204301in}}{\pgfqpoint{3.568330in}{3.193702in}}{\pgfqpoint{3.568330in}{3.182652in}}%
\pgfpathcurveto{\pgfqpoint{3.568330in}{3.171602in}}{\pgfqpoint{3.572720in}{3.161003in}}{\pgfqpoint{3.580534in}{3.153189in}}%
\pgfpathcurveto{\pgfqpoint{3.588347in}{3.145375in}}{\pgfqpoint{3.598946in}{3.140985in}}{\pgfqpoint{3.609996in}{3.140985in}}%
\pgfpathclose%
\pgfusepath{stroke,fill}%
\end{pgfscope}%
\begin{pgfscope}%
\pgfpathrectangle{\pgfqpoint{0.648703in}{0.548769in}}{\pgfqpoint{5.201297in}{3.102590in}}%
\pgfusepath{clip}%
\pgfsetbuttcap%
\pgfsetroundjoin%
\definecolor{currentfill}{rgb}{1.000000,0.498039,0.054902}%
\pgfsetfillcolor{currentfill}%
\pgfsetlinewidth{1.003750pt}%
\definecolor{currentstroke}{rgb}{1.000000,0.498039,0.054902}%
\pgfsetstrokecolor{currentstroke}%
\pgfsetdash{}{0pt}%
\pgfpathmoveto{\pgfqpoint{3.169208in}{3.145133in}}%
\pgfpathcurveto{\pgfqpoint{3.180259in}{3.145133in}}{\pgfqpoint{3.190858in}{3.149523in}}{\pgfqpoint{3.198671in}{3.157337in}}%
\pgfpathcurveto{\pgfqpoint{3.206485in}{3.165151in}}{\pgfqpoint{3.210875in}{3.175750in}}{\pgfqpoint{3.210875in}{3.186800in}}%
\pgfpathcurveto{\pgfqpoint{3.210875in}{3.197850in}}{\pgfqpoint{3.206485in}{3.208449in}}{\pgfqpoint{3.198671in}{3.216262in}}%
\pgfpathcurveto{\pgfqpoint{3.190858in}{3.224076in}}{\pgfqpoint{3.180259in}{3.228466in}}{\pgfqpoint{3.169208in}{3.228466in}}%
\pgfpathcurveto{\pgfqpoint{3.158158in}{3.228466in}}{\pgfqpoint{3.147559in}{3.224076in}}{\pgfqpoint{3.139746in}{3.216262in}}%
\pgfpathcurveto{\pgfqpoint{3.131932in}{3.208449in}}{\pgfqpoint{3.127542in}{3.197850in}}{\pgfqpoint{3.127542in}{3.186800in}}%
\pgfpathcurveto{\pgfqpoint{3.127542in}{3.175750in}}{\pgfqpoint{3.131932in}{3.165151in}}{\pgfqpoint{3.139746in}{3.157337in}}%
\pgfpathcurveto{\pgfqpoint{3.147559in}{3.149523in}}{\pgfqpoint{3.158158in}{3.145133in}}{\pgfqpoint{3.169208in}{3.145133in}}%
\pgfpathclose%
\pgfusepath{stroke,fill}%
\end{pgfscope}%
\begin{pgfscope}%
\pgfpathrectangle{\pgfqpoint{0.648703in}{0.548769in}}{\pgfqpoint{5.201297in}{3.102590in}}%
\pgfusepath{clip}%
\pgfsetbuttcap%
\pgfsetroundjoin%
\definecolor{currentfill}{rgb}{1.000000,0.498039,0.054902}%
\pgfsetfillcolor{currentfill}%
\pgfsetlinewidth{1.003750pt}%
\definecolor{currentstroke}{rgb}{1.000000,0.498039,0.054902}%
\pgfsetstrokecolor{currentstroke}%
\pgfsetdash{}{0pt}%
\pgfpathmoveto{\pgfqpoint{3.169208in}{3.145133in}}%
\pgfpathcurveto{\pgfqpoint{3.180259in}{3.145133in}}{\pgfqpoint{3.190858in}{3.149523in}}{\pgfqpoint{3.198671in}{3.157337in}}%
\pgfpathcurveto{\pgfqpoint{3.206485in}{3.165151in}}{\pgfqpoint{3.210875in}{3.175750in}}{\pgfqpoint{3.210875in}{3.186800in}}%
\pgfpathcurveto{\pgfqpoint{3.210875in}{3.197850in}}{\pgfqpoint{3.206485in}{3.208449in}}{\pgfqpoint{3.198671in}{3.216262in}}%
\pgfpathcurveto{\pgfqpoint{3.190858in}{3.224076in}}{\pgfqpoint{3.180259in}{3.228466in}}{\pgfqpoint{3.169208in}{3.228466in}}%
\pgfpathcurveto{\pgfqpoint{3.158158in}{3.228466in}}{\pgfqpoint{3.147559in}{3.224076in}}{\pgfqpoint{3.139746in}{3.216262in}}%
\pgfpathcurveto{\pgfqpoint{3.131932in}{3.208449in}}{\pgfqpoint{3.127542in}{3.197850in}}{\pgfqpoint{3.127542in}{3.186800in}}%
\pgfpathcurveto{\pgfqpoint{3.127542in}{3.175750in}}{\pgfqpoint{3.131932in}{3.165151in}}{\pgfqpoint{3.139746in}{3.157337in}}%
\pgfpathcurveto{\pgfqpoint{3.147559in}{3.149523in}}{\pgfqpoint{3.158158in}{3.145133in}}{\pgfqpoint{3.169208in}{3.145133in}}%
\pgfpathclose%
\pgfusepath{stroke,fill}%
\end{pgfscope}%
\begin{pgfscope}%
\pgfpathrectangle{\pgfqpoint{0.648703in}{0.548769in}}{\pgfqpoint{5.201297in}{3.102590in}}%
\pgfusepath{clip}%
\pgfsetbuttcap%
\pgfsetroundjoin%
\definecolor{currentfill}{rgb}{1.000000,0.498039,0.054902}%
\pgfsetfillcolor{currentfill}%
\pgfsetlinewidth{1.003750pt}%
\definecolor{currentstroke}{rgb}{1.000000,0.498039,0.054902}%
\pgfsetstrokecolor{currentstroke}%
\pgfsetdash{}{0pt}%
\pgfpathmoveto{\pgfqpoint{2.568134in}{3.136837in}}%
\pgfpathcurveto{\pgfqpoint{2.579184in}{3.136837in}}{\pgfqpoint{2.589783in}{3.141228in}}{\pgfqpoint{2.597597in}{3.149041in}}%
\pgfpathcurveto{\pgfqpoint{2.605411in}{3.156855in}}{\pgfqpoint{2.609801in}{3.167454in}}{\pgfqpoint{2.609801in}{3.178504in}}%
\pgfpathcurveto{\pgfqpoint{2.609801in}{3.189554in}}{\pgfqpoint{2.605411in}{3.200153in}}{\pgfqpoint{2.597597in}{3.207967in}}%
\pgfpathcurveto{\pgfqpoint{2.589783in}{3.215780in}}{\pgfqpoint{2.579184in}{3.220171in}}{\pgfqpoint{2.568134in}{3.220171in}}%
\pgfpathcurveto{\pgfqpoint{2.557084in}{3.220171in}}{\pgfqpoint{2.546485in}{3.215780in}}{\pgfqpoint{2.538671in}{3.207967in}}%
\pgfpathcurveto{\pgfqpoint{2.530858in}{3.200153in}}{\pgfqpoint{2.526467in}{3.189554in}}{\pgfqpoint{2.526467in}{3.178504in}}%
\pgfpathcurveto{\pgfqpoint{2.526467in}{3.167454in}}{\pgfqpoint{2.530858in}{3.156855in}}{\pgfqpoint{2.538671in}{3.149041in}}%
\pgfpathcurveto{\pgfqpoint{2.546485in}{3.141228in}}{\pgfqpoint{2.557084in}{3.136837in}}{\pgfqpoint{2.568134in}{3.136837in}}%
\pgfpathclose%
\pgfusepath{stroke,fill}%
\end{pgfscope}%
\begin{pgfscope}%
\pgfpathrectangle{\pgfqpoint{0.648703in}{0.548769in}}{\pgfqpoint{5.201297in}{3.102590in}}%
\pgfusepath{clip}%
\pgfsetbuttcap%
\pgfsetroundjoin%
\definecolor{currentfill}{rgb}{1.000000,0.498039,0.054902}%
\pgfsetfillcolor{currentfill}%
\pgfsetlinewidth{1.003750pt}%
\definecolor{currentstroke}{rgb}{1.000000,0.498039,0.054902}%
\pgfsetstrokecolor{currentstroke}%
\pgfsetdash{}{0pt}%
\pgfpathmoveto{\pgfqpoint{3.770283in}{3.468665in}}%
\pgfpathcurveto{\pgfqpoint{3.781333in}{3.468665in}}{\pgfqpoint{3.791932in}{3.473055in}}{\pgfqpoint{3.799746in}{3.480869in}}%
\pgfpathcurveto{\pgfqpoint{3.807559in}{3.488683in}}{\pgfqpoint{3.811949in}{3.499282in}}{\pgfqpoint{3.811949in}{3.510332in}}%
\pgfpathcurveto{\pgfqpoint{3.811949in}{3.521382in}}{\pgfqpoint{3.807559in}{3.531981in}}{\pgfqpoint{3.799746in}{3.539795in}}%
\pgfpathcurveto{\pgfqpoint{3.791932in}{3.547608in}}{\pgfqpoint{3.781333in}{3.551998in}}{\pgfqpoint{3.770283in}{3.551998in}}%
\pgfpathcurveto{\pgfqpoint{3.759233in}{3.551998in}}{\pgfqpoint{3.748634in}{3.547608in}}{\pgfqpoint{3.740820in}{3.539795in}}%
\pgfpathcurveto{\pgfqpoint{3.733006in}{3.531981in}}{\pgfqpoint{3.728616in}{3.521382in}}{\pgfqpoint{3.728616in}{3.510332in}}%
\pgfpathcurveto{\pgfqpoint{3.728616in}{3.499282in}}{\pgfqpoint{3.733006in}{3.488683in}}{\pgfqpoint{3.740820in}{3.480869in}}%
\pgfpathcurveto{\pgfqpoint{3.748634in}{3.473055in}}{\pgfqpoint{3.759233in}{3.468665in}}{\pgfqpoint{3.770283in}{3.468665in}}%
\pgfpathclose%
\pgfusepath{stroke,fill}%
\end{pgfscope}%
\begin{pgfscope}%
\pgfpathrectangle{\pgfqpoint{0.648703in}{0.548769in}}{\pgfqpoint{5.201297in}{3.102590in}}%
\pgfusepath{clip}%
\pgfsetbuttcap%
\pgfsetroundjoin%
\definecolor{currentfill}{rgb}{1.000000,0.498039,0.054902}%
\pgfsetfillcolor{currentfill}%
\pgfsetlinewidth{1.003750pt}%
\definecolor{currentstroke}{rgb}{1.000000,0.498039,0.054902}%
\pgfsetstrokecolor{currentstroke}%
\pgfsetdash{}{0pt}%
\pgfpathmoveto{\pgfqpoint{3.329495in}{3.136837in}}%
\pgfpathcurveto{\pgfqpoint{3.340545in}{3.136837in}}{\pgfqpoint{3.351144in}{3.141228in}}{\pgfqpoint{3.358958in}{3.149041in}}%
\pgfpathcurveto{\pgfqpoint{3.366771in}{3.156855in}}{\pgfqpoint{3.371162in}{3.167454in}}{\pgfqpoint{3.371162in}{3.178504in}}%
\pgfpathcurveto{\pgfqpoint{3.371162in}{3.189554in}}{\pgfqpoint{3.366771in}{3.200153in}}{\pgfqpoint{3.358958in}{3.207967in}}%
\pgfpathcurveto{\pgfqpoint{3.351144in}{3.215780in}}{\pgfqpoint{3.340545in}{3.220171in}}{\pgfqpoint{3.329495in}{3.220171in}}%
\pgfpathcurveto{\pgfqpoint{3.318445in}{3.220171in}}{\pgfqpoint{3.307846in}{3.215780in}}{\pgfqpoint{3.300032in}{3.207967in}}%
\pgfpathcurveto{\pgfqpoint{3.292219in}{3.200153in}}{\pgfqpoint{3.287828in}{3.189554in}}{\pgfqpoint{3.287828in}{3.178504in}}%
\pgfpathcurveto{\pgfqpoint{3.287828in}{3.167454in}}{\pgfqpoint{3.292219in}{3.156855in}}{\pgfqpoint{3.300032in}{3.149041in}}%
\pgfpathcurveto{\pgfqpoint{3.307846in}{3.141228in}}{\pgfqpoint{3.318445in}{3.136837in}}{\pgfqpoint{3.329495in}{3.136837in}}%
\pgfpathclose%
\pgfusepath{stroke,fill}%
\end{pgfscope}%
\begin{pgfscope}%
\pgfpathrectangle{\pgfqpoint{0.648703in}{0.548769in}}{\pgfqpoint{5.201297in}{3.102590in}}%
\pgfusepath{clip}%
\pgfsetbuttcap%
\pgfsetroundjoin%
\definecolor{currentfill}{rgb}{1.000000,0.498039,0.054902}%
\pgfsetfillcolor{currentfill}%
\pgfsetlinewidth{1.003750pt}%
\definecolor{currentstroke}{rgb}{1.000000,0.498039,0.054902}%
\pgfsetstrokecolor{currentstroke}%
\pgfsetdash{}{0pt}%
\pgfpathmoveto{\pgfqpoint{4.411429in}{3.136837in}}%
\pgfpathcurveto{\pgfqpoint{4.422479in}{3.136837in}}{\pgfqpoint{4.433078in}{3.141228in}}{\pgfqpoint{4.440892in}{3.149041in}}%
\pgfpathcurveto{\pgfqpoint{4.448705in}{3.156855in}}{\pgfqpoint{4.453095in}{3.167454in}}{\pgfqpoint{4.453095in}{3.178504in}}%
\pgfpathcurveto{\pgfqpoint{4.453095in}{3.189554in}}{\pgfqpoint{4.448705in}{3.200153in}}{\pgfqpoint{4.440892in}{3.207967in}}%
\pgfpathcurveto{\pgfqpoint{4.433078in}{3.215780in}}{\pgfqpoint{4.422479in}{3.220171in}}{\pgfqpoint{4.411429in}{3.220171in}}%
\pgfpathcurveto{\pgfqpoint{4.400379in}{3.220171in}}{\pgfqpoint{4.389780in}{3.215780in}}{\pgfqpoint{4.381966in}{3.207967in}}%
\pgfpathcurveto{\pgfqpoint{4.374152in}{3.200153in}}{\pgfqpoint{4.369762in}{3.189554in}}{\pgfqpoint{4.369762in}{3.178504in}}%
\pgfpathcurveto{\pgfqpoint{4.369762in}{3.167454in}}{\pgfqpoint{4.374152in}{3.156855in}}{\pgfqpoint{4.381966in}{3.149041in}}%
\pgfpathcurveto{\pgfqpoint{4.389780in}{3.141228in}}{\pgfqpoint{4.400379in}{3.136837in}}{\pgfqpoint{4.411429in}{3.136837in}}%
\pgfpathclose%
\pgfusepath{stroke,fill}%
\end{pgfscope}%
\begin{pgfscope}%
\pgfpathrectangle{\pgfqpoint{0.648703in}{0.548769in}}{\pgfqpoint{5.201297in}{3.102590in}}%
\pgfusepath{clip}%
\pgfsetbuttcap%
\pgfsetroundjoin%
\definecolor{currentfill}{rgb}{1.000000,0.498039,0.054902}%
\pgfsetfillcolor{currentfill}%
\pgfsetlinewidth{1.003750pt}%
\definecolor{currentstroke}{rgb}{1.000000,0.498039,0.054902}%
\pgfsetstrokecolor{currentstroke}%
\pgfsetdash{}{0pt}%
\pgfpathmoveto{\pgfqpoint{2.968850in}{3.136837in}}%
\pgfpathcurveto{\pgfqpoint{2.979900in}{3.136837in}}{\pgfqpoint{2.990500in}{3.141228in}}{\pgfqpoint{2.998313in}{3.149041in}}%
\pgfpathcurveto{\pgfqpoint{3.006127in}{3.156855in}}{\pgfqpoint{3.010517in}{3.167454in}}{\pgfqpoint{3.010517in}{3.178504in}}%
\pgfpathcurveto{\pgfqpoint{3.010517in}{3.189554in}}{\pgfqpoint{3.006127in}{3.200153in}}{\pgfqpoint{2.998313in}{3.207967in}}%
\pgfpathcurveto{\pgfqpoint{2.990500in}{3.215780in}}{\pgfqpoint{2.979900in}{3.220171in}}{\pgfqpoint{2.968850in}{3.220171in}}%
\pgfpathcurveto{\pgfqpoint{2.957800in}{3.220171in}}{\pgfqpoint{2.947201in}{3.215780in}}{\pgfqpoint{2.939388in}{3.207967in}}%
\pgfpathcurveto{\pgfqpoint{2.931574in}{3.200153in}}{\pgfqpoint{2.927184in}{3.189554in}}{\pgfqpoint{2.927184in}{3.178504in}}%
\pgfpathcurveto{\pgfqpoint{2.927184in}{3.167454in}}{\pgfqpoint{2.931574in}{3.156855in}}{\pgfqpoint{2.939388in}{3.149041in}}%
\pgfpathcurveto{\pgfqpoint{2.947201in}{3.141228in}}{\pgfqpoint{2.957800in}{3.136837in}}{\pgfqpoint{2.968850in}{3.136837in}}%
\pgfpathclose%
\pgfusepath{stroke,fill}%
\end{pgfscope}%
\begin{pgfscope}%
\pgfpathrectangle{\pgfqpoint{0.648703in}{0.548769in}}{\pgfqpoint{5.201297in}{3.102590in}}%
\pgfusepath{clip}%
\pgfsetbuttcap%
\pgfsetroundjoin%
\definecolor{currentfill}{rgb}{0.839216,0.152941,0.156863}%
\pgfsetfillcolor{currentfill}%
\pgfsetlinewidth{1.003750pt}%
\definecolor{currentstroke}{rgb}{0.839216,0.152941,0.156863}%
\pgfsetstrokecolor{currentstroke}%
\pgfsetdash{}{0pt}%
\pgfpathmoveto{\pgfqpoint{3.770283in}{3.136837in}}%
\pgfpathcurveto{\pgfqpoint{3.781333in}{3.136837in}}{\pgfqpoint{3.791932in}{3.141228in}}{\pgfqpoint{3.799746in}{3.149041in}}%
\pgfpathcurveto{\pgfqpoint{3.807559in}{3.156855in}}{\pgfqpoint{3.811949in}{3.167454in}}{\pgfqpoint{3.811949in}{3.178504in}}%
\pgfpathcurveto{\pgfqpoint{3.811949in}{3.189554in}}{\pgfqpoint{3.807559in}{3.200153in}}{\pgfqpoint{3.799746in}{3.207967in}}%
\pgfpathcurveto{\pgfqpoint{3.791932in}{3.215780in}}{\pgfqpoint{3.781333in}{3.220171in}}{\pgfqpoint{3.770283in}{3.220171in}}%
\pgfpathcurveto{\pgfqpoint{3.759233in}{3.220171in}}{\pgfqpoint{3.748634in}{3.215780in}}{\pgfqpoint{3.740820in}{3.207967in}}%
\pgfpathcurveto{\pgfqpoint{3.733006in}{3.200153in}}{\pgfqpoint{3.728616in}{3.189554in}}{\pgfqpoint{3.728616in}{3.178504in}}%
\pgfpathcurveto{\pgfqpoint{3.728616in}{3.167454in}}{\pgfqpoint{3.733006in}{3.156855in}}{\pgfqpoint{3.740820in}{3.149041in}}%
\pgfpathcurveto{\pgfqpoint{3.748634in}{3.141228in}}{\pgfqpoint{3.759233in}{3.136837in}}{\pgfqpoint{3.770283in}{3.136837in}}%
\pgfpathclose%
\pgfusepath{stroke,fill}%
\end{pgfscope}%
\begin{pgfscope}%
\pgfpathrectangle{\pgfqpoint{0.648703in}{0.548769in}}{\pgfqpoint{5.201297in}{3.102590in}}%
\pgfusepath{clip}%
\pgfsetbuttcap%
\pgfsetroundjoin%
\definecolor{currentfill}{rgb}{1.000000,0.498039,0.054902}%
\pgfsetfillcolor{currentfill}%
\pgfsetlinewidth{1.003750pt}%
\definecolor{currentstroke}{rgb}{1.000000,0.498039,0.054902}%
\pgfsetstrokecolor{currentstroke}%
\pgfsetdash{}{0pt}%
\pgfpathmoveto{\pgfqpoint{3.169208in}{3.149281in}}%
\pgfpathcurveto{\pgfqpoint{3.180259in}{3.149281in}}{\pgfqpoint{3.190858in}{3.153671in}}{\pgfqpoint{3.198671in}{3.161485in}}%
\pgfpathcurveto{\pgfqpoint{3.206485in}{3.169298in}}{\pgfqpoint{3.210875in}{3.179897in}}{\pgfqpoint{3.210875in}{3.190948in}}%
\pgfpathcurveto{\pgfqpoint{3.210875in}{3.201998in}}{\pgfqpoint{3.206485in}{3.212597in}}{\pgfqpoint{3.198671in}{3.220410in}}%
\pgfpathcurveto{\pgfqpoint{3.190858in}{3.228224in}}{\pgfqpoint{3.180259in}{3.232614in}}{\pgfqpoint{3.169208in}{3.232614in}}%
\pgfpathcurveto{\pgfqpoint{3.158158in}{3.232614in}}{\pgfqpoint{3.147559in}{3.228224in}}{\pgfqpoint{3.139746in}{3.220410in}}%
\pgfpathcurveto{\pgfqpoint{3.131932in}{3.212597in}}{\pgfqpoint{3.127542in}{3.201998in}}{\pgfqpoint{3.127542in}{3.190948in}}%
\pgfpathcurveto{\pgfqpoint{3.127542in}{3.179897in}}{\pgfqpoint{3.131932in}{3.169298in}}{\pgfqpoint{3.139746in}{3.161485in}}%
\pgfpathcurveto{\pgfqpoint{3.147559in}{3.153671in}}{\pgfqpoint{3.158158in}{3.149281in}}{\pgfqpoint{3.169208in}{3.149281in}}%
\pgfpathclose%
\pgfusepath{stroke,fill}%
\end{pgfscope}%
\begin{pgfscope}%
\pgfpathrectangle{\pgfqpoint{0.648703in}{0.548769in}}{\pgfqpoint{5.201297in}{3.102590in}}%
\pgfusepath{clip}%
\pgfsetbuttcap%
\pgfsetroundjoin%
\definecolor{currentfill}{rgb}{1.000000,0.498039,0.054902}%
\pgfsetfillcolor{currentfill}%
\pgfsetlinewidth{1.003750pt}%
\definecolor{currentstroke}{rgb}{1.000000,0.498039,0.054902}%
\pgfsetstrokecolor{currentstroke}%
\pgfsetdash{}{0pt}%
\pgfpathmoveto{\pgfqpoint{3.489781in}{3.286160in}}%
\pgfpathcurveto{\pgfqpoint{3.500832in}{3.286160in}}{\pgfqpoint{3.511431in}{3.290550in}}{\pgfqpoint{3.519244in}{3.298364in}}%
\pgfpathcurveto{\pgfqpoint{3.527058in}{3.306177in}}{\pgfqpoint{3.531448in}{3.316776in}}{\pgfqpoint{3.531448in}{3.327827in}}%
\pgfpathcurveto{\pgfqpoint{3.531448in}{3.338877in}}{\pgfqpoint{3.527058in}{3.349476in}}{\pgfqpoint{3.519244in}{3.357289in}}%
\pgfpathcurveto{\pgfqpoint{3.511431in}{3.365103in}}{\pgfqpoint{3.500832in}{3.369493in}}{\pgfqpoint{3.489781in}{3.369493in}}%
\pgfpathcurveto{\pgfqpoint{3.478731in}{3.369493in}}{\pgfqpoint{3.468132in}{3.365103in}}{\pgfqpoint{3.460319in}{3.357289in}}%
\pgfpathcurveto{\pgfqpoint{3.452505in}{3.349476in}}{\pgfqpoint{3.448115in}{3.338877in}}{\pgfqpoint{3.448115in}{3.327827in}}%
\pgfpathcurveto{\pgfqpoint{3.448115in}{3.316776in}}{\pgfqpoint{3.452505in}{3.306177in}}{\pgfqpoint{3.460319in}{3.298364in}}%
\pgfpathcurveto{\pgfqpoint{3.468132in}{3.290550in}}{\pgfqpoint{3.478731in}{3.286160in}}{\pgfqpoint{3.489781in}{3.286160in}}%
\pgfpathclose%
\pgfusepath{stroke,fill}%
\end{pgfscope}%
\begin{pgfscope}%
\pgfpathrectangle{\pgfqpoint{0.648703in}{0.548769in}}{\pgfqpoint{5.201297in}{3.102590in}}%
\pgfusepath{clip}%
\pgfsetbuttcap%
\pgfsetroundjoin%
\definecolor{currentfill}{rgb}{1.000000,0.498039,0.054902}%
\pgfsetfillcolor{currentfill}%
\pgfsetlinewidth{1.003750pt}%
\definecolor{currentstroke}{rgb}{1.000000,0.498039,0.054902}%
\pgfsetstrokecolor{currentstroke}%
\pgfsetdash{}{0pt}%
\pgfpathmoveto{\pgfqpoint{3.489781in}{3.136837in}}%
\pgfpathcurveto{\pgfqpoint{3.500832in}{3.136837in}}{\pgfqpoint{3.511431in}{3.141228in}}{\pgfqpoint{3.519244in}{3.149041in}}%
\pgfpathcurveto{\pgfqpoint{3.527058in}{3.156855in}}{\pgfqpoint{3.531448in}{3.167454in}}{\pgfqpoint{3.531448in}{3.178504in}}%
\pgfpathcurveto{\pgfqpoint{3.531448in}{3.189554in}}{\pgfqpoint{3.527058in}{3.200153in}}{\pgfqpoint{3.519244in}{3.207967in}}%
\pgfpathcurveto{\pgfqpoint{3.511431in}{3.215780in}}{\pgfqpoint{3.500832in}{3.220171in}}{\pgfqpoint{3.489781in}{3.220171in}}%
\pgfpathcurveto{\pgfqpoint{3.478731in}{3.220171in}}{\pgfqpoint{3.468132in}{3.215780in}}{\pgfqpoint{3.460319in}{3.207967in}}%
\pgfpathcurveto{\pgfqpoint{3.452505in}{3.200153in}}{\pgfqpoint{3.448115in}{3.189554in}}{\pgfqpoint{3.448115in}{3.178504in}}%
\pgfpathcurveto{\pgfqpoint{3.448115in}{3.167454in}}{\pgfqpoint{3.452505in}{3.156855in}}{\pgfqpoint{3.460319in}{3.149041in}}%
\pgfpathcurveto{\pgfqpoint{3.468132in}{3.141228in}}{\pgfqpoint{3.478731in}{3.136837in}}{\pgfqpoint{3.489781in}{3.136837in}}%
\pgfpathclose%
\pgfusepath{stroke,fill}%
\end{pgfscope}%
\begin{pgfscope}%
\pgfpathrectangle{\pgfqpoint{0.648703in}{0.548769in}}{\pgfqpoint{5.201297in}{3.102590in}}%
\pgfusepath{clip}%
\pgfsetbuttcap%
\pgfsetroundjoin%
\definecolor{currentfill}{rgb}{1.000000,0.498039,0.054902}%
\pgfsetfillcolor{currentfill}%
\pgfsetlinewidth{1.003750pt}%
\definecolor{currentstroke}{rgb}{1.000000,0.498039,0.054902}%
\pgfsetstrokecolor{currentstroke}%
\pgfsetdash{}{0pt}%
\pgfpathmoveto{\pgfqpoint{2.928779in}{3.149281in}}%
\pgfpathcurveto{\pgfqpoint{2.939829in}{3.149281in}}{\pgfqpoint{2.950428in}{3.153671in}}{\pgfqpoint{2.958242in}{3.161485in}}%
\pgfpathcurveto{\pgfqpoint{2.966055in}{3.169298in}}{\pgfqpoint{2.970445in}{3.179897in}}{\pgfqpoint{2.970445in}{3.190948in}}%
\pgfpathcurveto{\pgfqpoint{2.970445in}{3.201998in}}{\pgfqpoint{2.966055in}{3.212597in}}{\pgfqpoint{2.958242in}{3.220410in}}%
\pgfpathcurveto{\pgfqpoint{2.950428in}{3.228224in}}{\pgfqpoint{2.939829in}{3.232614in}}{\pgfqpoint{2.928779in}{3.232614in}}%
\pgfpathcurveto{\pgfqpoint{2.917729in}{3.232614in}}{\pgfqpoint{2.907130in}{3.228224in}}{\pgfqpoint{2.899316in}{3.220410in}}%
\pgfpathcurveto{\pgfqpoint{2.891502in}{3.212597in}}{\pgfqpoint{2.887112in}{3.201998in}}{\pgfqpoint{2.887112in}{3.190948in}}%
\pgfpathcurveto{\pgfqpoint{2.887112in}{3.179897in}}{\pgfqpoint{2.891502in}{3.169298in}}{\pgfqpoint{2.899316in}{3.161485in}}%
\pgfpathcurveto{\pgfqpoint{2.907130in}{3.153671in}}{\pgfqpoint{2.917729in}{3.149281in}}{\pgfqpoint{2.928779in}{3.149281in}}%
\pgfpathclose%
\pgfusepath{stroke,fill}%
\end{pgfscope}%
\begin{pgfscope}%
\pgfpathrectangle{\pgfqpoint{0.648703in}{0.548769in}}{\pgfqpoint{5.201297in}{3.102590in}}%
\pgfusepath{clip}%
\pgfsetbuttcap%
\pgfsetroundjoin%
\definecolor{currentfill}{rgb}{0.121569,0.466667,0.705882}%
\pgfsetfillcolor{currentfill}%
\pgfsetlinewidth{1.003750pt}%
\definecolor{currentstroke}{rgb}{0.121569,0.466667,0.705882}%
\pgfsetstrokecolor{currentstroke}%
\pgfsetdash{}{0pt}%
\pgfpathmoveto{\pgfqpoint{2.928779in}{3.128542in}}%
\pgfpathcurveto{\pgfqpoint{2.939829in}{3.128542in}}{\pgfqpoint{2.950428in}{3.132932in}}{\pgfqpoint{2.958242in}{3.140746in}}%
\pgfpathcurveto{\pgfqpoint{2.966055in}{3.148559in}}{\pgfqpoint{2.970445in}{3.159158in}}{\pgfqpoint{2.970445in}{3.170208in}}%
\pgfpathcurveto{\pgfqpoint{2.970445in}{3.181258in}}{\pgfqpoint{2.966055in}{3.191857in}}{\pgfqpoint{2.958242in}{3.199671in}}%
\pgfpathcurveto{\pgfqpoint{2.950428in}{3.207485in}}{\pgfqpoint{2.939829in}{3.211875in}}{\pgfqpoint{2.928779in}{3.211875in}}%
\pgfpathcurveto{\pgfqpoint{2.917729in}{3.211875in}}{\pgfqpoint{2.907130in}{3.207485in}}{\pgfqpoint{2.899316in}{3.199671in}}%
\pgfpathcurveto{\pgfqpoint{2.891502in}{3.191857in}}{\pgfqpoint{2.887112in}{3.181258in}}{\pgfqpoint{2.887112in}{3.170208in}}%
\pgfpathcurveto{\pgfqpoint{2.887112in}{3.159158in}}{\pgfqpoint{2.891502in}{3.148559in}}{\pgfqpoint{2.899316in}{3.140746in}}%
\pgfpathcurveto{\pgfqpoint{2.907130in}{3.132932in}}{\pgfqpoint{2.917729in}{3.128542in}}{\pgfqpoint{2.928779in}{3.128542in}}%
\pgfpathclose%
\pgfusepath{stroke,fill}%
\end{pgfscope}%
\begin{pgfscope}%
\pgfpathrectangle{\pgfqpoint{0.648703in}{0.548769in}}{\pgfqpoint{5.201297in}{3.102590in}}%
\pgfusepath{clip}%
\pgfsetbuttcap%
\pgfsetroundjoin%
\definecolor{currentfill}{rgb}{1.000000,0.498039,0.054902}%
\pgfsetfillcolor{currentfill}%
\pgfsetlinewidth{1.003750pt}%
\definecolor{currentstroke}{rgb}{1.000000,0.498039,0.054902}%
\pgfsetstrokecolor{currentstroke}%
\pgfsetdash{}{0pt}%
\pgfpathmoveto{\pgfqpoint{3.810354in}{3.232238in}}%
\pgfpathcurveto{\pgfqpoint{3.821405in}{3.232238in}}{\pgfqpoint{3.832004in}{3.236628in}}{\pgfqpoint{3.839817in}{3.244442in}}%
\pgfpathcurveto{\pgfqpoint{3.847631in}{3.252255in}}{\pgfqpoint{3.852021in}{3.262854in}}{\pgfqpoint{3.852021in}{3.273905in}}%
\pgfpathcurveto{\pgfqpoint{3.852021in}{3.284955in}}{\pgfqpoint{3.847631in}{3.295554in}}{\pgfqpoint{3.839817in}{3.303367in}}%
\pgfpathcurveto{\pgfqpoint{3.832004in}{3.311181in}}{\pgfqpoint{3.821405in}{3.315571in}}{\pgfqpoint{3.810354in}{3.315571in}}%
\pgfpathcurveto{\pgfqpoint{3.799304in}{3.315571in}}{\pgfqpoint{3.788705in}{3.311181in}}{\pgfqpoint{3.780892in}{3.303367in}}%
\pgfpathcurveto{\pgfqpoint{3.773078in}{3.295554in}}{\pgfqpoint{3.768688in}{3.284955in}}{\pgfqpoint{3.768688in}{3.273905in}}%
\pgfpathcurveto{\pgfqpoint{3.768688in}{3.262854in}}{\pgfqpoint{3.773078in}{3.252255in}}{\pgfqpoint{3.780892in}{3.244442in}}%
\pgfpathcurveto{\pgfqpoint{3.788705in}{3.236628in}}{\pgfqpoint{3.799304in}{3.232238in}}{\pgfqpoint{3.810354in}{3.232238in}}%
\pgfpathclose%
\pgfusepath{stroke,fill}%
\end{pgfscope}%
\begin{pgfscope}%
\pgfpathrectangle{\pgfqpoint{0.648703in}{0.548769in}}{\pgfqpoint{5.201297in}{3.102590in}}%
\pgfusepath{clip}%
\pgfsetbuttcap%
\pgfsetroundjoin%
\definecolor{currentfill}{rgb}{0.121569,0.466667,0.705882}%
\pgfsetfillcolor{currentfill}%
\pgfsetlinewidth{1.003750pt}%
\definecolor{currentstroke}{rgb}{0.121569,0.466667,0.705882}%
\pgfsetstrokecolor{currentstroke}%
\pgfsetdash{}{0pt}%
\pgfpathmoveto{\pgfqpoint{3.930569in}{0.648129in}}%
\pgfpathcurveto{\pgfqpoint{3.941619in}{0.648129in}}{\pgfqpoint{3.952218in}{0.652519in}}{\pgfqpoint{3.960032in}{0.660333in}}%
\pgfpathcurveto{\pgfqpoint{3.967846in}{0.668146in}}{\pgfqpoint{3.972236in}{0.678745in}}{\pgfqpoint{3.972236in}{0.689796in}}%
\pgfpathcurveto{\pgfqpoint{3.972236in}{0.700846in}}{\pgfqpoint{3.967846in}{0.711445in}}{\pgfqpoint{3.960032in}{0.719258in}}%
\pgfpathcurveto{\pgfqpoint{3.952218in}{0.727072in}}{\pgfqpoint{3.941619in}{0.731462in}}{\pgfqpoint{3.930569in}{0.731462in}}%
\pgfpathcurveto{\pgfqpoint{3.919519in}{0.731462in}}{\pgfqpoint{3.908920in}{0.727072in}}{\pgfqpoint{3.901107in}{0.719258in}}%
\pgfpathcurveto{\pgfqpoint{3.893293in}{0.711445in}}{\pgfqpoint{3.888903in}{0.700846in}}{\pgfqpoint{3.888903in}{0.689796in}}%
\pgfpathcurveto{\pgfqpoint{3.888903in}{0.678745in}}{\pgfqpoint{3.893293in}{0.668146in}}{\pgfqpoint{3.901107in}{0.660333in}}%
\pgfpathcurveto{\pgfqpoint{3.908920in}{0.652519in}}{\pgfqpoint{3.919519in}{0.648129in}}{\pgfqpoint{3.930569in}{0.648129in}}%
\pgfpathclose%
\pgfusepath{stroke,fill}%
\end{pgfscope}%
\begin{pgfscope}%
\pgfpathrectangle{\pgfqpoint{0.648703in}{0.548769in}}{\pgfqpoint{5.201297in}{3.102590in}}%
\pgfusepath{clip}%
\pgfsetbuttcap%
\pgfsetroundjoin%
\definecolor{currentfill}{rgb}{1.000000,0.498039,0.054902}%
\pgfsetfillcolor{currentfill}%
\pgfsetlinewidth{1.003750pt}%
\definecolor{currentstroke}{rgb}{1.000000,0.498039,0.054902}%
\pgfsetstrokecolor{currentstroke}%
\pgfsetdash{}{0pt}%
\pgfpathmoveto{\pgfqpoint{2.127346in}{3.136837in}}%
\pgfpathcurveto{\pgfqpoint{2.138396in}{3.136837in}}{\pgfqpoint{2.148995in}{3.141228in}}{\pgfqpoint{2.156809in}{3.149041in}}%
\pgfpathcurveto{\pgfqpoint{2.164623in}{3.156855in}}{\pgfqpoint{2.169013in}{3.167454in}}{\pgfqpoint{2.169013in}{3.178504in}}%
\pgfpathcurveto{\pgfqpoint{2.169013in}{3.189554in}}{\pgfqpoint{2.164623in}{3.200153in}}{\pgfqpoint{2.156809in}{3.207967in}}%
\pgfpathcurveto{\pgfqpoint{2.148995in}{3.215780in}}{\pgfqpoint{2.138396in}{3.220171in}}{\pgfqpoint{2.127346in}{3.220171in}}%
\pgfpathcurveto{\pgfqpoint{2.116296in}{3.220171in}}{\pgfqpoint{2.105697in}{3.215780in}}{\pgfqpoint{2.097883in}{3.207967in}}%
\pgfpathcurveto{\pgfqpoint{2.090070in}{3.200153in}}{\pgfqpoint{2.085680in}{3.189554in}}{\pgfqpoint{2.085680in}{3.178504in}}%
\pgfpathcurveto{\pgfqpoint{2.085680in}{3.167454in}}{\pgfqpoint{2.090070in}{3.156855in}}{\pgfqpoint{2.097883in}{3.149041in}}%
\pgfpathcurveto{\pgfqpoint{2.105697in}{3.141228in}}{\pgfqpoint{2.116296in}{3.136837in}}{\pgfqpoint{2.127346in}{3.136837in}}%
\pgfpathclose%
\pgfusepath{stroke,fill}%
\end{pgfscope}%
\begin{pgfscope}%
\pgfpathrectangle{\pgfqpoint{0.648703in}{0.548769in}}{\pgfqpoint{5.201297in}{3.102590in}}%
\pgfusepath{clip}%
\pgfsetbuttcap%
\pgfsetroundjoin%
\definecolor{currentfill}{rgb}{1.000000,0.498039,0.054902}%
\pgfsetfillcolor{currentfill}%
\pgfsetlinewidth{1.003750pt}%
\definecolor{currentstroke}{rgb}{1.000000,0.498039,0.054902}%
\pgfsetstrokecolor{currentstroke}%
\pgfsetdash{}{0pt}%
\pgfpathmoveto{\pgfqpoint{3.169208in}{3.140985in}}%
\pgfpathcurveto{\pgfqpoint{3.180259in}{3.140985in}}{\pgfqpoint{3.190858in}{3.145375in}}{\pgfqpoint{3.198671in}{3.153189in}}%
\pgfpathcurveto{\pgfqpoint{3.206485in}{3.161003in}}{\pgfqpoint{3.210875in}{3.171602in}}{\pgfqpoint{3.210875in}{3.182652in}}%
\pgfpathcurveto{\pgfqpoint{3.210875in}{3.193702in}}{\pgfqpoint{3.206485in}{3.204301in}}{\pgfqpoint{3.198671in}{3.212115in}}%
\pgfpathcurveto{\pgfqpoint{3.190858in}{3.219928in}}{\pgfqpoint{3.180259in}{3.224319in}}{\pgfqpoint{3.169208in}{3.224319in}}%
\pgfpathcurveto{\pgfqpoint{3.158158in}{3.224319in}}{\pgfqpoint{3.147559in}{3.219928in}}{\pgfqpoint{3.139746in}{3.212115in}}%
\pgfpathcurveto{\pgfqpoint{3.131932in}{3.204301in}}{\pgfqpoint{3.127542in}{3.193702in}}{\pgfqpoint{3.127542in}{3.182652in}}%
\pgfpathcurveto{\pgfqpoint{3.127542in}{3.171602in}}{\pgfqpoint{3.131932in}{3.161003in}}{\pgfqpoint{3.139746in}{3.153189in}}%
\pgfpathcurveto{\pgfqpoint{3.147559in}{3.145375in}}{\pgfqpoint{3.158158in}{3.140985in}}{\pgfqpoint{3.169208in}{3.140985in}}%
\pgfpathclose%
\pgfusepath{stroke,fill}%
\end{pgfscope}%
\begin{pgfscope}%
\pgfpathrectangle{\pgfqpoint{0.648703in}{0.548769in}}{\pgfqpoint{5.201297in}{3.102590in}}%
\pgfusepath{clip}%
\pgfsetbuttcap%
\pgfsetroundjoin%
\definecolor{currentfill}{rgb}{1.000000,0.498039,0.054902}%
\pgfsetfillcolor{currentfill}%
\pgfsetlinewidth{1.003750pt}%
\definecolor{currentstroke}{rgb}{1.000000,0.498039,0.054902}%
\pgfsetstrokecolor{currentstroke}%
\pgfsetdash{}{0pt}%
\pgfpathmoveto{\pgfqpoint{3.369567in}{3.149281in}}%
\pgfpathcurveto{\pgfqpoint{3.380617in}{3.149281in}}{\pgfqpoint{3.391216in}{3.153671in}}{\pgfqpoint{3.399029in}{3.161485in}}%
\pgfpathcurveto{\pgfqpoint{3.406843in}{3.169298in}}{\pgfqpoint{3.411233in}{3.179897in}}{\pgfqpoint{3.411233in}{3.190948in}}%
\pgfpathcurveto{\pgfqpoint{3.411233in}{3.201998in}}{\pgfqpoint{3.406843in}{3.212597in}}{\pgfqpoint{3.399029in}{3.220410in}}%
\pgfpathcurveto{\pgfqpoint{3.391216in}{3.228224in}}{\pgfqpoint{3.380617in}{3.232614in}}{\pgfqpoint{3.369567in}{3.232614in}}%
\pgfpathcurveto{\pgfqpoint{3.358516in}{3.232614in}}{\pgfqpoint{3.347917in}{3.228224in}}{\pgfqpoint{3.340104in}{3.220410in}}%
\pgfpathcurveto{\pgfqpoint{3.332290in}{3.212597in}}{\pgfqpoint{3.327900in}{3.201998in}}{\pgfqpoint{3.327900in}{3.190948in}}%
\pgfpathcurveto{\pgfqpoint{3.327900in}{3.179897in}}{\pgfqpoint{3.332290in}{3.169298in}}{\pgfqpoint{3.340104in}{3.161485in}}%
\pgfpathcurveto{\pgfqpoint{3.347917in}{3.153671in}}{\pgfqpoint{3.358516in}{3.149281in}}{\pgfqpoint{3.369567in}{3.149281in}}%
\pgfpathclose%
\pgfusepath{stroke,fill}%
\end{pgfscope}%
\begin{pgfscope}%
\pgfpathrectangle{\pgfqpoint{0.648703in}{0.548769in}}{\pgfqpoint{5.201297in}{3.102590in}}%
\pgfusepath{clip}%
\pgfsetbuttcap%
\pgfsetroundjoin%
\definecolor{currentfill}{rgb}{1.000000,0.498039,0.054902}%
\pgfsetfillcolor{currentfill}%
\pgfsetlinewidth{1.003750pt}%
\definecolor{currentstroke}{rgb}{1.000000,0.498039,0.054902}%
\pgfsetstrokecolor{currentstroke}%
\pgfsetdash{}{0pt}%
\pgfpathmoveto{\pgfqpoint{3.409638in}{3.140985in}}%
\pgfpathcurveto{\pgfqpoint{3.420688in}{3.140985in}}{\pgfqpoint{3.431287in}{3.145375in}}{\pgfqpoint{3.439101in}{3.153189in}}%
\pgfpathcurveto{\pgfqpoint{3.446915in}{3.161003in}}{\pgfqpoint{3.451305in}{3.171602in}}{\pgfqpoint{3.451305in}{3.182652in}}%
\pgfpathcurveto{\pgfqpoint{3.451305in}{3.193702in}}{\pgfqpoint{3.446915in}{3.204301in}}{\pgfqpoint{3.439101in}{3.212115in}}%
\pgfpathcurveto{\pgfqpoint{3.431287in}{3.219928in}}{\pgfqpoint{3.420688in}{3.224319in}}{\pgfqpoint{3.409638in}{3.224319in}}%
\pgfpathcurveto{\pgfqpoint{3.398588in}{3.224319in}}{\pgfqpoint{3.387989in}{3.219928in}}{\pgfqpoint{3.380175in}{3.212115in}}%
\pgfpathcurveto{\pgfqpoint{3.372362in}{3.204301in}}{\pgfqpoint{3.367972in}{3.193702in}}{\pgfqpoint{3.367972in}{3.182652in}}%
\pgfpathcurveto{\pgfqpoint{3.367972in}{3.171602in}}{\pgfqpoint{3.372362in}{3.161003in}}{\pgfqpoint{3.380175in}{3.153189in}}%
\pgfpathcurveto{\pgfqpoint{3.387989in}{3.145375in}}{\pgfqpoint{3.398588in}{3.140985in}}{\pgfqpoint{3.409638in}{3.140985in}}%
\pgfpathclose%
\pgfusepath{stroke,fill}%
\end{pgfscope}%
\begin{pgfscope}%
\pgfpathrectangle{\pgfqpoint{0.648703in}{0.548769in}}{\pgfqpoint{5.201297in}{3.102590in}}%
\pgfusepath{clip}%
\pgfsetbuttcap%
\pgfsetroundjoin%
\definecolor{currentfill}{rgb}{1.000000,0.498039,0.054902}%
\pgfsetfillcolor{currentfill}%
\pgfsetlinewidth{1.003750pt}%
\definecolor{currentstroke}{rgb}{1.000000,0.498039,0.054902}%
\pgfsetstrokecolor{currentstroke}%
\pgfsetdash{}{0pt}%
\pgfpathmoveto{\pgfqpoint{3.770283in}{3.136837in}}%
\pgfpathcurveto{\pgfqpoint{3.781333in}{3.136837in}}{\pgfqpoint{3.791932in}{3.141228in}}{\pgfqpoint{3.799746in}{3.149041in}}%
\pgfpathcurveto{\pgfqpoint{3.807559in}{3.156855in}}{\pgfqpoint{3.811949in}{3.167454in}}{\pgfqpoint{3.811949in}{3.178504in}}%
\pgfpathcurveto{\pgfqpoint{3.811949in}{3.189554in}}{\pgfqpoint{3.807559in}{3.200153in}}{\pgfqpoint{3.799746in}{3.207967in}}%
\pgfpathcurveto{\pgfqpoint{3.791932in}{3.215780in}}{\pgfqpoint{3.781333in}{3.220171in}}{\pgfqpoint{3.770283in}{3.220171in}}%
\pgfpathcurveto{\pgfqpoint{3.759233in}{3.220171in}}{\pgfqpoint{3.748634in}{3.215780in}}{\pgfqpoint{3.740820in}{3.207967in}}%
\pgfpathcurveto{\pgfqpoint{3.733006in}{3.200153in}}{\pgfqpoint{3.728616in}{3.189554in}}{\pgfqpoint{3.728616in}{3.178504in}}%
\pgfpathcurveto{\pgfqpoint{3.728616in}{3.167454in}}{\pgfqpoint{3.733006in}{3.156855in}}{\pgfqpoint{3.740820in}{3.149041in}}%
\pgfpathcurveto{\pgfqpoint{3.748634in}{3.141228in}}{\pgfqpoint{3.759233in}{3.136837in}}{\pgfqpoint{3.770283in}{3.136837in}}%
\pgfpathclose%
\pgfusepath{stroke,fill}%
\end{pgfscope}%
\begin{pgfscope}%
\pgfpathrectangle{\pgfqpoint{0.648703in}{0.548769in}}{\pgfqpoint{5.201297in}{3.102590in}}%
\pgfusepath{clip}%
\pgfsetbuttcap%
\pgfsetroundjoin%
\definecolor{currentfill}{rgb}{1.000000,0.498039,0.054902}%
\pgfsetfillcolor{currentfill}%
\pgfsetlinewidth{1.003750pt}%
\definecolor{currentstroke}{rgb}{1.000000,0.498039,0.054902}%
\pgfsetstrokecolor{currentstroke}%
\pgfsetdash{}{0pt}%
\pgfpathmoveto{\pgfqpoint{3.569925in}{3.145133in}}%
\pgfpathcurveto{\pgfqpoint{3.580975in}{3.145133in}}{\pgfqpoint{3.591574in}{3.149523in}}{\pgfqpoint{3.599387in}{3.157337in}}%
\pgfpathcurveto{\pgfqpoint{3.607201in}{3.165151in}}{\pgfqpoint{3.611591in}{3.175750in}}{\pgfqpoint{3.611591in}{3.186800in}}%
\pgfpathcurveto{\pgfqpoint{3.611591in}{3.197850in}}{\pgfqpoint{3.607201in}{3.208449in}}{\pgfqpoint{3.599387in}{3.216262in}}%
\pgfpathcurveto{\pgfqpoint{3.591574in}{3.224076in}}{\pgfqpoint{3.580975in}{3.228466in}}{\pgfqpoint{3.569925in}{3.228466in}}%
\pgfpathcurveto{\pgfqpoint{3.558875in}{3.228466in}}{\pgfqpoint{3.548276in}{3.224076in}}{\pgfqpoint{3.540462in}{3.216262in}}%
\pgfpathcurveto{\pgfqpoint{3.532648in}{3.208449in}}{\pgfqpoint{3.528258in}{3.197850in}}{\pgfqpoint{3.528258in}{3.186800in}}%
\pgfpathcurveto{\pgfqpoint{3.528258in}{3.175750in}}{\pgfqpoint{3.532648in}{3.165151in}}{\pgfqpoint{3.540462in}{3.157337in}}%
\pgfpathcurveto{\pgfqpoint{3.548276in}{3.149523in}}{\pgfqpoint{3.558875in}{3.145133in}}{\pgfqpoint{3.569925in}{3.145133in}}%
\pgfpathclose%
\pgfusepath{stroke,fill}%
\end{pgfscope}%
\begin{pgfscope}%
\pgfpathrectangle{\pgfqpoint{0.648703in}{0.548769in}}{\pgfqpoint{5.201297in}{3.102590in}}%
\pgfusepath{clip}%
\pgfsetbuttcap%
\pgfsetroundjoin%
\definecolor{currentfill}{rgb}{0.121569,0.466667,0.705882}%
\pgfsetfillcolor{currentfill}%
\pgfsetlinewidth{1.003750pt}%
\definecolor{currentstroke}{rgb}{0.121569,0.466667,0.705882}%
\pgfsetstrokecolor{currentstroke}%
\pgfsetdash{}{0pt}%
\pgfpathmoveto{\pgfqpoint{3.449710in}{3.132690in}}%
\pgfpathcurveto{\pgfqpoint{3.460760in}{3.132690in}}{\pgfqpoint{3.471359in}{3.137080in}}{\pgfqpoint{3.479173in}{3.144893in}}%
\pgfpathcurveto{\pgfqpoint{3.486986in}{3.152707in}}{\pgfqpoint{3.491376in}{3.163306in}}{\pgfqpoint{3.491376in}{3.174356in}}%
\pgfpathcurveto{\pgfqpoint{3.491376in}{3.185406in}}{\pgfqpoint{3.486986in}{3.196005in}}{\pgfqpoint{3.479173in}{3.203819in}}%
\pgfpathcurveto{\pgfqpoint{3.471359in}{3.211633in}}{\pgfqpoint{3.460760in}{3.216023in}}{\pgfqpoint{3.449710in}{3.216023in}}%
\pgfpathcurveto{\pgfqpoint{3.438660in}{3.216023in}}{\pgfqpoint{3.428061in}{3.211633in}}{\pgfqpoint{3.420247in}{3.203819in}}%
\pgfpathcurveto{\pgfqpoint{3.412433in}{3.196005in}}{\pgfqpoint{3.408043in}{3.185406in}}{\pgfqpoint{3.408043in}{3.174356in}}%
\pgfpathcurveto{\pgfqpoint{3.408043in}{3.163306in}}{\pgfqpoint{3.412433in}{3.152707in}}{\pgfqpoint{3.420247in}{3.144893in}}%
\pgfpathcurveto{\pgfqpoint{3.428061in}{3.137080in}}{\pgfqpoint{3.438660in}{3.132690in}}{\pgfqpoint{3.449710in}{3.132690in}}%
\pgfpathclose%
\pgfusepath{stroke,fill}%
\end{pgfscope}%
\begin{pgfscope}%
\pgfpathrectangle{\pgfqpoint{0.648703in}{0.548769in}}{\pgfqpoint{5.201297in}{3.102590in}}%
\pgfusepath{clip}%
\pgfsetbuttcap%
\pgfsetroundjoin%
\definecolor{currentfill}{rgb}{1.000000,0.498039,0.054902}%
\pgfsetfillcolor{currentfill}%
\pgfsetlinewidth{1.003750pt}%
\definecolor{currentstroke}{rgb}{1.000000,0.498039,0.054902}%
\pgfsetstrokecolor{currentstroke}%
\pgfsetdash{}{0pt}%
\pgfpathmoveto{\pgfqpoint{3.249352in}{3.348378in}}%
\pgfpathcurveto{\pgfqpoint{3.260402in}{3.348378in}}{\pgfqpoint{3.271001in}{3.352768in}}{\pgfqpoint{3.278814in}{3.360581in}}%
\pgfpathcurveto{\pgfqpoint{3.286628in}{3.368395in}}{\pgfqpoint{3.291018in}{3.378994in}}{\pgfqpoint{3.291018in}{3.390044in}}%
\pgfpathcurveto{\pgfqpoint{3.291018in}{3.401094in}}{\pgfqpoint{3.286628in}{3.411693in}}{\pgfqpoint{3.278814in}{3.419507in}}%
\pgfpathcurveto{\pgfqpoint{3.271001in}{3.427321in}}{\pgfqpoint{3.260402in}{3.431711in}}{\pgfqpoint{3.249352in}{3.431711in}}%
\pgfpathcurveto{\pgfqpoint{3.238302in}{3.431711in}}{\pgfqpoint{3.227703in}{3.427321in}}{\pgfqpoint{3.219889in}{3.419507in}}%
\pgfpathcurveto{\pgfqpoint{3.212075in}{3.411693in}}{\pgfqpoint{3.207685in}{3.401094in}}{\pgfqpoint{3.207685in}{3.390044in}}%
\pgfpathcurveto{\pgfqpoint{3.207685in}{3.378994in}}{\pgfqpoint{3.212075in}{3.368395in}}{\pgfqpoint{3.219889in}{3.360581in}}%
\pgfpathcurveto{\pgfqpoint{3.227703in}{3.352768in}}{\pgfqpoint{3.238302in}{3.348378in}}{\pgfqpoint{3.249352in}{3.348378in}}%
\pgfpathclose%
\pgfusepath{stroke,fill}%
\end{pgfscope}%
\begin{pgfscope}%
\pgfsetbuttcap%
\pgfsetroundjoin%
\definecolor{currentfill}{rgb}{0.000000,0.000000,0.000000}%
\pgfsetfillcolor{currentfill}%
\pgfsetlinewidth{0.803000pt}%
\definecolor{currentstroke}{rgb}{0.000000,0.000000,0.000000}%
\pgfsetstrokecolor{currentstroke}%
\pgfsetdash{}{0pt}%
\pgfsys@defobject{currentmarker}{\pgfqpoint{0.000000in}{-0.048611in}}{\pgfqpoint{0.000000in}{0.000000in}}{%
\pgfpathmoveto{\pgfqpoint{0.000000in}{0.000000in}}%
\pgfpathlineto{\pgfqpoint{0.000000in}{-0.048611in}}%
\pgfusepath{stroke,fill}%
}%
\begin{pgfscope}%
\pgfsys@transformshift{1.085484in}{0.548769in}%
\pgfsys@useobject{currentmarker}{}%
\end{pgfscope}%
\end{pgfscope}%
\begin{pgfscope}%
\definecolor{textcolor}{rgb}{0.000000,0.000000,0.000000}%
\pgfsetstrokecolor{textcolor}%
\pgfsetfillcolor{textcolor}%
\pgftext[x=1.085484in,y=0.451547in,,top]{\color{textcolor}\sffamily\fontsize{10.000000}{12.000000}\selectfont \(\displaystyle {-60}\)}%
\end{pgfscope}%
\begin{pgfscope}%
\pgfsetbuttcap%
\pgfsetroundjoin%
\definecolor{currentfill}{rgb}{0.000000,0.000000,0.000000}%
\pgfsetfillcolor{currentfill}%
\pgfsetlinewidth{0.803000pt}%
\definecolor{currentstroke}{rgb}{0.000000,0.000000,0.000000}%
\pgfsetstrokecolor{currentstroke}%
\pgfsetdash{}{0pt}%
\pgfsys@defobject{currentmarker}{\pgfqpoint{0.000000in}{-0.048611in}}{\pgfqpoint{0.000000in}{0.000000in}}{%
\pgfpathmoveto{\pgfqpoint{0.000000in}{0.000000in}}%
\pgfpathlineto{\pgfqpoint{0.000000in}{-0.048611in}}%
\pgfusepath{stroke,fill}%
}%
\begin{pgfscope}%
\pgfsys@transformshift{1.886917in}{0.548769in}%
\pgfsys@useobject{currentmarker}{}%
\end{pgfscope}%
\end{pgfscope}%
\begin{pgfscope}%
\definecolor{textcolor}{rgb}{0.000000,0.000000,0.000000}%
\pgfsetstrokecolor{textcolor}%
\pgfsetfillcolor{textcolor}%
\pgftext[x=1.886917in,y=0.451547in,,top]{\color{textcolor}\sffamily\fontsize{10.000000}{12.000000}\selectfont \(\displaystyle {-40}\)}%
\end{pgfscope}%
\begin{pgfscope}%
\pgfsetbuttcap%
\pgfsetroundjoin%
\definecolor{currentfill}{rgb}{0.000000,0.000000,0.000000}%
\pgfsetfillcolor{currentfill}%
\pgfsetlinewidth{0.803000pt}%
\definecolor{currentstroke}{rgb}{0.000000,0.000000,0.000000}%
\pgfsetstrokecolor{currentstroke}%
\pgfsetdash{}{0pt}%
\pgfsys@defobject{currentmarker}{\pgfqpoint{0.000000in}{-0.048611in}}{\pgfqpoint{0.000000in}{0.000000in}}{%
\pgfpathmoveto{\pgfqpoint{0.000000in}{0.000000in}}%
\pgfpathlineto{\pgfqpoint{0.000000in}{-0.048611in}}%
\pgfusepath{stroke,fill}%
}%
\begin{pgfscope}%
\pgfsys@transformshift{2.688349in}{0.548769in}%
\pgfsys@useobject{currentmarker}{}%
\end{pgfscope}%
\end{pgfscope}%
\begin{pgfscope}%
\definecolor{textcolor}{rgb}{0.000000,0.000000,0.000000}%
\pgfsetstrokecolor{textcolor}%
\pgfsetfillcolor{textcolor}%
\pgftext[x=2.688349in,y=0.451547in,,top]{\color{textcolor}\sffamily\fontsize{10.000000}{12.000000}\selectfont \(\displaystyle {-20}\)}%
\end{pgfscope}%
\begin{pgfscope}%
\pgfsetbuttcap%
\pgfsetroundjoin%
\definecolor{currentfill}{rgb}{0.000000,0.000000,0.000000}%
\pgfsetfillcolor{currentfill}%
\pgfsetlinewidth{0.803000pt}%
\definecolor{currentstroke}{rgb}{0.000000,0.000000,0.000000}%
\pgfsetstrokecolor{currentstroke}%
\pgfsetdash{}{0pt}%
\pgfsys@defobject{currentmarker}{\pgfqpoint{0.000000in}{-0.048611in}}{\pgfqpoint{0.000000in}{0.000000in}}{%
\pgfpathmoveto{\pgfqpoint{0.000000in}{0.000000in}}%
\pgfpathlineto{\pgfqpoint{0.000000in}{-0.048611in}}%
\pgfusepath{stroke,fill}%
}%
\begin{pgfscope}%
\pgfsys@transformshift{3.489781in}{0.548769in}%
\pgfsys@useobject{currentmarker}{}%
\end{pgfscope}%
\end{pgfscope}%
\begin{pgfscope}%
\definecolor{textcolor}{rgb}{0.000000,0.000000,0.000000}%
\pgfsetstrokecolor{textcolor}%
\pgfsetfillcolor{textcolor}%
\pgftext[x=3.489781in,y=0.451547in,,top]{\color{textcolor}\sffamily\fontsize{10.000000}{12.000000}\selectfont \(\displaystyle {0}\)}%
\end{pgfscope}%
\begin{pgfscope}%
\pgfsetbuttcap%
\pgfsetroundjoin%
\definecolor{currentfill}{rgb}{0.000000,0.000000,0.000000}%
\pgfsetfillcolor{currentfill}%
\pgfsetlinewidth{0.803000pt}%
\definecolor{currentstroke}{rgb}{0.000000,0.000000,0.000000}%
\pgfsetstrokecolor{currentstroke}%
\pgfsetdash{}{0pt}%
\pgfsys@defobject{currentmarker}{\pgfqpoint{0.000000in}{-0.048611in}}{\pgfqpoint{0.000000in}{0.000000in}}{%
\pgfpathmoveto{\pgfqpoint{0.000000in}{0.000000in}}%
\pgfpathlineto{\pgfqpoint{0.000000in}{-0.048611in}}%
\pgfusepath{stroke,fill}%
}%
\begin{pgfscope}%
\pgfsys@transformshift{4.291214in}{0.548769in}%
\pgfsys@useobject{currentmarker}{}%
\end{pgfscope}%
\end{pgfscope}%
\begin{pgfscope}%
\definecolor{textcolor}{rgb}{0.000000,0.000000,0.000000}%
\pgfsetstrokecolor{textcolor}%
\pgfsetfillcolor{textcolor}%
\pgftext[x=4.291214in,y=0.451547in,,top]{\color{textcolor}\sffamily\fontsize{10.000000}{12.000000}\selectfont \(\displaystyle {20}\)}%
\end{pgfscope}%
\begin{pgfscope}%
\pgfsetbuttcap%
\pgfsetroundjoin%
\definecolor{currentfill}{rgb}{0.000000,0.000000,0.000000}%
\pgfsetfillcolor{currentfill}%
\pgfsetlinewidth{0.803000pt}%
\definecolor{currentstroke}{rgb}{0.000000,0.000000,0.000000}%
\pgfsetstrokecolor{currentstroke}%
\pgfsetdash{}{0pt}%
\pgfsys@defobject{currentmarker}{\pgfqpoint{0.000000in}{-0.048611in}}{\pgfqpoint{0.000000in}{0.000000in}}{%
\pgfpathmoveto{\pgfqpoint{0.000000in}{0.000000in}}%
\pgfpathlineto{\pgfqpoint{0.000000in}{-0.048611in}}%
\pgfusepath{stroke,fill}%
}%
\begin{pgfscope}%
\pgfsys@transformshift{5.092646in}{0.548769in}%
\pgfsys@useobject{currentmarker}{}%
\end{pgfscope}%
\end{pgfscope}%
\begin{pgfscope}%
\definecolor{textcolor}{rgb}{0.000000,0.000000,0.000000}%
\pgfsetstrokecolor{textcolor}%
\pgfsetfillcolor{textcolor}%
\pgftext[x=5.092646in,y=0.451547in,,top]{\color{textcolor}\sffamily\fontsize{10.000000}{12.000000}\selectfont \(\displaystyle {40}\)}%
\end{pgfscope}%
\begin{pgfscope}%
\definecolor{textcolor}{rgb}{0.000000,0.000000,0.000000}%
\pgfsetstrokecolor{textcolor}%
\pgfsetfillcolor{textcolor}%
\pgftext[x=3.249352in,y=0.272658in,,top]{\color{textcolor}\sffamily\fontsize{10.000000}{12.000000}\selectfont Ratio of Sources to Sinks}%
\end{pgfscope}%
\begin{pgfscope}%
\pgfsetbuttcap%
\pgfsetroundjoin%
\definecolor{currentfill}{rgb}{0.000000,0.000000,0.000000}%
\pgfsetfillcolor{currentfill}%
\pgfsetlinewidth{0.803000pt}%
\definecolor{currentstroke}{rgb}{0.000000,0.000000,0.000000}%
\pgfsetstrokecolor{currentstroke}%
\pgfsetdash{}{0pt}%
\pgfsys@defobject{currentmarker}{\pgfqpoint{-0.048611in}{0.000000in}}{\pgfqpoint{0.000000in}{0.000000in}}{%
\pgfpathmoveto{\pgfqpoint{0.000000in}{0.000000in}}%
\pgfpathlineto{\pgfqpoint{-0.048611in}{0.000000in}}%
\pgfusepath{stroke,fill}%
}%
\begin{pgfscope}%
\pgfsys@transformshift{0.648703in}{0.689796in}%
\pgfsys@useobject{currentmarker}{}%
\end{pgfscope}%
\end{pgfscope}%
\begin{pgfscope}%
\definecolor{textcolor}{rgb}{0.000000,0.000000,0.000000}%
\pgfsetstrokecolor{textcolor}%
\pgfsetfillcolor{textcolor}%
\pgftext[x=0.482036in, y=0.641601in, left, base]{\color{textcolor}\sffamily\fontsize{10.000000}{12.000000}\selectfont \(\displaystyle {0}\)}%
\end{pgfscope}%
\begin{pgfscope}%
\pgfsetbuttcap%
\pgfsetroundjoin%
\definecolor{currentfill}{rgb}{0.000000,0.000000,0.000000}%
\pgfsetfillcolor{currentfill}%
\pgfsetlinewidth{0.803000pt}%
\definecolor{currentstroke}{rgb}{0.000000,0.000000,0.000000}%
\pgfsetstrokecolor{currentstroke}%
\pgfsetdash{}{0pt}%
\pgfsys@defobject{currentmarker}{\pgfqpoint{-0.048611in}{0.000000in}}{\pgfqpoint{0.000000in}{0.000000in}}{%
\pgfpathmoveto{\pgfqpoint{0.000000in}{0.000000in}}%
\pgfpathlineto{\pgfqpoint{-0.048611in}{0.000000in}}%
\pgfusepath{stroke,fill}%
}%
\begin{pgfscope}%
\pgfsys@transformshift{0.648703in}{1.104580in}%
\pgfsys@useobject{currentmarker}{}%
\end{pgfscope}%
\end{pgfscope}%
\begin{pgfscope}%
\definecolor{textcolor}{rgb}{0.000000,0.000000,0.000000}%
\pgfsetstrokecolor{textcolor}%
\pgfsetfillcolor{textcolor}%
\pgftext[x=0.343147in, y=1.056386in, left, base]{\color{textcolor}\sffamily\fontsize{10.000000}{12.000000}\selectfont \(\displaystyle {100}\)}%
\end{pgfscope}%
\begin{pgfscope}%
\pgfsetbuttcap%
\pgfsetroundjoin%
\definecolor{currentfill}{rgb}{0.000000,0.000000,0.000000}%
\pgfsetfillcolor{currentfill}%
\pgfsetlinewidth{0.803000pt}%
\definecolor{currentstroke}{rgb}{0.000000,0.000000,0.000000}%
\pgfsetstrokecolor{currentstroke}%
\pgfsetdash{}{0pt}%
\pgfsys@defobject{currentmarker}{\pgfqpoint{-0.048611in}{0.000000in}}{\pgfqpoint{0.000000in}{0.000000in}}{%
\pgfpathmoveto{\pgfqpoint{0.000000in}{0.000000in}}%
\pgfpathlineto{\pgfqpoint{-0.048611in}{0.000000in}}%
\pgfusepath{stroke,fill}%
}%
\begin{pgfscope}%
\pgfsys@transformshift{0.648703in}{1.519365in}%
\pgfsys@useobject{currentmarker}{}%
\end{pgfscope}%
\end{pgfscope}%
\begin{pgfscope}%
\definecolor{textcolor}{rgb}{0.000000,0.000000,0.000000}%
\pgfsetstrokecolor{textcolor}%
\pgfsetfillcolor{textcolor}%
\pgftext[x=0.343147in, y=1.471171in, left, base]{\color{textcolor}\sffamily\fontsize{10.000000}{12.000000}\selectfont \(\displaystyle {200}\)}%
\end{pgfscope}%
\begin{pgfscope}%
\pgfsetbuttcap%
\pgfsetroundjoin%
\definecolor{currentfill}{rgb}{0.000000,0.000000,0.000000}%
\pgfsetfillcolor{currentfill}%
\pgfsetlinewidth{0.803000pt}%
\definecolor{currentstroke}{rgb}{0.000000,0.000000,0.000000}%
\pgfsetstrokecolor{currentstroke}%
\pgfsetdash{}{0pt}%
\pgfsys@defobject{currentmarker}{\pgfqpoint{-0.048611in}{0.000000in}}{\pgfqpoint{0.000000in}{0.000000in}}{%
\pgfpathmoveto{\pgfqpoint{0.000000in}{0.000000in}}%
\pgfpathlineto{\pgfqpoint{-0.048611in}{0.000000in}}%
\pgfusepath{stroke,fill}%
}%
\begin{pgfscope}%
\pgfsys@transformshift{0.648703in}{1.934150in}%
\pgfsys@useobject{currentmarker}{}%
\end{pgfscope}%
\end{pgfscope}%
\begin{pgfscope}%
\definecolor{textcolor}{rgb}{0.000000,0.000000,0.000000}%
\pgfsetstrokecolor{textcolor}%
\pgfsetfillcolor{textcolor}%
\pgftext[x=0.343147in, y=1.885955in, left, base]{\color{textcolor}\sffamily\fontsize{10.000000}{12.000000}\selectfont \(\displaystyle {300}\)}%
\end{pgfscope}%
\begin{pgfscope}%
\pgfsetbuttcap%
\pgfsetroundjoin%
\definecolor{currentfill}{rgb}{0.000000,0.000000,0.000000}%
\pgfsetfillcolor{currentfill}%
\pgfsetlinewidth{0.803000pt}%
\definecolor{currentstroke}{rgb}{0.000000,0.000000,0.000000}%
\pgfsetstrokecolor{currentstroke}%
\pgfsetdash{}{0pt}%
\pgfsys@defobject{currentmarker}{\pgfqpoint{-0.048611in}{0.000000in}}{\pgfqpoint{0.000000in}{0.000000in}}{%
\pgfpathmoveto{\pgfqpoint{0.000000in}{0.000000in}}%
\pgfpathlineto{\pgfqpoint{-0.048611in}{0.000000in}}%
\pgfusepath{stroke,fill}%
}%
\begin{pgfscope}%
\pgfsys@transformshift{0.648703in}{2.348935in}%
\pgfsys@useobject{currentmarker}{}%
\end{pgfscope}%
\end{pgfscope}%
\begin{pgfscope}%
\definecolor{textcolor}{rgb}{0.000000,0.000000,0.000000}%
\pgfsetstrokecolor{textcolor}%
\pgfsetfillcolor{textcolor}%
\pgftext[x=0.343147in, y=2.300740in, left, base]{\color{textcolor}\sffamily\fontsize{10.000000}{12.000000}\selectfont \(\displaystyle {400}\)}%
\end{pgfscope}%
\begin{pgfscope}%
\pgfsetbuttcap%
\pgfsetroundjoin%
\definecolor{currentfill}{rgb}{0.000000,0.000000,0.000000}%
\pgfsetfillcolor{currentfill}%
\pgfsetlinewidth{0.803000pt}%
\definecolor{currentstroke}{rgb}{0.000000,0.000000,0.000000}%
\pgfsetstrokecolor{currentstroke}%
\pgfsetdash{}{0pt}%
\pgfsys@defobject{currentmarker}{\pgfqpoint{-0.048611in}{0.000000in}}{\pgfqpoint{0.000000in}{0.000000in}}{%
\pgfpathmoveto{\pgfqpoint{0.000000in}{0.000000in}}%
\pgfpathlineto{\pgfqpoint{-0.048611in}{0.000000in}}%
\pgfusepath{stroke,fill}%
}%
\begin{pgfscope}%
\pgfsys@transformshift{0.648703in}{2.763719in}%
\pgfsys@useobject{currentmarker}{}%
\end{pgfscope}%
\end{pgfscope}%
\begin{pgfscope}%
\definecolor{textcolor}{rgb}{0.000000,0.000000,0.000000}%
\pgfsetstrokecolor{textcolor}%
\pgfsetfillcolor{textcolor}%
\pgftext[x=0.343147in, y=2.715525in, left, base]{\color{textcolor}\sffamily\fontsize{10.000000}{12.000000}\selectfont \(\displaystyle {500}\)}%
\end{pgfscope}%
\begin{pgfscope}%
\pgfsetbuttcap%
\pgfsetroundjoin%
\definecolor{currentfill}{rgb}{0.000000,0.000000,0.000000}%
\pgfsetfillcolor{currentfill}%
\pgfsetlinewidth{0.803000pt}%
\definecolor{currentstroke}{rgb}{0.000000,0.000000,0.000000}%
\pgfsetstrokecolor{currentstroke}%
\pgfsetdash{}{0pt}%
\pgfsys@defobject{currentmarker}{\pgfqpoint{-0.048611in}{0.000000in}}{\pgfqpoint{0.000000in}{0.000000in}}{%
\pgfpathmoveto{\pgfqpoint{0.000000in}{0.000000in}}%
\pgfpathlineto{\pgfqpoint{-0.048611in}{0.000000in}}%
\pgfusepath{stroke,fill}%
}%
\begin{pgfscope}%
\pgfsys@transformshift{0.648703in}{3.178504in}%
\pgfsys@useobject{currentmarker}{}%
\end{pgfscope}%
\end{pgfscope}%
\begin{pgfscope}%
\definecolor{textcolor}{rgb}{0.000000,0.000000,0.000000}%
\pgfsetstrokecolor{textcolor}%
\pgfsetfillcolor{textcolor}%
\pgftext[x=0.343147in, y=3.130310in, left, base]{\color{textcolor}\sffamily\fontsize{10.000000}{12.000000}\selectfont \(\displaystyle {600}\)}%
\end{pgfscope}%
\begin{pgfscope}%
\pgfsetbuttcap%
\pgfsetroundjoin%
\definecolor{currentfill}{rgb}{0.000000,0.000000,0.000000}%
\pgfsetfillcolor{currentfill}%
\pgfsetlinewidth{0.803000pt}%
\definecolor{currentstroke}{rgb}{0.000000,0.000000,0.000000}%
\pgfsetstrokecolor{currentstroke}%
\pgfsetdash{}{0pt}%
\pgfsys@defobject{currentmarker}{\pgfqpoint{-0.048611in}{0.000000in}}{\pgfqpoint{0.000000in}{0.000000in}}{%
\pgfpathmoveto{\pgfqpoint{0.000000in}{0.000000in}}%
\pgfpathlineto{\pgfqpoint{-0.048611in}{0.000000in}}%
\pgfusepath{stroke,fill}%
}%
\begin{pgfscope}%
\pgfsys@transformshift{0.648703in}{3.593289in}%
\pgfsys@useobject{currentmarker}{}%
\end{pgfscope}%
\end{pgfscope}%
\begin{pgfscope}%
\definecolor{textcolor}{rgb}{0.000000,0.000000,0.000000}%
\pgfsetstrokecolor{textcolor}%
\pgfsetfillcolor{textcolor}%
\pgftext[x=0.343147in, y=3.545094in, left, base]{\color{textcolor}\sffamily\fontsize{10.000000}{12.000000}\selectfont \(\displaystyle {700}\)}%
\end{pgfscope}%
\begin{pgfscope}%
\definecolor{textcolor}{rgb}{0.000000,0.000000,0.000000}%
\pgfsetstrokecolor{textcolor}%
\pgfsetfillcolor{textcolor}%
\pgftext[x=0.287592in,y=2.100064in,,bottom,rotate=90.000000]{\color{textcolor}\sffamily\fontsize{10.000000}{12.000000}\selectfont Data Flow Time (s)}%
\end{pgfscope}%
\begin{pgfscope}%
\pgfsetrectcap%
\pgfsetmiterjoin%
\pgfsetlinewidth{0.803000pt}%
\definecolor{currentstroke}{rgb}{0.000000,0.000000,0.000000}%
\pgfsetstrokecolor{currentstroke}%
\pgfsetdash{}{0pt}%
\pgfpathmoveto{\pgfqpoint{0.648703in}{0.548769in}}%
\pgfpathlineto{\pgfqpoint{0.648703in}{3.651359in}}%
\pgfusepath{stroke}%
\end{pgfscope}%
\begin{pgfscope}%
\pgfsetrectcap%
\pgfsetmiterjoin%
\pgfsetlinewidth{0.803000pt}%
\definecolor{currentstroke}{rgb}{0.000000,0.000000,0.000000}%
\pgfsetstrokecolor{currentstroke}%
\pgfsetdash{}{0pt}%
\pgfpathmoveto{\pgfqpoint{5.850000in}{0.548769in}}%
\pgfpathlineto{\pgfqpoint{5.850000in}{3.651359in}}%
\pgfusepath{stroke}%
\end{pgfscope}%
\begin{pgfscope}%
\pgfsetrectcap%
\pgfsetmiterjoin%
\pgfsetlinewidth{0.803000pt}%
\definecolor{currentstroke}{rgb}{0.000000,0.000000,0.000000}%
\pgfsetstrokecolor{currentstroke}%
\pgfsetdash{}{0pt}%
\pgfpathmoveto{\pgfqpoint{0.648703in}{0.548769in}}%
\pgfpathlineto{\pgfqpoint{5.850000in}{0.548769in}}%
\pgfusepath{stroke}%
\end{pgfscope}%
\begin{pgfscope}%
\pgfsetrectcap%
\pgfsetmiterjoin%
\pgfsetlinewidth{0.803000pt}%
\definecolor{currentstroke}{rgb}{0.000000,0.000000,0.000000}%
\pgfsetstrokecolor{currentstroke}%
\pgfsetdash{}{0pt}%
\pgfpathmoveto{\pgfqpoint{0.648703in}{3.651359in}}%
\pgfpathlineto{\pgfqpoint{5.850000in}{3.651359in}}%
\pgfusepath{stroke}%
\end{pgfscope}%
\begin{pgfscope}%
\definecolor{textcolor}{rgb}{0.000000,0.000000,0.000000}%
\pgfsetstrokecolor{textcolor}%
\pgfsetfillcolor{textcolor}%
\pgftext[x=3.249352in,y=3.734692in,,base]{\color{textcolor}\sffamily\fontsize{12.000000}{14.400000}\selectfont Forward}%
\end{pgfscope}%
\begin{pgfscope}%
\pgfsetbuttcap%
\pgfsetmiterjoin%
\definecolor{currentfill}{rgb}{1.000000,1.000000,1.000000}%
\pgfsetfillcolor{currentfill}%
\pgfsetfillopacity{0.800000}%
\pgfsetlinewidth{1.003750pt}%
\definecolor{currentstroke}{rgb}{0.800000,0.800000,0.800000}%
\pgfsetstrokecolor{currentstroke}%
\pgfsetstrokeopacity{0.800000}%
\pgfsetdash{}{0pt}%
\pgfpathmoveto{\pgfqpoint{0.745926in}{0.618213in}}%
\pgfpathlineto{\pgfqpoint{2.198287in}{0.618213in}}%
\pgfpathquadraticcurveto{\pgfqpoint{2.226065in}{0.618213in}}{\pgfqpoint{2.226065in}{0.645991in}}%
\pgfpathlineto{\pgfqpoint{2.226065in}{1.214463in}}%
\pgfpathquadraticcurveto{\pgfqpoint{2.226065in}{1.242241in}}{\pgfqpoint{2.198287in}{1.242241in}}%
\pgfpathlineto{\pgfqpoint{0.745926in}{1.242241in}}%
\pgfpathquadraticcurveto{\pgfqpoint{0.718148in}{1.242241in}}{\pgfqpoint{0.718148in}{1.214463in}}%
\pgfpathlineto{\pgfqpoint{0.718148in}{0.645991in}}%
\pgfpathquadraticcurveto{\pgfqpoint{0.718148in}{0.618213in}}{\pgfqpoint{0.745926in}{0.618213in}}%
\pgfpathclose%
\pgfusepath{stroke,fill}%
\end{pgfscope}%
\begin{pgfscope}%
\pgfsetbuttcap%
\pgfsetroundjoin%
\definecolor{currentfill}{rgb}{0.121569,0.466667,0.705882}%
\pgfsetfillcolor{currentfill}%
\pgfsetlinewidth{1.003750pt}%
\definecolor{currentstroke}{rgb}{0.121569,0.466667,0.705882}%
\pgfsetstrokecolor{currentstroke}%
\pgfsetdash{}{0pt}%
\pgfsys@defobject{currentmarker}{\pgfqpoint{-0.034722in}{-0.034722in}}{\pgfqpoint{0.034722in}{0.034722in}}{%
\pgfpathmoveto{\pgfqpoint{0.000000in}{-0.034722in}}%
\pgfpathcurveto{\pgfqpoint{0.009208in}{-0.034722in}}{\pgfqpoint{0.018041in}{-0.031064in}}{\pgfqpoint{0.024552in}{-0.024552in}}%
\pgfpathcurveto{\pgfqpoint{0.031064in}{-0.018041in}}{\pgfqpoint{0.034722in}{-0.009208in}}{\pgfqpoint{0.034722in}{0.000000in}}%
\pgfpathcurveto{\pgfqpoint{0.034722in}{0.009208in}}{\pgfqpoint{0.031064in}{0.018041in}}{\pgfqpoint{0.024552in}{0.024552in}}%
\pgfpathcurveto{\pgfqpoint{0.018041in}{0.031064in}}{\pgfqpoint{0.009208in}{0.034722in}}{\pgfqpoint{0.000000in}{0.034722in}}%
\pgfpathcurveto{\pgfqpoint{-0.009208in}{0.034722in}}{\pgfqpoint{-0.018041in}{0.031064in}}{\pgfqpoint{-0.024552in}{0.024552in}}%
\pgfpathcurveto{\pgfqpoint{-0.031064in}{0.018041in}}{\pgfqpoint{-0.034722in}{0.009208in}}{\pgfqpoint{-0.034722in}{0.000000in}}%
\pgfpathcurveto{\pgfqpoint{-0.034722in}{-0.009208in}}{\pgfqpoint{-0.031064in}{-0.018041in}}{\pgfqpoint{-0.024552in}{-0.024552in}}%
\pgfpathcurveto{\pgfqpoint{-0.018041in}{-0.031064in}}{\pgfqpoint{-0.009208in}{-0.034722in}}{\pgfqpoint{0.000000in}{-0.034722in}}%
\pgfpathclose%
\pgfusepath{stroke,fill}%
}%
\begin{pgfscope}%
\pgfsys@transformshift{0.912592in}{1.138074in}%
\pgfsys@useobject{currentmarker}{}%
\end{pgfscope}%
\end{pgfscope}%
\begin{pgfscope}%
\definecolor{textcolor}{rgb}{0.000000,0.000000,0.000000}%
\pgfsetstrokecolor{textcolor}%
\pgfsetfillcolor{textcolor}%
\pgftext[x=1.162592in,y=1.089463in,left,base]{\color{textcolor}\sffamily\fontsize{10.000000}{12.000000}\selectfont No Timeout}%
\end{pgfscope}%
\begin{pgfscope}%
\pgfsetbuttcap%
\pgfsetroundjoin%
\definecolor{currentfill}{rgb}{1.000000,0.498039,0.054902}%
\pgfsetfillcolor{currentfill}%
\pgfsetlinewidth{1.003750pt}%
\definecolor{currentstroke}{rgb}{1.000000,0.498039,0.054902}%
\pgfsetstrokecolor{currentstroke}%
\pgfsetdash{}{0pt}%
\pgfsys@defobject{currentmarker}{\pgfqpoint{-0.034722in}{-0.034722in}}{\pgfqpoint{0.034722in}{0.034722in}}{%
\pgfpathmoveto{\pgfqpoint{0.000000in}{-0.034722in}}%
\pgfpathcurveto{\pgfqpoint{0.009208in}{-0.034722in}}{\pgfqpoint{0.018041in}{-0.031064in}}{\pgfqpoint{0.024552in}{-0.024552in}}%
\pgfpathcurveto{\pgfqpoint{0.031064in}{-0.018041in}}{\pgfqpoint{0.034722in}{-0.009208in}}{\pgfqpoint{0.034722in}{0.000000in}}%
\pgfpathcurveto{\pgfqpoint{0.034722in}{0.009208in}}{\pgfqpoint{0.031064in}{0.018041in}}{\pgfqpoint{0.024552in}{0.024552in}}%
\pgfpathcurveto{\pgfqpoint{0.018041in}{0.031064in}}{\pgfqpoint{0.009208in}{0.034722in}}{\pgfqpoint{0.000000in}{0.034722in}}%
\pgfpathcurveto{\pgfqpoint{-0.009208in}{0.034722in}}{\pgfqpoint{-0.018041in}{0.031064in}}{\pgfqpoint{-0.024552in}{0.024552in}}%
\pgfpathcurveto{\pgfqpoint{-0.031064in}{0.018041in}}{\pgfqpoint{-0.034722in}{0.009208in}}{\pgfqpoint{-0.034722in}{0.000000in}}%
\pgfpathcurveto{\pgfqpoint{-0.034722in}{-0.009208in}}{\pgfqpoint{-0.031064in}{-0.018041in}}{\pgfqpoint{-0.024552in}{-0.024552in}}%
\pgfpathcurveto{\pgfqpoint{-0.018041in}{-0.031064in}}{\pgfqpoint{-0.009208in}{-0.034722in}}{\pgfqpoint{0.000000in}{-0.034722in}}%
\pgfpathclose%
\pgfusepath{stroke,fill}%
}%
\begin{pgfscope}%
\pgfsys@transformshift{0.912592in}{0.944463in}%
\pgfsys@useobject{currentmarker}{}%
\end{pgfscope}%
\end{pgfscope}%
\begin{pgfscope}%
\definecolor{textcolor}{rgb}{0.000000,0.000000,0.000000}%
\pgfsetstrokecolor{textcolor}%
\pgfsetfillcolor{textcolor}%
\pgftext[x=1.162592in,y=0.895852in,left,base]{\color{textcolor}\sffamily\fontsize{10.000000}{12.000000}\selectfont Time Timeout}%
\end{pgfscope}%
\begin{pgfscope}%
\pgfsetbuttcap%
\pgfsetroundjoin%
\definecolor{currentfill}{rgb}{0.839216,0.152941,0.156863}%
\pgfsetfillcolor{currentfill}%
\pgfsetlinewidth{1.003750pt}%
\definecolor{currentstroke}{rgb}{0.839216,0.152941,0.156863}%
\pgfsetstrokecolor{currentstroke}%
\pgfsetdash{}{0pt}%
\pgfsys@defobject{currentmarker}{\pgfqpoint{-0.034722in}{-0.034722in}}{\pgfqpoint{0.034722in}{0.034722in}}{%
\pgfpathmoveto{\pgfqpoint{0.000000in}{-0.034722in}}%
\pgfpathcurveto{\pgfqpoint{0.009208in}{-0.034722in}}{\pgfqpoint{0.018041in}{-0.031064in}}{\pgfqpoint{0.024552in}{-0.024552in}}%
\pgfpathcurveto{\pgfqpoint{0.031064in}{-0.018041in}}{\pgfqpoint{0.034722in}{-0.009208in}}{\pgfqpoint{0.034722in}{0.000000in}}%
\pgfpathcurveto{\pgfqpoint{0.034722in}{0.009208in}}{\pgfqpoint{0.031064in}{0.018041in}}{\pgfqpoint{0.024552in}{0.024552in}}%
\pgfpathcurveto{\pgfqpoint{0.018041in}{0.031064in}}{\pgfqpoint{0.009208in}{0.034722in}}{\pgfqpoint{0.000000in}{0.034722in}}%
\pgfpathcurveto{\pgfqpoint{-0.009208in}{0.034722in}}{\pgfqpoint{-0.018041in}{0.031064in}}{\pgfqpoint{-0.024552in}{0.024552in}}%
\pgfpathcurveto{\pgfqpoint{-0.031064in}{0.018041in}}{\pgfqpoint{-0.034722in}{0.009208in}}{\pgfqpoint{-0.034722in}{0.000000in}}%
\pgfpathcurveto{\pgfqpoint{-0.034722in}{-0.009208in}}{\pgfqpoint{-0.031064in}{-0.018041in}}{\pgfqpoint{-0.024552in}{-0.024552in}}%
\pgfpathcurveto{\pgfqpoint{-0.018041in}{-0.031064in}}{\pgfqpoint{-0.009208in}{-0.034722in}}{\pgfqpoint{0.000000in}{-0.034722in}}%
\pgfpathclose%
\pgfusepath{stroke,fill}%
}%
\begin{pgfscope}%
\pgfsys@transformshift{0.912592in}{0.750852in}%
\pgfsys@useobject{currentmarker}{}%
\end{pgfscope}%
\end{pgfscope}%
\begin{pgfscope}%
\definecolor{textcolor}{rgb}{0.000000,0.000000,0.000000}%
\pgfsetstrokecolor{textcolor}%
\pgfsetfillcolor{textcolor}%
\pgftext[x=1.162592in,y=0.702241in,left,base]{\color{textcolor}\sffamily\fontsize{10.000000}{12.000000}\selectfont Memory Timeout}%
\end{pgfscope}%
\end{pgfpicture}%
\makeatother%
\endgroup%

                }
            \end{subfigure}
            \qquad
            \begin{subfigure}[]{0.45\textwidth}
                \centering
                \resizebox{\columnwidth}{!}{
                    %% Creator: Matplotlib, PGF backend
%%
%% To include the figure in your LaTeX document, write
%%   \input{<filename>.pgf}
%%
%% Make sure the required packages are loaded in your preamble
%%   \usepackage{pgf}
%%
%% and, on pdftex
%%   \usepackage[utf8]{inputenc}\DeclareUnicodeCharacter{2212}{-}
%%
%% or, on luatex and xetex
%%   \usepackage{unicode-math}
%%
%% Figures using additional raster images can only be included by \input if
%% they are in the same directory as the main LaTeX file. For loading figures
%% from other directories you can use the `import` package
%%   \usepackage{import}
%%
%% and then include the figures with
%%   \import{<path to file>}{<filename>.pgf}
%%
%% Matplotlib used the following preamble
%%   \usepackage{amsmath}
%%   \usepackage{fontspec}
%%
\begingroup%
\makeatletter%
\begin{pgfpicture}%
\pgfpathrectangle{\pgfpointorigin}{\pgfqpoint{6.000000in}{4.000000in}}%
\pgfusepath{use as bounding box, clip}%
\begin{pgfscope}%
\pgfsetbuttcap%
\pgfsetmiterjoin%
\definecolor{currentfill}{rgb}{1.000000,1.000000,1.000000}%
\pgfsetfillcolor{currentfill}%
\pgfsetlinewidth{0.000000pt}%
\definecolor{currentstroke}{rgb}{1.000000,1.000000,1.000000}%
\pgfsetstrokecolor{currentstroke}%
\pgfsetdash{}{0pt}%
\pgfpathmoveto{\pgfqpoint{0.000000in}{0.000000in}}%
\pgfpathlineto{\pgfqpoint{6.000000in}{0.000000in}}%
\pgfpathlineto{\pgfqpoint{6.000000in}{4.000000in}}%
\pgfpathlineto{\pgfqpoint{0.000000in}{4.000000in}}%
\pgfpathclose%
\pgfusepath{fill}%
\end{pgfscope}%
\begin{pgfscope}%
\pgfsetbuttcap%
\pgfsetmiterjoin%
\definecolor{currentfill}{rgb}{1.000000,1.000000,1.000000}%
\pgfsetfillcolor{currentfill}%
\pgfsetlinewidth{0.000000pt}%
\definecolor{currentstroke}{rgb}{0.000000,0.000000,0.000000}%
\pgfsetstrokecolor{currentstroke}%
\pgfsetstrokeopacity{0.000000}%
\pgfsetdash{}{0pt}%
\pgfpathmoveto{\pgfqpoint{0.648703in}{0.548769in}}%
\pgfpathlineto{\pgfqpoint{5.850000in}{0.548769in}}%
\pgfpathlineto{\pgfqpoint{5.850000in}{3.651359in}}%
\pgfpathlineto{\pgfqpoint{0.648703in}{3.651359in}}%
\pgfpathclose%
\pgfusepath{fill}%
\end{pgfscope}%
\begin{pgfscope}%
\pgfpathrectangle{\pgfqpoint{0.648703in}{0.548769in}}{\pgfqpoint{5.201297in}{3.102590in}}%
\pgfusepath{clip}%
\pgfsetbuttcap%
\pgfsetroundjoin%
\definecolor{currentfill}{rgb}{0.121569,0.466667,0.705882}%
\pgfsetfillcolor{currentfill}%
\pgfsetlinewidth{1.003750pt}%
\definecolor{currentstroke}{rgb}{0.121569,0.466667,0.705882}%
\pgfsetstrokecolor{currentstroke}%
\pgfsetdash{}{0pt}%
\pgfpathmoveto{\pgfqpoint{1.046124in}{0.673501in}}%
\pgfpathcurveto{\pgfqpoint{1.057174in}{0.673501in}}{\pgfqpoint{1.067773in}{0.677891in}}{\pgfqpoint{1.075587in}{0.685705in}}%
\pgfpathcurveto{\pgfqpoint{1.083401in}{0.693519in}}{\pgfqpoint{1.087791in}{0.704118in}}{\pgfqpoint{1.087791in}{0.715168in}}%
\pgfpathcurveto{\pgfqpoint{1.087791in}{0.726218in}}{\pgfqpoint{1.083401in}{0.736817in}}{\pgfqpoint{1.075587in}{0.744631in}}%
\pgfpathcurveto{\pgfqpoint{1.067773in}{0.752444in}}{\pgfqpoint{1.057174in}{0.756834in}}{\pgfqpoint{1.046124in}{0.756834in}}%
\pgfpathcurveto{\pgfqpoint{1.035074in}{0.756834in}}{\pgfqpoint{1.024475in}{0.752444in}}{\pgfqpoint{1.016662in}{0.744631in}}%
\pgfpathcurveto{\pgfqpoint{1.008848in}{0.736817in}}{\pgfqpoint{1.004458in}{0.726218in}}{\pgfqpoint{1.004458in}{0.715168in}}%
\pgfpathcurveto{\pgfqpoint{1.004458in}{0.704118in}}{\pgfqpoint{1.008848in}{0.693519in}}{\pgfqpoint{1.016662in}{0.685705in}}%
\pgfpathcurveto{\pgfqpoint{1.024475in}{0.677891in}}{\pgfqpoint{1.035074in}{0.673501in}}{\pgfqpoint{1.046124in}{0.673501in}}%
\pgfpathclose%
\pgfusepath{stroke,fill}%
\end{pgfscope}%
\begin{pgfscope}%
\pgfpathrectangle{\pgfqpoint{0.648703in}{0.548769in}}{\pgfqpoint{5.201297in}{3.102590in}}%
\pgfusepath{clip}%
\pgfsetbuttcap%
\pgfsetroundjoin%
\definecolor{currentfill}{rgb}{0.121569,0.466667,0.705882}%
\pgfsetfillcolor{currentfill}%
\pgfsetlinewidth{1.003750pt}%
\definecolor{currentstroke}{rgb}{0.121569,0.466667,0.705882}%
\pgfsetstrokecolor{currentstroke}%
\pgfsetdash{}{0pt}%
\pgfpathmoveto{\pgfqpoint{1.317673in}{3.176886in}}%
\pgfpathcurveto{\pgfqpoint{1.328723in}{3.176886in}}{\pgfqpoint{1.339322in}{3.181276in}}{\pgfqpoint{1.347136in}{3.189089in}}%
\pgfpathcurveto{\pgfqpoint{1.354950in}{3.196903in}}{\pgfqpoint{1.359340in}{3.207502in}}{\pgfqpoint{1.359340in}{3.218552in}}%
\pgfpathcurveto{\pgfqpoint{1.359340in}{3.229602in}}{\pgfqpoint{1.354950in}{3.240201in}}{\pgfqpoint{1.347136in}{3.248015in}}%
\pgfpathcurveto{\pgfqpoint{1.339322in}{3.255829in}}{\pgfqpoint{1.328723in}{3.260219in}}{\pgfqpoint{1.317673in}{3.260219in}}%
\pgfpathcurveto{\pgfqpoint{1.306623in}{3.260219in}}{\pgfqpoint{1.296024in}{3.255829in}}{\pgfqpoint{1.288210in}{3.248015in}}%
\pgfpathcurveto{\pgfqpoint{1.280397in}{3.240201in}}{\pgfqpoint{1.276007in}{3.229602in}}{\pgfqpoint{1.276007in}{3.218552in}}%
\pgfpathcurveto{\pgfqpoint{1.276007in}{3.207502in}}{\pgfqpoint{1.280397in}{3.196903in}}{\pgfqpoint{1.288210in}{3.189089in}}%
\pgfpathcurveto{\pgfqpoint{1.296024in}{3.181276in}}{\pgfqpoint{1.306623in}{3.176886in}}{\pgfqpoint{1.317673in}{3.176886in}}%
\pgfpathclose%
\pgfusepath{stroke,fill}%
\end{pgfscope}%
\begin{pgfscope}%
\pgfpathrectangle{\pgfqpoint{0.648703in}{0.548769in}}{\pgfqpoint{5.201297in}{3.102590in}}%
\pgfusepath{clip}%
\pgfsetbuttcap%
\pgfsetroundjoin%
\definecolor{currentfill}{rgb}{1.000000,0.498039,0.054902}%
\pgfsetfillcolor{currentfill}%
\pgfsetlinewidth{1.003750pt}%
\definecolor{currentstroke}{rgb}{1.000000,0.498039,0.054902}%
\pgfsetstrokecolor{currentstroke}%
\pgfsetdash{}{0pt}%
\pgfpathmoveto{\pgfqpoint{0.909626in}{3.198029in}}%
\pgfpathcurveto{\pgfqpoint{0.920676in}{3.198029in}}{\pgfqpoint{0.931275in}{3.202419in}}{\pgfqpoint{0.939089in}{3.210233in}}%
\pgfpathcurveto{\pgfqpoint{0.946902in}{3.218046in}}{\pgfqpoint{0.951292in}{3.228646in}}{\pgfqpoint{0.951292in}{3.239696in}}%
\pgfpathcurveto{\pgfqpoint{0.951292in}{3.250746in}}{\pgfqpoint{0.946902in}{3.261345in}}{\pgfqpoint{0.939089in}{3.269158in}}%
\pgfpathcurveto{\pgfqpoint{0.931275in}{3.276972in}}{\pgfqpoint{0.920676in}{3.281362in}}{\pgfqpoint{0.909626in}{3.281362in}}%
\pgfpathcurveto{\pgfqpoint{0.898576in}{3.281362in}}{\pgfqpoint{0.887977in}{3.276972in}}{\pgfqpoint{0.880163in}{3.269158in}}%
\pgfpathcurveto{\pgfqpoint{0.872349in}{3.261345in}}{\pgfqpoint{0.867959in}{3.250746in}}{\pgfqpoint{0.867959in}{3.239696in}}%
\pgfpathcurveto{\pgfqpoint{0.867959in}{3.228646in}}{\pgfqpoint{0.872349in}{3.218046in}}{\pgfqpoint{0.880163in}{3.210233in}}%
\pgfpathcurveto{\pgfqpoint{0.887977in}{3.202419in}}{\pgfqpoint{0.898576in}{3.198029in}}{\pgfqpoint{0.909626in}{3.198029in}}%
\pgfpathclose%
\pgfusepath{stroke,fill}%
\end{pgfscope}%
\begin{pgfscope}%
\pgfpathrectangle{\pgfqpoint{0.648703in}{0.548769in}}{\pgfqpoint{5.201297in}{3.102590in}}%
\pgfusepath{clip}%
\pgfsetbuttcap%
\pgfsetroundjoin%
\definecolor{currentfill}{rgb}{1.000000,0.498039,0.054902}%
\pgfsetfillcolor{currentfill}%
\pgfsetlinewidth{1.003750pt}%
\definecolor{currentstroke}{rgb}{1.000000,0.498039,0.054902}%
\pgfsetstrokecolor{currentstroke}%
\pgfsetdash{}{0pt}%
\pgfpathmoveto{\pgfqpoint{1.516675in}{3.185343in}}%
\pgfpathcurveto{\pgfqpoint{1.527725in}{3.185343in}}{\pgfqpoint{1.538324in}{3.189733in}}{\pgfqpoint{1.546138in}{3.197547in}}%
\pgfpathcurveto{\pgfqpoint{1.553951in}{3.205360in}}{\pgfqpoint{1.558342in}{3.215959in}}{\pgfqpoint{1.558342in}{3.227010in}}%
\pgfpathcurveto{\pgfqpoint{1.558342in}{3.238060in}}{\pgfqpoint{1.553951in}{3.248659in}}{\pgfqpoint{1.546138in}{3.256472in}}%
\pgfpathcurveto{\pgfqpoint{1.538324in}{3.264286in}}{\pgfqpoint{1.527725in}{3.268676in}}{\pgfqpoint{1.516675in}{3.268676in}}%
\pgfpathcurveto{\pgfqpoint{1.505625in}{3.268676in}}{\pgfqpoint{1.495026in}{3.264286in}}{\pgfqpoint{1.487212in}{3.256472in}}%
\pgfpathcurveto{\pgfqpoint{1.479399in}{3.248659in}}{\pgfqpoint{1.475008in}{3.238060in}}{\pgfqpoint{1.475008in}{3.227010in}}%
\pgfpathcurveto{\pgfqpoint{1.475008in}{3.215959in}}{\pgfqpoint{1.479399in}{3.205360in}}{\pgfqpoint{1.487212in}{3.197547in}}%
\pgfpathcurveto{\pgfqpoint{1.495026in}{3.189733in}}{\pgfqpoint{1.505625in}{3.185343in}}{\pgfqpoint{1.516675in}{3.185343in}}%
\pgfpathclose%
\pgfusepath{stroke,fill}%
\end{pgfscope}%
\begin{pgfscope}%
\pgfpathrectangle{\pgfqpoint{0.648703in}{0.548769in}}{\pgfqpoint{5.201297in}{3.102590in}}%
\pgfusepath{clip}%
\pgfsetbuttcap%
\pgfsetroundjoin%
\definecolor{currentfill}{rgb}{0.121569,0.466667,0.705882}%
\pgfsetfillcolor{currentfill}%
\pgfsetlinewidth{1.003750pt}%
\definecolor{currentstroke}{rgb}{0.121569,0.466667,0.705882}%
\pgfsetstrokecolor{currentstroke}%
\pgfsetdash{}{0pt}%
\pgfpathmoveto{\pgfqpoint{1.130123in}{0.652358in}}%
\pgfpathcurveto{\pgfqpoint{1.141174in}{0.652358in}}{\pgfqpoint{1.151773in}{0.656748in}}{\pgfqpoint{1.159586in}{0.664562in}}%
\pgfpathcurveto{\pgfqpoint{1.167400in}{0.672375in}}{\pgfqpoint{1.171790in}{0.682974in}}{\pgfqpoint{1.171790in}{0.694024in}}%
\pgfpathcurveto{\pgfqpoint{1.171790in}{0.705074in}}{\pgfqpoint{1.167400in}{0.715673in}}{\pgfqpoint{1.159586in}{0.723487in}}%
\pgfpathcurveto{\pgfqpoint{1.151773in}{0.731301in}}{\pgfqpoint{1.141174in}{0.735691in}}{\pgfqpoint{1.130123in}{0.735691in}}%
\pgfpathcurveto{\pgfqpoint{1.119073in}{0.735691in}}{\pgfqpoint{1.108474in}{0.731301in}}{\pgfqpoint{1.100661in}{0.723487in}}%
\pgfpathcurveto{\pgfqpoint{1.092847in}{0.715673in}}{\pgfqpoint{1.088457in}{0.705074in}}{\pgfqpoint{1.088457in}{0.694024in}}%
\pgfpathcurveto{\pgfqpoint{1.088457in}{0.682974in}}{\pgfqpoint{1.092847in}{0.672375in}}{\pgfqpoint{1.100661in}{0.664562in}}%
\pgfpathcurveto{\pgfqpoint{1.108474in}{0.656748in}}{\pgfqpoint{1.119073in}{0.652358in}}{\pgfqpoint{1.130123in}{0.652358in}}%
\pgfpathclose%
\pgfusepath{stroke,fill}%
\end{pgfscope}%
\begin{pgfscope}%
\pgfpathrectangle{\pgfqpoint{0.648703in}{0.548769in}}{\pgfqpoint{5.201297in}{3.102590in}}%
\pgfusepath{clip}%
\pgfsetbuttcap%
\pgfsetroundjoin%
\definecolor{currentfill}{rgb}{0.121569,0.466667,0.705882}%
\pgfsetfillcolor{currentfill}%
\pgfsetlinewidth{1.003750pt}%
\definecolor{currentstroke}{rgb}{0.121569,0.466667,0.705882}%
\pgfsetstrokecolor{currentstroke}%
\pgfsetdash{}{0pt}%
\pgfpathmoveto{\pgfqpoint{2.397985in}{3.181114in}}%
\pgfpathcurveto{\pgfqpoint{2.409036in}{3.181114in}}{\pgfqpoint{2.419635in}{3.185504in}}{\pgfqpoint{2.427448in}{3.193318in}}%
\pgfpathcurveto{\pgfqpoint{2.435262in}{3.201132in}}{\pgfqpoint{2.439652in}{3.211731in}}{\pgfqpoint{2.439652in}{3.222781in}}%
\pgfpathcurveto{\pgfqpoint{2.439652in}{3.233831in}}{\pgfqpoint{2.435262in}{3.244430in}}{\pgfqpoint{2.427448in}{3.252244in}}%
\pgfpathcurveto{\pgfqpoint{2.419635in}{3.260057in}}{\pgfqpoint{2.409036in}{3.264448in}}{\pgfqpoint{2.397985in}{3.264448in}}%
\pgfpathcurveto{\pgfqpoint{2.386935in}{3.264448in}}{\pgfqpoint{2.376336in}{3.260057in}}{\pgfqpoint{2.368523in}{3.252244in}}%
\pgfpathcurveto{\pgfqpoint{2.360709in}{3.244430in}}{\pgfqpoint{2.356319in}{3.233831in}}{\pgfqpoint{2.356319in}{3.222781in}}%
\pgfpathcurveto{\pgfqpoint{2.356319in}{3.211731in}}{\pgfqpoint{2.360709in}{3.201132in}}{\pgfqpoint{2.368523in}{3.193318in}}%
\pgfpathcurveto{\pgfqpoint{2.376336in}{3.185504in}}{\pgfqpoint{2.386935in}{3.181114in}}{\pgfqpoint{2.397985in}{3.181114in}}%
\pgfpathclose%
\pgfusepath{stroke,fill}%
\end{pgfscope}%
\begin{pgfscope}%
\pgfpathrectangle{\pgfqpoint{0.648703in}{0.548769in}}{\pgfqpoint{5.201297in}{3.102590in}}%
\pgfusepath{clip}%
\pgfsetbuttcap%
\pgfsetroundjoin%
\definecolor{currentfill}{rgb}{1.000000,0.498039,0.054902}%
\pgfsetfillcolor{currentfill}%
\pgfsetlinewidth{1.003750pt}%
\definecolor{currentstroke}{rgb}{1.000000,0.498039,0.054902}%
\pgfsetstrokecolor{currentstroke}%
\pgfsetdash{}{0pt}%
\pgfpathmoveto{\pgfqpoint{1.527884in}{3.189572in}}%
\pgfpathcurveto{\pgfqpoint{1.538934in}{3.189572in}}{\pgfqpoint{1.549533in}{3.193962in}}{\pgfqpoint{1.557347in}{3.201775in}}%
\pgfpathcurveto{\pgfqpoint{1.565160in}{3.209589in}}{\pgfqpoint{1.569551in}{3.220188in}}{\pgfqpoint{1.569551in}{3.231238in}}%
\pgfpathcurveto{\pgfqpoint{1.569551in}{3.242288in}}{\pgfqpoint{1.565160in}{3.252887in}}{\pgfqpoint{1.557347in}{3.260701in}}%
\pgfpathcurveto{\pgfqpoint{1.549533in}{3.268515in}}{\pgfqpoint{1.538934in}{3.272905in}}{\pgfqpoint{1.527884in}{3.272905in}}%
\pgfpathcurveto{\pgfqpoint{1.516834in}{3.272905in}}{\pgfqpoint{1.506235in}{3.268515in}}{\pgfqpoint{1.498421in}{3.260701in}}%
\pgfpathcurveto{\pgfqpoint{1.490608in}{3.252887in}}{\pgfqpoint{1.486217in}{3.242288in}}{\pgfqpoint{1.486217in}{3.231238in}}%
\pgfpathcurveto{\pgfqpoint{1.486217in}{3.220188in}}{\pgfqpoint{1.490608in}{3.209589in}}{\pgfqpoint{1.498421in}{3.201775in}}%
\pgfpathcurveto{\pgfqpoint{1.506235in}{3.193962in}}{\pgfqpoint{1.516834in}{3.189572in}}{\pgfqpoint{1.527884in}{3.189572in}}%
\pgfpathclose%
\pgfusepath{stroke,fill}%
\end{pgfscope}%
\begin{pgfscope}%
\pgfpathrectangle{\pgfqpoint{0.648703in}{0.548769in}}{\pgfqpoint{5.201297in}{3.102590in}}%
\pgfusepath{clip}%
\pgfsetbuttcap%
\pgfsetroundjoin%
\definecolor{currentfill}{rgb}{0.121569,0.466667,0.705882}%
\pgfsetfillcolor{currentfill}%
\pgfsetlinewidth{1.003750pt}%
\definecolor{currentstroke}{rgb}{0.121569,0.466667,0.705882}%
\pgfsetstrokecolor{currentstroke}%
\pgfsetdash{}{0pt}%
\pgfpathmoveto{\pgfqpoint{2.096114in}{0.648129in}}%
\pgfpathcurveto{\pgfqpoint{2.107164in}{0.648129in}}{\pgfqpoint{2.117763in}{0.652519in}}{\pgfqpoint{2.125576in}{0.660333in}}%
\pgfpathcurveto{\pgfqpoint{2.133390in}{0.668146in}}{\pgfqpoint{2.137780in}{0.678745in}}{\pgfqpoint{2.137780in}{0.689796in}}%
\pgfpathcurveto{\pgfqpoint{2.137780in}{0.700846in}}{\pgfqpoint{2.133390in}{0.711445in}}{\pgfqpoint{2.125576in}{0.719258in}}%
\pgfpathcurveto{\pgfqpoint{2.117763in}{0.727072in}}{\pgfqpoint{2.107164in}{0.731462in}}{\pgfqpoint{2.096114in}{0.731462in}}%
\pgfpathcurveto{\pgfqpoint{2.085063in}{0.731462in}}{\pgfqpoint{2.074464in}{0.727072in}}{\pgfqpoint{2.066651in}{0.719258in}}%
\pgfpathcurveto{\pgfqpoint{2.058837in}{0.711445in}}{\pgfqpoint{2.054447in}{0.700846in}}{\pgfqpoint{2.054447in}{0.689796in}}%
\pgfpathcurveto{\pgfqpoint{2.054447in}{0.678745in}}{\pgfqpoint{2.058837in}{0.668146in}}{\pgfqpoint{2.066651in}{0.660333in}}%
\pgfpathcurveto{\pgfqpoint{2.074464in}{0.652519in}}{\pgfqpoint{2.085063in}{0.648129in}}{\pgfqpoint{2.096114in}{0.648129in}}%
\pgfpathclose%
\pgfusepath{stroke,fill}%
\end{pgfscope}%
\begin{pgfscope}%
\pgfpathrectangle{\pgfqpoint{0.648703in}{0.548769in}}{\pgfqpoint{5.201297in}{3.102590in}}%
\pgfusepath{clip}%
\pgfsetbuttcap%
\pgfsetroundjoin%
\definecolor{currentfill}{rgb}{0.121569,0.466667,0.705882}%
\pgfsetfillcolor{currentfill}%
\pgfsetlinewidth{1.003750pt}%
\definecolor{currentstroke}{rgb}{0.121569,0.466667,0.705882}%
\pgfsetstrokecolor{currentstroke}%
\pgfsetdash{}{0pt}%
\pgfpathmoveto{\pgfqpoint{1.407787in}{0.774990in}}%
\pgfpathcurveto{\pgfqpoint{1.418837in}{0.774990in}}{\pgfqpoint{1.429436in}{0.779380in}}{\pgfqpoint{1.437250in}{0.787194in}}%
\pgfpathcurveto{\pgfqpoint{1.445064in}{0.795007in}}{\pgfqpoint{1.449454in}{0.805606in}}{\pgfqpoint{1.449454in}{0.816656in}}%
\pgfpathcurveto{\pgfqpoint{1.449454in}{0.827706in}}{\pgfqpoint{1.445064in}{0.838305in}}{\pgfqpoint{1.437250in}{0.846119in}}%
\pgfpathcurveto{\pgfqpoint{1.429436in}{0.853933in}}{\pgfqpoint{1.418837in}{0.858323in}}{\pgfqpoint{1.407787in}{0.858323in}}%
\pgfpathcurveto{\pgfqpoint{1.396737in}{0.858323in}}{\pgfqpoint{1.386138in}{0.853933in}}{\pgfqpoint{1.378325in}{0.846119in}}%
\pgfpathcurveto{\pgfqpoint{1.370511in}{0.838305in}}{\pgfqpoint{1.366121in}{0.827706in}}{\pgfqpoint{1.366121in}{0.816656in}}%
\pgfpathcurveto{\pgfqpoint{1.366121in}{0.805606in}}{\pgfqpoint{1.370511in}{0.795007in}}{\pgfqpoint{1.378325in}{0.787194in}}%
\pgfpathcurveto{\pgfqpoint{1.386138in}{0.779380in}}{\pgfqpoint{1.396737in}{0.774990in}}{\pgfqpoint{1.407787in}{0.774990in}}%
\pgfpathclose%
\pgfusepath{stroke,fill}%
\end{pgfscope}%
\begin{pgfscope}%
\pgfpathrectangle{\pgfqpoint{0.648703in}{0.548769in}}{\pgfqpoint{5.201297in}{3.102590in}}%
\pgfusepath{clip}%
\pgfsetbuttcap%
\pgfsetroundjoin%
\definecolor{currentfill}{rgb}{0.121569,0.466667,0.705882}%
\pgfsetfillcolor{currentfill}%
\pgfsetlinewidth{1.003750pt}%
\definecolor{currentstroke}{rgb}{0.121569,0.466667,0.705882}%
\pgfsetstrokecolor{currentstroke}%
\pgfsetdash{}{0pt}%
\pgfpathmoveto{\pgfqpoint{1.865116in}{0.783447in}}%
\pgfpathcurveto{\pgfqpoint{1.876166in}{0.783447in}}{\pgfqpoint{1.886765in}{0.787837in}}{\pgfqpoint{1.894579in}{0.795651in}}%
\pgfpathcurveto{\pgfqpoint{1.902392in}{0.803465in}}{\pgfqpoint{1.906783in}{0.814064in}}{\pgfqpoint{1.906783in}{0.825114in}}%
\pgfpathcurveto{\pgfqpoint{1.906783in}{0.836164in}}{\pgfqpoint{1.902392in}{0.846763in}}{\pgfqpoint{1.894579in}{0.854576in}}%
\pgfpathcurveto{\pgfqpoint{1.886765in}{0.862390in}}{\pgfqpoint{1.876166in}{0.866780in}}{\pgfqpoint{1.865116in}{0.866780in}}%
\pgfpathcurveto{\pgfqpoint{1.854066in}{0.866780in}}{\pgfqpoint{1.843467in}{0.862390in}}{\pgfqpoint{1.835653in}{0.854576in}}%
\pgfpathcurveto{\pgfqpoint{1.827840in}{0.846763in}}{\pgfqpoint{1.823449in}{0.836164in}}{\pgfqpoint{1.823449in}{0.825114in}}%
\pgfpathcurveto{\pgfqpoint{1.823449in}{0.814064in}}{\pgfqpoint{1.827840in}{0.803465in}}{\pgfqpoint{1.835653in}{0.795651in}}%
\pgfpathcurveto{\pgfqpoint{1.843467in}{0.787837in}}{\pgfqpoint{1.854066in}{0.783447in}}{\pgfqpoint{1.865116in}{0.783447in}}%
\pgfpathclose%
\pgfusepath{stroke,fill}%
\end{pgfscope}%
\begin{pgfscope}%
\pgfpathrectangle{\pgfqpoint{0.648703in}{0.548769in}}{\pgfqpoint{5.201297in}{3.102590in}}%
\pgfusepath{clip}%
\pgfsetbuttcap%
\pgfsetroundjoin%
\definecolor{currentfill}{rgb}{0.121569,0.466667,0.705882}%
\pgfsetfillcolor{currentfill}%
\pgfsetlinewidth{1.003750pt}%
\definecolor{currentstroke}{rgb}{0.121569,0.466667,0.705882}%
\pgfsetstrokecolor{currentstroke}%
\pgfsetdash{}{0pt}%
\pgfpathmoveto{\pgfqpoint{0.999458in}{0.652358in}}%
\pgfpathcurveto{\pgfqpoint{1.010508in}{0.652358in}}{\pgfqpoint{1.021107in}{0.656748in}}{\pgfqpoint{1.028921in}{0.664562in}}%
\pgfpathcurveto{\pgfqpoint{1.036735in}{0.672375in}}{\pgfqpoint{1.041125in}{0.682974in}}{\pgfqpoint{1.041125in}{0.694024in}}%
\pgfpathcurveto{\pgfqpoint{1.041125in}{0.705074in}}{\pgfqpoint{1.036735in}{0.715673in}}{\pgfqpoint{1.028921in}{0.723487in}}%
\pgfpathcurveto{\pgfqpoint{1.021107in}{0.731301in}}{\pgfqpoint{1.010508in}{0.735691in}}{\pgfqpoint{0.999458in}{0.735691in}}%
\pgfpathcurveto{\pgfqpoint{0.988408in}{0.735691in}}{\pgfqpoint{0.977809in}{0.731301in}}{\pgfqpoint{0.969995in}{0.723487in}}%
\pgfpathcurveto{\pgfqpoint{0.962182in}{0.715673in}}{\pgfqpoint{0.957791in}{0.705074in}}{\pgfqpoint{0.957791in}{0.694024in}}%
\pgfpathcurveto{\pgfqpoint{0.957791in}{0.682974in}}{\pgfqpoint{0.962182in}{0.672375in}}{\pgfqpoint{0.969995in}{0.664562in}}%
\pgfpathcurveto{\pgfqpoint{0.977809in}{0.656748in}}{\pgfqpoint{0.988408in}{0.652358in}}{\pgfqpoint{0.999458in}{0.652358in}}%
\pgfpathclose%
\pgfusepath{stroke,fill}%
\end{pgfscope}%
\begin{pgfscope}%
\pgfpathrectangle{\pgfqpoint{0.648703in}{0.548769in}}{\pgfqpoint{5.201297in}{3.102590in}}%
\pgfusepath{clip}%
\pgfsetbuttcap%
\pgfsetroundjoin%
\definecolor{currentfill}{rgb}{0.121569,0.466667,0.705882}%
\pgfsetfillcolor{currentfill}%
\pgfsetlinewidth{1.003750pt}%
\definecolor{currentstroke}{rgb}{0.121569,0.466667,0.705882}%
\pgfsetstrokecolor{currentstroke}%
\pgfsetdash{}{0pt}%
\pgfpathmoveto{\pgfqpoint{1.042843in}{0.648129in}}%
\pgfpathcurveto{\pgfqpoint{1.053893in}{0.648129in}}{\pgfqpoint{1.064492in}{0.652519in}}{\pgfqpoint{1.072306in}{0.660333in}}%
\pgfpathcurveto{\pgfqpoint{1.080120in}{0.668146in}}{\pgfqpoint{1.084510in}{0.678745in}}{\pgfqpoint{1.084510in}{0.689796in}}%
\pgfpathcurveto{\pgfqpoint{1.084510in}{0.700846in}}{\pgfqpoint{1.080120in}{0.711445in}}{\pgfqpoint{1.072306in}{0.719258in}}%
\pgfpathcurveto{\pgfqpoint{1.064492in}{0.727072in}}{\pgfqpoint{1.053893in}{0.731462in}}{\pgfqpoint{1.042843in}{0.731462in}}%
\pgfpathcurveto{\pgfqpoint{1.031793in}{0.731462in}}{\pgfqpoint{1.021194in}{0.727072in}}{\pgfqpoint{1.013380in}{0.719258in}}%
\pgfpathcurveto{\pgfqpoint{1.005567in}{0.711445in}}{\pgfqpoint{1.001176in}{0.700846in}}{\pgfqpoint{1.001176in}{0.689796in}}%
\pgfpathcurveto{\pgfqpoint{1.001176in}{0.678745in}}{\pgfqpoint{1.005567in}{0.668146in}}{\pgfqpoint{1.013380in}{0.660333in}}%
\pgfpathcurveto{\pgfqpoint{1.021194in}{0.652519in}}{\pgfqpoint{1.031793in}{0.648129in}}{\pgfqpoint{1.042843in}{0.648129in}}%
\pgfpathclose%
\pgfusepath{stroke,fill}%
\end{pgfscope}%
\begin{pgfscope}%
\pgfpathrectangle{\pgfqpoint{0.648703in}{0.548769in}}{\pgfqpoint{5.201297in}{3.102590in}}%
\pgfusepath{clip}%
\pgfsetbuttcap%
\pgfsetroundjoin%
\definecolor{currentfill}{rgb}{0.121569,0.466667,0.705882}%
\pgfsetfillcolor{currentfill}%
\pgfsetlinewidth{1.003750pt}%
\definecolor{currentstroke}{rgb}{0.121569,0.466667,0.705882}%
\pgfsetstrokecolor{currentstroke}%
\pgfsetdash{}{0pt}%
\pgfpathmoveto{\pgfqpoint{1.326121in}{0.648129in}}%
\pgfpathcurveto{\pgfqpoint{1.337172in}{0.648129in}}{\pgfqpoint{1.347771in}{0.652519in}}{\pgfqpoint{1.355584in}{0.660333in}}%
\pgfpathcurveto{\pgfqpoint{1.363398in}{0.668146in}}{\pgfqpoint{1.367788in}{0.678745in}}{\pgfqpoint{1.367788in}{0.689796in}}%
\pgfpathcurveto{\pgfqpoint{1.367788in}{0.700846in}}{\pgfqpoint{1.363398in}{0.711445in}}{\pgfqpoint{1.355584in}{0.719258in}}%
\pgfpathcurveto{\pgfqpoint{1.347771in}{0.727072in}}{\pgfqpoint{1.337172in}{0.731462in}}{\pgfqpoint{1.326121in}{0.731462in}}%
\pgfpathcurveto{\pgfqpoint{1.315071in}{0.731462in}}{\pgfqpoint{1.304472in}{0.727072in}}{\pgfqpoint{1.296659in}{0.719258in}}%
\pgfpathcurveto{\pgfqpoint{1.288845in}{0.711445in}}{\pgfqpoint{1.284455in}{0.700846in}}{\pgfqpoint{1.284455in}{0.689796in}}%
\pgfpathcurveto{\pgfqpoint{1.284455in}{0.678745in}}{\pgfqpoint{1.288845in}{0.668146in}}{\pgfqpoint{1.296659in}{0.660333in}}%
\pgfpathcurveto{\pgfqpoint{1.304472in}{0.652519in}}{\pgfqpoint{1.315071in}{0.648129in}}{\pgfqpoint{1.326121in}{0.648129in}}%
\pgfpathclose%
\pgfusepath{stroke,fill}%
\end{pgfscope}%
\begin{pgfscope}%
\pgfpathrectangle{\pgfqpoint{0.648703in}{0.548769in}}{\pgfqpoint{5.201297in}{3.102590in}}%
\pgfusepath{clip}%
\pgfsetbuttcap%
\pgfsetroundjoin%
\definecolor{currentfill}{rgb}{1.000000,0.498039,0.054902}%
\pgfsetfillcolor{currentfill}%
\pgfsetlinewidth{1.003750pt}%
\definecolor{currentstroke}{rgb}{1.000000,0.498039,0.054902}%
\pgfsetstrokecolor{currentstroke}%
\pgfsetdash{}{0pt}%
\pgfpathmoveto{\pgfqpoint{1.277122in}{3.193800in}}%
\pgfpathcurveto{\pgfqpoint{1.288172in}{3.193800in}}{\pgfqpoint{1.298771in}{3.198191in}}{\pgfqpoint{1.306585in}{3.206004in}}%
\pgfpathcurveto{\pgfqpoint{1.314398in}{3.213818in}}{\pgfqpoint{1.318789in}{3.224417in}}{\pgfqpoint{1.318789in}{3.235467in}}%
\pgfpathcurveto{\pgfqpoint{1.318789in}{3.246517in}}{\pgfqpoint{1.314398in}{3.257116in}}{\pgfqpoint{1.306585in}{3.264930in}}%
\pgfpathcurveto{\pgfqpoint{1.298771in}{3.272743in}}{\pgfqpoint{1.288172in}{3.277134in}}{\pgfqpoint{1.277122in}{3.277134in}}%
\pgfpathcurveto{\pgfqpoint{1.266072in}{3.277134in}}{\pgfqpoint{1.255473in}{3.272743in}}{\pgfqpoint{1.247659in}{3.264930in}}%
\pgfpathcurveto{\pgfqpoint{1.239846in}{3.257116in}}{\pgfqpoint{1.235455in}{3.246517in}}{\pgfqpoint{1.235455in}{3.235467in}}%
\pgfpathcurveto{\pgfqpoint{1.235455in}{3.224417in}}{\pgfqpoint{1.239846in}{3.213818in}}{\pgfqpoint{1.247659in}{3.206004in}}%
\pgfpathcurveto{\pgfqpoint{1.255473in}{3.198191in}}{\pgfqpoint{1.266072in}{3.193800in}}{\pgfqpoint{1.277122in}{3.193800in}}%
\pgfpathclose%
\pgfusepath{stroke,fill}%
\end{pgfscope}%
\begin{pgfscope}%
\pgfpathrectangle{\pgfqpoint{0.648703in}{0.548769in}}{\pgfqpoint{5.201297in}{3.102590in}}%
\pgfusepath{clip}%
\pgfsetbuttcap%
\pgfsetroundjoin%
\definecolor{currentfill}{rgb}{1.000000,0.498039,0.054902}%
\pgfsetfillcolor{currentfill}%
\pgfsetlinewidth{1.003750pt}%
\definecolor{currentstroke}{rgb}{1.000000,0.498039,0.054902}%
\pgfsetstrokecolor{currentstroke}%
\pgfsetdash{}{0pt}%
\pgfpathmoveto{\pgfqpoint{2.096114in}{3.231859in}}%
\pgfpathcurveto{\pgfqpoint{2.107164in}{3.231859in}}{\pgfqpoint{2.117763in}{3.236249in}}{\pgfqpoint{2.125576in}{3.244062in}}%
\pgfpathcurveto{\pgfqpoint{2.133390in}{3.251876in}}{\pgfqpoint{2.137780in}{3.262475in}}{\pgfqpoint{2.137780in}{3.273525in}}%
\pgfpathcurveto{\pgfqpoint{2.137780in}{3.284575in}}{\pgfqpoint{2.133390in}{3.295174in}}{\pgfqpoint{2.125576in}{3.302988in}}%
\pgfpathcurveto{\pgfqpoint{2.117763in}{3.310802in}}{\pgfqpoint{2.107164in}{3.315192in}}{\pgfqpoint{2.096114in}{3.315192in}}%
\pgfpathcurveto{\pgfqpoint{2.085063in}{3.315192in}}{\pgfqpoint{2.074464in}{3.310802in}}{\pgfqpoint{2.066651in}{3.302988in}}%
\pgfpathcurveto{\pgfqpoint{2.058837in}{3.295174in}}{\pgfqpoint{2.054447in}{3.284575in}}{\pgfqpoint{2.054447in}{3.273525in}}%
\pgfpathcurveto{\pgfqpoint{2.054447in}{3.262475in}}{\pgfqpoint{2.058837in}{3.251876in}}{\pgfqpoint{2.066651in}{3.244062in}}%
\pgfpathcurveto{\pgfqpoint{2.074464in}{3.236249in}}{\pgfqpoint{2.085063in}{3.231859in}}{\pgfqpoint{2.096114in}{3.231859in}}%
\pgfpathclose%
\pgfusepath{stroke,fill}%
\end{pgfscope}%
\begin{pgfscope}%
\pgfpathrectangle{\pgfqpoint{0.648703in}{0.548769in}}{\pgfqpoint{5.201297in}{3.102590in}}%
\pgfusepath{clip}%
\pgfsetbuttcap%
\pgfsetroundjoin%
\definecolor{currentfill}{rgb}{0.121569,0.466667,0.705882}%
\pgfsetfillcolor{currentfill}%
\pgfsetlinewidth{1.003750pt}%
\definecolor{currentstroke}{rgb}{0.121569,0.466667,0.705882}%
\pgfsetstrokecolor{currentstroke}%
\pgfsetdash{}{0pt}%
\pgfpathmoveto{\pgfqpoint{1.504301in}{0.758075in}}%
\pgfpathcurveto{\pgfqpoint{1.515352in}{0.758075in}}{\pgfqpoint{1.525951in}{0.762465in}}{\pgfqpoint{1.533764in}{0.770279in}}%
\pgfpathcurveto{\pgfqpoint{1.541578in}{0.778092in}}{\pgfqpoint{1.545968in}{0.788691in}}{\pgfqpoint{1.545968in}{0.799742in}}%
\pgfpathcurveto{\pgfqpoint{1.545968in}{0.810792in}}{\pgfqpoint{1.541578in}{0.821391in}}{\pgfqpoint{1.533764in}{0.829204in}}%
\pgfpathcurveto{\pgfqpoint{1.525951in}{0.837018in}}{\pgfqpoint{1.515352in}{0.841408in}}{\pgfqpoint{1.504301in}{0.841408in}}%
\pgfpathcurveto{\pgfqpoint{1.493251in}{0.841408in}}{\pgfqpoint{1.482652in}{0.837018in}}{\pgfqpoint{1.474839in}{0.829204in}}%
\pgfpathcurveto{\pgfqpoint{1.467025in}{0.821391in}}{\pgfqpoint{1.462635in}{0.810792in}}{\pgfqpoint{1.462635in}{0.799742in}}%
\pgfpathcurveto{\pgfqpoint{1.462635in}{0.788691in}}{\pgfqpoint{1.467025in}{0.778092in}}{\pgfqpoint{1.474839in}{0.770279in}}%
\pgfpathcurveto{\pgfqpoint{1.482652in}{0.762465in}}{\pgfqpoint{1.493251in}{0.758075in}}{\pgfqpoint{1.504301in}{0.758075in}}%
\pgfpathclose%
\pgfusepath{stroke,fill}%
\end{pgfscope}%
\begin{pgfscope}%
\pgfpathrectangle{\pgfqpoint{0.648703in}{0.548769in}}{\pgfqpoint{5.201297in}{3.102590in}}%
\pgfusepath{clip}%
\pgfsetbuttcap%
\pgfsetroundjoin%
\definecolor{currentfill}{rgb}{1.000000,0.498039,0.054902}%
\pgfsetfillcolor{currentfill}%
\pgfsetlinewidth{1.003750pt}%
\definecolor{currentstroke}{rgb}{1.000000,0.498039,0.054902}%
\pgfsetstrokecolor{currentstroke}%
\pgfsetdash{}{0pt}%
\pgfpathmoveto{\pgfqpoint{1.222965in}{3.185343in}}%
\pgfpathcurveto{\pgfqpoint{1.234015in}{3.185343in}}{\pgfqpoint{1.244614in}{3.189733in}}{\pgfqpoint{1.252427in}{3.197547in}}%
\pgfpathcurveto{\pgfqpoint{1.260241in}{3.205360in}}{\pgfqpoint{1.264631in}{3.215959in}}{\pgfqpoint{1.264631in}{3.227010in}}%
\pgfpathcurveto{\pgfqpoint{1.264631in}{3.238060in}}{\pgfqpoint{1.260241in}{3.248659in}}{\pgfqpoint{1.252427in}{3.256472in}}%
\pgfpathcurveto{\pgfqpoint{1.244614in}{3.264286in}}{\pgfqpoint{1.234015in}{3.268676in}}{\pgfqpoint{1.222965in}{3.268676in}}%
\pgfpathcurveto{\pgfqpoint{1.211914in}{3.268676in}}{\pgfqpoint{1.201315in}{3.264286in}}{\pgfqpoint{1.193502in}{3.256472in}}%
\pgfpathcurveto{\pgfqpoint{1.185688in}{3.248659in}}{\pgfqpoint{1.181298in}{3.238060in}}{\pgfqpoint{1.181298in}{3.227010in}}%
\pgfpathcurveto{\pgfqpoint{1.181298in}{3.215959in}}{\pgfqpoint{1.185688in}{3.205360in}}{\pgfqpoint{1.193502in}{3.197547in}}%
\pgfpathcurveto{\pgfqpoint{1.201315in}{3.189733in}}{\pgfqpoint{1.211914in}{3.185343in}}{\pgfqpoint{1.222965in}{3.185343in}}%
\pgfpathclose%
\pgfusepath{stroke,fill}%
\end{pgfscope}%
\begin{pgfscope}%
\pgfpathrectangle{\pgfqpoint{0.648703in}{0.548769in}}{\pgfqpoint{5.201297in}{3.102590in}}%
\pgfusepath{clip}%
\pgfsetbuttcap%
\pgfsetroundjoin%
\definecolor{currentfill}{rgb}{1.000000,0.498039,0.054902}%
\pgfsetfillcolor{currentfill}%
\pgfsetlinewidth{1.003750pt}%
\definecolor{currentstroke}{rgb}{1.000000,0.498039,0.054902}%
\pgfsetstrokecolor{currentstroke}%
\pgfsetdash{}{0pt}%
\pgfpathmoveto{\pgfqpoint{1.413621in}{3.202258in}}%
\pgfpathcurveto{\pgfqpoint{1.424671in}{3.202258in}}{\pgfqpoint{1.435270in}{3.206648in}}{\pgfqpoint{1.443083in}{3.214462in}}%
\pgfpathcurveto{\pgfqpoint{1.450897in}{3.222275in}}{\pgfqpoint{1.455287in}{3.232874in}}{\pgfqpoint{1.455287in}{3.243924in}}%
\pgfpathcurveto{\pgfqpoint{1.455287in}{3.254974in}}{\pgfqpoint{1.450897in}{3.265573in}}{\pgfqpoint{1.443083in}{3.273387in}}%
\pgfpathcurveto{\pgfqpoint{1.435270in}{3.281201in}}{\pgfqpoint{1.424671in}{3.285591in}}{\pgfqpoint{1.413621in}{3.285591in}}%
\pgfpathcurveto{\pgfqpoint{1.402570in}{3.285591in}}{\pgfqpoint{1.391971in}{3.281201in}}{\pgfqpoint{1.384158in}{3.273387in}}%
\pgfpathcurveto{\pgfqpoint{1.376344in}{3.265573in}}{\pgfqpoint{1.371954in}{3.254974in}}{\pgfqpoint{1.371954in}{3.243924in}}%
\pgfpathcurveto{\pgfqpoint{1.371954in}{3.232874in}}{\pgfqpoint{1.376344in}{3.222275in}}{\pgfqpoint{1.384158in}{3.214462in}}%
\pgfpathcurveto{\pgfqpoint{1.391971in}{3.206648in}}{\pgfqpoint{1.402570in}{3.202258in}}{\pgfqpoint{1.413621in}{3.202258in}}%
\pgfpathclose%
\pgfusepath{stroke,fill}%
\end{pgfscope}%
\begin{pgfscope}%
\pgfpathrectangle{\pgfqpoint{0.648703in}{0.548769in}}{\pgfqpoint{5.201297in}{3.102590in}}%
\pgfusepath{clip}%
\pgfsetbuttcap%
\pgfsetroundjoin%
\definecolor{currentfill}{rgb}{0.121569,0.466667,0.705882}%
\pgfsetfillcolor{currentfill}%
\pgfsetlinewidth{1.003750pt}%
\definecolor{currentstroke}{rgb}{0.121569,0.466667,0.705882}%
\pgfsetstrokecolor{currentstroke}%
\pgfsetdash{}{0pt}%
\pgfpathmoveto{\pgfqpoint{0.898239in}{0.648129in}}%
\pgfpathcurveto{\pgfqpoint{0.909289in}{0.648129in}}{\pgfqpoint{0.919888in}{0.652519in}}{\pgfqpoint{0.927701in}{0.660333in}}%
\pgfpathcurveto{\pgfqpoint{0.935515in}{0.668146in}}{\pgfqpoint{0.939905in}{0.678745in}}{\pgfqpoint{0.939905in}{0.689796in}}%
\pgfpathcurveto{\pgfqpoint{0.939905in}{0.700846in}}{\pgfqpoint{0.935515in}{0.711445in}}{\pgfqpoint{0.927701in}{0.719258in}}%
\pgfpathcurveto{\pgfqpoint{0.919888in}{0.727072in}}{\pgfqpoint{0.909289in}{0.731462in}}{\pgfqpoint{0.898239in}{0.731462in}}%
\pgfpathcurveto{\pgfqpoint{0.887188in}{0.731462in}}{\pgfqpoint{0.876589in}{0.727072in}}{\pgfqpoint{0.868776in}{0.719258in}}%
\pgfpathcurveto{\pgfqpoint{0.860962in}{0.711445in}}{\pgfqpoint{0.856572in}{0.700846in}}{\pgfqpoint{0.856572in}{0.689796in}}%
\pgfpathcurveto{\pgfqpoint{0.856572in}{0.678745in}}{\pgfqpoint{0.860962in}{0.668146in}}{\pgfqpoint{0.868776in}{0.660333in}}%
\pgfpathcurveto{\pgfqpoint{0.876589in}{0.652519in}}{\pgfqpoint{0.887188in}{0.648129in}}{\pgfqpoint{0.898239in}{0.648129in}}%
\pgfpathclose%
\pgfusepath{stroke,fill}%
\end{pgfscope}%
\begin{pgfscope}%
\pgfpathrectangle{\pgfqpoint{0.648703in}{0.548769in}}{\pgfqpoint{5.201297in}{3.102590in}}%
\pgfusepath{clip}%
\pgfsetbuttcap%
\pgfsetroundjoin%
\definecolor{currentfill}{rgb}{1.000000,0.498039,0.054902}%
\pgfsetfillcolor{currentfill}%
\pgfsetlinewidth{1.003750pt}%
\definecolor{currentstroke}{rgb}{1.000000,0.498039,0.054902}%
\pgfsetstrokecolor{currentstroke}%
\pgfsetdash{}{0pt}%
\pgfpathmoveto{\pgfqpoint{2.096114in}{3.206486in}}%
\pgfpathcurveto{\pgfqpoint{2.107164in}{3.206486in}}{\pgfqpoint{2.117763in}{3.210877in}}{\pgfqpoint{2.125576in}{3.218690in}}%
\pgfpathcurveto{\pgfqpoint{2.133390in}{3.226504in}}{\pgfqpoint{2.137780in}{3.237103in}}{\pgfqpoint{2.137780in}{3.248153in}}%
\pgfpathcurveto{\pgfqpoint{2.137780in}{3.259203in}}{\pgfqpoint{2.133390in}{3.269802in}}{\pgfqpoint{2.125576in}{3.277616in}}%
\pgfpathcurveto{\pgfqpoint{2.117763in}{3.285429in}}{\pgfqpoint{2.107164in}{3.289820in}}{\pgfqpoint{2.096114in}{3.289820in}}%
\pgfpathcurveto{\pgfqpoint{2.085063in}{3.289820in}}{\pgfqpoint{2.074464in}{3.285429in}}{\pgfqpoint{2.066651in}{3.277616in}}%
\pgfpathcurveto{\pgfqpoint{2.058837in}{3.269802in}}{\pgfqpoint{2.054447in}{3.259203in}}{\pgfqpoint{2.054447in}{3.248153in}}%
\pgfpathcurveto{\pgfqpoint{2.054447in}{3.237103in}}{\pgfqpoint{2.058837in}{3.226504in}}{\pgfqpoint{2.066651in}{3.218690in}}%
\pgfpathcurveto{\pgfqpoint{2.074464in}{3.210877in}}{\pgfqpoint{2.085063in}{3.206486in}}{\pgfqpoint{2.096114in}{3.206486in}}%
\pgfpathclose%
\pgfusepath{stroke,fill}%
\end{pgfscope}%
\begin{pgfscope}%
\pgfpathrectangle{\pgfqpoint{0.648703in}{0.548769in}}{\pgfqpoint{5.201297in}{3.102590in}}%
\pgfusepath{clip}%
\pgfsetbuttcap%
\pgfsetroundjoin%
\definecolor{currentfill}{rgb}{0.121569,0.466667,0.705882}%
\pgfsetfillcolor{currentfill}%
\pgfsetlinewidth{1.003750pt}%
\definecolor{currentstroke}{rgb}{0.121569,0.466667,0.705882}%
\pgfsetstrokecolor{currentstroke}%
\pgfsetdash{}{0pt}%
\pgfpathmoveto{\pgfqpoint{3.041104in}{0.648129in}}%
\pgfpathcurveto{\pgfqpoint{3.052154in}{0.648129in}}{\pgfqpoint{3.062753in}{0.652519in}}{\pgfqpoint{3.070567in}{0.660333in}}%
\pgfpathcurveto{\pgfqpoint{3.078380in}{0.668146in}}{\pgfqpoint{3.082771in}{0.678745in}}{\pgfqpoint{3.082771in}{0.689796in}}%
\pgfpathcurveto{\pgfqpoint{3.082771in}{0.700846in}}{\pgfqpoint{3.078380in}{0.711445in}}{\pgfqpoint{3.070567in}{0.719258in}}%
\pgfpathcurveto{\pgfqpoint{3.062753in}{0.727072in}}{\pgfqpoint{3.052154in}{0.731462in}}{\pgfqpoint{3.041104in}{0.731462in}}%
\pgfpathcurveto{\pgfqpoint{3.030054in}{0.731462in}}{\pgfqpoint{3.019455in}{0.727072in}}{\pgfqpoint{3.011641in}{0.719258in}}%
\pgfpathcurveto{\pgfqpoint{3.003827in}{0.711445in}}{\pgfqpoint{2.999437in}{0.700846in}}{\pgfqpoint{2.999437in}{0.689796in}}%
\pgfpathcurveto{\pgfqpoint{2.999437in}{0.678745in}}{\pgfqpoint{3.003827in}{0.668146in}}{\pgfqpoint{3.011641in}{0.660333in}}%
\pgfpathcurveto{\pgfqpoint{3.019455in}{0.652519in}}{\pgfqpoint{3.030054in}{0.648129in}}{\pgfqpoint{3.041104in}{0.648129in}}%
\pgfpathclose%
\pgfusepath{stroke,fill}%
\end{pgfscope}%
\begin{pgfscope}%
\pgfpathrectangle{\pgfqpoint{0.648703in}{0.548769in}}{\pgfqpoint{5.201297in}{3.102590in}}%
\pgfusepath{clip}%
\pgfsetbuttcap%
\pgfsetroundjoin%
\definecolor{currentfill}{rgb}{0.121569,0.466667,0.705882}%
\pgfsetfillcolor{currentfill}%
\pgfsetlinewidth{1.003750pt}%
\definecolor{currentstroke}{rgb}{0.121569,0.466667,0.705882}%
\pgfsetstrokecolor{currentstroke}%
\pgfsetdash{}{0pt}%
\pgfpathmoveto{\pgfqpoint{0.897376in}{0.648129in}}%
\pgfpathcurveto{\pgfqpoint{0.908426in}{0.648129in}}{\pgfqpoint{0.919025in}{0.652519in}}{\pgfqpoint{0.926839in}{0.660333in}}%
\pgfpathcurveto{\pgfqpoint{0.934652in}{0.668146in}}{\pgfqpoint{0.939043in}{0.678745in}}{\pgfqpoint{0.939043in}{0.689796in}}%
\pgfpathcurveto{\pgfqpoint{0.939043in}{0.700846in}}{\pgfqpoint{0.934652in}{0.711445in}}{\pgfqpoint{0.926839in}{0.719258in}}%
\pgfpathcurveto{\pgfqpoint{0.919025in}{0.727072in}}{\pgfqpoint{0.908426in}{0.731462in}}{\pgfqpoint{0.897376in}{0.731462in}}%
\pgfpathcurveto{\pgfqpoint{0.886326in}{0.731462in}}{\pgfqpoint{0.875727in}{0.727072in}}{\pgfqpoint{0.867913in}{0.719258in}}%
\pgfpathcurveto{\pgfqpoint{0.860099in}{0.711445in}}{\pgfqpoint{0.855709in}{0.700846in}}{\pgfqpoint{0.855709in}{0.689796in}}%
\pgfpathcurveto{\pgfqpoint{0.855709in}{0.678745in}}{\pgfqpoint{0.860099in}{0.668146in}}{\pgfqpoint{0.867913in}{0.660333in}}%
\pgfpathcurveto{\pgfqpoint{0.875727in}{0.652519in}}{\pgfqpoint{0.886326in}{0.648129in}}{\pgfqpoint{0.897376in}{0.648129in}}%
\pgfpathclose%
\pgfusepath{stroke,fill}%
\end{pgfscope}%
\begin{pgfscope}%
\pgfpathrectangle{\pgfqpoint{0.648703in}{0.548769in}}{\pgfqpoint{5.201297in}{3.102590in}}%
\pgfusepath{clip}%
\pgfsetbuttcap%
\pgfsetroundjoin%
\definecolor{currentfill}{rgb}{0.121569,0.466667,0.705882}%
\pgfsetfillcolor{currentfill}%
\pgfsetlinewidth{1.003750pt}%
\definecolor{currentstroke}{rgb}{0.121569,0.466667,0.705882}%
\pgfsetstrokecolor{currentstroke}%
\pgfsetdash{}{0pt}%
\pgfpathmoveto{\pgfqpoint{1.734451in}{0.817277in}}%
\pgfpathcurveto{\pgfqpoint{1.745501in}{0.817277in}}{\pgfqpoint{1.756100in}{0.821667in}}{\pgfqpoint{1.763913in}{0.829480in}}%
\pgfpathcurveto{\pgfqpoint{1.771727in}{0.837294in}}{\pgfqpoint{1.776117in}{0.847893in}}{\pgfqpoint{1.776117in}{0.858943in}}%
\pgfpathcurveto{\pgfqpoint{1.776117in}{0.869993in}}{\pgfqpoint{1.771727in}{0.880592in}}{\pgfqpoint{1.763913in}{0.888406in}}%
\pgfpathcurveto{\pgfqpoint{1.756100in}{0.896220in}}{\pgfqpoint{1.745501in}{0.900610in}}{\pgfqpoint{1.734451in}{0.900610in}}%
\pgfpathcurveto{\pgfqpoint{1.723400in}{0.900610in}}{\pgfqpoint{1.712801in}{0.896220in}}{\pgfqpoint{1.704988in}{0.888406in}}%
\pgfpathcurveto{\pgfqpoint{1.697174in}{0.880592in}}{\pgfqpoint{1.692784in}{0.869993in}}{\pgfqpoint{1.692784in}{0.858943in}}%
\pgfpathcurveto{\pgfqpoint{1.692784in}{0.847893in}}{\pgfqpoint{1.697174in}{0.837294in}}{\pgfqpoint{1.704988in}{0.829480in}}%
\pgfpathcurveto{\pgfqpoint{1.712801in}{0.821667in}}{\pgfqpoint{1.723400in}{0.817277in}}{\pgfqpoint{1.734451in}{0.817277in}}%
\pgfpathclose%
\pgfusepath{stroke,fill}%
\end{pgfscope}%
\begin{pgfscope}%
\pgfpathrectangle{\pgfqpoint{0.648703in}{0.548769in}}{\pgfqpoint{5.201297in}{3.102590in}}%
\pgfusepath{clip}%
\pgfsetbuttcap%
\pgfsetroundjoin%
\definecolor{currentfill}{rgb}{1.000000,0.498039,0.054902}%
\pgfsetfillcolor{currentfill}%
\pgfsetlinewidth{1.003750pt}%
\definecolor{currentstroke}{rgb}{1.000000,0.498039,0.054902}%
\pgfsetstrokecolor{currentstroke}%
\pgfsetdash{}{0pt}%
\pgfpathmoveto{\pgfqpoint{1.636452in}{3.189572in}}%
\pgfpathcurveto{\pgfqpoint{1.647502in}{3.189572in}}{\pgfqpoint{1.658101in}{3.193962in}}{\pgfqpoint{1.665914in}{3.201775in}}%
\pgfpathcurveto{\pgfqpoint{1.673728in}{3.209589in}}{\pgfqpoint{1.678118in}{3.220188in}}{\pgfqpoint{1.678118in}{3.231238in}}%
\pgfpathcurveto{\pgfqpoint{1.678118in}{3.242288in}}{\pgfqpoint{1.673728in}{3.252887in}}{\pgfqpoint{1.665914in}{3.260701in}}%
\pgfpathcurveto{\pgfqpoint{1.658101in}{3.268515in}}{\pgfqpoint{1.647502in}{3.272905in}}{\pgfqpoint{1.636452in}{3.272905in}}%
\pgfpathcurveto{\pgfqpoint{1.625401in}{3.272905in}}{\pgfqpoint{1.614802in}{3.268515in}}{\pgfqpoint{1.606989in}{3.260701in}}%
\pgfpathcurveto{\pgfqpoint{1.599175in}{3.252887in}}{\pgfqpoint{1.594785in}{3.242288in}}{\pgfqpoint{1.594785in}{3.231238in}}%
\pgfpathcurveto{\pgfqpoint{1.594785in}{3.220188in}}{\pgfqpoint{1.599175in}{3.209589in}}{\pgfqpoint{1.606989in}{3.201775in}}%
\pgfpathcurveto{\pgfqpoint{1.614802in}{3.193962in}}{\pgfqpoint{1.625401in}{3.189572in}}{\pgfqpoint{1.636452in}{3.189572in}}%
\pgfpathclose%
\pgfusepath{stroke,fill}%
\end{pgfscope}%
\begin{pgfscope}%
\pgfpathrectangle{\pgfqpoint{0.648703in}{0.548769in}}{\pgfqpoint{5.201297in}{3.102590in}}%
\pgfusepath{clip}%
\pgfsetbuttcap%
\pgfsetroundjoin%
\definecolor{currentfill}{rgb}{1.000000,0.498039,0.054902}%
\pgfsetfillcolor{currentfill}%
\pgfsetlinewidth{1.003750pt}%
\definecolor{currentstroke}{rgb}{1.000000,0.498039,0.054902}%
\pgfsetstrokecolor{currentstroke}%
\pgfsetdash{}{0pt}%
\pgfpathmoveto{\pgfqpoint{1.040291in}{3.214944in}}%
\pgfpathcurveto{\pgfqpoint{1.051341in}{3.214944in}}{\pgfqpoint{1.061940in}{3.219334in}}{\pgfqpoint{1.069754in}{3.227148in}}%
\pgfpathcurveto{\pgfqpoint{1.077567in}{3.234961in}}{\pgfqpoint{1.081958in}{3.245560in}}{\pgfqpoint{1.081958in}{3.256610in}}%
\pgfpathcurveto{\pgfqpoint{1.081958in}{3.267661in}}{\pgfqpoint{1.077567in}{3.278260in}}{\pgfqpoint{1.069754in}{3.286073in}}%
\pgfpathcurveto{\pgfqpoint{1.061940in}{3.293887in}}{\pgfqpoint{1.051341in}{3.298277in}}{\pgfqpoint{1.040291in}{3.298277in}}%
\pgfpathcurveto{\pgfqpoint{1.029241in}{3.298277in}}{\pgfqpoint{1.018642in}{3.293887in}}{\pgfqpoint{1.010828in}{3.286073in}}%
\pgfpathcurveto{\pgfqpoint{1.003015in}{3.278260in}}{\pgfqpoint{0.998624in}{3.267661in}}{\pgfqpoint{0.998624in}{3.256610in}}%
\pgfpathcurveto{\pgfqpoint{0.998624in}{3.245560in}}{\pgfqpoint{1.003015in}{3.234961in}}{\pgfqpoint{1.010828in}{3.227148in}}%
\pgfpathcurveto{\pgfqpoint{1.018642in}{3.219334in}}{\pgfqpoint{1.029241in}{3.214944in}}{\pgfqpoint{1.040291in}{3.214944in}}%
\pgfpathclose%
\pgfusepath{stroke,fill}%
\end{pgfscope}%
\begin{pgfscope}%
\pgfpathrectangle{\pgfqpoint{0.648703in}{0.548769in}}{\pgfqpoint{5.201297in}{3.102590in}}%
\pgfusepath{clip}%
\pgfsetbuttcap%
\pgfsetroundjoin%
\definecolor{currentfill}{rgb}{0.121569,0.466667,0.705882}%
\pgfsetfillcolor{currentfill}%
\pgfsetlinewidth{1.003750pt}%
\definecolor{currentstroke}{rgb}{0.121569,0.466667,0.705882}%
\pgfsetstrokecolor{currentstroke}%
\pgfsetdash{}{0pt}%
\pgfpathmoveto{\pgfqpoint{1.046124in}{0.648129in}}%
\pgfpathcurveto{\pgfqpoint{1.057174in}{0.648129in}}{\pgfqpoint{1.067773in}{0.652519in}}{\pgfqpoint{1.075587in}{0.660333in}}%
\pgfpathcurveto{\pgfqpoint{1.083401in}{0.668146in}}{\pgfqpoint{1.087791in}{0.678745in}}{\pgfqpoint{1.087791in}{0.689796in}}%
\pgfpathcurveto{\pgfqpoint{1.087791in}{0.700846in}}{\pgfqpoint{1.083401in}{0.711445in}}{\pgfqpoint{1.075587in}{0.719258in}}%
\pgfpathcurveto{\pgfqpoint{1.067773in}{0.727072in}}{\pgfqpoint{1.057174in}{0.731462in}}{\pgfqpoint{1.046124in}{0.731462in}}%
\pgfpathcurveto{\pgfqpoint{1.035074in}{0.731462in}}{\pgfqpoint{1.024475in}{0.727072in}}{\pgfqpoint{1.016662in}{0.719258in}}%
\pgfpathcurveto{\pgfqpoint{1.008848in}{0.711445in}}{\pgfqpoint{1.004458in}{0.700846in}}{\pgfqpoint{1.004458in}{0.689796in}}%
\pgfpathcurveto{\pgfqpoint{1.004458in}{0.678745in}}{\pgfqpoint{1.008848in}{0.668146in}}{\pgfqpoint{1.016662in}{0.660333in}}%
\pgfpathcurveto{\pgfqpoint{1.024475in}{0.652519in}}{\pgfqpoint{1.035074in}{0.648129in}}{\pgfqpoint{1.046124in}{0.648129in}}%
\pgfpathclose%
\pgfusepath{stroke,fill}%
\end{pgfscope}%
\begin{pgfscope}%
\pgfpathrectangle{\pgfqpoint{0.648703in}{0.548769in}}{\pgfqpoint{5.201297in}{3.102590in}}%
\pgfusepath{clip}%
\pgfsetbuttcap%
\pgfsetroundjoin%
\definecolor{currentfill}{rgb}{0.121569,0.466667,0.705882}%
\pgfsetfillcolor{currentfill}%
\pgfsetlinewidth{1.003750pt}%
\definecolor{currentstroke}{rgb}{0.121569,0.466667,0.705882}%
\pgfsetstrokecolor{currentstroke}%
\pgfsetdash{}{0pt}%
\pgfpathmoveto{\pgfqpoint{1.865116in}{0.758075in}}%
\pgfpathcurveto{\pgfqpoint{1.876166in}{0.758075in}}{\pgfqpoint{1.886765in}{0.762465in}}{\pgfqpoint{1.894579in}{0.770279in}}%
\pgfpathcurveto{\pgfqpoint{1.902392in}{0.778092in}}{\pgfqpoint{1.906783in}{0.788691in}}{\pgfqpoint{1.906783in}{0.799742in}}%
\pgfpathcurveto{\pgfqpoint{1.906783in}{0.810792in}}{\pgfqpoint{1.902392in}{0.821391in}}{\pgfqpoint{1.894579in}{0.829204in}}%
\pgfpathcurveto{\pgfqpoint{1.886765in}{0.837018in}}{\pgfqpoint{1.876166in}{0.841408in}}{\pgfqpoint{1.865116in}{0.841408in}}%
\pgfpathcurveto{\pgfqpoint{1.854066in}{0.841408in}}{\pgfqpoint{1.843467in}{0.837018in}}{\pgfqpoint{1.835653in}{0.829204in}}%
\pgfpathcurveto{\pgfqpoint{1.827840in}{0.821391in}}{\pgfqpoint{1.823449in}{0.810792in}}{\pgfqpoint{1.823449in}{0.799742in}}%
\pgfpathcurveto{\pgfqpoint{1.823449in}{0.788691in}}{\pgfqpoint{1.827840in}{0.778092in}}{\pgfqpoint{1.835653in}{0.770279in}}%
\pgfpathcurveto{\pgfqpoint{1.843467in}{0.762465in}}{\pgfqpoint{1.854066in}{0.758075in}}{\pgfqpoint{1.865116in}{0.758075in}}%
\pgfpathclose%
\pgfusepath{stroke,fill}%
\end{pgfscope}%
\begin{pgfscope}%
\pgfpathrectangle{\pgfqpoint{0.648703in}{0.548769in}}{\pgfqpoint{5.201297in}{3.102590in}}%
\pgfusepath{clip}%
\pgfsetbuttcap%
\pgfsetroundjoin%
\definecolor{currentfill}{rgb}{1.000000,0.498039,0.054902}%
\pgfsetfillcolor{currentfill}%
\pgfsetlinewidth{1.003750pt}%
\definecolor{currentstroke}{rgb}{1.000000,0.498039,0.054902}%
\pgfsetstrokecolor{currentstroke}%
\pgfsetdash{}{0pt}%
\pgfpathmoveto{\pgfqpoint{1.816116in}{3.193800in}}%
\pgfpathcurveto{\pgfqpoint{1.827167in}{3.193800in}}{\pgfqpoint{1.837766in}{3.198191in}}{\pgfqpoint{1.845579in}{3.206004in}}%
\pgfpathcurveto{\pgfqpoint{1.853393in}{3.213818in}}{\pgfqpoint{1.857783in}{3.224417in}}{\pgfqpoint{1.857783in}{3.235467in}}%
\pgfpathcurveto{\pgfqpoint{1.857783in}{3.246517in}}{\pgfqpoint{1.853393in}{3.257116in}}{\pgfqpoint{1.845579in}{3.264930in}}%
\pgfpathcurveto{\pgfqpoint{1.837766in}{3.272743in}}{\pgfqpoint{1.827167in}{3.277134in}}{\pgfqpoint{1.816116in}{3.277134in}}%
\pgfpathcurveto{\pgfqpoint{1.805066in}{3.277134in}}{\pgfqpoint{1.794467in}{3.272743in}}{\pgfqpoint{1.786654in}{3.264930in}}%
\pgfpathcurveto{\pgfqpoint{1.778840in}{3.257116in}}{\pgfqpoint{1.774450in}{3.246517in}}{\pgfqpoint{1.774450in}{3.235467in}}%
\pgfpathcurveto{\pgfqpoint{1.774450in}{3.224417in}}{\pgfqpoint{1.778840in}{3.213818in}}{\pgfqpoint{1.786654in}{3.206004in}}%
\pgfpathcurveto{\pgfqpoint{1.794467in}{3.198191in}}{\pgfqpoint{1.805066in}{3.193800in}}{\pgfqpoint{1.816116in}{3.193800in}}%
\pgfpathclose%
\pgfusepath{stroke,fill}%
\end{pgfscope}%
\begin{pgfscope}%
\pgfpathrectangle{\pgfqpoint{0.648703in}{0.548769in}}{\pgfqpoint{5.201297in}{3.102590in}}%
\pgfusepath{clip}%
\pgfsetbuttcap%
\pgfsetroundjoin%
\definecolor{currentfill}{rgb}{0.121569,0.466667,0.705882}%
\pgfsetfillcolor{currentfill}%
\pgfsetlinewidth{1.003750pt}%
\definecolor{currentstroke}{rgb}{0.121569,0.466667,0.705882}%
\pgfsetstrokecolor{currentstroke}%
\pgfsetdash{}{0pt}%
\pgfpathmoveto{\pgfqpoint{0.983125in}{0.648129in}}%
\pgfpathcurveto{\pgfqpoint{0.994175in}{0.648129in}}{\pgfqpoint{1.004774in}{0.652519in}}{\pgfqpoint{1.012588in}{0.660333in}}%
\pgfpathcurveto{\pgfqpoint{1.020401in}{0.668146in}}{\pgfqpoint{1.024792in}{0.678745in}}{\pgfqpoint{1.024792in}{0.689796in}}%
\pgfpathcurveto{\pgfqpoint{1.024792in}{0.700846in}}{\pgfqpoint{1.020401in}{0.711445in}}{\pgfqpoint{1.012588in}{0.719258in}}%
\pgfpathcurveto{\pgfqpoint{1.004774in}{0.727072in}}{\pgfqpoint{0.994175in}{0.731462in}}{\pgfqpoint{0.983125in}{0.731462in}}%
\pgfpathcurveto{\pgfqpoint{0.972075in}{0.731462in}}{\pgfqpoint{0.961476in}{0.727072in}}{\pgfqpoint{0.953662in}{0.719258in}}%
\pgfpathcurveto{\pgfqpoint{0.945849in}{0.711445in}}{\pgfqpoint{0.941458in}{0.700846in}}{\pgfqpoint{0.941458in}{0.689796in}}%
\pgfpathcurveto{\pgfqpoint{0.941458in}{0.678745in}}{\pgfqpoint{0.945849in}{0.668146in}}{\pgfqpoint{0.953662in}{0.660333in}}%
\pgfpathcurveto{\pgfqpoint{0.961476in}{0.652519in}}{\pgfqpoint{0.972075in}{0.648129in}}{\pgfqpoint{0.983125in}{0.648129in}}%
\pgfpathclose%
\pgfusepath{stroke,fill}%
\end{pgfscope}%
\begin{pgfscope}%
\pgfpathrectangle{\pgfqpoint{0.648703in}{0.548769in}}{\pgfqpoint{5.201297in}{3.102590in}}%
\pgfusepath{clip}%
\pgfsetbuttcap%
\pgfsetroundjoin%
\definecolor{currentfill}{rgb}{1.000000,0.498039,0.054902}%
\pgfsetfillcolor{currentfill}%
\pgfsetlinewidth{1.003750pt}%
\definecolor{currentstroke}{rgb}{1.000000,0.498039,0.054902}%
\pgfsetstrokecolor{currentstroke}%
\pgfsetdash{}{0pt}%
\pgfpathmoveto{\pgfqpoint{1.455068in}{3.198029in}}%
\pgfpathcurveto{\pgfqpoint{1.466118in}{3.198029in}}{\pgfqpoint{1.476717in}{3.202419in}}{\pgfqpoint{1.484530in}{3.210233in}}%
\pgfpathcurveto{\pgfqpoint{1.492344in}{3.218046in}}{\pgfqpoint{1.496734in}{3.228646in}}{\pgfqpoint{1.496734in}{3.239696in}}%
\pgfpathcurveto{\pgfqpoint{1.496734in}{3.250746in}}{\pgfqpoint{1.492344in}{3.261345in}}{\pgfqpoint{1.484530in}{3.269158in}}%
\pgfpathcurveto{\pgfqpoint{1.476717in}{3.276972in}}{\pgfqpoint{1.466118in}{3.281362in}}{\pgfqpoint{1.455068in}{3.281362in}}%
\pgfpathcurveto{\pgfqpoint{1.444017in}{3.281362in}}{\pgfqpoint{1.433418in}{3.276972in}}{\pgfqpoint{1.425605in}{3.269158in}}%
\pgfpathcurveto{\pgfqpoint{1.417791in}{3.261345in}}{\pgfqpoint{1.413401in}{3.250746in}}{\pgfqpoint{1.413401in}{3.239696in}}%
\pgfpathcurveto{\pgfqpoint{1.413401in}{3.228646in}}{\pgfqpoint{1.417791in}{3.218046in}}{\pgfqpoint{1.425605in}{3.210233in}}%
\pgfpathcurveto{\pgfqpoint{1.433418in}{3.202419in}}{\pgfqpoint{1.444017in}{3.198029in}}{\pgfqpoint{1.455068in}{3.198029in}}%
\pgfpathclose%
\pgfusepath{stroke,fill}%
\end{pgfscope}%
\begin{pgfscope}%
\pgfpathrectangle{\pgfqpoint{0.648703in}{0.548769in}}{\pgfqpoint{5.201297in}{3.102590in}}%
\pgfusepath{clip}%
\pgfsetbuttcap%
\pgfsetroundjoin%
\definecolor{currentfill}{rgb}{1.000000,0.498039,0.054902}%
\pgfsetfillcolor{currentfill}%
\pgfsetlinewidth{1.003750pt}%
\definecolor{currentstroke}{rgb}{1.000000,0.498039,0.054902}%
\pgfsetstrokecolor{currentstroke}%
\pgfsetdash{}{0pt}%
\pgfpathmoveto{\pgfqpoint{1.441414in}{3.193800in}}%
\pgfpathcurveto{\pgfqpoint{1.452465in}{3.193800in}}{\pgfqpoint{1.463064in}{3.198191in}}{\pgfqpoint{1.470877in}{3.206004in}}%
\pgfpathcurveto{\pgfqpoint{1.478691in}{3.213818in}}{\pgfqpoint{1.483081in}{3.224417in}}{\pgfqpoint{1.483081in}{3.235467in}}%
\pgfpathcurveto{\pgfqpoint{1.483081in}{3.246517in}}{\pgfqpoint{1.478691in}{3.257116in}}{\pgfqpoint{1.470877in}{3.264930in}}%
\pgfpathcurveto{\pgfqpoint{1.463064in}{3.272743in}}{\pgfqpoint{1.452465in}{3.277134in}}{\pgfqpoint{1.441414in}{3.277134in}}%
\pgfpathcurveto{\pgfqpoint{1.430364in}{3.277134in}}{\pgfqpoint{1.419765in}{3.272743in}}{\pgfqpoint{1.411952in}{3.264930in}}%
\pgfpathcurveto{\pgfqpoint{1.404138in}{3.257116in}}{\pgfqpoint{1.399748in}{3.246517in}}{\pgfqpoint{1.399748in}{3.235467in}}%
\pgfpathcurveto{\pgfqpoint{1.399748in}{3.224417in}}{\pgfqpoint{1.404138in}{3.213818in}}{\pgfqpoint{1.411952in}{3.206004in}}%
\pgfpathcurveto{\pgfqpoint{1.419765in}{3.198191in}}{\pgfqpoint{1.430364in}{3.193800in}}{\pgfqpoint{1.441414in}{3.193800in}}%
\pgfpathclose%
\pgfusepath{stroke,fill}%
\end{pgfscope}%
\begin{pgfscope}%
\pgfpathrectangle{\pgfqpoint{0.648703in}{0.548769in}}{\pgfqpoint{5.201297in}{3.102590in}}%
\pgfusepath{clip}%
\pgfsetbuttcap%
\pgfsetroundjoin%
\definecolor{currentfill}{rgb}{0.121569,0.466667,0.705882}%
\pgfsetfillcolor{currentfill}%
\pgfsetlinewidth{1.003750pt}%
\definecolor{currentstroke}{rgb}{0.121569,0.466667,0.705882}%
\pgfsetstrokecolor{currentstroke}%
\pgfsetdash{}{0pt}%
\pgfpathmoveto{\pgfqpoint{1.064228in}{0.652358in}}%
\pgfpathcurveto{\pgfqpoint{1.075278in}{0.652358in}}{\pgfqpoint{1.085877in}{0.656748in}}{\pgfqpoint{1.093690in}{0.664562in}}%
\pgfpathcurveto{\pgfqpoint{1.101504in}{0.672375in}}{\pgfqpoint{1.105894in}{0.682974in}}{\pgfqpoint{1.105894in}{0.694024in}}%
\pgfpathcurveto{\pgfqpoint{1.105894in}{0.705074in}}{\pgfqpoint{1.101504in}{0.715673in}}{\pgfqpoint{1.093690in}{0.723487in}}%
\pgfpathcurveto{\pgfqpoint{1.085877in}{0.731301in}}{\pgfqpoint{1.075278in}{0.735691in}}{\pgfqpoint{1.064228in}{0.735691in}}%
\pgfpathcurveto{\pgfqpoint{1.053177in}{0.735691in}}{\pgfqpoint{1.042578in}{0.731301in}}{\pgfqpoint{1.034765in}{0.723487in}}%
\pgfpathcurveto{\pgfqpoint{1.026951in}{0.715673in}}{\pgfqpoint{1.022561in}{0.705074in}}{\pgfqpoint{1.022561in}{0.694024in}}%
\pgfpathcurveto{\pgfqpoint{1.022561in}{0.682974in}}{\pgfqpoint{1.026951in}{0.672375in}}{\pgfqpoint{1.034765in}{0.664562in}}%
\pgfpathcurveto{\pgfqpoint{1.042578in}{0.656748in}}{\pgfqpoint{1.053177in}{0.652358in}}{\pgfqpoint{1.064228in}{0.652358in}}%
\pgfpathclose%
\pgfusepath{stroke,fill}%
\end{pgfscope}%
\begin{pgfscope}%
\pgfpathrectangle{\pgfqpoint{0.648703in}{0.548769in}}{\pgfqpoint{5.201297in}{3.102590in}}%
\pgfusepath{clip}%
\pgfsetbuttcap%
\pgfsetroundjoin%
\definecolor{currentfill}{rgb}{0.121569,0.466667,0.705882}%
\pgfsetfillcolor{currentfill}%
\pgfsetlinewidth{1.003750pt}%
\definecolor{currentstroke}{rgb}{0.121569,0.466667,0.705882}%
\pgfsetstrokecolor{currentstroke}%
\pgfsetdash{}{0pt}%
\pgfpathmoveto{\pgfqpoint{1.130123in}{0.648129in}}%
\pgfpathcurveto{\pgfqpoint{1.141174in}{0.648129in}}{\pgfqpoint{1.151773in}{0.652519in}}{\pgfqpoint{1.159586in}{0.660333in}}%
\pgfpathcurveto{\pgfqpoint{1.167400in}{0.668146in}}{\pgfqpoint{1.171790in}{0.678745in}}{\pgfqpoint{1.171790in}{0.689796in}}%
\pgfpathcurveto{\pgfqpoint{1.171790in}{0.700846in}}{\pgfqpoint{1.167400in}{0.711445in}}{\pgfqpoint{1.159586in}{0.719258in}}%
\pgfpathcurveto{\pgfqpoint{1.151773in}{0.727072in}}{\pgfqpoint{1.141174in}{0.731462in}}{\pgfqpoint{1.130123in}{0.731462in}}%
\pgfpathcurveto{\pgfqpoint{1.119073in}{0.731462in}}{\pgfqpoint{1.108474in}{0.727072in}}{\pgfqpoint{1.100661in}{0.719258in}}%
\pgfpathcurveto{\pgfqpoint{1.092847in}{0.711445in}}{\pgfqpoint{1.088457in}{0.700846in}}{\pgfqpoint{1.088457in}{0.689796in}}%
\pgfpathcurveto{\pgfqpoint{1.088457in}{0.678745in}}{\pgfqpoint{1.092847in}{0.668146in}}{\pgfqpoint{1.100661in}{0.660333in}}%
\pgfpathcurveto{\pgfqpoint{1.108474in}{0.652519in}}{\pgfqpoint{1.119073in}{0.648129in}}{\pgfqpoint{1.130123in}{0.648129in}}%
\pgfpathclose%
\pgfusepath{stroke,fill}%
\end{pgfscope}%
\begin{pgfscope}%
\pgfpathrectangle{\pgfqpoint{0.648703in}{0.548769in}}{\pgfqpoint{5.201297in}{3.102590in}}%
\pgfusepath{clip}%
\pgfsetbuttcap%
\pgfsetroundjoin%
\definecolor{currentfill}{rgb}{0.121569,0.466667,0.705882}%
\pgfsetfillcolor{currentfill}%
\pgfsetlinewidth{1.003750pt}%
\definecolor{currentstroke}{rgb}{0.121569,0.466667,0.705882}%
\pgfsetstrokecolor{currentstroke}%
\pgfsetdash{}{0pt}%
\pgfpathmoveto{\pgfqpoint{1.203623in}{0.648129in}}%
\pgfpathcurveto{\pgfqpoint{1.214673in}{0.648129in}}{\pgfqpoint{1.225272in}{0.652519in}}{\pgfqpoint{1.233085in}{0.660333in}}%
\pgfpathcurveto{\pgfqpoint{1.240899in}{0.668146in}}{\pgfqpoint{1.245289in}{0.678745in}}{\pgfqpoint{1.245289in}{0.689796in}}%
\pgfpathcurveto{\pgfqpoint{1.245289in}{0.700846in}}{\pgfqpoint{1.240899in}{0.711445in}}{\pgfqpoint{1.233085in}{0.719258in}}%
\pgfpathcurveto{\pgfqpoint{1.225272in}{0.727072in}}{\pgfqpoint{1.214673in}{0.731462in}}{\pgfqpoint{1.203623in}{0.731462in}}%
\pgfpathcurveto{\pgfqpoint{1.192573in}{0.731462in}}{\pgfqpoint{1.181974in}{0.727072in}}{\pgfqpoint{1.174160in}{0.719258in}}%
\pgfpathcurveto{\pgfqpoint{1.166346in}{0.711445in}}{\pgfqpoint{1.161956in}{0.700846in}}{\pgfqpoint{1.161956in}{0.689796in}}%
\pgfpathcurveto{\pgfqpoint{1.161956in}{0.678745in}}{\pgfqpoint{1.166346in}{0.668146in}}{\pgfqpoint{1.174160in}{0.660333in}}%
\pgfpathcurveto{\pgfqpoint{1.181974in}{0.652519in}}{\pgfqpoint{1.192573in}{0.648129in}}{\pgfqpoint{1.203623in}{0.648129in}}%
\pgfpathclose%
\pgfusepath{stroke,fill}%
\end{pgfscope}%
\begin{pgfscope}%
\pgfpathrectangle{\pgfqpoint{0.648703in}{0.548769in}}{\pgfqpoint{5.201297in}{3.102590in}}%
\pgfusepath{clip}%
\pgfsetbuttcap%
\pgfsetroundjoin%
\definecolor{currentfill}{rgb}{0.121569,0.466667,0.705882}%
\pgfsetfillcolor{currentfill}%
\pgfsetlinewidth{1.003750pt}%
\definecolor{currentstroke}{rgb}{0.121569,0.466667,0.705882}%
\pgfsetstrokecolor{currentstroke}%
\pgfsetdash{}{0pt}%
\pgfpathmoveto{\pgfqpoint{1.991115in}{0.648129in}}%
\pgfpathcurveto{\pgfqpoint{2.002165in}{0.648129in}}{\pgfqpoint{2.012764in}{0.652519in}}{\pgfqpoint{2.020577in}{0.660333in}}%
\pgfpathcurveto{\pgfqpoint{2.028391in}{0.668146in}}{\pgfqpoint{2.032781in}{0.678745in}}{\pgfqpoint{2.032781in}{0.689796in}}%
\pgfpathcurveto{\pgfqpoint{2.032781in}{0.700846in}}{\pgfqpoint{2.028391in}{0.711445in}}{\pgfqpoint{2.020577in}{0.719258in}}%
\pgfpathcurveto{\pgfqpoint{2.012764in}{0.727072in}}{\pgfqpoint{2.002165in}{0.731462in}}{\pgfqpoint{1.991115in}{0.731462in}}%
\pgfpathcurveto{\pgfqpoint{1.980064in}{0.731462in}}{\pgfqpoint{1.969465in}{0.727072in}}{\pgfqpoint{1.961652in}{0.719258in}}%
\pgfpathcurveto{\pgfqpoint{1.953838in}{0.711445in}}{\pgfqpoint{1.949448in}{0.700846in}}{\pgfqpoint{1.949448in}{0.689796in}}%
\pgfpathcurveto{\pgfqpoint{1.949448in}{0.678745in}}{\pgfqpoint{1.953838in}{0.668146in}}{\pgfqpoint{1.961652in}{0.660333in}}%
\pgfpathcurveto{\pgfqpoint{1.969465in}{0.652519in}}{\pgfqpoint{1.980064in}{0.648129in}}{\pgfqpoint{1.991115in}{0.648129in}}%
\pgfpathclose%
\pgfusepath{stroke,fill}%
\end{pgfscope}%
\begin{pgfscope}%
\pgfpathrectangle{\pgfqpoint{0.648703in}{0.548769in}}{\pgfqpoint{5.201297in}{3.102590in}}%
\pgfusepath{clip}%
\pgfsetbuttcap%
\pgfsetroundjoin%
\definecolor{currentfill}{rgb}{0.121569,0.466667,0.705882}%
\pgfsetfillcolor{currentfill}%
\pgfsetlinewidth{1.003750pt}%
\definecolor{currentstroke}{rgb}{0.121569,0.466667,0.705882}%
\pgfsetstrokecolor{currentstroke}%
\pgfsetdash{}{0pt}%
\pgfpathmoveto{\pgfqpoint{1.046124in}{0.648129in}}%
\pgfpathcurveto{\pgfqpoint{1.057174in}{0.648129in}}{\pgfqpoint{1.067773in}{0.652519in}}{\pgfqpoint{1.075587in}{0.660333in}}%
\pgfpathcurveto{\pgfqpoint{1.083401in}{0.668146in}}{\pgfqpoint{1.087791in}{0.678745in}}{\pgfqpoint{1.087791in}{0.689796in}}%
\pgfpathcurveto{\pgfqpoint{1.087791in}{0.700846in}}{\pgfqpoint{1.083401in}{0.711445in}}{\pgfqpoint{1.075587in}{0.719258in}}%
\pgfpathcurveto{\pgfqpoint{1.067773in}{0.727072in}}{\pgfqpoint{1.057174in}{0.731462in}}{\pgfqpoint{1.046124in}{0.731462in}}%
\pgfpathcurveto{\pgfqpoint{1.035074in}{0.731462in}}{\pgfqpoint{1.024475in}{0.727072in}}{\pgfqpoint{1.016662in}{0.719258in}}%
\pgfpathcurveto{\pgfqpoint{1.008848in}{0.711445in}}{\pgfqpoint{1.004458in}{0.700846in}}{\pgfqpoint{1.004458in}{0.689796in}}%
\pgfpathcurveto{\pgfqpoint{1.004458in}{0.678745in}}{\pgfqpoint{1.008848in}{0.668146in}}{\pgfqpoint{1.016662in}{0.660333in}}%
\pgfpathcurveto{\pgfqpoint{1.024475in}{0.652519in}}{\pgfqpoint{1.035074in}{0.648129in}}{\pgfqpoint{1.046124in}{0.648129in}}%
\pgfpathclose%
\pgfusepath{stroke,fill}%
\end{pgfscope}%
\begin{pgfscope}%
\pgfpathrectangle{\pgfqpoint{0.648703in}{0.548769in}}{\pgfqpoint{5.201297in}{3.102590in}}%
\pgfusepath{clip}%
\pgfsetbuttcap%
\pgfsetroundjoin%
\definecolor{currentfill}{rgb}{0.121569,0.466667,0.705882}%
\pgfsetfillcolor{currentfill}%
\pgfsetlinewidth{1.003750pt}%
\definecolor{currentstroke}{rgb}{0.121569,0.466667,0.705882}%
\pgfsetstrokecolor{currentstroke}%
\pgfsetdash{}{0pt}%
\pgfpathmoveto{\pgfqpoint{1.326121in}{0.648129in}}%
\pgfpathcurveto{\pgfqpoint{1.337172in}{0.648129in}}{\pgfqpoint{1.347771in}{0.652519in}}{\pgfqpoint{1.355584in}{0.660333in}}%
\pgfpathcurveto{\pgfqpoint{1.363398in}{0.668146in}}{\pgfqpoint{1.367788in}{0.678745in}}{\pgfqpoint{1.367788in}{0.689796in}}%
\pgfpathcurveto{\pgfqpoint{1.367788in}{0.700846in}}{\pgfqpoint{1.363398in}{0.711445in}}{\pgfqpoint{1.355584in}{0.719258in}}%
\pgfpathcurveto{\pgfqpoint{1.347771in}{0.727072in}}{\pgfqpoint{1.337172in}{0.731462in}}{\pgfqpoint{1.326121in}{0.731462in}}%
\pgfpathcurveto{\pgfqpoint{1.315071in}{0.731462in}}{\pgfqpoint{1.304472in}{0.727072in}}{\pgfqpoint{1.296659in}{0.719258in}}%
\pgfpathcurveto{\pgfqpoint{1.288845in}{0.711445in}}{\pgfqpoint{1.284455in}{0.700846in}}{\pgfqpoint{1.284455in}{0.689796in}}%
\pgfpathcurveto{\pgfqpoint{1.284455in}{0.678745in}}{\pgfqpoint{1.288845in}{0.668146in}}{\pgfqpoint{1.296659in}{0.660333in}}%
\pgfpathcurveto{\pgfqpoint{1.304472in}{0.652519in}}{\pgfqpoint{1.315071in}{0.648129in}}{\pgfqpoint{1.326121in}{0.648129in}}%
\pgfpathclose%
\pgfusepath{stroke,fill}%
\end{pgfscope}%
\begin{pgfscope}%
\pgfpathrectangle{\pgfqpoint{0.648703in}{0.548769in}}{\pgfqpoint{5.201297in}{3.102590in}}%
\pgfusepath{clip}%
\pgfsetbuttcap%
\pgfsetroundjoin%
\definecolor{currentfill}{rgb}{0.121569,0.466667,0.705882}%
\pgfsetfillcolor{currentfill}%
\pgfsetlinewidth{1.003750pt}%
\definecolor{currentstroke}{rgb}{0.121569,0.466667,0.705882}%
\pgfsetstrokecolor{currentstroke}%
\pgfsetdash{}{0pt}%
\pgfpathmoveto{\pgfqpoint{1.635031in}{3.181114in}}%
\pgfpathcurveto{\pgfqpoint{1.646081in}{3.181114in}}{\pgfqpoint{1.656680in}{3.185504in}}{\pgfqpoint{1.664494in}{3.193318in}}%
\pgfpathcurveto{\pgfqpoint{1.672308in}{3.201132in}}{\pgfqpoint{1.676698in}{3.211731in}}{\pgfqpoint{1.676698in}{3.222781in}}%
\pgfpathcurveto{\pgfqpoint{1.676698in}{3.233831in}}{\pgfqpoint{1.672308in}{3.244430in}}{\pgfqpoint{1.664494in}{3.252244in}}%
\pgfpathcurveto{\pgfqpoint{1.656680in}{3.260057in}}{\pgfqpoint{1.646081in}{3.264448in}}{\pgfqpoint{1.635031in}{3.264448in}}%
\pgfpathcurveto{\pgfqpoint{1.623981in}{3.264448in}}{\pgfqpoint{1.613382in}{3.260057in}}{\pgfqpoint{1.605569in}{3.252244in}}%
\pgfpathcurveto{\pgfqpoint{1.597755in}{3.244430in}}{\pgfqpoint{1.593365in}{3.233831in}}{\pgfqpoint{1.593365in}{3.222781in}}%
\pgfpathcurveto{\pgfqpoint{1.593365in}{3.211731in}}{\pgfqpoint{1.597755in}{3.201132in}}{\pgfqpoint{1.605569in}{3.193318in}}%
\pgfpathcurveto{\pgfqpoint{1.613382in}{3.185504in}}{\pgfqpoint{1.623981in}{3.181114in}}{\pgfqpoint{1.635031in}{3.181114in}}%
\pgfpathclose%
\pgfusepath{stroke,fill}%
\end{pgfscope}%
\begin{pgfscope}%
\pgfpathrectangle{\pgfqpoint{0.648703in}{0.548769in}}{\pgfqpoint{5.201297in}{3.102590in}}%
\pgfusepath{clip}%
\pgfsetbuttcap%
\pgfsetroundjoin%
\definecolor{currentfill}{rgb}{1.000000,0.498039,0.054902}%
\pgfsetfillcolor{currentfill}%
\pgfsetlinewidth{1.003750pt}%
\definecolor{currentstroke}{rgb}{1.000000,0.498039,0.054902}%
\pgfsetstrokecolor{currentstroke}%
\pgfsetdash{}{0pt}%
\pgfpathmoveto{\pgfqpoint{1.604528in}{3.193800in}}%
\pgfpathcurveto{\pgfqpoint{1.615578in}{3.193800in}}{\pgfqpoint{1.626177in}{3.198191in}}{\pgfqpoint{1.633990in}{3.206004in}}%
\pgfpathcurveto{\pgfqpoint{1.641804in}{3.213818in}}{\pgfqpoint{1.646194in}{3.224417in}}{\pgfqpoint{1.646194in}{3.235467in}}%
\pgfpathcurveto{\pgfqpoint{1.646194in}{3.246517in}}{\pgfqpoint{1.641804in}{3.257116in}}{\pgfqpoint{1.633990in}{3.264930in}}%
\pgfpathcurveto{\pgfqpoint{1.626177in}{3.272743in}}{\pgfqpoint{1.615578in}{3.277134in}}{\pgfqpoint{1.604528in}{3.277134in}}%
\pgfpathcurveto{\pgfqpoint{1.593478in}{3.277134in}}{\pgfqpoint{1.582879in}{3.272743in}}{\pgfqpoint{1.575065in}{3.264930in}}%
\pgfpathcurveto{\pgfqpoint{1.567251in}{3.257116in}}{\pgfqpoint{1.562861in}{3.246517in}}{\pgfqpoint{1.562861in}{3.235467in}}%
\pgfpathcurveto{\pgfqpoint{1.562861in}{3.224417in}}{\pgfqpoint{1.567251in}{3.213818in}}{\pgfqpoint{1.575065in}{3.206004in}}%
\pgfpathcurveto{\pgfqpoint{1.582879in}{3.198191in}}{\pgfqpoint{1.593478in}{3.193800in}}{\pgfqpoint{1.604528in}{3.193800in}}%
\pgfpathclose%
\pgfusepath{stroke,fill}%
\end{pgfscope}%
\begin{pgfscope}%
\pgfpathrectangle{\pgfqpoint{0.648703in}{0.548769in}}{\pgfqpoint{5.201297in}{3.102590in}}%
\pgfusepath{clip}%
\pgfsetbuttcap%
\pgfsetroundjoin%
\definecolor{currentfill}{rgb}{1.000000,0.498039,0.054902}%
\pgfsetfillcolor{currentfill}%
\pgfsetlinewidth{1.003750pt}%
\definecolor{currentstroke}{rgb}{1.000000,0.498039,0.054902}%
\pgfsetstrokecolor{currentstroke}%
\pgfsetdash{}{0pt}%
\pgfpathmoveto{\pgfqpoint{1.828366in}{3.193800in}}%
\pgfpathcurveto{\pgfqpoint{1.839416in}{3.193800in}}{\pgfqpoint{1.850015in}{3.198191in}}{\pgfqpoint{1.857829in}{3.206004in}}%
\pgfpathcurveto{\pgfqpoint{1.865643in}{3.213818in}}{\pgfqpoint{1.870033in}{3.224417in}}{\pgfqpoint{1.870033in}{3.235467in}}%
\pgfpathcurveto{\pgfqpoint{1.870033in}{3.246517in}}{\pgfqpoint{1.865643in}{3.257116in}}{\pgfqpoint{1.857829in}{3.264930in}}%
\pgfpathcurveto{\pgfqpoint{1.850015in}{3.272743in}}{\pgfqpoint{1.839416in}{3.277134in}}{\pgfqpoint{1.828366in}{3.277134in}}%
\pgfpathcurveto{\pgfqpoint{1.817316in}{3.277134in}}{\pgfqpoint{1.806717in}{3.272743in}}{\pgfqpoint{1.798904in}{3.264930in}}%
\pgfpathcurveto{\pgfqpoint{1.791090in}{3.257116in}}{\pgfqpoint{1.786700in}{3.246517in}}{\pgfqpoint{1.786700in}{3.235467in}}%
\pgfpathcurveto{\pgfqpoint{1.786700in}{3.224417in}}{\pgfqpoint{1.791090in}{3.213818in}}{\pgfqpoint{1.798904in}{3.206004in}}%
\pgfpathcurveto{\pgfqpoint{1.806717in}{3.198191in}}{\pgfqpoint{1.817316in}{3.193800in}}{\pgfqpoint{1.828366in}{3.193800in}}%
\pgfpathclose%
\pgfusepath{stroke,fill}%
\end{pgfscope}%
\begin{pgfscope}%
\pgfpathrectangle{\pgfqpoint{0.648703in}{0.548769in}}{\pgfqpoint{5.201297in}{3.102590in}}%
\pgfusepath{clip}%
\pgfsetbuttcap%
\pgfsetroundjoin%
\definecolor{currentfill}{rgb}{0.121569,0.466667,0.705882}%
\pgfsetfillcolor{currentfill}%
\pgfsetlinewidth{1.003750pt}%
\definecolor{currentstroke}{rgb}{0.121569,0.466667,0.705882}%
\pgfsetstrokecolor{currentstroke}%
\pgfsetdash{}{0pt}%
\pgfpathmoveto{\pgfqpoint{0.958625in}{0.648129in}}%
\pgfpathcurveto{\pgfqpoint{0.969675in}{0.648129in}}{\pgfqpoint{0.980274in}{0.652519in}}{\pgfqpoint{0.988088in}{0.660333in}}%
\pgfpathcurveto{\pgfqpoint{0.995902in}{0.668146in}}{\pgfqpoint{1.000292in}{0.678745in}}{\pgfqpoint{1.000292in}{0.689796in}}%
\pgfpathcurveto{\pgfqpoint{1.000292in}{0.700846in}}{\pgfqpoint{0.995902in}{0.711445in}}{\pgfqpoint{0.988088in}{0.719258in}}%
\pgfpathcurveto{\pgfqpoint{0.980274in}{0.727072in}}{\pgfqpoint{0.969675in}{0.731462in}}{\pgfqpoint{0.958625in}{0.731462in}}%
\pgfpathcurveto{\pgfqpoint{0.947575in}{0.731462in}}{\pgfqpoint{0.936976in}{0.727072in}}{\pgfqpoint{0.929162in}{0.719258in}}%
\pgfpathcurveto{\pgfqpoint{0.921349in}{0.711445in}}{\pgfqpoint{0.916959in}{0.700846in}}{\pgfqpoint{0.916959in}{0.689796in}}%
\pgfpathcurveto{\pgfqpoint{0.916959in}{0.678745in}}{\pgfqpoint{0.921349in}{0.668146in}}{\pgfqpoint{0.929162in}{0.660333in}}%
\pgfpathcurveto{\pgfqpoint{0.936976in}{0.652519in}}{\pgfqpoint{0.947575in}{0.648129in}}{\pgfqpoint{0.958625in}{0.648129in}}%
\pgfpathclose%
\pgfusepath{stroke,fill}%
\end{pgfscope}%
\begin{pgfscope}%
\pgfpathrectangle{\pgfqpoint{0.648703in}{0.548769in}}{\pgfqpoint{5.201297in}{3.102590in}}%
\pgfusepath{clip}%
\pgfsetbuttcap%
\pgfsetroundjoin%
\definecolor{currentfill}{rgb}{0.121569,0.466667,0.705882}%
\pgfsetfillcolor{currentfill}%
\pgfsetlinewidth{1.003750pt}%
\definecolor{currentstroke}{rgb}{0.121569,0.466667,0.705882}%
\pgfsetstrokecolor{currentstroke}%
\pgfsetdash{}{0pt}%
\pgfpathmoveto{\pgfqpoint{1.081124in}{0.665044in}}%
\pgfpathcurveto{\pgfqpoint{1.092174in}{0.665044in}}{\pgfqpoint{1.102773in}{0.669434in}}{\pgfqpoint{1.110587in}{0.677248in}}%
\pgfpathcurveto{\pgfqpoint{1.118400in}{0.685061in}}{\pgfqpoint{1.122791in}{0.695660in}}{\pgfqpoint{1.122791in}{0.706710in}}%
\pgfpathcurveto{\pgfqpoint{1.122791in}{0.717760in}}{\pgfqpoint{1.118400in}{0.728360in}}{\pgfqpoint{1.110587in}{0.736173in}}%
\pgfpathcurveto{\pgfqpoint{1.102773in}{0.743987in}}{\pgfqpoint{1.092174in}{0.748377in}}{\pgfqpoint{1.081124in}{0.748377in}}%
\pgfpathcurveto{\pgfqpoint{1.070074in}{0.748377in}}{\pgfqpoint{1.059475in}{0.743987in}}{\pgfqpoint{1.051661in}{0.736173in}}%
\pgfpathcurveto{\pgfqpoint{1.043848in}{0.728360in}}{\pgfqpoint{1.039457in}{0.717760in}}{\pgfqpoint{1.039457in}{0.706710in}}%
\pgfpathcurveto{\pgfqpoint{1.039457in}{0.695660in}}{\pgfqpoint{1.043848in}{0.685061in}}{\pgfqpoint{1.051661in}{0.677248in}}%
\pgfpathcurveto{\pgfqpoint{1.059475in}{0.669434in}}{\pgfqpoint{1.070074in}{0.665044in}}{\pgfqpoint{1.081124in}{0.665044in}}%
\pgfpathclose%
\pgfusepath{stroke,fill}%
\end{pgfscope}%
\begin{pgfscope}%
\pgfpathrectangle{\pgfqpoint{0.648703in}{0.548769in}}{\pgfqpoint{5.201297in}{3.102590in}}%
\pgfusepath{clip}%
\pgfsetbuttcap%
\pgfsetroundjoin%
\definecolor{currentfill}{rgb}{1.000000,0.498039,0.054902}%
\pgfsetfillcolor{currentfill}%
\pgfsetlinewidth{1.003750pt}%
\definecolor{currentstroke}{rgb}{1.000000,0.498039,0.054902}%
\pgfsetstrokecolor{currentstroke}%
\pgfsetdash{}{0pt}%
\pgfpathmoveto{\pgfqpoint{2.600108in}{3.189572in}}%
\pgfpathcurveto{\pgfqpoint{2.611159in}{3.189572in}}{\pgfqpoint{2.621758in}{3.193962in}}{\pgfqpoint{2.629571in}{3.201775in}}%
\pgfpathcurveto{\pgfqpoint{2.637385in}{3.209589in}}{\pgfqpoint{2.641775in}{3.220188in}}{\pgfqpoint{2.641775in}{3.231238in}}%
\pgfpathcurveto{\pgfqpoint{2.641775in}{3.242288in}}{\pgfqpoint{2.637385in}{3.252887in}}{\pgfqpoint{2.629571in}{3.260701in}}%
\pgfpathcurveto{\pgfqpoint{2.621758in}{3.268515in}}{\pgfqpoint{2.611159in}{3.272905in}}{\pgfqpoint{2.600108in}{3.272905in}}%
\pgfpathcurveto{\pgfqpoint{2.589058in}{3.272905in}}{\pgfqpoint{2.578459in}{3.268515in}}{\pgfqpoint{2.570646in}{3.260701in}}%
\pgfpathcurveto{\pgfqpoint{2.562832in}{3.252887in}}{\pgfqpoint{2.558442in}{3.242288in}}{\pgfqpoint{2.558442in}{3.231238in}}%
\pgfpathcurveto{\pgfqpoint{2.558442in}{3.220188in}}{\pgfqpoint{2.562832in}{3.209589in}}{\pgfqpoint{2.570646in}{3.201775in}}%
\pgfpathcurveto{\pgfqpoint{2.578459in}{3.193962in}}{\pgfqpoint{2.589058in}{3.189572in}}{\pgfqpoint{2.600108in}{3.189572in}}%
\pgfpathclose%
\pgfusepath{stroke,fill}%
\end{pgfscope}%
\begin{pgfscope}%
\pgfpathrectangle{\pgfqpoint{0.648703in}{0.548769in}}{\pgfqpoint{5.201297in}{3.102590in}}%
\pgfusepath{clip}%
\pgfsetbuttcap%
\pgfsetroundjoin%
\definecolor{currentfill}{rgb}{0.121569,0.466667,0.705882}%
\pgfsetfillcolor{currentfill}%
\pgfsetlinewidth{1.003750pt}%
\definecolor{currentstroke}{rgb}{0.121569,0.466667,0.705882}%
\pgfsetstrokecolor{currentstroke}%
\pgfsetdash{}{0pt}%
\pgfpathmoveto{\pgfqpoint{1.865116in}{0.808819in}}%
\pgfpathcurveto{\pgfqpoint{1.876166in}{0.808819in}}{\pgfqpoint{1.886765in}{0.813209in}}{\pgfqpoint{1.894579in}{0.821023in}}%
\pgfpathcurveto{\pgfqpoint{1.902392in}{0.828837in}}{\pgfqpoint{1.906783in}{0.839436in}}{\pgfqpoint{1.906783in}{0.850486in}}%
\pgfpathcurveto{\pgfqpoint{1.906783in}{0.861536in}}{\pgfqpoint{1.902392in}{0.872135in}}{\pgfqpoint{1.894579in}{0.879949in}}%
\pgfpathcurveto{\pgfqpoint{1.886765in}{0.887762in}}{\pgfqpoint{1.876166in}{0.892152in}}{\pgfqpoint{1.865116in}{0.892152in}}%
\pgfpathcurveto{\pgfqpoint{1.854066in}{0.892152in}}{\pgfqpoint{1.843467in}{0.887762in}}{\pgfqpoint{1.835653in}{0.879949in}}%
\pgfpathcurveto{\pgfqpoint{1.827840in}{0.872135in}}{\pgfqpoint{1.823449in}{0.861536in}}{\pgfqpoint{1.823449in}{0.850486in}}%
\pgfpathcurveto{\pgfqpoint{1.823449in}{0.839436in}}{\pgfqpoint{1.827840in}{0.828837in}}{\pgfqpoint{1.835653in}{0.821023in}}%
\pgfpathcurveto{\pgfqpoint{1.843467in}{0.813209in}}{\pgfqpoint{1.854066in}{0.808819in}}{\pgfqpoint{1.865116in}{0.808819in}}%
\pgfpathclose%
\pgfusepath{stroke,fill}%
\end{pgfscope}%
\begin{pgfscope}%
\pgfpathrectangle{\pgfqpoint{0.648703in}{0.548769in}}{\pgfqpoint{5.201297in}{3.102590in}}%
\pgfusepath{clip}%
\pgfsetbuttcap%
\pgfsetroundjoin%
\definecolor{currentfill}{rgb}{0.839216,0.152941,0.156863}%
\pgfsetfillcolor{currentfill}%
\pgfsetlinewidth{1.003750pt}%
\definecolor{currentstroke}{rgb}{0.839216,0.152941,0.156863}%
\pgfsetstrokecolor{currentstroke}%
\pgfsetdash{}{0pt}%
\pgfpathmoveto{\pgfqpoint{1.479245in}{3.198029in}}%
\pgfpathcurveto{\pgfqpoint{1.490295in}{3.198029in}}{\pgfqpoint{1.500894in}{3.202419in}}{\pgfqpoint{1.508708in}{3.210233in}}%
\pgfpathcurveto{\pgfqpoint{1.516521in}{3.218046in}}{\pgfqpoint{1.520912in}{3.228646in}}{\pgfqpoint{1.520912in}{3.239696in}}%
\pgfpathcurveto{\pgfqpoint{1.520912in}{3.250746in}}{\pgfqpoint{1.516521in}{3.261345in}}{\pgfqpoint{1.508708in}{3.269158in}}%
\pgfpathcurveto{\pgfqpoint{1.500894in}{3.276972in}}{\pgfqpoint{1.490295in}{3.281362in}}{\pgfqpoint{1.479245in}{3.281362in}}%
\pgfpathcurveto{\pgfqpoint{1.468195in}{3.281362in}}{\pgfqpoint{1.457596in}{3.276972in}}{\pgfqpoint{1.449782in}{3.269158in}}%
\pgfpathcurveto{\pgfqpoint{1.441968in}{3.261345in}}{\pgfqpoint{1.437578in}{3.250746in}}{\pgfqpoint{1.437578in}{3.239696in}}%
\pgfpathcurveto{\pgfqpoint{1.437578in}{3.228646in}}{\pgfqpoint{1.441968in}{3.218046in}}{\pgfqpoint{1.449782in}{3.210233in}}%
\pgfpathcurveto{\pgfqpoint{1.457596in}{3.202419in}}{\pgfqpoint{1.468195in}{3.198029in}}{\pgfqpoint{1.479245in}{3.198029in}}%
\pgfpathclose%
\pgfusepath{stroke,fill}%
\end{pgfscope}%
\begin{pgfscope}%
\pgfpathrectangle{\pgfqpoint{0.648703in}{0.548769in}}{\pgfqpoint{5.201297in}{3.102590in}}%
\pgfusepath{clip}%
\pgfsetbuttcap%
\pgfsetroundjoin%
\definecolor{currentfill}{rgb}{0.121569,0.466667,0.705882}%
\pgfsetfillcolor{currentfill}%
\pgfsetlinewidth{1.003750pt}%
\definecolor{currentstroke}{rgb}{0.121569,0.466667,0.705882}%
\pgfsetstrokecolor{currentstroke}%
\pgfsetdash{}{0pt}%
\pgfpathmoveto{\pgfqpoint{1.448620in}{0.648129in}}%
\pgfpathcurveto{\pgfqpoint{1.459670in}{0.648129in}}{\pgfqpoint{1.470269in}{0.652519in}}{\pgfqpoint{1.478083in}{0.660333in}}%
\pgfpathcurveto{\pgfqpoint{1.485897in}{0.668146in}}{\pgfqpoint{1.490287in}{0.678745in}}{\pgfqpoint{1.490287in}{0.689796in}}%
\pgfpathcurveto{\pgfqpoint{1.490287in}{0.700846in}}{\pgfqpoint{1.485897in}{0.711445in}}{\pgfqpoint{1.478083in}{0.719258in}}%
\pgfpathcurveto{\pgfqpoint{1.470269in}{0.727072in}}{\pgfqpoint{1.459670in}{0.731462in}}{\pgfqpoint{1.448620in}{0.731462in}}%
\pgfpathcurveto{\pgfqpoint{1.437570in}{0.731462in}}{\pgfqpoint{1.426971in}{0.727072in}}{\pgfqpoint{1.419157in}{0.719258in}}%
\pgfpathcurveto{\pgfqpoint{1.411344in}{0.711445in}}{\pgfqpoint{1.406954in}{0.700846in}}{\pgfqpoint{1.406954in}{0.689796in}}%
\pgfpathcurveto{\pgfqpoint{1.406954in}{0.678745in}}{\pgfqpoint{1.411344in}{0.668146in}}{\pgfqpoint{1.419157in}{0.660333in}}%
\pgfpathcurveto{\pgfqpoint{1.426971in}{0.652519in}}{\pgfqpoint{1.437570in}{0.648129in}}{\pgfqpoint{1.448620in}{0.648129in}}%
\pgfpathclose%
\pgfusepath{stroke,fill}%
\end{pgfscope}%
\begin{pgfscope}%
\pgfpathrectangle{\pgfqpoint{0.648703in}{0.548769in}}{\pgfqpoint{5.201297in}{3.102590in}}%
\pgfusepath{clip}%
\pgfsetbuttcap%
\pgfsetroundjoin%
\definecolor{currentfill}{rgb}{1.000000,0.498039,0.054902}%
\pgfsetfillcolor{currentfill}%
\pgfsetlinewidth{1.003750pt}%
\definecolor{currentstroke}{rgb}{1.000000,0.498039,0.054902}%
\pgfsetstrokecolor{currentstroke}%
\pgfsetdash{}{0pt}%
\pgfpathmoveto{\pgfqpoint{2.796106in}{3.185343in}}%
\pgfpathcurveto{\pgfqpoint{2.807156in}{3.185343in}}{\pgfqpoint{2.817756in}{3.189733in}}{\pgfqpoint{2.825569in}{3.197547in}}%
\pgfpathcurveto{\pgfqpoint{2.833383in}{3.205360in}}{\pgfqpoint{2.837773in}{3.215959in}}{\pgfqpoint{2.837773in}{3.227010in}}%
\pgfpathcurveto{\pgfqpoint{2.837773in}{3.238060in}}{\pgfqpoint{2.833383in}{3.248659in}}{\pgfqpoint{2.825569in}{3.256472in}}%
\pgfpathcurveto{\pgfqpoint{2.817756in}{3.264286in}}{\pgfqpoint{2.807156in}{3.268676in}}{\pgfqpoint{2.796106in}{3.268676in}}%
\pgfpathcurveto{\pgfqpoint{2.785056in}{3.268676in}}{\pgfqpoint{2.774457in}{3.264286in}}{\pgfqpoint{2.766644in}{3.256472in}}%
\pgfpathcurveto{\pgfqpoint{2.758830in}{3.248659in}}{\pgfqpoint{2.754440in}{3.238060in}}{\pgfqpoint{2.754440in}{3.227010in}}%
\pgfpathcurveto{\pgfqpoint{2.754440in}{3.215959in}}{\pgfqpoint{2.758830in}{3.205360in}}{\pgfqpoint{2.766644in}{3.197547in}}%
\pgfpathcurveto{\pgfqpoint{2.774457in}{3.189733in}}{\pgfqpoint{2.785056in}{3.185343in}}{\pgfqpoint{2.796106in}{3.185343in}}%
\pgfpathclose%
\pgfusepath{stroke,fill}%
\end{pgfscope}%
\begin{pgfscope}%
\pgfpathrectangle{\pgfqpoint{0.648703in}{0.548769in}}{\pgfqpoint{5.201297in}{3.102590in}}%
\pgfusepath{clip}%
\pgfsetbuttcap%
\pgfsetroundjoin%
\definecolor{currentfill}{rgb}{0.121569,0.466667,0.705882}%
\pgfsetfillcolor{currentfill}%
\pgfsetlinewidth{1.003750pt}%
\definecolor{currentstroke}{rgb}{0.121569,0.466667,0.705882}%
\pgfsetstrokecolor{currentstroke}%
\pgfsetdash{}{0pt}%
\pgfpathmoveto{\pgfqpoint{0.958625in}{0.648129in}}%
\pgfpathcurveto{\pgfqpoint{0.969675in}{0.648129in}}{\pgfqpoint{0.980274in}{0.652519in}}{\pgfqpoint{0.988088in}{0.660333in}}%
\pgfpathcurveto{\pgfqpoint{0.995902in}{0.668146in}}{\pgfqpoint{1.000292in}{0.678745in}}{\pgfqpoint{1.000292in}{0.689796in}}%
\pgfpathcurveto{\pgfqpoint{1.000292in}{0.700846in}}{\pgfqpoint{0.995902in}{0.711445in}}{\pgfqpoint{0.988088in}{0.719258in}}%
\pgfpathcurveto{\pgfqpoint{0.980274in}{0.727072in}}{\pgfqpoint{0.969675in}{0.731462in}}{\pgfqpoint{0.958625in}{0.731462in}}%
\pgfpathcurveto{\pgfqpoint{0.947575in}{0.731462in}}{\pgfqpoint{0.936976in}{0.727072in}}{\pgfqpoint{0.929162in}{0.719258in}}%
\pgfpathcurveto{\pgfqpoint{0.921349in}{0.711445in}}{\pgfqpoint{0.916959in}{0.700846in}}{\pgfqpoint{0.916959in}{0.689796in}}%
\pgfpathcurveto{\pgfqpoint{0.916959in}{0.678745in}}{\pgfqpoint{0.921349in}{0.668146in}}{\pgfqpoint{0.929162in}{0.660333in}}%
\pgfpathcurveto{\pgfqpoint{0.936976in}{0.652519in}}{\pgfqpoint{0.947575in}{0.648129in}}{\pgfqpoint{0.958625in}{0.648129in}}%
\pgfpathclose%
\pgfusepath{stroke,fill}%
\end{pgfscope}%
\begin{pgfscope}%
\pgfpathrectangle{\pgfqpoint{0.648703in}{0.548769in}}{\pgfqpoint{5.201297in}{3.102590in}}%
\pgfusepath{clip}%
\pgfsetbuttcap%
\pgfsetroundjoin%
\definecolor{currentfill}{rgb}{1.000000,0.498039,0.054902}%
\pgfsetfillcolor{currentfill}%
\pgfsetlinewidth{1.003750pt}%
\definecolor{currentstroke}{rgb}{1.000000,0.498039,0.054902}%
\pgfsetstrokecolor{currentstroke}%
\pgfsetdash{}{0pt}%
\pgfpathmoveto{\pgfqpoint{1.142373in}{3.185343in}}%
\pgfpathcurveto{\pgfqpoint{1.153423in}{3.185343in}}{\pgfqpoint{1.164023in}{3.189733in}}{\pgfqpoint{1.171836in}{3.197547in}}%
\pgfpathcurveto{\pgfqpoint{1.179650in}{3.205360in}}{\pgfqpoint{1.184040in}{3.215959in}}{\pgfqpoint{1.184040in}{3.227010in}}%
\pgfpathcurveto{\pgfqpoint{1.184040in}{3.238060in}}{\pgfqpoint{1.179650in}{3.248659in}}{\pgfqpoint{1.171836in}{3.256472in}}%
\pgfpathcurveto{\pgfqpoint{1.164023in}{3.264286in}}{\pgfqpoint{1.153423in}{3.268676in}}{\pgfqpoint{1.142373in}{3.268676in}}%
\pgfpathcurveto{\pgfqpoint{1.131323in}{3.268676in}}{\pgfqpoint{1.120724in}{3.264286in}}{\pgfqpoint{1.112911in}{3.256472in}}%
\pgfpathcurveto{\pgfqpoint{1.105097in}{3.248659in}}{\pgfqpoint{1.100707in}{3.238060in}}{\pgfqpoint{1.100707in}{3.227010in}}%
\pgfpathcurveto{\pgfqpoint{1.100707in}{3.215959in}}{\pgfqpoint{1.105097in}{3.205360in}}{\pgfqpoint{1.112911in}{3.197547in}}%
\pgfpathcurveto{\pgfqpoint{1.120724in}{3.189733in}}{\pgfqpoint{1.131323in}{3.185343in}}{\pgfqpoint{1.142373in}{3.185343in}}%
\pgfpathclose%
\pgfusepath{stroke,fill}%
\end{pgfscope}%
\begin{pgfscope}%
\pgfpathrectangle{\pgfqpoint{0.648703in}{0.548769in}}{\pgfqpoint{5.201297in}{3.102590in}}%
\pgfusepath{clip}%
\pgfsetbuttcap%
\pgfsetroundjoin%
\definecolor{currentfill}{rgb}{0.121569,0.466667,0.705882}%
\pgfsetfillcolor{currentfill}%
\pgfsetlinewidth{1.003750pt}%
\definecolor{currentstroke}{rgb}{0.121569,0.466667,0.705882}%
\pgfsetstrokecolor{currentstroke}%
\pgfsetdash{}{0pt}%
\pgfpathmoveto{\pgfqpoint{0.969761in}{0.648129in}}%
\pgfpathcurveto{\pgfqpoint{0.980812in}{0.648129in}}{\pgfqpoint{0.991411in}{0.652519in}}{\pgfqpoint{0.999224in}{0.660333in}}%
\pgfpathcurveto{\pgfqpoint{1.007038in}{0.668146in}}{\pgfqpoint{1.011428in}{0.678745in}}{\pgfqpoint{1.011428in}{0.689796in}}%
\pgfpathcurveto{\pgfqpoint{1.011428in}{0.700846in}}{\pgfqpoint{1.007038in}{0.711445in}}{\pgfqpoint{0.999224in}{0.719258in}}%
\pgfpathcurveto{\pgfqpoint{0.991411in}{0.727072in}}{\pgfqpoint{0.980812in}{0.731462in}}{\pgfqpoint{0.969761in}{0.731462in}}%
\pgfpathcurveto{\pgfqpoint{0.958711in}{0.731462in}}{\pgfqpoint{0.948112in}{0.727072in}}{\pgfqpoint{0.940299in}{0.719258in}}%
\pgfpathcurveto{\pgfqpoint{0.932485in}{0.711445in}}{\pgfqpoint{0.928095in}{0.700846in}}{\pgfqpoint{0.928095in}{0.689796in}}%
\pgfpathcurveto{\pgfqpoint{0.928095in}{0.678745in}}{\pgfqpoint{0.932485in}{0.668146in}}{\pgfqpoint{0.940299in}{0.660333in}}%
\pgfpathcurveto{\pgfqpoint{0.948112in}{0.652519in}}{\pgfqpoint{0.958711in}{0.648129in}}{\pgfqpoint{0.969761in}{0.648129in}}%
\pgfpathclose%
\pgfusepath{stroke,fill}%
\end{pgfscope}%
\begin{pgfscope}%
\pgfpathrectangle{\pgfqpoint{0.648703in}{0.548769in}}{\pgfqpoint{5.201297in}{3.102590in}}%
\pgfusepath{clip}%
\pgfsetbuttcap%
\pgfsetroundjoin%
\definecolor{currentfill}{rgb}{1.000000,0.498039,0.054902}%
\pgfsetfillcolor{currentfill}%
\pgfsetlinewidth{1.003750pt}%
\definecolor{currentstroke}{rgb}{1.000000,0.498039,0.054902}%
\pgfsetstrokecolor{currentstroke}%
\pgfsetdash{}{0pt}%
\pgfpathmoveto{\pgfqpoint{1.130123in}{3.278374in}}%
\pgfpathcurveto{\pgfqpoint{1.141174in}{3.278374in}}{\pgfqpoint{1.151773in}{3.282764in}}{\pgfqpoint{1.159586in}{3.290578in}}%
\pgfpathcurveto{\pgfqpoint{1.167400in}{3.298392in}}{\pgfqpoint{1.171790in}{3.308991in}}{\pgfqpoint{1.171790in}{3.320041in}}%
\pgfpathcurveto{\pgfqpoint{1.171790in}{3.331091in}}{\pgfqpoint{1.167400in}{3.341690in}}{\pgfqpoint{1.159586in}{3.349504in}}%
\pgfpathcurveto{\pgfqpoint{1.151773in}{3.357317in}}{\pgfqpoint{1.141174in}{3.361707in}}{\pgfqpoint{1.130123in}{3.361707in}}%
\pgfpathcurveto{\pgfqpoint{1.119073in}{3.361707in}}{\pgfqpoint{1.108474in}{3.357317in}}{\pgfqpoint{1.100661in}{3.349504in}}%
\pgfpathcurveto{\pgfqpoint{1.092847in}{3.341690in}}{\pgfqpoint{1.088457in}{3.331091in}}{\pgfqpoint{1.088457in}{3.320041in}}%
\pgfpathcurveto{\pgfqpoint{1.088457in}{3.308991in}}{\pgfqpoint{1.092847in}{3.298392in}}{\pgfqpoint{1.100661in}{3.290578in}}%
\pgfpathcurveto{\pgfqpoint{1.108474in}{3.282764in}}{\pgfqpoint{1.119073in}{3.278374in}}{\pgfqpoint{1.130123in}{3.278374in}}%
\pgfpathclose%
\pgfusepath{stroke,fill}%
\end{pgfscope}%
\begin{pgfscope}%
\pgfpathrectangle{\pgfqpoint{0.648703in}{0.548769in}}{\pgfqpoint{5.201297in}{3.102590in}}%
\pgfusepath{clip}%
\pgfsetbuttcap%
\pgfsetroundjoin%
\definecolor{currentfill}{rgb}{0.121569,0.466667,0.705882}%
\pgfsetfillcolor{currentfill}%
\pgfsetlinewidth{1.003750pt}%
\definecolor{currentstroke}{rgb}{0.121569,0.466667,0.705882}%
\pgfsetstrokecolor{currentstroke}%
\pgfsetdash{}{0pt}%
\pgfpathmoveto{\pgfqpoint{0.983125in}{0.648129in}}%
\pgfpathcurveto{\pgfqpoint{0.994175in}{0.648129in}}{\pgfqpoint{1.004774in}{0.652519in}}{\pgfqpoint{1.012588in}{0.660333in}}%
\pgfpathcurveto{\pgfqpoint{1.020401in}{0.668146in}}{\pgfqpoint{1.024792in}{0.678745in}}{\pgfqpoint{1.024792in}{0.689796in}}%
\pgfpathcurveto{\pgfqpoint{1.024792in}{0.700846in}}{\pgfqpoint{1.020401in}{0.711445in}}{\pgfqpoint{1.012588in}{0.719258in}}%
\pgfpathcurveto{\pgfqpoint{1.004774in}{0.727072in}}{\pgfqpoint{0.994175in}{0.731462in}}{\pgfqpoint{0.983125in}{0.731462in}}%
\pgfpathcurveto{\pgfqpoint{0.972075in}{0.731462in}}{\pgfqpoint{0.961476in}{0.727072in}}{\pgfqpoint{0.953662in}{0.719258in}}%
\pgfpathcurveto{\pgfqpoint{0.945849in}{0.711445in}}{\pgfqpoint{0.941458in}{0.700846in}}{\pgfqpoint{0.941458in}{0.689796in}}%
\pgfpathcurveto{\pgfqpoint{0.941458in}{0.678745in}}{\pgfqpoint{0.945849in}{0.668146in}}{\pgfqpoint{0.953662in}{0.660333in}}%
\pgfpathcurveto{\pgfqpoint{0.961476in}{0.652519in}}{\pgfqpoint{0.972075in}{0.648129in}}{\pgfqpoint{0.983125in}{0.648129in}}%
\pgfpathclose%
\pgfusepath{stroke,fill}%
\end{pgfscope}%
\begin{pgfscope}%
\pgfpathrectangle{\pgfqpoint{0.648703in}{0.548769in}}{\pgfqpoint{5.201297in}{3.102590in}}%
\pgfusepath{clip}%
\pgfsetbuttcap%
\pgfsetroundjoin%
\definecolor{currentfill}{rgb}{1.000000,0.498039,0.054902}%
\pgfsetfillcolor{currentfill}%
\pgfsetlinewidth{1.003750pt}%
\definecolor{currentstroke}{rgb}{1.000000,0.498039,0.054902}%
\pgfsetstrokecolor{currentstroke}%
\pgfsetdash{}{0pt}%
\pgfpathmoveto{\pgfqpoint{1.662993in}{3.202258in}}%
\pgfpathcurveto{\pgfqpoint{1.674043in}{3.202258in}}{\pgfqpoint{1.684642in}{3.206648in}}{\pgfqpoint{1.692456in}{3.214462in}}%
\pgfpathcurveto{\pgfqpoint{1.700269in}{3.222275in}}{\pgfqpoint{1.704660in}{3.232874in}}{\pgfqpoint{1.704660in}{3.243924in}}%
\pgfpathcurveto{\pgfqpoint{1.704660in}{3.254974in}}{\pgfqpoint{1.700269in}{3.265573in}}{\pgfqpoint{1.692456in}{3.273387in}}%
\pgfpathcurveto{\pgfqpoint{1.684642in}{3.281201in}}{\pgfqpoint{1.674043in}{3.285591in}}{\pgfqpoint{1.662993in}{3.285591in}}%
\pgfpathcurveto{\pgfqpoint{1.651943in}{3.285591in}}{\pgfqpoint{1.641344in}{3.281201in}}{\pgfqpoint{1.633530in}{3.273387in}}%
\pgfpathcurveto{\pgfqpoint{1.625717in}{3.265573in}}{\pgfqpoint{1.621326in}{3.254974in}}{\pgfqpoint{1.621326in}{3.243924in}}%
\pgfpathcurveto{\pgfqpoint{1.621326in}{3.232874in}}{\pgfqpoint{1.625717in}{3.222275in}}{\pgfqpoint{1.633530in}{3.214462in}}%
\pgfpathcurveto{\pgfqpoint{1.641344in}{3.206648in}}{\pgfqpoint{1.651943in}{3.202258in}}{\pgfqpoint{1.662993in}{3.202258in}}%
\pgfpathclose%
\pgfusepath{stroke,fill}%
\end{pgfscope}%
\begin{pgfscope}%
\pgfpathrectangle{\pgfqpoint{0.648703in}{0.548769in}}{\pgfqpoint{5.201297in}{3.102590in}}%
\pgfusepath{clip}%
\pgfsetbuttcap%
\pgfsetroundjoin%
\definecolor{currentfill}{rgb}{1.000000,0.498039,0.054902}%
\pgfsetfillcolor{currentfill}%
\pgfsetlinewidth{1.003750pt}%
\definecolor{currentstroke}{rgb}{1.000000,0.498039,0.054902}%
\pgfsetstrokecolor{currentstroke}%
\pgfsetdash{}{0pt}%
\pgfpathmoveto{\pgfqpoint{1.111749in}{3.185343in}}%
\pgfpathcurveto{\pgfqpoint{1.122799in}{3.185343in}}{\pgfqpoint{1.133398in}{3.189733in}}{\pgfqpoint{1.141211in}{3.197547in}}%
\pgfpathcurveto{\pgfqpoint{1.149025in}{3.205360in}}{\pgfqpoint{1.153415in}{3.215959in}}{\pgfqpoint{1.153415in}{3.227010in}}%
\pgfpathcurveto{\pgfqpoint{1.153415in}{3.238060in}}{\pgfqpoint{1.149025in}{3.248659in}}{\pgfqpoint{1.141211in}{3.256472in}}%
\pgfpathcurveto{\pgfqpoint{1.133398in}{3.264286in}}{\pgfqpoint{1.122799in}{3.268676in}}{\pgfqpoint{1.111749in}{3.268676in}}%
\pgfpathcurveto{\pgfqpoint{1.100699in}{3.268676in}}{\pgfqpoint{1.090099in}{3.264286in}}{\pgfqpoint{1.082286in}{3.256472in}}%
\pgfpathcurveto{\pgfqpoint{1.074472in}{3.248659in}}{\pgfqpoint{1.070082in}{3.238060in}}{\pgfqpoint{1.070082in}{3.227010in}}%
\pgfpathcurveto{\pgfqpoint{1.070082in}{3.215959in}}{\pgfqpoint{1.074472in}{3.205360in}}{\pgfqpoint{1.082286in}{3.197547in}}%
\pgfpathcurveto{\pgfqpoint{1.090099in}{3.189733in}}{\pgfqpoint{1.100699in}{3.185343in}}{\pgfqpoint{1.111749in}{3.185343in}}%
\pgfpathclose%
\pgfusepath{stroke,fill}%
\end{pgfscope}%
\begin{pgfscope}%
\pgfpathrectangle{\pgfqpoint{0.648703in}{0.548769in}}{\pgfqpoint{5.201297in}{3.102590in}}%
\pgfusepath{clip}%
\pgfsetbuttcap%
\pgfsetroundjoin%
\definecolor{currentfill}{rgb}{0.121569,0.466667,0.705882}%
\pgfsetfillcolor{currentfill}%
\pgfsetlinewidth{1.003750pt}%
\definecolor{currentstroke}{rgb}{0.121569,0.466667,0.705882}%
\pgfsetstrokecolor{currentstroke}%
\pgfsetdash{}{0pt}%
\pgfpathmoveto{\pgfqpoint{1.466120in}{0.648129in}}%
\pgfpathcurveto{\pgfqpoint{1.477170in}{0.648129in}}{\pgfqpoint{1.487769in}{0.652519in}}{\pgfqpoint{1.495583in}{0.660333in}}%
\pgfpathcurveto{\pgfqpoint{1.503396in}{0.668146in}}{\pgfqpoint{1.507787in}{0.678745in}}{\pgfqpoint{1.507787in}{0.689796in}}%
\pgfpathcurveto{\pgfqpoint{1.507787in}{0.700846in}}{\pgfqpoint{1.503396in}{0.711445in}}{\pgfqpoint{1.495583in}{0.719258in}}%
\pgfpathcurveto{\pgfqpoint{1.487769in}{0.727072in}}{\pgfqpoint{1.477170in}{0.731462in}}{\pgfqpoint{1.466120in}{0.731462in}}%
\pgfpathcurveto{\pgfqpoint{1.455070in}{0.731462in}}{\pgfqpoint{1.444471in}{0.727072in}}{\pgfqpoint{1.436657in}{0.719258in}}%
\pgfpathcurveto{\pgfqpoint{1.428844in}{0.711445in}}{\pgfqpoint{1.424453in}{0.700846in}}{\pgfqpoint{1.424453in}{0.689796in}}%
\pgfpathcurveto{\pgfqpoint{1.424453in}{0.678745in}}{\pgfqpoint{1.428844in}{0.668146in}}{\pgfqpoint{1.436657in}{0.660333in}}%
\pgfpathcurveto{\pgfqpoint{1.444471in}{0.652519in}}{\pgfqpoint{1.455070in}{0.648129in}}{\pgfqpoint{1.466120in}{0.648129in}}%
\pgfpathclose%
\pgfusepath{stroke,fill}%
\end{pgfscope}%
\begin{pgfscope}%
\pgfpathrectangle{\pgfqpoint{0.648703in}{0.548769in}}{\pgfqpoint{5.201297in}{3.102590in}}%
\pgfusepath{clip}%
\pgfsetbuttcap%
\pgfsetroundjoin%
\definecolor{currentfill}{rgb}{1.000000,0.498039,0.054902}%
\pgfsetfillcolor{currentfill}%
\pgfsetlinewidth{1.003750pt}%
\definecolor{currentstroke}{rgb}{1.000000,0.498039,0.054902}%
\pgfsetstrokecolor{currentstroke}%
\pgfsetdash{}{0pt}%
\pgfpathmoveto{\pgfqpoint{3.163603in}{3.185343in}}%
\pgfpathcurveto{\pgfqpoint{3.174653in}{3.185343in}}{\pgfqpoint{3.185252in}{3.189733in}}{\pgfqpoint{3.193065in}{3.197547in}}%
\pgfpathcurveto{\pgfqpoint{3.200879in}{3.205360in}}{\pgfqpoint{3.205269in}{3.215959in}}{\pgfqpoint{3.205269in}{3.227010in}}%
\pgfpathcurveto{\pgfqpoint{3.205269in}{3.238060in}}{\pgfqpoint{3.200879in}{3.248659in}}{\pgfqpoint{3.193065in}{3.256472in}}%
\pgfpathcurveto{\pgfqpoint{3.185252in}{3.264286in}}{\pgfqpoint{3.174653in}{3.268676in}}{\pgfqpoint{3.163603in}{3.268676in}}%
\pgfpathcurveto{\pgfqpoint{3.152552in}{3.268676in}}{\pgfqpoint{3.141953in}{3.264286in}}{\pgfqpoint{3.134140in}{3.256472in}}%
\pgfpathcurveto{\pgfqpoint{3.126326in}{3.248659in}}{\pgfqpoint{3.121936in}{3.238060in}}{\pgfqpoint{3.121936in}{3.227010in}}%
\pgfpathcurveto{\pgfqpoint{3.121936in}{3.215959in}}{\pgfqpoint{3.126326in}{3.205360in}}{\pgfqpoint{3.134140in}{3.197547in}}%
\pgfpathcurveto{\pgfqpoint{3.141953in}{3.189733in}}{\pgfqpoint{3.152552in}{3.185343in}}{\pgfqpoint{3.163603in}{3.185343in}}%
\pgfpathclose%
\pgfusepath{stroke,fill}%
\end{pgfscope}%
\begin{pgfscope}%
\pgfpathrectangle{\pgfqpoint{0.648703in}{0.548769in}}{\pgfqpoint{5.201297in}{3.102590in}}%
\pgfusepath{clip}%
\pgfsetbuttcap%
\pgfsetroundjoin%
\definecolor{currentfill}{rgb}{1.000000,0.498039,0.054902}%
\pgfsetfillcolor{currentfill}%
\pgfsetlinewidth{1.003750pt}%
\definecolor{currentstroke}{rgb}{1.000000,0.498039,0.054902}%
\pgfsetstrokecolor{currentstroke}%
\pgfsetdash{}{0pt}%
\pgfpathmoveto{\pgfqpoint{2.397985in}{3.193800in}}%
\pgfpathcurveto{\pgfqpoint{2.409036in}{3.193800in}}{\pgfqpoint{2.419635in}{3.198191in}}{\pgfqpoint{2.427448in}{3.206004in}}%
\pgfpathcurveto{\pgfqpoint{2.435262in}{3.213818in}}{\pgfqpoint{2.439652in}{3.224417in}}{\pgfqpoint{2.439652in}{3.235467in}}%
\pgfpathcurveto{\pgfqpoint{2.439652in}{3.246517in}}{\pgfqpoint{2.435262in}{3.257116in}}{\pgfqpoint{2.427448in}{3.264930in}}%
\pgfpathcurveto{\pgfqpoint{2.419635in}{3.272743in}}{\pgfqpoint{2.409036in}{3.277134in}}{\pgfqpoint{2.397985in}{3.277134in}}%
\pgfpathcurveto{\pgfqpoint{2.386935in}{3.277134in}}{\pgfqpoint{2.376336in}{3.272743in}}{\pgfqpoint{2.368523in}{3.264930in}}%
\pgfpathcurveto{\pgfqpoint{2.360709in}{3.257116in}}{\pgfqpoint{2.356319in}{3.246517in}}{\pgfqpoint{2.356319in}{3.235467in}}%
\pgfpathcurveto{\pgfqpoint{2.356319in}{3.224417in}}{\pgfqpoint{2.360709in}{3.213818in}}{\pgfqpoint{2.368523in}{3.206004in}}%
\pgfpathcurveto{\pgfqpoint{2.376336in}{3.198191in}}{\pgfqpoint{2.386935in}{3.193800in}}{\pgfqpoint{2.397985in}{3.193800in}}%
\pgfpathclose%
\pgfusepath{stroke,fill}%
\end{pgfscope}%
\begin{pgfscope}%
\pgfpathrectangle{\pgfqpoint{0.648703in}{0.548769in}}{\pgfqpoint{5.201297in}{3.102590in}}%
\pgfusepath{clip}%
\pgfsetbuttcap%
\pgfsetroundjoin%
\definecolor{currentfill}{rgb}{0.121569,0.466667,0.705882}%
\pgfsetfillcolor{currentfill}%
\pgfsetlinewidth{1.003750pt}%
\definecolor{currentstroke}{rgb}{0.121569,0.466667,0.705882}%
\pgfsetstrokecolor{currentstroke}%
\pgfsetdash{}{0pt}%
\pgfpathmoveto{\pgfqpoint{1.620118in}{0.656586in}}%
\pgfpathcurveto{\pgfqpoint{1.631169in}{0.656586in}}{\pgfqpoint{1.641768in}{0.660977in}}{\pgfqpoint{1.649581in}{0.668790in}}%
\pgfpathcurveto{\pgfqpoint{1.657395in}{0.676604in}}{\pgfqpoint{1.661785in}{0.687203in}}{\pgfqpoint{1.661785in}{0.698253in}}%
\pgfpathcurveto{\pgfqpoint{1.661785in}{0.709303in}}{\pgfqpoint{1.657395in}{0.719902in}}{\pgfqpoint{1.649581in}{0.727716in}}%
\pgfpathcurveto{\pgfqpoint{1.641768in}{0.735529in}}{\pgfqpoint{1.631169in}{0.739920in}}{\pgfqpoint{1.620118in}{0.739920in}}%
\pgfpathcurveto{\pgfqpoint{1.609068in}{0.739920in}}{\pgfqpoint{1.598469in}{0.735529in}}{\pgfqpoint{1.590656in}{0.727716in}}%
\pgfpathcurveto{\pgfqpoint{1.582842in}{0.719902in}}{\pgfqpoint{1.578452in}{0.709303in}}{\pgfqpoint{1.578452in}{0.698253in}}%
\pgfpathcurveto{\pgfqpoint{1.578452in}{0.687203in}}{\pgfqpoint{1.582842in}{0.676604in}}{\pgfqpoint{1.590656in}{0.668790in}}%
\pgfpathcurveto{\pgfqpoint{1.598469in}{0.660977in}}{\pgfqpoint{1.609068in}{0.656586in}}{\pgfqpoint{1.620118in}{0.656586in}}%
\pgfpathclose%
\pgfusepath{stroke,fill}%
\end{pgfscope}%
\begin{pgfscope}%
\pgfpathrectangle{\pgfqpoint{0.648703in}{0.548769in}}{\pgfqpoint{5.201297in}{3.102590in}}%
\pgfusepath{clip}%
\pgfsetbuttcap%
\pgfsetroundjoin%
\definecolor{currentfill}{rgb}{1.000000,0.498039,0.054902}%
\pgfsetfillcolor{currentfill}%
\pgfsetlinewidth{1.003750pt}%
\definecolor{currentstroke}{rgb}{1.000000,0.498039,0.054902}%
\pgfsetstrokecolor{currentstroke}%
\pgfsetdash{}{0pt}%
\pgfpathmoveto{\pgfqpoint{1.892678in}{3.214944in}}%
\pgfpathcurveto{\pgfqpoint{1.903728in}{3.214944in}}{\pgfqpoint{1.914327in}{3.219334in}}{\pgfqpoint{1.922141in}{3.227148in}}%
\pgfpathcurveto{\pgfqpoint{1.929955in}{3.234961in}}{\pgfqpoint{1.934345in}{3.245560in}}{\pgfqpoint{1.934345in}{3.256610in}}%
\pgfpathcurveto{\pgfqpoint{1.934345in}{3.267661in}}{\pgfqpoint{1.929955in}{3.278260in}}{\pgfqpoint{1.922141in}{3.286073in}}%
\pgfpathcurveto{\pgfqpoint{1.914327in}{3.293887in}}{\pgfqpoint{1.903728in}{3.298277in}}{\pgfqpoint{1.892678in}{3.298277in}}%
\pgfpathcurveto{\pgfqpoint{1.881628in}{3.298277in}}{\pgfqpoint{1.871029in}{3.293887in}}{\pgfqpoint{1.863215in}{3.286073in}}%
\pgfpathcurveto{\pgfqpoint{1.855402in}{3.278260in}}{\pgfqpoint{1.851011in}{3.267661in}}{\pgfqpoint{1.851011in}{3.256610in}}%
\pgfpathcurveto{\pgfqpoint{1.851011in}{3.245560in}}{\pgfqpoint{1.855402in}{3.234961in}}{\pgfqpoint{1.863215in}{3.227148in}}%
\pgfpathcurveto{\pgfqpoint{1.871029in}{3.219334in}}{\pgfqpoint{1.881628in}{3.214944in}}{\pgfqpoint{1.892678in}{3.214944in}}%
\pgfpathclose%
\pgfusepath{stroke,fill}%
\end{pgfscope}%
\begin{pgfscope}%
\pgfpathrectangle{\pgfqpoint{0.648703in}{0.548769in}}{\pgfqpoint{5.201297in}{3.102590in}}%
\pgfusepath{clip}%
\pgfsetbuttcap%
\pgfsetroundjoin%
\definecolor{currentfill}{rgb}{1.000000,0.498039,0.054902}%
\pgfsetfillcolor{currentfill}%
\pgfsetlinewidth{1.003750pt}%
\definecolor{currentstroke}{rgb}{1.000000,0.498039,0.054902}%
\pgfsetstrokecolor{currentstroke}%
\pgfsetdash{}{0pt}%
\pgfpathmoveto{\pgfqpoint{1.525182in}{3.202258in}}%
\pgfpathcurveto{\pgfqpoint{1.536232in}{3.202258in}}{\pgfqpoint{1.546831in}{3.206648in}}{\pgfqpoint{1.554645in}{3.214462in}}%
\pgfpathcurveto{\pgfqpoint{1.562458in}{3.222275in}}{\pgfqpoint{1.566849in}{3.232874in}}{\pgfqpoint{1.566849in}{3.243924in}}%
\pgfpathcurveto{\pgfqpoint{1.566849in}{3.254974in}}{\pgfqpoint{1.562458in}{3.265573in}}{\pgfqpoint{1.554645in}{3.273387in}}%
\pgfpathcurveto{\pgfqpoint{1.546831in}{3.281201in}}{\pgfqpoint{1.536232in}{3.285591in}}{\pgfqpoint{1.525182in}{3.285591in}}%
\pgfpathcurveto{\pgfqpoint{1.514132in}{3.285591in}}{\pgfqpoint{1.503533in}{3.281201in}}{\pgfqpoint{1.495719in}{3.273387in}}%
\pgfpathcurveto{\pgfqpoint{1.487906in}{3.265573in}}{\pgfqpoint{1.483515in}{3.254974in}}{\pgfqpoint{1.483515in}{3.243924in}}%
\pgfpathcurveto{\pgfqpoint{1.483515in}{3.232874in}}{\pgfqpoint{1.487906in}{3.222275in}}{\pgfqpoint{1.495719in}{3.214462in}}%
\pgfpathcurveto{\pgfqpoint{1.503533in}{3.206648in}}{\pgfqpoint{1.514132in}{3.202258in}}{\pgfqpoint{1.525182in}{3.202258in}}%
\pgfpathclose%
\pgfusepath{stroke,fill}%
\end{pgfscope}%
\begin{pgfscope}%
\pgfpathrectangle{\pgfqpoint{0.648703in}{0.548769in}}{\pgfqpoint{5.201297in}{3.102590in}}%
\pgfusepath{clip}%
\pgfsetbuttcap%
\pgfsetroundjoin%
\definecolor{currentfill}{rgb}{1.000000,0.498039,0.054902}%
\pgfsetfillcolor{currentfill}%
\pgfsetlinewidth{1.003750pt}%
\definecolor{currentstroke}{rgb}{1.000000,0.498039,0.054902}%
\pgfsetstrokecolor{currentstroke}%
\pgfsetdash{}{0pt}%
\pgfpathmoveto{\pgfqpoint{1.433308in}{3.244545in}}%
\pgfpathcurveto{\pgfqpoint{1.444358in}{3.244545in}}{\pgfqpoint{1.454957in}{3.248935in}}{\pgfqpoint{1.462771in}{3.256748in}}%
\pgfpathcurveto{\pgfqpoint{1.470584in}{3.264562in}}{\pgfqpoint{1.474975in}{3.275161in}}{\pgfqpoint{1.474975in}{3.286211in}}%
\pgfpathcurveto{\pgfqpoint{1.474975in}{3.297261in}}{\pgfqpoint{1.470584in}{3.307860in}}{\pgfqpoint{1.462771in}{3.315674in}}%
\pgfpathcurveto{\pgfqpoint{1.454957in}{3.323488in}}{\pgfqpoint{1.444358in}{3.327878in}}{\pgfqpoint{1.433308in}{3.327878in}}%
\pgfpathcurveto{\pgfqpoint{1.422258in}{3.327878in}}{\pgfqpoint{1.411659in}{3.323488in}}{\pgfqpoint{1.403845in}{3.315674in}}%
\pgfpathcurveto{\pgfqpoint{1.396031in}{3.307860in}}{\pgfqpoint{1.391641in}{3.297261in}}{\pgfqpoint{1.391641in}{3.286211in}}%
\pgfpathcurveto{\pgfqpoint{1.391641in}{3.275161in}}{\pgfqpoint{1.396031in}{3.264562in}}{\pgfqpoint{1.403845in}{3.256748in}}%
\pgfpathcurveto{\pgfqpoint{1.411659in}{3.248935in}}{\pgfqpoint{1.422258in}{3.244545in}}{\pgfqpoint{1.433308in}{3.244545in}}%
\pgfpathclose%
\pgfusepath{stroke,fill}%
\end{pgfscope}%
\begin{pgfscope}%
\pgfpathrectangle{\pgfqpoint{0.648703in}{0.548769in}}{\pgfqpoint{5.201297in}{3.102590in}}%
\pgfusepath{clip}%
\pgfsetbuttcap%
\pgfsetroundjoin%
\definecolor{currentfill}{rgb}{1.000000,0.498039,0.054902}%
\pgfsetfillcolor{currentfill}%
\pgfsetlinewidth{1.003750pt}%
\definecolor{currentstroke}{rgb}{1.000000,0.498039,0.054902}%
\pgfsetstrokecolor{currentstroke}%
\pgfsetdash{}{0pt}%
\pgfpathmoveto{\pgfqpoint{2.096114in}{3.312204in}}%
\pgfpathcurveto{\pgfqpoint{2.107164in}{3.312204in}}{\pgfqpoint{2.117763in}{3.316594in}}{\pgfqpoint{2.125576in}{3.324407in}}%
\pgfpathcurveto{\pgfqpoint{2.133390in}{3.332221in}}{\pgfqpoint{2.137780in}{3.342820in}}{\pgfqpoint{2.137780in}{3.353870in}}%
\pgfpathcurveto{\pgfqpoint{2.137780in}{3.364920in}}{\pgfqpoint{2.133390in}{3.375519in}}{\pgfqpoint{2.125576in}{3.383333in}}%
\pgfpathcurveto{\pgfqpoint{2.117763in}{3.391147in}}{\pgfqpoint{2.107164in}{3.395537in}}{\pgfqpoint{2.096114in}{3.395537in}}%
\pgfpathcurveto{\pgfqpoint{2.085063in}{3.395537in}}{\pgfqpoint{2.074464in}{3.391147in}}{\pgfqpoint{2.066651in}{3.383333in}}%
\pgfpathcurveto{\pgfqpoint{2.058837in}{3.375519in}}{\pgfqpoint{2.054447in}{3.364920in}}{\pgfqpoint{2.054447in}{3.353870in}}%
\pgfpathcurveto{\pgfqpoint{2.054447in}{3.342820in}}{\pgfqpoint{2.058837in}{3.332221in}}{\pgfqpoint{2.066651in}{3.324407in}}%
\pgfpathcurveto{\pgfqpoint{2.074464in}{3.316594in}}{\pgfqpoint{2.085063in}{3.312204in}}{\pgfqpoint{2.096114in}{3.312204in}}%
\pgfpathclose%
\pgfusepath{stroke,fill}%
\end{pgfscope}%
\begin{pgfscope}%
\pgfpathrectangle{\pgfqpoint{0.648703in}{0.548769in}}{\pgfqpoint{5.201297in}{3.102590in}}%
\pgfusepath{clip}%
\pgfsetbuttcap%
\pgfsetroundjoin%
\definecolor{currentfill}{rgb}{1.000000,0.498039,0.054902}%
\pgfsetfillcolor{currentfill}%
\pgfsetlinewidth{1.003750pt}%
\definecolor{currentstroke}{rgb}{1.000000,0.498039,0.054902}%
\pgfsetstrokecolor{currentstroke}%
\pgfsetdash{}{0pt}%
\pgfpathmoveto{\pgfqpoint{4.878585in}{3.210715in}}%
\pgfpathcurveto{\pgfqpoint{4.889635in}{3.210715in}}{\pgfqpoint{4.900234in}{3.215105in}}{\pgfqpoint{4.908048in}{3.222919in}}%
\pgfpathcurveto{\pgfqpoint{4.915861in}{3.230733in}}{\pgfqpoint{4.920252in}{3.241332in}}{\pgfqpoint{4.920252in}{3.252382in}}%
\pgfpathcurveto{\pgfqpoint{4.920252in}{3.263432in}}{\pgfqpoint{4.915861in}{3.274031in}}{\pgfqpoint{4.908048in}{3.281844in}}%
\pgfpathcurveto{\pgfqpoint{4.900234in}{3.289658in}}{\pgfqpoint{4.889635in}{3.294048in}}{\pgfqpoint{4.878585in}{3.294048in}}%
\pgfpathcurveto{\pgfqpoint{4.867535in}{3.294048in}}{\pgfqpoint{4.856936in}{3.289658in}}{\pgfqpoint{4.849122in}{3.281844in}}%
\pgfpathcurveto{\pgfqpoint{4.841309in}{3.274031in}}{\pgfqpoint{4.836918in}{3.263432in}}{\pgfqpoint{4.836918in}{3.252382in}}%
\pgfpathcurveto{\pgfqpoint{4.836918in}{3.241332in}}{\pgfqpoint{4.841309in}{3.230733in}}{\pgfqpoint{4.849122in}{3.222919in}}%
\pgfpathcurveto{\pgfqpoint{4.856936in}{3.215105in}}{\pgfqpoint{4.867535in}{3.210715in}}{\pgfqpoint{4.878585in}{3.210715in}}%
\pgfpathclose%
\pgfusepath{stroke,fill}%
\end{pgfscope}%
\begin{pgfscope}%
\pgfpathrectangle{\pgfqpoint{0.648703in}{0.548769in}}{\pgfqpoint{5.201297in}{3.102590in}}%
\pgfusepath{clip}%
\pgfsetbuttcap%
\pgfsetroundjoin%
\definecolor{currentfill}{rgb}{1.000000,0.498039,0.054902}%
\pgfsetfillcolor{currentfill}%
\pgfsetlinewidth{1.003750pt}%
\definecolor{currentstroke}{rgb}{1.000000,0.498039,0.054902}%
\pgfsetstrokecolor{currentstroke}%
\pgfsetdash{}{0pt}%
\pgfpathmoveto{\pgfqpoint{1.514581in}{3.236087in}}%
\pgfpathcurveto{\pgfqpoint{1.525631in}{3.236087in}}{\pgfqpoint{1.536230in}{3.240477in}}{\pgfqpoint{1.544044in}{3.248291in}}%
\pgfpathcurveto{\pgfqpoint{1.551857in}{3.256105in}}{\pgfqpoint{1.556248in}{3.266704in}}{\pgfqpoint{1.556248in}{3.277754in}}%
\pgfpathcurveto{\pgfqpoint{1.556248in}{3.288804in}}{\pgfqpoint{1.551857in}{3.299403in}}{\pgfqpoint{1.544044in}{3.307217in}}%
\pgfpathcurveto{\pgfqpoint{1.536230in}{3.315030in}}{\pgfqpoint{1.525631in}{3.319421in}}{\pgfqpoint{1.514581in}{3.319421in}}%
\pgfpathcurveto{\pgfqpoint{1.503531in}{3.319421in}}{\pgfqpoint{1.492932in}{3.315030in}}{\pgfqpoint{1.485118in}{3.307217in}}%
\pgfpathcurveto{\pgfqpoint{1.477305in}{3.299403in}}{\pgfqpoint{1.472914in}{3.288804in}}{\pgfqpoint{1.472914in}{3.277754in}}%
\pgfpathcurveto{\pgfqpoint{1.472914in}{3.266704in}}{\pgfqpoint{1.477305in}{3.256105in}}{\pgfqpoint{1.485118in}{3.248291in}}%
\pgfpathcurveto{\pgfqpoint{1.492932in}{3.240477in}}{\pgfqpoint{1.503531in}{3.236087in}}{\pgfqpoint{1.514581in}{3.236087in}}%
\pgfpathclose%
\pgfusepath{stroke,fill}%
\end{pgfscope}%
\begin{pgfscope}%
\pgfpathrectangle{\pgfqpoint{0.648703in}{0.548769in}}{\pgfqpoint{5.201297in}{3.102590in}}%
\pgfusepath{clip}%
\pgfsetbuttcap%
\pgfsetroundjoin%
\definecolor{currentfill}{rgb}{0.121569,0.466667,0.705882}%
\pgfsetfillcolor{currentfill}%
\pgfsetlinewidth{1.003750pt}%
\definecolor{currentstroke}{rgb}{0.121569,0.466667,0.705882}%
\pgfsetstrokecolor{currentstroke}%
\pgfsetdash{}{0pt}%
\pgfpathmoveto{\pgfqpoint{1.326121in}{0.648129in}}%
\pgfpathcurveto{\pgfqpoint{1.337172in}{0.648129in}}{\pgfqpoint{1.347771in}{0.652519in}}{\pgfqpoint{1.355584in}{0.660333in}}%
\pgfpathcurveto{\pgfqpoint{1.363398in}{0.668146in}}{\pgfqpoint{1.367788in}{0.678745in}}{\pgfqpoint{1.367788in}{0.689796in}}%
\pgfpathcurveto{\pgfqpoint{1.367788in}{0.700846in}}{\pgfqpoint{1.363398in}{0.711445in}}{\pgfqpoint{1.355584in}{0.719258in}}%
\pgfpathcurveto{\pgfqpoint{1.347771in}{0.727072in}}{\pgfqpoint{1.337172in}{0.731462in}}{\pgfqpoint{1.326121in}{0.731462in}}%
\pgfpathcurveto{\pgfqpoint{1.315071in}{0.731462in}}{\pgfqpoint{1.304472in}{0.727072in}}{\pgfqpoint{1.296659in}{0.719258in}}%
\pgfpathcurveto{\pgfqpoint{1.288845in}{0.711445in}}{\pgfqpoint{1.284455in}{0.700846in}}{\pgfqpoint{1.284455in}{0.689796in}}%
\pgfpathcurveto{\pgfqpoint{1.284455in}{0.678745in}}{\pgfqpoint{1.288845in}{0.668146in}}{\pgfqpoint{1.296659in}{0.660333in}}%
\pgfpathcurveto{\pgfqpoint{1.304472in}{0.652519in}}{\pgfqpoint{1.315071in}{0.648129in}}{\pgfqpoint{1.326121in}{0.648129in}}%
\pgfpathclose%
\pgfusepath{stroke,fill}%
\end{pgfscope}%
\begin{pgfscope}%
\pgfpathrectangle{\pgfqpoint{0.648703in}{0.548769in}}{\pgfqpoint{5.201297in}{3.102590in}}%
\pgfusepath{clip}%
\pgfsetbuttcap%
\pgfsetroundjoin%
\definecolor{currentfill}{rgb}{0.121569,0.466667,0.705882}%
\pgfsetfillcolor{currentfill}%
\pgfsetlinewidth{1.003750pt}%
\definecolor{currentstroke}{rgb}{0.121569,0.466667,0.705882}%
\pgfsetstrokecolor{currentstroke}%
\pgfsetdash{}{0pt}%
\pgfpathmoveto{\pgfqpoint{0.999458in}{0.648129in}}%
\pgfpathcurveto{\pgfqpoint{1.010508in}{0.648129in}}{\pgfqpoint{1.021107in}{0.652519in}}{\pgfqpoint{1.028921in}{0.660333in}}%
\pgfpathcurveto{\pgfqpoint{1.036735in}{0.668146in}}{\pgfqpoint{1.041125in}{0.678745in}}{\pgfqpoint{1.041125in}{0.689796in}}%
\pgfpathcurveto{\pgfqpoint{1.041125in}{0.700846in}}{\pgfqpoint{1.036735in}{0.711445in}}{\pgfqpoint{1.028921in}{0.719258in}}%
\pgfpathcurveto{\pgfqpoint{1.021107in}{0.727072in}}{\pgfqpoint{1.010508in}{0.731462in}}{\pgfqpoint{0.999458in}{0.731462in}}%
\pgfpathcurveto{\pgfqpoint{0.988408in}{0.731462in}}{\pgfqpoint{0.977809in}{0.727072in}}{\pgfqpoint{0.969995in}{0.719258in}}%
\pgfpathcurveto{\pgfqpoint{0.962182in}{0.711445in}}{\pgfqpoint{0.957791in}{0.700846in}}{\pgfqpoint{0.957791in}{0.689796in}}%
\pgfpathcurveto{\pgfqpoint{0.957791in}{0.678745in}}{\pgfqpoint{0.962182in}{0.668146in}}{\pgfqpoint{0.969995in}{0.660333in}}%
\pgfpathcurveto{\pgfqpoint{0.977809in}{0.652519in}}{\pgfqpoint{0.988408in}{0.648129in}}{\pgfqpoint{0.999458in}{0.648129in}}%
\pgfpathclose%
\pgfusepath{stroke,fill}%
\end{pgfscope}%
\begin{pgfscope}%
\pgfpathrectangle{\pgfqpoint{0.648703in}{0.548769in}}{\pgfqpoint{5.201297in}{3.102590in}}%
\pgfusepath{clip}%
\pgfsetbuttcap%
\pgfsetroundjoin%
\definecolor{currentfill}{rgb}{0.121569,0.466667,0.705882}%
\pgfsetfillcolor{currentfill}%
\pgfsetlinewidth{1.003750pt}%
\definecolor{currentstroke}{rgb}{0.121569,0.466667,0.705882}%
\pgfsetstrokecolor{currentstroke}%
\pgfsetdash{}{0pt}%
\pgfpathmoveto{\pgfqpoint{1.676118in}{0.648129in}}%
\pgfpathcurveto{\pgfqpoint{1.687168in}{0.648129in}}{\pgfqpoint{1.697767in}{0.652519in}}{\pgfqpoint{1.705581in}{0.660333in}}%
\pgfpathcurveto{\pgfqpoint{1.713394in}{0.668146in}}{\pgfqpoint{1.717785in}{0.678745in}}{\pgfqpoint{1.717785in}{0.689796in}}%
\pgfpathcurveto{\pgfqpoint{1.717785in}{0.700846in}}{\pgfqpoint{1.713394in}{0.711445in}}{\pgfqpoint{1.705581in}{0.719258in}}%
\pgfpathcurveto{\pgfqpoint{1.697767in}{0.727072in}}{\pgfqpoint{1.687168in}{0.731462in}}{\pgfqpoint{1.676118in}{0.731462in}}%
\pgfpathcurveto{\pgfqpoint{1.665068in}{0.731462in}}{\pgfqpoint{1.654469in}{0.727072in}}{\pgfqpoint{1.646655in}{0.719258in}}%
\pgfpathcurveto{\pgfqpoint{1.638841in}{0.711445in}}{\pgfqpoint{1.634451in}{0.700846in}}{\pgfqpoint{1.634451in}{0.689796in}}%
\pgfpathcurveto{\pgfqpoint{1.634451in}{0.678745in}}{\pgfqpoint{1.638841in}{0.668146in}}{\pgfqpoint{1.646655in}{0.660333in}}%
\pgfpathcurveto{\pgfqpoint{1.654469in}{0.652519in}}{\pgfqpoint{1.665068in}{0.648129in}}{\pgfqpoint{1.676118in}{0.648129in}}%
\pgfpathclose%
\pgfusepath{stroke,fill}%
\end{pgfscope}%
\begin{pgfscope}%
\pgfpathrectangle{\pgfqpoint{0.648703in}{0.548769in}}{\pgfqpoint{5.201297in}{3.102590in}}%
\pgfusepath{clip}%
\pgfsetbuttcap%
\pgfsetroundjoin%
\definecolor{currentfill}{rgb}{1.000000,0.498039,0.054902}%
\pgfsetfillcolor{currentfill}%
\pgfsetlinewidth{1.003750pt}%
\definecolor{currentstroke}{rgb}{1.000000,0.498039,0.054902}%
\pgfsetstrokecolor{currentstroke}%
\pgfsetdash{}{0pt}%
\pgfpathmoveto{\pgfqpoint{1.118816in}{3.405235in}}%
\pgfpathcurveto{\pgfqpoint{1.129866in}{3.405235in}}{\pgfqpoint{1.140465in}{3.409625in}}{\pgfqpoint{1.148279in}{3.417439in}}%
\pgfpathcurveto{\pgfqpoint{1.156092in}{3.425252in}}{\pgfqpoint{1.160483in}{3.435851in}}{\pgfqpoint{1.160483in}{3.446901in}}%
\pgfpathcurveto{\pgfqpoint{1.160483in}{3.457952in}}{\pgfqpoint{1.156092in}{3.468551in}}{\pgfqpoint{1.148279in}{3.476364in}}%
\pgfpathcurveto{\pgfqpoint{1.140465in}{3.484178in}}{\pgfqpoint{1.129866in}{3.488568in}}{\pgfqpoint{1.118816in}{3.488568in}}%
\pgfpathcurveto{\pgfqpoint{1.107766in}{3.488568in}}{\pgfqpoint{1.097167in}{3.484178in}}{\pgfqpoint{1.089353in}{3.476364in}}%
\pgfpathcurveto{\pgfqpoint{1.081539in}{3.468551in}}{\pgfqpoint{1.077149in}{3.457952in}}{\pgfqpoint{1.077149in}{3.446901in}}%
\pgfpathcurveto{\pgfqpoint{1.077149in}{3.435851in}}{\pgfqpoint{1.081539in}{3.425252in}}{\pgfqpoint{1.089353in}{3.417439in}}%
\pgfpathcurveto{\pgfqpoint{1.097167in}{3.409625in}}{\pgfqpoint{1.107766in}{3.405235in}}{\pgfqpoint{1.118816in}{3.405235in}}%
\pgfpathclose%
\pgfusepath{stroke,fill}%
\end{pgfscope}%
\begin{pgfscope}%
\pgfpathrectangle{\pgfqpoint{0.648703in}{0.548769in}}{\pgfqpoint{5.201297in}{3.102590in}}%
\pgfusepath{clip}%
\pgfsetbuttcap%
\pgfsetroundjoin%
\definecolor{currentfill}{rgb}{1.000000,0.498039,0.054902}%
\pgfsetfillcolor{currentfill}%
\pgfsetlinewidth{1.003750pt}%
\definecolor{currentstroke}{rgb}{1.000000,0.498039,0.054902}%
\pgfsetstrokecolor{currentstroke}%
\pgfsetdash{}{0pt}%
\pgfpathmoveto{\pgfqpoint{1.527884in}{3.193800in}}%
\pgfpathcurveto{\pgfqpoint{1.538934in}{3.193800in}}{\pgfqpoint{1.549533in}{3.198191in}}{\pgfqpoint{1.557347in}{3.206004in}}%
\pgfpathcurveto{\pgfqpoint{1.565160in}{3.213818in}}{\pgfqpoint{1.569551in}{3.224417in}}{\pgfqpoint{1.569551in}{3.235467in}}%
\pgfpathcurveto{\pgfqpoint{1.569551in}{3.246517in}}{\pgfqpoint{1.565160in}{3.257116in}}{\pgfqpoint{1.557347in}{3.264930in}}%
\pgfpathcurveto{\pgfqpoint{1.549533in}{3.272743in}}{\pgfqpoint{1.538934in}{3.277134in}}{\pgfqpoint{1.527884in}{3.277134in}}%
\pgfpathcurveto{\pgfqpoint{1.516834in}{3.277134in}}{\pgfqpoint{1.506235in}{3.272743in}}{\pgfqpoint{1.498421in}{3.264930in}}%
\pgfpathcurveto{\pgfqpoint{1.490608in}{3.257116in}}{\pgfqpoint{1.486217in}{3.246517in}}{\pgfqpoint{1.486217in}{3.235467in}}%
\pgfpathcurveto{\pgfqpoint{1.486217in}{3.224417in}}{\pgfqpoint{1.490608in}{3.213818in}}{\pgfqpoint{1.498421in}{3.206004in}}%
\pgfpathcurveto{\pgfqpoint{1.506235in}{3.198191in}}{\pgfqpoint{1.516834in}{3.193800in}}{\pgfqpoint{1.527884in}{3.193800in}}%
\pgfpathclose%
\pgfusepath{stroke,fill}%
\end{pgfscope}%
\begin{pgfscope}%
\pgfpathrectangle{\pgfqpoint{0.648703in}{0.548769in}}{\pgfqpoint{5.201297in}{3.102590in}}%
\pgfusepath{clip}%
\pgfsetbuttcap%
\pgfsetroundjoin%
\definecolor{currentfill}{rgb}{1.000000,0.498039,0.054902}%
\pgfsetfillcolor{currentfill}%
\pgfsetlinewidth{1.003750pt}%
\definecolor{currentstroke}{rgb}{1.000000,0.498039,0.054902}%
\pgfsetstrokecolor{currentstroke}%
\pgfsetdash{}{0pt}%
\pgfpathmoveto{\pgfqpoint{1.231892in}{3.358719in}}%
\pgfpathcurveto{\pgfqpoint{1.242942in}{3.358719in}}{\pgfqpoint{1.253541in}{3.363109in}}{\pgfqpoint{1.261354in}{3.370923in}}%
\pgfpathcurveto{\pgfqpoint{1.269168in}{3.378737in}}{\pgfqpoint{1.273558in}{3.389336in}}{\pgfqpoint{1.273558in}{3.400386in}}%
\pgfpathcurveto{\pgfqpoint{1.273558in}{3.411436in}}{\pgfqpoint{1.269168in}{3.422035in}}{\pgfqpoint{1.261354in}{3.429849in}}%
\pgfpathcurveto{\pgfqpoint{1.253541in}{3.437662in}}{\pgfqpoint{1.242942in}{3.442053in}}{\pgfqpoint{1.231892in}{3.442053in}}%
\pgfpathcurveto{\pgfqpoint{1.220842in}{3.442053in}}{\pgfqpoint{1.210242in}{3.437662in}}{\pgfqpoint{1.202429in}{3.429849in}}%
\pgfpathcurveto{\pgfqpoint{1.194615in}{3.422035in}}{\pgfqpoint{1.190225in}{3.411436in}}{\pgfqpoint{1.190225in}{3.400386in}}%
\pgfpathcurveto{\pgfqpoint{1.190225in}{3.389336in}}{\pgfqpoint{1.194615in}{3.378737in}}{\pgfqpoint{1.202429in}{3.370923in}}%
\pgfpathcurveto{\pgfqpoint{1.210242in}{3.363109in}}{\pgfqpoint{1.220842in}{3.358719in}}{\pgfqpoint{1.231892in}{3.358719in}}%
\pgfpathclose%
\pgfusepath{stroke,fill}%
\end{pgfscope}%
\begin{pgfscope}%
\pgfpathrectangle{\pgfqpoint{0.648703in}{0.548769in}}{\pgfqpoint{5.201297in}{3.102590in}}%
\pgfusepath{clip}%
\pgfsetbuttcap%
\pgfsetroundjoin%
\definecolor{currentfill}{rgb}{0.121569,0.466667,0.705882}%
\pgfsetfillcolor{currentfill}%
\pgfsetlinewidth{1.003750pt}%
\definecolor{currentstroke}{rgb}{0.121569,0.466667,0.705882}%
\pgfsetstrokecolor{currentstroke}%
\pgfsetdash{}{0pt}%
\pgfpathmoveto{\pgfqpoint{2.306111in}{0.648129in}}%
\pgfpathcurveto{\pgfqpoint{2.317162in}{0.648129in}}{\pgfqpoint{2.327761in}{0.652519in}}{\pgfqpoint{2.335574in}{0.660333in}}%
\pgfpathcurveto{\pgfqpoint{2.343388in}{0.668146in}}{\pgfqpoint{2.347778in}{0.678745in}}{\pgfqpoint{2.347778in}{0.689796in}}%
\pgfpathcurveto{\pgfqpoint{2.347778in}{0.700846in}}{\pgfqpoint{2.343388in}{0.711445in}}{\pgfqpoint{2.335574in}{0.719258in}}%
\pgfpathcurveto{\pgfqpoint{2.327761in}{0.727072in}}{\pgfqpoint{2.317162in}{0.731462in}}{\pgfqpoint{2.306111in}{0.731462in}}%
\pgfpathcurveto{\pgfqpoint{2.295061in}{0.731462in}}{\pgfqpoint{2.284462in}{0.727072in}}{\pgfqpoint{2.276649in}{0.719258in}}%
\pgfpathcurveto{\pgfqpoint{2.268835in}{0.711445in}}{\pgfqpoint{2.264445in}{0.700846in}}{\pgfqpoint{2.264445in}{0.689796in}}%
\pgfpathcurveto{\pgfqpoint{2.264445in}{0.678745in}}{\pgfqpoint{2.268835in}{0.668146in}}{\pgfqpoint{2.276649in}{0.660333in}}%
\pgfpathcurveto{\pgfqpoint{2.284462in}{0.652519in}}{\pgfqpoint{2.295061in}{0.648129in}}{\pgfqpoint{2.306111in}{0.648129in}}%
\pgfpathclose%
\pgfusepath{stroke,fill}%
\end{pgfscope}%
\begin{pgfscope}%
\pgfpathrectangle{\pgfqpoint{0.648703in}{0.548769in}}{\pgfqpoint{5.201297in}{3.102590in}}%
\pgfusepath{clip}%
\pgfsetbuttcap%
\pgfsetroundjoin%
\definecolor{currentfill}{rgb}{0.121569,0.466667,0.705882}%
\pgfsetfillcolor{currentfill}%
\pgfsetlinewidth{1.003750pt}%
\definecolor{currentstroke}{rgb}{0.121569,0.466667,0.705882}%
\pgfsetstrokecolor{currentstroke}%
\pgfsetdash{}{0pt}%
\pgfpathmoveto{\pgfqpoint{1.865116in}{0.796133in}}%
\pgfpathcurveto{\pgfqpoint{1.876166in}{0.796133in}}{\pgfqpoint{1.886765in}{0.800523in}}{\pgfqpoint{1.894579in}{0.808337in}}%
\pgfpathcurveto{\pgfqpoint{1.902392in}{0.816151in}}{\pgfqpoint{1.906783in}{0.826750in}}{\pgfqpoint{1.906783in}{0.837800in}}%
\pgfpathcurveto{\pgfqpoint{1.906783in}{0.848850in}}{\pgfqpoint{1.902392in}{0.859449in}}{\pgfqpoint{1.894579in}{0.867263in}}%
\pgfpathcurveto{\pgfqpoint{1.886765in}{0.875076in}}{\pgfqpoint{1.876166in}{0.879466in}}{\pgfqpoint{1.865116in}{0.879466in}}%
\pgfpathcurveto{\pgfqpoint{1.854066in}{0.879466in}}{\pgfqpoint{1.843467in}{0.875076in}}{\pgfqpoint{1.835653in}{0.867263in}}%
\pgfpathcurveto{\pgfqpoint{1.827840in}{0.859449in}}{\pgfqpoint{1.823449in}{0.848850in}}{\pgfqpoint{1.823449in}{0.837800in}}%
\pgfpathcurveto{\pgfqpoint{1.823449in}{0.826750in}}{\pgfqpoint{1.827840in}{0.816151in}}{\pgfqpoint{1.835653in}{0.808337in}}%
\pgfpathcurveto{\pgfqpoint{1.843467in}{0.800523in}}{\pgfqpoint{1.854066in}{0.796133in}}{\pgfqpoint{1.865116in}{0.796133in}}%
\pgfpathclose%
\pgfusepath{stroke,fill}%
\end{pgfscope}%
\begin{pgfscope}%
\pgfpathrectangle{\pgfqpoint{0.648703in}{0.548769in}}{\pgfqpoint{5.201297in}{3.102590in}}%
\pgfusepath{clip}%
\pgfsetbuttcap%
\pgfsetroundjoin%
\definecolor{currentfill}{rgb}{0.121569,0.466667,0.705882}%
\pgfsetfillcolor{currentfill}%
\pgfsetlinewidth{1.003750pt}%
\definecolor{currentstroke}{rgb}{0.121569,0.466667,0.705882}%
\pgfsetstrokecolor{currentstroke}%
\pgfsetdash{}{0pt}%
\pgfpathmoveto{\pgfqpoint{1.203623in}{3.155742in}}%
\pgfpathcurveto{\pgfqpoint{1.214673in}{3.155742in}}{\pgfqpoint{1.225272in}{3.160132in}}{\pgfqpoint{1.233085in}{3.167946in}}%
\pgfpathcurveto{\pgfqpoint{1.240899in}{3.175760in}}{\pgfqpoint{1.245289in}{3.186359in}}{\pgfqpoint{1.245289in}{3.197409in}}%
\pgfpathcurveto{\pgfqpoint{1.245289in}{3.208459in}}{\pgfqpoint{1.240899in}{3.219058in}}{\pgfqpoint{1.233085in}{3.226872in}}%
\pgfpathcurveto{\pgfqpoint{1.225272in}{3.234685in}}{\pgfqpoint{1.214673in}{3.239075in}}{\pgfqpoint{1.203623in}{3.239075in}}%
\pgfpathcurveto{\pgfqpoint{1.192573in}{3.239075in}}{\pgfqpoint{1.181974in}{3.234685in}}{\pgfqpoint{1.174160in}{3.226872in}}%
\pgfpathcurveto{\pgfqpoint{1.166346in}{3.219058in}}{\pgfqpoint{1.161956in}{3.208459in}}{\pgfqpoint{1.161956in}{3.197409in}}%
\pgfpathcurveto{\pgfqpoint{1.161956in}{3.186359in}}{\pgfqpoint{1.166346in}{3.175760in}}{\pgfqpoint{1.174160in}{3.167946in}}%
\pgfpathcurveto{\pgfqpoint{1.181974in}{3.160132in}}{\pgfqpoint{1.192573in}{3.155742in}}{\pgfqpoint{1.203623in}{3.155742in}}%
\pgfpathclose%
\pgfusepath{stroke,fill}%
\end{pgfscope}%
\begin{pgfscope}%
\pgfpathrectangle{\pgfqpoint{0.648703in}{0.548769in}}{\pgfqpoint{5.201297in}{3.102590in}}%
\pgfusepath{clip}%
\pgfsetbuttcap%
\pgfsetroundjoin%
\definecolor{currentfill}{rgb}{0.121569,0.466667,0.705882}%
\pgfsetfillcolor{currentfill}%
\pgfsetlinewidth{1.003750pt}%
\definecolor{currentstroke}{rgb}{0.121569,0.466667,0.705882}%
\pgfsetstrokecolor{currentstroke}%
\pgfsetdash{}{0pt}%
\pgfpathmoveto{\pgfqpoint{1.466120in}{0.648129in}}%
\pgfpathcurveto{\pgfqpoint{1.477170in}{0.648129in}}{\pgfqpoint{1.487769in}{0.652519in}}{\pgfqpoint{1.495583in}{0.660333in}}%
\pgfpathcurveto{\pgfqpoint{1.503396in}{0.668146in}}{\pgfqpoint{1.507787in}{0.678745in}}{\pgfqpoint{1.507787in}{0.689796in}}%
\pgfpathcurveto{\pgfqpoint{1.507787in}{0.700846in}}{\pgfqpoint{1.503396in}{0.711445in}}{\pgfqpoint{1.495583in}{0.719258in}}%
\pgfpathcurveto{\pgfqpoint{1.487769in}{0.727072in}}{\pgfqpoint{1.477170in}{0.731462in}}{\pgfqpoint{1.466120in}{0.731462in}}%
\pgfpathcurveto{\pgfqpoint{1.455070in}{0.731462in}}{\pgfqpoint{1.444471in}{0.727072in}}{\pgfqpoint{1.436657in}{0.719258in}}%
\pgfpathcurveto{\pgfqpoint{1.428844in}{0.711445in}}{\pgfqpoint{1.424453in}{0.700846in}}{\pgfqpoint{1.424453in}{0.689796in}}%
\pgfpathcurveto{\pgfqpoint{1.424453in}{0.678745in}}{\pgfqpoint{1.428844in}{0.668146in}}{\pgfqpoint{1.436657in}{0.660333in}}%
\pgfpathcurveto{\pgfqpoint{1.444471in}{0.652519in}}{\pgfqpoint{1.455070in}{0.648129in}}{\pgfqpoint{1.466120in}{0.648129in}}%
\pgfpathclose%
\pgfusepath{stroke,fill}%
\end{pgfscope}%
\begin{pgfscope}%
\pgfpathrectangle{\pgfqpoint{0.648703in}{0.548769in}}{\pgfqpoint{5.201297in}{3.102590in}}%
\pgfusepath{clip}%
\pgfsetbuttcap%
\pgfsetroundjoin%
\definecolor{currentfill}{rgb}{0.121569,0.466667,0.705882}%
\pgfsetfillcolor{currentfill}%
\pgfsetlinewidth{1.003750pt}%
\definecolor{currentstroke}{rgb}{0.121569,0.466667,0.705882}%
\pgfsetstrokecolor{currentstroke}%
\pgfsetdash{}{0pt}%
\pgfpathmoveto{\pgfqpoint{5.246081in}{0.648129in}}%
\pgfpathcurveto{\pgfqpoint{5.257131in}{0.648129in}}{\pgfqpoint{5.267730in}{0.652519in}}{\pgfqpoint{5.275544in}{0.660333in}}%
\pgfpathcurveto{\pgfqpoint{5.283358in}{0.668146in}}{\pgfqpoint{5.287748in}{0.678745in}}{\pgfqpoint{5.287748in}{0.689796in}}%
\pgfpathcurveto{\pgfqpoint{5.287748in}{0.700846in}}{\pgfqpoint{5.283358in}{0.711445in}}{\pgfqpoint{5.275544in}{0.719258in}}%
\pgfpathcurveto{\pgfqpoint{5.267730in}{0.727072in}}{\pgfqpoint{5.257131in}{0.731462in}}{\pgfqpoint{5.246081in}{0.731462in}}%
\pgfpathcurveto{\pgfqpoint{5.235031in}{0.731462in}}{\pgfqpoint{5.224432in}{0.727072in}}{\pgfqpoint{5.216618in}{0.719258in}}%
\pgfpathcurveto{\pgfqpoint{5.208805in}{0.711445in}}{\pgfqpoint{5.204415in}{0.700846in}}{\pgfqpoint{5.204415in}{0.689796in}}%
\pgfpathcurveto{\pgfqpoint{5.204415in}{0.678745in}}{\pgfqpoint{5.208805in}{0.668146in}}{\pgfqpoint{5.216618in}{0.660333in}}%
\pgfpathcurveto{\pgfqpoint{5.224432in}{0.652519in}}{\pgfqpoint{5.235031in}{0.648129in}}{\pgfqpoint{5.246081in}{0.648129in}}%
\pgfpathclose%
\pgfusepath{stroke,fill}%
\end{pgfscope}%
\begin{pgfscope}%
\pgfpathrectangle{\pgfqpoint{0.648703in}{0.548769in}}{\pgfqpoint{5.201297in}{3.102590in}}%
\pgfusepath{clip}%
\pgfsetbuttcap%
\pgfsetroundjoin%
\definecolor{currentfill}{rgb}{0.121569,0.466667,0.705882}%
\pgfsetfillcolor{currentfill}%
\pgfsetlinewidth{1.003750pt}%
\definecolor{currentstroke}{rgb}{0.121569,0.466667,0.705882}%
\pgfsetstrokecolor{currentstroke}%
\pgfsetdash{}{0pt}%
\pgfpathmoveto{\pgfqpoint{1.571119in}{2.512981in}}%
\pgfpathcurveto{\pgfqpoint{1.582169in}{2.512981in}}{\pgfqpoint{1.592768in}{2.517371in}}{\pgfqpoint{1.600582in}{2.525185in}}%
\pgfpathcurveto{\pgfqpoint{1.608395in}{2.532999in}}{\pgfqpoint{1.612786in}{2.543598in}}{\pgfqpoint{1.612786in}{2.554648in}}%
\pgfpathcurveto{\pgfqpoint{1.612786in}{2.565698in}}{\pgfqpoint{1.608395in}{2.576297in}}{\pgfqpoint{1.600582in}{2.584111in}}%
\pgfpathcurveto{\pgfqpoint{1.592768in}{2.591924in}}{\pgfqpoint{1.582169in}{2.596315in}}{\pgfqpoint{1.571119in}{2.596315in}}%
\pgfpathcurveto{\pgfqpoint{1.560069in}{2.596315in}}{\pgfqpoint{1.549470in}{2.591924in}}{\pgfqpoint{1.541656in}{2.584111in}}%
\pgfpathcurveto{\pgfqpoint{1.533843in}{2.576297in}}{\pgfqpoint{1.529452in}{2.565698in}}{\pgfqpoint{1.529452in}{2.554648in}}%
\pgfpathcurveto{\pgfqpoint{1.529452in}{2.543598in}}{\pgfqpoint{1.533843in}{2.532999in}}{\pgfqpoint{1.541656in}{2.525185in}}%
\pgfpathcurveto{\pgfqpoint{1.549470in}{2.517371in}}{\pgfqpoint{1.560069in}{2.512981in}}{\pgfqpoint{1.571119in}{2.512981in}}%
\pgfpathclose%
\pgfusepath{stroke,fill}%
\end{pgfscope}%
\begin{pgfscope}%
\pgfpathrectangle{\pgfqpoint{0.648703in}{0.548769in}}{\pgfqpoint{5.201297in}{3.102590in}}%
\pgfusepath{clip}%
\pgfsetbuttcap%
\pgfsetroundjoin%
\definecolor{currentfill}{rgb}{1.000000,0.498039,0.054902}%
\pgfsetfillcolor{currentfill}%
\pgfsetlinewidth{1.003750pt}%
\definecolor{currentstroke}{rgb}{1.000000,0.498039,0.054902}%
\pgfsetstrokecolor{currentstroke}%
\pgfsetdash{}{0pt}%
\pgfpathmoveto{\pgfqpoint{0.885126in}{3.206486in}}%
\pgfpathcurveto{\pgfqpoint{0.896176in}{3.206486in}}{\pgfqpoint{0.906775in}{3.210877in}}{\pgfqpoint{0.914589in}{3.218690in}}%
\pgfpathcurveto{\pgfqpoint{0.922402in}{3.226504in}}{\pgfqpoint{0.926793in}{3.237103in}}{\pgfqpoint{0.926793in}{3.248153in}}%
\pgfpathcurveto{\pgfqpoint{0.926793in}{3.259203in}}{\pgfqpoint{0.922402in}{3.269802in}}{\pgfqpoint{0.914589in}{3.277616in}}%
\pgfpathcurveto{\pgfqpoint{0.906775in}{3.285429in}}{\pgfqpoint{0.896176in}{3.289820in}}{\pgfqpoint{0.885126in}{3.289820in}}%
\pgfpathcurveto{\pgfqpoint{0.874076in}{3.289820in}}{\pgfqpoint{0.863477in}{3.285429in}}{\pgfqpoint{0.855663in}{3.277616in}}%
\pgfpathcurveto{\pgfqpoint{0.847850in}{3.269802in}}{\pgfqpoint{0.843459in}{3.259203in}}{\pgfqpoint{0.843459in}{3.248153in}}%
\pgfpathcurveto{\pgfqpoint{0.843459in}{3.237103in}}{\pgfqpoint{0.847850in}{3.226504in}}{\pgfqpoint{0.855663in}{3.218690in}}%
\pgfpathcurveto{\pgfqpoint{0.863477in}{3.210877in}}{\pgfqpoint{0.874076in}{3.206486in}}{\pgfqpoint{0.885126in}{3.206486in}}%
\pgfpathclose%
\pgfusepath{stroke,fill}%
\end{pgfscope}%
\begin{pgfscope}%
\pgfpathrectangle{\pgfqpoint{0.648703in}{0.548769in}}{\pgfqpoint{5.201297in}{3.102590in}}%
\pgfusepath{clip}%
\pgfsetbuttcap%
\pgfsetroundjoin%
\definecolor{currentfill}{rgb}{1.000000,0.498039,0.054902}%
\pgfsetfillcolor{currentfill}%
\pgfsetlinewidth{1.003750pt}%
\definecolor{currentstroke}{rgb}{1.000000,0.498039,0.054902}%
\pgfsetstrokecolor{currentstroke}%
\pgfsetdash{}{0pt}%
\pgfpathmoveto{\pgfqpoint{2.701877in}{3.198029in}}%
\pgfpathcurveto{\pgfqpoint{2.712927in}{3.198029in}}{\pgfqpoint{2.723526in}{3.202419in}}{\pgfqpoint{2.731339in}{3.210233in}}%
\pgfpathcurveto{\pgfqpoint{2.739153in}{3.218046in}}{\pgfqpoint{2.743543in}{3.228646in}}{\pgfqpoint{2.743543in}{3.239696in}}%
\pgfpathcurveto{\pgfqpoint{2.743543in}{3.250746in}}{\pgfqpoint{2.739153in}{3.261345in}}{\pgfqpoint{2.731339in}{3.269158in}}%
\pgfpathcurveto{\pgfqpoint{2.723526in}{3.276972in}}{\pgfqpoint{2.712927in}{3.281362in}}{\pgfqpoint{2.701877in}{3.281362in}}%
\pgfpathcurveto{\pgfqpoint{2.690826in}{3.281362in}}{\pgfqpoint{2.680227in}{3.276972in}}{\pgfqpoint{2.672414in}{3.269158in}}%
\pgfpathcurveto{\pgfqpoint{2.664600in}{3.261345in}}{\pgfqpoint{2.660210in}{3.250746in}}{\pgfqpoint{2.660210in}{3.239696in}}%
\pgfpathcurveto{\pgfqpoint{2.660210in}{3.228646in}}{\pgfqpoint{2.664600in}{3.218046in}}{\pgfqpoint{2.672414in}{3.210233in}}%
\pgfpathcurveto{\pgfqpoint{2.680227in}{3.202419in}}{\pgfqpoint{2.690826in}{3.198029in}}{\pgfqpoint{2.701877in}{3.198029in}}%
\pgfpathclose%
\pgfusepath{stroke,fill}%
\end{pgfscope}%
\begin{pgfscope}%
\pgfpathrectangle{\pgfqpoint{0.648703in}{0.548769in}}{\pgfqpoint{5.201297in}{3.102590in}}%
\pgfusepath{clip}%
\pgfsetbuttcap%
\pgfsetroundjoin%
\definecolor{currentfill}{rgb}{1.000000,0.498039,0.054902}%
\pgfsetfillcolor{currentfill}%
\pgfsetlinewidth{1.003750pt}%
\definecolor{currentstroke}{rgb}{1.000000,0.498039,0.054902}%
\pgfsetstrokecolor{currentstroke}%
\pgfsetdash{}{0pt}%
\pgfpathmoveto{\pgfqpoint{1.865116in}{3.248773in}}%
\pgfpathcurveto{\pgfqpoint{1.876166in}{3.248773in}}{\pgfqpoint{1.886765in}{3.253164in}}{\pgfqpoint{1.894579in}{3.260977in}}%
\pgfpathcurveto{\pgfqpoint{1.902392in}{3.268791in}}{\pgfqpoint{1.906783in}{3.279390in}}{\pgfqpoint{1.906783in}{3.290440in}}%
\pgfpathcurveto{\pgfqpoint{1.906783in}{3.301490in}}{\pgfqpoint{1.902392in}{3.312089in}}{\pgfqpoint{1.894579in}{3.319903in}}%
\pgfpathcurveto{\pgfqpoint{1.886765in}{3.327716in}}{\pgfqpoint{1.876166in}{3.332107in}}{\pgfqpoint{1.865116in}{3.332107in}}%
\pgfpathcurveto{\pgfqpoint{1.854066in}{3.332107in}}{\pgfqpoint{1.843467in}{3.327716in}}{\pgfqpoint{1.835653in}{3.319903in}}%
\pgfpathcurveto{\pgfqpoint{1.827840in}{3.312089in}}{\pgfqpoint{1.823449in}{3.301490in}}{\pgfqpoint{1.823449in}{3.290440in}}%
\pgfpathcurveto{\pgfqpoint{1.823449in}{3.279390in}}{\pgfqpoint{1.827840in}{3.268791in}}{\pgfqpoint{1.835653in}{3.260977in}}%
\pgfpathcurveto{\pgfqpoint{1.843467in}{3.253164in}}{\pgfqpoint{1.854066in}{3.248773in}}{\pgfqpoint{1.865116in}{3.248773in}}%
\pgfpathclose%
\pgfusepath{stroke,fill}%
\end{pgfscope}%
\begin{pgfscope}%
\pgfpathrectangle{\pgfqpoint{0.648703in}{0.548769in}}{\pgfqpoint{5.201297in}{3.102590in}}%
\pgfusepath{clip}%
\pgfsetbuttcap%
\pgfsetroundjoin%
\definecolor{currentfill}{rgb}{1.000000,0.498039,0.054902}%
\pgfsetfillcolor{currentfill}%
\pgfsetlinewidth{1.003750pt}%
\definecolor{currentstroke}{rgb}{1.000000,0.498039,0.054902}%
\pgfsetstrokecolor{currentstroke}%
\pgfsetdash{}{0pt}%
\pgfpathmoveto{\pgfqpoint{1.175354in}{3.189572in}}%
\pgfpathcurveto{\pgfqpoint{1.186404in}{3.189572in}}{\pgfqpoint{1.197003in}{3.193962in}}{\pgfqpoint{1.204817in}{3.201775in}}%
\pgfpathcurveto{\pgfqpoint{1.212630in}{3.209589in}}{\pgfqpoint{1.217020in}{3.220188in}}{\pgfqpoint{1.217020in}{3.231238in}}%
\pgfpathcurveto{\pgfqpoint{1.217020in}{3.242288in}}{\pgfqpoint{1.212630in}{3.252887in}}{\pgfqpoint{1.204817in}{3.260701in}}%
\pgfpathcurveto{\pgfqpoint{1.197003in}{3.268515in}}{\pgfqpoint{1.186404in}{3.272905in}}{\pgfqpoint{1.175354in}{3.272905in}}%
\pgfpathcurveto{\pgfqpoint{1.164304in}{3.272905in}}{\pgfqpoint{1.153705in}{3.268515in}}{\pgfqpoint{1.145891in}{3.260701in}}%
\pgfpathcurveto{\pgfqpoint{1.138077in}{3.252887in}}{\pgfqpoint{1.133687in}{3.242288in}}{\pgfqpoint{1.133687in}{3.231238in}}%
\pgfpathcurveto{\pgfqpoint{1.133687in}{3.220188in}}{\pgfqpoint{1.138077in}{3.209589in}}{\pgfqpoint{1.145891in}{3.201775in}}%
\pgfpathcurveto{\pgfqpoint{1.153705in}{3.193962in}}{\pgfqpoint{1.164304in}{3.189572in}}{\pgfqpoint{1.175354in}{3.189572in}}%
\pgfpathclose%
\pgfusepath{stroke,fill}%
\end{pgfscope}%
\begin{pgfscope}%
\pgfpathrectangle{\pgfqpoint{0.648703in}{0.548769in}}{\pgfqpoint{5.201297in}{3.102590in}}%
\pgfusepath{clip}%
\pgfsetbuttcap%
\pgfsetroundjoin%
\definecolor{currentfill}{rgb}{0.121569,0.466667,0.705882}%
\pgfsetfillcolor{currentfill}%
\pgfsetlinewidth{1.003750pt}%
\definecolor{currentstroke}{rgb}{0.121569,0.466667,0.705882}%
\pgfsetstrokecolor{currentstroke}%
\pgfsetdash{}{0pt}%
\pgfpathmoveto{\pgfqpoint{1.658618in}{3.181114in}}%
\pgfpathcurveto{\pgfqpoint{1.669668in}{3.181114in}}{\pgfqpoint{1.680267in}{3.185504in}}{\pgfqpoint{1.688081in}{3.193318in}}%
\pgfpathcurveto{\pgfqpoint{1.695894in}{3.201132in}}{\pgfqpoint{1.700285in}{3.211731in}}{\pgfqpoint{1.700285in}{3.222781in}}%
\pgfpathcurveto{\pgfqpoint{1.700285in}{3.233831in}}{\pgfqpoint{1.695894in}{3.244430in}}{\pgfqpoint{1.688081in}{3.252244in}}%
\pgfpathcurveto{\pgfqpoint{1.680267in}{3.260057in}}{\pgfqpoint{1.669668in}{3.264448in}}{\pgfqpoint{1.658618in}{3.264448in}}%
\pgfpathcurveto{\pgfqpoint{1.647568in}{3.264448in}}{\pgfqpoint{1.636969in}{3.260057in}}{\pgfqpoint{1.629155in}{3.252244in}}%
\pgfpathcurveto{\pgfqpoint{1.621342in}{3.244430in}}{\pgfqpoint{1.616951in}{3.233831in}}{\pgfqpoint{1.616951in}{3.222781in}}%
\pgfpathcurveto{\pgfqpoint{1.616951in}{3.211731in}}{\pgfqpoint{1.621342in}{3.201132in}}{\pgfqpoint{1.629155in}{3.193318in}}%
\pgfpathcurveto{\pgfqpoint{1.636969in}{3.185504in}}{\pgfqpoint{1.647568in}{3.181114in}}{\pgfqpoint{1.658618in}{3.181114in}}%
\pgfpathclose%
\pgfusepath{stroke,fill}%
\end{pgfscope}%
\begin{pgfscope}%
\pgfpathrectangle{\pgfqpoint{0.648703in}{0.548769in}}{\pgfqpoint{5.201297in}{3.102590in}}%
\pgfusepath{clip}%
\pgfsetbuttcap%
\pgfsetroundjoin%
\definecolor{currentfill}{rgb}{0.839216,0.152941,0.156863}%
\pgfsetfillcolor{currentfill}%
\pgfsetlinewidth{1.003750pt}%
\definecolor{currentstroke}{rgb}{0.839216,0.152941,0.156863}%
\pgfsetstrokecolor{currentstroke}%
\pgfsetdash{}{0pt}%
\pgfpathmoveto{\pgfqpoint{2.088336in}{3.202258in}}%
\pgfpathcurveto{\pgfqpoint{2.099386in}{3.202258in}}{\pgfqpoint{2.109985in}{3.206648in}}{\pgfqpoint{2.117799in}{3.214462in}}%
\pgfpathcurveto{\pgfqpoint{2.125612in}{3.222275in}}{\pgfqpoint{2.130003in}{3.232874in}}{\pgfqpoint{2.130003in}{3.243924in}}%
\pgfpathcurveto{\pgfqpoint{2.130003in}{3.254974in}}{\pgfqpoint{2.125612in}{3.265573in}}{\pgfqpoint{2.117799in}{3.273387in}}%
\pgfpathcurveto{\pgfqpoint{2.109985in}{3.281201in}}{\pgfqpoint{2.099386in}{3.285591in}}{\pgfqpoint{2.088336in}{3.285591in}}%
\pgfpathcurveto{\pgfqpoint{2.077286in}{3.285591in}}{\pgfqpoint{2.066687in}{3.281201in}}{\pgfqpoint{2.058873in}{3.273387in}}%
\pgfpathcurveto{\pgfqpoint{2.051059in}{3.265573in}}{\pgfqpoint{2.046669in}{3.254974in}}{\pgfqpoint{2.046669in}{3.243924in}}%
\pgfpathcurveto{\pgfqpoint{2.046669in}{3.232874in}}{\pgfqpoint{2.051059in}{3.222275in}}{\pgfqpoint{2.058873in}{3.214462in}}%
\pgfpathcurveto{\pgfqpoint{2.066687in}{3.206648in}}{\pgfqpoint{2.077286in}{3.202258in}}{\pgfqpoint{2.088336in}{3.202258in}}%
\pgfpathclose%
\pgfusepath{stroke,fill}%
\end{pgfscope}%
\begin{pgfscope}%
\pgfpathrectangle{\pgfqpoint{0.648703in}{0.548769in}}{\pgfqpoint{5.201297in}{3.102590in}}%
\pgfusepath{clip}%
\pgfsetbuttcap%
\pgfsetroundjoin%
\definecolor{currentfill}{rgb}{1.000000,0.498039,0.054902}%
\pgfsetfillcolor{currentfill}%
\pgfsetlinewidth{1.003750pt}%
\definecolor{currentstroke}{rgb}{1.000000,0.498039,0.054902}%
\pgfsetstrokecolor{currentstroke}%
\pgfsetdash{}{0pt}%
\pgfpathmoveto{\pgfqpoint{1.249560in}{3.185343in}}%
\pgfpathcurveto{\pgfqpoint{1.260610in}{3.185343in}}{\pgfqpoint{1.271209in}{3.189733in}}{\pgfqpoint{1.279023in}{3.197547in}}%
\pgfpathcurveto{\pgfqpoint{1.286836in}{3.205360in}}{\pgfqpoint{1.291226in}{3.215959in}}{\pgfqpoint{1.291226in}{3.227010in}}%
\pgfpathcurveto{\pgfqpoint{1.291226in}{3.238060in}}{\pgfqpoint{1.286836in}{3.248659in}}{\pgfqpoint{1.279023in}{3.256472in}}%
\pgfpathcurveto{\pgfqpoint{1.271209in}{3.264286in}}{\pgfqpoint{1.260610in}{3.268676in}}{\pgfqpoint{1.249560in}{3.268676in}}%
\pgfpathcurveto{\pgfqpoint{1.238510in}{3.268676in}}{\pgfqpoint{1.227911in}{3.264286in}}{\pgfqpoint{1.220097in}{3.256472in}}%
\pgfpathcurveto{\pgfqpoint{1.212283in}{3.248659in}}{\pgfqpoint{1.207893in}{3.238060in}}{\pgfqpoint{1.207893in}{3.227010in}}%
\pgfpathcurveto{\pgfqpoint{1.207893in}{3.215959in}}{\pgfqpoint{1.212283in}{3.205360in}}{\pgfqpoint{1.220097in}{3.197547in}}%
\pgfpathcurveto{\pgfqpoint{1.227911in}{3.189733in}}{\pgfqpoint{1.238510in}{3.185343in}}{\pgfqpoint{1.249560in}{3.185343in}}%
\pgfpathclose%
\pgfusepath{stroke,fill}%
\end{pgfscope}%
\begin{pgfscope}%
\pgfpathrectangle{\pgfqpoint{0.648703in}{0.548769in}}{\pgfqpoint{5.201297in}{3.102590in}}%
\pgfusepath{clip}%
\pgfsetbuttcap%
\pgfsetroundjoin%
\definecolor{currentfill}{rgb}{1.000000,0.498039,0.054902}%
\pgfsetfillcolor{currentfill}%
\pgfsetlinewidth{1.003750pt}%
\definecolor{currentstroke}{rgb}{1.000000,0.498039,0.054902}%
\pgfsetstrokecolor{currentstroke}%
\pgfsetdash{}{0pt}%
\pgfpathmoveto{\pgfqpoint{2.918605in}{3.278374in}}%
\pgfpathcurveto{\pgfqpoint{2.929655in}{3.278374in}}{\pgfqpoint{2.940254in}{3.282764in}}{\pgfqpoint{2.948068in}{3.290578in}}%
\pgfpathcurveto{\pgfqpoint{2.955882in}{3.298392in}}{\pgfqpoint{2.960272in}{3.308991in}}{\pgfqpoint{2.960272in}{3.320041in}}%
\pgfpathcurveto{\pgfqpoint{2.960272in}{3.331091in}}{\pgfqpoint{2.955882in}{3.341690in}}{\pgfqpoint{2.948068in}{3.349504in}}%
\pgfpathcurveto{\pgfqpoint{2.940254in}{3.357317in}}{\pgfqpoint{2.929655in}{3.361707in}}{\pgfqpoint{2.918605in}{3.361707in}}%
\pgfpathcurveto{\pgfqpoint{2.907555in}{3.361707in}}{\pgfqpoint{2.896956in}{3.357317in}}{\pgfqpoint{2.889142in}{3.349504in}}%
\pgfpathcurveto{\pgfqpoint{2.881329in}{3.341690in}}{\pgfqpoint{2.876938in}{3.331091in}}{\pgfqpoint{2.876938in}{3.320041in}}%
\pgfpathcurveto{\pgfqpoint{2.876938in}{3.308991in}}{\pgfqpoint{2.881329in}{3.298392in}}{\pgfqpoint{2.889142in}{3.290578in}}%
\pgfpathcurveto{\pgfqpoint{2.896956in}{3.282764in}}{\pgfqpoint{2.907555in}{3.278374in}}{\pgfqpoint{2.918605in}{3.278374in}}%
\pgfpathclose%
\pgfusepath{stroke,fill}%
\end{pgfscope}%
\begin{pgfscope}%
\pgfpathrectangle{\pgfqpoint{0.648703in}{0.548769in}}{\pgfqpoint{5.201297in}{3.102590in}}%
\pgfusepath{clip}%
\pgfsetbuttcap%
\pgfsetroundjoin%
\definecolor{currentfill}{rgb}{0.121569,0.466667,0.705882}%
\pgfsetfillcolor{currentfill}%
\pgfsetlinewidth{1.003750pt}%
\definecolor{currentstroke}{rgb}{0.121569,0.466667,0.705882}%
\pgfsetstrokecolor{currentstroke}%
\pgfsetdash{}{0pt}%
\pgfpathmoveto{\pgfqpoint{1.979448in}{3.181114in}}%
\pgfpathcurveto{\pgfqpoint{1.990498in}{3.181114in}}{\pgfqpoint{2.001097in}{3.185504in}}{\pgfqpoint{2.008911in}{3.193318in}}%
\pgfpathcurveto{\pgfqpoint{2.016724in}{3.201132in}}{\pgfqpoint{2.021115in}{3.211731in}}{\pgfqpoint{2.021115in}{3.222781in}}%
\pgfpathcurveto{\pgfqpoint{2.021115in}{3.233831in}}{\pgfqpoint{2.016724in}{3.244430in}}{\pgfqpoint{2.008911in}{3.252244in}}%
\pgfpathcurveto{\pgfqpoint{2.001097in}{3.260057in}}{\pgfqpoint{1.990498in}{3.264448in}}{\pgfqpoint{1.979448in}{3.264448in}}%
\pgfpathcurveto{\pgfqpoint{1.968398in}{3.264448in}}{\pgfqpoint{1.957799in}{3.260057in}}{\pgfqpoint{1.949985in}{3.252244in}}%
\pgfpathcurveto{\pgfqpoint{1.942172in}{3.244430in}}{\pgfqpoint{1.937781in}{3.233831in}}{\pgfqpoint{1.937781in}{3.222781in}}%
\pgfpathcurveto{\pgfqpoint{1.937781in}{3.211731in}}{\pgfqpoint{1.942172in}{3.201132in}}{\pgfqpoint{1.949985in}{3.193318in}}%
\pgfpathcurveto{\pgfqpoint{1.957799in}{3.185504in}}{\pgfqpoint{1.968398in}{3.181114in}}{\pgfqpoint{1.979448in}{3.181114in}}%
\pgfpathclose%
\pgfusepath{stroke,fill}%
\end{pgfscope}%
\begin{pgfscope}%
\pgfpathrectangle{\pgfqpoint{0.648703in}{0.548769in}}{\pgfqpoint{5.201297in}{3.102590in}}%
\pgfusepath{clip}%
\pgfsetbuttcap%
\pgfsetroundjoin%
\definecolor{currentfill}{rgb}{1.000000,0.498039,0.054902}%
\pgfsetfillcolor{currentfill}%
\pgfsetlinewidth{1.003750pt}%
\definecolor{currentstroke}{rgb}{1.000000,0.498039,0.054902}%
\pgfsetstrokecolor{currentstroke}%
\pgfsetdash{}{0pt}%
\pgfpathmoveto{\pgfqpoint{1.005740in}{3.219172in}}%
\pgfpathcurveto{\pgfqpoint{1.016790in}{3.219172in}}{\pgfqpoint{1.027389in}{3.223563in}}{\pgfqpoint{1.035203in}{3.231376in}}%
\pgfpathcurveto{\pgfqpoint{1.043017in}{3.239190in}}{\pgfqpoint{1.047407in}{3.249789in}}{\pgfqpoint{1.047407in}{3.260839in}}%
\pgfpathcurveto{\pgfqpoint{1.047407in}{3.271889in}}{\pgfqpoint{1.043017in}{3.282488in}}{\pgfqpoint{1.035203in}{3.290302in}}%
\pgfpathcurveto{\pgfqpoint{1.027389in}{3.298116in}}{\pgfqpoint{1.016790in}{3.302506in}}{\pgfqpoint{1.005740in}{3.302506in}}%
\pgfpathcurveto{\pgfqpoint{0.994690in}{3.302506in}}{\pgfqpoint{0.984091in}{3.298116in}}{\pgfqpoint{0.976277in}{3.290302in}}%
\pgfpathcurveto{\pgfqpoint{0.968464in}{3.282488in}}{\pgfqpoint{0.964073in}{3.271889in}}{\pgfqpoint{0.964073in}{3.260839in}}%
\pgfpathcurveto{\pgfqpoint{0.964073in}{3.249789in}}{\pgfqpoint{0.968464in}{3.239190in}}{\pgfqpoint{0.976277in}{3.231376in}}%
\pgfpathcurveto{\pgfqpoint{0.984091in}{3.223563in}}{\pgfqpoint{0.994690in}{3.219172in}}{\pgfqpoint{1.005740in}{3.219172in}}%
\pgfpathclose%
\pgfusepath{stroke,fill}%
\end{pgfscope}%
\begin{pgfscope}%
\pgfpathrectangle{\pgfqpoint{0.648703in}{0.548769in}}{\pgfqpoint{5.201297in}{3.102590in}}%
\pgfusepath{clip}%
\pgfsetbuttcap%
\pgfsetroundjoin%
\definecolor{currentfill}{rgb}{1.000000,0.498039,0.054902}%
\pgfsetfillcolor{currentfill}%
\pgfsetlinewidth{1.003750pt}%
\definecolor{currentstroke}{rgb}{1.000000,0.498039,0.054902}%
\pgfsetstrokecolor{currentstroke}%
\pgfsetdash{}{0pt}%
\pgfpathmoveto{\pgfqpoint{2.148613in}{3.468665in}}%
\pgfpathcurveto{\pgfqpoint{2.159663in}{3.468665in}}{\pgfqpoint{2.170262in}{3.473055in}}{\pgfqpoint{2.178076in}{3.480869in}}%
\pgfpathcurveto{\pgfqpoint{2.185889in}{3.488683in}}{\pgfqpoint{2.190280in}{3.499282in}}{\pgfqpoint{2.190280in}{3.510332in}}%
\pgfpathcurveto{\pgfqpoint{2.190280in}{3.521382in}}{\pgfqpoint{2.185889in}{3.531981in}}{\pgfqpoint{2.178076in}{3.539795in}}%
\pgfpathcurveto{\pgfqpoint{2.170262in}{3.547608in}}{\pgfqpoint{2.159663in}{3.551998in}}{\pgfqpoint{2.148613in}{3.551998in}}%
\pgfpathcurveto{\pgfqpoint{2.137563in}{3.551998in}}{\pgfqpoint{2.126964in}{3.547608in}}{\pgfqpoint{2.119150in}{3.539795in}}%
\pgfpathcurveto{\pgfqpoint{2.111337in}{3.531981in}}{\pgfqpoint{2.106946in}{3.521382in}}{\pgfqpoint{2.106946in}{3.510332in}}%
\pgfpathcurveto{\pgfqpoint{2.106946in}{3.499282in}}{\pgfqpoint{2.111337in}{3.488683in}}{\pgfqpoint{2.119150in}{3.480869in}}%
\pgfpathcurveto{\pgfqpoint{2.126964in}{3.473055in}}{\pgfqpoint{2.137563in}{3.468665in}}{\pgfqpoint{2.148613in}{3.468665in}}%
\pgfpathclose%
\pgfusepath{stroke,fill}%
\end{pgfscope}%
\begin{pgfscope}%
\pgfpathrectangle{\pgfqpoint{0.648703in}{0.548769in}}{\pgfqpoint{5.201297in}{3.102590in}}%
\pgfusepath{clip}%
\pgfsetbuttcap%
\pgfsetroundjoin%
\definecolor{currentfill}{rgb}{1.000000,0.498039,0.054902}%
\pgfsetfillcolor{currentfill}%
\pgfsetlinewidth{1.003750pt}%
\definecolor{currentstroke}{rgb}{1.000000,0.498039,0.054902}%
\pgfsetstrokecolor{currentstroke}%
\pgfsetdash{}{0pt}%
\pgfpathmoveto{\pgfqpoint{1.437484in}{3.189572in}}%
\pgfpathcurveto{\pgfqpoint{1.448534in}{3.189572in}}{\pgfqpoint{1.459133in}{3.193962in}}{\pgfqpoint{1.466947in}{3.201775in}}%
\pgfpathcurveto{\pgfqpoint{1.474760in}{3.209589in}}{\pgfqpoint{1.479151in}{3.220188in}}{\pgfqpoint{1.479151in}{3.231238in}}%
\pgfpathcurveto{\pgfqpoint{1.479151in}{3.242288in}}{\pgfqpoint{1.474760in}{3.252887in}}{\pgfqpoint{1.466947in}{3.260701in}}%
\pgfpathcurveto{\pgfqpoint{1.459133in}{3.268515in}}{\pgfqpoint{1.448534in}{3.272905in}}{\pgfqpoint{1.437484in}{3.272905in}}%
\pgfpathcurveto{\pgfqpoint{1.426434in}{3.272905in}}{\pgfqpoint{1.415835in}{3.268515in}}{\pgfqpoint{1.408021in}{3.260701in}}%
\pgfpathcurveto{\pgfqpoint{1.400208in}{3.252887in}}{\pgfqpoint{1.395817in}{3.242288in}}{\pgfqpoint{1.395817in}{3.231238in}}%
\pgfpathcurveto{\pgfqpoint{1.395817in}{3.220188in}}{\pgfqpoint{1.400208in}{3.209589in}}{\pgfqpoint{1.408021in}{3.201775in}}%
\pgfpathcurveto{\pgfqpoint{1.415835in}{3.193962in}}{\pgfqpoint{1.426434in}{3.189572in}}{\pgfqpoint{1.437484in}{3.189572in}}%
\pgfpathclose%
\pgfusepath{stroke,fill}%
\end{pgfscope}%
\begin{pgfscope}%
\pgfpathrectangle{\pgfqpoint{0.648703in}{0.548769in}}{\pgfqpoint{5.201297in}{3.102590in}}%
\pgfusepath{clip}%
\pgfsetbuttcap%
\pgfsetroundjoin%
\definecolor{currentfill}{rgb}{1.000000,0.498039,0.054902}%
\pgfsetfillcolor{currentfill}%
\pgfsetlinewidth{1.003750pt}%
\definecolor{currentstroke}{rgb}{1.000000,0.498039,0.054902}%
\pgfsetstrokecolor{currentstroke}%
\pgfsetdash{}{0pt}%
\pgfpathmoveto{\pgfqpoint{1.404075in}{3.189572in}}%
\pgfpathcurveto{\pgfqpoint{1.415125in}{3.189572in}}{\pgfqpoint{1.425724in}{3.193962in}}{\pgfqpoint{1.433538in}{3.201775in}}%
\pgfpathcurveto{\pgfqpoint{1.441352in}{3.209589in}}{\pgfqpoint{1.445742in}{3.220188in}}{\pgfqpoint{1.445742in}{3.231238in}}%
\pgfpathcurveto{\pgfqpoint{1.445742in}{3.242288in}}{\pgfqpoint{1.441352in}{3.252887in}}{\pgfqpoint{1.433538in}{3.260701in}}%
\pgfpathcurveto{\pgfqpoint{1.425724in}{3.268515in}}{\pgfqpoint{1.415125in}{3.272905in}}{\pgfqpoint{1.404075in}{3.272905in}}%
\pgfpathcurveto{\pgfqpoint{1.393025in}{3.272905in}}{\pgfqpoint{1.382426in}{3.268515in}}{\pgfqpoint{1.374612in}{3.260701in}}%
\pgfpathcurveto{\pgfqpoint{1.366799in}{3.252887in}}{\pgfqpoint{1.362409in}{3.242288in}}{\pgfqpoint{1.362409in}{3.231238in}}%
\pgfpathcurveto{\pgfqpoint{1.362409in}{3.220188in}}{\pgfqpoint{1.366799in}{3.209589in}}{\pgfqpoint{1.374612in}{3.201775in}}%
\pgfpathcurveto{\pgfqpoint{1.382426in}{3.193962in}}{\pgfqpoint{1.393025in}{3.189572in}}{\pgfqpoint{1.404075in}{3.189572in}}%
\pgfpathclose%
\pgfusepath{stroke,fill}%
\end{pgfscope}%
\begin{pgfscope}%
\pgfpathrectangle{\pgfqpoint{0.648703in}{0.548769in}}{\pgfqpoint{5.201297in}{3.102590in}}%
\pgfusepath{clip}%
\pgfsetbuttcap%
\pgfsetroundjoin%
\definecolor{currentfill}{rgb}{0.121569,0.466667,0.705882}%
\pgfsetfillcolor{currentfill}%
\pgfsetlinewidth{1.003750pt}%
\definecolor{currentstroke}{rgb}{0.121569,0.466667,0.705882}%
\pgfsetstrokecolor{currentstroke}%
\pgfsetdash{}{0pt}%
\pgfpathmoveto{\pgfqpoint{1.214759in}{3.181114in}}%
\pgfpathcurveto{\pgfqpoint{1.225809in}{3.181114in}}{\pgfqpoint{1.236408in}{3.185504in}}{\pgfqpoint{1.244222in}{3.193318in}}%
\pgfpathcurveto{\pgfqpoint{1.252035in}{3.201132in}}{\pgfqpoint{1.256426in}{3.211731in}}{\pgfqpoint{1.256426in}{3.222781in}}%
\pgfpathcurveto{\pgfqpoint{1.256426in}{3.233831in}}{\pgfqpoint{1.252035in}{3.244430in}}{\pgfqpoint{1.244222in}{3.252244in}}%
\pgfpathcurveto{\pgfqpoint{1.236408in}{3.260057in}}{\pgfqpoint{1.225809in}{3.264448in}}{\pgfqpoint{1.214759in}{3.264448in}}%
\pgfpathcurveto{\pgfqpoint{1.203709in}{3.264448in}}{\pgfqpoint{1.193110in}{3.260057in}}{\pgfqpoint{1.185296in}{3.252244in}}%
\pgfpathcurveto{\pgfqpoint{1.177483in}{3.244430in}}{\pgfqpoint{1.173092in}{3.233831in}}{\pgfqpoint{1.173092in}{3.222781in}}%
\pgfpathcurveto{\pgfqpoint{1.173092in}{3.211731in}}{\pgfqpoint{1.177483in}{3.201132in}}{\pgfqpoint{1.185296in}{3.193318in}}%
\pgfpathcurveto{\pgfqpoint{1.193110in}{3.185504in}}{\pgfqpoint{1.203709in}{3.181114in}}{\pgfqpoint{1.214759in}{3.181114in}}%
\pgfpathclose%
\pgfusepath{stroke,fill}%
\end{pgfscope}%
\begin{pgfscope}%
\pgfpathrectangle{\pgfqpoint{0.648703in}{0.548769in}}{\pgfqpoint{5.201297in}{3.102590in}}%
\pgfusepath{clip}%
\pgfsetbuttcap%
\pgfsetroundjoin%
\definecolor{currentfill}{rgb}{1.000000,0.498039,0.054902}%
\pgfsetfillcolor{currentfill}%
\pgfsetlinewidth{1.003750pt}%
\definecolor{currentstroke}{rgb}{1.000000,0.498039,0.054902}%
\pgfsetstrokecolor{currentstroke}%
\pgfsetdash{}{0pt}%
\pgfpathmoveto{\pgfqpoint{1.458043in}{3.185343in}}%
\pgfpathcurveto{\pgfqpoint{1.469093in}{3.185343in}}{\pgfqpoint{1.479692in}{3.189733in}}{\pgfqpoint{1.487506in}{3.197547in}}%
\pgfpathcurveto{\pgfqpoint{1.495320in}{3.205360in}}{\pgfqpoint{1.499710in}{3.215959in}}{\pgfqpoint{1.499710in}{3.227010in}}%
\pgfpathcurveto{\pgfqpoint{1.499710in}{3.238060in}}{\pgfqpoint{1.495320in}{3.248659in}}{\pgfqpoint{1.487506in}{3.256472in}}%
\pgfpathcurveto{\pgfqpoint{1.479692in}{3.264286in}}{\pgfqpoint{1.469093in}{3.268676in}}{\pgfqpoint{1.458043in}{3.268676in}}%
\pgfpathcurveto{\pgfqpoint{1.446993in}{3.268676in}}{\pgfqpoint{1.436394in}{3.264286in}}{\pgfqpoint{1.428580in}{3.256472in}}%
\pgfpathcurveto{\pgfqpoint{1.420767in}{3.248659in}}{\pgfqpoint{1.416377in}{3.238060in}}{\pgfqpoint{1.416377in}{3.227010in}}%
\pgfpathcurveto{\pgfqpoint{1.416377in}{3.215959in}}{\pgfqpoint{1.420767in}{3.205360in}}{\pgfqpoint{1.428580in}{3.197547in}}%
\pgfpathcurveto{\pgfqpoint{1.436394in}{3.189733in}}{\pgfqpoint{1.446993in}{3.185343in}}{\pgfqpoint{1.458043in}{3.185343in}}%
\pgfpathclose%
\pgfusepath{stroke,fill}%
\end{pgfscope}%
\begin{pgfscope}%
\pgfpathrectangle{\pgfqpoint{0.648703in}{0.548769in}}{\pgfqpoint{5.201297in}{3.102590in}}%
\pgfusepath{clip}%
\pgfsetbuttcap%
\pgfsetroundjoin%
\definecolor{currentfill}{rgb}{1.000000,0.498039,0.054902}%
\pgfsetfillcolor{currentfill}%
\pgfsetlinewidth{1.003750pt}%
\definecolor{currentstroke}{rgb}{1.000000,0.498039,0.054902}%
\pgfsetstrokecolor{currentstroke}%
\pgfsetdash{}{0pt}%
\pgfpathmoveto{\pgfqpoint{3.426100in}{3.358719in}}%
\pgfpathcurveto{\pgfqpoint{3.437150in}{3.358719in}}{\pgfqpoint{3.447749in}{3.363109in}}{\pgfqpoint{3.455563in}{3.370923in}}%
\pgfpathcurveto{\pgfqpoint{3.463376in}{3.378737in}}{\pgfqpoint{3.467767in}{3.389336in}}{\pgfqpoint{3.467767in}{3.400386in}}%
\pgfpathcurveto{\pgfqpoint{3.467767in}{3.411436in}}{\pgfqpoint{3.463376in}{3.422035in}}{\pgfqpoint{3.455563in}{3.429849in}}%
\pgfpathcurveto{\pgfqpoint{3.447749in}{3.437662in}}{\pgfqpoint{3.437150in}{3.442053in}}{\pgfqpoint{3.426100in}{3.442053in}}%
\pgfpathcurveto{\pgfqpoint{3.415050in}{3.442053in}}{\pgfqpoint{3.404451in}{3.437662in}}{\pgfqpoint{3.396637in}{3.429849in}}%
\pgfpathcurveto{\pgfqpoint{3.388823in}{3.422035in}}{\pgfqpoint{3.384433in}{3.411436in}}{\pgfqpoint{3.384433in}{3.400386in}}%
\pgfpathcurveto{\pgfqpoint{3.384433in}{3.389336in}}{\pgfqpoint{3.388823in}{3.378737in}}{\pgfqpoint{3.396637in}{3.370923in}}%
\pgfpathcurveto{\pgfqpoint{3.404451in}{3.363109in}}{\pgfqpoint{3.415050in}{3.358719in}}{\pgfqpoint{3.426100in}{3.358719in}}%
\pgfpathclose%
\pgfusepath{stroke,fill}%
\end{pgfscope}%
\begin{pgfscope}%
\pgfpathrectangle{\pgfqpoint{0.648703in}{0.548769in}}{\pgfqpoint{5.201297in}{3.102590in}}%
\pgfusepath{clip}%
\pgfsetbuttcap%
\pgfsetroundjoin%
\definecolor{currentfill}{rgb}{1.000000,0.498039,0.054902}%
\pgfsetfillcolor{currentfill}%
\pgfsetlinewidth{1.003750pt}%
\definecolor{currentstroke}{rgb}{1.000000,0.498039,0.054902}%
\pgfsetstrokecolor{currentstroke}%
\pgfsetdash{}{0pt}%
\pgfpathmoveto{\pgfqpoint{1.599388in}{3.198029in}}%
\pgfpathcurveto{\pgfqpoint{1.610438in}{3.198029in}}{\pgfqpoint{1.621037in}{3.202419in}}{\pgfqpoint{1.628851in}{3.210233in}}%
\pgfpathcurveto{\pgfqpoint{1.636664in}{3.218046in}}{\pgfqpoint{1.641055in}{3.228646in}}{\pgfqpoint{1.641055in}{3.239696in}}%
\pgfpathcurveto{\pgfqpoint{1.641055in}{3.250746in}}{\pgfqpoint{1.636664in}{3.261345in}}{\pgfqpoint{1.628851in}{3.269158in}}%
\pgfpathcurveto{\pgfqpoint{1.621037in}{3.276972in}}{\pgfqpoint{1.610438in}{3.281362in}}{\pgfqpoint{1.599388in}{3.281362in}}%
\pgfpathcurveto{\pgfqpoint{1.588338in}{3.281362in}}{\pgfqpoint{1.577739in}{3.276972in}}{\pgfqpoint{1.569925in}{3.269158in}}%
\pgfpathcurveto{\pgfqpoint{1.562111in}{3.261345in}}{\pgfqpoint{1.557721in}{3.250746in}}{\pgfqpoint{1.557721in}{3.239696in}}%
\pgfpathcurveto{\pgfqpoint{1.557721in}{3.228646in}}{\pgfqpoint{1.562111in}{3.218046in}}{\pgfqpoint{1.569925in}{3.210233in}}%
\pgfpathcurveto{\pgfqpoint{1.577739in}{3.202419in}}{\pgfqpoint{1.588338in}{3.198029in}}{\pgfqpoint{1.599388in}{3.198029in}}%
\pgfpathclose%
\pgfusepath{stroke,fill}%
\end{pgfscope}%
\begin{pgfscope}%
\pgfpathrectangle{\pgfqpoint{0.648703in}{0.548769in}}{\pgfqpoint{5.201297in}{3.102590in}}%
\pgfusepath{clip}%
\pgfsetbuttcap%
\pgfsetroundjoin%
\definecolor{currentfill}{rgb}{0.839216,0.152941,0.156863}%
\pgfsetfillcolor{currentfill}%
\pgfsetlinewidth{1.003750pt}%
\definecolor{currentstroke}{rgb}{0.839216,0.152941,0.156863}%
\pgfsetstrokecolor{currentstroke}%
\pgfsetdash{}{0pt}%
\pgfpathmoveto{\pgfqpoint{1.938615in}{3.193800in}}%
\pgfpathcurveto{\pgfqpoint{1.949665in}{3.193800in}}{\pgfqpoint{1.960264in}{3.198191in}}{\pgfqpoint{1.968078in}{3.206004in}}%
\pgfpathcurveto{\pgfqpoint{1.975892in}{3.213818in}}{\pgfqpoint{1.980282in}{3.224417in}}{\pgfqpoint{1.980282in}{3.235467in}}%
\pgfpathcurveto{\pgfqpoint{1.980282in}{3.246517in}}{\pgfqpoint{1.975892in}{3.257116in}}{\pgfqpoint{1.968078in}{3.264930in}}%
\pgfpathcurveto{\pgfqpoint{1.960264in}{3.272743in}}{\pgfqpoint{1.949665in}{3.277134in}}{\pgfqpoint{1.938615in}{3.277134in}}%
\pgfpathcurveto{\pgfqpoint{1.927565in}{3.277134in}}{\pgfqpoint{1.916966in}{3.272743in}}{\pgfqpoint{1.909152in}{3.264930in}}%
\pgfpathcurveto{\pgfqpoint{1.901339in}{3.257116in}}{\pgfqpoint{1.896949in}{3.246517in}}{\pgfqpoint{1.896949in}{3.235467in}}%
\pgfpathcurveto{\pgfqpoint{1.896949in}{3.224417in}}{\pgfqpoint{1.901339in}{3.213818in}}{\pgfqpoint{1.909152in}{3.206004in}}%
\pgfpathcurveto{\pgfqpoint{1.916966in}{3.198191in}}{\pgfqpoint{1.927565in}{3.193800in}}{\pgfqpoint{1.938615in}{3.193800in}}%
\pgfpathclose%
\pgfusepath{stroke,fill}%
\end{pgfscope}%
\begin{pgfscope}%
\pgfpathrectangle{\pgfqpoint{0.648703in}{0.548769in}}{\pgfqpoint{5.201297in}{3.102590in}}%
\pgfusepath{clip}%
\pgfsetbuttcap%
\pgfsetroundjoin%
\definecolor{currentfill}{rgb}{1.000000,0.498039,0.054902}%
\pgfsetfillcolor{currentfill}%
\pgfsetlinewidth{1.003750pt}%
\definecolor{currentstroke}{rgb}{1.000000,0.498039,0.054902}%
\pgfsetstrokecolor{currentstroke}%
\pgfsetdash{}{0pt}%
\pgfpathmoveto{\pgfqpoint{1.489453in}{3.193800in}}%
\pgfpathcurveto{\pgfqpoint{1.500503in}{3.193800in}}{\pgfqpoint{1.511102in}{3.198191in}}{\pgfqpoint{1.518916in}{3.206004in}}%
\pgfpathcurveto{\pgfqpoint{1.526730in}{3.213818in}}{\pgfqpoint{1.531120in}{3.224417in}}{\pgfqpoint{1.531120in}{3.235467in}}%
\pgfpathcurveto{\pgfqpoint{1.531120in}{3.246517in}}{\pgfqpoint{1.526730in}{3.257116in}}{\pgfqpoint{1.518916in}{3.264930in}}%
\pgfpathcurveto{\pgfqpoint{1.511102in}{3.272743in}}{\pgfqpoint{1.500503in}{3.277134in}}{\pgfqpoint{1.489453in}{3.277134in}}%
\pgfpathcurveto{\pgfqpoint{1.478403in}{3.277134in}}{\pgfqpoint{1.467804in}{3.272743in}}{\pgfqpoint{1.459990in}{3.264930in}}%
\pgfpathcurveto{\pgfqpoint{1.452177in}{3.257116in}}{\pgfqpoint{1.447786in}{3.246517in}}{\pgfqpoint{1.447786in}{3.235467in}}%
\pgfpathcurveto{\pgfqpoint{1.447786in}{3.224417in}}{\pgfqpoint{1.452177in}{3.213818in}}{\pgfqpoint{1.459990in}{3.206004in}}%
\pgfpathcurveto{\pgfqpoint{1.467804in}{3.198191in}}{\pgfqpoint{1.478403in}{3.193800in}}{\pgfqpoint{1.489453in}{3.193800in}}%
\pgfpathclose%
\pgfusepath{stroke,fill}%
\end{pgfscope}%
\begin{pgfscope}%
\pgfpathrectangle{\pgfqpoint{0.648703in}{0.548769in}}{\pgfqpoint{5.201297in}{3.102590in}}%
\pgfusepath{clip}%
\pgfsetbuttcap%
\pgfsetroundjoin%
\definecolor{currentfill}{rgb}{1.000000,0.498039,0.054902}%
\pgfsetfillcolor{currentfill}%
\pgfsetlinewidth{1.003750pt}%
\definecolor{currentstroke}{rgb}{1.000000,0.498039,0.054902}%
\pgfsetstrokecolor{currentstroke}%
\pgfsetdash{}{0pt}%
\pgfpathmoveto{\pgfqpoint{1.368362in}{3.185343in}}%
\pgfpathcurveto{\pgfqpoint{1.379413in}{3.185343in}}{\pgfqpoint{1.390012in}{3.189733in}}{\pgfqpoint{1.397825in}{3.197547in}}%
\pgfpathcurveto{\pgfqpoint{1.405639in}{3.205360in}}{\pgfqpoint{1.410029in}{3.215959in}}{\pgfqpoint{1.410029in}{3.227010in}}%
\pgfpathcurveto{\pgfqpoint{1.410029in}{3.238060in}}{\pgfqpoint{1.405639in}{3.248659in}}{\pgfqpoint{1.397825in}{3.256472in}}%
\pgfpathcurveto{\pgfqpoint{1.390012in}{3.264286in}}{\pgfqpoint{1.379413in}{3.268676in}}{\pgfqpoint{1.368362in}{3.268676in}}%
\pgfpathcurveto{\pgfqpoint{1.357312in}{3.268676in}}{\pgfqpoint{1.346713in}{3.264286in}}{\pgfqpoint{1.338900in}{3.256472in}}%
\pgfpathcurveto{\pgfqpoint{1.331086in}{3.248659in}}{\pgfqpoint{1.326696in}{3.238060in}}{\pgfqpoint{1.326696in}{3.227010in}}%
\pgfpathcurveto{\pgfqpoint{1.326696in}{3.215959in}}{\pgfqpoint{1.331086in}{3.205360in}}{\pgfqpoint{1.338900in}{3.197547in}}%
\pgfpathcurveto{\pgfqpoint{1.346713in}{3.189733in}}{\pgfqpoint{1.357312in}{3.185343in}}{\pgfqpoint{1.368362in}{3.185343in}}%
\pgfpathclose%
\pgfusepath{stroke,fill}%
\end{pgfscope}%
\begin{pgfscope}%
\pgfpathrectangle{\pgfqpoint{0.648703in}{0.548769in}}{\pgfqpoint{5.201297in}{3.102590in}}%
\pgfusepath{clip}%
\pgfsetbuttcap%
\pgfsetroundjoin%
\definecolor{currentfill}{rgb}{1.000000,0.498039,0.054902}%
\pgfsetfillcolor{currentfill}%
\pgfsetlinewidth{1.003750pt}%
\definecolor{currentstroke}{rgb}{1.000000,0.498039,0.054902}%
\pgfsetstrokecolor{currentstroke}%
\pgfsetdash{}{0pt}%
\pgfpathmoveto{\pgfqpoint{1.606119in}{3.185343in}}%
\pgfpathcurveto{\pgfqpoint{1.617169in}{3.185343in}}{\pgfqpoint{1.627768in}{3.189733in}}{\pgfqpoint{1.635581in}{3.197547in}}%
\pgfpathcurveto{\pgfqpoint{1.643395in}{3.205360in}}{\pgfqpoint{1.647785in}{3.215959in}}{\pgfqpoint{1.647785in}{3.227010in}}%
\pgfpathcurveto{\pgfqpoint{1.647785in}{3.238060in}}{\pgfqpoint{1.643395in}{3.248659in}}{\pgfqpoint{1.635581in}{3.256472in}}%
\pgfpathcurveto{\pgfqpoint{1.627768in}{3.264286in}}{\pgfqpoint{1.617169in}{3.268676in}}{\pgfqpoint{1.606119in}{3.268676in}}%
\pgfpathcurveto{\pgfqpoint{1.595068in}{3.268676in}}{\pgfqpoint{1.584469in}{3.264286in}}{\pgfqpoint{1.576656in}{3.256472in}}%
\pgfpathcurveto{\pgfqpoint{1.568842in}{3.248659in}}{\pgfqpoint{1.564452in}{3.238060in}}{\pgfqpoint{1.564452in}{3.227010in}}%
\pgfpathcurveto{\pgfqpoint{1.564452in}{3.215959in}}{\pgfqpoint{1.568842in}{3.205360in}}{\pgfqpoint{1.576656in}{3.197547in}}%
\pgfpathcurveto{\pgfqpoint{1.584469in}{3.189733in}}{\pgfqpoint{1.595068in}{3.185343in}}{\pgfqpoint{1.606119in}{3.185343in}}%
\pgfpathclose%
\pgfusepath{stroke,fill}%
\end{pgfscope}%
\begin{pgfscope}%
\pgfpathrectangle{\pgfqpoint{0.648703in}{0.548769in}}{\pgfqpoint{5.201297in}{3.102590in}}%
\pgfusepath{clip}%
\pgfsetbuttcap%
\pgfsetroundjoin%
\definecolor{currentfill}{rgb}{1.000000,0.498039,0.054902}%
\pgfsetfillcolor{currentfill}%
\pgfsetlinewidth{1.003750pt}%
\definecolor{currentstroke}{rgb}{1.000000,0.498039,0.054902}%
\pgfsetstrokecolor{currentstroke}%
\pgfsetdash{}{0pt}%
\pgfpathmoveto{\pgfqpoint{1.617056in}{3.193800in}}%
\pgfpathcurveto{\pgfqpoint{1.628106in}{3.193800in}}{\pgfqpoint{1.638705in}{3.198191in}}{\pgfqpoint{1.646519in}{3.206004in}}%
\pgfpathcurveto{\pgfqpoint{1.654332in}{3.213818in}}{\pgfqpoint{1.658723in}{3.224417in}}{\pgfqpoint{1.658723in}{3.235467in}}%
\pgfpathcurveto{\pgfqpoint{1.658723in}{3.246517in}}{\pgfqpoint{1.654332in}{3.257116in}}{\pgfqpoint{1.646519in}{3.264930in}}%
\pgfpathcurveto{\pgfqpoint{1.638705in}{3.272743in}}{\pgfqpoint{1.628106in}{3.277134in}}{\pgfqpoint{1.617056in}{3.277134in}}%
\pgfpathcurveto{\pgfqpoint{1.606006in}{3.277134in}}{\pgfqpoint{1.595407in}{3.272743in}}{\pgfqpoint{1.587593in}{3.264930in}}%
\pgfpathcurveto{\pgfqpoint{1.579780in}{3.257116in}}{\pgfqpoint{1.575389in}{3.246517in}}{\pgfqpoint{1.575389in}{3.235467in}}%
\pgfpathcurveto{\pgfqpoint{1.575389in}{3.224417in}}{\pgfqpoint{1.579780in}{3.213818in}}{\pgfqpoint{1.587593in}{3.206004in}}%
\pgfpathcurveto{\pgfqpoint{1.595407in}{3.198191in}}{\pgfqpoint{1.606006in}{3.193800in}}{\pgfqpoint{1.617056in}{3.193800in}}%
\pgfpathclose%
\pgfusepath{stroke,fill}%
\end{pgfscope}%
\begin{pgfscope}%
\pgfpathrectangle{\pgfqpoint{0.648703in}{0.548769in}}{\pgfqpoint{5.201297in}{3.102590in}}%
\pgfusepath{clip}%
\pgfsetbuttcap%
\pgfsetroundjoin%
\definecolor{currentfill}{rgb}{0.839216,0.152941,0.156863}%
\pgfsetfillcolor{currentfill}%
\pgfsetlinewidth{1.003750pt}%
\definecolor{currentstroke}{rgb}{0.839216,0.152941,0.156863}%
\pgfsetstrokecolor{currentstroke}%
\pgfsetdash{}{0pt}%
\pgfpathmoveto{\pgfqpoint{2.632775in}{3.189572in}}%
\pgfpathcurveto{\pgfqpoint{2.643825in}{3.189572in}}{\pgfqpoint{2.654424in}{3.193962in}}{\pgfqpoint{2.662237in}{3.201775in}}%
\pgfpathcurveto{\pgfqpoint{2.670051in}{3.209589in}}{\pgfqpoint{2.674441in}{3.220188in}}{\pgfqpoint{2.674441in}{3.231238in}}%
\pgfpathcurveto{\pgfqpoint{2.674441in}{3.242288in}}{\pgfqpoint{2.670051in}{3.252887in}}{\pgfqpoint{2.662237in}{3.260701in}}%
\pgfpathcurveto{\pgfqpoint{2.654424in}{3.268515in}}{\pgfqpoint{2.643825in}{3.272905in}}{\pgfqpoint{2.632775in}{3.272905in}}%
\pgfpathcurveto{\pgfqpoint{2.621725in}{3.272905in}}{\pgfqpoint{2.611126in}{3.268515in}}{\pgfqpoint{2.603312in}{3.260701in}}%
\pgfpathcurveto{\pgfqpoint{2.595498in}{3.252887in}}{\pgfqpoint{2.591108in}{3.242288in}}{\pgfqpoint{2.591108in}{3.231238in}}%
\pgfpathcurveto{\pgfqpoint{2.591108in}{3.220188in}}{\pgfqpoint{2.595498in}{3.209589in}}{\pgfqpoint{2.603312in}{3.201775in}}%
\pgfpathcurveto{\pgfqpoint{2.611126in}{3.193962in}}{\pgfqpoint{2.621725in}{3.189572in}}{\pgfqpoint{2.632775in}{3.189572in}}%
\pgfpathclose%
\pgfusepath{stroke,fill}%
\end{pgfscope}%
\begin{pgfscope}%
\pgfpathrectangle{\pgfqpoint{0.648703in}{0.548769in}}{\pgfqpoint{5.201297in}{3.102590in}}%
\pgfusepath{clip}%
\pgfsetbuttcap%
\pgfsetroundjoin%
\definecolor{currentfill}{rgb}{1.000000,0.498039,0.054902}%
\pgfsetfillcolor{currentfill}%
\pgfsetlinewidth{1.003750pt}%
\definecolor{currentstroke}{rgb}{1.000000,0.498039,0.054902}%
\pgfsetstrokecolor{currentstroke}%
\pgfsetdash{}{0pt}%
\pgfpathmoveto{\pgfqpoint{1.947578in}{3.198029in}}%
\pgfpathcurveto{\pgfqpoint{1.958629in}{3.198029in}}{\pgfqpoint{1.969228in}{3.202419in}}{\pgfqpoint{1.977041in}{3.210233in}}%
\pgfpathcurveto{\pgfqpoint{1.984855in}{3.218046in}}{\pgfqpoint{1.989245in}{3.228646in}}{\pgfqpoint{1.989245in}{3.239696in}}%
\pgfpathcurveto{\pgfqpoint{1.989245in}{3.250746in}}{\pgfqpoint{1.984855in}{3.261345in}}{\pgfqpoint{1.977041in}{3.269158in}}%
\pgfpathcurveto{\pgfqpoint{1.969228in}{3.276972in}}{\pgfqpoint{1.958629in}{3.281362in}}{\pgfqpoint{1.947578in}{3.281362in}}%
\pgfpathcurveto{\pgfqpoint{1.936528in}{3.281362in}}{\pgfqpoint{1.925929in}{3.276972in}}{\pgfqpoint{1.918116in}{3.269158in}}%
\pgfpathcurveto{\pgfqpoint{1.910302in}{3.261345in}}{\pgfqpoint{1.905912in}{3.250746in}}{\pgfqpoint{1.905912in}{3.239696in}}%
\pgfpathcurveto{\pgfqpoint{1.905912in}{3.228646in}}{\pgfqpoint{1.910302in}{3.218046in}}{\pgfqpoint{1.918116in}{3.210233in}}%
\pgfpathcurveto{\pgfqpoint{1.925929in}{3.202419in}}{\pgfqpoint{1.936528in}{3.198029in}}{\pgfqpoint{1.947578in}{3.198029in}}%
\pgfpathclose%
\pgfusepath{stroke,fill}%
\end{pgfscope}%
\begin{pgfscope}%
\pgfpathrectangle{\pgfqpoint{0.648703in}{0.548769in}}{\pgfqpoint{5.201297in}{3.102590in}}%
\pgfusepath{clip}%
\pgfsetbuttcap%
\pgfsetroundjoin%
\definecolor{currentfill}{rgb}{0.121569,0.466667,0.705882}%
\pgfsetfillcolor{currentfill}%
\pgfsetlinewidth{1.003750pt}%
\definecolor{currentstroke}{rgb}{0.121569,0.466667,0.705882}%
\pgfsetstrokecolor{currentstroke}%
\pgfsetdash{}{0pt}%
\pgfpathmoveto{\pgfqpoint{1.424120in}{3.181114in}}%
\pgfpathcurveto{\pgfqpoint{1.435171in}{3.181114in}}{\pgfqpoint{1.445770in}{3.185504in}}{\pgfqpoint{1.453583in}{3.193318in}}%
\pgfpathcurveto{\pgfqpoint{1.461397in}{3.201132in}}{\pgfqpoint{1.465787in}{3.211731in}}{\pgfqpoint{1.465787in}{3.222781in}}%
\pgfpathcurveto{\pgfqpoint{1.465787in}{3.233831in}}{\pgfqpoint{1.461397in}{3.244430in}}{\pgfqpoint{1.453583in}{3.252244in}}%
\pgfpathcurveto{\pgfqpoint{1.445770in}{3.260057in}}{\pgfqpoint{1.435171in}{3.264448in}}{\pgfqpoint{1.424120in}{3.264448in}}%
\pgfpathcurveto{\pgfqpoint{1.413070in}{3.264448in}}{\pgfqpoint{1.402471in}{3.260057in}}{\pgfqpoint{1.394658in}{3.252244in}}%
\pgfpathcurveto{\pgfqpoint{1.386844in}{3.244430in}}{\pgfqpoint{1.382454in}{3.233831in}}{\pgfqpoint{1.382454in}{3.222781in}}%
\pgfpathcurveto{\pgfqpoint{1.382454in}{3.211731in}}{\pgfqpoint{1.386844in}{3.201132in}}{\pgfqpoint{1.394658in}{3.193318in}}%
\pgfpathcurveto{\pgfqpoint{1.402471in}{3.185504in}}{\pgfqpoint{1.413070in}{3.181114in}}{\pgfqpoint{1.424120in}{3.181114in}}%
\pgfpathclose%
\pgfusepath{stroke,fill}%
\end{pgfscope}%
\begin{pgfscope}%
\pgfpathrectangle{\pgfqpoint{0.648703in}{0.548769in}}{\pgfqpoint{5.201297in}{3.102590in}}%
\pgfusepath{clip}%
\pgfsetbuttcap%
\pgfsetroundjoin%
\definecolor{currentfill}{rgb}{0.121569,0.466667,0.705882}%
\pgfsetfillcolor{currentfill}%
\pgfsetlinewidth{1.003750pt}%
\definecolor{currentstroke}{rgb}{0.121569,0.466667,0.705882}%
\pgfsetstrokecolor{currentstroke}%
\pgfsetdash{}{0pt}%
\pgfpathmoveto{\pgfqpoint{2.673608in}{0.939909in}}%
\pgfpathcurveto{\pgfqpoint{2.684658in}{0.939909in}}{\pgfqpoint{2.695257in}{0.944299in}}{\pgfqpoint{2.703070in}{0.952112in}}%
\pgfpathcurveto{\pgfqpoint{2.710884in}{0.959926in}}{\pgfqpoint{2.715274in}{0.970525in}}{\pgfqpoint{2.715274in}{0.981575in}}%
\pgfpathcurveto{\pgfqpoint{2.715274in}{0.992625in}}{\pgfqpoint{2.710884in}{1.003224in}}{\pgfqpoint{2.703070in}{1.011038in}}%
\pgfpathcurveto{\pgfqpoint{2.695257in}{1.018852in}}{\pgfqpoint{2.684658in}{1.023242in}}{\pgfqpoint{2.673608in}{1.023242in}}%
\pgfpathcurveto{\pgfqpoint{2.662557in}{1.023242in}}{\pgfqpoint{2.651958in}{1.018852in}}{\pgfqpoint{2.644145in}{1.011038in}}%
\pgfpathcurveto{\pgfqpoint{2.636331in}{1.003224in}}{\pgfqpoint{2.631941in}{0.992625in}}{\pgfqpoint{2.631941in}{0.981575in}}%
\pgfpathcurveto{\pgfqpoint{2.631941in}{0.970525in}}{\pgfqpoint{2.636331in}{0.959926in}}{\pgfqpoint{2.644145in}{0.952112in}}%
\pgfpathcurveto{\pgfqpoint{2.651958in}{0.944299in}}{\pgfqpoint{2.662557in}{0.939909in}}{\pgfqpoint{2.673608in}{0.939909in}}%
\pgfpathclose%
\pgfusepath{stroke,fill}%
\end{pgfscope}%
\begin{pgfscope}%
\pgfpathrectangle{\pgfqpoint{0.648703in}{0.548769in}}{\pgfqpoint{5.201297in}{3.102590in}}%
\pgfusepath{clip}%
\pgfsetbuttcap%
\pgfsetroundjoin%
\definecolor{currentfill}{rgb}{0.839216,0.152941,0.156863}%
\pgfsetfillcolor{currentfill}%
\pgfsetlinewidth{1.003750pt}%
\definecolor{currentstroke}{rgb}{0.839216,0.152941,0.156863}%
\pgfsetstrokecolor{currentstroke}%
\pgfsetdash{}{0pt}%
\pgfpathmoveto{\pgfqpoint{1.655926in}{3.210715in}}%
\pgfpathcurveto{\pgfqpoint{1.666976in}{3.210715in}}{\pgfqpoint{1.677575in}{3.215105in}}{\pgfqpoint{1.685389in}{3.222919in}}%
\pgfpathcurveto{\pgfqpoint{1.693202in}{3.230733in}}{\pgfqpoint{1.697592in}{3.241332in}}{\pgfqpoint{1.697592in}{3.252382in}}%
\pgfpathcurveto{\pgfqpoint{1.697592in}{3.263432in}}{\pgfqpoint{1.693202in}{3.274031in}}{\pgfqpoint{1.685389in}{3.281844in}}%
\pgfpathcurveto{\pgfqpoint{1.677575in}{3.289658in}}{\pgfqpoint{1.666976in}{3.294048in}}{\pgfqpoint{1.655926in}{3.294048in}}%
\pgfpathcurveto{\pgfqpoint{1.644876in}{3.294048in}}{\pgfqpoint{1.634277in}{3.289658in}}{\pgfqpoint{1.626463in}{3.281844in}}%
\pgfpathcurveto{\pgfqpoint{1.618649in}{3.274031in}}{\pgfqpoint{1.614259in}{3.263432in}}{\pgfqpoint{1.614259in}{3.252382in}}%
\pgfpathcurveto{\pgfqpoint{1.614259in}{3.241332in}}{\pgfqpoint{1.618649in}{3.230733in}}{\pgfqpoint{1.626463in}{3.222919in}}%
\pgfpathcurveto{\pgfqpoint{1.634277in}{3.215105in}}{\pgfqpoint{1.644876in}{3.210715in}}{\pgfqpoint{1.655926in}{3.210715in}}%
\pgfpathclose%
\pgfusepath{stroke,fill}%
\end{pgfscope}%
\begin{pgfscope}%
\pgfpathrectangle{\pgfqpoint{0.648703in}{0.548769in}}{\pgfqpoint{5.201297in}{3.102590in}}%
\pgfusepath{clip}%
\pgfsetbuttcap%
\pgfsetroundjoin%
\definecolor{currentfill}{rgb}{1.000000,0.498039,0.054902}%
\pgfsetfillcolor{currentfill}%
\pgfsetlinewidth{1.003750pt}%
\definecolor{currentstroke}{rgb}{1.000000,0.498039,0.054902}%
\pgfsetstrokecolor{currentstroke}%
\pgfsetdash{}{0pt}%
\pgfpathmoveto{\pgfqpoint{1.771571in}{3.202258in}}%
\pgfpathcurveto{\pgfqpoint{1.782622in}{3.202258in}}{\pgfqpoint{1.793221in}{3.206648in}}{\pgfqpoint{1.801034in}{3.214462in}}%
\pgfpathcurveto{\pgfqpoint{1.808848in}{3.222275in}}{\pgfqpoint{1.813238in}{3.232874in}}{\pgfqpoint{1.813238in}{3.243924in}}%
\pgfpathcurveto{\pgfqpoint{1.813238in}{3.254974in}}{\pgfqpoint{1.808848in}{3.265573in}}{\pgfqpoint{1.801034in}{3.273387in}}%
\pgfpathcurveto{\pgfqpoint{1.793221in}{3.281201in}}{\pgfqpoint{1.782622in}{3.285591in}}{\pgfqpoint{1.771571in}{3.285591in}}%
\pgfpathcurveto{\pgfqpoint{1.760521in}{3.285591in}}{\pgfqpoint{1.749922in}{3.281201in}}{\pgfqpoint{1.742109in}{3.273387in}}%
\pgfpathcurveto{\pgfqpoint{1.734295in}{3.265573in}}{\pgfqpoint{1.729905in}{3.254974in}}{\pgfqpoint{1.729905in}{3.243924in}}%
\pgfpathcurveto{\pgfqpoint{1.729905in}{3.232874in}}{\pgfqpoint{1.734295in}{3.222275in}}{\pgfqpoint{1.742109in}{3.214462in}}%
\pgfpathcurveto{\pgfqpoint{1.749922in}{3.206648in}}{\pgfqpoint{1.760521in}{3.202258in}}{\pgfqpoint{1.771571in}{3.202258in}}%
\pgfpathclose%
\pgfusepath{stroke,fill}%
\end{pgfscope}%
\begin{pgfscope}%
\pgfpathrectangle{\pgfqpoint{0.648703in}{0.548769in}}{\pgfqpoint{5.201297in}{3.102590in}}%
\pgfusepath{clip}%
\pgfsetbuttcap%
\pgfsetroundjoin%
\definecolor{currentfill}{rgb}{1.000000,0.498039,0.054902}%
\pgfsetfillcolor{currentfill}%
\pgfsetlinewidth{1.003750pt}%
\definecolor{currentstroke}{rgb}{1.000000,0.498039,0.054902}%
\pgfsetstrokecolor{currentstroke}%
\pgfsetdash{}{0pt}%
\pgfpathmoveto{\pgfqpoint{1.816116in}{3.206486in}}%
\pgfpathcurveto{\pgfqpoint{1.827167in}{3.206486in}}{\pgfqpoint{1.837766in}{3.210877in}}{\pgfqpoint{1.845579in}{3.218690in}}%
\pgfpathcurveto{\pgfqpoint{1.853393in}{3.226504in}}{\pgfqpoint{1.857783in}{3.237103in}}{\pgfqpoint{1.857783in}{3.248153in}}%
\pgfpathcurveto{\pgfqpoint{1.857783in}{3.259203in}}{\pgfqpoint{1.853393in}{3.269802in}}{\pgfqpoint{1.845579in}{3.277616in}}%
\pgfpathcurveto{\pgfqpoint{1.837766in}{3.285429in}}{\pgfqpoint{1.827167in}{3.289820in}}{\pgfqpoint{1.816116in}{3.289820in}}%
\pgfpathcurveto{\pgfqpoint{1.805066in}{3.289820in}}{\pgfqpoint{1.794467in}{3.285429in}}{\pgfqpoint{1.786654in}{3.277616in}}%
\pgfpathcurveto{\pgfqpoint{1.778840in}{3.269802in}}{\pgfqpoint{1.774450in}{3.259203in}}{\pgfqpoint{1.774450in}{3.248153in}}%
\pgfpathcurveto{\pgfqpoint{1.774450in}{3.237103in}}{\pgfqpoint{1.778840in}{3.226504in}}{\pgfqpoint{1.786654in}{3.218690in}}%
\pgfpathcurveto{\pgfqpoint{1.794467in}{3.210877in}}{\pgfqpoint{1.805066in}{3.206486in}}{\pgfqpoint{1.816116in}{3.206486in}}%
\pgfpathclose%
\pgfusepath{stroke,fill}%
\end{pgfscope}%
\begin{pgfscope}%
\pgfpathrectangle{\pgfqpoint{0.648703in}{0.548769in}}{\pgfqpoint{5.201297in}{3.102590in}}%
\pgfusepath{clip}%
\pgfsetbuttcap%
\pgfsetroundjoin%
\definecolor{currentfill}{rgb}{1.000000,0.498039,0.054902}%
\pgfsetfillcolor{currentfill}%
\pgfsetlinewidth{1.003750pt}%
\definecolor{currentstroke}{rgb}{1.000000,0.498039,0.054902}%
\pgfsetstrokecolor{currentstroke}%
\pgfsetdash{}{0pt}%
\pgfpathmoveto{\pgfqpoint{1.203623in}{3.198029in}}%
\pgfpathcurveto{\pgfqpoint{1.214673in}{3.198029in}}{\pgfqpoint{1.225272in}{3.202419in}}{\pgfqpoint{1.233085in}{3.210233in}}%
\pgfpathcurveto{\pgfqpoint{1.240899in}{3.218046in}}{\pgfqpoint{1.245289in}{3.228646in}}{\pgfqpoint{1.245289in}{3.239696in}}%
\pgfpathcurveto{\pgfqpoint{1.245289in}{3.250746in}}{\pgfqpoint{1.240899in}{3.261345in}}{\pgfqpoint{1.233085in}{3.269158in}}%
\pgfpathcurveto{\pgfqpoint{1.225272in}{3.276972in}}{\pgfqpoint{1.214673in}{3.281362in}}{\pgfqpoint{1.203623in}{3.281362in}}%
\pgfpathcurveto{\pgfqpoint{1.192573in}{3.281362in}}{\pgfqpoint{1.181974in}{3.276972in}}{\pgfqpoint{1.174160in}{3.269158in}}%
\pgfpathcurveto{\pgfqpoint{1.166346in}{3.261345in}}{\pgfqpoint{1.161956in}{3.250746in}}{\pgfqpoint{1.161956in}{3.239696in}}%
\pgfpathcurveto{\pgfqpoint{1.161956in}{3.228646in}}{\pgfqpoint{1.166346in}{3.218046in}}{\pgfqpoint{1.174160in}{3.210233in}}%
\pgfpathcurveto{\pgfqpoint{1.181974in}{3.202419in}}{\pgfqpoint{1.192573in}{3.198029in}}{\pgfqpoint{1.203623in}{3.198029in}}%
\pgfpathclose%
\pgfusepath{stroke,fill}%
\end{pgfscope}%
\begin{pgfscope}%
\pgfpathrectangle{\pgfqpoint{0.648703in}{0.548769in}}{\pgfqpoint{5.201297in}{3.102590in}}%
\pgfusepath{clip}%
\pgfsetbuttcap%
\pgfsetroundjoin%
\definecolor{currentfill}{rgb}{1.000000,0.498039,0.054902}%
\pgfsetfillcolor{currentfill}%
\pgfsetlinewidth{1.003750pt}%
\definecolor{currentstroke}{rgb}{1.000000,0.498039,0.054902}%
\pgfsetstrokecolor{currentstroke}%
\pgfsetdash{}{0pt}%
\pgfpathmoveto{\pgfqpoint{3.122770in}{3.202258in}}%
\pgfpathcurveto{\pgfqpoint{3.133820in}{3.202258in}}{\pgfqpoint{3.144419in}{3.206648in}}{\pgfqpoint{3.152232in}{3.214462in}}%
\pgfpathcurveto{\pgfqpoint{3.160046in}{3.222275in}}{\pgfqpoint{3.164436in}{3.232874in}}{\pgfqpoint{3.164436in}{3.243924in}}%
\pgfpathcurveto{\pgfqpoint{3.164436in}{3.254974in}}{\pgfqpoint{3.160046in}{3.265573in}}{\pgfqpoint{3.152232in}{3.273387in}}%
\pgfpathcurveto{\pgfqpoint{3.144419in}{3.281201in}}{\pgfqpoint{3.133820in}{3.285591in}}{\pgfqpoint{3.122770in}{3.285591in}}%
\pgfpathcurveto{\pgfqpoint{3.111720in}{3.285591in}}{\pgfqpoint{3.101121in}{3.281201in}}{\pgfqpoint{3.093307in}{3.273387in}}%
\pgfpathcurveto{\pgfqpoint{3.085493in}{3.265573in}}{\pgfqpoint{3.081103in}{3.254974in}}{\pgfqpoint{3.081103in}{3.243924in}}%
\pgfpathcurveto{\pgfqpoint{3.081103in}{3.232874in}}{\pgfqpoint{3.085493in}{3.222275in}}{\pgfqpoint{3.093307in}{3.214462in}}%
\pgfpathcurveto{\pgfqpoint{3.101121in}{3.206648in}}{\pgfqpoint{3.111720in}{3.202258in}}{\pgfqpoint{3.122770in}{3.202258in}}%
\pgfpathclose%
\pgfusepath{stroke,fill}%
\end{pgfscope}%
\begin{pgfscope}%
\pgfpathrectangle{\pgfqpoint{0.648703in}{0.548769in}}{\pgfqpoint{5.201297in}{3.102590in}}%
\pgfusepath{clip}%
\pgfsetbuttcap%
\pgfsetroundjoin%
\definecolor{currentfill}{rgb}{1.000000,0.498039,0.054902}%
\pgfsetfillcolor{currentfill}%
\pgfsetlinewidth{1.003750pt}%
\definecolor{currentstroke}{rgb}{1.000000,0.498039,0.054902}%
\pgfsetstrokecolor{currentstroke}%
\pgfsetdash{}{0pt}%
\pgfpathmoveto{\pgfqpoint{1.803222in}{3.189572in}}%
\pgfpathcurveto{\pgfqpoint{1.814272in}{3.189572in}}{\pgfqpoint{1.824871in}{3.193962in}}{\pgfqpoint{1.832685in}{3.201775in}}%
\pgfpathcurveto{\pgfqpoint{1.840498in}{3.209589in}}{\pgfqpoint{1.844888in}{3.220188in}}{\pgfqpoint{1.844888in}{3.231238in}}%
\pgfpathcurveto{\pgfqpoint{1.844888in}{3.242288in}}{\pgfqpoint{1.840498in}{3.252887in}}{\pgfqpoint{1.832685in}{3.260701in}}%
\pgfpathcurveto{\pgfqpoint{1.824871in}{3.268515in}}{\pgfqpoint{1.814272in}{3.272905in}}{\pgfqpoint{1.803222in}{3.272905in}}%
\pgfpathcurveto{\pgfqpoint{1.792172in}{3.272905in}}{\pgfqpoint{1.781573in}{3.268515in}}{\pgfqpoint{1.773759in}{3.260701in}}%
\pgfpathcurveto{\pgfqpoint{1.765945in}{3.252887in}}{\pgfqpoint{1.761555in}{3.242288in}}{\pgfqpoint{1.761555in}{3.231238in}}%
\pgfpathcurveto{\pgfqpoint{1.761555in}{3.220188in}}{\pgfqpoint{1.765945in}{3.209589in}}{\pgfqpoint{1.773759in}{3.201775in}}%
\pgfpathcurveto{\pgfqpoint{1.781573in}{3.193962in}}{\pgfqpoint{1.792172in}{3.189572in}}{\pgfqpoint{1.803222in}{3.189572in}}%
\pgfpathclose%
\pgfusepath{stroke,fill}%
\end{pgfscope}%
\begin{pgfscope}%
\pgfpathrectangle{\pgfqpoint{0.648703in}{0.548769in}}{\pgfqpoint{5.201297in}{3.102590in}}%
\pgfusepath{clip}%
\pgfsetbuttcap%
\pgfsetroundjoin%
\definecolor{currentfill}{rgb}{0.121569,0.466667,0.705882}%
\pgfsetfillcolor{currentfill}%
\pgfsetlinewidth{1.003750pt}%
\definecolor{currentstroke}{rgb}{0.121569,0.466667,0.705882}%
\pgfsetstrokecolor{currentstroke}%
\pgfsetdash{}{0pt}%
\pgfpathmoveto{\pgfqpoint{2.061114in}{3.181114in}}%
\pgfpathcurveto{\pgfqpoint{2.072164in}{3.181114in}}{\pgfqpoint{2.082763in}{3.185504in}}{\pgfqpoint{2.090577in}{3.193318in}}%
\pgfpathcurveto{\pgfqpoint{2.098390in}{3.201132in}}{\pgfqpoint{2.102781in}{3.211731in}}{\pgfqpoint{2.102781in}{3.222781in}}%
\pgfpathcurveto{\pgfqpoint{2.102781in}{3.233831in}}{\pgfqpoint{2.098390in}{3.244430in}}{\pgfqpoint{2.090577in}{3.252244in}}%
\pgfpathcurveto{\pgfqpoint{2.082763in}{3.260057in}}{\pgfqpoint{2.072164in}{3.264448in}}{\pgfqpoint{2.061114in}{3.264448in}}%
\pgfpathcurveto{\pgfqpoint{2.050064in}{3.264448in}}{\pgfqpoint{2.039465in}{3.260057in}}{\pgfqpoint{2.031651in}{3.252244in}}%
\pgfpathcurveto{\pgfqpoint{2.023838in}{3.244430in}}{\pgfqpoint{2.019447in}{3.233831in}}{\pgfqpoint{2.019447in}{3.222781in}}%
\pgfpathcurveto{\pgfqpoint{2.019447in}{3.211731in}}{\pgfqpoint{2.023838in}{3.201132in}}{\pgfqpoint{2.031651in}{3.193318in}}%
\pgfpathcurveto{\pgfqpoint{2.039465in}{3.185504in}}{\pgfqpoint{2.050064in}{3.181114in}}{\pgfqpoint{2.061114in}{3.181114in}}%
\pgfpathclose%
\pgfusepath{stroke,fill}%
\end{pgfscope}%
\begin{pgfscope}%
\pgfpathrectangle{\pgfqpoint{0.648703in}{0.548769in}}{\pgfqpoint{5.201297in}{3.102590in}}%
\pgfusepath{clip}%
\pgfsetbuttcap%
\pgfsetroundjoin%
\definecolor{currentfill}{rgb}{1.000000,0.498039,0.054902}%
\pgfsetfillcolor{currentfill}%
\pgfsetlinewidth{1.003750pt}%
\definecolor{currentstroke}{rgb}{1.000000,0.498039,0.054902}%
\pgfsetstrokecolor{currentstroke}%
\pgfsetdash{}{0pt}%
\pgfpathmoveto{\pgfqpoint{1.377700in}{3.244545in}}%
\pgfpathcurveto{\pgfqpoint{1.388750in}{3.244545in}}{\pgfqpoint{1.399349in}{3.248935in}}{\pgfqpoint{1.407163in}{3.256748in}}%
\pgfpathcurveto{\pgfqpoint{1.414976in}{3.264562in}}{\pgfqpoint{1.419367in}{3.275161in}}{\pgfqpoint{1.419367in}{3.286211in}}%
\pgfpathcurveto{\pgfqpoint{1.419367in}{3.297261in}}{\pgfqpoint{1.414976in}{3.307860in}}{\pgfqpoint{1.407163in}{3.315674in}}%
\pgfpathcurveto{\pgfqpoint{1.399349in}{3.323488in}}{\pgfqpoint{1.388750in}{3.327878in}}{\pgfqpoint{1.377700in}{3.327878in}}%
\pgfpathcurveto{\pgfqpoint{1.366650in}{3.327878in}}{\pgfqpoint{1.356051in}{3.323488in}}{\pgfqpoint{1.348237in}{3.315674in}}%
\pgfpathcurveto{\pgfqpoint{1.340423in}{3.307860in}}{\pgfqpoint{1.336033in}{3.297261in}}{\pgfqpoint{1.336033in}{3.286211in}}%
\pgfpathcurveto{\pgfqpoint{1.336033in}{3.275161in}}{\pgfqpoint{1.340423in}{3.264562in}}{\pgfqpoint{1.348237in}{3.256748in}}%
\pgfpathcurveto{\pgfqpoint{1.356051in}{3.248935in}}{\pgfqpoint{1.366650in}{3.244545in}}{\pgfqpoint{1.377700in}{3.244545in}}%
\pgfpathclose%
\pgfusepath{stroke,fill}%
\end{pgfscope}%
\begin{pgfscope}%
\pgfpathrectangle{\pgfqpoint{0.648703in}{0.548769in}}{\pgfqpoint{5.201297in}{3.102590in}}%
\pgfusepath{clip}%
\pgfsetbuttcap%
\pgfsetroundjoin%
\definecolor{currentfill}{rgb}{1.000000,0.498039,0.054902}%
\pgfsetfillcolor{currentfill}%
\pgfsetlinewidth{1.003750pt}%
\definecolor{currentstroke}{rgb}{1.000000,0.498039,0.054902}%
\pgfsetstrokecolor{currentstroke}%
\pgfsetdash{}{0pt}%
\pgfpathmoveto{\pgfqpoint{1.693618in}{3.189572in}}%
\pgfpathcurveto{\pgfqpoint{1.704668in}{3.189572in}}{\pgfqpoint{1.715267in}{3.193962in}}{\pgfqpoint{1.723080in}{3.201775in}}%
\pgfpathcurveto{\pgfqpoint{1.730894in}{3.209589in}}{\pgfqpoint{1.735284in}{3.220188in}}{\pgfqpoint{1.735284in}{3.231238in}}%
\pgfpathcurveto{\pgfqpoint{1.735284in}{3.242288in}}{\pgfqpoint{1.730894in}{3.252887in}}{\pgfqpoint{1.723080in}{3.260701in}}%
\pgfpathcurveto{\pgfqpoint{1.715267in}{3.268515in}}{\pgfqpoint{1.704668in}{3.272905in}}{\pgfqpoint{1.693618in}{3.272905in}}%
\pgfpathcurveto{\pgfqpoint{1.682568in}{3.272905in}}{\pgfqpoint{1.671969in}{3.268515in}}{\pgfqpoint{1.664155in}{3.260701in}}%
\pgfpathcurveto{\pgfqpoint{1.656341in}{3.252887in}}{\pgfqpoint{1.651951in}{3.242288in}}{\pgfqpoint{1.651951in}{3.231238in}}%
\pgfpathcurveto{\pgfqpoint{1.651951in}{3.220188in}}{\pgfqpoint{1.656341in}{3.209589in}}{\pgfqpoint{1.664155in}{3.201775in}}%
\pgfpathcurveto{\pgfqpoint{1.671969in}{3.193962in}}{\pgfqpoint{1.682568in}{3.189572in}}{\pgfqpoint{1.693618in}{3.189572in}}%
\pgfpathclose%
\pgfusepath{stroke,fill}%
\end{pgfscope}%
\begin{pgfscope}%
\pgfpathrectangle{\pgfqpoint{0.648703in}{0.548769in}}{\pgfqpoint{5.201297in}{3.102590in}}%
\pgfusepath{clip}%
\pgfsetbuttcap%
\pgfsetroundjoin%
\definecolor{currentfill}{rgb}{1.000000,0.498039,0.054902}%
\pgfsetfillcolor{currentfill}%
\pgfsetlinewidth{1.003750pt}%
\definecolor{currentstroke}{rgb}{1.000000,0.498039,0.054902}%
\pgfsetstrokecolor{currentstroke}%
\pgfsetdash{}{0pt}%
\pgfpathmoveto{\pgfqpoint{1.203623in}{3.210715in}}%
\pgfpathcurveto{\pgfqpoint{1.214673in}{3.210715in}}{\pgfqpoint{1.225272in}{3.215105in}}{\pgfqpoint{1.233085in}{3.222919in}}%
\pgfpathcurveto{\pgfqpoint{1.240899in}{3.230733in}}{\pgfqpoint{1.245289in}{3.241332in}}{\pgfqpoint{1.245289in}{3.252382in}}%
\pgfpathcurveto{\pgfqpoint{1.245289in}{3.263432in}}{\pgfqpoint{1.240899in}{3.274031in}}{\pgfqpoint{1.233085in}{3.281844in}}%
\pgfpathcurveto{\pgfqpoint{1.225272in}{3.289658in}}{\pgfqpoint{1.214673in}{3.294048in}}{\pgfqpoint{1.203623in}{3.294048in}}%
\pgfpathcurveto{\pgfqpoint{1.192573in}{3.294048in}}{\pgfqpoint{1.181974in}{3.289658in}}{\pgfqpoint{1.174160in}{3.281844in}}%
\pgfpathcurveto{\pgfqpoint{1.166346in}{3.274031in}}{\pgfqpoint{1.161956in}{3.263432in}}{\pgfqpoint{1.161956in}{3.252382in}}%
\pgfpathcurveto{\pgfqpoint{1.161956in}{3.241332in}}{\pgfqpoint{1.166346in}{3.230733in}}{\pgfqpoint{1.174160in}{3.222919in}}%
\pgfpathcurveto{\pgfqpoint{1.181974in}{3.215105in}}{\pgfqpoint{1.192573in}{3.210715in}}{\pgfqpoint{1.203623in}{3.210715in}}%
\pgfpathclose%
\pgfusepath{stroke,fill}%
\end{pgfscope}%
\begin{pgfscope}%
\pgfpathrectangle{\pgfqpoint{0.648703in}{0.548769in}}{\pgfqpoint{5.201297in}{3.102590in}}%
\pgfusepath{clip}%
\pgfsetbuttcap%
\pgfsetroundjoin%
\definecolor{currentfill}{rgb}{1.000000,0.498039,0.054902}%
\pgfsetfillcolor{currentfill}%
\pgfsetlinewidth{1.003750pt}%
\definecolor{currentstroke}{rgb}{1.000000,0.498039,0.054902}%
\pgfsetstrokecolor{currentstroke}%
\pgfsetdash{}{0pt}%
\pgfpathmoveto{\pgfqpoint{1.381443in}{3.189572in}}%
\pgfpathcurveto{\pgfqpoint{1.392494in}{3.189572in}}{\pgfqpoint{1.403093in}{3.193962in}}{\pgfqpoint{1.410906in}{3.201775in}}%
\pgfpathcurveto{\pgfqpoint{1.418720in}{3.209589in}}{\pgfqpoint{1.423110in}{3.220188in}}{\pgfqpoint{1.423110in}{3.231238in}}%
\pgfpathcurveto{\pgfqpoint{1.423110in}{3.242288in}}{\pgfqpoint{1.418720in}{3.252887in}}{\pgfqpoint{1.410906in}{3.260701in}}%
\pgfpathcurveto{\pgfqpoint{1.403093in}{3.268515in}}{\pgfqpoint{1.392494in}{3.272905in}}{\pgfqpoint{1.381443in}{3.272905in}}%
\pgfpathcurveto{\pgfqpoint{1.370393in}{3.272905in}}{\pgfqpoint{1.359794in}{3.268515in}}{\pgfqpoint{1.351981in}{3.260701in}}%
\pgfpathcurveto{\pgfqpoint{1.344167in}{3.252887in}}{\pgfqpoint{1.339777in}{3.242288in}}{\pgfqpoint{1.339777in}{3.231238in}}%
\pgfpathcurveto{\pgfqpoint{1.339777in}{3.220188in}}{\pgfqpoint{1.344167in}{3.209589in}}{\pgfqpoint{1.351981in}{3.201775in}}%
\pgfpathcurveto{\pgfqpoint{1.359794in}{3.193962in}}{\pgfqpoint{1.370393in}{3.189572in}}{\pgfqpoint{1.381443in}{3.189572in}}%
\pgfpathclose%
\pgfusepath{stroke,fill}%
\end{pgfscope}%
\begin{pgfscope}%
\pgfpathrectangle{\pgfqpoint{0.648703in}{0.548769in}}{\pgfqpoint{5.201297in}{3.102590in}}%
\pgfusepath{clip}%
\pgfsetbuttcap%
\pgfsetroundjoin%
\definecolor{currentfill}{rgb}{0.839216,0.152941,0.156863}%
\pgfsetfillcolor{currentfill}%
\pgfsetlinewidth{1.003750pt}%
\definecolor{currentstroke}{rgb}{0.839216,0.152941,0.156863}%
\pgfsetstrokecolor{currentstroke}%
\pgfsetdash{}{0pt}%
\pgfpathmoveto{\pgfqpoint{1.239652in}{3.181114in}}%
\pgfpathcurveto{\pgfqpoint{1.250702in}{3.181114in}}{\pgfqpoint{1.261301in}{3.185504in}}{\pgfqpoint{1.269115in}{3.193318in}}%
\pgfpathcurveto{\pgfqpoint{1.276928in}{3.201132in}}{\pgfqpoint{1.281318in}{3.211731in}}{\pgfqpoint{1.281318in}{3.222781in}}%
\pgfpathcurveto{\pgfqpoint{1.281318in}{3.233831in}}{\pgfqpoint{1.276928in}{3.244430in}}{\pgfqpoint{1.269115in}{3.252244in}}%
\pgfpathcurveto{\pgfqpoint{1.261301in}{3.260057in}}{\pgfqpoint{1.250702in}{3.264448in}}{\pgfqpoint{1.239652in}{3.264448in}}%
\pgfpathcurveto{\pgfqpoint{1.228602in}{3.264448in}}{\pgfqpoint{1.218003in}{3.260057in}}{\pgfqpoint{1.210189in}{3.252244in}}%
\pgfpathcurveto{\pgfqpoint{1.202375in}{3.244430in}}{\pgfqpoint{1.197985in}{3.233831in}}{\pgfqpoint{1.197985in}{3.222781in}}%
\pgfpathcurveto{\pgfqpoint{1.197985in}{3.211731in}}{\pgfqpoint{1.202375in}{3.201132in}}{\pgfqpoint{1.210189in}{3.193318in}}%
\pgfpathcurveto{\pgfqpoint{1.218003in}{3.185504in}}{\pgfqpoint{1.228602in}{3.181114in}}{\pgfqpoint{1.239652in}{3.181114in}}%
\pgfpathclose%
\pgfusepath{stroke,fill}%
\end{pgfscope}%
\begin{pgfscope}%
\pgfpathrectangle{\pgfqpoint{0.648703in}{0.548769in}}{\pgfqpoint{5.201297in}{3.102590in}}%
\pgfusepath{clip}%
\pgfsetbuttcap%
\pgfsetroundjoin%
\definecolor{currentfill}{rgb}{1.000000,0.498039,0.054902}%
\pgfsetfillcolor{currentfill}%
\pgfsetlinewidth{1.003750pt}%
\definecolor{currentstroke}{rgb}{1.000000,0.498039,0.054902}%
\pgfsetstrokecolor{currentstroke}%
\pgfsetdash{}{0pt}%
\pgfpathmoveto{\pgfqpoint{1.938615in}{3.202258in}}%
\pgfpathcurveto{\pgfqpoint{1.949665in}{3.202258in}}{\pgfqpoint{1.960264in}{3.206648in}}{\pgfqpoint{1.968078in}{3.214462in}}%
\pgfpathcurveto{\pgfqpoint{1.975892in}{3.222275in}}{\pgfqpoint{1.980282in}{3.232874in}}{\pgfqpoint{1.980282in}{3.243924in}}%
\pgfpathcurveto{\pgfqpoint{1.980282in}{3.254974in}}{\pgfqpoint{1.975892in}{3.265573in}}{\pgfqpoint{1.968078in}{3.273387in}}%
\pgfpathcurveto{\pgfqpoint{1.960264in}{3.281201in}}{\pgfqpoint{1.949665in}{3.285591in}}{\pgfqpoint{1.938615in}{3.285591in}}%
\pgfpathcurveto{\pgfqpoint{1.927565in}{3.285591in}}{\pgfqpoint{1.916966in}{3.281201in}}{\pgfqpoint{1.909152in}{3.273387in}}%
\pgfpathcurveto{\pgfqpoint{1.901339in}{3.265573in}}{\pgfqpoint{1.896949in}{3.254974in}}{\pgfqpoint{1.896949in}{3.243924in}}%
\pgfpathcurveto{\pgfqpoint{1.896949in}{3.232874in}}{\pgfqpoint{1.901339in}{3.222275in}}{\pgfqpoint{1.909152in}{3.214462in}}%
\pgfpathcurveto{\pgfqpoint{1.916966in}{3.206648in}}{\pgfqpoint{1.927565in}{3.202258in}}{\pgfqpoint{1.938615in}{3.202258in}}%
\pgfpathclose%
\pgfusepath{stroke,fill}%
\end{pgfscope}%
\begin{pgfscope}%
\pgfpathrectangle{\pgfqpoint{0.648703in}{0.548769in}}{\pgfqpoint{5.201297in}{3.102590in}}%
\pgfusepath{clip}%
\pgfsetbuttcap%
\pgfsetroundjoin%
\definecolor{currentfill}{rgb}{0.839216,0.152941,0.156863}%
\pgfsetfillcolor{currentfill}%
\pgfsetlinewidth{1.003750pt}%
\definecolor{currentstroke}{rgb}{0.839216,0.152941,0.156863}%
\pgfsetstrokecolor{currentstroke}%
\pgfsetdash{}{0pt}%
\pgfpathmoveto{\pgfqpoint{1.375121in}{3.265688in}}%
\pgfpathcurveto{\pgfqpoint{1.386171in}{3.265688in}}{\pgfqpoint{1.396770in}{3.270078in}}{\pgfqpoint{1.404584in}{3.277892in}}%
\pgfpathcurveto{\pgfqpoint{1.412397in}{3.285706in}}{\pgfqpoint{1.416788in}{3.296305in}}{\pgfqpoint{1.416788in}{3.307355in}}%
\pgfpathcurveto{\pgfqpoint{1.416788in}{3.318405in}}{\pgfqpoint{1.412397in}{3.329004in}}{\pgfqpoint{1.404584in}{3.336817in}}%
\pgfpathcurveto{\pgfqpoint{1.396770in}{3.344631in}}{\pgfqpoint{1.386171in}{3.349021in}}{\pgfqpoint{1.375121in}{3.349021in}}%
\pgfpathcurveto{\pgfqpoint{1.364071in}{3.349021in}}{\pgfqpoint{1.353472in}{3.344631in}}{\pgfqpoint{1.345658in}{3.336817in}}%
\pgfpathcurveto{\pgfqpoint{1.337845in}{3.329004in}}{\pgfqpoint{1.333454in}{3.318405in}}{\pgfqpoint{1.333454in}{3.307355in}}%
\pgfpathcurveto{\pgfqpoint{1.333454in}{3.296305in}}{\pgfqpoint{1.337845in}{3.285706in}}{\pgfqpoint{1.345658in}{3.277892in}}%
\pgfpathcurveto{\pgfqpoint{1.353472in}{3.270078in}}{\pgfqpoint{1.364071in}{3.265688in}}{\pgfqpoint{1.375121in}{3.265688in}}%
\pgfpathclose%
\pgfusepath{stroke,fill}%
\end{pgfscope}%
\begin{pgfscope}%
\pgfpathrectangle{\pgfqpoint{0.648703in}{0.548769in}}{\pgfqpoint{5.201297in}{3.102590in}}%
\pgfusepath{clip}%
\pgfsetbuttcap%
\pgfsetroundjoin%
\definecolor{currentfill}{rgb}{0.121569,0.466667,0.705882}%
\pgfsetfillcolor{currentfill}%
\pgfsetlinewidth{1.003750pt}%
\definecolor{currentstroke}{rgb}{0.121569,0.466667,0.705882}%
\pgfsetstrokecolor{currentstroke}%
\pgfsetdash{}{0pt}%
\pgfpathmoveto{\pgfqpoint{3.261602in}{2.808990in}}%
\pgfpathcurveto{\pgfqpoint{3.272652in}{2.808990in}}{\pgfqpoint{3.283251in}{2.813380in}}{\pgfqpoint{3.291064in}{2.821193in}}%
\pgfpathcurveto{\pgfqpoint{3.298878in}{2.829007in}}{\pgfqpoint{3.303268in}{2.839606in}}{\pgfqpoint{3.303268in}{2.850656in}}%
\pgfpathcurveto{\pgfqpoint{3.303268in}{2.861706in}}{\pgfqpoint{3.298878in}{2.872305in}}{\pgfqpoint{3.291064in}{2.880119in}}%
\pgfpathcurveto{\pgfqpoint{3.283251in}{2.887933in}}{\pgfqpoint{3.272652in}{2.892323in}}{\pgfqpoint{3.261602in}{2.892323in}}%
\pgfpathcurveto{\pgfqpoint{3.250551in}{2.892323in}}{\pgfqpoint{3.239952in}{2.887933in}}{\pgfqpoint{3.232139in}{2.880119in}}%
\pgfpathcurveto{\pgfqpoint{3.224325in}{2.872305in}}{\pgfqpoint{3.219935in}{2.861706in}}{\pgfqpoint{3.219935in}{2.850656in}}%
\pgfpathcurveto{\pgfqpoint{3.219935in}{2.839606in}}{\pgfqpoint{3.224325in}{2.829007in}}{\pgfqpoint{3.232139in}{2.821193in}}%
\pgfpathcurveto{\pgfqpoint{3.239952in}{2.813380in}}{\pgfqpoint{3.250551in}{2.808990in}}{\pgfqpoint{3.261602in}{2.808990in}}%
\pgfpathclose%
\pgfusepath{stroke,fill}%
\end{pgfscope}%
\begin{pgfscope}%
\pgfpathrectangle{\pgfqpoint{0.648703in}{0.548769in}}{\pgfqpoint{5.201297in}{3.102590in}}%
\pgfusepath{clip}%
\pgfsetbuttcap%
\pgfsetroundjoin%
\definecolor{currentfill}{rgb}{1.000000,0.498039,0.054902}%
\pgfsetfillcolor{currentfill}%
\pgfsetlinewidth{1.003750pt}%
\definecolor{currentstroke}{rgb}{1.000000,0.498039,0.054902}%
\pgfsetstrokecolor{currentstroke}%
\pgfsetdash{}{0pt}%
\pgfpathmoveto{\pgfqpoint{1.136805in}{3.185343in}}%
\pgfpathcurveto{\pgfqpoint{1.147855in}{3.185343in}}{\pgfqpoint{1.158454in}{3.189733in}}{\pgfqpoint{1.166268in}{3.197547in}}%
\pgfpathcurveto{\pgfqpoint{1.174082in}{3.205360in}}{\pgfqpoint{1.178472in}{3.215959in}}{\pgfqpoint{1.178472in}{3.227010in}}%
\pgfpathcurveto{\pgfqpoint{1.178472in}{3.238060in}}{\pgfqpoint{1.174082in}{3.248659in}}{\pgfqpoint{1.166268in}{3.256472in}}%
\pgfpathcurveto{\pgfqpoint{1.158454in}{3.264286in}}{\pgfqpoint{1.147855in}{3.268676in}}{\pgfqpoint{1.136805in}{3.268676in}}%
\pgfpathcurveto{\pgfqpoint{1.125755in}{3.268676in}}{\pgfqpoint{1.115156in}{3.264286in}}{\pgfqpoint{1.107342in}{3.256472in}}%
\pgfpathcurveto{\pgfqpoint{1.099529in}{3.248659in}}{\pgfqpoint{1.095139in}{3.238060in}}{\pgfqpoint{1.095139in}{3.227010in}}%
\pgfpathcurveto{\pgfqpoint{1.095139in}{3.215959in}}{\pgfqpoint{1.099529in}{3.205360in}}{\pgfqpoint{1.107342in}{3.197547in}}%
\pgfpathcurveto{\pgfqpoint{1.115156in}{3.189733in}}{\pgfqpoint{1.125755in}{3.185343in}}{\pgfqpoint{1.136805in}{3.185343in}}%
\pgfpathclose%
\pgfusepath{stroke,fill}%
\end{pgfscope}%
\begin{pgfscope}%
\pgfpathrectangle{\pgfqpoint{0.648703in}{0.548769in}}{\pgfqpoint{5.201297in}{3.102590in}}%
\pgfusepath{clip}%
\pgfsetbuttcap%
\pgfsetroundjoin%
\definecolor{currentfill}{rgb}{1.000000,0.498039,0.054902}%
\pgfsetfillcolor{currentfill}%
\pgfsetlinewidth{1.003750pt}%
\definecolor{currentstroke}{rgb}{1.000000,0.498039,0.054902}%
\pgfsetstrokecolor{currentstroke}%
\pgfsetdash{}{0pt}%
\pgfpathmoveto{\pgfqpoint{1.856949in}{3.193800in}}%
\pgfpathcurveto{\pgfqpoint{1.867999in}{3.193800in}}{\pgfqpoint{1.878599in}{3.198191in}}{\pgfqpoint{1.886412in}{3.206004in}}%
\pgfpathcurveto{\pgfqpoint{1.894226in}{3.213818in}}{\pgfqpoint{1.898616in}{3.224417in}}{\pgfqpoint{1.898616in}{3.235467in}}%
\pgfpathcurveto{\pgfqpoint{1.898616in}{3.246517in}}{\pgfqpoint{1.894226in}{3.257116in}}{\pgfqpoint{1.886412in}{3.264930in}}%
\pgfpathcurveto{\pgfqpoint{1.878599in}{3.272743in}}{\pgfqpoint{1.867999in}{3.277134in}}{\pgfqpoint{1.856949in}{3.277134in}}%
\pgfpathcurveto{\pgfqpoint{1.845899in}{3.277134in}}{\pgfqpoint{1.835300in}{3.272743in}}{\pgfqpoint{1.827487in}{3.264930in}}%
\pgfpathcurveto{\pgfqpoint{1.819673in}{3.257116in}}{\pgfqpoint{1.815283in}{3.246517in}}{\pgfqpoint{1.815283in}{3.235467in}}%
\pgfpathcurveto{\pgfqpoint{1.815283in}{3.224417in}}{\pgfqpoint{1.819673in}{3.213818in}}{\pgfqpoint{1.827487in}{3.206004in}}%
\pgfpathcurveto{\pgfqpoint{1.835300in}{3.198191in}}{\pgfqpoint{1.845899in}{3.193800in}}{\pgfqpoint{1.856949in}{3.193800in}}%
\pgfpathclose%
\pgfusepath{stroke,fill}%
\end{pgfscope}%
\begin{pgfscope}%
\pgfpathrectangle{\pgfqpoint{0.648703in}{0.548769in}}{\pgfqpoint{5.201297in}{3.102590in}}%
\pgfusepath{clip}%
\pgfsetbuttcap%
\pgfsetroundjoin%
\definecolor{currentfill}{rgb}{1.000000,0.498039,0.054902}%
\pgfsetfillcolor{currentfill}%
\pgfsetlinewidth{1.003750pt}%
\definecolor{currentstroke}{rgb}{1.000000,0.498039,0.054902}%
\pgfsetstrokecolor{currentstroke}%
\pgfsetdash{}{0pt}%
\pgfpathmoveto{\pgfqpoint{1.036579in}{3.362948in}}%
\pgfpathcurveto{\pgfqpoint{1.047629in}{3.362948in}}{\pgfqpoint{1.058228in}{3.367338in}}{\pgfqpoint{1.066042in}{3.375152in}}%
\pgfpathcurveto{\pgfqpoint{1.073855in}{3.382965in}}{\pgfqpoint{1.078246in}{3.393564in}}{\pgfqpoint{1.078246in}{3.404615in}}%
\pgfpathcurveto{\pgfqpoint{1.078246in}{3.415665in}}{\pgfqpoint{1.073855in}{3.426264in}}{\pgfqpoint{1.066042in}{3.434077in}}%
\pgfpathcurveto{\pgfqpoint{1.058228in}{3.441891in}}{\pgfqpoint{1.047629in}{3.446281in}}{\pgfqpoint{1.036579in}{3.446281in}}%
\pgfpathcurveto{\pgfqpoint{1.025529in}{3.446281in}}{\pgfqpoint{1.014930in}{3.441891in}}{\pgfqpoint{1.007116in}{3.434077in}}%
\pgfpathcurveto{\pgfqpoint{0.999303in}{3.426264in}}{\pgfqpoint{0.994912in}{3.415665in}}{\pgfqpoint{0.994912in}{3.404615in}}%
\pgfpathcurveto{\pgfqpoint{0.994912in}{3.393564in}}{\pgfqpoint{0.999303in}{3.382965in}}{\pgfqpoint{1.007116in}{3.375152in}}%
\pgfpathcurveto{\pgfqpoint{1.014930in}{3.367338in}}{\pgfqpoint{1.025529in}{3.362948in}}{\pgfqpoint{1.036579in}{3.362948in}}%
\pgfpathclose%
\pgfusepath{stroke,fill}%
\end{pgfscope}%
\begin{pgfscope}%
\pgfpathrectangle{\pgfqpoint{0.648703in}{0.548769in}}{\pgfqpoint{5.201297in}{3.102590in}}%
\pgfusepath{clip}%
\pgfsetbuttcap%
\pgfsetroundjoin%
\definecolor{currentfill}{rgb}{1.000000,0.498039,0.054902}%
\pgfsetfillcolor{currentfill}%
\pgfsetlinewidth{1.003750pt}%
\definecolor{currentstroke}{rgb}{1.000000,0.498039,0.054902}%
\pgfsetstrokecolor{currentstroke}%
\pgfsetdash{}{0pt}%
\pgfpathmoveto{\pgfqpoint{1.571119in}{3.257231in}}%
\pgfpathcurveto{\pgfqpoint{1.582169in}{3.257231in}}{\pgfqpoint{1.592768in}{3.261621in}}{\pgfqpoint{1.600582in}{3.269435in}}%
\pgfpathcurveto{\pgfqpoint{1.608395in}{3.277248in}}{\pgfqpoint{1.612786in}{3.287847in}}{\pgfqpoint{1.612786in}{3.298897in}}%
\pgfpathcurveto{\pgfqpoint{1.612786in}{3.309947in}}{\pgfqpoint{1.608395in}{3.320546in}}{\pgfqpoint{1.600582in}{3.328360in}}%
\pgfpathcurveto{\pgfqpoint{1.592768in}{3.336174in}}{\pgfqpoint{1.582169in}{3.340564in}}{\pgfqpoint{1.571119in}{3.340564in}}%
\pgfpathcurveto{\pgfqpoint{1.560069in}{3.340564in}}{\pgfqpoint{1.549470in}{3.336174in}}{\pgfqpoint{1.541656in}{3.328360in}}%
\pgfpathcurveto{\pgfqpoint{1.533843in}{3.320546in}}{\pgfqpoint{1.529452in}{3.309947in}}{\pgfqpoint{1.529452in}{3.298897in}}%
\pgfpathcurveto{\pgfqpoint{1.529452in}{3.287847in}}{\pgfqpoint{1.533843in}{3.277248in}}{\pgfqpoint{1.541656in}{3.269435in}}%
\pgfpathcurveto{\pgfqpoint{1.549470in}{3.261621in}}{\pgfqpoint{1.560069in}{3.257231in}}{\pgfqpoint{1.571119in}{3.257231in}}%
\pgfpathclose%
\pgfusepath{stroke,fill}%
\end{pgfscope}%
\begin{pgfscope}%
\pgfpathrectangle{\pgfqpoint{0.648703in}{0.548769in}}{\pgfqpoint{5.201297in}{3.102590in}}%
\pgfusepath{clip}%
\pgfsetbuttcap%
\pgfsetroundjoin%
\definecolor{currentfill}{rgb}{1.000000,0.498039,0.054902}%
\pgfsetfillcolor{currentfill}%
\pgfsetlinewidth{1.003750pt}%
\definecolor{currentstroke}{rgb}{1.000000,0.498039,0.054902}%
\pgfsetstrokecolor{currentstroke}%
\pgfsetdash{}{0pt}%
\pgfpathmoveto{\pgfqpoint{1.571119in}{3.189572in}}%
\pgfpathcurveto{\pgfqpoint{1.582169in}{3.189572in}}{\pgfqpoint{1.592768in}{3.193962in}}{\pgfqpoint{1.600582in}{3.201775in}}%
\pgfpathcurveto{\pgfqpoint{1.608395in}{3.209589in}}{\pgfqpoint{1.612786in}{3.220188in}}{\pgfqpoint{1.612786in}{3.231238in}}%
\pgfpathcurveto{\pgfqpoint{1.612786in}{3.242288in}}{\pgfqpoint{1.608395in}{3.252887in}}{\pgfqpoint{1.600582in}{3.260701in}}%
\pgfpathcurveto{\pgfqpoint{1.592768in}{3.268515in}}{\pgfqpoint{1.582169in}{3.272905in}}{\pgfqpoint{1.571119in}{3.272905in}}%
\pgfpathcurveto{\pgfqpoint{1.560069in}{3.272905in}}{\pgfqpoint{1.549470in}{3.268515in}}{\pgfqpoint{1.541656in}{3.260701in}}%
\pgfpathcurveto{\pgfqpoint{1.533843in}{3.252887in}}{\pgfqpoint{1.529452in}{3.242288in}}{\pgfqpoint{1.529452in}{3.231238in}}%
\pgfpathcurveto{\pgfqpoint{1.529452in}{3.220188in}}{\pgfqpoint{1.533843in}{3.209589in}}{\pgfqpoint{1.541656in}{3.201775in}}%
\pgfpathcurveto{\pgfqpoint{1.549470in}{3.193962in}}{\pgfqpoint{1.560069in}{3.189572in}}{\pgfqpoint{1.571119in}{3.189572in}}%
\pgfpathclose%
\pgfusepath{stroke,fill}%
\end{pgfscope}%
\begin{pgfscope}%
\pgfpathrectangle{\pgfqpoint{0.648703in}{0.548769in}}{\pgfqpoint{5.201297in}{3.102590in}}%
\pgfusepath{clip}%
\pgfsetbuttcap%
\pgfsetroundjoin%
\definecolor{currentfill}{rgb}{1.000000,0.498039,0.054902}%
\pgfsetfillcolor{currentfill}%
\pgfsetlinewidth{1.003750pt}%
\definecolor{currentstroke}{rgb}{1.000000,0.498039,0.054902}%
\pgfsetstrokecolor{currentstroke}%
\pgfsetdash{}{0pt}%
\pgfpathmoveto{\pgfqpoint{1.081124in}{3.189572in}}%
\pgfpathcurveto{\pgfqpoint{1.092174in}{3.189572in}}{\pgfqpoint{1.102773in}{3.193962in}}{\pgfqpoint{1.110587in}{3.201775in}}%
\pgfpathcurveto{\pgfqpoint{1.118400in}{3.209589in}}{\pgfqpoint{1.122791in}{3.220188in}}{\pgfqpoint{1.122791in}{3.231238in}}%
\pgfpathcurveto{\pgfqpoint{1.122791in}{3.242288in}}{\pgfqpoint{1.118400in}{3.252887in}}{\pgfqpoint{1.110587in}{3.260701in}}%
\pgfpathcurveto{\pgfqpoint{1.102773in}{3.268515in}}{\pgfqpoint{1.092174in}{3.272905in}}{\pgfqpoint{1.081124in}{3.272905in}}%
\pgfpathcurveto{\pgfqpoint{1.070074in}{3.272905in}}{\pgfqpoint{1.059475in}{3.268515in}}{\pgfqpoint{1.051661in}{3.260701in}}%
\pgfpathcurveto{\pgfqpoint{1.043848in}{3.252887in}}{\pgfqpoint{1.039457in}{3.242288in}}{\pgfqpoint{1.039457in}{3.231238in}}%
\pgfpathcurveto{\pgfqpoint{1.039457in}{3.220188in}}{\pgfqpoint{1.043848in}{3.209589in}}{\pgfqpoint{1.051661in}{3.201775in}}%
\pgfpathcurveto{\pgfqpoint{1.059475in}{3.193962in}}{\pgfqpoint{1.070074in}{3.189572in}}{\pgfqpoint{1.081124in}{3.189572in}}%
\pgfpathclose%
\pgfusepath{stroke,fill}%
\end{pgfscope}%
\begin{pgfscope}%
\pgfpathrectangle{\pgfqpoint{0.648703in}{0.548769in}}{\pgfqpoint{5.201297in}{3.102590in}}%
\pgfusepath{clip}%
\pgfsetbuttcap%
\pgfsetroundjoin%
\definecolor{currentfill}{rgb}{0.121569,0.466667,0.705882}%
\pgfsetfillcolor{currentfill}%
\pgfsetlinewidth{1.003750pt}%
\definecolor{currentstroke}{rgb}{0.121569,0.466667,0.705882}%
\pgfsetstrokecolor{currentstroke}%
\pgfsetdash{}{0pt}%
\pgfpathmoveto{\pgfqpoint{1.337258in}{3.181114in}}%
\pgfpathcurveto{\pgfqpoint{1.348308in}{3.181114in}}{\pgfqpoint{1.358907in}{3.185504in}}{\pgfqpoint{1.366720in}{3.193318in}}%
\pgfpathcurveto{\pgfqpoint{1.374534in}{3.201132in}}{\pgfqpoint{1.378924in}{3.211731in}}{\pgfqpoint{1.378924in}{3.222781in}}%
\pgfpathcurveto{\pgfqpoint{1.378924in}{3.233831in}}{\pgfqpoint{1.374534in}{3.244430in}}{\pgfqpoint{1.366720in}{3.252244in}}%
\pgfpathcurveto{\pgfqpoint{1.358907in}{3.260057in}}{\pgfqpoint{1.348308in}{3.264448in}}{\pgfqpoint{1.337258in}{3.264448in}}%
\pgfpathcurveto{\pgfqpoint{1.326208in}{3.264448in}}{\pgfqpoint{1.315609in}{3.260057in}}{\pgfqpoint{1.307795in}{3.252244in}}%
\pgfpathcurveto{\pgfqpoint{1.299981in}{3.244430in}}{\pgfqpoint{1.295591in}{3.233831in}}{\pgfqpoint{1.295591in}{3.222781in}}%
\pgfpathcurveto{\pgfqpoint{1.295591in}{3.211731in}}{\pgfqpoint{1.299981in}{3.201132in}}{\pgfqpoint{1.307795in}{3.193318in}}%
\pgfpathcurveto{\pgfqpoint{1.315609in}{3.185504in}}{\pgfqpoint{1.326208in}{3.181114in}}{\pgfqpoint{1.337258in}{3.181114in}}%
\pgfpathclose%
\pgfusepath{stroke,fill}%
\end{pgfscope}%
\begin{pgfscope}%
\pgfpathrectangle{\pgfqpoint{0.648703in}{0.548769in}}{\pgfqpoint{5.201297in}{3.102590in}}%
\pgfusepath{clip}%
\pgfsetbuttcap%
\pgfsetroundjoin%
\definecolor{currentfill}{rgb}{1.000000,0.498039,0.054902}%
\pgfsetfillcolor{currentfill}%
\pgfsetlinewidth{1.003750pt}%
\definecolor{currentstroke}{rgb}{1.000000,0.498039,0.054902}%
\pgfsetstrokecolor{currentstroke}%
\pgfsetdash{}{0pt}%
\pgfpathmoveto{\pgfqpoint{1.826768in}{3.185343in}}%
\pgfpathcurveto{\pgfqpoint{1.837819in}{3.185343in}}{\pgfqpoint{1.848418in}{3.189733in}}{\pgfqpoint{1.856231in}{3.197547in}}%
\pgfpathcurveto{\pgfqpoint{1.864045in}{3.205360in}}{\pgfqpoint{1.868435in}{3.215959in}}{\pgfqpoint{1.868435in}{3.227010in}}%
\pgfpathcurveto{\pgfqpoint{1.868435in}{3.238060in}}{\pgfqpoint{1.864045in}{3.248659in}}{\pgfqpoint{1.856231in}{3.256472in}}%
\pgfpathcurveto{\pgfqpoint{1.848418in}{3.264286in}}{\pgfqpoint{1.837819in}{3.268676in}}{\pgfqpoint{1.826768in}{3.268676in}}%
\pgfpathcurveto{\pgfqpoint{1.815718in}{3.268676in}}{\pgfqpoint{1.805119in}{3.264286in}}{\pgfqpoint{1.797306in}{3.256472in}}%
\pgfpathcurveto{\pgfqpoint{1.789492in}{3.248659in}}{\pgfqpoint{1.785102in}{3.238060in}}{\pgfqpoint{1.785102in}{3.227010in}}%
\pgfpathcurveto{\pgfqpoint{1.785102in}{3.215959in}}{\pgfqpoint{1.789492in}{3.205360in}}{\pgfqpoint{1.797306in}{3.197547in}}%
\pgfpathcurveto{\pgfqpoint{1.805119in}{3.189733in}}{\pgfqpoint{1.815718in}{3.185343in}}{\pgfqpoint{1.826768in}{3.185343in}}%
\pgfpathclose%
\pgfusepath{stroke,fill}%
\end{pgfscope}%
\begin{pgfscope}%
\pgfpathrectangle{\pgfqpoint{0.648703in}{0.548769in}}{\pgfqpoint{5.201297in}{3.102590in}}%
\pgfusepath{clip}%
\pgfsetbuttcap%
\pgfsetroundjoin%
\definecolor{currentfill}{rgb}{0.121569,0.466667,0.705882}%
\pgfsetfillcolor{currentfill}%
\pgfsetlinewidth{1.003750pt}%
\definecolor{currentstroke}{rgb}{0.121569,0.466667,0.705882}%
\pgfsetstrokecolor{currentstroke}%
\pgfsetdash{}{0pt}%
\pgfpathmoveto{\pgfqpoint{5.613577in}{0.681958in}}%
\pgfpathcurveto{\pgfqpoint{5.624628in}{0.681958in}}{\pgfqpoint{5.635227in}{0.686349in}}{\pgfqpoint{5.643040in}{0.694162in}}%
\pgfpathcurveto{\pgfqpoint{5.650854in}{0.701976in}}{\pgfqpoint{5.655244in}{0.712575in}}{\pgfqpoint{5.655244in}{0.723625in}}%
\pgfpathcurveto{\pgfqpoint{5.655244in}{0.734675in}}{\pgfqpoint{5.650854in}{0.745274in}}{\pgfqpoint{5.643040in}{0.753088in}}%
\pgfpathcurveto{\pgfqpoint{5.635227in}{0.760902in}}{\pgfqpoint{5.624628in}{0.765292in}}{\pgfqpoint{5.613577in}{0.765292in}}%
\pgfpathcurveto{\pgfqpoint{5.602527in}{0.765292in}}{\pgfqpoint{5.591928in}{0.760902in}}{\pgfqpoint{5.584115in}{0.753088in}}%
\pgfpathcurveto{\pgfqpoint{5.576301in}{0.745274in}}{\pgfqpoint{5.571911in}{0.734675in}}{\pgfqpoint{5.571911in}{0.723625in}}%
\pgfpathcurveto{\pgfqpoint{5.571911in}{0.712575in}}{\pgfqpoint{5.576301in}{0.701976in}}{\pgfqpoint{5.584115in}{0.694162in}}%
\pgfpathcurveto{\pgfqpoint{5.591928in}{0.686349in}}{\pgfqpoint{5.602527in}{0.681958in}}{\pgfqpoint{5.613577in}{0.681958in}}%
\pgfpathclose%
\pgfusepath{stroke,fill}%
\end{pgfscope}%
\begin{pgfscope}%
\pgfpathrectangle{\pgfqpoint{0.648703in}{0.548769in}}{\pgfqpoint{5.201297in}{3.102590in}}%
\pgfusepath{clip}%
\pgfsetbuttcap%
\pgfsetroundjoin%
\definecolor{currentfill}{rgb}{0.839216,0.152941,0.156863}%
\pgfsetfillcolor{currentfill}%
\pgfsetlinewidth{1.003750pt}%
\definecolor{currentstroke}{rgb}{0.839216,0.152941,0.156863}%
\pgfsetstrokecolor{currentstroke}%
\pgfsetdash{}{0pt}%
\pgfpathmoveto{\pgfqpoint{1.050499in}{3.214944in}}%
\pgfpathcurveto{\pgfqpoint{1.061549in}{3.214944in}}{\pgfqpoint{1.072148in}{3.219334in}}{\pgfqpoint{1.079962in}{3.227148in}}%
\pgfpathcurveto{\pgfqpoint{1.087776in}{3.234961in}}{\pgfqpoint{1.092166in}{3.245560in}}{\pgfqpoint{1.092166in}{3.256610in}}%
\pgfpathcurveto{\pgfqpoint{1.092166in}{3.267661in}}{\pgfqpoint{1.087776in}{3.278260in}}{\pgfqpoint{1.079962in}{3.286073in}}%
\pgfpathcurveto{\pgfqpoint{1.072148in}{3.293887in}}{\pgfqpoint{1.061549in}{3.298277in}}{\pgfqpoint{1.050499in}{3.298277in}}%
\pgfpathcurveto{\pgfqpoint{1.039449in}{3.298277in}}{\pgfqpoint{1.028850in}{3.293887in}}{\pgfqpoint{1.021037in}{3.286073in}}%
\pgfpathcurveto{\pgfqpoint{1.013223in}{3.278260in}}{\pgfqpoint{1.008833in}{3.267661in}}{\pgfqpoint{1.008833in}{3.256610in}}%
\pgfpathcurveto{\pgfqpoint{1.008833in}{3.245560in}}{\pgfqpoint{1.013223in}{3.234961in}}{\pgfqpoint{1.021037in}{3.227148in}}%
\pgfpathcurveto{\pgfqpoint{1.028850in}{3.219334in}}{\pgfqpoint{1.039449in}{3.214944in}}{\pgfqpoint{1.050499in}{3.214944in}}%
\pgfpathclose%
\pgfusepath{stroke,fill}%
\end{pgfscope}%
\begin{pgfscope}%
\pgfpathrectangle{\pgfqpoint{0.648703in}{0.548769in}}{\pgfqpoint{5.201297in}{3.102590in}}%
\pgfusepath{clip}%
\pgfsetbuttcap%
\pgfsetroundjoin%
\definecolor{currentfill}{rgb}{1.000000,0.498039,0.054902}%
\pgfsetfillcolor{currentfill}%
\pgfsetlinewidth{1.003750pt}%
\definecolor{currentstroke}{rgb}{1.000000,0.498039,0.054902}%
\pgfsetstrokecolor{currentstroke}%
\pgfsetdash{}{0pt}%
\pgfpathmoveto{\pgfqpoint{0.917792in}{3.185343in}}%
\pgfpathcurveto{\pgfqpoint{0.928842in}{3.185343in}}{\pgfqpoint{0.939441in}{3.189733in}}{\pgfqpoint{0.947255in}{3.197547in}}%
\pgfpathcurveto{\pgfqpoint{0.955069in}{3.205360in}}{\pgfqpoint{0.959459in}{3.215959in}}{\pgfqpoint{0.959459in}{3.227010in}}%
\pgfpathcurveto{\pgfqpoint{0.959459in}{3.238060in}}{\pgfqpoint{0.955069in}{3.248659in}}{\pgfqpoint{0.947255in}{3.256472in}}%
\pgfpathcurveto{\pgfqpoint{0.939441in}{3.264286in}}{\pgfqpoint{0.928842in}{3.268676in}}{\pgfqpoint{0.917792in}{3.268676in}}%
\pgfpathcurveto{\pgfqpoint{0.906742in}{3.268676in}}{\pgfqpoint{0.896143in}{3.264286in}}{\pgfqpoint{0.888330in}{3.256472in}}%
\pgfpathcurveto{\pgfqpoint{0.880516in}{3.248659in}}{\pgfqpoint{0.876126in}{3.238060in}}{\pgfqpoint{0.876126in}{3.227010in}}%
\pgfpathcurveto{\pgfqpoint{0.876126in}{3.215959in}}{\pgfqpoint{0.880516in}{3.205360in}}{\pgfqpoint{0.888330in}{3.197547in}}%
\pgfpathcurveto{\pgfqpoint{0.896143in}{3.189733in}}{\pgfqpoint{0.906742in}{3.185343in}}{\pgfqpoint{0.917792in}{3.185343in}}%
\pgfpathclose%
\pgfusepath{stroke,fill}%
\end{pgfscope}%
\begin{pgfscope}%
\pgfpathrectangle{\pgfqpoint{0.648703in}{0.548769in}}{\pgfqpoint{5.201297in}{3.102590in}}%
\pgfusepath{clip}%
\pgfsetbuttcap%
\pgfsetroundjoin%
\definecolor{currentfill}{rgb}{0.121569,0.466667,0.705882}%
\pgfsetfillcolor{currentfill}%
\pgfsetlinewidth{1.003750pt}%
\definecolor{currentstroke}{rgb}{0.121569,0.466667,0.705882}%
\pgfsetstrokecolor{currentstroke}%
\pgfsetdash{}{0pt}%
\pgfpathmoveto{\pgfqpoint{1.295497in}{1.138657in}}%
\pgfpathcurveto{\pgfqpoint{1.306547in}{1.138657in}}{\pgfqpoint{1.317146in}{1.143047in}}{\pgfqpoint{1.324960in}{1.150861in}}%
\pgfpathcurveto{\pgfqpoint{1.332773in}{1.158674in}}{\pgfqpoint{1.337163in}{1.169274in}}{\pgfqpoint{1.337163in}{1.180324in}}%
\pgfpathcurveto{\pgfqpoint{1.337163in}{1.191374in}}{\pgfqpoint{1.332773in}{1.201973in}}{\pgfqpoint{1.324960in}{1.209786in}}%
\pgfpathcurveto{\pgfqpoint{1.317146in}{1.217600in}}{\pgfqpoint{1.306547in}{1.221990in}}{\pgfqpoint{1.295497in}{1.221990in}}%
\pgfpathcurveto{\pgfqpoint{1.284447in}{1.221990in}}{\pgfqpoint{1.273848in}{1.217600in}}{\pgfqpoint{1.266034in}{1.209786in}}%
\pgfpathcurveto{\pgfqpoint{1.258220in}{1.201973in}}{\pgfqpoint{1.253830in}{1.191374in}}{\pgfqpoint{1.253830in}{1.180324in}}%
\pgfpathcurveto{\pgfqpoint{1.253830in}{1.169274in}}{\pgfqpoint{1.258220in}{1.158674in}}{\pgfqpoint{1.266034in}{1.150861in}}%
\pgfpathcurveto{\pgfqpoint{1.273848in}{1.143047in}}{\pgfqpoint{1.284447in}{1.138657in}}{\pgfqpoint{1.295497in}{1.138657in}}%
\pgfpathclose%
\pgfusepath{stroke,fill}%
\end{pgfscope}%
\begin{pgfscope}%
\pgfpathrectangle{\pgfqpoint{0.648703in}{0.548769in}}{\pgfqpoint{5.201297in}{3.102590in}}%
\pgfusepath{clip}%
\pgfsetbuttcap%
\pgfsetroundjoin%
\definecolor{currentfill}{rgb}{0.121569,0.466667,0.705882}%
\pgfsetfillcolor{currentfill}%
\pgfsetlinewidth{1.003750pt}%
\definecolor{currentstroke}{rgb}{0.121569,0.466667,0.705882}%
\pgfsetstrokecolor{currentstroke}%
\pgfsetdash{}{0pt}%
\pgfpathmoveto{\pgfqpoint{1.504301in}{1.087913in}}%
\pgfpathcurveto{\pgfqpoint{1.515352in}{1.087913in}}{\pgfqpoint{1.525951in}{1.092303in}}{\pgfqpoint{1.533764in}{1.100117in}}%
\pgfpathcurveto{\pgfqpoint{1.541578in}{1.107930in}}{\pgfqpoint{1.545968in}{1.118529in}}{\pgfqpoint{1.545968in}{1.129579in}}%
\pgfpathcurveto{\pgfqpoint{1.545968in}{1.140629in}}{\pgfqpoint{1.541578in}{1.151229in}}{\pgfqpoint{1.533764in}{1.159042in}}%
\pgfpathcurveto{\pgfqpoint{1.525951in}{1.166856in}}{\pgfqpoint{1.515352in}{1.171246in}}{\pgfqpoint{1.504301in}{1.171246in}}%
\pgfpathcurveto{\pgfqpoint{1.493251in}{1.171246in}}{\pgfqpoint{1.482652in}{1.166856in}}{\pgfqpoint{1.474839in}{1.159042in}}%
\pgfpathcurveto{\pgfqpoint{1.467025in}{1.151229in}}{\pgfqpoint{1.462635in}{1.140629in}}{\pgfqpoint{1.462635in}{1.129579in}}%
\pgfpathcurveto{\pgfqpoint{1.462635in}{1.118529in}}{\pgfqpoint{1.467025in}{1.107930in}}{\pgfqpoint{1.474839in}{1.100117in}}%
\pgfpathcurveto{\pgfqpoint{1.482652in}{1.092303in}}{\pgfqpoint{1.493251in}{1.087913in}}{\pgfqpoint{1.504301in}{1.087913in}}%
\pgfpathclose%
\pgfusepath{stroke,fill}%
\end{pgfscope}%
\begin{pgfscope}%
\pgfpathrectangle{\pgfqpoint{0.648703in}{0.548769in}}{\pgfqpoint{5.201297in}{3.102590in}}%
\pgfusepath{clip}%
\pgfsetbuttcap%
\pgfsetroundjoin%
\definecolor{currentfill}{rgb}{1.000000,0.498039,0.054902}%
\pgfsetfillcolor{currentfill}%
\pgfsetlinewidth{1.003750pt}%
\definecolor{currentstroke}{rgb}{1.000000,0.498039,0.054902}%
\pgfsetstrokecolor{currentstroke}%
\pgfsetdash{}{0pt}%
\pgfpathmoveto{\pgfqpoint{1.999865in}{3.189572in}}%
\pgfpathcurveto{\pgfqpoint{2.010915in}{3.189572in}}{\pgfqpoint{2.021514in}{3.193962in}}{\pgfqpoint{2.029327in}{3.201775in}}%
\pgfpathcurveto{\pgfqpoint{2.037141in}{3.209589in}}{\pgfqpoint{2.041531in}{3.220188in}}{\pgfqpoint{2.041531in}{3.231238in}}%
\pgfpathcurveto{\pgfqpoint{2.041531in}{3.242288in}}{\pgfqpoint{2.037141in}{3.252887in}}{\pgfqpoint{2.029327in}{3.260701in}}%
\pgfpathcurveto{\pgfqpoint{2.021514in}{3.268515in}}{\pgfqpoint{2.010915in}{3.272905in}}{\pgfqpoint{1.999865in}{3.272905in}}%
\pgfpathcurveto{\pgfqpoint{1.988814in}{3.272905in}}{\pgfqpoint{1.978215in}{3.268515in}}{\pgfqpoint{1.970402in}{3.260701in}}%
\pgfpathcurveto{\pgfqpoint{1.962588in}{3.252887in}}{\pgfqpoint{1.958198in}{3.242288in}}{\pgfqpoint{1.958198in}{3.231238in}}%
\pgfpathcurveto{\pgfqpoint{1.958198in}{3.220188in}}{\pgfqpoint{1.962588in}{3.209589in}}{\pgfqpoint{1.970402in}{3.201775in}}%
\pgfpathcurveto{\pgfqpoint{1.978215in}{3.193962in}}{\pgfqpoint{1.988814in}{3.189572in}}{\pgfqpoint{1.999865in}{3.189572in}}%
\pgfpathclose%
\pgfusepath{stroke,fill}%
\end{pgfscope}%
\begin{pgfscope}%
\pgfpathrectangle{\pgfqpoint{0.648703in}{0.548769in}}{\pgfqpoint{5.201297in}{3.102590in}}%
\pgfusepath{clip}%
\pgfsetbuttcap%
\pgfsetroundjoin%
\definecolor{currentfill}{rgb}{0.121569,0.466667,0.705882}%
\pgfsetfillcolor{currentfill}%
\pgfsetlinewidth{1.003750pt}%
\definecolor{currentstroke}{rgb}{0.121569,0.466667,0.705882}%
\pgfsetstrokecolor{currentstroke}%
\pgfsetdash{}{0pt}%
\pgfpathmoveto{\pgfqpoint{1.657589in}{1.024482in}}%
\pgfpathcurveto{\pgfqpoint{1.668639in}{1.024482in}}{\pgfqpoint{1.679238in}{1.028873in}}{\pgfqpoint{1.687051in}{1.036686in}}%
\pgfpathcurveto{\pgfqpoint{1.694865in}{1.044500in}}{\pgfqpoint{1.699255in}{1.055099in}}{\pgfqpoint{1.699255in}{1.066149in}}%
\pgfpathcurveto{\pgfqpoint{1.699255in}{1.077199in}}{\pgfqpoint{1.694865in}{1.087798in}}{\pgfqpoint{1.687051in}{1.095612in}}%
\pgfpathcurveto{\pgfqpoint{1.679238in}{1.103425in}}{\pgfqpoint{1.668639in}{1.107816in}}{\pgfqpoint{1.657589in}{1.107816in}}%
\pgfpathcurveto{\pgfqpoint{1.646539in}{1.107816in}}{\pgfqpoint{1.635939in}{1.103425in}}{\pgfqpoint{1.628126in}{1.095612in}}%
\pgfpathcurveto{\pgfqpoint{1.620312in}{1.087798in}}{\pgfqpoint{1.615922in}{1.077199in}}{\pgfqpoint{1.615922in}{1.066149in}}%
\pgfpathcurveto{\pgfqpoint{1.615922in}{1.055099in}}{\pgfqpoint{1.620312in}{1.044500in}}{\pgfqpoint{1.628126in}{1.036686in}}%
\pgfpathcurveto{\pgfqpoint{1.635939in}{1.028873in}}{\pgfqpoint{1.646539in}{1.024482in}}{\pgfqpoint{1.657589in}{1.024482in}}%
\pgfpathclose%
\pgfusepath{stroke,fill}%
\end{pgfscope}%
\begin{pgfscope}%
\pgfpathrectangle{\pgfqpoint{0.648703in}{0.548769in}}{\pgfqpoint{5.201297in}{3.102590in}}%
\pgfusepath{clip}%
\pgfsetbuttcap%
\pgfsetroundjoin%
\definecolor{currentfill}{rgb}{0.121569,0.466667,0.705882}%
\pgfsetfillcolor{currentfill}%
\pgfsetlinewidth{1.003750pt}%
\definecolor{currentstroke}{rgb}{0.121569,0.466667,0.705882}%
\pgfsetstrokecolor{currentstroke}%
\pgfsetdash{}{0pt}%
\pgfpathmoveto{\pgfqpoint{1.550119in}{0.813048in}}%
\pgfpathcurveto{\pgfqpoint{1.561169in}{0.813048in}}{\pgfqpoint{1.571768in}{0.817438in}}{\pgfqpoint{1.579582in}{0.825252in}}%
\pgfpathcurveto{\pgfqpoint{1.587396in}{0.833065in}}{\pgfqpoint{1.591786in}{0.843664in}}{\pgfqpoint{1.591786in}{0.854715in}}%
\pgfpathcurveto{\pgfqpoint{1.591786in}{0.865765in}}{\pgfqpoint{1.587396in}{0.876364in}}{\pgfqpoint{1.579582in}{0.884177in}}%
\pgfpathcurveto{\pgfqpoint{1.571768in}{0.891991in}}{\pgfqpoint{1.561169in}{0.896381in}}{\pgfqpoint{1.550119in}{0.896381in}}%
\pgfpathcurveto{\pgfqpoint{1.539069in}{0.896381in}}{\pgfqpoint{1.528470in}{0.891991in}}{\pgfqpoint{1.520656in}{0.884177in}}%
\pgfpathcurveto{\pgfqpoint{1.512843in}{0.876364in}}{\pgfqpoint{1.508452in}{0.865765in}}{\pgfqpoint{1.508452in}{0.854715in}}%
\pgfpathcurveto{\pgfqpoint{1.508452in}{0.843664in}}{\pgfqpoint{1.512843in}{0.833065in}}{\pgfqpoint{1.520656in}{0.825252in}}%
\pgfpathcurveto{\pgfqpoint{1.528470in}{0.817438in}}{\pgfqpoint{1.539069in}{0.813048in}}{\pgfqpoint{1.550119in}{0.813048in}}%
\pgfpathclose%
\pgfusepath{stroke,fill}%
\end{pgfscope}%
\begin{pgfscope}%
\pgfpathrectangle{\pgfqpoint{0.648703in}{0.548769in}}{\pgfqpoint{5.201297in}{3.102590in}}%
\pgfusepath{clip}%
\pgfsetbuttcap%
\pgfsetroundjoin%
\definecolor{currentfill}{rgb}{0.839216,0.152941,0.156863}%
\pgfsetfillcolor{currentfill}%
\pgfsetlinewidth{1.003750pt}%
\definecolor{currentstroke}{rgb}{0.839216,0.152941,0.156863}%
\pgfsetstrokecolor{currentstroke}%
\pgfsetdash{}{0pt}%
\pgfpathmoveto{\pgfqpoint{1.231892in}{3.202258in}}%
\pgfpathcurveto{\pgfqpoint{1.242942in}{3.202258in}}{\pgfqpoint{1.253541in}{3.206648in}}{\pgfqpoint{1.261354in}{3.214462in}}%
\pgfpathcurveto{\pgfqpoint{1.269168in}{3.222275in}}{\pgfqpoint{1.273558in}{3.232874in}}{\pgfqpoint{1.273558in}{3.243924in}}%
\pgfpathcurveto{\pgfqpoint{1.273558in}{3.254974in}}{\pgfqpoint{1.269168in}{3.265573in}}{\pgfqpoint{1.261354in}{3.273387in}}%
\pgfpathcurveto{\pgfqpoint{1.253541in}{3.281201in}}{\pgfqpoint{1.242942in}{3.285591in}}{\pgfqpoint{1.231892in}{3.285591in}}%
\pgfpathcurveto{\pgfqpoint{1.220842in}{3.285591in}}{\pgfqpoint{1.210242in}{3.281201in}}{\pgfqpoint{1.202429in}{3.273387in}}%
\pgfpathcurveto{\pgfqpoint{1.194615in}{3.265573in}}{\pgfqpoint{1.190225in}{3.254974in}}{\pgfqpoint{1.190225in}{3.243924in}}%
\pgfpathcurveto{\pgfqpoint{1.190225in}{3.232874in}}{\pgfqpoint{1.194615in}{3.222275in}}{\pgfqpoint{1.202429in}{3.214462in}}%
\pgfpathcurveto{\pgfqpoint{1.210242in}{3.206648in}}{\pgfqpoint{1.220842in}{3.202258in}}{\pgfqpoint{1.231892in}{3.202258in}}%
\pgfpathclose%
\pgfusepath{stroke,fill}%
\end{pgfscope}%
\begin{pgfscope}%
\pgfsetbuttcap%
\pgfsetroundjoin%
\definecolor{currentfill}{rgb}{0.000000,0.000000,0.000000}%
\pgfsetfillcolor{currentfill}%
\pgfsetlinewidth{0.803000pt}%
\definecolor{currentstroke}{rgb}{0.000000,0.000000,0.000000}%
\pgfsetstrokecolor{currentstroke}%
\pgfsetdash{}{0pt}%
\pgfsys@defobject{currentmarker}{\pgfqpoint{0.000000in}{-0.048611in}}{\pgfqpoint{0.000000in}{0.000000in}}{%
\pgfpathmoveto{\pgfqpoint{0.000000in}{0.000000in}}%
\pgfpathlineto{\pgfqpoint{0.000000in}{-0.048611in}}%
\pgfusepath{stroke,fill}%
}%
\begin{pgfscope}%
\pgfsys@transformshift{0.836126in}{0.548769in}%
\pgfsys@useobject{currentmarker}{}%
\end{pgfscope}%
\end{pgfscope}%
\begin{pgfscope}%
\definecolor{textcolor}{rgb}{0.000000,0.000000,0.000000}%
\pgfsetstrokecolor{textcolor}%
\pgfsetfillcolor{textcolor}%
\pgftext[x=0.836126in,y=0.451547in,,top]{\color{textcolor}\sffamily\fontsize{10.000000}{12.000000}\selectfont \(\displaystyle {-1}\)}%
\end{pgfscope}%
\begin{pgfscope}%
\pgfsetbuttcap%
\pgfsetroundjoin%
\definecolor{currentfill}{rgb}{0.000000,0.000000,0.000000}%
\pgfsetfillcolor{currentfill}%
\pgfsetlinewidth{0.803000pt}%
\definecolor{currentstroke}{rgb}{0.000000,0.000000,0.000000}%
\pgfsetstrokecolor{currentstroke}%
\pgfsetdash{}{0pt}%
\pgfsys@defobject{currentmarker}{\pgfqpoint{0.000000in}{-0.048611in}}{\pgfqpoint{0.000000in}{0.000000in}}{%
\pgfpathmoveto{\pgfqpoint{0.000000in}{0.000000in}}%
\pgfpathlineto{\pgfqpoint{0.000000in}{-0.048611in}}%
\pgfusepath{stroke,fill}%
}%
\begin{pgfscope}%
\pgfsys@transformshift{1.571119in}{0.548769in}%
\pgfsys@useobject{currentmarker}{}%
\end{pgfscope}%
\end{pgfscope}%
\begin{pgfscope}%
\definecolor{textcolor}{rgb}{0.000000,0.000000,0.000000}%
\pgfsetstrokecolor{textcolor}%
\pgfsetfillcolor{textcolor}%
\pgftext[x=1.571119in,y=0.451547in,,top]{\color{textcolor}\sffamily\fontsize{10.000000}{12.000000}\selectfont \(\displaystyle {0}\)}%
\end{pgfscope}%
\begin{pgfscope}%
\pgfsetbuttcap%
\pgfsetroundjoin%
\definecolor{currentfill}{rgb}{0.000000,0.000000,0.000000}%
\pgfsetfillcolor{currentfill}%
\pgfsetlinewidth{0.803000pt}%
\definecolor{currentstroke}{rgb}{0.000000,0.000000,0.000000}%
\pgfsetstrokecolor{currentstroke}%
\pgfsetdash{}{0pt}%
\pgfsys@defobject{currentmarker}{\pgfqpoint{0.000000in}{-0.048611in}}{\pgfqpoint{0.000000in}{0.000000in}}{%
\pgfpathmoveto{\pgfqpoint{0.000000in}{0.000000in}}%
\pgfpathlineto{\pgfqpoint{0.000000in}{-0.048611in}}%
\pgfusepath{stroke,fill}%
}%
\begin{pgfscope}%
\pgfsys@transformshift{2.306111in}{0.548769in}%
\pgfsys@useobject{currentmarker}{}%
\end{pgfscope}%
\end{pgfscope}%
\begin{pgfscope}%
\definecolor{textcolor}{rgb}{0.000000,0.000000,0.000000}%
\pgfsetstrokecolor{textcolor}%
\pgfsetfillcolor{textcolor}%
\pgftext[x=2.306111in,y=0.451547in,,top]{\color{textcolor}\sffamily\fontsize{10.000000}{12.000000}\selectfont \(\displaystyle {1}\)}%
\end{pgfscope}%
\begin{pgfscope}%
\pgfsetbuttcap%
\pgfsetroundjoin%
\definecolor{currentfill}{rgb}{0.000000,0.000000,0.000000}%
\pgfsetfillcolor{currentfill}%
\pgfsetlinewidth{0.803000pt}%
\definecolor{currentstroke}{rgb}{0.000000,0.000000,0.000000}%
\pgfsetstrokecolor{currentstroke}%
\pgfsetdash{}{0pt}%
\pgfsys@defobject{currentmarker}{\pgfqpoint{0.000000in}{-0.048611in}}{\pgfqpoint{0.000000in}{0.000000in}}{%
\pgfpathmoveto{\pgfqpoint{0.000000in}{0.000000in}}%
\pgfpathlineto{\pgfqpoint{0.000000in}{-0.048611in}}%
\pgfusepath{stroke,fill}%
}%
\begin{pgfscope}%
\pgfsys@transformshift{3.041104in}{0.548769in}%
\pgfsys@useobject{currentmarker}{}%
\end{pgfscope}%
\end{pgfscope}%
\begin{pgfscope}%
\definecolor{textcolor}{rgb}{0.000000,0.000000,0.000000}%
\pgfsetstrokecolor{textcolor}%
\pgfsetfillcolor{textcolor}%
\pgftext[x=3.041104in,y=0.451547in,,top]{\color{textcolor}\sffamily\fontsize{10.000000}{12.000000}\selectfont \(\displaystyle {2}\)}%
\end{pgfscope}%
\begin{pgfscope}%
\pgfsetbuttcap%
\pgfsetroundjoin%
\definecolor{currentfill}{rgb}{0.000000,0.000000,0.000000}%
\pgfsetfillcolor{currentfill}%
\pgfsetlinewidth{0.803000pt}%
\definecolor{currentstroke}{rgb}{0.000000,0.000000,0.000000}%
\pgfsetstrokecolor{currentstroke}%
\pgfsetdash{}{0pt}%
\pgfsys@defobject{currentmarker}{\pgfqpoint{0.000000in}{-0.048611in}}{\pgfqpoint{0.000000in}{0.000000in}}{%
\pgfpathmoveto{\pgfqpoint{0.000000in}{0.000000in}}%
\pgfpathlineto{\pgfqpoint{0.000000in}{-0.048611in}}%
\pgfusepath{stroke,fill}%
}%
\begin{pgfscope}%
\pgfsys@transformshift{3.776096in}{0.548769in}%
\pgfsys@useobject{currentmarker}{}%
\end{pgfscope}%
\end{pgfscope}%
\begin{pgfscope}%
\definecolor{textcolor}{rgb}{0.000000,0.000000,0.000000}%
\pgfsetstrokecolor{textcolor}%
\pgfsetfillcolor{textcolor}%
\pgftext[x=3.776096in,y=0.451547in,,top]{\color{textcolor}\sffamily\fontsize{10.000000}{12.000000}\selectfont \(\displaystyle {3}\)}%
\end{pgfscope}%
\begin{pgfscope}%
\pgfsetbuttcap%
\pgfsetroundjoin%
\definecolor{currentfill}{rgb}{0.000000,0.000000,0.000000}%
\pgfsetfillcolor{currentfill}%
\pgfsetlinewidth{0.803000pt}%
\definecolor{currentstroke}{rgb}{0.000000,0.000000,0.000000}%
\pgfsetstrokecolor{currentstroke}%
\pgfsetdash{}{0pt}%
\pgfsys@defobject{currentmarker}{\pgfqpoint{0.000000in}{-0.048611in}}{\pgfqpoint{0.000000in}{0.000000in}}{%
\pgfpathmoveto{\pgfqpoint{0.000000in}{0.000000in}}%
\pgfpathlineto{\pgfqpoint{0.000000in}{-0.048611in}}%
\pgfusepath{stroke,fill}%
}%
\begin{pgfscope}%
\pgfsys@transformshift{4.511089in}{0.548769in}%
\pgfsys@useobject{currentmarker}{}%
\end{pgfscope}%
\end{pgfscope}%
\begin{pgfscope}%
\definecolor{textcolor}{rgb}{0.000000,0.000000,0.000000}%
\pgfsetstrokecolor{textcolor}%
\pgfsetfillcolor{textcolor}%
\pgftext[x=4.511089in,y=0.451547in,,top]{\color{textcolor}\sffamily\fontsize{10.000000}{12.000000}\selectfont \(\displaystyle {4}\)}%
\end{pgfscope}%
\begin{pgfscope}%
\pgfsetbuttcap%
\pgfsetroundjoin%
\definecolor{currentfill}{rgb}{0.000000,0.000000,0.000000}%
\pgfsetfillcolor{currentfill}%
\pgfsetlinewidth{0.803000pt}%
\definecolor{currentstroke}{rgb}{0.000000,0.000000,0.000000}%
\pgfsetstrokecolor{currentstroke}%
\pgfsetdash{}{0pt}%
\pgfsys@defobject{currentmarker}{\pgfqpoint{0.000000in}{-0.048611in}}{\pgfqpoint{0.000000in}{0.000000in}}{%
\pgfpathmoveto{\pgfqpoint{0.000000in}{0.000000in}}%
\pgfpathlineto{\pgfqpoint{0.000000in}{-0.048611in}}%
\pgfusepath{stroke,fill}%
}%
\begin{pgfscope}%
\pgfsys@transformshift{5.246081in}{0.548769in}%
\pgfsys@useobject{currentmarker}{}%
\end{pgfscope}%
\end{pgfscope}%
\begin{pgfscope}%
\definecolor{textcolor}{rgb}{0.000000,0.000000,0.000000}%
\pgfsetstrokecolor{textcolor}%
\pgfsetfillcolor{textcolor}%
\pgftext[x=5.246081in,y=0.451547in,,top]{\color{textcolor}\sffamily\fontsize{10.000000}{12.000000}\selectfont \(\displaystyle {5}\)}%
\end{pgfscope}%
\begin{pgfscope}%
\definecolor{textcolor}{rgb}{0.000000,0.000000,0.000000}%
\pgfsetstrokecolor{textcolor}%
\pgfsetfillcolor{textcolor}%
\pgftext[x=3.249352in,y=0.272658in,,top]{\color{textcolor}\sffamily\fontsize{10.000000}{12.000000}\selectfont Ratio of Sources to Sinks}%
\end{pgfscope}%
\begin{pgfscope}%
\pgfsetbuttcap%
\pgfsetroundjoin%
\definecolor{currentfill}{rgb}{0.000000,0.000000,0.000000}%
\pgfsetfillcolor{currentfill}%
\pgfsetlinewidth{0.803000pt}%
\definecolor{currentstroke}{rgb}{0.000000,0.000000,0.000000}%
\pgfsetstrokecolor{currentstroke}%
\pgfsetdash{}{0pt}%
\pgfsys@defobject{currentmarker}{\pgfqpoint{-0.048611in}{0.000000in}}{\pgfqpoint{0.000000in}{0.000000in}}{%
\pgfpathmoveto{\pgfqpoint{0.000000in}{0.000000in}}%
\pgfpathlineto{\pgfqpoint{-0.048611in}{0.000000in}}%
\pgfusepath{stroke,fill}%
}%
\begin{pgfscope}%
\pgfsys@transformshift{0.648703in}{0.689796in}%
\pgfsys@useobject{currentmarker}{}%
\end{pgfscope}%
\end{pgfscope}%
\begin{pgfscope}%
\definecolor{textcolor}{rgb}{0.000000,0.000000,0.000000}%
\pgfsetstrokecolor{textcolor}%
\pgfsetfillcolor{textcolor}%
\pgftext[x=0.482036in, y=0.641601in, left, base]{\color{textcolor}\sffamily\fontsize{10.000000}{12.000000}\selectfont \(\displaystyle {0}\)}%
\end{pgfscope}%
\begin{pgfscope}%
\pgfsetbuttcap%
\pgfsetroundjoin%
\definecolor{currentfill}{rgb}{0.000000,0.000000,0.000000}%
\pgfsetfillcolor{currentfill}%
\pgfsetlinewidth{0.803000pt}%
\definecolor{currentstroke}{rgb}{0.000000,0.000000,0.000000}%
\pgfsetstrokecolor{currentstroke}%
\pgfsetdash{}{0pt}%
\pgfsys@defobject{currentmarker}{\pgfqpoint{-0.048611in}{0.000000in}}{\pgfqpoint{0.000000in}{0.000000in}}{%
\pgfpathmoveto{\pgfqpoint{0.000000in}{0.000000in}}%
\pgfpathlineto{\pgfqpoint{-0.048611in}{0.000000in}}%
\pgfusepath{stroke,fill}%
}%
\begin{pgfscope}%
\pgfsys@transformshift{0.648703in}{1.112665in}%
\pgfsys@useobject{currentmarker}{}%
\end{pgfscope}%
\end{pgfscope}%
\begin{pgfscope}%
\definecolor{textcolor}{rgb}{0.000000,0.000000,0.000000}%
\pgfsetstrokecolor{textcolor}%
\pgfsetfillcolor{textcolor}%
\pgftext[x=0.343147in, y=1.064470in, left, base]{\color{textcolor}\sffamily\fontsize{10.000000}{12.000000}\selectfont \(\displaystyle {100}\)}%
\end{pgfscope}%
\begin{pgfscope}%
\pgfsetbuttcap%
\pgfsetroundjoin%
\definecolor{currentfill}{rgb}{0.000000,0.000000,0.000000}%
\pgfsetfillcolor{currentfill}%
\pgfsetlinewidth{0.803000pt}%
\definecolor{currentstroke}{rgb}{0.000000,0.000000,0.000000}%
\pgfsetstrokecolor{currentstroke}%
\pgfsetdash{}{0pt}%
\pgfsys@defobject{currentmarker}{\pgfqpoint{-0.048611in}{0.000000in}}{\pgfqpoint{0.000000in}{0.000000in}}{%
\pgfpathmoveto{\pgfqpoint{0.000000in}{0.000000in}}%
\pgfpathlineto{\pgfqpoint{-0.048611in}{0.000000in}}%
\pgfusepath{stroke,fill}%
}%
\begin{pgfscope}%
\pgfsys@transformshift{0.648703in}{1.535534in}%
\pgfsys@useobject{currentmarker}{}%
\end{pgfscope}%
\end{pgfscope}%
\begin{pgfscope}%
\definecolor{textcolor}{rgb}{0.000000,0.000000,0.000000}%
\pgfsetstrokecolor{textcolor}%
\pgfsetfillcolor{textcolor}%
\pgftext[x=0.343147in, y=1.487339in, left, base]{\color{textcolor}\sffamily\fontsize{10.000000}{12.000000}\selectfont \(\displaystyle {200}\)}%
\end{pgfscope}%
\begin{pgfscope}%
\pgfsetbuttcap%
\pgfsetroundjoin%
\definecolor{currentfill}{rgb}{0.000000,0.000000,0.000000}%
\pgfsetfillcolor{currentfill}%
\pgfsetlinewidth{0.803000pt}%
\definecolor{currentstroke}{rgb}{0.000000,0.000000,0.000000}%
\pgfsetstrokecolor{currentstroke}%
\pgfsetdash{}{0pt}%
\pgfsys@defobject{currentmarker}{\pgfqpoint{-0.048611in}{0.000000in}}{\pgfqpoint{0.000000in}{0.000000in}}{%
\pgfpathmoveto{\pgfqpoint{0.000000in}{0.000000in}}%
\pgfpathlineto{\pgfqpoint{-0.048611in}{0.000000in}}%
\pgfusepath{stroke,fill}%
}%
\begin{pgfscope}%
\pgfsys@transformshift{0.648703in}{1.958403in}%
\pgfsys@useobject{currentmarker}{}%
\end{pgfscope}%
\end{pgfscope}%
\begin{pgfscope}%
\definecolor{textcolor}{rgb}{0.000000,0.000000,0.000000}%
\pgfsetstrokecolor{textcolor}%
\pgfsetfillcolor{textcolor}%
\pgftext[x=0.343147in, y=1.910208in, left, base]{\color{textcolor}\sffamily\fontsize{10.000000}{12.000000}\selectfont \(\displaystyle {300}\)}%
\end{pgfscope}%
\begin{pgfscope}%
\pgfsetbuttcap%
\pgfsetroundjoin%
\definecolor{currentfill}{rgb}{0.000000,0.000000,0.000000}%
\pgfsetfillcolor{currentfill}%
\pgfsetlinewidth{0.803000pt}%
\definecolor{currentstroke}{rgb}{0.000000,0.000000,0.000000}%
\pgfsetstrokecolor{currentstroke}%
\pgfsetdash{}{0pt}%
\pgfsys@defobject{currentmarker}{\pgfqpoint{-0.048611in}{0.000000in}}{\pgfqpoint{0.000000in}{0.000000in}}{%
\pgfpathmoveto{\pgfqpoint{0.000000in}{0.000000in}}%
\pgfpathlineto{\pgfqpoint{-0.048611in}{0.000000in}}%
\pgfusepath{stroke,fill}%
}%
\begin{pgfscope}%
\pgfsys@transformshift{0.648703in}{2.381272in}%
\pgfsys@useobject{currentmarker}{}%
\end{pgfscope}%
\end{pgfscope}%
\begin{pgfscope}%
\definecolor{textcolor}{rgb}{0.000000,0.000000,0.000000}%
\pgfsetstrokecolor{textcolor}%
\pgfsetfillcolor{textcolor}%
\pgftext[x=0.343147in, y=2.333077in, left, base]{\color{textcolor}\sffamily\fontsize{10.000000}{12.000000}\selectfont \(\displaystyle {400}\)}%
\end{pgfscope}%
\begin{pgfscope}%
\pgfsetbuttcap%
\pgfsetroundjoin%
\definecolor{currentfill}{rgb}{0.000000,0.000000,0.000000}%
\pgfsetfillcolor{currentfill}%
\pgfsetlinewidth{0.803000pt}%
\definecolor{currentstroke}{rgb}{0.000000,0.000000,0.000000}%
\pgfsetstrokecolor{currentstroke}%
\pgfsetdash{}{0pt}%
\pgfsys@defobject{currentmarker}{\pgfqpoint{-0.048611in}{0.000000in}}{\pgfqpoint{0.000000in}{0.000000in}}{%
\pgfpathmoveto{\pgfqpoint{0.000000in}{0.000000in}}%
\pgfpathlineto{\pgfqpoint{-0.048611in}{0.000000in}}%
\pgfusepath{stroke,fill}%
}%
\begin{pgfscope}%
\pgfsys@transformshift{0.648703in}{2.804141in}%
\pgfsys@useobject{currentmarker}{}%
\end{pgfscope}%
\end{pgfscope}%
\begin{pgfscope}%
\definecolor{textcolor}{rgb}{0.000000,0.000000,0.000000}%
\pgfsetstrokecolor{textcolor}%
\pgfsetfillcolor{textcolor}%
\pgftext[x=0.343147in, y=2.755946in, left, base]{\color{textcolor}\sffamily\fontsize{10.000000}{12.000000}\selectfont \(\displaystyle {500}\)}%
\end{pgfscope}%
\begin{pgfscope}%
\pgfsetbuttcap%
\pgfsetroundjoin%
\definecolor{currentfill}{rgb}{0.000000,0.000000,0.000000}%
\pgfsetfillcolor{currentfill}%
\pgfsetlinewidth{0.803000pt}%
\definecolor{currentstroke}{rgb}{0.000000,0.000000,0.000000}%
\pgfsetstrokecolor{currentstroke}%
\pgfsetdash{}{0pt}%
\pgfsys@defobject{currentmarker}{\pgfqpoint{-0.048611in}{0.000000in}}{\pgfqpoint{0.000000in}{0.000000in}}{%
\pgfpathmoveto{\pgfqpoint{0.000000in}{0.000000in}}%
\pgfpathlineto{\pgfqpoint{-0.048611in}{0.000000in}}%
\pgfusepath{stroke,fill}%
}%
\begin{pgfscope}%
\pgfsys@transformshift{0.648703in}{3.227010in}%
\pgfsys@useobject{currentmarker}{}%
\end{pgfscope}%
\end{pgfscope}%
\begin{pgfscope}%
\definecolor{textcolor}{rgb}{0.000000,0.000000,0.000000}%
\pgfsetstrokecolor{textcolor}%
\pgfsetfillcolor{textcolor}%
\pgftext[x=0.343147in, y=3.178815in, left, base]{\color{textcolor}\sffamily\fontsize{10.000000}{12.000000}\selectfont \(\displaystyle {600}\)}%
\end{pgfscope}%
\begin{pgfscope}%
\pgfsetbuttcap%
\pgfsetroundjoin%
\definecolor{currentfill}{rgb}{0.000000,0.000000,0.000000}%
\pgfsetfillcolor{currentfill}%
\pgfsetlinewidth{0.803000pt}%
\definecolor{currentstroke}{rgb}{0.000000,0.000000,0.000000}%
\pgfsetstrokecolor{currentstroke}%
\pgfsetdash{}{0pt}%
\pgfsys@defobject{currentmarker}{\pgfqpoint{-0.048611in}{0.000000in}}{\pgfqpoint{0.000000in}{0.000000in}}{%
\pgfpathmoveto{\pgfqpoint{0.000000in}{0.000000in}}%
\pgfpathlineto{\pgfqpoint{-0.048611in}{0.000000in}}%
\pgfusepath{stroke,fill}%
}%
\begin{pgfscope}%
\pgfsys@transformshift{0.648703in}{3.649879in}%
\pgfsys@useobject{currentmarker}{}%
\end{pgfscope}%
\end{pgfscope}%
\begin{pgfscope}%
\definecolor{textcolor}{rgb}{0.000000,0.000000,0.000000}%
\pgfsetstrokecolor{textcolor}%
\pgfsetfillcolor{textcolor}%
\pgftext[x=0.343147in, y=3.601684in, left, base]{\color{textcolor}\sffamily\fontsize{10.000000}{12.000000}\selectfont \(\displaystyle {700}\)}%
\end{pgfscope}%
\begin{pgfscope}%
\definecolor{textcolor}{rgb}{0.000000,0.000000,0.000000}%
\pgfsetstrokecolor{textcolor}%
\pgfsetfillcolor{textcolor}%
\pgftext[x=0.287592in,y=2.100064in,,bottom,rotate=90.000000]{\color{textcolor}\sffamily\fontsize{10.000000}{12.000000}\selectfont Data Flow Time (s)}%
\end{pgfscope}%
\begin{pgfscope}%
\pgfsetrectcap%
\pgfsetmiterjoin%
\pgfsetlinewidth{0.803000pt}%
\definecolor{currentstroke}{rgb}{0.000000,0.000000,0.000000}%
\pgfsetstrokecolor{currentstroke}%
\pgfsetdash{}{0pt}%
\pgfpathmoveto{\pgfqpoint{0.648703in}{0.548769in}}%
\pgfpathlineto{\pgfqpoint{0.648703in}{3.651359in}}%
\pgfusepath{stroke}%
\end{pgfscope}%
\begin{pgfscope}%
\pgfsetrectcap%
\pgfsetmiterjoin%
\pgfsetlinewidth{0.803000pt}%
\definecolor{currentstroke}{rgb}{0.000000,0.000000,0.000000}%
\pgfsetstrokecolor{currentstroke}%
\pgfsetdash{}{0pt}%
\pgfpathmoveto{\pgfqpoint{5.850000in}{0.548769in}}%
\pgfpathlineto{\pgfqpoint{5.850000in}{3.651359in}}%
\pgfusepath{stroke}%
\end{pgfscope}%
\begin{pgfscope}%
\pgfsetrectcap%
\pgfsetmiterjoin%
\pgfsetlinewidth{0.803000pt}%
\definecolor{currentstroke}{rgb}{0.000000,0.000000,0.000000}%
\pgfsetstrokecolor{currentstroke}%
\pgfsetdash{}{0pt}%
\pgfpathmoveto{\pgfqpoint{0.648703in}{0.548769in}}%
\pgfpathlineto{\pgfqpoint{5.850000in}{0.548769in}}%
\pgfusepath{stroke}%
\end{pgfscope}%
\begin{pgfscope}%
\pgfsetrectcap%
\pgfsetmiterjoin%
\pgfsetlinewidth{0.803000pt}%
\definecolor{currentstroke}{rgb}{0.000000,0.000000,0.000000}%
\pgfsetstrokecolor{currentstroke}%
\pgfsetdash{}{0pt}%
\pgfpathmoveto{\pgfqpoint{0.648703in}{3.651359in}}%
\pgfpathlineto{\pgfqpoint{5.850000in}{3.651359in}}%
\pgfusepath{stroke}%
\end{pgfscope}%
\begin{pgfscope}%
\definecolor{textcolor}{rgb}{0.000000,0.000000,0.000000}%
\pgfsetstrokecolor{textcolor}%
\pgfsetfillcolor{textcolor}%
\pgftext[x=3.249352in,y=3.734692in,,base]{\color{textcolor}\sffamily\fontsize{12.000000}{14.400000}\selectfont Backward}%
\end{pgfscope}%
\begin{pgfscope}%
\pgfsetbuttcap%
\pgfsetmiterjoin%
\definecolor{currentfill}{rgb}{1.000000,1.000000,1.000000}%
\pgfsetfillcolor{currentfill}%
\pgfsetfillopacity{0.800000}%
\pgfsetlinewidth{1.003750pt}%
\definecolor{currentstroke}{rgb}{0.800000,0.800000,0.800000}%
\pgfsetstrokecolor{currentstroke}%
\pgfsetstrokeopacity{0.800000}%
\pgfsetdash{}{0pt}%
\pgfpathmoveto{\pgfqpoint{4.300417in}{1.788050in}}%
\pgfpathlineto{\pgfqpoint{5.752778in}{1.788050in}}%
\pgfpathquadraticcurveto{\pgfqpoint{5.780556in}{1.788050in}}{\pgfqpoint{5.780556in}{1.815828in}}%
\pgfpathlineto{\pgfqpoint{5.780556in}{2.384300in}}%
\pgfpathquadraticcurveto{\pgfqpoint{5.780556in}{2.412078in}}{\pgfqpoint{5.752778in}{2.412078in}}%
\pgfpathlineto{\pgfqpoint{4.300417in}{2.412078in}}%
\pgfpathquadraticcurveto{\pgfqpoint{4.272639in}{2.412078in}}{\pgfqpoint{4.272639in}{2.384300in}}%
\pgfpathlineto{\pgfqpoint{4.272639in}{1.815828in}}%
\pgfpathquadraticcurveto{\pgfqpoint{4.272639in}{1.788050in}}{\pgfqpoint{4.300417in}{1.788050in}}%
\pgfpathclose%
\pgfusepath{stroke,fill}%
\end{pgfscope}%
\begin{pgfscope}%
\pgfsetbuttcap%
\pgfsetroundjoin%
\definecolor{currentfill}{rgb}{0.121569,0.466667,0.705882}%
\pgfsetfillcolor{currentfill}%
\pgfsetlinewidth{1.003750pt}%
\definecolor{currentstroke}{rgb}{0.121569,0.466667,0.705882}%
\pgfsetstrokecolor{currentstroke}%
\pgfsetdash{}{0pt}%
\pgfsys@defobject{currentmarker}{\pgfqpoint{-0.034722in}{-0.034722in}}{\pgfqpoint{0.034722in}{0.034722in}}{%
\pgfpathmoveto{\pgfqpoint{0.000000in}{-0.034722in}}%
\pgfpathcurveto{\pgfqpoint{0.009208in}{-0.034722in}}{\pgfqpoint{0.018041in}{-0.031064in}}{\pgfqpoint{0.024552in}{-0.024552in}}%
\pgfpathcurveto{\pgfqpoint{0.031064in}{-0.018041in}}{\pgfqpoint{0.034722in}{-0.009208in}}{\pgfqpoint{0.034722in}{0.000000in}}%
\pgfpathcurveto{\pgfqpoint{0.034722in}{0.009208in}}{\pgfqpoint{0.031064in}{0.018041in}}{\pgfqpoint{0.024552in}{0.024552in}}%
\pgfpathcurveto{\pgfqpoint{0.018041in}{0.031064in}}{\pgfqpoint{0.009208in}{0.034722in}}{\pgfqpoint{0.000000in}{0.034722in}}%
\pgfpathcurveto{\pgfqpoint{-0.009208in}{0.034722in}}{\pgfqpoint{-0.018041in}{0.031064in}}{\pgfqpoint{-0.024552in}{0.024552in}}%
\pgfpathcurveto{\pgfqpoint{-0.031064in}{0.018041in}}{\pgfqpoint{-0.034722in}{0.009208in}}{\pgfqpoint{-0.034722in}{0.000000in}}%
\pgfpathcurveto{\pgfqpoint{-0.034722in}{-0.009208in}}{\pgfqpoint{-0.031064in}{-0.018041in}}{\pgfqpoint{-0.024552in}{-0.024552in}}%
\pgfpathcurveto{\pgfqpoint{-0.018041in}{-0.031064in}}{\pgfqpoint{-0.009208in}{-0.034722in}}{\pgfqpoint{0.000000in}{-0.034722in}}%
\pgfpathclose%
\pgfusepath{stroke,fill}%
}%
\begin{pgfscope}%
\pgfsys@transformshift{4.467083in}{2.307911in}%
\pgfsys@useobject{currentmarker}{}%
\end{pgfscope}%
\end{pgfscope}%
\begin{pgfscope}%
\definecolor{textcolor}{rgb}{0.000000,0.000000,0.000000}%
\pgfsetstrokecolor{textcolor}%
\pgfsetfillcolor{textcolor}%
\pgftext[x=4.717083in,y=2.259300in,left,base]{\color{textcolor}\sffamily\fontsize{10.000000}{12.000000}\selectfont No Timeout}%
\end{pgfscope}%
\begin{pgfscope}%
\pgfsetbuttcap%
\pgfsetroundjoin%
\definecolor{currentfill}{rgb}{1.000000,0.498039,0.054902}%
\pgfsetfillcolor{currentfill}%
\pgfsetlinewidth{1.003750pt}%
\definecolor{currentstroke}{rgb}{1.000000,0.498039,0.054902}%
\pgfsetstrokecolor{currentstroke}%
\pgfsetdash{}{0pt}%
\pgfsys@defobject{currentmarker}{\pgfqpoint{-0.034722in}{-0.034722in}}{\pgfqpoint{0.034722in}{0.034722in}}{%
\pgfpathmoveto{\pgfqpoint{0.000000in}{-0.034722in}}%
\pgfpathcurveto{\pgfqpoint{0.009208in}{-0.034722in}}{\pgfqpoint{0.018041in}{-0.031064in}}{\pgfqpoint{0.024552in}{-0.024552in}}%
\pgfpathcurveto{\pgfqpoint{0.031064in}{-0.018041in}}{\pgfqpoint{0.034722in}{-0.009208in}}{\pgfqpoint{0.034722in}{0.000000in}}%
\pgfpathcurveto{\pgfqpoint{0.034722in}{0.009208in}}{\pgfqpoint{0.031064in}{0.018041in}}{\pgfqpoint{0.024552in}{0.024552in}}%
\pgfpathcurveto{\pgfqpoint{0.018041in}{0.031064in}}{\pgfqpoint{0.009208in}{0.034722in}}{\pgfqpoint{0.000000in}{0.034722in}}%
\pgfpathcurveto{\pgfqpoint{-0.009208in}{0.034722in}}{\pgfqpoint{-0.018041in}{0.031064in}}{\pgfqpoint{-0.024552in}{0.024552in}}%
\pgfpathcurveto{\pgfqpoint{-0.031064in}{0.018041in}}{\pgfqpoint{-0.034722in}{0.009208in}}{\pgfqpoint{-0.034722in}{0.000000in}}%
\pgfpathcurveto{\pgfqpoint{-0.034722in}{-0.009208in}}{\pgfqpoint{-0.031064in}{-0.018041in}}{\pgfqpoint{-0.024552in}{-0.024552in}}%
\pgfpathcurveto{\pgfqpoint{-0.018041in}{-0.031064in}}{\pgfqpoint{-0.009208in}{-0.034722in}}{\pgfqpoint{0.000000in}{-0.034722in}}%
\pgfpathclose%
\pgfusepath{stroke,fill}%
}%
\begin{pgfscope}%
\pgfsys@transformshift{4.467083in}{2.114300in}%
\pgfsys@useobject{currentmarker}{}%
\end{pgfscope}%
\end{pgfscope}%
\begin{pgfscope}%
\definecolor{textcolor}{rgb}{0.000000,0.000000,0.000000}%
\pgfsetstrokecolor{textcolor}%
\pgfsetfillcolor{textcolor}%
\pgftext[x=4.717083in,y=2.065689in,left,base]{\color{textcolor}\sffamily\fontsize{10.000000}{12.000000}\selectfont Time Timeout}%
\end{pgfscope}%
\begin{pgfscope}%
\pgfsetbuttcap%
\pgfsetroundjoin%
\definecolor{currentfill}{rgb}{0.839216,0.152941,0.156863}%
\pgfsetfillcolor{currentfill}%
\pgfsetlinewidth{1.003750pt}%
\definecolor{currentstroke}{rgb}{0.839216,0.152941,0.156863}%
\pgfsetstrokecolor{currentstroke}%
\pgfsetdash{}{0pt}%
\pgfsys@defobject{currentmarker}{\pgfqpoint{-0.034722in}{-0.034722in}}{\pgfqpoint{0.034722in}{0.034722in}}{%
\pgfpathmoveto{\pgfqpoint{0.000000in}{-0.034722in}}%
\pgfpathcurveto{\pgfqpoint{0.009208in}{-0.034722in}}{\pgfqpoint{0.018041in}{-0.031064in}}{\pgfqpoint{0.024552in}{-0.024552in}}%
\pgfpathcurveto{\pgfqpoint{0.031064in}{-0.018041in}}{\pgfqpoint{0.034722in}{-0.009208in}}{\pgfqpoint{0.034722in}{0.000000in}}%
\pgfpathcurveto{\pgfqpoint{0.034722in}{0.009208in}}{\pgfqpoint{0.031064in}{0.018041in}}{\pgfqpoint{0.024552in}{0.024552in}}%
\pgfpathcurveto{\pgfqpoint{0.018041in}{0.031064in}}{\pgfqpoint{0.009208in}{0.034722in}}{\pgfqpoint{0.000000in}{0.034722in}}%
\pgfpathcurveto{\pgfqpoint{-0.009208in}{0.034722in}}{\pgfqpoint{-0.018041in}{0.031064in}}{\pgfqpoint{-0.024552in}{0.024552in}}%
\pgfpathcurveto{\pgfqpoint{-0.031064in}{0.018041in}}{\pgfqpoint{-0.034722in}{0.009208in}}{\pgfqpoint{-0.034722in}{0.000000in}}%
\pgfpathcurveto{\pgfqpoint{-0.034722in}{-0.009208in}}{\pgfqpoint{-0.031064in}{-0.018041in}}{\pgfqpoint{-0.024552in}{-0.024552in}}%
\pgfpathcurveto{\pgfqpoint{-0.018041in}{-0.031064in}}{\pgfqpoint{-0.009208in}{-0.034722in}}{\pgfqpoint{0.000000in}{-0.034722in}}%
\pgfpathclose%
\pgfusepath{stroke,fill}%
}%
\begin{pgfscope}%
\pgfsys@transformshift{4.467083in}{1.920689in}%
\pgfsys@useobject{currentmarker}{}%
\end{pgfscope}%
\end{pgfscope}%
\begin{pgfscope}%
\definecolor{textcolor}{rgb}{0.000000,0.000000,0.000000}%
\pgfsetstrokecolor{textcolor}%
\pgfsetfillcolor{textcolor}%
\pgftext[x=4.717083in,y=1.872078in,left,base]{\color{textcolor}\sffamily\fontsize{10.000000}{12.000000}\selectfont Memory Timeout}%
\end{pgfscope}%
\end{pgfpicture}%
\makeatother%
\endgroup%

                }
            \end{subfigure}
            \caption{Ratio}
            \label{f:dfratio}
        \end{subfigure}
        \caption{Data Flow Time in Comparison to Sources, Sinks and the Ratio of Those}
        \label{f:dftoss}
    \end{figure}

    Even though Arzt's evaluation also showed no correlation between the code size \cite{Arzt2017PhD}, we do for completeness also compare the runtime to the number of statements, methods and classes.
    Note that these refer to the Jimple intermediate representation and not Java.
    \autoref{f:dftocodesize} includes all graphs with the existing implementation being on the left side and our implementation on the right side.
    The notation are the same as before, with the x-axis swapped out.
    The number of statements is uniformly distributed with some outliers.
    If we consider our data as two linear data sets with a structural break between the two groups, the linear regressions have a slope of close to 0.
    Resulting, the number of statements, methods and classes do not have an impact on the runtime.

    \begin{figure}[tbp]
        \centering
        \begin{subfigure}[b]{\textwidth}
            \centering
            \begin{subfigure}[]{0.45\textwidth}
                \centering
                \resizebox{\columnwidth}{!}{
                    %% Creator: Matplotlib, PGF backend
%%
%% To include the figure in your LaTeX document, write
%%   \input{<filename>.pgf}
%%
%% Make sure the required packages are loaded in your preamble
%%   \usepackage{pgf}
%%
%% and, on pdftex
%%   \usepackage[utf8]{inputenc}\DeclareUnicodeCharacter{2212}{-}
%%
%% or, on luatex and xetex
%%   \usepackage{unicode-math}
%%
%% Figures using additional raster images can only be included by \input if
%% they are in the same directory as the main LaTeX file. For loading figures
%% from other directories you can use the `import` package
%%   \usepackage{import}
%%
%% and then include the figures with
%%   \import{<path to file>}{<filename>.pgf}
%%
%% Matplotlib used the following preamble
%%   \usepackage{amsmath}
%%   \usepackage{fontspec}
%%
\begingroup%
\makeatletter%
\begin{pgfpicture}%
\pgfpathrectangle{\pgfpointorigin}{\pgfqpoint{6.000000in}{4.000000in}}%
\pgfusepath{use as bounding box, clip}%
\begin{pgfscope}%
\pgfsetbuttcap%
\pgfsetmiterjoin%
\definecolor{currentfill}{rgb}{1.000000,1.000000,1.000000}%
\pgfsetfillcolor{currentfill}%
\pgfsetlinewidth{0.000000pt}%
\definecolor{currentstroke}{rgb}{1.000000,1.000000,1.000000}%
\pgfsetstrokecolor{currentstroke}%
\pgfsetdash{}{0pt}%
\pgfpathmoveto{\pgfqpoint{0.000000in}{0.000000in}}%
\pgfpathlineto{\pgfqpoint{6.000000in}{0.000000in}}%
\pgfpathlineto{\pgfqpoint{6.000000in}{4.000000in}}%
\pgfpathlineto{\pgfqpoint{0.000000in}{4.000000in}}%
\pgfpathclose%
\pgfusepath{fill}%
\end{pgfscope}%
\begin{pgfscope}%
\pgfsetbuttcap%
\pgfsetmiterjoin%
\definecolor{currentfill}{rgb}{1.000000,1.000000,1.000000}%
\pgfsetfillcolor{currentfill}%
\pgfsetlinewidth{0.000000pt}%
\definecolor{currentstroke}{rgb}{0.000000,0.000000,0.000000}%
\pgfsetstrokecolor{currentstroke}%
\pgfsetstrokeopacity{0.000000}%
\pgfsetdash{}{0pt}%
\pgfpathmoveto{\pgfqpoint{0.648703in}{0.548769in}}%
\pgfpathlineto{\pgfqpoint{5.850000in}{0.548769in}}%
\pgfpathlineto{\pgfqpoint{5.850000in}{3.651359in}}%
\pgfpathlineto{\pgfqpoint{0.648703in}{3.651359in}}%
\pgfpathclose%
\pgfusepath{fill}%
\end{pgfscope}%
\begin{pgfscope}%
\pgfpathrectangle{\pgfqpoint{0.648703in}{0.548769in}}{\pgfqpoint{5.201297in}{3.102590in}}%
\pgfusepath{clip}%
\pgfsetbuttcap%
\pgfsetroundjoin%
\definecolor{currentfill}{rgb}{0.121569,0.466667,0.705882}%
\pgfsetfillcolor{currentfill}%
\pgfsetlinewidth{1.003750pt}%
\definecolor{currentstroke}{rgb}{0.121569,0.466667,0.705882}%
\pgfsetstrokecolor{currentstroke}%
\pgfsetdash{}{0pt}%
\pgfpathmoveto{\pgfqpoint{1.113780in}{0.648129in}}%
\pgfpathcurveto{\pgfqpoint{1.124830in}{0.648129in}}{\pgfqpoint{1.135429in}{0.652519in}}{\pgfqpoint{1.143243in}{0.660333in}}%
\pgfpathcurveto{\pgfqpoint{1.151056in}{0.668146in}}{\pgfqpoint{1.155447in}{0.678745in}}{\pgfqpoint{1.155447in}{0.689796in}}%
\pgfpathcurveto{\pgfqpoint{1.155447in}{0.700846in}}{\pgfqpoint{1.151056in}{0.711445in}}{\pgfqpoint{1.143243in}{0.719258in}}%
\pgfpathcurveto{\pgfqpoint{1.135429in}{0.727072in}}{\pgfqpoint{1.124830in}{0.731462in}}{\pgfqpoint{1.113780in}{0.731462in}}%
\pgfpathcurveto{\pgfqpoint{1.102730in}{0.731462in}}{\pgfqpoint{1.092131in}{0.727072in}}{\pgfqpoint{1.084317in}{0.719258in}}%
\pgfpathcurveto{\pgfqpoint{1.076504in}{0.711445in}}{\pgfqpoint{1.072113in}{0.700846in}}{\pgfqpoint{1.072113in}{0.689796in}}%
\pgfpathcurveto{\pgfqpoint{1.072113in}{0.678745in}}{\pgfqpoint{1.076504in}{0.668146in}}{\pgfqpoint{1.084317in}{0.660333in}}%
\pgfpathcurveto{\pgfqpoint{1.092131in}{0.652519in}}{\pgfqpoint{1.102730in}{0.648129in}}{\pgfqpoint{1.113780in}{0.648129in}}%
\pgfpathclose%
\pgfusepath{stroke,fill}%
\end{pgfscope}%
\begin{pgfscope}%
\pgfpathrectangle{\pgfqpoint{0.648703in}{0.548769in}}{\pgfqpoint{5.201297in}{3.102590in}}%
\pgfusepath{clip}%
\pgfsetbuttcap%
\pgfsetroundjoin%
\definecolor{currentfill}{rgb}{1.000000,0.498039,0.054902}%
\pgfsetfillcolor{currentfill}%
\pgfsetlinewidth{1.003750pt}%
\definecolor{currentstroke}{rgb}{1.000000,0.498039,0.054902}%
\pgfsetstrokecolor{currentstroke}%
\pgfsetdash{}{0pt}%
\pgfpathmoveto{\pgfqpoint{2.819926in}{3.124394in}}%
\pgfpathcurveto{\pgfqpoint{2.830976in}{3.124394in}}{\pgfqpoint{2.841575in}{3.128784in}}{\pgfqpoint{2.849389in}{3.136598in}}%
\pgfpathcurveto{\pgfqpoint{2.857202in}{3.144411in}}{\pgfqpoint{2.861593in}{3.155010in}}{\pgfqpoint{2.861593in}{3.166060in}}%
\pgfpathcurveto{\pgfqpoint{2.861593in}{3.177111in}}{\pgfqpoint{2.857202in}{3.187710in}}{\pgfqpoint{2.849389in}{3.195523in}}%
\pgfpathcurveto{\pgfqpoint{2.841575in}{3.203337in}}{\pgfqpoint{2.830976in}{3.207727in}}{\pgfqpoint{2.819926in}{3.207727in}}%
\pgfpathcurveto{\pgfqpoint{2.808876in}{3.207727in}}{\pgfqpoint{2.798277in}{3.203337in}}{\pgfqpoint{2.790463in}{3.195523in}}%
\pgfpathcurveto{\pgfqpoint{2.782649in}{3.187710in}}{\pgfqpoint{2.778259in}{3.177111in}}{\pgfqpoint{2.778259in}{3.166060in}}%
\pgfpathcurveto{\pgfqpoint{2.778259in}{3.155010in}}{\pgfqpoint{2.782649in}{3.144411in}}{\pgfqpoint{2.790463in}{3.136598in}}%
\pgfpathcurveto{\pgfqpoint{2.798277in}{3.128784in}}{\pgfqpoint{2.808876in}{3.124394in}}{\pgfqpoint{2.819926in}{3.124394in}}%
\pgfpathclose%
\pgfusepath{stroke,fill}%
\end{pgfscope}%
\begin{pgfscope}%
\pgfpathrectangle{\pgfqpoint{0.648703in}{0.548769in}}{\pgfqpoint{5.201297in}{3.102590in}}%
\pgfusepath{clip}%
\pgfsetbuttcap%
\pgfsetroundjoin%
\definecolor{currentfill}{rgb}{1.000000,0.498039,0.054902}%
\pgfsetfillcolor{currentfill}%
\pgfsetlinewidth{1.003750pt}%
\definecolor{currentstroke}{rgb}{1.000000,0.498039,0.054902}%
\pgfsetstrokecolor{currentstroke}%
\pgfsetdash{}{0pt}%
\pgfpathmoveto{\pgfqpoint{1.266468in}{3.140985in}}%
\pgfpathcurveto{\pgfqpoint{1.277518in}{3.140985in}}{\pgfqpoint{1.288117in}{3.145375in}}{\pgfqpoint{1.295931in}{3.153189in}}%
\pgfpathcurveto{\pgfqpoint{1.303744in}{3.161003in}}{\pgfqpoint{1.308135in}{3.171602in}}{\pgfqpoint{1.308135in}{3.182652in}}%
\pgfpathcurveto{\pgfqpoint{1.308135in}{3.193702in}}{\pgfqpoint{1.303744in}{3.204301in}}{\pgfqpoint{1.295931in}{3.212115in}}%
\pgfpathcurveto{\pgfqpoint{1.288117in}{3.219928in}}{\pgfqpoint{1.277518in}{3.224319in}}{\pgfqpoint{1.266468in}{3.224319in}}%
\pgfpathcurveto{\pgfqpoint{1.255418in}{3.224319in}}{\pgfqpoint{1.244819in}{3.219928in}}{\pgfqpoint{1.237005in}{3.212115in}}%
\pgfpathcurveto{\pgfqpoint{1.229192in}{3.204301in}}{\pgfqpoint{1.224801in}{3.193702in}}{\pgfqpoint{1.224801in}{3.182652in}}%
\pgfpathcurveto{\pgfqpoint{1.224801in}{3.171602in}}{\pgfqpoint{1.229192in}{3.161003in}}{\pgfqpoint{1.237005in}{3.153189in}}%
\pgfpathcurveto{\pgfqpoint{1.244819in}{3.145375in}}{\pgfqpoint{1.255418in}{3.140985in}}{\pgfqpoint{1.266468in}{3.140985in}}%
\pgfpathclose%
\pgfusepath{stroke,fill}%
\end{pgfscope}%
\begin{pgfscope}%
\pgfpathrectangle{\pgfqpoint{0.648703in}{0.548769in}}{\pgfqpoint{5.201297in}{3.102590in}}%
\pgfusepath{clip}%
\pgfsetbuttcap%
\pgfsetroundjoin%
\definecolor{currentfill}{rgb}{1.000000,0.498039,0.054902}%
\pgfsetfillcolor{currentfill}%
\pgfsetlinewidth{1.003750pt}%
\definecolor{currentstroke}{rgb}{1.000000,0.498039,0.054902}%
\pgfsetstrokecolor{currentstroke}%
\pgfsetdash{}{0pt}%
\pgfpathmoveto{\pgfqpoint{1.523022in}{3.132690in}}%
\pgfpathcurveto{\pgfqpoint{1.534072in}{3.132690in}}{\pgfqpoint{1.544671in}{3.137080in}}{\pgfqpoint{1.552485in}{3.144893in}}%
\pgfpathcurveto{\pgfqpoint{1.560298in}{3.152707in}}{\pgfqpoint{1.564689in}{3.163306in}}{\pgfqpoint{1.564689in}{3.174356in}}%
\pgfpathcurveto{\pgfqpoint{1.564689in}{3.185406in}}{\pgfqpoint{1.560298in}{3.196005in}}{\pgfqpoint{1.552485in}{3.203819in}}%
\pgfpathcurveto{\pgfqpoint{1.544671in}{3.211633in}}{\pgfqpoint{1.534072in}{3.216023in}}{\pgfqpoint{1.523022in}{3.216023in}}%
\pgfpathcurveto{\pgfqpoint{1.511972in}{3.216023in}}{\pgfqpoint{1.501373in}{3.211633in}}{\pgfqpoint{1.493559in}{3.203819in}}%
\pgfpathcurveto{\pgfqpoint{1.485746in}{3.196005in}}{\pgfqpoint{1.481355in}{3.185406in}}{\pgfqpoint{1.481355in}{3.174356in}}%
\pgfpathcurveto{\pgfqpoint{1.481355in}{3.163306in}}{\pgfqpoint{1.485746in}{3.152707in}}{\pgfqpoint{1.493559in}{3.144893in}}%
\pgfpathcurveto{\pgfqpoint{1.501373in}{3.137080in}}{\pgfqpoint{1.511972in}{3.132690in}}{\pgfqpoint{1.523022in}{3.132690in}}%
\pgfpathclose%
\pgfusepath{stroke,fill}%
\end{pgfscope}%
\begin{pgfscope}%
\pgfpathrectangle{\pgfqpoint{0.648703in}{0.548769in}}{\pgfqpoint{5.201297in}{3.102590in}}%
\pgfusepath{clip}%
\pgfsetbuttcap%
\pgfsetroundjoin%
\definecolor{currentfill}{rgb}{1.000000,0.498039,0.054902}%
\pgfsetfillcolor{currentfill}%
\pgfsetlinewidth{1.003750pt}%
\definecolor{currentstroke}{rgb}{1.000000,0.498039,0.054902}%
\pgfsetstrokecolor{currentstroke}%
\pgfsetdash{}{0pt}%
\pgfpathmoveto{\pgfqpoint{1.355123in}{3.136837in}}%
\pgfpathcurveto{\pgfqpoint{1.366173in}{3.136837in}}{\pgfqpoint{1.376773in}{3.141228in}}{\pgfqpoint{1.384586in}{3.149041in}}%
\pgfpathcurveto{\pgfqpoint{1.392400in}{3.156855in}}{\pgfqpoint{1.396790in}{3.167454in}}{\pgfqpoint{1.396790in}{3.178504in}}%
\pgfpathcurveto{\pgfqpoint{1.396790in}{3.189554in}}{\pgfqpoint{1.392400in}{3.200153in}}{\pgfqpoint{1.384586in}{3.207967in}}%
\pgfpathcurveto{\pgfqpoint{1.376773in}{3.215780in}}{\pgfqpoint{1.366173in}{3.220171in}}{\pgfqpoint{1.355123in}{3.220171in}}%
\pgfpathcurveto{\pgfqpoint{1.344073in}{3.220171in}}{\pgfqpoint{1.333474in}{3.215780in}}{\pgfqpoint{1.325661in}{3.207967in}}%
\pgfpathcurveto{\pgfqpoint{1.317847in}{3.200153in}}{\pgfqpoint{1.313457in}{3.189554in}}{\pgfqpoint{1.313457in}{3.178504in}}%
\pgfpathcurveto{\pgfqpoint{1.313457in}{3.167454in}}{\pgfqpoint{1.317847in}{3.156855in}}{\pgfqpoint{1.325661in}{3.149041in}}%
\pgfpathcurveto{\pgfqpoint{1.333474in}{3.141228in}}{\pgfqpoint{1.344073in}{3.136837in}}{\pgfqpoint{1.355123in}{3.136837in}}%
\pgfpathclose%
\pgfusepath{stroke,fill}%
\end{pgfscope}%
\begin{pgfscope}%
\pgfpathrectangle{\pgfqpoint{0.648703in}{0.548769in}}{\pgfqpoint{5.201297in}{3.102590in}}%
\pgfusepath{clip}%
\pgfsetbuttcap%
\pgfsetroundjoin%
\definecolor{currentfill}{rgb}{1.000000,0.498039,0.054902}%
\pgfsetfillcolor{currentfill}%
\pgfsetlinewidth{1.003750pt}%
\definecolor{currentstroke}{rgb}{1.000000,0.498039,0.054902}%
\pgfsetstrokecolor{currentstroke}%
\pgfsetdash{}{0pt}%
\pgfpathmoveto{\pgfqpoint{1.676996in}{3.132690in}}%
\pgfpathcurveto{\pgfqpoint{1.688046in}{3.132690in}}{\pgfqpoint{1.698645in}{3.137080in}}{\pgfqpoint{1.706459in}{3.144893in}}%
\pgfpathcurveto{\pgfqpoint{1.714273in}{3.152707in}}{\pgfqpoint{1.718663in}{3.163306in}}{\pgfqpoint{1.718663in}{3.174356in}}%
\pgfpathcurveto{\pgfqpoint{1.718663in}{3.185406in}}{\pgfqpoint{1.714273in}{3.196005in}}{\pgfqpoint{1.706459in}{3.203819in}}%
\pgfpathcurveto{\pgfqpoint{1.698645in}{3.211633in}}{\pgfqpoint{1.688046in}{3.216023in}}{\pgfqpoint{1.676996in}{3.216023in}}%
\pgfpathcurveto{\pgfqpoint{1.665946in}{3.216023in}}{\pgfqpoint{1.655347in}{3.211633in}}{\pgfqpoint{1.647534in}{3.203819in}}%
\pgfpathcurveto{\pgfqpoint{1.639720in}{3.196005in}}{\pgfqpoint{1.635330in}{3.185406in}}{\pgfqpoint{1.635330in}{3.174356in}}%
\pgfpathcurveto{\pgfqpoint{1.635330in}{3.163306in}}{\pgfqpoint{1.639720in}{3.152707in}}{\pgfqpoint{1.647534in}{3.144893in}}%
\pgfpathcurveto{\pgfqpoint{1.655347in}{3.137080in}}{\pgfqpoint{1.665946in}{3.132690in}}{\pgfqpoint{1.676996in}{3.132690in}}%
\pgfpathclose%
\pgfusepath{stroke,fill}%
\end{pgfscope}%
\begin{pgfscope}%
\pgfpathrectangle{\pgfqpoint{0.648703in}{0.548769in}}{\pgfqpoint{5.201297in}{3.102590in}}%
\pgfusepath{clip}%
\pgfsetbuttcap%
\pgfsetroundjoin%
\definecolor{currentfill}{rgb}{1.000000,0.498039,0.054902}%
\pgfsetfillcolor{currentfill}%
\pgfsetlinewidth{1.003750pt}%
\definecolor{currentstroke}{rgb}{1.000000,0.498039,0.054902}%
\pgfsetstrokecolor{currentstroke}%
\pgfsetdash{}{0pt}%
\pgfpathmoveto{\pgfqpoint{2.813506in}{3.128542in}}%
\pgfpathcurveto{\pgfqpoint{2.824556in}{3.128542in}}{\pgfqpoint{2.835155in}{3.132932in}}{\pgfqpoint{2.842969in}{3.140746in}}%
\pgfpathcurveto{\pgfqpoint{2.850782in}{3.148559in}}{\pgfqpoint{2.855173in}{3.159158in}}{\pgfqpoint{2.855173in}{3.170208in}}%
\pgfpathcurveto{\pgfqpoint{2.855173in}{3.181258in}}{\pgfqpoint{2.850782in}{3.191857in}}{\pgfqpoint{2.842969in}{3.199671in}}%
\pgfpathcurveto{\pgfqpoint{2.835155in}{3.207485in}}{\pgfqpoint{2.824556in}{3.211875in}}{\pgfqpoint{2.813506in}{3.211875in}}%
\pgfpathcurveto{\pgfqpoint{2.802456in}{3.211875in}}{\pgfqpoint{2.791857in}{3.207485in}}{\pgfqpoint{2.784043in}{3.199671in}}%
\pgfpathcurveto{\pgfqpoint{2.776230in}{3.191857in}}{\pgfqpoint{2.771839in}{3.181258in}}{\pgfqpoint{2.771839in}{3.170208in}}%
\pgfpathcurveto{\pgfqpoint{2.771839in}{3.159158in}}{\pgfqpoint{2.776230in}{3.148559in}}{\pgfqpoint{2.784043in}{3.140746in}}%
\pgfpathcurveto{\pgfqpoint{2.791857in}{3.132932in}}{\pgfqpoint{2.802456in}{3.128542in}}{\pgfqpoint{2.813506in}{3.128542in}}%
\pgfpathclose%
\pgfusepath{stroke,fill}%
\end{pgfscope}%
\begin{pgfscope}%
\pgfpathrectangle{\pgfqpoint{0.648703in}{0.548769in}}{\pgfqpoint{5.201297in}{3.102590in}}%
\pgfusepath{clip}%
\pgfsetbuttcap%
\pgfsetroundjoin%
\definecolor{currentfill}{rgb}{1.000000,0.498039,0.054902}%
\pgfsetfillcolor{currentfill}%
\pgfsetlinewidth{1.003750pt}%
\definecolor{currentstroke}{rgb}{1.000000,0.498039,0.054902}%
\pgfsetstrokecolor{currentstroke}%
\pgfsetdash{}{0pt}%
\pgfpathmoveto{\pgfqpoint{1.474790in}{3.149281in}}%
\pgfpathcurveto{\pgfqpoint{1.485840in}{3.149281in}}{\pgfqpoint{1.496439in}{3.153671in}}{\pgfqpoint{1.504252in}{3.161485in}}%
\pgfpathcurveto{\pgfqpoint{1.512066in}{3.169298in}}{\pgfqpoint{1.516456in}{3.179897in}}{\pgfqpoint{1.516456in}{3.190948in}}%
\pgfpathcurveto{\pgfqpoint{1.516456in}{3.201998in}}{\pgfqpoint{1.512066in}{3.212597in}}{\pgfqpoint{1.504252in}{3.220410in}}%
\pgfpathcurveto{\pgfqpoint{1.496439in}{3.228224in}}{\pgfqpoint{1.485840in}{3.232614in}}{\pgfqpoint{1.474790in}{3.232614in}}%
\pgfpathcurveto{\pgfqpoint{1.463740in}{3.232614in}}{\pgfqpoint{1.453141in}{3.228224in}}{\pgfqpoint{1.445327in}{3.220410in}}%
\pgfpathcurveto{\pgfqpoint{1.437513in}{3.212597in}}{\pgfqpoint{1.433123in}{3.201998in}}{\pgfqpoint{1.433123in}{3.190948in}}%
\pgfpathcurveto{\pgfqpoint{1.433123in}{3.179897in}}{\pgfqpoint{1.437513in}{3.169298in}}{\pgfqpoint{1.445327in}{3.161485in}}%
\pgfpathcurveto{\pgfqpoint{1.453141in}{3.153671in}}{\pgfqpoint{1.463740in}{3.149281in}}{\pgfqpoint{1.474790in}{3.149281in}}%
\pgfpathclose%
\pgfusepath{stroke,fill}%
\end{pgfscope}%
\begin{pgfscope}%
\pgfpathrectangle{\pgfqpoint{0.648703in}{0.548769in}}{\pgfqpoint{5.201297in}{3.102590in}}%
\pgfusepath{clip}%
\pgfsetbuttcap%
\pgfsetroundjoin%
\definecolor{currentfill}{rgb}{1.000000,0.498039,0.054902}%
\pgfsetfillcolor{currentfill}%
\pgfsetlinewidth{1.003750pt}%
\definecolor{currentstroke}{rgb}{1.000000,0.498039,0.054902}%
\pgfsetstrokecolor{currentstroke}%
\pgfsetdash{}{0pt}%
\pgfpathmoveto{\pgfqpoint{1.690545in}{3.240534in}}%
\pgfpathcurveto{\pgfqpoint{1.701595in}{3.240534in}}{\pgfqpoint{1.712194in}{3.244924in}}{\pgfqpoint{1.720007in}{3.252737in}}%
\pgfpathcurveto{\pgfqpoint{1.727821in}{3.260551in}}{\pgfqpoint{1.732211in}{3.271150in}}{\pgfqpoint{1.732211in}{3.282200in}}%
\pgfpathcurveto{\pgfqpoint{1.732211in}{3.293250in}}{\pgfqpoint{1.727821in}{3.303849in}}{\pgfqpoint{1.720007in}{3.311663in}}%
\pgfpathcurveto{\pgfqpoint{1.712194in}{3.319477in}}{\pgfqpoint{1.701595in}{3.323867in}}{\pgfqpoint{1.690545in}{3.323867in}}%
\pgfpathcurveto{\pgfqpoint{1.679494in}{3.323867in}}{\pgfqpoint{1.668895in}{3.319477in}}{\pgfqpoint{1.661082in}{3.311663in}}%
\pgfpathcurveto{\pgfqpoint{1.653268in}{3.303849in}}{\pgfqpoint{1.648878in}{3.293250in}}{\pgfqpoint{1.648878in}{3.282200in}}%
\pgfpathcurveto{\pgfqpoint{1.648878in}{3.271150in}}{\pgfqpoint{1.653268in}{3.260551in}}{\pgfqpoint{1.661082in}{3.252737in}}%
\pgfpathcurveto{\pgfqpoint{1.668895in}{3.244924in}}{\pgfqpoint{1.679494in}{3.240534in}}{\pgfqpoint{1.690545in}{3.240534in}}%
\pgfpathclose%
\pgfusepath{stroke,fill}%
\end{pgfscope}%
\begin{pgfscope}%
\pgfpathrectangle{\pgfqpoint{0.648703in}{0.548769in}}{\pgfqpoint{5.201297in}{3.102590in}}%
\pgfusepath{clip}%
\pgfsetbuttcap%
\pgfsetroundjoin%
\definecolor{currentfill}{rgb}{0.121569,0.466667,0.705882}%
\pgfsetfillcolor{currentfill}%
\pgfsetlinewidth{1.003750pt}%
\definecolor{currentstroke}{rgb}{0.121569,0.466667,0.705882}%
\pgfsetstrokecolor{currentstroke}%
\pgfsetdash{}{0pt}%
\pgfpathmoveto{\pgfqpoint{0.930200in}{0.664720in}}%
\pgfpathcurveto{\pgfqpoint{0.941250in}{0.664720in}}{\pgfqpoint{0.951849in}{0.669111in}}{\pgfqpoint{0.959663in}{0.676924in}}%
\pgfpathcurveto{\pgfqpoint{0.967476in}{0.684738in}}{\pgfqpoint{0.971866in}{0.695337in}}{\pgfqpoint{0.971866in}{0.706387in}}%
\pgfpathcurveto{\pgfqpoint{0.971866in}{0.717437in}}{\pgfqpoint{0.967476in}{0.728036in}}{\pgfqpoint{0.959663in}{0.735850in}}%
\pgfpathcurveto{\pgfqpoint{0.951849in}{0.743663in}}{\pgfqpoint{0.941250in}{0.748054in}}{\pgfqpoint{0.930200in}{0.748054in}}%
\pgfpathcurveto{\pgfqpoint{0.919150in}{0.748054in}}{\pgfqpoint{0.908551in}{0.743663in}}{\pgfqpoint{0.900737in}{0.735850in}}%
\pgfpathcurveto{\pgfqpoint{0.892923in}{0.728036in}}{\pgfqpoint{0.888533in}{0.717437in}}{\pgfqpoint{0.888533in}{0.706387in}}%
\pgfpathcurveto{\pgfqpoint{0.888533in}{0.695337in}}{\pgfqpoint{0.892923in}{0.684738in}}{\pgfqpoint{0.900737in}{0.676924in}}%
\pgfpathcurveto{\pgfqpoint{0.908551in}{0.669111in}}{\pgfqpoint{0.919150in}{0.664720in}}{\pgfqpoint{0.930200in}{0.664720in}}%
\pgfpathclose%
\pgfusepath{stroke,fill}%
\end{pgfscope}%
\begin{pgfscope}%
\pgfpathrectangle{\pgfqpoint{0.648703in}{0.548769in}}{\pgfqpoint{5.201297in}{3.102590in}}%
\pgfusepath{clip}%
\pgfsetbuttcap%
\pgfsetroundjoin%
\definecolor{currentfill}{rgb}{1.000000,0.498039,0.054902}%
\pgfsetfillcolor{currentfill}%
\pgfsetlinewidth{1.003750pt}%
\definecolor{currentstroke}{rgb}{1.000000,0.498039,0.054902}%
\pgfsetstrokecolor{currentstroke}%
\pgfsetdash{}{0pt}%
\pgfpathmoveto{\pgfqpoint{1.239482in}{3.157577in}}%
\pgfpathcurveto{\pgfqpoint{1.250532in}{3.157577in}}{\pgfqpoint{1.261131in}{3.161967in}}{\pgfqpoint{1.268945in}{3.169780in}}%
\pgfpathcurveto{\pgfqpoint{1.276759in}{3.177594in}}{\pgfqpoint{1.281149in}{3.188193in}}{\pgfqpoint{1.281149in}{3.199243in}}%
\pgfpathcurveto{\pgfqpoint{1.281149in}{3.210293in}}{\pgfqpoint{1.276759in}{3.220892in}}{\pgfqpoint{1.268945in}{3.228706in}}%
\pgfpathcurveto{\pgfqpoint{1.261131in}{3.236520in}}{\pgfqpoint{1.250532in}{3.240910in}}{\pgfqpoint{1.239482in}{3.240910in}}%
\pgfpathcurveto{\pgfqpoint{1.228432in}{3.240910in}}{\pgfqpoint{1.217833in}{3.236520in}}{\pgfqpoint{1.210020in}{3.228706in}}%
\pgfpathcurveto{\pgfqpoint{1.202206in}{3.220892in}}{\pgfqpoint{1.197816in}{3.210293in}}{\pgfqpoint{1.197816in}{3.199243in}}%
\pgfpathcurveto{\pgfqpoint{1.197816in}{3.188193in}}{\pgfqpoint{1.202206in}{3.177594in}}{\pgfqpoint{1.210020in}{3.169780in}}%
\pgfpathcurveto{\pgfqpoint{1.217833in}{3.161967in}}{\pgfqpoint{1.228432in}{3.157577in}}{\pgfqpoint{1.239482in}{3.157577in}}%
\pgfpathclose%
\pgfusepath{stroke,fill}%
\end{pgfscope}%
\begin{pgfscope}%
\pgfpathrectangle{\pgfqpoint{0.648703in}{0.548769in}}{\pgfqpoint{5.201297in}{3.102590in}}%
\pgfusepath{clip}%
\pgfsetbuttcap%
\pgfsetroundjoin%
\definecolor{currentfill}{rgb}{1.000000,0.498039,0.054902}%
\pgfsetfillcolor{currentfill}%
\pgfsetlinewidth{1.003750pt}%
\definecolor{currentstroke}{rgb}{1.000000,0.498039,0.054902}%
\pgfsetstrokecolor{currentstroke}%
\pgfsetdash{}{0pt}%
\pgfpathmoveto{\pgfqpoint{1.274598in}{3.140985in}}%
\pgfpathcurveto{\pgfqpoint{1.285648in}{3.140985in}}{\pgfqpoint{1.296247in}{3.145375in}}{\pgfqpoint{1.304060in}{3.153189in}}%
\pgfpathcurveto{\pgfqpoint{1.311874in}{3.161003in}}{\pgfqpoint{1.316264in}{3.171602in}}{\pgfqpoint{1.316264in}{3.182652in}}%
\pgfpathcurveto{\pgfqpoint{1.316264in}{3.193702in}}{\pgfqpoint{1.311874in}{3.204301in}}{\pgfqpoint{1.304060in}{3.212115in}}%
\pgfpathcurveto{\pgfqpoint{1.296247in}{3.219928in}}{\pgfqpoint{1.285648in}{3.224319in}}{\pgfqpoint{1.274598in}{3.224319in}}%
\pgfpathcurveto{\pgfqpoint{1.263548in}{3.224319in}}{\pgfqpoint{1.252949in}{3.219928in}}{\pgfqpoint{1.245135in}{3.212115in}}%
\pgfpathcurveto{\pgfqpoint{1.237321in}{3.204301in}}{\pgfqpoint{1.232931in}{3.193702in}}{\pgfqpoint{1.232931in}{3.182652in}}%
\pgfpathcurveto{\pgfqpoint{1.232931in}{3.171602in}}{\pgfqpoint{1.237321in}{3.161003in}}{\pgfqpoint{1.245135in}{3.153189in}}%
\pgfpathcurveto{\pgfqpoint{1.252949in}{3.145375in}}{\pgfqpoint{1.263548in}{3.140985in}}{\pgfqpoint{1.274598in}{3.140985in}}%
\pgfpathclose%
\pgfusepath{stroke,fill}%
\end{pgfscope}%
\begin{pgfscope}%
\pgfpathrectangle{\pgfqpoint{0.648703in}{0.548769in}}{\pgfqpoint{5.201297in}{3.102590in}}%
\pgfusepath{clip}%
\pgfsetbuttcap%
\pgfsetroundjoin%
\definecolor{currentfill}{rgb}{1.000000,0.498039,0.054902}%
\pgfsetfillcolor{currentfill}%
\pgfsetlinewidth{1.003750pt}%
\definecolor{currentstroke}{rgb}{1.000000,0.498039,0.054902}%
\pgfsetstrokecolor{currentstroke}%
\pgfsetdash{}{0pt}%
\pgfpathmoveto{\pgfqpoint{1.634147in}{3.136837in}}%
\pgfpathcurveto{\pgfqpoint{1.645197in}{3.136837in}}{\pgfqpoint{1.655796in}{3.141228in}}{\pgfqpoint{1.663610in}{3.149041in}}%
\pgfpathcurveto{\pgfqpoint{1.671423in}{3.156855in}}{\pgfqpoint{1.675814in}{3.167454in}}{\pgfqpoint{1.675814in}{3.178504in}}%
\pgfpathcurveto{\pgfqpoint{1.675814in}{3.189554in}}{\pgfqpoint{1.671423in}{3.200153in}}{\pgfqpoint{1.663610in}{3.207967in}}%
\pgfpathcurveto{\pgfqpoint{1.655796in}{3.215780in}}{\pgfqpoint{1.645197in}{3.220171in}}{\pgfqpoint{1.634147in}{3.220171in}}%
\pgfpathcurveto{\pgfqpoint{1.623097in}{3.220171in}}{\pgfqpoint{1.612498in}{3.215780in}}{\pgfqpoint{1.604684in}{3.207967in}}%
\pgfpathcurveto{\pgfqpoint{1.596871in}{3.200153in}}{\pgfqpoint{1.592480in}{3.189554in}}{\pgfqpoint{1.592480in}{3.178504in}}%
\pgfpathcurveto{\pgfqpoint{1.592480in}{3.167454in}}{\pgfqpoint{1.596871in}{3.156855in}}{\pgfqpoint{1.604684in}{3.149041in}}%
\pgfpathcurveto{\pgfqpoint{1.612498in}{3.141228in}}{\pgfqpoint{1.623097in}{3.136837in}}{\pgfqpoint{1.634147in}{3.136837in}}%
\pgfpathclose%
\pgfusepath{stroke,fill}%
\end{pgfscope}%
\begin{pgfscope}%
\pgfpathrectangle{\pgfqpoint{0.648703in}{0.548769in}}{\pgfqpoint{5.201297in}{3.102590in}}%
\pgfusepath{clip}%
\pgfsetbuttcap%
\pgfsetroundjoin%
\definecolor{currentfill}{rgb}{1.000000,0.498039,0.054902}%
\pgfsetfillcolor{currentfill}%
\pgfsetlinewidth{1.003750pt}%
\definecolor{currentstroke}{rgb}{1.000000,0.498039,0.054902}%
\pgfsetstrokecolor{currentstroke}%
\pgfsetdash{}{0pt}%
\pgfpathmoveto{\pgfqpoint{1.522226in}{3.136837in}}%
\pgfpathcurveto{\pgfqpoint{1.533277in}{3.136837in}}{\pgfqpoint{1.543876in}{3.141228in}}{\pgfqpoint{1.551689in}{3.149041in}}%
\pgfpathcurveto{\pgfqpoint{1.559503in}{3.156855in}}{\pgfqpoint{1.563893in}{3.167454in}}{\pgfqpoint{1.563893in}{3.178504in}}%
\pgfpathcurveto{\pgfqpoint{1.563893in}{3.189554in}}{\pgfqpoint{1.559503in}{3.200153in}}{\pgfqpoint{1.551689in}{3.207967in}}%
\pgfpathcurveto{\pgfqpoint{1.543876in}{3.215780in}}{\pgfqpoint{1.533277in}{3.220171in}}{\pgfqpoint{1.522226in}{3.220171in}}%
\pgfpathcurveto{\pgfqpoint{1.511176in}{3.220171in}}{\pgfqpoint{1.500577in}{3.215780in}}{\pgfqpoint{1.492764in}{3.207967in}}%
\pgfpathcurveto{\pgfqpoint{1.484950in}{3.200153in}}{\pgfqpoint{1.480560in}{3.189554in}}{\pgfqpoint{1.480560in}{3.178504in}}%
\pgfpathcurveto{\pgfqpoint{1.480560in}{3.167454in}}{\pgfqpoint{1.484950in}{3.156855in}}{\pgfqpoint{1.492764in}{3.149041in}}%
\pgfpathcurveto{\pgfqpoint{1.500577in}{3.141228in}}{\pgfqpoint{1.511176in}{3.136837in}}{\pgfqpoint{1.522226in}{3.136837in}}%
\pgfpathclose%
\pgfusepath{stroke,fill}%
\end{pgfscope}%
\begin{pgfscope}%
\pgfpathrectangle{\pgfqpoint{0.648703in}{0.548769in}}{\pgfqpoint{5.201297in}{3.102590in}}%
\pgfusepath{clip}%
\pgfsetbuttcap%
\pgfsetroundjoin%
\definecolor{currentfill}{rgb}{1.000000,0.498039,0.054902}%
\pgfsetfillcolor{currentfill}%
\pgfsetlinewidth{1.003750pt}%
\definecolor{currentstroke}{rgb}{1.000000,0.498039,0.054902}%
\pgfsetstrokecolor{currentstroke}%
\pgfsetdash{}{0pt}%
\pgfpathmoveto{\pgfqpoint{1.322687in}{3.136837in}}%
\pgfpathcurveto{\pgfqpoint{1.333738in}{3.136837in}}{\pgfqpoint{1.344337in}{3.141228in}}{\pgfqpoint{1.352150in}{3.149041in}}%
\pgfpathcurveto{\pgfqpoint{1.359964in}{3.156855in}}{\pgfqpoint{1.364354in}{3.167454in}}{\pgfqpoint{1.364354in}{3.178504in}}%
\pgfpathcurveto{\pgfqpoint{1.364354in}{3.189554in}}{\pgfqpoint{1.359964in}{3.200153in}}{\pgfqpoint{1.352150in}{3.207967in}}%
\pgfpathcurveto{\pgfqpoint{1.344337in}{3.215780in}}{\pgfqpoint{1.333738in}{3.220171in}}{\pgfqpoint{1.322687in}{3.220171in}}%
\pgfpathcurveto{\pgfqpoint{1.311637in}{3.220171in}}{\pgfqpoint{1.301038in}{3.215780in}}{\pgfqpoint{1.293225in}{3.207967in}}%
\pgfpathcurveto{\pgfqpoint{1.285411in}{3.200153in}}{\pgfqpoint{1.281021in}{3.189554in}}{\pgfqpoint{1.281021in}{3.178504in}}%
\pgfpathcurveto{\pgfqpoint{1.281021in}{3.167454in}}{\pgfqpoint{1.285411in}{3.156855in}}{\pgfqpoint{1.293225in}{3.149041in}}%
\pgfpathcurveto{\pgfqpoint{1.301038in}{3.141228in}}{\pgfqpoint{1.311637in}{3.136837in}}{\pgfqpoint{1.322687in}{3.136837in}}%
\pgfpathclose%
\pgfusepath{stroke,fill}%
\end{pgfscope}%
\begin{pgfscope}%
\pgfpathrectangle{\pgfqpoint{0.648703in}{0.548769in}}{\pgfqpoint{5.201297in}{3.102590in}}%
\pgfusepath{clip}%
\pgfsetbuttcap%
\pgfsetroundjoin%
\definecolor{currentfill}{rgb}{1.000000,0.498039,0.054902}%
\pgfsetfillcolor{currentfill}%
\pgfsetlinewidth{1.003750pt}%
\definecolor{currentstroke}{rgb}{1.000000,0.498039,0.054902}%
\pgfsetstrokecolor{currentstroke}%
\pgfsetdash{}{0pt}%
\pgfpathmoveto{\pgfqpoint{1.304219in}{3.136837in}}%
\pgfpathcurveto{\pgfqpoint{1.315270in}{3.136837in}}{\pgfqpoint{1.325869in}{3.141228in}}{\pgfqpoint{1.333682in}{3.149041in}}%
\pgfpathcurveto{\pgfqpoint{1.341496in}{3.156855in}}{\pgfqpoint{1.345886in}{3.167454in}}{\pgfqpoint{1.345886in}{3.178504in}}%
\pgfpathcurveto{\pgfqpoint{1.345886in}{3.189554in}}{\pgfqpoint{1.341496in}{3.200153in}}{\pgfqpoint{1.333682in}{3.207967in}}%
\pgfpathcurveto{\pgfqpoint{1.325869in}{3.215780in}}{\pgfqpoint{1.315270in}{3.220171in}}{\pgfqpoint{1.304219in}{3.220171in}}%
\pgfpathcurveto{\pgfqpoint{1.293169in}{3.220171in}}{\pgfqpoint{1.282570in}{3.215780in}}{\pgfqpoint{1.274757in}{3.207967in}}%
\pgfpathcurveto{\pgfqpoint{1.266943in}{3.200153in}}{\pgfqpoint{1.262553in}{3.189554in}}{\pgfqpoint{1.262553in}{3.178504in}}%
\pgfpathcurveto{\pgfqpoint{1.262553in}{3.167454in}}{\pgfqpoint{1.266943in}{3.156855in}}{\pgfqpoint{1.274757in}{3.149041in}}%
\pgfpathcurveto{\pgfqpoint{1.282570in}{3.141228in}}{\pgfqpoint{1.293169in}{3.136837in}}{\pgfqpoint{1.304219in}{3.136837in}}%
\pgfpathclose%
\pgfusepath{stroke,fill}%
\end{pgfscope}%
\begin{pgfscope}%
\pgfpathrectangle{\pgfqpoint{0.648703in}{0.548769in}}{\pgfqpoint{5.201297in}{3.102590in}}%
\pgfusepath{clip}%
\pgfsetbuttcap%
\pgfsetroundjoin%
\definecolor{currentfill}{rgb}{1.000000,0.498039,0.054902}%
\pgfsetfillcolor{currentfill}%
\pgfsetlinewidth{1.003750pt}%
\definecolor{currentstroke}{rgb}{1.000000,0.498039,0.054902}%
\pgfsetstrokecolor{currentstroke}%
\pgfsetdash{}{0pt}%
\pgfpathmoveto{\pgfqpoint{1.346689in}{3.136837in}}%
\pgfpathcurveto{\pgfqpoint{1.357739in}{3.136837in}}{\pgfqpoint{1.368338in}{3.141228in}}{\pgfqpoint{1.376152in}{3.149041in}}%
\pgfpathcurveto{\pgfqpoint{1.383965in}{3.156855in}}{\pgfqpoint{1.388356in}{3.167454in}}{\pgfqpoint{1.388356in}{3.178504in}}%
\pgfpathcurveto{\pgfqpoint{1.388356in}{3.189554in}}{\pgfqpoint{1.383965in}{3.200153in}}{\pgfqpoint{1.376152in}{3.207967in}}%
\pgfpathcurveto{\pgfqpoint{1.368338in}{3.215780in}}{\pgfqpoint{1.357739in}{3.220171in}}{\pgfqpoint{1.346689in}{3.220171in}}%
\pgfpathcurveto{\pgfqpoint{1.335639in}{3.220171in}}{\pgfqpoint{1.325040in}{3.215780in}}{\pgfqpoint{1.317226in}{3.207967in}}%
\pgfpathcurveto{\pgfqpoint{1.309412in}{3.200153in}}{\pgfqpoint{1.305022in}{3.189554in}}{\pgfqpoint{1.305022in}{3.178504in}}%
\pgfpathcurveto{\pgfqpoint{1.305022in}{3.167454in}}{\pgfqpoint{1.309412in}{3.156855in}}{\pgfqpoint{1.317226in}{3.149041in}}%
\pgfpathcurveto{\pgfqpoint{1.325040in}{3.141228in}}{\pgfqpoint{1.335639in}{3.136837in}}{\pgfqpoint{1.346689in}{3.136837in}}%
\pgfpathclose%
\pgfusepath{stroke,fill}%
\end{pgfscope}%
\begin{pgfscope}%
\pgfpathrectangle{\pgfqpoint{0.648703in}{0.548769in}}{\pgfqpoint{5.201297in}{3.102590in}}%
\pgfusepath{clip}%
\pgfsetbuttcap%
\pgfsetroundjoin%
\definecolor{currentfill}{rgb}{0.121569,0.466667,0.705882}%
\pgfsetfillcolor{currentfill}%
\pgfsetlinewidth{1.003750pt}%
\definecolor{currentstroke}{rgb}{0.121569,0.466667,0.705882}%
\pgfsetstrokecolor{currentstroke}%
\pgfsetdash{}{0pt}%
\pgfpathmoveto{\pgfqpoint{1.135177in}{2.846488in}}%
\pgfpathcurveto{\pgfqpoint{1.146227in}{2.846488in}}{\pgfqpoint{1.156826in}{2.850878in}}{\pgfqpoint{1.164640in}{2.858692in}}%
\pgfpathcurveto{\pgfqpoint{1.172453in}{2.866506in}}{\pgfqpoint{1.176844in}{2.877105in}}{\pgfqpoint{1.176844in}{2.888155in}}%
\pgfpathcurveto{\pgfqpoint{1.176844in}{2.899205in}}{\pgfqpoint{1.172453in}{2.909804in}}{\pgfqpoint{1.164640in}{2.917617in}}%
\pgfpathcurveto{\pgfqpoint{1.156826in}{2.925431in}}{\pgfqpoint{1.146227in}{2.929821in}}{\pgfqpoint{1.135177in}{2.929821in}}%
\pgfpathcurveto{\pgfqpoint{1.124127in}{2.929821in}}{\pgfqpoint{1.113528in}{2.925431in}}{\pgfqpoint{1.105714in}{2.917617in}}%
\pgfpathcurveto{\pgfqpoint{1.097901in}{2.909804in}}{\pgfqpoint{1.093510in}{2.899205in}}{\pgfqpoint{1.093510in}{2.888155in}}%
\pgfpathcurveto{\pgfqpoint{1.093510in}{2.877105in}}{\pgfqpoint{1.097901in}{2.866506in}}{\pgfqpoint{1.105714in}{2.858692in}}%
\pgfpathcurveto{\pgfqpoint{1.113528in}{2.850878in}}{\pgfqpoint{1.124127in}{2.846488in}}{\pgfqpoint{1.135177in}{2.846488in}}%
\pgfpathclose%
\pgfusepath{stroke,fill}%
\end{pgfscope}%
\begin{pgfscope}%
\pgfpathrectangle{\pgfqpoint{0.648703in}{0.548769in}}{\pgfqpoint{5.201297in}{3.102590in}}%
\pgfusepath{clip}%
\pgfsetbuttcap%
\pgfsetroundjoin%
\definecolor{currentfill}{rgb}{0.121569,0.466667,0.705882}%
\pgfsetfillcolor{currentfill}%
\pgfsetlinewidth{1.003750pt}%
\definecolor{currentstroke}{rgb}{0.121569,0.466667,0.705882}%
\pgfsetstrokecolor{currentstroke}%
\pgfsetdash{}{0pt}%
\pgfpathmoveto{\pgfqpoint{5.613577in}{3.074620in}}%
\pgfpathcurveto{\pgfqpoint{5.624628in}{3.074620in}}{\pgfqpoint{5.635227in}{3.079010in}}{\pgfqpoint{5.643040in}{3.086824in}}%
\pgfpathcurveto{\pgfqpoint{5.650854in}{3.094637in}}{\pgfqpoint{5.655244in}{3.105236in}}{\pgfqpoint{5.655244in}{3.116286in}}%
\pgfpathcurveto{\pgfqpoint{5.655244in}{3.127336in}}{\pgfqpoint{5.650854in}{3.137935in}}{\pgfqpoint{5.643040in}{3.145749in}}%
\pgfpathcurveto{\pgfqpoint{5.635227in}{3.153563in}}{\pgfqpoint{5.624628in}{3.157953in}}{\pgfqpoint{5.613577in}{3.157953in}}%
\pgfpathcurveto{\pgfqpoint{5.602527in}{3.157953in}}{\pgfqpoint{5.591928in}{3.153563in}}{\pgfqpoint{5.584115in}{3.145749in}}%
\pgfpathcurveto{\pgfqpoint{5.576301in}{3.137935in}}{\pgfqpoint{5.571911in}{3.127336in}}{\pgfqpoint{5.571911in}{3.116286in}}%
\pgfpathcurveto{\pgfqpoint{5.571911in}{3.105236in}}{\pgfqpoint{5.576301in}{3.094637in}}{\pgfqpoint{5.584115in}{3.086824in}}%
\pgfpathcurveto{\pgfqpoint{5.591928in}{3.079010in}}{\pgfqpoint{5.602527in}{3.074620in}}{\pgfqpoint{5.613577in}{3.074620in}}%
\pgfpathclose%
\pgfusepath{stroke,fill}%
\end{pgfscope}%
\begin{pgfscope}%
\pgfpathrectangle{\pgfqpoint{0.648703in}{0.548769in}}{\pgfqpoint{5.201297in}{3.102590in}}%
\pgfusepath{clip}%
\pgfsetbuttcap%
\pgfsetroundjoin%
\definecolor{currentfill}{rgb}{1.000000,0.498039,0.054902}%
\pgfsetfillcolor{currentfill}%
\pgfsetlinewidth{1.003750pt}%
\definecolor{currentstroke}{rgb}{1.000000,0.498039,0.054902}%
\pgfsetstrokecolor{currentstroke}%
\pgfsetdash{}{0pt}%
\pgfpathmoveto{\pgfqpoint{1.300855in}{3.315195in}}%
\pgfpathcurveto{\pgfqpoint{1.311905in}{3.315195in}}{\pgfqpoint{1.322504in}{3.319585in}}{\pgfqpoint{1.330318in}{3.327399in}}%
\pgfpathcurveto{\pgfqpoint{1.338132in}{3.335212in}}{\pgfqpoint{1.342522in}{3.345811in}}{\pgfqpoint{1.342522in}{3.356861in}}%
\pgfpathcurveto{\pgfqpoint{1.342522in}{3.367912in}}{\pgfqpoint{1.338132in}{3.378511in}}{\pgfqpoint{1.330318in}{3.386324in}}%
\pgfpathcurveto{\pgfqpoint{1.322504in}{3.394138in}}{\pgfqpoint{1.311905in}{3.398528in}}{\pgfqpoint{1.300855in}{3.398528in}}%
\pgfpathcurveto{\pgfqpoint{1.289805in}{3.398528in}}{\pgfqpoint{1.279206in}{3.394138in}}{\pgfqpoint{1.271392in}{3.386324in}}%
\pgfpathcurveto{\pgfqpoint{1.263579in}{3.378511in}}{\pgfqpoint{1.259188in}{3.367912in}}{\pgfqpoint{1.259188in}{3.356861in}}%
\pgfpathcurveto{\pgfqpoint{1.259188in}{3.345811in}}{\pgfqpoint{1.263579in}{3.335212in}}{\pgfqpoint{1.271392in}{3.327399in}}%
\pgfpathcurveto{\pgfqpoint{1.279206in}{3.319585in}}{\pgfqpoint{1.289805in}{3.315195in}}{\pgfqpoint{1.300855in}{3.315195in}}%
\pgfpathclose%
\pgfusepath{stroke,fill}%
\end{pgfscope}%
\begin{pgfscope}%
\pgfpathrectangle{\pgfqpoint{0.648703in}{0.548769in}}{\pgfqpoint{5.201297in}{3.102590in}}%
\pgfusepath{clip}%
\pgfsetbuttcap%
\pgfsetroundjoin%
\definecolor{currentfill}{rgb}{0.121569,0.466667,0.705882}%
\pgfsetfillcolor{currentfill}%
\pgfsetlinewidth{1.003750pt}%
\definecolor{currentstroke}{rgb}{0.121569,0.466667,0.705882}%
\pgfsetstrokecolor{currentstroke}%
\pgfsetdash{}{0pt}%
\pgfpathmoveto{\pgfqpoint{1.100576in}{0.648129in}}%
\pgfpathcurveto{\pgfqpoint{1.111626in}{0.648129in}}{\pgfqpoint{1.122225in}{0.652519in}}{\pgfqpoint{1.130039in}{0.660333in}}%
\pgfpathcurveto{\pgfqpoint{1.137852in}{0.668146in}}{\pgfqpoint{1.142243in}{0.678745in}}{\pgfqpoint{1.142243in}{0.689796in}}%
\pgfpathcurveto{\pgfqpoint{1.142243in}{0.700846in}}{\pgfqpoint{1.137852in}{0.711445in}}{\pgfqpoint{1.130039in}{0.719258in}}%
\pgfpathcurveto{\pgfqpoint{1.122225in}{0.727072in}}{\pgfqpoint{1.111626in}{0.731462in}}{\pgfqpoint{1.100576in}{0.731462in}}%
\pgfpathcurveto{\pgfqpoint{1.089526in}{0.731462in}}{\pgfqpoint{1.078927in}{0.727072in}}{\pgfqpoint{1.071113in}{0.719258in}}%
\pgfpathcurveto{\pgfqpoint{1.063300in}{0.711445in}}{\pgfqpoint{1.058909in}{0.700846in}}{\pgfqpoint{1.058909in}{0.689796in}}%
\pgfpathcurveto{\pgfqpoint{1.058909in}{0.678745in}}{\pgfqpoint{1.063300in}{0.668146in}}{\pgfqpoint{1.071113in}{0.660333in}}%
\pgfpathcurveto{\pgfqpoint{1.078927in}{0.652519in}}{\pgfqpoint{1.089526in}{0.648129in}}{\pgfqpoint{1.100576in}{0.648129in}}%
\pgfpathclose%
\pgfusepath{stroke,fill}%
\end{pgfscope}%
\begin{pgfscope}%
\pgfpathrectangle{\pgfqpoint{0.648703in}{0.548769in}}{\pgfqpoint{5.201297in}{3.102590in}}%
\pgfusepath{clip}%
\pgfsetbuttcap%
\pgfsetroundjoin%
\definecolor{currentfill}{rgb}{0.121569,0.466667,0.705882}%
\pgfsetfillcolor{currentfill}%
\pgfsetlinewidth{1.003750pt}%
\definecolor{currentstroke}{rgb}{0.121569,0.466667,0.705882}%
\pgfsetstrokecolor{currentstroke}%
\pgfsetdash{}{0pt}%
\pgfpathmoveto{\pgfqpoint{1.505809in}{1.166610in}}%
\pgfpathcurveto{\pgfqpoint{1.516859in}{1.166610in}}{\pgfqpoint{1.527458in}{1.171000in}}{\pgfqpoint{1.535271in}{1.178814in}}%
\pgfpathcurveto{\pgfqpoint{1.543085in}{1.186627in}}{\pgfqpoint{1.547475in}{1.197226in}}{\pgfqpoint{1.547475in}{1.208277in}}%
\pgfpathcurveto{\pgfqpoint{1.547475in}{1.219327in}}{\pgfqpoint{1.543085in}{1.229926in}}{\pgfqpoint{1.535271in}{1.237739in}}%
\pgfpathcurveto{\pgfqpoint{1.527458in}{1.245553in}}{\pgfqpoint{1.516859in}{1.249943in}}{\pgfqpoint{1.505809in}{1.249943in}}%
\pgfpathcurveto{\pgfqpoint{1.494758in}{1.249943in}}{\pgfqpoint{1.484159in}{1.245553in}}{\pgfqpoint{1.476346in}{1.237739in}}%
\pgfpathcurveto{\pgfqpoint{1.468532in}{1.229926in}}{\pgfqpoint{1.464142in}{1.219327in}}{\pgfqpoint{1.464142in}{1.208277in}}%
\pgfpathcurveto{\pgfqpoint{1.464142in}{1.197226in}}{\pgfqpoint{1.468532in}{1.186627in}}{\pgfqpoint{1.476346in}{1.178814in}}%
\pgfpathcurveto{\pgfqpoint{1.484159in}{1.171000in}}{\pgfqpoint{1.494758in}{1.166610in}}{\pgfqpoint{1.505809in}{1.166610in}}%
\pgfpathclose%
\pgfusepath{stroke,fill}%
\end{pgfscope}%
\begin{pgfscope}%
\pgfpathrectangle{\pgfqpoint{0.648703in}{0.548769in}}{\pgfqpoint{5.201297in}{3.102590in}}%
\pgfusepath{clip}%
\pgfsetbuttcap%
\pgfsetroundjoin%
\definecolor{currentfill}{rgb}{1.000000,0.498039,0.054902}%
\pgfsetfillcolor{currentfill}%
\pgfsetlinewidth{1.003750pt}%
\definecolor{currentstroke}{rgb}{1.000000,0.498039,0.054902}%
\pgfsetstrokecolor{currentstroke}%
\pgfsetdash{}{0pt}%
\pgfpathmoveto{\pgfqpoint{1.098086in}{3.136837in}}%
\pgfpathcurveto{\pgfqpoint{1.109137in}{3.136837in}}{\pgfqpoint{1.119736in}{3.141228in}}{\pgfqpoint{1.127549in}{3.149041in}}%
\pgfpathcurveto{\pgfqpoint{1.135363in}{3.156855in}}{\pgfqpoint{1.139753in}{3.167454in}}{\pgfqpoint{1.139753in}{3.178504in}}%
\pgfpathcurveto{\pgfqpoint{1.139753in}{3.189554in}}{\pgfqpoint{1.135363in}{3.200153in}}{\pgfqpoint{1.127549in}{3.207967in}}%
\pgfpathcurveto{\pgfqpoint{1.119736in}{3.215780in}}{\pgfqpoint{1.109137in}{3.220171in}}{\pgfqpoint{1.098086in}{3.220171in}}%
\pgfpathcurveto{\pgfqpoint{1.087036in}{3.220171in}}{\pgfqpoint{1.076437in}{3.215780in}}{\pgfqpoint{1.068624in}{3.207967in}}%
\pgfpathcurveto{\pgfqpoint{1.060810in}{3.200153in}}{\pgfqpoint{1.056420in}{3.189554in}}{\pgfqpoint{1.056420in}{3.178504in}}%
\pgfpathcurveto{\pgfqpoint{1.056420in}{3.167454in}}{\pgfqpoint{1.060810in}{3.156855in}}{\pgfqpoint{1.068624in}{3.149041in}}%
\pgfpathcurveto{\pgfqpoint{1.076437in}{3.141228in}}{\pgfqpoint{1.087036in}{3.136837in}}{\pgfqpoint{1.098086in}{3.136837in}}%
\pgfpathclose%
\pgfusepath{stroke,fill}%
\end{pgfscope}%
\begin{pgfscope}%
\pgfpathrectangle{\pgfqpoint{0.648703in}{0.548769in}}{\pgfqpoint{5.201297in}{3.102590in}}%
\pgfusepath{clip}%
\pgfsetbuttcap%
\pgfsetroundjoin%
\definecolor{currentfill}{rgb}{1.000000,0.498039,0.054902}%
\pgfsetfillcolor{currentfill}%
\pgfsetlinewidth{1.003750pt}%
\definecolor{currentstroke}{rgb}{1.000000,0.498039,0.054902}%
\pgfsetstrokecolor{currentstroke}%
\pgfsetdash{}{0pt}%
\pgfpathmoveto{\pgfqpoint{2.470565in}{3.132690in}}%
\pgfpathcurveto{\pgfqpoint{2.481615in}{3.132690in}}{\pgfqpoint{2.492214in}{3.137080in}}{\pgfqpoint{2.500027in}{3.144893in}}%
\pgfpathcurveto{\pgfqpoint{2.507841in}{3.152707in}}{\pgfqpoint{2.512231in}{3.163306in}}{\pgfqpoint{2.512231in}{3.174356in}}%
\pgfpathcurveto{\pgfqpoint{2.512231in}{3.185406in}}{\pgfqpoint{2.507841in}{3.196005in}}{\pgfqpoint{2.500027in}{3.203819in}}%
\pgfpathcurveto{\pgfqpoint{2.492214in}{3.211633in}}{\pgfqpoint{2.481615in}{3.216023in}}{\pgfqpoint{2.470565in}{3.216023in}}%
\pgfpathcurveto{\pgfqpoint{2.459514in}{3.216023in}}{\pgfqpoint{2.448915in}{3.211633in}}{\pgfqpoint{2.441102in}{3.203819in}}%
\pgfpathcurveto{\pgfqpoint{2.433288in}{3.196005in}}{\pgfqpoint{2.428898in}{3.185406in}}{\pgfqpoint{2.428898in}{3.174356in}}%
\pgfpathcurveto{\pgfqpoint{2.428898in}{3.163306in}}{\pgfqpoint{2.433288in}{3.152707in}}{\pgfqpoint{2.441102in}{3.144893in}}%
\pgfpathcurveto{\pgfqpoint{2.448915in}{3.137080in}}{\pgfqpoint{2.459514in}{3.132690in}}{\pgfqpoint{2.470565in}{3.132690in}}%
\pgfpathclose%
\pgfusepath{stroke,fill}%
\end{pgfscope}%
\begin{pgfscope}%
\pgfpathrectangle{\pgfqpoint{0.648703in}{0.548769in}}{\pgfqpoint{5.201297in}{3.102590in}}%
\pgfusepath{clip}%
\pgfsetbuttcap%
\pgfsetroundjoin%
\definecolor{currentfill}{rgb}{0.839216,0.152941,0.156863}%
\pgfsetfillcolor{currentfill}%
\pgfsetlinewidth{1.003750pt}%
\definecolor{currentstroke}{rgb}{0.839216,0.152941,0.156863}%
\pgfsetstrokecolor{currentstroke}%
\pgfsetdash{}{0pt}%
\pgfpathmoveto{\pgfqpoint{2.017278in}{3.149281in}}%
\pgfpathcurveto{\pgfqpoint{2.028328in}{3.149281in}}{\pgfqpoint{2.038927in}{3.153671in}}{\pgfqpoint{2.046741in}{3.161485in}}%
\pgfpathcurveto{\pgfqpoint{2.054554in}{3.169298in}}{\pgfqpoint{2.058945in}{3.179897in}}{\pgfqpoint{2.058945in}{3.190948in}}%
\pgfpathcurveto{\pgfqpoint{2.058945in}{3.201998in}}{\pgfqpoint{2.054554in}{3.212597in}}{\pgfqpoint{2.046741in}{3.220410in}}%
\pgfpathcurveto{\pgfqpoint{2.038927in}{3.228224in}}{\pgfqpoint{2.028328in}{3.232614in}}{\pgfqpoint{2.017278in}{3.232614in}}%
\pgfpathcurveto{\pgfqpoint{2.006228in}{3.232614in}}{\pgfqpoint{1.995629in}{3.228224in}}{\pgfqpoint{1.987815in}{3.220410in}}%
\pgfpathcurveto{\pgfqpoint{1.980002in}{3.212597in}}{\pgfqpoint{1.975611in}{3.201998in}}{\pgfqpoint{1.975611in}{3.190948in}}%
\pgfpathcurveto{\pgfqpoint{1.975611in}{3.179897in}}{\pgfqpoint{1.980002in}{3.169298in}}{\pgfqpoint{1.987815in}{3.161485in}}%
\pgfpathcurveto{\pgfqpoint{1.995629in}{3.153671in}}{\pgfqpoint{2.006228in}{3.149281in}}{\pgfqpoint{2.017278in}{3.149281in}}%
\pgfpathclose%
\pgfusepath{stroke,fill}%
\end{pgfscope}%
\begin{pgfscope}%
\pgfpathrectangle{\pgfqpoint{0.648703in}{0.548769in}}{\pgfqpoint{5.201297in}{3.102590in}}%
\pgfusepath{clip}%
\pgfsetbuttcap%
\pgfsetroundjoin%
\definecolor{currentfill}{rgb}{1.000000,0.498039,0.054902}%
\pgfsetfillcolor{currentfill}%
\pgfsetlinewidth{1.003750pt}%
\definecolor{currentstroke}{rgb}{1.000000,0.498039,0.054902}%
\pgfsetstrokecolor{currentstroke}%
\pgfsetdash{}{0pt}%
\pgfpathmoveto{\pgfqpoint{1.318943in}{3.174168in}}%
\pgfpathcurveto{\pgfqpoint{1.329993in}{3.174168in}}{\pgfqpoint{1.340592in}{3.178558in}}{\pgfqpoint{1.348406in}{3.186372in}}%
\pgfpathcurveto{\pgfqpoint{1.356220in}{3.194185in}}{\pgfqpoint{1.360610in}{3.204785in}}{\pgfqpoint{1.360610in}{3.215835in}}%
\pgfpathcurveto{\pgfqpoint{1.360610in}{3.226885in}}{\pgfqpoint{1.356220in}{3.237484in}}{\pgfqpoint{1.348406in}{3.245297in}}%
\pgfpathcurveto{\pgfqpoint{1.340592in}{3.253111in}}{\pgfqpoint{1.329993in}{3.257501in}}{\pgfqpoint{1.318943in}{3.257501in}}%
\pgfpathcurveto{\pgfqpoint{1.307893in}{3.257501in}}{\pgfqpoint{1.297294in}{3.253111in}}{\pgfqpoint{1.289480in}{3.245297in}}%
\pgfpathcurveto{\pgfqpoint{1.281667in}{3.237484in}}{\pgfqpoint{1.277277in}{3.226885in}}{\pgfqpoint{1.277277in}{3.215835in}}%
\pgfpathcurveto{\pgfqpoint{1.277277in}{3.204785in}}{\pgfqpoint{1.281667in}{3.194185in}}{\pgfqpoint{1.289480in}{3.186372in}}%
\pgfpathcurveto{\pgfqpoint{1.297294in}{3.178558in}}{\pgfqpoint{1.307893in}{3.174168in}}{\pgfqpoint{1.318943in}{3.174168in}}%
\pgfpathclose%
\pgfusepath{stroke,fill}%
\end{pgfscope}%
\begin{pgfscope}%
\pgfpathrectangle{\pgfqpoint{0.648703in}{0.548769in}}{\pgfqpoint{5.201297in}{3.102590in}}%
\pgfusepath{clip}%
\pgfsetbuttcap%
\pgfsetroundjoin%
\definecolor{currentfill}{rgb}{0.121569,0.466667,0.705882}%
\pgfsetfillcolor{currentfill}%
\pgfsetlinewidth{1.003750pt}%
\definecolor{currentstroke}{rgb}{0.121569,0.466667,0.705882}%
\pgfsetstrokecolor{currentstroke}%
\pgfsetdash{}{0pt}%
\pgfpathmoveto{\pgfqpoint{0.935717in}{0.656425in}}%
\pgfpathcurveto{\pgfqpoint{0.946767in}{0.656425in}}{\pgfqpoint{0.957366in}{0.660815in}}{\pgfqpoint{0.965180in}{0.668629in}}%
\pgfpathcurveto{\pgfqpoint{0.972994in}{0.676442in}}{\pgfqpoint{0.977384in}{0.687041in}}{\pgfqpoint{0.977384in}{0.698091in}}%
\pgfpathcurveto{\pgfqpoint{0.977384in}{0.709141in}}{\pgfqpoint{0.972994in}{0.719740in}}{\pgfqpoint{0.965180in}{0.727554in}}%
\pgfpathcurveto{\pgfqpoint{0.957366in}{0.735368in}}{\pgfqpoint{0.946767in}{0.739758in}}{\pgfqpoint{0.935717in}{0.739758in}}%
\pgfpathcurveto{\pgfqpoint{0.924667in}{0.739758in}}{\pgfqpoint{0.914068in}{0.735368in}}{\pgfqpoint{0.906254in}{0.727554in}}%
\pgfpathcurveto{\pgfqpoint{0.898441in}{0.719740in}}{\pgfqpoint{0.894051in}{0.709141in}}{\pgfqpoint{0.894051in}{0.698091in}}%
\pgfpathcurveto{\pgfqpoint{0.894051in}{0.687041in}}{\pgfqpoint{0.898441in}{0.676442in}}{\pgfqpoint{0.906254in}{0.668629in}}%
\pgfpathcurveto{\pgfqpoint{0.914068in}{0.660815in}}{\pgfqpoint{0.924667in}{0.656425in}}{\pgfqpoint{0.935717in}{0.656425in}}%
\pgfpathclose%
\pgfusepath{stroke,fill}%
\end{pgfscope}%
\begin{pgfscope}%
\pgfpathrectangle{\pgfqpoint{0.648703in}{0.548769in}}{\pgfqpoint{5.201297in}{3.102590in}}%
\pgfusepath{clip}%
\pgfsetbuttcap%
\pgfsetroundjoin%
\definecolor{currentfill}{rgb}{1.000000,0.498039,0.054902}%
\pgfsetfillcolor{currentfill}%
\pgfsetlinewidth{1.003750pt}%
\definecolor{currentstroke}{rgb}{1.000000,0.498039,0.054902}%
\pgfsetstrokecolor{currentstroke}%
\pgfsetdash{}{0pt}%
\pgfpathmoveto{\pgfqpoint{2.391452in}{3.124394in}}%
\pgfpathcurveto{\pgfqpoint{2.402502in}{3.124394in}}{\pgfqpoint{2.413101in}{3.128784in}}{\pgfqpoint{2.420915in}{3.136598in}}%
\pgfpathcurveto{\pgfqpoint{2.428728in}{3.144411in}}{\pgfqpoint{2.433119in}{3.155010in}}{\pgfqpoint{2.433119in}{3.166060in}}%
\pgfpathcurveto{\pgfqpoint{2.433119in}{3.177111in}}{\pgfqpoint{2.428728in}{3.187710in}}{\pgfqpoint{2.420915in}{3.195523in}}%
\pgfpathcurveto{\pgfqpoint{2.413101in}{3.203337in}}{\pgfqpoint{2.402502in}{3.207727in}}{\pgfqpoint{2.391452in}{3.207727in}}%
\pgfpathcurveto{\pgfqpoint{2.380402in}{3.207727in}}{\pgfqpoint{2.369803in}{3.203337in}}{\pgfqpoint{2.361989in}{3.195523in}}%
\pgfpathcurveto{\pgfqpoint{2.354176in}{3.187710in}}{\pgfqpoint{2.349785in}{3.177111in}}{\pgfqpoint{2.349785in}{3.166060in}}%
\pgfpathcurveto{\pgfqpoint{2.349785in}{3.155010in}}{\pgfqpoint{2.354176in}{3.144411in}}{\pgfqpoint{2.361989in}{3.136598in}}%
\pgfpathcurveto{\pgfqpoint{2.369803in}{3.128784in}}{\pgfqpoint{2.380402in}{3.124394in}}{\pgfqpoint{2.391452in}{3.124394in}}%
\pgfpathclose%
\pgfusepath{stroke,fill}%
\end{pgfscope}%
\begin{pgfscope}%
\pgfpathrectangle{\pgfqpoint{0.648703in}{0.548769in}}{\pgfqpoint{5.201297in}{3.102590in}}%
\pgfusepath{clip}%
\pgfsetbuttcap%
\pgfsetroundjoin%
\definecolor{currentfill}{rgb}{0.121569,0.466667,0.705882}%
\pgfsetfillcolor{currentfill}%
\pgfsetlinewidth{1.003750pt}%
\definecolor{currentstroke}{rgb}{0.121569,0.466667,0.705882}%
\pgfsetstrokecolor{currentstroke}%
\pgfsetdash{}{0pt}%
\pgfpathmoveto{\pgfqpoint{0.949444in}{0.648129in}}%
\pgfpathcurveto{\pgfqpoint{0.960494in}{0.648129in}}{\pgfqpoint{0.971093in}{0.652519in}}{\pgfqpoint{0.978906in}{0.660333in}}%
\pgfpathcurveto{\pgfqpoint{0.986720in}{0.668146in}}{\pgfqpoint{0.991110in}{0.678745in}}{\pgfqpoint{0.991110in}{0.689796in}}%
\pgfpathcurveto{\pgfqpoint{0.991110in}{0.700846in}}{\pgfqpoint{0.986720in}{0.711445in}}{\pgfqpoint{0.978906in}{0.719258in}}%
\pgfpathcurveto{\pgfqpoint{0.971093in}{0.727072in}}{\pgfqpoint{0.960494in}{0.731462in}}{\pgfqpoint{0.949444in}{0.731462in}}%
\pgfpathcurveto{\pgfqpoint{0.938393in}{0.731462in}}{\pgfqpoint{0.927794in}{0.727072in}}{\pgfqpoint{0.919981in}{0.719258in}}%
\pgfpathcurveto{\pgfqpoint{0.912167in}{0.711445in}}{\pgfqpoint{0.907777in}{0.700846in}}{\pgfqpoint{0.907777in}{0.689796in}}%
\pgfpathcurveto{\pgfqpoint{0.907777in}{0.678745in}}{\pgfqpoint{0.912167in}{0.668146in}}{\pgfqpoint{0.919981in}{0.660333in}}%
\pgfpathcurveto{\pgfqpoint{0.927794in}{0.652519in}}{\pgfqpoint{0.938393in}{0.648129in}}{\pgfqpoint{0.949444in}{0.648129in}}%
\pgfpathclose%
\pgfusepath{stroke,fill}%
\end{pgfscope}%
\begin{pgfscope}%
\pgfpathrectangle{\pgfqpoint{0.648703in}{0.548769in}}{\pgfqpoint{5.201297in}{3.102590in}}%
\pgfusepath{clip}%
\pgfsetbuttcap%
\pgfsetroundjoin%
\definecolor{currentfill}{rgb}{1.000000,0.498039,0.054902}%
\pgfsetfillcolor{currentfill}%
\pgfsetlinewidth{1.003750pt}%
\definecolor{currentstroke}{rgb}{1.000000,0.498039,0.054902}%
\pgfsetstrokecolor{currentstroke}%
\pgfsetdash{}{0pt}%
\pgfpathmoveto{\pgfqpoint{1.469969in}{3.136837in}}%
\pgfpathcurveto{\pgfqpoint{1.481019in}{3.136837in}}{\pgfqpoint{1.491618in}{3.141228in}}{\pgfqpoint{1.499432in}{3.149041in}}%
\pgfpathcurveto{\pgfqpoint{1.507245in}{3.156855in}}{\pgfqpoint{1.511636in}{3.167454in}}{\pgfqpoint{1.511636in}{3.178504in}}%
\pgfpathcurveto{\pgfqpoint{1.511636in}{3.189554in}}{\pgfqpoint{1.507245in}{3.200153in}}{\pgfqpoint{1.499432in}{3.207967in}}%
\pgfpathcurveto{\pgfqpoint{1.491618in}{3.215780in}}{\pgfqpoint{1.481019in}{3.220171in}}{\pgfqpoint{1.469969in}{3.220171in}}%
\pgfpathcurveto{\pgfqpoint{1.458919in}{3.220171in}}{\pgfqpoint{1.448320in}{3.215780in}}{\pgfqpoint{1.440506in}{3.207967in}}%
\pgfpathcurveto{\pgfqpoint{1.432692in}{3.200153in}}{\pgfqpoint{1.428302in}{3.189554in}}{\pgfqpoint{1.428302in}{3.178504in}}%
\pgfpathcurveto{\pgfqpoint{1.428302in}{3.167454in}}{\pgfqpoint{1.432692in}{3.156855in}}{\pgfqpoint{1.440506in}{3.149041in}}%
\pgfpathcurveto{\pgfqpoint{1.448320in}{3.141228in}}{\pgfqpoint{1.458919in}{3.136837in}}{\pgfqpoint{1.469969in}{3.136837in}}%
\pgfpathclose%
\pgfusepath{stroke,fill}%
\end{pgfscope}%
\begin{pgfscope}%
\pgfpathrectangle{\pgfqpoint{0.648703in}{0.548769in}}{\pgfqpoint{5.201297in}{3.102590in}}%
\pgfusepath{clip}%
\pgfsetbuttcap%
\pgfsetroundjoin%
\definecolor{currentfill}{rgb}{1.000000,0.498039,0.054902}%
\pgfsetfillcolor{currentfill}%
\pgfsetlinewidth{1.003750pt}%
\definecolor{currentstroke}{rgb}{1.000000,0.498039,0.054902}%
\pgfsetstrokecolor{currentstroke}%
\pgfsetdash{}{0pt}%
\pgfpathmoveto{\pgfqpoint{1.282363in}{3.140985in}}%
\pgfpathcurveto{\pgfqpoint{1.293413in}{3.140985in}}{\pgfqpoint{1.304012in}{3.145375in}}{\pgfqpoint{1.311826in}{3.153189in}}%
\pgfpathcurveto{\pgfqpoint{1.319640in}{3.161003in}}{\pgfqpoint{1.324030in}{3.171602in}}{\pgfqpoint{1.324030in}{3.182652in}}%
\pgfpathcurveto{\pgfqpoint{1.324030in}{3.193702in}}{\pgfqpoint{1.319640in}{3.204301in}}{\pgfqpoint{1.311826in}{3.212115in}}%
\pgfpathcurveto{\pgfqpoint{1.304012in}{3.219928in}}{\pgfqpoint{1.293413in}{3.224319in}}{\pgfqpoint{1.282363in}{3.224319in}}%
\pgfpathcurveto{\pgfqpoint{1.271313in}{3.224319in}}{\pgfqpoint{1.260714in}{3.219928in}}{\pgfqpoint{1.252901in}{3.212115in}}%
\pgfpathcurveto{\pgfqpoint{1.245087in}{3.204301in}}{\pgfqpoint{1.240697in}{3.193702in}}{\pgfqpoint{1.240697in}{3.182652in}}%
\pgfpathcurveto{\pgfqpoint{1.240697in}{3.171602in}}{\pgfqpoint{1.245087in}{3.161003in}}{\pgfqpoint{1.252901in}{3.153189in}}%
\pgfpathcurveto{\pgfqpoint{1.260714in}{3.145375in}}{\pgfqpoint{1.271313in}{3.140985in}}{\pgfqpoint{1.282363in}{3.140985in}}%
\pgfpathclose%
\pgfusepath{stroke,fill}%
\end{pgfscope}%
\begin{pgfscope}%
\pgfpathrectangle{\pgfqpoint{0.648703in}{0.548769in}}{\pgfqpoint{5.201297in}{3.102590in}}%
\pgfusepath{clip}%
\pgfsetbuttcap%
\pgfsetroundjoin%
\definecolor{currentfill}{rgb}{1.000000,0.498039,0.054902}%
\pgfsetfillcolor{currentfill}%
\pgfsetlinewidth{1.003750pt}%
\definecolor{currentstroke}{rgb}{1.000000,0.498039,0.054902}%
\pgfsetstrokecolor{currentstroke}%
\pgfsetdash{}{0pt}%
\pgfpathmoveto{\pgfqpoint{1.887685in}{3.165872in}}%
\pgfpathcurveto{\pgfqpoint{1.898735in}{3.165872in}}{\pgfqpoint{1.909334in}{3.170263in}}{\pgfqpoint{1.917148in}{3.178076in}}%
\pgfpathcurveto{\pgfqpoint{1.924961in}{3.185890in}}{\pgfqpoint{1.929352in}{3.196489in}}{\pgfqpoint{1.929352in}{3.207539in}}%
\pgfpathcurveto{\pgfqpoint{1.929352in}{3.218589in}}{\pgfqpoint{1.924961in}{3.229188in}}{\pgfqpoint{1.917148in}{3.237002in}}%
\pgfpathcurveto{\pgfqpoint{1.909334in}{3.244815in}}{\pgfqpoint{1.898735in}{3.249206in}}{\pgfqpoint{1.887685in}{3.249206in}}%
\pgfpathcurveto{\pgfqpoint{1.876635in}{3.249206in}}{\pgfqpoint{1.866036in}{3.244815in}}{\pgfqpoint{1.858222in}{3.237002in}}%
\pgfpathcurveto{\pgfqpoint{1.850408in}{3.229188in}}{\pgfqpoint{1.846018in}{3.218589in}}{\pgfqpoint{1.846018in}{3.207539in}}%
\pgfpathcurveto{\pgfqpoint{1.846018in}{3.196489in}}{\pgfqpoint{1.850408in}{3.185890in}}{\pgfqpoint{1.858222in}{3.178076in}}%
\pgfpathcurveto{\pgfqpoint{1.866036in}{3.170263in}}{\pgfqpoint{1.876635in}{3.165872in}}{\pgfqpoint{1.887685in}{3.165872in}}%
\pgfpathclose%
\pgfusepath{stroke,fill}%
\end{pgfscope}%
\begin{pgfscope}%
\pgfpathrectangle{\pgfqpoint{0.648703in}{0.548769in}}{\pgfqpoint{5.201297in}{3.102590in}}%
\pgfusepath{clip}%
\pgfsetbuttcap%
\pgfsetroundjoin%
\definecolor{currentfill}{rgb}{1.000000,0.498039,0.054902}%
\pgfsetfillcolor{currentfill}%
\pgfsetlinewidth{1.003750pt}%
\definecolor{currentstroke}{rgb}{1.000000,0.498039,0.054902}%
\pgfsetstrokecolor{currentstroke}%
\pgfsetdash{}{0pt}%
\pgfpathmoveto{\pgfqpoint{1.412855in}{3.145133in}}%
\pgfpathcurveto{\pgfqpoint{1.423905in}{3.145133in}}{\pgfqpoint{1.434504in}{3.149523in}}{\pgfqpoint{1.442318in}{3.157337in}}%
\pgfpathcurveto{\pgfqpoint{1.450131in}{3.165151in}}{\pgfqpoint{1.454521in}{3.175750in}}{\pgfqpoint{1.454521in}{3.186800in}}%
\pgfpathcurveto{\pgfqpoint{1.454521in}{3.197850in}}{\pgfqpoint{1.450131in}{3.208449in}}{\pgfqpoint{1.442318in}{3.216262in}}%
\pgfpathcurveto{\pgfqpoint{1.434504in}{3.224076in}}{\pgfqpoint{1.423905in}{3.228466in}}{\pgfqpoint{1.412855in}{3.228466in}}%
\pgfpathcurveto{\pgfqpoint{1.401805in}{3.228466in}}{\pgfqpoint{1.391206in}{3.224076in}}{\pgfqpoint{1.383392in}{3.216262in}}%
\pgfpathcurveto{\pgfqpoint{1.375578in}{3.208449in}}{\pgfqpoint{1.371188in}{3.197850in}}{\pgfqpoint{1.371188in}{3.186800in}}%
\pgfpathcurveto{\pgfqpoint{1.371188in}{3.175750in}}{\pgfqpoint{1.375578in}{3.165151in}}{\pgfqpoint{1.383392in}{3.157337in}}%
\pgfpathcurveto{\pgfqpoint{1.391206in}{3.149523in}}{\pgfqpoint{1.401805in}{3.145133in}}{\pgfqpoint{1.412855in}{3.145133in}}%
\pgfpathclose%
\pgfusepath{stroke,fill}%
\end{pgfscope}%
\begin{pgfscope}%
\pgfpathrectangle{\pgfqpoint{0.648703in}{0.548769in}}{\pgfqpoint{5.201297in}{3.102590in}}%
\pgfusepath{clip}%
\pgfsetbuttcap%
\pgfsetroundjoin%
\definecolor{currentfill}{rgb}{1.000000,0.498039,0.054902}%
\pgfsetfillcolor{currentfill}%
\pgfsetlinewidth{1.003750pt}%
\definecolor{currentstroke}{rgb}{1.000000,0.498039,0.054902}%
\pgfsetstrokecolor{currentstroke}%
\pgfsetdash{}{0pt}%
\pgfpathmoveto{\pgfqpoint{1.421994in}{3.157577in}}%
\pgfpathcurveto{\pgfqpoint{1.433044in}{3.157577in}}{\pgfqpoint{1.443643in}{3.161967in}}{\pgfqpoint{1.451457in}{3.169780in}}%
\pgfpathcurveto{\pgfqpoint{1.459270in}{3.177594in}}{\pgfqpoint{1.463661in}{3.188193in}}{\pgfqpoint{1.463661in}{3.199243in}}%
\pgfpathcurveto{\pgfqpoint{1.463661in}{3.210293in}}{\pgfqpoint{1.459270in}{3.220892in}}{\pgfqpoint{1.451457in}{3.228706in}}%
\pgfpathcurveto{\pgfqpoint{1.443643in}{3.236520in}}{\pgfqpoint{1.433044in}{3.240910in}}{\pgfqpoint{1.421994in}{3.240910in}}%
\pgfpathcurveto{\pgfqpoint{1.410944in}{3.240910in}}{\pgfqpoint{1.400345in}{3.236520in}}{\pgfqpoint{1.392531in}{3.228706in}}%
\pgfpathcurveto{\pgfqpoint{1.384717in}{3.220892in}}{\pgfqpoint{1.380327in}{3.210293in}}{\pgfqpoint{1.380327in}{3.199243in}}%
\pgfpathcurveto{\pgfqpoint{1.380327in}{3.188193in}}{\pgfqpoint{1.384717in}{3.177594in}}{\pgfqpoint{1.392531in}{3.169780in}}%
\pgfpathcurveto{\pgfqpoint{1.400345in}{3.161967in}}{\pgfqpoint{1.410944in}{3.157577in}}{\pgfqpoint{1.421994in}{3.157577in}}%
\pgfpathclose%
\pgfusepath{stroke,fill}%
\end{pgfscope}%
\begin{pgfscope}%
\pgfpathrectangle{\pgfqpoint{0.648703in}{0.548769in}}{\pgfqpoint{5.201297in}{3.102590in}}%
\pgfusepath{clip}%
\pgfsetbuttcap%
\pgfsetroundjoin%
\definecolor{currentfill}{rgb}{1.000000,0.498039,0.054902}%
\pgfsetfillcolor{currentfill}%
\pgfsetlinewidth{1.003750pt}%
\definecolor{currentstroke}{rgb}{1.000000,0.498039,0.054902}%
\pgfsetstrokecolor{currentstroke}%
\pgfsetdash{}{0pt}%
\pgfpathmoveto{\pgfqpoint{1.451893in}{3.323490in}}%
\pgfpathcurveto{\pgfqpoint{1.462943in}{3.323490in}}{\pgfqpoint{1.473542in}{3.327881in}}{\pgfqpoint{1.481355in}{3.335694in}}%
\pgfpathcurveto{\pgfqpoint{1.489169in}{3.343508in}}{\pgfqpoint{1.493559in}{3.354107in}}{\pgfqpoint{1.493559in}{3.365157in}}%
\pgfpathcurveto{\pgfqpoint{1.493559in}{3.376207in}}{\pgfqpoint{1.489169in}{3.386806in}}{\pgfqpoint{1.481355in}{3.394620in}}%
\pgfpathcurveto{\pgfqpoint{1.473542in}{3.402434in}}{\pgfqpoint{1.462943in}{3.406824in}}{\pgfqpoint{1.451893in}{3.406824in}}%
\pgfpathcurveto{\pgfqpoint{1.440843in}{3.406824in}}{\pgfqpoint{1.430243in}{3.402434in}}{\pgfqpoint{1.422430in}{3.394620in}}%
\pgfpathcurveto{\pgfqpoint{1.414616in}{3.386806in}}{\pgfqpoint{1.410226in}{3.376207in}}{\pgfqpoint{1.410226in}{3.365157in}}%
\pgfpathcurveto{\pgfqpoint{1.410226in}{3.354107in}}{\pgfqpoint{1.414616in}{3.343508in}}{\pgfqpoint{1.422430in}{3.335694in}}%
\pgfpathcurveto{\pgfqpoint{1.430243in}{3.327881in}}{\pgfqpoint{1.440843in}{3.323490in}}{\pgfqpoint{1.451893in}{3.323490in}}%
\pgfpathclose%
\pgfusepath{stroke,fill}%
\end{pgfscope}%
\begin{pgfscope}%
\pgfpathrectangle{\pgfqpoint{0.648703in}{0.548769in}}{\pgfqpoint{5.201297in}{3.102590in}}%
\pgfusepath{clip}%
\pgfsetbuttcap%
\pgfsetroundjoin%
\definecolor{currentfill}{rgb}{0.121569,0.466667,0.705882}%
\pgfsetfillcolor{currentfill}%
\pgfsetlinewidth{1.003750pt}%
\definecolor{currentstroke}{rgb}{0.121569,0.466667,0.705882}%
\pgfsetstrokecolor{currentstroke}%
\pgfsetdash{}{0pt}%
\pgfpathmoveto{\pgfqpoint{1.028932in}{2.410964in}}%
\pgfpathcurveto{\pgfqpoint{1.039982in}{2.410964in}}{\pgfqpoint{1.050581in}{2.415354in}}{\pgfqpoint{1.058395in}{2.423168in}}%
\pgfpathcurveto{\pgfqpoint{1.066209in}{2.430982in}}{\pgfqpoint{1.070599in}{2.441581in}}{\pgfqpoint{1.070599in}{2.452631in}}%
\pgfpathcurveto{\pgfqpoint{1.070599in}{2.463681in}}{\pgfqpoint{1.066209in}{2.474280in}}{\pgfqpoint{1.058395in}{2.482094in}}%
\pgfpathcurveto{\pgfqpoint{1.050581in}{2.489907in}}{\pgfqpoint{1.039982in}{2.494297in}}{\pgfqpoint{1.028932in}{2.494297in}}%
\pgfpathcurveto{\pgfqpoint{1.017882in}{2.494297in}}{\pgfqpoint{1.007283in}{2.489907in}}{\pgfqpoint{0.999469in}{2.482094in}}%
\pgfpathcurveto{\pgfqpoint{0.991656in}{2.474280in}}{\pgfqpoint{0.987266in}{2.463681in}}{\pgfqpoint{0.987266in}{2.452631in}}%
\pgfpathcurveto{\pgfqpoint{0.987266in}{2.441581in}}{\pgfqpoint{0.991656in}{2.430982in}}{\pgfqpoint{0.999469in}{2.423168in}}%
\pgfpathcurveto{\pgfqpoint{1.007283in}{2.415354in}}{\pgfqpoint{1.017882in}{2.410964in}}{\pgfqpoint{1.028932in}{2.410964in}}%
\pgfpathclose%
\pgfusepath{stroke,fill}%
\end{pgfscope}%
\begin{pgfscope}%
\pgfpathrectangle{\pgfqpoint{0.648703in}{0.548769in}}{\pgfqpoint{5.201297in}{3.102590in}}%
\pgfusepath{clip}%
\pgfsetbuttcap%
\pgfsetroundjoin%
\definecolor{currentfill}{rgb}{1.000000,0.498039,0.054902}%
\pgfsetfillcolor{currentfill}%
\pgfsetlinewidth{1.003750pt}%
\definecolor{currentstroke}{rgb}{1.000000,0.498039,0.054902}%
\pgfsetstrokecolor{currentstroke}%
\pgfsetdash{}{0pt}%
\pgfpathmoveto{\pgfqpoint{1.119175in}{3.199055in}}%
\pgfpathcurveto{\pgfqpoint{1.130225in}{3.199055in}}{\pgfqpoint{1.140824in}{3.203445in}}{\pgfqpoint{1.148638in}{3.211259in}}%
\pgfpathcurveto{\pgfqpoint{1.156451in}{3.219073in}}{\pgfqpoint{1.160841in}{3.229672in}}{\pgfqpoint{1.160841in}{3.240722in}}%
\pgfpathcurveto{\pgfqpoint{1.160841in}{3.251772in}}{\pgfqpoint{1.156451in}{3.262371in}}{\pgfqpoint{1.148638in}{3.270185in}}%
\pgfpathcurveto{\pgfqpoint{1.140824in}{3.277998in}}{\pgfqpoint{1.130225in}{3.282388in}}{\pgfqpoint{1.119175in}{3.282388in}}%
\pgfpathcurveto{\pgfqpoint{1.108125in}{3.282388in}}{\pgfqpoint{1.097526in}{3.277998in}}{\pgfqpoint{1.089712in}{3.270185in}}%
\pgfpathcurveto{\pgfqpoint{1.081898in}{3.262371in}}{\pgfqpoint{1.077508in}{3.251772in}}{\pgfqpoint{1.077508in}{3.240722in}}%
\pgfpathcurveto{\pgfqpoint{1.077508in}{3.229672in}}{\pgfqpoint{1.081898in}{3.219073in}}{\pgfqpoint{1.089712in}{3.211259in}}%
\pgfpathcurveto{\pgfqpoint{1.097526in}{3.203445in}}{\pgfqpoint{1.108125in}{3.199055in}}{\pgfqpoint{1.119175in}{3.199055in}}%
\pgfpathclose%
\pgfusepath{stroke,fill}%
\end{pgfscope}%
\begin{pgfscope}%
\pgfpathrectangle{\pgfqpoint{0.648703in}{0.548769in}}{\pgfqpoint{5.201297in}{3.102590in}}%
\pgfusepath{clip}%
\pgfsetbuttcap%
\pgfsetroundjoin%
\definecolor{currentfill}{rgb}{1.000000,0.498039,0.054902}%
\pgfsetfillcolor{currentfill}%
\pgfsetlinewidth{1.003750pt}%
\definecolor{currentstroke}{rgb}{1.000000,0.498039,0.054902}%
\pgfsetstrokecolor{currentstroke}%
\pgfsetdash{}{0pt}%
\pgfpathmoveto{\pgfqpoint{2.757983in}{3.124394in}}%
\pgfpathcurveto{\pgfqpoint{2.769033in}{3.124394in}}{\pgfqpoint{2.779632in}{3.128784in}}{\pgfqpoint{2.787446in}{3.136598in}}%
\pgfpathcurveto{\pgfqpoint{2.795259in}{3.144411in}}{\pgfqpoint{2.799650in}{3.155010in}}{\pgfqpoint{2.799650in}{3.166060in}}%
\pgfpathcurveto{\pgfqpoint{2.799650in}{3.177111in}}{\pgfqpoint{2.795259in}{3.187710in}}{\pgfqpoint{2.787446in}{3.195523in}}%
\pgfpathcurveto{\pgfqpoint{2.779632in}{3.203337in}}{\pgfqpoint{2.769033in}{3.207727in}}{\pgfqpoint{2.757983in}{3.207727in}}%
\pgfpathcurveto{\pgfqpoint{2.746933in}{3.207727in}}{\pgfqpoint{2.736334in}{3.203337in}}{\pgfqpoint{2.728520in}{3.195523in}}%
\pgfpathcurveto{\pgfqpoint{2.720707in}{3.187710in}}{\pgfqpoint{2.716316in}{3.177111in}}{\pgfqpoint{2.716316in}{3.166060in}}%
\pgfpathcurveto{\pgfqpoint{2.716316in}{3.155010in}}{\pgfqpoint{2.720707in}{3.144411in}}{\pgfqpoint{2.728520in}{3.136598in}}%
\pgfpathcurveto{\pgfqpoint{2.736334in}{3.128784in}}{\pgfqpoint{2.746933in}{3.124394in}}{\pgfqpoint{2.757983in}{3.124394in}}%
\pgfpathclose%
\pgfusepath{stroke,fill}%
\end{pgfscope}%
\begin{pgfscope}%
\pgfpathrectangle{\pgfqpoint{0.648703in}{0.548769in}}{\pgfqpoint{5.201297in}{3.102590in}}%
\pgfusepath{clip}%
\pgfsetbuttcap%
\pgfsetroundjoin%
\definecolor{currentfill}{rgb}{1.000000,0.498039,0.054902}%
\pgfsetfillcolor{currentfill}%
\pgfsetlinewidth{1.003750pt}%
\definecolor{currentstroke}{rgb}{1.000000,0.498039,0.054902}%
\pgfsetstrokecolor{currentstroke}%
\pgfsetdash{}{0pt}%
\pgfpathmoveto{\pgfqpoint{2.253603in}{3.132690in}}%
\pgfpathcurveto{\pgfqpoint{2.264653in}{3.132690in}}{\pgfqpoint{2.275252in}{3.137080in}}{\pgfqpoint{2.283065in}{3.144893in}}%
\pgfpathcurveto{\pgfqpoint{2.290879in}{3.152707in}}{\pgfqpoint{2.295269in}{3.163306in}}{\pgfqpoint{2.295269in}{3.174356in}}%
\pgfpathcurveto{\pgfqpoint{2.295269in}{3.185406in}}{\pgfqpoint{2.290879in}{3.196005in}}{\pgfqpoint{2.283065in}{3.203819in}}%
\pgfpathcurveto{\pgfqpoint{2.275252in}{3.211633in}}{\pgfqpoint{2.264653in}{3.216023in}}{\pgfqpoint{2.253603in}{3.216023in}}%
\pgfpathcurveto{\pgfqpoint{2.242552in}{3.216023in}}{\pgfqpoint{2.231953in}{3.211633in}}{\pgfqpoint{2.224140in}{3.203819in}}%
\pgfpathcurveto{\pgfqpoint{2.216326in}{3.196005in}}{\pgfqpoint{2.211936in}{3.185406in}}{\pgfqpoint{2.211936in}{3.174356in}}%
\pgfpathcurveto{\pgfqpoint{2.211936in}{3.163306in}}{\pgfqpoint{2.216326in}{3.152707in}}{\pgfqpoint{2.224140in}{3.144893in}}%
\pgfpathcurveto{\pgfqpoint{2.231953in}{3.137080in}}{\pgfqpoint{2.242552in}{3.132690in}}{\pgfqpoint{2.253603in}{3.132690in}}%
\pgfpathclose%
\pgfusepath{stroke,fill}%
\end{pgfscope}%
\begin{pgfscope}%
\pgfpathrectangle{\pgfqpoint{0.648703in}{0.548769in}}{\pgfqpoint{5.201297in}{3.102590in}}%
\pgfusepath{clip}%
\pgfsetbuttcap%
\pgfsetroundjoin%
\definecolor{currentfill}{rgb}{1.000000,0.498039,0.054902}%
\pgfsetfillcolor{currentfill}%
\pgfsetlinewidth{1.003750pt}%
\definecolor{currentstroke}{rgb}{1.000000,0.498039,0.054902}%
\pgfsetstrokecolor{currentstroke}%
\pgfsetdash{}{0pt}%
\pgfpathmoveto{\pgfqpoint{1.672266in}{3.132690in}}%
\pgfpathcurveto{\pgfqpoint{1.683317in}{3.132690in}}{\pgfqpoint{1.693916in}{3.137080in}}{\pgfqpoint{1.701729in}{3.144893in}}%
\pgfpathcurveto{\pgfqpoint{1.709543in}{3.152707in}}{\pgfqpoint{1.713933in}{3.163306in}}{\pgfqpoint{1.713933in}{3.174356in}}%
\pgfpathcurveto{\pgfqpoint{1.713933in}{3.185406in}}{\pgfqpoint{1.709543in}{3.196005in}}{\pgfqpoint{1.701729in}{3.203819in}}%
\pgfpathcurveto{\pgfqpoint{1.693916in}{3.211633in}}{\pgfqpoint{1.683317in}{3.216023in}}{\pgfqpoint{1.672266in}{3.216023in}}%
\pgfpathcurveto{\pgfqpoint{1.661216in}{3.216023in}}{\pgfqpoint{1.650617in}{3.211633in}}{\pgfqpoint{1.642804in}{3.203819in}}%
\pgfpathcurveto{\pgfqpoint{1.634990in}{3.196005in}}{\pgfqpoint{1.630600in}{3.185406in}}{\pgfqpoint{1.630600in}{3.174356in}}%
\pgfpathcurveto{\pgfqpoint{1.630600in}{3.163306in}}{\pgfqpoint{1.634990in}{3.152707in}}{\pgfqpoint{1.642804in}{3.144893in}}%
\pgfpathcurveto{\pgfqpoint{1.650617in}{3.137080in}}{\pgfqpoint{1.661216in}{3.132690in}}{\pgfqpoint{1.672266in}{3.132690in}}%
\pgfpathclose%
\pgfusepath{stroke,fill}%
\end{pgfscope}%
\begin{pgfscope}%
\pgfpathrectangle{\pgfqpoint{0.648703in}{0.548769in}}{\pgfqpoint{5.201297in}{3.102590in}}%
\pgfusepath{clip}%
\pgfsetbuttcap%
\pgfsetroundjoin%
\definecolor{currentfill}{rgb}{1.000000,0.498039,0.054902}%
\pgfsetfillcolor{currentfill}%
\pgfsetlinewidth{1.003750pt}%
\definecolor{currentstroke}{rgb}{1.000000,0.498039,0.054902}%
\pgfsetstrokecolor{currentstroke}%
\pgfsetdash{}{0pt}%
\pgfpathmoveto{\pgfqpoint{1.640242in}{3.136837in}}%
\pgfpathcurveto{\pgfqpoint{1.651292in}{3.136837in}}{\pgfqpoint{1.661891in}{3.141228in}}{\pgfqpoint{1.669705in}{3.149041in}}%
\pgfpathcurveto{\pgfqpoint{1.677519in}{3.156855in}}{\pgfqpoint{1.681909in}{3.167454in}}{\pgfqpoint{1.681909in}{3.178504in}}%
\pgfpathcurveto{\pgfqpoint{1.681909in}{3.189554in}}{\pgfqpoint{1.677519in}{3.200153in}}{\pgfqpoint{1.669705in}{3.207967in}}%
\pgfpathcurveto{\pgfqpoint{1.661891in}{3.215780in}}{\pgfqpoint{1.651292in}{3.220171in}}{\pgfqpoint{1.640242in}{3.220171in}}%
\pgfpathcurveto{\pgfqpoint{1.629192in}{3.220171in}}{\pgfqpoint{1.618593in}{3.215780in}}{\pgfqpoint{1.610779in}{3.207967in}}%
\pgfpathcurveto{\pgfqpoint{1.602966in}{3.200153in}}{\pgfqpoint{1.598576in}{3.189554in}}{\pgfqpoint{1.598576in}{3.178504in}}%
\pgfpathcurveto{\pgfqpoint{1.598576in}{3.167454in}}{\pgfqpoint{1.602966in}{3.156855in}}{\pgfqpoint{1.610779in}{3.149041in}}%
\pgfpathcurveto{\pgfqpoint{1.618593in}{3.141228in}}{\pgfqpoint{1.629192in}{3.136837in}}{\pgfqpoint{1.640242in}{3.136837in}}%
\pgfpathclose%
\pgfusepath{stroke,fill}%
\end{pgfscope}%
\begin{pgfscope}%
\pgfpathrectangle{\pgfqpoint{0.648703in}{0.548769in}}{\pgfqpoint{5.201297in}{3.102590in}}%
\pgfusepath{clip}%
\pgfsetbuttcap%
\pgfsetroundjoin%
\definecolor{currentfill}{rgb}{1.000000,0.498039,0.054902}%
\pgfsetfillcolor{currentfill}%
\pgfsetlinewidth{1.003750pt}%
\definecolor{currentstroke}{rgb}{1.000000,0.498039,0.054902}%
\pgfsetstrokecolor{currentstroke}%
\pgfsetdash{}{0pt}%
\pgfpathmoveto{\pgfqpoint{1.082813in}{3.244681in}}%
\pgfpathcurveto{\pgfqpoint{1.093863in}{3.244681in}}{\pgfqpoint{1.104462in}{3.249072in}}{\pgfqpoint{1.112275in}{3.256885in}}%
\pgfpathcurveto{\pgfqpoint{1.120089in}{3.264699in}}{\pgfqpoint{1.124479in}{3.275298in}}{\pgfqpoint{1.124479in}{3.286348in}}%
\pgfpathcurveto{\pgfqpoint{1.124479in}{3.297398in}}{\pgfqpoint{1.120089in}{3.307997in}}{\pgfqpoint{1.112275in}{3.315811in}}%
\pgfpathcurveto{\pgfqpoint{1.104462in}{3.323624in}}{\pgfqpoint{1.093863in}{3.328015in}}{\pgfqpoint{1.082813in}{3.328015in}}%
\pgfpathcurveto{\pgfqpoint{1.071762in}{3.328015in}}{\pgfqpoint{1.061163in}{3.323624in}}{\pgfqpoint{1.053350in}{3.315811in}}%
\pgfpathcurveto{\pgfqpoint{1.045536in}{3.307997in}}{\pgfqpoint{1.041146in}{3.297398in}}{\pgfqpoint{1.041146in}{3.286348in}}%
\pgfpathcurveto{\pgfqpoint{1.041146in}{3.275298in}}{\pgfqpoint{1.045536in}{3.264699in}}{\pgfqpoint{1.053350in}{3.256885in}}%
\pgfpathcurveto{\pgfqpoint{1.061163in}{3.249072in}}{\pgfqpoint{1.071762in}{3.244681in}}{\pgfqpoint{1.082813in}{3.244681in}}%
\pgfpathclose%
\pgfusepath{stroke,fill}%
\end{pgfscope}%
\begin{pgfscope}%
\pgfpathrectangle{\pgfqpoint{0.648703in}{0.548769in}}{\pgfqpoint{5.201297in}{3.102590in}}%
\pgfusepath{clip}%
\pgfsetbuttcap%
\pgfsetroundjoin%
\definecolor{currentfill}{rgb}{1.000000,0.498039,0.054902}%
\pgfsetfillcolor{currentfill}%
\pgfsetlinewidth{1.003750pt}%
\definecolor{currentstroke}{rgb}{1.000000,0.498039,0.054902}%
\pgfsetstrokecolor{currentstroke}%
\pgfsetdash{}{0pt}%
\pgfpathmoveto{\pgfqpoint{1.371502in}{3.190759in}}%
\pgfpathcurveto{\pgfqpoint{1.382552in}{3.190759in}}{\pgfqpoint{1.393151in}{3.195150in}}{\pgfqpoint{1.400964in}{3.202963in}}%
\pgfpathcurveto{\pgfqpoint{1.408778in}{3.210777in}}{\pgfqpoint{1.413168in}{3.221376in}}{\pgfqpoint{1.413168in}{3.232426in}}%
\pgfpathcurveto{\pgfqpoint{1.413168in}{3.243476in}}{\pgfqpoint{1.408778in}{3.254075in}}{\pgfqpoint{1.400964in}{3.261889in}}%
\pgfpathcurveto{\pgfqpoint{1.393151in}{3.269702in}}{\pgfqpoint{1.382552in}{3.274093in}}{\pgfqpoint{1.371502in}{3.274093in}}%
\pgfpathcurveto{\pgfqpoint{1.360451in}{3.274093in}}{\pgfqpoint{1.349852in}{3.269702in}}{\pgfqpoint{1.342039in}{3.261889in}}%
\pgfpathcurveto{\pgfqpoint{1.334225in}{3.254075in}}{\pgfqpoint{1.329835in}{3.243476in}}{\pgfqpoint{1.329835in}{3.232426in}}%
\pgfpathcurveto{\pgfqpoint{1.329835in}{3.221376in}}{\pgfqpoint{1.334225in}{3.210777in}}{\pgfqpoint{1.342039in}{3.202963in}}%
\pgfpathcurveto{\pgfqpoint{1.349852in}{3.195150in}}{\pgfqpoint{1.360451in}{3.190759in}}{\pgfqpoint{1.371502in}{3.190759in}}%
\pgfpathclose%
\pgfusepath{stroke,fill}%
\end{pgfscope}%
\begin{pgfscope}%
\pgfpathrectangle{\pgfqpoint{0.648703in}{0.548769in}}{\pgfqpoint{5.201297in}{3.102590in}}%
\pgfusepath{clip}%
\pgfsetbuttcap%
\pgfsetroundjoin%
\definecolor{currentfill}{rgb}{0.121569,0.466667,0.705882}%
\pgfsetfillcolor{currentfill}%
\pgfsetlinewidth{1.003750pt}%
\definecolor{currentstroke}{rgb}{0.121569,0.466667,0.705882}%
\pgfsetstrokecolor{currentstroke}%
\pgfsetdash{}{0pt}%
\pgfpathmoveto{\pgfqpoint{0.935717in}{0.664720in}}%
\pgfpathcurveto{\pgfqpoint{0.946767in}{0.664720in}}{\pgfqpoint{0.957366in}{0.669111in}}{\pgfqpoint{0.965180in}{0.676924in}}%
\pgfpathcurveto{\pgfqpoint{0.972994in}{0.684738in}}{\pgfqpoint{0.977384in}{0.695337in}}{\pgfqpoint{0.977384in}{0.706387in}}%
\pgfpathcurveto{\pgfqpoint{0.977384in}{0.717437in}}{\pgfqpoint{0.972994in}{0.728036in}}{\pgfqpoint{0.965180in}{0.735850in}}%
\pgfpathcurveto{\pgfqpoint{0.957366in}{0.743663in}}{\pgfqpoint{0.946767in}{0.748054in}}{\pgfqpoint{0.935717in}{0.748054in}}%
\pgfpathcurveto{\pgfqpoint{0.924667in}{0.748054in}}{\pgfqpoint{0.914068in}{0.743663in}}{\pgfqpoint{0.906254in}{0.735850in}}%
\pgfpathcurveto{\pgfqpoint{0.898441in}{0.728036in}}{\pgfqpoint{0.894051in}{0.717437in}}{\pgfqpoint{0.894051in}{0.706387in}}%
\pgfpathcurveto{\pgfqpoint{0.894051in}{0.695337in}}{\pgfqpoint{0.898441in}{0.684738in}}{\pgfqpoint{0.906254in}{0.676924in}}%
\pgfpathcurveto{\pgfqpoint{0.914068in}{0.669111in}}{\pgfqpoint{0.924667in}{0.664720in}}{\pgfqpoint{0.935717in}{0.664720in}}%
\pgfpathclose%
\pgfusepath{stroke,fill}%
\end{pgfscope}%
\begin{pgfscope}%
\pgfpathrectangle{\pgfqpoint{0.648703in}{0.548769in}}{\pgfqpoint{5.201297in}{3.102590in}}%
\pgfusepath{clip}%
\pgfsetbuttcap%
\pgfsetroundjoin%
\definecolor{currentfill}{rgb}{1.000000,0.498039,0.054902}%
\pgfsetfillcolor{currentfill}%
\pgfsetlinewidth{1.003750pt}%
\definecolor{currentstroke}{rgb}{1.000000,0.498039,0.054902}%
\pgfsetstrokecolor{currentstroke}%
\pgfsetdash{}{0pt}%
\pgfpathmoveto{\pgfqpoint{1.466466in}{3.145133in}}%
\pgfpathcurveto{\pgfqpoint{1.477516in}{3.145133in}}{\pgfqpoint{1.488115in}{3.149523in}}{\pgfqpoint{1.495929in}{3.157337in}}%
\pgfpathcurveto{\pgfqpoint{1.503742in}{3.165151in}}{\pgfqpoint{1.508133in}{3.175750in}}{\pgfqpoint{1.508133in}{3.186800in}}%
\pgfpathcurveto{\pgfqpoint{1.508133in}{3.197850in}}{\pgfqpoint{1.503742in}{3.208449in}}{\pgfqpoint{1.495929in}{3.216262in}}%
\pgfpathcurveto{\pgfqpoint{1.488115in}{3.224076in}}{\pgfqpoint{1.477516in}{3.228466in}}{\pgfqpoint{1.466466in}{3.228466in}}%
\pgfpathcurveto{\pgfqpoint{1.455416in}{3.228466in}}{\pgfqpoint{1.444817in}{3.224076in}}{\pgfqpoint{1.437003in}{3.216262in}}%
\pgfpathcurveto{\pgfqpoint{1.429190in}{3.208449in}}{\pgfqpoint{1.424799in}{3.197850in}}{\pgfqpoint{1.424799in}{3.186800in}}%
\pgfpathcurveto{\pgfqpoint{1.424799in}{3.175750in}}{\pgfqpoint{1.429190in}{3.165151in}}{\pgfqpoint{1.437003in}{3.157337in}}%
\pgfpathcurveto{\pgfqpoint{1.444817in}{3.149523in}}{\pgfqpoint{1.455416in}{3.145133in}}{\pgfqpoint{1.466466in}{3.145133in}}%
\pgfpathclose%
\pgfusepath{stroke,fill}%
\end{pgfscope}%
\begin{pgfscope}%
\pgfpathrectangle{\pgfqpoint{0.648703in}{0.548769in}}{\pgfqpoint{5.201297in}{3.102590in}}%
\pgfusepath{clip}%
\pgfsetbuttcap%
\pgfsetroundjoin%
\definecolor{currentfill}{rgb}{0.121569,0.466667,0.705882}%
\pgfsetfillcolor{currentfill}%
\pgfsetlinewidth{1.003750pt}%
\definecolor{currentstroke}{rgb}{0.121569,0.466667,0.705882}%
\pgfsetstrokecolor{currentstroke}%
\pgfsetdash{}{0pt}%
\pgfpathmoveto{\pgfqpoint{0.965351in}{0.648129in}}%
\pgfpathcurveto{\pgfqpoint{0.976401in}{0.648129in}}{\pgfqpoint{0.987000in}{0.652519in}}{\pgfqpoint{0.994814in}{0.660333in}}%
\pgfpathcurveto{\pgfqpoint{1.002627in}{0.668146in}}{\pgfqpoint{1.007017in}{0.678745in}}{\pgfqpoint{1.007017in}{0.689796in}}%
\pgfpathcurveto{\pgfqpoint{1.007017in}{0.700846in}}{\pgfqpoint{1.002627in}{0.711445in}}{\pgfqpoint{0.994814in}{0.719258in}}%
\pgfpathcurveto{\pgfqpoint{0.987000in}{0.727072in}}{\pgfqpoint{0.976401in}{0.731462in}}{\pgfqpoint{0.965351in}{0.731462in}}%
\pgfpathcurveto{\pgfqpoint{0.954301in}{0.731462in}}{\pgfqpoint{0.943702in}{0.727072in}}{\pgfqpoint{0.935888in}{0.719258in}}%
\pgfpathcurveto{\pgfqpoint{0.928074in}{0.711445in}}{\pgfqpoint{0.923684in}{0.700846in}}{\pgfqpoint{0.923684in}{0.689796in}}%
\pgfpathcurveto{\pgfqpoint{0.923684in}{0.678745in}}{\pgfqpoint{0.928074in}{0.668146in}}{\pgfqpoint{0.935888in}{0.660333in}}%
\pgfpathcurveto{\pgfqpoint{0.943702in}{0.652519in}}{\pgfqpoint{0.954301in}{0.648129in}}{\pgfqpoint{0.965351in}{0.648129in}}%
\pgfpathclose%
\pgfusepath{stroke,fill}%
\end{pgfscope}%
\begin{pgfscope}%
\pgfpathrectangle{\pgfqpoint{0.648703in}{0.548769in}}{\pgfqpoint{5.201297in}{3.102590in}}%
\pgfusepath{clip}%
\pgfsetbuttcap%
\pgfsetroundjoin%
\definecolor{currentfill}{rgb}{1.000000,0.498039,0.054902}%
\pgfsetfillcolor{currentfill}%
\pgfsetlinewidth{1.003750pt}%
\definecolor{currentstroke}{rgb}{1.000000,0.498039,0.054902}%
\pgfsetstrokecolor{currentstroke}%
\pgfsetdash{}{0pt}%
\pgfpathmoveto{\pgfqpoint{1.814727in}{3.149281in}}%
\pgfpathcurveto{\pgfqpoint{1.825777in}{3.149281in}}{\pgfqpoint{1.836376in}{3.153671in}}{\pgfqpoint{1.844190in}{3.161485in}}%
\pgfpathcurveto{\pgfqpoint{1.852003in}{3.169298in}}{\pgfqpoint{1.856394in}{3.179897in}}{\pgfqpoint{1.856394in}{3.190948in}}%
\pgfpathcurveto{\pgfqpoint{1.856394in}{3.201998in}}{\pgfqpoint{1.852003in}{3.212597in}}{\pgfqpoint{1.844190in}{3.220410in}}%
\pgfpathcurveto{\pgfqpoint{1.836376in}{3.228224in}}{\pgfqpoint{1.825777in}{3.232614in}}{\pgfqpoint{1.814727in}{3.232614in}}%
\pgfpathcurveto{\pgfqpoint{1.803677in}{3.232614in}}{\pgfqpoint{1.793078in}{3.228224in}}{\pgfqpoint{1.785264in}{3.220410in}}%
\pgfpathcurveto{\pgfqpoint{1.777451in}{3.212597in}}{\pgfqpoint{1.773060in}{3.201998in}}{\pgfqpoint{1.773060in}{3.190948in}}%
\pgfpathcurveto{\pgfqpoint{1.773060in}{3.179897in}}{\pgfqpoint{1.777451in}{3.169298in}}{\pgfqpoint{1.785264in}{3.161485in}}%
\pgfpathcurveto{\pgfqpoint{1.793078in}{3.153671in}}{\pgfqpoint{1.803677in}{3.149281in}}{\pgfqpoint{1.814727in}{3.149281in}}%
\pgfpathclose%
\pgfusepath{stroke,fill}%
\end{pgfscope}%
\begin{pgfscope}%
\pgfpathrectangle{\pgfqpoint{0.648703in}{0.548769in}}{\pgfqpoint{5.201297in}{3.102590in}}%
\pgfusepath{clip}%
\pgfsetbuttcap%
\pgfsetroundjoin%
\definecolor{currentfill}{rgb}{1.000000,0.498039,0.054902}%
\pgfsetfillcolor{currentfill}%
\pgfsetlinewidth{1.003750pt}%
\definecolor{currentstroke}{rgb}{1.000000,0.498039,0.054902}%
\pgfsetstrokecolor{currentstroke}%
\pgfsetdash{}{0pt}%
\pgfpathmoveto{\pgfqpoint{2.320881in}{3.124394in}}%
\pgfpathcurveto{\pgfqpoint{2.331931in}{3.124394in}}{\pgfqpoint{2.342530in}{3.128784in}}{\pgfqpoint{2.350344in}{3.136598in}}%
\pgfpathcurveto{\pgfqpoint{2.358157in}{3.144411in}}{\pgfqpoint{2.362547in}{3.155010in}}{\pgfqpoint{2.362547in}{3.166060in}}%
\pgfpathcurveto{\pgfqpoint{2.362547in}{3.177111in}}{\pgfqpoint{2.358157in}{3.187710in}}{\pgfqpoint{2.350344in}{3.195523in}}%
\pgfpathcurveto{\pgfqpoint{2.342530in}{3.203337in}}{\pgfqpoint{2.331931in}{3.207727in}}{\pgfqpoint{2.320881in}{3.207727in}}%
\pgfpathcurveto{\pgfqpoint{2.309831in}{3.207727in}}{\pgfqpoint{2.299232in}{3.203337in}}{\pgfqpoint{2.291418in}{3.195523in}}%
\pgfpathcurveto{\pgfqpoint{2.283604in}{3.187710in}}{\pgfqpoint{2.279214in}{3.177111in}}{\pgfqpoint{2.279214in}{3.166060in}}%
\pgfpathcurveto{\pgfqpoint{2.279214in}{3.155010in}}{\pgfqpoint{2.283604in}{3.144411in}}{\pgfqpoint{2.291418in}{3.136598in}}%
\pgfpathcurveto{\pgfqpoint{2.299232in}{3.128784in}}{\pgfqpoint{2.309831in}{3.124394in}}{\pgfqpoint{2.320881in}{3.124394in}}%
\pgfpathclose%
\pgfusepath{stroke,fill}%
\end{pgfscope}%
\begin{pgfscope}%
\pgfpathrectangle{\pgfqpoint{0.648703in}{0.548769in}}{\pgfqpoint{5.201297in}{3.102590in}}%
\pgfusepath{clip}%
\pgfsetbuttcap%
\pgfsetroundjoin%
\definecolor{currentfill}{rgb}{1.000000,0.498039,0.054902}%
\pgfsetfillcolor{currentfill}%
\pgfsetlinewidth{1.003750pt}%
\definecolor{currentstroke}{rgb}{1.000000,0.498039,0.054902}%
\pgfsetstrokecolor{currentstroke}%
\pgfsetdash{}{0pt}%
\pgfpathmoveto{\pgfqpoint{1.707762in}{3.136837in}}%
\pgfpathcurveto{\pgfqpoint{1.718812in}{3.136837in}}{\pgfqpoint{1.729411in}{3.141228in}}{\pgfqpoint{1.737225in}{3.149041in}}%
\pgfpathcurveto{\pgfqpoint{1.745038in}{3.156855in}}{\pgfqpoint{1.749429in}{3.167454in}}{\pgfqpoint{1.749429in}{3.178504in}}%
\pgfpathcurveto{\pgfqpoint{1.749429in}{3.189554in}}{\pgfqpoint{1.745038in}{3.200153in}}{\pgfqpoint{1.737225in}{3.207967in}}%
\pgfpathcurveto{\pgfqpoint{1.729411in}{3.215780in}}{\pgfqpoint{1.718812in}{3.220171in}}{\pgfqpoint{1.707762in}{3.220171in}}%
\pgfpathcurveto{\pgfqpoint{1.696712in}{3.220171in}}{\pgfqpoint{1.686113in}{3.215780in}}{\pgfqpoint{1.678299in}{3.207967in}}%
\pgfpathcurveto{\pgfqpoint{1.670485in}{3.200153in}}{\pgfqpoint{1.666095in}{3.189554in}}{\pgfqpoint{1.666095in}{3.178504in}}%
\pgfpathcurveto{\pgfqpoint{1.666095in}{3.167454in}}{\pgfqpoint{1.670485in}{3.156855in}}{\pgfqpoint{1.678299in}{3.149041in}}%
\pgfpathcurveto{\pgfqpoint{1.686113in}{3.141228in}}{\pgfqpoint{1.696712in}{3.136837in}}{\pgfqpoint{1.707762in}{3.136837in}}%
\pgfpathclose%
\pgfusepath{stroke,fill}%
\end{pgfscope}%
\begin{pgfscope}%
\pgfpathrectangle{\pgfqpoint{0.648703in}{0.548769in}}{\pgfqpoint{5.201297in}{3.102590in}}%
\pgfusepath{clip}%
\pgfsetbuttcap%
\pgfsetroundjoin%
\definecolor{currentfill}{rgb}{1.000000,0.498039,0.054902}%
\pgfsetfillcolor{currentfill}%
\pgfsetlinewidth{1.003750pt}%
\definecolor{currentstroke}{rgb}{1.000000,0.498039,0.054902}%
\pgfsetstrokecolor{currentstroke}%
\pgfsetdash{}{0pt}%
\pgfpathmoveto{\pgfqpoint{1.489541in}{3.140985in}}%
\pgfpathcurveto{\pgfqpoint{1.500591in}{3.140985in}}{\pgfqpoint{1.511190in}{3.145375in}}{\pgfqpoint{1.519004in}{3.153189in}}%
\pgfpathcurveto{\pgfqpoint{1.526818in}{3.161003in}}{\pgfqpoint{1.531208in}{3.171602in}}{\pgfqpoint{1.531208in}{3.182652in}}%
\pgfpathcurveto{\pgfqpoint{1.531208in}{3.193702in}}{\pgfqpoint{1.526818in}{3.204301in}}{\pgfqpoint{1.519004in}{3.212115in}}%
\pgfpathcurveto{\pgfqpoint{1.511190in}{3.219928in}}{\pgfqpoint{1.500591in}{3.224319in}}{\pgfqpoint{1.489541in}{3.224319in}}%
\pgfpathcurveto{\pgfqpoint{1.478491in}{3.224319in}}{\pgfqpoint{1.467892in}{3.219928in}}{\pgfqpoint{1.460078in}{3.212115in}}%
\pgfpathcurveto{\pgfqpoint{1.452265in}{3.204301in}}{\pgfqpoint{1.447875in}{3.193702in}}{\pgfqpoint{1.447875in}{3.182652in}}%
\pgfpathcurveto{\pgfqpoint{1.447875in}{3.171602in}}{\pgfqpoint{1.452265in}{3.161003in}}{\pgfqpoint{1.460078in}{3.153189in}}%
\pgfpathcurveto{\pgfqpoint{1.467892in}{3.145375in}}{\pgfqpoint{1.478491in}{3.140985in}}{\pgfqpoint{1.489541in}{3.140985in}}%
\pgfpathclose%
\pgfusepath{stroke,fill}%
\end{pgfscope}%
\begin{pgfscope}%
\pgfpathrectangle{\pgfqpoint{0.648703in}{0.548769in}}{\pgfqpoint{5.201297in}{3.102590in}}%
\pgfusepath{clip}%
\pgfsetbuttcap%
\pgfsetroundjoin%
\definecolor{currentfill}{rgb}{1.000000,0.498039,0.054902}%
\pgfsetfillcolor{currentfill}%
\pgfsetlinewidth{1.003750pt}%
\definecolor{currentstroke}{rgb}{1.000000,0.498039,0.054902}%
\pgfsetstrokecolor{currentstroke}%
\pgfsetdash{}{0pt}%
\pgfpathmoveto{\pgfqpoint{1.191381in}{3.145133in}}%
\pgfpathcurveto{\pgfqpoint{1.202431in}{3.145133in}}{\pgfqpoint{1.213030in}{3.149523in}}{\pgfqpoint{1.220843in}{3.157337in}}%
\pgfpathcurveto{\pgfqpoint{1.228657in}{3.165151in}}{\pgfqpoint{1.233047in}{3.175750in}}{\pgfqpoint{1.233047in}{3.186800in}}%
\pgfpathcurveto{\pgfqpoint{1.233047in}{3.197850in}}{\pgfqpoint{1.228657in}{3.208449in}}{\pgfqpoint{1.220843in}{3.216262in}}%
\pgfpathcurveto{\pgfqpoint{1.213030in}{3.224076in}}{\pgfqpoint{1.202431in}{3.228466in}}{\pgfqpoint{1.191381in}{3.228466in}}%
\pgfpathcurveto{\pgfqpoint{1.180331in}{3.228466in}}{\pgfqpoint{1.169731in}{3.224076in}}{\pgfqpoint{1.161918in}{3.216262in}}%
\pgfpathcurveto{\pgfqpoint{1.154104in}{3.208449in}}{\pgfqpoint{1.149714in}{3.197850in}}{\pgfqpoint{1.149714in}{3.186800in}}%
\pgfpathcurveto{\pgfqpoint{1.149714in}{3.175750in}}{\pgfqpoint{1.154104in}{3.165151in}}{\pgfqpoint{1.161918in}{3.157337in}}%
\pgfpathcurveto{\pgfqpoint{1.169731in}{3.149523in}}{\pgfqpoint{1.180331in}{3.145133in}}{\pgfqpoint{1.191381in}{3.145133in}}%
\pgfpathclose%
\pgfusepath{stroke,fill}%
\end{pgfscope}%
\begin{pgfscope}%
\pgfpathrectangle{\pgfqpoint{0.648703in}{0.548769in}}{\pgfqpoint{5.201297in}{3.102590in}}%
\pgfusepath{clip}%
\pgfsetbuttcap%
\pgfsetroundjoin%
\definecolor{currentfill}{rgb}{1.000000,0.498039,0.054902}%
\pgfsetfillcolor{currentfill}%
\pgfsetlinewidth{1.003750pt}%
\definecolor{currentstroke}{rgb}{1.000000,0.498039,0.054902}%
\pgfsetstrokecolor{currentstroke}%
\pgfsetdash{}{0pt}%
\pgfpathmoveto{\pgfqpoint{1.445275in}{3.132690in}}%
\pgfpathcurveto{\pgfqpoint{1.456325in}{3.132690in}}{\pgfqpoint{1.466924in}{3.137080in}}{\pgfqpoint{1.474738in}{3.144893in}}%
\pgfpathcurveto{\pgfqpoint{1.482551in}{3.152707in}}{\pgfqpoint{1.486942in}{3.163306in}}{\pgfqpoint{1.486942in}{3.174356in}}%
\pgfpathcurveto{\pgfqpoint{1.486942in}{3.185406in}}{\pgfqpoint{1.482551in}{3.196005in}}{\pgfqpoint{1.474738in}{3.203819in}}%
\pgfpathcurveto{\pgfqpoint{1.466924in}{3.211633in}}{\pgfqpoint{1.456325in}{3.216023in}}{\pgfqpoint{1.445275in}{3.216023in}}%
\pgfpathcurveto{\pgfqpoint{1.434225in}{3.216023in}}{\pgfqpoint{1.423626in}{3.211633in}}{\pgfqpoint{1.415812in}{3.203819in}}%
\pgfpathcurveto{\pgfqpoint{1.407998in}{3.196005in}}{\pgfqpoint{1.403608in}{3.185406in}}{\pgfqpoint{1.403608in}{3.174356in}}%
\pgfpathcurveto{\pgfqpoint{1.403608in}{3.163306in}}{\pgfqpoint{1.407998in}{3.152707in}}{\pgfqpoint{1.415812in}{3.144893in}}%
\pgfpathcurveto{\pgfqpoint{1.423626in}{3.137080in}}{\pgfqpoint{1.434225in}{3.132690in}}{\pgfqpoint{1.445275in}{3.132690in}}%
\pgfpathclose%
\pgfusepath{stroke,fill}%
\end{pgfscope}%
\begin{pgfscope}%
\pgfpathrectangle{\pgfqpoint{0.648703in}{0.548769in}}{\pgfqpoint{5.201297in}{3.102590in}}%
\pgfusepath{clip}%
\pgfsetbuttcap%
\pgfsetroundjoin%
\definecolor{currentfill}{rgb}{0.121569,0.466667,0.705882}%
\pgfsetfillcolor{currentfill}%
\pgfsetlinewidth{1.003750pt}%
\definecolor{currentstroke}{rgb}{0.121569,0.466667,0.705882}%
\pgfsetstrokecolor{currentstroke}%
\pgfsetdash{}{0pt}%
\pgfpathmoveto{\pgfqpoint{1.065192in}{0.660572in}}%
\pgfpathcurveto{\pgfqpoint{1.076242in}{0.660572in}}{\pgfqpoint{1.086841in}{0.664963in}}{\pgfqpoint{1.094654in}{0.672776in}}%
\pgfpathcurveto{\pgfqpoint{1.102468in}{0.680590in}}{\pgfqpoint{1.106858in}{0.691189in}}{\pgfqpoint{1.106858in}{0.702239in}}%
\pgfpathcurveto{\pgfqpoint{1.106858in}{0.713289in}}{\pgfqpoint{1.102468in}{0.723888in}}{\pgfqpoint{1.094654in}{0.731702in}}%
\pgfpathcurveto{\pgfqpoint{1.086841in}{0.739516in}}{\pgfqpoint{1.076242in}{0.743906in}}{\pgfqpoint{1.065192in}{0.743906in}}%
\pgfpathcurveto{\pgfqpoint{1.054141in}{0.743906in}}{\pgfqpoint{1.043542in}{0.739516in}}{\pgfqpoint{1.035729in}{0.731702in}}%
\pgfpathcurveto{\pgfqpoint{1.027915in}{0.723888in}}{\pgfqpoint{1.023525in}{0.713289in}}{\pgfqpoint{1.023525in}{0.702239in}}%
\pgfpathcurveto{\pgfqpoint{1.023525in}{0.691189in}}{\pgfqpoint{1.027915in}{0.680590in}}{\pgfqpoint{1.035729in}{0.672776in}}%
\pgfpathcurveto{\pgfqpoint{1.043542in}{0.664963in}}{\pgfqpoint{1.054141in}{0.660572in}}{\pgfqpoint{1.065192in}{0.660572in}}%
\pgfpathclose%
\pgfusepath{stroke,fill}%
\end{pgfscope}%
\begin{pgfscope}%
\pgfpathrectangle{\pgfqpoint{0.648703in}{0.548769in}}{\pgfqpoint{5.201297in}{3.102590in}}%
\pgfusepath{clip}%
\pgfsetbuttcap%
\pgfsetroundjoin%
\definecolor{currentfill}{rgb}{1.000000,0.498039,0.054902}%
\pgfsetfillcolor{currentfill}%
\pgfsetlinewidth{1.003750pt}%
\definecolor{currentstroke}{rgb}{1.000000,0.498039,0.054902}%
\pgfsetstrokecolor{currentstroke}%
\pgfsetdash{}{0pt}%
\pgfpathmoveto{\pgfqpoint{1.137639in}{3.136837in}}%
\pgfpathcurveto{\pgfqpoint{1.148689in}{3.136837in}}{\pgfqpoint{1.159288in}{3.141228in}}{\pgfqpoint{1.167102in}{3.149041in}}%
\pgfpathcurveto{\pgfqpoint{1.174915in}{3.156855in}}{\pgfqpoint{1.179306in}{3.167454in}}{\pgfqpoint{1.179306in}{3.178504in}}%
\pgfpathcurveto{\pgfqpoint{1.179306in}{3.189554in}}{\pgfqpoint{1.174915in}{3.200153in}}{\pgfqpoint{1.167102in}{3.207967in}}%
\pgfpathcurveto{\pgfqpoint{1.159288in}{3.215780in}}{\pgfqpoint{1.148689in}{3.220171in}}{\pgfqpoint{1.137639in}{3.220171in}}%
\pgfpathcurveto{\pgfqpoint{1.126589in}{3.220171in}}{\pgfqpoint{1.115990in}{3.215780in}}{\pgfqpoint{1.108176in}{3.207967in}}%
\pgfpathcurveto{\pgfqpoint{1.100362in}{3.200153in}}{\pgfqpoint{1.095972in}{3.189554in}}{\pgfqpoint{1.095972in}{3.178504in}}%
\pgfpathcurveto{\pgfqpoint{1.095972in}{3.167454in}}{\pgfqpoint{1.100362in}{3.156855in}}{\pgfqpoint{1.108176in}{3.149041in}}%
\pgfpathcurveto{\pgfqpoint{1.115990in}{3.141228in}}{\pgfqpoint{1.126589in}{3.136837in}}{\pgfqpoint{1.137639in}{3.136837in}}%
\pgfpathclose%
\pgfusepath{stroke,fill}%
\end{pgfscope}%
\begin{pgfscope}%
\pgfpathrectangle{\pgfqpoint{0.648703in}{0.548769in}}{\pgfqpoint{5.201297in}{3.102590in}}%
\pgfusepath{clip}%
\pgfsetbuttcap%
\pgfsetroundjoin%
\definecolor{currentfill}{rgb}{1.000000,0.498039,0.054902}%
\pgfsetfillcolor{currentfill}%
\pgfsetlinewidth{1.003750pt}%
\definecolor{currentstroke}{rgb}{1.000000,0.498039,0.054902}%
\pgfsetstrokecolor{currentstroke}%
\pgfsetdash{}{0pt}%
\pgfpathmoveto{\pgfqpoint{1.368236in}{3.157577in}}%
\pgfpathcurveto{\pgfqpoint{1.379286in}{3.157577in}}{\pgfqpoint{1.389885in}{3.161967in}}{\pgfqpoint{1.397699in}{3.169780in}}%
\pgfpathcurveto{\pgfqpoint{1.405513in}{3.177594in}}{\pgfqpoint{1.409903in}{3.188193in}}{\pgfqpoint{1.409903in}{3.199243in}}%
\pgfpathcurveto{\pgfqpoint{1.409903in}{3.210293in}}{\pgfqpoint{1.405513in}{3.220892in}}{\pgfqpoint{1.397699in}{3.228706in}}%
\pgfpathcurveto{\pgfqpoint{1.389885in}{3.236520in}}{\pgfqpoint{1.379286in}{3.240910in}}{\pgfqpoint{1.368236in}{3.240910in}}%
\pgfpathcurveto{\pgfqpoint{1.357186in}{3.240910in}}{\pgfqpoint{1.346587in}{3.236520in}}{\pgfqpoint{1.338773in}{3.228706in}}%
\pgfpathcurveto{\pgfqpoint{1.330960in}{3.220892in}}{\pgfqpoint{1.326570in}{3.210293in}}{\pgfqpoint{1.326570in}{3.199243in}}%
\pgfpathcurveto{\pgfqpoint{1.326570in}{3.188193in}}{\pgfqpoint{1.330960in}{3.177594in}}{\pgfqpoint{1.338773in}{3.169780in}}%
\pgfpathcurveto{\pgfqpoint{1.346587in}{3.161967in}}{\pgfqpoint{1.357186in}{3.157577in}}{\pgfqpoint{1.368236in}{3.157577in}}%
\pgfpathclose%
\pgfusepath{stroke,fill}%
\end{pgfscope}%
\begin{pgfscope}%
\pgfpathrectangle{\pgfqpoint{0.648703in}{0.548769in}}{\pgfqpoint{5.201297in}{3.102590in}}%
\pgfusepath{clip}%
\pgfsetbuttcap%
\pgfsetroundjoin%
\definecolor{currentfill}{rgb}{0.121569,0.466667,0.705882}%
\pgfsetfillcolor{currentfill}%
\pgfsetlinewidth{1.003750pt}%
\definecolor{currentstroke}{rgb}{0.121569,0.466667,0.705882}%
\pgfsetstrokecolor{currentstroke}%
\pgfsetdash{}{0pt}%
\pgfpathmoveto{\pgfqpoint{1.088389in}{0.648129in}}%
\pgfpathcurveto{\pgfqpoint{1.099440in}{0.648129in}}{\pgfqpoint{1.110039in}{0.652519in}}{\pgfqpoint{1.117852in}{0.660333in}}%
\pgfpathcurveto{\pgfqpoint{1.125666in}{0.668146in}}{\pgfqpoint{1.130056in}{0.678745in}}{\pgfqpoint{1.130056in}{0.689796in}}%
\pgfpathcurveto{\pgfqpoint{1.130056in}{0.700846in}}{\pgfqpoint{1.125666in}{0.711445in}}{\pgfqpoint{1.117852in}{0.719258in}}%
\pgfpathcurveto{\pgfqpoint{1.110039in}{0.727072in}}{\pgfqpoint{1.099440in}{0.731462in}}{\pgfqpoint{1.088389in}{0.731462in}}%
\pgfpathcurveto{\pgfqpoint{1.077339in}{0.731462in}}{\pgfqpoint{1.066740in}{0.727072in}}{\pgfqpoint{1.058927in}{0.719258in}}%
\pgfpathcurveto{\pgfqpoint{1.051113in}{0.711445in}}{\pgfqpoint{1.046723in}{0.700846in}}{\pgfqpoint{1.046723in}{0.689796in}}%
\pgfpathcurveto{\pgfqpoint{1.046723in}{0.678745in}}{\pgfqpoint{1.051113in}{0.668146in}}{\pgfqpoint{1.058927in}{0.660333in}}%
\pgfpathcurveto{\pgfqpoint{1.066740in}{0.652519in}}{\pgfqpoint{1.077339in}{0.648129in}}{\pgfqpoint{1.088389in}{0.648129in}}%
\pgfpathclose%
\pgfusepath{stroke,fill}%
\end{pgfscope}%
\begin{pgfscope}%
\pgfpathrectangle{\pgfqpoint{0.648703in}{0.548769in}}{\pgfqpoint{5.201297in}{3.102590in}}%
\pgfusepath{clip}%
\pgfsetbuttcap%
\pgfsetroundjoin%
\definecolor{currentfill}{rgb}{0.121569,0.466667,0.705882}%
\pgfsetfillcolor{currentfill}%
\pgfsetlinewidth{1.003750pt}%
\definecolor{currentstroke}{rgb}{0.121569,0.466667,0.705882}%
\pgfsetstrokecolor{currentstroke}%
\pgfsetdash{}{0pt}%
\pgfpathmoveto{\pgfqpoint{0.941346in}{0.648129in}}%
\pgfpathcurveto{\pgfqpoint{0.952396in}{0.648129in}}{\pgfqpoint{0.962995in}{0.652519in}}{\pgfqpoint{0.970808in}{0.660333in}}%
\pgfpathcurveto{\pgfqpoint{0.978622in}{0.668146in}}{\pgfqpoint{0.983012in}{0.678745in}}{\pgfqpoint{0.983012in}{0.689796in}}%
\pgfpathcurveto{\pgfqpoint{0.983012in}{0.700846in}}{\pgfqpoint{0.978622in}{0.711445in}}{\pgfqpoint{0.970808in}{0.719258in}}%
\pgfpathcurveto{\pgfqpoint{0.962995in}{0.727072in}}{\pgfqpoint{0.952396in}{0.731462in}}{\pgfqpoint{0.941346in}{0.731462in}}%
\pgfpathcurveto{\pgfqpoint{0.930295in}{0.731462in}}{\pgfqpoint{0.919696in}{0.727072in}}{\pgfqpoint{0.911883in}{0.719258in}}%
\pgfpathcurveto{\pgfqpoint{0.904069in}{0.711445in}}{\pgfqpoint{0.899679in}{0.700846in}}{\pgfqpoint{0.899679in}{0.689796in}}%
\pgfpathcurveto{\pgfqpoint{0.899679in}{0.678745in}}{\pgfqpoint{0.904069in}{0.668146in}}{\pgfqpoint{0.911883in}{0.660333in}}%
\pgfpathcurveto{\pgfqpoint{0.919696in}{0.652519in}}{\pgfqpoint{0.930295in}{0.648129in}}{\pgfqpoint{0.941346in}{0.648129in}}%
\pgfpathclose%
\pgfusepath{stroke,fill}%
\end{pgfscope}%
\begin{pgfscope}%
\pgfpathrectangle{\pgfqpoint{0.648703in}{0.548769in}}{\pgfqpoint{5.201297in}{3.102590in}}%
\pgfusepath{clip}%
\pgfsetbuttcap%
\pgfsetroundjoin%
\definecolor{currentfill}{rgb}{1.000000,0.498039,0.054902}%
\pgfsetfillcolor{currentfill}%
\pgfsetlinewidth{1.003750pt}%
\definecolor{currentstroke}{rgb}{1.000000,0.498039,0.054902}%
\pgfsetstrokecolor{currentstroke}%
\pgfsetdash{}{0pt}%
\pgfpathmoveto{\pgfqpoint{2.154823in}{3.132690in}}%
\pgfpathcurveto{\pgfqpoint{2.165873in}{3.132690in}}{\pgfqpoint{2.176472in}{3.137080in}}{\pgfqpoint{2.184285in}{3.144893in}}%
\pgfpathcurveto{\pgfqpoint{2.192099in}{3.152707in}}{\pgfqpoint{2.196489in}{3.163306in}}{\pgfqpoint{2.196489in}{3.174356in}}%
\pgfpathcurveto{\pgfqpoint{2.196489in}{3.185406in}}{\pgfqpoint{2.192099in}{3.196005in}}{\pgfqpoint{2.184285in}{3.203819in}}%
\pgfpathcurveto{\pgfqpoint{2.176472in}{3.211633in}}{\pgfqpoint{2.165873in}{3.216023in}}{\pgfqpoint{2.154823in}{3.216023in}}%
\pgfpathcurveto{\pgfqpoint{2.143772in}{3.216023in}}{\pgfqpoint{2.133173in}{3.211633in}}{\pgfqpoint{2.125360in}{3.203819in}}%
\pgfpathcurveto{\pgfqpoint{2.117546in}{3.196005in}}{\pgfqpoint{2.113156in}{3.185406in}}{\pgfqpoint{2.113156in}{3.174356in}}%
\pgfpathcurveto{\pgfqpoint{2.113156in}{3.163306in}}{\pgfqpoint{2.117546in}{3.152707in}}{\pgfqpoint{2.125360in}{3.144893in}}%
\pgfpathcurveto{\pgfqpoint{2.133173in}{3.137080in}}{\pgfqpoint{2.143772in}{3.132690in}}{\pgfqpoint{2.154823in}{3.132690in}}%
\pgfpathclose%
\pgfusepath{stroke,fill}%
\end{pgfscope}%
\begin{pgfscope}%
\pgfpathrectangle{\pgfqpoint{0.648703in}{0.548769in}}{\pgfqpoint{5.201297in}{3.102590in}}%
\pgfusepath{clip}%
\pgfsetbuttcap%
\pgfsetroundjoin%
\definecolor{currentfill}{rgb}{1.000000,0.498039,0.054902}%
\pgfsetfillcolor{currentfill}%
\pgfsetlinewidth{1.003750pt}%
\definecolor{currentstroke}{rgb}{1.000000,0.498039,0.054902}%
\pgfsetstrokecolor{currentstroke}%
\pgfsetdash{}{0pt}%
\pgfpathmoveto{\pgfqpoint{1.931294in}{3.219794in}}%
\pgfpathcurveto{\pgfqpoint{1.942344in}{3.219794in}}{\pgfqpoint{1.952943in}{3.224185in}}{\pgfqpoint{1.960757in}{3.231998in}}%
\pgfpathcurveto{\pgfqpoint{1.968571in}{3.239812in}}{\pgfqpoint{1.972961in}{3.250411in}}{\pgfqpoint{1.972961in}{3.261461in}}%
\pgfpathcurveto{\pgfqpoint{1.972961in}{3.272511in}}{\pgfqpoint{1.968571in}{3.283110in}}{\pgfqpoint{1.960757in}{3.290924in}}%
\pgfpathcurveto{\pgfqpoint{1.952943in}{3.298737in}}{\pgfqpoint{1.942344in}{3.303128in}}{\pgfqpoint{1.931294in}{3.303128in}}%
\pgfpathcurveto{\pgfqpoint{1.920244in}{3.303128in}}{\pgfqpoint{1.909645in}{3.298737in}}{\pgfqpoint{1.901831in}{3.290924in}}%
\pgfpathcurveto{\pgfqpoint{1.894018in}{3.283110in}}{\pgfqpoint{1.889628in}{3.272511in}}{\pgfqpoint{1.889628in}{3.261461in}}%
\pgfpathcurveto{\pgfqpoint{1.889628in}{3.250411in}}{\pgfqpoint{1.894018in}{3.239812in}}{\pgfqpoint{1.901831in}{3.231998in}}%
\pgfpathcurveto{\pgfqpoint{1.909645in}{3.224185in}}{\pgfqpoint{1.920244in}{3.219794in}}{\pgfqpoint{1.931294in}{3.219794in}}%
\pgfpathclose%
\pgfusepath{stroke,fill}%
\end{pgfscope}%
\begin{pgfscope}%
\pgfpathrectangle{\pgfqpoint{0.648703in}{0.548769in}}{\pgfqpoint{5.201297in}{3.102590in}}%
\pgfusepath{clip}%
\pgfsetbuttcap%
\pgfsetroundjoin%
\definecolor{currentfill}{rgb}{1.000000,0.498039,0.054902}%
\pgfsetfillcolor{currentfill}%
\pgfsetlinewidth{1.003750pt}%
\definecolor{currentstroke}{rgb}{1.000000,0.498039,0.054902}%
\pgfsetstrokecolor{currentstroke}%
\pgfsetdash{}{0pt}%
\pgfpathmoveto{\pgfqpoint{1.378499in}{3.140985in}}%
\pgfpathcurveto{\pgfqpoint{1.389549in}{3.140985in}}{\pgfqpoint{1.400148in}{3.145375in}}{\pgfqpoint{1.407962in}{3.153189in}}%
\pgfpathcurveto{\pgfqpoint{1.415776in}{3.161003in}}{\pgfqpoint{1.420166in}{3.171602in}}{\pgfqpoint{1.420166in}{3.182652in}}%
\pgfpathcurveto{\pgfqpoint{1.420166in}{3.193702in}}{\pgfqpoint{1.415776in}{3.204301in}}{\pgfqpoint{1.407962in}{3.212115in}}%
\pgfpathcurveto{\pgfqpoint{1.400148in}{3.219928in}}{\pgfqpoint{1.389549in}{3.224319in}}{\pgfqpoint{1.378499in}{3.224319in}}%
\pgfpathcurveto{\pgfqpoint{1.367449in}{3.224319in}}{\pgfqpoint{1.356850in}{3.219928in}}{\pgfqpoint{1.349037in}{3.212115in}}%
\pgfpathcurveto{\pgfqpoint{1.341223in}{3.204301in}}{\pgfqpoint{1.336833in}{3.193702in}}{\pgfqpoint{1.336833in}{3.182652in}}%
\pgfpathcurveto{\pgfqpoint{1.336833in}{3.171602in}}{\pgfqpoint{1.341223in}{3.161003in}}{\pgfqpoint{1.349037in}{3.153189in}}%
\pgfpathcurveto{\pgfqpoint{1.356850in}{3.145375in}}{\pgfqpoint{1.367449in}{3.140985in}}{\pgfqpoint{1.378499in}{3.140985in}}%
\pgfpathclose%
\pgfusepath{stroke,fill}%
\end{pgfscope}%
\begin{pgfscope}%
\pgfpathrectangle{\pgfqpoint{0.648703in}{0.548769in}}{\pgfqpoint{5.201297in}{3.102590in}}%
\pgfusepath{clip}%
\pgfsetbuttcap%
\pgfsetroundjoin%
\definecolor{currentfill}{rgb}{1.000000,0.498039,0.054902}%
\pgfsetfillcolor{currentfill}%
\pgfsetlinewidth{1.003750pt}%
\definecolor{currentstroke}{rgb}{1.000000,0.498039,0.054902}%
\pgfsetstrokecolor{currentstroke}%
\pgfsetdash{}{0pt}%
\pgfpathmoveto{\pgfqpoint{1.487515in}{3.145133in}}%
\pgfpathcurveto{\pgfqpoint{1.498565in}{3.145133in}}{\pgfqpoint{1.509164in}{3.149523in}}{\pgfqpoint{1.516977in}{3.157337in}}%
\pgfpathcurveto{\pgfqpoint{1.524791in}{3.165151in}}{\pgfqpoint{1.529181in}{3.175750in}}{\pgfqpoint{1.529181in}{3.186800in}}%
\pgfpathcurveto{\pgfqpoint{1.529181in}{3.197850in}}{\pgfqpoint{1.524791in}{3.208449in}}{\pgfqpoint{1.516977in}{3.216262in}}%
\pgfpathcurveto{\pgfqpoint{1.509164in}{3.224076in}}{\pgfqpoint{1.498565in}{3.228466in}}{\pgfqpoint{1.487515in}{3.228466in}}%
\pgfpathcurveto{\pgfqpoint{1.476465in}{3.228466in}}{\pgfqpoint{1.465866in}{3.224076in}}{\pgfqpoint{1.458052in}{3.216262in}}%
\pgfpathcurveto{\pgfqpoint{1.450238in}{3.208449in}}{\pgfqpoint{1.445848in}{3.197850in}}{\pgfqpoint{1.445848in}{3.186800in}}%
\pgfpathcurveto{\pgfqpoint{1.445848in}{3.175750in}}{\pgfqpoint{1.450238in}{3.165151in}}{\pgfqpoint{1.458052in}{3.157337in}}%
\pgfpathcurveto{\pgfqpoint{1.465866in}{3.149523in}}{\pgfqpoint{1.476465in}{3.145133in}}{\pgfqpoint{1.487515in}{3.145133in}}%
\pgfpathclose%
\pgfusepath{stroke,fill}%
\end{pgfscope}%
\begin{pgfscope}%
\pgfpathrectangle{\pgfqpoint{0.648703in}{0.548769in}}{\pgfqpoint{5.201297in}{3.102590in}}%
\pgfusepath{clip}%
\pgfsetbuttcap%
\pgfsetroundjoin%
\definecolor{currentfill}{rgb}{1.000000,0.498039,0.054902}%
\pgfsetfillcolor{currentfill}%
\pgfsetlinewidth{1.003750pt}%
\definecolor{currentstroke}{rgb}{1.000000,0.498039,0.054902}%
\pgfsetstrokecolor{currentstroke}%
\pgfsetdash{}{0pt}%
\pgfpathmoveto{\pgfqpoint{1.367468in}{3.136837in}}%
\pgfpathcurveto{\pgfqpoint{1.378518in}{3.136837in}}{\pgfqpoint{1.389118in}{3.141228in}}{\pgfqpoint{1.396931in}{3.149041in}}%
\pgfpathcurveto{\pgfqpoint{1.404745in}{3.156855in}}{\pgfqpoint{1.409135in}{3.167454in}}{\pgfqpoint{1.409135in}{3.178504in}}%
\pgfpathcurveto{\pgfqpoint{1.409135in}{3.189554in}}{\pgfqpoint{1.404745in}{3.200153in}}{\pgfqpoint{1.396931in}{3.207967in}}%
\pgfpathcurveto{\pgfqpoint{1.389118in}{3.215780in}}{\pgfqpoint{1.378518in}{3.220171in}}{\pgfqpoint{1.367468in}{3.220171in}}%
\pgfpathcurveto{\pgfqpoint{1.356418in}{3.220171in}}{\pgfqpoint{1.345819in}{3.215780in}}{\pgfqpoint{1.338006in}{3.207967in}}%
\pgfpathcurveto{\pgfqpoint{1.330192in}{3.200153in}}{\pgfqpoint{1.325802in}{3.189554in}}{\pgfqpoint{1.325802in}{3.178504in}}%
\pgfpathcurveto{\pgfqpoint{1.325802in}{3.167454in}}{\pgfqpoint{1.330192in}{3.156855in}}{\pgfqpoint{1.338006in}{3.149041in}}%
\pgfpathcurveto{\pgfqpoint{1.345819in}{3.141228in}}{\pgfqpoint{1.356418in}{3.136837in}}{\pgfqpoint{1.367468in}{3.136837in}}%
\pgfpathclose%
\pgfusepath{stroke,fill}%
\end{pgfscope}%
\begin{pgfscope}%
\pgfpathrectangle{\pgfqpoint{0.648703in}{0.548769in}}{\pgfqpoint{5.201297in}{3.102590in}}%
\pgfusepath{clip}%
\pgfsetbuttcap%
\pgfsetroundjoin%
\definecolor{currentfill}{rgb}{0.121569,0.466667,0.705882}%
\pgfsetfillcolor{currentfill}%
\pgfsetlinewidth{1.003750pt}%
\definecolor{currentstroke}{rgb}{0.121569,0.466667,0.705882}%
\pgfsetstrokecolor{currentstroke}%
\pgfsetdash{}{0pt}%
\pgfpathmoveto{\pgfqpoint{0.939434in}{0.648129in}}%
\pgfpathcurveto{\pgfqpoint{0.950484in}{0.648129in}}{\pgfqpoint{0.961083in}{0.652519in}}{\pgfqpoint{0.968897in}{0.660333in}}%
\pgfpathcurveto{\pgfqpoint{0.976710in}{0.668146in}}{\pgfqpoint{0.981100in}{0.678745in}}{\pgfqpoint{0.981100in}{0.689796in}}%
\pgfpathcurveto{\pgfqpoint{0.981100in}{0.700846in}}{\pgfqpoint{0.976710in}{0.711445in}}{\pgfqpoint{0.968897in}{0.719258in}}%
\pgfpathcurveto{\pgfqpoint{0.961083in}{0.727072in}}{\pgfqpoint{0.950484in}{0.731462in}}{\pgfqpoint{0.939434in}{0.731462in}}%
\pgfpathcurveto{\pgfqpoint{0.928384in}{0.731462in}}{\pgfqpoint{0.917785in}{0.727072in}}{\pgfqpoint{0.909971in}{0.719258in}}%
\pgfpathcurveto{\pgfqpoint{0.902157in}{0.711445in}}{\pgfqpoint{0.897767in}{0.700846in}}{\pgfqpoint{0.897767in}{0.689796in}}%
\pgfpathcurveto{\pgfqpoint{0.897767in}{0.678745in}}{\pgfqpoint{0.902157in}{0.668146in}}{\pgfqpoint{0.909971in}{0.660333in}}%
\pgfpathcurveto{\pgfqpoint{0.917785in}{0.652519in}}{\pgfqpoint{0.928384in}{0.648129in}}{\pgfqpoint{0.939434in}{0.648129in}}%
\pgfpathclose%
\pgfusepath{stroke,fill}%
\end{pgfscope}%
\begin{pgfscope}%
\pgfpathrectangle{\pgfqpoint{0.648703in}{0.548769in}}{\pgfqpoint{5.201297in}{3.102590in}}%
\pgfusepath{clip}%
\pgfsetbuttcap%
\pgfsetroundjoin%
\definecolor{currentfill}{rgb}{1.000000,0.498039,0.054902}%
\pgfsetfillcolor{currentfill}%
\pgfsetlinewidth{1.003750pt}%
\definecolor{currentstroke}{rgb}{1.000000,0.498039,0.054902}%
\pgfsetstrokecolor{currentstroke}%
\pgfsetdash{}{0pt}%
\pgfpathmoveto{\pgfqpoint{1.880553in}{3.132690in}}%
\pgfpathcurveto{\pgfqpoint{1.891603in}{3.132690in}}{\pgfqpoint{1.902202in}{3.137080in}}{\pgfqpoint{1.910015in}{3.144893in}}%
\pgfpathcurveto{\pgfqpoint{1.917829in}{3.152707in}}{\pgfqpoint{1.922219in}{3.163306in}}{\pgfqpoint{1.922219in}{3.174356in}}%
\pgfpathcurveto{\pgfqpoint{1.922219in}{3.185406in}}{\pgfqpoint{1.917829in}{3.196005in}}{\pgfqpoint{1.910015in}{3.203819in}}%
\pgfpathcurveto{\pgfqpoint{1.902202in}{3.211633in}}{\pgfqpoint{1.891603in}{3.216023in}}{\pgfqpoint{1.880553in}{3.216023in}}%
\pgfpathcurveto{\pgfqpoint{1.869502in}{3.216023in}}{\pgfqpoint{1.858903in}{3.211633in}}{\pgfqpoint{1.851090in}{3.203819in}}%
\pgfpathcurveto{\pgfqpoint{1.843276in}{3.196005in}}{\pgfqpoint{1.838886in}{3.185406in}}{\pgfqpoint{1.838886in}{3.174356in}}%
\pgfpathcurveto{\pgfqpoint{1.838886in}{3.163306in}}{\pgfqpoint{1.843276in}{3.152707in}}{\pgfqpoint{1.851090in}{3.144893in}}%
\pgfpathcurveto{\pgfqpoint{1.858903in}{3.137080in}}{\pgfqpoint{1.869502in}{3.132690in}}{\pgfqpoint{1.880553in}{3.132690in}}%
\pgfpathclose%
\pgfusepath{stroke,fill}%
\end{pgfscope}%
\begin{pgfscope}%
\pgfpathrectangle{\pgfqpoint{0.648703in}{0.548769in}}{\pgfqpoint{5.201297in}{3.102590in}}%
\pgfusepath{clip}%
\pgfsetbuttcap%
\pgfsetroundjoin%
\definecolor{currentfill}{rgb}{1.000000,0.498039,0.054902}%
\pgfsetfillcolor{currentfill}%
\pgfsetlinewidth{1.003750pt}%
\definecolor{currentstroke}{rgb}{1.000000,0.498039,0.054902}%
\pgfsetstrokecolor{currentstroke}%
\pgfsetdash{}{0pt}%
\pgfpathmoveto{\pgfqpoint{1.340673in}{3.145133in}}%
\pgfpathcurveto{\pgfqpoint{1.351723in}{3.145133in}}{\pgfqpoint{1.362322in}{3.149523in}}{\pgfqpoint{1.370135in}{3.157337in}}%
\pgfpathcurveto{\pgfqpoint{1.377949in}{3.165151in}}{\pgfqpoint{1.382339in}{3.175750in}}{\pgfqpoint{1.382339in}{3.186800in}}%
\pgfpathcurveto{\pgfqpoint{1.382339in}{3.197850in}}{\pgfqpoint{1.377949in}{3.208449in}}{\pgfqpoint{1.370135in}{3.216262in}}%
\pgfpathcurveto{\pgfqpoint{1.362322in}{3.224076in}}{\pgfqpoint{1.351723in}{3.228466in}}{\pgfqpoint{1.340673in}{3.228466in}}%
\pgfpathcurveto{\pgfqpoint{1.329623in}{3.228466in}}{\pgfqpoint{1.319024in}{3.224076in}}{\pgfqpoint{1.311210in}{3.216262in}}%
\pgfpathcurveto{\pgfqpoint{1.303396in}{3.208449in}}{\pgfqpoint{1.299006in}{3.197850in}}{\pgfqpoint{1.299006in}{3.186800in}}%
\pgfpathcurveto{\pgfqpoint{1.299006in}{3.175750in}}{\pgfqpoint{1.303396in}{3.165151in}}{\pgfqpoint{1.311210in}{3.157337in}}%
\pgfpathcurveto{\pgfqpoint{1.319024in}{3.149523in}}{\pgfqpoint{1.329623in}{3.145133in}}{\pgfqpoint{1.340673in}{3.145133in}}%
\pgfpathclose%
\pgfusepath{stroke,fill}%
\end{pgfscope}%
\begin{pgfscope}%
\pgfpathrectangle{\pgfqpoint{0.648703in}{0.548769in}}{\pgfqpoint{5.201297in}{3.102590in}}%
\pgfusepath{clip}%
\pgfsetbuttcap%
\pgfsetroundjoin%
\definecolor{currentfill}{rgb}{1.000000,0.498039,0.054902}%
\pgfsetfillcolor{currentfill}%
\pgfsetlinewidth{1.003750pt}%
\definecolor{currentstroke}{rgb}{1.000000,0.498039,0.054902}%
\pgfsetstrokecolor{currentstroke}%
\pgfsetdash{}{0pt}%
\pgfpathmoveto{\pgfqpoint{1.578557in}{3.132690in}}%
\pgfpathcurveto{\pgfqpoint{1.589607in}{3.132690in}}{\pgfqpoint{1.600206in}{3.137080in}}{\pgfqpoint{1.608020in}{3.144893in}}%
\pgfpathcurveto{\pgfqpoint{1.615833in}{3.152707in}}{\pgfqpoint{1.620223in}{3.163306in}}{\pgfqpoint{1.620223in}{3.174356in}}%
\pgfpathcurveto{\pgfqpoint{1.620223in}{3.185406in}}{\pgfqpoint{1.615833in}{3.196005in}}{\pgfqpoint{1.608020in}{3.203819in}}%
\pgfpathcurveto{\pgfqpoint{1.600206in}{3.211633in}}{\pgfqpoint{1.589607in}{3.216023in}}{\pgfqpoint{1.578557in}{3.216023in}}%
\pgfpathcurveto{\pgfqpoint{1.567507in}{3.216023in}}{\pgfqpoint{1.556908in}{3.211633in}}{\pgfqpoint{1.549094in}{3.203819in}}%
\pgfpathcurveto{\pgfqpoint{1.541280in}{3.196005in}}{\pgfqpoint{1.536890in}{3.185406in}}{\pgfqpoint{1.536890in}{3.174356in}}%
\pgfpathcurveto{\pgfqpoint{1.536890in}{3.163306in}}{\pgfqpoint{1.541280in}{3.152707in}}{\pgfqpoint{1.549094in}{3.144893in}}%
\pgfpathcurveto{\pgfqpoint{1.556908in}{3.137080in}}{\pgfqpoint{1.567507in}{3.132690in}}{\pgfqpoint{1.578557in}{3.132690in}}%
\pgfpathclose%
\pgfusepath{stroke,fill}%
\end{pgfscope}%
\begin{pgfscope}%
\pgfpathrectangle{\pgfqpoint{0.648703in}{0.548769in}}{\pgfqpoint{5.201297in}{3.102590in}}%
\pgfusepath{clip}%
\pgfsetbuttcap%
\pgfsetroundjoin%
\definecolor{currentfill}{rgb}{1.000000,0.498039,0.054902}%
\pgfsetfillcolor{currentfill}%
\pgfsetlinewidth{1.003750pt}%
\definecolor{currentstroke}{rgb}{1.000000,0.498039,0.054902}%
\pgfsetstrokecolor{currentstroke}%
\pgfsetdash{}{0pt}%
\pgfpathmoveto{\pgfqpoint{1.095866in}{3.157577in}}%
\pgfpathcurveto{\pgfqpoint{1.106916in}{3.157577in}}{\pgfqpoint{1.117515in}{3.161967in}}{\pgfqpoint{1.125329in}{3.169780in}}%
\pgfpathcurveto{\pgfqpoint{1.133142in}{3.177594in}}{\pgfqpoint{1.137533in}{3.188193in}}{\pgfqpoint{1.137533in}{3.199243in}}%
\pgfpathcurveto{\pgfqpoint{1.137533in}{3.210293in}}{\pgfqpoint{1.133142in}{3.220892in}}{\pgfqpoint{1.125329in}{3.228706in}}%
\pgfpathcurveto{\pgfqpoint{1.117515in}{3.236520in}}{\pgfqpoint{1.106916in}{3.240910in}}{\pgfqpoint{1.095866in}{3.240910in}}%
\pgfpathcurveto{\pgfqpoint{1.084816in}{3.240910in}}{\pgfqpoint{1.074217in}{3.236520in}}{\pgfqpoint{1.066403in}{3.228706in}}%
\pgfpathcurveto{\pgfqpoint{1.058590in}{3.220892in}}{\pgfqpoint{1.054199in}{3.210293in}}{\pgfqpoint{1.054199in}{3.199243in}}%
\pgfpathcurveto{\pgfqpoint{1.054199in}{3.188193in}}{\pgfqpoint{1.058590in}{3.177594in}}{\pgfqpoint{1.066403in}{3.169780in}}%
\pgfpathcurveto{\pgfqpoint{1.074217in}{3.161967in}}{\pgfqpoint{1.084816in}{3.157577in}}{\pgfqpoint{1.095866in}{3.157577in}}%
\pgfpathclose%
\pgfusepath{stroke,fill}%
\end{pgfscope}%
\begin{pgfscope}%
\pgfpathrectangle{\pgfqpoint{0.648703in}{0.548769in}}{\pgfqpoint{5.201297in}{3.102590in}}%
\pgfusepath{clip}%
\pgfsetbuttcap%
\pgfsetroundjoin%
\definecolor{currentfill}{rgb}{1.000000,0.498039,0.054902}%
\pgfsetfillcolor{currentfill}%
\pgfsetlinewidth{1.003750pt}%
\definecolor{currentstroke}{rgb}{1.000000,0.498039,0.054902}%
\pgfsetstrokecolor{currentstroke}%
\pgfsetdash{}{0pt}%
\pgfpathmoveto{\pgfqpoint{1.372808in}{3.145133in}}%
\pgfpathcurveto{\pgfqpoint{1.383858in}{3.145133in}}{\pgfqpoint{1.394457in}{3.149523in}}{\pgfqpoint{1.402270in}{3.157337in}}%
\pgfpathcurveto{\pgfqpoint{1.410084in}{3.165151in}}{\pgfqpoint{1.414474in}{3.175750in}}{\pgfqpoint{1.414474in}{3.186800in}}%
\pgfpathcurveto{\pgfqpoint{1.414474in}{3.197850in}}{\pgfqpoint{1.410084in}{3.208449in}}{\pgfqpoint{1.402270in}{3.216262in}}%
\pgfpathcurveto{\pgfqpoint{1.394457in}{3.224076in}}{\pgfqpoint{1.383858in}{3.228466in}}{\pgfqpoint{1.372808in}{3.228466in}}%
\pgfpathcurveto{\pgfqpoint{1.361758in}{3.228466in}}{\pgfqpoint{1.351159in}{3.224076in}}{\pgfqpoint{1.343345in}{3.216262in}}%
\pgfpathcurveto{\pgfqpoint{1.335531in}{3.208449in}}{\pgfqpoint{1.331141in}{3.197850in}}{\pgfqpoint{1.331141in}{3.186800in}}%
\pgfpathcurveto{\pgfqpoint{1.331141in}{3.175750in}}{\pgfqpoint{1.335531in}{3.165151in}}{\pgfqpoint{1.343345in}{3.157337in}}%
\pgfpathcurveto{\pgfqpoint{1.351159in}{3.149523in}}{\pgfqpoint{1.361758in}{3.145133in}}{\pgfqpoint{1.372808in}{3.145133in}}%
\pgfpathclose%
\pgfusepath{stroke,fill}%
\end{pgfscope}%
\begin{pgfscope}%
\pgfpathrectangle{\pgfqpoint{0.648703in}{0.548769in}}{\pgfqpoint{5.201297in}{3.102590in}}%
\pgfusepath{clip}%
\pgfsetbuttcap%
\pgfsetroundjoin%
\definecolor{currentfill}{rgb}{1.000000,0.498039,0.054902}%
\pgfsetfillcolor{currentfill}%
\pgfsetlinewidth{1.003750pt}%
\definecolor{currentstroke}{rgb}{1.000000,0.498039,0.054902}%
\pgfsetstrokecolor{currentstroke}%
\pgfsetdash{}{0pt}%
\pgfpathmoveto{\pgfqpoint{1.274772in}{3.190759in}}%
\pgfpathcurveto{\pgfqpoint{1.285822in}{3.190759in}}{\pgfqpoint{1.296421in}{3.195150in}}{\pgfqpoint{1.304235in}{3.202963in}}%
\pgfpathcurveto{\pgfqpoint{1.312048in}{3.210777in}}{\pgfqpoint{1.316439in}{3.221376in}}{\pgfqpoint{1.316439in}{3.232426in}}%
\pgfpathcurveto{\pgfqpoint{1.316439in}{3.243476in}}{\pgfqpoint{1.312048in}{3.254075in}}{\pgfqpoint{1.304235in}{3.261889in}}%
\pgfpathcurveto{\pgfqpoint{1.296421in}{3.269702in}}{\pgfqpoint{1.285822in}{3.274093in}}{\pgfqpoint{1.274772in}{3.274093in}}%
\pgfpathcurveto{\pgfqpoint{1.263722in}{3.274093in}}{\pgfqpoint{1.253123in}{3.269702in}}{\pgfqpoint{1.245309in}{3.261889in}}%
\pgfpathcurveto{\pgfqpoint{1.237495in}{3.254075in}}{\pgfqpoint{1.233105in}{3.243476in}}{\pgfqpoint{1.233105in}{3.232426in}}%
\pgfpathcurveto{\pgfqpoint{1.233105in}{3.221376in}}{\pgfqpoint{1.237495in}{3.210777in}}{\pgfqpoint{1.245309in}{3.202963in}}%
\pgfpathcurveto{\pgfqpoint{1.253123in}{3.195150in}}{\pgfqpoint{1.263722in}{3.190759in}}{\pgfqpoint{1.274772in}{3.190759in}}%
\pgfpathclose%
\pgfusepath{stroke,fill}%
\end{pgfscope}%
\begin{pgfscope}%
\pgfpathrectangle{\pgfqpoint{0.648703in}{0.548769in}}{\pgfqpoint{5.201297in}{3.102590in}}%
\pgfusepath{clip}%
\pgfsetbuttcap%
\pgfsetroundjoin%
\definecolor{currentfill}{rgb}{1.000000,0.498039,0.054902}%
\pgfsetfillcolor{currentfill}%
\pgfsetlinewidth{1.003750pt}%
\definecolor{currentstroke}{rgb}{1.000000,0.498039,0.054902}%
\pgfsetstrokecolor{currentstroke}%
\pgfsetdash{}{0pt}%
\pgfpathmoveto{\pgfqpoint{1.499496in}{3.207351in}}%
\pgfpathcurveto{\pgfqpoint{1.510546in}{3.207351in}}{\pgfqpoint{1.521145in}{3.211741in}}{\pgfqpoint{1.528958in}{3.219555in}}%
\pgfpathcurveto{\pgfqpoint{1.536772in}{3.227368in}}{\pgfqpoint{1.541162in}{3.237967in}}{\pgfqpoint{1.541162in}{3.249017in}}%
\pgfpathcurveto{\pgfqpoint{1.541162in}{3.260068in}}{\pgfqpoint{1.536772in}{3.270667in}}{\pgfqpoint{1.528958in}{3.278480in}}%
\pgfpathcurveto{\pgfqpoint{1.521145in}{3.286294in}}{\pgfqpoint{1.510546in}{3.290684in}}{\pgfqpoint{1.499496in}{3.290684in}}%
\pgfpathcurveto{\pgfqpoint{1.488445in}{3.290684in}}{\pgfqpoint{1.477846in}{3.286294in}}{\pgfqpoint{1.470033in}{3.278480in}}%
\pgfpathcurveto{\pgfqpoint{1.462219in}{3.270667in}}{\pgfqpoint{1.457829in}{3.260068in}}{\pgfqpoint{1.457829in}{3.249017in}}%
\pgfpathcurveto{\pgfqpoint{1.457829in}{3.237967in}}{\pgfqpoint{1.462219in}{3.227368in}}{\pgfqpoint{1.470033in}{3.219555in}}%
\pgfpathcurveto{\pgfqpoint{1.477846in}{3.211741in}}{\pgfqpoint{1.488445in}{3.207351in}}{\pgfqpoint{1.499496in}{3.207351in}}%
\pgfpathclose%
\pgfusepath{stroke,fill}%
\end{pgfscope}%
\begin{pgfscope}%
\pgfpathrectangle{\pgfqpoint{0.648703in}{0.548769in}}{\pgfqpoint{5.201297in}{3.102590in}}%
\pgfusepath{clip}%
\pgfsetbuttcap%
\pgfsetroundjoin%
\definecolor{currentfill}{rgb}{0.121569,0.466667,0.705882}%
\pgfsetfillcolor{currentfill}%
\pgfsetlinewidth{1.003750pt}%
\definecolor{currentstroke}{rgb}{0.121569,0.466667,0.705882}%
\pgfsetstrokecolor{currentstroke}%
\pgfsetdash{}{0pt}%
\pgfpathmoveto{\pgfqpoint{0.885126in}{0.648129in}}%
\pgfpathcurveto{\pgfqpoint{0.896176in}{0.648129in}}{\pgfqpoint{0.906775in}{0.652519in}}{\pgfqpoint{0.914589in}{0.660333in}}%
\pgfpathcurveto{\pgfqpoint{0.922402in}{0.668146in}}{\pgfqpoint{0.926793in}{0.678745in}}{\pgfqpoint{0.926793in}{0.689796in}}%
\pgfpathcurveto{\pgfqpoint{0.926793in}{0.700846in}}{\pgfqpoint{0.922402in}{0.711445in}}{\pgfqpoint{0.914589in}{0.719258in}}%
\pgfpathcurveto{\pgfqpoint{0.906775in}{0.727072in}}{\pgfqpoint{0.896176in}{0.731462in}}{\pgfqpoint{0.885126in}{0.731462in}}%
\pgfpathcurveto{\pgfqpoint{0.874076in}{0.731462in}}{\pgfqpoint{0.863477in}{0.727072in}}{\pgfqpoint{0.855663in}{0.719258in}}%
\pgfpathcurveto{\pgfqpoint{0.847850in}{0.711445in}}{\pgfqpoint{0.843459in}{0.700846in}}{\pgfqpoint{0.843459in}{0.689796in}}%
\pgfpathcurveto{\pgfqpoint{0.843459in}{0.678745in}}{\pgfqpoint{0.847850in}{0.668146in}}{\pgfqpoint{0.855663in}{0.660333in}}%
\pgfpathcurveto{\pgfqpoint{0.863477in}{0.652519in}}{\pgfqpoint{0.874076in}{0.648129in}}{\pgfqpoint{0.885126in}{0.648129in}}%
\pgfpathclose%
\pgfusepath{stroke,fill}%
\end{pgfscope}%
\begin{pgfscope}%
\pgfpathrectangle{\pgfqpoint{0.648703in}{0.548769in}}{\pgfqpoint{5.201297in}{3.102590in}}%
\pgfusepath{clip}%
\pgfsetbuttcap%
\pgfsetroundjoin%
\definecolor{currentfill}{rgb}{0.121569,0.466667,0.705882}%
\pgfsetfillcolor{currentfill}%
\pgfsetlinewidth{1.003750pt}%
\definecolor{currentstroke}{rgb}{0.121569,0.466667,0.705882}%
\pgfsetstrokecolor{currentstroke}%
\pgfsetdash{}{0pt}%
\pgfpathmoveto{\pgfqpoint{0.930200in}{0.656425in}}%
\pgfpathcurveto{\pgfqpoint{0.941250in}{0.656425in}}{\pgfqpoint{0.951849in}{0.660815in}}{\pgfqpoint{0.959663in}{0.668629in}}%
\pgfpathcurveto{\pgfqpoint{0.967476in}{0.676442in}}{\pgfqpoint{0.971866in}{0.687041in}}{\pgfqpoint{0.971866in}{0.698091in}}%
\pgfpathcurveto{\pgfqpoint{0.971866in}{0.709141in}}{\pgfqpoint{0.967476in}{0.719740in}}{\pgfqpoint{0.959663in}{0.727554in}}%
\pgfpathcurveto{\pgfqpoint{0.951849in}{0.735368in}}{\pgfqpoint{0.941250in}{0.739758in}}{\pgfqpoint{0.930200in}{0.739758in}}%
\pgfpathcurveto{\pgfqpoint{0.919150in}{0.739758in}}{\pgfqpoint{0.908551in}{0.735368in}}{\pgfqpoint{0.900737in}{0.727554in}}%
\pgfpathcurveto{\pgfqpoint{0.892923in}{0.719740in}}{\pgfqpoint{0.888533in}{0.709141in}}{\pgfqpoint{0.888533in}{0.698091in}}%
\pgfpathcurveto{\pgfqpoint{0.888533in}{0.687041in}}{\pgfqpoint{0.892923in}{0.676442in}}{\pgfqpoint{0.900737in}{0.668629in}}%
\pgfpathcurveto{\pgfqpoint{0.908551in}{0.660815in}}{\pgfqpoint{0.919150in}{0.656425in}}{\pgfqpoint{0.930200in}{0.656425in}}%
\pgfpathclose%
\pgfusepath{stroke,fill}%
\end{pgfscope}%
\begin{pgfscope}%
\pgfpathrectangle{\pgfqpoint{0.648703in}{0.548769in}}{\pgfqpoint{5.201297in}{3.102590in}}%
\pgfusepath{clip}%
\pgfsetbuttcap%
\pgfsetroundjoin%
\definecolor{currentfill}{rgb}{1.000000,0.498039,0.054902}%
\pgfsetfillcolor{currentfill}%
\pgfsetlinewidth{1.003750pt}%
\definecolor{currentstroke}{rgb}{1.000000,0.498039,0.054902}%
\pgfsetstrokecolor{currentstroke}%
\pgfsetdash{}{0pt}%
\pgfpathmoveto{\pgfqpoint{4.806473in}{3.099507in}}%
\pgfpathcurveto{\pgfqpoint{4.817523in}{3.099507in}}{\pgfqpoint{4.828122in}{3.103897in}}{\pgfqpoint{4.835936in}{3.111711in}}%
\pgfpathcurveto{\pgfqpoint{4.843749in}{3.119524in}}{\pgfqpoint{4.848139in}{3.130123in}}{\pgfqpoint{4.848139in}{3.141173in}}%
\pgfpathcurveto{\pgfqpoint{4.848139in}{3.152224in}}{\pgfqpoint{4.843749in}{3.162823in}}{\pgfqpoint{4.835936in}{3.170636in}}%
\pgfpathcurveto{\pgfqpoint{4.828122in}{3.178450in}}{\pgfqpoint{4.817523in}{3.182840in}}{\pgfqpoint{4.806473in}{3.182840in}}%
\pgfpathcurveto{\pgfqpoint{4.795423in}{3.182840in}}{\pgfqpoint{4.784824in}{3.178450in}}{\pgfqpoint{4.777010in}{3.170636in}}%
\pgfpathcurveto{\pgfqpoint{4.769196in}{3.162823in}}{\pgfqpoint{4.764806in}{3.152224in}}{\pgfqpoint{4.764806in}{3.141173in}}%
\pgfpathcurveto{\pgfqpoint{4.764806in}{3.130123in}}{\pgfqpoint{4.769196in}{3.119524in}}{\pgfqpoint{4.777010in}{3.111711in}}%
\pgfpathcurveto{\pgfqpoint{4.784824in}{3.103897in}}{\pgfqpoint{4.795423in}{3.099507in}}{\pgfqpoint{4.806473in}{3.099507in}}%
\pgfpathclose%
\pgfusepath{stroke,fill}%
\end{pgfscope}%
\begin{pgfscope}%
\pgfpathrectangle{\pgfqpoint{0.648703in}{0.548769in}}{\pgfqpoint{5.201297in}{3.102590in}}%
\pgfusepath{clip}%
\pgfsetbuttcap%
\pgfsetroundjoin%
\definecolor{currentfill}{rgb}{1.000000,0.498039,0.054902}%
\pgfsetfillcolor{currentfill}%
\pgfsetlinewidth{1.003750pt}%
\definecolor{currentstroke}{rgb}{1.000000,0.498039,0.054902}%
\pgfsetstrokecolor{currentstroke}%
\pgfsetdash{}{0pt}%
\pgfpathmoveto{\pgfqpoint{1.379714in}{3.136837in}}%
\pgfpathcurveto{\pgfqpoint{1.390765in}{3.136837in}}{\pgfqpoint{1.401364in}{3.141228in}}{\pgfqpoint{1.409177in}{3.149041in}}%
\pgfpathcurveto{\pgfqpoint{1.416991in}{3.156855in}}{\pgfqpoint{1.421381in}{3.167454in}}{\pgfqpoint{1.421381in}{3.178504in}}%
\pgfpathcurveto{\pgfqpoint{1.421381in}{3.189554in}}{\pgfqpoint{1.416991in}{3.200153in}}{\pgfqpoint{1.409177in}{3.207967in}}%
\pgfpathcurveto{\pgfqpoint{1.401364in}{3.215780in}}{\pgfqpoint{1.390765in}{3.220171in}}{\pgfqpoint{1.379714in}{3.220171in}}%
\pgfpathcurveto{\pgfqpoint{1.368664in}{3.220171in}}{\pgfqpoint{1.358065in}{3.215780in}}{\pgfqpoint{1.350252in}{3.207967in}}%
\pgfpathcurveto{\pgfqpoint{1.342438in}{3.200153in}}{\pgfqpoint{1.338048in}{3.189554in}}{\pgfqpoint{1.338048in}{3.178504in}}%
\pgfpathcurveto{\pgfqpoint{1.338048in}{3.167454in}}{\pgfqpoint{1.342438in}{3.156855in}}{\pgfqpoint{1.350252in}{3.149041in}}%
\pgfpathcurveto{\pgfqpoint{1.358065in}{3.141228in}}{\pgfqpoint{1.368664in}{3.136837in}}{\pgfqpoint{1.379714in}{3.136837in}}%
\pgfpathclose%
\pgfusepath{stroke,fill}%
\end{pgfscope}%
\begin{pgfscope}%
\pgfpathrectangle{\pgfqpoint{0.648703in}{0.548769in}}{\pgfqpoint{5.201297in}{3.102590in}}%
\pgfusepath{clip}%
\pgfsetbuttcap%
\pgfsetroundjoin%
\definecolor{currentfill}{rgb}{0.121569,0.466667,0.705882}%
\pgfsetfillcolor{currentfill}%
\pgfsetlinewidth{1.003750pt}%
\definecolor{currentstroke}{rgb}{0.121569,0.466667,0.705882}%
\pgfsetstrokecolor{currentstroke}%
\pgfsetdash{}{0pt}%
\pgfpathmoveto{\pgfqpoint{0.966261in}{0.648129in}}%
\pgfpathcurveto{\pgfqpoint{0.977311in}{0.648129in}}{\pgfqpoint{0.987910in}{0.652519in}}{\pgfqpoint{0.995724in}{0.660333in}}%
\pgfpathcurveto{\pgfqpoint{1.003538in}{0.668146in}}{\pgfqpoint{1.007928in}{0.678745in}}{\pgfqpoint{1.007928in}{0.689796in}}%
\pgfpathcurveto{\pgfqpoint{1.007928in}{0.700846in}}{\pgfqpoint{1.003538in}{0.711445in}}{\pgfqpoint{0.995724in}{0.719258in}}%
\pgfpathcurveto{\pgfqpoint{0.987910in}{0.727072in}}{\pgfqpoint{0.977311in}{0.731462in}}{\pgfqpoint{0.966261in}{0.731462in}}%
\pgfpathcurveto{\pgfqpoint{0.955211in}{0.731462in}}{\pgfqpoint{0.944612in}{0.727072in}}{\pgfqpoint{0.936798in}{0.719258in}}%
\pgfpathcurveto{\pgfqpoint{0.928985in}{0.711445in}}{\pgfqpoint{0.924594in}{0.700846in}}{\pgfqpoint{0.924594in}{0.689796in}}%
\pgfpathcurveto{\pgfqpoint{0.924594in}{0.678745in}}{\pgfqpoint{0.928985in}{0.668146in}}{\pgfqpoint{0.936798in}{0.660333in}}%
\pgfpathcurveto{\pgfqpoint{0.944612in}{0.652519in}}{\pgfqpoint{0.955211in}{0.648129in}}{\pgfqpoint{0.966261in}{0.648129in}}%
\pgfpathclose%
\pgfusepath{stroke,fill}%
\end{pgfscope}%
\begin{pgfscope}%
\pgfpathrectangle{\pgfqpoint{0.648703in}{0.548769in}}{\pgfqpoint{5.201297in}{3.102590in}}%
\pgfusepath{clip}%
\pgfsetbuttcap%
\pgfsetroundjoin%
\definecolor{currentfill}{rgb}{0.121569,0.466667,0.705882}%
\pgfsetfillcolor{currentfill}%
\pgfsetlinewidth{1.003750pt}%
\definecolor{currentstroke}{rgb}{0.121569,0.466667,0.705882}%
\pgfsetstrokecolor{currentstroke}%
\pgfsetdash{}{0pt}%
\pgfpathmoveto{\pgfqpoint{0.891985in}{0.648129in}}%
\pgfpathcurveto{\pgfqpoint{0.903035in}{0.648129in}}{\pgfqpoint{0.913634in}{0.652519in}}{\pgfqpoint{0.921448in}{0.660333in}}%
\pgfpathcurveto{\pgfqpoint{0.929262in}{0.668146in}}{\pgfqpoint{0.933652in}{0.678745in}}{\pgfqpoint{0.933652in}{0.689796in}}%
\pgfpathcurveto{\pgfqpoint{0.933652in}{0.700846in}}{\pgfqpoint{0.929262in}{0.711445in}}{\pgfqpoint{0.921448in}{0.719258in}}%
\pgfpathcurveto{\pgfqpoint{0.913634in}{0.727072in}}{\pgfqpoint{0.903035in}{0.731462in}}{\pgfqpoint{0.891985in}{0.731462in}}%
\pgfpathcurveto{\pgfqpoint{0.880935in}{0.731462in}}{\pgfqpoint{0.870336in}{0.727072in}}{\pgfqpoint{0.862522in}{0.719258in}}%
\pgfpathcurveto{\pgfqpoint{0.854709in}{0.711445in}}{\pgfqpoint{0.850319in}{0.700846in}}{\pgfqpoint{0.850319in}{0.689796in}}%
\pgfpathcurveto{\pgfqpoint{0.850319in}{0.678745in}}{\pgfqpoint{0.854709in}{0.668146in}}{\pgfqpoint{0.862522in}{0.660333in}}%
\pgfpathcurveto{\pgfqpoint{0.870336in}{0.652519in}}{\pgfqpoint{0.880935in}{0.648129in}}{\pgfqpoint{0.891985in}{0.648129in}}%
\pgfpathclose%
\pgfusepath{stroke,fill}%
\end{pgfscope}%
\begin{pgfscope}%
\pgfpathrectangle{\pgfqpoint{0.648703in}{0.548769in}}{\pgfqpoint{5.201297in}{3.102590in}}%
\pgfusepath{clip}%
\pgfsetbuttcap%
\pgfsetroundjoin%
\definecolor{currentfill}{rgb}{0.839216,0.152941,0.156863}%
\pgfsetfillcolor{currentfill}%
\pgfsetlinewidth{1.003750pt}%
\definecolor{currentstroke}{rgb}{0.839216,0.152941,0.156863}%
\pgfsetstrokecolor{currentstroke}%
\pgfsetdash{}{0pt}%
\pgfpathmoveto{\pgfqpoint{1.115688in}{3.410595in}}%
\pgfpathcurveto{\pgfqpoint{1.126738in}{3.410595in}}{\pgfqpoint{1.137337in}{3.414986in}}{\pgfqpoint{1.145151in}{3.422799in}}%
\pgfpathcurveto{\pgfqpoint{1.152964in}{3.430613in}}{\pgfqpoint{1.157354in}{3.441212in}}{\pgfqpoint{1.157354in}{3.452262in}}%
\pgfpathcurveto{\pgfqpoint{1.157354in}{3.463312in}}{\pgfqpoint{1.152964in}{3.473911in}}{\pgfqpoint{1.145151in}{3.481725in}}%
\pgfpathcurveto{\pgfqpoint{1.137337in}{3.489538in}}{\pgfqpoint{1.126738in}{3.493929in}}{\pgfqpoint{1.115688in}{3.493929in}}%
\pgfpathcurveto{\pgfqpoint{1.104638in}{3.493929in}}{\pgfqpoint{1.094039in}{3.489538in}}{\pgfqpoint{1.086225in}{3.481725in}}%
\pgfpathcurveto{\pgfqpoint{1.078411in}{3.473911in}}{\pgfqpoint{1.074021in}{3.463312in}}{\pgfqpoint{1.074021in}{3.452262in}}%
\pgfpathcurveto{\pgfqpoint{1.074021in}{3.441212in}}{\pgfqpoint{1.078411in}{3.430613in}}{\pgfqpoint{1.086225in}{3.422799in}}%
\pgfpathcurveto{\pgfqpoint{1.094039in}{3.414986in}}{\pgfqpoint{1.104638in}{3.410595in}}{\pgfqpoint{1.115688in}{3.410595in}}%
\pgfpathclose%
\pgfusepath{stroke,fill}%
\end{pgfscope}%
\begin{pgfscope}%
\pgfpathrectangle{\pgfqpoint{0.648703in}{0.548769in}}{\pgfqpoint{5.201297in}{3.102590in}}%
\pgfusepath{clip}%
\pgfsetbuttcap%
\pgfsetroundjoin%
\definecolor{currentfill}{rgb}{1.000000,0.498039,0.054902}%
\pgfsetfillcolor{currentfill}%
\pgfsetlinewidth{1.003750pt}%
\definecolor{currentstroke}{rgb}{1.000000,0.498039,0.054902}%
\pgfsetstrokecolor{currentstroke}%
\pgfsetdash{}{0pt}%
\pgfpathmoveto{\pgfqpoint{2.060879in}{3.128542in}}%
\pgfpathcurveto{\pgfqpoint{2.071929in}{3.128542in}}{\pgfqpoint{2.082528in}{3.132932in}}{\pgfqpoint{2.090342in}{3.140746in}}%
\pgfpathcurveto{\pgfqpoint{2.098156in}{3.148559in}}{\pgfqpoint{2.102546in}{3.159158in}}{\pgfqpoint{2.102546in}{3.170208in}}%
\pgfpathcurveto{\pgfqpoint{2.102546in}{3.181258in}}{\pgfqpoint{2.098156in}{3.191857in}}{\pgfqpoint{2.090342in}{3.199671in}}%
\pgfpathcurveto{\pgfqpoint{2.082528in}{3.207485in}}{\pgfqpoint{2.071929in}{3.211875in}}{\pgfqpoint{2.060879in}{3.211875in}}%
\pgfpathcurveto{\pgfqpoint{2.049829in}{3.211875in}}{\pgfqpoint{2.039230in}{3.207485in}}{\pgfqpoint{2.031417in}{3.199671in}}%
\pgfpathcurveto{\pgfqpoint{2.023603in}{3.191857in}}{\pgfqpoint{2.019213in}{3.181258in}}{\pgfqpoint{2.019213in}{3.170208in}}%
\pgfpathcurveto{\pgfqpoint{2.019213in}{3.159158in}}{\pgfqpoint{2.023603in}{3.148559in}}{\pgfqpoint{2.031417in}{3.140746in}}%
\pgfpathcurveto{\pgfqpoint{2.039230in}{3.132932in}}{\pgfqpoint{2.049829in}{3.128542in}}{\pgfqpoint{2.060879in}{3.128542in}}%
\pgfpathclose%
\pgfusepath{stroke,fill}%
\end{pgfscope}%
\begin{pgfscope}%
\pgfpathrectangle{\pgfqpoint{0.648703in}{0.548769in}}{\pgfqpoint{5.201297in}{3.102590in}}%
\pgfusepath{clip}%
\pgfsetbuttcap%
\pgfsetroundjoin%
\definecolor{currentfill}{rgb}{1.000000,0.498039,0.054902}%
\pgfsetfillcolor{currentfill}%
\pgfsetlinewidth{1.003750pt}%
\definecolor{currentstroke}{rgb}{1.000000,0.498039,0.054902}%
\pgfsetstrokecolor{currentstroke}%
\pgfsetdash{}{0pt}%
\pgfpathmoveto{\pgfqpoint{1.726131in}{3.315195in}}%
\pgfpathcurveto{\pgfqpoint{1.737181in}{3.315195in}}{\pgfqpoint{1.747780in}{3.319585in}}{\pgfqpoint{1.755594in}{3.327399in}}%
\pgfpathcurveto{\pgfqpoint{1.763407in}{3.335212in}}{\pgfqpoint{1.767798in}{3.345811in}}{\pgfqpoint{1.767798in}{3.356861in}}%
\pgfpathcurveto{\pgfqpoint{1.767798in}{3.367912in}}{\pgfqpoint{1.763407in}{3.378511in}}{\pgfqpoint{1.755594in}{3.386324in}}%
\pgfpathcurveto{\pgfqpoint{1.747780in}{3.394138in}}{\pgfqpoint{1.737181in}{3.398528in}}{\pgfqpoint{1.726131in}{3.398528in}}%
\pgfpathcurveto{\pgfqpoint{1.715081in}{3.398528in}}{\pgfqpoint{1.704482in}{3.394138in}}{\pgfqpoint{1.696668in}{3.386324in}}%
\pgfpathcurveto{\pgfqpoint{1.688855in}{3.378511in}}{\pgfqpoint{1.684464in}{3.367912in}}{\pgfqpoint{1.684464in}{3.356861in}}%
\pgfpathcurveto{\pgfqpoint{1.684464in}{3.345811in}}{\pgfqpoint{1.688855in}{3.335212in}}{\pgfqpoint{1.696668in}{3.327399in}}%
\pgfpathcurveto{\pgfqpoint{1.704482in}{3.319585in}}{\pgfqpoint{1.715081in}{3.315195in}}{\pgfqpoint{1.726131in}{3.315195in}}%
\pgfpathclose%
\pgfusepath{stroke,fill}%
\end{pgfscope}%
\begin{pgfscope}%
\pgfpathrectangle{\pgfqpoint{0.648703in}{0.548769in}}{\pgfqpoint{5.201297in}{3.102590in}}%
\pgfusepath{clip}%
\pgfsetbuttcap%
\pgfsetroundjoin%
\definecolor{currentfill}{rgb}{1.000000,0.498039,0.054902}%
\pgfsetfillcolor{currentfill}%
\pgfsetlinewidth{1.003750pt}%
\definecolor{currentstroke}{rgb}{1.000000,0.498039,0.054902}%
\pgfsetstrokecolor{currentstroke}%
\pgfsetdash{}{0pt}%
\pgfpathmoveto{\pgfqpoint{1.335753in}{3.136837in}}%
\pgfpathcurveto{\pgfqpoint{1.346803in}{3.136837in}}{\pgfqpoint{1.357402in}{3.141228in}}{\pgfqpoint{1.365216in}{3.149041in}}%
\pgfpathcurveto{\pgfqpoint{1.373029in}{3.156855in}}{\pgfqpoint{1.377420in}{3.167454in}}{\pgfqpoint{1.377420in}{3.178504in}}%
\pgfpathcurveto{\pgfqpoint{1.377420in}{3.189554in}}{\pgfqpoint{1.373029in}{3.200153in}}{\pgfqpoint{1.365216in}{3.207967in}}%
\pgfpathcurveto{\pgfqpoint{1.357402in}{3.215780in}}{\pgfqpoint{1.346803in}{3.220171in}}{\pgfqpoint{1.335753in}{3.220171in}}%
\pgfpathcurveto{\pgfqpoint{1.324703in}{3.220171in}}{\pgfqpoint{1.314104in}{3.215780in}}{\pgfqpoint{1.306290in}{3.207967in}}%
\pgfpathcurveto{\pgfqpoint{1.298476in}{3.200153in}}{\pgfqpoint{1.294086in}{3.189554in}}{\pgfqpoint{1.294086in}{3.178504in}}%
\pgfpathcurveto{\pgfqpoint{1.294086in}{3.167454in}}{\pgfqpoint{1.298476in}{3.156855in}}{\pgfqpoint{1.306290in}{3.149041in}}%
\pgfpathcurveto{\pgfqpoint{1.314104in}{3.141228in}}{\pgfqpoint{1.324703in}{3.136837in}}{\pgfqpoint{1.335753in}{3.136837in}}%
\pgfpathclose%
\pgfusepath{stroke,fill}%
\end{pgfscope}%
\begin{pgfscope}%
\pgfpathrectangle{\pgfqpoint{0.648703in}{0.548769in}}{\pgfqpoint{5.201297in}{3.102590in}}%
\pgfusepath{clip}%
\pgfsetbuttcap%
\pgfsetroundjoin%
\definecolor{currentfill}{rgb}{1.000000,0.498039,0.054902}%
\pgfsetfillcolor{currentfill}%
\pgfsetlinewidth{1.003750pt}%
\definecolor{currentstroke}{rgb}{1.000000,0.498039,0.054902}%
\pgfsetstrokecolor{currentstroke}%
\pgfsetdash{}{0pt}%
\pgfpathmoveto{\pgfqpoint{1.798345in}{3.136837in}}%
\pgfpathcurveto{\pgfqpoint{1.809395in}{3.136837in}}{\pgfqpoint{1.819994in}{3.141228in}}{\pgfqpoint{1.827808in}{3.149041in}}%
\pgfpathcurveto{\pgfqpoint{1.835621in}{3.156855in}}{\pgfqpoint{1.840011in}{3.167454in}}{\pgfqpoint{1.840011in}{3.178504in}}%
\pgfpathcurveto{\pgfqpoint{1.840011in}{3.189554in}}{\pgfqpoint{1.835621in}{3.200153in}}{\pgfqpoint{1.827808in}{3.207967in}}%
\pgfpathcurveto{\pgfqpoint{1.819994in}{3.215780in}}{\pgfqpoint{1.809395in}{3.220171in}}{\pgfqpoint{1.798345in}{3.220171in}}%
\pgfpathcurveto{\pgfqpoint{1.787295in}{3.220171in}}{\pgfqpoint{1.776696in}{3.215780in}}{\pgfqpoint{1.768882in}{3.207967in}}%
\pgfpathcurveto{\pgfqpoint{1.761068in}{3.200153in}}{\pgfqpoint{1.756678in}{3.189554in}}{\pgfqpoint{1.756678in}{3.178504in}}%
\pgfpathcurveto{\pgfqpoint{1.756678in}{3.167454in}}{\pgfqpoint{1.761068in}{3.156855in}}{\pgfqpoint{1.768882in}{3.149041in}}%
\pgfpathcurveto{\pgfqpoint{1.776696in}{3.141228in}}{\pgfqpoint{1.787295in}{3.136837in}}{\pgfqpoint{1.798345in}{3.136837in}}%
\pgfpathclose%
\pgfusepath{stroke,fill}%
\end{pgfscope}%
\begin{pgfscope}%
\pgfpathrectangle{\pgfqpoint{0.648703in}{0.548769in}}{\pgfqpoint{5.201297in}{3.102590in}}%
\pgfusepath{clip}%
\pgfsetbuttcap%
\pgfsetroundjoin%
\definecolor{currentfill}{rgb}{1.000000,0.498039,0.054902}%
\pgfsetfillcolor{currentfill}%
\pgfsetlinewidth{1.003750pt}%
\definecolor{currentstroke}{rgb}{1.000000,0.498039,0.054902}%
\pgfsetstrokecolor{currentstroke}%
\pgfsetdash{}{0pt}%
\pgfpathmoveto{\pgfqpoint{1.897398in}{3.140985in}}%
\pgfpathcurveto{\pgfqpoint{1.908448in}{3.140985in}}{\pgfqpoint{1.919047in}{3.145375in}}{\pgfqpoint{1.926861in}{3.153189in}}%
\pgfpathcurveto{\pgfqpoint{1.934674in}{3.161003in}}{\pgfqpoint{1.939065in}{3.171602in}}{\pgfqpoint{1.939065in}{3.182652in}}%
\pgfpathcurveto{\pgfqpoint{1.939065in}{3.193702in}}{\pgfqpoint{1.934674in}{3.204301in}}{\pgfqpoint{1.926861in}{3.212115in}}%
\pgfpathcurveto{\pgfqpoint{1.919047in}{3.219928in}}{\pgfqpoint{1.908448in}{3.224319in}}{\pgfqpoint{1.897398in}{3.224319in}}%
\pgfpathcurveto{\pgfqpoint{1.886348in}{3.224319in}}{\pgfqpoint{1.875749in}{3.219928in}}{\pgfqpoint{1.867935in}{3.212115in}}%
\pgfpathcurveto{\pgfqpoint{1.860121in}{3.204301in}}{\pgfqpoint{1.855731in}{3.193702in}}{\pgfqpoint{1.855731in}{3.182652in}}%
\pgfpathcurveto{\pgfqpoint{1.855731in}{3.171602in}}{\pgfqpoint{1.860121in}{3.161003in}}{\pgfqpoint{1.867935in}{3.153189in}}%
\pgfpathcurveto{\pgfqpoint{1.875749in}{3.145375in}}{\pgfqpoint{1.886348in}{3.140985in}}{\pgfqpoint{1.897398in}{3.140985in}}%
\pgfpathclose%
\pgfusepath{stroke,fill}%
\end{pgfscope}%
\begin{pgfscope}%
\pgfpathrectangle{\pgfqpoint{0.648703in}{0.548769in}}{\pgfqpoint{5.201297in}{3.102590in}}%
\pgfusepath{clip}%
\pgfsetbuttcap%
\pgfsetroundjoin%
\definecolor{currentfill}{rgb}{1.000000,0.498039,0.054902}%
\pgfsetfillcolor{currentfill}%
\pgfsetlinewidth{1.003750pt}%
\definecolor{currentstroke}{rgb}{1.000000,0.498039,0.054902}%
\pgfsetstrokecolor{currentstroke}%
\pgfsetdash{}{0pt}%
\pgfpathmoveto{\pgfqpoint{1.494607in}{3.136837in}}%
\pgfpathcurveto{\pgfqpoint{1.505658in}{3.136837in}}{\pgfqpoint{1.516257in}{3.141228in}}{\pgfqpoint{1.524070in}{3.149041in}}%
\pgfpathcurveto{\pgfqpoint{1.531884in}{3.156855in}}{\pgfqpoint{1.536274in}{3.167454in}}{\pgfqpoint{1.536274in}{3.178504in}}%
\pgfpathcurveto{\pgfqpoint{1.536274in}{3.189554in}}{\pgfqpoint{1.531884in}{3.200153in}}{\pgfqpoint{1.524070in}{3.207967in}}%
\pgfpathcurveto{\pgfqpoint{1.516257in}{3.215780in}}{\pgfqpoint{1.505658in}{3.220171in}}{\pgfqpoint{1.494607in}{3.220171in}}%
\pgfpathcurveto{\pgfqpoint{1.483557in}{3.220171in}}{\pgfqpoint{1.472958in}{3.215780in}}{\pgfqpoint{1.465145in}{3.207967in}}%
\pgfpathcurveto{\pgfqpoint{1.457331in}{3.200153in}}{\pgfqpoint{1.452941in}{3.189554in}}{\pgfqpoint{1.452941in}{3.178504in}}%
\pgfpathcurveto{\pgfqpoint{1.452941in}{3.167454in}}{\pgfqpoint{1.457331in}{3.156855in}}{\pgfqpoint{1.465145in}{3.149041in}}%
\pgfpathcurveto{\pgfqpoint{1.472958in}{3.141228in}}{\pgfqpoint{1.483557in}{3.136837in}}{\pgfqpoint{1.494607in}{3.136837in}}%
\pgfpathclose%
\pgfusepath{stroke,fill}%
\end{pgfscope}%
\begin{pgfscope}%
\pgfpathrectangle{\pgfqpoint{0.648703in}{0.548769in}}{\pgfqpoint{5.201297in}{3.102590in}}%
\pgfusepath{clip}%
\pgfsetbuttcap%
\pgfsetroundjoin%
\definecolor{currentfill}{rgb}{1.000000,0.498039,0.054902}%
\pgfsetfillcolor{currentfill}%
\pgfsetlinewidth{1.003750pt}%
\definecolor{currentstroke}{rgb}{1.000000,0.498039,0.054902}%
\pgfsetstrokecolor{currentstroke}%
\pgfsetdash{}{0pt}%
\pgfpathmoveto{\pgfqpoint{1.628170in}{3.356673in}}%
\pgfpathcurveto{\pgfqpoint{1.639220in}{3.356673in}}{\pgfqpoint{1.649819in}{3.361064in}}{\pgfqpoint{1.657633in}{3.368877in}}%
\pgfpathcurveto{\pgfqpoint{1.665447in}{3.376691in}}{\pgfqpoint{1.669837in}{3.387290in}}{\pgfqpoint{1.669837in}{3.398340in}}%
\pgfpathcurveto{\pgfqpoint{1.669837in}{3.409390in}}{\pgfqpoint{1.665447in}{3.419989in}}{\pgfqpoint{1.657633in}{3.427803in}}%
\pgfpathcurveto{\pgfqpoint{1.649819in}{3.435616in}}{\pgfqpoint{1.639220in}{3.440007in}}{\pgfqpoint{1.628170in}{3.440007in}}%
\pgfpathcurveto{\pgfqpoint{1.617120in}{3.440007in}}{\pgfqpoint{1.606521in}{3.435616in}}{\pgfqpoint{1.598708in}{3.427803in}}%
\pgfpathcurveto{\pgfqpoint{1.590894in}{3.419989in}}{\pgfqpoint{1.586504in}{3.409390in}}{\pgfqpoint{1.586504in}{3.398340in}}%
\pgfpathcurveto{\pgfqpoint{1.586504in}{3.387290in}}{\pgfqpoint{1.590894in}{3.376691in}}{\pgfqpoint{1.598708in}{3.368877in}}%
\pgfpathcurveto{\pgfqpoint{1.606521in}{3.361064in}}{\pgfqpoint{1.617120in}{3.356673in}}{\pgfqpoint{1.628170in}{3.356673in}}%
\pgfpathclose%
\pgfusepath{stroke,fill}%
\end{pgfscope}%
\begin{pgfscope}%
\pgfpathrectangle{\pgfqpoint{0.648703in}{0.548769in}}{\pgfqpoint{5.201297in}{3.102590in}}%
\pgfusepath{clip}%
\pgfsetbuttcap%
\pgfsetroundjoin%
\definecolor{currentfill}{rgb}{1.000000,0.498039,0.054902}%
\pgfsetfillcolor{currentfill}%
\pgfsetlinewidth{1.003750pt}%
\definecolor{currentstroke}{rgb}{1.000000,0.498039,0.054902}%
\pgfsetstrokecolor{currentstroke}%
\pgfsetdash{}{0pt}%
\pgfpathmoveto{\pgfqpoint{2.142280in}{3.219794in}}%
\pgfpathcurveto{\pgfqpoint{2.153330in}{3.219794in}}{\pgfqpoint{2.163929in}{3.224185in}}{\pgfqpoint{2.171742in}{3.231998in}}%
\pgfpathcurveto{\pgfqpoint{2.179556in}{3.239812in}}{\pgfqpoint{2.183946in}{3.250411in}}{\pgfqpoint{2.183946in}{3.261461in}}%
\pgfpathcurveto{\pgfqpoint{2.183946in}{3.272511in}}{\pgfqpoint{2.179556in}{3.283110in}}{\pgfqpoint{2.171742in}{3.290924in}}%
\pgfpathcurveto{\pgfqpoint{2.163929in}{3.298737in}}{\pgfqpoint{2.153330in}{3.303128in}}{\pgfqpoint{2.142280in}{3.303128in}}%
\pgfpathcurveto{\pgfqpoint{2.131230in}{3.303128in}}{\pgfqpoint{2.120631in}{3.298737in}}{\pgfqpoint{2.112817in}{3.290924in}}%
\pgfpathcurveto{\pgfqpoint{2.105003in}{3.283110in}}{\pgfqpoint{2.100613in}{3.272511in}}{\pgfqpoint{2.100613in}{3.261461in}}%
\pgfpathcurveto{\pgfqpoint{2.100613in}{3.250411in}}{\pgfqpoint{2.105003in}{3.239812in}}{\pgfqpoint{2.112817in}{3.231998in}}%
\pgfpathcurveto{\pgfqpoint{2.120631in}{3.224185in}}{\pgfqpoint{2.131230in}{3.219794in}}{\pgfqpoint{2.142280in}{3.219794in}}%
\pgfpathclose%
\pgfusepath{stroke,fill}%
\end{pgfscope}%
\begin{pgfscope}%
\pgfpathrectangle{\pgfqpoint{0.648703in}{0.548769in}}{\pgfqpoint{5.201297in}{3.102590in}}%
\pgfusepath{clip}%
\pgfsetbuttcap%
\pgfsetroundjoin%
\definecolor{currentfill}{rgb}{1.000000,0.498039,0.054902}%
\pgfsetfillcolor{currentfill}%
\pgfsetlinewidth{1.003750pt}%
\definecolor{currentstroke}{rgb}{1.000000,0.498039,0.054902}%
\pgfsetstrokecolor{currentstroke}%
\pgfsetdash{}{0pt}%
\pgfpathmoveto{\pgfqpoint{1.826728in}{3.120246in}}%
\pgfpathcurveto{\pgfqpoint{1.837778in}{3.120246in}}{\pgfqpoint{1.848377in}{3.124636in}}{\pgfqpoint{1.856190in}{3.132450in}}%
\pgfpathcurveto{\pgfqpoint{1.864004in}{3.140263in}}{\pgfqpoint{1.868394in}{3.150862in}}{\pgfqpoint{1.868394in}{3.161913in}}%
\pgfpathcurveto{\pgfqpoint{1.868394in}{3.172963in}}{\pgfqpoint{1.864004in}{3.183562in}}{\pgfqpoint{1.856190in}{3.191375in}}%
\pgfpathcurveto{\pgfqpoint{1.848377in}{3.199189in}}{\pgfqpoint{1.837778in}{3.203579in}}{\pgfqpoint{1.826728in}{3.203579in}}%
\pgfpathcurveto{\pgfqpoint{1.815678in}{3.203579in}}{\pgfqpoint{1.805078in}{3.199189in}}{\pgfqpoint{1.797265in}{3.191375in}}%
\pgfpathcurveto{\pgfqpoint{1.789451in}{3.183562in}}{\pgfqpoint{1.785061in}{3.172963in}}{\pgfqpoint{1.785061in}{3.161913in}}%
\pgfpathcurveto{\pgfqpoint{1.785061in}{3.150862in}}{\pgfqpoint{1.789451in}{3.140263in}}{\pgfqpoint{1.797265in}{3.132450in}}%
\pgfpathcurveto{\pgfqpoint{1.805078in}{3.124636in}}{\pgfqpoint{1.815678in}{3.120246in}}{\pgfqpoint{1.826728in}{3.120246in}}%
\pgfpathclose%
\pgfusepath{stroke,fill}%
\end{pgfscope}%
\begin{pgfscope}%
\pgfpathrectangle{\pgfqpoint{0.648703in}{0.548769in}}{\pgfqpoint{5.201297in}{3.102590in}}%
\pgfusepath{clip}%
\pgfsetbuttcap%
\pgfsetroundjoin%
\definecolor{currentfill}{rgb}{1.000000,0.498039,0.054902}%
\pgfsetfillcolor{currentfill}%
\pgfsetlinewidth{1.003750pt}%
\definecolor{currentstroke}{rgb}{1.000000,0.498039,0.054902}%
\pgfsetstrokecolor{currentstroke}%
\pgfsetdash{}{0pt}%
\pgfpathmoveto{\pgfqpoint{1.513543in}{3.140985in}}%
\pgfpathcurveto{\pgfqpoint{1.524593in}{3.140985in}}{\pgfqpoint{1.535192in}{3.145375in}}{\pgfqpoint{1.543005in}{3.153189in}}%
\pgfpathcurveto{\pgfqpoint{1.550819in}{3.161003in}}{\pgfqpoint{1.555209in}{3.171602in}}{\pgfqpoint{1.555209in}{3.182652in}}%
\pgfpathcurveto{\pgfqpoint{1.555209in}{3.193702in}}{\pgfqpoint{1.550819in}{3.204301in}}{\pgfqpoint{1.543005in}{3.212115in}}%
\pgfpathcurveto{\pgfqpoint{1.535192in}{3.219928in}}{\pgfqpoint{1.524593in}{3.224319in}}{\pgfqpoint{1.513543in}{3.224319in}}%
\pgfpathcurveto{\pgfqpoint{1.502492in}{3.224319in}}{\pgfqpoint{1.491893in}{3.219928in}}{\pgfqpoint{1.484080in}{3.212115in}}%
\pgfpathcurveto{\pgfqpoint{1.476266in}{3.204301in}}{\pgfqpoint{1.471876in}{3.193702in}}{\pgfqpoint{1.471876in}{3.182652in}}%
\pgfpathcurveto{\pgfqpoint{1.471876in}{3.171602in}}{\pgfqpoint{1.476266in}{3.161003in}}{\pgfqpoint{1.484080in}{3.153189in}}%
\pgfpathcurveto{\pgfqpoint{1.491893in}{3.145375in}}{\pgfqpoint{1.502492in}{3.140985in}}{\pgfqpoint{1.513543in}{3.140985in}}%
\pgfpathclose%
\pgfusepath{stroke,fill}%
\end{pgfscope}%
\begin{pgfscope}%
\pgfpathrectangle{\pgfqpoint{0.648703in}{0.548769in}}{\pgfqpoint{5.201297in}{3.102590in}}%
\pgfusepath{clip}%
\pgfsetbuttcap%
\pgfsetroundjoin%
\definecolor{currentfill}{rgb}{1.000000,0.498039,0.054902}%
\pgfsetfillcolor{currentfill}%
\pgfsetlinewidth{1.003750pt}%
\definecolor{currentstroke}{rgb}{1.000000,0.498039,0.054902}%
\pgfsetstrokecolor{currentstroke}%
\pgfsetdash{}{0pt}%
\pgfpathmoveto{\pgfqpoint{1.504111in}{3.145133in}}%
\pgfpathcurveto{\pgfqpoint{1.515161in}{3.145133in}}{\pgfqpoint{1.525760in}{3.149523in}}{\pgfqpoint{1.533573in}{3.157337in}}%
\pgfpathcurveto{\pgfqpoint{1.541387in}{3.165151in}}{\pgfqpoint{1.545777in}{3.175750in}}{\pgfqpoint{1.545777in}{3.186800in}}%
\pgfpathcurveto{\pgfqpoint{1.545777in}{3.197850in}}{\pgfqpoint{1.541387in}{3.208449in}}{\pgfqpoint{1.533573in}{3.216262in}}%
\pgfpathcurveto{\pgfqpoint{1.525760in}{3.224076in}}{\pgfqpoint{1.515161in}{3.228466in}}{\pgfqpoint{1.504111in}{3.228466in}}%
\pgfpathcurveto{\pgfqpoint{1.493060in}{3.228466in}}{\pgfqpoint{1.482461in}{3.224076in}}{\pgfqpoint{1.474648in}{3.216262in}}%
\pgfpathcurveto{\pgfqpoint{1.466834in}{3.208449in}}{\pgfqpoint{1.462444in}{3.197850in}}{\pgfqpoint{1.462444in}{3.186800in}}%
\pgfpathcurveto{\pgfqpoint{1.462444in}{3.175750in}}{\pgfqpoint{1.466834in}{3.165151in}}{\pgfqpoint{1.474648in}{3.157337in}}%
\pgfpathcurveto{\pgfqpoint{1.482461in}{3.149523in}}{\pgfqpoint{1.493060in}{3.145133in}}{\pgfqpoint{1.504111in}{3.145133in}}%
\pgfpathclose%
\pgfusepath{stroke,fill}%
\end{pgfscope}%
\begin{pgfscope}%
\pgfpathrectangle{\pgfqpoint{0.648703in}{0.548769in}}{\pgfqpoint{5.201297in}{3.102590in}}%
\pgfusepath{clip}%
\pgfsetbuttcap%
\pgfsetroundjoin%
\definecolor{currentfill}{rgb}{1.000000,0.498039,0.054902}%
\pgfsetfillcolor{currentfill}%
\pgfsetlinewidth{1.003750pt}%
\definecolor{currentstroke}{rgb}{1.000000,0.498039,0.054902}%
\pgfsetstrokecolor{currentstroke}%
\pgfsetdash{}{0pt}%
\pgfpathmoveto{\pgfqpoint{1.769459in}{3.136837in}}%
\pgfpathcurveto{\pgfqpoint{1.780509in}{3.136837in}}{\pgfqpoint{1.791108in}{3.141228in}}{\pgfqpoint{1.798922in}{3.149041in}}%
\pgfpathcurveto{\pgfqpoint{1.806736in}{3.156855in}}{\pgfqpoint{1.811126in}{3.167454in}}{\pgfqpoint{1.811126in}{3.178504in}}%
\pgfpathcurveto{\pgfqpoint{1.811126in}{3.189554in}}{\pgfqpoint{1.806736in}{3.200153in}}{\pgfqpoint{1.798922in}{3.207967in}}%
\pgfpathcurveto{\pgfqpoint{1.791108in}{3.215780in}}{\pgfqpoint{1.780509in}{3.220171in}}{\pgfqpoint{1.769459in}{3.220171in}}%
\pgfpathcurveto{\pgfqpoint{1.758409in}{3.220171in}}{\pgfqpoint{1.747810in}{3.215780in}}{\pgfqpoint{1.739996in}{3.207967in}}%
\pgfpathcurveto{\pgfqpoint{1.732183in}{3.200153in}}{\pgfqpoint{1.727793in}{3.189554in}}{\pgfqpoint{1.727793in}{3.178504in}}%
\pgfpathcurveto{\pgfqpoint{1.727793in}{3.167454in}}{\pgfqpoint{1.732183in}{3.156855in}}{\pgfqpoint{1.739996in}{3.149041in}}%
\pgfpathcurveto{\pgfqpoint{1.747810in}{3.141228in}}{\pgfqpoint{1.758409in}{3.136837in}}{\pgfqpoint{1.769459in}{3.136837in}}%
\pgfpathclose%
\pgfusepath{stroke,fill}%
\end{pgfscope}%
\begin{pgfscope}%
\pgfpathrectangle{\pgfqpoint{0.648703in}{0.548769in}}{\pgfqpoint{5.201297in}{3.102590in}}%
\pgfusepath{clip}%
\pgfsetbuttcap%
\pgfsetroundjoin%
\definecolor{currentfill}{rgb}{1.000000,0.498039,0.054902}%
\pgfsetfillcolor{currentfill}%
\pgfsetlinewidth{1.003750pt}%
\definecolor{currentstroke}{rgb}{1.000000,0.498039,0.054902}%
\pgfsetstrokecolor{currentstroke}%
\pgfsetdash{}{0pt}%
\pgfpathmoveto{\pgfqpoint{2.454174in}{3.132690in}}%
\pgfpathcurveto{\pgfqpoint{2.465225in}{3.132690in}}{\pgfqpoint{2.475824in}{3.137080in}}{\pgfqpoint{2.483637in}{3.144893in}}%
\pgfpathcurveto{\pgfqpoint{2.491451in}{3.152707in}}{\pgfqpoint{2.495841in}{3.163306in}}{\pgfqpoint{2.495841in}{3.174356in}}%
\pgfpathcurveto{\pgfqpoint{2.495841in}{3.185406in}}{\pgfqpoint{2.491451in}{3.196005in}}{\pgfqpoint{2.483637in}{3.203819in}}%
\pgfpathcurveto{\pgfqpoint{2.475824in}{3.211633in}}{\pgfqpoint{2.465225in}{3.216023in}}{\pgfqpoint{2.454174in}{3.216023in}}%
\pgfpathcurveto{\pgfqpoint{2.443124in}{3.216023in}}{\pgfqpoint{2.432525in}{3.211633in}}{\pgfqpoint{2.424712in}{3.203819in}}%
\pgfpathcurveto{\pgfqpoint{2.416898in}{3.196005in}}{\pgfqpoint{2.412508in}{3.185406in}}{\pgfqpoint{2.412508in}{3.174356in}}%
\pgfpathcurveto{\pgfqpoint{2.412508in}{3.163306in}}{\pgfqpoint{2.416898in}{3.152707in}}{\pgfqpoint{2.424712in}{3.144893in}}%
\pgfpathcurveto{\pgfqpoint{2.432525in}{3.137080in}}{\pgfqpoint{2.443124in}{3.132690in}}{\pgfqpoint{2.454174in}{3.132690in}}%
\pgfpathclose%
\pgfusepath{stroke,fill}%
\end{pgfscope}%
\begin{pgfscope}%
\pgfpathrectangle{\pgfqpoint{0.648703in}{0.548769in}}{\pgfqpoint{5.201297in}{3.102590in}}%
\pgfusepath{clip}%
\pgfsetbuttcap%
\pgfsetroundjoin%
\definecolor{currentfill}{rgb}{1.000000,0.498039,0.054902}%
\pgfsetfillcolor{currentfill}%
\pgfsetlinewidth{1.003750pt}%
\definecolor{currentstroke}{rgb}{1.000000,0.498039,0.054902}%
\pgfsetstrokecolor{currentstroke}%
\pgfsetdash{}{0pt}%
\pgfpathmoveto{\pgfqpoint{1.940315in}{3.132690in}}%
\pgfpathcurveto{\pgfqpoint{1.951365in}{3.132690in}}{\pgfqpoint{1.961964in}{3.137080in}}{\pgfqpoint{1.969777in}{3.144893in}}%
\pgfpathcurveto{\pgfqpoint{1.977591in}{3.152707in}}{\pgfqpoint{1.981981in}{3.163306in}}{\pgfqpoint{1.981981in}{3.174356in}}%
\pgfpathcurveto{\pgfqpoint{1.981981in}{3.185406in}}{\pgfqpoint{1.977591in}{3.196005in}}{\pgfqpoint{1.969777in}{3.203819in}}%
\pgfpathcurveto{\pgfqpoint{1.961964in}{3.211633in}}{\pgfqpoint{1.951365in}{3.216023in}}{\pgfqpoint{1.940315in}{3.216023in}}%
\pgfpathcurveto{\pgfqpoint{1.929264in}{3.216023in}}{\pgfqpoint{1.918665in}{3.211633in}}{\pgfqpoint{1.910852in}{3.203819in}}%
\pgfpathcurveto{\pgfqpoint{1.903038in}{3.196005in}}{\pgfqpoint{1.898648in}{3.185406in}}{\pgfqpoint{1.898648in}{3.174356in}}%
\pgfpathcurveto{\pgfqpoint{1.898648in}{3.163306in}}{\pgfqpoint{1.903038in}{3.152707in}}{\pgfqpoint{1.910852in}{3.144893in}}%
\pgfpathcurveto{\pgfqpoint{1.918665in}{3.137080in}}{\pgfqpoint{1.929264in}{3.132690in}}{\pgfqpoint{1.940315in}{3.132690in}}%
\pgfpathclose%
\pgfusepath{stroke,fill}%
\end{pgfscope}%
\begin{pgfscope}%
\pgfpathrectangle{\pgfqpoint{0.648703in}{0.548769in}}{\pgfqpoint{5.201297in}{3.102590in}}%
\pgfusepath{clip}%
\pgfsetbuttcap%
\pgfsetroundjoin%
\definecolor{currentfill}{rgb}{1.000000,0.498039,0.054902}%
\pgfsetfillcolor{currentfill}%
\pgfsetlinewidth{1.003750pt}%
\definecolor{currentstroke}{rgb}{1.000000,0.498039,0.054902}%
\pgfsetstrokecolor{currentstroke}%
\pgfsetdash{}{0pt}%
\pgfpathmoveto{\pgfqpoint{2.129642in}{3.136837in}}%
\pgfpathcurveto{\pgfqpoint{2.140692in}{3.136837in}}{\pgfqpoint{2.151291in}{3.141228in}}{\pgfqpoint{2.159105in}{3.149041in}}%
\pgfpathcurveto{\pgfqpoint{2.166918in}{3.156855in}}{\pgfqpoint{2.171308in}{3.167454in}}{\pgfqpoint{2.171308in}{3.178504in}}%
\pgfpathcurveto{\pgfqpoint{2.171308in}{3.189554in}}{\pgfqpoint{2.166918in}{3.200153in}}{\pgfqpoint{2.159105in}{3.207967in}}%
\pgfpathcurveto{\pgfqpoint{2.151291in}{3.215780in}}{\pgfqpoint{2.140692in}{3.220171in}}{\pgfqpoint{2.129642in}{3.220171in}}%
\pgfpathcurveto{\pgfqpoint{2.118592in}{3.220171in}}{\pgfqpoint{2.107993in}{3.215780in}}{\pgfqpoint{2.100179in}{3.207967in}}%
\pgfpathcurveto{\pgfqpoint{2.092365in}{3.200153in}}{\pgfqpoint{2.087975in}{3.189554in}}{\pgfqpoint{2.087975in}{3.178504in}}%
\pgfpathcurveto{\pgfqpoint{2.087975in}{3.167454in}}{\pgfqpoint{2.092365in}{3.156855in}}{\pgfqpoint{2.100179in}{3.149041in}}%
\pgfpathcurveto{\pgfqpoint{2.107993in}{3.141228in}}{\pgfqpoint{2.118592in}{3.136837in}}{\pgfqpoint{2.129642in}{3.136837in}}%
\pgfpathclose%
\pgfusepath{stroke,fill}%
\end{pgfscope}%
\begin{pgfscope}%
\pgfpathrectangle{\pgfqpoint{0.648703in}{0.548769in}}{\pgfqpoint{5.201297in}{3.102590in}}%
\pgfusepath{clip}%
\pgfsetbuttcap%
\pgfsetroundjoin%
\definecolor{currentfill}{rgb}{0.839216,0.152941,0.156863}%
\pgfsetfillcolor{currentfill}%
\pgfsetlinewidth{1.003750pt}%
\definecolor{currentstroke}{rgb}{0.839216,0.152941,0.156863}%
\pgfsetstrokecolor{currentstroke}%
\pgfsetdash{}{0pt}%
\pgfpathmoveto{\pgfqpoint{2.312050in}{3.120246in}}%
\pgfpathcurveto{\pgfqpoint{2.323101in}{3.120246in}}{\pgfqpoint{2.333700in}{3.124636in}}{\pgfqpoint{2.341513in}{3.132450in}}%
\pgfpathcurveto{\pgfqpoint{2.349327in}{3.140263in}}{\pgfqpoint{2.353717in}{3.150862in}}{\pgfqpoint{2.353717in}{3.161913in}}%
\pgfpathcurveto{\pgfqpoint{2.353717in}{3.172963in}}{\pgfqpoint{2.349327in}{3.183562in}}{\pgfqpoint{2.341513in}{3.191375in}}%
\pgfpathcurveto{\pgfqpoint{2.333700in}{3.199189in}}{\pgfqpoint{2.323101in}{3.203579in}}{\pgfqpoint{2.312050in}{3.203579in}}%
\pgfpathcurveto{\pgfqpoint{2.301000in}{3.203579in}}{\pgfqpoint{2.290401in}{3.199189in}}{\pgfqpoint{2.282588in}{3.191375in}}%
\pgfpathcurveto{\pgfqpoint{2.274774in}{3.183562in}}{\pgfqpoint{2.270384in}{3.172963in}}{\pgfqpoint{2.270384in}{3.161913in}}%
\pgfpathcurveto{\pgfqpoint{2.270384in}{3.150862in}}{\pgfqpoint{2.274774in}{3.140263in}}{\pgfqpoint{2.282588in}{3.132450in}}%
\pgfpathcurveto{\pgfqpoint{2.290401in}{3.124636in}}{\pgfqpoint{2.301000in}{3.120246in}}{\pgfqpoint{2.312050in}{3.120246in}}%
\pgfpathclose%
\pgfusepath{stroke,fill}%
\end{pgfscope}%
\begin{pgfscope}%
\pgfpathrectangle{\pgfqpoint{0.648703in}{0.548769in}}{\pgfqpoint{5.201297in}{3.102590in}}%
\pgfusepath{clip}%
\pgfsetbuttcap%
\pgfsetroundjoin%
\definecolor{currentfill}{rgb}{1.000000,0.498039,0.054902}%
\pgfsetfillcolor{currentfill}%
\pgfsetlinewidth{1.003750pt}%
\definecolor{currentstroke}{rgb}{1.000000,0.498039,0.054902}%
\pgfsetstrokecolor{currentstroke}%
\pgfsetdash{}{0pt}%
\pgfpathmoveto{\pgfqpoint{1.396477in}{3.244681in}}%
\pgfpathcurveto{\pgfqpoint{1.407527in}{3.244681in}}{\pgfqpoint{1.418126in}{3.249072in}}{\pgfqpoint{1.425939in}{3.256885in}}%
\pgfpathcurveto{\pgfqpoint{1.433753in}{3.264699in}}{\pgfqpoint{1.438143in}{3.275298in}}{\pgfqpoint{1.438143in}{3.286348in}}%
\pgfpathcurveto{\pgfqpoint{1.438143in}{3.297398in}}{\pgfqpoint{1.433753in}{3.307997in}}{\pgfqpoint{1.425939in}{3.315811in}}%
\pgfpathcurveto{\pgfqpoint{1.418126in}{3.323624in}}{\pgfqpoint{1.407527in}{3.328015in}}{\pgfqpoint{1.396477in}{3.328015in}}%
\pgfpathcurveto{\pgfqpoint{1.385426in}{3.328015in}}{\pgfqpoint{1.374827in}{3.323624in}}{\pgfqpoint{1.367014in}{3.315811in}}%
\pgfpathcurveto{\pgfqpoint{1.359200in}{3.307997in}}{\pgfqpoint{1.354810in}{3.297398in}}{\pgfqpoint{1.354810in}{3.286348in}}%
\pgfpathcurveto{\pgfqpoint{1.354810in}{3.275298in}}{\pgfqpoint{1.359200in}{3.264699in}}{\pgfqpoint{1.367014in}{3.256885in}}%
\pgfpathcurveto{\pgfqpoint{1.374827in}{3.249072in}}{\pgfqpoint{1.385426in}{3.244681in}}{\pgfqpoint{1.396477in}{3.244681in}}%
\pgfpathclose%
\pgfusepath{stroke,fill}%
\end{pgfscope}%
\begin{pgfscope}%
\pgfpathrectangle{\pgfqpoint{0.648703in}{0.548769in}}{\pgfqpoint{5.201297in}{3.102590in}}%
\pgfusepath{clip}%
\pgfsetbuttcap%
\pgfsetroundjoin%
\definecolor{currentfill}{rgb}{1.000000,0.498039,0.054902}%
\pgfsetfillcolor{currentfill}%
\pgfsetlinewidth{1.003750pt}%
\definecolor{currentstroke}{rgb}{1.000000,0.498039,0.054902}%
\pgfsetstrokecolor{currentstroke}%
\pgfsetdash{}{0pt}%
\pgfpathmoveto{\pgfqpoint{2.760639in}{3.128542in}}%
\pgfpathcurveto{\pgfqpoint{2.771689in}{3.128542in}}{\pgfqpoint{2.782288in}{3.132932in}}{\pgfqpoint{2.790102in}{3.140746in}}%
\pgfpathcurveto{\pgfqpoint{2.797915in}{3.148559in}}{\pgfqpoint{2.802306in}{3.159158in}}{\pgfqpoint{2.802306in}{3.170208in}}%
\pgfpathcurveto{\pgfqpoint{2.802306in}{3.181258in}}{\pgfqpoint{2.797915in}{3.191857in}}{\pgfqpoint{2.790102in}{3.199671in}}%
\pgfpathcurveto{\pgfqpoint{2.782288in}{3.207485in}}{\pgfqpoint{2.771689in}{3.211875in}}{\pgfqpoint{2.760639in}{3.211875in}}%
\pgfpathcurveto{\pgfqpoint{2.749589in}{3.211875in}}{\pgfqpoint{2.738990in}{3.207485in}}{\pgfqpoint{2.731176in}{3.199671in}}%
\pgfpathcurveto{\pgfqpoint{2.723363in}{3.191857in}}{\pgfqpoint{2.718972in}{3.181258in}}{\pgfqpoint{2.718972in}{3.170208in}}%
\pgfpathcurveto{\pgfqpoint{2.718972in}{3.159158in}}{\pgfqpoint{2.723363in}{3.148559in}}{\pgfqpoint{2.731176in}{3.140746in}}%
\pgfpathcurveto{\pgfqpoint{2.738990in}{3.132932in}}{\pgfqpoint{2.749589in}{3.128542in}}{\pgfqpoint{2.760639in}{3.128542in}}%
\pgfpathclose%
\pgfusepath{stroke,fill}%
\end{pgfscope}%
\begin{pgfscope}%
\pgfpathrectangle{\pgfqpoint{0.648703in}{0.548769in}}{\pgfqpoint{5.201297in}{3.102590in}}%
\pgfusepath{clip}%
\pgfsetbuttcap%
\pgfsetroundjoin%
\definecolor{currentfill}{rgb}{1.000000,0.498039,0.054902}%
\pgfsetfillcolor{currentfill}%
\pgfsetlinewidth{1.003750pt}%
\definecolor{currentstroke}{rgb}{1.000000,0.498039,0.054902}%
\pgfsetstrokecolor{currentstroke}%
\pgfsetdash{}{0pt}%
\pgfpathmoveto{\pgfqpoint{1.942400in}{3.136837in}}%
\pgfpathcurveto{\pgfqpoint{1.953451in}{3.136837in}}{\pgfqpoint{1.964050in}{3.141228in}}{\pgfqpoint{1.971863in}{3.149041in}}%
\pgfpathcurveto{\pgfqpoint{1.979677in}{3.156855in}}{\pgfqpoint{1.984067in}{3.167454in}}{\pgfqpoint{1.984067in}{3.178504in}}%
\pgfpathcurveto{\pgfqpoint{1.984067in}{3.189554in}}{\pgfqpoint{1.979677in}{3.200153in}}{\pgfqpoint{1.971863in}{3.207967in}}%
\pgfpathcurveto{\pgfqpoint{1.964050in}{3.215780in}}{\pgfqpoint{1.953451in}{3.220171in}}{\pgfqpoint{1.942400in}{3.220171in}}%
\pgfpathcurveto{\pgfqpoint{1.931350in}{3.220171in}}{\pgfqpoint{1.920751in}{3.215780in}}{\pgfqpoint{1.912938in}{3.207967in}}%
\pgfpathcurveto{\pgfqpoint{1.905124in}{3.200153in}}{\pgfqpoint{1.900734in}{3.189554in}}{\pgfqpoint{1.900734in}{3.178504in}}%
\pgfpathcurveto{\pgfqpoint{1.900734in}{3.167454in}}{\pgfqpoint{1.905124in}{3.156855in}}{\pgfqpoint{1.912938in}{3.149041in}}%
\pgfpathcurveto{\pgfqpoint{1.920751in}{3.141228in}}{\pgfqpoint{1.931350in}{3.136837in}}{\pgfqpoint{1.942400in}{3.136837in}}%
\pgfpathclose%
\pgfusepath{stroke,fill}%
\end{pgfscope}%
\begin{pgfscope}%
\pgfpathrectangle{\pgfqpoint{0.648703in}{0.548769in}}{\pgfqpoint{5.201297in}{3.102590in}}%
\pgfusepath{clip}%
\pgfsetbuttcap%
\pgfsetroundjoin%
\definecolor{currentfill}{rgb}{1.000000,0.498039,0.054902}%
\pgfsetfillcolor{currentfill}%
\pgfsetlinewidth{1.003750pt}%
\definecolor{currentstroke}{rgb}{1.000000,0.498039,0.054902}%
\pgfsetstrokecolor{currentstroke}%
\pgfsetdash{}{0pt}%
\pgfpathmoveto{\pgfqpoint{1.942286in}{3.136837in}}%
\pgfpathcurveto{\pgfqpoint{1.953336in}{3.136837in}}{\pgfqpoint{1.963935in}{3.141228in}}{\pgfqpoint{1.971748in}{3.149041in}}%
\pgfpathcurveto{\pgfqpoint{1.979562in}{3.156855in}}{\pgfqpoint{1.983952in}{3.167454in}}{\pgfqpoint{1.983952in}{3.178504in}}%
\pgfpathcurveto{\pgfqpoint{1.983952in}{3.189554in}}{\pgfqpoint{1.979562in}{3.200153in}}{\pgfqpoint{1.971748in}{3.207967in}}%
\pgfpathcurveto{\pgfqpoint{1.963935in}{3.215780in}}{\pgfqpoint{1.953336in}{3.220171in}}{\pgfqpoint{1.942286in}{3.220171in}}%
\pgfpathcurveto{\pgfqpoint{1.931235in}{3.220171in}}{\pgfqpoint{1.920636in}{3.215780in}}{\pgfqpoint{1.912823in}{3.207967in}}%
\pgfpathcurveto{\pgfqpoint{1.905009in}{3.200153in}}{\pgfqpoint{1.900619in}{3.189554in}}{\pgfqpoint{1.900619in}{3.178504in}}%
\pgfpathcurveto{\pgfqpoint{1.900619in}{3.167454in}}{\pgfqpoint{1.905009in}{3.156855in}}{\pgfqpoint{1.912823in}{3.149041in}}%
\pgfpathcurveto{\pgfqpoint{1.920636in}{3.141228in}}{\pgfqpoint{1.931235in}{3.136837in}}{\pgfqpoint{1.942286in}{3.136837in}}%
\pgfpathclose%
\pgfusepath{stroke,fill}%
\end{pgfscope}%
\begin{pgfscope}%
\pgfpathrectangle{\pgfqpoint{0.648703in}{0.548769in}}{\pgfqpoint{5.201297in}{3.102590in}}%
\pgfusepath{clip}%
\pgfsetbuttcap%
\pgfsetroundjoin%
\definecolor{currentfill}{rgb}{1.000000,0.498039,0.054902}%
\pgfsetfillcolor{currentfill}%
\pgfsetlinewidth{1.003750pt}%
\definecolor{currentstroke}{rgb}{1.000000,0.498039,0.054902}%
\pgfsetstrokecolor{currentstroke}%
\pgfsetdash{}{0pt}%
\pgfpathmoveto{\pgfqpoint{2.434555in}{3.136837in}}%
\pgfpathcurveto{\pgfqpoint{2.445605in}{3.136837in}}{\pgfqpoint{2.456204in}{3.141228in}}{\pgfqpoint{2.464017in}{3.149041in}}%
\pgfpathcurveto{\pgfqpoint{2.471831in}{3.156855in}}{\pgfqpoint{2.476221in}{3.167454in}}{\pgfqpoint{2.476221in}{3.178504in}}%
\pgfpathcurveto{\pgfqpoint{2.476221in}{3.189554in}}{\pgfqpoint{2.471831in}{3.200153in}}{\pgfqpoint{2.464017in}{3.207967in}}%
\pgfpathcurveto{\pgfqpoint{2.456204in}{3.215780in}}{\pgfqpoint{2.445605in}{3.220171in}}{\pgfqpoint{2.434555in}{3.220171in}}%
\pgfpathcurveto{\pgfqpoint{2.423505in}{3.220171in}}{\pgfqpoint{2.412905in}{3.215780in}}{\pgfqpoint{2.405092in}{3.207967in}}%
\pgfpathcurveto{\pgfqpoint{2.397278in}{3.200153in}}{\pgfqpoint{2.392888in}{3.189554in}}{\pgfqpoint{2.392888in}{3.178504in}}%
\pgfpathcurveto{\pgfqpoint{2.392888in}{3.167454in}}{\pgfqpoint{2.397278in}{3.156855in}}{\pgfqpoint{2.405092in}{3.149041in}}%
\pgfpathcurveto{\pgfqpoint{2.412905in}{3.141228in}}{\pgfqpoint{2.423505in}{3.136837in}}{\pgfqpoint{2.434555in}{3.136837in}}%
\pgfpathclose%
\pgfusepath{stroke,fill}%
\end{pgfscope}%
\begin{pgfscope}%
\pgfpathrectangle{\pgfqpoint{0.648703in}{0.548769in}}{\pgfqpoint{5.201297in}{3.102590in}}%
\pgfusepath{clip}%
\pgfsetbuttcap%
\pgfsetroundjoin%
\definecolor{currentfill}{rgb}{1.000000,0.498039,0.054902}%
\pgfsetfillcolor{currentfill}%
\pgfsetlinewidth{1.003750pt}%
\definecolor{currentstroke}{rgb}{1.000000,0.498039,0.054902}%
\pgfsetstrokecolor{currentstroke}%
\pgfsetdash{}{0pt}%
\pgfpathmoveto{\pgfqpoint{2.079886in}{3.132690in}}%
\pgfpathcurveto{\pgfqpoint{2.090936in}{3.132690in}}{\pgfqpoint{2.101535in}{3.137080in}}{\pgfqpoint{2.109348in}{3.144893in}}%
\pgfpathcurveto{\pgfqpoint{2.117162in}{3.152707in}}{\pgfqpoint{2.121552in}{3.163306in}}{\pgfqpoint{2.121552in}{3.174356in}}%
\pgfpathcurveto{\pgfqpoint{2.121552in}{3.185406in}}{\pgfqpoint{2.117162in}{3.196005in}}{\pgfqpoint{2.109348in}{3.203819in}}%
\pgfpathcurveto{\pgfqpoint{2.101535in}{3.211633in}}{\pgfqpoint{2.090936in}{3.216023in}}{\pgfqpoint{2.079886in}{3.216023in}}%
\pgfpathcurveto{\pgfqpoint{2.068836in}{3.216023in}}{\pgfqpoint{2.058237in}{3.211633in}}{\pgfqpoint{2.050423in}{3.203819in}}%
\pgfpathcurveto{\pgfqpoint{2.042609in}{3.196005in}}{\pgfqpoint{2.038219in}{3.185406in}}{\pgfqpoint{2.038219in}{3.174356in}}%
\pgfpathcurveto{\pgfqpoint{2.038219in}{3.163306in}}{\pgfqpoint{2.042609in}{3.152707in}}{\pgfqpoint{2.050423in}{3.144893in}}%
\pgfpathcurveto{\pgfqpoint{2.058237in}{3.137080in}}{\pgfqpoint{2.068836in}{3.132690in}}{\pgfqpoint{2.079886in}{3.132690in}}%
\pgfpathclose%
\pgfusepath{stroke,fill}%
\end{pgfscope}%
\begin{pgfscope}%
\pgfpathrectangle{\pgfqpoint{0.648703in}{0.548769in}}{\pgfqpoint{5.201297in}{3.102590in}}%
\pgfusepath{clip}%
\pgfsetbuttcap%
\pgfsetroundjoin%
\definecolor{currentfill}{rgb}{1.000000,0.498039,0.054902}%
\pgfsetfillcolor{currentfill}%
\pgfsetlinewidth{1.003750pt}%
\definecolor{currentstroke}{rgb}{1.000000,0.498039,0.054902}%
\pgfsetstrokecolor{currentstroke}%
\pgfsetdash{}{0pt}%
\pgfpathmoveto{\pgfqpoint{2.143681in}{3.107802in}}%
\pgfpathcurveto{\pgfqpoint{2.154731in}{3.107802in}}{\pgfqpoint{2.165330in}{3.112193in}}{\pgfqpoint{2.173144in}{3.120006in}}%
\pgfpathcurveto{\pgfqpoint{2.180957in}{3.127820in}}{\pgfqpoint{2.185347in}{3.138419in}}{\pgfqpoint{2.185347in}{3.149469in}}%
\pgfpathcurveto{\pgfqpoint{2.185347in}{3.160519in}}{\pgfqpoint{2.180957in}{3.171118in}}{\pgfqpoint{2.173144in}{3.178932in}}%
\pgfpathcurveto{\pgfqpoint{2.165330in}{3.186745in}}{\pgfqpoint{2.154731in}{3.191136in}}{\pgfqpoint{2.143681in}{3.191136in}}%
\pgfpathcurveto{\pgfqpoint{2.132631in}{3.191136in}}{\pgfqpoint{2.122032in}{3.186745in}}{\pgfqpoint{2.114218in}{3.178932in}}%
\pgfpathcurveto{\pgfqpoint{2.106404in}{3.171118in}}{\pgfqpoint{2.102014in}{3.160519in}}{\pgfqpoint{2.102014in}{3.149469in}}%
\pgfpathcurveto{\pgfqpoint{2.102014in}{3.138419in}}{\pgfqpoint{2.106404in}{3.127820in}}{\pgfqpoint{2.114218in}{3.120006in}}%
\pgfpathcurveto{\pgfqpoint{2.122032in}{3.112193in}}{\pgfqpoint{2.132631in}{3.107802in}}{\pgfqpoint{2.143681in}{3.107802in}}%
\pgfpathclose%
\pgfusepath{stroke,fill}%
\end{pgfscope}%
\begin{pgfscope}%
\pgfpathrectangle{\pgfqpoint{0.648703in}{0.548769in}}{\pgfqpoint{5.201297in}{3.102590in}}%
\pgfusepath{clip}%
\pgfsetbuttcap%
\pgfsetroundjoin%
\definecolor{currentfill}{rgb}{1.000000,0.498039,0.054902}%
\pgfsetfillcolor{currentfill}%
\pgfsetlinewidth{1.003750pt}%
\definecolor{currentstroke}{rgb}{1.000000,0.498039,0.054902}%
\pgfsetstrokecolor{currentstroke}%
\pgfsetdash{}{0pt}%
\pgfpathmoveto{\pgfqpoint{2.057911in}{3.132690in}}%
\pgfpathcurveto{\pgfqpoint{2.068961in}{3.132690in}}{\pgfqpoint{2.079560in}{3.137080in}}{\pgfqpoint{2.087374in}{3.144893in}}%
\pgfpathcurveto{\pgfqpoint{2.095187in}{3.152707in}}{\pgfqpoint{2.099577in}{3.163306in}}{\pgfqpoint{2.099577in}{3.174356in}}%
\pgfpathcurveto{\pgfqpoint{2.099577in}{3.185406in}}{\pgfqpoint{2.095187in}{3.196005in}}{\pgfqpoint{2.087374in}{3.203819in}}%
\pgfpathcurveto{\pgfqpoint{2.079560in}{3.211633in}}{\pgfqpoint{2.068961in}{3.216023in}}{\pgfqpoint{2.057911in}{3.216023in}}%
\pgfpathcurveto{\pgfqpoint{2.046861in}{3.216023in}}{\pgfqpoint{2.036262in}{3.211633in}}{\pgfqpoint{2.028448in}{3.203819in}}%
\pgfpathcurveto{\pgfqpoint{2.020634in}{3.196005in}}{\pgfqpoint{2.016244in}{3.185406in}}{\pgfqpoint{2.016244in}{3.174356in}}%
\pgfpathcurveto{\pgfqpoint{2.016244in}{3.163306in}}{\pgfqpoint{2.020634in}{3.152707in}}{\pgfqpoint{2.028448in}{3.144893in}}%
\pgfpathcurveto{\pgfqpoint{2.036262in}{3.137080in}}{\pgfqpoint{2.046861in}{3.132690in}}{\pgfqpoint{2.057911in}{3.132690in}}%
\pgfpathclose%
\pgfusepath{stroke,fill}%
\end{pgfscope}%
\begin{pgfscope}%
\pgfpathrectangle{\pgfqpoint{0.648703in}{0.548769in}}{\pgfqpoint{5.201297in}{3.102590in}}%
\pgfusepath{clip}%
\pgfsetbuttcap%
\pgfsetroundjoin%
\definecolor{currentfill}{rgb}{0.121569,0.466667,0.705882}%
\pgfsetfillcolor{currentfill}%
\pgfsetlinewidth{1.003750pt}%
\definecolor{currentstroke}{rgb}{0.121569,0.466667,0.705882}%
\pgfsetstrokecolor{currentstroke}%
\pgfsetdash{}{0pt}%
\pgfpathmoveto{\pgfqpoint{1.052819in}{0.648129in}}%
\pgfpathcurveto{\pgfqpoint{1.063869in}{0.648129in}}{\pgfqpoint{1.074468in}{0.652519in}}{\pgfqpoint{1.082282in}{0.660333in}}%
\pgfpathcurveto{\pgfqpoint{1.090095in}{0.668146in}}{\pgfqpoint{1.094485in}{0.678745in}}{\pgfqpoint{1.094485in}{0.689796in}}%
\pgfpathcurveto{\pgfqpoint{1.094485in}{0.700846in}}{\pgfqpoint{1.090095in}{0.711445in}}{\pgfqpoint{1.082282in}{0.719258in}}%
\pgfpathcurveto{\pgfqpoint{1.074468in}{0.727072in}}{\pgfqpoint{1.063869in}{0.731462in}}{\pgfqpoint{1.052819in}{0.731462in}}%
\pgfpathcurveto{\pgfqpoint{1.041769in}{0.731462in}}{\pgfqpoint{1.031170in}{0.727072in}}{\pgfqpoint{1.023356in}{0.719258in}}%
\pgfpathcurveto{\pgfqpoint{1.015542in}{0.711445in}}{\pgfqpoint{1.011152in}{0.700846in}}{\pgfqpoint{1.011152in}{0.689796in}}%
\pgfpathcurveto{\pgfqpoint{1.011152in}{0.678745in}}{\pgfqpoint{1.015542in}{0.668146in}}{\pgfqpoint{1.023356in}{0.660333in}}%
\pgfpathcurveto{\pgfqpoint{1.031170in}{0.652519in}}{\pgfqpoint{1.041769in}{0.648129in}}{\pgfqpoint{1.052819in}{0.648129in}}%
\pgfpathclose%
\pgfusepath{stroke,fill}%
\end{pgfscope}%
\begin{pgfscope}%
\pgfpathrectangle{\pgfqpoint{0.648703in}{0.548769in}}{\pgfqpoint{5.201297in}{3.102590in}}%
\pgfusepath{clip}%
\pgfsetbuttcap%
\pgfsetroundjoin%
\definecolor{currentfill}{rgb}{1.000000,0.498039,0.054902}%
\pgfsetfillcolor{currentfill}%
\pgfsetlinewidth{1.003750pt}%
\definecolor{currentstroke}{rgb}{1.000000,0.498039,0.054902}%
\pgfsetstrokecolor{currentstroke}%
\pgfsetdash{}{0pt}%
\pgfpathmoveto{\pgfqpoint{1.946339in}{3.140985in}}%
\pgfpathcurveto{\pgfqpoint{1.957389in}{3.140985in}}{\pgfqpoint{1.967988in}{3.145375in}}{\pgfqpoint{1.975801in}{3.153189in}}%
\pgfpathcurveto{\pgfqpoint{1.983615in}{3.161003in}}{\pgfqpoint{1.988005in}{3.171602in}}{\pgfqpoint{1.988005in}{3.182652in}}%
\pgfpathcurveto{\pgfqpoint{1.988005in}{3.193702in}}{\pgfqpoint{1.983615in}{3.204301in}}{\pgfqpoint{1.975801in}{3.212115in}}%
\pgfpathcurveto{\pgfqpoint{1.967988in}{3.219928in}}{\pgfqpoint{1.957389in}{3.224319in}}{\pgfqpoint{1.946339in}{3.224319in}}%
\pgfpathcurveto{\pgfqpoint{1.935288in}{3.224319in}}{\pgfqpoint{1.924689in}{3.219928in}}{\pgfqpoint{1.916876in}{3.212115in}}%
\pgfpathcurveto{\pgfqpoint{1.909062in}{3.204301in}}{\pgfqpoint{1.904672in}{3.193702in}}{\pgfqpoint{1.904672in}{3.182652in}}%
\pgfpathcurveto{\pgfqpoint{1.904672in}{3.171602in}}{\pgfqpoint{1.909062in}{3.161003in}}{\pgfqpoint{1.916876in}{3.153189in}}%
\pgfpathcurveto{\pgfqpoint{1.924689in}{3.145375in}}{\pgfqpoint{1.935288in}{3.140985in}}{\pgfqpoint{1.946339in}{3.140985in}}%
\pgfpathclose%
\pgfusepath{stroke,fill}%
\end{pgfscope}%
\begin{pgfscope}%
\pgfpathrectangle{\pgfqpoint{0.648703in}{0.548769in}}{\pgfqpoint{5.201297in}{3.102590in}}%
\pgfusepath{clip}%
\pgfsetbuttcap%
\pgfsetroundjoin%
\definecolor{currentfill}{rgb}{1.000000,0.498039,0.054902}%
\pgfsetfillcolor{currentfill}%
\pgfsetlinewidth{1.003750pt}%
\definecolor{currentstroke}{rgb}{1.000000,0.498039,0.054902}%
\pgfsetstrokecolor{currentstroke}%
\pgfsetdash{}{0pt}%
\pgfpathmoveto{\pgfqpoint{1.483893in}{3.145133in}}%
\pgfpathcurveto{\pgfqpoint{1.494943in}{3.145133in}}{\pgfqpoint{1.505542in}{3.149523in}}{\pgfqpoint{1.513356in}{3.157337in}}%
\pgfpathcurveto{\pgfqpoint{1.521170in}{3.165151in}}{\pgfqpoint{1.525560in}{3.175750in}}{\pgfqpoint{1.525560in}{3.186800in}}%
\pgfpathcurveto{\pgfqpoint{1.525560in}{3.197850in}}{\pgfqpoint{1.521170in}{3.208449in}}{\pgfqpoint{1.513356in}{3.216262in}}%
\pgfpathcurveto{\pgfqpoint{1.505542in}{3.224076in}}{\pgfqpoint{1.494943in}{3.228466in}}{\pgfqpoint{1.483893in}{3.228466in}}%
\pgfpathcurveto{\pgfqpoint{1.472843in}{3.228466in}}{\pgfqpoint{1.462244in}{3.224076in}}{\pgfqpoint{1.454430in}{3.216262in}}%
\pgfpathcurveto{\pgfqpoint{1.446617in}{3.208449in}}{\pgfqpoint{1.442226in}{3.197850in}}{\pgfqpoint{1.442226in}{3.186800in}}%
\pgfpathcurveto{\pgfqpoint{1.442226in}{3.175750in}}{\pgfqpoint{1.446617in}{3.165151in}}{\pgfqpoint{1.454430in}{3.157337in}}%
\pgfpathcurveto{\pgfqpoint{1.462244in}{3.149523in}}{\pgfqpoint{1.472843in}{3.145133in}}{\pgfqpoint{1.483893in}{3.145133in}}%
\pgfpathclose%
\pgfusepath{stroke,fill}%
\end{pgfscope}%
\begin{pgfscope}%
\pgfpathrectangle{\pgfqpoint{0.648703in}{0.548769in}}{\pgfqpoint{5.201297in}{3.102590in}}%
\pgfusepath{clip}%
\pgfsetbuttcap%
\pgfsetroundjoin%
\definecolor{currentfill}{rgb}{1.000000,0.498039,0.054902}%
\pgfsetfillcolor{currentfill}%
\pgfsetlinewidth{1.003750pt}%
\definecolor{currentstroke}{rgb}{1.000000,0.498039,0.054902}%
\pgfsetstrokecolor{currentstroke}%
\pgfsetdash{}{0pt}%
\pgfpathmoveto{\pgfqpoint{1.408715in}{3.140985in}}%
\pgfpathcurveto{\pgfqpoint{1.419765in}{3.140985in}}{\pgfqpoint{1.430364in}{3.145375in}}{\pgfqpoint{1.438178in}{3.153189in}}%
\pgfpathcurveto{\pgfqpoint{1.445991in}{3.161003in}}{\pgfqpoint{1.450381in}{3.171602in}}{\pgfqpoint{1.450381in}{3.182652in}}%
\pgfpathcurveto{\pgfqpoint{1.450381in}{3.193702in}}{\pgfqpoint{1.445991in}{3.204301in}}{\pgfqpoint{1.438178in}{3.212115in}}%
\pgfpathcurveto{\pgfqpoint{1.430364in}{3.219928in}}{\pgfqpoint{1.419765in}{3.224319in}}{\pgfqpoint{1.408715in}{3.224319in}}%
\pgfpathcurveto{\pgfqpoint{1.397665in}{3.224319in}}{\pgfqpoint{1.387066in}{3.219928in}}{\pgfqpoint{1.379252in}{3.212115in}}%
\pgfpathcurveto{\pgfqpoint{1.371438in}{3.204301in}}{\pgfqpoint{1.367048in}{3.193702in}}{\pgfqpoint{1.367048in}{3.182652in}}%
\pgfpathcurveto{\pgfqpoint{1.367048in}{3.171602in}}{\pgfqpoint{1.371438in}{3.161003in}}{\pgfqpoint{1.379252in}{3.153189in}}%
\pgfpathcurveto{\pgfqpoint{1.387066in}{3.145375in}}{\pgfqpoint{1.397665in}{3.140985in}}{\pgfqpoint{1.408715in}{3.140985in}}%
\pgfpathclose%
\pgfusepath{stroke,fill}%
\end{pgfscope}%
\begin{pgfscope}%
\pgfpathrectangle{\pgfqpoint{0.648703in}{0.548769in}}{\pgfqpoint{5.201297in}{3.102590in}}%
\pgfusepath{clip}%
\pgfsetbuttcap%
\pgfsetroundjoin%
\definecolor{currentfill}{rgb}{0.121569,0.466667,0.705882}%
\pgfsetfillcolor{currentfill}%
\pgfsetlinewidth{1.003750pt}%
\definecolor{currentstroke}{rgb}{0.121569,0.466667,0.705882}%
\pgfsetstrokecolor{currentstroke}%
\pgfsetdash{}{0pt}%
\pgfpathmoveto{\pgfqpoint{1.219506in}{0.697903in}}%
\pgfpathcurveto{\pgfqpoint{1.230556in}{0.697903in}}{\pgfqpoint{1.241155in}{0.702293in}}{\pgfqpoint{1.248969in}{0.710107in}}%
\pgfpathcurveto{\pgfqpoint{1.256783in}{0.717921in}}{\pgfqpoint{1.261173in}{0.728520in}}{\pgfqpoint{1.261173in}{0.739570in}}%
\pgfpathcurveto{\pgfqpoint{1.261173in}{0.750620in}}{\pgfqpoint{1.256783in}{0.761219in}}{\pgfqpoint{1.248969in}{0.769033in}}%
\pgfpathcurveto{\pgfqpoint{1.241155in}{0.776846in}}{\pgfqpoint{1.230556in}{0.781236in}}{\pgfqpoint{1.219506in}{0.781236in}}%
\pgfpathcurveto{\pgfqpoint{1.208456in}{0.781236in}}{\pgfqpoint{1.197857in}{0.776846in}}{\pgfqpoint{1.190043in}{0.769033in}}%
\pgfpathcurveto{\pgfqpoint{1.182230in}{0.761219in}}{\pgfqpoint{1.177840in}{0.750620in}}{\pgfqpoint{1.177840in}{0.739570in}}%
\pgfpathcurveto{\pgfqpoint{1.177840in}{0.728520in}}{\pgfqpoint{1.182230in}{0.717921in}}{\pgfqpoint{1.190043in}{0.710107in}}%
\pgfpathcurveto{\pgfqpoint{1.197857in}{0.702293in}}{\pgfqpoint{1.208456in}{0.697903in}}{\pgfqpoint{1.219506in}{0.697903in}}%
\pgfpathclose%
\pgfusepath{stroke,fill}%
\end{pgfscope}%
\begin{pgfscope}%
\pgfpathrectangle{\pgfqpoint{0.648703in}{0.548769in}}{\pgfqpoint{5.201297in}{3.102590in}}%
\pgfusepath{clip}%
\pgfsetbuttcap%
\pgfsetroundjoin%
\definecolor{currentfill}{rgb}{1.000000,0.498039,0.054902}%
\pgfsetfillcolor{currentfill}%
\pgfsetlinewidth{1.003750pt}%
\definecolor{currentstroke}{rgb}{1.000000,0.498039,0.054902}%
\pgfsetstrokecolor{currentstroke}%
\pgfsetdash{}{0pt}%
\pgfpathmoveto{\pgfqpoint{1.557330in}{3.140985in}}%
\pgfpathcurveto{\pgfqpoint{1.568380in}{3.140985in}}{\pgfqpoint{1.578979in}{3.145375in}}{\pgfqpoint{1.586793in}{3.153189in}}%
\pgfpathcurveto{\pgfqpoint{1.594606in}{3.161003in}}{\pgfqpoint{1.598997in}{3.171602in}}{\pgfqpoint{1.598997in}{3.182652in}}%
\pgfpathcurveto{\pgfqpoint{1.598997in}{3.193702in}}{\pgfqpoint{1.594606in}{3.204301in}}{\pgfqpoint{1.586793in}{3.212115in}}%
\pgfpathcurveto{\pgfqpoint{1.578979in}{3.219928in}}{\pgfqpoint{1.568380in}{3.224319in}}{\pgfqpoint{1.557330in}{3.224319in}}%
\pgfpathcurveto{\pgfqpoint{1.546280in}{3.224319in}}{\pgfqpoint{1.535681in}{3.219928in}}{\pgfqpoint{1.527867in}{3.212115in}}%
\pgfpathcurveto{\pgfqpoint{1.520054in}{3.204301in}}{\pgfqpoint{1.515663in}{3.193702in}}{\pgfqpoint{1.515663in}{3.182652in}}%
\pgfpathcurveto{\pgfqpoint{1.515663in}{3.171602in}}{\pgfqpoint{1.520054in}{3.161003in}}{\pgfqpoint{1.527867in}{3.153189in}}%
\pgfpathcurveto{\pgfqpoint{1.535681in}{3.145375in}}{\pgfqpoint{1.546280in}{3.140985in}}{\pgfqpoint{1.557330in}{3.140985in}}%
\pgfpathclose%
\pgfusepath{stroke,fill}%
\end{pgfscope}%
\begin{pgfscope}%
\pgfpathrectangle{\pgfqpoint{0.648703in}{0.548769in}}{\pgfqpoint{5.201297in}{3.102590in}}%
\pgfusepath{clip}%
\pgfsetbuttcap%
\pgfsetroundjoin%
\definecolor{currentfill}{rgb}{1.000000,0.498039,0.054902}%
\pgfsetfillcolor{currentfill}%
\pgfsetlinewidth{1.003750pt}%
\definecolor{currentstroke}{rgb}{1.000000,0.498039,0.054902}%
\pgfsetstrokecolor{currentstroke}%
\pgfsetdash{}{0pt}%
\pgfpathmoveto{\pgfqpoint{2.222501in}{3.132690in}}%
\pgfpathcurveto{\pgfqpoint{2.233551in}{3.132690in}}{\pgfqpoint{2.244150in}{3.137080in}}{\pgfqpoint{2.251963in}{3.144893in}}%
\pgfpathcurveto{\pgfqpoint{2.259777in}{3.152707in}}{\pgfqpoint{2.264167in}{3.163306in}}{\pgfqpoint{2.264167in}{3.174356in}}%
\pgfpathcurveto{\pgfqpoint{2.264167in}{3.185406in}}{\pgfqpoint{2.259777in}{3.196005in}}{\pgfqpoint{2.251963in}{3.203819in}}%
\pgfpathcurveto{\pgfqpoint{2.244150in}{3.211633in}}{\pgfqpoint{2.233551in}{3.216023in}}{\pgfqpoint{2.222501in}{3.216023in}}%
\pgfpathcurveto{\pgfqpoint{2.211450in}{3.216023in}}{\pgfqpoint{2.200851in}{3.211633in}}{\pgfqpoint{2.193038in}{3.203819in}}%
\pgfpathcurveto{\pgfqpoint{2.185224in}{3.196005in}}{\pgfqpoint{2.180834in}{3.185406in}}{\pgfqpoint{2.180834in}{3.174356in}}%
\pgfpathcurveto{\pgfqpoint{2.180834in}{3.163306in}}{\pgfqpoint{2.185224in}{3.152707in}}{\pgfqpoint{2.193038in}{3.144893in}}%
\pgfpathcurveto{\pgfqpoint{2.200851in}{3.137080in}}{\pgfqpoint{2.211450in}{3.132690in}}{\pgfqpoint{2.222501in}{3.132690in}}%
\pgfpathclose%
\pgfusepath{stroke,fill}%
\end{pgfscope}%
\begin{pgfscope}%
\pgfpathrectangle{\pgfqpoint{0.648703in}{0.548769in}}{\pgfqpoint{5.201297in}{3.102590in}}%
\pgfusepath{clip}%
\pgfsetbuttcap%
\pgfsetroundjoin%
\definecolor{currentfill}{rgb}{1.000000,0.498039,0.054902}%
\pgfsetfillcolor{currentfill}%
\pgfsetlinewidth{1.003750pt}%
\definecolor{currentstroke}{rgb}{1.000000,0.498039,0.054902}%
\pgfsetstrokecolor{currentstroke}%
\pgfsetdash{}{0pt}%
\pgfpathmoveto{\pgfqpoint{2.066527in}{3.140985in}}%
\pgfpathcurveto{\pgfqpoint{2.077578in}{3.140985in}}{\pgfqpoint{2.088177in}{3.145375in}}{\pgfqpoint{2.095990in}{3.153189in}}%
\pgfpathcurveto{\pgfqpoint{2.103804in}{3.161003in}}{\pgfqpoint{2.108194in}{3.171602in}}{\pgfqpoint{2.108194in}{3.182652in}}%
\pgfpathcurveto{\pgfqpoint{2.108194in}{3.193702in}}{\pgfqpoint{2.103804in}{3.204301in}}{\pgfqpoint{2.095990in}{3.212115in}}%
\pgfpathcurveto{\pgfqpoint{2.088177in}{3.219928in}}{\pgfqpoint{2.077578in}{3.224319in}}{\pgfqpoint{2.066527in}{3.224319in}}%
\pgfpathcurveto{\pgfqpoint{2.055477in}{3.224319in}}{\pgfqpoint{2.044878in}{3.219928in}}{\pgfqpoint{2.037065in}{3.212115in}}%
\pgfpathcurveto{\pgfqpoint{2.029251in}{3.204301in}}{\pgfqpoint{2.024861in}{3.193702in}}{\pgfqpoint{2.024861in}{3.182652in}}%
\pgfpathcurveto{\pgfqpoint{2.024861in}{3.171602in}}{\pgfqpoint{2.029251in}{3.161003in}}{\pgfqpoint{2.037065in}{3.153189in}}%
\pgfpathcurveto{\pgfqpoint{2.044878in}{3.145375in}}{\pgfqpoint{2.055477in}{3.140985in}}{\pgfqpoint{2.066527in}{3.140985in}}%
\pgfpathclose%
\pgfusepath{stroke,fill}%
\end{pgfscope}%
\begin{pgfscope}%
\pgfpathrectangle{\pgfqpoint{0.648703in}{0.548769in}}{\pgfqpoint{5.201297in}{3.102590in}}%
\pgfusepath{clip}%
\pgfsetbuttcap%
\pgfsetroundjoin%
\definecolor{currentfill}{rgb}{1.000000,0.498039,0.054902}%
\pgfsetfillcolor{currentfill}%
\pgfsetlinewidth{1.003750pt}%
\definecolor{currentstroke}{rgb}{1.000000,0.498039,0.054902}%
\pgfsetstrokecolor{currentstroke}%
\pgfsetdash{}{0pt}%
\pgfpathmoveto{\pgfqpoint{1.786871in}{3.132690in}}%
\pgfpathcurveto{\pgfqpoint{1.797921in}{3.132690in}}{\pgfqpoint{1.808520in}{3.137080in}}{\pgfqpoint{1.816333in}{3.144893in}}%
\pgfpathcurveto{\pgfqpoint{1.824147in}{3.152707in}}{\pgfqpoint{1.828537in}{3.163306in}}{\pgfqpoint{1.828537in}{3.174356in}}%
\pgfpathcurveto{\pgfqpoint{1.828537in}{3.185406in}}{\pgfqpoint{1.824147in}{3.196005in}}{\pgfqpoint{1.816333in}{3.203819in}}%
\pgfpathcurveto{\pgfqpoint{1.808520in}{3.211633in}}{\pgfqpoint{1.797921in}{3.216023in}}{\pgfqpoint{1.786871in}{3.216023in}}%
\pgfpathcurveto{\pgfqpoint{1.775820in}{3.216023in}}{\pgfqpoint{1.765221in}{3.211633in}}{\pgfqpoint{1.757408in}{3.203819in}}%
\pgfpathcurveto{\pgfqpoint{1.749594in}{3.196005in}}{\pgfqpoint{1.745204in}{3.185406in}}{\pgfqpoint{1.745204in}{3.174356in}}%
\pgfpathcurveto{\pgfqpoint{1.745204in}{3.163306in}}{\pgfqpoint{1.749594in}{3.152707in}}{\pgfqpoint{1.757408in}{3.144893in}}%
\pgfpathcurveto{\pgfqpoint{1.765221in}{3.137080in}}{\pgfqpoint{1.775820in}{3.132690in}}{\pgfqpoint{1.786871in}{3.132690in}}%
\pgfpathclose%
\pgfusepath{stroke,fill}%
\end{pgfscope}%
\begin{pgfscope}%
\pgfpathrectangle{\pgfqpoint{0.648703in}{0.548769in}}{\pgfqpoint{5.201297in}{3.102590in}}%
\pgfusepath{clip}%
\pgfsetbuttcap%
\pgfsetroundjoin%
\definecolor{currentfill}{rgb}{1.000000,0.498039,0.054902}%
\pgfsetfillcolor{currentfill}%
\pgfsetlinewidth{1.003750pt}%
\definecolor{currentstroke}{rgb}{1.000000,0.498039,0.054902}%
\pgfsetstrokecolor{currentstroke}%
\pgfsetdash{}{0pt}%
\pgfpathmoveto{\pgfqpoint{1.423134in}{3.140985in}}%
\pgfpathcurveto{\pgfqpoint{1.434184in}{3.140985in}}{\pgfqpoint{1.444783in}{3.145375in}}{\pgfqpoint{1.452597in}{3.153189in}}%
\pgfpathcurveto{\pgfqpoint{1.460410in}{3.161003in}}{\pgfqpoint{1.464800in}{3.171602in}}{\pgfqpoint{1.464800in}{3.182652in}}%
\pgfpathcurveto{\pgfqpoint{1.464800in}{3.193702in}}{\pgfqpoint{1.460410in}{3.204301in}}{\pgfqpoint{1.452597in}{3.212115in}}%
\pgfpathcurveto{\pgfqpoint{1.444783in}{3.219928in}}{\pgfqpoint{1.434184in}{3.224319in}}{\pgfqpoint{1.423134in}{3.224319in}}%
\pgfpathcurveto{\pgfqpoint{1.412084in}{3.224319in}}{\pgfqpoint{1.401485in}{3.219928in}}{\pgfqpoint{1.393671in}{3.212115in}}%
\pgfpathcurveto{\pgfqpoint{1.385857in}{3.204301in}}{\pgfqpoint{1.381467in}{3.193702in}}{\pgfqpoint{1.381467in}{3.182652in}}%
\pgfpathcurveto{\pgfqpoint{1.381467in}{3.171602in}}{\pgfqpoint{1.385857in}{3.161003in}}{\pgfqpoint{1.393671in}{3.153189in}}%
\pgfpathcurveto{\pgfqpoint{1.401485in}{3.145375in}}{\pgfqpoint{1.412084in}{3.140985in}}{\pgfqpoint{1.423134in}{3.140985in}}%
\pgfpathclose%
\pgfusepath{stroke,fill}%
\end{pgfscope}%
\begin{pgfscope}%
\pgfpathrectangle{\pgfqpoint{0.648703in}{0.548769in}}{\pgfqpoint{5.201297in}{3.102590in}}%
\pgfusepath{clip}%
\pgfsetbuttcap%
\pgfsetroundjoin%
\definecolor{currentfill}{rgb}{1.000000,0.498039,0.054902}%
\pgfsetfillcolor{currentfill}%
\pgfsetlinewidth{1.003750pt}%
\definecolor{currentstroke}{rgb}{1.000000,0.498039,0.054902}%
\pgfsetstrokecolor{currentstroke}%
\pgfsetdash{}{0pt}%
\pgfpathmoveto{\pgfqpoint{1.729099in}{3.145133in}}%
\pgfpathcurveto{\pgfqpoint{1.740150in}{3.145133in}}{\pgfqpoint{1.750749in}{3.149523in}}{\pgfqpoint{1.758562in}{3.157337in}}%
\pgfpathcurveto{\pgfqpoint{1.766376in}{3.165151in}}{\pgfqpoint{1.770766in}{3.175750in}}{\pgfqpoint{1.770766in}{3.186800in}}%
\pgfpathcurveto{\pgfqpoint{1.770766in}{3.197850in}}{\pgfqpoint{1.766376in}{3.208449in}}{\pgfqpoint{1.758562in}{3.216262in}}%
\pgfpathcurveto{\pgfqpoint{1.750749in}{3.224076in}}{\pgfqpoint{1.740150in}{3.228466in}}{\pgfqpoint{1.729099in}{3.228466in}}%
\pgfpathcurveto{\pgfqpoint{1.718049in}{3.228466in}}{\pgfqpoint{1.707450in}{3.224076in}}{\pgfqpoint{1.699637in}{3.216262in}}%
\pgfpathcurveto{\pgfqpoint{1.691823in}{3.208449in}}{\pgfqpoint{1.687433in}{3.197850in}}{\pgfqpoint{1.687433in}{3.186800in}}%
\pgfpathcurveto{\pgfqpoint{1.687433in}{3.175750in}}{\pgfqpoint{1.691823in}{3.165151in}}{\pgfqpoint{1.699637in}{3.157337in}}%
\pgfpathcurveto{\pgfqpoint{1.707450in}{3.149523in}}{\pgfqpoint{1.718049in}{3.145133in}}{\pgfqpoint{1.729099in}{3.145133in}}%
\pgfpathclose%
\pgfusepath{stroke,fill}%
\end{pgfscope}%
\begin{pgfscope}%
\pgfpathrectangle{\pgfqpoint{0.648703in}{0.548769in}}{\pgfqpoint{5.201297in}{3.102590in}}%
\pgfusepath{clip}%
\pgfsetbuttcap%
\pgfsetroundjoin%
\definecolor{currentfill}{rgb}{1.000000,0.498039,0.054902}%
\pgfsetfillcolor{currentfill}%
\pgfsetlinewidth{1.003750pt}%
\definecolor{currentstroke}{rgb}{1.000000,0.498039,0.054902}%
\pgfsetstrokecolor{currentstroke}%
\pgfsetdash{}{0pt}%
\pgfpathmoveto{\pgfqpoint{1.788968in}{3.145133in}}%
\pgfpathcurveto{\pgfqpoint{1.800018in}{3.145133in}}{\pgfqpoint{1.810617in}{3.149523in}}{\pgfqpoint{1.818431in}{3.157337in}}%
\pgfpathcurveto{\pgfqpoint{1.826245in}{3.165151in}}{\pgfqpoint{1.830635in}{3.175750in}}{\pgfqpoint{1.830635in}{3.186800in}}%
\pgfpathcurveto{\pgfqpoint{1.830635in}{3.197850in}}{\pgfqpoint{1.826245in}{3.208449in}}{\pgfqpoint{1.818431in}{3.216262in}}%
\pgfpathcurveto{\pgfqpoint{1.810617in}{3.224076in}}{\pgfqpoint{1.800018in}{3.228466in}}{\pgfqpoint{1.788968in}{3.228466in}}%
\pgfpathcurveto{\pgfqpoint{1.777918in}{3.228466in}}{\pgfqpoint{1.767319in}{3.224076in}}{\pgfqpoint{1.759505in}{3.216262in}}%
\pgfpathcurveto{\pgfqpoint{1.751692in}{3.208449in}}{\pgfqpoint{1.747302in}{3.197850in}}{\pgfqpoint{1.747302in}{3.186800in}}%
\pgfpathcurveto{\pgfqpoint{1.747302in}{3.175750in}}{\pgfqpoint{1.751692in}{3.165151in}}{\pgfqpoint{1.759505in}{3.157337in}}%
\pgfpathcurveto{\pgfqpoint{1.767319in}{3.149523in}}{\pgfqpoint{1.777918in}{3.145133in}}{\pgfqpoint{1.788968in}{3.145133in}}%
\pgfpathclose%
\pgfusepath{stroke,fill}%
\end{pgfscope}%
\begin{pgfscope}%
\pgfpathrectangle{\pgfqpoint{0.648703in}{0.548769in}}{\pgfqpoint{5.201297in}{3.102590in}}%
\pgfusepath{clip}%
\pgfsetbuttcap%
\pgfsetroundjoin%
\definecolor{currentfill}{rgb}{1.000000,0.498039,0.054902}%
\pgfsetfillcolor{currentfill}%
\pgfsetlinewidth{1.003750pt}%
\definecolor{currentstroke}{rgb}{1.000000,0.498039,0.054902}%
\pgfsetstrokecolor{currentstroke}%
\pgfsetdash{}{0pt}%
\pgfpathmoveto{\pgfqpoint{1.991123in}{3.136837in}}%
\pgfpathcurveto{\pgfqpoint{2.002174in}{3.136837in}}{\pgfqpoint{2.012773in}{3.141228in}}{\pgfqpoint{2.020586in}{3.149041in}}%
\pgfpathcurveto{\pgfqpoint{2.028400in}{3.156855in}}{\pgfqpoint{2.032790in}{3.167454in}}{\pgfqpoint{2.032790in}{3.178504in}}%
\pgfpathcurveto{\pgfqpoint{2.032790in}{3.189554in}}{\pgfqpoint{2.028400in}{3.200153in}}{\pgfqpoint{2.020586in}{3.207967in}}%
\pgfpathcurveto{\pgfqpoint{2.012773in}{3.215780in}}{\pgfqpoint{2.002174in}{3.220171in}}{\pgfqpoint{1.991123in}{3.220171in}}%
\pgfpathcurveto{\pgfqpoint{1.980073in}{3.220171in}}{\pgfqpoint{1.969474in}{3.215780in}}{\pgfqpoint{1.961661in}{3.207967in}}%
\pgfpathcurveto{\pgfqpoint{1.953847in}{3.200153in}}{\pgfqpoint{1.949457in}{3.189554in}}{\pgfqpoint{1.949457in}{3.178504in}}%
\pgfpathcurveto{\pgfqpoint{1.949457in}{3.167454in}}{\pgfqpoint{1.953847in}{3.156855in}}{\pgfqpoint{1.961661in}{3.149041in}}%
\pgfpathcurveto{\pgfqpoint{1.969474in}{3.141228in}}{\pgfqpoint{1.980073in}{3.136837in}}{\pgfqpoint{1.991123in}{3.136837in}}%
\pgfpathclose%
\pgfusepath{stroke,fill}%
\end{pgfscope}%
\begin{pgfscope}%
\pgfpathrectangle{\pgfqpoint{0.648703in}{0.548769in}}{\pgfqpoint{5.201297in}{3.102590in}}%
\pgfusepath{clip}%
\pgfsetbuttcap%
\pgfsetroundjoin%
\definecolor{currentfill}{rgb}{1.000000,0.498039,0.054902}%
\pgfsetfillcolor{currentfill}%
\pgfsetlinewidth{1.003750pt}%
\definecolor{currentstroke}{rgb}{1.000000,0.498039,0.054902}%
\pgfsetstrokecolor{currentstroke}%
\pgfsetdash{}{0pt}%
\pgfpathmoveto{\pgfqpoint{1.493163in}{3.468665in}}%
\pgfpathcurveto{\pgfqpoint{1.504213in}{3.468665in}}{\pgfqpoint{1.514812in}{3.473055in}}{\pgfqpoint{1.522626in}{3.480869in}}%
\pgfpathcurveto{\pgfqpoint{1.530439in}{3.488683in}}{\pgfqpoint{1.534829in}{3.499282in}}{\pgfqpoint{1.534829in}{3.510332in}}%
\pgfpathcurveto{\pgfqpoint{1.534829in}{3.521382in}}{\pgfqpoint{1.530439in}{3.531981in}}{\pgfqpoint{1.522626in}{3.539795in}}%
\pgfpathcurveto{\pgfqpoint{1.514812in}{3.547608in}}{\pgfqpoint{1.504213in}{3.551998in}}{\pgfqpoint{1.493163in}{3.551998in}}%
\pgfpathcurveto{\pgfqpoint{1.482113in}{3.551998in}}{\pgfqpoint{1.471514in}{3.547608in}}{\pgfqpoint{1.463700in}{3.539795in}}%
\pgfpathcurveto{\pgfqpoint{1.455886in}{3.531981in}}{\pgfqpoint{1.451496in}{3.521382in}}{\pgfqpoint{1.451496in}{3.510332in}}%
\pgfpathcurveto{\pgfqpoint{1.451496in}{3.499282in}}{\pgfqpoint{1.455886in}{3.488683in}}{\pgfqpoint{1.463700in}{3.480869in}}%
\pgfpathcurveto{\pgfqpoint{1.471514in}{3.473055in}}{\pgfqpoint{1.482113in}{3.468665in}}{\pgfqpoint{1.493163in}{3.468665in}}%
\pgfpathclose%
\pgfusepath{stroke,fill}%
\end{pgfscope}%
\begin{pgfscope}%
\pgfpathrectangle{\pgfqpoint{0.648703in}{0.548769in}}{\pgfqpoint{5.201297in}{3.102590in}}%
\pgfusepath{clip}%
\pgfsetbuttcap%
\pgfsetroundjoin%
\definecolor{currentfill}{rgb}{1.000000,0.498039,0.054902}%
\pgfsetfillcolor{currentfill}%
\pgfsetlinewidth{1.003750pt}%
\definecolor{currentstroke}{rgb}{1.000000,0.498039,0.054902}%
\pgfsetstrokecolor{currentstroke}%
\pgfsetdash{}{0pt}%
\pgfpathmoveto{\pgfqpoint{1.820830in}{3.136837in}}%
\pgfpathcurveto{\pgfqpoint{1.831880in}{3.136837in}}{\pgfqpoint{1.842479in}{3.141228in}}{\pgfqpoint{1.850293in}{3.149041in}}%
\pgfpathcurveto{\pgfqpoint{1.858107in}{3.156855in}}{\pgfqpoint{1.862497in}{3.167454in}}{\pgfqpoint{1.862497in}{3.178504in}}%
\pgfpathcurveto{\pgfqpoint{1.862497in}{3.189554in}}{\pgfqpoint{1.858107in}{3.200153in}}{\pgfqpoint{1.850293in}{3.207967in}}%
\pgfpathcurveto{\pgfqpoint{1.842479in}{3.215780in}}{\pgfqpoint{1.831880in}{3.220171in}}{\pgfqpoint{1.820830in}{3.220171in}}%
\pgfpathcurveto{\pgfqpoint{1.809780in}{3.220171in}}{\pgfqpoint{1.799181in}{3.215780in}}{\pgfqpoint{1.791367in}{3.207967in}}%
\pgfpathcurveto{\pgfqpoint{1.783554in}{3.200153in}}{\pgfqpoint{1.779164in}{3.189554in}}{\pgfqpoint{1.779164in}{3.178504in}}%
\pgfpathcurveto{\pgfqpoint{1.779164in}{3.167454in}}{\pgfqpoint{1.783554in}{3.156855in}}{\pgfqpoint{1.791367in}{3.149041in}}%
\pgfpathcurveto{\pgfqpoint{1.799181in}{3.141228in}}{\pgfqpoint{1.809780in}{3.136837in}}{\pgfqpoint{1.820830in}{3.136837in}}%
\pgfpathclose%
\pgfusepath{stroke,fill}%
\end{pgfscope}%
\begin{pgfscope}%
\pgfpathrectangle{\pgfqpoint{0.648703in}{0.548769in}}{\pgfqpoint{5.201297in}{3.102590in}}%
\pgfusepath{clip}%
\pgfsetbuttcap%
\pgfsetroundjoin%
\definecolor{currentfill}{rgb}{1.000000,0.498039,0.054902}%
\pgfsetfillcolor{currentfill}%
\pgfsetlinewidth{1.003750pt}%
\definecolor{currentstroke}{rgb}{1.000000,0.498039,0.054902}%
\pgfsetstrokecolor{currentstroke}%
\pgfsetdash{}{0pt}%
\pgfpathmoveto{\pgfqpoint{2.102901in}{3.136837in}}%
\pgfpathcurveto{\pgfqpoint{2.113952in}{3.136837in}}{\pgfqpoint{2.124551in}{3.141228in}}{\pgfqpoint{2.132364in}{3.149041in}}%
\pgfpathcurveto{\pgfqpoint{2.140178in}{3.156855in}}{\pgfqpoint{2.144568in}{3.167454in}}{\pgfqpoint{2.144568in}{3.178504in}}%
\pgfpathcurveto{\pgfqpoint{2.144568in}{3.189554in}}{\pgfqpoint{2.140178in}{3.200153in}}{\pgfqpoint{2.132364in}{3.207967in}}%
\pgfpathcurveto{\pgfqpoint{2.124551in}{3.215780in}}{\pgfqpoint{2.113952in}{3.220171in}}{\pgfqpoint{2.102901in}{3.220171in}}%
\pgfpathcurveto{\pgfqpoint{2.091851in}{3.220171in}}{\pgfqpoint{2.081252in}{3.215780in}}{\pgfqpoint{2.073439in}{3.207967in}}%
\pgfpathcurveto{\pgfqpoint{2.065625in}{3.200153in}}{\pgfqpoint{2.061235in}{3.189554in}}{\pgfqpoint{2.061235in}{3.178504in}}%
\pgfpathcurveto{\pgfqpoint{2.061235in}{3.167454in}}{\pgfqpoint{2.065625in}{3.156855in}}{\pgfqpoint{2.073439in}{3.149041in}}%
\pgfpathcurveto{\pgfqpoint{2.081252in}{3.141228in}}{\pgfqpoint{2.091851in}{3.136837in}}{\pgfqpoint{2.102901in}{3.136837in}}%
\pgfpathclose%
\pgfusepath{stroke,fill}%
\end{pgfscope}%
\begin{pgfscope}%
\pgfpathrectangle{\pgfqpoint{0.648703in}{0.548769in}}{\pgfqpoint{5.201297in}{3.102590in}}%
\pgfusepath{clip}%
\pgfsetbuttcap%
\pgfsetroundjoin%
\definecolor{currentfill}{rgb}{1.000000,0.498039,0.054902}%
\pgfsetfillcolor{currentfill}%
\pgfsetlinewidth{1.003750pt}%
\definecolor{currentstroke}{rgb}{1.000000,0.498039,0.054902}%
\pgfsetstrokecolor{currentstroke}%
\pgfsetdash{}{0pt}%
\pgfpathmoveto{\pgfqpoint{1.910115in}{3.136837in}}%
\pgfpathcurveto{\pgfqpoint{1.921165in}{3.136837in}}{\pgfqpoint{1.931764in}{3.141228in}}{\pgfqpoint{1.939578in}{3.149041in}}%
\pgfpathcurveto{\pgfqpoint{1.947391in}{3.156855in}}{\pgfqpoint{1.951782in}{3.167454in}}{\pgfqpoint{1.951782in}{3.178504in}}%
\pgfpathcurveto{\pgfqpoint{1.951782in}{3.189554in}}{\pgfqpoint{1.947391in}{3.200153in}}{\pgfqpoint{1.939578in}{3.207967in}}%
\pgfpathcurveto{\pgfqpoint{1.931764in}{3.215780in}}{\pgfqpoint{1.921165in}{3.220171in}}{\pgfqpoint{1.910115in}{3.220171in}}%
\pgfpathcurveto{\pgfqpoint{1.899065in}{3.220171in}}{\pgfqpoint{1.888466in}{3.215780in}}{\pgfqpoint{1.880652in}{3.207967in}}%
\pgfpathcurveto{\pgfqpoint{1.872839in}{3.200153in}}{\pgfqpoint{1.868448in}{3.189554in}}{\pgfqpoint{1.868448in}{3.178504in}}%
\pgfpathcurveto{\pgfqpoint{1.868448in}{3.167454in}}{\pgfqpoint{1.872839in}{3.156855in}}{\pgfqpoint{1.880652in}{3.149041in}}%
\pgfpathcurveto{\pgfqpoint{1.888466in}{3.141228in}}{\pgfqpoint{1.899065in}{3.136837in}}{\pgfqpoint{1.910115in}{3.136837in}}%
\pgfpathclose%
\pgfusepath{stroke,fill}%
\end{pgfscope}%
\begin{pgfscope}%
\pgfpathrectangle{\pgfqpoint{0.648703in}{0.548769in}}{\pgfqpoint{5.201297in}{3.102590in}}%
\pgfusepath{clip}%
\pgfsetbuttcap%
\pgfsetroundjoin%
\definecolor{currentfill}{rgb}{0.839216,0.152941,0.156863}%
\pgfsetfillcolor{currentfill}%
\pgfsetlinewidth{1.003750pt}%
\definecolor{currentstroke}{rgb}{0.839216,0.152941,0.156863}%
\pgfsetstrokecolor{currentstroke}%
\pgfsetdash{}{0pt}%
\pgfpathmoveto{\pgfqpoint{1.676905in}{3.136837in}}%
\pgfpathcurveto{\pgfqpoint{1.687955in}{3.136837in}}{\pgfqpoint{1.698554in}{3.141228in}}{\pgfqpoint{1.706368in}{3.149041in}}%
\pgfpathcurveto{\pgfqpoint{1.714182in}{3.156855in}}{\pgfqpoint{1.718572in}{3.167454in}}{\pgfqpoint{1.718572in}{3.178504in}}%
\pgfpathcurveto{\pgfqpoint{1.718572in}{3.189554in}}{\pgfqpoint{1.714182in}{3.200153in}}{\pgfqpoint{1.706368in}{3.207967in}}%
\pgfpathcurveto{\pgfqpoint{1.698554in}{3.215780in}}{\pgfqpoint{1.687955in}{3.220171in}}{\pgfqpoint{1.676905in}{3.220171in}}%
\pgfpathcurveto{\pgfqpoint{1.665855in}{3.220171in}}{\pgfqpoint{1.655256in}{3.215780in}}{\pgfqpoint{1.647442in}{3.207967in}}%
\pgfpathcurveto{\pgfqpoint{1.639629in}{3.200153in}}{\pgfqpoint{1.635239in}{3.189554in}}{\pgfqpoint{1.635239in}{3.178504in}}%
\pgfpathcurveto{\pgfqpoint{1.635239in}{3.167454in}}{\pgfqpoint{1.639629in}{3.156855in}}{\pgfqpoint{1.647442in}{3.149041in}}%
\pgfpathcurveto{\pgfqpoint{1.655256in}{3.141228in}}{\pgfqpoint{1.665855in}{3.136837in}}{\pgfqpoint{1.676905in}{3.136837in}}%
\pgfpathclose%
\pgfusepath{stroke,fill}%
\end{pgfscope}%
\begin{pgfscope}%
\pgfpathrectangle{\pgfqpoint{0.648703in}{0.548769in}}{\pgfqpoint{5.201297in}{3.102590in}}%
\pgfusepath{clip}%
\pgfsetbuttcap%
\pgfsetroundjoin%
\definecolor{currentfill}{rgb}{1.000000,0.498039,0.054902}%
\pgfsetfillcolor{currentfill}%
\pgfsetlinewidth{1.003750pt}%
\definecolor{currentstroke}{rgb}{1.000000,0.498039,0.054902}%
\pgfsetstrokecolor{currentstroke}%
\pgfsetdash{}{0pt}%
\pgfpathmoveto{\pgfqpoint{1.098633in}{3.149281in}}%
\pgfpathcurveto{\pgfqpoint{1.109683in}{3.149281in}}{\pgfqpoint{1.120282in}{3.153671in}}{\pgfqpoint{1.128095in}{3.161485in}}%
\pgfpathcurveto{\pgfqpoint{1.135909in}{3.169298in}}{\pgfqpoint{1.140299in}{3.179897in}}{\pgfqpoint{1.140299in}{3.190948in}}%
\pgfpathcurveto{\pgfqpoint{1.140299in}{3.201998in}}{\pgfqpoint{1.135909in}{3.212597in}}{\pgfqpoint{1.128095in}{3.220410in}}%
\pgfpathcurveto{\pgfqpoint{1.120282in}{3.228224in}}{\pgfqpoint{1.109683in}{3.232614in}}{\pgfqpoint{1.098633in}{3.232614in}}%
\pgfpathcurveto{\pgfqpoint{1.087583in}{3.232614in}}{\pgfqpoint{1.076984in}{3.228224in}}{\pgfqpoint{1.069170in}{3.220410in}}%
\pgfpathcurveto{\pgfqpoint{1.061356in}{3.212597in}}{\pgfqpoint{1.056966in}{3.201998in}}{\pgfqpoint{1.056966in}{3.190948in}}%
\pgfpathcurveto{\pgfqpoint{1.056966in}{3.179897in}}{\pgfqpoint{1.061356in}{3.169298in}}{\pgfqpoint{1.069170in}{3.161485in}}%
\pgfpathcurveto{\pgfqpoint{1.076984in}{3.153671in}}{\pgfqpoint{1.087583in}{3.149281in}}{\pgfqpoint{1.098633in}{3.149281in}}%
\pgfpathclose%
\pgfusepath{stroke,fill}%
\end{pgfscope}%
\begin{pgfscope}%
\pgfpathrectangle{\pgfqpoint{0.648703in}{0.548769in}}{\pgfqpoint{5.201297in}{3.102590in}}%
\pgfusepath{clip}%
\pgfsetbuttcap%
\pgfsetroundjoin%
\definecolor{currentfill}{rgb}{1.000000,0.498039,0.054902}%
\pgfsetfillcolor{currentfill}%
\pgfsetlinewidth{1.003750pt}%
\definecolor{currentstroke}{rgb}{1.000000,0.498039,0.054902}%
\pgfsetstrokecolor{currentstroke}%
\pgfsetdash{}{0pt}%
\pgfpathmoveto{\pgfqpoint{2.178230in}{3.286160in}}%
\pgfpathcurveto{\pgfqpoint{2.189280in}{3.286160in}}{\pgfqpoint{2.199879in}{3.290550in}}{\pgfqpoint{2.207693in}{3.298364in}}%
\pgfpathcurveto{\pgfqpoint{2.215507in}{3.306177in}}{\pgfqpoint{2.219897in}{3.316776in}}{\pgfqpoint{2.219897in}{3.327827in}}%
\pgfpathcurveto{\pgfqpoint{2.219897in}{3.338877in}}{\pgfqpoint{2.215507in}{3.349476in}}{\pgfqpoint{2.207693in}{3.357289in}}%
\pgfpathcurveto{\pgfqpoint{2.199879in}{3.365103in}}{\pgfqpoint{2.189280in}{3.369493in}}{\pgfqpoint{2.178230in}{3.369493in}}%
\pgfpathcurveto{\pgfqpoint{2.167180in}{3.369493in}}{\pgfqpoint{2.156581in}{3.365103in}}{\pgfqpoint{2.148767in}{3.357289in}}%
\pgfpathcurveto{\pgfqpoint{2.140954in}{3.349476in}}{\pgfqpoint{2.136564in}{3.338877in}}{\pgfqpoint{2.136564in}{3.327827in}}%
\pgfpathcurveto{\pgfqpoint{2.136564in}{3.316776in}}{\pgfqpoint{2.140954in}{3.306177in}}{\pgfqpoint{2.148767in}{3.298364in}}%
\pgfpathcurveto{\pgfqpoint{2.156581in}{3.290550in}}{\pgfqpoint{2.167180in}{3.286160in}}{\pgfqpoint{2.178230in}{3.286160in}}%
\pgfpathclose%
\pgfusepath{stroke,fill}%
\end{pgfscope}%
\begin{pgfscope}%
\pgfpathrectangle{\pgfqpoint{0.648703in}{0.548769in}}{\pgfqpoint{5.201297in}{3.102590in}}%
\pgfusepath{clip}%
\pgfsetbuttcap%
\pgfsetroundjoin%
\definecolor{currentfill}{rgb}{1.000000,0.498039,0.054902}%
\pgfsetfillcolor{currentfill}%
\pgfsetlinewidth{1.003750pt}%
\definecolor{currentstroke}{rgb}{1.000000,0.498039,0.054902}%
\pgfsetstrokecolor{currentstroke}%
\pgfsetdash{}{0pt}%
\pgfpathmoveto{\pgfqpoint{1.365390in}{3.136837in}}%
\pgfpathcurveto{\pgfqpoint{1.376441in}{3.136837in}}{\pgfqpoint{1.387040in}{3.141228in}}{\pgfqpoint{1.394853in}{3.149041in}}%
\pgfpathcurveto{\pgfqpoint{1.402667in}{3.156855in}}{\pgfqpoint{1.407057in}{3.167454in}}{\pgfqpoint{1.407057in}{3.178504in}}%
\pgfpathcurveto{\pgfqpoint{1.407057in}{3.189554in}}{\pgfqpoint{1.402667in}{3.200153in}}{\pgfqpoint{1.394853in}{3.207967in}}%
\pgfpathcurveto{\pgfqpoint{1.387040in}{3.215780in}}{\pgfqpoint{1.376441in}{3.220171in}}{\pgfqpoint{1.365390in}{3.220171in}}%
\pgfpathcurveto{\pgfqpoint{1.354340in}{3.220171in}}{\pgfqpoint{1.343741in}{3.215780in}}{\pgfqpoint{1.335928in}{3.207967in}}%
\pgfpathcurveto{\pgfqpoint{1.328114in}{3.200153in}}{\pgfqpoint{1.323724in}{3.189554in}}{\pgfqpoint{1.323724in}{3.178504in}}%
\pgfpathcurveto{\pgfqpoint{1.323724in}{3.167454in}}{\pgfqpoint{1.328114in}{3.156855in}}{\pgfqpoint{1.335928in}{3.149041in}}%
\pgfpathcurveto{\pgfqpoint{1.343741in}{3.141228in}}{\pgfqpoint{1.354340in}{3.136837in}}{\pgfqpoint{1.365390in}{3.136837in}}%
\pgfpathclose%
\pgfusepath{stroke,fill}%
\end{pgfscope}%
\begin{pgfscope}%
\pgfpathrectangle{\pgfqpoint{0.648703in}{0.548769in}}{\pgfqpoint{5.201297in}{3.102590in}}%
\pgfusepath{clip}%
\pgfsetbuttcap%
\pgfsetroundjoin%
\definecolor{currentfill}{rgb}{1.000000,0.498039,0.054902}%
\pgfsetfillcolor{currentfill}%
\pgfsetlinewidth{1.003750pt}%
\definecolor{currentstroke}{rgb}{1.000000,0.498039,0.054902}%
\pgfsetstrokecolor{currentstroke}%
\pgfsetdash{}{0pt}%
\pgfpathmoveto{\pgfqpoint{1.654978in}{3.149281in}}%
\pgfpathcurveto{\pgfqpoint{1.666028in}{3.149281in}}{\pgfqpoint{1.676627in}{3.153671in}}{\pgfqpoint{1.684441in}{3.161485in}}%
\pgfpathcurveto{\pgfqpoint{1.692254in}{3.169298in}}{\pgfqpoint{1.696645in}{3.179897in}}{\pgfqpoint{1.696645in}{3.190948in}}%
\pgfpathcurveto{\pgfqpoint{1.696645in}{3.201998in}}{\pgfqpoint{1.692254in}{3.212597in}}{\pgfqpoint{1.684441in}{3.220410in}}%
\pgfpathcurveto{\pgfqpoint{1.676627in}{3.228224in}}{\pgfqpoint{1.666028in}{3.232614in}}{\pgfqpoint{1.654978in}{3.232614in}}%
\pgfpathcurveto{\pgfqpoint{1.643928in}{3.232614in}}{\pgfqpoint{1.633329in}{3.228224in}}{\pgfqpoint{1.625515in}{3.220410in}}%
\pgfpathcurveto{\pgfqpoint{1.617702in}{3.212597in}}{\pgfqpoint{1.613311in}{3.201998in}}{\pgfqpoint{1.613311in}{3.190948in}}%
\pgfpathcurveto{\pgfqpoint{1.613311in}{3.179897in}}{\pgfqpoint{1.617702in}{3.169298in}}{\pgfqpoint{1.625515in}{3.161485in}}%
\pgfpathcurveto{\pgfqpoint{1.633329in}{3.153671in}}{\pgfqpoint{1.643928in}{3.149281in}}{\pgfqpoint{1.654978in}{3.149281in}}%
\pgfpathclose%
\pgfusepath{stroke,fill}%
\end{pgfscope}%
\begin{pgfscope}%
\pgfpathrectangle{\pgfqpoint{0.648703in}{0.548769in}}{\pgfqpoint{5.201297in}{3.102590in}}%
\pgfusepath{clip}%
\pgfsetbuttcap%
\pgfsetroundjoin%
\definecolor{currentfill}{rgb}{1.000000,0.498039,0.054902}%
\pgfsetfillcolor{currentfill}%
\pgfsetlinewidth{1.003750pt}%
\definecolor{currentstroke}{rgb}{1.000000,0.498039,0.054902}%
\pgfsetstrokecolor{currentstroke}%
\pgfsetdash{}{0pt}%
\pgfpathmoveto{\pgfqpoint{2.324372in}{3.128542in}}%
\pgfpathcurveto{\pgfqpoint{2.335422in}{3.128542in}}{\pgfqpoint{2.346021in}{3.132932in}}{\pgfqpoint{2.353834in}{3.140746in}}%
\pgfpathcurveto{\pgfqpoint{2.361648in}{3.148559in}}{\pgfqpoint{2.366038in}{3.159158in}}{\pgfqpoint{2.366038in}{3.170208in}}%
\pgfpathcurveto{\pgfqpoint{2.366038in}{3.181258in}}{\pgfqpoint{2.361648in}{3.191857in}}{\pgfqpoint{2.353834in}{3.199671in}}%
\pgfpathcurveto{\pgfqpoint{2.346021in}{3.207485in}}{\pgfqpoint{2.335422in}{3.211875in}}{\pgfqpoint{2.324372in}{3.211875in}}%
\pgfpathcurveto{\pgfqpoint{2.313322in}{3.211875in}}{\pgfqpoint{2.302723in}{3.207485in}}{\pgfqpoint{2.294909in}{3.199671in}}%
\pgfpathcurveto{\pgfqpoint{2.287095in}{3.191857in}}{\pgfqpoint{2.282705in}{3.181258in}}{\pgfqpoint{2.282705in}{3.170208in}}%
\pgfpathcurveto{\pgfqpoint{2.282705in}{3.159158in}}{\pgfqpoint{2.287095in}{3.148559in}}{\pgfqpoint{2.294909in}{3.140746in}}%
\pgfpathcurveto{\pgfqpoint{2.302723in}{3.132932in}}{\pgfqpoint{2.313322in}{3.128542in}}{\pgfqpoint{2.324372in}{3.128542in}}%
\pgfpathclose%
\pgfusepath{stroke,fill}%
\end{pgfscope}%
\begin{pgfscope}%
\pgfpathrectangle{\pgfqpoint{0.648703in}{0.548769in}}{\pgfqpoint{5.201297in}{3.102590in}}%
\pgfusepath{clip}%
\pgfsetbuttcap%
\pgfsetroundjoin%
\definecolor{currentfill}{rgb}{1.000000,0.498039,0.054902}%
\pgfsetfillcolor{currentfill}%
\pgfsetlinewidth{1.003750pt}%
\definecolor{currentstroke}{rgb}{1.000000,0.498039,0.054902}%
\pgfsetstrokecolor{currentstroke}%
\pgfsetdash{}{0pt}%
\pgfpathmoveto{\pgfqpoint{1.871299in}{3.232238in}}%
\pgfpathcurveto{\pgfqpoint{1.882349in}{3.232238in}}{\pgfqpoint{1.892948in}{3.236628in}}{\pgfqpoint{1.900762in}{3.244442in}}%
\pgfpathcurveto{\pgfqpoint{1.908575in}{3.252255in}}{\pgfqpoint{1.912965in}{3.262854in}}{\pgfqpoint{1.912965in}{3.273905in}}%
\pgfpathcurveto{\pgfqpoint{1.912965in}{3.284955in}}{\pgfqpoint{1.908575in}{3.295554in}}{\pgfqpoint{1.900762in}{3.303367in}}%
\pgfpathcurveto{\pgfqpoint{1.892948in}{3.311181in}}{\pgfqpoint{1.882349in}{3.315571in}}{\pgfqpoint{1.871299in}{3.315571in}}%
\pgfpathcurveto{\pgfqpoint{1.860249in}{3.315571in}}{\pgfqpoint{1.849650in}{3.311181in}}{\pgfqpoint{1.841836in}{3.303367in}}%
\pgfpathcurveto{\pgfqpoint{1.834022in}{3.295554in}}{\pgfqpoint{1.829632in}{3.284955in}}{\pgfqpoint{1.829632in}{3.273905in}}%
\pgfpathcurveto{\pgfqpoint{1.829632in}{3.262854in}}{\pgfqpoint{1.834022in}{3.252255in}}{\pgfqpoint{1.841836in}{3.244442in}}%
\pgfpathcurveto{\pgfqpoint{1.849650in}{3.236628in}}{\pgfqpoint{1.860249in}{3.232238in}}{\pgfqpoint{1.871299in}{3.232238in}}%
\pgfpathclose%
\pgfusepath{stroke,fill}%
\end{pgfscope}%
\begin{pgfscope}%
\pgfpathrectangle{\pgfqpoint{0.648703in}{0.548769in}}{\pgfqpoint{5.201297in}{3.102590in}}%
\pgfusepath{clip}%
\pgfsetbuttcap%
\pgfsetroundjoin%
\definecolor{currentfill}{rgb}{0.121569,0.466667,0.705882}%
\pgfsetfillcolor{currentfill}%
\pgfsetlinewidth{1.003750pt}%
\definecolor{currentstroke}{rgb}{0.121569,0.466667,0.705882}%
\pgfsetstrokecolor{currentstroke}%
\pgfsetdash{}{0pt}%
\pgfpathmoveto{\pgfqpoint{1.227248in}{0.648129in}}%
\pgfpathcurveto{\pgfqpoint{1.238298in}{0.648129in}}{\pgfqpoint{1.248897in}{0.652519in}}{\pgfqpoint{1.256711in}{0.660333in}}%
\pgfpathcurveto{\pgfqpoint{1.264524in}{0.668146in}}{\pgfqpoint{1.268915in}{0.678745in}}{\pgfqpoint{1.268915in}{0.689796in}}%
\pgfpathcurveto{\pgfqpoint{1.268915in}{0.700846in}}{\pgfqpoint{1.264524in}{0.711445in}}{\pgfqpoint{1.256711in}{0.719258in}}%
\pgfpathcurveto{\pgfqpoint{1.248897in}{0.727072in}}{\pgfqpoint{1.238298in}{0.731462in}}{\pgfqpoint{1.227248in}{0.731462in}}%
\pgfpathcurveto{\pgfqpoint{1.216198in}{0.731462in}}{\pgfqpoint{1.205599in}{0.727072in}}{\pgfqpoint{1.197785in}{0.719258in}}%
\pgfpathcurveto{\pgfqpoint{1.189972in}{0.711445in}}{\pgfqpoint{1.185581in}{0.700846in}}{\pgfqpoint{1.185581in}{0.689796in}}%
\pgfpathcurveto{\pgfqpoint{1.185581in}{0.678745in}}{\pgfqpoint{1.189972in}{0.668146in}}{\pgfqpoint{1.197785in}{0.660333in}}%
\pgfpathcurveto{\pgfqpoint{1.205599in}{0.652519in}}{\pgfqpoint{1.216198in}{0.648129in}}{\pgfqpoint{1.227248in}{0.648129in}}%
\pgfpathclose%
\pgfusepath{stroke,fill}%
\end{pgfscope}%
\begin{pgfscope}%
\pgfpathrectangle{\pgfqpoint{0.648703in}{0.548769in}}{\pgfqpoint{5.201297in}{3.102590in}}%
\pgfusepath{clip}%
\pgfsetbuttcap%
\pgfsetroundjoin%
\definecolor{currentfill}{rgb}{1.000000,0.498039,0.054902}%
\pgfsetfillcolor{currentfill}%
\pgfsetlinewidth{1.003750pt}%
\definecolor{currentstroke}{rgb}{1.000000,0.498039,0.054902}%
\pgfsetstrokecolor{currentstroke}%
\pgfsetdash{}{0pt}%
\pgfpathmoveto{\pgfqpoint{2.173089in}{3.136837in}}%
\pgfpathcurveto{\pgfqpoint{2.184139in}{3.136837in}}{\pgfqpoint{2.194738in}{3.141228in}}{\pgfqpoint{2.202552in}{3.149041in}}%
\pgfpathcurveto{\pgfqpoint{2.210365in}{3.156855in}}{\pgfqpoint{2.214755in}{3.167454in}}{\pgfqpoint{2.214755in}{3.178504in}}%
\pgfpathcurveto{\pgfqpoint{2.214755in}{3.189554in}}{\pgfqpoint{2.210365in}{3.200153in}}{\pgfqpoint{2.202552in}{3.207967in}}%
\pgfpathcurveto{\pgfqpoint{2.194738in}{3.215780in}}{\pgfqpoint{2.184139in}{3.220171in}}{\pgfqpoint{2.173089in}{3.220171in}}%
\pgfpathcurveto{\pgfqpoint{2.162039in}{3.220171in}}{\pgfqpoint{2.151440in}{3.215780in}}{\pgfqpoint{2.143626in}{3.207967in}}%
\pgfpathcurveto{\pgfqpoint{2.135812in}{3.200153in}}{\pgfqpoint{2.131422in}{3.189554in}}{\pgfqpoint{2.131422in}{3.178504in}}%
\pgfpathcurveto{\pgfqpoint{2.131422in}{3.167454in}}{\pgfqpoint{2.135812in}{3.156855in}}{\pgfqpoint{2.143626in}{3.149041in}}%
\pgfpathcurveto{\pgfqpoint{2.151440in}{3.141228in}}{\pgfqpoint{2.162039in}{3.136837in}}{\pgfqpoint{2.173089in}{3.136837in}}%
\pgfpathclose%
\pgfusepath{stroke,fill}%
\end{pgfscope}%
\begin{pgfscope}%
\pgfpathrectangle{\pgfqpoint{0.648703in}{0.548769in}}{\pgfqpoint{5.201297in}{3.102590in}}%
\pgfusepath{clip}%
\pgfsetbuttcap%
\pgfsetroundjoin%
\definecolor{currentfill}{rgb}{1.000000,0.498039,0.054902}%
\pgfsetfillcolor{currentfill}%
\pgfsetlinewidth{1.003750pt}%
\definecolor{currentstroke}{rgb}{1.000000,0.498039,0.054902}%
\pgfsetstrokecolor{currentstroke}%
\pgfsetdash{}{0pt}%
\pgfpathmoveto{\pgfqpoint{1.189089in}{3.140985in}}%
\pgfpathcurveto{\pgfqpoint{1.200139in}{3.140985in}}{\pgfqpoint{1.210738in}{3.145375in}}{\pgfqpoint{1.218552in}{3.153189in}}%
\pgfpathcurveto{\pgfqpoint{1.226365in}{3.161003in}}{\pgfqpoint{1.230756in}{3.171602in}}{\pgfqpoint{1.230756in}{3.182652in}}%
\pgfpathcurveto{\pgfqpoint{1.230756in}{3.193702in}}{\pgfqpoint{1.226365in}{3.204301in}}{\pgfqpoint{1.218552in}{3.212115in}}%
\pgfpathcurveto{\pgfqpoint{1.210738in}{3.219928in}}{\pgfqpoint{1.200139in}{3.224319in}}{\pgfqpoint{1.189089in}{3.224319in}}%
\pgfpathcurveto{\pgfqpoint{1.178039in}{3.224319in}}{\pgfqpoint{1.167440in}{3.219928in}}{\pgfqpoint{1.159626in}{3.212115in}}%
\pgfpathcurveto{\pgfqpoint{1.151813in}{3.204301in}}{\pgfqpoint{1.147422in}{3.193702in}}{\pgfqpoint{1.147422in}{3.182652in}}%
\pgfpathcurveto{\pgfqpoint{1.147422in}{3.171602in}}{\pgfqpoint{1.151813in}{3.161003in}}{\pgfqpoint{1.159626in}{3.153189in}}%
\pgfpathcurveto{\pgfqpoint{1.167440in}{3.145375in}}{\pgfqpoint{1.178039in}{3.140985in}}{\pgfqpoint{1.189089in}{3.140985in}}%
\pgfpathclose%
\pgfusepath{stroke,fill}%
\end{pgfscope}%
\begin{pgfscope}%
\pgfpathrectangle{\pgfqpoint{0.648703in}{0.548769in}}{\pgfqpoint{5.201297in}{3.102590in}}%
\pgfusepath{clip}%
\pgfsetbuttcap%
\pgfsetroundjoin%
\definecolor{currentfill}{rgb}{1.000000,0.498039,0.054902}%
\pgfsetfillcolor{currentfill}%
\pgfsetlinewidth{1.003750pt}%
\definecolor{currentstroke}{rgb}{1.000000,0.498039,0.054902}%
\pgfsetstrokecolor{currentstroke}%
\pgfsetdash{}{0pt}%
\pgfpathmoveto{\pgfqpoint{1.170356in}{3.149281in}}%
\pgfpathcurveto{\pgfqpoint{1.181406in}{3.149281in}}{\pgfqpoint{1.192005in}{3.153671in}}{\pgfqpoint{1.199819in}{3.161485in}}%
\pgfpathcurveto{\pgfqpoint{1.207632in}{3.169298in}}{\pgfqpoint{1.212022in}{3.179897in}}{\pgfqpoint{1.212022in}{3.190948in}}%
\pgfpathcurveto{\pgfqpoint{1.212022in}{3.201998in}}{\pgfqpoint{1.207632in}{3.212597in}}{\pgfqpoint{1.199819in}{3.220410in}}%
\pgfpathcurveto{\pgfqpoint{1.192005in}{3.228224in}}{\pgfqpoint{1.181406in}{3.232614in}}{\pgfqpoint{1.170356in}{3.232614in}}%
\pgfpathcurveto{\pgfqpoint{1.159306in}{3.232614in}}{\pgfqpoint{1.148707in}{3.228224in}}{\pgfqpoint{1.140893in}{3.220410in}}%
\pgfpathcurveto{\pgfqpoint{1.133079in}{3.212597in}}{\pgfqpoint{1.128689in}{3.201998in}}{\pgfqpoint{1.128689in}{3.190948in}}%
\pgfpathcurveto{\pgfqpoint{1.128689in}{3.179897in}}{\pgfqpoint{1.133079in}{3.169298in}}{\pgfqpoint{1.140893in}{3.161485in}}%
\pgfpathcurveto{\pgfqpoint{1.148707in}{3.153671in}}{\pgfqpoint{1.159306in}{3.149281in}}{\pgfqpoint{1.170356in}{3.149281in}}%
\pgfpathclose%
\pgfusepath{stroke,fill}%
\end{pgfscope}%
\begin{pgfscope}%
\pgfpathrectangle{\pgfqpoint{0.648703in}{0.548769in}}{\pgfqpoint{5.201297in}{3.102590in}}%
\pgfusepath{clip}%
\pgfsetbuttcap%
\pgfsetroundjoin%
\definecolor{currentfill}{rgb}{1.000000,0.498039,0.054902}%
\pgfsetfillcolor{currentfill}%
\pgfsetlinewidth{1.003750pt}%
\definecolor{currentstroke}{rgb}{1.000000,0.498039,0.054902}%
\pgfsetstrokecolor{currentstroke}%
\pgfsetdash{}{0pt}%
\pgfpathmoveto{\pgfqpoint{2.065696in}{3.140985in}}%
\pgfpathcurveto{\pgfqpoint{2.076746in}{3.140985in}}{\pgfqpoint{2.087345in}{3.145375in}}{\pgfqpoint{2.095159in}{3.153189in}}%
\pgfpathcurveto{\pgfqpoint{2.102973in}{3.161003in}}{\pgfqpoint{2.107363in}{3.171602in}}{\pgfqpoint{2.107363in}{3.182652in}}%
\pgfpathcurveto{\pgfqpoint{2.107363in}{3.193702in}}{\pgfqpoint{2.102973in}{3.204301in}}{\pgfqpoint{2.095159in}{3.212115in}}%
\pgfpathcurveto{\pgfqpoint{2.087345in}{3.219928in}}{\pgfqpoint{2.076746in}{3.224319in}}{\pgfqpoint{2.065696in}{3.224319in}}%
\pgfpathcurveto{\pgfqpoint{2.054646in}{3.224319in}}{\pgfqpoint{2.044047in}{3.219928in}}{\pgfqpoint{2.036233in}{3.212115in}}%
\pgfpathcurveto{\pgfqpoint{2.028420in}{3.204301in}}{\pgfqpoint{2.024030in}{3.193702in}}{\pgfqpoint{2.024030in}{3.182652in}}%
\pgfpathcurveto{\pgfqpoint{2.024030in}{3.171602in}}{\pgfqpoint{2.028420in}{3.161003in}}{\pgfqpoint{2.036233in}{3.153189in}}%
\pgfpathcurveto{\pgfqpoint{2.044047in}{3.145375in}}{\pgfqpoint{2.054646in}{3.140985in}}{\pgfqpoint{2.065696in}{3.140985in}}%
\pgfpathclose%
\pgfusepath{stroke,fill}%
\end{pgfscope}%
\begin{pgfscope}%
\pgfpathrectangle{\pgfqpoint{0.648703in}{0.548769in}}{\pgfqpoint{5.201297in}{3.102590in}}%
\pgfusepath{clip}%
\pgfsetbuttcap%
\pgfsetroundjoin%
\definecolor{currentfill}{rgb}{1.000000,0.498039,0.054902}%
\pgfsetfillcolor{currentfill}%
\pgfsetlinewidth{1.003750pt}%
\definecolor{currentstroke}{rgb}{1.000000,0.498039,0.054902}%
\pgfsetstrokecolor{currentstroke}%
\pgfsetdash{}{0pt}%
\pgfpathmoveto{\pgfqpoint{1.739845in}{3.136837in}}%
\pgfpathcurveto{\pgfqpoint{1.750896in}{3.136837in}}{\pgfqpoint{1.761495in}{3.141228in}}{\pgfqpoint{1.769308in}{3.149041in}}%
\pgfpathcurveto{\pgfqpoint{1.777122in}{3.156855in}}{\pgfqpoint{1.781512in}{3.167454in}}{\pgfqpoint{1.781512in}{3.178504in}}%
\pgfpathcurveto{\pgfqpoint{1.781512in}{3.189554in}}{\pgfqpoint{1.777122in}{3.200153in}}{\pgfqpoint{1.769308in}{3.207967in}}%
\pgfpathcurveto{\pgfqpoint{1.761495in}{3.215780in}}{\pgfqpoint{1.750896in}{3.220171in}}{\pgfqpoint{1.739845in}{3.220171in}}%
\pgfpathcurveto{\pgfqpoint{1.728795in}{3.220171in}}{\pgfqpoint{1.718196in}{3.215780in}}{\pgfqpoint{1.710383in}{3.207967in}}%
\pgfpathcurveto{\pgfqpoint{1.702569in}{3.200153in}}{\pgfqpoint{1.698179in}{3.189554in}}{\pgfqpoint{1.698179in}{3.178504in}}%
\pgfpathcurveto{\pgfqpoint{1.698179in}{3.167454in}}{\pgfqpoint{1.702569in}{3.156855in}}{\pgfqpoint{1.710383in}{3.149041in}}%
\pgfpathcurveto{\pgfqpoint{1.718196in}{3.141228in}}{\pgfqpoint{1.728795in}{3.136837in}}{\pgfqpoint{1.739845in}{3.136837in}}%
\pgfpathclose%
\pgfusepath{stroke,fill}%
\end{pgfscope}%
\begin{pgfscope}%
\pgfpathrectangle{\pgfqpoint{0.648703in}{0.548769in}}{\pgfqpoint{5.201297in}{3.102590in}}%
\pgfusepath{clip}%
\pgfsetbuttcap%
\pgfsetroundjoin%
\definecolor{currentfill}{rgb}{1.000000,0.498039,0.054902}%
\pgfsetfillcolor{currentfill}%
\pgfsetlinewidth{1.003750pt}%
\definecolor{currentstroke}{rgb}{1.000000,0.498039,0.054902}%
\pgfsetstrokecolor{currentstroke}%
\pgfsetdash{}{0pt}%
\pgfpathmoveto{\pgfqpoint{1.664287in}{3.145133in}}%
\pgfpathcurveto{\pgfqpoint{1.675337in}{3.145133in}}{\pgfqpoint{1.685936in}{3.149523in}}{\pgfqpoint{1.693750in}{3.157337in}}%
\pgfpathcurveto{\pgfqpoint{1.701564in}{3.165151in}}{\pgfqpoint{1.705954in}{3.175750in}}{\pgfqpoint{1.705954in}{3.186800in}}%
\pgfpathcurveto{\pgfqpoint{1.705954in}{3.197850in}}{\pgfqpoint{1.701564in}{3.208449in}}{\pgfqpoint{1.693750in}{3.216262in}}%
\pgfpathcurveto{\pgfqpoint{1.685936in}{3.224076in}}{\pgfqpoint{1.675337in}{3.228466in}}{\pgfqpoint{1.664287in}{3.228466in}}%
\pgfpathcurveto{\pgfqpoint{1.653237in}{3.228466in}}{\pgfqpoint{1.642638in}{3.224076in}}{\pgfqpoint{1.634824in}{3.216262in}}%
\pgfpathcurveto{\pgfqpoint{1.627011in}{3.208449in}}{\pgfqpoint{1.622620in}{3.197850in}}{\pgfqpoint{1.622620in}{3.186800in}}%
\pgfpathcurveto{\pgfqpoint{1.622620in}{3.175750in}}{\pgfqpoint{1.627011in}{3.165151in}}{\pgfqpoint{1.634824in}{3.157337in}}%
\pgfpathcurveto{\pgfqpoint{1.642638in}{3.149523in}}{\pgfqpoint{1.653237in}{3.145133in}}{\pgfqpoint{1.664287in}{3.145133in}}%
\pgfpathclose%
\pgfusepath{stroke,fill}%
\end{pgfscope}%
\begin{pgfscope}%
\pgfpathrectangle{\pgfqpoint{0.648703in}{0.548769in}}{\pgfqpoint{5.201297in}{3.102590in}}%
\pgfusepath{clip}%
\pgfsetbuttcap%
\pgfsetroundjoin%
\definecolor{currentfill}{rgb}{1.000000,0.498039,0.054902}%
\pgfsetfillcolor{currentfill}%
\pgfsetlinewidth{1.003750pt}%
\definecolor{currentstroke}{rgb}{1.000000,0.498039,0.054902}%
\pgfsetstrokecolor{currentstroke}%
\pgfsetdash{}{0pt}%
\pgfpathmoveto{\pgfqpoint{2.149020in}{3.132690in}}%
\pgfpathcurveto{\pgfqpoint{2.160070in}{3.132690in}}{\pgfqpoint{2.170669in}{3.137080in}}{\pgfqpoint{2.178483in}{3.144893in}}%
\pgfpathcurveto{\pgfqpoint{2.186297in}{3.152707in}}{\pgfqpoint{2.190687in}{3.163306in}}{\pgfqpoint{2.190687in}{3.174356in}}%
\pgfpathcurveto{\pgfqpoint{2.190687in}{3.185406in}}{\pgfqpoint{2.186297in}{3.196005in}}{\pgfqpoint{2.178483in}{3.203819in}}%
\pgfpathcurveto{\pgfqpoint{2.170669in}{3.211633in}}{\pgfqpoint{2.160070in}{3.216023in}}{\pgfqpoint{2.149020in}{3.216023in}}%
\pgfpathcurveto{\pgfqpoint{2.137970in}{3.216023in}}{\pgfqpoint{2.127371in}{3.211633in}}{\pgfqpoint{2.119557in}{3.203819in}}%
\pgfpathcurveto{\pgfqpoint{2.111744in}{3.196005in}}{\pgfqpoint{2.107353in}{3.185406in}}{\pgfqpoint{2.107353in}{3.174356in}}%
\pgfpathcurveto{\pgfqpoint{2.107353in}{3.163306in}}{\pgfqpoint{2.111744in}{3.152707in}}{\pgfqpoint{2.119557in}{3.144893in}}%
\pgfpathcurveto{\pgfqpoint{2.127371in}{3.137080in}}{\pgfqpoint{2.137970in}{3.132690in}}{\pgfqpoint{2.149020in}{3.132690in}}%
\pgfpathclose%
\pgfusepath{stroke,fill}%
\end{pgfscope}%
\begin{pgfscope}%
\pgfpathrectangle{\pgfqpoint{0.648703in}{0.548769in}}{\pgfqpoint{5.201297in}{3.102590in}}%
\pgfusepath{clip}%
\pgfsetbuttcap%
\pgfsetroundjoin%
\definecolor{currentfill}{rgb}{1.000000,0.498039,0.054902}%
\pgfsetfillcolor{currentfill}%
\pgfsetlinewidth{1.003750pt}%
\definecolor{currentstroke}{rgb}{1.000000,0.498039,0.054902}%
\pgfsetstrokecolor{currentstroke}%
\pgfsetdash{}{0pt}%
\pgfpathmoveto{\pgfqpoint{2.053834in}{3.348378in}}%
\pgfpathcurveto{\pgfqpoint{2.064884in}{3.348378in}}{\pgfqpoint{2.075483in}{3.352768in}}{\pgfqpoint{2.083297in}{3.360581in}}%
\pgfpathcurveto{\pgfqpoint{2.091110in}{3.368395in}}{\pgfqpoint{2.095501in}{3.378994in}}{\pgfqpoint{2.095501in}{3.390044in}}%
\pgfpathcurveto{\pgfqpoint{2.095501in}{3.401094in}}{\pgfqpoint{2.091110in}{3.411693in}}{\pgfqpoint{2.083297in}{3.419507in}}%
\pgfpathcurveto{\pgfqpoint{2.075483in}{3.427321in}}{\pgfqpoint{2.064884in}{3.431711in}}{\pgfqpoint{2.053834in}{3.431711in}}%
\pgfpathcurveto{\pgfqpoint{2.042784in}{3.431711in}}{\pgfqpoint{2.032185in}{3.427321in}}{\pgfqpoint{2.024371in}{3.419507in}}%
\pgfpathcurveto{\pgfqpoint{2.016558in}{3.411693in}}{\pgfqpoint{2.012167in}{3.401094in}}{\pgfqpoint{2.012167in}{3.390044in}}%
\pgfpathcurveto{\pgfqpoint{2.012167in}{3.378994in}}{\pgfqpoint{2.016558in}{3.368395in}}{\pgfqpoint{2.024371in}{3.360581in}}%
\pgfpathcurveto{\pgfqpoint{2.032185in}{3.352768in}}{\pgfqpoint{2.042784in}{3.348378in}}{\pgfqpoint{2.053834in}{3.348378in}}%
\pgfpathclose%
\pgfusepath{stroke,fill}%
\end{pgfscope}%
\begin{pgfscope}%
\pgfsetbuttcap%
\pgfsetroundjoin%
\definecolor{currentfill}{rgb}{0.000000,0.000000,0.000000}%
\pgfsetfillcolor{currentfill}%
\pgfsetlinewidth{0.803000pt}%
\definecolor{currentstroke}{rgb}{0.000000,0.000000,0.000000}%
\pgfsetstrokecolor{currentstroke}%
\pgfsetdash{}{0pt}%
\pgfsys@defobject{currentmarker}{\pgfqpoint{0.000000in}{-0.048611in}}{\pgfqpoint{0.000000in}{0.000000in}}{%
\pgfpathmoveto{\pgfqpoint{0.000000in}{0.000000in}}%
\pgfpathlineto{\pgfqpoint{0.000000in}{-0.048611in}}%
\pgfusepath{stroke,fill}%
}%
\begin{pgfscope}%
\pgfsys@transformshift{0.872963in}{0.548769in}%
\pgfsys@useobject{currentmarker}{}%
\end{pgfscope}%
\end{pgfscope}%
\begin{pgfscope}%
\definecolor{textcolor}{rgb}{0.000000,0.000000,0.000000}%
\pgfsetstrokecolor{textcolor}%
\pgfsetfillcolor{textcolor}%
\pgftext[x=0.872963in,y=0.451547in,,top]{\color{textcolor}\sffamily\fontsize{10.000000}{12.000000}\selectfont \(\displaystyle {0.0}\)}%
\end{pgfscope}%
\begin{pgfscope}%
\pgfsetbuttcap%
\pgfsetroundjoin%
\definecolor{currentfill}{rgb}{0.000000,0.000000,0.000000}%
\pgfsetfillcolor{currentfill}%
\pgfsetlinewidth{0.803000pt}%
\definecolor{currentstroke}{rgb}{0.000000,0.000000,0.000000}%
\pgfsetstrokecolor{currentstroke}%
\pgfsetdash{}{0pt}%
\pgfsys@defobject{currentmarker}{\pgfqpoint{0.000000in}{-0.048611in}}{\pgfqpoint{0.000000in}{0.000000in}}{%
\pgfpathmoveto{\pgfqpoint{0.000000in}{0.000000in}}%
\pgfpathlineto{\pgfqpoint{0.000000in}{-0.048611in}}%
\pgfusepath{stroke,fill}%
}%
\begin{pgfscope}%
\pgfsys@transformshift{1.664564in}{0.548769in}%
\pgfsys@useobject{currentmarker}{}%
\end{pgfscope}%
\end{pgfscope}%
\begin{pgfscope}%
\definecolor{textcolor}{rgb}{0.000000,0.000000,0.000000}%
\pgfsetstrokecolor{textcolor}%
\pgfsetfillcolor{textcolor}%
\pgftext[x=1.664564in,y=0.451547in,,top]{\color{textcolor}\sffamily\fontsize{10.000000}{12.000000}\selectfont \(\displaystyle {0.2}\)}%
\end{pgfscope}%
\begin{pgfscope}%
\pgfsetbuttcap%
\pgfsetroundjoin%
\definecolor{currentfill}{rgb}{0.000000,0.000000,0.000000}%
\pgfsetfillcolor{currentfill}%
\pgfsetlinewidth{0.803000pt}%
\definecolor{currentstroke}{rgb}{0.000000,0.000000,0.000000}%
\pgfsetstrokecolor{currentstroke}%
\pgfsetdash{}{0pt}%
\pgfsys@defobject{currentmarker}{\pgfqpoint{0.000000in}{-0.048611in}}{\pgfqpoint{0.000000in}{0.000000in}}{%
\pgfpathmoveto{\pgfqpoint{0.000000in}{0.000000in}}%
\pgfpathlineto{\pgfqpoint{0.000000in}{-0.048611in}}%
\pgfusepath{stroke,fill}%
}%
\begin{pgfscope}%
\pgfsys@transformshift{2.456165in}{0.548769in}%
\pgfsys@useobject{currentmarker}{}%
\end{pgfscope}%
\end{pgfscope}%
\begin{pgfscope}%
\definecolor{textcolor}{rgb}{0.000000,0.000000,0.000000}%
\pgfsetstrokecolor{textcolor}%
\pgfsetfillcolor{textcolor}%
\pgftext[x=2.456165in,y=0.451547in,,top]{\color{textcolor}\sffamily\fontsize{10.000000}{12.000000}\selectfont \(\displaystyle {0.4}\)}%
\end{pgfscope}%
\begin{pgfscope}%
\pgfsetbuttcap%
\pgfsetroundjoin%
\definecolor{currentfill}{rgb}{0.000000,0.000000,0.000000}%
\pgfsetfillcolor{currentfill}%
\pgfsetlinewidth{0.803000pt}%
\definecolor{currentstroke}{rgb}{0.000000,0.000000,0.000000}%
\pgfsetstrokecolor{currentstroke}%
\pgfsetdash{}{0pt}%
\pgfsys@defobject{currentmarker}{\pgfqpoint{0.000000in}{-0.048611in}}{\pgfqpoint{0.000000in}{0.000000in}}{%
\pgfpathmoveto{\pgfqpoint{0.000000in}{0.000000in}}%
\pgfpathlineto{\pgfqpoint{0.000000in}{-0.048611in}}%
\pgfusepath{stroke,fill}%
}%
\begin{pgfscope}%
\pgfsys@transformshift{3.247767in}{0.548769in}%
\pgfsys@useobject{currentmarker}{}%
\end{pgfscope}%
\end{pgfscope}%
\begin{pgfscope}%
\definecolor{textcolor}{rgb}{0.000000,0.000000,0.000000}%
\pgfsetstrokecolor{textcolor}%
\pgfsetfillcolor{textcolor}%
\pgftext[x=3.247767in,y=0.451547in,,top]{\color{textcolor}\sffamily\fontsize{10.000000}{12.000000}\selectfont \(\displaystyle {0.6}\)}%
\end{pgfscope}%
\begin{pgfscope}%
\pgfsetbuttcap%
\pgfsetroundjoin%
\definecolor{currentfill}{rgb}{0.000000,0.000000,0.000000}%
\pgfsetfillcolor{currentfill}%
\pgfsetlinewidth{0.803000pt}%
\definecolor{currentstroke}{rgb}{0.000000,0.000000,0.000000}%
\pgfsetstrokecolor{currentstroke}%
\pgfsetdash{}{0pt}%
\pgfsys@defobject{currentmarker}{\pgfqpoint{0.000000in}{-0.048611in}}{\pgfqpoint{0.000000in}{0.000000in}}{%
\pgfpathmoveto{\pgfqpoint{0.000000in}{0.000000in}}%
\pgfpathlineto{\pgfqpoint{0.000000in}{-0.048611in}}%
\pgfusepath{stroke,fill}%
}%
\begin{pgfscope}%
\pgfsys@transformshift{4.039368in}{0.548769in}%
\pgfsys@useobject{currentmarker}{}%
\end{pgfscope}%
\end{pgfscope}%
\begin{pgfscope}%
\definecolor{textcolor}{rgb}{0.000000,0.000000,0.000000}%
\pgfsetstrokecolor{textcolor}%
\pgfsetfillcolor{textcolor}%
\pgftext[x=4.039368in,y=0.451547in,,top]{\color{textcolor}\sffamily\fontsize{10.000000}{12.000000}\selectfont \(\displaystyle {0.8}\)}%
\end{pgfscope}%
\begin{pgfscope}%
\pgfsetbuttcap%
\pgfsetroundjoin%
\definecolor{currentfill}{rgb}{0.000000,0.000000,0.000000}%
\pgfsetfillcolor{currentfill}%
\pgfsetlinewidth{0.803000pt}%
\definecolor{currentstroke}{rgb}{0.000000,0.000000,0.000000}%
\pgfsetstrokecolor{currentstroke}%
\pgfsetdash{}{0pt}%
\pgfsys@defobject{currentmarker}{\pgfqpoint{0.000000in}{-0.048611in}}{\pgfqpoint{0.000000in}{0.000000in}}{%
\pgfpathmoveto{\pgfqpoint{0.000000in}{0.000000in}}%
\pgfpathlineto{\pgfqpoint{0.000000in}{-0.048611in}}%
\pgfusepath{stroke,fill}%
}%
\begin{pgfscope}%
\pgfsys@transformshift{4.830969in}{0.548769in}%
\pgfsys@useobject{currentmarker}{}%
\end{pgfscope}%
\end{pgfscope}%
\begin{pgfscope}%
\definecolor{textcolor}{rgb}{0.000000,0.000000,0.000000}%
\pgfsetstrokecolor{textcolor}%
\pgfsetfillcolor{textcolor}%
\pgftext[x=4.830969in,y=0.451547in,,top]{\color{textcolor}\sffamily\fontsize{10.000000}{12.000000}\selectfont \(\displaystyle {1.0}\)}%
\end{pgfscope}%
\begin{pgfscope}%
\pgfsetbuttcap%
\pgfsetroundjoin%
\definecolor{currentfill}{rgb}{0.000000,0.000000,0.000000}%
\pgfsetfillcolor{currentfill}%
\pgfsetlinewidth{0.803000pt}%
\definecolor{currentstroke}{rgb}{0.000000,0.000000,0.000000}%
\pgfsetstrokecolor{currentstroke}%
\pgfsetdash{}{0pt}%
\pgfsys@defobject{currentmarker}{\pgfqpoint{0.000000in}{-0.048611in}}{\pgfqpoint{0.000000in}{0.000000in}}{%
\pgfpathmoveto{\pgfqpoint{0.000000in}{0.000000in}}%
\pgfpathlineto{\pgfqpoint{0.000000in}{-0.048611in}}%
\pgfusepath{stroke,fill}%
}%
\begin{pgfscope}%
\pgfsys@transformshift{5.622570in}{0.548769in}%
\pgfsys@useobject{currentmarker}{}%
\end{pgfscope}%
\end{pgfscope}%
\begin{pgfscope}%
\definecolor{textcolor}{rgb}{0.000000,0.000000,0.000000}%
\pgfsetstrokecolor{textcolor}%
\pgfsetfillcolor{textcolor}%
\pgftext[x=5.622570in,y=0.451547in,,top]{\color{textcolor}\sffamily\fontsize{10.000000}{12.000000}\selectfont \(\displaystyle {1.2}\)}%
\end{pgfscope}%
\begin{pgfscope}%
\definecolor{textcolor}{rgb}{0.000000,0.000000,0.000000}%
\pgfsetstrokecolor{textcolor}%
\pgfsetfillcolor{textcolor}%
\pgftext[x=3.249352in,y=0.272658in,,top]{\color{textcolor}\sffamily\fontsize{10.000000}{12.000000}\selectfont Statements}%
\end{pgfscope}%
\begin{pgfscope}%
\definecolor{textcolor}{rgb}{0.000000,0.000000,0.000000}%
\pgfsetstrokecolor{textcolor}%
\pgfsetfillcolor{textcolor}%
\pgftext[x=5.850000in,y=0.286547in,right,top]{\color{textcolor}\sffamily\fontsize{10.000000}{12.000000}\selectfont \(\displaystyle \times{10^{6}}{}\)}%
\end{pgfscope}%
\begin{pgfscope}%
\pgfsetbuttcap%
\pgfsetroundjoin%
\definecolor{currentfill}{rgb}{0.000000,0.000000,0.000000}%
\pgfsetfillcolor{currentfill}%
\pgfsetlinewidth{0.803000pt}%
\definecolor{currentstroke}{rgb}{0.000000,0.000000,0.000000}%
\pgfsetstrokecolor{currentstroke}%
\pgfsetdash{}{0pt}%
\pgfsys@defobject{currentmarker}{\pgfqpoint{-0.048611in}{0.000000in}}{\pgfqpoint{0.000000in}{0.000000in}}{%
\pgfpathmoveto{\pgfqpoint{0.000000in}{0.000000in}}%
\pgfpathlineto{\pgfqpoint{-0.048611in}{0.000000in}}%
\pgfusepath{stroke,fill}%
}%
\begin{pgfscope}%
\pgfsys@transformshift{0.648703in}{0.689796in}%
\pgfsys@useobject{currentmarker}{}%
\end{pgfscope}%
\end{pgfscope}%
\begin{pgfscope}%
\definecolor{textcolor}{rgb}{0.000000,0.000000,0.000000}%
\pgfsetstrokecolor{textcolor}%
\pgfsetfillcolor{textcolor}%
\pgftext[x=0.482036in, y=0.641601in, left, base]{\color{textcolor}\sffamily\fontsize{10.000000}{12.000000}\selectfont \(\displaystyle {0}\)}%
\end{pgfscope}%
\begin{pgfscope}%
\pgfsetbuttcap%
\pgfsetroundjoin%
\definecolor{currentfill}{rgb}{0.000000,0.000000,0.000000}%
\pgfsetfillcolor{currentfill}%
\pgfsetlinewidth{0.803000pt}%
\definecolor{currentstroke}{rgb}{0.000000,0.000000,0.000000}%
\pgfsetstrokecolor{currentstroke}%
\pgfsetdash{}{0pt}%
\pgfsys@defobject{currentmarker}{\pgfqpoint{-0.048611in}{0.000000in}}{\pgfqpoint{0.000000in}{0.000000in}}{%
\pgfpathmoveto{\pgfqpoint{0.000000in}{0.000000in}}%
\pgfpathlineto{\pgfqpoint{-0.048611in}{0.000000in}}%
\pgfusepath{stroke,fill}%
}%
\begin{pgfscope}%
\pgfsys@transformshift{0.648703in}{1.104580in}%
\pgfsys@useobject{currentmarker}{}%
\end{pgfscope}%
\end{pgfscope}%
\begin{pgfscope}%
\definecolor{textcolor}{rgb}{0.000000,0.000000,0.000000}%
\pgfsetstrokecolor{textcolor}%
\pgfsetfillcolor{textcolor}%
\pgftext[x=0.343147in, y=1.056386in, left, base]{\color{textcolor}\sffamily\fontsize{10.000000}{12.000000}\selectfont \(\displaystyle {100}\)}%
\end{pgfscope}%
\begin{pgfscope}%
\pgfsetbuttcap%
\pgfsetroundjoin%
\definecolor{currentfill}{rgb}{0.000000,0.000000,0.000000}%
\pgfsetfillcolor{currentfill}%
\pgfsetlinewidth{0.803000pt}%
\definecolor{currentstroke}{rgb}{0.000000,0.000000,0.000000}%
\pgfsetstrokecolor{currentstroke}%
\pgfsetdash{}{0pt}%
\pgfsys@defobject{currentmarker}{\pgfqpoint{-0.048611in}{0.000000in}}{\pgfqpoint{0.000000in}{0.000000in}}{%
\pgfpathmoveto{\pgfqpoint{0.000000in}{0.000000in}}%
\pgfpathlineto{\pgfqpoint{-0.048611in}{0.000000in}}%
\pgfusepath{stroke,fill}%
}%
\begin{pgfscope}%
\pgfsys@transformshift{0.648703in}{1.519365in}%
\pgfsys@useobject{currentmarker}{}%
\end{pgfscope}%
\end{pgfscope}%
\begin{pgfscope}%
\definecolor{textcolor}{rgb}{0.000000,0.000000,0.000000}%
\pgfsetstrokecolor{textcolor}%
\pgfsetfillcolor{textcolor}%
\pgftext[x=0.343147in, y=1.471171in, left, base]{\color{textcolor}\sffamily\fontsize{10.000000}{12.000000}\selectfont \(\displaystyle {200}\)}%
\end{pgfscope}%
\begin{pgfscope}%
\pgfsetbuttcap%
\pgfsetroundjoin%
\definecolor{currentfill}{rgb}{0.000000,0.000000,0.000000}%
\pgfsetfillcolor{currentfill}%
\pgfsetlinewidth{0.803000pt}%
\definecolor{currentstroke}{rgb}{0.000000,0.000000,0.000000}%
\pgfsetstrokecolor{currentstroke}%
\pgfsetdash{}{0pt}%
\pgfsys@defobject{currentmarker}{\pgfqpoint{-0.048611in}{0.000000in}}{\pgfqpoint{0.000000in}{0.000000in}}{%
\pgfpathmoveto{\pgfqpoint{0.000000in}{0.000000in}}%
\pgfpathlineto{\pgfqpoint{-0.048611in}{0.000000in}}%
\pgfusepath{stroke,fill}%
}%
\begin{pgfscope}%
\pgfsys@transformshift{0.648703in}{1.934150in}%
\pgfsys@useobject{currentmarker}{}%
\end{pgfscope}%
\end{pgfscope}%
\begin{pgfscope}%
\definecolor{textcolor}{rgb}{0.000000,0.000000,0.000000}%
\pgfsetstrokecolor{textcolor}%
\pgfsetfillcolor{textcolor}%
\pgftext[x=0.343147in, y=1.885955in, left, base]{\color{textcolor}\sffamily\fontsize{10.000000}{12.000000}\selectfont \(\displaystyle {300}\)}%
\end{pgfscope}%
\begin{pgfscope}%
\pgfsetbuttcap%
\pgfsetroundjoin%
\definecolor{currentfill}{rgb}{0.000000,0.000000,0.000000}%
\pgfsetfillcolor{currentfill}%
\pgfsetlinewidth{0.803000pt}%
\definecolor{currentstroke}{rgb}{0.000000,0.000000,0.000000}%
\pgfsetstrokecolor{currentstroke}%
\pgfsetdash{}{0pt}%
\pgfsys@defobject{currentmarker}{\pgfqpoint{-0.048611in}{0.000000in}}{\pgfqpoint{0.000000in}{0.000000in}}{%
\pgfpathmoveto{\pgfqpoint{0.000000in}{0.000000in}}%
\pgfpathlineto{\pgfqpoint{-0.048611in}{0.000000in}}%
\pgfusepath{stroke,fill}%
}%
\begin{pgfscope}%
\pgfsys@transformshift{0.648703in}{2.348935in}%
\pgfsys@useobject{currentmarker}{}%
\end{pgfscope}%
\end{pgfscope}%
\begin{pgfscope}%
\definecolor{textcolor}{rgb}{0.000000,0.000000,0.000000}%
\pgfsetstrokecolor{textcolor}%
\pgfsetfillcolor{textcolor}%
\pgftext[x=0.343147in, y=2.300740in, left, base]{\color{textcolor}\sffamily\fontsize{10.000000}{12.000000}\selectfont \(\displaystyle {400}\)}%
\end{pgfscope}%
\begin{pgfscope}%
\pgfsetbuttcap%
\pgfsetroundjoin%
\definecolor{currentfill}{rgb}{0.000000,0.000000,0.000000}%
\pgfsetfillcolor{currentfill}%
\pgfsetlinewidth{0.803000pt}%
\definecolor{currentstroke}{rgb}{0.000000,0.000000,0.000000}%
\pgfsetstrokecolor{currentstroke}%
\pgfsetdash{}{0pt}%
\pgfsys@defobject{currentmarker}{\pgfqpoint{-0.048611in}{0.000000in}}{\pgfqpoint{0.000000in}{0.000000in}}{%
\pgfpathmoveto{\pgfqpoint{0.000000in}{0.000000in}}%
\pgfpathlineto{\pgfqpoint{-0.048611in}{0.000000in}}%
\pgfusepath{stroke,fill}%
}%
\begin{pgfscope}%
\pgfsys@transformshift{0.648703in}{2.763719in}%
\pgfsys@useobject{currentmarker}{}%
\end{pgfscope}%
\end{pgfscope}%
\begin{pgfscope}%
\definecolor{textcolor}{rgb}{0.000000,0.000000,0.000000}%
\pgfsetstrokecolor{textcolor}%
\pgfsetfillcolor{textcolor}%
\pgftext[x=0.343147in, y=2.715525in, left, base]{\color{textcolor}\sffamily\fontsize{10.000000}{12.000000}\selectfont \(\displaystyle {500}\)}%
\end{pgfscope}%
\begin{pgfscope}%
\pgfsetbuttcap%
\pgfsetroundjoin%
\definecolor{currentfill}{rgb}{0.000000,0.000000,0.000000}%
\pgfsetfillcolor{currentfill}%
\pgfsetlinewidth{0.803000pt}%
\definecolor{currentstroke}{rgb}{0.000000,0.000000,0.000000}%
\pgfsetstrokecolor{currentstroke}%
\pgfsetdash{}{0pt}%
\pgfsys@defobject{currentmarker}{\pgfqpoint{-0.048611in}{0.000000in}}{\pgfqpoint{0.000000in}{0.000000in}}{%
\pgfpathmoveto{\pgfqpoint{0.000000in}{0.000000in}}%
\pgfpathlineto{\pgfqpoint{-0.048611in}{0.000000in}}%
\pgfusepath{stroke,fill}%
}%
\begin{pgfscope}%
\pgfsys@transformshift{0.648703in}{3.178504in}%
\pgfsys@useobject{currentmarker}{}%
\end{pgfscope}%
\end{pgfscope}%
\begin{pgfscope}%
\definecolor{textcolor}{rgb}{0.000000,0.000000,0.000000}%
\pgfsetstrokecolor{textcolor}%
\pgfsetfillcolor{textcolor}%
\pgftext[x=0.343147in, y=3.130310in, left, base]{\color{textcolor}\sffamily\fontsize{10.000000}{12.000000}\selectfont \(\displaystyle {600}\)}%
\end{pgfscope}%
\begin{pgfscope}%
\pgfsetbuttcap%
\pgfsetroundjoin%
\definecolor{currentfill}{rgb}{0.000000,0.000000,0.000000}%
\pgfsetfillcolor{currentfill}%
\pgfsetlinewidth{0.803000pt}%
\definecolor{currentstroke}{rgb}{0.000000,0.000000,0.000000}%
\pgfsetstrokecolor{currentstroke}%
\pgfsetdash{}{0pt}%
\pgfsys@defobject{currentmarker}{\pgfqpoint{-0.048611in}{0.000000in}}{\pgfqpoint{0.000000in}{0.000000in}}{%
\pgfpathmoveto{\pgfqpoint{0.000000in}{0.000000in}}%
\pgfpathlineto{\pgfqpoint{-0.048611in}{0.000000in}}%
\pgfusepath{stroke,fill}%
}%
\begin{pgfscope}%
\pgfsys@transformshift{0.648703in}{3.593289in}%
\pgfsys@useobject{currentmarker}{}%
\end{pgfscope}%
\end{pgfscope}%
\begin{pgfscope}%
\definecolor{textcolor}{rgb}{0.000000,0.000000,0.000000}%
\pgfsetstrokecolor{textcolor}%
\pgfsetfillcolor{textcolor}%
\pgftext[x=0.343147in, y=3.545094in, left, base]{\color{textcolor}\sffamily\fontsize{10.000000}{12.000000}\selectfont \(\displaystyle {700}\)}%
\end{pgfscope}%
\begin{pgfscope}%
\definecolor{textcolor}{rgb}{0.000000,0.000000,0.000000}%
\pgfsetstrokecolor{textcolor}%
\pgfsetfillcolor{textcolor}%
\pgftext[x=0.287592in,y=2.100064in,,bottom,rotate=90.000000]{\color{textcolor}\sffamily\fontsize{10.000000}{12.000000}\selectfont Data Flow Time (s)}%
\end{pgfscope}%
\begin{pgfscope}%
\pgfsetrectcap%
\pgfsetmiterjoin%
\pgfsetlinewidth{0.803000pt}%
\definecolor{currentstroke}{rgb}{0.000000,0.000000,0.000000}%
\pgfsetstrokecolor{currentstroke}%
\pgfsetdash{}{0pt}%
\pgfpathmoveto{\pgfqpoint{0.648703in}{0.548769in}}%
\pgfpathlineto{\pgfqpoint{0.648703in}{3.651359in}}%
\pgfusepath{stroke}%
\end{pgfscope}%
\begin{pgfscope}%
\pgfsetrectcap%
\pgfsetmiterjoin%
\pgfsetlinewidth{0.803000pt}%
\definecolor{currentstroke}{rgb}{0.000000,0.000000,0.000000}%
\pgfsetstrokecolor{currentstroke}%
\pgfsetdash{}{0pt}%
\pgfpathmoveto{\pgfqpoint{5.850000in}{0.548769in}}%
\pgfpathlineto{\pgfqpoint{5.850000in}{3.651359in}}%
\pgfusepath{stroke}%
\end{pgfscope}%
\begin{pgfscope}%
\pgfsetrectcap%
\pgfsetmiterjoin%
\pgfsetlinewidth{0.803000pt}%
\definecolor{currentstroke}{rgb}{0.000000,0.000000,0.000000}%
\pgfsetstrokecolor{currentstroke}%
\pgfsetdash{}{0pt}%
\pgfpathmoveto{\pgfqpoint{0.648703in}{0.548769in}}%
\pgfpathlineto{\pgfqpoint{5.850000in}{0.548769in}}%
\pgfusepath{stroke}%
\end{pgfscope}%
\begin{pgfscope}%
\pgfsetrectcap%
\pgfsetmiterjoin%
\pgfsetlinewidth{0.803000pt}%
\definecolor{currentstroke}{rgb}{0.000000,0.000000,0.000000}%
\pgfsetstrokecolor{currentstroke}%
\pgfsetdash{}{0pt}%
\pgfpathmoveto{\pgfqpoint{0.648703in}{3.651359in}}%
\pgfpathlineto{\pgfqpoint{5.850000in}{3.651359in}}%
\pgfusepath{stroke}%
\end{pgfscope}%
\begin{pgfscope}%
\definecolor{textcolor}{rgb}{0.000000,0.000000,0.000000}%
\pgfsetstrokecolor{textcolor}%
\pgfsetfillcolor{textcolor}%
\pgftext[x=3.249352in,y=3.734692in,,base]{\color{textcolor}\sffamily\fontsize{12.000000}{14.400000}\selectfont Forward}%
\end{pgfscope}%
\begin{pgfscope}%
\pgfsetbuttcap%
\pgfsetmiterjoin%
\definecolor{currentfill}{rgb}{1.000000,1.000000,1.000000}%
\pgfsetfillcolor{currentfill}%
\pgfsetfillopacity{0.800000}%
\pgfsetlinewidth{1.003750pt}%
\definecolor{currentstroke}{rgb}{0.800000,0.800000,0.800000}%
\pgfsetstrokecolor{currentstroke}%
\pgfsetstrokeopacity{0.800000}%
\pgfsetdash{}{0pt}%
\pgfpathmoveto{\pgfqpoint{4.300417in}{0.618213in}}%
\pgfpathlineto{\pgfqpoint{5.752778in}{0.618213in}}%
\pgfpathquadraticcurveto{\pgfqpoint{5.780556in}{0.618213in}}{\pgfqpoint{5.780556in}{0.645991in}}%
\pgfpathlineto{\pgfqpoint{5.780556in}{1.214463in}}%
\pgfpathquadraticcurveto{\pgfqpoint{5.780556in}{1.242241in}}{\pgfqpoint{5.752778in}{1.242241in}}%
\pgfpathlineto{\pgfqpoint{4.300417in}{1.242241in}}%
\pgfpathquadraticcurveto{\pgfqpoint{4.272639in}{1.242241in}}{\pgfqpoint{4.272639in}{1.214463in}}%
\pgfpathlineto{\pgfqpoint{4.272639in}{0.645991in}}%
\pgfpathquadraticcurveto{\pgfqpoint{4.272639in}{0.618213in}}{\pgfqpoint{4.300417in}{0.618213in}}%
\pgfpathclose%
\pgfusepath{stroke,fill}%
\end{pgfscope}%
\begin{pgfscope}%
\pgfsetbuttcap%
\pgfsetroundjoin%
\definecolor{currentfill}{rgb}{0.121569,0.466667,0.705882}%
\pgfsetfillcolor{currentfill}%
\pgfsetlinewidth{1.003750pt}%
\definecolor{currentstroke}{rgb}{0.121569,0.466667,0.705882}%
\pgfsetstrokecolor{currentstroke}%
\pgfsetdash{}{0pt}%
\pgfsys@defobject{currentmarker}{\pgfqpoint{-0.034722in}{-0.034722in}}{\pgfqpoint{0.034722in}{0.034722in}}{%
\pgfpathmoveto{\pgfqpoint{0.000000in}{-0.034722in}}%
\pgfpathcurveto{\pgfqpoint{0.009208in}{-0.034722in}}{\pgfqpoint{0.018041in}{-0.031064in}}{\pgfqpoint{0.024552in}{-0.024552in}}%
\pgfpathcurveto{\pgfqpoint{0.031064in}{-0.018041in}}{\pgfqpoint{0.034722in}{-0.009208in}}{\pgfqpoint{0.034722in}{0.000000in}}%
\pgfpathcurveto{\pgfqpoint{0.034722in}{0.009208in}}{\pgfqpoint{0.031064in}{0.018041in}}{\pgfqpoint{0.024552in}{0.024552in}}%
\pgfpathcurveto{\pgfqpoint{0.018041in}{0.031064in}}{\pgfqpoint{0.009208in}{0.034722in}}{\pgfqpoint{0.000000in}{0.034722in}}%
\pgfpathcurveto{\pgfqpoint{-0.009208in}{0.034722in}}{\pgfqpoint{-0.018041in}{0.031064in}}{\pgfqpoint{-0.024552in}{0.024552in}}%
\pgfpathcurveto{\pgfqpoint{-0.031064in}{0.018041in}}{\pgfqpoint{-0.034722in}{0.009208in}}{\pgfqpoint{-0.034722in}{0.000000in}}%
\pgfpathcurveto{\pgfqpoint{-0.034722in}{-0.009208in}}{\pgfqpoint{-0.031064in}{-0.018041in}}{\pgfqpoint{-0.024552in}{-0.024552in}}%
\pgfpathcurveto{\pgfqpoint{-0.018041in}{-0.031064in}}{\pgfqpoint{-0.009208in}{-0.034722in}}{\pgfqpoint{0.000000in}{-0.034722in}}%
\pgfpathclose%
\pgfusepath{stroke,fill}%
}%
\begin{pgfscope}%
\pgfsys@transformshift{4.467083in}{1.138074in}%
\pgfsys@useobject{currentmarker}{}%
\end{pgfscope}%
\end{pgfscope}%
\begin{pgfscope}%
\definecolor{textcolor}{rgb}{0.000000,0.000000,0.000000}%
\pgfsetstrokecolor{textcolor}%
\pgfsetfillcolor{textcolor}%
\pgftext[x=4.717083in,y=1.089463in,left,base]{\color{textcolor}\sffamily\fontsize{10.000000}{12.000000}\selectfont No Timeout}%
\end{pgfscope}%
\begin{pgfscope}%
\pgfsetbuttcap%
\pgfsetroundjoin%
\definecolor{currentfill}{rgb}{1.000000,0.498039,0.054902}%
\pgfsetfillcolor{currentfill}%
\pgfsetlinewidth{1.003750pt}%
\definecolor{currentstroke}{rgb}{1.000000,0.498039,0.054902}%
\pgfsetstrokecolor{currentstroke}%
\pgfsetdash{}{0pt}%
\pgfsys@defobject{currentmarker}{\pgfqpoint{-0.034722in}{-0.034722in}}{\pgfqpoint{0.034722in}{0.034722in}}{%
\pgfpathmoveto{\pgfqpoint{0.000000in}{-0.034722in}}%
\pgfpathcurveto{\pgfqpoint{0.009208in}{-0.034722in}}{\pgfqpoint{0.018041in}{-0.031064in}}{\pgfqpoint{0.024552in}{-0.024552in}}%
\pgfpathcurveto{\pgfqpoint{0.031064in}{-0.018041in}}{\pgfqpoint{0.034722in}{-0.009208in}}{\pgfqpoint{0.034722in}{0.000000in}}%
\pgfpathcurveto{\pgfqpoint{0.034722in}{0.009208in}}{\pgfqpoint{0.031064in}{0.018041in}}{\pgfqpoint{0.024552in}{0.024552in}}%
\pgfpathcurveto{\pgfqpoint{0.018041in}{0.031064in}}{\pgfqpoint{0.009208in}{0.034722in}}{\pgfqpoint{0.000000in}{0.034722in}}%
\pgfpathcurveto{\pgfqpoint{-0.009208in}{0.034722in}}{\pgfqpoint{-0.018041in}{0.031064in}}{\pgfqpoint{-0.024552in}{0.024552in}}%
\pgfpathcurveto{\pgfqpoint{-0.031064in}{0.018041in}}{\pgfqpoint{-0.034722in}{0.009208in}}{\pgfqpoint{-0.034722in}{0.000000in}}%
\pgfpathcurveto{\pgfqpoint{-0.034722in}{-0.009208in}}{\pgfqpoint{-0.031064in}{-0.018041in}}{\pgfqpoint{-0.024552in}{-0.024552in}}%
\pgfpathcurveto{\pgfqpoint{-0.018041in}{-0.031064in}}{\pgfqpoint{-0.009208in}{-0.034722in}}{\pgfqpoint{0.000000in}{-0.034722in}}%
\pgfpathclose%
\pgfusepath{stroke,fill}%
}%
\begin{pgfscope}%
\pgfsys@transformshift{4.467083in}{0.944463in}%
\pgfsys@useobject{currentmarker}{}%
\end{pgfscope}%
\end{pgfscope}%
\begin{pgfscope}%
\definecolor{textcolor}{rgb}{0.000000,0.000000,0.000000}%
\pgfsetstrokecolor{textcolor}%
\pgfsetfillcolor{textcolor}%
\pgftext[x=4.717083in,y=0.895852in,left,base]{\color{textcolor}\sffamily\fontsize{10.000000}{12.000000}\selectfont Time Timeout}%
\end{pgfscope}%
\begin{pgfscope}%
\pgfsetbuttcap%
\pgfsetroundjoin%
\definecolor{currentfill}{rgb}{0.839216,0.152941,0.156863}%
\pgfsetfillcolor{currentfill}%
\pgfsetlinewidth{1.003750pt}%
\definecolor{currentstroke}{rgb}{0.839216,0.152941,0.156863}%
\pgfsetstrokecolor{currentstroke}%
\pgfsetdash{}{0pt}%
\pgfsys@defobject{currentmarker}{\pgfqpoint{-0.034722in}{-0.034722in}}{\pgfqpoint{0.034722in}{0.034722in}}{%
\pgfpathmoveto{\pgfqpoint{0.000000in}{-0.034722in}}%
\pgfpathcurveto{\pgfqpoint{0.009208in}{-0.034722in}}{\pgfqpoint{0.018041in}{-0.031064in}}{\pgfqpoint{0.024552in}{-0.024552in}}%
\pgfpathcurveto{\pgfqpoint{0.031064in}{-0.018041in}}{\pgfqpoint{0.034722in}{-0.009208in}}{\pgfqpoint{0.034722in}{0.000000in}}%
\pgfpathcurveto{\pgfqpoint{0.034722in}{0.009208in}}{\pgfqpoint{0.031064in}{0.018041in}}{\pgfqpoint{0.024552in}{0.024552in}}%
\pgfpathcurveto{\pgfqpoint{0.018041in}{0.031064in}}{\pgfqpoint{0.009208in}{0.034722in}}{\pgfqpoint{0.000000in}{0.034722in}}%
\pgfpathcurveto{\pgfqpoint{-0.009208in}{0.034722in}}{\pgfqpoint{-0.018041in}{0.031064in}}{\pgfqpoint{-0.024552in}{0.024552in}}%
\pgfpathcurveto{\pgfqpoint{-0.031064in}{0.018041in}}{\pgfqpoint{-0.034722in}{0.009208in}}{\pgfqpoint{-0.034722in}{0.000000in}}%
\pgfpathcurveto{\pgfqpoint{-0.034722in}{-0.009208in}}{\pgfqpoint{-0.031064in}{-0.018041in}}{\pgfqpoint{-0.024552in}{-0.024552in}}%
\pgfpathcurveto{\pgfqpoint{-0.018041in}{-0.031064in}}{\pgfqpoint{-0.009208in}{-0.034722in}}{\pgfqpoint{0.000000in}{-0.034722in}}%
\pgfpathclose%
\pgfusepath{stroke,fill}%
}%
\begin{pgfscope}%
\pgfsys@transformshift{4.467083in}{0.750852in}%
\pgfsys@useobject{currentmarker}{}%
\end{pgfscope}%
\end{pgfscope}%
\begin{pgfscope}%
\definecolor{textcolor}{rgb}{0.000000,0.000000,0.000000}%
\pgfsetstrokecolor{textcolor}%
\pgfsetfillcolor{textcolor}%
\pgftext[x=4.717083in,y=0.702241in,left,base]{\color{textcolor}\sffamily\fontsize{10.000000}{12.000000}\selectfont Memory Timeout}%
\end{pgfscope}%
\end{pgfpicture}%
\makeatother%
\endgroup%

                }
            \end{subfigure}
            \qquad
            \begin{subfigure}[]{0.45\textwidth}
                \centering
                \resizebox{\columnwidth}{!}{
                    %% Creator: Matplotlib, PGF backend
%%
%% To include the figure in your LaTeX document, write
%%   \input{<filename>.pgf}
%%
%% Make sure the required packages are loaded in your preamble
%%   \usepackage{pgf}
%%
%% and, on pdftex
%%   \usepackage[utf8]{inputenc}\DeclareUnicodeCharacter{2212}{-}
%%
%% or, on luatex and xetex
%%   \usepackage{unicode-math}
%%
%% Figures using additional raster images can only be included by \input if
%% they are in the same directory as the main LaTeX file. For loading figures
%% from other directories you can use the `import` package
%%   \usepackage{import}
%%
%% and then include the figures with
%%   \import{<path to file>}{<filename>.pgf}
%%
%% Matplotlib used the following preamble
%%   \usepackage{amsmath}
%%   \usepackage{fontspec}
%%
\begingroup%
\makeatletter%
\begin{pgfpicture}%
\pgfpathrectangle{\pgfpointorigin}{\pgfqpoint{6.000000in}{4.000000in}}%
\pgfusepath{use as bounding box, clip}%
\begin{pgfscope}%
\pgfsetbuttcap%
\pgfsetmiterjoin%
\definecolor{currentfill}{rgb}{1.000000,1.000000,1.000000}%
\pgfsetfillcolor{currentfill}%
\pgfsetlinewidth{0.000000pt}%
\definecolor{currentstroke}{rgb}{1.000000,1.000000,1.000000}%
\pgfsetstrokecolor{currentstroke}%
\pgfsetdash{}{0pt}%
\pgfpathmoveto{\pgfqpoint{0.000000in}{0.000000in}}%
\pgfpathlineto{\pgfqpoint{6.000000in}{0.000000in}}%
\pgfpathlineto{\pgfqpoint{6.000000in}{4.000000in}}%
\pgfpathlineto{\pgfqpoint{0.000000in}{4.000000in}}%
\pgfpathclose%
\pgfusepath{fill}%
\end{pgfscope}%
\begin{pgfscope}%
\pgfsetbuttcap%
\pgfsetmiterjoin%
\definecolor{currentfill}{rgb}{1.000000,1.000000,1.000000}%
\pgfsetfillcolor{currentfill}%
\pgfsetlinewidth{0.000000pt}%
\definecolor{currentstroke}{rgb}{0.000000,0.000000,0.000000}%
\pgfsetstrokecolor{currentstroke}%
\pgfsetstrokeopacity{0.000000}%
\pgfsetdash{}{0pt}%
\pgfpathmoveto{\pgfqpoint{0.648703in}{0.548769in}}%
\pgfpathlineto{\pgfqpoint{5.850000in}{0.548769in}}%
\pgfpathlineto{\pgfqpoint{5.850000in}{3.651359in}}%
\pgfpathlineto{\pgfqpoint{0.648703in}{3.651359in}}%
\pgfpathclose%
\pgfusepath{fill}%
\end{pgfscope}%
\begin{pgfscope}%
\pgfpathrectangle{\pgfqpoint{0.648703in}{0.548769in}}{\pgfqpoint{5.201297in}{3.102590in}}%
\pgfusepath{clip}%
\pgfsetbuttcap%
\pgfsetroundjoin%
\definecolor{currentfill}{rgb}{0.121569,0.466667,0.705882}%
\pgfsetfillcolor{currentfill}%
\pgfsetlinewidth{1.003750pt}%
\definecolor{currentstroke}{rgb}{0.121569,0.466667,0.705882}%
\pgfsetstrokecolor{currentstroke}%
\pgfsetdash{}{0pt}%
\pgfpathmoveto{\pgfqpoint{1.113780in}{0.673501in}}%
\pgfpathcurveto{\pgfqpoint{1.124830in}{0.673501in}}{\pgfqpoint{1.135429in}{0.677891in}}{\pgfqpoint{1.143243in}{0.685705in}}%
\pgfpathcurveto{\pgfqpoint{1.151056in}{0.693519in}}{\pgfqpoint{1.155447in}{0.704118in}}{\pgfqpoint{1.155447in}{0.715168in}}%
\pgfpathcurveto{\pgfqpoint{1.155447in}{0.726218in}}{\pgfqpoint{1.151056in}{0.736817in}}{\pgfqpoint{1.143243in}{0.744631in}}%
\pgfpathcurveto{\pgfqpoint{1.135429in}{0.752444in}}{\pgfqpoint{1.124830in}{0.756834in}}{\pgfqpoint{1.113780in}{0.756834in}}%
\pgfpathcurveto{\pgfqpoint{1.102730in}{0.756834in}}{\pgfqpoint{1.092131in}{0.752444in}}{\pgfqpoint{1.084317in}{0.744631in}}%
\pgfpathcurveto{\pgfqpoint{1.076504in}{0.736817in}}{\pgfqpoint{1.072113in}{0.726218in}}{\pgfqpoint{1.072113in}{0.715168in}}%
\pgfpathcurveto{\pgfqpoint{1.072113in}{0.704118in}}{\pgfqpoint{1.076504in}{0.693519in}}{\pgfqpoint{1.084317in}{0.685705in}}%
\pgfpathcurveto{\pgfqpoint{1.092131in}{0.677891in}}{\pgfqpoint{1.102730in}{0.673501in}}{\pgfqpoint{1.113780in}{0.673501in}}%
\pgfpathclose%
\pgfusepath{stroke,fill}%
\end{pgfscope}%
\begin{pgfscope}%
\pgfpathrectangle{\pgfqpoint{0.648703in}{0.548769in}}{\pgfqpoint{5.201297in}{3.102590in}}%
\pgfusepath{clip}%
\pgfsetbuttcap%
\pgfsetroundjoin%
\definecolor{currentfill}{rgb}{0.121569,0.466667,0.705882}%
\pgfsetfillcolor{currentfill}%
\pgfsetlinewidth{1.003750pt}%
\definecolor{currentstroke}{rgb}{0.121569,0.466667,0.705882}%
\pgfsetstrokecolor{currentstroke}%
\pgfsetdash{}{0pt}%
\pgfpathmoveto{\pgfqpoint{2.819926in}{3.176886in}}%
\pgfpathcurveto{\pgfqpoint{2.830976in}{3.176886in}}{\pgfqpoint{2.841575in}{3.181276in}}{\pgfqpoint{2.849389in}{3.189089in}}%
\pgfpathcurveto{\pgfqpoint{2.857202in}{3.196903in}}{\pgfqpoint{2.861593in}{3.207502in}}{\pgfqpoint{2.861593in}{3.218552in}}%
\pgfpathcurveto{\pgfqpoint{2.861593in}{3.229602in}}{\pgfqpoint{2.857202in}{3.240201in}}{\pgfqpoint{2.849389in}{3.248015in}}%
\pgfpathcurveto{\pgfqpoint{2.841575in}{3.255829in}}{\pgfqpoint{2.830976in}{3.260219in}}{\pgfqpoint{2.819926in}{3.260219in}}%
\pgfpathcurveto{\pgfqpoint{2.808876in}{3.260219in}}{\pgfqpoint{2.798277in}{3.255829in}}{\pgfqpoint{2.790463in}{3.248015in}}%
\pgfpathcurveto{\pgfqpoint{2.782649in}{3.240201in}}{\pgfqpoint{2.778259in}{3.229602in}}{\pgfqpoint{2.778259in}{3.218552in}}%
\pgfpathcurveto{\pgfqpoint{2.778259in}{3.207502in}}{\pgfqpoint{2.782649in}{3.196903in}}{\pgfqpoint{2.790463in}{3.189089in}}%
\pgfpathcurveto{\pgfqpoint{2.798277in}{3.181276in}}{\pgfqpoint{2.808876in}{3.176886in}}{\pgfqpoint{2.819926in}{3.176886in}}%
\pgfpathclose%
\pgfusepath{stroke,fill}%
\end{pgfscope}%
\begin{pgfscope}%
\pgfpathrectangle{\pgfqpoint{0.648703in}{0.548769in}}{\pgfqpoint{5.201297in}{3.102590in}}%
\pgfusepath{clip}%
\pgfsetbuttcap%
\pgfsetroundjoin%
\definecolor{currentfill}{rgb}{1.000000,0.498039,0.054902}%
\pgfsetfillcolor{currentfill}%
\pgfsetlinewidth{1.003750pt}%
\definecolor{currentstroke}{rgb}{1.000000,0.498039,0.054902}%
\pgfsetstrokecolor{currentstroke}%
\pgfsetdash{}{0pt}%
\pgfpathmoveto{\pgfqpoint{1.266468in}{3.198029in}}%
\pgfpathcurveto{\pgfqpoint{1.277518in}{3.198029in}}{\pgfqpoint{1.288117in}{3.202419in}}{\pgfqpoint{1.295931in}{3.210233in}}%
\pgfpathcurveto{\pgfqpoint{1.303744in}{3.218046in}}{\pgfqpoint{1.308135in}{3.228646in}}{\pgfqpoint{1.308135in}{3.239696in}}%
\pgfpathcurveto{\pgfqpoint{1.308135in}{3.250746in}}{\pgfqpoint{1.303744in}{3.261345in}}{\pgfqpoint{1.295931in}{3.269158in}}%
\pgfpathcurveto{\pgfqpoint{1.288117in}{3.276972in}}{\pgfqpoint{1.277518in}{3.281362in}}{\pgfqpoint{1.266468in}{3.281362in}}%
\pgfpathcurveto{\pgfqpoint{1.255418in}{3.281362in}}{\pgfqpoint{1.244819in}{3.276972in}}{\pgfqpoint{1.237005in}{3.269158in}}%
\pgfpathcurveto{\pgfqpoint{1.229192in}{3.261345in}}{\pgfqpoint{1.224801in}{3.250746in}}{\pgfqpoint{1.224801in}{3.239696in}}%
\pgfpathcurveto{\pgfqpoint{1.224801in}{3.228646in}}{\pgfqpoint{1.229192in}{3.218046in}}{\pgfqpoint{1.237005in}{3.210233in}}%
\pgfpathcurveto{\pgfqpoint{1.244819in}{3.202419in}}{\pgfqpoint{1.255418in}{3.198029in}}{\pgfqpoint{1.266468in}{3.198029in}}%
\pgfpathclose%
\pgfusepath{stroke,fill}%
\end{pgfscope}%
\begin{pgfscope}%
\pgfpathrectangle{\pgfqpoint{0.648703in}{0.548769in}}{\pgfqpoint{5.201297in}{3.102590in}}%
\pgfusepath{clip}%
\pgfsetbuttcap%
\pgfsetroundjoin%
\definecolor{currentfill}{rgb}{1.000000,0.498039,0.054902}%
\pgfsetfillcolor{currentfill}%
\pgfsetlinewidth{1.003750pt}%
\definecolor{currentstroke}{rgb}{1.000000,0.498039,0.054902}%
\pgfsetstrokecolor{currentstroke}%
\pgfsetdash{}{0pt}%
\pgfpathmoveto{\pgfqpoint{1.523022in}{3.185343in}}%
\pgfpathcurveto{\pgfqpoint{1.534072in}{3.185343in}}{\pgfqpoint{1.544671in}{3.189733in}}{\pgfqpoint{1.552485in}{3.197547in}}%
\pgfpathcurveto{\pgfqpoint{1.560298in}{3.205360in}}{\pgfqpoint{1.564689in}{3.215959in}}{\pgfqpoint{1.564689in}{3.227010in}}%
\pgfpathcurveto{\pgfqpoint{1.564689in}{3.238060in}}{\pgfqpoint{1.560298in}{3.248659in}}{\pgfqpoint{1.552485in}{3.256472in}}%
\pgfpathcurveto{\pgfqpoint{1.544671in}{3.264286in}}{\pgfqpoint{1.534072in}{3.268676in}}{\pgfqpoint{1.523022in}{3.268676in}}%
\pgfpathcurveto{\pgfqpoint{1.511972in}{3.268676in}}{\pgfqpoint{1.501373in}{3.264286in}}{\pgfqpoint{1.493559in}{3.256472in}}%
\pgfpathcurveto{\pgfqpoint{1.485746in}{3.248659in}}{\pgfqpoint{1.481355in}{3.238060in}}{\pgfqpoint{1.481355in}{3.227010in}}%
\pgfpathcurveto{\pgfqpoint{1.481355in}{3.215959in}}{\pgfqpoint{1.485746in}{3.205360in}}{\pgfqpoint{1.493559in}{3.197547in}}%
\pgfpathcurveto{\pgfqpoint{1.501373in}{3.189733in}}{\pgfqpoint{1.511972in}{3.185343in}}{\pgfqpoint{1.523022in}{3.185343in}}%
\pgfpathclose%
\pgfusepath{stroke,fill}%
\end{pgfscope}%
\begin{pgfscope}%
\pgfpathrectangle{\pgfqpoint{0.648703in}{0.548769in}}{\pgfqpoint{5.201297in}{3.102590in}}%
\pgfusepath{clip}%
\pgfsetbuttcap%
\pgfsetroundjoin%
\definecolor{currentfill}{rgb}{0.121569,0.466667,0.705882}%
\pgfsetfillcolor{currentfill}%
\pgfsetlinewidth{1.003750pt}%
\definecolor{currentstroke}{rgb}{0.121569,0.466667,0.705882}%
\pgfsetstrokecolor{currentstroke}%
\pgfsetdash{}{0pt}%
\pgfpathmoveto{\pgfqpoint{1.355123in}{0.652358in}}%
\pgfpathcurveto{\pgfqpoint{1.366173in}{0.652358in}}{\pgfqpoint{1.376773in}{0.656748in}}{\pgfqpoint{1.384586in}{0.664562in}}%
\pgfpathcurveto{\pgfqpoint{1.392400in}{0.672375in}}{\pgfqpoint{1.396790in}{0.682974in}}{\pgfqpoint{1.396790in}{0.694024in}}%
\pgfpathcurveto{\pgfqpoint{1.396790in}{0.705074in}}{\pgfqpoint{1.392400in}{0.715673in}}{\pgfqpoint{1.384586in}{0.723487in}}%
\pgfpathcurveto{\pgfqpoint{1.376773in}{0.731301in}}{\pgfqpoint{1.366173in}{0.735691in}}{\pgfqpoint{1.355123in}{0.735691in}}%
\pgfpathcurveto{\pgfqpoint{1.344073in}{0.735691in}}{\pgfqpoint{1.333474in}{0.731301in}}{\pgfqpoint{1.325661in}{0.723487in}}%
\pgfpathcurveto{\pgfqpoint{1.317847in}{0.715673in}}{\pgfqpoint{1.313457in}{0.705074in}}{\pgfqpoint{1.313457in}{0.694024in}}%
\pgfpathcurveto{\pgfqpoint{1.313457in}{0.682974in}}{\pgfqpoint{1.317847in}{0.672375in}}{\pgfqpoint{1.325661in}{0.664562in}}%
\pgfpathcurveto{\pgfqpoint{1.333474in}{0.656748in}}{\pgfqpoint{1.344073in}{0.652358in}}{\pgfqpoint{1.355123in}{0.652358in}}%
\pgfpathclose%
\pgfusepath{stroke,fill}%
\end{pgfscope}%
\begin{pgfscope}%
\pgfpathrectangle{\pgfqpoint{0.648703in}{0.548769in}}{\pgfqpoint{5.201297in}{3.102590in}}%
\pgfusepath{clip}%
\pgfsetbuttcap%
\pgfsetroundjoin%
\definecolor{currentfill}{rgb}{0.121569,0.466667,0.705882}%
\pgfsetfillcolor{currentfill}%
\pgfsetlinewidth{1.003750pt}%
\definecolor{currentstroke}{rgb}{0.121569,0.466667,0.705882}%
\pgfsetstrokecolor{currentstroke}%
\pgfsetdash{}{0pt}%
\pgfpathmoveto{\pgfqpoint{1.676996in}{3.181114in}}%
\pgfpathcurveto{\pgfqpoint{1.688046in}{3.181114in}}{\pgfqpoint{1.698645in}{3.185504in}}{\pgfqpoint{1.706459in}{3.193318in}}%
\pgfpathcurveto{\pgfqpoint{1.714273in}{3.201132in}}{\pgfqpoint{1.718663in}{3.211731in}}{\pgfqpoint{1.718663in}{3.222781in}}%
\pgfpathcurveto{\pgfqpoint{1.718663in}{3.233831in}}{\pgfqpoint{1.714273in}{3.244430in}}{\pgfqpoint{1.706459in}{3.252244in}}%
\pgfpathcurveto{\pgfqpoint{1.698645in}{3.260057in}}{\pgfqpoint{1.688046in}{3.264448in}}{\pgfqpoint{1.676996in}{3.264448in}}%
\pgfpathcurveto{\pgfqpoint{1.665946in}{3.264448in}}{\pgfqpoint{1.655347in}{3.260057in}}{\pgfqpoint{1.647534in}{3.252244in}}%
\pgfpathcurveto{\pgfqpoint{1.639720in}{3.244430in}}{\pgfqpoint{1.635330in}{3.233831in}}{\pgfqpoint{1.635330in}{3.222781in}}%
\pgfpathcurveto{\pgfqpoint{1.635330in}{3.211731in}}{\pgfqpoint{1.639720in}{3.201132in}}{\pgfqpoint{1.647534in}{3.193318in}}%
\pgfpathcurveto{\pgfqpoint{1.655347in}{3.185504in}}{\pgfqpoint{1.665946in}{3.181114in}}{\pgfqpoint{1.676996in}{3.181114in}}%
\pgfpathclose%
\pgfusepath{stroke,fill}%
\end{pgfscope}%
\begin{pgfscope}%
\pgfpathrectangle{\pgfqpoint{0.648703in}{0.548769in}}{\pgfqpoint{5.201297in}{3.102590in}}%
\pgfusepath{clip}%
\pgfsetbuttcap%
\pgfsetroundjoin%
\definecolor{currentfill}{rgb}{1.000000,0.498039,0.054902}%
\pgfsetfillcolor{currentfill}%
\pgfsetlinewidth{1.003750pt}%
\definecolor{currentstroke}{rgb}{1.000000,0.498039,0.054902}%
\pgfsetstrokecolor{currentstroke}%
\pgfsetdash{}{0pt}%
\pgfpathmoveto{\pgfqpoint{2.813506in}{3.189572in}}%
\pgfpathcurveto{\pgfqpoint{2.824556in}{3.189572in}}{\pgfqpoint{2.835155in}{3.193962in}}{\pgfqpoint{2.842969in}{3.201775in}}%
\pgfpathcurveto{\pgfqpoint{2.850782in}{3.209589in}}{\pgfqpoint{2.855173in}{3.220188in}}{\pgfqpoint{2.855173in}{3.231238in}}%
\pgfpathcurveto{\pgfqpoint{2.855173in}{3.242288in}}{\pgfqpoint{2.850782in}{3.252887in}}{\pgfqpoint{2.842969in}{3.260701in}}%
\pgfpathcurveto{\pgfqpoint{2.835155in}{3.268515in}}{\pgfqpoint{2.824556in}{3.272905in}}{\pgfqpoint{2.813506in}{3.272905in}}%
\pgfpathcurveto{\pgfqpoint{2.802456in}{3.272905in}}{\pgfqpoint{2.791857in}{3.268515in}}{\pgfqpoint{2.784043in}{3.260701in}}%
\pgfpathcurveto{\pgfqpoint{2.776230in}{3.252887in}}{\pgfqpoint{2.771839in}{3.242288in}}{\pgfqpoint{2.771839in}{3.231238in}}%
\pgfpathcurveto{\pgfqpoint{2.771839in}{3.220188in}}{\pgfqpoint{2.776230in}{3.209589in}}{\pgfqpoint{2.784043in}{3.201775in}}%
\pgfpathcurveto{\pgfqpoint{2.791857in}{3.193962in}}{\pgfqpoint{2.802456in}{3.189572in}}{\pgfqpoint{2.813506in}{3.189572in}}%
\pgfpathclose%
\pgfusepath{stroke,fill}%
\end{pgfscope}%
\begin{pgfscope}%
\pgfpathrectangle{\pgfqpoint{0.648703in}{0.548769in}}{\pgfqpoint{5.201297in}{3.102590in}}%
\pgfusepath{clip}%
\pgfsetbuttcap%
\pgfsetroundjoin%
\definecolor{currentfill}{rgb}{0.121569,0.466667,0.705882}%
\pgfsetfillcolor{currentfill}%
\pgfsetlinewidth{1.003750pt}%
\definecolor{currentstroke}{rgb}{0.121569,0.466667,0.705882}%
\pgfsetstrokecolor{currentstroke}%
\pgfsetdash{}{0pt}%
\pgfpathmoveto{\pgfqpoint{1.474790in}{0.648129in}}%
\pgfpathcurveto{\pgfqpoint{1.485840in}{0.648129in}}{\pgfqpoint{1.496439in}{0.652519in}}{\pgfqpoint{1.504252in}{0.660333in}}%
\pgfpathcurveto{\pgfqpoint{1.512066in}{0.668146in}}{\pgfqpoint{1.516456in}{0.678745in}}{\pgfqpoint{1.516456in}{0.689796in}}%
\pgfpathcurveto{\pgfqpoint{1.516456in}{0.700846in}}{\pgfqpoint{1.512066in}{0.711445in}}{\pgfqpoint{1.504252in}{0.719258in}}%
\pgfpathcurveto{\pgfqpoint{1.496439in}{0.727072in}}{\pgfqpoint{1.485840in}{0.731462in}}{\pgfqpoint{1.474790in}{0.731462in}}%
\pgfpathcurveto{\pgfqpoint{1.463740in}{0.731462in}}{\pgfqpoint{1.453141in}{0.727072in}}{\pgfqpoint{1.445327in}{0.719258in}}%
\pgfpathcurveto{\pgfqpoint{1.437513in}{0.711445in}}{\pgfqpoint{1.433123in}{0.700846in}}{\pgfqpoint{1.433123in}{0.689796in}}%
\pgfpathcurveto{\pgfqpoint{1.433123in}{0.678745in}}{\pgfqpoint{1.437513in}{0.668146in}}{\pgfqpoint{1.445327in}{0.660333in}}%
\pgfpathcurveto{\pgfqpoint{1.453141in}{0.652519in}}{\pgfqpoint{1.463740in}{0.648129in}}{\pgfqpoint{1.474790in}{0.648129in}}%
\pgfpathclose%
\pgfusepath{stroke,fill}%
\end{pgfscope}%
\begin{pgfscope}%
\pgfpathrectangle{\pgfqpoint{0.648703in}{0.548769in}}{\pgfqpoint{5.201297in}{3.102590in}}%
\pgfusepath{clip}%
\pgfsetbuttcap%
\pgfsetroundjoin%
\definecolor{currentfill}{rgb}{0.121569,0.466667,0.705882}%
\pgfsetfillcolor{currentfill}%
\pgfsetlinewidth{1.003750pt}%
\definecolor{currentstroke}{rgb}{0.121569,0.466667,0.705882}%
\pgfsetstrokecolor{currentstroke}%
\pgfsetdash{}{0pt}%
\pgfpathmoveto{\pgfqpoint{1.690545in}{0.774990in}}%
\pgfpathcurveto{\pgfqpoint{1.701595in}{0.774990in}}{\pgfqpoint{1.712194in}{0.779380in}}{\pgfqpoint{1.720007in}{0.787194in}}%
\pgfpathcurveto{\pgfqpoint{1.727821in}{0.795007in}}{\pgfqpoint{1.732211in}{0.805606in}}{\pgfqpoint{1.732211in}{0.816656in}}%
\pgfpathcurveto{\pgfqpoint{1.732211in}{0.827706in}}{\pgfqpoint{1.727821in}{0.838305in}}{\pgfqpoint{1.720007in}{0.846119in}}%
\pgfpathcurveto{\pgfqpoint{1.712194in}{0.853933in}}{\pgfqpoint{1.701595in}{0.858323in}}{\pgfqpoint{1.690545in}{0.858323in}}%
\pgfpathcurveto{\pgfqpoint{1.679494in}{0.858323in}}{\pgfqpoint{1.668895in}{0.853933in}}{\pgfqpoint{1.661082in}{0.846119in}}%
\pgfpathcurveto{\pgfqpoint{1.653268in}{0.838305in}}{\pgfqpoint{1.648878in}{0.827706in}}{\pgfqpoint{1.648878in}{0.816656in}}%
\pgfpathcurveto{\pgfqpoint{1.648878in}{0.805606in}}{\pgfqpoint{1.653268in}{0.795007in}}{\pgfqpoint{1.661082in}{0.787194in}}%
\pgfpathcurveto{\pgfqpoint{1.668895in}{0.779380in}}{\pgfqpoint{1.679494in}{0.774990in}}{\pgfqpoint{1.690545in}{0.774990in}}%
\pgfpathclose%
\pgfusepath{stroke,fill}%
\end{pgfscope}%
\begin{pgfscope}%
\pgfpathrectangle{\pgfqpoint{0.648703in}{0.548769in}}{\pgfqpoint{5.201297in}{3.102590in}}%
\pgfusepath{clip}%
\pgfsetbuttcap%
\pgfsetroundjoin%
\definecolor{currentfill}{rgb}{0.121569,0.466667,0.705882}%
\pgfsetfillcolor{currentfill}%
\pgfsetlinewidth{1.003750pt}%
\definecolor{currentstroke}{rgb}{0.121569,0.466667,0.705882}%
\pgfsetstrokecolor{currentstroke}%
\pgfsetdash{}{0pt}%
\pgfpathmoveto{\pgfqpoint{0.930200in}{0.783447in}}%
\pgfpathcurveto{\pgfqpoint{0.941250in}{0.783447in}}{\pgfqpoint{0.951849in}{0.787837in}}{\pgfqpoint{0.959663in}{0.795651in}}%
\pgfpathcurveto{\pgfqpoint{0.967476in}{0.803465in}}{\pgfqpoint{0.971866in}{0.814064in}}{\pgfqpoint{0.971866in}{0.825114in}}%
\pgfpathcurveto{\pgfqpoint{0.971866in}{0.836164in}}{\pgfqpoint{0.967476in}{0.846763in}}{\pgfqpoint{0.959663in}{0.854576in}}%
\pgfpathcurveto{\pgfqpoint{0.951849in}{0.862390in}}{\pgfqpoint{0.941250in}{0.866780in}}{\pgfqpoint{0.930200in}{0.866780in}}%
\pgfpathcurveto{\pgfqpoint{0.919150in}{0.866780in}}{\pgfqpoint{0.908551in}{0.862390in}}{\pgfqpoint{0.900737in}{0.854576in}}%
\pgfpathcurveto{\pgfqpoint{0.892923in}{0.846763in}}{\pgfqpoint{0.888533in}{0.836164in}}{\pgfqpoint{0.888533in}{0.825114in}}%
\pgfpathcurveto{\pgfqpoint{0.888533in}{0.814064in}}{\pgfqpoint{0.892923in}{0.803465in}}{\pgfqpoint{0.900737in}{0.795651in}}%
\pgfpathcurveto{\pgfqpoint{0.908551in}{0.787837in}}{\pgfqpoint{0.919150in}{0.783447in}}{\pgfqpoint{0.930200in}{0.783447in}}%
\pgfpathclose%
\pgfusepath{stroke,fill}%
\end{pgfscope}%
\begin{pgfscope}%
\pgfpathrectangle{\pgfqpoint{0.648703in}{0.548769in}}{\pgfqpoint{5.201297in}{3.102590in}}%
\pgfusepath{clip}%
\pgfsetbuttcap%
\pgfsetroundjoin%
\definecolor{currentfill}{rgb}{0.121569,0.466667,0.705882}%
\pgfsetfillcolor{currentfill}%
\pgfsetlinewidth{1.003750pt}%
\definecolor{currentstroke}{rgb}{0.121569,0.466667,0.705882}%
\pgfsetstrokecolor{currentstroke}%
\pgfsetdash{}{0pt}%
\pgfpathmoveto{\pgfqpoint{1.239482in}{0.652358in}}%
\pgfpathcurveto{\pgfqpoint{1.250532in}{0.652358in}}{\pgfqpoint{1.261131in}{0.656748in}}{\pgfqpoint{1.268945in}{0.664562in}}%
\pgfpathcurveto{\pgfqpoint{1.276759in}{0.672375in}}{\pgfqpoint{1.281149in}{0.682974in}}{\pgfqpoint{1.281149in}{0.694024in}}%
\pgfpathcurveto{\pgfqpoint{1.281149in}{0.705074in}}{\pgfqpoint{1.276759in}{0.715673in}}{\pgfqpoint{1.268945in}{0.723487in}}%
\pgfpathcurveto{\pgfqpoint{1.261131in}{0.731301in}}{\pgfqpoint{1.250532in}{0.735691in}}{\pgfqpoint{1.239482in}{0.735691in}}%
\pgfpathcurveto{\pgfqpoint{1.228432in}{0.735691in}}{\pgfqpoint{1.217833in}{0.731301in}}{\pgfqpoint{1.210020in}{0.723487in}}%
\pgfpathcurveto{\pgfqpoint{1.202206in}{0.715673in}}{\pgfqpoint{1.197816in}{0.705074in}}{\pgfqpoint{1.197816in}{0.694024in}}%
\pgfpathcurveto{\pgfqpoint{1.197816in}{0.682974in}}{\pgfqpoint{1.202206in}{0.672375in}}{\pgfqpoint{1.210020in}{0.664562in}}%
\pgfpathcurveto{\pgfqpoint{1.217833in}{0.656748in}}{\pgfqpoint{1.228432in}{0.652358in}}{\pgfqpoint{1.239482in}{0.652358in}}%
\pgfpathclose%
\pgfusepath{stroke,fill}%
\end{pgfscope}%
\begin{pgfscope}%
\pgfpathrectangle{\pgfqpoint{0.648703in}{0.548769in}}{\pgfqpoint{5.201297in}{3.102590in}}%
\pgfusepath{clip}%
\pgfsetbuttcap%
\pgfsetroundjoin%
\definecolor{currentfill}{rgb}{0.121569,0.466667,0.705882}%
\pgfsetfillcolor{currentfill}%
\pgfsetlinewidth{1.003750pt}%
\definecolor{currentstroke}{rgb}{0.121569,0.466667,0.705882}%
\pgfsetstrokecolor{currentstroke}%
\pgfsetdash{}{0pt}%
\pgfpathmoveto{\pgfqpoint{1.274598in}{0.648129in}}%
\pgfpathcurveto{\pgfqpoint{1.285648in}{0.648129in}}{\pgfqpoint{1.296247in}{0.652519in}}{\pgfqpoint{1.304060in}{0.660333in}}%
\pgfpathcurveto{\pgfqpoint{1.311874in}{0.668146in}}{\pgfqpoint{1.316264in}{0.678745in}}{\pgfqpoint{1.316264in}{0.689796in}}%
\pgfpathcurveto{\pgfqpoint{1.316264in}{0.700846in}}{\pgfqpoint{1.311874in}{0.711445in}}{\pgfqpoint{1.304060in}{0.719258in}}%
\pgfpathcurveto{\pgfqpoint{1.296247in}{0.727072in}}{\pgfqpoint{1.285648in}{0.731462in}}{\pgfqpoint{1.274598in}{0.731462in}}%
\pgfpathcurveto{\pgfqpoint{1.263548in}{0.731462in}}{\pgfqpoint{1.252949in}{0.727072in}}{\pgfqpoint{1.245135in}{0.719258in}}%
\pgfpathcurveto{\pgfqpoint{1.237321in}{0.711445in}}{\pgfqpoint{1.232931in}{0.700846in}}{\pgfqpoint{1.232931in}{0.689796in}}%
\pgfpathcurveto{\pgfqpoint{1.232931in}{0.678745in}}{\pgfqpoint{1.237321in}{0.668146in}}{\pgfqpoint{1.245135in}{0.660333in}}%
\pgfpathcurveto{\pgfqpoint{1.252949in}{0.652519in}}{\pgfqpoint{1.263548in}{0.648129in}}{\pgfqpoint{1.274598in}{0.648129in}}%
\pgfpathclose%
\pgfusepath{stroke,fill}%
\end{pgfscope}%
\begin{pgfscope}%
\pgfpathrectangle{\pgfqpoint{0.648703in}{0.548769in}}{\pgfqpoint{5.201297in}{3.102590in}}%
\pgfusepath{clip}%
\pgfsetbuttcap%
\pgfsetroundjoin%
\definecolor{currentfill}{rgb}{0.121569,0.466667,0.705882}%
\pgfsetfillcolor{currentfill}%
\pgfsetlinewidth{1.003750pt}%
\definecolor{currentstroke}{rgb}{0.121569,0.466667,0.705882}%
\pgfsetstrokecolor{currentstroke}%
\pgfsetdash{}{0pt}%
\pgfpathmoveto{\pgfqpoint{1.634147in}{0.648129in}}%
\pgfpathcurveto{\pgfqpoint{1.645197in}{0.648129in}}{\pgfqpoint{1.655796in}{0.652519in}}{\pgfqpoint{1.663610in}{0.660333in}}%
\pgfpathcurveto{\pgfqpoint{1.671423in}{0.668146in}}{\pgfqpoint{1.675814in}{0.678745in}}{\pgfqpoint{1.675814in}{0.689796in}}%
\pgfpathcurveto{\pgfqpoint{1.675814in}{0.700846in}}{\pgfqpoint{1.671423in}{0.711445in}}{\pgfqpoint{1.663610in}{0.719258in}}%
\pgfpathcurveto{\pgfqpoint{1.655796in}{0.727072in}}{\pgfqpoint{1.645197in}{0.731462in}}{\pgfqpoint{1.634147in}{0.731462in}}%
\pgfpathcurveto{\pgfqpoint{1.623097in}{0.731462in}}{\pgfqpoint{1.612498in}{0.727072in}}{\pgfqpoint{1.604684in}{0.719258in}}%
\pgfpathcurveto{\pgfqpoint{1.596871in}{0.711445in}}{\pgfqpoint{1.592480in}{0.700846in}}{\pgfqpoint{1.592480in}{0.689796in}}%
\pgfpathcurveto{\pgfqpoint{1.592480in}{0.678745in}}{\pgfqpoint{1.596871in}{0.668146in}}{\pgfqpoint{1.604684in}{0.660333in}}%
\pgfpathcurveto{\pgfqpoint{1.612498in}{0.652519in}}{\pgfqpoint{1.623097in}{0.648129in}}{\pgfqpoint{1.634147in}{0.648129in}}%
\pgfpathclose%
\pgfusepath{stroke,fill}%
\end{pgfscope}%
\begin{pgfscope}%
\pgfpathrectangle{\pgfqpoint{0.648703in}{0.548769in}}{\pgfqpoint{5.201297in}{3.102590in}}%
\pgfusepath{clip}%
\pgfsetbuttcap%
\pgfsetroundjoin%
\definecolor{currentfill}{rgb}{1.000000,0.498039,0.054902}%
\pgfsetfillcolor{currentfill}%
\pgfsetlinewidth{1.003750pt}%
\definecolor{currentstroke}{rgb}{1.000000,0.498039,0.054902}%
\pgfsetstrokecolor{currentstroke}%
\pgfsetdash{}{0pt}%
\pgfpathmoveto{\pgfqpoint{1.522226in}{3.193800in}}%
\pgfpathcurveto{\pgfqpoint{1.533277in}{3.193800in}}{\pgfqpoint{1.543876in}{3.198191in}}{\pgfqpoint{1.551689in}{3.206004in}}%
\pgfpathcurveto{\pgfqpoint{1.559503in}{3.213818in}}{\pgfqpoint{1.563893in}{3.224417in}}{\pgfqpoint{1.563893in}{3.235467in}}%
\pgfpathcurveto{\pgfqpoint{1.563893in}{3.246517in}}{\pgfqpoint{1.559503in}{3.257116in}}{\pgfqpoint{1.551689in}{3.264930in}}%
\pgfpathcurveto{\pgfqpoint{1.543876in}{3.272743in}}{\pgfqpoint{1.533277in}{3.277134in}}{\pgfqpoint{1.522226in}{3.277134in}}%
\pgfpathcurveto{\pgfqpoint{1.511176in}{3.277134in}}{\pgfqpoint{1.500577in}{3.272743in}}{\pgfqpoint{1.492764in}{3.264930in}}%
\pgfpathcurveto{\pgfqpoint{1.484950in}{3.257116in}}{\pgfqpoint{1.480560in}{3.246517in}}{\pgfqpoint{1.480560in}{3.235467in}}%
\pgfpathcurveto{\pgfqpoint{1.480560in}{3.224417in}}{\pgfqpoint{1.484950in}{3.213818in}}{\pgfqpoint{1.492764in}{3.206004in}}%
\pgfpathcurveto{\pgfqpoint{1.500577in}{3.198191in}}{\pgfqpoint{1.511176in}{3.193800in}}{\pgfqpoint{1.522226in}{3.193800in}}%
\pgfpathclose%
\pgfusepath{stroke,fill}%
\end{pgfscope}%
\begin{pgfscope}%
\pgfpathrectangle{\pgfqpoint{0.648703in}{0.548769in}}{\pgfqpoint{5.201297in}{3.102590in}}%
\pgfusepath{clip}%
\pgfsetbuttcap%
\pgfsetroundjoin%
\definecolor{currentfill}{rgb}{1.000000,0.498039,0.054902}%
\pgfsetfillcolor{currentfill}%
\pgfsetlinewidth{1.003750pt}%
\definecolor{currentstroke}{rgb}{1.000000,0.498039,0.054902}%
\pgfsetstrokecolor{currentstroke}%
\pgfsetdash{}{0pt}%
\pgfpathmoveto{\pgfqpoint{1.322687in}{3.231859in}}%
\pgfpathcurveto{\pgfqpoint{1.333738in}{3.231859in}}{\pgfqpoint{1.344337in}{3.236249in}}{\pgfqpoint{1.352150in}{3.244062in}}%
\pgfpathcurveto{\pgfqpoint{1.359964in}{3.251876in}}{\pgfqpoint{1.364354in}{3.262475in}}{\pgfqpoint{1.364354in}{3.273525in}}%
\pgfpathcurveto{\pgfqpoint{1.364354in}{3.284575in}}{\pgfqpoint{1.359964in}{3.295174in}}{\pgfqpoint{1.352150in}{3.302988in}}%
\pgfpathcurveto{\pgfqpoint{1.344337in}{3.310802in}}{\pgfqpoint{1.333738in}{3.315192in}}{\pgfqpoint{1.322687in}{3.315192in}}%
\pgfpathcurveto{\pgfqpoint{1.311637in}{3.315192in}}{\pgfqpoint{1.301038in}{3.310802in}}{\pgfqpoint{1.293225in}{3.302988in}}%
\pgfpathcurveto{\pgfqpoint{1.285411in}{3.295174in}}{\pgfqpoint{1.281021in}{3.284575in}}{\pgfqpoint{1.281021in}{3.273525in}}%
\pgfpathcurveto{\pgfqpoint{1.281021in}{3.262475in}}{\pgfqpoint{1.285411in}{3.251876in}}{\pgfqpoint{1.293225in}{3.244062in}}%
\pgfpathcurveto{\pgfqpoint{1.301038in}{3.236249in}}{\pgfqpoint{1.311637in}{3.231859in}}{\pgfqpoint{1.322687in}{3.231859in}}%
\pgfpathclose%
\pgfusepath{stroke,fill}%
\end{pgfscope}%
\begin{pgfscope}%
\pgfpathrectangle{\pgfqpoint{0.648703in}{0.548769in}}{\pgfqpoint{5.201297in}{3.102590in}}%
\pgfusepath{clip}%
\pgfsetbuttcap%
\pgfsetroundjoin%
\definecolor{currentfill}{rgb}{0.121569,0.466667,0.705882}%
\pgfsetfillcolor{currentfill}%
\pgfsetlinewidth{1.003750pt}%
\definecolor{currentstroke}{rgb}{0.121569,0.466667,0.705882}%
\pgfsetstrokecolor{currentstroke}%
\pgfsetdash{}{0pt}%
\pgfpathmoveto{\pgfqpoint{1.304219in}{0.758075in}}%
\pgfpathcurveto{\pgfqpoint{1.315270in}{0.758075in}}{\pgfqpoint{1.325869in}{0.762465in}}{\pgfqpoint{1.333682in}{0.770279in}}%
\pgfpathcurveto{\pgfqpoint{1.341496in}{0.778092in}}{\pgfqpoint{1.345886in}{0.788691in}}{\pgfqpoint{1.345886in}{0.799742in}}%
\pgfpathcurveto{\pgfqpoint{1.345886in}{0.810792in}}{\pgfqpoint{1.341496in}{0.821391in}}{\pgfqpoint{1.333682in}{0.829204in}}%
\pgfpathcurveto{\pgfqpoint{1.325869in}{0.837018in}}{\pgfqpoint{1.315270in}{0.841408in}}{\pgfqpoint{1.304219in}{0.841408in}}%
\pgfpathcurveto{\pgfqpoint{1.293169in}{0.841408in}}{\pgfqpoint{1.282570in}{0.837018in}}{\pgfqpoint{1.274757in}{0.829204in}}%
\pgfpathcurveto{\pgfqpoint{1.266943in}{0.821391in}}{\pgfqpoint{1.262553in}{0.810792in}}{\pgfqpoint{1.262553in}{0.799742in}}%
\pgfpathcurveto{\pgfqpoint{1.262553in}{0.788691in}}{\pgfqpoint{1.266943in}{0.778092in}}{\pgfqpoint{1.274757in}{0.770279in}}%
\pgfpathcurveto{\pgfqpoint{1.282570in}{0.762465in}}{\pgfqpoint{1.293169in}{0.758075in}}{\pgfqpoint{1.304219in}{0.758075in}}%
\pgfpathclose%
\pgfusepath{stroke,fill}%
\end{pgfscope}%
\begin{pgfscope}%
\pgfpathrectangle{\pgfqpoint{0.648703in}{0.548769in}}{\pgfqpoint{5.201297in}{3.102590in}}%
\pgfusepath{clip}%
\pgfsetbuttcap%
\pgfsetroundjoin%
\definecolor{currentfill}{rgb}{1.000000,0.498039,0.054902}%
\pgfsetfillcolor{currentfill}%
\pgfsetlinewidth{1.003750pt}%
\definecolor{currentstroke}{rgb}{1.000000,0.498039,0.054902}%
\pgfsetstrokecolor{currentstroke}%
\pgfsetdash{}{0pt}%
\pgfpathmoveto{\pgfqpoint{1.346689in}{3.185343in}}%
\pgfpathcurveto{\pgfqpoint{1.357739in}{3.185343in}}{\pgfqpoint{1.368338in}{3.189733in}}{\pgfqpoint{1.376152in}{3.197547in}}%
\pgfpathcurveto{\pgfqpoint{1.383965in}{3.205360in}}{\pgfqpoint{1.388356in}{3.215959in}}{\pgfqpoint{1.388356in}{3.227010in}}%
\pgfpathcurveto{\pgfqpoint{1.388356in}{3.238060in}}{\pgfqpoint{1.383965in}{3.248659in}}{\pgfqpoint{1.376152in}{3.256472in}}%
\pgfpathcurveto{\pgfqpoint{1.368338in}{3.264286in}}{\pgfqpoint{1.357739in}{3.268676in}}{\pgfqpoint{1.346689in}{3.268676in}}%
\pgfpathcurveto{\pgfqpoint{1.335639in}{3.268676in}}{\pgfqpoint{1.325040in}{3.264286in}}{\pgfqpoint{1.317226in}{3.256472in}}%
\pgfpathcurveto{\pgfqpoint{1.309412in}{3.248659in}}{\pgfqpoint{1.305022in}{3.238060in}}{\pgfqpoint{1.305022in}{3.227010in}}%
\pgfpathcurveto{\pgfqpoint{1.305022in}{3.215959in}}{\pgfqpoint{1.309412in}{3.205360in}}{\pgfqpoint{1.317226in}{3.197547in}}%
\pgfpathcurveto{\pgfqpoint{1.325040in}{3.189733in}}{\pgfqpoint{1.335639in}{3.185343in}}{\pgfqpoint{1.346689in}{3.185343in}}%
\pgfpathclose%
\pgfusepath{stroke,fill}%
\end{pgfscope}%
\begin{pgfscope}%
\pgfpathrectangle{\pgfqpoint{0.648703in}{0.548769in}}{\pgfqpoint{5.201297in}{3.102590in}}%
\pgfusepath{clip}%
\pgfsetbuttcap%
\pgfsetroundjoin%
\definecolor{currentfill}{rgb}{1.000000,0.498039,0.054902}%
\pgfsetfillcolor{currentfill}%
\pgfsetlinewidth{1.003750pt}%
\definecolor{currentstroke}{rgb}{1.000000,0.498039,0.054902}%
\pgfsetstrokecolor{currentstroke}%
\pgfsetdash{}{0pt}%
\pgfpathmoveto{\pgfqpoint{1.135177in}{3.202258in}}%
\pgfpathcurveto{\pgfqpoint{1.146227in}{3.202258in}}{\pgfqpoint{1.156826in}{3.206648in}}{\pgfqpoint{1.164640in}{3.214462in}}%
\pgfpathcurveto{\pgfqpoint{1.172453in}{3.222275in}}{\pgfqpoint{1.176844in}{3.232874in}}{\pgfqpoint{1.176844in}{3.243924in}}%
\pgfpathcurveto{\pgfqpoint{1.176844in}{3.254974in}}{\pgfqpoint{1.172453in}{3.265573in}}{\pgfqpoint{1.164640in}{3.273387in}}%
\pgfpathcurveto{\pgfqpoint{1.156826in}{3.281201in}}{\pgfqpoint{1.146227in}{3.285591in}}{\pgfqpoint{1.135177in}{3.285591in}}%
\pgfpathcurveto{\pgfqpoint{1.124127in}{3.285591in}}{\pgfqpoint{1.113528in}{3.281201in}}{\pgfqpoint{1.105714in}{3.273387in}}%
\pgfpathcurveto{\pgfqpoint{1.097901in}{3.265573in}}{\pgfqpoint{1.093510in}{3.254974in}}{\pgfqpoint{1.093510in}{3.243924in}}%
\pgfpathcurveto{\pgfqpoint{1.093510in}{3.232874in}}{\pgfqpoint{1.097901in}{3.222275in}}{\pgfqpoint{1.105714in}{3.214462in}}%
\pgfpathcurveto{\pgfqpoint{1.113528in}{3.206648in}}{\pgfqpoint{1.124127in}{3.202258in}}{\pgfqpoint{1.135177in}{3.202258in}}%
\pgfpathclose%
\pgfusepath{stroke,fill}%
\end{pgfscope}%
\begin{pgfscope}%
\pgfpathrectangle{\pgfqpoint{0.648703in}{0.548769in}}{\pgfqpoint{5.201297in}{3.102590in}}%
\pgfusepath{clip}%
\pgfsetbuttcap%
\pgfsetroundjoin%
\definecolor{currentfill}{rgb}{0.121569,0.466667,0.705882}%
\pgfsetfillcolor{currentfill}%
\pgfsetlinewidth{1.003750pt}%
\definecolor{currentstroke}{rgb}{0.121569,0.466667,0.705882}%
\pgfsetstrokecolor{currentstroke}%
\pgfsetdash{}{0pt}%
\pgfpathmoveto{\pgfqpoint{5.613577in}{0.648129in}}%
\pgfpathcurveto{\pgfqpoint{5.624628in}{0.648129in}}{\pgfqpoint{5.635227in}{0.652519in}}{\pgfqpoint{5.643040in}{0.660333in}}%
\pgfpathcurveto{\pgfqpoint{5.650854in}{0.668146in}}{\pgfqpoint{5.655244in}{0.678745in}}{\pgfqpoint{5.655244in}{0.689796in}}%
\pgfpathcurveto{\pgfqpoint{5.655244in}{0.700846in}}{\pgfqpoint{5.650854in}{0.711445in}}{\pgfqpoint{5.643040in}{0.719258in}}%
\pgfpathcurveto{\pgfqpoint{5.635227in}{0.727072in}}{\pgfqpoint{5.624628in}{0.731462in}}{\pgfqpoint{5.613577in}{0.731462in}}%
\pgfpathcurveto{\pgfqpoint{5.602527in}{0.731462in}}{\pgfqpoint{5.591928in}{0.727072in}}{\pgfqpoint{5.584115in}{0.719258in}}%
\pgfpathcurveto{\pgfqpoint{5.576301in}{0.711445in}}{\pgfqpoint{5.571911in}{0.700846in}}{\pgfqpoint{5.571911in}{0.689796in}}%
\pgfpathcurveto{\pgfqpoint{5.571911in}{0.678745in}}{\pgfqpoint{5.576301in}{0.668146in}}{\pgfqpoint{5.584115in}{0.660333in}}%
\pgfpathcurveto{\pgfqpoint{5.591928in}{0.652519in}}{\pgfqpoint{5.602527in}{0.648129in}}{\pgfqpoint{5.613577in}{0.648129in}}%
\pgfpathclose%
\pgfusepath{stroke,fill}%
\end{pgfscope}%
\begin{pgfscope}%
\pgfpathrectangle{\pgfqpoint{0.648703in}{0.548769in}}{\pgfqpoint{5.201297in}{3.102590in}}%
\pgfusepath{clip}%
\pgfsetbuttcap%
\pgfsetroundjoin%
\definecolor{currentfill}{rgb}{1.000000,0.498039,0.054902}%
\pgfsetfillcolor{currentfill}%
\pgfsetlinewidth{1.003750pt}%
\definecolor{currentstroke}{rgb}{1.000000,0.498039,0.054902}%
\pgfsetstrokecolor{currentstroke}%
\pgfsetdash{}{0pt}%
\pgfpathmoveto{\pgfqpoint{1.300855in}{3.206486in}}%
\pgfpathcurveto{\pgfqpoint{1.311905in}{3.206486in}}{\pgfqpoint{1.322504in}{3.210877in}}{\pgfqpoint{1.330318in}{3.218690in}}%
\pgfpathcurveto{\pgfqpoint{1.338132in}{3.226504in}}{\pgfqpoint{1.342522in}{3.237103in}}{\pgfqpoint{1.342522in}{3.248153in}}%
\pgfpathcurveto{\pgfqpoint{1.342522in}{3.259203in}}{\pgfqpoint{1.338132in}{3.269802in}}{\pgfqpoint{1.330318in}{3.277616in}}%
\pgfpathcurveto{\pgfqpoint{1.322504in}{3.285429in}}{\pgfqpoint{1.311905in}{3.289820in}}{\pgfqpoint{1.300855in}{3.289820in}}%
\pgfpathcurveto{\pgfqpoint{1.289805in}{3.289820in}}{\pgfqpoint{1.279206in}{3.285429in}}{\pgfqpoint{1.271392in}{3.277616in}}%
\pgfpathcurveto{\pgfqpoint{1.263579in}{3.269802in}}{\pgfqpoint{1.259188in}{3.259203in}}{\pgfqpoint{1.259188in}{3.248153in}}%
\pgfpathcurveto{\pgfqpoint{1.259188in}{3.237103in}}{\pgfqpoint{1.263579in}{3.226504in}}{\pgfqpoint{1.271392in}{3.218690in}}%
\pgfpathcurveto{\pgfqpoint{1.279206in}{3.210877in}}{\pgfqpoint{1.289805in}{3.206486in}}{\pgfqpoint{1.300855in}{3.206486in}}%
\pgfpathclose%
\pgfusepath{stroke,fill}%
\end{pgfscope}%
\begin{pgfscope}%
\pgfpathrectangle{\pgfqpoint{0.648703in}{0.548769in}}{\pgfqpoint{5.201297in}{3.102590in}}%
\pgfusepath{clip}%
\pgfsetbuttcap%
\pgfsetroundjoin%
\definecolor{currentfill}{rgb}{0.121569,0.466667,0.705882}%
\pgfsetfillcolor{currentfill}%
\pgfsetlinewidth{1.003750pt}%
\definecolor{currentstroke}{rgb}{0.121569,0.466667,0.705882}%
\pgfsetstrokecolor{currentstroke}%
\pgfsetdash{}{0pt}%
\pgfpathmoveto{\pgfqpoint{1.100576in}{0.648129in}}%
\pgfpathcurveto{\pgfqpoint{1.111626in}{0.648129in}}{\pgfqpoint{1.122225in}{0.652519in}}{\pgfqpoint{1.130039in}{0.660333in}}%
\pgfpathcurveto{\pgfqpoint{1.137852in}{0.668146in}}{\pgfqpoint{1.142243in}{0.678745in}}{\pgfqpoint{1.142243in}{0.689796in}}%
\pgfpathcurveto{\pgfqpoint{1.142243in}{0.700846in}}{\pgfqpoint{1.137852in}{0.711445in}}{\pgfqpoint{1.130039in}{0.719258in}}%
\pgfpathcurveto{\pgfqpoint{1.122225in}{0.727072in}}{\pgfqpoint{1.111626in}{0.731462in}}{\pgfqpoint{1.100576in}{0.731462in}}%
\pgfpathcurveto{\pgfqpoint{1.089526in}{0.731462in}}{\pgfqpoint{1.078927in}{0.727072in}}{\pgfqpoint{1.071113in}{0.719258in}}%
\pgfpathcurveto{\pgfqpoint{1.063300in}{0.711445in}}{\pgfqpoint{1.058909in}{0.700846in}}{\pgfqpoint{1.058909in}{0.689796in}}%
\pgfpathcurveto{\pgfqpoint{1.058909in}{0.678745in}}{\pgfqpoint{1.063300in}{0.668146in}}{\pgfqpoint{1.071113in}{0.660333in}}%
\pgfpathcurveto{\pgfqpoint{1.078927in}{0.652519in}}{\pgfqpoint{1.089526in}{0.648129in}}{\pgfqpoint{1.100576in}{0.648129in}}%
\pgfpathclose%
\pgfusepath{stroke,fill}%
\end{pgfscope}%
\begin{pgfscope}%
\pgfpathrectangle{\pgfqpoint{0.648703in}{0.548769in}}{\pgfqpoint{5.201297in}{3.102590in}}%
\pgfusepath{clip}%
\pgfsetbuttcap%
\pgfsetroundjoin%
\definecolor{currentfill}{rgb}{0.121569,0.466667,0.705882}%
\pgfsetfillcolor{currentfill}%
\pgfsetlinewidth{1.003750pt}%
\definecolor{currentstroke}{rgb}{0.121569,0.466667,0.705882}%
\pgfsetstrokecolor{currentstroke}%
\pgfsetdash{}{0pt}%
\pgfpathmoveto{\pgfqpoint{1.505809in}{0.648129in}}%
\pgfpathcurveto{\pgfqpoint{1.516859in}{0.648129in}}{\pgfqpoint{1.527458in}{0.652519in}}{\pgfqpoint{1.535271in}{0.660333in}}%
\pgfpathcurveto{\pgfqpoint{1.543085in}{0.668146in}}{\pgfqpoint{1.547475in}{0.678745in}}{\pgfqpoint{1.547475in}{0.689796in}}%
\pgfpathcurveto{\pgfqpoint{1.547475in}{0.700846in}}{\pgfqpoint{1.543085in}{0.711445in}}{\pgfqpoint{1.535271in}{0.719258in}}%
\pgfpathcurveto{\pgfqpoint{1.527458in}{0.727072in}}{\pgfqpoint{1.516859in}{0.731462in}}{\pgfqpoint{1.505809in}{0.731462in}}%
\pgfpathcurveto{\pgfqpoint{1.494758in}{0.731462in}}{\pgfqpoint{1.484159in}{0.727072in}}{\pgfqpoint{1.476346in}{0.719258in}}%
\pgfpathcurveto{\pgfqpoint{1.468532in}{0.711445in}}{\pgfqpoint{1.464142in}{0.700846in}}{\pgfqpoint{1.464142in}{0.689796in}}%
\pgfpathcurveto{\pgfqpoint{1.464142in}{0.678745in}}{\pgfqpoint{1.468532in}{0.668146in}}{\pgfqpoint{1.476346in}{0.660333in}}%
\pgfpathcurveto{\pgfqpoint{1.484159in}{0.652519in}}{\pgfqpoint{1.494758in}{0.648129in}}{\pgfqpoint{1.505809in}{0.648129in}}%
\pgfpathclose%
\pgfusepath{stroke,fill}%
\end{pgfscope}%
\begin{pgfscope}%
\pgfpathrectangle{\pgfqpoint{0.648703in}{0.548769in}}{\pgfqpoint{5.201297in}{3.102590in}}%
\pgfusepath{clip}%
\pgfsetbuttcap%
\pgfsetroundjoin%
\definecolor{currentfill}{rgb}{0.121569,0.466667,0.705882}%
\pgfsetfillcolor{currentfill}%
\pgfsetlinewidth{1.003750pt}%
\definecolor{currentstroke}{rgb}{0.121569,0.466667,0.705882}%
\pgfsetstrokecolor{currentstroke}%
\pgfsetdash{}{0pt}%
\pgfpathmoveto{\pgfqpoint{1.098086in}{0.817277in}}%
\pgfpathcurveto{\pgfqpoint{1.109137in}{0.817277in}}{\pgfqpoint{1.119736in}{0.821667in}}{\pgfqpoint{1.127549in}{0.829480in}}%
\pgfpathcurveto{\pgfqpoint{1.135363in}{0.837294in}}{\pgfqpoint{1.139753in}{0.847893in}}{\pgfqpoint{1.139753in}{0.858943in}}%
\pgfpathcurveto{\pgfqpoint{1.139753in}{0.869993in}}{\pgfqpoint{1.135363in}{0.880592in}}{\pgfqpoint{1.127549in}{0.888406in}}%
\pgfpathcurveto{\pgfqpoint{1.119736in}{0.896220in}}{\pgfqpoint{1.109137in}{0.900610in}}{\pgfqpoint{1.098086in}{0.900610in}}%
\pgfpathcurveto{\pgfqpoint{1.087036in}{0.900610in}}{\pgfqpoint{1.076437in}{0.896220in}}{\pgfqpoint{1.068624in}{0.888406in}}%
\pgfpathcurveto{\pgfqpoint{1.060810in}{0.880592in}}{\pgfqpoint{1.056420in}{0.869993in}}{\pgfqpoint{1.056420in}{0.858943in}}%
\pgfpathcurveto{\pgfqpoint{1.056420in}{0.847893in}}{\pgfqpoint{1.060810in}{0.837294in}}{\pgfqpoint{1.068624in}{0.829480in}}%
\pgfpathcurveto{\pgfqpoint{1.076437in}{0.821667in}}{\pgfqpoint{1.087036in}{0.817277in}}{\pgfqpoint{1.098086in}{0.817277in}}%
\pgfpathclose%
\pgfusepath{stroke,fill}%
\end{pgfscope}%
\begin{pgfscope}%
\pgfpathrectangle{\pgfqpoint{0.648703in}{0.548769in}}{\pgfqpoint{5.201297in}{3.102590in}}%
\pgfusepath{clip}%
\pgfsetbuttcap%
\pgfsetroundjoin%
\definecolor{currentfill}{rgb}{1.000000,0.498039,0.054902}%
\pgfsetfillcolor{currentfill}%
\pgfsetlinewidth{1.003750pt}%
\definecolor{currentstroke}{rgb}{1.000000,0.498039,0.054902}%
\pgfsetstrokecolor{currentstroke}%
\pgfsetdash{}{0pt}%
\pgfpathmoveto{\pgfqpoint{2.470565in}{3.189572in}}%
\pgfpathcurveto{\pgfqpoint{2.481615in}{3.189572in}}{\pgfqpoint{2.492214in}{3.193962in}}{\pgfqpoint{2.500027in}{3.201775in}}%
\pgfpathcurveto{\pgfqpoint{2.507841in}{3.209589in}}{\pgfqpoint{2.512231in}{3.220188in}}{\pgfqpoint{2.512231in}{3.231238in}}%
\pgfpathcurveto{\pgfqpoint{2.512231in}{3.242288in}}{\pgfqpoint{2.507841in}{3.252887in}}{\pgfqpoint{2.500027in}{3.260701in}}%
\pgfpathcurveto{\pgfqpoint{2.492214in}{3.268515in}}{\pgfqpoint{2.481615in}{3.272905in}}{\pgfqpoint{2.470565in}{3.272905in}}%
\pgfpathcurveto{\pgfqpoint{2.459514in}{3.272905in}}{\pgfqpoint{2.448915in}{3.268515in}}{\pgfqpoint{2.441102in}{3.260701in}}%
\pgfpathcurveto{\pgfqpoint{2.433288in}{3.252887in}}{\pgfqpoint{2.428898in}{3.242288in}}{\pgfqpoint{2.428898in}{3.231238in}}%
\pgfpathcurveto{\pgfqpoint{2.428898in}{3.220188in}}{\pgfqpoint{2.433288in}{3.209589in}}{\pgfqpoint{2.441102in}{3.201775in}}%
\pgfpathcurveto{\pgfqpoint{2.448915in}{3.193962in}}{\pgfqpoint{2.459514in}{3.189572in}}{\pgfqpoint{2.470565in}{3.189572in}}%
\pgfpathclose%
\pgfusepath{stroke,fill}%
\end{pgfscope}%
\begin{pgfscope}%
\pgfpathrectangle{\pgfqpoint{0.648703in}{0.548769in}}{\pgfqpoint{5.201297in}{3.102590in}}%
\pgfusepath{clip}%
\pgfsetbuttcap%
\pgfsetroundjoin%
\definecolor{currentfill}{rgb}{1.000000,0.498039,0.054902}%
\pgfsetfillcolor{currentfill}%
\pgfsetlinewidth{1.003750pt}%
\definecolor{currentstroke}{rgb}{1.000000,0.498039,0.054902}%
\pgfsetstrokecolor{currentstroke}%
\pgfsetdash{}{0pt}%
\pgfpathmoveto{\pgfqpoint{2.017278in}{3.214944in}}%
\pgfpathcurveto{\pgfqpoint{2.028328in}{3.214944in}}{\pgfqpoint{2.038927in}{3.219334in}}{\pgfqpoint{2.046741in}{3.227148in}}%
\pgfpathcurveto{\pgfqpoint{2.054554in}{3.234961in}}{\pgfqpoint{2.058945in}{3.245560in}}{\pgfqpoint{2.058945in}{3.256610in}}%
\pgfpathcurveto{\pgfqpoint{2.058945in}{3.267661in}}{\pgfqpoint{2.054554in}{3.278260in}}{\pgfqpoint{2.046741in}{3.286073in}}%
\pgfpathcurveto{\pgfqpoint{2.038927in}{3.293887in}}{\pgfqpoint{2.028328in}{3.298277in}}{\pgfqpoint{2.017278in}{3.298277in}}%
\pgfpathcurveto{\pgfqpoint{2.006228in}{3.298277in}}{\pgfqpoint{1.995629in}{3.293887in}}{\pgfqpoint{1.987815in}{3.286073in}}%
\pgfpathcurveto{\pgfqpoint{1.980002in}{3.278260in}}{\pgfqpoint{1.975611in}{3.267661in}}{\pgfqpoint{1.975611in}{3.256610in}}%
\pgfpathcurveto{\pgfqpoint{1.975611in}{3.245560in}}{\pgfqpoint{1.980002in}{3.234961in}}{\pgfqpoint{1.987815in}{3.227148in}}%
\pgfpathcurveto{\pgfqpoint{1.995629in}{3.219334in}}{\pgfqpoint{2.006228in}{3.214944in}}{\pgfqpoint{2.017278in}{3.214944in}}%
\pgfpathclose%
\pgfusepath{stroke,fill}%
\end{pgfscope}%
\begin{pgfscope}%
\pgfpathrectangle{\pgfqpoint{0.648703in}{0.548769in}}{\pgfqpoint{5.201297in}{3.102590in}}%
\pgfusepath{clip}%
\pgfsetbuttcap%
\pgfsetroundjoin%
\definecolor{currentfill}{rgb}{0.121569,0.466667,0.705882}%
\pgfsetfillcolor{currentfill}%
\pgfsetlinewidth{1.003750pt}%
\definecolor{currentstroke}{rgb}{0.121569,0.466667,0.705882}%
\pgfsetstrokecolor{currentstroke}%
\pgfsetdash{}{0pt}%
\pgfpathmoveto{\pgfqpoint{1.318943in}{0.648129in}}%
\pgfpathcurveto{\pgfqpoint{1.329993in}{0.648129in}}{\pgfqpoint{1.340592in}{0.652519in}}{\pgfqpoint{1.348406in}{0.660333in}}%
\pgfpathcurveto{\pgfqpoint{1.356220in}{0.668146in}}{\pgfqpoint{1.360610in}{0.678745in}}{\pgfqpoint{1.360610in}{0.689796in}}%
\pgfpathcurveto{\pgfqpoint{1.360610in}{0.700846in}}{\pgfqpoint{1.356220in}{0.711445in}}{\pgfqpoint{1.348406in}{0.719258in}}%
\pgfpathcurveto{\pgfqpoint{1.340592in}{0.727072in}}{\pgfqpoint{1.329993in}{0.731462in}}{\pgfqpoint{1.318943in}{0.731462in}}%
\pgfpathcurveto{\pgfqpoint{1.307893in}{0.731462in}}{\pgfqpoint{1.297294in}{0.727072in}}{\pgfqpoint{1.289480in}{0.719258in}}%
\pgfpathcurveto{\pgfqpoint{1.281667in}{0.711445in}}{\pgfqpoint{1.277277in}{0.700846in}}{\pgfqpoint{1.277277in}{0.689796in}}%
\pgfpathcurveto{\pgfqpoint{1.277277in}{0.678745in}}{\pgfqpoint{1.281667in}{0.668146in}}{\pgfqpoint{1.289480in}{0.660333in}}%
\pgfpathcurveto{\pgfqpoint{1.297294in}{0.652519in}}{\pgfqpoint{1.307893in}{0.648129in}}{\pgfqpoint{1.318943in}{0.648129in}}%
\pgfpathclose%
\pgfusepath{stroke,fill}%
\end{pgfscope}%
\begin{pgfscope}%
\pgfpathrectangle{\pgfqpoint{0.648703in}{0.548769in}}{\pgfqpoint{5.201297in}{3.102590in}}%
\pgfusepath{clip}%
\pgfsetbuttcap%
\pgfsetroundjoin%
\definecolor{currentfill}{rgb}{0.121569,0.466667,0.705882}%
\pgfsetfillcolor{currentfill}%
\pgfsetlinewidth{1.003750pt}%
\definecolor{currentstroke}{rgb}{0.121569,0.466667,0.705882}%
\pgfsetstrokecolor{currentstroke}%
\pgfsetdash{}{0pt}%
\pgfpathmoveto{\pgfqpoint{0.935717in}{0.758075in}}%
\pgfpathcurveto{\pgfqpoint{0.946767in}{0.758075in}}{\pgfqpoint{0.957366in}{0.762465in}}{\pgfqpoint{0.965180in}{0.770279in}}%
\pgfpathcurveto{\pgfqpoint{0.972994in}{0.778092in}}{\pgfqpoint{0.977384in}{0.788691in}}{\pgfqpoint{0.977384in}{0.799742in}}%
\pgfpathcurveto{\pgfqpoint{0.977384in}{0.810792in}}{\pgfqpoint{0.972994in}{0.821391in}}{\pgfqpoint{0.965180in}{0.829204in}}%
\pgfpathcurveto{\pgfqpoint{0.957366in}{0.837018in}}{\pgfqpoint{0.946767in}{0.841408in}}{\pgfqpoint{0.935717in}{0.841408in}}%
\pgfpathcurveto{\pgfqpoint{0.924667in}{0.841408in}}{\pgfqpoint{0.914068in}{0.837018in}}{\pgfqpoint{0.906254in}{0.829204in}}%
\pgfpathcurveto{\pgfqpoint{0.898441in}{0.821391in}}{\pgfqpoint{0.894051in}{0.810792in}}{\pgfqpoint{0.894051in}{0.799742in}}%
\pgfpathcurveto{\pgfqpoint{0.894051in}{0.788691in}}{\pgfqpoint{0.898441in}{0.778092in}}{\pgfqpoint{0.906254in}{0.770279in}}%
\pgfpathcurveto{\pgfqpoint{0.914068in}{0.762465in}}{\pgfqpoint{0.924667in}{0.758075in}}{\pgfqpoint{0.935717in}{0.758075in}}%
\pgfpathclose%
\pgfusepath{stroke,fill}%
\end{pgfscope}%
\begin{pgfscope}%
\pgfpathrectangle{\pgfqpoint{0.648703in}{0.548769in}}{\pgfqpoint{5.201297in}{3.102590in}}%
\pgfusepath{clip}%
\pgfsetbuttcap%
\pgfsetroundjoin%
\definecolor{currentfill}{rgb}{1.000000,0.498039,0.054902}%
\pgfsetfillcolor{currentfill}%
\pgfsetlinewidth{1.003750pt}%
\definecolor{currentstroke}{rgb}{1.000000,0.498039,0.054902}%
\pgfsetstrokecolor{currentstroke}%
\pgfsetdash{}{0pt}%
\pgfpathmoveto{\pgfqpoint{2.391452in}{3.193800in}}%
\pgfpathcurveto{\pgfqpoint{2.402502in}{3.193800in}}{\pgfqpoint{2.413101in}{3.198191in}}{\pgfqpoint{2.420915in}{3.206004in}}%
\pgfpathcurveto{\pgfqpoint{2.428728in}{3.213818in}}{\pgfqpoint{2.433119in}{3.224417in}}{\pgfqpoint{2.433119in}{3.235467in}}%
\pgfpathcurveto{\pgfqpoint{2.433119in}{3.246517in}}{\pgfqpoint{2.428728in}{3.257116in}}{\pgfqpoint{2.420915in}{3.264930in}}%
\pgfpathcurveto{\pgfqpoint{2.413101in}{3.272743in}}{\pgfqpoint{2.402502in}{3.277134in}}{\pgfqpoint{2.391452in}{3.277134in}}%
\pgfpathcurveto{\pgfqpoint{2.380402in}{3.277134in}}{\pgfqpoint{2.369803in}{3.272743in}}{\pgfqpoint{2.361989in}{3.264930in}}%
\pgfpathcurveto{\pgfqpoint{2.354176in}{3.257116in}}{\pgfqpoint{2.349785in}{3.246517in}}{\pgfqpoint{2.349785in}{3.235467in}}%
\pgfpathcurveto{\pgfqpoint{2.349785in}{3.224417in}}{\pgfqpoint{2.354176in}{3.213818in}}{\pgfqpoint{2.361989in}{3.206004in}}%
\pgfpathcurveto{\pgfqpoint{2.369803in}{3.198191in}}{\pgfqpoint{2.380402in}{3.193800in}}{\pgfqpoint{2.391452in}{3.193800in}}%
\pgfpathclose%
\pgfusepath{stroke,fill}%
\end{pgfscope}%
\begin{pgfscope}%
\pgfpathrectangle{\pgfqpoint{0.648703in}{0.548769in}}{\pgfqpoint{5.201297in}{3.102590in}}%
\pgfusepath{clip}%
\pgfsetbuttcap%
\pgfsetroundjoin%
\definecolor{currentfill}{rgb}{0.121569,0.466667,0.705882}%
\pgfsetfillcolor{currentfill}%
\pgfsetlinewidth{1.003750pt}%
\definecolor{currentstroke}{rgb}{0.121569,0.466667,0.705882}%
\pgfsetstrokecolor{currentstroke}%
\pgfsetdash{}{0pt}%
\pgfpathmoveto{\pgfqpoint{0.949444in}{0.648129in}}%
\pgfpathcurveto{\pgfqpoint{0.960494in}{0.648129in}}{\pgfqpoint{0.971093in}{0.652519in}}{\pgfqpoint{0.978906in}{0.660333in}}%
\pgfpathcurveto{\pgfqpoint{0.986720in}{0.668146in}}{\pgfqpoint{0.991110in}{0.678745in}}{\pgfqpoint{0.991110in}{0.689796in}}%
\pgfpathcurveto{\pgfqpoint{0.991110in}{0.700846in}}{\pgfqpoint{0.986720in}{0.711445in}}{\pgfqpoint{0.978906in}{0.719258in}}%
\pgfpathcurveto{\pgfqpoint{0.971093in}{0.727072in}}{\pgfqpoint{0.960494in}{0.731462in}}{\pgfqpoint{0.949444in}{0.731462in}}%
\pgfpathcurveto{\pgfqpoint{0.938393in}{0.731462in}}{\pgfqpoint{0.927794in}{0.727072in}}{\pgfqpoint{0.919981in}{0.719258in}}%
\pgfpathcurveto{\pgfqpoint{0.912167in}{0.711445in}}{\pgfqpoint{0.907777in}{0.700846in}}{\pgfqpoint{0.907777in}{0.689796in}}%
\pgfpathcurveto{\pgfqpoint{0.907777in}{0.678745in}}{\pgfqpoint{0.912167in}{0.668146in}}{\pgfqpoint{0.919981in}{0.660333in}}%
\pgfpathcurveto{\pgfqpoint{0.927794in}{0.652519in}}{\pgfqpoint{0.938393in}{0.648129in}}{\pgfqpoint{0.949444in}{0.648129in}}%
\pgfpathclose%
\pgfusepath{stroke,fill}%
\end{pgfscope}%
\begin{pgfscope}%
\pgfpathrectangle{\pgfqpoint{0.648703in}{0.548769in}}{\pgfqpoint{5.201297in}{3.102590in}}%
\pgfusepath{clip}%
\pgfsetbuttcap%
\pgfsetroundjoin%
\definecolor{currentfill}{rgb}{1.000000,0.498039,0.054902}%
\pgfsetfillcolor{currentfill}%
\pgfsetlinewidth{1.003750pt}%
\definecolor{currentstroke}{rgb}{1.000000,0.498039,0.054902}%
\pgfsetstrokecolor{currentstroke}%
\pgfsetdash{}{0pt}%
\pgfpathmoveto{\pgfqpoint{1.469969in}{3.198029in}}%
\pgfpathcurveto{\pgfqpoint{1.481019in}{3.198029in}}{\pgfqpoint{1.491618in}{3.202419in}}{\pgfqpoint{1.499432in}{3.210233in}}%
\pgfpathcurveto{\pgfqpoint{1.507245in}{3.218046in}}{\pgfqpoint{1.511636in}{3.228646in}}{\pgfqpoint{1.511636in}{3.239696in}}%
\pgfpathcurveto{\pgfqpoint{1.511636in}{3.250746in}}{\pgfqpoint{1.507245in}{3.261345in}}{\pgfqpoint{1.499432in}{3.269158in}}%
\pgfpathcurveto{\pgfqpoint{1.491618in}{3.276972in}}{\pgfqpoint{1.481019in}{3.281362in}}{\pgfqpoint{1.469969in}{3.281362in}}%
\pgfpathcurveto{\pgfqpoint{1.458919in}{3.281362in}}{\pgfqpoint{1.448320in}{3.276972in}}{\pgfqpoint{1.440506in}{3.269158in}}%
\pgfpathcurveto{\pgfqpoint{1.432692in}{3.261345in}}{\pgfqpoint{1.428302in}{3.250746in}}{\pgfqpoint{1.428302in}{3.239696in}}%
\pgfpathcurveto{\pgfqpoint{1.428302in}{3.228646in}}{\pgfqpoint{1.432692in}{3.218046in}}{\pgfqpoint{1.440506in}{3.210233in}}%
\pgfpathcurveto{\pgfqpoint{1.448320in}{3.202419in}}{\pgfqpoint{1.458919in}{3.198029in}}{\pgfqpoint{1.469969in}{3.198029in}}%
\pgfpathclose%
\pgfusepath{stroke,fill}%
\end{pgfscope}%
\begin{pgfscope}%
\pgfpathrectangle{\pgfqpoint{0.648703in}{0.548769in}}{\pgfqpoint{5.201297in}{3.102590in}}%
\pgfusepath{clip}%
\pgfsetbuttcap%
\pgfsetroundjoin%
\definecolor{currentfill}{rgb}{1.000000,0.498039,0.054902}%
\pgfsetfillcolor{currentfill}%
\pgfsetlinewidth{1.003750pt}%
\definecolor{currentstroke}{rgb}{1.000000,0.498039,0.054902}%
\pgfsetstrokecolor{currentstroke}%
\pgfsetdash{}{0pt}%
\pgfpathmoveto{\pgfqpoint{1.282363in}{3.193800in}}%
\pgfpathcurveto{\pgfqpoint{1.293413in}{3.193800in}}{\pgfqpoint{1.304012in}{3.198191in}}{\pgfqpoint{1.311826in}{3.206004in}}%
\pgfpathcurveto{\pgfqpoint{1.319640in}{3.213818in}}{\pgfqpoint{1.324030in}{3.224417in}}{\pgfqpoint{1.324030in}{3.235467in}}%
\pgfpathcurveto{\pgfqpoint{1.324030in}{3.246517in}}{\pgfqpoint{1.319640in}{3.257116in}}{\pgfqpoint{1.311826in}{3.264930in}}%
\pgfpathcurveto{\pgfqpoint{1.304012in}{3.272743in}}{\pgfqpoint{1.293413in}{3.277134in}}{\pgfqpoint{1.282363in}{3.277134in}}%
\pgfpathcurveto{\pgfqpoint{1.271313in}{3.277134in}}{\pgfqpoint{1.260714in}{3.272743in}}{\pgfqpoint{1.252901in}{3.264930in}}%
\pgfpathcurveto{\pgfqpoint{1.245087in}{3.257116in}}{\pgfqpoint{1.240697in}{3.246517in}}{\pgfqpoint{1.240697in}{3.235467in}}%
\pgfpathcurveto{\pgfqpoint{1.240697in}{3.224417in}}{\pgfqpoint{1.245087in}{3.213818in}}{\pgfqpoint{1.252901in}{3.206004in}}%
\pgfpathcurveto{\pgfqpoint{1.260714in}{3.198191in}}{\pgfqpoint{1.271313in}{3.193800in}}{\pgfqpoint{1.282363in}{3.193800in}}%
\pgfpathclose%
\pgfusepath{stroke,fill}%
\end{pgfscope}%
\begin{pgfscope}%
\pgfpathrectangle{\pgfqpoint{0.648703in}{0.548769in}}{\pgfqpoint{5.201297in}{3.102590in}}%
\pgfusepath{clip}%
\pgfsetbuttcap%
\pgfsetroundjoin%
\definecolor{currentfill}{rgb}{0.121569,0.466667,0.705882}%
\pgfsetfillcolor{currentfill}%
\pgfsetlinewidth{1.003750pt}%
\definecolor{currentstroke}{rgb}{0.121569,0.466667,0.705882}%
\pgfsetstrokecolor{currentstroke}%
\pgfsetdash{}{0pt}%
\pgfpathmoveto{\pgfqpoint{1.887685in}{0.652358in}}%
\pgfpathcurveto{\pgfqpoint{1.898735in}{0.652358in}}{\pgfqpoint{1.909334in}{0.656748in}}{\pgfqpoint{1.917148in}{0.664562in}}%
\pgfpathcurveto{\pgfqpoint{1.924961in}{0.672375in}}{\pgfqpoint{1.929352in}{0.682974in}}{\pgfqpoint{1.929352in}{0.694024in}}%
\pgfpathcurveto{\pgfqpoint{1.929352in}{0.705074in}}{\pgfqpoint{1.924961in}{0.715673in}}{\pgfqpoint{1.917148in}{0.723487in}}%
\pgfpathcurveto{\pgfqpoint{1.909334in}{0.731301in}}{\pgfqpoint{1.898735in}{0.735691in}}{\pgfqpoint{1.887685in}{0.735691in}}%
\pgfpathcurveto{\pgfqpoint{1.876635in}{0.735691in}}{\pgfqpoint{1.866036in}{0.731301in}}{\pgfqpoint{1.858222in}{0.723487in}}%
\pgfpathcurveto{\pgfqpoint{1.850408in}{0.715673in}}{\pgfqpoint{1.846018in}{0.705074in}}{\pgfqpoint{1.846018in}{0.694024in}}%
\pgfpathcurveto{\pgfqpoint{1.846018in}{0.682974in}}{\pgfqpoint{1.850408in}{0.672375in}}{\pgfqpoint{1.858222in}{0.664562in}}%
\pgfpathcurveto{\pgfqpoint{1.866036in}{0.656748in}}{\pgfqpoint{1.876635in}{0.652358in}}{\pgfqpoint{1.887685in}{0.652358in}}%
\pgfpathclose%
\pgfusepath{stroke,fill}%
\end{pgfscope}%
\begin{pgfscope}%
\pgfpathrectangle{\pgfqpoint{0.648703in}{0.548769in}}{\pgfqpoint{5.201297in}{3.102590in}}%
\pgfusepath{clip}%
\pgfsetbuttcap%
\pgfsetroundjoin%
\definecolor{currentfill}{rgb}{0.121569,0.466667,0.705882}%
\pgfsetfillcolor{currentfill}%
\pgfsetlinewidth{1.003750pt}%
\definecolor{currentstroke}{rgb}{0.121569,0.466667,0.705882}%
\pgfsetstrokecolor{currentstroke}%
\pgfsetdash{}{0pt}%
\pgfpathmoveto{\pgfqpoint{1.412855in}{0.648129in}}%
\pgfpathcurveto{\pgfqpoint{1.423905in}{0.648129in}}{\pgfqpoint{1.434504in}{0.652519in}}{\pgfqpoint{1.442318in}{0.660333in}}%
\pgfpathcurveto{\pgfqpoint{1.450131in}{0.668146in}}{\pgfqpoint{1.454521in}{0.678745in}}{\pgfqpoint{1.454521in}{0.689796in}}%
\pgfpathcurveto{\pgfqpoint{1.454521in}{0.700846in}}{\pgfqpoint{1.450131in}{0.711445in}}{\pgfqpoint{1.442318in}{0.719258in}}%
\pgfpathcurveto{\pgfqpoint{1.434504in}{0.727072in}}{\pgfqpoint{1.423905in}{0.731462in}}{\pgfqpoint{1.412855in}{0.731462in}}%
\pgfpathcurveto{\pgfqpoint{1.401805in}{0.731462in}}{\pgfqpoint{1.391206in}{0.727072in}}{\pgfqpoint{1.383392in}{0.719258in}}%
\pgfpathcurveto{\pgfqpoint{1.375578in}{0.711445in}}{\pgfqpoint{1.371188in}{0.700846in}}{\pgfqpoint{1.371188in}{0.689796in}}%
\pgfpathcurveto{\pgfqpoint{1.371188in}{0.678745in}}{\pgfqpoint{1.375578in}{0.668146in}}{\pgfqpoint{1.383392in}{0.660333in}}%
\pgfpathcurveto{\pgfqpoint{1.391206in}{0.652519in}}{\pgfqpoint{1.401805in}{0.648129in}}{\pgfqpoint{1.412855in}{0.648129in}}%
\pgfpathclose%
\pgfusepath{stroke,fill}%
\end{pgfscope}%
\begin{pgfscope}%
\pgfpathrectangle{\pgfqpoint{0.648703in}{0.548769in}}{\pgfqpoint{5.201297in}{3.102590in}}%
\pgfusepath{clip}%
\pgfsetbuttcap%
\pgfsetroundjoin%
\definecolor{currentfill}{rgb}{0.121569,0.466667,0.705882}%
\pgfsetfillcolor{currentfill}%
\pgfsetlinewidth{1.003750pt}%
\definecolor{currentstroke}{rgb}{0.121569,0.466667,0.705882}%
\pgfsetstrokecolor{currentstroke}%
\pgfsetdash{}{0pt}%
\pgfpathmoveto{\pgfqpoint{1.421994in}{0.648129in}}%
\pgfpathcurveto{\pgfqpoint{1.433044in}{0.648129in}}{\pgfqpoint{1.443643in}{0.652519in}}{\pgfqpoint{1.451457in}{0.660333in}}%
\pgfpathcurveto{\pgfqpoint{1.459270in}{0.668146in}}{\pgfqpoint{1.463661in}{0.678745in}}{\pgfqpoint{1.463661in}{0.689796in}}%
\pgfpathcurveto{\pgfqpoint{1.463661in}{0.700846in}}{\pgfqpoint{1.459270in}{0.711445in}}{\pgfqpoint{1.451457in}{0.719258in}}%
\pgfpathcurveto{\pgfqpoint{1.443643in}{0.727072in}}{\pgfqpoint{1.433044in}{0.731462in}}{\pgfqpoint{1.421994in}{0.731462in}}%
\pgfpathcurveto{\pgfqpoint{1.410944in}{0.731462in}}{\pgfqpoint{1.400345in}{0.727072in}}{\pgfqpoint{1.392531in}{0.719258in}}%
\pgfpathcurveto{\pgfqpoint{1.384717in}{0.711445in}}{\pgfqpoint{1.380327in}{0.700846in}}{\pgfqpoint{1.380327in}{0.689796in}}%
\pgfpathcurveto{\pgfqpoint{1.380327in}{0.678745in}}{\pgfqpoint{1.384717in}{0.668146in}}{\pgfqpoint{1.392531in}{0.660333in}}%
\pgfpathcurveto{\pgfqpoint{1.400345in}{0.652519in}}{\pgfqpoint{1.410944in}{0.648129in}}{\pgfqpoint{1.421994in}{0.648129in}}%
\pgfpathclose%
\pgfusepath{stroke,fill}%
\end{pgfscope}%
\begin{pgfscope}%
\pgfpathrectangle{\pgfqpoint{0.648703in}{0.548769in}}{\pgfqpoint{5.201297in}{3.102590in}}%
\pgfusepath{clip}%
\pgfsetbuttcap%
\pgfsetroundjoin%
\definecolor{currentfill}{rgb}{0.121569,0.466667,0.705882}%
\pgfsetfillcolor{currentfill}%
\pgfsetlinewidth{1.003750pt}%
\definecolor{currentstroke}{rgb}{0.121569,0.466667,0.705882}%
\pgfsetstrokecolor{currentstroke}%
\pgfsetdash{}{0pt}%
\pgfpathmoveto{\pgfqpoint{1.451893in}{0.648129in}}%
\pgfpathcurveto{\pgfqpoint{1.462943in}{0.648129in}}{\pgfqpoint{1.473542in}{0.652519in}}{\pgfqpoint{1.481355in}{0.660333in}}%
\pgfpathcurveto{\pgfqpoint{1.489169in}{0.668146in}}{\pgfqpoint{1.493559in}{0.678745in}}{\pgfqpoint{1.493559in}{0.689796in}}%
\pgfpathcurveto{\pgfqpoint{1.493559in}{0.700846in}}{\pgfqpoint{1.489169in}{0.711445in}}{\pgfqpoint{1.481355in}{0.719258in}}%
\pgfpathcurveto{\pgfqpoint{1.473542in}{0.727072in}}{\pgfqpoint{1.462943in}{0.731462in}}{\pgfqpoint{1.451893in}{0.731462in}}%
\pgfpathcurveto{\pgfqpoint{1.440843in}{0.731462in}}{\pgfqpoint{1.430243in}{0.727072in}}{\pgfqpoint{1.422430in}{0.719258in}}%
\pgfpathcurveto{\pgfqpoint{1.414616in}{0.711445in}}{\pgfqpoint{1.410226in}{0.700846in}}{\pgfqpoint{1.410226in}{0.689796in}}%
\pgfpathcurveto{\pgfqpoint{1.410226in}{0.678745in}}{\pgfqpoint{1.414616in}{0.668146in}}{\pgfqpoint{1.422430in}{0.660333in}}%
\pgfpathcurveto{\pgfqpoint{1.430243in}{0.652519in}}{\pgfqpoint{1.440843in}{0.648129in}}{\pgfqpoint{1.451893in}{0.648129in}}%
\pgfpathclose%
\pgfusepath{stroke,fill}%
\end{pgfscope}%
\begin{pgfscope}%
\pgfpathrectangle{\pgfqpoint{0.648703in}{0.548769in}}{\pgfqpoint{5.201297in}{3.102590in}}%
\pgfusepath{clip}%
\pgfsetbuttcap%
\pgfsetroundjoin%
\definecolor{currentfill}{rgb}{0.121569,0.466667,0.705882}%
\pgfsetfillcolor{currentfill}%
\pgfsetlinewidth{1.003750pt}%
\definecolor{currentstroke}{rgb}{0.121569,0.466667,0.705882}%
\pgfsetstrokecolor{currentstroke}%
\pgfsetdash{}{0pt}%
\pgfpathmoveto{\pgfqpoint{1.028932in}{0.648129in}}%
\pgfpathcurveto{\pgfqpoint{1.039982in}{0.648129in}}{\pgfqpoint{1.050581in}{0.652519in}}{\pgfqpoint{1.058395in}{0.660333in}}%
\pgfpathcurveto{\pgfqpoint{1.066209in}{0.668146in}}{\pgfqpoint{1.070599in}{0.678745in}}{\pgfqpoint{1.070599in}{0.689796in}}%
\pgfpathcurveto{\pgfqpoint{1.070599in}{0.700846in}}{\pgfqpoint{1.066209in}{0.711445in}}{\pgfqpoint{1.058395in}{0.719258in}}%
\pgfpathcurveto{\pgfqpoint{1.050581in}{0.727072in}}{\pgfqpoint{1.039982in}{0.731462in}}{\pgfqpoint{1.028932in}{0.731462in}}%
\pgfpathcurveto{\pgfqpoint{1.017882in}{0.731462in}}{\pgfqpoint{1.007283in}{0.727072in}}{\pgfqpoint{0.999469in}{0.719258in}}%
\pgfpathcurveto{\pgfqpoint{0.991656in}{0.711445in}}{\pgfqpoint{0.987266in}{0.700846in}}{\pgfqpoint{0.987266in}{0.689796in}}%
\pgfpathcurveto{\pgfqpoint{0.987266in}{0.678745in}}{\pgfqpoint{0.991656in}{0.668146in}}{\pgfqpoint{0.999469in}{0.660333in}}%
\pgfpathcurveto{\pgfqpoint{1.007283in}{0.652519in}}{\pgfqpoint{1.017882in}{0.648129in}}{\pgfqpoint{1.028932in}{0.648129in}}%
\pgfpathclose%
\pgfusepath{stroke,fill}%
\end{pgfscope}%
\begin{pgfscope}%
\pgfpathrectangle{\pgfqpoint{0.648703in}{0.548769in}}{\pgfqpoint{5.201297in}{3.102590in}}%
\pgfusepath{clip}%
\pgfsetbuttcap%
\pgfsetroundjoin%
\definecolor{currentfill}{rgb}{0.121569,0.466667,0.705882}%
\pgfsetfillcolor{currentfill}%
\pgfsetlinewidth{1.003750pt}%
\definecolor{currentstroke}{rgb}{0.121569,0.466667,0.705882}%
\pgfsetstrokecolor{currentstroke}%
\pgfsetdash{}{0pt}%
\pgfpathmoveto{\pgfqpoint{1.119175in}{0.648129in}}%
\pgfpathcurveto{\pgfqpoint{1.130225in}{0.648129in}}{\pgfqpoint{1.140824in}{0.652519in}}{\pgfqpoint{1.148638in}{0.660333in}}%
\pgfpathcurveto{\pgfqpoint{1.156451in}{0.668146in}}{\pgfqpoint{1.160841in}{0.678745in}}{\pgfqpoint{1.160841in}{0.689796in}}%
\pgfpathcurveto{\pgfqpoint{1.160841in}{0.700846in}}{\pgfqpoint{1.156451in}{0.711445in}}{\pgfqpoint{1.148638in}{0.719258in}}%
\pgfpathcurveto{\pgfqpoint{1.140824in}{0.727072in}}{\pgfqpoint{1.130225in}{0.731462in}}{\pgfqpoint{1.119175in}{0.731462in}}%
\pgfpathcurveto{\pgfqpoint{1.108125in}{0.731462in}}{\pgfqpoint{1.097526in}{0.727072in}}{\pgfqpoint{1.089712in}{0.719258in}}%
\pgfpathcurveto{\pgfqpoint{1.081898in}{0.711445in}}{\pgfqpoint{1.077508in}{0.700846in}}{\pgfqpoint{1.077508in}{0.689796in}}%
\pgfpathcurveto{\pgfqpoint{1.077508in}{0.678745in}}{\pgfqpoint{1.081898in}{0.668146in}}{\pgfqpoint{1.089712in}{0.660333in}}%
\pgfpathcurveto{\pgfqpoint{1.097526in}{0.652519in}}{\pgfqpoint{1.108125in}{0.648129in}}{\pgfqpoint{1.119175in}{0.648129in}}%
\pgfpathclose%
\pgfusepath{stroke,fill}%
\end{pgfscope}%
\begin{pgfscope}%
\pgfpathrectangle{\pgfqpoint{0.648703in}{0.548769in}}{\pgfqpoint{5.201297in}{3.102590in}}%
\pgfusepath{clip}%
\pgfsetbuttcap%
\pgfsetroundjoin%
\definecolor{currentfill}{rgb}{0.121569,0.466667,0.705882}%
\pgfsetfillcolor{currentfill}%
\pgfsetlinewidth{1.003750pt}%
\definecolor{currentstroke}{rgb}{0.121569,0.466667,0.705882}%
\pgfsetstrokecolor{currentstroke}%
\pgfsetdash{}{0pt}%
\pgfpathmoveto{\pgfqpoint{2.757983in}{3.181114in}}%
\pgfpathcurveto{\pgfqpoint{2.769033in}{3.181114in}}{\pgfqpoint{2.779632in}{3.185504in}}{\pgfqpoint{2.787446in}{3.193318in}}%
\pgfpathcurveto{\pgfqpoint{2.795259in}{3.201132in}}{\pgfqpoint{2.799650in}{3.211731in}}{\pgfqpoint{2.799650in}{3.222781in}}%
\pgfpathcurveto{\pgfqpoint{2.799650in}{3.233831in}}{\pgfqpoint{2.795259in}{3.244430in}}{\pgfqpoint{2.787446in}{3.252244in}}%
\pgfpathcurveto{\pgfqpoint{2.779632in}{3.260057in}}{\pgfqpoint{2.769033in}{3.264448in}}{\pgfqpoint{2.757983in}{3.264448in}}%
\pgfpathcurveto{\pgfqpoint{2.746933in}{3.264448in}}{\pgfqpoint{2.736334in}{3.260057in}}{\pgfqpoint{2.728520in}{3.252244in}}%
\pgfpathcurveto{\pgfqpoint{2.720707in}{3.244430in}}{\pgfqpoint{2.716316in}{3.233831in}}{\pgfqpoint{2.716316in}{3.222781in}}%
\pgfpathcurveto{\pgfqpoint{2.716316in}{3.211731in}}{\pgfqpoint{2.720707in}{3.201132in}}{\pgfqpoint{2.728520in}{3.193318in}}%
\pgfpathcurveto{\pgfqpoint{2.736334in}{3.185504in}}{\pgfqpoint{2.746933in}{3.181114in}}{\pgfqpoint{2.757983in}{3.181114in}}%
\pgfpathclose%
\pgfusepath{stroke,fill}%
\end{pgfscope}%
\begin{pgfscope}%
\pgfpathrectangle{\pgfqpoint{0.648703in}{0.548769in}}{\pgfqpoint{5.201297in}{3.102590in}}%
\pgfusepath{clip}%
\pgfsetbuttcap%
\pgfsetroundjoin%
\definecolor{currentfill}{rgb}{1.000000,0.498039,0.054902}%
\pgfsetfillcolor{currentfill}%
\pgfsetlinewidth{1.003750pt}%
\definecolor{currentstroke}{rgb}{1.000000,0.498039,0.054902}%
\pgfsetstrokecolor{currentstroke}%
\pgfsetdash{}{0pt}%
\pgfpathmoveto{\pgfqpoint{2.253603in}{3.193800in}}%
\pgfpathcurveto{\pgfqpoint{2.264653in}{3.193800in}}{\pgfqpoint{2.275252in}{3.198191in}}{\pgfqpoint{2.283065in}{3.206004in}}%
\pgfpathcurveto{\pgfqpoint{2.290879in}{3.213818in}}{\pgfqpoint{2.295269in}{3.224417in}}{\pgfqpoint{2.295269in}{3.235467in}}%
\pgfpathcurveto{\pgfqpoint{2.295269in}{3.246517in}}{\pgfqpoint{2.290879in}{3.257116in}}{\pgfqpoint{2.283065in}{3.264930in}}%
\pgfpathcurveto{\pgfqpoint{2.275252in}{3.272743in}}{\pgfqpoint{2.264653in}{3.277134in}}{\pgfqpoint{2.253603in}{3.277134in}}%
\pgfpathcurveto{\pgfqpoint{2.242552in}{3.277134in}}{\pgfqpoint{2.231953in}{3.272743in}}{\pgfqpoint{2.224140in}{3.264930in}}%
\pgfpathcurveto{\pgfqpoint{2.216326in}{3.257116in}}{\pgfqpoint{2.211936in}{3.246517in}}{\pgfqpoint{2.211936in}{3.235467in}}%
\pgfpathcurveto{\pgfqpoint{2.211936in}{3.224417in}}{\pgfqpoint{2.216326in}{3.213818in}}{\pgfqpoint{2.224140in}{3.206004in}}%
\pgfpathcurveto{\pgfqpoint{2.231953in}{3.198191in}}{\pgfqpoint{2.242552in}{3.193800in}}{\pgfqpoint{2.253603in}{3.193800in}}%
\pgfpathclose%
\pgfusepath{stroke,fill}%
\end{pgfscope}%
\begin{pgfscope}%
\pgfpathrectangle{\pgfqpoint{0.648703in}{0.548769in}}{\pgfqpoint{5.201297in}{3.102590in}}%
\pgfusepath{clip}%
\pgfsetbuttcap%
\pgfsetroundjoin%
\definecolor{currentfill}{rgb}{1.000000,0.498039,0.054902}%
\pgfsetfillcolor{currentfill}%
\pgfsetlinewidth{1.003750pt}%
\definecolor{currentstroke}{rgb}{1.000000,0.498039,0.054902}%
\pgfsetstrokecolor{currentstroke}%
\pgfsetdash{}{0pt}%
\pgfpathmoveto{\pgfqpoint{1.672266in}{3.193800in}}%
\pgfpathcurveto{\pgfqpoint{1.683317in}{3.193800in}}{\pgfqpoint{1.693916in}{3.198191in}}{\pgfqpoint{1.701729in}{3.206004in}}%
\pgfpathcurveto{\pgfqpoint{1.709543in}{3.213818in}}{\pgfqpoint{1.713933in}{3.224417in}}{\pgfqpoint{1.713933in}{3.235467in}}%
\pgfpathcurveto{\pgfqpoint{1.713933in}{3.246517in}}{\pgfqpoint{1.709543in}{3.257116in}}{\pgfqpoint{1.701729in}{3.264930in}}%
\pgfpathcurveto{\pgfqpoint{1.693916in}{3.272743in}}{\pgfqpoint{1.683317in}{3.277134in}}{\pgfqpoint{1.672266in}{3.277134in}}%
\pgfpathcurveto{\pgfqpoint{1.661216in}{3.277134in}}{\pgfqpoint{1.650617in}{3.272743in}}{\pgfqpoint{1.642804in}{3.264930in}}%
\pgfpathcurveto{\pgfqpoint{1.634990in}{3.257116in}}{\pgfqpoint{1.630600in}{3.246517in}}{\pgfqpoint{1.630600in}{3.235467in}}%
\pgfpathcurveto{\pgfqpoint{1.630600in}{3.224417in}}{\pgfqpoint{1.634990in}{3.213818in}}{\pgfqpoint{1.642804in}{3.206004in}}%
\pgfpathcurveto{\pgfqpoint{1.650617in}{3.198191in}}{\pgfqpoint{1.661216in}{3.193800in}}{\pgfqpoint{1.672266in}{3.193800in}}%
\pgfpathclose%
\pgfusepath{stroke,fill}%
\end{pgfscope}%
\begin{pgfscope}%
\pgfpathrectangle{\pgfqpoint{0.648703in}{0.548769in}}{\pgfqpoint{5.201297in}{3.102590in}}%
\pgfusepath{clip}%
\pgfsetbuttcap%
\pgfsetroundjoin%
\definecolor{currentfill}{rgb}{0.121569,0.466667,0.705882}%
\pgfsetfillcolor{currentfill}%
\pgfsetlinewidth{1.003750pt}%
\definecolor{currentstroke}{rgb}{0.121569,0.466667,0.705882}%
\pgfsetstrokecolor{currentstroke}%
\pgfsetdash{}{0pt}%
\pgfpathmoveto{\pgfqpoint{1.640242in}{0.648129in}}%
\pgfpathcurveto{\pgfqpoint{1.651292in}{0.648129in}}{\pgfqpoint{1.661891in}{0.652519in}}{\pgfqpoint{1.669705in}{0.660333in}}%
\pgfpathcurveto{\pgfqpoint{1.677519in}{0.668146in}}{\pgfqpoint{1.681909in}{0.678745in}}{\pgfqpoint{1.681909in}{0.689796in}}%
\pgfpathcurveto{\pgfqpoint{1.681909in}{0.700846in}}{\pgfqpoint{1.677519in}{0.711445in}}{\pgfqpoint{1.669705in}{0.719258in}}%
\pgfpathcurveto{\pgfqpoint{1.661891in}{0.727072in}}{\pgfqpoint{1.651292in}{0.731462in}}{\pgfqpoint{1.640242in}{0.731462in}}%
\pgfpathcurveto{\pgfqpoint{1.629192in}{0.731462in}}{\pgfqpoint{1.618593in}{0.727072in}}{\pgfqpoint{1.610779in}{0.719258in}}%
\pgfpathcurveto{\pgfqpoint{1.602966in}{0.711445in}}{\pgfqpoint{1.598576in}{0.700846in}}{\pgfqpoint{1.598576in}{0.689796in}}%
\pgfpathcurveto{\pgfqpoint{1.598576in}{0.678745in}}{\pgfqpoint{1.602966in}{0.668146in}}{\pgfqpoint{1.610779in}{0.660333in}}%
\pgfpathcurveto{\pgfqpoint{1.618593in}{0.652519in}}{\pgfqpoint{1.629192in}{0.648129in}}{\pgfqpoint{1.640242in}{0.648129in}}%
\pgfpathclose%
\pgfusepath{stroke,fill}%
\end{pgfscope}%
\begin{pgfscope}%
\pgfpathrectangle{\pgfqpoint{0.648703in}{0.548769in}}{\pgfqpoint{5.201297in}{3.102590in}}%
\pgfusepath{clip}%
\pgfsetbuttcap%
\pgfsetroundjoin%
\definecolor{currentfill}{rgb}{0.121569,0.466667,0.705882}%
\pgfsetfillcolor{currentfill}%
\pgfsetlinewidth{1.003750pt}%
\definecolor{currentstroke}{rgb}{0.121569,0.466667,0.705882}%
\pgfsetstrokecolor{currentstroke}%
\pgfsetdash{}{0pt}%
\pgfpathmoveto{\pgfqpoint{1.082813in}{0.665044in}}%
\pgfpathcurveto{\pgfqpoint{1.093863in}{0.665044in}}{\pgfqpoint{1.104462in}{0.669434in}}{\pgfqpoint{1.112275in}{0.677248in}}%
\pgfpathcurveto{\pgfqpoint{1.120089in}{0.685061in}}{\pgfqpoint{1.124479in}{0.695660in}}{\pgfqpoint{1.124479in}{0.706710in}}%
\pgfpathcurveto{\pgfqpoint{1.124479in}{0.717760in}}{\pgfqpoint{1.120089in}{0.728360in}}{\pgfqpoint{1.112275in}{0.736173in}}%
\pgfpathcurveto{\pgfqpoint{1.104462in}{0.743987in}}{\pgfqpoint{1.093863in}{0.748377in}}{\pgfqpoint{1.082813in}{0.748377in}}%
\pgfpathcurveto{\pgfqpoint{1.071762in}{0.748377in}}{\pgfqpoint{1.061163in}{0.743987in}}{\pgfqpoint{1.053350in}{0.736173in}}%
\pgfpathcurveto{\pgfqpoint{1.045536in}{0.728360in}}{\pgfqpoint{1.041146in}{0.717760in}}{\pgfqpoint{1.041146in}{0.706710in}}%
\pgfpathcurveto{\pgfqpoint{1.041146in}{0.695660in}}{\pgfqpoint{1.045536in}{0.685061in}}{\pgfqpoint{1.053350in}{0.677248in}}%
\pgfpathcurveto{\pgfqpoint{1.061163in}{0.669434in}}{\pgfqpoint{1.071762in}{0.665044in}}{\pgfqpoint{1.082813in}{0.665044in}}%
\pgfpathclose%
\pgfusepath{stroke,fill}%
\end{pgfscope}%
\begin{pgfscope}%
\pgfpathrectangle{\pgfqpoint{0.648703in}{0.548769in}}{\pgfqpoint{5.201297in}{3.102590in}}%
\pgfusepath{clip}%
\pgfsetbuttcap%
\pgfsetroundjoin%
\definecolor{currentfill}{rgb}{1.000000,0.498039,0.054902}%
\pgfsetfillcolor{currentfill}%
\pgfsetlinewidth{1.003750pt}%
\definecolor{currentstroke}{rgb}{1.000000,0.498039,0.054902}%
\pgfsetstrokecolor{currentstroke}%
\pgfsetdash{}{0pt}%
\pgfpathmoveto{\pgfqpoint{1.371502in}{3.189572in}}%
\pgfpathcurveto{\pgfqpoint{1.382552in}{3.189572in}}{\pgfqpoint{1.393151in}{3.193962in}}{\pgfqpoint{1.400964in}{3.201775in}}%
\pgfpathcurveto{\pgfqpoint{1.408778in}{3.209589in}}{\pgfqpoint{1.413168in}{3.220188in}}{\pgfqpoint{1.413168in}{3.231238in}}%
\pgfpathcurveto{\pgfqpoint{1.413168in}{3.242288in}}{\pgfqpoint{1.408778in}{3.252887in}}{\pgfqpoint{1.400964in}{3.260701in}}%
\pgfpathcurveto{\pgfqpoint{1.393151in}{3.268515in}}{\pgfqpoint{1.382552in}{3.272905in}}{\pgfqpoint{1.371502in}{3.272905in}}%
\pgfpathcurveto{\pgfqpoint{1.360451in}{3.272905in}}{\pgfqpoint{1.349852in}{3.268515in}}{\pgfqpoint{1.342039in}{3.260701in}}%
\pgfpathcurveto{\pgfqpoint{1.334225in}{3.252887in}}{\pgfqpoint{1.329835in}{3.242288in}}{\pgfqpoint{1.329835in}{3.231238in}}%
\pgfpathcurveto{\pgfqpoint{1.329835in}{3.220188in}}{\pgfqpoint{1.334225in}{3.209589in}}{\pgfqpoint{1.342039in}{3.201775in}}%
\pgfpathcurveto{\pgfqpoint{1.349852in}{3.193962in}}{\pgfqpoint{1.360451in}{3.189572in}}{\pgfqpoint{1.371502in}{3.189572in}}%
\pgfpathclose%
\pgfusepath{stroke,fill}%
\end{pgfscope}%
\begin{pgfscope}%
\pgfpathrectangle{\pgfqpoint{0.648703in}{0.548769in}}{\pgfqpoint{5.201297in}{3.102590in}}%
\pgfusepath{clip}%
\pgfsetbuttcap%
\pgfsetroundjoin%
\definecolor{currentfill}{rgb}{0.121569,0.466667,0.705882}%
\pgfsetfillcolor{currentfill}%
\pgfsetlinewidth{1.003750pt}%
\definecolor{currentstroke}{rgb}{0.121569,0.466667,0.705882}%
\pgfsetstrokecolor{currentstroke}%
\pgfsetdash{}{0pt}%
\pgfpathmoveto{\pgfqpoint{0.935717in}{0.808819in}}%
\pgfpathcurveto{\pgfqpoint{0.946767in}{0.808819in}}{\pgfqpoint{0.957366in}{0.813209in}}{\pgfqpoint{0.965180in}{0.821023in}}%
\pgfpathcurveto{\pgfqpoint{0.972994in}{0.828837in}}{\pgfqpoint{0.977384in}{0.839436in}}{\pgfqpoint{0.977384in}{0.850486in}}%
\pgfpathcurveto{\pgfqpoint{0.977384in}{0.861536in}}{\pgfqpoint{0.972994in}{0.872135in}}{\pgfqpoint{0.965180in}{0.879949in}}%
\pgfpathcurveto{\pgfqpoint{0.957366in}{0.887762in}}{\pgfqpoint{0.946767in}{0.892152in}}{\pgfqpoint{0.935717in}{0.892152in}}%
\pgfpathcurveto{\pgfqpoint{0.924667in}{0.892152in}}{\pgfqpoint{0.914068in}{0.887762in}}{\pgfqpoint{0.906254in}{0.879949in}}%
\pgfpathcurveto{\pgfqpoint{0.898441in}{0.872135in}}{\pgfqpoint{0.894051in}{0.861536in}}{\pgfqpoint{0.894051in}{0.850486in}}%
\pgfpathcurveto{\pgfqpoint{0.894051in}{0.839436in}}{\pgfqpoint{0.898441in}{0.828837in}}{\pgfqpoint{0.906254in}{0.821023in}}%
\pgfpathcurveto{\pgfqpoint{0.914068in}{0.813209in}}{\pgfqpoint{0.924667in}{0.808819in}}{\pgfqpoint{0.935717in}{0.808819in}}%
\pgfpathclose%
\pgfusepath{stroke,fill}%
\end{pgfscope}%
\begin{pgfscope}%
\pgfpathrectangle{\pgfqpoint{0.648703in}{0.548769in}}{\pgfqpoint{5.201297in}{3.102590in}}%
\pgfusepath{clip}%
\pgfsetbuttcap%
\pgfsetroundjoin%
\definecolor{currentfill}{rgb}{0.839216,0.152941,0.156863}%
\pgfsetfillcolor{currentfill}%
\pgfsetlinewidth{1.003750pt}%
\definecolor{currentstroke}{rgb}{0.839216,0.152941,0.156863}%
\pgfsetstrokecolor{currentstroke}%
\pgfsetdash{}{0pt}%
\pgfpathmoveto{\pgfqpoint{1.466466in}{3.198029in}}%
\pgfpathcurveto{\pgfqpoint{1.477516in}{3.198029in}}{\pgfqpoint{1.488115in}{3.202419in}}{\pgfqpoint{1.495929in}{3.210233in}}%
\pgfpathcurveto{\pgfqpoint{1.503742in}{3.218046in}}{\pgfqpoint{1.508133in}{3.228646in}}{\pgfqpoint{1.508133in}{3.239696in}}%
\pgfpathcurveto{\pgfqpoint{1.508133in}{3.250746in}}{\pgfqpoint{1.503742in}{3.261345in}}{\pgfqpoint{1.495929in}{3.269158in}}%
\pgfpathcurveto{\pgfqpoint{1.488115in}{3.276972in}}{\pgfqpoint{1.477516in}{3.281362in}}{\pgfqpoint{1.466466in}{3.281362in}}%
\pgfpathcurveto{\pgfqpoint{1.455416in}{3.281362in}}{\pgfqpoint{1.444817in}{3.276972in}}{\pgfqpoint{1.437003in}{3.269158in}}%
\pgfpathcurveto{\pgfqpoint{1.429190in}{3.261345in}}{\pgfqpoint{1.424799in}{3.250746in}}{\pgfqpoint{1.424799in}{3.239696in}}%
\pgfpathcurveto{\pgfqpoint{1.424799in}{3.228646in}}{\pgfqpoint{1.429190in}{3.218046in}}{\pgfqpoint{1.437003in}{3.210233in}}%
\pgfpathcurveto{\pgfqpoint{1.444817in}{3.202419in}}{\pgfqpoint{1.455416in}{3.198029in}}{\pgfqpoint{1.466466in}{3.198029in}}%
\pgfpathclose%
\pgfusepath{stroke,fill}%
\end{pgfscope}%
\begin{pgfscope}%
\pgfpathrectangle{\pgfqpoint{0.648703in}{0.548769in}}{\pgfqpoint{5.201297in}{3.102590in}}%
\pgfusepath{clip}%
\pgfsetbuttcap%
\pgfsetroundjoin%
\definecolor{currentfill}{rgb}{0.121569,0.466667,0.705882}%
\pgfsetfillcolor{currentfill}%
\pgfsetlinewidth{1.003750pt}%
\definecolor{currentstroke}{rgb}{0.121569,0.466667,0.705882}%
\pgfsetstrokecolor{currentstroke}%
\pgfsetdash{}{0pt}%
\pgfpathmoveto{\pgfqpoint{0.965351in}{0.648129in}}%
\pgfpathcurveto{\pgfqpoint{0.976401in}{0.648129in}}{\pgfqpoint{0.987000in}{0.652519in}}{\pgfqpoint{0.994814in}{0.660333in}}%
\pgfpathcurveto{\pgfqpoint{1.002627in}{0.668146in}}{\pgfqpoint{1.007017in}{0.678745in}}{\pgfqpoint{1.007017in}{0.689796in}}%
\pgfpathcurveto{\pgfqpoint{1.007017in}{0.700846in}}{\pgfqpoint{1.002627in}{0.711445in}}{\pgfqpoint{0.994814in}{0.719258in}}%
\pgfpathcurveto{\pgfqpoint{0.987000in}{0.727072in}}{\pgfqpoint{0.976401in}{0.731462in}}{\pgfqpoint{0.965351in}{0.731462in}}%
\pgfpathcurveto{\pgfqpoint{0.954301in}{0.731462in}}{\pgfqpoint{0.943702in}{0.727072in}}{\pgfqpoint{0.935888in}{0.719258in}}%
\pgfpathcurveto{\pgfqpoint{0.928074in}{0.711445in}}{\pgfqpoint{0.923684in}{0.700846in}}{\pgfqpoint{0.923684in}{0.689796in}}%
\pgfpathcurveto{\pgfqpoint{0.923684in}{0.678745in}}{\pgfqpoint{0.928074in}{0.668146in}}{\pgfqpoint{0.935888in}{0.660333in}}%
\pgfpathcurveto{\pgfqpoint{0.943702in}{0.652519in}}{\pgfqpoint{0.954301in}{0.648129in}}{\pgfqpoint{0.965351in}{0.648129in}}%
\pgfpathclose%
\pgfusepath{stroke,fill}%
\end{pgfscope}%
\begin{pgfscope}%
\pgfpathrectangle{\pgfqpoint{0.648703in}{0.548769in}}{\pgfqpoint{5.201297in}{3.102590in}}%
\pgfusepath{clip}%
\pgfsetbuttcap%
\pgfsetroundjoin%
\definecolor{currentfill}{rgb}{1.000000,0.498039,0.054902}%
\pgfsetfillcolor{currentfill}%
\pgfsetlinewidth{1.003750pt}%
\definecolor{currentstroke}{rgb}{1.000000,0.498039,0.054902}%
\pgfsetstrokecolor{currentstroke}%
\pgfsetdash{}{0pt}%
\pgfpathmoveto{\pgfqpoint{1.814727in}{3.185343in}}%
\pgfpathcurveto{\pgfqpoint{1.825777in}{3.185343in}}{\pgfqpoint{1.836376in}{3.189733in}}{\pgfqpoint{1.844190in}{3.197547in}}%
\pgfpathcurveto{\pgfqpoint{1.852003in}{3.205360in}}{\pgfqpoint{1.856394in}{3.215959in}}{\pgfqpoint{1.856394in}{3.227010in}}%
\pgfpathcurveto{\pgfqpoint{1.856394in}{3.238060in}}{\pgfqpoint{1.852003in}{3.248659in}}{\pgfqpoint{1.844190in}{3.256472in}}%
\pgfpathcurveto{\pgfqpoint{1.836376in}{3.264286in}}{\pgfqpoint{1.825777in}{3.268676in}}{\pgfqpoint{1.814727in}{3.268676in}}%
\pgfpathcurveto{\pgfqpoint{1.803677in}{3.268676in}}{\pgfqpoint{1.793078in}{3.264286in}}{\pgfqpoint{1.785264in}{3.256472in}}%
\pgfpathcurveto{\pgfqpoint{1.777451in}{3.248659in}}{\pgfqpoint{1.773060in}{3.238060in}}{\pgfqpoint{1.773060in}{3.227010in}}%
\pgfpathcurveto{\pgfqpoint{1.773060in}{3.215959in}}{\pgfqpoint{1.777451in}{3.205360in}}{\pgfqpoint{1.785264in}{3.197547in}}%
\pgfpathcurveto{\pgfqpoint{1.793078in}{3.189733in}}{\pgfqpoint{1.803677in}{3.185343in}}{\pgfqpoint{1.814727in}{3.185343in}}%
\pgfpathclose%
\pgfusepath{stroke,fill}%
\end{pgfscope}%
\begin{pgfscope}%
\pgfpathrectangle{\pgfqpoint{0.648703in}{0.548769in}}{\pgfqpoint{5.201297in}{3.102590in}}%
\pgfusepath{clip}%
\pgfsetbuttcap%
\pgfsetroundjoin%
\definecolor{currentfill}{rgb}{0.121569,0.466667,0.705882}%
\pgfsetfillcolor{currentfill}%
\pgfsetlinewidth{1.003750pt}%
\definecolor{currentstroke}{rgb}{0.121569,0.466667,0.705882}%
\pgfsetstrokecolor{currentstroke}%
\pgfsetdash{}{0pt}%
\pgfpathmoveto{\pgfqpoint{2.320881in}{0.648129in}}%
\pgfpathcurveto{\pgfqpoint{2.331931in}{0.648129in}}{\pgfqpoint{2.342530in}{0.652519in}}{\pgfqpoint{2.350344in}{0.660333in}}%
\pgfpathcurveto{\pgfqpoint{2.358157in}{0.668146in}}{\pgfqpoint{2.362547in}{0.678745in}}{\pgfqpoint{2.362547in}{0.689796in}}%
\pgfpathcurveto{\pgfqpoint{2.362547in}{0.700846in}}{\pgfqpoint{2.358157in}{0.711445in}}{\pgfqpoint{2.350344in}{0.719258in}}%
\pgfpathcurveto{\pgfqpoint{2.342530in}{0.727072in}}{\pgfqpoint{2.331931in}{0.731462in}}{\pgfqpoint{2.320881in}{0.731462in}}%
\pgfpathcurveto{\pgfqpoint{2.309831in}{0.731462in}}{\pgfqpoint{2.299232in}{0.727072in}}{\pgfqpoint{2.291418in}{0.719258in}}%
\pgfpathcurveto{\pgfqpoint{2.283604in}{0.711445in}}{\pgfqpoint{2.279214in}{0.700846in}}{\pgfqpoint{2.279214in}{0.689796in}}%
\pgfpathcurveto{\pgfqpoint{2.279214in}{0.678745in}}{\pgfqpoint{2.283604in}{0.668146in}}{\pgfqpoint{2.291418in}{0.660333in}}%
\pgfpathcurveto{\pgfqpoint{2.299232in}{0.652519in}}{\pgfqpoint{2.309831in}{0.648129in}}{\pgfqpoint{2.320881in}{0.648129in}}%
\pgfpathclose%
\pgfusepath{stroke,fill}%
\end{pgfscope}%
\begin{pgfscope}%
\pgfpathrectangle{\pgfqpoint{0.648703in}{0.548769in}}{\pgfqpoint{5.201297in}{3.102590in}}%
\pgfusepath{clip}%
\pgfsetbuttcap%
\pgfsetroundjoin%
\definecolor{currentfill}{rgb}{1.000000,0.498039,0.054902}%
\pgfsetfillcolor{currentfill}%
\pgfsetlinewidth{1.003750pt}%
\definecolor{currentstroke}{rgb}{1.000000,0.498039,0.054902}%
\pgfsetstrokecolor{currentstroke}%
\pgfsetdash{}{0pt}%
\pgfpathmoveto{\pgfqpoint{1.707762in}{3.185343in}}%
\pgfpathcurveto{\pgfqpoint{1.718812in}{3.185343in}}{\pgfqpoint{1.729411in}{3.189733in}}{\pgfqpoint{1.737225in}{3.197547in}}%
\pgfpathcurveto{\pgfqpoint{1.745038in}{3.205360in}}{\pgfqpoint{1.749429in}{3.215959in}}{\pgfqpoint{1.749429in}{3.227010in}}%
\pgfpathcurveto{\pgfqpoint{1.749429in}{3.238060in}}{\pgfqpoint{1.745038in}{3.248659in}}{\pgfqpoint{1.737225in}{3.256472in}}%
\pgfpathcurveto{\pgfqpoint{1.729411in}{3.264286in}}{\pgfqpoint{1.718812in}{3.268676in}}{\pgfqpoint{1.707762in}{3.268676in}}%
\pgfpathcurveto{\pgfqpoint{1.696712in}{3.268676in}}{\pgfqpoint{1.686113in}{3.264286in}}{\pgfqpoint{1.678299in}{3.256472in}}%
\pgfpathcurveto{\pgfqpoint{1.670485in}{3.248659in}}{\pgfqpoint{1.666095in}{3.238060in}}{\pgfqpoint{1.666095in}{3.227010in}}%
\pgfpathcurveto{\pgfqpoint{1.666095in}{3.215959in}}{\pgfqpoint{1.670485in}{3.205360in}}{\pgfqpoint{1.678299in}{3.197547in}}%
\pgfpathcurveto{\pgfqpoint{1.686113in}{3.189733in}}{\pgfqpoint{1.696712in}{3.185343in}}{\pgfqpoint{1.707762in}{3.185343in}}%
\pgfpathclose%
\pgfusepath{stroke,fill}%
\end{pgfscope}%
\begin{pgfscope}%
\pgfpathrectangle{\pgfqpoint{0.648703in}{0.548769in}}{\pgfqpoint{5.201297in}{3.102590in}}%
\pgfusepath{clip}%
\pgfsetbuttcap%
\pgfsetroundjoin%
\definecolor{currentfill}{rgb}{0.121569,0.466667,0.705882}%
\pgfsetfillcolor{currentfill}%
\pgfsetlinewidth{1.003750pt}%
\definecolor{currentstroke}{rgb}{0.121569,0.466667,0.705882}%
\pgfsetstrokecolor{currentstroke}%
\pgfsetdash{}{0pt}%
\pgfpathmoveto{\pgfqpoint{1.489541in}{0.648129in}}%
\pgfpathcurveto{\pgfqpoint{1.500591in}{0.648129in}}{\pgfqpoint{1.511190in}{0.652519in}}{\pgfqpoint{1.519004in}{0.660333in}}%
\pgfpathcurveto{\pgfqpoint{1.526818in}{0.668146in}}{\pgfqpoint{1.531208in}{0.678745in}}{\pgfqpoint{1.531208in}{0.689796in}}%
\pgfpathcurveto{\pgfqpoint{1.531208in}{0.700846in}}{\pgfqpoint{1.526818in}{0.711445in}}{\pgfqpoint{1.519004in}{0.719258in}}%
\pgfpathcurveto{\pgfqpoint{1.511190in}{0.727072in}}{\pgfqpoint{1.500591in}{0.731462in}}{\pgfqpoint{1.489541in}{0.731462in}}%
\pgfpathcurveto{\pgfqpoint{1.478491in}{0.731462in}}{\pgfqpoint{1.467892in}{0.727072in}}{\pgfqpoint{1.460078in}{0.719258in}}%
\pgfpathcurveto{\pgfqpoint{1.452265in}{0.711445in}}{\pgfqpoint{1.447875in}{0.700846in}}{\pgfqpoint{1.447875in}{0.689796in}}%
\pgfpathcurveto{\pgfqpoint{1.447875in}{0.678745in}}{\pgfqpoint{1.452265in}{0.668146in}}{\pgfqpoint{1.460078in}{0.660333in}}%
\pgfpathcurveto{\pgfqpoint{1.467892in}{0.652519in}}{\pgfqpoint{1.478491in}{0.648129in}}{\pgfqpoint{1.489541in}{0.648129in}}%
\pgfpathclose%
\pgfusepath{stroke,fill}%
\end{pgfscope}%
\begin{pgfscope}%
\pgfpathrectangle{\pgfqpoint{0.648703in}{0.548769in}}{\pgfqpoint{5.201297in}{3.102590in}}%
\pgfusepath{clip}%
\pgfsetbuttcap%
\pgfsetroundjoin%
\definecolor{currentfill}{rgb}{1.000000,0.498039,0.054902}%
\pgfsetfillcolor{currentfill}%
\pgfsetlinewidth{1.003750pt}%
\definecolor{currentstroke}{rgb}{1.000000,0.498039,0.054902}%
\pgfsetstrokecolor{currentstroke}%
\pgfsetdash{}{0pt}%
\pgfpathmoveto{\pgfqpoint{1.191381in}{3.278374in}}%
\pgfpathcurveto{\pgfqpoint{1.202431in}{3.278374in}}{\pgfqpoint{1.213030in}{3.282764in}}{\pgfqpoint{1.220843in}{3.290578in}}%
\pgfpathcurveto{\pgfqpoint{1.228657in}{3.298392in}}{\pgfqpoint{1.233047in}{3.308991in}}{\pgfqpoint{1.233047in}{3.320041in}}%
\pgfpathcurveto{\pgfqpoint{1.233047in}{3.331091in}}{\pgfqpoint{1.228657in}{3.341690in}}{\pgfqpoint{1.220843in}{3.349504in}}%
\pgfpathcurveto{\pgfqpoint{1.213030in}{3.357317in}}{\pgfqpoint{1.202431in}{3.361707in}}{\pgfqpoint{1.191381in}{3.361707in}}%
\pgfpathcurveto{\pgfqpoint{1.180331in}{3.361707in}}{\pgfqpoint{1.169731in}{3.357317in}}{\pgfqpoint{1.161918in}{3.349504in}}%
\pgfpathcurveto{\pgfqpoint{1.154104in}{3.341690in}}{\pgfqpoint{1.149714in}{3.331091in}}{\pgfqpoint{1.149714in}{3.320041in}}%
\pgfpathcurveto{\pgfqpoint{1.149714in}{3.308991in}}{\pgfqpoint{1.154104in}{3.298392in}}{\pgfqpoint{1.161918in}{3.290578in}}%
\pgfpathcurveto{\pgfqpoint{1.169731in}{3.282764in}}{\pgfqpoint{1.180331in}{3.278374in}}{\pgfqpoint{1.191381in}{3.278374in}}%
\pgfpathclose%
\pgfusepath{stroke,fill}%
\end{pgfscope}%
\begin{pgfscope}%
\pgfpathrectangle{\pgfqpoint{0.648703in}{0.548769in}}{\pgfqpoint{5.201297in}{3.102590in}}%
\pgfusepath{clip}%
\pgfsetbuttcap%
\pgfsetroundjoin%
\definecolor{currentfill}{rgb}{0.121569,0.466667,0.705882}%
\pgfsetfillcolor{currentfill}%
\pgfsetlinewidth{1.003750pt}%
\definecolor{currentstroke}{rgb}{0.121569,0.466667,0.705882}%
\pgfsetstrokecolor{currentstroke}%
\pgfsetdash{}{0pt}%
\pgfpathmoveto{\pgfqpoint{1.445275in}{0.648129in}}%
\pgfpathcurveto{\pgfqpoint{1.456325in}{0.648129in}}{\pgfqpoint{1.466924in}{0.652519in}}{\pgfqpoint{1.474738in}{0.660333in}}%
\pgfpathcurveto{\pgfqpoint{1.482551in}{0.668146in}}{\pgfqpoint{1.486942in}{0.678745in}}{\pgfqpoint{1.486942in}{0.689796in}}%
\pgfpathcurveto{\pgfqpoint{1.486942in}{0.700846in}}{\pgfqpoint{1.482551in}{0.711445in}}{\pgfqpoint{1.474738in}{0.719258in}}%
\pgfpathcurveto{\pgfqpoint{1.466924in}{0.727072in}}{\pgfqpoint{1.456325in}{0.731462in}}{\pgfqpoint{1.445275in}{0.731462in}}%
\pgfpathcurveto{\pgfqpoint{1.434225in}{0.731462in}}{\pgfqpoint{1.423626in}{0.727072in}}{\pgfqpoint{1.415812in}{0.719258in}}%
\pgfpathcurveto{\pgfqpoint{1.407998in}{0.711445in}}{\pgfqpoint{1.403608in}{0.700846in}}{\pgfqpoint{1.403608in}{0.689796in}}%
\pgfpathcurveto{\pgfqpoint{1.403608in}{0.678745in}}{\pgfqpoint{1.407998in}{0.668146in}}{\pgfqpoint{1.415812in}{0.660333in}}%
\pgfpathcurveto{\pgfqpoint{1.423626in}{0.652519in}}{\pgfqpoint{1.434225in}{0.648129in}}{\pgfqpoint{1.445275in}{0.648129in}}%
\pgfpathclose%
\pgfusepath{stroke,fill}%
\end{pgfscope}%
\begin{pgfscope}%
\pgfpathrectangle{\pgfqpoint{0.648703in}{0.548769in}}{\pgfqpoint{5.201297in}{3.102590in}}%
\pgfusepath{clip}%
\pgfsetbuttcap%
\pgfsetroundjoin%
\definecolor{currentfill}{rgb}{1.000000,0.498039,0.054902}%
\pgfsetfillcolor{currentfill}%
\pgfsetlinewidth{1.003750pt}%
\definecolor{currentstroke}{rgb}{1.000000,0.498039,0.054902}%
\pgfsetstrokecolor{currentstroke}%
\pgfsetdash{}{0pt}%
\pgfpathmoveto{\pgfqpoint{1.065192in}{3.202258in}}%
\pgfpathcurveto{\pgfqpoint{1.076242in}{3.202258in}}{\pgfqpoint{1.086841in}{3.206648in}}{\pgfqpoint{1.094654in}{3.214462in}}%
\pgfpathcurveto{\pgfqpoint{1.102468in}{3.222275in}}{\pgfqpoint{1.106858in}{3.232874in}}{\pgfqpoint{1.106858in}{3.243924in}}%
\pgfpathcurveto{\pgfqpoint{1.106858in}{3.254974in}}{\pgfqpoint{1.102468in}{3.265573in}}{\pgfqpoint{1.094654in}{3.273387in}}%
\pgfpathcurveto{\pgfqpoint{1.086841in}{3.281201in}}{\pgfqpoint{1.076242in}{3.285591in}}{\pgfqpoint{1.065192in}{3.285591in}}%
\pgfpathcurveto{\pgfqpoint{1.054141in}{3.285591in}}{\pgfqpoint{1.043542in}{3.281201in}}{\pgfqpoint{1.035729in}{3.273387in}}%
\pgfpathcurveto{\pgfqpoint{1.027915in}{3.265573in}}{\pgfqpoint{1.023525in}{3.254974in}}{\pgfqpoint{1.023525in}{3.243924in}}%
\pgfpathcurveto{\pgfqpoint{1.023525in}{3.232874in}}{\pgfqpoint{1.027915in}{3.222275in}}{\pgfqpoint{1.035729in}{3.214462in}}%
\pgfpathcurveto{\pgfqpoint{1.043542in}{3.206648in}}{\pgfqpoint{1.054141in}{3.202258in}}{\pgfqpoint{1.065192in}{3.202258in}}%
\pgfpathclose%
\pgfusepath{stroke,fill}%
\end{pgfscope}%
\begin{pgfscope}%
\pgfpathrectangle{\pgfqpoint{0.648703in}{0.548769in}}{\pgfqpoint{5.201297in}{3.102590in}}%
\pgfusepath{clip}%
\pgfsetbuttcap%
\pgfsetroundjoin%
\definecolor{currentfill}{rgb}{1.000000,0.498039,0.054902}%
\pgfsetfillcolor{currentfill}%
\pgfsetlinewidth{1.003750pt}%
\definecolor{currentstroke}{rgb}{1.000000,0.498039,0.054902}%
\pgfsetstrokecolor{currentstroke}%
\pgfsetdash{}{0pt}%
\pgfpathmoveto{\pgfqpoint{1.137639in}{3.185343in}}%
\pgfpathcurveto{\pgfqpoint{1.148689in}{3.185343in}}{\pgfqpoint{1.159288in}{3.189733in}}{\pgfqpoint{1.167102in}{3.197547in}}%
\pgfpathcurveto{\pgfqpoint{1.174915in}{3.205360in}}{\pgfqpoint{1.179306in}{3.215959in}}{\pgfqpoint{1.179306in}{3.227010in}}%
\pgfpathcurveto{\pgfqpoint{1.179306in}{3.238060in}}{\pgfqpoint{1.174915in}{3.248659in}}{\pgfqpoint{1.167102in}{3.256472in}}%
\pgfpathcurveto{\pgfqpoint{1.159288in}{3.264286in}}{\pgfqpoint{1.148689in}{3.268676in}}{\pgfqpoint{1.137639in}{3.268676in}}%
\pgfpathcurveto{\pgfqpoint{1.126589in}{3.268676in}}{\pgfqpoint{1.115990in}{3.264286in}}{\pgfqpoint{1.108176in}{3.256472in}}%
\pgfpathcurveto{\pgfqpoint{1.100362in}{3.248659in}}{\pgfqpoint{1.095972in}{3.238060in}}{\pgfqpoint{1.095972in}{3.227010in}}%
\pgfpathcurveto{\pgfqpoint{1.095972in}{3.215959in}}{\pgfqpoint{1.100362in}{3.205360in}}{\pgfqpoint{1.108176in}{3.197547in}}%
\pgfpathcurveto{\pgfqpoint{1.115990in}{3.189733in}}{\pgfqpoint{1.126589in}{3.185343in}}{\pgfqpoint{1.137639in}{3.185343in}}%
\pgfpathclose%
\pgfusepath{stroke,fill}%
\end{pgfscope}%
\begin{pgfscope}%
\pgfpathrectangle{\pgfqpoint{0.648703in}{0.548769in}}{\pgfqpoint{5.201297in}{3.102590in}}%
\pgfusepath{clip}%
\pgfsetbuttcap%
\pgfsetroundjoin%
\definecolor{currentfill}{rgb}{0.121569,0.466667,0.705882}%
\pgfsetfillcolor{currentfill}%
\pgfsetlinewidth{1.003750pt}%
\definecolor{currentstroke}{rgb}{0.121569,0.466667,0.705882}%
\pgfsetstrokecolor{currentstroke}%
\pgfsetdash{}{0pt}%
\pgfpathmoveto{\pgfqpoint{1.368236in}{0.648129in}}%
\pgfpathcurveto{\pgfqpoint{1.379286in}{0.648129in}}{\pgfqpoint{1.389885in}{0.652519in}}{\pgfqpoint{1.397699in}{0.660333in}}%
\pgfpathcurveto{\pgfqpoint{1.405513in}{0.668146in}}{\pgfqpoint{1.409903in}{0.678745in}}{\pgfqpoint{1.409903in}{0.689796in}}%
\pgfpathcurveto{\pgfqpoint{1.409903in}{0.700846in}}{\pgfqpoint{1.405513in}{0.711445in}}{\pgfqpoint{1.397699in}{0.719258in}}%
\pgfpathcurveto{\pgfqpoint{1.389885in}{0.727072in}}{\pgfqpoint{1.379286in}{0.731462in}}{\pgfqpoint{1.368236in}{0.731462in}}%
\pgfpathcurveto{\pgfqpoint{1.357186in}{0.731462in}}{\pgfqpoint{1.346587in}{0.727072in}}{\pgfqpoint{1.338773in}{0.719258in}}%
\pgfpathcurveto{\pgfqpoint{1.330960in}{0.711445in}}{\pgfqpoint{1.326570in}{0.700846in}}{\pgfqpoint{1.326570in}{0.689796in}}%
\pgfpathcurveto{\pgfqpoint{1.326570in}{0.678745in}}{\pgfqpoint{1.330960in}{0.668146in}}{\pgfqpoint{1.338773in}{0.660333in}}%
\pgfpathcurveto{\pgfqpoint{1.346587in}{0.652519in}}{\pgfqpoint{1.357186in}{0.648129in}}{\pgfqpoint{1.368236in}{0.648129in}}%
\pgfpathclose%
\pgfusepath{stroke,fill}%
\end{pgfscope}%
\begin{pgfscope}%
\pgfpathrectangle{\pgfqpoint{0.648703in}{0.548769in}}{\pgfqpoint{5.201297in}{3.102590in}}%
\pgfusepath{clip}%
\pgfsetbuttcap%
\pgfsetroundjoin%
\definecolor{currentfill}{rgb}{1.000000,0.498039,0.054902}%
\pgfsetfillcolor{currentfill}%
\pgfsetlinewidth{1.003750pt}%
\definecolor{currentstroke}{rgb}{1.000000,0.498039,0.054902}%
\pgfsetstrokecolor{currentstroke}%
\pgfsetdash{}{0pt}%
\pgfpathmoveto{\pgfqpoint{1.088389in}{3.185343in}}%
\pgfpathcurveto{\pgfqpoint{1.099440in}{3.185343in}}{\pgfqpoint{1.110039in}{3.189733in}}{\pgfqpoint{1.117852in}{3.197547in}}%
\pgfpathcurveto{\pgfqpoint{1.125666in}{3.205360in}}{\pgfqpoint{1.130056in}{3.215959in}}{\pgfqpoint{1.130056in}{3.227010in}}%
\pgfpathcurveto{\pgfqpoint{1.130056in}{3.238060in}}{\pgfqpoint{1.125666in}{3.248659in}}{\pgfqpoint{1.117852in}{3.256472in}}%
\pgfpathcurveto{\pgfqpoint{1.110039in}{3.264286in}}{\pgfqpoint{1.099440in}{3.268676in}}{\pgfqpoint{1.088389in}{3.268676in}}%
\pgfpathcurveto{\pgfqpoint{1.077339in}{3.268676in}}{\pgfqpoint{1.066740in}{3.264286in}}{\pgfqpoint{1.058927in}{3.256472in}}%
\pgfpathcurveto{\pgfqpoint{1.051113in}{3.248659in}}{\pgfqpoint{1.046723in}{3.238060in}}{\pgfqpoint{1.046723in}{3.227010in}}%
\pgfpathcurveto{\pgfqpoint{1.046723in}{3.215959in}}{\pgfqpoint{1.051113in}{3.205360in}}{\pgfqpoint{1.058927in}{3.197547in}}%
\pgfpathcurveto{\pgfqpoint{1.066740in}{3.189733in}}{\pgfqpoint{1.077339in}{3.185343in}}{\pgfqpoint{1.088389in}{3.185343in}}%
\pgfpathclose%
\pgfusepath{stroke,fill}%
\end{pgfscope}%
\begin{pgfscope}%
\pgfpathrectangle{\pgfqpoint{0.648703in}{0.548769in}}{\pgfqpoint{5.201297in}{3.102590in}}%
\pgfusepath{clip}%
\pgfsetbuttcap%
\pgfsetroundjoin%
\definecolor{currentfill}{rgb}{1.000000,0.498039,0.054902}%
\pgfsetfillcolor{currentfill}%
\pgfsetlinewidth{1.003750pt}%
\definecolor{currentstroke}{rgb}{1.000000,0.498039,0.054902}%
\pgfsetstrokecolor{currentstroke}%
\pgfsetdash{}{0pt}%
\pgfpathmoveto{\pgfqpoint{0.941346in}{3.193800in}}%
\pgfpathcurveto{\pgfqpoint{0.952396in}{3.193800in}}{\pgfqpoint{0.962995in}{3.198191in}}{\pgfqpoint{0.970808in}{3.206004in}}%
\pgfpathcurveto{\pgfqpoint{0.978622in}{3.213818in}}{\pgfqpoint{0.983012in}{3.224417in}}{\pgfqpoint{0.983012in}{3.235467in}}%
\pgfpathcurveto{\pgfqpoint{0.983012in}{3.246517in}}{\pgfqpoint{0.978622in}{3.257116in}}{\pgfqpoint{0.970808in}{3.264930in}}%
\pgfpathcurveto{\pgfqpoint{0.962995in}{3.272743in}}{\pgfqpoint{0.952396in}{3.277134in}}{\pgfqpoint{0.941346in}{3.277134in}}%
\pgfpathcurveto{\pgfqpoint{0.930295in}{3.277134in}}{\pgfqpoint{0.919696in}{3.272743in}}{\pgfqpoint{0.911883in}{3.264930in}}%
\pgfpathcurveto{\pgfqpoint{0.904069in}{3.257116in}}{\pgfqpoint{0.899679in}{3.246517in}}{\pgfqpoint{0.899679in}{3.235467in}}%
\pgfpathcurveto{\pgfqpoint{0.899679in}{3.224417in}}{\pgfqpoint{0.904069in}{3.213818in}}{\pgfqpoint{0.911883in}{3.206004in}}%
\pgfpathcurveto{\pgfqpoint{0.919696in}{3.198191in}}{\pgfqpoint{0.930295in}{3.193800in}}{\pgfqpoint{0.941346in}{3.193800in}}%
\pgfpathclose%
\pgfusepath{stroke,fill}%
\end{pgfscope}%
\begin{pgfscope}%
\pgfpathrectangle{\pgfqpoint{0.648703in}{0.548769in}}{\pgfqpoint{5.201297in}{3.102590in}}%
\pgfusepath{clip}%
\pgfsetbuttcap%
\pgfsetroundjoin%
\definecolor{currentfill}{rgb}{0.121569,0.466667,0.705882}%
\pgfsetfillcolor{currentfill}%
\pgfsetlinewidth{1.003750pt}%
\definecolor{currentstroke}{rgb}{0.121569,0.466667,0.705882}%
\pgfsetstrokecolor{currentstroke}%
\pgfsetdash{}{0pt}%
\pgfpathmoveto{\pgfqpoint{2.154823in}{0.656586in}}%
\pgfpathcurveto{\pgfqpoint{2.165873in}{0.656586in}}{\pgfqpoint{2.176472in}{0.660977in}}{\pgfqpoint{2.184285in}{0.668790in}}%
\pgfpathcurveto{\pgfqpoint{2.192099in}{0.676604in}}{\pgfqpoint{2.196489in}{0.687203in}}{\pgfqpoint{2.196489in}{0.698253in}}%
\pgfpathcurveto{\pgfqpoint{2.196489in}{0.709303in}}{\pgfqpoint{2.192099in}{0.719902in}}{\pgfqpoint{2.184285in}{0.727716in}}%
\pgfpathcurveto{\pgfqpoint{2.176472in}{0.735529in}}{\pgfqpoint{2.165873in}{0.739920in}}{\pgfqpoint{2.154823in}{0.739920in}}%
\pgfpathcurveto{\pgfqpoint{2.143772in}{0.739920in}}{\pgfqpoint{2.133173in}{0.735529in}}{\pgfqpoint{2.125360in}{0.727716in}}%
\pgfpathcurveto{\pgfqpoint{2.117546in}{0.719902in}}{\pgfqpoint{2.113156in}{0.709303in}}{\pgfqpoint{2.113156in}{0.698253in}}%
\pgfpathcurveto{\pgfqpoint{2.113156in}{0.687203in}}{\pgfqpoint{2.117546in}{0.676604in}}{\pgfqpoint{2.125360in}{0.668790in}}%
\pgfpathcurveto{\pgfqpoint{2.133173in}{0.660977in}}{\pgfqpoint{2.143772in}{0.656586in}}{\pgfqpoint{2.154823in}{0.656586in}}%
\pgfpathclose%
\pgfusepath{stroke,fill}%
\end{pgfscope}%
\begin{pgfscope}%
\pgfpathrectangle{\pgfqpoint{0.648703in}{0.548769in}}{\pgfqpoint{5.201297in}{3.102590in}}%
\pgfusepath{clip}%
\pgfsetbuttcap%
\pgfsetroundjoin%
\definecolor{currentfill}{rgb}{1.000000,0.498039,0.054902}%
\pgfsetfillcolor{currentfill}%
\pgfsetlinewidth{1.003750pt}%
\definecolor{currentstroke}{rgb}{1.000000,0.498039,0.054902}%
\pgfsetstrokecolor{currentstroke}%
\pgfsetdash{}{0pt}%
\pgfpathmoveto{\pgfqpoint{1.931294in}{3.214944in}}%
\pgfpathcurveto{\pgfqpoint{1.942344in}{3.214944in}}{\pgfqpoint{1.952943in}{3.219334in}}{\pgfqpoint{1.960757in}{3.227148in}}%
\pgfpathcurveto{\pgfqpoint{1.968571in}{3.234961in}}{\pgfqpoint{1.972961in}{3.245560in}}{\pgfqpoint{1.972961in}{3.256610in}}%
\pgfpathcurveto{\pgfqpoint{1.972961in}{3.267661in}}{\pgfqpoint{1.968571in}{3.278260in}}{\pgfqpoint{1.960757in}{3.286073in}}%
\pgfpathcurveto{\pgfqpoint{1.952943in}{3.293887in}}{\pgfqpoint{1.942344in}{3.298277in}}{\pgfqpoint{1.931294in}{3.298277in}}%
\pgfpathcurveto{\pgfqpoint{1.920244in}{3.298277in}}{\pgfqpoint{1.909645in}{3.293887in}}{\pgfqpoint{1.901831in}{3.286073in}}%
\pgfpathcurveto{\pgfqpoint{1.894018in}{3.278260in}}{\pgfqpoint{1.889628in}{3.267661in}}{\pgfqpoint{1.889628in}{3.256610in}}%
\pgfpathcurveto{\pgfqpoint{1.889628in}{3.245560in}}{\pgfqpoint{1.894018in}{3.234961in}}{\pgfqpoint{1.901831in}{3.227148in}}%
\pgfpathcurveto{\pgfqpoint{1.909645in}{3.219334in}}{\pgfqpoint{1.920244in}{3.214944in}}{\pgfqpoint{1.931294in}{3.214944in}}%
\pgfpathclose%
\pgfusepath{stroke,fill}%
\end{pgfscope}%
\begin{pgfscope}%
\pgfpathrectangle{\pgfqpoint{0.648703in}{0.548769in}}{\pgfqpoint{5.201297in}{3.102590in}}%
\pgfusepath{clip}%
\pgfsetbuttcap%
\pgfsetroundjoin%
\definecolor{currentfill}{rgb}{1.000000,0.498039,0.054902}%
\pgfsetfillcolor{currentfill}%
\pgfsetlinewidth{1.003750pt}%
\definecolor{currentstroke}{rgb}{1.000000,0.498039,0.054902}%
\pgfsetstrokecolor{currentstroke}%
\pgfsetdash{}{0pt}%
\pgfpathmoveto{\pgfqpoint{1.378499in}{3.202258in}}%
\pgfpathcurveto{\pgfqpoint{1.389549in}{3.202258in}}{\pgfqpoint{1.400148in}{3.206648in}}{\pgfqpoint{1.407962in}{3.214462in}}%
\pgfpathcurveto{\pgfqpoint{1.415776in}{3.222275in}}{\pgfqpoint{1.420166in}{3.232874in}}{\pgfqpoint{1.420166in}{3.243924in}}%
\pgfpathcurveto{\pgfqpoint{1.420166in}{3.254974in}}{\pgfqpoint{1.415776in}{3.265573in}}{\pgfqpoint{1.407962in}{3.273387in}}%
\pgfpathcurveto{\pgfqpoint{1.400148in}{3.281201in}}{\pgfqpoint{1.389549in}{3.285591in}}{\pgfqpoint{1.378499in}{3.285591in}}%
\pgfpathcurveto{\pgfqpoint{1.367449in}{3.285591in}}{\pgfqpoint{1.356850in}{3.281201in}}{\pgfqpoint{1.349037in}{3.273387in}}%
\pgfpathcurveto{\pgfqpoint{1.341223in}{3.265573in}}{\pgfqpoint{1.336833in}{3.254974in}}{\pgfqpoint{1.336833in}{3.243924in}}%
\pgfpathcurveto{\pgfqpoint{1.336833in}{3.232874in}}{\pgfqpoint{1.341223in}{3.222275in}}{\pgfqpoint{1.349037in}{3.214462in}}%
\pgfpathcurveto{\pgfqpoint{1.356850in}{3.206648in}}{\pgfqpoint{1.367449in}{3.202258in}}{\pgfqpoint{1.378499in}{3.202258in}}%
\pgfpathclose%
\pgfusepath{stroke,fill}%
\end{pgfscope}%
\begin{pgfscope}%
\pgfpathrectangle{\pgfqpoint{0.648703in}{0.548769in}}{\pgfqpoint{5.201297in}{3.102590in}}%
\pgfusepath{clip}%
\pgfsetbuttcap%
\pgfsetroundjoin%
\definecolor{currentfill}{rgb}{1.000000,0.498039,0.054902}%
\pgfsetfillcolor{currentfill}%
\pgfsetlinewidth{1.003750pt}%
\definecolor{currentstroke}{rgb}{1.000000,0.498039,0.054902}%
\pgfsetstrokecolor{currentstroke}%
\pgfsetdash{}{0pt}%
\pgfpathmoveto{\pgfqpoint{1.487515in}{3.244545in}}%
\pgfpathcurveto{\pgfqpoint{1.498565in}{3.244545in}}{\pgfqpoint{1.509164in}{3.248935in}}{\pgfqpoint{1.516977in}{3.256748in}}%
\pgfpathcurveto{\pgfqpoint{1.524791in}{3.264562in}}{\pgfqpoint{1.529181in}{3.275161in}}{\pgfqpoint{1.529181in}{3.286211in}}%
\pgfpathcurveto{\pgfqpoint{1.529181in}{3.297261in}}{\pgfqpoint{1.524791in}{3.307860in}}{\pgfqpoint{1.516977in}{3.315674in}}%
\pgfpathcurveto{\pgfqpoint{1.509164in}{3.323488in}}{\pgfqpoint{1.498565in}{3.327878in}}{\pgfqpoint{1.487515in}{3.327878in}}%
\pgfpathcurveto{\pgfqpoint{1.476465in}{3.327878in}}{\pgfqpoint{1.465866in}{3.323488in}}{\pgfqpoint{1.458052in}{3.315674in}}%
\pgfpathcurveto{\pgfqpoint{1.450238in}{3.307860in}}{\pgfqpoint{1.445848in}{3.297261in}}{\pgfqpoint{1.445848in}{3.286211in}}%
\pgfpathcurveto{\pgfqpoint{1.445848in}{3.275161in}}{\pgfqpoint{1.450238in}{3.264562in}}{\pgfqpoint{1.458052in}{3.256748in}}%
\pgfpathcurveto{\pgfqpoint{1.465866in}{3.248935in}}{\pgfqpoint{1.476465in}{3.244545in}}{\pgfqpoint{1.487515in}{3.244545in}}%
\pgfpathclose%
\pgfusepath{stroke,fill}%
\end{pgfscope}%
\begin{pgfscope}%
\pgfpathrectangle{\pgfqpoint{0.648703in}{0.548769in}}{\pgfqpoint{5.201297in}{3.102590in}}%
\pgfusepath{clip}%
\pgfsetbuttcap%
\pgfsetroundjoin%
\definecolor{currentfill}{rgb}{1.000000,0.498039,0.054902}%
\pgfsetfillcolor{currentfill}%
\pgfsetlinewidth{1.003750pt}%
\definecolor{currentstroke}{rgb}{1.000000,0.498039,0.054902}%
\pgfsetstrokecolor{currentstroke}%
\pgfsetdash{}{0pt}%
\pgfpathmoveto{\pgfqpoint{1.367468in}{3.312204in}}%
\pgfpathcurveto{\pgfqpoint{1.378518in}{3.312204in}}{\pgfqpoint{1.389118in}{3.316594in}}{\pgfqpoint{1.396931in}{3.324407in}}%
\pgfpathcurveto{\pgfqpoint{1.404745in}{3.332221in}}{\pgfqpoint{1.409135in}{3.342820in}}{\pgfqpoint{1.409135in}{3.353870in}}%
\pgfpathcurveto{\pgfqpoint{1.409135in}{3.364920in}}{\pgfqpoint{1.404745in}{3.375519in}}{\pgfqpoint{1.396931in}{3.383333in}}%
\pgfpathcurveto{\pgfqpoint{1.389118in}{3.391147in}}{\pgfqpoint{1.378518in}{3.395537in}}{\pgfqpoint{1.367468in}{3.395537in}}%
\pgfpathcurveto{\pgfqpoint{1.356418in}{3.395537in}}{\pgfqpoint{1.345819in}{3.391147in}}{\pgfqpoint{1.338006in}{3.383333in}}%
\pgfpathcurveto{\pgfqpoint{1.330192in}{3.375519in}}{\pgfqpoint{1.325802in}{3.364920in}}{\pgfqpoint{1.325802in}{3.353870in}}%
\pgfpathcurveto{\pgfqpoint{1.325802in}{3.342820in}}{\pgfqpoint{1.330192in}{3.332221in}}{\pgfqpoint{1.338006in}{3.324407in}}%
\pgfpathcurveto{\pgfqpoint{1.345819in}{3.316594in}}{\pgfqpoint{1.356418in}{3.312204in}}{\pgfqpoint{1.367468in}{3.312204in}}%
\pgfpathclose%
\pgfusepath{stroke,fill}%
\end{pgfscope}%
\begin{pgfscope}%
\pgfpathrectangle{\pgfqpoint{0.648703in}{0.548769in}}{\pgfqpoint{5.201297in}{3.102590in}}%
\pgfusepath{clip}%
\pgfsetbuttcap%
\pgfsetroundjoin%
\definecolor{currentfill}{rgb}{1.000000,0.498039,0.054902}%
\pgfsetfillcolor{currentfill}%
\pgfsetlinewidth{1.003750pt}%
\definecolor{currentstroke}{rgb}{1.000000,0.498039,0.054902}%
\pgfsetstrokecolor{currentstroke}%
\pgfsetdash{}{0pt}%
\pgfpathmoveto{\pgfqpoint{0.939434in}{3.210715in}}%
\pgfpathcurveto{\pgfqpoint{0.950484in}{3.210715in}}{\pgfqpoint{0.961083in}{3.215105in}}{\pgfqpoint{0.968897in}{3.222919in}}%
\pgfpathcurveto{\pgfqpoint{0.976710in}{3.230733in}}{\pgfqpoint{0.981100in}{3.241332in}}{\pgfqpoint{0.981100in}{3.252382in}}%
\pgfpathcurveto{\pgfqpoint{0.981100in}{3.263432in}}{\pgfqpoint{0.976710in}{3.274031in}}{\pgfqpoint{0.968897in}{3.281844in}}%
\pgfpathcurveto{\pgfqpoint{0.961083in}{3.289658in}}{\pgfqpoint{0.950484in}{3.294048in}}{\pgfqpoint{0.939434in}{3.294048in}}%
\pgfpathcurveto{\pgfqpoint{0.928384in}{3.294048in}}{\pgfqpoint{0.917785in}{3.289658in}}{\pgfqpoint{0.909971in}{3.281844in}}%
\pgfpathcurveto{\pgfqpoint{0.902157in}{3.274031in}}{\pgfqpoint{0.897767in}{3.263432in}}{\pgfqpoint{0.897767in}{3.252382in}}%
\pgfpathcurveto{\pgfqpoint{0.897767in}{3.241332in}}{\pgfqpoint{0.902157in}{3.230733in}}{\pgfqpoint{0.909971in}{3.222919in}}%
\pgfpathcurveto{\pgfqpoint{0.917785in}{3.215105in}}{\pgfqpoint{0.928384in}{3.210715in}}{\pgfqpoint{0.939434in}{3.210715in}}%
\pgfpathclose%
\pgfusepath{stroke,fill}%
\end{pgfscope}%
\begin{pgfscope}%
\pgfpathrectangle{\pgfqpoint{0.648703in}{0.548769in}}{\pgfqpoint{5.201297in}{3.102590in}}%
\pgfusepath{clip}%
\pgfsetbuttcap%
\pgfsetroundjoin%
\definecolor{currentfill}{rgb}{1.000000,0.498039,0.054902}%
\pgfsetfillcolor{currentfill}%
\pgfsetlinewidth{1.003750pt}%
\definecolor{currentstroke}{rgb}{1.000000,0.498039,0.054902}%
\pgfsetstrokecolor{currentstroke}%
\pgfsetdash{}{0pt}%
\pgfpathmoveto{\pgfqpoint{1.880553in}{3.236087in}}%
\pgfpathcurveto{\pgfqpoint{1.891603in}{3.236087in}}{\pgfqpoint{1.902202in}{3.240477in}}{\pgfqpoint{1.910015in}{3.248291in}}%
\pgfpathcurveto{\pgfqpoint{1.917829in}{3.256105in}}{\pgfqpoint{1.922219in}{3.266704in}}{\pgfqpoint{1.922219in}{3.277754in}}%
\pgfpathcurveto{\pgfqpoint{1.922219in}{3.288804in}}{\pgfqpoint{1.917829in}{3.299403in}}{\pgfqpoint{1.910015in}{3.307217in}}%
\pgfpathcurveto{\pgfqpoint{1.902202in}{3.315030in}}{\pgfqpoint{1.891603in}{3.319421in}}{\pgfqpoint{1.880553in}{3.319421in}}%
\pgfpathcurveto{\pgfqpoint{1.869502in}{3.319421in}}{\pgfqpoint{1.858903in}{3.315030in}}{\pgfqpoint{1.851090in}{3.307217in}}%
\pgfpathcurveto{\pgfqpoint{1.843276in}{3.299403in}}{\pgfqpoint{1.838886in}{3.288804in}}{\pgfqpoint{1.838886in}{3.277754in}}%
\pgfpathcurveto{\pgfqpoint{1.838886in}{3.266704in}}{\pgfqpoint{1.843276in}{3.256105in}}{\pgfqpoint{1.851090in}{3.248291in}}%
\pgfpathcurveto{\pgfqpoint{1.858903in}{3.240477in}}{\pgfqpoint{1.869502in}{3.236087in}}{\pgfqpoint{1.880553in}{3.236087in}}%
\pgfpathclose%
\pgfusepath{stroke,fill}%
\end{pgfscope}%
\begin{pgfscope}%
\pgfpathrectangle{\pgfqpoint{0.648703in}{0.548769in}}{\pgfqpoint{5.201297in}{3.102590in}}%
\pgfusepath{clip}%
\pgfsetbuttcap%
\pgfsetroundjoin%
\definecolor{currentfill}{rgb}{0.121569,0.466667,0.705882}%
\pgfsetfillcolor{currentfill}%
\pgfsetlinewidth{1.003750pt}%
\definecolor{currentstroke}{rgb}{0.121569,0.466667,0.705882}%
\pgfsetstrokecolor{currentstroke}%
\pgfsetdash{}{0pt}%
\pgfpathmoveto{\pgfqpoint{1.340673in}{0.648129in}}%
\pgfpathcurveto{\pgfqpoint{1.351723in}{0.648129in}}{\pgfqpoint{1.362322in}{0.652519in}}{\pgfqpoint{1.370135in}{0.660333in}}%
\pgfpathcurveto{\pgfqpoint{1.377949in}{0.668146in}}{\pgfqpoint{1.382339in}{0.678745in}}{\pgfqpoint{1.382339in}{0.689796in}}%
\pgfpathcurveto{\pgfqpoint{1.382339in}{0.700846in}}{\pgfqpoint{1.377949in}{0.711445in}}{\pgfqpoint{1.370135in}{0.719258in}}%
\pgfpathcurveto{\pgfqpoint{1.362322in}{0.727072in}}{\pgfqpoint{1.351723in}{0.731462in}}{\pgfqpoint{1.340673in}{0.731462in}}%
\pgfpathcurveto{\pgfqpoint{1.329623in}{0.731462in}}{\pgfqpoint{1.319024in}{0.727072in}}{\pgfqpoint{1.311210in}{0.719258in}}%
\pgfpathcurveto{\pgfqpoint{1.303396in}{0.711445in}}{\pgfqpoint{1.299006in}{0.700846in}}{\pgfqpoint{1.299006in}{0.689796in}}%
\pgfpathcurveto{\pgfqpoint{1.299006in}{0.678745in}}{\pgfqpoint{1.303396in}{0.668146in}}{\pgfqpoint{1.311210in}{0.660333in}}%
\pgfpathcurveto{\pgfqpoint{1.319024in}{0.652519in}}{\pgfqpoint{1.329623in}{0.648129in}}{\pgfqpoint{1.340673in}{0.648129in}}%
\pgfpathclose%
\pgfusepath{stroke,fill}%
\end{pgfscope}%
\begin{pgfscope}%
\pgfpathrectangle{\pgfqpoint{0.648703in}{0.548769in}}{\pgfqpoint{5.201297in}{3.102590in}}%
\pgfusepath{clip}%
\pgfsetbuttcap%
\pgfsetroundjoin%
\definecolor{currentfill}{rgb}{0.121569,0.466667,0.705882}%
\pgfsetfillcolor{currentfill}%
\pgfsetlinewidth{1.003750pt}%
\definecolor{currentstroke}{rgb}{0.121569,0.466667,0.705882}%
\pgfsetstrokecolor{currentstroke}%
\pgfsetdash{}{0pt}%
\pgfpathmoveto{\pgfqpoint{1.578557in}{0.648129in}}%
\pgfpathcurveto{\pgfqpoint{1.589607in}{0.648129in}}{\pgfqpoint{1.600206in}{0.652519in}}{\pgfqpoint{1.608020in}{0.660333in}}%
\pgfpathcurveto{\pgfqpoint{1.615833in}{0.668146in}}{\pgfqpoint{1.620223in}{0.678745in}}{\pgfqpoint{1.620223in}{0.689796in}}%
\pgfpathcurveto{\pgfqpoint{1.620223in}{0.700846in}}{\pgfqpoint{1.615833in}{0.711445in}}{\pgfqpoint{1.608020in}{0.719258in}}%
\pgfpathcurveto{\pgfqpoint{1.600206in}{0.727072in}}{\pgfqpoint{1.589607in}{0.731462in}}{\pgfqpoint{1.578557in}{0.731462in}}%
\pgfpathcurveto{\pgfqpoint{1.567507in}{0.731462in}}{\pgfqpoint{1.556908in}{0.727072in}}{\pgfqpoint{1.549094in}{0.719258in}}%
\pgfpathcurveto{\pgfqpoint{1.541280in}{0.711445in}}{\pgfqpoint{1.536890in}{0.700846in}}{\pgfqpoint{1.536890in}{0.689796in}}%
\pgfpathcurveto{\pgfqpoint{1.536890in}{0.678745in}}{\pgfqpoint{1.541280in}{0.668146in}}{\pgfqpoint{1.549094in}{0.660333in}}%
\pgfpathcurveto{\pgfqpoint{1.556908in}{0.652519in}}{\pgfqpoint{1.567507in}{0.648129in}}{\pgfqpoint{1.578557in}{0.648129in}}%
\pgfpathclose%
\pgfusepath{stroke,fill}%
\end{pgfscope}%
\begin{pgfscope}%
\pgfpathrectangle{\pgfqpoint{0.648703in}{0.548769in}}{\pgfqpoint{5.201297in}{3.102590in}}%
\pgfusepath{clip}%
\pgfsetbuttcap%
\pgfsetroundjoin%
\definecolor{currentfill}{rgb}{0.121569,0.466667,0.705882}%
\pgfsetfillcolor{currentfill}%
\pgfsetlinewidth{1.003750pt}%
\definecolor{currentstroke}{rgb}{0.121569,0.466667,0.705882}%
\pgfsetstrokecolor{currentstroke}%
\pgfsetdash{}{0pt}%
\pgfpathmoveto{\pgfqpoint{1.095866in}{0.648129in}}%
\pgfpathcurveto{\pgfqpoint{1.106916in}{0.648129in}}{\pgfqpoint{1.117515in}{0.652519in}}{\pgfqpoint{1.125329in}{0.660333in}}%
\pgfpathcurveto{\pgfqpoint{1.133142in}{0.668146in}}{\pgfqpoint{1.137533in}{0.678745in}}{\pgfqpoint{1.137533in}{0.689796in}}%
\pgfpathcurveto{\pgfqpoint{1.137533in}{0.700846in}}{\pgfqpoint{1.133142in}{0.711445in}}{\pgfqpoint{1.125329in}{0.719258in}}%
\pgfpathcurveto{\pgfqpoint{1.117515in}{0.727072in}}{\pgfqpoint{1.106916in}{0.731462in}}{\pgfqpoint{1.095866in}{0.731462in}}%
\pgfpathcurveto{\pgfqpoint{1.084816in}{0.731462in}}{\pgfqpoint{1.074217in}{0.727072in}}{\pgfqpoint{1.066403in}{0.719258in}}%
\pgfpathcurveto{\pgfqpoint{1.058590in}{0.711445in}}{\pgfqpoint{1.054199in}{0.700846in}}{\pgfqpoint{1.054199in}{0.689796in}}%
\pgfpathcurveto{\pgfqpoint{1.054199in}{0.678745in}}{\pgfqpoint{1.058590in}{0.668146in}}{\pgfqpoint{1.066403in}{0.660333in}}%
\pgfpathcurveto{\pgfqpoint{1.074217in}{0.652519in}}{\pgfqpoint{1.084816in}{0.648129in}}{\pgfqpoint{1.095866in}{0.648129in}}%
\pgfpathclose%
\pgfusepath{stroke,fill}%
\end{pgfscope}%
\begin{pgfscope}%
\pgfpathrectangle{\pgfqpoint{0.648703in}{0.548769in}}{\pgfqpoint{5.201297in}{3.102590in}}%
\pgfusepath{clip}%
\pgfsetbuttcap%
\pgfsetroundjoin%
\definecolor{currentfill}{rgb}{1.000000,0.498039,0.054902}%
\pgfsetfillcolor{currentfill}%
\pgfsetlinewidth{1.003750pt}%
\definecolor{currentstroke}{rgb}{1.000000,0.498039,0.054902}%
\pgfsetstrokecolor{currentstroke}%
\pgfsetdash{}{0pt}%
\pgfpathmoveto{\pgfqpoint{1.372808in}{3.405235in}}%
\pgfpathcurveto{\pgfqpoint{1.383858in}{3.405235in}}{\pgfqpoint{1.394457in}{3.409625in}}{\pgfqpoint{1.402270in}{3.417439in}}%
\pgfpathcurveto{\pgfqpoint{1.410084in}{3.425252in}}{\pgfqpoint{1.414474in}{3.435851in}}{\pgfqpoint{1.414474in}{3.446901in}}%
\pgfpathcurveto{\pgfqpoint{1.414474in}{3.457952in}}{\pgfqpoint{1.410084in}{3.468551in}}{\pgfqpoint{1.402270in}{3.476364in}}%
\pgfpathcurveto{\pgfqpoint{1.394457in}{3.484178in}}{\pgfqpoint{1.383858in}{3.488568in}}{\pgfqpoint{1.372808in}{3.488568in}}%
\pgfpathcurveto{\pgfqpoint{1.361758in}{3.488568in}}{\pgfqpoint{1.351159in}{3.484178in}}{\pgfqpoint{1.343345in}{3.476364in}}%
\pgfpathcurveto{\pgfqpoint{1.335531in}{3.468551in}}{\pgfqpoint{1.331141in}{3.457952in}}{\pgfqpoint{1.331141in}{3.446901in}}%
\pgfpathcurveto{\pgfqpoint{1.331141in}{3.435851in}}{\pgfqpoint{1.335531in}{3.425252in}}{\pgfqpoint{1.343345in}{3.417439in}}%
\pgfpathcurveto{\pgfqpoint{1.351159in}{3.409625in}}{\pgfqpoint{1.361758in}{3.405235in}}{\pgfqpoint{1.372808in}{3.405235in}}%
\pgfpathclose%
\pgfusepath{stroke,fill}%
\end{pgfscope}%
\begin{pgfscope}%
\pgfpathrectangle{\pgfqpoint{0.648703in}{0.548769in}}{\pgfqpoint{5.201297in}{3.102590in}}%
\pgfusepath{clip}%
\pgfsetbuttcap%
\pgfsetroundjoin%
\definecolor{currentfill}{rgb}{1.000000,0.498039,0.054902}%
\pgfsetfillcolor{currentfill}%
\pgfsetlinewidth{1.003750pt}%
\definecolor{currentstroke}{rgb}{1.000000,0.498039,0.054902}%
\pgfsetstrokecolor{currentstroke}%
\pgfsetdash{}{0pt}%
\pgfpathmoveto{\pgfqpoint{1.274772in}{3.193800in}}%
\pgfpathcurveto{\pgfqpoint{1.285822in}{3.193800in}}{\pgfqpoint{1.296421in}{3.198191in}}{\pgfqpoint{1.304235in}{3.206004in}}%
\pgfpathcurveto{\pgfqpoint{1.312048in}{3.213818in}}{\pgfqpoint{1.316439in}{3.224417in}}{\pgfqpoint{1.316439in}{3.235467in}}%
\pgfpathcurveto{\pgfqpoint{1.316439in}{3.246517in}}{\pgfqpoint{1.312048in}{3.257116in}}{\pgfqpoint{1.304235in}{3.264930in}}%
\pgfpathcurveto{\pgfqpoint{1.296421in}{3.272743in}}{\pgfqpoint{1.285822in}{3.277134in}}{\pgfqpoint{1.274772in}{3.277134in}}%
\pgfpathcurveto{\pgfqpoint{1.263722in}{3.277134in}}{\pgfqpoint{1.253123in}{3.272743in}}{\pgfqpoint{1.245309in}{3.264930in}}%
\pgfpathcurveto{\pgfqpoint{1.237495in}{3.257116in}}{\pgfqpoint{1.233105in}{3.246517in}}{\pgfqpoint{1.233105in}{3.235467in}}%
\pgfpathcurveto{\pgfqpoint{1.233105in}{3.224417in}}{\pgfqpoint{1.237495in}{3.213818in}}{\pgfqpoint{1.245309in}{3.206004in}}%
\pgfpathcurveto{\pgfqpoint{1.253123in}{3.198191in}}{\pgfqpoint{1.263722in}{3.193800in}}{\pgfqpoint{1.274772in}{3.193800in}}%
\pgfpathclose%
\pgfusepath{stroke,fill}%
\end{pgfscope}%
\begin{pgfscope}%
\pgfpathrectangle{\pgfqpoint{0.648703in}{0.548769in}}{\pgfqpoint{5.201297in}{3.102590in}}%
\pgfusepath{clip}%
\pgfsetbuttcap%
\pgfsetroundjoin%
\definecolor{currentfill}{rgb}{1.000000,0.498039,0.054902}%
\pgfsetfillcolor{currentfill}%
\pgfsetlinewidth{1.003750pt}%
\definecolor{currentstroke}{rgb}{1.000000,0.498039,0.054902}%
\pgfsetstrokecolor{currentstroke}%
\pgfsetdash{}{0pt}%
\pgfpathmoveto{\pgfqpoint{1.499496in}{3.358719in}}%
\pgfpathcurveto{\pgfqpoint{1.510546in}{3.358719in}}{\pgfqpoint{1.521145in}{3.363109in}}{\pgfqpoint{1.528958in}{3.370923in}}%
\pgfpathcurveto{\pgfqpoint{1.536772in}{3.378737in}}{\pgfqpoint{1.541162in}{3.389336in}}{\pgfqpoint{1.541162in}{3.400386in}}%
\pgfpathcurveto{\pgfqpoint{1.541162in}{3.411436in}}{\pgfqpoint{1.536772in}{3.422035in}}{\pgfqpoint{1.528958in}{3.429849in}}%
\pgfpathcurveto{\pgfqpoint{1.521145in}{3.437662in}}{\pgfqpoint{1.510546in}{3.442053in}}{\pgfqpoint{1.499496in}{3.442053in}}%
\pgfpathcurveto{\pgfqpoint{1.488445in}{3.442053in}}{\pgfqpoint{1.477846in}{3.437662in}}{\pgfqpoint{1.470033in}{3.429849in}}%
\pgfpathcurveto{\pgfqpoint{1.462219in}{3.422035in}}{\pgfqpoint{1.457829in}{3.411436in}}{\pgfqpoint{1.457829in}{3.400386in}}%
\pgfpathcurveto{\pgfqpoint{1.457829in}{3.389336in}}{\pgfqpoint{1.462219in}{3.378737in}}{\pgfqpoint{1.470033in}{3.370923in}}%
\pgfpathcurveto{\pgfqpoint{1.477846in}{3.363109in}}{\pgfqpoint{1.488445in}{3.358719in}}{\pgfqpoint{1.499496in}{3.358719in}}%
\pgfpathclose%
\pgfusepath{stroke,fill}%
\end{pgfscope}%
\begin{pgfscope}%
\pgfpathrectangle{\pgfqpoint{0.648703in}{0.548769in}}{\pgfqpoint{5.201297in}{3.102590in}}%
\pgfusepath{clip}%
\pgfsetbuttcap%
\pgfsetroundjoin%
\definecolor{currentfill}{rgb}{0.121569,0.466667,0.705882}%
\pgfsetfillcolor{currentfill}%
\pgfsetlinewidth{1.003750pt}%
\definecolor{currentstroke}{rgb}{0.121569,0.466667,0.705882}%
\pgfsetstrokecolor{currentstroke}%
\pgfsetdash{}{0pt}%
\pgfpathmoveto{\pgfqpoint{0.885126in}{0.648129in}}%
\pgfpathcurveto{\pgfqpoint{0.896176in}{0.648129in}}{\pgfqpoint{0.906775in}{0.652519in}}{\pgfqpoint{0.914589in}{0.660333in}}%
\pgfpathcurveto{\pgfqpoint{0.922402in}{0.668146in}}{\pgfqpoint{0.926793in}{0.678745in}}{\pgfqpoint{0.926793in}{0.689796in}}%
\pgfpathcurveto{\pgfqpoint{0.926793in}{0.700846in}}{\pgfqpoint{0.922402in}{0.711445in}}{\pgfqpoint{0.914589in}{0.719258in}}%
\pgfpathcurveto{\pgfqpoint{0.906775in}{0.727072in}}{\pgfqpoint{0.896176in}{0.731462in}}{\pgfqpoint{0.885126in}{0.731462in}}%
\pgfpathcurveto{\pgfqpoint{0.874076in}{0.731462in}}{\pgfqpoint{0.863477in}{0.727072in}}{\pgfqpoint{0.855663in}{0.719258in}}%
\pgfpathcurveto{\pgfqpoint{0.847850in}{0.711445in}}{\pgfqpoint{0.843459in}{0.700846in}}{\pgfqpoint{0.843459in}{0.689796in}}%
\pgfpathcurveto{\pgfqpoint{0.843459in}{0.678745in}}{\pgfqpoint{0.847850in}{0.668146in}}{\pgfqpoint{0.855663in}{0.660333in}}%
\pgfpathcurveto{\pgfqpoint{0.863477in}{0.652519in}}{\pgfqpoint{0.874076in}{0.648129in}}{\pgfqpoint{0.885126in}{0.648129in}}%
\pgfpathclose%
\pgfusepath{stroke,fill}%
\end{pgfscope}%
\begin{pgfscope}%
\pgfpathrectangle{\pgfqpoint{0.648703in}{0.548769in}}{\pgfqpoint{5.201297in}{3.102590in}}%
\pgfusepath{clip}%
\pgfsetbuttcap%
\pgfsetroundjoin%
\definecolor{currentfill}{rgb}{0.121569,0.466667,0.705882}%
\pgfsetfillcolor{currentfill}%
\pgfsetlinewidth{1.003750pt}%
\definecolor{currentstroke}{rgb}{0.121569,0.466667,0.705882}%
\pgfsetstrokecolor{currentstroke}%
\pgfsetdash{}{0pt}%
\pgfpathmoveto{\pgfqpoint{0.930200in}{0.796133in}}%
\pgfpathcurveto{\pgfqpoint{0.941250in}{0.796133in}}{\pgfqpoint{0.951849in}{0.800523in}}{\pgfqpoint{0.959663in}{0.808337in}}%
\pgfpathcurveto{\pgfqpoint{0.967476in}{0.816151in}}{\pgfqpoint{0.971866in}{0.826750in}}{\pgfqpoint{0.971866in}{0.837800in}}%
\pgfpathcurveto{\pgfqpoint{0.971866in}{0.848850in}}{\pgfqpoint{0.967476in}{0.859449in}}{\pgfqpoint{0.959663in}{0.867263in}}%
\pgfpathcurveto{\pgfqpoint{0.951849in}{0.875076in}}{\pgfqpoint{0.941250in}{0.879466in}}{\pgfqpoint{0.930200in}{0.879466in}}%
\pgfpathcurveto{\pgfqpoint{0.919150in}{0.879466in}}{\pgfqpoint{0.908551in}{0.875076in}}{\pgfqpoint{0.900737in}{0.867263in}}%
\pgfpathcurveto{\pgfqpoint{0.892923in}{0.859449in}}{\pgfqpoint{0.888533in}{0.848850in}}{\pgfqpoint{0.888533in}{0.837800in}}%
\pgfpathcurveto{\pgfqpoint{0.888533in}{0.826750in}}{\pgfqpoint{0.892923in}{0.816151in}}{\pgfqpoint{0.900737in}{0.808337in}}%
\pgfpathcurveto{\pgfqpoint{0.908551in}{0.800523in}}{\pgfqpoint{0.919150in}{0.796133in}}{\pgfqpoint{0.930200in}{0.796133in}}%
\pgfpathclose%
\pgfusepath{stroke,fill}%
\end{pgfscope}%
\begin{pgfscope}%
\pgfpathrectangle{\pgfqpoint{0.648703in}{0.548769in}}{\pgfqpoint{5.201297in}{3.102590in}}%
\pgfusepath{clip}%
\pgfsetbuttcap%
\pgfsetroundjoin%
\definecolor{currentfill}{rgb}{0.121569,0.466667,0.705882}%
\pgfsetfillcolor{currentfill}%
\pgfsetlinewidth{1.003750pt}%
\definecolor{currentstroke}{rgb}{0.121569,0.466667,0.705882}%
\pgfsetstrokecolor{currentstroke}%
\pgfsetdash{}{0pt}%
\pgfpathmoveto{\pgfqpoint{4.806473in}{3.155742in}}%
\pgfpathcurveto{\pgfqpoint{4.817523in}{3.155742in}}{\pgfqpoint{4.828122in}{3.160132in}}{\pgfqpoint{4.835936in}{3.167946in}}%
\pgfpathcurveto{\pgfqpoint{4.843749in}{3.175760in}}{\pgfqpoint{4.848139in}{3.186359in}}{\pgfqpoint{4.848139in}{3.197409in}}%
\pgfpathcurveto{\pgfqpoint{4.848139in}{3.208459in}}{\pgfqpoint{4.843749in}{3.219058in}}{\pgfqpoint{4.835936in}{3.226872in}}%
\pgfpathcurveto{\pgfqpoint{4.828122in}{3.234685in}}{\pgfqpoint{4.817523in}{3.239075in}}{\pgfqpoint{4.806473in}{3.239075in}}%
\pgfpathcurveto{\pgfqpoint{4.795423in}{3.239075in}}{\pgfqpoint{4.784824in}{3.234685in}}{\pgfqpoint{4.777010in}{3.226872in}}%
\pgfpathcurveto{\pgfqpoint{4.769196in}{3.219058in}}{\pgfqpoint{4.764806in}{3.208459in}}{\pgfqpoint{4.764806in}{3.197409in}}%
\pgfpathcurveto{\pgfqpoint{4.764806in}{3.186359in}}{\pgfqpoint{4.769196in}{3.175760in}}{\pgfqpoint{4.777010in}{3.167946in}}%
\pgfpathcurveto{\pgfqpoint{4.784824in}{3.160132in}}{\pgfqpoint{4.795423in}{3.155742in}}{\pgfqpoint{4.806473in}{3.155742in}}%
\pgfpathclose%
\pgfusepath{stroke,fill}%
\end{pgfscope}%
\begin{pgfscope}%
\pgfpathrectangle{\pgfqpoint{0.648703in}{0.548769in}}{\pgfqpoint{5.201297in}{3.102590in}}%
\pgfusepath{clip}%
\pgfsetbuttcap%
\pgfsetroundjoin%
\definecolor{currentfill}{rgb}{0.121569,0.466667,0.705882}%
\pgfsetfillcolor{currentfill}%
\pgfsetlinewidth{1.003750pt}%
\definecolor{currentstroke}{rgb}{0.121569,0.466667,0.705882}%
\pgfsetstrokecolor{currentstroke}%
\pgfsetdash{}{0pt}%
\pgfpathmoveto{\pgfqpoint{1.379714in}{0.648129in}}%
\pgfpathcurveto{\pgfqpoint{1.390765in}{0.648129in}}{\pgfqpoint{1.401364in}{0.652519in}}{\pgfqpoint{1.409177in}{0.660333in}}%
\pgfpathcurveto{\pgfqpoint{1.416991in}{0.668146in}}{\pgfqpoint{1.421381in}{0.678745in}}{\pgfqpoint{1.421381in}{0.689796in}}%
\pgfpathcurveto{\pgfqpoint{1.421381in}{0.700846in}}{\pgfqpoint{1.416991in}{0.711445in}}{\pgfqpoint{1.409177in}{0.719258in}}%
\pgfpathcurveto{\pgfqpoint{1.401364in}{0.727072in}}{\pgfqpoint{1.390765in}{0.731462in}}{\pgfqpoint{1.379714in}{0.731462in}}%
\pgfpathcurveto{\pgfqpoint{1.368664in}{0.731462in}}{\pgfqpoint{1.358065in}{0.727072in}}{\pgfqpoint{1.350252in}{0.719258in}}%
\pgfpathcurveto{\pgfqpoint{1.342438in}{0.711445in}}{\pgfqpoint{1.338048in}{0.700846in}}{\pgfqpoint{1.338048in}{0.689796in}}%
\pgfpathcurveto{\pgfqpoint{1.338048in}{0.678745in}}{\pgfqpoint{1.342438in}{0.668146in}}{\pgfqpoint{1.350252in}{0.660333in}}%
\pgfpathcurveto{\pgfqpoint{1.358065in}{0.652519in}}{\pgfqpoint{1.368664in}{0.648129in}}{\pgfqpoint{1.379714in}{0.648129in}}%
\pgfpathclose%
\pgfusepath{stroke,fill}%
\end{pgfscope}%
\begin{pgfscope}%
\pgfpathrectangle{\pgfqpoint{0.648703in}{0.548769in}}{\pgfqpoint{5.201297in}{3.102590in}}%
\pgfusepath{clip}%
\pgfsetbuttcap%
\pgfsetroundjoin%
\definecolor{currentfill}{rgb}{0.121569,0.466667,0.705882}%
\pgfsetfillcolor{currentfill}%
\pgfsetlinewidth{1.003750pt}%
\definecolor{currentstroke}{rgb}{0.121569,0.466667,0.705882}%
\pgfsetstrokecolor{currentstroke}%
\pgfsetdash{}{0pt}%
\pgfpathmoveto{\pgfqpoint{0.966261in}{0.648129in}}%
\pgfpathcurveto{\pgfqpoint{0.977311in}{0.648129in}}{\pgfqpoint{0.987910in}{0.652519in}}{\pgfqpoint{0.995724in}{0.660333in}}%
\pgfpathcurveto{\pgfqpoint{1.003538in}{0.668146in}}{\pgfqpoint{1.007928in}{0.678745in}}{\pgfqpoint{1.007928in}{0.689796in}}%
\pgfpathcurveto{\pgfqpoint{1.007928in}{0.700846in}}{\pgfqpoint{1.003538in}{0.711445in}}{\pgfqpoint{0.995724in}{0.719258in}}%
\pgfpathcurveto{\pgfqpoint{0.987910in}{0.727072in}}{\pgfqpoint{0.977311in}{0.731462in}}{\pgfqpoint{0.966261in}{0.731462in}}%
\pgfpathcurveto{\pgfqpoint{0.955211in}{0.731462in}}{\pgfqpoint{0.944612in}{0.727072in}}{\pgfqpoint{0.936798in}{0.719258in}}%
\pgfpathcurveto{\pgfqpoint{0.928985in}{0.711445in}}{\pgfqpoint{0.924594in}{0.700846in}}{\pgfqpoint{0.924594in}{0.689796in}}%
\pgfpathcurveto{\pgfqpoint{0.924594in}{0.678745in}}{\pgfqpoint{0.928985in}{0.668146in}}{\pgfqpoint{0.936798in}{0.660333in}}%
\pgfpathcurveto{\pgfqpoint{0.944612in}{0.652519in}}{\pgfqpoint{0.955211in}{0.648129in}}{\pgfqpoint{0.966261in}{0.648129in}}%
\pgfpathclose%
\pgfusepath{stroke,fill}%
\end{pgfscope}%
\begin{pgfscope}%
\pgfpathrectangle{\pgfqpoint{0.648703in}{0.548769in}}{\pgfqpoint{5.201297in}{3.102590in}}%
\pgfusepath{clip}%
\pgfsetbuttcap%
\pgfsetroundjoin%
\definecolor{currentfill}{rgb}{0.121569,0.466667,0.705882}%
\pgfsetfillcolor{currentfill}%
\pgfsetlinewidth{1.003750pt}%
\definecolor{currentstroke}{rgb}{0.121569,0.466667,0.705882}%
\pgfsetstrokecolor{currentstroke}%
\pgfsetdash{}{0pt}%
\pgfpathmoveto{\pgfqpoint{0.891985in}{2.512981in}}%
\pgfpathcurveto{\pgfqpoint{0.903035in}{2.512981in}}{\pgfqpoint{0.913634in}{2.517371in}}{\pgfqpoint{0.921448in}{2.525185in}}%
\pgfpathcurveto{\pgfqpoint{0.929262in}{2.532999in}}{\pgfqpoint{0.933652in}{2.543598in}}{\pgfqpoint{0.933652in}{2.554648in}}%
\pgfpathcurveto{\pgfqpoint{0.933652in}{2.565698in}}{\pgfqpoint{0.929262in}{2.576297in}}{\pgfqpoint{0.921448in}{2.584111in}}%
\pgfpathcurveto{\pgfqpoint{0.913634in}{2.591924in}}{\pgfqpoint{0.903035in}{2.596315in}}{\pgfqpoint{0.891985in}{2.596315in}}%
\pgfpathcurveto{\pgfqpoint{0.880935in}{2.596315in}}{\pgfqpoint{0.870336in}{2.591924in}}{\pgfqpoint{0.862522in}{2.584111in}}%
\pgfpathcurveto{\pgfqpoint{0.854709in}{2.576297in}}{\pgfqpoint{0.850319in}{2.565698in}}{\pgfqpoint{0.850319in}{2.554648in}}%
\pgfpathcurveto{\pgfqpoint{0.850319in}{2.543598in}}{\pgfqpoint{0.854709in}{2.532999in}}{\pgfqpoint{0.862522in}{2.525185in}}%
\pgfpathcurveto{\pgfqpoint{0.870336in}{2.517371in}}{\pgfqpoint{0.880935in}{2.512981in}}{\pgfqpoint{0.891985in}{2.512981in}}%
\pgfpathclose%
\pgfusepath{stroke,fill}%
\end{pgfscope}%
\begin{pgfscope}%
\pgfpathrectangle{\pgfqpoint{0.648703in}{0.548769in}}{\pgfqpoint{5.201297in}{3.102590in}}%
\pgfusepath{clip}%
\pgfsetbuttcap%
\pgfsetroundjoin%
\definecolor{currentfill}{rgb}{1.000000,0.498039,0.054902}%
\pgfsetfillcolor{currentfill}%
\pgfsetlinewidth{1.003750pt}%
\definecolor{currentstroke}{rgb}{1.000000,0.498039,0.054902}%
\pgfsetstrokecolor{currentstroke}%
\pgfsetdash{}{0pt}%
\pgfpathmoveto{\pgfqpoint{1.115688in}{3.206486in}}%
\pgfpathcurveto{\pgfqpoint{1.126738in}{3.206486in}}{\pgfqpoint{1.137337in}{3.210877in}}{\pgfqpoint{1.145151in}{3.218690in}}%
\pgfpathcurveto{\pgfqpoint{1.152964in}{3.226504in}}{\pgfqpoint{1.157354in}{3.237103in}}{\pgfqpoint{1.157354in}{3.248153in}}%
\pgfpathcurveto{\pgfqpoint{1.157354in}{3.259203in}}{\pgfqpoint{1.152964in}{3.269802in}}{\pgfqpoint{1.145151in}{3.277616in}}%
\pgfpathcurveto{\pgfqpoint{1.137337in}{3.285429in}}{\pgfqpoint{1.126738in}{3.289820in}}{\pgfqpoint{1.115688in}{3.289820in}}%
\pgfpathcurveto{\pgfqpoint{1.104638in}{3.289820in}}{\pgfqpoint{1.094039in}{3.285429in}}{\pgfqpoint{1.086225in}{3.277616in}}%
\pgfpathcurveto{\pgfqpoint{1.078411in}{3.269802in}}{\pgfqpoint{1.074021in}{3.259203in}}{\pgfqpoint{1.074021in}{3.248153in}}%
\pgfpathcurveto{\pgfqpoint{1.074021in}{3.237103in}}{\pgfqpoint{1.078411in}{3.226504in}}{\pgfqpoint{1.086225in}{3.218690in}}%
\pgfpathcurveto{\pgfqpoint{1.094039in}{3.210877in}}{\pgfqpoint{1.104638in}{3.206486in}}{\pgfqpoint{1.115688in}{3.206486in}}%
\pgfpathclose%
\pgfusepath{stroke,fill}%
\end{pgfscope}%
\begin{pgfscope}%
\pgfpathrectangle{\pgfqpoint{0.648703in}{0.548769in}}{\pgfqpoint{5.201297in}{3.102590in}}%
\pgfusepath{clip}%
\pgfsetbuttcap%
\pgfsetroundjoin%
\definecolor{currentfill}{rgb}{1.000000,0.498039,0.054902}%
\pgfsetfillcolor{currentfill}%
\pgfsetlinewidth{1.003750pt}%
\definecolor{currentstroke}{rgb}{1.000000,0.498039,0.054902}%
\pgfsetstrokecolor{currentstroke}%
\pgfsetdash{}{0pt}%
\pgfpathmoveto{\pgfqpoint{2.060879in}{3.198029in}}%
\pgfpathcurveto{\pgfqpoint{2.071929in}{3.198029in}}{\pgfqpoint{2.082528in}{3.202419in}}{\pgfqpoint{2.090342in}{3.210233in}}%
\pgfpathcurveto{\pgfqpoint{2.098156in}{3.218046in}}{\pgfqpoint{2.102546in}{3.228646in}}{\pgfqpoint{2.102546in}{3.239696in}}%
\pgfpathcurveto{\pgfqpoint{2.102546in}{3.250746in}}{\pgfqpoint{2.098156in}{3.261345in}}{\pgfqpoint{2.090342in}{3.269158in}}%
\pgfpathcurveto{\pgfqpoint{2.082528in}{3.276972in}}{\pgfqpoint{2.071929in}{3.281362in}}{\pgfqpoint{2.060879in}{3.281362in}}%
\pgfpathcurveto{\pgfqpoint{2.049829in}{3.281362in}}{\pgfqpoint{2.039230in}{3.276972in}}{\pgfqpoint{2.031417in}{3.269158in}}%
\pgfpathcurveto{\pgfqpoint{2.023603in}{3.261345in}}{\pgfqpoint{2.019213in}{3.250746in}}{\pgfqpoint{2.019213in}{3.239696in}}%
\pgfpathcurveto{\pgfqpoint{2.019213in}{3.228646in}}{\pgfqpoint{2.023603in}{3.218046in}}{\pgfqpoint{2.031417in}{3.210233in}}%
\pgfpathcurveto{\pgfqpoint{2.039230in}{3.202419in}}{\pgfqpoint{2.049829in}{3.198029in}}{\pgfqpoint{2.060879in}{3.198029in}}%
\pgfpathclose%
\pgfusepath{stroke,fill}%
\end{pgfscope}%
\begin{pgfscope}%
\pgfpathrectangle{\pgfqpoint{0.648703in}{0.548769in}}{\pgfqpoint{5.201297in}{3.102590in}}%
\pgfusepath{clip}%
\pgfsetbuttcap%
\pgfsetroundjoin%
\definecolor{currentfill}{rgb}{1.000000,0.498039,0.054902}%
\pgfsetfillcolor{currentfill}%
\pgfsetlinewidth{1.003750pt}%
\definecolor{currentstroke}{rgb}{1.000000,0.498039,0.054902}%
\pgfsetstrokecolor{currentstroke}%
\pgfsetdash{}{0pt}%
\pgfpathmoveto{\pgfqpoint{1.726131in}{3.248773in}}%
\pgfpathcurveto{\pgfqpoint{1.737181in}{3.248773in}}{\pgfqpoint{1.747780in}{3.253164in}}{\pgfqpoint{1.755594in}{3.260977in}}%
\pgfpathcurveto{\pgfqpoint{1.763407in}{3.268791in}}{\pgfqpoint{1.767798in}{3.279390in}}{\pgfqpoint{1.767798in}{3.290440in}}%
\pgfpathcurveto{\pgfqpoint{1.767798in}{3.301490in}}{\pgfqpoint{1.763407in}{3.312089in}}{\pgfqpoint{1.755594in}{3.319903in}}%
\pgfpathcurveto{\pgfqpoint{1.747780in}{3.327716in}}{\pgfqpoint{1.737181in}{3.332107in}}{\pgfqpoint{1.726131in}{3.332107in}}%
\pgfpathcurveto{\pgfqpoint{1.715081in}{3.332107in}}{\pgfqpoint{1.704482in}{3.327716in}}{\pgfqpoint{1.696668in}{3.319903in}}%
\pgfpathcurveto{\pgfqpoint{1.688855in}{3.312089in}}{\pgfqpoint{1.684464in}{3.301490in}}{\pgfqpoint{1.684464in}{3.290440in}}%
\pgfpathcurveto{\pgfqpoint{1.684464in}{3.279390in}}{\pgfqpoint{1.688855in}{3.268791in}}{\pgfqpoint{1.696668in}{3.260977in}}%
\pgfpathcurveto{\pgfqpoint{1.704482in}{3.253164in}}{\pgfqpoint{1.715081in}{3.248773in}}{\pgfqpoint{1.726131in}{3.248773in}}%
\pgfpathclose%
\pgfusepath{stroke,fill}%
\end{pgfscope}%
\begin{pgfscope}%
\pgfpathrectangle{\pgfqpoint{0.648703in}{0.548769in}}{\pgfqpoint{5.201297in}{3.102590in}}%
\pgfusepath{clip}%
\pgfsetbuttcap%
\pgfsetroundjoin%
\definecolor{currentfill}{rgb}{1.000000,0.498039,0.054902}%
\pgfsetfillcolor{currentfill}%
\pgfsetlinewidth{1.003750pt}%
\definecolor{currentstroke}{rgb}{1.000000,0.498039,0.054902}%
\pgfsetstrokecolor{currentstroke}%
\pgfsetdash{}{0pt}%
\pgfpathmoveto{\pgfqpoint{1.335753in}{3.189572in}}%
\pgfpathcurveto{\pgfqpoint{1.346803in}{3.189572in}}{\pgfqpoint{1.357402in}{3.193962in}}{\pgfqpoint{1.365216in}{3.201775in}}%
\pgfpathcurveto{\pgfqpoint{1.373029in}{3.209589in}}{\pgfqpoint{1.377420in}{3.220188in}}{\pgfqpoint{1.377420in}{3.231238in}}%
\pgfpathcurveto{\pgfqpoint{1.377420in}{3.242288in}}{\pgfqpoint{1.373029in}{3.252887in}}{\pgfqpoint{1.365216in}{3.260701in}}%
\pgfpathcurveto{\pgfqpoint{1.357402in}{3.268515in}}{\pgfqpoint{1.346803in}{3.272905in}}{\pgfqpoint{1.335753in}{3.272905in}}%
\pgfpathcurveto{\pgfqpoint{1.324703in}{3.272905in}}{\pgfqpoint{1.314104in}{3.268515in}}{\pgfqpoint{1.306290in}{3.260701in}}%
\pgfpathcurveto{\pgfqpoint{1.298476in}{3.252887in}}{\pgfqpoint{1.294086in}{3.242288in}}{\pgfqpoint{1.294086in}{3.231238in}}%
\pgfpathcurveto{\pgfqpoint{1.294086in}{3.220188in}}{\pgfqpoint{1.298476in}{3.209589in}}{\pgfqpoint{1.306290in}{3.201775in}}%
\pgfpathcurveto{\pgfqpoint{1.314104in}{3.193962in}}{\pgfqpoint{1.324703in}{3.189572in}}{\pgfqpoint{1.335753in}{3.189572in}}%
\pgfpathclose%
\pgfusepath{stroke,fill}%
\end{pgfscope}%
\begin{pgfscope}%
\pgfpathrectangle{\pgfqpoint{0.648703in}{0.548769in}}{\pgfqpoint{5.201297in}{3.102590in}}%
\pgfusepath{clip}%
\pgfsetbuttcap%
\pgfsetroundjoin%
\definecolor{currentfill}{rgb}{0.121569,0.466667,0.705882}%
\pgfsetfillcolor{currentfill}%
\pgfsetlinewidth{1.003750pt}%
\definecolor{currentstroke}{rgb}{0.121569,0.466667,0.705882}%
\pgfsetstrokecolor{currentstroke}%
\pgfsetdash{}{0pt}%
\pgfpathmoveto{\pgfqpoint{1.798345in}{3.181114in}}%
\pgfpathcurveto{\pgfqpoint{1.809395in}{3.181114in}}{\pgfqpoint{1.819994in}{3.185504in}}{\pgfqpoint{1.827808in}{3.193318in}}%
\pgfpathcurveto{\pgfqpoint{1.835621in}{3.201132in}}{\pgfqpoint{1.840011in}{3.211731in}}{\pgfqpoint{1.840011in}{3.222781in}}%
\pgfpathcurveto{\pgfqpoint{1.840011in}{3.233831in}}{\pgfqpoint{1.835621in}{3.244430in}}{\pgfqpoint{1.827808in}{3.252244in}}%
\pgfpathcurveto{\pgfqpoint{1.819994in}{3.260057in}}{\pgfqpoint{1.809395in}{3.264448in}}{\pgfqpoint{1.798345in}{3.264448in}}%
\pgfpathcurveto{\pgfqpoint{1.787295in}{3.264448in}}{\pgfqpoint{1.776696in}{3.260057in}}{\pgfqpoint{1.768882in}{3.252244in}}%
\pgfpathcurveto{\pgfqpoint{1.761068in}{3.244430in}}{\pgfqpoint{1.756678in}{3.233831in}}{\pgfqpoint{1.756678in}{3.222781in}}%
\pgfpathcurveto{\pgfqpoint{1.756678in}{3.211731in}}{\pgfqpoint{1.761068in}{3.201132in}}{\pgfqpoint{1.768882in}{3.193318in}}%
\pgfpathcurveto{\pgfqpoint{1.776696in}{3.185504in}}{\pgfqpoint{1.787295in}{3.181114in}}{\pgfqpoint{1.798345in}{3.181114in}}%
\pgfpathclose%
\pgfusepath{stroke,fill}%
\end{pgfscope}%
\begin{pgfscope}%
\pgfpathrectangle{\pgfqpoint{0.648703in}{0.548769in}}{\pgfqpoint{5.201297in}{3.102590in}}%
\pgfusepath{clip}%
\pgfsetbuttcap%
\pgfsetroundjoin%
\definecolor{currentfill}{rgb}{0.839216,0.152941,0.156863}%
\pgfsetfillcolor{currentfill}%
\pgfsetlinewidth{1.003750pt}%
\definecolor{currentstroke}{rgb}{0.839216,0.152941,0.156863}%
\pgfsetstrokecolor{currentstroke}%
\pgfsetdash{}{0pt}%
\pgfpathmoveto{\pgfqpoint{1.897398in}{3.202258in}}%
\pgfpathcurveto{\pgfqpoint{1.908448in}{3.202258in}}{\pgfqpoint{1.919047in}{3.206648in}}{\pgfqpoint{1.926861in}{3.214462in}}%
\pgfpathcurveto{\pgfqpoint{1.934674in}{3.222275in}}{\pgfqpoint{1.939065in}{3.232874in}}{\pgfqpoint{1.939065in}{3.243924in}}%
\pgfpathcurveto{\pgfqpoint{1.939065in}{3.254974in}}{\pgfqpoint{1.934674in}{3.265573in}}{\pgfqpoint{1.926861in}{3.273387in}}%
\pgfpathcurveto{\pgfqpoint{1.919047in}{3.281201in}}{\pgfqpoint{1.908448in}{3.285591in}}{\pgfqpoint{1.897398in}{3.285591in}}%
\pgfpathcurveto{\pgfqpoint{1.886348in}{3.285591in}}{\pgfqpoint{1.875749in}{3.281201in}}{\pgfqpoint{1.867935in}{3.273387in}}%
\pgfpathcurveto{\pgfqpoint{1.860121in}{3.265573in}}{\pgfqpoint{1.855731in}{3.254974in}}{\pgfqpoint{1.855731in}{3.243924in}}%
\pgfpathcurveto{\pgfqpoint{1.855731in}{3.232874in}}{\pgfqpoint{1.860121in}{3.222275in}}{\pgfqpoint{1.867935in}{3.214462in}}%
\pgfpathcurveto{\pgfqpoint{1.875749in}{3.206648in}}{\pgfqpoint{1.886348in}{3.202258in}}{\pgfqpoint{1.897398in}{3.202258in}}%
\pgfpathclose%
\pgfusepath{stroke,fill}%
\end{pgfscope}%
\begin{pgfscope}%
\pgfpathrectangle{\pgfqpoint{0.648703in}{0.548769in}}{\pgfqpoint{5.201297in}{3.102590in}}%
\pgfusepath{clip}%
\pgfsetbuttcap%
\pgfsetroundjoin%
\definecolor{currentfill}{rgb}{1.000000,0.498039,0.054902}%
\pgfsetfillcolor{currentfill}%
\pgfsetlinewidth{1.003750pt}%
\definecolor{currentstroke}{rgb}{1.000000,0.498039,0.054902}%
\pgfsetstrokecolor{currentstroke}%
\pgfsetdash{}{0pt}%
\pgfpathmoveto{\pgfqpoint{1.494607in}{3.185343in}}%
\pgfpathcurveto{\pgfqpoint{1.505658in}{3.185343in}}{\pgfqpoint{1.516257in}{3.189733in}}{\pgfqpoint{1.524070in}{3.197547in}}%
\pgfpathcurveto{\pgfqpoint{1.531884in}{3.205360in}}{\pgfqpoint{1.536274in}{3.215959in}}{\pgfqpoint{1.536274in}{3.227010in}}%
\pgfpathcurveto{\pgfqpoint{1.536274in}{3.238060in}}{\pgfqpoint{1.531884in}{3.248659in}}{\pgfqpoint{1.524070in}{3.256472in}}%
\pgfpathcurveto{\pgfqpoint{1.516257in}{3.264286in}}{\pgfqpoint{1.505658in}{3.268676in}}{\pgfqpoint{1.494607in}{3.268676in}}%
\pgfpathcurveto{\pgfqpoint{1.483557in}{3.268676in}}{\pgfqpoint{1.472958in}{3.264286in}}{\pgfqpoint{1.465145in}{3.256472in}}%
\pgfpathcurveto{\pgfqpoint{1.457331in}{3.248659in}}{\pgfqpoint{1.452941in}{3.238060in}}{\pgfqpoint{1.452941in}{3.227010in}}%
\pgfpathcurveto{\pgfqpoint{1.452941in}{3.215959in}}{\pgfqpoint{1.457331in}{3.205360in}}{\pgfqpoint{1.465145in}{3.197547in}}%
\pgfpathcurveto{\pgfqpoint{1.472958in}{3.189733in}}{\pgfqpoint{1.483557in}{3.185343in}}{\pgfqpoint{1.494607in}{3.185343in}}%
\pgfpathclose%
\pgfusepath{stroke,fill}%
\end{pgfscope}%
\begin{pgfscope}%
\pgfpathrectangle{\pgfqpoint{0.648703in}{0.548769in}}{\pgfqpoint{5.201297in}{3.102590in}}%
\pgfusepath{clip}%
\pgfsetbuttcap%
\pgfsetroundjoin%
\definecolor{currentfill}{rgb}{1.000000,0.498039,0.054902}%
\pgfsetfillcolor{currentfill}%
\pgfsetlinewidth{1.003750pt}%
\definecolor{currentstroke}{rgb}{1.000000,0.498039,0.054902}%
\pgfsetstrokecolor{currentstroke}%
\pgfsetdash{}{0pt}%
\pgfpathmoveto{\pgfqpoint{1.628170in}{3.278374in}}%
\pgfpathcurveto{\pgfqpoint{1.639220in}{3.278374in}}{\pgfqpoint{1.649819in}{3.282764in}}{\pgfqpoint{1.657633in}{3.290578in}}%
\pgfpathcurveto{\pgfqpoint{1.665447in}{3.298392in}}{\pgfqpoint{1.669837in}{3.308991in}}{\pgfqpoint{1.669837in}{3.320041in}}%
\pgfpathcurveto{\pgfqpoint{1.669837in}{3.331091in}}{\pgfqpoint{1.665447in}{3.341690in}}{\pgfqpoint{1.657633in}{3.349504in}}%
\pgfpathcurveto{\pgfqpoint{1.649819in}{3.357317in}}{\pgfqpoint{1.639220in}{3.361707in}}{\pgfqpoint{1.628170in}{3.361707in}}%
\pgfpathcurveto{\pgfqpoint{1.617120in}{3.361707in}}{\pgfqpoint{1.606521in}{3.357317in}}{\pgfqpoint{1.598708in}{3.349504in}}%
\pgfpathcurveto{\pgfqpoint{1.590894in}{3.341690in}}{\pgfqpoint{1.586504in}{3.331091in}}{\pgfqpoint{1.586504in}{3.320041in}}%
\pgfpathcurveto{\pgfqpoint{1.586504in}{3.308991in}}{\pgfqpoint{1.590894in}{3.298392in}}{\pgfqpoint{1.598708in}{3.290578in}}%
\pgfpathcurveto{\pgfqpoint{1.606521in}{3.282764in}}{\pgfqpoint{1.617120in}{3.278374in}}{\pgfqpoint{1.628170in}{3.278374in}}%
\pgfpathclose%
\pgfusepath{stroke,fill}%
\end{pgfscope}%
\begin{pgfscope}%
\pgfpathrectangle{\pgfqpoint{0.648703in}{0.548769in}}{\pgfqpoint{5.201297in}{3.102590in}}%
\pgfusepath{clip}%
\pgfsetbuttcap%
\pgfsetroundjoin%
\definecolor{currentfill}{rgb}{0.121569,0.466667,0.705882}%
\pgfsetfillcolor{currentfill}%
\pgfsetlinewidth{1.003750pt}%
\definecolor{currentstroke}{rgb}{0.121569,0.466667,0.705882}%
\pgfsetstrokecolor{currentstroke}%
\pgfsetdash{}{0pt}%
\pgfpathmoveto{\pgfqpoint{2.142280in}{3.181114in}}%
\pgfpathcurveto{\pgfqpoint{2.153330in}{3.181114in}}{\pgfqpoint{2.163929in}{3.185504in}}{\pgfqpoint{2.171742in}{3.193318in}}%
\pgfpathcurveto{\pgfqpoint{2.179556in}{3.201132in}}{\pgfqpoint{2.183946in}{3.211731in}}{\pgfqpoint{2.183946in}{3.222781in}}%
\pgfpathcurveto{\pgfqpoint{2.183946in}{3.233831in}}{\pgfqpoint{2.179556in}{3.244430in}}{\pgfqpoint{2.171742in}{3.252244in}}%
\pgfpathcurveto{\pgfqpoint{2.163929in}{3.260057in}}{\pgfqpoint{2.153330in}{3.264448in}}{\pgfqpoint{2.142280in}{3.264448in}}%
\pgfpathcurveto{\pgfqpoint{2.131230in}{3.264448in}}{\pgfqpoint{2.120631in}{3.260057in}}{\pgfqpoint{2.112817in}{3.252244in}}%
\pgfpathcurveto{\pgfqpoint{2.105003in}{3.244430in}}{\pgfqpoint{2.100613in}{3.233831in}}{\pgfqpoint{2.100613in}{3.222781in}}%
\pgfpathcurveto{\pgfqpoint{2.100613in}{3.211731in}}{\pgfqpoint{2.105003in}{3.201132in}}{\pgfqpoint{2.112817in}{3.193318in}}%
\pgfpathcurveto{\pgfqpoint{2.120631in}{3.185504in}}{\pgfqpoint{2.131230in}{3.181114in}}{\pgfqpoint{2.142280in}{3.181114in}}%
\pgfpathclose%
\pgfusepath{stroke,fill}%
\end{pgfscope}%
\begin{pgfscope}%
\pgfpathrectangle{\pgfqpoint{0.648703in}{0.548769in}}{\pgfqpoint{5.201297in}{3.102590in}}%
\pgfusepath{clip}%
\pgfsetbuttcap%
\pgfsetroundjoin%
\definecolor{currentfill}{rgb}{1.000000,0.498039,0.054902}%
\pgfsetfillcolor{currentfill}%
\pgfsetlinewidth{1.003750pt}%
\definecolor{currentstroke}{rgb}{1.000000,0.498039,0.054902}%
\pgfsetstrokecolor{currentstroke}%
\pgfsetdash{}{0pt}%
\pgfpathmoveto{\pgfqpoint{1.826728in}{3.219172in}}%
\pgfpathcurveto{\pgfqpoint{1.837778in}{3.219172in}}{\pgfqpoint{1.848377in}{3.223563in}}{\pgfqpoint{1.856190in}{3.231376in}}%
\pgfpathcurveto{\pgfqpoint{1.864004in}{3.239190in}}{\pgfqpoint{1.868394in}{3.249789in}}{\pgfqpoint{1.868394in}{3.260839in}}%
\pgfpathcurveto{\pgfqpoint{1.868394in}{3.271889in}}{\pgfqpoint{1.864004in}{3.282488in}}{\pgfqpoint{1.856190in}{3.290302in}}%
\pgfpathcurveto{\pgfqpoint{1.848377in}{3.298116in}}{\pgfqpoint{1.837778in}{3.302506in}}{\pgfqpoint{1.826728in}{3.302506in}}%
\pgfpathcurveto{\pgfqpoint{1.815678in}{3.302506in}}{\pgfqpoint{1.805078in}{3.298116in}}{\pgfqpoint{1.797265in}{3.290302in}}%
\pgfpathcurveto{\pgfqpoint{1.789451in}{3.282488in}}{\pgfqpoint{1.785061in}{3.271889in}}{\pgfqpoint{1.785061in}{3.260839in}}%
\pgfpathcurveto{\pgfqpoint{1.785061in}{3.249789in}}{\pgfqpoint{1.789451in}{3.239190in}}{\pgfqpoint{1.797265in}{3.231376in}}%
\pgfpathcurveto{\pgfqpoint{1.805078in}{3.223563in}}{\pgfqpoint{1.815678in}{3.219172in}}{\pgfqpoint{1.826728in}{3.219172in}}%
\pgfpathclose%
\pgfusepath{stroke,fill}%
\end{pgfscope}%
\begin{pgfscope}%
\pgfpathrectangle{\pgfqpoint{0.648703in}{0.548769in}}{\pgfqpoint{5.201297in}{3.102590in}}%
\pgfusepath{clip}%
\pgfsetbuttcap%
\pgfsetroundjoin%
\definecolor{currentfill}{rgb}{1.000000,0.498039,0.054902}%
\pgfsetfillcolor{currentfill}%
\pgfsetlinewidth{1.003750pt}%
\definecolor{currentstroke}{rgb}{1.000000,0.498039,0.054902}%
\pgfsetstrokecolor{currentstroke}%
\pgfsetdash{}{0pt}%
\pgfpathmoveto{\pgfqpoint{1.513543in}{3.468665in}}%
\pgfpathcurveto{\pgfqpoint{1.524593in}{3.468665in}}{\pgfqpoint{1.535192in}{3.473055in}}{\pgfqpoint{1.543005in}{3.480869in}}%
\pgfpathcurveto{\pgfqpoint{1.550819in}{3.488683in}}{\pgfqpoint{1.555209in}{3.499282in}}{\pgfqpoint{1.555209in}{3.510332in}}%
\pgfpathcurveto{\pgfqpoint{1.555209in}{3.521382in}}{\pgfqpoint{1.550819in}{3.531981in}}{\pgfqpoint{1.543005in}{3.539795in}}%
\pgfpathcurveto{\pgfqpoint{1.535192in}{3.547608in}}{\pgfqpoint{1.524593in}{3.551998in}}{\pgfqpoint{1.513543in}{3.551998in}}%
\pgfpathcurveto{\pgfqpoint{1.502492in}{3.551998in}}{\pgfqpoint{1.491893in}{3.547608in}}{\pgfqpoint{1.484080in}{3.539795in}}%
\pgfpathcurveto{\pgfqpoint{1.476266in}{3.531981in}}{\pgfqpoint{1.471876in}{3.521382in}}{\pgfqpoint{1.471876in}{3.510332in}}%
\pgfpathcurveto{\pgfqpoint{1.471876in}{3.499282in}}{\pgfqpoint{1.476266in}{3.488683in}}{\pgfqpoint{1.484080in}{3.480869in}}%
\pgfpathcurveto{\pgfqpoint{1.491893in}{3.473055in}}{\pgfqpoint{1.502492in}{3.468665in}}{\pgfqpoint{1.513543in}{3.468665in}}%
\pgfpathclose%
\pgfusepath{stroke,fill}%
\end{pgfscope}%
\begin{pgfscope}%
\pgfpathrectangle{\pgfqpoint{0.648703in}{0.548769in}}{\pgfqpoint{5.201297in}{3.102590in}}%
\pgfusepath{clip}%
\pgfsetbuttcap%
\pgfsetroundjoin%
\definecolor{currentfill}{rgb}{1.000000,0.498039,0.054902}%
\pgfsetfillcolor{currentfill}%
\pgfsetlinewidth{1.003750pt}%
\definecolor{currentstroke}{rgb}{1.000000,0.498039,0.054902}%
\pgfsetstrokecolor{currentstroke}%
\pgfsetdash{}{0pt}%
\pgfpathmoveto{\pgfqpoint{1.504111in}{3.189572in}}%
\pgfpathcurveto{\pgfqpoint{1.515161in}{3.189572in}}{\pgfqpoint{1.525760in}{3.193962in}}{\pgfqpoint{1.533573in}{3.201775in}}%
\pgfpathcurveto{\pgfqpoint{1.541387in}{3.209589in}}{\pgfqpoint{1.545777in}{3.220188in}}{\pgfqpoint{1.545777in}{3.231238in}}%
\pgfpathcurveto{\pgfqpoint{1.545777in}{3.242288in}}{\pgfqpoint{1.541387in}{3.252887in}}{\pgfqpoint{1.533573in}{3.260701in}}%
\pgfpathcurveto{\pgfqpoint{1.525760in}{3.268515in}}{\pgfqpoint{1.515161in}{3.272905in}}{\pgfqpoint{1.504111in}{3.272905in}}%
\pgfpathcurveto{\pgfqpoint{1.493060in}{3.272905in}}{\pgfqpoint{1.482461in}{3.268515in}}{\pgfqpoint{1.474648in}{3.260701in}}%
\pgfpathcurveto{\pgfqpoint{1.466834in}{3.252887in}}{\pgfqpoint{1.462444in}{3.242288in}}{\pgfqpoint{1.462444in}{3.231238in}}%
\pgfpathcurveto{\pgfqpoint{1.462444in}{3.220188in}}{\pgfqpoint{1.466834in}{3.209589in}}{\pgfqpoint{1.474648in}{3.201775in}}%
\pgfpathcurveto{\pgfqpoint{1.482461in}{3.193962in}}{\pgfqpoint{1.493060in}{3.189572in}}{\pgfqpoint{1.504111in}{3.189572in}}%
\pgfpathclose%
\pgfusepath{stroke,fill}%
\end{pgfscope}%
\begin{pgfscope}%
\pgfpathrectangle{\pgfqpoint{0.648703in}{0.548769in}}{\pgfqpoint{5.201297in}{3.102590in}}%
\pgfusepath{clip}%
\pgfsetbuttcap%
\pgfsetroundjoin%
\definecolor{currentfill}{rgb}{1.000000,0.498039,0.054902}%
\pgfsetfillcolor{currentfill}%
\pgfsetlinewidth{1.003750pt}%
\definecolor{currentstroke}{rgb}{1.000000,0.498039,0.054902}%
\pgfsetstrokecolor{currentstroke}%
\pgfsetdash{}{0pt}%
\pgfpathmoveto{\pgfqpoint{1.769459in}{3.189572in}}%
\pgfpathcurveto{\pgfqpoint{1.780509in}{3.189572in}}{\pgfqpoint{1.791108in}{3.193962in}}{\pgfqpoint{1.798922in}{3.201775in}}%
\pgfpathcurveto{\pgfqpoint{1.806736in}{3.209589in}}{\pgfqpoint{1.811126in}{3.220188in}}{\pgfqpoint{1.811126in}{3.231238in}}%
\pgfpathcurveto{\pgfqpoint{1.811126in}{3.242288in}}{\pgfqpoint{1.806736in}{3.252887in}}{\pgfqpoint{1.798922in}{3.260701in}}%
\pgfpathcurveto{\pgfqpoint{1.791108in}{3.268515in}}{\pgfqpoint{1.780509in}{3.272905in}}{\pgfqpoint{1.769459in}{3.272905in}}%
\pgfpathcurveto{\pgfqpoint{1.758409in}{3.272905in}}{\pgfqpoint{1.747810in}{3.268515in}}{\pgfqpoint{1.739996in}{3.260701in}}%
\pgfpathcurveto{\pgfqpoint{1.732183in}{3.252887in}}{\pgfqpoint{1.727793in}{3.242288in}}{\pgfqpoint{1.727793in}{3.231238in}}%
\pgfpathcurveto{\pgfqpoint{1.727793in}{3.220188in}}{\pgfqpoint{1.732183in}{3.209589in}}{\pgfqpoint{1.739996in}{3.201775in}}%
\pgfpathcurveto{\pgfqpoint{1.747810in}{3.193962in}}{\pgfqpoint{1.758409in}{3.189572in}}{\pgfqpoint{1.769459in}{3.189572in}}%
\pgfpathclose%
\pgfusepath{stroke,fill}%
\end{pgfscope}%
\begin{pgfscope}%
\pgfpathrectangle{\pgfqpoint{0.648703in}{0.548769in}}{\pgfqpoint{5.201297in}{3.102590in}}%
\pgfusepath{clip}%
\pgfsetbuttcap%
\pgfsetroundjoin%
\definecolor{currentfill}{rgb}{0.121569,0.466667,0.705882}%
\pgfsetfillcolor{currentfill}%
\pgfsetlinewidth{1.003750pt}%
\definecolor{currentstroke}{rgb}{0.121569,0.466667,0.705882}%
\pgfsetstrokecolor{currentstroke}%
\pgfsetdash{}{0pt}%
\pgfpathmoveto{\pgfqpoint{2.454174in}{3.181114in}}%
\pgfpathcurveto{\pgfqpoint{2.465225in}{3.181114in}}{\pgfqpoint{2.475824in}{3.185504in}}{\pgfqpoint{2.483637in}{3.193318in}}%
\pgfpathcurveto{\pgfqpoint{2.491451in}{3.201132in}}{\pgfqpoint{2.495841in}{3.211731in}}{\pgfqpoint{2.495841in}{3.222781in}}%
\pgfpathcurveto{\pgfqpoint{2.495841in}{3.233831in}}{\pgfqpoint{2.491451in}{3.244430in}}{\pgfqpoint{2.483637in}{3.252244in}}%
\pgfpathcurveto{\pgfqpoint{2.475824in}{3.260057in}}{\pgfqpoint{2.465225in}{3.264448in}}{\pgfqpoint{2.454174in}{3.264448in}}%
\pgfpathcurveto{\pgfqpoint{2.443124in}{3.264448in}}{\pgfqpoint{2.432525in}{3.260057in}}{\pgfqpoint{2.424712in}{3.252244in}}%
\pgfpathcurveto{\pgfqpoint{2.416898in}{3.244430in}}{\pgfqpoint{2.412508in}{3.233831in}}{\pgfqpoint{2.412508in}{3.222781in}}%
\pgfpathcurveto{\pgfqpoint{2.412508in}{3.211731in}}{\pgfqpoint{2.416898in}{3.201132in}}{\pgfqpoint{2.424712in}{3.193318in}}%
\pgfpathcurveto{\pgfqpoint{2.432525in}{3.185504in}}{\pgfqpoint{2.443124in}{3.181114in}}{\pgfqpoint{2.454174in}{3.181114in}}%
\pgfpathclose%
\pgfusepath{stroke,fill}%
\end{pgfscope}%
\begin{pgfscope}%
\pgfpathrectangle{\pgfqpoint{0.648703in}{0.548769in}}{\pgfqpoint{5.201297in}{3.102590in}}%
\pgfusepath{clip}%
\pgfsetbuttcap%
\pgfsetroundjoin%
\definecolor{currentfill}{rgb}{1.000000,0.498039,0.054902}%
\pgfsetfillcolor{currentfill}%
\pgfsetlinewidth{1.003750pt}%
\definecolor{currentstroke}{rgb}{1.000000,0.498039,0.054902}%
\pgfsetstrokecolor{currentstroke}%
\pgfsetdash{}{0pt}%
\pgfpathmoveto{\pgfqpoint{1.940315in}{3.185343in}}%
\pgfpathcurveto{\pgfqpoint{1.951365in}{3.185343in}}{\pgfqpoint{1.961964in}{3.189733in}}{\pgfqpoint{1.969777in}{3.197547in}}%
\pgfpathcurveto{\pgfqpoint{1.977591in}{3.205360in}}{\pgfqpoint{1.981981in}{3.215959in}}{\pgfqpoint{1.981981in}{3.227010in}}%
\pgfpathcurveto{\pgfqpoint{1.981981in}{3.238060in}}{\pgfqpoint{1.977591in}{3.248659in}}{\pgfqpoint{1.969777in}{3.256472in}}%
\pgfpathcurveto{\pgfqpoint{1.961964in}{3.264286in}}{\pgfqpoint{1.951365in}{3.268676in}}{\pgfqpoint{1.940315in}{3.268676in}}%
\pgfpathcurveto{\pgfqpoint{1.929264in}{3.268676in}}{\pgfqpoint{1.918665in}{3.264286in}}{\pgfqpoint{1.910852in}{3.256472in}}%
\pgfpathcurveto{\pgfqpoint{1.903038in}{3.248659in}}{\pgfqpoint{1.898648in}{3.238060in}}{\pgfqpoint{1.898648in}{3.227010in}}%
\pgfpathcurveto{\pgfqpoint{1.898648in}{3.215959in}}{\pgfqpoint{1.903038in}{3.205360in}}{\pgfqpoint{1.910852in}{3.197547in}}%
\pgfpathcurveto{\pgfqpoint{1.918665in}{3.189733in}}{\pgfqpoint{1.929264in}{3.185343in}}{\pgfqpoint{1.940315in}{3.185343in}}%
\pgfpathclose%
\pgfusepath{stroke,fill}%
\end{pgfscope}%
\begin{pgfscope}%
\pgfpathrectangle{\pgfqpoint{0.648703in}{0.548769in}}{\pgfqpoint{5.201297in}{3.102590in}}%
\pgfusepath{clip}%
\pgfsetbuttcap%
\pgfsetroundjoin%
\definecolor{currentfill}{rgb}{1.000000,0.498039,0.054902}%
\pgfsetfillcolor{currentfill}%
\pgfsetlinewidth{1.003750pt}%
\definecolor{currentstroke}{rgb}{1.000000,0.498039,0.054902}%
\pgfsetstrokecolor{currentstroke}%
\pgfsetdash{}{0pt}%
\pgfpathmoveto{\pgfqpoint{2.129642in}{3.358719in}}%
\pgfpathcurveto{\pgfqpoint{2.140692in}{3.358719in}}{\pgfqpoint{2.151291in}{3.363109in}}{\pgfqpoint{2.159105in}{3.370923in}}%
\pgfpathcurveto{\pgfqpoint{2.166918in}{3.378737in}}{\pgfqpoint{2.171308in}{3.389336in}}{\pgfqpoint{2.171308in}{3.400386in}}%
\pgfpathcurveto{\pgfqpoint{2.171308in}{3.411436in}}{\pgfqpoint{2.166918in}{3.422035in}}{\pgfqpoint{2.159105in}{3.429849in}}%
\pgfpathcurveto{\pgfqpoint{2.151291in}{3.437662in}}{\pgfqpoint{2.140692in}{3.442053in}}{\pgfqpoint{2.129642in}{3.442053in}}%
\pgfpathcurveto{\pgfqpoint{2.118592in}{3.442053in}}{\pgfqpoint{2.107993in}{3.437662in}}{\pgfqpoint{2.100179in}{3.429849in}}%
\pgfpathcurveto{\pgfqpoint{2.092365in}{3.422035in}}{\pgfqpoint{2.087975in}{3.411436in}}{\pgfqpoint{2.087975in}{3.400386in}}%
\pgfpathcurveto{\pgfqpoint{2.087975in}{3.389336in}}{\pgfqpoint{2.092365in}{3.378737in}}{\pgfqpoint{2.100179in}{3.370923in}}%
\pgfpathcurveto{\pgfqpoint{2.107993in}{3.363109in}}{\pgfqpoint{2.118592in}{3.358719in}}{\pgfqpoint{2.129642in}{3.358719in}}%
\pgfpathclose%
\pgfusepath{stroke,fill}%
\end{pgfscope}%
\begin{pgfscope}%
\pgfpathrectangle{\pgfqpoint{0.648703in}{0.548769in}}{\pgfqpoint{5.201297in}{3.102590in}}%
\pgfusepath{clip}%
\pgfsetbuttcap%
\pgfsetroundjoin%
\definecolor{currentfill}{rgb}{1.000000,0.498039,0.054902}%
\pgfsetfillcolor{currentfill}%
\pgfsetlinewidth{1.003750pt}%
\definecolor{currentstroke}{rgb}{1.000000,0.498039,0.054902}%
\pgfsetstrokecolor{currentstroke}%
\pgfsetdash{}{0pt}%
\pgfpathmoveto{\pgfqpoint{2.312050in}{3.198029in}}%
\pgfpathcurveto{\pgfqpoint{2.323101in}{3.198029in}}{\pgfqpoint{2.333700in}{3.202419in}}{\pgfqpoint{2.341513in}{3.210233in}}%
\pgfpathcurveto{\pgfqpoint{2.349327in}{3.218046in}}{\pgfqpoint{2.353717in}{3.228646in}}{\pgfqpoint{2.353717in}{3.239696in}}%
\pgfpathcurveto{\pgfqpoint{2.353717in}{3.250746in}}{\pgfqpoint{2.349327in}{3.261345in}}{\pgfqpoint{2.341513in}{3.269158in}}%
\pgfpathcurveto{\pgfqpoint{2.333700in}{3.276972in}}{\pgfqpoint{2.323101in}{3.281362in}}{\pgfqpoint{2.312050in}{3.281362in}}%
\pgfpathcurveto{\pgfqpoint{2.301000in}{3.281362in}}{\pgfqpoint{2.290401in}{3.276972in}}{\pgfqpoint{2.282588in}{3.269158in}}%
\pgfpathcurveto{\pgfqpoint{2.274774in}{3.261345in}}{\pgfqpoint{2.270384in}{3.250746in}}{\pgfqpoint{2.270384in}{3.239696in}}%
\pgfpathcurveto{\pgfqpoint{2.270384in}{3.228646in}}{\pgfqpoint{2.274774in}{3.218046in}}{\pgfqpoint{2.282588in}{3.210233in}}%
\pgfpathcurveto{\pgfqpoint{2.290401in}{3.202419in}}{\pgfqpoint{2.301000in}{3.198029in}}{\pgfqpoint{2.312050in}{3.198029in}}%
\pgfpathclose%
\pgfusepath{stroke,fill}%
\end{pgfscope}%
\begin{pgfscope}%
\pgfpathrectangle{\pgfqpoint{0.648703in}{0.548769in}}{\pgfqpoint{5.201297in}{3.102590in}}%
\pgfusepath{clip}%
\pgfsetbuttcap%
\pgfsetroundjoin%
\definecolor{currentfill}{rgb}{0.839216,0.152941,0.156863}%
\pgfsetfillcolor{currentfill}%
\pgfsetlinewidth{1.003750pt}%
\definecolor{currentstroke}{rgb}{0.839216,0.152941,0.156863}%
\pgfsetstrokecolor{currentstroke}%
\pgfsetdash{}{0pt}%
\pgfpathmoveto{\pgfqpoint{1.396477in}{3.193800in}}%
\pgfpathcurveto{\pgfqpoint{1.407527in}{3.193800in}}{\pgfqpoint{1.418126in}{3.198191in}}{\pgfqpoint{1.425939in}{3.206004in}}%
\pgfpathcurveto{\pgfqpoint{1.433753in}{3.213818in}}{\pgfqpoint{1.438143in}{3.224417in}}{\pgfqpoint{1.438143in}{3.235467in}}%
\pgfpathcurveto{\pgfqpoint{1.438143in}{3.246517in}}{\pgfqpoint{1.433753in}{3.257116in}}{\pgfqpoint{1.425939in}{3.264930in}}%
\pgfpathcurveto{\pgfqpoint{1.418126in}{3.272743in}}{\pgfqpoint{1.407527in}{3.277134in}}{\pgfqpoint{1.396477in}{3.277134in}}%
\pgfpathcurveto{\pgfqpoint{1.385426in}{3.277134in}}{\pgfqpoint{1.374827in}{3.272743in}}{\pgfqpoint{1.367014in}{3.264930in}}%
\pgfpathcurveto{\pgfqpoint{1.359200in}{3.257116in}}{\pgfqpoint{1.354810in}{3.246517in}}{\pgfqpoint{1.354810in}{3.235467in}}%
\pgfpathcurveto{\pgfqpoint{1.354810in}{3.224417in}}{\pgfqpoint{1.359200in}{3.213818in}}{\pgfqpoint{1.367014in}{3.206004in}}%
\pgfpathcurveto{\pgfqpoint{1.374827in}{3.198191in}}{\pgfqpoint{1.385426in}{3.193800in}}{\pgfqpoint{1.396477in}{3.193800in}}%
\pgfpathclose%
\pgfusepath{stroke,fill}%
\end{pgfscope}%
\begin{pgfscope}%
\pgfpathrectangle{\pgfqpoint{0.648703in}{0.548769in}}{\pgfqpoint{5.201297in}{3.102590in}}%
\pgfusepath{clip}%
\pgfsetbuttcap%
\pgfsetroundjoin%
\definecolor{currentfill}{rgb}{1.000000,0.498039,0.054902}%
\pgfsetfillcolor{currentfill}%
\pgfsetlinewidth{1.003750pt}%
\definecolor{currentstroke}{rgb}{1.000000,0.498039,0.054902}%
\pgfsetstrokecolor{currentstroke}%
\pgfsetdash{}{0pt}%
\pgfpathmoveto{\pgfqpoint{2.760639in}{3.193800in}}%
\pgfpathcurveto{\pgfqpoint{2.771689in}{3.193800in}}{\pgfqpoint{2.782288in}{3.198191in}}{\pgfqpoint{2.790102in}{3.206004in}}%
\pgfpathcurveto{\pgfqpoint{2.797915in}{3.213818in}}{\pgfqpoint{2.802306in}{3.224417in}}{\pgfqpoint{2.802306in}{3.235467in}}%
\pgfpathcurveto{\pgfqpoint{2.802306in}{3.246517in}}{\pgfqpoint{2.797915in}{3.257116in}}{\pgfqpoint{2.790102in}{3.264930in}}%
\pgfpathcurveto{\pgfqpoint{2.782288in}{3.272743in}}{\pgfqpoint{2.771689in}{3.277134in}}{\pgfqpoint{2.760639in}{3.277134in}}%
\pgfpathcurveto{\pgfqpoint{2.749589in}{3.277134in}}{\pgfqpoint{2.738990in}{3.272743in}}{\pgfqpoint{2.731176in}{3.264930in}}%
\pgfpathcurveto{\pgfqpoint{2.723363in}{3.257116in}}{\pgfqpoint{2.718972in}{3.246517in}}{\pgfqpoint{2.718972in}{3.235467in}}%
\pgfpathcurveto{\pgfqpoint{2.718972in}{3.224417in}}{\pgfqpoint{2.723363in}{3.213818in}}{\pgfqpoint{2.731176in}{3.206004in}}%
\pgfpathcurveto{\pgfqpoint{2.738990in}{3.198191in}}{\pgfqpoint{2.749589in}{3.193800in}}{\pgfqpoint{2.760639in}{3.193800in}}%
\pgfpathclose%
\pgfusepath{stroke,fill}%
\end{pgfscope}%
\begin{pgfscope}%
\pgfpathrectangle{\pgfqpoint{0.648703in}{0.548769in}}{\pgfqpoint{5.201297in}{3.102590in}}%
\pgfusepath{clip}%
\pgfsetbuttcap%
\pgfsetroundjoin%
\definecolor{currentfill}{rgb}{1.000000,0.498039,0.054902}%
\pgfsetfillcolor{currentfill}%
\pgfsetlinewidth{1.003750pt}%
\definecolor{currentstroke}{rgb}{1.000000,0.498039,0.054902}%
\pgfsetstrokecolor{currentstroke}%
\pgfsetdash{}{0pt}%
\pgfpathmoveto{\pgfqpoint{1.942400in}{3.185343in}}%
\pgfpathcurveto{\pgfqpoint{1.953451in}{3.185343in}}{\pgfqpoint{1.964050in}{3.189733in}}{\pgfqpoint{1.971863in}{3.197547in}}%
\pgfpathcurveto{\pgfqpoint{1.979677in}{3.205360in}}{\pgfqpoint{1.984067in}{3.215959in}}{\pgfqpoint{1.984067in}{3.227010in}}%
\pgfpathcurveto{\pgfqpoint{1.984067in}{3.238060in}}{\pgfqpoint{1.979677in}{3.248659in}}{\pgfqpoint{1.971863in}{3.256472in}}%
\pgfpathcurveto{\pgfqpoint{1.964050in}{3.264286in}}{\pgfqpoint{1.953451in}{3.268676in}}{\pgfqpoint{1.942400in}{3.268676in}}%
\pgfpathcurveto{\pgfqpoint{1.931350in}{3.268676in}}{\pgfqpoint{1.920751in}{3.264286in}}{\pgfqpoint{1.912938in}{3.256472in}}%
\pgfpathcurveto{\pgfqpoint{1.905124in}{3.248659in}}{\pgfqpoint{1.900734in}{3.238060in}}{\pgfqpoint{1.900734in}{3.227010in}}%
\pgfpathcurveto{\pgfqpoint{1.900734in}{3.215959in}}{\pgfqpoint{1.905124in}{3.205360in}}{\pgfqpoint{1.912938in}{3.197547in}}%
\pgfpathcurveto{\pgfqpoint{1.920751in}{3.189733in}}{\pgfqpoint{1.931350in}{3.185343in}}{\pgfqpoint{1.942400in}{3.185343in}}%
\pgfpathclose%
\pgfusepath{stroke,fill}%
\end{pgfscope}%
\begin{pgfscope}%
\pgfpathrectangle{\pgfqpoint{0.648703in}{0.548769in}}{\pgfqpoint{5.201297in}{3.102590in}}%
\pgfusepath{clip}%
\pgfsetbuttcap%
\pgfsetroundjoin%
\definecolor{currentfill}{rgb}{1.000000,0.498039,0.054902}%
\pgfsetfillcolor{currentfill}%
\pgfsetlinewidth{1.003750pt}%
\definecolor{currentstroke}{rgb}{1.000000,0.498039,0.054902}%
\pgfsetstrokecolor{currentstroke}%
\pgfsetdash{}{0pt}%
\pgfpathmoveto{\pgfqpoint{1.942286in}{3.185343in}}%
\pgfpathcurveto{\pgfqpoint{1.953336in}{3.185343in}}{\pgfqpoint{1.963935in}{3.189733in}}{\pgfqpoint{1.971748in}{3.197547in}}%
\pgfpathcurveto{\pgfqpoint{1.979562in}{3.205360in}}{\pgfqpoint{1.983952in}{3.215959in}}{\pgfqpoint{1.983952in}{3.227010in}}%
\pgfpathcurveto{\pgfqpoint{1.983952in}{3.238060in}}{\pgfqpoint{1.979562in}{3.248659in}}{\pgfqpoint{1.971748in}{3.256472in}}%
\pgfpathcurveto{\pgfqpoint{1.963935in}{3.264286in}}{\pgfqpoint{1.953336in}{3.268676in}}{\pgfqpoint{1.942286in}{3.268676in}}%
\pgfpathcurveto{\pgfqpoint{1.931235in}{3.268676in}}{\pgfqpoint{1.920636in}{3.264286in}}{\pgfqpoint{1.912823in}{3.256472in}}%
\pgfpathcurveto{\pgfqpoint{1.905009in}{3.248659in}}{\pgfqpoint{1.900619in}{3.238060in}}{\pgfqpoint{1.900619in}{3.227010in}}%
\pgfpathcurveto{\pgfqpoint{1.900619in}{3.215959in}}{\pgfqpoint{1.905009in}{3.205360in}}{\pgfqpoint{1.912823in}{3.197547in}}%
\pgfpathcurveto{\pgfqpoint{1.920636in}{3.189733in}}{\pgfqpoint{1.931235in}{3.185343in}}{\pgfqpoint{1.942286in}{3.185343in}}%
\pgfpathclose%
\pgfusepath{stroke,fill}%
\end{pgfscope}%
\begin{pgfscope}%
\pgfpathrectangle{\pgfqpoint{0.648703in}{0.548769in}}{\pgfqpoint{5.201297in}{3.102590in}}%
\pgfusepath{clip}%
\pgfsetbuttcap%
\pgfsetroundjoin%
\definecolor{currentfill}{rgb}{1.000000,0.498039,0.054902}%
\pgfsetfillcolor{currentfill}%
\pgfsetlinewidth{1.003750pt}%
\definecolor{currentstroke}{rgb}{1.000000,0.498039,0.054902}%
\pgfsetstrokecolor{currentstroke}%
\pgfsetdash{}{0pt}%
\pgfpathmoveto{\pgfqpoint{2.434555in}{3.193800in}}%
\pgfpathcurveto{\pgfqpoint{2.445605in}{3.193800in}}{\pgfqpoint{2.456204in}{3.198191in}}{\pgfqpoint{2.464017in}{3.206004in}}%
\pgfpathcurveto{\pgfqpoint{2.471831in}{3.213818in}}{\pgfqpoint{2.476221in}{3.224417in}}{\pgfqpoint{2.476221in}{3.235467in}}%
\pgfpathcurveto{\pgfqpoint{2.476221in}{3.246517in}}{\pgfqpoint{2.471831in}{3.257116in}}{\pgfqpoint{2.464017in}{3.264930in}}%
\pgfpathcurveto{\pgfqpoint{2.456204in}{3.272743in}}{\pgfqpoint{2.445605in}{3.277134in}}{\pgfqpoint{2.434555in}{3.277134in}}%
\pgfpathcurveto{\pgfqpoint{2.423505in}{3.277134in}}{\pgfqpoint{2.412905in}{3.272743in}}{\pgfqpoint{2.405092in}{3.264930in}}%
\pgfpathcurveto{\pgfqpoint{2.397278in}{3.257116in}}{\pgfqpoint{2.392888in}{3.246517in}}{\pgfqpoint{2.392888in}{3.235467in}}%
\pgfpathcurveto{\pgfqpoint{2.392888in}{3.224417in}}{\pgfqpoint{2.397278in}{3.213818in}}{\pgfqpoint{2.405092in}{3.206004in}}%
\pgfpathcurveto{\pgfqpoint{2.412905in}{3.198191in}}{\pgfqpoint{2.423505in}{3.193800in}}{\pgfqpoint{2.434555in}{3.193800in}}%
\pgfpathclose%
\pgfusepath{stroke,fill}%
\end{pgfscope}%
\begin{pgfscope}%
\pgfpathrectangle{\pgfqpoint{0.648703in}{0.548769in}}{\pgfqpoint{5.201297in}{3.102590in}}%
\pgfusepath{clip}%
\pgfsetbuttcap%
\pgfsetroundjoin%
\definecolor{currentfill}{rgb}{0.839216,0.152941,0.156863}%
\pgfsetfillcolor{currentfill}%
\pgfsetlinewidth{1.003750pt}%
\definecolor{currentstroke}{rgb}{0.839216,0.152941,0.156863}%
\pgfsetstrokecolor{currentstroke}%
\pgfsetdash{}{0pt}%
\pgfpathmoveto{\pgfqpoint{2.079886in}{3.189572in}}%
\pgfpathcurveto{\pgfqpoint{2.090936in}{3.189572in}}{\pgfqpoint{2.101535in}{3.193962in}}{\pgfqpoint{2.109348in}{3.201775in}}%
\pgfpathcurveto{\pgfqpoint{2.117162in}{3.209589in}}{\pgfqpoint{2.121552in}{3.220188in}}{\pgfqpoint{2.121552in}{3.231238in}}%
\pgfpathcurveto{\pgfqpoint{2.121552in}{3.242288in}}{\pgfqpoint{2.117162in}{3.252887in}}{\pgfqpoint{2.109348in}{3.260701in}}%
\pgfpathcurveto{\pgfqpoint{2.101535in}{3.268515in}}{\pgfqpoint{2.090936in}{3.272905in}}{\pgfqpoint{2.079886in}{3.272905in}}%
\pgfpathcurveto{\pgfqpoint{2.068836in}{3.272905in}}{\pgfqpoint{2.058237in}{3.268515in}}{\pgfqpoint{2.050423in}{3.260701in}}%
\pgfpathcurveto{\pgfqpoint{2.042609in}{3.252887in}}{\pgfqpoint{2.038219in}{3.242288in}}{\pgfqpoint{2.038219in}{3.231238in}}%
\pgfpathcurveto{\pgfqpoint{2.038219in}{3.220188in}}{\pgfqpoint{2.042609in}{3.209589in}}{\pgfqpoint{2.050423in}{3.201775in}}%
\pgfpathcurveto{\pgfqpoint{2.058237in}{3.193962in}}{\pgfqpoint{2.068836in}{3.189572in}}{\pgfqpoint{2.079886in}{3.189572in}}%
\pgfpathclose%
\pgfusepath{stroke,fill}%
\end{pgfscope}%
\begin{pgfscope}%
\pgfpathrectangle{\pgfqpoint{0.648703in}{0.548769in}}{\pgfqpoint{5.201297in}{3.102590in}}%
\pgfusepath{clip}%
\pgfsetbuttcap%
\pgfsetroundjoin%
\definecolor{currentfill}{rgb}{1.000000,0.498039,0.054902}%
\pgfsetfillcolor{currentfill}%
\pgfsetlinewidth{1.003750pt}%
\definecolor{currentstroke}{rgb}{1.000000,0.498039,0.054902}%
\pgfsetstrokecolor{currentstroke}%
\pgfsetdash{}{0pt}%
\pgfpathmoveto{\pgfqpoint{2.143681in}{3.198029in}}%
\pgfpathcurveto{\pgfqpoint{2.154731in}{3.198029in}}{\pgfqpoint{2.165330in}{3.202419in}}{\pgfqpoint{2.173144in}{3.210233in}}%
\pgfpathcurveto{\pgfqpoint{2.180957in}{3.218046in}}{\pgfqpoint{2.185347in}{3.228646in}}{\pgfqpoint{2.185347in}{3.239696in}}%
\pgfpathcurveto{\pgfqpoint{2.185347in}{3.250746in}}{\pgfqpoint{2.180957in}{3.261345in}}{\pgfqpoint{2.173144in}{3.269158in}}%
\pgfpathcurveto{\pgfqpoint{2.165330in}{3.276972in}}{\pgfqpoint{2.154731in}{3.281362in}}{\pgfqpoint{2.143681in}{3.281362in}}%
\pgfpathcurveto{\pgfqpoint{2.132631in}{3.281362in}}{\pgfqpoint{2.122032in}{3.276972in}}{\pgfqpoint{2.114218in}{3.269158in}}%
\pgfpathcurveto{\pgfqpoint{2.106404in}{3.261345in}}{\pgfqpoint{2.102014in}{3.250746in}}{\pgfqpoint{2.102014in}{3.239696in}}%
\pgfpathcurveto{\pgfqpoint{2.102014in}{3.228646in}}{\pgfqpoint{2.106404in}{3.218046in}}{\pgfqpoint{2.114218in}{3.210233in}}%
\pgfpathcurveto{\pgfqpoint{2.122032in}{3.202419in}}{\pgfqpoint{2.132631in}{3.198029in}}{\pgfqpoint{2.143681in}{3.198029in}}%
\pgfpathclose%
\pgfusepath{stroke,fill}%
\end{pgfscope}%
\begin{pgfscope}%
\pgfpathrectangle{\pgfqpoint{0.648703in}{0.548769in}}{\pgfqpoint{5.201297in}{3.102590in}}%
\pgfusepath{clip}%
\pgfsetbuttcap%
\pgfsetroundjoin%
\definecolor{currentfill}{rgb}{0.121569,0.466667,0.705882}%
\pgfsetfillcolor{currentfill}%
\pgfsetlinewidth{1.003750pt}%
\definecolor{currentstroke}{rgb}{0.121569,0.466667,0.705882}%
\pgfsetstrokecolor{currentstroke}%
\pgfsetdash{}{0pt}%
\pgfpathmoveto{\pgfqpoint{2.057911in}{3.181114in}}%
\pgfpathcurveto{\pgfqpoint{2.068961in}{3.181114in}}{\pgfqpoint{2.079560in}{3.185504in}}{\pgfqpoint{2.087374in}{3.193318in}}%
\pgfpathcurveto{\pgfqpoint{2.095187in}{3.201132in}}{\pgfqpoint{2.099577in}{3.211731in}}{\pgfqpoint{2.099577in}{3.222781in}}%
\pgfpathcurveto{\pgfqpoint{2.099577in}{3.233831in}}{\pgfqpoint{2.095187in}{3.244430in}}{\pgfqpoint{2.087374in}{3.252244in}}%
\pgfpathcurveto{\pgfqpoint{2.079560in}{3.260057in}}{\pgfqpoint{2.068961in}{3.264448in}}{\pgfqpoint{2.057911in}{3.264448in}}%
\pgfpathcurveto{\pgfqpoint{2.046861in}{3.264448in}}{\pgfqpoint{2.036262in}{3.260057in}}{\pgfqpoint{2.028448in}{3.252244in}}%
\pgfpathcurveto{\pgfqpoint{2.020634in}{3.244430in}}{\pgfqpoint{2.016244in}{3.233831in}}{\pgfqpoint{2.016244in}{3.222781in}}%
\pgfpathcurveto{\pgfqpoint{2.016244in}{3.211731in}}{\pgfqpoint{2.020634in}{3.201132in}}{\pgfqpoint{2.028448in}{3.193318in}}%
\pgfpathcurveto{\pgfqpoint{2.036262in}{3.185504in}}{\pgfqpoint{2.046861in}{3.181114in}}{\pgfqpoint{2.057911in}{3.181114in}}%
\pgfpathclose%
\pgfusepath{stroke,fill}%
\end{pgfscope}%
\begin{pgfscope}%
\pgfpathrectangle{\pgfqpoint{0.648703in}{0.548769in}}{\pgfqpoint{5.201297in}{3.102590in}}%
\pgfusepath{clip}%
\pgfsetbuttcap%
\pgfsetroundjoin%
\definecolor{currentfill}{rgb}{0.121569,0.466667,0.705882}%
\pgfsetfillcolor{currentfill}%
\pgfsetlinewidth{1.003750pt}%
\definecolor{currentstroke}{rgb}{0.121569,0.466667,0.705882}%
\pgfsetstrokecolor{currentstroke}%
\pgfsetdash{}{0pt}%
\pgfpathmoveto{\pgfqpoint{1.052819in}{0.939909in}}%
\pgfpathcurveto{\pgfqpoint{1.063869in}{0.939909in}}{\pgfqpoint{1.074468in}{0.944299in}}{\pgfqpoint{1.082282in}{0.952112in}}%
\pgfpathcurveto{\pgfqpoint{1.090095in}{0.959926in}}{\pgfqpoint{1.094485in}{0.970525in}}{\pgfqpoint{1.094485in}{0.981575in}}%
\pgfpathcurveto{\pgfqpoint{1.094485in}{0.992625in}}{\pgfqpoint{1.090095in}{1.003224in}}{\pgfqpoint{1.082282in}{1.011038in}}%
\pgfpathcurveto{\pgfqpoint{1.074468in}{1.018852in}}{\pgfqpoint{1.063869in}{1.023242in}}{\pgfqpoint{1.052819in}{1.023242in}}%
\pgfpathcurveto{\pgfqpoint{1.041769in}{1.023242in}}{\pgfqpoint{1.031170in}{1.018852in}}{\pgfqpoint{1.023356in}{1.011038in}}%
\pgfpathcurveto{\pgfqpoint{1.015542in}{1.003224in}}{\pgfqpoint{1.011152in}{0.992625in}}{\pgfqpoint{1.011152in}{0.981575in}}%
\pgfpathcurveto{\pgfqpoint{1.011152in}{0.970525in}}{\pgfqpoint{1.015542in}{0.959926in}}{\pgfqpoint{1.023356in}{0.952112in}}%
\pgfpathcurveto{\pgfqpoint{1.031170in}{0.944299in}}{\pgfqpoint{1.041769in}{0.939909in}}{\pgfqpoint{1.052819in}{0.939909in}}%
\pgfpathclose%
\pgfusepath{stroke,fill}%
\end{pgfscope}%
\begin{pgfscope}%
\pgfpathrectangle{\pgfqpoint{0.648703in}{0.548769in}}{\pgfqpoint{5.201297in}{3.102590in}}%
\pgfusepath{clip}%
\pgfsetbuttcap%
\pgfsetroundjoin%
\definecolor{currentfill}{rgb}{0.839216,0.152941,0.156863}%
\pgfsetfillcolor{currentfill}%
\pgfsetlinewidth{1.003750pt}%
\definecolor{currentstroke}{rgb}{0.839216,0.152941,0.156863}%
\pgfsetstrokecolor{currentstroke}%
\pgfsetdash{}{0pt}%
\pgfpathmoveto{\pgfqpoint{1.946339in}{3.210715in}}%
\pgfpathcurveto{\pgfqpoint{1.957389in}{3.210715in}}{\pgfqpoint{1.967988in}{3.215105in}}{\pgfqpoint{1.975801in}{3.222919in}}%
\pgfpathcurveto{\pgfqpoint{1.983615in}{3.230733in}}{\pgfqpoint{1.988005in}{3.241332in}}{\pgfqpoint{1.988005in}{3.252382in}}%
\pgfpathcurveto{\pgfqpoint{1.988005in}{3.263432in}}{\pgfqpoint{1.983615in}{3.274031in}}{\pgfqpoint{1.975801in}{3.281844in}}%
\pgfpathcurveto{\pgfqpoint{1.967988in}{3.289658in}}{\pgfqpoint{1.957389in}{3.294048in}}{\pgfqpoint{1.946339in}{3.294048in}}%
\pgfpathcurveto{\pgfqpoint{1.935288in}{3.294048in}}{\pgfqpoint{1.924689in}{3.289658in}}{\pgfqpoint{1.916876in}{3.281844in}}%
\pgfpathcurveto{\pgfqpoint{1.909062in}{3.274031in}}{\pgfqpoint{1.904672in}{3.263432in}}{\pgfqpoint{1.904672in}{3.252382in}}%
\pgfpathcurveto{\pgfqpoint{1.904672in}{3.241332in}}{\pgfqpoint{1.909062in}{3.230733in}}{\pgfqpoint{1.916876in}{3.222919in}}%
\pgfpathcurveto{\pgfqpoint{1.924689in}{3.215105in}}{\pgfqpoint{1.935288in}{3.210715in}}{\pgfqpoint{1.946339in}{3.210715in}}%
\pgfpathclose%
\pgfusepath{stroke,fill}%
\end{pgfscope}%
\begin{pgfscope}%
\pgfpathrectangle{\pgfqpoint{0.648703in}{0.548769in}}{\pgfqpoint{5.201297in}{3.102590in}}%
\pgfusepath{clip}%
\pgfsetbuttcap%
\pgfsetroundjoin%
\definecolor{currentfill}{rgb}{1.000000,0.498039,0.054902}%
\pgfsetfillcolor{currentfill}%
\pgfsetlinewidth{1.003750pt}%
\definecolor{currentstroke}{rgb}{1.000000,0.498039,0.054902}%
\pgfsetstrokecolor{currentstroke}%
\pgfsetdash{}{0pt}%
\pgfpathmoveto{\pgfqpoint{1.483893in}{3.202258in}}%
\pgfpathcurveto{\pgfqpoint{1.494943in}{3.202258in}}{\pgfqpoint{1.505542in}{3.206648in}}{\pgfqpoint{1.513356in}{3.214462in}}%
\pgfpathcurveto{\pgfqpoint{1.521170in}{3.222275in}}{\pgfqpoint{1.525560in}{3.232874in}}{\pgfqpoint{1.525560in}{3.243924in}}%
\pgfpathcurveto{\pgfqpoint{1.525560in}{3.254974in}}{\pgfqpoint{1.521170in}{3.265573in}}{\pgfqpoint{1.513356in}{3.273387in}}%
\pgfpathcurveto{\pgfqpoint{1.505542in}{3.281201in}}{\pgfqpoint{1.494943in}{3.285591in}}{\pgfqpoint{1.483893in}{3.285591in}}%
\pgfpathcurveto{\pgfqpoint{1.472843in}{3.285591in}}{\pgfqpoint{1.462244in}{3.281201in}}{\pgfqpoint{1.454430in}{3.273387in}}%
\pgfpathcurveto{\pgfqpoint{1.446617in}{3.265573in}}{\pgfqpoint{1.442226in}{3.254974in}}{\pgfqpoint{1.442226in}{3.243924in}}%
\pgfpathcurveto{\pgfqpoint{1.442226in}{3.232874in}}{\pgfqpoint{1.446617in}{3.222275in}}{\pgfqpoint{1.454430in}{3.214462in}}%
\pgfpathcurveto{\pgfqpoint{1.462244in}{3.206648in}}{\pgfqpoint{1.472843in}{3.202258in}}{\pgfqpoint{1.483893in}{3.202258in}}%
\pgfpathclose%
\pgfusepath{stroke,fill}%
\end{pgfscope}%
\begin{pgfscope}%
\pgfpathrectangle{\pgfqpoint{0.648703in}{0.548769in}}{\pgfqpoint{5.201297in}{3.102590in}}%
\pgfusepath{clip}%
\pgfsetbuttcap%
\pgfsetroundjoin%
\definecolor{currentfill}{rgb}{1.000000,0.498039,0.054902}%
\pgfsetfillcolor{currentfill}%
\pgfsetlinewidth{1.003750pt}%
\definecolor{currentstroke}{rgb}{1.000000,0.498039,0.054902}%
\pgfsetstrokecolor{currentstroke}%
\pgfsetdash{}{0pt}%
\pgfpathmoveto{\pgfqpoint{1.408715in}{3.206486in}}%
\pgfpathcurveto{\pgfqpoint{1.419765in}{3.206486in}}{\pgfqpoint{1.430364in}{3.210877in}}{\pgfqpoint{1.438178in}{3.218690in}}%
\pgfpathcurveto{\pgfqpoint{1.445991in}{3.226504in}}{\pgfqpoint{1.450381in}{3.237103in}}{\pgfqpoint{1.450381in}{3.248153in}}%
\pgfpathcurveto{\pgfqpoint{1.450381in}{3.259203in}}{\pgfqpoint{1.445991in}{3.269802in}}{\pgfqpoint{1.438178in}{3.277616in}}%
\pgfpathcurveto{\pgfqpoint{1.430364in}{3.285429in}}{\pgfqpoint{1.419765in}{3.289820in}}{\pgfqpoint{1.408715in}{3.289820in}}%
\pgfpathcurveto{\pgfqpoint{1.397665in}{3.289820in}}{\pgfqpoint{1.387066in}{3.285429in}}{\pgfqpoint{1.379252in}{3.277616in}}%
\pgfpathcurveto{\pgfqpoint{1.371438in}{3.269802in}}{\pgfqpoint{1.367048in}{3.259203in}}{\pgfqpoint{1.367048in}{3.248153in}}%
\pgfpathcurveto{\pgfqpoint{1.367048in}{3.237103in}}{\pgfqpoint{1.371438in}{3.226504in}}{\pgfqpoint{1.379252in}{3.218690in}}%
\pgfpathcurveto{\pgfqpoint{1.387066in}{3.210877in}}{\pgfqpoint{1.397665in}{3.206486in}}{\pgfqpoint{1.408715in}{3.206486in}}%
\pgfpathclose%
\pgfusepath{stroke,fill}%
\end{pgfscope}%
\begin{pgfscope}%
\pgfpathrectangle{\pgfqpoint{0.648703in}{0.548769in}}{\pgfqpoint{5.201297in}{3.102590in}}%
\pgfusepath{clip}%
\pgfsetbuttcap%
\pgfsetroundjoin%
\definecolor{currentfill}{rgb}{1.000000,0.498039,0.054902}%
\pgfsetfillcolor{currentfill}%
\pgfsetlinewidth{1.003750pt}%
\definecolor{currentstroke}{rgb}{1.000000,0.498039,0.054902}%
\pgfsetstrokecolor{currentstroke}%
\pgfsetdash{}{0pt}%
\pgfpathmoveto{\pgfqpoint{1.219506in}{3.198029in}}%
\pgfpathcurveto{\pgfqpoint{1.230556in}{3.198029in}}{\pgfqpoint{1.241155in}{3.202419in}}{\pgfqpoint{1.248969in}{3.210233in}}%
\pgfpathcurveto{\pgfqpoint{1.256783in}{3.218046in}}{\pgfqpoint{1.261173in}{3.228646in}}{\pgfqpoint{1.261173in}{3.239696in}}%
\pgfpathcurveto{\pgfqpoint{1.261173in}{3.250746in}}{\pgfqpoint{1.256783in}{3.261345in}}{\pgfqpoint{1.248969in}{3.269158in}}%
\pgfpathcurveto{\pgfqpoint{1.241155in}{3.276972in}}{\pgfqpoint{1.230556in}{3.281362in}}{\pgfqpoint{1.219506in}{3.281362in}}%
\pgfpathcurveto{\pgfqpoint{1.208456in}{3.281362in}}{\pgfqpoint{1.197857in}{3.276972in}}{\pgfqpoint{1.190043in}{3.269158in}}%
\pgfpathcurveto{\pgfqpoint{1.182230in}{3.261345in}}{\pgfqpoint{1.177840in}{3.250746in}}{\pgfqpoint{1.177840in}{3.239696in}}%
\pgfpathcurveto{\pgfqpoint{1.177840in}{3.228646in}}{\pgfqpoint{1.182230in}{3.218046in}}{\pgfqpoint{1.190043in}{3.210233in}}%
\pgfpathcurveto{\pgfqpoint{1.197857in}{3.202419in}}{\pgfqpoint{1.208456in}{3.198029in}}{\pgfqpoint{1.219506in}{3.198029in}}%
\pgfpathclose%
\pgfusepath{stroke,fill}%
\end{pgfscope}%
\begin{pgfscope}%
\pgfpathrectangle{\pgfqpoint{0.648703in}{0.548769in}}{\pgfqpoint{5.201297in}{3.102590in}}%
\pgfusepath{clip}%
\pgfsetbuttcap%
\pgfsetroundjoin%
\definecolor{currentfill}{rgb}{1.000000,0.498039,0.054902}%
\pgfsetfillcolor{currentfill}%
\pgfsetlinewidth{1.003750pt}%
\definecolor{currentstroke}{rgb}{1.000000,0.498039,0.054902}%
\pgfsetstrokecolor{currentstroke}%
\pgfsetdash{}{0pt}%
\pgfpathmoveto{\pgfqpoint{1.557330in}{3.202258in}}%
\pgfpathcurveto{\pgfqpoint{1.568380in}{3.202258in}}{\pgfqpoint{1.578979in}{3.206648in}}{\pgfqpoint{1.586793in}{3.214462in}}%
\pgfpathcurveto{\pgfqpoint{1.594606in}{3.222275in}}{\pgfqpoint{1.598997in}{3.232874in}}{\pgfqpoint{1.598997in}{3.243924in}}%
\pgfpathcurveto{\pgfqpoint{1.598997in}{3.254974in}}{\pgfqpoint{1.594606in}{3.265573in}}{\pgfqpoint{1.586793in}{3.273387in}}%
\pgfpathcurveto{\pgfqpoint{1.578979in}{3.281201in}}{\pgfqpoint{1.568380in}{3.285591in}}{\pgfqpoint{1.557330in}{3.285591in}}%
\pgfpathcurveto{\pgfqpoint{1.546280in}{3.285591in}}{\pgfqpoint{1.535681in}{3.281201in}}{\pgfqpoint{1.527867in}{3.273387in}}%
\pgfpathcurveto{\pgfqpoint{1.520054in}{3.265573in}}{\pgfqpoint{1.515663in}{3.254974in}}{\pgfqpoint{1.515663in}{3.243924in}}%
\pgfpathcurveto{\pgfqpoint{1.515663in}{3.232874in}}{\pgfqpoint{1.520054in}{3.222275in}}{\pgfqpoint{1.527867in}{3.214462in}}%
\pgfpathcurveto{\pgfqpoint{1.535681in}{3.206648in}}{\pgfqpoint{1.546280in}{3.202258in}}{\pgfqpoint{1.557330in}{3.202258in}}%
\pgfpathclose%
\pgfusepath{stroke,fill}%
\end{pgfscope}%
\begin{pgfscope}%
\pgfpathrectangle{\pgfqpoint{0.648703in}{0.548769in}}{\pgfqpoint{5.201297in}{3.102590in}}%
\pgfusepath{clip}%
\pgfsetbuttcap%
\pgfsetroundjoin%
\definecolor{currentfill}{rgb}{1.000000,0.498039,0.054902}%
\pgfsetfillcolor{currentfill}%
\pgfsetlinewidth{1.003750pt}%
\definecolor{currentstroke}{rgb}{1.000000,0.498039,0.054902}%
\pgfsetstrokecolor{currentstroke}%
\pgfsetdash{}{0pt}%
\pgfpathmoveto{\pgfqpoint{2.222501in}{3.189572in}}%
\pgfpathcurveto{\pgfqpoint{2.233551in}{3.189572in}}{\pgfqpoint{2.244150in}{3.193962in}}{\pgfqpoint{2.251963in}{3.201775in}}%
\pgfpathcurveto{\pgfqpoint{2.259777in}{3.209589in}}{\pgfqpoint{2.264167in}{3.220188in}}{\pgfqpoint{2.264167in}{3.231238in}}%
\pgfpathcurveto{\pgfqpoint{2.264167in}{3.242288in}}{\pgfqpoint{2.259777in}{3.252887in}}{\pgfqpoint{2.251963in}{3.260701in}}%
\pgfpathcurveto{\pgfqpoint{2.244150in}{3.268515in}}{\pgfqpoint{2.233551in}{3.272905in}}{\pgfqpoint{2.222501in}{3.272905in}}%
\pgfpathcurveto{\pgfqpoint{2.211450in}{3.272905in}}{\pgfqpoint{2.200851in}{3.268515in}}{\pgfqpoint{2.193038in}{3.260701in}}%
\pgfpathcurveto{\pgfqpoint{2.185224in}{3.252887in}}{\pgfqpoint{2.180834in}{3.242288in}}{\pgfqpoint{2.180834in}{3.231238in}}%
\pgfpathcurveto{\pgfqpoint{2.180834in}{3.220188in}}{\pgfqpoint{2.185224in}{3.209589in}}{\pgfqpoint{2.193038in}{3.201775in}}%
\pgfpathcurveto{\pgfqpoint{2.200851in}{3.193962in}}{\pgfqpoint{2.211450in}{3.189572in}}{\pgfqpoint{2.222501in}{3.189572in}}%
\pgfpathclose%
\pgfusepath{stroke,fill}%
\end{pgfscope}%
\begin{pgfscope}%
\pgfpathrectangle{\pgfqpoint{0.648703in}{0.548769in}}{\pgfqpoint{5.201297in}{3.102590in}}%
\pgfusepath{clip}%
\pgfsetbuttcap%
\pgfsetroundjoin%
\definecolor{currentfill}{rgb}{0.121569,0.466667,0.705882}%
\pgfsetfillcolor{currentfill}%
\pgfsetlinewidth{1.003750pt}%
\definecolor{currentstroke}{rgb}{0.121569,0.466667,0.705882}%
\pgfsetstrokecolor{currentstroke}%
\pgfsetdash{}{0pt}%
\pgfpathmoveto{\pgfqpoint{2.066527in}{3.181114in}}%
\pgfpathcurveto{\pgfqpoint{2.077578in}{3.181114in}}{\pgfqpoint{2.088177in}{3.185504in}}{\pgfqpoint{2.095990in}{3.193318in}}%
\pgfpathcurveto{\pgfqpoint{2.103804in}{3.201132in}}{\pgfqpoint{2.108194in}{3.211731in}}{\pgfqpoint{2.108194in}{3.222781in}}%
\pgfpathcurveto{\pgfqpoint{2.108194in}{3.233831in}}{\pgfqpoint{2.103804in}{3.244430in}}{\pgfqpoint{2.095990in}{3.252244in}}%
\pgfpathcurveto{\pgfqpoint{2.088177in}{3.260057in}}{\pgfqpoint{2.077578in}{3.264448in}}{\pgfqpoint{2.066527in}{3.264448in}}%
\pgfpathcurveto{\pgfqpoint{2.055477in}{3.264448in}}{\pgfqpoint{2.044878in}{3.260057in}}{\pgfqpoint{2.037065in}{3.252244in}}%
\pgfpathcurveto{\pgfqpoint{2.029251in}{3.244430in}}{\pgfqpoint{2.024861in}{3.233831in}}{\pgfqpoint{2.024861in}{3.222781in}}%
\pgfpathcurveto{\pgfqpoint{2.024861in}{3.211731in}}{\pgfqpoint{2.029251in}{3.201132in}}{\pgfqpoint{2.037065in}{3.193318in}}%
\pgfpathcurveto{\pgfqpoint{2.044878in}{3.185504in}}{\pgfqpoint{2.055477in}{3.181114in}}{\pgfqpoint{2.066527in}{3.181114in}}%
\pgfpathclose%
\pgfusepath{stroke,fill}%
\end{pgfscope}%
\begin{pgfscope}%
\pgfpathrectangle{\pgfqpoint{0.648703in}{0.548769in}}{\pgfqpoint{5.201297in}{3.102590in}}%
\pgfusepath{clip}%
\pgfsetbuttcap%
\pgfsetroundjoin%
\definecolor{currentfill}{rgb}{1.000000,0.498039,0.054902}%
\pgfsetfillcolor{currentfill}%
\pgfsetlinewidth{1.003750pt}%
\definecolor{currentstroke}{rgb}{1.000000,0.498039,0.054902}%
\pgfsetstrokecolor{currentstroke}%
\pgfsetdash{}{0pt}%
\pgfpathmoveto{\pgfqpoint{1.786871in}{3.244545in}}%
\pgfpathcurveto{\pgfqpoint{1.797921in}{3.244545in}}{\pgfqpoint{1.808520in}{3.248935in}}{\pgfqpoint{1.816333in}{3.256748in}}%
\pgfpathcurveto{\pgfqpoint{1.824147in}{3.264562in}}{\pgfqpoint{1.828537in}{3.275161in}}{\pgfqpoint{1.828537in}{3.286211in}}%
\pgfpathcurveto{\pgfqpoint{1.828537in}{3.297261in}}{\pgfqpoint{1.824147in}{3.307860in}}{\pgfqpoint{1.816333in}{3.315674in}}%
\pgfpathcurveto{\pgfqpoint{1.808520in}{3.323488in}}{\pgfqpoint{1.797921in}{3.327878in}}{\pgfqpoint{1.786871in}{3.327878in}}%
\pgfpathcurveto{\pgfqpoint{1.775820in}{3.327878in}}{\pgfqpoint{1.765221in}{3.323488in}}{\pgfqpoint{1.757408in}{3.315674in}}%
\pgfpathcurveto{\pgfqpoint{1.749594in}{3.307860in}}{\pgfqpoint{1.745204in}{3.297261in}}{\pgfqpoint{1.745204in}{3.286211in}}%
\pgfpathcurveto{\pgfqpoint{1.745204in}{3.275161in}}{\pgfqpoint{1.749594in}{3.264562in}}{\pgfqpoint{1.757408in}{3.256748in}}%
\pgfpathcurveto{\pgfqpoint{1.765221in}{3.248935in}}{\pgfqpoint{1.775820in}{3.244545in}}{\pgfqpoint{1.786871in}{3.244545in}}%
\pgfpathclose%
\pgfusepath{stroke,fill}%
\end{pgfscope}%
\begin{pgfscope}%
\pgfpathrectangle{\pgfqpoint{0.648703in}{0.548769in}}{\pgfqpoint{5.201297in}{3.102590in}}%
\pgfusepath{clip}%
\pgfsetbuttcap%
\pgfsetroundjoin%
\definecolor{currentfill}{rgb}{1.000000,0.498039,0.054902}%
\pgfsetfillcolor{currentfill}%
\pgfsetlinewidth{1.003750pt}%
\definecolor{currentstroke}{rgb}{1.000000,0.498039,0.054902}%
\pgfsetstrokecolor{currentstroke}%
\pgfsetdash{}{0pt}%
\pgfpathmoveto{\pgfqpoint{1.423134in}{3.189572in}}%
\pgfpathcurveto{\pgfqpoint{1.434184in}{3.189572in}}{\pgfqpoint{1.444783in}{3.193962in}}{\pgfqpoint{1.452597in}{3.201775in}}%
\pgfpathcurveto{\pgfqpoint{1.460410in}{3.209589in}}{\pgfqpoint{1.464800in}{3.220188in}}{\pgfqpoint{1.464800in}{3.231238in}}%
\pgfpathcurveto{\pgfqpoint{1.464800in}{3.242288in}}{\pgfqpoint{1.460410in}{3.252887in}}{\pgfqpoint{1.452597in}{3.260701in}}%
\pgfpathcurveto{\pgfqpoint{1.444783in}{3.268515in}}{\pgfqpoint{1.434184in}{3.272905in}}{\pgfqpoint{1.423134in}{3.272905in}}%
\pgfpathcurveto{\pgfqpoint{1.412084in}{3.272905in}}{\pgfqpoint{1.401485in}{3.268515in}}{\pgfqpoint{1.393671in}{3.260701in}}%
\pgfpathcurveto{\pgfqpoint{1.385857in}{3.252887in}}{\pgfqpoint{1.381467in}{3.242288in}}{\pgfqpoint{1.381467in}{3.231238in}}%
\pgfpathcurveto{\pgfqpoint{1.381467in}{3.220188in}}{\pgfqpoint{1.385857in}{3.209589in}}{\pgfqpoint{1.393671in}{3.201775in}}%
\pgfpathcurveto{\pgfqpoint{1.401485in}{3.193962in}}{\pgfqpoint{1.412084in}{3.189572in}}{\pgfqpoint{1.423134in}{3.189572in}}%
\pgfpathclose%
\pgfusepath{stroke,fill}%
\end{pgfscope}%
\begin{pgfscope}%
\pgfpathrectangle{\pgfqpoint{0.648703in}{0.548769in}}{\pgfqpoint{5.201297in}{3.102590in}}%
\pgfusepath{clip}%
\pgfsetbuttcap%
\pgfsetroundjoin%
\definecolor{currentfill}{rgb}{1.000000,0.498039,0.054902}%
\pgfsetfillcolor{currentfill}%
\pgfsetlinewidth{1.003750pt}%
\definecolor{currentstroke}{rgb}{1.000000,0.498039,0.054902}%
\pgfsetstrokecolor{currentstroke}%
\pgfsetdash{}{0pt}%
\pgfpathmoveto{\pgfqpoint{1.729099in}{3.210715in}}%
\pgfpathcurveto{\pgfqpoint{1.740150in}{3.210715in}}{\pgfqpoint{1.750749in}{3.215105in}}{\pgfqpoint{1.758562in}{3.222919in}}%
\pgfpathcurveto{\pgfqpoint{1.766376in}{3.230733in}}{\pgfqpoint{1.770766in}{3.241332in}}{\pgfqpoint{1.770766in}{3.252382in}}%
\pgfpathcurveto{\pgfqpoint{1.770766in}{3.263432in}}{\pgfqpoint{1.766376in}{3.274031in}}{\pgfqpoint{1.758562in}{3.281844in}}%
\pgfpathcurveto{\pgfqpoint{1.750749in}{3.289658in}}{\pgfqpoint{1.740150in}{3.294048in}}{\pgfqpoint{1.729099in}{3.294048in}}%
\pgfpathcurveto{\pgfqpoint{1.718049in}{3.294048in}}{\pgfqpoint{1.707450in}{3.289658in}}{\pgfqpoint{1.699637in}{3.281844in}}%
\pgfpathcurveto{\pgfqpoint{1.691823in}{3.274031in}}{\pgfqpoint{1.687433in}{3.263432in}}{\pgfqpoint{1.687433in}{3.252382in}}%
\pgfpathcurveto{\pgfqpoint{1.687433in}{3.241332in}}{\pgfqpoint{1.691823in}{3.230733in}}{\pgfqpoint{1.699637in}{3.222919in}}%
\pgfpathcurveto{\pgfqpoint{1.707450in}{3.215105in}}{\pgfqpoint{1.718049in}{3.210715in}}{\pgfqpoint{1.729099in}{3.210715in}}%
\pgfpathclose%
\pgfusepath{stroke,fill}%
\end{pgfscope}%
\begin{pgfscope}%
\pgfpathrectangle{\pgfqpoint{0.648703in}{0.548769in}}{\pgfqpoint{5.201297in}{3.102590in}}%
\pgfusepath{clip}%
\pgfsetbuttcap%
\pgfsetroundjoin%
\definecolor{currentfill}{rgb}{1.000000,0.498039,0.054902}%
\pgfsetfillcolor{currentfill}%
\pgfsetlinewidth{1.003750pt}%
\definecolor{currentstroke}{rgb}{1.000000,0.498039,0.054902}%
\pgfsetstrokecolor{currentstroke}%
\pgfsetdash{}{0pt}%
\pgfpathmoveto{\pgfqpoint{1.788968in}{3.189572in}}%
\pgfpathcurveto{\pgfqpoint{1.800018in}{3.189572in}}{\pgfqpoint{1.810617in}{3.193962in}}{\pgfqpoint{1.818431in}{3.201775in}}%
\pgfpathcurveto{\pgfqpoint{1.826245in}{3.209589in}}{\pgfqpoint{1.830635in}{3.220188in}}{\pgfqpoint{1.830635in}{3.231238in}}%
\pgfpathcurveto{\pgfqpoint{1.830635in}{3.242288in}}{\pgfqpoint{1.826245in}{3.252887in}}{\pgfqpoint{1.818431in}{3.260701in}}%
\pgfpathcurveto{\pgfqpoint{1.810617in}{3.268515in}}{\pgfqpoint{1.800018in}{3.272905in}}{\pgfqpoint{1.788968in}{3.272905in}}%
\pgfpathcurveto{\pgfqpoint{1.777918in}{3.272905in}}{\pgfqpoint{1.767319in}{3.268515in}}{\pgfqpoint{1.759505in}{3.260701in}}%
\pgfpathcurveto{\pgfqpoint{1.751692in}{3.252887in}}{\pgfqpoint{1.747302in}{3.242288in}}{\pgfqpoint{1.747302in}{3.231238in}}%
\pgfpathcurveto{\pgfqpoint{1.747302in}{3.220188in}}{\pgfqpoint{1.751692in}{3.209589in}}{\pgfqpoint{1.759505in}{3.201775in}}%
\pgfpathcurveto{\pgfqpoint{1.767319in}{3.193962in}}{\pgfqpoint{1.777918in}{3.189572in}}{\pgfqpoint{1.788968in}{3.189572in}}%
\pgfpathclose%
\pgfusepath{stroke,fill}%
\end{pgfscope}%
\begin{pgfscope}%
\pgfpathrectangle{\pgfqpoint{0.648703in}{0.548769in}}{\pgfqpoint{5.201297in}{3.102590in}}%
\pgfusepath{clip}%
\pgfsetbuttcap%
\pgfsetroundjoin%
\definecolor{currentfill}{rgb}{0.839216,0.152941,0.156863}%
\pgfsetfillcolor{currentfill}%
\pgfsetlinewidth{1.003750pt}%
\definecolor{currentstroke}{rgb}{0.839216,0.152941,0.156863}%
\pgfsetstrokecolor{currentstroke}%
\pgfsetdash{}{0pt}%
\pgfpathmoveto{\pgfqpoint{1.991123in}{3.181114in}}%
\pgfpathcurveto{\pgfqpoint{2.002174in}{3.181114in}}{\pgfqpoint{2.012773in}{3.185504in}}{\pgfqpoint{2.020586in}{3.193318in}}%
\pgfpathcurveto{\pgfqpoint{2.028400in}{3.201132in}}{\pgfqpoint{2.032790in}{3.211731in}}{\pgfqpoint{2.032790in}{3.222781in}}%
\pgfpathcurveto{\pgfqpoint{2.032790in}{3.233831in}}{\pgfqpoint{2.028400in}{3.244430in}}{\pgfqpoint{2.020586in}{3.252244in}}%
\pgfpathcurveto{\pgfqpoint{2.012773in}{3.260057in}}{\pgfqpoint{2.002174in}{3.264448in}}{\pgfqpoint{1.991123in}{3.264448in}}%
\pgfpathcurveto{\pgfqpoint{1.980073in}{3.264448in}}{\pgfqpoint{1.969474in}{3.260057in}}{\pgfqpoint{1.961661in}{3.252244in}}%
\pgfpathcurveto{\pgfqpoint{1.953847in}{3.244430in}}{\pgfqpoint{1.949457in}{3.233831in}}{\pgfqpoint{1.949457in}{3.222781in}}%
\pgfpathcurveto{\pgfqpoint{1.949457in}{3.211731in}}{\pgfqpoint{1.953847in}{3.201132in}}{\pgfqpoint{1.961661in}{3.193318in}}%
\pgfpathcurveto{\pgfqpoint{1.969474in}{3.185504in}}{\pgfqpoint{1.980073in}{3.181114in}}{\pgfqpoint{1.991123in}{3.181114in}}%
\pgfpathclose%
\pgfusepath{stroke,fill}%
\end{pgfscope}%
\begin{pgfscope}%
\pgfpathrectangle{\pgfqpoint{0.648703in}{0.548769in}}{\pgfqpoint{5.201297in}{3.102590in}}%
\pgfusepath{clip}%
\pgfsetbuttcap%
\pgfsetroundjoin%
\definecolor{currentfill}{rgb}{1.000000,0.498039,0.054902}%
\pgfsetfillcolor{currentfill}%
\pgfsetlinewidth{1.003750pt}%
\definecolor{currentstroke}{rgb}{1.000000,0.498039,0.054902}%
\pgfsetstrokecolor{currentstroke}%
\pgfsetdash{}{0pt}%
\pgfpathmoveto{\pgfqpoint{1.493163in}{3.202258in}}%
\pgfpathcurveto{\pgfqpoint{1.504213in}{3.202258in}}{\pgfqpoint{1.514812in}{3.206648in}}{\pgfqpoint{1.522626in}{3.214462in}}%
\pgfpathcurveto{\pgfqpoint{1.530439in}{3.222275in}}{\pgfqpoint{1.534829in}{3.232874in}}{\pgfqpoint{1.534829in}{3.243924in}}%
\pgfpathcurveto{\pgfqpoint{1.534829in}{3.254974in}}{\pgfqpoint{1.530439in}{3.265573in}}{\pgfqpoint{1.522626in}{3.273387in}}%
\pgfpathcurveto{\pgfqpoint{1.514812in}{3.281201in}}{\pgfqpoint{1.504213in}{3.285591in}}{\pgfqpoint{1.493163in}{3.285591in}}%
\pgfpathcurveto{\pgfqpoint{1.482113in}{3.285591in}}{\pgfqpoint{1.471514in}{3.281201in}}{\pgfqpoint{1.463700in}{3.273387in}}%
\pgfpathcurveto{\pgfqpoint{1.455886in}{3.265573in}}{\pgfqpoint{1.451496in}{3.254974in}}{\pgfqpoint{1.451496in}{3.243924in}}%
\pgfpathcurveto{\pgfqpoint{1.451496in}{3.232874in}}{\pgfqpoint{1.455886in}{3.222275in}}{\pgfqpoint{1.463700in}{3.214462in}}%
\pgfpathcurveto{\pgfqpoint{1.471514in}{3.206648in}}{\pgfqpoint{1.482113in}{3.202258in}}{\pgfqpoint{1.493163in}{3.202258in}}%
\pgfpathclose%
\pgfusepath{stroke,fill}%
\end{pgfscope}%
\begin{pgfscope}%
\pgfpathrectangle{\pgfqpoint{0.648703in}{0.548769in}}{\pgfqpoint{5.201297in}{3.102590in}}%
\pgfusepath{clip}%
\pgfsetbuttcap%
\pgfsetroundjoin%
\definecolor{currentfill}{rgb}{0.839216,0.152941,0.156863}%
\pgfsetfillcolor{currentfill}%
\pgfsetlinewidth{1.003750pt}%
\definecolor{currentstroke}{rgb}{0.839216,0.152941,0.156863}%
\pgfsetstrokecolor{currentstroke}%
\pgfsetdash{}{0pt}%
\pgfpathmoveto{\pgfqpoint{1.820830in}{3.265688in}}%
\pgfpathcurveto{\pgfqpoint{1.831880in}{3.265688in}}{\pgfqpoint{1.842479in}{3.270078in}}{\pgfqpoint{1.850293in}{3.277892in}}%
\pgfpathcurveto{\pgfqpoint{1.858107in}{3.285706in}}{\pgfqpoint{1.862497in}{3.296305in}}{\pgfqpoint{1.862497in}{3.307355in}}%
\pgfpathcurveto{\pgfqpoint{1.862497in}{3.318405in}}{\pgfqpoint{1.858107in}{3.329004in}}{\pgfqpoint{1.850293in}{3.336817in}}%
\pgfpathcurveto{\pgfqpoint{1.842479in}{3.344631in}}{\pgfqpoint{1.831880in}{3.349021in}}{\pgfqpoint{1.820830in}{3.349021in}}%
\pgfpathcurveto{\pgfqpoint{1.809780in}{3.349021in}}{\pgfqpoint{1.799181in}{3.344631in}}{\pgfqpoint{1.791367in}{3.336817in}}%
\pgfpathcurveto{\pgfqpoint{1.783554in}{3.329004in}}{\pgfqpoint{1.779164in}{3.318405in}}{\pgfqpoint{1.779164in}{3.307355in}}%
\pgfpathcurveto{\pgfqpoint{1.779164in}{3.296305in}}{\pgfqpoint{1.783554in}{3.285706in}}{\pgfqpoint{1.791367in}{3.277892in}}%
\pgfpathcurveto{\pgfqpoint{1.799181in}{3.270078in}}{\pgfqpoint{1.809780in}{3.265688in}}{\pgfqpoint{1.820830in}{3.265688in}}%
\pgfpathclose%
\pgfusepath{stroke,fill}%
\end{pgfscope}%
\begin{pgfscope}%
\pgfpathrectangle{\pgfqpoint{0.648703in}{0.548769in}}{\pgfqpoint{5.201297in}{3.102590in}}%
\pgfusepath{clip}%
\pgfsetbuttcap%
\pgfsetroundjoin%
\definecolor{currentfill}{rgb}{0.121569,0.466667,0.705882}%
\pgfsetfillcolor{currentfill}%
\pgfsetlinewidth{1.003750pt}%
\definecolor{currentstroke}{rgb}{0.121569,0.466667,0.705882}%
\pgfsetstrokecolor{currentstroke}%
\pgfsetdash{}{0pt}%
\pgfpathmoveto{\pgfqpoint{2.102901in}{2.808990in}}%
\pgfpathcurveto{\pgfqpoint{2.113952in}{2.808990in}}{\pgfqpoint{2.124551in}{2.813380in}}{\pgfqpoint{2.132364in}{2.821193in}}%
\pgfpathcurveto{\pgfqpoint{2.140178in}{2.829007in}}{\pgfqpoint{2.144568in}{2.839606in}}{\pgfqpoint{2.144568in}{2.850656in}}%
\pgfpathcurveto{\pgfqpoint{2.144568in}{2.861706in}}{\pgfqpoint{2.140178in}{2.872305in}}{\pgfqpoint{2.132364in}{2.880119in}}%
\pgfpathcurveto{\pgfqpoint{2.124551in}{2.887933in}}{\pgfqpoint{2.113952in}{2.892323in}}{\pgfqpoint{2.102901in}{2.892323in}}%
\pgfpathcurveto{\pgfqpoint{2.091851in}{2.892323in}}{\pgfqpoint{2.081252in}{2.887933in}}{\pgfqpoint{2.073439in}{2.880119in}}%
\pgfpathcurveto{\pgfqpoint{2.065625in}{2.872305in}}{\pgfqpoint{2.061235in}{2.861706in}}{\pgfqpoint{2.061235in}{2.850656in}}%
\pgfpathcurveto{\pgfqpoint{2.061235in}{2.839606in}}{\pgfqpoint{2.065625in}{2.829007in}}{\pgfqpoint{2.073439in}{2.821193in}}%
\pgfpathcurveto{\pgfqpoint{2.081252in}{2.813380in}}{\pgfqpoint{2.091851in}{2.808990in}}{\pgfqpoint{2.102901in}{2.808990in}}%
\pgfpathclose%
\pgfusepath{stroke,fill}%
\end{pgfscope}%
\begin{pgfscope}%
\pgfpathrectangle{\pgfqpoint{0.648703in}{0.548769in}}{\pgfqpoint{5.201297in}{3.102590in}}%
\pgfusepath{clip}%
\pgfsetbuttcap%
\pgfsetroundjoin%
\definecolor{currentfill}{rgb}{1.000000,0.498039,0.054902}%
\pgfsetfillcolor{currentfill}%
\pgfsetlinewidth{1.003750pt}%
\definecolor{currentstroke}{rgb}{1.000000,0.498039,0.054902}%
\pgfsetstrokecolor{currentstroke}%
\pgfsetdash{}{0pt}%
\pgfpathmoveto{\pgfqpoint{1.910115in}{3.185343in}}%
\pgfpathcurveto{\pgfqpoint{1.921165in}{3.185343in}}{\pgfqpoint{1.931764in}{3.189733in}}{\pgfqpoint{1.939578in}{3.197547in}}%
\pgfpathcurveto{\pgfqpoint{1.947391in}{3.205360in}}{\pgfqpoint{1.951782in}{3.215959in}}{\pgfqpoint{1.951782in}{3.227010in}}%
\pgfpathcurveto{\pgfqpoint{1.951782in}{3.238060in}}{\pgfqpoint{1.947391in}{3.248659in}}{\pgfqpoint{1.939578in}{3.256472in}}%
\pgfpathcurveto{\pgfqpoint{1.931764in}{3.264286in}}{\pgfqpoint{1.921165in}{3.268676in}}{\pgfqpoint{1.910115in}{3.268676in}}%
\pgfpathcurveto{\pgfqpoint{1.899065in}{3.268676in}}{\pgfqpoint{1.888466in}{3.264286in}}{\pgfqpoint{1.880652in}{3.256472in}}%
\pgfpathcurveto{\pgfqpoint{1.872839in}{3.248659in}}{\pgfqpoint{1.868448in}{3.238060in}}{\pgfqpoint{1.868448in}{3.227010in}}%
\pgfpathcurveto{\pgfqpoint{1.868448in}{3.215959in}}{\pgfqpoint{1.872839in}{3.205360in}}{\pgfqpoint{1.880652in}{3.197547in}}%
\pgfpathcurveto{\pgfqpoint{1.888466in}{3.189733in}}{\pgfqpoint{1.899065in}{3.185343in}}{\pgfqpoint{1.910115in}{3.185343in}}%
\pgfpathclose%
\pgfusepath{stroke,fill}%
\end{pgfscope}%
\begin{pgfscope}%
\pgfpathrectangle{\pgfqpoint{0.648703in}{0.548769in}}{\pgfqpoint{5.201297in}{3.102590in}}%
\pgfusepath{clip}%
\pgfsetbuttcap%
\pgfsetroundjoin%
\definecolor{currentfill}{rgb}{1.000000,0.498039,0.054902}%
\pgfsetfillcolor{currentfill}%
\pgfsetlinewidth{1.003750pt}%
\definecolor{currentstroke}{rgb}{1.000000,0.498039,0.054902}%
\pgfsetstrokecolor{currentstroke}%
\pgfsetdash{}{0pt}%
\pgfpathmoveto{\pgfqpoint{1.676905in}{3.193800in}}%
\pgfpathcurveto{\pgfqpoint{1.687955in}{3.193800in}}{\pgfqpoint{1.698554in}{3.198191in}}{\pgfqpoint{1.706368in}{3.206004in}}%
\pgfpathcurveto{\pgfqpoint{1.714182in}{3.213818in}}{\pgfqpoint{1.718572in}{3.224417in}}{\pgfqpoint{1.718572in}{3.235467in}}%
\pgfpathcurveto{\pgfqpoint{1.718572in}{3.246517in}}{\pgfqpoint{1.714182in}{3.257116in}}{\pgfqpoint{1.706368in}{3.264930in}}%
\pgfpathcurveto{\pgfqpoint{1.698554in}{3.272743in}}{\pgfqpoint{1.687955in}{3.277134in}}{\pgfqpoint{1.676905in}{3.277134in}}%
\pgfpathcurveto{\pgfqpoint{1.665855in}{3.277134in}}{\pgfqpoint{1.655256in}{3.272743in}}{\pgfqpoint{1.647442in}{3.264930in}}%
\pgfpathcurveto{\pgfqpoint{1.639629in}{3.257116in}}{\pgfqpoint{1.635239in}{3.246517in}}{\pgfqpoint{1.635239in}{3.235467in}}%
\pgfpathcurveto{\pgfqpoint{1.635239in}{3.224417in}}{\pgfqpoint{1.639629in}{3.213818in}}{\pgfqpoint{1.647442in}{3.206004in}}%
\pgfpathcurveto{\pgfqpoint{1.655256in}{3.198191in}}{\pgfqpoint{1.665855in}{3.193800in}}{\pgfqpoint{1.676905in}{3.193800in}}%
\pgfpathclose%
\pgfusepath{stroke,fill}%
\end{pgfscope}%
\begin{pgfscope}%
\pgfpathrectangle{\pgfqpoint{0.648703in}{0.548769in}}{\pgfqpoint{5.201297in}{3.102590in}}%
\pgfusepath{clip}%
\pgfsetbuttcap%
\pgfsetroundjoin%
\definecolor{currentfill}{rgb}{1.000000,0.498039,0.054902}%
\pgfsetfillcolor{currentfill}%
\pgfsetlinewidth{1.003750pt}%
\definecolor{currentstroke}{rgb}{1.000000,0.498039,0.054902}%
\pgfsetstrokecolor{currentstroke}%
\pgfsetdash{}{0pt}%
\pgfpathmoveto{\pgfqpoint{1.098633in}{3.362948in}}%
\pgfpathcurveto{\pgfqpoint{1.109683in}{3.362948in}}{\pgfqpoint{1.120282in}{3.367338in}}{\pgfqpoint{1.128095in}{3.375152in}}%
\pgfpathcurveto{\pgfqpoint{1.135909in}{3.382965in}}{\pgfqpoint{1.140299in}{3.393564in}}{\pgfqpoint{1.140299in}{3.404615in}}%
\pgfpathcurveto{\pgfqpoint{1.140299in}{3.415665in}}{\pgfqpoint{1.135909in}{3.426264in}}{\pgfqpoint{1.128095in}{3.434077in}}%
\pgfpathcurveto{\pgfqpoint{1.120282in}{3.441891in}}{\pgfqpoint{1.109683in}{3.446281in}}{\pgfqpoint{1.098633in}{3.446281in}}%
\pgfpathcurveto{\pgfqpoint{1.087583in}{3.446281in}}{\pgfqpoint{1.076984in}{3.441891in}}{\pgfqpoint{1.069170in}{3.434077in}}%
\pgfpathcurveto{\pgfqpoint{1.061356in}{3.426264in}}{\pgfqpoint{1.056966in}{3.415665in}}{\pgfqpoint{1.056966in}{3.404615in}}%
\pgfpathcurveto{\pgfqpoint{1.056966in}{3.393564in}}{\pgfqpoint{1.061356in}{3.382965in}}{\pgfqpoint{1.069170in}{3.375152in}}%
\pgfpathcurveto{\pgfqpoint{1.076984in}{3.367338in}}{\pgfqpoint{1.087583in}{3.362948in}}{\pgfqpoint{1.098633in}{3.362948in}}%
\pgfpathclose%
\pgfusepath{stroke,fill}%
\end{pgfscope}%
\begin{pgfscope}%
\pgfpathrectangle{\pgfqpoint{0.648703in}{0.548769in}}{\pgfqpoint{5.201297in}{3.102590in}}%
\pgfusepath{clip}%
\pgfsetbuttcap%
\pgfsetroundjoin%
\definecolor{currentfill}{rgb}{1.000000,0.498039,0.054902}%
\pgfsetfillcolor{currentfill}%
\pgfsetlinewidth{1.003750pt}%
\definecolor{currentstroke}{rgb}{1.000000,0.498039,0.054902}%
\pgfsetstrokecolor{currentstroke}%
\pgfsetdash{}{0pt}%
\pgfpathmoveto{\pgfqpoint{2.178230in}{3.257231in}}%
\pgfpathcurveto{\pgfqpoint{2.189280in}{3.257231in}}{\pgfqpoint{2.199879in}{3.261621in}}{\pgfqpoint{2.207693in}{3.269435in}}%
\pgfpathcurveto{\pgfqpoint{2.215507in}{3.277248in}}{\pgfqpoint{2.219897in}{3.287847in}}{\pgfqpoint{2.219897in}{3.298897in}}%
\pgfpathcurveto{\pgfqpoint{2.219897in}{3.309947in}}{\pgfqpoint{2.215507in}{3.320546in}}{\pgfqpoint{2.207693in}{3.328360in}}%
\pgfpathcurveto{\pgfqpoint{2.199879in}{3.336174in}}{\pgfqpoint{2.189280in}{3.340564in}}{\pgfqpoint{2.178230in}{3.340564in}}%
\pgfpathcurveto{\pgfqpoint{2.167180in}{3.340564in}}{\pgfqpoint{2.156581in}{3.336174in}}{\pgfqpoint{2.148767in}{3.328360in}}%
\pgfpathcurveto{\pgfqpoint{2.140954in}{3.320546in}}{\pgfqpoint{2.136564in}{3.309947in}}{\pgfqpoint{2.136564in}{3.298897in}}%
\pgfpathcurveto{\pgfqpoint{2.136564in}{3.287847in}}{\pgfqpoint{2.140954in}{3.277248in}}{\pgfqpoint{2.148767in}{3.269435in}}%
\pgfpathcurveto{\pgfqpoint{2.156581in}{3.261621in}}{\pgfqpoint{2.167180in}{3.257231in}}{\pgfqpoint{2.178230in}{3.257231in}}%
\pgfpathclose%
\pgfusepath{stroke,fill}%
\end{pgfscope}%
\begin{pgfscope}%
\pgfpathrectangle{\pgfqpoint{0.648703in}{0.548769in}}{\pgfqpoint{5.201297in}{3.102590in}}%
\pgfusepath{clip}%
\pgfsetbuttcap%
\pgfsetroundjoin%
\definecolor{currentfill}{rgb}{1.000000,0.498039,0.054902}%
\pgfsetfillcolor{currentfill}%
\pgfsetlinewidth{1.003750pt}%
\definecolor{currentstroke}{rgb}{1.000000,0.498039,0.054902}%
\pgfsetstrokecolor{currentstroke}%
\pgfsetdash{}{0pt}%
\pgfpathmoveto{\pgfqpoint{1.365390in}{3.189572in}}%
\pgfpathcurveto{\pgfqpoint{1.376441in}{3.189572in}}{\pgfqpoint{1.387040in}{3.193962in}}{\pgfqpoint{1.394853in}{3.201775in}}%
\pgfpathcurveto{\pgfqpoint{1.402667in}{3.209589in}}{\pgfqpoint{1.407057in}{3.220188in}}{\pgfqpoint{1.407057in}{3.231238in}}%
\pgfpathcurveto{\pgfqpoint{1.407057in}{3.242288in}}{\pgfqpoint{1.402667in}{3.252887in}}{\pgfqpoint{1.394853in}{3.260701in}}%
\pgfpathcurveto{\pgfqpoint{1.387040in}{3.268515in}}{\pgfqpoint{1.376441in}{3.272905in}}{\pgfqpoint{1.365390in}{3.272905in}}%
\pgfpathcurveto{\pgfqpoint{1.354340in}{3.272905in}}{\pgfqpoint{1.343741in}{3.268515in}}{\pgfqpoint{1.335928in}{3.260701in}}%
\pgfpathcurveto{\pgfqpoint{1.328114in}{3.252887in}}{\pgfqpoint{1.323724in}{3.242288in}}{\pgfqpoint{1.323724in}{3.231238in}}%
\pgfpathcurveto{\pgfqpoint{1.323724in}{3.220188in}}{\pgfqpoint{1.328114in}{3.209589in}}{\pgfqpoint{1.335928in}{3.201775in}}%
\pgfpathcurveto{\pgfqpoint{1.343741in}{3.193962in}}{\pgfqpoint{1.354340in}{3.189572in}}{\pgfqpoint{1.365390in}{3.189572in}}%
\pgfpathclose%
\pgfusepath{stroke,fill}%
\end{pgfscope}%
\begin{pgfscope}%
\pgfpathrectangle{\pgfqpoint{0.648703in}{0.548769in}}{\pgfqpoint{5.201297in}{3.102590in}}%
\pgfusepath{clip}%
\pgfsetbuttcap%
\pgfsetroundjoin%
\definecolor{currentfill}{rgb}{1.000000,0.498039,0.054902}%
\pgfsetfillcolor{currentfill}%
\pgfsetlinewidth{1.003750pt}%
\definecolor{currentstroke}{rgb}{1.000000,0.498039,0.054902}%
\pgfsetstrokecolor{currentstroke}%
\pgfsetdash{}{0pt}%
\pgfpathmoveto{\pgfqpoint{1.871299in}{3.185343in}}%
\pgfpathcurveto{\pgfqpoint{1.882349in}{3.185343in}}{\pgfqpoint{1.892948in}{3.189733in}}{\pgfqpoint{1.900762in}{3.197547in}}%
\pgfpathcurveto{\pgfqpoint{1.908575in}{3.205360in}}{\pgfqpoint{1.912965in}{3.215959in}}{\pgfqpoint{1.912965in}{3.227010in}}%
\pgfpathcurveto{\pgfqpoint{1.912965in}{3.238060in}}{\pgfqpoint{1.908575in}{3.248659in}}{\pgfqpoint{1.900762in}{3.256472in}}%
\pgfpathcurveto{\pgfqpoint{1.892948in}{3.264286in}}{\pgfqpoint{1.882349in}{3.268676in}}{\pgfqpoint{1.871299in}{3.268676in}}%
\pgfpathcurveto{\pgfqpoint{1.860249in}{3.268676in}}{\pgfqpoint{1.849650in}{3.264286in}}{\pgfqpoint{1.841836in}{3.256472in}}%
\pgfpathcurveto{\pgfqpoint{1.834022in}{3.248659in}}{\pgfqpoint{1.829632in}{3.238060in}}{\pgfqpoint{1.829632in}{3.227010in}}%
\pgfpathcurveto{\pgfqpoint{1.829632in}{3.215959in}}{\pgfqpoint{1.834022in}{3.205360in}}{\pgfqpoint{1.841836in}{3.197547in}}%
\pgfpathcurveto{\pgfqpoint{1.849650in}{3.189733in}}{\pgfqpoint{1.860249in}{3.185343in}}{\pgfqpoint{1.871299in}{3.185343in}}%
\pgfpathclose%
\pgfusepath{stroke,fill}%
\end{pgfscope}%
\begin{pgfscope}%
\pgfpathrectangle{\pgfqpoint{0.648703in}{0.548769in}}{\pgfqpoint{5.201297in}{3.102590in}}%
\pgfusepath{clip}%
\pgfsetbuttcap%
\pgfsetroundjoin%
\definecolor{currentfill}{rgb}{0.121569,0.466667,0.705882}%
\pgfsetfillcolor{currentfill}%
\pgfsetlinewidth{1.003750pt}%
\definecolor{currentstroke}{rgb}{0.121569,0.466667,0.705882}%
\pgfsetstrokecolor{currentstroke}%
\pgfsetdash{}{0pt}%
\pgfpathmoveto{\pgfqpoint{1.227248in}{0.681958in}}%
\pgfpathcurveto{\pgfqpoint{1.238298in}{0.681958in}}{\pgfqpoint{1.248897in}{0.686349in}}{\pgfqpoint{1.256711in}{0.694162in}}%
\pgfpathcurveto{\pgfqpoint{1.264524in}{0.701976in}}{\pgfqpoint{1.268915in}{0.712575in}}{\pgfqpoint{1.268915in}{0.723625in}}%
\pgfpathcurveto{\pgfqpoint{1.268915in}{0.734675in}}{\pgfqpoint{1.264524in}{0.745274in}}{\pgfqpoint{1.256711in}{0.753088in}}%
\pgfpathcurveto{\pgfqpoint{1.248897in}{0.760902in}}{\pgfqpoint{1.238298in}{0.765292in}}{\pgfqpoint{1.227248in}{0.765292in}}%
\pgfpathcurveto{\pgfqpoint{1.216198in}{0.765292in}}{\pgfqpoint{1.205599in}{0.760902in}}{\pgfqpoint{1.197785in}{0.753088in}}%
\pgfpathcurveto{\pgfqpoint{1.189972in}{0.745274in}}{\pgfqpoint{1.185581in}{0.734675in}}{\pgfqpoint{1.185581in}{0.723625in}}%
\pgfpathcurveto{\pgfqpoint{1.185581in}{0.712575in}}{\pgfqpoint{1.189972in}{0.701976in}}{\pgfqpoint{1.197785in}{0.694162in}}%
\pgfpathcurveto{\pgfqpoint{1.205599in}{0.686349in}}{\pgfqpoint{1.216198in}{0.681958in}}{\pgfqpoint{1.227248in}{0.681958in}}%
\pgfpathclose%
\pgfusepath{stroke,fill}%
\end{pgfscope}%
\begin{pgfscope}%
\pgfpathrectangle{\pgfqpoint{0.648703in}{0.548769in}}{\pgfqpoint{5.201297in}{3.102590in}}%
\pgfusepath{clip}%
\pgfsetbuttcap%
\pgfsetroundjoin%
\definecolor{currentfill}{rgb}{0.839216,0.152941,0.156863}%
\pgfsetfillcolor{currentfill}%
\pgfsetlinewidth{1.003750pt}%
\definecolor{currentstroke}{rgb}{0.839216,0.152941,0.156863}%
\pgfsetstrokecolor{currentstroke}%
\pgfsetdash{}{0pt}%
\pgfpathmoveto{\pgfqpoint{2.173089in}{3.214944in}}%
\pgfpathcurveto{\pgfqpoint{2.184139in}{3.214944in}}{\pgfqpoint{2.194738in}{3.219334in}}{\pgfqpoint{2.202552in}{3.227148in}}%
\pgfpathcurveto{\pgfqpoint{2.210365in}{3.234961in}}{\pgfqpoint{2.214755in}{3.245560in}}{\pgfqpoint{2.214755in}{3.256610in}}%
\pgfpathcurveto{\pgfqpoint{2.214755in}{3.267661in}}{\pgfqpoint{2.210365in}{3.278260in}}{\pgfqpoint{2.202552in}{3.286073in}}%
\pgfpathcurveto{\pgfqpoint{2.194738in}{3.293887in}}{\pgfqpoint{2.184139in}{3.298277in}}{\pgfqpoint{2.173089in}{3.298277in}}%
\pgfpathcurveto{\pgfqpoint{2.162039in}{3.298277in}}{\pgfqpoint{2.151440in}{3.293887in}}{\pgfqpoint{2.143626in}{3.286073in}}%
\pgfpathcurveto{\pgfqpoint{2.135812in}{3.278260in}}{\pgfqpoint{2.131422in}{3.267661in}}{\pgfqpoint{2.131422in}{3.256610in}}%
\pgfpathcurveto{\pgfqpoint{2.131422in}{3.245560in}}{\pgfqpoint{2.135812in}{3.234961in}}{\pgfqpoint{2.143626in}{3.227148in}}%
\pgfpathcurveto{\pgfqpoint{2.151440in}{3.219334in}}{\pgfqpoint{2.162039in}{3.214944in}}{\pgfqpoint{2.173089in}{3.214944in}}%
\pgfpathclose%
\pgfusepath{stroke,fill}%
\end{pgfscope}%
\begin{pgfscope}%
\pgfpathrectangle{\pgfqpoint{0.648703in}{0.548769in}}{\pgfqpoint{5.201297in}{3.102590in}}%
\pgfusepath{clip}%
\pgfsetbuttcap%
\pgfsetroundjoin%
\definecolor{currentfill}{rgb}{1.000000,0.498039,0.054902}%
\pgfsetfillcolor{currentfill}%
\pgfsetlinewidth{1.003750pt}%
\definecolor{currentstroke}{rgb}{1.000000,0.498039,0.054902}%
\pgfsetstrokecolor{currentstroke}%
\pgfsetdash{}{0pt}%
\pgfpathmoveto{\pgfqpoint{1.189089in}{3.185343in}}%
\pgfpathcurveto{\pgfqpoint{1.200139in}{3.185343in}}{\pgfqpoint{1.210738in}{3.189733in}}{\pgfqpoint{1.218552in}{3.197547in}}%
\pgfpathcurveto{\pgfqpoint{1.226365in}{3.205360in}}{\pgfqpoint{1.230756in}{3.215959in}}{\pgfqpoint{1.230756in}{3.227010in}}%
\pgfpathcurveto{\pgfqpoint{1.230756in}{3.238060in}}{\pgfqpoint{1.226365in}{3.248659in}}{\pgfqpoint{1.218552in}{3.256472in}}%
\pgfpathcurveto{\pgfqpoint{1.210738in}{3.264286in}}{\pgfqpoint{1.200139in}{3.268676in}}{\pgfqpoint{1.189089in}{3.268676in}}%
\pgfpathcurveto{\pgfqpoint{1.178039in}{3.268676in}}{\pgfqpoint{1.167440in}{3.264286in}}{\pgfqpoint{1.159626in}{3.256472in}}%
\pgfpathcurveto{\pgfqpoint{1.151813in}{3.248659in}}{\pgfqpoint{1.147422in}{3.238060in}}{\pgfqpoint{1.147422in}{3.227010in}}%
\pgfpathcurveto{\pgfqpoint{1.147422in}{3.215959in}}{\pgfqpoint{1.151813in}{3.205360in}}{\pgfqpoint{1.159626in}{3.197547in}}%
\pgfpathcurveto{\pgfqpoint{1.167440in}{3.189733in}}{\pgfqpoint{1.178039in}{3.185343in}}{\pgfqpoint{1.189089in}{3.185343in}}%
\pgfpathclose%
\pgfusepath{stroke,fill}%
\end{pgfscope}%
\begin{pgfscope}%
\pgfpathrectangle{\pgfqpoint{0.648703in}{0.548769in}}{\pgfqpoint{5.201297in}{3.102590in}}%
\pgfusepath{clip}%
\pgfsetbuttcap%
\pgfsetroundjoin%
\definecolor{currentfill}{rgb}{0.121569,0.466667,0.705882}%
\pgfsetfillcolor{currentfill}%
\pgfsetlinewidth{1.003750pt}%
\definecolor{currentstroke}{rgb}{0.121569,0.466667,0.705882}%
\pgfsetstrokecolor{currentstroke}%
\pgfsetdash{}{0pt}%
\pgfpathmoveto{\pgfqpoint{1.170356in}{1.138657in}}%
\pgfpathcurveto{\pgfqpoint{1.181406in}{1.138657in}}{\pgfqpoint{1.192005in}{1.143047in}}{\pgfqpoint{1.199819in}{1.150861in}}%
\pgfpathcurveto{\pgfqpoint{1.207632in}{1.158674in}}{\pgfqpoint{1.212022in}{1.169274in}}{\pgfqpoint{1.212022in}{1.180324in}}%
\pgfpathcurveto{\pgfqpoint{1.212022in}{1.191374in}}{\pgfqpoint{1.207632in}{1.201973in}}{\pgfqpoint{1.199819in}{1.209786in}}%
\pgfpathcurveto{\pgfqpoint{1.192005in}{1.217600in}}{\pgfqpoint{1.181406in}{1.221990in}}{\pgfqpoint{1.170356in}{1.221990in}}%
\pgfpathcurveto{\pgfqpoint{1.159306in}{1.221990in}}{\pgfqpoint{1.148707in}{1.217600in}}{\pgfqpoint{1.140893in}{1.209786in}}%
\pgfpathcurveto{\pgfqpoint{1.133079in}{1.201973in}}{\pgfqpoint{1.128689in}{1.191374in}}{\pgfqpoint{1.128689in}{1.180324in}}%
\pgfpathcurveto{\pgfqpoint{1.128689in}{1.169274in}}{\pgfqpoint{1.133079in}{1.158674in}}{\pgfqpoint{1.140893in}{1.150861in}}%
\pgfpathcurveto{\pgfqpoint{1.148707in}{1.143047in}}{\pgfqpoint{1.159306in}{1.138657in}}{\pgfqpoint{1.170356in}{1.138657in}}%
\pgfpathclose%
\pgfusepath{stroke,fill}%
\end{pgfscope}%
\begin{pgfscope}%
\pgfpathrectangle{\pgfqpoint{0.648703in}{0.548769in}}{\pgfqpoint{5.201297in}{3.102590in}}%
\pgfusepath{clip}%
\pgfsetbuttcap%
\pgfsetroundjoin%
\definecolor{currentfill}{rgb}{0.121569,0.466667,0.705882}%
\pgfsetfillcolor{currentfill}%
\pgfsetlinewidth{1.003750pt}%
\definecolor{currentstroke}{rgb}{0.121569,0.466667,0.705882}%
\pgfsetstrokecolor{currentstroke}%
\pgfsetdash{}{0pt}%
\pgfpathmoveto{\pgfqpoint{2.065696in}{1.087913in}}%
\pgfpathcurveto{\pgfqpoint{2.076746in}{1.087913in}}{\pgfqpoint{2.087345in}{1.092303in}}{\pgfqpoint{2.095159in}{1.100117in}}%
\pgfpathcurveto{\pgfqpoint{2.102973in}{1.107930in}}{\pgfqpoint{2.107363in}{1.118529in}}{\pgfqpoint{2.107363in}{1.129579in}}%
\pgfpathcurveto{\pgfqpoint{2.107363in}{1.140629in}}{\pgfqpoint{2.102973in}{1.151229in}}{\pgfqpoint{2.095159in}{1.159042in}}%
\pgfpathcurveto{\pgfqpoint{2.087345in}{1.166856in}}{\pgfqpoint{2.076746in}{1.171246in}}{\pgfqpoint{2.065696in}{1.171246in}}%
\pgfpathcurveto{\pgfqpoint{2.054646in}{1.171246in}}{\pgfqpoint{2.044047in}{1.166856in}}{\pgfqpoint{2.036233in}{1.159042in}}%
\pgfpathcurveto{\pgfqpoint{2.028420in}{1.151229in}}{\pgfqpoint{2.024030in}{1.140629in}}{\pgfqpoint{2.024030in}{1.129579in}}%
\pgfpathcurveto{\pgfqpoint{2.024030in}{1.118529in}}{\pgfqpoint{2.028420in}{1.107930in}}{\pgfqpoint{2.036233in}{1.100117in}}%
\pgfpathcurveto{\pgfqpoint{2.044047in}{1.092303in}}{\pgfqpoint{2.054646in}{1.087913in}}{\pgfqpoint{2.065696in}{1.087913in}}%
\pgfpathclose%
\pgfusepath{stroke,fill}%
\end{pgfscope}%
\begin{pgfscope}%
\pgfpathrectangle{\pgfqpoint{0.648703in}{0.548769in}}{\pgfqpoint{5.201297in}{3.102590in}}%
\pgfusepath{clip}%
\pgfsetbuttcap%
\pgfsetroundjoin%
\definecolor{currentfill}{rgb}{0.121569,0.466667,0.705882}%
\pgfsetfillcolor{currentfill}%
\pgfsetlinewidth{1.003750pt}%
\definecolor{currentstroke}{rgb}{0.121569,0.466667,0.705882}%
\pgfsetstrokecolor{currentstroke}%
\pgfsetdash{}{0pt}%
\pgfpathmoveto{\pgfqpoint{1.664287in}{1.024482in}}%
\pgfpathcurveto{\pgfqpoint{1.675337in}{1.024482in}}{\pgfqpoint{1.685936in}{1.028873in}}{\pgfqpoint{1.693750in}{1.036686in}}%
\pgfpathcurveto{\pgfqpoint{1.701564in}{1.044500in}}{\pgfqpoint{1.705954in}{1.055099in}}{\pgfqpoint{1.705954in}{1.066149in}}%
\pgfpathcurveto{\pgfqpoint{1.705954in}{1.077199in}}{\pgfqpoint{1.701564in}{1.087798in}}{\pgfqpoint{1.693750in}{1.095612in}}%
\pgfpathcurveto{\pgfqpoint{1.685936in}{1.103425in}}{\pgfqpoint{1.675337in}{1.107816in}}{\pgfqpoint{1.664287in}{1.107816in}}%
\pgfpathcurveto{\pgfqpoint{1.653237in}{1.107816in}}{\pgfqpoint{1.642638in}{1.103425in}}{\pgfqpoint{1.634824in}{1.095612in}}%
\pgfpathcurveto{\pgfqpoint{1.627011in}{1.087798in}}{\pgfqpoint{1.622620in}{1.077199in}}{\pgfqpoint{1.622620in}{1.066149in}}%
\pgfpathcurveto{\pgfqpoint{1.622620in}{1.055099in}}{\pgfqpoint{1.627011in}{1.044500in}}{\pgfqpoint{1.634824in}{1.036686in}}%
\pgfpathcurveto{\pgfqpoint{1.642638in}{1.028873in}}{\pgfqpoint{1.653237in}{1.024482in}}{\pgfqpoint{1.664287in}{1.024482in}}%
\pgfpathclose%
\pgfusepath{stroke,fill}%
\end{pgfscope}%
\begin{pgfscope}%
\pgfpathrectangle{\pgfqpoint{0.648703in}{0.548769in}}{\pgfqpoint{5.201297in}{3.102590in}}%
\pgfusepath{clip}%
\pgfsetbuttcap%
\pgfsetroundjoin%
\definecolor{currentfill}{rgb}{0.121569,0.466667,0.705882}%
\pgfsetfillcolor{currentfill}%
\pgfsetlinewidth{1.003750pt}%
\definecolor{currentstroke}{rgb}{0.121569,0.466667,0.705882}%
\pgfsetstrokecolor{currentstroke}%
\pgfsetdash{}{0pt}%
\pgfpathmoveto{\pgfqpoint{2.149020in}{0.813048in}}%
\pgfpathcurveto{\pgfqpoint{2.160070in}{0.813048in}}{\pgfqpoint{2.170669in}{0.817438in}}{\pgfqpoint{2.178483in}{0.825252in}}%
\pgfpathcurveto{\pgfqpoint{2.186297in}{0.833065in}}{\pgfqpoint{2.190687in}{0.843664in}}{\pgfqpoint{2.190687in}{0.854715in}}%
\pgfpathcurveto{\pgfqpoint{2.190687in}{0.865765in}}{\pgfqpoint{2.186297in}{0.876364in}}{\pgfqpoint{2.178483in}{0.884177in}}%
\pgfpathcurveto{\pgfqpoint{2.170669in}{0.891991in}}{\pgfqpoint{2.160070in}{0.896381in}}{\pgfqpoint{2.149020in}{0.896381in}}%
\pgfpathcurveto{\pgfqpoint{2.137970in}{0.896381in}}{\pgfqpoint{2.127371in}{0.891991in}}{\pgfqpoint{2.119557in}{0.884177in}}%
\pgfpathcurveto{\pgfqpoint{2.111744in}{0.876364in}}{\pgfqpoint{2.107353in}{0.865765in}}{\pgfqpoint{2.107353in}{0.854715in}}%
\pgfpathcurveto{\pgfqpoint{2.107353in}{0.843664in}}{\pgfqpoint{2.111744in}{0.833065in}}{\pgfqpoint{2.119557in}{0.825252in}}%
\pgfpathcurveto{\pgfqpoint{2.127371in}{0.817438in}}{\pgfqpoint{2.137970in}{0.813048in}}{\pgfqpoint{2.149020in}{0.813048in}}%
\pgfpathclose%
\pgfusepath{stroke,fill}%
\end{pgfscope}%
\begin{pgfscope}%
\pgfpathrectangle{\pgfqpoint{0.648703in}{0.548769in}}{\pgfqpoint{5.201297in}{3.102590in}}%
\pgfusepath{clip}%
\pgfsetbuttcap%
\pgfsetroundjoin%
\definecolor{currentfill}{rgb}{0.839216,0.152941,0.156863}%
\pgfsetfillcolor{currentfill}%
\pgfsetlinewidth{1.003750pt}%
\definecolor{currentstroke}{rgb}{0.839216,0.152941,0.156863}%
\pgfsetstrokecolor{currentstroke}%
\pgfsetdash{}{0pt}%
\pgfpathmoveto{\pgfqpoint{2.053834in}{3.202258in}}%
\pgfpathcurveto{\pgfqpoint{2.064884in}{3.202258in}}{\pgfqpoint{2.075483in}{3.206648in}}{\pgfqpoint{2.083297in}{3.214462in}}%
\pgfpathcurveto{\pgfqpoint{2.091110in}{3.222275in}}{\pgfqpoint{2.095501in}{3.232874in}}{\pgfqpoint{2.095501in}{3.243924in}}%
\pgfpathcurveto{\pgfqpoint{2.095501in}{3.254974in}}{\pgfqpoint{2.091110in}{3.265573in}}{\pgfqpoint{2.083297in}{3.273387in}}%
\pgfpathcurveto{\pgfqpoint{2.075483in}{3.281201in}}{\pgfqpoint{2.064884in}{3.285591in}}{\pgfqpoint{2.053834in}{3.285591in}}%
\pgfpathcurveto{\pgfqpoint{2.042784in}{3.285591in}}{\pgfqpoint{2.032185in}{3.281201in}}{\pgfqpoint{2.024371in}{3.273387in}}%
\pgfpathcurveto{\pgfqpoint{2.016558in}{3.265573in}}{\pgfqpoint{2.012167in}{3.254974in}}{\pgfqpoint{2.012167in}{3.243924in}}%
\pgfpathcurveto{\pgfqpoint{2.012167in}{3.232874in}}{\pgfqpoint{2.016558in}{3.222275in}}{\pgfqpoint{2.024371in}{3.214462in}}%
\pgfpathcurveto{\pgfqpoint{2.032185in}{3.206648in}}{\pgfqpoint{2.042784in}{3.202258in}}{\pgfqpoint{2.053834in}{3.202258in}}%
\pgfpathclose%
\pgfusepath{stroke,fill}%
\end{pgfscope}%
\begin{pgfscope}%
\pgfsetbuttcap%
\pgfsetroundjoin%
\definecolor{currentfill}{rgb}{0.000000,0.000000,0.000000}%
\pgfsetfillcolor{currentfill}%
\pgfsetlinewidth{0.803000pt}%
\definecolor{currentstroke}{rgb}{0.000000,0.000000,0.000000}%
\pgfsetstrokecolor{currentstroke}%
\pgfsetdash{}{0pt}%
\pgfsys@defobject{currentmarker}{\pgfqpoint{0.000000in}{-0.048611in}}{\pgfqpoint{0.000000in}{0.000000in}}{%
\pgfpathmoveto{\pgfqpoint{0.000000in}{0.000000in}}%
\pgfpathlineto{\pgfqpoint{0.000000in}{-0.048611in}}%
\pgfusepath{stroke,fill}%
}%
\begin{pgfscope}%
\pgfsys@transformshift{0.872963in}{0.548769in}%
\pgfsys@useobject{currentmarker}{}%
\end{pgfscope}%
\end{pgfscope}%
\begin{pgfscope}%
\definecolor{textcolor}{rgb}{0.000000,0.000000,0.000000}%
\pgfsetstrokecolor{textcolor}%
\pgfsetfillcolor{textcolor}%
\pgftext[x=0.872963in,y=0.451547in,,top]{\color{textcolor}\sffamily\fontsize{10.000000}{12.000000}\selectfont \(\displaystyle {0.0}\)}%
\end{pgfscope}%
\begin{pgfscope}%
\pgfsetbuttcap%
\pgfsetroundjoin%
\definecolor{currentfill}{rgb}{0.000000,0.000000,0.000000}%
\pgfsetfillcolor{currentfill}%
\pgfsetlinewidth{0.803000pt}%
\definecolor{currentstroke}{rgb}{0.000000,0.000000,0.000000}%
\pgfsetstrokecolor{currentstroke}%
\pgfsetdash{}{0pt}%
\pgfsys@defobject{currentmarker}{\pgfqpoint{0.000000in}{-0.048611in}}{\pgfqpoint{0.000000in}{0.000000in}}{%
\pgfpathmoveto{\pgfqpoint{0.000000in}{0.000000in}}%
\pgfpathlineto{\pgfqpoint{0.000000in}{-0.048611in}}%
\pgfusepath{stroke,fill}%
}%
\begin{pgfscope}%
\pgfsys@transformshift{1.664564in}{0.548769in}%
\pgfsys@useobject{currentmarker}{}%
\end{pgfscope}%
\end{pgfscope}%
\begin{pgfscope}%
\definecolor{textcolor}{rgb}{0.000000,0.000000,0.000000}%
\pgfsetstrokecolor{textcolor}%
\pgfsetfillcolor{textcolor}%
\pgftext[x=1.664564in,y=0.451547in,,top]{\color{textcolor}\sffamily\fontsize{10.000000}{12.000000}\selectfont \(\displaystyle {0.2}\)}%
\end{pgfscope}%
\begin{pgfscope}%
\pgfsetbuttcap%
\pgfsetroundjoin%
\definecolor{currentfill}{rgb}{0.000000,0.000000,0.000000}%
\pgfsetfillcolor{currentfill}%
\pgfsetlinewidth{0.803000pt}%
\definecolor{currentstroke}{rgb}{0.000000,0.000000,0.000000}%
\pgfsetstrokecolor{currentstroke}%
\pgfsetdash{}{0pt}%
\pgfsys@defobject{currentmarker}{\pgfqpoint{0.000000in}{-0.048611in}}{\pgfqpoint{0.000000in}{0.000000in}}{%
\pgfpathmoveto{\pgfqpoint{0.000000in}{0.000000in}}%
\pgfpathlineto{\pgfqpoint{0.000000in}{-0.048611in}}%
\pgfusepath{stroke,fill}%
}%
\begin{pgfscope}%
\pgfsys@transformshift{2.456165in}{0.548769in}%
\pgfsys@useobject{currentmarker}{}%
\end{pgfscope}%
\end{pgfscope}%
\begin{pgfscope}%
\definecolor{textcolor}{rgb}{0.000000,0.000000,0.000000}%
\pgfsetstrokecolor{textcolor}%
\pgfsetfillcolor{textcolor}%
\pgftext[x=2.456165in,y=0.451547in,,top]{\color{textcolor}\sffamily\fontsize{10.000000}{12.000000}\selectfont \(\displaystyle {0.4}\)}%
\end{pgfscope}%
\begin{pgfscope}%
\pgfsetbuttcap%
\pgfsetroundjoin%
\definecolor{currentfill}{rgb}{0.000000,0.000000,0.000000}%
\pgfsetfillcolor{currentfill}%
\pgfsetlinewidth{0.803000pt}%
\definecolor{currentstroke}{rgb}{0.000000,0.000000,0.000000}%
\pgfsetstrokecolor{currentstroke}%
\pgfsetdash{}{0pt}%
\pgfsys@defobject{currentmarker}{\pgfqpoint{0.000000in}{-0.048611in}}{\pgfqpoint{0.000000in}{0.000000in}}{%
\pgfpathmoveto{\pgfqpoint{0.000000in}{0.000000in}}%
\pgfpathlineto{\pgfqpoint{0.000000in}{-0.048611in}}%
\pgfusepath{stroke,fill}%
}%
\begin{pgfscope}%
\pgfsys@transformshift{3.247767in}{0.548769in}%
\pgfsys@useobject{currentmarker}{}%
\end{pgfscope}%
\end{pgfscope}%
\begin{pgfscope}%
\definecolor{textcolor}{rgb}{0.000000,0.000000,0.000000}%
\pgfsetstrokecolor{textcolor}%
\pgfsetfillcolor{textcolor}%
\pgftext[x=3.247767in,y=0.451547in,,top]{\color{textcolor}\sffamily\fontsize{10.000000}{12.000000}\selectfont \(\displaystyle {0.6}\)}%
\end{pgfscope}%
\begin{pgfscope}%
\pgfsetbuttcap%
\pgfsetroundjoin%
\definecolor{currentfill}{rgb}{0.000000,0.000000,0.000000}%
\pgfsetfillcolor{currentfill}%
\pgfsetlinewidth{0.803000pt}%
\definecolor{currentstroke}{rgb}{0.000000,0.000000,0.000000}%
\pgfsetstrokecolor{currentstroke}%
\pgfsetdash{}{0pt}%
\pgfsys@defobject{currentmarker}{\pgfqpoint{0.000000in}{-0.048611in}}{\pgfqpoint{0.000000in}{0.000000in}}{%
\pgfpathmoveto{\pgfqpoint{0.000000in}{0.000000in}}%
\pgfpathlineto{\pgfqpoint{0.000000in}{-0.048611in}}%
\pgfusepath{stroke,fill}%
}%
\begin{pgfscope}%
\pgfsys@transformshift{4.039368in}{0.548769in}%
\pgfsys@useobject{currentmarker}{}%
\end{pgfscope}%
\end{pgfscope}%
\begin{pgfscope}%
\definecolor{textcolor}{rgb}{0.000000,0.000000,0.000000}%
\pgfsetstrokecolor{textcolor}%
\pgfsetfillcolor{textcolor}%
\pgftext[x=4.039368in,y=0.451547in,,top]{\color{textcolor}\sffamily\fontsize{10.000000}{12.000000}\selectfont \(\displaystyle {0.8}\)}%
\end{pgfscope}%
\begin{pgfscope}%
\pgfsetbuttcap%
\pgfsetroundjoin%
\definecolor{currentfill}{rgb}{0.000000,0.000000,0.000000}%
\pgfsetfillcolor{currentfill}%
\pgfsetlinewidth{0.803000pt}%
\definecolor{currentstroke}{rgb}{0.000000,0.000000,0.000000}%
\pgfsetstrokecolor{currentstroke}%
\pgfsetdash{}{0pt}%
\pgfsys@defobject{currentmarker}{\pgfqpoint{0.000000in}{-0.048611in}}{\pgfqpoint{0.000000in}{0.000000in}}{%
\pgfpathmoveto{\pgfqpoint{0.000000in}{0.000000in}}%
\pgfpathlineto{\pgfqpoint{0.000000in}{-0.048611in}}%
\pgfusepath{stroke,fill}%
}%
\begin{pgfscope}%
\pgfsys@transformshift{4.830969in}{0.548769in}%
\pgfsys@useobject{currentmarker}{}%
\end{pgfscope}%
\end{pgfscope}%
\begin{pgfscope}%
\definecolor{textcolor}{rgb}{0.000000,0.000000,0.000000}%
\pgfsetstrokecolor{textcolor}%
\pgfsetfillcolor{textcolor}%
\pgftext[x=4.830969in,y=0.451547in,,top]{\color{textcolor}\sffamily\fontsize{10.000000}{12.000000}\selectfont \(\displaystyle {1.0}\)}%
\end{pgfscope}%
\begin{pgfscope}%
\pgfsetbuttcap%
\pgfsetroundjoin%
\definecolor{currentfill}{rgb}{0.000000,0.000000,0.000000}%
\pgfsetfillcolor{currentfill}%
\pgfsetlinewidth{0.803000pt}%
\definecolor{currentstroke}{rgb}{0.000000,0.000000,0.000000}%
\pgfsetstrokecolor{currentstroke}%
\pgfsetdash{}{0pt}%
\pgfsys@defobject{currentmarker}{\pgfqpoint{0.000000in}{-0.048611in}}{\pgfqpoint{0.000000in}{0.000000in}}{%
\pgfpathmoveto{\pgfqpoint{0.000000in}{0.000000in}}%
\pgfpathlineto{\pgfqpoint{0.000000in}{-0.048611in}}%
\pgfusepath{stroke,fill}%
}%
\begin{pgfscope}%
\pgfsys@transformshift{5.622570in}{0.548769in}%
\pgfsys@useobject{currentmarker}{}%
\end{pgfscope}%
\end{pgfscope}%
\begin{pgfscope}%
\definecolor{textcolor}{rgb}{0.000000,0.000000,0.000000}%
\pgfsetstrokecolor{textcolor}%
\pgfsetfillcolor{textcolor}%
\pgftext[x=5.622570in,y=0.451547in,,top]{\color{textcolor}\sffamily\fontsize{10.000000}{12.000000}\selectfont \(\displaystyle {1.2}\)}%
\end{pgfscope}%
\begin{pgfscope}%
\definecolor{textcolor}{rgb}{0.000000,0.000000,0.000000}%
\pgfsetstrokecolor{textcolor}%
\pgfsetfillcolor{textcolor}%
\pgftext[x=3.249352in,y=0.272658in,,top]{\color{textcolor}\sffamily\fontsize{10.000000}{12.000000}\selectfont Statements}%
\end{pgfscope}%
\begin{pgfscope}%
\definecolor{textcolor}{rgb}{0.000000,0.000000,0.000000}%
\pgfsetstrokecolor{textcolor}%
\pgfsetfillcolor{textcolor}%
\pgftext[x=5.850000in,y=0.286547in,right,top]{\color{textcolor}\sffamily\fontsize{10.000000}{12.000000}\selectfont \(\displaystyle \times{10^{6}}{}\)}%
\end{pgfscope}%
\begin{pgfscope}%
\pgfsetbuttcap%
\pgfsetroundjoin%
\definecolor{currentfill}{rgb}{0.000000,0.000000,0.000000}%
\pgfsetfillcolor{currentfill}%
\pgfsetlinewidth{0.803000pt}%
\definecolor{currentstroke}{rgb}{0.000000,0.000000,0.000000}%
\pgfsetstrokecolor{currentstroke}%
\pgfsetdash{}{0pt}%
\pgfsys@defobject{currentmarker}{\pgfqpoint{-0.048611in}{0.000000in}}{\pgfqpoint{0.000000in}{0.000000in}}{%
\pgfpathmoveto{\pgfqpoint{0.000000in}{0.000000in}}%
\pgfpathlineto{\pgfqpoint{-0.048611in}{0.000000in}}%
\pgfusepath{stroke,fill}%
}%
\begin{pgfscope}%
\pgfsys@transformshift{0.648703in}{0.689796in}%
\pgfsys@useobject{currentmarker}{}%
\end{pgfscope}%
\end{pgfscope}%
\begin{pgfscope}%
\definecolor{textcolor}{rgb}{0.000000,0.000000,0.000000}%
\pgfsetstrokecolor{textcolor}%
\pgfsetfillcolor{textcolor}%
\pgftext[x=0.482036in, y=0.641601in, left, base]{\color{textcolor}\sffamily\fontsize{10.000000}{12.000000}\selectfont \(\displaystyle {0}\)}%
\end{pgfscope}%
\begin{pgfscope}%
\pgfsetbuttcap%
\pgfsetroundjoin%
\definecolor{currentfill}{rgb}{0.000000,0.000000,0.000000}%
\pgfsetfillcolor{currentfill}%
\pgfsetlinewidth{0.803000pt}%
\definecolor{currentstroke}{rgb}{0.000000,0.000000,0.000000}%
\pgfsetstrokecolor{currentstroke}%
\pgfsetdash{}{0pt}%
\pgfsys@defobject{currentmarker}{\pgfqpoint{-0.048611in}{0.000000in}}{\pgfqpoint{0.000000in}{0.000000in}}{%
\pgfpathmoveto{\pgfqpoint{0.000000in}{0.000000in}}%
\pgfpathlineto{\pgfqpoint{-0.048611in}{0.000000in}}%
\pgfusepath{stroke,fill}%
}%
\begin{pgfscope}%
\pgfsys@transformshift{0.648703in}{1.112665in}%
\pgfsys@useobject{currentmarker}{}%
\end{pgfscope}%
\end{pgfscope}%
\begin{pgfscope}%
\definecolor{textcolor}{rgb}{0.000000,0.000000,0.000000}%
\pgfsetstrokecolor{textcolor}%
\pgfsetfillcolor{textcolor}%
\pgftext[x=0.343147in, y=1.064470in, left, base]{\color{textcolor}\sffamily\fontsize{10.000000}{12.000000}\selectfont \(\displaystyle {100}\)}%
\end{pgfscope}%
\begin{pgfscope}%
\pgfsetbuttcap%
\pgfsetroundjoin%
\definecolor{currentfill}{rgb}{0.000000,0.000000,0.000000}%
\pgfsetfillcolor{currentfill}%
\pgfsetlinewidth{0.803000pt}%
\definecolor{currentstroke}{rgb}{0.000000,0.000000,0.000000}%
\pgfsetstrokecolor{currentstroke}%
\pgfsetdash{}{0pt}%
\pgfsys@defobject{currentmarker}{\pgfqpoint{-0.048611in}{0.000000in}}{\pgfqpoint{0.000000in}{0.000000in}}{%
\pgfpathmoveto{\pgfqpoint{0.000000in}{0.000000in}}%
\pgfpathlineto{\pgfqpoint{-0.048611in}{0.000000in}}%
\pgfusepath{stroke,fill}%
}%
\begin{pgfscope}%
\pgfsys@transformshift{0.648703in}{1.535534in}%
\pgfsys@useobject{currentmarker}{}%
\end{pgfscope}%
\end{pgfscope}%
\begin{pgfscope}%
\definecolor{textcolor}{rgb}{0.000000,0.000000,0.000000}%
\pgfsetstrokecolor{textcolor}%
\pgfsetfillcolor{textcolor}%
\pgftext[x=0.343147in, y=1.487339in, left, base]{\color{textcolor}\sffamily\fontsize{10.000000}{12.000000}\selectfont \(\displaystyle {200}\)}%
\end{pgfscope}%
\begin{pgfscope}%
\pgfsetbuttcap%
\pgfsetroundjoin%
\definecolor{currentfill}{rgb}{0.000000,0.000000,0.000000}%
\pgfsetfillcolor{currentfill}%
\pgfsetlinewidth{0.803000pt}%
\definecolor{currentstroke}{rgb}{0.000000,0.000000,0.000000}%
\pgfsetstrokecolor{currentstroke}%
\pgfsetdash{}{0pt}%
\pgfsys@defobject{currentmarker}{\pgfqpoint{-0.048611in}{0.000000in}}{\pgfqpoint{0.000000in}{0.000000in}}{%
\pgfpathmoveto{\pgfqpoint{0.000000in}{0.000000in}}%
\pgfpathlineto{\pgfqpoint{-0.048611in}{0.000000in}}%
\pgfusepath{stroke,fill}%
}%
\begin{pgfscope}%
\pgfsys@transformshift{0.648703in}{1.958403in}%
\pgfsys@useobject{currentmarker}{}%
\end{pgfscope}%
\end{pgfscope}%
\begin{pgfscope}%
\definecolor{textcolor}{rgb}{0.000000,0.000000,0.000000}%
\pgfsetstrokecolor{textcolor}%
\pgfsetfillcolor{textcolor}%
\pgftext[x=0.343147in, y=1.910208in, left, base]{\color{textcolor}\sffamily\fontsize{10.000000}{12.000000}\selectfont \(\displaystyle {300}\)}%
\end{pgfscope}%
\begin{pgfscope}%
\pgfsetbuttcap%
\pgfsetroundjoin%
\definecolor{currentfill}{rgb}{0.000000,0.000000,0.000000}%
\pgfsetfillcolor{currentfill}%
\pgfsetlinewidth{0.803000pt}%
\definecolor{currentstroke}{rgb}{0.000000,0.000000,0.000000}%
\pgfsetstrokecolor{currentstroke}%
\pgfsetdash{}{0pt}%
\pgfsys@defobject{currentmarker}{\pgfqpoint{-0.048611in}{0.000000in}}{\pgfqpoint{0.000000in}{0.000000in}}{%
\pgfpathmoveto{\pgfqpoint{0.000000in}{0.000000in}}%
\pgfpathlineto{\pgfqpoint{-0.048611in}{0.000000in}}%
\pgfusepath{stroke,fill}%
}%
\begin{pgfscope}%
\pgfsys@transformshift{0.648703in}{2.381272in}%
\pgfsys@useobject{currentmarker}{}%
\end{pgfscope}%
\end{pgfscope}%
\begin{pgfscope}%
\definecolor{textcolor}{rgb}{0.000000,0.000000,0.000000}%
\pgfsetstrokecolor{textcolor}%
\pgfsetfillcolor{textcolor}%
\pgftext[x=0.343147in, y=2.333077in, left, base]{\color{textcolor}\sffamily\fontsize{10.000000}{12.000000}\selectfont \(\displaystyle {400}\)}%
\end{pgfscope}%
\begin{pgfscope}%
\pgfsetbuttcap%
\pgfsetroundjoin%
\definecolor{currentfill}{rgb}{0.000000,0.000000,0.000000}%
\pgfsetfillcolor{currentfill}%
\pgfsetlinewidth{0.803000pt}%
\definecolor{currentstroke}{rgb}{0.000000,0.000000,0.000000}%
\pgfsetstrokecolor{currentstroke}%
\pgfsetdash{}{0pt}%
\pgfsys@defobject{currentmarker}{\pgfqpoint{-0.048611in}{0.000000in}}{\pgfqpoint{0.000000in}{0.000000in}}{%
\pgfpathmoveto{\pgfqpoint{0.000000in}{0.000000in}}%
\pgfpathlineto{\pgfqpoint{-0.048611in}{0.000000in}}%
\pgfusepath{stroke,fill}%
}%
\begin{pgfscope}%
\pgfsys@transformshift{0.648703in}{2.804141in}%
\pgfsys@useobject{currentmarker}{}%
\end{pgfscope}%
\end{pgfscope}%
\begin{pgfscope}%
\definecolor{textcolor}{rgb}{0.000000,0.000000,0.000000}%
\pgfsetstrokecolor{textcolor}%
\pgfsetfillcolor{textcolor}%
\pgftext[x=0.343147in, y=2.755946in, left, base]{\color{textcolor}\sffamily\fontsize{10.000000}{12.000000}\selectfont \(\displaystyle {500}\)}%
\end{pgfscope}%
\begin{pgfscope}%
\pgfsetbuttcap%
\pgfsetroundjoin%
\definecolor{currentfill}{rgb}{0.000000,0.000000,0.000000}%
\pgfsetfillcolor{currentfill}%
\pgfsetlinewidth{0.803000pt}%
\definecolor{currentstroke}{rgb}{0.000000,0.000000,0.000000}%
\pgfsetstrokecolor{currentstroke}%
\pgfsetdash{}{0pt}%
\pgfsys@defobject{currentmarker}{\pgfqpoint{-0.048611in}{0.000000in}}{\pgfqpoint{0.000000in}{0.000000in}}{%
\pgfpathmoveto{\pgfqpoint{0.000000in}{0.000000in}}%
\pgfpathlineto{\pgfqpoint{-0.048611in}{0.000000in}}%
\pgfusepath{stroke,fill}%
}%
\begin{pgfscope}%
\pgfsys@transformshift{0.648703in}{3.227010in}%
\pgfsys@useobject{currentmarker}{}%
\end{pgfscope}%
\end{pgfscope}%
\begin{pgfscope}%
\definecolor{textcolor}{rgb}{0.000000,0.000000,0.000000}%
\pgfsetstrokecolor{textcolor}%
\pgfsetfillcolor{textcolor}%
\pgftext[x=0.343147in, y=3.178815in, left, base]{\color{textcolor}\sffamily\fontsize{10.000000}{12.000000}\selectfont \(\displaystyle {600}\)}%
\end{pgfscope}%
\begin{pgfscope}%
\pgfsetbuttcap%
\pgfsetroundjoin%
\definecolor{currentfill}{rgb}{0.000000,0.000000,0.000000}%
\pgfsetfillcolor{currentfill}%
\pgfsetlinewidth{0.803000pt}%
\definecolor{currentstroke}{rgb}{0.000000,0.000000,0.000000}%
\pgfsetstrokecolor{currentstroke}%
\pgfsetdash{}{0pt}%
\pgfsys@defobject{currentmarker}{\pgfqpoint{-0.048611in}{0.000000in}}{\pgfqpoint{0.000000in}{0.000000in}}{%
\pgfpathmoveto{\pgfqpoint{0.000000in}{0.000000in}}%
\pgfpathlineto{\pgfqpoint{-0.048611in}{0.000000in}}%
\pgfusepath{stroke,fill}%
}%
\begin{pgfscope}%
\pgfsys@transformshift{0.648703in}{3.649879in}%
\pgfsys@useobject{currentmarker}{}%
\end{pgfscope}%
\end{pgfscope}%
\begin{pgfscope}%
\definecolor{textcolor}{rgb}{0.000000,0.000000,0.000000}%
\pgfsetstrokecolor{textcolor}%
\pgfsetfillcolor{textcolor}%
\pgftext[x=0.343147in, y=3.601684in, left, base]{\color{textcolor}\sffamily\fontsize{10.000000}{12.000000}\selectfont \(\displaystyle {700}\)}%
\end{pgfscope}%
\begin{pgfscope}%
\definecolor{textcolor}{rgb}{0.000000,0.000000,0.000000}%
\pgfsetstrokecolor{textcolor}%
\pgfsetfillcolor{textcolor}%
\pgftext[x=0.287592in,y=2.100064in,,bottom,rotate=90.000000]{\color{textcolor}\sffamily\fontsize{10.000000}{12.000000}\selectfont Data Flow Time (s)}%
\end{pgfscope}%
\begin{pgfscope}%
\pgfsetrectcap%
\pgfsetmiterjoin%
\pgfsetlinewidth{0.803000pt}%
\definecolor{currentstroke}{rgb}{0.000000,0.000000,0.000000}%
\pgfsetstrokecolor{currentstroke}%
\pgfsetdash{}{0pt}%
\pgfpathmoveto{\pgfqpoint{0.648703in}{0.548769in}}%
\pgfpathlineto{\pgfqpoint{0.648703in}{3.651359in}}%
\pgfusepath{stroke}%
\end{pgfscope}%
\begin{pgfscope}%
\pgfsetrectcap%
\pgfsetmiterjoin%
\pgfsetlinewidth{0.803000pt}%
\definecolor{currentstroke}{rgb}{0.000000,0.000000,0.000000}%
\pgfsetstrokecolor{currentstroke}%
\pgfsetdash{}{0pt}%
\pgfpathmoveto{\pgfqpoint{5.850000in}{0.548769in}}%
\pgfpathlineto{\pgfqpoint{5.850000in}{3.651359in}}%
\pgfusepath{stroke}%
\end{pgfscope}%
\begin{pgfscope}%
\pgfsetrectcap%
\pgfsetmiterjoin%
\pgfsetlinewidth{0.803000pt}%
\definecolor{currentstroke}{rgb}{0.000000,0.000000,0.000000}%
\pgfsetstrokecolor{currentstroke}%
\pgfsetdash{}{0pt}%
\pgfpathmoveto{\pgfqpoint{0.648703in}{0.548769in}}%
\pgfpathlineto{\pgfqpoint{5.850000in}{0.548769in}}%
\pgfusepath{stroke}%
\end{pgfscope}%
\begin{pgfscope}%
\pgfsetrectcap%
\pgfsetmiterjoin%
\pgfsetlinewidth{0.803000pt}%
\definecolor{currentstroke}{rgb}{0.000000,0.000000,0.000000}%
\pgfsetstrokecolor{currentstroke}%
\pgfsetdash{}{0pt}%
\pgfpathmoveto{\pgfqpoint{0.648703in}{3.651359in}}%
\pgfpathlineto{\pgfqpoint{5.850000in}{3.651359in}}%
\pgfusepath{stroke}%
\end{pgfscope}%
\begin{pgfscope}%
\definecolor{textcolor}{rgb}{0.000000,0.000000,0.000000}%
\pgfsetstrokecolor{textcolor}%
\pgfsetfillcolor{textcolor}%
\pgftext[x=3.249352in,y=3.734692in,,base]{\color{textcolor}\sffamily\fontsize{12.000000}{14.400000}\selectfont Backwards}%
\end{pgfscope}%
\begin{pgfscope}%
\pgfsetbuttcap%
\pgfsetmiterjoin%
\definecolor{currentfill}{rgb}{1.000000,1.000000,1.000000}%
\pgfsetfillcolor{currentfill}%
\pgfsetfillopacity{0.800000}%
\pgfsetlinewidth{1.003750pt}%
\definecolor{currentstroke}{rgb}{0.800000,0.800000,0.800000}%
\pgfsetstrokecolor{currentstroke}%
\pgfsetstrokeopacity{0.800000}%
\pgfsetdash{}{0pt}%
\pgfpathmoveto{\pgfqpoint{4.300417in}{1.788050in}}%
\pgfpathlineto{\pgfqpoint{5.752778in}{1.788050in}}%
\pgfpathquadraticcurveto{\pgfqpoint{5.780556in}{1.788050in}}{\pgfqpoint{5.780556in}{1.815828in}}%
\pgfpathlineto{\pgfqpoint{5.780556in}{2.384300in}}%
\pgfpathquadraticcurveto{\pgfqpoint{5.780556in}{2.412078in}}{\pgfqpoint{5.752778in}{2.412078in}}%
\pgfpathlineto{\pgfqpoint{4.300417in}{2.412078in}}%
\pgfpathquadraticcurveto{\pgfqpoint{4.272639in}{2.412078in}}{\pgfqpoint{4.272639in}{2.384300in}}%
\pgfpathlineto{\pgfqpoint{4.272639in}{1.815828in}}%
\pgfpathquadraticcurveto{\pgfqpoint{4.272639in}{1.788050in}}{\pgfqpoint{4.300417in}{1.788050in}}%
\pgfpathclose%
\pgfusepath{stroke,fill}%
\end{pgfscope}%
\begin{pgfscope}%
\pgfsetbuttcap%
\pgfsetroundjoin%
\definecolor{currentfill}{rgb}{0.121569,0.466667,0.705882}%
\pgfsetfillcolor{currentfill}%
\pgfsetlinewidth{1.003750pt}%
\definecolor{currentstroke}{rgb}{0.121569,0.466667,0.705882}%
\pgfsetstrokecolor{currentstroke}%
\pgfsetdash{}{0pt}%
\pgfsys@defobject{currentmarker}{\pgfqpoint{-0.034722in}{-0.034722in}}{\pgfqpoint{0.034722in}{0.034722in}}{%
\pgfpathmoveto{\pgfqpoint{0.000000in}{-0.034722in}}%
\pgfpathcurveto{\pgfqpoint{0.009208in}{-0.034722in}}{\pgfqpoint{0.018041in}{-0.031064in}}{\pgfqpoint{0.024552in}{-0.024552in}}%
\pgfpathcurveto{\pgfqpoint{0.031064in}{-0.018041in}}{\pgfqpoint{0.034722in}{-0.009208in}}{\pgfqpoint{0.034722in}{0.000000in}}%
\pgfpathcurveto{\pgfqpoint{0.034722in}{0.009208in}}{\pgfqpoint{0.031064in}{0.018041in}}{\pgfqpoint{0.024552in}{0.024552in}}%
\pgfpathcurveto{\pgfqpoint{0.018041in}{0.031064in}}{\pgfqpoint{0.009208in}{0.034722in}}{\pgfqpoint{0.000000in}{0.034722in}}%
\pgfpathcurveto{\pgfqpoint{-0.009208in}{0.034722in}}{\pgfqpoint{-0.018041in}{0.031064in}}{\pgfqpoint{-0.024552in}{0.024552in}}%
\pgfpathcurveto{\pgfqpoint{-0.031064in}{0.018041in}}{\pgfqpoint{-0.034722in}{0.009208in}}{\pgfqpoint{-0.034722in}{0.000000in}}%
\pgfpathcurveto{\pgfqpoint{-0.034722in}{-0.009208in}}{\pgfqpoint{-0.031064in}{-0.018041in}}{\pgfqpoint{-0.024552in}{-0.024552in}}%
\pgfpathcurveto{\pgfqpoint{-0.018041in}{-0.031064in}}{\pgfqpoint{-0.009208in}{-0.034722in}}{\pgfqpoint{0.000000in}{-0.034722in}}%
\pgfpathclose%
\pgfusepath{stroke,fill}%
}%
\begin{pgfscope}%
\pgfsys@transformshift{4.467083in}{2.307911in}%
\pgfsys@useobject{currentmarker}{}%
\end{pgfscope}%
\end{pgfscope}%
\begin{pgfscope}%
\definecolor{textcolor}{rgb}{0.000000,0.000000,0.000000}%
\pgfsetstrokecolor{textcolor}%
\pgfsetfillcolor{textcolor}%
\pgftext[x=4.717083in,y=2.259300in,left,base]{\color{textcolor}\sffamily\fontsize{10.000000}{12.000000}\selectfont No Timeout}%
\end{pgfscope}%
\begin{pgfscope}%
\pgfsetbuttcap%
\pgfsetroundjoin%
\definecolor{currentfill}{rgb}{1.000000,0.498039,0.054902}%
\pgfsetfillcolor{currentfill}%
\pgfsetlinewidth{1.003750pt}%
\definecolor{currentstroke}{rgb}{1.000000,0.498039,0.054902}%
\pgfsetstrokecolor{currentstroke}%
\pgfsetdash{}{0pt}%
\pgfsys@defobject{currentmarker}{\pgfqpoint{-0.034722in}{-0.034722in}}{\pgfqpoint{0.034722in}{0.034722in}}{%
\pgfpathmoveto{\pgfqpoint{0.000000in}{-0.034722in}}%
\pgfpathcurveto{\pgfqpoint{0.009208in}{-0.034722in}}{\pgfqpoint{0.018041in}{-0.031064in}}{\pgfqpoint{0.024552in}{-0.024552in}}%
\pgfpathcurveto{\pgfqpoint{0.031064in}{-0.018041in}}{\pgfqpoint{0.034722in}{-0.009208in}}{\pgfqpoint{0.034722in}{0.000000in}}%
\pgfpathcurveto{\pgfqpoint{0.034722in}{0.009208in}}{\pgfqpoint{0.031064in}{0.018041in}}{\pgfqpoint{0.024552in}{0.024552in}}%
\pgfpathcurveto{\pgfqpoint{0.018041in}{0.031064in}}{\pgfqpoint{0.009208in}{0.034722in}}{\pgfqpoint{0.000000in}{0.034722in}}%
\pgfpathcurveto{\pgfqpoint{-0.009208in}{0.034722in}}{\pgfqpoint{-0.018041in}{0.031064in}}{\pgfqpoint{-0.024552in}{0.024552in}}%
\pgfpathcurveto{\pgfqpoint{-0.031064in}{0.018041in}}{\pgfqpoint{-0.034722in}{0.009208in}}{\pgfqpoint{-0.034722in}{0.000000in}}%
\pgfpathcurveto{\pgfqpoint{-0.034722in}{-0.009208in}}{\pgfqpoint{-0.031064in}{-0.018041in}}{\pgfqpoint{-0.024552in}{-0.024552in}}%
\pgfpathcurveto{\pgfqpoint{-0.018041in}{-0.031064in}}{\pgfqpoint{-0.009208in}{-0.034722in}}{\pgfqpoint{0.000000in}{-0.034722in}}%
\pgfpathclose%
\pgfusepath{stroke,fill}%
}%
\begin{pgfscope}%
\pgfsys@transformshift{4.467083in}{2.114300in}%
\pgfsys@useobject{currentmarker}{}%
\end{pgfscope}%
\end{pgfscope}%
\begin{pgfscope}%
\definecolor{textcolor}{rgb}{0.000000,0.000000,0.000000}%
\pgfsetstrokecolor{textcolor}%
\pgfsetfillcolor{textcolor}%
\pgftext[x=4.717083in,y=2.065689in,left,base]{\color{textcolor}\sffamily\fontsize{10.000000}{12.000000}\selectfont Time Timeout}%
\end{pgfscope}%
\begin{pgfscope}%
\pgfsetbuttcap%
\pgfsetroundjoin%
\definecolor{currentfill}{rgb}{0.839216,0.152941,0.156863}%
\pgfsetfillcolor{currentfill}%
\pgfsetlinewidth{1.003750pt}%
\definecolor{currentstroke}{rgb}{0.839216,0.152941,0.156863}%
\pgfsetstrokecolor{currentstroke}%
\pgfsetdash{}{0pt}%
\pgfsys@defobject{currentmarker}{\pgfqpoint{-0.034722in}{-0.034722in}}{\pgfqpoint{0.034722in}{0.034722in}}{%
\pgfpathmoveto{\pgfqpoint{0.000000in}{-0.034722in}}%
\pgfpathcurveto{\pgfqpoint{0.009208in}{-0.034722in}}{\pgfqpoint{0.018041in}{-0.031064in}}{\pgfqpoint{0.024552in}{-0.024552in}}%
\pgfpathcurveto{\pgfqpoint{0.031064in}{-0.018041in}}{\pgfqpoint{0.034722in}{-0.009208in}}{\pgfqpoint{0.034722in}{0.000000in}}%
\pgfpathcurveto{\pgfqpoint{0.034722in}{0.009208in}}{\pgfqpoint{0.031064in}{0.018041in}}{\pgfqpoint{0.024552in}{0.024552in}}%
\pgfpathcurveto{\pgfqpoint{0.018041in}{0.031064in}}{\pgfqpoint{0.009208in}{0.034722in}}{\pgfqpoint{0.000000in}{0.034722in}}%
\pgfpathcurveto{\pgfqpoint{-0.009208in}{0.034722in}}{\pgfqpoint{-0.018041in}{0.031064in}}{\pgfqpoint{-0.024552in}{0.024552in}}%
\pgfpathcurveto{\pgfqpoint{-0.031064in}{0.018041in}}{\pgfqpoint{-0.034722in}{0.009208in}}{\pgfqpoint{-0.034722in}{0.000000in}}%
\pgfpathcurveto{\pgfqpoint{-0.034722in}{-0.009208in}}{\pgfqpoint{-0.031064in}{-0.018041in}}{\pgfqpoint{-0.024552in}{-0.024552in}}%
\pgfpathcurveto{\pgfqpoint{-0.018041in}{-0.031064in}}{\pgfqpoint{-0.009208in}{-0.034722in}}{\pgfqpoint{0.000000in}{-0.034722in}}%
\pgfpathclose%
\pgfusepath{stroke,fill}%
}%
\begin{pgfscope}%
\pgfsys@transformshift{4.467083in}{1.920689in}%
\pgfsys@useobject{currentmarker}{}%
\end{pgfscope}%
\end{pgfscope}%
\begin{pgfscope}%
\definecolor{textcolor}{rgb}{0.000000,0.000000,0.000000}%
\pgfsetstrokecolor{textcolor}%
\pgfsetfillcolor{textcolor}%
\pgftext[x=4.717083in,y=1.872078in,left,base]{\color{textcolor}\sffamily\fontsize{10.000000}{12.000000}\selectfont Memory Timeout}%
\end{pgfscope}%
\end{pgfpicture}%
\makeatother%
\endgroup%

                }
            \end{subfigure}
            \caption{Statements}
        \end{subfigure}
        \bigbreak
        \begin{subfigure}[b]{\textwidth}
            \centering
            \begin{subfigure}[]{0.45\textwidth}
                \centering
                \resizebox{\columnwidth}{!}{
                    %% Creator: Matplotlib, PGF backend
%%
%% To include the figure in your LaTeX document, write
%%   \input{<filename>.pgf}
%%
%% Make sure the required packages are loaded in your preamble
%%   \usepackage{pgf}
%%
%% and, on pdftex
%%   \usepackage[utf8]{inputenc}\DeclareUnicodeCharacter{2212}{-}
%%
%% or, on luatex and xetex
%%   \usepackage{unicode-math}
%%
%% Figures using additional raster images can only be included by \input if
%% they are in the same directory as the main LaTeX file. For loading figures
%% from other directories you can use the `import` package
%%   \usepackage{import}
%%
%% and then include the figures with
%%   \import{<path to file>}{<filename>.pgf}
%%
%% Matplotlib used the following preamble
%%   \usepackage{amsmath}
%%   \usepackage{fontspec}
%%
\begingroup%
\makeatletter%
\begin{pgfpicture}%
\pgfpathrectangle{\pgfpointorigin}{\pgfqpoint{6.000000in}{4.000000in}}%
\pgfusepath{use as bounding box, clip}%
\begin{pgfscope}%
\pgfsetbuttcap%
\pgfsetmiterjoin%
\definecolor{currentfill}{rgb}{1.000000,1.000000,1.000000}%
\pgfsetfillcolor{currentfill}%
\pgfsetlinewidth{0.000000pt}%
\definecolor{currentstroke}{rgb}{1.000000,1.000000,1.000000}%
\pgfsetstrokecolor{currentstroke}%
\pgfsetdash{}{0pt}%
\pgfpathmoveto{\pgfqpoint{0.000000in}{0.000000in}}%
\pgfpathlineto{\pgfqpoint{6.000000in}{0.000000in}}%
\pgfpathlineto{\pgfqpoint{6.000000in}{4.000000in}}%
\pgfpathlineto{\pgfqpoint{0.000000in}{4.000000in}}%
\pgfpathclose%
\pgfusepath{fill}%
\end{pgfscope}%
\begin{pgfscope}%
\pgfsetbuttcap%
\pgfsetmiterjoin%
\definecolor{currentfill}{rgb}{1.000000,1.000000,1.000000}%
\pgfsetfillcolor{currentfill}%
\pgfsetlinewidth{0.000000pt}%
\definecolor{currentstroke}{rgb}{0.000000,0.000000,0.000000}%
\pgfsetstrokecolor{currentstroke}%
\pgfsetstrokeopacity{0.000000}%
\pgfsetdash{}{0pt}%
\pgfpathmoveto{\pgfqpoint{0.648703in}{0.548769in}}%
\pgfpathlineto{\pgfqpoint{5.850000in}{0.548769in}}%
\pgfpathlineto{\pgfqpoint{5.850000in}{3.651359in}}%
\pgfpathlineto{\pgfqpoint{0.648703in}{3.651359in}}%
\pgfpathclose%
\pgfusepath{fill}%
\end{pgfscope}%
\begin{pgfscope}%
\pgfpathrectangle{\pgfqpoint{0.648703in}{0.548769in}}{\pgfqpoint{5.201297in}{3.102590in}}%
\pgfusepath{clip}%
\pgfsetbuttcap%
\pgfsetroundjoin%
\definecolor{currentfill}{rgb}{0.121569,0.466667,0.705882}%
\pgfsetfillcolor{currentfill}%
\pgfsetlinewidth{1.003750pt}%
\definecolor{currentstroke}{rgb}{0.121569,0.466667,0.705882}%
\pgfsetstrokecolor{currentstroke}%
\pgfsetdash{}{0pt}%
\pgfpathmoveto{\pgfqpoint{1.333773in}{0.648129in}}%
\pgfpathcurveto{\pgfqpoint{1.344823in}{0.648129in}}{\pgfqpoint{1.355422in}{0.652519in}}{\pgfqpoint{1.363236in}{0.660333in}}%
\pgfpathcurveto{\pgfqpoint{1.371050in}{0.668146in}}{\pgfqpoint{1.375440in}{0.678745in}}{\pgfqpoint{1.375440in}{0.689796in}}%
\pgfpathcurveto{\pgfqpoint{1.375440in}{0.700846in}}{\pgfqpoint{1.371050in}{0.711445in}}{\pgfqpoint{1.363236in}{0.719258in}}%
\pgfpathcurveto{\pgfqpoint{1.355422in}{0.727072in}}{\pgfqpoint{1.344823in}{0.731462in}}{\pgfqpoint{1.333773in}{0.731462in}}%
\pgfpathcurveto{\pgfqpoint{1.322723in}{0.731462in}}{\pgfqpoint{1.312124in}{0.727072in}}{\pgfqpoint{1.304310in}{0.719258in}}%
\pgfpathcurveto{\pgfqpoint{1.296497in}{0.711445in}}{\pgfqpoint{1.292106in}{0.700846in}}{\pgfqpoint{1.292106in}{0.689796in}}%
\pgfpathcurveto{\pgfqpoint{1.292106in}{0.678745in}}{\pgfqpoint{1.296497in}{0.668146in}}{\pgfqpoint{1.304310in}{0.660333in}}%
\pgfpathcurveto{\pgfqpoint{1.312124in}{0.652519in}}{\pgfqpoint{1.322723in}{0.648129in}}{\pgfqpoint{1.333773in}{0.648129in}}%
\pgfpathclose%
\pgfusepath{stroke,fill}%
\end{pgfscope}%
\begin{pgfscope}%
\pgfpathrectangle{\pgfqpoint{0.648703in}{0.548769in}}{\pgfqpoint{5.201297in}{3.102590in}}%
\pgfusepath{clip}%
\pgfsetbuttcap%
\pgfsetroundjoin%
\definecolor{currentfill}{rgb}{0.121569,0.466667,0.705882}%
\pgfsetfillcolor{currentfill}%
\pgfsetlinewidth{1.003750pt}%
\definecolor{currentstroke}{rgb}{0.121569,0.466667,0.705882}%
\pgfsetstrokecolor{currentstroke}%
\pgfsetdash{}{0pt}%
\pgfpathmoveto{\pgfqpoint{4.728025in}{3.124394in}}%
\pgfpathcurveto{\pgfqpoint{4.739075in}{3.124394in}}{\pgfqpoint{4.749674in}{3.128784in}}{\pgfqpoint{4.757488in}{3.136598in}}%
\pgfpathcurveto{\pgfqpoint{4.765301in}{3.144411in}}{\pgfqpoint{4.769692in}{3.155010in}}{\pgfqpoint{4.769692in}{3.166060in}}%
\pgfpathcurveto{\pgfqpoint{4.769692in}{3.177111in}}{\pgfqpoint{4.765301in}{3.187710in}}{\pgfqpoint{4.757488in}{3.195523in}}%
\pgfpathcurveto{\pgfqpoint{4.749674in}{3.203337in}}{\pgfqpoint{4.739075in}{3.207727in}}{\pgfqpoint{4.728025in}{3.207727in}}%
\pgfpathcurveto{\pgfqpoint{4.716975in}{3.207727in}}{\pgfqpoint{4.706376in}{3.203337in}}{\pgfqpoint{4.698562in}{3.195523in}}%
\pgfpathcurveto{\pgfqpoint{4.690749in}{3.187710in}}{\pgfqpoint{4.686358in}{3.177111in}}{\pgfqpoint{4.686358in}{3.166060in}}%
\pgfpathcurveto{\pgfqpoint{4.686358in}{3.155010in}}{\pgfqpoint{4.690749in}{3.144411in}}{\pgfqpoint{4.698562in}{3.136598in}}%
\pgfpathcurveto{\pgfqpoint{4.706376in}{3.128784in}}{\pgfqpoint{4.716975in}{3.124394in}}{\pgfqpoint{4.728025in}{3.124394in}}%
\pgfpathclose%
\pgfusepath{stroke,fill}%
\end{pgfscope}%
\begin{pgfscope}%
\pgfpathrectangle{\pgfqpoint{0.648703in}{0.548769in}}{\pgfqpoint{5.201297in}{3.102590in}}%
\pgfusepath{clip}%
\pgfsetbuttcap%
\pgfsetroundjoin%
\definecolor{currentfill}{rgb}{1.000000,0.498039,0.054902}%
\pgfsetfillcolor{currentfill}%
\pgfsetlinewidth{1.003750pt}%
\definecolor{currentstroke}{rgb}{1.000000,0.498039,0.054902}%
\pgfsetstrokecolor{currentstroke}%
\pgfsetdash{}{0pt}%
\pgfpathmoveto{\pgfqpoint{1.263609in}{3.140985in}}%
\pgfpathcurveto{\pgfqpoint{1.274660in}{3.140985in}}{\pgfqpoint{1.285259in}{3.145375in}}{\pgfqpoint{1.293072in}{3.153189in}}%
\pgfpathcurveto{\pgfqpoint{1.300886in}{3.161003in}}{\pgfqpoint{1.305276in}{3.171602in}}{\pgfqpoint{1.305276in}{3.182652in}}%
\pgfpathcurveto{\pgfqpoint{1.305276in}{3.193702in}}{\pgfqpoint{1.300886in}{3.204301in}}{\pgfqpoint{1.293072in}{3.212115in}}%
\pgfpathcurveto{\pgfqpoint{1.285259in}{3.219928in}}{\pgfqpoint{1.274660in}{3.224319in}}{\pgfqpoint{1.263609in}{3.224319in}}%
\pgfpathcurveto{\pgfqpoint{1.252559in}{3.224319in}}{\pgfqpoint{1.241960in}{3.219928in}}{\pgfqpoint{1.234147in}{3.212115in}}%
\pgfpathcurveto{\pgfqpoint{1.226333in}{3.204301in}}{\pgfqpoint{1.221943in}{3.193702in}}{\pgfqpoint{1.221943in}{3.182652in}}%
\pgfpathcurveto{\pgfqpoint{1.221943in}{3.171602in}}{\pgfqpoint{1.226333in}{3.161003in}}{\pgfqpoint{1.234147in}{3.153189in}}%
\pgfpathcurveto{\pgfqpoint{1.241960in}{3.145375in}}{\pgfqpoint{1.252559in}{3.140985in}}{\pgfqpoint{1.263609in}{3.140985in}}%
\pgfpathclose%
\pgfusepath{stroke,fill}%
\end{pgfscope}%
\begin{pgfscope}%
\pgfpathrectangle{\pgfqpoint{0.648703in}{0.548769in}}{\pgfqpoint{5.201297in}{3.102590in}}%
\pgfusepath{clip}%
\pgfsetbuttcap%
\pgfsetroundjoin%
\definecolor{currentfill}{rgb}{0.121569,0.466667,0.705882}%
\pgfsetfillcolor{currentfill}%
\pgfsetlinewidth{1.003750pt}%
\definecolor{currentstroke}{rgb}{0.121569,0.466667,0.705882}%
\pgfsetstrokecolor{currentstroke}%
\pgfsetdash{}{0pt}%
\pgfpathmoveto{\pgfqpoint{2.952495in}{3.132690in}}%
\pgfpathcurveto{\pgfqpoint{2.963545in}{3.132690in}}{\pgfqpoint{2.974144in}{3.137080in}}{\pgfqpoint{2.981958in}{3.144893in}}%
\pgfpathcurveto{\pgfqpoint{2.989772in}{3.152707in}}{\pgfqpoint{2.994162in}{3.163306in}}{\pgfqpoint{2.994162in}{3.174356in}}%
\pgfpathcurveto{\pgfqpoint{2.994162in}{3.185406in}}{\pgfqpoint{2.989772in}{3.196005in}}{\pgfqpoint{2.981958in}{3.203819in}}%
\pgfpathcurveto{\pgfqpoint{2.974144in}{3.211633in}}{\pgfqpoint{2.963545in}{3.216023in}}{\pgfqpoint{2.952495in}{3.216023in}}%
\pgfpathcurveto{\pgfqpoint{2.941445in}{3.216023in}}{\pgfqpoint{2.930846in}{3.211633in}}{\pgfqpoint{2.923032in}{3.203819in}}%
\pgfpathcurveto{\pgfqpoint{2.915219in}{3.196005in}}{\pgfqpoint{2.910828in}{3.185406in}}{\pgfqpoint{2.910828in}{3.174356in}}%
\pgfpathcurveto{\pgfqpoint{2.910828in}{3.163306in}}{\pgfqpoint{2.915219in}{3.152707in}}{\pgfqpoint{2.923032in}{3.144893in}}%
\pgfpathcurveto{\pgfqpoint{2.930846in}{3.137080in}}{\pgfqpoint{2.941445in}{3.132690in}}{\pgfqpoint{2.952495in}{3.132690in}}%
\pgfpathclose%
\pgfusepath{stroke,fill}%
\end{pgfscope}%
\begin{pgfscope}%
\pgfpathrectangle{\pgfqpoint{0.648703in}{0.548769in}}{\pgfqpoint{5.201297in}{3.102590in}}%
\pgfusepath{clip}%
\pgfsetbuttcap%
\pgfsetroundjoin%
\definecolor{currentfill}{rgb}{1.000000,0.498039,0.054902}%
\pgfsetfillcolor{currentfill}%
\pgfsetlinewidth{1.003750pt}%
\definecolor{currentstroke}{rgb}{1.000000,0.498039,0.054902}%
\pgfsetstrokecolor{currentstroke}%
\pgfsetdash{}{0pt}%
\pgfpathmoveto{\pgfqpoint{2.854240in}{3.136837in}}%
\pgfpathcurveto{\pgfqpoint{2.865290in}{3.136837in}}{\pgfqpoint{2.875889in}{3.141228in}}{\pgfqpoint{2.883703in}{3.149041in}}%
\pgfpathcurveto{\pgfqpoint{2.891516in}{3.156855in}}{\pgfqpoint{2.895906in}{3.167454in}}{\pgfqpoint{2.895906in}{3.178504in}}%
\pgfpathcurveto{\pgfqpoint{2.895906in}{3.189554in}}{\pgfqpoint{2.891516in}{3.200153in}}{\pgfqpoint{2.883703in}{3.207967in}}%
\pgfpathcurveto{\pgfqpoint{2.875889in}{3.215780in}}{\pgfqpoint{2.865290in}{3.220171in}}{\pgfqpoint{2.854240in}{3.220171in}}%
\pgfpathcurveto{\pgfqpoint{2.843190in}{3.220171in}}{\pgfqpoint{2.832591in}{3.215780in}}{\pgfqpoint{2.824777in}{3.207967in}}%
\pgfpathcurveto{\pgfqpoint{2.816963in}{3.200153in}}{\pgfqpoint{2.812573in}{3.189554in}}{\pgfqpoint{2.812573in}{3.178504in}}%
\pgfpathcurveto{\pgfqpoint{2.812573in}{3.167454in}}{\pgfqpoint{2.816963in}{3.156855in}}{\pgfqpoint{2.824777in}{3.149041in}}%
\pgfpathcurveto{\pgfqpoint{2.832591in}{3.141228in}}{\pgfqpoint{2.843190in}{3.136837in}}{\pgfqpoint{2.854240in}{3.136837in}}%
\pgfpathclose%
\pgfusepath{stroke,fill}%
\end{pgfscope}%
\begin{pgfscope}%
\pgfpathrectangle{\pgfqpoint{0.648703in}{0.548769in}}{\pgfqpoint{5.201297in}{3.102590in}}%
\pgfusepath{clip}%
\pgfsetbuttcap%
\pgfsetroundjoin%
\definecolor{currentfill}{rgb}{0.121569,0.466667,0.705882}%
\pgfsetfillcolor{currentfill}%
\pgfsetlinewidth{1.003750pt}%
\definecolor{currentstroke}{rgb}{0.121569,0.466667,0.705882}%
\pgfsetstrokecolor{currentstroke}%
\pgfsetdash{}{0pt}%
\pgfpathmoveto{\pgfqpoint{2.559526in}{3.132690in}}%
\pgfpathcurveto{\pgfqpoint{2.570576in}{3.132690in}}{\pgfqpoint{2.581175in}{3.137080in}}{\pgfqpoint{2.588989in}{3.144893in}}%
\pgfpathcurveto{\pgfqpoint{2.596802in}{3.152707in}}{\pgfqpoint{2.601193in}{3.163306in}}{\pgfqpoint{2.601193in}{3.174356in}}%
\pgfpathcurveto{\pgfqpoint{2.601193in}{3.185406in}}{\pgfqpoint{2.596802in}{3.196005in}}{\pgfqpoint{2.588989in}{3.203819in}}%
\pgfpathcurveto{\pgfqpoint{2.581175in}{3.211633in}}{\pgfqpoint{2.570576in}{3.216023in}}{\pgfqpoint{2.559526in}{3.216023in}}%
\pgfpathcurveto{\pgfqpoint{2.548476in}{3.216023in}}{\pgfqpoint{2.537877in}{3.211633in}}{\pgfqpoint{2.530063in}{3.203819in}}%
\pgfpathcurveto{\pgfqpoint{2.522250in}{3.196005in}}{\pgfqpoint{2.517859in}{3.185406in}}{\pgfqpoint{2.517859in}{3.174356in}}%
\pgfpathcurveto{\pgfqpoint{2.517859in}{3.163306in}}{\pgfqpoint{2.522250in}{3.152707in}}{\pgfqpoint{2.530063in}{3.144893in}}%
\pgfpathcurveto{\pgfqpoint{2.537877in}{3.137080in}}{\pgfqpoint{2.548476in}{3.132690in}}{\pgfqpoint{2.559526in}{3.132690in}}%
\pgfpathclose%
\pgfusepath{stroke,fill}%
\end{pgfscope}%
\begin{pgfscope}%
\pgfpathrectangle{\pgfqpoint{0.648703in}{0.548769in}}{\pgfqpoint{5.201297in}{3.102590in}}%
\pgfusepath{clip}%
\pgfsetbuttcap%
\pgfsetroundjoin%
\definecolor{currentfill}{rgb}{0.121569,0.466667,0.705882}%
\pgfsetfillcolor{currentfill}%
\pgfsetlinewidth{1.003750pt}%
\definecolor{currentstroke}{rgb}{0.121569,0.466667,0.705882}%
\pgfsetstrokecolor{currentstroke}%
\pgfsetdash{}{0pt}%
\pgfpathmoveto{\pgfqpoint{2.471488in}{3.128542in}}%
\pgfpathcurveto{\pgfqpoint{2.482538in}{3.128542in}}{\pgfqpoint{2.493137in}{3.132932in}}{\pgfqpoint{2.500951in}{3.140746in}}%
\pgfpathcurveto{\pgfqpoint{2.508765in}{3.148559in}}{\pgfqpoint{2.513155in}{3.159158in}}{\pgfqpoint{2.513155in}{3.170208in}}%
\pgfpathcurveto{\pgfqpoint{2.513155in}{3.181258in}}{\pgfqpoint{2.508765in}{3.191857in}}{\pgfqpoint{2.500951in}{3.199671in}}%
\pgfpathcurveto{\pgfqpoint{2.493137in}{3.207485in}}{\pgfqpoint{2.482538in}{3.211875in}}{\pgfqpoint{2.471488in}{3.211875in}}%
\pgfpathcurveto{\pgfqpoint{2.460438in}{3.211875in}}{\pgfqpoint{2.449839in}{3.207485in}}{\pgfqpoint{2.442025in}{3.199671in}}%
\pgfpathcurveto{\pgfqpoint{2.434212in}{3.191857in}}{\pgfqpoint{2.429822in}{3.181258in}}{\pgfqpoint{2.429822in}{3.170208in}}%
\pgfpathcurveto{\pgfqpoint{2.429822in}{3.159158in}}{\pgfqpoint{2.434212in}{3.148559in}}{\pgfqpoint{2.442025in}{3.140746in}}%
\pgfpathcurveto{\pgfqpoint{2.449839in}{3.132932in}}{\pgfqpoint{2.460438in}{3.128542in}}{\pgfqpoint{2.471488in}{3.128542in}}%
\pgfpathclose%
\pgfusepath{stroke,fill}%
\end{pgfscope}%
\begin{pgfscope}%
\pgfpathrectangle{\pgfqpoint{0.648703in}{0.548769in}}{\pgfqpoint{5.201297in}{3.102590in}}%
\pgfusepath{clip}%
\pgfsetbuttcap%
\pgfsetroundjoin%
\definecolor{currentfill}{rgb}{1.000000,0.498039,0.054902}%
\pgfsetfillcolor{currentfill}%
\pgfsetlinewidth{1.003750pt}%
\definecolor{currentstroke}{rgb}{1.000000,0.498039,0.054902}%
\pgfsetstrokecolor{currentstroke}%
\pgfsetdash{}{0pt}%
\pgfpathmoveto{\pgfqpoint{1.456784in}{3.149281in}}%
\pgfpathcurveto{\pgfqpoint{1.467834in}{3.149281in}}{\pgfqpoint{1.478433in}{3.153671in}}{\pgfqpoint{1.486247in}{3.161485in}}%
\pgfpathcurveto{\pgfqpoint{1.494060in}{3.169298in}}{\pgfqpoint{1.498451in}{3.179897in}}{\pgfqpoint{1.498451in}{3.190948in}}%
\pgfpathcurveto{\pgfqpoint{1.498451in}{3.201998in}}{\pgfqpoint{1.494060in}{3.212597in}}{\pgfqpoint{1.486247in}{3.220410in}}%
\pgfpathcurveto{\pgfqpoint{1.478433in}{3.228224in}}{\pgfqpoint{1.467834in}{3.232614in}}{\pgfqpoint{1.456784in}{3.232614in}}%
\pgfpathcurveto{\pgfqpoint{1.445734in}{3.232614in}}{\pgfqpoint{1.435135in}{3.228224in}}{\pgfqpoint{1.427321in}{3.220410in}}%
\pgfpathcurveto{\pgfqpoint{1.419508in}{3.212597in}}{\pgfqpoint{1.415117in}{3.201998in}}{\pgfqpoint{1.415117in}{3.190948in}}%
\pgfpathcurveto{\pgfqpoint{1.415117in}{3.179897in}}{\pgfqpoint{1.419508in}{3.169298in}}{\pgfqpoint{1.427321in}{3.161485in}}%
\pgfpathcurveto{\pgfqpoint{1.435135in}{3.153671in}}{\pgfqpoint{1.445734in}{3.149281in}}{\pgfqpoint{1.456784in}{3.149281in}}%
\pgfpathclose%
\pgfusepath{stroke,fill}%
\end{pgfscope}%
\begin{pgfscope}%
\pgfpathrectangle{\pgfqpoint{0.648703in}{0.548769in}}{\pgfqpoint{5.201297in}{3.102590in}}%
\pgfusepath{clip}%
\pgfsetbuttcap%
\pgfsetroundjoin%
\definecolor{currentfill}{rgb}{1.000000,0.498039,0.054902}%
\pgfsetfillcolor{currentfill}%
\pgfsetlinewidth{1.003750pt}%
\definecolor{currentstroke}{rgb}{1.000000,0.498039,0.054902}%
\pgfsetstrokecolor{currentstroke}%
\pgfsetdash{}{0pt}%
\pgfpathmoveto{\pgfqpoint{2.243419in}{3.240534in}}%
\pgfpathcurveto{\pgfqpoint{2.254469in}{3.240534in}}{\pgfqpoint{2.265068in}{3.244924in}}{\pgfqpoint{2.272882in}{3.252737in}}%
\pgfpathcurveto{\pgfqpoint{2.280695in}{3.260551in}}{\pgfqpoint{2.285086in}{3.271150in}}{\pgfqpoint{2.285086in}{3.282200in}}%
\pgfpathcurveto{\pgfqpoint{2.285086in}{3.293250in}}{\pgfqpoint{2.280695in}{3.303849in}}{\pgfqpoint{2.272882in}{3.311663in}}%
\pgfpathcurveto{\pgfqpoint{2.265068in}{3.319477in}}{\pgfqpoint{2.254469in}{3.323867in}}{\pgfqpoint{2.243419in}{3.323867in}}%
\pgfpathcurveto{\pgfqpoint{2.232369in}{3.323867in}}{\pgfqpoint{2.221770in}{3.319477in}}{\pgfqpoint{2.213956in}{3.311663in}}%
\pgfpathcurveto{\pgfqpoint{2.206143in}{3.303849in}}{\pgfqpoint{2.201752in}{3.293250in}}{\pgfqpoint{2.201752in}{3.282200in}}%
\pgfpathcurveto{\pgfqpoint{2.201752in}{3.271150in}}{\pgfqpoint{2.206143in}{3.260551in}}{\pgfqpoint{2.213956in}{3.252737in}}%
\pgfpathcurveto{\pgfqpoint{2.221770in}{3.244924in}}{\pgfqpoint{2.232369in}{3.240534in}}{\pgfqpoint{2.243419in}{3.240534in}}%
\pgfpathclose%
\pgfusepath{stroke,fill}%
\end{pgfscope}%
\begin{pgfscope}%
\pgfpathrectangle{\pgfqpoint{0.648703in}{0.548769in}}{\pgfqpoint{5.201297in}{3.102590in}}%
\pgfusepath{clip}%
\pgfsetbuttcap%
\pgfsetroundjoin%
\definecolor{currentfill}{rgb}{0.121569,0.466667,0.705882}%
\pgfsetfillcolor{currentfill}%
\pgfsetlinewidth{1.003750pt}%
\definecolor{currentstroke}{rgb}{0.121569,0.466667,0.705882}%
\pgfsetstrokecolor{currentstroke}%
\pgfsetdash{}{0pt}%
\pgfpathmoveto{\pgfqpoint{0.885178in}{0.664720in}}%
\pgfpathcurveto{\pgfqpoint{0.896228in}{0.664720in}}{\pgfqpoint{0.906827in}{0.669111in}}{\pgfqpoint{0.914641in}{0.676924in}}%
\pgfpathcurveto{\pgfqpoint{0.922455in}{0.684738in}}{\pgfqpoint{0.926845in}{0.695337in}}{\pgfqpoint{0.926845in}{0.706387in}}%
\pgfpathcurveto{\pgfqpoint{0.926845in}{0.717437in}}{\pgfqpoint{0.922455in}{0.728036in}}{\pgfqpoint{0.914641in}{0.735850in}}%
\pgfpathcurveto{\pgfqpoint{0.906827in}{0.743663in}}{\pgfqpoint{0.896228in}{0.748054in}}{\pgfqpoint{0.885178in}{0.748054in}}%
\pgfpathcurveto{\pgfqpoint{0.874128in}{0.748054in}}{\pgfqpoint{0.863529in}{0.743663in}}{\pgfqpoint{0.855715in}{0.735850in}}%
\pgfpathcurveto{\pgfqpoint{0.847902in}{0.728036in}}{\pgfqpoint{0.843512in}{0.717437in}}{\pgfqpoint{0.843512in}{0.706387in}}%
\pgfpathcurveto{\pgfqpoint{0.843512in}{0.695337in}}{\pgfqpoint{0.847902in}{0.684738in}}{\pgfqpoint{0.855715in}{0.676924in}}%
\pgfpathcurveto{\pgfqpoint{0.863529in}{0.669111in}}{\pgfqpoint{0.874128in}{0.664720in}}{\pgfqpoint{0.885178in}{0.664720in}}%
\pgfpathclose%
\pgfusepath{stroke,fill}%
\end{pgfscope}%
\begin{pgfscope}%
\pgfpathrectangle{\pgfqpoint{0.648703in}{0.548769in}}{\pgfqpoint{5.201297in}{3.102590in}}%
\pgfusepath{clip}%
\pgfsetbuttcap%
\pgfsetroundjoin%
\definecolor{currentfill}{rgb}{1.000000,0.498039,0.054902}%
\pgfsetfillcolor{currentfill}%
\pgfsetlinewidth{1.003750pt}%
\definecolor{currentstroke}{rgb}{1.000000,0.498039,0.054902}%
\pgfsetstrokecolor{currentstroke}%
\pgfsetdash{}{0pt}%
\pgfpathmoveto{\pgfqpoint{2.085061in}{3.157577in}}%
\pgfpathcurveto{\pgfqpoint{2.096111in}{3.157577in}}{\pgfqpoint{2.106710in}{3.161967in}}{\pgfqpoint{2.114523in}{3.169780in}}%
\pgfpathcurveto{\pgfqpoint{2.122337in}{3.177594in}}{\pgfqpoint{2.126727in}{3.188193in}}{\pgfqpoint{2.126727in}{3.199243in}}%
\pgfpathcurveto{\pgfqpoint{2.126727in}{3.210293in}}{\pgfqpoint{2.122337in}{3.220892in}}{\pgfqpoint{2.114523in}{3.228706in}}%
\pgfpathcurveto{\pgfqpoint{2.106710in}{3.236520in}}{\pgfqpoint{2.096111in}{3.240910in}}{\pgfqpoint{2.085061in}{3.240910in}}%
\pgfpathcurveto{\pgfqpoint{2.074011in}{3.240910in}}{\pgfqpoint{2.063412in}{3.236520in}}{\pgfqpoint{2.055598in}{3.228706in}}%
\pgfpathcurveto{\pgfqpoint{2.047784in}{3.220892in}}{\pgfqpoint{2.043394in}{3.210293in}}{\pgfqpoint{2.043394in}{3.199243in}}%
\pgfpathcurveto{\pgfqpoint{2.043394in}{3.188193in}}{\pgfqpoint{2.047784in}{3.177594in}}{\pgfqpoint{2.055598in}{3.169780in}}%
\pgfpathcurveto{\pgfqpoint{2.063412in}{3.161967in}}{\pgfqpoint{2.074011in}{3.157577in}}{\pgfqpoint{2.085061in}{3.157577in}}%
\pgfpathclose%
\pgfusepath{stroke,fill}%
\end{pgfscope}%
\begin{pgfscope}%
\pgfpathrectangle{\pgfqpoint{0.648703in}{0.548769in}}{\pgfqpoint{5.201297in}{3.102590in}}%
\pgfusepath{clip}%
\pgfsetbuttcap%
\pgfsetroundjoin%
\definecolor{currentfill}{rgb}{1.000000,0.498039,0.054902}%
\pgfsetfillcolor{currentfill}%
\pgfsetlinewidth{1.003750pt}%
\definecolor{currentstroke}{rgb}{1.000000,0.498039,0.054902}%
\pgfsetstrokecolor{currentstroke}%
\pgfsetdash{}{0pt}%
\pgfpathmoveto{\pgfqpoint{1.502950in}{3.140985in}}%
\pgfpathcurveto{\pgfqpoint{1.514000in}{3.140985in}}{\pgfqpoint{1.524599in}{3.145375in}}{\pgfqpoint{1.532413in}{3.153189in}}%
\pgfpathcurveto{\pgfqpoint{1.540226in}{3.161003in}}{\pgfqpoint{1.544617in}{3.171602in}}{\pgfqpoint{1.544617in}{3.182652in}}%
\pgfpathcurveto{\pgfqpoint{1.544617in}{3.193702in}}{\pgfqpoint{1.540226in}{3.204301in}}{\pgfqpoint{1.532413in}{3.212115in}}%
\pgfpathcurveto{\pgfqpoint{1.524599in}{3.219928in}}{\pgfqpoint{1.514000in}{3.224319in}}{\pgfqpoint{1.502950in}{3.224319in}}%
\pgfpathcurveto{\pgfqpoint{1.491900in}{3.224319in}}{\pgfqpoint{1.481301in}{3.219928in}}{\pgfqpoint{1.473487in}{3.212115in}}%
\pgfpathcurveto{\pgfqpoint{1.465674in}{3.204301in}}{\pgfqpoint{1.461283in}{3.193702in}}{\pgfqpoint{1.461283in}{3.182652in}}%
\pgfpathcurveto{\pgfqpoint{1.461283in}{3.171602in}}{\pgfqpoint{1.465674in}{3.161003in}}{\pgfqpoint{1.473487in}{3.153189in}}%
\pgfpathcurveto{\pgfqpoint{1.481301in}{3.145375in}}{\pgfqpoint{1.491900in}{3.140985in}}{\pgfqpoint{1.502950in}{3.140985in}}%
\pgfpathclose%
\pgfusepath{stroke,fill}%
\end{pgfscope}%
\begin{pgfscope}%
\pgfpathrectangle{\pgfqpoint{0.648703in}{0.548769in}}{\pgfqpoint{5.201297in}{3.102590in}}%
\pgfusepath{clip}%
\pgfsetbuttcap%
\pgfsetroundjoin%
\definecolor{currentfill}{rgb}{1.000000,0.498039,0.054902}%
\pgfsetfillcolor{currentfill}%
\pgfsetlinewidth{1.003750pt}%
\definecolor{currentstroke}{rgb}{1.000000,0.498039,0.054902}%
\pgfsetstrokecolor{currentstroke}%
\pgfsetdash{}{0pt}%
\pgfpathmoveto{\pgfqpoint{1.817794in}{3.136837in}}%
\pgfpathcurveto{\pgfqpoint{1.828844in}{3.136837in}}{\pgfqpoint{1.839443in}{3.141228in}}{\pgfqpoint{1.847257in}{3.149041in}}%
\pgfpathcurveto{\pgfqpoint{1.855070in}{3.156855in}}{\pgfqpoint{1.859461in}{3.167454in}}{\pgfqpoint{1.859461in}{3.178504in}}%
\pgfpathcurveto{\pgfqpoint{1.859461in}{3.189554in}}{\pgfqpoint{1.855070in}{3.200153in}}{\pgfqpoint{1.847257in}{3.207967in}}%
\pgfpathcurveto{\pgfqpoint{1.839443in}{3.215780in}}{\pgfqpoint{1.828844in}{3.220171in}}{\pgfqpoint{1.817794in}{3.220171in}}%
\pgfpathcurveto{\pgfqpoint{1.806744in}{3.220171in}}{\pgfqpoint{1.796145in}{3.215780in}}{\pgfqpoint{1.788331in}{3.207967in}}%
\pgfpathcurveto{\pgfqpoint{1.780518in}{3.200153in}}{\pgfqpoint{1.776127in}{3.189554in}}{\pgfqpoint{1.776127in}{3.178504in}}%
\pgfpathcurveto{\pgfqpoint{1.776127in}{3.167454in}}{\pgfqpoint{1.780518in}{3.156855in}}{\pgfqpoint{1.788331in}{3.149041in}}%
\pgfpathcurveto{\pgfqpoint{1.796145in}{3.141228in}}{\pgfqpoint{1.806744in}{3.136837in}}{\pgfqpoint{1.817794in}{3.136837in}}%
\pgfpathclose%
\pgfusepath{stroke,fill}%
\end{pgfscope}%
\begin{pgfscope}%
\pgfpathrectangle{\pgfqpoint{0.648703in}{0.548769in}}{\pgfqpoint{5.201297in}{3.102590in}}%
\pgfusepath{clip}%
\pgfsetbuttcap%
\pgfsetroundjoin%
\definecolor{currentfill}{rgb}{1.000000,0.498039,0.054902}%
\pgfsetfillcolor{currentfill}%
\pgfsetlinewidth{1.003750pt}%
\definecolor{currentstroke}{rgb}{1.000000,0.498039,0.054902}%
\pgfsetstrokecolor{currentstroke}%
\pgfsetdash{}{0pt}%
\pgfpathmoveto{\pgfqpoint{1.376490in}{3.136837in}}%
\pgfpathcurveto{\pgfqpoint{1.387540in}{3.136837in}}{\pgfqpoint{1.398139in}{3.141228in}}{\pgfqpoint{1.405953in}{3.149041in}}%
\pgfpathcurveto{\pgfqpoint{1.413766in}{3.156855in}}{\pgfqpoint{1.418156in}{3.167454in}}{\pgfqpoint{1.418156in}{3.178504in}}%
\pgfpathcurveto{\pgfqpoint{1.418156in}{3.189554in}}{\pgfqpoint{1.413766in}{3.200153in}}{\pgfqpoint{1.405953in}{3.207967in}}%
\pgfpathcurveto{\pgfqpoint{1.398139in}{3.215780in}}{\pgfqpoint{1.387540in}{3.220171in}}{\pgfqpoint{1.376490in}{3.220171in}}%
\pgfpathcurveto{\pgfqpoint{1.365440in}{3.220171in}}{\pgfqpoint{1.354841in}{3.215780in}}{\pgfqpoint{1.347027in}{3.207967in}}%
\pgfpathcurveto{\pgfqpoint{1.339213in}{3.200153in}}{\pgfqpoint{1.334823in}{3.189554in}}{\pgfqpoint{1.334823in}{3.178504in}}%
\pgfpathcurveto{\pgfqpoint{1.334823in}{3.167454in}}{\pgfqpoint{1.339213in}{3.156855in}}{\pgfqpoint{1.347027in}{3.149041in}}%
\pgfpathcurveto{\pgfqpoint{1.354841in}{3.141228in}}{\pgfqpoint{1.365440in}{3.136837in}}{\pgfqpoint{1.376490in}{3.136837in}}%
\pgfpathclose%
\pgfusepath{stroke,fill}%
\end{pgfscope}%
\begin{pgfscope}%
\pgfpathrectangle{\pgfqpoint{0.648703in}{0.548769in}}{\pgfqpoint{5.201297in}{3.102590in}}%
\pgfusepath{clip}%
\pgfsetbuttcap%
\pgfsetroundjoin%
\definecolor{currentfill}{rgb}{1.000000,0.498039,0.054902}%
\pgfsetfillcolor{currentfill}%
\pgfsetlinewidth{1.003750pt}%
\definecolor{currentstroke}{rgb}{1.000000,0.498039,0.054902}%
\pgfsetstrokecolor{currentstroke}%
\pgfsetdash{}{0pt}%
\pgfpathmoveto{\pgfqpoint{1.473752in}{3.136837in}}%
\pgfpathcurveto{\pgfqpoint{1.484802in}{3.136837in}}{\pgfqpoint{1.495401in}{3.141228in}}{\pgfqpoint{1.503215in}{3.149041in}}%
\pgfpathcurveto{\pgfqpoint{1.511029in}{3.156855in}}{\pgfqpoint{1.515419in}{3.167454in}}{\pgfqpoint{1.515419in}{3.178504in}}%
\pgfpathcurveto{\pgfqpoint{1.515419in}{3.189554in}}{\pgfqpoint{1.511029in}{3.200153in}}{\pgfqpoint{1.503215in}{3.207967in}}%
\pgfpathcurveto{\pgfqpoint{1.495401in}{3.215780in}}{\pgfqpoint{1.484802in}{3.220171in}}{\pgfqpoint{1.473752in}{3.220171in}}%
\pgfpathcurveto{\pgfqpoint{1.462702in}{3.220171in}}{\pgfqpoint{1.452103in}{3.215780in}}{\pgfqpoint{1.444289in}{3.207967in}}%
\pgfpathcurveto{\pgfqpoint{1.436476in}{3.200153in}}{\pgfqpoint{1.432085in}{3.189554in}}{\pgfqpoint{1.432085in}{3.178504in}}%
\pgfpathcurveto{\pgfqpoint{1.432085in}{3.167454in}}{\pgfqpoint{1.436476in}{3.156855in}}{\pgfqpoint{1.444289in}{3.149041in}}%
\pgfpathcurveto{\pgfqpoint{1.452103in}{3.141228in}}{\pgfqpoint{1.462702in}{3.136837in}}{\pgfqpoint{1.473752in}{3.136837in}}%
\pgfpathclose%
\pgfusepath{stroke,fill}%
\end{pgfscope}%
\begin{pgfscope}%
\pgfpathrectangle{\pgfqpoint{0.648703in}{0.548769in}}{\pgfqpoint{5.201297in}{3.102590in}}%
\pgfusepath{clip}%
\pgfsetbuttcap%
\pgfsetroundjoin%
\definecolor{currentfill}{rgb}{1.000000,0.498039,0.054902}%
\pgfsetfillcolor{currentfill}%
\pgfsetlinewidth{1.003750pt}%
\definecolor{currentstroke}{rgb}{1.000000,0.498039,0.054902}%
\pgfsetstrokecolor{currentstroke}%
\pgfsetdash{}{0pt}%
\pgfpathmoveto{\pgfqpoint{1.784319in}{3.136837in}}%
\pgfpathcurveto{\pgfqpoint{1.795369in}{3.136837in}}{\pgfqpoint{1.805968in}{3.141228in}}{\pgfqpoint{1.813782in}{3.149041in}}%
\pgfpathcurveto{\pgfqpoint{1.821596in}{3.156855in}}{\pgfqpoint{1.825986in}{3.167454in}}{\pgfqpoint{1.825986in}{3.178504in}}%
\pgfpathcurveto{\pgfqpoint{1.825986in}{3.189554in}}{\pgfqpoint{1.821596in}{3.200153in}}{\pgfqpoint{1.813782in}{3.207967in}}%
\pgfpathcurveto{\pgfqpoint{1.805968in}{3.215780in}}{\pgfqpoint{1.795369in}{3.220171in}}{\pgfqpoint{1.784319in}{3.220171in}}%
\pgfpathcurveto{\pgfqpoint{1.773269in}{3.220171in}}{\pgfqpoint{1.762670in}{3.215780in}}{\pgfqpoint{1.754856in}{3.207967in}}%
\pgfpathcurveto{\pgfqpoint{1.747043in}{3.200153in}}{\pgfqpoint{1.742653in}{3.189554in}}{\pgfqpoint{1.742653in}{3.178504in}}%
\pgfpathcurveto{\pgfqpoint{1.742653in}{3.167454in}}{\pgfqpoint{1.747043in}{3.156855in}}{\pgfqpoint{1.754856in}{3.149041in}}%
\pgfpathcurveto{\pgfqpoint{1.762670in}{3.141228in}}{\pgfqpoint{1.773269in}{3.136837in}}{\pgfqpoint{1.784319in}{3.136837in}}%
\pgfpathclose%
\pgfusepath{stroke,fill}%
\end{pgfscope}%
\begin{pgfscope}%
\pgfpathrectangle{\pgfqpoint{0.648703in}{0.548769in}}{\pgfqpoint{5.201297in}{3.102590in}}%
\pgfusepath{clip}%
\pgfsetbuttcap%
\pgfsetroundjoin%
\definecolor{currentfill}{rgb}{1.000000,0.498039,0.054902}%
\pgfsetfillcolor{currentfill}%
\pgfsetlinewidth{1.003750pt}%
\definecolor{currentstroke}{rgb}{1.000000,0.498039,0.054902}%
\pgfsetstrokecolor{currentstroke}%
\pgfsetdash{}{0pt}%
\pgfpathmoveto{\pgfqpoint{1.580039in}{3.136837in}}%
\pgfpathcurveto{\pgfqpoint{1.591089in}{3.136837in}}{\pgfqpoint{1.601688in}{3.141228in}}{\pgfqpoint{1.609501in}{3.149041in}}%
\pgfpathcurveto{\pgfqpoint{1.617315in}{3.156855in}}{\pgfqpoint{1.621705in}{3.167454in}}{\pgfqpoint{1.621705in}{3.178504in}}%
\pgfpathcurveto{\pgfqpoint{1.621705in}{3.189554in}}{\pgfqpoint{1.617315in}{3.200153in}}{\pgfqpoint{1.609501in}{3.207967in}}%
\pgfpathcurveto{\pgfqpoint{1.601688in}{3.215780in}}{\pgfqpoint{1.591089in}{3.220171in}}{\pgfqpoint{1.580039in}{3.220171in}}%
\pgfpathcurveto{\pgfqpoint{1.568989in}{3.220171in}}{\pgfqpoint{1.558390in}{3.215780in}}{\pgfqpoint{1.550576in}{3.207967in}}%
\pgfpathcurveto{\pgfqpoint{1.542762in}{3.200153in}}{\pgfqpoint{1.538372in}{3.189554in}}{\pgfqpoint{1.538372in}{3.178504in}}%
\pgfpathcurveto{\pgfqpoint{1.538372in}{3.167454in}}{\pgfqpoint{1.542762in}{3.156855in}}{\pgfqpoint{1.550576in}{3.149041in}}%
\pgfpathcurveto{\pgfqpoint{1.558390in}{3.141228in}}{\pgfqpoint{1.568989in}{3.136837in}}{\pgfqpoint{1.580039in}{3.136837in}}%
\pgfpathclose%
\pgfusepath{stroke,fill}%
\end{pgfscope}%
\begin{pgfscope}%
\pgfpathrectangle{\pgfqpoint{0.648703in}{0.548769in}}{\pgfqpoint{5.201297in}{3.102590in}}%
\pgfusepath{clip}%
\pgfsetbuttcap%
\pgfsetroundjoin%
\definecolor{currentfill}{rgb}{0.121569,0.466667,0.705882}%
\pgfsetfillcolor{currentfill}%
\pgfsetlinewidth{1.003750pt}%
\definecolor{currentstroke}{rgb}{0.121569,0.466667,0.705882}%
\pgfsetstrokecolor{currentstroke}%
\pgfsetdash{}{0pt}%
\pgfpathmoveto{\pgfqpoint{1.765748in}{2.846488in}}%
\pgfpathcurveto{\pgfqpoint{1.776798in}{2.846488in}}{\pgfqpoint{1.787397in}{2.850878in}}{\pgfqpoint{1.795211in}{2.858692in}}%
\pgfpathcurveto{\pgfqpoint{1.803025in}{2.866506in}}{\pgfqpoint{1.807415in}{2.877105in}}{\pgfqpoint{1.807415in}{2.888155in}}%
\pgfpathcurveto{\pgfqpoint{1.807415in}{2.899205in}}{\pgfqpoint{1.803025in}{2.909804in}}{\pgfqpoint{1.795211in}{2.917617in}}%
\pgfpathcurveto{\pgfqpoint{1.787397in}{2.925431in}}{\pgfqpoint{1.776798in}{2.929821in}}{\pgfqpoint{1.765748in}{2.929821in}}%
\pgfpathcurveto{\pgfqpoint{1.754698in}{2.929821in}}{\pgfqpoint{1.744099in}{2.925431in}}{\pgfqpoint{1.736285in}{2.917617in}}%
\pgfpathcurveto{\pgfqpoint{1.728472in}{2.909804in}}{\pgfqpoint{1.724082in}{2.899205in}}{\pgfqpoint{1.724082in}{2.888155in}}%
\pgfpathcurveto{\pgfqpoint{1.724082in}{2.877105in}}{\pgfqpoint{1.728472in}{2.866506in}}{\pgfqpoint{1.736285in}{2.858692in}}%
\pgfpathcurveto{\pgfqpoint{1.744099in}{2.850878in}}{\pgfqpoint{1.754698in}{2.846488in}}{\pgfqpoint{1.765748in}{2.846488in}}%
\pgfpathclose%
\pgfusepath{stroke,fill}%
\end{pgfscope}%
\begin{pgfscope}%
\pgfpathrectangle{\pgfqpoint{0.648703in}{0.548769in}}{\pgfqpoint{5.201297in}{3.102590in}}%
\pgfusepath{clip}%
\pgfsetbuttcap%
\pgfsetroundjoin%
\definecolor{currentfill}{rgb}{0.121569,0.466667,0.705882}%
\pgfsetfillcolor{currentfill}%
\pgfsetlinewidth{1.003750pt}%
\definecolor{currentstroke}{rgb}{0.121569,0.466667,0.705882}%
\pgfsetstrokecolor{currentstroke}%
\pgfsetdash{}{0pt}%
\pgfpathmoveto{\pgfqpoint{5.613577in}{3.074620in}}%
\pgfpathcurveto{\pgfqpoint{5.624628in}{3.074620in}}{\pgfqpoint{5.635227in}{3.079010in}}{\pgfqpoint{5.643040in}{3.086824in}}%
\pgfpathcurveto{\pgfqpoint{5.650854in}{3.094637in}}{\pgfqpoint{5.655244in}{3.105236in}}{\pgfqpoint{5.655244in}{3.116286in}}%
\pgfpathcurveto{\pgfqpoint{5.655244in}{3.127336in}}{\pgfqpoint{5.650854in}{3.137935in}}{\pgfqpoint{5.643040in}{3.145749in}}%
\pgfpathcurveto{\pgfqpoint{5.635227in}{3.153563in}}{\pgfqpoint{5.624628in}{3.157953in}}{\pgfqpoint{5.613577in}{3.157953in}}%
\pgfpathcurveto{\pgfqpoint{5.602527in}{3.157953in}}{\pgfqpoint{5.591928in}{3.153563in}}{\pgfqpoint{5.584115in}{3.145749in}}%
\pgfpathcurveto{\pgfqpoint{5.576301in}{3.137935in}}{\pgfqpoint{5.571911in}{3.127336in}}{\pgfqpoint{5.571911in}{3.116286in}}%
\pgfpathcurveto{\pgfqpoint{5.571911in}{3.105236in}}{\pgfqpoint{5.576301in}{3.094637in}}{\pgfqpoint{5.584115in}{3.086824in}}%
\pgfpathcurveto{\pgfqpoint{5.591928in}{3.079010in}}{\pgfqpoint{5.602527in}{3.074620in}}{\pgfqpoint{5.613577in}{3.074620in}}%
\pgfpathclose%
\pgfusepath{stroke,fill}%
\end{pgfscope}%
\begin{pgfscope}%
\pgfpathrectangle{\pgfqpoint{0.648703in}{0.548769in}}{\pgfqpoint{5.201297in}{3.102590in}}%
\pgfusepath{clip}%
\pgfsetbuttcap%
\pgfsetroundjoin%
\definecolor{currentfill}{rgb}{1.000000,0.498039,0.054902}%
\pgfsetfillcolor{currentfill}%
\pgfsetlinewidth{1.003750pt}%
\definecolor{currentstroke}{rgb}{1.000000,0.498039,0.054902}%
\pgfsetstrokecolor{currentstroke}%
\pgfsetdash{}{0pt}%
\pgfpathmoveto{\pgfqpoint{2.100356in}{3.315195in}}%
\pgfpathcurveto{\pgfqpoint{2.111407in}{3.315195in}}{\pgfqpoint{2.122006in}{3.319585in}}{\pgfqpoint{2.129819in}{3.327399in}}%
\pgfpathcurveto{\pgfqpoint{2.137633in}{3.335212in}}{\pgfqpoint{2.142023in}{3.345811in}}{\pgfqpoint{2.142023in}{3.356861in}}%
\pgfpathcurveto{\pgfqpoint{2.142023in}{3.367912in}}{\pgfqpoint{2.137633in}{3.378511in}}{\pgfqpoint{2.129819in}{3.386324in}}%
\pgfpathcurveto{\pgfqpoint{2.122006in}{3.394138in}}{\pgfqpoint{2.111407in}{3.398528in}}{\pgfqpoint{2.100356in}{3.398528in}}%
\pgfpathcurveto{\pgfqpoint{2.089306in}{3.398528in}}{\pgfqpoint{2.078707in}{3.394138in}}{\pgfqpoint{2.070894in}{3.386324in}}%
\pgfpathcurveto{\pgfqpoint{2.063080in}{3.378511in}}{\pgfqpoint{2.058690in}{3.367912in}}{\pgfqpoint{2.058690in}{3.356861in}}%
\pgfpathcurveto{\pgfqpoint{2.058690in}{3.345811in}}{\pgfqpoint{2.063080in}{3.335212in}}{\pgfqpoint{2.070894in}{3.327399in}}%
\pgfpathcurveto{\pgfqpoint{2.078707in}{3.319585in}}{\pgfqpoint{2.089306in}{3.315195in}}{\pgfqpoint{2.100356in}{3.315195in}}%
\pgfpathclose%
\pgfusepath{stroke,fill}%
\end{pgfscope}%
\begin{pgfscope}%
\pgfpathrectangle{\pgfqpoint{0.648703in}{0.548769in}}{\pgfqpoint{5.201297in}{3.102590in}}%
\pgfusepath{clip}%
\pgfsetbuttcap%
\pgfsetroundjoin%
\definecolor{currentfill}{rgb}{0.121569,0.466667,0.705882}%
\pgfsetfillcolor{currentfill}%
\pgfsetlinewidth{1.003750pt}%
\definecolor{currentstroke}{rgb}{0.121569,0.466667,0.705882}%
\pgfsetstrokecolor{currentstroke}%
\pgfsetdash{}{0pt}%
\pgfpathmoveto{\pgfqpoint{1.362143in}{0.648129in}}%
\pgfpathcurveto{\pgfqpoint{1.373194in}{0.648129in}}{\pgfqpoint{1.383793in}{0.652519in}}{\pgfqpoint{1.391606in}{0.660333in}}%
\pgfpathcurveto{\pgfqpoint{1.399420in}{0.668146in}}{\pgfqpoint{1.403810in}{0.678745in}}{\pgfqpoint{1.403810in}{0.689796in}}%
\pgfpathcurveto{\pgfqpoint{1.403810in}{0.700846in}}{\pgfqpoint{1.399420in}{0.711445in}}{\pgfqpoint{1.391606in}{0.719258in}}%
\pgfpathcurveto{\pgfqpoint{1.383793in}{0.727072in}}{\pgfqpoint{1.373194in}{0.731462in}}{\pgfqpoint{1.362143in}{0.731462in}}%
\pgfpathcurveto{\pgfqpoint{1.351093in}{0.731462in}}{\pgfqpoint{1.340494in}{0.727072in}}{\pgfqpoint{1.332681in}{0.719258in}}%
\pgfpathcurveto{\pgfqpoint{1.324867in}{0.711445in}}{\pgfqpoint{1.320477in}{0.700846in}}{\pgfqpoint{1.320477in}{0.689796in}}%
\pgfpathcurveto{\pgfqpoint{1.320477in}{0.678745in}}{\pgfqpoint{1.324867in}{0.668146in}}{\pgfqpoint{1.332681in}{0.660333in}}%
\pgfpathcurveto{\pgfqpoint{1.340494in}{0.652519in}}{\pgfqpoint{1.351093in}{0.648129in}}{\pgfqpoint{1.362143in}{0.648129in}}%
\pgfpathclose%
\pgfusepath{stroke,fill}%
\end{pgfscope}%
\begin{pgfscope}%
\pgfpathrectangle{\pgfqpoint{0.648703in}{0.548769in}}{\pgfqpoint{5.201297in}{3.102590in}}%
\pgfusepath{clip}%
\pgfsetbuttcap%
\pgfsetroundjoin%
\definecolor{currentfill}{rgb}{0.121569,0.466667,0.705882}%
\pgfsetfillcolor{currentfill}%
\pgfsetlinewidth{1.003750pt}%
\definecolor{currentstroke}{rgb}{0.121569,0.466667,0.705882}%
\pgfsetstrokecolor{currentstroke}%
\pgfsetdash{}{0pt}%
\pgfpathmoveto{\pgfqpoint{1.144179in}{1.166610in}}%
\pgfpathcurveto{\pgfqpoint{1.155229in}{1.166610in}}{\pgfqpoint{1.165828in}{1.171000in}}{\pgfqpoint{1.173641in}{1.178814in}}%
\pgfpathcurveto{\pgfqpoint{1.181455in}{1.186627in}}{\pgfqpoint{1.185845in}{1.197226in}}{\pgfqpoint{1.185845in}{1.208277in}}%
\pgfpathcurveto{\pgfqpoint{1.185845in}{1.219327in}}{\pgfqpoint{1.181455in}{1.229926in}}{\pgfqpoint{1.173641in}{1.237739in}}%
\pgfpathcurveto{\pgfqpoint{1.165828in}{1.245553in}}{\pgfqpoint{1.155229in}{1.249943in}}{\pgfqpoint{1.144179in}{1.249943in}}%
\pgfpathcurveto{\pgfqpoint{1.133129in}{1.249943in}}{\pgfqpoint{1.122529in}{1.245553in}}{\pgfqpoint{1.114716in}{1.237739in}}%
\pgfpathcurveto{\pgfqpoint{1.106902in}{1.229926in}}{\pgfqpoint{1.102512in}{1.219327in}}{\pgfqpoint{1.102512in}{1.208277in}}%
\pgfpathcurveto{\pgfqpoint{1.102512in}{1.197226in}}{\pgfqpoint{1.106902in}{1.186627in}}{\pgfqpoint{1.114716in}{1.178814in}}%
\pgfpathcurveto{\pgfqpoint{1.122529in}{1.171000in}}{\pgfqpoint{1.133129in}{1.166610in}}{\pgfqpoint{1.144179in}{1.166610in}}%
\pgfpathclose%
\pgfusepath{stroke,fill}%
\end{pgfscope}%
\begin{pgfscope}%
\pgfpathrectangle{\pgfqpoint{0.648703in}{0.548769in}}{\pgfqpoint{5.201297in}{3.102590in}}%
\pgfusepath{clip}%
\pgfsetbuttcap%
\pgfsetroundjoin%
\definecolor{currentfill}{rgb}{1.000000,0.498039,0.054902}%
\pgfsetfillcolor{currentfill}%
\pgfsetlinewidth{1.003750pt}%
\definecolor{currentstroke}{rgb}{1.000000,0.498039,0.054902}%
\pgfsetstrokecolor{currentstroke}%
\pgfsetdash{}{0pt}%
\pgfpathmoveto{\pgfqpoint{1.372239in}{3.136837in}}%
\pgfpathcurveto{\pgfqpoint{1.383289in}{3.136837in}}{\pgfqpoint{1.393888in}{3.141228in}}{\pgfqpoint{1.401702in}{3.149041in}}%
\pgfpathcurveto{\pgfqpoint{1.409515in}{3.156855in}}{\pgfqpoint{1.413906in}{3.167454in}}{\pgfqpoint{1.413906in}{3.178504in}}%
\pgfpathcurveto{\pgfqpoint{1.413906in}{3.189554in}}{\pgfqpoint{1.409515in}{3.200153in}}{\pgfqpoint{1.401702in}{3.207967in}}%
\pgfpathcurveto{\pgfqpoint{1.393888in}{3.215780in}}{\pgfqpoint{1.383289in}{3.220171in}}{\pgfqpoint{1.372239in}{3.220171in}}%
\pgfpathcurveto{\pgfqpoint{1.361189in}{3.220171in}}{\pgfqpoint{1.350590in}{3.215780in}}{\pgfqpoint{1.342776in}{3.207967in}}%
\pgfpathcurveto{\pgfqpoint{1.334963in}{3.200153in}}{\pgfqpoint{1.330572in}{3.189554in}}{\pgfqpoint{1.330572in}{3.178504in}}%
\pgfpathcurveto{\pgfqpoint{1.330572in}{3.167454in}}{\pgfqpoint{1.334963in}{3.156855in}}{\pgfqpoint{1.342776in}{3.149041in}}%
\pgfpathcurveto{\pgfqpoint{1.350590in}{3.141228in}}{\pgfqpoint{1.361189in}{3.136837in}}{\pgfqpoint{1.372239in}{3.136837in}}%
\pgfpathclose%
\pgfusepath{stroke,fill}%
\end{pgfscope}%
\begin{pgfscope}%
\pgfpathrectangle{\pgfqpoint{0.648703in}{0.548769in}}{\pgfqpoint{5.201297in}{3.102590in}}%
\pgfusepath{clip}%
\pgfsetbuttcap%
\pgfsetroundjoin%
\definecolor{currentfill}{rgb}{0.121569,0.466667,0.705882}%
\pgfsetfillcolor{currentfill}%
\pgfsetlinewidth{1.003750pt}%
\definecolor{currentstroke}{rgb}{0.121569,0.466667,0.705882}%
\pgfsetstrokecolor{currentstroke}%
\pgfsetdash{}{0pt}%
\pgfpathmoveto{\pgfqpoint{2.270979in}{3.132690in}}%
\pgfpathcurveto{\pgfqpoint{2.282029in}{3.132690in}}{\pgfqpoint{2.292628in}{3.137080in}}{\pgfqpoint{2.300442in}{3.144893in}}%
\pgfpathcurveto{\pgfqpoint{2.308256in}{3.152707in}}{\pgfqpoint{2.312646in}{3.163306in}}{\pgfqpoint{2.312646in}{3.174356in}}%
\pgfpathcurveto{\pgfqpoint{2.312646in}{3.185406in}}{\pgfqpoint{2.308256in}{3.196005in}}{\pgfqpoint{2.300442in}{3.203819in}}%
\pgfpathcurveto{\pgfqpoint{2.292628in}{3.211633in}}{\pgfqpoint{2.282029in}{3.216023in}}{\pgfqpoint{2.270979in}{3.216023in}}%
\pgfpathcurveto{\pgfqpoint{2.259929in}{3.216023in}}{\pgfqpoint{2.249330in}{3.211633in}}{\pgfqpoint{2.241517in}{3.203819in}}%
\pgfpathcurveto{\pgfqpoint{2.233703in}{3.196005in}}{\pgfqpoint{2.229313in}{3.185406in}}{\pgfqpoint{2.229313in}{3.174356in}}%
\pgfpathcurveto{\pgfqpoint{2.229313in}{3.163306in}}{\pgfqpoint{2.233703in}{3.152707in}}{\pgfqpoint{2.241517in}{3.144893in}}%
\pgfpathcurveto{\pgfqpoint{2.249330in}{3.137080in}}{\pgfqpoint{2.259929in}{3.132690in}}{\pgfqpoint{2.270979in}{3.132690in}}%
\pgfpathclose%
\pgfusepath{stroke,fill}%
\end{pgfscope}%
\begin{pgfscope}%
\pgfpathrectangle{\pgfqpoint{0.648703in}{0.548769in}}{\pgfqpoint{5.201297in}{3.102590in}}%
\pgfusepath{clip}%
\pgfsetbuttcap%
\pgfsetroundjoin%
\definecolor{currentfill}{rgb}{0.839216,0.152941,0.156863}%
\pgfsetfillcolor{currentfill}%
\pgfsetlinewidth{1.003750pt}%
\definecolor{currentstroke}{rgb}{0.839216,0.152941,0.156863}%
\pgfsetstrokecolor{currentstroke}%
\pgfsetdash{}{0pt}%
\pgfpathmoveto{\pgfqpoint{1.332327in}{3.149281in}}%
\pgfpathcurveto{\pgfqpoint{1.343377in}{3.149281in}}{\pgfqpoint{1.353976in}{3.153671in}}{\pgfqpoint{1.361790in}{3.161485in}}%
\pgfpathcurveto{\pgfqpoint{1.369604in}{3.169298in}}{\pgfqpoint{1.373994in}{3.179897in}}{\pgfqpoint{1.373994in}{3.190948in}}%
\pgfpathcurveto{\pgfqpoint{1.373994in}{3.201998in}}{\pgfqpoint{1.369604in}{3.212597in}}{\pgfqpoint{1.361790in}{3.220410in}}%
\pgfpathcurveto{\pgfqpoint{1.353976in}{3.228224in}}{\pgfqpoint{1.343377in}{3.232614in}}{\pgfqpoint{1.332327in}{3.232614in}}%
\pgfpathcurveto{\pgfqpoint{1.321277in}{3.232614in}}{\pgfqpoint{1.310678in}{3.228224in}}{\pgfqpoint{1.302864in}{3.220410in}}%
\pgfpathcurveto{\pgfqpoint{1.295051in}{3.212597in}}{\pgfqpoint{1.290661in}{3.201998in}}{\pgfqpoint{1.290661in}{3.190948in}}%
\pgfpathcurveto{\pgfqpoint{1.290661in}{3.179897in}}{\pgfqpoint{1.295051in}{3.169298in}}{\pgfqpoint{1.302864in}{3.161485in}}%
\pgfpathcurveto{\pgfqpoint{1.310678in}{3.153671in}}{\pgfqpoint{1.321277in}{3.149281in}}{\pgfqpoint{1.332327in}{3.149281in}}%
\pgfpathclose%
\pgfusepath{stroke,fill}%
\end{pgfscope}%
\begin{pgfscope}%
\pgfpathrectangle{\pgfqpoint{0.648703in}{0.548769in}}{\pgfqpoint{5.201297in}{3.102590in}}%
\pgfusepath{clip}%
\pgfsetbuttcap%
\pgfsetroundjoin%
\definecolor{currentfill}{rgb}{1.000000,0.498039,0.054902}%
\pgfsetfillcolor{currentfill}%
\pgfsetlinewidth{1.003750pt}%
\definecolor{currentstroke}{rgb}{1.000000,0.498039,0.054902}%
\pgfsetstrokecolor{currentstroke}%
\pgfsetdash{}{0pt}%
\pgfpathmoveto{\pgfqpoint{1.278583in}{3.174168in}}%
\pgfpathcurveto{\pgfqpoint{1.289633in}{3.174168in}}{\pgfqpoint{1.300232in}{3.178558in}}{\pgfqpoint{1.308046in}{3.186372in}}%
\pgfpathcurveto{\pgfqpoint{1.315859in}{3.194185in}}{\pgfqpoint{1.320250in}{3.204785in}}{\pgfqpoint{1.320250in}{3.215835in}}%
\pgfpathcurveto{\pgfqpoint{1.320250in}{3.226885in}}{\pgfqpoint{1.315859in}{3.237484in}}{\pgfqpoint{1.308046in}{3.245297in}}%
\pgfpathcurveto{\pgfqpoint{1.300232in}{3.253111in}}{\pgfqpoint{1.289633in}{3.257501in}}{\pgfqpoint{1.278583in}{3.257501in}}%
\pgfpathcurveto{\pgfqpoint{1.267533in}{3.257501in}}{\pgfqpoint{1.256934in}{3.253111in}}{\pgfqpoint{1.249120in}{3.245297in}}%
\pgfpathcurveto{\pgfqpoint{1.241306in}{3.237484in}}{\pgfqpoint{1.236916in}{3.226885in}}{\pgfqpoint{1.236916in}{3.215835in}}%
\pgfpathcurveto{\pgfqpoint{1.236916in}{3.204785in}}{\pgfqpoint{1.241306in}{3.194185in}}{\pgfqpoint{1.249120in}{3.186372in}}%
\pgfpathcurveto{\pgfqpoint{1.256934in}{3.178558in}}{\pgfqpoint{1.267533in}{3.174168in}}{\pgfqpoint{1.278583in}{3.174168in}}%
\pgfpathclose%
\pgfusepath{stroke,fill}%
\end{pgfscope}%
\begin{pgfscope}%
\pgfpathrectangle{\pgfqpoint{0.648703in}{0.548769in}}{\pgfqpoint{5.201297in}{3.102590in}}%
\pgfusepath{clip}%
\pgfsetbuttcap%
\pgfsetroundjoin%
\definecolor{currentfill}{rgb}{0.121569,0.466667,0.705882}%
\pgfsetfillcolor{currentfill}%
\pgfsetlinewidth{1.003750pt}%
\definecolor{currentstroke}{rgb}{0.121569,0.466667,0.705882}%
\pgfsetstrokecolor{currentstroke}%
\pgfsetdash{}{0pt}%
\pgfpathmoveto{\pgfqpoint{1.168830in}{0.656425in}}%
\pgfpathcurveto{\pgfqpoint{1.179880in}{0.656425in}}{\pgfqpoint{1.190479in}{0.660815in}}{\pgfqpoint{1.198292in}{0.668629in}}%
\pgfpathcurveto{\pgfqpoint{1.206106in}{0.676442in}}{\pgfqpoint{1.210496in}{0.687041in}}{\pgfqpoint{1.210496in}{0.698091in}}%
\pgfpathcurveto{\pgfqpoint{1.210496in}{0.709141in}}{\pgfqpoint{1.206106in}{0.719740in}}{\pgfqpoint{1.198292in}{0.727554in}}%
\pgfpathcurveto{\pgfqpoint{1.190479in}{0.735368in}}{\pgfqpoint{1.179880in}{0.739758in}}{\pgfqpoint{1.168830in}{0.739758in}}%
\pgfpathcurveto{\pgfqpoint{1.157779in}{0.739758in}}{\pgfqpoint{1.147180in}{0.735368in}}{\pgfqpoint{1.139367in}{0.727554in}}%
\pgfpathcurveto{\pgfqpoint{1.131553in}{0.719740in}}{\pgfqpoint{1.127163in}{0.709141in}}{\pgfqpoint{1.127163in}{0.698091in}}%
\pgfpathcurveto{\pgfqpoint{1.127163in}{0.687041in}}{\pgfqpoint{1.131553in}{0.676442in}}{\pgfqpoint{1.139367in}{0.668629in}}%
\pgfpathcurveto{\pgfqpoint{1.147180in}{0.660815in}}{\pgfqpoint{1.157779in}{0.656425in}}{\pgfqpoint{1.168830in}{0.656425in}}%
\pgfpathclose%
\pgfusepath{stroke,fill}%
\end{pgfscope}%
\begin{pgfscope}%
\pgfpathrectangle{\pgfqpoint{0.648703in}{0.548769in}}{\pgfqpoint{5.201297in}{3.102590in}}%
\pgfusepath{clip}%
\pgfsetbuttcap%
\pgfsetroundjoin%
\definecolor{currentfill}{rgb}{0.121569,0.466667,0.705882}%
\pgfsetfillcolor{currentfill}%
\pgfsetlinewidth{1.003750pt}%
\definecolor{currentstroke}{rgb}{0.121569,0.466667,0.705882}%
\pgfsetstrokecolor{currentstroke}%
\pgfsetdash{}{0pt}%
\pgfpathmoveto{\pgfqpoint{1.961100in}{3.124394in}}%
\pgfpathcurveto{\pgfqpoint{1.972151in}{3.124394in}}{\pgfqpoint{1.982750in}{3.128784in}}{\pgfqpoint{1.990563in}{3.136598in}}%
\pgfpathcurveto{\pgfqpoint{1.998377in}{3.144411in}}{\pgfqpoint{2.002767in}{3.155010in}}{\pgfqpoint{2.002767in}{3.166060in}}%
\pgfpathcurveto{\pgfqpoint{2.002767in}{3.177111in}}{\pgfqpoint{1.998377in}{3.187710in}}{\pgfqpoint{1.990563in}{3.195523in}}%
\pgfpathcurveto{\pgfqpoint{1.982750in}{3.203337in}}{\pgfqpoint{1.972151in}{3.207727in}}{\pgfqpoint{1.961100in}{3.207727in}}%
\pgfpathcurveto{\pgfqpoint{1.950050in}{3.207727in}}{\pgfqpoint{1.939451in}{3.203337in}}{\pgfqpoint{1.931638in}{3.195523in}}%
\pgfpathcurveto{\pgfqpoint{1.923824in}{3.187710in}}{\pgfqpoint{1.919434in}{3.177111in}}{\pgfqpoint{1.919434in}{3.166060in}}%
\pgfpathcurveto{\pgfqpoint{1.919434in}{3.155010in}}{\pgfqpoint{1.923824in}{3.144411in}}{\pgfqpoint{1.931638in}{3.136598in}}%
\pgfpathcurveto{\pgfqpoint{1.939451in}{3.128784in}}{\pgfqpoint{1.950050in}{3.124394in}}{\pgfqpoint{1.961100in}{3.124394in}}%
\pgfpathclose%
\pgfusepath{stroke,fill}%
\end{pgfscope}%
\begin{pgfscope}%
\pgfpathrectangle{\pgfqpoint{0.648703in}{0.548769in}}{\pgfqpoint{5.201297in}{3.102590in}}%
\pgfusepath{clip}%
\pgfsetbuttcap%
\pgfsetroundjoin%
\definecolor{currentfill}{rgb}{0.121569,0.466667,0.705882}%
\pgfsetfillcolor{currentfill}%
\pgfsetlinewidth{1.003750pt}%
\definecolor{currentstroke}{rgb}{0.121569,0.466667,0.705882}%
\pgfsetstrokecolor{currentstroke}%
\pgfsetdash{}{0pt}%
\pgfpathmoveto{\pgfqpoint{0.983625in}{0.648129in}}%
\pgfpathcurveto{\pgfqpoint{0.994675in}{0.648129in}}{\pgfqpoint{1.005274in}{0.652519in}}{\pgfqpoint{1.013088in}{0.660333in}}%
\pgfpathcurveto{\pgfqpoint{1.020902in}{0.668146in}}{\pgfqpoint{1.025292in}{0.678745in}}{\pgfqpoint{1.025292in}{0.689796in}}%
\pgfpathcurveto{\pgfqpoint{1.025292in}{0.700846in}}{\pgfqpoint{1.020902in}{0.711445in}}{\pgfqpoint{1.013088in}{0.719258in}}%
\pgfpathcurveto{\pgfqpoint{1.005274in}{0.727072in}}{\pgfqpoint{0.994675in}{0.731462in}}{\pgfqpoint{0.983625in}{0.731462in}}%
\pgfpathcurveto{\pgfqpoint{0.972575in}{0.731462in}}{\pgfqpoint{0.961976in}{0.727072in}}{\pgfqpoint{0.954162in}{0.719258in}}%
\pgfpathcurveto{\pgfqpoint{0.946349in}{0.711445in}}{\pgfqpoint{0.941959in}{0.700846in}}{\pgfqpoint{0.941959in}{0.689796in}}%
\pgfpathcurveto{\pgfqpoint{0.941959in}{0.678745in}}{\pgfqpoint{0.946349in}{0.668146in}}{\pgfqpoint{0.954162in}{0.660333in}}%
\pgfpathcurveto{\pgfqpoint{0.961976in}{0.652519in}}{\pgfqpoint{0.972575in}{0.648129in}}{\pgfqpoint{0.983625in}{0.648129in}}%
\pgfpathclose%
\pgfusepath{stroke,fill}%
\end{pgfscope}%
\begin{pgfscope}%
\pgfpathrectangle{\pgfqpoint{0.648703in}{0.548769in}}{\pgfqpoint{5.201297in}{3.102590in}}%
\pgfusepath{clip}%
\pgfsetbuttcap%
\pgfsetroundjoin%
\definecolor{currentfill}{rgb}{1.000000,0.498039,0.054902}%
\pgfsetfillcolor{currentfill}%
\pgfsetlinewidth{1.003750pt}%
\definecolor{currentstroke}{rgb}{1.000000,0.498039,0.054902}%
\pgfsetstrokecolor{currentstroke}%
\pgfsetdash{}{0pt}%
\pgfpathmoveto{\pgfqpoint{1.713328in}{3.136837in}}%
\pgfpathcurveto{\pgfqpoint{1.724378in}{3.136837in}}{\pgfqpoint{1.734977in}{3.141228in}}{\pgfqpoint{1.742791in}{3.149041in}}%
\pgfpathcurveto{\pgfqpoint{1.750604in}{3.156855in}}{\pgfqpoint{1.754995in}{3.167454in}}{\pgfqpoint{1.754995in}{3.178504in}}%
\pgfpathcurveto{\pgfqpoint{1.754995in}{3.189554in}}{\pgfqpoint{1.750604in}{3.200153in}}{\pgfqpoint{1.742791in}{3.207967in}}%
\pgfpathcurveto{\pgfqpoint{1.734977in}{3.215780in}}{\pgfqpoint{1.724378in}{3.220171in}}{\pgfqpoint{1.713328in}{3.220171in}}%
\pgfpathcurveto{\pgfqpoint{1.702278in}{3.220171in}}{\pgfqpoint{1.691679in}{3.215780in}}{\pgfqpoint{1.683865in}{3.207967in}}%
\pgfpathcurveto{\pgfqpoint{1.676052in}{3.200153in}}{\pgfqpoint{1.671661in}{3.189554in}}{\pgfqpoint{1.671661in}{3.178504in}}%
\pgfpathcurveto{\pgfqpoint{1.671661in}{3.167454in}}{\pgfqpoint{1.676052in}{3.156855in}}{\pgfqpoint{1.683865in}{3.149041in}}%
\pgfpathcurveto{\pgfqpoint{1.691679in}{3.141228in}}{\pgfqpoint{1.702278in}{3.136837in}}{\pgfqpoint{1.713328in}{3.136837in}}%
\pgfpathclose%
\pgfusepath{stroke,fill}%
\end{pgfscope}%
\begin{pgfscope}%
\pgfpathrectangle{\pgfqpoint{0.648703in}{0.548769in}}{\pgfqpoint{5.201297in}{3.102590in}}%
\pgfusepath{clip}%
\pgfsetbuttcap%
\pgfsetroundjoin%
\definecolor{currentfill}{rgb}{1.000000,0.498039,0.054902}%
\pgfsetfillcolor{currentfill}%
\pgfsetlinewidth{1.003750pt}%
\definecolor{currentstroke}{rgb}{1.000000,0.498039,0.054902}%
\pgfsetstrokecolor{currentstroke}%
\pgfsetdash{}{0pt}%
\pgfpathmoveto{\pgfqpoint{1.586093in}{3.140985in}}%
\pgfpathcurveto{\pgfqpoint{1.597143in}{3.140985in}}{\pgfqpoint{1.607742in}{3.145375in}}{\pgfqpoint{1.615555in}{3.153189in}}%
\pgfpathcurveto{\pgfqpoint{1.623369in}{3.161003in}}{\pgfqpoint{1.627759in}{3.171602in}}{\pgfqpoint{1.627759in}{3.182652in}}%
\pgfpathcurveto{\pgfqpoint{1.627759in}{3.193702in}}{\pgfqpoint{1.623369in}{3.204301in}}{\pgfqpoint{1.615555in}{3.212115in}}%
\pgfpathcurveto{\pgfqpoint{1.607742in}{3.219928in}}{\pgfqpoint{1.597143in}{3.224319in}}{\pgfqpoint{1.586093in}{3.224319in}}%
\pgfpathcurveto{\pgfqpoint{1.575042in}{3.224319in}}{\pgfqpoint{1.564443in}{3.219928in}}{\pgfqpoint{1.556630in}{3.212115in}}%
\pgfpathcurveto{\pgfqpoint{1.548816in}{3.204301in}}{\pgfqpoint{1.544426in}{3.193702in}}{\pgfqpoint{1.544426in}{3.182652in}}%
\pgfpathcurveto{\pgfqpoint{1.544426in}{3.171602in}}{\pgfqpoint{1.548816in}{3.161003in}}{\pgfqpoint{1.556630in}{3.153189in}}%
\pgfpathcurveto{\pgfqpoint{1.564443in}{3.145375in}}{\pgfqpoint{1.575042in}{3.140985in}}{\pgfqpoint{1.586093in}{3.140985in}}%
\pgfpathclose%
\pgfusepath{stroke,fill}%
\end{pgfscope}%
\begin{pgfscope}%
\pgfpathrectangle{\pgfqpoint{0.648703in}{0.548769in}}{\pgfqpoint{5.201297in}{3.102590in}}%
\pgfusepath{clip}%
\pgfsetbuttcap%
\pgfsetroundjoin%
\definecolor{currentfill}{rgb}{1.000000,0.498039,0.054902}%
\pgfsetfillcolor{currentfill}%
\pgfsetlinewidth{1.003750pt}%
\definecolor{currentstroke}{rgb}{1.000000,0.498039,0.054902}%
\pgfsetstrokecolor{currentstroke}%
\pgfsetdash{}{0pt}%
\pgfpathmoveto{\pgfqpoint{1.976762in}{3.165872in}}%
\pgfpathcurveto{\pgfqpoint{1.987812in}{3.165872in}}{\pgfqpoint{1.998411in}{3.170263in}}{\pgfqpoint{2.006225in}{3.178076in}}%
\pgfpathcurveto{\pgfqpoint{2.014038in}{3.185890in}}{\pgfqpoint{2.018429in}{3.196489in}}{\pgfqpoint{2.018429in}{3.207539in}}%
\pgfpathcurveto{\pgfqpoint{2.018429in}{3.218589in}}{\pgfqpoint{2.014038in}{3.229188in}}{\pgfqpoint{2.006225in}{3.237002in}}%
\pgfpathcurveto{\pgfqpoint{1.998411in}{3.244815in}}{\pgfqpoint{1.987812in}{3.249206in}}{\pgfqpoint{1.976762in}{3.249206in}}%
\pgfpathcurveto{\pgfqpoint{1.965712in}{3.249206in}}{\pgfqpoint{1.955113in}{3.244815in}}{\pgfqpoint{1.947299in}{3.237002in}}%
\pgfpathcurveto{\pgfqpoint{1.939486in}{3.229188in}}{\pgfqpoint{1.935095in}{3.218589in}}{\pgfqpoint{1.935095in}{3.207539in}}%
\pgfpathcurveto{\pgfqpoint{1.935095in}{3.196489in}}{\pgfqpoint{1.939486in}{3.185890in}}{\pgfqpoint{1.947299in}{3.178076in}}%
\pgfpathcurveto{\pgfqpoint{1.955113in}{3.170263in}}{\pgfqpoint{1.965712in}{3.165872in}}{\pgfqpoint{1.976762in}{3.165872in}}%
\pgfpathclose%
\pgfusepath{stroke,fill}%
\end{pgfscope}%
\begin{pgfscope}%
\pgfpathrectangle{\pgfqpoint{0.648703in}{0.548769in}}{\pgfqpoint{5.201297in}{3.102590in}}%
\pgfusepath{clip}%
\pgfsetbuttcap%
\pgfsetroundjoin%
\definecolor{currentfill}{rgb}{1.000000,0.498039,0.054902}%
\pgfsetfillcolor{currentfill}%
\pgfsetlinewidth{1.003750pt}%
\definecolor{currentstroke}{rgb}{1.000000,0.498039,0.054902}%
\pgfsetstrokecolor{currentstroke}%
\pgfsetdash{}{0pt}%
\pgfpathmoveto{\pgfqpoint{2.393433in}{3.145133in}}%
\pgfpathcurveto{\pgfqpoint{2.404483in}{3.145133in}}{\pgfqpoint{2.415082in}{3.149523in}}{\pgfqpoint{2.422895in}{3.157337in}}%
\pgfpathcurveto{\pgfqpoint{2.430709in}{3.165151in}}{\pgfqpoint{2.435099in}{3.175750in}}{\pgfqpoint{2.435099in}{3.186800in}}%
\pgfpathcurveto{\pgfqpoint{2.435099in}{3.197850in}}{\pgfqpoint{2.430709in}{3.208449in}}{\pgfqpoint{2.422895in}{3.216262in}}%
\pgfpathcurveto{\pgfqpoint{2.415082in}{3.224076in}}{\pgfqpoint{2.404483in}{3.228466in}}{\pgfqpoint{2.393433in}{3.228466in}}%
\pgfpathcurveto{\pgfqpoint{2.382383in}{3.228466in}}{\pgfqpoint{2.371783in}{3.224076in}}{\pgfqpoint{2.363970in}{3.216262in}}%
\pgfpathcurveto{\pgfqpoint{2.356156in}{3.208449in}}{\pgfqpoint{2.351766in}{3.197850in}}{\pgfqpoint{2.351766in}{3.186800in}}%
\pgfpathcurveto{\pgfqpoint{2.351766in}{3.175750in}}{\pgfqpoint{2.356156in}{3.165151in}}{\pgfqpoint{2.363970in}{3.157337in}}%
\pgfpathcurveto{\pgfqpoint{2.371783in}{3.149523in}}{\pgfqpoint{2.382383in}{3.145133in}}{\pgfqpoint{2.393433in}{3.145133in}}%
\pgfpathclose%
\pgfusepath{stroke,fill}%
\end{pgfscope}%
\begin{pgfscope}%
\pgfpathrectangle{\pgfqpoint{0.648703in}{0.548769in}}{\pgfqpoint{5.201297in}{3.102590in}}%
\pgfusepath{clip}%
\pgfsetbuttcap%
\pgfsetroundjoin%
\definecolor{currentfill}{rgb}{1.000000,0.498039,0.054902}%
\pgfsetfillcolor{currentfill}%
\pgfsetlinewidth{1.003750pt}%
\definecolor{currentstroke}{rgb}{1.000000,0.498039,0.054902}%
\pgfsetstrokecolor{currentstroke}%
\pgfsetdash{}{0pt}%
\pgfpathmoveto{\pgfqpoint{1.287398in}{3.157577in}}%
\pgfpathcurveto{\pgfqpoint{1.298448in}{3.157577in}}{\pgfqpoint{1.309047in}{3.161967in}}{\pgfqpoint{1.316861in}{3.169780in}}%
\pgfpathcurveto{\pgfqpoint{1.324674in}{3.177594in}}{\pgfqpoint{1.329065in}{3.188193in}}{\pgfqpoint{1.329065in}{3.199243in}}%
\pgfpathcurveto{\pgfqpoint{1.329065in}{3.210293in}}{\pgfqpoint{1.324674in}{3.220892in}}{\pgfqpoint{1.316861in}{3.228706in}}%
\pgfpathcurveto{\pgfqpoint{1.309047in}{3.236520in}}{\pgfqpoint{1.298448in}{3.240910in}}{\pgfqpoint{1.287398in}{3.240910in}}%
\pgfpathcurveto{\pgfqpoint{1.276348in}{3.240910in}}{\pgfqpoint{1.265749in}{3.236520in}}{\pgfqpoint{1.257935in}{3.228706in}}%
\pgfpathcurveto{\pgfqpoint{1.250122in}{3.220892in}}{\pgfqpoint{1.245731in}{3.210293in}}{\pgfqpoint{1.245731in}{3.199243in}}%
\pgfpathcurveto{\pgfqpoint{1.245731in}{3.188193in}}{\pgfqpoint{1.250122in}{3.177594in}}{\pgfqpoint{1.257935in}{3.169780in}}%
\pgfpathcurveto{\pgfqpoint{1.265749in}{3.161967in}}{\pgfqpoint{1.276348in}{3.157577in}}{\pgfqpoint{1.287398in}{3.157577in}}%
\pgfpathclose%
\pgfusepath{stroke,fill}%
\end{pgfscope}%
\begin{pgfscope}%
\pgfpathrectangle{\pgfqpoint{0.648703in}{0.548769in}}{\pgfqpoint{5.201297in}{3.102590in}}%
\pgfusepath{clip}%
\pgfsetbuttcap%
\pgfsetroundjoin%
\definecolor{currentfill}{rgb}{1.000000,0.498039,0.054902}%
\pgfsetfillcolor{currentfill}%
\pgfsetlinewidth{1.003750pt}%
\definecolor{currentstroke}{rgb}{1.000000,0.498039,0.054902}%
\pgfsetstrokecolor{currentstroke}%
\pgfsetdash{}{0pt}%
\pgfpathmoveto{\pgfqpoint{1.452376in}{3.323490in}}%
\pgfpathcurveto{\pgfqpoint{1.463427in}{3.323490in}}{\pgfqpoint{1.474026in}{3.327881in}}{\pgfqpoint{1.481839in}{3.335694in}}%
\pgfpathcurveto{\pgfqpoint{1.489653in}{3.343508in}}{\pgfqpoint{1.494043in}{3.354107in}}{\pgfqpoint{1.494043in}{3.365157in}}%
\pgfpathcurveto{\pgfqpoint{1.494043in}{3.376207in}}{\pgfqpoint{1.489653in}{3.386806in}}{\pgfqpoint{1.481839in}{3.394620in}}%
\pgfpathcurveto{\pgfqpoint{1.474026in}{3.402434in}}{\pgfqpoint{1.463427in}{3.406824in}}{\pgfqpoint{1.452376in}{3.406824in}}%
\pgfpathcurveto{\pgfqpoint{1.441326in}{3.406824in}}{\pgfqpoint{1.430727in}{3.402434in}}{\pgfqpoint{1.422914in}{3.394620in}}%
\pgfpathcurveto{\pgfqpoint{1.415100in}{3.386806in}}{\pgfqpoint{1.410710in}{3.376207in}}{\pgfqpoint{1.410710in}{3.365157in}}%
\pgfpathcurveto{\pgfqpoint{1.410710in}{3.354107in}}{\pgfqpoint{1.415100in}{3.343508in}}{\pgfqpoint{1.422914in}{3.335694in}}%
\pgfpathcurveto{\pgfqpoint{1.430727in}{3.327881in}}{\pgfqpoint{1.441326in}{3.323490in}}{\pgfqpoint{1.452376in}{3.323490in}}%
\pgfpathclose%
\pgfusepath{stroke,fill}%
\end{pgfscope}%
\begin{pgfscope}%
\pgfpathrectangle{\pgfqpoint{0.648703in}{0.548769in}}{\pgfqpoint{5.201297in}{3.102590in}}%
\pgfusepath{clip}%
\pgfsetbuttcap%
\pgfsetroundjoin%
\definecolor{currentfill}{rgb}{0.121569,0.466667,0.705882}%
\pgfsetfillcolor{currentfill}%
\pgfsetlinewidth{1.003750pt}%
\definecolor{currentstroke}{rgb}{0.121569,0.466667,0.705882}%
\pgfsetstrokecolor{currentstroke}%
\pgfsetdash{}{0pt}%
\pgfpathmoveto{\pgfqpoint{1.229159in}{2.410964in}}%
\pgfpathcurveto{\pgfqpoint{1.240209in}{2.410964in}}{\pgfqpoint{1.250808in}{2.415354in}}{\pgfqpoint{1.258622in}{2.423168in}}%
\pgfpathcurveto{\pgfqpoint{1.266435in}{2.430982in}}{\pgfqpoint{1.270826in}{2.441581in}}{\pgfqpoint{1.270826in}{2.452631in}}%
\pgfpathcurveto{\pgfqpoint{1.270826in}{2.463681in}}{\pgfqpoint{1.266435in}{2.474280in}}{\pgfqpoint{1.258622in}{2.482094in}}%
\pgfpathcurveto{\pgfqpoint{1.250808in}{2.489907in}}{\pgfqpoint{1.240209in}{2.494297in}}{\pgfqpoint{1.229159in}{2.494297in}}%
\pgfpathcurveto{\pgfqpoint{1.218109in}{2.494297in}}{\pgfqpoint{1.207510in}{2.489907in}}{\pgfqpoint{1.199696in}{2.482094in}}%
\pgfpathcurveto{\pgfqpoint{1.191883in}{2.474280in}}{\pgfqpoint{1.187492in}{2.463681in}}{\pgfqpoint{1.187492in}{2.452631in}}%
\pgfpathcurveto{\pgfqpoint{1.187492in}{2.441581in}}{\pgfqpoint{1.191883in}{2.430982in}}{\pgfqpoint{1.199696in}{2.423168in}}%
\pgfpathcurveto{\pgfqpoint{1.207510in}{2.415354in}}{\pgfqpoint{1.218109in}{2.410964in}}{\pgfqpoint{1.229159in}{2.410964in}}%
\pgfpathclose%
\pgfusepath{stroke,fill}%
\end{pgfscope}%
\begin{pgfscope}%
\pgfpathrectangle{\pgfqpoint{0.648703in}{0.548769in}}{\pgfqpoint{5.201297in}{3.102590in}}%
\pgfusepath{clip}%
\pgfsetbuttcap%
\pgfsetroundjoin%
\definecolor{currentfill}{rgb}{1.000000,0.498039,0.054902}%
\pgfsetfillcolor{currentfill}%
\pgfsetlinewidth{1.003750pt}%
\definecolor{currentstroke}{rgb}{1.000000,0.498039,0.054902}%
\pgfsetstrokecolor{currentstroke}%
\pgfsetdash{}{0pt}%
\pgfpathmoveto{\pgfqpoint{1.398170in}{3.199055in}}%
\pgfpathcurveto{\pgfqpoint{1.409221in}{3.199055in}}{\pgfqpoint{1.419820in}{3.203445in}}{\pgfqpoint{1.427633in}{3.211259in}}%
\pgfpathcurveto{\pgfqpoint{1.435447in}{3.219073in}}{\pgfqpoint{1.439837in}{3.229672in}}{\pgfqpoint{1.439837in}{3.240722in}}%
\pgfpathcurveto{\pgfqpoint{1.439837in}{3.251772in}}{\pgfqpoint{1.435447in}{3.262371in}}{\pgfqpoint{1.427633in}{3.270185in}}%
\pgfpathcurveto{\pgfqpoint{1.419820in}{3.277998in}}{\pgfqpoint{1.409221in}{3.282388in}}{\pgfqpoint{1.398170in}{3.282388in}}%
\pgfpathcurveto{\pgfqpoint{1.387120in}{3.282388in}}{\pgfqpoint{1.376521in}{3.277998in}}{\pgfqpoint{1.368708in}{3.270185in}}%
\pgfpathcurveto{\pgfqpoint{1.360894in}{3.262371in}}{\pgfqpoint{1.356504in}{3.251772in}}{\pgfqpoint{1.356504in}{3.240722in}}%
\pgfpathcurveto{\pgfqpoint{1.356504in}{3.229672in}}{\pgfqpoint{1.360894in}{3.219073in}}{\pgfqpoint{1.368708in}{3.211259in}}%
\pgfpathcurveto{\pgfqpoint{1.376521in}{3.203445in}}{\pgfqpoint{1.387120in}{3.199055in}}{\pgfqpoint{1.398170in}{3.199055in}}%
\pgfpathclose%
\pgfusepath{stroke,fill}%
\end{pgfscope}%
\begin{pgfscope}%
\pgfpathrectangle{\pgfqpoint{0.648703in}{0.548769in}}{\pgfqpoint{5.201297in}{3.102590in}}%
\pgfusepath{clip}%
\pgfsetbuttcap%
\pgfsetroundjoin%
\definecolor{currentfill}{rgb}{0.121569,0.466667,0.705882}%
\pgfsetfillcolor{currentfill}%
\pgfsetlinewidth{1.003750pt}%
\definecolor{currentstroke}{rgb}{0.121569,0.466667,0.705882}%
\pgfsetstrokecolor{currentstroke}%
\pgfsetdash{}{0pt}%
\pgfpathmoveto{\pgfqpoint{2.897601in}{3.124394in}}%
\pgfpathcurveto{\pgfqpoint{2.908651in}{3.124394in}}{\pgfqpoint{2.919250in}{3.128784in}}{\pgfqpoint{2.927064in}{3.136598in}}%
\pgfpathcurveto{\pgfqpoint{2.934877in}{3.144411in}}{\pgfqpoint{2.939268in}{3.155010in}}{\pgfqpoint{2.939268in}{3.166060in}}%
\pgfpathcurveto{\pgfqpoint{2.939268in}{3.177111in}}{\pgfqpoint{2.934877in}{3.187710in}}{\pgfqpoint{2.927064in}{3.195523in}}%
\pgfpathcurveto{\pgfqpoint{2.919250in}{3.203337in}}{\pgfqpoint{2.908651in}{3.207727in}}{\pgfqpoint{2.897601in}{3.207727in}}%
\pgfpathcurveto{\pgfqpoint{2.886551in}{3.207727in}}{\pgfqpoint{2.875952in}{3.203337in}}{\pgfqpoint{2.868138in}{3.195523in}}%
\pgfpathcurveto{\pgfqpoint{2.860325in}{3.187710in}}{\pgfqpoint{2.855934in}{3.177111in}}{\pgfqpoint{2.855934in}{3.166060in}}%
\pgfpathcurveto{\pgfqpoint{2.855934in}{3.155010in}}{\pgfqpoint{2.860325in}{3.144411in}}{\pgfqpoint{2.868138in}{3.136598in}}%
\pgfpathcurveto{\pgfqpoint{2.875952in}{3.128784in}}{\pgfqpoint{2.886551in}{3.124394in}}{\pgfqpoint{2.897601in}{3.124394in}}%
\pgfpathclose%
\pgfusepath{stroke,fill}%
\end{pgfscope}%
\begin{pgfscope}%
\pgfpathrectangle{\pgfqpoint{0.648703in}{0.548769in}}{\pgfqpoint{5.201297in}{3.102590in}}%
\pgfusepath{clip}%
\pgfsetbuttcap%
\pgfsetroundjoin%
\definecolor{currentfill}{rgb}{0.121569,0.466667,0.705882}%
\pgfsetfillcolor{currentfill}%
\pgfsetlinewidth{1.003750pt}%
\definecolor{currentstroke}{rgb}{0.121569,0.466667,0.705882}%
\pgfsetstrokecolor{currentstroke}%
\pgfsetdash{}{0pt}%
\pgfpathmoveto{\pgfqpoint{2.087831in}{3.132690in}}%
\pgfpathcurveto{\pgfqpoint{2.098881in}{3.132690in}}{\pgfqpoint{2.109480in}{3.137080in}}{\pgfqpoint{2.117293in}{3.144893in}}%
\pgfpathcurveto{\pgfqpoint{2.125107in}{3.152707in}}{\pgfqpoint{2.129497in}{3.163306in}}{\pgfqpoint{2.129497in}{3.174356in}}%
\pgfpathcurveto{\pgfqpoint{2.129497in}{3.185406in}}{\pgfqpoint{2.125107in}{3.196005in}}{\pgfqpoint{2.117293in}{3.203819in}}%
\pgfpathcurveto{\pgfqpoint{2.109480in}{3.211633in}}{\pgfqpoint{2.098881in}{3.216023in}}{\pgfqpoint{2.087831in}{3.216023in}}%
\pgfpathcurveto{\pgfqpoint{2.076781in}{3.216023in}}{\pgfqpoint{2.066181in}{3.211633in}}{\pgfqpoint{2.058368in}{3.203819in}}%
\pgfpathcurveto{\pgfqpoint{2.050554in}{3.196005in}}{\pgfqpoint{2.046164in}{3.185406in}}{\pgfqpoint{2.046164in}{3.174356in}}%
\pgfpathcurveto{\pgfqpoint{2.046164in}{3.163306in}}{\pgfqpoint{2.050554in}{3.152707in}}{\pgfqpoint{2.058368in}{3.144893in}}%
\pgfpathcurveto{\pgfqpoint{2.066181in}{3.137080in}}{\pgfqpoint{2.076781in}{3.132690in}}{\pgfqpoint{2.087831in}{3.132690in}}%
\pgfpathclose%
\pgfusepath{stroke,fill}%
\end{pgfscope}%
\begin{pgfscope}%
\pgfpathrectangle{\pgfqpoint{0.648703in}{0.548769in}}{\pgfqpoint{5.201297in}{3.102590in}}%
\pgfusepath{clip}%
\pgfsetbuttcap%
\pgfsetroundjoin%
\definecolor{currentfill}{rgb}{0.121569,0.466667,0.705882}%
\pgfsetfillcolor{currentfill}%
\pgfsetlinewidth{1.003750pt}%
\definecolor{currentstroke}{rgb}{0.121569,0.466667,0.705882}%
\pgfsetstrokecolor{currentstroke}%
\pgfsetdash{}{0pt}%
\pgfpathmoveto{\pgfqpoint{1.678773in}{3.132690in}}%
\pgfpathcurveto{\pgfqpoint{1.689823in}{3.132690in}}{\pgfqpoint{1.700422in}{3.137080in}}{\pgfqpoint{1.708236in}{3.144893in}}%
\pgfpathcurveto{\pgfqpoint{1.716049in}{3.152707in}}{\pgfqpoint{1.720440in}{3.163306in}}{\pgfqpoint{1.720440in}{3.174356in}}%
\pgfpathcurveto{\pgfqpoint{1.720440in}{3.185406in}}{\pgfqpoint{1.716049in}{3.196005in}}{\pgfqpoint{1.708236in}{3.203819in}}%
\pgfpathcurveto{\pgfqpoint{1.700422in}{3.211633in}}{\pgfqpoint{1.689823in}{3.216023in}}{\pgfqpoint{1.678773in}{3.216023in}}%
\pgfpathcurveto{\pgfqpoint{1.667723in}{3.216023in}}{\pgfqpoint{1.657124in}{3.211633in}}{\pgfqpoint{1.649310in}{3.203819in}}%
\pgfpathcurveto{\pgfqpoint{1.641497in}{3.196005in}}{\pgfqpoint{1.637106in}{3.185406in}}{\pgfqpoint{1.637106in}{3.174356in}}%
\pgfpathcurveto{\pgfqpoint{1.637106in}{3.163306in}}{\pgfqpoint{1.641497in}{3.152707in}}{\pgfqpoint{1.649310in}{3.144893in}}%
\pgfpathcurveto{\pgfqpoint{1.657124in}{3.137080in}}{\pgfqpoint{1.667723in}{3.132690in}}{\pgfqpoint{1.678773in}{3.132690in}}%
\pgfpathclose%
\pgfusepath{stroke,fill}%
\end{pgfscope}%
\begin{pgfscope}%
\pgfpathrectangle{\pgfqpoint{0.648703in}{0.548769in}}{\pgfqpoint{5.201297in}{3.102590in}}%
\pgfusepath{clip}%
\pgfsetbuttcap%
\pgfsetroundjoin%
\definecolor{currentfill}{rgb}{1.000000,0.498039,0.054902}%
\pgfsetfillcolor{currentfill}%
\pgfsetlinewidth{1.003750pt}%
\definecolor{currentstroke}{rgb}{1.000000,0.498039,0.054902}%
\pgfsetstrokecolor{currentstroke}%
\pgfsetdash{}{0pt}%
\pgfpathmoveto{\pgfqpoint{1.449659in}{3.136837in}}%
\pgfpathcurveto{\pgfqpoint{1.460709in}{3.136837in}}{\pgfqpoint{1.471308in}{3.141228in}}{\pgfqpoint{1.479121in}{3.149041in}}%
\pgfpathcurveto{\pgfqpoint{1.486935in}{3.156855in}}{\pgfqpoint{1.491325in}{3.167454in}}{\pgfqpoint{1.491325in}{3.178504in}}%
\pgfpathcurveto{\pgfqpoint{1.491325in}{3.189554in}}{\pgfqpoint{1.486935in}{3.200153in}}{\pgfqpoint{1.479121in}{3.207967in}}%
\pgfpathcurveto{\pgfqpoint{1.471308in}{3.215780in}}{\pgfqpoint{1.460709in}{3.220171in}}{\pgfqpoint{1.449659in}{3.220171in}}%
\pgfpathcurveto{\pgfqpoint{1.438609in}{3.220171in}}{\pgfqpoint{1.428010in}{3.215780in}}{\pgfqpoint{1.420196in}{3.207967in}}%
\pgfpathcurveto{\pgfqpoint{1.412382in}{3.200153in}}{\pgfqpoint{1.407992in}{3.189554in}}{\pgfqpoint{1.407992in}{3.178504in}}%
\pgfpathcurveto{\pgfqpoint{1.407992in}{3.167454in}}{\pgfqpoint{1.412382in}{3.156855in}}{\pgfqpoint{1.420196in}{3.149041in}}%
\pgfpathcurveto{\pgfqpoint{1.428010in}{3.141228in}}{\pgfqpoint{1.438609in}{3.136837in}}{\pgfqpoint{1.449659in}{3.136837in}}%
\pgfpathclose%
\pgfusepath{stroke,fill}%
\end{pgfscope}%
\begin{pgfscope}%
\pgfpathrectangle{\pgfqpoint{0.648703in}{0.548769in}}{\pgfqpoint{5.201297in}{3.102590in}}%
\pgfusepath{clip}%
\pgfsetbuttcap%
\pgfsetroundjoin%
\definecolor{currentfill}{rgb}{1.000000,0.498039,0.054902}%
\pgfsetfillcolor{currentfill}%
\pgfsetlinewidth{1.003750pt}%
\definecolor{currentstroke}{rgb}{1.000000,0.498039,0.054902}%
\pgfsetstrokecolor{currentstroke}%
\pgfsetdash{}{0pt}%
\pgfpathmoveto{\pgfqpoint{1.265822in}{3.244681in}}%
\pgfpathcurveto{\pgfqpoint{1.276872in}{3.244681in}}{\pgfqpoint{1.287471in}{3.249072in}}{\pgfqpoint{1.295285in}{3.256885in}}%
\pgfpathcurveto{\pgfqpoint{1.303098in}{3.264699in}}{\pgfqpoint{1.307489in}{3.275298in}}{\pgfqpoint{1.307489in}{3.286348in}}%
\pgfpathcurveto{\pgfqpoint{1.307489in}{3.297398in}}{\pgfqpoint{1.303098in}{3.307997in}}{\pgfqpoint{1.295285in}{3.315811in}}%
\pgfpathcurveto{\pgfqpoint{1.287471in}{3.323624in}}{\pgfqpoint{1.276872in}{3.328015in}}{\pgfqpoint{1.265822in}{3.328015in}}%
\pgfpathcurveto{\pgfqpoint{1.254772in}{3.328015in}}{\pgfqpoint{1.244173in}{3.323624in}}{\pgfqpoint{1.236359in}{3.315811in}}%
\pgfpathcurveto{\pgfqpoint{1.228545in}{3.307997in}}{\pgfqpoint{1.224155in}{3.297398in}}{\pgfqpoint{1.224155in}{3.286348in}}%
\pgfpathcurveto{\pgfqpoint{1.224155in}{3.275298in}}{\pgfqpoint{1.228545in}{3.264699in}}{\pgfqpoint{1.236359in}{3.256885in}}%
\pgfpathcurveto{\pgfqpoint{1.244173in}{3.249072in}}{\pgfqpoint{1.254772in}{3.244681in}}{\pgfqpoint{1.265822in}{3.244681in}}%
\pgfpathclose%
\pgfusepath{stroke,fill}%
\end{pgfscope}%
\begin{pgfscope}%
\pgfpathrectangle{\pgfqpoint{0.648703in}{0.548769in}}{\pgfqpoint{5.201297in}{3.102590in}}%
\pgfusepath{clip}%
\pgfsetbuttcap%
\pgfsetroundjoin%
\definecolor{currentfill}{rgb}{1.000000,0.498039,0.054902}%
\pgfsetfillcolor{currentfill}%
\pgfsetlinewidth{1.003750pt}%
\definecolor{currentstroke}{rgb}{1.000000,0.498039,0.054902}%
\pgfsetstrokecolor{currentstroke}%
\pgfsetdash{}{0pt}%
\pgfpathmoveto{\pgfqpoint{1.363903in}{3.190759in}}%
\pgfpathcurveto{\pgfqpoint{1.374953in}{3.190759in}}{\pgfqpoint{1.385552in}{3.195150in}}{\pgfqpoint{1.393366in}{3.202963in}}%
\pgfpathcurveto{\pgfqpoint{1.401179in}{3.210777in}}{\pgfqpoint{1.405570in}{3.221376in}}{\pgfqpoint{1.405570in}{3.232426in}}%
\pgfpathcurveto{\pgfqpoint{1.405570in}{3.243476in}}{\pgfqpoint{1.401179in}{3.254075in}}{\pgfqpoint{1.393366in}{3.261889in}}%
\pgfpathcurveto{\pgfqpoint{1.385552in}{3.269702in}}{\pgfqpoint{1.374953in}{3.274093in}}{\pgfqpoint{1.363903in}{3.274093in}}%
\pgfpathcurveto{\pgfqpoint{1.352853in}{3.274093in}}{\pgfqpoint{1.342254in}{3.269702in}}{\pgfqpoint{1.334440in}{3.261889in}}%
\pgfpathcurveto{\pgfqpoint{1.326627in}{3.254075in}}{\pgfqpoint{1.322236in}{3.243476in}}{\pgfqpoint{1.322236in}{3.232426in}}%
\pgfpathcurveto{\pgfqpoint{1.322236in}{3.221376in}}{\pgfqpoint{1.326627in}{3.210777in}}{\pgfqpoint{1.334440in}{3.202963in}}%
\pgfpathcurveto{\pgfqpoint{1.342254in}{3.195150in}}{\pgfqpoint{1.352853in}{3.190759in}}{\pgfqpoint{1.363903in}{3.190759in}}%
\pgfpathclose%
\pgfusepath{stroke,fill}%
\end{pgfscope}%
\begin{pgfscope}%
\pgfpathrectangle{\pgfqpoint{0.648703in}{0.548769in}}{\pgfqpoint{5.201297in}{3.102590in}}%
\pgfusepath{clip}%
\pgfsetbuttcap%
\pgfsetroundjoin%
\definecolor{currentfill}{rgb}{0.121569,0.466667,0.705882}%
\pgfsetfillcolor{currentfill}%
\pgfsetlinewidth{1.003750pt}%
\definecolor{currentstroke}{rgb}{0.121569,0.466667,0.705882}%
\pgfsetstrokecolor{currentstroke}%
\pgfsetdash{}{0pt}%
\pgfpathmoveto{\pgfqpoint{0.888183in}{0.664720in}}%
\pgfpathcurveto{\pgfqpoint{0.899234in}{0.664720in}}{\pgfqpoint{0.909833in}{0.669111in}}{\pgfqpoint{0.917646in}{0.676924in}}%
\pgfpathcurveto{\pgfqpoint{0.925460in}{0.684738in}}{\pgfqpoint{0.929850in}{0.695337in}}{\pgfqpoint{0.929850in}{0.706387in}}%
\pgfpathcurveto{\pgfqpoint{0.929850in}{0.717437in}}{\pgfqpoint{0.925460in}{0.728036in}}{\pgfqpoint{0.917646in}{0.735850in}}%
\pgfpathcurveto{\pgfqpoint{0.909833in}{0.743663in}}{\pgfqpoint{0.899234in}{0.748054in}}{\pgfqpoint{0.888183in}{0.748054in}}%
\pgfpathcurveto{\pgfqpoint{0.877133in}{0.748054in}}{\pgfqpoint{0.866534in}{0.743663in}}{\pgfqpoint{0.858721in}{0.735850in}}%
\pgfpathcurveto{\pgfqpoint{0.850907in}{0.728036in}}{\pgfqpoint{0.846517in}{0.717437in}}{\pgfqpoint{0.846517in}{0.706387in}}%
\pgfpathcurveto{\pgfqpoint{0.846517in}{0.695337in}}{\pgfqpoint{0.850907in}{0.684738in}}{\pgfqpoint{0.858721in}{0.676924in}}%
\pgfpathcurveto{\pgfqpoint{0.866534in}{0.669111in}}{\pgfqpoint{0.877133in}{0.664720in}}{\pgfqpoint{0.888183in}{0.664720in}}%
\pgfpathclose%
\pgfusepath{stroke,fill}%
\end{pgfscope}%
\begin{pgfscope}%
\pgfpathrectangle{\pgfqpoint{0.648703in}{0.548769in}}{\pgfqpoint{5.201297in}{3.102590in}}%
\pgfusepath{clip}%
\pgfsetbuttcap%
\pgfsetroundjoin%
\definecolor{currentfill}{rgb}{1.000000,0.498039,0.054902}%
\pgfsetfillcolor{currentfill}%
\pgfsetlinewidth{1.003750pt}%
\definecolor{currentstroke}{rgb}{1.000000,0.498039,0.054902}%
\pgfsetstrokecolor{currentstroke}%
\pgfsetdash{}{0pt}%
\pgfpathmoveto{\pgfqpoint{1.935796in}{3.145133in}}%
\pgfpathcurveto{\pgfqpoint{1.946846in}{3.145133in}}{\pgfqpoint{1.957445in}{3.149523in}}{\pgfqpoint{1.965259in}{3.157337in}}%
\pgfpathcurveto{\pgfqpoint{1.973073in}{3.165151in}}{\pgfqpoint{1.977463in}{3.175750in}}{\pgfqpoint{1.977463in}{3.186800in}}%
\pgfpathcurveto{\pgfqpoint{1.977463in}{3.197850in}}{\pgfqpoint{1.973073in}{3.208449in}}{\pgfqpoint{1.965259in}{3.216262in}}%
\pgfpathcurveto{\pgfqpoint{1.957445in}{3.224076in}}{\pgfqpoint{1.946846in}{3.228466in}}{\pgfqpoint{1.935796in}{3.228466in}}%
\pgfpathcurveto{\pgfqpoint{1.924746in}{3.228466in}}{\pgfqpoint{1.914147in}{3.224076in}}{\pgfqpoint{1.906333in}{3.216262in}}%
\pgfpathcurveto{\pgfqpoint{1.898520in}{3.208449in}}{\pgfqpoint{1.894130in}{3.197850in}}{\pgfqpoint{1.894130in}{3.186800in}}%
\pgfpathcurveto{\pgfqpoint{1.894130in}{3.175750in}}{\pgfqpoint{1.898520in}{3.165151in}}{\pgfqpoint{1.906333in}{3.157337in}}%
\pgfpathcurveto{\pgfqpoint{1.914147in}{3.149523in}}{\pgfqpoint{1.924746in}{3.145133in}}{\pgfqpoint{1.935796in}{3.145133in}}%
\pgfpathclose%
\pgfusepath{stroke,fill}%
\end{pgfscope}%
\begin{pgfscope}%
\pgfpathrectangle{\pgfqpoint{0.648703in}{0.548769in}}{\pgfqpoint{5.201297in}{3.102590in}}%
\pgfusepath{clip}%
\pgfsetbuttcap%
\pgfsetroundjoin%
\definecolor{currentfill}{rgb}{0.121569,0.466667,0.705882}%
\pgfsetfillcolor{currentfill}%
\pgfsetlinewidth{1.003750pt}%
\definecolor{currentstroke}{rgb}{0.121569,0.466667,0.705882}%
\pgfsetstrokecolor{currentstroke}%
\pgfsetdash{}{0pt}%
\pgfpathmoveto{\pgfqpoint{1.093675in}{0.648129in}}%
\pgfpathcurveto{\pgfqpoint{1.104725in}{0.648129in}}{\pgfqpoint{1.115324in}{0.652519in}}{\pgfqpoint{1.123137in}{0.660333in}}%
\pgfpathcurveto{\pgfqpoint{1.130951in}{0.668146in}}{\pgfqpoint{1.135341in}{0.678745in}}{\pgfqpoint{1.135341in}{0.689796in}}%
\pgfpathcurveto{\pgfqpoint{1.135341in}{0.700846in}}{\pgfqpoint{1.130951in}{0.711445in}}{\pgfqpoint{1.123137in}{0.719258in}}%
\pgfpathcurveto{\pgfqpoint{1.115324in}{0.727072in}}{\pgfqpoint{1.104725in}{0.731462in}}{\pgfqpoint{1.093675in}{0.731462in}}%
\pgfpathcurveto{\pgfqpoint{1.082625in}{0.731462in}}{\pgfqpoint{1.072026in}{0.727072in}}{\pgfqpoint{1.064212in}{0.719258in}}%
\pgfpathcurveto{\pgfqpoint{1.056398in}{0.711445in}}{\pgfqpoint{1.052008in}{0.700846in}}{\pgfqpoint{1.052008in}{0.689796in}}%
\pgfpathcurveto{\pgfqpoint{1.052008in}{0.678745in}}{\pgfqpoint{1.056398in}{0.668146in}}{\pgfqpoint{1.064212in}{0.660333in}}%
\pgfpathcurveto{\pgfqpoint{1.072026in}{0.652519in}}{\pgfqpoint{1.082625in}{0.648129in}}{\pgfqpoint{1.093675in}{0.648129in}}%
\pgfpathclose%
\pgfusepath{stroke,fill}%
\end{pgfscope}%
\begin{pgfscope}%
\pgfpathrectangle{\pgfqpoint{0.648703in}{0.548769in}}{\pgfqpoint{5.201297in}{3.102590in}}%
\pgfusepath{clip}%
\pgfsetbuttcap%
\pgfsetroundjoin%
\definecolor{currentfill}{rgb}{1.000000,0.498039,0.054902}%
\pgfsetfillcolor{currentfill}%
\pgfsetlinewidth{1.003750pt}%
\definecolor{currentstroke}{rgb}{1.000000,0.498039,0.054902}%
\pgfsetstrokecolor{currentstroke}%
\pgfsetdash{}{0pt}%
\pgfpathmoveto{\pgfqpoint{1.875850in}{3.149281in}}%
\pgfpathcurveto{\pgfqpoint{1.886900in}{3.149281in}}{\pgfqpoint{1.897499in}{3.153671in}}{\pgfqpoint{1.905313in}{3.161485in}}%
\pgfpathcurveto{\pgfqpoint{1.913126in}{3.169298in}}{\pgfqpoint{1.917517in}{3.179897in}}{\pgfqpoint{1.917517in}{3.190948in}}%
\pgfpathcurveto{\pgfqpoint{1.917517in}{3.201998in}}{\pgfqpoint{1.913126in}{3.212597in}}{\pgfqpoint{1.905313in}{3.220410in}}%
\pgfpathcurveto{\pgfqpoint{1.897499in}{3.228224in}}{\pgfqpoint{1.886900in}{3.232614in}}{\pgfqpoint{1.875850in}{3.232614in}}%
\pgfpathcurveto{\pgfqpoint{1.864800in}{3.232614in}}{\pgfqpoint{1.854201in}{3.228224in}}{\pgfqpoint{1.846387in}{3.220410in}}%
\pgfpathcurveto{\pgfqpoint{1.838574in}{3.212597in}}{\pgfqpoint{1.834183in}{3.201998in}}{\pgfqpoint{1.834183in}{3.190948in}}%
\pgfpathcurveto{\pgfqpoint{1.834183in}{3.179897in}}{\pgfqpoint{1.838574in}{3.169298in}}{\pgfqpoint{1.846387in}{3.161485in}}%
\pgfpathcurveto{\pgfqpoint{1.854201in}{3.153671in}}{\pgfqpoint{1.864800in}{3.149281in}}{\pgfqpoint{1.875850in}{3.149281in}}%
\pgfpathclose%
\pgfusepath{stroke,fill}%
\end{pgfscope}%
\begin{pgfscope}%
\pgfpathrectangle{\pgfqpoint{0.648703in}{0.548769in}}{\pgfqpoint{5.201297in}{3.102590in}}%
\pgfusepath{clip}%
\pgfsetbuttcap%
\pgfsetroundjoin%
\definecolor{currentfill}{rgb}{0.121569,0.466667,0.705882}%
\pgfsetfillcolor{currentfill}%
\pgfsetlinewidth{1.003750pt}%
\definecolor{currentstroke}{rgb}{0.121569,0.466667,0.705882}%
\pgfsetstrokecolor{currentstroke}%
\pgfsetdash{}{0pt}%
\pgfpathmoveto{\pgfqpoint{2.663034in}{3.124394in}}%
\pgfpathcurveto{\pgfqpoint{2.674084in}{3.124394in}}{\pgfqpoint{2.684683in}{3.128784in}}{\pgfqpoint{2.692497in}{3.136598in}}%
\pgfpathcurveto{\pgfqpoint{2.700310in}{3.144411in}}{\pgfqpoint{2.704700in}{3.155010in}}{\pgfqpoint{2.704700in}{3.166060in}}%
\pgfpathcurveto{\pgfqpoint{2.704700in}{3.177111in}}{\pgfqpoint{2.700310in}{3.187710in}}{\pgfqpoint{2.692497in}{3.195523in}}%
\pgfpathcurveto{\pgfqpoint{2.684683in}{3.203337in}}{\pgfqpoint{2.674084in}{3.207727in}}{\pgfqpoint{2.663034in}{3.207727in}}%
\pgfpathcurveto{\pgfqpoint{2.651984in}{3.207727in}}{\pgfqpoint{2.641385in}{3.203337in}}{\pgfqpoint{2.633571in}{3.195523in}}%
\pgfpathcurveto{\pgfqpoint{2.625757in}{3.187710in}}{\pgfqpoint{2.621367in}{3.177111in}}{\pgfqpoint{2.621367in}{3.166060in}}%
\pgfpathcurveto{\pgfqpoint{2.621367in}{3.155010in}}{\pgfqpoint{2.625757in}{3.144411in}}{\pgfqpoint{2.633571in}{3.136598in}}%
\pgfpathcurveto{\pgfqpoint{2.641385in}{3.128784in}}{\pgfqpoint{2.651984in}{3.124394in}}{\pgfqpoint{2.663034in}{3.124394in}}%
\pgfpathclose%
\pgfusepath{stroke,fill}%
\end{pgfscope}%
\begin{pgfscope}%
\pgfpathrectangle{\pgfqpoint{0.648703in}{0.548769in}}{\pgfqpoint{5.201297in}{3.102590in}}%
\pgfusepath{clip}%
\pgfsetbuttcap%
\pgfsetroundjoin%
\definecolor{currentfill}{rgb}{1.000000,0.498039,0.054902}%
\pgfsetfillcolor{currentfill}%
\pgfsetlinewidth{1.003750pt}%
\definecolor{currentstroke}{rgb}{1.000000,0.498039,0.054902}%
\pgfsetstrokecolor{currentstroke}%
\pgfsetdash{}{0pt}%
\pgfpathmoveto{\pgfqpoint{1.193123in}{3.136837in}}%
\pgfpathcurveto{\pgfqpoint{1.204174in}{3.136837in}}{\pgfqpoint{1.214773in}{3.141228in}}{\pgfqpoint{1.222586in}{3.149041in}}%
\pgfpathcurveto{\pgfqpoint{1.230400in}{3.156855in}}{\pgfqpoint{1.234790in}{3.167454in}}{\pgfqpoint{1.234790in}{3.178504in}}%
\pgfpathcurveto{\pgfqpoint{1.234790in}{3.189554in}}{\pgfqpoint{1.230400in}{3.200153in}}{\pgfqpoint{1.222586in}{3.207967in}}%
\pgfpathcurveto{\pgfqpoint{1.214773in}{3.215780in}}{\pgfqpoint{1.204174in}{3.220171in}}{\pgfqpoint{1.193123in}{3.220171in}}%
\pgfpathcurveto{\pgfqpoint{1.182073in}{3.220171in}}{\pgfqpoint{1.171474in}{3.215780in}}{\pgfqpoint{1.163661in}{3.207967in}}%
\pgfpathcurveto{\pgfqpoint{1.155847in}{3.200153in}}{\pgfqpoint{1.151457in}{3.189554in}}{\pgfqpoint{1.151457in}{3.178504in}}%
\pgfpathcurveto{\pgfqpoint{1.151457in}{3.167454in}}{\pgfqpoint{1.155847in}{3.156855in}}{\pgfqpoint{1.163661in}{3.149041in}}%
\pgfpathcurveto{\pgfqpoint{1.171474in}{3.141228in}}{\pgfqpoint{1.182073in}{3.136837in}}{\pgfqpoint{1.193123in}{3.136837in}}%
\pgfpathclose%
\pgfusepath{stroke,fill}%
\end{pgfscope}%
\begin{pgfscope}%
\pgfpathrectangle{\pgfqpoint{0.648703in}{0.548769in}}{\pgfqpoint{5.201297in}{3.102590in}}%
\pgfusepath{clip}%
\pgfsetbuttcap%
\pgfsetroundjoin%
\definecolor{currentfill}{rgb}{1.000000,0.498039,0.054902}%
\pgfsetfillcolor{currentfill}%
\pgfsetlinewidth{1.003750pt}%
\definecolor{currentstroke}{rgb}{1.000000,0.498039,0.054902}%
\pgfsetstrokecolor{currentstroke}%
\pgfsetdash{}{0pt}%
\pgfpathmoveto{\pgfqpoint{1.818752in}{3.140985in}}%
\pgfpathcurveto{\pgfqpoint{1.829802in}{3.140985in}}{\pgfqpoint{1.840401in}{3.145375in}}{\pgfqpoint{1.848215in}{3.153189in}}%
\pgfpathcurveto{\pgfqpoint{1.856029in}{3.161003in}}{\pgfqpoint{1.860419in}{3.171602in}}{\pgfqpoint{1.860419in}{3.182652in}}%
\pgfpathcurveto{\pgfqpoint{1.860419in}{3.193702in}}{\pgfqpoint{1.856029in}{3.204301in}}{\pgfqpoint{1.848215in}{3.212115in}}%
\pgfpathcurveto{\pgfqpoint{1.840401in}{3.219928in}}{\pgfqpoint{1.829802in}{3.224319in}}{\pgfqpoint{1.818752in}{3.224319in}}%
\pgfpathcurveto{\pgfqpoint{1.807702in}{3.224319in}}{\pgfqpoint{1.797103in}{3.219928in}}{\pgfqpoint{1.789289in}{3.212115in}}%
\pgfpathcurveto{\pgfqpoint{1.781476in}{3.204301in}}{\pgfqpoint{1.777085in}{3.193702in}}{\pgfqpoint{1.777085in}{3.182652in}}%
\pgfpathcurveto{\pgfqpoint{1.777085in}{3.171602in}}{\pgfqpoint{1.781476in}{3.161003in}}{\pgfqpoint{1.789289in}{3.153189in}}%
\pgfpathcurveto{\pgfqpoint{1.797103in}{3.145375in}}{\pgfqpoint{1.807702in}{3.140985in}}{\pgfqpoint{1.818752in}{3.140985in}}%
\pgfpathclose%
\pgfusepath{stroke,fill}%
\end{pgfscope}%
\begin{pgfscope}%
\pgfpathrectangle{\pgfqpoint{0.648703in}{0.548769in}}{\pgfqpoint{5.201297in}{3.102590in}}%
\pgfusepath{clip}%
\pgfsetbuttcap%
\pgfsetroundjoin%
\definecolor{currentfill}{rgb}{1.000000,0.498039,0.054902}%
\pgfsetfillcolor{currentfill}%
\pgfsetlinewidth{1.003750pt}%
\definecolor{currentstroke}{rgb}{1.000000,0.498039,0.054902}%
\pgfsetstrokecolor{currentstroke}%
\pgfsetdash{}{0pt}%
\pgfpathmoveto{\pgfqpoint{1.026664in}{3.145133in}}%
\pgfpathcurveto{\pgfqpoint{1.037714in}{3.145133in}}{\pgfqpoint{1.048313in}{3.149523in}}{\pgfqpoint{1.056127in}{3.157337in}}%
\pgfpathcurveto{\pgfqpoint{1.063941in}{3.165151in}}{\pgfqpoint{1.068331in}{3.175750in}}{\pgfqpoint{1.068331in}{3.186800in}}%
\pgfpathcurveto{\pgfqpoint{1.068331in}{3.197850in}}{\pgfqpoint{1.063941in}{3.208449in}}{\pgfqpoint{1.056127in}{3.216262in}}%
\pgfpathcurveto{\pgfqpoint{1.048313in}{3.224076in}}{\pgfqpoint{1.037714in}{3.228466in}}{\pgfqpoint{1.026664in}{3.228466in}}%
\pgfpathcurveto{\pgfqpoint{1.015614in}{3.228466in}}{\pgfqpoint{1.005015in}{3.224076in}}{\pgfqpoint{0.997201in}{3.216262in}}%
\pgfpathcurveto{\pgfqpoint{0.989388in}{3.208449in}}{\pgfqpoint{0.984998in}{3.197850in}}{\pgfqpoint{0.984998in}{3.186800in}}%
\pgfpathcurveto{\pgfqpoint{0.984998in}{3.175750in}}{\pgfqpoint{0.989388in}{3.165151in}}{\pgfqpoint{0.997201in}{3.157337in}}%
\pgfpathcurveto{\pgfqpoint{1.005015in}{3.149523in}}{\pgfqpoint{1.015614in}{3.145133in}}{\pgfqpoint{1.026664in}{3.145133in}}%
\pgfpathclose%
\pgfusepath{stroke,fill}%
\end{pgfscope}%
\begin{pgfscope}%
\pgfpathrectangle{\pgfqpoint{0.648703in}{0.548769in}}{\pgfqpoint{5.201297in}{3.102590in}}%
\pgfusepath{clip}%
\pgfsetbuttcap%
\pgfsetroundjoin%
\definecolor{currentfill}{rgb}{0.121569,0.466667,0.705882}%
\pgfsetfillcolor{currentfill}%
\pgfsetlinewidth{1.003750pt}%
\definecolor{currentstroke}{rgb}{0.121569,0.466667,0.705882}%
\pgfsetstrokecolor{currentstroke}%
\pgfsetdash{}{0pt}%
\pgfpathmoveto{\pgfqpoint{1.447655in}{3.132690in}}%
\pgfpathcurveto{\pgfqpoint{1.458705in}{3.132690in}}{\pgfqpoint{1.469304in}{3.137080in}}{\pgfqpoint{1.477118in}{3.144893in}}%
\pgfpathcurveto{\pgfqpoint{1.484932in}{3.152707in}}{\pgfqpoint{1.489322in}{3.163306in}}{\pgfqpoint{1.489322in}{3.174356in}}%
\pgfpathcurveto{\pgfqpoint{1.489322in}{3.185406in}}{\pgfqpoint{1.484932in}{3.196005in}}{\pgfqpoint{1.477118in}{3.203819in}}%
\pgfpathcurveto{\pgfqpoint{1.469304in}{3.211633in}}{\pgfqpoint{1.458705in}{3.216023in}}{\pgfqpoint{1.447655in}{3.216023in}}%
\pgfpathcurveto{\pgfqpoint{1.436605in}{3.216023in}}{\pgfqpoint{1.426006in}{3.211633in}}{\pgfqpoint{1.418192in}{3.203819in}}%
\pgfpathcurveto{\pgfqpoint{1.410379in}{3.196005in}}{\pgfqpoint{1.405989in}{3.185406in}}{\pgfqpoint{1.405989in}{3.174356in}}%
\pgfpathcurveto{\pgfqpoint{1.405989in}{3.163306in}}{\pgfqpoint{1.410379in}{3.152707in}}{\pgfqpoint{1.418192in}{3.144893in}}%
\pgfpathcurveto{\pgfqpoint{1.426006in}{3.137080in}}{\pgfqpoint{1.436605in}{3.132690in}}{\pgfqpoint{1.447655in}{3.132690in}}%
\pgfpathclose%
\pgfusepath{stroke,fill}%
\end{pgfscope}%
\begin{pgfscope}%
\pgfpathrectangle{\pgfqpoint{0.648703in}{0.548769in}}{\pgfqpoint{5.201297in}{3.102590in}}%
\pgfusepath{clip}%
\pgfsetbuttcap%
\pgfsetroundjoin%
\definecolor{currentfill}{rgb}{0.121569,0.466667,0.705882}%
\pgfsetfillcolor{currentfill}%
\pgfsetlinewidth{1.003750pt}%
\definecolor{currentstroke}{rgb}{0.121569,0.466667,0.705882}%
\pgfsetstrokecolor{currentstroke}%
\pgfsetdash{}{0pt}%
\pgfpathmoveto{\pgfqpoint{1.347013in}{0.660572in}}%
\pgfpathcurveto{\pgfqpoint{1.358063in}{0.660572in}}{\pgfqpoint{1.368662in}{0.664963in}}{\pgfqpoint{1.376476in}{0.672776in}}%
\pgfpathcurveto{\pgfqpoint{1.384290in}{0.680590in}}{\pgfqpoint{1.388680in}{0.691189in}}{\pgfqpoint{1.388680in}{0.702239in}}%
\pgfpathcurveto{\pgfqpoint{1.388680in}{0.713289in}}{\pgfqpoint{1.384290in}{0.723888in}}{\pgfqpoint{1.376476in}{0.731702in}}%
\pgfpathcurveto{\pgfqpoint{1.368662in}{0.739516in}}{\pgfqpoint{1.358063in}{0.743906in}}{\pgfqpoint{1.347013in}{0.743906in}}%
\pgfpathcurveto{\pgfqpoint{1.335963in}{0.743906in}}{\pgfqpoint{1.325364in}{0.739516in}}{\pgfqpoint{1.317550in}{0.731702in}}%
\pgfpathcurveto{\pgfqpoint{1.309737in}{0.723888in}}{\pgfqpoint{1.305347in}{0.713289in}}{\pgfqpoint{1.305347in}{0.702239in}}%
\pgfpathcurveto{\pgfqpoint{1.305347in}{0.691189in}}{\pgfqpoint{1.309737in}{0.680590in}}{\pgfqpoint{1.317550in}{0.672776in}}%
\pgfpathcurveto{\pgfqpoint{1.325364in}{0.664963in}}{\pgfqpoint{1.335963in}{0.660572in}}{\pgfqpoint{1.347013in}{0.660572in}}%
\pgfpathclose%
\pgfusepath{stroke,fill}%
\end{pgfscope}%
\begin{pgfscope}%
\pgfpathrectangle{\pgfqpoint{0.648703in}{0.548769in}}{\pgfqpoint{5.201297in}{3.102590in}}%
\pgfusepath{clip}%
\pgfsetbuttcap%
\pgfsetroundjoin%
\definecolor{currentfill}{rgb}{1.000000,0.498039,0.054902}%
\pgfsetfillcolor{currentfill}%
\pgfsetlinewidth{1.003750pt}%
\definecolor{currentstroke}{rgb}{1.000000,0.498039,0.054902}%
\pgfsetstrokecolor{currentstroke}%
\pgfsetdash{}{0pt}%
\pgfpathmoveto{\pgfqpoint{2.372492in}{3.136837in}}%
\pgfpathcurveto{\pgfqpoint{2.383543in}{3.136837in}}{\pgfqpoint{2.394142in}{3.141228in}}{\pgfqpoint{2.401955in}{3.149041in}}%
\pgfpathcurveto{\pgfqpoint{2.409769in}{3.156855in}}{\pgfqpoint{2.414159in}{3.167454in}}{\pgfqpoint{2.414159in}{3.178504in}}%
\pgfpathcurveto{\pgfqpoint{2.414159in}{3.189554in}}{\pgfqpoint{2.409769in}{3.200153in}}{\pgfqpoint{2.401955in}{3.207967in}}%
\pgfpathcurveto{\pgfqpoint{2.394142in}{3.215780in}}{\pgfqpoint{2.383543in}{3.220171in}}{\pgfqpoint{2.372492in}{3.220171in}}%
\pgfpathcurveto{\pgfqpoint{2.361442in}{3.220171in}}{\pgfqpoint{2.350843in}{3.215780in}}{\pgfqpoint{2.343030in}{3.207967in}}%
\pgfpathcurveto{\pgfqpoint{2.335216in}{3.200153in}}{\pgfqpoint{2.330826in}{3.189554in}}{\pgfqpoint{2.330826in}{3.178504in}}%
\pgfpathcurveto{\pgfqpoint{2.330826in}{3.167454in}}{\pgfqpoint{2.335216in}{3.156855in}}{\pgfqpoint{2.343030in}{3.149041in}}%
\pgfpathcurveto{\pgfqpoint{2.350843in}{3.141228in}}{\pgfqpoint{2.361442in}{3.136837in}}{\pgfqpoint{2.372492in}{3.136837in}}%
\pgfpathclose%
\pgfusepath{stroke,fill}%
\end{pgfscope}%
\begin{pgfscope}%
\pgfpathrectangle{\pgfqpoint{0.648703in}{0.548769in}}{\pgfqpoint{5.201297in}{3.102590in}}%
\pgfusepath{clip}%
\pgfsetbuttcap%
\pgfsetroundjoin%
\definecolor{currentfill}{rgb}{1.000000,0.498039,0.054902}%
\pgfsetfillcolor{currentfill}%
\pgfsetlinewidth{1.003750pt}%
\definecolor{currentstroke}{rgb}{1.000000,0.498039,0.054902}%
\pgfsetstrokecolor{currentstroke}%
\pgfsetdash{}{0pt}%
\pgfpathmoveto{\pgfqpoint{1.806322in}{3.157577in}}%
\pgfpathcurveto{\pgfqpoint{1.817372in}{3.157577in}}{\pgfqpoint{1.827971in}{3.161967in}}{\pgfqpoint{1.835785in}{3.169780in}}%
\pgfpathcurveto{\pgfqpoint{1.843599in}{3.177594in}}{\pgfqpoint{1.847989in}{3.188193in}}{\pgfqpoint{1.847989in}{3.199243in}}%
\pgfpathcurveto{\pgfqpoint{1.847989in}{3.210293in}}{\pgfqpoint{1.843599in}{3.220892in}}{\pgfqpoint{1.835785in}{3.228706in}}%
\pgfpathcurveto{\pgfqpoint{1.827971in}{3.236520in}}{\pgfqpoint{1.817372in}{3.240910in}}{\pgfqpoint{1.806322in}{3.240910in}}%
\pgfpathcurveto{\pgfqpoint{1.795272in}{3.240910in}}{\pgfqpoint{1.784673in}{3.236520in}}{\pgfqpoint{1.776859in}{3.228706in}}%
\pgfpathcurveto{\pgfqpoint{1.769046in}{3.220892in}}{\pgfqpoint{1.764655in}{3.210293in}}{\pgfqpoint{1.764655in}{3.199243in}}%
\pgfpathcurveto{\pgfqpoint{1.764655in}{3.188193in}}{\pgfqpoint{1.769046in}{3.177594in}}{\pgfqpoint{1.776859in}{3.169780in}}%
\pgfpathcurveto{\pgfqpoint{1.784673in}{3.161967in}}{\pgfqpoint{1.795272in}{3.157577in}}{\pgfqpoint{1.806322in}{3.157577in}}%
\pgfpathclose%
\pgfusepath{stroke,fill}%
\end{pgfscope}%
\begin{pgfscope}%
\pgfpathrectangle{\pgfqpoint{0.648703in}{0.548769in}}{\pgfqpoint{5.201297in}{3.102590in}}%
\pgfusepath{clip}%
\pgfsetbuttcap%
\pgfsetroundjoin%
\definecolor{currentfill}{rgb}{0.121569,0.466667,0.705882}%
\pgfsetfillcolor{currentfill}%
\pgfsetlinewidth{1.003750pt}%
\definecolor{currentstroke}{rgb}{0.121569,0.466667,0.705882}%
\pgfsetstrokecolor{currentstroke}%
\pgfsetdash{}{0pt}%
\pgfpathmoveto{\pgfqpoint{1.294654in}{0.648129in}}%
\pgfpathcurveto{\pgfqpoint{1.305704in}{0.648129in}}{\pgfqpoint{1.316303in}{0.652519in}}{\pgfqpoint{1.324117in}{0.660333in}}%
\pgfpathcurveto{\pgfqpoint{1.331930in}{0.668146in}}{\pgfqpoint{1.336321in}{0.678745in}}{\pgfqpoint{1.336321in}{0.689796in}}%
\pgfpathcurveto{\pgfqpoint{1.336321in}{0.700846in}}{\pgfqpoint{1.331930in}{0.711445in}}{\pgfqpoint{1.324117in}{0.719258in}}%
\pgfpathcurveto{\pgfqpoint{1.316303in}{0.727072in}}{\pgfqpoint{1.305704in}{0.731462in}}{\pgfqpoint{1.294654in}{0.731462in}}%
\pgfpathcurveto{\pgfqpoint{1.283604in}{0.731462in}}{\pgfqpoint{1.273005in}{0.727072in}}{\pgfqpoint{1.265191in}{0.719258in}}%
\pgfpathcurveto{\pgfqpoint{1.257378in}{0.711445in}}{\pgfqpoint{1.252987in}{0.700846in}}{\pgfqpoint{1.252987in}{0.689796in}}%
\pgfpathcurveto{\pgfqpoint{1.252987in}{0.678745in}}{\pgfqpoint{1.257378in}{0.668146in}}{\pgfqpoint{1.265191in}{0.660333in}}%
\pgfpathcurveto{\pgfqpoint{1.273005in}{0.652519in}}{\pgfqpoint{1.283604in}{0.648129in}}{\pgfqpoint{1.294654in}{0.648129in}}%
\pgfpathclose%
\pgfusepath{stroke,fill}%
\end{pgfscope}%
\begin{pgfscope}%
\pgfpathrectangle{\pgfqpoint{0.648703in}{0.548769in}}{\pgfqpoint{5.201297in}{3.102590in}}%
\pgfusepath{clip}%
\pgfsetbuttcap%
\pgfsetroundjoin%
\definecolor{currentfill}{rgb}{0.121569,0.466667,0.705882}%
\pgfsetfillcolor{currentfill}%
\pgfsetlinewidth{1.003750pt}%
\definecolor{currentstroke}{rgb}{0.121569,0.466667,0.705882}%
\pgfsetstrokecolor{currentstroke}%
\pgfsetdash{}{0pt}%
\pgfpathmoveto{\pgfqpoint{0.943077in}{0.648129in}}%
\pgfpathcurveto{\pgfqpoint{0.954128in}{0.648129in}}{\pgfqpoint{0.964727in}{0.652519in}}{\pgfqpoint{0.972540in}{0.660333in}}%
\pgfpathcurveto{\pgfqpoint{0.980354in}{0.668146in}}{\pgfqpoint{0.984744in}{0.678745in}}{\pgfqpoint{0.984744in}{0.689796in}}%
\pgfpathcurveto{\pgfqpoint{0.984744in}{0.700846in}}{\pgfqpoint{0.980354in}{0.711445in}}{\pgfqpoint{0.972540in}{0.719258in}}%
\pgfpathcurveto{\pgfqpoint{0.964727in}{0.727072in}}{\pgfqpoint{0.954128in}{0.731462in}}{\pgfqpoint{0.943077in}{0.731462in}}%
\pgfpathcurveto{\pgfqpoint{0.932027in}{0.731462in}}{\pgfqpoint{0.921428in}{0.727072in}}{\pgfqpoint{0.913615in}{0.719258in}}%
\pgfpathcurveto{\pgfqpoint{0.905801in}{0.711445in}}{\pgfqpoint{0.901411in}{0.700846in}}{\pgfqpoint{0.901411in}{0.689796in}}%
\pgfpathcurveto{\pgfqpoint{0.901411in}{0.678745in}}{\pgfqpoint{0.905801in}{0.668146in}}{\pgfqpoint{0.913615in}{0.660333in}}%
\pgfpathcurveto{\pgfqpoint{0.921428in}{0.652519in}}{\pgfqpoint{0.932027in}{0.648129in}}{\pgfqpoint{0.943077in}{0.648129in}}%
\pgfpathclose%
\pgfusepath{stroke,fill}%
\end{pgfscope}%
\begin{pgfscope}%
\pgfpathrectangle{\pgfqpoint{0.648703in}{0.548769in}}{\pgfqpoint{5.201297in}{3.102590in}}%
\pgfusepath{clip}%
\pgfsetbuttcap%
\pgfsetroundjoin%
\definecolor{currentfill}{rgb}{0.121569,0.466667,0.705882}%
\pgfsetfillcolor{currentfill}%
\pgfsetlinewidth{1.003750pt}%
\definecolor{currentstroke}{rgb}{0.121569,0.466667,0.705882}%
\pgfsetstrokecolor{currentstroke}%
\pgfsetdash{}{0pt}%
\pgfpathmoveto{\pgfqpoint{1.807986in}{3.132690in}}%
\pgfpathcurveto{\pgfqpoint{1.819036in}{3.132690in}}{\pgfqpoint{1.829635in}{3.137080in}}{\pgfqpoint{1.837449in}{3.144893in}}%
\pgfpathcurveto{\pgfqpoint{1.845262in}{3.152707in}}{\pgfqpoint{1.849653in}{3.163306in}}{\pgfqpoint{1.849653in}{3.174356in}}%
\pgfpathcurveto{\pgfqpoint{1.849653in}{3.185406in}}{\pgfqpoint{1.845262in}{3.196005in}}{\pgfqpoint{1.837449in}{3.203819in}}%
\pgfpathcurveto{\pgfqpoint{1.829635in}{3.211633in}}{\pgfqpoint{1.819036in}{3.216023in}}{\pgfqpoint{1.807986in}{3.216023in}}%
\pgfpathcurveto{\pgfqpoint{1.796936in}{3.216023in}}{\pgfqpoint{1.786337in}{3.211633in}}{\pgfqpoint{1.778523in}{3.203819in}}%
\pgfpathcurveto{\pgfqpoint{1.770709in}{3.196005in}}{\pgfqpoint{1.766319in}{3.185406in}}{\pgfqpoint{1.766319in}{3.174356in}}%
\pgfpathcurveto{\pgfqpoint{1.766319in}{3.163306in}}{\pgfqpoint{1.770709in}{3.152707in}}{\pgfqpoint{1.778523in}{3.144893in}}%
\pgfpathcurveto{\pgfqpoint{1.786337in}{3.137080in}}{\pgfqpoint{1.796936in}{3.132690in}}{\pgfqpoint{1.807986in}{3.132690in}}%
\pgfpathclose%
\pgfusepath{stroke,fill}%
\end{pgfscope}%
\begin{pgfscope}%
\pgfpathrectangle{\pgfqpoint{0.648703in}{0.548769in}}{\pgfqpoint{5.201297in}{3.102590in}}%
\pgfusepath{clip}%
\pgfsetbuttcap%
\pgfsetroundjoin%
\definecolor{currentfill}{rgb}{1.000000,0.498039,0.054902}%
\pgfsetfillcolor{currentfill}%
\pgfsetlinewidth{1.003750pt}%
\definecolor{currentstroke}{rgb}{1.000000,0.498039,0.054902}%
\pgfsetstrokecolor{currentstroke}%
\pgfsetdash{}{0pt}%
\pgfpathmoveto{\pgfqpoint{2.974707in}{3.219794in}}%
\pgfpathcurveto{\pgfqpoint{2.985757in}{3.219794in}}{\pgfqpoint{2.996356in}{3.224185in}}{\pgfqpoint{3.004170in}{3.231998in}}%
\pgfpathcurveto{\pgfqpoint{3.011984in}{3.239812in}}{\pgfqpoint{3.016374in}{3.250411in}}{\pgfqpoint{3.016374in}{3.261461in}}%
\pgfpathcurveto{\pgfqpoint{3.016374in}{3.272511in}}{\pgfqpoint{3.011984in}{3.283110in}}{\pgfqpoint{3.004170in}{3.290924in}}%
\pgfpathcurveto{\pgfqpoint{2.996356in}{3.298737in}}{\pgfqpoint{2.985757in}{3.303128in}}{\pgfqpoint{2.974707in}{3.303128in}}%
\pgfpathcurveto{\pgfqpoint{2.963657in}{3.303128in}}{\pgfqpoint{2.953058in}{3.298737in}}{\pgfqpoint{2.945244in}{3.290924in}}%
\pgfpathcurveto{\pgfqpoint{2.937431in}{3.283110in}}{\pgfqpoint{2.933040in}{3.272511in}}{\pgfqpoint{2.933040in}{3.261461in}}%
\pgfpathcurveto{\pgfqpoint{2.933040in}{3.250411in}}{\pgfqpoint{2.937431in}{3.239812in}}{\pgfqpoint{2.945244in}{3.231998in}}%
\pgfpathcurveto{\pgfqpoint{2.953058in}{3.224185in}}{\pgfqpoint{2.963657in}{3.219794in}}{\pgfqpoint{2.974707in}{3.219794in}}%
\pgfpathclose%
\pgfusepath{stroke,fill}%
\end{pgfscope}%
\begin{pgfscope}%
\pgfpathrectangle{\pgfqpoint{0.648703in}{0.548769in}}{\pgfqpoint{5.201297in}{3.102590in}}%
\pgfusepath{clip}%
\pgfsetbuttcap%
\pgfsetroundjoin%
\definecolor{currentfill}{rgb}{1.000000,0.498039,0.054902}%
\pgfsetfillcolor{currentfill}%
\pgfsetlinewidth{1.003750pt}%
\definecolor{currentstroke}{rgb}{1.000000,0.498039,0.054902}%
\pgfsetstrokecolor{currentstroke}%
\pgfsetdash{}{0pt}%
\pgfpathmoveto{\pgfqpoint{1.736524in}{3.140985in}}%
\pgfpathcurveto{\pgfqpoint{1.747574in}{3.140985in}}{\pgfqpoint{1.758173in}{3.145375in}}{\pgfqpoint{1.765987in}{3.153189in}}%
\pgfpathcurveto{\pgfqpoint{1.773801in}{3.161003in}}{\pgfqpoint{1.778191in}{3.171602in}}{\pgfqpoint{1.778191in}{3.182652in}}%
\pgfpathcurveto{\pgfqpoint{1.778191in}{3.193702in}}{\pgfqpoint{1.773801in}{3.204301in}}{\pgfqpoint{1.765987in}{3.212115in}}%
\pgfpathcurveto{\pgfqpoint{1.758173in}{3.219928in}}{\pgfqpoint{1.747574in}{3.224319in}}{\pgfqpoint{1.736524in}{3.224319in}}%
\pgfpathcurveto{\pgfqpoint{1.725474in}{3.224319in}}{\pgfqpoint{1.714875in}{3.219928in}}{\pgfqpoint{1.707061in}{3.212115in}}%
\pgfpathcurveto{\pgfqpoint{1.699248in}{3.204301in}}{\pgfqpoint{1.694858in}{3.193702in}}{\pgfqpoint{1.694858in}{3.182652in}}%
\pgfpathcurveto{\pgfqpoint{1.694858in}{3.171602in}}{\pgfqpoint{1.699248in}{3.161003in}}{\pgfqpoint{1.707061in}{3.153189in}}%
\pgfpathcurveto{\pgfqpoint{1.714875in}{3.145375in}}{\pgfqpoint{1.725474in}{3.140985in}}{\pgfqpoint{1.736524in}{3.140985in}}%
\pgfpathclose%
\pgfusepath{stroke,fill}%
\end{pgfscope}%
\begin{pgfscope}%
\pgfpathrectangle{\pgfqpoint{0.648703in}{0.548769in}}{\pgfqpoint{5.201297in}{3.102590in}}%
\pgfusepath{clip}%
\pgfsetbuttcap%
\pgfsetroundjoin%
\definecolor{currentfill}{rgb}{1.000000,0.498039,0.054902}%
\pgfsetfillcolor{currentfill}%
\pgfsetlinewidth{1.003750pt}%
\definecolor{currentstroke}{rgb}{1.000000,0.498039,0.054902}%
\pgfsetstrokecolor{currentstroke}%
\pgfsetdash{}{0pt}%
\pgfpathmoveto{\pgfqpoint{1.879482in}{3.145133in}}%
\pgfpathcurveto{\pgfqpoint{1.890532in}{3.145133in}}{\pgfqpoint{1.901131in}{3.149523in}}{\pgfqpoint{1.908945in}{3.157337in}}%
\pgfpathcurveto{\pgfqpoint{1.916759in}{3.165151in}}{\pgfqpoint{1.921149in}{3.175750in}}{\pgfqpoint{1.921149in}{3.186800in}}%
\pgfpathcurveto{\pgfqpoint{1.921149in}{3.197850in}}{\pgfqpoint{1.916759in}{3.208449in}}{\pgfqpoint{1.908945in}{3.216262in}}%
\pgfpathcurveto{\pgfqpoint{1.901131in}{3.224076in}}{\pgfqpoint{1.890532in}{3.228466in}}{\pgfqpoint{1.879482in}{3.228466in}}%
\pgfpathcurveto{\pgfqpoint{1.868432in}{3.228466in}}{\pgfqpoint{1.857833in}{3.224076in}}{\pgfqpoint{1.850020in}{3.216262in}}%
\pgfpathcurveto{\pgfqpoint{1.842206in}{3.208449in}}{\pgfqpoint{1.837816in}{3.197850in}}{\pgfqpoint{1.837816in}{3.186800in}}%
\pgfpathcurveto{\pgfqpoint{1.837816in}{3.175750in}}{\pgfqpoint{1.842206in}{3.165151in}}{\pgfqpoint{1.850020in}{3.157337in}}%
\pgfpathcurveto{\pgfqpoint{1.857833in}{3.149523in}}{\pgfqpoint{1.868432in}{3.145133in}}{\pgfqpoint{1.879482in}{3.145133in}}%
\pgfpathclose%
\pgfusepath{stroke,fill}%
\end{pgfscope}%
\begin{pgfscope}%
\pgfpathrectangle{\pgfqpoint{0.648703in}{0.548769in}}{\pgfqpoint{5.201297in}{3.102590in}}%
\pgfusepath{clip}%
\pgfsetbuttcap%
\pgfsetroundjoin%
\definecolor{currentfill}{rgb}{1.000000,0.498039,0.054902}%
\pgfsetfillcolor{currentfill}%
\pgfsetlinewidth{1.003750pt}%
\definecolor{currentstroke}{rgb}{1.000000,0.498039,0.054902}%
\pgfsetstrokecolor{currentstroke}%
\pgfsetdash{}{0pt}%
\pgfpathmoveto{\pgfqpoint{1.579115in}{3.136837in}}%
\pgfpathcurveto{\pgfqpoint{1.590165in}{3.136837in}}{\pgfqpoint{1.600765in}{3.141228in}}{\pgfqpoint{1.608578in}{3.149041in}}%
\pgfpathcurveto{\pgfqpoint{1.616392in}{3.156855in}}{\pgfqpoint{1.620782in}{3.167454in}}{\pgfqpoint{1.620782in}{3.178504in}}%
\pgfpathcurveto{\pgfqpoint{1.620782in}{3.189554in}}{\pgfqpoint{1.616392in}{3.200153in}}{\pgfqpoint{1.608578in}{3.207967in}}%
\pgfpathcurveto{\pgfqpoint{1.600765in}{3.215780in}}{\pgfqpoint{1.590165in}{3.220171in}}{\pgfqpoint{1.579115in}{3.220171in}}%
\pgfpathcurveto{\pgfqpoint{1.568065in}{3.220171in}}{\pgfqpoint{1.557466in}{3.215780in}}{\pgfqpoint{1.549653in}{3.207967in}}%
\pgfpathcurveto{\pgfqpoint{1.541839in}{3.200153in}}{\pgfqpoint{1.537449in}{3.189554in}}{\pgfqpoint{1.537449in}{3.178504in}}%
\pgfpathcurveto{\pgfqpoint{1.537449in}{3.167454in}}{\pgfqpoint{1.541839in}{3.156855in}}{\pgfqpoint{1.549653in}{3.149041in}}%
\pgfpathcurveto{\pgfqpoint{1.557466in}{3.141228in}}{\pgfqpoint{1.568065in}{3.136837in}}{\pgfqpoint{1.579115in}{3.136837in}}%
\pgfpathclose%
\pgfusepath{stroke,fill}%
\end{pgfscope}%
\begin{pgfscope}%
\pgfpathrectangle{\pgfqpoint{0.648703in}{0.548769in}}{\pgfqpoint{5.201297in}{3.102590in}}%
\pgfusepath{clip}%
\pgfsetbuttcap%
\pgfsetroundjoin%
\definecolor{currentfill}{rgb}{0.121569,0.466667,0.705882}%
\pgfsetfillcolor{currentfill}%
\pgfsetlinewidth{1.003750pt}%
\definecolor{currentstroke}{rgb}{0.121569,0.466667,0.705882}%
\pgfsetstrokecolor{currentstroke}%
\pgfsetdash{}{0pt}%
\pgfpathmoveto{\pgfqpoint{1.022413in}{0.648129in}}%
\pgfpathcurveto{\pgfqpoint{1.033464in}{0.648129in}}{\pgfqpoint{1.044063in}{0.652519in}}{\pgfqpoint{1.051876in}{0.660333in}}%
\pgfpathcurveto{\pgfqpoint{1.059690in}{0.668146in}}{\pgfqpoint{1.064080in}{0.678745in}}{\pgfqpoint{1.064080in}{0.689796in}}%
\pgfpathcurveto{\pgfqpoint{1.064080in}{0.700846in}}{\pgfqpoint{1.059690in}{0.711445in}}{\pgfqpoint{1.051876in}{0.719258in}}%
\pgfpathcurveto{\pgfqpoint{1.044063in}{0.727072in}}{\pgfqpoint{1.033464in}{0.731462in}}{\pgfqpoint{1.022413in}{0.731462in}}%
\pgfpathcurveto{\pgfqpoint{1.011363in}{0.731462in}}{\pgfqpoint{1.000764in}{0.727072in}}{\pgfqpoint{0.992951in}{0.719258in}}%
\pgfpathcurveto{\pgfqpoint{0.985137in}{0.711445in}}{\pgfqpoint{0.980747in}{0.700846in}}{\pgfqpoint{0.980747in}{0.689796in}}%
\pgfpathcurveto{\pgfqpoint{0.980747in}{0.678745in}}{\pgfqpoint{0.985137in}{0.668146in}}{\pgfqpoint{0.992951in}{0.660333in}}%
\pgfpathcurveto{\pgfqpoint{1.000764in}{0.652519in}}{\pgfqpoint{1.011363in}{0.648129in}}{\pgfqpoint{1.022413in}{0.648129in}}%
\pgfpathclose%
\pgfusepath{stroke,fill}%
\end{pgfscope}%
\begin{pgfscope}%
\pgfpathrectangle{\pgfqpoint{0.648703in}{0.548769in}}{\pgfqpoint{5.201297in}{3.102590in}}%
\pgfusepath{clip}%
\pgfsetbuttcap%
\pgfsetroundjoin%
\definecolor{currentfill}{rgb}{0.121569,0.466667,0.705882}%
\pgfsetfillcolor{currentfill}%
\pgfsetlinewidth{1.003750pt}%
\definecolor{currentstroke}{rgb}{0.121569,0.466667,0.705882}%
\pgfsetstrokecolor{currentstroke}%
\pgfsetdash{}{0pt}%
\pgfpathmoveto{\pgfqpoint{1.774624in}{3.132690in}}%
\pgfpathcurveto{\pgfqpoint{1.785674in}{3.132690in}}{\pgfqpoint{1.796273in}{3.137080in}}{\pgfqpoint{1.804087in}{3.144893in}}%
\pgfpathcurveto{\pgfqpoint{1.811901in}{3.152707in}}{\pgfqpoint{1.816291in}{3.163306in}}{\pgfqpoint{1.816291in}{3.174356in}}%
\pgfpathcurveto{\pgfqpoint{1.816291in}{3.185406in}}{\pgfqpoint{1.811901in}{3.196005in}}{\pgfqpoint{1.804087in}{3.203819in}}%
\pgfpathcurveto{\pgfqpoint{1.796273in}{3.211633in}}{\pgfqpoint{1.785674in}{3.216023in}}{\pgfqpoint{1.774624in}{3.216023in}}%
\pgfpathcurveto{\pgfqpoint{1.763574in}{3.216023in}}{\pgfqpoint{1.752975in}{3.211633in}}{\pgfqpoint{1.745162in}{3.203819in}}%
\pgfpathcurveto{\pgfqpoint{1.737348in}{3.196005in}}{\pgfqpoint{1.732958in}{3.185406in}}{\pgfqpoint{1.732958in}{3.174356in}}%
\pgfpathcurveto{\pgfqpoint{1.732958in}{3.163306in}}{\pgfqpoint{1.737348in}{3.152707in}}{\pgfqpoint{1.745162in}{3.144893in}}%
\pgfpathcurveto{\pgfqpoint{1.752975in}{3.137080in}}{\pgfqpoint{1.763574in}{3.132690in}}{\pgfqpoint{1.774624in}{3.132690in}}%
\pgfpathclose%
\pgfusepath{stroke,fill}%
\end{pgfscope}%
\begin{pgfscope}%
\pgfpathrectangle{\pgfqpoint{0.648703in}{0.548769in}}{\pgfqpoint{5.201297in}{3.102590in}}%
\pgfusepath{clip}%
\pgfsetbuttcap%
\pgfsetroundjoin%
\definecolor{currentfill}{rgb}{1.000000,0.498039,0.054902}%
\pgfsetfillcolor{currentfill}%
\pgfsetlinewidth{1.003750pt}%
\definecolor{currentstroke}{rgb}{1.000000,0.498039,0.054902}%
\pgfsetstrokecolor{currentstroke}%
\pgfsetdash{}{0pt}%
\pgfpathmoveto{\pgfqpoint{1.764442in}{3.145133in}}%
\pgfpathcurveto{\pgfqpoint{1.775492in}{3.145133in}}{\pgfqpoint{1.786091in}{3.149523in}}{\pgfqpoint{1.793904in}{3.157337in}}%
\pgfpathcurveto{\pgfqpoint{1.801718in}{3.165151in}}{\pgfqpoint{1.806108in}{3.175750in}}{\pgfqpoint{1.806108in}{3.186800in}}%
\pgfpathcurveto{\pgfqpoint{1.806108in}{3.197850in}}{\pgfqpoint{1.801718in}{3.208449in}}{\pgfqpoint{1.793904in}{3.216262in}}%
\pgfpathcurveto{\pgfqpoint{1.786091in}{3.224076in}}{\pgfqpoint{1.775492in}{3.228466in}}{\pgfqpoint{1.764442in}{3.228466in}}%
\pgfpathcurveto{\pgfqpoint{1.753392in}{3.228466in}}{\pgfqpoint{1.742792in}{3.224076in}}{\pgfqpoint{1.734979in}{3.216262in}}%
\pgfpathcurveto{\pgfqpoint{1.727165in}{3.208449in}}{\pgfqpoint{1.722775in}{3.197850in}}{\pgfqpoint{1.722775in}{3.186800in}}%
\pgfpathcurveto{\pgfqpoint{1.722775in}{3.175750in}}{\pgfqpoint{1.727165in}{3.165151in}}{\pgfqpoint{1.734979in}{3.157337in}}%
\pgfpathcurveto{\pgfqpoint{1.742792in}{3.149523in}}{\pgfqpoint{1.753392in}{3.145133in}}{\pgfqpoint{1.764442in}{3.145133in}}%
\pgfpathclose%
\pgfusepath{stroke,fill}%
\end{pgfscope}%
\begin{pgfscope}%
\pgfpathrectangle{\pgfqpoint{0.648703in}{0.548769in}}{\pgfqpoint{5.201297in}{3.102590in}}%
\pgfusepath{clip}%
\pgfsetbuttcap%
\pgfsetroundjoin%
\definecolor{currentfill}{rgb}{0.121569,0.466667,0.705882}%
\pgfsetfillcolor{currentfill}%
\pgfsetlinewidth{1.003750pt}%
\definecolor{currentstroke}{rgb}{0.121569,0.466667,0.705882}%
\pgfsetstrokecolor{currentstroke}%
\pgfsetdash{}{0pt}%
\pgfpathmoveto{\pgfqpoint{1.678207in}{3.132690in}}%
\pgfpathcurveto{\pgfqpoint{1.689257in}{3.132690in}}{\pgfqpoint{1.699856in}{3.137080in}}{\pgfqpoint{1.707670in}{3.144893in}}%
\pgfpathcurveto{\pgfqpoint{1.715483in}{3.152707in}}{\pgfqpoint{1.719874in}{3.163306in}}{\pgfqpoint{1.719874in}{3.174356in}}%
\pgfpathcurveto{\pgfqpoint{1.719874in}{3.185406in}}{\pgfqpoint{1.715483in}{3.196005in}}{\pgfqpoint{1.707670in}{3.203819in}}%
\pgfpathcurveto{\pgfqpoint{1.699856in}{3.211633in}}{\pgfqpoint{1.689257in}{3.216023in}}{\pgfqpoint{1.678207in}{3.216023in}}%
\pgfpathcurveto{\pgfqpoint{1.667157in}{3.216023in}}{\pgfqpoint{1.656558in}{3.211633in}}{\pgfqpoint{1.648744in}{3.203819in}}%
\pgfpathcurveto{\pgfqpoint{1.640931in}{3.196005in}}{\pgfqpoint{1.636540in}{3.185406in}}{\pgfqpoint{1.636540in}{3.174356in}}%
\pgfpathcurveto{\pgfqpoint{1.636540in}{3.163306in}}{\pgfqpoint{1.640931in}{3.152707in}}{\pgfqpoint{1.648744in}{3.144893in}}%
\pgfpathcurveto{\pgfqpoint{1.656558in}{3.137080in}}{\pgfqpoint{1.667157in}{3.132690in}}{\pgfqpoint{1.678207in}{3.132690in}}%
\pgfpathclose%
\pgfusepath{stroke,fill}%
\end{pgfscope}%
\begin{pgfscope}%
\pgfpathrectangle{\pgfqpoint{0.648703in}{0.548769in}}{\pgfqpoint{5.201297in}{3.102590in}}%
\pgfusepath{clip}%
\pgfsetbuttcap%
\pgfsetroundjoin%
\definecolor{currentfill}{rgb}{1.000000,0.498039,0.054902}%
\pgfsetfillcolor{currentfill}%
\pgfsetlinewidth{1.003750pt}%
\definecolor{currentstroke}{rgb}{1.000000,0.498039,0.054902}%
\pgfsetstrokecolor{currentstroke}%
\pgfsetdash{}{0pt}%
\pgfpathmoveto{\pgfqpoint{1.592199in}{3.157577in}}%
\pgfpathcurveto{\pgfqpoint{1.603249in}{3.157577in}}{\pgfqpoint{1.613848in}{3.161967in}}{\pgfqpoint{1.621661in}{3.169780in}}%
\pgfpathcurveto{\pgfqpoint{1.629475in}{3.177594in}}{\pgfqpoint{1.633865in}{3.188193in}}{\pgfqpoint{1.633865in}{3.199243in}}%
\pgfpathcurveto{\pgfqpoint{1.633865in}{3.210293in}}{\pgfqpoint{1.629475in}{3.220892in}}{\pgfqpoint{1.621661in}{3.228706in}}%
\pgfpathcurveto{\pgfqpoint{1.613848in}{3.236520in}}{\pgfqpoint{1.603249in}{3.240910in}}{\pgfqpoint{1.592199in}{3.240910in}}%
\pgfpathcurveto{\pgfqpoint{1.581149in}{3.240910in}}{\pgfqpoint{1.570549in}{3.236520in}}{\pgfqpoint{1.562736in}{3.228706in}}%
\pgfpathcurveto{\pgfqpoint{1.554922in}{3.220892in}}{\pgfqpoint{1.550532in}{3.210293in}}{\pgfqpoint{1.550532in}{3.199243in}}%
\pgfpathcurveto{\pgfqpoint{1.550532in}{3.188193in}}{\pgfqpoint{1.554922in}{3.177594in}}{\pgfqpoint{1.562736in}{3.169780in}}%
\pgfpathcurveto{\pgfqpoint{1.570549in}{3.161967in}}{\pgfqpoint{1.581149in}{3.157577in}}{\pgfqpoint{1.592199in}{3.157577in}}%
\pgfpathclose%
\pgfusepath{stroke,fill}%
\end{pgfscope}%
\begin{pgfscope}%
\pgfpathrectangle{\pgfqpoint{0.648703in}{0.548769in}}{\pgfqpoint{5.201297in}{3.102590in}}%
\pgfusepath{clip}%
\pgfsetbuttcap%
\pgfsetroundjoin%
\definecolor{currentfill}{rgb}{1.000000,0.498039,0.054902}%
\pgfsetfillcolor{currentfill}%
\pgfsetlinewidth{1.003750pt}%
\definecolor{currentstroke}{rgb}{1.000000,0.498039,0.054902}%
\pgfsetstrokecolor{currentstroke}%
\pgfsetdash{}{0pt}%
\pgfpathmoveto{\pgfqpoint{1.315237in}{3.145133in}}%
\pgfpathcurveto{\pgfqpoint{1.326287in}{3.145133in}}{\pgfqpoint{1.336886in}{3.149523in}}{\pgfqpoint{1.344700in}{3.157337in}}%
\pgfpathcurveto{\pgfqpoint{1.352513in}{3.165151in}}{\pgfqpoint{1.356904in}{3.175750in}}{\pgfqpoint{1.356904in}{3.186800in}}%
\pgfpathcurveto{\pgfqpoint{1.356904in}{3.197850in}}{\pgfqpoint{1.352513in}{3.208449in}}{\pgfqpoint{1.344700in}{3.216262in}}%
\pgfpathcurveto{\pgfqpoint{1.336886in}{3.224076in}}{\pgfqpoint{1.326287in}{3.228466in}}{\pgfqpoint{1.315237in}{3.228466in}}%
\pgfpathcurveto{\pgfqpoint{1.304187in}{3.228466in}}{\pgfqpoint{1.293588in}{3.224076in}}{\pgfqpoint{1.285774in}{3.216262in}}%
\pgfpathcurveto{\pgfqpoint{1.277961in}{3.208449in}}{\pgfqpoint{1.273570in}{3.197850in}}{\pgfqpoint{1.273570in}{3.186800in}}%
\pgfpathcurveto{\pgfqpoint{1.273570in}{3.175750in}}{\pgfqpoint{1.277961in}{3.165151in}}{\pgfqpoint{1.285774in}{3.157337in}}%
\pgfpathcurveto{\pgfqpoint{1.293588in}{3.149523in}}{\pgfqpoint{1.304187in}{3.145133in}}{\pgfqpoint{1.315237in}{3.145133in}}%
\pgfpathclose%
\pgfusepath{stroke,fill}%
\end{pgfscope}%
\begin{pgfscope}%
\pgfpathrectangle{\pgfqpoint{0.648703in}{0.548769in}}{\pgfqpoint{5.201297in}{3.102590in}}%
\pgfusepath{clip}%
\pgfsetbuttcap%
\pgfsetroundjoin%
\definecolor{currentfill}{rgb}{1.000000,0.498039,0.054902}%
\pgfsetfillcolor{currentfill}%
\pgfsetlinewidth{1.003750pt}%
\definecolor{currentstroke}{rgb}{1.000000,0.498039,0.054902}%
\pgfsetstrokecolor{currentstroke}%
\pgfsetdash{}{0pt}%
\pgfpathmoveto{\pgfqpoint{2.098919in}{3.190759in}}%
\pgfpathcurveto{\pgfqpoint{2.109969in}{3.190759in}}{\pgfqpoint{2.120568in}{3.195150in}}{\pgfqpoint{2.128382in}{3.202963in}}%
\pgfpathcurveto{\pgfqpoint{2.136196in}{3.210777in}}{\pgfqpoint{2.140586in}{3.221376in}}{\pgfqpoint{2.140586in}{3.232426in}}%
\pgfpathcurveto{\pgfqpoint{2.140586in}{3.243476in}}{\pgfqpoint{2.136196in}{3.254075in}}{\pgfqpoint{2.128382in}{3.261889in}}%
\pgfpathcurveto{\pgfqpoint{2.120568in}{3.269702in}}{\pgfqpoint{2.109969in}{3.274093in}}{\pgfqpoint{2.098919in}{3.274093in}}%
\pgfpathcurveto{\pgfqpoint{2.087869in}{3.274093in}}{\pgfqpoint{2.077270in}{3.269702in}}{\pgfqpoint{2.069456in}{3.261889in}}%
\pgfpathcurveto{\pgfqpoint{2.061643in}{3.254075in}}{\pgfqpoint{2.057253in}{3.243476in}}{\pgfqpoint{2.057253in}{3.232426in}}%
\pgfpathcurveto{\pgfqpoint{2.057253in}{3.221376in}}{\pgfqpoint{2.061643in}{3.210777in}}{\pgfqpoint{2.069456in}{3.202963in}}%
\pgfpathcurveto{\pgfqpoint{2.077270in}{3.195150in}}{\pgfqpoint{2.087869in}{3.190759in}}{\pgfqpoint{2.098919in}{3.190759in}}%
\pgfpathclose%
\pgfusepath{stroke,fill}%
\end{pgfscope}%
\begin{pgfscope}%
\pgfpathrectangle{\pgfqpoint{0.648703in}{0.548769in}}{\pgfqpoint{5.201297in}{3.102590in}}%
\pgfusepath{clip}%
\pgfsetbuttcap%
\pgfsetroundjoin%
\definecolor{currentfill}{rgb}{1.000000,0.498039,0.054902}%
\pgfsetfillcolor{currentfill}%
\pgfsetlinewidth{1.003750pt}%
\definecolor{currentstroke}{rgb}{1.000000,0.498039,0.054902}%
\pgfsetstrokecolor{currentstroke}%
\pgfsetdash{}{0pt}%
\pgfpathmoveto{\pgfqpoint{1.830694in}{3.207351in}}%
\pgfpathcurveto{\pgfqpoint{1.841744in}{3.207351in}}{\pgfqpoint{1.852343in}{3.211741in}}{\pgfqpoint{1.860157in}{3.219555in}}%
\pgfpathcurveto{\pgfqpoint{1.867971in}{3.227368in}}{\pgfqpoint{1.872361in}{3.237967in}}{\pgfqpoint{1.872361in}{3.249017in}}%
\pgfpathcurveto{\pgfqpoint{1.872361in}{3.260068in}}{\pgfqpoint{1.867971in}{3.270667in}}{\pgfqpoint{1.860157in}{3.278480in}}%
\pgfpathcurveto{\pgfqpoint{1.852343in}{3.286294in}}{\pgfqpoint{1.841744in}{3.290684in}}{\pgfqpoint{1.830694in}{3.290684in}}%
\pgfpathcurveto{\pgfqpoint{1.819644in}{3.290684in}}{\pgfqpoint{1.809045in}{3.286294in}}{\pgfqpoint{1.801232in}{3.278480in}}%
\pgfpathcurveto{\pgfqpoint{1.793418in}{3.270667in}}{\pgfqpoint{1.789028in}{3.260068in}}{\pgfqpoint{1.789028in}{3.249017in}}%
\pgfpathcurveto{\pgfqpoint{1.789028in}{3.237967in}}{\pgfqpoint{1.793418in}{3.227368in}}{\pgfqpoint{1.801232in}{3.219555in}}%
\pgfpathcurveto{\pgfqpoint{1.809045in}{3.211741in}}{\pgfqpoint{1.819644in}{3.207351in}}{\pgfqpoint{1.830694in}{3.207351in}}%
\pgfpathclose%
\pgfusepath{stroke,fill}%
\end{pgfscope}%
\begin{pgfscope}%
\pgfpathrectangle{\pgfqpoint{0.648703in}{0.548769in}}{\pgfqpoint{5.201297in}{3.102590in}}%
\pgfusepath{clip}%
\pgfsetbuttcap%
\pgfsetroundjoin%
\definecolor{currentfill}{rgb}{0.121569,0.466667,0.705882}%
\pgfsetfillcolor{currentfill}%
\pgfsetlinewidth{1.003750pt}%
\definecolor{currentstroke}{rgb}{0.121569,0.466667,0.705882}%
\pgfsetstrokecolor{currentstroke}%
\pgfsetdash{}{0pt}%
\pgfpathmoveto{\pgfqpoint{0.969392in}{0.648129in}}%
\pgfpathcurveto{\pgfqpoint{0.980442in}{0.648129in}}{\pgfqpoint{0.991041in}{0.652519in}}{\pgfqpoint{0.998855in}{0.660333in}}%
\pgfpathcurveto{\pgfqpoint{1.006669in}{0.668146in}}{\pgfqpoint{1.011059in}{0.678745in}}{\pgfqpoint{1.011059in}{0.689796in}}%
\pgfpathcurveto{\pgfqpoint{1.011059in}{0.700846in}}{\pgfqpoint{1.006669in}{0.711445in}}{\pgfqpoint{0.998855in}{0.719258in}}%
\pgfpathcurveto{\pgfqpoint{0.991041in}{0.727072in}}{\pgfqpoint{0.980442in}{0.731462in}}{\pgfqpoint{0.969392in}{0.731462in}}%
\pgfpathcurveto{\pgfqpoint{0.958342in}{0.731462in}}{\pgfqpoint{0.947743in}{0.727072in}}{\pgfqpoint{0.939929in}{0.719258in}}%
\pgfpathcurveto{\pgfqpoint{0.932116in}{0.711445in}}{\pgfqpoint{0.927725in}{0.700846in}}{\pgfqpoint{0.927725in}{0.689796in}}%
\pgfpathcurveto{\pgfqpoint{0.927725in}{0.678745in}}{\pgfqpoint{0.932116in}{0.668146in}}{\pgfqpoint{0.939929in}{0.660333in}}%
\pgfpathcurveto{\pgfqpoint{0.947743in}{0.652519in}}{\pgfqpoint{0.958342in}{0.648129in}}{\pgfqpoint{0.969392in}{0.648129in}}%
\pgfpathclose%
\pgfusepath{stroke,fill}%
\end{pgfscope}%
\begin{pgfscope}%
\pgfpathrectangle{\pgfqpoint{0.648703in}{0.548769in}}{\pgfqpoint{5.201297in}{3.102590in}}%
\pgfusepath{clip}%
\pgfsetbuttcap%
\pgfsetroundjoin%
\definecolor{currentfill}{rgb}{0.121569,0.466667,0.705882}%
\pgfsetfillcolor{currentfill}%
\pgfsetlinewidth{1.003750pt}%
\definecolor{currentstroke}{rgb}{0.121569,0.466667,0.705882}%
\pgfsetstrokecolor{currentstroke}%
\pgfsetdash{}{0pt}%
\pgfpathmoveto{\pgfqpoint{0.885126in}{0.656425in}}%
\pgfpathcurveto{\pgfqpoint{0.896176in}{0.656425in}}{\pgfqpoint{0.906775in}{0.660815in}}{\pgfqpoint{0.914589in}{0.668629in}}%
\pgfpathcurveto{\pgfqpoint{0.922402in}{0.676442in}}{\pgfqpoint{0.926793in}{0.687041in}}{\pgfqpoint{0.926793in}{0.698091in}}%
\pgfpathcurveto{\pgfqpoint{0.926793in}{0.709141in}}{\pgfqpoint{0.922402in}{0.719740in}}{\pgfqpoint{0.914589in}{0.727554in}}%
\pgfpathcurveto{\pgfqpoint{0.906775in}{0.735368in}}{\pgfqpoint{0.896176in}{0.739758in}}{\pgfqpoint{0.885126in}{0.739758in}}%
\pgfpathcurveto{\pgfqpoint{0.874076in}{0.739758in}}{\pgfqpoint{0.863477in}{0.735368in}}{\pgfqpoint{0.855663in}{0.727554in}}%
\pgfpathcurveto{\pgfqpoint{0.847850in}{0.719740in}}{\pgfqpoint{0.843459in}{0.709141in}}{\pgfqpoint{0.843459in}{0.698091in}}%
\pgfpathcurveto{\pgfqpoint{0.843459in}{0.687041in}}{\pgfqpoint{0.847850in}{0.676442in}}{\pgfqpoint{0.855663in}{0.668629in}}%
\pgfpathcurveto{\pgfqpoint{0.863477in}{0.660815in}}{\pgfqpoint{0.874076in}{0.656425in}}{\pgfqpoint{0.885126in}{0.656425in}}%
\pgfpathclose%
\pgfusepath{stroke,fill}%
\end{pgfscope}%
\begin{pgfscope}%
\pgfpathrectangle{\pgfqpoint{0.648703in}{0.548769in}}{\pgfqpoint{5.201297in}{3.102590in}}%
\pgfusepath{clip}%
\pgfsetbuttcap%
\pgfsetroundjoin%
\definecolor{currentfill}{rgb}{0.121569,0.466667,0.705882}%
\pgfsetfillcolor{currentfill}%
\pgfsetlinewidth{1.003750pt}%
\definecolor{currentstroke}{rgb}{0.121569,0.466667,0.705882}%
\pgfsetstrokecolor{currentstroke}%
\pgfsetdash{}{0pt}%
\pgfpathmoveto{\pgfqpoint{1.742012in}{3.099507in}}%
\pgfpathcurveto{\pgfqpoint{1.753062in}{3.099507in}}{\pgfqpoint{1.763661in}{3.103897in}}{\pgfqpoint{1.771475in}{3.111711in}}%
\pgfpathcurveto{\pgfqpoint{1.779288in}{3.119524in}}{\pgfqpoint{1.783679in}{3.130123in}}{\pgfqpoint{1.783679in}{3.141173in}}%
\pgfpathcurveto{\pgfqpoint{1.783679in}{3.152224in}}{\pgfqpoint{1.779288in}{3.162823in}}{\pgfqpoint{1.771475in}{3.170636in}}%
\pgfpathcurveto{\pgfqpoint{1.763661in}{3.178450in}}{\pgfqpoint{1.753062in}{3.182840in}}{\pgfqpoint{1.742012in}{3.182840in}}%
\pgfpathcurveto{\pgfqpoint{1.730962in}{3.182840in}}{\pgfqpoint{1.720363in}{3.178450in}}{\pgfqpoint{1.712549in}{3.170636in}}%
\pgfpathcurveto{\pgfqpoint{1.704736in}{3.162823in}}{\pgfqpoint{1.700345in}{3.152224in}}{\pgfqpoint{1.700345in}{3.141173in}}%
\pgfpathcurveto{\pgfqpoint{1.700345in}{3.130123in}}{\pgfqpoint{1.704736in}{3.119524in}}{\pgfqpoint{1.712549in}{3.111711in}}%
\pgfpathcurveto{\pgfqpoint{1.720363in}{3.103897in}}{\pgfqpoint{1.730962in}{3.099507in}}{\pgfqpoint{1.742012in}{3.099507in}}%
\pgfpathclose%
\pgfusepath{stroke,fill}%
\end{pgfscope}%
\begin{pgfscope}%
\pgfpathrectangle{\pgfqpoint{0.648703in}{0.548769in}}{\pgfqpoint{5.201297in}{3.102590in}}%
\pgfusepath{clip}%
\pgfsetbuttcap%
\pgfsetroundjoin%
\definecolor{currentfill}{rgb}{1.000000,0.498039,0.054902}%
\pgfsetfillcolor{currentfill}%
\pgfsetlinewidth{1.003750pt}%
\definecolor{currentstroke}{rgb}{1.000000,0.498039,0.054902}%
\pgfsetstrokecolor{currentstroke}%
\pgfsetdash{}{0pt}%
\pgfpathmoveto{\pgfqpoint{1.704321in}{3.136837in}}%
\pgfpathcurveto{\pgfqpoint{1.715371in}{3.136837in}}{\pgfqpoint{1.725970in}{3.141228in}}{\pgfqpoint{1.733784in}{3.149041in}}%
\pgfpathcurveto{\pgfqpoint{1.741598in}{3.156855in}}{\pgfqpoint{1.745988in}{3.167454in}}{\pgfqpoint{1.745988in}{3.178504in}}%
\pgfpathcurveto{\pgfqpoint{1.745988in}{3.189554in}}{\pgfqpoint{1.741598in}{3.200153in}}{\pgfqpoint{1.733784in}{3.207967in}}%
\pgfpathcurveto{\pgfqpoint{1.725970in}{3.215780in}}{\pgfqpoint{1.715371in}{3.220171in}}{\pgfqpoint{1.704321in}{3.220171in}}%
\pgfpathcurveto{\pgfqpoint{1.693271in}{3.220171in}}{\pgfqpoint{1.682672in}{3.215780in}}{\pgfqpoint{1.674858in}{3.207967in}}%
\pgfpathcurveto{\pgfqpoint{1.667045in}{3.200153in}}{\pgfqpoint{1.662655in}{3.189554in}}{\pgfqpoint{1.662655in}{3.178504in}}%
\pgfpathcurveto{\pgfqpoint{1.662655in}{3.167454in}}{\pgfqpoint{1.667045in}{3.156855in}}{\pgfqpoint{1.674858in}{3.149041in}}%
\pgfpathcurveto{\pgfqpoint{1.682672in}{3.141228in}}{\pgfqpoint{1.693271in}{3.136837in}}{\pgfqpoint{1.704321in}{3.136837in}}%
\pgfpathclose%
\pgfusepath{stroke,fill}%
\end{pgfscope}%
\begin{pgfscope}%
\pgfpathrectangle{\pgfqpoint{0.648703in}{0.548769in}}{\pgfqpoint{5.201297in}{3.102590in}}%
\pgfusepath{clip}%
\pgfsetbuttcap%
\pgfsetroundjoin%
\definecolor{currentfill}{rgb}{0.121569,0.466667,0.705882}%
\pgfsetfillcolor{currentfill}%
\pgfsetlinewidth{1.003750pt}%
\definecolor{currentstroke}{rgb}{0.121569,0.466667,0.705882}%
\pgfsetstrokecolor{currentstroke}%
\pgfsetdash{}{0pt}%
\pgfpathmoveto{\pgfqpoint{1.222426in}{0.648129in}}%
\pgfpathcurveto{\pgfqpoint{1.233476in}{0.648129in}}{\pgfqpoint{1.244075in}{0.652519in}}{\pgfqpoint{1.251889in}{0.660333in}}%
\pgfpathcurveto{\pgfqpoint{1.259702in}{0.668146in}}{\pgfqpoint{1.264092in}{0.678745in}}{\pgfqpoint{1.264092in}{0.689796in}}%
\pgfpathcurveto{\pgfqpoint{1.264092in}{0.700846in}}{\pgfqpoint{1.259702in}{0.711445in}}{\pgfqpoint{1.251889in}{0.719258in}}%
\pgfpathcurveto{\pgfqpoint{1.244075in}{0.727072in}}{\pgfqpoint{1.233476in}{0.731462in}}{\pgfqpoint{1.222426in}{0.731462in}}%
\pgfpathcurveto{\pgfqpoint{1.211376in}{0.731462in}}{\pgfqpoint{1.200777in}{0.727072in}}{\pgfqpoint{1.192963in}{0.719258in}}%
\pgfpathcurveto{\pgfqpoint{1.185149in}{0.711445in}}{\pgfqpoint{1.180759in}{0.700846in}}{\pgfqpoint{1.180759in}{0.689796in}}%
\pgfpathcurveto{\pgfqpoint{1.180759in}{0.678745in}}{\pgfqpoint{1.185149in}{0.668146in}}{\pgfqpoint{1.192963in}{0.660333in}}%
\pgfpathcurveto{\pgfqpoint{1.200777in}{0.652519in}}{\pgfqpoint{1.211376in}{0.648129in}}{\pgfqpoint{1.222426in}{0.648129in}}%
\pgfpathclose%
\pgfusepath{stroke,fill}%
\end{pgfscope}%
\begin{pgfscope}%
\pgfpathrectangle{\pgfqpoint{0.648703in}{0.548769in}}{\pgfqpoint{5.201297in}{3.102590in}}%
\pgfusepath{clip}%
\pgfsetbuttcap%
\pgfsetroundjoin%
\definecolor{currentfill}{rgb}{0.121569,0.466667,0.705882}%
\pgfsetfillcolor{currentfill}%
\pgfsetlinewidth{1.003750pt}%
\definecolor{currentstroke}{rgb}{0.121569,0.466667,0.705882}%
\pgfsetstrokecolor{currentstroke}%
\pgfsetdash{}{0pt}%
\pgfpathmoveto{\pgfqpoint{0.898349in}{0.648129in}}%
\pgfpathcurveto{\pgfqpoint{0.909399in}{0.648129in}}{\pgfqpoint{0.919998in}{0.652519in}}{\pgfqpoint{0.927811in}{0.660333in}}%
\pgfpathcurveto{\pgfqpoint{0.935625in}{0.668146in}}{\pgfqpoint{0.940015in}{0.678745in}}{\pgfqpoint{0.940015in}{0.689796in}}%
\pgfpathcurveto{\pgfqpoint{0.940015in}{0.700846in}}{\pgfqpoint{0.935625in}{0.711445in}}{\pgfqpoint{0.927811in}{0.719258in}}%
\pgfpathcurveto{\pgfqpoint{0.919998in}{0.727072in}}{\pgfqpoint{0.909399in}{0.731462in}}{\pgfqpoint{0.898349in}{0.731462in}}%
\pgfpathcurveto{\pgfqpoint{0.887299in}{0.731462in}}{\pgfqpoint{0.876699in}{0.727072in}}{\pgfqpoint{0.868886in}{0.719258in}}%
\pgfpathcurveto{\pgfqpoint{0.861072in}{0.711445in}}{\pgfqpoint{0.856682in}{0.700846in}}{\pgfqpoint{0.856682in}{0.689796in}}%
\pgfpathcurveto{\pgfqpoint{0.856682in}{0.678745in}}{\pgfqpoint{0.861072in}{0.668146in}}{\pgfqpoint{0.868886in}{0.660333in}}%
\pgfpathcurveto{\pgfqpoint{0.876699in}{0.652519in}}{\pgfqpoint{0.887299in}{0.648129in}}{\pgfqpoint{0.898349in}{0.648129in}}%
\pgfpathclose%
\pgfusepath{stroke,fill}%
\end{pgfscope}%
\begin{pgfscope}%
\pgfpathrectangle{\pgfqpoint{0.648703in}{0.548769in}}{\pgfqpoint{5.201297in}{3.102590in}}%
\pgfusepath{clip}%
\pgfsetbuttcap%
\pgfsetroundjoin%
\definecolor{currentfill}{rgb}{0.839216,0.152941,0.156863}%
\pgfsetfillcolor{currentfill}%
\pgfsetlinewidth{1.003750pt}%
\definecolor{currentstroke}{rgb}{0.839216,0.152941,0.156863}%
\pgfsetstrokecolor{currentstroke}%
\pgfsetdash{}{0pt}%
\pgfpathmoveto{\pgfqpoint{1.216041in}{3.410595in}}%
\pgfpathcurveto{\pgfqpoint{1.227091in}{3.410595in}}{\pgfqpoint{1.237690in}{3.414986in}}{\pgfqpoint{1.245504in}{3.422799in}}%
\pgfpathcurveto{\pgfqpoint{1.253317in}{3.430613in}}{\pgfqpoint{1.257708in}{3.441212in}}{\pgfqpoint{1.257708in}{3.452262in}}%
\pgfpathcurveto{\pgfqpoint{1.257708in}{3.463312in}}{\pgfqpoint{1.253317in}{3.473911in}}{\pgfqpoint{1.245504in}{3.481725in}}%
\pgfpathcurveto{\pgfqpoint{1.237690in}{3.489538in}}{\pgfqpoint{1.227091in}{3.493929in}}{\pgfqpoint{1.216041in}{3.493929in}}%
\pgfpathcurveto{\pgfqpoint{1.204991in}{3.493929in}}{\pgfqpoint{1.194392in}{3.489538in}}{\pgfqpoint{1.186578in}{3.481725in}}%
\pgfpathcurveto{\pgfqpoint{1.178765in}{3.473911in}}{\pgfqpoint{1.174374in}{3.463312in}}{\pgfqpoint{1.174374in}{3.452262in}}%
\pgfpathcurveto{\pgfqpoint{1.174374in}{3.441212in}}{\pgfqpoint{1.178765in}{3.430613in}}{\pgfqpoint{1.186578in}{3.422799in}}%
\pgfpathcurveto{\pgfqpoint{1.194392in}{3.414986in}}{\pgfqpoint{1.204991in}{3.410595in}}{\pgfqpoint{1.216041in}{3.410595in}}%
\pgfpathclose%
\pgfusepath{stroke,fill}%
\end{pgfscope}%
\begin{pgfscope}%
\pgfpathrectangle{\pgfqpoint{0.648703in}{0.548769in}}{\pgfqpoint{5.201297in}{3.102590in}}%
\pgfusepath{clip}%
\pgfsetbuttcap%
\pgfsetroundjoin%
\definecolor{currentfill}{rgb}{0.121569,0.466667,0.705882}%
\pgfsetfillcolor{currentfill}%
\pgfsetlinewidth{1.003750pt}%
\definecolor{currentstroke}{rgb}{0.121569,0.466667,0.705882}%
\pgfsetstrokecolor{currentstroke}%
\pgfsetdash{}{0pt}%
\pgfpathmoveto{\pgfqpoint{2.213045in}{3.128542in}}%
\pgfpathcurveto{\pgfqpoint{2.224095in}{3.128542in}}{\pgfqpoint{2.234694in}{3.132932in}}{\pgfqpoint{2.242508in}{3.140746in}}%
\pgfpathcurveto{\pgfqpoint{2.250322in}{3.148559in}}{\pgfqpoint{2.254712in}{3.159158in}}{\pgfqpoint{2.254712in}{3.170208in}}%
\pgfpathcurveto{\pgfqpoint{2.254712in}{3.181258in}}{\pgfqpoint{2.250322in}{3.191857in}}{\pgfqpoint{2.242508in}{3.199671in}}%
\pgfpathcurveto{\pgfqpoint{2.234694in}{3.207485in}}{\pgfqpoint{2.224095in}{3.211875in}}{\pgfqpoint{2.213045in}{3.211875in}}%
\pgfpathcurveto{\pgfqpoint{2.201995in}{3.211875in}}{\pgfqpoint{2.191396in}{3.207485in}}{\pgfqpoint{2.183582in}{3.199671in}}%
\pgfpathcurveto{\pgfqpoint{2.175769in}{3.191857in}}{\pgfqpoint{2.171379in}{3.181258in}}{\pgfqpoint{2.171379in}{3.170208in}}%
\pgfpathcurveto{\pgfqpoint{2.171379in}{3.159158in}}{\pgfqpoint{2.175769in}{3.148559in}}{\pgfqpoint{2.183582in}{3.140746in}}%
\pgfpathcurveto{\pgfqpoint{2.191396in}{3.132932in}}{\pgfqpoint{2.201995in}{3.128542in}}{\pgfqpoint{2.213045in}{3.128542in}}%
\pgfpathclose%
\pgfusepath{stroke,fill}%
\end{pgfscope}%
\begin{pgfscope}%
\pgfpathrectangle{\pgfqpoint{0.648703in}{0.548769in}}{\pgfqpoint{5.201297in}{3.102590in}}%
\pgfusepath{clip}%
\pgfsetbuttcap%
\pgfsetroundjoin%
\definecolor{currentfill}{rgb}{1.000000,0.498039,0.054902}%
\pgfsetfillcolor{currentfill}%
\pgfsetlinewidth{1.003750pt}%
\definecolor{currentstroke}{rgb}{1.000000,0.498039,0.054902}%
\pgfsetstrokecolor{currentstroke}%
\pgfsetdash{}{0pt}%
\pgfpathmoveto{\pgfqpoint{2.716691in}{3.315195in}}%
\pgfpathcurveto{\pgfqpoint{2.727741in}{3.315195in}}{\pgfqpoint{2.738340in}{3.319585in}}{\pgfqpoint{2.746154in}{3.327399in}}%
\pgfpathcurveto{\pgfqpoint{2.753967in}{3.335212in}}{\pgfqpoint{2.758358in}{3.345811in}}{\pgfqpoint{2.758358in}{3.356861in}}%
\pgfpathcurveto{\pgfqpoint{2.758358in}{3.367912in}}{\pgfqpoint{2.753967in}{3.378511in}}{\pgfqpoint{2.746154in}{3.386324in}}%
\pgfpathcurveto{\pgfqpoint{2.738340in}{3.394138in}}{\pgfqpoint{2.727741in}{3.398528in}}{\pgfqpoint{2.716691in}{3.398528in}}%
\pgfpathcurveto{\pgfqpoint{2.705641in}{3.398528in}}{\pgfqpoint{2.695042in}{3.394138in}}{\pgfqpoint{2.687228in}{3.386324in}}%
\pgfpathcurveto{\pgfqpoint{2.679415in}{3.378511in}}{\pgfqpoint{2.675024in}{3.367912in}}{\pgfqpoint{2.675024in}{3.356861in}}%
\pgfpathcurveto{\pgfqpoint{2.675024in}{3.345811in}}{\pgfqpoint{2.679415in}{3.335212in}}{\pgfqpoint{2.687228in}{3.327399in}}%
\pgfpathcurveto{\pgfqpoint{2.695042in}{3.319585in}}{\pgfqpoint{2.705641in}{3.315195in}}{\pgfqpoint{2.716691in}{3.315195in}}%
\pgfpathclose%
\pgfusepath{stroke,fill}%
\end{pgfscope}%
\begin{pgfscope}%
\pgfpathrectangle{\pgfqpoint{0.648703in}{0.548769in}}{\pgfqpoint{5.201297in}{3.102590in}}%
\pgfusepath{clip}%
\pgfsetbuttcap%
\pgfsetroundjoin%
\definecolor{currentfill}{rgb}{1.000000,0.498039,0.054902}%
\pgfsetfillcolor{currentfill}%
\pgfsetlinewidth{1.003750pt}%
\definecolor{currentstroke}{rgb}{1.000000,0.498039,0.054902}%
\pgfsetstrokecolor{currentstroke}%
\pgfsetdash{}{0pt}%
\pgfpathmoveto{\pgfqpoint{1.832497in}{3.136837in}}%
\pgfpathcurveto{\pgfqpoint{1.843548in}{3.136837in}}{\pgfqpoint{1.854147in}{3.141228in}}{\pgfqpoint{1.861960in}{3.149041in}}%
\pgfpathcurveto{\pgfqpoint{1.869774in}{3.156855in}}{\pgfqpoint{1.874164in}{3.167454in}}{\pgfqpoint{1.874164in}{3.178504in}}%
\pgfpathcurveto{\pgfqpoint{1.874164in}{3.189554in}}{\pgfqpoint{1.869774in}{3.200153in}}{\pgfqpoint{1.861960in}{3.207967in}}%
\pgfpathcurveto{\pgfqpoint{1.854147in}{3.215780in}}{\pgfqpoint{1.843548in}{3.220171in}}{\pgfqpoint{1.832497in}{3.220171in}}%
\pgfpathcurveto{\pgfqpoint{1.821447in}{3.220171in}}{\pgfqpoint{1.810848in}{3.215780in}}{\pgfqpoint{1.803035in}{3.207967in}}%
\pgfpathcurveto{\pgfqpoint{1.795221in}{3.200153in}}{\pgfqpoint{1.790831in}{3.189554in}}{\pgfqpoint{1.790831in}{3.178504in}}%
\pgfpathcurveto{\pgfqpoint{1.790831in}{3.167454in}}{\pgfqpoint{1.795221in}{3.156855in}}{\pgfqpoint{1.803035in}{3.149041in}}%
\pgfpathcurveto{\pgfqpoint{1.810848in}{3.141228in}}{\pgfqpoint{1.821447in}{3.136837in}}{\pgfqpoint{1.832497in}{3.136837in}}%
\pgfpathclose%
\pgfusepath{stroke,fill}%
\end{pgfscope}%
\begin{pgfscope}%
\pgfpathrectangle{\pgfqpoint{0.648703in}{0.548769in}}{\pgfqpoint{5.201297in}{3.102590in}}%
\pgfusepath{clip}%
\pgfsetbuttcap%
\pgfsetroundjoin%
\definecolor{currentfill}{rgb}{1.000000,0.498039,0.054902}%
\pgfsetfillcolor{currentfill}%
\pgfsetlinewidth{1.003750pt}%
\definecolor{currentstroke}{rgb}{1.000000,0.498039,0.054902}%
\pgfsetstrokecolor{currentstroke}%
\pgfsetdash{}{0pt}%
\pgfpathmoveto{\pgfqpoint{1.805852in}{3.136837in}}%
\pgfpathcurveto{\pgfqpoint{1.816902in}{3.136837in}}{\pgfqpoint{1.827501in}{3.141228in}}{\pgfqpoint{1.835315in}{3.149041in}}%
\pgfpathcurveto{\pgfqpoint{1.843128in}{3.156855in}}{\pgfqpoint{1.847518in}{3.167454in}}{\pgfqpoint{1.847518in}{3.178504in}}%
\pgfpathcurveto{\pgfqpoint{1.847518in}{3.189554in}}{\pgfqpoint{1.843128in}{3.200153in}}{\pgfqpoint{1.835315in}{3.207967in}}%
\pgfpathcurveto{\pgfqpoint{1.827501in}{3.215780in}}{\pgfqpoint{1.816902in}{3.220171in}}{\pgfqpoint{1.805852in}{3.220171in}}%
\pgfpathcurveto{\pgfqpoint{1.794802in}{3.220171in}}{\pgfqpoint{1.784203in}{3.215780in}}{\pgfqpoint{1.776389in}{3.207967in}}%
\pgfpathcurveto{\pgfqpoint{1.768575in}{3.200153in}}{\pgfqpoint{1.764185in}{3.189554in}}{\pgfqpoint{1.764185in}{3.178504in}}%
\pgfpathcurveto{\pgfqpoint{1.764185in}{3.167454in}}{\pgfqpoint{1.768575in}{3.156855in}}{\pgfqpoint{1.776389in}{3.149041in}}%
\pgfpathcurveto{\pgfqpoint{1.784203in}{3.141228in}}{\pgfqpoint{1.794802in}{3.136837in}}{\pgfqpoint{1.805852in}{3.136837in}}%
\pgfpathclose%
\pgfusepath{stroke,fill}%
\end{pgfscope}%
\begin{pgfscope}%
\pgfpathrectangle{\pgfqpoint{0.648703in}{0.548769in}}{\pgfqpoint{5.201297in}{3.102590in}}%
\pgfusepath{clip}%
\pgfsetbuttcap%
\pgfsetroundjoin%
\definecolor{currentfill}{rgb}{1.000000,0.498039,0.054902}%
\pgfsetfillcolor{currentfill}%
\pgfsetlinewidth{1.003750pt}%
\definecolor{currentstroke}{rgb}{1.000000,0.498039,0.054902}%
\pgfsetstrokecolor{currentstroke}%
\pgfsetdash{}{0pt}%
\pgfpathmoveto{\pgfqpoint{1.494788in}{3.140985in}}%
\pgfpathcurveto{\pgfqpoint{1.505838in}{3.140985in}}{\pgfqpoint{1.516437in}{3.145375in}}{\pgfqpoint{1.524251in}{3.153189in}}%
\pgfpathcurveto{\pgfqpoint{1.532065in}{3.161003in}}{\pgfqpoint{1.536455in}{3.171602in}}{\pgfqpoint{1.536455in}{3.182652in}}%
\pgfpathcurveto{\pgfqpoint{1.536455in}{3.193702in}}{\pgfqpoint{1.532065in}{3.204301in}}{\pgfqpoint{1.524251in}{3.212115in}}%
\pgfpathcurveto{\pgfqpoint{1.516437in}{3.219928in}}{\pgfqpoint{1.505838in}{3.224319in}}{\pgfqpoint{1.494788in}{3.224319in}}%
\pgfpathcurveto{\pgfqpoint{1.483738in}{3.224319in}}{\pgfqpoint{1.473139in}{3.219928in}}{\pgfqpoint{1.465325in}{3.212115in}}%
\pgfpathcurveto{\pgfqpoint{1.457512in}{3.204301in}}{\pgfqpoint{1.453122in}{3.193702in}}{\pgfqpoint{1.453122in}{3.182652in}}%
\pgfpathcurveto{\pgfqpoint{1.453122in}{3.171602in}}{\pgfqpoint{1.457512in}{3.161003in}}{\pgfqpoint{1.465325in}{3.153189in}}%
\pgfpathcurveto{\pgfqpoint{1.473139in}{3.145375in}}{\pgfqpoint{1.483738in}{3.140985in}}{\pgfqpoint{1.494788in}{3.140985in}}%
\pgfpathclose%
\pgfusepath{stroke,fill}%
\end{pgfscope}%
\begin{pgfscope}%
\pgfpathrectangle{\pgfqpoint{0.648703in}{0.548769in}}{\pgfqpoint{5.201297in}{3.102590in}}%
\pgfusepath{clip}%
\pgfsetbuttcap%
\pgfsetroundjoin%
\definecolor{currentfill}{rgb}{1.000000,0.498039,0.054902}%
\pgfsetfillcolor{currentfill}%
\pgfsetlinewidth{1.003750pt}%
\definecolor{currentstroke}{rgb}{1.000000,0.498039,0.054902}%
\pgfsetstrokecolor{currentstroke}%
\pgfsetdash{}{0pt}%
\pgfpathmoveto{\pgfqpoint{1.370680in}{3.136837in}}%
\pgfpathcurveto{\pgfqpoint{1.381730in}{3.136837in}}{\pgfqpoint{1.392329in}{3.141228in}}{\pgfqpoint{1.400143in}{3.149041in}}%
\pgfpathcurveto{\pgfqpoint{1.407956in}{3.156855in}}{\pgfqpoint{1.412347in}{3.167454in}}{\pgfqpoint{1.412347in}{3.178504in}}%
\pgfpathcurveto{\pgfqpoint{1.412347in}{3.189554in}}{\pgfqpoint{1.407956in}{3.200153in}}{\pgfqpoint{1.400143in}{3.207967in}}%
\pgfpathcurveto{\pgfqpoint{1.392329in}{3.215780in}}{\pgfqpoint{1.381730in}{3.220171in}}{\pgfqpoint{1.370680in}{3.220171in}}%
\pgfpathcurveto{\pgfqpoint{1.359630in}{3.220171in}}{\pgfqpoint{1.349031in}{3.215780in}}{\pgfqpoint{1.341217in}{3.207967in}}%
\pgfpathcurveto{\pgfqpoint{1.333403in}{3.200153in}}{\pgfqpoint{1.329013in}{3.189554in}}{\pgfqpoint{1.329013in}{3.178504in}}%
\pgfpathcurveto{\pgfqpoint{1.329013in}{3.167454in}}{\pgfqpoint{1.333403in}{3.156855in}}{\pgfqpoint{1.341217in}{3.149041in}}%
\pgfpathcurveto{\pgfqpoint{1.349031in}{3.141228in}}{\pgfqpoint{1.359630in}{3.136837in}}{\pgfqpoint{1.370680in}{3.136837in}}%
\pgfpathclose%
\pgfusepath{stroke,fill}%
\end{pgfscope}%
\begin{pgfscope}%
\pgfpathrectangle{\pgfqpoint{0.648703in}{0.548769in}}{\pgfqpoint{5.201297in}{3.102590in}}%
\pgfusepath{clip}%
\pgfsetbuttcap%
\pgfsetroundjoin%
\definecolor{currentfill}{rgb}{1.000000,0.498039,0.054902}%
\pgfsetfillcolor{currentfill}%
\pgfsetlinewidth{1.003750pt}%
\definecolor{currentstroke}{rgb}{1.000000,0.498039,0.054902}%
\pgfsetstrokecolor{currentstroke}%
\pgfsetdash{}{0pt}%
\pgfpathmoveto{\pgfqpoint{2.206556in}{3.356673in}}%
\pgfpathcurveto{\pgfqpoint{2.217606in}{3.356673in}}{\pgfqpoint{2.228205in}{3.361064in}}{\pgfqpoint{2.236019in}{3.368877in}}%
\pgfpathcurveto{\pgfqpoint{2.243832in}{3.376691in}}{\pgfqpoint{2.248223in}{3.387290in}}{\pgfqpoint{2.248223in}{3.398340in}}%
\pgfpathcurveto{\pgfqpoint{2.248223in}{3.409390in}}{\pgfqpoint{2.243832in}{3.419989in}}{\pgfqpoint{2.236019in}{3.427803in}}%
\pgfpathcurveto{\pgfqpoint{2.228205in}{3.435616in}}{\pgfqpoint{2.217606in}{3.440007in}}{\pgfqpoint{2.206556in}{3.440007in}}%
\pgfpathcurveto{\pgfqpoint{2.195506in}{3.440007in}}{\pgfqpoint{2.184907in}{3.435616in}}{\pgfqpoint{2.177093in}{3.427803in}}%
\pgfpathcurveto{\pgfqpoint{2.169279in}{3.419989in}}{\pgfqpoint{2.164889in}{3.409390in}}{\pgfqpoint{2.164889in}{3.398340in}}%
\pgfpathcurveto{\pgfqpoint{2.164889in}{3.387290in}}{\pgfqpoint{2.169279in}{3.376691in}}{\pgfqpoint{2.177093in}{3.368877in}}%
\pgfpathcurveto{\pgfqpoint{2.184907in}{3.361064in}}{\pgfqpoint{2.195506in}{3.356673in}}{\pgfqpoint{2.206556in}{3.356673in}}%
\pgfpathclose%
\pgfusepath{stroke,fill}%
\end{pgfscope}%
\begin{pgfscope}%
\pgfpathrectangle{\pgfqpoint{0.648703in}{0.548769in}}{\pgfqpoint{5.201297in}{3.102590in}}%
\pgfusepath{clip}%
\pgfsetbuttcap%
\pgfsetroundjoin%
\definecolor{currentfill}{rgb}{1.000000,0.498039,0.054902}%
\pgfsetfillcolor{currentfill}%
\pgfsetlinewidth{1.003750pt}%
\definecolor{currentstroke}{rgb}{1.000000,0.498039,0.054902}%
\pgfsetstrokecolor{currentstroke}%
\pgfsetdash{}{0pt}%
\pgfpathmoveto{\pgfqpoint{2.670551in}{3.219794in}}%
\pgfpathcurveto{\pgfqpoint{2.681601in}{3.219794in}}{\pgfqpoint{2.692200in}{3.224185in}}{\pgfqpoint{2.700014in}{3.231998in}}%
\pgfpathcurveto{\pgfqpoint{2.707827in}{3.239812in}}{\pgfqpoint{2.712218in}{3.250411in}}{\pgfqpoint{2.712218in}{3.261461in}}%
\pgfpathcurveto{\pgfqpoint{2.712218in}{3.272511in}}{\pgfqpoint{2.707827in}{3.283110in}}{\pgfqpoint{2.700014in}{3.290924in}}%
\pgfpathcurveto{\pgfqpoint{2.692200in}{3.298737in}}{\pgfqpoint{2.681601in}{3.303128in}}{\pgfqpoint{2.670551in}{3.303128in}}%
\pgfpathcurveto{\pgfqpoint{2.659501in}{3.303128in}}{\pgfqpoint{2.648902in}{3.298737in}}{\pgfqpoint{2.641088in}{3.290924in}}%
\pgfpathcurveto{\pgfqpoint{2.633275in}{3.283110in}}{\pgfqpoint{2.628884in}{3.272511in}}{\pgfqpoint{2.628884in}{3.261461in}}%
\pgfpathcurveto{\pgfqpoint{2.628884in}{3.250411in}}{\pgfqpoint{2.633275in}{3.239812in}}{\pgfqpoint{2.641088in}{3.231998in}}%
\pgfpathcurveto{\pgfqpoint{2.648902in}{3.224185in}}{\pgfqpoint{2.659501in}{3.219794in}}{\pgfqpoint{2.670551in}{3.219794in}}%
\pgfpathclose%
\pgfusepath{stroke,fill}%
\end{pgfscope}%
\begin{pgfscope}%
\pgfpathrectangle{\pgfqpoint{0.648703in}{0.548769in}}{\pgfqpoint{5.201297in}{3.102590in}}%
\pgfusepath{clip}%
\pgfsetbuttcap%
\pgfsetroundjoin%
\definecolor{currentfill}{rgb}{0.121569,0.466667,0.705882}%
\pgfsetfillcolor{currentfill}%
\pgfsetlinewidth{1.003750pt}%
\definecolor{currentstroke}{rgb}{0.121569,0.466667,0.705882}%
\pgfsetstrokecolor{currentstroke}%
\pgfsetdash{}{0pt}%
\pgfpathmoveto{\pgfqpoint{4.610092in}{3.120246in}}%
\pgfpathcurveto{\pgfqpoint{4.621143in}{3.120246in}}{\pgfqpoint{4.631742in}{3.124636in}}{\pgfqpoint{4.639555in}{3.132450in}}%
\pgfpathcurveto{\pgfqpoint{4.647369in}{3.140263in}}{\pgfqpoint{4.651759in}{3.150862in}}{\pgfqpoint{4.651759in}{3.161913in}}%
\pgfpathcurveto{\pgfqpoint{4.651759in}{3.172963in}}{\pgfqpoint{4.647369in}{3.183562in}}{\pgfqpoint{4.639555in}{3.191375in}}%
\pgfpathcurveto{\pgfqpoint{4.631742in}{3.199189in}}{\pgfqpoint{4.621143in}{3.203579in}}{\pgfqpoint{4.610092in}{3.203579in}}%
\pgfpathcurveto{\pgfqpoint{4.599042in}{3.203579in}}{\pgfqpoint{4.588443in}{3.199189in}}{\pgfqpoint{4.580630in}{3.191375in}}%
\pgfpathcurveto{\pgfqpoint{4.572816in}{3.183562in}}{\pgfqpoint{4.568426in}{3.172963in}}{\pgfqpoint{4.568426in}{3.161913in}}%
\pgfpathcurveto{\pgfqpoint{4.568426in}{3.150862in}}{\pgfqpoint{4.572816in}{3.140263in}}{\pgfqpoint{4.580630in}{3.132450in}}%
\pgfpathcurveto{\pgfqpoint{4.588443in}{3.124636in}}{\pgfqpoint{4.599042in}{3.120246in}}{\pgfqpoint{4.610092in}{3.120246in}}%
\pgfpathclose%
\pgfusepath{stroke,fill}%
\end{pgfscope}%
\begin{pgfscope}%
\pgfpathrectangle{\pgfqpoint{0.648703in}{0.548769in}}{\pgfqpoint{5.201297in}{3.102590in}}%
\pgfusepath{clip}%
\pgfsetbuttcap%
\pgfsetroundjoin%
\definecolor{currentfill}{rgb}{1.000000,0.498039,0.054902}%
\pgfsetfillcolor{currentfill}%
\pgfsetlinewidth{1.003750pt}%
\definecolor{currentstroke}{rgb}{1.000000,0.498039,0.054902}%
\pgfsetstrokecolor{currentstroke}%
\pgfsetdash{}{0pt}%
\pgfpathmoveto{\pgfqpoint{1.668730in}{3.140985in}}%
\pgfpathcurveto{\pgfqpoint{1.679780in}{3.140985in}}{\pgfqpoint{1.690379in}{3.145375in}}{\pgfqpoint{1.698193in}{3.153189in}}%
\pgfpathcurveto{\pgfqpoint{1.706006in}{3.161003in}}{\pgfqpoint{1.710396in}{3.171602in}}{\pgfqpoint{1.710396in}{3.182652in}}%
\pgfpathcurveto{\pgfqpoint{1.710396in}{3.193702in}}{\pgfqpoint{1.706006in}{3.204301in}}{\pgfqpoint{1.698193in}{3.212115in}}%
\pgfpathcurveto{\pgfqpoint{1.690379in}{3.219928in}}{\pgfqpoint{1.679780in}{3.224319in}}{\pgfqpoint{1.668730in}{3.224319in}}%
\pgfpathcurveto{\pgfqpoint{1.657680in}{3.224319in}}{\pgfqpoint{1.647081in}{3.219928in}}{\pgfqpoint{1.639267in}{3.212115in}}%
\pgfpathcurveto{\pgfqpoint{1.631453in}{3.204301in}}{\pgfqpoint{1.627063in}{3.193702in}}{\pgfqpoint{1.627063in}{3.182652in}}%
\pgfpathcurveto{\pgfqpoint{1.627063in}{3.171602in}}{\pgfqpoint{1.631453in}{3.161003in}}{\pgfqpoint{1.639267in}{3.153189in}}%
\pgfpathcurveto{\pgfqpoint{1.647081in}{3.145375in}}{\pgfqpoint{1.657680in}{3.140985in}}{\pgfqpoint{1.668730in}{3.140985in}}%
\pgfpathclose%
\pgfusepath{stroke,fill}%
\end{pgfscope}%
\begin{pgfscope}%
\pgfpathrectangle{\pgfqpoint{0.648703in}{0.548769in}}{\pgfqpoint{5.201297in}{3.102590in}}%
\pgfusepath{clip}%
\pgfsetbuttcap%
\pgfsetroundjoin%
\definecolor{currentfill}{rgb}{1.000000,0.498039,0.054902}%
\pgfsetfillcolor{currentfill}%
\pgfsetlinewidth{1.003750pt}%
\definecolor{currentstroke}{rgb}{1.000000,0.498039,0.054902}%
\pgfsetstrokecolor{currentstroke}%
\pgfsetdash{}{0pt}%
\pgfpathmoveto{\pgfqpoint{2.130791in}{3.145133in}}%
\pgfpathcurveto{\pgfqpoint{2.141841in}{3.145133in}}{\pgfqpoint{2.152440in}{3.149523in}}{\pgfqpoint{2.160254in}{3.157337in}}%
\pgfpathcurveto{\pgfqpoint{2.168068in}{3.165151in}}{\pgfqpoint{2.172458in}{3.175750in}}{\pgfqpoint{2.172458in}{3.186800in}}%
\pgfpathcurveto{\pgfqpoint{2.172458in}{3.197850in}}{\pgfqpoint{2.168068in}{3.208449in}}{\pgfqpoint{2.160254in}{3.216262in}}%
\pgfpathcurveto{\pgfqpoint{2.152440in}{3.224076in}}{\pgfqpoint{2.141841in}{3.228466in}}{\pgfqpoint{2.130791in}{3.228466in}}%
\pgfpathcurveto{\pgfqpoint{2.119741in}{3.228466in}}{\pgfqpoint{2.109142in}{3.224076in}}{\pgfqpoint{2.101328in}{3.216262in}}%
\pgfpathcurveto{\pgfqpoint{2.093515in}{3.208449in}}{\pgfqpoint{2.089125in}{3.197850in}}{\pgfqpoint{2.089125in}{3.186800in}}%
\pgfpathcurveto{\pgfqpoint{2.089125in}{3.175750in}}{\pgfqpoint{2.093515in}{3.165151in}}{\pgfqpoint{2.101328in}{3.157337in}}%
\pgfpathcurveto{\pgfqpoint{2.109142in}{3.149523in}}{\pgfqpoint{2.119741in}{3.145133in}}{\pgfqpoint{2.130791in}{3.145133in}}%
\pgfpathclose%
\pgfusepath{stroke,fill}%
\end{pgfscope}%
\begin{pgfscope}%
\pgfpathrectangle{\pgfqpoint{0.648703in}{0.548769in}}{\pgfqpoint{5.201297in}{3.102590in}}%
\pgfusepath{clip}%
\pgfsetbuttcap%
\pgfsetroundjoin%
\definecolor{currentfill}{rgb}{1.000000,0.498039,0.054902}%
\pgfsetfillcolor{currentfill}%
\pgfsetlinewidth{1.003750pt}%
\definecolor{currentstroke}{rgb}{1.000000,0.498039,0.054902}%
\pgfsetstrokecolor{currentstroke}%
\pgfsetdash{}{0pt}%
\pgfpathmoveto{\pgfqpoint{1.963008in}{3.136837in}}%
\pgfpathcurveto{\pgfqpoint{1.974058in}{3.136837in}}{\pgfqpoint{1.984657in}{3.141228in}}{\pgfqpoint{1.992471in}{3.149041in}}%
\pgfpathcurveto{\pgfqpoint{2.000284in}{3.156855in}}{\pgfqpoint{2.004675in}{3.167454in}}{\pgfqpoint{2.004675in}{3.178504in}}%
\pgfpathcurveto{\pgfqpoint{2.004675in}{3.189554in}}{\pgfqpoint{2.000284in}{3.200153in}}{\pgfqpoint{1.992471in}{3.207967in}}%
\pgfpathcurveto{\pgfqpoint{1.984657in}{3.215780in}}{\pgfqpoint{1.974058in}{3.220171in}}{\pgfqpoint{1.963008in}{3.220171in}}%
\pgfpathcurveto{\pgfqpoint{1.951958in}{3.220171in}}{\pgfqpoint{1.941359in}{3.215780in}}{\pgfqpoint{1.933545in}{3.207967in}}%
\pgfpathcurveto{\pgfqpoint{1.925732in}{3.200153in}}{\pgfqpoint{1.921341in}{3.189554in}}{\pgfqpoint{1.921341in}{3.178504in}}%
\pgfpathcurveto{\pgfqpoint{1.921341in}{3.167454in}}{\pgfqpoint{1.925732in}{3.156855in}}{\pgfqpoint{1.933545in}{3.149041in}}%
\pgfpathcurveto{\pgfqpoint{1.941359in}{3.141228in}}{\pgfqpoint{1.951958in}{3.136837in}}{\pgfqpoint{1.963008in}{3.136837in}}%
\pgfpathclose%
\pgfusepath{stroke,fill}%
\end{pgfscope}%
\begin{pgfscope}%
\pgfpathrectangle{\pgfqpoint{0.648703in}{0.548769in}}{\pgfqpoint{5.201297in}{3.102590in}}%
\pgfusepath{clip}%
\pgfsetbuttcap%
\pgfsetroundjoin%
\definecolor{currentfill}{rgb}{0.121569,0.466667,0.705882}%
\pgfsetfillcolor{currentfill}%
\pgfsetlinewidth{1.003750pt}%
\definecolor{currentstroke}{rgb}{0.121569,0.466667,0.705882}%
\pgfsetstrokecolor{currentstroke}%
\pgfsetdash{}{0pt}%
\pgfpathmoveto{\pgfqpoint{2.719435in}{3.132690in}}%
\pgfpathcurveto{\pgfqpoint{2.730485in}{3.132690in}}{\pgfqpoint{2.741084in}{3.137080in}}{\pgfqpoint{2.748898in}{3.144893in}}%
\pgfpathcurveto{\pgfqpoint{2.756711in}{3.152707in}}{\pgfqpoint{2.761102in}{3.163306in}}{\pgfqpoint{2.761102in}{3.174356in}}%
\pgfpathcurveto{\pgfqpoint{2.761102in}{3.185406in}}{\pgfqpoint{2.756711in}{3.196005in}}{\pgfqpoint{2.748898in}{3.203819in}}%
\pgfpathcurveto{\pgfqpoint{2.741084in}{3.211633in}}{\pgfqpoint{2.730485in}{3.216023in}}{\pgfqpoint{2.719435in}{3.216023in}}%
\pgfpathcurveto{\pgfqpoint{2.708385in}{3.216023in}}{\pgfqpoint{2.697786in}{3.211633in}}{\pgfqpoint{2.689972in}{3.203819in}}%
\pgfpathcurveto{\pgfqpoint{2.682158in}{3.196005in}}{\pgfqpoint{2.677768in}{3.185406in}}{\pgfqpoint{2.677768in}{3.174356in}}%
\pgfpathcurveto{\pgfqpoint{2.677768in}{3.163306in}}{\pgfqpoint{2.682158in}{3.152707in}}{\pgfqpoint{2.689972in}{3.144893in}}%
\pgfpathcurveto{\pgfqpoint{2.697786in}{3.137080in}}{\pgfqpoint{2.708385in}{3.132690in}}{\pgfqpoint{2.719435in}{3.132690in}}%
\pgfpathclose%
\pgfusepath{stroke,fill}%
\end{pgfscope}%
\begin{pgfscope}%
\pgfpathrectangle{\pgfqpoint{0.648703in}{0.548769in}}{\pgfqpoint{5.201297in}{3.102590in}}%
\pgfusepath{clip}%
\pgfsetbuttcap%
\pgfsetroundjoin%
\definecolor{currentfill}{rgb}{0.121569,0.466667,0.705882}%
\pgfsetfillcolor{currentfill}%
\pgfsetlinewidth{1.003750pt}%
\definecolor{currentstroke}{rgb}{0.121569,0.466667,0.705882}%
\pgfsetstrokecolor{currentstroke}%
\pgfsetdash{}{0pt}%
\pgfpathmoveto{\pgfqpoint{2.249786in}{3.132690in}}%
\pgfpathcurveto{\pgfqpoint{2.260837in}{3.132690in}}{\pgfqpoint{2.271436in}{3.137080in}}{\pgfqpoint{2.279249in}{3.144893in}}%
\pgfpathcurveto{\pgfqpoint{2.287063in}{3.152707in}}{\pgfqpoint{2.291453in}{3.163306in}}{\pgfqpoint{2.291453in}{3.174356in}}%
\pgfpathcurveto{\pgfqpoint{2.291453in}{3.185406in}}{\pgfqpoint{2.287063in}{3.196005in}}{\pgfqpoint{2.279249in}{3.203819in}}%
\pgfpathcurveto{\pgfqpoint{2.271436in}{3.211633in}}{\pgfqpoint{2.260837in}{3.216023in}}{\pgfqpoint{2.249786in}{3.216023in}}%
\pgfpathcurveto{\pgfqpoint{2.238736in}{3.216023in}}{\pgfqpoint{2.228137in}{3.211633in}}{\pgfqpoint{2.220324in}{3.203819in}}%
\pgfpathcurveto{\pgfqpoint{2.212510in}{3.196005in}}{\pgfqpoint{2.208120in}{3.185406in}}{\pgfqpoint{2.208120in}{3.174356in}}%
\pgfpathcurveto{\pgfqpoint{2.208120in}{3.163306in}}{\pgfqpoint{2.212510in}{3.152707in}}{\pgfqpoint{2.220324in}{3.144893in}}%
\pgfpathcurveto{\pgfqpoint{2.228137in}{3.137080in}}{\pgfqpoint{2.238736in}{3.132690in}}{\pgfqpoint{2.249786in}{3.132690in}}%
\pgfpathclose%
\pgfusepath{stroke,fill}%
\end{pgfscope}%
\begin{pgfscope}%
\pgfpathrectangle{\pgfqpoint{0.648703in}{0.548769in}}{\pgfqpoint{5.201297in}{3.102590in}}%
\pgfusepath{clip}%
\pgfsetbuttcap%
\pgfsetroundjoin%
\definecolor{currentfill}{rgb}{1.000000,0.498039,0.054902}%
\pgfsetfillcolor{currentfill}%
\pgfsetlinewidth{1.003750pt}%
\definecolor{currentstroke}{rgb}{1.000000,0.498039,0.054902}%
\pgfsetstrokecolor{currentstroke}%
\pgfsetdash{}{0pt}%
\pgfpathmoveto{\pgfqpoint{2.103518in}{3.136837in}}%
\pgfpathcurveto{\pgfqpoint{2.114569in}{3.136837in}}{\pgfqpoint{2.125168in}{3.141228in}}{\pgfqpoint{2.132981in}{3.149041in}}%
\pgfpathcurveto{\pgfqpoint{2.140795in}{3.156855in}}{\pgfqpoint{2.145185in}{3.167454in}}{\pgfqpoint{2.145185in}{3.178504in}}%
\pgfpathcurveto{\pgfqpoint{2.145185in}{3.189554in}}{\pgfqpoint{2.140795in}{3.200153in}}{\pgfqpoint{2.132981in}{3.207967in}}%
\pgfpathcurveto{\pgfqpoint{2.125168in}{3.215780in}}{\pgfqpoint{2.114569in}{3.220171in}}{\pgfqpoint{2.103518in}{3.220171in}}%
\pgfpathcurveto{\pgfqpoint{2.092468in}{3.220171in}}{\pgfqpoint{2.081869in}{3.215780in}}{\pgfqpoint{2.074056in}{3.207967in}}%
\pgfpathcurveto{\pgfqpoint{2.066242in}{3.200153in}}{\pgfqpoint{2.061852in}{3.189554in}}{\pgfqpoint{2.061852in}{3.178504in}}%
\pgfpathcurveto{\pgfqpoint{2.061852in}{3.167454in}}{\pgfqpoint{2.066242in}{3.156855in}}{\pgfqpoint{2.074056in}{3.149041in}}%
\pgfpathcurveto{\pgfqpoint{2.081869in}{3.141228in}}{\pgfqpoint{2.092468in}{3.136837in}}{\pgfqpoint{2.103518in}{3.136837in}}%
\pgfpathclose%
\pgfusepath{stroke,fill}%
\end{pgfscope}%
\begin{pgfscope}%
\pgfpathrectangle{\pgfqpoint{0.648703in}{0.548769in}}{\pgfqpoint{5.201297in}{3.102590in}}%
\pgfusepath{clip}%
\pgfsetbuttcap%
\pgfsetroundjoin%
\definecolor{currentfill}{rgb}{0.839216,0.152941,0.156863}%
\pgfsetfillcolor{currentfill}%
\pgfsetlinewidth{1.003750pt}%
\definecolor{currentstroke}{rgb}{0.839216,0.152941,0.156863}%
\pgfsetstrokecolor{currentstroke}%
\pgfsetdash{}{0pt}%
\pgfpathmoveto{\pgfqpoint{2.621267in}{3.120246in}}%
\pgfpathcurveto{\pgfqpoint{2.632317in}{3.120246in}}{\pgfqpoint{2.642916in}{3.124636in}}{\pgfqpoint{2.650729in}{3.132450in}}%
\pgfpathcurveto{\pgfqpoint{2.658543in}{3.140263in}}{\pgfqpoint{2.662933in}{3.150862in}}{\pgfqpoint{2.662933in}{3.161913in}}%
\pgfpathcurveto{\pgfqpoint{2.662933in}{3.172963in}}{\pgfqpoint{2.658543in}{3.183562in}}{\pgfqpoint{2.650729in}{3.191375in}}%
\pgfpathcurveto{\pgfqpoint{2.642916in}{3.199189in}}{\pgfqpoint{2.632317in}{3.203579in}}{\pgfqpoint{2.621267in}{3.203579in}}%
\pgfpathcurveto{\pgfqpoint{2.610216in}{3.203579in}}{\pgfqpoint{2.599617in}{3.199189in}}{\pgfqpoint{2.591804in}{3.191375in}}%
\pgfpathcurveto{\pgfqpoint{2.583990in}{3.183562in}}{\pgfqpoint{2.579600in}{3.172963in}}{\pgfqpoint{2.579600in}{3.161913in}}%
\pgfpathcurveto{\pgfqpoint{2.579600in}{3.150862in}}{\pgfqpoint{2.583990in}{3.140263in}}{\pgfqpoint{2.591804in}{3.132450in}}%
\pgfpathcurveto{\pgfqpoint{2.599617in}{3.124636in}}{\pgfqpoint{2.610216in}{3.120246in}}{\pgfqpoint{2.621267in}{3.120246in}}%
\pgfpathclose%
\pgfusepath{stroke,fill}%
\end{pgfscope}%
\begin{pgfscope}%
\pgfpathrectangle{\pgfqpoint{0.648703in}{0.548769in}}{\pgfqpoint{5.201297in}{3.102590in}}%
\pgfusepath{clip}%
\pgfsetbuttcap%
\pgfsetroundjoin%
\definecolor{currentfill}{rgb}{1.000000,0.498039,0.054902}%
\pgfsetfillcolor{currentfill}%
\pgfsetlinewidth{1.003750pt}%
\definecolor{currentstroke}{rgb}{1.000000,0.498039,0.054902}%
\pgfsetstrokecolor{currentstroke}%
\pgfsetdash{}{0pt}%
\pgfpathmoveto{\pgfqpoint{1.542313in}{3.244681in}}%
\pgfpathcurveto{\pgfqpoint{1.553363in}{3.244681in}}{\pgfqpoint{1.563962in}{3.249072in}}{\pgfqpoint{1.571776in}{3.256885in}}%
\pgfpathcurveto{\pgfqpoint{1.579590in}{3.264699in}}{\pgfqpoint{1.583980in}{3.275298in}}{\pgfqpoint{1.583980in}{3.286348in}}%
\pgfpathcurveto{\pgfqpoint{1.583980in}{3.297398in}}{\pgfqpoint{1.579590in}{3.307997in}}{\pgfqpoint{1.571776in}{3.315811in}}%
\pgfpathcurveto{\pgfqpoint{1.563962in}{3.323624in}}{\pgfqpoint{1.553363in}{3.328015in}}{\pgfqpoint{1.542313in}{3.328015in}}%
\pgfpathcurveto{\pgfqpoint{1.531263in}{3.328015in}}{\pgfqpoint{1.520664in}{3.323624in}}{\pgfqpoint{1.512850in}{3.315811in}}%
\pgfpathcurveto{\pgfqpoint{1.505037in}{3.307997in}}{\pgfqpoint{1.500646in}{3.297398in}}{\pgfqpoint{1.500646in}{3.286348in}}%
\pgfpathcurveto{\pgfqpoint{1.500646in}{3.275298in}}{\pgfqpoint{1.505037in}{3.264699in}}{\pgfqpoint{1.512850in}{3.256885in}}%
\pgfpathcurveto{\pgfqpoint{1.520664in}{3.249072in}}{\pgfqpoint{1.531263in}{3.244681in}}{\pgfqpoint{1.542313in}{3.244681in}}%
\pgfpathclose%
\pgfusepath{stroke,fill}%
\end{pgfscope}%
\begin{pgfscope}%
\pgfpathrectangle{\pgfqpoint{0.648703in}{0.548769in}}{\pgfqpoint{5.201297in}{3.102590in}}%
\pgfusepath{clip}%
\pgfsetbuttcap%
\pgfsetroundjoin%
\definecolor{currentfill}{rgb}{0.121569,0.466667,0.705882}%
\pgfsetfillcolor{currentfill}%
\pgfsetlinewidth{1.003750pt}%
\definecolor{currentstroke}{rgb}{0.121569,0.466667,0.705882}%
\pgfsetstrokecolor{currentstroke}%
\pgfsetdash{}{0pt}%
\pgfpathmoveto{\pgfqpoint{1.883202in}{3.128542in}}%
\pgfpathcurveto{\pgfqpoint{1.894252in}{3.128542in}}{\pgfqpoint{1.904851in}{3.132932in}}{\pgfqpoint{1.912664in}{3.140746in}}%
\pgfpathcurveto{\pgfqpoint{1.920478in}{3.148559in}}{\pgfqpoint{1.924868in}{3.159158in}}{\pgfqpoint{1.924868in}{3.170208in}}%
\pgfpathcurveto{\pgfqpoint{1.924868in}{3.181258in}}{\pgfqpoint{1.920478in}{3.191857in}}{\pgfqpoint{1.912664in}{3.199671in}}%
\pgfpathcurveto{\pgfqpoint{1.904851in}{3.207485in}}{\pgfqpoint{1.894252in}{3.211875in}}{\pgfqpoint{1.883202in}{3.211875in}}%
\pgfpathcurveto{\pgfqpoint{1.872152in}{3.211875in}}{\pgfqpoint{1.861553in}{3.207485in}}{\pgfqpoint{1.853739in}{3.199671in}}%
\pgfpathcurveto{\pgfqpoint{1.845925in}{3.191857in}}{\pgfqpoint{1.841535in}{3.181258in}}{\pgfqpoint{1.841535in}{3.170208in}}%
\pgfpathcurveto{\pgfqpoint{1.841535in}{3.159158in}}{\pgfqpoint{1.845925in}{3.148559in}}{\pgfqpoint{1.853739in}{3.140746in}}%
\pgfpathcurveto{\pgfqpoint{1.861553in}{3.132932in}}{\pgfqpoint{1.872152in}{3.128542in}}{\pgfqpoint{1.883202in}{3.128542in}}%
\pgfpathclose%
\pgfusepath{stroke,fill}%
\end{pgfscope}%
\begin{pgfscope}%
\pgfpathrectangle{\pgfqpoint{0.648703in}{0.548769in}}{\pgfqpoint{5.201297in}{3.102590in}}%
\pgfusepath{clip}%
\pgfsetbuttcap%
\pgfsetroundjoin%
\definecolor{currentfill}{rgb}{1.000000,0.498039,0.054902}%
\pgfsetfillcolor{currentfill}%
\pgfsetlinewidth{1.003750pt}%
\definecolor{currentstroke}{rgb}{1.000000,0.498039,0.054902}%
\pgfsetstrokecolor{currentstroke}%
\pgfsetdash{}{0pt}%
\pgfpathmoveto{\pgfqpoint{2.731508in}{3.136837in}}%
\pgfpathcurveto{\pgfqpoint{2.742558in}{3.136837in}}{\pgfqpoint{2.753157in}{3.141228in}}{\pgfqpoint{2.760970in}{3.149041in}}%
\pgfpathcurveto{\pgfqpoint{2.768784in}{3.156855in}}{\pgfqpoint{2.773174in}{3.167454in}}{\pgfqpoint{2.773174in}{3.178504in}}%
\pgfpathcurveto{\pgfqpoint{2.773174in}{3.189554in}}{\pgfqpoint{2.768784in}{3.200153in}}{\pgfqpoint{2.760970in}{3.207967in}}%
\pgfpathcurveto{\pgfqpoint{2.753157in}{3.215780in}}{\pgfqpoint{2.742558in}{3.220171in}}{\pgfqpoint{2.731508in}{3.220171in}}%
\pgfpathcurveto{\pgfqpoint{2.720458in}{3.220171in}}{\pgfqpoint{2.709859in}{3.215780in}}{\pgfqpoint{2.702045in}{3.207967in}}%
\pgfpathcurveto{\pgfqpoint{2.694231in}{3.200153in}}{\pgfqpoint{2.689841in}{3.189554in}}{\pgfqpoint{2.689841in}{3.178504in}}%
\pgfpathcurveto{\pgfqpoint{2.689841in}{3.167454in}}{\pgfqpoint{2.694231in}{3.156855in}}{\pgfqpoint{2.702045in}{3.149041in}}%
\pgfpathcurveto{\pgfqpoint{2.709859in}{3.141228in}}{\pgfqpoint{2.720458in}{3.136837in}}{\pgfqpoint{2.731508in}{3.136837in}}%
\pgfpathclose%
\pgfusepath{stroke,fill}%
\end{pgfscope}%
\begin{pgfscope}%
\pgfpathrectangle{\pgfqpoint{0.648703in}{0.548769in}}{\pgfqpoint{5.201297in}{3.102590in}}%
\pgfusepath{clip}%
\pgfsetbuttcap%
\pgfsetroundjoin%
\definecolor{currentfill}{rgb}{1.000000,0.498039,0.054902}%
\pgfsetfillcolor{currentfill}%
\pgfsetlinewidth{1.003750pt}%
\definecolor{currentstroke}{rgb}{1.000000,0.498039,0.054902}%
\pgfsetstrokecolor{currentstroke}%
\pgfsetdash{}{0pt}%
\pgfpathmoveto{\pgfqpoint{1.577277in}{3.136837in}}%
\pgfpathcurveto{\pgfqpoint{1.588328in}{3.136837in}}{\pgfqpoint{1.598927in}{3.141228in}}{\pgfqpoint{1.606740in}{3.149041in}}%
\pgfpathcurveto{\pgfqpoint{1.614554in}{3.156855in}}{\pgfqpoint{1.618944in}{3.167454in}}{\pgfqpoint{1.618944in}{3.178504in}}%
\pgfpathcurveto{\pgfqpoint{1.618944in}{3.189554in}}{\pgfqpoint{1.614554in}{3.200153in}}{\pgfqpoint{1.606740in}{3.207967in}}%
\pgfpathcurveto{\pgfqpoint{1.598927in}{3.215780in}}{\pgfqpoint{1.588328in}{3.220171in}}{\pgfqpoint{1.577277in}{3.220171in}}%
\pgfpathcurveto{\pgfqpoint{1.566227in}{3.220171in}}{\pgfqpoint{1.555628in}{3.215780in}}{\pgfqpoint{1.547815in}{3.207967in}}%
\pgfpathcurveto{\pgfqpoint{1.540001in}{3.200153in}}{\pgfqpoint{1.535611in}{3.189554in}}{\pgfqpoint{1.535611in}{3.178504in}}%
\pgfpathcurveto{\pgfqpoint{1.535611in}{3.167454in}}{\pgfqpoint{1.540001in}{3.156855in}}{\pgfqpoint{1.547815in}{3.149041in}}%
\pgfpathcurveto{\pgfqpoint{1.555628in}{3.141228in}}{\pgfqpoint{1.566227in}{3.136837in}}{\pgfqpoint{1.577277in}{3.136837in}}%
\pgfpathclose%
\pgfusepath{stroke,fill}%
\end{pgfscope}%
\begin{pgfscope}%
\pgfpathrectangle{\pgfqpoint{0.648703in}{0.548769in}}{\pgfqpoint{5.201297in}{3.102590in}}%
\pgfusepath{clip}%
\pgfsetbuttcap%
\pgfsetroundjoin%
\definecolor{currentfill}{rgb}{1.000000,0.498039,0.054902}%
\pgfsetfillcolor{currentfill}%
\pgfsetlinewidth{1.003750pt}%
\definecolor{currentstroke}{rgb}{1.000000,0.498039,0.054902}%
\pgfsetstrokecolor{currentstroke}%
\pgfsetdash{}{0pt}%
\pgfpathmoveto{\pgfqpoint{2.355315in}{3.136837in}}%
\pgfpathcurveto{\pgfqpoint{2.366365in}{3.136837in}}{\pgfqpoint{2.376964in}{3.141228in}}{\pgfqpoint{2.384778in}{3.149041in}}%
\pgfpathcurveto{\pgfqpoint{2.392592in}{3.156855in}}{\pgfqpoint{2.396982in}{3.167454in}}{\pgfqpoint{2.396982in}{3.178504in}}%
\pgfpathcurveto{\pgfqpoint{2.396982in}{3.189554in}}{\pgfqpoint{2.392592in}{3.200153in}}{\pgfqpoint{2.384778in}{3.207967in}}%
\pgfpathcurveto{\pgfqpoint{2.376964in}{3.215780in}}{\pgfqpoint{2.366365in}{3.220171in}}{\pgfqpoint{2.355315in}{3.220171in}}%
\pgfpathcurveto{\pgfqpoint{2.344265in}{3.220171in}}{\pgfqpoint{2.333666in}{3.215780in}}{\pgfqpoint{2.325852in}{3.207967in}}%
\pgfpathcurveto{\pgfqpoint{2.318039in}{3.200153in}}{\pgfqpoint{2.313648in}{3.189554in}}{\pgfqpoint{2.313648in}{3.178504in}}%
\pgfpathcurveto{\pgfqpoint{2.313648in}{3.167454in}}{\pgfqpoint{2.318039in}{3.156855in}}{\pgfqpoint{2.325852in}{3.149041in}}%
\pgfpathcurveto{\pgfqpoint{2.333666in}{3.141228in}}{\pgfqpoint{2.344265in}{3.136837in}}{\pgfqpoint{2.355315in}{3.136837in}}%
\pgfpathclose%
\pgfusepath{stroke,fill}%
\end{pgfscope}%
\begin{pgfscope}%
\pgfpathrectangle{\pgfqpoint{0.648703in}{0.548769in}}{\pgfqpoint{5.201297in}{3.102590in}}%
\pgfusepath{clip}%
\pgfsetbuttcap%
\pgfsetroundjoin%
\definecolor{currentfill}{rgb}{0.121569,0.466667,0.705882}%
\pgfsetfillcolor{currentfill}%
\pgfsetlinewidth{1.003750pt}%
\definecolor{currentstroke}{rgb}{0.121569,0.466667,0.705882}%
\pgfsetstrokecolor{currentstroke}%
\pgfsetdash{}{0pt}%
\pgfpathmoveto{\pgfqpoint{2.385645in}{3.132690in}}%
\pgfpathcurveto{\pgfqpoint{2.396696in}{3.132690in}}{\pgfqpoint{2.407295in}{3.137080in}}{\pgfqpoint{2.415108in}{3.144893in}}%
\pgfpathcurveto{\pgfqpoint{2.422922in}{3.152707in}}{\pgfqpoint{2.427312in}{3.163306in}}{\pgfqpoint{2.427312in}{3.174356in}}%
\pgfpathcurveto{\pgfqpoint{2.427312in}{3.185406in}}{\pgfqpoint{2.422922in}{3.196005in}}{\pgfqpoint{2.415108in}{3.203819in}}%
\pgfpathcurveto{\pgfqpoint{2.407295in}{3.211633in}}{\pgfqpoint{2.396696in}{3.216023in}}{\pgfqpoint{2.385645in}{3.216023in}}%
\pgfpathcurveto{\pgfqpoint{2.374595in}{3.216023in}}{\pgfqpoint{2.363996in}{3.211633in}}{\pgfqpoint{2.356183in}{3.203819in}}%
\pgfpathcurveto{\pgfqpoint{2.348369in}{3.196005in}}{\pgfqpoint{2.343979in}{3.185406in}}{\pgfqpoint{2.343979in}{3.174356in}}%
\pgfpathcurveto{\pgfqpoint{2.343979in}{3.163306in}}{\pgfqpoint{2.348369in}{3.152707in}}{\pgfqpoint{2.356183in}{3.144893in}}%
\pgfpathcurveto{\pgfqpoint{2.363996in}{3.137080in}}{\pgfqpoint{2.374595in}{3.132690in}}{\pgfqpoint{2.385645in}{3.132690in}}%
\pgfpathclose%
\pgfusepath{stroke,fill}%
\end{pgfscope}%
\begin{pgfscope}%
\pgfpathrectangle{\pgfqpoint{0.648703in}{0.548769in}}{\pgfqpoint{5.201297in}{3.102590in}}%
\pgfusepath{clip}%
\pgfsetbuttcap%
\pgfsetroundjoin%
\definecolor{currentfill}{rgb}{0.121569,0.466667,0.705882}%
\pgfsetfillcolor{currentfill}%
\pgfsetlinewidth{1.003750pt}%
\definecolor{currentstroke}{rgb}{0.121569,0.466667,0.705882}%
\pgfsetstrokecolor{currentstroke}%
\pgfsetdash{}{0pt}%
\pgfpathmoveto{\pgfqpoint{4.651738in}{3.107802in}}%
\pgfpathcurveto{\pgfqpoint{4.662788in}{3.107802in}}{\pgfqpoint{4.673387in}{3.112193in}}{\pgfqpoint{4.681201in}{3.120006in}}%
\pgfpathcurveto{\pgfqpoint{4.689014in}{3.127820in}}{\pgfqpoint{4.693404in}{3.138419in}}{\pgfqpoint{4.693404in}{3.149469in}}%
\pgfpathcurveto{\pgfqpoint{4.693404in}{3.160519in}}{\pgfqpoint{4.689014in}{3.171118in}}{\pgfqpoint{4.681201in}{3.178932in}}%
\pgfpathcurveto{\pgfqpoint{4.673387in}{3.186745in}}{\pgfqpoint{4.662788in}{3.191136in}}{\pgfqpoint{4.651738in}{3.191136in}}%
\pgfpathcurveto{\pgfqpoint{4.640688in}{3.191136in}}{\pgfqpoint{4.630089in}{3.186745in}}{\pgfqpoint{4.622275in}{3.178932in}}%
\pgfpathcurveto{\pgfqpoint{4.614461in}{3.171118in}}{\pgfqpoint{4.610071in}{3.160519in}}{\pgfqpoint{4.610071in}{3.149469in}}%
\pgfpathcurveto{\pgfqpoint{4.610071in}{3.138419in}}{\pgfqpoint{4.614461in}{3.127820in}}{\pgfqpoint{4.622275in}{3.120006in}}%
\pgfpathcurveto{\pgfqpoint{4.630089in}{3.112193in}}{\pgfqpoint{4.640688in}{3.107802in}}{\pgfqpoint{4.651738in}{3.107802in}}%
\pgfpathclose%
\pgfusepath{stroke,fill}%
\end{pgfscope}%
\begin{pgfscope}%
\pgfpathrectangle{\pgfqpoint{0.648703in}{0.548769in}}{\pgfqpoint{5.201297in}{3.102590in}}%
\pgfusepath{clip}%
\pgfsetbuttcap%
\pgfsetroundjoin%
\definecolor{currentfill}{rgb}{0.121569,0.466667,0.705882}%
\pgfsetfillcolor{currentfill}%
\pgfsetlinewidth{1.003750pt}%
\definecolor{currentstroke}{rgb}{0.121569,0.466667,0.705882}%
\pgfsetstrokecolor{currentstroke}%
\pgfsetdash{}{0pt}%
\pgfpathmoveto{\pgfqpoint{2.046891in}{3.132690in}}%
\pgfpathcurveto{\pgfqpoint{2.057941in}{3.132690in}}{\pgfqpoint{2.068540in}{3.137080in}}{\pgfqpoint{2.076354in}{3.144893in}}%
\pgfpathcurveto{\pgfqpoint{2.084167in}{3.152707in}}{\pgfqpoint{2.088558in}{3.163306in}}{\pgfqpoint{2.088558in}{3.174356in}}%
\pgfpathcurveto{\pgfqpoint{2.088558in}{3.185406in}}{\pgfqpoint{2.084167in}{3.196005in}}{\pgfqpoint{2.076354in}{3.203819in}}%
\pgfpathcurveto{\pgfqpoint{2.068540in}{3.211633in}}{\pgfqpoint{2.057941in}{3.216023in}}{\pgfqpoint{2.046891in}{3.216023in}}%
\pgfpathcurveto{\pgfqpoint{2.035841in}{3.216023in}}{\pgfqpoint{2.025242in}{3.211633in}}{\pgfqpoint{2.017428in}{3.203819in}}%
\pgfpathcurveto{\pgfqpoint{2.009615in}{3.196005in}}{\pgfqpoint{2.005224in}{3.185406in}}{\pgfqpoint{2.005224in}{3.174356in}}%
\pgfpathcurveto{\pgfqpoint{2.005224in}{3.163306in}}{\pgfqpoint{2.009615in}{3.152707in}}{\pgfqpoint{2.017428in}{3.144893in}}%
\pgfpathcurveto{\pgfqpoint{2.025242in}{3.137080in}}{\pgfqpoint{2.035841in}{3.132690in}}{\pgfqpoint{2.046891in}{3.132690in}}%
\pgfpathclose%
\pgfusepath{stroke,fill}%
\end{pgfscope}%
\begin{pgfscope}%
\pgfpathrectangle{\pgfqpoint{0.648703in}{0.548769in}}{\pgfqpoint{5.201297in}{3.102590in}}%
\pgfusepath{clip}%
\pgfsetbuttcap%
\pgfsetroundjoin%
\definecolor{currentfill}{rgb}{0.121569,0.466667,0.705882}%
\pgfsetfillcolor{currentfill}%
\pgfsetlinewidth{1.003750pt}%
\definecolor{currentstroke}{rgb}{0.121569,0.466667,0.705882}%
\pgfsetstrokecolor{currentstroke}%
\pgfsetdash{}{0pt}%
\pgfpathmoveto{\pgfqpoint{1.042997in}{0.648129in}}%
\pgfpathcurveto{\pgfqpoint{1.054047in}{0.648129in}}{\pgfqpoint{1.064646in}{0.652519in}}{\pgfqpoint{1.072459in}{0.660333in}}%
\pgfpathcurveto{\pgfqpoint{1.080273in}{0.668146in}}{\pgfqpoint{1.084663in}{0.678745in}}{\pgfqpoint{1.084663in}{0.689796in}}%
\pgfpathcurveto{\pgfqpoint{1.084663in}{0.700846in}}{\pgfqpoint{1.080273in}{0.711445in}}{\pgfqpoint{1.072459in}{0.719258in}}%
\pgfpathcurveto{\pgfqpoint{1.064646in}{0.727072in}}{\pgfqpoint{1.054047in}{0.731462in}}{\pgfqpoint{1.042997in}{0.731462in}}%
\pgfpathcurveto{\pgfqpoint{1.031946in}{0.731462in}}{\pgfqpoint{1.021347in}{0.727072in}}{\pgfqpoint{1.013534in}{0.719258in}}%
\pgfpathcurveto{\pgfqpoint{1.005720in}{0.711445in}}{\pgfqpoint{1.001330in}{0.700846in}}{\pgfqpoint{1.001330in}{0.689796in}}%
\pgfpathcurveto{\pgfqpoint{1.001330in}{0.678745in}}{\pgfqpoint{1.005720in}{0.668146in}}{\pgfqpoint{1.013534in}{0.660333in}}%
\pgfpathcurveto{\pgfqpoint{1.021347in}{0.652519in}}{\pgfqpoint{1.031946in}{0.648129in}}{\pgfqpoint{1.042997in}{0.648129in}}%
\pgfpathclose%
\pgfusepath{stroke,fill}%
\end{pgfscope}%
\begin{pgfscope}%
\pgfpathrectangle{\pgfqpoint{0.648703in}{0.548769in}}{\pgfqpoint{5.201297in}{3.102590in}}%
\pgfusepath{clip}%
\pgfsetbuttcap%
\pgfsetroundjoin%
\definecolor{currentfill}{rgb}{1.000000,0.498039,0.054902}%
\pgfsetfillcolor{currentfill}%
\pgfsetlinewidth{1.003750pt}%
\definecolor{currentstroke}{rgb}{1.000000,0.498039,0.054902}%
\pgfsetstrokecolor{currentstroke}%
\pgfsetdash{}{0pt}%
\pgfpathmoveto{\pgfqpoint{1.932747in}{3.140985in}}%
\pgfpathcurveto{\pgfqpoint{1.943798in}{3.140985in}}{\pgfqpoint{1.954397in}{3.145375in}}{\pgfqpoint{1.962210in}{3.153189in}}%
\pgfpathcurveto{\pgfqpoint{1.970024in}{3.161003in}}{\pgfqpoint{1.974414in}{3.171602in}}{\pgfqpoint{1.974414in}{3.182652in}}%
\pgfpathcurveto{\pgfqpoint{1.974414in}{3.193702in}}{\pgfqpoint{1.970024in}{3.204301in}}{\pgfqpoint{1.962210in}{3.212115in}}%
\pgfpathcurveto{\pgfqpoint{1.954397in}{3.219928in}}{\pgfqpoint{1.943798in}{3.224319in}}{\pgfqpoint{1.932747in}{3.224319in}}%
\pgfpathcurveto{\pgfqpoint{1.921697in}{3.224319in}}{\pgfqpoint{1.911098in}{3.219928in}}{\pgfqpoint{1.903285in}{3.212115in}}%
\pgfpathcurveto{\pgfqpoint{1.895471in}{3.204301in}}{\pgfqpoint{1.891081in}{3.193702in}}{\pgfqpoint{1.891081in}{3.182652in}}%
\pgfpathcurveto{\pgfqpoint{1.891081in}{3.171602in}}{\pgfqpoint{1.895471in}{3.161003in}}{\pgfqpoint{1.903285in}{3.153189in}}%
\pgfpathcurveto{\pgfqpoint{1.911098in}{3.145375in}}{\pgfqpoint{1.921697in}{3.140985in}}{\pgfqpoint{1.932747in}{3.140985in}}%
\pgfpathclose%
\pgfusepath{stroke,fill}%
\end{pgfscope}%
\begin{pgfscope}%
\pgfpathrectangle{\pgfqpoint{0.648703in}{0.548769in}}{\pgfqpoint{5.201297in}{3.102590in}}%
\pgfusepath{clip}%
\pgfsetbuttcap%
\pgfsetroundjoin%
\definecolor{currentfill}{rgb}{1.000000,0.498039,0.054902}%
\pgfsetfillcolor{currentfill}%
\pgfsetlinewidth{1.003750pt}%
\definecolor{currentstroke}{rgb}{1.000000,0.498039,0.054902}%
\pgfsetstrokecolor{currentstroke}%
\pgfsetdash{}{0pt}%
\pgfpathmoveto{\pgfqpoint{2.192758in}{3.145133in}}%
\pgfpathcurveto{\pgfqpoint{2.203808in}{3.145133in}}{\pgfqpoint{2.214407in}{3.149523in}}{\pgfqpoint{2.222221in}{3.157337in}}%
\pgfpathcurveto{\pgfqpoint{2.230035in}{3.165151in}}{\pgfqpoint{2.234425in}{3.175750in}}{\pgfqpoint{2.234425in}{3.186800in}}%
\pgfpathcurveto{\pgfqpoint{2.234425in}{3.197850in}}{\pgfqpoint{2.230035in}{3.208449in}}{\pgfqpoint{2.222221in}{3.216262in}}%
\pgfpathcurveto{\pgfqpoint{2.214407in}{3.224076in}}{\pgfqpoint{2.203808in}{3.228466in}}{\pgfqpoint{2.192758in}{3.228466in}}%
\pgfpathcurveto{\pgfqpoint{2.181708in}{3.228466in}}{\pgfqpoint{2.171109in}{3.224076in}}{\pgfqpoint{2.163296in}{3.216262in}}%
\pgfpathcurveto{\pgfqpoint{2.155482in}{3.208449in}}{\pgfqpoint{2.151092in}{3.197850in}}{\pgfqpoint{2.151092in}{3.186800in}}%
\pgfpathcurveto{\pgfqpoint{2.151092in}{3.175750in}}{\pgfqpoint{2.155482in}{3.165151in}}{\pgfqpoint{2.163296in}{3.157337in}}%
\pgfpathcurveto{\pgfqpoint{2.171109in}{3.149523in}}{\pgfqpoint{2.181708in}{3.145133in}}{\pgfqpoint{2.192758in}{3.145133in}}%
\pgfpathclose%
\pgfusepath{stroke,fill}%
\end{pgfscope}%
\begin{pgfscope}%
\pgfpathrectangle{\pgfqpoint{0.648703in}{0.548769in}}{\pgfqpoint{5.201297in}{3.102590in}}%
\pgfusepath{clip}%
\pgfsetbuttcap%
\pgfsetroundjoin%
\definecolor{currentfill}{rgb}{1.000000,0.498039,0.054902}%
\pgfsetfillcolor{currentfill}%
\pgfsetlinewidth{1.003750pt}%
\definecolor{currentstroke}{rgb}{1.000000,0.498039,0.054902}%
\pgfsetstrokecolor{currentstroke}%
\pgfsetdash{}{0pt}%
\pgfpathmoveto{\pgfqpoint{1.270952in}{3.140985in}}%
\pgfpathcurveto{\pgfqpoint{1.282003in}{3.140985in}}{\pgfqpoint{1.292602in}{3.145375in}}{\pgfqpoint{1.300415in}{3.153189in}}%
\pgfpathcurveto{\pgfqpoint{1.308229in}{3.161003in}}{\pgfqpoint{1.312619in}{3.171602in}}{\pgfqpoint{1.312619in}{3.182652in}}%
\pgfpathcurveto{\pgfqpoint{1.312619in}{3.193702in}}{\pgfqpoint{1.308229in}{3.204301in}}{\pgfqpoint{1.300415in}{3.212115in}}%
\pgfpathcurveto{\pgfqpoint{1.292602in}{3.219928in}}{\pgfqpoint{1.282003in}{3.224319in}}{\pgfqpoint{1.270952in}{3.224319in}}%
\pgfpathcurveto{\pgfqpoint{1.259902in}{3.224319in}}{\pgfqpoint{1.249303in}{3.219928in}}{\pgfqpoint{1.241490in}{3.212115in}}%
\pgfpathcurveto{\pgfqpoint{1.233676in}{3.204301in}}{\pgfqpoint{1.229286in}{3.193702in}}{\pgfqpoint{1.229286in}{3.182652in}}%
\pgfpathcurveto{\pgfqpoint{1.229286in}{3.171602in}}{\pgfqpoint{1.233676in}{3.161003in}}{\pgfqpoint{1.241490in}{3.153189in}}%
\pgfpathcurveto{\pgfqpoint{1.249303in}{3.145375in}}{\pgfqpoint{1.259902in}{3.140985in}}{\pgfqpoint{1.270952in}{3.140985in}}%
\pgfpathclose%
\pgfusepath{stroke,fill}%
\end{pgfscope}%
\begin{pgfscope}%
\pgfpathrectangle{\pgfqpoint{0.648703in}{0.548769in}}{\pgfqpoint{5.201297in}{3.102590in}}%
\pgfusepath{clip}%
\pgfsetbuttcap%
\pgfsetroundjoin%
\definecolor{currentfill}{rgb}{0.121569,0.466667,0.705882}%
\pgfsetfillcolor{currentfill}%
\pgfsetlinewidth{1.003750pt}%
\definecolor{currentstroke}{rgb}{0.121569,0.466667,0.705882}%
\pgfsetstrokecolor{currentstroke}%
\pgfsetdash{}{0pt}%
\pgfpathmoveto{\pgfqpoint{1.231999in}{0.697903in}}%
\pgfpathcurveto{\pgfqpoint{1.243049in}{0.697903in}}{\pgfqpoint{1.253648in}{0.702293in}}{\pgfqpoint{1.261462in}{0.710107in}}%
\pgfpathcurveto{\pgfqpoint{1.269275in}{0.717921in}}{\pgfqpoint{1.273665in}{0.728520in}}{\pgfqpoint{1.273665in}{0.739570in}}%
\pgfpathcurveto{\pgfqpoint{1.273665in}{0.750620in}}{\pgfqpoint{1.269275in}{0.761219in}}{\pgfqpoint{1.261462in}{0.769033in}}%
\pgfpathcurveto{\pgfqpoint{1.253648in}{0.776846in}}{\pgfqpoint{1.243049in}{0.781236in}}{\pgfqpoint{1.231999in}{0.781236in}}%
\pgfpathcurveto{\pgfqpoint{1.220949in}{0.781236in}}{\pgfqpoint{1.210350in}{0.776846in}}{\pgfqpoint{1.202536in}{0.769033in}}%
\pgfpathcurveto{\pgfqpoint{1.194722in}{0.761219in}}{\pgfqpoint{1.190332in}{0.750620in}}{\pgfqpoint{1.190332in}{0.739570in}}%
\pgfpathcurveto{\pgfqpoint{1.190332in}{0.728520in}}{\pgfqpoint{1.194722in}{0.717921in}}{\pgfqpoint{1.202536in}{0.710107in}}%
\pgfpathcurveto{\pgfqpoint{1.210350in}{0.702293in}}{\pgfqpoint{1.220949in}{0.697903in}}{\pgfqpoint{1.231999in}{0.697903in}}%
\pgfpathclose%
\pgfusepath{stroke,fill}%
\end{pgfscope}%
\begin{pgfscope}%
\pgfpathrectangle{\pgfqpoint{0.648703in}{0.548769in}}{\pgfqpoint{5.201297in}{3.102590in}}%
\pgfusepath{clip}%
\pgfsetbuttcap%
\pgfsetroundjoin%
\definecolor{currentfill}{rgb}{1.000000,0.498039,0.054902}%
\pgfsetfillcolor{currentfill}%
\pgfsetlinewidth{1.003750pt}%
\definecolor{currentstroke}{rgb}{1.000000,0.498039,0.054902}%
\pgfsetstrokecolor{currentstroke}%
\pgfsetdash{}{0pt}%
\pgfpathmoveto{\pgfqpoint{2.407108in}{3.140985in}}%
\pgfpathcurveto{\pgfqpoint{2.418158in}{3.140985in}}{\pgfqpoint{2.428757in}{3.145375in}}{\pgfqpoint{2.436571in}{3.153189in}}%
\pgfpathcurveto{\pgfqpoint{2.444385in}{3.161003in}}{\pgfqpoint{2.448775in}{3.171602in}}{\pgfqpoint{2.448775in}{3.182652in}}%
\pgfpathcurveto{\pgfqpoint{2.448775in}{3.193702in}}{\pgfqpoint{2.444385in}{3.204301in}}{\pgfqpoint{2.436571in}{3.212115in}}%
\pgfpathcurveto{\pgfqpoint{2.428757in}{3.219928in}}{\pgfqpoint{2.418158in}{3.224319in}}{\pgfqpoint{2.407108in}{3.224319in}}%
\pgfpathcurveto{\pgfqpoint{2.396058in}{3.224319in}}{\pgfqpoint{2.385459in}{3.219928in}}{\pgfqpoint{2.377645in}{3.212115in}}%
\pgfpathcurveto{\pgfqpoint{2.369832in}{3.204301in}}{\pgfqpoint{2.365442in}{3.193702in}}{\pgfqpoint{2.365442in}{3.182652in}}%
\pgfpathcurveto{\pgfqpoint{2.365442in}{3.171602in}}{\pgfqpoint{2.369832in}{3.161003in}}{\pgfqpoint{2.377645in}{3.153189in}}%
\pgfpathcurveto{\pgfqpoint{2.385459in}{3.145375in}}{\pgfqpoint{2.396058in}{3.140985in}}{\pgfqpoint{2.407108in}{3.140985in}}%
\pgfpathclose%
\pgfusepath{stroke,fill}%
\end{pgfscope}%
\begin{pgfscope}%
\pgfpathrectangle{\pgfqpoint{0.648703in}{0.548769in}}{\pgfqpoint{5.201297in}{3.102590in}}%
\pgfusepath{clip}%
\pgfsetbuttcap%
\pgfsetroundjoin%
\definecolor{currentfill}{rgb}{0.121569,0.466667,0.705882}%
\pgfsetfillcolor{currentfill}%
\pgfsetlinewidth{1.003750pt}%
\definecolor{currentstroke}{rgb}{0.121569,0.466667,0.705882}%
\pgfsetstrokecolor{currentstroke}%
\pgfsetdash{}{0pt}%
\pgfpathmoveto{\pgfqpoint{2.139241in}{3.132690in}}%
\pgfpathcurveto{\pgfqpoint{2.150291in}{3.132690in}}{\pgfqpoint{2.160890in}{3.137080in}}{\pgfqpoint{2.168703in}{3.144893in}}%
\pgfpathcurveto{\pgfqpoint{2.176517in}{3.152707in}}{\pgfqpoint{2.180907in}{3.163306in}}{\pgfqpoint{2.180907in}{3.174356in}}%
\pgfpathcurveto{\pgfqpoint{2.180907in}{3.185406in}}{\pgfqpoint{2.176517in}{3.196005in}}{\pgfqpoint{2.168703in}{3.203819in}}%
\pgfpathcurveto{\pgfqpoint{2.160890in}{3.211633in}}{\pgfqpoint{2.150291in}{3.216023in}}{\pgfqpoint{2.139241in}{3.216023in}}%
\pgfpathcurveto{\pgfqpoint{2.128190in}{3.216023in}}{\pgfqpoint{2.117591in}{3.211633in}}{\pgfqpoint{2.109778in}{3.203819in}}%
\pgfpathcurveto{\pgfqpoint{2.101964in}{3.196005in}}{\pgfqpoint{2.097574in}{3.185406in}}{\pgfqpoint{2.097574in}{3.174356in}}%
\pgfpathcurveto{\pgfqpoint{2.097574in}{3.163306in}}{\pgfqpoint{2.101964in}{3.152707in}}{\pgfqpoint{2.109778in}{3.144893in}}%
\pgfpathcurveto{\pgfqpoint{2.117591in}{3.137080in}}{\pgfqpoint{2.128190in}{3.132690in}}{\pgfqpoint{2.139241in}{3.132690in}}%
\pgfpathclose%
\pgfusepath{stroke,fill}%
\end{pgfscope}%
\begin{pgfscope}%
\pgfpathrectangle{\pgfqpoint{0.648703in}{0.548769in}}{\pgfqpoint{5.201297in}{3.102590in}}%
\pgfusepath{clip}%
\pgfsetbuttcap%
\pgfsetroundjoin%
\definecolor{currentfill}{rgb}{1.000000,0.498039,0.054902}%
\pgfsetfillcolor{currentfill}%
\pgfsetlinewidth{1.003750pt}%
\definecolor{currentstroke}{rgb}{1.000000,0.498039,0.054902}%
\pgfsetstrokecolor{currentstroke}%
\pgfsetdash{}{0pt}%
\pgfpathmoveto{\pgfqpoint{3.128762in}{3.140985in}}%
\pgfpathcurveto{\pgfqpoint{3.139813in}{3.140985in}}{\pgfqpoint{3.150412in}{3.145375in}}{\pgfqpoint{3.158225in}{3.153189in}}%
\pgfpathcurveto{\pgfqpoint{3.166039in}{3.161003in}}{\pgfqpoint{3.170429in}{3.171602in}}{\pgfqpoint{3.170429in}{3.182652in}}%
\pgfpathcurveto{\pgfqpoint{3.170429in}{3.193702in}}{\pgfqpoint{3.166039in}{3.204301in}}{\pgfqpoint{3.158225in}{3.212115in}}%
\pgfpathcurveto{\pgfqpoint{3.150412in}{3.219928in}}{\pgfqpoint{3.139813in}{3.224319in}}{\pgfqpoint{3.128762in}{3.224319in}}%
\pgfpathcurveto{\pgfqpoint{3.117712in}{3.224319in}}{\pgfqpoint{3.107113in}{3.219928in}}{\pgfqpoint{3.099300in}{3.212115in}}%
\pgfpathcurveto{\pgfqpoint{3.091486in}{3.204301in}}{\pgfqpoint{3.087096in}{3.193702in}}{\pgfqpoint{3.087096in}{3.182652in}}%
\pgfpathcurveto{\pgfqpoint{3.087096in}{3.171602in}}{\pgfqpoint{3.091486in}{3.161003in}}{\pgfqpoint{3.099300in}{3.153189in}}%
\pgfpathcurveto{\pgfqpoint{3.107113in}{3.145375in}}{\pgfqpoint{3.117712in}{3.140985in}}{\pgfqpoint{3.128762in}{3.140985in}}%
\pgfpathclose%
\pgfusepath{stroke,fill}%
\end{pgfscope}%
\begin{pgfscope}%
\pgfpathrectangle{\pgfqpoint{0.648703in}{0.548769in}}{\pgfqpoint{5.201297in}{3.102590in}}%
\pgfusepath{clip}%
\pgfsetbuttcap%
\pgfsetroundjoin%
\definecolor{currentfill}{rgb}{0.121569,0.466667,0.705882}%
\pgfsetfillcolor{currentfill}%
\pgfsetlinewidth{1.003750pt}%
\definecolor{currentstroke}{rgb}{0.121569,0.466667,0.705882}%
\pgfsetstrokecolor{currentstroke}%
\pgfsetdash{}{0pt}%
\pgfpathmoveto{\pgfqpoint{2.306510in}{3.132690in}}%
\pgfpathcurveto{\pgfqpoint{2.317560in}{3.132690in}}{\pgfqpoint{2.328159in}{3.137080in}}{\pgfqpoint{2.335973in}{3.144893in}}%
\pgfpathcurveto{\pgfqpoint{2.343786in}{3.152707in}}{\pgfqpoint{2.348176in}{3.163306in}}{\pgfqpoint{2.348176in}{3.174356in}}%
\pgfpathcurveto{\pgfqpoint{2.348176in}{3.185406in}}{\pgfqpoint{2.343786in}{3.196005in}}{\pgfqpoint{2.335973in}{3.203819in}}%
\pgfpathcurveto{\pgfqpoint{2.328159in}{3.211633in}}{\pgfqpoint{2.317560in}{3.216023in}}{\pgfqpoint{2.306510in}{3.216023in}}%
\pgfpathcurveto{\pgfqpoint{2.295460in}{3.216023in}}{\pgfqpoint{2.284861in}{3.211633in}}{\pgfqpoint{2.277047in}{3.203819in}}%
\pgfpathcurveto{\pgfqpoint{2.269233in}{3.196005in}}{\pgfqpoint{2.264843in}{3.185406in}}{\pgfqpoint{2.264843in}{3.174356in}}%
\pgfpathcurveto{\pgfqpoint{2.264843in}{3.163306in}}{\pgfqpoint{2.269233in}{3.152707in}}{\pgfqpoint{2.277047in}{3.144893in}}%
\pgfpathcurveto{\pgfqpoint{2.284861in}{3.137080in}}{\pgfqpoint{2.295460in}{3.132690in}}{\pgfqpoint{2.306510in}{3.132690in}}%
\pgfpathclose%
\pgfusepath{stroke,fill}%
\end{pgfscope}%
\begin{pgfscope}%
\pgfpathrectangle{\pgfqpoint{0.648703in}{0.548769in}}{\pgfqpoint{5.201297in}{3.102590in}}%
\pgfusepath{clip}%
\pgfsetbuttcap%
\pgfsetroundjoin%
\definecolor{currentfill}{rgb}{1.000000,0.498039,0.054902}%
\pgfsetfillcolor{currentfill}%
\pgfsetlinewidth{1.003750pt}%
\definecolor{currentstroke}{rgb}{1.000000,0.498039,0.054902}%
\pgfsetstrokecolor{currentstroke}%
\pgfsetdash{}{0pt}%
\pgfpathmoveto{\pgfqpoint{2.422996in}{3.140985in}}%
\pgfpathcurveto{\pgfqpoint{2.434046in}{3.140985in}}{\pgfqpoint{2.444646in}{3.145375in}}{\pgfqpoint{2.452459in}{3.153189in}}%
\pgfpathcurveto{\pgfqpoint{2.460273in}{3.161003in}}{\pgfqpoint{2.464663in}{3.171602in}}{\pgfqpoint{2.464663in}{3.182652in}}%
\pgfpathcurveto{\pgfqpoint{2.464663in}{3.193702in}}{\pgfqpoint{2.460273in}{3.204301in}}{\pgfqpoint{2.452459in}{3.212115in}}%
\pgfpathcurveto{\pgfqpoint{2.444646in}{3.219928in}}{\pgfqpoint{2.434046in}{3.224319in}}{\pgfqpoint{2.422996in}{3.224319in}}%
\pgfpathcurveto{\pgfqpoint{2.411946in}{3.224319in}}{\pgfqpoint{2.401347in}{3.219928in}}{\pgfqpoint{2.393534in}{3.212115in}}%
\pgfpathcurveto{\pgfqpoint{2.385720in}{3.204301in}}{\pgfqpoint{2.381330in}{3.193702in}}{\pgfqpoint{2.381330in}{3.182652in}}%
\pgfpathcurveto{\pgfqpoint{2.381330in}{3.171602in}}{\pgfqpoint{2.385720in}{3.161003in}}{\pgfqpoint{2.393534in}{3.153189in}}%
\pgfpathcurveto{\pgfqpoint{2.401347in}{3.145375in}}{\pgfqpoint{2.411946in}{3.140985in}}{\pgfqpoint{2.422996in}{3.140985in}}%
\pgfpathclose%
\pgfusepath{stroke,fill}%
\end{pgfscope}%
\begin{pgfscope}%
\pgfpathrectangle{\pgfqpoint{0.648703in}{0.548769in}}{\pgfqpoint{5.201297in}{3.102590in}}%
\pgfusepath{clip}%
\pgfsetbuttcap%
\pgfsetroundjoin%
\definecolor{currentfill}{rgb}{1.000000,0.498039,0.054902}%
\pgfsetfillcolor{currentfill}%
\pgfsetlinewidth{1.003750pt}%
\definecolor{currentstroke}{rgb}{1.000000,0.498039,0.054902}%
\pgfsetstrokecolor{currentstroke}%
\pgfsetdash{}{0pt}%
\pgfpathmoveto{\pgfqpoint{1.626327in}{3.145133in}}%
\pgfpathcurveto{\pgfqpoint{1.637377in}{3.145133in}}{\pgfqpoint{1.647976in}{3.149523in}}{\pgfqpoint{1.655789in}{3.157337in}}%
\pgfpathcurveto{\pgfqpoint{1.663603in}{3.165151in}}{\pgfqpoint{1.667993in}{3.175750in}}{\pgfqpoint{1.667993in}{3.186800in}}%
\pgfpathcurveto{\pgfqpoint{1.667993in}{3.197850in}}{\pgfqpoint{1.663603in}{3.208449in}}{\pgfqpoint{1.655789in}{3.216262in}}%
\pgfpathcurveto{\pgfqpoint{1.647976in}{3.224076in}}{\pgfqpoint{1.637377in}{3.228466in}}{\pgfqpoint{1.626327in}{3.228466in}}%
\pgfpathcurveto{\pgfqpoint{1.615277in}{3.228466in}}{\pgfqpoint{1.604678in}{3.224076in}}{\pgfqpoint{1.596864in}{3.216262in}}%
\pgfpathcurveto{\pgfqpoint{1.589050in}{3.208449in}}{\pgfqpoint{1.584660in}{3.197850in}}{\pgfqpoint{1.584660in}{3.186800in}}%
\pgfpathcurveto{\pgfqpoint{1.584660in}{3.175750in}}{\pgfqpoint{1.589050in}{3.165151in}}{\pgfqpoint{1.596864in}{3.157337in}}%
\pgfpathcurveto{\pgfqpoint{1.604678in}{3.149523in}}{\pgfqpoint{1.615277in}{3.145133in}}{\pgfqpoint{1.626327in}{3.145133in}}%
\pgfpathclose%
\pgfusepath{stroke,fill}%
\end{pgfscope}%
\begin{pgfscope}%
\pgfpathrectangle{\pgfqpoint{0.648703in}{0.548769in}}{\pgfqpoint{5.201297in}{3.102590in}}%
\pgfusepath{clip}%
\pgfsetbuttcap%
\pgfsetroundjoin%
\definecolor{currentfill}{rgb}{1.000000,0.498039,0.054902}%
\pgfsetfillcolor{currentfill}%
\pgfsetlinewidth{1.003750pt}%
\definecolor{currentstroke}{rgb}{1.000000,0.498039,0.054902}%
\pgfsetstrokecolor{currentstroke}%
\pgfsetdash{}{0pt}%
\pgfpathmoveto{\pgfqpoint{1.963592in}{3.145133in}}%
\pgfpathcurveto{\pgfqpoint{1.974642in}{3.145133in}}{\pgfqpoint{1.985241in}{3.149523in}}{\pgfqpoint{1.993054in}{3.157337in}}%
\pgfpathcurveto{\pgfqpoint{2.000868in}{3.165151in}}{\pgfqpoint{2.005258in}{3.175750in}}{\pgfqpoint{2.005258in}{3.186800in}}%
\pgfpathcurveto{\pgfqpoint{2.005258in}{3.197850in}}{\pgfqpoint{2.000868in}{3.208449in}}{\pgfqpoint{1.993054in}{3.216262in}}%
\pgfpathcurveto{\pgfqpoint{1.985241in}{3.224076in}}{\pgfqpoint{1.974642in}{3.228466in}}{\pgfqpoint{1.963592in}{3.228466in}}%
\pgfpathcurveto{\pgfqpoint{1.952542in}{3.228466in}}{\pgfqpoint{1.941942in}{3.224076in}}{\pgfqpoint{1.934129in}{3.216262in}}%
\pgfpathcurveto{\pgfqpoint{1.926315in}{3.208449in}}{\pgfqpoint{1.921925in}{3.197850in}}{\pgfqpoint{1.921925in}{3.186800in}}%
\pgfpathcurveto{\pgfqpoint{1.921925in}{3.175750in}}{\pgfqpoint{1.926315in}{3.165151in}}{\pgfqpoint{1.934129in}{3.157337in}}%
\pgfpathcurveto{\pgfqpoint{1.941942in}{3.149523in}}{\pgfqpoint{1.952542in}{3.145133in}}{\pgfqpoint{1.963592in}{3.145133in}}%
\pgfpathclose%
\pgfusepath{stroke,fill}%
\end{pgfscope}%
\begin{pgfscope}%
\pgfpathrectangle{\pgfqpoint{0.648703in}{0.548769in}}{\pgfqpoint{5.201297in}{3.102590in}}%
\pgfusepath{clip}%
\pgfsetbuttcap%
\pgfsetroundjoin%
\definecolor{currentfill}{rgb}{1.000000,0.498039,0.054902}%
\pgfsetfillcolor{currentfill}%
\pgfsetlinewidth{1.003750pt}%
\definecolor{currentstroke}{rgb}{1.000000,0.498039,0.054902}%
\pgfsetstrokecolor{currentstroke}%
\pgfsetdash{}{0pt}%
\pgfpathmoveto{\pgfqpoint{2.594821in}{3.136837in}}%
\pgfpathcurveto{\pgfqpoint{2.605871in}{3.136837in}}{\pgfqpoint{2.616470in}{3.141228in}}{\pgfqpoint{2.624284in}{3.149041in}}%
\pgfpathcurveto{\pgfqpoint{2.632098in}{3.156855in}}{\pgfqpoint{2.636488in}{3.167454in}}{\pgfqpoint{2.636488in}{3.178504in}}%
\pgfpathcurveto{\pgfqpoint{2.636488in}{3.189554in}}{\pgfqpoint{2.632098in}{3.200153in}}{\pgfqpoint{2.624284in}{3.207967in}}%
\pgfpathcurveto{\pgfqpoint{2.616470in}{3.215780in}}{\pgfqpoint{2.605871in}{3.220171in}}{\pgfqpoint{2.594821in}{3.220171in}}%
\pgfpathcurveto{\pgfqpoint{2.583771in}{3.220171in}}{\pgfqpoint{2.573172in}{3.215780in}}{\pgfqpoint{2.565358in}{3.207967in}}%
\pgfpathcurveto{\pgfqpoint{2.557545in}{3.200153in}}{\pgfqpoint{2.553155in}{3.189554in}}{\pgfqpoint{2.553155in}{3.178504in}}%
\pgfpathcurveto{\pgfqpoint{2.553155in}{3.167454in}}{\pgfqpoint{2.557545in}{3.156855in}}{\pgfqpoint{2.565358in}{3.149041in}}%
\pgfpathcurveto{\pgfqpoint{2.573172in}{3.141228in}}{\pgfqpoint{2.583771in}{3.136837in}}{\pgfqpoint{2.594821in}{3.136837in}}%
\pgfpathclose%
\pgfusepath{stroke,fill}%
\end{pgfscope}%
\begin{pgfscope}%
\pgfpathrectangle{\pgfqpoint{0.648703in}{0.548769in}}{\pgfqpoint{5.201297in}{3.102590in}}%
\pgfusepath{clip}%
\pgfsetbuttcap%
\pgfsetroundjoin%
\definecolor{currentfill}{rgb}{1.000000,0.498039,0.054902}%
\pgfsetfillcolor{currentfill}%
\pgfsetlinewidth{1.003750pt}%
\definecolor{currentstroke}{rgb}{1.000000,0.498039,0.054902}%
\pgfsetstrokecolor{currentstroke}%
\pgfsetdash{}{0pt}%
\pgfpathmoveto{\pgfqpoint{1.492175in}{3.468665in}}%
\pgfpathcurveto{\pgfqpoint{1.503225in}{3.468665in}}{\pgfqpoint{1.513824in}{3.473055in}}{\pgfqpoint{1.521638in}{3.480869in}}%
\pgfpathcurveto{\pgfqpoint{1.529451in}{3.488683in}}{\pgfqpoint{1.533842in}{3.499282in}}{\pgfqpoint{1.533842in}{3.510332in}}%
\pgfpathcurveto{\pgfqpoint{1.533842in}{3.521382in}}{\pgfqpoint{1.529451in}{3.531981in}}{\pgfqpoint{1.521638in}{3.539795in}}%
\pgfpathcurveto{\pgfqpoint{1.513824in}{3.547608in}}{\pgfqpoint{1.503225in}{3.551998in}}{\pgfqpoint{1.492175in}{3.551998in}}%
\pgfpathcurveto{\pgfqpoint{1.481125in}{3.551998in}}{\pgfqpoint{1.470526in}{3.547608in}}{\pgfqpoint{1.462712in}{3.539795in}}%
\pgfpathcurveto{\pgfqpoint{1.454899in}{3.531981in}}{\pgfqpoint{1.450508in}{3.521382in}}{\pgfqpoint{1.450508in}{3.510332in}}%
\pgfpathcurveto{\pgfqpoint{1.450508in}{3.499282in}}{\pgfqpoint{1.454899in}{3.488683in}}{\pgfqpoint{1.462712in}{3.480869in}}%
\pgfpathcurveto{\pgfqpoint{1.470526in}{3.473055in}}{\pgfqpoint{1.481125in}{3.468665in}}{\pgfqpoint{1.492175in}{3.468665in}}%
\pgfpathclose%
\pgfusepath{stroke,fill}%
\end{pgfscope}%
\begin{pgfscope}%
\pgfpathrectangle{\pgfqpoint{0.648703in}{0.548769in}}{\pgfqpoint{5.201297in}{3.102590in}}%
\pgfusepath{clip}%
\pgfsetbuttcap%
\pgfsetroundjoin%
\definecolor{currentfill}{rgb}{1.000000,0.498039,0.054902}%
\pgfsetfillcolor{currentfill}%
\pgfsetlinewidth{1.003750pt}%
\definecolor{currentstroke}{rgb}{1.000000,0.498039,0.054902}%
\pgfsetstrokecolor{currentstroke}%
\pgfsetdash{}{0pt}%
\pgfpathmoveto{\pgfqpoint{1.770783in}{3.136837in}}%
\pgfpathcurveto{\pgfqpoint{1.781833in}{3.136837in}}{\pgfqpoint{1.792432in}{3.141228in}}{\pgfqpoint{1.800246in}{3.149041in}}%
\pgfpathcurveto{\pgfqpoint{1.808059in}{3.156855in}}{\pgfqpoint{1.812450in}{3.167454in}}{\pgfqpoint{1.812450in}{3.178504in}}%
\pgfpathcurveto{\pgfqpoint{1.812450in}{3.189554in}}{\pgfqpoint{1.808059in}{3.200153in}}{\pgfqpoint{1.800246in}{3.207967in}}%
\pgfpathcurveto{\pgfqpoint{1.792432in}{3.215780in}}{\pgfqpoint{1.781833in}{3.220171in}}{\pgfqpoint{1.770783in}{3.220171in}}%
\pgfpathcurveto{\pgfqpoint{1.759733in}{3.220171in}}{\pgfqpoint{1.749134in}{3.215780in}}{\pgfqpoint{1.741320in}{3.207967in}}%
\pgfpathcurveto{\pgfqpoint{1.733507in}{3.200153in}}{\pgfqpoint{1.729116in}{3.189554in}}{\pgfqpoint{1.729116in}{3.178504in}}%
\pgfpathcurveto{\pgfqpoint{1.729116in}{3.167454in}}{\pgfqpoint{1.733507in}{3.156855in}}{\pgfqpoint{1.741320in}{3.149041in}}%
\pgfpathcurveto{\pgfqpoint{1.749134in}{3.141228in}}{\pgfqpoint{1.759733in}{3.136837in}}{\pgfqpoint{1.770783in}{3.136837in}}%
\pgfpathclose%
\pgfusepath{stroke,fill}%
\end{pgfscope}%
\begin{pgfscope}%
\pgfpathrectangle{\pgfqpoint{0.648703in}{0.548769in}}{\pgfqpoint{5.201297in}{3.102590in}}%
\pgfusepath{clip}%
\pgfsetbuttcap%
\pgfsetroundjoin%
\definecolor{currentfill}{rgb}{1.000000,0.498039,0.054902}%
\pgfsetfillcolor{currentfill}%
\pgfsetlinewidth{1.003750pt}%
\definecolor{currentstroke}{rgb}{1.000000,0.498039,0.054902}%
\pgfsetstrokecolor{currentstroke}%
\pgfsetdash{}{0pt}%
\pgfpathmoveto{\pgfqpoint{2.337972in}{3.136837in}}%
\pgfpathcurveto{\pgfqpoint{2.349023in}{3.136837in}}{\pgfqpoint{2.359622in}{3.141228in}}{\pgfqpoint{2.367435in}{3.149041in}}%
\pgfpathcurveto{\pgfqpoint{2.375249in}{3.156855in}}{\pgfqpoint{2.379639in}{3.167454in}}{\pgfqpoint{2.379639in}{3.178504in}}%
\pgfpathcurveto{\pgfqpoint{2.379639in}{3.189554in}}{\pgfqpoint{2.375249in}{3.200153in}}{\pgfqpoint{2.367435in}{3.207967in}}%
\pgfpathcurveto{\pgfqpoint{2.359622in}{3.215780in}}{\pgfqpoint{2.349023in}{3.220171in}}{\pgfqpoint{2.337972in}{3.220171in}}%
\pgfpathcurveto{\pgfqpoint{2.326922in}{3.220171in}}{\pgfqpoint{2.316323in}{3.215780in}}{\pgfqpoint{2.308510in}{3.207967in}}%
\pgfpathcurveto{\pgfqpoint{2.300696in}{3.200153in}}{\pgfqpoint{2.296306in}{3.189554in}}{\pgfqpoint{2.296306in}{3.178504in}}%
\pgfpathcurveto{\pgfqpoint{2.296306in}{3.167454in}}{\pgfqpoint{2.300696in}{3.156855in}}{\pgfqpoint{2.308510in}{3.149041in}}%
\pgfpathcurveto{\pgfqpoint{2.316323in}{3.141228in}}{\pgfqpoint{2.326922in}{3.136837in}}{\pgfqpoint{2.337972in}{3.136837in}}%
\pgfpathclose%
\pgfusepath{stroke,fill}%
\end{pgfscope}%
\begin{pgfscope}%
\pgfpathrectangle{\pgfqpoint{0.648703in}{0.548769in}}{\pgfqpoint{5.201297in}{3.102590in}}%
\pgfusepath{clip}%
\pgfsetbuttcap%
\pgfsetroundjoin%
\definecolor{currentfill}{rgb}{1.000000,0.498039,0.054902}%
\pgfsetfillcolor{currentfill}%
\pgfsetlinewidth{1.003750pt}%
\definecolor{currentstroke}{rgb}{1.000000,0.498039,0.054902}%
\pgfsetstrokecolor{currentstroke}%
\pgfsetdash{}{0pt}%
\pgfpathmoveto{\pgfqpoint{2.172480in}{3.136837in}}%
\pgfpathcurveto{\pgfqpoint{2.183530in}{3.136837in}}{\pgfqpoint{2.194129in}{3.141228in}}{\pgfqpoint{2.201943in}{3.149041in}}%
\pgfpathcurveto{\pgfqpoint{2.209756in}{3.156855in}}{\pgfqpoint{2.214147in}{3.167454in}}{\pgfqpoint{2.214147in}{3.178504in}}%
\pgfpathcurveto{\pgfqpoint{2.214147in}{3.189554in}}{\pgfqpoint{2.209756in}{3.200153in}}{\pgfqpoint{2.201943in}{3.207967in}}%
\pgfpathcurveto{\pgfqpoint{2.194129in}{3.215780in}}{\pgfqpoint{2.183530in}{3.220171in}}{\pgfqpoint{2.172480in}{3.220171in}}%
\pgfpathcurveto{\pgfqpoint{2.161430in}{3.220171in}}{\pgfqpoint{2.150831in}{3.215780in}}{\pgfqpoint{2.143017in}{3.207967in}}%
\pgfpathcurveto{\pgfqpoint{2.135204in}{3.200153in}}{\pgfqpoint{2.130813in}{3.189554in}}{\pgfqpoint{2.130813in}{3.178504in}}%
\pgfpathcurveto{\pgfqpoint{2.130813in}{3.167454in}}{\pgfqpoint{2.135204in}{3.156855in}}{\pgfqpoint{2.143017in}{3.149041in}}%
\pgfpathcurveto{\pgfqpoint{2.150831in}{3.141228in}}{\pgfqpoint{2.161430in}{3.136837in}}{\pgfqpoint{2.172480in}{3.136837in}}%
\pgfpathclose%
\pgfusepath{stroke,fill}%
\end{pgfscope}%
\begin{pgfscope}%
\pgfpathrectangle{\pgfqpoint{0.648703in}{0.548769in}}{\pgfqpoint{5.201297in}{3.102590in}}%
\pgfusepath{clip}%
\pgfsetbuttcap%
\pgfsetroundjoin%
\definecolor{currentfill}{rgb}{0.839216,0.152941,0.156863}%
\pgfsetfillcolor{currentfill}%
\pgfsetlinewidth{1.003750pt}%
\definecolor{currentstroke}{rgb}{0.839216,0.152941,0.156863}%
\pgfsetstrokecolor{currentstroke}%
\pgfsetdash{}{0pt}%
\pgfpathmoveto{\pgfqpoint{1.955543in}{3.136837in}}%
\pgfpathcurveto{\pgfqpoint{1.966593in}{3.136837in}}{\pgfqpoint{1.977192in}{3.141228in}}{\pgfqpoint{1.985006in}{3.149041in}}%
\pgfpathcurveto{\pgfqpoint{1.992819in}{3.156855in}}{\pgfqpoint{1.997210in}{3.167454in}}{\pgfqpoint{1.997210in}{3.178504in}}%
\pgfpathcurveto{\pgfqpoint{1.997210in}{3.189554in}}{\pgfqpoint{1.992819in}{3.200153in}}{\pgfqpoint{1.985006in}{3.207967in}}%
\pgfpathcurveto{\pgfqpoint{1.977192in}{3.215780in}}{\pgfqpoint{1.966593in}{3.220171in}}{\pgfqpoint{1.955543in}{3.220171in}}%
\pgfpathcurveto{\pgfqpoint{1.944493in}{3.220171in}}{\pgfqpoint{1.933894in}{3.215780in}}{\pgfqpoint{1.926080in}{3.207967in}}%
\pgfpathcurveto{\pgfqpoint{1.918267in}{3.200153in}}{\pgfqpoint{1.913876in}{3.189554in}}{\pgfqpoint{1.913876in}{3.178504in}}%
\pgfpathcurveto{\pgfqpoint{1.913876in}{3.167454in}}{\pgfqpoint{1.918267in}{3.156855in}}{\pgfqpoint{1.926080in}{3.149041in}}%
\pgfpathcurveto{\pgfqpoint{1.933894in}{3.141228in}}{\pgfqpoint{1.944493in}{3.136837in}}{\pgfqpoint{1.955543in}{3.136837in}}%
\pgfpathclose%
\pgfusepath{stroke,fill}%
\end{pgfscope}%
\begin{pgfscope}%
\pgfpathrectangle{\pgfqpoint{0.648703in}{0.548769in}}{\pgfqpoint{5.201297in}{3.102590in}}%
\pgfusepath{clip}%
\pgfsetbuttcap%
\pgfsetroundjoin%
\definecolor{currentfill}{rgb}{1.000000,0.498039,0.054902}%
\pgfsetfillcolor{currentfill}%
\pgfsetlinewidth{1.003750pt}%
\definecolor{currentstroke}{rgb}{1.000000,0.498039,0.054902}%
\pgfsetstrokecolor{currentstroke}%
\pgfsetdash{}{0pt}%
\pgfpathmoveto{\pgfqpoint{1.696290in}{3.149281in}}%
\pgfpathcurveto{\pgfqpoint{1.707340in}{3.149281in}}{\pgfqpoint{1.717939in}{3.153671in}}{\pgfqpoint{1.725753in}{3.161485in}}%
\pgfpathcurveto{\pgfqpoint{1.733566in}{3.169298in}}{\pgfqpoint{1.737957in}{3.179897in}}{\pgfqpoint{1.737957in}{3.190948in}}%
\pgfpathcurveto{\pgfqpoint{1.737957in}{3.201998in}}{\pgfqpoint{1.733566in}{3.212597in}}{\pgfqpoint{1.725753in}{3.220410in}}%
\pgfpathcurveto{\pgfqpoint{1.717939in}{3.228224in}}{\pgfqpoint{1.707340in}{3.232614in}}{\pgfqpoint{1.696290in}{3.232614in}}%
\pgfpathcurveto{\pgfqpoint{1.685240in}{3.232614in}}{\pgfqpoint{1.674641in}{3.228224in}}{\pgfqpoint{1.666827in}{3.220410in}}%
\pgfpathcurveto{\pgfqpoint{1.659014in}{3.212597in}}{\pgfqpoint{1.654623in}{3.201998in}}{\pgfqpoint{1.654623in}{3.190948in}}%
\pgfpathcurveto{\pgfqpoint{1.654623in}{3.179897in}}{\pgfqpoint{1.659014in}{3.169298in}}{\pgfqpoint{1.666827in}{3.161485in}}%
\pgfpathcurveto{\pgfqpoint{1.674641in}{3.153671in}}{\pgfqpoint{1.685240in}{3.149281in}}{\pgfqpoint{1.696290in}{3.149281in}}%
\pgfpathclose%
\pgfusepath{stroke,fill}%
\end{pgfscope}%
\begin{pgfscope}%
\pgfpathrectangle{\pgfqpoint{0.648703in}{0.548769in}}{\pgfqpoint{5.201297in}{3.102590in}}%
\pgfusepath{clip}%
\pgfsetbuttcap%
\pgfsetroundjoin%
\definecolor{currentfill}{rgb}{1.000000,0.498039,0.054902}%
\pgfsetfillcolor{currentfill}%
\pgfsetlinewidth{1.003750pt}%
\definecolor{currentstroke}{rgb}{1.000000,0.498039,0.054902}%
\pgfsetstrokecolor{currentstroke}%
\pgfsetdash{}{0pt}%
\pgfpathmoveto{\pgfqpoint{1.802089in}{3.286160in}}%
\pgfpathcurveto{\pgfqpoint{1.813139in}{3.286160in}}{\pgfqpoint{1.823738in}{3.290550in}}{\pgfqpoint{1.831552in}{3.298364in}}%
\pgfpathcurveto{\pgfqpoint{1.839365in}{3.306177in}}{\pgfqpoint{1.843755in}{3.316776in}}{\pgfqpoint{1.843755in}{3.327827in}}%
\pgfpathcurveto{\pgfqpoint{1.843755in}{3.338877in}}{\pgfqpoint{1.839365in}{3.349476in}}{\pgfqpoint{1.831552in}{3.357289in}}%
\pgfpathcurveto{\pgfqpoint{1.823738in}{3.365103in}}{\pgfqpoint{1.813139in}{3.369493in}}{\pgfqpoint{1.802089in}{3.369493in}}%
\pgfpathcurveto{\pgfqpoint{1.791039in}{3.369493in}}{\pgfqpoint{1.780440in}{3.365103in}}{\pgfqpoint{1.772626in}{3.357289in}}%
\pgfpathcurveto{\pgfqpoint{1.764812in}{3.349476in}}{\pgfqpoint{1.760422in}{3.338877in}}{\pgfqpoint{1.760422in}{3.327827in}}%
\pgfpathcurveto{\pgfqpoint{1.760422in}{3.316776in}}{\pgfqpoint{1.764812in}{3.306177in}}{\pgfqpoint{1.772626in}{3.298364in}}%
\pgfpathcurveto{\pgfqpoint{1.780440in}{3.290550in}}{\pgfqpoint{1.791039in}{3.286160in}}{\pgfqpoint{1.802089in}{3.286160in}}%
\pgfpathclose%
\pgfusepath{stroke,fill}%
\end{pgfscope}%
\begin{pgfscope}%
\pgfpathrectangle{\pgfqpoint{0.648703in}{0.548769in}}{\pgfqpoint{5.201297in}{3.102590in}}%
\pgfusepath{clip}%
\pgfsetbuttcap%
\pgfsetroundjoin%
\definecolor{currentfill}{rgb}{1.000000,0.498039,0.054902}%
\pgfsetfillcolor{currentfill}%
\pgfsetlinewidth{1.003750pt}%
\definecolor{currentstroke}{rgb}{1.000000,0.498039,0.054902}%
\pgfsetstrokecolor{currentstroke}%
\pgfsetdash{}{0pt}%
\pgfpathmoveto{\pgfqpoint{1.623147in}{3.136837in}}%
\pgfpathcurveto{\pgfqpoint{1.634197in}{3.136837in}}{\pgfqpoint{1.644796in}{3.141228in}}{\pgfqpoint{1.652610in}{3.149041in}}%
\pgfpathcurveto{\pgfqpoint{1.660424in}{3.156855in}}{\pgfqpoint{1.664814in}{3.167454in}}{\pgfqpoint{1.664814in}{3.178504in}}%
\pgfpathcurveto{\pgfqpoint{1.664814in}{3.189554in}}{\pgfqpoint{1.660424in}{3.200153in}}{\pgfqpoint{1.652610in}{3.207967in}}%
\pgfpathcurveto{\pgfqpoint{1.644796in}{3.215780in}}{\pgfqpoint{1.634197in}{3.220171in}}{\pgfqpoint{1.623147in}{3.220171in}}%
\pgfpathcurveto{\pgfqpoint{1.612097in}{3.220171in}}{\pgfqpoint{1.601498in}{3.215780in}}{\pgfqpoint{1.593685in}{3.207967in}}%
\pgfpathcurveto{\pgfqpoint{1.585871in}{3.200153in}}{\pgfqpoint{1.581481in}{3.189554in}}{\pgfqpoint{1.581481in}{3.178504in}}%
\pgfpathcurveto{\pgfqpoint{1.581481in}{3.167454in}}{\pgfqpoint{1.585871in}{3.156855in}}{\pgfqpoint{1.593685in}{3.149041in}}%
\pgfpathcurveto{\pgfqpoint{1.601498in}{3.141228in}}{\pgfqpoint{1.612097in}{3.136837in}}{\pgfqpoint{1.623147in}{3.136837in}}%
\pgfpathclose%
\pgfusepath{stroke,fill}%
\end{pgfscope}%
\begin{pgfscope}%
\pgfpathrectangle{\pgfqpoint{0.648703in}{0.548769in}}{\pgfqpoint{5.201297in}{3.102590in}}%
\pgfusepath{clip}%
\pgfsetbuttcap%
\pgfsetroundjoin%
\definecolor{currentfill}{rgb}{1.000000,0.498039,0.054902}%
\pgfsetfillcolor{currentfill}%
\pgfsetlinewidth{1.003750pt}%
\definecolor{currentstroke}{rgb}{1.000000,0.498039,0.054902}%
\pgfsetstrokecolor{currentstroke}%
\pgfsetdash{}{0pt}%
\pgfpathmoveto{\pgfqpoint{2.583140in}{3.149281in}}%
\pgfpathcurveto{\pgfqpoint{2.594191in}{3.149281in}}{\pgfqpoint{2.604790in}{3.153671in}}{\pgfqpoint{2.612603in}{3.161485in}}%
\pgfpathcurveto{\pgfqpoint{2.620417in}{3.169298in}}{\pgfqpoint{2.624807in}{3.179897in}}{\pgfqpoint{2.624807in}{3.190948in}}%
\pgfpathcurveto{\pgfqpoint{2.624807in}{3.201998in}}{\pgfqpoint{2.620417in}{3.212597in}}{\pgfqpoint{2.612603in}{3.220410in}}%
\pgfpathcurveto{\pgfqpoint{2.604790in}{3.228224in}}{\pgfqpoint{2.594191in}{3.232614in}}{\pgfqpoint{2.583140in}{3.232614in}}%
\pgfpathcurveto{\pgfqpoint{2.572090in}{3.232614in}}{\pgfqpoint{2.561491in}{3.228224in}}{\pgfqpoint{2.553678in}{3.220410in}}%
\pgfpathcurveto{\pgfqpoint{2.545864in}{3.212597in}}{\pgfqpoint{2.541474in}{3.201998in}}{\pgfqpoint{2.541474in}{3.190948in}}%
\pgfpathcurveto{\pgfqpoint{2.541474in}{3.179897in}}{\pgfqpoint{2.545864in}{3.169298in}}{\pgfqpoint{2.553678in}{3.161485in}}%
\pgfpathcurveto{\pgfqpoint{2.561491in}{3.153671in}}{\pgfqpoint{2.572090in}{3.149281in}}{\pgfqpoint{2.583140in}{3.149281in}}%
\pgfpathclose%
\pgfusepath{stroke,fill}%
\end{pgfscope}%
\begin{pgfscope}%
\pgfpathrectangle{\pgfqpoint{0.648703in}{0.548769in}}{\pgfqpoint{5.201297in}{3.102590in}}%
\pgfusepath{clip}%
\pgfsetbuttcap%
\pgfsetroundjoin%
\definecolor{currentfill}{rgb}{0.121569,0.466667,0.705882}%
\pgfsetfillcolor{currentfill}%
\pgfsetlinewidth{1.003750pt}%
\definecolor{currentstroke}{rgb}{0.121569,0.466667,0.705882}%
\pgfsetstrokecolor{currentstroke}%
\pgfsetdash{}{0pt}%
\pgfpathmoveto{\pgfqpoint{2.962695in}{3.128542in}}%
\pgfpathcurveto{\pgfqpoint{2.973745in}{3.128542in}}{\pgfqpoint{2.984344in}{3.132932in}}{\pgfqpoint{2.992158in}{3.140746in}}%
\pgfpathcurveto{\pgfqpoint{2.999972in}{3.148559in}}{\pgfqpoint{3.004362in}{3.159158in}}{\pgfqpoint{3.004362in}{3.170208in}}%
\pgfpathcurveto{\pgfqpoint{3.004362in}{3.181258in}}{\pgfqpoint{2.999972in}{3.191857in}}{\pgfqpoint{2.992158in}{3.199671in}}%
\pgfpathcurveto{\pgfqpoint{2.984344in}{3.207485in}}{\pgfqpoint{2.973745in}{3.211875in}}{\pgfqpoint{2.962695in}{3.211875in}}%
\pgfpathcurveto{\pgfqpoint{2.951645in}{3.211875in}}{\pgfqpoint{2.941046in}{3.207485in}}{\pgfqpoint{2.933232in}{3.199671in}}%
\pgfpathcurveto{\pgfqpoint{2.925419in}{3.191857in}}{\pgfqpoint{2.921029in}{3.181258in}}{\pgfqpoint{2.921029in}{3.170208in}}%
\pgfpathcurveto{\pgfqpoint{2.921029in}{3.159158in}}{\pgfqpoint{2.925419in}{3.148559in}}{\pgfqpoint{2.933232in}{3.140746in}}%
\pgfpathcurveto{\pgfqpoint{2.941046in}{3.132932in}}{\pgfqpoint{2.951645in}{3.128542in}}{\pgfqpoint{2.962695in}{3.128542in}}%
\pgfpathclose%
\pgfusepath{stroke,fill}%
\end{pgfscope}%
\begin{pgfscope}%
\pgfpathrectangle{\pgfqpoint{0.648703in}{0.548769in}}{\pgfqpoint{5.201297in}{3.102590in}}%
\pgfusepath{clip}%
\pgfsetbuttcap%
\pgfsetroundjoin%
\definecolor{currentfill}{rgb}{1.000000,0.498039,0.054902}%
\pgfsetfillcolor{currentfill}%
\pgfsetlinewidth{1.003750pt}%
\definecolor{currentstroke}{rgb}{1.000000,0.498039,0.054902}%
\pgfsetstrokecolor{currentstroke}%
\pgfsetdash{}{0pt}%
\pgfpathmoveto{\pgfqpoint{2.282547in}{3.232238in}}%
\pgfpathcurveto{\pgfqpoint{2.293597in}{3.232238in}}{\pgfqpoint{2.304196in}{3.236628in}}{\pgfqpoint{2.312010in}{3.244442in}}%
\pgfpathcurveto{\pgfqpoint{2.319823in}{3.252255in}}{\pgfqpoint{2.324214in}{3.262854in}}{\pgfqpoint{2.324214in}{3.273905in}}%
\pgfpathcurveto{\pgfqpoint{2.324214in}{3.284955in}}{\pgfqpoint{2.319823in}{3.295554in}}{\pgfqpoint{2.312010in}{3.303367in}}%
\pgfpathcurveto{\pgfqpoint{2.304196in}{3.311181in}}{\pgfqpoint{2.293597in}{3.315571in}}{\pgfqpoint{2.282547in}{3.315571in}}%
\pgfpathcurveto{\pgfqpoint{2.271497in}{3.315571in}}{\pgfqpoint{2.260898in}{3.311181in}}{\pgfqpoint{2.253084in}{3.303367in}}%
\pgfpathcurveto{\pgfqpoint{2.245271in}{3.295554in}}{\pgfqpoint{2.240880in}{3.284955in}}{\pgfqpoint{2.240880in}{3.273905in}}%
\pgfpathcurveto{\pgfqpoint{2.240880in}{3.262854in}}{\pgfqpoint{2.245271in}{3.252255in}}{\pgfqpoint{2.253084in}{3.244442in}}%
\pgfpathcurveto{\pgfqpoint{2.260898in}{3.236628in}}{\pgfqpoint{2.271497in}{3.232238in}}{\pgfqpoint{2.282547in}{3.232238in}}%
\pgfpathclose%
\pgfusepath{stroke,fill}%
\end{pgfscope}%
\begin{pgfscope}%
\pgfpathrectangle{\pgfqpoint{0.648703in}{0.548769in}}{\pgfqpoint{5.201297in}{3.102590in}}%
\pgfusepath{clip}%
\pgfsetbuttcap%
\pgfsetroundjoin%
\definecolor{currentfill}{rgb}{0.121569,0.466667,0.705882}%
\pgfsetfillcolor{currentfill}%
\pgfsetlinewidth{1.003750pt}%
\definecolor{currentstroke}{rgb}{0.121569,0.466667,0.705882}%
\pgfsetstrokecolor{currentstroke}%
\pgfsetdash{}{0pt}%
\pgfpathmoveto{\pgfqpoint{1.603549in}{0.648129in}}%
\pgfpathcurveto{\pgfqpoint{1.614599in}{0.648129in}}{\pgfqpoint{1.625198in}{0.652519in}}{\pgfqpoint{1.633011in}{0.660333in}}%
\pgfpathcurveto{\pgfqpoint{1.640825in}{0.668146in}}{\pgfqpoint{1.645215in}{0.678745in}}{\pgfqpoint{1.645215in}{0.689796in}}%
\pgfpathcurveto{\pgfqpoint{1.645215in}{0.700846in}}{\pgfqpoint{1.640825in}{0.711445in}}{\pgfqpoint{1.633011in}{0.719258in}}%
\pgfpathcurveto{\pgfqpoint{1.625198in}{0.727072in}}{\pgfqpoint{1.614599in}{0.731462in}}{\pgfqpoint{1.603549in}{0.731462in}}%
\pgfpathcurveto{\pgfqpoint{1.592498in}{0.731462in}}{\pgfqpoint{1.581899in}{0.727072in}}{\pgfqpoint{1.574086in}{0.719258in}}%
\pgfpathcurveto{\pgfqpoint{1.566272in}{0.711445in}}{\pgfqpoint{1.561882in}{0.700846in}}{\pgfqpoint{1.561882in}{0.689796in}}%
\pgfpathcurveto{\pgfqpoint{1.561882in}{0.678745in}}{\pgfqpoint{1.566272in}{0.668146in}}{\pgfqpoint{1.574086in}{0.660333in}}%
\pgfpathcurveto{\pgfqpoint{1.581899in}{0.652519in}}{\pgfqpoint{1.592498in}{0.648129in}}{\pgfqpoint{1.603549in}{0.648129in}}%
\pgfpathclose%
\pgfusepath{stroke,fill}%
\end{pgfscope}%
\begin{pgfscope}%
\pgfpathrectangle{\pgfqpoint{0.648703in}{0.548769in}}{\pgfqpoint{5.201297in}{3.102590in}}%
\pgfusepath{clip}%
\pgfsetbuttcap%
\pgfsetroundjoin%
\definecolor{currentfill}{rgb}{1.000000,0.498039,0.054902}%
\pgfsetfillcolor{currentfill}%
\pgfsetlinewidth{1.003750pt}%
\definecolor{currentstroke}{rgb}{1.000000,0.498039,0.054902}%
\pgfsetstrokecolor{currentstroke}%
\pgfsetdash{}{0pt}%
\pgfpathmoveto{\pgfqpoint{2.131697in}{3.136837in}}%
\pgfpathcurveto{\pgfqpoint{2.142747in}{3.136837in}}{\pgfqpoint{2.153346in}{3.141228in}}{\pgfqpoint{2.161160in}{3.149041in}}%
\pgfpathcurveto{\pgfqpoint{2.168974in}{3.156855in}}{\pgfqpoint{2.173364in}{3.167454in}}{\pgfqpoint{2.173364in}{3.178504in}}%
\pgfpathcurveto{\pgfqpoint{2.173364in}{3.189554in}}{\pgfqpoint{2.168974in}{3.200153in}}{\pgfqpoint{2.161160in}{3.207967in}}%
\pgfpathcurveto{\pgfqpoint{2.153346in}{3.215780in}}{\pgfqpoint{2.142747in}{3.220171in}}{\pgfqpoint{2.131697in}{3.220171in}}%
\pgfpathcurveto{\pgfqpoint{2.120647in}{3.220171in}}{\pgfqpoint{2.110048in}{3.215780in}}{\pgfqpoint{2.102234in}{3.207967in}}%
\pgfpathcurveto{\pgfqpoint{2.094421in}{3.200153in}}{\pgfqpoint{2.090030in}{3.189554in}}{\pgfqpoint{2.090030in}{3.178504in}}%
\pgfpathcurveto{\pgfqpoint{2.090030in}{3.167454in}}{\pgfqpoint{2.094421in}{3.156855in}}{\pgfqpoint{2.102234in}{3.149041in}}%
\pgfpathcurveto{\pgfqpoint{2.110048in}{3.141228in}}{\pgfqpoint{2.120647in}{3.136837in}}{\pgfqpoint{2.131697in}{3.136837in}}%
\pgfpathclose%
\pgfusepath{stroke,fill}%
\end{pgfscope}%
\begin{pgfscope}%
\pgfpathrectangle{\pgfqpoint{0.648703in}{0.548769in}}{\pgfqpoint{5.201297in}{3.102590in}}%
\pgfusepath{clip}%
\pgfsetbuttcap%
\pgfsetroundjoin%
\definecolor{currentfill}{rgb}{1.000000,0.498039,0.054902}%
\pgfsetfillcolor{currentfill}%
\pgfsetlinewidth{1.003750pt}%
\definecolor{currentstroke}{rgb}{1.000000,0.498039,0.054902}%
\pgfsetstrokecolor{currentstroke}%
\pgfsetdash{}{0pt}%
\pgfpathmoveto{\pgfqpoint{1.512175in}{3.140985in}}%
\pgfpathcurveto{\pgfqpoint{1.523225in}{3.140985in}}{\pgfqpoint{1.533824in}{3.145375in}}{\pgfqpoint{1.541637in}{3.153189in}}%
\pgfpathcurveto{\pgfqpoint{1.549451in}{3.161003in}}{\pgfqpoint{1.553841in}{3.171602in}}{\pgfqpoint{1.553841in}{3.182652in}}%
\pgfpathcurveto{\pgfqpoint{1.553841in}{3.193702in}}{\pgfqpoint{1.549451in}{3.204301in}}{\pgfqpoint{1.541637in}{3.212115in}}%
\pgfpathcurveto{\pgfqpoint{1.533824in}{3.219928in}}{\pgfqpoint{1.523225in}{3.224319in}}{\pgfqpoint{1.512175in}{3.224319in}}%
\pgfpathcurveto{\pgfqpoint{1.501124in}{3.224319in}}{\pgfqpoint{1.490525in}{3.219928in}}{\pgfqpoint{1.482712in}{3.212115in}}%
\pgfpathcurveto{\pgfqpoint{1.474898in}{3.204301in}}{\pgfqpoint{1.470508in}{3.193702in}}{\pgfqpoint{1.470508in}{3.182652in}}%
\pgfpathcurveto{\pgfqpoint{1.470508in}{3.171602in}}{\pgfqpoint{1.474898in}{3.161003in}}{\pgfqpoint{1.482712in}{3.153189in}}%
\pgfpathcurveto{\pgfqpoint{1.490525in}{3.145375in}}{\pgfqpoint{1.501124in}{3.140985in}}{\pgfqpoint{1.512175in}{3.140985in}}%
\pgfpathclose%
\pgfusepath{stroke,fill}%
\end{pgfscope}%
\begin{pgfscope}%
\pgfpathrectangle{\pgfqpoint{0.648703in}{0.548769in}}{\pgfqpoint{5.201297in}{3.102590in}}%
\pgfusepath{clip}%
\pgfsetbuttcap%
\pgfsetroundjoin%
\definecolor{currentfill}{rgb}{1.000000,0.498039,0.054902}%
\pgfsetfillcolor{currentfill}%
\pgfsetlinewidth{1.003750pt}%
\definecolor{currentstroke}{rgb}{1.000000,0.498039,0.054902}%
\pgfsetstrokecolor{currentstroke}%
\pgfsetdash{}{0pt}%
\pgfpathmoveto{\pgfqpoint{1.751672in}{3.149281in}}%
\pgfpathcurveto{\pgfqpoint{1.762722in}{3.149281in}}{\pgfqpoint{1.773321in}{3.153671in}}{\pgfqpoint{1.781135in}{3.161485in}}%
\pgfpathcurveto{\pgfqpoint{1.788948in}{3.169298in}}{\pgfqpoint{1.793339in}{3.179897in}}{\pgfqpoint{1.793339in}{3.190948in}}%
\pgfpathcurveto{\pgfqpoint{1.793339in}{3.201998in}}{\pgfqpoint{1.788948in}{3.212597in}}{\pgfqpoint{1.781135in}{3.220410in}}%
\pgfpathcurveto{\pgfqpoint{1.773321in}{3.228224in}}{\pgfqpoint{1.762722in}{3.232614in}}{\pgfqpoint{1.751672in}{3.232614in}}%
\pgfpathcurveto{\pgfqpoint{1.740622in}{3.232614in}}{\pgfqpoint{1.730023in}{3.228224in}}{\pgfqpoint{1.722209in}{3.220410in}}%
\pgfpathcurveto{\pgfqpoint{1.714396in}{3.212597in}}{\pgfqpoint{1.710005in}{3.201998in}}{\pgfqpoint{1.710005in}{3.190948in}}%
\pgfpathcurveto{\pgfqpoint{1.710005in}{3.179897in}}{\pgfqpoint{1.714396in}{3.169298in}}{\pgfqpoint{1.722209in}{3.161485in}}%
\pgfpathcurveto{\pgfqpoint{1.730023in}{3.153671in}}{\pgfqpoint{1.740622in}{3.149281in}}{\pgfqpoint{1.751672in}{3.149281in}}%
\pgfpathclose%
\pgfusepath{stroke,fill}%
\end{pgfscope}%
\begin{pgfscope}%
\pgfpathrectangle{\pgfqpoint{0.648703in}{0.548769in}}{\pgfqpoint{5.201297in}{3.102590in}}%
\pgfusepath{clip}%
\pgfsetbuttcap%
\pgfsetroundjoin%
\definecolor{currentfill}{rgb}{1.000000,0.498039,0.054902}%
\pgfsetfillcolor{currentfill}%
\pgfsetlinewidth{1.003750pt}%
\definecolor{currentstroke}{rgb}{1.000000,0.498039,0.054902}%
\pgfsetstrokecolor{currentstroke}%
\pgfsetdash{}{0pt}%
\pgfpathmoveto{\pgfqpoint{1.980473in}{3.140985in}}%
\pgfpathcurveto{\pgfqpoint{1.991523in}{3.140985in}}{\pgfqpoint{2.002122in}{3.145375in}}{\pgfqpoint{2.009936in}{3.153189in}}%
\pgfpathcurveto{\pgfqpoint{2.017749in}{3.161003in}}{\pgfqpoint{2.022139in}{3.171602in}}{\pgfqpoint{2.022139in}{3.182652in}}%
\pgfpathcurveto{\pgfqpoint{2.022139in}{3.193702in}}{\pgfqpoint{2.017749in}{3.204301in}}{\pgfqpoint{2.009936in}{3.212115in}}%
\pgfpathcurveto{\pgfqpoint{2.002122in}{3.219928in}}{\pgfqpoint{1.991523in}{3.224319in}}{\pgfqpoint{1.980473in}{3.224319in}}%
\pgfpathcurveto{\pgfqpoint{1.969423in}{3.224319in}}{\pgfqpoint{1.958824in}{3.219928in}}{\pgfqpoint{1.951010in}{3.212115in}}%
\pgfpathcurveto{\pgfqpoint{1.943196in}{3.204301in}}{\pgfqpoint{1.938806in}{3.193702in}}{\pgfqpoint{1.938806in}{3.182652in}}%
\pgfpathcurveto{\pgfqpoint{1.938806in}{3.171602in}}{\pgfqpoint{1.943196in}{3.161003in}}{\pgfqpoint{1.951010in}{3.153189in}}%
\pgfpathcurveto{\pgfqpoint{1.958824in}{3.145375in}}{\pgfqpoint{1.969423in}{3.140985in}}{\pgfqpoint{1.980473in}{3.140985in}}%
\pgfpathclose%
\pgfusepath{stroke,fill}%
\end{pgfscope}%
\begin{pgfscope}%
\pgfpathrectangle{\pgfqpoint{0.648703in}{0.548769in}}{\pgfqpoint{5.201297in}{3.102590in}}%
\pgfusepath{clip}%
\pgfsetbuttcap%
\pgfsetroundjoin%
\definecolor{currentfill}{rgb}{1.000000,0.498039,0.054902}%
\pgfsetfillcolor{currentfill}%
\pgfsetlinewidth{1.003750pt}%
\definecolor{currentstroke}{rgb}{1.000000,0.498039,0.054902}%
\pgfsetstrokecolor{currentstroke}%
\pgfsetdash{}{0pt}%
\pgfpathmoveto{\pgfqpoint{1.767168in}{3.136837in}}%
\pgfpathcurveto{\pgfqpoint{1.778218in}{3.136837in}}{\pgfqpoint{1.788817in}{3.141228in}}{\pgfqpoint{1.796631in}{3.149041in}}%
\pgfpathcurveto{\pgfqpoint{1.804444in}{3.156855in}}{\pgfqpoint{1.808835in}{3.167454in}}{\pgfqpoint{1.808835in}{3.178504in}}%
\pgfpathcurveto{\pgfqpoint{1.808835in}{3.189554in}}{\pgfqpoint{1.804444in}{3.200153in}}{\pgfqpoint{1.796631in}{3.207967in}}%
\pgfpathcurveto{\pgfqpoint{1.788817in}{3.215780in}}{\pgfqpoint{1.778218in}{3.220171in}}{\pgfqpoint{1.767168in}{3.220171in}}%
\pgfpathcurveto{\pgfqpoint{1.756118in}{3.220171in}}{\pgfqpoint{1.745519in}{3.215780in}}{\pgfqpoint{1.737705in}{3.207967in}}%
\pgfpathcurveto{\pgfqpoint{1.729892in}{3.200153in}}{\pgfqpoint{1.725501in}{3.189554in}}{\pgfqpoint{1.725501in}{3.178504in}}%
\pgfpathcurveto{\pgfqpoint{1.725501in}{3.167454in}}{\pgfqpoint{1.729892in}{3.156855in}}{\pgfqpoint{1.737705in}{3.149041in}}%
\pgfpathcurveto{\pgfqpoint{1.745519in}{3.141228in}}{\pgfqpoint{1.756118in}{3.136837in}}{\pgfqpoint{1.767168in}{3.136837in}}%
\pgfpathclose%
\pgfusepath{stroke,fill}%
\end{pgfscope}%
\begin{pgfscope}%
\pgfpathrectangle{\pgfqpoint{0.648703in}{0.548769in}}{\pgfqpoint{5.201297in}{3.102590in}}%
\pgfusepath{clip}%
\pgfsetbuttcap%
\pgfsetroundjoin%
\definecolor{currentfill}{rgb}{1.000000,0.498039,0.054902}%
\pgfsetfillcolor{currentfill}%
\pgfsetlinewidth{1.003750pt}%
\definecolor{currentstroke}{rgb}{1.000000,0.498039,0.054902}%
\pgfsetstrokecolor{currentstroke}%
\pgfsetdash{}{0pt}%
\pgfpathmoveto{\pgfqpoint{1.596023in}{3.145133in}}%
\pgfpathcurveto{\pgfqpoint{1.607073in}{3.145133in}}{\pgfqpoint{1.617672in}{3.149523in}}{\pgfqpoint{1.625485in}{3.157337in}}%
\pgfpathcurveto{\pgfqpoint{1.633299in}{3.165151in}}{\pgfqpoint{1.637689in}{3.175750in}}{\pgfqpoint{1.637689in}{3.186800in}}%
\pgfpathcurveto{\pgfqpoint{1.637689in}{3.197850in}}{\pgfqpoint{1.633299in}{3.208449in}}{\pgfqpoint{1.625485in}{3.216262in}}%
\pgfpathcurveto{\pgfqpoint{1.617672in}{3.224076in}}{\pgfqpoint{1.607073in}{3.228466in}}{\pgfqpoint{1.596023in}{3.228466in}}%
\pgfpathcurveto{\pgfqpoint{1.584972in}{3.228466in}}{\pgfqpoint{1.574373in}{3.224076in}}{\pgfqpoint{1.566560in}{3.216262in}}%
\pgfpathcurveto{\pgfqpoint{1.558746in}{3.208449in}}{\pgfqpoint{1.554356in}{3.197850in}}{\pgfqpoint{1.554356in}{3.186800in}}%
\pgfpathcurveto{\pgfqpoint{1.554356in}{3.175750in}}{\pgfqpoint{1.558746in}{3.165151in}}{\pgfqpoint{1.566560in}{3.157337in}}%
\pgfpathcurveto{\pgfqpoint{1.574373in}{3.149523in}}{\pgfqpoint{1.584972in}{3.145133in}}{\pgfqpoint{1.596023in}{3.145133in}}%
\pgfpathclose%
\pgfusepath{stroke,fill}%
\end{pgfscope}%
\begin{pgfscope}%
\pgfpathrectangle{\pgfqpoint{0.648703in}{0.548769in}}{\pgfqpoint{5.201297in}{3.102590in}}%
\pgfusepath{clip}%
\pgfsetbuttcap%
\pgfsetroundjoin%
\definecolor{currentfill}{rgb}{0.121569,0.466667,0.705882}%
\pgfsetfillcolor{currentfill}%
\pgfsetlinewidth{1.003750pt}%
\definecolor{currentstroke}{rgb}{0.121569,0.466667,0.705882}%
\pgfsetstrokecolor{currentstroke}%
\pgfsetdash{}{0pt}%
\pgfpathmoveto{\pgfqpoint{2.589969in}{3.132690in}}%
\pgfpathcurveto{\pgfqpoint{2.601020in}{3.132690in}}{\pgfqpoint{2.611619in}{3.137080in}}{\pgfqpoint{2.619432in}{3.144893in}}%
\pgfpathcurveto{\pgfqpoint{2.627246in}{3.152707in}}{\pgfqpoint{2.631636in}{3.163306in}}{\pgfqpoint{2.631636in}{3.174356in}}%
\pgfpathcurveto{\pgfqpoint{2.631636in}{3.185406in}}{\pgfqpoint{2.627246in}{3.196005in}}{\pgfqpoint{2.619432in}{3.203819in}}%
\pgfpathcurveto{\pgfqpoint{2.611619in}{3.211633in}}{\pgfqpoint{2.601020in}{3.216023in}}{\pgfqpoint{2.589969in}{3.216023in}}%
\pgfpathcurveto{\pgfqpoint{2.578919in}{3.216023in}}{\pgfqpoint{2.568320in}{3.211633in}}{\pgfqpoint{2.560507in}{3.203819in}}%
\pgfpathcurveto{\pgfqpoint{2.552693in}{3.196005in}}{\pgfqpoint{2.548303in}{3.185406in}}{\pgfqpoint{2.548303in}{3.174356in}}%
\pgfpathcurveto{\pgfqpoint{2.548303in}{3.163306in}}{\pgfqpoint{2.552693in}{3.152707in}}{\pgfqpoint{2.560507in}{3.144893in}}%
\pgfpathcurveto{\pgfqpoint{2.568320in}{3.137080in}}{\pgfqpoint{2.578919in}{3.132690in}}{\pgfqpoint{2.589969in}{3.132690in}}%
\pgfpathclose%
\pgfusepath{stroke,fill}%
\end{pgfscope}%
\begin{pgfscope}%
\pgfpathrectangle{\pgfqpoint{0.648703in}{0.548769in}}{\pgfqpoint{5.201297in}{3.102590in}}%
\pgfusepath{clip}%
\pgfsetbuttcap%
\pgfsetroundjoin%
\definecolor{currentfill}{rgb}{1.000000,0.498039,0.054902}%
\pgfsetfillcolor{currentfill}%
\pgfsetlinewidth{1.003750pt}%
\definecolor{currentstroke}{rgb}{1.000000,0.498039,0.054902}%
\pgfsetstrokecolor{currentstroke}%
\pgfsetdash{}{0pt}%
\pgfpathmoveto{\pgfqpoint{1.738641in}{3.348378in}}%
\pgfpathcurveto{\pgfqpoint{1.749691in}{3.348378in}}{\pgfqpoint{1.760290in}{3.352768in}}{\pgfqpoint{1.768104in}{3.360581in}}%
\pgfpathcurveto{\pgfqpoint{1.775917in}{3.368395in}}{\pgfqpoint{1.780308in}{3.378994in}}{\pgfqpoint{1.780308in}{3.390044in}}%
\pgfpathcurveto{\pgfqpoint{1.780308in}{3.401094in}}{\pgfqpoint{1.775917in}{3.411693in}}{\pgfqpoint{1.768104in}{3.419507in}}%
\pgfpathcurveto{\pgfqpoint{1.760290in}{3.427321in}}{\pgfqpoint{1.749691in}{3.431711in}}{\pgfqpoint{1.738641in}{3.431711in}}%
\pgfpathcurveto{\pgfqpoint{1.727591in}{3.431711in}}{\pgfqpoint{1.716992in}{3.427321in}}{\pgfqpoint{1.709178in}{3.419507in}}%
\pgfpathcurveto{\pgfqpoint{1.701365in}{3.411693in}}{\pgfqpoint{1.696974in}{3.401094in}}{\pgfqpoint{1.696974in}{3.390044in}}%
\pgfpathcurveto{\pgfqpoint{1.696974in}{3.378994in}}{\pgfqpoint{1.701365in}{3.368395in}}{\pgfqpoint{1.709178in}{3.360581in}}%
\pgfpathcurveto{\pgfqpoint{1.716992in}{3.352768in}}{\pgfqpoint{1.727591in}{3.348378in}}{\pgfqpoint{1.738641in}{3.348378in}}%
\pgfpathclose%
\pgfusepath{stroke,fill}%
\end{pgfscope}%
\begin{pgfscope}%
\pgfsetbuttcap%
\pgfsetroundjoin%
\definecolor{currentfill}{rgb}{0.000000,0.000000,0.000000}%
\pgfsetfillcolor{currentfill}%
\pgfsetlinewidth{0.803000pt}%
\definecolor{currentstroke}{rgb}{0.000000,0.000000,0.000000}%
\pgfsetstrokecolor{currentstroke}%
\pgfsetdash{}{0pt}%
\pgfsys@defobject{currentmarker}{\pgfqpoint{0.000000in}{-0.048611in}}{\pgfqpoint{0.000000in}{0.000000in}}{%
\pgfpathmoveto{\pgfqpoint{0.000000in}{0.000000in}}%
\pgfpathlineto{\pgfqpoint{0.000000in}{-0.048611in}}%
\pgfusepath{stroke,fill}%
}%
\begin{pgfscope}%
\pgfsys@transformshift{0.848324in}{0.548769in}%
\pgfsys@useobject{currentmarker}{}%
\end{pgfscope}%
\end{pgfscope}%
\begin{pgfscope}%
\definecolor{textcolor}{rgb}{0.000000,0.000000,0.000000}%
\pgfsetstrokecolor{textcolor}%
\pgfsetfillcolor{textcolor}%
\pgftext[x=0.848324in,y=0.451547in,,top]{\color{textcolor}\sffamily\fontsize{10.000000}{12.000000}\selectfont \(\displaystyle {0}\)}%
\end{pgfscope}%
\begin{pgfscope}%
\pgfsetbuttcap%
\pgfsetroundjoin%
\definecolor{currentfill}{rgb}{0.000000,0.000000,0.000000}%
\pgfsetfillcolor{currentfill}%
\pgfsetlinewidth{0.803000pt}%
\definecolor{currentstroke}{rgb}{0.000000,0.000000,0.000000}%
\pgfsetstrokecolor{currentstroke}%
\pgfsetdash{}{0pt}%
\pgfsys@defobject{currentmarker}{\pgfqpoint{0.000000in}{-0.048611in}}{\pgfqpoint{0.000000in}{0.000000in}}{%
\pgfpathmoveto{\pgfqpoint{0.000000in}{0.000000in}}%
\pgfpathlineto{\pgfqpoint{0.000000in}{-0.048611in}}%
\pgfusepath{stroke,fill}%
}%
\begin{pgfscope}%
\pgfsys@transformshift{1.719382in}{0.548769in}%
\pgfsys@useobject{currentmarker}{}%
\end{pgfscope}%
\end{pgfscope}%
\begin{pgfscope}%
\definecolor{textcolor}{rgb}{0.000000,0.000000,0.000000}%
\pgfsetstrokecolor{textcolor}%
\pgfsetfillcolor{textcolor}%
\pgftext[x=1.719382in,y=0.451547in,,top]{\color{textcolor}\sffamily\fontsize{10.000000}{12.000000}\selectfont \(\displaystyle {100000}\)}%
\end{pgfscope}%
\begin{pgfscope}%
\pgfsetbuttcap%
\pgfsetroundjoin%
\definecolor{currentfill}{rgb}{0.000000,0.000000,0.000000}%
\pgfsetfillcolor{currentfill}%
\pgfsetlinewidth{0.803000pt}%
\definecolor{currentstroke}{rgb}{0.000000,0.000000,0.000000}%
\pgfsetstrokecolor{currentstroke}%
\pgfsetdash{}{0pt}%
\pgfsys@defobject{currentmarker}{\pgfqpoint{0.000000in}{-0.048611in}}{\pgfqpoint{0.000000in}{0.000000in}}{%
\pgfpathmoveto{\pgfqpoint{0.000000in}{0.000000in}}%
\pgfpathlineto{\pgfqpoint{0.000000in}{-0.048611in}}%
\pgfusepath{stroke,fill}%
}%
\begin{pgfscope}%
\pgfsys@transformshift{2.590440in}{0.548769in}%
\pgfsys@useobject{currentmarker}{}%
\end{pgfscope}%
\end{pgfscope}%
\begin{pgfscope}%
\definecolor{textcolor}{rgb}{0.000000,0.000000,0.000000}%
\pgfsetstrokecolor{textcolor}%
\pgfsetfillcolor{textcolor}%
\pgftext[x=2.590440in,y=0.451547in,,top]{\color{textcolor}\sffamily\fontsize{10.000000}{12.000000}\selectfont \(\displaystyle {200000}\)}%
\end{pgfscope}%
\begin{pgfscope}%
\pgfsetbuttcap%
\pgfsetroundjoin%
\definecolor{currentfill}{rgb}{0.000000,0.000000,0.000000}%
\pgfsetfillcolor{currentfill}%
\pgfsetlinewidth{0.803000pt}%
\definecolor{currentstroke}{rgb}{0.000000,0.000000,0.000000}%
\pgfsetstrokecolor{currentstroke}%
\pgfsetdash{}{0pt}%
\pgfsys@defobject{currentmarker}{\pgfqpoint{0.000000in}{-0.048611in}}{\pgfqpoint{0.000000in}{0.000000in}}{%
\pgfpathmoveto{\pgfqpoint{0.000000in}{0.000000in}}%
\pgfpathlineto{\pgfqpoint{0.000000in}{-0.048611in}}%
\pgfusepath{stroke,fill}%
}%
\begin{pgfscope}%
\pgfsys@transformshift{3.461498in}{0.548769in}%
\pgfsys@useobject{currentmarker}{}%
\end{pgfscope}%
\end{pgfscope}%
\begin{pgfscope}%
\definecolor{textcolor}{rgb}{0.000000,0.000000,0.000000}%
\pgfsetstrokecolor{textcolor}%
\pgfsetfillcolor{textcolor}%
\pgftext[x=3.461498in,y=0.451547in,,top]{\color{textcolor}\sffamily\fontsize{10.000000}{12.000000}\selectfont \(\displaystyle {300000}\)}%
\end{pgfscope}%
\begin{pgfscope}%
\pgfsetbuttcap%
\pgfsetroundjoin%
\definecolor{currentfill}{rgb}{0.000000,0.000000,0.000000}%
\pgfsetfillcolor{currentfill}%
\pgfsetlinewidth{0.803000pt}%
\definecolor{currentstroke}{rgb}{0.000000,0.000000,0.000000}%
\pgfsetstrokecolor{currentstroke}%
\pgfsetdash{}{0pt}%
\pgfsys@defobject{currentmarker}{\pgfqpoint{0.000000in}{-0.048611in}}{\pgfqpoint{0.000000in}{0.000000in}}{%
\pgfpathmoveto{\pgfqpoint{0.000000in}{0.000000in}}%
\pgfpathlineto{\pgfqpoint{0.000000in}{-0.048611in}}%
\pgfusepath{stroke,fill}%
}%
\begin{pgfscope}%
\pgfsys@transformshift{4.332556in}{0.548769in}%
\pgfsys@useobject{currentmarker}{}%
\end{pgfscope}%
\end{pgfscope}%
\begin{pgfscope}%
\definecolor{textcolor}{rgb}{0.000000,0.000000,0.000000}%
\pgfsetstrokecolor{textcolor}%
\pgfsetfillcolor{textcolor}%
\pgftext[x=4.332556in,y=0.451547in,,top]{\color{textcolor}\sffamily\fontsize{10.000000}{12.000000}\selectfont \(\displaystyle {400000}\)}%
\end{pgfscope}%
\begin{pgfscope}%
\pgfsetbuttcap%
\pgfsetroundjoin%
\definecolor{currentfill}{rgb}{0.000000,0.000000,0.000000}%
\pgfsetfillcolor{currentfill}%
\pgfsetlinewidth{0.803000pt}%
\definecolor{currentstroke}{rgb}{0.000000,0.000000,0.000000}%
\pgfsetstrokecolor{currentstroke}%
\pgfsetdash{}{0pt}%
\pgfsys@defobject{currentmarker}{\pgfqpoint{0.000000in}{-0.048611in}}{\pgfqpoint{0.000000in}{0.000000in}}{%
\pgfpathmoveto{\pgfqpoint{0.000000in}{0.000000in}}%
\pgfpathlineto{\pgfqpoint{0.000000in}{-0.048611in}}%
\pgfusepath{stroke,fill}%
}%
\begin{pgfscope}%
\pgfsys@transformshift{5.203614in}{0.548769in}%
\pgfsys@useobject{currentmarker}{}%
\end{pgfscope}%
\end{pgfscope}%
\begin{pgfscope}%
\definecolor{textcolor}{rgb}{0.000000,0.000000,0.000000}%
\pgfsetstrokecolor{textcolor}%
\pgfsetfillcolor{textcolor}%
\pgftext[x=5.203614in,y=0.451547in,,top]{\color{textcolor}\sffamily\fontsize{10.000000}{12.000000}\selectfont \(\displaystyle {500000}\)}%
\end{pgfscope}%
\begin{pgfscope}%
\definecolor{textcolor}{rgb}{0.000000,0.000000,0.000000}%
\pgfsetstrokecolor{textcolor}%
\pgfsetfillcolor{textcolor}%
\pgftext[x=3.249352in,y=0.272658in,,top]{\color{textcolor}\sffamily\fontsize{10.000000}{12.000000}\selectfont Methods}%
\end{pgfscope}%
\begin{pgfscope}%
\pgfsetbuttcap%
\pgfsetroundjoin%
\definecolor{currentfill}{rgb}{0.000000,0.000000,0.000000}%
\pgfsetfillcolor{currentfill}%
\pgfsetlinewidth{0.803000pt}%
\definecolor{currentstroke}{rgb}{0.000000,0.000000,0.000000}%
\pgfsetstrokecolor{currentstroke}%
\pgfsetdash{}{0pt}%
\pgfsys@defobject{currentmarker}{\pgfqpoint{-0.048611in}{0.000000in}}{\pgfqpoint{0.000000in}{0.000000in}}{%
\pgfpathmoveto{\pgfqpoint{0.000000in}{0.000000in}}%
\pgfpathlineto{\pgfqpoint{-0.048611in}{0.000000in}}%
\pgfusepath{stroke,fill}%
}%
\begin{pgfscope}%
\pgfsys@transformshift{0.648703in}{0.689796in}%
\pgfsys@useobject{currentmarker}{}%
\end{pgfscope}%
\end{pgfscope}%
\begin{pgfscope}%
\definecolor{textcolor}{rgb}{0.000000,0.000000,0.000000}%
\pgfsetstrokecolor{textcolor}%
\pgfsetfillcolor{textcolor}%
\pgftext[x=0.482036in, y=0.641601in, left, base]{\color{textcolor}\sffamily\fontsize{10.000000}{12.000000}\selectfont \(\displaystyle {0}\)}%
\end{pgfscope}%
\begin{pgfscope}%
\pgfsetbuttcap%
\pgfsetroundjoin%
\definecolor{currentfill}{rgb}{0.000000,0.000000,0.000000}%
\pgfsetfillcolor{currentfill}%
\pgfsetlinewidth{0.803000pt}%
\definecolor{currentstroke}{rgb}{0.000000,0.000000,0.000000}%
\pgfsetstrokecolor{currentstroke}%
\pgfsetdash{}{0pt}%
\pgfsys@defobject{currentmarker}{\pgfqpoint{-0.048611in}{0.000000in}}{\pgfqpoint{0.000000in}{0.000000in}}{%
\pgfpathmoveto{\pgfqpoint{0.000000in}{0.000000in}}%
\pgfpathlineto{\pgfqpoint{-0.048611in}{0.000000in}}%
\pgfusepath{stroke,fill}%
}%
\begin{pgfscope}%
\pgfsys@transformshift{0.648703in}{1.104580in}%
\pgfsys@useobject{currentmarker}{}%
\end{pgfscope}%
\end{pgfscope}%
\begin{pgfscope}%
\definecolor{textcolor}{rgb}{0.000000,0.000000,0.000000}%
\pgfsetstrokecolor{textcolor}%
\pgfsetfillcolor{textcolor}%
\pgftext[x=0.343147in, y=1.056386in, left, base]{\color{textcolor}\sffamily\fontsize{10.000000}{12.000000}\selectfont \(\displaystyle {100}\)}%
\end{pgfscope}%
\begin{pgfscope}%
\pgfsetbuttcap%
\pgfsetroundjoin%
\definecolor{currentfill}{rgb}{0.000000,0.000000,0.000000}%
\pgfsetfillcolor{currentfill}%
\pgfsetlinewidth{0.803000pt}%
\definecolor{currentstroke}{rgb}{0.000000,0.000000,0.000000}%
\pgfsetstrokecolor{currentstroke}%
\pgfsetdash{}{0pt}%
\pgfsys@defobject{currentmarker}{\pgfqpoint{-0.048611in}{0.000000in}}{\pgfqpoint{0.000000in}{0.000000in}}{%
\pgfpathmoveto{\pgfqpoint{0.000000in}{0.000000in}}%
\pgfpathlineto{\pgfqpoint{-0.048611in}{0.000000in}}%
\pgfusepath{stroke,fill}%
}%
\begin{pgfscope}%
\pgfsys@transformshift{0.648703in}{1.519365in}%
\pgfsys@useobject{currentmarker}{}%
\end{pgfscope}%
\end{pgfscope}%
\begin{pgfscope}%
\definecolor{textcolor}{rgb}{0.000000,0.000000,0.000000}%
\pgfsetstrokecolor{textcolor}%
\pgfsetfillcolor{textcolor}%
\pgftext[x=0.343147in, y=1.471171in, left, base]{\color{textcolor}\sffamily\fontsize{10.000000}{12.000000}\selectfont \(\displaystyle {200}\)}%
\end{pgfscope}%
\begin{pgfscope}%
\pgfsetbuttcap%
\pgfsetroundjoin%
\definecolor{currentfill}{rgb}{0.000000,0.000000,0.000000}%
\pgfsetfillcolor{currentfill}%
\pgfsetlinewidth{0.803000pt}%
\definecolor{currentstroke}{rgb}{0.000000,0.000000,0.000000}%
\pgfsetstrokecolor{currentstroke}%
\pgfsetdash{}{0pt}%
\pgfsys@defobject{currentmarker}{\pgfqpoint{-0.048611in}{0.000000in}}{\pgfqpoint{0.000000in}{0.000000in}}{%
\pgfpathmoveto{\pgfqpoint{0.000000in}{0.000000in}}%
\pgfpathlineto{\pgfqpoint{-0.048611in}{0.000000in}}%
\pgfusepath{stroke,fill}%
}%
\begin{pgfscope}%
\pgfsys@transformshift{0.648703in}{1.934150in}%
\pgfsys@useobject{currentmarker}{}%
\end{pgfscope}%
\end{pgfscope}%
\begin{pgfscope}%
\definecolor{textcolor}{rgb}{0.000000,0.000000,0.000000}%
\pgfsetstrokecolor{textcolor}%
\pgfsetfillcolor{textcolor}%
\pgftext[x=0.343147in, y=1.885955in, left, base]{\color{textcolor}\sffamily\fontsize{10.000000}{12.000000}\selectfont \(\displaystyle {300}\)}%
\end{pgfscope}%
\begin{pgfscope}%
\pgfsetbuttcap%
\pgfsetroundjoin%
\definecolor{currentfill}{rgb}{0.000000,0.000000,0.000000}%
\pgfsetfillcolor{currentfill}%
\pgfsetlinewidth{0.803000pt}%
\definecolor{currentstroke}{rgb}{0.000000,0.000000,0.000000}%
\pgfsetstrokecolor{currentstroke}%
\pgfsetdash{}{0pt}%
\pgfsys@defobject{currentmarker}{\pgfqpoint{-0.048611in}{0.000000in}}{\pgfqpoint{0.000000in}{0.000000in}}{%
\pgfpathmoveto{\pgfqpoint{0.000000in}{0.000000in}}%
\pgfpathlineto{\pgfqpoint{-0.048611in}{0.000000in}}%
\pgfusepath{stroke,fill}%
}%
\begin{pgfscope}%
\pgfsys@transformshift{0.648703in}{2.348935in}%
\pgfsys@useobject{currentmarker}{}%
\end{pgfscope}%
\end{pgfscope}%
\begin{pgfscope}%
\definecolor{textcolor}{rgb}{0.000000,0.000000,0.000000}%
\pgfsetstrokecolor{textcolor}%
\pgfsetfillcolor{textcolor}%
\pgftext[x=0.343147in, y=2.300740in, left, base]{\color{textcolor}\sffamily\fontsize{10.000000}{12.000000}\selectfont \(\displaystyle {400}\)}%
\end{pgfscope}%
\begin{pgfscope}%
\pgfsetbuttcap%
\pgfsetroundjoin%
\definecolor{currentfill}{rgb}{0.000000,0.000000,0.000000}%
\pgfsetfillcolor{currentfill}%
\pgfsetlinewidth{0.803000pt}%
\definecolor{currentstroke}{rgb}{0.000000,0.000000,0.000000}%
\pgfsetstrokecolor{currentstroke}%
\pgfsetdash{}{0pt}%
\pgfsys@defobject{currentmarker}{\pgfqpoint{-0.048611in}{0.000000in}}{\pgfqpoint{0.000000in}{0.000000in}}{%
\pgfpathmoveto{\pgfqpoint{0.000000in}{0.000000in}}%
\pgfpathlineto{\pgfqpoint{-0.048611in}{0.000000in}}%
\pgfusepath{stroke,fill}%
}%
\begin{pgfscope}%
\pgfsys@transformshift{0.648703in}{2.763719in}%
\pgfsys@useobject{currentmarker}{}%
\end{pgfscope}%
\end{pgfscope}%
\begin{pgfscope}%
\definecolor{textcolor}{rgb}{0.000000,0.000000,0.000000}%
\pgfsetstrokecolor{textcolor}%
\pgfsetfillcolor{textcolor}%
\pgftext[x=0.343147in, y=2.715525in, left, base]{\color{textcolor}\sffamily\fontsize{10.000000}{12.000000}\selectfont \(\displaystyle {500}\)}%
\end{pgfscope}%
\begin{pgfscope}%
\pgfsetbuttcap%
\pgfsetroundjoin%
\definecolor{currentfill}{rgb}{0.000000,0.000000,0.000000}%
\pgfsetfillcolor{currentfill}%
\pgfsetlinewidth{0.803000pt}%
\definecolor{currentstroke}{rgb}{0.000000,0.000000,0.000000}%
\pgfsetstrokecolor{currentstroke}%
\pgfsetdash{}{0pt}%
\pgfsys@defobject{currentmarker}{\pgfqpoint{-0.048611in}{0.000000in}}{\pgfqpoint{0.000000in}{0.000000in}}{%
\pgfpathmoveto{\pgfqpoint{0.000000in}{0.000000in}}%
\pgfpathlineto{\pgfqpoint{-0.048611in}{0.000000in}}%
\pgfusepath{stroke,fill}%
}%
\begin{pgfscope}%
\pgfsys@transformshift{0.648703in}{3.178504in}%
\pgfsys@useobject{currentmarker}{}%
\end{pgfscope}%
\end{pgfscope}%
\begin{pgfscope}%
\definecolor{textcolor}{rgb}{0.000000,0.000000,0.000000}%
\pgfsetstrokecolor{textcolor}%
\pgfsetfillcolor{textcolor}%
\pgftext[x=0.343147in, y=3.130310in, left, base]{\color{textcolor}\sffamily\fontsize{10.000000}{12.000000}\selectfont \(\displaystyle {600}\)}%
\end{pgfscope}%
\begin{pgfscope}%
\pgfsetbuttcap%
\pgfsetroundjoin%
\definecolor{currentfill}{rgb}{0.000000,0.000000,0.000000}%
\pgfsetfillcolor{currentfill}%
\pgfsetlinewidth{0.803000pt}%
\definecolor{currentstroke}{rgb}{0.000000,0.000000,0.000000}%
\pgfsetstrokecolor{currentstroke}%
\pgfsetdash{}{0pt}%
\pgfsys@defobject{currentmarker}{\pgfqpoint{-0.048611in}{0.000000in}}{\pgfqpoint{0.000000in}{0.000000in}}{%
\pgfpathmoveto{\pgfqpoint{0.000000in}{0.000000in}}%
\pgfpathlineto{\pgfqpoint{-0.048611in}{0.000000in}}%
\pgfusepath{stroke,fill}%
}%
\begin{pgfscope}%
\pgfsys@transformshift{0.648703in}{3.593289in}%
\pgfsys@useobject{currentmarker}{}%
\end{pgfscope}%
\end{pgfscope}%
\begin{pgfscope}%
\definecolor{textcolor}{rgb}{0.000000,0.000000,0.000000}%
\pgfsetstrokecolor{textcolor}%
\pgfsetfillcolor{textcolor}%
\pgftext[x=0.343147in, y=3.545094in, left, base]{\color{textcolor}\sffamily\fontsize{10.000000}{12.000000}\selectfont \(\displaystyle {700}\)}%
\end{pgfscope}%
\begin{pgfscope}%
\definecolor{textcolor}{rgb}{0.000000,0.000000,0.000000}%
\pgfsetstrokecolor{textcolor}%
\pgfsetfillcolor{textcolor}%
\pgftext[x=0.287592in,y=2.100064in,,bottom,rotate=90.000000]{\color{textcolor}\sffamily\fontsize{10.000000}{12.000000}\selectfont Data Flow Time (s)}%
\end{pgfscope}%
\begin{pgfscope}%
\pgfsetrectcap%
\pgfsetmiterjoin%
\pgfsetlinewidth{0.803000pt}%
\definecolor{currentstroke}{rgb}{0.000000,0.000000,0.000000}%
\pgfsetstrokecolor{currentstroke}%
\pgfsetdash{}{0pt}%
\pgfpathmoveto{\pgfqpoint{0.648703in}{0.548769in}}%
\pgfpathlineto{\pgfqpoint{0.648703in}{3.651359in}}%
\pgfusepath{stroke}%
\end{pgfscope}%
\begin{pgfscope}%
\pgfsetrectcap%
\pgfsetmiterjoin%
\pgfsetlinewidth{0.803000pt}%
\definecolor{currentstroke}{rgb}{0.000000,0.000000,0.000000}%
\pgfsetstrokecolor{currentstroke}%
\pgfsetdash{}{0pt}%
\pgfpathmoveto{\pgfqpoint{5.850000in}{0.548769in}}%
\pgfpathlineto{\pgfqpoint{5.850000in}{3.651359in}}%
\pgfusepath{stroke}%
\end{pgfscope}%
\begin{pgfscope}%
\pgfsetrectcap%
\pgfsetmiterjoin%
\pgfsetlinewidth{0.803000pt}%
\definecolor{currentstroke}{rgb}{0.000000,0.000000,0.000000}%
\pgfsetstrokecolor{currentstroke}%
\pgfsetdash{}{0pt}%
\pgfpathmoveto{\pgfqpoint{0.648703in}{0.548769in}}%
\pgfpathlineto{\pgfqpoint{5.850000in}{0.548769in}}%
\pgfusepath{stroke}%
\end{pgfscope}%
\begin{pgfscope}%
\pgfsetrectcap%
\pgfsetmiterjoin%
\pgfsetlinewidth{0.803000pt}%
\definecolor{currentstroke}{rgb}{0.000000,0.000000,0.000000}%
\pgfsetstrokecolor{currentstroke}%
\pgfsetdash{}{0pt}%
\pgfpathmoveto{\pgfqpoint{0.648703in}{3.651359in}}%
\pgfpathlineto{\pgfqpoint{5.850000in}{3.651359in}}%
\pgfusepath{stroke}%
\end{pgfscope}%
\begin{pgfscope}%
\definecolor{textcolor}{rgb}{0.000000,0.000000,0.000000}%
\pgfsetstrokecolor{textcolor}%
\pgfsetfillcolor{textcolor}%
\pgftext[x=3.249352in,y=3.734692in,,base]{\color{textcolor}\sffamily\fontsize{12.000000}{14.400000}\selectfont Forward}%
\end{pgfscope}%
\begin{pgfscope}%
\pgfsetbuttcap%
\pgfsetmiterjoin%
\definecolor{currentfill}{rgb}{1.000000,1.000000,1.000000}%
\pgfsetfillcolor{currentfill}%
\pgfsetfillopacity{0.800000}%
\pgfsetlinewidth{1.003750pt}%
\definecolor{currentstroke}{rgb}{0.800000,0.800000,0.800000}%
\pgfsetstrokecolor{currentstroke}%
\pgfsetstrokeopacity{0.800000}%
\pgfsetdash{}{0pt}%
\pgfpathmoveto{\pgfqpoint{4.300417in}{0.618213in}}%
\pgfpathlineto{\pgfqpoint{5.752778in}{0.618213in}}%
\pgfpathquadraticcurveto{\pgfqpoint{5.780556in}{0.618213in}}{\pgfqpoint{5.780556in}{0.645991in}}%
\pgfpathlineto{\pgfqpoint{5.780556in}{1.214463in}}%
\pgfpathquadraticcurveto{\pgfqpoint{5.780556in}{1.242241in}}{\pgfqpoint{5.752778in}{1.242241in}}%
\pgfpathlineto{\pgfqpoint{4.300417in}{1.242241in}}%
\pgfpathquadraticcurveto{\pgfqpoint{4.272639in}{1.242241in}}{\pgfqpoint{4.272639in}{1.214463in}}%
\pgfpathlineto{\pgfqpoint{4.272639in}{0.645991in}}%
\pgfpathquadraticcurveto{\pgfqpoint{4.272639in}{0.618213in}}{\pgfqpoint{4.300417in}{0.618213in}}%
\pgfpathclose%
\pgfusepath{stroke,fill}%
\end{pgfscope}%
\begin{pgfscope}%
\pgfsetbuttcap%
\pgfsetroundjoin%
\definecolor{currentfill}{rgb}{0.121569,0.466667,0.705882}%
\pgfsetfillcolor{currentfill}%
\pgfsetlinewidth{1.003750pt}%
\definecolor{currentstroke}{rgb}{0.121569,0.466667,0.705882}%
\pgfsetstrokecolor{currentstroke}%
\pgfsetdash{}{0pt}%
\pgfsys@defobject{currentmarker}{\pgfqpoint{-0.034722in}{-0.034722in}}{\pgfqpoint{0.034722in}{0.034722in}}{%
\pgfpathmoveto{\pgfqpoint{0.000000in}{-0.034722in}}%
\pgfpathcurveto{\pgfqpoint{0.009208in}{-0.034722in}}{\pgfqpoint{0.018041in}{-0.031064in}}{\pgfqpoint{0.024552in}{-0.024552in}}%
\pgfpathcurveto{\pgfqpoint{0.031064in}{-0.018041in}}{\pgfqpoint{0.034722in}{-0.009208in}}{\pgfqpoint{0.034722in}{0.000000in}}%
\pgfpathcurveto{\pgfqpoint{0.034722in}{0.009208in}}{\pgfqpoint{0.031064in}{0.018041in}}{\pgfqpoint{0.024552in}{0.024552in}}%
\pgfpathcurveto{\pgfqpoint{0.018041in}{0.031064in}}{\pgfqpoint{0.009208in}{0.034722in}}{\pgfqpoint{0.000000in}{0.034722in}}%
\pgfpathcurveto{\pgfqpoint{-0.009208in}{0.034722in}}{\pgfqpoint{-0.018041in}{0.031064in}}{\pgfqpoint{-0.024552in}{0.024552in}}%
\pgfpathcurveto{\pgfqpoint{-0.031064in}{0.018041in}}{\pgfqpoint{-0.034722in}{0.009208in}}{\pgfqpoint{-0.034722in}{0.000000in}}%
\pgfpathcurveto{\pgfqpoint{-0.034722in}{-0.009208in}}{\pgfqpoint{-0.031064in}{-0.018041in}}{\pgfqpoint{-0.024552in}{-0.024552in}}%
\pgfpathcurveto{\pgfqpoint{-0.018041in}{-0.031064in}}{\pgfqpoint{-0.009208in}{-0.034722in}}{\pgfqpoint{0.000000in}{-0.034722in}}%
\pgfpathclose%
\pgfusepath{stroke,fill}%
}%
\begin{pgfscope}%
\pgfsys@transformshift{4.467083in}{1.138074in}%
\pgfsys@useobject{currentmarker}{}%
\end{pgfscope}%
\end{pgfscope}%
\begin{pgfscope}%
\definecolor{textcolor}{rgb}{0.000000,0.000000,0.000000}%
\pgfsetstrokecolor{textcolor}%
\pgfsetfillcolor{textcolor}%
\pgftext[x=4.717083in,y=1.089463in,left,base]{\color{textcolor}\sffamily\fontsize{10.000000}{12.000000}\selectfont No Timeout}%
\end{pgfscope}%
\begin{pgfscope}%
\pgfsetbuttcap%
\pgfsetroundjoin%
\definecolor{currentfill}{rgb}{1.000000,0.498039,0.054902}%
\pgfsetfillcolor{currentfill}%
\pgfsetlinewidth{1.003750pt}%
\definecolor{currentstroke}{rgb}{1.000000,0.498039,0.054902}%
\pgfsetstrokecolor{currentstroke}%
\pgfsetdash{}{0pt}%
\pgfsys@defobject{currentmarker}{\pgfqpoint{-0.034722in}{-0.034722in}}{\pgfqpoint{0.034722in}{0.034722in}}{%
\pgfpathmoveto{\pgfqpoint{0.000000in}{-0.034722in}}%
\pgfpathcurveto{\pgfqpoint{0.009208in}{-0.034722in}}{\pgfqpoint{0.018041in}{-0.031064in}}{\pgfqpoint{0.024552in}{-0.024552in}}%
\pgfpathcurveto{\pgfqpoint{0.031064in}{-0.018041in}}{\pgfqpoint{0.034722in}{-0.009208in}}{\pgfqpoint{0.034722in}{0.000000in}}%
\pgfpathcurveto{\pgfqpoint{0.034722in}{0.009208in}}{\pgfqpoint{0.031064in}{0.018041in}}{\pgfqpoint{0.024552in}{0.024552in}}%
\pgfpathcurveto{\pgfqpoint{0.018041in}{0.031064in}}{\pgfqpoint{0.009208in}{0.034722in}}{\pgfqpoint{0.000000in}{0.034722in}}%
\pgfpathcurveto{\pgfqpoint{-0.009208in}{0.034722in}}{\pgfqpoint{-0.018041in}{0.031064in}}{\pgfqpoint{-0.024552in}{0.024552in}}%
\pgfpathcurveto{\pgfqpoint{-0.031064in}{0.018041in}}{\pgfqpoint{-0.034722in}{0.009208in}}{\pgfqpoint{-0.034722in}{0.000000in}}%
\pgfpathcurveto{\pgfqpoint{-0.034722in}{-0.009208in}}{\pgfqpoint{-0.031064in}{-0.018041in}}{\pgfqpoint{-0.024552in}{-0.024552in}}%
\pgfpathcurveto{\pgfqpoint{-0.018041in}{-0.031064in}}{\pgfqpoint{-0.009208in}{-0.034722in}}{\pgfqpoint{0.000000in}{-0.034722in}}%
\pgfpathclose%
\pgfusepath{stroke,fill}%
}%
\begin{pgfscope}%
\pgfsys@transformshift{4.467083in}{0.944463in}%
\pgfsys@useobject{currentmarker}{}%
\end{pgfscope}%
\end{pgfscope}%
\begin{pgfscope}%
\definecolor{textcolor}{rgb}{0.000000,0.000000,0.000000}%
\pgfsetstrokecolor{textcolor}%
\pgfsetfillcolor{textcolor}%
\pgftext[x=4.717083in,y=0.895852in,left,base]{\color{textcolor}\sffamily\fontsize{10.000000}{12.000000}\selectfont Time Timeout}%
\end{pgfscope}%
\begin{pgfscope}%
\pgfsetbuttcap%
\pgfsetroundjoin%
\definecolor{currentfill}{rgb}{0.839216,0.152941,0.156863}%
\pgfsetfillcolor{currentfill}%
\pgfsetlinewidth{1.003750pt}%
\definecolor{currentstroke}{rgb}{0.839216,0.152941,0.156863}%
\pgfsetstrokecolor{currentstroke}%
\pgfsetdash{}{0pt}%
\pgfsys@defobject{currentmarker}{\pgfqpoint{-0.034722in}{-0.034722in}}{\pgfqpoint{0.034722in}{0.034722in}}{%
\pgfpathmoveto{\pgfqpoint{0.000000in}{-0.034722in}}%
\pgfpathcurveto{\pgfqpoint{0.009208in}{-0.034722in}}{\pgfqpoint{0.018041in}{-0.031064in}}{\pgfqpoint{0.024552in}{-0.024552in}}%
\pgfpathcurveto{\pgfqpoint{0.031064in}{-0.018041in}}{\pgfqpoint{0.034722in}{-0.009208in}}{\pgfqpoint{0.034722in}{0.000000in}}%
\pgfpathcurveto{\pgfqpoint{0.034722in}{0.009208in}}{\pgfqpoint{0.031064in}{0.018041in}}{\pgfqpoint{0.024552in}{0.024552in}}%
\pgfpathcurveto{\pgfqpoint{0.018041in}{0.031064in}}{\pgfqpoint{0.009208in}{0.034722in}}{\pgfqpoint{0.000000in}{0.034722in}}%
\pgfpathcurveto{\pgfqpoint{-0.009208in}{0.034722in}}{\pgfqpoint{-0.018041in}{0.031064in}}{\pgfqpoint{-0.024552in}{0.024552in}}%
\pgfpathcurveto{\pgfqpoint{-0.031064in}{0.018041in}}{\pgfqpoint{-0.034722in}{0.009208in}}{\pgfqpoint{-0.034722in}{0.000000in}}%
\pgfpathcurveto{\pgfqpoint{-0.034722in}{-0.009208in}}{\pgfqpoint{-0.031064in}{-0.018041in}}{\pgfqpoint{-0.024552in}{-0.024552in}}%
\pgfpathcurveto{\pgfqpoint{-0.018041in}{-0.031064in}}{\pgfqpoint{-0.009208in}{-0.034722in}}{\pgfqpoint{0.000000in}{-0.034722in}}%
\pgfpathclose%
\pgfusepath{stroke,fill}%
}%
\begin{pgfscope}%
\pgfsys@transformshift{4.467083in}{0.750852in}%
\pgfsys@useobject{currentmarker}{}%
\end{pgfscope}%
\end{pgfscope}%
\begin{pgfscope}%
\definecolor{textcolor}{rgb}{0.000000,0.000000,0.000000}%
\pgfsetstrokecolor{textcolor}%
\pgfsetfillcolor{textcolor}%
\pgftext[x=4.717083in,y=0.702241in,left,base]{\color{textcolor}\sffamily\fontsize{10.000000}{12.000000}\selectfont Memory Timeout}%
\end{pgfscope}%
\end{pgfpicture}%
\makeatother%
\endgroup%

                }
            \end{subfigure}
            \qquad
            \begin{subfigure}[]{0.45\textwidth}
                \centering
                \resizebox{\columnwidth}{!}{
                    %% Creator: Matplotlib, PGF backend
%%
%% To include the figure in your LaTeX document, write
%%   \input{<filename>.pgf}
%%
%% Make sure the required packages are loaded in your preamble
%%   \usepackage{pgf}
%%
%% and, on pdftex
%%   \usepackage[utf8]{inputenc}\DeclareUnicodeCharacter{2212}{-}
%%
%% or, on luatex and xetex
%%   \usepackage{unicode-math}
%%
%% Figures using additional raster images can only be included by \input if
%% they are in the same directory as the main LaTeX file. For loading figures
%% from other directories you can use the `import` package
%%   \usepackage{import}
%%
%% and then include the figures with
%%   \import{<path to file>}{<filename>.pgf}
%%
%% Matplotlib used the following preamble
%%   \usepackage{amsmath}
%%   \usepackage{fontspec}
%%
\begingroup%
\makeatletter%
\begin{pgfpicture}%
\pgfpathrectangle{\pgfpointorigin}{\pgfqpoint{6.000000in}{4.000000in}}%
\pgfusepath{use as bounding box, clip}%
\begin{pgfscope}%
\pgfsetbuttcap%
\pgfsetmiterjoin%
\definecolor{currentfill}{rgb}{1.000000,1.000000,1.000000}%
\pgfsetfillcolor{currentfill}%
\pgfsetlinewidth{0.000000pt}%
\definecolor{currentstroke}{rgb}{1.000000,1.000000,1.000000}%
\pgfsetstrokecolor{currentstroke}%
\pgfsetdash{}{0pt}%
\pgfpathmoveto{\pgfqpoint{0.000000in}{0.000000in}}%
\pgfpathlineto{\pgfqpoint{6.000000in}{0.000000in}}%
\pgfpathlineto{\pgfqpoint{6.000000in}{4.000000in}}%
\pgfpathlineto{\pgfqpoint{0.000000in}{4.000000in}}%
\pgfpathclose%
\pgfusepath{fill}%
\end{pgfscope}%
\begin{pgfscope}%
\pgfsetbuttcap%
\pgfsetmiterjoin%
\definecolor{currentfill}{rgb}{1.000000,1.000000,1.000000}%
\pgfsetfillcolor{currentfill}%
\pgfsetlinewidth{0.000000pt}%
\definecolor{currentstroke}{rgb}{0.000000,0.000000,0.000000}%
\pgfsetstrokecolor{currentstroke}%
\pgfsetstrokeopacity{0.000000}%
\pgfsetdash{}{0pt}%
\pgfpathmoveto{\pgfqpoint{0.648703in}{0.548769in}}%
\pgfpathlineto{\pgfqpoint{5.850000in}{0.548769in}}%
\pgfpathlineto{\pgfqpoint{5.850000in}{3.651359in}}%
\pgfpathlineto{\pgfqpoint{0.648703in}{3.651359in}}%
\pgfpathclose%
\pgfusepath{fill}%
\end{pgfscope}%
\begin{pgfscope}%
\pgfpathrectangle{\pgfqpoint{0.648703in}{0.548769in}}{\pgfqpoint{5.201297in}{3.102590in}}%
\pgfusepath{clip}%
\pgfsetbuttcap%
\pgfsetroundjoin%
\definecolor{currentfill}{rgb}{0.121569,0.466667,0.705882}%
\pgfsetfillcolor{currentfill}%
\pgfsetlinewidth{1.003750pt}%
\definecolor{currentstroke}{rgb}{0.121569,0.466667,0.705882}%
\pgfsetstrokecolor{currentstroke}%
\pgfsetdash{}{0pt}%
\pgfpathmoveto{\pgfqpoint{1.333773in}{0.673501in}}%
\pgfpathcurveto{\pgfqpoint{1.344823in}{0.673501in}}{\pgfqpoint{1.355422in}{0.677891in}}{\pgfqpoint{1.363236in}{0.685705in}}%
\pgfpathcurveto{\pgfqpoint{1.371050in}{0.693519in}}{\pgfqpoint{1.375440in}{0.704118in}}{\pgfqpoint{1.375440in}{0.715168in}}%
\pgfpathcurveto{\pgfqpoint{1.375440in}{0.726218in}}{\pgfqpoint{1.371050in}{0.736817in}}{\pgfqpoint{1.363236in}{0.744631in}}%
\pgfpathcurveto{\pgfqpoint{1.355422in}{0.752444in}}{\pgfqpoint{1.344823in}{0.756834in}}{\pgfqpoint{1.333773in}{0.756834in}}%
\pgfpathcurveto{\pgfqpoint{1.322723in}{0.756834in}}{\pgfqpoint{1.312124in}{0.752444in}}{\pgfqpoint{1.304310in}{0.744631in}}%
\pgfpathcurveto{\pgfqpoint{1.296497in}{0.736817in}}{\pgfqpoint{1.292106in}{0.726218in}}{\pgfqpoint{1.292106in}{0.715168in}}%
\pgfpathcurveto{\pgfqpoint{1.292106in}{0.704118in}}{\pgfqpoint{1.296497in}{0.693519in}}{\pgfqpoint{1.304310in}{0.685705in}}%
\pgfpathcurveto{\pgfqpoint{1.312124in}{0.677891in}}{\pgfqpoint{1.322723in}{0.673501in}}{\pgfqpoint{1.333773in}{0.673501in}}%
\pgfpathclose%
\pgfusepath{stroke,fill}%
\end{pgfscope}%
\begin{pgfscope}%
\pgfpathrectangle{\pgfqpoint{0.648703in}{0.548769in}}{\pgfqpoint{5.201297in}{3.102590in}}%
\pgfusepath{clip}%
\pgfsetbuttcap%
\pgfsetroundjoin%
\definecolor{currentfill}{rgb}{0.121569,0.466667,0.705882}%
\pgfsetfillcolor{currentfill}%
\pgfsetlinewidth{1.003750pt}%
\definecolor{currentstroke}{rgb}{0.121569,0.466667,0.705882}%
\pgfsetstrokecolor{currentstroke}%
\pgfsetdash{}{0pt}%
\pgfpathmoveto{\pgfqpoint{4.728025in}{3.176886in}}%
\pgfpathcurveto{\pgfqpoint{4.739075in}{3.176886in}}{\pgfqpoint{4.749674in}{3.181276in}}{\pgfqpoint{4.757488in}{3.189089in}}%
\pgfpathcurveto{\pgfqpoint{4.765301in}{3.196903in}}{\pgfqpoint{4.769692in}{3.207502in}}{\pgfqpoint{4.769692in}{3.218552in}}%
\pgfpathcurveto{\pgfqpoint{4.769692in}{3.229602in}}{\pgfqpoint{4.765301in}{3.240201in}}{\pgfqpoint{4.757488in}{3.248015in}}%
\pgfpathcurveto{\pgfqpoint{4.749674in}{3.255829in}}{\pgfqpoint{4.739075in}{3.260219in}}{\pgfqpoint{4.728025in}{3.260219in}}%
\pgfpathcurveto{\pgfqpoint{4.716975in}{3.260219in}}{\pgfqpoint{4.706376in}{3.255829in}}{\pgfqpoint{4.698562in}{3.248015in}}%
\pgfpathcurveto{\pgfqpoint{4.690749in}{3.240201in}}{\pgfqpoint{4.686358in}{3.229602in}}{\pgfqpoint{4.686358in}{3.218552in}}%
\pgfpathcurveto{\pgfqpoint{4.686358in}{3.207502in}}{\pgfqpoint{4.690749in}{3.196903in}}{\pgfqpoint{4.698562in}{3.189089in}}%
\pgfpathcurveto{\pgfqpoint{4.706376in}{3.181276in}}{\pgfqpoint{4.716975in}{3.176886in}}{\pgfqpoint{4.728025in}{3.176886in}}%
\pgfpathclose%
\pgfusepath{stroke,fill}%
\end{pgfscope}%
\begin{pgfscope}%
\pgfpathrectangle{\pgfqpoint{0.648703in}{0.548769in}}{\pgfqpoint{5.201297in}{3.102590in}}%
\pgfusepath{clip}%
\pgfsetbuttcap%
\pgfsetroundjoin%
\definecolor{currentfill}{rgb}{1.000000,0.498039,0.054902}%
\pgfsetfillcolor{currentfill}%
\pgfsetlinewidth{1.003750pt}%
\definecolor{currentstroke}{rgb}{1.000000,0.498039,0.054902}%
\pgfsetstrokecolor{currentstroke}%
\pgfsetdash{}{0pt}%
\pgfpathmoveto{\pgfqpoint{1.263609in}{3.198029in}}%
\pgfpathcurveto{\pgfqpoint{1.274660in}{3.198029in}}{\pgfqpoint{1.285259in}{3.202419in}}{\pgfqpoint{1.293072in}{3.210233in}}%
\pgfpathcurveto{\pgfqpoint{1.300886in}{3.218046in}}{\pgfqpoint{1.305276in}{3.228646in}}{\pgfqpoint{1.305276in}{3.239696in}}%
\pgfpathcurveto{\pgfqpoint{1.305276in}{3.250746in}}{\pgfqpoint{1.300886in}{3.261345in}}{\pgfqpoint{1.293072in}{3.269158in}}%
\pgfpathcurveto{\pgfqpoint{1.285259in}{3.276972in}}{\pgfqpoint{1.274660in}{3.281362in}}{\pgfqpoint{1.263609in}{3.281362in}}%
\pgfpathcurveto{\pgfqpoint{1.252559in}{3.281362in}}{\pgfqpoint{1.241960in}{3.276972in}}{\pgfqpoint{1.234147in}{3.269158in}}%
\pgfpathcurveto{\pgfqpoint{1.226333in}{3.261345in}}{\pgfqpoint{1.221943in}{3.250746in}}{\pgfqpoint{1.221943in}{3.239696in}}%
\pgfpathcurveto{\pgfqpoint{1.221943in}{3.228646in}}{\pgfqpoint{1.226333in}{3.218046in}}{\pgfqpoint{1.234147in}{3.210233in}}%
\pgfpathcurveto{\pgfqpoint{1.241960in}{3.202419in}}{\pgfqpoint{1.252559in}{3.198029in}}{\pgfqpoint{1.263609in}{3.198029in}}%
\pgfpathclose%
\pgfusepath{stroke,fill}%
\end{pgfscope}%
\begin{pgfscope}%
\pgfpathrectangle{\pgfqpoint{0.648703in}{0.548769in}}{\pgfqpoint{5.201297in}{3.102590in}}%
\pgfusepath{clip}%
\pgfsetbuttcap%
\pgfsetroundjoin%
\definecolor{currentfill}{rgb}{1.000000,0.498039,0.054902}%
\pgfsetfillcolor{currentfill}%
\pgfsetlinewidth{1.003750pt}%
\definecolor{currentstroke}{rgb}{1.000000,0.498039,0.054902}%
\pgfsetstrokecolor{currentstroke}%
\pgfsetdash{}{0pt}%
\pgfpathmoveto{\pgfqpoint{2.952495in}{3.185343in}}%
\pgfpathcurveto{\pgfqpoint{2.963545in}{3.185343in}}{\pgfqpoint{2.974144in}{3.189733in}}{\pgfqpoint{2.981958in}{3.197547in}}%
\pgfpathcurveto{\pgfqpoint{2.989772in}{3.205360in}}{\pgfqpoint{2.994162in}{3.215959in}}{\pgfqpoint{2.994162in}{3.227010in}}%
\pgfpathcurveto{\pgfqpoint{2.994162in}{3.238060in}}{\pgfqpoint{2.989772in}{3.248659in}}{\pgfqpoint{2.981958in}{3.256472in}}%
\pgfpathcurveto{\pgfqpoint{2.974144in}{3.264286in}}{\pgfqpoint{2.963545in}{3.268676in}}{\pgfqpoint{2.952495in}{3.268676in}}%
\pgfpathcurveto{\pgfqpoint{2.941445in}{3.268676in}}{\pgfqpoint{2.930846in}{3.264286in}}{\pgfqpoint{2.923032in}{3.256472in}}%
\pgfpathcurveto{\pgfqpoint{2.915219in}{3.248659in}}{\pgfqpoint{2.910828in}{3.238060in}}{\pgfqpoint{2.910828in}{3.227010in}}%
\pgfpathcurveto{\pgfqpoint{2.910828in}{3.215959in}}{\pgfqpoint{2.915219in}{3.205360in}}{\pgfqpoint{2.923032in}{3.197547in}}%
\pgfpathcurveto{\pgfqpoint{2.930846in}{3.189733in}}{\pgfqpoint{2.941445in}{3.185343in}}{\pgfqpoint{2.952495in}{3.185343in}}%
\pgfpathclose%
\pgfusepath{stroke,fill}%
\end{pgfscope}%
\begin{pgfscope}%
\pgfpathrectangle{\pgfqpoint{0.648703in}{0.548769in}}{\pgfqpoint{5.201297in}{3.102590in}}%
\pgfusepath{clip}%
\pgfsetbuttcap%
\pgfsetroundjoin%
\definecolor{currentfill}{rgb}{0.121569,0.466667,0.705882}%
\pgfsetfillcolor{currentfill}%
\pgfsetlinewidth{1.003750pt}%
\definecolor{currentstroke}{rgb}{0.121569,0.466667,0.705882}%
\pgfsetstrokecolor{currentstroke}%
\pgfsetdash{}{0pt}%
\pgfpathmoveto{\pgfqpoint{2.854240in}{0.652358in}}%
\pgfpathcurveto{\pgfqpoint{2.865290in}{0.652358in}}{\pgfqpoint{2.875889in}{0.656748in}}{\pgfqpoint{2.883703in}{0.664562in}}%
\pgfpathcurveto{\pgfqpoint{2.891516in}{0.672375in}}{\pgfqpoint{2.895906in}{0.682974in}}{\pgfqpoint{2.895906in}{0.694024in}}%
\pgfpathcurveto{\pgfqpoint{2.895906in}{0.705074in}}{\pgfqpoint{2.891516in}{0.715673in}}{\pgfqpoint{2.883703in}{0.723487in}}%
\pgfpathcurveto{\pgfqpoint{2.875889in}{0.731301in}}{\pgfqpoint{2.865290in}{0.735691in}}{\pgfqpoint{2.854240in}{0.735691in}}%
\pgfpathcurveto{\pgfqpoint{2.843190in}{0.735691in}}{\pgfqpoint{2.832591in}{0.731301in}}{\pgfqpoint{2.824777in}{0.723487in}}%
\pgfpathcurveto{\pgfqpoint{2.816963in}{0.715673in}}{\pgfqpoint{2.812573in}{0.705074in}}{\pgfqpoint{2.812573in}{0.694024in}}%
\pgfpathcurveto{\pgfqpoint{2.812573in}{0.682974in}}{\pgfqpoint{2.816963in}{0.672375in}}{\pgfqpoint{2.824777in}{0.664562in}}%
\pgfpathcurveto{\pgfqpoint{2.832591in}{0.656748in}}{\pgfqpoint{2.843190in}{0.652358in}}{\pgfqpoint{2.854240in}{0.652358in}}%
\pgfpathclose%
\pgfusepath{stroke,fill}%
\end{pgfscope}%
\begin{pgfscope}%
\pgfpathrectangle{\pgfqpoint{0.648703in}{0.548769in}}{\pgfqpoint{5.201297in}{3.102590in}}%
\pgfusepath{clip}%
\pgfsetbuttcap%
\pgfsetroundjoin%
\definecolor{currentfill}{rgb}{0.121569,0.466667,0.705882}%
\pgfsetfillcolor{currentfill}%
\pgfsetlinewidth{1.003750pt}%
\definecolor{currentstroke}{rgb}{0.121569,0.466667,0.705882}%
\pgfsetstrokecolor{currentstroke}%
\pgfsetdash{}{0pt}%
\pgfpathmoveto{\pgfqpoint{2.559526in}{3.181114in}}%
\pgfpathcurveto{\pgfqpoint{2.570576in}{3.181114in}}{\pgfqpoint{2.581175in}{3.185504in}}{\pgfqpoint{2.588989in}{3.193318in}}%
\pgfpathcurveto{\pgfqpoint{2.596802in}{3.201132in}}{\pgfqpoint{2.601193in}{3.211731in}}{\pgfqpoint{2.601193in}{3.222781in}}%
\pgfpathcurveto{\pgfqpoint{2.601193in}{3.233831in}}{\pgfqpoint{2.596802in}{3.244430in}}{\pgfqpoint{2.588989in}{3.252244in}}%
\pgfpathcurveto{\pgfqpoint{2.581175in}{3.260057in}}{\pgfqpoint{2.570576in}{3.264448in}}{\pgfqpoint{2.559526in}{3.264448in}}%
\pgfpathcurveto{\pgfqpoint{2.548476in}{3.264448in}}{\pgfqpoint{2.537877in}{3.260057in}}{\pgfqpoint{2.530063in}{3.252244in}}%
\pgfpathcurveto{\pgfqpoint{2.522250in}{3.244430in}}{\pgfqpoint{2.517859in}{3.233831in}}{\pgfqpoint{2.517859in}{3.222781in}}%
\pgfpathcurveto{\pgfqpoint{2.517859in}{3.211731in}}{\pgfqpoint{2.522250in}{3.201132in}}{\pgfqpoint{2.530063in}{3.193318in}}%
\pgfpathcurveto{\pgfqpoint{2.537877in}{3.185504in}}{\pgfqpoint{2.548476in}{3.181114in}}{\pgfqpoint{2.559526in}{3.181114in}}%
\pgfpathclose%
\pgfusepath{stroke,fill}%
\end{pgfscope}%
\begin{pgfscope}%
\pgfpathrectangle{\pgfqpoint{0.648703in}{0.548769in}}{\pgfqpoint{5.201297in}{3.102590in}}%
\pgfusepath{clip}%
\pgfsetbuttcap%
\pgfsetroundjoin%
\definecolor{currentfill}{rgb}{1.000000,0.498039,0.054902}%
\pgfsetfillcolor{currentfill}%
\pgfsetlinewidth{1.003750pt}%
\definecolor{currentstroke}{rgb}{1.000000,0.498039,0.054902}%
\pgfsetstrokecolor{currentstroke}%
\pgfsetdash{}{0pt}%
\pgfpathmoveto{\pgfqpoint{2.471488in}{3.189572in}}%
\pgfpathcurveto{\pgfqpoint{2.482538in}{3.189572in}}{\pgfqpoint{2.493137in}{3.193962in}}{\pgfqpoint{2.500951in}{3.201775in}}%
\pgfpathcurveto{\pgfqpoint{2.508765in}{3.209589in}}{\pgfqpoint{2.513155in}{3.220188in}}{\pgfqpoint{2.513155in}{3.231238in}}%
\pgfpathcurveto{\pgfqpoint{2.513155in}{3.242288in}}{\pgfqpoint{2.508765in}{3.252887in}}{\pgfqpoint{2.500951in}{3.260701in}}%
\pgfpathcurveto{\pgfqpoint{2.493137in}{3.268515in}}{\pgfqpoint{2.482538in}{3.272905in}}{\pgfqpoint{2.471488in}{3.272905in}}%
\pgfpathcurveto{\pgfqpoint{2.460438in}{3.272905in}}{\pgfqpoint{2.449839in}{3.268515in}}{\pgfqpoint{2.442025in}{3.260701in}}%
\pgfpathcurveto{\pgfqpoint{2.434212in}{3.252887in}}{\pgfqpoint{2.429822in}{3.242288in}}{\pgfqpoint{2.429822in}{3.231238in}}%
\pgfpathcurveto{\pgfqpoint{2.429822in}{3.220188in}}{\pgfqpoint{2.434212in}{3.209589in}}{\pgfqpoint{2.442025in}{3.201775in}}%
\pgfpathcurveto{\pgfqpoint{2.449839in}{3.193962in}}{\pgfqpoint{2.460438in}{3.189572in}}{\pgfqpoint{2.471488in}{3.189572in}}%
\pgfpathclose%
\pgfusepath{stroke,fill}%
\end{pgfscope}%
\begin{pgfscope}%
\pgfpathrectangle{\pgfqpoint{0.648703in}{0.548769in}}{\pgfqpoint{5.201297in}{3.102590in}}%
\pgfusepath{clip}%
\pgfsetbuttcap%
\pgfsetroundjoin%
\definecolor{currentfill}{rgb}{0.121569,0.466667,0.705882}%
\pgfsetfillcolor{currentfill}%
\pgfsetlinewidth{1.003750pt}%
\definecolor{currentstroke}{rgb}{0.121569,0.466667,0.705882}%
\pgfsetstrokecolor{currentstroke}%
\pgfsetdash{}{0pt}%
\pgfpathmoveto{\pgfqpoint{1.456784in}{0.648129in}}%
\pgfpathcurveto{\pgfqpoint{1.467834in}{0.648129in}}{\pgfqpoint{1.478433in}{0.652519in}}{\pgfqpoint{1.486247in}{0.660333in}}%
\pgfpathcurveto{\pgfqpoint{1.494060in}{0.668146in}}{\pgfqpoint{1.498451in}{0.678745in}}{\pgfqpoint{1.498451in}{0.689796in}}%
\pgfpathcurveto{\pgfqpoint{1.498451in}{0.700846in}}{\pgfqpoint{1.494060in}{0.711445in}}{\pgfqpoint{1.486247in}{0.719258in}}%
\pgfpathcurveto{\pgfqpoint{1.478433in}{0.727072in}}{\pgfqpoint{1.467834in}{0.731462in}}{\pgfqpoint{1.456784in}{0.731462in}}%
\pgfpathcurveto{\pgfqpoint{1.445734in}{0.731462in}}{\pgfqpoint{1.435135in}{0.727072in}}{\pgfqpoint{1.427321in}{0.719258in}}%
\pgfpathcurveto{\pgfqpoint{1.419508in}{0.711445in}}{\pgfqpoint{1.415117in}{0.700846in}}{\pgfqpoint{1.415117in}{0.689796in}}%
\pgfpathcurveto{\pgfqpoint{1.415117in}{0.678745in}}{\pgfqpoint{1.419508in}{0.668146in}}{\pgfqpoint{1.427321in}{0.660333in}}%
\pgfpathcurveto{\pgfqpoint{1.435135in}{0.652519in}}{\pgfqpoint{1.445734in}{0.648129in}}{\pgfqpoint{1.456784in}{0.648129in}}%
\pgfpathclose%
\pgfusepath{stroke,fill}%
\end{pgfscope}%
\begin{pgfscope}%
\pgfpathrectangle{\pgfqpoint{0.648703in}{0.548769in}}{\pgfqpoint{5.201297in}{3.102590in}}%
\pgfusepath{clip}%
\pgfsetbuttcap%
\pgfsetroundjoin%
\definecolor{currentfill}{rgb}{0.121569,0.466667,0.705882}%
\pgfsetfillcolor{currentfill}%
\pgfsetlinewidth{1.003750pt}%
\definecolor{currentstroke}{rgb}{0.121569,0.466667,0.705882}%
\pgfsetstrokecolor{currentstroke}%
\pgfsetdash{}{0pt}%
\pgfpathmoveto{\pgfqpoint{2.243419in}{0.774990in}}%
\pgfpathcurveto{\pgfqpoint{2.254469in}{0.774990in}}{\pgfqpoint{2.265068in}{0.779380in}}{\pgfqpoint{2.272882in}{0.787194in}}%
\pgfpathcurveto{\pgfqpoint{2.280695in}{0.795007in}}{\pgfqpoint{2.285086in}{0.805606in}}{\pgfqpoint{2.285086in}{0.816656in}}%
\pgfpathcurveto{\pgfqpoint{2.285086in}{0.827706in}}{\pgfqpoint{2.280695in}{0.838305in}}{\pgfqpoint{2.272882in}{0.846119in}}%
\pgfpathcurveto{\pgfqpoint{2.265068in}{0.853933in}}{\pgfqpoint{2.254469in}{0.858323in}}{\pgfqpoint{2.243419in}{0.858323in}}%
\pgfpathcurveto{\pgfqpoint{2.232369in}{0.858323in}}{\pgfqpoint{2.221770in}{0.853933in}}{\pgfqpoint{2.213956in}{0.846119in}}%
\pgfpathcurveto{\pgfqpoint{2.206143in}{0.838305in}}{\pgfqpoint{2.201752in}{0.827706in}}{\pgfqpoint{2.201752in}{0.816656in}}%
\pgfpathcurveto{\pgfqpoint{2.201752in}{0.805606in}}{\pgfqpoint{2.206143in}{0.795007in}}{\pgfqpoint{2.213956in}{0.787194in}}%
\pgfpathcurveto{\pgfqpoint{2.221770in}{0.779380in}}{\pgfqpoint{2.232369in}{0.774990in}}{\pgfqpoint{2.243419in}{0.774990in}}%
\pgfpathclose%
\pgfusepath{stroke,fill}%
\end{pgfscope}%
\begin{pgfscope}%
\pgfpathrectangle{\pgfqpoint{0.648703in}{0.548769in}}{\pgfqpoint{5.201297in}{3.102590in}}%
\pgfusepath{clip}%
\pgfsetbuttcap%
\pgfsetroundjoin%
\definecolor{currentfill}{rgb}{0.121569,0.466667,0.705882}%
\pgfsetfillcolor{currentfill}%
\pgfsetlinewidth{1.003750pt}%
\definecolor{currentstroke}{rgb}{0.121569,0.466667,0.705882}%
\pgfsetstrokecolor{currentstroke}%
\pgfsetdash{}{0pt}%
\pgfpathmoveto{\pgfqpoint{0.885178in}{0.783447in}}%
\pgfpathcurveto{\pgfqpoint{0.896228in}{0.783447in}}{\pgfqpoint{0.906827in}{0.787837in}}{\pgfqpoint{0.914641in}{0.795651in}}%
\pgfpathcurveto{\pgfqpoint{0.922455in}{0.803465in}}{\pgfqpoint{0.926845in}{0.814064in}}{\pgfqpoint{0.926845in}{0.825114in}}%
\pgfpathcurveto{\pgfqpoint{0.926845in}{0.836164in}}{\pgfqpoint{0.922455in}{0.846763in}}{\pgfqpoint{0.914641in}{0.854576in}}%
\pgfpathcurveto{\pgfqpoint{0.906827in}{0.862390in}}{\pgfqpoint{0.896228in}{0.866780in}}{\pgfqpoint{0.885178in}{0.866780in}}%
\pgfpathcurveto{\pgfqpoint{0.874128in}{0.866780in}}{\pgfqpoint{0.863529in}{0.862390in}}{\pgfqpoint{0.855715in}{0.854576in}}%
\pgfpathcurveto{\pgfqpoint{0.847902in}{0.846763in}}{\pgfqpoint{0.843512in}{0.836164in}}{\pgfqpoint{0.843512in}{0.825114in}}%
\pgfpathcurveto{\pgfqpoint{0.843512in}{0.814064in}}{\pgfqpoint{0.847902in}{0.803465in}}{\pgfqpoint{0.855715in}{0.795651in}}%
\pgfpathcurveto{\pgfqpoint{0.863529in}{0.787837in}}{\pgfqpoint{0.874128in}{0.783447in}}{\pgfqpoint{0.885178in}{0.783447in}}%
\pgfpathclose%
\pgfusepath{stroke,fill}%
\end{pgfscope}%
\begin{pgfscope}%
\pgfpathrectangle{\pgfqpoint{0.648703in}{0.548769in}}{\pgfqpoint{5.201297in}{3.102590in}}%
\pgfusepath{clip}%
\pgfsetbuttcap%
\pgfsetroundjoin%
\definecolor{currentfill}{rgb}{0.121569,0.466667,0.705882}%
\pgfsetfillcolor{currentfill}%
\pgfsetlinewidth{1.003750pt}%
\definecolor{currentstroke}{rgb}{0.121569,0.466667,0.705882}%
\pgfsetstrokecolor{currentstroke}%
\pgfsetdash{}{0pt}%
\pgfpathmoveto{\pgfqpoint{2.085061in}{0.652358in}}%
\pgfpathcurveto{\pgfqpoint{2.096111in}{0.652358in}}{\pgfqpoint{2.106710in}{0.656748in}}{\pgfqpoint{2.114523in}{0.664562in}}%
\pgfpathcurveto{\pgfqpoint{2.122337in}{0.672375in}}{\pgfqpoint{2.126727in}{0.682974in}}{\pgfqpoint{2.126727in}{0.694024in}}%
\pgfpathcurveto{\pgfqpoint{2.126727in}{0.705074in}}{\pgfqpoint{2.122337in}{0.715673in}}{\pgfqpoint{2.114523in}{0.723487in}}%
\pgfpathcurveto{\pgfqpoint{2.106710in}{0.731301in}}{\pgfqpoint{2.096111in}{0.735691in}}{\pgfqpoint{2.085061in}{0.735691in}}%
\pgfpathcurveto{\pgfqpoint{2.074011in}{0.735691in}}{\pgfqpoint{2.063412in}{0.731301in}}{\pgfqpoint{2.055598in}{0.723487in}}%
\pgfpathcurveto{\pgfqpoint{2.047784in}{0.715673in}}{\pgfqpoint{2.043394in}{0.705074in}}{\pgfqpoint{2.043394in}{0.694024in}}%
\pgfpathcurveto{\pgfqpoint{2.043394in}{0.682974in}}{\pgfqpoint{2.047784in}{0.672375in}}{\pgfqpoint{2.055598in}{0.664562in}}%
\pgfpathcurveto{\pgfqpoint{2.063412in}{0.656748in}}{\pgfqpoint{2.074011in}{0.652358in}}{\pgfqpoint{2.085061in}{0.652358in}}%
\pgfpathclose%
\pgfusepath{stroke,fill}%
\end{pgfscope}%
\begin{pgfscope}%
\pgfpathrectangle{\pgfqpoint{0.648703in}{0.548769in}}{\pgfqpoint{5.201297in}{3.102590in}}%
\pgfusepath{clip}%
\pgfsetbuttcap%
\pgfsetroundjoin%
\definecolor{currentfill}{rgb}{0.121569,0.466667,0.705882}%
\pgfsetfillcolor{currentfill}%
\pgfsetlinewidth{1.003750pt}%
\definecolor{currentstroke}{rgb}{0.121569,0.466667,0.705882}%
\pgfsetstrokecolor{currentstroke}%
\pgfsetdash{}{0pt}%
\pgfpathmoveto{\pgfqpoint{1.502950in}{0.648129in}}%
\pgfpathcurveto{\pgfqpoint{1.514000in}{0.648129in}}{\pgfqpoint{1.524599in}{0.652519in}}{\pgfqpoint{1.532413in}{0.660333in}}%
\pgfpathcurveto{\pgfqpoint{1.540226in}{0.668146in}}{\pgfqpoint{1.544617in}{0.678745in}}{\pgfqpoint{1.544617in}{0.689796in}}%
\pgfpathcurveto{\pgfqpoint{1.544617in}{0.700846in}}{\pgfqpoint{1.540226in}{0.711445in}}{\pgfqpoint{1.532413in}{0.719258in}}%
\pgfpathcurveto{\pgfqpoint{1.524599in}{0.727072in}}{\pgfqpoint{1.514000in}{0.731462in}}{\pgfqpoint{1.502950in}{0.731462in}}%
\pgfpathcurveto{\pgfqpoint{1.491900in}{0.731462in}}{\pgfqpoint{1.481301in}{0.727072in}}{\pgfqpoint{1.473487in}{0.719258in}}%
\pgfpathcurveto{\pgfqpoint{1.465674in}{0.711445in}}{\pgfqpoint{1.461283in}{0.700846in}}{\pgfqpoint{1.461283in}{0.689796in}}%
\pgfpathcurveto{\pgfqpoint{1.461283in}{0.678745in}}{\pgfqpoint{1.465674in}{0.668146in}}{\pgfqpoint{1.473487in}{0.660333in}}%
\pgfpathcurveto{\pgfqpoint{1.481301in}{0.652519in}}{\pgfqpoint{1.491900in}{0.648129in}}{\pgfqpoint{1.502950in}{0.648129in}}%
\pgfpathclose%
\pgfusepath{stroke,fill}%
\end{pgfscope}%
\begin{pgfscope}%
\pgfpathrectangle{\pgfqpoint{0.648703in}{0.548769in}}{\pgfqpoint{5.201297in}{3.102590in}}%
\pgfusepath{clip}%
\pgfsetbuttcap%
\pgfsetroundjoin%
\definecolor{currentfill}{rgb}{0.121569,0.466667,0.705882}%
\pgfsetfillcolor{currentfill}%
\pgfsetlinewidth{1.003750pt}%
\definecolor{currentstroke}{rgb}{0.121569,0.466667,0.705882}%
\pgfsetstrokecolor{currentstroke}%
\pgfsetdash{}{0pt}%
\pgfpathmoveto{\pgfqpoint{1.817794in}{0.648129in}}%
\pgfpathcurveto{\pgfqpoint{1.828844in}{0.648129in}}{\pgfqpoint{1.839443in}{0.652519in}}{\pgfqpoint{1.847257in}{0.660333in}}%
\pgfpathcurveto{\pgfqpoint{1.855070in}{0.668146in}}{\pgfqpoint{1.859461in}{0.678745in}}{\pgfqpoint{1.859461in}{0.689796in}}%
\pgfpathcurveto{\pgfqpoint{1.859461in}{0.700846in}}{\pgfqpoint{1.855070in}{0.711445in}}{\pgfqpoint{1.847257in}{0.719258in}}%
\pgfpathcurveto{\pgfqpoint{1.839443in}{0.727072in}}{\pgfqpoint{1.828844in}{0.731462in}}{\pgfqpoint{1.817794in}{0.731462in}}%
\pgfpathcurveto{\pgfqpoint{1.806744in}{0.731462in}}{\pgfqpoint{1.796145in}{0.727072in}}{\pgfqpoint{1.788331in}{0.719258in}}%
\pgfpathcurveto{\pgfqpoint{1.780518in}{0.711445in}}{\pgfqpoint{1.776127in}{0.700846in}}{\pgfqpoint{1.776127in}{0.689796in}}%
\pgfpathcurveto{\pgfqpoint{1.776127in}{0.678745in}}{\pgfqpoint{1.780518in}{0.668146in}}{\pgfqpoint{1.788331in}{0.660333in}}%
\pgfpathcurveto{\pgfqpoint{1.796145in}{0.652519in}}{\pgfqpoint{1.806744in}{0.648129in}}{\pgfqpoint{1.817794in}{0.648129in}}%
\pgfpathclose%
\pgfusepath{stroke,fill}%
\end{pgfscope}%
\begin{pgfscope}%
\pgfpathrectangle{\pgfqpoint{0.648703in}{0.548769in}}{\pgfqpoint{5.201297in}{3.102590in}}%
\pgfusepath{clip}%
\pgfsetbuttcap%
\pgfsetroundjoin%
\definecolor{currentfill}{rgb}{1.000000,0.498039,0.054902}%
\pgfsetfillcolor{currentfill}%
\pgfsetlinewidth{1.003750pt}%
\definecolor{currentstroke}{rgb}{1.000000,0.498039,0.054902}%
\pgfsetstrokecolor{currentstroke}%
\pgfsetdash{}{0pt}%
\pgfpathmoveto{\pgfqpoint{1.376490in}{3.193800in}}%
\pgfpathcurveto{\pgfqpoint{1.387540in}{3.193800in}}{\pgfqpoint{1.398139in}{3.198191in}}{\pgfqpoint{1.405953in}{3.206004in}}%
\pgfpathcurveto{\pgfqpoint{1.413766in}{3.213818in}}{\pgfqpoint{1.418156in}{3.224417in}}{\pgfqpoint{1.418156in}{3.235467in}}%
\pgfpathcurveto{\pgfqpoint{1.418156in}{3.246517in}}{\pgfqpoint{1.413766in}{3.257116in}}{\pgfqpoint{1.405953in}{3.264930in}}%
\pgfpathcurveto{\pgfqpoint{1.398139in}{3.272743in}}{\pgfqpoint{1.387540in}{3.277134in}}{\pgfqpoint{1.376490in}{3.277134in}}%
\pgfpathcurveto{\pgfqpoint{1.365440in}{3.277134in}}{\pgfqpoint{1.354841in}{3.272743in}}{\pgfqpoint{1.347027in}{3.264930in}}%
\pgfpathcurveto{\pgfqpoint{1.339213in}{3.257116in}}{\pgfqpoint{1.334823in}{3.246517in}}{\pgfqpoint{1.334823in}{3.235467in}}%
\pgfpathcurveto{\pgfqpoint{1.334823in}{3.224417in}}{\pgfqpoint{1.339213in}{3.213818in}}{\pgfqpoint{1.347027in}{3.206004in}}%
\pgfpathcurveto{\pgfqpoint{1.354841in}{3.198191in}}{\pgfqpoint{1.365440in}{3.193800in}}{\pgfqpoint{1.376490in}{3.193800in}}%
\pgfpathclose%
\pgfusepath{stroke,fill}%
\end{pgfscope}%
\begin{pgfscope}%
\pgfpathrectangle{\pgfqpoint{0.648703in}{0.548769in}}{\pgfqpoint{5.201297in}{3.102590in}}%
\pgfusepath{clip}%
\pgfsetbuttcap%
\pgfsetroundjoin%
\definecolor{currentfill}{rgb}{1.000000,0.498039,0.054902}%
\pgfsetfillcolor{currentfill}%
\pgfsetlinewidth{1.003750pt}%
\definecolor{currentstroke}{rgb}{1.000000,0.498039,0.054902}%
\pgfsetstrokecolor{currentstroke}%
\pgfsetdash{}{0pt}%
\pgfpathmoveto{\pgfqpoint{1.473752in}{3.231859in}}%
\pgfpathcurveto{\pgfqpoint{1.484802in}{3.231859in}}{\pgfqpoint{1.495401in}{3.236249in}}{\pgfqpoint{1.503215in}{3.244062in}}%
\pgfpathcurveto{\pgfqpoint{1.511029in}{3.251876in}}{\pgfqpoint{1.515419in}{3.262475in}}{\pgfqpoint{1.515419in}{3.273525in}}%
\pgfpathcurveto{\pgfqpoint{1.515419in}{3.284575in}}{\pgfqpoint{1.511029in}{3.295174in}}{\pgfqpoint{1.503215in}{3.302988in}}%
\pgfpathcurveto{\pgfqpoint{1.495401in}{3.310802in}}{\pgfqpoint{1.484802in}{3.315192in}}{\pgfqpoint{1.473752in}{3.315192in}}%
\pgfpathcurveto{\pgfqpoint{1.462702in}{3.315192in}}{\pgfqpoint{1.452103in}{3.310802in}}{\pgfqpoint{1.444289in}{3.302988in}}%
\pgfpathcurveto{\pgfqpoint{1.436476in}{3.295174in}}{\pgfqpoint{1.432085in}{3.284575in}}{\pgfqpoint{1.432085in}{3.273525in}}%
\pgfpathcurveto{\pgfqpoint{1.432085in}{3.262475in}}{\pgfqpoint{1.436476in}{3.251876in}}{\pgfqpoint{1.444289in}{3.244062in}}%
\pgfpathcurveto{\pgfqpoint{1.452103in}{3.236249in}}{\pgfqpoint{1.462702in}{3.231859in}}{\pgfqpoint{1.473752in}{3.231859in}}%
\pgfpathclose%
\pgfusepath{stroke,fill}%
\end{pgfscope}%
\begin{pgfscope}%
\pgfpathrectangle{\pgfqpoint{0.648703in}{0.548769in}}{\pgfqpoint{5.201297in}{3.102590in}}%
\pgfusepath{clip}%
\pgfsetbuttcap%
\pgfsetroundjoin%
\definecolor{currentfill}{rgb}{0.121569,0.466667,0.705882}%
\pgfsetfillcolor{currentfill}%
\pgfsetlinewidth{1.003750pt}%
\definecolor{currentstroke}{rgb}{0.121569,0.466667,0.705882}%
\pgfsetstrokecolor{currentstroke}%
\pgfsetdash{}{0pt}%
\pgfpathmoveto{\pgfqpoint{1.784319in}{0.758075in}}%
\pgfpathcurveto{\pgfqpoint{1.795369in}{0.758075in}}{\pgfqpoint{1.805968in}{0.762465in}}{\pgfqpoint{1.813782in}{0.770279in}}%
\pgfpathcurveto{\pgfqpoint{1.821596in}{0.778092in}}{\pgfqpoint{1.825986in}{0.788691in}}{\pgfqpoint{1.825986in}{0.799742in}}%
\pgfpathcurveto{\pgfqpoint{1.825986in}{0.810792in}}{\pgfqpoint{1.821596in}{0.821391in}}{\pgfqpoint{1.813782in}{0.829204in}}%
\pgfpathcurveto{\pgfqpoint{1.805968in}{0.837018in}}{\pgfqpoint{1.795369in}{0.841408in}}{\pgfqpoint{1.784319in}{0.841408in}}%
\pgfpathcurveto{\pgfqpoint{1.773269in}{0.841408in}}{\pgfqpoint{1.762670in}{0.837018in}}{\pgfqpoint{1.754856in}{0.829204in}}%
\pgfpathcurveto{\pgfqpoint{1.747043in}{0.821391in}}{\pgfqpoint{1.742653in}{0.810792in}}{\pgfqpoint{1.742653in}{0.799742in}}%
\pgfpathcurveto{\pgfqpoint{1.742653in}{0.788691in}}{\pgfqpoint{1.747043in}{0.778092in}}{\pgfqpoint{1.754856in}{0.770279in}}%
\pgfpathcurveto{\pgfqpoint{1.762670in}{0.762465in}}{\pgfqpoint{1.773269in}{0.758075in}}{\pgfqpoint{1.784319in}{0.758075in}}%
\pgfpathclose%
\pgfusepath{stroke,fill}%
\end{pgfscope}%
\begin{pgfscope}%
\pgfpathrectangle{\pgfqpoint{0.648703in}{0.548769in}}{\pgfqpoint{5.201297in}{3.102590in}}%
\pgfusepath{clip}%
\pgfsetbuttcap%
\pgfsetroundjoin%
\definecolor{currentfill}{rgb}{1.000000,0.498039,0.054902}%
\pgfsetfillcolor{currentfill}%
\pgfsetlinewidth{1.003750pt}%
\definecolor{currentstroke}{rgb}{1.000000,0.498039,0.054902}%
\pgfsetstrokecolor{currentstroke}%
\pgfsetdash{}{0pt}%
\pgfpathmoveto{\pgfqpoint{1.580039in}{3.185343in}}%
\pgfpathcurveto{\pgfqpoint{1.591089in}{3.185343in}}{\pgfqpoint{1.601688in}{3.189733in}}{\pgfqpoint{1.609501in}{3.197547in}}%
\pgfpathcurveto{\pgfqpoint{1.617315in}{3.205360in}}{\pgfqpoint{1.621705in}{3.215959in}}{\pgfqpoint{1.621705in}{3.227010in}}%
\pgfpathcurveto{\pgfqpoint{1.621705in}{3.238060in}}{\pgfqpoint{1.617315in}{3.248659in}}{\pgfqpoint{1.609501in}{3.256472in}}%
\pgfpathcurveto{\pgfqpoint{1.601688in}{3.264286in}}{\pgfqpoint{1.591089in}{3.268676in}}{\pgfqpoint{1.580039in}{3.268676in}}%
\pgfpathcurveto{\pgfqpoint{1.568989in}{3.268676in}}{\pgfqpoint{1.558390in}{3.264286in}}{\pgfqpoint{1.550576in}{3.256472in}}%
\pgfpathcurveto{\pgfqpoint{1.542762in}{3.248659in}}{\pgfqpoint{1.538372in}{3.238060in}}{\pgfqpoint{1.538372in}{3.227010in}}%
\pgfpathcurveto{\pgfqpoint{1.538372in}{3.215959in}}{\pgfqpoint{1.542762in}{3.205360in}}{\pgfqpoint{1.550576in}{3.197547in}}%
\pgfpathcurveto{\pgfqpoint{1.558390in}{3.189733in}}{\pgfqpoint{1.568989in}{3.185343in}}{\pgfqpoint{1.580039in}{3.185343in}}%
\pgfpathclose%
\pgfusepath{stroke,fill}%
\end{pgfscope}%
\begin{pgfscope}%
\pgfpathrectangle{\pgfqpoint{0.648703in}{0.548769in}}{\pgfqpoint{5.201297in}{3.102590in}}%
\pgfusepath{clip}%
\pgfsetbuttcap%
\pgfsetroundjoin%
\definecolor{currentfill}{rgb}{1.000000,0.498039,0.054902}%
\pgfsetfillcolor{currentfill}%
\pgfsetlinewidth{1.003750pt}%
\definecolor{currentstroke}{rgb}{1.000000,0.498039,0.054902}%
\pgfsetstrokecolor{currentstroke}%
\pgfsetdash{}{0pt}%
\pgfpathmoveto{\pgfqpoint{1.765748in}{3.202258in}}%
\pgfpathcurveto{\pgfqpoint{1.776798in}{3.202258in}}{\pgfqpoint{1.787397in}{3.206648in}}{\pgfqpoint{1.795211in}{3.214462in}}%
\pgfpathcurveto{\pgfqpoint{1.803025in}{3.222275in}}{\pgfqpoint{1.807415in}{3.232874in}}{\pgfqpoint{1.807415in}{3.243924in}}%
\pgfpathcurveto{\pgfqpoint{1.807415in}{3.254974in}}{\pgfqpoint{1.803025in}{3.265573in}}{\pgfqpoint{1.795211in}{3.273387in}}%
\pgfpathcurveto{\pgfqpoint{1.787397in}{3.281201in}}{\pgfqpoint{1.776798in}{3.285591in}}{\pgfqpoint{1.765748in}{3.285591in}}%
\pgfpathcurveto{\pgfqpoint{1.754698in}{3.285591in}}{\pgfqpoint{1.744099in}{3.281201in}}{\pgfqpoint{1.736285in}{3.273387in}}%
\pgfpathcurveto{\pgfqpoint{1.728472in}{3.265573in}}{\pgfqpoint{1.724082in}{3.254974in}}{\pgfqpoint{1.724082in}{3.243924in}}%
\pgfpathcurveto{\pgfqpoint{1.724082in}{3.232874in}}{\pgfqpoint{1.728472in}{3.222275in}}{\pgfqpoint{1.736285in}{3.214462in}}%
\pgfpathcurveto{\pgfqpoint{1.744099in}{3.206648in}}{\pgfqpoint{1.754698in}{3.202258in}}{\pgfqpoint{1.765748in}{3.202258in}}%
\pgfpathclose%
\pgfusepath{stroke,fill}%
\end{pgfscope}%
\begin{pgfscope}%
\pgfpathrectangle{\pgfqpoint{0.648703in}{0.548769in}}{\pgfqpoint{5.201297in}{3.102590in}}%
\pgfusepath{clip}%
\pgfsetbuttcap%
\pgfsetroundjoin%
\definecolor{currentfill}{rgb}{0.121569,0.466667,0.705882}%
\pgfsetfillcolor{currentfill}%
\pgfsetlinewidth{1.003750pt}%
\definecolor{currentstroke}{rgb}{0.121569,0.466667,0.705882}%
\pgfsetstrokecolor{currentstroke}%
\pgfsetdash{}{0pt}%
\pgfpathmoveto{\pgfqpoint{5.613577in}{0.648129in}}%
\pgfpathcurveto{\pgfqpoint{5.624628in}{0.648129in}}{\pgfqpoint{5.635227in}{0.652519in}}{\pgfqpoint{5.643040in}{0.660333in}}%
\pgfpathcurveto{\pgfqpoint{5.650854in}{0.668146in}}{\pgfqpoint{5.655244in}{0.678745in}}{\pgfqpoint{5.655244in}{0.689796in}}%
\pgfpathcurveto{\pgfqpoint{5.655244in}{0.700846in}}{\pgfqpoint{5.650854in}{0.711445in}}{\pgfqpoint{5.643040in}{0.719258in}}%
\pgfpathcurveto{\pgfqpoint{5.635227in}{0.727072in}}{\pgfqpoint{5.624628in}{0.731462in}}{\pgfqpoint{5.613577in}{0.731462in}}%
\pgfpathcurveto{\pgfqpoint{5.602527in}{0.731462in}}{\pgfqpoint{5.591928in}{0.727072in}}{\pgfqpoint{5.584115in}{0.719258in}}%
\pgfpathcurveto{\pgfqpoint{5.576301in}{0.711445in}}{\pgfqpoint{5.571911in}{0.700846in}}{\pgfqpoint{5.571911in}{0.689796in}}%
\pgfpathcurveto{\pgfqpoint{5.571911in}{0.678745in}}{\pgfqpoint{5.576301in}{0.668146in}}{\pgfqpoint{5.584115in}{0.660333in}}%
\pgfpathcurveto{\pgfqpoint{5.591928in}{0.652519in}}{\pgfqpoint{5.602527in}{0.648129in}}{\pgfqpoint{5.613577in}{0.648129in}}%
\pgfpathclose%
\pgfusepath{stroke,fill}%
\end{pgfscope}%
\begin{pgfscope}%
\pgfpathrectangle{\pgfqpoint{0.648703in}{0.548769in}}{\pgfqpoint{5.201297in}{3.102590in}}%
\pgfusepath{clip}%
\pgfsetbuttcap%
\pgfsetroundjoin%
\definecolor{currentfill}{rgb}{1.000000,0.498039,0.054902}%
\pgfsetfillcolor{currentfill}%
\pgfsetlinewidth{1.003750pt}%
\definecolor{currentstroke}{rgb}{1.000000,0.498039,0.054902}%
\pgfsetstrokecolor{currentstroke}%
\pgfsetdash{}{0pt}%
\pgfpathmoveto{\pgfqpoint{2.100356in}{3.206486in}}%
\pgfpathcurveto{\pgfqpoint{2.111407in}{3.206486in}}{\pgfqpoint{2.122006in}{3.210877in}}{\pgfqpoint{2.129819in}{3.218690in}}%
\pgfpathcurveto{\pgfqpoint{2.137633in}{3.226504in}}{\pgfqpoint{2.142023in}{3.237103in}}{\pgfqpoint{2.142023in}{3.248153in}}%
\pgfpathcurveto{\pgfqpoint{2.142023in}{3.259203in}}{\pgfqpoint{2.137633in}{3.269802in}}{\pgfqpoint{2.129819in}{3.277616in}}%
\pgfpathcurveto{\pgfqpoint{2.122006in}{3.285429in}}{\pgfqpoint{2.111407in}{3.289820in}}{\pgfqpoint{2.100356in}{3.289820in}}%
\pgfpathcurveto{\pgfqpoint{2.089306in}{3.289820in}}{\pgfqpoint{2.078707in}{3.285429in}}{\pgfqpoint{2.070894in}{3.277616in}}%
\pgfpathcurveto{\pgfqpoint{2.063080in}{3.269802in}}{\pgfqpoint{2.058690in}{3.259203in}}{\pgfqpoint{2.058690in}{3.248153in}}%
\pgfpathcurveto{\pgfqpoint{2.058690in}{3.237103in}}{\pgfqpoint{2.063080in}{3.226504in}}{\pgfqpoint{2.070894in}{3.218690in}}%
\pgfpathcurveto{\pgfqpoint{2.078707in}{3.210877in}}{\pgfqpoint{2.089306in}{3.206486in}}{\pgfqpoint{2.100356in}{3.206486in}}%
\pgfpathclose%
\pgfusepath{stroke,fill}%
\end{pgfscope}%
\begin{pgfscope}%
\pgfpathrectangle{\pgfqpoint{0.648703in}{0.548769in}}{\pgfqpoint{5.201297in}{3.102590in}}%
\pgfusepath{clip}%
\pgfsetbuttcap%
\pgfsetroundjoin%
\definecolor{currentfill}{rgb}{0.121569,0.466667,0.705882}%
\pgfsetfillcolor{currentfill}%
\pgfsetlinewidth{1.003750pt}%
\definecolor{currentstroke}{rgb}{0.121569,0.466667,0.705882}%
\pgfsetstrokecolor{currentstroke}%
\pgfsetdash{}{0pt}%
\pgfpathmoveto{\pgfqpoint{1.362143in}{0.648129in}}%
\pgfpathcurveto{\pgfqpoint{1.373194in}{0.648129in}}{\pgfqpoint{1.383793in}{0.652519in}}{\pgfqpoint{1.391606in}{0.660333in}}%
\pgfpathcurveto{\pgfqpoint{1.399420in}{0.668146in}}{\pgfqpoint{1.403810in}{0.678745in}}{\pgfqpoint{1.403810in}{0.689796in}}%
\pgfpathcurveto{\pgfqpoint{1.403810in}{0.700846in}}{\pgfqpoint{1.399420in}{0.711445in}}{\pgfqpoint{1.391606in}{0.719258in}}%
\pgfpathcurveto{\pgfqpoint{1.383793in}{0.727072in}}{\pgfqpoint{1.373194in}{0.731462in}}{\pgfqpoint{1.362143in}{0.731462in}}%
\pgfpathcurveto{\pgfqpoint{1.351093in}{0.731462in}}{\pgfqpoint{1.340494in}{0.727072in}}{\pgfqpoint{1.332681in}{0.719258in}}%
\pgfpathcurveto{\pgfqpoint{1.324867in}{0.711445in}}{\pgfqpoint{1.320477in}{0.700846in}}{\pgfqpoint{1.320477in}{0.689796in}}%
\pgfpathcurveto{\pgfqpoint{1.320477in}{0.678745in}}{\pgfqpoint{1.324867in}{0.668146in}}{\pgfqpoint{1.332681in}{0.660333in}}%
\pgfpathcurveto{\pgfqpoint{1.340494in}{0.652519in}}{\pgfqpoint{1.351093in}{0.648129in}}{\pgfqpoint{1.362143in}{0.648129in}}%
\pgfpathclose%
\pgfusepath{stroke,fill}%
\end{pgfscope}%
\begin{pgfscope}%
\pgfpathrectangle{\pgfqpoint{0.648703in}{0.548769in}}{\pgfqpoint{5.201297in}{3.102590in}}%
\pgfusepath{clip}%
\pgfsetbuttcap%
\pgfsetroundjoin%
\definecolor{currentfill}{rgb}{0.121569,0.466667,0.705882}%
\pgfsetfillcolor{currentfill}%
\pgfsetlinewidth{1.003750pt}%
\definecolor{currentstroke}{rgb}{0.121569,0.466667,0.705882}%
\pgfsetstrokecolor{currentstroke}%
\pgfsetdash{}{0pt}%
\pgfpathmoveto{\pgfqpoint{1.144179in}{0.648129in}}%
\pgfpathcurveto{\pgfqpoint{1.155229in}{0.648129in}}{\pgfqpoint{1.165828in}{0.652519in}}{\pgfqpoint{1.173641in}{0.660333in}}%
\pgfpathcurveto{\pgfqpoint{1.181455in}{0.668146in}}{\pgfqpoint{1.185845in}{0.678745in}}{\pgfqpoint{1.185845in}{0.689796in}}%
\pgfpathcurveto{\pgfqpoint{1.185845in}{0.700846in}}{\pgfqpoint{1.181455in}{0.711445in}}{\pgfqpoint{1.173641in}{0.719258in}}%
\pgfpathcurveto{\pgfqpoint{1.165828in}{0.727072in}}{\pgfqpoint{1.155229in}{0.731462in}}{\pgfqpoint{1.144179in}{0.731462in}}%
\pgfpathcurveto{\pgfqpoint{1.133129in}{0.731462in}}{\pgfqpoint{1.122529in}{0.727072in}}{\pgfqpoint{1.114716in}{0.719258in}}%
\pgfpathcurveto{\pgfqpoint{1.106902in}{0.711445in}}{\pgfqpoint{1.102512in}{0.700846in}}{\pgfqpoint{1.102512in}{0.689796in}}%
\pgfpathcurveto{\pgfqpoint{1.102512in}{0.678745in}}{\pgfqpoint{1.106902in}{0.668146in}}{\pgfqpoint{1.114716in}{0.660333in}}%
\pgfpathcurveto{\pgfqpoint{1.122529in}{0.652519in}}{\pgfqpoint{1.133129in}{0.648129in}}{\pgfqpoint{1.144179in}{0.648129in}}%
\pgfpathclose%
\pgfusepath{stroke,fill}%
\end{pgfscope}%
\begin{pgfscope}%
\pgfpathrectangle{\pgfqpoint{0.648703in}{0.548769in}}{\pgfqpoint{5.201297in}{3.102590in}}%
\pgfusepath{clip}%
\pgfsetbuttcap%
\pgfsetroundjoin%
\definecolor{currentfill}{rgb}{0.121569,0.466667,0.705882}%
\pgfsetfillcolor{currentfill}%
\pgfsetlinewidth{1.003750pt}%
\definecolor{currentstroke}{rgb}{0.121569,0.466667,0.705882}%
\pgfsetstrokecolor{currentstroke}%
\pgfsetdash{}{0pt}%
\pgfpathmoveto{\pgfqpoint{1.372239in}{0.817277in}}%
\pgfpathcurveto{\pgfqpoint{1.383289in}{0.817277in}}{\pgfqpoint{1.393888in}{0.821667in}}{\pgfqpoint{1.401702in}{0.829480in}}%
\pgfpathcurveto{\pgfqpoint{1.409515in}{0.837294in}}{\pgfqpoint{1.413906in}{0.847893in}}{\pgfqpoint{1.413906in}{0.858943in}}%
\pgfpathcurveto{\pgfqpoint{1.413906in}{0.869993in}}{\pgfqpoint{1.409515in}{0.880592in}}{\pgfqpoint{1.401702in}{0.888406in}}%
\pgfpathcurveto{\pgfqpoint{1.393888in}{0.896220in}}{\pgfqpoint{1.383289in}{0.900610in}}{\pgfqpoint{1.372239in}{0.900610in}}%
\pgfpathcurveto{\pgfqpoint{1.361189in}{0.900610in}}{\pgfqpoint{1.350590in}{0.896220in}}{\pgfqpoint{1.342776in}{0.888406in}}%
\pgfpathcurveto{\pgfqpoint{1.334963in}{0.880592in}}{\pgfqpoint{1.330572in}{0.869993in}}{\pgfqpoint{1.330572in}{0.858943in}}%
\pgfpathcurveto{\pgfqpoint{1.330572in}{0.847893in}}{\pgfqpoint{1.334963in}{0.837294in}}{\pgfqpoint{1.342776in}{0.829480in}}%
\pgfpathcurveto{\pgfqpoint{1.350590in}{0.821667in}}{\pgfqpoint{1.361189in}{0.817277in}}{\pgfqpoint{1.372239in}{0.817277in}}%
\pgfpathclose%
\pgfusepath{stroke,fill}%
\end{pgfscope}%
\begin{pgfscope}%
\pgfpathrectangle{\pgfqpoint{0.648703in}{0.548769in}}{\pgfqpoint{5.201297in}{3.102590in}}%
\pgfusepath{clip}%
\pgfsetbuttcap%
\pgfsetroundjoin%
\definecolor{currentfill}{rgb}{1.000000,0.498039,0.054902}%
\pgfsetfillcolor{currentfill}%
\pgfsetlinewidth{1.003750pt}%
\definecolor{currentstroke}{rgb}{1.000000,0.498039,0.054902}%
\pgfsetstrokecolor{currentstroke}%
\pgfsetdash{}{0pt}%
\pgfpathmoveto{\pgfqpoint{2.270979in}{3.189572in}}%
\pgfpathcurveto{\pgfqpoint{2.282029in}{3.189572in}}{\pgfqpoint{2.292628in}{3.193962in}}{\pgfqpoint{2.300442in}{3.201775in}}%
\pgfpathcurveto{\pgfqpoint{2.308256in}{3.209589in}}{\pgfqpoint{2.312646in}{3.220188in}}{\pgfqpoint{2.312646in}{3.231238in}}%
\pgfpathcurveto{\pgfqpoint{2.312646in}{3.242288in}}{\pgfqpoint{2.308256in}{3.252887in}}{\pgfqpoint{2.300442in}{3.260701in}}%
\pgfpathcurveto{\pgfqpoint{2.292628in}{3.268515in}}{\pgfqpoint{2.282029in}{3.272905in}}{\pgfqpoint{2.270979in}{3.272905in}}%
\pgfpathcurveto{\pgfqpoint{2.259929in}{3.272905in}}{\pgfqpoint{2.249330in}{3.268515in}}{\pgfqpoint{2.241517in}{3.260701in}}%
\pgfpathcurveto{\pgfqpoint{2.233703in}{3.252887in}}{\pgfqpoint{2.229313in}{3.242288in}}{\pgfqpoint{2.229313in}{3.231238in}}%
\pgfpathcurveto{\pgfqpoint{2.229313in}{3.220188in}}{\pgfqpoint{2.233703in}{3.209589in}}{\pgfqpoint{2.241517in}{3.201775in}}%
\pgfpathcurveto{\pgfqpoint{2.249330in}{3.193962in}}{\pgfqpoint{2.259929in}{3.189572in}}{\pgfqpoint{2.270979in}{3.189572in}}%
\pgfpathclose%
\pgfusepath{stroke,fill}%
\end{pgfscope}%
\begin{pgfscope}%
\pgfpathrectangle{\pgfqpoint{0.648703in}{0.548769in}}{\pgfqpoint{5.201297in}{3.102590in}}%
\pgfusepath{clip}%
\pgfsetbuttcap%
\pgfsetroundjoin%
\definecolor{currentfill}{rgb}{1.000000,0.498039,0.054902}%
\pgfsetfillcolor{currentfill}%
\pgfsetlinewidth{1.003750pt}%
\definecolor{currentstroke}{rgb}{1.000000,0.498039,0.054902}%
\pgfsetstrokecolor{currentstroke}%
\pgfsetdash{}{0pt}%
\pgfpathmoveto{\pgfqpoint{1.332327in}{3.214944in}}%
\pgfpathcurveto{\pgfqpoint{1.343377in}{3.214944in}}{\pgfqpoint{1.353976in}{3.219334in}}{\pgfqpoint{1.361790in}{3.227148in}}%
\pgfpathcurveto{\pgfqpoint{1.369604in}{3.234961in}}{\pgfqpoint{1.373994in}{3.245560in}}{\pgfqpoint{1.373994in}{3.256610in}}%
\pgfpathcurveto{\pgfqpoint{1.373994in}{3.267661in}}{\pgfqpoint{1.369604in}{3.278260in}}{\pgfqpoint{1.361790in}{3.286073in}}%
\pgfpathcurveto{\pgfqpoint{1.353976in}{3.293887in}}{\pgfqpoint{1.343377in}{3.298277in}}{\pgfqpoint{1.332327in}{3.298277in}}%
\pgfpathcurveto{\pgfqpoint{1.321277in}{3.298277in}}{\pgfqpoint{1.310678in}{3.293887in}}{\pgfqpoint{1.302864in}{3.286073in}}%
\pgfpathcurveto{\pgfqpoint{1.295051in}{3.278260in}}{\pgfqpoint{1.290661in}{3.267661in}}{\pgfqpoint{1.290661in}{3.256610in}}%
\pgfpathcurveto{\pgfqpoint{1.290661in}{3.245560in}}{\pgfqpoint{1.295051in}{3.234961in}}{\pgfqpoint{1.302864in}{3.227148in}}%
\pgfpathcurveto{\pgfqpoint{1.310678in}{3.219334in}}{\pgfqpoint{1.321277in}{3.214944in}}{\pgfqpoint{1.332327in}{3.214944in}}%
\pgfpathclose%
\pgfusepath{stroke,fill}%
\end{pgfscope}%
\begin{pgfscope}%
\pgfpathrectangle{\pgfqpoint{0.648703in}{0.548769in}}{\pgfqpoint{5.201297in}{3.102590in}}%
\pgfusepath{clip}%
\pgfsetbuttcap%
\pgfsetroundjoin%
\definecolor{currentfill}{rgb}{0.121569,0.466667,0.705882}%
\pgfsetfillcolor{currentfill}%
\pgfsetlinewidth{1.003750pt}%
\definecolor{currentstroke}{rgb}{0.121569,0.466667,0.705882}%
\pgfsetstrokecolor{currentstroke}%
\pgfsetdash{}{0pt}%
\pgfpathmoveto{\pgfqpoint{1.278583in}{0.648129in}}%
\pgfpathcurveto{\pgfqpoint{1.289633in}{0.648129in}}{\pgfqpoint{1.300232in}{0.652519in}}{\pgfqpoint{1.308046in}{0.660333in}}%
\pgfpathcurveto{\pgfqpoint{1.315859in}{0.668146in}}{\pgfqpoint{1.320250in}{0.678745in}}{\pgfqpoint{1.320250in}{0.689796in}}%
\pgfpathcurveto{\pgfqpoint{1.320250in}{0.700846in}}{\pgfqpoint{1.315859in}{0.711445in}}{\pgfqpoint{1.308046in}{0.719258in}}%
\pgfpathcurveto{\pgfqpoint{1.300232in}{0.727072in}}{\pgfqpoint{1.289633in}{0.731462in}}{\pgfqpoint{1.278583in}{0.731462in}}%
\pgfpathcurveto{\pgfqpoint{1.267533in}{0.731462in}}{\pgfqpoint{1.256934in}{0.727072in}}{\pgfqpoint{1.249120in}{0.719258in}}%
\pgfpathcurveto{\pgfqpoint{1.241306in}{0.711445in}}{\pgfqpoint{1.236916in}{0.700846in}}{\pgfqpoint{1.236916in}{0.689796in}}%
\pgfpathcurveto{\pgfqpoint{1.236916in}{0.678745in}}{\pgfqpoint{1.241306in}{0.668146in}}{\pgfqpoint{1.249120in}{0.660333in}}%
\pgfpathcurveto{\pgfqpoint{1.256934in}{0.652519in}}{\pgfqpoint{1.267533in}{0.648129in}}{\pgfqpoint{1.278583in}{0.648129in}}%
\pgfpathclose%
\pgfusepath{stroke,fill}%
\end{pgfscope}%
\begin{pgfscope}%
\pgfpathrectangle{\pgfqpoint{0.648703in}{0.548769in}}{\pgfqpoint{5.201297in}{3.102590in}}%
\pgfusepath{clip}%
\pgfsetbuttcap%
\pgfsetroundjoin%
\definecolor{currentfill}{rgb}{0.121569,0.466667,0.705882}%
\pgfsetfillcolor{currentfill}%
\pgfsetlinewidth{1.003750pt}%
\definecolor{currentstroke}{rgb}{0.121569,0.466667,0.705882}%
\pgfsetstrokecolor{currentstroke}%
\pgfsetdash{}{0pt}%
\pgfpathmoveto{\pgfqpoint{1.168830in}{0.758075in}}%
\pgfpathcurveto{\pgfqpoint{1.179880in}{0.758075in}}{\pgfqpoint{1.190479in}{0.762465in}}{\pgfqpoint{1.198292in}{0.770279in}}%
\pgfpathcurveto{\pgfqpoint{1.206106in}{0.778092in}}{\pgfqpoint{1.210496in}{0.788691in}}{\pgfqpoint{1.210496in}{0.799742in}}%
\pgfpathcurveto{\pgfqpoint{1.210496in}{0.810792in}}{\pgfqpoint{1.206106in}{0.821391in}}{\pgfqpoint{1.198292in}{0.829204in}}%
\pgfpathcurveto{\pgfqpoint{1.190479in}{0.837018in}}{\pgfqpoint{1.179880in}{0.841408in}}{\pgfqpoint{1.168830in}{0.841408in}}%
\pgfpathcurveto{\pgfqpoint{1.157779in}{0.841408in}}{\pgfqpoint{1.147180in}{0.837018in}}{\pgfqpoint{1.139367in}{0.829204in}}%
\pgfpathcurveto{\pgfqpoint{1.131553in}{0.821391in}}{\pgfqpoint{1.127163in}{0.810792in}}{\pgfqpoint{1.127163in}{0.799742in}}%
\pgfpathcurveto{\pgfqpoint{1.127163in}{0.788691in}}{\pgfqpoint{1.131553in}{0.778092in}}{\pgfqpoint{1.139367in}{0.770279in}}%
\pgfpathcurveto{\pgfqpoint{1.147180in}{0.762465in}}{\pgfqpoint{1.157779in}{0.758075in}}{\pgfqpoint{1.168830in}{0.758075in}}%
\pgfpathclose%
\pgfusepath{stroke,fill}%
\end{pgfscope}%
\begin{pgfscope}%
\pgfpathrectangle{\pgfqpoint{0.648703in}{0.548769in}}{\pgfqpoint{5.201297in}{3.102590in}}%
\pgfusepath{clip}%
\pgfsetbuttcap%
\pgfsetroundjoin%
\definecolor{currentfill}{rgb}{1.000000,0.498039,0.054902}%
\pgfsetfillcolor{currentfill}%
\pgfsetlinewidth{1.003750pt}%
\definecolor{currentstroke}{rgb}{1.000000,0.498039,0.054902}%
\pgfsetstrokecolor{currentstroke}%
\pgfsetdash{}{0pt}%
\pgfpathmoveto{\pgfqpoint{1.961100in}{3.193800in}}%
\pgfpathcurveto{\pgfqpoint{1.972151in}{3.193800in}}{\pgfqpoint{1.982750in}{3.198191in}}{\pgfqpoint{1.990563in}{3.206004in}}%
\pgfpathcurveto{\pgfqpoint{1.998377in}{3.213818in}}{\pgfqpoint{2.002767in}{3.224417in}}{\pgfqpoint{2.002767in}{3.235467in}}%
\pgfpathcurveto{\pgfqpoint{2.002767in}{3.246517in}}{\pgfqpoint{1.998377in}{3.257116in}}{\pgfqpoint{1.990563in}{3.264930in}}%
\pgfpathcurveto{\pgfqpoint{1.982750in}{3.272743in}}{\pgfqpoint{1.972151in}{3.277134in}}{\pgfqpoint{1.961100in}{3.277134in}}%
\pgfpathcurveto{\pgfqpoint{1.950050in}{3.277134in}}{\pgfqpoint{1.939451in}{3.272743in}}{\pgfqpoint{1.931638in}{3.264930in}}%
\pgfpathcurveto{\pgfqpoint{1.923824in}{3.257116in}}{\pgfqpoint{1.919434in}{3.246517in}}{\pgfqpoint{1.919434in}{3.235467in}}%
\pgfpathcurveto{\pgfqpoint{1.919434in}{3.224417in}}{\pgfqpoint{1.923824in}{3.213818in}}{\pgfqpoint{1.931638in}{3.206004in}}%
\pgfpathcurveto{\pgfqpoint{1.939451in}{3.198191in}}{\pgfqpoint{1.950050in}{3.193800in}}{\pgfqpoint{1.961100in}{3.193800in}}%
\pgfpathclose%
\pgfusepath{stroke,fill}%
\end{pgfscope}%
\begin{pgfscope}%
\pgfpathrectangle{\pgfqpoint{0.648703in}{0.548769in}}{\pgfqpoint{5.201297in}{3.102590in}}%
\pgfusepath{clip}%
\pgfsetbuttcap%
\pgfsetroundjoin%
\definecolor{currentfill}{rgb}{0.121569,0.466667,0.705882}%
\pgfsetfillcolor{currentfill}%
\pgfsetlinewidth{1.003750pt}%
\definecolor{currentstroke}{rgb}{0.121569,0.466667,0.705882}%
\pgfsetstrokecolor{currentstroke}%
\pgfsetdash{}{0pt}%
\pgfpathmoveto{\pgfqpoint{0.983625in}{0.648129in}}%
\pgfpathcurveto{\pgfqpoint{0.994675in}{0.648129in}}{\pgfqpoint{1.005274in}{0.652519in}}{\pgfqpoint{1.013088in}{0.660333in}}%
\pgfpathcurveto{\pgfqpoint{1.020902in}{0.668146in}}{\pgfqpoint{1.025292in}{0.678745in}}{\pgfqpoint{1.025292in}{0.689796in}}%
\pgfpathcurveto{\pgfqpoint{1.025292in}{0.700846in}}{\pgfqpoint{1.020902in}{0.711445in}}{\pgfqpoint{1.013088in}{0.719258in}}%
\pgfpathcurveto{\pgfqpoint{1.005274in}{0.727072in}}{\pgfqpoint{0.994675in}{0.731462in}}{\pgfqpoint{0.983625in}{0.731462in}}%
\pgfpathcurveto{\pgfqpoint{0.972575in}{0.731462in}}{\pgfqpoint{0.961976in}{0.727072in}}{\pgfqpoint{0.954162in}{0.719258in}}%
\pgfpathcurveto{\pgfqpoint{0.946349in}{0.711445in}}{\pgfqpoint{0.941959in}{0.700846in}}{\pgfqpoint{0.941959in}{0.689796in}}%
\pgfpathcurveto{\pgfqpoint{0.941959in}{0.678745in}}{\pgfqpoint{0.946349in}{0.668146in}}{\pgfqpoint{0.954162in}{0.660333in}}%
\pgfpathcurveto{\pgfqpoint{0.961976in}{0.652519in}}{\pgfqpoint{0.972575in}{0.648129in}}{\pgfqpoint{0.983625in}{0.648129in}}%
\pgfpathclose%
\pgfusepath{stroke,fill}%
\end{pgfscope}%
\begin{pgfscope}%
\pgfpathrectangle{\pgfqpoint{0.648703in}{0.548769in}}{\pgfqpoint{5.201297in}{3.102590in}}%
\pgfusepath{clip}%
\pgfsetbuttcap%
\pgfsetroundjoin%
\definecolor{currentfill}{rgb}{1.000000,0.498039,0.054902}%
\pgfsetfillcolor{currentfill}%
\pgfsetlinewidth{1.003750pt}%
\definecolor{currentstroke}{rgb}{1.000000,0.498039,0.054902}%
\pgfsetstrokecolor{currentstroke}%
\pgfsetdash{}{0pt}%
\pgfpathmoveto{\pgfqpoint{1.713328in}{3.198029in}}%
\pgfpathcurveto{\pgfqpoint{1.724378in}{3.198029in}}{\pgfqpoint{1.734977in}{3.202419in}}{\pgfqpoint{1.742791in}{3.210233in}}%
\pgfpathcurveto{\pgfqpoint{1.750604in}{3.218046in}}{\pgfqpoint{1.754995in}{3.228646in}}{\pgfqpoint{1.754995in}{3.239696in}}%
\pgfpathcurveto{\pgfqpoint{1.754995in}{3.250746in}}{\pgfqpoint{1.750604in}{3.261345in}}{\pgfqpoint{1.742791in}{3.269158in}}%
\pgfpathcurveto{\pgfqpoint{1.734977in}{3.276972in}}{\pgfqpoint{1.724378in}{3.281362in}}{\pgfqpoint{1.713328in}{3.281362in}}%
\pgfpathcurveto{\pgfqpoint{1.702278in}{3.281362in}}{\pgfqpoint{1.691679in}{3.276972in}}{\pgfqpoint{1.683865in}{3.269158in}}%
\pgfpathcurveto{\pgfqpoint{1.676052in}{3.261345in}}{\pgfqpoint{1.671661in}{3.250746in}}{\pgfqpoint{1.671661in}{3.239696in}}%
\pgfpathcurveto{\pgfqpoint{1.671661in}{3.228646in}}{\pgfqpoint{1.676052in}{3.218046in}}{\pgfqpoint{1.683865in}{3.210233in}}%
\pgfpathcurveto{\pgfqpoint{1.691679in}{3.202419in}}{\pgfqpoint{1.702278in}{3.198029in}}{\pgfqpoint{1.713328in}{3.198029in}}%
\pgfpathclose%
\pgfusepath{stroke,fill}%
\end{pgfscope}%
\begin{pgfscope}%
\pgfpathrectangle{\pgfqpoint{0.648703in}{0.548769in}}{\pgfqpoint{5.201297in}{3.102590in}}%
\pgfusepath{clip}%
\pgfsetbuttcap%
\pgfsetroundjoin%
\definecolor{currentfill}{rgb}{1.000000,0.498039,0.054902}%
\pgfsetfillcolor{currentfill}%
\pgfsetlinewidth{1.003750pt}%
\definecolor{currentstroke}{rgb}{1.000000,0.498039,0.054902}%
\pgfsetstrokecolor{currentstroke}%
\pgfsetdash{}{0pt}%
\pgfpathmoveto{\pgfqpoint{1.586093in}{3.193800in}}%
\pgfpathcurveto{\pgfqpoint{1.597143in}{3.193800in}}{\pgfqpoint{1.607742in}{3.198191in}}{\pgfqpoint{1.615555in}{3.206004in}}%
\pgfpathcurveto{\pgfqpoint{1.623369in}{3.213818in}}{\pgfqpoint{1.627759in}{3.224417in}}{\pgfqpoint{1.627759in}{3.235467in}}%
\pgfpathcurveto{\pgfqpoint{1.627759in}{3.246517in}}{\pgfqpoint{1.623369in}{3.257116in}}{\pgfqpoint{1.615555in}{3.264930in}}%
\pgfpathcurveto{\pgfqpoint{1.607742in}{3.272743in}}{\pgfqpoint{1.597143in}{3.277134in}}{\pgfqpoint{1.586093in}{3.277134in}}%
\pgfpathcurveto{\pgfqpoint{1.575042in}{3.277134in}}{\pgfqpoint{1.564443in}{3.272743in}}{\pgfqpoint{1.556630in}{3.264930in}}%
\pgfpathcurveto{\pgfqpoint{1.548816in}{3.257116in}}{\pgfqpoint{1.544426in}{3.246517in}}{\pgfqpoint{1.544426in}{3.235467in}}%
\pgfpathcurveto{\pgfqpoint{1.544426in}{3.224417in}}{\pgfqpoint{1.548816in}{3.213818in}}{\pgfqpoint{1.556630in}{3.206004in}}%
\pgfpathcurveto{\pgfqpoint{1.564443in}{3.198191in}}{\pgfqpoint{1.575042in}{3.193800in}}{\pgfqpoint{1.586093in}{3.193800in}}%
\pgfpathclose%
\pgfusepath{stroke,fill}%
\end{pgfscope}%
\begin{pgfscope}%
\pgfpathrectangle{\pgfqpoint{0.648703in}{0.548769in}}{\pgfqpoint{5.201297in}{3.102590in}}%
\pgfusepath{clip}%
\pgfsetbuttcap%
\pgfsetroundjoin%
\definecolor{currentfill}{rgb}{0.121569,0.466667,0.705882}%
\pgfsetfillcolor{currentfill}%
\pgfsetlinewidth{1.003750pt}%
\definecolor{currentstroke}{rgb}{0.121569,0.466667,0.705882}%
\pgfsetstrokecolor{currentstroke}%
\pgfsetdash{}{0pt}%
\pgfpathmoveto{\pgfqpoint{1.976762in}{0.652358in}}%
\pgfpathcurveto{\pgfqpoint{1.987812in}{0.652358in}}{\pgfqpoint{1.998411in}{0.656748in}}{\pgfqpoint{2.006225in}{0.664562in}}%
\pgfpathcurveto{\pgfqpoint{2.014038in}{0.672375in}}{\pgfqpoint{2.018429in}{0.682974in}}{\pgfqpoint{2.018429in}{0.694024in}}%
\pgfpathcurveto{\pgfqpoint{2.018429in}{0.705074in}}{\pgfqpoint{2.014038in}{0.715673in}}{\pgfqpoint{2.006225in}{0.723487in}}%
\pgfpathcurveto{\pgfqpoint{1.998411in}{0.731301in}}{\pgfqpoint{1.987812in}{0.735691in}}{\pgfqpoint{1.976762in}{0.735691in}}%
\pgfpathcurveto{\pgfqpoint{1.965712in}{0.735691in}}{\pgfqpoint{1.955113in}{0.731301in}}{\pgfqpoint{1.947299in}{0.723487in}}%
\pgfpathcurveto{\pgfqpoint{1.939486in}{0.715673in}}{\pgfqpoint{1.935095in}{0.705074in}}{\pgfqpoint{1.935095in}{0.694024in}}%
\pgfpathcurveto{\pgfqpoint{1.935095in}{0.682974in}}{\pgfqpoint{1.939486in}{0.672375in}}{\pgfqpoint{1.947299in}{0.664562in}}%
\pgfpathcurveto{\pgfqpoint{1.955113in}{0.656748in}}{\pgfqpoint{1.965712in}{0.652358in}}{\pgfqpoint{1.976762in}{0.652358in}}%
\pgfpathclose%
\pgfusepath{stroke,fill}%
\end{pgfscope}%
\begin{pgfscope}%
\pgfpathrectangle{\pgfqpoint{0.648703in}{0.548769in}}{\pgfqpoint{5.201297in}{3.102590in}}%
\pgfusepath{clip}%
\pgfsetbuttcap%
\pgfsetroundjoin%
\definecolor{currentfill}{rgb}{0.121569,0.466667,0.705882}%
\pgfsetfillcolor{currentfill}%
\pgfsetlinewidth{1.003750pt}%
\definecolor{currentstroke}{rgb}{0.121569,0.466667,0.705882}%
\pgfsetstrokecolor{currentstroke}%
\pgfsetdash{}{0pt}%
\pgfpathmoveto{\pgfqpoint{2.393433in}{0.648129in}}%
\pgfpathcurveto{\pgfqpoint{2.404483in}{0.648129in}}{\pgfqpoint{2.415082in}{0.652519in}}{\pgfqpoint{2.422895in}{0.660333in}}%
\pgfpathcurveto{\pgfqpoint{2.430709in}{0.668146in}}{\pgfqpoint{2.435099in}{0.678745in}}{\pgfqpoint{2.435099in}{0.689796in}}%
\pgfpathcurveto{\pgfqpoint{2.435099in}{0.700846in}}{\pgfqpoint{2.430709in}{0.711445in}}{\pgfqpoint{2.422895in}{0.719258in}}%
\pgfpathcurveto{\pgfqpoint{2.415082in}{0.727072in}}{\pgfqpoint{2.404483in}{0.731462in}}{\pgfqpoint{2.393433in}{0.731462in}}%
\pgfpathcurveto{\pgfqpoint{2.382383in}{0.731462in}}{\pgfqpoint{2.371783in}{0.727072in}}{\pgfqpoint{2.363970in}{0.719258in}}%
\pgfpathcurveto{\pgfqpoint{2.356156in}{0.711445in}}{\pgfqpoint{2.351766in}{0.700846in}}{\pgfqpoint{2.351766in}{0.689796in}}%
\pgfpathcurveto{\pgfqpoint{2.351766in}{0.678745in}}{\pgfqpoint{2.356156in}{0.668146in}}{\pgfqpoint{2.363970in}{0.660333in}}%
\pgfpathcurveto{\pgfqpoint{2.371783in}{0.652519in}}{\pgfqpoint{2.382383in}{0.648129in}}{\pgfqpoint{2.393433in}{0.648129in}}%
\pgfpathclose%
\pgfusepath{stroke,fill}%
\end{pgfscope}%
\begin{pgfscope}%
\pgfpathrectangle{\pgfqpoint{0.648703in}{0.548769in}}{\pgfqpoint{5.201297in}{3.102590in}}%
\pgfusepath{clip}%
\pgfsetbuttcap%
\pgfsetroundjoin%
\definecolor{currentfill}{rgb}{0.121569,0.466667,0.705882}%
\pgfsetfillcolor{currentfill}%
\pgfsetlinewidth{1.003750pt}%
\definecolor{currentstroke}{rgb}{0.121569,0.466667,0.705882}%
\pgfsetstrokecolor{currentstroke}%
\pgfsetdash{}{0pt}%
\pgfpathmoveto{\pgfqpoint{1.287398in}{0.648129in}}%
\pgfpathcurveto{\pgfqpoint{1.298448in}{0.648129in}}{\pgfqpoint{1.309047in}{0.652519in}}{\pgfqpoint{1.316861in}{0.660333in}}%
\pgfpathcurveto{\pgfqpoint{1.324674in}{0.668146in}}{\pgfqpoint{1.329065in}{0.678745in}}{\pgfqpoint{1.329065in}{0.689796in}}%
\pgfpathcurveto{\pgfqpoint{1.329065in}{0.700846in}}{\pgfqpoint{1.324674in}{0.711445in}}{\pgfqpoint{1.316861in}{0.719258in}}%
\pgfpathcurveto{\pgfqpoint{1.309047in}{0.727072in}}{\pgfqpoint{1.298448in}{0.731462in}}{\pgfqpoint{1.287398in}{0.731462in}}%
\pgfpathcurveto{\pgfqpoint{1.276348in}{0.731462in}}{\pgfqpoint{1.265749in}{0.727072in}}{\pgfqpoint{1.257935in}{0.719258in}}%
\pgfpathcurveto{\pgfqpoint{1.250122in}{0.711445in}}{\pgfqpoint{1.245731in}{0.700846in}}{\pgfqpoint{1.245731in}{0.689796in}}%
\pgfpathcurveto{\pgfqpoint{1.245731in}{0.678745in}}{\pgfqpoint{1.250122in}{0.668146in}}{\pgfqpoint{1.257935in}{0.660333in}}%
\pgfpathcurveto{\pgfqpoint{1.265749in}{0.652519in}}{\pgfqpoint{1.276348in}{0.648129in}}{\pgfqpoint{1.287398in}{0.648129in}}%
\pgfpathclose%
\pgfusepath{stroke,fill}%
\end{pgfscope}%
\begin{pgfscope}%
\pgfpathrectangle{\pgfqpoint{0.648703in}{0.548769in}}{\pgfqpoint{5.201297in}{3.102590in}}%
\pgfusepath{clip}%
\pgfsetbuttcap%
\pgfsetroundjoin%
\definecolor{currentfill}{rgb}{0.121569,0.466667,0.705882}%
\pgfsetfillcolor{currentfill}%
\pgfsetlinewidth{1.003750pt}%
\definecolor{currentstroke}{rgb}{0.121569,0.466667,0.705882}%
\pgfsetstrokecolor{currentstroke}%
\pgfsetdash{}{0pt}%
\pgfpathmoveto{\pgfqpoint{1.452376in}{0.648129in}}%
\pgfpathcurveto{\pgfqpoint{1.463427in}{0.648129in}}{\pgfqpoint{1.474026in}{0.652519in}}{\pgfqpoint{1.481839in}{0.660333in}}%
\pgfpathcurveto{\pgfqpoint{1.489653in}{0.668146in}}{\pgfqpoint{1.494043in}{0.678745in}}{\pgfqpoint{1.494043in}{0.689796in}}%
\pgfpathcurveto{\pgfqpoint{1.494043in}{0.700846in}}{\pgfqpoint{1.489653in}{0.711445in}}{\pgfqpoint{1.481839in}{0.719258in}}%
\pgfpathcurveto{\pgfqpoint{1.474026in}{0.727072in}}{\pgfqpoint{1.463427in}{0.731462in}}{\pgfqpoint{1.452376in}{0.731462in}}%
\pgfpathcurveto{\pgfqpoint{1.441326in}{0.731462in}}{\pgfqpoint{1.430727in}{0.727072in}}{\pgfqpoint{1.422914in}{0.719258in}}%
\pgfpathcurveto{\pgfqpoint{1.415100in}{0.711445in}}{\pgfqpoint{1.410710in}{0.700846in}}{\pgfqpoint{1.410710in}{0.689796in}}%
\pgfpathcurveto{\pgfqpoint{1.410710in}{0.678745in}}{\pgfqpoint{1.415100in}{0.668146in}}{\pgfqpoint{1.422914in}{0.660333in}}%
\pgfpathcurveto{\pgfqpoint{1.430727in}{0.652519in}}{\pgfqpoint{1.441326in}{0.648129in}}{\pgfqpoint{1.452376in}{0.648129in}}%
\pgfpathclose%
\pgfusepath{stroke,fill}%
\end{pgfscope}%
\begin{pgfscope}%
\pgfpathrectangle{\pgfqpoint{0.648703in}{0.548769in}}{\pgfqpoint{5.201297in}{3.102590in}}%
\pgfusepath{clip}%
\pgfsetbuttcap%
\pgfsetroundjoin%
\definecolor{currentfill}{rgb}{0.121569,0.466667,0.705882}%
\pgfsetfillcolor{currentfill}%
\pgfsetlinewidth{1.003750pt}%
\definecolor{currentstroke}{rgb}{0.121569,0.466667,0.705882}%
\pgfsetstrokecolor{currentstroke}%
\pgfsetdash{}{0pt}%
\pgfpathmoveto{\pgfqpoint{1.229159in}{0.648129in}}%
\pgfpathcurveto{\pgfqpoint{1.240209in}{0.648129in}}{\pgfqpoint{1.250808in}{0.652519in}}{\pgfqpoint{1.258622in}{0.660333in}}%
\pgfpathcurveto{\pgfqpoint{1.266435in}{0.668146in}}{\pgfqpoint{1.270826in}{0.678745in}}{\pgfqpoint{1.270826in}{0.689796in}}%
\pgfpathcurveto{\pgfqpoint{1.270826in}{0.700846in}}{\pgfqpoint{1.266435in}{0.711445in}}{\pgfqpoint{1.258622in}{0.719258in}}%
\pgfpathcurveto{\pgfqpoint{1.250808in}{0.727072in}}{\pgfqpoint{1.240209in}{0.731462in}}{\pgfqpoint{1.229159in}{0.731462in}}%
\pgfpathcurveto{\pgfqpoint{1.218109in}{0.731462in}}{\pgfqpoint{1.207510in}{0.727072in}}{\pgfqpoint{1.199696in}{0.719258in}}%
\pgfpathcurveto{\pgfqpoint{1.191883in}{0.711445in}}{\pgfqpoint{1.187492in}{0.700846in}}{\pgfqpoint{1.187492in}{0.689796in}}%
\pgfpathcurveto{\pgfqpoint{1.187492in}{0.678745in}}{\pgfqpoint{1.191883in}{0.668146in}}{\pgfqpoint{1.199696in}{0.660333in}}%
\pgfpathcurveto{\pgfqpoint{1.207510in}{0.652519in}}{\pgfqpoint{1.218109in}{0.648129in}}{\pgfqpoint{1.229159in}{0.648129in}}%
\pgfpathclose%
\pgfusepath{stroke,fill}%
\end{pgfscope}%
\begin{pgfscope}%
\pgfpathrectangle{\pgfqpoint{0.648703in}{0.548769in}}{\pgfqpoint{5.201297in}{3.102590in}}%
\pgfusepath{clip}%
\pgfsetbuttcap%
\pgfsetroundjoin%
\definecolor{currentfill}{rgb}{0.121569,0.466667,0.705882}%
\pgfsetfillcolor{currentfill}%
\pgfsetlinewidth{1.003750pt}%
\definecolor{currentstroke}{rgb}{0.121569,0.466667,0.705882}%
\pgfsetstrokecolor{currentstroke}%
\pgfsetdash{}{0pt}%
\pgfpathmoveto{\pgfqpoint{1.398170in}{0.648129in}}%
\pgfpathcurveto{\pgfqpoint{1.409221in}{0.648129in}}{\pgfqpoint{1.419820in}{0.652519in}}{\pgfqpoint{1.427633in}{0.660333in}}%
\pgfpathcurveto{\pgfqpoint{1.435447in}{0.668146in}}{\pgfqpoint{1.439837in}{0.678745in}}{\pgfqpoint{1.439837in}{0.689796in}}%
\pgfpathcurveto{\pgfqpoint{1.439837in}{0.700846in}}{\pgfqpoint{1.435447in}{0.711445in}}{\pgfqpoint{1.427633in}{0.719258in}}%
\pgfpathcurveto{\pgfqpoint{1.419820in}{0.727072in}}{\pgfqpoint{1.409221in}{0.731462in}}{\pgfqpoint{1.398170in}{0.731462in}}%
\pgfpathcurveto{\pgfqpoint{1.387120in}{0.731462in}}{\pgfqpoint{1.376521in}{0.727072in}}{\pgfqpoint{1.368708in}{0.719258in}}%
\pgfpathcurveto{\pgfqpoint{1.360894in}{0.711445in}}{\pgfqpoint{1.356504in}{0.700846in}}{\pgfqpoint{1.356504in}{0.689796in}}%
\pgfpathcurveto{\pgfqpoint{1.356504in}{0.678745in}}{\pgfqpoint{1.360894in}{0.668146in}}{\pgfqpoint{1.368708in}{0.660333in}}%
\pgfpathcurveto{\pgfqpoint{1.376521in}{0.652519in}}{\pgfqpoint{1.387120in}{0.648129in}}{\pgfqpoint{1.398170in}{0.648129in}}%
\pgfpathclose%
\pgfusepath{stroke,fill}%
\end{pgfscope}%
\begin{pgfscope}%
\pgfpathrectangle{\pgfqpoint{0.648703in}{0.548769in}}{\pgfqpoint{5.201297in}{3.102590in}}%
\pgfusepath{clip}%
\pgfsetbuttcap%
\pgfsetroundjoin%
\definecolor{currentfill}{rgb}{0.121569,0.466667,0.705882}%
\pgfsetfillcolor{currentfill}%
\pgfsetlinewidth{1.003750pt}%
\definecolor{currentstroke}{rgb}{0.121569,0.466667,0.705882}%
\pgfsetstrokecolor{currentstroke}%
\pgfsetdash{}{0pt}%
\pgfpathmoveto{\pgfqpoint{2.897601in}{3.181114in}}%
\pgfpathcurveto{\pgfqpoint{2.908651in}{3.181114in}}{\pgfqpoint{2.919250in}{3.185504in}}{\pgfqpoint{2.927064in}{3.193318in}}%
\pgfpathcurveto{\pgfqpoint{2.934877in}{3.201132in}}{\pgfqpoint{2.939268in}{3.211731in}}{\pgfqpoint{2.939268in}{3.222781in}}%
\pgfpathcurveto{\pgfqpoint{2.939268in}{3.233831in}}{\pgfqpoint{2.934877in}{3.244430in}}{\pgfqpoint{2.927064in}{3.252244in}}%
\pgfpathcurveto{\pgfqpoint{2.919250in}{3.260057in}}{\pgfqpoint{2.908651in}{3.264448in}}{\pgfqpoint{2.897601in}{3.264448in}}%
\pgfpathcurveto{\pgfqpoint{2.886551in}{3.264448in}}{\pgfqpoint{2.875952in}{3.260057in}}{\pgfqpoint{2.868138in}{3.252244in}}%
\pgfpathcurveto{\pgfqpoint{2.860325in}{3.244430in}}{\pgfqpoint{2.855934in}{3.233831in}}{\pgfqpoint{2.855934in}{3.222781in}}%
\pgfpathcurveto{\pgfqpoint{2.855934in}{3.211731in}}{\pgfqpoint{2.860325in}{3.201132in}}{\pgfqpoint{2.868138in}{3.193318in}}%
\pgfpathcurveto{\pgfqpoint{2.875952in}{3.185504in}}{\pgfqpoint{2.886551in}{3.181114in}}{\pgfqpoint{2.897601in}{3.181114in}}%
\pgfpathclose%
\pgfusepath{stroke,fill}%
\end{pgfscope}%
\begin{pgfscope}%
\pgfpathrectangle{\pgfqpoint{0.648703in}{0.548769in}}{\pgfqpoint{5.201297in}{3.102590in}}%
\pgfusepath{clip}%
\pgfsetbuttcap%
\pgfsetroundjoin%
\definecolor{currentfill}{rgb}{1.000000,0.498039,0.054902}%
\pgfsetfillcolor{currentfill}%
\pgfsetlinewidth{1.003750pt}%
\definecolor{currentstroke}{rgb}{1.000000,0.498039,0.054902}%
\pgfsetstrokecolor{currentstroke}%
\pgfsetdash{}{0pt}%
\pgfpathmoveto{\pgfqpoint{2.087831in}{3.193800in}}%
\pgfpathcurveto{\pgfqpoint{2.098881in}{3.193800in}}{\pgfqpoint{2.109480in}{3.198191in}}{\pgfqpoint{2.117293in}{3.206004in}}%
\pgfpathcurveto{\pgfqpoint{2.125107in}{3.213818in}}{\pgfqpoint{2.129497in}{3.224417in}}{\pgfqpoint{2.129497in}{3.235467in}}%
\pgfpathcurveto{\pgfqpoint{2.129497in}{3.246517in}}{\pgfqpoint{2.125107in}{3.257116in}}{\pgfqpoint{2.117293in}{3.264930in}}%
\pgfpathcurveto{\pgfqpoint{2.109480in}{3.272743in}}{\pgfqpoint{2.098881in}{3.277134in}}{\pgfqpoint{2.087831in}{3.277134in}}%
\pgfpathcurveto{\pgfqpoint{2.076781in}{3.277134in}}{\pgfqpoint{2.066181in}{3.272743in}}{\pgfqpoint{2.058368in}{3.264930in}}%
\pgfpathcurveto{\pgfqpoint{2.050554in}{3.257116in}}{\pgfqpoint{2.046164in}{3.246517in}}{\pgfqpoint{2.046164in}{3.235467in}}%
\pgfpathcurveto{\pgfqpoint{2.046164in}{3.224417in}}{\pgfqpoint{2.050554in}{3.213818in}}{\pgfqpoint{2.058368in}{3.206004in}}%
\pgfpathcurveto{\pgfqpoint{2.066181in}{3.198191in}}{\pgfqpoint{2.076781in}{3.193800in}}{\pgfqpoint{2.087831in}{3.193800in}}%
\pgfpathclose%
\pgfusepath{stroke,fill}%
\end{pgfscope}%
\begin{pgfscope}%
\pgfpathrectangle{\pgfqpoint{0.648703in}{0.548769in}}{\pgfqpoint{5.201297in}{3.102590in}}%
\pgfusepath{clip}%
\pgfsetbuttcap%
\pgfsetroundjoin%
\definecolor{currentfill}{rgb}{1.000000,0.498039,0.054902}%
\pgfsetfillcolor{currentfill}%
\pgfsetlinewidth{1.003750pt}%
\definecolor{currentstroke}{rgb}{1.000000,0.498039,0.054902}%
\pgfsetstrokecolor{currentstroke}%
\pgfsetdash{}{0pt}%
\pgfpathmoveto{\pgfqpoint{1.678773in}{3.193800in}}%
\pgfpathcurveto{\pgfqpoint{1.689823in}{3.193800in}}{\pgfqpoint{1.700422in}{3.198191in}}{\pgfqpoint{1.708236in}{3.206004in}}%
\pgfpathcurveto{\pgfqpoint{1.716049in}{3.213818in}}{\pgfqpoint{1.720440in}{3.224417in}}{\pgfqpoint{1.720440in}{3.235467in}}%
\pgfpathcurveto{\pgfqpoint{1.720440in}{3.246517in}}{\pgfqpoint{1.716049in}{3.257116in}}{\pgfqpoint{1.708236in}{3.264930in}}%
\pgfpathcurveto{\pgfqpoint{1.700422in}{3.272743in}}{\pgfqpoint{1.689823in}{3.277134in}}{\pgfqpoint{1.678773in}{3.277134in}}%
\pgfpathcurveto{\pgfqpoint{1.667723in}{3.277134in}}{\pgfqpoint{1.657124in}{3.272743in}}{\pgfqpoint{1.649310in}{3.264930in}}%
\pgfpathcurveto{\pgfqpoint{1.641497in}{3.257116in}}{\pgfqpoint{1.637106in}{3.246517in}}{\pgfqpoint{1.637106in}{3.235467in}}%
\pgfpathcurveto{\pgfqpoint{1.637106in}{3.224417in}}{\pgfqpoint{1.641497in}{3.213818in}}{\pgfqpoint{1.649310in}{3.206004in}}%
\pgfpathcurveto{\pgfqpoint{1.657124in}{3.198191in}}{\pgfqpoint{1.667723in}{3.193800in}}{\pgfqpoint{1.678773in}{3.193800in}}%
\pgfpathclose%
\pgfusepath{stroke,fill}%
\end{pgfscope}%
\begin{pgfscope}%
\pgfpathrectangle{\pgfqpoint{0.648703in}{0.548769in}}{\pgfqpoint{5.201297in}{3.102590in}}%
\pgfusepath{clip}%
\pgfsetbuttcap%
\pgfsetroundjoin%
\definecolor{currentfill}{rgb}{0.121569,0.466667,0.705882}%
\pgfsetfillcolor{currentfill}%
\pgfsetlinewidth{1.003750pt}%
\definecolor{currentstroke}{rgb}{0.121569,0.466667,0.705882}%
\pgfsetstrokecolor{currentstroke}%
\pgfsetdash{}{0pt}%
\pgfpathmoveto{\pgfqpoint{1.449659in}{0.648129in}}%
\pgfpathcurveto{\pgfqpoint{1.460709in}{0.648129in}}{\pgfqpoint{1.471308in}{0.652519in}}{\pgfqpoint{1.479121in}{0.660333in}}%
\pgfpathcurveto{\pgfqpoint{1.486935in}{0.668146in}}{\pgfqpoint{1.491325in}{0.678745in}}{\pgfqpoint{1.491325in}{0.689796in}}%
\pgfpathcurveto{\pgfqpoint{1.491325in}{0.700846in}}{\pgfqpoint{1.486935in}{0.711445in}}{\pgfqpoint{1.479121in}{0.719258in}}%
\pgfpathcurveto{\pgfqpoint{1.471308in}{0.727072in}}{\pgfqpoint{1.460709in}{0.731462in}}{\pgfqpoint{1.449659in}{0.731462in}}%
\pgfpathcurveto{\pgfqpoint{1.438609in}{0.731462in}}{\pgfqpoint{1.428010in}{0.727072in}}{\pgfqpoint{1.420196in}{0.719258in}}%
\pgfpathcurveto{\pgfqpoint{1.412382in}{0.711445in}}{\pgfqpoint{1.407992in}{0.700846in}}{\pgfqpoint{1.407992in}{0.689796in}}%
\pgfpathcurveto{\pgfqpoint{1.407992in}{0.678745in}}{\pgfqpoint{1.412382in}{0.668146in}}{\pgfqpoint{1.420196in}{0.660333in}}%
\pgfpathcurveto{\pgfqpoint{1.428010in}{0.652519in}}{\pgfqpoint{1.438609in}{0.648129in}}{\pgfqpoint{1.449659in}{0.648129in}}%
\pgfpathclose%
\pgfusepath{stroke,fill}%
\end{pgfscope}%
\begin{pgfscope}%
\pgfpathrectangle{\pgfqpoint{0.648703in}{0.548769in}}{\pgfqpoint{5.201297in}{3.102590in}}%
\pgfusepath{clip}%
\pgfsetbuttcap%
\pgfsetroundjoin%
\definecolor{currentfill}{rgb}{0.121569,0.466667,0.705882}%
\pgfsetfillcolor{currentfill}%
\pgfsetlinewidth{1.003750pt}%
\definecolor{currentstroke}{rgb}{0.121569,0.466667,0.705882}%
\pgfsetstrokecolor{currentstroke}%
\pgfsetdash{}{0pt}%
\pgfpathmoveto{\pgfqpoint{1.265822in}{0.665044in}}%
\pgfpathcurveto{\pgfqpoint{1.276872in}{0.665044in}}{\pgfqpoint{1.287471in}{0.669434in}}{\pgfqpoint{1.295285in}{0.677248in}}%
\pgfpathcurveto{\pgfqpoint{1.303098in}{0.685061in}}{\pgfqpoint{1.307489in}{0.695660in}}{\pgfqpoint{1.307489in}{0.706710in}}%
\pgfpathcurveto{\pgfqpoint{1.307489in}{0.717760in}}{\pgfqpoint{1.303098in}{0.728360in}}{\pgfqpoint{1.295285in}{0.736173in}}%
\pgfpathcurveto{\pgfqpoint{1.287471in}{0.743987in}}{\pgfqpoint{1.276872in}{0.748377in}}{\pgfqpoint{1.265822in}{0.748377in}}%
\pgfpathcurveto{\pgfqpoint{1.254772in}{0.748377in}}{\pgfqpoint{1.244173in}{0.743987in}}{\pgfqpoint{1.236359in}{0.736173in}}%
\pgfpathcurveto{\pgfqpoint{1.228545in}{0.728360in}}{\pgfqpoint{1.224155in}{0.717760in}}{\pgfqpoint{1.224155in}{0.706710in}}%
\pgfpathcurveto{\pgfqpoint{1.224155in}{0.695660in}}{\pgfqpoint{1.228545in}{0.685061in}}{\pgfqpoint{1.236359in}{0.677248in}}%
\pgfpathcurveto{\pgfqpoint{1.244173in}{0.669434in}}{\pgfqpoint{1.254772in}{0.665044in}}{\pgfqpoint{1.265822in}{0.665044in}}%
\pgfpathclose%
\pgfusepath{stroke,fill}%
\end{pgfscope}%
\begin{pgfscope}%
\pgfpathrectangle{\pgfqpoint{0.648703in}{0.548769in}}{\pgfqpoint{5.201297in}{3.102590in}}%
\pgfusepath{clip}%
\pgfsetbuttcap%
\pgfsetroundjoin%
\definecolor{currentfill}{rgb}{1.000000,0.498039,0.054902}%
\pgfsetfillcolor{currentfill}%
\pgfsetlinewidth{1.003750pt}%
\definecolor{currentstroke}{rgb}{1.000000,0.498039,0.054902}%
\pgfsetstrokecolor{currentstroke}%
\pgfsetdash{}{0pt}%
\pgfpathmoveto{\pgfqpoint{1.363903in}{3.189572in}}%
\pgfpathcurveto{\pgfqpoint{1.374953in}{3.189572in}}{\pgfqpoint{1.385552in}{3.193962in}}{\pgfqpoint{1.393366in}{3.201775in}}%
\pgfpathcurveto{\pgfqpoint{1.401179in}{3.209589in}}{\pgfqpoint{1.405570in}{3.220188in}}{\pgfqpoint{1.405570in}{3.231238in}}%
\pgfpathcurveto{\pgfqpoint{1.405570in}{3.242288in}}{\pgfqpoint{1.401179in}{3.252887in}}{\pgfqpoint{1.393366in}{3.260701in}}%
\pgfpathcurveto{\pgfqpoint{1.385552in}{3.268515in}}{\pgfqpoint{1.374953in}{3.272905in}}{\pgfqpoint{1.363903in}{3.272905in}}%
\pgfpathcurveto{\pgfqpoint{1.352853in}{3.272905in}}{\pgfqpoint{1.342254in}{3.268515in}}{\pgfqpoint{1.334440in}{3.260701in}}%
\pgfpathcurveto{\pgfqpoint{1.326627in}{3.252887in}}{\pgfqpoint{1.322236in}{3.242288in}}{\pgfqpoint{1.322236in}{3.231238in}}%
\pgfpathcurveto{\pgfqpoint{1.322236in}{3.220188in}}{\pgfqpoint{1.326627in}{3.209589in}}{\pgfqpoint{1.334440in}{3.201775in}}%
\pgfpathcurveto{\pgfqpoint{1.342254in}{3.193962in}}{\pgfqpoint{1.352853in}{3.189572in}}{\pgfqpoint{1.363903in}{3.189572in}}%
\pgfpathclose%
\pgfusepath{stroke,fill}%
\end{pgfscope}%
\begin{pgfscope}%
\pgfpathrectangle{\pgfqpoint{0.648703in}{0.548769in}}{\pgfqpoint{5.201297in}{3.102590in}}%
\pgfusepath{clip}%
\pgfsetbuttcap%
\pgfsetroundjoin%
\definecolor{currentfill}{rgb}{0.121569,0.466667,0.705882}%
\pgfsetfillcolor{currentfill}%
\pgfsetlinewidth{1.003750pt}%
\definecolor{currentstroke}{rgb}{0.121569,0.466667,0.705882}%
\pgfsetstrokecolor{currentstroke}%
\pgfsetdash{}{0pt}%
\pgfpathmoveto{\pgfqpoint{0.888183in}{0.808819in}}%
\pgfpathcurveto{\pgfqpoint{0.899234in}{0.808819in}}{\pgfqpoint{0.909833in}{0.813209in}}{\pgfqpoint{0.917646in}{0.821023in}}%
\pgfpathcurveto{\pgfqpoint{0.925460in}{0.828837in}}{\pgfqpoint{0.929850in}{0.839436in}}{\pgfqpoint{0.929850in}{0.850486in}}%
\pgfpathcurveto{\pgfqpoint{0.929850in}{0.861536in}}{\pgfqpoint{0.925460in}{0.872135in}}{\pgfqpoint{0.917646in}{0.879949in}}%
\pgfpathcurveto{\pgfqpoint{0.909833in}{0.887762in}}{\pgfqpoint{0.899234in}{0.892152in}}{\pgfqpoint{0.888183in}{0.892152in}}%
\pgfpathcurveto{\pgfqpoint{0.877133in}{0.892152in}}{\pgfqpoint{0.866534in}{0.887762in}}{\pgfqpoint{0.858721in}{0.879949in}}%
\pgfpathcurveto{\pgfqpoint{0.850907in}{0.872135in}}{\pgfqpoint{0.846517in}{0.861536in}}{\pgfqpoint{0.846517in}{0.850486in}}%
\pgfpathcurveto{\pgfqpoint{0.846517in}{0.839436in}}{\pgfqpoint{0.850907in}{0.828837in}}{\pgfqpoint{0.858721in}{0.821023in}}%
\pgfpathcurveto{\pgfqpoint{0.866534in}{0.813209in}}{\pgfqpoint{0.877133in}{0.808819in}}{\pgfqpoint{0.888183in}{0.808819in}}%
\pgfpathclose%
\pgfusepath{stroke,fill}%
\end{pgfscope}%
\begin{pgfscope}%
\pgfpathrectangle{\pgfqpoint{0.648703in}{0.548769in}}{\pgfqpoint{5.201297in}{3.102590in}}%
\pgfusepath{clip}%
\pgfsetbuttcap%
\pgfsetroundjoin%
\definecolor{currentfill}{rgb}{0.839216,0.152941,0.156863}%
\pgfsetfillcolor{currentfill}%
\pgfsetlinewidth{1.003750pt}%
\definecolor{currentstroke}{rgb}{0.839216,0.152941,0.156863}%
\pgfsetstrokecolor{currentstroke}%
\pgfsetdash{}{0pt}%
\pgfpathmoveto{\pgfqpoint{1.935796in}{3.198029in}}%
\pgfpathcurveto{\pgfqpoint{1.946846in}{3.198029in}}{\pgfqpoint{1.957445in}{3.202419in}}{\pgfqpoint{1.965259in}{3.210233in}}%
\pgfpathcurveto{\pgfqpoint{1.973073in}{3.218046in}}{\pgfqpoint{1.977463in}{3.228646in}}{\pgfqpoint{1.977463in}{3.239696in}}%
\pgfpathcurveto{\pgfqpoint{1.977463in}{3.250746in}}{\pgfqpoint{1.973073in}{3.261345in}}{\pgfqpoint{1.965259in}{3.269158in}}%
\pgfpathcurveto{\pgfqpoint{1.957445in}{3.276972in}}{\pgfqpoint{1.946846in}{3.281362in}}{\pgfqpoint{1.935796in}{3.281362in}}%
\pgfpathcurveto{\pgfqpoint{1.924746in}{3.281362in}}{\pgfqpoint{1.914147in}{3.276972in}}{\pgfqpoint{1.906333in}{3.269158in}}%
\pgfpathcurveto{\pgfqpoint{1.898520in}{3.261345in}}{\pgfqpoint{1.894130in}{3.250746in}}{\pgfqpoint{1.894130in}{3.239696in}}%
\pgfpathcurveto{\pgfqpoint{1.894130in}{3.228646in}}{\pgfqpoint{1.898520in}{3.218046in}}{\pgfqpoint{1.906333in}{3.210233in}}%
\pgfpathcurveto{\pgfqpoint{1.914147in}{3.202419in}}{\pgfqpoint{1.924746in}{3.198029in}}{\pgfqpoint{1.935796in}{3.198029in}}%
\pgfpathclose%
\pgfusepath{stroke,fill}%
\end{pgfscope}%
\begin{pgfscope}%
\pgfpathrectangle{\pgfqpoint{0.648703in}{0.548769in}}{\pgfqpoint{5.201297in}{3.102590in}}%
\pgfusepath{clip}%
\pgfsetbuttcap%
\pgfsetroundjoin%
\definecolor{currentfill}{rgb}{0.121569,0.466667,0.705882}%
\pgfsetfillcolor{currentfill}%
\pgfsetlinewidth{1.003750pt}%
\definecolor{currentstroke}{rgb}{0.121569,0.466667,0.705882}%
\pgfsetstrokecolor{currentstroke}%
\pgfsetdash{}{0pt}%
\pgfpathmoveto{\pgfqpoint{1.093675in}{0.648129in}}%
\pgfpathcurveto{\pgfqpoint{1.104725in}{0.648129in}}{\pgfqpoint{1.115324in}{0.652519in}}{\pgfqpoint{1.123137in}{0.660333in}}%
\pgfpathcurveto{\pgfqpoint{1.130951in}{0.668146in}}{\pgfqpoint{1.135341in}{0.678745in}}{\pgfqpoint{1.135341in}{0.689796in}}%
\pgfpathcurveto{\pgfqpoint{1.135341in}{0.700846in}}{\pgfqpoint{1.130951in}{0.711445in}}{\pgfqpoint{1.123137in}{0.719258in}}%
\pgfpathcurveto{\pgfqpoint{1.115324in}{0.727072in}}{\pgfqpoint{1.104725in}{0.731462in}}{\pgfqpoint{1.093675in}{0.731462in}}%
\pgfpathcurveto{\pgfqpoint{1.082625in}{0.731462in}}{\pgfqpoint{1.072026in}{0.727072in}}{\pgfqpoint{1.064212in}{0.719258in}}%
\pgfpathcurveto{\pgfqpoint{1.056398in}{0.711445in}}{\pgfqpoint{1.052008in}{0.700846in}}{\pgfqpoint{1.052008in}{0.689796in}}%
\pgfpathcurveto{\pgfqpoint{1.052008in}{0.678745in}}{\pgfqpoint{1.056398in}{0.668146in}}{\pgfqpoint{1.064212in}{0.660333in}}%
\pgfpathcurveto{\pgfqpoint{1.072026in}{0.652519in}}{\pgfqpoint{1.082625in}{0.648129in}}{\pgfqpoint{1.093675in}{0.648129in}}%
\pgfpathclose%
\pgfusepath{stroke,fill}%
\end{pgfscope}%
\begin{pgfscope}%
\pgfpathrectangle{\pgfqpoint{0.648703in}{0.548769in}}{\pgfqpoint{5.201297in}{3.102590in}}%
\pgfusepath{clip}%
\pgfsetbuttcap%
\pgfsetroundjoin%
\definecolor{currentfill}{rgb}{1.000000,0.498039,0.054902}%
\pgfsetfillcolor{currentfill}%
\pgfsetlinewidth{1.003750pt}%
\definecolor{currentstroke}{rgb}{1.000000,0.498039,0.054902}%
\pgfsetstrokecolor{currentstroke}%
\pgfsetdash{}{0pt}%
\pgfpathmoveto{\pgfqpoint{1.875850in}{3.185343in}}%
\pgfpathcurveto{\pgfqpoint{1.886900in}{3.185343in}}{\pgfqpoint{1.897499in}{3.189733in}}{\pgfqpoint{1.905313in}{3.197547in}}%
\pgfpathcurveto{\pgfqpoint{1.913126in}{3.205360in}}{\pgfqpoint{1.917517in}{3.215959in}}{\pgfqpoint{1.917517in}{3.227010in}}%
\pgfpathcurveto{\pgfqpoint{1.917517in}{3.238060in}}{\pgfqpoint{1.913126in}{3.248659in}}{\pgfqpoint{1.905313in}{3.256472in}}%
\pgfpathcurveto{\pgfqpoint{1.897499in}{3.264286in}}{\pgfqpoint{1.886900in}{3.268676in}}{\pgfqpoint{1.875850in}{3.268676in}}%
\pgfpathcurveto{\pgfqpoint{1.864800in}{3.268676in}}{\pgfqpoint{1.854201in}{3.264286in}}{\pgfqpoint{1.846387in}{3.256472in}}%
\pgfpathcurveto{\pgfqpoint{1.838574in}{3.248659in}}{\pgfqpoint{1.834183in}{3.238060in}}{\pgfqpoint{1.834183in}{3.227010in}}%
\pgfpathcurveto{\pgfqpoint{1.834183in}{3.215959in}}{\pgfqpoint{1.838574in}{3.205360in}}{\pgfqpoint{1.846387in}{3.197547in}}%
\pgfpathcurveto{\pgfqpoint{1.854201in}{3.189733in}}{\pgfqpoint{1.864800in}{3.185343in}}{\pgfqpoint{1.875850in}{3.185343in}}%
\pgfpathclose%
\pgfusepath{stroke,fill}%
\end{pgfscope}%
\begin{pgfscope}%
\pgfpathrectangle{\pgfqpoint{0.648703in}{0.548769in}}{\pgfqpoint{5.201297in}{3.102590in}}%
\pgfusepath{clip}%
\pgfsetbuttcap%
\pgfsetroundjoin%
\definecolor{currentfill}{rgb}{0.121569,0.466667,0.705882}%
\pgfsetfillcolor{currentfill}%
\pgfsetlinewidth{1.003750pt}%
\definecolor{currentstroke}{rgb}{0.121569,0.466667,0.705882}%
\pgfsetstrokecolor{currentstroke}%
\pgfsetdash{}{0pt}%
\pgfpathmoveto{\pgfqpoint{2.663034in}{0.648129in}}%
\pgfpathcurveto{\pgfqpoint{2.674084in}{0.648129in}}{\pgfqpoint{2.684683in}{0.652519in}}{\pgfqpoint{2.692497in}{0.660333in}}%
\pgfpathcurveto{\pgfqpoint{2.700310in}{0.668146in}}{\pgfqpoint{2.704700in}{0.678745in}}{\pgfqpoint{2.704700in}{0.689796in}}%
\pgfpathcurveto{\pgfqpoint{2.704700in}{0.700846in}}{\pgfqpoint{2.700310in}{0.711445in}}{\pgfqpoint{2.692497in}{0.719258in}}%
\pgfpathcurveto{\pgfqpoint{2.684683in}{0.727072in}}{\pgfqpoint{2.674084in}{0.731462in}}{\pgfqpoint{2.663034in}{0.731462in}}%
\pgfpathcurveto{\pgfqpoint{2.651984in}{0.731462in}}{\pgfqpoint{2.641385in}{0.727072in}}{\pgfqpoint{2.633571in}{0.719258in}}%
\pgfpathcurveto{\pgfqpoint{2.625757in}{0.711445in}}{\pgfqpoint{2.621367in}{0.700846in}}{\pgfqpoint{2.621367in}{0.689796in}}%
\pgfpathcurveto{\pgfqpoint{2.621367in}{0.678745in}}{\pgfqpoint{2.625757in}{0.668146in}}{\pgfqpoint{2.633571in}{0.660333in}}%
\pgfpathcurveto{\pgfqpoint{2.641385in}{0.652519in}}{\pgfqpoint{2.651984in}{0.648129in}}{\pgfqpoint{2.663034in}{0.648129in}}%
\pgfpathclose%
\pgfusepath{stroke,fill}%
\end{pgfscope}%
\begin{pgfscope}%
\pgfpathrectangle{\pgfqpoint{0.648703in}{0.548769in}}{\pgfqpoint{5.201297in}{3.102590in}}%
\pgfusepath{clip}%
\pgfsetbuttcap%
\pgfsetroundjoin%
\definecolor{currentfill}{rgb}{1.000000,0.498039,0.054902}%
\pgfsetfillcolor{currentfill}%
\pgfsetlinewidth{1.003750pt}%
\definecolor{currentstroke}{rgb}{1.000000,0.498039,0.054902}%
\pgfsetstrokecolor{currentstroke}%
\pgfsetdash{}{0pt}%
\pgfpathmoveto{\pgfqpoint{1.193123in}{3.185343in}}%
\pgfpathcurveto{\pgfqpoint{1.204174in}{3.185343in}}{\pgfqpoint{1.214773in}{3.189733in}}{\pgfqpoint{1.222586in}{3.197547in}}%
\pgfpathcurveto{\pgfqpoint{1.230400in}{3.205360in}}{\pgfqpoint{1.234790in}{3.215959in}}{\pgfqpoint{1.234790in}{3.227010in}}%
\pgfpathcurveto{\pgfqpoint{1.234790in}{3.238060in}}{\pgfqpoint{1.230400in}{3.248659in}}{\pgfqpoint{1.222586in}{3.256472in}}%
\pgfpathcurveto{\pgfqpoint{1.214773in}{3.264286in}}{\pgfqpoint{1.204174in}{3.268676in}}{\pgfqpoint{1.193123in}{3.268676in}}%
\pgfpathcurveto{\pgfqpoint{1.182073in}{3.268676in}}{\pgfqpoint{1.171474in}{3.264286in}}{\pgfqpoint{1.163661in}{3.256472in}}%
\pgfpathcurveto{\pgfqpoint{1.155847in}{3.248659in}}{\pgfqpoint{1.151457in}{3.238060in}}{\pgfqpoint{1.151457in}{3.227010in}}%
\pgfpathcurveto{\pgfqpoint{1.151457in}{3.215959in}}{\pgfqpoint{1.155847in}{3.205360in}}{\pgfqpoint{1.163661in}{3.197547in}}%
\pgfpathcurveto{\pgfqpoint{1.171474in}{3.189733in}}{\pgfqpoint{1.182073in}{3.185343in}}{\pgfqpoint{1.193123in}{3.185343in}}%
\pgfpathclose%
\pgfusepath{stroke,fill}%
\end{pgfscope}%
\begin{pgfscope}%
\pgfpathrectangle{\pgfqpoint{0.648703in}{0.548769in}}{\pgfqpoint{5.201297in}{3.102590in}}%
\pgfusepath{clip}%
\pgfsetbuttcap%
\pgfsetroundjoin%
\definecolor{currentfill}{rgb}{0.121569,0.466667,0.705882}%
\pgfsetfillcolor{currentfill}%
\pgfsetlinewidth{1.003750pt}%
\definecolor{currentstroke}{rgb}{0.121569,0.466667,0.705882}%
\pgfsetstrokecolor{currentstroke}%
\pgfsetdash{}{0pt}%
\pgfpathmoveto{\pgfqpoint{1.818752in}{0.648129in}}%
\pgfpathcurveto{\pgfqpoint{1.829802in}{0.648129in}}{\pgfqpoint{1.840401in}{0.652519in}}{\pgfqpoint{1.848215in}{0.660333in}}%
\pgfpathcurveto{\pgfqpoint{1.856029in}{0.668146in}}{\pgfqpoint{1.860419in}{0.678745in}}{\pgfqpoint{1.860419in}{0.689796in}}%
\pgfpathcurveto{\pgfqpoint{1.860419in}{0.700846in}}{\pgfqpoint{1.856029in}{0.711445in}}{\pgfqpoint{1.848215in}{0.719258in}}%
\pgfpathcurveto{\pgfqpoint{1.840401in}{0.727072in}}{\pgfqpoint{1.829802in}{0.731462in}}{\pgfqpoint{1.818752in}{0.731462in}}%
\pgfpathcurveto{\pgfqpoint{1.807702in}{0.731462in}}{\pgfqpoint{1.797103in}{0.727072in}}{\pgfqpoint{1.789289in}{0.719258in}}%
\pgfpathcurveto{\pgfqpoint{1.781476in}{0.711445in}}{\pgfqpoint{1.777085in}{0.700846in}}{\pgfqpoint{1.777085in}{0.689796in}}%
\pgfpathcurveto{\pgfqpoint{1.777085in}{0.678745in}}{\pgfqpoint{1.781476in}{0.668146in}}{\pgfqpoint{1.789289in}{0.660333in}}%
\pgfpathcurveto{\pgfqpoint{1.797103in}{0.652519in}}{\pgfqpoint{1.807702in}{0.648129in}}{\pgfqpoint{1.818752in}{0.648129in}}%
\pgfpathclose%
\pgfusepath{stroke,fill}%
\end{pgfscope}%
\begin{pgfscope}%
\pgfpathrectangle{\pgfqpoint{0.648703in}{0.548769in}}{\pgfqpoint{5.201297in}{3.102590in}}%
\pgfusepath{clip}%
\pgfsetbuttcap%
\pgfsetroundjoin%
\definecolor{currentfill}{rgb}{1.000000,0.498039,0.054902}%
\pgfsetfillcolor{currentfill}%
\pgfsetlinewidth{1.003750pt}%
\definecolor{currentstroke}{rgb}{1.000000,0.498039,0.054902}%
\pgfsetstrokecolor{currentstroke}%
\pgfsetdash{}{0pt}%
\pgfpathmoveto{\pgfqpoint{1.026664in}{3.278374in}}%
\pgfpathcurveto{\pgfqpoint{1.037714in}{3.278374in}}{\pgfqpoint{1.048313in}{3.282764in}}{\pgfqpoint{1.056127in}{3.290578in}}%
\pgfpathcurveto{\pgfqpoint{1.063941in}{3.298392in}}{\pgfqpoint{1.068331in}{3.308991in}}{\pgfqpoint{1.068331in}{3.320041in}}%
\pgfpathcurveto{\pgfqpoint{1.068331in}{3.331091in}}{\pgfqpoint{1.063941in}{3.341690in}}{\pgfqpoint{1.056127in}{3.349504in}}%
\pgfpathcurveto{\pgfqpoint{1.048313in}{3.357317in}}{\pgfqpoint{1.037714in}{3.361707in}}{\pgfqpoint{1.026664in}{3.361707in}}%
\pgfpathcurveto{\pgfqpoint{1.015614in}{3.361707in}}{\pgfqpoint{1.005015in}{3.357317in}}{\pgfqpoint{0.997201in}{3.349504in}}%
\pgfpathcurveto{\pgfqpoint{0.989388in}{3.341690in}}{\pgfqpoint{0.984998in}{3.331091in}}{\pgfqpoint{0.984998in}{3.320041in}}%
\pgfpathcurveto{\pgfqpoint{0.984998in}{3.308991in}}{\pgfqpoint{0.989388in}{3.298392in}}{\pgfqpoint{0.997201in}{3.290578in}}%
\pgfpathcurveto{\pgfqpoint{1.005015in}{3.282764in}}{\pgfqpoint{1.015614in}{3.278374in}}{\pgfqpoint{1.026664in}{3.278374in}}%
\pgfpathclose%
\pgfusepath{stroke,fill}%
\end{pgfscope}%
\begin{pgfscope}%
\pgfpathrectangle{\pgfqpoint{0.648703in}{0.548769in}}{\pgfqpoint{5.201297in}{3.102590in}}%
\pgfusepath{clip}%
\pgfsetbuttcap%
\pgfsetroundjoin%
\definecolor{currentfill}{rgb}{0.121569,0.466667,0.705882}%
\pgfsetfillcolor{currentfill}%
\pgfsetlinewidth{1.003750pt}%
\definecolor{currentstroke}{rgb}{0.121569,0.466667,0.705882}%
\pgfsetstrokecolor{currentstroke}%
\pgfsetdash{}{0pt}%
\pgfpathmoveto{\pgfqpoint{1.447655in}{0.648129in}}%
\pgfpathcurveto{\pgfqpoint{1.458705in}{0.648129in}}{\pgfqpoint{1.469304in}{0.652519in}}{\pgfqpoint{1.477118in}{0.660333in}}%
\pgfpathcurveto{\pgfqpoint{1.484932in}{0.668146in}}{\pgfqpoint{1.489322in}{0.678745in}}{\pgfqpoint{1.489322in}{0.689796in}}%
\pgfpathcurveto{\pgfqpoint{1.489322in}{0.700846in}}{\pgfqpoint{1.484932in}{0.711445in}}{\pgfqpoint{1.477118in}{0.719258in}}%
\pgfpathcurveto{\pgfqpoint{1.469304in}{0.727072in}}{\pgfqpoint{1.458705in}{0.731462in}}{\pgfqpoint{1.447655in}{0.731462in}}%
\pgfpathcurveto{\pgfqpoint{1.436605in}{0.731462in}}{\pgfqpoint{1.426006in}{0.727072in}}{\pgfqpoint{1.418192in}{0.719258in}}%
\pgfpathcurveto{\pgfqpoint{1.410379in}{0.711445in}}{\pgfqpoint{1.405989in}{0.700846in}}{\pgfqpoint{1.405989in}{0.689796in}}%
\pgfpathcurveto{\pgfqpoint{1.405989in}{0.678745in}}{\pgfqpoint{1.410379in}{0.668146in}}{\pgfqpoint{1.418192in}{0.660333in}}%
\pgfpathcurveto{\pgfqpoint{1.426006in}{0.652519in}}{\pgfqpoint{1.436605in}{0.648129in}}{\pgfqpoint{1.447655in}{0.648129in}}%
\pgfpathclose%
\pgfusepath{stroke,fill}%
\end{pgfscope}%
\begin{pgfscope}%
\pgfpathrectangle{\pgfqpoint{0.648703in}{0.548769in}}{\pgfqpoint{5.201297in}{3.102590in}}%
\pgfusepath{clip}%
\pgfsetbuttcap%
\pgfsetroundjoin%
\definecolor{currentfill}{rgb}{1.000000,0.498039,0.054902}%
\pgfsetfillcolor{currentfill}%
\pgfsetlinewidth{1.003750pt}%
\definecolor{currentstroke}{rgb}{1.000000,0.498039,0.054902}%
\pgfsetstrokecolor{currentstroke}%
\pgfsetdash{}{0pt}%
\pgfpathmoveto{\pgfqpoint{1.347013in}{3.202258in}}%
\pgfpathcurveto{\pgfqpoint{1.358063in}{3.202258in}}{\pgfqpoint{1.368662in}{3.206648in}}{\pgfqpoint{1.376476in}{3.214462in}}%
\pgfpathcurveto{\pgfqpoint{1.384290in}{3.222275in}}{\pgfqpoint{1.388680in}{3.232874in}}{\pgfqpoint{1.388680in}{3.243924in}}%
\pgfpathcurveto{\pgfqpoint{1.388680in}{3.254974in}}{\pgfqpoint{1.384290in}{3.265573in}}{\pgfqpoint{1.376476in}{3.273387in}}%
\pgfpathcurveto{\pgfqpoint{1.368662in}{3.281201in}}{\pgfqpoint{1.358063in}{3.285591in}}{\pgfqpoint{1.347013in}{3.285591in}}%
\pgfpathcurveto{\pgfqpoint{1.335963in}{3.285591in}}{\pgfqpoint{1.325364in}{3.281201in}}{\pgfqpoint{1.317550in}{3.273387in}}%
\pgfpathcurveto{\pgfqpoint{1.309737in}{3.265573in}}{\pgfqpoint{1.305347in}{3.254974in}}{\pgfqpoint{1.305347in}{3.243924in}}%
\pgfpathcurveto{\pgfqpoint{1.305347in}{3.232874in}}{\pgfqpoint{1.309737in}{3.222275in}}{\pgfqpoint{1.317550in}{3.214462in}}%
\pgfpathcurveto{\pgfqpoint{1.325364in}{3.206648in}}{\pgfqpoint{1.335963in}{3.202258in}}{\pgfqpoint{1.347013in}{3.202258in}}%
\pgfpathclose%
\pgfusepath{stroke,fill}%
\end{pgfscope}%
\begin{pgfscope}%
\pgfpathrectangle{\pgfqpoint{0.648703in}{0.548769in}}{\pgfqpoint{5.201297in}{3.102590in}}%
\pgfusepath{clip}%
\pgfsetbuttcap%
\pgfsetroundjoin%
\definecolor{currentfill}{rgb}{1.000000,0.498039,0.054902}%
\pgfsetfillcolor{currentfill}%
\pgfsetlinewidth{1.003750pt}%
\definecolor{currentstroke}{rgb}{1.000000,0.498039,0.054902}%
\pgfsetstrokecolor{currentstroke}%
\pgfsetdash{}{0pt}%
\pgfpathmoveto{\pgfqpoint{2.372492in}{3.185343in}}%
\pgfpathcurveto{\pgfqpoint{2.383543in}{3.185343in}}{\pgfqpoint{2.394142in}{3.189733in}}{\pgfqpoint{2.401955in}{3.197547in}}%
\pgfpathcurveto{\pgfqpoint{2.409769in}{3.205360in}}{\pgfqpoint{2.414159in}{3.215959in}}{\pgfqpoint{2.414159in}{3.227010in}}%
\pgfpathcurveto{\pgfqpoint{2.414159in}{3.238060in}}{\pgfqpoint{2.409769in}{3.248659in}}{\pgfqpoint{2.401955in}{3.256472in}}%
\pgfpathcurveto{\pgfqpoint{2.394142in}{3.264286in}}{\pgfqpoint{2.383543in}{3.268676in}}{\pgfqpoint{2.372492in}{3.268676in}}%
\pgfpathcurveto{\pgfqpoint{2.361442in}{3.268676in}}{\pgfqpoint{2.350843in}{3.264286in}}{\pgfqpoint{2.343030in}{3.256472in}}%
\pgfpathcurveto{\pgfqpoint{2.335216in}{3.248659in}}{\pgfqpoint{2.330826in}{3.238060in}}{\pgfqpoint{2.330826in}{3.227010in}}%
\pgfpathcurveto{\pgfqpoint{2.330826in}{3.215959in}}{\pgfqpoint{2.335216in}{3.205360in}}{\pgfqpoint{2.343030in}{3.197547in}}%
\pgfpathcurveto{\pgfqpoint{2.350843in}{3.189733in}}{\pgfqpoint{2.361442in}{3.185343in}}{\pgfqpoint{2.372492in}{3.185343in}}%
\pgfpathclose%
\pgfusepath{stroke,fill}%
\end{pgfscope}%
\begin{pgfscope}%
\pgfpathrectangle{\pgfqpoint{0.648703in}{0.548769in}}{\pgfqpoint{5.201297in}{3.102590in}}%
\pgfusepath{clip}%
\pgfsetbuttcap%
\pgfsetroundjoin%
\definecolor{currentfill}{rgb}{0.121569,0.466667,0.705882}%
\pgfsetfillcolor{currentfill}%
\pgfsetlinewidth{1.003750pt}%
\definecolor{currentstroke}{rgb}{0.121569,0.466667,0.705882}%
\pgfsetstrokecolor{currentstroke}%
\pgfsetdash{}{0pt}%
\pgfpathmoveto{\pgfqpoint{1.806322in}{0.648129in}}%
\pgfpathcurveto{\pgfqpoint{1.817372in}{0.648129in}}{\pgfqpoint{1.827971in}{0.652519in}}{\pgfqpoint{1.835785in}{0.660333in}}%
\pgfpathcurveto{\pgfqpoint{1.843599in}{0.668146in}}{\pgfqpoint{1.847989in}{0.678745in}}{\pgfqpoint{1.847989in}{0.689796in}}%
\pgfpathcurveto{\pgfqpoint{1.847989in}{0.700846in}}{\pgfqpoint{1.843599in}{0.711445in}}{\pgfqpoint{1.835785in}{0.719258in}}%
\pgfpathcurveto{\pgfqpoint{1.827971in}{0.727072in}}{\pgfqpoint{1.817372in}{0.731462in}}{\pgfqpoint{1.806322in}{0.731462in}}%
\pgfpathcurveto{\pgfqpoint{1.795272in}{0.731462in}}{\pgfqpoint{1.784673in}{0.727072in}}{\pgfqpoint{1.776859in}{0.719258in}}%
\pgfpathcurveto{\pgfqpoint{1.769046in}{0.711445in}}{\pgfqpoint{1.764655in}{0.700846in}}{\pgfqpoint{1.764655in}{0.689796in}}%
\pgfpathcurveto{\pgfqpoint{1.764655in}{0.678745in}}{\pgfqpoint{1.769046in}{0.668146in}}{\pgfqpoint{1.776859in}{0.660333in}}%
\pgfpathcurveto{\pgfqpoint{1.784673in}{0.652519in}}{\pgfqpoint{1.795272in}{0.648129in}}{\pgfqpoint{1.806322in}{0.648129in}}%
\pgfpathclose%
\pgfusepath{stroke,fill}%
\end{pgfscope}%
\begin{pgfscope}%
\pgfpathrectangle{\pgfqpoint{0.648703in}{0.548769in}}{\pgfqpoint{5.201297in}{3.102590in}}%
\pgfusepath{clip}%
\pgfsetbuttcap%
\pgfsetroundjoin%
\definecolor{currentfill}{rgb}{1.000000,0.498039,0.054902}%
\pgfsetfillcolor{currentfill}%
\pgfsetlinewidth{1.003750pt}%
\definecolor{currentstroke}{rgb}{1.000000,0.498039,0.054902}%
\pgfsetstrokecolor{currentstroke}%
\pgfsetdash{}{0pt}%
\pgfpathmoveto{\pgfqpoint{1.294654in}{3.185343in}}%
\pgfpathcurveto{\pgfqpoint{1.305704in}{3.185343in}}{\pgfqpoint{1.316303in}{3.189733in}}{\pgfqpoint{1.324117in}{3.197547in}}%
\pgfpathcurveto{\pgfqpoint{1.331930in}{3.205360in}}{\pgfqpoint{1.336321in}{3.215959in}}{\pgfqpoint{1.336321in}{3.227010in}}%
\pgfpathcurveto{\pgfqpoint{1.336321in}{3.238060in}}{\pgfqpoint{1.331930in}{3.248659in}}{\pgfqpoint{1.324117in}{3.256472in}}%
\pgfpathcurveto{\pgfqpoint{1.316303in}{3.264286in}}{\pgfqpoint{1.305704in}{3.268676in}}{\pgfqpoint{1.294654in}{3.268676in}}%
\pgfpathcurveto{\pgfqpoint{1.283604in}{3.268676in}}{\pgfqpoint{1.273005in}{3.264286in}}{\pgfqpoint{1.265191in}{3.256472in}}%
\pgfpathcurveto{\pgfqpoint{1.257378in}{3.248659in}}{\pgfqpoint{1.252987in}{3.238060in}}{\pgfqpoint{1.252987in}{3.227010in}}%
\pgfpathcurveto{\pgfqpoint{1.252987in}{3.215959in}}{\pgfqpoint{1.257378in}{3.205360in}}{\pgfqpoint{1.265191in}{3.197547in}}%
\pgfpathcurveto{\pgfqpoint{1.273005in}{3.189733in}}{\pgfqpoint{1.283604in}{3.185343in}}{\pgfqpoint{1.294654in}{3.185343in}}%
\pgfpathclose%
\pgfusepath{stroke,fill}%
\end{pgfscope}%
\begin{pgfscope}%
\pgfpathrectangle{\pgfqpoint{0.648703in}{0.548769in}}{\pgfqpoint{5.201297in}{3.102590in}}%
\pgfusepath{clip}%
\pgfsetbuttcap%
\pgfsetroundjoin%
\definecolor{currentfill}{rgb}{1.000000,0.498039,0.054902}%
\pgfsetfillcolor{currentfill}%
\pgfsetlinewidth{1.003750pt}%
\definecolor{currentstroke}{rgb}{1.000000,0.498039,0.054902}%
\pgfsetstrokecolor{currentstroke}%
\pgfsetdash{}{0pt}%
\pgfpathmoveto{\pgfqpoint{0.943077in}{3.193800in}}%
\pgfpathcurveto{\pgfqpoint{0.954128in}{3.193800in}}{\pgfqpoint{0.964727in}{3.198191in}}{\pgfqpoint{0.972540in}{3.206004in}}%
\pgfpathcurveto{\pgfqpoint{0.980354in}{3.213818in}}{\pgfqpoint{0.984744in}{3.224417in}}{\pgfqpoint{0.984744in}{3.235467in}}%
\pgfpathcurveto{\pgfqpoint{0.984744in}{3.246517in}}{\pgfqpoint{0.980354in}{3.257116in}}{\pgfqpoint{0.972540in}{3.264930in}}%
\pgfpathcurveto{\pgfqpoint{0.964727in}{3.272743in}}{\pgfqpoint{0.954128in}{3.277134in}}{\pgfqpoint{0.943077in}{3.277134in}}%
\pgfpathcurveto{\pgfqpoint{0.932027in}{3.277134in}}{\pgfqpoint{0.921428in}{3.272743in}}{\pgfqpoint{0.913615in}{3.264930in}}%
\pgfpathcurveto{\pgfqpoint{0.905801in}{3.257116in}}{\pgfqpoint{0.901411in}{3.246517in}}{\pgfqpoint{0.901411in}{3.235467in}}%
\pgfpathcurveto{\pgfqpoint{0.901411in}{3.224417in}}{\pgfqpoint{0.905801in}{3.213818in}}{\pgfqpoint{0.913615in}{3.206004in}}%
\pgfpathcurveto{\pgfqpoint{0.921428in}{3.198191in}}{\pgfqpoint{0.932027in}{3.193800in}}{\pgfqpoint{0.943077in}{3.193800in}}%
\pgfpathclose%
\pgfusepath{stroke,fill}%
\end{pgfscope}%
\begin{pgfscope}%
\pgfpathrectangle{\pgfqpoint{0.648703in}{0.548769in}}{\pgfqpoint{5.201297in}{3.102590in}}%
\pgfusepath{clip}%
\pgfsetbuttcap%
\pgfsetroundjoin%
\definecolor{currentfill}{rgb}{0.121569,0.466667,0.705882}%
\pgfsetfillcolor{currentfill}%
\pgfsetlinewidth{1.003750pt}%
\definecolor{currentstroke}{rgb}{0.121569,0.466667,0.705882}%
\pgfsetstrokecolor{currentstroke}%
\pgfsetdash{}{0pt}%
\pgfpathmoveto{\pgfqpoint{1.807986in}{0.656586in}}%
\pgfpathcurveto{\pgfqpoint{1.819036in}{0.656586in}}{\pgfqpoint{1.829635in}{0.660977in}}{\pgfqpoint{1.837449in}{0.668790in}}%
\pgfpathcurveto{\pgfqpoint{1.845262in}{0.676604in}}{\pgfqpoint{1.849653in}{0.687203in}}{\pgfqpoint{1.849653in}{0.698253in}}%
\pgfpathcurveto{\pgfqpoint{1.849653in}{0.709303in}}{\pgfqpoint{1.845262in}{0.719902in}}{\pgfqpoint{1.837449in}{0.727716in}}%
\pgfpathcurveto{\pgfqpoint{1.829635in}{0.735529in}}{\pgfqpoint{1.819036in}{0.739920in}}{\pgfqpoint{1.807986in}{0.739920in}}%
\pgfpathcurveto{\pgfqpoint{1.796936in}{0.739920in}}{\pgfqpoint{1.786337in}{0.735529in}}{\pgfqpoint{1.778523in}{0.727716in}}%
\pgfpathcurveto{\pgfqpoint{1.770709in}{0.719902in}}{\pgfqpoint{1.766319in}{0.709303in}}{\pgfqpoint{1.766319in}{0.698253in}}%
\pgfpathcurveto{\pgfqpoint{1.766319in}{0.687203in}}{\pgfqpoint{1.770709in}{0.676604in}}{\pgfqpoint{1.778523in}{0.668790in}}%
\pgfpathcurveto{\pgfqpoint{1.786337in}{0.660977in}}{\pgfqpoint{1.796936in}{0.656586in}}{\pgfqpoint{1.807986in}{0.656586in}}%
\pgfpathclose%
\pgfusepath{stroke,fill}%
\end{pgfscope}%
\begin{pgfscope}%
\pgfpathrectangle{\pgfqpoint{0.648703in}{0.548769in}}{\pgfqpoint{5.201297in}{3.102590in}}%
\pgfusepath{clip}%
\pgfsetbuttcap%
\pgfsetroundjoin%
\definecolor{currentfill}{rgb}{1.000000,0.498039,0.054902}%
\pgfsetfillcolor{currentfill}%
\pgfsetlinewidth{1.003750pt}%
\definecolor{currentstroke}{rgb}{1.000000,0.498039,0.054902}%
\pgfsetstrokecolor{currentstroke}%
\pgfsetdash{}{0pt}%
\pgfpathmoveto{\pgfqpoint{2.974707in}{3.214944in}}%
\pgfpathcurveto{\pgfqpoint{2.985757in}{3.214944in}}{\pgfqpoint{2.996356in}{3.219334in}}{\pgfqpoint{3.004170in}{3.227148in}}%
\pgfpathcurveto{\pgfqpoint{3.011984in}{3.234961in}}{\pgfqpoint{3.016374in}{3.245560in}}{\pgfqpoint{3.016374in}{3.256610in}}%
\pgfpathcurveto{\pgfqpoint{3.016374in}{3.267661in}}{\pgfqpoint{3.011984in}{3.278260in}}{\pgfqpoint{3.004170in}{3.286073in}}%
\pgfpathcurveto{\pgfqpoint{2.996356in}{3.293887in}}{\pgfqpoint{2.985757in}{3.298277in}}{\pgfqpoint{2.974707in}{3.298277in}}%
\pgfpathcurveto{\pgfqpoint{2.963657in}{3.298277in}}{\pgfqpoint{2.953058in}{3.293887in}}{\pgfqpoint{2.945244in}{3.286073in}}%
\pgfpathcurveto{\pgfqpoint{2.937431in}{3.278260in}}{\pgfqpoint{2.933040in}{3.267661in}}{\pgfqpoint{2.933040in}{3.256610in}}%
\pgfpathcurveto{\pgfqpoint{2.933040in}{3.245560in}}{\pgfqpoint{2.937431in}{3.234961in}}{\pgfqpoint{2.945244in}{3.227148in}}%
\pgfpathcurveto{\pgfqpoint{2.953058in}{3.219334in}}{\pgfqpoint{2.963657in}{3.214944in}}{\pgfqpoint{2.974707in}{3.214944in}}%
\pgfpathclose%
\pgfusepath{stroke,fill}%
\end{pgfscope}%
\begin{pgfscope}%
\pgfpathrectangle{\pgfqpoint{0.648703in}{0.548769in}}{\pgfqpoint{5.201297in}{3.102590in}}%
\pgfusepath{clip}%
\pgfsetbuttcap%
\pgfsetroundjoin%
\definecolor{currentfill}{rgb}{1.000000,0.498039,0.054902}%
\pgfsetfillcolor{currentfill}%
\pgfsetlinewidth{1.003750pt}%
\definecolor{currentstroke}{rgb}{1.000000,0.498039,0.054902}%
\pgfsetstrokecolor{currentstroke}%
\pgfsetdash{}{0pt}%
\pgfpathmoveto{\pgfqpoint{1.736524in}{3.202258in}}%
\pgfpathcurveto{\pgfqpoint{1.747574in}{3.202258in}}{\pgfqpoint{1.758173in}{3.206648in}}{\pgfqpoint{1.765987in}{3.214462in}}%
\pgfpathcurveto{\pgfqpoint{1.773801in}{3.222275in}}{\pgfqpoint{1.778191in}{3.232874in}}{\pgfqpoint{1.778191in}{3.243924in}}%
\pgfpathcurveto{\pgfqpoint{1.778191in}{3.254974in}}{\pgfqpoint{1.773801in}{3.265573in}}{\pgfqpoint{1.765987in}{3.273387in}}%
\pgfpathcurveto{\pgfqpoint{1.758173in}{3.281201in}}{\pgfqpoint{1.747574in}{3.285591in}}{\pgfqpoint{1.736524in}{3.285591in}}%
\pgfpathcurveto{\pgfqpoint{1.725474in}{3.285591in}}{\pgfqpoint{1.714875in}{3.281201in}}{\pgfqpoint{1.707061in}{3.273387in}}%
\pgfpathcurveto{\pgfqpoint{1.699248in}{3.265573in}}{\pgfqpoint{1.694858in}{3.254974in}}{\pgfqpoint{1.694858in}{3.243924in}}%
\pgfpathcurveto{\pgfqpoint{1.694858in}{3.232874in}}{\pgfqpoint{1.699248in}{3.222275in}}{\pgfqpoint{1.707061in}{3.214462in}}%
\pgfpathcurveto{\pgfqpoint{1.714875in}{3.206648in}}{\pgfqpoint{1.725474in}{3.202258in}}{\pgfqpoint{1.736524in}{3.202258in}}%
\pgfpathclose%
\pgfusepath{stroke,fill}%
\end{pgfscope}%
\begin{pgfscope}%
\pgfpathrectangle{\pgfqpoint{0.648703in}{0.548769in}}{\pgfqpoint{5.201297in}{3.102590in}}%
\pgfusepath{clip}%
\pgfsetbuttcap%
\pgfsetroundjoin%
\definecolor{currentfill}{rgb}{1.000000,0.498039,0.054902}%
\pgfsetfillcolor{currentfill}%
\pgfsetlinewidth{1.003750pt}%
\definecolor{currentstroke}{rgb}{1.000000,0.498039,0.054902}%
\pgfsetstrokecolor{currentstroke}%
\pgfsetdash{}{0pt}%
\pgfpathmoveto{\pgfqpoint{1.879482in}{3.244545in}}%
\pgfpathcurveto{\pgfqpoint{1.890532in}{3.244545in}}{\pgfqpoint{1.901131in}{3.248935in}}{\pgfqpoint{1.908945in}{3.256748in}}%
\pgfpathcurveto{\pgfqpoint{1.916759in}{3.264562in}}{\pgfqpoint{1.921149in}{3.275161in}}{\pgfqpoint{1.921149in}{3.286211in}}%
\pgfpathcurveto{\pgfqpoint{1.921149in}{3.297261in}}{\pgfqpoint{1.916759in}{3.307860in}}{\pgfqpoint{1.908945in}{3.315674in}}%
\pgfpathcurveto{\pgfqpoint{1.901131in}{3.323488in}}{\pgfqpoint{1.890532in}{3.327878in}}{\pgfqpoint{1.879482in}{3.327878in}}%
\pgfpathcurveto{\pgfqpoint{1.868432in}{3.327878in}}{\pgfqpoint{1.857833in}{3.323488in}}{\pgfqpoint{1.850020in}{3.315674in}}%
\pgfpathcurveto{\pgfqpoint{1.842206in}{3.307860in}}{\pgfqpoint{1.837816in}{3.297261in}}{\pgfqpoint{1.837816in}{3.286211in}}%
\pgfpathcurveto{\pgfqpoint{1.837816in}{3.275161in}}{\pgfqpoint{1.842206in}{3.264562in}}{\pgfqpoint{1.850020in}{3.256748in}}%
\pgfpathcurveto{\pgfqpoint{1.857833in}{3.248935in}}{\pgfqpoint{1.868432in}{3.244545in}}{\pgfqpoint{1.879482in}{3.244545in}}%
\pgfpathclose%
\pgfusepath{stroke,fill}%
\end{pgfscope}%
\begin{pgfscope}%
\pgfpathrectangle{\pgfqpoint{0.648703in}{0.548769in}}{\pgfqpoint{5.201297in}{3.102590in}}%
\pgfusepath{clip}%
\pgfsetbuttcap%
\pgfsetroundjoin%
\definecolor{currentfill}{rgb}{1.000000,0.498039,0.054902}%
\pgfsetfillcolor{currentfill}%
\pgfsetlinewidth{1.003750pt}%
\definecolor{currentstroke}{rgb}{1.000000,0.498039,0.054902}%
\pgfsetstrokecolor{currentstroke}%
\pgfsetdash{}{0pt}%
\pgfpathmoveto{\pgfqpoint{1.579115in}{3.312204in}}%
\pgfpathcurveto{\pgfqpoint{1.590165in}{3.312204in}}{\pgfqpoint{1.600765in}{3.316594in}}{\pgfqpoint{1.608578in}{3.324407in}}%
\pgfpathcurveto{\pgfqpoint{1.616392in}{3.332221in}}{\pgfqpoint{1.620782in}{3.342820in}}{\pgfqpoint{1.620782in}{3.353870in}}%
\pgfpathcurveto{\pgfqpoint{1.620782in}{3.364920in}}{\pgfqpoint{1.616392in}{3.375519in}}{\pgfqpoint{1.608578in}{3.383333in}}%
\pgfpathcurveto{\pgfqpoint{1.600765in}{3.391147in}}{\pgfqpoint{1.590165in}{3.395537in}}{\pgfqpoint{1.579115in}{3.395537in}}%
\pgfpathcurveto{\pgfqpoint{1.568065in}{3.395537in}}{\pgfqpoint{1.557466in}{3.391147in}}{\pgfqpoint{1.549653in}{3.383333in}}%
\pgfpathcurveto{\pgfqpoint{1.541839in}{3.375519in}}{\pgfqpoint{1.537449in}{3.364920in}}{\pgfqpoint{1.537449in}{3.353870in}}%
\pgfpathcurveto{\pgfqpoint{1.537449in}{3.342820in}}{\pgfqpoint{1.541839in}{3.332221in}}{\pgfqpoint{1.549653in}{3.324407in}}%
\pgfpathcurveto{\pgfqpoint{1.557466in}{3.316594in}}{\pgfqpoint{1.568065in}{3.312204in}}{\pgfqpoint{1.579115in}{3.312204in}}%
\pgfpathclose%
\pgfusepath{stroke,fill}%
\end{pgfscope}%
\begin{pgfscope}%
\pgfpathrectangle{\pgfqpoint{0.648703in}{0.548769in}}{\pgfqpoint{5.201297in}{3.102590in}}%
\pgfusepath{clip}%
\pgfsetbuttcap%
\pgfsetroundjoin%
\definecolor{currentfill}{rgb}{1.000000,0.498039,0.054902}%
\pgfsetfillcolor{currentfill}%
\pgfsetlinewidth{1.003750pt}%
\definecolor{currentstroke}{rgb}{1.000000,0.498039,0.054902}%
\pgfsetstrokecolor{currentstroke}%
\pgfsetdash{}{0pt}%
\pgfpathmoveto{\pgfqpoint{1.022413in}{3.210715in}}%
\pgfpathcurveto{\pgfqpoint{1.033464in}{3.210715in}}{\pgfqpoint{1.044063in}{3.215105in}}{\pgfqpoint{1.051876in}{3.222919in}}%
\pgfpathcurveto{\pgfqpoint{1.059690in}{3.230733in}}{\pgfqpoint{1.064080in}{3.241332in}}{\pgfqpoint{1.064080in}{3.252382in}}%
\pgfpathcurveto{\pgfqpoint{1.064080in}{3.263432in}}{\pgfqpoint{1.059690in}{3.274031in}}{\pgfqpoint{1.051876in}{3.281844in}}%
\pgfpathcurveto{\pgfqpoint{1.044063in}{3.289658in}}{\pgfqpoint{1.033464in}{3.294048in}}{\pgfqpoint{1.022413in}{3.294048in}}%
\pgfpathcurveto{\pgfqpoint{1.011363in}{3.294048in}}{\pgfqpoint{1.000764in}{3.289658in}}{\pgfqpoint{0.992951in}{3.281844in}}%
\pgfpathcurveto{\pgfqpoint{0.985137in}{3.274031in}}{\pgfqpoint{0.980747in}{3.263432in}}{\pgfqpoint{0.980747in}{3.252382in}}%
\pgfpathcurveto{\pgfqpoint{0.980747in}{3.241332in}}{\pgfqpoint{0.985137in}{3.230733in}}{\pgfqpoint{0.992951in}{3.222919in}}%
\pgfpathcurveto{\pgfqpoint{1.000764in}{3.215105in}}{\pgfqpoint{1.011363in}{3.210715in}}{\pgfqpoint{1.022413in}{3.210715in}}%
\pgfpathclose%
\pgfusepath{stroke,fill}%
\end{pgfscope}%
\begin{pgfscope}%
\pgfpathrectangle{\pgfqpoint{0.648703in}{0.548769in}}{\pgfqpoint{5.201297in}{3.102590in}}%
\pgfusepath{clip}%
\pgfsetbuttcap%
\pgfsetroundjoin%
\definecolor{currentfill}{rgb}{1.000000,0.498039,0.054902}%
\pgfsetfillcolor{currentfill}%
\pgfsetlinewidth{1.003750pt}%
\definecolor{currentstroke}{rgb}{1.000000,0.498039,0.054902}%
\pgfsetstrokecolor{currentstroke}%
\pgfsetdash{}{0pt}%
\pgfpathmoveto{\pgfqpoint{1.774624in}{3.236087in}}%
\pgfpathcurveto{\pgfqpoint{1.785674in}{3.236087in}}{\pgfqpoint{1.796273in}{3.240477in}}{\pgfqpoint{1.804087in}{3.248291in}}%
\pgfpathcurveto{\pgfqpoint{1.811901in}{3.256105in}}{\pgfqpoint{1.816291in}{3.266704in}}{\pgfqpoint{1.816291in}{3.277754in}}%
\pgfpathcurveto{\pgfqpoint{1.816291in}{3.288804in}}{\pgfqpoint{1.811901in}{3.299403in}}{\pgfqpoint{1.804087in}{3.307217in}}%
\pgfpathcurveto{\pgfqpoint{1.796273in}{3.315030in}}{\pgfqpoint{1.785674in}{3.319421in}}{\pgfqpoint{1.774624in}{3.319421in}}%
\pgfpathcurveto{\pgfqpoint{1.763574in}{3.319421in}}{\pgfqpoint{1.752975in}{3.315030in}}{\pgfqpoint{1.745162in}{3.307217in}}%
\pgfpathcurveto{\pgfqpoint{1.737348in}{3.299403in}}{\pgfqpoint{1.732958in}{3.288804in}}{\pgfqpoint{1.732958in}{3.277754in}}%
\pgfpathcurveto{\pgfqpoint{1.732958in}{3.266704in}}{\pgfqpoint{1.737348in}{3.256105in}}{\pgfqpoint{1.745162in}{3.248291in}}%
\pgfpathcurveto{\pgfqpoint{1.752975in}{3.240477in}}{\pgfqpoint{1.763574in}{3.236087in}}{\pgfqpoint{1.774624in}{3.236087in}}%
\pgfpathclose%
\pgfusepath{stroke,fill}%
\end{pgfscope}%
\begin{pgfscope}%
\pgfpathrectangle{\pgfqpoint{0.648703in}{0.548769in}}{\pgfqpoint{5.201297in}{3.102590in}}%
\pgfusepath{clip}%
\pgfsetbuttcap%
\pgfsetroundjoin%
\definecolor{currentfill}{rgb}{0.121569,0.466667,0.705882}%
\pgfsetfillcolor{currentfill}%
\pgfsetlinewidth{1.003750pt}%
\definecolor{currentstroke}{rgb}{0.121569,0.466667,0.705882}%
\pgfsetstrokecolor{currentstroke}%
\pgfsetdash{}{0pt}%
\pgfpathmoveto{\pgfqpoint{1.764442in}{0.648129in}}%
\pgfpathcurveto{\pgfqpoint{1.775492in}{0.648129in}}{\pgfqpoint{1.786091in}{0.652519in}}{\pgfqpoint{1.793904in}{0.660333in}}%
\pgfpathcurveto{\pgfqpoint{1.801718in}{0.668146in}}{\pgfqpoint{1.806108in}{0.678745in}}{\pgfqpoint{1.806108in}{0.689796in}}%
\pgfpathcurveto{\pgfqpoint{1.806108in}{0.700846in}}{\pgfqpoint{1.801718in}{0.711445in}}{\pgfqpoint{1.793904in}{0.719258in}}%
\pgfpathcurveto{\pgfqpoint{1.786091in}{0.727072in}}{\pgfqpoint{1.775492in}{0.731462in}}{\pgfqpoint{1.764442in}{0.731462in}}%
\pgfpathcurveto{\pgfqpoint{1.753392in}{0.731462in}}{\pgfqpoint{1.742792in}{0.727072in}}{\pgfqpoint{1.734979in}{0.719258in}}%
\pgfpathcurveto{\pgfqpoint{1.727165in}{0.711445in}}{\pgfqpoint{1.722775in}{0.700846in}}{\pgfqpoint{1.722775in}{0.689796in}}%
\pgfpathcurveto{\pgfqpoint{1.722775in}{0.678745in}}{\pgfqpoint{1.727165in}{0.668146in}}{\pgfqpoint{1.734979in}{0.660333in}}%
\pgfpathcurveto{\pgfqpoint{1.742792in}{0.652519in}}{\pgfqpoint{1.753392in}{0.648129in}}{\pgfqpoint{1.764442in}{0.648129in}}%
\pgfpathclose%
\pgfusepath{stroke,fill}%
\end{pgfscope}%
\begin{pgfscope}%
\pgfpathrectangle{\pgfqpoint{0.648703in}{0.548769in}}{\pgfqpoint{5.201297in}{3.102590in}}%
\pgfusepath{clip}%
\pgfsetbuttcap%
\pgfsetroundjoin%
\definecolor{currentfill}{rgb}{0.121569,0.466667,0.705882}%
\pgfsetfillcolor{currentfill}%
\pgfsetlinewidth{1.003750pt}%
\definecolor{currentstroke}{rgb}{0.121569,0.466667,0.705882}%
\pgfsetstrokecolor{currentstroke}%
\pgfsetdash{}{0pt}%
\pgfpathmoveto{\pgfqpoint{1.678207in}{0.648129in}}%
\pgfpathcurveto{\pgfqpoint{1.689257in}{0.648129in}}{\pgfqpoint{1.699856in}{0.652519in}}{\pgfqpoint{1.707670in}{0.660333in}}%
\pgfpathcurveto{\pgfqpoint{1.715483in}{0.668146in}}{\pgfqpoint{1.719874in}{0.678745in}}{\pgfqpoint{1.719874in}{0.689796in}}%
\pgfpathcurveto{\pgfqpoint{1.719874in}{0.700846in}}{\pgfqpoint{1.715483in}{0.711445in}}{\pgfqpoint{1.707670in}{0.719258in}}%
\pgfpathcurveto{\pgfqpoint{1.699856in}{0.727072in}}{\pgfqpoint{1.689257in}{0.731462in}}{\pgfqpoint{1.678207in}{0.731462in}}%
\pgfpathcurveto{\pgfqpoint{1.667157in}{0.731462in}}{\pgfqpoint{1.656558in}{0.727072in}}{\pgfqpoint{1.648744in}{0.719258in}}%
\pgfpathcurveto{\pgfqpoint{1.640931in}{0.711445in}}{\pgfqpoint{1.636540in}{0.700846in}}{\pgfqpoint{1.636540in}{0.689796in}}%
\pgfpathcurveto{\pgfqpoint{1.636540in}{0.678745in}}{\pgfqpoint{1.640931in}{0.668146in}}{\pgfqpoint{1.648744in}{0.660333in}}%
\pgfpathcurveto{\pgfqpoint{1.656558in}{0.652519in}}{\pgfqpoint{1.667157in}{0.648129in}}{\pgfqpoint{1.678207in}{0.648129in}}%
\pgfpathclose%
\pgfusepath{stroke,fill}%
\end{pgfscope}%
\begin{pgfscope}%
\pgfpathrectangle{\pgfqpoint{0.648703in}{0.548769in}}{\pgfqpoint{5.201297in}{3.102590in}}%
\pgfusepath{clip}%
\pgfsetbuttcap%
\pgfsetroundjoin%
\definecolor{currentfill}{rgb}{0.121569,0.466667,0.705882}%
\pgfsetfillcolor{currentfill}%
\pgfsetlinewidth{1.003750pt}%
\definecolor{currentstroke}{rgb}{0.121569,0.466667,0.705882}%
\pgfsetstrokecolor{currentstroke}%
\pgfsetdash{}{0pt}%
\pgfpathmoveto{\pgfqpoint{1.592199in}{0.648129in}}%
\pgfpathcurveto{\pgfqpoint{1.603249in}{0.648129in}}{\pgfqpoint{1.613848in}{0.652519in}}{\pgfqpoint{1.621661in}{0.660333in}}%
\pgfpathcurveto{\pgfqpoint{1.629475in}{0.668146in}}{\pgfqpoint{1.633865in}{0.678745in}}{\pgfqpoint{1.633865in}{0.689796in}}%
\pgfpathcurveto{\pgfqpoint{1.633865in}{0.700846in}}{\pgfqpoint{1.629475in}{0.711445in}}{\pgfqpoint{1.621661in}{0.719258in}}%
\pgfpathcurveto{\pgfqpoint{1.613848in}{0.727072in}}{\pgfqpoint{1.603249in}{0.731462in}}{\pgfqpoint{1.592199in}{0.731462in}}%
\pgfpathcurveto{\pgfqpoint{1.581149in}{0.731462in}}{\pgfqpoint{1.570549in}{0.727072in}}{\pgfqpoint{1.562736in}{0.719258in}}%
\pgfpathcurveto{\pgfqpoint{1.554922in}{0.711445in}}{\pgfqpoint{1.550532in}{0.700846in}}{\pgfqpoint{1.550532in}{0.689796in}}%
\pgfpathcurveto{\pgfqpoint{1.550532in}{0.678745in}}{\pgfqpoint{1.554922in}{0.668146in}}{\pgfqpoint{1.562736in}{0.660333in}}%
\pgfpathcurveto{\pgfqpoint{1.570549in}{0.652519in}}{\pgfqpoint{1.581149in}{0.648129in}}{\pgfqpoint{1.592199in}{0.648129in}}%
\pgfpathclose%
\pgfusepath{stroke,fill}%
\end{pgfscope}%
\begin{pgfscope}%
\pgfpathrectangle{\pgfqpoint{0.648703in}{0.548769in}}{\pgfqpoint{5.201297in}{3.102590in}}%
\pgfusepath{clip}%
\pgfsetbuttcap%
\pgfsetroundjoin%
\definecolor{currentfill}{rgb}{1.000000,0.498039,0.054902}%
\pgfsetfillcolor{currentfill}%
\pgfsetlinewidth{1.003750pt}%
\definecolor{currentstroke}{rgb}{1.000000,0.498039,0.054902}%
\pgfsetstrokecolor{currentstroke}%
\pgfsetdash{}{0pt}%
\pgfpathmoveto{\pgfqpoint{1.315237in}{3.405235in}}%
\pgfpathcurveto{\pgfqpoint{1.326287in}{3.405235in}}{\pgfqpoint{1.336886in}{3.409625in}}{\pgfqpoint{1.344700in}{3.417439in}}%
\pgfpathcurveto{\pgfqpoint{1.352513in}{3.425252in}}{\pgfqpoint{1.356904in}{3.435851in}}{\pgfqpoint{1.356904in}{3.446901in}}%
\pgfpathcurveto{\pgfqpoint{1.356904in}{3.457952in}}{\pgfqpoint{1.352513in}{3.468551in}}{\pgfqpoint{1.344700in}{3.476364in}}%
\pgfpathcurveto{\pgfqpoint{1.336886in}{3.484178in}}{\pgfqpoint{1.326287in}{3.488568in}}{\pgfqpoint{1.315237in}{3.488568in}}%
\pgfpathcurveto{\pgfqpoint{1.304187in}{3.488568in}}{\pgfqpoint{1.293588in}{3.484178in}}{\pgfqpoint{1.285774in}{3.476364in}}%
\pgfpathcurveto{\pgfqpoint{1.277961in}{3.468551in}}{\pgfqpoint{1.273570in}{3.457952in}}{\pgfqpoint{1.273570in}{3.446901in}}%
\pgfpathcurveto{\pgfqpoint{1.273570in}{3.435851in}}{\pgfqpoint{1.277961in}{3.425252in}}{\pgfqpoint{1.285774in}{3.417439in}}%
\pgfpathcurveto{\pgfqpoint{1.293588in}{3.409625in}}{\pgfqpoint{1.304187in}{3.405235in}}{\pgfqpoint{1.315237in}{3.405235in}}%
\pgfpathclose%
\pgfusepath{stroke,fill}%
\end{pgfscope}%
\begin{pgfscope}%
\pgfpathrectangle{\pgfqpoint{0.648703in}{0.548769in}}{\pgfqpoint{5.201297in}{3.102590in}}%
\pgfusepath{clip}%
\pgfsetbuttcap%
\pgfsetroundjoin%
\definecolor{currentfill}{rgb}{1.000000,0.498039,0.054902}%
\pgfsetfillcolor{currentfill}%
\pgfsetlinewidth{1.003750pt}%
\definecolor{currentstroke}{rgb}{1.000000,0.498039,0.054902}%
\pgfsetstrokecolor{currentstroke}%
\pgfsetdash{}{0pt}%
\pgfpathmoveto{\pgfqpoint{2.098919in}{3.193800in}}%
\pgfpathcurveto{\pgfqpoint{2.109969in}{3.193800in}}{\pgfqpoint{2.120568in}{3.198191in}}{\pgfqpoint{2.128382in}{3.206004in}}%
\pgfpathcurveto{\pgfqpoint{2.136196in}{3.213818in}}{\pgfqpoint{2.140586in}{3.224417in}}{\pgfqpoint{2.140586in}{3.235467in}}%
\pgfpathcurveto{\pgfqpoint{2.140586in}{3.246517in}}{\pgfqpoint{2.136196in}{3.257116in}}{\pgfqpoint{2.128382in}{3.264930in}}%
\pgfpathcurveto{\pgfqpoint{2.120568in}{3.272743in}}{\pgfqpoint{2.109969in}{3.277134in}}{\pgfqpoint{2.098919in}{3.277134in}}%
\pgfpathcurveto{\pgfqpoint{2.087869in}{3.277134in}}{\pgfqpoint{2.077270in}{3.272743in}}{\pgfqpoint{2.069456in}{3.264930in}}%
\pgfpathcurveto{\pgfqpoint{2.061643in}{3.257116in}}{\pgfqpoint{2.057253in}{3.246517in}}{\pgfqpoint{2.057253in}{3.235467in}}%
\pgfpathcurveto{\pgfqpoint{2.057253in}{3.224417in}}{\pgfqpoint{2.061643in}{3.213818in}}{\pgfqpoint{2.069456in}{3.206004in}}%
\pgfpathcurveto{\pgfqpoint{2.077270in}{3.198191in}}{\pgfqpoint{2.087869in}{3.193800in}}{\pgfqpoint{2.098919in}{3.193800in}}%
\pgfpathclose%
\pgfusepath{stroke,fill}%
\end{pgfscope}%
\begin{pgfscope}%
\pgfpathrectangle{\pgfqpoint{0.648703in}{0.548769in}}{\pgfqpoint{5.201297in}{3.102590in}}%
\pgfusepath{clip}%
\pgfsetbuttcap%
\pgfsetroundjoin%
\definecolor{currentfill}{rgb}{1.000000,0.498039,0.054902}%
\pgfsetfillcolor{currentfill}%
\pgfsetlinewidth{1.003750pt}%
\definecolor{currentstroke}{rgb}{1.000000,0.498039,0.054902}%
\pgfsetstrokecolor{currentstroke}%
\pgfsetdash{}{0pt}%
\pgfpathmoveto{\pgfqpoint{1.830694in}{3.358719in}}%
\pgfpathcurveto{\pgfqpoint{1.841744in}{3.358719in}}{\pgfqpoint{1.852343in}{3.363109in}}{\pgfqpoint{1.860157in}{3.370923in}}%
\pgfpathcurveto{\pgfqpoint{1.867971in}{3.378737in}}{\pgfqpoint{1.872361in}{3.389336in}}{\pgfqpoint{1.872361in}{3.400386in}}%
\pgfpathcurveto{\pgfqpoint{1.872361in}{3.411436in}}{\pgfqpoint{1.867971in}{3.422035in}}{\pgfqpoint{1.860157in}{3.429849in}}%
\pgfpathcurveto{\pgfqpoint{1.852343in}{3.437662in}}{\pgfqpoint{1.841744in}{3.442053in}}{\pgfqpoint{1.830694in}{3.442053in}}%
\pgfpathcurveto{\pgfqpoint{1.819644in}{3.442053in}}{\pgfqpoint{1.809045in}{3.437662in}}{\pgfqpoint{1.801232in}{3.429849in}}%
\pgfpathcurveto{\pgfqpoint{1.793418in}{3.422035in}}{\pgfqpoint{1.789028in}{3.411436in}}{\pgfqpoint{1.789028in}{3.400386in}}%
\pgfpathcurveto{\pgfqpoint{1.789028in}{3.389336in}}{\pgfqpoint{1.793418in}{3.378737in}}{\pgfqpoint{1.801232in}{3.370923in}}%
\pgfpathcurveto{\pgfqpoint{1.809045in}{3.363109in}}{\pgfqpoint{1.819644in}{3.358719in}}{\pgfqpoint{1.830694in}{3.358719in}}%
\pgfpathclose%
\pgfusepath{stroke,fill}%
\end{pgfscope}%
\begin{pgfscope}%
\pgfpathrectangle{\pgfqpoint{0.648703in}{0.548769in}}{\pgfqpoint{5.201297in}{3.102590in}}%
\pgfusepath{clip}%
\pgfsetbuttcap%
\pgfsetroundjoin%
\definecolor{currentfill}{rgb}{0.121569,0.466667,0.705882}%
\pgfsetfillcolor{currentfill}%
\pgfsetlinewidth{1.003750pt}%
\definecolor{currentstroke}{rgb}{0.121569,0.466667,0.705882}%
\pgfsetstrokecolor{currentstroke}%
\pgfsetdash{}{0pt}%
\pgfpathmoveto{\pgfqpoint{0.969392in}{0.648129in}}%
\pgfpathcurveto{\pgfqpoint{0.980442in}{0.648129in}}{\pgfqpoint{0.991041in}{0.652519in}}{\pgfqpoint{0.998855in}{0.660333in}}%
\pgfpathcurveto{\pgfqpoint{1.006669in}{0.668146in}}{\pgfqpoint{1.011059in}{0.678745in}}{\pgfqpoint{1.011059in}{0.689796in}}%
\pgfpathcurveto{\pgfqpoint{1.011059in}{0.700846in}}{\pgfqpoint{1.006669in}{0.711445in}}{\pgfqpoint{0.998855in}{0.719258in}}%
\pgfpathcurveto{\pgfqpoint{0.991041in}{0.727072in}}{\pgfqpoint{0.980442in}{0.731462in}}{\pgfqpoint{0.969392in}{0.731462in}}%
\pgfpathcurveto{\pgfqpoint{0.958342in}{0.731462in}}{\pgfqpoint{0.947743in}{0.727072in}}{\pgfqpoint{0.939929in}{0.719258in}}%
\pgfpathcurveto{\pgfqpoint{0.932116in}{0.711445in}}{\pgfqpoint{0.927725in}{0.700846in}}{\pgfqpoint{0.927725in}{0.689796in}}%
\pgfpathcurveto{\pgfqpoint{0.927725in}{0.678745in}}{\pgfqpoint{0.932116in}{0.668146in}}{\pgfqpoint{0.939929in}{0.660333in}}%
\pgfpathcurveto{\pgfqpoint{0.947743in}{0.652519in}}{\pgfqpoint{0.958342in}{0.648129in}}{\pgfqpoint{0.969392in}{0.648129in}}%
\pgfpathclose%
\pgfusepath{stroke,fill}%
\end{pgfscope}%
\begin{pgfscope}%
\pgfpathrectangle{\pgfqpoint{0.648703in}{0.548769in}}{\pgfqpoint{5.201297in}{3.102590in}}%
\pgfusepath{clip}%
\pgfsetbuttcap%
\pgfsetroundjoin%
\definecolor{currentfill}{rgb}{0.121569,0.466667,0.705882}%
\pgfsetfillcolor{currentfill}%
\pgfsetlinewidth{1.003750pt}%
\definecolor{currentstroke}{rgb}{0.121569,0.466667,0.705882}%
\pgfsetstrokecolor{currentstroke}%
\pgfsetdash{}{0pt}%
\pgfpathmoveto{\pgfqpoint{0.885126in}{0.796133in}}%
\pgfpathcurveto{\pgfqpoint{0.896176in}{0.796133in}}{\pgfqpoint{0.906775in}{0.800523in}}{\pgfqpoint{0.914589in}{0.808337in}}%
\pgfpathcurveto{\pgfqpoint{0.922402in}{0.816151in}}{\pgfqpoint{0.926793in}{0.826750in}}{\pgfqpoint{0.926793in}{0.837800in}}%
\pgfpathcurveto{\pgfqpoint{0.926793in}{0.848850in}}{\pgfqpoint{0.922402in}{0.859449in}}{\pgfqpoint{0.914589in}{0.867263in}}%
\pgfpathcurveto{\pgfqpoint{0.906775in}{0.875076in}}{\pgfqpoint{0.896176in}{0.879466in}}{\pgfqpoint{0.885126in}{0.879466in}}%
\pgfpathcurveto{\pgfqpoint{0.874076in}{0.879466in}}{\pgfqpoint{0.863477in}{0.875076in}}{\pgfqpoint{0.855663in}{0.867263in}}%
\pgfpathcurveto{\pgfqpoint{0.847850in}{0.859449in}}{\pgfqpoint{0.843459in}{0.848850in}}{\pgfqpoint{0.843459in}{0.837800in}}%
\pgfpathcurveto{\pgfqpoint{0.843459in}{0.826750in}}{\pgfqpoint{0.847850in}{0.816151in}}{\pgfqpoint{0.855663in}{0.808337in}}%
\pgfpathcurveto{\pgfqpoint{0.863477in}{0.800523in}}{\pgfqpoint{0.874076in}{0.796133in}}{\pgfqpoint{0.885126in}{0.796133in}}%
\pgfpathclose%
\pgfusepath{stroke,fill}%
\end{pgfscope}%
\begin{pgfscope}%
\pgfpathrectangle{\pgfqpoint{0.648703in}{0.548769in}}{\pgfqpoint{5.201297in}{3.102590in}}%
\pgfusepath{clip}%
\pgfsetbuttcap%
\pgfsetroundjoin%
\definecolor{currentfill}{rgb}{0.121569,0.466667,0.705882}%
\pgfsetfillcolor{currentfill}%
\pgfsetlinewidth{1.003750pt}%
\definecolor{currentstroke}{rgb}{0.121569,0.466667,0.705882}%
\pgfsetstrokecolor{currentstroke}%
\pgfsetdash{}{0pt}%
\pgfpathmoveto{\pgfqpoint{1.742012in}{3.155742in}}%
\pgfpathcurveto{\pgfqpoint{1.753062in}{3.155742in}}{\pgfqpoint{1.763661in}{3.160132in}}{\pgfqpoint{1.771475in}{3.167946in}}%
\pgfpathcurveto{\pgfqpoint{1.779288in}{3.175760in}}{\pgfqpoint{1.783679in}{3.186359in}}{\pgfqpoint{1.783679in}{3.197409in}}%
\pgfpathcurveto{\pgfqpoint{1.783679in}{3.208459in}}{\pgfqpoint{1.779288in}{3.219058in}}{\pgfqpoint{1.771475in}{3.226872in}}%
\pgfpathcurveto{\pgfqpoint{1.763661in}{3.234685in}}{\pgfqpoint{1.753062in}{3.239075in}}{\pgfqpoint{1.742012in}{3.239075in}}%
\pgfpathcurveto{\pgfqpoint{1.730962in}{3.239075in}}{\pgfqpoint{1.720363in}{3.234685in}}{\pgfqpoint{1.712549in}{3.226872in}}%
\pgfpathcurveto{\pgfqpoint{1.704736in}{3.219058in}}{\pgfqpoint{1.700345in}{3.208459in}}{\pgfqpoint{1.700345in}{3.197409in}}%
\pgfpathcurveto{\pgfqpoint{1.700345in}{3.186359in}}{\pgfqpoint{1.704736in}{3.175760in}}{\pgfqpoint{1.712549in}{3.167946in}}%
\pgfpathcurveto{\pgfqpoint{1.720363in}{3.160132in}}{\pgfqpoint{1.730962in}{3.155742in}}{\pgfqpoint{1.742012in}{3.155742in}}%
\pgfpathclose%
\pgfusepath{stroke,fill}%
\end{pgfscope}%
\begin{pgfscope}%
\pgfpathrectangle{\pgfqpoint{0.648703in}{0.548769in}}{\pgfqpoint{5.201297in}{3.102590in}}%
\pgfusepath{clip}%
\pgfsetbuttcap%
\pgfsetroundjoin%
\definecolor{currentfill}{rgb}{0.121569,0.466667,0.705882}%
\pgfsetfillcolor{currentfill}%
\pgfsetlinewidth{1.003750pt}%
\definecolor{currentstroke}{rgb}{0.121569,0.466667,0.705882}%
\pgfsetstrokecolor{currentstroke}%
\pgfsetdash{}{0pt}%
\pgfpathmoveto{\pgfqpoint{1.704321in}{0.648129in}}%
\pgfpathcurveto{\pgfqpoint{1.715371in}{0.648129in}}{\pgfqpoint{1.725970in}{0.652519in}}{\pgfqpoint{1.733784in}{0.660333in}}%
\pgfpathcurveto{\pgfqpoint{1.741598in}{0.668146in}}{\pgfqpoint{1.745988in}{0.678745in}}{\pgfqpoint{1.745988in}{0.689796in}}%
\pgfpathcurveto{\pgfqpoint{1.745988in}{0.700846in}}{\pgfqpoint{1.741598in}{0.711445in}}{\pgfqpoint{1.733784in}{0.719258in}}%
\pgfpathcurveto{\pgfqpoint{1.725970in}{0.727072in}}{\pgfqpoint{1.715371in}{0.731462in}}{\pgfqpoint{1.704321in}{0.731462in}}%
\pgfpathcurveto{\pgfqpoint{1.693271in}{0.731462in}}{\pgfqpoint{1.682672in}{0.727072in}}{\pgfqpoint{1.674858in}{0.719258in}}%
\pgfpathcurveto{\pgfqpoint{1.667045in}{0.711445in}}{\pgfqpoint{1.662655in}{0.700846in}}{\pgfqpoint{1.662655in}{0.689796in}}%
\pgfpathcurveto{\pgfqpoint{1.662655in}{0.678745in}}{\pgfqpoint{1.667045in}{0.668146in}}{\pgfqpoint{1.674858in}{0.660333in}}%
\pgfpathcurveto{\pgfqpoint{1.682672in}{0.652519in}}{\pgfqpoint{1.693271in}{0.648129in}}{\pgfqpoint{1.704321in}{0.648129in}}%
\pgfpathclose%
\pgfusepath{stroke,fill}%
\end{pgfscope}%
\begin{pgfscope}%
\pgfpathrectangle{\pgfqpoint{0.648703in}{0.548769in}}{\pgfqpoint{5.201297in}{3.102590in}}%
\pgfusepath{clip}%
\pgfsetbuttcap%
\pgfsetroundjoin%
\definecolor{currentfill}{rgb}{0.121569,0.466667,0.705882}%
\pgfsetfillcolor{currentfill}%
\pgfsetlinewidth{1.003750pt}%
\definecolor{currentstroke}{rgb}{0.121569,0.466667,0.705882}%
\pgfsetstrokecolor{currentstroke}%
\pgfsetdash{}{0pt}%
\pgfpathmoveto{\pgfqpoint{1.222426in}{0.648129in}}%
\pgfpathcurveto{\pgfqpoint{1.233476in}{0.648129in}}{\pgfqpoint{1.244075in}{0.652519in}}{\pgfqpoint{1.251889in}{0.660333in}}%
\pgfpathcurveto{\pgfqpoint{1.259702in}{0.668146in}}{\pgfqpoint{1.264092in}{0.678745in}}{\pgfqpoint{1.264092in}{0.689796in}}%
\pgfpathcurveto{\pgfqpoint{1.264092in}{0.700846in}}{\pgfqpoint{1.259702in}{0.711445in}}{\pgfqpoint{1.251889in}{0.719258in}}%
\pgfpathcurveto{\pgfqpoint{1.244075in}{0.727072in}}{\pgfqpoint{1.233476in}{0.731462in}}{\pgfqpoint{1.222426in}{0.731462in}}%
\pgfpathcurveto{\pgfqpoint{1.211376in}{0.731462in}}{\pgfqpoint{1.200777in}{0.727072in}}{\pgfqpoint{1.192963in}{0.719258in}}%
\pgfpathcurveto{\pgfqpoint{1.185149in}{0.711445in}}{\pgfqpoint{1.180759in}{0.700846in}}{\pgfqpoint{1.180759in}{0.689796in}}%
\pgfpathcurveto{\pgfqpoint{1.180759in}{0.678745in}}{\pgfqpoint{1.185149in}{0.668146in}}{\pgfqpoint{1.192963in}{0.660333in}}%
\pgfpathcurveto{\pgfqpoint{1.200777in}{0.652519in}}{\pgfqpoint{1.211376in}{0.648129in}}{\pgfqpoint{1.222426in}{0.648129in}}%
\pgfpathclose%
\pgfusepath{stroke,fill}%
\end{pgfscope}%
\begin{pgfscope}%
\pgfpathrectangle{\pgfqpoint{0.648703in}{0.548769in}}{\pgfqpoint{5.201297in}{3.102590in}}%
\pgfusepath{clip}%
\pgfsetbuttcap%
\pgfsetroundjoin%
\definecolor{currentfill}{rgb}{0.121569,0.466667,0.705882}%
\pgfsetfillcolor{currentfill}%
\pgfsetlinewidth{1.003750pt}%
\definecolor{currentstroke}{rgb}{0.121569,0.466667,0.705882}%
\pgfsetstrokecolor{currentstroke}%
\pgfsetdash{}{0pt}%
\pgfpathmoveto{\pgfqpoint{0.898349in}{2.512981in}}%
\pgfpathcurveto{\pgfqpoint{0.909399in}{2.512981in}}{\pgfqpoint{0.919998in}{2.517371in}}{\pgfqpoint{0.927811in}{2.525185in}}%
\pgfpathcurveto{\pgfqpoint{0.935625in}{2.532999in}}{\pgfqpoint{0.940015in}{2.543598in}}{\pgfqpoint{0.940015in}{2.554648in}}%
\pgfpathcurveto{\pgfqpoint{0.940015in}{2.565698in}}{\pgfqpoint{0.935625in}{2.576297in}}{\pgfqpoint{0.927811in}{2.584111in}}%
\pgfpathcurveto{\pgfqpoint{0.919998in}{2.591924in}}{\pgfqpoint{0.909399in}{2.596315in}}{\pgfqpoint{0.898349in}{2.596315in}}%
\pgfpathcurveto{\pgfqpoint{0.887299in}{2.596315in}}{\pgfqpoint{0.876699in}{2.591924in}}{\pgfqpoint{0.868886in}{2.584111in}}%
\pgfpathcurveto{\pgfqpoint{0.861072in}{2.576297in}}{\pgfqpoint{0.856682in}{2.565698in}}{\pgfqpoint{0.856682in}{2.554648in}}%
\pgfpathcurveto{\pgfqpoint{0.856682in}{2.543598in}}{\pgfqpoint{0.861072in}{2.532999in}}{\pgfqpoint{0.868886in}{2.525185in}}%
\pgfpathcurveto{\pgfqpoint{0.876699in}{2.517371in}}{\pgfqpoint{0.887299in}{2.512981in}}{\pgfqpoint{0.898349in}{2.512981in}}%
\pgfpathclose%
\pgfusepath{stroke,fill}%
\end{pgfscope}%
\begin{pgfscope}%
\pgfpathrectangle{\pgfqpoint{0.648703in}{0.548769in}}{\pgfqpoint{5.201297in}{3.102590in}}%
\pgfusepath{clip}%
\pgfsetbuttcap%
\pgfsetroundjoin%
\definecolor{currentfill}{rgb}{1.000000,0.498039,0.054902}%
\pgfsetfillcolor{currentfill}%
\pgfsetlinewidth{1.003750pt}%
\definecolor{currentstroke}{rgb}{1.000000,0.498039,0.054902}%
\pgfsetstrokecolor{currentstroke}%
\pgfsetdash{}{0pt}%
\pgfpathmoveto{\pgfqpoint{1.216041in}{3.206486in}}%
\pgfpathcurveto{\pgfqpoint{1.227091in}{3.206486in}}{\pgfqpoint{1.237690in}{3.210877in}}{\pgfqpoint{1.245504in}{3.218690in}}%
\pgfpathcurveto{\pgfqpoint{1.253317in}{3.226504in}}{\pgfqpoint{1.257708in}{3.237103in}}{\pgfqpoint{1.257708in}{3.248153in}}%
\pgfpathcurveto{\pgfqpoint{1.257708in}{3.259203in}}{\pgfqpoint{1.253317in}{3.269802in}}{\pgfqpoint{1.245504in}{3.277616in}}%
\pgfpathcurveto{\pgfqpoint{1.237690in}{3.285429in}}{\pgfqpoint{1.227091in}{3.289820in}}{\pgfqpoint{1.216041in}{3.289820in}}%
\pgfpathcurveto{\pgfqpoint{1.204991in}{3.289820in}}{\pgfqpoint{1.194392in}{3.285429in}}{\pgfqpoint{1.186578in}{3.277616in}}%
\pgfpathcurveto{\pgfqpoint{1.178765in}{3.269802in}}{\pgfqpoint{1.174374in}{3.259203in}}{\pgfqpoint{1.174374in}{3.248153in}}%
\pgfpathcurveto{\pgfqpoint{1.174374in}{3.237103in}}{\pgfqpoint{1.178765in}{3.226504in}}{\pgfqpoint{1.186578in}{3.218690in}}%
\pgfpathcurveto{\pgfqpoint{1.194392in}{3.210877in}}{\pgfqpoint{1.204991in}{3.206486in}}{\pgfqpoint{1.216041in}{3.206486in}}%
\pgfpathclose%
\pgfusepath{stroke,fill}%
\end{pgfscope}%
\begin{pgfscope}%
\pgfpathrectangle{\pgfqpoint{0.648703in}{0.548769in}}{\pgfqpoint{5.201297in}{3.102590in}}%
\pgfusepath{clip}%
\pgfsetbuttcap%
\pgfsetroundjoin%
\definecolor{currentfill}{rgb}{1.000000,0.498039,0.054902}%
\pgfsetfillcolor{currentfill}%
\pgfsetlinewidth{1.003750pt}%
\definecolor{currentstroke}{rgb}{1.000000,0.498039,0.054902}%
\pgfsetstrokecolor{currentstroke}%
\pgfsetdash{}{0pt}%
\pgfpathmoveto{\pgfqpoint{2.213045in}{3.198029in}}%
\pgfpathcurveto{\pgfqpoint{2.224095in}{3.198029in}}{\pgfqpoint{2.234694in}{3.202419in}}{\pgfqpoint{2.242508in}{3.210233in}}%
\pgfpathcurveto{\pgfqpoint{2.250322in}{3.218046in}}{\pgfqpoint{2.254712in}{3.228646in}}{\pgfqpoint{2.254712in}{3.239696in}}%
\pgfpathcurveto{\pgfqpoint{2.254712in}{3.250746in}}{\pgfqpoint{2.250322in}{3.261345in}}{\pgfqpoint{2.242508in}{3.269158in}}%
\pgfpathcurveto{\pgfqpoint{2.234694in}{3.276972in}}{\pgfqpoint{2.224095in}{3.281362in}}{\pgfqpoint{2.213045in}{3.281362in}}%
\pgfpathcurveto{\pgfqpoint{2.201995in}{3.281362in}}{\pgfqpoint{2.191396in}{3.276972in}}{\pgfqpoint{2.183582in}{3.269158in}}%
\pgfpathcurveto{\pgfqpoint{2.175769in}{3.261345in}}{\pgfqpoint{2.171379in}{3.250746in}}{\pgfqpoint{2.171379in}{3.239696in}}%
\pgfpathcurveto{\pgfqpoint{2.171379in}{3.228646in}}{\pgfqpoint{2.175769in}{3.218046in}}{\pgfqpoint{2.183582in}{3.210233in}}%
\pgfpathcurveto{\pgfqpoint{2.191396in}{3.202419in}}{\pgfqpoint{2.201995in}{3.198029in}}{\pgfqpoint{2.213045in}{3.198029in}}%
\pgfpathclose%
\pgfusepath{stroke,fill}%
\end{pgfscope}%
\begin{pgfscope}%
\pgfpathrectangle{\pgfqpoint{0.648703in}{0.548769in}}{\pgfqpoint{5.201297in}{3.102590in}}%
\pgfusepath{clip}%
\pgfsetbuttcap%
\pgfsetroundjoin%
\definecolor{currentfill}{rgb}{1.000000,0.498039,0.054902}%
\pgfsetfillcolor{currentfill}%
\pgfsetlinewidth{1.003750pt}%
\definecolor{currentstroke}{rgb}{1.000000,0.498039,0.054902}%
\pgfsetstrokecolor{currentstroke}%
\pgfsetdash{}{0pt}%
\pgfpathmoveto{\pgfqpoint{2.716691in}{3.248773in}}%
\pgfpathcurveto{\pgfqpoint{2.727741in}{3.248773in}}{\pgfqpoint{2.738340in}{3.253164in}}{\pgfqpoint{2.746154in}{3.260977in}}%
\pgfpathcurveto{\pgfqpoint{2.753967in}{3.268791in}}{\pgfqpoint{2.758358in}{3.279390in}}{\pgfqpoint{2.758358in}{3.290440in}}%
\pgfpathcurveto{\pgfqpoint{2.758358in}{3.301490in}}{\pgfqpoint{2.753967in}{3.312089in}}{\pgfqpoint{2.746154in}{3.319903in}}%
\pgfpathcurveto{\pgfqpoint{2.738340in}{3.327716in}}{\pgfqpoint{2.727741in}{3.332107in}}{\pgfqpoint{2.716691in}{3.332107in}}%
\pgfpathcurveto{\pgfqpoint{2.705641in}{3.332107in}}{\pgfqpoint{2.695042in}{3.327716in}}{\pgfqpoint{2.687228in}{3.319903in}}%
\pgfpathcurveto{\pgfqpoint{2.679415in}{3.312089in}}{\pgfqpoint{2.675024in}{3.301490in}}{\pgfqpoint{2.675024in}{3.290440in}}%
\pgfpathcurveto{\pgfqpoint{2.675024in}{3.279390in}}{\pgfqpoint{2.679415in}{3.268791in}}{\pgfqpoint{2.687228in}{3.260977in}}%
\pgfpathcurveto{\pgfqpoint{2.695042in}{3.253164in}}{\pgfqpoint{2.705641in}{3.248773in}}{\pgfqpoint{2.716691in}{3.248773in}}%
\pgfpathclose%
\pgfusepath{stroke,fill}%
\end{pgfscope}%
\begin{pgfscope}%
\pgfpathrectangle{\pgfqpoint{0.648703in}{0.548769in}}{\pgfqpoint{5.201297in}{3.102590in}}%
\pgfusepath{clip}%
\pgfsetbuttcap%
\pgfsetroundjoin%
\definecolor{currentfill}{rgb}{1.000000,0.498039,0.054902}%
\pgfsetfillcolor{currentfill}%
\pgfsetlinewidth{1.003750pt}%
\definecolor{currentstroke}{rgb}{1.000000,0.498039,0.054902}%
\pgfsetstrokecolor{currentstroke}%
\pgfsetdash{}{0pt}%
\pgfpathmoveto{\pgfqpoint{1.832497in}{3.189572in}}%
\pgfpathcurveto{\pgfqpoint{1.843548in}{3.189572in}}{\pgfqpoint{1.854147in}{3.193962in}}{\pgfqpoint{1.861960in}{3.201775in}}%
\pgfpathcurveto{\pgfqpoint{1.869774in}{3.209589in}}{\pgfqpoint{1.874164in}{3.220188in}}{\pgfqpoint{1.874164in}{3.231238in}}%
\pgfpathcurveto{\pgfqpoint{1.874164in}{3.242288in}}{\pgfqpoint{1.869774in}{3.252887in}}{\pgfqpoint{1.861960in}{3.260701in}}%
\pgfpathcurveto{\pgfqpoint{1.854147in}{3.268515in}}{\pgfqpoint{1.843548in}{3.272905in}}{\pgfqpoint{1.832497in}{3.272905in}}%
\pgfpathcurveto{\pgfqpoint{1.821447in}{3.272905in}}{\pgfqpoint{1.810848in}{3.268515in}}{\pgfqpoint{1.803035in}{3.260701in}}%
\pgfpathcurveto{\pgfqpoint{1.795221in}{3.252887in}}{\pgfqpoint{1.790831in}{3.242288in}}{\pgfqpoint{1.790831in}{3.231238in}}%
\pgfpathcurveto{\pgfqpoint{1.790831in}{3.220188in}}{\pgfqpoint{1.795221in}{3.209589in}}{\pgfqpoint{1.803035in}{3.201775in}}%
\pgfpathcurveto{\pgfqpoint{1.810848in}{3.193962in}}{\pgfqpoint{1.821447in}{3.189572in}}{\pgfqpoint{1.832497in}{3.189572in}}%
\pgfpathclose%
\pgfusepath{stroke,fill}%
\end{pgfscope}%
\begin{pgfscope}%
\pgfpathrectangle{\pgfqpoint{0.648703in}{0.548769in}}{\pgfqpoint{5.201297in}{3.102590in}}%
\pgfusepath{clip}%
\pgfsetbuttcap%
\pgfsetroundjoin%
\definecolor{currentfill}{rgb}{0.121569,0.466667,0.705882}%
\pgfsetfillcolor{currentfill}%
\pgfsetlinewidth{1.003750pt}%
\definecolor{currentstroke}{rgb}{0.121569,0.466667,0.705882}%
\pgfsetstrokecolor{currentstroke}%
\pgfsetdash{}{0pt}%
\pgfpathmoveto{\pgfqpoint{1.805852in}{3.181114in}}%
\pgfpathcurveto{\pgfqpoint{1.816902in}{3.181114in}}{\pgfqpoint{1.827501in}{3.185504in}}{\pgfqpoint{1.835315in}{3.193318in}}%
\pgfpathcurveto{\pgfqpoint{1.843128in}{3.201132in}}{\pgfqpoint{1.847518in}{3.211731in}}{\pgfqpoint{1.847518in}{3.222781in}}%
\pgfpathcurveto{\pgfqpoint{1.847518in}{3.233831in}}{\pgfqpoint{1.843128in}{3.244430in}}{\pgfqpoint{1.835315in}{3.252244in}}%
\pgfpathcurveto{\pgfqpoint{1.827501in}{3.260057in}}{\pgfqpoint{1.816902in}{3.264448in}}{\pgfqpoint{1.805852in}{3.264448in}}%
\pgfpathcurveto{\pgfqpoint{1.794802in}{3.264448in}}{\pgfqpoint{1.784203in}{3.260057in}}{\pgfqpoint{1.776389in}{3.252244in}}%
\pgfpathcurveto{\pgfqpoint{1.768575in}{3.244430in}}{\pgfqpoint{1.764185in}{3.233831in}}{\pgfqpoint{1.764185in}{3.222781in}}%
\pgfpathcurveto{\pgfqpoint{1.764185in}{3.211731in}}{\pgfqpoint{1.768575in}{3.201132in}}{\pgfqpoint{1.776389in}{3.193318in}}%
\pgfpathcurveto{\pgfqpoint{1.784203in}{3.185504in}}{\pgfqpoint{1.794802in}{3.181114in}}{\pgfqpoint{1.805852in}{3.181114in}}%
\pgfpathclose%
\pgfusepath{stroke,fill}%
\end{pgfscope}%
\begin{pgfscope}%
\pgfpathrectangle{\pgfqpoint{0.648703in}{0.548769in}}{\pgfqpoint{5.201297in}{3.102590in}}%
\pgfusepath{clip}%
\pgfsetbuttcap%
\pgfsetroundjoin%
\definecolor{currentfill}{rgb}{0.839216,0.152941,0.156863}%
\pgfsetfillcolor{currentfill}%
\pgfsetlinewidth{1.003750pt}%
\definecolor{currentstroke}{rgb}{0.839216,0.152941,0.156863}%
\pgfsetstrokecolor{currentstroke}%
\pgfsetdash{}{0pt}%
\pgfpathmoveto{\pgfqpoint{1.494788in}{3.202258in}}%
\pgfpathcurveto{\pgfqpoint{1.505838in}{3.202258in}}{\pgfqpoint{1.516437in}{3.206648in}}{\pgfqpoint{1.524251in}{3.214462in}}%
\pgfpathcurveto{\pgfqpoint{1.532065in}{3.222275in}}{\pgfqpoint{1.536455in}{3.232874in}}{\pgfqpoint{1.536455in}{3.243924in}}%
\pgfpathcurveto{\pgfqpoint{1.536455in}{3.254974in}}{\pgfqpoint{1.532065in}{3.265573in}}{\pgfqpoint{1.524251in}{3.273387in}}%
\pgfpathcurveto{\pgfqpoint{1.516437in}{3.281201in}}{\pgfqpoint{1.505838in}{3.285591in}}{\pgfqpoint{1.494788in}{3.285591in}}%
\pgfpathcurveto{\pgfqpoint{1.483738in}{3.285591in}}{\pgfqpoint{1.473139in}{3.281201in}}{\pgfqpoint{1.465325in}{3.273387in}}%
\pgfpathcurveto{\pgfqpoint{1.457512in}{3.265573in}}{\pgfqpoint{1.453122in}{3.254974in}}{\pgfqpoint{1.453122in}{3.243924in}}%
\pgfpathcurveto{\pgfqpoint{1.453122in}{3.232874in}}{\pgfqpoint{1.457512in}{3.222275in}}{\pgfqpoint{1.465325in}{3.214462in}}%
\pgfpathcurveto{\pgfqpoint{1.473139in}{3.206648in}}{\pgfqpoint{1.483738in}{3.202258in}}{\pgfqpoint{1.494788in}{3.202258in}}%
\pgfpathclose%
\pgfusepath{stroke,fill}%
\end{pgfscope}%
\begin{pgfscope}%
\pgfpathrectangle{\pgfqpoint{0.648703in}{0.548769in}}{\pgfqpoint{5.201297in}{3.102590in}}%
\pgfusepath{clip}%
\pgfsetbuttcap%
\pgfsetroundjoin%
\definecolor{currentfill}{rgb}{1.000000,0.498039,0.054902}%
\pgfsetfillcolor{currentfill}%
\pgfsetlinewidth{1.003750pt}%
\definecolor{currentstroke}{rgb}{1.000000,0.498039,0.054902}%
\pgfsetstrokecolor{currentstroke}%
\pgfsetdash{}{0pt}%
\pgfpathmoveto{\pgfqpoint{1.370680in}{3.185343in}}%
\pgfpathcurveto{\pgfqpoint{1.381730in}{3.185343in}}{\pgfqpoint{1.392329in}{3.189733in}}{\pgfqpoint{1.400143in}{3.197547in}}%
\pgfpathcurveto{\pgfqpoint{1.407956in}{3.205360in}}{\pgfqpoint{1.412347in}{3.215959in}}{\pgfqpoint{1.412347in}{3.227010in}}%
\pgfpathcurveto{\pgfqpoint{1.412347in}{3.238060in}}{\pgfqpoint{1.407956in}{3.248659in}}{\pgfqpoint{1.400143in}{3.256472in}}%
\pgfpathcurveto{\pgfqpoint{1.392329in}{3.264286in}}{\pgfqpoint{1.381730in}{3.268676in}}{\pgfqpoint{1.370680in}{3.268676in}}%
\pgfpathcurveto{\pgfqpoint{1.359630in}{3.268676in}}{\pgfqpoint{1.349031in}{3.264286in}}{\pgfqpoint{1.341217in}{3.256472in}}%
\pgfpathcurveto{\pgfqpoint{1.333403in}{3.248659in}}{\pgfqpoint{1.329013in}{3.238060in}}{\pgfqpoint{1.329013in}{3.227010in}}%
\pgfpathcurveto{\pgfqpoint{1.329013in}{3.215959in}}{\pgfqpoint{1.333403in}{3.205360in}}{\pgfqpoint{1.341217in}{3.197547in}}%
\pgfpathcurveto{\pgfqpoint{1.349031in}{3.189733in}}{\pgfqpoint{1.359630in}{3.185343in}}{\pgfqpoint{1.370680in}{3.185343in}}%
\pgfpathclose%
\pgfusepath{stroke,fill}%
\end{pgfscope}%
\begin{pgfscope}%
\pgfpathrectangle{\pgfqpoint{0.648703in}{0.548769in}}{\pgfqpoint{5.201297in}{3.102590in}}%
\pgfusepath{clip}%
\pgfsetbuttcap%
\pgfsetroundjoin%
\definecolor{currentfill}{rgb}{1.000000,0.498039,0.054902}%
\pgfsetfillcolor{currentfill}%
\pgfsetlinewidth{1.003750pt}%
\definecolor{currentstroke}{rgb}{1.000000,0.498039,0.054902}%
\pgfsetstrokecolor{currentstroke}%
\pgfsetdash{}{0pt}%
\pgfpathmoveto{\pgfqpoint{2.206556in}{3.278374in}}%
\pgfpathcurveto{\pgfqpoint{2.217606in}{3.278374in}}{\pgfqpoint{2.228205in}{3.282764in}}{\pgfqpoint{2.236019in}{3.290578in}}%
\pgfpathcurveto{\pgfqpoint{2.243832in}{3.298392in}}{\pgfqpoint{2.248223in}{3.308991in}}{\pgfqpoint{2.248223in}{3.320041in}}%
\pgfpathcurveto{\pgfqpoint{2.248223in}{3.331091in}}{\pgfqpoint{2.243832in}{3.341690in}}{\pgfqpoint{2.236019in}{3.349504in}}%
\pgfpathcurveto{\pgfqpoint{2.228205in}{3.357317in}}{\pgfqpoint{2.217606in}{3.361707in}}{\pgfqpoint{2.206556in}{3.361707in}}%
\pgfpathcurveto{\pgfqpoint{2.195506in}{3.361707in}}{\pgfqpoint{2.184907in}{3.357317in}}{\pgfqpoint{2.177093in}{3.349504in}}%
\pgfpathcurveto{\pgfqpoint{2.169279in}{3.341690in}}{\pgfqpoint{2.164889in}{3.331091in}}{\pgfqpoint{2.164889in}{3.320041in}}%
\pgfpathcurveto{\pgfqpoint{2.164889in}{3.308991in}}{\pgfqpoint{2.169279in}{3.298392in}}{\pgfqpoint{2.177093in}{3.290578in}}%
\pgfpathcurveto{\pgfqpoint{2.184907in}{3.282764in}}{\pgfqpoint{2.195506in}{3.278374in}}{\pgfqpoint{2.206556in}{3.278374in}}%
\pgfpathclose%
\pgfusepath{stroke,fill}%
\end{pgfscope}%
\begin{pgfscope}%
\pgfpathrectangle{\pgfqpoint{0.648703in}{0.548769in}}{\pgfqpoint{5.201297in}{3.102590in}}%
\pgfusepath{clip}%
\pgfsetbuttcap%
\pgfsetroundjoin%
\definecolor{currentfill}{rgb}{0.121569,0.466667,0.705882}%
\pgfsetfillcolor{currentfill}%
\pgfsetlinewidth{1.003750pt}%
\definecolor{currentstroke}{rgb}{0.121569,0.466667,0.705882}%
\pgfsetstrokecolor{currentstroke}%
\pgfsetdash{}{0pt}%
\pgfpathmoveto{\pgfqpoint{2.670551in}{3.181114in}}%
\pgfpathcurveto{\pgfqpoint{2.681601in}{3.181114in}}{\pgfqpoint{2.692200in}{3.185504in}}{\pgfqpoint{2.700014in}{3.193318in}}%
\pgfpathcurveto{\pgfqpoint{2.707827in}{3.201132in}}{\pgfqpoint{2.712218in}{3.211731in}}{\pgfqpoint{2.712218in}{3.222781in}}%
\pgfpathcurveto{\pgfqpoint{2.712218in}{3.233831in}}{\pgfqpoint{2.707827in}{3.244430in}}{\pgfqpoint{2.700014in}{3.252244in}}%
\pgfpathcurveto{\pgfqpoint{2.692200in}{3.260057in}}{\pgfqpoint{2.681601in}{3.264448in}}{\pgfqpoint{2.670551in}{3.264448in}}%
\pgfpathcurveto{\pgfqpoint{2.659501in}{3.264448in}}{\pgfqpoint{2.648902in}{3.260057in}}{\pgfqpoint{2.641088in}{3.252244in}}%
\pgfpathcurveto{\pgfqpoint{2.633275in}{3.244430in}}{\pgfqpoint{2.628884in}{3.233831in}}{\pgfqpoint{2.628884in}{3.222781in}}%
\pgfpathcurveto{\pgfqpoint{2.628884in}{3.211731in}}{\pgfqpoint{2.633275in}{3.201132in}}{\pgfqpoint{2.641088in}{3.193318in}}%
\pgfpathcurveto{\pgfqpoint{2.648902in}{3.185504in}}{\pgfqpoint{2.659501in}{3.181114in}}{\pgfqpoint{2.670551in}{3.181114in}}%
\pgfpathclose%
\pgfusepath{stroke,fill}%
\end{pgfscope}%
\begin{pgfscope}%
\pgfpathrectangle{\pgfqpoint{0.648703in}{0.548769in}}{\pgfqpoint{5.201297in}{3.102590in}}%
\pgfusepath{clip}%
\pgfsetbuttcap%
\pgfsetroundjoin%
\definecolor{currentfill}{rgb}{1.000000,0.498039,0.054902}%
\pgfsetfillcolor{currentfill}%
\pgfsetlinewidth{1.003750pt}%
\definecolor{currentstroke}{rgb}{1.000000,0.498039,0.054902}%
\pgfsetstrokecolor{currentstroke}%
\pgfsetdash{}{0pt}%
\pgfpathmoveto{\pgfqpoint{4.610092in}{3.219172in}}%
\pgfpathcurveto{\pgfqpoint{4.621143in}{3.219172in}}{\pgfqpoint{4.631742in}{3.223563in}}{\pgfqpoint{4.639555in}{3.231376in}}%
\pgfpathcurveto{\pgfqpoint{4.647369in}{3.239190in}}{\pgfqpoint{4.651759in}{3.249789in}}{\pgfqpoint{4.651759in}{3.260839in}}%
\pgfpathcurveto{\pgfqpoint{4.651759in}{3.271889in}}{\pgfqpoint{4.647369in}{3.282488in}}{\pgfqpoint{4.639555in}{3.290302in}}%
\pgfpathcurveto{\pgfqpoint{4.631742in}{3.298116in}}{\pgfqpoint{4.621143in}{3.302506in}}{\pgfqpoint{4.610092in}{3.302506in}}%
\pgfpathcurveto{\pgfqpoint{4.599042in}{3.302506in}}{\pgfqpoint{4.588443in}{3.298116in}}{\pgfqpoint{4.580630in}{3.290302in}}%
\pgfpathcurveto{\pgfqpoint{4.572816in}{3.282488in}}{\pgfqpoint{4.568426in}{3.271889in}}{\pgfqpoint{4.568426in}{3.260839in}}%
\pgfpathcurveto{\pgfqpoint{4.568426in}{3.249789in}}{\pgfqpoint{4.572816in}{3.239190in}}{\pgfqpoint{4.580630in}{3.231376in}}%
\pgfpathcurveto{\pgfqpoint{4.588443in}{3.223563in}}{\pgfqpoint{4.599042in}{3.219172in}}{\pgfqpoint{4.610092in}{3.219172in}}%
\pgfpathclose%
\pgfusepath{stroke,fill}%
\end{pgfscope}%
\begin{pgfscope}%
\pgfpathrectangle{\pgfqpoint{0.648703in}{0.548769in}}{\pgfqpoint{5.201297in}{3.102590in}}%
\pgfusepath{clip}%
\pgfsetbuttcap%
\pgfsetroundjoin%
\definecolor{currentfill}{rgb}{1.000000,0.498039,0.054902}%
\pgfsetfillcolor{currentfill}%
\pgfsetlinewidth{1.003750pt}%
\definecolor{currentstroke}{rgb}{1.000000,0.498039,0.054902}%
\pgfsetstrokecolor{currentstroke}%
\pgfsetdash{}{0pt}%
\pgfpathmoveto{\pgfqpoint{1.668730in}{3.468665in}}%
\pgfpathcurveto{\pgfqpoint{1.679780in}{3.468665in}}{\pgfqpoint{1.690379in}{3.473055in}}{\pgfqpoint{1.698193in}{3.480869in}}%
\pgfpathcurveto{\pgfqpoint{1.706006in}{3.488683in}}{\pgfqpoint{1.710396in}{3.499282in}}{\pgfqpoint{1.710396in}{3.510332in}}%
\pgfpathcurveto{\pgfqpoint{1.710396in}{3.521382in}}{\pgfqpoint{1.706006in}{3.531981in}}{\pgfqpoint{1.698193in}{3.539795in}}%
\pgfpathcurveto{\pgfqpoint{1.690379in}{3.547608in}}{\pgfqpoint{1.679780in}{3.551998in}}{\pgfqpoint{1.668730in}{3.551998in}}%
\pgfpathcurveto{\pgfqpoint{1.657680in}{3.551998in}}{\pgfqpoint{1.647081in}{3.547608in}}{\pgfqpoint{1.639267in}{3.539795in}}%
\pgfpathcurveto{\pgfqpoint{1.631453in}{3.531981in}}{\pgfqpoint{1.627063in}{3.521382in}}{\pgfqpoint{1.627063in}{3.510332in}}%
\pgfpathcurveto{\pgfqpoint{1.627063in}{3.499282in}}{\pgfqpoint{1.631453in}{3.488683in}}{\pgfqpoint{1.639267in}{3.480869in}}%
\pgfpathcurveto{\pgfqpoint{1.647081in}{3.473055in}}{\pgfqpoint{1.657680in}{3.468665in}}{\pgfqpoint{1.668730in}{3.468665in}}%
\pgfpathclose%
\pgfusepath{stroke,fill}%
\end{pgfscope}%
\begin{pgfscope}%
\pgfpathrectangle{\pgfqpoint{0.648703in}{0.548769in}}{\pgfqpoint{5.201297in}{3.102590in}}%
\pgfusepath{clip}%
\pgfsetbuttcap%
\pgfsetroundjoin%
\definecolor{currentfill}{rgb}{1.000000,0.498039,0.054902}%
\pgfsetfillcolor{currentfill}%
\pgfsetlinewidth{1.003750pt}%
\definecolor{currentstroke}{rgb}{1.000000,0.498039,0.054902}%
\pgfsetstrokecolor{currentstroke}%
\pgfsetdash{}{0pt}%
\pgfpathmoveto{\pgfqpoint{2.130791in}{3.189572in}}%
\pgfpathcurveto{\pgfqpoint{2.141841in}{3.189572in}}{\pgfqpoint{2.152440in}{3.193962in}}{\pgfqpoint{2.160254in}{3.201775in}}%
\pgfpathcurveto{\pgfqpoint{2.168068in}{3.209589in}}{\pgfqpoint{2.172458in}{3.220188in}}{\pgfqpoint{2.172458in}{3.231238in}}%
\pgfpathcurveto{\pgfqpoint{2.172458in}{3.242288in}}{\pgfqpoint{2.168068in}{3.252887in}}{\pgfqpoint{2.160254in}{3.260701in}}%
\pgfpathcurveto{\pgfqpoint{2.152440in}{3.268515in}}{\pgfqpoint{2.141841in}{3.272905in}}{\pgfqpoint{2.130791in}{3.272905in}}%
\pgfpathcurveto{\pgfqpoint{2.119741in}{3.272905in}}{\pgfqpoint{2.109142in}{3.268515in}}{\pgfqpoint{2.101328in}{3.260701in}}%
\pgfpathcurveto{\pgfqpoint{2.093515in}{3.252887in}}{\pgfqpoint{2.089125in}{3.242288in}}{\pgfqpoint{2.089125in}{3.231238in}}%
\pgfpathcurveto{\pgfqpoint{2.089125in}{3.220188in}}{\pgfqpoint{2.093515in}{3.209589in}}{\pgfqpoint{2.101328in}{3.201775in}}%
\pgfpathcurveto{\pgfqpoint{2.109142in}{3.193962in}}{\pgfqpoint{2.119741in}{3.189572in}}{\pgfqpoint{2.130791in}{3.189572in}}%
\pgfpathclose%
\pgfusepath{stroke,fill}%
\end{pgfscope}%
\begin{pgfscope}%
\pgfpathrectangle{\pgfqpoint{0.648703in}{0.548769in}}{\pgfqpoint{5.201297in}{3.102590in}}%
\pgfusepath{clip}%
\pgfsetbuttcap%
\pgfsetroundjoin%
\definecolor{currentfill}{rgb}{1.000000,0.498039,0.054902}%
\pgfsetfillcolor{currentfill}%
\pgfsetlinewidth{1.003750pt}%
\definecolor{currentstroke}{rgb}{1.000000,0.498039,0.054902}%
\pgfsetstrokecolor{currentstroke}%
\pgfsetdash{}{0pt}%
\pgfpathmoveto{\pgfqpoint{1.963008in}{3.189572in}}%
\pgfpathcurveto{\pgfqpoint{1.974058in}{3.189572in}}{\pgfqpoint{1.984657in}{3.193962in}}{\pgfqpoint{1.992471in}{3.201775in}}%
\pgfpathcurveto{\pgfqpoint{2.000284in}{3.209589in}}{\pgfqpoint{2.004675in}{3.220188in}}{\pgfqpoint{2.004675in}{3.231238in}}%
\pgfpathcurveto{\pgfqpoint{2.004675in}{3.242288in}}{\pgfqpoint{2.000284in}{3.252887in}}{\pgfqpoint{1.992471in}{3.260701in}}%
\pgfpathcurveto{\pgfqpoint{1.984657in}{3.268515in}}{\pgfqpoint{1.974058in}{3.272905in}}{\pgfqpoint{1.963008in}{3.272905in}}%
\pgfpathcurveto{\pgfqpoint{1.951958in}{3.272905in}}{\pgfqpoint{1.941359in}{3.268515in}}{\pgfqpoint{1.933545in}{3.260701in}}%
\pgfpathcurveto{\pgfqpoint{1.925732in}{3.252887in}}{\pgfqpoint{1.921341in}{3.242288in}}{\pgfqpoint{1.921341in}{3.231238in}}%
\pgfpathcurveto{\pgfqpoint{1.921341in}{3.220188in}}{\pgfqpoint{1.925732in}{3.209589in}}{\pgfqpoint{1.933545in}{3.201775in}}%
\pgfpathcurveto{\pgfqpoint{1.941359in}{3.193962in}}{\pgfqpoint{1.951958in}{3.189572in}}{\pgfqpoint{1.963008in}{3.189572in}}%
\pgfpathclose%
\pgfusepath{stroke,fill}%
\end{pgfscope}%
\begin{pgfscope}%
\pgfpathrectangle{\pgfqpoint{0.648703in}{0.548769in}}{\pgfqpoint{5.201297in}{3.102590in}}%
\pgfusepath{clip}%
\pgfsetbuttcap%
\pgfsetroundjoin%
\definecolor{currentfill}{rgb}{0.121569,0.466667,0.705882}%
\pgfsetfillcolor{currentfill}%
\pgfsetlinewidth{1.003750pt}%
\definecolor{currentstroke}{rgb}{0.121569,0.466667,0.705882}%
\pgfsetstrokecolor{currentstroke}%
\pgfsetdash{}{0pt}%
\pgfpathmoveto{\pgfqpoint{2.719435in}{3.181114in}}%
\pgfpathcurveto{\pgfqpoint{2.730485in}{3.181114in}}{\pgfqpoint{2.741084in}{3.185504in}}{\pgfqpoint{2.748898in}{3.193318in}}%
\pgfpathcurveto{\pgfqpoint{2.756711in}{3.201132in}}{\pgfqpoint{2.761102in}{3.211731in}}{\pgfqpoint{2.761102in}{3.222781in}}%
\pgfpathcurveto{\pgfqpoint{2.761102in}{3.233831in}}{\pgfqpoint{2.756711in}{3.244430in}}{\pgfqpoint{2.748898in}{3.252244in}}%
\pgfpathcurveto{\pgfqpoint{2.741084in}{3.260057in}}{\pgfqpoint{2.730485in}{3.264448in}}{\pgfqpoint{2.719435in}{3.264448in}}%
\pgfpathcurveto{\pgfqpoint{2.708385in}{3.264448in}}{\pgfqpoint{2.697786in}{3.260057in}}{\pgfqpoint{2.689972in}{3.252244in}}%
\pgfpathcurveto{\pgfqpoint{2.682158in}{3.244430in}}{\pgfqpoint{2.677768in}{3.233831in}}{\pgfqpoint{2.677768in}{3.222781in}}%
\pgfpathcurveto{\pgfqpoint{2.677768in}{3.211731in}}{\pgfqpoint{2.682158in}{3.201132in}}{\pgfqpoint{2.689972in}{3.193318in}}%
\pgfpathcurveto{\pgfqpoint{2.697786in}{3.185504in}}{\pgfqpoint{2.708385in}{3.181114in}}{\pgfqpoint{2.719435in}{3.181114in}}%
\pgfpathclose%
\pgfusepath{stroke,fill}%
\end{pgfscope}%
\begin{pgfscope}%
\pgfpathrectangle{\pgfqpoint{0.648703in}{0.548769in}}{\pgfqpoint{5.201297in}{3.102590in}}%
\pgfusepath{clip}%
\pgfsetbuttcap%
\pgfsetroundjoin%
\definecolor{currentfill}{rgb}{1.000000,0.498039,0.054902}%
\pgfsetfillcolor{currentfill}%
\pgfsetlinewidth{1.003750pt}%
\definecolor{currentstroke}{rgb}{1.000000,0.498039,0.054902}%
\pgfsetstrokecolor{currentstroke}%
\pgfsetdash{}{0pt}%
\pgfpathmoveto{\pgfqpoint{2.249786in}{3.185343in}}%
\pgfpathcurveto{\pgfqpoint{2.260837in}{3.185343in}}{\pgfqpoint{2.271436in}{3.189733in}}{\pgfqpoint{2.279249in}{3.197547in}}%
\pgfpathcurveto{\pgfqpoint{2.287063in}{3.205360in}}{\pgfqpoint{2.291453in}{3.215959in}}{\pgfqpoint{2.291453in}{3.227010in}}%
\pgfpathcurveto{\pgfqpoint{2.291453in}{3.238060in}}{\pgfqpoint{2.287063in}{3.248659in}}{\pgfqpoint{2.279249in}{3.256472in}}%
\pgfpathcurveto{\pgfqpoint{2.271436in}{3.264286in}}{\pgfqpoint{2.260837in}{3.268676in}}{\pgfqpoint{2.249786in}{3.268676in}}%
\pgfpathcurveto{\pgfqpoint{2.238736in}{3.268676in}}{\pgfqpoint{2.228137in}{3.264286in}}{\pgfqpoint{2.220324in}{3.256472in}}%
\pgfpathcurveto{\pgfqpoint{2.212510in}{3.248659in}}{\pgfqpoint{2.208120in}{3.238060in}}{\pgfqpoint{2.208120in}{3.227010in}}%
\pgfpathcurveto{\pgfqpoint{2.208120in}{3.215959in}}{\pgfqpoint{2.212510in}{3.205360in}}{\pgfqpoint{2.220324in}{3.197547in}}%
\pgfpathcurveto{\pgfqpoint{2.228137in}{3.189733in}}{\pgfqpoint{2.238736in}{3.185343in}}{\pgfqpoint{2.249786in}{3.185343in}}%
\pgfpathclose%
\pgfusepath{stroke,fill}%
\end{pgfscope}%
\begin{pgfscope}%
\pgfpathrectangle{\pgfqpoint{0.648703in}{0.548769in}}{\pgfqpoint{5.201297in}{3.102590in}}%
\pgfusepath{clip}%
\pgfsetbuttcap%
\pgfsetroundjoin%
\definecolor{currentfill}{rgb}{1.000000,0.498039,0.054902}%
\pgfsetfillcolor{currentfill}%
\pgfsetlinewidth{1.003750pt}%
\definecolor{currentstroke}{rgb}{1.000000,0.498039,0.054902}%
\pgfsetstrokecolor{currentstroke}%
\pgfsetdash{}{0pt}%
\pgfpathmoveto{\pgfqpoint{2.103518in}{3.358719in}}%
\pgfpathcurveto{\pgfqpoint{2.114569in}{3.358719in}}{\pgfqpoint{2.125168in}{3.363109in}}{\pgfqpoint{2.132981in}{3.370923in}}%
\pgfpathcurveto{\pgfqpoint{2.140795in}{3.378737in}}{\pgfqpoint{2.145185in}{3.389336in}}{\pgfqpoint{2.145185in}{3.400386in}}%
\pgfpathcurveto{\pgfqpoint{2.145185in}{3.411436in}}{\pgfqpoint{2.140795in}{3.422035in}}{\pgfqpoint{2.132981in}{3.429849in}}%
\pgfpathcurveto{\pgfqpoint{2.125168in}{3.437662in}}{\pgfqpoint{2.114569in}{3.442053in}}{\pgfqpoint{2.103518in}{3.442053in}}%
\pgfpathcurveto{\pgfqpoint{2.092468in}{3.442053in}}{\pgfqpoint{2.081869in}{3.437662in}}{\pgfqpoint{2.074056in}{3.429849in}}%
\pgfpathcurveto{\pgfqpoint{2.066242in}{3.422035in}}{\pgfqpoint{2.061852in}{3.411436in}}{\pgfqpoint{2.061852in}{3.400386in}}%
\pgfpathcurveto{\pgfqpoint{2.061852in}{3.389336in}}{\pgfqpoint{2.066242in}{3.378737in}}{\pgfqpoint{2.074056in}{3.370923in}}%
\pgfpathcurveto{\pgfqpoint{2.081869in}{3.363109in}}{\pgfqpoint{2.092468in}{3.358719in}}{\pgfqpoint{2.103518in}{3.358719in}}%
\pgfpathclose%
\pgfusepath{stroke,fill}%
\end{pgfscope}%
\begin{pgfscope}%
\pgfpathrectangle{\pgfqpoint{0.648703in}{0.548769in}}{\pgfqpoint{5.201297in}{3.102590in}}%
\pgfusepath{clip}%
\pgfsetbuttcap%
\pgfsetroundjoin%
\definecolor{currentfill}{rgb}{1.000000,0.498039,0.054902}%
\pgfsetfillcolor{currentfill}%
\pgfsetlinewidth{1.003750pt}%
\definecolor{currentstroke}{rgb}{1.000000,0.498039,0.054902}%
\pgfsetstrokecolor{currentstroke}%
\pgfsetdash{}{0pt}%
\pgfpathmoveto{\pgfqpoint{2.621267in}{3.198029in}}%
\pgfpathcurveto{\pgfqpoint{2.632317in}{3.198029in}}{\pgfqpoint{2.642916in}{3.202419in}}{\pgfqpoint{2.650729in}{3.210233in}}%
\pgfpathcurveto{\pgfqpoint{2.658543in}{3.218046in}}{\pgfqpoint{2.662933in}{3.228646in}}{\pgfqpoint{2.662933in}{3.239696in}}%
\pgfpathcurveto{\pgfqpoint{2.662933in}{3.250746in}}{\pgfqpoint{2.658543in}{3.261345in}}{\pgfqpoint{2.650729in}{3.269158in}}%
\pgfpathcurveto{\pgfqpoint{2.642916in}{3.276972in}}{\pgfqpoint{2.632317in}{3.281362in}}{\pgfqpoint{2.621267in}{3.281362in}}%
\pgfpathcurveto{\pgfqpoint{2.610216in}{3.281362in}}{\pgfqpoint{2.599617in}{3.276972in}}{\pgfqpoint{2.591804in}{3.269158in}}%
\pgfpathcurveto{\pgfqpoint{2.583990in}{3.261345in}}{\pgfqpoint{2.579600in}{3.250746in}}{\pgfqpoint{2.579600in}{3.239696in}}%
\pgfpathcurveto{\pgfqpoint{2.579600in}{3.228646in}}{\pgfqpoint{2.583990in}{3.218046in}}{\pgfqpoint{2.591804in}{3.210233in}}%
\pgfpathcurveto{\pgfqpoint{2.599617in}{3.202419in}}{\pgfqpoint{2.610216in}{3.198029in}}{\pgfqpoint{2.621267in}{3.198029in}}%
\pgfpathclose%
\pgfusepath{stroke,fill}%
\end{pgfscope}%
\begin{pgfscope}%
\pgfpathrectangle{\pgfqpoint{0.648703in}{0.548769in}}{\pgfqpoint{5.201297in}{3.102590in}}%
\pgfusepath{clip}%
\pgfsetbuttcap%
\pgfsetroundjoin%
\definecolor{currentfill}{rgb}{0.839216,0.152941,0.156863}%
\pgfsetfillcolor{currentfill}%
\pgfsetlinewidth{1.003750pt}%
\definecolor{currentstroke}{rgb}{0.839216,0.152941,0.156863}%
\pgfsetstrokecolor{currentstroke}%
\pgfsetdash{}{0pt}%
\pgfpathmoveto{\pgfqpoint{1.542313in}{3.193800in}}%
\pgfpathcurveto{\pgfqpoint{1.553363in}{3.193800in}}{\pgfqpoint{1.563962in}{3.198191in}}{\pgfqpoint{1.571776in}{3.206004in}}%
\pgfpathcurveto{\pgfqpoint{1.579590in}{3.213818in}}{\pgfqpoint{1.583980in}{3.224417in}}{\pgfqpoint{1.583980in}{3.235467in}}%
\pgfpathcurveto{\pgfqpoint{1.583980in}{3.246517in}}{\pgfqpoint{1.579590in}{3.257116in}}{\pgfqpoint{1.571776in}{3.264930in}}%
\pgfpathcurveto{\pgfqpoint{1.563962in}{3.272743in}}{\pgfqpoint{1.553363in}{3.277134in}}{\pgfqpoint{1.542313in}{3.277134in}}%
\pgfpathcurveto{\pgfqpoint{1.531263in}{3.277134in}}{\pgfqpoint{1.520664in}{3.272743in}}{\pgfqpoint{1.512850in}{3.264930in}}%
\pgfpathcurveto{\pgfqpoint{1.505037in}{3.257116in}}{\pgfqpoint{1.500646in}{3.246517in}}{\pgfqpoint{1.500646in}{3.235467in}}%
\pgfpathcurveto{\pgfqpoint{1.500646in}{3.224417in}}{\pgfqpoint{1.505037in}{3.213818in}}{\pgfqpoint{1.512850in}{3.206004in}}%
\pgfpathcurveto{\pgfqpoint{1.520664in}{3.198191in}}{\pgfqpoint{1.531263in}{3.193800in}}{\pgfqpoint{1.542313in}{3.193800in}}%
\pgfpathclose%
\pgfusepath{stroke,fill}%
\end{pgfscope}%
\begin{pgfscope}%
\pgfpathrectangle{\pgfqpoint{0.648703in}{0.548769in}}{\pgfqpoint{5.201297in}{3.102590in}}%
\pgfusepath{clip}%
\pgfsetbuttcap%
\pgfsetroundjoin%
\definecolor{currentfill}{rgb}{1.000000,0.498039,0.054902}%
\pgfsetfillcolor{currentfill}%
\pgfsetlinewidth{1.003750pt}%
\definecolor{currentstroke}{rgb}{1.000000,0.498039,0.054902}%
\pgfsetstrokecolor{currentstroke}%
\pgfsetdash{}{0pt}%
\pgfpathmoveto{\pgfqpoint{1.883202in}{3.193800in}}%
\pgfpathcurveto{\pgfqpoint{1.894252in}{3.193800in}}{\pgfqpoint{1.904851in}{3.198191in}}{\pgfqpoint{1.912664in}{3.206004in}}%
\pgfpathcurveto{\pgfqpoint{1.920478in}{3.213818in}}{\pgfqpoint{1.924868in}{3.224417in}}{\pgfqpoint{1.924868in}{3.235467in}}%
\pgfpathcurveto{\pgfqpoint{1.924868in}{3.246517in}}{\pgfqpoint{1.920478in}{3.257116in}}{\pgfqpoint{1.912664in}{3.264930in}}%
\pgfpathcurveto{\pgfqpoint{1.904851in}{3.272743in}}{\pgfqpoint{1.894252in}{3.277134in}}{\pgfqpoint{1.883202in}{3.277134in}}%
\pgfpathcurveto{\pgfqpoint{1.872152in}{3.277134in}}{\pgfqpoint{1.861553in}{3.272743in}}{\pgfqpoint{1.853739in}{3.264930in}}%
\pgfpathcurveto{\pgfqpoint{1.845925in}{3.257116in}}{\pgfqpoint{1.841535in}{3.246517in}}{\pgfqpoint{1.841535in}{3.235467in}}%
\pgfpathcurveto{\pgfqpoint{1.841535in}{3.224417in}}{\pgfqpoint{1.845925in}{3.213818in}}{\pgfqpoint{1.853739in}{3.206004in}}%
\pgfpathcurveto{\pgfqpoint{1.861553in}{3.198191in}}{\pgfqpoint{1.872152in}{3.193800in}}{\pgfqpoint{1.883202in}{3.193800in}}%
\pgfpathclose%
\pgfusepath{stroke,fill}%
\end{pgfscope}%
\begin{pgfscope}%
\pgfpathrectangle{\pgfqpoint{0.648703in}{0.548769in}}{\pgfqpoint{5.201297in}{3.102590in}}%
\pgfusepath{clip}%
\pgfsetbuttcap%
\pgfsetroundjoin%
\definecolor{currentfill}{rgb}{1.000000,0.498039,0.054902}%
\pgfsetfillcolor{currentfill}%
\pgfsetlinewidth{1.003750pt}%
\definecolor{currentstroke}{rgb}{1.000000,0.498039,0.054902}%
\pgfsetstrokecolor{currentstroke}%
\pgfsetdash{}{0pt}%
\pgfpathmoveto{\pgfqpoint{2.731508in}{3.185343in}}%
\pgfpathcurveto{\pgfqpoint{2.742558in}{3.185343in}}{\pgfqpoint{2.753157in}{3.189733in}}{\pgfqpoint{2.760970in}{3.197547in}}%
\pgfpathcurveto{\pgfqpoint{2.768784in}{3.205360in}}{\pgfqpoint{2.773174in}{3.215959in}}{\pgfqpoint{2.773174in}{3.227010in}}%
\pgfpathcurveto{\pgfqpoint{2.773174in}{3.238060in}}{\pgfqpoint{2.768784in}{3.248659in}}{\pgfqpoint{2.760970in}{3.256472in}}%
\pgfpathcurveto{\pgfqpoint{2.753157in}{3.264286in}}{\pgfqpoint{2.742558in}{3.268676in}}{\pgfqpoint{2.731508in}{3.268676in}}%
\pgfpathcurveto{\pgfqpoint{2.720458in}{3.268676in}}{\pgfqpoint{2.709859in}{3.264286in}}{\pgfqpoint{2.702045in}{3.256472in}}%
\pgfpathcurveto{\pgfqpoint{2.694231in}{3.248659in}}{\pgfqpoint{2.689841in}{3.238060in}}{\pgfqpoint{2.689841in}{3.227010in}}%
\pgfpathcurveto{\pgfqpoint{2.689841in}{3.215959in}}{\pgfqpoint{2.694231in}{3.205360in}}{\pgfqpoint{2.702045in}{3.197547in}}%
\pgfpathcurveto{\pgfqpoint{2.709859in}{3.189733in}}{\pgfqpoint{2.720458in}{3.185343in}}{\pgfqpoint{2.731508in}{3.185343in}}%
\pgfpathclose%
\pgfusepath{stroke,fill}%
\end{pgfscope}%
\begin{pgfscope}%
\pgfpathrectangle{\pgfqpoint{0.648703in}{0.548769in}}{\pgfqpoint{5.201297in}{3.102590in}}%
\pgfusepath{clip}%
\pgfsetbuttcap%
\pgfsetroundjoin%
\definecolor{currentfill}{rgb}{1.000000,0.498039,0.054902}%
\pgfsetfillcolor{currentfill}%
\pgfsetlinewidth{1.003750pt}%
\definecolor{currentstroke}{rgb}{1.000000,0.498039,0.054902}%
\pgfsetstrokecolor{currentstroke}%
\pgfsetdash{}{0pt}%
\pgfpathmoveto{\pgfqpoint{1.577277in}{3.185343in}}%
\pgfpathcurveto{\pgfqpoint{1.588328in}{3.185343in}}{\pgfqpoint{1.598927in}{3.189733in}}{\pgfqpoint{1.606740in}{3.197547in}}%
\pgfpathcurveto{\pgfqpoint{1.614554in}{3.205360in}}{\pgfqpoint{1.618944in}{3.215959in}}{\pgfqpoint{1.618944in}{3.227010in}}%
\pgfpathcurveto{\pgfqpoint{1.618944in}{3.238060in}}{\pgfqpoint{1.614554in}{3.248659in}}{\pgfqpoint{1.606740in}{3.256472in}}%
\pgfpathcurveto{\pgfqpoint{1.598927in}{3.264286in}}{\pgfqpoint{1.588328in}{3.268676in}}{\pgfqpoint{1.577277in}{3.268676in}}%
\pgfpathcurveto{\pgfqpoint{1.566227in}{3.268676in}}{\pgfqpoint{1.555628in}{3.264286in}}{\pgfqpoint{1.547815in}{3.256472in}}%
\pgfpathcurveto{\pgfqpoint{1.540001in}{3.248659in}}{\pgfqpoint{1.535611in}{3.238060in}}{\pgfqpoint{1.535611in}{3.227010in}}%
\pgfpathcurveto{\pgfqpoint{1.535611in}{3.215959in}}{\pgfqpoint{1.540001in}{3.205360in}}{\pgfqpoint{1.547815in}{3.197547in}}%
\pgfpathcurveto{\pgfqpoint{1.555628in}{3.189733in}}{\pgfqpoint{1.566227in}{3.185343in}}{\pgfqpoint{1.577277in}{3.185343in}}%
\pgfpathclose%
\pgfusepath{stroke,fill}%
\end{pgfscope}%
\begin{pgfscope}%
\pgfpathrectangle{\pgfqpoint{0.648703in}{0.548769in}}{\pgfqpoint{5.201297in}{3.102590in}}%
\pgfusepath{clip}%
\pgfsetbuttcap%
\pgfsetroundjoin%
\definecolor{currentfill}{rgb}{1.000000,0.498039,0.054902}%
\pgfsetfillcolor{currentfill}%
\pgfsetlinewidth{1.003750pt}%
\definecolor{currentstroke}{rgb}{1.000000,0.498039,0.054902}%
\pgfsetstrokecolor{currentstroke}%
\pgfsetdash{}{0pt}%
\pgfpathmoveto{\pgfqpoint{2.355315in}{3.193800in}}%
\pgfpathcurveto{\pgfqpoint{2.366365in}{3.193800in}}{\pgfqpoint{2.376964in}{3.198191in}}{\pgfqpoint{2.384778in}{3.206004in}}%
\pgfpathcurveto{\pgfqpoint{2.392592in}{3.213818in}}{\pgfqpoint{2.396982in}{3.224417in}}{\pgfqpoint{2.396982in}{3.235467in}}%
\pgfpathcurveto{\pgfqpoint{2.396982in}{3.246517in}}{\pgfqpoint{2.392592in}{3.257116in}}{\pgfqpoint{2.384778in}{3.264930in}}%
\pgfpathcurveto{\pgfqpoint{2.376964in}{3.272743in}}{\pgfqpoint{2.366365in}{3.277134in}}{\pgfqpoint{2.355315in}{3.277134in}}%
\pgfpathcurveto{\pgfqpoint{2.344265in}{3.277134in}}{\pgfqpoint{2.333666in}{3.272743in}}{\pgfqpoint{2.325852in}{3.264930in}}%
\pgfpathcurveto{\pgfqpoint{2.318039in}{3.257116in}}{\pgfqpoint{2.313648in}{3.246517in}}{\pgfqpoint{2.313648in}{3.235467in}}%
\pgfpathcurveto{\pgfqpoint{2.313648in}{3.224417in}}{\pgfqpoint{2.318039in}{3.213818in}}{\pgfqpoint{2.325852in}{3.206004in}}%
\pgfpathcurveto{\pgfqpoint{2.333666in}{3.198191in}}{\pgfqpoint{2.344265in}{3.193800in}}{\pgfqpoint{2.355315in}{3.193800in}}%
\pgfpathclose%
\pgfusepath{stroke,fill}%
\end{pgfscope}%
\begin{pgfscope}%
\pgfpathrectangle{\pgfqpoint{0.648703in}{0.548769in}}{\pgfqpoint{5.201297in}{3.102590in}}%
\pgfusepath{clip}%
\pgfsetbuttcap%
\pgfsetroundjoin%
\definecolor{currentfill}{rgb}{0.839216,0.152941,0.156863}%
\pgfsetfillcolor{currentfill}%
\pgfsetlinewidth{1.003750pt}%
\definecolor{currentstroke}{rgb}{0.839216,0.152941,0.156863}%
\pgfsetstrokecolor{currentstroke}%
\pgfsetdash{}{0pt}%
\pgfpathmoveto{\pgfqpoint{2.385645in}{3.189572in}}%
\pgfpathcurveto{\pgfqpoint{2.396696in}{3.189572in}}{\pgfqpoint{2.407295in}{3.193962in}}{\pgfqpoint{2.415108in}{3.201775in}}%
\pgfpathcurveto{\pgfqpoint{2.422922in}{3.209589in}}{\pgfqpoint{2.427312in}{3.220188in}}{\pgfqpoint{2.427312in}{3.231238in}}%
\pgfpathcurveto{\pgfqpoint{2.427312in}{3.242288in}}{\pgfqpoint{2.422922in}{3.252887in}}{\pgfqpoint{2.415108in}{3.260701in}}%
\pgfpathcurveto{\pgfqpoint{2.407295in}{3.268515in}}{\pgfqpoint{2.396696in}{3.272905in}}{\pgfqpoint{2.385645in}{3.272905in}}%
\pgfpathcurveto{\pgfqpoint{2.374595in}{3.272905in}}{\pgfqpoint{2.363996in}{3.268515in}}{\pgfqpoint{2.356183in}{3.260701in}}%
\pgfpathcurveto{\pgfqpoint{2.348369in}{3.252887in}}{\pgfqpoint{2.343979in}{3.242288in}}{\pgfqpoint{2.343979in}{3.231238in}}%
\pgfpathcurveto{\pgfqpoint{2.343979in}{3.220188in}}{\pgfqpoint{2.348369in}{3.209589in}}{\pgfqpoint{2.356183in}{3.201775in}}%
\pgfpathcurveto{\pgfqpoint{2.363996in}{3.193962in}}{\pgfqpoint{2.374595in}{3.189572in}}{\pgfqpoint{2.385645in}{3.189572in}}%
\pgfpathclose%
\pgfusepath{stroke,fill}%
\end{pgfscope}%
\begin{pgfscope}%
\pgfpathrectangle{\pgfqpoint{0.648703in}{0.548769in}}{\pgfqpoint{5.201297in}{3.102590in}}%
\pgfusepath{clip}%
\pgfsetbuttcap%
\pgfsetroundjoin%
\definecolor{currentfill}{rgb}{1.000000,0.498039,0.054902}%
\pgfsetfillcolor{currentfill}%
\pgfsetlinewidth{1.003750pt}%
\definecolor{currentstroke}{rgb}{1.000000,0.498039,0.054902}%
\pgfsetstrokecolor{currentstroke}%
\pgfsetdash{}{0pt}%
\pgfpathmoveto{\pgfqpoint{4.651738in}{3.198029in}}%
\pgfpathcurveto{\pgfqpoint{4.662788in}{3.198029in}}{\pgfqpoint{4.673387in}{3.202419in}}{\pgfqpoint{4.681201in}{3.210233in}}%
\pgfpathcurveto{\pgfqpoint{4.689014in}{3.218046in}}{\pgfqpoint{4.693404in}{3.228646in}}{\pgfqpoint{4.693404in}{3.239696in}}%
\pgfpathcurveto{\pgfqpoint{4.693404in}{3.250746in}}{\pgfqpoint{4.689014in}{3.261345in}}{\pgfqpoint{4.681201in}{3.269158in}}%
\pgfpathcurveto{\pgfqpoint{4.673387in}{3.276972in}}{\pgfqpoint{4.662788in}{3.281362in}}{\pgfqpoint{4.651738in}{3.281362in}}%
\pgfpathcurveto{\pgfqpoint{4.640688in}{3.281362in}}{\pgfqpoint{4.630089in}{3.276972in}}{\pgfqpoint{4.622275in}{3.269158in}}%
\pgfpathcurveto{\pgfqpoint{4.614461in}{3.261345in}}{\pgfqpoint{4.610071in}{3.250746in}}{\pgfqpoint{4.610071in}{3.239696in}}%
\pgfpathcurveto{\pgfqpoint{4.610071in}{3.228646in}}{\pgfqpoint{4.614461in}{3.218046in}}{\pgfqpoint{4.622275in}{3.210233in}}%
\pgfpathcurveto{\pgfqpoint{4.630089in}{3.202419in}}{\pgfqpoint{4.640688in}{3.198029in}}{\pgfqpoint{4.651738in}{3.198029in}}%
\pgfpathclose%
\pgfusepath{stroke,fill}%
\end{pgfscope}%
\begin{pgfscope}%
\pgfpathrectangle{\pgfqpoint{0.648703in}{0.548769in}}{\pgfqpoint{5.201297in}{3.102590in}}%
\pgfusepath{clip}%
\pgfsetbuttcap%
\pgfsetroundjoin%
\definecolor{currentfill}{rgb}{0.121569,0.466667,0.705882}%
\pgfsetfillcolor{currentfill}%
\pgfsetlinewidth{1.003750pt}%
\definecolor{currentstroke}{rgb}{0.121569,0.466667,0.705882}%
\pgfsetstrokecolor{currentstroke}%
\pgfsetdash{}{0pt}%
\pgfpathmoveto{\pgfqpoint{2.046891in}{3.181114in}}%
\pgfpathcurveto{\pgfqpoint{2.057941in}{3.181114in}}{\pgfqpoint{2.068540in}{3.185504in}}{\pgfqpoint{2.076354in}{3.193318in}}%
\pgfpathcurveto{\pgfqpoint{2.084167in}{3.201132in}}{\pgfqpoint{2.088558in}{3.211731in}}{\pgfqpoint{2.088558in}{3.222781in}}%
\pgfpathcurveto{\pgfqpoint{2.088558in}{3.233831in}}{\pgfqpoint{2.084167in}{3.244430in}}{\pgfqpoint{2.076354in}{3.252244in}}%
\pgfpathcurveto{\pgfqpoint{2.068540in}{3.260057in}}{\pgfqpoint{2.057941in}{3.264448in}}{\pgfqpoint{2.046891in}{3.264448in}}%
\pgfpathcurveto{\pgfqpoint{2.035841in}{3.264448in}}{\pgfqpoint{2.025242in}{3.260057in}}{\pgfqpoint{2.017428in}{3.252244in}}%
\pgfpathcurveto{\pgfqpoint{2.009615in}{3.244430in}}{\pgfqpoint{2.005224in}{3.233831in}}{\pgfqpoint{2.005224in}{3.222781in}}%
\pgfpathcurveto{\pgfqpoint{2.005224in}{3.211731in}}{\pgfqpoint{2.009615in}{3.201132in}}{\pgfqpoint{2.017428in}{3.193318in}}%
\pgfpathcurveto{\pgfqpoint{2.025242in}{3.185504in}}{\pgfqpoint{2.035841in}{3.181114in}}{\pgfqpoint{2.046891in}{3.181114in}}%
\pgfpathclose%
\pgfusepath{stroke,fill}%
\end{pgfscope}%
\begin{pgfscope}%
\pgfpathrectangle{\pgfqpoint{0.648703in}{0.548769in}}{\pgfqpoint{5.201297in}{3.102590in}}%
\pgfusepath{clip}%
\pgfsetbuttcap%
\pgfsetroundjoin%
\definecolor{currentfill}{rgb}{0.121569,0.466667,0.705882}%
\pgfsetfillcolor{currentfill}%
\pgfsetlinewidth{1.003750pt}%
\definecolor{currentstroke}{rgb}{0.121569,0.466667,0.705882}%
\pgfsetstrokecolor{currentstroke}%
\pgfsetdash{}{0pt}%
\pgfpathmoveto{\pgfqpoint{1.042997in}{0.939909in}}%
\pgfpathcurveto{\pgfqpoint{1.054047in}{0.939909in}}{\pgfqpoint{1.064646in}{0.944299in}}{\pgfqpoint{1.072459in}{0.952112in}}%
\pgfpathcurveto{\pgfqpoint{1.080273in}{0.959926in}}{\pgfqpoint{1.084663in}{0.970525in}}{\pgfqpoint{1.084663in}{0.981575in}}%
\pgfpathcurveto{\pgfqpoint{1.084663in}{0.992625in}}{\pgfqpoint{1.080273in}{1.003224in}}{\pgfqpoint{1.072459in}{1.011038in}}%
\pgfpathcurveto{\pgfqpoint{1.064646in}{1.018852in}}{\pgfqpoint{1.054047in}{1.023242in}}{\pgfqpoint{1.042997in}{1.023242in}}%
\pgfpathcurveto{\pgfqpoint{1.031946in}{1.023242in}}{\pgfqpoint{1.021347in}{1.018852in}}{\pgfqpoint{1.013534in}{1.011038in}}%
\pgfpathcurveto{\pgfqpoint{1.005720in}{1.003224in}}{\pgfqpoint{1.001330in}{0.992625in}}{\pgfqpoint{1.001330in}{0.981575in}}%
\pgfpathcurveto{\pgfqpoint{1.001330in}{0.970525in}}{\pgfqpoint{1.005720in}{0.959926in}}{\pgfqpoint{1.013534in}{0.952112in}}%
\pgfpathcurveto{\pgfqpoint{1.021347in}{0.944299in}}{\pgfqpoint{1.031946in}{0.939909in}}{\pgfqpoint{1.042997in}{0.939909in}}%
\pgfpathclose%
\pgfusepath{stroke,fill}%
\end{pgfscope}%
\begin{pgfscope}%
\pgfpathrectangle{\pgfqpoint{0.648703in}{0.548769in}}{\pgfqpoint{5.201297in}{3.102590in}}%
\pgfusepath{clip}%
\pgfsetbuttcap%
\pgfsetroundjoin%
\definecolor{currentfill}{rgb}{0.839216,0.152941,0.156863}%
\pgfsetfillcolor{currentfill}%
\pgfsetlinewidth{1.003750pt}%
\definecolor{currentstroke}{rgb}{0.839216,0.152941,0.156863}%
\pgfsetstrokecolor{currentstroke}%
\pgfsetdash{}{0pt}%
\pgfpathmoveto{\pgfqpoint{1.932747in}{3.210715in}}%
\pgfpathcurveto{\pgfqpoint{1.943798in}{3.210715in}}{\pgfqpoint{1.954397in}{3.215105in}}{\pgfqpoint{1.962210in}{3.222919in}}%
\pgfpathcurveto{\pgfqpoint{1.970024in}{3.230733in}}{\pgfqpoint{1.974414in}{3.241332in}}{\pgfqpoint{1.974414in}{3.252382in}}%
\pgfpathcurveto{\pgfqpoint{1.974414in}{3.263432in}}{\pgfqpoint{1.970024in}{3.274031in}}{\pgfqpoint{1.962210in}{3.281844in}}%
\pgfpathcurveto{\pgfqpoint{1.954397in}{3.289658in}}{\pgfqpoint{1.943798in}{3.294048in}}{\pgfqpoint{1.932747in}{3.294048in}}%
\pgfpathcurveto{\pgfqpoint{1.921697in}{3.294048in}}{\pgfqpoint{1.911098in}{3.289658in}}{\pgfqpoint{1.903285in}{3.281844in}}%
\pgfpathcurveto{\pgfqpoint{1.895471in}{3.274031in}}{\pgfqpoint{1.891081in}{3.263432in}}{\pgfqpoint{1.891081in}{3.252382in}}%
\pgfpathcurveto{\pgfqpoint{1.891081in}{3.241332in}}{\pgfqpoint{1.895471in}{3.230733in}}{\pgfqpoint{1.903285in}{3.222919in}}%
\pgfpathcurveto{\pgfqpoint{1.911098in}{3.215105in}}{\pgfqpoint{1.921697in}{3.210715in}}{\pgfqpoint{1.932747in}{3.210715in}}%
\pgfpathclose%
\pgfusepath{stroke,fill}%
\end{pgfscope}%
\begin{pgfscope}%
\pgfpathrectangle{\pgfqpoint{0.648703in}{0.548769in}}{\pgfqpoint{5.201297in}{3.102590in}}%
\pgfusepath{clip}%
\pgfsetbuttcap%
\pgfsetroundjoin%
\definecolor{currentfill}{rgb}{1.000000,0.498039,0.054902}%
\pgfsetfillcolor{currentfill}%
\pgfsetlinewidth{1.003750pt}%
\definecolor{currentstroke}{rgb}{1.000000,0.498039,0.054902}%
\pgfsetstrokecolor{currentstroke}%
\pgfsetdash{}{0pt}%
\pgfpathmoveto{\pgfqpoint{2.192758in}{3.202258in}}%
\pgfpathcurveto{\pgfqpoint{2.203808in}{3.202258in}}{\pgfqpoint{2.214407in}{3.206648in}}{\pgfqpoint{2.222221in}{3.214462in}}%
\pgfpathcurveto{\pgfqpoint{2.230035in}{3.222275in}}{\pgfqpoint{2.234425in}{3.232874in}}{\pgfqpoint{2.234425in}{3.243924in}}%
\pgfpathcurveto{\pgfqpoint{2.234425in}{3.254974in}}{\pgfqpoint{2.230035in}{3.265573in}}{\pgfqpoint{2.222221in}{3.273387in}}%
\pgfpathcurveto{\pgfqpoint{2.214407in}{3.281201in}}{\pgfqpoint{2.203808in}{3.285591in}}{\pgfqpoint{2.192758in}{3.285591in}}%
\pgfpathcurveto{\pgfqpoint{2.181708in}{3.285591in}}{\pgfqpoint{2.171109in}{3.281201in}}{\pgfqpoint{2.163296in}{3.273387in}}%
\pgfpathcurveto{\pgfqpoint{2.155482in}{3.265573in}}{\pgfqpoint{2.151092in}{3.254974in}}{\pgfqpoint{2.151092in}{3.243924in}}%
\pgfpathcurveto{\pgfqpoint{2.151092in}{3.232874in}}{\pgfqpoint{2.155482in}{3.222275in}}{\pgfqpoint{2.163296in}{3.214462in}}%
\pgfpathcurveto{\pgfqpoint{2.171109in}{3.206648in}}{\pgfqpoint{2.181708in}{3.202258in}}{\pgfqpoint{2.192758in}{3.202258in}}%
\pgfpathclose%
\pgfusepath{stroke,fill}%
\end{pgfscope}%
\begin{pgfscope}%
\pgfpathrectangle{\pgfqpoint{0.648703in}{0.548769in}}{\pgfqpoint{5.201297in}{3.102590in}}%
\pgfusepath{clip}%
\pgfsetbuttcap%
\pgfsetroundjoin%
\definecolor{currentfill}{rgb}{1.000000,0.498039,0.054902}%
\pgfsetfillcolor{currentfill}%
\pgfsetlinewidth{1.003750pt}%
\definecolor{currentstroke}{rgb}{1.000000,0.498039,0.054902}%
\pgfsetstrokecolor{currentstroke}%
\pgfsetdash{}{0pt}%
\pgfpathmoveto{\pgfqpoint{1.270952in}{3.206486in}}%
\pgfpathcurveto{\pgfqpoint{1.282003in}{3.206486in}}{\pgfqpoint{1.292602in}{3.210877in}}{\pgfqpoint{1.300415in}{3.218690in}}%
\pgfpathcurveto{\pgfqpoint{1.308229in}{3.226504in}}{\pgfqpoint{1.312619in}{3.237103in}}{\pgfqpoint{1.312619in}{3.248153in}}%
\pgfpathcurveto{\pgfqpoint{1.312619in}{3.259203in}}{\pgfqpoint{1.308229in}{3.269802in}}{\pgfqpoint{1.300415in}{3.277616in}}%
\pgfpathcurveto{\pgfqpoint{1.292602in}{3.285429in}}{\pgfqpoint{1.282003in}{3.289820in}}{\pgfqpoint{1.270952in}{3.289820in}}%
\pgfpathcurveto{\pgfqpoint{1.259902in}{3.289820in}}{\pgfqpoint{1.249303in}{3.285429in}}{\pgfqpoint{1.241490in}{3.277616in}}%
\pgfpathcurveto{\pgfqpoint{1.233676in}{3.269802in}}{\pgfqpoint{1.229286in}{3.259203in}}{\pgfqpoint{1.229286in}{3.248153in}}%
\pgfpathcurveto{\pgfqpoint{1.229286in}{3.237103in}}{\pgfqpoint{1.233676in}{3.226504in}}{\pgfqpoint{1.241490in}{3.218690in}}%
\pgfpathcurveto{\pgfqpoint{1.249303in}{3.210877in}}{\pgfqpoint{1.259902in}{3.206486in}}{\pgfqpoint{1.270952in}{3.206486in}}%
\pgfpathclose%
\pgfusepath{stroke,fill}%
\end{pgfscope}%
\begin{pgfscope}%
\pgfpathrectangle{\pgfqpoint{0.648703in}{0.548769in}}{\pgfqpoint{5.201297in}{3.102590in}}%
\pgfusepath{clip}%
\pgfsetbuttcap%
\pgfsetroundjoin%
\definecolor{currentfill}{rgb}{1.000000,0.498039,0.054902}%
\pgfsetfillcolor{currentfill}%
\pgfsetlinewidth{1.003750pt}%
\definecolor{currentstroke}{rgb}{1.000000,0.498039,0.054902}%
\pgfsetstrokecolor{currentstroke}%
\pgfsetdash{}{0pt}%
\pgfpathmoveto{\pgfqpoint{1.231999in}{3.198029in}}%
\pgfpathcurveto{\pgfqpoint{1.243049in}{3.198029in}}{\pgfqpoint{1.253648in}{3.202419in}}{\pgfqpoint{1.261462in}{3.210233in}}%
\pgfpathcurveto{\pgfqpoint{1.269275in}{3.218046in}}{\pgfqpoint{1.273665in}{3.228646in}}{\pgfqpoint{1.273665in}{3.239696in}}%
\pgfpathcurveto{\pgfqpoint{1.273665in}{3.250746in}}{\pgfqpoint{1.269275in}{3.261345in}}{\pgfqpoint{1.261462in}{3.269158in}}%
\pgfpathcurveto{\pgfqpoint{1.253648in}{3.276972in}}{\pgfqpoint{1.243049in}{3.281362in}}{\pgfqpoint{1.231999in}{3.281362in}}%
\pgfpathcurveto{\pgfqpoint{1.220949in}{3.281362in}}{\pgfqpoint{1.210350in}{3.276972in}}{\pgfqpoint{1.202536in}{3.269158in}}%
\pgfpathcurveto{\pgfqpoint{1.194722in}{3.261345in}}{\pgfqpoint{1.190332in}{3.250746in}}{\pgfqpoint{1.190332in}{3.239696in}}%
\pgfpathcurveto{\pgfqpoint{1.190332in}{3.228646in}}{\pgfqpoint{1.194722in}{3.218046in}}{\pgfqpoint{1.202536in}{3.210233in}}%
\pgfpathcurveto{\pgfqpoint{1.210350in}{3.202419in}}{\pgfqpoint{1.220949in}{3.198029in}}{\pgfqpoint{1.231999in}{3.198029in}}%
\pgfpathclose%
\pgfusepath{stroke,fill}%
\end{pgfscope}%
\begin{pgfscope}%
\pgfpathrectangle{\pgfqpoint{0.648703in}{0.548769in}}{\pgfqpoint{5.201297in}{3.102590in}}%
\pgfusepath{clip}%
\pgfsetbuttcap%
\pgfsetroundjoin%
\definecolor{currentfill}{rgb}{1.000000,0.498039,0.054902}%
\pgfsetfillcolor{currentfill}%
\pgfsetlinewidth{1.003750pt}%
\definecolor{currentstroke}{rgb}{1.000000,0.498039,0.054902}%
\pgfsetstrokecolor{currentstroke}%
\pgfsetdash{}{0pt}%
\pgfpathmoveto{\pgfqpoint{2.407108in}{3.202258in}}%
\pgfpathcurveto{\pgfqpoint{2.418158in}{3.202258in}}{\pgfqpoint{2.428757in}{3.206648in}}{\pgfqpoint{2.436571in}{3.214462in}}%
\pgfpathcurveto{\pgfqpoint{2.444385in}{3.222275in}}{\pgfqpoint{2.448775in}{3.232874in}}{\pgfqpoint{2.448775in}{3.243924in}}%
\pgfpathcurveto{\pgfqpoint{2.448775in}{3.254974in}}{\pgfqpoint{2.444385in}{3.265573in}}{\pgfqpoint{2.436571in}{3.273387in}}%
\pgfpathcurveto{\pgfqpoint{2.428757in}{3.281201in}}{\pgfqpoint{2.418158in}{3.285591in}}{\pgfqpoint{2.407108in}{3.285591in}}%
\pgfpathcurveto{\pgfqpoint{2.396058in}{3.285591in}}{\pgfqpoint{2.385459in}{3.281201in}}{\pgfqpoint{2.377645in}{3.273387in}}%
\pgfpathcurveto{\pgfqpoint{2.369832in}{3.265573in}}{\pgfqpoint{2.365442in}{3.254974in}}{\pgfqpoint{2.365442in}{3.243924in}}%
\pgfpathcurveto{\pgfqpoint{2.365442in}{3.232874in}}{\pgfqpoint{2.369832in}{3.222275in}}{\pgfqpoint{2.377645in}{3.214462in}}%
\pgfpathcurveto{\pgfqpoint{2.385459in}{3.206648in}}{\pgfqpoint{2.396058in}{3.202258in}}{\pgfqpoint{2.407108in}{3.202258in}}%
\pgfpathclose%
\pgfusepath{stroke,fill}%
\end{pgfscope}%
\begin{pgfscope}%
\pgfpathrectangle{\pgfqpoint{0.648703in}{0.548769in}}{\pgfqpoint{5.201297in}{3.102590in}}%
\pgfusepath{clip}%
\pgfsetbuttcap%
\pgfsetroundjoin%
\definecolor{currentfill}{rgb}{1.000000,0.498039,0.054902}%
\pgfsetfillcolor{currentfill}%
\pgfsetlinewidth{1.003750pt}%
\definecolor{currentstroke}{rgb}{1.000000,0.498039,0.054902}%
\pgfsetstrokecolor{currentstroke}%
\pgfsetdash{}{0pt}%
\pgfpathmoveto{\pgfqpoint{2.139241in}{3.189572in}}%
\pgfpathcurveto{\pgfqpoint{2.150291in}{3.189572in}}{\pgfqpoint{2.160890in}{3.193962in}}{\pgfqpoint{2.168703in}{3.201775in}}%
\pgfpathcurveto{\pgfqpoint{2.176517in}{3.209589in}}{\pgfqpoint{2.180907in}{3.220188in}}{\pgfqpoint{2.180907in}{3.231238in}}%
\pgfpathcurveto{\pgfqpoint{2.180907in}{3.242288in}}{\pgfqpoint{2.176517in}{3.252887in}}{\pgfqpoint{2.168703in}{3.260701in}}%
\pgfpathcurveto{\pgfqpoint{2.160890in}{3.268515in}}{\pgfqpoint{2.150291in}{3.272905in}}{\pgfqpoint{2.139241in}{3.272905in}}%
\pgfpathcurveto{\pgfqpoint{2.128190in}{3.272905in}}{\pgfqpoint{2.117591in}{3.268515in}}{\pgfqpoint{2.109778in}{3.260701in}}%
\pgfpathcurveto{\pgfqpoint{2.101964in}{3.252887in}}{\pgfqpoint{2.097574in}{3.242288in}}{\pgfqpoint{2.097574in}{3.231238in}}%
\pgfpathcurveto{\pgfqpoint{2.097574in}{3.220188in}}{\pgfqpoint{2.101964in}{3.209589in}}{\pgfqpoint{2.109778in}{3.201775in}}%
\pgfpathcurveto{\pgfqpoint{2.117591in}{3.193962in}}{\pgfqpoint{2.128190in}{3.189572in}}{\pgfqpoint{2.139241in}{3.189572in}}%
\pgfpathclose%
\pgfusepath{stroke,fill}%
\end{pgfscope}%
\begin{pgfscope}%
\pgfpathrectangle{\pgfqpoint{0.648703in}{0.548769in}}{\pgfqpoint{5.201297in}{3.102590in}}%
\pgfusepath{clip}%
\pgfsetbuttcap%
\pgfsetroundjoin%
\definecolor{currentfill}{rgb}{0.121569,0.466667,0.705882}%
\pgfsetfillcolor{currentfill}%
\pgfsetlinewidth{1.003750pt}%
\definecolor{currentstroke}{rgb}{0.121569,0.466667,0.705882}%
\pgfsetstrokecolor{currentstroke}%
\pgfsetdash{}{0pt}%
\pgfpathmoveto{\pgfqpoint{3.128762in}{3.181114in}}%
\pgfpathcurveto{\pgfqpoint{3.139813in}{3.181114in}}{\pgfqpoint{3.150412in}{3.185504in}}{\pgfqpoint{3.158225in}{3.193318in}}%
\pgfpathcurveto{\pgfqpoint{3.166039in}{3.201132in}}{\pgfqpoint{3.170429in}{3.211731in}}{\pgfqpoint{3.170429in}{3.222781in}}%
\pgfpathcurveto{\pgfqpoint{3.170429in}{3.233831in}}{\pgfqpoint{3.166039in}{3.244430in}}{\pgfqpoint{3.158225in}{3.252244in}}%
\pgfpathcurveto{\pgfqpoint{3.150412in}{3.260057in}}{\pgfqpoint{3.139813in}{3.264448in}}{\pgfqpoint{3.128762in}{3.264448in}}%
\pgfpathcurveto{\pgfqpoint{3.117712in}{3.264448in}}{\pgfqpoint{3.107113in}{3.260057in}}{\pgfqpoint{3.099300in}{3.252244in}}%
\pgfpathcurveto{\pgfqpoint{3.091486in}{3.244430in}}{\pgfqpoint{3.087096in}{3.233831in}}{\pgfqpoint{3.087096in}{3.222781in}}%
\pgfpathcurveto{\pgfqpoint{3.087096in}{3.211731in}}{\pgfqpoint{3.091486in}{3.201132in}}{\pgfqpoint{3.099300in}{3.193318in}}%
\pgfpathcurveto{\pgfqpoint{3.107113in}{3.185504in}}{\pgfqpoint{3.117712in}{3.181114in}}{\pgfqpoint{3.128762in}{3.181114in}}%
\pgfpathclose%
\pgfusepath{stroke,fill}%
\end{pgfscope}%
\begin{pgfscope}%
\pgfpathrectangle{\pgfqpoint{0.648703in}{0.548769in}}{\pgfqpoint{5.201297in}{3.102590in}}%
\pgfusepath{clip}%
\pgfsetbuttcap%
\pgfsetroundjoin%
\definecolor{currentfill}{rgb}{1.000000,0.498039,0.054902}%
\pgfsetfillcolor{currentfill}%
\pgfsetlinewidth{1.003750pt}%
\definecolor{currentstroke}{rgb}{1.000000,0.498039,0.054902}%
\pgfsetstrokecolor{currentstroke}%
\pgfsetdash{}{0pt}%
\pgfpathmoveto{\pgfqpoint{2.306510in}{3.244545in}}%
\pgfpathcurveto{\pgfqpoint{2.317560in}{3.244545in}}{\pgfqpoint{2.328159in}{3.248935in}}{\pgfqpoint{2.335973in}{3.256748in}}%
\pgfpathcurveto{\pgfqpoint{2.343786in}{3.264562in}}{\pgfqpoint{2.348176in}{3.275161in}}{\pgfqpoint{2.348176in}{3.286211in}}%
\pgfpathcurveto{\pgfqpoint{2.348176in}{3.297261in}}{\pgfqpoint{2.343786in}{3.307860in}}{\pgfqpoint{2.335973in}{3.315674in}}%
\pgfpathcurveto{\pgfqpoint{2.328159in}{3.323488in}}{\pgfqpoint{2.317560in}{3.327878in}}{\pgfqpoint{2.306510in}{3.327878in}}%
\pgfpathcurveto{\pgfqpoint{2.295460in}{3.327878in}}{\pgfqpoint{2.284861in}{3.323488in}}{\pgfqpoint{2.277047in}{3.315674in}}%
\pgfpathcurveto{\pgfqpoint{2.269233in}{3.307860in}}{\pgfqpoint{2.264843in}{3.297261in}}{\pgfqpoint{2.264843in}{3.286211in}}%
\pgfpathcurveto{\pgfqpoint{2.264843in}{3.275161in}}{\pgfqpoint{2.269233in}{3.264562in}}{\pgfqpoint{2.277047in}{3.256748in}}%
\pgfpathcurveto{\pgfqpoint{2.284861in}{3.248935in}}{\pgfqpoint{2.295460in}{3.244545in}}{\pgfqpoint{2.306510in}{3.244545in}}%
\pgfpathclose%
\pgfusepath{stroke,fill}%
\end{pgfscope}%
\begin{pgfscope}%
\pgfpathrectangle{\pgfqpoint{0.648703in}{0.548769in}}{\pgfqpoint{5.201297in}{3.102590in}}%
\pgfusepath{clip}%
\pgfsetbuttcap%
\pgfsetroundjoin%
\definecolor{currentfill}{rgb}{1.000000,0.498039,0.054902}%
\pgfsetfillcolor{currentfill}%
\pgfsetlinewidth{1.003750pt}%
\definecolor{currentstroke}{rgb}{1.000000,0.498039,0.054902}%
\pgfsetstrokecolor{currentstroke}%
\pgfsetdash{}{0pt}%
\pgfpathmoveto{\pgfqpoint{2.422996in}{3.189572in}}%
\pgfpathcurveto{\pgfqpoint{2.434046in}{3.189572in}}{\pgfqpoint{2.444646in}{3.193962in}}{\pgfqpoint{2.452459in}{3.201775in}}%
\pgfpathcurveto{\pgfqpoint{2.460273in}{3.209589in}}{\pgfqpoint{2.464663in}{3.220188in}}{\pgfqpoint{2.464663in}{3.231238in}}%
\pgfpathcurveto{\pgfqpoint{2.464663in}{3.242288in}}{\pgfqpoint{2.460273in}{3.252887in}}{\pgfqpoint{2.452459in}{3.260701in}}%
\pgfpathcurveto{\pgfqpoint{2.444646in}{3.268515in}}{\pgfqpoint{2.434046in}{3.272905in}}{\pgfqpoint{2.422996in}{3.272905in}}%
\pgfpathcurveto{\pgfqpoint{2.411946in}{3.272905in}}{\pgfqpoint{2.401347in}{3.268515in}}{\pgfqpoint{2.393534in}{3.260701in}}%
\pgfpathcurveto{\pgfqpoint{2.385720in}{3.252887in}}{\pgfqpoint{2.381330in}{3.242288in}}{\pgfqpoint{2.381330in}{3.231238in}}%
\pgfpathcurveto{\pgfqpoint{2.381330in}{3.220188in}}{\pgfqpoint{2.385720in}{3.209589in}}{\pgfqpoint{2.393534in}{3.201775in}}%
\pgfpathcurveto{\pgfqpoint{2.401347in}{3.193962in}}{\pgfqpoint{2.411946in}{3.189572in}}{\pgfqpoint{2.422996in}{3.189572in}}%
\pgfpathclose%
\pgfusepath{stroke,fill}%
\end{pgfscope}%
\begin{pgfscope}%
\pgfpathrectangle{\pgfqpoint{0.648703in}{0.548769in}}{\pgfqpoint{5.201297in}{3.102590in}}%
\pgfusepath{clip}%
\pgfsetbuttcap%
\pgfsetroundjoin%
\definecolor{currentfill}{rgb}{1.000000,0.498039,0.054902}%
\pgfsetfillcolor{currentfill}%
\pgfsetlinewidth{1.003750pt}%
\definecolor{currentstroke}{rgb}{1.000000,0.498039,0.054902}%
\pgfsetstrokecolor{currentstroke}%
\pgfsetdash{}{0pt}%
\pgfpathmoveto{\pgfqpoint{1.626327in}{3.210715in}}%
\pgfpathcurveto{\pgfqpoint{1.637377in}{3.210715in}}{\pgfqpoint{1.647976in}{3.215105in}}{\pgfqpoint{1.655789in}{3.222919in}}%
\pgfpathcurveto{\pgfqpoint{1.663603in}{3.230733in}}{\pgfqpoint{1.667993in}{3.241332in}}{\pgfqpoint{1.667993in}{3.252382in}}%
\pgfpathcurveto{\pgfqpoint{1.667993in}{3.263432in}}{\pgfqpoint{1.663603in}{3.274031in}}{\pgfqpoint{1.655789in}{3.281844in}}%
\pgfpathcurveto{\pgfqpoint{1.647976in}{3.289658in}}{\pgfqpoint{1.637377in}{3.294048in}}{\pgfqpoint{1.626327in}{3.294048in}}%
\pgfpathcurveto{\pgfqpoint{1.615277in}{3.294048in}}{\pgfqpoint{1.604678in}{3.289658in}}{\pgfqpoint{1.596864in}{3.281844in}}%
\pgfpathcurveto{\pgfqpoint{1.589050in}{3.274031in}}{\pgfqpoint{1.584660in}{3.263432in}}{\pgfqpoint{1.584660in}{3.252382in}}%
\pgfpathcurveto{\pgfqpoint{1.584660in}{3.241332in}}{\pgfqpoint{1.589050in}{3.230733in}}{\pgfqpoint{1.596864in}{3.222919in}}%
\pgfpathcurveto{\pgfqpoint{1.604678in}{3.215105in}}{\pgfqpoint{1.615277in}{3.210715in}}{\pgfqpoint{1.626327in}{3.210715in}}%
\pgfpathclose%
\pgfusepath{stroke,fill}%
\end{pgfscope}%
\begin{pgfscope}%
\pgfpathrectangle{\pgfqpoint{0.648703in}{0.548769in}}{\pgfqpoint{5.201297in}{3.102590in}}%
\pgfusepath{clip}%
\pgfsetbuttcap%
\pgfsetroundjoin%
\definecolor{currentfill}{rgb}{1.000000,0.498039,0.054902}%
\pgfsetfillcolor{currentfill}%
\pgfsetlinewidth{1.003750pt}%
\definecolor{currentstroke}{rgb}{1.000000,0.498039,0.054902}%
\pgfsetstrokecolor{currentstroke}%
\pgfsetdash{}{0pt}%
\pgfpathmoveto{\pgfqpoint{1.963592in}{3.189572in}}%
\pgfpathcurveto{\pgfqpoint{1.974642in}{3.189572in}}{\pgfqpoint{1.985241in}{3.193962in}}{\pgfqpoint{1.993054in}{3.201775in}}%
\pgfpathcurveto{\pgfqpoint{2.000868in}{3.209589in}}{\pgfqpoint{2.005258in}{3.220188in}}{\pgfqpoint{2.005258in}{3.231238in}}%
\pgfpathcurveto{\pgfqpoint{2.005258in}{3.242288in}}{\pgfqpoint{2.000868in}{3.252887in}}{\pgfqpoint{1.993054in}{3.260701in}}%
\pgfpathcurveto{\pgfqpoint{1.985241in}{3.268515in}}{\pgfqpoint{1.974642in}{3.272905in}}{\pgfqpoint{1.963592in}{3.272905in}}%
\pgfpathcurveto{\pgfqpoint{1.952542in}{3.272905in}}{\pgfqpoint{1.941942in}{3.268515in}}{\pgfqpoint{1.934129in}{3.260701in}}%
\pgfpathcurveto{\pgfqpoint{1.926315in}{3.252887in}}{\pgfqpoint{1.921925in}{3.242288in}}{\pgfqpoint{1.921925in}{3.231238in}}%
\pgfpathcurveto{\pgfqpoint{1.921925in}{3.220188in}}{\pgfqpoint{1.926315in}{3.209589in}}{\pgfqpoint{1.934129in}{3.201775in}}%
\pgfpathcurveto{\pgfqpoint{1.941942in}{3.193962in}}{\pgfqpoint{1.952542in}{3.189572in}}{\pgfqpoint{1.963592in}{3.189572in}}%
\pgfpathclose%
\pgfusepath{stroke,fill}%
\end{pgfscope}%
\begin{pgfscope}%
\pgfpathrectangle{\pgfqpoint{0.648703in}{0.548769in}}{\pgfqpoint{5.201297in}{3.102590in}}%
\pgfusepath{clip}%
\pgfsetbuttcap%
\pgfsetroundjoin%
\definecolor{currentfill}{rgb}{0.839216,0.152941,0.156863}%
\pgfsetfillcolor{currentfill}%
\pgfsetlinewidth{1.003750pt}%
\definecolor{currentstroke}{rgb}{0.839216,0.152941,0.156863}%
\pgfsetstrokecolor{currentstroke}%
\pgfsetdash{}{0pt}%
\pgfpathmoveto{\pgfqpoint{2.594821in}{3.181114in}}%
\pgfpathcurveto{\pgfqpoint{2.605871in}{3.181114in}}{\pgfqpoint{2.616470in}{3.185504in}}{\pgfqpoint{2.624284in}{3.193318in}}%
\pgfpathcurveto{\pgfqpoint{2.632098in}{3.201132in}}{\pgfqpoint{2.636488in}{3.211731in}}{\pgfqpoint{2.636488in}{3.222781in}}%
\pgfpathcurveto{\pgfqpoint{2.636488in}{3.233831in}}{\pgfqpoint{2.632098in}{3.244430in}}{\pgfqpoint{2.624284in}{3.252244in}}%
\pgfpathcurveto{\pgfqpoint{2.616470in}{3.260057in}}{\pgfqpoint{2.605871in}{3.264448in}}{\pgfqpoint{2.594821in}{3.264448in}}%
\pgfpathcurveto{\pgfqpoint{2.583771in}{3.264448in}}{\pgfqpoint{2.573172in}{3.260057in}}{\pgfqpoint{2.565358in}{3.252244in}}%
\pgfpathcurveto{\pgfqpoint{2.557545in}{3.244430in}}{\pgfqpoint{2.553155in}{3.233831in}}{\pgfqpoint{2.553155in}{3.222781in}}%
\pgfpathcurveto{\pgfqpoint{2.553155in}{3.211731in}}{\pgfqpoint{2.557545in}{3.201132in}}{\pgfqpoint{2.565358in}{3.193318in}}%
\pgfpathcurveto{\pgfqpoint{2.573172in}{3.185504in}}{\pgfqpoint{2.583771in}{3.181114in}}{\pgfqpoint{2.594821in}{3.181114in}}%
\pgfpathclose%
\pgfusepath{stroke,fill}%
\end{pgfscope}%
\begin{pgfscope}%
\pgfpathrectangle{\pgfqpoint{0.648703in}{0.548769in}}{\pgfqpoint{5.201297in}{3.102590in}}%
\pgfusepath{clip}%
\pgfsetbuttcap%
\pgfsetroundjoin%
\definecolor{currentfill}{rgb}{1.000000,0.498039,0.054902}%
\pgfsetfillcolor{currentfill}%
\pgfsetlinewidth{1.003750pt}%
\definecolor{currentstroke}{rgb}{1.000000,0.498039,0.054902}%
\pgfsetstrokecolor{currentstroke}%
\pgfsetdash{}{0pt}%
\pgfpathmoveto{\pgfqpoint{1.492175in}{3.202258in}}%
\pgfpathcurveto{\pgfqpoint{1.503225in}{3.202258in}}{\pgfqpoint{1.513824in}{3.206648in}}{\pgfqpoint{1.521638in}{3.214462in}}%
\pgfpathcurveto{\pgfqpoint{1.529451in}{3.222275in}}{\pgfqpoint{1.533842in}{3.232874in}}{\pgfqpoint{1.533842in}{3.243924in}}%
\pgfpathcurveto{\pgfqpoint{1.533842in}{3.254974in}}{\pgfqpoint{1.529451in}{3.265573in}}{\pgfqpoint{1.521638in}{3.273387in}}%
\pgfpathcurveto{\pgfqpoint{1.513824in}{3.281201in}}{\pgfqpoint{1.503225in}{3.285591in}}{\pgfqpoint{1.492175in}{3.285591in}}%
\pgfpathcurveto{\pgfqpoint{1.481125in}{3.285591in}}{\pgfqpoint{1.470526in}{3.281201in}}{\pgfqpoint{1.462712in}{3.273387in}}%
\pgfpathcurveto{\pgfqpoint{1.454899in}{3.265573in}}{\pgfqpoint{1.450508in}{3.254974in}}{\pgfqpoint{1.450508in}{3.243924in}}%
\pgfpathcurveto{\pgfqpoint{1.450508in}{3.232874in}}{\pgfqpoint{1.454899in}{3.222275in}}{\pgfqpoint{1.462712in}{3.214462in}}%
\pgfpathcurveto{\pgfqpoint{1.470526in}{3.206648in}}{\pgfqpoint{1.481125in}{3.202258in}}{\pgfqpoint{1.492175in}{3.202258in}}%
\pgfpathclose%
\pgfusepath{stroke,fill}%
\end{pgfscope}%
\begin{pgfscope}%
\pgfpathrectangle{\pgfqpoint{0.648703in}{0.548769in}}{\pgfqpoint{5.201297in}{3.102590in}}%
\pgfusepath{clip}%
\pgfsetbuttcap%
\pgfsetroundjoin%
\definecolor{currentfill}{rgb}{0.839216,0.152941,0.156863}%
\pgfsetfillcolor{currentfill}%
\pgfsetlinewidth{1.003750pt}%
\definecolor{currentstroke}{rgb}{0.839216,0.152941,0.156863}%
\pgfsetstrokecolor{currentstroke}%
\pgfsetdash{}{0pt}%
\pgfpathmoveto{\pgfqpoint{1.770783in}{3.265688in}}%
\pgfpathcurveto{\pgfqpoint{1.781833in}{3.265688in}}{\pgfqpoint{1.792432in}{3.270078in}}{\pgfqpoint{1.800246in}{3.277892in}}%
\pgfpathcurveto{\pgfqpoint{1.808059in}{3.285706in}}{\pgfqpoint{1.812450in}{3.296305in}}{\pgfqpoint{1.812450in}{3.307355in}}%
\pgfpathcurveto{\pgfqpoint{1.812450in}{3.318405in}}{\pgfqpoint{1.808059in}{3.329004in}}{\pgfqpoint{1.800246in}{3.336817in}}%
\pgfpathcurveto{\pgfqpoint{1.792432in}{3.344631in}}{\pgfqpoint{1.781833in}{3.349021in}}{\pgfqpoint{1.770783in}{3.349021in}}%
\pgfpathcurveto{\pgfqpoint{1.759733in}{3.349021in}}{\pgfqpoint{1.749134in}{3.344631in}}{\pgfqpoint{1.741320in}{3.336817in}}%
\pgfpathcurveto{\pgfqpoint{1.733507in}{3.329004in}}{\pgfqpoint{1.729116in}{3.318405in}}{\pgfqpoint{1.729116in}{3.307355in}}%
\pgfpathcurveto{\pgfqpoint{1.729116in}{3.296305in}}{\pgfqpoint{1.733507in}{3.285706in}}{\pgfqpoint{1.741320in}{3.277892in}}%
\pgfpathcurveto{\pgfqpoint{1.749134in}{3.270078in}}{\pgfqpoint{1.759733in}{3.265688in}}{\pgfqpoint{1.770783in}{3.265688in}}%
\pgfpathclose%
\pgfusepath{stroke,fill}%
\end{pgfscope}%
\begin{pgfscope}%
\pgfpathrectangle{\pgfqpoint{0.648703in}{0.548769in}}{\pgfqpoint{5.201297in}{3.102590in}}%
\pgfusepath{clip}%
\pgfsetbuttcap%
\pgfsetroundjoin%
\definecolor{currentfill}{rgb}{0.121569,0.466667,0.705882}%
\pgfsetfillcolor{currentfill}%
\pgfsetlinewidth{1.003750pt}%
\definecolor{currentstroke}{rgb}{0.121569,0.466667,0.705882}%
\pgfsetstrokecolor{currentstroke}%
\pgfsetdash{}{0pt}%
\pgfpathmoveto{\pgfqpoint{2.337972in}{2.808990in}}%
\pgfpathcurveto{\pgfqpoint{2.349023in}{2.808990in}}{\pgfqpoint{2.359622in}{2.813380in}}{\pgfqpoint{2.367435in}{2.821193in}}%
\pgfpathcurveto{\pgfqpoint{2.375249in}{2.829007in}}{\pgfqpoint{2.379639in}{2.839606in}}{\pgfqpoint{2.379639in}{2.850656in}}%
\pgfpathcurveto{\pgfqpoint{2.379639in}{2.861706in}}{\pgfqpoint{2.375249in}{2.872305in}}{\pgfqpoint{2.367435in}{2.880119in}}%
\pgfpathcurveto{\pgfqpoint{2.359622in}{2.887933in}}{\pgfqpoint{2.349023in}{2.892323in}}{\pgfqpoint{2.337972in}{2.892323in}}%
\pgfpathcurveto{\pgfqpoint{2.326922in}{2.892323in}}{\pgfqpoint{2.316323in}{2.887933in}}{\pgfqpoint{2.308510in}{2.880119in}}%
\pgfpathcurveto{\pgfqpoint{2.300696in}{2.872305in}}{\pgfqpoint{2.296306in}{2.861706in}}{\pgfqpoint{2.296306in}{2.850656in}}%
\pgfpathcurveto{\pgfqpoint{2.296306in}{2.839606in}}{\pgfqpoint{2.300696in}{2.829007in}}{\pgfqpoint{2.308510in}{2.821193in}}%
\pgfpathcurveto{\pgfqpoint{2.316323in}{2.813380in}}{\pgfqpoint{2.326922in}{2.808990in}}{\pgfqpoint{2.337972in}{2.808990in}}%
\pgfpathclose%
\pgfusepath{stroke,fill}%
\end{pgfscope}%
\begin{pgfscope}%
\pgfpathrectangle{\pgfqpoint{0.648703in}{0.548769in}}{\pgfqpoint{5.201297in}{3.102590in}}%
\pgfusepath{clip}%
\pgfsetbuttcap%
\pgfsetroundjoin%
\definecolor{currentfill}{rgb}{1.000000,0.498039,0.054902}%
\pgfsetfillcolor{currentfill}%
\pgfsetlinewidth{1.003750pt}%
\definecolor{currentstroke}{rgb}{1.000000,0.498039,0.054902}%
\pgfsetstrokecolor{currentstroke}%
\pgfsetdash{}{0pt}%
\pgfpathmoveto{\pgfqpoint{2.172480in}{3.185343in}}%
\pgfpathcurveto{\pgfqpoint{2.183530in}{3.185343in}}{\pgfqpoint{2.194129in}{3.189733in}}{\pgfqpoint{2.201943in}{3.197547in}}%
\pgfpathcurveto{\pgfqpoint{2.209756in}{3.205360in}}{\pgfqpoint{2.214147in}{3.215959in}}{\pgfqpoint{2.214147in}{3.227010in}}%
\pgfpathcurveto{\pgfqpoint{2.214147in}{3.238060in}}{\pgfqpoint{2.209756in}{3.248659in}}{\pgfqpoint{2.201943in}{3.256472in}}%
\pgfpathcurveto{\pgfqpoint{2.194129in}{3.264286in}}{\pgfqpoint{2.183530in}{3.268676in}}{\pgfqpoint{2.172480in}{3.268676in}}%
\pgfpathcurveto{\pgfqpoint{2.161430in}{3.268676in}}{\pgfqpoint{2.150831in}{3.264286in}}{\pgfqpoint{2.143017in}{3.256472in}}%
\pgfpathcurveto{\pgfqpoint{2.135204in}{3.248659in}}{\pgfqpoint{2.130813in}{3.238060in}}{\pgfqpoint{2.130813in}{3.227010in}}%
\pgfpathcurveto{\pgfqpoint{2.130813in}{3.215959in}}{\pgfqpoint{2.135204in}{3.205360in}}{\pgfqpoint{2.143017in}{3.197547in}}%
\pgfpathcurveto{\pgfqpoint{2.150831in}{3.189733in}}{\pgfqpoint{2.161430in}{3.185343in}}{\pgfqpoint{2.172480in}{3.185343in}}%
\pgfpathclose%
\pgfusepath{stroke,fill}%
\end{pgfscope}%
\begin{pgfscope}%
\pgfpathrectangle{\pgfqpoint{0.648703in}{0.548769in}}{\pgfqpoint{5.201297in}{3.102590in}}%
\pgfusepath{clip}%
\pgfsetbuttcap%
\pgfsetroundjoin%
\definecolor{currentfill}{rgb}{1.000000,0.498039,0.054902}%
\pgfsetfillcolor{currentfill}%
\pgfsetlinewidth{1.003750pt}%
\definecolor{currentstroke}{rgb}{1.000000,0.498039,0.054902}%
\pgfsetstrokecolor{currentstroke}%
\pgfsetdash{}{0pt}%
\pgfpathmoveto{\pgfqpoint{1.955543in}{3.193800in}}%
\pgfpathcurveto{\pgfqpoint{1.966593in}{3.193800in}}{\pgfqpoint{1.977192in}{3.198191in}}{\pgfqpoint{1.985006in}{3.206004in}}%
\pgfpathcurveto{\pgfqpoint{1.992819in}{3.213818in}}{\pgfqpoint{1.997210in}{3.224417in}}{\pgfqpoint{1.997210in}{3.235467in}}%
\pgfpathcurveto{\pgfqpoint{1.997210in}{3.246517in}}{\pgfqpoint{1.992819in}{3.257116in}}{\pgfqpoint{1.985006in}{3.264930in}}%
\pgfpathcurveto{\pgfqpoint{1.977192in}{3.272743in}}{\pgfqpoint{1.966593in}{3.277134in}}{\pgfqpoint{1.955543in}{3.277134in}}%
\pgfpathcurveto{\pgfqpoint{1.944493in}{3.277134in}}{\pgfqpoint{1.933894in}{3.272743in}}{\pgfqpoint{1.926080in}{3.264930in}}%
\pgfpathcurveto{\pgfqpoint{1.918267in}{3.257116in}}{\pgfqpoint{1.913876in}{3.246517in}}{\pgfqpoint{1.913876in}{3.235467in}}%
\pgfpathcurveto{\pgfqpoint{1.913876in}{3.224417in}}{\pgfqpoint{1.918267in}{3.213818in}}{\pgfqpoint{1.926080in}{3.206004in}}%
\pgfpathcurveto{\pgfqpoint{1.933894in}{3.198191in}}{\pgfqpoint{1.944493in}{3.193800in}}{\pgfqpoint{1.955543in}{3.193800in}}%
\pgfpathclose%
\pgfusepath{stroke,fill}%
\end{pgfscope}%
\begin{pgfscope}%
\pgfpathrectangle{\pgfqpoint{0.648703in}{0.548769in}}{\pgfqpoint{5.201297in}{3.102590in}}%
\pgfusepath{clip}%
\pgfsetbuttcap%
\pgfsetroundjoin%
\definecolor{currentfill}{rgb}{1.000000,0.498039,0.054902}%
\pgfsetfillcolor{currentfill}%
\pgfsetlinewidth{1.003750pt}%
\definecolor{currentstroke}{rgb}{1.000000,0.498039,0.054902}%
\pgfsetstrokecolor{currentstroke}%
\pgfsetdash{}{0pt}%
\pgfpathmoveto{\pgfqpoint{1.696290in}{3.362948in}}%
\pgfpathcurveto{\pgfqpoint{1.707340in}{3.362948in}}{\pgfqpoint{1.717939in}{3.367338in}}{\pgfqpoint{1.725753in}{3.375152in}}%
\pgfpathcurveto{\pgfqpoint{1.733566in}{3.382965in}}{\pgfqpoint{1.737957in}{3.393564in}}{\pgfqpoint{1.737957in}{3.404615in}}%
\pgfpathcurveto{\pgfqpoint{1.737957in}{3.415665in}}{\pgfqpoint{1.733566in}{3.426264in}}{\pgfqpoint{1.725753in}{3.434077in}}%
\pgfpathcurveto{\pgfqpoint{1.717939in}{3.441891in}}{\pgfqpoint{1.707340in}{3.446281in}}{\pgfqpoint{1.696290in}{3.446281in}}%
\pgfpathcurveto{\pgfqpoint{1.685240in}{3.446281in}}{\pgfqpoint{1.674641in}{3.441891in}}{\pgfqpoint{1.666827in}{3.434077in}}%
\pgfpathcurveto{\pgfqpoint{1.659014in}{3.426264in}}{\pgfqpoint{1.654623in}{3.415665in}}{\pgfqpoint{1.654623in}{3.404615in}}%
\pgfpathcurveto{\pgfqpoint{1.654623in}{3.393564in}}{\pgfqpoint{1.659014in}{3.382965in}}{\pgfqpoint{1.666827in}{3.375152in}}%
\pgfpathcurveto{\pgfqpoint{1.674641in}{3.367338in}}{\pgfqpoint{1.685240in}{3.362948in}}{\pgfqpoint{1.696290in}{3.362948in}}%
\pgfpathclose%
\pgfusepath{stroke,fill}%
\end{pgfscope}%
\begin{pgfscope}%
\pgfpathrectangle{\pgfqpoint{0.648703in}{0.548769in}}{\pgfqpoint{5.201297in}{3.102590in}}%
\pgfusepath{clip}%
\pgfsetbuttcap%
\pgfsetroundjoin%
\definecolor{currentfill}{rgb}{1.000000,0.498039,0.054902}%
\pgfsetfillcolor{currentfill}%
\pgfsetlinewidth{1.003750pt}%
\definecolor{currentstroke}{rgb}{1.000000,0.498039,0.054902}%
\pgfsetstrokecolor{currentstroke}%
\pgfsetdash{}{0pt}%
\pgfpathmoveto{\pgfqpoint{1.802089in}{3.257231in}}%
\pgfpathcurveto{\pgfqpoint{1.813139in}{3.257231in}}{\pgfqpoint{1.823738in}{3.261621in}}{\pgfqpoint{1.831552in}{3.269435in}}%
\pgfpathcurveto{\pgfqpoint{1.839365in}{3.277248in}}{\pgfqpoint{1.843755in}{3.287847in}}{\pgfqpoint{1.843755in}{3.298897in}}%
\pgfpathcurveto{\pgfqpoint{1.843755in}{3.309947in}}{\pgfqpoint{1.839365in}{3.320546in}}{\pgfqpoint{1.831552in}{3.328360in}}%
\pgfpathcurveto{\pgfqpoint{1.823738in}{3.336174in}}{\pgfqpoint{1.813139in}{3.340564in}}{\pgfqpoint{1.802089in}{3.340564in}}%
\pgfpathcurveto{\pgfqpoint{1.791039in}{3.340564in}}{\pgfqpoint{1.780440in}{3.336174in}}{\pgfqpoint{1.772626in}{3.328360in}}%
\pgfpathcurveto{\pgfqpoint{1.764812in}{3.320546in}}{\pgfqpoint{1.760422in}{3.309947in}}{\pgfqpoint{1.760422in}{3.298897in}}%
\pgfpathcurveto{\pgfqpoint{1.760422in}{3.287847in}}{\pgfqpoint{1.764812in}{3.277248in}}{\pgfqpoint{1.772626in}{3.269435in}}%
\pgfpathcurveto{\pgfqpoint{1.780440in}{3.261621in}}{\pgfqpoint{1.791039in}{3.257231in}}{\pgfqpoint{1.802089in}{3.257231in}}%
\pgfpathclose%
\pgfusepath{stroke,fill}%
\end{pgfscope}%
\begin{pgfscope}%
\pgfpathrectangle{\pgfqpoint{0.648703in}{0.548769in}}{\pgfqpoint{5.201297in}{3.102590in}}%
\pgfusepath{clip}%
\pgfsetbuttcap%
\pgfsetroundjoin%
\definecolor{currentfill}{rgb}{1.000000,0.498039,0.054902}%
\pgfsetfillcolor{currentfill}%
\pgfsetlinewidth{1.003750pt}%
\definecolor{currentstroke}{rgb}{1.000000,0.498039,0.054902}%
\pgfsetstrokecolor{currentstroke}%
\pgfsetdash{}{0pt}%
\pgfpathmoveto{\pgfqpoint{1.623147in}{3.189572in}}%
\pgfpathcurveto{\pgfqpoint{1.634197in}{3.189572in}}{\pgfqpoint{1.644796in}{3.193962in}}{\pgfqpoint{1.652610in}{3.201775in}}%
\pgfpathcurveto{\pgfqpoint{1.660424in}{3.209589in}}{\pgfqpoint{1.664814in}{3.220188in}}{\pgfqpoint{1.664814in}{3.231238in}}%
\pgfpathcurveto{\pgfqpoint{1.664814in}{3.242288in}}{\pgfqpoint{1.660424in}{3.252887in}}{\pgfqpoint{1.652610in}{3.260701in}}%
\pgfpathcurveto{\pgfqpoint{1.644796in}{3.268515in}}{\pgfqpoint{1.634197in}{3.272905in}}{\pgfqpoint{1.623147in}{3.272905in}}%
\pgfpathcurveto{\pgfqpoint{1.612097in}{3.272905in}}{\pgfqpoint{1.601498in}{3.268515in}}{\pgfqpoint{1.593685in}{3.260701in}}%
\pgfpathcurveto{\pgfqpoint{1.585871in}{3.252887in}}{\pgfqpoint{1.581481in}{3.242288in}}{\pgfqpoint{1.581481in}{3.231238in}}%
\pgfpathcurveto{\pgfqpoint{1.581481in}{3.220188in}}{\pgfqpoint{1.585871in}{3.209589in}}{\pgfqpoint{1.593685in}{3.201775in}}%
\pgfpathcurveto{\pgfqpoint{1.601498in}{3.193962in}}{\pgfqpoint{1.612097in}{3.189572in}}{\pgfqpoint{1.623147in}{3.189572in}}%
\pgfpathclose%
\pgfusepath{stroke,fill}%
\end{pgfscope}%
\begin{pgfscope}%
\pgfpathrectangle{\pgfqpoint{0.648703in}{0.548769in}}{\pgfqpoint{5.201297in}{3.102590in}}%
\pgfusepath{clip}%
\pgfsetbuttcap%
\pgfsetroundjoin%
\definecolor{currentfill}{rgb}{1.000000,0.498039,0.054902}%
\pgfsetfillcolor{currentfill}%
\pgfsetlinewidth{1.003750pt}%
\definecolor{currentstroke}{rgb}{1.000000,0.498039,0.054902}%
\pgfsetstrokecolor{currentstroke}%
\pgfsetdash{}{0pt}%
\pgfpathmoveto{\pgfqpoint{2.583140in}{3.189572in}}%
\pgfpathcurveto{\pgfqpoint{2.594191in}{3.189572in}}{\pgfqpoint{2.604790in}{3.193962in}}{\pgfqpoint{2.612603in}{3.201775in}}%
\pgfpathcurveto{\pgfqpoint{2.620417in}{3.209589in}}{\pgfqpoint{2.624807in}{3.220188in}}{\pgfqpoint{2.624807in}{3.231238in}}%
\pgfpathcurveto{\pgfqpoint{2.624807in}{3.242288in}}{\pgfqpoint{2.620417in}{3.252887in}}{\pgfqpoint{2.612603in}{3.260701in}}%
\pgfpathcurveto{\pgfqpoint{2.604790in}{3.268515in}}{\pgfqpoint{2.594191in}{3.272905in}}{\pgfqpoint{2.583140in}{3.272905in}}%
\pgfpathcurveto{\pgfqpoint{2.572090in}{3.272905in}}{\pgfqpoint{2.561491in}{3.268515in}}{\pgfqpoint{2.553678in}{3.260701in}}%
\pgfpathcurveto{\pgfqpoint{2.545864in}{3.252887in}}{\pgfqpoint{2.541474in}{3.242288in}}{\pgfqpoint{2.541474in}{3.231238in}}%
\pgfpathcurveto{\pgfqpoint{2.541474in}{3.220188in}}{\pgfqpoint{2.545864in}{3.209589in}}{\pgfqpoint{2.553678in}{3.201775in}}%
\pgfpathcurveto{\pgfqpoint{2.561491in}{3.193962in}}{\pgfqpoint{2.572090in}{3.189572in}}{\pgfqpoint{2.583140in}{3.189572in}}%
\pgfpathclose%
\pgfusepath{stroke,fill}%
\end{pgfscope}%
\begin{pgfscope}%
\pgfpathrectangle{\pgfqpoint{0.648703in}{0.548769in}}{\pgfqpoint{5.201297in}{3.102590in}}%
\pgfusepath{clip}%
\pgfsetbuttcap%
\pgfsetroundjoin%
\definecolor{currentfill}{rgb}{0.121569,0.466667,0.705882}%
\pgfsetfillcolor{currentfill}%
\pgfsetlinewidth{1.003750pt}%
\definecolor{currentstroke}{rgb}{0.121569,0.466667,0.705882}%
\pgfsetstrokecolor{currentstroke}%
\pgfsetdash{}{0pt}%
\pgfpathmoveto{\pgfqpoint{2.962695in}{3.181114in}}%
\pgfpathcurveto{\pgfqpoint{2.973745in}{3.181114in}}{\pgfqpoint{2.984344in}{3.185504in}}{\pgfqpoint{2.992158in}{3.193318in}}%
\pgfpathcurveto{\pgfqpoint{2.999972in}{3.201132in}}{\pgfqpoint{3.004362in}{3.211731in}}{\pgfqpoint{3.004362in}{3.222781in}}%
\pgfpathcurveto{\pgfqpoint{3.004362in}{3.233831in}}{\pgfqpoint{2.999972in}{3.244430in}}{\pgfqpoint{2.992158in}{3.252244in}}%
\pgfpathcurveto{\pgfqpoint{2.984344in}{3.260057in}}{\pgfqpoint{2.973745in}{3.264448in}}{\pgfqpoint{2.962695in}{3.264448in}}%
\pgfpathcurveto{\pgfqpoint{2.951645in}{3.264448in}}{\pgfqpoint{2.941046in}{3.260057in}}{\pgfqpoint{2.933232in}{3.252244in}}%
\pgfpathcurveto{\pgfqpoint{2.925419in}{3.244430in}}{\pgfqpoint{2.921029in}{3.233831in}}{\pgfqpoint{2.921029in}{3.222781in}}%
\pgfpathcurveto{\pgfqpoint{2.921029in}{3.211731in}}{\pgfqpoint{2.925419in}{3.201132in}}{\pgfqpoint{2.933232in}{3.193318in}}%
\pgfpathcurveto{\pgfqpoint{2.941046in}{3.185504in}}{\pgfqpoint{2.951645in}{3.181114in}}{\pgfqpoint{2.962695in}{3.181114in}}%
\pgfpathclose%
\pgfusepath{stroke,fill}%
\end{pgfscope}%
\begin{pgfscope}%
\pgfpathrectangle{\pgfqpoint{0.648703in}{0.548769in}}{\pgfqpoint{5.201297in}{3.102590in}}%
\pgfusepath{clip}%
\pgfsetbuttcap%
\pgfsetroundjoin%
\definecolor{currentfill}{rgb}{1.000000,0.498039,0.054902}%
\pgfsetfillcolor{currentfill}%
\pgfsetlinewidth{1.003750pt}%
\definecolor{currentstroke}{rgb}{1.000000,0.498039,0.054902}%
\pgfsetstrokecolor{currentstroke}%
\pgfsetdash{}{0pt}%
\pgfpathmoveto{\pgfqpoint{2.282547in}{3.185343in}}%
\pgfpathcurveto{\pgfqpoint{2.293597in}{3.185343in}}{\pgfqpoint{2.304196in}{3.189733in}}{\pgfqpoint{2.312010in}{3.197547in}}%
\pgfpathcurveto{\pgfqpoint{2.319823in}{3.205360in}}{\pgfqpoint{2.324214in}{3.215959in}}{\pgfqpoint{2.324214in}{3.227010in}}%
\pgfpathcurveto{\pgfqpoint{2.324214in}{3.238060in}}{\pgfqpoint{2.319823in}{3.248659in}}{\pgfqpoint{2.312010in}{3.256472in}}%
\pgfpathcurveto{\pgfqpoint{2.304196in}{3.264286in}}{\pgfqpoint{2.293597in}{3.268676in}}{\pgfqpoint{2.282547in}{3.268676in}}%
\pgfpathcurveto{\pgfqpoint{2.271497in}{3.268676in}}{\pgfqpoint{2.260898in}{3.264286in}}{\pgfqpoint{2.253084in}{3.256472in}}%
\pgfpathcurveto{\pgfqpoint{2.245271in}{3.248659in}}{\pgfqpoint{2.240880in}{3.238060in}}{\pgfqpoint{2.240880in}{3.227010in}}%
\pgfpathcurveto{\pgfqpoint{2.240880in}{3.215959in}}{\pgfqpoint{2.245271in}{3.205360in}}{\pgfqpoint{2.253084in}{3.197547in}}%
\pgfpathcurveto{\pgfqpoint{2.260898in}{3.189733in}}{\pgfqpoint{2.271497in}{3.185343in}}{\pgfqpoint{2.282547in}{3.185343in}}%
\pgfpathclose%
\pgfusepath{stroke,fill}%
\end{pgfscope}%
\begin{pgfscope}%
\pgfpathrectangle{\pgfqpoint{0.648703in}{0.548769in}}{\pgfqpoint{5.201297in}{3.102590in}}%
\pgfusepath{clip}%
\pgfsetbuttcap%
\pgfsetroundjoin%
\definecolor{currentfill}{rgb}{0.121569,0.466667,0.705882}%
\pgfsetfillcolor{currentfill}%
\pgfsetlinewidth{1.003750pt}%
\definecolor{currentstroke}{rgb}{0.121569,0.466667,0.705882}%
\pgfsetstrokecolor{currentstroke}%
\pgfsetdash{}{0pt}%
\pgfpathmoveto{\pgfqpoint{1.603549in}{0.681958in}}%
\pgfpathcurveto{\pgfqpoint{1.614599in}{0.681958in}}{\pgfqpoint{1.625198in}{0.686349in}}{\pgfqpoint{1.633011in}{0.694162in}}%
\pgfpathcurveto{\pgfqpoint{1.640825in}{0.701976in}}{\pgfqpoint{1.645215in}{0.712575in}}{\pgfqpoint{1.645215in}{0.723625in}}%
\pgfpathcurveto{\pgfqpoint{1.645215in}{0.734675in}}{\pgfqpoint{1.640825in}{0.745274in}}{\pgfqpoint{1.633011in}{0.753088in}}%
\pgfpathcurveto{\pgfqpoint{1.625198in}{0.760902in}}{\pgfqpoint{1.614599in}{0.765292in}}{\pgfqpoint{1.603549in}{0.765292in}}%
\pgfpathcurveto{\pgfqpoint{1.592498in}{0.765292in}}{\pgfqpoint{1.581899in}{0.760902in}}{\pgfqpoint{1.574086in}{0.753088in}}%
\pgfpathcurveto{\pgfqpoint{1.566272in}{0.745274in}}{\pgfqpoint{1.561882in}{0.734675in}}{\pgfqpoint{1.561882in}{0.723625in}}%
\pgfpathcurveto{\pgfqpoint{1.561882in}{0.712575in}}{\pgfqpoint{1.566272in}{0.701976in}}{\pgfqpoint{1.574086in}{0.694162in}}%
\pgfpathcurveto{\pgfqpoint{1.581899in}{0.686349in}}{\pgfqpoint{1.592498in}{0.681958in}}{\pgfqpoint{1.603549in}{0.681958in}}%
\pgfpathclose%
\pgfusepath{stroke,fill}%
\end{pgfscope}%
\begin{pgfscope}%
\pgfpathrectangle{\pgfqpoint{0.648703in}{0.548769in}}{\pgfqpoint{5.201297in}{3.102590in}}%
\pgfusepath{clip}%
\pgfsetbuttcap%
\pgfsetroundjoin%
\definecolor{currentfill}{rgb}{0.839216,0.152941,0.156863}%
\pgfsetfillcolor{currentfill}%
\pgfsetlinewidth{1.003750pt}%
\definecolor{currentstroke}{rgb}{0.839216,0.152941,0.156863}%
\pgfsetstrokecolor{currentstroke}%
\pgfsetdash{}{0pt}%
\pgfpathmoveto{\pgfqpoint{2.131697in}{3.214944in}}%
\pgfpathcurveto{\pgfqpoint{2.142747in}{3.214944in}}{\pgfqpoint{2.153346in}{3.219334in}}{\pgfqpoint{2.161160in}{3.227148in}}%
\pgfpathcurveto{\pgfqpoint{2.168974in}{3.234961in}}{\pgfqpoint{2.173364in}{3.245560in}}{\pgfqpoint{2.173364in}{3.256610in}}%
\pgfpathcurveto{\pgfqpoint{2.173364in}{3.267661in}}{\pgfqpoint{2.168974in}{3.278260in}}{\pgfqpoint{2.161160in}{3.286073in}}%
\pgfpathcurveto{\pgfqpoint{2.153346in}{3.293887in}}{\pgfqpoint{2.142747in}{3.298277in}}{\pgfqpoint{2.131697in}{3.298277in}}%
\pgfpathcurveto{\pgfqpoint{2.120647in}{3.298277in}}{\pgfqpoint{2.110048in}{3.293887in}}{\pgfqpoint{2.102234in}{3.286073in}}%
\pgfpathcurveto{\pgfqpoint{2.094421in}{3.278260in}}{\pgfqpoint{2.090030in}{3.267661in}}{\pgfqpoint{2.090030in}{3.256610in}}%
\pgfpathcurveto{\pgfqpoint{2.090030in}{3.245560in}}{\pgfqpoint{2.094421in}{3.234961in}}{\pgfqpoint{2.102234in}{3.227148in}}%
\pgfpathcurveto{\pgfqpoint{2.110048in}{3.219334in}}{\pgfqpoint{2.120647in}{3.214944in}}{\pgfqpoint{2.131697in}{3.214944in}}%
\pgfpathclose%
\pgfusepath{stroke,fill}%
\end{pgfscope}%
\begin{pgfscope}%
\pgfpathrectangle{\pgfqpoint{0.648703in}{0.548769in}}{\pgfqpoint{5.201297in}{3.102590in}}%
\pgfusepath{clip}%
\pgfsetbuttcap%
\pgfsetroundjoin%
\definecolor{currentfill}{rgb}{1.000000,0.498039,0.054902}%
\pgfsetfillcolor{currentfill}%
\pgfsetlinewidth{1.003750pt}%
\definecolor{currentstroke}{rgb}{1.000000,0.498039,0.054902}%
\pgfsetstrokecolor{currentstroke}%
\pgfsetdash{}{0pt}%
\pgfpathmoveto{\pgfqpoint{1.512175in}{3.185343in}}%
\pgfpathcurveto{\pgfqpoint{1.523225in}{3.185343in}}{\pgfqpoint{1.533824in}{3.189733in}}{\pgfqpoint{1.541637in}{3.197547in}}%
\pgfpathcurveto{\pgfqpoint{1.549451in}{3.205360in}}{\pgfqpoint{1.553841in}{3.215959in}}{\pgfqpoint{1.553841in}{3.227010in}}%
\pgfpathcurveto{\pgfqpoint{1.553841in}{3.238060in}}{\pgfqpoint{1.549451in}{3.248659in}}{\pgfqpoint{1.541637in}{3.256472in}}%
\pgfpathcurveto{\pgfqpoint{1.533824in}{3.264286in}}{\pgfqpoint{1.523225in}{3.268676in}}{\pgfqpoint{1.512175in}{3.268676in}}%
\pgfpathcurveto{\pgfqpoint{1.501124in}{3.268676in}}{\pgfqpoint{1.490525in}{3.264286in}}{\pgfqpoint{1.482712in}{3.256472in}}%
\pgfpathcurveto{\pgfqpoint{1.474898in}{3.248659in}}{\pgfqpoint{1.470508in}{3.238060in}}{\pgfqpoint{1.470508in}{3.227010in}}%
\pgfpathcurveto{\pgfqpoint{1.470508in}{3.215959in}}{\pgfqpoint{1.474898in}{3.205360in}}{\pgfqpoint{1.482712in}{3.197547in}}%
\pgfpathcurveto{\pgfqpoint{1.490525in}{3.189733in}}{\pgfqpoint{1.501124in}{3.185343in}}{\pgfqpoint{1.512175in}{3.185343in}}%
\pgfpathclose%
\pgfusepath{stroke,fill}%
\end{pgfscope}%
\begin{pgfscope}%
\pgfpathrectangle{\pgfqpoint{0.648703in}{0.548769in}}{\pgfqpoint{5.201297in}{3.102590in}}%
\pgfusepath{clip}%
\pgfsetbuttcap%
\pgfsetroundjoin%
\definecolor{currentfill}{rgb}{0.121569,0.466667,0.705882}%
\pgfsetfillcolor{currentfill}%
\pgfsetlinewidth{1.003750pt}%
\definecolor{currentstroke}{rgb}{0.121569,0.466667,0.705882}%
\pgfsetstrokecolor{currentstroke}%
\pgfsetdash{}{0pt}%
\pgfpathmoveto{\pgfqpoint{1.751672in}{1.138657in}}%
\pgfpathcurveto{\pgfqpoint{1.762722in}{1.138657in}}{\pgfqpoint{1.773321in}{1.143047in}}{\pgfqpoint{1.781135in}{1.150861in}}%
\pgfpathcurveto{\pgfqpoint{1.788948in}{1.158674in}}{\pgfqpoint{1.793339in}{1.169274in}}{\pgfqpoint{1.793339in}{1.180324in}}%
\pgfpathcurveto{\pgfqpoint{1.793339in}{1.191374in}}{\pgfqpoint{1.788948in}{1.201973in}}{\pgfqpoint{1.781135in}{1.209786in}}%
\pgfpathcurveto{\pgfqpoint{1.773321in}{1.217600in}}{\pgfqpoint{1.762722in}{1.221990in}}{\pgfqpoint{1.751672in}{1.221990in}}%
\pgfpathcurveto{\pgfqpoint{1.740622in}{1.221990in}}{\pgfqpoint{1.730023in}{1.217600in}}{\pgfqpoint{1.722209in}{1.209786in}}%
\pgfpathcurveto{\pgfqpoint{1.714396in}{1.201973in}}{\pgfqpoint{1.710005in}{1.191374in}}{\pgfqpoint{1.710005in}{1.180324in}}%
\pgfpathcurveto{\pgfqpoint{1.710005in}{1.169274in}}{\pgfqpoint{1.714396in}{1.158674in}}{\pgfqpoint{1.722209in}{1.150861in}}%
\pgfpathcurveto{\pgfqpoint{1.730023in}{1.143047in}}{\pgfqpoint{1.740622in}{1.138657in}}{\pgfqpoint{1.751672in}{1.138657in}}%
\pgfpathclose%
\pgfusepath{stroke,fill}%
\end{pgfscope}%
\begin{pgfscope}%
\pgfpathrectangle{\pgfqpoint{0.648703in}{0.548769in}}{\pgfqpoint{5.201297in}{3.102590in}}%
\pgfusepath{clip}%
\pgfsetbuttcap%
\pgfsetroundjoin%
\definecolor{currentfill}{rgb}{0.121569,0.466667,0.705882}%
\pgfsetfillcolor{currentfill}%
\pgfsetlinewidth{1.003750pt}%
\definecolor{currentstroke}{rgb}{0.121569,0.466667,0.705882}%
\pgfsetstrokecolor{currentstroke}%
\pgfsetdash{}{0pt}%
\pgfpathmoveto{\pgfqpoint{1.980473in}{1.087913in}}%
\pgfpathcurveto{\pgfqpoint{1.991523in}{1.087913in}}{\pgfqpoint{2.002122in}{1.092303in}}{\pgfqpoint{2.009936in}{1.100117in}}%
\pgfpathcurveto{\pgfqpoint{2.017749in}{1.107930in}}{\pgfqpoint{2.022139in}{1.118529in}}{\pgfqpoint{2.022139in}{1.129579in}}%
\pgfpathcurveto{\pgfqpoint{2.022139in}{1.140629in}}{\pgfqpoint{2.017749in}{1.151229in}}{\pgfqpoint{2.009936in}{1.159042in}}%
\pgfpathcurveto{\pgfqpoint{2.002122in}{1.166856in}}{\pgfqpoint{1.991523in}{1.171246in}}{\pgfqpoint{1.980473in}{1.171246in}}%
\pgfpathcurveto{\pgfqpoint{1.969423in}{1.171246in}}{\pgfqpoint{1.958824in}{1.166856in}}{\pgfqpoint{1.951010in}{1.159042in}}%
\pgfpathcurveto{\pgfqpoint{1.943196in}{1.151229in}}{\pgfqpoint{1.938806in}{1.140629in}}{\pgfqpoint{1.938806in}{1.129579in}}%
\pgfpathcurveto{\pgfqpoint{1.938806in}{1.118529in}}{\pgfqpoint{1.943196in}{1.107930in}}{\pgfqpoint{1.951010in}{1.100117in}}%
\pgfpathcurveto{\pgfqpoint{1.958824in}{1.092303in}}{\pgfqpoint{1.969423in}{1.087913in}}{\pgfqpoint{1.980473in}{1.087913in}}%
\pgfpathclose%
\pgfusepath{stroke,fill}%
\end{pgfscope}%
\begin{pgfscope}%
\pgfpathrectangle{\pgfqpoint{0.648703in}{0.548769in}}{\pgfqpoint{5.201297in}{3.102590in}}%
\pgfusepath{clip}%
\pgfsetbuttcap%
\pgfsetroundjoin%
\definecolor{currentfill}{rgb}{1.000000,0.498039,0.054902}%
\pgfsetfillcolor{currentfill}%
\pgfsetlinewidth{1.003750pt}%
\definecolor{currentstroke}{rgb}{1.000000,0.498039,0.054902}%
\pgfsetstrokecolor{currentstroke}%
\pgfsetdash{}{0pt}%
\pgfpathmoveto{\pgfqpoint{1.767168in}{3.189572in}}%
\pgfpathcurveto{\pgfqpoint{1.778218in}{3.189572in}}{\pgfqpoint{1.788817in}{3.193962in}}{\pgfqpoint{1.796631in}{3.201775in}}%
\pgfpathcurveto{\pgfqpoint{1.804444in}{3.209589in}}{\pgfqpoint{1.808835in}{3.220188in}}{\pgfqpoint{1.808835in}{3.231238in}}%
\pgfpathcurveto{\pgfqpoint{1.808835in}{3.242288in}}{\pgfqpoint{1.804444in}{3.252887in}}{\pgfqpoint{1.796631in}{3.260701in}}%
\pgfpathcurveto{\pgfqpoint{1.788817in}{3.268515in}}{\pgfqpoint{1.778218in}{3.272905in}}{\pgfqpoint{1.767168in}{3.272905in}}%
\pgfpathcurveto{\pgfqpoint{1.756118in}{3.272905in}}{\pgfqpoint{1.745519in}{3.268515in}}{\pgfqpoint{1.737705in}{3.260701in}}%
\pgfpathcurveto{\pgfqpoint{1.729892in}{3.252887in}}{\pgfqpoint{1.725501in}{3.242288in}}{\pgfqpoint{1.725501in}{3.231238in}}%
\pgfpathcurveto{\pgfqpoint{1.725501in}{3.220188in}}{\pgfqpoint{1.729892in}{3.209589in}}{\pgfqpoint{1.737705in}{3.201775in}}%
\pgfpathcurveto{\pgfqpoint{1.745519in}{3.193962in}}{\pgfqpoint{1.756118in}{3.189572in}}{\pgfqpoint{1.767168in}{3.189572in}}%
\pgfpathclose%
\pgfusepath{stroke,fill}%
\end{pgfscope}%
\begin{pgfscope}%
\pgfpathrectangle{\pgfqpoint{0.648703in}{0.548769in}}{\pgfqpoint{5.201297in}{3.102590in}}%
\pgfusepath{clip}%
\pgfsetbuttcap%
\pgfsetroundjoin%
\definecolor{currentfill}{rgb}{0.121569,0.466667,0.705882}%
\pgfsetfillcolor{currentfill}%
\pgfsetlinewidth{1.003750pt}%
\definecolor{currentstroke}{rgb}{0.121569,0.466667,0.705882}%
\pgfsetstrokecolor{currentstroke}%
\pgfsetdash{}{0pt}%
\pgfpathmoveto{\pgfqpoint{1.596023in}{1.024482in}}%
\pgfpathcurveto{\pgfqpoint{1.607073in}{1.024482in}}{\pgfqpoint{1.617672in}{1.028873in}}{\pgfqpoint{1.625485in}{1.036686in}}%
\pgfpathcurveto{\pgfqpoint{1.633299in}{1.044500in}}{\pgfqpoint{1.637689in}{1.055099in}}{\pgfqpoint{1.637689in}{1.066149in}}%
\pgfpathcurveto{\pgfqpoint{1.637689in}{1.077199in}}{\pgfqpoint{1.633299in}{1.087798in}}{\pgfqpoint{1.625485in}{1.095612in}}%
\pgfpathcurveto{\pgfqpoint{1.617672in}{1.103425in}}{\pgfqpoint{1.607073in}{1.107816in}}{\pgfqpoint{1.596023in}{1.107816in}}%
\pgfpathcurveto{\pgfqpoint{1.584972in}{1.107816in}}{\pgfqpoint{1.574373in}{1.103425in}}{\pgfqpoint{1.566560in}{1.095612in}}%
\pgfpathcurveto{\pgfqpoint{1.558746in}{1.087798in}}{\pgfqpoint{1.554356in}{1.077199in}}{\pgfqpoint{1.554356in}{1.066149in}}%
\pgfpathcurveto{\pgfqpoint{1.554356in}{1.055099in}}{\pgfqpoint{1.558746in}{1.044500in}}{\pgfqpoint{1.566560in}{1.036686in}}%
\pgfpathcurveto{\pgfqpoint{1.574373in}{1.028873in}}{\pgfqpoint{1.584972in}{1.024482in}}{\pgfqpoint{1.596023in}{1.024482in}}%
\pgfpathclose%
\pgfusepath{stroke,fill}%
\end{pgfscope}%
\begin{pgfscope}%
\pgfpathrectangle{\pgfqpoint{0.648703in}{0.548769in}}{\pgfqpoint{5.201297in}{3.102590in}}%
\pgfusepath{clip}%
\pgfsetbuttcap%
\pgfsetroundjoin%
\definecolor{currentfill}{rgb}{0.121569,0.466667,0.705882}%
\pgfsetfillcolor{currentfill}%
\pgfsetlinewidth{1.003750pt}%
\definecolor{currentstroke}{rgb}{0.121569,0.466667,0.705882}%
\pgfsetstrokecolor{currentstroke}%
\pgfsetdash{}{0pt}%
\pgfpathmoveto{\pgfqpoint{2.589969in}{0.813048in}}%
\pgfpathcurveto{\pgfqpoint{2.601020in}{0.813048in}}{\pgfqpoint{2.611619in}{0.817438in}}{\pgfqpoint{2.619432in}{0.825252in}}%
\pgfpathcurveto{\pgfqpoint{2.627246in}{0.833065in}}{\pgfqpoint{2.631636in}{0.843664in}}{\pgfqpoint{2.631636in}{0.854715in}}%
\pgfpathcurveto{\pgfqpoint{2.631636in}{0.865765in}}{\pgfqpoint{2.627246in}{0.876364in}}{\pgfqpoint{2.619432in}{0.884177in}}%
\pgfpathcurveto{\pgfqpoint{2.611619in}{0.891991in}}{\pgfqpoint{2.601020in}{0.896381in}}{\pgfqpoint{2.589969in}{0.896381in}}%
\pgfpathcurveto{\pgfqpoint{2.578919in}{0.896381in}}{\pgfqpoint{2.568320in}{0.891991in}}{\pgfqpoint{2.560507in}{0.884177in}}%
\pgfpathcurveto{\pgfqpoint{2.552693in}{0.876364in}}{\pgfqpoint{2.548303in}{0.865765in}}{\pgfqpoint{2.548303in}{0.854715in}}%
\pgfpathcurveto{\pgfqpoint{2.548303in}{0.843664in}}{\pgfqpoint{2.552693in}{0.833065in}}{\pgfqpoint{2.560507in}{0.825252in}}%
\pgfpathcurveto{\pgfqpoint{2.568320in}{0.817438in}}{\pgfqpoint{2.578919in}{0.813048in}}{\pgfqpoint{2.589969in}{0.813048in}}%
\pgfpathclose%
\pgfusepath{stroke,fill}%
\end{pgfscope}%
\begin{pgfscope}%
\pgfpathrectangle{\pgfqpoint{0.648703in}{0.548769in}}{\pgfqpoint{5.201297in}{3.102590in}}%
\pgfusepath{clip}%
\pgfsetbuttcap%
\pgfsetroundjoin%
\definecolor{currentfill}{rgb}{0.839216,0.152941,0.156863}%
\pgfsetfillcolor{currentfill}%
\pgfsetlinewidth{1.003750pt}%
\definecolor{currentstroke}{rgb}{0.839216,0.152941,0.156863}%
\pgfsetstrokecolor{currentstroke}%
\pgfsetdash{}{0pt}%
\pgfpathmoveto{\pgfqpoint{1.738641in}{3.202258in}}%
\pgfpathcurveto{\pgfqpoint{1.749691in}{3.202258in}}{\pgfqpoint{1.760290in}{3.206648in}}{\pgfqpoint{1.768104in}{3.214462in}}%
\pgfpathcurveto{\pgfqpoint{1.775917in}{3.222275in}}{\pgfqpoint{1.780308in}{3.232874in}}{\pgfqpoint{1.780308in}{3.243924in}}%
\pgfpathcurveto{\pgfqpoint{1.780308in}{3.254974in}}{\pgfqpoint{1.775917in}{3.265573in}}{\pgfqpoint{1.768104in}{3.273387in}}%
\pgfpathcurveto{\pgfqpoint{1.760290in}{3.281201in}}{\pgfqpoint{1.749691in}{3.285591in}}{\pgfqpoint{1.738641in}{3.285591in}}%
\pgfpathcurveto{\pgfqpoint{1.727591in}{3.285591in}}{\pgfqpoint{1.716992in}{3.281201in}}{\pgfqpoint{1.709178in}{3.273387in}}%
\pgfpathcurveto{\pgfqpoint{1.701365in}{3.265573in}}{\pgfqpoint{1.696974in}{3.254974in}}{\pgfqpoint{1.696974in}{3.243924in}}%
\pgfpathcurveto{\pgfqpoint{1.696974in}{3.232874in}}{\pgfqpoint{1.701365in}{3.222275in}}{\pgfqpoint{1.709178in}{3.214462in}}%
\pgfpathcurveto{\pgfqpoint{1.716992in}{3.206648in}}{\pgfqpoint{1.727591in}{3.202258in}}{\pgfqpoint{1.738641in}{3.202258in}}%
\pgfpathclose%
\pgfusepath{stroke,fill}%
\end{pgfscope}%
\begin{pgfscope}%
\pgfsetbuttcap%
\pgfsetroundjoin%
\definecolor{currentfill}{rgb}{0.000000,0.000000,0.000000}%
\pgfsetfillcolor{currentfill}%
\pgfsetlinewidth{0.803000pt}%
\definecolor{currentstroke}{rgb}{0.000000,0.000000,0.000000}%
\pgfsetstrokecolor{currentstroke}%
\pgfsetdash{}{0pt}%
\pgfsys@defobject{currentmarker}{\pgfqpoint{0.000000in}{-0.048611in}}{\pgfqpoint{0.000000in}{0.000000in}}{%
\pgfpathmoveto{\pgfqpoint{0.000000in}{0.000000in}}%
\pgfpathlineto{\pgfqpoint{0.000000in}{-0.048611in}}%
\pgfusepath{stroke,fill}%
}%
\begin{pgfscope}%
\pgfsys@transformshift{0.848324in}{0.548769in}%
\pgfsys@useobject{currentmarker}{}%
\end{pgfscope}%
\end{pgfscope}%
\begin{pgfscope}%
\definecolor{textcolor}{rgb}{0.000000,0.000000,0.000000}%
\pgfsetstrokecolor{textcolor}%
\pgfsetfillcolor{textcolor}%
\pgftext[x=0.848324in,y=0.451547in,,top]{\color{textcolor}\sffamily\fontsize{10.000000}{12.000000}\selectfont \(\displaystyle {0}\)}%
\end{pgfscope}%
\begin{pgfscope}%
\pgfsetbuttcap%
\pgfsetroundjoin%
\definecolor{currentfill}{rgb}{0.000000,0.000000,0.000000}%
\pgfsetfillcolor{currentfill}%
\pgfsetlinewidth{0.803000pt}%
\definecolor{currentstroke}{rgb}{0.000000,0.000000,0.000000}%
\pgfsetstrokecolor{currentstroke}%
\pgfsetdash{}{0pt}%
\pgfsys@defobject{currentmarker}{\pgfqpoint{0.000000in}{-0.048611in}}{\pgfqpoint{0.000000in}{0.000000in}}{%
\pgfpathmoveto{\pgfqpoint{0.000000in}{0.000000in}}%
\pgfpathlineto{\pgfqpoint{0.000000in}{-0.048611in}}%
\pgfusepath{stroke,fill}%
}%
\begin{pgfscope}%
\pgfsys@transformshift{1.719382in}{0.548769in}%
\pgfsys@useobject{currentmarker}{}%
\end{pgfscope}%
\end{pgfscope}%
\begin{pgfscope}%
\definecolor{textcolor}{rgb}{0.000000,0.000000,0.000000}%
\pgfsetstrokecolor{textcolor}%
\pgfsetfillcolor{textcolor}%
\pgftext[x=1.719382in,y=0.451547in,,top]{\color{textcolor}\sffamily\fontsize{10.000000}{12.000000}\selectfont \(\displaystyle {100000}\)}%
\end{pgfscope}%
\begin{pgfscope}%
\pgfsetbuttcap%
\pgfsetroundjoin%
\definecolor{currentfill}{rgb}{0.000000,0.000000,0.000000}%
\pgfsetfillcolor{currentfill}%
\pgfsetlinewidth{0.803000pt}%
\definecolor{currentstroke}{rgb}{0.000000,0.000000,0.000000}%
\pgfsetstrokecolor{currentstroke}%
\pgfsetdash{}{0pt}%
\pgfsys@defobject{currentmarker}{\pgfqpoint{0.000000in}{-0.048611in}}{\pgfqpoint{0.000000in}{0.000000in}}{%
\pgfpathmoveto{\pgfqpoint{0.000000in}{0.000000in}}%
\pgfpathlineto{\pgfqpoint{0.000000in}{-0.048611in}}%
\pgfusepath{stroke,fill}%
}%
\begin{pgfscope}%
\pgfsys@transformshift{2.590440in}{0.548769in}%
\pgfsys@useobject{currentmarker}{}%
\end{pgfscope}%
\end{pgfscope}%
\begin{pgfscope}%
\definecolor{textcolor}{rgb}{0.000000,0.000000,0.000000}%
\pgfsetstrokecolor{textcolor}%
\pgfsetfillcolor{textcolor}%
\pgftext[x=2.590440in,y=0.451547in,,top]{\color{textcolor}\sffamily\fontsize{10.000000}{12.000000}\selectfont \(\displaystyle {200000}\)}%
\end{pgfscope}%
\begin{pgfscope}%
\pgfsetbuttcap%
\pgfsetroundjoin%
\definecolor{currentfill}{rgb}{0.000000,0.000000,0.000000}%
\pgfsetfillcolor{currentfill}%
\pgfsetlinewidth{0.803000pt}%
\definecolor{currentstroke}{rgb}{0.000000,0.000000,0.000000}%
\pgfsetstrokecolor{currentstroke}%
\pgfsetdash{}{0pt}%
\pgfsys@defobject{currentmarker}{\pgfqpoint{0.000000in}{-0.048611in}}{\pgfqpoint{0.000000in}{0.000000in}}{%
\pgfpathmoveto{\pgfqpoint{0.000000in}{0.000000in}}%
\pgfpathlineto{\pgfqpoint{0.000000in}{-0.048611in}}%
\pgfusepath{stroke,fill}%
}%
\begin{pgfscope}%
\pgfsys@transformshift{3.461498in}{0.548769in}%
\pgfsys@useobject{currentmarker}{}%
\end{pgfscope}%
\end{pgfscope}%
\begin{pgfscope}%
\definecolor{textcolor}{rgb}{0.000000,0.000000,0.000000}%
\pgfsetstrokecolor{textcolor}%
\pgfsetfillcolor{textcolor}%
\pgftext[x=3.461498in,y=0.451547in,,top]{\color{textcolor}\sffamily\fontsize{10.000000}{12.000000}\selectfont \(\displaystyle {300000}\)}%
\end{pgfscope}%
\begin{pgfscope}%
\pgfsetbuttcap%
\pgfsetroundjoin%
\definecolor{currentfill}{rgb}{0.000000,0.000000,0.000000}%
\pgfsetfillcolor{currentfill}%
\pgfsetlinewidth{0.803000pt}%
\definecolor{currentstroke}{rgb}{0.000000,0.000000,0.000000}%
\pgfsetstrokecolor{currentstroke}%
\pgfsetdash{}{0pt}%
\pgfsys@defobject{currentmarker}{\pgfqpoint{0.000000in}{-0.048611in}}{\pgfqpoint{0.000000in}{0.000000in}}{%
\pgfpathmoveto{\pgfqpoint{0.000000in}{0.000000in}}%
\pgfpathlineto{\pgfqpoint{0.000000in}{-0.048611in}}%
\pgfusepath{stroke,fill}%
}%
\begin{pgfscope}%
\pgfsys@transformshift{4.332556in}{0.548769in}%
\pgfsys@useobject{currentmarker}{}%
\end{pgfscope}%
\end{pgfscope}%
\begin{pgfscope}%
\definecolor{textcolor}{rgb}{0.000000,0.000000,0.000000}%
\pgfsetstrokecolor{textcolor}%
\pgfsetfillcolor{textcolor}%
\pgftext[x=4.332556in,y=0.451547in,,top]{\color{textcolor}\sffamily\fontsize{10.000000}{12.000000}\selectfont \(\displaystyle {400000}\)}%
\end{pgfscope}%
\begin{pgfscope}%
\pgfsetbuttcap%
\pgfsetroundjoin%
\definecolor{currentfill}{rgb}{0.000000,0.000000,0.000000}%
\pgfsetfillcolor{currentfill}%
\pgfsetlinewidth{0.803000pt}%
\definecolor{currentstroke}{rgb}{0.000000,0.000000,0.000000}%
\pgfsetstrokecolor{currentstroke}%
\pgfsetdash{}{0pt}%
\pgfsys@defobject{currentmarker}{\pgfqpoint{0.000000in}{-0.048611in}}{\pgfqpoint{0.000000in}{0.000000in}}{%
\pgfpathmoveto{\pgfqpoint{0.000000in}{0.000000in}}%
\pgfpathlineto{\pgfqpoint{0.000000in}{-0.048611in}}%
\pgfusepath{stroke,fill}%
}%
\begin{pgfscope}%
\pgfsys@transformshift{5.203614in}{0.548769in}%
\pgfsys@useobject{currentmarker}{}%
\end{pgfscope}%
\end{pgfscope}%
\begin{pgfscope}%
\definecolor{textcolor}{rgb}{0.000000,0.000000,0.000000}%
\pgfsetstrokecolor{textcolor}%
\pgfsetfillcolor{textcolor}%
\pgftext[x=5.203614in,y=0.451547in,,top]{\color{textcolor}\sffamily\fontsize{10.000000}{12.000000}\selectfont \(\displaystyle {500000}\)}%
\end{pgfscope}%
\begin{pgfscope}%
\definecolor{textcolor}{rgb}{0.000000,0.000000,0.000000}%
\pgfsetstrokecolor{textcolor}%
\pgfsetfillcolor{textcolor}%
\pgftext[x=3.249352in,y=0.272658in,,top]{\color{textcolor}\sffamily\fontsize{10.000000}{12.000000}\selectfont Methods}%
\end{pgfscope}%
\begin{pgfscope}%
\pgfsetbuttcap%
\pgfsetroundjoin%
\definecolor{currentfill}{rgb}{0.000000,0.000000,0.000000}%
\pgfsetfillcolor{currentfill}%
\pgfsetlinewidth{0.803000pt}%
\definecolor{currentstroke}{rgb}{0.000000,0.000000,0.000000}%
\pgfsetstrokecolor{currentstroke}%
\pgfsetdash{}{0pt}%
\pgfsys@defobject{currentmarker}{\pgfqpoint{-0.048611in}{0.000000in}}{\pgfqpoint{0.000000in}{0.000000in}}{%
\pgfpathmoveto{\pgfqpoint{0.000000in}{0.000000in}}%
\pgfpathlineto{\pgfqpoint{-0.048611in}{0.000000in}}%
\pgfusepath{stroke,fill}%
}%
\begin{pgfscope}%
\pgfsys@transformshift{0.648703in}{0.689796in}%
\pgfsys@useobject{currentmarker}{}%
\end{pgfscope}%
\end{pgfscope}%
\begin{pgfscope}%
\definecolor{textcolor}{rgb}{0.000000,0.000000,0.000000}%
\pgfsetstrokecolor{textcolor}%
\pgfsetfillcolor{textcolor}%
\pgftext[x=0.482036in, y=0.641601in, left, base]{\color{textcolor}\sffamily\fontsize{10.000000}{12.000000}\selectfont \(\displaystyle {0}\)}%
\end{pgfscope}%
\begin{pgfscope}%
\pgfsetbuttcap%
\pgfsetroundjoin%
\definecolor{currentfill}{rgb}{0.000000,0.000000,0.000000}%
\pgfsetfillcolor{currentfill}%
\pgfsetlinewidth{0.803000pt}%
\definecolor{currentstroke}{rgb}{0.000000,0.000000,0.000000}%
\pgfsetstrokecolor{currentstroke}%
\pgfsetdash{}{0pt}%
\pgfsys@defobject{currentmarker}{\pgfqpoint{-0.048611in}{0.000000in}}{\pgfqpoint{0.000000in}{0.000000in}}{%
\pgfpathmoveto{\pgfqpoint{0.000000in}{0.000000in}}%
\pgfpathlineto{\pgfqpoint{-0.048611in}{0.000000in}}%
\pgfusepath{stroke,fill}%
}%
\begin{pgfscope}%
\pgfsys@transformshift{0.648703in}{1.112665in}%
\pgfsys@useobject{currentmarker}{}%
\end{pgfscope}%
\end{pgfscope}%
\begin{pgfscope}%
\definecolor{textcolor}{rgb}{0.000000,0.000000,0.000000}%
\pgfsetstrokecolor{textcolor}%
\pgfsetfillcolor{textcolor}%
\pgftext[x=0.343147in, y=1.064470in, left, base]{\color{textcolor}\sffamily\fontsize{10.000000}{12.000000}\selectfont \(\displaystyle {100}\)}%
\end{pgfscope}%
\begin{pgfscope}%
\pgfsetbuttcap%
\pgfsetroundjoin%
\definecolor{currentfill}{rgb}{0.000000,0.000000,0.000000}%
\pgfsetfillcolor{currentfill}%
\pgfsetlinewidth{0.803000pt}%
\definecolor{currentstroke}{rgb}{0.000000,0.000000,0.000000}%
\pgfsetstrokecolor{currentstroke}%
\pgfsetdash{}{0pt}%
\pgfsys@defobject{currentmarker}{\pgfqpoint{-0.048611in}{0.000000in}}{\pgfqpoint{0.000000in}{0.000000in}}{%
\pgfpathmoveto{\pgfqpoint{0.000000in}{0.000000in}}%
\pgfpathlineto{\pgfqpoint{-0.048611in}{0.000000in}}%
\pgfusepath{stroke,fill}%
}%
\begin{pgfscope}%
\pgfsys@transformshift{0.648703in}{1.535534in}%
\pgfsys@useobject{currentmarker}{}%
\end{pgfscope}%
\end{pgfscope}%
\begin{pgfscope}%
\definecolor{textcolor}{rgb}{0.000000,0.000000,0.000000}%
\pgfsetstrokecolor{textcolor}%
\pgfsetfillcolor{textcolor}%
\pgftext[x=0.343147in, y=1.487339in, left, base]{\color{textcolor}\sffamily\fontsize{10.000000}{12.000000}\selectfont \(\displaystyle {200}\)}%
\end{pgfscope}%
\begin{pgfscope}%
\pgfsetbuttcap%
\pgfsetroundjoin%
\definecolor{currentfill}{rgb}{0.000000,0.000000,0.000000}%
\pgfsetfillcolor{currentfill}%
\pgfsetlinewidth{0.803000pt}%
\definecolor{currentstroke}{rgb}{0.000000,0.000000,0.000000}%
\pgfsetstrokecolor{currentstroke}%
\pgfsetdash{}{0pt}%
\pgfsys@defobject{currentmarker}{\pgfqpoint{-0.048611in}{0.000000in}}{\pgfqpoint{0.000000in}{0.000000in}}{%
\pgfpathmoveto{\pgfqpoint{0.000000in}{0.000000in}}%
\pgfpathlineto{\pgfqpoint{-0.048611in}{0.000000in}}%
\pgfusepath{stroke,fill}%
}%
\begin{pgfscope}%
\pgfsys@transformshift{0.648703in}{1.958403in}%
\pgfsys@useobject{currentmarker}{}%
\end{pgfscope}%
\end{pgfscope}%
\begin{pgfscope}%
\definecolor{textcolor}{rgb}{0.000000,0.000000,0.000000}%
\pgfsetstrokecolor{textcolor}%
\pgfsetfillcolor{textcolor}%
\pgftext[x=0.343147in, y=1.910208in, left, base]{\color{textcolor}\sffamily\fontsize{10.000000}{12.000000}\selectfont \(\displaystyle {300}\)}%
\end{pgfscope}%
\begin{pgfscope}%
\pgfsetbuttcap%
\pgfsetroundjoin%
\definecolor{currentfill}{rgb}{0.000000,0.000000,0.000000}%
\pgfsetfillcolor{currentfill}%
\pgfsetlinewidth{0.803000pt}%
\definecolor{currentstroke}{rgb}{0.000000,0.000000,0.000000}%
\pgfsetstrokecolor{currentstroke}%
\pgfsetdash{}{0pt}%
\pgfsys@defobject{currentmarker}{\pgfqpoint{-0.048611in}{0.000000in}}{\pgfqpoint{0.000000in}{0.000000in}}{%
\pgfpathmoveto{\pgfqpoint{0.000000in}{0.000000in}}%
\pgfpathlineto{\pgfqpoint{-0.048611in}{0.000000in}}%
\pgfusepath{stroke,fill}%
}%
\begin{pgfscope}%
\pgfsys@transformshift{0.648703in}{2.381272in}%
\pgfsys@useobject{currentmarker}{}%
\end{pgfscope}%
\end{pgfscope}%
\begin{pgfscope}%
\definecolor{textcolor}{rgb}{0.000000,0.000000,0.000000}%
\pgfsetstrokecolor{textcolor}%
\pgfsetfillcolor{textcolor}%
\pgftext[x=0.343147in, y=2.333077in, left, base]{\color{textcolor}\sffamily\fontsize{10.000000}{12.000000}\selectfont \(\displaystyle {400}\)}%
\end{pgfscope}%
\begin{pgfscope}%
\pgfsetbuttcap%
\pgfsetroundjoin%
\definecolor{currentfill}{rgb}{0.000000,0.000000,0.000000}%
\pgfsetfillcolor{currentfill}%
\pgfsetlinewidth{0.803000pt}%
\definecolor{currentstroke}{rgb}{0.000000,0.000000,0.000000}%
\pgfsetstrokecolor{currentstroke}%
\pgfsetdash{}{0pt}%
\pgfsys@defobject{currentmarker}{\pgfqpoint{-0.048611in}{0.000000in}}{\pgfqpoint{0.000000in}{0.000000in}}{%
\pgfpathmoveto{\pgfqpoint{0.000000in}{0.000000in}}%
\pgfpathlineto{\pgfqpoint{-0.048611in}{0.000000in}}%
\pgfusepath{stroke,fill}%
}%
\begin{pgfscope}%
\pgfsys@transformshift{0.648703in}{2.804141in}%
\pgfsys@useobject{currentmarker}{}%
\end{pgfscope}%
\end{pgfscope}%
\begin{pgfscope}%
\definecolor{textcolor}{rgb}{0.000000,0.000000,0.000000}%
\pgfsetstrokecolor{textcolor}%
\pgfsetfillcolor{textcolor}%
\pgftext[x=0.343147in, y=2.755946in, left, base]{\color{textcolor}\sffamily\fontsize{10.000000}{12.000000}\selectfont \(\displaystyle {500}\)}%
\end{pgfscope}%
\begin{pgfscope}%
\pgfsetbuttcap%
\pgfsetroundjoin%
\definecolor{currentfill}{rgb}{0.000000,0.000000,0.000000}%
\pgfsetfillcolor{currentfill}%
\pgfsetlinewidth{0.803000pt}%
\definecolor{currentstroke}{rgb}{0.000000,0.000000,0.000000}%
\pgfsetstrokecolor{currentstroke}%
\pgfsetdash{}{0pt}%
\pgfsys@defobject{currentmarker}{\pgfqpoint{-0.048611in}{0.000000in}}{\pgfqpoint{0.000000in}{0.000000in}}{%
\pgfpathmoveto{\pgfqpoint{0.000000in}{0.000000in}}%
\pgfpathlineto{\pgfqpoint{-0.048611in}{0.000000in}}%
\pgfusepath{stroke,fill}%
}%
\begin{pgfscope}%
\pgfsys@transformshift{0.648703in}{3.227010in}%
\pgfsys@useobject{currentmarker}{}%
\end{pgfscope}%
\end{pgfscope}%
\begin{pgfscope}%
\definecolor{textcolor}{rgb}{0.000000,0.000000,0.000000}%
\pgfsetstrokecolor{textcolor}%
\pgfsetfillcolor{textcolor}%
\pgftext[x=0.343147in, y=3.178815in, left, base]{\color{textcolor}\sffamily\fontsize{10.000000}{12.000000}\selectfont \(\displaystyle {600}\)}%
\end{pgfscope}%
\begin{pgfscope}%
\pgfsetbuttcap%
\pgfsetroundjoin%
\definecolor{currentfill}{rgb}{0.000000,0.000000,0.000000}%
\pgfsetfillcolor{currentfill}%
\pgfsetlinewidth{0.803000pt}%
\definecolor{currentstroke}{rgb}{0.000000,0.000000,0.000000}%
\pgfsetstrokecolor{currentstroke}%
\pgfsetdash{}{0pt}%
\pgfsys@defobject{currentmarker}{\pgfqpoint{-0.048611in}{0.000000in}}{\pgfqpoint{0.000000in}{0.000000in}}{%
\pgfpathmoveto{\pgfqpoint{0.000000in}{0.000000in}}%
\pgfpathlineto{\pgfqpoint{-0.048611in}{0.000000in}}%
\pgfusepath{stroke,fill}%
}%
\begin{pgfscope}%
\pgfsys@transformshift{0.648703in}{3.649879in}%
\pgfsys@useobject{currentmarker}{}%
\end{pgfscope}%
\end{pgfscope}%
\begin{pgfscope}%
\definecolor{textcolor}{rgb}{0.000000,0.000000,0.000000}%
\pgfsetstrokecolor{textcolor}%
\pgfsetfillcolor{textcolor}%
\pgftext[x=0.343147in, y=3.601684in, left, base]{\color{textcolor}\sffamily\fontsize{10.000000}{12.000000}\selectfont \(\displaystyle {700}\)}%
\end{pgfscope}%
\begin{pgfscope}%
\definecolor{textcolor}{rgb}{0.000000,0.000000,0.000000}%
\pgfsetstrokecolor{textcolor}%
\pgfsetfillcolor{textcolor}%
\pgftext[x=0.287592in,y=2.100064in,,bottom,rotate=90.000000]{\color{textcolor}\sffamily\fontsize{10.000000}{12.000000}\selectfont Data Flow Time (s)}%
\end{pgfscope}%
\begin{pgfscope}%
\pgfsetrectcap%
\pgfsetmiterjoin%
\pgfsetlinewidth{0.803000pt}%
\definecolor{currentstroke}{rgb}{0.000000,0.000000,0.000000}%
\pgfsetstrokecolor{currentstroke}%
\pgfsetdash{}{0pt}%
\pgfpathmoveto{\pgfqpoint{0.648703in}{0.548769in}}%
\pgfpathlineto{\pgfqpoint{0.648703in}{3.651359in}}%
\pgfusepath{stroke}%
\end{pgfscope}%
\begin{pgfscope}%
\pgfsetrectcap%
\pgfsetmiterjoin%
\pgfsetlinewidth{0.803000pt}%
\definecolor{currentstroke}{rgb}{0.000000,0.000000,0.000000}%
\pgfsetstrokecolor{currentstroke}%
\pgfsetdash{}{0pt}%
\pgfpathmoveto{\pgfqpoint{5.850000in}{0.548769in}}%
\pgfpathlineto{\pgfqpoint{5.850000in}{3.651359in}}%
\pgfusepath{stroke}%
\end{pgfscope}%
\begin{pgfscope}%
\pgfsetrectcap%
\pgfsetmiterjoin%
\pgfsetlinewidth{0.803000pt}%
\definecolor{currentstroke}{rgb}{0.000000,0.000000,0.000000}%
\pgfsetstrokecolor{currentstroke}%
\pgfsetdash{}{0pt}%
\pgfpathmoveto{\pgfqpoint{0.648703in}{0.548769in}}%
\pgfpathlineto{\pgfqpoint{5.850000in}{0.548769in}}%
\pgfusepath{stroke}%
\end{pgfscope}%
\begin{pgfscope}%
\pgfsetrectcap%
\pgfsetmiterjoin%
\pgfsetlinewidth{0.803000pt}%
\definecolor{currentstroke}{rgb}{0.000000,0.000000,0.000000}%
\pgfsetstrokecolor{currentstroke}%
\pgfsetdash{}{0pt}%
\pgfpathmoveto{\pgfqpoint{0.648703in}{3.651359in}}%
\pgfpathlineto{\pgfqpoint{5.850000in}{3.651359in}}%
\pgfusepath{stroke}%
\end{pgfscope}%
\begin{pgfscope}%
\definecolor{textcolor}{rgb}{0.000000,0.000000,0.000000}%
\pgfsetstrokecolor{textcolor}%
\pgfsetfillcolor{textcolor}%
\pgftext[x=3.249352in,y=3.734692in,,base]{\color{textcolor}\sffamily\fontsize{12.000000}{14.400000}\selectfont Backward}%
\end{pgfscope}%
\begin{pgfscope}%
\pgfsetbuttcap%
\pgfsetmiterjoin%
\definecolor{currentfill}{rgb}{1.000000,1.000000,1.000000}%
\pgfsetfillcolor{currentfill}%
\pgfsetfillopacity{0.800000}%
\pgfsetlinewidth{1.003750pt}%
\definecolor{currentstroke}{rgb}{0.800000,0.800000,0.800000}%
\pgfsetstrokecolor{currentstroke}%
\pgfsetstrokeopacity{0.800000}%
\pgfsetdash{}{0pt}%
\pgfpathmoveto{\pgfqpoint{4.300417in}{1.788050in}}%
\pgfpathlineto{\pgfqpoint{5.752778in}{1.788050in}}%
\pgfpathquadraticcurveto{\pgfqpoint{5.780556in}{1.788050in}}{\pgfqpoint{5.780556in}{1.815828in}}%
\pgfpathlineto{\pgfqpoint{5.780556in}{2.384300in}}%
\pgfpathquadraticcurveto{\pgfqpoint{5.780556in}{2.412078in}}{\pgfqpoint{5.752778in}{2.412078in}}%
\pgfpathlineto{\pgfqpoint{4.300417in}{2.412078in}}%
\pgfpathquadraticcurveto{\pgfqpoint{4.272639in}{2.412078in}}{\pgfqpoint{4.272639in}{2.384300in}}%
\pgfpathlineto{\pgfqpoint{4.272639in}{1.815828in}}%
\pgfpathquadraticcurveto{\pgfqpoint{4.272639in}{1.788050in}}{\pgfqpoint{4.300417in}{1.788050in}}%
\pgfpathclose%
\pgfusepath{stroke,fill}%
\end{pgfscope}%
\begin{pgfscope}%
\pgfsetbuttcap%
\pgfsetroundjoin%
\definecolor{currentfill}{rgb}{0.121569,0.466667,0.705882}%
\pgfsetfillcolor{currentfill}%
\pgfsetlinewidth{1.003750pt}%
\definecolor{currentstroke}{rgb}{0.121569,0.466667,0.705882}%
\pgfsetstrokecolor{currentstroke}%
\pgfsetdash{}{0pt}%
\pgfsys@defobject{currentmarker}{\pgfqpoint{-0.034722in}{-0.034722in}}{\pgfqpoint{0.034722in}{0.034722in}}{%
\pgfpathmoveto{\pgfqpoint{0.000000in}{-0.034722in}}%
\pgfpathcurveto{\pgfqpoint{0.009208in}{-0.034722in}}{\pgfqpoint{0.018041in}{-0.031064in}}{\pgfqpoint{0.024552in}{-0.024552in}}%
\pgfpathcurveto{\pgfqpoint{0.031064in}{-0.018041in}}{\pgfqpoint{0.034722in}{-0.009208in}}{\pgfqpoint{0.034722in}{0.000000in}}%
\pgfpathcurveto{\pgfqpoint{0.034722in}{0.009208in}}{\pgfqpoint{0.031064in}{0.018041in}}{\pgfqpoint{0.024552in}{0.024552in}}%
\pgfpathcurveto{\pgfqpoint{0.018041in}{0.031064in}}{\pgfqpoint{0.009208in}{0.034722in}}{\pgfqpoint{0.000000in}{0.034722in}}%
\pgfpathcurveto{\pgfqpoint{-0.009208in}{0.034722in}}{\pgfqpoint{-0.018041in}{0.031064in}}{\pgfqpoint{-0.024552in}{0.024552in}}%
\pgfpathcurveto{\pgfqpoint{-0.031064in}{0.018041in}}{\pgfqpoint{-0.034722in}{0.009208in}}{\pgfqpoint{-0.034722in}{0.000000in}}%
\pgfpathcurveto{\pgfqpoint{-0.034722in}{-0.009208in}}{\pgfqpoint{-0.031064in}{-0.018041in}}{\pgfqpoint{-0.024552in}{-0.024552in}}%
\pgfpathcurveto{\pgfqpoint{-0.018041in}{-0.031064in}}{\pgfqpoint{-0.009208in}{-0.034722in}}{\pgfqpoint{0.000000in}{-0.034722in}}%
\pgfpathclose%
\pgfusepath{stroke,fill}%
}%
\begin{pgfscope}%
\pgfsys@transformshift{4.467083in}{2.307911in}%
\pgfsys@useobject{currentmarker}{}%
\end{pgfscope}%
\end{pgfscope}%
\begin{pgfscope}%
\definecolor{textcolor}{rgb}{0.000000,0.000000,0.000000}%
\pgfsetstrokecolor{textcolor}%
\pgfsetfillcolor{textcolor}%
\pgftext[x=4.717083in,y=2.259300in,left,base]{\color{textcolor}\sffamily\fontsize{10.000000}{12.000000}\selectfont No Timeout}%
\end{pgfscope}%
\begin{pgfscope}%
\pgfsetbuttcap%
\pgfsetroundjoin%
\definecolor{currentfill}{rgb}{1.000000,0.498039,0.054902}%
\pgfsetfillcolor{currentfill}%
\pgfsetlinewidth{1.003750pt}%
\definecolor{currentstroke}{rgb}{1.000000,0.498039,0.054902}%
\pgfsetstrokecolor{currentstroke}%
\pgfsetdash{}{0pt}%
\pgfsys@defobject{currentmarker}{\pgfqpoint{-0.034722in}{-0.034722in}}{\pgfqpoint{0.034722in}{0.034722in}}{%
\pgfpathmoveto{\pgfqpoint{0.000000in}{-0.034722in}}%
\pgfpathcurveto{\pgfqpoint{0.009208in}{-0.034722in}}{\pgfqpoint{0.018041in}{-0.031064in}}{\pgfqpoint{0.024552in}{-0.024552in}}%
\pgfpathcurveto{\pgfqpoint{0.031064in}{-0.018041in}}{\pgfqpoint{0.034722in}{-0.009208in}}{\pgfqpoint{0.034722in}{0.000000in}}%
\pgfpathcurveto{\pgfqpoint{0.034722in}{0.009208in}}{\pgfqpoint{0.031064in}{0.018041in}}{\pgfqpoint{0.024552in}{0.024552in}}%
\pgfpathcurveto{\pgfqpoint{0.018041in}{0.031064in}}{\pgfqpoint{0.009208in}{0.034722in}}{\pgfqpoint{0.000000in}{0.034722in}}%
\pgfpathcurveto{\pgfqpoint{-0.009208in}{0.034722in}}{\pgfqpoint{-0.018041in}{0.031064in}}{\pgfqpoint{-0.024552in}{0.024552in}}%
\pgfpathcurveto{\pgfqpoint{-0.031064in}{0.018041in}}{\pgfqpoint{-0.034722in}{0.009208in}}{\pgfqpoint{-0.034722in}{0.000000in}}%
\pgfpathcurveto{\pgfqpoint{-0.034722in}{-0.009208in}}{\pgfqpoint{-0.031064in}{-0.018041in}}{\pgfqpoint{-0.024552in}{-0.024552in}}%
\pgfpathcurveto{\pgfqpoint{-0.018041in}{-0.031064in}}{\pgfqpoint{-0.009208in}{-0.034722in}}{\pgfqpoint{0.000000in}{-0.034722in}}%
\pgfpathclose%
\pgfusepath{stroke,fill}%
}%
\begin{pgfscope}%
\pgfsys@transformshift{4.467083in}{2.114300in}%
\pgfsys@useobject{currentmarker}{}%
\end{pgfscope}%
\end{pgfscope}%
\begin{pgfscope}%
\definecolor{textcolor}{rgb}{0.000000,0.000000,0.000000}%
\pgfsetstrokecolor{textcolor}%
\pgfsetfillcolor{textcolor}%
\pgftext[x=4.717083in,y=2.065689in,left,base]{\color{textcolor}\sffamily\fontsize{10.000000}{12.000000}\selectfont Time Timeout}%
\end{pgfscope}%
\begin{pgfscope}%
\pgfsetbuttcap%
\pgfsetroundjoin%
\definecolor{currentfill}{rgb}{0.839216,0.152941,0.156863}%
\pgfsetfillcolor{currentfill}%
\pgfsetlinewidth{1.003750pt}%
\definecolor{currentstroke}{rgb}{0.839216,0.152941,0.156863}%
\pgfsetstrokecolor{currentstroke}%
\pgfsetdash{}{0pt}%
\pgfsys@defobject{currentmarker}{\pgfqpoint{-0.034722in}{-0.034722in}}{\pgfqpoint{0.034722in}{0.034722in}}{%
\pgfpathmoveto{\pgfqpoint{0.000000in}{-0.034722in}}%
\pgfpathcurveto{\pgfqpoint{0.009208in}{-0.034722in}}{\pgfqpoint{0.018041in}{-0.031064in}}{\pgfqpoint{0.024552in}{-0.024552in}}%
\pgfpathcurveto{\pgfqpoint{0.031064in}{-0.018041in}}{\pgfqpoint{0.034722in}{-0.009208in}}{\pgfqpoint{0.034722in}{0.000000in}}%
\pgfpathcurveto{\pgfqpoint{0.034722in}{0.009208in}}{\pgfqpoint{0.031064in}{0.018041in}}{\pgfqpoint{0.024552in}{0.024552in}}%
\pgfpathcurveto{\pgfqpoint{0.018041in}{0.031064in}}{\pgfqpoint{0.009208in}{0.034722in}}{\pgfqpoint{0.000000in}{0.034722in}}%
\pgfpathcurveto{\pgfqpoint{-0.009208in}{0.034722in}}{\pgfqpoint{-0.018041in}{0.031064in}}{\pgfqpoint{-0.024552in}{0.024552in}}%
\pgfpathcurveto{\pgfqpoint{-0.031064in}{0.018041in}}{\pgfqpoint{-0.034722in}{0.009208in}}{\pgfqpoint{-0.034722in}{0.000000in}}%
\pgfpathcurveto{\pgfqpoint{-0.034722in}{-0.009208in}}{\pgfqpoint{-0.031064in}{-0.018041in}}{\pgfqpoint{-0.024552in}{-0.024552in}}%
\pgfpathcurveto{\pgfqpoint{-0.018041in}{-0.031064in}}{\pgfqpoint{-0.009208in}{-0.034722in}}{\pgfqpoint{0.000000in}{-0.034722in}}%
\pgfpathclose%
\pgfusepath{stroke,fill}%
}%
\begin{pgfscope}%
\pgfsys@transformshift{4.467083in}{1.920689in}%
\pgfsys@useobject{currentmarker}{}%
\end{pgfscope}%
\end{pgfscope}%
\begin{pgfscope}%
\definecolor{textcolor}{rgb}{0.000000,0.000000,0.000000}%
\pgfsetstrokecolor{textcolor}%
\pgfsetfillcolor{textcolor}%
\pgftext[x=4.717083in,y=1.872078in,left,base]{\color{textcolor}\sffamily\fontsize{10.000000}{12.000000}\selectfont Memory Timeout}%
\end{pgfscope}%
\end{pgfpicture}%
\makeatother%
\endgroup%

                }
            \end{subfigure}
            \caption{Methods}
        \end{subfigure}
        \bigbreak
        \begin{subfigure}[b]{\textwidth}
            \centering
            \begin{subfigure}[]{0.45\textwidth}
                \centering
                \resizebox{\columnwidth}{!}{
                    %% Creator: Matplotlib, PGF backend
%%
%% To include the figure in your LaTeX document, write
%%   \input{<filename>.pgf}
%%
%% Make sure the required packages are loaded in your preamble
%%   \usepackage{pgf}
%%
%% and, on pdftex
%%   \usepackage[utf8]{inputenc}\DeclareUnicodeCharacter{2212}{-}
%%
%% or, on luatex and xetex
%%   \usepackage{unicode-math}
%%
%% Figures using additional raster images can only be included by \input if
%% they are in the same directory as the main LaTeX file. For loading figures
%% from other directories you can use the `import` package
%%   \usepackage{import}
%%
%% and then include the figures with
%%   \import{<path to file>}{<filename>.pgf}
%%
%% Matplotlib used the following preamble
%%   \usepackage{amsmath}
%%   \usepackage{fontspec}
%%
\begingroup%
\makeatletter%
\begin{pgfpicture}%
\pgfpathrectangle{\pgfpointorigin}{\pgfqpoint{6.000000in}{4.000000in}}%
\pgfusepath{use as bounding box, clip}%
\begin{pgfscope}%
\pgfsetbuttcap%
\pgfsetmiterjoin%
\definecolor{currentfill}{rgb}{1.000000,1.000000,1.000000}%
\pgfsetfillcolor{currentfill}%
\pgfsetlinewidth{0.000000pt}%
\definecolor{currentstroke}{rgb}{1.000000,1.000000,1.000000}%
\pgfsetstrokecolor{currentstroke}%
\pgfsetdash{}{0pt}%
\pgfpathmoveto{\pgfqpoint{0.000000in}{0.000000in}}%
\pgfpathlineto{\pgfqpoint{6.000000in}{0.000000in}}%
\pgfpathlineto{\pgfqpoint{6.000000in}{4.000000in}}%
\pgfpathlineto{\pgfqpoint{0.000000in}{4.000000in}}%
\pgfpathclose%
\pgfusepath{fill}%
\end{pgfscope}%
\begin{pgfscope}%
\pgfsetbuttcap%
\pgfsetmiterjoin%
\definecolor{currentfill}{rgb}{1.000000,1.000000,1.000000}%
\pgfsetfillcolor{currentfill}%
\pgfsetlinewidth{0.000000pt}%
\definecolor{currentstroke}{rgb}{0.000000,0.000000,0.000000}%
\pgfsetstrokecolor{currentstroke}%
\pgfsetstrokeopacity{0.000000}%
\pgfsetdash{}{0pt}%
\pgfpathmoveto{\pgfqpoint{0.648703in}{0.548769in}}%
\pgfpathlineto{\pgfqpoint{5.850000in}{0.548769in}}%
\pgfpathlineto{\pgfqpoint{5.850000in}{3.651359in}}%
\pgfpathlineto{\pgfqpoint{0.648703in}{3.651359in}}%
\pgfpathclose%
\pgfusepath{fill}%
\end{pgfscope}%
\begin{pgfscope}%
\pgfpathrectangle{\pgfqpoint{0.648703in}{0.548769in}}{\pgfqpoint{5.201297in}{3.102590in}}%
\pgfusepath{clip}%
\pgfsetbuttcap%
\pgfsetroundjoin%
\definecolor{currentfill}{rgb}{0.121569,0.466667,0.705882}%
\pgfsetfillcolor{currentfill}%
\pgfsetlinewidth{1.003750pt}%
\definecolor{currentstroke}{rgb}{0.121569,0.466667,0.705882}%
\pgfsetstrokecolor{currentstroke}%
\pgfsetdash{}{0pt}%
\pgfpathmoveto{\pgfqpoint{1.215563in}{0.648129in}}%
\pgfpathcurveto{\pgfqpoint{1.226613in}{0.648129in}}{\pgfqpoint{1.237213in}{0.652519in}}{\pgfqpoint{1.245026in}{0.660333in}}%
\pgfpathcurveto{\pgfqpoint{1.252840in}{0.668146in}}{\pgfqpoint{1.257230in}{0.678745in}}{\pgfqpoint{1.257230in}{0.689796in}}%
\pgfpathcurveto{\pgfqpoint{1.257230in}{0.700846in}}{\pgfqpoint{1.252840in}{0.711445in}}{\pgfqpoint{1.245026in}{0.719258in}}%
\pgfpathcurveto{\pgfqpoint{1.237213in}{0.727072in}}{\pgfqpoint{1.226613in}{0.731462in}}{\pgfqpoint{1.215563in}{0.731462in}}%
\pgfpathcurveto{\pgfqpoint{1.204513in}{0.731462in}}{\pgfqpoint{1.193914in}{0.727072in}}{\pgfqpoint{1.186101in}{0.719258in}}%
\pgfpathcurveto{\pgfqpoint{1.178287in}{0.711445in}}{\pgfqpoint{1.173897in}{0.700846in}}{\pgfqpoint{1.173897in}{0.689796in}}%
\pgfpathcurveto{\pgfqpoint{1.173897in}{0.678745in}}{\pgfqpoint{1.178287in}{0.668146in}}{\pgfqpoint{1.186101in}{0.660333in}}%
\pgfpathcurveto{\pgfqpoint{1.193914in}{0.652519in}}{\pgfqpoint{1.204513in}{0.648129in}}{\pgfqpoint{1.215563in}{0.648129in}}%
\pgfpathclose%
\pgfusepath{stroke,fill}%
\end{pgfscope}%
\begin{pgfscope}%
\pgfpathrectangle{\pgfqpoint{0.648703in}{0.548769in}}{\pgfqpoint{5.201297in}{3.102590in}}%
\pgfusepath{clip}%
\pgfsetbuttcap%
\pgfsetroundjoin%
\definecolor{currentfill}{rgb}{0.121569,0.466667,0.705882}%
\pgfsetfillcolor{currentfill}%
\pgfsetlinewidth{1.003750pt}%
\definecolor{currentstroke}{rgb}{0.121569,0.466667,0.705882}%
\pgfsetstrokecolor{currentstroke}%
\pgfsetdash{}{0pt}%
\pgfpathmoveto{\pgfqpoint{3.894777in}{3.124394in}}%
\pgfpathcurveto{\pgfqpoint{3.905827in}{3.124394in}}{\pgfqpoint{3.916426in}{3.128784in}}{\pgfqpoint{3.924240in}{3.136598in}}%
\pgfpathcurveto{\pgfqpoint{3.932053in}{3.144411in}}{\pgfqpoint{3.936444in}{3.155010in}}{\pgfqpoint{3.936444in}{3.166060in}}%
\pgfpathcurveto{\pgfqpoint{3.936444in}{3.177111in}}{\pgfqpoint{3.932053in}{3.187710in}}{\pgfqpoint{3.924240in}{3.195523in}}%
\pgfpathcurveto{\pgfqpoint{3.916426in}{3.203337in}}{\pgfqpoint{3.905827in}{3.207727in}}{\pgfqpoint{3.894777in}{3.207727in}}%
\pgfpathcurveto{\pgfqpoint{3.883727in}{3.207727in}}{\pgfqpoint{3.873128in}{3.203337in}}{\pgfqpoint{3.865314in}{3.195523in}}%
\pgfpathcurveto{\pgfqpoint{3.857501in}{3.187710in}}{\pgfqpoint{3.853110in}{3.177111in}}{\pgfqpoint{3.853110in}{3.166060in}}%
\pgfpathcurveto{\pgfqpoint{3.853110in}{3.155010in}}{\pgfqpoint{3.857501in}{3.144411in}}{\pgfqpoint{3.865314in}{3.136598in}}%
\pgfpathcurveto{\pgfqpoint{3.873128in}{3.128784in}}{\pgfqpoint{3.883727in}{3.124394in}}{\pgfqpoint{3.894777in}{3.124394in}}%
\pgfpathclose%
\pgfusepath{stroke,fill}%
\end{pgfscope}%
\begin{pgfscope}%
\pgfpathrectangle{\pgfqpoint{0.648703in}{0.548769in}}{\pgfqpoint{5.201297in}{3.102590in}}%
\pgfusepath{clip}%
\pgfsetbuttcap%
\pgfsetroundjoin%
\definecolor{currentfill}{rgb}{1.000000,0.498039,0.054902}%
\pgfsetfillcolor{currentfill}%
\pgfsetlinewidth{1.003750pt}%
\definecolor{currentstroke}{rgb}{1.000000,0.498039,0.054902}%
\pgfsetstrokecolor{currentstroke}%
\pgfsetdash{}{0pt}%
\pgfpathmoveto{\pgfqpoint{1.224080in}{3.140985in}}%
\pgfpathcurveto{\pgfqpoint{1.235130in}{3.140985in}}{\pgfqpoint{1.245729in}{3.145375in}}{\pgfqpoint{1.253542in}{3.153189in}}%
\pgfpathcurveto{\pgfqpoint{1.261356in}{3.161003in}}{\pgfqpoint{1.265746in}{3.171602in}}{\pgfqpoint{1.265746in}{3.182652in}}%
\pgfpathcurveto{\pgfqpoint{1.265746in}{3.193702in}}{\pgfqpoint{1.261356in}{3.204301in}}{\pgfqpoint{1.253542in}{3.212115in}}%
\pgfpathcurveto{\pgfqpoint{1.245729in}{3.219928in}}{\pgfqpoint{1.235130in}{3.224319in}}{\pgfqpoint{1.224080in}{3.224319in}}%
\pgfpathcurveto{\pgfqpoint{1.213029in}{3.224319in}}{\pgfqpoint{1.202430in}{3.219928in}}{\pgfqpoint{1.194617in}{3.212115in}}%
\pgfpathcurveto{\pgfqpoint{1.186803in}{3.204301in}}{\pgfqpoint{1.182413in}{3.193702in}}{\pgfqpoint{1.182413in}{3.182652in}}%
\pgfpathcurveto{\pgfqpoint{1.182413in}{3.171602in}}{\pgfqpoint{1.186803in}{3.161003in}}{\pgfqpoint{1.194617in}{3.153189in}}%
\pgfpathcurveto{\pgfqpoint{1.202430in}{3.145375in}}{\pgfqpoint{1.213029in}{3.140985in}}{\pgfqpoint{1.224080in}{3.140985in}}%
\pgfpathclose%
\pgfusepath{stroke,fill}%
\end{pgfscope}%
\begin{pgfscope}%
\pgfpathrectangle{\pgfqpoint{0.648703in}{0.548769in}}{\pgfqpoint{5.201297in}{3.102590in}}%
\pgfusepath{clip}%
\pgfsetbuttcap%
\pgfsetroundjoin%
\definecolor{currentfill}{rgb}{0.121569,0.466667,0.705882}%
\pgfsetfillcolor{currentfill}%
\pgfsetlinewidth{1.003750pt}%
\definecolor{currentstroke}{rgb}{0.121569,0.466667,0.705882}%
\pgfsetstrokecolor{currentstroke}%
\pgfsetdash{}{0pt}%
\pgfpathmoveto{\pgfqpoint{2.495891in}{3.132690in}}%
\pgfpathcurveto{\pgfqpoint{2.506941in}{3.132690in}}{\pgfqpoint{2.517540in}{3.137080in}}{\pgfqpoint{2.525354in}{3.144893in}}%
\pgfpathcurveto{\pgfqpoint{2.533168in}{3.152707in}}{\pgfqpoint{2.537558in}{3.163306in}}{\pgfqpoint{2.537558in}{3.174356in}}%
\pgfpathcurveto{\pgfqpoint{2.537558in}{3.185406in}}{\pgfqpoint{2.533168in}{3.196005in}}{\pgfqpoint{2.525354in}{3.203819in}}%
\pgfpathcurveto{\pgfqpoint{2.517540in}{3.211633in}}{\pgfqpoint{2.506941in}{3.216023in}}{\pgfqpoint{2.495891in}{3.216023in}}%
\pgfpathcurveto{\pgfqpoint{2.484841in}{3.216023in}}{\pgfqpoint{2.474242in}{3.211633in}}{\pgfqpoint{2.466428in}{3.203819in}}%
\pgfpathcurveto{\pgfqpoint{2.458615in}{3.196005in}}{\pgfqpoint{2.454225in}{3.185406in}}{\pgfqpoint{2.454225in}{3.174356in}}%
\pgfpathcurveto{\pgfqpoint{2.454225in}{3.163306in}}{\pgfqpoint{2.458615in}{3.152707in}}{\pgfqpoint{2.466428in}{3.144893in}}%
\pgfpathcurveto{\pgfqpoint{2.474242in}{3.137080in}}{\pgfqpoint{2.484841in}{3.132690in}}{\pgfqpoint{2.495891in}{3.132690in}}%
\pgfpathclose%
\pgfusepath{stroke,fill}%
\end{pgfscope}%
\begin{pgfscope}%
\pgfpathrectangle{\pgfqpoint{0.648703in}{0.548769in}}{\pgfqpoint{5.201297in}{3.102590in}}%
\pgfusepath{clip}%
\pgfsetbuttcap%
\pgfsetroundjoin%
\definecolor{currentfill}{rgb}{1.000000,0.498039,0.054902}%
\pgfsetfillcolor{currentfill}%
\pgfsetlinewidth{1.003750pt}%
\definecolor{currentstroke}{rgb}{1.000000,0.498039,0.054902}%
\pgfsetstrokecolor{currentstroke}%
\pgfsetdash{}{0pt}%
\pgfpathmoveto{\pgfqpoint{2.644323in}{3.136837in}}%
\pgfpathcurveto{\pgfqpoint{2.655373in}{3.136837in}}{\pgfqpoint{2.665972in}{3.141228in}}{\pgfqpoint{2.673785in}{3.149041in}}%
\pgfpathcurveto{\pgfqpoint{2.681599in}{3.156855in}}{\pgfqpoint{2.685989in}{3.167454in}}{\pgfqpoint{2.685989in}{3.178504in}}%
\pgfpathcurveto{\pgfqpoint{2.685989in}{3.189554in}}{\pgfqpoint{2.681599in}{3.200153in}}{\pgfqpoint{2.673785in}{3.207967in}}%
\pgfpathcurveto{\pgfqpoint{2.665972in}{3.215780in}}{\pgfqpoint{2.655373in}{3.220171in}}{\pgfqpoint{2.644323in}{3.220171in}}%
\pgfpathcurveto{\pgfqpoint{2.633273in}{3.220171in}}{\pgfqpoint{2.622674in}{3.215780in}}{\pgfqpoint{2.614860in}{3.207967in}}%
\pgfpathcurveto{\pgfqpoint{2.607046in}{3.200153in}}{\pgfqpoint{2.602656in}{3.189554in}}{\pgfqpoint{2.602656in}{3.178504in}}%
\pgfpathcurveto{\pgfqpoint{2.602656in}{3.167454in}}{\pgfqpoint{2.607046in}{3.156855in}}{\pgfqpoint{2.614860in}{3.149041in}}%
\pgfpathcurveto{\pgfqpoint{2.622674in}{3.141228in}}{\pgfqpoint{2.633273in}{3.136837in}}{\pgfqpoint{2.644323in}{3.136837in}}%
\pgfpathclose%
\pgfusepath{stroke,fill}%
\end{pgfscope}%
\begin{pgfscope}%
\pgfpathrectangle{\pgfqpoint{0.648703in}{0.548769in}}{\pgfqpoint{5.201297in}{3.102590in}}%
\pgfusepath{clip}%
\pgfsetbuttcap%
\pgfsetroundjoin%
\definecolor{currentfill}{rgb}{0.121569,0.466667,0.705882}%
\pgfsetfillcolor{currentfill}%
\pgfsetlinewidth{1.003750pt}%
\definecolor{currentstroke}{rgb}{0.121569,0.466667,0.705882}%
\pgfsetstrokecolor{currentstroke}%
\pgfsetdash{}{0pt}%
\pgfpathmoveto{\pgfqpoint{2.110923in}{3.132690in}}%
\pgfpathcurveto{\pgfqpoint{2.121974in}{3.132690in}}{\pgfqpoint{2.132573in}{3.137080in}}{\pgfqpoint{2.140386in}{3.144893in}}%
\pgfpathcurveto{\pgfqpoint{2.148200in}{3.152707in}}{\pgfqpoint{2.152590in}{3.163306in}}{\pgfqpoint{2.152590in}{3.174356in}}%
\pgfpathcurveto{\pgfqpoint{2.152590in}{3.185406in}}{\pgfqpoint{2.148200in}{3.196005in}}{\pgfqpoint{2.140386in}{3.203819in}}%
\pgfpathcurveto{\pgfqpoint{2.132573in}{3.211633in}}{\pgfqpoint{2.121974in}{3.216023in}}{\pgfqpoint{2.110923in}{3.216023in}}%
\pgfpathcurveto{\pgfqpoint{2.099873in}{3.216023in}}{\pgfqpoint{2.089274in}{3.211633in}}{\pgfqpoint{2.081461in}{3.203819in}}%
\pgfpathcurveto{\pgfqpoint{2.073647in}{3.196005in}}{\pgfqpoint{2.069257in}{3.185406in}}{\pgfqpoint{2.069257in}{3.174356in}}%
\pgfpathcurveto{\pgfqpoint{2.069257in}{3.163306in}}{\pgfqpoint{2.073647in}{3.152707in}}{\pgfqpoint{2.081461in}{3.144893in}}%
\pgfpathcurveto{\pgfqpoint{2.089274in}{3.137080in}}{\pgfqpoint{2.099873in}{3.132690in}}{\pgfqpoint{2.110923in}{3.132690in}}%
\pgfpathclose%
\pgfusepath{stroke,fill}%
\end{pgfscope}%
\begin{pgfscope}%
\pgfpathrectangle{\pgfqpoint{0.648703in}{0.548769in}}{\pgfqpoint{5.201297in}{3.102590in}}%
\pgfusepath{clip}%
\pgfsetbuttcap%
\pgfsetroundjoin%
\definecolor{currentfill}{rgb}{0.121569,0.466667,0.705882}%
\pgfsetfillcolor{currentfill}%
\pgfsetlinewidth{1.003750pt}%
\definecolor{currentstroke}{rgb}{0.121569,0.466667,0.705882}%
\pgfsetstrokecolor{currentstroke}%
\pgfsetdash{}{0pt}%
\pgfpathmoveto{\pgfqpoint{2.253781in}{3.128542in}}%
\pgfpathcurveto{\pgfqpoint{2.264832in}{3.128542in}}{\pgfqpoint{2.275431in}{3.132932in}}{\pgfqpoint{2.283244in}{3.140746in}}%
\pgfpathcurveto{\pgfqpoint{2.291058in}{3.148559in}}{\pgfqpoint{2.295448in}{3.159158in}}{\pgfqpoint{2.295448in}{3.170208in}}%
\pgfpathcurveto{\pgfqpoint{2.295448in}{3.181258in}}{\pgfqpoint{2.291058in}{3.191857in}}{\pgfqpoint{2.283244in}{3.199671in}}%
\pgfpathcurveto{\pgfqpoint{2.275431in}{3.207485in}}{\pgfqpoint{2.264832in}{3.211875in}}{\pgfqpoint{2.253781in}{3.211875in}}%
\pgfpathcurveto{\pgfqpoint{2.242731in}{3.211875in}}{\pgfqpoint{2.232132in}{3.207485in}}{\pgfqpoint{2.224319in}{3.199671in}}%
\pgfpathcurveto{\pgfqpoint{2.216505in}{3.191857in}}{\pgfqpoint{2.212115in}{3.181258in}}{\pgfqpoint{2.212115in}{3.170208in}}%
\pgfpathcurveto{\pgfqpoint{2.212115in}{3.159158in}}{\pgfqpoint{2.216505in}{3.148559in}}{\pgfqpoint{2.224319in}{3.140746in}}%
\pgfpathcurveto{\pgfqpoint{2.232132in}{3.132932in}}{\pgfqpoint{2.242731in}{3.128542in}}{\pgfqpoint{2.253781in}{3.128542in}}%
\pgfpathclose%
\pgfusepath{stroke,fill}%
\end{pgfscope}%
\begin{pgfscope}%
\pgfpathrectangle{\pgfqpoint{0.648703in}{0.548769in}}{\pgfqpoint{5.201297in}{3.102590in}}%
\pgfusepath{clip}%
\pgfsetbuttcap%
\pgfsetroundjoin%
\definecolor{currentfill}{rgb}{1.000000,0.498039,0.054902}%
\pgfsetfillcolor{currentfill}%
\pgfsetlinewidth{1.003750pt}%
\definecolor{currentstroke}{rgb}{1.000000,0.498039,0.054902}%
\pgfsetstrokecolor{currentstroke}%
\pgfsetdash{}{0pt}%
\pgfpathmoveto{\pgfqpoint{1.341255in}{3.149281in}}%
\pgfpathcurveto{\pgfqpoint{1.352305in}{3.149281in}}{\pgfqpoint{1.362904in}{3.153671in}}{\pgfqpoint{1.370718in}{3.161485in}}%
\pgfpathcurveto{\pgfqpoint{1.378532in}{3.169298in}}{\pgfqpoint{1.382922in}{3.179897in}}{\pgfqpoint{1.382922in}{3.190948in}}%
\pgfpathcurveto{\pgfqpoint{1.382922in}{3.201998in}}{\pgfqpoint{1.378532in}{3.212597in}}{\pgfqpoint{1.370718in}{3.220410in}}%
\pgfpathcurveto{\pgfqpoint{1.362904in}{3.228224in}}{\pgfqpoint{1.352305in}{3.232614in}}{\pgfqpoint{1.341255in}{3.232614in}}%
\pgfpathcurveto{\pgfqpoint{1.330205in}{3.232614in}}{\pgfqpoint{1.319606in}{3.228224in}}{\pgfqpoint{1.311793in}{3.220410in}}%
\pgfpathcurveto{\pgfqpoint{1.303979in}{3.212597in}}{\pgfqpoint{1.299589in}{3.201998in}}{\pgfqpoint{1.299589in}{3.190948in}}%
\pgfpathcurveto{\pgfqpoint{1.299589in}{3.179897in}}{\pgfqpoint{1.303979in}{3.169298in}}{\pgfqpoint{1.311793in}{3.161485in}}%
\pgfpathcurveto{\pgfqpoint{1.319606in}{3.153671in}}{\pgfqpoint{1.330205in}{3.149281in}}{\pgfqpoint{1.341255in}{3.149281in}}%
\pgfpathclose%
\pgfusepath{stroke,fill}%
\end{pgfscope}%
\begin{pgfscope}%
\pgfpathrectangle{\pgfqpoint{0.648703in}{0.548769in}}{\pgfqpoint{5.201297in}{3.102590in}}%
\pgfusepath{clip}%
\pgfsetbuttcap%
\pgfsetroundjoin%
\definecolor{currentfill}{rgb}{1.000000,0.498039,0.054902}%
\pgfsetfillcolor{currentfill}%
\pgfsetlinewidth{1.003750pt}%
\definecolor{currentstroke}{rgb}{1.000000,0.498039,0.054902}%
\pgfsetstrokecolor{currentstroke}%
\pgfsetdash{}{0pt}%
\pgfpathmoveto{\pgfqpoint{1.954332in}{3.240534in}}%
\pgfpathcurveto{\pgfqpoint{1.965382in}{3.240534in}}{\pgfqpoint{1.975982in}{3.244924in}}{\pgfqpoint{1.983795in}{3.252737in}}%
\pgfpathcurveto{\pgfqpoint{1.991609in}{3.260551in}}{\pgfqpoint{1.995999in}{3.271150in}}{\pgfqpoint{1.995999in}{3.282200in}}%
\pgfpathcurveto{\pgfqpoint{1.995999in}{3.293250in}}{\pgfqpoint{1.991609in}{3.303849in}}{\pgfqpoint{1.983795in}{3.311663in}}%
\pgfpathcurveto{\pgfqpoint{1.975982in}{3.319477in}}{\pgfqpoint{1.965382in}{3.323867in}}{\pgfqpoint{1.954332in}{3.323867in}}%
\pgfpathcurveto{\pgfqpoint{1.943282in}{3.323867in}}{\pgfqpoint{1.932683in}{3.319477in}}{\pgfqpoint{1.924870in}{3.311663in}}%
\pgfpathcurveto{\pgfqpoint{1.917056in}{3.303849in}}{\pgfqpoint{1.912666in}{3.293250in}}{\pgfqpoint{1.912666in}{3.282200in}}%
\pgfpathcurveto{\pgfqpoint{1.912666in}{3.271150in}}{\pgfqpoint{1.917056in}{3.260551in}}{\pgfqpoint{1.924870in}{3.252737in}}%
\pgfpathcurveto{\pgfqpoint{1.932683in}{3.244924in}}{\pgfqpoint{1.943282in}{3.240534in}}{\pgfqpoint{1.954332in}{3.240534in}}%
\pgfpathclose%
\pgfusepath{stroke,fill}%
\end{pgfscope}%
\begin{pgfscope}%
\pgfpathrectangle{\pgfqpoint{0.648703in}{0.548769in}}{\pgfqpoint{5.201297in}{3.102590in}}%
\pgfusepath{clip}%
\pgfsetbuttcap%
\pgfsetroundjoin%
\definecolor{currentfill}{rgb}{0.121569,0.466667,0.705882}%
\pgfsetfillcolor{currentfill}%
\pgfsetlinewidth{1.003750pt}%
\definecolor{currentstroke}{rgb}{0.121569,0.466667,0.705882}%
\pgfsetstrokecolor{currentstroke}%
\pgfsetdash{}{0pt}%
\pgfpathmoveto{\pgfqpoint{0.885215in}{0.664720in}}%
\pgfpathcurveto{\pgfqpoint{0.896265in}{0.664720in}}{\pgfqpoint{0.906864in}{0.669111in}}{\pgfqpoint{0.914678in}{0.676924in}}%
\pgfpathcurveto{\pgfqpoint{0.922492in}{0.684738in}}{\pgfqpoint{0.926882in}{0.695337in}}{\pgfqpoint{0.926882in}{0.706387in}}%
\pgfpathcurveto{\pgfqpoint{0.926882in}{0.717437in}}{\pgfqpoint{0.922492in}{0.728036in}}{\pgfqpoint{0.914678in}{0.735850in}}%
\pgfpathcurveto{\pgfqpoint{0.906864in}{0.743663in}}{\pgfqpoint{0.896265in}{0.748054in}}{\pgfqpoint{0.885215in}{0.748054in}}%
\pgfpathcurveto{\pgfqpoint{0.874165in}{0.748054in}}{\pgfqpoint{0.863566in}{0.743663in}}{\pgfqpoint{0.855752in}{0.735850in}}%
\pgfpathcurveto{\pgfqpoint{0.847939in}{0.728036in}}{\pgfqpoint{0.843548in}{0.717437in}}{\pgfqpoint{0.843548in}{0.706387in}}%
\pgfpathcurveto{\pgfqpoint{0.843548in}{0.695337in}}{\pgfqpoint{0.847939in}{0.684738in}}{\pgfqpoint{0.855752in}{0.676924in}}%
\pgfpathcurveto{\pgfqpoint{0.863566in}{0.669111in}}{\pgfqpoint{0.874165in}{0.664720in}}{\pgfqpoint{0.885215in}{0.664720in}}%
\pgfpathclose%
\pgfusepath{stroke,fill}%
\end{pgfscope}%
\begin{pgfscope}%
\pgfpathrectangle{\pgfqpoint{0.648703in}{0.548769in}}{\pgfqpoint{5.201297in}{3.102590in}}%
\pgfusepath{clip}%
\pgfsetbuttcap%
\pgfsetroundjoin%
\definecolor{currentfill}{rgb}{1.000000,0.498039,0.054902}%
\pgfsetfillcolor{currentfill}%
\pgfsetlinewidth{1.003750pt}%
\definecolor{currentstroke}{rgb}{1.000000,0.498039,0.054902}%
\pgfsetstrokecolor{currentstroke}%
\pgfsetdash{}{0pt}%
\pgfpathmoveto{\pgfqpoint{2.013009in}{3.157577in}}%
\pgfpathcurveto{\pgfqpoint{2.024060in}{3.157577in}}{\pgfqpoint{2.034659in}{3.161967in}}{\pgfqpoint{2.042472in}{3.169780in}}%
\pgfpathcurveto{\pgfqpoint{2.050286in}{3.177594in}}{\pgfqpoint{2.054676in}{3.188193in}}{\pgfqpoint{2.054676in}{3.199243in}}%
\pgfpathcurveto{\pgfqpoint{2.054676in}{3.210293in}}{\pgfqpoint{2.050286in}{3.220892in}}{\pgfqpoint{2.042472in}{3.228706in}}%
\pgfpathcurveto{\pgfqpoint{2.034659in}{3.236520in}}{\pgfqpoint{2.024060in}{3.240910in}}{\pgfqpoint{2.013009in}{3.240910in}}%
\pgfpathcurveto{\pgfqpoint{2.001959in}{3.240910in}}{\pgfqpoint{1.991360in}{3.236520in}}{\pgfqpoint{1.983547in}{3.228706in}}%
\pgfpathcurveto{\pgfqpoint{1.975733in}{3.220892in}}{\pgfqpoint{1.971343in}{3.210293in}}{\pgfqpoint{1.971343in}{3.199243in}}%
\pgfpathcurveto{\pgfqpoint{1.971343in}{3.188193in}}{\pgfqpoint{1.975733in}{3.177594in}}{\pgfqpoint{1.983547in}{3.169780in}}%
\pgfpathcurveto{\pgfqpoint{1.991360in}{3.161967in}}{\pgfqpoint{2.001959in}{3.157577in}}{\pgfqpoint{2.013009in}{3.157577in}}%
\pgfpathclose%
\pgfusepath{stroke,fill}%
\end{pgfscope}%
\begin{pgfscope}%
\pgfpathrectangle{\pgfqpoint{0.648703in}{0.548769in}}{\pgfqpoint{5.201297in}{3.102590in}}%
\pgfusepath{clip}%
\pgfsetbuttcap%
\pgfsetroundjoin%
\definecolor{currentfill}{rgb}{1.000000,0.498039,0.054902}%
\pgfsetfillcolor{currentfill}%
\pgfsetlinewidth{1.003750pt}%
\definecolor{currentstroke}{rgb}{1.000000,0.498039,0.054902}%
\pgfsetstrokecolor{currentstroke}%
\pgfsetdash{}{0pt}%
\pgfpathmoveto{\pgfqpoint{1.431589in}{3.140985in}}%
\pgfpathcurveto{\pgfqpoint{1.442640in}{3.140985in}}{\pgfqpoint{1.453239in}{3.145375in}}{\pgfqpoint{1.461052in}{3.153189in}}%
\pgfpathcurveto{\pgfqpoint{1.468866in}{3.161003in}}{\pgfqpoint{1.473256in}{3.171602in}}{\pgfqpoint{1.473256in}{3.182652in}}%
\pgfpathcurveto{\pgfqpoint{1.473256in}{3.193702in}}{\pgfqpoint{1.468866in}{3.204301in}}{\pgfqpoint{1.461052in}{3.212115in}}%
\pgfpathcurveto{\pgfqpoint{1.453239in}{3.219928in}}{\pgfqpoint{1.442640in}{3.224319in}}{\pgfqpoint{1.431589in}{3.224319in}}%
\pgfpathcurveto{\pgfqpoint{1.420539in}{3.224319in}}{\pgfqpoint{1.409940in}{3.219928in}}{\pgfqpoint{1.402127in}{3.212115in}}%
\pgfpathcurveto{\pgfqpoint{1.394313in}{3.204301in}}{\pgfqpoint{1.389923in}{3.193702in}}{\pgfqpoint{1.389923in}{3.182652in}}%
\pgfpathcurveto{\pgfqpoint{1.389923in}{3.171602in}}{\pgfqpoint{1.394313in}{3.161003in}}{\pgfqpoint{1.402127in}{3.153189in}}%
\pgfpathcurveto{\pgfqpoint{1.409940in}{3.145375in}}{\pgfqpoint{1.420539in}{3.140985in}}{\pgfqpoint{1.431589in}{3.140985in}}%
\pgfpathclose%
\pgfusepath{stroke,fill}%
\end{pgfscope}%
\begin{pgfscope}%
\pgfpathrectangle{\pgfqpoint{0.648703in}{0.548769in}}{\pgfqpoint{5.201297in}{3.102590in}}%
\pgfusepath{clip}%
\pgfsetbuttcap%
\pgfsetroundjoin%
\definecolor{currentfill}{rgb}{1.000000,0.498039,0.054902}%
\pgfsetfillcolor{currentfill}%
\pgfsetlinewidth{1.003750pt}%
\definecolor{currentstroke}{rgb}{1.000000,0.498039,0.054902}%
\pgfsetstrokecolor{currentstroke}%
\pgfsetdash{}{0pt}%
\pgfpathmoveto{\pgfqpoint{1.616137in}{3.136837in}}%
\pgfpathcurveto{\pgfqpoint{1.627187in}{3.136837in}}{\pgfqpoint{1.637786in}{3.141228in}}{\pgfqpoint{1.645600in}{3.149041in}}%
\pgfpathcurveto{\pgfqpoint{1.653413in}{3.156855in}}{\pgfqpoint{1.657803in}{3.167454in}}{\pgfqpoint{1.657803in}{3.178504in}}%
\pgfpathcurveto{\pgfqpoint{1.657803in}{3.189554in}}{\pgfqpoint{1.653413in}{3.200153in}}{\pgfqpoint{1.645600in}{3.207967in}}%
\pgfpathcurveto{\pgfqpoint{1.637786in}{3.215780in}}{\pgfqpoint{1.627187in}{3.220171in}}{\pgfqpoint{1.616137in}{3.220171in}}%
\pgfpathcurveto{\pgfqpoint{1.605087in}{3.220171in}}{\pgfqpoint{1.594488in}{3.215780in}}{\pgfqpoint{1.586674in}{3.207967in}}%
\pgfpathcurveto{\pgfqpoint{1.578860in}{3.200153in}}{\pgfqpoint{1.574470in}{3.189554in}}{\pgfqpoint{1.574470in}{3.178504in}}%
\pgfpathcurveto{\pgfqpoint{1.574470in}{3.167454in}}{\pgfqpoint{1.578860in}{3.156855in}}{\pgfqpoint{1.586674in}{3.149041in}}%
\pgfpathcurveto{\pgfqpoint{1.594488in}{3.141228in}}{\pgfqpoint{1.605087in}{3.136837in}}{\pgfqpoint{1.616137in}{3.136837in}}%
\pgfpathclose%
\pgfusepath{stroke,fill}%
\end{pgfscope}%
\begin{pgfscope}%
\pgfpathrectangle{\pgfqpoint{0.648703in}{0.548769in}}{\pgfqpoint{5.201297in}{3.102590in}}%
\pgfusepath{clip}%
\pgfsetbuttcap%
\pgfsetroundjoin%
\definecolor{currentfill}{rgb}{1.000000,0.498039,0.054902}%
\pgfsetfillcolor{currentfill}%
\pgfsetlinewidth{1.003750pt}%
\definecolor{currentstroke}{rgb}{1.000000,0.498039,0.054902}%
\pgfsetstrokecolor{currentstroke}%
\pgfsetdash{}{0pt}%
\pgfpathmoveto{\pgfqpoint{1.379779in}{3.136837in}}%
\pgfpathcurveto{\pgfqpoint{1.390829in}{3.136837in}}{\pgfqpoint{1.401428in}{3.141228in}}{\pgfqpoint{1.409242in}{3.149041in}}%
\pgfpathcurveto{\pgfqpoint{1.417055in}{3.156855in}}{\pgfqpoint{1.421446in}{3.167454in}}{\pgfqpoint{1.421446in}{3.178504in}}%
\pgfpathcurveto{\pgfqpoint{1.421446in}{3.189554in}}{\pgfqpoint{1.417055in}{3.200153in}}{\pgfqpoint{1.409242in}{3.207967in}}%
\pgfpathcurveto{\pgfqpoint{1.401428in}{3.215780in}}{\pgfqpoint{1.390829in}{3.220171in}}{\pgfqpoint{1.379779in}{3.220171in}}%
\pgfpathcurveto{\pgfqpoint{1.368729in}{3.220171in}}{\pgfqpoint{1.358130in}{3.215780in}}{\pgfqpoint{1.350316in}{3.207967in}}%
\pgfpathcurveto{\pgfqpoint{1.342502in}{3.200153in}}{\pgfqpoint{1.338112in}{3.189554in}}{\pgfqpoint{1.338112in}{3.178504in}}%
\pgfpathcurveto{\pgfqpoint{1.338112in}{3.167454in}}{\pgfqpoint{1.342502in}{3.156855in}}{\pgfqpoint{1.350316in}{3.149041in}}%
\pgfpathcurveto{\pgfqpoint{1.358130in}{3.141228in}}{\pgfqpoint{1.368729in}{3.136837in}}{\pgfqpoint{1.379779in}{3.136837in}}%
\pgfpathclose%
\pgfusepath{stroke,fill}%
\end{pgfscope}%
\begin{pgfscope}%
\pgfpathrectangle{\pgfqpoint{0.648703in}{0.548769in}}{\pgfqpoint{5.201297in}{3.102590in}}%
\pgfusepath{clip}%
\pgfsetbuttcap%
\pgfsetroundjoin%
\definecolor{currentfill}{rgb}{1.000000,0.498039,0.054902}%
\pgfsetfillcolor{currentfill}%
\pgfsetlinewidth{1.003750pt}%
\definecolor{currentstroke}{rgb}{1.000000,0.498039,0.054902}%
\pgfsetstrokecolor{currentstroke}%
\pgfsetdash{}{0pt}%
\pgfpathmoveto{\pgfqpoint{1.391059in}{3.136837in}}%
\pgfpathcurveto{\pgfqpoint{1.402110in}{3.136837in}}{\pgfqpoint{1.412709in}{3.141228in}}{\pgfqpoint{1.420522in}{3.149041in}}%
\pgfpathcurveto{\pgfqpoint{1.428336in}{3.156855in}}{\pgfqpoint{1.432726in}{3.167454in}}{\pgfqpoint{1.432726in}{3.178504in}}%
\pgfpathcurveto{\pgfqpoint{1.432726in}{3.189554in}}{\pgfqpoint{1.428336in}{3.200153in}}{\pgfqpoint{1.420522in}{3.207967in}}%
\pgfpathcurveto{\pgfqpoint{1.412709in}{3.215780in}}{\pgfqpoint{1.402110in}{3.220171in}}{\pgfqpoint{1.391059in}{3.220171in}}%
\pgfpathcurveto{\pgfqpoint{1.380009in}{3.220171in}}{\pgfqpoint{1.369410in}{3.215780in}}{\pgfqpoint{1.361597in}{3.207967in}}%
\pgfpathcurveto{\pgfqpoint{1.353783in}{3.200153in}}{\pgfqpoint{1.349393in}{3.189554in}}{\pgfqpoint{1.349393in}{3.178504in}}%
\pgfpathcurveto{\pgfqpoint{1.349393in}{3.167454in}}{\pgfqpoint{1.353783in}{3.156855in}}{\pgfqpoint{1.361597in}{3.149041in}}%
\pgfpathcurveto{\pgfqpoint{1.369410in}{3.141228in}}{\pgfqpoint{1.380009in}{3.136837in}}{\pgfqpoint{1.391059in}{3.136837in}}%
\pgfpathclose%
\pgfusepath{stroke,fill}%
\end{pgfscope}%
\begin{pgfscope}%
\pgfpathrectangle{\pgfqpoint{0.648703in}{0.548769in}}{\pgfqpoint{5.201297in}{3.102590in}}%
\pgfusepath{clip}%
\pgfsetbuttcap%
\pgfsetroundjoin%
\definecolor{currentfill}{rgb}{1.000000,0.498039,0.054902}%
\pgfsetfillcolor{currentfill}%
\pgfsetlinewidth{1.003750pt}%
\definecolor{currentstroke}{rgb}{1.000000,0.498039,0.054902}%
\pgfsetstrokecolor{currentstroke}%
\pgfsetdash{}{0pt}%
\pgfpathmoveto{\pgfqpoint{1.624073in}{3.136837in}}%
\pgfpathcurveto{\pgfqpoint{1.635123in}{3.136837in}}{\pgfqpoint{1.645722in}{3.141228in}}{\pgfqpoint{1.653536in}{3.149041in}}%
\pgfpathcurveto{\pgfqpoint{1.661350in}{3.156855in}}{\pgfqpoint{1.665740in}{3.167454in}}{\pgfqpoint{1.665740in}{3.178504in}}%
\pgfpathcurveto{\pgfqpoint{1.665740in}{3.189554in}}{\pgfqpoint{1.661350in}{3.200153in}}{\pgfqpoint{1.653536in}{3.207967in}}%
\pgfpathcurveto{\pgfqpoint{1.645722in}{3.215780in}}{\pgfqpoint{1.635123in}{3.220171in}}{\pgfqpoint{1.624073in}{3.220171in}}%
\pgfpathcurveto{\pgfqpoint{1.613023in}{3.220171in}}{\pgfqpoint{1.602424in}{3.215780in}}{\pgfqpoint{1.594611in}{3.207967in}}%
\pgfpathcurveto{\pgfqpoint{1.586797in}{3.200153in}}{\pgfqpoint{1.582407in}{3.189554in}}{\pgfqpoint{1.582407in}{3.178504in}}%
\pgfpathcurveto{\pgfqpoint{1.582407in}{3.167454in}}{\pgfqpoint{1.586797in}{3.156855in}}{\pgfqpoint{1.594611in}{3.149041in}}%
\pgfpathcurveto{\pgfqpoint{1.602424in}{3.141228in}}{\pgfqpoint{1.613023in}{3.136837in}}{\pgfqpoint{1.624073in}{3.136837in}}%
\pgfpathclose%
\pgfusepath{stroke,fill}%
\end{pgfscope}%
\begin{pgfscope}%
\pgfpathrectangle{\pgfqpoint{0.648703in}{0.548769in}}{\pgfqpoint{5.201297in}{3.102590in}}%
\pgfusepath{clip}%
\pgfsetbuttcap%
\pgfsetroundjoin%
\definecolor{currentfill}{rgb}{1.000000,0.498039,0.054902}%
\pgfsetfillcolor{currentfill}%
\pgfsetlinewidth{1.003750pt}%
\definecolor{currentstroke}{rgb}{1.000000,0.498039,0.054902}%
\pgfsetstrokecolor{currentstroke}%
\pgfsetdash{}{0pt}%
\pgfpathmoveto{\pgfqpoint{1.437876in}{3.136837in}}%
\pgfpathcurveto{\pgfqpoint{1.448926in}{3.136837in}}{\pgfqpoint{1.459525in}{3.141228in}}{\pgfqpoint{1.467339in}{3.149041in}}%
\pgfpathcurveto{\pgfqpoint{1.475153in}{3.156855in}}{\pgfqpoint{1.479543in}{3.167454in}}{\pgfqpoint{1.479543in}{3.178504in}}%
\pgfpathcurveto{\pgfqpoint{1.479543in}{3.189554in}}{\pgfqpoint{1.475153in}{3.200153in}}{\pgfqpoint{1.467339in}{3.207967in}}%
\pgfpathcurveto{\pgfqpoint{1.459525in}{3.215780in}}{\pgfqpoint{1.448926in}{3.220171in}}{\pgfqpoint{1.437876in}{3.220171in}}%
\pgfpathcurveto{\pgfqpoint{1.426826in}{3.220171in}}{\pgfqpoint{1.416227in}{3.215780in}}{\pgfqpoint{1.408413in}{3.207967in}}%
\pgfpathcurveto{\pgfqpoint{1.400600in}{3.200153in}}{\pgfqpoint{1.396210in}{3.189554in}}{\pgfqpoint{1.396210in}{3.178504in}}%
\pgfpathcurveto{\pgfqpoint{1.396210in}{3.167454in}}{\pgfqpoint{1.400600in}{3.156855in}}{\pgfqpoint{1.408413in}{3.149041in}}%
\pgfpathcurveto{\pgfqpoint{1.416227in}{3.141228in}}{\pgfqpoint{1.426826in}{3.136837in}}{\pgfqpoint{1.437876in}{3.136837in}}%
\pgfpathclose%
\pgfusepath{stroke,fill}%
\end{pgfscope}%
\begin{pgfscope}%
\pgfpathrectangle{\pgfqpoint{0.648703in}{0.548769in}}{\pgfqpoint{5.201297in}{3.102590in}}%
\pgfusepath{clip}%
\pgfsetbuttcap%
\pgfsetroundjoin%
\definecolor{currentfill}{rgb}{0.121569,0.466667,0.705882}%
\pgfsetfillcolor{currentfill}%
\pgfsetlinewidth{1.003750pt}%
\definecolor{currentstroke}{rgb}{0.121569,0.466667,0.705882}%
\pgfsetstrokecolor{currentstroke}%
\pgfsetdash{}{0pt}%
\pgfpathmoveto{\pgfqpoint{1.593843in}{2.846488in}}%
\pgfpathcurveto{\pgfqpoint{1.604893in}{2.846488in}}{\pgfqpoint{1.615492in}{2.850878in}}{\pgfqpoint{1.623306in}{2.858692in}}%
\pgfpathcurveto{\pgfqpoint{1.631119in}{2.866506in}}{\pgfqpoint{1.635510in}{2.877105in}}{\pgfqpoint{1.635510in}{2.888155in}}%
\pgfpathcurveto{\pgfqpoint{1.635510in}{2.899205in}}{\pgfqpoint{1.631119in}{2.909804in}}{\pgfqpoint{1.623306in}{2.917617in}}%
\pgfpathcurveto{\pgfqpoint{1.615492in}{2.925431in}}{\pgfqpoint{1.604893in}{2.929821in}}{\pgfqpoint{1.593843in}{2.929821in}}%
\pgfpathcurveto{\pgfqpoint{1.582793in}{2.929821in}}{\pgfqpoint{1.572194in}{2.925431in}}{\pgfqpoint{1.564380in}{2.917617in}}%
\pgfpathcurveto{\pgfqpoint{1.556567in}{2.909804in}}{\pgfqpoint{1.552176in}{2.899205in}}{\pgfqpoint{1.552176in}{2.888155in}}%
\pgfpathcurveto{\pgfqpoint{1.552176in}{2.877105in}}{\pgfqpoint{1.556567in}{2.866506in}}{\pgfqpoint{1.564380in}{2.858692in}}%
\pgfpathcurveto{\pgfqpoint{1.572194in}{2.850878in}}{\pgfqpoint{1.582793in}{2.846488in}}{\pgfqpoint{1.593843in}{2.846488in}}%
\pgfpathclose%
\pgfusepath{stroke,fill}%
\end{pgfscope}%
\begin{pgfscope}%
\pgfpathrectangle{\pgfqpoint{0.648703in}{0.548769in}}{\pgfqpoint{5.201297in}{3.102590in}}%
\pgfusepath{clip}%
\pgfsetbuttcap%
\pgfsetroundjoin%
\definecolor{currentfill}{rgb}{0.121569,0.466667,0.705882}%
\pgfsetfillcolor{currentfill}%
\pgfsetlinewidth{1.003750pt}%
\definecolor{currentstroke}{rgb}{0.121569,0.466667,0.705882}%
\pgfsetstrokecolor{currentstroke}%
\pgfsetdash{}{0pt}%
\pgfpathmoveto{\pgfqpoint{5.613577in}{3.074620in}}%
\pgfpathcurveto{\pgfqpoint{5.624628in}{3.074620in}}{\pgfqpoint{5.635227in}{3.079010in}}{\pgfqpoint{5.643040in}{3.086824in}}%
\pgfpathcurveto{\pgfqpoint{5.650854in}{3.094637in}}{\pgfqpoint{5.655244in}{3.105236in}}{\pgfqpoint{5.655244in}{3.116286in}}%
\pgfpathcurveto{\pgfqpoint{5.655244in}{3.127336in}}{\pgfqpoint{5.650854in}{3.137935in}}{\pgfqpoint{5.643040in}{3.145749in}}%
\pgfpathcurveto{\pgfqpoint{5.635227in}{3.153563in}}{\pgfqpoint{5.624628in}{3.157953in}}{\pgfqpoint{5.613577in}{3.157953in}}%
\pgfpathcurveto{\pgfqpoint{5.602527in}{3.157953in}}{\pgfqpoint{5.591928in}{3.153563in}}{\pgfqpoint{5.584115in}{3.145749in}}%
\pgfpathcurveto{\pgfqpoint{5.576301in}{3.137935in}}{\pgfqpoint{5.571911in}{3.127336in}}{\pgfqpoint{5.571911in}{3.116286in}}%
\pgfpathcurveto{\pgfqpoint{5.571911in}{3.105236in}}{\pgfqpoint{5.576301in}{3.094637in}}{\pgfqpoint{5.584115in}{3.086824in}}%
\pgfpathcurveto{\pgfqpoint{5.591928in}{3.079010in}}{\pgfqpoint{5.602527in}{3.074620in}}{\pgfqpoint{5.613577in}{3.074620in}}%
\pgfpathclose%
\pgfusepath{stroke,fill}%
\end{pgfscope}%
\begin{pgfscope}%
\pgfpathrectangle{\pgfqpoint{0.648703in}{0.548769in}}{\pgfqpoint{5.201297in}{3.102590in}}%
\pgfusepath{clip}%
\pgfsetbuttcap%
\pgfsetroundjoin%
\definecolor{currentfill}{rgb}{1.000000,0.498039,0.054902}%
\pgfsetfillcolor{currentfill}%
\pgfsetlinewidth{1.003750pt}%
\definecolor{currentstroke}{rgb}{1.000000,0.498039,0.054902}%
\pgfsetstrokecolor{currentstroke}%
\pgfsetdash{}{0pt}%
\pgfpathmoveto{\pgfqpoint{1.885177in}{3.315195in}}%
\pgfpathcurveto{\pgfqpoint{1.896227in}{3.315195in}}{\pgfqpoint{1.906826in}{3.319585in}}{\pgfqpoint{1.914640in}{3.327399in}}%
\pgfpathcurveto{\pgfqpoint{1.922454in}{3.335212in}}{\pgfqpoint{1.926844in}{3.345811in}}{\pgfqpoint{1.926844in}{3.356861in}}%
\pgfpathcurveto{\pgfqpoint{1.926844in}{3.367912in}}{\pgfqpoint{1.922454in}{3.378511in}}{\pgfqpoint{1.914640in}{3.386324in}}%
\pgfpathcurveto{\pgfqpoint{1.906826in}{3.394138in}}{\pgfqpoint{1.896227in}{3.398528in}}{\pgfqpoint{1.885177in}{3.398528in}}%
\pgfpathcurveto{\pgfqpoint{1.874127in}{3.398528in}}{\pgfqpoint{1.863528in}{3.394138in}}{\pgfqpoint{1.855714in}{3.386324in}}%
\pgfpathcurveto{\pgfqpoint{1.847901in}{3.378511in}}{\pgfqpoint{1.843511in}{3.367912in}}{\pgfqpoint{1.843511in}{3.356861in}}%
\pgfpathcurveto{\pgfqpoint{1.843511in}{3.345811in}}{\pgfqpoint{1.847901in}{3.335212in}}{\pgfqpoint{1.855714in}{3.327399in}}%
\pgfpathcurveto{\pgfqpoint{1.863528in}{3.319585in}}{\pgfqpoint{1.874127in}{3.315195in}}{\pgfqpoint{1.885177in}{3.315195in}}%
\pgfpathclose%
\pgfusepath{stroke,fill}%
\end{pgfscope}%
\begin{pgfscope}%
\pgfpathrectangle{\pgfqpoint{0.648703in}{0.548769in}}{\pgfqpoint{5.201297in}{3.102590in}}%
\pgfusepath{clip}%
\pgfsetbuttcap%
\pgfsetroundjoin%
\definecolor{currentfill}{rgb}{0.121569,0.466667,0.705882}%
\pgfsetfillcolor{currentfill}%
\pgfsetlinewidth{1.003750pt}%
\definecolor{currentstroke}{rgb}{0.121569,0.466667,0.705882}%
\pgfsetstrokecolor{currentstroke}%
\pgfsetdash{}{0pt}%
\pgfpathmoveto{\pgfqpoint{1.238927in}{0.648129in}}%
\pgfpathcurveto{\pgfqpoint{1.249977in}{0.648129in}}{\pgfqpoint{1.260576in}{0.652519in}}{\pgfqpoint{1.268390in}{0.660333in}}%
\pgfpathcurveto{\pgfqpoint{1.276204in}{0.668146in}}{\pgfqpoint{1.280594in}{0.678745in}}{\pgfqpoint{1.280594in}{0.689796in}}%
\pgfpathcurveto{\pgfqpoint{1.280594in}{0.700846in}}{\pgfqpoint{1.276204in}{0.711445in}}{\pgfqpoint{1.268390in}{0.719258in}}%
\pgfpathcurveto{\pgfqpoint{1.260576in}{0.727072in}}{\pgfqpoint{1.249977in}{0.731462in}}{\pgfqpoint{1.238927in}{0.731462in}}%
\pgfpathcurveto{\pgfqpoint{1.227877in}{0.731462in}}{\pgfqpoint{1.217278in}{0.727072in}}{\pgfqpoint{1.209464in}{0.719258in}}%
\pgfpathcurveto{\pgfqpoint{1.201651in}{0.711445in}}{\pgfqpoint{1.197261in}{0.700846in}}{\pgfqpoint{1.197261in}{0.689796in}}%
\pgfpathcurveto{\pgfqpoint{1.197261in}{0.678745in}}{\pgfqpoint{1.201651in}{0.668146in}}{\pgfqpoint{1.209464in}{0.660333in}}%
\pgfpathcurveto{\pgfqpoint{1.217278in}{0.652519in}}{\pgfqpoint{1.227877in}{0.648129in}}{\pgfqpoint{1.238927in}{0.648129in}}%
\pgfpathclose%
\pgfusepath{stroke,fill}%
\end{pgfscope}%
\begin{pgfscope}%
\pgfpathrectangle{\pgfqpoint{0.648703in}{0.548769in}}{\pgfqpoint{5.201297in}{3.102590in}}%
\pgfusepath{clip}%
\pgfsetbuttcap%
\pgfsetroundjoin%
\definecolor{currentfill}{rgb}{0.121569,0.466667,0.705882}%
\pgfsetfillcolor{currentfill}%
\pgfsetlinewidth{1.003750pt}%
\definecolor{currentstroke}{rgb}{0.121569,0.466667,0.705882}%
\pgfsetstrokecolor{currentstroke}%
\pgfsetdash{}{0pt}%
\pgfpathmoveto{\pgfqpoint{1.194072in}{1.166610in}}%
\pgfpathcurveto{\pgfqpoint{1.205122in}{1.166610in}}{\pgfqpoint{1.215721in}{1.171000in}}{\pgfqpoint{1.223535in}{1.178814in}}%
\pgfpathcurveto{\pgfqpoint{1.231349in}{1.186627in}}{\pgfqpoint{1.235739in}{1.197226in}}{\pgfqpoint{1.235739in}{1.208277in}}%
\pgfpathcurveto{\pgfqpoint{1.235739in}{1.219327in}}{\pgfqpoint{1.231349in}{1.229926in}}{\pgfqpoint{1.223535in}{1.237739in}}%
\pgfpathcurveto{\pgfqpoint{1.215721in}{1.245553in}}{\pgfqpoint{1.205122in}{1.249943in}}{\pgfqpoint{1.194072in}{1.249943in}}%
\pgfpathcurveto{\pgfqpoint{1.183022in}{1.249943in}}{\pgfqpoint{1.172423in}{1.245553in}}{\pgfqpoint{1.164609in}{1.237739in}}%
\pgfpathcurveto{\pgfqpoint{1.156796in}{1.229926in}}{\pgfqpoint{1.152406in}{1.219327in}}{\pgfqpoint{1.152406in}{1.208277in}}%
\pgfpathcurveto{\pgfqpoint{1.152406in}{1.197226in}}{\pgfqpoint{1.156796in}{1.186627in}}{\pgfqpoint{1.164609in}{1.178814in}}%
\pgfpathcurveto{\pgfqpoint{1.172423in}{1.171000in}}{\pgfqpoint{1.183022in}{1.166610in}}{\pgfqpoint{1.194072in}{1.166610in}}%
\pgfpathclose%
\pgfusepath{stroke,fill}%
\end{pgfscope}%
\begin{pgfscope}%
\pgfpathrectangle{\pgfqpoint{0.648703in}{0.548769in}}{\pgfqpoint{5.201297in}{3.102590in}}%
\pgfusepath{clip}%
\pgfsetbuttcap%
\pgfsetroundjoin%
\definecolor{currentfill}{rgb}{1.000000,0.498039,0.054902}%
\pgfsetfillcolor{currentfill}%
\pgfsetlinewidth{1.003750pt}%
\definecolor{currentstroke}{rgb}{1.000000,0.498039,0.054902}%
\pgfsetstrokecolor{currentstroke}%
\pgfsetdash{}{0pt}%
\pgfpathmoveto{\pgfqpoint{1.257163in}{3.136837in}}%
\pgfpathcurveto{\pgfqpoint{1.268214in}{3.136837in}}{\pgfqpoint{1.278813in}{3.141228in}}{\pgfqpoint{1.286626in}{3.149041in}}%
\pgfpathcurveto{\pgfqpoint{1.294440in}{3.156855in}}{\pgfqpoint{1.298830in}{3.167454in}}{\pgfqpoint{1.298830in}{3.178504in}}%
\pgfpathcurveto{\pgfqpoint{1.298830in}{3.189554in}}{\pgfqpoint{1.294440in}{3.200153in}}{\pgfqpoint{1.286626in}{3.207967in}}%
\pgfpathcurveto{\pgfqpoint{1.278813in}{3.215780in}}{\pgfqpoint{1.268214in}{3.220171in}}{\pgfqpoint{1.257163in}{3.220171in}}%
\pgfpathcurveto{\pgfqpoint{1.246113in}{3.220171in}}{\pgfqpoint{1.235514in}{3.215780in}}{\pgfqpoint{1.227701in}{3.207967in}}%
\pgfpathcurveto{\pgfqpoint{1.219887in}{3.200153in}}{\pgfqpoint{1.215497in}{3.189554in}}{\pgfqpoint{1.215497in}{3.178504in}}%
\pgfpathcurveto{\pgfqpoint{1.215497in}{3.167454in}}{\pgfqpoint{1.219887in}{3.156855in}}{\pgfqpoint{1.227701in}{3.149041in}}%
\pgfpathcurveto{\pgfqpoint{1.235514in}{3.141228in}}{\pgfqpoint{1.246113in}{3.136837in}}{\pgfqpoint{1.257163in}{3.136837in}}%
\pgfpathclose%
\pgfusepath{stroke,fill}%
\end{pgfscope}%
\begin{pgfscope}%
\pgfpathrectangle{\pgfqpoint{0.648703in}{0.548769in}}{\pgfqpoint{5.201297in}{3.102590in}}%
\pgfusepath{clip}%
\pgfsetbuttcap%
\pgfsetroundjoin%
\definecolor{currentfill}{rgb}{0.121569,0.466667,0.705882}%
\pgfsetfillcolor{currentfill}%
\pgfsetlinewidth{1.003750pt}%
\definecolor{currentstroke}{rgb}{0.121569,0.466667,0.705882}%
\pgfsetstrokecolor{currentstroke}%
\pgfsetdash{}{0pt}%
\pgfpathmoveto{\pgfqpoint{2.080782in}{3.132690in}}%
\pgfpathcurveto{\pgfqpoint{2.091832in}{3.132690in}}{\pgfqpoint{2.102431in}{3.137080in}}{\pgfqpoint{2.110245in}{3.144893in}}%
\pgfpathcurveto{\pgfqpoint{2.118059in}{3.152707in}}{\pgfqpoint{2.122449in}{3.163306in}}{\pgfqpoint{2.122449in}{3.174356in}}%
\pgfpathcurveto{\pgfqpoint{2.122449in}{3.185406in}}{\pgfqpoint{2.118059in}{3.196005in}}{\pgfqpoint{2.110245in}{3.203819in}}%
\pgfpathcurveto{\pgfqpoint{2.102431in}{3.211633in}}{\pgfqpoint{2.091832in}{3.216023in}}{\pgfqpoint{2.080782in}{3.216023in}}%
\pgfpathcurveto{\pgfqpoint{2.069732in}{3.216023in}}{\pgfqpoint{2.059133in}{3.211633in}}{\pgfqpoint{2.051319in}{3.203819in}}%
\pgfpathcurveto{\pgfqpoint{2.043506in}{3.196005in}}{\pgfqpoint{2.039116in}{3.185406in}}{\pgfqpoint{2.039116in}{3.174356in}}%
\pgfpathcurveto{\pgfqpoint{2.039116in}{3.163306in}}{\pgfqpoint{2.043506in}{3.152707in}}{\pgfqpoint{2.051319in}{3.144893in}}%
\pgfpathcurveto{\pgfqpoint{2.059133in}{3.137080in}}{\pgfqpoint{2.069732in}{3.132690in}}{\pgfqpoint{2.080782in}{3.132690in}}%
\pgfpathclose%
\pgfusepath{stroke,fill}%
\end{pgfscope}%
\begin{pgfscope}%
\pgfpathrectangle{\pgfqpoint{0.648703in}{0.548769in}}{\pgfqpoint{5.201297in}{3.102590in}}%
\pgfusepath{clip}%
\pgfsetbuttcap%
\pgfsetroundjoin%
\definecolor{currentfill}{rgb}{0.839216,0.152941,0.156863}%
\pgfsetfillcolor{currentfill}%
\pgfsetlinewidth{1.003750pt}%
\definecolor{currentstroke}{rgb}{0.839216,0.152941,0.156863}%
\pgfsetstrokecolor{currentstroke}%
\pgfsetdash{}{0pt}%
\pgfpathmoveto{\pgfqpoint{1.434175in}{3.149281in}}%
\pgfpathcurveto{\pgfqpoint{1.445226in}{3.149281in}}{\pgfqpoint{1.455825in}{3.153671in}}{\pgfqpoint{1.463638in}{3.161485in}}%
\pgfpathcurveto{\pgfqpoint{1.471452in}{3.169298in}}{\pgfqpoint{1.475842in}{3.179897in}}{\pgfqpoint{1.475842in}{3.190948in}}%
\pgfpathcurveto{\pgfqpoint{1.475842in}{3.201998in}}{\pgfqpoint{1.471452in}{3.212597in}}{\pgfqpoint{1.463638in}{3.220410in}}%
\pgfpathcurveto{\pgfqpoint{1.455825in}{3.228224in}}{\pgfqpoint{1.445226in}{3.232614in}}{\pgfqpoint{1.434175in}{3.232614in}}%
\pgfpathcurveto{\pgfqpoint{1.423125in}{3.232614in}}{\pgfqpoint{1.412526in}{3.228224in}}{\pgfqpoint{1.404713in}{3.220410in}}%
\pgfpathcurveto{\pgfqpoint{1.396899in}{3.212597in}}{\pgfqpoint{1.392509in}{3.201998in}}{\pgfqpoint{1.392509in}{3.190948in}}%
\pgfpathcurveto{\pgfqpoint{1.392509in}{3.179897in}}{\pgfqpoint{1.396899in}{3.169298in}}{\pgfqpoint{1.404713in}{3.161485in}}%
\pgfpathcurveto{\pgfqpoint{1.412526in}{3.153671in}}{\pgfqpoint{1.423125in}{3.149281in}}{\pgfqpoint{1.434175in}{3.149281in}}%
\pgfpathclose%
\pgfusepath{stroke,fill}%
\end{pgfscope}%
\begin{pgfscope}%
\pgfpathrectangle{\pgfqpoint{0.648703in}{0.548769in}}{\pgfqpoint{5.201297in}{3.102590in}}%
\pgfusepath{clip}%
\pgfsetbuttcap%
\pgfsetroundjoin%
\definecolor{currentfill}{rgb}{1.000000,0.498039,0.054902}%
\pgfsetfillcolor{currentfill}%
\pgfsetlinewidth{1.003750pt}%
\definecolor{currentstroke}{rgb}{1.000000,0.498039,0.054902}%
\pgfsetstrokecolor{currentstroke}%
\pgfsetdash{}{0pt}%
\pgfpathmoveto{\pgfqpoint{1.217614in}{3.174168in}}%
\pgfpathcurveto{\pgfqpoint{1.228665in}{3.174168in}}{\pgfqpoint{1.239264in}{3.178558in}}{\pgfqpoint{1.247077in}{3.186372in}}%
\pgfpathcurveto{\pgfqpoint{1.254891in}{3.194185in}}{\pgfqpoint{1.259281in}{3.204785in}}{\pgfqpoint{1.259281in}{3.215835in}}%
\pgfpathcurveto{\pgfqpoint{1.259281in}{3.226885in}}{\pgfqpoint{1.254891in}{3.237484in}}{\pgfqpoint{1.247077in}{3.245297in}}%
\pgfpathcurveto{\pgfqpoint{1.239264in}{3.253111in}}{\pgfqpoint{1.228665in}{3.257501in}}{\pgfqpoint{1.217614in}{3.257501in}}%
\pgfpathcurveto{\pgfqpoint{1.206564in}{3.257501in}}{\pgfqpoint{1.195965in}{3.253111in}}{\pgfqpoint{1.188152in}{3.245297in}}%
\pgfpathcurveto{\pgfqpoint{1.180338in}{3.237484in}}{\pgfqpoint{1.175948in}{3.226885in}}{\pgfqpoint{1.175948in}{3.215835in}}%
\pgfpathcurveto{\pgfqpoint{1.175948in}{3.204785in}}{\pgfqpoint{1.180338in}{3.194185in}}{\pgfqpoint{1.188152in}{3.186372in}}%
\pgfpathcurveto{\pgfqpoint{1.195965in}{3.178558in}}{\pgfqpoint{1.206564in}{3.174168in}}{\pgfqpoint{1.217614in}{3.174168in}}%
\pgfpathclose%
\pgfusepath{stroke,fill}%
\end{pgfscope}%
\begin{pgfscope}%
\pgfpathrectangle{\pgfqpoint{0.648703in}{0.548769in}}{\pgfqpoint{5.201297in}{3.102590in}}%
\pgfusepath{clip}%
\pgfsetbuttcap%
\pgfsetroundjoin%
\definecolor{currentfill}{rgb}{0.121569,0.466667,0.705882}%
\pgfsetfillcolor{currentfill}%
\pgfsetlinewidth{1.003750pt}%
\definecolor{currentstroke}{rgb}{0.121569,0.466667,0.705882}%
\pgfsetstrokecolor{currentstroke}%
\pgfsetdash{}{0pt}%
\pgfpathmoveto{\pgfqpoint{1.165715in}{0.656425in}}%
\pgfpathcurveto{\pgfqpoint{1.176765in}{0.656425in}}{\pgfqpoint{1.187364in}{0.660815in}}{\pgfqpoint{1.195177in}{0.668629in}}%
\pgfpathcurveto{\pgfqpoint{1.202991in}{0.676442in}}{\pgfqpoint{1.207381in}{0.687041in}}{\pgfqpoint{1.207381in}{0.698091in}}%
\pgfpathcurveto{\pgfqpoint{1.207381in}{0.709141in}}{\pgfqpoint{1.202991in}{0.719740in}}{\pgfqpoint{1.195177in}{0.727554in}}%
\pgfpathcurveto{\pgfqpoint{1.187364in}{0.735368in}}{\pgfqpoint{1.176765in}{0.739758in}}{\pgfqpoint{1.165715in}{0.739758in}}%
\pgfpathcurveto{\pgfqpoint{1.154664in}{0.739758in}}{\pgfqpoint{1.144065in}{0.735368in}}{\pgfqpoint{1.136252in}{0.727554in}}%
\pgfpathcurveto{\pgfqpoint{1.128438in}{0.719740in}}{\pgfqpoint{1.124048in}{0.709141in}}{\pgfqpoint{1.124048in}{0.698091in}}%
\pgfpathcurveto{\pgfqpoint{1.124048in}{0.687041in}}{\pgfqpoint{1.128438in}{0.676442in}}{\pgfqpoint{1.136252in}{0.668629in}}%
\pgfpathcurveto{\pgfqpoint{1.144065in}{0.660815in}}{\pgfqpoint{1.154664in}{0.656425in}}{\pgfqpoint{1.165715in}{0.656425in}}%
\pgfpathclose%
\pgfusepath{stroke,fill}%
\end{pgfscope}%
\begin{pgfscope}%
\pgfpathrectangle{\pgfqpoint{0.648703in}{0.548769in}}{\pgfqpoint{5.201297in}{3.102590in}}%
\pgfusepath{clip}%
\pgfsetbuttcap%
\pgfsetroundjoin%
\definecolor{currentfill}{rgb}{0.121569,0.466667,0.705882}%
\pgfsetfillcolor{currentfill}%
\pgfsetlinewidth{1.003750pt}%
\definecolor{currentstroke}{rgb}{0.121569,0.466667,0.705882}%
\pgfsetstrokecolor{currentstroke}%
\pgfsetdash{}{0pt}%
\pgfpathmoveto{\pgfqpoint{2.385314in}{3.124394in}}%
\pgfpathcurveto{\pgfqpoint{2.396364in}{3.124394in}}{\pgfqpoint{2.406964in}{3.128784in}}{\pgfqpoint{2.414777in}{3.136598in}}%
\pgfpathcurveto{\pgfqpoint{2.422591in}{3.144411in}}{\pgfqpoint{2.426981in}{3.155010in}}{\pgfqpoint{2.426981in}{3.166060in}}%
\pgfpathcurveto{\pgfqpoint{2.426981in}{3.177111in}}{\pgfqpoint{2.422591in}{3.187710in}}{\pgfqpoint{2.414777in}{3.195523in}}%
\pgfpathcurveto{\pgfqpoint{2.406964in}{3.203337in}}{\pgfqpoint{2.396364in}{3.207727in}}{\pgfqpoint{2.385314in}{3.207727in}}%
\pgfpathcurveto{\pgfqpoint{2.374264in}{3.207727in}}{\pgfqpoint{2.363665in}{3.203337in}}{\pgfqpoint{2.355852in}{3.195523in}}%
\pgfpathcurveto{\pgfqpoint{2.348038in}{3.187710in}}{\pgfqpoint{2.343648in}{3.177111in}}{\pgfqpoint{2.343648in}{3.166060in}}%
\pgfpathcurveto{\pgfqpoint{2.343648in}{3.155010in}}{\pgfqpoint{2.348038in}{3.144411in}}{\pgfqpoint{2.355852in}{3.136598in}}%
\pgfpathcurveto{\pgfqpoint{2.363665in}{3.128784in}}{\pgfqpoint{2.374264in}{3.124394in}}{\pgfqpoint{2.385314in}{3.124394in}}%
\pgfpathclose%
\pgfusepath{stroke,fill}%
\end{pgfscope}%
\begin{pgfscope}%
\pgfpathrectangle{\pgfqpoint{0.648703in}{0.548769in}}{\pgfqpoint{5.201297in}{3.102590in}}%
\pgfusepath{clip}%
\pgfsetbuttcap%
\pgfsetroundjoin%
\definecolor{currentfill}{rgb}{0.121569,0.466667,0.705882}%
\pgfsetfillcolor{currentfill}%
\pgfsetlinewidth{1.003750pt}%
\definecolor{currentstroke}{rgb}{0.121569,0.466667,0.705882}%
\pgfsetstrokecolor{currentstroke}%
\pgfsetdash{}{0pt}%
\pgfpathmoveto{\pgfqpoint{0.947950in}{0.648129in}}%
\pgfpathcurveto{\pgfqpoint{0.959000in}{0.648129in}}{\pgfqpoint{0.969599in}{0.652519in}}{\pgfqpoint{0.977412in}{0.660333in}}%
\pgfpathcurveto{\pgfqpoint{0.985226in}{0.668146in}}{\pgfqpoint{0.989616in}{0.678745in}}{\pgfqpoint{0.989616in}{0.689796in}}%
\pgfpathcurveto{\pgfqpoint{0.989616in}{0.700846in}}{\pgfqpoint{0.985226in}{0.711445in}}{\pgfqpoint{0.977412in}{0.719258in}}%
\pgfpathcurveto{\pgfqpoint{0.969599in}{0.727072in}}{\pgfqpoint{0.959000in}{0.731462in}}{\pgfqpoint{0.947950in}{0.731462in}}%
\pgfpathcurveto{\pgfqpoint{0.936900in}{0.731462in}}{\pgfqpoint{0.926301in}{0.727072in}}{\pgfqpoint{0.918487in}{0.719258in}}%
\pgfpathcurveto{\pgfqpoint{0.910673in}{0.711445in}}{\pgfqpoint{0.906283in}{0.700846in}}{\pgfqpoint{0.906283in}{0.689796in}}%
\pgfpathcurveto{\pgfqpoint{0.906283in}{0.678745in}}{\pgfqpoint{0.910673in}{0.668146in}}{\pgfqpoint{0.918487in}{0.660333in}}%
\pgfpathcurveto{\pgfqpoint{0.926301in}{0.652519in}}{\pgfqpoint{0.936900in}{0.648129in}}{\pgfqpoint{0.947950in}{0.648129in}}%
\pgfpathclose%
\pgfusepath{stroke,fill}%
\end{pgfscope}%
\begin{pgfscope}%
\pgfpathrectangle{\pgfqpoint{0.648703in}{0.548769in}}{\pgfqpoint{5.201297in}{3.102590in}}%
\pgfusepath{clip}%
\pgfsetbuttcap%
\pgfsetroundjoin%
\definecolor{currentfill}{rgb}{1.000000,0.498039,0.054902}%
\pgfsetfillcolor{currentfill}%
\pgfsetlinewidth{1.003750pt}%
\definecolor{currentstroke}{rgb}{1.000000,0.498039,0.054902}%
\pgfsetstrokecolor{currentstroke}%
\pgfsetdash{}{0pt}%
\pgfpathmoveto{\pgfqpoint{1.673922in}{3.136837in}}%
\pgfpathcurveto{\pgfqpoint{1.684972in}{3.136837in}}{\pgfqpoint{1.695571in}{3.141228in}}{\pgfqpoint{1.703385in}{3.149041in}}%
\pgfpathcurveto{\pgfqpoint{1.711198in}{3.156855in}}{\pgfqpoint{1.715589in}{3.167454in}}{\pgfqpoint{1.715589in}{3.178504in}}%
\pgfpathcurveto{\pgfqpoint{1.715589in}{3.189554in}}{\pgfqpoint{1.711198in}{3.200153in}}{\pgfqpoint{1.703385in}{3.207967in}}%
\pgfpathcurveto{\pgfqpoint{1.695571in}{3.215780in}}{\pgfqpoint{1.684972in}{3.220171in}}{\pgfqpoint{1.673922in}{3.220171in}}%
\pgfpathcurveto{\pgfqpoint{1.662872in}{3.220171in}}{\pgfqpoint{1.652273in}{3.215780in}}{\pgfqpoint{1.644459in}{3.207967in}}%
\pgfpathcurveto{\pgfqpoint{1.636646in}{3.200153in}}{\pgfqpoint{1.632255in}{3.189554in}}{\pgfqpoint{1.632255in}{3.178504in}}%
\pgfpathcurveto{\pgfqpoint{1.632255in}{3.167454in}}{\pgfqpoint{1.636646in}{3.156855in}}{\pgfqpoint{1.644459in}{3.149041in}}%
\pgfpathcurveto{\pgfqpoint{1.652273in}{3.141228in}}{\pgfqpoint{1.662872in}{3.136837in}}{\pgfqpoint{1.673922in}{3.136837in}}%
\pgfpathclose%
\pgfusepath{stroke,fill}%
\end{pgfscope}%
\begin{pgfscope}%
\pgfpathrectangle{\pgfqpoint{0.648703in}{0.548769in}}{\pgfqpoint{5.201297in}{3.102590in}}%
\pgfusepath{clip}%
\pgfsetbuttcap%
\pgfsetroundjoin%
\definecolor{currentfill}{rgb}{1.000000,0.498039,0.054902}%
\pgfsetfillcolor{currentfill}%
\pgfsetlinewidth{1.003750pt}%
\definecolor{currentstroke}{rgb}{1.000000,0.498039,0.054902}%
\pgfsetstrokecolor{currentstroke}%
\pgfsetdash{}{0pt}%
\pgfpathmoveto{\pgfqpoint{1.471361in}{3.140985in}}%
\pgfpathcurveto{\pgfqpoint{1.482412in}{3.140985in}}{\pgfqpoint{1.493011in}{3.145375in}}{\pgfqpoint{1.500824in}{3.153189in}}%
\pgfpathcurveto{\pgfqpoint{1.508638in}{3.161003in}}{\pgfqpoint{1.513028in}{3.171602in}}{\pgfqpoint{1.513028in}{3.182652in}}%
\pgfpathcurveto{\pgfqpoint{1.513028in}{3.193702in}}{\pgfqpoint{1.508638in}{3.204301in}}{\pgfqpoint{1.500824in}{3.212115in}}%
\pgfpathcurveto{\pgfqpoint{1.493011in}{3.219928in}}{\pgfqpoint{1.482412in}{3.224319in}}{\pgfqpoint{1.471361in}{3.224319in}}%
\pgfpathcurveto{\pgfqpoint{1.460311in}{3.224319in}}{\pgfqpoint{1.449712in}{3.219928in}}{\pgfqpoint{1.441899in}{3.212115in}}%
\pgfpathcurveto{\pgfqpoint{1.434085in}{3.204301in}}{\pgfqpoint{1.429695in}{3.193702in}}{\pgfqpoint{1.429695in}{3.182652in}}%
\pgfpathcurveto{\pgfqpoint{1.429695in}{3.171602in}}{\pgfqpoint{1.434085in}{3.161003in}}{\pgfqpoint{1.441899in}{3.153189in}}%
\pgfpathcurveto{\pgfqpoint{1.449712in}{3.145375in}}{\pgfqpoint{1.460311in}{3.140985in}}{\pgfqpoint{1.471361in}{3.140985in}}%
\pgfpathclose%
\pgfusepath{stroke,fill}%
\end{pgfscope}%
\begin{pgfscope}%
\pgfpathrectangle{\pgfqpoint{0.648703in}{0.548769in}}{\pgfqpoint{5.201297in}{3.102590in}}%
\pgfusepath{clip}%
\pgfsetbuttcap%
\pgfsetroundjoin%
\definecolor{currentfill}{rgb}{1.000000,0.498039,0.054902}%
\pgfsetfillcolor{currentfill}%
\pgfsetlinewidth{1.003750pt}%
\definecolor{currentstroke}{rgb}{1.000000,0.498039,0.054902}%
\pgfsetstrokecolor{currentstroke}%
\pgfsetdash{}{0pt}%
\pgfpathmoveto{\pgfqpoint{1.801888in}{3.165872in}}%
\pgfpathcurveto{\pgfqpoint{1.812938in}{3.165872in}}{\pgfqpoint{1.823537in}{3.170263in}}{\pgfqpoint{1.831351in}{3.178076in}}%
\pgfpathcurveto{\pgfqpoint{1.839164in}{3.185890in}}{\pgfqpoint{1.843555in}{3.196489in}}{\pgfqpoint{1.843555in}{3.207539in}}%
\pgfpathcurveto{\pgfqpoint{1.843555in}{3.218589in}}{\pgfqpoint{1.839164in}{3.229188in}}{\pgfqpoint{1.831351in}{3.237002in}}%
\pgfpathcurveto{\pgfqpoint{1.823537in}{3.244815in}}{\pgfqpoint{1.812938in}{3.249206in}}{\pgfqpoint{1.801888in}{3.249206in}}%
\pgfpathcurveto{\pgfqpoint{1.790838in}{3.249206in}}{\pgfqpoint{1.780239in}{3.244815in}}{\pgfqpoint{1.772425in}{3.237002in}}%
\pgfpathcurveto{\pgfqpoint{1.764612in}{3.229188in}}{\pgfqpoint{1.760221in}{3.218589in}}{\pgfqpoint{1.760221in}{3.207539in}}%
\pgfpathcurveto{\pgfqpoint{1.760221in}{3.196489in}}{\pgfqpoint{1.764612in}{3.185890in}}{\pgfqpoint{1.772425in}{3.178076in}}%
\pgfpathcurveto{\pgfqpoint{1.780239in}{3.170263in}}{\pgfqpoint{1.790838in}{3.165872in}}{\pgfqpoint{1.801888in}{3.165872in}}%
\pgfpathclose%
\pgfusepath{stroke,fill}%
\end{pgfscope}%
\begin{pgfscope}%
\pgfpathrectangle{\pgfqpoint{0.648703in}{0.548769in}}{\pgfqpoint{5.201297in}{3.102590in}}%
\pgfusepath{clip}%
\pgfsetbuttcap%
\pgfsetroundjoin%
\definecolor{currentfill}{rgb}{1.000000,0.498039,0.054902}%
\pgfsetfillcolor{currentfill}%
\pgfsetlinewidth{1.003750pt}%
\definecolor{currentstroke}{rgb}{1.000000,0.498039,0.054902}%
\pgfsetstrokecolor{currentstroke}%
\pgfsetdash{}{0pt}%
\pgfpathmoveto{\pgfqpoint{1.964142in}{3.145133in}}%
\pgfpathcurveto{\pgfqpoint{1.975192in}{3.145133in}}{\pgfqpoint{1.985791in}{3.149523in}}{\pgfqpoint{1.993604in}{3.157337in}}%
\pgfpathcurveto{\pgfqpoint{2.001418in}{3.165151in}}{\pgfqpoint{2.005808in}{3.175750in}}{\pgfqpoint{2.005808in}{3.186800in}}%
\pgfpathcurveto{\pgfqpoint{2.005808in}{3.197850in}}{\pgfqpoint{2.001418in}{3.208449in}}{\pgfqpoint{1.993604in}{3.216262in}}%
\pgfpathcurveto{\pgfqpoint{1.985791in}{3.224076in}}{\pgfqpoint{1.975192in}{3.228466in}}{\pgfqpoint{1.964142in}{3.228466in}}%
\pgfpathcurveto{\pgfqpoint{1.953091in}{3.228466in}}{\pgfqpoint{1.942492in}{3.224076in}}{\pgfqpoint{1.934679in}{3.216262in}}%
\pgfpathcurveto{\pgfqpoint{1.926865in}{3.208449in}}{\pgfqpoint{1.922475in}{3.197850in}}{\pgfqpoint{1.922475in}{3.186800in}}%
\pgfpathcurveto{\pgfqpoint{1.922475in}{3.175750in}}{\pgfqpoint{1.926865in}{3.165151in}}{\pgfqpoint{1.934679in}{3.157337in}}%
\pgfpathcurveto{\pgfqpoint{1.942492in}{3.149523in}}{\pgfqpoint{1.953091in}{3.145133in}}{\pgfqpoint{1.964142in}{3.145133in}}%
\pgfpathclose%
\pgfusepath{stroke,fill}%
\end{pgfscope}%
\begin{pgfscope}%
\pgfpathrectangle{\pgfqpoint{0.648703in}{0.548769in}}{\pgfqpoint{5.201297in}{3.102590in}}%
\pgfusepath{clip}%
\pgfsetbuttcap%
\pgfsetroundjoin%
\definecolor{currentfill}{rgb}{1.000000,0.498039,0.054902}%
\pgfsetfillcolor{currentfill}%
\pgfsetlinewidth{1.003750pt}%
\definecolor{currentstroke}{rgb}{1.000000,0.498039,0.054902}%
\pgfsetstrokecolor{currentstroke}%
\pgfsetdash{}{0pt}%
\pgfpathmoveto{\pgfqpoint{1.226487in}{3.157577in}}%
\pgfpathcurveto{\pgfqpoint{1.237537in}{3.157577in}}{\pgfqpoint{1.248136in}{3.161967in}}{\pgfqpoint{1.255950in}{3.169780in}}%
\pgfpathcurveto{\pgfqpoint{1.263764in}{3.177594in}}{\pgfqpoint{1.268154in}{3.188193in}}{\pgfqpoint{1.268154in}{3.199243in}}%
\pgfpathcurveto{\pgfqpoint{1.268154in}{3.210293in}}{\pgfqpoint{1.263764in}{3.220892in}}{\pgfqpoint{1.255950in}{3.228706in}}%
\pgfpathcurveto{\pgfqpoint{1.248136in}{3.236520in}}{\pgfqpoint{1.237537in}{3.240910in}}{\pgfqpoint{1.226487in}{3.240910in}}%
\pgfpathcurveto{\pgfqpoint{1.215437in}{3.240910in}}{\pgfqpoint{1.204838in}{3.236520in}}{\pgfqpoint{1.197025in}{3.228706in}}%
\pgfpathcurveto{\pgfqpoint{1.189211in}{3.220892in}}{\pgfqpoint{1.184821in}{3.210293in}}{\pgfqpoint{1.184821in}{3.199243in}}%
\pgfpathcurveto{\pgfqpoint{1.184821in}{3.188193in}}{\pgfqpoint{1.189211in}{3.177594in}}{\pgfqpoint{1.197025in}{3.169780in}}%
\pgfpathcurveto{\pgfqpoint{1.204838in}{3.161967in}}{\pgfqpoint{1.215437in}{3.157577in}}{\pgfqpoint{1.226487in}{3.157577in}}%
\pgfpathclose%
\pgfusepath{stroke,fill}%
\end{pgfscope}%
\begin{pgfscope}%
\pgfpathrectangle{\pgfqpoint{0.648703in}{0.548769in}}{\pgfqpoint{5.201297in}{3.102590in}}%
\pgfusepath{clip}%
\pgfsetbuttcap%
\pgfsetroundjoin%
\definecolor{currentfill}{rgb}{1.000000,0.498039,0.054902}%
\pgfsetfillcolor{currentfill}%
\pgfsetlinewidth{1.003750pt}%
\definecolor{currentstroke}{rgb}{1.000000,0.498039,0.054902}%
\pgfsetstrokecolor{currentstroke}%
\pgfsetdash{}{0pt}%
\pgfpathmoveto{\pgfqpoint{1.502885in}{3.323490in}}%
\pgfpathcurveto{\pgfqpoint{1.513935in}{3.323490in}}{\pgfqpoint{1.524534in}{3.327881in}}{\pgfqpoint{1.532347in}{3.335694in}}%
\pgfpathcurveto{\pgfqpoint{1.540161in}{3.343508in}}{\pgfqpoint{1.544551in}{3.354107in}}{\pgfqpoint{1.544551in}{3.365157in}}%
\pgfpathcurveto{\pgfqpoint{1.544551in}{3.376207in}}{\pgfqpoint{1.540161in}{3.386806in}}{\pgfqpoint{1.532347in}{3.394620in}}%
\pgfpathcurveto{\pgfqpoint{1.524534in}{3.402434in}}{\pgfqpoint{1.513935in}{3.406824in}}{\pgfqpoint{1.502885in}{3.406824in}}%
\pgfpathcurveto{\pgfqpoint{1.491835in}{3.406824in}}{\pgfqpoint{1.481236in}{3.402434in}}{\pgfqpoint{1.473422in}{3.394620in}}%
\pgfpathcurveto{\pgfqpoint{1.465608in}{3.386806in}}{\pgfqpoint{1.461218in}{3.376207in}}{\pgfqpoint{1.461218in}{3.365157in}}%
\pgfpathcurveto{\pgfqpoint{1.461218in}{3.354107in}}{\pgfqpoint{1.465608in}{3.343508in}}{\pgfqpoint{1.473422in}{3.335694in}}%
\pgfpathcurveto{\pgfqpoint{1.481236in}{3.327881in}}{\pgfqpoint{1.491835in}{3.323490in}}{\pgfqpoint{1.502885in}{3.323490in}}%
\pgfpathclose%
\pgfusepath{stroke,fill}%
\end{pgfscope}%
\begin{pgfscope}%
\pgfpathrectangle{\pgfqpoint{0.648703in}{0.548769in}}{\pgfqpoint{5.201297in}{3.102590in}}%
\pgfusepath{clip}%
\pgfsetbuttcap%
\pgfsetroundjoin%
\definecolor{currentfill}{rgb}{0.121569,0.466667,0.705882}%
\pgfsetfillcolor{currentfill}%
\pgfsetlinewidth{1.003750pt}%
\definecolor{currentstroke}{rgb}{0.121569,0.466667,0.705882}%
\pgfsetstrokecolor{currentstroke}%
\pgfsetdash{}{0pt}%
\pgfpathmoveto{\pgfqpoint{1.198932in}{2.410964in}}%
\pgfpathcurveto{\pgfqpoint{1.209982in}{2.410964in}}{\pgfqpoint{1.220581in}{2.415354in}}{\pgfqpoint{1.228395in}{2.423168in}}%
\pgfpathcurveto{\pgfqpoint{1.236209in}{2.430982in}}{\pgfqpoint{1.240599in}{2.441581in}}{\pgfqpoint{1.240599in}{2.452631in}}%
\pgfpathcurveto{\pgfqpoint{1.240599in}{2.463681in}}{\pgfqpoint{1.236209in}{2.474280in}}{\pgfqpoint{1.228395in}{2.482094in}}%
\pgfpathcurveto{\pgfqpoint{1.220581in}{2.489907in}}{\pgfqpoint{1.209982in}{2.494297in}}{\pgfqpoint{1.198932in}{2.494297in}}%
\pgfpathcurveto{\pgfqpoint{1.187882in}{2.494297in}}{\pgfqpoint{1.177283in}{2.489907in}}{\pgfqpoint{1.169469in}{2.482094in}}%
\pgfpathcurveto{\pgfqpoint{1.161656in}{2.474280in}}{\pgfqpoint{1.157266in}{2.463681in}}{\pgfqpoint{1.157266in}{2.452631in}}%
\pgfpathcurveto{\pgfqpoint{1.157266in}{2.441581in}}{\pgfqpoint{1.161656in}{2.430982in}}{\pgfqpoint{1.169469in}{2.423168in}}%
\pgfpathcurveto{\pgfqpoint{1.177283in}{2.415354in}}{\pgfqpoint{1.187882in}{2.410964in}}{\pgfqpoint{1.198932in}{2.410964in}}%
\pgfpathclose%
\pgfusepath{stroke,fill}%
\end{pgfscope}%
\begin{pgfscope}%
\pgfpathrectangle{\pgfqpoint{0.648703in}{0.548769in}}{\pgfqpoint{5.201297in}{3.102590in}}%
\pgfusepath{clip}%
\pgfsetbuttcap%
\pgfsetroundjoin%
\definecolor{currentfill}{rgb}{1.000000,0.498039,0.054902}%
\pgfsetfillcolor{currentfill}%
\pgfsetlinewidth{1.003750pt}%
\definecolor{currentstroke}{rgb}{1.000000,0.498039,0.054902}%
\pgfsetstrokecolor{currentstroke}%
\pgfsetdash{}{0pt}%
\pgfpathmoveto{\pgfqpoint{1.276871in}{3.199055in}}%
\pgfpathcurveto{\pgfqpoint{1.287921in}{3.199055in}}{\pgfqpoint{1.298520in}{3.203445in}}{\pgfqpoint{1.306334in}{3.211259in}}%
\pgfpathcurveto{\pgfqpoint{1.314147in}{3.219073in}}{\pgfqpoint{1.318538in}{3.229672in}}{\pgfqpoint{1.318538in}{3.240722in}}%
\pgfpathcurveto{\pgfqpoint{1.318538in}{3.251772in}}{\pgfqpoint{1.314147in}{3.262371in}}{\pgfqpoint{1.306334in}{3.270185in}}%
\pgfpathcurveto{\pgfqpoint{1.298520in}{3.277998in}}{\pgfqpoint{1.287921in}{3.282388in}}{\pgfqpoint{1.276871in}{3.282388in}}%
\pgfpathcurveto{\pgfqpoint{1.265821in}{3.282388in}}{\pgfqpoint{1.255222in}{3.277998in}}{\pgfqpoint{1.247408in}{3.270185in}}%
\pgfpathcurveto{\pgfqpoint{1.239595in}{3.262371in}}{\pgfqpoint{1.235204in}{3.251772in}}{\pgfqpoint{1.235204in}{3.240722in}}%
\pgfpathcurveto{\pgfqpoint{1.235204in}{3.229672in}}{\pgfqpoint{1.239595in}{3.219073in}}{\pgfqpoint{1.247408in}{3.211259in}}%
\pgfpathcurveto{\pgfqpoint{1.255222in}{3.203445in}}{\pgfqpoint{1.265821in}{3.199055in}}{\pgfqpoint{1.276871in}{3.199055in}}%
\pgfpathclose%
\pgfusepath{stroke,fill}%
\end{pgfscope}%
\begin{pgfscope}%
\pgfpathrectangle{\pgfqpoint{0.648703in}{0.548769in}}{\pgfqpoint{5.201297in}{3.102590in}}%
\pgfusepath{clip}%
\pgfsetbuttcap%
\pgfsetroundjoin%
\definecolor{currentfill}{rgb}{0.121569,0.466667,0.705882}%
\pgfsetfillcolor{currentfill}%
\pgfsetlinewidth{1.003750pt}%
\definecolor{currentstroke}{rgb}{0.121569,0.466667,0.705882}%
\pgfsetstrokecolor{currentstroke}%
\pgfsetdash{}{0pt}%
\pgfpathmoveto{\pgfqpoint{2.544982in}{3.124394in}}%
\pgfpathcurveto{\pgfqpoint{2.556032in}{3.124394in}}{\pgfqpoint{2.566631in}{3.128784in}}{\pgfqpoint{2.574445in}{3.136598in}}%
\pgfpathcurveto{\pgfqpoint{2.582258in}{3.144411in}}{\pgfqpoint{2.586649in}{3.155010in}}{\pgfqpoint{2.586649in}{3.166060in}}%
\pgfpathcurveto{\pgfqpoint{2.586649in}{3.177111in}}{\pgfqpoint{2.582258in}{3.187710in}}{\pgfqpoint{2.574445in}{3.195523in}}%
\pgfpathcurveto{\pgfqpoint{2.566631in}{3.203337in}}{\pgfqpoint{2.556032in}{3.207727in}}{\pgfqpoint{2.544982in}{3.207727in}}%
\pgfpathcurveto{\pgfqpoint{2.533932in}{3.207727in}}{\pgfqpoint{2.523333in}{3.203337in}}{\pgfqpoint{2.515519in}{3.195523in}}%
\pgfpathcurveto{\pgfqpoint{2.507706in}{3.187710in}}{\pgfqpoint{2.503315in}{3.177111in}}{\pgfqpoint{2.503315in}{3.166060in}}%
\pgfpathcurveto{\pgfqpoint{2.503315in}{3.155010in}}{\pgfqpoint{2.507706in}{3.144411in}}{\pgfqpoint{2.515519in}{3.136598in}}%
\pgfpathcurveto{\pgfqpoint{2.523333in}{3.128784in}}{\pgfqpoint{2.533932in}{3.124394in}}{\pgfqpoint{2.544982in}{3.124394in}}%
\pgfpathclose%
\pgfusepath{stroke,fill}%
\end{pgfscope}%
\begin{pgfscope}%
\pgfpathrectangle{\pgfqpoint{0.648703in}{0.548769in}}{\pgfqpoint{5.201297in}{3.102590in}}%
\pgfusepath{clip}%
\pgfsetbuttcap%
\pgfsetroundjoin%
\definecolor{currentfill}{rgb}{0.121569,0.466667,0.705882}%
\pgfsetfillcolor{currentfill}%
\pgfsetlinewidth{1.003750pt}%
\definecolor{currentstroke}{rgb}{0.121569,0.466667,0.705882}%
\pgfsetstrokecolor{currentstroke}%
\pgfsetdash{}{0pt}%
\pgfpathmoveto{\pgfqpoint{2.006544in}{3.132690in}}%
\pgfpathcurveto{\pgfqpoint{2.017594in}{3.132690in}}{\pgfqpoint{2.028193in}{3.137080in}}{\pgfqpoint{2.036007in}{3.144893in}}%
\pgfpathcurveto{\pgfqpoint{2.043821in}{3.152707in}}{\pgfqpoint{2.048211in}{3.163306in}}{\pgfqpoint{2.048211in}{3.174356in}}%
\pgfpathcurveto{\pgfqpoint{2.048211in}{3.185406in}}{\pgfqpoint{2.043821in}{3.196005in}}{\pgfqpoint{2.036007in}{3.203819in}}%
\pgfpathcurveto{\pgfqpoint{2.028193in}{3.211633in}}{\pgfqpoint{2.017594in}{3.216023in}}{\pgfqpoint{2.006544in}{3.216023in}}%
\pgfpathcurveto{\pgfqpoint{1.995494in}{3.216023in}}{\pgfqpoint{1.984895in}{3.211633in}}{\pgfqpoint{1.977081in}{3.203819in}}%
\pgfpathcurveto{\pgfqpoint{1.969268in}{3.196005in}}{\pgfqpoint{1.964878in}{3.185406in}}{\pgfqpoint{1.964878in}{3.174356in}}%
\pgfpathcurveto{\pgfqpoint{1.964878in}{3.163306in}}{\pgfqpoint{1.969268in}{3.152707in}}{\pgfqpoint{1.977081in}{3.144893in}}%
\pgfpathcurveto{\pgfqpoint{1.984895in}{3.137080in}}{\pgfqpoint{1.995494in}{3.132690in}}{\pgfqpoint{2.006544in}{3.132690in}}%
\pgfpathclose%
\pgfusepath{stroke,fill}%
\end{pgfscope}%
\begin{pgfscope}%
\pgfpathrectangle{\pgfqpoint{0.648703in}{0.548769in}}{\pgfqpoint{5.201297in}{3.102590in}}%
\pgfusepath{clip}%
\pgfsetbuttcap%
\pgfsetroundjoin%
\definecolor{currentfill}{rgb}{0.121569,0.466667,0.705882}%
\pgfsetfillcolor{currentfill}%
\pgfsetlinewidth{1.003750pt}%
\definecolor{currentstroke}{rgb}{0.121569,0.466667,0.705882}%
\pgfsetstrokecolor{currentstroke}%
\pgfsetdash{}{0pt}%
\pgfpathmoveto{\pgfqpoint{1.593085in}{3.132690in}}%
\pgfpathcurveto{\pgfqpoint{1.604135in}{3.132690in}}{\pgfqpoint{1.614734in}{3.137080in}}{\pgfqpoint{1.622548in}{3.144893in}}%
\pgfpathcurveto{\pgfqpoint{1.630361in}{3.152707in}}{\pgfqpoint{1.634752in}{3.163306in}}{\pgfqpoint{1.634752in}{3.174356in}}%
\pgfpathcurveto{\pgfqpoint{1.634752in}{3.185406in}}{\pgfqpoint{1.630361in}{3.196005in}}{\pgfqpoint{1.622548in}{3.203819in}}%
\pgfpathcurveto{\pgfqpoint{1.614734in}{3.211633in}}{\pgfqpoint{1.604135in}{3.216023in}}{\pgfqpoint{1.593085in}{3.216023in}}%
\pgfpathcurveto{\pgfqpoint{1.582035in}{3.216023in}}{\pgfqpoint{1.571436in}{3.211633in}}{\pgfqpoint{1.563622in}{3.203819in}}%
\pgfpathcurveto{\pgfqpoint{1.555809in}{3.196005in}}{\pgfqpoint{1.551418in}{3.185406in}}{\pgfqpoint{1.551418in}{3.174356in}}%
\pgfpathcurveto{\pgfqpoint{1.551418in}{3.163306in}}{\pgfqpoint{1.555809in}{3.152707in}}{\pgfqpoint{1.563622in}{3.144893in}}%
\pgfpathcurveto{\pgfqpoint{1.571436in}{3.137080in}}{\pgfqpoint{1.582035in}{3.132690in}}{\pgfqpoint{1.593085in}{3.132690in}}%
\pgfpathclose%
\pgfusepath{stroke,fill}%
\end{pgfscope}%
\begin{pgfscope}%
\pgfpathrectangle{\pgfqpoint{0.648703in}{0.548769in}}{\pgfqpoint{5.201297in}{3.102590in}}%
\pgfusepath{clip}%
\pgfsetbuttcap%
\pgfsetroundjoin%
\definecolor{currentfill}{rgb}{1.000000,0.498039,0.054902}%
\pgfsetfillcolor{currentfill}%
\pgfsetlinewidth{1.003750pt}%
\definecolor{currentstroke}{rgb}{1.000000,0.498039,0.054902}%
\pgfsetstrokecolor{currentstroke}%
\pgfsetdash{}{0pt}%
\pgfpathmoveto{\pgfqpoint{1.387626in}{3.136837in}}%
\pgfpathcurveto{\pgfqpoint{1.398676in}{3.136837in}}{\pgfqpoint{1.409275in}{3.141228in}}{\pgfqpoint{1.417089in}{3.149041in}}%
\pgfpathcurveto{\pgfqpoint{1.424903in}{3.156855in}}{\pgfqpoint{1.429293in}{3.167454in}}{\pgfqpoint{1.429293in}{3.178504in}}%
\pgfpathcurveto{\pgfqpoint{1.429293in}{3.189554in}}{\pgfqpoint{1.424903in}{3.200153in}}{\pgfqpoint{1.417089in}{3.207967in}}%
\pgfpathcurveto{\pgfqpoint{1.409275in}{3.215780in}}{\pgfqpoint{1.398676in}{3.220171in}}{\pgfqpoint{1.387626in}{3.220171in}}%
\pgfpathcurveto{\pgfqpoint{1.376576in}{3.220171in}}{\pgfqpoint{1.365977in}{3.215780in}}{\pgfqpoint{1.358163in}{3.207967in}}%
\pgfpathcurveto{\pgfqpoint{1.350350in}{3.200153in}}{\pgfqpoint{1.345960in}{3.189554in}}{\pgfqpoint{1.345960in}{3.178504in}}%
\pgfpathcurveto{\pgfqpoint{1.345960in}{3.167454in}}{\pgfqpoint{1.350350in}{3.156855in}}{\pgfqpoint{1.358163in}{3.149041in}}%
\pgfpathcurveto{\pgfqpoint{1.365977in}{3.141228in}}{\pgfqpoint{1.376576in}{3.136837in}}{\pgfqpoint{1.387626in}{3.136837in}}%
\pgfpathclose%
\pgfusepath{stroke,fill}%
\end{pgfscope}%
\begin{pgfscope}%
\pgfpathrectangle{\pgfqpoint{0.648703in}{0.548769in}}{\pgfqpoint{5.201297in}{3.102590in}}%
\pgfusepath{clip}%
\pgfsetbuttcap%
\pgfsetroundjoin%
\definecolor{currentfill}{rgb}{1.000000,0.498039,0.054902}%
\pgfsetfillcolor{currentfill}%
\pgfsetlinewidth{1.003750pt}%
\definecolor{currentstroke}{rgb}{1.000000,0.498039,0.054902}%
\pgfsetstrokecolor{currentstroke}%
\pgfsetdash{}{0pt}%
\pgfpathmoveto{\pgfqpoint{1.261132in}{3.244681in}}%
\pgfpathcurveto{\pgfqpoint{1.272182in}{3.244681in}}{\pgfqpoint{1.282781in}{3.249072in}}{\pgfqpoint{1.290594in}{3.256885in}}%
\pgfpathcurveto{\pgfqpoint{1.298408in}{3.264699in}}{\pgfqpoint{1.302798in}{3.275298in}}{\pgfqpoint{1.302798in}{3.286348in}}%
\pgfpathcurveto{\pgfqpoint{1.302798in}{3.297398in}}{\pgfqpoint{1.298408in}{3.307997in}}{\pgfqpoint{1.290594in}{3.315811in}}%
\pgfpathcurveto{\pgfqpoint{1.282781in}{3.323624in}}{\pgfqpoint{1.272182in}{3.328015in}}{\pgfqpoint{1.261132in}{3.328015in}}%
\pgfpathcurveto{\pgfqpoint{1.250082in}{3.328015in}}{\pgfqpoint{1.239483in}{3.323624in}}{\pgfqpoint{1.231669in}{3.315811in}}%
\pgfpathcurveto{\pgfqpoint{1.223855in}{3.307997in}}{\pgfqpoint{1.219465in}{3.297398in}}{\pgfqpoint{1.219465in}{3.286348in}}%
\pgfpathcurveto{\pgfqpoint{1.219465in}{3.275298in}}{\pgfqpoint{1.223855in}{3.264699in}}{\pgfqpoint{1.231669in}{3.256885in}}%
\pgfpathcurveto{\pgfqpoint{1.239483in}{3.249072in}}{\pgfqpoint{1.250082in}{3.244681in}}{\pgfqpoint{1.261132in}{3.244681in}}%
\pgfpathclose%
\pgfusepath{stroke,fill}%
\end{pgfscope}%
\begin{pgfscope}%
\pgfpathrectangle{\pgfqpoint{0.648703in}{0.548769in}}{\pgfqpoint{5.201297in}{3.102590in}}%
\pgfusepath{clip}%
\pgfsetbuttcap%
\pgfsetroundjoin%
\definecolor{currentfill}{rgb}{1.000000,0.498039,0.054902}%
\pgfsetfillcolor{currentfill}%
\pgfsetlinewidth{1.003750pt}%
\definecolor{currentstroke}{rgb}{1.000000,0.498039,0.054902}%
\pgfsetstrokecolor{currentstroke}%
\pgfsetdash{}{0pt}%
\pgfpathmoveto{\pgfqpoint{1.276871in}{3.190759in}}%
\pgfpathcurveto{\pgfqpoint{1.287921in}{3.190759in}}{\pgfqpoint{1.298520in}{3.195150in}}{\pgfqpoint{1.306334in}{3.202963in}}%
\pgfpathcurveto{\pgfqpoint{1.314147in}{3.210777in}}{\pgfqpoint{1.318538in}{3.221376in}}{\pgfqpoint{1.318538in}{3.232426in}}%
\pgfpathcurveto{\pgfqpoint{1.318538in}{3.243476in}}{\pgfqpoint{1.314147in}{3.254075in}}{\pgfqpoint{1.306334in}{3.261889in}}%
\pgfpathcurveto{\pgfqpoint{1.298520in}{3.269702in}}{\pgfqpoint{1.287921in}{3.274093in}}{\pgfqpoint{1.276871in}{3.274093in}}%
\pgfpathcurveto{\pgfqpoint{1.265821in}{3.274093in}}{\pgfqpoint{1.255222in}{3.269702in}}{\pgfqpoint{1.247408in}{3.261889in}}%
\pgfpathcurveto{\pgfqpoint{1.239595in}{3.254075in}}{\pgfqpoint{1.235204in}{3.243476in}}{\pgfqpoint{1.235204in}{3.232426in}}%
\pgfpathcurveto{\pgfqpoint{1.235204in}{3.221376in}}{\pgfqpoint{1.239595in}{3.210777in}}{\pgfqpoint{1.247408in}{3.202963in}}%
\pgfpathcurveto{\pgfqpoint{1.255222in}{3.195150in}}{\pgfqpoint{1.265821in}{3.190759in}}{\pgfqpoint{1.276871in}{3.190759in}}%
\pgfpathclose%
\pgfusepath{stroke,fill}%
\end{pgfscope}%
\begin{pgfscope}%
\pgfpathrectangle{\pgfqpoint{0.648703in}{0.548769in}}{\pgfqpoint{5.201297in}{3.102590in}}%
\pgfusepath{clip}%
\pgfsetbuttcap%
\pgfsetroundjoin%
\definecolor{currentfill}{rgb}{0.121569,0.466667,0.705882}%
\pgfsetfillcolor{currentfill}%
\pgfsetlinewidth{1.003750pt}%
\definecolor{currentstroke}{rgb}{0.121569,0.466667,0.705882}%
\pgfsetstrokecolor{currentstroke}%
\pgfsetdash{}{0pt}%
\pgfpathmoveto{\pgfqpoint{0.889139in}{0.664720in}}%
\pgfpathcurveto{\pgfqpoint{0.900189in}{0.664720in}}{\pgfqpoint{0.910788in}{0.669111in}}{\pgfqpoint{0.918602in}{0.676924in}}%
\pgfpathcurveto{\pgfqpoint{0.926415in}{0.684738in}}{\pgfqpoint{0.930806in}{0.695337in}}{\pgfqpoint{0.930806in}{0.706387in}}%
\pgfpathcurveto{\pgfqpoint{0.930806in}{0.717437in}}{\pgfqpoint{0.926415in}{0.728036in}}{\pgfqpoint{0.918602in}{0.735850in}}%
\pgfpathcurveto{\pgfqpoint{0.910788in}{0.743663in}}{\pgfqpoint{0.900189in}{0.748054in}}{\pgfqpoint{0.889139in}{0.748054in}}%
\pgfpathcurveto{\pgfqpoint{0.878089in}{0.748054in}}{\pgfqpoint{0.867490in}{0.743663in}}{\pgfqpoint{0.859676in}{0.735850in}}%
\pgfpathcurveto{\pgfqpoint{0.851862in}{0.728036in}}{\pgfqpoint{0.847472in}{0.717437in}}{\pgfqpoint{0.847472in}{0.706387in}}%
\pgfpathcurveto{\pgfqpoint{0.847472in}{0.695337in}}{\pgfqpoint{0.851862in}{0.684738in}}{\pgfqpoint{0.859676in}{0.676924in}}%
\pgfpathcurveto{\pgfqpoint{0.867490in}{0.669111in}}{\pgfqpoint{0.878089in}{0.664720in}}{\pgfqpoint{0.889139in}{0.664720in}}%
\pgfpathclose%
\pgfusepath{stroke,fill}%
\end{pgfscope}%
\begin{pgfscope}%
\pgfpathrectangle{\pgfqpoint{0.648703in}{0.548769in}}{\pgfqpoint{5.201297in}{3.102590in}}%
\pgfusepath{clip}%
\pgfsetbuttcap%
\pgfsetroundjoin%
\definecolor{currentfill}{rgb}{1.000000,0.498039,0.054902}%
\pgfsetfillcolor{currentfill}%
\pgfsetlinewidth{1.003750pt}%
\definecolor{currentstroke}{rgb}{1.000000,0.498039,0.054902}%
\pgfsetstrokecolor{currentstroke}%
\pgfsetdash{}{0pt}%
\pgfpathmoveto{\pgfqpoint{1.774868in}{3.145133in}}%
\pgfpathcurveto{\pgfqpoint{1.785918in}{3.145133in}}{\pgfqpoint{1.796517in}{3.149523in}}{\pgfqpoint{1.804331in}{3.157337in}}%
\pgfpathcurveto{\pgfqpoint{1.812144in}{3.165151in}}{\pgfqpoint{1.816535in}{3.175750in}}{\pgfqpoint{1.816535in}{3.186800in}}%
\pgfpathcurveto{\pgfqpoint{1.816535in}{3.197850in}}{\pgfqpoint{1.812144in}{3.208449in}}{\pgfqpoint{1.804331in}{3.216262in}}%
\pgfpathcurveto{\pgfqpoint{1.796517in}{3.224076in}}{\pgfqpoint{1.785918in}{3.228466in}}{\pgfqpoint{1.774868in}{3.228466in}}%
\pgfpathcurveto{\pgfqpoint{1.763818in}{3.228466in}}{\pgfqpoint{1.753219in}{3.224076in}}{\pgfqpoint{1.745405in}{3.216262in}}%
\pgfpathcurveto{\pgfqpoint{1.737592in}{3.208449in}}{\pgfqpoint{1.733201in}{3.197850in}}{\pgfqpoint{1.733201in}{3.186800in}}%
\pgfpathcurveto{\pgfqpoint{1.733201in}{3.175750in}}{\pgfqpoint{1.737592in}{3.165151in}}{\pgfqpoint{1.745405in}{3.157337in}}%
\pgfpathcurveto{\pgfqpoint{1.753219in}{3.149523in}}{\pgfqpoint{1.763818in}{3.145133in}}{\pgfqpoint{1.774868in}{3.145133in}}%
\pgfpathclose%
\pgfusepath{stroke,fill}%
\end{pgfscope}%
\begin{pgfscope}%
\pgfpathrectangle{\pgfqpoint{0.648703in}{0.548769in}}{\pgfqpoint{5.201297in}{3.102590in}}%
\pgfusepath{clip}%
\pgfsetbuttcap%
\pgfsetroundjoin%
\definecolor{currentfill}{rgb}{0.121569,0.466667,0.705882}%
\pgfsetfillcolor{currentfill}%
\pgfsetlinewidth{1.003750pt}%
\definecolor{currentstroke}{rgb}{0.121569,0.466667,0.705882}%
\pgfsetstrokecolor{currentstroke}%
\pgfsetdash{}{0pt}%
\pgfpathmoveto{\pgfqpoint{1.061068in}{0.648129in}}%
\pgfpathcurveto{\pgfqpoint{1.072118in}{0.648129in}}{\pgfqpoint{1.082717in}{0.652519in}}{\pgfqpoint{1.090531in}{0.660333in}}%
\pgfpathcurveto{\pgfqpoint{1.098344in}{0.668146in}}{\pgfqpoint{1.102735in}{0.678745in}}{\pgfqpoint{1.102735in}{0.689796in}}%
\pgfpathcurveto{\pgfqpoint{1.102735in}{0.700846in}}{\pgfqpoint{1.098344in}{0.711445in}}{\pgfqpoint{1.090531in}{0.719258in}}%
\pgfpathcurveto{\pgfqpoint{1.082717in}{0.727072in}}{\pgfqpoint{1.072118in}{0.731462in}}{\pgfqpoint{1.061068in}{0.731462in}}%
\pgfpathcurveto{\pgfqpoint{1.050018in}{0.731462in}}{\pgfqpoint{1.039419in}{0.727072in}}{\pgfqpoint{1.031605in}{0.719258in}}%
\pgfpathcurveto{\pgfqpoint{1.023792in}{0.711445in}}{\pgfqpoint{1.019401in}{0.700846in}}{\pgfqpoint{1.019401in}{0.689796in}}%
\pgfpathcurveto{\pgfqpoint{1.019401in}{0.678745in}}{\pgfqpoint{1.023792in}{0.668146in}}{\pgfqpoint{1.031605in}{0.660333in}}%
\pgfpathcurveto{\pgfqpoint{1.039419in}{0.652519in}}{\pgfqpoint{1.050018in}{0.648129in}}{\pgfqpoint{1.061068in}{0.648129in}}%
\pgfpathclose%
\pgfusepath{stroke,fill}%
\end{pgfscope}%
\begin{pgfscope}%
\pgfpathrectangle{\pgfqpoint{0.648703in}{0.548769in}}{\pgfqpoint{5.201297in}{3.102590in}}%
\pgfusepath{clip}%
\pgfsetbuttcap%
\pgfsetroundjoin%
\definecolor{currentfill}{rgb}{1.000000,0.498039,0.054902}%
\pgfsetfillcolor{currentfill}%
\pgfsetlinewidth{1.003750pt}%
\definecolor{currentstroke}{rgb}{1.000000,0.498039,0.054902}%
\pgfsetstrokecolor{currentstroke}%
\pgfsetdash{}{0pt}%
\pgfpathmoveto{\pgfqpoint{1.744459in}{3.149281in}}%
\pgfpathcurveto{\pgfqpoint{1.755509in}{3.149281in}}{\pgfqpoint{1.766109in}{3.153671in}}{\pgfqpoint{1.773922in}{3.161485in}}%
\pgfpathcurveto{\pgfqpoint{1.781736in}{3.169298in}}{\pgfqpoint{1.786126in}{3.179897in}}{\pgfqpoint{1.786126in}{3.190948in}}%
\pgfpathcurveto{\pgfqpoint{1.786126in}{3.201998in}}{\pgfqpoint{1.781736in}{3.212597in}}{\pgfqpoint{1.773922in}{3.220410in}}%
\pgfpathcurveto{\pgfqpoint{1.766109in}{3.228224in}}{\pgfqpoint{1.755509in}{3.232614in}}{\pgfqpoint{1.744459in}{3.232614in}}%
\pgfpathcurveto{\pgfqpoint{1.733409in}{3.232614in}}{\pgfqpoint{1.722810in}{3.228224in}}{\pgfqpoint{1.714997in}{3.220410in}}%
\pgfpathcurveto{\pgfqpoint{1.707183in}{3.212597in}}{\pgfqpoint{1.702793in}{3.201998in}}{\pgfqpoint{1.702793in}{3.190948in}}%
\pgfpathcurveto{\pgfqpoint{1.702793in}{3.179897in}}{\pgfqpoint{1.707183in}{3.169298in}}{\pgfqpoint{1.714997in}{3.161485in}}%
\pgfpathcurveto{\pgfqpoint{1.722810in}{3.153671in}}{\pgfqpoint{1.733409in}{3.149281in}}{\pgfqpoint{1.744459in}{3.149281in}}%
\pgfpathclose%
\pgfusepath{stroke,fill}%
\end{pgfscope}%
\begin{pgfscope}%
\pgfpathrectangle{\pgfqpoint{0.648703in}{0.548769in}}{\pgfqpoint{5.201297in}{3.102590in}}%
\pgfusepath{clip}%
\pgfsetbuttcap%
\pgfsetroundjoin%
\definecolor{currentfill}{rgb}{0.121569,0.466667,0.705882}%
\pgfsetfillcolor{currentfill}%
\pgfsetlinewidth{1.003750pt}%
\definecolor{currentstroke}{rgb}{0.121569,0.466667,0.705882}%
\pgfsetstrokecolor{currentstroke}%
\pgfsetdash{}{0pt}%
\pgfpathmoveto{\pgfqpoint{2.243838in}{3.124394in}}%
\pgfpathcurveto{\pgfqpoint{2.254889in}{3.124394in}}{\pgfqpoint{2.265488in}{3.128784in}}{\pgfqpoint{2.273301in}{3.136598in}}%
\pgfpathcurveto{\pgfqpoint{2.281115in}{3.144411in}}{\pgfqpoint{2.285505in}{3.155010in}}{\pgfqpoint{2.285505in}{3.166060in}}%
\pgfpathcurveto{\pgfqpoint{2.285505in}{3.177111in}}{\pgfqpoint{2.281115in}{3.187710in}}{\pgfqpoint{2.273301in}{3.195523in}}%
\pgfpathcurveto{\pgfqpoint{2.265488in}{3.203337in}}{\pgfqpoint{2.254889in}{3.207727in}}{\pgfqpoint{2.243838in}{3.207727in}}%
\pgfpathcurveto{\pgfqpoint{2.232788in}{3.207727in}}{\pgfqpoint{2.222189in}{3.203337in}}{\pgfqpoint{2.214376in}{3.195523in}}%
\pgfpathcurveto{\pgfqpoint{2.206562in}{3.187710in}}{\pgfqpoint{2.202172in}{3.177111in}}{\pgfqpoint{2.202172in}{3.166060in}}%
\pgfpathcurveto{\pgfqpoint{2.202172in}{3.155010in}}{\pgfqpoint{2.206562in}{3.144411in}}{\pgfqpoint{2.214376in}{3.136598in}}%
\pgfpathcurveto{\pgfqpoint{2.222189in}{3.128784in}}{\pgfqpoint{2.232788in}{3.124394in}}{\pgfqpoint{2.243838in}{3.124394in}}%
\pgfpathclose%
\pgfusepath{stroke,fill}%
\end{pgfscope}%
\begin{pgfscope}%
\pgfpathrectangle{\pgfqpoint{0.648703in}{0.548769in}}{\pgfqpoint{5.201297in}{3.102590in}}%
\pgfusepath{clip}%
\pgfsetbuttcap%
\pgfsetroundjoin%
\definecolor{currentfill}{rgb}{1.000000,0.498039,0.054902}%
\pgfsetfillcolor{currentfill}%
\pgfsetlinewidth{1.003750pt}%
\definecolor{currentstroke}{rgb}{1.000000,0.498039,0.054902}%
\pgfsetstrokecolor{currentstroke}%
\pgfsetdash{}{0pt}%
\pgfpathmoveto{\pgfqpoint{1.349147in}{3.136837in}}%
\pgfpathcurveto{\pgfqpoint{1.360197in}{3.136837in}}{\pgfqpoint{1.370796in}{3.141228in}}{\pgfqpoint{1.378610in}{3.149041in}}%
\pgfpathcurveto{\pgfqpoint{1.386424in}{3.156855in}}{\pgfqpoint{1.390814in}{3.167454in}}{\pgfqpoint{1.390814in}{3.178504in}}%
\pgfpathcurveto{\pgfqpoint{1.390814in}{3.189554in}}{\pgfqpoint{1.386424in}{3.200153in}}{\pgfqpoint{1.378610in}{3.207967in}}%
\pgfpathcurveto{\pgfqpoint{1.370796in}{3.215780in}}{\pgfqpoint{1.360197in}{3.220171in}}{\pgfqpoint{1.349147in}{3.220171in}}%
\pgfpathcurveto{\pgfqpoint{1.338097in}{3.220171in}}{\pgfqpoint{1.327498in}{3.215780in}}{\pgfqpoint{1.319684in}{3.207967in}}%
\pgfpathcurveto{\pgfqpoint{1.311871in}{3.200153in}}{\pgfqpoint{1.307481in}{3.189554in}}{\pgfqpoint{1.307481in}{3.178504in}}%
\pgfpathcurveto{\pgfqpoint{1.307481in}{3.167454in}}{\pgfqpoint{1.311871in}{3.156855in}}{\pgfqpoint{1.319684in}{3.149041in}}%
\pgfpathcurveto{\pgfqpoint{1.327498in}{3.141228in}}{\pgfqpoint{1.338097in}{3.136837in}}{\pgfqpoint{1.349147in}{3.136837in}}%
\pgfpathclose%
\pgfusepath{stroke,fill}%
\end{pgfscope}%
\begin{pgfscope}%
\pgfpathrectangle{\pgfqpoint{0.648703in}{0.548769in}}{\pgfqpoint{5.201297in}{3.102590in}}%
\pgfusepath{clip}%
\pgfsetbuttcap%
\pgfsetroundjoin%
\definecolor{currentfill}{rgb}{1.000000,0.498039,0.054902}%
\pgfsetfillcolor{currentfill}%
\pgfsetlinewidth{1.003750pt}%
\definecolor{currentstroke}{rgb}{1.000000,0.498039,0.054902}%
\pgfsetstrokecolor{currentstroke}%
\pgfsetdash{}{0pt}%
\pgfpathmoveto{\pgfqpoint{1.816111in}{3.140985in}}%
\pgfpathcurveto{\pgfqpoint{1.827161in}{3.140985in}}{\pgfqpoint{1.837761in}{3.145375in}}{\pgfqpoint{1.845574in}{3.153189in}}%
\pgfpathcurveto{\pgfqpoint{1.853388in}{3.161003in}}{\pgfqpoint{1.857778in}{3.171602in}}{\pgfqpoint{1.857778in}{3.182652in}}%
\pgfpathcurveto{\pgfqpoint{1.857778in}{3.193702in}}{\pgfqpoint{1.853388in}{3.204301in}}{\pgfqpoint{1.845574in}{3.212115in}}%
\pgfpathcurveto{\pgfqpoint{1.837761in}{3.219928in}}{\pgfqpoint{1.827161in}{3.224319in}}{\pgfqpoint{1.816111in}{3.224319in}}%
\pgfpathcurveto{\pgfqpoint{1.805061in}{3.224319in}}{\pgfqpoint{1.794462in}{3.219928in}}{\pgfqpoint{1.786649in}{3.212115in}}%
\pgfpathcurveto{\pgfqpoint{1.778835in}{3.204301in}}{\pgfqpoint{1.774445in}{3.193702in}}{\pgfqpoint{1.774445in}{3.182652in}}%
\pgfpathcurveto{\pgfqpoint{1.774445in}{3.171602in}}{\pgfqpoint{1.778835in}{3.161003in}}{\pgfqpoint{1.786649in}{3.153189in}}%
\pgfpathcurveto{\pgfqpoint{1.794462in}{3.145375in}}{\pgfqpoint{1.805061in}{3.140985in}}{\pgfqpoint{1.816111in}{3.140985in}}%
\pgfpathclose%
\pgfusepath{stroke,fill}%
\end{pgfscope}%
\begin{pgfscope}%
\pgfpathrectangle{\pgfqpoint{0.648703in}{0.548769in}}{\pgfqpoint{5.201297in}{3.102590in}}%
\pgfusepath{clip}%
\pgfsetbuttcap%
\pgfsetroundjoin%
\definecolor{currentfill}{rgb}{1.000000,0.498039,0.054902}%
\pgfsetfillcolor{currentfill}%
\pgfsetlinewidth{1.003750pt}%
\definecolor{currentstroke}{rgb}{1.000000,0.498039,0.054902}%
\pgfsetstrokecolor{currentstroke}%
\pgfsetdash{}{0pt}%
\pgfpathmoveto{\pgfqpoint{1.037303in}{3.145133in}}%
\pgfpathcurveto{\pgfqpoint{1.048353in}{3.145133in}}{\pgfqpoint{1.058952in}{3.149523in}}{\pgfqpoint{1.066766in}{3.157337in}}%
\pgfpathcurveto{\pgfqpoint{1.074579in}{3.165151in}}{\pgfqpoint{1.078970in}{3.175750in}}{\pgfqpoint{1.078970in}{3.186800in}}%
\pgfpathcurveto{\pgfqpoint{1.078970in}{3.197850in}}{\pgfqpoint{1.074579in}{3.208449in}}{\pgfqpoint{1.066766in}{3.216262in}}%
\pgfpathcurveto{\pgfqpoint{1.058952in}{3.224076in}}{\pgfqpoint{1.048353in}{3.228466in}}{\pgfqpoint{1.037303in}{3.228466in}}%
\pgfpathcurveto{\pgfqpoint{1.026253in}{3.228466in}}{\pgfqpoint{1.015654in}{3.224076in}}{\pgfqpoint{1.007840in}{3.216262in}}%
\pgfpathcurveto{\pgfqpoint{1.000026in}{3.208449in}}{\pgfqpoint{0.995636in}{3.197850in}}{\pgfqpoint{0.995636in}{3.186800in}}%
\pgfpathcurveto{\pgfqpoint{0.995636in}{3.175750in}}{\pgfqpoint{1.000026in}{3.165151in}}{\pgfqpoint{1.007840in}{3.157337in}}%
\pgfpathcurveto{\pgfqpoint{1.015654in}{3.149523in}}{\pgfqpoint{1.026253in}{3.145133in}}{\pgfqpoint{1.037303in}{3.145133in}}%
\pgfpathclose%
\pgfusepath{stroke,fill}%
\end{pgfscope}%
\begin{pgfscope}%
\pgfpathrectangle{\pgfqpoint{0.648703in}{0.548769in}}{\pgfqpoint{5.201297in}{3.102590in}}%
\pgfusepath{clip}%
\pgfsetbuttcap%
\pgfsetroundjoin%
\definecolor{currentfill}{rgb}{0.121569,0.466667,0.705882}%
\pgfsetfillcolor{currentfill}%
\pgfsetlinewidth{1.003750pt}%
\definecolor{currentstroke}{rgb}{0.121569,0.466667,0.705882}%
\pgfsetstrokecolor{currentstroke}%
\pgfsetdash{}{0pt}%
\pgfpathmoveto{\pgfqpoint{1.409207in}{3.132690in}}%
\pgfpathcurveto{\pgfqpoint{1.420257in}{3.132690in}}{\pgfqpoint{1.430856in}{3.137080in}}{\pgfqpoint{1.438669in}{3.144893in}}%
\pgfpathcurveto{\pgfqpoint{1.446483in}{3.152707in}}{\pgfqpoint{1.450873in}{3.163306in}}{\pgfqpoint{1.450873in}{3.174356in}}%
\pgfpathcurveto{\pgfqpoint{1.450873in}{3.185406in}}{\pgfqpoint{1.446483in}{3.196005in}}{\pgfqpoint{1.438669in}{3.203819in}}%
\pgfpathcurveto{\pgfqpoint{1.430856in}{3.211633in}}{\pgfqpoint{1.420257in}{3.216023in}}{\pgfqpoint{1.409207in}{3.216023in}}%
\pgfpathcurveto{\pgfqpoint{1.398156in}{3.216023in}}{\pgfqpoint{1.387557in}{3.211633in}}{\pgfqpoint{1.379744in}{3.203819in}}%
\pgfpathcurveto{\pgfqpoint{1.371930in}{3.196005in}}{\pgfqpoint{1.367540in}{3.185406in}}{\pgfqpoint{1.367540in}{3.174356in}}%
\pgfpathcurveto{\pgfqpoint{1.367540in}{3.163306in}}{\pgfqpoint{1.371930in}{3.152707in}}{\pgfqpoint{1.379744in}{3.144893in}}%
\pgfpathcurveto{\pgfqpoint{1.387557in}{3.137080in}}{\pgfqpoint{1.398156in}{3.132690in}}{\pgfqpoint{1.409207in}{3.132690in}}%
\pgfpathclose%
\pgfusepath{stroke,fill}%
\end{pgfscope}%
\begin{pgfscope}%
\pgfpathrectangle{\pgfqpoint{0.648703in}{0.548769in}}{\pgfqpoint{5.201297in}{3.102590in}}%
\pgfusepath{clip}%
\pgfsetbuttcap%
\pgfsetroundjoin%
\definecolor{currentfill}{rgb}{0.121569,0.466667,0.705882}%
\pgfsetfillcolor{currentfill}%
\pgfsetlinewidth{1.003750pt}%
\definecolor{currentstroke}{rgb}{0.121569,0.466667,0.705882}%
\pgfsetstrokecolor{currentstroke}%
\pgfsetdash{}{0pt}%
\pgfpathmoveto{\pgfqpoint{1.406264in}{0.660572in}}%
\pgfpathcurveto{\pgfqpoint{1.417314in}{0.660572in}}{\pgfqpoint{1.427913in}{0.664963in}}{\pgfqpoint{1.435727in}{0.672776in}}%
\pgfpathcurveto{\pgfqpoint{1.443540in}{0.680590in}}{\pgfqpoint{1.447930in}{0.691189in}}{\pgfqpoint{1.447930in}{0.702239in}}%
\pgfpathcurveto{\pgfqpoint{1.447930in}{0.713289in}}{\pgfqpoint{1.443540in}{0.723888in}}{\pgfqpoint{1.435727in}{0.731702in}}%
\pgfpathcurveto{\pgfqpoint{1.427913in}{0.739516in}}{\pgfqpoint{1.417314in}{0.743906in}}{\pgfqpoint{1.406264in}{0.743906in}}%
\pgfpathcurveto{\pgfqpoint{1.395214in}{0.743906in}}{\pgfqpoint{1.384615in}{0.739516in}}{\pgfqpoint{1.376801in}{0.731702in}}%
\pgfpathcurveto{\pgfqpoint{1.368987in}{0.723888in}}{\pgfqpoint{1.364597in}{0.713289in}}{\pgfqpoint{1.364597in}{0.702239in}}%
\pgfpathcurveto{\pgfqpoint{1.364597in}{0.691189in}}{\pgfqpoint{1.368987in}{0.680590in}}{\pgfqpoint{1.376801in}{0.672776in}}%
\pgfpathcurveto{\pgfqpoint{1.384615in}{0.664963in}}{\pgfqpoint{1.395214in}{0.660572in}}{\pgfqpoint{1.406264in}{0.660572in}}%
\pgfpathclose%
\pgfusepath{stroke,fill}%
\end{pgfscope}%
\begin{pgfscope}%
\pgfpathrectangle{\pgfqpoint{0.648703in}{0.548769in}}{\pgfqpoint{5.201297in}{3.102590in}}%
\pgfusepath{clip}%
\pgfsetbuttcap%
\pgfsetroundjoin%
\definecolor{currentfill}{rgb}{1.000000,0.498039,0.054902}%
\pgfsetfillcolor{currentfill}%
\pgfsetlinewidth{1.003750pt}%
\definecolor{currentstroke}{rgb}{1.000000,0.498039,0.054902}%
\pgfsetstrokecolor{currentstroke}%
\pgfsetdash{}{0pt}%
\pgfpathmoveto{\pgfqpoint{1.994060in}{3.136837in}}%
\pgfpathcurveto{\pgfqpoint{2.005110in}{3.136837in}}{\pgfqpoint{2.015709in}{3.141228in}}{\pgfqpoint{2.023523in}{3.149041in}}%
\pgfpathcurveto{\pgfqpoint{2.031336in}{3.156855in}}{\pgfqpoint{2.035726in}{3.167454in}}{\pgfqpoint{2.035726in}{3.178504in}}%
\pgfpathcurveto{\pgfqpoint{2.035726in}{3.189554in}}{\pgfqpoint{2.031336in}{3.200153in}}{\pgfqpoint{2.023523in}{3.207967in}}%
\pgfpathcurveto{\pgfqpoint{2.015709in}{3.215780in}}{\pgfqpoint{2.005110in}{3.220171in}}{\pgfqpoint{1.994060in}{3.220171in}}%
\pgfpathcurveto{\pgfqpoint{1.983010in}{3.220171in}}{\pgfqpoint{1.972411in}{3.215780in}}{\pgfqpoint{1.964597in}{3.207967in}}%
\pgfpathcurveto{\pgfqpoint{1.956783in}{3.200153in}}{\pgfqpoint{1.952393in}{3.189554in}}{\pgfqpoint{1.952393in}{3.178504in}}%
\pgfpathcurveto{\pgfqpoint{1.952393in}{3.167454in}}{\pgfqpoint{1.956783in}{3.156855in}}{\pgfqpoint{1.964597in}{3.149041in}}%
\pgfpathcurveto{\pgfqpoint{1.972411in}{3.141228in}}{\pgfqpoint{1.983010in}{3.136837in}}{\pgfqpoint{1.994060in}{3.136837in}}%
\pgfpathclose%
\pgfusepath{stroke,fill}%
\end{pgfscope}%
\begin{pgfscope}%
\pgfpathrectangle{\pgfqpoint{0.648703in}{0.548769in}}{\pgfqpoint{5.201297in}{3.102590in}}%
\pgfusepath{clip}%
\pgfsetbuttcap%
\pgfsetroundjoin%
\definecolor{currentfill}{rgb}{1.000000,0.498039,0.054902}%
\pgfsetfillcolor{currentfill}%
\pgfsetlinewidth{1.003750pt}%
\definecolor{currentstroke}{rgb}{1.000000,0.498039,0.054902}%
\pgfsetstrokecolor{currentstroke}%
\pgfsetdash{}{0pt}%
\pgfpathmoveto{\pgfqpoint{1.627685in}{3.157577in}}%
\pgfpathcurveto{\pgfqpoint{1.638735in}{3.157577in}}{\pgfqpoint{1.649334in}{3.161967in}}{\pgfqpoint{1.657148in}{3.169780in}}%
\pgfpathcurveto{\pgfqpoint{1.664961in}{3.177594in}}{\pgfqpoint{1.669352in}{3.188193in}}{\pgfqpoint{1.669352in}{3.199243in}}%
\pgfpathcurveto{\pgfqpoint{1.669352in}{3.210293in}}{\pgfqpoint{1.664961in}{3.220892in}}{\pgfqpoint{1.657148in}{3.228706in}}%
\pgfpathcurveto{\pgfqpoint{1.649334in}{3.236520in}}{\pgfqpoint{1.638735in}{3.240910in}}{\pgfqpoint{1.627685in}{3.240910in}}%
\pgfpathcurveto{\pgfqpoint{1.616635in}{3.240910in}}{\pgfqpoint{1.606036in}{3.236520in}}{\pgfqpoint{1.598222in}{3.228706in}}%
\pgfpathcurveto{\pgfqpoint{1.590408in}{3.220892in}}{\pgfqpoint{1.586018in}{3.210293in}}{\pgfqpoint{1.586018in}{3.199243in}}%
\pgfpathcurveto{\pgfqpoint{1.586018in}{3.188193in}}{\pgfqpoint{1.590408in}{3.177594in}}{\pgfqpoint{1.598222in}{3.169780in}}%
\pgfpathcurveto{\pgfqpoint{1.606036in}{3.161967in}}{\pgfqpoint{1.616635in}{3.157577in}}{\pgfqpoint{1.627685in}{3.157577in}}%
\pgfpathclose%
\pgfusepath{stroke,fill}%
\end{pgfscope}%
\begin{pgfscope}%
\pgfpathrectangle{\pgfqpoint{0.648703in}{0.548769in}}{\pgfqpoint{5.201297in}{3.102590in}}%
\pgfusepath{clip}%
\pgfsetbuttcap%
\pgfsetroundjoin%
\definecolor{currentfill}{rgb}{0.121569,0.466667,0.705882}%
\pgfsetfillcolor{currentfill}%
\pgfsetlinewidth{1.003750pt}%
\definecolor{currentstroke}{rgb}{0.121569,0.466667,0.705882}%
\pgfsetstrokecolor{currentstroke}%
\pgfsetdash{}{0pt}%
\pgfpathmoveto{\pgfqpoint{1.312318in}{0.648129in}}%
\pgfpathcurveto{\pgfqpoint{1.323368in}{0.648129in}}{\pgfqpoint{1.333967in}{0.652519in}}{\pgfqpoint{1.341781in}{0.660333in}}%
\pgfpathcurveto{\pgfqpoint{1.349594in}{0.668146in}}{\pgfqpoint{1.353985in}{0.678745in}}{\pgfqpoint{1.353985in}{0.689796in}}%
\pgfpathcurveto{\pgfqpoint{1.353985in}{0.700846in}}{\pgfqpoint{1.349594in}{0.711445in}}{\pgfqpoint{1.341781in}{0.719258in}}%
\pgfpathcurveto{\pgfqpoint{1.333967in}{0.727072in}}{\pgfqpoint{1.323368in}{0.731462in}}{\pgfqpoint{1.312318in}{0.731462in}}%
\pgfpathcurveto{\pgfqpoint{1.301268in}{0.731462in}}{\pgfqpoint{1.290669in}{0.727072in}}{\pgfqpoint{1.282855in}{0.719258in}}%
\pgfpathcurveto{\pgfqpoint{1.275042in}{0.711445in}}{\pgfqpoint{1.270651in}{0.700846in}}{\pgfqpoint{1.270651in}{0.689796in}}%
\pgfpathcurveto{\pgfqpoint{1.270651in}{0.678745in}}{\pgfqpoint{1.275042in}{0.668146in}}{\pgfqpoint{1.282855in}{0.660333in}}%
\pgfpathcurveto{\pgfqpoint{1.290669in}{0.652519in}}{\pgfqpoint{1.301268in}{0.648129in}}{\pgfqpoint{1.312318in}{0.648129in}}%
\pgfpathclose%
\pgfusepath{stroke,fill}%
\end{pgfscope}%
\begin{pgfscope}%
\pgfpathrectangle{\pgfqpoint{0.648703in}{0.548769in}}{\pgfqpoint{5.201297in}{3.102590in}}%
\pgfusepath{clip}%
\pgfsetbuttcap%
\pgfsetroundjoin%
\definecolor{currentfill}{rgb}{0.121569,0.466667,0.705882}%
\pgfsetfillcolor{currentfill}%
\pgfsetlinewidth{1.003750pt}%
\definecolor{currentstroke}{rgb}{0.121569,0.466667,0.705882}%
\pgfsetstrokecolor{currentstroke}%
\pgfsetdash{}{0pt}%
\pgfpathmoveto{\pgfqpoint{0.946612in}{0.648129in}}%
\pgfpathcurveto{\pgfqpoint{0.957662in}{0.648129in}}{\pgfqpoint{0.968261in}{0.652519in}}{\pgfqpoint{0.976075in}{0.660333in}}%
\pgfpathcurveto{\pgfqpoint{0.983888in}{0.668146in}}{\pgfqpoint{0.988279in}{0.678745in}}{\pgfqpoint{0.988279in}{0.689796in}}%
\pgfpathcurveto{\pgfqpoint{0.988279in}{0.700846in}}{\pgfqpoint{0.983888in}{0.711445in}}{\pgfqpoint{0.976075in}{0.719258in}}%
\pgfpathcurveto{\pgfqpoint{0.968261in}{0.727072in}}{\pgfqpoint{0.957662in}{0.731462in}}{\pgfqpoint{0.946612in}{0.731462in}}%
\pgfpathcurveto{\pgfqpoint{0.935562in}{0.731462in}}{\pgfqpoint{0.924963in}{0.727072in}}{\pgfqpoint{0.917149in}{0.719258in}}%
\pgfpathcurveto{\pgfqpoint{0.909336in}{0.711445in}}{\pgfqpoint{0.904945in}{0.700846in}}{\pgfqpoint{0.904945in}{0.689796in}}%
\pgfpathcurveto{\pgfqpoint{0.904945in}{0.678745in}}{\pgfqpoint{0.909336in}{0.668146in}}{\pgfqpoint{0.917149in}{0.660333in}}%
\pgfpathcurveto{\pgfqpoint{0.924963in}{0.652519in}}{\pgfqpoint{0.935562in}{0.648129in}}{\pgfqpoint{0.946612in}{0.648129in}}%
\pgfpathclose%
\pgfusepath{stroke,fill}%
\end{pgfscope}%
\begin{pgfscope}%
\pgfpathrectangle{\pgfqpoint{0.648703in}{0.548769in}}{\pgfqpoint{5.201297in}{3.102590in}}%
\pgfusepath{clip}%
\pgfsetbuttcap%
\pgfsetroundjoin%
\definecolor{currentfill}{rgb}{0.121569,0.466667,0.705882}%
\pgfsetfillcolor{currentfill}%
\pgfsetlinewidth{1.003750pt}%
\definecolor{currentstroke}{rgb}{0.121569,0.466667,0.705882}%
\pgfsetstrokecolor{currentstroke}%
\pgfsetdash{}{0pt}%
\pgfpathmoveto{\pgfqpoint{1.731574in}{3.132690in}}%
\pgfpathcurveto{\pgfqpoint{1.742624in}{3.132690in}}{\pgfqpoint{1.753223in}{3.137080in}}{\pgfqpoint{1.761036in}{3.144893in}}%
\pgfpathcurveto{\pgfqpoint{1.768850in}{3.152707in}}{\pgfqpoint{1.773240in}{3.163306in}}{\pgfqpoint{1.773240in}{3.174356in}}%
\pgfpathcurveto{\pgfqpoint{1.773240in}{3.185406in}}{\pgfqpoint{1.768850in}{3.196005in}}{\pgfqpoint{1.761036in}{3.203819in}}%
\pgfpathcurveto{\pgfqpoint{1.753223in}{3.211633in}}{\pgfqpoint{1.742624in}{3.216023in}}{\pgfqpoint{1.731574in}{3.216023in}}%
\pgfpathcurveto{\pgfqpoint{1.720523in}{3.216023in}}{\pgfqpoint{1.709924in}{3.211633in}}{\pgfqpoint{1.702111in}{3.203819in}}%
\pgfpathcurveto{\pgfqpoint{1.694297in}{3.196005in}}{\pgfqpoint{1.689907in}{3.185406in}}{\pgfqpoint{1.689907in}{3.174356in}}%
\pgfpathcurveto{\pgfqpoint{1.689907in}{3.163306in}}{\pgfqpoint{1.694297in}{3.152707in}}{\pgfqpoint{1.702111in}{3.144893in}}%
\pgfpathcurveto{\pgfqpoint{1.709924in}{3.137080in}}{\pgfqpoint{1.720523in}{3.132690in}}{\pgfqpoint{1.731574in}{3.132690in}}%
\pgfpathclose%
\pgfusepath{stroke,fill}%
\end{pgfscope}%
\begin{pgfscope}%
\pgfpathrectangle{\pgfqpoint{0.648703in}{0.548769in}}{\pgfqpoint{5.201297in}{3.102590in}}%
\pgfusepath{clip}%
\pgfsetbuttcap%
\pgfsetroundjoin%
\definecolor{currentfill}{rgb}{1.000000,0.498039,0.054902}%
\pgfsetfillcolor{currentfill}%
\pgfsetlinewidth{1.003750pt}%
\definecolor{currentstroke}{rgb}{1.000000,0.498039,0.054902}%
\pgfsetstrokecolor{currentstroke}%
\pgfsetdash{}{0pt}%
\pgfpathmoveto{\pgfqpoint{2.470833in}{3.219794in}}%
\pgfpathcurveto{\pgfqpoint{2.481883in}{3.219794in}}{\pgfqpoint{2.492482in}{3.224185in}}{\pgfqpoint{2.500296in}{3.231998in}}%
\pgfpathcurveto{\pgfqpoint{2.508109in}{3.239812in}}{\pgfqpoint{2.512500in}{3.250411in}}{\pgfqpoint{2.512500in}{3.261461in}}%
\pgfpathcurveto{\pgfqpoint{2.512500in}{3.272511in}}{\pgfqpoint{2.508109in}{3.283110in}}{\pgfqpoint{2.500296in}{3.290924in}}%
\pgfpathcurveto{\pgfqpoint{2.492482in}{3.298737in}}{\pgfqpoint{2.481883in}{3.303128in}}{\pgfqpoint{2.470833in}{3.303128in}}%
\pgfpathcurveto{\pgfqpoint{2.459783in}{3.303128in}}{\pgfqpoint{2.449184in}{3.298737in}}{\pgfqpoint{2.441370in}{3.290924in}}%
\pgfpathcurveto{\pgfqpoint{2.433557in}{3.283110in}}{\pgfqpoint{2.429166in}{3.272511in}}{\pgfqpoint{2.429166in}{3.261461in}}%
\pgfpathcurveto{\pgfqpoint{2.429166in}{3.250411in}}{\pgfqpoint{2.433557in}{3.239812in}}{\pgfqpoint{2.441370in}{3.231998in}}%
\pgfpathcurveto{\pgfqpoint{2.449184in}{3.224185in}}{\pgfqpoint{2.459783in}{3.219794in}}{\pgfqpoint{2.470833in}{3.219794in}}%
\pgfpathclose%
\pgfusepath{stroke,fill}%
\end{pgfscope}%
\begin{pgfscope}%
\pgfpathrectangle{\pgfqpoint{0.648703in}{0.548769in}}{\pgfqpoint{5.201297in}{3.102590in}}%
\pgfusepath{clip}%
\pgfsetbuttcap%
\pgfsetroundjoin%
\definecolor{currentfill}{rgb}{1.000000,0.498039,0.054902}%
\pgfsetfillcolor{currentfill}%
\pgfsetlinewidth{1.003750pt}%
\definecolor{currentstroke}{rgb}{1.000000,0.498039,0.054902}%
\pgfsetstrokecolor{currentstroke}%
\pgfsetdash{}{0pt}%
\pgfpathmoveto{\pgfqpoint{1.635488in}{3.140985in}}%
\pgfpathcurveto{\pgfqpoint{1.646538in}{3.140985in}}{\pgfqpoint{1.657137in}{3.145375in}}{\pgfqpoint{1.664950in}{3.153189in}}%
\pgfpathcurveto{\pgfqpoint{1.672764in}{3.161003in}}{\pgfqpoint{1.677154in}{3.171602in}}{\pgfqpoint{1.677154in}{3.182652in}}%
\pgfpathcurveto{\pgfqpoint{1.677154in}{3.193702in}}{\pgfqpoint{1.672764in}{3.204301in}}{\pgfqpoint{1.664950in}{3.212115in}}%
\pgfpathcurveto{\pgfqpoint{1.657137in}{3.219928in}}{\pgfqpoint{1.646538in}{3.224319in}}{\pgfqpoint{1.635488in}{3.224319in}}%
\pgfpathcurveto{\pgfqpoint{1.624438in}{3.224319in}}{\pgfqpoint{1.613839in}{3.219928in}}{\pgfqpoint{1.606025in}{3.212115in}}%
\pgfpathcurveto{\pgfqpoint{1.598211in}{3.204301in}}{\pgfqpoint{1.593821in}{3.193702in}}{\pgfqpoint{1.593821in}{3.182652in}}%
\pgfpathcurveto{\pgfqpoint{1.593821in}{3.171602in}}{\pgfqpoint{1.598211in}{3.161003in}}{\pgfqpoint{1.606025in}{3.153189in}}%
\pgfpathcurveto{\pgfqpoint{1.613839in}{3.145375in}}{\pgfqpoint{1.624438in}{3.140985in}}{\pgfqpoint{1.635488in}{3.140985in}}%
\pgfpathclose%
\pgfusepath{stroke,fill}%
\end{pgfscope}%
\begin{pgfscope}%
\pgfpathrectangle{\pgfqpoint{0.648703in}{0.548769in}}{\pgfqpoint{5.201297in}{3.102590in}}%
\pgfusepath{clip}%
\pgfsetbuttcap%
\pgfsetroundjoin%
\definecolor{currentfill}{rgb}{1.000000,0.498039,0.054902}%
\pgfsetfillcolor{currentfill}%
\pgfsetlinewidth{1.003750pt}%
\definecolor{currentstroke}{rgb}{1.000000,0.498039,0.054902}%
\pgfsetstrokecolor{currentstroke}%
\pgfsetdash{}{0pt}%
\pgfpathmoveto{\pgfqpoint{1.771479in}{3.145133in}}%
\pgfpathcurveto{\pgfqpoint{1.782529in}{3.145133in}}{\pgfqpoint{1.793128in}{3.149523in}}{\pgfqpoint{1.800942in}{3.157337in}}%
\pgfpathcurveto{\pgfqpoint{1.808756in}{3.165151in}}{\pgfqpoint{1.813146in}{3.175750in}}{\pgfqpoint{1.813146in}{3.186800in}}%
\pgfpathcurveto{\pgfqpoint{1.813146in}{3.197850in}}{\pgfqpoint{1.808756in}{3.208449in}}{\pgfqpoint{1.800942in}{3.216262in}}%
\pgfpathcurveto{\pgfqpoint{1.793128in}{3.224076in}}{\pgfqpoint{1.782529in}{3.228466in}}{\pgfqpoint{1.771479in}{3.228466in}}%
\pgfpathcurveto{\pgfqpoint{1.760429in}{3.228466in}}{\pgfqpoint{1.749830in}{3.224076in}}{\pgfqpoint{1.742017in}{3.216262in}}%
\pgfpathcurveto{\pgfqpoint{1.734203in}{3.208449in}}{\pgfqpoint{1.729813in}{3.197850in}}{\pgfqpoint{1.729813in}{3.186800in}}%
\pgfpathcurveto{\pgfqpoint{1.729813in}{3.175750in}}{\pgfqpoint{1.734203in}{3.165151in}}{\pgfqpoint{1.742017in}{3.157337in}}%
\pgfpathcurveto{\pgfqpoint{1.749830in}{3.149523in}}{\pgfqpoint{1.760429in}{3.145133in}}{\pgfqpoint{1.771479in}{3.145133in}}%
\pgfpathclose%
\pgfusepath{stroke,fill}%
\end{pgfscope}%
\begin{pgfscope}%
\pgfpathrectangle{\pgfqpoint{0.648703in}{0.548769in}}{\pgfqpoint{5.201297in}{3.102590in}}%
\pgfusepath{clip}%
\pgfsetbuttcap%
\pgfsetroundjoin%
\definecolor{currentfill}{rgb}{1.000000,0.498039,0.054902}%
\pgfsetfillcolor{currentfill}%
\pgfsetlinewidth{1.003750pt}%
\definecolor{currentstroke}{rgb}{1.000000,0.498039,0.054902}%
\pgfsetstrokecolor{currentstroke}%
\pgfsetdash{}{0pt}%
\pgfpathmoveto{\pgfqpoint{1.512293in}{3.136837in}}%
\pgfpathcurveto{\pgfqpoint{1.523343in}{3.136837in}}{\pgfqpoint{1.533942in}{3.141228in}}{\pgfqpoint{1.541755in}{3.149041in}}%
\pgfpathcurveto{\pgfqpoint{1.549569in}{3.156855in}}{\pgfqpoint{1.553959in}{3.167454in}}{\pgfqpoint{1.553959in}{3.178504in}}%
\pgfpathcurveto{\pgfqpoint{1.553959in}{3.189554in}}{\pgfqpoint{1.549569in}{3.200153in}}{\pgfqpoint{1.541755in}{3.207967in}}%
\pgfpathcurveto{\pgfqpoint{1.533942in}{3.215780in}}{\pgfqpoint{1.523343in}{3.220171in}}{\pgfqpoint{1.512293in}{3.220171in}}%
\pgfpathcurveto{\pgfqpoint{1.501243in}{3.220171in}}{\pgfqpoint{1.490643in}{3.215780in}}{\pgfqpoint{1.482830in}{3.207967in}}%
\pgfpathcurveto{\pgfqpoint{1.475016in}{3.200153in}}{\pgfqpoint{1.470626in}{3.189554in}}{\pgfqpoint{1.470626in}{3.178504in}}%
\pgfpathcurveto{\pgfqpoint{1.470626in}{3.167454in}}{\pgfqpoint{1.475016in}{3.156855in}}{\pgfqpoint{1.482830in}{3.149041in}}%
\pgfpathcurveto{\pgfqpoint{1.490643in}{3.141228in}}{\pgfqpoint{1.501243in}{3.136837in}}{\pgfqpoint{1.512293in}{3.136837in}}%
\pgfpathclose%
\pgfusepath{stroke,fill}%
\end{pgfscope}%
\begin{pgfscope}%
\pgfpathrectangle{\pgfqpoint{0.648703in}{0.548769in}}{\pgfqpoint{5.201297in}{3.102590in}}%
\pgfusepath{clip}%
\pgfsetbuttcap%
\pgfsetroundjoin%
\definecolor{currentfill}{rgb}{0.121569,0.466667,0.705882}%
\pgfsetfillcolor{currentfill}%
\pgfsetlinewidth{1.003750pt}%
\definecolor{currentstroke}{rgb}{0.121569,0.466667,0.705882}%
\pgfsetstrokecolor{currentstroke}%
\pgfsetdash{}{0pt}%
\pgfpathmoveto{\pgfqpoint{0.997219in}{0.648129in}}%
\pgfpathcurveto{\pgfqpoint{1.008269in}{0.648129in}}{\pgfqpoint{1.018868in}{0.652519in}}{\pgfqpoint{1.026682in}{0.660333in}}%
\pgfpathcurveto{\pgfqpoint{1.034495in}{0.668146in}}{\pgfqpoint{1.038885in}{0.678745in}}{\pgfqpoint{1.038885in}{0.689796in}}%
\pgfpathcurveto{\pgfqpoint{1.038885in}{0.700846in}}{\pgfqpoint{1.034495in}{0.711445in}}{\pgfqpoint{1.026682in}{0.719258in}}%
\pgfpathcurveto{\pgfqpoint{1.018868in}{0.727072in}}{\pgfqpoint{1.008269in}{0.731462in}}{\pgfqpoint{0.997219in}{0.731462in}}%
\pgfpathcurveto{\pgfqpoint{0.986169in}{0.731462in}}{\pgfqpoint{0.975570in}{0.727072in}}{\pgfqpoint{0.967756in}{0.719258in}}%
\pgfpathcurveto{\pgfqpoint{0.959942in}{0.711445in}}{\pgfqpoint{0.955552in}{0.700846in}}{\pgfqpoint{0.955552in}{0.689796in}}%
\pgfpathcurveto{\pgfqpoint{0.955552in}{0.678745in}}{\pgfqpoint{0.959942in}{0.668146in}}{\pgfqpoint{0.967756in}{0.660333in}}%
\pgfpathcurveto{\pgfqpoint{0.975570in}{0.652519in}}{\pgfqpoint{0.986169in}{0.648129in}}{\pgfqpoint{0.997219in}{0.648129in}}%
\pgfpathclose%
\pgfusepath{stroke,fill}%
\end{pgfscope}%
\begin{pgfscope}%
\pgfpathrectangle{\pgfqpoint{0.648703in}{0.548769in}}{\pgfqpoint{5.201297in}{3.102590in}}%
\pgfusepath{clip}%
\pgfsetbuttcap%
\pgfsetroundjoin%
\definecolor{currentfill}{rgb}{0.121569,0.466667,0.705882}%
\pgfsetfillcolor{currentfill}%
\pgfsetlinewidth{1.003750pt}%
\definecolor{currentstroke}{rgb}{0.121569,0.466667,0.705882}%
\pgfsetstrokecolor{currentstroke}%
\pgfsetdash{}{0pt}%
\pgfpathmoveto{\pgfqpoint{1.614755in}{3.132690in}}%
\pgfpathcurveto{\pgfqpoint{1.625805in}{3.132690in}}{\pgfqpoint{1.636404in}{3.137080in}}{\pgfqpoint{1.644217in}{3.144893in}}%
\pgfpathcurveto{\pgfqpoint{1.652031in}{3.152707in}}{\pgfqpoint{1.656421in}{3.163306in}}{\pgfqpoint{1.656421in}{3.174356in}}%
\pgfpathcurveto{\pgfqpoint{1.656421in}{3.185406in}}{\pgfqpoint{1.652031in}{3.196005in}}{\pgfqpoint{1.644217in}{3.203819in}}%
\pgfpathcurveto{\pgfqpoint{1.636404in}{3.211633in}}{\pgfqpoint{1.625805in}{3.216023in}}{\pgfqpoint{1.614755in}{3.216023in}}%
\pgfpathcurveto{\pgfqpoint{1.603704in}{3.216023in}}{\pgfqpoint{1.593105in}{3.211633in}}{\pgfqpoint{1.585292in}{3.203819in}}%
\pgfpathcurveto{\pgfqpoint{1.577478in}{3.196005in}}{\pgfqpoint{1.573088in}{3.185406in}}{\pgfqpoint{1.573088in}{3.174356in}}%
\pgfpathcurveto{\pgfqpoint{1.573088in}{3.163306in}}{\pgfqpoint{1.577478in}{3.152707in}}{\pgfqpoint{1.585292in}{3.144893in}}%
\pgfpathcurveto{\pgfqpoint{1.593105in}{3.137080in}}{\pgfqpoint{1.603704in}{3.132690in}}{\pgfqpoint{1.614755in}{3.132690in}}%
\pgfpathclose%
\pgfusepath{stroke,fill}%
\end{pgfscope}%
\begin{pgfscope}%
\pgfpathrectangle{\pgfqpoint{0.648703in}{0.548769in}}{\pgfqpoint{5.201297in}{3.102590in}}%
\pgfusepath{clip}%
\pgfsetbuttcap%
\pgfsetroundjoin%
\definecolor{currentfill}{rgb}{1.000000,0.498039,0.054902}%
\pgfsetfillcolor{currentfill}%
\pgfsetlinewidth{1.003750pt}%
\definecolor{currentstroke}{rgb}{1.000000,0.498039,0.054902}%
\pgfsetstrokecolor{currentstroke}%
\pgfsetdash{}{0pt}%
\pgfpathmoveto{\pgfqpoint{1.609092in}{3.145133in}}%
\pgfpathcurveto{\pgfqpoint{1.620142in}{3.145133in}}{\pgfqpoint{1.630741in}{3.149523in}}{\pgfqpoint{1.638555in}{3.157337in}}%
\pgfpathcurveto{\pgfqpoint{1.646368in}{3.165151in}}{\pgfqpoint{1.650759in}{3.175750in}}{\pgfqpoint{1.650759in}{3.186800in}}%
\pgfpathcurveto{\pgfqpoint{1.650759in}{3.197850in}}{\pgfqpoint{1.646368in}{3.208449in}}{\pgfqpoint{1.638555in}{3.216262in}}%
\pgfpathcurveto{\pgfqpoint{1.630741in}{3.224076in}}{\pgfqpoint{1.620142in}{3.228466in}}{\pgfqpoint{1.609092in}{3.228466in}}%
\pgfpathcurveto{\pgfqpoint{1.598042in}{3.228466in}}{\pgfqpoint{1.587443in}{3.224076in}}{\pgfqpoint{1.579629in}{3.216262in}}%
\pgfpathcurveto{\pgfqpoint{1.571816in}{3.208449in}}{\pgfqpoint{1.567425in}{3.197850in}}{\pgfqpoint{1.567425in}{3.186800in}}%
\pgfpathcurveto{\pgfqpoint{1.567425in}{3.175750in}}{\pgfqpoint{1.571816in}{3.165151in}}{\pgfqpoint{1.579629in}{3.157337in}}%
\pgfpathcurveto{\pgfqpoint{1.587443in}{3.149523in}}{\pgfqpoint{1.598042in}{3.145133in}}{\pgfqpoint{1.609092in}{3.145133in}}%
\pgfpathclose%
\pgfusepath{stroke,fill}%
\end{pgfscope}%
\begin{pgfscope}%
\pgfpathrectangle{\pgfqpoint{0.648703in}{0.548769in}}{\pgfqpoint{5.201297in}{3.102590in}}%
\pgfusepath{clip}%
\pgfsetbuttcap%
\pgfsetroundjoin%
\definecolor{currentfill}{rgb}{0.121569,0.466667,0.705882}%
\pgfsetfillcolor{currentfill}%
\pgfsetlinewidth{1.003750pt}%
\definecolor{currentstroke}{rgb}{0.121569,0.466667,0.705882}%
\pgfsetstrokecolor{currentstroke}%
\pgfsetdash{}{0pt}%
\pgfpathmoveto{\pgfqpoint{1.460437in}{3.132690in}}%
\pgfpathcurveto{\pgfqpoint{1.471488in}{3.132690in}}{\pgfqpoint{1.482087in}{3.137080in}}{\pgfqpoint{1.489900in}{3.144893in}}%
\pgfpathcurveto{\pgfqpoint{1.497714in}{3.152707in}}{\pgfqpoint{1.502104in}{3.163306in}}{\pgfqpoint{1.502104in}{3.174356in}}%
\pgfpathcurveto{\pgfqpoint{1.502104in}{3.185406in}}{\pgfqpoint{1.497714in}{3.196005in}}{\pgfqpoint{1.489900in}{3.203819in}}%
\pgfpathcurveto{\pgfqpoint{1.482087in}{3.211633in}}{\pgfqpoint{1.471488in}{3.216023in}}{\pgfqpoint{1.460437in}{3.216023in}}%
\pgfpathcurveto{\pgfqpoint{1.449387in}{3.216023in}}{\pgfqpoint{1.438788in}{3.211633in}}{\pgfqpoint{1.430975in}{3.203819in}}%
\pgfpathcurveto{\pgfqpoint{1.423161in}{3.196005in}}{\pgfqpoint{1.418771in}{3.185406in}}{\pgfqpoint{1.418771in}{3.174356in}}%
\pgfpathcurveto{\pgfqpoint{1.418771in}{3.163306in}}{\pgfqpoint{1.423161in}{3.152707in}}{\pgfqpoint{1.430975in}{3.144893in}}%
\pgfpathcurveto{\pgfqpoint{1.438788in}{3.137080in}}{\pgfqpoint{1.449387in}{3.132690in}}{\pgfqpoint{1.460437in}{3.132690in}}%
\pgfpathclose%
\pgfusepath{stroke,fill}%
\end{pgfscope}%
\begin{pgfscope}%
\pgfpathrectangle{\pgfqpoint{0.648703in}{0.548769in}}{\pgfqpoint{5.201297in}{3.102590in}}%
\pgfusepath{clip}%
\pgfsetbuttcap%
\pgfsetroundjoin%
\definecolor{currentfill}{rgb}{1.000000,0.498039,0.054902}%
\pgfsetfillcolor{currentfill}%
\pgfsetlinewidth{1.003750pt}%
\definecolor{currentstroke}{rgb}{1.000000,0.498039,0.054902}%
\pgfsetstrokecolor{currentstroke}%
\pgfsetdash{}{0pt}%
\pgfpathmoveto{\pgfqpoint{1.456960in}{3.157577in}}%
\pgfpathcurveto{\pgfqpoint{1.468010in}{3.157577in}}{\pgfqpoint{1.478609in}{3.161967in}}{\pgfqpoint{1.486422in}{3.169780in}}%
\pgfpathcurveto{\pgfqpoint{1.494236in}{3.177594in}}{\pgfqpoint{1.498626in}{3.188193in}}{\pgfqpoint{1.498626in}{3.199243in}}%
\pgfpathcurveto{\pgfqpoint{1.498626in}{3.210293in}}{\pgfqpoint{1.494236in}{3.220892in}}{\pgfqpoint{1.486422in}{3.228706in}}%
\pgfpathcurveto{\pgfqpoint{1.478609in}{3.236520in}}{\pgfqpoint{1.468010in}{3.240910in}}{\pgfqpoint{1.456960in}{3.240910in}}%
\pgfpathcurveto{\pgfqpoint{1.445910in}{3.240910in}}{\pgfqpoint{1.435311in}{3.236520in}}{\pgfqpoint{1.427497in}{3.228706in}}%
\pgfpathcurveto{\pgfqpoint{1.419683in}{3.220892in}}{\pgfqpoint{1.415293in}{3.210293in}}{\pgfqpoint{1.415293in}{3.199243in}}%
\pgfpathcurveto{\pgfqpoint{1.415293in}{3.188193in}}{\pgfqpoint{1.419683in}{3.177594in}}{\pgfqpoint{1.427497in}{3.169780in}}%
\pgfpathcurveto{\pgfqpoint{1.435311in}{3.161967in}}{\pgfqpoint{1.445910in}{3.157577in}}{\pgfqpoint{1.456960in}{3.157577in}}%
\pgfpathclose%
\pgfusepath{stroke,fill}%
\end{pgfscope}%
\begin{pgfscope}%
\pgfpathrectangle{\pgfqpoint{0.648703in}{0.548769in}}{\pgfqpoint{5.201297in}{3.102590in}}%
\pgfusepath{clip}%
\pgfsetbuttcap%
\pgfsetroundjoin%
\definecolor{currentfill}{rgb}{1.000000,0.498039,0.054902}%
\pgfsetfillcolor{currentfill}%
\pgfsetlinewidth{1.003750pt}%
\definecolor{currentstroke}{rgb}{1.000000,0.498039,0.054902}%
\pgfsetstrokecolor{currentstroke}%
\pgfsetdash{}{0pt}%
\pgfpathmoveto{\pgfqpoint{1.313121in}{3.145133in}}%
\pgfpathcurveto{\pgfqpoint{1.324171in}{3.145133in}}{\pgfqpoint{1.334770in}{3.149523in}}{\pgfqpoint{1.342583in}{3.157337in}}%
\pgfpathcurveto{\pgfqpoint{1.350397in}{3.165151in}}{\pgfqpoint{1.354787in}{3.175750in}}{\pgfqpoint{1.354787in}{3.186800in}}%
\pgfpathcurveto{\pgfqpoint{1.354787in}{3.197850in}}{\pgfqpoint{1.350397in}{3.208449in}}{\pgfqpoint{1.342583in}{3.216262in}}%
\pgfpathcurveto{\pgfqpoint{1.334770in}{3.224076in}}{\pgfqpoint{1.324171in}{3.228466in}}{\pgfqpoint{1.313121in}{3.228466in}}%
\pgfpathcurveto{\pgfqpoint{1.302071in}{3.228466in}}{\pgfqpoint{1.291471in}{3.224076in}}{\pgfqpoint{1.283658in}{3.216262in}}%
\pgfpathcurveto{\pgfqpoint{1.275844in}{3.208449in}}{\pgfqpoint{1.271454in}{3.197850in}}{\pgfqpoint{1.271454in}{3.186800in}}%
\pgfpathcurveto{\pgfqpoint{1.271454in}{3.175750in}}{\pgfqpoint{1.275844in}{3.165151in}}{\pgfqpoint{1.283658in}{3.157337in}}%
\pgfpathcurveto{\pgfqpoint{1.291471in}{3.149523in}}{\pgfqpoint{1.302071in}{3.145133in}}{\pgfqpoint{1.313121in}{3.145133in}}%
\pgfpathclose%
\pgfusepath{stroke,fill}%
\end{pgfscope}%
\begin{pgfscope}%
\pgfpathrectangle{\pgfqpoint{0.648703in}{0.548769in}}{\pgfqpoint{5.201297in}{3.102590in}}%
\pgfusepath{clip}%
\pgfsetbuttcap%
\pgfsetroundjoin%
\definecolor{currentfill}{rgb}{1.000000,0.498039,0.054902}%
\pgfsetfillcolor{currentfill}%
\pgfsetlinewidth{1.003750pt}%
\definecolor{currentstroke}{rgb}{1.000000,0.498039,0.054902}%
\pgfsetstrokecolor{currentstroke}%
\pgfsetdash{}{0pt}%
\pgfpathmoveto{\pgfqpoint{1.744771in}{3.190759in}}%
\pgfpathcurveto{\pgfqpoint{1.755822in}{3.190759in}}{\pgfqpoint{1.766421in}{3.195150in}}{\pgfqpoint{1.774234in}{3.202963in}}%
\pgfpathcurveto{\pgfqpoint{1.782048in}{3.210777in}}{\pgfqpoint{1.786438in}{3.221376in}}{\pgfqpoint{1.786438in}{3.232426in}}%
\pgfpathcurveto{\pgfqpoint{1.786438in}{3.243476in}}{\pgfqpoint{1.782048in}{3.254075in}}{\pgfqpoint{1.774234in}{3.261889in}}%
\pgfpathcurveto{\pgfqpoint{1.766421in}{3.269702in}}{\pgfqpoint{1.755822in}{3.274093in}}{\pgfqpoint{1.744771in}{3.274093in}}%
\pgfpathcurveto{\pgfqpoint{1.733721in}{3.274093in}}{\pgfqpoint{1.723122in}{3.269702in}}{\pgfqpoint{1.715309in}{3.261889in}}%
\pgfpathcurveto{\pgfqpoint{1.707495in}{3.254075in}}{\pgfqpoint{1.703105in}{3.243476in}}{\pgfqpoint{1.703105in}{3.232426in}}%
\pgfpathcurveto{\pgfqpoint{1.703105in}{3.221376in}}{\pgfqpoint{1.707495in}{3.210777in}}{\pgfqpoint{1.715309in}{3.202963in}}%
\pgfpathcurveto{\pgfqpoint{1.723122in}{3.195150in}}{\pgfqpoint{1.733721in}{3.190759in}}{\pgfqpoint{1.744771in}{3.190759in}}%
\pgfpathclose%
\pgfusepath{stroke,fill}%
\end{pgfscope}%
\begin{pgfscope}%
\pgfpathrectangle{\pgfqpoint{0.648703in}{0.548769in}}{\pgfqpoint{5.201297in}{3.102590in}}%
\pgfusepath{clip}%
\pgfsetbuttcap%
\pgfsetroundjoin%
\definecolor{currentfill}{rgb}{1.000000,0.498039,0.054902}%
\pgfsetfillcolor{currentfill}%
\pgfsetlinewidth{1.003750pt}%
\definecolor{currentstroke}{rgb}{1.000000,0.498039,0.054902}%
\pgfsetstrokecolor{currentstroke}%
\pgfsetdash{}{0pt}%
\pgfpathmoveto{\pgfqpoint{1.621978in}{3.207351in}}%
\pgfpathcurveto{\pgfqpoint{1.633028in}{3.207351in}}{\pgfqpoint{1.643627in}{3.211741in}}{\pgfqpoint{1.651440in}{3.219555in}}%
\pgfpathcurveto{\pgfqpoint{1.659254in}{3.227368in}}{\pgfqpoint{1.663644in}{3.237967in}}{\pgfqpoint{1.663644in}{3.249017in}}%
\pgfpathcurveto{\pgfqpoint{1.663644in}{3.260068in}}{\pgfqpoint{1.659254in}{3.270667in}}{\pgfqpoint{1.651440in}{3.278480in}}%
\pgfpathcurveto{\pgfqpoint{1.643627in}{3.286294in}}{\pgfqpoint{1.633028in}{3.290684in}}{\pgfqpoint{1.621978in}{3.290684in}}%
\pgfpathcurveto{\pgfqpoint{1.610928in}{3.290684in}}{\pgfqpoint{1.600329in}{3.286294in}}{\pgfqpoint{1.592515in}{3.278480in}}%
\pgfpathcurveto{\pgfqpoint{1.584701in}{3.270667in}}{\pgfqpoint{1.580311in}{3.260068in}}{\pgfqpoint{1.580311in}{3.249017in}}%
\pgfpathcurveto{\pgfqpoint{1.580311in}{3.237967in}}{\pgfqpoint{1.584701in}{3.227368in}}{\pgfqpoint{1.592515in}{3.219555in}}%
\pgfpathcurveto{\pgfqpoint{1.600329in}{3.211741in}}{\pgfqpoint{1.610928in}{3.207351in}}{\pgfqpoint{1.621978in}{3.207351in}}%
\pgfpathclose%
\pgfusepath{stroke,fill}%
\end{pgfscope}%
\begin{pgfscope}%
\pgfpathrectangle{\pgfqpoint{0.648703in}{0.548769in}}{\pgfqpoint{5.201297in}{3.102590in}}%
\pgfusepath{clip}%
\pgfsetbuttcap%
\pgfsetroundjoin%
\definecolor{currentfill}{rgb}{0.121569,0.466667,0.705882}%
\pgfsetfillcolor{currentfill}%
\pgfsetlinewidth{1.003750pt}%
\definecolor{currentstroke}{rgb}{0.121569,0.466667,0.705882}%
\pgfsetstrokecolor{currentstroke}%
\pgfsetdash{}{0pt}%
\pgfpathmoveto{\pgfqpoint{0.946032in}{0.648129in}}%
\pgfpathcurveto{\pgfqpoint{0.957083in}{0.648129in}}{\pgfqpoint{0.967682in}{0.652519in}}{\pgfqpoint{0.975495in}{0.660333in}}%
\pgfpathcurveto{\pgfqpoint{0.983309in}{0.668146in}}{\pgfqpoint{0.987699in}{0.678745in}}{\pgfqpoint{0.987699in}{0.689796in}}%
\pgfpathcurveto{\pgfqpoint{0.987699in}{0.700846in}}{\pgfqpoint{0.983309in}{0.711445in}}{\pgfqpoint{0.975495in}{0.719258in}}%
\pgfpathcurveto{\pgfqpoint{0.967682in}{0.727072in}}{\pgfqpoint{0.957083in}{0.731462in}}{\pgfqpoint{0.946032in}{0.731462in}}%
\pgfpathcurveto{\pgfqpoint{0.934982in}{0.731462in}}{\pgfqpoint{0.924383in}{0.727072in}}{\pgfqpoint{0.916570in}{0.719258in}}%
\pgfpathcurveto{\pgfqpoint{0.908756in}{0.711445in}}{\pgfqpoint{0.904366in}{0.700846in}}{\pgfqpoint{0.904366in}{0.689796in}}%
\pgfpathcurveto{\pgfqpoint{0.904366in}{0.678745in}}{\pgfqpoint{0.908756in}{0.668146in}}{\pgfqpoint{0.916570in}{0.660333in}}%
\pgfpathcurveto{\pgfqpoint{0.924383in}{0.652519in}}{\pgfqpoint{0.934982in}{0.648129in}}{\pgfqpoint{0.946032in}{0.648129in}}%
\pgfpathclose%
\pgfusepath{stroke,fill}%
\end{pgfscope}%
\begin{pgfscope}%
\pgfpathrectangle{\pgfqpoint{0.648703in}{0.548769in}}{\pgfqpoint{5.201297in}{3.102590in}}%
\pgfusepath{clip}%
\pgfsetbuttcap%
\pgfsetroundjoin%
\definecolor{currentfill}{rgb}{0.121569,0.466667,0.705882}%
\pgfsetfillcolor{currentfill}%
\pgfsetlinewidth{1.003750pt}%
\definecolor{currentstroke}{rgb}{0.121569,0.466667,0.705882}%
\pgfsetstrokecolor{currentstroke}%
\pgfsetdash{}{0pt}%
\pgfpathmoveto{\pgfqpoint{0.885126in}{0.656425in}}%
\pgfpathcurveto{\pgfqpoint{0.896176in}{0.656425in}}{\pgfqpoint{0.906775in}{0.660815in}}{\pgfqpoint{0.914589in}{0.668629in}}%
\pgfpathcurveto{\pgfqpoint{0.922402in}{0.676442in}}{\pgfqpoint{0.926793in}{0.687041in}}{\pgfqpoint{0.926793in}{0.698091in}}%
\pgfpathcurveto{\pgfqpoint{0.926793in}{0.709141in}}{\pgfqpoint{0.922402in}{0.719740in}}{\pgfqpoint{0.914589in}{0.727554in}}%
\pgfpathcurveto{\pgfqpoint{0.906775in}{0.735368in}}{\pgfqpoint{0.896176in}{0.739758in}}{\pgfqpoint{0.885126in}{0.739758in}}%
\pgfpathcurveto{\pgfqpoint{0.874076in}{0.739758in}}{\pgfqpoint{0.863477in}{0.735368in}}{\pgfqpoint{0.855663in}{0.727554in}}%
\pgfpathcurveto{\pgfqpoint{0.847850in}{0.719740in}}{\pgfqpoint{0.843459in}{0.709141in}}{\pgfqpoint{0.843459in}{0.698091in}}%
\pgfpathcurveto{\pgfqpoint{0.843459in}{0.687041in}}{\pgfqpoint{0.847850in}{0.676442in}}{\pgfqpoint{0.855663in}{0.668629in}}%
\pgfpathcurveto{\pgfqpoint{0.863477in}{0.660815in}}{\pgfqpoint{0.874076in}{0.656425in}}{\pgfqpoint{0.885126in}{0.656425in}}%
\pgfpathclose%
\pgfusepath{stroke,fill}%
\end{pgfscope}%
\begin{pgfscope}%
\pgfpathrectangle{\pgfqpoint{0.648703in}{0.548769in}}{\pgfqpoint{5.201297in}{3.102590in}}%
\pgfusepath{clip}%
\pgfsetbuttcap%
\pgfsetroundjoin%
\definecolor{currentfill}{rgb}{0.121569,0.466667,0.705882}%
\pgfsetfillcolor{currentfill}%
\pgfsetlinewidth{1.003750pt}%
\definecolor{currentstroke}{rgb}{0.121569,0.466667,0.705882}%
\pgfsetstrokecolor{currentstroke}%
\pgfsetdash{}{0pt}%
\pgfpathmoveto{\pgfqpoint{1.957632in}{3.099507in}}%
\pgfpathcurveto{\pgfqpoint{1.968682in}{3.099507in}}{\pgfqpoint{1.979281in}{3.103897in}}{\pgfqpoint{1.987095in}{3.111711in}}%
\pgfpathcurveto{\pgfqpoint{1.994908in}{3.119524in}}{\pgfqpoint{1.999298in}{3.130123in}}{\pgfqpoint{1.999298in}{3.141173in}}%
\pgfpathcurveto{\pgfqpoint{1.999298in}{3.152224in}}{\pgfqpoint{1.994908in}{3.162823in}}{\pgfqpoint{1.987095in}{3.170636in}}%
\pgfpathcurveto{\pgfqpoint{1.979281in}{3.178450in}}{\pgfqpoint{1.968682in}{3.182840in}}{\pgfqpoint{1.957632in}{3.182840in}}%
\pgfpathcurveto{\pgfqpoint{1.946582in}{3.182840in}}{\pgfqpoint{1.935983in}{3.178450in}}{\pgfqpoint{1.928169in}{3.170636in}}%
\pgfpathcurveto{\pgfqpoint{1.920355in}{3.162823in}}{\pgfqpoint{1.915965in}{3.152224in}}{\pgfqpoint{1.915965in}{3.141173in}}%
\pgfpathcurveto{\pgfqpoint{1.915965in}{3.130123in}}{\pgfqpoint{1.920355in}{3.119524in}}{\pgfqpoint{1.928169in}{3.111711in}}%
\pgfpathcurveto{\pgfqpoint{1.935983in}{3.103897in}}{\pgfqpoint{1.946582in}{3.099507in}}{\pgfqpoint{1.957632in}{3.099507in}}%
\pgfpathclose%
\pgfusepath{stroke,fill}%
\end{pgfscope}%
\begin{pgfscope}%
\pgfpathrectangle{\pgfqpoint{0.648703in}{0.548769in}}{\pgfqpoint{5.201297in}{3.102590in}}%
\pgfusepath{clip}%
\pgfsetbuttcap%
\pgfsetroundjoin%
\definecolor{currentfill}{rgb}{1.000000,0.498039,0.054902}%
\pgfsetfillcolor{currentfill}%
\pgfsetlinewidth{1.003750pt}%
\definecolor{currentstroke}{rgb}{1.000000,0.498039,0.054902}%
\pgfsetstrokecolor{currentstroke}%
\pgfsetdash{}{0pt}%
\pgfpathmoveto{\pgfqpoint{1.575919in}{3.136837in}}%
\pgfpathcurveto{\pgfqpoint{1.586969in}{3.136837in}}{\pgfqpoint{1.597568in}{3.141228in}}{\pgfqpoint{1.605382in}{3.149041in}}%
\pgfpathcurveto{\pgfqpoint{1.613195in}{3.156855in}}{\pgfqpoint{1.617586in}{3.167454in}}{\pgfqpoint{1.617586in}{3.178504in}}%
\pgfpathcurveto{\pgfqpoint{1.617586in}{3.189554in}}{\pgfqpoint{1.613195in}{3.200153in}}{\pgfqpoint{1.605382in}{3.207967in}}%
\pgfpathcurveto{\pgfqpoint{1.597568in}{3.215780in}}{\pgfqpoint{1.586969in}{3.220171in}}{\pgfqpoint{1.575919in}{3.220171in}}%
\pgfpathcurveto{\pgfqpoint{1.564869in}{3.220171in}}{\pgfqpoint{1.554270in}{3.215780in}}{\pgfqpoint{1.546456in}{3.207967in}}%
\pgfpathcurveto{\pgfqpoint{1.538643in}{3.200153in}}{\pgfqpoint{1.534252in}{3.189554in}}{\pgfqpoint{1.534252in}{3.178504in}}%
\pgfpathcurveto{\pgfqpoint{1.534252in}{3.167454in}}{\pgfqpoint{1.538643in}{3.156855in}}{\pgfqpoint{1.546456in}{3.149041in}}%
\pgfpathcurveto{\pgfqpoint{1.554270in}{3.141228in}}{\pgfqpoint{1.564869in}{3.136837in}}{\pgfqpoint{1.575919in}{3.136837in}}%
\pgfpathclose%
\pgfusepath{stroke,fill}%
\end{pgfscope}%
\begin{pgfscope}%
\pgfpathrectangle{\pgfqpoint{0.648703in}{0.548769in}}{\pgfqpoint{5.201297in}{3.102590in}}%
\pgfusepath{clip}%
\pgfsetbuttcap%
\pgfsetroundjoin%
\definecolor{currentfill}{rgb}{0.121569,0.466667,0.705882}%
\pgfsetfillcolor{currentfill}%
\pgfsetlinewidth{1.003750pt}%
\definecolor{currentstroke}{rgb}{0.121569,0.466667,0.705882}%
\pgfsetstrokecolor{currentstroke}%
\pgfsetdash{}{0pt}%
\pgfpathmoveto{\pgfqpoint{1.160899in}{0.648129in}}%
\pgfpathcurveto{\pgfqpoint{1.171949in}{0.648129in}}{\pgfqpoint{1.182548in}{0.652519in}}{\pgfqpoint{1.190362in}{0.660333in}}%
\pgfpathcurveto{\pgfqpoint{1.198176in}{0.668146in}}{\pgfqpoint{1.202566in}{0.678745in}}{\pgfqpoint{1.202566in}{0.689796in}}%
\pgfpathcurveto{\pgfqpoint{1.202566in}{0.700846in}}{\pgfqpoint{1.198176in}{0.711445in}}{\pgfqpoint{1.190362in}{0.719258in}}%
\pgfpathcurveto{\pgfqpoint{1.182548in}{0.727072in}}{\pgfqpoint{1.171949in}{0.731462in}}{\pgfqpoint{1.160899in}{0.731462in}}%
\pgfpathcurveto{\pgfqpoint{1.149849in}{0.731462in}}{\pgfqpoint{1.139250in}{0.727072in}}{\pgfqpoint{1.131436in}{0.719258in}}%
\pgfpathcurveto{\pgfqpoint{1.123623in}{0.711445in}}{\pgfqpoint{1.119233in}{0.700846in}}{\pgfqpoint{1.119233in}{0.689796in}}%
\pgfpathcurveto{\pgfqpoint{1.119233in}{0.678745in}}{\pgfqpoint{1.123623in}{0.668146in}}{\pgfqpoint{1.131436in}{0.660333in}}%
\pgfpathcurveto{\pgfqpoint{1.139250in}{0.652519in}}{\pgfqpoint{1.149849in}{0.648129in}}{\pgfqpoint{1.160899in}{0.648129in}}%
\pgfpathclose%
\pgfusepath{stroke,fill}%
\end{pgfscope}%
\begin{pgfscope}%
\pgfpathrectangle{\pgfqpoint{0.648703in}{0.548769in}}{\pgfqpoint{5.201297in}{3.102590in}}%
\pgfusepath{clip}%
\pgfsetbuttcap%
\pgfsetroundjoin%
\definecolor{currentfill}{rgb}{0.121569,0.466667,0.705882}%
\pgfsetfillcolor{currentfill}%
\pgfsetlinewidth{1.003750pt}%
\definecolor{currentstroke}{rgb}{0.121569,0.466667,0.705882}%
\pgfsetstrokecolor{currentstroke}%
\pgfsetdash{}{0pt}%
\pgfpathmoveto{\pgfqpoint{0.900642in}{0.648129in}}%
\pgfpathcurveto{\pgfqpoint{0.911693in}{0.648129in}}{\pgfqpoint{0.922292in}{0.652519in}}{\pgfqpoint{0.930105in}{0.660333in}}%
\pgfpathcurveto{\pgfqpoint{0.937919in}{0.668146in}}{\pgfqpoint{0.942309in}{0.678745in}}{\pgfqpoint{0.942309in}{0.689796in}}%
\pgfpathcurveto{\pgfqpoint{0.942309in}{0.700846in}}{\pgfqpoint{0.937919in}{0.711445in}}{\pgfqpoint{0.930105in}{0.719258in}}%
\pgfpathcurveto{\pgfqpoint{0.922292in}{0.727072in}}{\pgfqpoint{0.911693in}{0.731462in}}{\pgfqpoint{0.900642in}{0.731462in}}%
\pgfpathcurveto{\pgfqpoint{0.889592in}{0.731462in}}{\pgfqpoint{0.878993in}{0.727072in}}{\pgfqpoint{0.871180in}{0.719258in}}%
\pgfpathcurveto{\pgfqpoint{0.863366in}{0.711445in}}{\pgfqpoint{0.858976in}{0.700846in}}{\pgfqpoint{0.858976in}{0.689796in}}%
\pgfpathcurveto{\pgfqpoint{0.858976in}{0.678745in}}{\pgfqpoint{0.863366in}{0.668146in}}{\pgfqpoint{0.871180in}{0.660333in}}%
\pgfpathcurveto{\pgfqpoint{0.878993in}{0.652519in}}{\pgfqpoint{0.889592in}{0.648129in}}{\pgfqpoint{0.900642in}{0.648129in}}%
\pgfpathclose%
\pgfusepath{stroke,fill}%
\end{pgfscope}%
\begin{pgfscope}%
\pgfpathrectangle{\pgfqpoint{0.648703in}{0.548769in}}{\pgfqpoint{5.201297in}{3.102590in}}%
\pgfusepath{clip}%
\pgfsetbuttcap%
\pgfsetroundjoin%
\definecolor{currentfill}{rgb}{0.839216,0.152941,0.156863}%
\pgfsetfillcolor{currentfill}%
\pgfsetlinewidth{1.003750pt}%
\definecolor{currentstroke}{rgb}{0.839216,0.152941,0.156863}%
\pgfsetstrokecolor{currentstroke}%
\pgfsetdash{}{0pt}%
\pgfpathmoveto{\pgfqpoint{1.195009in}{3.410595in}}%
\pgfpathcurveto{\pgfqpoint{1.206059in}{3.410595in}}{\pgfqpoint{1.216658in}{3.414986in}}{\pgfqpoint{1.224471in}{3.422799in}}%
\pgfpathcurveto{\pgfqpoint{1.232285in}{3.430613in}}{\pgfqpoint{1.236675in}{3.441212in}}{\pgfqpoint{1.236675in}{3.452262in}}%
\pgfpathcurveto{\pgfqpoint{1.236675in}{3.463312in}}{\pgfqpoint{1.232285in}{3.473911in}}{\pgfqpoint{1.224471in}{3.481725in}}%
\pgfpathcurveto{\pgfqpoint{1.216658in}{3.489538in}}{\pgfqpoint{1.206059in}{3.493929in}}{\pgfqpoint{1.195009in}{3.493929in}}%
\pgfpathcurveto{\pgfqpoint{1.183958in}{3.493929in}}{\pgfqpoint{1.173359in}{3.489538in}}{\pgfqpoint{1.165546in}{3.481725in}}%
\pgfpathcurveto{\pgfqpoint{1.157732in}{3.473911in}}{\pgfqpoint{1.153342in}{3.463312in}}{\pgfqpoint{1.153342in}{3.452262in}}%
\pgfpathcurveto{\pgfqpoint{1.153342in}{3.441212in}}{\pgfqpoint{1.157732in}{3.430613in}}{\pgfqpoint{1.165546in}{3.422799in}}%
\pgfpathcurveto{\pgfqpoint{1.173359in}{3.414986in}}{\pgfqpoint{1.183958in}{3.410595in}}{\pgfqpoint{1.195009in}{3.410595in}}%
\pgfpathclose%
\pgfusepath{stroke,fill}%
\end{pgfscope}%
\begin{pgfscope}%
\pgfpathrectangle{\pgfqpoint{0.648703in}{0.548769in}}{\pgfqpoint{5.201297in}{3.102590in}}%
\pgfusepath{clip}%
\pgfsetbuttcap%
\pgfsetroundjoin%
\definecolor{currentfill}{rgb}{0.121569,0.466667,0.705882}%
\pgfsetfillcolor{currentfill}%
\pgfsetlinewidth{1.003750pt}%
\definecolor{currentstroke}{rgb}{0.121569,0.466667,0.705882}%
\pgfsetstrokecolor{currentstroke}%
\pgfsetdash{}{0pt}%
\pgfpathmoveto{\pgfqpoint{1.885579in}{3.128542in}}%
\pgfpathcurveto{\pgfqpoint{1.896629in}{3.128542in}}{\pgfqpoint{1.907228in}{3.132932in}}{\pgfqpoint{1.915041in}{3.140746in}}%
\pgfpathcurveto{\pgfqpoint{1.922855in}{3.148559in}}{\pgfqpoint{1.927245in}{3.159158in}}{\pgfqpoint{1.927245in}{3.170208in}}%
\pgfpathcurveto{\pgfqpoint{1.927245in}{3.181258in}}{\pgfqpoint{1.922855in}{3.191857in}}{\pgfqpoint{1.915041in}{3.199671in}}%
\pgfpathcurveto{\pgfqpoint{1.907228in}{3.207485in}}{\pgfqpoint{1.896629in}{3.211875in}}{\pgfqpoint{1.885579in}{3.211875in}}%
\pgfpathcurveto{\pgfqpoint{1.874528in}{3.211875in}}{\pgfqpoint{1.863929in}{3.207485in}}{\pgfqpoint{1.856116in}{3.199671in}}%
\pgfpathcurveto{\pgfqpoint{1.848302in}{3.191857in}}{\pgfqpoint{1.843912in}{3.181258in}}{\pgfqpoint{1.843912in}{3.170208in}}%
\pgfpathcurveto{\pgfqpoint{1.843912in}{3.159158in}}{\pgfqpoint{1.848302in}{3.148559in}}{\pgfqpoint{1.856116in}{3.140746in}}%
\pgfpathcurveto{\pgfqpoint{1.863929in}{3.132932in}}{\pgfqpoint{1.874528in}{3.128542in}}{\pgfqpoint{1.885579in}{3.128542in}}%
\pgfpathclose%
\pgfusepath{stroke,fill}%
\end{pgfscope}%
\begin{pgfscope}%
\pgfpathrectangle{\pgfqpoint{0.648703in}{0.548769in}}{\pgfqpoint{5.201297in}{3.102590in}}%
\pgfusepath{clip}%
\pgfsetbuttcap%
\pgfsetroundjoin%
\definecolor{currentfill}{rgb}{1.000000,0.498039,0.054902}%
\pgfsetfillcolor{currentfill}%
\pgfsetlinewidth{1.003750pt}%
\definecolor{currentstroke}{rgb}{1.000000,0.498039,0.054902}%
\pgfsetstrokecolor{currentstroke}%
\pgfsetdash{}{0pt}%
\pgfpathmoveto{\pgfqpoint{2.272553in}{3.315195in}}%
\pgfpathcurveto{\pgfqpoint{2.283603in}{3.315195in}}{\pgfqpoint{2.294202in}{3.319585in}}{\pgfqpoint{2.302016in}{3.327399in}}%
\pgfpathcurveto{\pgfqpoint{2.309829in}{3.335212in}}{\pgfqpoint{2.314219in}{3.345811in}}{\pgfqpoint{2.314219in}{3.356861in}}%
\pgfpathcurveto{\pgfqpoint{2.314219in}{3.367912in}}{\pgfqpoint{2.309829in}{3.378511in}}{\pgfqpoint{2.302016in}{3.386324in}}%
\pgfpathcurveto{\pgfqpoint{2.294202in}{3.394138in}}{\pgfqpoint{2.283603in}{3.398528in}}{\pgfqpoint{2.272553in}{3.398528in}}%
\pgfpathcurveto{\pgfqpoint{2.261503in}{3.398528in}}{\pgfqpoint{2.250904in}{3.394138in}}{\pgfqpoint{2.243090in}{3.386324in}}%
\pgfpathcurveto{\pgfqpoint{2.235276in}{3.378511in}}{\pgfqpoint{2.230886in}{3.367912in}}{\pgfqpoint{2.230886in}{3.356861in}}%
\pgfpathcurveto{\pgfqpoint{2.230886in}{3.345811in}}{\pgfqpoint{2.235276in}{3.335212in}}{\pgfqpoint{2.243090in}{3.327399in}}%
\pgfpathcurveto{\pgfqpoint{2.250904in}{3.319585in}}{\pgfqpoint{2.261503in}{3.315195in}}{\pgfqpoint{2.272553in}{3.315195in}}%
\pgfpathclose%
\pgfusepath{stroke,fill}%
\end{pgfscope}%
\begin{pgfscope}%
\pgfpathrectangle{\pgfqpoint{0.648703in}{0.548769in}}{\pgfqpoint{5.201297in}{3.102590in}}%
\pgfusepath{clip}%
\pgfsetbuttcap%
\pgfsetroundjoin%
\definecolor{currentfill}{rgb}{1.000000,0.498039,0.054902}%
\pgfsetfillcolor{currentfill}%
\pgfsetlinewidth{1.003750pt}%
\definecolor{currentstroke}{rgb}{1.000000,0.498039,0.054902}%
\pgfsetstrokecolor{currentstroke}%
\pgfsetdash{}{0pt}%
\pgfpathmoveto{\pgfqpoint{1.770231in}{3.136837in}}%
\pgfpathcurveto{\pgfqpoint{1.781281in}{3.136837in}}{\pgfqpoint{1.791880in}{3.141228in}}{\pgfqpoint{1.799694in}{3.149041in}}%
\pgfpathcurveto{\pgfqpoint{1.807507in}{3.156855in}}{\pgfqpoint{1.811898in}{3.167454in}}{\pgfqpoint{1.811898in}{3.178504in}}%
\pgfpathcurveto{\pgfqpoint{1.811898in}{3.189554in}}{\pgfqpoint{1.807507in}{3.200153in}}{\pgfqpoint{1.799694in}{3.207967in}}%
\pgfpathcurveto{\pgfqpoint{1.791880in}{3.215780in}}{\pgfqpoint{1.781281in}{3.220171in}}{\pgfqpoint{1.770231in}{3.220171in}}%
\pgfpathcurveto{\pgfqpoint{1.759181in}{3.220171in}}{\pgfqpoint{1.748582in}{3.215780in}}{\pgfqpoint{1.740768in}{3.207967in}}%
\pgfpathcurveto{\pgfqpoint{1.732954in}{3.200153in}}{\pgfqpoint{1.728564in}{3.189554in}}{\pgfqpoint{1.728564in}{3.178504in}}%
\pgfpathcurveto{\pgfqpoint{1.728564in}{3.167454in}}{\pgfqpoint{1.732954in}{3.156855in}}{\pgfqpoint{1.740768in}{3.149041in}}%
\pgfpathcurveto{\pgfqpoint{1.748582in}{3.141228in}}{\pgfqpoint{1.759181in}{3.136837in}}{\pgfqpoint{1.770231in}{3.136837in}}%
\pgfpathclose%
\pgfusepath{stroke,fill}%
\end{pgfscope}%
\begin{pgfscope}%
\pgfpathrectangle{\pgfqpoint{0.648703in}{0.548769in}}{\pgfqpoint{5.201297in}{3.102590in}}%
\pgfusepath{clip}%
\pgfsetbuttcap%
\pgfsetroundjoin%
\definecolor{currentfill}{rgb}{1.000000,0.498039,0.054902}%
\pgfsetfillcolor{currentfill}%
\pgfsetlinewidth{1.003750pt}%
\definecolor{currentstroke}{rgb}{1.000000,0.498039,0.054902}%
\pgfsetstrokecolor{currentstroke}%
\pgfsetdash{}{0pt}%
\pgfpathmoveto{\pgfqpoint{1.675483in}{3.136837in}}%
\pgfpathcurveto{\pgfqpoint{1.686533in}{3.136837in}}{\pgfqpoint{1.697132in}{3.141228in}}{\pgfqpoint{1.704945in}{3.149041in}}%
\pgfpathcurveto{\pgfqpoint{1.712759in}{3.156855in}}{\pgfqpoint{1.717149in}{3.167454in}}{\pgfqpoint{1.717149in}{3.178504in}}%
\pgfpathcurveto{\pgfqpoint{1.717149in}{3.189554in}}{\pgfqpoint{1.712759in}{3.200153in}}{\pgfqpoint{1.704945in}{3.207967in}}%
\pgfpathcurveto{\pgfqpoint{1.697132in}{3.215780in}}{\pgfqpoint{1.686533in}{3.220171in}}{\pgfqpoint{1.675483in}{3.220171in}}%
\pgfpathcurveto{\pgfqpoint{1.664432in}{3.220171in}}{\pgfqpoint{1.653833in}{3.215780in}}{\pgfqpoint{1.646020in}{3.207967in}}%
\pgfpathcurveto{\pgfqpoint{1.638206in}{3.200153in}}{\pgfqpoint{1.633816in}{3.189554in}}{\pgfqpoint{1.633816in}{3.178504in}}%
\pgfpathcurveto{\pgfqpoint{1.633816in}{3.167454in}}{\pgfqpoint{1.638206in}{3.156855in}}{\pgfqpoint{1.646020in}{3.149041in}}%
\pgfpathcurveto{\pgfqpoint{1.653833in}{3.141228in}}{\pgfqpoint{1.664432in}{3.136837in}}{\pgfqpoint{1.675483in}{3.136837in}}%
\pgfpathclose%
\pgfusepath{stroke,fill}%
\end{pgfscope}%
\begin{pgfscope}%
\pgfpathrectangle{\pgfqpoint{0.648703in}{0.548769in}}{\pgfqpoint{5.201297in}{3.102590in}}%
\pgfusepath{clip}%
\pgfsetbuttcap%
\pgfsetroundjoin%
\definecolor{currentfill}{rgb}{1.000000,0.498039,0.054902}%
\pgfsetfillcolor{currentfill}%
\pgfsetlinewidth{1.003750pt}%
\definecolor{currentstroke}{rgb}{1.000000,0.498039,0.054902}%
\pgfsetstrokecolor{currentstroke}%
\pgfsetdash{}{0pt}%
\pgfpathmoveto{\pgfqpoint{1.619213in}{3.140985in}}%
\pgfpathcurveto{\pgfqpoint{1.630263in}{3.140985in}}{\pgfqpoint{1.640862in}{3.145375in}}{\pgfqpoint{1.648676in}{3.153189in}}%
\pgfpathcurveto{\pgfqpoint{1.656490in}{3.161003in}}{\pgfqpoint{1.660880in}{3.171602in}}{\pgfqpoint{1.660880in}{3.182652in}}%
\pgfpathcurveto{\pgfqpoint{1.660880in}{3.193702in}}{\pgfqpoint{1.656490in}{3.204301in}}{\pgfqpoint{1.648676in}{3.212115in}}%
\pgfpathcurveto{\pgfqpoint{1.640862in}{3.219928in}}{\pgfqpoint{1.630263in}{3.224319in}}{\pgfqpoint{1.619213in}{3.224319in}}%
\pgfpathcurveto{\pgfqpoint{1.608163in}{3.224319in}}{\pgfqpoint{1.597564in}{3.219928in}}{\pgfqpoint{1.589751in}{3.212115in}}%
\pgfpathcurveto{\pgfqpoint{1.581937in}{3.204301in}}{\pgfqpoint{1.577547in}{3.193702in}}{\pgfqpoint{1.577547in}{3.182652in}}%
\pgfpathcurveto{\pgfqpoint{1.577547in}{3.171602in}}{\pgfqpoint{1.581937in}{3.161003in}}{\pgfqpoint{1.589751in}{3.153189in}}%
\pgfpathcurveto{\pgfqpoint{1.597564in}{3.145375in}}{\pgfqpoint{1.608163in}{3.140985in}}{\pgfqpoint{1.619213in}{3.140985in}}%
\pgfpathclose%
\pgfusepath{stroke,fill}%
\end{pgfscope}%
\begin{pgfscope}%
\pgfpathrectangle{\pgfqpoint{0.648703in}{0.548769in}}{\pgfqpoint{5.201297in}{3.102590in}}%
\pgfusepath{clip}%
\pgfsetbuttcap%
\pgfsetroundjoin%
\definecolor{currentfill}{rgb}{1.000000,0.498039,0.054902}%
\pgfsetfillcolor{currentfill}%
\pgfsetlinewidth{1.003750pt}%
\definecolor{currentstroke}{rgb}{1.000000,0.498039,0.054902}%
\pgfsetstrokecolor{currentstroke}%
\pgfsetdash{}{0pt}%
\pgfpathmoveto{\pgfqpoint{1.336306in}{3.136837in}}%
\pgfpathcurveto{\pgfqpoint{1.347356in}{3.136837in}}{\pgfqpoint{1.357955in}{3.141228in}}{\pgfqpoint{1.365769in}{3.149041in}}%
\pgfpathcurveto{\pgfqpoint{1.373583in}{3.156855in}}{\pgfqpoint{1.377973in}{3.167454in}}{\pgfqpoint{1.377973in}{3.178504in}}%
\pgfpathcurveto{\pgfqpoint{1.377973in}{3.189554in}}{\pgfqpoint{1.373583in}{3.200153in}}{\pgfqpoint{1.365769in}{3.207967in}}%
\pgfpathcurveto{\pgfqpoint{1.357955in}{3.215780in}}{\pgfqpoint{1.347356in}{3.220171in}}{\pgfqpoint{1.336306in}{3.220171in}}%
\pgfpathcurveto{\pgfqpoint{1.325256in}{3.220171in}}{\pgfqpoint{1.314657in}{3.215780in}}{\pgfqpoint{1.306843in}{3.207967in}}%
\pgfpathcurveto{\pgfqpoint{1.299030in}{3.200153in}}{\pgfqpoint{1.294639in}{3.189554in}}{\pgfqpoint{1.294639in}{3.178504in}}%
\pgfpathcurveto{\pgfqpoint{1.294639in}{3.167454in}}{\pgfqpoint{1.299030in}{3.156855in}}{\pgfqpoint{1.306843in}{3.149041in}}%
\pgfpathcurveto{\pgfqpoint{1.314657in}{3.141228in}}{\pgfqpoint{1.325256in}{3.136837in}}{\pgfqpoint{1.336306in}{3.136837in}}%
\pgfpathclose%
\pgfusepath{stroke,fill}%
\end{pgfscope}%
\begin{pgfscope}%
\pgfpathrectangle{\pgfqpoint{0.648703in}{0.548769in}}{\pgfqpoint{5.201297in}{3.102590in}}%
\pgfusepath{clip}%
\pgfsetbuttcap%
\pgfsetroundjoin%
\definecolor{currentfill}{rgb}{1.000000,0.498039,0.054902}%
\pgfsetfillcolor{currentfill}%
\pgfsetlinewidth{1.003750pt}%
\definecolor{currentstroke}{rgb}{1.000000,0.498039,0.054902}%
\pgfsetstrokecolor{currentstroke}%
\pgfsetdash{}{0pt}%
\pgfpathmoveto{\pgfqpoint{2.080738in}{3.356673in}}%
\pgfpathcurveto{\pgfqpoint{2.091788in}{3.356673in}}{\pgfqpoint{2.102387in}{3.361064in}}{\pgfqpoint{2.110200in}{3.368877in}}%
\pgfpathcurveto{\pgfqpoint{2.118014in}{3.376691in}}{\pgfqpoint{2.122404in}{3.387290in}}{\pgfqpoint{2.122404in}{3.398340in}}%
\pgfpathcurveto{\pgfqpoint{2.122404in}{3.409390in}}{\pgfqpoint{2.118014in}{3.419989in}}{\pgfqpoint{2.110200in}{3.427803in}}%
\pgfpathcurveto{\pgfqpoint{2.102387in}{3.435616in}}{\pgfqpoint{2.091788in}{3.440007in}}{\pgfqpoint{2.080738in}{3.440007in}}%
\pgfpathcurveto{\pgfqpoint{2.069688in}{3.440007in}}{\pgfqpoint{2.059089in}{3.435616in}}{\pgfqpoint{2.051275in}{3.427803in}}%
\pgfpathcurveto{\pgfqpoint{2.043461in}{3.419989in}}{\pgfqpoint{2.039071in}{3.409390in}}{\pgfqpoint{2.039071in}{3.398340in}}%
\pgfpathcurveto{\pgfqpoint{2.039071in}{3.387290in}}{\pgfqpoint{2.043461in}{3.376691in}}{\pgfqpoint{2.051275in}{3.368877in}}%
\pgfpathcurveto{\pgfqpoint{2.059089in}{3.361064in}}{\pgfqpoint{2.069688in}{3.356673in}}{\pgfqpoint{2.080738in}{3.356673in}}%
\pgfpathclose%
\pgfusepath{stroke,fill}%
\end{pgfscope}%
\begin{pgfscope}%
\pgfpathrectangle{\pgfqpoint{0.648703in}{0.548769in}}{\pgfqpoint{5.201297in}{3.102590in}}%
\pgfusepath{clip}%
\pgfsetbuttcap%
\pgfsetroundjoin%
\definecolor{currentfill}{rgb}{1.000000,0.498039,0.054902}%
\pgfsetfillcolor{currentfill}%
\pgfsetlinewidth{1.003750pt}%
\definecolor{currentstroke}{rgb}{1.000000,0.498039,0.054902}%
\pgfsetstrokecolor{currentstroke}%
\pgfsetdash{}{0pt}%
\pgfpathmoveto{\pgfqpoint{2.405914in}{3.219794in}}%
\pgfpathcurveto{\pgfqpoint{2.416964in}{3.219794in}}{\pgfqpoint{2.427563in}{3.224185in}}{\pgfqpoint{2.435377in}{3.231998in}}%
\pgfpathcurveto{\pgfqpoint{2.443190in}{3.239812in}}{\pgfqpoint{2.447580in}{3.250411in}}{\pgfqpoint{2.447580in}{3.261461in}}%
\pgfpathcurveto{\pgfqpoint{2.447580in}{3.272511in}}{\pgfqpoint{2.443190in}{3.283110in}}{\pgfqpoint{2.435377in}{3.290924in}}%
\pgfpathcurveto{\pgfqpoint{2.427563in}{3.298737in}}{\pgfqpoint{2.416964in}{3.303128in}}{\pgfqpoint{2.405914in}{3.303128in}}%
\pgfpathcurveto{\pgfqpoint{2.394864in}{3.303128in}}{\pgfqpoint{2.384265in}{3.298737in}}{\pgfqpoint{2.376451in}{3.290924in}}%
\pgfpathcurveto{\pgfqpoint{2.368637in}{3.283110in}}{\pgfqpoint{2.364247in}{3.272511in}}{\pgfqpoint{2.364247in}{3.261461in}}%
\pgfpathcurveto{\pgfqpoint{2.364247in}{3.250411in}}{\pgfqpoint{2.368637in}{3.239812in}}{\pgfqpoint{2.376451in}{3.231998in}}%
\pgfpathcurveto{\pgfqpoint{2.384265in}{3.224185in}}{\pgfqpoint{2.394864in}{3.219794in}}{\pgfqpoint{2.405914in}{3.219794in}}%
\pgfpathclose%
\pgfusepath{stroke,fill}%
\end{pgfscope}%
\begin{pgfscope}%
\pgfpathrectangle{\pgfqpoint{0.648703in}{0.548769in}}{\pgfqpoint{5.201297in}{3.102590in}}%
\pgfusepath{clip}%
\pgfsetbuttcap%
\pgfsetroundjoin%
\definecolor{currentfill}{rgb}{0.121569,0.466667,0.705882}%
\pgfsetfillcolor{currentfill}%
\pgfsetlinewidth{1.003750pt}%
\definecolor{currentstroke}{rgb}{0.121569,0.466667,0.705882}%
\pgfsetstrokecolor{currentstroke}%
\pgfsetdash{}{0pt}%
\pgfpathmoveto{\pgfqpoint{4.289955in}{3.120246in}}%
\pgfpathcurveto{\pgfqpoint{4.301005in}{3.120246in}}{\pgfqpoint{4.311604in}{3.124636in}}{\pgfqpoint{4.319418in}{3.132450in}}%
\pgfpathcurveto{\pgfqpoint{4.327232in}{3.140263in}}{\pgfqpoint{4.331622in}{3.150862in}}{\pgfqpoint{4.331622in}{3.161913in}}%
\pgfpathcurveto{\pgfqpoint{4.331622in}{3.172963in}}{\pgfqpoint{4.327232in}{3.183562in}}{\pgfqpoint{4.319418in}{3.191375in}}%
\pgfpathcurveto{\pgfqpoint{4.311604in}{3.199189in}}{\pgfqpoint{4.301005in}{3.203579in}}{\pgfqpoint{4.289955in}{3.203579in}}%
\pgfpathcurveto{\pgfqpoint{4.278905in}{3.203579in}}{\pgfqpoint{4.268306in}{3.199189in}}{\pgfqpoint{4.260492in}{3.191375in}}%
\pgfpathcurveto{\pgfqpoint{4.252679in}{3.183562in}}{\pgfqpoint{4.248289in}{3.172963in}}{\pgfqpoint{4.248289in}{3.161913in}}%
\pgfpathcurveto{\pgfqpoint{4.248289in}{3.150862in}}{\pgfqpoint{4.252679in}{3.140263in}}{\pgfqpoint{4.260492in}{3.132450in}}%
\pgfpathcurveto{\pgfqpoint{4.268306in}{3.124636in}}{\pgfqpoint{4.278905in}{3.120246in}}{\pgfqpoint{4.289955in}{3.120246in}}%
\pgfpathclose%
\pgfusepath{stroke,fill}%
\end{pgfscope}%
\begin{pgfscope}%
\pgfpathrectangle{\pgfqpoint{0.648703in}{0.548769in}}{\pgfqpoint{5.201297in}{3.102590in}}%
\pgfusepath{clip}%
\pgfsetbuttcap%
\pgfsetroundjoin%
\definecolor{currentfill}{rgb}{1.000000,0.498039,0.054902}%
\pgfsetfillcolor{currentfill}%
\pgfsetlinewidth{1.003750pt}%
\definecolor{currentstroke}{rgb}{1.000000,0.498039,0.054902}%
\pgfsetstrokecolor{currentstroke}%
\pgfsetdash{}{0pt}%
\pgfpathmoveto{\pgfqpoint{1.565619in}{3.140985in}}%
\pgfpathcurveto{\pgfqpoint{1.576669in}{3.140985in}}{\pgfqpoint{1.587268in}{3.145375in}}{\pgfqpoint{1.595082in}{3.153189in}}%
\pgfpathcurveto{\pgfqpoint{1.602896in}{3.161003in}}{\pgfqpoint{1.607286in}{3.171602in}}{\pgfqpoint{1.607286in}{3.182652in}}%
\pgfpathcurveto{\pgfqpoint{1.607286in}{3.193702in}}{\pgfqpoint{1.602896in}{3.204301in}}{\pgfqpoint{1.595082in}{3.212115in}}%
\pgfpathcurveto{\pgfqpoint{1.587268in}{3.219928in}}{\pgfqpoint{1.576669in}{3.224319in}}{\pgfqpoint{1.565619in}{3.224319in}}%
\pgfpathcurveto{\pgfqpoint{1.554569in}{3.224319in}}{\pgfqpoint{1.543970in}{3.219928in}}{\pgfqpoint{1.536156in}{3.212115in}}%
\pgfpathcurveto{\pgfqpoint{1.528343in}{3.204301in}}{\pgfqpoint{1.523953in}{3.193702in}}{\pgfqpoint{1.523953in}{3.182652in}}%
\pgfpathcurveto{\pgfqpoint{1.523953in}{3.171602in}}{\pgfqpoint{1.528343in}{3.161003in}}{\pgfqpoint{1.536156in}{3.153189in}}%
\pgfpathcurveto{\pgfqpoint{1.543970in}{3.145375in}}{\pgfqpoint{1.554569in}{3.140985in}}{\pgfqpoint{1.565619in}{3.140985in}}%
\pgfpathclose%
\pgfusepath{stroke,fill}%
\end{pgfscope}%
\begin{pgfscope}%
\pgfpathrectangle{\pgfqpoint{0.648703in}{0.548769in}}{\pgfqpoint{5.201297in}{3.102590in}}%
\pgfusepath{clip}%
\pgfsetbuttcap%
\pgfsetroundjoin%
\definecolor{currentfill}{rgb}{1.000000,0.498039,0.054902}%
\pgfsetfillcolor{currentfill}%
\pgfsetlinewidth{1.003750pt}%
\definecolor{currentstroke}{rgb}{1.000000,0.498039,0.054902}%
\pgfsetstrokecolor{currentstroke}%
\pgfsetdash{}{0pt}%
\pgfpathmoveto{\pgfqpoint{1.890662in}{3.145133in}}%
\pgfpathcurveto{\pgfqpoint{1.901712in}{3.145133in}}{\pgfqpoint{1.912311in}{3.149523in}}{\pgfqpoint{1.920124in}{3.157337in}}%
\pgfpathcurveto{\pgfqpoint{1.927938in}{3.165151in}}{\pgfqpoint{1.932328in}{3.175750in}}{\pgfqpoint{1.932328in}{3.186800in}}%
\pgfpathcurveto{\pgfqpoint{1.932328in}{3.197850in}}{\pgfqpoint{1.927938in}{3.208449in}}{\pgfqpoint{1.920124in}{3.216262in}}%
\pgfpathcurveto{\pgfqpoint{1.912311in}{3.224076in}}{\pgfqpoint{1.901712in}{3.228466in}}{\pgfqpoint{1.890662in}{3.228466in}}%
\pgfpathcurveto{\pgfqpoint{1.879611in}{3.228466in}}{\pgfqpoint{1.869012in}{3.224076in}}{\pgfqpoint{1.861199in}{3.216262in}}%
\pgfpathcurveto{\pgfqpoint{1.853385in}{3.208449in}}{\pgfqpoint{1.848995in}{3.197850in}}{\pgfqpoint{1.848995in}{3.186800in}}%
\pgfpathcurveto{\pgfqpoint{1.848995in}{3.175750in}}{\pgfqpoint{1.853385in}{3.165151in}}{\pgfqpoint{1.861199in}{3.157337in}}%
\pgfpathcurveto{\pgfqpoint{1.869012in}{3.149523in}}{\pgfqpoint{1.879611in}{3.145133in}}{\pgfqpoint{1.890662in}{3.145133in}}%
\pgfpathclose%
\pgfusepath{stroke,fill}%
\end{pgfscope}%
\begin{pgfscope}%
\pgfpathrectangle{\pgfqpoint{0.648703in}{0.548769in}}{\pgfqpoint{5.201297in}{3.102590in}}%
\pgfusepath{clip}%
\pgfsetbuttcap%
\pgfsetroundjoin%
\definecolor{currentfill}{rgb}{1.000000,0.498039,0.054902}%
\pgfsetfillcolor{currentfill}%
\pgfsetlinewidth{1.003750pt}%
\definecolor{currentstroke}{rgb}{1.000000,0.498039,0.054902}%
\pgfsetstrokecolor{currentstroke}%
\pgfsetdash{}{0pt}%
\pgfpathmoveto{\pgfqpoint{1.949918in}{3.136837in}}%
\pgfpathcurveto{\pgfqpoint{1.960968in}{3.136837in}}{\pgfqpoint{1.971567in}{3.141228in}}{\pgfqpoint{1.979381in}{3.149041in}}%
\pgfpathcurveto{\pgfqpoint{1.987195in}{3.156855in}}{\pgfqpoint{1.991585in}{3.167454in}}{\pgfqpoint{1.991585in}{3.178504in}}%
\pgfpathcurveto{\pgfqpoint{1.991585in}{3.189554in}}{\pgfqpoint{1.987195in}{3.200153in}}{\pgfqpoint{1.979381in}{3.207967in}}%
\pgfpathcurveto{\pgfqpoint{1.971567in}{3.215780in}}{\pgfqpoint{1.960968in}{3.220171in}}{\pgfqpoint{1.949918in}{3.220171in}}%
\pgfpathcurveto{\pgfqpoint{1.938868in}{3.220171in}}{\pgfqpoint{1.928269in}{3.215780in}}{\pgfqpoint{1.920455in}{3.207967in}}%
\pgfpathcurveto{\pgfqpoint{1.912642in}{3.200153in}}{\pgfqpoint{1.908252in}{3.189554in}}{\pgfqpoint{1.908252in}{3.178504in}}%
\pgfpathcurveto{\pgfqpoint{1.908252in}{3.167454in}}{\pgfqpoint{1.912642in}{3.156855in}}{\pgfqpoint{1.920455in}{3.149041in}}%
\pgfpathcurveto{\pgfqpoint{1.928269in}{3.141228in}}{\pgfqpoint{1.938868in}{3.136837in}}{\pgfqpoint{1.949918in}{3.136837in}}%
\pgfpathclose%
\pgfusepath{stroke,fill}%
\end{pgfscope}%
\begin{pgfscope}%
\pgfpathrectangle{\pgfqpoint{0.648703in}{0.548769in}}{\pgfqpoint{5.201297in}{3.102590in}}%
\pgfusepath{clip}%
\pgfsetbuttcap%
\pgfsetroundjoin%
\definecolor{currentfill}{rgb}{0.121569,0.466667,0.705882}%
\pgfsetfillcolor{currentfill}%
\pgfsetlinewidth{1.003750pt}%
\definecolor{currentstroke}{rgb}{0.121569,0.466667,0.705882}%
\pgfsetstrokecolor{currentstroke}%
\pgfsetdash{}{0pt}%
\pgfpathmoveto{\pgfqpoint{2.320083in}{3.132690in}}%
\pgfpathcurveto{\pgfqpoint{2.331133in}{3.132690in}}{\pgfqpoint{2.341732in}{3.137080in}}{\pgfqpoint{2.349546in}{3.144893in}}%
\pgfpathcurveto{\pgfqpoint{2.357359in}{3.152707in}}{\pgfqpoint{2.361750in}{3.163306in}}{\pgfqpoint{2.361750in}{3.174356in}}%
\pgfpathcurveto{\pgfqpoint{2.361750in}{3.185406in}}{\pgfqpoint{2.357359in}{3.196005in}}{\pgfqpoint{2.349546in}{3.203819in}}%
\pgfpathcurveto{\pgfqpoint{2.341732in}{3.211633in}}{\pgfqpoint{2.331133in}{3.216023in}}{\pgfqpoint{2.320083in}{3.216023in}}%
\pgfpathcurveto{\pgfqpoint{2.309033in}{3.216023in}}{\pgfqpoint{2.298434in}{3.211633in}}{\pgfqpoint{2.290620in}{3.203819in}}%
\pgfpathcurveto{\pgfqpoint{2.282807in}{3.196005in}}{\pgfqpoint{2.278416in}{3.185406in}}{\pgfqpoint{2.278416in}{3.174356in}}%
\pgfpathcurveto{\pgfqpoint{2.278416in}{3.163306in}}{\pgfqpoint{2.282807in}{3.152707in}}{\pgfqpoint{2.290620in}{3.144893in}}%
\pgfpathcurveto{\pgfqpoint{2.298434in}{3.137080in}}{\pgfqpoint{2.309033in}{3.132690in}}{\pgfqpoint{2.320083in}{3.132690in}}%
\pgfpathclose%
\pgfusepath{stroke,fill}%
\end{pgfscope}%
\begin{pgfscope}%
\pgfpathrectangle{\pgfqpoint{0.648703in}{0.548769in}}{\pgfqpoint{5.201297in}{3.102590in}}%
\pgfusepath{clip}%
\pgfsetbuttcap%
\pgfsetroundjoin%
\definecolor{currentfill}{rgb}{0.121569,0.466667,0.705882}%
\pgfsetfillcolor{currentfill}%
\pgfsetlinewidth{1.003750pt}%
\definecolor{currentstroke}{rgb}{0.121569,0.466667,0.705882}%
\pgfsetstrokecolor{currentstroke}%
\pgfsetdash{}{0pt}%
\pgfpathmoveto{\pgfqpoint{1.976180in}{3.132690in}}%
\pgfpathcurveto{\pgfqpoint{1.987230in}{3.132690in}}{\pgfqpoint{1.997829in}{3.137080in}}{\pgfqpoint{2.005643in}{3.144893in}}%
\pgfpathcurveto{\pgfqpoint{2.013457in}{3.152707in}}{\pgfqpoint{2.017847in}{3.163306in}}{\pgfqpoint{2.017847in}{3.174356in}}%
\pgfpathcurveto{\pgfqpoint{2.017847in}{3.185406in}}{\pgfqpoint{2.013457in}{3.196005in}}{\pgfqpoint{2.005643in}{3.203819in}}%
\pgfpathcurveto{\pgfqpoint{1.997829in}{3.211633in}}{\pgfqpoint{1.987230in}{3.216023in}}{\pgfqpoint{1.976180in}{3.216023in}}%
\pgfpathcurveto{\pgfqpoint{1.965130in}{3.216023in}}{\pgfqpoint{1.954531in}{3.211633in}}{\pgfqpoint{1.946717in}{3.203819in}}%
\pgfpathcurveto{\pgfqpoint{1.938904in}{3.196005in}}{\pgfqpoint{1.934514in}{3.185406in}}{\pgfqpoint{1.934514in}{3.174356in}}%
\pgfpathcurveto{\pgfqpoint{1.934514in}{3.163306in}}{\pgfqpoint{1.938904in}{3.152707in}}{\pgfqpoint{1.946717in}{3.144893in}}%
\pgfpathcurveto{\pgfqpoint{1.954531in}{3.137080in}}{\pgfqpoint{1.965130in}{3.132690in}}{\pgfqpoint{1.976180in}{3.132690in}}%
\pgfpathclose%
\pgfusepath{stroke,fill}%
\end{pgfscope}%
\begin{pgfscope}%
\pgfpathrectangle{\pgfqpoint{0.648703in}{0.548769in}}{\pgfqpoint{5.201297in}{3.102590in}}%
\pgfusepath{clip}%
\pgfsetbuttcap%
\pgfsetroundjoin%
\definecolor{currentfill}{rgb}{1.000000,0.498039,0.054902}%
\pgfsetfillcolor{currentfill}%
\pgfsetlinewidth{1.003750pt}%
\definecolor{currentstroke}{rgb}{1.000000,0.498039,0.054902}%
\pgfsetstrokecolor{currentstroke}%
\pgfsetdash{}{0pt}%
\pgfpathmoveto{\pgfqpoint{1.970161in}{3.136837in}}%
\pgfpathcurveto{\pgfqpoint{1.981211in}{3.136837in}}{\pgfqpoint{1.991810in}{3.141228in}}{\pgfqpoint{1.999624in}{3.149041in}}%
\pgfpathcurveto{\pgfqpoint{2.007437in}{3.156855in}}{\pgfqpoint{2.011828in}{3.167454in}}{\pgfqpoint{2.011828in}{3.178504in}}%
\pgfpathcurveto{\pgfqpoint{2.011828in}{3.189554in}}{\pgfqpoint{2.007437in}{3.200153in}}{\pgfqpoint{1.999624in}{3.207967in}}%
\pgfpathcurveto{\pgfqpoint{1.991810in}{3.215780in}}{\pgfqpoint{1.981211in}{3.220171in}}{\pgfqpoint{1.970161in}{3.220171in}}%
\pgfpathcurveto{\pgfqpoint{1.959111in}{3.220171in}}{\pgfqpoint{1.948512in}{3.215780in}}{\pgfqpoint{1.940698in}{3.207967in}}%
\pgfpathcurveto{\pgfqpoint{1.932884in}{3.200153in}}{\pgfqpoint{1.928494in}{3.189554in}}{\pgfqpoint{1.928494in}{3.178504in}}%
\pgfpathcurveto{\pgfqpoint{1.928494in}{3.167454in}}{\pgfqpoint{1.932884in}{3.156855in}}{\pgfqpoint{1.940698in}{3.149041in}}%
\pgfpathcurveto{\pgfqpoint{1.948512in}{3.141228in}}{\pgfqpoint{1.959111in}{3.136837in}}{\pgfqpoint{1.970161in}{3.136837in}}%
\pgfpathclose%
\pgfusepath{stroke,fill}%
\end{pgfscope}%
\begin{pgfscope}%
\pgfpathrectangle{\pgfqpoint{0.648703in}{0.548769in}}{\pgfqpoint{5.201297in}{3.102590in}}%
\pgfusepath{clip}%
\pgfsetbuttcap%
\pgfsetroundjoin%
\definecolor{currentfill}{rgb}{0.839216,0.152941,0.156863}%
\pgfsetfillcolor{currentfill}%
\pgfsetlinewidth{1.003750pt}%
\definecolor{currentstroke}{rgb}{0.839216,0.152941,0.156863}%
\pgfsetstrokecolor{currentstroke}%
\pgfsetdash{}{0pt}%
\pgfpathmoveto{\pgfqpoint{2.648380in}{3.120246in}}%
\pgfpathcurveto{\pgfqpoint{2.659430in}{3.120246in}}{\pgfqpoint{2.670029in}{3.124636in}}{\pgfqpoint{2.677843in}{3.132450in}}%
\pgfpathcurveto{\pgfqpoint{2.685657in}{3.140263in}}{\pgfqpoint{2.690047in}{3.150862in}}{\pgfqpoint{2.690047in}{3.161913in}}%
\pgfpathcurveto{\pgfqpoint{2.690047in}{3.172963in}}{\pgfqpoint{2.685657in}{3.183562in}}{\pgfqpoint{2.677843in}{3.191375in}}%
\pgfpathcurveto{\pgfqpoint{2.670029in}{3.199189in}}{\pgfqpoint{2.659430in}{3.203579in}}{\pgfqpoint{2.648380in}{3.203579in}}%
\pgfpathcurveto{\pgfqpoint{2.637330in}{3.203579in}}{\pgfqpoint{2.626731in}{3.199189in}}{\pgfqpoint{2.618917in}{3.191375in}}%
\pgfpathcurveto{\pgfqpoint{2.611104in}{3.183562in}}{\pgfqpoint{2.606713in}{3.172963in}}{\pgfqpoint{2.606713in}{3.161913in}}%
\pgfpathcurveto{\pgfqpoint{2.606713in}{3.150862in}}{\pgfqpoint{2.611104in}{3.140263in}}{\pgfqpoint{2.618917in}{3.132450in}}%
\pgfpathcurveto{\pgfqpoint{2.626731in}{3.124636in}}{\pgfqpoint{2.637330in}{3.120246in}}{\pgfqpoint{2.648380in}{3.120246in}}%
\pgfpathclose%
\pgfusepath{stroke,fill}%
\end{pgfscope}%
\begin{pgfscope}%
\pgfpathrectangle{\pgfqpoint{0.648703in}{0.548769in}}{\pgfqpoint{5.201297in}{3.102590in}}%
\pgfusepath{clip}%
\pgfsetbuttcap%
\pgfsetroundjoin%
\definecolor{currentfill}{rgb}{1.000000,0.498039,0.054902}%
\pgfsetfillcolor{currentfill}%
\pgfsetlinewidth{1.003750pt}%
\definecolor{currentstroke}{rgb}{1.000000,0.498039,0.054902}%
\pgfsetstrokecolor{currentstroke}%
\pgfsetdash{}{0pt}%
\pgfpathmoveto{\pgfqpoint{1.447730in}{3.244681in}}%
\pgfpathcurveto{\pgfqpoint{1.458780in}{3.244681in}}{\pgfqpoint{1.469379in}{3.249072in}}{\pgfqpoint{1.477193in}{3.256885in}}%
\pgfpathcurveto{\pgfqpoint{1.485006in}{3.264699in}}{\pgfqpoint{1.489397in}{3.275298in}}{\pgfqpoint{1.489397in}{3.286348in}}%
\pgfpathcurveto{\pgfqpoint{1.489397in}{3.297398in}}{\pgfqpoint{1.485006in}{3.307997in}}{\pgfqpoint{1.477193in}{3.315811in}}%
\pgfpathcurveto{\pgfqpoint{1.469379in}{3.323624in}}{\pgfqpoint{1.458780in}{3.328015in}}{\pgfqpoint{1.447730in}{3.328015in}}%
\pgfpathcurveto{\pgfqpoint{1.436680in}{3.328015in}}{\pgfqpoint{1.426081in}{3.323624in}}{\pgfqpoint{1.418267in}{3.315811in}}%
\pgfpathcurveto{\pgfqpoint{1.410454in}{3.307997in}}{\pgfqpoint{1.406063in}{3.297398in}}{\pgfqpoint{1.406063in}{3.286348in}}%
\pgfpathcurveto{\pgfqpoint{1.406063in}{3.275298in}}{\pgfqpoint{1.410454in}{3.264699in}}{\pgfqpoint{1.418267in}{3.256885in}}%
\pgfpathcurveto{\pgfqpoint{1.426081in}{3.249072in}}{\pgfqpoint{1.436680in}{3.244681in}}{\pgfqpoint{1.447730in}{3.244681in}}%
\pgfpathclose%
\pgfusepath{stroke,fill}%
\end{pgfscope}%
\begin{pgfscope}%
\pgfpathrectangle{\pgfqpoint{0.648703in}{0.548769in}}{\pgfqpoint{5.201297in}{3.102590in}}%
\pgfusepath{clip}%
\pgfsetbuttcap%
\pgfsetroundjoin%
\definecolor{currentfill}{rgb}{0.121569,0.466667,0.705882}%
\pgfsetfillcolor{currentfill}%
\pgfsetlinewidth{1.003750pt}%
\definecolor{currentstroke}{rgb}{0.121569,0.466667,0.705882}%
\pgfsetstrokecolor{currentstroke}%
\pgfsetdash{}{0pt}%
\pgfpathmoveto{\pgfqpoint{2.029061in}{3.128542in}}%
\pgfpathcurveto{\pgfqpoint{2.040111in}{3.128542in}}{\pgfqpoint{2.050710in}{3.132932in}}{\pgfqpoint{2.058524in}{3.140746in}}%
\pgfpathcurveto{\pgfqpoint{2.066337in}{3.148559in}}{\pgfqpoint{2.070728in}{3.159158in}}{\pgfqpoint{2.070728in}{3.170208in}}%
\pgfpathcurveto{\pgfqpoint{2.070728in}{3.181258in}}{\pgfqpoint{2.066337in}{3.191857in}}{\pgfqpoint{2.058524in}{3.199671in}}%
\pgfpathcurveto{\pgfqpoint{2.050710in}{3.207485in}}{\pgfqpoint{2.040111in}{3.211875in}}{\pgfqpoint{2.029061in}{3.211875in}}%
\pgfpathcurveto{\pgfqpoint{2.018011in}{3.211875in}}{\pgfqpoint{2.007412in}{3.207485in}}{\pgfqpoint{1.999598in}{3.199671in}}%
\pgfpathcurveto{\pgfqpoint{1.991784in}{3.191857in}}{\pgfqpoint{1.987394in}{3.181258in}}{\pgfqpoint{1.987394in}{3.170208in}}%
\pgfpathcurveto{\pgfqpoint{1.987394in}{3.159158in}}{\pgfqpoint{1.991784in}{3.148559in}}{\pgfqpoint{1.999598in}{3.140746in}}%
\pgfpathcurveto{\pgfqpoint{2.007412in}{3.132932in}}{\pgfqpoint{2.018011in}{3.128542in}}{\pgfqpoint{2.029061in}{3.128542in}}%
\pgfpathclose%
\pgfusepath{stroke,fill}%
\end{pgfscope}%
\begin{pgfscope}%
\pgfpathrectangle{\pgfqpoint{0.648703in}{0.548769in}}{\pgfqpoint{5.201297in}{3.102590in}}%
\pgfusepath{clip}%
\pgfsetbuttcap%
\pgfsetroundjoin%
\definecolor{currentfill}{rgb}{1.000000,0.498039,0.054902}%
\pgfsetfillcolor{currentfill}%
\pgfsetlinewidth{1.003750pt}%
\definecolor{currentstroke}{rgb}{1.000000,0.498039,0.054902}%
\pgfsetstrokecolor{currentstroke}%
\pgfsetdash{}{0pt}%
\pgfpathmoveto{\pgfqpoint{2.396194in}{3.136837in}}%
\pgfpathcurveto{\pgfqpoint{2.407244in}{3.136837in}}{\pgfqpoint{2.417843in}{3.141228in}}{\pgfqpoint{2.425656in}{3.149041in}}%
\pgfpathcurveto{\pgfqpoint{2.433470in}{3.156855in}}{\pgfqpoint{2.437860in}{3.167454in}}{\pgfqpoint{2.437860in}{3.178504in}}%
\pgfpathcurveto{\pgfqpoint{2.437860in}{3.189554in}}{\pgfqpoint{2.433470in}{3.200153in}}{\pgfqpoint{2.425656in}{3.207967in}}%
\pgfpathcurveto{\pgfqpoint{2.417843in}{3.215780in}}{\pgfqpoint{2.407244in}{3.220171in}}{\pgfqpoint{2.396194in}{3.220171in}}%
\pgfpathcurveto{\pgfqpoint{2.385144in}{3.220171in}}{\pgfqpoint{2.374545in}{3.215780in}}{\pgfqpoint{2.366731in}{3.207967in}}%
\pgfpathcurveto{\pgfqpoint{2.358917in}{3.200153in}}{\pgfqpoint{2.354527in}{3.189554in}}{\pgfqpoint{2.354527in}{3.178504in}}%
\pgfpathcurveto{\pgfqpoint{2.354527in}{3.167454in}}{\pgfqpoint{2.358917in}{3.156855in}}{\pgfqpoint{2.366731in}{3.149041in}}%
\pgfpathcurveto{\pgfqpoint{2.374545in}{3.141228in}}{\pgfqpoint{2.385144in}{3.136837in}}{\pgfqpoint{2.396194in}{3.136837in}}%
\pgfpathclose%
\pgfusepath{stroke,fill}%
\end{pgfscope}%
\begin{pgfscope}%
\pgfpathrectangle{\pgfqpoint{0.648703in}{0.548769in}}{\pgfqpoint{5.201297in}{3.102590in}}%
\pgfusepath{clip}%
\pgfsetbuttcap%
\pgfsetroundjoin%
\definecolor{currentfill}{rgb}{1.000000,0.498039,0.054902}%
\pgfsetfillcolor{currentfill}%
\pgfsetlinewidth{1.003750pt}%
\definecolor{currentstroke}{rgb}{1.000000,0.498039,0.054902}%
\pgfsetstrokecolor{currentstroke}%
\pgfsetdash{}{0pt}%
\pgfpathmoveto{\pgfqpoint{1.770900in}{3.136837in}}%
\pgfpathcurveto{\pgfqpoint{1.781950in}{3.136837in}}{\pgfqpoint{1.792549in}{3.141228in}}{\pgfqpoint{1.800362in}{3.149041in}}%
\pgfpathcurveto{\pgfqpoint{1.808176in}{3.156855in}}{\pgfqpoint{1.812566in}{3.167454in}}{\pgfqpoint{1.812566in}{3.178504in}}%
\pgfpathcurveto{\pgfqpoint{1.812566in}{3.189554in}}{\pgfqpoint{1.808176in}{3.200153in}}{\pgfqpoint{1.800362in}{3.207967in}}%
\pgfpathcurveto{\pgfqpoint{1.792549in}{3.215780in}}{\pgfqpoint{1.781950in}{3.220171in}}{\pgfqpoint{1.770900in}{3.220171in}}%
\pgfpathcurveto{\pgfqpoint{1.759850in}{3.220171in}}{\pgfqpoint{1.749251in}{3.215780in}}{\pgfqpoint{1.741437in}{3.207967in}}%
\pgfpathcurveto{\pgfqpoint{1.733623in}{3.200153in}}{\pgfqpoint{1.729233in}{3.189554in}}{\pgfqpoint{1.729233in}{3.178504in}}%
\pgfpathcurveto{\pgfqpoint{1.729233in}{3.167454in}}{\pgfqpoint{1.733623in}{3.156855in}}{\pgfqpoint{1.741437in}{3.149041in}}%
\pgfpathcurveto{\pgfqpoint{1.749251in}{3.141228in}}{\pgfqpoint{1.759850in}{3.136837in}}{\pgfqpoint{1.770900in}{3.136837in}}%
\pgfpathclose%
\pgfusepath{stroke,fill}%
\end{pgfscope}%
\begin{pgfscope}%
\pgfpathrectangle{\pgfqpoint{0.648703in}{0.548769in}}{\pgfqpoint{5.201297in}{3.102590in}}%
\pgfusepath{clip}%
\pgfsetbuttcap%
\pgfsetroundjoin%
\definecolor{currentfill}{rgb}{1.000000,0.498039,0.054902}%
\pgfsetfillcolor{currentfill}%
\pgfsetlinewidth{1.003750pt}%
\definecolor{currentstroke}{rgb}{1.000000,0.498039,0.054902}%
\pgfsetstrokecolor{currentstroke}%
\pgfsetdash{}{0pt}%
\pgfpathmoveto{\pgfqpoint{2.117389in}{3.136837in}}%
\pgfpathcurveto{\pgfqpoint{2.128439in}{3.136837in}}{\pgfqpoint{2.139038in}{3.141228in}}{\pgfqpoint{2.146851in}{3.149041in}}%
\pgfpathcurveto{\pgfqpoint{2.154665in}{3.156855in}}{\pgfqpoint{2.159055in}{3.167454in}}{\pgfqpoint{2.159055in}{3.178504in}}%
\pgfpathcurveto{\pgfqpoint{2.159055in}{3.189554in}}{\pgfqpoint{2.154665in}{3.200153in}}{\pgfqpoint{2.146851in}{3.207967in}}%
\pgfpathcurveto{\pgfqpoint{2.139038in}{3.215780in}}{\pgfqpoint{2.128439in}{3.220171in}}{\pgfqpoint{2.117389in}{3.220171in}}%
\pgfpathcurveto{\pgfqpoint{2.106338in}{3.220171in}}{\pgfqpoint{2.095739in}{3.215780in}}{\pgfqpoint{2.087926in}{3.207967in}}%
\pgfpathcurveto{\pgfqpoint{2.080112in}{3.200153in}}{\pgfqpoint{2.075722in}{3.189554in}}{\pgfqpoint{2.075722in}{3.178504in}}%
\pgfpathcurveto{\pgfqpoint{2.075722in}{3.167454in}}{\pgfqpoint{2.080112in}{3.156855in}}{\pgfqpoint{2.087926in}{3.149041in}}%
\pgfpathcurveto{\pgfqpoint{2.095739in}{3.141228in}}{\pgfqpoint{2.106338in}{3.136837in}}{\pgfqpoint{2.117389in}{3.136837in}}%
\pgfpathclose%
\pgfusepath{stroke,fill}%
\end{pgfscope}%
\begin{pgfscope}%
\pgfpathrectangle{\pgfqpoint{0.648703in}{0.548769in}}{\pgfqpoint{5.201297in}{3.102590in}}%
\pgfusepath{clip}%
\pgfsetbuttcap%
\pgfsetroundjoin%
\definecolor{currentfill}{rgb}{0.121569,0.466667,0.705882}%
\pgfsetfillcolor{currentfill}%
\pgfsetlinewidth{1.003750pt}%
\definecolor{currentstroke}{rgb}{0.121569,0.466667,0.705882}%
\pgfsetstrokecolor{currentstroke}%
\pgfsetdash{}{0pt}%
\pgfpathmoveto{\pgfqpoint{2.049304in}{3.132690in}}%
\pgfpathcurveto{\pgfqpoint{2.060354in}{3.132690in}}{\pgfqpoint{2.070953in}{3.137080in}}{\pgfqpoint{2.078766in}{3.144893in}}%
\pgfpathcurveto{\pgfqpoint{2.086580in}{3.152707in}}{\pgfqpoint{2.090970in}{3.163306in}}{\pgfqpoint{2.090970in}{3.174356in}}%
\pgfpathcurveto{\pgfqpoint{2.090970in}{3.185406in}}{\pgfqpoint{2.086580in}{3.196005in}}{\pgfqpoint{2.078766in}{3.203819in}}%
\pgfpathcurveto{\pgfqpoint{2.070953in}{3.211633in}}{\pgfqpoint{2.060354in}{3.216023in}}{\pgfqpoint{2.049304in}{3.216023in}}%
\pgfpathcurveto{\pgfqpoint{2.038253in}{3.216023in}}{\pgfqpoint{2.027654in}{3.211633in}}{\pgfqpoint{2.019841in}{3.203819in}}%
\pgfpathcurveto{\pgfqpoint{2.012027in}{3.196005in}}{\pgfqpoint{2.007637in}{3.185406in}}{\pgfqpoint{2.007637in}{3.174356in}}%
\pgfpathcurveto{\pgfqpoint{2.007637in}{3.163306in}}{\pgfqpoint{2.012027in}{3.152707in}}{\pgfqpoint{2.019841in}{3.144893in}}%
\pgfpathcurveto{\pgfqpoint{2.027654in}{3.137080in}}{\pgfqpoint{2.038253in}{3.132690in}}{\pgfqpoint{2.049304in}{3.132690in}}%
\pgfpathclose%
\pgfusepath{stroke,fill}%
\end{pgfscope}%
\begin{pgfscope}%
\pgfpathrectangle{\pgfqpoint{0.648703in}{0.548769in}}{\pgfqpoint{5.201297in}{3.102590in}}%
\pgfusepath{clip}%
\pgfsetbuttcap%
\pgfsetroundjoin%
\definecolor{currentfill}{rgb}{0.121569,0.466667,0.705882}%
\pgfsetfillcolor{currentfill}%
\pgfsetlinewidth{1.003750pt}%
\definecolor{currentstroke}{rgb}{0.121569,0.466667,0.705882}%
\pgfsetstrokecolor{currentstroke}%
\pgfsetdash{}{0pt}%
\pgfpathmoveto{\pgfqpoint{4.035316in}{3.107802in}}%
\pgfpathcurveto{\pgfqpoint{4.046367in}{3.107802in}}{\pgfqpoint{4.056966in}{3.112193in}}{\pgfqpoint{4.064779in}{3.120006in}}%
\pgfpathcurveto{\pgfqpoint{4.072593in}{3.127820in}}{\pgfqpoint{4.076983in}{3.138419in}}{\pgfqpoint{4.076983in}{3.149469in}}%
\pgfpathcurveto{\pgfqpoint{4.076983in}{3.160519in}}{\pgfqpoint{4.072593in}{3.171118in}}{\pgfqpoint{4.064779in}{3.178932in}}%
\pgfpathcurveto{\pgfqpoint{4.056966in}{3.186745in}}{\pgfqpoint{4.046367in}{3.191136in}}{\pgfqpoint{4.035316in}{3.191136in}}%
\pgfpathcurveto{\pgfqpoint{4.024266in}{3.191136in}}{\pgfqpoint{4.013667in}{3.186745in}}{\pgfqpoint{4.005854in}{3.178932in}}%
\pgfpathcurveto{\pgfqpoint{3.998040in}{3.171118in}}{\pgfqpoint{3.993650in}{3.160519in}}{\pgfqpoint{3.993650in}{3.149469in}}%
\pgfpathcurveto{\pgfqpoint{3.993650in}{3.138419in}}{\pgfqpoint{3.998040in}{3.127820in}}{\pgfqpoint{4.005854in}{3.120006in}}%
\pgfpathcurveto{\pgfqpoint{4.013667in}{3.112193in}}{\pgfqpoint{4.024266in}{3.107802in}}{\pgfqpoint{4.035316in}{3.107802in}}%
\pgfpathclose%
\pgfusepath{stroke,fill}%
\end{pgfscope}%
\begin{pgfscope}%
\pgfpathrectangle{\pgfqpoint{0.648703in}{0.548769in}}{\pgfqpoint{5.201297in}{3.102590in}}%
\pgfusepath{clip}%
\pgfsetbuttcap%
\pgfsetroundjoin%
\definecolor{currentfill}{rgb}{0.121569,0.466667,0.705882}%
\pgfsetfillcolor{currentfill}%
\pgfsetlinewidth{1.003750pt}%
\definecolor{currentstroke}{rgb}{0.121569,0.466667,0.705882}%
\pgfsetstrokecolor{currentstroke}%
\pgfsetdash{}{0pt}%
\pgfpathmoveto{\pgfqpoint{1.904751in}{3.132690in}}%
\pgfpathcurveto{\pgfqpoint{1.915801in}{3.132690in}}{\pgfqpoint{1.926400in}{3.137080in}}{\pgfqpoint{1.934214in}{3.144893in}}%
\pgfpathcurveto{\pgfqpoint{1.942028in}{3.152707in}}{\pgfqpoint{1.946418in}{3.163306in}}{\pgfqpoint{1.946418in}{3.174356in}}%
\pgfpathcurveto{\pgfqpoint{1.946418in}{3.185406in}}{\pgfqpoint{1.942028in}{3.196005in}}{\pgfqpoint{1.934214in}{3.203819in}}%
\pgfpathcurveto{\pgfqpoint{1.926400in}{3.211633in}}{\pgfqpoint{1.915801in}{3.216023in}}{\pgfqpoint{1.904751in}{3.216023in}}%
\pgfpathcurveto{\pgfqpoint{1.893701in}{3.216023in}}{\pgfqpoint{1.883102in}{3.211633in}}{\pgfqpoint{1.875288in}{3.203819in}}%
\pgfpathcurveto{\pgfqpoint{1.867475in}{3.196005in}}{\pgfqpoint{1.863084in}{3.185406in}}{\pgfqpoint{1.863084in}{3.174356in}}%
\pgfpathcurveto{\pgfqpoint{1.863084in}{3.163306in}}{\pgfqpoint{1.867475in}{3.152707in}}{\pgfqpoint{1.875288in}{3.144893in}}%
\pgfpathcurveto{\pgfqpoint{1.883102in}{3.137080in}}{\pgfqpoint{1.893701in}{3.132690in}}{\pgfqpoint{1.904751in}{3.132690in}}%
\pgfpathclose%
\pgfusepath{stroke,fill}%
\end{pgfscope}%
\begin{pgfscope}%
\pgfpathrectangle{\pgfqpoint{0.648703in}{0.548769in}}{\pgfqpoint{5.201297in}{3.102590in}}%
\pgfusepath{clip}%
\pgfsetbuttcap%
\pgfsetroundjoin%
\definecolor{currentfill}{rgb}{0.121569,0.466667,0.705882}%
\pgfsetfillcolor{currentfill}%
\pgfsetlinewidth{1.003750pt}%
\definecolor{currentstroke}{rgb}{0.121569,0.466667,0.705882}%
\pgfsetstrokecolor{currentstroke}%
\pgfsetdash{}{0pt}%
\pgfpathmoveto{\pgfqpoint{1.022455in}{0.648129in}}%
\pgfpathcurveto{\pgfqpoint{1.033505in}{0.648129in}}{\pgfqpoint{1.044104in}{0.652519in}}{\pgfqpoint{1.051918in}{0.660333in}}%
\pgfpathcurveto{\pgfqpoint{1.059732in}{0.668146in}}{\pgfqpoint{1.064122in}{0.678745in}}{\pgfqpoint{1.064122in}{0.689796in}}%
\pgfpathcurveto{\pgfqpoint{1.064122in}{0.700846in}}{\pgfqpoint{1.059732in}{0.711445in}}{\pgfqpoint{1.051918in}{0.719258in}}%
\pgfpathcurveto{\pgfqpoint{1.044104in}{0.727072in}}{\pgfqpoint{1.033505in}{0.731462in}}{\pgfqpoint{1.022455in}{0.731462in}}%
\pgfpathcurveto{\pgfqpoint{1.011405in}{0.731462in}}{\pgfqpoint{1.000806in}{0.727072in}}{\pgfqpoint{0.992992in}{0.719258in}}%
\pgfpathcurveto{\pgfqpoint{0.985179in}{0.711445in}}{\pgfqpoint{0.980789in}{0.700846in}}{\pgfqpoint{0.980789in}{0.689796in}}%
\pgfpathcurveto{\pgfqpoint{0.980789in}{0.678745in}}{\pgfqpoint{0.985179in}{0.668146in}}{\pgfqpoint{0.992992in}{0.660333in}}%
\pgfpathcurveto{\pgfqpoint{1.000806in}{0.652519in}}{\pgfqpoint{1.011405in}{0.648129in}}{\pgfqpoint{1.022455in}{0.648129in}}%
\pgfpathclose%
\pgfusepath{stroke,fill}%
\end{pgfscope}%
\begin{pgfscope}%
\pgfpathrectangle{\pgfqpoint{0.648703in}{0.548769in}}{\pgfqpoint{5.201297in}{3.102590in}}%
\pgfusepath{clip}%
\pgfsetbuttcap%
\pgfsetroundjoin%
\definecolor{currentfill}{rgb}{1.000000,0.498039,0.054902}%
\pgfsetfillcolor{currentfill}%
\pgfsetlinewidth{1.003750pt}%
\definecolor{currentstroke}{rgb}{1.000000,0.498039,0.054902}%
\pgfsetstrokecolor{currentstroke}%
\pgfsetdash{}{0pt}%
\pgfpathmoveto{\pgfqpoint{2.235545in}{3.140985in}}%
\pgfpathcurveto{\pgfqpoint{2.246595in}{3.140985in}}{\pgfqpoint{2.257194in}{3.145375in}}{\pgfqpoint{2.265008in}{3.153189in}}%
\pgfpathcurveto{\pgfqpoint{2.272822in}{3.161003in}}{\pgfqpoint{2.277212in}{3.171602in}}{\pgfqpoint{2.277212in}{3.182652in}}%
\pgfpathcurveto{\pgfqpoint{2.277212in}{3.193702in}}{\pgfqpoint{2.272822in}{3.204301in}}{\pgfqpoint{2.265008in}{3.212115in}}%
\pgfpathcurveto{\pgfqpoint{2.257194in}{3.219928in}}{\pgfqpoint{2.246595in}{3.224319in}}{\pgfqpoint{2.235545in}{3.224319in}}%
\pgfpathcurveto{\pgfqpoint{2.224495in}{3.224319in}}{\pgfqpoint{2.213896in}{3.219928in}}{\pgfqpoint{2.206082in}{3.212115in}}%
\pgfpathcurveto{\pgfqpoint{2.198269in}{3.204301in}}{\pgfqpoint{2.193879in}{3.193702in}}{\pgfqpoint{2.193879in}{3.182652in}}%
\pgfpathcurveto{\pgfqpoint{2.193879in}{3.171602in}}{\pgfqpoint{2.198269in}{3.161003in}}{\pgfqpoint{2.206082in}{3.153189in}}%
\pgfpathcurveto{\pgfqpoint{2.213896in}{3.145375in}}{\pgfqpoint{2.224495in}{3.140985in}}{\pgfqpoint{2.235545in}{3.140985in}}%
\pgfpathclose%
\pgfusepath{stroke,fill}%
\end{pgfscope}%
\begin{pgfscope}%
\pgfpathrectangle{\pgfqpoint{0.648703in}{0.548769in}}{\pgfqpoint{5.201297in}{3.102590in}}%
\pgfusepath{clip}%
\pgfsetbuttcap%
\pgfsetroundjoin%
\definecolor{currentfill}{rgb}{1.000000,0.498039,0.054902}%
\pgfsetfillcolor{currentfill}%
\pgfsetlinewidth{1.003750pt}%
\definecolor{currentstroke}{rgb}{1.000000,0.498039,0.054902}%
\pgfsetstrokecolor{currentstroke}%
\pgfsetdash{}{0pt}%
\pgfpathmoveto{\pgfqpoint{1.860119in}{3.145133in}}%
\pgfpathcurveto{\pgfqpoint{1.871169in}{3.145133in}}{\pgfqpoint{1.881768in}{3.149523in}}{\pgfqpoint{1.889582in}{3.157337in}}%
\pgfpathcurveto{\pgfqpoint{1.897396in}{3.165151in}}{\pgfqpoint{1.901786in}{3.175750in}}{\pgfqpoint{1.901786in}{3.186800in}}%
\pgfpathcurveto{\pgfqpoint{1.901786in}{3.197850in}}{\pgfqpoint{1.897396in}{3.208449in}}{\pgfqpoint{1.889582in}{3.216262in}}%
\pgfpathcurveto{\pgfqpoint{1.881768in}{3.224076in}}{\pgfqpoint{1.871169in}{3.228466in}}{\pgfqpoint{1.860119in}{3.228466in}}%
\pgfpathcurveto{\pgfqpoint{1.849069in}{3.228466in}}{\pgfqpoint{1.838470in}{3.224076in}}{\pgfqpoint{1.830656in}{3.216262in}}%
\pgfpathcurveto{\pgfqpoint{1.822843in}{3.208449in}}{\pgfqpoint{1.818452in}{3.197850in}}{\pgfqpoint{1.818452in}{3.186800in}}%
\pgfpathcurveto{\pgfqpoint{1.818452in}{3.175750in}}{\pgfqpoint{1.822843in}{3.165151in}}{\pgfqpoint{1.830656in}{3.157337in}}%
\pgfpathcurveto{\pgfqpoint{1.838470in}{3.149523in}}{\pgfqpoint{1.849069in}{3.145133in}}{\pgfqpoint{1.860119in}{3.145133in}}%
\pgfpathclose%
\pgfusepath{stroke,fill}%
\end{pgfscope}%
\begin{pgfscope}%
\pgfpathrectangle{\pgfqpoint{0.648703in}{0.548769in}}{\pgfqpoint{5.201297in}{3.102590in}}%
\pgfusepath{clip}%
\pgfsetbuttcap%
\pgfsetroundjoin%
\definecolor{currentfill}{rgb}{1.000000,0.498039,0.054902}%
\pgfsetfillcolor{currentfill}%
\pgfsetlinewidth{1.003750pt}%
\definecolor{currentstroke}{rgb}{1.000000,0.498039,0.054902}%
\pgfsetstrokecolor{currentstroke}%
\pgfsetdash{}{0pt}%
\pgfpathmoveto{\pgfqpoint{1.374651in}{3.140985in}}%
\pgfpathcurveto{\pgfqpoint{1.385701in}{3.140985in}}{\pgfqpoint{1.396300in}{3.145375in}}{\pgfqpoint{1.404114in}{3.153189in}}%
\pgfpathcurveto{\pgfqpoint{1.411928in}{3.161003in}}{\pgfqpoint{1.416318in}{3.171602in}}{\pgfqpoint{1.416318in}{3.182652in}}%
\pgfpathcurveto{\pgfqpoint{1.416318in}{3.193702in}}{\pgfqpoint{1.411928in}{3.204301in}}{\pgfqpoint{1.404114in}{3.212115in}}%
\pgfpathcurveto{\pgfqpoint{1.396300in}{3.219928in}}{\pgfqpoint{1.385701in}{3.224319in}}{\pgfqpoint{1.374651in}{3.224319in}}%
\pgfpathcurveto{\pgfqpoint{1.363601in}{3.224319in}}{\pgfqpoint{1.353002in}{3.219928in}}{\pgfqpoint{1.345189in}{3.212115in}}%
\pgfpathcurveto{\pgfqpoint{1.337375in}{3.204301in}}{\pgfqpoint{1.332985in}{3.193702in}}{\pgfqpoint{1.332985in}{3.182652in}}%
\pgfpathcurveto{\pgfqpoint{1.332985in}{3.171602in}}{\pgfqpoint{1.337375in}{3.161003in}}{\pgfqpoint{1.345189in}{3.153189in}}%
\pgfpathcurveto{\pgfqpoint{1.353002in}{3.145375in}}{\pgfqpoint{1.363601in}{3.140985in}}{\pgfqpoint{1.374651in}{3.140985in}}%
\pgfpathclose%
\pgfusepath{stroke,fill}%
\end{pgfscope}%
\begin{pgfscope}%
\pgfpathrectangle{\pgfqpoint{0.648703in}{0.548769in}}{\pgfqpoint{5.201297in}{3.102590in}}%
\pgfusepath{clip}%
\pgfsetbuttcap%
\pgfsetroundjoin%
\definecolor{currentfill}{rgb}{0.121569,0.466667,0.705882}%
\pgfsetfillcolor{currentfill}%
\pgfsetlinewidth{1.003750pt}%
\definecolor{currentstroke}{rgb}{0.121569,0.466667,0.705882}%
\pgfsetstrokecolor{currentstroke}%
\pgfsetdash{}{0pt}%
\pgfpathmoveto{\pgfqpoint{1.158224in}{0.697903in}}%
\pgfpathcurveto{\pgfqpoint{1.169274in}{0.697903in}}{\pgfqpoint{1.179873in}{0.702293in}}{\pgfqpoint{1.187687in}{0.710107in}}%
\pgfpathcurveto{\pgfqpoint{1.195500in}{0.717921in}}{\pgfqpoint{1.199891in}{0.728520in}}{\pgfqpoint{1.199891in}{0.739570in}}%
\pgfpathcurveto{\pgfqpoint{1.199891in}{0.750620in}}{\pgfqpoint{1.195500in}{0.761219in}}{\pgfqpoint{1.187687in}{0.769033in}}%
\pgfpathcurveto{\pgfqpoint{1.179873in}{0.776846in}}{\pgfqpoint{1.169274in}{0.781236in}}{\pgfqpoint{1.158224in}{0.781236in}}%
\pgfpathcurveto{\pgfqpoint{1.147174in}{0.781236in}}{\pgfqpoint{1.136575in}{0.776846in}}{\pgfqpoint{1.128761in}{0.769033in}}%
\pgfpathcurveto{\pgfqpoint{1.120948in}{0.761219in}}{\pgfqpoint{1.116557in}{0.750620in}}{\pgfqpoint{1.116557in}{0.739570in}}%
\pgfpathcurveto{\pgfqpoint{1.116557in}{0.728520in}}{\pgfqpoint{1.120948in}{0.717921in}}{\pgfqpoint{1.128761in}{0.710107in}}%
\pgfpathcurveto{\pgfqpoint{1.136575in}{0.702293in}}{\pgfqpoint{1.147174in}{0.697903in}}{\pgfqpoint{1.158224in}{0.697903in}}%
\pgfpathclose%
\pgfusepath{stroke,fill}%
\end{pgfscope}%
\begin{pgfscope}%
\pgfpathrectangle{\pgfqpoint{0.648703in}{0.548769in}}{\pgfqpoint{5.201297in}{3.102590in}}%
\pgfusepath{clip}%
\pgfsetbuttcap%
\pgfsetroundjoin%
\definecolor{currentfill}{rgb}{1.000000,0.498039,0.054902}%
\pgfsetfillcolor{currentfill}%
\pgfsetlinewidth{1.003750pt}%
\definecolor{currentstroke}{rgb}{1.000000,0.498039,0.054902}%
\pgfsetstrokecolor{currentstroke}%
\pgfsetdash{}{0pt}%
\pgfpathmoveto{\pgfqpoint{2.092999in}{3.140985in}}%
\pgfpathcurveto{\pgfqpoint{2.104049in}{3.140985in}}{\pgfqpoint{2.114648in}{3.145375in}}{\pgfqpoint{2.122462in}{3.153189in}}%
\pgfpathcurveto{\pgfqpoint{2.130276in}{3.161003in}}{\pgfqpoint{2.134666in}{3.171602in}}{\pgfqpoint{2.134666in}{3.182652in}}%
\pgfpathcurveto{\pgfqpoint{2.134666in}{3.193702in}}{\pgfqpoint{2.130276in}{3.204301in}}{\pgfqpoint{2.122462in}{3.212115in}}%
\pgfpathcurveto{\pgfqpoint{2.114648in}{3.219928in}}{\pgfqpoint{2.104049in}{3.224319in}}{\pgfqpoint{2.092999in}{3.224319in}}%
\pgfpathcurveto{\pgfqpoint{2.081949in}{3.224319in}}{\pgfqpoint{2.071350in}{3.219928in}}{\pgfqpoint{2.063536in}{3.212115in}}%
\pgfpathcurveto{\pgfqpoint{2.055723in}{3.204301in}}{\pgfqpoint{2.051333in}{3.193702in}}{\pgfqpoint{2.051333in}{3.182652in}}%
\pgfpathcurveto{\pgfqpoint{2.051333in}{3.171602in}}{\pgfqpoint{2.055723in}{3.161003in}}{\pgfqpoint{2.063536in}{3.153189in}}%
\pgfpathcurveto{\pgfqpoint{2.071350in}{3.145375in}}{\pgfqpoint{2.081949in}{3.140985in}}{\pgfqpoint{2.092999in}{3.140985in}}%
\pgfpathclose%
\pgfusepath{stroke,fill}%
\end{pgfscope}%
\begin{pgfscope}%
\pgfpathrectangle{\pgfqpoint{0.648703in}{0.548769in}}{\pgfqpoint{5.201297in}{3.102590in}}%
\pgfusepath{clip}%
\pgfsetbuttcap%
\pgfsetroundjoin%
\definecolor{currentfill}{rgb}{0.121569,0.466667,0.705882}%
\pgfsetfillcolor{currentfill}%
\pgfsetlinewidth{1.003750pt}%
\definecolor{currentstroke}{rgb}{0.121569,0.466667,0.705882}%
\pgfsetstrokecolor{currentstroke}%
\pgfsetdash{}{0pt}%
\pgfpathmoveto{\pgfqpoint{1.972702in}{3.132690in}}%
\pgfpathcurveto{\pgfqpoint{1.983752in}{3.132690in}}{\pgfqpoint{1.994352in}{3.137080in}}{\pgfqpoint{2.002165in}{3.144893in}}%
\pgfpathcurveto{\pgfqpoint{2.009979in}{3.152707in}}{\pgfqpoint{2.014369in}{3.163306in}}{\pgfqpoint{2.014369in}{3.174356in}}%
\pgfpathcurveto{\pgfqpoint{2.014369in}{3.185406in}}{\pgfqpoint{2.009979in}{3.196005in}}{\pgfqpoint{2.002165in}{3.203819in}}%
\pgfpathcurveto{\pgfqpoint{1.994352in}{3.211633in}}{\pgfqpoint{1.983752in}{3.216023in}}{\pgfqpoint{1.972702in}{3.216023in}}%
\pgfpathcurveto{\pgfqpoint{1.961652in}{3.216023in}}{\pgfqpoint{1.951053in}{3.211633in}}{\pgfqpoint{1.943240in}{3.203819in}}%
\pgfpathcurveto{\pgfqpoint{1.935426in}{3.196005in}}{\pgfqpoint{1.931036in}{3.185406in}}{\pgfqpoint{1.931036in}{3.174356in}}%
\pgfpathcurveto{\pgfqpoint{1.931036in}{3.163306in}}{\pgfqpoint{1.935426in}{3.152707in}}{\pgfqpoint{1.943240in}{3.144893in}}%
\pgfpathcurveto{\pgfqpoint{1.951053in}{3.137080in}}{\pgfqpoint{1.961652in}{3.132690in}}{\pgfqpoint{1.972702in}{3.132690in}}%
\pgfpathclose%
\pgfusepath{stroke,fill}%
\end{pgfscope}%
\begin{pgfscope}%
\pgfpathrectangle{\pgfqpoint{0.648703in}{0.548769in}}{\pgfqpoint{5.201297in}{3.102590in}}%
\pgfusepath{clip}%
\pgfsetbuttcap%
\pgfsetroundjoin%
\definecolor{currentfill}{rgb}{1.000000,0.498039,0.054902}%
\pgfsetfillcolor{currentfill}%
\pgfsetlinewidth{1.003750pt}%
\definecolor{currentstroke}{rgb}{1.000000,0.498039,0.054902}%
\pgfsetstrokecolor{currentstroke}%
\pgfsetdash{}{0pt}%
\pgfpathmoveto{\pgfqpoint{2.803455in}{3.140985in}}%
\pgfpathcurveto{\pgfqpoint{2.814505in}{3.140985in}}{\pgfqpoint{2.825104in}{3.145375in}}{\pgfqpoint{2.832918in}{3.153189in}}%
\pgfpathcurveto{\pgfqpoint{2.840732in}{3.161003in}}{\pgfqpoint{2.845122in}{3.171602in}}{\pgfqpoint{2.845122in}{3.182652in}}%
\pgfpathcurveto{\pgfqpoint{2.845122in}{3.193702in}}{\pgfqpoint{2.840732in}{3.204301in}}{\pgfqpoint{2.832918in}{3.212115in}}%
\pgfpathcurveto{\pgfqpoint{2.825104in}{3.219928in}}{\pgfqpoint{2.814505in}{3.224319in}}{\pgfqpoint{2.803455in}{3.224319in}}%
\pgfpathcurveto{\pgfqpoint{2.792405in}{3.224319in}}{\pgfqpoint{2.781806in}{3.219928in}}{\pgfqpoint{2.773992in}{3.212115in}}%
\pgfpathcurveto{\pgfqpoint{2.766179in}{3.204301in}}{\pgfqpoint{2.761789in}{3.193702in}}{\pgfqpoint{2.761789in}{3.182652in}}%
\pgfpathcurveto{\pgfqpoint{2.761789in}{3.171602in}}{\pgfqpoint{2.766179in}{3.161003in}}{\pgfqpoint{2.773992in}{3.153189in}}%
\pgfpathcurveto{\pgfqpoint{2.781806in}{3.145375in}}{\pgfqpoint{2.792405in}{3.140985in}}{\pgfqpoint{2.803455in}{3.140985in}}%
\pgfpathclose%
\pgfusepath{stroke,fill}%
\end{pgfscope}%
\begin{pgfscope}%
\pgfpathrectangle{\pgfqpoint{0.648703in}{0.548769in}}{\pgfqpoint{5.201297in}{3.102590in}}%
\pgfusepath{clip}%
\pgfsetbuttcap%
\pgfsetroundjoin%
\definecolor{currentfill}{rgb}{0.121569,0.466667,0.705882}%
\pgfsetfillcolor{currentfill}%
\pgfsetlinewidth{1.003750pt}%
\definecolor{currentstroke}{rgb}{0.121569,0.466667,0.705882}%
\pgfsetstrokecolor{currentstroke}%
\pgfsetdash{}{0pt}%
\pgfpathmoveto{\pgfqpoint{1.990716in}{3.132690in}}%
\pgfpathcurveto{\pgfqpoint{2.001766in}{3.132690in}}{\pgfqpoint{2.012365in}{3.137080in}}{\pgfqpoint{2.020178in}{3.144893in}}%
\pgfpathcurveto{\pgfqpoint{2.027992in}{3.152707in}}{\pgfqpoint{2.032382in}{3.163306in}}{\pgfqpoint{2.032382in}{3.174356in}}%
\pgfpathcurveto{\pgfqpoint{2.032382in}{3.185406in}}{\pgfqpoint{2.027992in}{3.196005in}}{\pgfqpoint{2.020178in}{3.203819in}}%
\pgfpathcurveto{\pgfqpoint{2.012365in}{3.211633in}}{\pgfqpoint{2.001766in}{3.216023in}}{\pgfqpoint{1.990716in}{3.216023in}}%
\pgfpathcurveto{\pgfqpoint{1.979666in}{3.216023in}}{\pgfqpoint{1.969067in}{3.211633in}}{\pgfqpoint{1.961253in}{3.203819in}}%
\pgfpathcurveto{\pgfqpoint{1.953439in}{3.196005in}}{\pgfqpoint{1.949049in}{3.185406in}}{\pgfqpoint{1.949049in}{3.174356in}}%
\pgfpathcurveto{\pgfqpoint{1.949049in}{3.163306in}}{\pgfqpoint{1.953439in}{3.152707in}}{\pgfqpoint{1.961253in}{3.144893in}}%
\pgfpathcurveto{\pgfqpoint{1.969067in}{3.137080in}}{\pgfqpoint{1.979666in}{3.132690in}}{\pgfqpoint{1.990716in}{3.132690in}}%
\pgfpathclose%
\pgfusepath{stroke,fill}%
\end{pgfscope}%
\begin{pgfscope}%
\pgfpathrectangle{\pgfqpoint{0.648703in}{0.548769in}}{\pgfqpoint{5.201297in}{3.102590in}}%
\pgfusepath{clip}%
\pgfsetbuttcap%
\pgfsetroundjoin%
\definecolor{currentfill}{rgb}{1.000000,0.498039,0.054902}%
\pgfsetfillcolor{currentfill}%
\pgfsetlinewidth{1.003750pt}%
\definecolor{currentstroke}{rgb}{1.000000,0.498039,0.054902}%
\pgfsetstrokecolor{currentstroke}%
\pgfsetdash{}{0pt}%
\pgfpathmoveto{\pgfqpoint{1.997983in}{3.140985in}}%
\pgfpathcurveto{\pgfqpoint{2.009034in}{3.140985in}}{\pgfqpoint{2.019633in}{3.145375in}}{\pgfqpoint{2.027446in}{3.153189in}}%
\pgfpathcurveto{\pgfqpoint{2.035260in}{3.161003in}}{\pgfqpoint{2.039650in}{3.171602in}}{\pgfqpoint{2.039650in}{3.182652in}}%
\pgfpathcurveto{\pgfqpoint{2.039650in}{3.193702in}}{\pgfqpoint{2.035260in}{3.204301in}}{\pgfqpoint{2.027446in}{3.212115in}}%
\pgfpathcurveto{\pgfqpoint{2.019633in}{3.219928in}}{\pgfqpoint{2.009034in}{3.224319in}}{\pgfqpoint{1.997983in}{3.224319in}}%
\pgfpathcurveto{\pgfqpoint{1.986933in}{3.224319in}}{\pgfqpoint{1.976334in}{3.219928in}}{\pgfqpoint{1.968521in}{3.212115in}}%
\pgfpathcurveto{\pgfqpoint{1.960707in}{3.204301in}}{\pgfqpoint{1.956317in}{3.193702in}}{\pgfqpoint{1.956317in}{3.182652in}}%
\pgfpathcurveto{\pgfqpoint{1.956317in}{3.171602in}}{\pgfqpoint{1.960707in}{3.161003in}}{\pgfqpoint{1.968521in}{3.153189in}}%
\pgfpathcurveto{\pgfqpoint{1.976334in}{3.145375in}}{\pgfqpoint{1.986933in}{3.140985in}}{\pgfqpoint{1.997983in}{3.140985in}}%
\pgfpathclose%
\pgfusepath{stroke,fill}%
\end{pgfscope}%
\begin{pgfscope}%
\pgfpathrectangle{\pgfqpoint{0.648703in}{0.548769in}}{\pgfqpoint{5.201297in}{3.102590in}}%
\pgfusepath{clip}%
\pgfsetbuttcap%
\pgfsetroundjoin%
\definecolor{currentfill}{rgb}{1.000000,0.498039,0.054902}%
\pgfsetfillcolor{currentfill}%
\pgfsetlinewidth{1.003750pt}%
\definecolor{currentstroke}{rgb}{1.000000,0.498039,0.054902}%
\pgfsetstrokecolor{currentstroke}%
\pgfsetdash{}{0pt}%
\pgfpathmoveto{\pgfqpoint{1.500745in}{3.145133in}}%
\pgfpathcurveto{\pgfqpoint{1.511795in}{3.145133in}}{\pgfqpoint{1.522394in}{3.149523in}}{\pgfqpoint{1.530207in}{3.157337in}}%
\pgfpathcurveto{\pgfqpoint{1.538021in}{3.165151in}}{\pgfqpoint{1.542411in}{3.175750in}}{\pgfqpoint{1.542411in}{3.186800in}}%
\pgfpathcurveto{\pgfqpoint{1.542411in}{3.197850in}}{\pgfqpoint{1.538021in}{3.208449in}}{\pgfqpoint{1.530207in}{3.216262in}}%
\pgfpathcurveto{\pgfqpoint{1.522394in}{3.224076in}}{\pgfqpoint{1.511795in}{3.228466in}}{\pgfqpoint{1.500745in}{3.228466in}}%
\pgfpathcurveto{\pgfqpoint{1.489694in}{3.228466in}}{\pgfqpoint{1.479095in}{3.224076in}}{\pgfqpoint{1.471282in}{3.216262in}}%
\pgfpathcurveto{\pgfqpoint{1.463468in}{3.208449in}}{\pgfqpoint{1.459078in}{3.197850in}}{\pgfqpoint{1.459078in}{3.186800in}}%
\pgfpathcurveto{\pgfqpoint{1.459078in}{3.175750in}}{\pgfqpoint{1.463468in}{3.165151in}}{\pgfqpoint{1.471282in}{3.157337in}}%
\pgfpathcurveto{\pgfqpoint{1.479095in}{3.149523in}}{\pgfqpoint{1.489694in}{3.145133in}}{\pgfqpoint{1.500745in}{3.145133in}}%
\pgfpathclose%
\pgfusepath{stroke,fill}%
\end{pgfscope}%
\begin{pgfscope}%
\pgfpathrectangle{\pgfqpoint{0.648703in}{0.548769in}}{\pgfqpoint{5.201297in}{3.102590in}}%
\pgfusepath{clip}%
\pgfsetbuttcap%
\pgfsetroundjoin%
\definecolor{currentfill}{rgb}{1.000000,0.498039,0.054902}%
\pgfsetfillcolor{currentfill}%
\pgfsetlinewidth{1.003750pt}%
\definecolor{currentstroke}{rgb}{1.000000,0.498039,0.054902}%
\pgfsetstrokecolor{currentstroke}%
\pgfsetdash{}{0pt}%
\pgfpathmoveto{\pgfqpoint{1.876438in}{3.145133in}}%
\pgfpathcurveto{\pgfqpoint{1.887488in}{3.145133in}}{\pgfqpoint{1.898087in}{3.149523in}}{\pgfqpoint{1.905901in}{3.157337in}}%
\pgfpathcurveto{\pgfqpoint{1.913715in}{3.165151in}}{\pgfqpoint{1.918105in}{3.175750in}}{\pgfqpoint{1.918105in}{3.186800in}}%
\pgfpathcurveto{\pgfqpoint{1.918105in}{3.197850in}}{\pgfqpoint{1.913715in}{3.208449in}}{\pgfqpoint{1.905901in}{3.216262in}}%
\pgfpathcurveto{\pgfqpoint{1.898087in}{3.224076in}}{\pgfqpoint{1.887488in}{3.228466in}}{\pgfqpoint{1.876438in}{3.228466in}}%
\pgfpathcurveto{\pgfqpoint{1.865388in}{3.228466in}}{\pgfqpoint{1.854789in}{3.224076in}}{\pgfqpoint{1.846975in}{3.216262in}}%
\pgfpathcurveto{\pgfqpoint{1.839162in}{3.208449in}}{\pgfqpoint{1.834771in}{3.197850in}}{\pgfqpoint{1.834771in}{3.186800in}}%
\pgfpathcurveto{\pgfqpoint{1.834771in}{3.175750in}}{\pgfqpoint{1.839162in}{3.165151in}}{\pgfqpoint{1.846975in}{3.157337in}}%
\pgfpathcurveto{\pgfqpoint{1.854789in}{3.149523in}}{\pgfqpoint{1.865388in}{3.145133in}}{\pgfqpoint{1.876438in}{3.145133in}}%
\pgfpathclose%
\pgfusepath{stroke,fill}%
\end{pgfscope}%
\begin{pgfscope}%
\pgfpathrectangle{\pgfqpoint{0.648703in}{0.548769in}}{\pgfqpoint{5.201297in}{3.102590in}}%
\pgfusepath{clip}%
\pgfsetbuttcap%
\pgfsetroundjoin%
\definecolor{currentfill}{rgb}{1.000000,0.498039,0.054902}%
\pgfsetfillcolor{currentfill}%
\pgfsetlinewidth{1.003750pt}%
\definecolor{currentstroke}{rgb}{1.000000,0.498039,0.054902}%
\pgfsetstrokecolor{currentstroke}%
\pgfsetdash{}{0pt}%
\pgfpathmoveto{\pgfqpoint{2.381881in}{3.136837in}}%
\pgfpathcurveto{\pgfqpoint{2.392931in}{3.136837in}}{\pgfqpoint{2.403530in}{3.141228in}}{\pgfqpoint{2.411344in}{3.149041in}}%
\pgfpathcurveto{\pgfqpoint{2.419158in}{3.156855in}}{\pgfqpoint{2.423548in}{3.167454in}}{\pgfqpoint{2.423548in}{3.178504in}}%
\pgfpathcurveto{\pgfqpoint{2.423548in}{3.189554in}}{\pgfqpoint{2.419158in}{3.200153in}}{\pgfqpoint{2.411344in}{3.207967in}}%
\pgfpathcurveto{\pgfqpoint{2.403530in}{3.215780in}}{\pgfqpoint{2.392931in}{3.220171in}}{\pgfqpoint{2.381881in}{3.220171in}}%
\pgfpathcurveto{\pgfqpoint{2.370831in}{3.220171in}}{\pgfqpoint{2.360232in}{3.215780in}}{\pgfqpoint{2.352418in}{3.207967in}}%
\pgfpathcurveto{\pgfqpoint{2.344605in}{3.200153in}}{\pgfqpoint{2.340214in}{3.189554in}}{\pgfqpoint{2.340214in}{3.178504in}}%
\pgfpathcurveto{\pgfqpoint{2.340214in}{3.167454in}}{\pgfqpoint{2.344605in}{3.156855in}}{\pgfqpoint{2.352418in}{3.149041in}}%
\pgfpathcurveto{\pgfqpoint{2.360232in}{3.141228in}}{\pgfqpoint{2.370831in}{3.136837in}}{\pgfqpoint{2.381881in}{3.136837in}}%
\pgfpathclose%
\pgfusepath{stroke,fill}%
\end{pgfscope}%
\begin{pgfscope}%
\pgfpathrectangle{\pgfqpoint{0.648703in}{0.548769in}}{\pgfqpoint{5.201297in}{3.102590in}}%
\pgfusepath{clip}%
\pgfsetbuttcap%
\pgfsetroundjoin%
\definecolor{currentfill}{rgb}{1.000000,0.498039,0.054902}%
\pgfsetfillcolor{currentfill}%
\pgfsetlinewidth{1.003750pt}%
\definecolor{currentstroke}{rgb}{1.000000,0.498039,0.054902}%
\pgfsetstrokecolor{currentstroke}%
\pgfsetdash{}{0pt}%
\pgfpathmoveto{\pgfqpoint{1.686273in}{3.468665in}}%
\pgfpathcurveto{\pgfqpoint{1.697323in}{3.468665in}}{\pgfqpoint{1.707922in}{3.473055in}}{\pgfqpoint{1.715736in}{3.480869in}}%
\pgfpathcurveto{\pgfqpoint{1.723549in}{3.488683in}}{\pgfqpoint{1.727939in}{3.499282in}}{\pgfqpoint{1.727939in}{3.510332in}}%
\pgfpathcurveto{\pgfqpoint{1.727939in}{3.521382in}}{\pgfqpoint{1.723549in}{3.531981in}}{\pgfqpoint{1.715736in}{3.539795in}}%
\pgfpathcurveto{\pgfqpoint{1.707922in}{3.547608in}}{\pgfqpoint{1.697323in}{3.551998in}}{\pgfqpoint{1.686273in}{3.551998in}}%
\pgfpathcurveto{\pgfqpoint{1.675223in}{3.551998in}}{\pgfqpoint{1.664624in}{3.547608in}}{\pgfqpoint{1.656810in}{3.539795in}}%
\pgfpathcurveto{\pgfqpoint{1.648996in}{3.531981in}}{\pgfqpoint{1.644606in}{3.521382in}}{\pgfqpoint{1.644606in}{3.510332in}}%
\pgfpathcurveto{\pgfqpoint{1.644606in}{3.499282in}}{\pgfqpoint{1.648996in}{3.488683in}}{\pgfqpoint{1.656810in}{3.480869in}}%
\pgfpathcurveto{\pgfqpoint{1.664624in}{3.473055in}}{\pgfqpoint{1.675223in}{3.468665in}}{\pgfqpoint{1.686273in}{3.468665in}}%
\pgfpathclose%
\pgfusepath{stroke,fill}%
\end{pgfscope}%
\begin{pgfscope}%
\pgfpathrectangle{\pgfqpoint{0.648703in}{0.548769in}}{\pgfqpoint{5.201297in}{3.102590in}}%
\pgfusepath{clip}%
\pgfsetbuttcap%
\pgfsetroundjoin%
\definecolor{currentfill}{rgb}{1.000000,0.498039,0.054902}%
\pgfsetfillcolor{currentfill}%
\pgfsetlinewidth{1.003750pt}%
\definecolor{currentstroke}{rgb}{1.000000,0.498039,0.054902}%
\pgfsetstrokecolor{currentstroke}%
\pgfsetdash{}{0pt}%
\pgfpathmoveto{\pgfqpoint{1.684133in}{3.136837in}}%
\pgfpathcurveto{\pgfqpoint{1.695183in}{3.136837in}}{\pgfqpoint{1.705782in}{3.141228in}}{\pgfqpoint{1.713595in}{3.149041in}}%
\pgfpathcurveto{\pgfqpoint{1.721409in}{3.156855in}}{\pgfqpoint{1.725799in}{3.167454in}}{\pgfqpoint{1.725799in}{3.178504in}}%
\pgfpathcurveto{\pgfqpoint{1.725799in}{3.189554in}}{\pgfqpoint{1.721409in}{3.200153in}}{\pgfqpoint{1.713595in}{3.207967in}}%
\pgfpathcurveto{\pgfqpoint{1.705782in}{3.215780in}}{\pgfqpoint{1.695183in}{3.220171in}}{\pgfqpoint{1.684133in}{3.220171in}}%
\pgfpathcurveto{\pgfqpoint{1.673082in}{3.220171in}}{\pgfqpoint{1.662483in}{3.215780in}}{\pgfqpoint{1.654670in}{3.207967in}}%
\pgfpathcurveto{\pgfqpoint{1.646856in}{3.200153in}}{\pgfqpoint{1.642466in}{3.189554in}}{\pgfqpoint{1.642466in}{3.178504in}}%
\pgfpathcurveto{\pgfqpoint{1.642466in}{3.167454in}}{\pgfqpoint{1.646856in}{3.156855in}}{\pgfqpoint{1.654670in}{3.149041in}}%
\pgfpathcurveto{\pgfqpoint{1.662483in}{3.141228in}}{\pgfqpoint{1.673082in}{3.136837in}}{\pgfqpoint{1.684133in}{3.136837in}}%
\pgfpathclose%
\pgfusepath{stroke,fill}%
\end{pgfscope}%
\begin{pgfscope}%
\pgfpathrectangle{\pgfqpoint{0.648703in}{0.548769in}}{\pgfqpoint{5.201297in}{3.102590in}}%
\pgfusepath{clip}%
\pgfsetbuttcap%
\pgfsetroundjoin%
\definecolor{currentfill}{rgb}{1.000000,0.498039,0.054902}%
\pgfsetfillcolor{currentfill}%
\pgfsetlinewidth{1.003750pt}%
\definecolor{currentstroke}{rgb}{1.000000,0.498039,0.054902}%
\pgfsetstrokecolor{currentstroke}%
\pgfsetdash{}{0pt}%
\pgfpathmoveto{\pgfqpoint{2.141689in}{3.136837in}}%
\pgfpathcurveto{\pgfqpoint{2.152739in}{3.136837in}}{\pgfqpoint{2.163338in}{3.141228in}}{\pgfqpoint{2.171151in}{3.149041in}}%
\pgfpathcurveto{\pgfqpoint{2.178965in}{3.156855in}}{\pgfqpoint{2.183355in}{3.167454in}}{\pgfqpoint{2.183355in}{3.178504in}}%
\pgfpathcurveto{\pgfqpoint{2.183355in}{3.189554in}}{\pgfqpoint{2.178965in}{3.200153in}}{\pgfqpoint{2.171151in}{3.207967in}}%
\pgfpathcurveto{\pgfqpoint{2.163338in}{3.215780in}}{\pgfqpoint{2.152739in}{3.220171in}}{\pgfqpoint{2.141689in}{3.220171in}}%
\pgfpathcurveto{\pgfqpoint{2.130639in}{3.220171in}}{\pgfqpoint{2.120040in}{3.215780in}}{\pgfqpoint{2.112226in}{3.207967in}}%
\pgfpathcurveto{\pgfqpoint{2.104412in}{3.200153in}}{\pgfqpoint{2.100022in}{3.189554in}}{\pgfqpoint{2.100022in}{3.178504in}}%
\pgfpathcurveto{\pgfqpoint{2.100022in}{3.167454in}}{\pgfqpoint{2.104412in}{3.156855in}}{\pgfqpoint{2.112226in}{3.149041in}}%
\pgfpathcurveto{\pgfqpoint{2.120040in}{3.141228in}}{\pgfqpoint{2.130639in}{3.136837in}}{\pgfqpoint{2.141689in}{3.136837in}}%
\pgfpathclose%
\pgfusepath{stroke,fill}%
\end{pgfscope}%
\begin{pgfscope}%
\pgfpathrectangle{\pgfqpoint{0.648703in}{0.548769in}}{\pgfqpoint{5.201297in}{3.102590in}}%
\pgfusepath{clip}%
\pgfsetbuttcap%
\pgfsetroundjoin%
\definecolor{currentfill}{rgb}{1.000000,0.498039,0.054902}%
\pgfsetfillcolor{currentfill}%
\pgfsetlinewidth{1.003750pt}%
\definecolor{currentstroke}{rgb}{1.000000,0.498039,0.054902}%
\pgfsetstrokecolor{currentstroke}%
\pgfsetdash{}{0pt}%
\pgfpathmoveto{\pgfqpoint{1.749810in}{3.136837in}}%
\pgfpathcurveto{\pgfqpoint{1.760860in}{3.136837in}}{\pgfqpoint{1.771459in}{3.141228in}}{\pgfqpoint{1.779273in}{3.149041in}}%
\pgfpathcurveto{\pgfqpoint{1.787086in}{3.156855in}}{\pgfqpoint{1.791477in}{3.167454in}}{\pgfqpoint{1.791477in}{3.178504in}}%
\pgfpathcurveto{\pgfqpoint{1.791477in}{3.189554in}}{\pgfqpoint{1.787086in}{3.200153in}}{\pgfqpoint{1.779273in}{3.207967in}}%
\pgfpathcurveto{\pgfqpoint{1.771459in}{3.215780in}}{\pgfqpoint{1.760860in}{3.220171in}}{\pgfqpoint{1.749810in}{3.220171in}}%
\pgfpathcurveto{\pgfqpoint{1.738760in}{3.220171in}}{\pgfqpoint{1.728161in}{3.215780in}}{\pgfqpoint{1.720347in}{3.207967in}}%
\pgfpathcurveto{\pgfqpoint{1.712533in}{3.200153in}}{\pgfqpoint{1.708143in}{3.189554in}}{\pgfqpoint{1.708143in}{3.178504in}}%
\pgfpathcurveto{\pgfqpoint{1.708143in}{3.167454in}}{\pgfqpoint{1.712533in}{3.156855in}}{\pgfqpoint{1.720347in}{3.149041in}}%
\pgfpathcurveto{\pgfqpoint{1.728161in}{3.141228in}}{\pgfqpoint{1.738760in}{3.136837in}}{\pgfqpoint{1.749810in}{3.136837in}}%
\pgfpathclose%
\pgfusepath{stroke,fill}%
\end{pgfscope}%
\begin{pgfscope}%
\pgfpathrectangle{\pgfqpoint{0.648703in}{0.548769in}}{\pgfqpoint{5.201297in}{3.102590in}}%
\pgfusepath{clip}%
\pgfsetbuttcap%
\pgfsetroundjoin%
\definecolor{currentfill}{rgb}{0.839216,0.152941,0.156863}%
\pgfsetfillcolor{currentfill}%
\pgfsetlinewidth{1.003750pt}%
\definecolor{currentstroke}{rgb}{0.839216,0.152941,0.156863}%
\pgfsetstrokecolor{currentstroke}%
\pgfsetdash{}{0pt}%
\pgfpathmoveto{\pgfqpoint{1.871756in}{3.136837in}}%
\pgfpathcurveto{\pgfqpoint{1.882807in}{3.136837in}}{\pgfqpoint{1.893406in}{3.141228in}}{\pgfqpoint{1.901219in}{3.149041in}}%
\pgfpathcurveto{\pgfqpoint{1.909033in}{3.156855in}}{\pgfqpoint{1.913423in}{3.167454in}}{\pgfqpoint{1.913423in}{3.178504in}}%
\pgfpathcurveto{\pgfqpoint{1.913423in}{3.189554in}}{\pgfqpoint{1.909033in}{3.200153in}}{\pgfqpoint{1.901219in}{3.207967in}}%
\pgfpathcurveto{\pgfqpoint{1.893406in}{3.215780in}}{\pgfqpoint{1.882807in}{3.220171in}}{\pgfqpoint{1.871756in}{3.220171in}}%
\pgfpathcurveto{\pgfqpoint{1.860706in}{3.220171in}}{\pgfqpoint{1.850107in}{3.215780in}}{\pgfqpoint{1.842294in}{3.207967in}}%
\pgfpathcurveto{\pgfqpoint{1.834480in}{3.200153in}}{\pgfqpoint{1.830090in}{3.189554in}}{\pgfqpoint{1.830090in}{3.178504in}}%
\pgfpathcurveto{\pgfqpoint{1.830090in}{3.167454in}}{\pgfqpoint{1.834480in}{3.156855in}}{\pgfqpoint{1.842294in}{3.149041in}}%
\pgfpathcurveto{\pgfqpoint{1.850107in}{3.141228in}}{\pgfqpoint{1.860706in}{3.136837in}}{\pgfqpoint{1.871756in}{3.136837in}}%
\pgfpathclose%
\pgfusepath{stroke,fill}%
\end{pgfscope}%
\begin{pgfscope}%
\pgfpathrectangle{\pgfqpoint{0.648703in}{0.548769in}}{\pgfqpoint{5.201297in}{3.102590in}}%
\pgfusepath{clip}%
\pgfsetbuttcap%
\pgfsetroundjoin%
\definecolor{currentfill}{rgb}{1.000000,0.498039,0.054902}%
\pgfsetfillcolor{currentfill}%
\pgfsetlinewidth{1.003750pt}%
\definecolor{currentstroke}{rgb}{1.000000,0.498039,0.054902}%
\pgfsetstrokecolor{currentstroke}%
\pgfsetdash{}{0pt}%
\pgfpathmoveto{\pgfqpoint{1.586620in}{3.149281in}}%
\pgfpathcurveto{\pgfqpoint{1.597670in}{3.149281in}}{\pgfqpoint{1.608269in}{3.153671in}}{\pgfqpoint{1.616083in}{3.161485in}}%
\pgfpathcurveto{\pgfqpoint{1.623896in}{3.169298in}}{\pgfqpoint{1.628287in}{3.179897in}}{\pgfqpoint{1.628287in}{3.190948in}}%
\pgfpathcurveto{\pgfqpoint{1.628287in}{3.201998in}}{\pgfqpoint{1.623896in}{3.212597in}}{\pgfqpoint{1.616083in}{3.220410in}}%
\pgfpathcurveto{\pgfqpoint{1.608269in}{3.228224in}}{\pgfqpoint{1.597670in}{3.232614in}}{\pgfqpoint{1.586620in}{3.232614in}}%
\pgfpathcurveto{\pgfqpoint{1.575570in}{3.232614in}}{\pgfqpoint{1.564971in}{3.228224in}}{\pgfqpoint{1.557157in}{3.220410in}}%
\pgfpathcurveto{\pgfqpoint{1.549343in}{3.212597in}}{\pgfqpoint{1.544953in}{3.201998in}}{\pgfqpoint{1.544953in}{3.190948in}}%
\pgfpathcurveto{\pgfqpoint{1.544953in}{3.179897in}}{\pgfqpoint{1.549343in}{3.169298in}}{\pgfqpoint{1.557157in}{3.161485in}}%
\pgfpathcurveto{\pgfqpoint{1.564971in}{3.153671in}}{\pgfqpoint{1.575570in}{3.149281in}}{\pgfqpoint{1.586620in}{3.149281in}}%
\pgfpathclose%
\pgfusepath{stroke,fill}%
\end{pgfscope}%
\begin{pgfscope}%
\pgfpathrectangle{\pgfqpoint{0.648703in}{0.548769in}}{\pgfqpoint{5.201297in}{3.102590in}}%
\pgfusepath{clip}%
\pgfsetbuttcap%
\pgfsetroundjoin%
\definecolor{currentfill}{rgb}{1.000000,0.498039,0.054902}%
\pgfsetfillcolor{currentfill}%
\pgfsetlinewidth{1.003750pt}%
\definecolor{currentstroke}{rgb}{1.000000,0.498039,0.054902}%
\pgfsetstrokecolor{currentstroke}%
\pgfsetdash{}{0pt}%
\pgfpathmoveto{\pgfqpoint{1.759664in}{3.286160in}}%
\pgfpathcurveto{\pgfqpoint{1.770714in}{3.286160in}}{\pgfqpoint{1.781313in}{3.290550in}}{\pgfqpoint{1.789126in}{3.298364in}}%
\pgfpathcurveto{\pgfqpoint{1.796940in}{3.306177in}}{\pgfqpoint{1.801330in}{3.316776in}}{\pgfqpoint{1.801330in}{3.327827in}}%
\pgfpathcurveto{\pgfqpoint{1.801330in}{3.338877in}}{\pgfqpoint{1.796940in}{3.349476in}}{\pgfqpoint{1.789126in}{3.357289in}}%
\pgfpathcurveto{\pgfqpoint{1.781313in}{3.365103in}}{\pgfqpoint{1.770714in}{3.369493in}}{\pgfqpoint{1.759664in}{3.369493in}}%
\pgfpathcurveto{\pgfqpoint{1.748614in}{3.369493in}}{\pgfqpoint{1.738015in}{3.365103in}}{\pgfqpoint{1.730201in}{3.357289in}}%
\pgfpathcurveto{\pgfqpoint{1.722387in}{3.349476in}}{\pgfqpoint{1.717997in}{3.338877in}}{\pgfqpoint{1.717997in}{3.327827in}}%
\pgfpathcurveto{\pgfqpoint{1.717997in}{3.316776in}}{\pgfqpoint{1.722387in}{3.306177in}}{\pgfqpoint{1.730201in}{3.298364in}}%
\pgfpathcurveto{\pgfqpoint{1.738015in}{3.290550in}}{\pgfqpoint{1.748614in}{3.286160in}}{\pgfqpoint{1.759664in}{3.286160in}}%
\pgfpathclose%
\pgfusepath{stroke,fill}%
\end{pgfscope}%
\begin{pgfscope}%
\pgfpathrectangle{\pgfqpoint{0.648703in}{0.548769in}}{\pgfqpoint{5.201297in}{3.102590in}}%
\pgfusepath{clip}%
\pgfsetbuttcap%
\pgfsetroundjoin%
\definecolor{currentfill}{rgb}{1.000000,0.498039,0.054902}%
\pgfsetfillcolor{currentfill}%
\pgfsetlinewidth{1.003750pt}%
\definecolor{currentstroke}{rgb}{1.000000,0.498039,0.054902}%
\pgfsetstrokecolor{currentstroke}%
\pgfsetdash{}{0pt}%
\pgfpathmoveto{\pgfqpoint{1.642800in}{3.136837in}}%
\pgfpathcurveto{\pgfqpoint{1.653850in}{3.136837in}}{\pgfqpoint{1.664449in}{3.141228in}}{\pgfqpoint{1.672263in}{3.149041in}}%
\pgfpathcurveto{\pgfqpoint{1.680076in}{3.156855in}}{\pgfqpoint{1.684467in}{3.167454in}}{\pgfqpoint{1.684467in}{3.178504in}}%
\pgfpathcurveto{\pgfqpoint{1.684467in}{3.189554in}}{\pgfqpoint{1.680076in}{3.200153in}}{\pgfqpoint{1.672263in}{3.207967in}}%
\pgfpathcurveto{\pgfqpoint{1.664449in}{3.215780in}}{\pgfqpoint{1.653850in}{3.220171in}}{\pgfqpoint{1.642800in}{3.220171in}}%
\pgfpathcurveto{\pgfqpoint{1.631750in}{3.220171in}}{\pgfqpoint{1.621151in}{3.215780in}}{\pgfqpoint{1.613337in}{3.207967in}}%
\pgfpathcurveto{\pgfqpoint{1.605524in}{3.200153in}}{\pgfqpoint{1.601133in}{3.189554in}}{\pgfqpoint{1.601133in}{3.178504in}}%
\pgfpathcurveto{\pgfqpoint{1.601133in}{3.167454in}}{\pgfqpoint{1.605524in}{3.156855in}}{\pgfqpoint{1.613337in}{3.149041in}}%
\pgfpathcurveto{\pgfqpoint{1.621151in}{3.141228in}}{\pgfqpoint{1.631750in}{3.136837in}}{\pgfqpoint{1.642800in}{3.136837in}}%
\pgfpathclose%
\pgfusepath{stroke,fill}%
\end{pgfscope}%
\begin{pgfscope}%
\pgfpathrectangle{\pgfqpoint{0.648703in}{0.548769in}}{\pgfqpoint{5.201297in}{3.102590in}}%
\pgfusepath{clip}%
\pgfsetbuttcap%
\pgfsetroundjoin%
\definecolor{currentfill}{rgb}{1.000000,0.498039,0.054902}%
\pgfsetfillcolor{currentfill}%
\pgfsetlinewidth{1.003750pt}%
\definecolor{currentstroke}{rgb}{1.000000,0.498039,0.054902}%
\pgfsetstrokecolor{currentstroke}%
\pgfsetdash{}{0pt}%
\pgfpathmoveto{\pgfqpoint{2.127376in}{3.149281in}}%
\pgfpathcurveto{\pgfqpoint{2.138426in}{3.149281in}}{\pgfqpoint{2.149025in}{3.153671in}}{\pgfqpoint{2.156839in}{3.161485in}}%
\pgfpathcurveto{\pgfqpoint{2.164653in}{3.169298in}}{\pgfqpoint{2.169043in}{3.179897in}}{\pgfqpoint{2.169043in}{3.190948in}}%
\pgfpathcurveto{\pgfqpoint{2.169043in}{3.201998in}}{\pgfqpoint{2.164653in}{3.212597in}}{\pgfqpoint{2.156839in}{3.220410in}}%
\pgfpathcurveto{\pgfqpoint{2.149025in}{3.228224in}}{\pgfqpoint{2.138426in}{3.232614in}}{\pgfqpoint{2.127376in}{3.232614in}}%
\pgfpathcurveto{\pgfqpoint{2.116326in}{3.232614in}}{\pgfqpoint{2.105727in}{3.228224in}}{\pgfqpoint{2.097913in}{3.220410in}}%
\pgfpathcurveto{\pgfqpoint{2.090100in}{3.212597in}}{\pgfqpoint{2.085709in}{3.201998in}}{\pgfqpoint{2.085709in}{3.190948in}}%
\pgfpathcurveto{\pgfqpoint{2.085709in}{3.179897in}}{\pgfqpoint{2.090100in}{3.169298in}}{\pgfqpoint{2.097913in}{3.161485in}}%
\pgfpathcurveto{\pgfqpoint{2.105727in}{3.153671in}}{\pgfqpoint{2.116326in}{3.149281in}}{\pgfqpoint{2.127376in}{3.149281in}}%
\pgfpathclose%
\pgfusepath{stroke,fill}%
\end{pgfscope}%
\begin{pgfscope}%
\pgfpathrectangle{\pgfqpoint{0.648703in}{0.548769in}}{\pgfqpoint{5.201297in}{3.102590in}}%
\pgfusepath{clip}%
\pgfsetbuttcap%
\pgfsetroundjoin%
\definecolor{currentfill}{rgb}{0.121569,0.466667,0.705882}%
\pgfsetfillcolor{currentfill}%
\pgfsetlinewidth{1.003750pt}%
\definecolor{currentstroke}{rgb}{0.121569,0.466667,0.705882}%
\pgfsetstrokecolor{currentstroke}%
\pgfsetdash{}{0pt}%
\pgfpathmoveto{\pgfqpoint{2.442877in}{3.128542in}}%
\pgfpathcurveto{\pgfqpoint{2.453927in}{3.128542in}}{\pgfqpoint{2.464526in}{3.132932in}}{\pgfqpoint{2.472340in}{3.140746in}}%
\pgfpathcurveto{\pgfqpoint{2.480153in}{3.148559in}}{\pgfqpoint{2.484543in}{3.159158in}}{\pgfqpoint{2.484543in}{3.170208in}}%
\pgfpathcurveto{\pgfqpoint{2.484543in}{3.181258in}}{\pgfqpoint{2.480153in}{3.191857in}}{\pgfqpoint{2.472340in}{3.199671in}}%
\pgfpathcurveto{\pgfqpoint{2.464526in}{3.207485in}}{\pgfqpoint{2.453927in}{3.211875in}}{\pgfqpoint{2.442877in}{3.211875in}}%
\pgfpathcurveto{\pgfqpoint{2.431827in}{3.211875in}}{\pgfqpoint{2.421228in}{3.207485in}}{\pgfqpoint{2.413414in}{3.199671in}}%
\pgfpathcurveto{\pgfqpoint{2.405600in}{3.191857in}}{\pgfqpoint{2.401210in}{3.181258in}}{\pgfqpoint{2.401210in}{3.170208in}}%
\pgfpathcurveto{\pgfqpoint{2.401210in}{3.159158in}}{\pgfqpoint{2.405600in}{3.148559in}}{\pgfqpoint{2.413414in}{3.140746in}}%
\pgfpathcurveto{\pgfqpoint{2.421228in}{3.132932in}}{\pgfqpoint{2.431827in}{3.128542in}}{\pgfqpoint{2.442877in}{3.128542in}}%
\pgfpathclose%
\pgfusepath{stroke,fill}%
\end{pgfscope}%
\begin{pgfscope}%
\pgfpathrectangle{\pgfqpoint{0.648703in}{0.548769in}}{\pgfqpoint{5.201297in}{3.102590in}}%
\pgfusepath{clip}%
\pgfsetbuttcap%
\pgfsetroundjoin%
\definecolor{currentfill}{rgb}{1.000000,0.498039,0.054902}%
\pgfsetfillcolor{currentfill}%
\pgfsetlinewidth{1.003750pt}%
\definecolor{currentstroke}{rgb}{1.000000,0.498039,0.054902}%
\pgfsetstrokecolor{currentstroke}%
\pgfsetdash{}{0pt}%
\pgfpathmoveto{\pgfqpoint{2.247183in}{3.232238in}}%
\pgfpathcurveto{\pgfqpoint{2.258233in}{3.232238in}}{\pgfqpoint{2.268832in}{3.236628in}}{\pgfqpoint{2.276645in}{3.244442in}}%
\pgfpathcurveto{\pgfqpoint{2.284459in}{3.252255in}}{\pgfqpoint{2.288849in}{3.262854in}}{\pgfqpoint{2.288849in}{3.273905in}}%
\pgfpathcurveto{\pgfqpoint{2.288849in}{3.284955in}}{\pgfqpoint{2.284459in}{3.295554in}}{\pgfqpoint{2.276645in}{3.303367in}}%
\pgfpathcurveto{\pgfqpoint{2.268832in}{3.311181in}}{\pgfqpoint{2.258233in}{3.315571in}}{\pgfqpoint{2.247183in}{3.315571in}}%
\pgfpathcurveto{\pgfqpoint{2.236132in}{3.315571in}}{\pgfqpoint{2.225533in}{3.311181in}}{\pgfqpoint{2.217720in}{3.303367in}}%
\pgfpathcurveto{\pgfqpoint{2.209906in}{3.295554in}}{\pgfqpoint{2.205516in}{3.284955in}}{\pgfqpoint{2.205516in}{3.273905in}}%
\pgfpathcurveto{\pgfqpoint{2.205516in}{3.262854in}}{\pgfqpoint{2.209906in}{3.252255in}}{\pgfqpoint{2.217720in}{3.244442in}}%
\pgfpathcurveto{\pgfqpoint{2.225533in}{3.236628in}}{\pgfqpoint{2.236132in}{3.232238in}}{\pgfqpoint{2.247183in}{3.232238in}}%
\pgfpathclose%
\pgfusepath{stroke,fill}%
\end{pgfscope}%
\begin{pgfscope}%
\pgfpathrectangle{\pgfqpoint{0.648703in}{0.548769in}}{\pgfqpoint{5.201297in}{3.102590in}}%
\pgfusepath{clip}%
\pgfsetbuttcap%
\pgfsetroundjoin%
\definecolor{currentfill}{rgb}{0.121569,0.466667,0.705882}%
\pgfsetfillcolor{currentfill}%
\pgfsetlinewidth{1.003750pt}%
\definecolor{currentstroke}{rgb}{0.121569,0.466667,0.705882}%
\pgfsetstrokecolor{currentstroke}%
\pgfsetdash{}{0pt}%
\pgfpathmoveto{\pgfqpoint{1.542345in}{0.648129in}}%
\pgfpathcurveto{\pgfqpoint{1.553395in}{0.648129in}}{\pgfqpoint{1.563994in}{0.652519in}}{\pgfqpoint{1.571807in}{0.660333in}}%
\pgfpathcurveto{\pgfqpoint{1.579621in}{0.668146in}}{\pgfqpoint{1.584011in}{0.678745in}}{\pgfqpoint{1.584011in}{0.689796in}}%
\pgfpathcurveto{\pgfqpoint{1.584011in}{0.700846in}}{\pgfqpoint{1.579621in}{0.711445in}}{\pgfqpoint{1.571807in}{0.719258in}}%
\pgfpathcurveto{\pgfqpoint{1.563994in}{0.727072in}}{\pgfqpoint{1.553395in}{0.731462in}}{\pgfqpoint{1.542345in}{0.731462in}}%
\pgfpathcurveto{\pgfqpoint{1.531294in}{0.731462in}}{\pgfqpoint{1.520695in}{0.727072in}}{\pgfqpoint{1.512882in}{0.719258in}}%
\pgfpathcurveto{\pgfqpoint{1.505068in}{0.711445in}}{\pgfqpoint{1.500678in}{0.700846in}}{\pgfqpoint{1.500678in}{0.689796in}}%
\pgfpathcurveto{\pgfqpoint{1.500678in}{0.678745in}}{\pgfqpoint{1.505068in}{0.668146in}}{\pgfqpoint{1.512882in}{0.660333in}}%
\pgfpathcurveto{\pgfqpoint{1.520695in}{0.652519in}}{\pgfqpoint{1.531294in}{0.648129in}}{\pgfqpoint{1.542345in}{0.648129in}}%
\pgfpathclose%
\pgfusepath{stroke,fill}%
\end{pgfscope}%
\begin{pgfscope}%
\pgfpathrectangle{\pgfqpoint{0.648703in}{0.548769in}}{\pgfqpoint{5.201297in}{3.102590in}}%
\pgfusepath{clip}%
\pgfsetbuttcap%
\pgfsetroundjoin%
\definecolor{currentfill}{rgb}{1.000000,0.498039,0.054902}%
\pgfsetfillcolor{currentfill}%
\pgfsetlinewidth{1.003750pt}%
\definecolor{currentstroke}{rgb}{1.000000,0.498039,0.054902}%
\pgfsetstrokecolor{currentstroke}%
\pgfsetdash{}{0pt}%
\pgfpathmoveto{\pgfqpoint{1.986257in}{3.136837in}}%
\pgfpathcurveto{\pgfqpoint{1.997307in}{3.136837in}}{\pgfqpoint{2.007906in}{3.141228in}}{\pgfqpoint{2.015720in}{3.149041in}}%
\pgfpathcurveto{\pgfqpoint{2.023533in}{3.156855in}}{\pgfqpoint{2.027924in}{3.167454in}}{\pgfqpoint{2.027924in}{3.178504in}}%
\pgfpathcurveto{\pgfqpoint{2.027924in}{3.189554in}}{\pgfqpoint{2.023533in}{3.200153in}}{\pgfqpoint{2.015720in}{3.207967in}}%
\pgfpathcurveto{\pgfqpoint{2.007906in}{3.215780in}}{\pgfqpoint{1.997307in}{3.220171in}}{\pgfqpoint{1.986257in}{3.220171in}}%
\pgfpathcurveto{\pgfqpoint{1.975207in}{3.220171in}}{\pgfqpoint{1.964608in}{3.215780in}}{\pgfqpoint{1.956794in}{3.207967in}}%
\pgfpathcurveto{\pgfqpoint{1.948981in}{3.200153in}}{\pgfqpoint{1.944590in}{3.189554in}}{\pgfqpoint{1.944590in}{3.178504in}}%
\pgfpathcurveto{\pgfqpoint{1.944590in}{3.167454in}}{\pgfqpoint{1.948981in}{3.156855in}}{\pgfqpoint{1.956794in}{3.149041in}}%
\pgfpathcurveto{\pgfqpoint{1.964608in}{3.141228in}}{\pgfqpoint{1.975207in}{3.136837in}}{\pgfqpoint{1.986257in}{3.136837in}}%
\pgfpathclose%
\pgfusepath{stroke,fill}%
\end{pgfscope}%
\begin{pgfscope}%
\pgfpathrectangle{\pgfqpoint{0.648703in}{0.548769in}}{\pgfqpoint{5.201297in}{3.102590in}}%
\pgfusepath{clip}%
\pgfsetbuttcap%
\pgfsetroundjoin%
\definecolor{currentfill}{rgb}{1.000000,0.498039,0.054902}%
\pgfsetfillcolor{currentfill}%
\pgfsetlinewidth{1.003750pt}%
\definecolor{currentstroke}{rgb}{1.000000,0.498039,0.054902}%
\pgfsetstrokecolor{currentstroke}%
\pgfsetdash{}{0pt}%
\pgfpathmoveto{\pgfqpoint{1.362657in}{3.140985in}}%
\pgfpathcurveto{\pgfqpoint{1.373707in}{3.140985in}}{\pgfqpoint{1.384306in}{3.145375in}}{\pgfqpoint{1.392120in}{3.153189in}}%
\pgfpathcurveto{\pgfqpoint{1.399934in}{3.161003in}}{\pgfqpoint{1.404324in}{3.171602in}}{\pgfqpoint{1.404324in}{3.182652in}}%
\pgfpathcurveto{\pgfqpoint{1.404324in}{3.193702in}}{\pgfqpoint{1.399934in}{3.204301in}}{\pgfqpoint{1.392120in}{3.212115in}}%
\pgfpathcurveto{\pgfqpoint{1.384306in}{3.219928in}}{\pgfqpoint{1.373707in}{3.224319in}}{\pgfqpoint{1.362657in}{3.224319in}}%
\pgfpathcurveto{\pgfqpoint{1.351607in}{3.224319in}}{\pgfqpoint{1.341008in}{3.219928in}}{\pgfqpoint{1.333194in}{3.212115in}}%
\pgfpathcurveto{\pgfqpoint{1.325381in}{3.204301in}}{\pgfqpoint{1.320991in}{3.193702in}}{\pgfqpoint{1.320991in}{3.182652in}}%
\pgfpathcurveto{\pgfqpoint{1.320991in}{3.171602in}}{\pgfqpoint{1.325381in}{3.161003in}}{\pgfqpoint{1.333194in}{3.153189in}}%
\pgfpathcurveto{\pgfqpoint{1.341008in}{3.145375in}}{\pgfqpoint{1.351607in}{3.140985in}}{\pgfqpoint{1.362657in}{3.140985in}}%
\pgfpathclose%
\pgfusepath{stroke,fill}%
\end{pgfscope}%
\begin{pgfscope}%
\pgfpathrectangle{\pgfqpoint{0.648703in}{0.548769in}}{\pgfqpoint{5.201297in}{3.102590in}}%
\pgfusepath{clip}%
\pgfsetbuttcap%
\pgfsetroundjoin%
\definecolor{currentfill}{rgb}{1.000000,0.498039,0.054902}%
\pgfsetfillcolor{currentfill}%
\pgfsetlinewidth{1.003750pt}%
\definecolor{currentstroke}{rgb}{1.000000,0.498039,0.054902}%
\pgfsetstrokecolor{currentstroke}%
\pgfsetdash{}{0pt}%
\pgfpathmoveto{\pgfqpoint{1.585996in}{3.149281in}}%
\pgfpathcurveto{\pgfqpoint{1.597046in}{3.149281in}}{\pgfqpoint{1.607645in}{3.153671in}}{\pgfqpoint{1.615458in}{3.161485in}}%
\pgfpathcurveto{\pgfqpoint{1.623272in}{3.169298in}}{\pgfqpoint{1.627662in}{3.179897in}}{\pgfqpoint{1.627662in}{3.190948in}}%
\pgfpathcurveto{\pgfqpoint{1.627662in}{3.201998in}}{\pgfqpoint{1.623272in}{3.212597in}}{\pgfqpoint{1.615458in}{3.220410in}}%
\pgfpathcurveto{\pgfqpoint{1.607645in}{3.228224in}}{\pgfqpoint{1.597046in}{3.232614in}}{\pgfqpoint{1.585996in}{3.232614in}}%
\pgfpathcurveto{\pgfqpoint{1.574946in}{3.232614in}}{\pgfqpoint{1.564347in}{3.228224in}}{\pgfqpoint{1.556533in}{3.220410in}}%
\pgfpathcurveto{\pgfqpoint{1.548719in}{3.212597in}}{\pgfqpoint{1.544329in}{3.201998in}}{\pgfqpoint{1.544329in}{3.190948in}}%
\pgfpathcurveto{\pgfqpoint{1.544329in}{3.179897in}}{\pgfqpoint{1.548719in}{3.169298in}}{\pgfqpoint{1.556533in}{3.161485in}}%
\pgfpathcurveto{\pgfqpoint{1.564347in}{3.153671in}}{\pgfqpoint{1.574946in}{3.149281in}}{\pgfqpoint{1.585996in}{3.149281in}}%
\pgfpathclose%
\pgfusepath{stroke,fill}%
\end{pgfscope}%
\begin{pgfscope}%
\pgfpathrectangle{\pgfqpoint{0.648703in}{0.548769in}}{\pgfqpoint{5.201297in}{3.102590in}}%
\pgfusepath{clip}%
\pgfsetbuttcap%
\pgfsetroundjoin%
\definecolor{currentfill}{rgb}{1.000000,0.498039,0.054902}%
\pgfsetfillcolor{currentfill}%
\pgfsetlinewidth{1.003750pt}%
\definecolor{currentstroke}{rgb}{1.000000,0.498039,0.054902}%
\pgfsetstrokecolor{currentstroke}%
\pgfsetdash{}{0pt}%
\pgfpathmoveto{\pgfqpoint{1.800729in}{3.140985in}}%
\pgfpathcurveto{\pgfqpoint{1.811779in}{3.140985in}}{\pgfqpoint{1.822378in}{3.145375in}}{\pgfqpoint{1.830191in}{3.153189in}}%
\pgfpathcurveto{\pgfqpoint{1.838005in}{3.161003in}}{\pgfqpoint{1.842395in}{3.171602in}}{\pgfqpoint{1.842395in}{3.182652in}}%
\pgfpathcurveto{\pgfqpoint{1.842395in}{3.193702in}}{\pgfqpoint{1.838005in}{3.204301in}}{\pgfqpoint{1.830191in}{3.212115in}}%
\pgfpathcurveto{\pgfqpoint{1.822378in}{3.219928in}}{\pgfqpoint{1.811779in}{3.224319in}}{\pgfqpoint{1.800729in}{3.224319in}}%
\pgfpathcurveto{\pgfqpoint{1.789679in}{3.224319in}}{\pgfqpoint{1.779080in}{3.219928in}}{\pgfqpoint{1.771266in}{3.212115in}}%
\pgfpathcurveto{\pgfqpoint{1.763452in}{3.204301in}}{\pgfqpoint{1.759062in}{3.193702in}}{\pgfqpoint{1.759062in}{3.182652in}}%
\pgfpathcurveto{\pgfqpoint{1.759062in}{3.171602in}}{\pgfqpoint{1.763452in}{3.161003in}}{\pgfqpoint{1.771266in}{3.153189in}}%
\pgfpathcurveto{\pgfqpoint{1.779080in}{3.145375in}}{\pgfqpoint{1.789679in}{3.140985in}}{\pgfqpoint{1.800729in}{3.140985in}}%
\pgfpathclose%
\pgfusepath{stroke,fill}%
\end{pgfscope}%
\begin{pgfscope}%
\pgfpathrectangle{\pgfqpoint{0.648703in}{0.548769in}}{\pgfqpoint{5.201297in}{3.102590in}}%
\pgfusepath{clip}%
\pgfsetbuttcap%
\pgfsetroundjoin%
\definecolor{currentfill}{rgb}{1.000000,0.498039,0.054902}%
\pgfsetfillcolor{currentfill}%
\pgfsetlinewidth{1.003750pt}%
\definecolor{currentstroke}{rgb}{1.000000,0.498039,0.054902}%
\pgfsetstrokecolor{currentstroke}%
\pgfsetdash{}{0pt}%
\pgfpathmoveto{\pgfqpoint{1.643513in}{3.136837in}}%
\pgfpathcurveto{\pgfqpoint{1.654564in}{3.136837in}}{\pgfqpoint{1.665163in}{3.141228in}}{\pgfqpoint{1.672976in}{3.149041in}}%
\pgfpathcurveto{\pgfqpoint{1.680790in}{3.156855in}}{\pgfqpoint{1.685180in}{3.167454in}}{\pgfqpoint{1.685180in}{3.178504in}}%
\pgfpathcurveto{\pgfqpoint{1.685180in}{3.189554in}}{\pgfqpoint{1.680790in}{3.200153in}}{\pgfqpoint{1.672976in}{3.207967in}}%
\pgfpathcurveto{\pgfqpoint{1.665163in}{3.215780in}}{\pgfqpoint{1.654564in}{3.220171in}}{\pgfqpoint{1.643513in}{3.220171in}}%
\pgfpathcurveto{\pgfqpoint{1.632463in}{3.220171in}}{\pgfqpoint{1.621864in}{3.215780in}}{\pgfqpoint{1.614051in}{3.207967in}}%
\pgfpathcurveto{\pgfqpoint{1.606237in}{3.200153in}}{\pgfqpoint{1.601847in}{3.189554in}}{\pgfqpoint{1.601847in}{3.178504in}}%
\pgfpathcurveto{\pgfqpoint{1.601847in}{3.167454in}}{\pgfqpoint{1.606237in}{3.156855in}}{\pgfqpoint{1.614051in}{3.149041in}}%
\pgfpathcurveto{\pgfqpoint{1.621864in}{3.141228in}}{\pgfqpoint{1.632463in}{3.136837in}}{\pgfqpoint{1.643513in}{3.136837in}}%
\pgfpathclose%
\pgfusepath{stroke,fill}%
\end{pgfscope}%
\begin{pgfscope}%
\pgfpathrectangle{\pgfqpoint{0.648703in}{0.548769in}}{\pgfqpoint{5.201297in}{3.102590in}}%
\pgfusepath{clip}%
\pgfsetbuttcap%
\pgfsetroundjoin%
\definecolor{currentfill}{rgb}{1.000000,0.498039,0.054902}%
\pgfsetfillcolor{currentfill}%
\pgfsetlinewidth{1.003750pt}%
\definecolor{currentstroke}{rgb}{1.000000,0.498039,0.054902}%
\pgfsetstrokecolor{currentstroke}%
\pgfsetdash{}{0pt}%
\pgfpathmoveto{\pgfqpoint{1.524733in}{3.145133in}}%
\pgfpathcurveto{\pgfqpoint{1.535783in}{3.145133in}}{\pgfqpoint{1.546382in}{3.149523in}}{\pgfqpoint{1.554195in}{3.157337in}}%
\pgfpathcurveto{\pgfqpoint{1.562009in}{3.165151in}}{\pgfqpoint{1.566399in}{3.175750in}}{\pgfqpoint{1.566399in}{3.186800in}}%
\pgfpathcurveto{\pgfqpoint{1.566399in}{3.197850in}}{\pgfqpoint{1.562009in}{3.208449in}}{\pgfqpoint{1.554195in}{3.216262in}}%
\pgfpathcurveto{\pgfqpoint{1.546382in}{3.224076in}}{\pgfqpoint{1.535783in}{3.228466in}}{\pgfqpoint{1.524733in}{3.228466in}}%
\pgfpathcurveto{\pgfqpoint{1.513682in}{3.228466in}}{\pgfqpoint{1.503083in}{3.224076in}}{\pgfqpoint{1.495270in}{3.216262in}}%
\pgfpathcurveto{\pgfqpoint{1.487456in}{3.208449in}}{\pgfqpoint{1.483066in}{3.197850in}}{\pgfqpoint{1.483066in}{3.186800in}}%
\pgfpathcurveto{\pgfqpoint{1.483066in}{3.175750in}}{\pgfqpoint{1.487456in}{3.165151in}}{\pgfqpoint{1.495270in}{3.157337in}}%
\pgfpathcurveto{\pgfqpoint{1.503083in}{3.149523in}}{\pgfqpoint{1.513682in}{3.145133in}}{\pgfqpoint{1.524733in}{3.145133in}}%
\pgfpathclose%
\pgfusepath{stroke,fill}%
\end{pgfscope}%
\begin{pgfscope}%
\pgfpathrectangle{\pgfqpoint{0.648703in}{0.548769in}}{\pgfqpoint{5.201297in}{3.102590in}}%
\pgfusepath{clip}%
\pgfsetbuttcap%
\pgfsetroundjoin%
\definecolor{currentfill}{rgb}{0.121569,0.466667,0.705882}%
\pgfsetfillcolor{currentfill}%
\pgfsetlinewidth{1.003750pt}%
\definecolor{currentstroke}{rgb}{0.121569,0.466667,0.705882}%
\pgfsetstrokecolor{currentstroke}%
\pgfsetdash{}{0pt}%
\pgfpathmoveto{\pgfqpoint{2.260737in}{3.132690in}}%
\pgfpathcurveto{\pgfqpoint{2.271787in}{3.132690in}}{\pgfqpoint{2.282386in}{3.137080in}}{\pgfqpoint{2.290200in}{3.144893in}}%
\pgfpathcurveto{\pgfqpoint{2.298014in}{3.152707in}}{\pgfqpoint{2.302404in}{3.163306in}}{\pgfqpoint{2.302404in}{3.174356in}}%
\pgfpathcurveto{\pgfqpoint{2.302404in}{3.185406in}}{\pgfqpoint{2.298014in}{3.196005in}}{\pgfqpoint{2.290200in}{3.203819in}}%
\pgfpathcurveto{\pgfqpoint{2.282386in}{3.211633in}}{\pgfqpoint{2.271787in}{3.216023in}}{\pgfqpoint{2.260737in}{3.216023in}}%
\pgfpathcurveto{\pgfqpoint{2.249687in}{3.216023in}}{\pgfqpoint{2.239088in}{3.211633in}}{\pgfqpoint{2.231274in}{3.203819in}}%
\pgfpathcurveto{\pgfqpoint{2.223461in}{3.196005in}}{\pgfqpoint{2.219070in}{3.185406in}}{\pgfqpoint{2.219070in}{3.174356in}}%
\pgfpathcurveto{\pgfqpoint{2.219070in}{3.163306in}}{\pgfqpoint{2.223461in}{3.152707in}}{\pgfqpoint{2.231274in}{3.144893in}}%
\pgfpathcurveto{\pgfqpoint{2.239088in}{3.137080in}}{\pgfqpoint{2.249687in}{3.132690in}}{\pgfqpoint{2.260737in}{3.132690in}}%
\pgfpathclose%
\pgfusepath{stroke,fill}%
\end{pgfscope}%
\begin{pgfscope}%
\pgfpathrectangle{\pgfqpoint{0.648703in}{0.548769in}}{\pgfqpoint{5.201297in}{3.102590in}}%
\pgfusepath{clip}%
\pgfsetbuttcap%
\pgfsetroundjoin%
\definecolor{currentfill}{rgb}{1.000000,0.498039,0.054902}%
\pgfsetfillcolor{currentfill}%
\pgfsetlinewidth{1.003750pt}%
\definecolor{currentstroke}{rgb}{1.000000,0.498039,0.054902}%
\pgfsetstrokecolor{currentstroke}%
\pgfsetdash{}{0pt}%
\pgfpathmoveto{\pgfqpoint{1.779639in}{3.348378in}}%
\pgfpathcurveto{\pgfqpoint{1.790689in}{3.348378in}}{\pgfqpoint{1.801288in}{3.352768in}}{\pgfqpoint{1.809102in}{3.360581in}}%
\pgfpathcurveto{\pgfqpoint{1.816915in}{3.368395in}}{\pgfqpoint{1.821305in}{3.378994in}}{\pgfqpoint{1.821305in}{3.390044in}}%
\pgfpathcurveto{\pgfqpoint{1.821305in}{3.401094in}}{\pgfqpoint{1.816915in}{3.411693in}}{\pgfqpoint{1.809102in}{3.419507in}}%
\pgfpathcurveto{\pgfqpoint{1.801288in}{3.427321in}}{\pgfqpoint{1.790689in}{3.431711in}}{\pgfqpoint{1.779639in}{3.431711in}}%
\pgfpathcurveto{\pgfqpoint{1.768589in}{3.431711in}}{\pgfqpoint{1.757990in}{3.427321in}}{\pgfqpoint{1.750176in}{3.419507in}}%
\pgfpathcurveto{\pgfqpoint{1.742362in}{3.411693in}}{\pgfqpoint{1.737972in}{3.401094in}}{\pgfqpoint{1.737972in}{3.390044in}}%
\pgfpathcurveto{\pgfqpoint{1.737972in}{3.378994in}}{\pgfqpoint{1.742362in}{3.368395in}}{\pgfqpoint{1.750176in}{3.360581in}}%
\pgfpathcurveto{\pgfqpoint{1.757990in}{3.352768in}}{\pgfqpoint{1.768589in}{3.348378in}}{\pgfqpoint{1.779639in}{3.348378in}}%
\pgfpathclose%
\pgfusepath{stroke,fill}%
\end{pgfscope}%
\begin{pgfscope}%
\pgfsetbuttcap%
\pgfsetroundjoin%
\definecolor{currentfill}{rgb}{0.000000,0.000000,0.000000}%
\pgfsetfillcolor{currentfill}%
\pgfsetlinewidth{0.803000pt}%
\definecolor{currentstroke}{rgb}{0.000000,0.000000,0.000000}%
\pgfsetstrokecolor{currentstroke}%
\pgfsetdash{}{0pt}%
\pgfsys@defobject{currentmarker}{\pgfqpoint{0.000000in}{-0.048611in}}{\pgfqpoint{0.000000in}{0.000000in}}{%
\pgfpathmoveto{\pgfqpoint{0.000000in}{0.000000in}}%
\pgfpathlineto{\pgfqpoint{0.000000in}{-0.048611in}}%
\pgfusepath{stroke,fill}%
}%
\begin{pgfscope}%
\pgfsys@transformshift{0.854628in}{0.548769in}%
\pgfsys@useobject{currentmarker}{}%
\end{pgfscope}%
\end{pgfscope}%
\begin{pgfscope}%
\definecolor{textcolor}{rgb}{0.000000,0.000000,0.000000}%
\pgfsetstrokecolor{textcolor}%
\pgfsetfillcolor{textcolor}%
\pgftext[x=0.854628in,y=0.451547in,,top]{\color{textcolor}\sffamily\fontsize{10.000000}{12.000000}\selectfont \(\displaystyle {0}\)}%
\end{pgfscope}%
\begin{pgfscope}%
\pgfsetbuttcap%
\pgfsetroundjoin%
\definecolor{currentfill}{rgb}{0.000000,0.000000,0.000000}%
\pgfsetfillcolor{currentfill}%
\pgfsetlinewidth{0.803000pt}%
\definecolor{currentstroke}{rgb}{0.000000,0.000000,0.000000}%
\pgfsetstrokecolor{currentstroke}%
\pgfsetdash{}{0pt}%
\pgfsys@defobject{currentmarker}{\pgfqpoint{0.000000in}{-0.048611in}}{\pgfqpoint{0.000000in}{0.000000in}}{%
\pgfpathmoveto{\pgfqpoint{0.000000in}{0.000000in}}%
\pgfpathlineto{\pgfqpoint{0.000000in}{-0.048611in}}%
\pgfusepath{stroke,fill}%
}%
\begin{pgfscope}%
\pgfsys@transformshift{1.746377in}{0.548769in}%
\pgfsys@useobject{currentmarker}{}%
\end{pgfscope}%
\end{pgfscope}%
\begin{pgfscope}%
\definecolor{textcolor}{rgb}{0.000000,0.000000,0.000000}%
\pgfsetstrokecolor{textcolor}%
\pgfsetfillcolor{textcolor}%
\pgftext[x=1.746377in,y=0.451547in,,top]{\color{textcolor}\sffamily\fontsize{10.000000}{12.000000}\selectfont \(\displaystyle {20000}\)}%
\end{pgfscope}%
\begin{pgfscope}%
\pgfsetbuttcap%
\pgfsetroundjoin%
\definecolor{currentfill}{rgb}{0.000000,0.000000,0.000000}%
\pgfsetfillcolor{currentfill}%
\pgfsetlinewidth{0.803000pt}%
\definecolor{currentstroke}{rgb}{0.000000,0.000000,0.000000}%
\pgfsetstrokecolor{currentstroke}%
\pgfsetdash{}{0pt}%
\pgfsys@defobject{currentmarker}{\pgfqpoint{0.000000in}{-0.048611in}}{\pgfqpoint{0.000000in}{0.000000in}}{%
\pgfpathmoveto{\pgfqpoint{0.000000in}{0.000000in}}%
\pgfpathlineto{\pgfqpoint{0.000000in}{-0.048611in}}%
\pgfusepath{stroke,fill}%
}%
\begin{pgfscope}%
\pgfsys@transformshift{2.638125in}{0.548769in}%
\pgfsys@useobject{currentmarker}{}%
\end{pgfscope}%
\end{pgfscope}%
\begin{pgfscope}%
\definecolor{textcolor}{rgb}{0.000000,0.000000,0.000000}%
\pgfsetstrokecolor{textcolor}%
\pgfsetfillcolor{textcolor}%
\pgftext[x=2.638125in,y=0.451547in,,top]{\color{textcolor}\sffamily\fontsize{10.000000}{12.000000}\selectfont \(\displaystyle {40000}\)}%
\end{pgfscope}%
\begin{pgfscope}%
\pgfsetbuttcap%
\pgfsetroundjoin%
\definecolor{currentfill}{rgb}{0.000000,0.000000,0.000000}%
\pgfsetfillcolor{currentfill}%
\pgfsetlinewidth{0.803000pt}%
\definecolor{currentstroke}{rgb}{0.000000,0.000000,0.000000}%
\pgfsetstrokecolor{currentstroke}%
\pgfsetdash{}{0pt}%
\pgfsys@defobject{currentmarker}{\pgfqpoint{0.000000in}{-0.048611in}}{\pgfqpoint{0.000000in}{0.000000in}}{%
\pgfpathmoveto{\pgfqpoint{0.000000in}{0.000000in}}%
\pgfpathlineto{\pgfqpoint{0.000000in}{-0.048611in}}%
\pgfusepath{stroke,fill}%
}%
\begin{pgfscope}%
\pgfsys@transformshift{3.529873in}{0.548769in}%
\pgfsys@useobject{currentmarker}{}%
\end{pgfscope}%
\end{pgfscope}%
\begin{pgfscope}%
\definecolor{textcolor}{rgb}{0.000000,0.000000,0.000000}%
\pgfsetstrokecolor{textcolor}%
\pgfsetfillcolor{textcolor}%
\pgftext[x=3.529873in,y=0.451547in,,top]{\color{textcolor}\sffamily\fontsize{10.000000}{12.000000}\selectfont \(\displaystyle {60000}\)}%
\end{pgfscope}%
\begin{pgfscope}%
\pgfsetbuttcap%
\pgfsetroundjoin%
\definecolor{currentfill}{rgb}{0.000000,0.000000,0.000000}%
\pgfsetfillcolor{currentfill}%
\pgfsetlinewidth{0.803000pt}%
\definecolor{currentstroke}{rgb}{0.000000,0.000000,0.000000}%
\pgfsetstrokecolor{currentstroke}%
\pgfsetdash{}{0pt}%
\pgfsys@defobject{currentmarker}{\pgfqpoint{0.000000in}{-0.048611in}}{\pgfqpoint{0.000000in}{0.000000in}}{%
\pgfpathmoveto{\pgfqpoint{0.000000in}{0.000000in}}%
\pgfpathlineto{\pgfqpoint{0.000000in}{-0.048611in}}%
\pgfusepath{stroke,fill}%
}%
\begin{pgfscope}%
\pgfsys@transformshift{4.421622in}{0.548769in}%
\pgfsys@useobject{currentmarker}{}%
\end{pgfscope}%
\end{pgfscope}%
\begin{pgfscope}%
\definecolor{textcolor}{rgb}{0.000000,0.000000,0.000000}%
\pgfsetstrokecolor{textcolor}%
\pgfsetfillcolor{textcolor}%
\pgftext[x=4.421622in,y=0.451547in,,top]{\color{textcolor}\sffamily\fontsize{10.000000}{12.000000}\selectfont \(\displaystyle {80000}\)}%
\end{pgfscope}%
\begin{pgfscope}%
\pgfsetbuttcap%
\pgfsetroundjoin%
\definecolor{currentfill}{rgb}{0.000000,0.000000,0.000000}%
\pgfsetfillcolor{currentfill}%
\pgfsetlinewidth{0.803000pt}%
\definecolor{currentstroke}{rgb}{0.000000,0.000000,0.000000}%
\pgfsetstrokecolor{currentstroke}%
\pgfsetdash{}{0pt}%
\pgfsys@defobject{currentmarker}{\pgfqpoint{0.000000in}{-0.048611in}}{\pgfqpoint{0.000000in}{0.000000in}}{%
\pgfpathmoveto{\pgfqpoint{0.000000in}{0.000000in}}%
\pgfpathlineto{\pgfqpoint{0.000000in}{-0.048611in}}%
\pgfusepath{stroke,fill}%
}%
\begin{pgfscope}%
\pgfsys@transformshift{5.313370in}{0.548769in}%
\pgfsys@useobject{currentmarker}{}%
\end{pgfscope}%
\end{pgfscope}%
\begin{pgfscope}%
\definecolor{textcolor}{rgb}{0.000000,0.000000,0.000000}%
\pgfsetstrokecolor{textcolor}%
\pgfsetfillcolor{textcolor}%
\pgftext[x=5.313370in,y=0.451547in,,top]{\color{textcolor}\sffamily\fontsize{10.000000}{12.000000}\selectfont \(\displaystyle {100000}\)}%
\end{pgfscope}%
\begin{pgfscope}%
\definecolor{textcolor}{rgb}{0.000000,0.000000,0.000000}%
\pgfsetstrokecolor{textcolor}%
\pgfsetfillcolor{textcolor}%
\pgftext[x=3.249352in,y=0.272658in,,top]{\color{textcolor}\sffamily\fontsize{10.000000}{12.000000}\selectfont Classes}%
\end{pgfscope}%
\begin{pgfscope}%
\pgfsetbuttcap%
\pgfsetroundjoin%
\definecolor{currentfill}{rgb}{0.000000,0.000000,0.000000}%
\pgfsetfillcolor{currentfill}%
\pgfsetlinewidth{0.803000pt}%
\definecolor{currentstroke}{rgb}{0.000000,0.000000,0.000000}%
\pgfsetstrokecolor{currentstroke}%
\pgfsetdash{}{0pt}%
\pgfsys@defobject{currentmarker}{\pgfqpoint{-0.048611in}{0.000000in}}{\pgfqpoint{0.000000in}{0.000000in}}{%
\pgfpathmoveto{\pgfqpoint{0.000000in}{0.000000in}}%
\pgfpathlineto{\pgfqpoint{-0.048611in}{0.000000in}}%
\pgfusepath{stroke,fill}%
}%
\begin{pgfscope}%
\pgfsys@transformshift{0.648703in}{0.689796in}%
\pgfsys@useobject{currentmarker}{}%
\end{pgfscope}%
\end{pgfscope}%
\begin{pgfscope}%
\definecolor{textcolor}{rgb}{0.000000,0.000000,0.000000}%
\pgfsetstrokecolor{textcolor}%
\pgfsetfillcolor{textcolor}%
\pgftext[x=0.482036in, y=0.641601in, left, base]{\color{textcolor}\sffamily\fontsize{10.000000}{12.000000}\selectfont \(\displaystyle {0}\)}%
\end{pgfscope}%
\begin{pgfscope}%
\pgfsetbuttcap%
\pgfsetroundjoin%
\definecolor{currentfill}{rgb}{0.000000,0.000000,0.000000}%
\pgfsetfillcolor{currentfill}%
\pgfsetlinewidth{0.803000pt}%
\definecolor{currentstroke}{rgb}{0.000000,0.000000,0.000000}%
\pgfsetstrokecolor{currentstroke}%
\pgfsetdash{}{0pt}%
\pgfsys@defobject{currentmarker}{\pgfqpoint{-0.048611in}{0.000000in}}{\pgfqpoint{0.000000in}{0.000000in}}{%
\pgfpathmoveto{\pgfqpoint{0.000000in}{0.000000in}}%
\pgfpathlineto{\pgfqpoint{-0.048611in}{0.000000in}}%
\pgfusepath{stroke,fill}%
}%
\begin{pgfscope}%
\pgfsys@transformshift{0.648703in}{1.104580in}%
\pgfsys@useobject{currentmarker}{}%
\end{pgfscope}%
\end{pgfscope}%
\begin{pgfscope}%
\definecolor{textcolor}{rgb}{0.000000,0.000000,0.000000}%
\pgfsetstrokecolor{textcolor}%
\pgfsetfillcolor{textcolor}%
\pgftext[x=0.343147in, y=1.056386in, left, base]{\color{textcolor}\sffamily\fontsize{10.000000}{12.000000}\selectfont \(\displaystyle {100}\)}%
\end{pgfscope}%
\begin{pgfscope}%
\pgfsetbuttcap%
\pgfsetroundjoin%
\definecolor{currentfill}{rgb}{0.000000,0.000000,0.000000}%
\pgfsetfillcolor{currentfill}%
\pgfsetlinewidth{0.803000pt}%
\definecolor{currentstroke}{rgb}{0.000000,0.000000,0.000000}%
\pgfsetstrokecolor{currentstroke}%
\pgfsetdash{}{0pt}%
\pgfsys@defobject{currentmarker}{\pgfqpoint{-0.048611in}{0.000000in}}{\pgfqpoint{0.000000in}{0.000000in}}{%
\pgfpathmoveto{\pgfqpoint{0.000000in}{0.000000in}}%
\pgfpathlineto{\pgfqpoint{-0.048611in}{0.000000in}}%
\pgfusepath{stroke,fill}%
}%
\begin{pgfscope}%
\pgfsys@transformshift{0.648703in}{1.519365in}%
\pgfsys@useobject{currentmarker}{}%
\end{pgfscope}%
\end{pgfscope}%
\begin{pgfscope}%
\definecolor{textcolor}{rgb}{0.000000,0.000000,0.000000}%
\pgfsetstrokecolor{textcolor}%
\pgfsetfillcolor{textcolor}%
\pgftext[x=0.343147in, y=1.471171in, left, base]{\color{textcolor}\sffamily\fontsize{10.000000}{12.000000}\selectfont \(\displaystyle {200}\)}%
\end{pgfscope}%
\begin{pgfscope}%
\pgfsetbuttcap%
\pgfsetroundjoin%
\definecolor{currentfill}{rgb}{0.000000,0.000000,0.000000}%
\pgfsetfillcolor{currentfill}%
\pgfsetlinewidth{0.803000pt}%
\definecolor{currentstroke}{rgb}{0.000000,0.000000,0.000000}%
\pgfsetstrokecolor{currentstroke}%
\pgfsetdash{}{0pt}%
\pgfsys@defobject{currentmarker}{\pgfqpoint{-0.048611in}{0.000000in}}{\pgfqpoint{0.000000in}{0.000000in}}{%
\pgfpathmoveto{\pgfqpoint{0.000000in}{0.000000in}}%
\pgfpathlineto{\pgfqpoint{-0.048611in}{0.000000in}}%
\pgfusepath{stroke,fill}%
}%
\begin{pgfscope}%
\pgfsys@transformshift{0.648703in}{1.934150in}%
\pgfsys@useobject{currentmarker}{}%
\end{pgfscope}%
\end{pgfscope}%
\begin{pgfscope}%
\definecolor{textcolor}{rgb}{0.000000,0.000000,0.000000}%
\pgfsetstrokecolor{textcolor}%
\pgfsetfillcolor{textcolor}%
\pgftext[x=0.343147in, y=1.885955in, left, base]{\color{textcolor}\sffamily\fontsize{10.000000}{12.000000}\selectfont \(\displaystyle {300}\)}%
\end{pgfscope}%
\begin{pgfscope}%
\pgfsetbuttcap%
\pgfsetroundjoin%
\definecolor{currentfill}{rgb}{0.000000,0.000000,0.000000}%
\pgfsetfillcolor{currentfill}%
\pgfsetlinewidth{0.803000pt}%
\definecolor{currentstroke}{rgb}{0.000000,0.000000,0.000000}%
\pgfsetstrokecolor{currentstroke}%
\pgfsetdash{}{0pt}%
\pgfsys@defobject{currentmarker}{\pgfqpoint{-0.048611in}{0.000000in}}{\pgfqpoint{0.000000in}{0.000000in}}{%
\pgfpathmoveto{\pgfqpoint{0.000000in}{0.000000in}}%
\pgfpathlineto{\pgfqpoint{-0.048611in}{0.000000in}}%
\pgfusepath{stroke,fill}%
}%
\begin{pgfscope}%
\pgfsys@transformshift{0.648703in}{2.348935in}%
\pgfsys@useobject{currentmarker}{}%
\end{pgfscope}%
\end{pgfscope}%
\begin{pgfscope}%
\definecolor{textcolor}{rgb}{0.000000,0.000000,0.000000}%
\pgfsetstrokecolor{textcolor}%
\pgfsetfillcolor{textcolor}%
\pgftext[x=0.343147in, y=2.300740in, left, base]{\color{textcolor}\sffamily\fontsize{10.000000}{12.000000}\selectfont \(\displaystyle {400}\)}%
\end{pgfscope}%
\begin{pgfscope}%
\pgfsetbuttcap%
\pgfsetroundjoin%
\definecolor{currentfill}{rgb}{0.000000,0.000000,0.000000}%
\pgfsetfillcolor{currentfill}%
\pgfsetlinewidth{0.803000pt}%
\definecolor{currentstroke}{rgb}{0.000000,0.000000,0.000000}%
\pgfsetstrokecolor{currentstroke}%
\pgfsetdash{}{0pt}%
\pgfsys@defobject{currentmarker}{\pgfqpoint{-0.048611in}{0.000000in}}{\pgfqpoint{0.000000in}{0.000000in}}{%
\pgfpathmoveto{\pgfqpoint{0.000000in}{0.000000in}}%
\pgfpathlineto{\pgfqpoint{-0.048611in}{0.000000in}}%
\pgfusepath{stroke,fill}%
}%
\begin{pgfscope}%
\pgfsys@transformshift{0.648703in}{2.763719in}%
\pgfsys@useobject{currentmarker}{}%
\end{pgfscope}%
\end{pgfscope}%
\begin{pgfscope}%
\definecolor{textcolor}{rgb}{0.000000,0.000000,0.000000}%
\pgfsetstrokecolor{textcolor}%
\pgfsetfillcolor{textcolor}%
\pgftext[x=0.343147in, y=2.715525in, left, base]{\color{textcolor}\sffamily\fontsize{10.000000}{12.000000}\selectfont \(\displaystyle {500}\)}%
\end{pgfscope}%
\begin{pgfscope}%
\pgfsetbuttcap%
\pgfsetroundjoin%
\definecolor{currentfill}{rgb}{0.000000,0.000000,0.000000}%
\pgfsetfillcolor{currentfill}%
\pgfsetlinewidth{0.803000pt}%
\definecolor{currentstroke}{rgb}{0.000000,0.000000,0.000000}%
\pgfsetstrokecolor{currentstroke}%
\pgfsetdash{}{0pt}%
\pgfsys@defobject{currentmarker}{\pgfqpoint{-0.048611in}{0.000000in}}{\pgfqpoint{0.000000in}{0.000000in}}{%
\pgfpathmoveto{\pgfqpoint{0.000000in}{0.000000in}}%
\pgfpathlineto{\pgfqpoint{-0.048611in}{0.000000in}}%
\pgfusepath{stroke,fill}%
}%
\begin{pgfscope}%
\pgfsys@transformshift{0.648703in}{3.178504in}%
\pgfsys@useobject{currentmarker}{}%
\end{pgfscope}%
\end{pgfscope}%
\begin{pgfscope}%
\definecolor{textcolor}{rgb}{0.000000,0.000000,0.000000}%
\pgfsetstrokecolor{textcolor}%
\pgfsetfillcolor{textcolor}%
\pgftext[x=0.343147in, y=3.130310in, left, base]{\color{textcolor}\sffamily\fontsize{10.000000}{12.000000}\selectfont \(\displaystyle {600}\)}%
\end{pgfscope}%
\begin{pgfscope}%
\pgfsetbuttcap%
\pgfsetroundjoin%
\definecolor{currentfill}{rgb}{0.000000,0.000000,0.000000}%
\pgfsetfillcolor{currentfill}%
\pgfsetlinewidth{0.803000pt}%
\definecolor{currentstroke}{rgb}{0.000000,0.000000,0.000000}%
\pgfsetstrokecolor{currentstroke}%
\pgfsetdash{}{0pt}%
\pgfsys@defobject{currentmarker}{\pgfqpoint{-0.048611in}{0.000000in}}{\pgfqpoint{0.000000in}{0.000000in}}{%
\pgfpathmoveto{\pgfqpoint{0.000000in}{0.000000in}}%
\pgfpathlineto{\pgfqpoint{-0.048611in}{0.000000in}}%
\pgfusepath{stroke,fill}%
}%
\begin{pgfscope}%
\pgfsys@transformshift{0.648703in}{3.593289in}%
\pgfsys@useobject{currentmarker}{}%
\end{pgfscope}%
\end{pgfscope}%
\begin{pgfscope}%
\definecolor{textcolor}{rgb}{0.000000,0.000000,0.000000}%
\pgfsetstrokecolor{textcolor}%
\pgfsetfillcolor{textcolor}%
\pgftext[x=0.343147in, y=3.545094in, left, base]{\color{textcolor}\sffamily\fontsize{10.000000}{12.000000}\selectfont \(\displaystyle {700}\)}%
\end{pgfscope}%
\begin{pgfscope}%
\definecolor{textcolor}{rgb}{0.000000,0.000000,0.000000}%
\pgfsetstrokecolor{textcolor}%
\pgfsetfillcolor{textcolor}%
\pgftext[x=0.287592in,y=2.100064in,,bottom,rotate=90.000000]{\color{textcolor}\sffamily\fontsize{10.000000}{12.000000}\selectfont Data Flow Time (s)}%
\end{pgfscope}%
\begin{pgfscope}%
\pgfsetrectcap%
\pgfsetmiterjoin%
\pgfsetlinewidth{0.803000pt}%
\definecolor{currentstroke}{rgb}{0.000000,0.000000,0.000000}%
\pgfsetstrokecolor{currentstroke}%
\pgfsetdash{}{0pt}%
\pgfpathmoveto{\pgfqpoint{0.648703in}{0.548769in}}%
\pgfpathlineto{\pgfqpoint{0.648703in}{3.651359in}}%
\pgfusepath{stroke}%
\end{pgfscope}%
\begin{pgfscope}%
\pgfsetrectcap%
\pgfsetmiterjoin%
\pgfsetlinewidth{0.803000pt}%
\definecolor{currentstroke}{rgb}{0.000000,0.000000,0.000000}%
\pgfsetstrokecolor{currentstroke}%
\pgfsetdash{}{0pt}%
\pgfpathmoveto{\pgfqpoint{5.850000in}{0.548769in}}%
\pgfpathlineto{\pgfqpoint{5.850000in}{3.651359in}}%
\pgfusepath{stroke}%
\end{pgfscope}%
\begin{pgfscope}%
\pgfsetrectcap%
\pgfsetmiterjoin%
\pgfsetlinewidth{0.803000pt}%
\definecolor{currentstroke}{rgb}{0.000000,0.000000,0.000000}%
\pgfsetstrokecolor{currentstroke}%
\pgfsetdash{}{0pt}%
\pgfpathmoveto{\pgfqpoint{0.648703in}{0.548769in}}%
\pgfpathlineto{\pgfqpoint{5.850000in}{0.548769in}}%
\pgfusepath{stroke}%
\end{pgfscope}%
\begin{pgfscope}%
\pgfsetrectcap%
\pgfsetmiterjoin%
\pgfsetlinewidth{0.803000pt}%
\definecolor{currentstroke}{rgb}{0.000000,0.000000,0.000000}%
\pgfsetstrokecolor{currentstroke}%
\pgfsetdash{}{0pt}%
\pgfpathmoveto{\pgfqpoint{0.648703in}{3.651359in}}%
\pgfpathlineto{\pgfqpoint{5.850000in}{3.651359in}}%
\pgfusepath{stroke}%
\end{pgfscope}%
\begin{pgfscope}%
\definecolor{textcolor}{rgb}{0.000000,0.000000,0.000000}%
\pgfsetstrokecolor{textcolor}%
\pgfsetfillcolor{textcolor}%
\pgftext[x=3.249352in,y=3.734692in,,base]{\color{textcolor}\sffamily\fontsize{12.000000}{14.400000}\selectfont Forward}%
\end{pgfscope}%
\begin{pgfscope}%
\pgfsetbuttcap%
\pgfsetmiterjoin%
\definecolor{currentfill}{rgb}{1.000000,1.000000,1.000000}%
\pgfsetfillcolor{currentfill}%
\pgfsetfillopacity{0.800000}%
\pgfsetlinewidth{1.003750pt}%
\definecolor{currentstroke}{rgb}{0.800000,0.800000,0.800000}%
\pgfsetstrokecolor{currentstroke}%
\pgfsetstrokeopacity{0.800000}%
\pgfsetdash{}{0pt}%
\pgfpathmoveto{\pgfqpoint{4.300417in}{0.618213in}}%
\pgfpathlineto{\pgfqpoint{5.752778in}{0.618213in}}%
\pgfpathquadraticcurveto{\pgfqpoint{5.780556in}{0.618213in}}{\pgfqpoint{5.780556in}{0.645991in}}%
\pgfpathlineto{\pgfqpoint{5.780556in}{1.214463in}}%
\pgfpathquadraticcurveto{\pgfqpoint{5.780556in}{1.242241in}}{\pgfqpoint{5.752778in}{1.242241in}}%
\pgfpathlineto{\pgfqpoint{4.300417in}{1.242241in}}%
\pgfpathquadraticcurveto{\pgfqpoint{4.272639in}{1.242241in}}{\pgfqpoint{4.272639in}{1.214463in}}%
\pgfpathlineto{\pgfqpoint{4.272639in}{0.645991in}}%
\pgfpathquadraticcurveto{\pgfqpoint{4.272639in}{0.618213in}}{\pgfqpoint{4.300417in}{0.618213in}}%
\pgfpathclose%
\pgfusepath{stroke,fill}%
\end{pgfscope}%
\begin{pgfscope}%
\pgfsetbuttcap%
\pgfsetroundjoin%
\definecolor{currentfill}{rgb}{0.121569,0.466667,0.705882}%
\pgfsetfillcolor{currentfill}%
\pgfsetlinewidth{1.003750pt}%
\definecolor{currentstroke}{rgb}{0.121569,0.466667,0.705882}%
\pgfsetstrokecolor{currentstroke}%
\pgfsetdash{}{0pt}%
\pgfsys@defobject{currentmarker}{\pgfqpoint{-0.034722in}{-0.034722in}}{\pgfqpoint{0.034722in}{0.034722in}}{%
\pgfpathmoveto{\pgfqpoint{0.000000in}{-0.034722in}}%
\pgfpathcurveto{\pgfqpoint{0.009208in}{-0.034722in}}{\pgfqpoint{0.018041in}{-0.031064in}}{\pgfqpoint{0.024552in}{-0.024552in}}%
\pgfpathcurveto{\pgfqpoint{0.031064in}{-0.018041in}}{\pgfqpoint{0.034722in}{-0.009208in}}{\pgfqpoint{0.034722in}{0.000000in}}%
\pgfpathcurveto{\pgfqpoint{0.034722in}{0.009208in}}{\pgfqpoint{0.031064in}{0.018041in}}{\pgfqpoint{0.024552in}{0.024552in}}%
\pgfpathcurveto{\pgfqpoint{0.018041in}{0.031064in}}{\pgfqpoint{0.009208in}{0.034722in}}{\pgfqpoint{0.000000in}{0.034722in}}%
\pgfpathcurveto{\pgfqpoint{-0.009208in}{0.034722in}}{\pgfqpoint{-0.018041in}{0.031064in}}{\pgfqpoint{-0.024552in}{0.024552in}}%
\pgfpathcurveto{\pgfqpoint{-0.031064in}{0.018041in}}{\pgfqpoint{-0.034722in}{0.009208in}}{\pgfqpoint{-0.034722in}{0.000000in}}%
\pgfpathcurveto{\pgfqpoint{-0.034722in}{-0.009208in}}{\pgfqpoint{-0.031064in}{-0.018041in}}{\pgfqpoint{-0.024552in}{-0.024552in}}%
\pgfpathcurveto{\pgfqpoint{-0.018041in}{-0.031064in}}{\pgfqpoint{-0.009208in}{-0.034722in}}{\pgfqpoint{0.000000in}{-0.034722in}}%
\pgfpathclose%
\pgfusepath{stroke,fill}%
}%
\begin{pgfscope}%
\pgfsys@transformshift{4.467083in}{1.138074in}%
\pgfsys@useobject{currentmarker}{}%
\end{pgfscope}%
\end{pgfscope}%
\begin{pgfscope}%
\definecolor{textcolor}{rgb}{0.000000,0.000000,0.000000}%
\pgfsetstrokecolor{textcolor}%
\pgfsetfillcolor{textcolor}%
\pgftext[x=4.717083in,y=1.089463in,left,base]{\color{textcolor}\sffamily\fontsize{10.000000}{12.000000}\selectfont No Timeout}%
\end{pgfscope}%
\begin{pgfscope}%
\pgfsetbuttcap%
\pgfsetroundjoin%
\definecolor{currentfill}{rgb}{1.000000,0.498039,0.054902}%
\pgfsetfillcolor{currentfill}%
\pgfsetlinewidth{1.003750pt}%
\definecolor{currentstroke}{rgb}{1.000000,0.498039,0.054902}%
\pgfsetstrokecolor{currentstroke}%
\pgfsetdash{}{0pt}%
\pgfsys@defobject{currentmarker}{\pgfqpoint{-0.034722in}{-0.034722in}}{\pgfqpoint{0.034722in}{0.034722in}}{%
\pgfpathmoveto{\pgfqpoint{0.000000in}{-0.034722in}}%
\pgfpathcurveto{\pgfqpoint{0.009208in}{-0.034722in}}{\pgfqpoint{0.018041in}{-0.031064in}}{\pgfqpoint{0.024552in}{-0.024552in}}%
\pgfpathcurveto{\pgfqpoint{0.031064in}{-0.018041in}}{\pgfqpoint{0.034722in}{-0.009208in}}{\pgfqpoint{0.034722in}{0.000000in}}%
\pgfpathcurveto{\pgfqpoint{0.034722in}{0.009208in}}{\pgfqpoint{0.031064in}{0.018041in}}{\pgfqpoint{0.024552in}{0.024552in}}%
\pgfpathcurveto{\pgfqpoint{0.018041in}{0.031064in}}{\pgfqpoint{0.009208in}{0.034722in}}{\pgfqpoint{0.000000in}{0.034722in}}%
\pgfpathcurveto{\pgfqpoint{-0.009208in}{0.034722in}}{\pgfqpoint{-0.018041in}{0.031064in}}{\pgfqpoint{-0.024552in}{0.024552in}}%
\pgfpathcurveto{\pgfqpoint{-0.031064in}{0.018041in}}{\pgfqpoint{-0.034722in}{0.009208in}}{\pgfqpoint{-0.034722in}{0.000000in}}%
\pgfpathcurveto{\pgfqpoint{-0.034722in}{-0.009208in}}{\pgfqpoint{-0.031064in}{-0.018041in}}{\pgfqpoint{-0.024552in}{-0.024552in}}%
\pgfpathcurveto{\pgfqpoint{-0.018041in}{-0.031064in}}{\pgfqpoint{-0.009208in}{-0.034722in}}{\pgfqpoint{0.000000in}{-0.034722in}}%
\pgfpathclose%
\pgfusepath{stroke,fill}%
}%
\begin{pgfscope}%
\pgfsys@transformshift{4.467083in}{0.944463in}%
\pgfsys@useobject{currentmarker}{}%
\end{pgfscope}%
\end{pgfscope}%
\begin{pgfscope}%
\definecolor{textcolor}{rgb}{0.000000,0.000000,0.000000}%
\pgfsetstrokecolor{textcolor}%
\pgfsetfillcolor{textcolor}%
\pgftext[x=4.717083in,y=0.895852in,left,base]{\color{textcolor}\sffamily\fontsize{10.000000}{12.000000}\selectfont Time Timeout}%
\end{pgfscope}%
\begin{pgfscope}%
\pgfsetbuttcap%
\pgfsetroundjoin%
\definecolor{currentfill}{rgb}{0.839216,0.152941,0.156863}%
\pgfsetfillcolor{currentfill}%
\pgfsetlinewidth{1.003750pt}%
\definecolor{currentstroke}{rgb}{0.839216,0.152941,0.156863}%
\pgfsetstrokecolor{currentstroke}%
\pgfsetdash{}{0pt}%
\pgfsys@defobject{currentmarker}{\pgfqpoint{-0.034722in}{-0.034722in}}{\pgfqpoint{0.034722in}{0.034722in}}{%
\pgfpathmoveto{\pgfqpoint{0.000000in}{-0.034722in}}%
\pgfpathcurveto{\pgfqpoint{0.009208in}{-0.034722in}}{\pgfqpoint{0.018041in}{-0.031064in}}{\pgfqpoint{0.024552in}{-0.024552in}}%
\pgfpathcurveto{\pgfqpoint{0.031064in}{-0.018041in}}{\pgfqpoint{0.034722in}{-0.009208in}}{\pgfqpoint{0.034722in}{0.000000in}}%
\pgfpathcurveto{\pgfqpoint{0.034722in}{0.009208in}}{\pgfqpoint{0.031064in}{0.018041in}}{\pgfqpoint{0.024552in}{0.024552in}}%
\pgfpathcurveto{\pgfqpoint{0.018041in}{0.031064in}}{\pgfqpoint{0.009208in}{0.034722in}}{\pgfqpoint{0.000000in}{0.034722in}}%
\pgfpathcurveto{\pgfqpoint{-0.009208in}{0.034722in}}{\pgfqpoint{-0.018041in}{0.031064in}}{\pgfqpoint{-0.024552in}{0.024552in}}%
\pgfpathcurveto{\pgfqpoint{-0.031064in}{0.018041in}}{\pgfqpoint{-0.034722in}{0.009208in}}{\pgfqpoint{-0.034722in}{0.000000in}}%
\pgfpathcurveto{\pgfqpoint{-0.034722in}{-0.009208in}}{\pgfqpoint{-0.031064in}{-0.018041in}}{\pgfqpoint{-0.024552in}{-0.024552in}}%
\pgfpathcurveto{\pgfqpoint{-0.018041in}{-0.031064in}}{\pgfqpoint{-0.009208in}{-0.034722in}}{\pgfqpoint{0.000000in}{-0.034722in}}%
\pgfpathclose%
\pgfusepath{stroke,fill}%
}%
\begin{pgfscope}%
\pgfsys@transformshift{4.467083in}{0.750852in}%
\pgfsys@useobject{currentmarker}{}%
\end{pgfscope}%
\end{pgfscope}%
\begin{pgfscope}%
\definecolor{textcolor}{rgb}{0.000000,0.000000,0.000000}%
\pgfsetstrokecolor{textcolor}%
\pgfsetfillcolor{textcolor}%
\pgftext[x=4.717083in,y=0.702241in,left,base]{\color{textcolor}\sffamily\fontsize{10.000000}{12.000000}\selectfont Memory Timeout}%
\end{pgfscope}%
\end{pgfpicture}%
\makeatother%
\endgroup%

                }
            \end{subfigure}
            \qquad
            \begin{subfigure}[]{0.45\textwidth}
                \centering
                \resizebox{\columnwidth}{!}{
                    %% Creator: Matplotlib, PGF backend
%%
%% To include the figure in your LaTeX document, write
%%   \input{<filename>.pgf}
%%
%% Make sure the required packages are loaded in your preamble
%%   \usepackage{pgf}
%%
%% and, on pdftex
%%   \usepackage[utf8]{inputenc}\DeclareUnicodeCharacter{2212}{-}
%%
%% or, on luatex and xetex
%%   \usepackage{unicode-math}
%%
%% Figures using additional raster images can only be included by \input if
%% they are in the same directory as the main LaTeX file. For loading figures
%% from other directories you can use the `import` package
%%   \usepackage{import}
%%
%% and then include the figures with
%%   \import{<path to file>}{<filename>.pgf}
%%
%% Matplotlib used the following preamble
%%   \usepackage{amsmath}
%%   \usepackage{fontspec}
%%
\begingroup%
\makeatletter%
\begin{pgfpicture}%
\pgfpathrectangle{\pgfpointorigin}{\pgfqpoint{6.000000in}{4.000000in}}%
\pgfusepath{use as bounding box, clip}%
\begin{pgfscope}%
\pgfsetbuttcap%
\pgfsetmiterjoin%
\definecolor{currentfill}{rgb}{1.000000,1.000000,1.000000}%
\pgfsetfillcolor{currentfill}%
\pgfsetlinewidth{0.000000pt}%
\definecolor{currentstroke}{rgb}{1.000000,1.000000,1.000000}%
\pgfsetstrokecolor{currentstroke}%
\pgfsetdash{}{0pt}%
\pgfpathmoveto{\pgfqpoint{0.000000in}{0.000000in}}%
\pgfpathlineto{\pgfqpoint{6.000000in}{0.000000in}}%
\pgfpathlineto{\pgfqpoint{6.000000in}{4.000000in}}%
\pgfpathlineto{\pgfqpoint{0.000000in}{4.000000in}}%
\pgfpathclose%
\pgfusepath{fill}%
\end{pgfscope}%
\begin{pgfscope}%
\pgfsetbuttcap%
\pgfsetmiterjoin%
\definecolor{currentfill}{rgb}{1.000000,1.000000,1.000000}%
\pgfsetfillcolor{currentfill}%
\pgfsetlinewidth{0.000000pt}%
\definecolor{currentstroke}{rgb}{0.000000,0.000000,0.000000}%
\pgfsetstrokecolor{currentstroke}%
\pgfsetstrokeopacity{0.000000}%
\pgfsetdash{}{0pt}%
\pgfpathmoveto{\pgfqpoint{0.648703in}{0.548769in}}%
\pgfpathlineto{\pgfqpoint{5.850000in}{0.548769in}}%
\pgfpathlineto{\pgfqpoint{5.850000in}{3.651359in}}%
\pgfpathlineto{\pgfqpoint{0.648703in}{3.651359in}}%
\pgfpathclose%
\pgfusepath{fill}%
\end{pgfscope}%
\begin{pgfscope}%
\pgfpathrectangle{\pgfqpoint{0.648703in}{0.548769in}}{\pgfqpoint{5.201297in}{3.102590in}}%
\pgfusepath{clip}%
\pgfsetbuttcap%
\pgfsetroundjoin%
\definecolor{currentfill}{rgb}{0.121569,0.466667,0.705882}%
\pgfsetfillcolor{currentfill}%
\pgfsetlinewidth{1.003750pt}%
\definecolor{currentstroke}{rgb}{0.121569,0.466667,0.705882}%
\pgfsetstrokecolor{currentstroke}%
\pgfsetdash{}{0pt}%
\pgfpathmoveto{\pgfqpoint{1.215563in}{0.673501in}}%
\pgfpathcurveto{\pgfqpoint{1.226613in}{0.673501in}}{\pgfqpoint{1.237213in}{0.677891in}}{\pgfqpoint{1.245026in}{0.685705in}}%
\pgfpathcurveto{\pgfqpoint{1.252840in}{0.693519in}}{\pgfqpoint{1.257230in}{0.704118in}}{\pgfqpoint{1.257230in}{0.715168in}}%
\pgfpathcurveto{\pgfqpoint{1.257230in}{0.726218in}}{\pgfqpoint{1.252840in}{0.736817in}}{\pgfqpoint{1.245026in}{0.744631in}}%
\pgfpathcurveto{\pgfqpoint{1.237213in}{0.752444in}}{\pgfqpoint{1.226613in}{0.756834in}}{\pgfqpoint{1.215563in}{0.756834in}}%
\pgfpathcurveto{\pgfqpoint{1.204513in}{0.756834in}}{\pgfqpoint{1.193914in}{0.752444in}}{\pgfqpoint{1.186101in}{0.744631in}}%
\pgfpathcurveto{\pgfqpoint{1.178287in}{0.736817in}}{\pgfqpoint{1.173897in}{0.726218in}}{\pgfqpoint{1.173897in}{0.715168in}}%
\pgfpathcurveto{\pgfqpoint{1.173897in}{0.704118in}}{\pgfqpoint{1.178287in}{0.693519in}}{\pgfqpoint{1.186101in}{0.685705in}}%
\pgfpathcurveto{\pgfqpoint{1.193914in}{0.677891in}}{\pgfqpoint{1.204513in}{0.673501in}}{\pgfqpoint{1.215563in}{0.673501in}}%
\pgfpathclose%
\pgfusepath{stroke,fill}%
\end{pgfscope}%
\begin{pgfscope}%
\pgfpathrectangle{\pgfqpoint{0.648703in}{0.548769in}}{\pgfqpoint{5.201297in}{3.102590in}}%
\pgfusepath{clip}%
\pgfsetbuttcap%
\pgfsetroundjoin%
\definecolor{currentfill}{rgb}{0.121569,0.466667,0.705882}%
\pgfsetfillcolor{currentfill}%
\pgfsetlinewidth{1.003750pt}%
\definecolor{currentstroke}{rgb}{0.121569,0.466667,0.705882}%
\pgfsetstrokecolor{currentstroke}%
\pgfsetdash{}{0pt}%
\pgfpathmoveto{\pgfqpoint{3.894777in}{3.176886in}}%
\pgfpathcurveto{\pgfqpoint{3.905827in}{3.176886in}}{\pgfqpoint{3.916426in}{3.181276in}}{\pgfqpoint{3.924240in}{3.189089in}}%
\pgfpathcurveto{\pgfqpoint{3.932053in}{3.196903in}}{\pgfqpoint{3.936444in}{3.207502in}}{\pgfqpoint{3.936444in}{3.218552in}}%
\pgfpathcurveto{\pgfqpoint{3.936444in}{3.229602in}}{\pgfqpoint{3.932053in}{3.240201in}}{\pgfqpoint{3.924240in}{3.248015in}}%
\pgfpathcurveto{\pgfqpoint{3.916426in}{3.255829in}}{\pgfqpoint{3.905827in}{3.260219in}}{\pgfqpoint{3.894777in}{3.260219in}}%
\pgfpathcurveto{\pgfqpoint{3.883727in}{3.260219in}}{\pgfqpoint{3.873128in}{3.255829in}}{\pgfqpoint{3.865314in}{3.248015in}}%
\pgfpathcurveto{\pgfqpoint{3.857501in}{3.240201in}}{\pgfqpoint{3.853110in}{3.229602in}}{\pgfqpoint{3.853110in}{3.218552in}}%
\pgfpathcurveto{\pgfqpoint{3.853110in}{3.207502in}}{\pgfqpoint{3.857501in}{3.196903in}}{\pgfqpoint{3.865314in}{3.189089in}}%
\pgfpathcurveto{\pgfqpoint{3.873128in}{3.181276in}}{\pgfqpoint{3.883727in}{3.176886in}}{\pgfqpoint{3.894777in}{3.176886in}}%
\pgfpathclose%
\pgfusepath{stroke,fill}%
\end{pgfscope}%
\begin{pgfscope}%
\pgfpathrectangle{\pgfqpoint{0.648703in}{0.548769in}}{\pgfqpoint{5.201297in}{3.102590in}}%
\pgfusepath{clip}%
\pgfsetbuttcap%
\pgfsetroundjoin%
\definecolor{currentfill}{rgb}{1.000000,0.498039,0.054902}%
\pgfsetfillcolor{currentfill}%
\pgfsetlinewidth{1.003750pt}%
\definecolor{currentstroke}{rgb}{1.000000,0.498039,0.054902}%
\pgfsetstrokecolor{currentstroke}%
\pgfsetdash{}{0pt}%
\pgfpathmoveto{\pgfqpoint{1.224080in}{3.198029in}}%
\pgfpathcurveto{\pgfqpoint{1.235130in}{3.198029in}}{\pgfqpoint{1.245729in}{3.202419in}}{\pgfqpoint{1.253542in}{3.210233in}}%
\pgfpathcurveto{\pgfqpoint{1.261356in}{3.218046in}}{\pgfqpoint{1.265746in}{3.228646in}}{\pgfqpoint{1.265746in}{3.239696in}}%
\pgfpathcurveto{\pgfqpoint{1.265746in}{3.250746in}}{\pgfqpoint{1.261356in}{3.261345in}}{\pgfqpoint{1.253542in}{3.269158in}}%
\pgfpathcurveto{\pgfqpoint{1.245729in}{3.276972in}}{\pgfqpoint{1.235130in}{3.281362in}}{\pgfqpoint{1.224080in}{3.281362in}}%
\pgfpathcurveto{\pgfqpoint{1.213029in}{3.281362in}}{\pgfqpoint{1.202430in}{3.276972in}}{\pgfqpoint{1.194617in}{3.269158in}}%
\pgfpathcurveto{\pgfqpoint{1.186803in}{3.261345in}}{\pgfqpoint{1.182413in}{3.250746in}}{\pgfqpoint{1.182413in}{3.239696in}}%
\pgfpathcurveto{\pgfqpoint{1.182413in}{3.228646in}}{\pgfqpoint{1.186803in}{3.218046in}}{\pgfqpoint{1.194617in}{3.210233in}}%
\pgfpathcurveto{\pgfqpoint{1.202430in}{3.202419in}}{\pgfqpoint{1.213029in}{3.198029in}}{\pgfqpoint{1.224080in}{3.198029in}}%
\pgfpathclose%
\pgfusepath{stroke,fill}%
\end{pgfscope}%
\begin{pgfscope}%
\pgfpathrectangle{\pgfqpoint{0.648703in}{0.548769in}}{\pgfqpoint{5.201297in}{3.102590in}}%
\pgfusepath{clip}%
\pgfsetbuttcap%
\pgfsetroundjoin%
\definecolor{currentfill}{rgb}{1.000000,0.498039,0.054902}%
\pgfsetfillcolor{currentfill}%
\pgfsetlinewidth{1.003750pt}%
\definecolor{currentstroke}{rgb}{1.000000,0.498039,0.054902}%
\pgfsetstrokecolor{currentstroke}%
\pgfsetdash{}{0pt}%
\pgfpathmoveto{\pgfqpoint{2.495891in}{3.185343in}}%
\pgfpathcurveto{\pgfqpoint{2.506941in}{3.185343in}}{\pgfqpoint{2.517540in}{3.189733in}}{\pgfqpoint{2.525354in}{3.197547in}}%
\pgfpathcurveto{\pgfqpoint{2.533168in}{3.205360in}}{\pgfqpoint{2.537558in}{3.215959in}}{\pgfqpoint{2.537558in}{3.227010in}}%
\pgfpathcurveto{\pgfqpoint{2.537558in}{3.238060in}}{\pgfqpoint{2.533168in}{3.248659in}}{\pgfqpoint{2.525354in}{3.256472in}}%
\pgfpathcurveto{\pgfqpoint{2.517540in}{3.264286in}}{\pgfqpoint{2.506941in}{3.268676in}}{\pgfqpoint{2.495891in}{3.268676in}}%
\pgfpathcurveto{\pgfqpoint{2.484841in}{3.268676in}}{\pgfqpoint{2.474242in}{3.264286in}}{\pgfqpoint{2.466428in}{3.256472in}}%
\pgfpathcurveto{\pgfqpoint{2.458615in}{3.248659in}}{\pgfqpoint{2.454225in}{3.238060in}}{\pgfqpoint{2.454225in}{3.227010in}}%
\pgfpathcurveto{\pgfqpoint{2.454225in}{3.215959in}}{\pgfqpoint{2.458615in}{3.205360in}}{\pgfqpoint{2.466428in}{3.197547in}}%
\pgfpathcurveto{\pgfqpoint{2.474242in}{3.189733in}}{\pgfqpoint{2.484841in}{3.185343in}}{\pgfqpoint{2.495891in}{3.185343in}}%
\pgfpathclose%
\pgfusepath{stroke,fill}%
\end{pgfscope}%
\begin{pgfscope}%
\pgfpathrectangle{\pgfqpoint{0.648703in}{0.548769in}}{\pgfqpoint{5.201297in}{3.102590in}}%
\pgfusepath{clip}%
\pgfsetbuttcap%
\pgfsetroundjoin%
\definecolor{currentfill}{rgb}{0.121569,0.466667,0.705882}%
\pgfsetfillcolor{currentfill}%
\pgfsetlinewidth{1.003750pt}%
\definecolor{currentstroke}{rgb}{0.121569,0.466667,0.705882}%
\pgfsetstrokecolor{currentstroke}%
\pgfsetdash{}{0pt}%
\pgfpathmoveto{\pgfqpoint{2.644323in}{0.652358in}}%
\pgfpathcurveto{\pgfqpoint{2.655373in}{0.652358in}}{\pgfqpoint{2.665972in}{0.656748in}}{\pgfqpoint{2.673785in}{0.664562in}}%
\pgfpathcurveto{\pgfqpoint{2.681599in}{0.672375in}}{\pgfqpoint{2.685989in}{0.682974in}}{\pgfqpoint{2.685989in}{0.694024in}}%
\pgfpathcurveto{\pgfqpoint{2.685989in}{0.705074in}}{\pgfqpoint{2.681599in}{0.715673in}}{\pgfqpoint{2.673785in}{0.723487in}}%
\pgfpathcurveto{\pgfqpoint{2.665972in}{0.731301in}}{\pgfqpoint{2.655373in}{0.735691in}}{\pgfqpoint{2.644323in}{0.735691in}}%
\pgfpathcurveto{\pgfqpoint{2.633273in}{0.735691in}}{\pgfqpoint{2.622674in}{0.731301in}}{\pgfqpoint{2.614860in}{0.723487in}}%
\pgfpathcurveto{\pgfqpoint{2.607046in}{0.715673in}}{\pgfqpoint{2.602656in}{0.705074in}}{\pgfqpoint{2.602656in}{0.694024in}}%
\pgfpathcurveto{\pgfqpoint{2.602656in}{0.682974in}}{\pgfqpoint{2.607046in}{0.672375in}}{\pgfqpoint{2.614860in}{0.664562in}}%
\pgfpathcurveto{\pgfqpoint{2.622674in}{0.656748in}}{\pgfqpoint{2.633273in}{0.652358in}}{\pgfqpoint{2.644323in}{0.652358in}}%
\pgfpathclose%
\pgfusepath{stroke,fill}%
\end{pgfscope}%
\begin{pgfscope}%
\pgfpathrectangle{\pgfqpoint{0.648703in}{0.548769in}}{\pgfqpoint{5.201297in}{3.102590in}}%
\pgfusepath{clip}%
\pgfsetbuttcap%
\pgfsetroundjoin%
\definecolor{currentfill}{rgb}{0.121569,0.466667,0.705882}%
\pgfsetfillcolor{currentfill}%
\pgfsetlinewidth{1.003750pt}%
\definecolor{currentstroke}{rgb}{0.121569,0.466667,0.705882}%
\pgfsetstrokecolor{currentstroke}%
\pgfsetdash{}{0pt}%
\pgfpathmoveto{\pgfqpoint{2.110923in}{3.181114in}}%
\pgfpathcurveto{\pgfqpoint{2.121974in}{3.181114in}}{\pgfqpoint{2.132573in}{3.185504in}}{\pgfqpoint{2.140386in}{3.193318in}}%
\pgfpathcurveto{\pgfqpoint{2.148200in}{3.201132in}}{\pgfqpoint{2.152590in}{3.211731in}}{\pgfqpoint{2.152590in}{3.222781in}}%
\pgfpathcurveto{\pgfqpoint{2.152590in}{3.233831in}}{\pgfqpoint{2.148200in}{3.244430in}}{\pgfqpoint{2.140386in}{3.252244in}}%
\pgfpathcurveto{\pgfqpoint{2.132573in}{3.260057in}}{\pgfqpoint{2.121974in}{3.264448in}}{\pgfqpoint{2.110923in}{3.264448in}}%
\pgfpathcurveto{\pgfqpoint{2.099873in}{3.264448in}}{\pgfqpoint{2.089274in}{3.260057in}}{\pgfqpoint{2.081461in}{3.252244in}}%
\pgfpathcurveto{\pgfqpoint{2.073647in}{3.244430in}}{\pgfqpoint{2.069257in}{3.233831in}}{\pgfqpoint{2.069257in}{3.222781in}}%
\pgfpathcurveto{\pgfqpoint{2.069257in}{3.211731in}}{\pgfqpoint{2.073647in}{3.201132in}}{\pgfqpoint{2.081461in}{3.193318in}}%
\pgfpathcurveto{\pgfqpoint{2.089274in}{3.185504in}}{\pgfqpoint{2.099873in}{3.181114in}}{\pgfqpoint{2.110923in}{3.181114in}}%
\pgfpathclose%
\pgfusepath{stroke,fill}%
\end{pgfscope}%
\begin{pgfscope}%
\pgfpathrectangle{\pgfqpoint{0.648703in}{0.548769in}}{\pgfqpoint{5.201297in}{3.102590in}}%
\pgfusepath{clip}%
\pgfsetbuttcap%
\pgfsetroundjoin%
\definecolor{currentfill}{rgb}{1.000000,0.498039,0.054902}%
\pgfsetfillcolor{currentfill}%
\pgfsetlinewidth{1.003750pt}%
\definecolor{currentstroke}{rgb}{1.000000,0.498039,0.054902}%
\pgfsetstrokecolor{currentstroke}%
\pgfsetdash{}{0pt}%
\pgfpathmoveto{\pgfqpoint{2.253781in}{3.189572in}}%
\pgfpathcurveto{\pgfqpoint{2.264832in}{3.189572in}}{\pgfqpoint{2.275431in}{3.193962in}}{\pgfqpoint{2.283244in}{3.201775in}}%
\pgfpathcurveto{\pgfqpoint{2.291058in}{3.209589in}}{\pgfqpoint{2.295448in}{3.220188in}}{\pgfqpoint{2.295448in}{3.231238in}}%
\pgfpathcurveto{\pgfqpoint{2.295448in}{3.242288in}}{\pgfqpoint{2.291058in}{3.252887in}}{\pgfqpoint{2.283244in}{3.260701in}}%
\pgfpathcurveto{\pgfqpoint{2.275431in}{3.268515in}}{\pgfqpoint{2.264832in}{3.272905in}}{\pgfqpoint{2.253781in}{3.272905in}}%
\pgfpathcurveto{\pgfqpoint{2.242731in}{3.272905in}}{\pgfqpoint{2.232132in}{3.268515in}}{\pgfqpoint{2.224319in}{3.260701in}}%
\pgfpathcurveto{\pgfqpoint{2.216505in}{3.252887in}}{\pgfqpoint{2.212115in}{3.242288in}}{\pgfqpoint{2.212115in}{3.231238in}}%
\pgfpathcurveto{\pgfqpoint{2.212115in}{3.220188in}}{\pgfqpoint{2.216505in}{3.209589in}}{\pgfqpoint{2.224319in}{3.201775in}}%
\pgfpathcurveto{\pgfqpoint{2.232132in}{3.193962in}}{\pgfqpoint{2.242731in}{3.189572in}}{\pgfqpoint{2.253781in}{3.189572in}}%
\pgfpathclose%
\pgfusepath{stroke,fill}%
\end{pgfscope}%
\begin{pgfscope}%
\pgfpathrectangle{\pgfqpoint{0.648703in}{0.548769in}}{\pgfqpoint{5.201297in}{3.102590in}}%
\pgfusepath{clip}%
\pgfsetbuttcap%
\pgfsetroundjoin%
\definecolor{currentfill}{rgb}{0.121569,0.466667,0.705882}%
\pgfsetfillcolor{currentfill}%
\pgfsetlinewidth{1.003750pt}%
\definecolor{currentstroke}{rgb}{0.121569,0.466667,0.705882}%
\pgfsetstrokecolor{currentstroke}%
\pgfsetdash{}{0pt}%
\pgfpathmoveto{\pgfqpoint{1.341255in}{0.648129in}}%
\pgfpathcurveto{\pgfqpoint{1.352305in}{0.648129in}}{\pgfqpoint{1.362904in}{0.652519in}}{\pgfqpoint{1.370718in}{0.660333in}}%
\pgfpathcurveto{\pgfqpoint{1.378532in}{0.668146in}}{\pgfqpoint{1.382922in}{0.678745in}}{\pgfqpoint{1.382922in}{0.689796in}}%
\pgfpathcurveto{\pgfqpoint{1.382922in}{0.700846in}}{\pgfqpoint{1.378532in}{0.711445in}}{\pgfqpoint{1.370718in}{0.719258in}}%
\pgfpathcurveto{\pgfqpoint{1.362904in}{0.727072in}}{\pgfqpoint{1.352305in}{0.731462in}}{\pgfqpoint{1.341255in}{0.731462in}}%
\pgfpathcurveto{\pgfqpoint{1.330205in}{0.731462in}}{\pgfqpoint{1.319606in}{0.727072in}}{\pgfqpoint{1.311793in}{0.719258in}}%
\pgfpathcurveto{\pgfqpoint{1.303979in}{0.711445in}}{\pgfqpoint{1.299589in}{0.700846in}}{\pgfqpoint{1.299589in}{0.689796in}}%
\pgfpathcurveto{\pgfqpoint{1.299589in}{0.678745in}}{\pgfqpoint{1.303979in}{0.668146in}}{\pgfqpoint{1.311793in}{0.660333in}}%
\pgfpathcurveto{\pgfqpoint{1.319606in}{0.652519in}}{\pgfqpoint{1.330205in}{0.648129in}}{\pgfqpoint{1.341255in}{0.648129in}}%
\pgfpathclose%
\pgfusepath{stroke,fill}%
\end{pgfscope}%
\begin{pgfscope}%
\pgfpathrectangle{\pgfqpoint{0.648703in}{0.548769in}}{\pgfqpoint{5.201297in}{3.102590in}}%
\pgfusepath{clip}%
\pgfsetbuttcap%
\pgfsetroundjoin%
\definecolor{currentfill}{rgb}{0.121569,0.466667,0.705882}%
\pgfsetfillcolor{currentfill}%
\pgfsetlinewidth{1.003750pt}%
\definecolor{currentstroke}{rgb}{0.121569,0.466667,0.705882}%
\pgfsetstrokecolor{currentstroke}%
\pgfsetdash{}{0pt}%
\pgfpathmoveto{\pgfqpoint{1.954332in}{0.774990in}}%
\pgfpathcurveto{\pgfqpoint{1.965382in}{0.774990in}}{\pgfqpoint{1.975982in}{0.779380in}}{\pgfqpoint{1.983795in}{0.787194in}}%
\pgfpathcurveto{\pgfqpoint{1.991609in}{0.795007in}}{\pgfqpoint{1.995999in}{0.805606in}}{\pgfqpoint{1.995999in}{0.816656in}}%
\pgfpathcurveto{\pgfqpoint{1.995999in}{0.827706in}}{\pgfqpoint{1.991609in}{0.838305in}}{\pgfqpoint{1.983795in}{0.846119in}}%
\pgfpathcurveto{\pgfqpoint{1.975982in}{0.853933in}}{\pgfqpoint{1.965382in}{0.858323in}}{\pgfqpoint{1.954332in}{0.858323in}}%
\pgfpathcurveto{\pgfqpoint{1.943282in}{0.858323in}}{\pgfqpoint{1.932683in}{0.853933in}}{\pgfqpoint{1.924870in}{0.846119in}}%
\pgfpathcurveto{\pgfqpoint{1.917056in}{0.838305in}}{\pgfqpoint{1.912666in}{0.827706in}}{\pgfqpoint{1.912666in}{0.816656in}}%
\pgfpathcurveto{\pgfqpoint{1.912666in}{0.805606in}}{\pgfqpoint{1.917056in}{0.795007in}}{\pgfqpoint{1.924870in}{0.787194in}}%
\pgfpathcurveto{\pgfqpoint{1.932683in}{0.779380in}}{\pgfqpoint{1.943282in}{0.774990in}}{\pgfqpoint{1.954332in}{0.774990in}}%
\pgfpathclose%
\pgfusepath{stroke,fill}%
\end{pgfscope}%
\begin{pgfscope}%
\pgfpathrectangle{\pgfqpoint{0.648703in}{0.548769in}}{\pgfqpoint{5.201297in}{3.102590in}}%
\pgfusepath{clip}%
\pgfsetbuttcap%
\pgfsetroundjoin%
\definecolor{currentfill}{rgb}{0.121569,0.466667,0.705882}%
\pgfsetfillcolor{currentfill}%
\pgfsetlinewidth{1.003750pt}%
\definecolor{currentstroke}{rgb}{0.121569,0.466667,0.705882}%
\pgfsetstrokecolor{currentstroke}%
\pgfsetdash{}{0pt}%
\pgfpathmoveto{\pgfqpoint{0.885215in}{0.783447in}}%
\pgfpathcurveto{\pgfqpoint{0.896265in}{0.783447in}}{\pgfqpoint{0.906864in}{0.787837in}}{\pgfqpoint{0.914678in}{0.795651in}}%
\pgfpathcurveto{\pgfqpoint{0.922492in}{0.803465in}}{\pgfqpoint{0.926882in}{0.814064in}}{\pgfqpoint{0.926882in}{0.825114in}}%
\pgfpathcurveto{\pgfqpoint{0.926882in}{0.836164in}}{\pgfqpoint{0.922492in}{0.846763in}}{\pgfqpoint{0.914678in}{0.854576in}}%
\pgfpathcurveto{\pgfqpoint{0.906864in}{0.862390in}}{\pgfqpoint{0.896265in}{0.866780in}}{\pgfqpoint{0.885215in}{0.866780in}}%
\pgfpathcurveto{\pgfqpoint{0.874165in}{0.866780in}}{\pgfqpoint{0.863566in}{0.862390in}}{\pgfqpoint{0.855752in}{0.854576in}}%
\pgfpathcurveto{\pgfqpoint{0.847939in}{0.846763in}}{\pgfqpoint{0.843548in}{0.836164in}}{\pgfqpoint{0.843548in}{0.825114in}}%
\pgfpathcurveto{\pgfqpoint{0.843548in}{0.814064in}}{\pgfqpoint{0.847939in}{0.803465in}}{\pgfqpoint{0.855752in}{0.795651in}}%
\pgfpathcurveto{\pgfqpoint{0.863566in}{0.787837in}}{\pgfqpoint{0.874165in}{0.783447in}}{\pgfqpoint{0.885215in}{0.783447in}}%
\pgfpathclose%
\pgfusepath{stroke,fill}%
\end{pgfscope}%
\begin{pgfscope}%
\pgfpathrectangle{\pgfqpoint{0.648703in}{0.548769in}}{\pgfqpoint{5.201297in}{3.102590in}}%
\pgfusepath{clip}%
\pgfsetbuttcap%
\pgfsetroundjoin%
\definecolor{currentfill}{rgb}{0.121569,0.466667,0.705882}%
\pgfsetfillcolor{currentfill}%
\pgfsetlinewidth{1.003750pt}%
\definecolor{currentstroke}{rgb}{0.121569,0.466667,0.705882}%
\pgfsetstrokecolor{currentstroke}%
\pgfsetdash{}{0pt}%
\pgfpathmoveto{\pgfqpoint{2.013009in}{0.652358in}}%
\pgfpathcurveto{\pgfqpoint{2.024060in}{0.652358in}}{\pgfqpoint{2.034659in}{0.656748in}}{\pgfqpoint{2.042472in}{0.664562in}}%
\pgfpathcurveto{\pgfqpoint{2.050286in}{0.672375in}}{\pgfqpoint{2.054676in}{0.682974in}}{\pgfqpoint{2.054676in}{0.694024in}}%
\pgfpathcurveto{\pgfqpoint{2.054676in}{0.705074in}}{\pgfqpoint{2.050286in}{0.715673in}}{\pgfqpoint{2.042472in}{0.723487in}}%
\pgfpathcurveto{\pgfqpoint{2.034659in}{0.731301in}}{\pgfqpoint{2.024060in}{0.735691in}}{\pgfqpoint{2.013009in}{0.735691in}}%
\pgfpathcurveto{\pgfqpoint{2.001959in}{0.735691in}}{\pgfqpoint{1.991360in}{0.731301in}}{\pgfqpoint{1.983547in}{0.723487in}}%
\pgfpathcurveto{\pgfqpoint{1.975733in}{0.715673in}}{\pgfqpoint{1.971343in}{0.705074in}}{\pgfqpoint{1.971343in}{0.694024in}}%
\pgfpathcurveto{\pgfqpoint{1.971343in}{0.682974in}}{\pgfqpoint{1.975733in}{0.672375in}}{\pgfqpoint{1.983547in}{0.664562in}}%
\pgfpathcurveto{\pgfqpoint{1.991360in}{0.656748in}}{\pgfqpoint{2.001959in}{0.652358in}}{\pgfqpoint{2.013009in}{0.652358in}}%
\pgfpathclose%
\pgfusepath{stroke,fill}%
\end{pgfscope}%
\begin{pgfscope}%
\pgfpathrectangle{\pgfqpoint{0.648703in}{0.548769in}}{\pgfqpoint{5.201297in}{3.102590in}}%
\pgfusepath{clip}%
\pgfsetbuttcap%
\pgfsetroundjoin%
\definecolor{currentfill}{rgb}{0.121569,0.466667,0.705882}%
\pgfsetfillcolor{currentfill}%
\pgfsetlinewidth{1.003750pt}%
\definecolor{currentstroke}{rgb}{0.121569,0.466667,0.705882}%
\pgfsetstrokecolor{currentstroke}%
\pgfsetdash{}{0pt}%
\pgfpathmoveto{\pgfqpoint{1.431589in}{0.648129in}}%
\pgfpathcurveto{\pgfqpoint{1.442640in}{0.648129in}}{\pgfqpoint{1.453239in}{0.652519in}}{\pgfqpoint{1.461052in}{0.660333in}}%
\pgfpathcurveto{\pgfqpoint{1.468866in}{0.668146in}}{\pgfqpoint{1.473256in}{0.678745in}}{\pgfqpoint{1.473256in}{0.689796in}}%
\pgfpathcurveto{\pgfqpoint{1.473256in}{0.700846in}}{\pgfqpoint{1.468866in}{0.711445in}}{\pgfqpoint{1.461052in}{0.719258in}}%
\pgfpathcurveto{\pgfqpoint{1.453239in}{0.727072in}}{\pgfqpoint{1.442640in}{0.731462in}}{\pgfqpoint{1.431589in}{0.731462in}}%
\pgfpathcurveto{\pgfqpoint{1.420539in}{0.731462in}}{\pgfqpoint{1.409940in}{0.727072in}}{\pgfqpoint{1.402127in}{0.719258in}}%
\pgfpathcurveto{\pgfqpoint{1.394313in}{0.711445in}}{\pgfqpoint{1.389923in}{0.700846in}}{\pgfqpoint{1.389923in}{0.689796in}}%
\pgfpathcurveto{\pgfqpoint{1.389923in}{0.678745in}}{\pgfqpoint{1.394313in}{0.668146in}}{\pgfqpoint{1.402127in}{0.660333in}}%
\pgfpathcurveto{\pgfqpoint{1.409940in}{0.652519in}}{\pgfqpoint{1.420539in}{0.648129in}}{\pgfqpoint{1.431589in}{0.648129in}}%
\pgfpathclose%
\pgfusepath{stroke,fill}%
\end{pgfscope}%
\begin{pgfscope}%
\pgfpathrectangle{\pgfqpoint{0.648703in}{0.548769in}}{\pgfqpoint{5.201297in}{3.102590in}}%
\pgfusepath{clip}%
\pgfsetbuttcap%
\pgfsetroundjoin%
\definecolor{currentfill}{rgb}{0.121569,0.466667,0.705882}%
\pgfsetfillcolor{currentfill}%
\pgfsetlinewidth{1.003750pt}%
\definecolor{currentstroke}{rgb}{0.121569,0.466667,0.705882}%
\pgfsetstrokecolor{currentstroke}%
\pgfsetdash{}{0pt}%
\pgfpathmoveto{\pgfqpoint{1.616137in}{0.648129in}}%
\pgfpathcurveto{\pgfqpoint{1.627187in}{0.648129in}}{\pgfqpoint{1.637786in}{0.652519in}}{\pgfqpoint{1.645600in}{0.660333in}}%
\pgfpathcurveto{\pgfqpoint{1.653413in}{0.668146in}}{\pgfqpoint{1.657803in}{0.678745in}}{\pgfqpoint{1.657803in}{0.689796in}}%
\pgfpathcurveto{\pgfqpoint{1.657803in}{0.700846in}}{\pgfqpoint{1.653413in}{0.711445in}}{\pgfqpoint{1.645600in}{0.719258in}}%
\pgfpathcurveto{\pgfqpoint{1.637786in}{0.727072in}}{\pgfqpoint{1.627187in}{0.731462in}}{\pgfqpoint{1.616137in}{0.731462in}}%
\pgfpathcurveto{\pgfqpoint{1.605087in}{0.731462in}}{\pgfqpoint{1.594488in}{0.727072in}}{\pgfqpoint{1.586674in}{0.719258in}}%
\pgfpathcurveto{\pgfqpoint{1.578860in}{0.711445in}}{\pgfqpoint{1.574470in}{0.700846in}}{\pgfqpoint{1.574470in}{0.689796in}}%
\pgfpathcurveto{\pgfqpoint{1.574470in}{0.678745in}}{\pgfqpoint{1.578860in}{0.668146in}}{\pgfqpoint{1.586674in}{0.660333in}}%
\pgfpathcurveto{\pgfqpoint{1.594488in}{0.652519in}}{\pgfqpoint{1.605087in}{0.648129in}}{\pgfqpoint{1.616137in}{0.648129in}}%
\pgfpathclose%
\pgfusepath{stroke,fill}%
\end{pgfscope}%
\begin{pgfscope}%
\pgfpathrectangle{\pgfqpoint{0.648703in}{0.548769in}}{\pgfqpoint{5.201297in}{3.102590in}}%
\pgfusepath{clip}%
\pgfsetbuttcap%
\pgfsetroundjoin%
\definecolor{currentfill}{rgb}{1.000000,0.498039,0.054902}%
\pgfsetfillcolor{currentfill}%
\pgfsetlinewidth{1.003750pt}%
\definecolor{currentstroke}{rgb}{1.000000,0.498039,0.054902}%
\pgfsetstrokecolor{currentstroke}%
\pgfsetdash{}{0pt}%
\pgfpathmoveto{\pgfqpoint{1.379779in}{3.193800in}}%
\pgfpathcurveto{\pgfqpoint{1.390829in}{3.193800in}}{\pgfqpoint{1.401428in}{3.198191in}}{\pgfqpoint{1.409242in}{3.206004in}}%
\pgfpathcurveto{\pgfqpoint{1.417055in}{3.213818in}}{\pgfqpoint{1.421446in}{3.224417in}}{\pgfqpoint{1.421446in}{3.235467in}}%
\pgfpathcurveto{\pgfqpoint{1.421446in}{3.246517in}}{\pgfqpoint{1.417055in}{3.257116in}}{\pgfqpoint{1.409242in}{3.264930in}}%
\pgfpathcurveto{\pgfqpoint{1.401428in}{3.272743in}}{\pgfqpoint{1.390829in}{3.277134in}}{\pgfqpoint{1.379779in}{3.277134in}}%
\pgfpathcurveto{\pgfqpoint{1.368729in}{3.277134in}}{\pgfqpoint{1.358130in}{3.272743in}}{\pgfqpoint{1.350316in}{3.264930in}}%
\pgfpathcurveto{\pgfqpoint{1.342502in}{3.257116in}}{\pgfqpoint{1.338112in}{3.246517in}}{\pgfqpoint{1.338112in}{3.235467in}}%
\pgfpathcurveto{\pgfqpoint{1.338112in}{3.224417in}}{\pgfqpoint{1.342502in}{3.213818in}}{\pgfqpoint{1.350316in}{3.206004in}}%
\pgfpathcurveto{\pgfqpoint{1.358130in}{3.198191in}}{\pgfqpoint{1.368729in}{3.193800in}}{\pgfqpoint{1.379779in}{3.193800in}}%
\pgfpathclose%
\pgfusepath{stroke,fill}%
\end{pgfscope}%
\begin{pgfscope}%
\pgfpathrectangle{\pgfqpoint{0.648703in}{0.548769in}}{\pgfqpoint{5.201297in}{3.102590in}}%
\pgfusepath{clip}%
\pgfsetbuttcap%
\pgfsetroundjoin%
\definecolor{currentfill}{rgb}{1.000000,0.498039,0.054902}%
\pgfsetfillcolor{currentfill}%
\pgfsetlinewidth{1.003750pt}%
\definecolor{currentstroke}{rgb}{1.000000,0.498039,0.054902}%
\pgfsetstrokecolor{currentstroke}%
\pgfsetdash{}{0pt}%
\pgfpathmoveto{\pgfqpoint{1.391059in}{3.231859in}}%
\pgfpathcurveto{\pgfqpoint{1.402110in}{3.231859in}}{\pgfqpoint{1.412709in}{3.236249in}}{\pgfqpoint{1.420522in}{3.244062in}}%
\pgfpathcurveto{\pgfqpoint{1.428336in}{3.251876in}}{\pgfqpoint{1.432726in}{3.262475in}}{\pgfqpoint{1.432726in}{3.273525in}}%
\pgfpathcurveto{\pgfqpoint{1.432726in}{3.284575in}}{\pgfqpoint{1.428336in}{3.295174in}}{\pgfqpoint{1.420522in}{3.302988in}}%
\pgfpathcurveto{\pgfqpoint{1.412709in}{3.310802in}}{\pgfqpoint{1.402110in}{3.315192in}}{\pgfqpoint{1.391059in}{3.315192in}}%
\pgfpathcurveto{\pgfqpoint{1.380009in}{3.315192in}}{\pgfqpoint{1.369410in}{3.310802in}}{\pgfqpoint{1.361597in}{3.302988in}}%
\pgfpathcurveto{\pgfqpoint{1.353783in}{3.295174in}}{\pgfqpoint{1.349393in}{3.284575in}}{\pgfqpoint{1.349393in}{3.273525in}}%
\pgfpathcurveto{\pgfqpoint{1.349393in}{3.262475in}}{\pgfqpoint{1.353783in}{3.251876in}}{\pgfqpoint{1.361597in}{3.244062in}}%
\pgfpathcurveto{\pgfqpoint{1.369410in}{3.236249in}}{\pgfqpoint{1.380009in}{3.231859in}}{\pgfqpoint{1.391059in}{3.231859in}}%
\pgfpathclose%
\pgfusepath{stroke,fill}%
\end{pgfscope}%
\begin{pgfscope}%
\pgfpathrectangle{\pgfqpoint{0.648703in}{0.548769in}}{\pgfqpoint{5.201297in}{3.102590in}}%
\pgfusepath{clip}%
\pgfsetbuttcap%
\pgfsetroundjoin%
\definecolor{currentfill}{rgb}{0.121569,0.466667,0.705882}%
\pgfsetfillcolor{currentfill}%
\pgfsetlinewidth{1.003750pt}%
\definecolor{currentstroke}{rgb}{0.121569,0.466667,0.705882}%
\pgfsetstrokecolor{currentstroke}%
\pgfsetdash{}{0pt}%
\pgfpathmoveto{\pgfqpoint{1.624073in}{0.758075in}}%
\pgfpathcurveto{\pgfqpoint{1.635123in}{0.758075in}}{\pgfqpoint{1.645722in}{0.762465in}}{\pgfqpoint{1.653536in}{0.770279in}}%
\pgfpathcurveto{\pgfqpoint{1.661350in}{0.778092in}}{\pgfqpoint{1.665740in}{0.788691in}}{\pgfqpoint{1.665740in}{0.799742in}}%
\pgfpathcurveto{\pgfqpoint{1.665740in}{0.810792in}}{\pgfqpoint{1.661350in}{0.821391in}}{\pgfqpoint{1.653536in}{0.829204in}}%
\pgfpathcurveto{\pgfqpoint{1.645722in}{0.837018in}}{\pgfqpoint{1.635123in}{0.841408in}}{\pgfqpoint{1.624073in}{0.841408in}}%
\pgfpathcurveto{\pgfqpoint{1.613023in}{0.841408in}}{\pgfqpoint{1.602424in}{0.837018in}}{\pgfqpoint{1.594611in}{0.829204in}}%
\pgfpathcurveto{\pgfqpoint{1.586797in}{0.821391in}}{\pgfqpoint{1.582407in}{0.810792in}}{\pgfqpoint{1.582407in}{0.799742in}}%
\pgfpathcurveto{\pgfqpoint{1.582407in}{0.788691in}}{\pgfqpoint{1.586797in}{0.778092in}}{\pgfqpoint{1.594611in}{0.770279in}}%
\pgfpathcurveto{\pgfqpoint{1.602424in}{0.762465in}}{\pgfqpoint{1.613023in}{0.758075in}}{\pgfqpoint{1.624073in}{0.758075in}}%
\pgfpathclose%
\pgfusepath{stroke,fill}%
\end{pgfscope}%
\begin{pgfscope}%
\pgfpathrectangle{\pgfqpoint{0.648703in}{0.548769in}}{\pgfqpoint{5.201297in}{3.102590in}}%
\pgfusepath{clip}%
\pgfsetbuttcap%
\pgfsetroundjoin%
\definecolor{currentfill}{rgb}{1.000000,0.498039,0.054902}%
\pgfsetfillcolor{currentfill}%
\pgfsetlinewidth{1.003750pt}%
\definecolor{currentstroke}{rgb}{1.000000,0.498039,0.054902}%
\pgfsetstrokecolor{currentstroke}%
\pgfsetdash{}{0pt}%
\pgfpathmoveto{\pgfqpoint{1.437876in}{3.185343in}}%
\pgfpathcurveto{\pgfqpoint{1.448926in}{3.185343in}}{\pgfqpoint{1.459525in}{3.189733in}}{\pgfqpoint{1.467339in}{3.197547in}}%
\pgfpathcurveto{\pgfqpoint{1.475153in}{3.205360in}}{\pgfqpoint{1.479543in}{3.215959in}}{\pgfqpoint{1.479543in}{3.227010in}}%
\pgfpathcurveto{\pgfqpoint{1.479543in}{3.238060in}}{\pgfqpoint{1.475153in}{3.248659in}}{\pgfqpoint{1.467339in}{3.256472in}}%
\pgfpathcurveto{\pgfqpoint{1.459525in}{3.264286in}}{\pgfqpoint{1.448926in}{3.268676in}}{\pgfqpoint{1.437876in}{3.268676in}}%
\pgfpathcurveto{\pgfqpoint{1.426826in}{3.268676in}}{\pgfqpoint{1.416227in}{3.264286in}}{\pgfqpoint{1.408413in}{3.256472in}}%
\pgfpathcurveto{\pgfqpoint{1.400600in}{3.248659in}}{\pgfqpoint{1.396210in}{3.238060in}}{\pgfqpoint{1.396210in}{3.227010in}}%
\pgfpathcurveto{\pgfqpoint{1.396210in}{3.215959in}}{\pgfqpoint{1.400600in}{3.205360in}}{\pgfqpoint{1.408413in}{3.197547in}}%
\pgfpathcurveto{\pgfqpoint{1.416227in}{3.189733in}}{\pgfqpoint{1.426826in}{3.185343in}}{\pgfqpoint{1.437876in}{3.185343in}}%
\pgfpathclose%
\pgfusepath{stroke,fill}%
\end{pgfscope}%
\begin{pgfscope}%
\pgfpathrectangle{\pgfqpoint{0.648703in}{0.548769in}}{\pgfqpoint{5.201297in}{3.102590in}}%
\pgfusepath{clip}%
\pgfsetbuttcap%
\pgfsetroundjoin%
\definecolor{currentfill}{rgb}{1.000000,0.498039,0.054902}%
\pgfsetfillcolor{currentfill}%
\pgfsetlinewidth{1.003750pt}%
\definecolor{currentstroke}{rgb}{1.000000,0.498039,0.054902}%
\pgfsetstrokecolor{currentstroke}%
\pgfsetdash{}{0pt}%
\pgfpathmoveto{\pgfqpoint{1.593843in}{3.202258in}}%
\pgfpathcurveto{\pgfqpoint{1.604893in}{3.202258in}}{\pgfqpoint{1.615492in}{3.206648in}}{\pgfqpoint{1.623306in}{3.214462in}}%
\pgfpathcurveto{\pgfqpoint{1.631119in}{3.222275in}}{\pgfqpoint{1.635510in}{3.232874in}}{\pgfqpoint{1.635510in}{3.243924in}}%
\pgfpathcurveto{\pgfqpoint{1.635510in}{3.254974in}}{\pgfqpoint{1.631119in}{3.265573in}}{\pgfqpoint{1.623306in}{3.273387in}}%
\pgfpathcurveto{\pgfqpoint{1.615492in}{3.281201in}}{\pgfqpoint{1.604893in}{3.285591in}}{\pgfqpoint{1.593843in}{3.285591in}}%
\pgfpathcurveto{\pgfqpoint{1.582793in}{3.285591in}}{\pgfqpoint{1.572194in}{3.281201in}}{\pgfqpoint{1.564380in}{3.273387in}}%
\pgfpathcurveto{\pgfqpoint{1.556567in}{3.265573in}}{\pgfqpoint{1.552176in}{3.254974in}}{\pgfqpoint{1.552176in}{3.243924in}}%
\pgfpathcurveto{\pgfqpoint{1.552176in}{3.232874in}}{\pgfqpoint{1.556567in}{3.222275in}}{\pgfqpoint{1.564380in}{3.214462in}}%
\pgfpathcurveto{\pgfqpoint{1.572194in}{3.206648in}}{\pgfqpoint{1.582793in}{3.202258in}}{\pgfqpoint{1.593843in}{3.202258in}}%
\pgfpathclose%
\pgfusepath{stroke,fill}%
\end{pgfscope}%
\begin{pgfscope}%
\pgfpathrectangle{\pgfqpoint{0.648703in}{0.548769in}}{\pgfqpoint{5.201297in}{3.102590in}}%
\pgfusepath{clip}%
\pgfsetbuttcap%
\pgfsetroundjoin%
\definecolor{currentfill}{rgb}{0.121569,0.466667,0.705882}%
\pgfsetfillcolor{currentfill}%
\pgfsetlinewidth{1.003750pt}%
\definecolor{currentstroke}{rgb}{0.121569,0.466667,0.705882}%
\pgfsetstrokecolor{currentstroke}%
\pgfsetdash{}{0pt}%
\pgfpathmoveto{\pgfqpoint{5.613577in}{0.648129in}}%
\pgfpathcurveto{\pgfqpoint{5.624628in}{0.648129in}}{\pgfqpoint{5.635227in}{0.652519in}}{\pgfqpoint{5.643040in}{0.660333in}}%
\pgfpathcurveto{\pgfqpoint{5.650854in}{0.668146in}}{\pgfqpoint{5.655244in}{0.678745in}}{\pgfqpoint{5.655244in}{0.689796in}}%
\pgfpathcurveto{\pgfqpoint{5.655244in}{0.700846in}}{\pgfqpoint{5.650854in}{0.711445in}}{\pgfqpoint{5.643040in}{0.719258in}}%
\pgfpathcurveto{\pgfqpoint{5.635227in}{0.727072in}}{\pgfqpoint{5.624628in}{0.731462in}}{\pgfqpoint{5.613577in}{0.731462in}}%
\pgfpathcurveto{\pgfqpoint{5.602527in}{0.731462in}}{\pgfqpoint{5.591928in}{0.727072in}}{\pgfqpoint{5.584115in}{0.719258in}}%
\pgfpathcurveto{\pgfqpoint{5.576301in}{0.711445in}}{\pgfqpoint{5.571911in}{0.700846in}}{\pgfqpoint{5.571911in}{0.689796in}}%
\pgfpathcurveto{\pgfqpoint{5.571911in}{0.678745in}}{\pgfqpoint{5.576301in}{0.668146in}}{\pgfqpoint{5.584115in}{0.660333in}}%
\pgfpathcurveto{\pgfqpoint{5.591928in}{0.652519in}}{\pgfqpoint{5.602527in}{0.648129in}}{\pgfqpoint{5.613577in}{0.648129in}}%
\pgfpathclose%
\pgfusepath{stroke,fill}%
\end{pgfscope}%
\begin{pgfscope}%
\pgfpathrectangle{\pgfqpoint{0.648703in}{0.548769in}}{\pgfqpoint{5.201297in}{3.102590in}}%
\pgfusepath{clip}%
\pgfsetbuttcap%
\pgfsetroundjoin%
\definecolor{currentfill}{rgb}{1.000000,0.498039,0.054902}%
\pgfsetfillcolor{currentfill}%
\pgfsetlinewidth{1.003750pt}%
\definecolor{currentstroke}{rgb}{1.000000,0.498039,0.054902}%
\pgfsetstrokecolor{currentstroke}%
\pgfsetdash{}{0pt}%
\pgfpathmoveto{\pgfqpoint{1.885177in}{3.206486in}}%
\pgfpathcurveto{\pgfqpoint{1.896227in}{3.206486in}}{\pgfqpoint{1.906826in}{3.210877in}}{\pgfqpoint{1.914640in}{3.218690in}}%
\pgfpathcurveto{\pgfqpoint{1.922454in}{3.226504in}}{\pgfqpoint{1.926844in}{3.237103in}}{\pgfqpoint{1.926844in}{3.248153in}}%
\pgfpathcurveto{\pgfqpoint{1.926844in}{3.259203in}}{\pgfqpoint{1.922454in}{3.269802in}}{\pgfqpoint{1.914640in}{3.277616in}}%
\pgfpathcurveto{\pgfqpoint{1.906826in}{3.285429in}}{\pgfqpoint{1.896227in}{3.289820in}}{\pgfqpoint{1.885177in}{3.289820in}}%
\pgfpathcurveto{\pgfqpoint{1.874127in}{3.289820in}}{\pgfqpoint{1.863528in}{3.285429in}}{\pgfqpoint{1.855714in}{3.277616in}}%
\pgfpathcurveto{\pgfqpoint{1.847901in}{3.269802in}}{\pgfqpoint{1.843511in}{3.259203in}}{\pgfqpoint{1.843511in}{3.248153in}}%
\pgfpathcurveto{\pgfqpoint{1.843511in}{3.237103in}}{\pgfqpoint{1.847901in}{3.226504in}}{\pgfqpoint{1.855714in}{3.218690in}}%
\pgfpathcurveto{\pgfqpoint{1.863528in}{3.210877in}}{\pgfqpoint{1.874127in}{3.206486in}}{\pgfqpoint{1.885177in}{3.206486in}}%
\pgfpathclose%
\pgfusepath{stroke,fill}%
\end{pgfscope}%
\begin{pgfscope}%
\pgfpathrectangle{\pgfqpoint{0.648703in}{0.548769in}}{\pgfqpoint{5.201297in}{3.102590in}}%
\pgfusepath{clip}%
\pgfsetbuttcap%
\pgfsetroundjoin%
\definecolor{currentfill}{rgb}{0.121569,0.466667,0.705882}%
\pgfsetfillcolor{currentfill}%
\pgfsetlinewidth{1.003750pt}%
\definecolor{currentstroke}{rgb}{0.121569,0.466667,0.705882}%
\pgfsetstrokecolor{currentstroke}%
\pgfsetdash{}{0pt}%
\pgfpathmoveto{\pgfqpoint{1.238927in}{0.648129in}}%
\pgfpathcurveto{\pgfqpoint{1.249977in}{0.648129in}}{\pgfqpoint{1.260576in}{0.652519in}}{\pgfqpoint{1.268390in}{0.660333in}}%
\pgfpathcurveto{\pgfqpoint{1.276204in}{0.668146in}}{\pgfqpoint{1.280594in}{0.678745in}}{\pgfqpoint{1.280594in}{0.689796in}}%
\pgfpathcurveto{\pgfqpoint{1.280594in}{0.700846in}}{\pgfqpoint{1.276204in}{0.711445in}}{\pgfqpoint{1.268390in}{0.719258in}}%
\pgfpathcurveto{\pgfqpoint{1.260576in}{0.727072in}}{\pgfqpoint{1.249977in}{0.731462in}}{\pgfqpoint{1.238927in}{0.731462in}}%
\pgfpathcurveto{\pgfqpoint{1.227877in}{0.731462in}}{\pgfqpoint{1.217278in}{0.727072in}}{\pgfqpoint{1.209464in}{0.719258in}}%
\pgfpathcurveto{\pgfqpoint{1.201651in}{0.711445in}}{\pgfqpoint{1.197261in}{0.700846in}}{\pgfqpoint{1.197261in}{0.689796in}}%
\pgfpathcurveto{\pgfqpoint{1.197261in}{0.678745in}}{\pgfqpoint{1.201651in}{0.668146in}}{\pgfqpoint{1.209464in}{0.660333in}}%
\pgfpathcurveto{\pgfqpoint{1.217278in}{0.652519in}}{\pgfqpoint{1.227877in}{0.648129in}}{\pgfqpoint{1.238927in}{0.648129in}}%
\pgfpathclose%
\pgfusepath{stroke,fill}%
\end{pgfscope}%
\begin{pgfscope}%
\pgfpathrectangle{\pgfqpoint{0.648703in}{0.548769in}}{\pgfqpoint{5.201297in}{3.102590in}}%
\pgfusepath{clip}%
\pgfsetbuttcap%
\pgfsetroundjoin%
\definecolor{currentfill}{rgb}{0.121569,0.466667,0.705882}%
\pgfsetfillcolor{currentfill}%
\pgfsetlinewidth{1.003750pt}%
\definecolor{currentstroke}{rgb}{0.121569,0.466667,0.705882}%
\pgfsetstrokecolor{currentstroke}%
\pgfsetdash{}{0pt}%
\pgfpathmoveto{\pgfqpoint{1.194072in}{0.648129in}}%
\pgfpathcurveto{\pgfqpoint{1.205122in}{0.648129in}}{\pgfqpoint{1.215721in}{0.652519in}}{\pgfqpoint{1.223535in}{0.660333in}}%
\pgfpathcurveto{\pgfqpoint{1.231349in}{0.668146in}}{\pgfqpoint{1.235739in}{0.678745in}}{\pgfqpoint{1.235739in}{0.689796in}}%
\pgfpathcurveto{\pgfqpoint{1.235739in}{0.700846in}}{\pgfqpoint{1.231349in}{0.711445in}}{\pgfqpoint{1.223535in}{0.719258in}}%
\pgfpathcurveto{\pgfqpoint{1.215721in}{0.727072in}}{\pgfqpoint{1.205122in}{0.731462in}}{\pgfqpoint{1.194072in}{0.731462in}}%
\pgfpathcurveto{\pgfqpoint{1.183022in}{0.731462in}}{\pgfqpoint{1.172423in}{0.727072in}}{\pgfqpoint{1.164609in}{0.719258in}}%
\pgfpathcurveto{\pgfqpoint{1.156796in}{0.711445in}}{\pgfqpoint{1.152406in}{0.700846in}}{\pgfqpoint{1.152406in}{0.689796in}}%
\pgfpathcurveto{\pgfqpoint{1.152406in}{0.678745in}}{\pgfqpoint{1.156796in}{0.668146in}}{\pgfqpoint{1.164609in}{0.660333in}}%
\pgfpathcurveto{\pgfqpoint{1.172423in}{0.652519in}}{\pgfqpoint{1.183022in}{0.648129in}}{\pgfqpoint{1.194072in}{0.648129in}}%
\pgfpathclose%
\pgfusepath{stroke,fill}%
\end{pgfscope}%
\begin{pgfscope}%
\pgfpathrectangle{\pgfqpoint{0.648703in}{0.548769in}}{\pgfqpoint{5.201297in}{3.102590in}}%
\pgfusepath{clip}%
\pgfsetbuttcap%
\pgfsetroundjoin%
\definecolor{currentfill}{rgb}{0.121569,0.466667,0.705882}%
\pgfsetfillcolor{currentfill}%
\pgfsetlinewidth{1.003750pt}%
\definecolor{currentstroke}{rgb}{0.121569,0.466667,0.705882}%
\pgfsetstrokecolor{currentstroke}%
\pgfsetdash{}{0pt}%
\pgfpathmoveto{\pgfqpoint{1.257163in}{0.817277in}}%
\pgfpathcurveto{\pgfqpoint{1.268214in}{0.817277in}}{\pgfqpoint{1.278813in}{0.821667in}}{\pgfqpoint{1.286626in}{0.829480in}}%
\pgfpathcurveto{\pgfqpoint{1.294440in}{0.837294in}}{\pgfqpoint{1.298830in}{0.847893in}}{\pgfqpoint{1.298830in}{0.858943in}}%
\pgfpathcurveto{\pgfqpoint{1.298830in}{0.869993in}}{\pgfqpoint{1.294440in}{0.880592in}}{\pgfqpoint{1.286626in}{0.888406in}}%
\pgfpathcurveto{\pgfqpoint{1.278813in}{0.896220in}}{\pgfqpoint{1.268214in}{0.900610in}}{\pgfqpoint{1.257163in}{0.900610in}}%
\pgfpathcurveto{\pgfqpoint{1.246113in}{0.900610in}}{\pgfqpoint{1.235514in}{0.896220in}}{\pgfqpoint{1.227701in}{0.888406in}}%
\pgfpathcurveto{\pgfqpoint{1.219887in}{0.880592in}}{\pgfqpoint{1.215497in}{0.869993in}}{\pgfqpoint{1.215497in}{0.858943in}}%
\pgfpathcurveto{\pgfqpoint{1.215497in}{0.847893in}}{\pgfqpoint{1.219887in}{0.837294in}}{\pgfqpoint{1.227701in}{0.829480in}}%
\pgfpathcurveto{\pgfqpoint{1.235514in}{0.821667in}}{\pgfqpoint{1.246113in}{0.817277in}}{\pgfqpoint{1.257163in}{0.817277in}}%
\pgfpathclose%
\pgfusepath{stroke,fill}%
\end{pgfscope}%
\begin{pgfscope}%
\pgfpathrectangle{\pgfqpoint{0.648703in}{0.548769in}}{\pgfqpoint{5.201297in}{3.102590in}}%
\pgfusepath{clip}%
\pgfsetbuttcap%
\pgfsetroundjoin%
\definecolor{currentfill}{rgb}{1.000000,0.498039,0.054902}%
\pgfsetfillcolor{currentfill}%
\pgfsetlinewidth{1.003750pt}%
\definecolor{currentstroke}{rgb}{1.000000,0.498039,0.054902}%
\pgfsetstrokecolor{currentstroke}%
\pgfsetdash{}{0pt}%
\pgfpathmoveto{\pgfqpoint{2.080782in}{3.189572in}}%
\pgfpathcurveto{\pgfqpoint{2.091832in}{3.189572in}}{\pgfqpoint{2.102431in}{3.193962in}}{\pgfqpoint{2.110245in}{3.201775in}}%
\pgfpathcurveto{\pgfqpoint{2.118059in}{3.209589in}}{\pgfqpoint{2.122449in}{3.220188in}}{\pgfqpoint{2.122449in}{3.231238in}}%
\pgfpathcurveto{\pgfqpoint{2.122449in}{3.242288in}}{\pgfqpoint{2.118059in}{3.252887in}}{\pgfqpoint{2.110245in}{3.260701in}}%
\pgfpathcurveto{\pgfqpoint{2.102431in}{3.268515in}}{\pgfqpoint{2.091832in}{3.272905in}}{\pgfqpoint{2.080782in}{3.272905in}}%
\pgfpathcurveto{\pgfqpoint{2.069732in}{3.272905in}}{\pgfqpoint{2.059133in}{3.268515in}}{\pgfqpoint{2.051319in}{3.260701in}}%
\pgfpathcurveto{\pgfqpoint{2.043506in}{3.252887in}}{\pgfqpoint{2.039116in}{3.242288in}}{\pgfqpoint{2.039116in}{3.231238in}}%
\pgfpathcurveto{\pgfqpoint{2.039116in}{3.220188in}}{\pgfqpoint{2.043506in}{3.209589in}}{\pgfqpoint{2.051319in}{3.201775in}}%
\pgfpathcurveto{\pgfqpoint{2.059133in}{3.193962in}}{\pgfqpoint{2.069732in}{3.189572in}}{\pgfqpoint{2.080782in}{3.189572in}}%
\pgfpathclose%
\pgfusepath{stroke,fill}%
\end{pgfscope}%
\begin{pgfscope}%
\pgfpathrectangle{\pgfqpoint{0.648703in}{0.548769in}}{\pgfqpoint{5.201297in}{3.102590in}}%
\pgfusepath{clip}%
\pgfsetbuttcap%
\pgfsetroundjoin%
\definecolor{currentfill}{rgb}{1.000000,0.498039,0.054902}%
\pgfsetfillcolor{currentfill}%
\pgfsetlinewidth{1.003750pt}%
\definecolor{currentstroke}{rgb}{1.000000,0.498039,0.054902}%
\pgfsetstrokecolor{currentstroke}%
\pgfsetdash{}{0pt}%
\pgfpathmoveto{\pgfqpoint{1.434175in}{3.214944in}}%
\pgfpathcurveto{\pgfqpoint{1.445226in}{3.214944in}}{\pgfqpoint{1.455825in}{3.219334in}}{\pgfqpoint{1.463638in}{3.227148in}}%
\pgfpathcurveto{\pgfqpoint{1.471452in}{3.234961in}}{\pgfqpoint{1.475842in}{3.245560in}}{\pgfqpoint{1.475842in}{3.256610in}}%
\pgfpathcurveto{\pgfqpoint{1.475842in}{3.267661in}}{\pgfqpoint{1.471452in}{3.278260in}}{\pgfqpoint{1.463638in}{3.286073in}}%
\pgfpathcurveto{\pgfqpoint{1.455825in}{3.293887in}}{\pgfqpoint{1.445226in}{3.298277in}}{\pgfqpoint{1.434175in}{3.298277in}}%
\pgfpathcurveto{\pgfqpoint{1.423125in}{3.298277in}}{\pgfqpoint{1.412526in}{3.293887in}}{\pgfqpoint{1.404713in}{3.286073in}}%
\pgfpathcurveto{\pgfqpoint{1.396899in}{3.278260in}}{\pgfqpoint{1.392509in}{3.267661in}}{\pgfqpoint{1.392509in}{3.256610in}}%
\pgfpathcurveto{\pgfqpoint{1.392509in}{3.245560in}}{\pgfqpoint{1.396899in}{3.234961in}}{\pgfqpoint{1.404713in}{3.227148in}}%
\pgfpathcurveto{\pgfqpoint{1.412526in}{3.219334in}}{\pgfqpoint{1.423125in}{3.214944in}}{\pgfqpoint{1.434175in}{3.214944in}}%
\pgfpathclose%
\pgfusepath{stroke,fill}%
\end{pgfscope}%
\begin{pgfscope}%
\pgfpathrectangle{\pgfqpoint{0.648703in}{0.548769in}}{\pgfqpoint{5.201297in}{3.102590in}}%
\pgfusepath{clip}%
\pgfsetbuttcap%
\pgfsetroundjoin%
\definecolor{currentfill}{rgb}{0.121569,0.466667,0.705882}%
\pgfsetfillcolor{currentfill}%
\pgfsetlinewidth{1.003750pt}%
\definecolor{currentstroke}{rgb}{0.121569,0.466667,0.705882}%
\pgfsetstrokecolor{currentstroke}%
\pgfsetdash{}{0pt}%
\pgfpathmoveto{\pgfqpoint{1.217614in}{0.648129in}}%
\pgfpathcurveto{\pgfqpoint{1.228665in}{0.648129in}}{\pgfqpoint{1.239264in}{0.652519in}}{\pgfqpoint{1.247077in}{0.660333in}}%
\pgfpathcurveto{\pgfqpoint{1.254891in}{0.668146in}}{\pgfqpoint{1.259281in}{0.678745in}}{\pgfqpoint{1.259281in}{0.689796in}}%
\pgfpathcurveto{\pgfqpoint{1.259281in}{0.700846in}}{\pgfqpoint{1.254891in}{0.711445in}}{\pgfqpoint{1.247077in}{0.719258in}}%
\pgfpathcurveto{\pgfqpoint{1.239264in}{0.727072in}}{\pgfqpoint{1.228665in}{0.731462in}}{\pgfqpoint{1.217614in}{0.731462in}}%
\pgfpathcurveto{\pgfqpoint{1.206564in}{0.731462in}}{\pgfqpoint{1.195965in}{0.727072in}}{\pgfqpoint{1.188152in}{0.719258in}}%
\pgfpathcurveto{\pgfqpoint{1.180338in}{0.711445in}}{\pgfqpoint{1.175948in}{0.700846in}}{\pgfqpoint{1.175948in}{0.689796in}}%
\pgfpathcurveto{\pgfqpoint{1.175948in}{0.678745in}}{\pgfqpoint{1.180338in}{0.668146in}}{\pgfqpoint{1.188152in}{0.660333in}}%
\pgfpathcurveto{\pgfqpoint{1.195965in}{0.652519in}}{\pgfqpoint{1.206564in}{0.648129in}}{\pgfqpoint{1.217614in}{0.648129in}}%
\pgfpathclose%
\pgfusepath{stroke,fill}%
\end{pgfscope}%
\begin{pgfscope}%
\pgfpathrectangle{\pgfqpoint{0.648703in}{0.548769in}}{\pgfqpoint{5.201297in}{3.102590in}}%
\pgfusepath{clip}%
\pgfsetbuttcap%
\pgfsetroundjoin%
\definecolor{currentfill}{rgb}{0.121569,0.466667,0.705882}%
\pgfsetfillcolor{currentfill}%
\pgfsetlinewidth{1.003750pt}%
\definecolor{currentstroke}{rgb}{0.121569,0.466667,0.705882}%
\pgfsetstrokecolor{currentstroke}%
\pgfsetdash{}{0pt}%
\pgfpathmoveto{\pgfqpoint{1.165715in}{0.758075in}}%
\pgfpathcurveto{\pgfqpoint{1.176765in}{0.758075in}}{\pgfqpoint{1.187364in}{0.762465in}}{\pgfqpoint{1.195177in}{0.770279in}}%
\pgfpathcurveto{\pgfqpoint{1.202991in}{0.778092in}}{\pgfqpoint{1.207381in}{0.788691in}}{\pgfqpoint{1.207381in}{0.799742in}}%
\pgfpathcurveto{\pgfqpoint{1.207381in}{0.810792in}}{\pgfqpoint{1.202991in}{0.821391in}}{\pgfqpoint{1.195177in}{0.829204in}}%
\pgfpathcurveto{\pgfqpoint{1.187364in}{0.837018in}}{\pgfqpoint{1.176765in}{0.841408in}}{\pgfqpoint{1.165715in}{0.841408in}}%
\pgfpathcurveto{\pgfqpoint{1.154664in}{0.841408in}}{\pgfqpoint{1.144065in}{0.837018in}}{\pgfqpoint{1.136252in}{0.829204in}}%
\pgfpathcurveto{\pgfqpoint{1.128438in}{0.821391in}}{\pgfqpoint{1.124048in}{0.810792in}}{\pgfqpoint{1.124048in}{0.799742in}}%
\pgfpathcurveto{\pgfqpoint{1.124048in}{0.788691in}}{\pgfqpoint{1.128438in}{0.778092in}}{\pgfqpoint{1.136252in}{0.770279in}}%
\pgfpathcurveto{\pgfqpoint{1.144065in}{0.762465in}}{\pgfqpoint{1.154664in}{0.758075in}}{\pgfqpoint{1.165715in}{0.758075in}}%
\pgfpathclose%
\pgfusepath{stroke,fill}%
\end{pgfscope}%
\begin{pgfscope}%
\pgfpathrectangle{\pgfqpoint{0.648703in}{0.548769in}}{\pgfqpoint{5.201297in}{3.102590in}}%
\pgfusepath{clip}%
\pgfsetbuttcap%
\pgfsetroundjoin%
\definecolor{currentfill}{rgb}{1.000000,0.498039,0.054902}%
\pgfsetfillcolor{currentfill}%
\pgfsetlinewidth{1.003750pt}%
\definecolor{currentstroke}{rgb}{1.000000,0.498039,0.054902}%
\pgfsetstrokecolor{currentstroke}%
\pgfsetdash{}{0pt}%
\pgfpathmoveto{\pgfqpoint{2.385314in}{3.193800in}}%
\pgfpathcurveto{\pgfqpoint{2.396364in}{3.193800in}}{\pgfqpoint{2.406964in}{3.198191in}}{\pgfqpoint{2.414777in}{3.206004in}}%
\pgfpathcurveto{\pgfqpoint{2.422591in}{3.213818in}}{\pgfqpoint{2.426981in}{3.224417in}}{\pgfqpoint{2.426981in}{3.235467in}}%
\pgfpathcurveto{\pgfqpoint{2.426981in}{3.246517in}}{\pgfqpoint{2.422591in}{3.257116in}}{\pgfqpoint{2.414777in}{3.264930in}}%
\pgfpathcurveto{\pgfqpoint{2.406964in}{3.272743in}}{\pgfqpoint{2.396364in}{3.277134in}}{\pgfqpoint{2.385314in}{3.277134in}}%
\pgfpathcurveto{\pgfqpoint{2.374264in}{3.277134in}}{\pgfqpoint{2.363665in}{3.272743in}}{\pgfqpoint{2.355852in}{3.264930in}}%
\pgfpathcurveto{\pgfqpoint{2.348038in}{3.257116in}}{\pgfqpoint{2.343648in}{3.246517in}}{\pgfqpoint{2.343648in}{3.235467in}}%
\pgfpathcurveto{\pgfqpoint{2.343648in}{3.224417in}}{\pgfqpoint{2.348038in}{3.213818in}}{\pgfqpoint{2.355852in}{3.206004in}}%
\pgfpathcurveto{\pgfqpoint{2.363665in}{3.198191in}}{\pgfqpoint{2.374264in}{3.193800in}}{\pgfqpoint{2.385314in}{3.193800in}}%
\pgfpathclose%
\pgfusepath{stroke,fill}%
\end{pgfscope}%
\begin{pgfscope}%
\pgfpathrectangle{\pgfqpoint{0.648703in}{0.548769in}}{\pgfqpoint{5.201297in}{3.102590in}}%
\pgfusepath{clip}%
\pgfsetbuttcap%
\pgfsetroundjoin%
\definecolor{currentfill}{rgb}{0.121569,0.466667,0.705882}%
\pgfsetfillcolor{currentfill}%
\pgfsetlinewidth{1.003750pt}%
\definecolor{currentstroke}{rgb}{0.121569,0.466667,0.705882}%
\pgfsetstrokecolor{currentstroke}%
\pgfsetdash{}{0pt}%
\pgfpathmoveto{\pgfqpoint{0.947950in}{0.648129in}}%
\pgfpathcurveto{\pgfqpoint{0.959000in}{0.648129in}}{\pgfqpoint{0.969599in}{0.652519in}}{\pgfqpoint{0.977412in}{0.660333in}}%
\pgfpathcurveto{\pgfqpoint{0.985226in}{0.668146in}}{\pgfqpoint{0.989616in}{0.678745in}}{\pgfqpoint{0.989616in}{0.689796in}}%
\pgfpathcurveto{\pgfqpoint{0.989616in}{0.700846in}}{\pgfqpoint{0.985226in}{0.711445in}}{\pgfqpoint{0.977412in}{0.719258in}}%
\pgfpathcurveto{\pgfqpoint{0.969599in}{0.727072in}}{\pgfqpoint{0.959000in}{0.731462in}}{\pgfqpoint{0.947950in}{0.731462in}}%
\pgfpathcurveto{\pgfqpoint{0.936900in}{0.731462in}}{\pgfqpoint{0.926301in}{0.727072in}}{\pgfqpoint{0.918487in}{0.719258in}}%
\pgfpathcurveto{\pgfqpoint{0.910673in}{0.711445in}}{\pgfqpoint{0.906283in}{0.700846in}}{\pgfqpoint{0.906283in}{0.689796in}}%
\pgfpathcurveto{\pgfqpoint{0.906283in}{0.678745in}}{\pgfqpoint{0.910673in}{0.668146in}}{\pgfqpoint{0.918487in}{0.660333in}}%
\pgfpathcurveto{\pgfqpoint{0.926301in}{0.652519in}}{\pgfqpoint{0.936900in}{0.648129in}}{\pgfqpoint{0.947950in}{0.648129in}}%
\pgfpathclose%
\pgfusepath{stroke,fill}%
\end{pgfscope}%
\begin{pgfscope}%
\pgfpathrectangle{\pgfqpoint{0.648703in}{0.548769in}}{\pgfqpoint{5.201297in}{3.102590in}}%
\pgfusepath{clip}%
\pgfsetbuttcap%
\pgfsetroundjoin%
\definecolor{currentfill}{rgb}{1.000000,0.498039,0.054902}%
\pgfsetfillcolor{currentfill}%
\pgfsetlinewidth{1.003750pt}%
\definecolor{currentstroke}{rgb}{1.000000,0.498039,0.054902}%
\pgfsetstrokecolor{currentstroke}%
\pgfsetdash{}{0pt}%
\pgfpathmoveto{\pgfqpoint{1.673922in}{3.198029in}}%
\pgfpathcurveto{\pgfqpoint{1.684972in}{3.198029in}}{\pgfqpoint{1.695571in}{3.202419in}}{\pgfqpoint{1.703385in}{3.210233in}}%
\pgfpathcurveto{\pgfqpoint{1.711198in}{3.218046in}}{\pgfqpoint{1.715589in}{3.228646in}}{\pgfqpoint{1.715589in}{3.239696in}}%
\pgfpathcurveto{\pgfqpoint{1.715589in}{3.250746in}}{\pgfqpoint{1.711198in}{3.261345in}}{\pgfqpoint{1.703385in}{3.269158in}}%
\pgfpathcurveto{\pgfqpoint{1.695571in}{3.276972in}}{\pgfqpoint{1.684972in}{3.281362in}}{\pgfqpoint{1.673922in}{3.281362in}}%
\pgfpathcurveto{\pgfqpoint{1.662872in}{3.281362in}}{\pgfqpoint{1.652273in}{3.276972in}}{\pgfqpoint{1.644459in}{3.269158in}}%
\pgfpathcurveto{\pgfqpoint{1.636646in}{3.261345in}}{\pgfqpoint{1.632255in}{3.250746in}}{\pgfqpoint{1.632255in}{3.239696in}}%
\pgfpathcurveto{\pgfqpoint{1.632255in}{3.228646in}}{\pgfqpoint{1.636646in}{3.218046in}}{\pgfqpoint{1.644459in}{3.210233in}}%
\pgfpathcurveto{\pgfqpoint{1.652273in}{3.202419in}}{\pgfqpoint{1.662872in}{3.198029in}}{\pgfqpoint{1.673922in}{3.198029in}}%
\pgfpathclose%
\pgfusepath{stroke,fill}%
\end{pgfscope}%
\begin{pgfscope}%
\pgfpathrectangle{\pgfqpoint{0.648703in}{0.548769in}}{\pgfqpoint{5.201297in}{3.102590in}}%
\pgfusepath{clip}%
\pgfsetbuttcap%
\pgfsetroundjoin%
\definecolor{currentfill}{rgb}{1.000000,0.498039,0.054902}%
\pgfsetfillcolor{currentfill}%
\pgfsetlinewidth{1.003750pt}%
\definecolor{currentstroke}{rgb}{1.000000,0.498039,0.054902}%
\pgfsetstrokecolor{currentstroke}%
\pgfsetdash{}{0pt}%
\pgfpathmoveto{\pgfqpoint{1.471361in}{3.193800in}}%
\pgfpathcurveto{\pgfqpoint{1.482412in}{3.193800in}}{\pgfqpoint{1.493011in}{3.198191in}}{\pgfqpoint{1.500824in}{3.206004in}}%
\pgfpathcurveto{\pgfqpoint{1.508638in}{3.213818in}}{\pgfqpoint{1.513028in}{3.224417in}}{\pgfqpoint{1.513028in}{3.235467in}}%
\pgfpathcurveto{\pgfqpoint{1.513028in}{3.246517in}}{\pgfqpoint{1.508638in}{3.257116in}}{\pgfqpoint{1.500824in}{3.264930in}}%
\pgfpathcurveto{\pgfqpoint{1.493011in}{3.272743in}}{\pgfqpoint{1.482412in}{3.277134in}}{\pgfqpoint{1.471361in}{3.277134in}}%
\pgfpathcurveto{\pgfqpoint{1.460311in}{3.277134in}}{\pgfqpoint{1.449712in}{3.272743in}}{\pgfqpoint{1.441899in}{3.264930in}}%
\pgfpathcurveto{\pgfqpoint{1.434085in}{3.257116in}}{\pgfqpoint{1.429695in}{3.246517in}}{\pgfqpoint{1.429695in}{3.235467in}}%
\pgfpathcurveto{\pgfqpoint{1.429695in}{3.224417in}}{\pgfqpoint{1.434085in}{3.213818in}}{\pgfqpoint{1.441899in}{3.206004in}}%
\pgfpathcurveto{\pgfqpoint{1.449712in}{3.198191in}}{\pgfqpoint{1.460311in}{3.193800in}}{\pgfqpoint{1.471361in}{3.193800in}}%
\pgfpathclose%
\pgfusepath{stroke,fill}%
\end{pgfscope}%
\begin{pgfscope}%
\pgfpathrectangle{\pgfqpoint{0.648703in}{0.548769in}}{\pgfqpoint{5.201297in}{3.102590in}}%
\pgfusepath{clip}%
\pgfsetbuttcap%
\pgfsetroundjoin%
\definecolor{currentfill}{rgb}{0.121569,0.466667,0.705882}%
\pgfsetfillcolor{currentfill}%
\pgfsetlinewidth{1.003750pt}%
\definecolor{currentstroke}{rgb}{0.121569,0.466667,0.705882}%
\pgfsetstrokecolor{currentstroke}%
\pgfsetdash{}{0pt}%
\pgfpathmoveto{\pgfqpoint{1.801888in}{0.652358in}}%
\pgfpathcurveto{\pgfqpoint{1.812938in}{0.652358in}}{\pgfqpoint{1.823537in}{0.656748in}}{\pgfqpoint{1.831351in}{0.664562in}}%
\pgfpathcurveto{\pgfqpoint{1.839164in}{0.672375in}}{\pgfqpoint{1.843555in}{0.682974in}}{\pgfqpoint{1.843555in}{0.694024in}}%
\pgfpathcurveto{\pgfqpoint{1.843555in}{0.705074in}}{\pgfqpoint{1.839164in}{0.715673in}}{\pgfqpoint{1.831351in}{0.723487in}}%
\pgfpathcurveto{\pgfqpoint{1.823537in}{0.731301in}}{\pgfqpoint{1.812938in}{0.735691in}}{\pgfqpoint{1.801888in}{0.735691in}}%
\pgfpathcurveto{\pgfqpoint{1.790838in}{0.735691in}}{\pgfqpoint{1.780239in}{0.731301in}}{\pgfqpoint{1.772425in}{0.723487in}}%
\pgfpathcurveto{\pgfqpoint{1.764612in}{0.715673in}}{\pgfqpoint{1.760221in}{0.705074in}}{\pgfqpoint{1.760221in}{0.694024in}}%
\pgfpathcurveto{\pgfqpoint{1.760221in}{0.682974in}}{\pgfqpoint{1.764612in}{0.672375in}}{\pgfqpoint{1.772425in}{0.664562in}}%
\pgfpathcurveto{\pgfqpoint{1.780239in}{0.656748in}}{\pgfqpoint{1.790838in}{0.652358in}}{\pgfqpoint{1.801888in}{0.652358in}}%
\pgfpathclose%
\pgfusepath{stroke,fill}%
\end{pgfscope}%
\begin{pgfscope}%
\pgfpathrectangle{\pgfqpoint{0.648703in}{0.548769in}}{\pgfqpoint{5.201297in}{3.102590in}}%
\pgfusepath{clip}%
\pgfsetbuttcap%
\pgfsetroundjoin%
\definecolor{currentfill}{rgb}{0.121569,0.466667,0.705882}%
\pgfsetfillcolor{currentfill}%
\pgfsetlinewidth{1.003750pt}%
\definecolor{currentstroke}{rgb}{0.121569,0.466667,0.705882}%
\pgfsetstrokecolor{currentstroke}%
\pgfsetdash{}{0pt}%
\pgfpathmoveto{\pgfqpoint{1.964142in}{0.648129in}}%
\pgfpathcurveto{\pgfqpoint{1.975192in}{0.648129in}}{\pgfqpoint{1.985791in}{0.652519in}}{\pgfqpoint{1.993604in}{0.660333in}}%
\pgfpathcurveto{\pgfqpoint{2.001418in}{0.668146in}}{\pgfqpoint{2.005808in}{0.678745in}}{\pgfqpoint{2.005808in}{0.689796in}}%
\pgfpathcurveto{\pgfqpoint{2.005808in}{0.700846in}}{\pgfqpoint{2.001418in}{0.711445in}}{\pgfqpoint{1.993604in}{0.719258in}}%
\pgfpathcurveto{\pgfqpoint{1.985791in}{0.727072in}}{\pgfqpoint{1.975192in}{0.731462in}}{\pgfqpoint{1.964142in}{0.731462in}}%
\pgfpathcurveto{\pgfqpoint{1.953091in}{0.731462in}}{\pgfqpoint{1.942492in}{0.727072in}}{\pgfqpoint{1.934679in}{0.719258in}}%
\pgfpathcurveto{\pgfqpoint{1.926865in}{0.711445in}}{\pgfqpoint{1.922475in}{0.700846in}}{\pgfqpoint{1.922475in}{0.689796in}}%
\pgfpathcurveto{\pgfqpoint{1.922475in}{0.678745in}}{\pgfqpoint{1.926865in}{0.668146in}}{\pgfqpoint{1.934679in}{0.660333in}}%
\pgfpathcurveto{\pgfqpoint{1.942492in}{0.652519in}}{\pgfqpoint{1.953091in}{0.648129in}}{\pgfqpoint{1.964142in}{0.648129in}}%
\pgfpathclose%
\pgfusepath{stroke,fill}%
\end{pgfscope}%
\begin{pgfscope}%
\pgfpathrectangle{\pgfqpoint{0.648703in}{0.548769in}}{\pgfqpoint{5.201297in}{3.102590in}}%
\pgfusepath{clip}%
\pgfsetbuttcap%
\pgfsetroundjoin%
\definecolor{currentfill}{rgb}{0.121569,0.466667,0.705882}%
\pgfsetfillcolor{currentfill}%
\pgfsetlinewidth{1.003750pt}%
\definecolor{currentstroke}{rgb}{0.121569,0.466667,0.705882}%
\pgfsetstrokecolor{currentstroke}%
\pgfsetdash{}{0pt}%
\pgfpathmoveto{\pgfqpoint{1.226487in}{0.648129in}}%
\pgfpathcurveto{\pgfqpoint{1.237537in}{0.648129in}}{\pgfqpoint{1.248136in}{0.652519in}}{\pgfqpoint{1.255950in}{0.660333in}}%
\pgfpathcurveto{\pgfqpoint{1.263764in}{0.668146in}}{\pgfqpoint{1.268154in}{0.678745in}}{\pgfqpoint{1.268154in}{0.689796in}}%
\pgfpathcurveto{\pgfqpoint{1.268154in}{0.700846in}}{\pgfqpoint{1.263764in}{0.711445in}}{\pgfqpoint{1.255950in}{0.719258in}}%
\pgfpathcurveto{\pgfqpoint{1.248136in}{0.727072in}}{\pgfqpoint{1.237537in}{0.731462in}}{\pgfqpoint{1.226487in}{0.731462in}}%
\pgfpathcurveto{\pgfqpoint{1.215437in}{0.731462in}}{\pgfqpoint{1.204838in}{0.727072in}}{\pgfqpoint{1.197025in}{0.719258in}}%
\pgfpathcurveto{\pgfqpoint{1.189211in}{0.711445in}}{\pgfqpoint{1.184821in}{0.700846in}}{\pgfqpoint{1.184821in}{0.689796in}}%
\pgfpathcurveto{\pgfqpoint{1.184821in}{0.678745in}}{\pgfqpoint{1.189211in}{0.668146in}}{\pgfqpoint{1.197025in}{0.660333in}}%
\pgfpathcurveto{\pgfqpoint{1.204838in}{0.652519in}}{\pgfqpoint{1.215437in}{0.648129in}}{\pgfqpoint{1.226487in}{0.648129in}}%
\pgfpathclose%
\pgfusepath{stroke,fill}%
\end{pgfscope}%
\begin{pgfscope}%
\pgfpathrectangle{\pgfqpoint{0.648703in}{0.548769in}}{\pgfqpoint{5.201297in}{3.102590in}}%
\pgfusepath{clip}%
\pgfsetbuttcap%
\pgfsetroundjoin%
\definecolor{currentfill}{rgb}{0.121569,0.466667,0.705882}%
\pgfsetfillcolor{currentfill}%
\pgfsetlinewidth{1.003750pt}%
\definecolor{currentstroke}{rgb}{0.121569,0.466667,0.705882}%
\pgfsetstrokecolor{currentstroke}%
\pgfsetdash{}{0pt}%
\pgfpathmoveto{\pgfqpoint{1.502885in}{0.648129in}}%
\pgfpathcurveto{\pgfqpoint{1.513935in}{0.648129in}}{\pgfqpoint{1.524534in}{0.652519in}}{\pgfqpoint{1.532347in}{0.660333in}}%
\pgfpathcurveto{\pgfqpoint{1.540161in}{0.668146in}}{\pgfqpoint{1.544551in}{0.678745in}}{\pgfqpoint{1.544551in}{0.689796in}}%
\pgfpathcurveto{\pgfqpoint{1.544551in}{0.700846in}}{\pgfqpoint{1.540161in}{0.711445in}}{\pgfqpoint{1.532347in}{0.719258in}}%
\pgfpathcurveto{\pgfqpoint{1.524534in}{0.727072in}}{\pgfqpoint{1.513935in}{0.731462in}}{\pgfqpoint{1.502885in}{0.731462in}}%
\pgfpathcurveto{\pgfqpoint{1.491835in}{0.731462in}}{\pgfqpoint{1.481236in}{0.727072in}}{\pgfqpoint{1.473422in}{0.719258in}}%
\pgfpathcurveto{\pgfqpoint{1.465608in}{0.711445in}}{\pgfqpoint{1.461218in}{0.700846in}}{\pgfqpoint{1.461218in}{0.689796in}}%
\pgfpathcurveto{\pgfqpoint{1.461218in}{0.678745in}}{\pgfqpoint{1.465608in}{0.668146in}}{\pgfqpoint{1.473422in}{0.660333in}}%
\pgfpathcurveto{\pgfqpoint{1.481236in}{0.652519in}}{\pgfqpoint{1.491835in}{0.648129in}}{\pgfqpoint{1.502885in}{0.648129in}}%
\pgfpathclose%
\pgfusepath{stroke,fill}%
\end{pgfscope}%
\begin{pgfscope}%
\pgfpathrectangle{\pgfqpoint{0.648703in}{0.548769in}}{\pgfqpoint{5.201297in}{3.102590in}}%
\pgfusepath{clip}%
\pgfsetbuttcap%
\pgfsetroundjoin%
\definecolor{currentfill}{rgb}{0.121569,0.466667,0.705882}%
\pgfsetfillcolor{currentfill}%
\pgfsetlinewidth{1.003750pt}%
\definecolor{currentstroke}{rgb}{0.121569,0.466667,0.705882}%
\pgfsetstrokecolor{currentstroke}%
\pgfsetdash{}{0pt}%
\pgfpathmoveto{\pgfqpoint{1.198932in}{0.648129in}}%
\pgfpathcurveto{\pgfqpoint{1.209982in}{0.648129in}}{\pgfqpoint{1.220581in}{0.652519in}}{\pgfqpoint{1.228395in}{0.660333in}}%
\pgfpathcurveto{\pgfqpoint{1.236209in}{0.668146in}}{\pgfqpoint{1.240599in}{0.678745in}}{\pgfqpoint{1.240599in}{0.689796in}}%
\pgfpathcurveto{\pgfqpoint{1.240599in}{0.700846in}}{\pgfqpoint{1.236209in}{0.711445in}}{\pgfqpoint{1.228395in}{0.719258in}}%
\pgfpathcurveto{\pgfqpoint{1.220581in}{0.727072in}}{\pgfqpoint{1.209982in}{0.731462in}}{\pgfqpoint{1.198932in}{0.731462in}}%
\pgfpathcurveto{\pgfqpoint{1.187882in}{0.731462in}}{\pgfqpoint{1.177283in}{0.727072in}}{\pgfqpoint{1.169469in}{0.719258in}}%
\pgfpathcurveto{\pgfqpoint{1.161656in}{0.711445in}}{\pgfqpoint{1.157266in}{0.700846in}}{\pgfqpoint{1.157266in}{0.689796in}}%
\pgfpathcurveto{\pgfqpoint{1.157266in}{0.678745in}}{\pgfqpoint{1.161656in}{0.668146in}}{\pgfqpoint{1.169469in}{0.660333in}}%
\pgfpathcurveto{\pgfqpoint{1.177283in}{0.652519in}}{\pgfqpoint{1.187882in}{0.648129in}}{\pgfqpoint{1.198932in}{0.648129in}}%
\pgfpathclose%
\pgfusepath{stroke,fill}%
\end{pgfscope}%
\begin{pgfscope}%
\pgfpathrectangle{\pgfqpoint{0.648703in}{0.548769in}}{\pgfqpoint{5.201297in}{3.102590in}}%
\pgfusepath{clip}%
\pgfsetbuttcap%
\pgfsetroundjoin%
\definecolor{currentfill}{rgb}{0.121569,0.466667,0.705882}%
\pgfsetfillcolor{currentfill}%
\pgfsetlinewidth{1.003750pt}%
\definecolor{currentstroke}{rgb}{0.121569,0.466667,0.705882}%
\pgfsetstrokecolor{currentstroke}%
\pgfsetdash{}{0pt}%
\pgfpathmoveto{\pgfqpoint{1.276871in}{0.648129in}}%
\pgfpathcurveto{\pgfqpoint{1.287921in}{0.648129in}}{\pgfqpoint{1.298520in}{0.652519in}}{\pgfqpoint{1.306334in}{0.660333in}}%
\pgfpathcurveto{\pgfqpoint{1.314147in}{0.668146in}}{\pgfqpoint{1.318538in}{0.678745in}}{\pgfqpoint{1.318538in}{0.689796in}}%
\pgfpathcurveto{\pgfqpoint{1.318538in}{0.700846in}}{\pgfqpoint{1.314147in}{0.711445in}}{\pgfqpoint{1.306334in}{0.719258in}}%
\pgfpathcurveto{\pgfqpoint{1.298520in}{0.727072in}}{\pgfqpoint{1.287921in}{0.731462in}}{\pgfqpoint{1.276871in}{0.731462in}}%
\pgfpathcurveto{\pgfqpoint{1.265821in}{0.731462in}}{\pgfqpoint{1.255222in}{0.727072in}}{\pgfqpoint{1.247408in}{0.719258in}}%
\pgfpathcurveto{\pgfqpoint{1.239595in}{0.711445in}}{\pgfqpoint{1.235204in}{0.700846in}}{\pgfqpoint{1.235204in}{0.689796in}}%
\pgfpathcurveto{\pgfqpoint{1.235204in}{0.678745in}}{\pgfqpoint{1.239595in}{0.668146in}}{\pgfqpoint{1.247408in}{0.660333in}}%
\pgfpathcurveto{\pgfqpoint{1.255222in}{0.652519in}}{\pgfqpoint{1.265821in}{0.648129in}}{\pgfqpoint{1.276871in}{0.648129in}}%
\pgfpathclose%
\pgfusepath{stroke,fill}%
\end{pgfscope}%
\begin{pgfscope}%
\pgfpathrectangle{\pgfqpoint{0.648703in}{0.548769in}}{\pgfqpoint{5.201297in}{3.102590in}}%
\pgfusepath{clip}%
\pgfsetbuttcap%
\pgfsetroundjoin%
\definecolor{currentfill}{rgb}{0.121569,0.466667,0.705882}%
\pgfsetfillcolor{currentfill}%
\pgfsetlinewidth{1.003750pt}%
\definecolor{currentstroke}{rgb}{0.121569,0.466667,0.705882}%
\pgfsetstrokecolor{currentstroke}%
\pgfsetdash{}{0pt}%
\pgfpathmoveto{\pgfqpoint{2.544982in}{3.181114in}}%
\pgfpathcurveto{\pgfqpoint{2.556032in}{3.181114in}}{\pgfqpoint{2.566631in}{3.185504in}}{\pgfqpoint{2.574445in}{3.193318in}}%
\pgfpathcurveto{\pgfqpoint{2.582258in}{3.201132in}}{\pgfqpoint{2.586649in}{3.211731in}}{\pgfqpoint{2.586649in}{3.222781in}}%
\pgfpathcurveto{\pgfqpoint{2.586649in}{3.233831in}}{\pgfqpoint{2.582258in}{3.244430in}}{\pgfqpoint{2.574445in}{3.252244in}}%
\pgfpathcurveto{\pgfqpoint{2.566631in}{3.260057in}}{\pgfqpoint{2.556032in}{3.264448in}}{\pgfqpoint{2.544982in}{3.264448in}}%
\pgfpathcurveto{\pgfqpoint{2.533932in}{3.264448in}}{\pgfqpoint{2.523333in}{3.260057in}}{\pgfqpoint{2.515519in}{3.252244in}}%
\pgfpathcurveto{\pgfqpoint{2.507706in}{3.244430in}}{\pgfqpoint{2.503315in}{3.233831in}}{\pgfqpoint{2.503315in}{3.222781in}}%
\pgfpathcurveto{\pgfqpoint{2.503315in}{3.211731in}}{\pgfqpoint{2.507706in}{3.201132in}}{\pgfqpoint{2.515519in}{3.193318in}}%
\pgfpathcurveto{\pgfqpoint{2.523333in}{3.185504in}}{\pgfqpoint{2.533932in}{3.181114in}}{\pgfqpoint{2.544982in}{3.181114in}}%
\pgfpathclose%
\pgfusepath{stroke,fill}%
\end{pgfscope}%
\begin{pgfscope}%
\pgfpathrectangle{\pgfqpoint{0.648703in}{0.548769in}}{\pgfqpoint{5.201297in}{3.102590in}}%
\pgfusepath{clip}%
\pgfsetbuttcap%
\pgfsetroundjoin%
\definecolor{currentfill}{rgb}{1.000000,0.498039,0.054902}%
\pgfsetfillcolor{currentfill}%
\pgfsetlinewidth{1.003750pt}%
\definecolor{currentstroke}{rgb}{1.000000,0.498039,0.054902}%
\pgfsetstrokecolor{currentstroke}%
\pgfsetdash{}{0pt}%
\pgfpathmoveto{\pgfqpoint{2.006544in}{3.193800in}}%
\pgfpathcurveto{\pgfqpoint{2.017594in}{3.193800in}}{\pgfqpoint{2.028193in}{3.198191in}}{\pgfqpoint{2.036007in}{3.206004in}}%
\pgfpathcurveto{\pgfqpoint{2.043821in}{3.213818in}}{\pgfqpoint{2.048211in}{3.224417in}}{\pgfqpoint{2.048211in}{3.235467in}}%
\pgfpathcurveto{\pgfqpoint{2.048211in}{3.246517in}}{\pgfqpoint{2.043821in}{3.257116in}}{\pgfqpoint{2.036007in}{3.264930in}}%
\pgfpathcurveto{\pgfqpoint{2.028193in}{3.272743in}}{\pgfqpoint{2.017594in}{3.277134in}}{\pgfqpoint{2.006544in}{3.277134in}}%
\pgfpathcurveto{\pgfqpoint{1.995494in}{3.277134in}}{\pgfqpoint{1.984895in}{3.272743in}}{\pgfqpoint{1.977081in}{3.264930in}}%
\pgfpathcurveto{\pgfqpoint{1.969268in}{3.257116in}}{\pgfqpoint{1.964878in}{3.246517in}}{\pgfqpoint{1.964878in}{3.235467in}}%
\pgfpathcurveto{\pgfqpoint{1.964878in}{3.224417in}}{\pgfqpoint{1.969268in}{3.213818in}}{\pgfqpoint{1.977081in}{3.206004in}}%
\pgfpathcurveto{\pgfqpoint{1.984895in}{3.198191in}}{\pgfqpoint{1.995494in}{3.193800in}}{\pgfqpoint{2.006544in}{3.193800in}}%
\pgfpathclose%
\pgfusepath{stroke,fill}%
\end{pgfscope}%
\begin{pgfscope}%
\pgfpathrectangle{\pgfqpoint{0.648703in}{0.548769in}}{\pgfqpoint{5.201297in}{3.102590in}}%
\pgfusepath{clip}%
\pgfsetbuttcap%
\pgfsetroundjoin%
\definecolor{currentfill}{rgb}{1.000000,0.498039,0.054902}%
\pgfsetfillcolor{currentfill}%
\pgfsetlinewidth{1.003750pt}%
\definecolor{currentstroke}{rgb}{1.000000,0.498039,0.054902}%
\pgfsetstrokecolor{currentstroke}%
\pgfsetdash{}{0pt}%
\pgfpathmoveto{\pgfqpoint{1.593085in}{3.193800in}}%
\pgfpathcurveto{\pgfqpoint{1.604135in}{3.193800in}}{\pgfqpoint{1.614734in}{3.198191in}}{\pgfqpoint{1.622548in}{3.206004in}}%
\pgfpathcurveto{\pgfqpoint{1.630361in}{3.213818in}}{\pgfqpoint{1.634752in}{3.224417in}}{\pgfqpoint{1.634752in}{3.235467in}}%
\pgfpathcurveto{\pgfqpoint{1.634752in}{3.246517in}}{\pgfqpoint{1.630361in}{3.257116in}}{\pgfqpoint{1.622548in}{3.264930in}}%
\pgfpathcurveto{\pgfqpoint{1.614734in}{3.272743in}}{\pgfqpoint{1.604135in}{3.277134in}}{\pgfqpoint{1.593085in}{3.277134in}}%
\pgfpathcurveto{\pgfqpoint{1.582035in}{3.277134in}}{\pgfqpoint{1.571436in}{3.272743in}}{\pgfqpoint{1.563622in}{3.264930in}}%
\pgfpathcurveto{\pgfqpoint{1.555809in}{3.257116in}}{\pgfqpoint{1.551418in}{3.246517in}}{\pgfqpoint{1.551418in}{3.235467in}}%
\pgfpathcurveto{\pgfqpoint{1.551418in}{3.224417in}}{\pgfqpoint{1.555809in}{3.213818in}}{\pgfqpoint{1.563622in}{3.206004in}}%
\pgfpathcurveto{\pgfqpoint{1.571436in}{3.198191in}}{\pgfqpoint{1.582035in}{3.193800in}}{\pgfqpoint{1.593085in}{3.193800in}}%
\pgfpathclose%
\pgfusepath{stroke,fill}%
\end{pgfscope}%
\begin{pgfscope}%
\pgfpathrectangle{\pgfqpoint{0.648703in}{0.548769in}}{\pgfqpoint{5.201297in}{3.102590in}}%
\pgfusepath{clip}%
\pgfsetbuttcap%
\pgfsetroundjoin%
\definecolor{currentfill}{rgb}{0.121569,0.466667,0.705882}%
\pgfsetfillcolor{currentfill}%
\pgfsetlinewidth{1.003750pt}%
\definecolor{currentstroke}{rgb}{0.121569,0.466667,0.705882}%
\pgfsetstrokecolor{currentstroke}%
\pgfsetdash{}{0pt}%
\pgfpathmoveto{\pgfqpoint{1.387626in}{0.648129in}}%
\pgfpathcurveto{\pgfqpoint{1.398676in}{0.648129in}}{\pgfqpoint{1.409275in}{0.652519in}}{\pgfqpoint{1.417089in}{0.660333in}}%
\pgfpathcurveto{\pgfqpoint{1.424903in}{0.668146in}}{\pgfqpoint{1.429293in}{0.678745in}}{\pgfqpoint{1.429293in}{0.689796in}}%
\pgfpathcurveto{\pgfqpoint{1.429293in}{0.700846in}}{\pgfqpoint{1.424903in}{0.711445in}}{\pgfqpoint{1.417089in}{0.719258in}}%
\pgfpathcurveto{\pgfqpoint{1.409275in}{0.727072in}}{\pgfqpoint{1.398676in}{0.731462in}}{\pgfqpoint{1.387626in}{0.731462in}}%
\pgfpathcurveto{\pgfqpoint{1.376576in}{0.731462in}}{\pgfqpoint{1.365977in}{0.727072in}}{\pgfqpoint{1.358163in}{0.719258in}}%
\pgfpathcurveto{\pgfqpoint{1.350350in}{0.711445in}}{\pgfqpoint{1.345960in}{0.700846in}}{\pgfqpoint{1.345960in}{0.689796in}}%
\pgfpathcurveto{\pgfqpoint{1.345960in}{0.678745in}}{\pgfqpoint{1.350350in}{0.668146in}}{\pgfqpoint{1.358163in}{0.660333in}}%
\pgfpathcurveto{\pgfqpoint{1.365977in}{0.652519in}}{\pgfqpoint{1.376576in}{0.648129in}}{\pgfqpoint{1.387626in}{0.648129in}}%
\pgfpathclose%
\pgfusepath{stroke,fill}%
\end{pgfscope}%
\begin{pgfscope}%
\pgfpathrectangle{\pgfqpoint{0.648703in}{0.548769in}}{\pgfqpoint{5.201297in}{3.102590in}}%
\pgfusepath{clip}%
\pgfsetbuttcap%
\pgfsetroundjoin%
\definecolor{currentfill}{rgb}{0.121569,0.466667,0.705882}%
\pgfsetfillcolor{currentfill}%
\pgfsetlinewidth{1.003750pt}%
\definecolor{currentstroke}{rgb}{0.121569,0.466667,0.705882}%
\pgfsetstrokecolor{currentstroke}%
\pgfsetdash{}{0pt}%
\pgfpathmoveto{\pgfqpoint{1.261132in}{0.665044in}}%
\pgfpathcurveto{\pgfqpoint{1.272182in}{0.665044in}}{\pgfqpoint{1.282781in}{0.669434in}}{\pgfqpoint{1.290594in}{0.677248in}}%
\pgfpathcurveto{\pgfqpoint{1.298408in}{0.685061in}}{\pgfqpoint{1.302798in}{0.695660in}}{\pgfqpoint{1.302798in}{0.706710in}}%
\pgfpathcurveto{\pgfqpoint{1.302798in}{0.717760in}}{\pgfqpoint{1.298408in}{0.728360in}}{\pgfqpoint{1.290594in}{0.736173in}}%
\pgfpathcurveto{\pgfqpoint{1.282781in}{0.743987in}}{\pgfqpoint{1.272182in}{0.748377in}}{\pgfqpoint{1.261132in}{0.748377in}}%
\pgfpathcurveto{\pgfqpoint{1.250082in}{0.748377in}}{\pgfqpoint{1.239483in}{0.743987in}}{\pgfqpoint{1.231669in}{0.736173in}}%
\pgfpathcurveto{\pgfqpoint{1.223855in}{0.728360in}}{\pgfqpoint{1.219465in}{0.717760in}}{\pgfqpoint{1.219465in}{0.706710in}}%
\pgfpathcurveto{\pgfqpoint{1.219465in}{0.695660in}}{\pgfqpoint{1.223855in}{0.685061in}}{\pgfqpoint{1.231669in}{0.677248in}}%
\pgfpathcurveto{\pgfqpoint{1.239483in}{0.669434in}}{\pgfqpoint{1.250082in}{0.665044in}}{\pgfqpoint{1.261132in}{0.665044in}}%
\pgfpathclose%
\pgfusepath{stroke,fill}%
\end{pgfscope}%
\begin{pgfscope}%
\pgfpathrectangle{\pgfqpoint{0.648703in}{0.548769in}}{\pgfqpoint{5.201297in}{3.102590in}}%
\pgfusepath{clip}%
\pgfsetbuttcap%
\pgfsetroundjoin%
\definecolor{currentfill}{rgb}{1.000000,0.498039,0.054902}%
\pgfsetfillcolor{currentfill}%
\pgfsetlinewidth{1.003750pt}%
\definecolor{currentstroke}{rgb}{1.000000,0.498039,0.054902}%
\pgfsetstrokecolor{currentstroke}%
\pgfsetdash{}{0pt}%
\pgfpathmoveto{\pgfqpoint{1.276871in}{3.189572in}}%
\pgfpathcurveto{\pgfqpoint{1.287921in}{3.189572in}}{\pgfqpoint{1.298520in}{3.193962in}}{\pgfqpoint{1.306334in}{3.201775in}}%
\pgfpathcurveto{\pgfqpoint{1.314147in}{3.209589in}}{\pgfqpoint{1.318538in}{3.220188in}}{\pgfqpoint{1.318538in}{3.231238in}}%
\pgfpathcurveto{\pgfqpoint{1.318538in}{3.242288in}}{\pgfqpoint{1.314147in}{3.252887in}}{\pgfqpoint{1.306334in}{3.260701in}}%
\pgfpathcurveto{\pgfqpoint{1.298520in}{3.268515in}}{\pgfqpoint{1.287921in}{3.272905in}}{\pgfqpoint{1.276871in}{3.272905in}}%
\pgfpathcurveto{\pgfqpoint{1.265821in}{3.272905in}}{\pgfqpoint{1.255222in}{3.268515in}}{\pgfqpoint{1.247408in}{3.260701in}}%
\pgfpathcurveto{\pgfqpoint{1.239595in}{3.252887in}}{\pgfqpoint{1.235204in}{3.242288in}}{\pgfqpoint{1.235204in}{3.231238in}}%
\pgfpathcurveto{\pgfqpoint{1.235204in}{3.220188in}}{\pgfqpoint{1.239595in}{3.209589in}}{\pgfqpoint{1.247408in}{3.201775in}}%
\pgfpathcurveto{\pgfqpoint{1.255222in}{3.193962in}}{\pgfqpoint{1.265821in}{3.189572in}}{\pgfqpoint{1.276871in}{3.189572in}}%
\pgfpathclose%
\pgfusepath{stroke,fill}%
\end{pgfscope}%
\begin{pgfscope}%
\pgfpathrectangle{\pgfqpoint{0.648703in}{0.548769in}}{\pgfqpoint{5.201297in}{3.102590in}}%
\pgfusepath{clip}%
\pgfsetbuttcap%
\pgfsetroundjoin%
\definecolor{currentfill}{rgb}{0.121569,0.466667,0.705882}%
\pgfsetfillcolor{currentfill}%
\pgfsetlinewidth{1.003750pt}%
\definecolor{currentstroke}{rgb}{0.121569,0.466667,0.705882}%
\pgfsetstrokecolor{currentstroke}%
\pgfsetdash{}{0pt}%
\pgfpathmoveto{\pgfqpoint{0.889139in}{0.808819in}}%
\pgfpathcurveto{\pgfqpoint{0.900189in}{0.808819in}}{\pgfqpoint{0.910788in}{0.813209in}}{\pgfqpoint{0.918602in}{0.821023in}}%
\pgfpathcurveto{\pgfqpoint{0.926415in}{0.828837in}}{\pgfqpoint{0.930806in}{0.839436in}}{\pgfqpoint{0.930806in}{0.850486in}}%
\pgfpathcurveto{\pgfqpoint{0.930806in}{0.861536in}}{\pgfqpoint{0.926415in}{0.872135in}}{\pgfqpoint{0.918602in}{0.879949in}}%
\pgfpathcurveto{\pgfqpoint{0.910788in}{0.887762in}}{\pgfqpoint{0.900189in}{0.892152in}}{\pgfqpoint{0.889139in}{0.892152in}}%
\pgfpathcurveto{\pgfqpoint{0.878089in}{0.892152in}}{\pgfqpoint{0.867490in}{0.887762in}}{\pgfqpoint{0.859676in}{0.879949in}}%
\pgfpathcurveto{\pgfqpoint{0.851862in}{0.872135in}}{\pgfqpoint{0.847472in}{0.861536in}}{\pgfqpoint{0.847472in}{0.850486in}}%
\pgfpathcurveto{\pgfqpoint{0.847472in}{0.839436in}}{\pgfqpoint{0.851862in}{0.828837in}}{\pgfqpoint{0.859676in}{0.821023in}}%
\pgfpathcurveto{\pgfqpoint{0.867490in}{0.813209in}}{\pgfqpoint{0.878089in}{0.808819in}}{\pgfqpoint{0.889139in}{0.808819in}}%
\pgfpathclose%
\pgfusepath{stroke,fill}%
\end{pgfscope}%
\begin{pgfscope}%
\pgfpathrectangle{\pgfqpoint{0.648703in}{0.548769in}}{\pgfqpoint{5.201297in}{3.102590in}}%
\pgfusepath{clip}%
\pgfsetbuttcap%
\pgfsetroundjoin%
\definecolor{currentfill}{rgb}{0.839216,0.152941,0.156863}%
\pgfsetfillcolor{currentfill}%
\pgfsetlinewidth{1.003750pt}%
\definecolor{currentstroke}{rgb}{0.839216,0.152941,0.156863}%
\pgfsetstrokecolor{currentstroke}%
\pgfsetdash{}{0pt}%
\pgfpathmoveto{\pgfqpoint{1.774868in}{3.198029in}}%
\pgfpathcurveto{\pgfqpoint{1.785918in}{3.198029in}}{\pgfqpoint{1.796517in}{3.202419in}}{\pgfqpoint{1.804331in}{3.210233in}}%
\pgfpathcurveto{\pgfqpoint{1.812144in}{3.218046in}}{\pgfqpoint{1.816535in}{3.228646in}}{\pgfqpoint{1.816535in}{3.239696in}}%
\pgfpathcurveto{\pgfqpoint{1.816535in}{3.250746in}}{\pgfqpoint{1.812144in}{3.261345in}}{\pgfqpoint{1.804331in}{3.269158in}}%
\pgfpathcurveto{\pgfqpoint{1.796517in}{3.276972in}}{\pgfqpoint{1.785918in}{3.281362in}}{\pgfqpoint{1.774868in}{3.281362in}}%
\pgfpathcurveto{\pgfqpoint{1.763818in}{3.281362in}}{\pgfqpoint{1.753219in}{3.276972in}}{\pgfqpoint{1.745405in}{3.269158in}}%
\pgfpathcurveto{\pgfqpoint{1.737592in}{3.261345in}}{\pgfqpoint{1.733201in}{3.250746in}}{\pgfqpoint{1.733201in}{3.239696in}}%
\pgfpathcurveto{\pgfqpoint{1.733201in}{3.228646in}}{\pgfqpoint{1.737592in}{3.218046in}}{\pgfqpoint{1.745405in}{3.210233in}}%
\pgfpathcurveto{\pgfqpoint{1.753219in}{3.202419in}}{\pgfqpoint{1.763818in}{3.198029in}}{\pgfqpoint{1.774868in}{3.198029in}}%
\pgfpathclose%
\pgfusepath{stroke,fill}%
\end{pgfscope}%
\begin{pgfscope}%
\pgfpathrectangle{\pgfqpoint{0.648703in}{0.548769in}}{\pgfqpoint{5.201297in}{3.102590in}}%
\pgfusepath{clip}%
\pgfsetbuttcap%
\pgfsetroundjoin%
\definecolor{currentfill}{rgb}{0.121569,0.466667,0.705882}%
\pgfsetfillcolor{currentfill}%
\pgfsetlinewidth{1.003750pt}%
\definecolor{currentstroke}{rgb}{0.121569,0.466667,0.705882}%
\pgfsetstrokecolor{currentstroke}%
\pgfsetdash{}{0pt}%
\pgfpathmoveto{\pgfqpoint{1.061068in}{0.648129in}}%
\pgfpathcurveto{\pgfqpoint{1.072118in}{0.648129in}}{\pgfqpoint{1.082717in}{0.652519in}}{\pgfqpoint{1.090531in}{0.660333in}}%
\pgfpathcurveto{\pgfqpoint{1.098344in}{0.668146in}}{\pgfqpoint{1.102735in}{0.678745in}}{\pgfqpoint{1.102735in}{0.689796in}}%
\pgfpathcurveto{\pgfqpoint{1.102735in}{0.700846in}}{\pgfqpoint{1.098344in}{0.711445in}}{\pgfqpoint{1.090531in}{0.719258in}}%
\pgfpathcurveto{\pgfqpoint{1.082717in}{0.727072in}}{\pgfqpoint{1.072118in}{0.731462in}}{\pgfqpoint{1.061068in}{0.731462in}}%
\pgfpathcurveto{\pgfqpoint{1.050018in}{0.731462in}}{\pgfqpoint{1.039419in}{0.727072in}}{\pgfqpoint{1.031605in}{0.719258in}}%
\pgfpathcurveto{\pgfqpoint{1.023792in}{0.711445in}}{\pgfqpoint{1.019401in}{0.700846in}}{\pgfqpoint{1.019401in}{0.689796in}}%
\pgfpathcurveto{\pgfqpoint{1.019401in}{0.678745in}}{\pgfqpoint{1.023792in}{0.668146in}}{\pgfqpoint{1.031605in}{0.660333in}}%
\pgfpathcurveto{\pgfqpoint{1.039419in}{0.652519in}}{\pgfqpoint{1.050018in}{0.648129in}}{\pgfqpoint{1.061068in}{0.648129in}}%
\pgfpathclose%
\pgfusepath{stroke,fill}%
\end{pgfscope}%
\begin{pgfscope}%
\pgfpathrectangle{\pgfqpoint{0.648703in}{0.548769in}}{\pgfqpoint{5.201297in}{3.102590in}}%
\pgfusepath{clip}%
\pgfsetbuttcap%
\pgfsetroundjoin%
\definecolor{currentfill}{rgb}{1.000000,0.498039,0.054902}%
\pgfsetfillcolor{currentfill}%
\pgfsetlinewidth{1.003750pt}%
\definecolor{currentstroke}{rgb}{1.000000,0.498039,0.054902}%
\pgfsetstrokecolor{currentstroke}%
\pgfsetdash{}{0pt}%
\pgfpathmoveto{\pgfqpoint{1.744459in}{3.185343in}}%
\pgfpathcurveto{\pgfqpoint{1.755509in}{3.185343in}}{\pgfqpoint{1.766109in}{3.189733in}}{\pgfqpoint{1.773922in}{3.197547in}}%
\pgfpathcurveto{\pgfqpoint{1.781736in}{3.205360in}}{\pgfqpoint{1.786126in}{3.215959in}}{\pgfqpoint{1.786126in}{3.227010in}}%
\pgfpathcurveto{\pgfqpoint{1.786126in}{3.238060in}}{\pgfqpoint{1.781736in}{3.248659in}}{\pgfqpoint{1.773922in}{3.256472in}}%
\pgfpathcurveto{\pgfqpoint{1.766109in}{3.264286in}}{\pgfqpoint{1.755509in}{3.268676in}}{\pgfqpoint{1.744459in}{3.268676in}}%
\pgfpathcurveto{\pgfqpoint{1.733409in}{3.268676in}}{\pgfqpoint{1.722810in}{3.264286in}}{\pgfqpoint{1.714997in}{3.256472in}}%
\pgfpathcurveto{\pgfqpoint{1.707183in}{3.248659in}}{\pgfqpoint{1.702793in}{3.238060in}}{\pgfqpoint{1.702793in}{3.227010in}}%
\pgfpathcurveto{\pgfqpoint{1.702793in}{3.215959in}}{\pgfqpoint{1.707183in}{3.205360in}}{\pgfqpoint{1.714997in}{3.197547in}}%
\pgfpathcurveto{\pgfqpoint{1.722810in}{3.189733in}}{\pgfqpoint{1.733409in}{3.185343in}}{\pgfqpoint{1.744459in}{3.185343in}}%
\pgfpathclose%
\pgfusepath{stroke,fill}%
\end{pgfscope}%
\begin{pgfscope}%
\pgfpathrectangle{\pgfqpoint{0.648703in}{0.548769in}}{\pgfqpoint{5.201297in}{3.102590in}}%
\pgfusepath{clip}%
\pgfsetbuttcap%
\pgfsetroundjoin%
\definecolor{currentfill}{rgb}{0.121569,0.466667,0.705882}%
\pgfsetfillcolor{currentfill}%
\pgfsetlinewidth{1.003750pt}%
\definecolor{currentstroke}{rgb}{0.121569,0.466667,0.705882}%
\pgfsetstrokecolor{currentstroke}%
\pgfsetdash{}{0pt}%
\pgfpathmoveto{\pgfqpoint{2.243838in}{0.648129in}}%
\pgfpathcurveto{\pgfqpoint{2.254889in}{0.648129in}}{\pgfqpoint{2.265488in}{0.652519in}}{\pgfqpoint{2.273301in}{0.660333in}}%
\pgfpathcurveto{\pgfqpoint{2.281115in}{0.668146in}}{\pgfqpoint{2.285505in}{0.678745in}}{\pgfqpoint{2.285505in}{0.689796in}}%
\pgfpathcurveto{\pgfqpoint{2.285505in}{0.700846in}}{\pgfqpoint{2.281115in}{0.711445in}}{\pgfqpoint{2.273301in}{0.719258in}}%
\pgfpathcurveto{\pgfqpoint{2.265488in}{0.727072in}}{\pgfqpoint{2.254889in}{0.731462in}}{\pgfqpoint{2.243838in}{0.731462in}}%
\pgfpathcurveto{\pgfqpoint{2.232788in}{0.731462in}}{\pgfqpoint{2.222189in}{0.727072in}}{\pgfqpoint{2.214376in}{0.719258in}}%
\pgfpathcurveto{\pgfqpoint{2.206562in}{0.711445in}}{\pgfqpoint{2.202172in}{0.700846in}}{\pgfqpoint{2.202172in}{0.689796in}}%
\pgfpathcurveto{\pgfqpoint{2.202172in}{0.678745in}}{\pgfqpoint{2.206562in}{0.668146in}}{\pgfqpoint{2.214376in}{0.660333in}}%
\pgfpathcurveto{\pgfqpoint{2.222189in}{0.652519in}}{\pgfqpoint{2.232788in}{0.648129in}}{\pgfqpoint{2.243838in}{0.648129in}}%
\pgfpathclose%
\pgfusepath{stroke,fill}%
\end{pgfscope}%
\begin{pgfscope}%
\pgfpathrectangle{\pgfqpoint{0.648703in}{0.548769in}}{\pgfqpoint{5.201297in}{3.102590in}}%
\pgfusepath{clip}%
\pgfsetbuttcap%
\pgfsetroundjoin%
\definecolor{currentfill}{rgb}{1.000000,0.498039,0.054902}%
\pgfsetfillcolor{currentfill}%
\pgfsetlinewidth{1.003750pt}%
\definecolor{currentstroke}{rgb}{1.000000,0.498039,0.054902}%
\pgfsetstrokecolor{currentstroke}%
\pgfsetdash{}{0pt}%
\pgfpathmoveto{\pgfqpoint{1.349147in}{3.185343in}}%
\pgfpathcurveto{\pgfqpoint{1.360197in}{3.185343in}}{\pgfqpoint{1.370796in}{3.189733in}}{\pgfqpoint{1.378610in}{3.197547in}}%
\pgfpathcurveto{\pgfqpoint{1.386424in}{3.205360in}}{\pgfqpoint{1.390814in}{3.215959in}}{\pgfqpoint{1.390814in}{3.227010in}}%
\pgfpathcurveto{\pgfqpoint{1.390814in}{3.238060in}}{\pgfqpoint{1.386424in}{3.248659in}}{\pgfqpoint{1.378610in}{3.256472in}}%
\pgfpathcurveto{\pgfqpoint{1.370796in}{3.264286in}}{\pgfqpoint{1.360197in}{3.268676in}}{\pgfqpoint{1.349147in}{3.268676in}}%
\pgfpathcurveto{\pgfqpoint{1.338097in}{3.268676in}}{\pgfqpoint{1.327498in}{3.264286in}}{\pgfqpoint{1.319684in}{3.256472in}}%
\pgfpathcurveto{\pgfqpoint{1.311871in}{3.248659in}}{\pgfqpoint{1.307481in}{3.238060in}}{\pgfqpoint{1.307481in}{3.227010in}}%
\pgfpathcurveto{\pgfqpoint{1.307481in}{3.215959in}}{\pgfqpoint{1.311871in}{3.205360in}}{\pgfqpoint{1.319684in}{3.197547in}}%
\pgfpathcurveto{\pgfqpoint{1.327498in}{3.189733in}}{\pgfqpoint{1.338097in}{3.185343in}}{\pgfqpoint{1.349147in}{3.185343in}}%
\pgfpathclose%
\pgfusepath{stroke,fill}%
\end{pgfscope}%
\begin{pgfscope}%
\pgfpathrectangle{\pgfqpoint{0.648703in}{0.548769in}}{\pgfqpoint{5.201297in}{3.102590in}}%
\pgfusepath{clip}%
\pgfsetbuttcap%
\pgfsetroundjoin%
\definecolor{currentfill}{rgb}{0.121569,0.466667,0.705882}%
\pgfsetfillcolor{currentfill}%
\pgfsetlinewidth{1.003750pt}%
\definecolor{currentstroke}{rgb}{0.121569,0.466667,0.705882}%
\pgfsetstrokecolor{currentstroke}%
\pgfsetdash{}{0pt}%
\pgfpathmoveto{\pgfqpoint{1.816111in}{0.648129in}}%
\pgfpathcurveto{\pgfqpoint{1.827161in}{0.648129in}}{\pgfqpoint{1.837761in}{0.652519in}}{\pgfqpoint{1.845574in}{0.660333in}}%
\pgfpathcurveto{\pgfqpoint{1.853388in}{0.668146in}}{\pgfqpoint{1.857778in}{0.678745in}}{\pgfqpoint{1.857778in}{0.689796in}}%
\pgfpathcurveto{\pgfqpoint{1.857778in}{0.700846in}}{\pgfqpoint{1.853388in}{0.711445in}}{\pgfqpoint{1.845574in}{0.719258in}}%
\pgfpathcurveto{\pgfqpoint{1.837761in}{0.727072in}}{\pgfqpoint{1.827161in}{0.731462in}}{\pgfqpoint{1.816111in}{0.731462in}}%
\pgfpathcurveto{\pgfqpoint{1.805061in}{0.731462in}}{\pgfqpoint{1.794462in}{0.727072in}}{\pgfqpoint{1.786649in}{0.719258in}}%
\pgfpathcurveto{\pgfqpoint{1.778835in}{0.711445in}}{\pgfqpoint{1.774445in}{0.700846in}}{\pgfqpoint{1.774445in}{0.689796in}}%
\pgfpathcurveto{\pgfqpoint{1.774445in}{0.678745in}}{\pgfqpoint{1.778835in}{0.668146in}}{\pgfqpoint{1.786649in}{0.660333in}}%
\pgfpathcurveto{\pgfqpoint{1.794462in}{0.652519in}}{\pgfqpoint{1.805061in}{0.648129in}}{\pgfqpoint{1.816111in}{0.648129in}}%
\pgfpathclose%
\pgfusepath{stroke,fill}%
\end{pgfscope}%
\begin{pgfscope}%
\pgfpathrectangle{\pgfqpoint{0.648703in}{0.548769in}}{\pgfqpoint{5.201297in}{3.102590in}}%
\pgfusepath{clip}%
\pgfsetbuttcap%
\pgfsetroundjoin%
\definecolor{currentfill}{rgb}{1.000000,0.498039,0.054902}%
\pgfsetfillcolor{currentfill}%
\pgfsetlinewidth{1.003750pt}%
\definecolor{currentstroke}{rgb}{1.000000,0.498039,0.054902}%
\pgfsetstrokecolor{currentstroke}%
\pgfsetdash{}{0pt}%
\pgfpathmoveto{\pgfqpoint{1.037303in}{3.278374in}}%
\pgfpathcurveto{\pgfqpoint{1.048353in}{3.278374in}}{\pgfqpoint{1.058952in}{3.282764in}}{\pgfqpoint{1.066766in}{3.290578in}}%
\pgfpathcurveto{\pgfqpoint{1.074579in}{3.298392in}}{\pgfqpoint{1.078970in}{3.308991in}}{\pgfqpoint{1.078970in}{3.320041in}}%
\pgfpathcurveto{\pgfqpoint{1.078970in}{3.331091in}}{\pgfqpoint{1.074579in}{3.341690in}}{\pgfqpoint{1.066766in}{3.349504in}}%
\pgfpathcurveto{\pgfqpoint{1.058952in}{3.357317in}}{\pgfqpoint{1.048353in}{3.361707in}}{\pgfqpoint{1.037303in}{3.361707in}}%
\pgfpathcurveto{\pgfqpoint{1.026253in}{3.361707in}}{\pgfqpoint{1.015654in}{3.357317in}}{\pgfqpoint{1.007840in}{3.349504in}}%
\pgfpathcurveto{\pgfqpoint{1.000026in}{3.341690in}}{\pgfqpoint{0.995636in}{3.331091in}}{\pgfqpoint{0.995636in}{3.320041in}}%
\pgfpathcurveto{\pgfqpoint{0.995636in}{3.308991in}}{\pgfqpoint{1.000026in}{3.298392in}}{\pgfqpoint{1.007840in}{3.290578in}}%
\pgfpathcurveto{\pgfqpoint{1.015654in}{3.282764in}}{\pgfqpoint{1.026253in}{3.278374in}}{\pgfqpoint{1.037303in}{3.278374in}}%
\pgfpathclose%
\pgfusepath{stroke,fill}%
\end{pgfscope}%
\begin{pgfscope}%
\pgfpathrectangle{\pgfqpoint{0.648703in}{0.548769in}}{\pgfqpoint{5.201297in}{3.102590in}}%
\pgfusepath{clip}%
\pgfsetbuttcap%
\pgfsetroundjoin%
\definecolor{currentfill}{rgb}{0.121569,0.466667,0.705882}%
\pgfsetfillcolor{currentfill}%
\pgfsetlinewidth{1.003750pt}%
\definecolor{currentstroke}{rgb}{0.121569,0.466667,0.705882}%
\pgfsetstrokecolor{currentstroke}%
\pgfsetdash{}{0pt}%
\pgfpathmoveto{\pgfqpoint{1.409207in}{0.648129in}}%
\pgfpathcurveto{\pgfqpoint{1.420257in}{0.648129in}}{\pgfqpoint{1.430856in}{0.652519in}}{\pgfqpoint{1.438669in}{0.660333in}}%
\pgfpathcurveto{\pgfqpoint{1.446483in}{0.668146in}}{\pgfqpoint{1.450873in}{0.678745in}}{\pgfqpoint{1.450873in}{0.689796in}}%
\pgfpathcurveto{\pgfqpoint{1.450873in}{0.700846in}}{\pgfqpoint{1.446483in}{0.711445in}}{\pgfqpoint{1.438669in}{0.719258in}}%
\pgfpathcurveto{\pgfqpoint{1.430856in}{0.727072in}}{\pgfqpoint{1.420257in}{0.731462in}}{\pgfqpoint{1.409207in}{0.731462in}}%
\pgfpathcurveto{\pgfqpoint{1.398156in}{0.731462in}}{\pgfqpoint{1.387557in}{0.727072in}}{\pgfqpoint{1.379744in}{0.719258in}}%
\pgfpathcurveto{\pgfqpoint{1.371930in}{0.711445in}}{\pgfqpoint{1.367540in}{0.700846in}}{\pgfqpoint{1.367540in}{0.689796in}}%
\pgfpathcurveto{\pgfqpoint{1.367540in}{0.678745in}}{\pgfqpoint{1.371930in}{0.668146in}}{\pgfqpoint{1.379744in}{0.660333in}}%
\pgfpathcurveto{\pgfqpoint{1.387557in}{0.652519in}}{\pgfqpoint{1.398156in}{0.648129in}}{\pgfqpoint{1.409207in}{0.648129in}}%
\pgfpathclose%
\pgfusepath{stroke,fill}%
\end{pgfscope}%
\begin{pgfscope}%
\pgfpathrectangle{\pgfqpoint{0.648703in}{0.548769in}}{\pgfqpoint{5.201297in}{3.102590in}}%
\pgfusepath{clip}%
\pgfsetbuttcap%
\pgfsetroundjoin%
\definecolor{currentfill}{rgb}{1.000000,0.498039,0.054902}%
\pgfsetfillcolor{currentfill}%
\pgfsetlinewidth{1.003750pt}%
\definecolor{currentstroke}{rgb}{1.000000,0.498039,0.054902}%
\pgfsetstrokecolor{currentstroke}%
\pgfsetdash{}{0pt}%
\pgfpathmoveto{\pgfqpoint{1.406264in}{3.202258in}}%
\pgfpathcurveto{\pgfqpoint{1.417314in}{3.202258in}}{\pgfqpoint{1.427913in}{3.206648in}}{\pgfqpoint{1.435727in}{3.214462in}}%
\pgfpathcurveto{\pgfqpoint{1.443540in}{3.222275in}}{\pgfqpoint{1.447930in}{3.232874in}}{\pgfqpoint{1.447930in}{3.243924in}}%
\pgfpathcurveto{\pgfqpoint{1.447930in}{3.254974in}}{\pgfqpoint{1.443540in}{3.265573in}}{\pgfqpoint{1.435727in}{3.273387in}}%
\pgfpathcurveto{\pgfqpoint{1.427913in}{3.281201in}}{\pgfqpoint{1.417314in}{3.285591in}}{\pgfqpoint{1.406264in}{3.285591in}}%
\pgfpathcurveto{\pgfqpoint{1.395214in}{3.285591in}}{\pgfqpoint{1.384615in}{3.281201in}}{\pgfqpoint{1.376801in}{3.273387in}}%
\pgfpathcurveto{\pgfqpoint{1.368987in}{3.265573in}}{\pgfqpoint{1.364597in}{3.254974in}}{\pgfqpoint{1.364597in}{3.243924in}}%
\pgfpathcurveto{\pgfqpoint{1.364597in}{3.232874in}}{\pgfqpoint{1.368987in}{3.222275in}}{\pgfqpoint{1.376801in}{3.214462in}}%
\pgfpathcurveto{\pgfqpoint{1.384615in}{3.206648in}}{\pgfqpoint{1.395214in}{3.202258in}}{\pgfqpoint{1.406264in}{3.202258in}}%
\pgfpathclose%
\pgfusepath{stroke,fill}%
\end{pgfscope}%
\begin{pgfscope}%
\pgfpathrectangle{\pgfqpoint{0.648703in}{0.548769in}}{\pgfqpoint{5.201297in}{3.102590in}}%
\pgfusepath{clip}%
\pgfsetbuttcap%
\pgfsetroundjoin%
\definecolor{currentfill}{rgb}{1.000000,0.498039,0.054902}%
\pgfsetfillcolor{currentfill}%
\pgfsetlinewidth{1.003750pt}%
\definecolor{currentstroke}{rgb}{1.000000,0.498039,0.054902}%
\pgfsetstrokecolor{currentstroke}%
\pgfsetdash{}{0pt}%
\pgfpathmoveto{\pgfqpoint{1.994060in}{3.185343in}}%
\pgfpathcurveto{\pgfqpoint{2.005110in}{3.185343in}}{\pgfqpoint{2.015709in}{3.189733in}}{\pgfqpoint{2.023523in}{3.197547in}}%
\pgfpathcurveto{\pgfqpoint{2.031336in}{3.205360in}}{\pgfqpoint{2.035726in}{3.215959in}}{\pgfqpoint{2.035726in}{3.227010in}}%
\pgfpathcurveto{\pgfqpoint{2.035726in}{3.238060in}}{\pgfqpoint{2.031336in}{3.248659in}}{\pgfqpoint{2.023523in}{3.256472in}}%
\pgfpathcurveto{\pgfqpoint{2.015709in}{3.264286in}}{\pgfqpoint{2.005110in}{3.268676in}}{\pgfqpoint{1.994060in}{3.268676in}}%
\pgfpathcurveto{\pgfqpoint{1.983010in}{3.268676in}}{\pgfqpoint{1.972411in}{3.264286in}}{\pgfqpoint{1.964597in}{3.256472in}}%
\pgfpathcurveto{\pgfqpoint{1.956783in}{3.248659in}}{\pgfqpoint{1.952393in}{3.238060in}}{\pgfqpoint{1.952393in}{3.227010in}}%
\pgfpathcurveto{\pgfqpoint{1.952393in}{3.215959in}}{\pgfqpoint{1.956783in}{3.205360in}}{\pgfqpoint{1.964597in}{3.197547in}}%
\pgfpathcurveto{\pgfqpoint{1.972411in}{3.189733in}}{\pgfqpoint{1.983010in}{3.185343in}}{\pgfqpoint{1.994060in}{3.185343in}}%
\pgfpathclose%
\pgfusepath{stroke,fill}%
\end{pgfscope}%
\begin{pgfscope}%
\pgfpathrectangle{\pgfqpoint{0.648703in}{0.548769in}}{\pgfqpoint{5.201297in}{3.102590in}}%
\pgfusepath{clip}%
\pgfsetbuttcap%
\pgfsetroundjoin%
\definecolor{currentfill}{rgb}{0.121569,0.466667,0.705882}%
\pgfsetfillcolor{currentfill}%
\pgfsetlinewidth{1.003750pt}%
\definecolor{currentstroke}{rgb}{0.121569,0.466667,0.705882}%
\pgfsetstrokecolor{currentstroke}%
\pgfsetdash{}{0pt}%
\pgfpathmoveto{\pgfqpoint{1.627685in}{0.648129in}}%
\pgfpathcurveto{\pgfqpoint{1.638735in}{0.648129in}}{\pgfqpoint{1.649334in}{0.652519in}}{\pgfqpoint{1.657148in}{0.660333in}}%
\pgfpathcurveto{\pgfqpoint{1.664961in}{0.668146in}}{\pgfqpoint{1.669352in}{0.678745in}}{\pgfqpoint{1.669352in}{0.689796in}}%
\pgfpathcurveto{\pgfqpoint{1.669352in}{0.700846in}}{\pgfqpoint{1.664961in}{0.711445in}}{\pgfqpoint{1.657148in}{0.719258in}}%
\pgfpathcurveto{\pgfqpoint{1.649334in}{0.727072in}}{\pgfqpoint{1.638735in}{0.731462in}}{\pgfqpoint{1.627685in}{0.731462in}}%
\pgfpathcurveto{\pgfqpoint{1.616635in}{0.731462in}}{\pgfqpoint{1.606036in}{0.727072in}}{\pgfqpoint{1.598222in}{0.719258in}}%
\pgfpathcurveto{\pgfqpoint{1.590408in}{0.711445in}}{\pgfqpoint{1.586018in}{0.700846in}}{\pgfqpoint{1.586018in}{0.689796in}}%
\pgfpathcurveto{\pgfqpoint{1.586018in}{0.678745in}}{\pgfqpoint{1.590408in}{0.668146in}}{\pgfqpoint{1.598222in}{0.660333in}}%
\pgfpathcurveto{\pgfqpoint{1.606036in}{0.652519in}}{\pgfqpoint{1.616635in}{0.648129in}}{\pgfqpoint{1.627685in}{0.648129in}}%
\pgfpathclose%
\pgfusepath{stroke,fill}%
\end{pgfscope}%
\begin{pgfscope}%
\pgfpathrectangle{\pgfqpoint{0.648703in}{0.548769in}}{\pgfqpoint{5.201297in}{3.102590in}}%
\pgfusepath{clip}%
\pgfsetbuttcap%
\pgfsetroundjoin%
\definecolor{currentfill}{rgb}{1.000000,0.498039,0.054902}%
\pgfsetfillcolor{currentfill}%
\pgfsetlinewidth{1.003750pt}%
\definecolor{currentstroke}{rgb}{1.000000,0.498039,0.054902}%
\pgfsetstrokecolor{currentstroke}%
\pgfsetdash{}{0pt}%
\pgfpathmoveto{\pgfqpoint{1.312318in}{3.185343in}}%
\pgfpathcurveto{\pgfqpoint{1.323368in}{3.185343in}}{\pgfqpoint{1.333967in}{3.189733in}}{\pgfqpoint{1.341781in}{3.197547in}}%
\pgfpathcurveto{\pgfqpoint{1.349594in}{3.205360in}}{\pgfqpoint{1.353985in}{3.215959in}}{\pgfqpoint{1.353985in}{3.227010in}}%
\pgfpathcurveto{\pgfqpoint{1.353985in}{3.238060in}}{\pgfqpoint{1.349594in}{3.248659in}}{\pgfqpoint{1.341781in}{3.256472in}}%
\pgfpathcurveto{\pgfqpoint{1.333967in}{3.264286in}}{\pgfqpoint{1.323368in}{3.268676in}}{\pgfqpoint{1.312318in}{3.268676in}}%
\pgfpathcurveto{\pgfqpoint{1.301268in}{3.268676in}}{\pgfqpoint{1.290669in}{3.264286in}}{\pgfqpoint{1.282855in}{3.256472in}}%
\pgfpathcurveto{\pgfqpoint{1.275042in}{3.248659in}}{\pgfqpoint{1.270651in}{3.238060in}}{\pgfqpoint{1.270651in}{3.227010in}}%
\pgfpathcurveto{\pgfqpoint{1.270651in}{3.215959in}}{\pgfqpoint{1.275042in}{3.205360in}}{\pgfqpoint{1.282855in}{3.197547in}}%
\pgfpathcurveto{\pgfqpoint{1.290669in}{3.189733in}}{\pgfqpoint{1.301268in}{3.185343in}}{\pgfqpoint{1.312318in}{3.185343in}}%
\pgfpathclose%
\pgfusepath{stroke,fill}%
\end{pgfscope}%
\begin{pgfscope}%
\pgfpathrectangle{\pgfqpoint{0.648703in}{0.548769in}}{\pgfqpoint{5.201297in}{3.102590in}}%
\pgfusepath{clip}%
\pgfsetbuttcap%
\pgfsetroundjoin%
\definecolor{currentfill}{rgb}{1.000000,0.498039,0.054902}%
\pgfsetfillcolor{currentfill}%
\pgfsetlinewidth{1.003750pt}%
\definecolor{currentstroke}{rgb}{1.000000,0.498039,0.054902}%
\pgfsetstrokecolor{currentstroke}%
\pgfsetdash{}{0pt}%
\pgfpathmoveto{\pgfqpoint{0.946612in}{3.193800in}}%
\pgfpathcurveto{\pgfqpoint{0.957662in}{3.193800in}}{\pgfqpoint{0.968261in}{3.198191in}}{\pgfqpoint{0.976075in}{3.206004in}}%
\pgfpathcurveto{\pgfqpoint{0.983888in}{3.213818in}}{\pgfqpoint{0.988279in}{3.224417in}}{\pgfqpoint{0.988279in}{3.235467in}}%
\pgfpathcurveto{\pgfqpoint{0.988279in}{3.246517in}}{\pgfqpoint{0.983888in}{3.257116in}}{\pgfqpoint{0.976075in}{3.264930in}}%
\pgfpathcurveto{\pgfqpoint{0.968261in}{3.272743in}}{\pgfqpoint{0.957662in}{3.277134in}}{\pgfqpoint{0.946612in}{3.277134in}}%
\pgfpathcurveto{\pgfqpoint{0.935562in}{3.277134in}}{\pgfqpoint{0.924963in}{3.272743in}}{\pgfqpoint{0.917149in}{3.264930in}}%
\pgfpathcurveto{\pgfqpoint{0.909336in}{3.257116in}}{\pgfqpoint{0.904945in}{3.246517in}}{\pgfqpoint{0.904945in}{3.235467in}}%
\pgfpathcurveto{\pgfqpoint{0.904945in}{3.224417in}}{\pgfqpoint{0.909336in}{3.213818in}}{\pgfqpoint{0.917149in}{3.206004in}}%
\pgfpathcurveto{\pgfqpoint{0.924963in}{3.198191in}}{\pgfqpoint{0.935562in}{3.193800in}}{\pgfqpoint{0.946612in}{3.193800in}}%
\pgfpathclose%
\pgfusepath{stroke,fill}%
\end{pgfscope}%
\begin{pgfscope}%
\pgfpathrectangle{\pgfqpoint{0.648703in}{0.548769in}}{\pgfqpoint{5.201297in}{3.102590in}}%
\pgfusepath{clip}%
\pgfsetbuttcap%
\pgfsetroundjoin%
\definecolor{currentfill}{rgb}{0.121569,0.466667,0.705882}%
\pgfsetfillcolor{currentfill}%
\pgfsetlinewidth{1.003750pt}%
\definecolor{currentstroke}{rgb}{0.121569,0.466667,0.705882}%
\pgfsetstrokecolor{currentstroke}%
\pgfsetdash{}{0pt}%
\pgfpathmoveto{\pgfqpoint{1.731574in}{0.656586in}}%
\pgfpathcurveto{\pgfqpoint{1.742624in}{0.656586in}}{\pgfqpoint{1.753223in}{0.660977in}}{\pgfqpoint{1.761036in}{0.668790in}}%
\pgfpathcurveto{\pgfqpoint{1.768850in}{0.676604in}}{\pgfqpoint{1.773240in}{0.687203in}}{\pgfqpoint{1.773240in}{0.698253in}}%
\pgfpathcurveto{\pgfqpoint{1.773240in}{0.709303in}}{\pgfqpoint{1.768850in}{0.719902in}}{\pgfqpoint{1.761036in}{0.727716in}}%
\pgfpathcurveto{\pgfqpoint{1.753223in}{0.735529in}}{\pgfqpoint{1.742624in}{0.739920in}}{\pgfqpoint{1.731574in}{0.739920in}}%
\pgfpathcurveto{\pgfqpoint{1.720523in}{0.739920in}}{\pgfqpoint{1.709924in}{0.735529in}}{\pgfqpoint{1.702111in}{0.727716in}}%
\pgfpathcurveto{\pgfqpoint{1.694297in}{0.719902in}}{\pgfqpoint{1.689907in}{0.709303in}}{\pgfqpoint{1.689907in}{0.698253in}}%
\pgfpathcurveto{\pgfqpoint{1.689907in}{0.687203in}}{\pgfqpoint{1.694297in}{0.676604in}}{\pgfqpoint{1.702111in}{0.668790in}}%
\pgfpathcurveto{\pgfqpoint{1.709924in}{0.660977in}}{\pgfqpoint{1.720523in}{0.656586in}}{\pgfqpoint{1.731574in}{0.656586in}}%
\pgfpathclose%
\pgfusepath{stroke,fill}%
\end{pgfscope}%
\begin{pgfscope}%
\pgfpathrectangle{\pgfqpoint{0.648703in}{0.548769in}}{\pgfqpoint{5.201297in}{3.102590in}}%
\pgfusepath{clip}%
\pgfsetbuttcap%
\pgfsetroundjoin%
\definecolor{currentfill}{rgb}{1.000000,0.498039,0.054902}%
\pgfsetfillcolor{currentfill}%
\pgfsetlinewidth{1.003750pt}%
\definecolor{currentstroke}{rgb}{1.000000,0.498039,0.054902}%
\pgfsetstrokecolor{currentstroke}%
\pgfsetdash{}{0pt}%
\pgfpathmoveto{\pgfqpoint{2.470833in}{3.214944in}}%
\pgfpathcurveto{\pgfqpoint{2.481883in}{3.214944in}}{\pgfqpoint{2.492482in}{3.219334in}}{\pgfqpoint{2.500296in}{3.227148in}}%
\pgfpathcurveto{\pgfqpoint{2.508109in}{3.234961in}}{\pgfqpoint{2.512500in}{3.245560in}}{\pgfqpoint{2.512500in}{3.256610in}}%
\pgfpathcurveto{\pgfqpoint{2.512500in}{3.267661in}}{\pgfqpoint{2.508109in}{3.278260in}}{\pgfqpoint{2.500296in}{3.286073in}}%
\pgfpathcurveto{\pgfqpoint{2.492482in}{3.293887in}}{\pgfqpoint{2.481883in}{3.298277in}}{\pgfqpoint{2.470833in}{3.298277in}}%
\pgfpathcurveto{\pgfqpoint{2.459783in}{3.298277in}}{\pgfqpoint{2.449184in}{3.293887in}}{\pgfqpoint{2.441370in}{3.286073in}}%
\pgfpathcurveto{\pgfqpoint{2.433557in}{3.278260in}}{\pgfqpoint{2.429166in}{3.267661in}}{\pgfqpoint{2.429166in}{3.256610in}}%
\pgfpathcurveto{\pgfqpoint{2.429166in}{3.245560in}}{\pgfqpoint{2.433557in}{3.234961in}}{\pgfqpoint{2.441370in}{3.227148in}}%
\pgfpathcurveto{\pgfqpoint{2.449184in}{3.219334in}}{\pgfqpoint{2.459783in}{3.214944in}}{\pgfqpoint{2.470833in}{3.214944in}}%
\pgfpathclose%
\pgfusepath{stroke,fill}%
\end{pgfscope}%
\begin{pgfscope}%
\pgfpathrectangle{\pgfqpoint{0.648703in}{0.548769in}}{\pgfqpoint{5.201297in}{3.102590in}}%
\pgfusepath{clip}%
\pgfsetbuttcap%
\pgfsetroundjoin%
\definecolor{currentfill}{rgb}{1.000000,0.498039,0.054902}%
\pgfsetfillcolor{currentfill}%
\pgfsetlinewidth{1.003750pt}%
\definecolor{currentstroke}{rgb}{1.000000,0.498039,0.054902}%
\pgfsetstrokecolor{currentstroke}%
\pgfsetdash{}{0pt}%
\pgfpathmoveto{\pgfqpoint{1.635488in}{3.202258in}}%
\pgfpathcurveto{\pgfqpoint{1.646538in}{3.202258in}}{\pgfqpoint{1.657137in}{3.206648in}}{\pgfqpoint{1.664950in}{3.214462in}}%
\pgfpathcurveto{\pgfqpoint{1.672764in}{3.222275in}}{\pgfqpoint{1.677154in}{3.232874in}}{\pgfqpoint{1.677154in}{3.243924in}}%
\pgfpathcurveto{\pgfqpoint{1.677154in}{3.254974in}}{\pgfqpoint{1.672764in}{3.265573in}}{\pgfqpoint{1.664950in}{3.273387in}}%
\pgfpathcurveto{\pgfqpoint{1.657137in}{3.281201in}}{\pgfqpoint{1.646538in}{3.285591in}}{\pgfqpoint{1.635488in}{3.285591in}}%
\pgfpathcurveto{\pgfqpoint{1.624438in}{3.285591in}}{\pgfqpoint{1.613839in}{3.281201in}}{\pgfqpoint{1.606025in}{3.273387in}}%
\pgfpathcurveto{\pgfqpoint{1.598211in}{3.265573in}}{\pgfqpoint{1.593821in}{3.254974in}}{\pgfqpoint{1.593821in}{3.243924in}}%
\pgfpathcurveto{\pgfqpoint{1.593821in}{3.232874in}}{\pgfqpoint{1.598211in}{3.222275in}}{\pgfqpoint{1.606025in}{3.214462in}}%
\pgfpathcurveto{\pgfqpoint{1.613839in}{3.206648in}}{\pgfqpoint{1.624438in}{3.202258in}}{\pgfqpoint{1.635488in}{3.202258in}}%
\pgfpathclose%
\pgfusepath{stroke,fill}%
\end{pgfscope}%
\begin{pgfscope}%
\pgfpathrectangle{\pgfqpoint{0.648703in}{0.548769in}}{\pgfqpoint{5.201297in}{3.102590in}}%
\pgfusepath{clip}%
\pgfsetbuttcap%
\pgfsetroundjoin%
\definecolor{currentfill}{rgb}{1.000000,0.498039,0.054902}%
\pgfsetfillcolor{currentfill}%
\pgfsetlinewidth{1.003750pt}%
\definecolor{currentstroke}{rgb}{1.000000,0.498039,0.054902}%
\pgfsetstrokecolor{currentstroke}%
\pgfsetdash{}{0pt}%
\pgfpathmoveto{\pgfqpoint{1.771479in}{3.244545in}}%
\pgfpathcurveto{\pgfqpoint{1.782529in}{3.244545in}}{\pgfqpoint{1.793128in}{3.248935in}}{\pgfqpoint{1.800942in}{3.256748in}}%
\pgfpathcurveto{\pgfqpoint{1.808756in}{3.264562in}}{\pgfqpoint{1.813146in}{3.275161in}}{\pgfqpoint{1.813146in}{3.286211in}}%
\pgfpathcurveto{\pgfqpoint{1.813146in}{3.297261in}}{\pgfqpoint{1.808756in}{3.307860in}}{\pgfqpoint{1.800942in}{3.315674in}}%
\pgfpathcurveto{\pgfqpoint{1.793128in}{3.323488in}}{\pgfqpoint{1.782529in}{3.327878in}}{\pgfqpoint{1.771479in}{3.327878in}}%
\pgfpathcurveto{\pgfqpoint{1.760429in}{3.327878in}}{\pgfqpoint{1.749830in}{3.323488in}}{\pgfqpoint{1.742017in}{3.315674in}}%
\pgfpathcurveto{\pgfqpoint{1.734203in}{3.307860in}}{\pgfqpoint{1.729813in}{3.297261in}}{\pgfqpoint{1.729813in}{3.286211in}}%
\pgfpathcurveto{\pgfqpoint{1.729813in}{3.275161in}}{\pgfqpoint{1.734203in}{3.264562in}}{\pgfqpoint{1.742017in}{3.256748in}}%
\pgfpathcurveto{\pgfqpoint{1.749830in}{3.248935in}}{\pgfqpoint{1.760429in}{3.244545in}}{\pgfqpoint{1.771479in}{3.244545in}}%
\pgfpathclose%
\pgfusepath{stroke,fill}%
\end{pgfscope}%
\begin{pgfscope}%
\pgfpathrectangle{\pgfqpoint{0.648703in}{0.548769in}}{\pgfqpoint{5.201297in}{3.102590in}}%
\pgfusepath{clip}%
\pgfsetbuttcap%
\pgfsetroundjoin%
\definecolor{currentfill}{rgb}{1.000000,0.498039,0.054902}%
\pgfsetfillcolor{currentfill}%
\pgfsetlinewidth{1.003750pt}%
\definecolor{currentstroke}{rgb}{1.000000,0.498039,0.054902}%
\pgfsetstrokecolor{currentstroke}%
\pgfsetdash{}{0pt}%
\pgfpathmoveto{\pgfqpoint{1.512293in}{3.312204in}}%
\pgfpathcurveto{\pgfqpoint{1.523343in}{3.312204in}}{\pgfqpoint{1.533942in}{3.316594in}}{\pgfqpoint{1.541755in}{3.324407in}}%
\pgfpathcurveto{\pgfqpoint{1.549569in}{3.332221in}}{\pgfqpoint{1.553959in}{3.342820in}}{\pgfqpoint{1.553959in}{3.353870in}}%
\pgfpathcurveto{\pgfqpoint{1.553959in}{3.364920in}}{\pgfqpoint{1.549569in}{3.375519in}}{\pgfqpoint{1.541755in}{3.383333in}}%
\pgfpathcurveto{\pgfqpoint{1.533942in}{3.391147in}}{\pgfqpoint{1.523343in}{3.395537in}}{\pgfqpoint{1.512293in}{3.395537in}}%
\pgfpathcurveto{\pgfqpoint{1.501243in}{3.395537in}}{\pgfqpoint{1.490643in}{3.391147in}}{\pgfqpoint{1.482830in}{3.383333in}}%
\pgfpathcurveto{\pgfqpoint{1.475016in}{3.375519in}}{\pgfqpoint{1.470626in}{3.364920in}}{\pgfqpoint{1.470626in}{3.353870in}}%
\pgfpathcurveto{\pgfqpoint{1.470626in}{3.342820in}}{\pgfqpoint{1.475016in}{3.332221in}}{\pgfqpoint{1.482830in}{3.324407in}}%
\pgfpathcurveto{\pgfqpoint{1.490643in}{3.316594in}}{\pgfqpoint{1.501243in}{3.312204in}}{\pgfqpoint{1.512293in}{3.312204in}}%
\pgfpathclose%
\pgfusepath{stroke,fill}%
\end{pgfscope}%
\begin{pgfscope}%
\pgfpathrectangle{\pgfqpoint{0.648703in}{0.548769in}}{\pgfqpoint{5.201297in}{3.102590in}}%
\pgfusepath{clip}%
\pgfsetbuttcap%
\pgfsetroundjoin%
\definecolor{currentfill}{rgb}{1.000000,0.498039,0.054902}%
\pgfsetfillcolor{currentfill}%
\pgfsetlinewidth{1.003750pt}%
\definecolor{currentstroke}{rgb}{1.000000,0.498039,0.054902}%
\pgfsetstrokecolor{currentstroke}%
\pgfsetdash{}{0pt}%
\pgfpathmoveto{\pgfqpoint{0.997219in}{3.210715in}}%
\pgfpathcurveto{\pgfqpoint{1.008269in}{3.210715in}}{\pgfqpoint{1.018868in}{3.215105in}}{\pgfqpoint{1.026682in}{3.222919in}}%
\pgfpathcurveto{\pgfqpoint{1.034495in}{3.230733in}}{\pgfqpoint{1.038885in}{3.241332in}}{\pgfqpoint{1.038885in}{3.252382in}}%
\pgfpathcurveto{\pgfqpoint{1.038885in}{3.263432in}}{\pgfqpoint{1.034495in}{3.274031in}}{\pgfqpoint{1.026682in}{3.281844in}}%
\pgfpathcurveto{\pgfqpoint{1.018868in}{3.289658in}}{\pgfqpoint{1.008269in}{3.294048in}}{\pgfqpoint{0.997219in}{3.294048in}}%
\pgfpathcurveto{\pgfqpoint{0.986169in}{3.294048in}}{\pgfqpoint{0.975570in}{3.289658in}}{\pgfqpoint{0.967756in}{3.281844in}}%
\pgfpathcurveto{\pgfqpoint{0.959942in}{3.274031in}}{\pgfqpoint{0.955552in}{3.263432in}}{\pgfqpoint{0.955552in}{3.252382in}}%
\pgfpathcurveto{\pgfqpoint{0.955552in}{3.241332in}}{\pgfqpoint{0.959942in}{3.230733in}}{\pgfqpoint{0.967756in}{3.222919in}}%
\pgfpathcurveto{\pgfqpoint{0.975570in}{3.215105in}}{\pgfqpoint{0.986169in}{3.210715in}}{\pgfqpoint{0.997219in}{3.210715in}}%
\pgfpathclose%
\pgfusepath{stroke,fill}%
\end{pgfscope}%
\begin{pgfscope}%
\pgfpathrectangle{\pgfqpoint{0.648703in}{0.548769in}}{\pgfqpoint{5.201297in}{3.102590in}}%
\pgfusepath{clip}%
\pgfsetbuttcap%
\pgfsetroundjoin%
\definecolor{currentfill}{rgb}{1.000000,0.498039,0.054902}%
\pgfsetfillcolor{currentfill}%
\pgfsetlinewidth{1.003750pt}%
\definecolor{currentstroke}{rgb}{1.000000,0.498039,0.054902}%
\pgfsetstrokecolor{currentstroke}%
\pgfsetdash{}{0pt}%
\pgfpathmoveto{\pgfqpoint{1.614755in}{3.236087in}}%
\pgfpathcurveto{\pgfqpoint{1.625805in}{3.236087in}}{\pgfqpoint{1.636404in}{3.240477in}}{\pgfqpoint{1.644217in}{3.248291in}}%
\pgfpathcurveto{\pgfqpoint{1.652031in}{3.256105in}}{\pgfqpoint{1.656421in}{3.266704in}}{\pgfqpoint{1.656421in}{3.277754in}}%
\pgfpathcurveto{\pgfqpoint{1.656421in}{3.288804in}}{\pgfqpoint{1.652031in}{3.299403in}}{\pgfqpoint{1.644217in}{3.307217in}}%
\pgfpathcurveto{\pgfqpoint{1.636404in}{3.315030in}}{\pgfqpoint{1.625805in}{3.319421in}}{\pgfqpoint{1.614755in}{3.319421in}}%
\pgfpathcurveto{\pgfqpoint{1.603704in}{3.319421in}}{\pgfqpoint{1.593105in}{3.315030in}}{\pgfqpoint{1.585292in}{3.307217in}}%
\pgfpathcurveto{\pgfqpoint{1.577478in}{3.299403in}}{\pgfqpoint{1.573088in}{3.288804in}}{\pgfqpoint{1.573088in}{3.277754in}}%
\pgfpathcurveto{\pgfqpoint{1.573088in}{3.266704in}}{\pgfqpoint{1.577478in}{3.256105in}}{\pgfqpoint{1.585292in}{3.248291in}}%
\pgfpathcurveto{\pgfqpoint{1.593105in}{3.240477in}}{\pgfqpoint{1.603704in}{3.236087in}}{\pgfqpoint{1.614755in}{3.236087in}}%
\pgfpathclose%
\pgfusepath{stroke,fill}%
\end{pgfscope}%
\begin{pgfscope}%
\pgfpathrectangle{\pgfqpoint{0.648703in}{0.548769in}}{\pgfqpoint{5.201297in}{3.102590in}}%
\pgfusepath{clip}%
\pgfsetbuttcap%
\pgfsetroundjoin%
\definecolor{currentfill}{rgb}{0.121569,0.466667,0.705882}%
\pgfsetfillcolor{currentfill}%
\pgfsetlinewidth{1.003750pt}%
\definecolor{currentstroke}{rgb}{0.121569,0.466667,0.705882}%
\pgfsetstrokecolor{currentstroke}%
\pgfsetdash{}{0pt}%
\pgfpathmoveto{\pgfqpoint{1.609092in}{0.648129in}}%
\pgfpathcurveto{\pgfqpoint{1.620142in}{0.648129in}}{\pgfqpoint{1.630741in}{0.652519in}}{\pgfqpoint{1.638555in}{0.660333in}}%
\pgfpathcurveto{\pgfqpoint{1.646368in}{0.668146in}}{\pgfqpoint{1.650759in}{0.678745in}}{\pgfqpoint{1.650759in}{0.689796in}}%
\pgfpathcurveto{\pgfqpoint{1.650759in}{0.700846in}}{\pgfqpoint{1.646368in}{0.711445in}}{\pgfqpoint{1.638555in}{0.719258in}}%
\pgfpathcurveto{\pgfqpoint{1.630741in}{0.727072in}}{\pgfqpoint{1.620142in}{0.731462in}}{\pgfqpoint{1.609092in}{0.731462in}}%
\pgfpathcurveto{\pgfqpoint{1.598042in}{0.731462in}}{\pgfqpoint{1.587443in}{0.727072in}}{\pgfqpoint{1.579629in}{0.719258in}}%
\pgfpathcurveto{\pgfqpoint{1.571816in}{0.711445in}}{\pgfqpoint{1.567425in}{0.700846in}}{\pgfqpoint{1.567425in}{0.689796in}}%
\pgfpathcurveto{\pgfqpoint{1.567425in}{0.678745in}}{\pgfqpoint{1.571816in}{0.668146in}}{\pgfqpoint{1.579629in}{0.660333in}}%
\pgfpathcurveto{\pgfqpoint{1.587443in}{0.652519in}}{\pgfqpoint{1.598042in}{0.648129in}}{\pgfqpoint{1.609092in}{0.648129in}}%
\pgfpathclose%
\pgfusepath{stroke,fill}%
\end{pgfscope}%
\begin{pgfscope}%
\pgfpathrectangle{\pgfqpoint{0.648703in}{0.548769in}}{\pgfqpoint{5.201297in}{3.102590in}}%
\pgfusepath{clip}%
\pgfsetbuttcap%
\pgfsetroundjoin%
\definecolor{currentfill}{rgb}{0.121569,0.466667,0.705882}%
\pgfsetfillcolor{currentfill}%
\pgfsetlinewidth{1.003750pt}%
\definecolor{currentstroke}{rgb}{0.121569,0.466667,0.705882}%
\pgfsetstrokecolor{currentstroke}%
\pgfsetdash{}{0pt}%
\pgfpathmoveto{\pgfqpoint{1.460437in}{0.648129in}}%
\pgfpathcurveto{\pgfqpoint{1.471488in}{0.648129in}}{\pgfqpoint{1.482087in}{0.652519in}}{\pgfqpoint{1.489900in}{0.660333in}}%
\pgfpathcurveto{\pgfqpoint{1.497714in}{0.668146in}}{\pgfqpoint{1.502104in}{0.678745in}}{\pgfqpoint{1.502104in}{0.689796in}}%
\pgfpathcurveto{\pgfqpoint{1.502104in}{0.700846in}}{\pgfqpoint{1.497714in}{0.711445in}}{\pgfqpoint{1.489900in}{0.719258in}}%
\pgfpathcurveto{\pgfqpoint{1.482087in}{0.727072in}}{\pgfqpoint{1.471488in}{0.731462in}}{\pgfqpoint{1.460437in}{0.731462in}}%
\pgfpathcurveto{\pgfqpoint{1.449387in}{0.731462in}}{\pgfqpoint{1.438788in}{0.727072in}}{\pgfqpoint{1.430975in}{0.719258in}}%
\pgfpathcurveto{\pgfqpoint{1.423161in}{0.711445in}}{\pgfqpoint{1.418771in}{0.700846in}}{\pgfqpoint{1.418771in}{0.689796in}}%
\pgfpathcurveto{\pgfqpoint{1.418771in}{0.678745in}}{\pgfqpoint{1.423161in}{0.668146in}}{\pgfqpoint{1.430975in}{0.660333in}}%
\pgfpathcurveto{\pgfqpoint{1.438788in}{0.652519in}}{\pgfqpoint{1.449387in}{0.648129in}}{\pgfqpoint{1.460437in}{0.648129in}}%
\pgfpathclose%
\pgfusepath{stroke,fill}%
\end{pgfscope}%
\begin{pgfscope}%
\pgfpathrectangle{\pgfqpoint{0.648703in}{0.548769in}}{\pgfqpoint{5.201297in}{3.102590in}}%
\pgfusepath{clip}%
\pgfsetbuttcap%
\pgfsetroundjoin%
\definecolor{currentfill}{rgb}{0.121569,0.466667,0.705882}%
\pgfsetfillcolor{currentfill}%
\pgfsetlinewidth{1.003750pt}%
\definecolor{currentstroke}{rgb}{0.121569,0.466667,0.705882}%
\pgfsetstrokecolor{currentstroke}%
\pgfsetdash{}{0pt}%
\pgfpathmoveto{\pgfqpoint{1.456960in}{0.648129in}}%
\pgfpathcurveto{\pgfqpoint{1.468010in}{0.648129in}}{\pgfqpoint{1.478609in}{0.652519in}}{\pgfqpoint{1.486422in}{0.660333in}}%
\pgfpathcurveto{\pgfqpoint{1.494236in}{0.668146in}}{\pgfqpoint{1.498626in}{0.678745in}}{\pgfqpoint{1.498626in}{0.689796in}}%
\pgfpathcurveto{\pgfqpoint{1.498626in}{0.700846in}}{\pgfqpoint{1.494236in}{0.711445in}}{\pgfqpoint{1.486422in}{0.719258in}}%
\pgfpathcurveto{\pgfqpoint{1.478609in}{0.727072in}}{\pgfqpoint{1.468010in}{0.731462in}}{\pgfqpoint{1.456960in}{0.731462in}}%
\pgfpathcurveto{\pgfqpoint{1.445910in}{0.731462in}}{\pgfqpoint{1.435311in}{0.727072in}}{\pgfqpoint{1.427497in}{0.719258in}}%
\pgfpathcurveto{\pgfqpoint{1.419683in}{0.711445in}}{\pgfqpoint{1.415293in}{0.700846in}}{\pgfqpoint{1.415293in}{0.689796in}}%
\pgfpathcurveto{\pgfqpoint{1.415293in}{0.678745in}}{\pgfqpoint{1.419683in}{0.668146in}}{\pgfqpoint{1.427497in}{0.660333in}}%
\pgfpathcurveto{\pgfqpoint{1.435311in}{0.652519in}}{\pgfqpoint{1.445910in}{0.648129in}}{\pgfqpoint{1.456960in}{0.648129in}}%
\pgfpathclose%
\pgfusepath{stroke,fill}%
\end{pgfscope}%
\begin{pgfscope}%
\pgfpathrectangle{\pgfqpoint{0.648703in}{0.548769in}}{\pgfqpoint{5.201297in}{3.102590in}}%
\pgfusepath{clip}%
\pgfsetbuttcap%
\pgfsetroundjoin%
\definecolor{currentfill}{rgb}{1.000000,0.498039,0.054902}%
\pgfsetfillcolor{currentfill}%
\pgfsetlinewidth{1.003750pt}%
\definecolor{currentstroke}{rgb}{1.000000,0.498039,0.054902}%
\pgfsetstrokecolor{currentstroke}%
\pgfsetdash{}{0pt}%
\pgfpathmoveto{\pgfqpoint{1.313121in}{3.405235in}}%
\pgfpathcurveto{\pgfqpoint{1.324171in}{3.405235in}}{\pgfqpoint{1.334770in}{3.409625in}}{\pgfqpoint{1.342583in}{3.417439in}}%
\pgfpathcurveto{\pgfqpoint{1.350397in}{3.425252in}}{\pgfqpoint{1.354787in}{3.435851in}}{\pgfqpoint{1.354787in}{3.446901in}}%
\pgfpathcurveto{\pgfqpoint{1.354787in}{3.457952in}}{\pgfqpoint{1.350397in}{3.468551in}}{\pgfqpoint{1.342583in}{3.476364in}}%
\pgfpathcurveto{\pgfqpoint{1.334770in}{3.484178in}}{\pgfqpoint{1.324171in}{3.488568in}}{\pgfqpoint{1.313121in}{3.488568in}}%
\pgfpathcurveto{\pgfqpoint{1.302071in}{3.488568in}}{\pgfqpoint{1.291471in}{3.484178in}}{\pgfqpoint{1.283658in}{3.476364in}}%
\pgfpathcurveto{\pgfqpoint{1.275844in}{3.468551in}}{\pgfqpoint{1.271454in}{3.457952in}}{\pgfqpoint{1.271454in}{3.446901in}}%
\pgfpathcurveto{\pgfqpoint{1.271454in}{3.435851in}}{\pgfqpoint{1.275844in}{3.425252in}}{\pgfqpoint{1.283658in}{3.417439in}}%
\pgfpathcurveto{\pgfqpoint{1.291471in}{3.409625in}}{\pgfqpoint{1.302071in}{3.405235in}}{\pgfqpoint{1.313121in}{3.405235in}}%
\pgfpathclose%
\pgfusepath{stroke,fill}%
\end{pgfscope}%
\begin{pgfscope}%
\pgfpathrectangle{\pgfqpoint{0.648703in}{0.548769in}}{\pgfqpoint{5.201297in}{3.102590in}}%
\pgfusepath{clip}%
\pgfsetbuttcap%
\pgfsetroundjoin%
\definecolor{currentfill}{rgb}{1.000000,0.498039,0.054902}%
\pgfsetfillcolor{currentfill}%
\pgfsetlinewidth{1.003750pt}%
\definecolor{currentstroke}{rgb}{1.000000,0.498039,0.054902}%
\pgfsetstrokecolor{currentstroke}%
\pgfsetdash{}{0pt}%
\pgfpathmoveto{\pgfqpoint{1.744771in}{3.193800in}}%
\pgfpathcurveto{\pgfqpoint{1.755822in}{3.193800in}}{\pgfqpoint{1.766421in}{3.198191in}}{\pgfqpoint{1.774234in}{3.206004in}}%
\pgfpathcurveto{\pgfqpoint{1.782048in}{3.213818in}}{\pgfqpoint{1.786438in}{3.224417in}}{\pgfqpoint{1.786438in}{3.235467in}}%
\pgfpathcurveto{\pgfqpoint{1.786438in}{3.246517in}}{\pgfqpoint{1.782048in}{3.257116in}}{\pgfqpoint{1.774234in}{3.264930in}}%
\pgfpathcurveto{\pgfqpoint{1.766421in}{3.272743in}}{\pgfqpoint{1.755822in}{3.277134in}}{\pgfqpoint{1.744771in}{3.277134in}}%
\pgfpathcurveto{\pgfqpoint{1.733721in}{3.277134in}}{\pgfqpoint{1.723122in}{3.272743in}}{\pgfqpoint{1.715309in}{3.264930in}}%
\pgfpathcurveto{\pgfqpoint{1.707495in}{3.257116in}}{\pgfqpoint{1.703105in}{3.246517in}}{\pgfqpoint{1.703105in}{3.235467in}}%
\pgfpathcurveto{\pgfqpoint{1.703105in}{3.224417in}}{\pgfqpoint{1.707495in}{3.213818in}}{\pgfqpoint{1.715309in}{3.206004in}}%
\pgfpathcurveto{\pgfqpoint{1.723122in}{3.198191in}}{\pgfqpoint{1.733721in}{3.193800in}}{\pgfqpoint{1.744771in}{3.193800in}}%
\pgfpathclose%
\pgfusepath{stroke,fill}%
\end{pgfscope}%
\begin{pgfscope}%
\pgfpathrectangle{\pgfqpoint{0.648703in}{0.548769in}}{\pgfqpoint{5.201297in}{3.102590in}}%
\pgfusepath{clip}%
\pgfsetbuttcap%
\pgfsetroundjoin%
\definecolor{currentfill}{rgb}{1.000000,0.498039,0.054902}%
\pgfsetfillcolor{currentfill}%
\pgfsetlinewidth{1.003750pt}%
\definecolor{currentstroke}{rgb}{1.000000,0.498039,0.054902}%
\pgfsetstrokecolor{currentstroke}%
\pgfsetdash{}{0pt}%
\pgfpathmoveto{\pgfqpoint{1.621978in}{3.358719in}}%
\pgfpathcurveto{\pgfqpoint{1.633028in}{3.358719in}}{\pgfqpoint{1.643627in}{3.363109in}}{\pgfqpoint{1.651440in}{3.370923in}}%
\pgfpathcurveto{\pgfqpoint{1.659254in}{3.378737in}}{\pgfqpoint{1.663644in}{3.389336in}}{\pgfqpoint{1.663644in}{3.400386in}}%
\pgfpathcurveto{\pgfqpoint{1.663644in}{3.411436in}}{\pgfqpoint{1.659254in}{3.422035in}}{\pgfqpoint{1.651440in}{3.429849in}}%
\pgfpathcurveto{\pgfqpoint{1.643627in}{3.437662in}}{\pgfqpoint{1.633028in}{3.442053in}}{\pgfqpoint{1.621978in}{3.442053in}}%
\pgfpathcurveto{\pgfqpoint{1.610928in}{3.442053in}}{\pgfqpoint{1.600329in}{3.437662in}}{\pgfqpoint{1.592515in}{3.429849in}}%
\pgfpathcurveto{\pgfqpoint{1.584701in}{3.422035in}}{\pgfqpoint{1.580311in}{3.411436in}}{\pgfqpoint{1.580311in}{3.400386in}}%
\pgfpathcurveto{\pgfqpoint{1.580311in}{3.389336in}}{\pgfqpoint{1.584701in}{3.378737in}}{\pgfqpoint{1.592515in}{3.370923in}}%
\pgfpathcurveto{\pgfqpoint{1.600329in}{3.363109in}}{\pgfqpoint{1.610928in}{3.358719in}}{\pgfqpoint{1.621978in}{3.358719in}}%
\pgfpathclose%
\pgfusepath{stroke,fill}%
\end{pgfscope}%
\begin{pgfscope}%
\pgfpathrectangle{\pgfqpoint{0.648703in}{0.548769in}}{\pgfqpoint{5.201297in}{3.102590in}}%
\pgfusepath{clip}%
\pgfsetbuttcap%
\pgfsetroundjoin%
\definecolor{currentfill}{rgb}{0.121569,0.466667,0.705882}%
\pgfsetfillcolor{currentfill}%
\pgfsetlinewidth{1.003750pt}%
\definecolor{currentstroke}{rgb}{0.121569,0.466667,0.705882}%
\pgfsetstrokecolor{currentstroke}%
\pgfsetdash{}{0pt}%
\pgfpathmoveto{\pgfqpoint{0.946032in}{0.648129in}}%
\pgfpathcurveto{\pgfqpoint{0.957083in}{0.648129in}}{\pgfqpoint{0.967682in}{0.652519in}}{\pgfqpoint{0.975495in}{0.660333in}}%
\pgfpathcurveto{\pgfqpoint{0.983309in}{0.668146in}}{\pgfqpoint{0.987699in}{0.678745in}}{\pgfqpoint{0.987699in}{0.689796in}}%
\pgfpathcurveto{\pgfqpoint{0.987699in}{0.700846in}}{\pgfqpoint{0.983309in}{0.711445in}}{\pgfqpoint{0.975495in}{0.719258in}}%
\pgfpathcurveto{\pgfqpoint{0.967682in}{0.727072in}}{\pgfqpoint{0.957083in}{0.731462in}}{\pgfqpoint{0.946032in}{0.731462in}}%
\pgfpathcurveto{\pgfqpoint{0.934982in}{0.731462in}}{\pgfqpoint{0.924383in}{0.727072in}}{\pgfqpoint{0.916570in}{0.719258in}}%
\pgfpathcurveto{\pgfqpoint{0.908756in}{0.711445in}}{\pgfqpoint{0.904366in}{0.700846in}}{\pgfqpoint{0.904366in}{0.689796in}}%
\pgfpathcurveto{\pgfqpoint{0.904366in}{0.678745in}}{\pgfqpoint{0.908756in}{0.668146in}}{\pgfqpoint{0.916570in}{0.660333in}}%
\pgfpathcurveto{\pgfqpoint{0.924383in}{0.652519in}}{\pgfqpoint{0.934982in}{0.648129in}}{\pgfqpoint{0.946032in}{0.648129in}}%
\pgfpathclose%
\pgfusepath{stroke,fill}%
\end{pgfscope}%
\begin{pgfscope}%
\pgfpathrectangle{\pgfqpoint{0.648703in}{0.548769in}}{\pgfqpoint{5.201297in}{3.102590in}}%
\pgfusepath{clip}%
\pgfsetbuttcap%
\pgfsetroundjoin%
\definecolor{currentfill}{rgb}{0.121569,0.466667,0.705882}%
\pgfsetfillcolor{currentfill}%
\pgfsetlinewidth{1.003750pt}%
\definecolor{currentstroke}{rgb}{0.121569,0.466667,0.705882}%
\pgfsetstrokecolor{currentstroke}%
\pgfsetdash{}{0pt}%
\pgfpathmoveto{\pgfqpoint{0.885126in}{0.796133in}}%
\pgfpathcurveto{\pgfqpoint{0.896176in}{0.796133in}}{\pgfqpoint{0.906775in}{0.800523in}}{\pgfqpoint{0.914589in}{0.808337in}}%
\pgfpathcurveto{\pgfqpoint{0.922402in}{0.816151in}}{\pgfqpoint{0.926793in}{0.826750in}}{\pgfqpoint{0.926793in}{0.837800in}}%
\pgfpathcurveto{\pgfqpoint{0.926793in}{0.848850in}}{\pgfqpoint{0.922402in}{0.859449in}}{\pgfqpoint{0.914589in}{0.867263in}}%
\pgfpathcurveto{\pgfqpoint{0.906775in}{0.875076in}}{\pgfqpoint{0.896176in}{0.879466in}}{\pgfqpoint{0.885126in}{0.879466in}}%
\pgfpathcurveto{\pgfqpoint{0.874076in}{0.879466in}}{\pgfqpoint{0.863477in}{0.875076in}}{\pgfqpoint{0.855663in}{0.867263in}}%
\pgfpathcurveto{\pgfqpoint{0.847850in}{0.859449in}}{\pgfqpoint{0.843459in}{0.848850in}}{\pgfqpoint{0.843459in}{0.837800in}}%
\pgfpathcurveto{\pgfqpoint{0.843459in}{0.826750in}}{\pgfqpoint{0.847850in}{0.816151in}}{\pgfqpoint{0.855663in}{0.808337in}}%
\pgfpathcurveto{\pgfqpoint{0.863477in}{0.800523in}}{\pgfqpoint{0.874076in}{0.796133in}}{\pgfqpoint{0.885126in}{0.796133in}}%
\pgfpathclose%
\pgfusepath{stroke,fill}%
\end{pgfscope}%
\begin{pgfscope}%
\pgfpathrectangle{\pgfqpoint{0.648703in}{0.548769in}}{\pgfqpoint{5.201297in}{3.102590in}}%
\pgfusepath{clip}%
\pgfsetbuttcap%
\pgfsetroundjoin%
\definecolor{currentfill}{rgb}{0.121569,0.466667,0.705882}%
\pgfsetfillcolor{currentfill}%
\pgfsetlinewidth{1.003750pt}%
\definecolor{currentstroke}{rgb}{0.121569,0.466667,0.705882}%
\pgfsetstrokecolor{currentstroke}%
\pgfsetdash{}{0pt}%
\pgfpathmoveto{\pgfqpoint{1.957632in}{3.155742in}}%
\pgfpathcurveto{\pgfqpoint{1.968682in}{3.155742in}}{\pgfqpoint{1.979281in}{3.160132in}}{\pgfqpoint{1.987095in}{3.167946in}}%
\pgfpathcurveto{\pgfqpoint{1.994908in}{3.175760in}}{\pgfqpoint{1.999298in}{3.186359in}}{\pgfqpoint{1.999298in}{3.197409in}}%
\pgfpathcurveto{\pgfqpoint{1.999298in}{3.208459in}}{\pgfqpoint{1.994908in}{3.219058in}}{\pgfqpoint{1.987095in}{3.226872in}}%
\pgfpathcurveto{\pgfqpoint{1.979281in}{3.234685in}}{\pgfqpoint{1.968682in}{3.239075in}}{\pgfqpoint{1.957632in}{3.239075in}}%
\pgfpathcurveto{\pgfqpoint{1.946582in}{3.239075in}}{\pgfqpoint{1.935983in}{3.234685in}}{\pgfqpoint{1.928169in}{3.226872in}}%
\pgfpathcurveto{\pgfqpoint{1.920355in}{3.219058in}}{\pgfqpoint{1.915965in}{3.208459in}}{\pgfqpoint{1.915965in}{3.197409in}}%
\pgfpathcurveto{\pgfqpoint{1.915965in}{3.186359in}}{\pgfqpoint{1.920355in}{3.175760in}}{\pgfqpoint{1.928169in}{3.167946in}}%
\pgfpathcurveto{\pgfqpoint{1.935983in}{3.160132in}}{\pgfqpoint{1.946582in}{3.155742in}}{\pgfqpoint{1.957632in}{3.155742in}}%
\pgfpathclose%
\pgfusepath{stroke,fill}%
\end{pgfscope}%
\begin{pgfscope}%
\pgfpathrectangle{\pgfqpoint{0.648703in}{0.548769in}}{\pgfqpoint{5.201297in}{3.102590in}}%
\pgfusepath{clip}%
\pgfsetbuttcap%
\pgfsetroundjoin%
\definecolor{currentfill}{rgb}{0.121569,0.466667,0.705882}%
\pgfsetfillcolor{currentfill}%
\pgfsetlinewidth{1.003750pt}%
\definecolor{currentstroke}{rgb}{0.121569,0.466667,0.705882}%
\pgfsetstrokecolor{currentstroke}%
\pgfsetdash{}{0pt}%
\pgfpathmoveto{\pgfqpoint{1.575919in}{0.648129in}}%
\pgfpathcurveto{\pgfqpoint{1.586969in}{0.648129in}}{\pgfqpoint{1.597568in}{0.652519in}}{\pgfqpoint{1.605382in}{0.660333in}}%
\pgfpathcurveto{\pgfqpoint{1.613195in}{0.668146in}}{\pgfqpoint{1.617586in}{0.678745in}}{\pgfqpoint{1.617586in}{0.689796in}}%
\pgfpathcurveto{\pgfqpoint{1.617586in}{0.700846in}}{\pgfqpoint{1.613195in}{0.711445in}}{\pgfqpoint{1.605382in}{0.719258in}}%
\pgfpathcurveto{\pgfqpoint{1.597568in}{0.727072in}}{\pgfqpoint{1.586969in}{0.731462in}}{\pgfqpoint{1.575919in}{0.731462in}}%
\pgfpathcurveto{\pgfqpoint{1.564869in}{0.731462in}}{\pgfqpoint{1.554270in}{0.727072in}}{\pgfqpoint{1.546456in}{0.719258in}}%
\pgfpathcurveto{\pgfqpoint{1.538643in}{0.711445in}}{\pgfqpoint{1.534252in}{0.700846in}}{\pgfqpoint{1.534252in}{0.689796in}}%
\pgfpathcurveto{\pgfqpoint{1.534252in}{0.678745in}}{\pgfqpoint{1.538643in}{0.668146in}}{\pgfqpoint{1.546456in}{0.660333in}}%
\pgfpathcurveto{\pgfqpoint{1.554270in}{0.652519in}}{\pgfqpoint{1.564869in}{0.648129in}}{\pgfqpoint{1.575919in}{0.648129in}}%
\pgfpathclose%
\pgfusepath{stroke,fill}%
\end{pgfscope}%
\begin{pgfscope}%
\pgfpathrectangle{\pgfqpoint{0.648703in}{0.548769in}}{\pgfqpoint{5.201297in}{3.102590in}}%
\pgfusepath{clip}%
\pgfsetbuttcap%
\pgfsetroundjoin%
\definecolor{currentfill}{rgb}{0.121569,0.466667,0.705882}%
\pgfsetfillcolor{currentfill}%
\pgfsetlinewidth{1.003750pt}%
\definecolor{currentstroke}{rgb}{0.121569,0.466667,0.705882}%
\pgfsetstrokecolor{currentstroke}%
\pgfsetdash{}{0pt}%
\pgfpathmoveto{\pgfqpoint{1.160899in}{0.648129in}}%
\pgfpathcurveto{\pgfqpoint{1.171949in}{0.648129in}}{\pgfqpoint{1.182548in}{0.652519in}}{\pgfqpoint{1.190362in}{0.660333in}}%
\pgfpathcurveto{\pgfqpoint{1.198176in}{0.668146in}}{\pgfqpoint{1.202566in}{0.678745in}}{\pgfqpoint{1.202566in}{0.689796in}}%
\pgfpathcurveto{\pgfqpoint{1.202566in}{0.700846in}}{\pgfqpoint{1.198176in}{0.711445in}}{\pgfqpoint{1.190362in}{0.719258in}}%
\pgfpathcurveto{\pgfqpoint{1.182548in}{0.727072in}}{\pgfqpoint{1.171949in}{0.731462in}}{\pgfqpoint{1.160899in}{0.731462in}}%
\pgfpathcurveto{\pgfqpoint{1.149849in}{0.731462in}}{\pgfqpoint{1.139250in}{0.727072in}}{\pgfqpoint{1.131436in}{0.719258in}}%
\pgfpathcurveto{\pgfqpoint{1.123623in}{0.711445in}}{\pgfqpoint{1.119233in}{0.700846in}}{\pgfqpoint{1.119233in}{0.689796in}}%
\pgfpathcurveto{\pgfqpoint{1.119233in}{0.678745in}}{\pgfqpoint{1.123623in}{0.668146in}}{\pgfqpoint{1.131436in}{0.660333in}}%
\pgfpathcurveto{\pgfqpoint{1.139250in}{0.652519in}}{\pgfqpoint{1.149849in}{0.648129in}}{\pgfqpoint{1.160899in}{0.648129in}}%
\pgfpathclose%
\pgfusepath{stroke,fill}%
\end{pgfscope}%
\begin{pgfscope}%
\pgfpathrectangle{\pgfqpoint{0.648703in}{0.548769in}}{\pgfqpoint{5.201297in}{3.102590in}}%
\pgfusepath{clip}%
\pgfsetbuttcap%
\pgfsetroundjoin%
\definecolor{currentfill}{rgb}{0.121569,0.466667,0.705882}%
\pgfsetfillcolor{currentfill}%
\pgfsetlinewidth{1.003750pt}%
\definecolor{currentstroke}{rgb}{0.121569,0.466667,0.705882}%
\pgfsetstrokecolor{currentstroke}%
\pgfsetdash{}{0pt}%
\pgfpathmoveto{\pgfqpoint{0.900642in}{2.512981in}}%
\pgfpathcurveto{\pgfqpoint{0.911693in}{2.512981in}}{\pgfqpoint{0.922292in}{2.517371in}}{\pgfqpoint{0.930105in}{2.525185in}}%
\pgfpathcurveto{\pgfqpoint{0.937919in}{2.532999in}}{\pgfqpoint{0.942309in}{2.543598in}}{\pgfqpoint{0.942309in}{2.554648in}}%
\pgfpathcurveto{\pgfqpoint{0.942309in}{2.565698in}}{\pgfqpoint{0.937919in}{2.576297in}}{\pgfqpoint{0.930105in}{2.584111in}}%
\pgfpathcurveto{\pgfqpoint{0.922292in}{2.591924in}}{\pgfqpoint{0.911693in}{2.596315in}}{\pgfqpoint{0.900642in}{2.596315in}}%
\pgfpathcurveto{\pgfqpoint{0.889592in}{2.596315in}}{\pgfqpoint{0.878993in}{2.591924in}}{\pgfqpoint{0.871180in}{2.584111in}}%
\pgfpathcurveto{\pgfqpoint{0.863366in}{2.576297in}}{\pgfqpoint{0.858976in}{2.565698in}}{\pgfqpoint{0.858976in}{2.554648in}}%
\pgfpathcurveto{\pgfqpoint{0.858976in}{2.543598in}}{\pgfqpoint{0.863366in}{2.532999in}}{\pgfqpoint{0.871180in}{2.525185in}}%
\pgfpathcurveto{\pgfqpoint{0.878993in}{2.517371in}}{\pgfqpoint{0.889592in}{2.512981in}}{\pgfqpoint{0.900642in}{2.512981in}}%
\pgfpathclose%
\pgfusepath{stroke,fill}%
\end{pgfscope}%
\begin{pgfscope}%
\pgfpathrectangle{\pgfqpoint{0.648703in}{0.548769in}}{\pgfqpoint{5.201297in}{3.102590in}}%
\pgfusepath{clip}%
\pgfsetbuttcap%
\pgfsetroundjoin%
\definecolor{currentfill}{rgb}{1.000000,0.498039,0.054902}%
\pgfsetfillcolor{currentfill}%
\pgfsetlinewidth{1.003750pt}%
\definecolor{currentstroke}{rgb}{1.000000,0.498039,0.054902}%
\pgfsetstrokecolor{currentstroke}%
\pgfsetdash{}{0pt}%
\pgfpathmoveto{\pgfqpoint{1.195009in}{3.206486in}}%
\pgfpathcurveto{\pgfqpoint{1.206059in}{3.206486in}}{\pgfqpoint{1.216658in}{3.210877in}}{\pgfqpoint{1.224471in}{3.218690in}}%
\pgfpathcurveto{\pgfqpoint{1.232285in}{3.226504in}}{\pgfqpoint{1.236675in}{3.237103in}}{\pgfqpoint{1.236675in}{3.248153in}}%
\pgfpathcurveto{\pgfqpoint{1.236675in}{3.259203in}}{\pgfqpoint{1.232285in}{3.269802in}}{\pgfqpoint{1.224471in}{3.277616in}}%
\pgfpathcurveto{\pgfqpoint{1.216658in}{3.285429in}}{\pgfqpoint{1.206059in}{3.289820in}}{\pgfqpoint{1.195009in}{3.289820in}}%
\pgfpathcurveto{\pgfqpoint{1.183958in}{3.289820in}}{\pgfqpoint{1.173359in}{3.285429in}}{\pgfqpoint{1.165546in}{3.277616in}}%
\pgfpathcurveto{\pgfqpoint{1.157732in}{3.269802in}}{\pgfqpoint{1.153342in}{3.259203in}}{\pgfqpoint{1.153342in}{3.248153in}}%
\pgfpathcurveto{\pgfqpoint{1.153342in}{3.237103in}}{\pgfqpoint{1.157732in}{3.226504in}}{\pgfqpoint{1.165546in}{3.218690in}}%
\pgfpathcurveto{\pgfqpoint{1.173359in}{3.210877in}}{\pgfqpoint{1.183958in}{3.206486in}}{\pgfqpoint{1.195009in}{3.206486in}}%
\pgfpathclose%
\pgfusepath{stroke,fill}%
\end{pgfscope}%
\begin{pgfscope}%
\pgfpathrectangle{\pgfqpoint{0.648703in}{0.548769in}}{\pgfqpoint{5.201297in}{3.102590in}}%
\pgfusepath{clip}%
\pgfsetbuttcap%
\pgfsetroundjoin%
\definecolor{currentfill}{rgb}{1.000000,0.498039,0.054902}%
\pgfsetfillcolor{currentfill}%
\pgfsetlinewidth{1.003750pt}%
\definecolor{currentstroke}{rgb}{1.000000,0.498039,0.054902}%
\pgfsetstrokecolor{currentstroke}%
\pgfsetdash{}{0pt}%
\pgfpathmoveto{\pgfqpoint{1.885579in}{3.198029in}}%
\pgfpathcurveto{\pgfqpoint{1.896629in}{3.198029in}}{\pgfqpoint{1.907228in}{3.202419in}}{\pgfqpoint{1.915041in}{3.210233in}}%
\pgfpathcurveto{\pgfqpoint{1.922855in}{3.218046in}}{\pgfqpoint{1.927245in}{3.228646in}}{\pgfqpoint{1.927245in}{3.239696in}}%
\pgfpathcurveto{\pgfqpoint{1.927245in}{3.250746in}}{\pgfqpoint{1.922855in}{3.261345in}}{\pgfqpoint{1.915041in}{3.269158in}}%
\pgfpathcurveto{\pgfqpoint{1.907228in}{3.276972in}}{\pgfqpoint{1.896629in}{3.281362in}}{\pgfqpoint{1.885579in}{3.281362in}}%
\pgfpathcurveto{\pgfqpoint{1.874528in}{3.281362in}}{\pgfqpoint{1.863929in}{3.276972in}}{\pgfqpoint{1.856116in}{3.269158in}}%
\pgfpathcurveto{\pgfqpoint{1.848302in}{3.261345in}}{\pgfqpoint{1.843912in}{3.250746in}}{\pgfqpoint{1.843912in}{3.239696in}}%
\pgfpathcurveto{\pgfqpoint{1.843912in}{3.228646in}}{\pgfqpoint{1.848302in}{3.218046in}}{\pgfqpoint{1.856116in}{3.210233in}}%
\pgfpathcurveto{\pgfqpoint{1.863929in}{3.202419in}}{\pgfqpoint{1.874528in}{3.198029in}}{\pgfqpoint{1.885579in}{3.198029in}}%
\pgfpathclose%
\pgfusepath{stroke,fill}%
\end{pgfscope}%
\begin{pgfscope}%
\pgfpathrectangle{\pgfqpoint{0.648703in}{0.548769in}}{\pgfqpoint{5.201297in}{3.102590in}}%
\pgfusepath{clip}%
\pgfsetbuttcap%
\pgfsetroundjoin%
\definecolor{currentfill}{rgb}{1.000000,0.498039,0.054902}%
\pgfsetfillcolor{currentfill}%
\pgfsetlinewidth{1.003750pt}%
\definecolor{currentstroke}{rgb}{1.000000,0.498039,0.054902}%
\pgfsetstrokecolor{currentstroke}%
\pgfsetdash{}{0pt}%
\pgfpathmoveto{\pgfqpoint{2.272553in}{3.248773in}}%
\pgfpathcurveto{\pgfqpoint{2.283603in}{3.248773in}}{\pgfqpoint{2.294202in}{3.253164in}}{\pgfqpoint{2.302016in}{3.260977in}}%
\pgfpathcurveto{\pgfqpoint{2.309829in}{3.268791in}}{\pgfqpoint{2.314219in}{3.279390in}}{\pgfqpoint{2.314219in}{3.290440in}}%
\pgfpathcurveto{\pgfqpoint{2.314219in}{3.301490in}}{\pgfqpoint{2.309829in}{3.312089in}}{\pgfqpoint{2.302016in}{3.319903in}}%
\pgfpathcurveto{\pgfqpoint{2.294202in}{3.327716in}}{\pgfqpoint{2.283603in}{3.332107in}}{\pgfqpoint{2.272553in}{3.332107in}}%
\pgfpathcurveto{\pgfqpoint{2.261503in}{3.332107in}}{\pgfqpoint{2.250904in}{3.327716in}}{\pgfqpoint{2.243090in}{3.319903in}}%
\pgfpathcurveto{\pgfqpoint{2.235276in}{3.312089in}}{\pgfqpoint{2.230886in}{3.301490in}}{\pgfqpoint{2.230886in}{3.290440in}}%
\pgfpathcurveto{\pgfqpoint{2.230886in}{3.279390in}}{\pgfqpoint{2.235276in}{3.268791in}}{\pgfqpoint{2.243090in}{3.260977in}}%
\pgfpathcurveto{\pgfqpoint{2.250904in}{3.253164in}}{\pgfqpoint{2.261503in}{3.248773in}}{\pgfqpoint{2.272553in}{3.248773in}}%
\pgfpathclose%
\pgfusepath{stroke,fill}%
\end{pgfscope}%
\begin{pgfscope}%
\pgfpathrectangle{\pgfqpoint{0.648703in}{0.548769in}}{\pgfqpoint{5.201297in}{3.102590in}}%
\pgfusepath{clip}%
\pgfsetbuttcap%
\pgfsetroundjoin%
\definecolor{currentfill}{rgb}{1.000000,0.498039,0.054902}%
\pgfsetfillcolor{currentfill}%
\pgfsetlinewidth{1.003750pt}%
\definecolor{currentstroke}{rgb}{1.000000,0.498039,0.054902}%
\pgfsetstrokecolor{currentstroke}%
\pgfsetdash{}{0pt}%
\pgfpathmoveto{\pgfqpoint{1.770231in}{3.189572in}}%
\pgfpathcurveto{\pgfqpoint{1.781281in}{3.189572in}}{\pgfqpoint{1.791880in}{3.193962in}}{\pgfqpoint{1.799694in}{3.201775in}}%
\pgfpathcurveto{\pgfqpoint{1.807507in}{3.209589in}}{\pgfqpoint{1.811898in}{3.220188in}}{\pgfqpoint{1.811898in}{3.231238in}}%
\pgfpathcurveto{\pgfqpoint{1.811898in}{3.242288in}}{\pgfqpoint{1.807507in}{3.252887in}}{\pgfqpoint{1.799694in}{3.260701in}}%
\pgfpathcurveto{\pgfqpoint{1.791880in}{3.268515in}}{\pgfqpoint{1.781281in}{3.272905in}}{\pgfqpoint{1.770231in}{3.272905in}}%
\pgfpathcurveto{\pgfqpoint{1.759181in}{3.272905in}}{\pgfqpoint{1.748582in}{3.268515in}}{\pgfqpoint{1.740768in}{3.260701in}}%
\pgfpathcurveto{\pgfqpoint{1.732954in}{3.252887in}}{\pgfqpoint{1.728564in}{3.242288in}}{\pgfqpoint{1.728564in}{3.231238in}}%
\pgfpathcurveto{\pgfqpoint{1.728564in}{3.220188in}}{\pgfqpoint{1.732954in}{3.209589in}}{\pgfqpoint{1.740768in}{3.201775in}}%
\pgfpathcurveto{\pgfqpoint{1.748582in}{3.193962in}}{\pgfqpoint{1.759181in}{3.189572in}}{\pgfqpoint{1.770231in}{3.189572in}}%
\pgfpathclose%
\pgfusepath{stroke,fill}%
\end{pgfscope}%
\begin{pgfscope}%
\pgfpathrectangle{\pgfqpoint{0.648703in}{0.548769in}}{\pgfqpoint{5.201297in}{3.102590in}}%
\pgfusepath{clip}%
\pgfsetbuttcap%
\pgfsetroundjoin%
\definecolor{currentfill}{rgb}{0.121569,0.466667,0.705882}%
\pgfsetfillcolor{currentfill}%
\pgfsetlinewidth{1.003750pt}%
\definecolor{currentstroke}{rgb}{0.121569,0.466667,0.705882}%
\pgfsetstrokecolor{currentstroke}%
\pgfsetdash{}{0pt}%
\pgfpathmoveto{\pgfqpoint{1.675483in}{3.181114in}}%
\pgfpathcurveto{\pgfqpoint{1.686533in}{3.181114in}}{\pgfqpoint{1.697132in}{3.185504in}}{\pgfqpoint{1.704945in}{3.193318in}}%
\pgfpathcurveto{\pgfqpoint{1.712759in}{3.201132in}}{\pgfqpoint{1.717149in}{3.211731in}}{\pgfqpoint{1.717149in}{3.222781in}}%
\pgfpathcurveto{\pgfqpoint{1.717149in}{3.233831in}}{\pgfqpoint{1.712759in}{3.244430in}}{\pgfqpoint{1.704945in}{3.252244in}}%
\pgfpathcurveto{\pgfqpoint{1.697132in}{3.260057in}}{\pgfqpoint{1.686533in}{3.264448in}}{\pgfqpoint{1.675483in}{3.264448in}}%
\pgfpathcurveto{\pgfqpoint{1.664432in}{3.264448in}}{\pgfqpoint{1.653833in}{3.260057in}}{\pgfqpoint{1.646020in}{3.252244in}}%
\pgfpathcurveto{\pgfqpoint{1.638206in}{3.244430in}}{\pgfqpoint{1.633816in}{3.233831in}}{\pgfqpoint{1.633816in}{3.222781in}}%
\pgfpathcurveto{\pgfqpoint{1.633816in}{3.211731in}}{\pgfqpoint{1.638206in}{3.201132in}}{\pgfqpoint{1.646020in}{3.193318in}}%
\pgfpathcurveto{\pgfqpoint{1.653833in}{3.185504in}}{\pgfqpoint{1.664432in}{3.181114in}}{\pgfqpoint{1.675483in}{3.181114in}}%
\pgfpathclose%
\pgfusepath{stroke,fill}%
\end{pgfscope}%
\begin{pgfscope}%
\pgfpathrectangle{\pgfqpoint{0.648703in}{0.548769in}}{\pgfqpoint{5.201297in}{3.102590in}}%
\pgfusepath{clip}%
\pgfsetbuttcap%
\pgfsetroundjoin%
\definecolor{currentfill}{rgb}{0.839216,0.152941,0.156863}%
\pgfsetfillcolor{currentfill}%
\pgfsetlinewidth{1.003750pt}%
\definecolor{currentstroke}{rgb}{0.839216,0.152941,0.156863}%
\pgfsetstrokecolor{currentstroke}%
\pgfsetdash{}{0pt}%
\pgfpathmoveto{\pgfqpoint{1.619213in}{3.202258in}}%
\pgfpathcurveto{\pgfqpoint{1.630263in}{3.202258in}}{\pgfqpoint{1.640862in}{3.206648in}}{\pgfqpoint{1.648676in}{3.214462in}}%
\pgfpathcurveto{\pgfqpoint{1.656490in}{3.222275in}}{\pgfqpoint{1.660880in}{3.232874in}}{\pgfqpoint{1.660880in}{3.243924in}}%
\pgfpathcurveto{\pgfqpoint{1.660880in}{3.254974in}}{\pgfqpoint{1.656490in}{3.265573in}}{\pgfqpoint{1.648676in}{3.273387in}}%
\pgfpathcurveto{\pgfqpoint{1.640862in}{3.281201in}}{\pgfqpoint{1.630263in}{3.285591in}}{\pgfqpoint{1.619213in}{3.285591in}}%
\pgfpathcurveto{\pgfqpoint{1.608163in}{3.285591in}}{\pgfqpoint{1.597564in}{3.281201in}}{\pgfqpoint{1.589751in}{3.273387in}}%
\pgfpathcurveto{\pgfqpoint{1.581937in}{3.265573in}}{\pgfqpoint{1.577547in}{3.254974in}}{\pgfqpoint{1.577547in}{3.243924in}}%
\pgfpathcurveto{\pgfqpoint{1.577547in}{3.232874in}}{\pgfqpoint{1.581937in}{3.222275in}}{\pgfqpoint{1.589751in}{3.214462in}}%
\pgfpathcurveto{\pgfqpoint{1.597564in}{3.206648in}}{\pgfqpoint{1.608163in}{3.202258in}}{\pgfqpoint{1.619213in}{3.202258in}}%
\pgfpathclose%
\pgfusepath{stroke,fill}%
\end{pgfscope}%
\begin{pgfscope}%
\pgfpathrectangle{\pgfqpoint{0.648703in}{0.548769in}}{\pgfqpoint{5.201297in}{3.102590in}}%
\pgfusepath{clip}%
\pgfsetbuttcap%
\pgfsetroundjoin%
\definecolor{currentfill}{rgb}{1.000000,0.498039,0.054902}%
\pgfsetfillcolor{currentfill}%
\pgfsetlinewidth{1.003750pt}%
\definecolor{currentstroke}{rgb}{1.000000,0.498039,0.054902}%
\pgfsetstrokecolor{currentstroke}%
\pgfsetdash{}{0pt}%
\pgfpathmoveto{\pgfqpoint{1.336306in}{3.185343in}}%
\pgfpathcurveto{\pgfqpoint{1.347356in}{3.185343in}}{\pgfqpoint{1.357955in}{3.189733in}}{\pgfqpoint{1.365769in}{3.197547in}}%
\pgfpathcurveto{\pgfqpoint{1.373583in}{3.205360in}}{\pgfqpoint{1.377973in}{3.215959in}}{\pgfqpoint{1.377973in}{3.227010in}}%
\pgfpathcurveto{\pgfqpoint{1.377973in}{3.238060in}}{\pgfqpoint{1.373583in}{3.248659in}}{\pgfqpoint{1.365769in}{3.256472in}}%
\pgfpathcurveto{\pgfqpoint{1.357955in}{3.264286in}}{\pgfqpoint{1.347356in}{3.268676in}}{\pgfqpoint{1.336306in}{3.268676in}}%
\pgfpathcurveto{\pgfqpoint{1.325256in}{3.268676in}}{\pgfqpoint{1.314657in}{3.264286in}}{\pgfqpoint{1.306843in}{3.256472in}}%
\pgfpathcurveto{\pgfqpoint{1.299030in}{3.248659in}}{\pgfqpoint{1.294639in}{3.238060in}}{\pgfqpoint{1.294639in}{3.227010in}}%
\pgfpathcurveto{\pgfqpoint{1.294639in}{3.215959in}}{\pgfqpoint{1.299030in}{3.205360in}}{\pgfqpoint{1.306843in}{3.197547in}}%
\pgfpathcurveto{\pgfqpoint{1.314657in}{3.189733in}}{\pgfqpoint{1.325256in}{3.185343in}}{\pgfqpoint{1.336306in}{3.185343in}}%
\pgfpathclose%
\pgfusepath{stroke,fill}%
\end{pgfscope}%
\begin{pgfscope}%
\pgfpathrectangle{\pgfqpoint{0.648703in}{0.548769in}}{\pgfqpoint{5.201297in}{3.102590in}}%
\pgfusepath{clip}%
\pgfsetbuttcap%
\pgfsetroundjoin%
\definecolor{currentfill}{rgb}{1.000000,0.498039,0.054902}%
\pgfsetfillcolor{currentfill}%
\pgfsetlinewidth{1.003750pt}%
\definecolor{currentstroke}{rgb}{1.000000,0.498039,0.054902}%
\pgfsetstrokecolor{currentstroke}%
\pgfsetdash{}{0pt}%
\pgfpathmoveto{\pgfqpoint{2.080738in}{3.278374in}}%
\pgfpathcurveto{\pgfqpoint{2.091788in}{3.278374in}}{\pgfqpoint{2.102387in}{3.282764in}}{\pgfqpoint{2.110200in}{3.290578in}}%
\pgfpathcurveto{\pgfqpoint{2.118014in}{3.298392in}}{\pgfqpoint{2.122404in}{3.308991in}}{\pgfqpoint{2.122404in}{3.320041in}}%
\pgfpathcurveto{\pgfqpoint{2.122404in}{3.331091in}}{\pgfqpoint{2.118014in}{3.341690in}}{\pgfqpoint{2.110200in}{3.349504in}}%
\pgfpathcurveto{\pgfqpoint{2.102387in}{3.357317in}}{\pgfqpoint{2.091788in}{3.361707in}}{\pgfqpoint{2.080738in}{3.361707in}}%
\pgfpathcurveto{\pgfqpoint{2.069688in}{3.361707in}}{\pgfqpoint{2.059089in}{3.357317in}}{\pgfqpoint{2.051275in}{3.349504in}}%
\pgfpathcurveto{\pgfqpoint{2.043461in}{3.341690in}}{\pgfqpoint{2.039071in}{3.331091in}}{\pgfqpoint{2.039071in}{3.320041in}}%
\pgfpathcurveto{\pgfqpoint{2.039071in}{3.308991in}}{\pgfqpoint{2.043461in}{3.298392in}}{\pgfqpoint{2.051275in}{3.290578in}}%
\pgfpathcurveto{\pgfqpoint{2.059089in}{3.282764in}}{\pgfqpoint{2.069688in}{3.278374in}}{\pgfqpoint{2.080738in}{3.278374in}}%
\pgfpathclose%
\pgfusepath{stroke,fill}%
\end{pgfscope}%
\begin{pgfscope}%
\pgfpathrectangle{\pgfqpoint{0.648703in}{0.548769in}}{\pgfqpoint{5.201297in}{3.102590in}}%
\pgfusepath{clip}%
\pgfsetbuttcap%
\pgfsetroundjoin%
\definecolor{currentfill}{rgb}{0.121569,0.466667,0.705882}%
\pgfsetfillcolor{currentfill}%
\pgfsetlinewidth{1.003750pt}%
\definecolor{currentstroke}{rgb}{0.121569,0.466667,0.705882}%
\pgfsetstrokecolor{currentstroke}%
\pgfsetdash{}{0pt}%
\pgfpathmoveto{\pgfqpoint{2.405914in}{3.181114in}}%
\pgfpathcurveto{\pgfqpoint{2.416964in}{3.181114in}}{\pgfqpoint{2.427563in}{3.185504in}}{\pgfqpoint{2.435377in}{3.193318in}}%
\pgfpathcurveto{\pgfqpoint{2.443190in}{3.201132in}}{\pgfqpoint{2.447580in}{3.211731in}}{\pgfqpoint{2.447580in}{3.222781in}}%
\pgfpathcurveto{\pgfqpoint{2.447580in}{3.233831in}}{\pgfqpoint{2.443190in}{3.244430in}}{\pgfqpoint{2.435377in}{3.252244in}}%
\pgfpathcurveto{\pgfqpoint{2.427563in}{3.260057in}}{\pgfqpoint{2.416964in}{3.264448in}}{\pgfqpoint{2.405914in}{3.264448in}}%
\pgfpathcurveto{\pgfqpoint{2.394864in}{3.264448in}}{\pgfqpoint{2.384265in}{3.260057in}}{\pgfqpoint{2.376451in}{3.252244in}}%
\pgfpathcurveto{\pgfqpoint{2.368637in}{3.244430in}}{\pgfqpoint{2.364247in}{3.233831in}}{\pgfqpoint{2.364247in}{3.222781in}}%
\pgfpathcurveto{\pgfqpoint{2.364247in}{3.211731in}}{\pgfqpoint{2.368637in}{3.201132in}}{\pgfqpoint{2.376451in}{3.193318in}}%
\pgfpathcurveto{\pgfqpoint{2.384265in}{3.185504in}}{\pgfqpoint{2.394864in}{3.181114in}}{\pgfqpoint{2.405914in}{3.181114in}}%
\pgfpathclose%
\pgfusepath{stroke,fill}%
\end{pgfscope}%
\begin{pgfscope}%
\pgfpathrectangle{\pgfqpoint{0.648703in}{0.548769in}}{\pgfqpoint{5.201297in}{3.102590in}}%
\pgfusepath{clip}%
\pgfsetbuttcap%
\pgfsetroundjoin%
\definecolor{currentfill}{rgb}{1.000000,0.498039,0.054902}%
\pgfsetfillcolor{currentfill}%
\pgfsetlinewidth{1.003750pt}%
\definecolor{currentstroke}{rgb}{1.000000,0.498039,0.054902}%
\pgfsetstrokecolor{currentstroke}%
\pgfsetdash{}{0pt}%
\pgfpathmoveto{\pgfqpoint{4.289955in}{3.219172in}}%
\pgfpathcurveto{\pgfqpoint{4.301005in}{3.219172in}}{\pgfqpoint{4.311604in}{3.223563in}}{\pgfqpoint{4.319418in}{3.231376in}}%
\pgfpathcurveto{\pgfqpoint{4.327232in}{3.239190in}}{\pgfqpoint{4.331622in}{3.249789in}}{\pgfqpoint{4.331622in}{3.260839in}}%
\pgfpathcurveto{\pgfqpoint{4.331622in}{3.271889in}}{\pgfqpoint{4.327232in}{3.282488in}}{\pgfqpoint{4.319418in}{3.290302in}}%
\pgfpathcurveto{\pgfqpoint{4.311604in}{3.298116in}}{\pgfqpoint{4.301005in}{3.302506in}}{\pgfqpoint{4.289955in}{3.302506in}}%
\pgfpathcurveto{\pgfqpoint{4.278905in}{3.302506in}}{\pgfqpoint{4.268306in}{3.298116in}}{\pgfqpoint{4.260492in}{3.290302in}}%
\pgfpathcurveto{\pgfqpoint{4.252679in}{3.282488in}}{\pgfqpoint{4.248289in}{3.271889in}}{\pgfqpoint{4.248289in}{3.260839in}}%
\pgfpathcurveto{\pgfqpoint{4.248289in}{3.249789in}}{\pgfqpoint{4.252679in}{3.239190in}}{\pgfqpoint{4.260492in}{3.231376in}}%
\pgfpathcurveto{\pgfqpoint{4.268306in}{3.223563in}}{\pgfqpoint{4.278905in}{3.219172in}}{\pgfqpoint{4.289955in}{3.219172in}}%
\pgfpathclose%
\pgfusepath{stroke,fill}%
\end{pgfscope}%
\begin{pgfscope}%
\pgfpathrectangle{\pgfqpoint{0.648703in}{0.548769in}}{\pgfqpoint{5.201297in}{3.102590in}}%
\pgfusepath{clip}%
\pgfsetbuttcap%
\pgfsetroundjoin%
\definecolor{currentfill}{rgb}{1.000000,0.498039,0.054902}%
\pgfsetfillcolor{currentfill}%
\pgfsetlinewidth{1.003750pt}%
\definecolor{currentstroke}{rgb}{1.000000,0.498039,0.054902}%
\pgfsetstrokecolor{currentstroke}%
\pgfsetdash{}{0pt}%
\pgfpathmoveto{\pgfqpoint{1.565619in}{3.468665in}}%
\pgfpathcurveto{\pgfqpoint{1.576669in}{3.468665in}}{\pgfqpoint{1.587268in}{3.473055in}}{\pgfqpoint{1.595082in}{3.480869in}}%
\pgfpathcurveto{\pgfqpoint{1.602896in}{3.488683in}}{\pgfqpoint{1.607286in}{3.499282in}}{\pgfqpoint{1.607286in}{3.510332in}}%
\pgfpathcurveto{\pgfqpoint{1.607286in}{3.521382in}}{\pgfqpoint{1.602896in}{3.531981in}}{\pgfqpoint{1.595082in}{3.539795in}}%
\pgfpathcurveto{\pgfqpoint{1.587268in}{3.547608in}}{\pgfqpoint{1.576669in}{3.551998in}}{\pgfqpoint{1.565619in}{3.551998in}}%
\pgfpathcurveto{\pgfqpoint{1.554569in}{3.551998in}}{\pgfqpoint{1.543970in}{3.547608in}}{\pgfqpoint{1.536156in}{3.539795in}}%
\pgfpathcurveto{\pgfqpoint{1.528343in}{3.531981in}}{\pgfqpoint{1.523953in}{3.521382in}}{\pgfqpoint{1.523953in}{3.510332in}}%
\pgfpathcurveto{\pgfqpoint{1.523953in}{3.499282in}}{\pgfqpoint{1.528343in}{3.488683in}}{\pgfqpoint{1.536156in}{3.480869in}}%
\pgfpathcurveto{\pgfqpoint{1.543970in}{3.473055in}}{\pgfqpoint{1.554569in}{3.468665in}}{\pgfqpoint{1.565619in}{3.468665in}}%
\pgfpathclose%
\pgfusepath{stroke,fill}%
\end{pgfscope}%
\begin{pgfscope}%
\pgfpathrectangle{\pgfqpoint{0.648703in}{0.548769in}}{\pgfqpoint{5.201297in}{3.102590in}}%
\pgfusepath{clip}%
\pgfsetbuttcap%
\pgfsetroundjoin%
\definecolor{currentfill}{rgb}{1.000000,0.498039,0.054902}%
\pgfsetfillcolor{currentfill}%
\pgfsetlinewidth{1.003750pt}%
\definecolor{currentstroke}{rgb}{1.000000,0.498039,0.054902}%
\pgfsetstrokecolor{currentstroke}%
\pgfsetdash{}{0pt}%
\pgfpathmoveto{\pgfqpoint{1.890662in}{3.189572in}}%
\pgfpathcurveto{\pgfqpoint{1.901712in}{3.189572in}}{\pgfqpoint{1.912311in}{3.193962in}}{\pgfqpoint{1.920124in}{3.201775in}}%
\pgfpathcurveto{\pgfqpoint{1.927938in}{3.209589in}}{\pgfqpoint{1.932328in}{3.220188in}}{\pgfqpoint{1.932328in}{3.231238in}}%
\pgfpathcurveto{\pgfqpoint{1.932328in}{3.242288in}}{\pgfqpoint{1.927938in}{3.252887in}}{\pgfqpoint{1.920124in}{3.260701in}}%
\pgfpathcurveto{\pgfqpoint{1.912311in}{3.268515in}}{\pgfqpoint{1.901712in}{3.272905in}}{\pgfqpoint{1.890662in}{3.272905in}}%
\pgfpathcurveto{\pgfqpoint{1.879611in}{3.272905in}}{\pgfqpoint{1.869012in}{3.268515in}}{\pgfqpoint{1.861199in}{3.260701in}}%
\pgfpathcurveto{\pgfqpoint{1.853385in}{3.252887in}}{\pgfqpoint{1.848995in}{3.242288in}}{\pgfqpoint{1.848995in}{3.231238in}}%
\pgfpathcurveto{\pgfqpoint{1.848995in}{3.220188in}}{\pgfqpoint{1.853385in}{3.209589in}}{\pgfqpoint{1.861199in}{3.201775in}}%
\pgfpathcurveto{\pgfqpoint{1.869012in}{3.193962in}}{\pgfqpoint{1.879611in}{3.189572in}}{\pgfqpoint{1.890662in}{3.189572in}}%
\pgfpathclose%
\pgfusepath{stroke,fill}%
\end{pgfscope}%
\begin{pgfscope}%
\pgfpathrectangle{\pgfqpoint{0.648703in}{0.548769in}}{\pgfqpoint{5.201297in}{3.102590in}}%
\pgfusepath{clip}%
\pgfsetbuttcap%
\pgfsetroundjoin%
\definecolor{currentfill}{rgb}{1.000000,0.498039,0.054902}%
\pgfsetfillcolor{currentfill}%
\pgfsetlinewidth{1.003750pt}%
\definecolor{currentstroke}{rgb}{1.000000,0.498039,0.054902}%
\pgfsetstrokecolor{currentstroke}%
\pgfsetdash{}{0pt}%
\pgfpathmoveto{\pgfqpoint{1.949918in}{3.189572in}}%
\pgfpathcurveto{\pgfqpoint{1.960968in}{3.189572in}}{\pgfqpoint{1.971567in}{3.193962in}}{\pgfqpoint{1.979381in}{3.201775in}}%
\pgfpathcurveto{\pgfqpoint{1.987195in}{3.209589in}}{\pgfqpoint{1.991585in}{3.220188in}}{\pgfqpoint{1.991585in}{3.231238in}}%
\pgfpathcurveto{\pgfqpoint{1.991585in}{3.242288in}}{\pgfqpoint{1.987195in}{3.252887in}}{\pgfqpoint{1.979381in}{3.260701in}}%
\pgfpathcurveto{\pgfqpoint{1.971567in}{3.268515in}}{\pgfqpoint{1.960968in}{3.272905in}}{\pgfqpoint{1.949918in}{3.272905in}}%
\pgfpathcurveto{\pgfqpoint{1.938868in}{3.272905in}}{\pgfqpoint{1.928269in}{3.268515in}}{\pgfqpoint{1.920455in}{3.260701in}}%
\pgfpathcurveto{\pgfqpoint{1.912642in}{3.252887in}}{\pgfqpoint{1.908252in}{3.242288in}}{\pgfqpoint{1.908252in}{3.231238in}}%
\pgfpathcurveto{\pgfqpoint{1.908252in}{3.220188in}}{\pgfqpoint{1.912642in}{3.209589in}}{\pgfqpoint{1.920455in}{3.201775in}}%
\pgfpathcurveto{\pgfqpoint{1.928269in}{3.193962in}}{\pgfqpoint{1.938868in}{3.189572in}}{\pgfqpoint{1.949918in}{3.189572in}}%
\pgfpathclose%
\pgfusepath{stroke,fill}%
\end{pgfscope}%
\begin{pgfscope}%
\pgfpathrectangle{\pgfqpoint{0.648703in}{0.548769in}}{\pgfqpoint{5.201297in}{3.102590in}}%
\pgfusepath{clip}%
\pgfsetbuttcap%
\pgfsetroundjoin%
\definecolor{currentfill}{rgb}{0.121569,0.466667,0.705882}%
\pgfsetfillcolor{currentfill}%
\pgfsetlinewidth{1.003750pt}%
\definecolor{currentstroke}{rgb}{0.121569,0.466667,0.705882}%
\pgfsetstrokecolor{currentstroke}%
\pgfsetdash{}{0pt}%
\pgfpathmoveto{\pgfqpoint{2.320083in}{3.181114in}}%
\pgfpathcurveto{\pgfqpoint{2.331133in}{3.181114in}}{\pgfqpoint{2.341732in}{3.185504in}}{\pgfqpoint{2.349546in}{3.193318in}}%
\pgfpathcurveto{\pgfqpoint{2.357359in}{3.201132in}}{\pgfqpoint{2.361750in}{3.211731in}}{\pgfqpoint{2.361750in}{3.222781in}}%
\pgfpathcurveto{\pgfqpoint{2.361750in}{3.233831in}}{\pgfqpoint{2.357359in}{3.244430in}}{\pgfqpoint{2.349546in}{3.252244in}}%
\pgfpathcurveto{\pgfqpoint{2.341732in}{3.260057in}}{\pgfqpoint{2.331133in}{3.264448in}}{\pgfqpoint{2.320083in}{3.264448in}}%
\pgfpathcurveto{\pgfqpoint{2.309033in}{3.264448in}}{\pgfqpoint{2.298434in}{3.260057in}}{\pgfqpoint{2.290620in}{3.252244in}}%
\pgfpathcurveto{\pgfqpoint{2.282807in}{3.244430in}}{\pgfqpoint{2.278416in}{3.233831in}}{\pgfqpoint{2.278416in}{3.222781in}}%
\pgfpathcurveto{\pgfqpoint{2.278416in}{3.211731in}}{\pgfqpoint{2.282807in}{3.201132in}}{\pgfqpoint{2.290620in}{3.193318in}}%
\pgfpathcurveto{\pgfqpoint{2.298434in}{3.185504in}}{\pgfqpoint{2.309033in}{3.181114in}}{\pgfqpoint{2.320083in}{3.181114in}}%
\pgfpathclose%
\pgfusepath{stroke,fill}%
\end{pgfscope}%
\begin{pgfscope}%
\pgfpathrectangle{\pgfqpoint{0.648703in}{0.548769in}}{\pgfqpoint{5.201297in}{3.102590in}}%
\pgfusepath{clip}%
\pgfsetbuttcap%
\pgfsetroundjoin%
\definecolor{currentfill}{rgb}{1.000000,0.498039,0.054902}%
\pgfsetfillcolor{currentfill}%
\pgfsetlinewidth{1.003750pt}%
\definecolor{currentstroke}{rgb}{1.000000,0.498039,0.054902}%
\pgfsetstrokecolor{currentstroke}%
\pgfsetdash{}{0pt}%
\pgfpathmoveto{\pgfqpoint{1.976180in}{3.185343in}}%
\pgfpathcurveto{\pgfqpoint{1.987230in}{3.185343in}}{\pgfqpoint{1.997829in}{3.189733in}}{\pgfqpoint{2.005643in}{3.197547in}}%
\pgfpathcurveto{\pgfqpoint{2.013457in}{3.205360in}}{\pgfqpoint{2.017847in}{3.215959in}}{\pgfqpoint{2.017847in}{3.227010in}}%
\pgfpathcurveto{\pgfqpoint{2.017847in}{3.238060in}}{\pgfqpoint{2.013457in}{3.248659in}}{\pgfqpoint{2.005643in}{3.256472in}}%
\pgfpathcurveto{\pgfqpoint{1.997829in}{3.264286in}}{\pgfqpoint{1.987230in}{3.268676in}}{\pgfqpoint{1.976180in}{3.268676in}}%
\pgfpathcurveto{\pgfqpoint{1.965130in}{3.268676in}}{\pgfqpoint{1.954531in}{3.264286in}}{\pgfqpoint{1.946717in}{3.256472in}}%
\pgfpathcurveto{\pgfqpoint{1.938904in}{3.248659in}}{\pgfqpoint{1.934514in}{3.238060in}}{\pgfqpoint{1.934514in}{3.227010in}}%
\pgfpathcurveto{\pgfqpoint{1.934514in}{3.215959in}}{\pgfqpoint{1.938904in}{3.205360in}}{\pgfqpoint{1.946717in}{3.197547in}}%
\pgfpathcurveto{\pgfqpoint{1.954531in}{3.189733in}}{\pgfqpoint{1.965130in}{3.185343in}}{\pgfqpoint{1.976180in}{3.185343in}}%
\pgfpathclose%
\pgfusepath{stroke,fill}%
\end{pgfscope}%
\begin{pgfscope}%
\pgfpathrectangle{\pgfqpoint{0.648703in}{0.548769in}}{\pgfqpoint{5.201297in}{3.102590in}}%
\pgfusepath{clip}%
\pgfsetbuttcap%
\pgfsetroundjoin%
\definecolor{currentfill}{rgb}{1.000000,0.498039,0.054902}%
\pgfsetfillcolor{currentfill}%
\pgfsetlinewidth{1.003750pt}%
\definecolor{currentstroke}{rgb}{1.000000,0.498039,0.054902}%
\pgfsetstrokecolor{currentstroke}%
\pgfsetdash{}{0pt}%
\pgfpathmoveto{\pgfqpoint{1.970161in}{3.358719in}}%
\pgfpathcurveto{\pgfqpoint{1.981211in}{3.358719in}}{\pgfqpoint{1.991810in}{3.363109in}}{\pgfqpoint{1.999624in}{3.370923in}}%
\pgfpathcurveto{\pgfqpoint{2.007437in}{3.378737in}}{\pgfqpoint{2.011828in}{3.389336in}}{\pgfqpoint{2.011828in}{3.400386in}}%
\pgfpathcurveto{\pgfqpoint{2.011828in}{3.411436in}}{\pgfqpoint{2.007437in}{3.422035in}}{\pgfqpoint{1.999624in}{3.429849in}}%
\pgfpathcurveto{\pgfqpoint{1.991810in}{3.437662in}}{\pgfqpoint{1.981211in}{3.442053in}}{\pgfqpoint{1.970161in}{3.442053in}}%
\pgfpathcurveto{\pgfqpoint{1.959111in}{3.442053in}}{\pgfqpoint{1.948512in}{3.437662in}}{\pgfqpoint{1.940698in}{3.429849in}}%
\pgfpathcurveto{\pgfqpoint{1.932884in}{3.422035in}}{\pgfqpoint{1.928494in}{3.411436in}}{\pgfqpoint{1.928494in}{3.400386in}}%
\pgfpathcurveto{\pgfqpoint{1.928494in}{3.389336in}}{\pgfqpoint{1.932884in}{3.378737in}}{\pgfqpoint{1.940698in}{3.370923in}}%
\pgfpathcurveto{\pgfqpoint{1.948512in}{3.363109in}}{\pgfqpoint{1.959111in}{3.358719in}}{\pgfqpoint{1.970161in}{3.358719in}}%
\pgfpathclose%
\pgfusepath{stroke,fill}%
\end{pgfscope}%
\begin{pgfscope}%
\pgfpathrectangle{\pgfqpoint{0.648703in}{0.548769in}}{\pgfqpoint{5.201297in}{3.102590in}}%
\pgfusepath{clip}%
\pgfsetbuttcap%
\pgfsetroundjoin%
\definecolor{currentfill}{rgb}{1.000000,0.498039,0.054902}%
\pgfsetfillcolor{currentfill}%
\pgfsetlinewidth{1.003750pt}%
\definecolor{currentstroke}{rgb}{1.000000,0.498039,0.054902}%
\pgfsetstrokecolor{currentstroke}%
\pgfsetdash{}{0pt}%
\pgfpathmoveto{\pgfqpoint{2.648380in}{3.198029in}}%
\pgfpathcurveto{\pgfqpoint{2.659430in}{3.198029in}}{\pgfqpoint{2.670029in}{3.202419in}}{\pgfqpoint{2.677843in}{3.210233in}}%
\pgfpathcurveto{\pgfqpoint{2.685657in}{3.218046in}}{\pgfqpoint{2.690047in}{3.228646in}}{\pgfqpoint{2.690047in}{3.239696in}}%
\pgfpathcurveto{\pgfqpoint{2.690047in}{3.250746in}}{\pgfqpoint{2.685657in}{3.261345in}}{\pgfqpoint{2.677843in}{3.269158in}}%
\pgfpathcurveto{\pgfqpoint{2.670029in}{3.276972in}}{\pgfqpoint{2.659430in}{3.281362in}}{\pgfqpoint{2.648380in}{3.281362in}}%
\pgfpathcurveto{\pgfqpoint{2.637330in}{3.281362in}}{\pgfqpoint{2.626731in}{3.276972in}}{\pgfqpoint{2.618917in}{3.269158in}}%
\pgfpathcurveto{\pgfqpoint{2.611104in}{3.261345in}}{\pgfqpoint{2.606713in}{3.250746in}}{\pgfqpoint{2.606713in}{3.239696in}}%
\pgfpathcurveto{\pgfqpoint{2.606713in}{3.228646in}}{\pgfqpoint{2.611104in}{3.218046in}}{\pgfqpoint{2.618917in}{3.210233in}}%
\pgfpathcurveto{\pgfqpoint{2.626731in}{3.202419in}}{\pgfqpoint{2.637330in}{3.198029in}}{\pgfqpoint{2.648380in}{3.198029in}}%
\pgfpathclose%
\pgfusepath{stroke,fill}%
\end{pgfscope}%
\begin{pgfscope}%
\pgfpathrectangle{\pgfqpoint{0.648703in}{0.548769in}}{\pgfqpoint{5.201297in}{3.102590in}}%
\pgfusepath{clip}%
\pgfsetbuttcap%
\pgfsetroundjoin%
\definecolor{currentfill}{rgb}{0.839216,0.152941,0.156863}%
\pgfsetfillcolor{currentfill}%
\pgfsetlinewidth{1.003750pt}%
\definecolor{currentstroke}{rgb}{0.839216,0.152941,0.156863}%
\pgfsetstrokecolor{currentstroke}%
\pgfsetdash{}{0pt}%
\pgfpathmoveto{\pgfqpoint{1.447730in}{3.193800in}}%
\pgfpathcurveto{\pgfqpoint{1.458780in}{3.193800in}}{\pgfqpoint{1.469379in}{3.198191in}}{\pgfqpoint{1.477193in}{3.206004in}}%
\pgfpathcurveto{\pgfqpoint{1.485006in}{3.213818in}}{\pgfqpoint{1.489397in}{3.224417in}}{\pgfqpoint{1.489397in}{3.235467in}}%
\pgfpathcurveto{\pgfqpoint{1.489397in}{3.246517in}}{\pgfqpoint{1.485006in}{3.257116in}}{\pgfqpoint{1.477193in}{3.264930in}}%
\pgfpathcurveto{\pgfqpoint{1.469379in}{3.272743in}}{\pgfqpoint{1.458780in}{3.277134in}}{\pgfqpoint{1.447730in}{3.277134in}}%
\pgfpathcurveto{\pgfqpoint{1.436680in}{3.277134in}}{\pgfqpoint{1.426081in}{3.272743in}}{\pgfqpoint{1.418267in}{3.264930in}}%
\pgfpathcurveto{\pgfqpoint{1.410454in}{3.257116in}}{\pgfqpoint{1.406063in}{3.246517in}}{\pgfqpoint{1.406063in}{3.235467in}}%
\pgfpathcurveto{\pgfqpoint{1.406063in}{3.224417in}}{\pgfqpoint{1.410454in}{3.213818in}}{\pgfqpoint{1.418267in}{3.206004in}}%
\pgfpathcurveto{\pgfqpoint{1.426081in}{3.198191in}}{\pgfqpoint{1.436680in}{3.193800in}}{\pgfqpoint{1.447730in}{3.193800in}}%
\pgfpathclose%
\pgfusepath{stroke,fill}%
\end{pgfscope}%
\begin{pgfscope}%
\pgfpathrectangle{\pgfqpoint{0.648703in}{0.548769in}}{\pgfqpoint{5.201297in}{3.102590in}}%
\pgfusepath{clip}%
\pgfsetbuttcap%
\pgfsetroundjoin%
\definecolor{currentfill}{rgb}{1.000000,0.498039,0.054902}%
\pgfsetfillcolor{currentfill}%
\pgfsetlinewidth{1.003750pt}%
\definecolor{currentstroke}{rgb}{1.000000,0.498039,0.054902}%
\pgfsetstrokecolor{currentstroke}%
\pgfsetdash{}{0pt}%
\pgfpathmoveto{\pgfqpoint{2.029061in}{3.193800in}}%
\pgfpathcurveto{\pgfqpoint{2.040111in}{3.193800in}}{\pgfqpoint{2.050710in}{3.198191in}}{\pgfqpoint{2.058524in}{3.206004in}}%
\pgfpathcurveto{\pgfqpoint{2.066337in}{3.213818in}}{\pgfqpoint{2.070728in}{3.224417in}}{\pgfqpoint{2.070728in}{3.235467in}}%
\pgfpathcurveto{\pgfqpoint{2.070728in}{3.246517in}}{\pgfqpoint{2.066337in}{3.257116in}}{\pgfqpoint{2.058524in}{3.264930in}}%
\pgfpathcurveto{\pgfqpoint{2.050710in}{3.272743in}}{\pgfqpoint{2.040111in}{3.277134in}}{\pgfqpoint{2.029061in}{3.277134in}}%
\pgfpathcurveto{\pgfqpoint{2.018011in}{3.277134in}}{\pgfqpoint{2.007412in}{3.272743in}}{\pgfqpoint{1.999598in}{3.264930in}}%
\pgfpathcurveto{\pgfqpoint{1.991784in}{3.257116in}}{\pgfqpoint{1.987394in}{3.246517in}}{\pgfqpoint{1.987394in}{3.235467in}}%
\pgfpathcurveto{\pgfqpoint{1.987394in}{3.224417in}}{\pgfqpoint{1.991784in}{3.213818in}}{\pgfqpoint{1.999598in}{3.206004in}}%
\pgfpathcurveto{\pgfqpoint{2.007412in}{3.198191in}}{\pgfqpoint{2.018011in}{3.193800in}}{\pgfqpoint{2.029061in}{3.193800in}}%
\pgfpathclose%
\pgfusepath{stroke,fill}%
\end{pgfscope}%
\begin{pgfscope}%
\pgfpathrectangle{\pgfqpoint{0.648703in}{0.548769in}}{\pgfqpoint{5.201297in}{3.102590in}}%
\pgfusepath{clip}%
\pgfsetbuttcap%
\pgfsetroundjoin%
\definecolor{currentfill}{rgb}{1.000000,0.498039,0.054902}%
\pgfsetfillcolor{currentfill}%
\pgfsetlinewidth{1.003750pt}%
\definecolor{currentstroke}{rgb}{1.000000,0.498039,0.054902}%
\pgfsetstrokecolor{currentstroke}%
\pgfsetdash{}{0pt}%
\pgfpathmoveto{\pgfqpoint{2.396194in}{3.185343in}}%
\pgfpathcurveto{\pgfqpoint{2.407244in}{3.185343in}}{\pgfqpoint{2.417843in}{3.189733in}}{\pgfqpoint{2.425656in}{3.197547in}}%
\pgfpathcurveto{\pgfqpoint{2.433470in}{3.205360in}}{\pgfqpoint{2.437860in}{3.215959in}}{\pgfqpoint{2.437860in}{3.227010in}}%
\pgfpathcurveto{\pgfqpoint{2.437860in}{3.238060in}}{\pgfqpoint{2.433470in}{3.248659in}}{\pgfqpoint{2.425656in}{3.256472in}}%
\pgfpathcurveto{\pgfqpoint{2.417843in}{3.264286in}}{\pgfqpoint{2.407244in}{3.268676in}}{\pgfqpoint{2.396194in}{3.268676in}}%
\pgfpathcurveto{\pgfqpoint{2.385144in}{3.268676in}}{\pgfqpoint{2.374545in}{3.264286in}}{\pgfqpoint{2.366731in}{3.256472in}}%
\pgfpathcurveto{\pgfqpoint{2.358917in}{3.248659in}}{\pgfqpoint{2.354527in}{3.238060in}}{\pgfqpoint{2.354527in}{3.227010in}}%
\pgfpathcurveto{\pgfqpoint{2.354527in}{3.215959in}}{\pgfqpoint{2.358917in}{3.205360in}}{\pgfqpoint{2.366731in}{3.197547in}}%
\pgfpathcurveto{\pgfqpoint{2.374545in}{3.189733in}}{\pgfqpoint{2.385144in}{3.185343in}}{\pgfqpoint{2.396194in}{3.185343in}}%
\pgfpathclose%
\pgfusepath{stroke,fill}%
\end{pgfscope}%
\begin{pgfscope}%
\pgfpathrectangle{\pgfqpoint{0.648703in}{0.548769in}}{\pgfqpoint{5.201297in}{3.102590in}}%
\pgfusepath{clip}%
\pgfsetbuttcap%
\pgfsetroundjoin%
\definecolor{currentfill}{rgb}{1.000000,0.498039,0.054902}%
\pgfsetfillcolor{currentfill}%
\pgfsetlinewidth{1.003750pt}%
\definecolor{currentstroke}{rgb}{1.000000,0.498039,0.054902}%
\pgfsetstrokecolor{currentstroke}%
\pgfsetdash{}{0pt}%
\pgfpathmoveto{\pgfqpoint{1.770900in}{3.185343in}}%
\pgfpathcurveto{\pgfqpoint{1.781950in}{3.185343in}}{\pgfqpoint{1.792549in}{3.189733in}}{\pgfqpoint{1.800362in}{3.197547in}}%
\pgfpathcurveto{\pgfqpoint{1.808176in}{3.205360in}}{\pgfqpoint{1.812566in}{3.215959in}}{\pgfqpoint{1.812566in}{3.227010in}}%
\pgfpathcurveto{\pgfqpoint{1.812566in}{3.238060in}}{\pgfqpoint{1.808176in}{3.248659in}}{\pgfqpoint{1.800362in}{3.256472in}}%
\pgfpathcurveto{\pgfqpoint{1.792549in}{3.264286in}}{\pgfqpoint{1.781950in}{3.268676in}}{\pgfqpoint{1.770900in}{3.268676in}}%
\pgfpathcurveto{\pgfqpoint{1.759850in}{3.268676in}}{\pgfqpoint{1.749251in}{3.264286in}}{\pgfqpoint{1.741437in}{3.256472in}}%
\pgfpathcurveto{\pgfqpoint{1.733623in}{3.248659in}}{\pgfqpoint{1.729233in}{3.238060in}}{\pgfqpoint{1.729233in}{3.227010in}}%
\pgfpathcurveto{\pgfqpoint{1.729233in}{3.215959in}}{\pgfqpoint{1.733623in}{3.205360in}}{\pgfqpoint{1.741437in}{3.197547in}}%
\pgfpathcurveto{\pgfqpoint{1.749251in}{3.189733in}}{\pgfqpoint{1.759850in}{3.185343in}}{\pgfqpoint{1.770900in}{3.185343in}}%
\pgfpathclose%
\pgfusepath{stroke,fill}%
\end{pgfscope}%
\begin{pgfscope}%
\pgfpathrectangle{\pgfqpoint{0.648703in}{0.548769in}}{\pgfqpoint{5.201297in}{3.102590in}}%
\pgfusepath{clip}%
\pgfsetbuttcap%
\pgfsetroundjoin%
\definecolor{currentfill}{rgb}{1.000000,0.498039,0.054902}%
\pgfsetfillcolor{currentfill}%
\pgfsetlinewidth{1.003750pt}%
\definecolor{currentstroke}{rgb}{1.000000,0.498039,0.054902}%
\pgfsetstrokecolor{currentstroke}%
\pgfsetdash{}{0pt}%
\pgfpathmoveto{\pgfqpoint{2.117389in}{3.193800in}}%
\pgfpathcurveto{\pgfqpoint{2.128439in}{3.193800in}}{\pgfqpoint{2.139038in}{3.198191in}}{\pgfqpoint{2.146851in}{3.206004in}}%
\pgfpathcurveto{\pgfqpoint{2.154665in}{3.213818in}}{\pgfqpoint{2.159055in}{3.224417in}}{\pgfqpoint{2.159055in}{3.235467in}}%
\pgfpathcurveto{\pgfqpoint{2.159055in}{3.246517in}}{\pgfqpoint{2.154665in}{3.257116in}}{\pgfqpoint{2.146851in}{3.264930in}}%
\pgfpathcurveto{\pgfqpoint{2.139038in}{3.272743in}}{\pgfqpoint{2.128439in}{3.277134in}}{\pgfqpoint{2.117389in}{3.277134in}}%
\pgfpathcurveto{\pgfqpoint{2.106338in}{3.277134in}}{\pgfqpoint{2.095739in}{3.272743in}}{\pgfqpoint{2.087926in}{3.264930in}}%
\pgfpathcurveto{\pgfqpoint{2.080112in}{3.257116in}}{\pgfqpoint{2.075722in}{3.246517in}}{\pgfqpoint{2.075722in}{3.235467in}}%
\pgfpathcurveto{\pgfqpoint{2.075722in}{3.224417in}}{\pgfqpoint{2.080112in}{3.213818in}}{\pgfqpoint{2.087926in}{3.206004in}}%
\pgfpathcurveto{\pgfqpoint{2.095739in}{3.198191in}}{\pgfqpoint{2.106338in}{3.193800in}}{\pgfqpoint{2.117389in}{3.193800in}}%
\pgfpathclose%
\pgfusepath{stroke,fill}%
\end{pgfscope}%
\begin{pgfscope}%
\pgfpathrectangle{\pgfqpoint{0.648703in}{0.548769in}}{\pgfqpoint{5.201297in}{3.102590in}}%
\pgfusepath{clip}%
\pgfsetbuttcap%
\pgfsetroundjoin%
\definecolor{currentfill}{rgb}{0.839216,0.152941,0.156863}%
\pgfsetfillcolor{currentfill}%
\pgfsetlinewidth{1.003750pt}%
\definecolor{currentstroke}{rgb}{0.839216,0.152941,0.156863}%
\pgfsetstrokecolor{currentstroke}%
\pgfsetdash{}{0pt}%
\pgfpathmoveto{\pgfqpoint{2.049304in}{3.189572in}}%
\pgfpathcurveto{\pgfqpoint{2.060354in}{3.189572in}}{\pgfqpoint{2.070953in}{3.193962in}}{\pgfqpoint{2.078766in}{3.201775in}}%
\pgfpathcurveto{\pgfqpoint{2.086580in}{3.209589in}}{\pgfqpoint{2.090970in}{3.220188in}}{\pgfqpoint{2.090970in}{3.231238in}}%
\pgfpathcurveto{\pgfqpoint{2.090970in}{3.242288in}}{\pgfqpoint{2.086580in}{3.252887in}}{\pgfqpoint{2.078766in}{3.260701in}}%
\pgfpathcurveto{\pgfqpoint{2.070953in}{3.268515in}}{\pgfqpoint{2.060354in}{3.272905in}}{\pgfqpoint{2.049304in}{3.272905in}}%
\pgfpathcurveto{\pgfqpoint{2.038253in}{3.272905in}}{\pgfqpoint{2.027654in}{3.268515in}}{\pgfqpoint{2.019841in}{3.260701in}}%
\pgfpathcurveto{\pgfqpoint{2.012027in}{3.252887in}}{\pgfqpoint{2.007637in}{3.242288in}}{\pgfqpoint{2.007637in}{3.231238in}}%
\pgfpathcurveto{\pgfqpoint{2.007637in}{3.220188in}}{\pgfqpoint{2.012027in}{3.209589in}}{\pgfqpoint{2.019841in}{3.201775in}}%
\pgfpathcurveto{\pgfqpoint{2.027654in}{3.193962in}}{\pgfqpoint{2.038253in}{3.189572in}}{\pgfqpoint{2.049304in}{3.189572in}}%
\pgfpathclose%
\pgfusepath{stroke,fill}%
\end{pgfscope}%
\begin{pgfscope}%
\pgfpathrectangle{\pgfqpoint{0.648703in}{0.548769in}}{\pgfqpoint{5.201297in}{3.102590in}}%
\pgfusepath{clip}%
\pgfsetbuttcap%
\pgfsetroundjoin%
\definecolor{currentfill}{rgb}{1.000000,0.498039,0.054902}%
\pgfsetfillcolor{currentfill}%
\pgfsetlinewidth{1.003750pt}%
\definecolor{currentstroke}{rgb}{1.000000,0.498039,0.054902}%
\pgfsetstrokecolor{currentstroke}%
\pgfsetdash{}{0pt}%
\pgfpathmoveto{\pgfqpoint{4.035316in}{3.198029in}}%
\pgfpathcurveto{\pgfqpoint{4.046367in}{3.198029in}}{\pgfqpoint{4.056966in}{3.202419in}}{\pgfqpoint{4.064779in}{3.210233in}}%
\pgfpathcurveto{\pgfqpoint{4.072593in}{3.218046in}}{\pgfqpoint{4.076983in}{3.228646in}}{\pgfqpoint{4.076983in}{3.239696in}}%
\pgfpathcurveto{\pgfqpoint{4.076983in}{3.250746in}}{\pgfqpoint{4.072593in}{3.261345in}}{\pgfqpoint{4.064779in}{3.269158in}}%
\pgfpathcurveto{\pgfqpoint{4.056966in}{3.276972in}}{\pgfqpoint{4.046367in}{3.281362in}}{\pgfqpoint{4.035316in}{3.281362in}}%
\pgfpathcurveto{\pgfqpoint{4.024266in}{3.281362in}}{\pgfqpoint{4.013667in}{3.276972in}}{\pgfqpoint{4.005854in}{3.269158in}}%
\pgfpathcurveto{\pgfqpoint{3.998040in}{3.261345in}}{\pgfqpoint{3.993650in}{3.250746in}}{\pgfqpoint{3.993650in}{3.239696in}}%
\pgfpathcurveto{\pgfqpoint{3.993650in}{3.228646in}}{\pgfqpoint{3.998040in}{3.218046in}}{\pgfqpoint{4.005854in}{3.210233in}}%
\pgfpathcurveto{\pgfqpoint{4.013667in}{3.202419in}}{\pgfqpoint{4.024266in}{3.198029in}}{\pgfqpoint{4.035316in}{3.198029in}}%
\pgfpathclose%
\pgfusepath{stroke,fill}%
\end{pgfscope}%
\begin{pgfscope}%
\pgfpathrectangle{\pgfqpoint{0.648703in}{0.548769in}}{\pgfqpoint{5.201297in}{3.102590in}}%
\pgfusepath{clip}%
\pgfsetbuttcap%
\pgfsetroundjoin%
\definecolor{currentfill}{rgb}{0.121569,0.466667,0.705882}%
\pgfsetfillcolor{currentfill}%
\pgfsetlinewidth{1.003750pt}%
\definecolor{currentstroke}{rgb}{0.121569,0.466667,0.705882}%
\pgfsetstrokecolor{currentstroke}%
\pgfsetdash{}{0pt}%
\pgfpathmoveto{\pgfqpoint{1.904751in}{3.181114in}}%
\pgfpathcurveto{\pgfqpoint{1.915801in}{3.181114in}}{\pgfqpoint{1.926400in}{3.185504in}}{\pgfqpoint{1.934214in}{3.193318in}}%
\pgfpathcurveto{\pgfqpoint{1.942028in}{3.201132in}}{\pgfqpoint{1.946418in}{3.211731in}}{\pgfqpoint{1.946418in}{3.222781in}}%
\pgfpathcurveto{\pgfqpoint{1.946418in}{3.233831in}}{\pgfqpoint{1.942028in}{3.244430in}}{\pgfqpoint{1.934214in}{3.252244in}}%
\pgfpathcurveto{\pgfqpoint{1.926400in}{3.260057in}}{\pgfqpoint{1.915801in}{3.264448in}}{\pgfqpoint{1.904751in}{3.264448in}}%
\pgfpathcurveto{\pgfqpoint{1.893701in}{3.264448in}}{\pgfqpoint{1.883102in}{3.260057in}}{\pgfqpoint{1.875288in}{3.252244in}}%
\pgfpathcurveto{\pgfqpoint{1.867475in}{3.244430in}}{\pgfqpoint{1.863084in}{3.233831in}}{\pgfqpoint{1.863084in}{3.222781in}}%
\pgfpathcurveto{\pgfqpoint{1.863084in}{3.211731in}}{\pgfqpoint{1.867475in}{3.201132in}}{\pgfqpoint{1.875288in}{3.193318in}}%
\pgfpathcurveto{\pgfqpoint{1.883102in}{3.185504in}}{\pgfqpoint{1.893701in}{3.181114in}}{\pgfqpoint{1.904751in}{3.181114in}}%
\pgfpathclose%
\pgfusepath{stroke,fill}%
\end{pgfscope}%
\begin{pgfscope}%
\pgfpathrectangle{\pgfqpoint{0.648703in}{0.548769in}}{\pgfqpoint{5.201297in}{3.102590in}}%
\pgfusepath{clip}%
\pgfsetbuttcap%
\pgfsetroundjoin%
\definecolor{currentfill}{rgb}{0.121569,0.466667,0.705882}%
\pgfsetfillcolor{currentfill}%
\pgfsetlinewidth{1.003750pt}%
\definecolor{currentstroke}{rgb}{0.121569,0.466667,0.705882}%
\pgfsetstrokecolor{currentstroke}%
\pgfsetdash{}{0pt}%
\pgfpathmoveto{\pgfqpoint{1.022455in}{0.939909in}}%
\pgfpathcurveto{\pgfqpoint{1.033505in}{0.939909in}}{\pgfqpoint{1.044104in}{0.944299in}}{\pgfqpoint{1.051918in}{0.952112in}}%
\pgfpathcurveto{\pgfqpoint{1.059732in}{0.959926in}}{\pgfqpoint{1.064122in}{0.970525in}}{\pgfqpoint{1.064122in}{0.981575in}}%
\pgfpathcurveto{\pgfqpoint{1.064122in}{0.992625in}}{\pgfqpoint{1.059732in}{1.003224in}}{\pgfqpoint{1.051918in}{1.011038in}}%
\pgfpathcurveto{\pgfqpoint{1.044104in}{1.018852in}}{\pgfqpoint{1.033505in}{1.023242in}}{\pgfqpoint{1.022455in}{1.023242in}}%
\pgfpathcurveto{\pgfqpoint{1.011405in}{1.023242in}}{\pgfqpoint{1.000806in}{1.018852in}}{\pgfqpoint{0.992992in}{1.011038in}}%
\pgfpathcurveto{\pgfqpoint{0.985179in}{1.003224in}}{\pgfqpoint{0.980789in}{0.992625in}}{\pgfqpoint{0.980789in}{0.981575in}}%
\pgfpathcurveto{\pgfqpoint{0.980789in}{0.970525in}}{\pgfqpoint{0.985179in}{0.959926in}}{\pgfqpoint{0.992992in}{0.952112in}}%
\pgfpathcurveto{\pgfqpoint{1.000806in}{0.944299in}}{\pgfqpoint{1.011405in}{0.939909in}}{\pgfqpoint{1.022455in}{0.939909in}}%
\pgfpathclose%
\pgfusepath{stroke,fill}%
\end{pgfscope}%
\begin{pgfscope}%
\pgfpathrectangle{\pgfqpoint{0.648703in}{0.548769in}}{\pgfqpoint{5.201297in}{3.102590in}}%
\pgfusepath{clip}%
\pgfsetbuttcap%
\pgfsetroundjoin%
\definecolor{currentfill}{rgb}{0.839216,0.152941,0.156863}%
\pgfsetfillcolor{currentfill}%
\pgfsetlinewidth{1.003750pt}%
\definecolor{currentstroke}{rgb}{0.839216,0.152941,0.156863}%
\pgfsetstrokecolor{currentstroke}%
\pgfsetdash{}{0pt}%
\pgfpathmoveto{\pgfqpoint{2.235545in}{3.210715in}}%
\pgfpathcurveto{\pgfqpoint{2.246595in}{3.210715in}}{\pgfqpoint{2.257194in}{3.215105in}}{\pgfqpoint{2.265008in}{3.222919in}}%
\pgfpathcurveto{\pgfqpoint{2.272822in}{3.230733in}}{\pgfqpoint{2.277212in}{3.241332in}}{\pgfqpoint{2.277212in}{3.252382in}}%
\pgfpathcurveto{\pgfqpoint{2.277212in}{3.263432in}}{\pgfqpoint{2.272822in}{3.274031in}}{\pgfqpoint{2.265008in}{3.281844in}}%
\pgfpathcurveto{\pgfqpoint{2.257194in}{3.289658in}}{\pgfqpoint{2.246595in}{3.294048in}}{\pgfqpoint{2.235545in}{3.294048in}}%
\pgfpathcurveto{\pgfqpoint{2.224495in}{3.294048in}}{\pgfqpoint{2.213896in}{3.289658in}}{\pgfqpoint{2.206082in}{3.281844in}}%
\pgfpathcurveto{\pgfqpoint{2.198269in}{3.274031in}}{\pgfqpoint{2.193879in}{3.263432in}}{\pgfqpoint{2.193879in}{3.252382in}}%
\pgfpathcurveto{\pgfqpoint{2.193879in}{3.241332in}}{\pgfqpoint{2.198269in}{3.230733in}}{\pgfqpoint{2.206082in}{3.222919in}}%
\pgfpathcurveto{\pgfqpoint{2.213896in}{3.215105in}}{\pgfqpoint{2.224495in}{3.210715in}}{\pgfqpoint{2.235545in}{3.210715in}}%
\pgfpathclose%
\pgfusepath{stroke,fill}%
\end{pgfscope}%
\begin{pgfscope}%
\pgfpathrectangle{\pgfqpoint{0.648703in}{0.548769in}}{\pgfqpoint{5.201297in}{3.102590in}}%
\pgfusepath{clip}%
\pgfsetbuttcap%
\pgfsetroundjoin%
\definecolor{currentfill}{rgb}{1.000000,0.498039,0.054902}%
\pgfsetfillcolor{currentfill}%
\pgfsetlinewidth{1.003750pt}%
\definecolor{currentstroke}{rgb}{1.000000,0.498039,0.054902}%
\pgfsetstrokecolor{currentstroke}%
\pgfsetdash{}{0pt}%
\pgfpathmoveto{\pgfqpoint{1.860119in}{3.202258in}}%
\pgfpathcurveto{\pgfqpoint{1.871169in}{3.202258in}}{\pgfqpoint{1.881768in}{3.206648in}}{\pgfqpoint{1.889582in}{3.214462in}}%
\pgfpathcurveto{\pgfqpoint{1.897396in}{3.222275in}}{\pgfqpoint{1.901786in}{3.232874in}}{\pgfqpoint{1.901786in}{3.243924in}}%
\pgfpathcurveto{\pgfqpoint{1.901786in}{3.254974in}}{\pgfqpoint{1.897396in}{3.265573in}}{\pgfqpoint{1.889582in}{3.273387in}}%
\pgfpathcurveto{\pgfqpoint{1.881768in}{3.281201in}}{\pgfqpoint{1.871169in}{3.285591in}}{\pgfqpoint{1.860119in}{3.285591in}}%
\pgfpathcurveto{\pgfqpoint{1.849069in}{3.285591in}}{\pgfqpoint{1.838470in}{3.281201in}}{\pgfqpoint{1.830656in}{3.273387in}}%
\pgfpathcurveto{\pgfqpoint{1.822843in}{3.265573in}}{\pgfqpoint{1.818452in}{3.254974in}}{\pgfqpoint{1.818452in}{3.243924in}}%
\pgfpathcurveto{\pgfqpoint{1.818452in}{3.232874in}}{\pgfqpoint{1.822843in}{3.222275in}}{\pgfqpoint{1.830656in}{3.214462in}}%
\pgfpathcurveto{\pgfqpoint{1.838470in}{3.206648in}}{\pgfqpoint{1.849069in}{3.202258in}}{\pgfqpoint{1.860119in}{3.202258in}}%
\pgfpathclose%
\pgfusepath{stroke,fill}%
\end{pgfscope}%
\begin{pgfscope}%
\pgfpathrectangle{\pgfqpoint{0.648703in}{0.548769in}}{\pgfqpoint{5.201297in}{3.102590in}}%
\pgfusepath{clip}%
\pgfsetbuttcap%
\pgfsetroundjoin%
\definecolor{currentfill}{rgb}{1.000000,0.498039,0.054902}%
\pgfsetfillcolor{currentfill}%
\pgfsetlinewidth{1.003750pt}%
\definecolor{currentstroke}{rgb}{1.000000,0.498039,0.054902}%
\pgfsetstrokecolor{currentstroke}%
\pgfsetdash{}{0pt}%
\pgfpathmoveto{\pgfqpoint{1.374651in}{3.206486in}}%
\pgfpathcurveto{\pgfqpoint{1.385701in}{3.206486in}}{\pgfqpoint{1.396300in}{3.210877in}}{\pgfqpoint{1.404114in}{3.218690in}}%
\pgfpathcurveto{\pgfqpoint{1.411928in}{3.226504in}}{\pgfqpoint{1.416318in}{3.237103in}}{\pgfqpoint{1.416318in}{3.248153in}}%
\pgfpathcurveto{\pgfqpoint{1.416318in}{3.259203in}}{\pgfqpoint{1.411928in}{3.269802in}}{\pgfqpoint{1.404114in}{3.277616in}}%
\pgfpathcurveto{\pgfqpoint{1.396300in}{3.285429in}}{\pgfqpoint{1.385701in}{3.289820in}}{\pgfqpoint{1.374651in}{3.289820in}}%
\pgfpathcurveto{\pgfqpoint{1.363601in}{3.289820in}}{\pgfqpoint{1.353002in}{3.285429in}}{\pgfqpoint{1.345189in}{3.277616in}}%
\pgfpathcurveto{\pgfqpoint{1.337375in}{3.269802in}}{\pgfqpoint{1.332985in}{3.259203in}}{\pgfqpoint{1.332985in}{3.248153in}}%
\pgfpathcurveto{\pgfqpoint{1.332985in}{3.237103in}}{\pgfqpoint{1.337375in}{3.226504in}}{\pgfqpoint{1.345189in}{3.218690in}}%
\pgfpathcurveto{\pgfqpoint{1.353002in}{3.210877in}}{\pgfqpoint{1.363601in}{3.206486in}}{\pgfqpoint{1.374651in}{3.206486in}}%
\pgfpathclose%
\pgfusepath{stroke,fill}%
\end{pgfscope}%
\begin{pgfscope}%
\pgfpathrectangle{\pgfqpoint{0.648703in}{0.548769in}}{\pgfqpoint{5.201297in}{3.102590in}}%
\pgfusepath{clip}%
\pgfsetbuttcap%
\pgfsetroundjoin%
\definecolor{currentfill}{rgb}{1.000000,0.498039,0.054902}%
\pgfsetfillcolor{currentfill}%
\pgfsetlinewidth{1.003750pt}%
\definecolor{currentstroke}{rgb}{1.000000,0.498039,0.054902}%
\pgfsetstrokecolor{currentstroke}%
\pgfsetdash{}{0pt}%
\pgfpathmoveto{\pgfqpoint{1.158224in}{3.198029in}}%
\pgfpathcurveto{\pgfqpoint{1.169274in}{3.198029in}}{\pgfqpoint{1.179873in}{3.202419in}}{\pgfqpoint{1.187687in}{3.210233in}}%
\pgfpathcurveto{\pgfqpoint{1.195500in}{3.218046in}}{\pgfqpoint{1.199891in}{3.228646in}}{\pgfqpoint{1.199891in}{3.239696in}}%
\pgfpathcurveto{\pgfqpoint{1.199891in}{3.250746in}}{\pgfqpoint{1.195500in}{3.261345in}}{\pgfqpoint{1.187687in}{3.269158in}}%
\pgfpathcurveto{\pgfqpoint{1.179873in}{3.276972in}}{\pgfqpoint{1.169274in}{3.281362in}}{\pgfqpoint{1.158224in}{3.281362in}}%
\pgfpathcurveto{\pgfqpoint{1.147174in}{3.281362in}}{\pgfqpoint{1.136575in}{3.276972in}}{\pgfqpoint{1.128761in}{3.269158in}}%
\pgfpathcurveto{\pgfqpoint{1.120948in}{3.261345in}}{\pgfqpoint{1.116557in}{3.250746in}}{\pgfqpoint{1.116557in}{3.239696in}}%
\pgfpathcurveto{\pgfqpoint{1.116557in}{3.228646in}}{\pgfqpoint{1.120948in}{3.218046in}}{\pgfqpoint{1.128761in}{3.210233in}}%
\pgfpathcurveto{\pgfqpoint{1.136575in}{3.202419in}}{\pgfqpoint{1.147174in}{3.198029in}}{\pgfqpoint{1.158224in}{3.198029in}}%
\pgfpathclose%
\pgfusepath{stroke,fill}%
\end{pgfscope}%
\begin{pgfscope}%
\pgfpathrectangle{\pgfqpoint{0.648703in}{0.548769in}}{\pgfqpoint{5.201297in}{3.102590in}}%
\pgfusepath{clip}%
\pgfsetbuttcap%
\pgfsetroundjoin%
\definecolor{currentfill}{rgb}{1.000000,0.498039,0.054902}%
\pgfsetfillcolor{currentfill}%
\pgfsetlinewidth{1.003750pt}%
\definecolor{currentstroke}{rgb}{1.000000,0.498039,0.054902}%
\pgfsetstrokecolor{currentstroke}%
\pgfsetdash{}{0pt}%
\pgfpathmoveto{\pgfqpoint{2.092999in}{3.202258in}}%
\pgfpathcurveto{\pgfqpoint{2.104049in}{3.202258in}}{\pgfqpoint{2.114648in}{3.206648in}}{\pgfqpoint{2.122462in}{3.214462in}}%
\pgfpathcurveto{\pgfqpoint{2.130276in}{3.222275in}}{\pgfqpoint{2.134666in}{3.232874in}}{\pgfqpoint{2.134666in}{3.243924in}}%
\pgfpathcurveto{\pgfqpoint{2.134666in}{3.254974in}}{\pgfqpoint{2.130276in}{3.265573in}}{\pgfqpoint{2.122462in}{3.273387in}}%
\pgfpathcurveto{\pgfqpoint{2.114648in}{3.281201in}}{\pgfqpoint{2.104049in}{3.285591in}}{\pgfqpoint{2.092999in}{3.285591in}}%
\pgfpathcurveto{\pgfqpoint{2.081949in}{3.285591in}}{\pgfqpoint{2.071350in}{3.281201in}}{\pgfqpoint{2.063536in}{3.273387in}}%
\pgfpathcurveto{\pgfqpoint{2.055723in}{3.265573in}}{\pgfqpoint{2.051333in}{3.254974in}}{\pgfqpoint{2.051333in}{3.243924in}}%
\pgfpathcurveto{\pgfqpoint{2.051333in}{3.232874in}}{\pgfqpoint{2.055723in}{3.222275in}}{\pgfqpoint{2.063536in}{3.214462in}}%
\pgfpathcurveto{\pgfqpoint{2.071350in}{3.206648in}}{\pgfqpoint{2.081949in}{3.202258in}}{\pgfqpoint{2.092999in}{3.202258in}}%
\pgfpathclose%
\pgfusepath{stroke,fill}%
\end{pgfscope}%
\begin{pgfscope}%
\pgfpathrectangle{\pgfqpoint{0.648703in}{0.548769in}}{\pgfqpoint{5.201297in}{3.102590in}}%
\pgfusepath{clip}%
\pgfsetbuttcap%
\pgfsetroundjoin%
\definecolor{currentfill}{rgb}{1.000000,0.498039,0.054902}%
\pgfsetfillcolor{currentfill}%
\pgfsetlinewidth{1.003750pt}%
\definecolor{currentstroke}{rgb}{1.000000,0.498039,0.054902}%
\pgfsetstrokecolor{currentstroke}%
\pgfsetdash{}{0pt}%
\pgfpathmoveto{\pgfqpoint{1.972702in}{3.189572in}}%
\pgfpathcurveto{\pgfqpoint{1.983752in}{3.189572in}}{\pgfqpoint{1.994352in}{3.193962in}}{\pgfqpoint{2.002165in}{3.201775in}}%
\pgfpathcurveto{\pgfqpoint{2.009979in}{3.209589in}}{\pgfqpoint{2.014369in}{3.220188in}}{\pgfqpoint{2.014369in}{3.231238in}}%
\pgfpathcurveto{\pgfqpoint{2.014369in}{3.242288in}}{\pgfqpoint{2.009979in}{3.252887in}}{\pgfqpoint{2.002165in}{3.260701in}}%
\pgfpathcurveto{\pgfqpoint{1.994352in}{3.268515in}}{\pgfqpoint{1.983752in}{3.272905in}}{\pgfqpoint{1.972702in}{3.272905in}}%
\pgfpathcurveto{\pgfqpoint{1.961652in}{3.272905in}}{\pgfqpoint{1.951053in}{3.268515in}}{\pgfqpoint{1.943240in}{3.260701in}}%
\pgfpathcurveto{\pgfqpoint{1.935426in}{3.252887in}}{\pgfqpoint{1.931036in}{3.242288in}}{\pgfqpoint{1.931036in}{3.231238in}}%
\pgfpathcurveto{\pgfqpoint{1.931036in}{3.220188in}}{\pgfqpoint{1.935426in}{3.209589in}}{\pgfqpoint{1.943240in}{3.201775in}}%
\pgfpathcurveto{\pgfqpoint{1.951053in}{3.193962in}}{\pgfqpoint{1.961652in}{3.189572in}}{\pgfqpoint{1.972702in}{3.189572in}}%
\pgfpathclose%
\pgfusepath{stroke,fill}%
\end{pgfscope}%
\begin{pgfscope}%
\pgfpathrectangle{\pgfqpoint{0.648703in}{0.548769in}}{\pgfqpoint{5.201297in}{3.102590in}}%
\pgfusepath{clip}%
\pgfsetbuttcap%
\pgfsetroundjoin%
\definecolor{currentfill}{rgb}{0.121569,0.466667,0.705882}%
\pgfsetfillcolor{currentfill}%
\pgfsetlinewidth{1.003750pt}%
\definecolor{currentstroke}{rgb}{0.121569,0.466667,0.705882}%
\pgfsetstrokecolor{currentstroke}%
\pgfsetdash{}{0pt}%
\pgfpathmoveto{\pgfqpoint{2.803455in}{3.181114in}}%
\pgfpathcurveto{\pgfqpoint{2.814505in}{3.181114in}}{\pgfqpoint{2.825104in}{3.185504in}}{\pgfqpoint{2.832918in}{3.193318in}}%
\pgfpathcurveto{\pgfqpoint{2.840732in}{3.201132in}}{\pgfqpoint{2.845122in}{3.211731in}}{\pgfqpoint{2.845122in}{3.222781in}}%
\pgfpathcurveto{\pgfqpoint{2.845122in}{3.233831in}}{\pgfqpoint{2.840732in}{3.244430in}}{\pgfqpoint{2.832918in}{3.252244in}}%
\pgfpathcurveto{\pgfqpoint{2.825104in}{3.260057in}}{\pgfqpoint{2.814505in}{3.264448in}}{\pgfqpoint{2.803455in}{3.264448in}}%
\pgfpathcurveto{\pgfqpoint{2.792405in}{3.264448in}}{\pgfqpoint{2.781806in}{3.260057in}}{\pgfqpoint{2.773992in}{3.252244in}}%
\pgfpathcurveto{\pgfqpoint{2.766179in}{3.244430in}}{\pgfqpoint{2.761789in}{3.233831in}}{\pgfqpoint{2.761789in}{3.222781in}}%
\pgfpathcurveto{\pgfqpoint{2.761789in}{3.211731in}}{\pgfqpoint{2.766179in}{3.201132in}}{\pgfqpoint{2.773992in}{3.193318in}}%
\pgfpathcurveto{\pgfqpoint{2.781806in}{3.185504in}}{\pgfqpoint{2.792405in}{3.181114in}}{\pgfqpoint{2.803455in}{3.181114in}}%
\pgfpathclose%
\pgfusepath{stroke,fill}%
\end{pgfscope}%
\begin{pgfscope}%
\pgfpathrectangle{\pgfqpoint{0.648703in}{0.548769in}}{\pgfqpoint{5.201297in}{3.102590in}}%
\pgfusepath{clip}%
\pgfsetbuttcap%
\pgfsetroundjoin%
\definecolor{currentfill}{rgb}{1.000000,0.498039,0.054902}%
\pgfsetfillcolor{currentfill}%
\pgfsetlinewidth{1.003750pt}%
\definecolor{currentstroke}{rgb}{1.000000,0.498039,0.054902}%
\pgfsetstrokecolor{currentstroke}%
\pgfsetdash{}{0pt}%
\pgfpathmoveto{\pgfqpoint{1.990716in}{3.244545in}}%
\pgfpathcurveto{\pgfqpoint{2.001766in}{3.244545in}}{\pgfqpoint{2.012365in}{3.248935in}}{\pgfqpoint{2.020178in}{3.256748in}}%
\pgfpathcurveto{\pgfqpoint{2.027992in}{3.264562in}}{\pgfqpoint{2.032382in}{3.275161in}}{\pgfqpoint{2.032382in}{3.286211in}}%
\pgfpathcurveto{\pgfqpoint{2.032382in}{3.297261in}}{\pgfqpoint{2.027992in}{3.307860in}}{\pgfqpoint{2.020178in}{3.315674in}}%
\pgfpathcurveto{\pgfqpoint{2.012365in}{3.323488in}}{\pgfqpoint{2.001766in}{3.327878in}}{\pgfqpoint{1.990716in}{3.327878in}}%
\pgfpathcurveto{\pgfqpoint{1.979666in}{3.327878in}}{\pgfqpoint{1.969067in}{3.323488in}}{\pgfqpoint{1.961253in}{3.315674in}}%
\pgfpathcurveto{\pgfqpoint{1.953439in}{3.307860in}}{\pgfqpoint{1.949049in}{3.297261in}}{\pgfqpoint{1.949049in}{3.286211in}}%
\pgfpathcurveto{\pgfqpoint{1.949049in}{3.275161in}}{\pgfqpoint{1.953439in}{3.264562in}}{\pgfqpoint{1.961253in}{3.256748in}}%
\pgfpathcurveto{\pgfqpoint{1.969067in}{3.248935in}}{\pgfqpoint{1.979666in}{3.244545in}}{\pgfqpoint{1.990716in}{3.244545in}}%
\pgfpathclose%
\pgfusepath{stroke,fill}%
\end{pgfscope}%
\begin{pgfscope}%
\pgfpathrectangle{\pgfqpoint{0.648703in}{0.548769in}}{\pgfqpoint{5.201297in}{3.102590in}}%
\pgfusepath{clip}%
\pgfsetbuttcap%
\pgfsetroundjoin%
\definecolor{currentfill}{rgb}{1.000000,0.498039,0.054902}%
\pgfsetfillcolor{currentfill}%
\pgfsetlinewidth{1.003750pt}%
\definecolor{currentstroke}{rgb}{1.000000,0.498039,0.054902}%
\pgfsetstrokecolor{currentstroke}%
\pgfsetdash{}{0pt}%
\pgfpathmoveto{\pgfqpoint{1.997983in}{3.189572in}}%
\pgfpathcurveto{\pgfqpoint{2.009034in}{3.189572in}}{\pgfqpoint{2.019633in}{3.193962in}}{\pgfqpoint{2.027446in}{3.201775in}}%
\pgfpathcurveto{\pgfqpoint{2.035260in}{3.209589in}}{\pgfqpoint{2.039650in}{3.220188in}}{\pgfqpoint{2.039650in}{3.231238in}}%
\pgfpathcurveto{\pgfqpoint{2.039650in}{3.242288in}}{\pgfqpoint{2.035260in}{3.252887in}}{\pgfqpoint{2.027446in}{3.260701in}}%
\pgfpathcurveto{\pgfqpoint{2.019633in}{3.268515in}}{\pgfqpoint{2.009034in}{3.272905in}}{\pgfqpoint{1.997983in}{3.272905in}}%
\pgfpathcurveto{\pgfqpoint{1.986933in}{3.272905in}}{\pgfqpoint{1.976334in}{3.268515in}}{\pgfqpoint{1.968521in}{3.260701in}}%
\pgfpathcurveto{\pgfqpoint{1.960707in}{3.252887in}}{\pgfqpoint{1.956317in}{3.242288in}}{\pgfqpoint{1.956317in}{3.231238in}}%
\pgfpathcurveto{\pgfqpoint{1.956317in}{3.220188in}}{\pgfqpoint{1.960707in}{3.209589in}}{\pgfqpoint{1.968521in}{3.201775in}}%
\pgfpathcurveto{\pgfqpoint{1.976334in}{3.193962in}}{\pgfqpoint{1.986933in}{3.189572in}}{\pgfqpoint{1.997983in}{3.189572in}}%
\pgfpathclose%
\pgfusepath{stroke,fill}%
\end{pgfscope}%
\begin{pgfscope}%
\pgfpathrectangle{\pgfqpoint{0.648703in}{0.548769in}}{\pgfqpoint{5.201297in}{3.102590in}}%
\pgfusepath{clip}%
\pgfsetbuttcap%
\pgfsetroundjoin%
\definecolor{currentfill}{rgb}{1.000000,0.498039,0.054902}%
\pgfsetfillcolor{currentfill}%
\pgfsetlinewidth{1.003750pt}%
\definecolor{currentstroke}{rgb}{1.000000,0.498039,0.054902}%
\pgfsetstrokecolor{currentstroke}%
\pgfsetdash{}{0pt}%
\pgfpathmoveto{\pgfqpoint{1.500745in}{3.210715in}}%
\pgfpathcurveto{\pgfqpoint{1.511795in}{3.210715in}}{\pgfqpoint{1.522394in}{3.215105in}}{\pgfqpoint{1.530207in}{3.222919in}}%
\pgfpathcurveto{\pgfqpoint{1.538021in}{3.230733in}}{\pgfqpoint{1.542411in}{3.241332in}}{\pgfqpoint{1.542411in}{3.252382in}}%
\pgfpathcurveto{\pgfqpoint{1.542411in}{3.263432in}}{\pgfqpoint{1.538021in}{3.274031in}}{\pgfqpoint{1.530207in}{3.281844in}}%
\pgfpathcurveto{\pgfqpoint{1.522394in}{3.289658in}}{\pgfqpoint{1.511795in}{3.294048in}}{\pgfqpoint{1.500745in}{3.294048in}}%
\pgfpathcurveto{\pgfqpoint{1.489694in}{3.294048in}}{\pgfqpoint{1.479095in}{3.289658in}}{\pgfqpoint{1.471282in}{3.281844in}}%
\pgfpathcurveto{\pgfqpoint{1.463468in}{3.274031in}}{\pgfqpoint{1.459078in}{3.263432in}}{\pgfqpoint{1.459078in}{3.252382in}}%
\pgfpathcurveto{\pgfqpoint{1.459078in}{3.241332in}}{\pgfqpoint{1.463468in}{3.230733in}}{\pgfqpoint{1.471282in}{3.222919in}}%
\pgfpathcurveto{\pgfqpoint{1.479095in}{3.215105in}}{\pgfqpoint{1.489694in}{3.210715in}}{\pgfqpoint{1.500745in}{3.210715in}}%
\pgfpathclose%
\pgfusepath{stroke,fill}%
\end{pgfscope}%
\begin{pgfscope}%
\pgfpathrectangle{\pgfqpoint{0.648703in}{0.548769in}}{\pgfqpoint{5.201297in}{3.102590in}}%
\pgfusepath{clip}%
\pgfsetbuttcap%
\pgfsetroundjoin%
\definecolor{currentfill}{rgb}{1.000000,0.498039,0.054902}%
\pgfsetfillcolor{currentfill}%
\pgfsetlinewidth{1.003750pt}%
\definecolor{currentstroke}{rgb}{1.000000,0.498039,0.054902}%
\pgfsetstrokecolor{currentstroke}%
\pgfsetdash{}{0pt}%
\pgfpathmoveto{\pgfqpoint{1.876438in}{3.189572in}}%
\pgfpathcurveto{\pgfqpoint{1.887488in}{3.189572in}}{\pgfqpoint{1.898087in}{3.193962in}}{\pgfqpoint{1.905901in}{3.201775in}}%
\pgfpathcurveto{\pgfqpoint{1.913715in}{3.209589in}}{\pgfqpoint{1.918105in}{3.220188in}}{\pgfqpoint{1.918105in}{3.231238in}}%
\pgfpathcurveto{\pgfqpoint{1.918105in}{3.242288in}}{\pgfqpoint{1.913715in}{3.252887in}}{\pgfqpoint{1.905901in}{3.260701in}}%
\pgfpathcurveto{\pgfqpoint{1.898087in}{3.268515in}}{\pgfqpoint{1.887488in}{3.272905in}}{\pgfqpoint{1.876438in}{3.272905in}}%
\pgfpathcurveto{\pgfqpoint{1.865388in}{3.272905in}}{\pgfqpoint{1.854789in}{3.268515in}}{\pgfqpoint{1.846975in}{3.260701in}}%
\pgfpathcurveto{\pgfqpoint{1.839162in}{3.252887in}}{\pgfqpoint{1.834771in}{3.242288in}}{\pgfqpoint{1.834771in}{3.231238in}}%
\pgfpathcurveto{\pgfqpoint{1.834771in}{3.220188in}}{\pgfqpoint{1.839162in}{3.209589in}}{\pgfqpoint{1.846975in}{3.201775in}}%
\pgfpathcurveto{\pgfqpoint{1.854789in}{3.193962in}}{\pgfqpoint{1.865388in}{3.189572in}}{\pgfqpoint{1.876438in}{3.189572in}}%
\pgfpathclose%
\pgfusepath{stroke,fill}%
\end{pgfscope}%
\begin{pgfscope}%
\pgfpathrectangle{\pgfqpoint{0.648703in}{0.548769in}}{\pgfqpoint{5.201297in}{3.102590in}}%
\pgfusepath{clip}%
\pgfsetbuttcap%
\pgfsetroundjoin%
\definecolor{currentfill}{rgb}{0.839216,0.152941,0.156863}%
\pgfsetfillcolor{currentfill}%
\pgfsetlinewidth{1.003750pt}%
\definecolor{currentstroke}{rgb}{0.839216,0.152941,0.156863}%
\pgfsetstrokecolor{currentstroke}%
\pgfsetdash{}{0pt}%
\pgfpathmoveto{\pgfqpoint{2.381881in}{3.181114in}}%
\pgfpathcurveto{\pgfqpoint{2.392931in}{3.181114in}}{\pgfqpoint{2.403530in}{3.185504in}}{\pgfqpoint{2.411344in}{3.193318in}}%
\pgfpathcurveto{\pgfqpoint{2.419158in}{3.201132in}}{\pgfqpoint{2.423548in}{3.211731in}}{\pgfqpoint{2.423548in}{3.222781in}}%
\pgfpathcurveto{\pgfqpoint{2.423548in}{3.233831in}}{\pgfqpoint{2.419158in}{3.244430in}}{\pgfqpoint{2.411344in}{3.252244in}}%
\pgfpathcurveto{\pgfqpoint{2.403530in}{3.260057in}}{\pgfqpoint{2.392931in}{3.264448in}}{\pgfqpoint{2.381881in}{3.264448in}}%
\pgfpathcurveto{\pgfqpoint{2.370831in}{3.264448in}}{\pgfqpoint{2.360232in}{3.260057in}}{\pgfqpoint{2.352418in}{3.252244in}}%
\pgfpathcurveto{\pgfqpoint{2.344605in}{3.244430in}}{\pgfqpoint{2.340214in}{3.233831in}}{\pgfqpoint{2.340214in}{3.222781in}}%
\pgfpathcurveto{\pgfqpoint{2.340214in}{3.211731in}}{\pgfqpoint{2.344605in}{3.201132in}}{\pgfqpoint{2.352418in}{3.193318in}}%
\pgfpathcurveto{\pgfqpoint{2.360232in}{3.185504in}}{\pgfqpoint{2.370831in}{3.181114in}}{\pgfqpoint{2.381881in}{3.181114in}}%
\pgfpathclose%
\pgfusepath{stroke,fill}%
\end{pgfscope}%
\begin{pgfscope}%
\pgfpathrectangle{\pgfqpoint{0.648703in}{0.548769in}}{\pgfqpoint{5.201297in}{3.102590in}}%
\pgfusepath{clip}%
\pgfsetbuttcap%
\pgfsetroundjoin%
\definecolor{currentfill}{rgb}{1.000000,0.498039,0.054902}%
\pgfsetfillcolor{currentfill}%
\pgfsetlinewidth{1.003750pt}%
\definecolor{currentstroke}{rgb}{1.000000,0.498039,0.054902}%
\pgfsetstrokecolor{currentstroke}%
\pgfsetdash{}{0pt}%
\pgfpathmoveto{\pgfqpoint{1.686273in}{3.202258in}}%
\pgfpathcurveto{\pgfqpoint{1.697323in}{3.202258in}}{\pgfqpoint{1.707922in}{3.206648in}}{\pgfqpoint{1.715736in}{3.214462in}}%
\pgfpathcurveto{\pgfqpoint{1.723549in}{3.222275in}}{\pgfqpoint{1.727939in}{3.232874in}}{\pgfqpoint{1.727939in}{3.243924in}}%
\pgfpathcurveto{\pgfqpoint{1.727939in}{3.254974in}}{\pgfqpoint{1.723549in}{3.265573in}}{\pgfqpoint{1.715736in}{3.273387in}}%
\pgfpathcurveto{\pgfqpoint{1.707922in}{3.281201in}}{\pgfqpoint{1.697323in}{3.285591in}}{\pgfqpoint{1.686273in}{3.285591in}}%
\pgfpathcurveto{\pgfqpoint{1.675223in}{3.285591in}}{\pgfqpoint{1.664624in}{3.281201in}}{\pgfqpoint{1.656810in}{3.273387in}}%
\pgfpathcurveto{\pgfqpoint{1.648996in}{3.265573in}}{\pgfqpoint{1.644606in}{3.254974in}}{\pgfqpoint{1.644606in}{3.243924in}}%
\pgfpathcurveto{\pgfqpoint{1.644606in}{3.232874in}}{\pgfqpoint{1.648996in}{3.222275in}}{\pgfqpoint{1.656810in}{3.214462in}}%
\pgfpathcurveto{\pgfqpoint{1.664624in}{3.206648in}}{\pgfqpoint{1.675223in}{3.202258in}}{\pgfqpoint{1.686273in}{3.202258in}}%
\pgfpathclose%
\pgfusepath{stroke,fill}%
\end{pgfscope}%
\begin{pgfscope}%
\pgfpathrectangle{\pgfqpoint{0.648703in}{0.548769in}}{\pgfqpoint{5.201297in}{3.102590in}}%
\pgfusepath{clip}%
\pgfsetbuttcap%
\pgfsetroundjoin%
\definecolor{currentfill}{rgb}{0.839216,0.152941,0.156863}%
\pgfsetfillcolor{currentfill}%
\pgfsetlinewidth{1.003750pt}%
\definecolor{currentstroke}{rgb}{0.839216,0.152941,0.156863}%
\pgfsetstrokecolor{currentstroke}%
\pgfsetdash{}{0pt}%
\pgfpathmoveto{\pgfqpoint{1.684133in}{3.265688in}}%
\pgfpathcurveto{\pgfqpoint{1.695183in}{3.265688in}}{\pgfqpoint{1.705782in}{3.270078in}}{\pgfqpoint{1.713595in}{3.277892in}}%
\pgfpathcurveto{\pgfqpoint{1.721409in}{3.285706in}}{\pgfqpoint{1.725799in}{3.296305in}}{\pgfqpoint{1.725799in}{3.307355in}}%
\pgfpathcurveto{\pgfqpoint{1.725799in}{3.318405in}}{\pgfqpoint{1.721409in}{3.329004in}}{\pgfqpoint{1.713595in}{3.336817in}}%
\pgfpathcurveto{\pgfqpoint{1.705782in}{3.344631in}}{\pgfqpoint{1.695183in}{3.349021in}}{\pgfqpoint{1.684133in}{3.349021in}}%
\pgfpathcurveto{\pgfqpoint{1.673082in}{3.349021in}}{\pgfqpoint{1.662483in}{3.344631in}}{\pgfqpoint{1.654670in}{3.336817in}}%
\pgfpathcurveto{\pgfqpoint{1.646856in}{3.329004in}}{\pgfqpoint{1.642466in}{3.318405in}}{\pgfqpoint{1.642466in}{3.307355in}}%
\pgfpathcurveto{\pgfqpoint{1.642466in}{3.296305in}}{\pgfqpoint{1.646856in}{3.285706in}}{\pgfqpoint{1.654670in}{3.277892in}}%
\pgfpathcurveto{\pgfqpoint{1.662483in}{3.270078in}}{\pgfqpoint{1.673082in}{3.265688in}}{\pgfqpoint{1.684133in}{3.265688in}}%
\pgfpathclose%
\pgfusepath{stroke,fill}%
\end{pgfscope}%
\begin{pgfscope}%
\pgfpathrectangle{\pgfqpoint{0.648703in}{0.548769in}}{\pgfqpoint{5.201297in}{3.102590in}}%
\pgfusepath{clip}%
\pgfsetbuttcap%
\pgfsetroundjoin%
\definecolor{currentfill}{rgb}{0.121569,0.466667,0.705882}%
\pgfsetfillcolor{currentfill}%
\pgfsetlinewidth{1.003750pt}%
\definecolor{currentstroke}{rgb}{0.121569,0.466667,0.705882}%
\pgfsetstrokecolor{currentstroke}%
\pgfsetdash{}{0pt}%
\pgfpathmoveto{\pgfqpoint{2.141689in}{2.808990in}}%
\pgfpathcurveto{\pgfqpoint{2.152739in}{2.808990in}}{\pgfqpoint{2.163338in}{2.813380in}}{\pgfqpoint{2.171151in}{2.821193in}}%
\pgfpathcurveto{\pgfqpoint{2.178965in}{2.829007in}}{\pgfqpoint{2.183355in}{2.839606in}}{\pgfqpoint{2.183355in}{2.850656in}}%
\pgfpathcurveto{\pgfqpoint{2.183355in}{2.861706in}}{\pgfqpoint{2.178965in}{2.872305in}}{\pgfqpoint{2.171151in}{2.880119in}}%
\pgfpathcurveto{\pgfqpoint{2.163338in}{2.887933in}}{\pgfqpoint{2.152739in}{2.892323in}}{\pgfqpoint{2.141689in}{2.892323in}}%
\pgfpathcurveto{\pgfqpoint{2.130639in}{2.892323in}}{\pgfqpoint{2.120040in}{2.887933in}}{\pgfqpoint{2.112226in}{2.880119in}}%
\pgfpathcurveto{\pgfqpoint{2.104412in}{2.872305in}}{\pgfqpoint{2.100022in}{2.861706in}}{\pgfqpoint{2.100022in}{2.850656in}}%
\pgfpathcurveto{\pgfqpoint{2.100022in}{2.839606in}}{\pgfqpoint{2.104412in}{2.829007in}}{\pgfqpoint{2.112226in}{2.821193in}}%
\pgfpathcurveto{\pgfqpoint{2.120040in}{2.813380in}}{\pgfqpoint{2.130639in}{2.808990in}}{\pgfqpoint{2.141689in}{2.808990in}}%
\pgfpathclose%
\pgfusepath{stroke,fill}%
\end{pgfscope}%
\begin{pgfscope}%
\pgfpathrectangle{\pgfqpoint{0.648703in}{0.548769in}}{\pgfqpoint{5.201297in}{3.102590in}}%
\pgfusepath{clip}%
\pgfsetbuttcap%
\pgfsetroundjoin%
\definecolor{currentfill}{rgb}{1.000000,0.498039,0.054902}%
\pgfsetfillcolor{currentfill}%
\pgfsetlinewidth{1.003750pt}%
\definecolor{currentstroke}{rgb}{1.000000,0.498039,0.054902}%
\pgfsetstrokecolor{currentstroke}%
\pgfsetdash{}{0pt}%
\pgfpathmoveto{\pgfqpoint{1.749810in}{3.185343in}}%
\pgfpathcurveto{\pgfqpoint{1.760860in}{3.185343in}}{\pgfqpoint{1.771459in}{3.189733in}}{\pgfqpoint{1.779273in}{3.197547in}}%
\pgfpathcurveto{\pgfqpoint{1.787086in}{3.205360in}}{\pgfqpoint{1.791477in}{3.215959in}}{\pgfqpoint{1.791477in}{3.227010in}}%
\pgfpathcurveto{\pgfqpoint{1.791477in}{3.238060in}}{\pgfqpoint{1.787086in}{3.248659in}}{\pgfqpoint{1.779273in}{3.256472in}}%
\pgfpathcurveto{\pgfqpoint{1.771459in}{3.264286in}}{\pgfqpoint{1.760860in}{3.268676in}}{\pgfqpoint{1.749810in}{3.268676in}}%
\pgfpathcurveto{\pgfqpoint{1.738760in}{3.268676in}}{\pgfqpoint{1.728161in}{3.264286in}}{\pgfqpoint{1.720347in}{3.256472in}}%
\pgfpathcurveto{\pgfqpoint{1.712533in}{3.248659in}}{\pgfqpoint{1.708143in}{3.238060in}}{\pgfqpoint{1.708143in}{3.227010in}}%
\pgfpathcurveto{\pgfqpoint{1.708143in}{3.215959in}}{\pgfqpoint{1.712533in}{3.205360in}}{\pgfqpoint{1.720347in}{3.197547in}}%
\pgfpathcurveto{\pgfqpoint{1.728161in}{3.189733in}}{\pgfqpoint{1.738760in}{3.185343in}}{\pgfqpoint{1.749810in}{3.185343in}}%
\pgfpathclose%
\pgfusepath{stroke,fill}%
\end{pgfscope}%
\begin{pgfscope}%
\pgfpathrectangle{\pgfqpoint{0.648703in}{0.548769in}}{\pgfqpoint{5.201297in}{3.102590in}}%
\pgfusepath{clip}%
\pgfsetbuttcap%
\pgfsetroundjoin%
\definecolor{currentfill}{rgb}{1.000000,0.498039,0.054902}%
\pgfsetfillcolor{currentfill}%
\pgfsetlinewidth{1.003750pt}%
\definecolor{currentstroke}{rgb}{1.000000,0.498039,0.054902}%
\pgfsetstrokecolor{currentstroke}%
\pgfsetdash{}{0pt}%
\pgfpathmoveto{\pgfqpoint{1.871756in}{3.193800in}}%
\pgfpathcurveto{\pgfqpoint{1.882807in}{3.193800in}}{\pgfqpoint{1.893406in}{3.198191in}}{\pgfqpoint{1.901219in}{3.206004in}}%
\pgfpathcurveto{\pgfqpoint{1.909033in}{3.213818in}}{\pgfqpoint{1.913423in}{3.224417in}}{\pgfqpoint{1.913423in}{3.235467in}}%
\pgfpathcurveto{\pgfqpoint{1.913423in}{3.246517in}}{\pgfqpoint{1.909033in}{3.257116in}}{\pgfqpoint{1.901219in}{3.264930in}}%
\pgfpathcurveto{\pgfqpoint{1.893406in}{3.272743in}}{\pgfqpoint{1.882807in}{3.277134in}}{\pgfqpoint{1.871756in}{3.277134in}}%
\pgfpathcurveto{\pgfqpoint{1.860706in}{3.277134in}}{\pgfqpoint{1.850107in}{3.272743in}}{\pgfqpoint{1.842294in}{3.264930in}}%
\pgfpathcurveto{\pgfqpoint{1.834480in}{3.257116in}}{\pgfqpoint{1.830090in}{3.246517in}}{\pgfqpoint{1.830090in}{3.235467in}}%
\pgfpathcurveto{\pgfqpoint{1.830090in}{3.224417in}}{\pgfqpoint{1.834480in}{3.213818in}}{\pgfqpoint{1.842294in}{3.206004in}}%
\pgfpathcurveto{\pgfqpoint{1.850107in}{3.198191in}}{\pgfqpoint{1.860706in}{3.193800in}}{\pgfqpoint{1.871756in}{3.193800in}}%
\pgfpathclose%
\pgfusepath{stroke,fill}%
\end{pgfscope}%
\begin{pgfscope}%
\pgfpathrectangle{\pgfqpoint{0.648703in}{0.548769in}}{\pgfqpoint{5.201297in}{3.102590in}}%
\pgfusepath{clip}%
\pgfsetbuttcap%
\pgfsetroundjoin%
\definecolor{currentfill}{rgb}{1.000000,0.498039,0.054902}%
\pgfsetfillcolor{currentfill}%
\pgfsetlinewidth{1.003750pt}%
\definecolor{currentstroke}{rgb}{1.000000,0.498039,0.054902}%
\pgfsetstrokecolor{currentstroke}%
\pgfsetdash{}{0pt}%
\pgfpathmoveto{\pgfqpoint{1.586620in}{3.362948in}}%
\pgfpathcurveto{\pgfqpoint{1.597670in}{3.362948in}}{\pgfqpoint{1.608269in}{3.367338in}}{\pgfqpoint{1.616083in}{3.375152in}}%
\pgfpathcurveto{\pgfqpoint{1.623896in}{3.382965in}}{\pgfqpoint{1.628287in}{3.393564in}}{\pgfqpoint{1.628287in}{3.404615in}}%
\pgfpathcurveto{\pgfqpoint{1.628287in}{3.415665in}}{\pgfqpoint{1.623896in}{3.426264in}}{\pgfqpoint{1.616083in}{3.434077in}}%
\pgfpathcurveto{\pgfqpoint{1.608269in}{3.441891in}}{\pgfqpoint{1.597670in}{3.446281in}}{\pgfqpoint{1.586620in}{3.446281in}}%
\pgfpathcurveto{\pgfqpoint{1.575570in}{3.446281in}}{\pgfqpoint{1.564971in}{3.441891in}}{\pgfqpoint{1.557157in}{3.434077in}}%
\pgfpathcurveto{\pgfqpoint{1.549343in}{3.426264in}}{\pgfqpoint{1.544953in}{3.415665in}}{\pgfqpoint{1.544953in}{3.404615in}}%
\pgfpathcurveto{\pgfqpoint{1.544953in}{3.393564in}}{\pgfqpoint{1.549343in}{3.382965in}}{\pgfqpoint{1.557157in}{3.375152in}}%
\pgfpathcurveto{\pgfqpoint{1.564971in}{3.367338in}}{\pgfqpoint{1.575570in}{3.362948in}}{\pgfqpoint{1.586620in}{3.362948in}}%
\pgfpathclose%
\pgfusepath{stroke,fill}%
\end{pgfscope}%
\begin{pgfscope}%
\pgfpathrectangle{\pgfqpoint{0.648703in}{0.548769in}}{\pgfqpoint{5.201297in}{3.102590in}}%
\pgfusepath{clip}%
\pgfsetbuttcap%
\pgfsetroundjoin%
\definecolor{currentfill}{rgb}{1.000000,0.498039,0.054902}%
\pgfsetfillcolor{currentfill}%
\pgfsetlinewidth{1.003750pt}%
\definecolor{currentstroke}{rgb}{1.000000,0.498039,0.054902}%
\pgfsetstrokecolor{currentstroke}%
\pgfsetdash{}{0pt}%
\pgfpathmoveto{\pgfqpoint{1.759664in}{3.257231in}}%
\pgfpathcurveto{\pgfqpoint{1.770714in}{3.257231in}}{\pgfqpoint{1.781313in}{3.261621in}}{\pgfqpoint{1.789126in}{3.269435in}}%
\pgfpathcurveto{\pgfqpoint{1.796940in}{3.277248in}}{\pgfqpoint{1.801330in}{3.287847in}}{\pgfqpoint{1.801330in}{3.298897in}}%
\pgfpathcurveto{\pgfqpoint{1.801330in}{3.309947in}}{\pgfqpoint{1.796940in}{3.320546in}}{\pgfqpoint{1.789126in}{3.328360in}}%
\pgfpathcurveto{\pgfqpoint{1.781313in}{3.336174in}}{\pgfqpoint{1.770714in}{3.340564in}}{\pgfqpoint{1.759664in}{3.340564in}}%
\pgfpathcurveto{\pgfqpoint{1.748614in}{3.340564in}}{\pgfqpoint{1.738015in}{3.336174in}}{\pgfqpoint{1.730201in}{3.328360in}}%
\pgfpathcurveto{\pgfqpoint{1.722387in}{3.320546in}}{\pgfqpoint{1.717997in}{3.309947in}}{\pgfqpoint{1.717997in}{3.298897in}}%
\pgfpathcurveto{\pgfqpoint{1.717997in}{3.287847in}}{\pgfqpoint{1.722387in}{3.277248in}}{\pgfqpoint{1.730201in}{3.269435in}}%
\pgfpathcurveto{\pgfqpoint{1.738015in}{3.261621in}}{\pgfqpoint{1.748614in}{3.257231in}}{\pgfqpoint{1.759664in}{3.257231in}}%
\pgfpathclose%
\pgfusepath{stroke,fill}%
\end{pgfscope}%
\begin{pgfscope}%
\pgfpathrectangle{\pgfqpoint{0.648703in}{0.548769in}}{\pgfqpoint{5.201297in}{3.102590in}}%
\pgfusepath{clip}%
\pgfsetbuttcap%
\pgfsetroundjoin%
\definecolor{currentfill}{rgb}{1.000000,0.498039,0.054902}%
\pgfsetfillcolor{currentfill}%
\pgfsetlinewidth{1.003750pt}%
\definecolor{currentstroke}{rgb}{1.000000,0.498039,0.054902}%
\pgfsetstrokecolor{currentstroke}%
\pgfsetdash{}{0pt}%
\pgfpathmoveto{\pgfqpoint{1.642800in}{3.189572in}}%
\pgfpathcurveto{\pgfqpoint{1.653850in}{3.189572in}}{\pgfqpoint{1.664449in}{3.193962in}}{\pgfqpoint{1.672263in}{3.201775in}}%
\pgfpathcurveto{\pgfqpoint{1.680076in}{3.209589in}}{\pgfqpoint{1.684467in}{3.220188in}}{\pgfqpoint{1.684467in}{3.231238in}}%
\pgfpathcurveto{\pgfqpoint{1.684467in}{3.242288in}}{\pgfqpoint{1.680076in}{3.252887in}}{\pgfqpoint{1.672263in}{3.260701in}}%
\pgfpathcurveto{\pgfqpoint{1.664449in}{3.268515in}}{\pgfqpoint{1.653850in}{3.272905in}}{\pgfqpoint{1.642800in}{3.272905in}}%
\pgfpathcurveto{\pgfqpoint{1.631750in}{3.272905in}}{\pgfqpoint{1.621151in}{3.268515in}}{\pgfqpoint{1.613337in}{3.260701in}}%
\pgfpathcurveto{\pgfqpoint{1.605524in}{3.252887in}}{\pgfqpoint{1.601133in}{3.242288in}}{\pgfqpoint{1.601133in}{3.231238in}}%
\pgfpathcurveto{\pgfqpoint{1.601133in}{3.220188in}}{\pgfqpoint{1.605524in}{3.209589in}}{\pgfqpoint{1.613337in}{3.201775in}}%
\pgfpathcurveto{\pgfqpoint{1.621151in}{3.193962in}}{\pgfqpoint{1.631750in}{3.189572in}}{\pgfqpoint{1.642800in}{3.189572in}}%
\pgfpathclose%
\pgfusepath{stroke,fill}%
\end{pgfscope}%
\begin{pgfscope}%
\pgfpathrectangle{\pgfqpoint{0.648703in}{0.548769in}}{\pgfqpoint{5.201297in}{3.102590in}}%
\pgfusepath{clip}%
\pgfsetbuttcap%
\pgfsetroundjoin%
\definecolor{currentfill}{rgb}{1.000000,0.498039,0.054902}%
\pgfsetfillcolor{currentfill}%
\pgfsetlinewidth{1.003750pt}%
\definecolor{currentstroke}{rgb}{1.000000,0.498039,0.054902}%
\pgfsetstrokecolor{currentstroke}%
\pgfsetdash{}{0pt}%
\pgfpathmoveto{\pgfqpoint{2.127376in}{3.189572in}}%
\pgfpathcurveto{\pgfqpoint{2.138426in}{3.189572in}}{\pgfqpoint{2.149025in}{3.193962in}}{\pgfqpoint{2.156839in}{3.201775in}}%
\pgfpathcurveto{\pgfqpoint{2.164653in}{3.209589in}}{\pgfqpoint{2.169043in}{3.220188in}}{\pgfqpoint{2.169043in}{3.231238in}}%
\pgfpathcurveto{\pgfqpoint{2.169043in}{3.242288in}}{\pgfqpoint{2.164653in}{3.252887in}}{\pgfqpoint{2.156839in}{3.260701in}}%
\pgfpathcurveto{\pgfqpoint{2.149025in}{3.268515in}}{\pgfqpoint{2.138426in}{3.272905in}}{\pgfqpoint{2.127376in}{3.272905in}}%
\pgfpathcurveto{\pgfqpoint{2.116326in}{3.272905in}}{\pgfqpoint{2.105727in}{3.268515in}}{\pgfqpoint{2.097913in}{3.260701in}}%
\pgfpathcurveto{\pgfqpoint{2.090100in}{3.252887in}}{\pgfqpoint{2.085709in}{3.242288in}}{\pgfqpoint{2.085709in}{3.231238in}}%
\pgfpathcurveto{\pgfqpoint{2.085709in}{3.220188in}}{\pgfqpoint{2.090100in}{3.209589in}}{\pgfqpoint{2.097913in}{3.201775in}}%
\pgfpathcurveto{\pgfqpoint{2.105727in}{3.193962in}}{\pgfqpoint{2.116326in}{3.189572in}}{\pgfqpoint{2.127376in}{3.189572in}}%
\pgfpathclose%
\pgfusepath{stroke,fill}%
\end{pgfscope}%
\begin{pgfscope}%
\pgfpathrectangle{\pgfqpoint{0.648703in}{0.548769in}}{\pgfqpoint{5.201297in}{3.102590in}}%
\pgfusepath{clip}%
\pgfsetbuttcap%
\pgfsetroundjoin%
\definecolor{currentfill}{rgb}{0.121569,0.466667,0.705882}%
\pgfsetfillcolor{currentfill}%
\pgfsetlinewidth{1.003750pt}%
\definecolor{currentstroke}{rgb}{0.121569,0.466667,0.705882}%
\pgfsetstrokecolor{currentstroke}%
\pgfsetdash{}{0pt}%
\pgfpathmoveto{\pgfqpoint{2.442877in}{3.181114in}}%
\pgfpathcurveto{\pgfqpoint{2.453927in}{3.181114in}}{\pgfqpoint{2.464526in}{3.185504in}}{\pgfqpoint{2.472340in}{3.193318in}}%
\pgfpathcurveto{\pgfqpoint{2.480153in}{3.201132in}}{\pgfqpoint{2.484543in}{3.211731in}}{\pgfqpoint{2.484543in}{3.222781in}}%
\pgfpathcurveto{\pgfqpoint{2.484543in}{3.233831in}}{\pgfqpoint{2.480153in}{3.244430in}}{\pgfqpoint{2.472340in}{3.252244in}}%
\pgfpathcurveto{\pgfqpoint{2.464526in}{3.260057in}}{\pgfqpoint{2.453927in}{3.264448in}}{\pgfqpoint{2.442877in}{3.264448in}}%
\pgfpathcurveto{\pgfqpoint{2.431827in}{3.264448in}}{\pgfqpoint{2.421228in}{3.260057in}}{\pgfqpoint{2.413414in}{3.252244in}}%
\pgfpathcurveto{\pgfqpoint{2.405600in}{3.244430in}}{\pgfqpoint{2.401210in}{3.233831in}}{\pgfqpoint{2.401210in}{3.222781in}}%
\pgfpathcurveto{\pgfqpoint{2.401210in}{3.211731in}}{\pgfqpoint{2.405600in}{3.201132in}}{\pgfqpoint{2.413414in}{3.193318in}}%
\pgfpathcurveto{\pgfqpoint{2.421228in}{3.185504in}}{\pgfqpoint{2.431827in}{3.181114in}}{\pgfqpoint{2.442877in}{3.181114in}}%
\pgfpathclose%
\pgfusepath{stroke,fill}%
\end{pgfscope}%
\begin{pgfscope}%
\pgfpathrectangle{\pgfqpoint{0.648703in}{0.548769in}}{\pgfqpoint{5.201297in}{3.102590in}}%
\pgfusepath{clip}%
\pgfsetbuttcap%
\pgfsetroundjoin%
\definecolor{currentfill}{rgb}{1.000000,0.498039,0.054902}%
\pgfsetfillcolor{currentfill}%
\pgfsetlinewidth{1.003750pt}%
\definecolor{currentstroke}{rgb}{1.000000,0.498039,0.054902}%
\pgfsetstrokecolor{currentstroke}%
\pgfsetdash{}{0pt}%
\pgfpathmoveto{\pgfqpoint{2.247183in}{3.185343in}}%
\pgfpathcurveto{\pgfqpoint{2.258233in}{3.185343in}}{\pgfqpoint{2.268832in}{3.189733in}}{\pgfqpoint{2.276645in}{3.197547in}}%
\pgfpathcurveto{\pgfqpoint{2.284459in}{3.205360in}}{\pgfqpoint{2.288849in}{3.215959in}}{\pgfqpoint{2.288849in}{3.227010in}}%
\pgfpathcurveto{\pgfqpoint{2.288849in}{3.238060in}}{\pgfqpoint{2.284459in}{3.248659in}}{\pgfqpoint{2.276645in}{3.256472in}}%
\pgfpathcurveto{\pgfqpoint{2.268832in}{3.264286in}}{\pgfqpoint{2.258233in}{3.268676in}}{\pgfqpoint{2.247183in}{3.268676in}}%
\pgfpathcurveto{\pgfqpoint{2.236132in}{3.268676in}}{\pgfqpoint{2.225533in}{3.264286in}}{\pgfqpoint{2.217720in}{3.256472in}}%
\pgfpathcurveto{\pgfqpoint{2.209906in}{3.248659in}}{\pgfqpoint{2.205516in}{3.238060in}}{\pgfqpoint{2.205516in}{3.227010in}}%
\pgfpathcurveto{\pgfqpoint{2.205516in}{3.215959in}}{\pgfqpoint{2.209906in}{3.205360in}}{\pgfqpoint{2.217720in}{3.197547in}}%
\pgfpathcurveto{\pgfqpoint{2.225533in}{3.189733in}}{\pgfqpoint{2.236132in}{3.185343in}}{\pgfqpoint{2.247183in}{3.185343in}}%
\pgfpathclose%
\pgfusepath{stroke,fill}%
\end{pgfscope}%
\begin{pgfscope}%
\pgfpathrectangle{\pgfqpoint{0.648703in}{0.548769in}}{\pgfqpoint{5.201297in}{3.102590in}}%
\pgfusepath{clip}%
\pgfsetbuttcap%
\pgfsetroundjoin%
\definecolor{currentfill}{rgb}{0.121569,0.466667,0.705882}%
\pgfsetfillcolor{currentfill}%
\pgfsetlinewidth{1.003750pt}%
\definecolor{currentstroke}{rgb}{0.121569,0.466667,0.705882}%
\pgfsetstrokecolor{currentstroke}%
\pgfsetdash{}{0pt}%
\pgfpathmoveto{\pgfqpoint{1.542345in}{0.681958in}}%
\pgfpathcurveto{\pgfqpoint{1.553395in}{0.681958in}}{\pgfqpoint{1.563994in}{0.686349in}}{\pgfqpoint{1.571807in}{0.694162in}}%
\pgfpathcurveto{\pgfqpoint{1.579621in}{0.701976in}}{\pgfqpoint{1.584011in}{0.712575in}}{\pgfqpoint{1.584011in}{0.723625in}}%
\pgfpathcurveto{\pgfqpoint{1.584011in}{0.734675in}}{\pgfqpoint{1.579621in}{0.745274in}}{\pgfqpoint{1.571807in}{0.753088in}}%
\pgfpathcurveto{\pgfqpoint{1.563994in}{0.760902in}}{\pgfqpoint{1.553395in}{0.765292in}}{\pgfqpoint{1.542345in}{0.765292in}}%
\pgfpathcurveto{\pgfqpoint{1.531294in}{0.765292in}}{\pgfqpoint{1.520695in}{0.760902in}}{\pgfqpoint{1.512882in}{0.753088in}}%
\pgfpathcurveto{\pgfqpoint{1.505068in}{0.745274in}}{\pgfqpoint{1.500678in}{0.734675in}}{\pgfqpoint{1.500678in}{0.723625in}}%
\pgfpathcurveto{\pgfqpoint{1.500678in}{0.712575in}}{\pgfqpoint{1.505068in}{0.701976in}}{\pgfqpoint{1.512882in}{0.694162in}}%
\pgfpathcurveto{\pgfqpoint{1.520695in}{0.686349in}}{\pgfqpoint{1.531294in}{0.681958in}}{\pgfqpoint{1.542345in}{0.681958in}}%
\pgfpathclose%
\pgfusepath{stroke,fill}%
\end{pgfscope}%
\begin{pgfscope}%
\pgfpathrectangle{\pgfqpoint{0.648703in}{0.548769in}}{\pgfqpoint{5.201297in}{3.102590in}}%
\pgfusepath{clip}%
\pgfsetbuttcap%
\pgfsetroundjoin%
\definecolor{currentfill}{rgb}{0.839216,0.152941,0.156863}%
\pgfsetfillcolor{currentfill}%
\pgfsetlinewidth{1.003750pt}%
\definecolor{currentstroke}{rgb}{0.839216,0.152941,0.156863}%
\pgfsetstrokecolor{currentstroke}%
\pgfsetdash{}{0pt}%
\pgfpathmoveto{\pgfqpoint{1.986257in}{3.214944in}}%
\pgfpathcurveto{\pgfqpoint{1.997307in}{3.214944in}}{\pgfqpoint{2.007906in}{3.219334in}}{\pgfqpoint{2.015720in}{3.227148in}}%
\pgfpathcurveto{\pgfqpoint{2.023533in}{3.234961in}}{\pgfqpoint{2.027924in}{3.245560in}}{\pgfqpoint{2.027924in}{3.256610in}}%
\pgfpathcurveto{\pgfqpoint{2.027924in}{3.267661in}}{\pgfqpoint{2.023533in}{3.278260in}}{\pgfqpoint{2.015720in}{3.286073in}}%
\pgfpathcurveto{\pgfqpoint{2.007906in}{3.293887in}}{\pgfqpoint{1.997307in}{3.298277in}}{\pgfqpoint{1.986257in}{3.298277in}}%
\pgfpathcurveto{\pgfqpoint{1.975207in}{3.298277in}}{\pgfqpoint{1.964608in}{3.293887in}}{\pgfqpoint{1.956794in}{3.286073in}}%
\pgfpathcurveto{\pgfqpoint{1.948981in}{3.278260in}}{\pgfqpoint{1.944590in}{3.267661in}}{\pgfqpoint{1.944590in}{3.256610in}}%
\pgfpathcurveto{\pgfqpoint{1.944590in}{3.245560in}}{\pgfqpoint{1.948981in}{3.234961in}}{\pgfqpoint{1.956794in}{3.227148in}}%
\pgfpathcurveto{\pgfqpoint{1.964608in}{3.219334in}}{\pgfqpoint{1.975207in}{3.214944in}}{\pgfqpoint{1.986257in}{3.214944in}}%
\pgfpathclose%
\pgfusepath{stroke,fill}%
\end{pgfscope}%
\begin{pgfscope}%
\pgfpathrectangle{\pgfqpoint{0.648703in}{0.548769in}}{\pgfqpoint{5.201297in}{3.102590in}}%
\pgfusepath{clip}%
\pgfsetbuttcap%
\pgfsetroundjoin%
\definecolor{currentfill}{rgb}{1.000000,0.498039,0.054902}%
\pgfsetfillcolor{currentfill}%
\pgfsetlinewidth{1.003750pt}%
\definecolor{currentstroke}{rgb}{1.000000,0.498039,0.054902}%
\pgfsetstrokecolor{currentstroke}%
\pgfsetdash{}{0pt}%
\pgfpathmoveto{\pgfqpoint{1.362657in}{3.185343in}}%
\pgfpathcurveto{\pgfqpoint{1.373707in}{3.185343in}}{\pgfqpoint{1.384306in}{3.189733in}}{\pgfqpoint{1.392120in}{3.197547in}}%
\pgfpathcurveto{\pgfqpoint{1.399934in}{3.205360in}}{\pgfqpoint{1.404324in}{3.215959in}}{\pgfqpoint{1.404324in}{3.227010in}}%
\pgfpathcurveto{\pgfqpoint{1.404324in}{3.238060in}}{\pgfqpoint{1.399934in}{3.248659in}}{\pgfqpoint{1.392120in}{3.256472in}}%
\pgfpathcurveto{\pgfqpoint{1.384306in}{3.264286in}}{\pgfqpoint{1.373707in}{3.268676in}}{\pgfqpoint{1.362657in}{3.268676in}}%
\pgfpathcurveto{\pgfqpoint{1.351607in}{3.268676in}}{\pgfqpoint{1.341008in}{3.264286in}}{\pgfqpoint{1.333194in}{3.256472in}}%
\pgfpathcurveto{\pgfqpoint{1.325381in}{3.248659in}}{\pgfqpoint{1.320991in}{3.238060in}}{\pgfqpoint{1.320991in}{3.227010in}}%
\pgfpathcurveto{\pgfqpoint{1.320991in}{3.215959in}}{\pgfqpoint{1.325381in}{3.205360in}}{\pgfqpoint{1.333194in}{3.197547in}}%
\pgfpathcurveto{\pgfqpoint{1.341008in}{3.189733in}}{\pgfqpoint{1.351607in}{3.185343in}}{\pgfqpoint{1.362657in}{3.185343in}}%
\pgfpathclose%
\pgfusepath{stroke,fill}%
\end{pgfscope}%
\begin{pgfscope}%
\pgfpathrectangle{\pgfqpoint{0.648703in}{0.548769in}}{\pgfqpoint{5.201297in}{3.102590in}}%
\pgfusepath{clip}%
\pgfsetbuttcap%
\pgfsetroundjoin%
\definecolor{currentfill}{rgb}{0.121569,0.466667,0.705882}%
\pgfsetfillcolor{currentfill}%
\pgfsetlinewidth{1.003750pt}%
\definecolor{currentstroke}{rgb}{0.121569,0.466667,0.705882}%
\pgfsetstrokecolor{currentstroke}%
\pgfsetdash{}{0pt}%
\pgfpathmoveto{\pgfqpoint{1.585996in}{1.138657in}}%
\pgfpathcurveto{\pgfqpoint{1.597046in}{1.138657in}}{\pgfqpoint{1.607645in}{1.143047in}}{\pgfqpoint{1.615458in}{1.150861in}}%
\pgfpathcurveto{\pgfqpoint{1.623272in}{1.158674in}}{\pgfqpoint{1.627662in}{1.169274in}}{\pgfqpoint{1.627662in}{1.180324in}}%
\pgfpathcurveto{\pgfqpoint{1.627662in}{1.191374in}}{\pgfqpoint{1.623272in}{1.201973in}}{\pgfqpoint{1.615458in}{1.209786in}}%
\pgfpathcurveto{\pgfqpoint{1.607645in}{1.217600in}}{\pgfqpoint{1.597046in}{1.221990in}}{\pgfqpoint{1.585996in}{1.221990in}}%
\pgfpathcurveto{\pgfqpoint{1.574946in}{1.221990in}}{\pgfqpoint{1.564347in}{1.217600in}}{\pgfqpoint{1.556533in}{1.209786in}}%
\pgfpathcurveto{\pgfqpoint{1.548719in}{1.201973in}}{\pgfqpoint{1.544329in}{1.191374in}}{\pgfqpoint{1.544329in}{1.180324in}}%
\pgfpathcurveto{\pgfqpoint{1.544329in}{1.169274in}}{\pgfqpoint{1.548719in}{1.158674in}}{\pgfqpoint{1.556533in}{1.150861in}}%
\pgfpathcurveto{\pgfqpoint{1.564347in}{1.143047in}}{\pgfqpoint{1.574946in}{1.138657in}}{\pgfqpoint{1.585996in}{1.138657in}}%
\pgfpathclose%
\pgfusepath{stroke,fill}%
\end{pgfscope}%
\begin{pgfscope}%
\pgfpathrectangle{\pgfqpoint{0.648703in}{0.548769in}}{\pgfqpoint{5.201297in}{3.102590in}}%
\pgfusepath{clip}%
\pgfsetbuttcap%
\pgfsetroundjoin%
\definecolor{currentfill}{rgb}{0.121569,0.466667,0.705882}%
\pgfsetfillcolor{currentfill}%
\pgfsetlinewidth{1.003750pt}%
\definecolor{currentstroke}{rgb}{0.121569,0.466667,0.705882}%
\pgfsetstrokecolor{currentstroke}%
\pgfsetdash{}{0pt}%
\pgfpathmoveto{\pgfqpoint{1.800729in}{1.087913in}}%
\pgfpathcurveto{\pgfqpoint{1.811779in}{1.087913in}}{\pgfqpoint{1.822378in}{1.092303in}}{\pgfqpoint{1.830191in}{1.100117in}}%
\pgfpathcurveto{\pgfqpoint{1.838005in}{1.107930in}}{\pgfqpoint{1.842395in}{1.118529in}}{\pgfqpoint{1.842395in}{1.129579in}}%
\pgfpathcurveto{\pgfqpoint{1.842395in}{1.140629in}}{\pgfqpoint{1.838005in}{1.151229in}}{\pgfqpoint{1.830191in}{1.159042in}}%
\pgfpathcurveto{\pgfqpoint{1.822378in}{1.166856in}}{\pgfqpoint{1.811779in}{1.171246in}}{\pgfqpoint{1.800729in}{1.171246in}}%
\pgfpathcurveto{\pgfqpoint{1.789679in}{1.171246in}}{\pgfqpoint{1.779080in}{1.166856in}}{\pgfqpoint{1.771266in}{1.159042in}}%
\pgfpathcurveto{\pgfqpoint{1.763452in}{1.151229in}}{\pgfqpoint{1.759062in}{1.140629in}}{\pgfqpoint{1.759062in}{1.129579in}}%
\pgfpathcurveto{\pgfqpoint{1.759062in}{1.118529in}}{\pgfqpoint{1.763452in}{1.107930in}}{\pgfqpoint{1.771266in}{1.100117in}}%
\pgfpathcurveto{\pgfqpoint{1.779080in}{1.092303in}}{\pgfqpoint{1.789679in}{1.087913in}}{\pgfqpoint{1.800729in}{1.087913in}}%
\pgfpathclose%
\pgfusepath{stroke,fill}%
\end{pgfscope}%
\begin{pgfscope}%
\pgfpathrectangle{\pgfqpoint{0.648703in}{0.548769in}}{\pgfqpoint{5.201297in}{3.102590in}}%
\pgfusepath{clip}%
\pgfsetbuttcap%
\pgfsetroundjoin%
\definecolor{currentfill}{rgb}{1.000000,0.498039,0.054902}%
\pgfsetfillcolor{currentfill}%
\pgfsetlinewidth{1.003750pt}%
\definecolor{currentstroke}{rgb}{1.000000,0.498039,0.054902}%
\pgfsetstrokecolor{currentstroke}%
\pgfsetdash{}{0pt}%
\pgfpathmoveto{\pgfqpoint{1.643513in}{3.189572in}}%
\pgfpathcurveto{\pgfqpoint{1.654564in}{3.189572in}}{\pgfqpoint{1.665163in}{3.193962in}}{\pgfqpoint{1.672976in}{3.201775in}}%
\pgfpathcurveto{\pgfqpoint{1.680790in}{3.209589in}}{\pgfqpoint{1.685180in}{3.220188in}}{\pgfqpoint{1.685180in}{3.231238in}}%
\pgfpathcurveto{\pgfqpoint{1.685180in}{3.242288in}}{\pgfqpoint{1.680790in}{3.252887in}}{\pgfqpoint{1.672976in}{3.260701in}}%
\pgfpathcurveto{\pgfqpoint{1.665163in}{3.268515in}}{\pgfqpoint{1.654564in}{3.272905in}}{\pgfqpoint{1.643513in}{3.272905in}}%
\pgfpathcurveto{\pgfqpoint{1.632463in}{3.272905in}}{\pgfqpoint{1.621864in}{3.268515in}}{\pgfqpoint{1.614051in}{3.260701in}}%
\pgfpathcurveto{\pgfqpoint{1.606237in}{3.252887in}}{\pgfqpoint{1.601847in}{3.242288in}}{\pgfqpoint{1.601847in}{3.231238in}}%
\pgfpathcurveto{\pgfqpoint{1.601847in}{3.220188in}}{\pgfqpoint{1.606237in}{3.209589in}}{\pgfqpoint{1.614051in}{3.201775in}}%
\pgfpathcurveto{\pgfqpoint{1.621864in}{3.193962in}}{\pgfqpoint{1.632463in}{3.189572in}}{\pgfqpoint{1.643513in}{3.189572in}}%
\pgfpathclose%
\pgfusepath{stroke,fill}%
\end{pgfscope}%
\begin{pgfscope}%
\pgfpathrectangle{\pgfqpoint{0.648703in}{0.548769in}}{\pgfqpoint{5.201297in}{3.102590in}}%
\pgfusepath{clip}%
\pgfsetbuttcap%
\pgfsetroundjoin%
\definecolor{currentfill}{rgb}{0.121569,0.466667,0.705882}%
\pgfsetfillcolor{currentfill}%
\pgfsetlinewidth{1.003750pt}%
\definecolor{currentstroke}{rgb}{0.121569,0.466667,0.705882}%
\pgfsetstrokecolor{currentstroke}%
\pgfsetdash{}{0pt}%
\pgfpathmoveto{\pgfqpoint{1.524733in}{1.024482in}}%
\pgfpathcurveto{\pgfqpoint{1.535783in}{1.024482in}}{\pgfqpoint{1.546382in}{1.028873in}}{\pgfqpoint{1.554195in}{1.036686in}}%
\pgfpathcurveto{\pgfqpoint{1.562009in}{1.044500in}}{\pgfqpoint{1.566399in}{1.055099in}}{\pgfqpoint{1.566399in}{1.066149in}}%
\pgfpathcurveto{\pgfqpoint{1.566399in}{1.077199in}}{\pgfqpoint{1.562009in}{1.087798in}}{\pgfqpoint{1.554195in}{1.095612in}}%
\pgfpathcurveto{\pgfqpoint{1.546382in}{1.103425in}}{\pgfqpoint{1.535783in}{1.107816in}}{\pgfqpoint{1.524733in}{1.107816in}}%
\pgfpathcurveto{\pgfqpoint{1.513682in}{1.107816in}}{\pgfqpoint{1.503083in}{1.103425in}}{\pgfqpoint{1.495270in}{1.095612in}}%
\pgfpathcurveto{\pgfqpoint{1.487456in}{1.087798in}}{\pgfqpoint{1.483066in}{1.077199in}}{\pgfqpoint{1.483066in}{1.066149in}}%
\pgfpathcurveto{\pgfqpoint{1.483066in}{1.055099in}}{\pgfqpoint{1.487456in}{1.044500in}}{\pgfqpoint{1.495270in}{1.036686in}}%
\pgfpathcurveto{\pgfqpoint{1.503083in}{1.028873in}}{\pgfqpoint{1.513682in}{1.024482in}}{\pgfqpoint{1.524733in}{1.024482in}}%
\pgfpathclose%
\pgfusepath{stroke,fill}%
\end{pgfscope}%
\begin{pgfscope}%
\pgfpathrectangle{\pgfqpoint{0.648703in}{0.548769in}}{\pgfqpoint{5.201297in}{3.102590in}}%
\pgfusepath{clip}%
\pgfsetbuttcap%
\pgfsetroundjoin%
\definecolor{currentfill}{rgb}{0.121569,0.466667,0.705882}%
\pgfsetfillcolor{currentfill}%
\pgfsetlinewidth{1.003750pt}%
\definecolor{currentstroke}{rgb}{0.121569,0.466667,0.705882}%
\pgfsetstrokecolor{currentstroke}%
\pgfsetdash{}{0pt}%
\pgfpathmoveto{\pgfqpoint{2.260737in}{0.813048in}}%
\pgfpathcurveto{\pgfqpoint{2.271787in}{0.813048in}}{\pgfqpoint{2.282386in}{0.817438in}}{\pgfqpoint{2.290200in}{0.825252in}}%
\pgfpathcurveto{\pgfqpoint{2.298014in}{0.833065in}}{\pgfqpoint{2.302404in}{0.843664in}}{\pgfqpoint{2.302404in}{0.854715in}}%
\pgfpathcurveto{\pgfqpoint{2.302404in}{0.865765in}}{\pgfqpoint{2.298014in}{0.876364in}}{\pgfqpoint{2.290200in}{0.884177in}}%
\pgfpathcurveto{\pgfqpoint{2.282386in}{0.891991in}}{\pgfqpoint{2.271787in}{0.896381in}}{\pgfqpoint{2.260737in}{0.896381in}}%
\pgfpathcurveto{\pgfqpoint{2.249687in}{0.896381in}}{\pgfqpoint{2.239088in}{0.891991in}}{\pgfqpoint{2.231274in}{0.884177in}}%
\pgfpathcurveto{\pgfqpoint{2.223461in}{0.876364in}}{\pgfqpoint{2.219070in}{0.865765in}}{\pgfqpoint{2.219070in}{0.854715in}}%
\pgfpathcurveto{\pgfqpoint{2.219070in}{0.843664in}}{\pgfqpoint{2.223461in}{0.833065in}}{\pgfqpoint{2.231274in}{0.825252in}}%
\pgfpathcurveto{\pgfqpoint{2.239088in}{0.817438in}}{\pgfqpoint{2.249687in}{0.813048in}}{\pgfqpoint{2.260737in}{0.813048in}}%
\pgfpathclose%
\pgfusepath{stroke,fill}%
\end{pgfscope}%
\begin{pgfscope}%
\pgfpathrectangle{\pgfqpoint{0.648703in}{0.548769in}}{\pgfqpoint{5.201297in}{3.102590in}}%
\pgfusepath{clip}%
\pgfsetbuttcap%
\pgfsetroundjoin%
\definecolor{currentfill}{rgb}{0.839216,0.152941,0.156863}%
\pgfsetfillcolor{currentfill}%
\pgfsetlinewidth{1.003750pt}%
\definecolor{currentstroke}{rgb}{0.839216,0.152941,0.156863}%
\pgfsetstrokecolor{currentstroke}%
\pgfsetdash{}{0pt}%
\pgfpathmoveto{\pgfqpoint{1.779639in}{3.202258in}}%
\pgfpathcurveto{\pgfqpoint{1.790689in}{3.202258in}}{\pgfqpoint{1.801288in}{3.206648in}}{\pgfqpoint{1.809102in}{3.214462in}}%
\pgfpathcurveto{\pgfqpoint{1.816915in}{3.222275in}}{\pgfqpoint{1.821305in}{3.232874in}}{\pgfqpoint{1.821305in}{3.243924in}}%
\pgfpathcurveto{\pgfqpoint{1.821305in}{3.254974in}}{\pgfqpoint{1.816915in}{3.265573in}}{\pgfqpoint{1.809102in}{3.273387in}}%
\pgfpathcurveto{\pgfqpoint{1.801288in}{3.281201in}}{\pgfqpoint{1.790689in}{3.285591in}}{\pgfqpoint{1.779639in}{3.285591in}}%
\pgfpathcurveto{\pgfqpoint{1.768589in}{3.285591in}}{\pgfqpoint{1.757990in}{3.281201in}}{\pgfqpoint{1.750176in}{3.273387in}}%
\pgfpathcurveto{\pgfqpoint{1.742362in}{3.265573in}}{\pgfqpoint{1.737972in}{3.254974in}}{\pgfqpoint{1.737972in}{3.243924in}}%
\pgfpathcurveto{\pgfqpoint{1.737972in}{3.232874in}}{\pgfqpoint{1.742362in}{3.222275in}}{\pgfqpoint{1.750176in}{3.214462in}}%
\pgfpathcurveto{\pgfqpoint{1.757990in}{3.206648in}}{\pgfqpoint{1.768589in}{3.202258in}}{\pgfqpoint{1.779639in}{3.202258in}}%
\pgfpathclose%
\pgfusepath{stroke,fill}%
\end{pgfscope}%
\begin{pgfscope}%
\pgfsetbuttcap%
\pgfsetroundjoin%
\definecolor{currentfill}{rgb}{0.000000,0.000000,0.000000}%
\pgfsetfillcolor{currentfill}%
\pgfsetlinewidth{0.803000pt}%
\definecolor{currentstroke}{rgb}{0.000000,0.000000,0.000000}%
\pgfsetstrokecolor{currentstroke}%
\pgfsetdash{}{0pt}%
\pgfsys@defobject{currentmarker}{\pgfqpoint{0.000000in}{-0.048611in}}{\pgfqpoint{0.000000in}{0.000000in}}{%
\pgfpathmoveto{\pgfqpoint{0.000000in}{0.000000in}}%
\pgfpathlineto{\pgfqpoint{0.000000in}{-0.048611in}}%
\pgfusepath{stroke,fill}%
}%
\begin{pgfscope}%
\pgfsys@transformshift{0.854628in}{0.548769in}%
\pgfsys@useobject{currentmarker}{}%
\end{pgfscope}%
\end{pgfscope}%
\begin{pgfscope}%
\definecolor{textcolor}{rgb}{0.000000,0.000000,0.000000}%
\pgfsetstrokecolor{textcolor}%
\pgfsetfillcolor{textcolor}%
\pgftext[x=0.854628in,y=0.451547in,,top]{\color{textcolor}\sffamily\fontsize{10.000000}{12.000000}\selectfont \(\displaystyle {0}\)}%
\end{pgfscope}%
\begin{pgfscope}%
\pgfsetbuttcap%
\pgfsetroundjoin%
\definecolor{currentfill}{rgb}{0.000000,0.000000,0.000000}%
\pgfsetfillcolor{currentfill}%
\pgfsetlinewidth{0.803000pt}%
\definecolor{currentstroke}{rgb}{0.000000,0.000000,0.000000}%
\pgfsetstrokecolor{currentstroke}%
\pgfsetdash{}{0pt}%
\pgfsys@defobject{currentmarker}{\pgfqpoint{0.000000in}{-0.048611in}}{\pgfqpoint{0.000000in}{0.000000in}}{%
\pgfpathmoveto{\pgfqpoint{0.000000in}{0.000000in}}%
\pgfpathlineto{\pgfqpoint{0.000000in}{-0.048611in}}%
\pgfusepath{stroke,fill}%
}%
\begin{pgfscope}%
\pgfsys@transformshift{1.746377in}{0.548769in}%
\pgfsys@useobject{currentmarker}{}%
\end{pgfscope}%
\end{pgfscope}%
\begin{pgfscope}%
\definecolor{textcolor}{rgb}{0.000000,0.000000,0.000000}%
\pgfsetstrokecolor{textcolor}%
\pgfsetfillcolor{textcolor}%
\pgftext[x=1.746377in,y=0.451547in,,top]{\color{textcolor}\sffamily\fontsize{10.000000}{12.000000}\selectfont \(\displaystyle {20000}\)}%
\end{pgfscope}%
\begin{pgfscope}%
\pgfsetbuttcap%
\pgfsetroundjoin%
\definecolor{currentfill}{rgb}{0.000000,0.000000,0.000000}%
\pgfsetfillcolor{currentfill}%
\pgfsetlinewidth{0.803000pt}%
\definecolor{currentstroke}{rgb}{0.000000,0.000000,0.000000}%
\pgfsetstrokecolor{currentstroke}%
\pgfsetdash{}{0pt}%
\pgfsys@defobject{currentmarker}{\pgfqpoint{0.000000in}{-0.048611in}}{\pgfqpoint{0.000000in}{0.000000in}}{%
\pgfpathmoveto{\pgfqpoint{0.000000in}{0.000000in}}%
\pgfpathlineto{\pgfqpoint{0.000000in}{-0.048611in}}%
\pgfusepath{stroke,fill}%
}%
\begin{pgfscope}%
\pgfsys@transformshift{2.638125in}{0.548769in}%
\pgfsys@useobject{currentmarker}{}%
\end{pgfscope}%
\end{pgfscope}%
\begin{pgfscope}%
\definecolor{textcolor}{rgb}{0.000000,0.000000,0.000000}%
\pgfsetstrokecolor{textcolor}%
\pgfsetfillcolor{textcolor}%
\pgftext[x=2.638125in,y=0.451547in,,top]{\color{textcolor}\sffamily\fontsize{10.000000}{12.000000}\selectfont \(\displaystyle {40000}\)}%
\end{pgfscope}%
\begin{pgfscope}%
\pgfsetbuttcap%
\pgfsetroundjoin%
\definecolor{currentfill}{rgb}{0.000000,0.000000,0.000000}%
\pgfsetfillcolor{currentfill}%
\pgfsetlinewidth{0.803000pt}%
\definecolor{currentstroke}{rgb}{0.000000,0.000000,0.000000}%
\pgfsetstrokecolor{currentstroke}%
\pgfsetdash{}{0pt}%
\pgfsys@defobject{currentmarker}{\pgfqpoint{0.000000in}{-0.048611in}}{\pgfqpoint{0.000000in}{0.000000in}}{%
\pgfpathmoveto{\pgfqpoint{0.000000in}{0.000000in}}%
\pgfpathlineto{\pgfqpoint{0.000000in}{-0.048611in}}%
\pgfusepath{stroke,fill}%
}%
\begin{pgfscope}%
\pgfsys@transformshift{3.529873in}{0.548769in}%
\pgfsys@useobject{currentmarker}{}%
\end{pgfscope}%
\end{pgfscope}%
\begin{pgfscope}%
\definecolor{textcolor}{rgb}{0.000000,0.000000,0.000000}%
\pgfsetstrokecolor{textcolor}%
\pgfsetfillcolor{textcolor}%
\pgftext[x=3.529873in,y=0.451547in,,top]{\color{textcolor}\sffamily\fontsize{10.000000}{12.000000}\selectfont \(\displaystyle {60000}\)}%
\end{pgfscope}%
\begin{pgfscope}%
\pgfsetbuttcap%
\pgfsetroundjoin%
\definecolor{currentfill}{rgb}{0.000000,0.000000,0.000000}%
\pgfsetfillcolor{currentfill}%
\pgfsetlinewidth{0.803000pt}%
\definecolor{currentstroke}{rgb}{0.000000,0.000000,0.000000}%
\pgfsetstrokecolor{currentstroke}%
\pgfsetdash{}{0pt}%
\pgfsys@defobject{currentmarker}{\pgfqpoint{0.000000in}{-0.048611in}}{\pgfqpoint{0.000000in}{0.000000in}}{%
\pgfpathmoveto{\pgfqpoint{0.000000in}{0.000000in}}%
\pgfpathlineto{\pgfqpoint{0.000000in}{-0.048611in}}%
\pgfusepath{stroke,fill}%
}%
\begin{pgfscope}%
\pgfsys@transformshift{4.421622in}{0.548769in}%
\pgfsys@useobject{currentmarker}{}%
\end{pgfscope}%
\end{pgfscope}%
\begin{pgfscope}%
\definecolor{textcolor}{rgb}{0.000000,0.000000,0.000000}%
\pgfsetstrokecolor{textcolor}%
\pgfsetfillcolor{textcolor}%
\pgftext[x=4.421622in,y=0.451547in,,top]{\color{textcolor}\sffamily\fontsize{10.000000}{12.000000}\selectfont \(\displaystyle {80000}\)}%
\end{pgfscope}%
\begin{pgfscope}%
\pgfsetbuttcap%
\pgfsetroundjoin%
\definecolor{currentfill}{rgb}{0.000000,0.000000,0.000000}%
\pgfsetfillcolor{currentfill}%
\pgfsetlinewidth{0.803000pt}%
\definecolor{currentstroke}{rgb}{0.000000,0.000000,0.000000}%
\pgfsetstrokecolor{currentstroke}%
\pgfsetdash{}{0pt}%
\pgfsys@defobject{currentmarker}{\pgfqpoint{0.000000in}{-0.048611in}}{\pgfqpoint{0.000000in}{0.000000in}}{%
\pgfpathmoveto{\pgfqpoint{0.000000in}{0.000000in}}%
\pgfpathlineto{\pgfqpoint{0.000000in}{-0.048611in}}%
\pgfusepath{stroke,fill}%
}%
\begin{pgfscope}%
\pgfsys@transformshift{5.313370in}{0.548769in}%
\pgfsys@useobject{currentmarker}{}%
\end{pgfscope}%
\end{pgfscope}%
\begin{pgfscope}%
\definecolor{textcolor}{rgb}{0.000000,0.000000,0.000000}%
\pgfsetstrokecolor{textcolor}%
\pgfsetfillcolor{textcolor}%
\pgftext[x=5.313370in,y=0.451547in,,top]{\color{textcolor}\sffamily\fontsize{10.000000}{12.000000}\selectfont \(\displaystyle {100000}\)}%
\end{pgfscope}%
\begin{pgfscope}%
\definecolor{textcolor}{rgb}{0.000000,0.000000,0.000000}%
\pgfsetstrokecolor{textcolor}%
\pgfsetfillcolor{textcolor}%
\pgftext[x=3.249352in,y=0.272658in,,top]{\color{textcolor}\sffamily\fontsize{10.000000}{12.000000}\selectfont Classes}%
\end{pgfscope}%
\begin{pgfscope}%
\pgfsetbuttcap%
\pgfsetroundjoin%
\definecolor{currentfill}{rgb}{0.000000,0.000000,0.000000}%
\pgfsetfillcolor{currentfill}%
\pgfsetlinewidth{0.803000pt}%
\definecolor{currentstroke}{rgb}{0.000000,0.000000,0.000000}%
\pgfsetstrokecolor{currentstroke}%
\pgfsetdash{}{0pt}%
\pgfsys@defobject{currentmarker}{\pgfqpoint{-0.048611in}{0.000000in}}{\pgfqpoint{0.000000in}{0.000000in}}{%
\pgfpathmoveto{\pgfqpoint{0.000000in}{0.000000in}}%
\pgfpathlineto{\pgfqpoint{-0.048611in}{0.000000in}}%
\pgfusepath{stroke,fill}%
}%
\begin{pgfscope}%
\pgfsys@transformshift{0.648703in}{0.689796in}%
\pgfsys@useobject{currentmarker}{}%
\end{pgfscope}%
\end{pgfscope}%
\begin{pgfscope}%
\definecolor{textcolor}{rgb}{0.000000,0.000000,0.000000}%
\pgfsetstrokecolor{textcolor}%
\pgfsetfillcolor{textcolor}%
\pgftext[x=0.482036in, y=0.641601in, left, base]{\color{textcolor}\sffamily\fontsize{10.000000}{12.000000}\selectfont \(\displaystyle {0}\)}%
\end{pgfscope}%
\begin{pgfscope}%
\pgfsetbuttcap%
\pgfsetroundjoin%
\definecolor{currentfill}{rgb}{0.000000,0.000000,0.000000}%
\pgfsetfillcolor{currentfill}%
\pgfsetlinewidth{0.803000pt}%
\definecolor{currentstroke}{rgb}{0.000000,0.000000,0.000000}%
\pgfsetstrokecolor{currentstroke}%
\pgfsetdash{}{0pt}%
\pgfsys@defobject{currentmarker}{\pgfqpoint{-0.048611in}{0.000000in}}{\pgfqpoint{0.000000in}{0.000000in}}{%
\pgfpathmoveto{\pgfqpoint{0.000000in}{0.000000in}}%
\pgfpathlineto{\pgfqpoint{-0.048611in}{0.000000in}}%
\pgfusepath{stroke,fill}%
}%
\begin{pgfscope}%
\pgfsys@transformshift{0.648703in}{1.112665in}%
\pgfsys@useobject{currentmarker}{}%
\end{pgfscope}%
\end{pgfscope}%
\begin{pgfscope}%
\definecolor{textcolor}{rgb}{0.000000,0.000000,0.000000}%
\pgfsetstrokecolor{textcolor}%
\pgfsetfillcolor{textcolor}%
\pgftext[x=0.343147in, y=1.064470in, left, base]{\color{textcolor}\sffamily\fontsize{10.000000}{12.000000}\selectfont \(\displaystyle {100}\)}%
\end{pgfscope}%
\begin{pgfscope}%
\pgfsetbuttcap%
\pgfsetroundjoin%
\definecolor{currentfill}{rgb}{0.000000,0.000000,0.000000}%
\pgfsetfillcolor{currentfill}%
\pgfsetlinewidth{0.803000pt}%
\definecolor{currentstroke}{rgb}{0.000000,0.000000,0.000000}%
\pgfsetstrokecolor{currentstroke}%
\pgfsetdash{}{0pt}%
\pgfsys@defobject{currentmarker}{\pgfqpoint{-0.048611in}{0.000000in}}{\pgfqpoint{0.000000in}{0.000000in}}{%
\pgfpathmoveto{\pgfqpoint{0.000000in}{0.000000in}}%
\pgfpathlineto{\pgfqpoint{-0.048611in}{0.000000in}}%
\pgfusepath{stroke,fill}%
}%
\begin{pgfscope}%
\pgfsys@transformshift{0.648703in}{1.535534in}%
\pgfsys@useobject{currentmarker}{}%
\end{pgfscope}%
\end{pgfscope}%
\begin{pgfscope}%
\definecolor{textcolor}{rgb}{0.000000,0.000000,0.000000}%
\pgfsetstrokecolor{textcolor}%
\pgfsetfillcolor{textcolor}%
\pgftext[x=0.343147in, y=1.487339in, left, base]{\color{textcolor}\sffamily\fontsize{10.000000}{12.000000}\selectfont \(\displaystyle {200}\)}%
\end{pgfscope}%
\begin{pgfscope}%
\pgfsetbuttcap%
\pgfsetroundjoin%
\definecolor{currentfill}{rgb}{0.000000,0.000000,0.000000}%
\pgfsetfillcolor{currentfill}%
\pgfsetlinewidth{0.803000pt}%
\definecolor{currentstroke}{rgb}{0.000000,0.000000,0.000000}%
\pgfsetstrokecolor{currentstroke}%
\pgfsetdash{}{0pt}%
\pgfsys@defobject{currentmarker}{\pgfqpoint{-0.048611in}{0.000000in}}{\pgfqpoint{0.000000in}{0.000000in}}{%
\pgfpathmoveto{\pgfqpoint{0.000000in}{0.000000in}}%
\pgfpathlineto{\pgfqpoint{-0.048611in}{0.000000in}}%
\pgfusepath{stroke,fill}%
}%
\begin{pgfscope}%
\pgfsys@transformshift{0.648703in}{1.958403in}%
\pgfsys@useobject{currentmarker}{}%
\end{pgfscope}%
\end{pgfscope}%
\begin{pgfscope}%
\definecolor{textcolor}{rgb}{0.000000,0.000000,0.000000}%
\pgfsetstrokecolor{textcolor}%
\pgfsetfillcolor{textcolor}%
\pgftext[x=0.343147in, y=1.910208in, left, base]{\color{textcolor}\sffamily\fontsize{10.000000}{12.000000}\selectfont \(\displaystyle {300}\)}%
\end{pgfscope}%
\begin{pgfscope}%
\pgfsetbuttcap%
\pgfsetroundjoin%
\definecolor{currentfill}{rgb}{0.000000,0.000000,0.000000}%
\pgfsetfillcolor{currentfill}%
\pgfsetlinewidth{0.803000pt}%
\definecolor{currentstroke}{rgb}{0.000000,0.000000,0.000000}%
\pgfsetstrokecolor{currentstroke}%
\pgfsetdash{}{0pt}%
\pgfsys@defobject{currentmarker}{\pgfqpoint{-0.048611in}{0.000000in}}{\pgfqpoint{0.000000in}{0.000000in}}{%
\pgfpathmoveto{\pgfqpoint{0.000000in}{0.000000in}}%
\pgfpathlineto{\pgfqpoint{-0.048611in}{0.000000in}}%
\pgfusepath{stroke,fill}%
}%
\begin{pgfscope}%
\pgfsys@transformshift{0.648703in}{2.381272in}%
\pgfsys@useobject{currentmarker}{}%
\end{pgfscope}%
\end{pgfscope}%
\begin{pgfscope}%
\definecolor{textcolor}{rgb}{0.000000,0.000000,0.000000}%
\pgfsetstrokecolor{textcolor}%
\pgfsetfillcolor{textcolor}%
\pgftext[x=0.343147in, y=2.333077in, left, base]{\color{textcolor}\sffamily\fontsize{10.000000}{12.000000}\selectfont \(\displaystyle {400}\)}%
\end{pgfscope}%
\begin{pgfscope}%
\pgfsetbuttcap%
\pgfsetroundjoin%
\definecolor{currentfill}{rgb}{0.000000,0.000000,0.000000}%
\pgfsetfillcolor{currentfill}%
\pgfsetlinewidth{0.803000pt}%
\definecolor{currentstroke}{rgb}{0.000000,0.000000,0.000000}%
\pgfsetstrokecolor{currentstroke}%
\pgfsetdash{}{0pt}%
\pgfsys@defobject{currentmarker}{\pgfqpoint{-0.048611in}{0.000000in}}{\pgfqpoint{0.000000in}{0.000000in}}{%
\pgfpathmoveto{\pgfqpoint{0.000000in}{0.000000in}}%
\pgfpathlineto{\pgfqpoint{-0.048611in}{0.000000in}}%
\pgfusepath{stroke,fill}%
}%
\begin{pgfscope}%
\pgfsys@transformshift{0.648703in}{2.804141in}%
\pgfsys@useobject{currentmarker}{}%
\end{pgfscope}%
\end{pgfscope}%
\begin{pgfscope}%
\definecolor{textcolor}{rgb}{0.000000,0.000000,0.000000}%
\pgfsetstrokecolor{textcolor}%
\pgfsetfillcolor{textcolor}%
\pgftext[x=0.343147in, y=2.755946in, left, base]{\color{textcolor}\sffamily\fontsize{10.000000}{12.000000}\selectfont \(\displaystyle {500}\)}%
\end{pgfscope}%
\begin{pgfscope}%
\pgfsetbuttcap%
\pgfsetroundjoin%
\definecolor{currentfill}{rgb}{0.000000,0.000000,0.000000}%
\pgfsetfillcolor{currentfill}%
\pgfsetlinewidth{0.803000pt}%
\definecolor{currentstroke}{rgb}{0.000000,0.000000,0.000000}%
\pgfsetstrokecolor{currentstroke}%
\pgfsetdash{}{0pt}%
\pgfsys@defobject{currentmarker}{\pgfqpoint{-0.048611in}{0.000000in}}{\pgfqpoint{0.000000in}{0.000000in}}{%
\pgfpathmoveto{\pgfqpoint{0.000000in}{0.000000in}}%
\pgfpathlineto{\pgfqpoint{-0.048611in}{0.000000in}}%
\pgfusepath{stroke,fill}%
}%
\begin{pgfscope}%
\pgfsys@transformshift{0.648703in}{3.227010in}%
\pgfsys@useobject{currentmarker}{}%
\end{pgfscope}%
\end{pgfscope}%
\begin{pgfscope}%
\definecolor{textcolor}{rgb}{0.000000,0.000000,0.000000}%
\pgfsetstrokecolor{textcolor}%
\pgfsetfillcolor{textcolor}%
\pgftext[x=0.343147in, y=3.178815in, left, base]{\color{textcolor}\sffamily\fontsize{10.000000}{12.000000}\selectfont \(\displaystyle {600}\)}%
\end{pgfscope}%
\begin{pgfscope}%
\pgfsetbuttcap%
\pgfsetroundjoin%
\definecolor{currentfill}{rgb}{0.000000,0.000000,0.000000}%
\pgfsetfillcolor{currentfill}%
\pgfsetlinewidth{0.803000pt}%
\definecolor{currentstroke}{rgb}{0.000000,0.000000,0.000000}%
\pgfsetstrokecolor{currentstroke}%
\pgfsetdash{}{0pt}%
\pgfsys@defobject{currentmarker}{\pgfqpoint{-0.048611in}{0.000000in}}{\pgfqpoint{0.000000in}{0.000000in}}{%
\pgfpathmoveto{\pgfqpoint{0.000000in}{0.000000in}}%
\pgfpathlineto{\pgfqpoint{-0.048611in}{0.000000in}}%
\pgfusepath{stroke,fill}%
}%
\begin{pgfscope}%
\pgfsys@transformshift{0.648703in}{3.649879in}%
\pgfsys@useobject{currentmarker}{}%
\end{pgfscope}%
\end{pgfscope}%
\begin{pgfscope}%
\definecolor{textcolor}{rgb}{0.000000,0.000000,0.000000}%
\pgfsetstrokecolor{textcolor}%
\pgfsetfillcolor{textcolor}%
\pgftext[x=0.343147in, y=3.601684in, left, base]{\color{textcolor}\sffamily\fontsize{10.000000}{12.000000}\selectfont \(\displaystyle {700}\)}%
\end{pgfscope}%
\begin{pgfscope}%
\definecolor{textcolor}{rgb}{0.000000,0.000000,0.000000}%
\pgfsetstrokecolor{textcolor}%
\pgfsetfillcolor{textcolor}%
\pgftext[x=0.287592in,y=2.100064in,,bottom,rotate=90.000000]{\color{textcolor}\sffamily\fontsize{10.000000}{12.000000}\selectfont Data Flow Time (s)}%
\end{pgfscope}%
\begin{pgfscope}%
\pgfsetrectcap%
\pgfsetmiterjoin%
\pgfsetlinewidth{0.803000pt}%
\definecolor{currentstroke}{rgb}{0.000000,0.000000,0.000000}%
\pgfsetstrokecolor{currentstroke}%
\pgfsetdash{}{0pt}%
\pgfpathmoveto{\pgfqpoint{0.648703in}{0.548769in}}%
\pgfpathlineto{\pgfqpoint{0.648703in}{3.651359in}}%
\pgfusepath{stroke}%
\end{pgfscope}%
\begin{pgfscope}%
\pgfsetrectcap%
\pgfsetmiterjoin%
\pgfsetlinewidth{0.803000pt}%
\definecolor{currentstroke}{rgb}{0.000000,0.000000,0.000000}%
\pgfsetstrokecolor{currentstroke}%
\pgfsetdash{}{0pt}%
\pgfpathmoveto{\pgfqpoint{5.850000in}{0.548769in}}%
\pgfpathlineto{\pgfqpoint{5.850000in}{3.651359in}}%
\pgfusepath{stroke}%
\end{pgfscope}%
\begin{pgfscope}%
\pgfsetrectcap%
\pgfsetmiterjoin%
\pgfsetlinewidth{0.803000pt}%
\definecolor{currentstroke}{rgb}{0.000000,0.000000,0.000000}%
\pgfsetstrokecolor{currentstroke}%
\pgfsetdash{}{0pt}%
\pgfpathmoveto{\pgfqpoint{0.648703in}{0.548769in}}%
\pgfpathlineto{\pgfqpoint{5.850000in}{0.548769in}}%
\pgfusepath{stroke}%
\end{pgfscope}%
\begin{pgfscope}%
\pgfsetrectcap%
\pgfsetmiterjoin%
\pgfsetlinewidth{0.803000pt}%
\definecolor{currentstroke}{rgb}{0.000000,0.000000,0.000000}%
\pgfsetstrokecolor{currentstroke}%
\pgfsetdash{}{0pt}%
\pgfpathmoveto{\pgfqpoint{0.648703in}{3.651359in}}%
\pgfpathlineto{\pgfqpoint{5.850000in}{3.651359in}}%
\pgfusepath{stroke}%
\end{pgfscope}%
\begin{pgfscope}%
\definecolor{textcolor}{rgb}{0.000000,0.000000,0.000000}%
\pgfsetstrokecolor{textcolor}%
\pgfsetfillcolor{textcolor}%
\pgftext[x=3.249352in,y=3.734692in,,base]{\color{textcolor}\sffamily\fontsize{12.000000}{14.400000}\selectfont Backward}%
\end{pgfscope}%
\begin{pgfscope}%
\pgfsetbuttcap%
\pgfsetmiterjoin%
\definecolor{currentfill}{rgb}{1.000000,1.000000,1.000000}%
\pgfsetfillcolor{currentfill}%
\pgfsetfillopacity{0.800000}%
\pgfsetlinewidth{1.003750pt}%
\definecolor{currentstroke}{rgb}{0.800000,0.800000,0.800000}%
\pgfsetstrokecolor{currentstroke}%
\pgfsetstrokeopacity{0.800000}%
\pgfsetdash{}{0pt}%
\pgfpathmoveto{\pgfqpoint{4.300417in}{1.788050in}}%
\pgfpathlineto{\pgfqpoint{5.752778in}{1.788050in}}%
\pgfpathquadraticcurveto{\pgfqpoint{5.780556in}{1.788050in}}{\pgfqpoint{5.780556in}{1.815828in}}%
\pgfpathlineto{\pgfqpoint{5.780556in}{2.384300in}}%
\pgfpathquadraticcurveto{\pgfqpoint{5.780556in}{2.412078in}}{\pgfqpoint{5.752778in}{2.412078in}}%
\pgfpathlineto{\pgfqpoint{4.300417in}{2.412078in}}%
\pgfpathquadraticcurveto{\pgfqpoint{4.272639in}{2.412078in}}{\pgfqpoint{4.272639in}{2.384300in}}%
\pgfpathlineto{\pgfqpoint{4.272639in}{1.815828in}}%
\pgfpathquadraticcurveto{\pgfqpoint{4.272639in}{1.788050in}}{\pgfqpoint{4.300417in}{1.788050in}}%
\pgfpathclose%
\pgfusepath{stroke,fill}%
\end{pgfscope}%
\begin{pgfscope}%
\pgfsetbuttcap%
\pgfsetroundjoin%
\definecolor{currentfill}{rgb}{0.121569,0.466667,0.705882}%
\pgfsetfillcolor{currentfill}%
\pgfsetlinewidth{1.003750pt}%
\definecolor{currentstroke}{rgb}{0.121569,0.466667,0.705882}%
\pgfsetstrokecolor{currentstroke}%
\pgfsetdash{}{0pt}%
\pgfsys@defobject{currentmarker}{\pgfqpoint{-0.034722in}{-0.034722in}}{\pgfqpoint{0.034722in}{0.034722in}}{%
\pgfpathmoveto{\pgfqpoint{0.000000in}{-0.034722in}}%
\pgfpathcurveto{\pgfqpoint{0.009208in}{-0.034722in}}{\pgfqpoint{0.018041in}{-0.031064in}}{\pgfqpoint{0.024552in}{-0.024552in}}%
\pgfpathcurveto{\pgfqpoint{0.031064in}{-0.018041in}}{\pgfqpoint{0.034722in}{-0.009208in}}{\pgfqpoint{0.034722in}{0.000000in}}%
\pgfpathcurveto{\pgfqpoint{0.034722in}{0.009208in}}{\pgfqpoint{0.031064in}{0.018041in}}{\pgfqpoint{0.024552in}{0.024552in}}%
\pgfpathcurveto{\pgfqpoint{0.018041in}{0.031064in}}{\pgfqpoint{0.009208in}{0.034722in}}{\pgfqpoint{0.000000in}{0.034722in}}%
\pgfpathcurveto{\pgfqpoint{-0.009208in}{0.034722in}}{\pgfqpoint{-0.018041in}{0.031064in}}{\pgfqpoint{-0.024552in}{0.024552in}}%
\pgfpathcurveto{\pgfqpoint{-0.031064in}{0.018041in}}{\pgfqpoint{-0.034722in}{0.009208in}}{\pgfqpoint{-0.034722in}{0.000000in}}%
\pgfpathcurveto{\pgfqpoint{-0.034722in}{-0.009208in}}{\pgfqpoint{-0.031064in}{-0.018041in}}{\pgfqpoint{-0.024552in}{-0.024552in}}%
\pgfpathcurveto{\pgfqpoint{-0.018041in}{-0.031064in}}{\pgfqpoint{-0.009208in}{-0.034722in}}{\pgfqpoint{0.000000in}{-0.034722in}}%
\pgfpathclose%
\pgfusepath{stroke,fill}%
}%
\begin{pgfscope}%
\pgfsys@transformshift{4.467083in}{2.307911in}%
\pgfsys@useobject{currentmarker}{}%
\end{pgfscope}%
\end{pgfscope}%
\begin{pgfscope}%
\definecolor{textcolor}{rgb}{0.000000,0.000000,0.000000}%
\pgfsetstrokecolor{textcolor}%
\pgfsetfillcolor{textcolor}%
\pgftext[x=4.717083in,y=2.259300in,left,base]{\color{textcolor}\sffamily\fontsize{10.000000}{12.000000}\selectfont No Timeout}%
\end{pgfscope}%
\begin{pgfscope}%
\pgfsetbuttcap%
\pgfsetroundjoin%
\definecolor{currentfill}{rgb}{1.000000,0.498039,0.054902}%
\pgfsetfillcolor{currentfill}%
\pgfsetlinewidth{1.003750pt}%
\definecolor{currentstroke}{rgb}{1.000000,0.498039,0.054902}%
\pgfsetstrokecolor{currentstroke}%
\pgfsetdash{}{0pt}%
\pgfsys@defobject{currentmarker}{\pgfqpoint{-0.034722in}{-0.034722in}}{\pgfqpoint{0.034722in}{0.034722in}}{%
\pgfpathmoveto{\pgfqpoint{0.000000in}{-0.034722in}}%
\pgfpathcurveto{\pgfqpoint{0.009208in}{-0.034722in}}{\pgfqpoint{0.018041in}{-0.031064in}}{\pgfqpoint{0.024552in}{-0.024552in}}%
\pgfpathcurveto{\pgfqpoint{0.031064in}{-0.018041in}}{\pgfqpoint{0.034722in}{-0.009208in}}{\pgfqpoint{0.034722in}{0.000000in}}%
\pgfpathcurveto{\pgfqpoint{0.034722in}{0.009208in}}{\pgfqpoint{0.031064in}{0.018041in}}{\pgfqpoint{0.024552in}{0.024552in}}%
\pgfpathcurveto{\pgfqpoint{0.018041in}{0.031064in}}{\pgfqpoint{0.009208in}{0.034722in}}{\pgfqpoint{0.000000in}{0.034722in}}%
\pgfpathcurveto{\pgfqpoint{-0.009208in}{0.034722in}}{\pgfqpoint{-0.018041in}{0.031064in}}{\pgfqpoint{-0.024552in}{0.024552in}}%
\pgfpathcurveto{\pgfqpoint{-0.031064in}{0.018041in}}{\pgfqpoint{-0.034722in}{0.009208in}}{\pgfqpoint{-0.034722in}{0.000000in}}%
\pgfpathcurveto{\pgfqpoint{-0.034722in}{-0.009208in}}{\pgfqpoint{-0.031064in}{-0.018041in}}{\pgfqpoint{-0.024552in}{-0.024552in}}%
\pgfpathcurveto{\pgfqpoint{-0.018041in}{-0.031064in}}{\pgfqpoint{-0.009208in}{-0.034722in}}{\pgfqpoint{0.000000in}{-0.034722in}}%
\pgfpathclose%
\pgfusepath{stroke,fill}%
}%
\begin{pgfscope}%
\pgfsys@transformshift{4.467083in}{2.114300in}%
\pgfsys@useobject{currentmarker}{}%
\end{pgfscope}%
\end{pgfscope}%
\begin{pgfscope}%
\definecolor{textcolor}{rgb}{0.000000,0.000000,0.000000}%
\pgfsetstrokecolor{textcolor}%
\pgfsetfillcolor{textcolor}%
\pgftext[x=4.717083in,y=2.065689in,left,base]{\color{textcolor}\sffamily\fontsize{10.000000}{12.000000}\selectfont Time Timeout}%
\end{pgfscope}%
\begin{pgfscope}%
\pgfsetbuttcap%
\pgfsetroundjoin%
\definecolor{currentfill}{rgb}{0.839216,0.152941,0.156863}%
\pgfsetfillcolor{currentfill}%
\pgfsetlinewidth{1.003750pt}%
\definecolor{currentstroke}{rgb}{0.839216,0.152941,0.156863}%
\pgfsetstrokecolor{currentstroke}%
\pgfsetdash{}{0pt}%
\pgfsys@defobject{currentmarker}{\pgfqpoint{-0.034722in}{-0.034722in}}{\pgfqpoint{0.034722in}{0.034722in}}{%
\pgfpathmoveto{\pgfqpoint{0.000000in}{-0.034722in}}%
\pgfpathcurveto{\pgfqpoint{0.009208in}{-0.034722in}}{\pgfqpoint{0.018041in}{-0.031064in}}{\pgfqpoint{0.024552in}{-0.024552in}}%
\pgfpathcurveto{\pgfqpoint{0.031064in}{-0.018041in}}{\pgfqpoint{0.034722in}{-0.009208in}}{\pgfqpoint{0.034722in}{0.000000in}}%
\pgfpathcurveto{\pgfqpoint{0.034722in}{0.009208in}}{\pgfqpoint{0.031064in}{0.018041in}}{\pgfqpoint{0.024552in}{0.024552in}}%
\pgfpathcurveto{\pgfqpoint{0.018041in}{0.031064in}}{\pgfqpoint{0.009208in}{0.034722in}}{\pgfqpoint{0.000000in}{0.034722in}}%
\pgfpathcurveto{\pgfqpoint{-0.009208in}{0.034722in}}{\pgfqpoint{-0.018041in}{0.031064in}}{\pgfqpoint{-0.024552in}{0.024552in}}%
\pgfpathcurveto{\pgfqpoint{-0.031064in}{0.018041in}}{\pgfqpoint{-0.034722in}{0.009208in}}{\pgfqpoint{-0.034722in}{0.000000in}}%
\pgfpathcurveto{\pgfqpoint{-0.034722in}{-0.009208in}}{\pgfqpoint{-0.031064in}{-0.018041in}}{\pgfqpoint{-0.024552in}{-0.024552in}}%
\pgfpathcurveto{\pgfqpoint{-0.018041in}{-0.031064in}}{\pgfqpoint{-0.009208in}{-0.034722in}}{\pgfqpoint{0.000000in}{-0.034722in}}%
\pgfpathclose%
\pgfusepath{stroke,fill}%
}%
\begin{pgfscope}%
\pgfsys@transformshift{4.467083in}{1.920689in}%
\pgfsys@useobject{currentmarker}{}%
\end{pgfscope}%
\end{pgfscope}%
\begin{pgfscope}%
\definecolor{textcolor}{rgb}{0.000000,0.000000,0.000000}%
\pgfsetstrokecolor{textcolor}%
\pgfsetfillcolor{textcolor}%
\pgftext[x=4.717083in,y=1.872078in,left,base]{\color{textcolor}\sffamily\fontsize{10.000000}{12.000000}\selectfont Memory Timeout}%
\end{pgfscope}%
\end{pgfpicture}%
\makeatother%
\endgroup%

                }
            \end{subfigure}
            \caption{Classes}
        \end{subfigure}
        \caption{Data Flow Time in Comparison to Code Size}
        \label{f:dftocodesize}
    \end{figure}

    Because the above mentioned parameters do not influence the runtime, we did further investigate to find a parameter to decide the favorable direction.
    First, we looked the methods containing sources and sinks.
    We counted the number of statements of the method, call statements and callers of the method and compared these numbers between sources and sinks.
    A advantage in those did not result in a faster analysis.
    Next, we implemented a fast intraprocedural taint analysis. It omits access paths and aliasing.
    Method calls are overapproximated in a similar fashion to the \code{EasyTaintWrapper}.
    We then counted the taints flowing into the callees and callers.
    Also, we did count the number of taints in the method.
    The drawback is that this only works when the state explosion happens inside the first method and this is not the case in the app set.
    Again, we could not find any resilient correlation.
    At this point, we run out of easily computable facts about an app that could correlate with the runtime and decided to leave the question up for future work.

    Finally, we compare the number of edges in the exploded supergraph, referred to as taint propagations in \autoref{s:complexity}.
    Note that the edges in the exploded supergraph are only known after the analysis, making them useless for predictions.
    In \autoref{f:dfedgestotal} we plot the edge count on the x-axis to the data flow time on the y-axis.
    In both graphs, we observe a linear correlation for the apps with a runtime below 500 seconds.
    Then there is a structural break and after that the apps time out.
    Because a linear regression does not really fit well for our diverse data set, we decided to fit a function using the least squares method. We achieved a $r^2$ measure of greater than $0.9$ for four degree polynomials and above.
    However, the good fitting curve seems overfitted to us because the apps not being close to timeouts have a good fitting linear correlation.
    When we look at the point where the structural break happens, we notice that backward the timeouts start after roughly $3 \cdot 10^7$ propagations.
    Forward on the other hand only gets to around $2 \cdot 10^7$ edge propagations before reaching a timeout.
    Such a large difference is unintuitive because the computation cost should not be much different.
    We split the edges up by IFDS problems in \autoref{f:dfedgesi} and \autoref{f:dfedgesa}.
    Interestingly, all curves have a similar steep curve with the exception of the backward alias analysis being more shallow.
    This gives a possible explanation linked to the ratio of infoflow and alias edges.
    The alias flow functions are way simpler and thus, should also cost less to compute.
    The backward analysis has a ratio biased toward the alias edges which could explain the higher edge count possible in ten minutes.
    Why the structural break happens could not be conclusively clarified in this work, so it is also hard to finally reason whether this also holds without such a structural break.

    \begin{figure}[tbp]
        \begin{subfigure}[b]{\textwidth}
            \centering
            \begin{subfigure}[]{0.45\textwidth}
                \centering
                \resizebox{\columnwidth}{!}{
                    %% Creator: Matplotlib, PGF backend
%%
%% To include the figure in your LaTeX document, write
%%   \input{<filename>.pgf}
%%
%% Make sure the required packages are loaded in your preamble
%%   \usepackage{pgf}
%%
%% and, on pdftex
%%   \usepackage[utf8]{inputenc}\DeclareUnicodeCharacter{2212}{-}
%%
%% or, on luatex and xetex
%%   \usepackage{unicode-math}
%%
%% Figures using additional raster images can only be included by \input if
%% they are in the same directory as the main LaTeX file. For loading figures
%% from other directories you can use the `import` package
%%   \usepackage{import}
%%
%% and then include the figures with
%%   \import{<path to file>}{<filename>.pgf}
%%
%% Matplotlib used the following preamble
%%   \usepackage{amsmath}
%%   \usepackage{fontspec}
%%
\begingroup%
\makeatletter%
\begin{pgfpicture}%
\pgfpathrectangle{\pgfpointorigin}{\pgfqpoint{6.000000in}{4.000000in}}%
\pgfusepath{use as bounding box, clip}%
\begin{pgfscope}%
\pgfsetbuttcap%
\pgfsetmiterjoin%
\definecolor{currentfill}{rgb}{1.000000,1.000000,1.000000}%
\pgfsetfillcolor{currentfill}%
\pgfsetlinewidth{0.000000pt}%
\definecolor{currentstroke}{rgb}{1.000000,1.000000,1.000000}%
\pgfsetstrokecolor{currentstroke}%
\pgfsetdash{}{0pt}%
\pgfpathmoveto{\pgfqpoint{0.000000in}{0.000000in}}%
\pgfpathlineto{\pgfqpoint{6.000000in}{0.000000in}}%
\pgfpathlineto{\pgfqpoint{6.000000in}{4.000000in}}%
\pgfpathlineto{\pgfqpoint{0.000000in}{4.000000in}}%
\pgfpathclose%
\pgfusepath{fill}%
\end{pgfscope}%
\begin{pgfscope}%
\pgfsetbuttcap%
\pgfsetmiterjoin%
\definecolor{currentfill}{rgb}{1.000000,1.000000,1.000000}%
\pgfsetfillcolor{currentfill}%
\pgfsetlinewidth{0.000000pt}%
\definecolor{currentstroke}{rgb}{0.000000,0.000000,0.000000}%
\pgfsetstrokecolor{currentstroke}%
\pgfsetstrokeopacity{0.000000}%
\pgfsetdash{}{0pt}%
\pgfpathmoveto{\pgfqpoint{0.648703in}{0.548769in}}%
\pgfpathlineto{\pgfqpoint{5.761597in}{0.548769in}}%
\pgfpathlineto{\pgfqpoint{5.761597in}{3.651359in}}%
\pgfpathlineto{\pgfqpoint{0.648703in}{3.651359in}}%
\pgfpathclose%
\pgfusepath{fill}%
\end{pgfscope}%
\begin{pgfscope}%
\pgfpathrectangle{\pgfqpoint{0.648703in}{0.548769in}}{\pgfqpoint{5.112893in}{3.102590in}}%
\pgfusepath{clip}%
\pgfsetbuttcap%
\pgfsetroundjoin%
\definecolor{currentfill}{rgb}{0.121569,0.466667,0.705882}%
\pgfsetfillcolor{currentfill}%
\pgfsetlinewidth{1.003750pt}%
\definecolor{currentstroke}{rgb}{0.121569,0.466667,0.705882}%
\pgfsetstrokecolor{currentstroke}%
\pgfsetdash{}{0pt}%
\pgfpathmoveto{\pgfqpoint{0.873194in}{0.648129in}}%
\pgfpathcurveto{\pgfqpoint{0.884245in}{0.648129in}}{\pgfqpoint{0.894844in}{0.652519in}}{\pgfqpoint{0.902657in}{0.660333in}}%
\pgfpathcurveto{\pgfqpoint{0.910471in}{0.668146in}}{\pgfqpoint{0.914861in}{0.678745in}}{\pgfqpoint{0.914861in}{0.689796in}}%
\pgfpathcurveto{\pgfqpoint{0.914861in}{0.700846in}}{\pgfqpoint{0.910471in}{0.711445in}}{\pgfqpoint{0.902657in}{0.719258in}}%
\pgfpathcurveto{\pgfqpoint{0.894844in}{0.727072in}}{\pgfqpoint{0.884245in}{0.731462in}}{\pgfqpoint{0.873194in}{0.731462in}}%
\pgfpathcurveto{\pgfqpoint{0.862144in}{0.731462in}}{\pgfqpoint{0.851545in}{0.727072in}}{\pgfqpoint{0.843732in}{0.719258in}}%
\pgfpathcurveto{\pgfqpoint{0.835918in}{0.711445in}}{\pgfqpoint{0.831528in}{0.700846in}}{\pgfqpoint{0.831528in}{0.689796in}}%
\pgfpathcurveto{\pgfqpoint{0.831528in}{0.678745in}}{\pgfqpoint{0.835918in}{0.668146in}}{\pgfqpoint{0.843732in}{0.660333in}}%
\pgfpathcurveto{\pgfqpoint{0.851545in}{0.652519in}}{\pgfqpoint{0.862144in}{0.648129in}}{\pgfqpoint{0.873194in}{0.648129in}}%
\pgfpathclose%
\pgfusepath{stroke,fill}%
\end{pgfscope}%
\begin{pgfscope}%
\pgfpathrectangle{\pgfqpoint{0.648703in}{0.548769in}}{\pgfqpoint{5.112893in}{3.102590in}}%
\pgfusepath{clip}%
\pgfsetbuttcap%
\pgfsetroundjoin%
\definecolor{currentfill}{rgb}{0.121569,0.466667,0.705882}%
\pgfsetfillcolor{currentfill}%
\pgfsetlinewidth{1.003750pt}%
\definecolor{currentstroke}{rgb}{0.121569,0.466667,0.705882}%
\pgfsetstrokecolor{currentstroke}%
\pgfsetdash{}{0pt}%
\pgfpathmoveto{\pgfqpoint{2.092728in}{3.124394in}}%
\pgfpathcurveto{\pgfqpoint{2.103778in}{3.124394in}}{\pgfqpoint{2.114377in}{3.128784in}}{\pgfqpoint{2.122191in}{3.136598in}}%
\pgfpathcurveto{\pgfqpoint{2.130004in}{3.144411in}}{\pgfqpoint{2.134395in}{3.155010in}}{\pgfqpoint{2.134395in}{3.166060in}}%
\pgfpathcurveto{\pgfqpoint{2.134395in}{3.177111in}}{\pgfqpoint{2.130004in}{3.187710in}}{\pgfqpoint{2.122191in}{3.195523in}}%
\pgfpathcurveto{\pgfqpoint{2.114377in}{3.203337in}}{\pgfqpoint{2.103778in}{3.207727in}}{\pgfqpoint{2.092728in}{3.207727in}}%
\pgfpathcurveto{\pgfqpoint{2.081678in}{3.207727in}}{\pgfqpoint{2.071079in}{3.203337in}}{\pgfqpoint{2.063265in}{3.195523in}}%
\pgfpathcurveto{\pgfqpoint{2.055452in}{3.187710in}}{\pgfqpoint{2.051061in}{3.177111in}}{\pgfqpoint{2.051061in}{3.166060in}}%
\pgfpathcurveto{\pgfqpoint{2.051061in}{3.155010in}}{\pgfqpoint{2.055452in}{3.144411in}}{\pgfqpoint{2.063265in}{3.136598in}}%
\pgfpathcurveto{\pgfqpoint{2.071079in}{3.128784in}}{\pgfqpoint{2.081678in}{3.124394in}}{\pgfqpoint{2.092728in}{3.124394in}}%
\pgfpathclose%
\pgfusepath{stroke,fill}%
\end{pgfscope}%
\begin{pgfscope}%
\pgfpathrectangle{\pgfqpoint{0.648703in}{0.548769in}}{\pgfqpoint{5.112893in}{3.102590in}}%
\pgfusepath{clip}%
\pgfsetbuttcap%
\pgfsetroundjoin%
\definecolor{currentfill}{rgb}{1.000000,0.498039,0.054902}%
\pgfsetfillcolor{currentfill}%
\pgfsetlinewidth{1.003750pt}%
\definecolor{currentstroke}{rgb}{1.000000,0.498039,0.054902}%
\pgfsetstrokecolor{currentstroke}%
\pgfsetdash{}{0pt}%
\pgfpathmoveto{\pgfqpoint{4.118865in}{3.140985in}}%
\pgfpathcurveto{\pgfqpoint{4.129915in}{3.140985in}}{\pgfqpoint{4.140514in}{3.145375in}}{\pgfqpoint{4.148328in}{3.153189in}}%
\pgfpathcurveto{\pgfqpoint{4.156141in}{3.161003in}}{\pgfqpoint{4.160532in}{3.171602in}}{\pgfqpoint{4.160532in}{3.182652in}}%
\pgfpathcurveto{\pgfqpoint{4.160532in}{3.193702in}}{\pgfqpoint{4.156141in}{3.204301in}}{\pgfqpoint{4.148328in}{3.212115in}}%
\pgfpathcurveto{\pgfqpoint{4.140514in}{3.219928in}}{\pgfqpoint{4.129915in}{3.224319in}}{\pgfqpoint{4.118865in}{3.224319in}}%
\pgfpathcurveto{\pgfqpoint{4.107815in}{3.224319in}}{\pgfqpoint{4.097216in}{3.219928in}}{\pgfqpoint{4.089402in}{3.212115in}}%
\pgfpathcurveto{\pgfqpoint{4.081588in}{3.204301in}}{\pgfqpoint{4.077198in}{3.193702in}}{\pgfqpoint{4.077198in}{3.182652in}}%
\pgfpathcurveto{\pgfqpoint{4.077198in}{3.171602in}}{\pgfqpoint{4.081588in}{3.161003in}}{\pgfqpoint{4.089402in}{3.153189in}}%
\pgfpathcurveto{\pgfqpoint{4.097216in}{3.145375in}}{\pgfqpoint{4.107815in}{3.140985in}}{\pgfqpoint{4.118865in}{3.140985in}}%
\pgfpathclose%
\pgfusepath{stroke,fill}%
\end{pgfscope}%
\begin{pgfscope}%
\pgfpathrectangle{\pgfqpoint{0.648703in}{0.548769in}}{\pgfqpoint{5.112893in}{3.102590in}}%
\pgfusepath{clip}%
\pgfsetbuttcap%
\pgfsetroundjoin%
\definecolor{currentfill}{rgb}{0.121569,0.466667,0.705882}%
\pgfsetfillcolor{currentfill}%
\pgfsetlinewidth{1.003750pt}%
\definecolor{currentstroke}{rgb}{0.121569,0.466667,0.705882}%
\pgfsetstrokecolor{currentstroke}%
\pgfsetdash{}{0pt}%
\pgfpathmoveto{\pgfqpoint{3.701793in}{3.132690in}}%
\pgfpathcurveto{\pgfqpoint{3.712843in}{3.132690in}}{\pgfqpoint{3.723442in}{3.137080in}}{\pgfqpoint{3.731255in}{3.144893in}}%
\pgfpathcurveto{\pgfqpoint{3.739069in}{3.152707in}}{\pgfqpoint{3.743459in}{3.163306in}}{\pgfqpoint{3.743459in}{3.174356in}}%
\pgfpathcurveto{\pgfqpoint{3.743459in}{3.185406in}}{\pgfqpoint{3.739069in}{3.196005in}}{\pgfqpoint{3.731255in}{3.203819in}}%
\pgfpathcurveto{\pgfqpoint{3.723442in}{3.211633in}}{\pgfqpoint{3.712843in}{3.216023in}}{\pgfqpoint{3.701793in}{3.216023in}}%
\pgfpathcurveto{\pgfqpoint{3.690743in}{3.216023in}}{\pgfqpoint{3.680144in}{3.211633in}}{\pgfqpoint{3.672330in}{3.203819in}}%
\pgfpathcurveto{\pgfqpoint{3.664516in}{3.196005in}}{\pgfqpoint{3.660126in}{3.185406in}}{\pgfqpoint{3.660126in}{3.174356in}}%
\pgfpathcurveto{\pgfqpoint{3.660126in}{3.163306in}}{\pgfqpoint{3.664516in}{3.152707in}}{\pgfqpoint{3.672330in}{3.144893in}}%
\pgfpathcurveto{\pgfqpoint{3.680144in}{3.137080in}}{\pgfqpoint{3.690743in}{3.132690in}}{\pgfqpoint{3.701793in}{3.132690in}}%
\pgfpathclose%
\pgfusepath{stroke,fill}%
\end{pgfscope}%
\begin{pgfscope}%
\pgfpathrectangle{\pgfqpoint{0.648703in}{0.548769in}}{\pgfqpoint{5.112893in}{3.102590in}}%
\pgfusepath{clip}%
\pgfsetbuttcap%
\pgfsetroundjoin%
\definecolor{currentfill}{rgb}{1.000000,0.498039,0.054902}%
\pgfsetfillcolor{currentfill}%
\pgfsetlinewidth{1.003750pt}%
\definecolor{currentstroke}{rgb}{1.000000,0.498039,0.054902}%
\pgfsetstrokecolor{currentstroke}%
\pgfsetdash{}{0pt}%
\pgfpathmoveto{\pgfqpoint{4.554640in}{3.136837in}}%
\pgfpathcurveto{\pgfqpoint{4.565690in}{3.136837in}}{\pgfqpoint{4.576289in}{3.141228in}}{\pgfqpoint{4.584103in}{3.149041in}}%
\pgfpathcurveto{\pgfqpoint{4.591917in}{3.156855in}}{\pgfqpoint{4.596307in}{3.167454in}}{\pgfqpoint{4.596307in}{3.178504in}}%
\pgfpathcurveto{\pgfqpoint{4.596307in}{3.189554in}}{\pgfqpoint{4.591917in}{3.200153in}}{\pgfqpoint{4.584103in}{3.207967in}}%
\pgfpathcurveto{\pgfqpoint{4.576289in}{3.215780in}}{\pgfqpoint{4.565690in}{3.220171in}}{\pgfqpoint{4.554640in}{3.220171in}}%
\pgfpathcurveto{\pgfqpoint{4.543590in}{3.220171in}}{\pgfqpoint{4.532991in}{3.215780in}}{\pgfqpoint{4.525177in}{3.207967in}}%
\pgfpathcurveto{\pgfqpoint{4.517364in}{3.200153in}}{\pgfqpoint{4.512974in}{3.189554in}}{\pgfqpoint{4.512974in}{3.178504in}}%
\pgfpathcurveto{\pgfqpoint{4.512974in}{3.167454in}}{\pgfqpoint{4.517364in}{3.156855in}}{\pgfqpoint{4.525177in}{3.149041in}}%
\pgfpathcurveto{\pgfqpoint{4.532991in}{3.141228in}}{\pgfqpoint{4.543590in}{3.136837in}}{\pgfqpoint{4.554640in}{3.136837in}}%
\pgfpathclose%
\pgfusepath{stroke,fill}%
\end{pgfscope}%
\begin{pgfscope}%
\pgfpathrectangle{\pgfqpoint{0.648703in}{0.548769in}}{\pgfqpoint{5.112893in}{3.102590in}}%
\pgfusepath{clip}%
\pgfsetbuttcap%
\pgfsetroundjoin%
\definecolor{currentfill}{rgb}{0.121569,0.466667,0.705882}%
\pgfsetfillcolor{currentfill}%
\pgfsetlinewidth{1.003750pt}%
\definecolor{currentstroke}{rgb}{0.121569,0.466667,0.705882}%
\pgfsetstrokecolor{currentstroke}%
\pgfsetdash{}{0pt}%
\pgfpathmoveto{\pgfqpoint{2.905752in}{3.132690in}}%
\pgfpathcurveto{\pgfqpoint{2.916803in}{3.132690in}}{\pgfqpoint{2.927402in}{3.137080in}}{\pgfqpoint{2.935215in}{3.144893in}}%
\pgfpathcurveto{\pgfqpoint{2.943029in}{3.152707in}}{\pgfqpoint{2.947419in}{3.163306in}}{\pgfqpoint{2.947419in}{3.174356in}}%
\pgfpathcurveto{\pgfqpoint{2.947419in}{3.185406in}}{\pgfqpoint{2.943029in}{3.196005in}}{\pgfqpoint{2.935215in}{3.203819in}}%
\pgfpathcurveto{\pgfqpoint{2.927402in}{3.211633in}}{\pgfqpoint{2.916803in}{3.216023in}}{\pgfqpoint{2.905752in}{3.216023in}}%
\pgfpathcurveto{\pgfqpoint{2.894702in}{3.216023in}}{\pgfqpoint{2.884103in}{3.211633in}}{\pgfqpoint{2.876290in}{3.203819in}}%
\pgfpathcurveto{\pgfqpoint{2.868476in}{3.196005in}}{\pgfqpoint{2.864086in}{3.185406in}}{\pgfqpoint{2.864086in}{3.174356in}}%
\pgfpathcurveto{\pgfqpoint{2.864086in}{3.163306in}}{\pgfqpoint{2.868476in}{3.152707in}}{\pgfqpoint{2.876290in}{3.144893in}}%
\pgfpathcurveto{\pgfqpoint{2.884103in}{3.137080in}}{\pgfqpoint{2.894702in}{3.132690in}}{\pgfqpoint{2.905752in}{3.132690in}}%
\pgfpathclose%
\pgfusepath{stroke,fill}%
\end{pgfscope}%
\begin{pgfscope}%
\pgfpathrectangle{\pgfqpoint{0.648703in}{0.548769in}}{\pgfqpoint{5.112893in}{3.102590in}}%
\pgfusepath{clip}%
\pgfsetbuttcap%
\pgfsetroundjoin%
\definecolor{currentfill}{rgb}{0.121569,0.466667,0.705882}%
\pgfsetfillcolor{currentfill}%
\pgfsetlinewidth{1.003750pt}%
\definecolor{currentstroke}{rgb}{0.121569,0.466667,0.705882}%
\pgfsetstrokecolor{currentstroke}%
\pgfsetdash{}{0pt}%
\pgfpathmoveto{\pgfqpoint{3.514491in}{3.128542in}}%
\pgfpathcurveto{\pgfqpoint{3.525541in}{3.128542in}}{\pgfqpoint{3.536140in}{3.132932in}}{\pgfqpoint{3.543954in}{3.140746in}}%
\pgfpathcurveto{\pgfqpoint{3.551767in}{3.148559in}}{\pgfqpoint{3.556158in}{3.159158in}}{\pgfqpoint{3.556158in}{3.170208in}}%
\pgfpathcurveto{\pgfqpoint{3.556158in}{3.181258in}}{\pgfqpoint{3.551767in}{3.191857in}}{\pgfqpoint{3.543954in}{3.199671in}}%
\pgfpathcurveto{\pgfqpoint{3.536140in}{3.207485in}}{\pgfqpoint{3.525541in}{3.211875in}}{\pgfqpoint{3.514491in}{3.211875in}}%
\pgfpathcurveto{\pgfqpoint{3.503441in}{3.211875in}}{\pgfqpoint{3.492842in}{3.207485in}}{\pgfqpoint{3.485028in}{3.199671in}}%
\pgfpathcurveto{\pgfqpoint{3.477215in}{3.191857in}}{\pgfqpoint{3.472824in}{3.181258in}}{\pgfqpoint{3.472824in}{3.170208in}}%
\pgfpathcurveto{\pgfqpoint{3.472824in}{3.159158in}}{\pgfqpoint{3.477215in}{3.148559in}}{\pgfqpoint{3.485028in}{3.140746in}}%
\pgfpathcurveto{\pgfqpoint{3.492842in}{3.132932in}}{\pgfqpoint{3.503441in}{3.128542in}}{\pgfqpoint{3.514491in}{3.128542in}}%
\pgfpathclose%
\pgfusepath{stroke,fill}%
\end{pgfscope}%
\begin{pgfscope}%
\pgfpathrectangle{\pgfqpoint{0.648703in}{0.548769in}}{\pgfqpoint{5.112893in}{3.102590in}}%
\pgfusepath{clip}%
\pgfsetbuttcap%
\pgfsetroundjoin%
\definecolor{currentfill}{rgb}{1.000000,0.498039,0.054902}%
\pgfsetfillcolor{currentfill}%
\pgfsetlinewidth{1.003750pt}%
\definecolor{currentstroke}{rgb}{1.000000,0.498039,0.054902}%
\pgfsetstrokecolor{currentstroke}%
\pgfsetdash{}{0pt}%
\pgfpathmoveto{\pgfqpoint{4.219075in}{3.149281in}}%
\pgfpathcurveto{\pgfqpoint{4.230125in}{3.149281in}}{\pgfqpoint{4.240724in}{3.153671in}}{\pgfqpoint{4.248537in}{3.161485in}}%
\pgfpathcurveto{\pgfqpoint{4.256351in}{3.169298in}}{\pgfqpoint{4.260741in}{3.179897in}}{\pgfqpoint{4.260741in}{3.190948in}}%
\pgfpathcurveto{\pgfqpoint{4.260741in}{3.201998in}}{\pgfqpoint{4.256351in}{3.212597in}}{\pgfqpoint{4.248537in}{3.220410in}}%
\pgfpathcurveto{\pgfqpoint{4.240724in}{3.228224in}}{\pgfqpoint{4.230125in}{3.232614in}}{\pgfqpoint{4.219075in}{3.232614in}}%
\pgfpathcurveto{\pgfqpoint{4.208025in}{3.232614in}}{\pgfqpoint{4.197426in}{3.228224in}}{\pgfqpoint{4.189612in}{3.220410in}}%
\pgfpathcurveto{\pgfqpoint{4.181798in}{3.212597in}}{\pgfqpoint{4.177408in}{3.201998in}}{\pgfqpoint{4.177408in}{3.190948in}}%
\pgfpathcurveto{\pgfqpoint{4.177408in}{3.179897in}}{\pgfqpoint{4.181798in}{3.169298in}}{\pgfqpoint{4.189612in}{3.161485in}}%
\pgfpathcurveto{\pgfqpoint{4.197426in}{3.153671in}}{\pgfqpoint{4.208025in}{3.149281in}}{\pgfqpoint{4.219075in}{3.149281in}}%
\pgfpathclose%
\pgfusepath{stroke,fill}%
\end{pgfscope}%
\begin{pgfscope}%
\pgfpathrectangle{\pgfqpoint{0.648703in}{0.548769in}}{\pgfqpoint{5.112893in}{3.102590in}}%
\pgfusepath{clip}%
\pgfsetbuttcap%
\pgfsetroundjoin%
\definecolor{currentfill}{rgb}{1.000000,0.498039,0.054902}%
\pgfsetfillcolor{currentfill}%
\pgfsetlinewidth{1.003750pt}%
\definecolor{currentstroke}{rgb}{1.000000,0.498039,0.054902}%
\pgfsetstrokecolor{currentstroke}%
\pgfsetdash{}{0pt}%
\pgfpathmoveto{\pgfqpoint{4.178170in}{3.240534in}}%
\pgfpathcurveto{\pgfqpoint{4.189220in}{3.240534in}}{\pgfqpoint{4.199819in}{3.244924in}}{\pgfqpoint{4.207633in}{3.252737in}}%
\pgfpathcurveto{\pgfqpoint{4.215446in}{3.260551in}}{\pgfqpoint{4.219836in}{3.271150in}}{\pgfqpoint{4.219836in}{3.282200in}}%
\pgfpathcurveto{\pgfqpoint{4.219836in}{3.293250in}}{\pgfqpoint{4.215446in}{3.303849in}}{\pgfqpoint{4.207633in}{3.311663in}}%
\pgfpathcurveto{\pgfqpoint{4.199819in}{3.319477in}}{\pgfqpoint{4.189220in}{3.323867in}}{\pgfqpoint{4.178170in}{3.323867in}}%
\pgfpathcurveto{\pgfqpoint{4.167120in}{3.323867in}}{\pgfqpoint{4.156521in}{3.319477in}}{\pgfqpoint{4.148707in}{3.311663in}}%
\pgfpathcurveto{\pgfqpoint{4.140893in}{3.303849in}}{\pgfqpoint{4.136503in}{3.293250in}}{\pgfqpoint{4.136503in}{3.282200in}}%
\pgfpathcurveto{\pgfqpoint{4.136503in}{3.271150in}}{\pgfqpoint{4.140893in}{3.260551in}}{\pgfqpoint{4.148707in}{3.252737in}}%
\pgfpathcurveto{\pgfqpoint{4.156521in}{3.244924in}}{\pgfqpoint{4.167120in}{3.240534in}}{\pgfqpoint{4.178170in}{3.240534in}}%
\pgfpathclose%
\pgfusepath{stroke,fill}%
\end{pgfscope}%
\begin{pgfscope}%
\pgfpathrectangle{\pgfqpoint{0.648703in}{0.548769in}}{\pgfqpoint{5.112893in}{3.102590in}}%
\pgfusepath{clip}%
\pgfsetbuttcap%
\pgfsetroundjoin%
\definecolor{currentfill}{rgb}{0.121569,0.466667,0.705882}%
\pgfsetfillcolor{currentfill}%
\pgfsetlinewidth{1.003750pt}%
\definecolor{currentstroke}{rgb}{0.121569,0.466667,0.705882}%
\pgfsetstrokecolor{currentstroke}%
\pgfsetdash{}{0pt}%
\pgfpathmoveto{\pgfqpoint{0.908456in}{0.664720in}}%
\pgfpathcurveto{\pgfqpoint{0.919506in}{0.664720in}}{\pgfqpoint{0.930105in}{0.669111in}}{\pgfqpoint{0.937919in}{0.676924in}}%
\pgfpathcurveto{\pgfqpoint{0.945732in}{0.684738in}}{\pgfqpoint{0.950123in}{0.695337in}}{\pgfqpoint{0.950123in}{0.706387in}}%
\pgfpathcurveto{\pgfqpoint{0.950123in}{0.717437in}}{\pgfqpoint{0.945732in}{0.728036in}}{\pgfqpoint{0.937919in}{0.735850in}}%
\pgfpathcurveto{\pgfqpoint{0.930105in}{0.743663in}}{\pgfqpoint{0.919506in}{0.748054in}}{\pgfqpoint{0.908456in}{0.748054in}}%
\pgfpathcurveto{\pgfqpoint{0.897406in}{0.748054in}}{\pgfqpoint{0.886807in}{0.743663in}}{\pgfqpoint{0.878993in}{0.735850in}}%
\pgfpathcurveto{\pgfqpoint{0.871179in}{0.728036in}}{\pgfqpoint{0.866789in}{0.717437in}}{\pgfqpoint{0.866789in}{0.706387in}}%
\pgfpathcurveto{\pgfqpoint{0.866789in}{0.695337in}}{\pgfqpoint{0.871179in}{0.684738in}}{\pgfqpoint{0.878993in}{0.676924in}}%
\pgfpathcurveto{\pgfqpoint{0.886807in}{0.669111in}}{\pgfqpoint{0.897406in}{0.664720in}}{\pgfqpoint{0.908456in}{0.664720in}}%
\pgfpathclose%
\pgfusepath{stroke,fill}%
\end{pgfscope}%
\begin{pgfscope}%
\pgfpathrectangle{\pgfqpoint{0.648703in}{0.548769in}}{\pgfqpoint{5.112893in}{3.102590in}}%
\pgfusepath{clip}%
\pgfsetbuttcap%
\pgfsetroundjoin%
\definecolor{currentfill}{rgb}{1.000000,0.498039,0.054902}%
\pgfsetfillcolor{currentfill}%
\pgfsetlinewidth{1.003750pt}%
\definecolor{currentstroke}{rgb}{1.000000,0.498039,0.054902}%
\pgfsetstrokecolor{currentstroke}%
\pgfsetdash{}{0pt}%
\pgfpathmoveto{\pgfqpoint{4.405148in}{3.157577in}}%
\pgfpathcurveto{\pgfqpoint{4.416198in}{3.157577in}}{\pgfqpoint{4.426797in}{3.161967in}}{\pgfqpoint{4.434611in}{3.169780in}}%
\pgfpathcurveto{\pgfqpoint{4.442424in}{3.177594in}}{\pgfqpoint{4.446815in}{3.188193in}}{\pgfqpoint{4.446815in}{3.199243in}}%
\pgfpathcurveto{\pgfqpoint{4.446815in}{3.210293in}}{\pgfqpoint{4.442424in}{3.220892in}}{\pgfqpoint{4.434611in}{3.228706in}}%
\pgfpathcurveto{\pgfqpoint{4.426797in}{3.236520in}}{\pgfqpoint{4.416198in}{3.240910in}}{\pgfqpoint{4.405148in}{3.240910in}}%
\pgfpathcurveto{\pgfqpoint{4.394098in}{3.240910in}}{\pgfqpoint{4.383499in}{3.236520in}}{\pgfqpoint{4.375685in}{3.228706in}}%
\pgfpathcurveto{\pgfqpoint{4.367871in}{3.220892in}}{\pgfqpoint{4.363481in}{3.210293in}}{\pgfqpoint{4.363481in}{3.199243in}}%
\pgfpathcurveto{\pgfqpoint{4.363481in}{3.188193in}}{\pgfqpoint{4.367871in}{3.177594in}}{\pgfqpoint{4.375685in}{3.169780in}}%
\pgfpathcurveto{\pgfqpoint{4.383499in}{3.161967in}}{\pgfqpoint{4.394098in}{3.157577in}}{\pgfqpoint{4.405148in}{3.157577in}}%
\pgfpathclose%
\pgfusepath{stroke,fill}%
\end{pgfscope}%
\begin{pgfscope}%
\pgfpathrectangle{\pgfqpoint{0.648703in}{0.548769in}}{\pgfqpoint{5.112893in}{3.102590in}}%
\pgfusepath{clip}%
\pgfsetbuttcap%
\pgfsetroundjoin%
\definecolor{currentfill}{rgb}{1.000000,0.498039,0.054902}%
\pgfsetfillcolor{currentfill}%
\pgfsetlinewidth{1.003750pt}%
\definecolor{currentstroke}{rgb}{1.000000,0.498039,0.054902}%
\pgfsetstrokecolor{currentstroke}%
\pgfsetdash{}{0pt}%
\pgfpathmoveto{\pgfqpoint{3.906803in}{3.140985in}}%
\pgfpathcurveto{\pgfqpoint{3.917853in}{3.140985in}}{\pgfqpoint{3.928452in}{3.145375in}}{\pgfqpoint{3.936266in}{3.153189in}}%
\pgfpathcurveto{\pgfqpoint{3.944079in}{3.161003in}}{\pgfqpoint{3.948469in}{3.171602in}}{\pgfqpoint{3.948469in}{3.182652in}}%
\pgfpathcurveto{\pgfqpoint{3.948469in}{3.193702in}}{\pgfqpoint{3.944079in}{3.204301in}}{\pgfqpoint{3.936266in}{3.212115in}}%
\pgfpathcurveto{\pgfqpoint{3.928452in}{3.219928in}}{\pgfqpoint{3.917853in}{3.224319in}}{\pgfqpoint{3.906803in}{3.224319in}}%
\pgfpathcurveto{\pgfqpoint{3.895753in}{3.224319in}}{\pgfqpoint{3.885154in}{3.219928in}}{\pgfqpoint{3.877340in}{3.212115in}}%
\pgfpathcurveto{\pgfqpoint{3.869526in}{3.204301in}}{\pgfqpoint{3.865136in}{3.193702in}}{\pgfqpoint{3.865136in}{3.182652in}}%
\pgfpathcurveto{\pgfqpoint{3.865136in}{3.171602in}}{\pgfqpoint{3.869526in}{3.161003in}}{\pgfqpoint{3.877340in}{3.153189in}}%
\pgfpathcurveto{\pgfqpoint{3.885154in}{3.145375in}}{\pgfqpoint{3.895753in}{3.140985in}}{\pgfqpoint{3.906803in}{3.140985in}}%
\pgfpathclose%
\pgfusepath{stroke,fill}%
\end{pgfscope}%
\begin{pgfscope}%
\pgfpathrectangle{\pgfqpoint{0.648703in}{0.548769in}}{\pgfqpoint{5.112893in}{3.102590in}}%
\pgfusepath{clip}%
\pgfsetbuttcap%
\pgfsetroundjoin%
\definecolor{currentfill}{rgb}{1.000000,0.498039,0.054902}%
\pgfsetfillcolor{currentfill}%
\pgfsetlinewidth{1.003750pt}%
\definecolor{currentstroke}{rgb}{1.000000,0.498039,0.054902}%
\pgfsetstrokecolor{currentstroke}%
\pgfsetdash{}{0pt}%
\pgfpathmoveto{\pgfqpoint{3.926292in}{3.136837in}}%
\pgfpathcurveto{\pgfqpoint{3.937343in}{3.136837in}}{\pgfqpoint{3.947942in}{3.141228in}}{\pgfqpoint{3.955755in}{3.149041in}}%
\pgfpathcurveto{\pgfqpoint{3.963569in}{3.156855in}}{\pgfqpoint{3.967959in}{3.167454in}}{\pgfqpoint{3.967959in}{3.178504in}}%
\pgfpathcurveto{\pgfqpoint{3.967959in}{3.189554in}}{\pgfqpoint{3.963569in}{3.200153in}}{\pgfqpoint{3.955755in}{3.207967in}}%
\pgfpathcurveto{\pgfqpoint{3.947942in}{3.215780in}}{\pgfqpoint{3.937343in}{3.220171in}}{\pgfqpoint{3.926292in}{3.220171in}}%
\pgfpathcurveto{\pgfqpoint{3.915242in}{3.220171in}}{\pgfqpoint{3.904643in}{3.215780in}}{\pgfqpoint{3.896830in}{3.207967in}}%
\pgfpathcurveto{\pgfqpoint{3.889016in}{3.200153in}}{\pgfqpoint{3.884626in}{3.189554in}}{\pgfqpoint{3.884626in}{3.178504in}}%
\pgfpathcurveto{\pgfqpoint{3.884626in}{3.167454in}}{\pgfqpoint{3.889016in}{3.156855in}}{\pgfqpoint{3.896830in}{3.149041in}}%
\pgfpathcurveto{\pgfqpoint{3.904643in}{3.141228in}}{\pgfqpoint{3.915242in}{3.136837in}}{\pgfqpoint{3.926292in}{3.136837in}}%
\pgfpathclose%
\pgfusepath{stroke,fill}%
\end{pgfscope}%
\begin{pgfscope}%
\pgfpathrectangle{\pgfqpoint{0.648703in}{0.548769in}}{\pgfqpoint{5.112893in}{3.102590in}}%
\pgfusepath{clip}%
\pgfsetbuttcap%
\pgfsetroundjoin%
\definecolor{currentfill}{rgb}{1.000000,0.498039,0.054902}%
\pgfsetfillcolor{currentfill}%
\pgfsetlinewidth{1.003750pt}%
\definecolor{currentstroke}{rgb}{1.000000,0.498039,0.054902}%
\pgfsetstrokecolor{currentstroke}%
\pgfsetdash{}{0pt}%
\pgfpathmoveto{\pgfqpoint{4.169426in}{3.136837in}}%
\pgfpathcurveto{\pgfqpoint{4.180476in}{3.136837in}}{\pgfqpoint{4.191075in}{3.141228in}}{\pgfqpoint{4.198889in}{3.149041in}}%
\pgfpathcurveto{\pgfqpoint{4.206702in}{3.156855in}}{\pgfqpoint{4.211093in}{3.167454in}}{\pgfqpoint{4.211093in}{3.178504in}}%
\pgfpathcurveto{\pgfqpoint{4.211093in}{3.189554in}}{\pgfqpoint{4.206702in}{3.200153in}}{\pgfqpoint{4.198889in}{3.207967in}}%
\pgfpathcurveto{\pgfqpoint{4.191075in}{3.215780in}}{\pgfqpoint{4.180476in}{3.220171in}}{\pgfqpoint{4.169426in}{3.220171in}}%
\pgfpathcurveto{\pgfqpoint{4.158376in}{3.220171in}}{\pgfqpoint{4.147777in}{3.215780in}}{\pgfqpoint{4.139963in}{3.207967in}}%
\pgfpathcurveto{\pgfqpoint{4.132149in}{3.200153in}}{\pgfqpoint{4.127759in}{3.189554in}}{\pgfqpoint{4.127759in}{3.178504in}}%
\pgfpathcurveto{\pgfqpoint{4.127759in}{3.167454in}}{\pgfqpoint{4.132149in}{3.156855in}}{\pgfqpoint{4.139963in}{3.149041in}}%
\pgfpathcurveto{\pgfqpoint{4.147777in}{3.141228in}}{\pgfqpoint{4.158376in}{3.136837in}}{\pgfqpoint{4.169426in}{3.136837in}}%
\pgfpathclose%
\pgfusepath{stroke,fill}%
\end{pgfscope}%
\begin{pgfscope}%
\pgfpathrectangle{\pgfqpoint{0.648703in}{0.548769in}}{\pgfqpoint{5.112893in}{3.102590in}}%
\pgfusepath{clip}%
\pgfsetbuttcap%
\pgfsetroundjoin%
\definecolor{currentfill}{rgb}{1.000000,0.498039,0.054902}%
\pgfsetfillcolor{currentfill}%
\pgfsetlinewidth{1.003750pt}%
\definecolor{currentstroke}{rgb}{1.000000,0.498039,0.054902}%
\pgfsetstrokecolor{currentstroke}%
\pgfsetdash{}{0pt}%
\pgfpathmoveto{\pgfqpoint{1.885299in}{3.136837in}}%
\pgfpathcurveto{\pgfqpoint{1.896349in}{3.136837in}}{\pgfqpoint{1.906948in}{3.141228in}}{\pgfqpoint{1.914762in}{3.149041in}}%
\pgfpathcurveto{\pgfqpoint{1.922576in}{3.156855in}}{\pgfqpoint{1.926966in}{3.167454in}}{\pgfqpoint{1.926966in}{3.178504in}}%
\pgfpathcurveto{\pgfqpoint{1.926966in}{3.189554in}}{\pgfqpoint{1.922576in}{3.200153in}}{\pgfqpoint{1.914762in}{3.207967in}}%
\pgfpathcurveto{\pgfqpoint{1.906948in}{3.215780in}}{\pgfqpoint{1.896349in}{3.220171in}}{\pgfqpoint{1.885299in}{3.220171in}}%
\pgfpathcurveto{\pgfqpoint{1.874249in}{3.220171in}}{\pgfqpoint{1.863650in}{3.215780in}}{\pgfqpoint{1.855836in}{3.207967in}}%
\pgfpathcurveto{\pgfqpoint{1.848023in}{3.200153in}}{\pgfqpoint{1.843632in}{3.189554in}}{\pgfqpoint{1.843632in}{3.178504in}}%
\pgfpathcurveto{\pgfqpoint{1.843632in}{3.167454in}}{\pgfqpoint{1.848023in}{3.156855in}}{\pgfqpoint{1.855836in}{3.149041in}}%
\pgfpathcurveto{\pgfqpoint{1.863650in}{3.141228in}}{\pgfqpoint{1.874249in}{3.136837in}}{\pgfqpoint{1.885299in}{3.136837in}}%
\pgfpathclose%
\pgfusepath{stroke,fill}%
\end{pgfscope}%
\begin{pgfscope}%
\pgfpathrectangle{\pgfqpoint{0.648703in}{0.548769in}}{\pgfqpoint{5.112893in}{3.102590in}}%
\pgfusepath{clip}%
\pgfsetbuttcap%
\pgfsetroundjoin%
\definecolor{currentfill}{rgb}{1.000000,0.498039,0.054902}%
\pgfsetfillcolor{currentfill}%
\pgfsetlinewidth{1.003750pt}%
\definecolor{currentstroke}{rgb}{1.000000,0.498039,0.054902}%
\pgfsetstrokecolor{currentstroke}%
\pgfsetdash{}{0pt}%
\pgfpathmoveto{\pgfqpoint{2.792535in}{3.136837in}}%
\pgfpathcurveto{\pgfqpoint{2.803585in}{3.136837in}}{\pgfqpoint{2.814184in}{3.141228in}}{\pgfqpoint{2.821997in}{3.149041in}}%
\pgfpathcurveto{\pgfqpoint{2.829811in}{3.156855in}}{\pgfqpoint{2.834201in}{3.167454in}}{\pgfqpoint{2.834201in}{3.178504in}}%
\pgfpathcurveto{\pgfqpoint{2.834201in}{3.189554in}}{\pgfqpoint{2.829811in}{3.200153in}}{\pgfqpoint{2.821997in}{3.207967in}}%
\pgfpathcurveto{\pgfqpoint{2.814184in}{3.215780in}}{\pgfqpoint{2.803585in}{3.220171in}}{\pgfqpoint{2.792535in}{3.220171in}}%
\pgfpathcurveto{\pgfqpoint{2.781485in}{3.220171in}}{\pgfqpoint{2.770886in}{3.215780in}}{\pgfqpoint{2.763072in}{3.207967in}}%
\pgfpathcurveto{\pgfqpoint{2.755258in}{3.200153in}}{\pgfqpoint{2.750868in}{3.189554in}}{\pgfqpoint{2.750868in}{3.178504in}}%
\pgfpathcurveto{\pgfqpoint{2.750868in}{3.167454in}}{\pgfqpoint{2.755258in}{3.156855in}}{\pgfqpoint{2.763072in}{3.149041in}}%
\pgfpathcurveto{\pgfqpoint{2.770886in}{3.141228in}}{\pgfqpoint{2.781485in}{3.136837in}}{\pgfqpoint{2.792535in}{3.136837in}}%
\pgfpathclose%
\pgfusepath{stroke,fill}%
\end{pgfscope}%
\begin{pgfscope}%
\pgfpathrectangle{\pgfqpoint{0.648703in}{0.548769in}}{\pgfqpoint{5.112893in}{3.102590in}}%
\pgfusepath{clip}%
\pgfsetbuttcap%
\pgfsetroundjoin%
\definecolor{currentfill}{rgb}{1.000000,0.498039,0.054902}%
\pgfsetfillcolor{currentfill}%
\pgfsetlinewidth{1.003750pt}%
\definecolor{currentstroke}{rgb}{1.000000,0.498039,0.054902}%
\pgfsetstrokecolor{currentstroke}%
\pgfsetdash{}{0pt}%
\pgfpathmoveto{\pgfqpoint{4.349133in}{3.136837in}}%
\pgfpathcurveto{\pgfqpoint{4.360183in}{3.136837in}}{\pgfqpoint{4.370782in}{3.141228in}}{\pgfqpoint{4.378596in}{3.149041in}}%
\pgfpathcurveto{\pgfqpoint{4.386410in}{3.156855in}}{\pgfqpoint{4.390800in}{3.167454in}}{\pgfqpoint{4.390800in}{3.178504in}}%
\pgfpathcurveto{\pgfqpoint{4.390800in}{3.189554in}}{\pgfqpoint{4.386410in}{3.200153in}}{\pgfqpoint{4.378596in}{3.207967in}}%
\pgfpathcurveto{\pgfqpoint{4.370782in}{3.215780in}}{\pgfqpoint{4.360183in}{3.220171in}}{\pgfqpoint{4.349133in}{3.220171in}}%
\pgfpathcurveto{\pgfqpoint{4.338083in}{3.220171in}}{\pgfqpoint{4.327484in}{3.215780in}}{\pgfqpoint{4.319671in}{3.207967in}}%
\pgfpathcurveto{\pgfqpoint{4.311857in}{3.200153in}}{\pgfqpoint{4.307467in}{3.189554in}}{\pgfqpoint{4.307467in}{3.178504in}}%
\pgfpathcurveto{\pgfqpoint{4.307467in}{3.167454in}}{\pgfqpoint{4.311857in}{3.156855in}}{\pgfqpoint{4.319671in}{3.149041in}}%
\pgfpathcurveto{\pgfqpoint{4.327484in}{3.141228in}}{\pgfqpoint{4.338083in}{3.136837in}}{\pgfqpoint{4.349133in}{3.136837in}}%
\pgfpathclose%
\pgfusepath{stroke,fill}%
\end{pgfscope}%
\begin{pgfscope}%
\pgfpathrectangle{\pgfqpoint{0.648703in}{0.548769in}}{\pgfqpoint{5.112893in}{3.102590in}}%
\pgfusepath{clip}%
\pgfsetbuttcap%
\pgfsetroundjoin%
\definecolor{currentfill}{rgb}{0.121569,0.466667,0.705882}%
\pgfsetfillcolor{currentfill}%
\pgfsetlinewidth{1.003750pt}%
\definecolor{currentstroke}{rgb}{0.121569,0.466667,0.705882}%
\pgfsetstrokecolor{currentstroke}%
\pgfsetdash{}{0pt}%
\pgfpathmoveto{\pgfqpoint{4.212985in}{2.846488in}}%
\pgfpathcurveto{\pgfqpoint{4.224035in}{2.846488in}}{\pgfqpoint{4.234635in}{2.850878in}}{\pgfqpoint{4.242448in}{2.858692in}}%
\pgfpathcurveto{\pgfqpoint{4.250262in}{2.866506in}}{\pgfqpoint{4.254652in}{2.877105in}}{\pgfqpoint{4.254652in}{2.888155in}}%
\pgfpathcurveto{\pgfqpoint{4.254652in}{2.899205in}}{\pgfqpoint{4.250262in}{2.909804in}}{\pgfqpoint{4.242448in}{2.917617in}}%
\pgfpathcurveto{\pgfqpoint{4.234635in}{2.925431in}}{\pgfqpoint{4.224035in}{2.929821in}}{\pgfqpoint{4.212985in}{2.929821in}}%
\pgfpathcurveto{\pgfqpoint{4.201935in}{2.929821in}}{\pgfqpoint{4.191336in}{2.925431in}}{\pgfqpoint{4.183523in}{2.917617in}}%
\pgfpathcurveto{\pgfqpoint{4.175709in}{2.909804in}}{\pgfqpoint{4.171319in}{2.899205in}}{\pgfqpoint{4.171319in}{2.888155in}}%
\pgfpathcurveto{\pgfqpoint{4.171319in}{2.877105in}}{\pgfqpoint{4.175709in}{2.866506in}}{\pgfqpoint{4.183523in}{2.858692in}}%
\pgfpathcurveto{\pgfqpoint{4.191336in}{2.850878in}}{\pgfqpoint{4.201935in}{2.846488in}}{\pgfqpoint{4.212985in}{2.846488in}}%
\pgfpathclose%
\pgfusepath{stroke,fill}%
\end{pgfscope}%
\begin{pgfscope}%
\pgfpathrectangle{\pgfqpoint{0.648703in}{0.548769in}}{\pgfqpoint{5.112893in}{3.102590in}}%
\pgfusepath{clip}%
\pgfsetbuttcap%
\pgfsetroundjoin%
\definecolor{currentfill}{rgb}{0.121569,0.466667,0.705882}%
\pgfsetfillcolor{currentfill}%
\pgfsetlinewidth{1.003750pt}%
\definecolor{currentstroke}{rgb}{0.121569,0.466667,0.705882}%
\pgfsetstrokecolor{currentstroke}%
\pgfsetdash{}{0pt}%
\pgfpathmoveto{\pgfqpoint{1.500322in}{3.074620in}}%
\pgfpathcurveto{\pgfqpoint{1.511373in}{3.074620in}}{\pgfqpoint{1.521972in}{3.079010in}}{\pgfqpoint{1.529785in}{3.086824in}}%
\pgfpathcurveto{\pgfqpoint{1.537599in}{3.094637in}}{\pgfqpoint{1.541989in}{3.105236in}}{\pgfqpoint{1.541989in}{3.116286in}}%
\pgfpathcurveto{\pgfqpoint{1.541989in}{3.127336in}}{\pgfqpoint{1.537599in}{3.137935in}}{\pgfqpoint{1.529785in}{3.145749in}}%
\pgfpathcurveto{\pgfqpoint{1.521972in}{3.153563in}}{\pgfqpoint{1.511373in}{3.157953in}}{\pgfqpoint{1.500322in}{3.157953in}}%
\pgfpathcurveto{\pgfqpoint{1.489272in}{3.157953in}}{\pgfqpoint{1.478673in}{3.153563in}}{\pgfqpoint{1.470860in}{3.145749in}}%
\pgfpathcurveto{\pgfqpoint{1.463046in}{3.137935in}}{\pgfqpoint{1.458656in}{3.127336in}}{\pgfqpoint{1.458656in}{3.116286in}}%
\pgfpathcurveto{\pgfqpoint{1.458656in}{3.105236in}}{\pgfqpoint{1.463046in}{3.094637in}}{\pgfqpoint{1.470860in}{3.086824in}}%
\pgfpathcurveto{\pgfqpoint{1.478673in}{3.079010in}}{\pgfqpoint{1.489272in}{3.074620in}}{\pgfqpoint{1.500322in}{3.074620in}}%
\pgfpathclose%
\pgfusepath{stroke,fill}%
\end{pgfscope}%
\begin{pgfscope}%
\pgfpathrectangle{\pgfqpoint{0.648703in}{0.548769in}}{\pgfqpoint{5.112893in}{3.102590in}}%
\pgfusepath{clip}%
\pgfsetbuttcap%
\pgfsetroundjoin%
\definecolor{currentfill}{rgb}{1.000000,0.498039,0.054902}%
\pgfsetfillcolor{currentfill}%
\pgfsetlinewidth{1.003750pt}%
\definecolor{currentstroke}{rgb}{1.000000,0.498039,0.054902}%
\pgfsetstrokecolor{currentstroke}%
\pgfsetdash{}{0pt}%
\pgfpathmoveto{\pgfqpoint{4.076071in}{3.315195in}}%
\pgfpathcurveto{\pgfqpoint{4.087121in}{3.315195in}}{\pgfqpoint{4.097720in}{3.319585in}}{\pgfqpoint{4.105534in}{3.327399in}}%
\pgfpathcurveto{\pgfqpoint{4.113347in}{3.335212in}}{\pgfqpoint{4.117737in}{3.345811in}}{\pgfqpoint{4.117737in}{3.356861in}}%
\pgfpathcurveto{\pgfqpoint{4.117737in}{3.367912in}}{\pgfqpoint{4.113347in}{3.378511in}}{\pgfqpoint{4.105534in}{3.386324in}}%
\pgfpathcurveto{\pgfqpoint{4.097720in}{3.394138in}}{\pgfqpoint{4.087121in}{3.398528in}}{\pgfqpoint{4.076071in}{3.398528in}}%
\pgfpathcurveto{\pgfqpoint{4.065021in}{3.398528in}}{\pgfqpoint{4.054422in}{3.394138in}}{\pgfqpoint{4.046608in}{3.386324in}}%
\pgfpathcurveto{\pgfqpoint{4.038794in}{3.378511in}}{\pgfqpoint{4.034404in}{3.367912in}}{\pgfqpoint{4.034404in}{3.356861in}}%
\pgfpathcurveto{\pgfqpoint{4.034404in}{3.345811in}}{\pgfqpoint{4.038794in}{3.335212in}}{\pgfqpoint{4.046608in}{3.327399in}}%
\pgfpathcurveto{\pgfqpoint{4.054422in}{3.319585in}}{\pgfqpoint{4.065021in}{3.315195in}}{\pgfqpoint{4.076071in}{3.315195in}}%
\pgfpathclose%
\pgfusepath{stroke,fill}%
\end{pgfscope}%
\begin{pgfscope}%
\pgfpathrectangle{\pgfqpoint{0.648703in}{0.548769in}}{\pgfqpoint{5.112893in}{3.102590in}}%
\pgfusepath{clip}%
\pgfsetbuttcap%
\pgfsetroundjoin%
\definecolor{currentfill}{rgb}{0.121569,0.466667,0.705882}%
\pgfsetfillcolor{currentfill}%
\pgfsetlinewidth{1.003750pt}%
\definecolor{currentstroke}{rgb}{0.121569,0.466667,0.705882}%
\pgfsetstrokecolor{currentstroke}%
\pgfsetdash{}{0pt}%
\pgfpathmoveto{\pgfqpoint{0.873459in}{0.648129in}}%
\pgfpathcurveto{\pgfqpoint{0.884509in}{0.648129in}}{\pgfqpoint{0.895108in}{0.652519in}}{\pgfqpoint{0.902922in}{0.660333in}}%
\pgfpathcurveto{\pgfqpoint{0.910735in}{0.668146in}}{\pgfqpoint{0.915126in}{0.678745in}}{\pgfqpoint{0.915126in}{0.689796in}}%
\pgfpathcurveto{\pgfqpoint{0.915126in}{0.700846in}}{\pgfqpoint{0.910735in}{0.711445in}}{\pgfqpoint{0.902922in}{0.719258in}}%
\pgfpathcurveto{\pgfqpoint{0.895108in}{0.727072in}}{\pgfqpoint{0.884509in}{0.731462in}}{\pgfqpoint{0.873459in}{0.731462in}}%
\pgfpathcurveto{\pgfqpoint{0.862409in}{0.731462in}}{\pgfqpoint{0.851810in}{0.727072in}}{\pgfqpoint{0.843996in}{0.719258in}}%
\pgfpathcurveto{\pgfqpoint{0.836183in}{0.711445in}}{\pgfqpoint{0.831792in}{0.700846in}}{\pgfqpoint{0.831792in}{0.689796in}}%
\pgfpathcurveto{\pgfqpoint{0.831792in}{0.678745in}}{\pgfqpoint{0.836183in}{0.668146in}}{\pgfqpoint{0.843996in}{0.660333in}}%
\pgfpathcurveto{\pgfqpoint{0.851810in}{0.652519in}}{\pgfqpoint{0.862409in}{0.648129in}}{\pgfqpoint{0.873459in}{0.648129in}}%
\pgfpathclose%
\pgfusepath{stroke,fill}%
\end{pgfscope}%
\begin{pgfscope}%
\pgfpathrectangle{\pgfqpoint{0.648703in}{0.548769in}}{\pgfqpoint{5.112893in}{3.102590in}}%
\pgfusepath{clip}%
\pgfsetbuttcap%
\pgfsetroundjoin%
\definecolor{currentfill}{rgb}{0.121569,0.466667,0.705882}%
\pgfsetfillcolor{currentfill}%
\pgfsetlinewidth{1.003750pt}%
\definecolor{currentstroke}{rgb}{0.121569,0.466667,0.705882}%
\pgfsetstrokecolor{currentstroke}%
\pgfsetdash{}{0pt}%
\pgfpathmoveto{\pgfqpoint{1.968955in}{1.166610in}}%
\pgfpathcurveto{\pgfqpoint{1.980005in}{1.166610in}}{\pgfqpoint{1.990604in}{1.171000in}}{\pgfqpoint{1.998417in}{1.178814in}}%
\pgfpathcurveto{\pgfqpoint{2.006231in}{1.186627in}}{\pgfqpoint{2.010621in}{1.197226in}}{\pgfqpoint{2.010621in}{1.208277in}}%
\pgfpathcurveto{\pgfqpoint{2.010621in}{1.219327in}}{\pgfqpoint{2.006231in}{1.229926in}}{\pgfqpoint{1.998417in}{1.237739in}}%
\pgfpathcurveto{\pgfqpoint{1.990604in}{1.245553in}}{\pgfqpoint{1.980005in}{1.249943in}}{\pgfqpoint{1.968955in}{1.249943in}}%
\pgfpathcurveto{\pgfqpoint{1.957904in}{1.249943in}}{\pgfqpoint{1.947305in}{1.245553in}}{\pgfqpoint{1.939492in}{1.237739in}}%
\pgfpathcurveto{\pgfqpoint{1.931678in}{1.229926in}}{\pgfqpoint{1.927288in}{1.219327in}}{\pgfqpoint{1.927288in}{1.208277in}}%
\pgfpathcurveto{\pgfqpoint{1.927288in}{1.197226in}}{\pgfqpoint{1.931678in}{1.186627in}}{\pgfqpoint{1.939492in}{1.178814in}}%
\pgfpathcurveto{\pgfqpoint{1.947305in}{1.171000in}}{\pgfqpoint{1.957904in}{1.166610in}}{\pgfqpoint{1.968955in}{1.166610in}}%
\pgfpathclose%
\pgfusepath{stroke,fill}%
\end{pgfscope}%
\begin{pgfscope}%
\pgfpathrectangle{\pgfqpoint{0.648703in}{0.548769in}}{\pgfqpoint{5.112893in}{3.102590in}}%
\pgfusepath{clip}%
\pgfsetbuttcap%
\pgfsetroundjoin%
\definecolor{currentfill}{rgb}{1.000000,0.498039,0.054902}%
\pgfsetfillcolor{currentfill}%
\pgfsetlinewidth{1.003750pt}%
\definecolor{currentstroke}{rgb}{1.000000,0.498039,0.054902}%
\pgfsetstrokecolor{currentstroke}%
\pgfsetdash{}{0pt}%
\pgfpathmoveto{\pgfqpoint{4.012639in}{3.136837in}}%
\pgfpathcurveto{\pgfqpoint{4.023689in}{3.136837in}}{\pgfqpoint{4.034288in}{3.141228in}}{\pgfqpoint{4.042101in}{3.149041in}}%
\pgfpathcurveto{\pgfqpoint{4.049915in}{3.156855in}}{\pgfqpoint{4.054305in}{3.167454in}}{\pgfqpoint{4.054305in}{3.178504in}}%
\pgfpathcurveto{\pgfqpoint{4.054305in}{3.189554in}}{\pgfqpoint{4.049915in}{3.200153in}}{\pgfqpoint{4.042101in}{3.207967in}}%
\pgfpathcurveto{\pgfqpoint{4.034288in}{3.215780in}}{\pgfqpoint{4.023689in}{3.220171in}}{\pgfqpoint{4.012639in}{3.220171in}}%
\pgfpathcurveto{\pgfqpoint{4.001588in}{3.220171in}}{\pgfqpoint{3.990989in}{3.215780in}}{\pgfqpoint{3.983176in}{3.207967in}}%
\pgfpathcurveto{\pgfqpoint{3.975362in}{3.200153in}}{\pgfqpoint{3.970972in}{3.189554in}}{\pgfqpoint{3.970972in}{3.178504in}}%
\pgfpathcurveto{\pgfqpoint{3.970972in}{3.167454in}}{\pgfqpoint{3.975362in}{3.156855in}}{\pgfqpoint{3.983176in}{3.149041in}}%
\pgfpathcurveto{\pgfqpoint{3.990989in}{3.141228in}}{\pgfqpoint{4.001588in}{3.136837in}}{\pgfqpoint{4.012639in}{3.136837in}}%
\pgfpathclose%
\pgfusepath{stroke,fill}%
\end{pgfscope}%
\begin{pgfscope}%
\pgfpathrectangle{\pgfqpoint{0.648703in}{0.548769in}}{\pgfqpoint{5.112893in}{3.102590in}}%
\pgfusepath{clip}%
\pgfsetbuttcap%
\pgfsetroundjoin%
\definecolor{currentfill}{rgb}{0.121569,0.466667,0.705882}%
\pgfsetfillcolor{currentfill}%
\pgfsetlinewidth{1.003750pt}%
\definecolor{currentstroke}{rgb}{0.121569,0.466667,0.705882}%
\pgfsetstrokecolor{currentstroke}%
\pgfsetdash{}{0pt}%
\pgfpathmoveto{\pgfqpoint{2.389483in}{3.132690in}}%
\pgfpathcurveto{\pgfqpoint{2.400533in}{3.132690in}}{\pgfqpoint{2.411132in}{3.137080in}}{\pgfqpoint{2.418946in}{3.144893in}}%
\pgfpathcurveto{\pgfqpoint{2.426759in}{3.152707in}}{\pgfqpoint{2.431150in}{3.163306in}}{\pgfqpoint{2.431150in}{3.174356in}}%
\pgfpathcurveto{\pgfqpoint{2.431150in}{3.185406in}}{\pgfqpoint{2.426759in}{3.196005in}}{\pgfqpoint{2.418946in}{3.203819in}}%
\pgfpathcurveto{\pgfqpoint{2.411132in}{3.211633in}}{\pgfqpoint{2.400533in}{3.216023in}}{\pgfqpoint{2.389483in}{3.216023in}}%
\pgfpathcurveto{\pgfqpoint{2.378433in}{3.216023in}}{\pgfqpoint{2.367834in}{3.211633in}}{\pgfqpoint{2.360020in}{3.203819in}}%
\pgfpathcurveto{\pgfqpoint{2.352207in}{3.196005in}}{\pgfqpoint{2.347816in}{3.185406in}}{\pgfqpoint{2.347816in}{3.174356in}}%
\pgfpathcurveto{\pgfqpoint{2.347816in}{3.163306in}}{\pgfqpoint{2.352207in}{3.152707in}}{\pgfqpoint{2.360020in}{3.144893in}}%
\pgfpathcurveto{\pgfqpoint{2.367834in}{3.137080in}}{\pgfqpoint{2.378433in}{3.132690in}}{\pgfqpoint{2.389483in}{3.132690in}}%
\pgfpathclose%
\pgfusepath{stroke,fill}%
\end{pgfscope}%
\begin{pgfscope}%
\pgfpathrectangle{\pgfqpoint{0.648703in}{0.548769in}}{\pgfqpoint{5.112893in}{3.102590in}}%
\pgfusepath{clip}%
\pgfsetbuttcap%
\pgfsetroundjoin%
\definecolor{currentfill}{rgb}{0.839216,0.152941,0.156863}%
\pgfsetfillcolor{currentfill}%
\pgfsetlinewidth{1.003750pt}%
\definecolor{currentstroke}{rgb}{0.839216,0.152941,0.156863}%
\pgfsetstrokecolor{currentstroke}%
\pgfsetdash{}{0pt}%
\pgfpathmoveto{\pgfqpoint{4.732279in}{3.149281in}}%
\pgfpathcurveto{\pgfqpoint{4.743329in}{3.149281in}}{\pgfqpoint{4.753928in}{3.153671in}}{\pgfqpoint{4.761742in}{3.161485in}}%
\pgfpathcurveto{\pgfqpoint{4.769555in}{3.169298in}}{\pgfqpoint{4.773945in}{3.179897in}}{\pgfqpoint{4.773945in}{3.190948in}}%
\pgfpathcurveto{\pgfqpoint{4.773945in}{3.201998in}}{\pgfqpoint{4.769555in}{3.212597in}}{\pgfqpoint{4.761742in}{3.220410in}}%
\pgfpathcurveto{\pgfqpoint{4.753928in}{3.228224in}}{\pgfqpoint{4.743329in}{3.232614in}}{\pgfqpoint{4.732279in}{3.232614in}}%
\pgfpathcurveto{\pgfqpoint{4.721229in}{3.232614in}}{\pgfqpoint{4.710630in}{3.228224in}}{\pgfqpoint{4.702816in}{3.220410in}}%
\pgfpathcurveto{\pgfqpoint{4.695002in}{3.212597in}}{\pgfqpoint{4.690612in}{3.201998in}}{\pgfqpoint{4.690612in}{3.190948in}}%
\pgfpathcurveto{\pgfqpoint{4.690612in}{3.179897in}}{\pgfqpoint{4.695002in}{3.169298in}}{\pgfqpoint{4.702816in}{3.161485in}}%
\pgfpathcurveto{\pgfqpoint{4.710630in}{3.153671in}}{\pgfqpoint{4.721229in}{3.149281in}}{\pgfqpoint{4.732279in}{3.149281in}}%
\pgfpathclose%
\pgfusepath{stroke,fill}%
\end{pgfscope}%
\begin{pgfscope}%
\pgfpathrectangle{\pgfqpoint{0.648703in}{0.548769in}}{\pgfqpoint{5.112893in}{3.102590in}}%
\pgfusepath{clip}%
\pgfsetbuttcap%
\pgfsetroundjoin%
\definecolor{currentfill}{rgb}{1.000000,0.498039,0.054902}%
\pgfsetfillcolor{currentfill}%
\pgfsetlinewidth{1.003750pt}%
\definecolor{currentstroke}{rgb}{1.000000,0.498039,0.054902}%
\pgfsetstrokecolor{currentstroke}%
\pgfsetdash{}{0pt}%
\pgfpathmoveto{\pgfqpoint{4.050596in}{3.174168in}}%
\pgfpathcurveto{\pgfqpoint{4.061646in}{3.174168in}}{\pgfqpoint{4.072245in}{3.178558in}}{\pgfqpoint{4.080059in}{3.186372in}}%
\pgfpathcurveto{\pgfqpoint{4.087873in}{3.194185in}}{\pgfqpoint{4.092263in}{3.204785in}}{\pgfqpoint{4.092263in}{3.215835in}}%
\pgfpathcurveto{\pgfqpoint{4.092263in}{3.226885in}}{\pgfqpoint{4.087873in}{3.237484in}}{\pgfqpoint{4.080059in}{3.245297in}}%
\pgfpathcurveto{\pgfqpoint{4.072245in}{3.253111in}}{\pgfqpoint{4.061646in}{3.257501in}}{\pgfqpoint{4.050596in}{3.257501in}}%
\pgfpathcurveto{\pgfqpoint{4.039546in}{3.257501in}}{\pgfqpoint{4.028947in}{3.253111in}}{\pgfqpoint{4.021133in}{3.245297in}}%
\pgfpathcurveto{\pgfqpoint{4.013320in}{3.237484in}}{\pgfqpoint{4.008929in}{3.226885in}}{\pgfqpoint{4.008929in}{3.215835in}}%
\pgfpathcurveto{\pgfqpoint{4.008929in}{3.204785in}}{\pgfqpoint{4.013320in}{3.194185in}}{\pgfqpoint{4.021133in}{3.186372in}}%
\pgfpathcurveto{\pgfqpoint{4.028947in}{3.178558in}}{\pgfqpoint{4.039546in}{3.174168in}}{\pgfqpoint{4.050596in}{3.174168in}}%
\pgfpathclose%
\pgfusepath{stroke,fill}%
\end{pgfscope}%
\begin{pgfscope}%
\pgfpathrectangle{\pgfqpoint{0.648703in}{0.548769in}}{\pgfqpoint{5.112893in}{3.102590in}}%
\pgfusepath{clip}%
\pgfsetbuttcap%
\pgfsetroundjoin%
\definecolor{currentfill}{rgb}{0.121569,0.466667,0.705882}%
\pgfsetfillcolor{currentfill}%
\pgfsetlinewidth{1.003750pt}%
\definecolor{currentstroke}{rgb}{0.121569,0.466667,0.705882}%
\pgfsetstrokecolor{currentstroke}%
\pgfsetdash{}{0pt}%
\pgfpathmoveto{\pgfqpoint{0.911596in}{0.656425in}}%
\pgfpathcurveto{\pgfqpoint{0.922646in}{0.656425in}}{\pgfqpoint{0.933245in}{0.660815in}}{\pgfqpoint{0.941059in}{0.668629in}}%
\pgfpathcurveto{\pgfqpoint{0.948872in}{0.676442in}}{\pgfqpoint{0.953263in}{0.687041in}}{\pgfqpoint{0.953263in}{0.698091in}}%
\pgfpathcurveto{\pgfqpoint{0.953263in}{0.709141in}}{\pgfqpoint{0.948872in}{0.719740in}}{\pgfqpoint{0.941059in}{0.727554in}}%
\pgfpathcurveto{\pgfqpoint{0.933245in}{0.735368in}}{\pgfqpoint{0.922646in}{0.739758in}}{\pgfqpoint{0.911596in}{0.739758in}}%
\pgfpathcurveto{\pgfqpoint{0.900546in}{0.739758in}}{\pgfqpoint{0.889947in}{0.735368in}}{\pgfqpoint{0.882133in}{0.727554in}}%
\pgfpathcurveto{\pgfqpoint{0.874320in}{0.719740in}}{\pgfqpoint{0.869929in}{0.709141in}}{\pgfqpoint{0.869929in}{0.698091in}}%
\pgfpathcurveto{\pgfqpoint{0.869929in}{0.687041in}}{\pgfqpoint{0.874320in}{0.676442in}}{\pgfqpoint{0.882133in}{0.668629in}}%
\pgfpathcurveto{\pgfqpoint{0.889947in}{0.660815in}}{\pgfqpoint{0.900546in}{0.656425in}}{\pgfqpoint{0.911596in}{0.656425in}}%
\pgfpathclose%
\pgfusepath{stroke,fill}%
\end{pgfscope}%
\begin{pgfscope}%
\pgfpathrectangle{\pgfqpoint{0.648703in}{0.548769in}}{\pgfqpoint{5.112893in}{3.102590in}}%
\pgfusepath{clip}%
\pgfsetbuttcap%
\pgfsetroundjoin%
\definecolor{currentfill}{rgb}{0.121569,0.466667,0.705882}%
\pgfsetfillcolor{currentfill}%
\pgfsetlinewidth{1.003750pt}%
\definecolor{currentstroke}{rgb}{0.121569,0.466667,0.705882}%
\pgfsetstrokecolor{currentstroke}%
\pgfsetdash{}{0pt}%
\pgfpathmoveto{\pgfqpoint{1.909636in}{3.124394in}}%
\pgfpathcurveto{\pgfqpoint{1.920686in}{3.124394in}}{\pgfqpoint{1.931285in}{3.128784in}}{\pgfqpoint{1.939099in}{3.136598in}}%
\pgfpathcurveto{\pgfqpoint{1.946912in}{3.144411in}}{\pgfqpoint{1.951303in}{3.155010in}}{\pgfqpoint{1.951303in}{3.166060in}}%
\pgfpathcurveto{\pgfqpoint{1.951303in}{3.177111in}}{\pgfqpoint{1.946912in}{3.187710in}}{\pgfqpoint{1.939099in}{3.195523in}}%
\pgfpathcurveto{\pgfqpoint{1.931285in}{3.203337in}}{\pgfqpoint{1.920686in}{3.207727in}}{\pgfqpoint{1.909636in}{3.207727in}}%
\pgfpathcurveto{\pgfqpoint{1.898586in}{3.207727in}}{\pgfqpoint{1.887987in}{3.203337in}}{\pgfqpoint{1.880173in}{3.195523in}}%
\pgfpathcurveto{\pgfqpoint{1.872360in}{3.187710in}}{\pgfqpoint{1.867969in}{3.177111in}}{\pgfqpoint{1.867969in}{3.166060in}}%
\pgfpathcurveto{\pgfqpoint{1.867969in}{3.155010in}}{\pgfqpoint{1.872360in}{3.144411in}}{\pgfqpoint{1.880173in}{3.136598in}}%
\pgfpathcurveto{\pgfqpoint{1.887987in}{3.128784in}}{\pgfqpoint{1.898586in}{3.124394in}}{\pgfqpoint{1.909636in}{3.124394in}}%
\pgfpathclose%
\pgfusepath{stroke,fill}%
\end{pgfscope}%
\begin{pgfscope}%
\pgfpathrectangle{\pgfqpoint{0.648703in}{0.548769in}}{\pgfqpoint{5.112893in}{3.102590in}}%
\pgfusepath{clip}%
\pgfsetbuttcap%
\pgfsetroundjoin%
\definecolor{currentfill}{rgb}{0.121569,0.466667,0.705882}%
\pgfsetfillcolor{currentfill}%
\pgfsetlinewidth{1.003750pt}%
\definecolor{currentstroke}{rgb}{0.121569,0.466667,0.705882}%
\pgfsetstrokecolor{currentstroke}%
\pgfsetdash{}{0pt}%
\pgfpathmoveto{\pgfqpoint{0.873476in}{0.648129in}}%
\pgfpathcurveto{\pgfqpoint{0.884526in}{0.648129in}}{\pgfqpoint{0.895125in}{0.652519in}}{\pgfqpoint{0.902939in}{0.660333in}}%
\pgfpathcurveto{\pgfqpoint{0.910753in}{0.668146in}}{\pgfqpoint{0.915143in}{0.678745in}}{\pgfqpoint{0.915143in}{0.689796in}}%
\pgfpathcurveto{\pgfqpoint{0.915143in}{0.700846in}}{\pgfqpoint{0.910753in}{0.711445in}}{\pgfqpoint{0.902939in}{0.719258in}}%
\pgfpathcurveto{\pgfqpoint{0.895125in}{0.727072in}}{\pgfqpoint{0.884526in}{0.731462in}}{\pgfqpoint{0.873476in}{0.731462in}}%
\pgfpathcurveto{\pgfqpoint{0.862426in}{0.731462in}}{\pgfqpoint{0.851827in}{0.727072in}}{\pgfqpoint{0.844013in}{0.719258in}}%
\pgfpathcurveto{\pgfqpoint{0.836200in}{0.711445in}}{\pgfqpoint{0.831809in}{0.700846in}}{\pgfqpoint{0.831809in}{0.689796in}}%
\pgfpathcurveto{\pgfqpoint{0.831809in}{0.678745in}}{\pgfqpoint{0.836200in}{0.668146in}}{\pgfqpoint{0.844013in}{0.660333in}}%
\pgfpathcurveto{\pgfqpoint{0.851827in}{0.652519in}}{\pgfqpoint{0.862426in}{0.648129in}}{\pgfqpoint{0.873476in}{0.648129in}}%
\pgfpathclose%
\pgfusepath{stroke,fill}%
\end{pgfscope}%
\begin{pgfscope}%
\pgfpathrectangle{\pgfqpoint{0.648703in}{0.548769in}}{\pgfqpoint{5.112893in}{3.102590in}}%
\pgfusepath{clip}%
\pgfsetbuttcap%
\pgfsetroundjoin%
\definecolor{currentfill}{rgb}{1.000000,0.498039,0.054902}%
\pgfsetfillcolor{currentfill}%
\pgfsetlinewidth{1.003750pt}%
\definecolor{currentstroke}{rgb}{1.000000,0.498039,0.054902}%
\pgfsetstrokecolor{currentstroke}%
\pgfsetdash{}{0pt}%
\pgfpathmoveto{\pgfqpoint{2.243334in}{3.136837in}}%
\pgfpathcurveto{\pgfqpoint{2.254384in}{3.136837in}}{\pgfqpoint{2.264983in}{3.141228in}}{\pgfqpoint{2.272797in}{3.149041in}}%
\pgfpathcurveto{\pgfqpoint{2.280610in}{3.156855in}}{\pgfqpoint{2.285001in}{3.167454in}}{\pgfqpoint{2.285001in}{3.178504in}}%
\pgfpathcurveto{\pgfqpoint{2.285001in}{3.189554in}}{\pgfqpoint{2.280610in}{3.200153in}}{\pgfqpoint{2.272797in}{3.207967in}}%
\pgfpathcurveto{\pgfqpoint{2.264983in}{3.215780in}}{\pgfqpoint{2.254384in}{3.220171in}}{\pgfqpoint{2.243334in}{3.220171in}}%
\pgfpathcurveto{\pgfqpoint{2.232284in}{3.220171in}}{\pgfqpoint{2.221685in}{3.215780in}}{\pgfqpoint{2.213871in}{3.207967in}}%
\pgfpathcurveto{\pgfqpoint{2.206058in}{3.200153in}}{\pgfqpoint{2.201667in}{3.189554in}}{\pgfqpoint{2.201667in}{3.178504in}}%
\pgfpathcurveto{\pgfqpoint{2.201667in}{3.167454in}}{\pgfqpoint{2.206058in}{3.156855in}}{\pgfqpoint{2.213871in}{3.149041in}}%
\pgfpathcurveto{\pgfqpoint{2.221685in}{3.141228in}}{\pgfqpoint{2.232284in}{3.136837in}}{\pgfqpoint{2.243334in}{3.136837in}}%
\pgfpathclose%
\pgfusepath{stroke,fill}%
\end{pgfscope}%
\begin{pgfscope}%
\pgfpathrectangle{\pgfqpoint{0.648703in}{0.548769in}}{\pgfqpoint{5.112893in}{3.102590in}}%
\pgfusepath{clip}%
\pgfsetbuttcap%
\pgfsetroundjoin%
\definecolor{currentfill}{rgb}{1.000000,0.498039,0.054902}%
\pgfsetfillcolor{currentfill}%
\pgfsetlinewidth{1.003750pt}%
\definecolor{currentstroke}{rgb}{1.000000,0.498039,0.054902}%
\pgfsetstrokecolor{currentstroke}%
\pgfsetdash{}{0pt}%
\pgfpathmoveto{\pgfqpoint{4.124375in}{3.140985in}}%
\pgfpathcurveto{\pgfqpoint{4.135425in}{3.140985in}}{\pgfqpoint{4.146024in}{3.145375in}}{\pgfqpoint{4.153837in}{3.153189in}}%
\pgfpathcurveto{\pgfqpoint{4.161651in}{3.161003in}}{\pgfqpoint{4.166041in}{3.171602in}}{\pgfqpoint{4.166041in}{3.182652in}}%
\pgfpathcurveto{\pgfqpoint{4.166041in}{3.193702in}}{\pgfqpoint{4.161651in}{3.204301in}}{\pgfqpoint{4.153837in}{3.212115in}}%
\pgfpathcurveto{\pgfqpoint{4.146024in}{3.219928in}}{\pgfqpoint{4.135425in}{3.224319in}}{\pgfqpoint{4.124375in}{3.224319in}}%
\pgfpathcurveto{\pgfqpoint{4.113325in}{3.224319in}}{\pgfqpoint{4.102726in}{3.219928in}}{\pgfqpoint{4.094912in}{3.212115in}}%
\pgfpathcurveto{\pgfqpoint{4.087098in}{3.204301in}}{\pgfqpoint{4.082708in}{3.193702in}}{\pgfqpoint{4.082708in}{3.182652in}}%
\pgfpathcurveto{\pgfqpoint{4.082708in}{3.171602in}}{\pgfqpoint{4.087098in}{3.161003in}}{\pgfqpoint{4.094912in}{3.153189in}}%
\pgfpathcurveto{\pgfqpoint{4.102726in}{3.145375in}}{\pgfqpoint{4.113325in}{3.140985in}}{\pgfqpoint{4.124375in}{3.140985in}}%
\pgfpathclose%
\pgfusepath{stroke,fill}%
\end{pgfscope}%
\begin{pgfscope}%
\pgfpathrectangle{\pgfqpoint{0.648703in}{0.548769in}}{\pgfqpoint{5.112893in}{3.102590in}}%
\pgfusepath{clip}%
\pgfsetbuttcap%
\pgfsetroundjoin%
\definecolor{currentfill}{rgb}{1.000000,0.498039,0.054902}%
\pgfsetfillcolor{currentfill}%
\pgfsetlinewidth{1.003750pt}%
\definecolor{currentstroke}{rgb}{1.000000,0.498039,0.054902}%
\pgfsetstrokecolor{currentstroke}%
\pgfsetdash{}{0pt}%
\pgfpathmoveto{\pgfqpoint{2.444167in}{3.165872in}}%
\pgfpathcurveto{\pgfqpoint{2.455217in}{3.165872in}}{\pgfqpoint{2.465816in}{3.170263in}}{\pgfqpoint{2.473630in}{3.178076in}}%
\pgfpathcurveto{\pgfqpoint{2.481444in}{3.185890in}}{\pgfqpoint{2.485834in}{3.196489in}}{\pgfqpoint{2.485834in}{3.207539in}}%
\pgfpathcurveto{\pgfqpoint{2.485834in}{3.218589in}}{\pgfqpoint{2.481444in}{3.229188in}}{\pgfqpoint{2.473630in}{3.237002in}}%
\pgfpathcurveto{\pgfqpoint{2.465816in}{3.244815in}}{\pgfqpoint{2.455217in}{3.249206in}}{\pgfqpoint{2.444167in}{3.249206in}}%
\pgfpathcurveto{\pgfqpoint{2.433117in}{3.249206in}}{\pgfqpoint{2.422518in}{3.244815in}}{\pgfqpoint{2.414705in}{3.237002in}}%
\pgfpathcurveto{\pgfqpoint{2.406891in}{3.229188in}}{\pgfqpoint{2.402501in}{3.218589in}}{\pgfqpoint{2.402501in}{3.207539in}}%
\pgfpathcurveto{\pgfqpoint{2.402501in}{3.196489in}}{\pgfqpoint{2.406891in}{3.185890in}}{\pgfqpoint{2.414705in}{3.178076in}}%
\pgfpathcurveto{\pgfqpoint{2.422518in}{3.170263in}}{\pgfqpoint{2.433117in}{3.165872in}}{\pgfqpoint{2.444167in}{3.165872in}}%
\pgfpathclose%
\pgfusepath{stroke,fill}%
\end{pgfscope}%
\begin{pgfscope}%
\pgfpathrectangle{\pgfqpoint{0.648703in}{0.548769in}}{\pgfqpoint{5.112893in}{3.102590in}}%
\pgfusepath{clip}%
\pgfsetbuttcap%
\pgfsetroundjoin%
\definecolor{currentfill}{rgb}{1.000000,0.498039,0.054902}%
\pgfsetfillcolor{currentfill}%
\pgfsetlinewidth{1.003750pt}%
\definecolor{currentstroke}{rgb}{1.000000,0.498039,0.054902}%
\pgfsetstrokecolor{currentstroke}%
\pgfsetdash{}{0pt}%
\pgfpathmoveto{\pgfqpoint{4.115889in}{3.145133in}}%
\pgfpathcurveto{\pgfqpoint{4.126939in}{3.145133in}}{\pgfqpoint{4.137538in}{3.149523in}}{\pgfqpoint{4.145352in}{3.157337in}}%
\pgfpathcurveto{\pgfqpoint{4.153165in}{3.165151in}}{\pgfqpoint{4.157555in}{3.175750in}}{\pgfqpoint{4.157555in}{3.186800in}}%
\pgfpathcurveto{\pgfqpoint{4.157555in}{3.197850in}}{\pgfqpoint{4.153165in}{3.208449in}}{\pgfqpoint{4.145352in}{3.216262in}}%
\pgfpathcurveto{\pgfqpoint{4.137538in}{3.224076in}}{\pgfqpoint{4.126939in}{3.228466in}}{\pgfqpoint{4.115889in}{3.228466in}}%
\pgfpathcurveto{\pgfqpoint{4.104839in}{3.228466in}}{\pgfqpoint{4.094240in}{3.224076in}}{\pgfqpoint{4.086426in}{3.216262in}}%
\pgfpathcurveto{\pgfqpoint{4.078612in}{3.208449in}}{\pgfqpoint{4.074222in}{3.197850in}}{\pgfqpoint{4.074222in}{3.186800in}}%
\pgfpathcurveto{\pgfqpoint{4.074222in}{3.175750in}}{\pgfqpoint{4.078612in}{3.165151in}}{\pgfqpoint{4.086426in}{3.157337in}}%
\pgfpathcurveto{\pgfqpoint{4.094240in}{3.149523in}}{\pgfqpoint{4.104839in}{3.145133in}}{\pgfqpoint{4.115889in}{3.145133in}}%
\pgfpathclose%
\pgfusepath{stroke,fill}%
\end{pgfscope}%
\begin{pgfscope}%
\pgfpathrectangle{\pgfqpoint{0.648703in}{0.548769in}}{\pgfqpoint{5.112893in}{3.102590in}}%
\pgfusepath{clip}%
\pgfsetbuttcap%
\pgfsetroundjoin%
\definecolor{currentfill}{rgb}{1.000000,0.498039,0.054902}%
\pgfsetfillcolor{currentfill}%
\pgfsetlinewidth{1.003750pt}%
\definecolor{currentstroke}{rgb}{1.000000,0.498039,0.054902}%
\pgfsetstrokecolor{currentstroke}%
\pgfsetdash{}{0pt}%
\pgfpathmoveto{\pgfqpoint{4.366049in}{3.157577in}}%
\pgfpathcurveto{\pgfqpoint{4.377099in}{3.157577in}}{\pgfqpoint{4.387698in}{3.161967in}}{\pgfqpoint{4.395512in}{3.169780in}}%
\pgfpathcurveto{\pgfqpoint{4.403325in}{3.177594in}}{\pgfqpoint{4.407715in}{3.188193in}}{\pgfqpoint{4.407715in}{3.199243in}}%
\pgfpathcurveto{\pgfqpoint{4.407715in}{3.210293in}}{\pgfqpoint{4.403325in}{3.220892in}}{\pgfqpoint{4.395512in}{3.228706in}}%
\pgfpathcurveto{\pgfqpoint{4.387698in}{3.236520in}}{\pgfqpoint{4.377099in}{3.240910in}}{\pgfqpoint{4.366049in}{3.240910in}}%
\pgfpathcurveto{\pgfqpoint{4.354999in}{3.240910in}}{\pgfqpoint{4.344400in}{3.236520in}}{\pgfqpoint{4.336586in}{3.228706in}}%
\pgfpathcurveto{\pgfqpoint{4.328772in}{3.220892in}}{\pgfqpoint{4.324382in}{3.210293in}}{\pgfqpoint{4.324382in}{3.199243in}}%
\pgfpathcurveto{\pgfqpoint{4.324382in}{3.188193in}}{\pgfqpoint{4.328772in}{3.177594in}}{\pgfqpoint{4.336586in}{3.169780in}}%
\pgfpathcurveto{\pgfqpoint{4.344400in}{3.161967in}}{\pgfqpoint{4.354999in}{3.157577in}}{\pgfqpoint{4.366049in}{3.157577in}}%
\pgfpathclose%
\pgfusepath{stroke,fill}%
\end{pgfscope}%
\begin{pgfscope}%
\pgfpathrectangle{\pgfqpoint{0.648703in}{0.548769in}}{\pgfqpoint{5.112893in}{3.102590in}}%
\pgfusepath{clip}%
\pgfsetbuttcap%
\pgfsetroundjoin%
\definecolor{currentfill}{rgb}{1.000000,0.498039,0.054902}%
\pgfsetfillcolor{currentfill}%
\pgfsetlinewidth{1.003750pt}%
\definecolor{currentstroke}{rgb}{1.000000,0.498039,0.054902}%
\pgfsetstrokecolor{currentstroke}%
\pgfsetdash{}{0pt}%
\pgfpathmoveto{\pgfqpoint{4.330052in}{3.323490in}}%
\pgfpathcurveto{\pgfqpoint{4.341102in}{3.323490in}}{\pgfqpoint{4.351701in}{3.327881in}}{\pgfqpoint{4.359515in}{3.335694in}}%
\pgfpathcurveto{\pgfqpoint{4.367329in}{3.343508in}}{\pgfqpoint{4.371719in}{3.354107in}}{\pgfqpoint{4.371719in}{3.365157in}}%
\pgfpathcurveto{\pgfqpoint{4.371719in}{3.376207in}}{\pgfqpoint{4.367329in}{3.386806in}}{\pgfqpoint{4.359515in}{3.394620in}}%
\pgfpathcurveto{\pgfqpoint{4.351701in}{3.402434in}}{\pgfqpoint{4.341102in}{3.406824in}}{\pgfqpoint{4.330052in}{3.406824in}}%
\pgfpathcurveto{\pgfqpoint{4.319002in}{3.406824in}}{\pgfqpoint{4.308403in}{3.402434in}}{\pgfqpoint{4.300590in}{3.394620in}}%
\pgfpathcurveto{\pgfqpoint{4.292776in}{3.386806in}}{\pgfqpoint{4.288386in}{3.376207in}}{\pgfqpoint{4.288386in}{3.365157in}}%
\pgfpathcurveto{\pgfqpoint{4.288386in}{3.354107in}}{\pgfqpoint{4.292776in}{3.343508in}}{\pgfqpoint{4.300590in}{3.335694in}}%
\pgfpathcurveto{\pgfqpoint{4.308403in}{3.327881in}}{\pgfqpoint{4.319002in}{3.323490in}}{\pgfqpoint{4.330052in}{3.323490in}}%
\pgfpathclose%
\pgfusepath{stroke,fill}%
\end{pgfscope}%
\begin{pgfscope}%
\pgfpathrectangle{\pgfqpoint{0.648703in}{0.548769in}}{\pgfqpoint{5.112893in}{3.102590in}}%
\pgfusepath{clip}%
\pgfsetbuttcap%
\pgfsetroundjoin%
\definecolor{currentfill}{rgb}{0.121569,0.466667,0.705882}%
\pgfsetfillcolor{currentfill}%
\pgfsetlinewidth{1.003750pt}%
\definecolor{currentstroke}{rgb}{0.121569,0.466667,0.705882}%
\pgfsetstrokecolor{currentstroke}%
\pgfsetdash{}{0pt}%
\pgfpathmoveto{\pgfqpoint{3.550149in}{2.410964in}}%
\pgfpathcurveto{\pgfqpoint{3.561200in}{2.410964in}}{\pgfqpoint{3.571799in}{2.415354in}}{\pgfqpoint{3.579612in}{2.423168in}}%
\pgfpathcurveto{\pgfqpoint{3.587426in}{2.430982in}}{\pgfqpoint{3.591816in}{2.441581in}}{\pgfqpoint{3.591816in}{2.452631in}}%
\pgfpathcurveto{\pgfqpoint{3.591816in}{2.463681in}}{\pgfqpoint{3.587426in}{2.474280in}}{\pgfqpoint{3.579612in}{2.482094in}}%
\pgfpathcurveto{\pgfqpoint{3.571799in}{2.489907in}}{\pgfqpoint{3.561200in}{2.494297in}}{\pgfqpoint{3.550149in}{2.494297in}}%
\pgfpathcurveto{\pgfqpoint{3.539099in}{2.494297in}}{\pgfqpoint{3.528500in}{2.489907in}}{\pgfqpoint{3.520687in}{2.482094in}}%
\pgfpathcurveto{\pgfqpoint{3.512873in}{2.474280in}}{\pgfqpoint{3.508483in}{2.463681in}}{\pgfqpoint{3.508483in}{2.452631in}}%
\pgfpathcurveto{\pgfqpoint{3.508483in}{2.441581in}}{\pgfqpoint{3.512873in}{2.430982in}}{\pgfqpoint{3.520687in}{2.423168in}}%
\pgfpathcurveto{\pgfqpoint{3.528500in}{2.415354in}}{\pgfqpoint{3.539099in}{2.410964in}}{\pgfqpoint{3.550149in}{2.410964in}}%
\pgfpathclose%
\pgfusepath{stroke,fill}%
\end{pgfscope}%
\begin{pgfscope}%
\pgfpathrectangle{\pgfqpoint{0.648703in}{0.548769in}}{\pgfqpoint{5.112893in}{3.102590in}}%
\pgfusepath{clip}%
\pgfsetbuttcap%
\pgfsetroundjoin%
\definecolor{currentfill}{rgb}{1.000000,0.498039,0.054902}%
\pgfsetfillcolor{currentfill}%
\pgfsetlinewidth{1.003750pt}%
\definecolor{currentstroke}{rgb}{1.000000,0.498039,0.054902}%
\pgfsetstrokecolor{currentstroke}%
\pgfsetdash{}{0pt}%
\pgfpathmoveto{\pgfqpoint{4.324210in}{3.199055in}}%
\pgfpathcurveto{\pgfqpoint{4.335260in}{3.199055in}}{\pgfqpoint{4.345859in}{3.203445in}}{\pgfqpoint{4.353673in}{3.211259in}}%
\pgfpathcurveto{\pgfqpoint{4.361487in}{3.219073in}}{\pgfqpoint{4.365877in}{3.229672in}}{\pgfqpoint{4.365877in}{3.240722in}}%
\pgfpathcurveto{\pgfqpoint{4.365877in}{3.251772in}}{\pgfqpoint{4.361487in}{3.262371in}}{\pgfqpoint{4.353673in}{3.270185in}}%
\pgfpathcurveto{\pgfqpoint{4.345859in}{3.277998in}}{\pgfqpoint{4.335260in}{3.282388in}}{\pgfqpoint{4.324210in}{3.282388in}}%
\pgfpathcurveto{\pgfqpoint{4.313160in}{3.282388in}}{\pgfqpoint{4.302561in}{3.277998in}}{\pgfqpoint{4.294748in}{3.270185in}}%
\pgfpathcurveto{\pgfqpoint{4.286934in}{3.262371in}}{\pgfqpoint{4.282544in}{3.251772in}}{\pgfqpoint{4.282544in}{3.240722in}}%
\pgfpathcurveto{\pgfqpoint{4.282544in}{3.229672in}}{\pgfqpoint{4.286934in}{3.219073in}}{\pgfqpoint{4.294748in}{3.211259in}}%
\pgfpathcurveto{\pgfqpoint{4.302561in}{3.203445in}}{\pgfqpoint{4.313160in}{3.199055in}}{\pgfqpoint{4.324210in}{3.199055in}}%
\pgfpathclose%
\pgfusepath{stroke,fill}%
\end{pgfscope}%
\begin{pgfscope}%
\pgfpathrectangle{\pgfqpoint{0.648703in}{0.548769in}}{\pgfqpoint{5.112893in}{3.102590in}}%
\pgfusepath{clip}%
\pgfsetbuttcap%
\pgfsetroundjoin%
\definecolor{currentfill}{rgb}{0.121569,0.466667,0.705882}%
\pgfsetfillcolor{currentfill}%
\pgfsetlinewidth{1.003750pt}%
\definecolor{currentstroke}{rgb}{0.121569,0.466667,0.705882}%
\pgfsetstrokecolor{currentstroke}%
\pgfsetdash{}{0pt}%
\pgfpathmoveto{\pgfqpoint{1.925643in}{3.124394in}}%
\pgfpathcurveto{\pgfqpoint{1.936693in}{3.124394in}}{\pgfqpoint{1.947292in}{3.128784in}}{\pgfqpoint{1.955106in}{3.136598in}}%
\pgfpathcurveto{\pgfqpoint{1.962919in}{3.144411in}}{\pgfqpoint{1.967310in}{3.155010in}}{\pgfqpoint{1.967310in}{3.166060in}}%
\pgfpathcurveto{\pgfqpoint{1.967310in}{3.177111in}}{\pgfqpoint{1.962919in}{3.187710in}}{\pgfqpoint{1.955106in}{3.195523in}}%
\pgfpathcurveto{\pgfqpoint{1.947292in}{3.203337in}}{\pgfqpoint{1.936693in}{3.207727in}}{\pgfqpoint{1.925643in}{3.207727in}}%
\pgfpathcurveto{\pgfqpoint{1.914593in}{3.207727in}}{\pgfqpoint{1.903994in}{3.203337in}}{\pgfqpoint{1.896180in}{3.195523in}}%
\pgfpathcurveto{\pgfqpoint{1.888366in}{3.187710in}}{\pgfqpoint{1.883976in}{3.177111in}}{\pgfqpoint{1.883976in}{3.166060in}}%
\pgfpathcurveto{\pgfqpoint{1.883976in}{3.155010in}}{\pgfqpoint{1.888366in}{3.144411in}}{\pgfqpoint{1.896180in}{3.136598in}}%
\pgfpathcurveto{\pgfqpoint{1.903994in}{3.128784in}}{\pgfqpoint{1.914593in}{3.124394in}}{\pgfqpoint{1.925643in}{3.124394in}}%
\pgfpathclose%
\pgfusepath{stroke,fill}%
\end{pgfscope}%
\begin{pgfscope}%
\pgfpathrectangle{\pgfqpoint{0.648703in}{0.548769in}}{\pgfqpoint{5.112893in}{3.102590in}}%
\pgfusepath{clip}%
\pgfsetbuttcap%
\pgfsetroundjoin%
\definecolor{currentfill}{rgb}{0.121569,0.466667,0.705882}%
\pgfsetfillcolor{currentfill}%
\pgfsetlinewidth{1.003750pt}%
\definecolor{currentstroke}{rgb}{0.121569,0.466667,0.705882}%
\pgfsetstrokecolor{currentstroke}%
\pgfsetdash{}{0pt}%
\pgfpathmoveto{\pgfqpoint{3.010237in}{3.132690in}}%
\pgfpathcurveto{\pgfqpoint{3.021287in}{3.132690in}}{\pgfqpoint{3.031886in}{3.137080in}}{\pgfqpoint{3.039700in}{3.144893in}}%
\pgfpathcurveto{\pgfqpoint{3.047513in}{3.152707in}}{\pgfqpoint{3.051904in}{3.163306in}}{\pgfqpoint{3.051904in}{3.174356in}}%
\pgfpathcurveto{\pgfqpoint{3.051904in}{3.185406in}}{\pgfqpoint{3.047513in}{3.196005in}}{\pgfqpoint{3.039700in}{3.203819in}}%
\pgfpathcurveto{\pgfqpoint{3.031886in}{3.211633in}}{\pgfqpoint{3.021287in}{3.216023in}}{\pgfqpoint{3.010237in}{3.216023in}}%
\pgfpathcurveto{\pgfqpoint{2.999187in}{3.216023in}}{\pgfqpoint{2.988588in}{3.211633in}}{\pgfqpoint{2.980774in}{3.203819in}}%
\pgfpathcurveto{\pgfqpoint{2.972961in}{3.196005in}}{\pgfqpoint{2.968570in}{3.185406in}}{\pgfqpoint{2.968570in}{3.174356in}}%
\pgfpathcurveto{\pgfqpoint{2.968570in}{3.163306in}}{\pgfqpoint{2.972961in}{3.152707in}}{\pgfqpoint{2.980774in}{3.144893in}}%
\pgfpathcurveto{\pgfqpoint{2.988588in}{3.137080in}}{\pgfqpoint{2.999187in}{3.132690in}}{\pgfqpoint{3.010237in}{3.132690in}}%
\pgfpathclose%
\pgfusepath{stroke,fill}%
\end{pgfscope}%
\begin{pgfscope}%
\pgfpathrectangle{\pgfqpoint{0.648703in}{0.548769in}}{\pgfqpoint{5.112893in}{3.102590in}}%
\pgfusepath{clip}%
\pgfsetbuttcap%
\pgfsetroundjoin%
\definecolor{currentfill}{rgb}{0.121569,0.466667,0.705882}%
\pgfsetfillcolor{currentfill}%
\pgfsetlinewidth{1.003750pt}%
\definecolor{currentstroke}{rgb}{0.121569,0.466667,0.705882}%
\pgfsetstrokecolor{currentstroke}%
\pgfsetdash{}{0pt}%
\pgfpathmoveto{\pgfqpoint{1.887202in}{3.132690in}}%
\pgfpathcurveto{\pgfqpoint{1.898252in}{3.132690in}}{\pgfqpoint{1.908851in}{3.137080in}}{\pgfqpoint{1.916664in}{3.144893in}}%
\pgfpathcurveto{\pgfqpoint{1.924478in}{3.152707in}}{\pgfqpoint{1.928868in}{3.163306in}}{\pgfqpoint{1.928868in}{3.174356in}}%
\pgfpathcurveto{\pgfqpoint{1.928868in}{3.185406in}}{\pgfqpoint{1.924478in}{3.196005in}}{\pgfqpoint{1.916664in}{3.203819in}}%
\pgfpathcurveto{\pgfqpoint{1.908851in}{3.211633in}}{\pgfqpoint{1.898252in}{3.216023in}}{\pgfqpoint{1.887202in}{3.216023in}}%
\pgfpathcurveto{\pgfqpoint{1.876152in}{3.216023in}}{\pgfqpoint{1.865553in}{3.211633in}}{\pgfqpoint{1.857739in}{3.203819in}}%
\pgfpathcurveto{\pgfqpoint{1.849925in}{3.196005in}}{\pgfqpoint{1.845535in}{3.185406in}}{\pgfqpoint{1.845535in}{3.174356in}}%
\pgfpathcurveto{\pgfqpoint{1.845535in}{3.163306in}}{\pgfqpoint{1.849925in}{3.152707in}}{\pgfqpoint{1.857739in}{3.144893in}}%
\pgfpathcurveto{\pgfqpoint{1.865553in}{3.137080in}}{\pgfqpoint{1.876152in}{3.132690in}}{\pgfqpoint{1.887202in}{3.132690in}}%
\pgfpathclose%
\pgfusepath{stroke,fill}%
\end{pgfscope}%
\begin{pgfscope}%
\pgfpathrectangle{\pgfqpoint{0.648703in}{0.548769in}}{\pgfqpoint{5.112893in}{3.102590in}}%
\pgfusepath{clip}%
\pgfsetbuttcap%
\pgfsetroundjoin%
\definecolor{currentfill}{rgb}{1.000000,0.498039,0.054902}%
\pgfsetfillcolor{currentfill}%
\pgfsetlinewidth{1.003750pt}%
\definecolor{currentstroke}{rgb}{1.000000,0.498039,0.054902}%
\pgfsetstrokecolor{currentstroke}%
\pgfsetdash{}{0pt}%
\pgfpathmoveto{\pgfqpoint{3.710101in}{3.136837in}}%
\pgfpathcurveto{\pgfqpoint{3.721152in}{3.136837in}}{\pgfqpoint{3.731751in}{3.141228in}}{\pgfqpoint{3.739564in}{3.149041in}}%
\pgfpathcurveto{\pgfqpoint{3.747378in}{3.156855in}}{\pgfqpoint{3.751768in}{3.167454in}}{\pgfqpoint{3.751768in}{3.178504in}}%
\pgfpathcurveto{\pgfqpoint{3.751768in}{3.189554in}}{\pgfqpoint{3.747378in}{3.200153in}}{\pgfqpoint{3.739564in}{3.207967in}}%
\pgfpathcurveto{\pgfqpoint{3.731751in}{3.215780in}}{\pgfqpoint{3.721152in}{3.220171in}}{\pgfqpoint{3.710101in}{3.220171in}}%
\pgfpathcurveto{\pgfqpoint{3.699051in}{3.220171in}}{\pgfqpoint{3.688452in}{3.215780in}}{\pgfqpoint{3.680639in}{3.207967in}}%
\pgfpathcurveto{\pgfqpoint{3.672825in}{3.200153in}}{\pgfqpoint{3.668435in}{3.189554in}}{\pgfqpoint{3.668435in}{3.178504in}}%
\pgfpathcurveto{\pgfqpoint{3.668435in}{3.167454in}}{\pgfqpoint{3.672825in}{3.156855in}}{\pgfqpoint{3.680639in}{3.149041in}}%
\pgfpathcurveto{\pgfqpoint{3.688452in}{3.141228in}}{\pgfqpoint{3.699051in}{3.136837in}}{\pgfqpoint{3.710101in}{3.136837in}}%
\pgfpathclose%
\pgfusepath{stroke,fill}%
\end{pgfscope}%
\begin{pgfscope}%
\pgfpathrectangle{\pgfqpoint{0.648703in}{0.548769in}}{\pgfqpoint{5.112893in}{3.102590in}}%
\pgfusepath{clip}%
\pgfsetbuttcap%
\pgfsetroundjoin%
\definecolor{currentfill}{rgb}{1.000000,0.498039,0.054902}%
\pgfsetfillcolor{currentfill}%
\pgfsetlinewidth{1.003750pt}%
\definecolor{currentstroke}{rgb}{1.000000,0.498039,0.054902}%
\pgfsetstrokecolor{currentstroke}%
\pgfsetdash{}{0pt}%
\pgfpathmoveto{\pgfqpoint{4.510399in}{3.244681in}}%
\pgfpathcurveto{\pgfqpoint{4.521449in}{3.244681in}}{\pgfqpoint{4.532048in}{3.249072in}}{\pgfqpoint{4.539862in}{3.256885in}}%
\pgfpathcurveto{\pgfqpoint{4.547675in}{3.264699in}}{\pgfqpoint{4.552066in}{3.275298in}}{\pgfqpoint{4.552066in}{3.286348in}}%
\pgfpathcurveto{\pgfqpoint{4.552066in}{3.297398in}}{\pgfqpoint{4.547675in}{3.307997in}}{\pgfqpoint{4.539862in}{3.315811in}}%
\pgfpathcurveto{\pgfqpoint{4.532048in}{3.323624in}}{\pgfqpoint{4.521449in}{3.328015in}}{\pgfqpoint{4.510399in}{3.328015in}}%
\pgfpathcurveto{\pgfqpoint{4.499349in}{3.328015in}}{\pgfqpoint{4.488750in}{3.323624in}}{\pgfqpoint{4.480936in}{3.315811in}}%
\pgfpathcurveto{\pgfqpoint{4.473122in}{3.307997in}}{\pgfqpoint{4.468732in}{3.297398in}}{\pgfqpoint{4.468732in}{3.286348in}}%
\pgfpathcurveto{\pgfqpoint{4.468732in}{3.275298in}}{\pgfqpoint{4.473122in}{3.264699in}}{\pgfqpoint{4.480936in}{3.256885in}}%
\pgfpathcurveto{\pgfqpoint{4.488750in}{3.249072in}}{\pgfqpoint{4.499349in}{3.244681in}}{\pgfqpoint{4.510399in}{3.244681in}}%
\pgfpathclose%
\pgfusepath{stroke,fill}%
\end{pgfscope}%
\begin{pgfscope}%
\pgfpathrectangle{\pgfqpoint{0.648703in}{0.548769in}}{\pgfqpoint{5.112893in}{3.102590in}}%
\pgfusepath{clip}%
\pgfsetbuttcap%
\pgfsetroundjoin%
\definecolor{currentfill}{rgb}{1.000000,0.498039,0.054902}%
\pgfsetfillcolor{currentfill}%
\pgfsetlinewidth{1.003750pt}%
\definecolor{currentstroke}{rgb}{1.000000,0.498039,0.054902}%
\pgfsetstrokecolor{currentstroke}%
\pgfsetdash{}{0pt}%
\pgfpathmoveto{\pgfqpoint{4.205239in}{3.190759in}}%
\pgfpathcurveto{\pgfqpoint{4.216289in}{3.190759in}}{\pgfqpoint{4.226888in}{3.195150in}}{\pgfqpoint{4.234702in}{3.202963in}}%
\pgfpathcurveto{\pgfqpoint{4.242515in}{3.210777in}}{\pgfqpoint{4.246905in}{3.221376in}}{\pgfqpoint{4.246905in}{3.232426in}}%
\pgfpathcurveto{\pgfqpoint{4.246905in}{3.243476in}}{\pgfqpoint{4.242515in}{3.254075in}}{\pgfqpoint{4.234702in}{3.261889in}}%
\pgfpathcurveto{\pgfqpoint{4.226888in}{3.269702in}}{\pgfqpoint{4.216289in}{3.274093in}}{\pgfqpoint{4.205239in}{3.274093in}}%
\pgfpathcurveto{\pgfqpoint{4.194189in}{3.274093in}}{\pgfqpoint{4.183590in}{3.269702in}}{\pgfqpoint{4.175776in}{3.261889in}}%
\pgfpathcurveto{\pgfqpoint{4.167962in}{3.254075in}}{\pgfqpoint{4.163572in}{3.243476in}}{\pgfqpoint{4.163572in}{3.232426in}}%
\pgfpathcurveto{\pgfqpoint{4.163572in}{3.221376in}}{\pgfqpoint{4.167962in}{3.210777in}}{\pgfqpoint{4.175776in}{3.202963in}}%
\pgfpathcurveto{\pgfqpoint{4.183590in}{3.195150in}}{\pgfqpoint{4.194189in}{3.190759in}}{\pgfqpoint{4.205239in}{3.190759in}}%
\pgfpathclose%
\pgfusepath{stroke,fill}%
\end{pgfscope}%
\begin{pgfscope}%
\pgfpathrectangle{\pgfqpoint{0.648703in}{0.548769in}}{\pgfqpoint{5.112893in}{3.102590in}}%
\pgfusepath{clip}%
\pgfsetbuttcap%
\pgfsetroundjoin%
\definecolor{currentfill}{rgb}{0.121569,0.466667,0.705882}%
\pgfsetfillcolor{currentfill}%
\pgfsetlinewidth{1.003750pt}%
\definecolor{currentstroke}{rgb}{0.121569,0.466667,0.705882}%
\pgfsetstrokecolor{currentstroke}%
\pgfsetdash{}{0pt}%
\pgfpathmoveto{\pgfqpoint{0.904418in}{0.664720in}}%
\pgfpathcurveto{\pgfqpoint{0.915468in}{0.664720in}}{\pgfqpoint{0.926067in}{0.669111in}}{\pgfqpoint{0.933880in}{0.676924in}}%
\pgfpathcurveto{\pgfqpoint{0.941694in}{0.684738in}}{\pgfqpoint{0.946084in}{0.695337in}}{\pgfqpoint{0.946084in}{0.706387in}}%
\pgfpathcurveto{\pgfqpoint{0.946084in}{0.717437in}}{\pgfqpoint{0.941694in}{0.728036in}}{\pgfqpoint{0.933880in}{0.735850in}}%
\pgfpathcurveto{\pgfqpoint{0.926067in}{0.743663in}}{\pgfqpoint{0.915468in}{0.748054in}}{\pgfqpoint{0.904418in}{0.748054in}}%
\pgfpathcurveto{\pgfqpoint{0.893367in}{0.748054in}}{\pgfqpoint{0.882768in}{0.743663in}}{\pgfqpoint{0.874955in}{0.735850in}}%
\pgfpathcurveto{\pgfqpoint{0.867141in}{0.728036in}}{\pgfqpoint{0.862751in}{0.717437in}}{\pgfqpoint{0.862751in}{0.706387in}}%
\pgfpathcurveto{\pgfqpoint{0.862751in}{0.695337in}}{\pgfqpoint{0.867141in}{0.684738in}}{\pgfqpoint{0.874955in}{0.676924in}}%
\pgfpathcurveto{\pgfqpoint{0.882768in}{0.669111in}}{\pgfqpoint{0.893367in}{0.664720in}}{\pgfqpoint{0.904418in}{0.664720in}}%
\pgfpathclose%
\pgfusepath{stroke,fill}%
\end{pgfscope}%
\begin{pgfscope}%
\pgfpathrectangle{\pgfqpoint{0.648703in}{0.548769in}}{\pgfqpoint{5.112893in}{3.102590in}}%
\pgfusepath{clip}%
\pgfsetbuttcap%
\pgfsetroundjoin%
\definecolor{currentfill}{rgb}{1.000000,0.498039,0.054902}%
\pgfsetfillcolor{currentfill}%
\pgfsetlinewidth{1.003750pt}%
\definecolor{currentstroke}{rgb}{1.000000,0.498039,0.054902}%
\pgfsetstrokecolor{currentstroke}%
\pgfsetdash{}{0pt}%
\pgfpathmoveto{\pgfqpoint{4.466343in}{3.145133in}}%
\pgfpathcurveto{\pgfqpoint{4.477393in}{3.145133in}}{\pgfqpoint{4.487992in}{3.149523in}}{\pgfqpoint{4.495805in}{3.157337in}}%
\pgfpathcurveto{\pgfqpoint{4.503619in}{3.165151in}}{\pgfqpoint{4.508009in}{3.175750in}}{\pgfqpoint{4.508009in}{3.186800in}}%
\pgfpathcurveto{\pgfqpoint{4.508009in}{3.197850in}}{\pgfqpoint{4.503619in}{3.208449in}}{\pgfqpoint{4.495805in}{3.216262in}}%
\pgfpathcurveto{\pgfqpoint{4.487992in}{3.224076in}}{\pgfqpoint{4.477393in}{3.228466in}}{\pgfqpoint{4.466343in}{3.228466in}}%
\pgfpathcurveto{\pgfqpoint{4.455292in}{3.228466in}}{\pgfqpoint{4.444693in}{3.224076in}}{\pgfqpoint{4.436880in}{3.216262in}}%
\pgfpathcurveto{\pgfqpoint{4.429066in}{3.208449in}}{\pgfqpoint{4.424676in}{3.197850in}}{\pgfqpoint{4.424676in}{3.186800in}}%
\pgfpathcurveto{\pgfqpoint{4.424676in}{3.175750in}}{\pgfqpoint{4.429066in}{3.165151in}}{\pgfqpoint{4.436880in}{3.157337in}}%
\pgfpathcurveto{\pgfqpoint{4.444693in}{3.149523in}}{\pgfqpoint{4.455292in}{3.145133in}}{\pgfqpoint{4.466343in}{3.145133in}}%
\pgfpathclose%
\pgfusepath{stroke,fill}%
\end{pgfscope}%
\begin{pgfscope}%
\pgfpathrectangle{\pgfqpoint{0.648703in}{0.548769in}}{\pgfqpoint{5.112893in}{3.102590in}}%
\pgfusepath{clip}%
\pgfsetbuttcap%
\pgfsetroundjoin%
\definecolor{currentfill}{rgb}{0.121569,0.466667,0.705882}%
\pgfsetfillcolor{currentfill}%
\pgfsetlinewidth{1.003750pt}%
\definecolor{currentstroke}{rgb}{0.121569,0.466667,0.705882}%
\pgfsetstrokecolor{currentstroke}%
\pgfsetdash{}{0pt}%
\pgfpathmoveto{\pgfqpoint{0.873307in}{0.648129in}}%
\pgfpathcurveto{\pgfqpoint{0.884358in}{0.648129in}}{\pgfqpoint{0.894957in}{0.652519in}}{\pgfqpoint{0.902770in}{0.660333in}}%
\pgfpathcurveto{\pgfqpoint{0.910584in}{0.668146in}}{\pgfqpoint{0.914974in}{0.678745in}}{\pgfqpoint{0.914974in}{0.689796in}}%
\pgfpathcurveto{\pgfqpoint{0.914974in}{0.700846in}}{\pgfqpoint{0.910584in}{0.711445in}}{\pgfqpoint{0.902770in}{0.719258in}}%
\pgfpathcurveto{\pgfqpoint{0.894957in}{0.727072in}}{\pgfqpoint{0.884358in}{0.731462in}}{\pgfqpoint{0.873307in}{0.731462in}}%
\pgfpathcurveto{\pgfqpoint{0.862257in}{0.731462in}}{\pgfqpoint{0.851658in}{0.727072in}}{\pgfqpoint{0.843845in}{0.719258in}}%
\pgfpathcurveto{\pgfqpoint{0.836031in}{0.711445in}}{\pgfqpoint{0.831641in}{0.700846in}}{\pgfqpoint{0.831641in}{0.689796in}}%
\pgfpathcurveto{\pgfqpoint{0.831641in}{0.678745in}}{\pgfqpoint{0.836031in}{0.668146in}}{\pgfqpoint{0.843845in}{0.660333in}}%
\pgfpathcurveto{\pgfqpoint{0.851658in}{0.652519in}}{\pgfqpoint{0.862257in}{0.648129in}}{\pgfqpoint{0.873307in}{0.648129in}}%
\pgfpathclose%
\pgfusepath{stroke,fill}%
\end{pgfscope}%
\begin{pgfscope}%
\pgfpathrectangle{\pgfqpoint{0.648703in}{0.548769in}}{\pgfqpoint{5.112893in}{3.102590in}}%
\pgfusepath{clip}%
\pgfsetbuttcap%
\pgfsetroundjoin%
\definecolor{currentfill}{rgb}{1.000000,0.498039,0.054902}%
\pgfsetfillcolor{currentfill}%
\pgfsetlinewidth{1.003750pt}%
\definecolor{currentstroke}{rgb}{1.000000,0.498039,0.054902}%
\pgfsetstrokecolor{currentstroke}%
\pgfsetdash{}{0pt}%
\pgfpathmoveto{\pgfqpoint{4.262499in}{3.149281in}}%
\pgfpathcurveto{\pgfqpoint{4.273549in}{3.149281in}}{\pgfqpoint{4.284148in}{3.153671in}}{\pgfqpoint{4.291962in}{3.161485in}}%
\pgfpathcurveto{\pgfqpoint{4.299776in}{3.169298in}}{\pgfqpoint{4.304166in}{3.179897in}}{\pgfqpoint{4.304166in}{3.190948in}}%
\pgfpathcurveto{\pgfqpoint{4.304166in}{3.201998in}}{\pgfqpoint{4.299776in}{3.212597in}}{\pgfqpoint{4.291962in}{3.220410in}}%
\pgfpathcurveto{\pgfqpoint{4.284148in}{3.228224in}}{\pgfqpoint{4.273549in}{3.232614in}}{\pgfqpoint{4.262499in}{3.232614in}}%
\pgfpathcurveto{\pgfqpoint{4.251449in}{3.232614in}}{\pgfqpoint{4.240850in}{3.228224in}}{\pgfqpoint{4.233037in}{3.220410in}}%
\pgfpathcurveto{\pgfqpoint{4.225223in}{3.212597in}}{\pgfqpoint{4.220833in}{3.201998in}}{\pgfqpoint{4.220833in}{3.190948in}}%
\pgfpathcurveto{\pgfqpoint{4.220833in}{3.179897in}}{\pgfqpoint{4.225223in}{3.169298in}}{\pgfqpoint{4.233037in}{3.161485in}}%
\pgfpathcurveto{\pgfqpoint{4.240850in}{3.153671in}}{\pgfqpoint{4.251449in}{3.149281in}}{\pgfqpoint{4.262499in}{3.149281in}}%
\pgfpathclose%
\pgfusepath{stroke,fill}%
\end{pgfscope}%
\begin{pgfscope}%
\pgfpathrectangle{\pgfqpoint{0.648703in}{0.548769in}}{\pgfqpoint{5.112893in}{3.102590in}}%
\pgfusepath{clip}%
\pgfsetbuttcap%
\pgfsetroundjoin%
\definecolor{currentfill}{rgb}{0.121569,0.466667,0.705882}%
\pgfsetfillcolor{currentfill}%
\pgfsetlinewidth{1.003750pt}%
\definecolor{currentstroke}{rgb}{0.121569,0.466667,0.705882}%
\pgfsetstrokecolor{currentstroke}%
\pgfsetdash{}{0pt}%
\pgfpathmoveto{\pgfqpoint{4.235707in}{3.124394in}}%
\pgfpathcurveto{\pgfqpoint{4.246758in}{3.124394in}}{\pgfqpoint{4.257357in}{3.128784in}}{\pgfqpoint{4.265170in}{3.136598in}}%
\pgfpathcurveto{\pgfqpoint{4.272984in}{3.144411in}}{\pgfqpoint{4.277374in}{3.155010in}}{\pgfqpoint{4.277374in}{3.166060in}}%
\pgfpathcurveto{\pgfqpoint{4.277374in}{3.177111in}}{\pgfqpoint{4.272984in}{3.187710in}}{\pgfqpoint{4.265170in}{3.195523in}}%
\pgfpathcurveto{\pgfqpoint{4.257357in}{3.203337in}}{\pgfqpoint{4.246758in}{3.207727in}}{\pgfqpoint{4.235707in}{3.207727in}}%
\pgfpathcurveto{\pgfqpoint{4.224657in}{3.207727in}}{\pgfqpoint{4.214058in}{3.203337in}}{\pgfqpoint{4.206245in}{3.195523in}}%
\pgfpathcurveto{\pgfqpoint{4.198431in}{3.187710in}}{\pgfqpoint{4.194041in}{3.177111in}}{\pgfqpoint{4.194041in}{3.166060in}}%
\pgfpathcurveto{\pgfqpoint{4.194041in}{3.155010in}}{\pgfqpoint{4.198431in}{3.144411in}}{\pgfqpoint{4.206245in}{3.136598in}}%
\pgfpathcurveto{\pgfqpoint{4.214058in}{3.128784in}}{\pgfqpoint{4.224657in}{3.124394in}}{\pgfqpoint{4.235707in}{3.124394in}}%
\pgfpathclose%
\pgfusepath{stroke,fill}%
\end{pgfscope}%
\begin{pgfscope}%
\pgfpathrectangle{\pgfqpoint{0.648703in}{0.548769in}}{\pgfqpoint{5.112893in}{3.102590in}}%
\pgfusepath{clip}%
\pgfsetbuttcap%
\pgfsetroundjoin%
\definecolor{currentfill}{rgb}{1.000000,0.498039,0.054902}%
\pgfsetfillcolor{currentfill}%
\pgfsetlinewidth{1.003750pt}%
\definecolor{currentstroke}{rgb}{1.000000,0.498039,0.054902}%
\pgfsetstrokecolor{currentstroke}%
\pgfsetdash{}{0pt}%
\pgfpathmoveto{\pgfqpoint{4.106820in}{3.136837in}}%
\pgfpathcurveto{\pgfqpoint{4.117871in}{3.136837in}}{\pgfqpoint{4.128470in}{3.141228in}}{\pgfqpoint{4.136283in}{3.149041in}}%
\pgfpathcurveto{\pgfqpoint{4.144097in}{3.156855in}}{\pgfqpoint{4.148487in}{3.167454in}}{\pgfqpoint{4.148487in}{3.178504in}}%
\pgfpathcurveto{\pgfqpoint{4.148487in}{3.189554in}}{\pgfqpoint{4.144097in}{3.200153in}}{\pgfqpoint{4.136283in}{3.207967in}}%
\pgfpathcurveto{\pgfqpoint{4.128470in}{3.215780in}}{\pgfqpoint{4.117871in}{3.220171in}}{\pgfqpoint{4.106820in}{3.220171in}}%
\pgfpathcurveto{\pgfqpoint{4.095770in}{3.220171in}}{\pgfqpoint{4.085171in}{3.215780in}}{\pgfqpoint{4.077358in}{3.207967in}}%
\pgfpathcurveto{\pgfqpoint{4.069544in}{3.200153in}}{\pgfqpoint{4.065154in}{3.189554in}}{\pgfqpoint{4.065154in}{3.178504in}}%
\pgfpathcurveto{\pgfqpoint{4.065154in}{3.167454in}}{\pgfqpoint{4.069544in}{3.156855in}}{\pgfqpoint{4.077358in}{3.149041in}}%
\pgfpathcurveto{\pgfqpoint{4.085171in}{3.141228in}}{\pgfqpoint{4.095770in}{3.136837in}}{\pgfqpoint{4.106820in}{3.136837in}}%
\pgfpathclose%
\pgfusepath{stroke,fill}%
\end{pgfscope}%
\begin{pgfscope}%
\pgfpathrectangle{\pgfqpoint{0.648703in}{0.548769in}}{\pgfqpoint{5.112893in}{3.102590in}}%
\pgfusepath{clip}%
\pgfsetbuttcap%
\pgfsetroundjoin%
\definecolor{currentfill}{rgb}{1.000000,0.498039,0.054902}%
\pgfsetfillcolor{currentfill}%
\pgfsetlinewidth{1.003750pt}%
\definecolor{currentstroke}{rgb}{1.000000,0.498039,0.054902}%
\pgfsetstrokecolor{currentstroke}%
\pgfsetdash{}{0pt}%
\pgfpathmoveto{\pgfqpoint{2.605452in}{3.140985in}}%
\pgfpathcurveto{\pgfqpoint{2.616502in}{3.140985in}}{\pgfqpoint{2.627101in}{3.145375in}}{\pgfqpoint{2.634915in}{3.153189in}}%
\pgfpathcurveto{\pgfqpoint{2.642728in}{3.161003in}}{\pgfqpoint{2.647118in}{3.171602in}}{\pgfqpoint{2.647118in}{3.182652in}}%
\pgfpathcurveto{\pgfqpoint{2.647118in}{3.193702in}}{\pgfqpoint{2.642728in}{3.204301in}}{\pgfqpoint{2.634915in}{3.212115in}}%
\pgfpathcurveto{\pgfqpoint{2.627101in}{3.219928in}}{\pgfqpoint{2.616502in}{3.224319in}}{\pgfqpoint{2.605452in}{3.224319in}}%
\pgfpathcurveto{\pgfqpoint{2.594402in}{3.224319in}}{\pgfqpoint{2.583803in}{3.219928in}}{\pgfqpoint{2.575989in}{3.212115in}}%
\pgfpathcurveto{\pgfqpoint{2.568175in}{3.204301in}}{\pgfqpoint{2.563785in}{3.193702in}}{\pgfqpoint{2.563785in}{3.182652in}}%
\pgfpathcurveto{\pgfqpoint{2.563785in}{3.171602in}}{\pgfqpoint{2.568175in}{3.161003in}}{\pgfqpoint{2.575989in}{3.153189in}}%
\pgfpathcurveto{\pgfqpoint{2.583803in}{3.145375in}}{\pgfqpoint{2.594402in}{3.140985in}}{\pgfqpoint{2.605452in}{3.140985in}}%
\pgfpathclose%
\pgfusepath{stroke,fill}%
\end{pgfscope}%
\begin{pgfscope}%
\pgfpathrectangle{\pgfqpoint{0.648703in}{0.548769in}}{\pgfqpoint{5.112893in}{3.102590in}}%
\pgfusepath{clip}%
\pgfsetbuttcap%
\pgfsetroundjoin%
\definecolor{currentfill}{rgb}{1.000000,0.498039,0.054902}%
\pgfsetfillcolor{currentfill}%
\pgfsetlinewidth{1.003750pt}%
\definecolor{currentstroke}{rgb}{1.000000,0.498039,0.054902}%
\pgfsetstrokecolor{currentstroke}%
\pgfsetdash{}{0pt}%
\pgfpathmoveto{\pgfqpoint{3.968661in}{3.145133in}}%
\pgfpathcurveto{\pgfqpoint{3.979711in}{3.145133in}}{\pgfqpoint{3.990310in}{3.149523in}}{\pgfqpoint{3.998124in}{3.157337in}}%
\pgfpathcurveto{\pgfqpoint{4.005937in}{3.165151in}}{\pgfqpoint{4.010328in}{3.175750in}}{\pgfqpoint{4.010328in}{3.186800in}}%
\pgfpathcurveto{\pgfqpoint{4.010328in}{3.197850in}}{\pgfqpoint{4.005937in}{3.208449in}}{\pgfqpoint{3.998124in}{3.216262in}}%
\pgfpathcurveto{\pgfqpoint{3.990310in}{3.224076in}}{\pgfqpoint{3.979711in}{3.228466in}}{\pgfqpoint{3.968661in}{3.228466in}}%
\pgfpathcurveto{\pgfqpoint{3.957611in}{3.228466in}}{\pgfqpoint{3.947012in}{3.224076in}}{\pgfqpoint{3.939198in}{3.216262in}}%
\pgfpathcurveto{\pgfqpoint{3.931385in}{3.208449in}}{\pgfqpoint{3.926994in}{3.197850in}}{\pgfqpoint{3.926994in}{3.186800in}}%
\pgfpathcurveto{\pgfqpoint{3.926994in}{3.175750in}}{\pgfqpoint{3.931385in}{3.165151in}}{\pgfqpoint{3.939198in}{3.157337in}}%
\pgfpathcurveto{\pgfqpoint{3.947012in}{3.149523in}}{\pgfqpoint{3.957611in}{3.145133in}}{\pgfqpoint{3.968661in}{3.145133in}}%
\pgfpathclose%
\pgfusepath{stroke,fill}%
\end{pgfscope}%
\begin{pgfscope}%
\pgfpathrectangle{\pgfqpoint{0.648703in}{0.548769in}}{\pgfqpoint{5.112893in}{3.102590in}}%
\pgfusepath{clip}%
\pgfsetbuttcap%
\pgfsetroundjoin%
\definecolor{currentfill}{rgb}{0.121569,0.466667,0.705882}%
\pgfsetfillcolor{currentfill}%
\pgfsetlinewidth{1.003750pt}%
\definecolor{currentstroke}{rgb}{0.121569,0.466667,0.705882}%
\pgfsetstrokecolor{currentstroke}%
\pgfsetdash{}{0pt}%
\pgfpathmoveto{\pgfqpoint{2.960698in}{3.132690in}}%
\pgfpathcurveto{\pgfqpoint{2.971748in}{3.132690in}}{\pgfqpoint{2.982347in}{3.137080in}}{\pgfqpoint{2.990161in}{3.144893in}}%
\pgfpathcurveto{\pgfqpoint{2.997974in}{3.152707in}}{\pgfqpoint{3.002365in}{3.163306in}}{\pgfqpoint{3.002365in}{3.174356in}}%
\pgfpathcurveto{\pgfqpoint{3.002365in}{3.185406in}}{\pgfqpoint{2.997974in}{3.196005in}}{\pgfqpoint{2.990161in}{3.203819in}}%
\pgfpathcurveto{\pgfqpoint{2.982347in}{3.211633in}}{\pgfqpoint{2.971748in}{3.216023in}}{\pgfqpoint{2.960698in}{3.216023in}}%
\pgfpathcurveto{\pgfqpoint{2.949648in}{3.216023in}}{\pgfqpoint{2.939049in}{3.211633in}}{\pgfqpoint{2.931235in}{3.203819in}}%
\pgfpathcurveto{\pgfqpoint{2.923422in}{3.196005in}}{\pgfqpoint{2.919031in}{3.185406in}}{\pgfqpoint{2.919031in}{3.174356in}}%
\pgfpathcurveto{\pgfqpoint{2.919031in}{3.163306in}}{\pgfqpoint{2.923422in}{3.152707in}}{\pgfqpoint{2.931235in}{3.144893in}}%
\pgfpathcurveto{\pgfqpoint{2.939049in}{3.137080in}}{\pgfqpoint{2.949648in}{3.132690in}}{\pgfqpoint{2.960698in}{3.132690in}}%
\pgfpathclose%
\pgfusepath{stroke,fill}%
\end{pgfscope}%
\begin{pgfscope}%
\pgfpathrectangle{\pgfqpoint{0.648703in}{0.548769in}}{\pgfqpoint{5.112893in}{3.102590in}}%
\pgfusepath{clip}%
\pgfsetbuttcap%
\pgfsetroundjoin%
\definecolor{currentfill}{rgb}{0.121569,0.466667,0.705882}%
\pgfsetfillcolor{currentfill}%
\pgfsetlinewidth{1.003750pt}%
\definecolor{currentstroke}{rgb}{0.121569,0.466667,0.705882}%
\pgfsetstrokecolor{currentstroke}%
\pgfsetdash{}{0pt}%
\pgfpathmoveto{\pgfqpoint{0.924052in}{0.660572in}}%
\pgfpathcurveto{\pgfqpoint{0.935102in}{0.660572in}}{\pgfqpoint{0.945701in}{0.664963in}}{\pgfqpoint{0.953514in}{0.672776in}}%
\pgfpathcurveto{\pgfqpoint{0.961328in}{0.680590in}}{\pgfqpoint{0.965718in}{0.691189in}}{\pgfqpoint{0.965718in}{0.702239in}}%
\pgfpathcurveto{\pgfqpoint{0.965718in}{0.713289in}}{\pgfqpoint{0.961328in}{0.723888in}}{\pgfqpoint{0.953514in}{0.731702in}}%
\pgfpathcurveto{\pgfqpoint{0.945701in}{0.739516in}}{\pgfqpoint{0.935102in}{0.743906in}}{\pgfqpoint{0.924052in}{0.743906in}}%
\pgfpathcurveto{\pgfqpoint{0.913001in}{0.743906in}}{\pgfqpoint{0.902402in}{0.739516in}}{\pgfqpoint{0.894589in}{0.731702in}}%
\pgfpathcurveto{\pgfqpoint{0.886775in}{0.723888in}}{\pgfqpoint{0.882385in}{0.713289in}}{\pgfqpoint{0.882385in}{0.702239in}}%
\pgfpathcurveto{\pgfqpoint{0.882385in}{0.691189in}}{\pgfqpoint{0.886775in}{0.680590in}}{\pgfqpoint{0.894589in}{0.672776in}}%
\pgfpathcurveto{\pgfqpoint{0.902402in}{0.664963in}}{\pgfqpoint{0.913001in}{0.660572in}}{\pgfqpoint{0.924052in}{0.660572in}}%
\pgfpathclose%
\pgfusepath{stroke,fill}%
\end{pgfscope}%
\begin{pgfscope}%
\pgfpathrectangle{\pgfqpoint{0.648703in}{0.548769in}}{\pgfqpoint{5.112893in}{3.102590in}}%
\pgfusepath{clip}%
\pgfsetbuttcap%
\pgfsetroundjoin%
\definecolor{currentfill}{rgb}{1.000000,0.498039,0.054902}%
\pgfsetfillcolor{currentfill}%
\pgfsetlinewidth{1.003750pt}%
\definecolor{currentstroke}{rgb}{1.000000,0.498039,0.054902}%
\pgfsetstrokecolor{currentstroke}%
\pgfsetdash{}{0pt}%
\pgfpathmoveto{\pgfqpoint{4.294385in}{3.136837in}}%
\pgfpathcurveto{\pgfqpoint{4.305435in}{3.136837in}}{\pgfqpoint{4.316035in}{3.141228in}}{\pgfqpoint{4.323848in}{3.149041in}}%
\pgfpathcurveto{\pgfqpoint{4.331662in}{3.156855in}}{\pgfqpoint{4.336052in}{3.167454in}}{\pgfqpoint{4.336052in}{3.178504in}}%
\pgfpathcurveto{\pgfqpoint{4.336052in}{3.189554in}}{\pgfqpoint{4.331662in}{3.200153in}}{\pgfqpoint{4.323848in}{3.207967in}}%
\pgfpathcurveto{\pgfqpoint{4.316035in}{3.215780in}}{\pgfqpoint{4.305435in}{3.220171in}}{\pgfqpoint{4.294385in}{3.220171in}}%
\pgfpathcurveto{\pgfqpoint{4.283335in}{3.220171in}}{\pgfqpoint{4.272736in}{3.215780in}}{\pgfqpoint{4.264923in}{3.207967in}}%
\pgfpathcurveto{\pgfqpoint{4.257109in}{3.200153in}}{\pgfqpoint{4.252719in}{3.189554in}}{\pgfqpoint{4.252719in}{3.178504in}}%
\pgfpathcurveto{\pgfqpoint{4.252719in}{3.167454in}}{\pgfqpoint{4.257109in}{3.156855in}}{\pgfqpoint{4.264923in}{3.149041in}}%
\pgfpathcurveto{\pgfqpoint{4.272736in}{3.141228in}}{\pgfqpoint{4.283335in}{3.136837in}}{\pgfqpoint{4.294385in}{3.136837in}}%
\pgfpathclose%
\pgfusepath{stroke,fill}%
\end{pgfscope}%
\begin{pgfscope}%
\pgfpathrectangle{\pgfqpoint{0.648703in}{0.548769in}}{\pgfqpoint{5.112893in}{3.102590in}}%
\pgfusepath{clip}%
\pgfsetbuttcap%
\pgfsetroundjoin%
\definecolor{currentfill}{rgb}{1.000000,0.498039,0.054902}%
\pgfsetfillcolor{currentfill}%
\pgfsetlinewidth{1.003750pt}%
\definecolor{currentstroke}{rgb}{1.000000,0.498039,0.054902}%
\pgfsetstrokecolor{currentstroke}%
\pgfsetdash{}{0pt}%
\pgfpathmoveto{\pgfqpoint{4.245531in}{3.157577in}}%
\pgfpathcurveto{\pgfqpoint{4.256581in}{3.157577in}}{\pgfqpoint{4.267180in}{3.161967in}}{\pgfqpoint{4.274994in}{3.169780in}}%
\pgfpathcurveto{\pgfqpoint{4.282808in}{3.177594in}}{\pgfqpoint{4.287198in}{3.188193in}}{\pgfqpoint{4.287198in}{3.199243in}}%
\pgfpathcurveto{\pgfqpoint{4.287198in}{3.210293in}}{\pgfqpoint{4.282808in}{3.220892in}}{\pgfqpoint{4.274994in}{3.228706in}}%
\pgfpathcurveto{\pgfqpoint{4.267180in}{3.236520in}}{\pgfqpoint{4.256581in}{3.240910in}}{\pgfqpoint{4.245531in}{3.240910in}}%
\pgfpathcurveto{\pgfqpoint{4.234481in}{3.240910in}}{\pgfqpoint{4.223882in}{3.236520in}}{\pgfqpoint{4.216068in}{3.228706in}}%
\pgfpathcurveto{\pgfqpoint{4.208255in}{3.220892in}}{\pgfqpoint{4.203864in}{3.210293in}}{\pgfqpoint{4.203864in}{3.199243in}}%
\pgfpathcurveto{\pgfqpoint{4.203864in}{3.188193in}}{\pgfqpoint{4.208255in}{3.177594in}}{\pgfqpoint{4.216068in}{3.169780in}}%
\pgfpathcurveto{\pgfqpoint{4.223882in}{3.161967in}}{\pgfqpoint{4.234481in}{3.157577in}}{\pgfqpoint{4.245531in}{3.157577in}}%
\pgfpathclose%
\pgfusepath{stroke,fill}%
\end{pgfscope}%
\begin{pgfscope}%
\pgfpathrectangle{\pgfqpoint{0.648703in}{0.548769in}}{\pgfqpoint{5.112893in}{3.102590in}}%
\pgfusepath{clip}%
\pgfsetbuttcap%
\pgfsetroundjoin%
\definecolor{currentfill}{rgb}{0.121569,0.466667,0.705882}%
\pgfsetfillcolor{currentfill}%
\pgfsetlinewidth{1.003750pt}%
\definecolor{currentstroke}{rgb}{0.121569,0.466667,0.705882}%
\pgfsetstrokecolor{currentstroke}%
\pgfsetdash{}{0pt}%
\pgfpathmoveto{\pgfqpoint{0.873129in}{0.648129in}}%
\pgfpathcurveto{\pgfqpoint{0.884180in}{0.648129in}}{\pgfqpoint{0.894779in}{0.652519in}}{\pgfqpoint{0.902592in}{0.660333in}}%
\pgfpathcurveto{\pgfqpoint{0.910406in}{0.668146in}}{\pgfqpoint{0.914796in}{0.678745in}}{\pgfqpoint{0.914796in}{0.689796in}}%
\pgfpathcurveto{\pgfqpoint{0.914796in}{0.700846in}}{\pgfqpoint{0.910406in}{0.711445in}}{\pgfqpoint{0.902592in}{0.719258in}}%
\pgfpathcurveto{\pgfqpoint{0.894779in}{0.727072in}}{\pgfqpoint{0.884180in}{0.731462in}}{\pgfqpoint{0.873129in}{0.731462in}}%
\pgfpathcurveto{\pgfqpoint{0.862079in}{0.731462in}}{\pgfqpoint{0.851480in}{0.727072in}}{\pgfqpoint{0.843667in}{0.719258in}}%
\pgfpathcurveto{\pgfqpoint{0.835853in}{0.711445in}}{\pgfqpoint{0.831463in}{0.700846in}}{\pgfqpoint{0.831463in}{0.689796in}}%
\pgfpathcurveto{\pgfqpoint{0.831463in}{0.678745in}}{\pgfqpoint{0.835853in}{0.668146in}}{\pgfqpoint{0.843667in}{0.660333in}}%
\pgfpathcurveto{\pgfqpoint{0.851480in}{0.652519in}}{\pgfqpoint{0.862079in}{0.648129in}}{\pgfqpoint{0.873129in}{0.648129in}}%
\pgfpathclose%
\pgfusepath{stroke,fill}%
\end{pgfscope}%
\begin{pgfscope}%
\pgfpathrectangle{\pgfqpoint{0.648703in}{0.548769in}}{\pgfqpoint{5.112893in}{3.102590in}}%
\pgfusepath{clip}%
\pgfsetbuttcap%
\pgfsetroundjoin%
\definecolor{currentfill}{rgb}{0.121569,0.466667,0.705882}%
\pgfsetfillcolor{currentfill}%
\pgfsetlinewidth{1.003750pt}%
\definecolor{currentstroke}{rgb}{0.121569,0.466667,0.705882}%
\pgfsetstrokecolor{currentstroke}%
\pgfsetdash{}{0pt}%
\pgfpathmoveto{\pgfqpoint{0.876392in}{0.648129in}}%
\pgfpathcurveto{\pgfqpoint{0.887442in}{0.648129in}}{\pgfqpoint{0.898041in}{0.652519in}}{\pgfqpoint{0.905855in}{0.660333in}}%
\pgfpathcurveto{\pgfqpoint{0.913669in}{0.668146in}}{\pgfqpoint{0.918059in}{0.678745in}}{\pgfqpoint{0.918059in}{0.689796in}}%
\pgfpathcurveto{\pgfqpoint{0.918059in}{0.700846in}}{\pgfqpoint{0.913669in}{0.711445in}}{\pgfqpoint{0.905855in}{0.719258in}}%
\pgfpathcurveto{\pgfqpoint{0.898041in}{0.727072in}}{\pgfqpoint{0.887442in}{0.731462in}}{\pgfqpoint{0.876392in}{0.731462in}}%
\pgfpathcurveto{\pgfqpoint{0.865342in}{0.731462in}}{\pgfqpoint{0.854743in}{0.727072in}}{\pgfqpoint{0.846930in}{0.719258in}}%
\pgfpathcurveto{\pgfqpoint{0.839116in}{0.711445in}}{\pgfqpoint{0.834726in}{0.700846in}}{\pgfqpoint{0.834726in}{0.689796in}}%
\pgfpathcurveto{\pgfqpoint{0.834726in}{0.678745in}}{\pgfqpoint{0.839116in}{0.668146in}}{\pgfqpoint{0.846930in}{0.660333in}}%
\pgfpathcurveto{\pgfqpoint{0.854743in}{0.652519in}}{\pgfqpoint{0.865342in}{0.648129in}}{\pgfqpoint{0.876392in}{0.648129in}}%
\pgfpathclose%
\pgfusepath{stroke,fill}%
\end{pgfscope}%
\begin{pgfscope}%
\pgfpathrectangle{\pgfqpoint{0.648703in}{0.548769in}}{\pgfqpoint{5.112893in}{3.102590in}}%
\pgfusepath{clip}%
\pgfsetbuttcap%
\pgfsetroundjoin%
\definecolor{currentfill}{rgb}{0.121569,0.466667,0.705882}%
\pgfsetfillcolor{currentfill}%
\pgfsetlinewidth{1.003750pt}%
\definecolor{currentstroke}{rgb}{0.121569,0.466667,0.705882}%
\pgfsetstrokecolor{currentstroke}%
\pgfsetdash{}{0pt}%
\pgfpathmoveto{\pgfqpoint{4.048078in}{3.132690in}}%
\pgfpathcurveto{\pgfqpoint{4.059128in}{3.132690in}}{\pgfqpoint{4.069727in}{3.137080in}}{\pgfqpoint{4.077541in}{3.144893in}}%
\pgfpathcurveto{\pgfqpoint{4.085354in}{3.152707in}}{\pgfqpoint{4.089745in}{3.163306in}}{\pgfqpoint{4.089745in}{3.174356in}}%
\pgfpathcurveto{\pgfqpoint{4.089745in}{3.185406in}}{\pgfqpoint{4.085354in}{3.196005in}}{\pgfqpoint{4.077541in}{3.203819in}}%
\pgfpathcurveto{\pgfqpoint{4.069727in}{3.211633in}}{\pgfqpoint{4.059128in}{3.216023in}}{\pgfqpoint{4.048078in}{3.216023in}}%
\pgfpathcurveto{\pgfqpoint{4.037028in}{3.216023in}}{\pgfqpoint{4.026429in}{3.211633in}}{\pgfqpoint{4.018615in}{3.203819in}}%
\pgfpathcurveto{\pgfqpoint{4.010801in}{3.196005in}}{\pgfqpoint{4.006411in}{3.185406in}}{\pgfqpoint{4.006411in}{3.174356in}}%
\pgfpathcurveto{\pgfqpoint{4.006411in}{3.163306in}}{\pgfqpoint{4.010801in}{3.152707in}}{\pgfqpoint{4.018615in}{3.144893in}}%
\pgfpathcurveto{\pgfqpoint{4.026429in}{3.137080in}}{\pgfqpoint{4.037028in}{3.132690in}}{\pgfqpoint{4.048078in}{3.132690in}}%
\pgfpathclose%
\pgfusepath{stroke,fill}%
\end{pgfscope}%
\begin{pgfscope}%
\pgfpathrectangle{\pgfqpoint{0.648703in}{0.548769in}}{\pgfqpoint{5.112893in}{3.102590in}}%
\pgfusepath{clip}%
\pgfsetbuttcap%
\pgfsetroundjoin%
\definecolor{currentfill}{rgb}{1.000000,0.498039,0.054902}%
\pgfsetfillcolor{currentfill}%
\pgfsetlinewidth{1.003750pt}%
\definecolor{currentstroke}{rgb}{1.000000,0.498039,0.054902}%
\pgfsetstrokecolor{currentstroke}%
\pgfsetdash{}{0pt}%
\pgfpathmoveto{\pgfqpoint{4.099220in}{3.219794in}}%
\pgfpathcurveto{\pgfqpoint{4.110270in}{3.219794in}}{\pgfqpoint{4.120869in}{3.224185in}}{\pgfqpoint{4.128682in}{3.231998in}}%
\pgfpathcurveto{\pgfqpoint{4.136496in}{3.239812in}}{\pgfqpoint{4.140886in}{3.250411in}}{\pgfqpoint{4.140886in}{3.261461in}}%
\pgfpathcurveto{\pgfqpoint{4.140886in}{3.272511in}}{\pgfqpoint{4.136496in}{3.283110in}}{\pgfqpoint{4.128682in}{3.290924in}}%
\pgfpathcurveto{\pgfqpoint{4.120869in}{3.298737in}}{\pgfqpoint{4.110270in}{3.303128in}}{\pgfqpoint{4.099220in}{3.303128in}}%
\pgfpathcurveto{\pgfqpoint{4.088169in}{3.303128in}}{\pgfqpoint{4.077570in}{3.298737in}}{\pgfqpoint{4.069757in}{3.290924in}}%
\pgfpathcurveto{\pgfqpoint{4.061943in}{3.283110in}}{\pgfqpoint{4.057553in}{3.272511in}}{\pgfqpoint{4.057553in}{3.261461in}}%
\pgfpathcurveto{\pgfqpoint{4.057553in}{3.250411in}}{\pgfqpoint{4.061943in}{3.239812in}}{\pgfqpoint{4.069757in}{3.231998in}}%
\pgfpathcurveto{\pgfqpoint{4.077570in}{3.224185in}}{\pgfqpoint{4.088169in}{3.219794in}}{\pgfqpoint{4.099220in}{3.219794in}}%
\pgfpathclose%
\pgfusepath{stroke,fill}%
\end{pgfscope}%
\begin{pgfscope}%
\pgfpathrectangle{\pgfqpoint{0.648703in}{0.548769in}}{\pgfqpoint{5.112893in}{3.102590in}}%
\pgfusepath{clip}%
\pgfsetbuttcap%
\pgfsetroundjoin%
\definecolor{currentfill}{rgb}{1.000000,0.498039,0.054902}%
\pgfsetfillcolor{currentfill}%
\pgfsetlinewidth{1.003750pt}%
\definecolor{currentstroke}{rgb}{1.000000,0.498039,0.054902}%
\pgfsetstrokecolor{currentstroke}%
\pgfsetdash{}{0pt}%
\pgfpathmoveto{\pgfqpoint{3.432399in}{3.140985in}}%
\pgfpathcurveto{\pgfqpoint{3.443449in}{3.140985in}}{\pgfqpoint{3.454048in}{3.145375in}}{\pgfqpoint{3.461862in}{3.153189in}}%
\pgfpathcurveto{\pgfqpoint{3.469675in}{3.161003in}}{\pgfqpoint{3.474066in}{3.171602in}}{\pgfqpoint{3.474066in}{3.182652in}}%
\pgfpathcurveto{\pgfqpoint{3.474066in}{3.193702in}}{\pgfqpoint{3.469675in}{3.204301in}}{\pgfqpoint{3.461862in}{3.212115in}}%
\pgfpathcurveto{\pgfqpoint{3.454048in}{3.219928in}}{\pgfqpoint{3.443449in}{3.224319in}}{\pgfqpoint{3.432399in}{3.224319in}}%
\pgfpathcurveto{\pgfqpoint{3.421349in}{3.224319in}}{\pgfqpoint{3.410750in}{3.219928in}}{\pgfqpoint{3.402936in}{3.212115in}}%
\pgfpathcurveto{\pgfqpoint{3.395123in}{3.204301in}}{\pgfqpoint{3.390732in}{3.193702in}}{\pgfqpoint{3.390732in}{3.182652in}}%
\pgfpathcurveto{\pgfqpoint{3.390732in}{3.171602in}}{\pgfqpoint{3.395123in}{3.161003in}}{\pgfqpoint{3.402936in}{3.153189in}}%
\pgfpathcurveto{\pgfqpoint{3.410750in}{3.145375in}}{\pgfqpoint{3.421349in}{3.140985in}}{\pgfqpoint{3.432399in}{3.140985in}}%
\pgfpathclose%
\pgfusepath{stroke,fill}%
\end{pgfscope}%
\begin{pgfscope}%
\pgfpathrectangle{\pgfqpoint{0.648703in}{0.548769in}}{\pgfqpoint{5.112893in}{3.102590in}}%
\pgfusepath{clip}%
\pgfsetbuttcap%
\pgfsetroundjoin%
\definecolor{currentfill}{rgb}{1.000000,0.498039,0.054902}%
\pgfsetfillcolor{currentfill}%
\pgfsetlinewidth{1.003750pt}%
\definecolor{currentstroke}{rgb}{1.000000,0.498039,0.054902}%
\pgfsetstrokecolor{currentstroke}%
\pgfsetdash{}{0pt}%
\pgfpathmoveto{\pgfqpoint{2.594286in}{3.145133in}}%
\pgfpathcurveto{\pgfqpoint{2.605336in}{3.145133in}}{\pgfqpoint{2.615935in}{3.149523in}}{\pgfqpoint{2.623748in}{3.157337in}}%
\pgfpathcurveto{\pgfqpoint{2.631562in}{3.165151in}}{\pgfqpoint{2.635952in}{3.175750in}}{\pgfqpoint{2.635952in}{3.186800in}}%
\pgfpathcurveto{\pgfqpoint{2.635952in}{3.197850in}}{\pgfqpoint{2.631562in}{3.208449in}}{\pgfqpoint{2.623748in}{3.216262in}}%
\pgfpathcurveto{\pgfqpoint{2.615935in}{3.224076in}}{\pgfqpoint{2.605336in}{3.228466in}}{\pgfqpoint{2.594286in}{3.228466in}}%
\pgfpathcurveto{\pgfqpoint{2.583235in}{3.228466in}}{\pgfqpoint{2.572636in}{3.224076in}}{\pgfqpoint{2.564823in}{3.216262in}}%
\pgfpathcurveto{\pgfqpoint{2.557009in}{3.208449in}}{\pgfqpoint{2.552619in}{3.197850in}}{\pgfqpoint{2.552619in}{3.186800in}}%
\pgfpathcurveto{\pgfqpoint{2.552619in}{3.175750in}}{\pgfqpoint{2.557009in}{3.165151in}}{\pgfqpoint{2.564823in}{3.157337in}}%
\pgfpathcurveto{\pgfqpoint{2.572636in}{3.149523in}}{\pgfqpoint{2.583235in}{3.145133in}}{\pgfqpoint{2.594286in}{3.145133in}}%
\pgfpathclose%
\pgfusepath{stroke,fill}%
\end{pgfscope}%
\begin{pgfscope}%
\pgfpathrectangle{\pgfqpoint{0.648703in}{0.548769in}}{\pgfqpoint{5.112893in}{3.102590in}}%
\pgfusepath{clip}%
\pgfsetbuttcap%
\pgfsetroundjoin%
\definecolor{currentfill}{rgb}{1.000000,0.498039,0.054902}%
\pgfsetfillcolor{currentfill}%
\pgfsetlinewidth{1.003750pt}%
\definecolor{currentstroke}{rgb}{1.000000,0.498039,0.054902}%
\pgfsetstrokecolor{currentstroke}%
\pgfsetdash{}{0pt}%
\pgfpathmoveto{\pgfqpoint{4.174432in}{3.136837in}}%
\pgfpathcurveto{\pgfqpoint{4.185482in}{3.136837in}}{\pgfqpoint{4.196081in}{3.141228in}}{\pgfqpoint{4.203895in}{3.149041in}}%
\pgfpathcurveto{\pgfqpoint{4.211708in}{3.156855in}}{\pgfqpoint{4.216098in}{3.167454in}}{\pgfqpoint{4.216098in}{3.178504in}}%
\pgfpathcurveto{\pgfqpoint{4.216098in}{3.189554in}}{\pgfqpoint{4.211708in}{3.200153in}}{\pgfqpoint{4.203895in}{3.207967in}}%
\pgfpathcurveto{\pgfqpoint{4.196081in}{3.215780in}}{\pgfqpoint{4.185482in}{3.220171in}}{\pgfqpoint{4.174432in}{3.220171in}}%
\pgfpathcurveto{\pgfqpoint{4.163382in}{3.220171in}}{\pgfqpoint{4.152783in}{3.215780in}}{\pgfqpoint{4.144969in}{3.207967in}}%
\pgfpathcurveto{\pgfqpoint{4.137155in}{3.200153in}}{\pgfqpoint{4.132765in}{3.189554in}}{\pgfqpoint{4.132765in}{3.178504in}}%
\pgfpathcurveto{\pgfqpoint{4.132765in}{3.167454in}}{\pgfqpoint{4.137155in}{3.156855in}}{\pgfqpoint{4.144969in}{3.149041in}}%
\pgfpathcurveto{\pgfqpoint{4.152783in}{3.141228in}}{\pgfqpoint{4.163382in}{3.136837in}}{\pgfqpoint{4.174432in}{3.136837in}}%
\pgfpathclose%
\pgfusepath{stroke,fill}%
\end{pgfscope}%
\begin{pgfscope}%
\pgfpathrectangle{\pgfqpoint{0.648703in}{0.548769in}}{\pgfqpoint{5.112893in}{3.102590in}}%
\pgfusepath{clip}%
\pgfsetbuttcap%
\pgfsetroundjoin%
\definecolor{currentfill}{rgb}{0.121569,0.466667,0.705882}%
\pgfsetfillcolor{currentfill}%
\pgfsetlinewidth{1.003750pt}%
\definecolor{currentstroke}{rgb}{0.121569,0.466667,0.705882}%
\pgfsetstrokecolor{currentstroke}%
\pgfsetdash{}{0pt}%
\pgfpathmoveto{\pgfqpoint{0.873125in}{0.648129in}}%
\pgfpathcurveto{\pgfqpoint{0.884175in}{0.648129in}}{\pgfqpoint{0.894774in}{0.652519in}}{\pgfqpoint{0.902588in}{0.660333in}}%
\pgfpathcurveto{\pgfqpoint{0.910402in}{0.668146in}}{\pgfqpoint{0.914792in}{0.678745in}}{\pgfqpoint{0.914792in}{0.689796in}}%
\pgfpathcurveto{\pgfqpoint{0.914792in}{0.700846in}}{\pgfqpoint{0.910402in}{0.711445in}}{\pgfqpoint{0.902588in}{0.719258in}}%
\pgfpathcurveto{\pgfqpoint{0.894774in}{0.727072in}}{\pgfqpoint{0.884175in}{0.731462in}}{\pgfqpoint{0.873125in}{0.731462in}}%
\pgfpathcurveto{\pgfqpoint{0.862075in}{0.731462in}}{\pgfqpoint{0.851476in}{0.727072in}}{\pgfqpoint{0.843662in}{0.719258in}}%
\pgfpathcurveto{\pgfqpoint{0.835849in}{0.711445in}}{\pgfqpoint{0.831458in}{0.700846in}}{\pgfqpoint{0.831458in}{0.689796in}}%
\pgfpathcurveto{\pgfqpoint{0.831458in}{0.678745in}}{\pgfqpoint{0.835849in}{0.668146in}}{\pgfqpoint{0.843662in}{0.660333in}}%
\pgfpathcurveto{\pgfqpoint{0.851476in}{0.652519in}}{\pgfqpoint{0.862075in}{0.648129in}}{\pgfqpoint{0.873125in}{0.648129in}}%
\pgfpathclose%
\pgfusepath{stroke,fill}%
\end{pgfscope}%
\begin{pgfscope}%
\pgfpathrectangle{\pgfqpoint{0.648703in}{0.548769in}}{\pgfqpoint{5.112893in}{3.102590in}}%
\pgfusepath{clip}%
\pgfsetbuttcap%
\pgfsetroundjoin%
\definecolor{currentfill}{rgb}{0.121569,0.466667,0.705882}%
\pgfsetfillcolor{currentfill}%
\pgfsetlinewidth{1.003750pt}%
\definecolor{currentstroke}{rgb}{0.121569,0.466667,0.705882}%
\pgfsetstrokecolor{currentstroke}%
\pgfsetdash{}{0pt}%
\pgfpathmoveto{\pgfqpoint{4.275665in}{3.132690in}}%
\pgfpathcurveto{\pgfqpoint{4.286715in}{3.132690in}}{\pgfqpoint{4.297314in}{3.137080in}}{\pgfqpoint{4.305127in}{3.144893in}}%
\pgfpathcurveto{\pgfqpoint{4.312941in}{3.152707in}}{\pgfqpoint{4.317331in}{3.163306in}}{\pgfqpoint{4.317331in}{3.174356in}}%
\pgfpathcurveto{\pgfqpoint{4.317331in}{3.185406in}}{\pgfqpoint{4.312941in}{3.196005in}}{\pgfqpoint{4.305127in}{3.203819in}}%
\pgfpathcurveto{\pgfqpoint{4.297314in}{3.211633in}}{\pgfqpoint{4.286715in}{3.216023in}}{\pgfqpoint{4.275665in}{3.216023in}}%
\pgfpathcurveto{\pgfqpoint{4.264614in}{3.216023in}}{\pgfqpoint{4.254015in}{3.211633in}}{\pgfqpoint{4.246202in}{3.203819in}}%
\pgfpathcurveto{\pgfqpoint{4.238388in}{3.196005in}}{\pgfqpoint{4.233998in}{3.185406in}}{\pgfqpoint{4.233998in}{3.174356in}}%
\pgfpathcurveto{\pgfqpoint{4.233998in}{3.163306in}}{\pgfqpoint{4.238388in}{3.152707in}}{\pgfqpoint{4.246202in}{3.144893in}}%
\pgfpathcurveto{\pgfqpoint{4.254015in}{3.137080in}}{\pgfqpoint{4.264614in}{3.132690in}}{\pgfqpoint{4.275665in}{3.132690in}}%
\pgfpathclose%
\pgfusepath{stroke,fill}%
\end{pgfscope}%
\begin{pgfscope}%
\pgfpathrectangle{\pgfqpoint{0.648703in}{0.548769in}}{\pgfqpoint{5.112893in}{3.102590in}}%
\pgfusepath{clip}%
\pgfsetbuttcap%
\pgfsetroundjoin%
\definecolor{currentfill}{rgb}{1.000000,0.498039,0.054902}%
\pgfsetfillcolor{currentfill}%
\pgfsetlinewidth{1.003750pt}%
\definecolor{currentstroke}{rgb}{1.000000,0.498039,0.054902}%
\pgfsetstrokecolor{currentstroke}%
\pgfsetdash{}{0pt}%
\pgfpathmoveto{\pgfqpoint{4.365216in}{3.145133in}}%
\pgfpathcurveto{\pgfqpoint{4.376266in}{3.145133in}}{\pgfqpoint{4.386865in}{3.149523in}}{\pgfqpoint{4.394678in}{3.157337in}}%
\pgfpathcurveto{\pgfqpoint{4.402492in}{3.165151in}}{\pgfqpoint{4.406882in}{3.175750in}}{\pgfqpoint{4.406882in}{3.186800in}}%
\pgfpathcurveto{\pgfqpoint{4.406882in}{3.197850in}}{\pgfqpoint{4.402492in}{3.208449in}}{\pgfqpoint{4.394678in}{3.216262in}}%
\pgfpathcurveto{\pgfqpoint{4.386865in}{3.224076in}}{\pgfqpoint{4.376266in}{3.228466in}}{\pgfqpoint{4.365216in}{3.228466in}}%
\pgfpathcurveto{\pgfqpoint{4.354165in}{3.228466in}}{\pgfqpoint{4.343566in}{3.224076in}}{\pgfqpoint{4.335753in}{3.216262in}}%
\pgfpathcurveto{\pgfqpoint{4.327939in}{3.208449in}}{\pgfqpoint{4.323549in}{3.197850in}}{\pgfqpoint{4.323549in}{3.186800in}}%
\pgfpathcurveto{\pgfqpoint{4.323549in}{3.175750in}}{\pgfqpoint{4.327939in}{3.165151in}}{\pgfqpoint{4.335753in}{3.157337in}}%
\pgfpathcurveto{\pgfqpoint{4.343566in}{3.149523in}}{\pgfqpoint{4.354165in}{3.145133in}}{\pgfqpoint{4.365216in}{3.145133in}}%
\pgfpathclose%
\pgfusepath{stroke,fill}%
\end{pgfscope}%
\begin{pgfscope}%
\pgfpathrectangle{\pgfqpoint{0.648703in}{0.548769in}}{\pgfqpoint{5.112893in}{3.102590in}}%
\pgfusepath{clip}%
\pgfsetbuttcap%
\pgfsetroundjoin%
\definecolor{currentfill}{rgb}{0.121569,0.466667,0.705882}%
\pgfsetfillcolor{currentfill}%
\pgfsetlinewidth{1.003750pt}%
\definecolor{currentstroke}{rgb}{0.121569,0.466667,0.705882}%
\pgfsetstrokecolor{currentstroke}%
\pgfsetdash{}{0pt}%
\pgfpathmoveto{\pgfqpoint{2.607751in}{3.132690in}}%
\pgfpathcurveto{\pgfqpoint{2.618801in}{3.132690in}}{\pgfqpoint{2.629400in}{3.137080in}}{\pgfqpoint{2.637214in}{3.144893in}}%
\pgfpathcurveto{\pgfqpoint{2.645027in}{3.152707in}}{\pgfqpoint{2.649417in}{3.163306in}}{\pgfqpoint{2.649417in}{3.174356in}}%
\pgfpathcurveto{\pgfqpoint{2.649417in}{3.185406in}}{\pgfqpoint{2.645027in}{3.196005in}}{\pgfqpoint{2.637214in}{3.203819in}}%
\pgfpathcurveto{\pgfqpoint{2.629400in}{3.211633in}}{\pgfqpoint{2.618801in}{3.216023in}}{\pgfqpoint{2.607751in}{3.216023in}}%
\pgfpathcurveto{\pgfqpoint{2.596701in}{3.216023in}}{\pgfqpoint{2.586102in}{3.211633in}}{\pgfqpoint{2.578288in}{3.203819in}}%
\pgfpathcurveto{\pgfqpoint{2.570474in}{3.196005in}}{\pgfqpoint{2.566084in}{3.185406in}}{\pgfqpoint{2.566084in}{3.174356in}}%
\pgfpathcurveto{\pgfqpoint{2.566084in}{3.163306in}}{\pgfqpoint{2.570474in}{3.152707in}}{\pgfqpoint{2.578288in}{3.144893in}}%
\pgfpathcurveto{\pgfqpoint{2.586102in}{3.137080in}}{\pgfqpoint{2.596701in}{3.132690in}}{\pgfqpoint{2.607751in}{3.132690in}}%
\pgfpathclose%
\pgfusepath{stroke,fill}%
\end{pgfscope}%
\begin{pgfscope}%
\pgfpathrectangle{\pgfqpoint{0.648703in}{0.548769in}}{\pgfqpoint{5.112893in}{3.102590in}}%
\pgfusepath{clip}%
\pgfsetbuttcap%
\pgfsetroundjoin%
\definecolor{currentfill}{rgb}{1.000000,0.498039,0.054902}%
\pgfsetfillcolor{currentfill}%
\pgfsetlinewidth{1.003750pt}%
\definecolor{currentstroke}{rgb}{1.000000,0.498039,0.054902}%
\pgfsetstrokecolor{currentstroke}%
\pgfsetdash{}{0pt}%
\pgfpathmoveto{\pgfqpoint{5.361255in}{3.157577in}}%
\pgfpathcurveto{\pgfqpoint{5.372305in}{3.157577in}}{\pgfqpoint{5.382904in}{3.161967in}}{\pgfqpoint{5.390718in}{3.169780in}}%
\pgfpathcurveto{\pgfqpoint{5.398531in}{3.177594in}}{\pgfqpoint{5.402922in}{3.188193in}}{\pgfqpoint{5.402922in}{3.199243in}}%
\pgfpathcurveto{\pgfqpoint{5.402922in}{3.210293in}}{\pgfqpoint{5.398531in}{3.220892in}}{\pgfqpoint{5.390718in}{3.228706in}}%
\pgfpathcurveto{\pgfqpoint{5.382904in}{3.236520in}}{\pgfqpoint{5.372305in}{3.240910in}}{\pgfqpoint{5.361255in}{3.240910in}}%
\pgfpathcurveto{\pgfqpoint{5.350205in}{3.240910in}}{\pgfqpoint{5.339606in}{3.236520in}}{\pgfqpoint{5.331792in}{3.228706in}}%
\pgfpathcurveto{\pgfqpoint{5.323979in}{3.220892in}}{\pgfqpoint{5.319588in}{3.210293in}}{\pgfqpoint{5.319588in}{3.199243in}}%
\pgfpathcurveto{\pgfqpoint{5.319588in}{3.188193in}}{\pgfqpoint{5.323979in}{3.177594in}}{\pgfqpoint{5.331792in}{3.169780in}}%
\pgfpathcurveto{\pgfqpoint{5.339606in}{3.161967in}}{\pgfqpoint{5.350205in}{3.157577in}}{\pgfqpoint{5.361255in}{3.157577in}}%
\pgfpathclose%
\pgfusepath{stroke,fill}%
\end{pgfscope}%
\begin{pgfscope}%
\pgfpathrectangle{\pgfqpoint{0.648703in}{0.548769in}}{\pgfqpoint{5.112893in}{3.102590in}}%
\pgfusepath{clip}%
\pgfsetbuttcap%
\pgfsetroundjoin%
\definecolor{currentfill}{rgb}{1.000000,0.498039,0.054902}%
\pgfsetfillcolor{currentfill}%
\pgfsetlinewidth{1.003750pt}%
\definecolor{currentstroke}{rgb}{1.000000,0.498039,0.054902}%
\pgfsetstrokecolor{currentstroke}%
\pgfsetdash{}{0pt}%
\pgfpathmoveto{\pgfqpoint{4.152046in}{3.145133in}}%
\pgfpathcurveto{\pgfqpoint{4.163096in}{3.145133in}}{\pgfqpoint{4.173695in}{3.149523in}}{\pgfqpoint{4.181509in}{3.157337in}}%
\pgfpathcurveto{\pgfqpoint{4.189322in}{3.165151in}}{\pgfqpoint{4.193713in}{3.175750in}}{\pgfqpoint{4.193713in}{3.186800in}}%
\pgfpathcurveto{\pgfqpoint{4.193713in}{3.197850in}}{\pgfqpoint{4.189322in}{3.208449in}}{\pgfqpoint{4.181509in}{3.216262in}}%
\pgfpathcurveto{\pgfqpoint{4.173695in}{3.224076in}}{\pgfqpoint{4.163096in}{3.228466in}}{\pgfqpoint{4.152046in}{3.228466in}}%
\pgfpathcurveto{\pgfqpoint{4.140996in}{3.228466in}}{\pgfqpoint{4.130397in}{3.224076in}}{\pgfqpoint{4.122583in}{3.216262in}}%
\pgfpathcurveto{\pgfqpoint{4.114769in}{3.208449in}}{\pgfqpoint{4.110379in}{3.197850in}}{\pgfqpoint{4.110379in}{3.186800in}}%
\pgfpathcurveto{\pgfqpoint{4.110379in}{3.175750in}}{\pgfqpoint{4.114769in}{3.165151in}}{\pgfqpoint{4.122583in}{3.157337in}}%
\pgfpathcurveto{\pgfqpoint{4.130397in}{3.149523in}}{\pgfqpoint{4.140996in}{3.145133in}}{\pgfqpoint{4.152046in}{3.145133in}}%
\pgfpathclose%
\pgfusepath{stroke,fill}%
\end{pgfscope}%
\begin{pgfscope}%
\pgfpathrectangle{\pgfqpoint{0.648703in}{0.548769in}}{\pgfqpoint{5.112893in}{3.102590in}}%
\pgfusepath{clip}%
\pgfsetbuttcap%
\pgfsetroundjoin%
\definecolor{currentfill}{rgb}{1.000000,0.498039,0.054902}%
\pgfsetfillcolor{currentfill}%
\pgfsetlinewidth{1.003750pt}%
\definecolor{currentstroke}{rgb}{1.000000,0.498039,0.054902}%
\pgfsetstrokecolor{currentstroke}%
\pgfsetdash{}{0pt}%
\pgfpathmoveto{\pgfqpoint{4.302207in}{3.190759in}}%
\pgfpathcurveto{\pgfqpoint{4.313257in}{3.190759in}}{\pgfqpoint{4.323856in}{3.195150in}}{\pgfqpoint{4.331670in}{3.202963in}}%
\pgfpathcurveto{\pgfqpoint{4.339484in}{3.210777in}}{\pgfqpoint{4.343874in}{3.221376in}}{\pgfqpoint{4.343874in}{3.232426in}}%
\pgfpathcurveto{\pgfqpoint{4.343874in}{3.243476in}}{\pgfqpoint{4.339484in}{3.254075in}}{\pgfqpoint{4.331670in}{3.261889in}}%
\pgfpathcurveto{\pgfqpoint{4.323856in}{3.269702in}}{\pgfqpoint{4.313257in}{3.274093in}}{\pgfqpoint{4.302207in}{3.274093in}}%
\pgfpathcurveto{\pgfqpoint{4.291157in}{3.274093in}}{\pgfqpoint{4.280558in}{3.269702in}}{\pgfqpoint{4.272744in}{3.261889in}}%
\pgfpathcurveto{\pgfqpoint{4.264931in}{3.254075in}}{\pgfqpoint{4.260541in}{3.243476in}}{\pgfqpoint{4.260541in}{3.232426in}}%
\pgfpathcurveto{\pgfqpoint{4.260541in}{3.221376in}}{\pgfqpoint{4.264931in}{3.210777in}}{\pgfqpoint{4.272744in}{3.202963in}}%
\pgfpathcurveto{\pgfqpoint{4.280558in}{3.195150in}}{\pgfqpoint{4.291157in}{3.190759in}}{\pgfqpoint{4.302207in}{3.190759in}}%
\pgfpathclose%
\pgfusepath{stroke,fill}%
\end{pgfscope}%
\begin{pgfscope}%
\pgfpathrectangle{\pgfqpoint{0.648703in}{0.548769in}}{\pgfqpoint{5.112893in}{3.102590in}}%
\pgfusepath{clip}%
\pgfsetbuttcap%
\pgfsetroundjoin%
\definecolor{currentfill}{rgb}{1.000000,0.498039,0.054902}%
\pgfsetfillcolor{currentfill}%
\pgfsetlinewidth{1.003750pt}%
\definecolor{currentstroke}{rgb}{1.000000,0.498039,0.054902}%
\pgfsetstrokecolor{currentstroke}%
\pgfsetdash{}{0pt}%
\pgfpathmoveto{\pgfqpoint{4.045968in}{3.207351in}}%
\pgfpathcurveto{\pgfqpoint{4.057018in}{3.207351in}}{\pgfqpoint{4.067617in}{3.211741in}}{\pgfqpoint{4.075430in}{3.219555in}}%
\pgfpathcurveto{\pgfqpoint{4.083244in}{3.227368in}}{\pgfqpoint{4.087634in}{3.237967in}}{\pgfqpoint{4.087634in}{3.249017in}}%
\pgfpathcurveto{\pgfqpoint{4.087634in}{3.260068in}}{\pgfqpoint{4.083244in}{3.270667in}}{\pgfqpoint{4.075430in}{3.278480in}}%
\pgfpathcurveto{\pgfqpoint{4.067617in}{3.286294in}}{\pgfqpoint{4.057018in}{3.290684in}}{\pgfqpoint{4.045968in}{3.290684in}}%
\pgfpathcurveto{\pgfqpoint{4.034917in}{3.290684in}}{\pgfqpoint{4.024318in}{3.286294in}}{\pgfqpoint{4.016505in}{3.278480in}}%
\pgfpathcurveto{\pgfqpoint{4.008691in}{3.270667in}}{\pgfqpoint{4.004301in}{3.260068in}}{\pgfqpoint{4.004301in}{3.249017in}}%
\pgfpathcurveto{\pgfqpoint{4.004301in}{3.237967in}}{\pgfqpoint{4.008691in}{3.227368in}}{\pgfqpoint{4.016505in}{3.219555in}}%
\pgfpathcurveto{\pgfqpoint{4.024318in}{3.211741in}}{\pgfqpoint{4.034917in}{3.207351in}}{\pgfqpoint{4.045968in}{3.207351in}}%
\pgfpathclose%
\pgfusepath{stroke,fill}%
\end{pgfscope}%
\begin{pgfscope}%
\pgfpathrectangle{\pgfqpoint{0.648703in}{0.548769in}}{\pgfqpoint{5.112893in}{3.102590in}}%
\pgfusepath{clip}%
\pgfsetbuttcap%
\pgfsetroundjoin%
\definecolor{currentfill}{rgb}{0.121569,0.466667,0.705882}%
\pgfsetfillcolor{currentfill}%
\pgfsetlinewidth{1.003750pt}%
\definecolor{currentstroke}{rgb}{0.121569,0.466667,0.705882}%
\pgfsetstrokecolor{currentstroke}%
\pgfsetdash{}{0pt}%
\pgfpathmoveto{\pgfqpoint{0.873111in}{0.648129in}}%
\pgfpathcurveto{\pgfqpoint{0.884161in}{0.648129in}}{\pgfqpoint{0.894760in}{0.652519in}}{\pgfqpoint{0.902573in}{0.660333in}}%
\pgfpathcurveto{\pgfqpoint{0.910387in}{0.668146in}}{\pgfqpoint{0.914777in}{0.678745in}}{\pgfqpoint{0.914777in}{0.689796in}}%
\pgfpathcurveto{\pgfqpoint{0.914777in}{0.700846in}}{\pgfqpoint{0.910387in}{0.711445in}}{\pgfqpoint{0.902573in}{0.719258in}}%
\pgfpathcurveto{\pgfqpoint{0.894760in}{0.727072in}}{\pgfqpoint{0.884161in}{0.731462in}}{\pgfqpoint{0.873111in}{0.731462in}}%
\pgfpathcurveto{\pgfqpoint{0.862061in}{0.731462in}}{\pgfqpoint{0.851461in}{0.727072in}}{\pgfqpoint{0.843648in}{0.719258in}}%
\pgfpathcurveto{\pgfqpoint{0.835834in}{0.711445in}}{\pgfqpoint{0.831444in}{0.700846in}}{\pgfqpoint{0.831444in}{0.689796in}}%
\pgfpathcurveto{\pgfqpoint{0.831444in}{0.678745in}}{\pgfqpoint{0.835834in}{0.668146in}}{\pgfqpoint{0.843648in}{0.660333in}}%
\pgfpathcurveto{\pgfqpoint{0.851461in}{0.652519in}}{\pgfqpoint{0.862061in}{0.648129in}}{\pgfqpoint{0.873111in}{0.648129in}}%
\pgfpathclose%
\pgfusepath{stroke,fill}%
\end{pgfscope}%
\begin{pgfscope}%
\pgfpathrectangle{\pgfqpoint{0.648703in}{0.548769in}}{\pgfqpoint{5.112893in}{3.102590in}}%
\pgfusepath{clip}%
\pgfsetbuttcap%
\pgfsetroundjoin%
\definecolor{currentfill}{rgb}{0.121569,0.466667,0.705882}%
\pgfsetfillcolor{currentfill}%
\pgfsetlinewidth{1.003750pt}%
\definecolor{currentstroke}{rgb}{0.121569,0.466667,0.705882}%
\pgfsetstrokecolor{currentstroke}%
\pgfsetdash{}{0pt}%
\pgfpathmoveto{\pgfqpoint{0.902102in}{0.656425in}}%
\pgfpathcurveto{\pgfqpoint{0.913153in}{0.656425in}}{\pgfqpoint{0.923752in}{0.660815in}}{\pgfqpoint{0.931565in}{0.668629in}}%
\pgfpathcurveto{\pgfqpoint{0.939379in}{0.676442in}}{\pgfqpoint{0.943769in}{0.687041in}}{\pgfqpoint{0.943769in}{0.698091in}}%
\pgfpathcurveto{\pgfqpoint{0.943769in}{0.709141in}}{\pgfqpoint{0.939379in}{0.719740in}}{\pgfqpoint{0.931565in}{0.727554in}}%
\pgfpathcurveto{\pgfqpoint{0.923752in}{0.735368in}}{\pgfqpoint{0.913153in}{0.739758in}}{\pgfqpoint{0.902102in}{0.739758in}}%
\pgfpathcurveto{\pgfqpoint{0.891052in}{0.739758in}}{\pgfqpoint{0.880453in}{0.735368in}}{\pgfqpoint{0.872640in}{0.727554in}}%
\pgfpathcurveto{\pgfqpoint{0.864826in}{0.719740in}}{\pgfqpoint{0.860436in}{0.709141in}}{\pgfqpoint{0.860436in}{0.698091in}}%
\pgfpathcurveto{\pgfqpoint{0.860436in}{0.687041in}}{\pgfqpoint{0.864826in}{0.676442in}}{\pgfqpoint{0.872640in}{0.668629in}}%
\pgfpathcurveto{\pgfqpoint{0.880453in}{0.660815in}}{\pgfqpoint{0.891052in}{0.656425in}}{\pgfqpoint{0.902102in}{0.656425in}}%
\pgfpathclose%
\pgfusepath{stroke,fill}%
\end{pgfscope}%
\begin{pgfscope}%
\pgfpathrectangle{\pgfqpoint{0.648703in}{0.548769in}}{\pgfqpoint{5.112893in}{3.102590in}}%
\pgfusepath{clip}%
\pgfsetbuttcap%
\pgfsetroundjoin%
\definecolor{currentfill}{rgb}{0.121569,0.466667,0.705882}%
\pgfsetfillcolor{currentfill}%
\pgfsetlinewidth{1.003750pt}%
\definecolor{currentstroke}{rgb}{0.121569,0.466667,0.705882}%
\pgfsetstrokecolor{currentstroke}%
\pgfsetdash{}{0pt}%
\pgfpathmoveto{\pgfqpoint{1.224633in}{3.099507in}}%
\pgfpathcurveto{\pgfqpoint{1.235684in}{3.099507in}}{\pgfqpoint{1.246283in}{3.103897in}}{\pgfqpoint{1.254096in}{3.111711in}}%
\pgfpathcurveto{\pgfqpoint{1.261910in}{3.119524in}}{\pgfqpoint{1.266300in}{3.130123in}}{\pgfqpoint{1.266300in}{3.141173in}}%
\pgfpathcurveto{\pgfqpoint{1.266300in}{3.152224in}}{\pgfqpoint{1.261910in}{3.162823in}}{\pgfqpoint{1.254096in}{3.170636in}}%
\pgfpathcurveto{\pgfqpoint{1.246283in}{3.178450in}}{\pgfqpoint{1.235684in}{3.182840in}}{\pgfqpoint{1.224633in}{3.182840in}}%
\pgfpathcurveto{\pgfqpoint{1.213583in}{3.182840in}}{\pgfqpoint{1.202984in}{3.178450in}}{\pgfqpoint{1.195171in}{3.170636in}}%
\pgfpathcurveto{\pgfqpoint{1.187357in}{3.162823in}}{\pgfqpoint{1.182967in}{3.152224in}}{\pgfqpoint{1.182967in}{3.141173in}}%
\pgfpathcurveto{\pgfqpoint{1.182967in}{3.130123in}}{\pgfqpoint{1.187357in}{3.119524in}}{\pgfqpoint{1.195171in}{3.111711in}}%
\pgfpathcurveto{\pgfqpoint{1.202984in}{3.103897in}}{\pgfqpoint{1.213583in}{3.099507in}}{\pgfqpoint{1.224633in}{3.099507in}}%
\pgfpathclose%
\pgfusepath{stroke,fill}%
\end{pgfscope}%
\begin{pgfscope}%
\pgfpathrectangle{\pgfqpoint{0.648703in}{0.548769in}}{\pgfqpoint{5.112893in}{3.102590in}}%
\pgfusepath{clip}%
\pgfsetbuttcap%
\pgfsetroundjoin%
\definecolor{currentfill}{rgb}{1.000000,0.498039,0.054902}%
\pgfsetfillcolor{currentfill}%
\pgfsetlinewidth{1.003750pt}%
\definecolor{currentstroke}{rgb}{1.000000,0.498039,0.054902}%
\pgfsetstrokecolor{currentstroke}%
\pgfsetdash{}{0pt}%
\pgfpathmoveto{\pgfqpoint{4.646763in}{3.136837in}}%
\pgfpathcurveto{\pgfqpoint{4.657813in}{3.136837in}}{\pgfqpoint{4.668412in}{3.141228in}}{\pgfqpoint{4.676225in}{3.149041in}}%
\pgfpathcurveto{\pgfqpoint{4.684039in}{3.156855in}}{\pgfqpoint{4.688429in}{3.167454in}}{\pgfqpoint{4.688429in}{3.178504in}}%
\pgfpathcurveto{\pgfqpoint{4.688429in}{3.189554in}}{\pgfqpoint{4.684039in}{3.200153in}}{\pgfqpoint{4.676225in}{3.207967in}}%
\pgfpathcurveto{\pgfqpoint{4.668412in}{3.215780in}}{\pgfqpoint{4.657813in}{3.220171in}}{\pgfqpoint{4.646763in}{3.220171in}}%
\pgfpathcurveto{\pgfqpoint{4.635713in}{3.220171in}}{\pgfqpoint{4.625114in}{3.215780in}}{\pgfqpoint{4.617300in}{3.207967in}}%
\pgfpathcurveto{\pgfqpoint{4.609486in}{3.200153in}}{\pgfqpoint{4.605096in}{3.189554in}}{\pgfqpoint{4.605096in}{3.178504in}}%
\pgfpathcurveto{\pgfqpoint{4.605096in}{3.167454in}}{\pgfqpoint{4.609486in}{3.156855in}}{\pgfqpoint{4.617300in}{3.149041in}}%
\pgfpathcurveto{\pgfqpoint{4.625114in}{3.141228in}}{\pgfqpoint{4.635713in}{3.136837in}}{\pgfqpoint{4.646763in}{3.136837in}}%
\pgfpathclose%
\pgfusepath{stroke,fill}%
\end{pgfscope}%
\begin{pgfscope}%
\pgfpathrectangle{\pgfqpoint{0.648703in}{0.548769in}}{\pgfqpoint{5.112893in}{3.102590in}}%
\pgfusepath{clip}%
\pgfsetbuttcap%
\pgfsetroundjoin%
\definecolor{currentfill}{rgb}{0.121569,0.466667,0.705882}%
\pgfsetfillcolor{currentfill}%
\pgfsetlinewidth{1.003750pt}%
\definecolor{currentstroke}{rgb}{0.121569,0.466667,0.705882}%
\pgfsetstrokecolor{currentstroke}%
\pgfsetdash{}{0pt}%
\pgfpathmoveto{\pgfqpoint{0.873114in}{0.648129in}}%
\pgfpathcurveto{\pgfqpoint{0.884164in}{0.648129in}}{\pgfqpoint{0.894763in}{0.652519in}}{\pgfqpoint{0.902577in}{0.660333in}}%
\pgfpathcurveto{\pgfqpoint{0.910390in}{0.668146in}}{\pgfqpoint{0.914781in}{0.678745in}}{\pgfqpoint{0.914781in}{0.689796in}}%
\pgfpathcurveto{\pgfqpoint{0.914781in}{0.700846in}}{\pgfqpoint{0.910390in}{0.711445in}}{\pgfqpoint{0.902577in}{0.719258in}}%
\pgfpathcurveto{\pgfqpoint{0.894763in}{0.727072in}}{\pgfqpoint{0.884164in}{0.731462in}}{\pgfqpoint{0.873114in}{0.731462in}}%
\pgfpathcurveto{\pgfqpoint{0.862064in}{0.731462in}}{\pgfqpoint{0.851465in}{0.727072in}}{\pgfqpoint{0.843651in}{0.719258in}}%
\pgfpathcurveto{\pgfqpoint{0.835838in}{0.711445in}}{\pgfqpoint{0.831447in}{0.700846in}}{\pgfqpoint{0.831447in}{0.689796in}}%
\pgfpathcurveto{\pgfqpoint{0.831447in}{0.678745in}}{\pgfqpoint{0.835838in}{0.668146in}}{\pgfqpoint{0.843651in}{0.660333in}}%
\pgfpathcurveto{\pgfqpoint{0.851465in}{0.652519in}}{\pgfqpoint{0.862064in}{0.648129in}}{\pgfqpoint{0.873114in}{0.648129in}}%
\pgfpathclose%
\pgfusepath{stroke,fill}%
\end{pgfscope}%
\begin{pgfscope}%
\pgfpathrectangle{\pgfqpoint{0.648703in}{0.548769in}}{\pgfqpoint{5.112893in}{3.102590in}}%
\pgfusepath{clip}%
\pgfsetbuttcap%
\pgfsetroundjoin%
\definecolor{currentfill}{rgb}{0.121569,0.466667,0.705882}%
\pgfsetfillcolor{currentfill}%
\pgfsetlinewidth{1.003750pt}%
\definecolor{currentstroke}{rgb}{0.121569,0.466667,0.705882}%
\pgfsetstrokecolor{currentstroke}%
\pgfsetdash{}{0pt}%
\pgfpathmoveto{\pgfqpoint{0.873125in}{0.648129in}}%
\pgfpathcurveto{\pgfqpoint{0.884175in}{0.648129in}}{\pgfqpoint{0.894774in}{0.652519in}}{\pgfqpoint{0.902588in}{0.660333in}}%
\pgfpathcurveto{\pgfqpoint{0.910402in}{0.668146in}}{\pgfqpoint{0.914792in}{0.678745in}}{\pgfqpoint{0.914792in}{0.689796in}}%
\pgfpathcurveto{\pgfqpoint{0.914792in}{0.700846in}}{\pgfqpoint{0.910402in}{0.711445in}}{\pgfqpoint{0.902588in}{0.719258in}}%
\pgfpathcurveto{\pgfqpoint{0.894774in}{0.727072in}}{\pgfqpoint{0.884175in}{0.731462in}}{\pgfqpoint{0.873125in}{0.731462in}}%
\pgfpathcurveto{\pgfqpoint{0.862075in}{0.731462in}}{\pgfqpoint{0.851476in}{0.727072in}}{\pgfqpoint{0.843662in}{0.719258in}}%
\pgfpathcurveto{\pgfqpoint{0.835849in}{0.711445in}}{\pgfqpoint{0.831458in}{0.700846in}}{\pgfqpoint{0.831458in}{0.689796in}}%
\pgfpathcurveto{\pgfqpoint{0.831458in}{0.678745in}}{\pgfqpoint{0.835849in}{0.668146in}}{\pgfqpoint{0.843662in}{0.660333in}}%
\pgfpathcurveto{\pgfqpoint{0.851476in}{0.652519in}}{\pgfqpoint{0.862075in}{0.648129in}}{\pgfqpoint{0.873125in}{0.648129in}}%
\pgfpathclose%
\pgfusepath{stroke,fill}%
\end{pgfscope}%
\begin{pgfscope}%
\pgfpathrectangle{\pgfqpoint{0.648703in}{0.548769in}}{\pgfqpoint{5.112893in}{3.102590in}}%
\pgfusepath{clip}%
\pgfsetbuttcap%
\pgfsetroundjoin%
\definecolor{currentfill}{rgb}{0.839216,0.152941,0.156863}%
\pgfsetfillcolor{currentfill}%
\pgfsetlinewidth{1.003750pt}%
\definecolor{currentstroke}{rgb}{0.839216,0.152941,0.156863}%
\pgfsetstrokecolor{currentstroke}%
\pgfsetdash{}{0pt}%
\pgfpathmoveto{\pgfqpoint{4.277626in}{3.410595in}}%
\pgfpathcurveto{\pgfqpoint{4.288676in}{3.410595in}}{\pgfqpoint{4.299275in}{3.414986in}}{\pgfqpoint{4.307089in}{3.422799in}}%
\pgfpathcurveto{\pgfqpoint{4.314902in}{3.430613in}}{\pgfqpoint{4.319293in}{3.441212in}}{\pgfqpoint{4.319293in}{3.452262in}}%
\pgfpathcurveto{\pgfqpoint{4.319293in}{3.463312in}}{\pgfqpoint{4.314902in}{3.473911in}}{\pgfqpoint{4.307089in}{3.481725in}}%
\pgfpathcurveto{\pgfqpoint{4.299275in}{3.489538in}}{\pgfqpoint{4.288676in}{3.493929in}}{\pgfqpoint{4.277626in}{3.493929in}}%
\pgfpathcurveto{\pgfqpoint{4.266576in}{3.493929in}}{\pgfqpoint{4.255977in}{3.489538in}}{\pgfqpoint{4.248163in}{3.481725in}}%
\pgfpathcurveto{\pgfqpoint{4.240350in}{3.473911in}}{\pgfqpoint{4.235959in}{3.463312in}}{\pgfqpoint{4.235959in}{3.452262in}}%
\pgfpathcurveto{\pgfqpoint{4.235959in}{3.441212in}}{\pgfqpoint{4.240350in}{3.430613in}}{\pgfqpoint{4.248163in}{3.422799in}}%
\pgfpathcurveto{\pgfqpoint{4.255977in}{3.414986in}}{\pgfqpoint{4.266576in}{3.410595in}}{\pgfqpoint{4.277626in}{3.410595in}}%
\pgfpathclose%
\pgfusepath{stroke,fill}%
\end{pgfscope}%
\begin{pgfscope}%
\pgfpathrectangle{\pgfqpoint{0.648703in}{0.548769in}}{\pgfqpoint{5.112893in}{3.102590in}}%
\pgfusepath{clip}%
\pgfsetbuttcap%
\pgfsetroundjoin%
\definecolor{currentfill}{rgb}{0.121569,0.466667,0.705882}%
\pgfsetfillcolor{currentfill}%
\pgfsetlinewidth{1.003750pt}%
\definecolor{currentstroke}{rgb}{0.121569,0.466667,0.705882}%
\pgfsetstrokecolor{currentstroke}%
\pgfsetdash{}{0pt}%
\pgfpathmoveto{\pgfqpoint{4.105742in}{3.128542in}}%
\pgfpathcurveto{\pgfqpoint{4.116793in}{3.128542in}}{\pgfqpoint{4.127392in}{3.132932in}}{\pgfqpoint{4.135205in}{3.140746in}}%
\pgfpathcurveto{\pgfqpoint{4.143019in}{3.148559in}}{\pgfqpoint{4.147409in}{3.159158in}}{\pgfqpoint{4.147409in}{3.170208in}}%
\pgfpathcurveto{\pgfqpoint{4.147409in}{3.181258in}}{\pgfqpoint{4.143019in}{3.191857in}}{\pgfqpoint{4.135205in}{3.199671in}}%
\pgfpathcurveto{\pgfqpoint{4.127392in}{3.207485in}}{\pgfqpoint{4.116793in}{3.211875in}}{\pgfqpoint{4.105742in}{3.211875in}}%
\pgfpathcurveto{\pgfqpoint{4.094692in}{3.211875in}}{\pgfqpoint{4.084093in}{3.207485in}}{\pgfqpoint{4.076280in}{3.199671in}}%
\pgfpathcurveto{\pgfqpoint{4.068466in}{3.191857in}}{\pgfqpoint{4.064076in}{3.181258in}}{\pgfqpoint{4.064076in}{3.170208in}}%
\pgfpathcurveto{\pgfqpoint{4.064076in}{3.159158in}}{\pgfqpoint{4.068466in}{3.148559in}}{\pgfqpoint{4.076280in}{3.140746in}}%
\pgfpathcurveto{\pgfqpoint{4.084093in}{3.132932in}}{\pgfqpoint{4.094692in}{3.128542in}}{\pgfqpoint{4.105742in}{3.128542in}}%
\pgfpathclose%
\pgfusepath{stroke,fill}%
\end{pgfscope}%
\begin{pgfscope}%
\pgfpathrectangle{\pgfqpoint{0.648703in}{0.548769in}}{\pgfqpoint{5.112893in}{3.102590in}}%
\pgfusepath{clip}%
\pgfsetbuttcap%
\pgfsetroundjoin%
\definecolor{currentfill}{rgb}{1.000000,0.498039,0.054902}%
\pgfsetfillcolor{currentfill}%
\pgfsetlinewidth{1.003750pt}%
\definecolor{currentstroke}{rgb}{1.000000,0.498039,0.054902}%
\pgfsetstrokecolor{currentstroke}%
\pgfsetdash{}{0pt}%
\pgfpathmoveto{\pgfqpoint{4.218601in}{3.315195in}}%
\pgfpathcurveto{\pgfqpoint{4.229651in}{3.315195in}}{\pgfqpoint{4.240250in}{3.319585in}}{\pgfqpoint{4.248064in}{3.327399in}}%
\pgfpathcurveto{\pgfqpoint{4.255878in}{3.335212in}}{\pgfqpoint{4.260268in}{3.345811in}}{\pgfqpoint{4.260268in}{3.356861in}}%
\pgfpathcurveto{\pgfqpoint{4.260268in}{3.367912in}}{\pgfqpoint{4.255878in}{3.378511in}}{\pgfqpoint{4.248064in}{3.386324in}}%
\pgfpathcurveto{\pgfqpoint{4.240250in}{3.394138in}}{\pgfqpoint{4.229651in}{3.398528in}}{\pgfqpoint{4.218601in}{3.398528in}}%
\pgfpathcurveto{\pgfqpoint{4.207551in}{3.398528in}}{\pgfqpoint{4.196952in}{3.394138in}}{\pgfqpoint{4.189138in}{3.386324in}}%
\pgfpathcurveto{\pgfqpoint{4.181325in}{3.378511in}}{\pgfqpoint{4.176935in}{3.367912in}}{\pgfqpoint{4.176935in}{3.356861in}}%
\pgfpathcurveto{\pgfqpoint{4.176935in}{3.345811in}}{\pgfqpoint{4.181325in}{3.335212in}}{\pgfqpoint{4.189138in}{3.327399in}}%
\pgfpathcurveto{\pgfqpoint{4.196952in}{3.319585in}}{\pgfqpoint{4.207551in}{3.315195in}}{\pgfqpoint{4.218601in}{3.315195in}}%
\pgfpathclose%
\pgfusepath{stroke,fill}%
\end{pgfscope}%
\begin{pgfscope}%
\pgfpathrectangle{\pgfqpoint{0.648703in}{0.548769in}}{\pgfqpoint{5.112893in}{3.102590in}}%
\pgfusepath{clip}%
\pgfsetbuttcap%
\pgfsetroundjoin%
\definecolor{currentfill}{rgb}{1.000000,0.498039,0.054902}%
\pgfsetfillcolor{currentfill}%
\pgfsetlinewidth{1.003750pt}%
\definecolor{currentstroke}{rgb}{1.000000,0.498039,0.054902}%
\pgfsetstrokecolor{currentstroke}%
\pgfsetdash{}{0pt}%
\pgfpathmoveto{\pgfqpoint{3.933325in}{3.136837in}}%
\pgfpathcurveto{\pgfqpoint{3.944375in}{3.136837in}}{\pgfqpoint{3.954975in}{3.141228in}}{\pgfqpoint{3.962788in}{3.149041in}}%
\pgfpathcurveto{\pgfqpoint{3.970602in}{3.156855in}}{\pgfqpoint{3.974992in}{3.167454in}}{\pgfqpoint{3.974992in}{3.178504in}}%
\pgfpathcurveto{\pgfqpoint{3.974992in}{3.189554in}}{\pgfqpoint{3.970602in}{3.200153in}}{\pgfqpoint{3.962788in}{3.207967in}}%
\pgfpathcurveto{\pgfqpoint{3.954975in}{3.215780in}}{\pgfqpoint{3.944375in}{3.220171in}}{\pgfqpoint{3.933325in}{3.220171in}}%
\pgfpathcurveto{\pgfqpoint{3.922275in}{3.220171in}}{\pgfqpoint{3.911676in}{3.215780in}}{\pgfqpoint{3.903863in}{3.207967in}}%
\pgfpathcurveto{\pgfqpoint{3.896049in}{3.200153in}}{\pgfqpoint{3.891659in}{3.189554in}}{\pgfqpoint{3.891659in}{3.178504in}}%
\pgfpathcurveto{\pgfqpoint{3.891659in}{3.167454in}}{\pgfqpoint{3.896049in}{3.156855in}}{\pgfqpoint{3.903863in}{3.149041in}}%
\pgfpathcurveto{\pgfqpoint{3.911676in}{3.141228in}}{\pgfqpoint{3.922275in}{3.136837in}}{\pgfqpoint{3.933325in}{3.136837in}}%
\pgfpathclose%
\pgfusepath{stroke,fill}%
\end{pgfscope}%
\begin{pgfscope}%
\pgfpathrectangle{\pgfqpoint{0.648703in}{0.548769in}}{\pgfqpoint{5.112893in}{3.102590in}}%
\pgfusepath{clip}%
\pgfsetbuttcap%
\pgfsetroundjoin%
\definecolor{currentfill}{rgb}{1.000000,0.498039,0.054902}%
\pgfsetfillcolor{currentfill}%
\pgfsetlinewidth{1.003750pt}%
\definecolor{currentstroke}{rgb}{1.000000,0.498039,0.054902}%
\pgfsetstrokecolor{currentstroke}%
\pgfsetdash{}{0pt}%
\pgfpathmoveto{\pgfqpoint{4.137020in}{3.136837in}}%
\pgfpathcurveto{\pgfqpoint{4.148070in}{3.136837in}}{\pgfqpoint{4.158669in}{3.141228in}}{\pgfqpoint{4.166482in}{3.149041in}}%
\pgfpathcurveto{\pgfqpoint{4.174296in}{3.156855in}}{\pgfqpoint{4.178686in}{3.167454in}}{\pgfqpoint{4.178686in}{3.178504in}}%
\pgfpathcurveto{\pgfqpoint{4.178686in}{3.189554in}}{\pgfqpoint{4.174296in}{3.200153in}}{\pgfqpoint{4.166482in}{3.207967in}}%
\pgfpathcurveto{\pgfqpoint{4.158669in}{3.215780in}}{\pgfqpoint{4.148070in}{3.220171in}}{\pgfqpoint{4.137020in}{3.220171in}}%
\pgfpathcurveto{\pgfqpoint{4.125970in}{3.220171in}}{\pgfqpoint{4.115371in}{3.215780in}}{\pgfqpoint{4.107557in}{3.207967in}}%
\pgfpathcurveto{\pgfqpoint{4.099743in}{3.200153in}}{\pgfqpoint{4.095353in}{3.189554in}}{\pgfqpoint{4.095353in}{3.178504in}}%
\pgfpathcurveto{\pgfqpoint{4.095353in}{3.167454in}}{\pgfqpoint{4.099743in}{3.156855in}}{\pgfqpoint{4.107557in}{3.149041in}}%
\pgfpathcurveto{\pgfqpoint{4.115371in}{3.141228in}}{\pgfqpoint{4.125970in}{3.136837in}}{\pgfqpoint{4.137020in}{3.136837in}}%
\pgfpathclose%
\pgfusepath{stroke,fill}%
\end{pgfscope}%
\begin{pgfscope}%
\pgfpathrectangle{\pgfqpoint{0.648703in}{0.548769in}}{\pgfqpoint{5.112893in}{3.102590in}}%
\pgfusepath{clip}%
\pgfsetbuttcap%
\pgfsetroundjoin%
\definecolor{currentfill}{rgb}{1.000000,0.498039,0.054902}%
\pgfsetfillcolor{currentfill}%
\pgfsetlinewidth{1.003750pt}%
\definecolor{currentstroke}{rgb}{1.000000,0.498039,0.054902}%
\pgfsetstrokecolor{currentstroke}%
\pgfsetdash{}{0pt}%
\pgfpathmoveto{\pgfqpoint{2.135871in}{3.140985in}}%
\pgfpathcurveto{\pgfqpoint{2.146921in}{3.140985in}}{\pgfqpoint{2.157520in}{3.145375in}}{\pgfqpoint{2.165334in}{3.153189in}}%
\pgfpathcurveto{\pgfqpoint{2.173148in}{3.161003in}}{\pgfqpoint{2.177538in}{3.171602in}}{\pgfqpoint{2.177538in}{3.182652in}}%
\pgfpathcurveto{\pgfqpoint{2.177538in}{3.193702in}}{\pgfqpoint{2.173148in}{3.204301in}}{\pgfqpoint{2.165334in}{3.212115in}}%
\pgfpathcurveto{\pgfqpoint{2.157520in}{3.219928in}}{\pgfqpoint{2.146921in}{3.224319in}}{\pgfqpoint{2.135871in}{3.224319in}}%
\pgfpathcurveto{\pgfqpoint{2.124821in}{3.224319in}}{\pgfqpoint{2.114222in}{3.219928in}}{\pgfqpoint{2.106408in}{3.212115in}}%
\pgfpathcurveto{\pgfqpoint{2.098595in}{3.204301in}}{\pgfqpoint{2.094205in}{3.193702in}}{\pgfqpoint{2.094205in}{3.182652in}}%
\pgfpathcurveto{\pgfqpoint{2.094205in}{3.171602in}}{\pgfqpoint{2.098595in}{3.161003in}}{\pgfqpoint{2.106408in}{3.153189in}}%
\pgfpathcurveto{\pgfqpoint{2.114222in}{3.145375in}}{\pgfqpoint{2.124821in}{3.140985in}}{\pgfqpoint{2.135871in}{3.140985in}}%
\pgfpathclose%
\pgfusepath{stroke,fill}%
\end{pgfscope}%
\begin{pgfscope}%
\pgfpathrectangle{\pgfqpoint{0.648703in}{0.548769in}}{\pgfqpoint{5.112893in}{3.102590in}}%
\pgfusepath{clip}%
\pgfsetbuttcap%
\pgfsetroundjoin%
\definecolor{currentfill}{rgb}{1.000000,0.498039,0.054902}%
\pgfsetfillcolor{currentfill}%
\pgfsetlinewidth{1.003750pt}%
\definecolor{currentstroke}{rgb}{1.000000,0.498039,0.054902}%
\pgfsetstrokecolor{currentstroke}%
\pgfsetdash{}{0pt}%
\pgfpathmoveto{\pgfqpoint{4.159057in}{3.136837in}}%
\pgfpathcurveto{\pgfqpoint{4.170107in}{3.136837in}}{\pgfqpoint{4.180706in}{3.141228in}}{\pgfqpoint{4.188519in}{3.149041in}}%
\pgfpathcurveto{\pgfqpoint{4.196333in}{3.156855in}}{\pgfqpoint{4.200723in}{3.167454in}}{\pgfqpoint{4.200723in}{3.178504in}}%
\pgfpathcurveto{\pgfqpoint{4.200723in}{3.189554in}}{\pgfqpoint{4.196333in}{3.200153in}}{\pgfqpoint{4.188519in}{3.207967in}}%
\pgfpathcurveto{\pgfqpoint{4.180706in}{3.215780in}}{\pgfqpoint{4.170107in}{3.220171in}}{\pgfqpoint{4.159057in}{3.220171in}}%
\pgfpathcurveto{\pgfqpoint{4.148006in}{3.220171in}}{\pgfqpoint{4.137407in}{3.215780in}}{\pgfqpoint{4.129594in}{3.207967in}}%
\pgfpathcurveto{\pgfqpoint{4.121780in}{3.200153in}}{\pgfqpoint{4.117390in}{3.189554in}}{\pgfqpoint{4.117390in}{3.178504in}}%
\pgfpathcurveto{\pgfqpoint{4.117390in}{3.167454in}}{\pgfqpoint{4.121780in}{3.156855in}}{\pgfqpoint{4.129594in}{3.149041in}}%
\pgfpathcurveto{\pgfqpoint{4.137407in}{3.141228in}}{\pgfqpoint{4.148006in}{3.136837in}}{\pgfqpoint{4.159057in}{3.136837in}}%
\pgfpathclose%
\pgfusepath{stroke,fill}%
\end{pgfscope}%
\begin{pgfscope}%
\pgfpathrectangle{\pgfqpoint{0.648703in}{0.548769in}}{\pgfqpoint{5.112893in}{3.102590in}}%
\pgfusepath{clip}%
\pgfsetbuttcap%
\pgfsetroundjoin%
\definecolor{currentfill}{rgb}{1.000000,0.498039,0.054902}%
\pgfsetfillcolor{currentfill}%
\pgfsetlinewidth{1.003750pt}%
\definecolor{currentstroke}{rgb}{1.000000,0.498039,0.054902}%
\pgfsetstrokecolor{currentstroke}%
\pgfsetdash{}{0pt}%
\pgfpathmoveto{\pgfqpoint{4.346462in}{3.356673in}}%
\pgfpathcurveto{\pgfqpoint{4.357512in}{3.356673in}}{\pgfqpoint{4.368111in}{3.361064in}}{\pgfqpoint{4.375925in}{3.368877in}}%
\pgfpathcurveto{\pgfqpoint{4.383739in}{3.376691in}}{\pgfqpoint{4.388129in}{3.387290in}}{\pgfqpoint{4.388129in}{3.398340in}}%
\pgfpathcurveto{\pgfqpoint{4.388129in}{3.409390in}}{\pgfqpoint{4.383739in}{3.419989in}}{\pgfqpoint{4.375925in}{3.427803in}}%
\pgfpathcurveto{\pgfqpoint{4.368111in}{3.435616in}}{\pgfqpoint{4.357512in}{3.440007in}}{\pgfqpoint{4.346462in}{3.440007in}}%
\pgfpathcurveto{\pgfqpoint{4.335412in}{3.440007in}}{\pgfqpoint{4.324813in}{3.435616in}}{\pgfqpoint{4.316999in}{3.427803in}}%
\pgfpathcurveto{\pgfqpoint{4.309186in}{3.419989in}}{\pgfqpoint{4.304796in}{3.409390in}}{\pgfqpoint{4.304796in}{3.398340in}}%
\pgfpathcurveto{\pgfqpoint{4.304796in}{3.387290in}}{\pgfqpoint{4.309186in}{3.376691in}}{\pgfqpoint{4.316999in}{3.368877in}}%
\pgfpathcurveto{\pgfqpoint{4.324813in}{3.361064in}}{\pgfqpoint{4.335412in}{3.356673in}}{\pgfqpoint{4.346462in}{3.356673in}}%
\pgfpathclose%
\pgfusepath{stroke,fill}%
\end{pgfscope}%
\begin{pgfscope}%
\pgfpathrectangle{\pgfqpoint{0.648703in}{0.548769in}}{\pgfqpoint{5.112893in}{3.102590in}}%
\pgfusepath{clip}%
\pgfsetbuttcap%
\pgfsetroundjoin%
\definecolor{currentfill}{rgb}{1.000000,0.498039,0.054902}%
\pgfsetfillcolor{currentfill}%
\pgfsetlinewidth{1.003750pt}%
\definecolor{currentstroke}{rgb}{1.000000,0.498039,0.054902}%
\pgfsetstrokecolor{currentstroke}%
\pgfsetdash{}{0pt}%
\pgfpathmoveto{\pgfqpoint{2.525088in}{3.219794in}}%
\pgfpathcurveto{\pgfqpoint{2.536138in}{3.219794in}}{\pgfqpoint{2.546738in}{3.224185in}}{\pgfqpoint{2.554551in}{3.231998in}}%
\pgfpathcurveto{\pgfqpoint{2.562365in}{3.239812in}}{\pgfqpoint{2.566755in}{3.250411in}}{\pgfqpoint{2.566755in}{3.261461in}}%
\pgfpathcurveto{\pgfqpoint{2.566755in}{3.272511in}}{\pgfqpoint{2.562365in}{3.283110in}}{\pgfqpoint{2.554551in}{3.290924in}}%
\pgfpathcurveto{\pgfqpoint{2.546738in}{3.298737in}}{\pgfqpoint{2.536138in}{3.303128in}}{\pgfqpoint{2.525088in}{3.303128in}}%
\pgfpathcurveto{\pgfqpoint{2.514038in}{3.303128in}}{\pgfqpoint{2.503439in}{3.298737in}}{\pgfqpoint{2.495626in}{3.290924in}}%
\pgfpathcurveto{\pgfqpoint{2.487812in}{3.283110in}}{\pgfqpoint{2.483422in}{3.272511in}}{\pgfqpoint{2.483422in}{3.261461in}}%
\pgfpathcurveto{\pgfqpoint{2.483422in}{3.250411in}}{\pgfqpoint{2.487812in}{3.239812in}}{\pgfqpoint{2.495626in}{3.231998in}}%
\pgfpathcurveto{\pgfqpoint{2.503439in}{3.224185in}}{\pgfqpoint{2.514038in}{3.219794in}}{\pgfqpoint{2.525088in}{3.219794in}}%
\pgfpathclose%
\pgfusepath{stroke,fill}%
\end{pgfscope}%
\begin{pgfscope}%
\pgfpathrectangle{\pgfqpoint{0.648703in}{0.548769in}}{\pgfqpoint{5.112893in}{3.102590in}}%
\pgfusepath{clip}%
\pgfsetbuttcap%
\pgfsetroundjoin%
\definecolor{currentfill}{rgb}{0.121569,0.466667,0.705882}%
\pgfsetfillcolor{currentfill}%
\pgfsetlinewidth{1.003750pt}%
\definecolor{currentstroke}{rgb}{0.121569,0.466667,0.705882}%
\pgfsetstrokecolor{currentstroke}%
\pgfsetdash{}{0pt}%
\pgfpathmoveto{\pgfqpoint{3.483006in}{3.120246in}}%
\pgfpathcurveto{\pgfqpoint{3.494056in}{3.120246in}}{\pgfqpoint{3.504655in}{3.124636in}}{\pgfqpoint{3.512469in}{3.132450in}}%
\pgfpathcurveto{\pgfqpoint{3.520282in}{3.140263in}}{\pgfqpoint{3.524673in}{3.150862in}}{\pgfqpoint{3.524673in}{3.161913in}}%
\pgfpathcurveto{\pgfqpoint{3.524673in}{3.172963in}}{\pgfqpoint{3.520282in}{3.183562in}}{\pgfqpoint{3.512469in}{3.191375in}}%
\pgfpathcurveto{\pgfqpoint{3.504655in}{3.199189in}}{\pgfqpoint{3.494056in}{3.203579in}}{\pgfqpoint{3.483006in}{3.203579in}}%
\pgfpathcurveto{\pgfqpoint{3.471956in}{3.203579in}}{\pgfqpoint{3.461357in}{3.199189in}}{\pgfqpoint{3.453543in}{3.191375in}}%
\pgfpathcurveto{\pgfqpoint{3.445730in}{3.183562in}}{\pgfqpoint{3.441339in}{3.172963in}}{\pgfqpoint{3.441339in}{3.161913in}}%
\pgfpathcurveto{\pgfqpoint{3.441339in}{3.150862in}}{\pgfqpoint{3.445730in}{3.140263in}}{\pgfqpoint{3.453543in}{3.132450in}}%
\pgfpathcurveto{\pgfqpoint{3.461357in}{3.124636in}}{\pgfqpoint{3.471956in}{3.120246in}}{\pgfqpoint{3.483006in}{3.120246in}}%
\pgfpathclose%
\pgfusepath{stroke,fill}%
\end{pgfscope}%
\begin{pgfscope}%
\pgfpathrectangle{\pgfqpoint{0.648703in}{0.548769in}}{\pgfqpoint{5.112893in}{3.102590in}}%
\pgfusepath{clip}%
\pgfsetbuttcap%
\pgfsetroundjoin%
\definecolor{currentfill}{rgb}{1.000000,0.498039,0.054902}%
\pgfsetfillcolor{currentfill}%
\pgfsetlinewidth{1.003750pt}%
\definecolor{currentstroke}{rgb}{1.000000,0.498039,0.054902}%
\pgfsetstrokecolor{currentstroke}%
\pgfsetdash{}{0pt}%
\pgfpathmoveto{\pgfqpoint{4.237059in}{3.140985in}}%
\pgfpathcurveto{\pgfqpoint{4.248109in}{3.140985in}}{\pgfqpoint{4.258708in}{3.145375in}}{\pgfqpoint{4.266521in}{3.153189in}}%
\pgfpathcurveto{\pgfqpoint{4.274335in}{3.161003in}}{\pgfqpoint{4.278725in}{3.171602in}}{\pgfqpoint{4.278725in}{3.182652in}}%
\pgfpathcurveto{\pgfqpoint{4.278725in}{3.193702in}}{\pgfqpoint{4.274335in}{3.204301in}}{\pgfqpoint{4.266521in}{3.212115in}}%
\pgfpathcurveto{\pgfqpoint{4.258708in}{3.219928in}}{\pgfqpoint{4.248109in}{3.224319in}}{\pgfqpoint{4.237059in}{3.224319in}}%
\pgfpathcurveto{\pgfqpoint{4.226009in}{3.224319in}}{\pgfqpoint{4.215409in}{3.219928in}}{\pgfqpoint{4.207596in}{3.212115in}}%
\pgfpathcurveto{\pgfqpoint{4.199782in}{3.204301in}}{\pgfqpoint{4.195392in}{3.193702in}}{\pgfqpoint{4.195392in}{3.182652in}}%
\pgfpathcurveto{\pgfqpoint{4.195392in}{3.171602in}}{\pgfqpoint{4.199782in}{3.161003in}}{\pgfqpoint{4.207596in}{3.153189in}}%
\pgfpathcurveto{\pgfqpoint{4.215409in}{3.145375in}}{\pgfqpoint{4.226009in}{3.140985in}}{\pgfqpoint{4.237059in}{3.140985in}}%
\pgfpathclose%
\pgfusepath{stroke,fill}%
\end{pgfscope}%
\begin{pgfscope}%
\pgfpathrectangle{\pgfqpoint{0.648703in}{0.548769in}}{\pgfqpoint{5.112893in}{3.102590in}}%
\pgfusepath{clip}%
\pgfsetbuttcap%
\pgfsetroundjoin%
\definecolor{currentfill}{rgb}{1.000000,0.498039,0.054902}%
\pgfsetfillcolor{currentfill}%
\pgfsetlinewidth{1.003750pt}%
\definecolor{currentstroke}{rgb}{1.000000,0.498039,0.054902}%
\pgfsetstrokecolor{currentstroke}%
\pgfsetdash{}{0pt}%
\pgfpathmoveto{\pgfqpoint{4.417793in}{3.145133in}}%
\pgfpathcurveto{\pgfqpoint{4.428843in}{3.145133in}}{\pgfqpoint{4.439442in}{3.149523in}}{\pgfqpoint{4.447255in}{3.157337in}}%
\pgfpathcurveto{\pgfqpoint{4.455069in}{3.165151in}}{\pgfqpoint{4.459459in}{3.175750in}}{\pgfqpoint{4.459459in}{3.186800in}}%
\pgfpathcurveto{\pgfqpoint{4.459459in}{3.197850in}}{\pgfqpoint{4.455069in}{3.208449in}}{\pgfqpoint{4.447255in}{3.216262in}}%
\pgfpathcurveto{\pgfqpoint{4.439442in}{3.224076in}}{\pgfqpoint{4.428843in}{3.228466in}}{\pgfqpoint{4.417793in}{3.228466in}}%
\pgfpathcurveto{\pgfqpoint{4.406743in}{3.228466in}}{\pgfqpoint{4.396143in}{3.224076in}}{\pgfqpoint{4.388330in}{3.216262in}}%
\pgfpathcurveto{\pgfqpoint{4.380516in}{3.208449in}}{\pgfqpoint{4.376126in}{3.197850in}}{\pgfqpoint{4.376126in}{3.186800in}}%
\pgfpathcurveto{\pgfqpoint{4.376126in}{3.175750in}}{\pgfqpoint{4.380516in}{3.165151in}}{\pgfqpoint{4.388330in}{3.157337in}}%
\pgfpathcurveto{\pgfqpoint{4.396143in}{3.149523in}}{\pgfqpoint{4.406743in}{3.145133in}}{\pgfqpoint{4.417793in}{3.145133in}}%
\pgfpathclose%
\pgfusepath{stroke,fill}%
\end{pgfscope}%
\begin{pgfscope}%
\pgfpathrectangle{\pgfqpoint{0.648703in}{0.548769in}}{\pgfqpoint{5.112893in}{3.102590in}}%
\pgfusepath{clip}%
\pgfsetbuttcap%
\pgfsetroundjoin%
\definecolor{currentfill}{rgb}{1.000000,0.498039,0.054902}%
\pgfsetfillcolor{currentfill}%
\pgfsetlinewidth{1.003750pt}%
\definecolor{currentstroke}{rgb}{1.000000,0.498039,0.054902}%
\pgfsetstrokecolor{currentstroke}%
\pgfsetdash{}{0pt}%
\pgfpathmoveto{\pgfqpoint{1.609598in}{3.136837in}}%
\pgfpathcurveto{\pgfqpoint{1.620648in}{3.136837in}}{\pgfqpoint{1.631247in}{3.141228in}}{\pgfqpoint{1.639061in}{3.149041in}}%
\pgfpathcurveto{\pgfqpoint{1.646874in}{3.156855in}}{\pgfqpoint{1.651265in}{3.167454in}}{\pgfqpoint{1.651265in}{3.178504in}}%
\pgfpathcurveto{\pgfqpoint{1.651265in}{3.189554in}}{\pgfqpoint{1.646874in}{3.200153in}}{\pgfqpoint{1.639061in}{3.207967in}}%
\pgfpathcurveto{\pgfqpoint{1.631247in}{3.215780in}}{\pgfqpoint{1.620648in}{3.220171in}}{\pgfqpoint{1.609598in}{3.220171in}}%
\pgfpathcurveto{\pgfqpoint{1.598548in}{3.220171in}}{\pgfqpoint{1.587949in}{3.215780in}}{\pgfqpoint{1.580135in}{3.207967in}}%
\pgfpathcurveto{\pgfqpoint{1.572322in}{3.200153in}}{\pgfqpoint{1.567931in}{3.189554in}}{\pgfqpoint{1.567931in}{3.178504in}}%
\pgfpathcurveto{\pgfqpoint{1.567931in}{3.167454in}}{\pgfqpoint{1.572322in}{3.156855in}}{\pgfqpoint{1.580135in}{3.149041in}}%
\pgfpathcurveto{\pgfqpoint{1.587949in}{3.141228in}}{\pgfqpoint{1.598548in}{3.136837in}}{\pgfqpoint{1.609598in}{3.136837in}}%
\pgfpathclose%
\pgfusepath{stroke,fill}%
\end{pgfscope}%
\begin{pgfscope}%
\pgfpathrectangle{\pgfqpoint{0.648703in}{0.548769in}}{\pgfqpoint{5.112893in}{3.102590in}}%
\pgfusepath{clip}%
\pgfsetbuttcap%
\pgfsetroundjoin%
\definecolor{currentfill}{rgb}{0.121569,0.466667,0.705882}%
\pgfsetfillcolor{currentfill}%
\pgfsetlinewidth{1.003750pt}%
\definecolor{currentstroke}{rgb}{0.121569,0.466667,0.705882}%
\pgfsetstrokecolor{currentstroke}%
\pgfsetdash{}{0pt}%
\pgfpathmoveto{\pgfqpoint{2.029496in}{3.132690in}}%
\pgfpathcurveto{\pgfqpoint{2.040546in}{3.132690in}}{\pgfqpoint{2.051145in}{3.137080in}}{\pgfqpoint{2.058959in}{3.144893in}}%
\pgfpathcurveto{\pgfqpoint{2.066773in}{3.152707in}}{\pgfqpoint{2.071163in}{3.163306in}}{\pgfqpoint{2.071163in}{3.174356in}}%
\pgfpathcurveto{\pgfqpoint{2.071163in}{3.185406in}}{\pgfqpoint{2.066773in}{3.196005in}}{\pgfqpoint{2.058959in}{3.203819in}}%
\pgfpathcurveto{\pgfqpoint{2.051145in}{3.211633in}}{\pgfqpoint{2.040546in}{3.216023in}}{\pgfqpoint{2.029496in}{3.216023in}}%
\pgfpathcurveto{\pgfqpoint{2.018446in}{3.216023in}}{\pgfqpoint{2.007847in}{3.211633in}}{\pgfqpoint{2.000033in}{3.203819in}}%
\pgfpathcurveto{\pgfqpoint{1.992220in}{3.196005in}}{\pgfqpoint{1.987830in}{3.185406in}}{\pgfqpoint{1.987830in}{3.174356in}}%
\pgfpathcurveto{\pgfqpoint{1.987830in}{3.163306in}}{\pgfqpoint{1.992220in}{3.152707in}}{\pgfqpoint{2.000033in}{3.144893in}}%
\pgfpathcurveto{\pgfqpoint{2.007847in}{3.137080in}}{\pgfqpoint{2.018446in}{3.132690in}}{\pgfqpoint{2.029496in}{3.132690in}}%
\pgfpathclose%
\pgfusepath{stroke,fill}%
\end{pgfscope}%
\begin{pgfscope}%
\pgfpathrectangle{\pgfqpoint{0.648703in}{0.548769in}}{\pgfqpoint{5.112893in}{3.102590in}}%
\pgfusepath{clip}%
\pgfsetbuttcap%
\pgfsetroundjoin%
\definecolor{currentfill}{rgb}{0.121569,0.466667,0.705882}%
\pgfsetfillcolor{currentfill}%
\pgfsetlinewidth{1.003750pt}%
\definecolor{currentstroke}{rgb}{0.121569,0.466667,0.705882}%
\pgfsetstrokecolor{currentstroke}%
\pgfsetdash{}{0pt}%
\pgfpathmoveto{\pgfqpoint{3.934257in}{3.132690in}}%
\pgfpathcurveto{\pgfqpoint{3.945307in}{3.132690in}}{\pgfqpoint{3.955906in}{3.137080in}}{\pgfqpoint{3.963720in}{3.144893in}}%
\pgfpathcurveto{\pgfqpoint{3.971533in}{3.152707in}}{\pgfqpoint{3.975924in}{3.163306in}}{\pgfqpoint{3.975924in}{3.174356in}}%
\pgfpathcurveto{\pgfqpoint{3.975924in}{3.185406in}}{\pgfqpoint{3.971533in}{3.196005in}}{\pgfqpoint{3.963720in}{3.203819in}}%
\pgfpathcurveto{\pgfqpoint{3.955906in}{3.211633in}}{\pgfqpoint{3.945307in}{3.216023in}}{\pgfqpoint{3.934257in}{3.216023in}}%
\pgfpathcurveto{\pgfqpoint{3.923207in}{3.216023in}}{\pgfqpoint{3.912608in}{3.211633in}}{\pgfqpoint{3.904794in}{3.203819in}}%
\pgfpathcurveto{\pgfqpoint{3.896981in}{3.196005in}}{\pgfqpoint{3.892590in}{3.185406in}}{\pgfqpoint{3.892590in}{3.174356in}}%
\pgfpathcurveto{\pgfqpoint{3.892590in}{3.163306in}}{\pgfqpoint{3.896981in}{3.152707in}}{\pgfqpoint{3.904794in}{3.144893in}}%
\pgfpathcurveto{\pgfqpoint{3.912608in}{3.137080in}}{\pgfqpoint{3.923207in}{3.132690in}}{\pgfqpoint{3.934257in}{3.132690in}}%
\pgfpathclose%
\pgfusepath{stroke,fill}%
\end{pgfscope}%
\begin{pgfscope}%
\pgfpathrectangle{\pgfqpoint{0.648703in}{0.548769in}}{\pgfqpoint{5.112893in}{3.102590in}}%
\pgfusepath{clip}%
\pgfsetbuttcap%
\pgfsetroundjoin%
\definecolor{currentfill}{rgb}{1.000000,0.498039,0.054902}%
\pgfsetfillcolor{currentfill}%
\pgfsetlinewidth{1.003750pt}%
\definecolor{currentstroke}{rgb}{1.000000,0.498039,0.054902}%
\pgfsetstrokecolor{currentstroke}%
\pgfsetdash{}{0pt}%
\pgfpathmoveto{\pgfqpoint{3.507277in}{3.136837in}}%
\pgfpathcurveto{\pgfqpoint{3.518327in}{3.136837in}}{\pgfqpoint{3.528926in}{3.141228in}}{\pgfqpoint{3.536739in}{3.149041in}}%
\pgfpathcurveto{\pgfqpoint{3.544553in}{3.156855in}}{\pgfqpoint{3.548943in}{3.167454in}}{\pgfqpoint{3.548943in}{3.178504in}}%
\pgfpathcurveto{\pgfqpoint{3.548943in}{3.189554in}}{\pgfqpoint{3.544553in}{3.200153in}}{\pgfqpoint{3.536739in}{3.207967in}}%
\pgfpathcurveto{\pgfqpoint{3.528926in}{3.215780in}}{\pgfqpoint{3.518327in}{3.220171in}}{\pgfqpoint{3.507277in}{3.220171in}}%
\pgfpathcurveto{\pgfqpoint{3.496226in}{3.220171in}}{\pgfqpoint{3.485627in}{3.215780in}}{\pgfqpoint{3.477814in}{3.207967in}}%
\pgfpathcurveto{\pgfqpoint{3.470000in}{3.200153in}}{\pgfqpoint{3.465610in}{3.189554in}}{\pgfqpoint{3.465610in}{3.178504in}}%
\pgfpathcurveto{\pgfqpoint{3.465610in}{3.167454in}}{\pgfqpoint{3.470000in}{3.156855in}}{\pgfqpoint{3.477814in}{3.149041in}}%
\pgfpathcurveto{\pgfqpoint{3.485627in}{3.141228in}}{\pgfqpoint{3.496226in}{3.136837in}}{\pgfqpoint{3.507277in}{3.136837in}}%
\pgfpathclose%
\pgfusepath{stroke,fill}%
\end{pgfscope}%
\begin{pgfscope}%
\pgfpathrectangle{\pgfqpoint{0.648703in}{0.548769in}}{\pgfqpoint{5.112893in}{3.102590in}}%
\pgfusepath{clip}%
\pgfsetbuttcap%
\pgfsetroundjoin%
\definecolor{currentfill}{rgb}{0.839216,0.152941,0.156863}%
\pgfsetfillcolor{currentfill}%
\pgfsetlinewidth{1.003750pt}%
\definecolor{currentstroke}{rgb}{0.839216,0.152941,0.156863}%
\pgfsetstrokecolor{currentstroke}%
\pgfsetdash{}{0pt}%
\pgfpathmoveto{\pgfqpoint{3.261580in}{3.120246in}}%
\pgfpathcurveto{\pgfqpoint{3.272630in}{3.120246in}}{\pgfqpoint{3.283229in}{3.124636in}}{\pgfqpoint{3.291043in}{3.132450in}}%
\pgfpathcurveto{\pgfqpoint{3.298856in}{3.140263in}}{\pgfqpoint{3.303247in}{3.150862in}}{\pgfqpoint{3.303247in}{3.161913in}}%
\pgfpathcurveto{\pgfqpoint{3.303247in}{3.172963in}}{\pgfqpoint{3.298856in}{3.183562in}}{\pgfqpoint{3.291043in}{3.191375in}}%
\pgfpathcurveto{\pgfqpoint{3.283229in}{3.199189in}}{\pgfqpoint{3.272630in}{3.203579in}}{\pgfqpoint{3.261580in}{3.203579in}}%
\pgfpathcurveto{\pgfqpoint{3.250530in}{3.203579in}}{\pgfqpoint{3.239931in}{3.199189in}}{\pgfqpoint{3.232117in}{3.191375in}}%
\pgfpathcurveto{\pgfqpoint{3.224304in}{3.183562in}}{\pgfqpoint{3.219913in}{3.172963in}}{\pgfqpoint{3.219913in}{3.161913in}}%
\pgfpathcurveto{\pgfqpoint{3.219913in}{3.150862in}}{\pgfqpoint{3.224304in}{3.140263in}}{\pgfqpoint{3.232117in}{3.132450in}}%
\pgfpathcurveto{\pgfqpoint{3.239931in}{3.124636in}}{\pgfqpoint{3.250530in}{3.120246in}}{\pgfqpoint{3.261580in}{3.120246in}}%
\pgfpathclose%
\pgfusepath{stroke,fill}%
\end{pgfscope}%
\begin{pgfscope}%
\pgfpathrectangle{\pgfqpoint{0.648703in}{0.548769in}}{\pgfqpoint{5.112893in}{3.102590in}}%
\pgfusepath{clip}%
\pgfsetbuttcap%
\pgfsetroundjoin%
\definecolor{currentfill}{rgb}{1.000000,0.498039,0.054902}%
\pgfsetfillcolor{currentfill}%
\pgfsetlinewidth{1.003750pt}%
\definecolor{currentstroke}{rgb}{1.000000,0.498039,0.054902}%
\pgfsetstrokecolor{currentstroke}%
\pgfsetdash{}{0pt}%
\pgfpathmoveto{\pgfqpoint{4.602865in}{3.244681in}}%
\pgfpathcurveto{\pgfqpoint{4.613915in}{3.244681in}}{\pgfqpoint{4.624514in}{3.249072in}}{\pgfqpoint{4.632327in}{3.256885in}}%
\pgfpathcurveto{\pgfqpoint{4.640141in}{3.264699in}}{\pgfqpoint{4.644531in}{3.275298in}}{\pgfqpoint{4.644531in}{3.286348in}}%
\pgfpathcurveto{\pgfqpoint{4.644531in}{3.297398in}}{\pgfqpoint{4.640141in}{3.307997in}}{\pgfqpoint{4.632327in}{3.315811in}}%
\pgfpathcurveto{\pgfqpoint{4.624514in}{3.323624in}}{\pgfqpoint{4.613915in}{3.328015in}}{\pgfqpoint{4.602865in}{3.328015in}}%
\pgfpathcurveto{\pgfqpoint{4.591815in}{3.328015in}}{\pgfqpoint{4.581216in}{3.323624in}}{\pgfqpoint{4.573402in}{3.315811in}}%
\pgfpathcurveto{\pgfqpoint{4.565588in}{3.307997in}}{\pgfqpoint{4.561198in}{3.297398in}}{\pgfqpoint{4.561198in}{3.286348in}}%
\pgfpathcurveto{\pgfqpoint{4.561198in}{3.275298in}}{\pgfqpoint{4.565588in}{3.264699in}}{\pgfqpoint{4.573402in}{3.256885in}}%
\pgfpathcurveto{\pgfqpoint{4.581216in}{3.249072in}}{\pgfqpoint{4.591815in}{3.244681in}}{\pgfqpoint{4.602865in}{3.244681in}}%
\pgfpathclose%
\pgfusepath{stroke,fill}%
\end{pgfscope}%
\begin{pgfscope}%
\pgfpathrectangle{\pgfqpoint{0.648703in}{0.548769in}}{\pgfqpoint{5.112893in}{3.102590in}}%
\pgfusepath{clip}%
\pgfsetbuttcap%
\pgfsetroundjoin%
\definecolor{currentfill}{rgb}{0.121569,0.466667,0.705882}%
\pgfsetfillcolor{currentfill}%
\pgfsetlinewidth{1.003750pt}%
\definecolor{currentstroke}{rgb}{0.121569,0.466667,0.705882}%
\pgfsetstrokecolor{currentstroke}%
\pgfsetdash{}{0pt}%
\pgfpathmoveto{\pgfqpoint{3.831154in}{3.128542in}}%
\pgfpathcurveto{\pgfqpoint{3.842204in}{3.128542in}}{\pgfqpoint{3.852804in}{3.132932in}}{\pgfqpoint{3.860617in}{3.140746in}}%
\pgfpathcurveto{\pgfqpoint{3.868431in}{3.148559in}}{\pgfqpoint{3.872821in}{3.159158in}}{\pgfqpoint{3.872821in}{3.170208in}}%
\pgfpathcurveto{\pgfqpoint{3.872821in}{3.181258in}}{\pgfqpoint{3.868431in}{3.191857in}}{\pgfqpoint{3.860617in}{3.199671in}}%
\pgfpathcurveto{\pgfqpoint{3.852804in}{3.207485in}}{\pgfqpoint{3.842204in}{3.211875in}}{\pgfqpoint{3.831154in}{3.211875in}}%
\pgfpathcurveto{\pgfqpoint{3.820104in}{3.211875in}}{\pgfqpoint{3.809505in}{3.207485in}}{\pgfqpoint{3.801692in}{3.199671in}}%
\pgfpathcurveto{\pgfqpoint{3.793878in}{3.191857in}}{\pgfqpoint{3.789488in}{3.181258in}}{\pgfqpoint{3.789488in}{3.170208in}}%
\pgfpathcurveto{\pgfqpoint{3.789488in}{3.159158in}}{\pgfqpoint{3.793878in}{3.148559in}}{\pgfqpoint{3.801692in}{3.140746in}}%
\pgfpathcurveto{\pgfqpoint{3.809505in}{3.132932in}}{\pgfqpoint{3.820104in}{3.128542in}}{\pgfqpoint{3.831154in}{3.128542in}}%
\pgfpathclose%
\pgfusepath{stroke,fill}%
\end{pgfscope}%
\begin{pgfscope}%
\pgfpathrectangle{\pgfqpoint{0.648703in}{0.548769in}}{\pgfqpoint{5.112893in}{3.102590in}}%
\pgfusepath{clip}%
\pgfsetbuttcap%
\pgfsetroundjoin%
\definecolor{currentfill}{rgb}{1.000000,0.498039,0.054902}%
\pgfsetfillcolor{currentfill}%
\pgfsetlinewidth{1.003750pt}%
\definecolor{currentstroke}{rgb}{1.000000,0.498039,0.054902}%
\pgfsetstrokecolor{currentstroke}%
\pgfsetdash{}{0pt}%
\pgfpathmoveto{\pgfqpoint{2.759793in}{3.136837in}}%
\pgfpathcurveto{\pgfqpoint{2.770844in}{3.136837in}}{\pgfqpoint{2.781443in}{3.141228in}}{\pgfqpoint{2.789256in}{3.149041in}}%
\pgfpathcurveto{\pgfqpoint{2.797070in}{3.156855in}}{\pgfqpoint{2.801460in}{3.167454in}}{\pgfqpoint{2.801460in}{3.178504in}}%
\pgfpathcurveto{\pgfqpoint{2.801460in}{3.189554in}}{\pgfqpoint{2.797070in}{3.200153in}}{\pgfqpoint{2.789256in}{3.207967in}}%
\pgfpathcurveto{\pgfqpoint{2.781443in}{3.215780in}}{\pgfqpoint{2.770844in}{3.220171in}}{\pgfqpoint{2.759793in}{3.220171in}}%
\pgfpathcurveto{\pgfqpoint{2.748743in}{3.220171in}}{\pgfqpoint{2.738144in}{3.215780in}}{\pgfqpoint{2.730331in}{3.207967in}}%
\pgfpathcurveto{\pgfqpoint{2.722517in}{3.200153in}}{\pgfqpoint{2.718127in}{3.189554in}}{\pgfqpoint{2.718127in}{3.178504in}}%
\pgfpathcurveto{\pgfqpoint{2.718127in}{3.167454in}}{\pgfqpoint{2.722517in}{3.156855in}}{\pgfqpoint{2.730331in}{3.149041in}}%
\pgfpathcurveto{\pgfqpoint{2.738144in}{3.141228in}}{\pgfqpoint{2.748743in}{3.136837in}}{\pgfqpoint{2.759793in}{3.136837in}}%
\pgfpathclose%
\pgfusepath{stroke,fill}%
\end{pgfscope}%
\begin{pgfscope}%
\pgfpathrectangle{\pgfqpoint{0.648703in}{0.548769in}}{\pgfqpoint{5.112893in}{3.102590in}}%
\pgfusepath{clip}%
\pgfsetbuttcap%
\pgfsetroundjoin%
\definecolor{currentfill}{rgb}{1.000000,0.498039,0.054902}%
\pgfsetfillcolor{currentfill}%
\pgfsetlinewidth{1.003750pt}%
\definecolor{currentstroke}{rgb}{1.000000,0.498039,0.054902}%
\pgfsetstrokecolor{currentstroke}%
\pgfsetdash{}{0pt}%
\pgfpathmoveto{\pgfqpoint{2.321715in}{3.136837in}}%
\pgfpathcurveto{\pgfqpoint{2.332765in}{3.136837in}}{\pgfqpoint{2.343364in}{3.141228in}}{\pgfqpoint{2.351178in}{3.149041in}}%
\pgfpathcurveto{\pgfqpoint{2.358992in}{3.156855in}}{\pgfqpoint{2.363382in}{3.167454in}}{\pgfqpoint{2.363382in}{3.178504in}}%
\pgfpathcurveto{\pgfqpoint{2.363382in}{3.189554in}}{\pgfqpoint{2.358992in}{3.200153in}}{\pgfqpoint{2.351178in}{3.207967in}}%
\pgfpathcurveto{\pgfqpoint{2.343364in}{3.215780in}}{\pgfqpoint{2.332765in}{3.220171in}}{\pgfqpoint{2.321715in}{3.220171in}}%
\pgfpathcurveto{\pgfqpoint{2.310665in}{3.220171in}}{\pgfqpoint{2.300066in}{3.215780in}}{\pgfqpoint{2.292253in}{3.207967in}}%
\pgfpathcurveto{\pgfqpoint{2.284439in}{3.200153in}}{\pgfqpoint{2.280049in}{3.189554in}}{\pgfqpoint{2.280049in}{3.178504in}}%
\pgfpathcurveto{\pgfqpoint{2.280049in}{3.167454in}}{\pgfqpoint{2.284439in}{3.156855in}}{\pgfqpoint{2.292253in}{3.149041in}}%
\pgfpathcurveto{\pgfqpoint{2.300066in}{3.141228in}}{\pgfqpoint{2.310665in}{3.136837in}}{\pgfqpoint{2.321715in}{3.136837in}}%
\pgfpathclose%
\pgfusepath{stroke,fill}%
\end{pgfscope}%
\begin{pgfscope}%
\pgfpathrectangle{\pgfqpoint{0.648703in}{0.548769in}}{\pgfqpoint{5.112893in}{3.102590in}}%
\pgfusepath{clip}%
\pgfsetbuttcap%
\pgfsetroundjoin%
\definecolor{currentfill}{rgb}{1.000000,0.498039,0.054902}%
\pgfsetfillcolor{currentfill}%
\pgfsetlinewidth{1.003750pt}%
\definecolor{currentstroke}{rgb}{1.000000,0.498039,0.054902}%
\pgfsetstrokecolor{currentstroke}%
\pgfsetdash{}{0pt}%
\pgfpathmoveto{\pgfqpoint{2.227388in}{3.136837in}}%
\pgfpathcurveto{\pgfqpoint{2.238438in}{3.136837in}}{\pgfqpoint{2.249037in}{3.141228in}}{\pgfqpoint{2.256851in}{3.149041in}}%
\pgfpathcurveto{\pgfqpoint{2.264664in}{3.156855in}}{\pgfqpoint{2.269055in}{3.167454in}}{\pgfqpoint{2.269055in}{3.178504in}}%
\pgfpathcurveto{\pgfqpoint{2.269055in}{3.189554in}}{\pgfqpoint{2.264664in}{3.200153in}}{\pgfqpoint{2.256851in}{3.207967in}}%
\pgfpathcurveto{\pgfqpoint{2.249037in}{3.215780in}}{\pgfqpoint{2.238438in}{3.220171in}}{\pgfqpoint{2.227388in}{3.220171in}}%
\pgfpathcurveto{\pgfqpoint{2.216338in}{3.220171in}}{\pgfqpoint{2.205739in}{3.215780in}}{\pgfqpoint{2.197925in}{3.207967in}}%
\pgfpathcurveto{\pgfqpoint{2.190112in}{3.200153in}}{\pgfqpoint{2.185721in}{3.189554in}}{\pgfqpoint{2.185721in}{3.178504in}}%
\pgfpathcurveto{\pgfqpoint{2.185721in}{3.167454in}}{\pgfqpoint{2.190112in}{3.156855in}}{\pgfqpoint{2.197925in}{3.149041in}}%
\pgfpathcurveto{\pgfqpoint{2.205739in}{3.141228in}}{\pgfqpoint{2.216338in}{3.136837in}}{\pgfqpoint{2.227388in}{3.136837in}}%
\pgfpathclose%
\pgfusepath{stroke,fill}%
\end{pgfscope}%
\begin{pgfscope}%
\pgfpathrectangle{\pgfqpoint{0.648703in}{0.548769in}}{\pgfqpoint{5.112893in}{3.102590in}}%
\pgfusepath{clip}%
\pgfsetbuttcap%
\pgfsetroundjoin%
\definecolor{currentfill}{rgb}{0.121569,0.466667,0.705882}%
\pgfsetfillcolor{currentfill}%
\pgfsetlinewidth{1.003750pt}%
\definecolor{currentstroke}{rgb}{0.121569,0.466667,0.705882}%
\pgfsetstrokecolor{currentstroke}%
\pgfsetdash{}{0pt}%
\pgfpathmoveto{\pgfqpoint{3.654830in}{3.132690in}}%
\pgfpathcurveto{\pgfqpoint{3.665880in}{3.132690in}}{\pgfqpoint{3.676479in}{3.137080in}}{\pgfqpoint{3.684293in}{3.144893in}}%
\pgfpathcurveto{\pgfqpoint{3.692107in}{3.152707in}}{\pgfqpoint{3.696497in}{3.163306in}}{\pgfqpoint{3.696497in}{3.174356in}}%
\pgfpathcurveto{\pgfqpoint{3.696497in}{3.185406in}}{\pgfqpoint{3.692107in}{3.196005in}}{\pgfqpoint{3.684293in}{3.203819in}}%
\pgfpathcurveto{\pgfqpoint{3.676479in}{3.211633in}}{\pgfqpoint{3.665880in}{3.216023in}}{\pgfqpoint{3.654830in}{3.216023in}}%
\pgfpathcurveto{\pgfqpoint{3.643780in}{3.216023in}}{\pgfqpoint{3.633181in}{3.211633in}}{\pgfqpoint{3.625367in}{3.203819in}}%
\pgfpathcurveto{\pgfqpoint{3.617554in}{3.196005in}}{\pgfqpoint{3.613164in}{3.185406in}}{\pgfqpoint{3.613164in}{3.174356in}}%
\pgfpathcurveto{\pgfqpoint{3.613164in}{3.163306in}}{\pgfqpoint{3.617554in}{3.152707in}}{\pgfqpoint{3.625367in}{3.144893in}}%
\pgfpathcurveto{\pgfqpoint{3.633181in}{3.137080in}}{\pgfqpoint{3.643780in}{3.132690in}}{\pgfqpoint{3.654830in}{3.132690in}}%
\pgfpathclose%
\pgfusepath{stroke,fill}%
\end{pgfscope}%
\begin{pgfscope}%
\pgfpathrectangle{\pgfqpoint{0.648703in}{0.548769in}}{\pgfqpoint{5.112893in}{3.102590in}}%
\pgfusepath{clip}%
\pgfsetbuttcap%
\pgfsetroundjoin%
\definecolor{currentfill}{rgb}{0.121569,0.466667,0.705882}%
\pgfsetfillcolor{currentfill}%
\pgfsetlinewidth{1.003750pt}%
\definecolor{currentstroke}{rgb}{0.121569,0.466667,0.705882}%
\pgfsetstrokecolor{currentstroke}%
\pgfsetdash{}{0pt}%
\pgfpathmoveto{\pgfqpoint{3.965041in}{3.107802in}}%
\pgfpathcurveto{\pgfqpoint{3.976091in}{3.107802in}}{\pgfqpoint{3.986690in}{3.112193in}}{\pgfqpoint{3.994503in}{3.120006in}}%
\pgfpathcurveto{\pgfqpoint{4.002317in}{3.127820in}}{\pgfqpoint{4.006707in}{3.138419in}}{\pgfqpoint{4.006707in}{3.149469in}}%
\pgfpathcurveto{\pgfqpoint{4.006707in}{3.160519in}}{\pgfqpoint{4.002317in}{3.171118in}}{\pgfqpoint{3.994503in}{3.178932in}}%
\pgfpathcurveto{\pgfqpoint{3.986690in}{3.186745in}}{\pgfqpoint{3.976091in}{3.191136in}}{\pgfqpoint{3.965041in}{3.191136in}}%
\pgfpathcurveto{\pgfqpoint{3.953990in}{3.191136in}}{\pgfqpoint{3.943391in}{3.186745in}}{\pgfqpoint{3.935578in}{3.178932in}}%
\pgfpathcurveto{\pgfqpoint{3.927764in}{3.171118in}}{\pgfqpoint{3.923374in}{3.160519in}}{\pgfqpoint{3.923374in}{3.149469in}}%
\pgfpathcurveto{\pgfqpoint{3.923374in}{3.138419in}}{\pgfqpoint{3.927764in}{3.127820in}}{\pgfqpoint{3.935578in}{3.120006in}}%
\pgfpathcurveto{\pgfqpoint{3.943391in}{3.112193in}}{\pgfqpoint{3.953990in}{3.107802in}}{\pgfqpoint{3.965041in}{3.107802in}}%
\pgfpathclose%
\pgfusepath{stroke,fill}%
\end{pgfscope}%
\begin{pgfscope}%
\pgfpathrectangle{\pgfqpoint{0.648703in}{0.548769in}}{\pgfqpoint{5.112893in}{3.102590in}}%
\pgfusepath{clip}%
\pgfsetbuttcap%
\pgfsetroundjoin%
\definecolor{currentfill}{rgb}{0.121569,0.466667,0.705882}%
\pgfsetfillcolor{currentfill}%
\pgfsetlinewidth{1.003750pt}%
\definecolor{currentstroke}{rgb}{0.121569,0.466667,0.705882}%
\pgfsetstrokecolor{currentstroke}%
\pgfsetdash{}{0pt}%
\pgfpathmoveto{\pgfqpoint{4.057762in}{3.132690in}}%
\pgfpathcurveto{\pgfqpoint{4.068813in}{3.132690in}}{\pgfqpoint{4.079412in}{3.137080in}}{\pgfqpoint{4.087225in}{3.144893in}}%
\pgfpathcurveto{\pgfqpoint{4.095039in}{3.152707in}}{\pgfqpoint{4.099429in}{3.163306in}}{\pgfqpoint{4.099429in}{3.174356in}}%
\pgfpathcurveto{\pgfqpoint{4.099429in}{3.185406in}}{\pgfqpoint{4.095039in}{3.196005in}}{\pgfqpoint{4.087225in}{3.203819in}}%
\pgfpathcurveto{\pgfqpoint{4.079412in}{3.211633in}}{\pgfqpoint{4.068813in}{3.216023in}}{\pgfqpoint{4.057762in}{3.216023in}}%
\pgfpathcurveto{\pgfqpoint{4.046712in}{3.216023in}}{\pgfqpoint{4.036113in}{3.211633in}}{\pgfqpoint{4.028300in}{3.203819in}}%
\pgfpathcurveto{\pgfqpoint{4.020486in}{3.196005in}}{\pgfqpoint{4.016096in}{3.185406in}}{\pgfqpoint{4.016096in}{3.174356in}}%
\pgfpathcurveto{\pgfqpoint{4.016096in}{3.163306in}}{\pgfqpoint{4.020486in}{3.152707in}}{\pgfqpoint{4.028300in}{3.144893in}}%
\pgfpathcurveto{\pgfqpoint{4.036113in}{3.137080in}}{\pgfqpoint{4.046712in}{3.132690in}}{\pgfqpoint{4.057762in}{3.132690in}}%
\pgfpathclose%
\pgfusepath{stroke,fill}%
\end{pgfscope}%
\begin{pgfscope}%
\pgfpathrectangle{\pgfqpoint{0.648703in}{0.548769in}}{\pgfqpoint{5.112893in}{3.102590in}}%
\pgfusepath{clip}%
\pgfsetbuttcap%
\pgfsetroundjoin%
\definecolor{currentfill}{rgb}{0.121569,0.466667,0.705882}%
\pgfsetfillcolor{currentfill}%
\pgfsetlinewidth{1.003750pt}%
\definecolor{currentstroke}{rgb}{0.121569,0.466667,0.705882}%
\pgfsetstrokecolor{currentstroke}%
\pgfsetdash{}{0pt}%
\pgfpathmoveto{\pgfqpoint{0.873122in}{0.648129in}}%
\pgfpathcurveto{\pgfqpoint{0.884172in}{0.648129in}}{\pgfqpoint{0.894771in}{0.652519in}}{\pgfqpoint{0.902584in}{0.660333in}}%
\pgfpathcurveto{\pgfqpoint{0.910398in}{0.668146in}}{\pgfqpoint{0.914788in}{0.678745in}}{\pgfqpoint{0.914788in}{0.689796in}}%
\pgfpathcurveto{\pgfqpoint{0.914788in}{0.700846in}}{\pgfqpoint{0.910398in}{0.711445in}}{\pgfqpoint{0.902584in}{0.719258in}}%
\pgfpathcurveto{\pgfqpoint{0.894771in}{0.727072in}}{\pgfqpoint{0.884172in}{0.731462in}}{\pgfqpoint{0.873122in}{0.731462in}}%
\pgfpathcurveto{\pgfqpoint{0.862071in}{0.731462in}}{\pgfqpoint{0.851472in}{0.727072in}}{\pgfqpoint{0.843659in}{0.719258in}}%
\pgfpathcurveto{\pgfqpoint{0.835845in}{0.711445in}}{\pgfqpoint{0.831455in}{0.700846in}}{\pgfqpoint{0.831455in}{0.689796in}}%
\pgfpathcurveto{\pgfqpoint{0.831455in}{0.678745in}}{\pgfqpoint{0.835845in}{0.668146in}}{\pgfqpoint{0.843659in}{0.660333in}}%
\pgfpathcurveto{\pgfqpoint{0.851472in}{0.652519in}}{\pgfqpoint{0.862071in}{0.648129in}}{\pgfqpoint{0.873122in}{0.648129in}}%
\pgfpathclose%
\pgfusepath{stroke,fill}%
\end{pgfscope}%
\begin{pgfscope}%
\pgfpathrectangle{\pgfqpoint{0.648703in}{0.548769in}}{\pgfqpoint{5.112893in}{3.102590in}}%
\pgfusepath{clip}%
\pgfsetbuttcap%
\pgfsetroundjoin%
\definecolor{currentfill}{rgb}{1.000000,0.498039,0.054902}%
\pgfsetfillcolor{currentfill}%
\pgfsetlinewidth{1.003750pt}%
\definecolor{currentstroke}{rgb}{1.000000,0.498039,0.054902}%
\pgfsetstrokecolor{currentstroke}%
\pgfsetdash{}{0pt}%
\pgfpathmoveto{\pgfqpoint{1.718907in}{3.140985in}}%
\pgfpathcurveto{\pgfqpoint{1.729957in}{3.140985in}}{\pgfqpoint{1.740556in}{3.145375in}}{\pgfqpoint{1.748370in}{3.153189in}}%
\pgfpathcurveto{\pgfqpoint{1.756183in}{3.161003in}}{\pgfqpoint{1.760574in}{3.171602in}}{\pgfqpoint{1.760574in}{3.182652in}}%
\pgfpathcurveto{\pgfqpoint{1.760574in}{3.193702in}}{\pgfqpoint{1.756183in}{3.204301in}}{\pgfqpoint{1.748370in}{3.212115in}}%
\pgfpathcurveto{\pgfqpoint{1.740556in}{3.219928in}}{\pgfqpoint{1.729957in}{3.224319in}}{\pgfqpoint{1.718907in}{3.224319in}}%
\pgfpathcurveto{\pgfqpoint{1.707857in}{3.224319in}}{\pgfqpoint{1.697258in}{3.219928in}}{\pgfqpoint{1.689444in}{3.212115in}}%
\pgfpathcurveto{\pgfqpoint{1.681631in}{3.204301in}}{\pgfqpoint{1.677240in}{3.193702in}}{\pgfqpoint{1.677240in}{3.182652in}}%
\pgfpathcurveto{\pgfqpoint{1.677240in}{3.171602in}}{\pgfqpoint{1.681631in}{3.161003in}}{\pgfqpoint{1.689444in}{3.153189in}}%
\pgfpathcurveto{\pgfqpoint{1.697258in}{3.145375in}}{\pgfqpoint{1.707857in}{3.140985in}}{\pgfqpoint{1.718907in}{3.140985in}}%
\pgfpathclose%
\pgfusepath{stroke,fill}%
\end{pgfscope}%
\begin{pgfscope}%
\pgfpathrectangle{\pgfqpoint{0.648703in}{0.548769in}}{\pgfqpoint{5.112893in}{3.102590in}}%
\pgfusepath{clip}%
\pgfsetbuttcap%
\pgfsetroundjoin%
\definecolor{currentfill}{rgb}{1.000000,0.498039,0.054902}%
\pgfsetfillcolor{currentfill}%
\pgfsetlinewidth{1.003750pt}%
\definecolor{currentstroke}{rgb}{1.000000,0.498039,0.054902}%
\pgfsetstrokecolor{currentstroke}%
\pgfsetdash{}{0pt}%
\pgfpathmoveto{\pgfqpoint{4.197899in}{3.145133in}}%
\pgfpathcurveto{\pgfqpoint{4.208949in}{3.145133in}}{\pgfqpoint{4.219548in}{3.149523in}}{\pgfqpoint{4.227362in}{3.157337in}}%
\pgfpathcurveto{\pgfqpoint{4.235175in}{3.165151in}}{\pgfqpoint{4.239566in}{3.175750in}}{\pgfqpoint{4.239566in}{3.186800in}}%
\pgfpathcurveto{\pgfqpoint{4.239566in}{3.197850in}}{\pgfqpoint{4.235175in}{3.208449in}}{\pgfqpoint{4.227362in}{3.216262in}}%
\pgfpathcurveto{\pgfqpoint{4.219548in}{3.224076in}}{\pgfqpoint{4.208949in}{3.228466in}}{\pgfqpoint{4.197899in}{3.228466in}}%
\pgfpathcurveto{\pgfqpoint{4.186849in}{3.228466in}}{\pgfqpoint{4.176250in}{3.224076in}}{\pgfqpoint{4.168436in}{3.216262in}}%
\pgfpathcurveto{\pgfqpoint{4.160623in}{3.208449in}}{\pgfqpoint{4.156232in}{3.197850in}}{\pgfqpoint{4.156232in}{3.186800in}}%
\pgfpathcurveto{\pgfqpoint{4.156232in}{3.175750in}}{\pgfqpoint{4.160623in}{3.165151in}}{\pgfqpoint{4.168436in}{3.157337in}}%
\pgfpathcurveto{\pgfqpoint{4.176250in}{3.149523in}}{\pgfqpoint{4.186849in}{3.145133in}}{\pgfqpoint{4.197899in}{3.145133in}}%
\pgfpathclose%
\pgfusepath{stroke,fill}%
\end{pgfscope}%
\begin{pgfscope}%
\pgfpathrectangle{\pgfqpoint{0.648703in}{0.548769in}}{\pgfqpoint{5.112893in}{3.102590in}}%
\pgfusepath{clip}%
\pgfsetbuttcap%
\pgfsetroundjoin%
\definecolor{currentfill}{rgb}{1.000000,0.498039,0.054902}%
\pgfsetfillcolor{currentfill}%
\pgfsetlinewidth{1.003750pt}%
\definecolor{currentstroke}{rgb}{1.000000,0.498039,0.054902}%
\pgfsetstrokecolor{currentstroke}%
\pgfsetdash{}{0pt}%
\pgfpathmoveto{\pgfqpoint{4.492390in}{3.140985in}}%
\pgfpathcurveto{\pgfqpoint{4.503440in}{3.140985in}}{\pgfqpoint{4.514039in}{3.145375in}}{\pgfqpoint{4.521853in}{3.153189in}}%
\pgfpathcurveto{\pgfqpoint{4.529667in}{3.161003in}}{\pgfqpoint{4.534057in}{3.171602in}}{\pgfqpoint{4.534057in}{3.182652in}}%
\pgfpathcurveto{\pgfqpoint{4.534057in}{3.193702in}}{\pgfqpoint{4.529667in}{3.204301in}}{\pgfqpoint{4.521853in}{3.212115in}}%
\pgfpathcurveto{\pgfqpoint{4.514039in}{3.219928in}}{\pgfqpoint{4.503440in}{3.224319in}}{\pgfqpoint{4.492390in}{3.224319in}}%
\pgfpathcurveto{\pgfqpoint{4.481340in}{3.224319in}}{\pgfqpoint{4.470741in}{3.219928in}}{\pgfqpoint{4.462927in}{3.212115in}}%
\pgfpathcurveto{\pgfqpoint{4.455114in}{3.204301in}}{\pgfqpoint{4.450724in}{3.193702in}}{\pgfqpoint{4.450724in}{3.182652in}}%
\pgfpathcurveto{\pgfqpoint{4.450724in}{3.171602in}}{\pgfqpoint{4.455114in}{3.161003in}}{\pgfqpoint{4.462927in}{3.153189in}}%
\pgfpathcurveto{\pgfqpoint{4.470741in}{3.145375in}}{\pgfqpoint{4.481340in}{3.140985in}}{\pgfqpoint{4.492390in}{3.140985in}}%
\pgfpathclose%
\pgfusepath{stroke,fill}%
\end{pgfscope}%
\begin{pgfscope}%
\pgfpathrectangle{\pgfqpoint{0.648703in}{0.548769in}}{\pgfqpoint{5.112893in}{3.102590in}}%
\pgfusepath{clip}%
\pgfsetbuttcap%
\pgfsetroundjoin%
\definecolor{currentfill}{rgb}{0.121569,0.466667,0.705882}%
\pgfsetfillcolor{currentfill}%
\pgfsetlinewidth{1.003750pt}%
\definecolor{currentstroke}{rgb}{0.121569,0.466667,0.705882}%
\pgfsetstrokecolor{currentstroke}%
\pgfsetdash{}{0pt}%
\pgfpathmoveto{\pgfqpoint{1.026801in}{0.697903in}}%
\pgfpathcurveto{\pgfqpoint{1.037851in}{0.697903in}}{\pgfqpoint{1.048450in}{0.702293in}}{\pgfqpoint{1.056264in}{0.710107in}}%
\pgfpathcurveto{\pgfqpoint{1.064078in}{0.717921in}}{\pgfqpoint{1.068468in}{0.728520in}}{\pgfqpoint{1.068468in}{0.739570in}}%
\pgfpathcurveto{\pgfqpoint{1.068468in}{0.750620in}}{\pgfqpoint{1.064078in}{0.761219in}}{\pgfqpoint{1.056264in}{0.769033in}}%
\pgfpathcurveto{\pgfqpoint{1.048450in}{0.776846in}}{\pgfqpoint{1.037851in}{0.781236in}}{\pgfqpoint{1.026801in}{0.781236in}}%
\pgfpathcurveto{\pgfqpoint{1.015751in}{0.781236in}}{\pgfqpoint{1.005152in}{0.776846in}}{\pgfqpoint{0.997338in}{0.769033in}}%
\pgfpathcurveto{\pgfqpoint{0.989525in}{0.761219in}}{\pgfqpoint{0.985134in}{0.750620in}}{\pgfqpoint{0.985134in}{0.739570in}}%
\pgfpathcurveto{\pgfqpoint{0.985134in}{0.728520in}}{\pgfqpoint{0.989525in}{0.717921in}}{\pgfqpoint{0.997338in}{0.710107in}}%
\pgfpathcurveto{\pgfqpoint{1.005152in}{0.702293in}}{\pgfqpoint{1.015751in}{0.697903in}}{\pgfqpoint{1.026801in}{0.697903in}}%
\pgfpathclose%
\pgfusepath{stroke,fill}%
\end{pgfscope}%
\begin{pgfscope}%
\pgfpathrectangle{\pgfqpoint{0.648703in}{0.548769in}}{\pgfqpoint{5.112893in}{3.102590in}}%
\pgfusepath{clip}%
\pgfsetbuttcap%
\pgfsetroundjoin%
\definecolor{currentfill}{rgb}{1.000000,0.498039,0.054902}%
\pgfsetfillcolor{currentfill}%
\pgfsetlinewidth{1.003750pt}%
\definecolor{currentstroke}{rgb}{1.000000,0.498039,0.054902}%
\pgfsetstrokecolor{currentstroke}%
\pgfsetdash{}{0pt}%
\pgfpathmoveto{\pgfqpoint{1.968945in}{3.140985in}}%
\pgfpathcurveto{\pgfqpoint{1.979995in}{3.140985in}}{\pgfqpoint{1.990594in}{3.145375in}}{\pgfqpoint{1.998408in}{3.153189in}}%
\pgfpathcurveto{\pgfqpoint{2.006222in}{3.161003in}}{\pgfqpoint{2.010612in}{3.171602in}}{\pgfqpoint{2.010612in}{3.182652in}}%
\pgfpathcurveto{\pgfqpoint{2.010612in}{3.193702in}}{\pgfqpoint{2.006222in}{3.204301in}}{\pgfqpoint{1.998408in}{3.212115in}}%
\pgfpathcurveto{\pgfqpoint{1.990594in}{3.219928in}}{\pgfqpoint{1.979995in}{3.224319in}}{\pgfqpoint{1.968945in}{3.224319in}}%
\pgfpathcurveto{\pgfqpoint{1.957895in}{3.224319in}}{\pgfqpoint{1.947296in}{3.219928in}}{\pgfqpoint{1.939482in}{3.212115in}}%
\pgfpathcurveto{\pgfqpoint{1.931669in}{3.204301in}}{\pgfqpoint{1.927278in}{3.193702in}}{\pgfqpoint{1.927278in}{3.182652in}}%
\pgfpathcurveto{\pgfqpoint{1.927278in}{3.171602in}}{\pgfqpoint{1.931669in}{3.161003in}}{\pgfqpoint{1.939482in}{3.153189in}}%
\pgfpathcurveto{\pgfqpoint{1.947296in}{3.145375in}}{\pgfqpoint{1.957895in}{3.140985in}}{\pgfqpoint{1.968945in}{3.140985in}}%
\pgfpathclose%
\pgfusepath{stroke,fill}%
\end{pgfscope}%
\begin{pgfscope}%
\pgfpathrectangle{\pgfqpoint{0.648703in}{0.548769in}}{\pgfqpoint{5.112893in}{3.102590in}}%
\pgfusepath{clip}%
\pgfsetbuttcap%
\pgfsetroundjoin%
\definecolor{currentfill}{rgb}{0.121569,0.466667,0.705882}%
\pgfsetfillcolor{currentfill}%
\pgfsetlinewidth{1.003750pt}%
\definecolor{currentstroke}{rgb}{0.121569,0.466667,0.705882}%
\pgfsetstrokecolor{currentstroke}%
\pgfsetdash{}{0pt}%
\pgfpathmoveto{\pgfqpoint{3.336160in}{3.132690in}}%
\pgfpathcurveto{\pgfqpoint{3.347210in}{3.132690in}}{\pgfqpoint{3.357809in}{3.137080in}}{\pgfqpoint{3.365623in}{3.144893in}}%
\pgfpathcurveto{\pgfqpoint{3.373436in}{3.152707in}}{\pgfqpoint{3.377827in}{3.163306in}}{\pgfqpoint{3.377827in}{3.174356in}}%
\pgfpathcurveto{\pgfqpoint{3.377827in}{3.185406in}}{\pgfqpoint{3.373436in}{3.196005in}}{\pgfqpoint{3.365623in}{3.203819in}}%
\pgfpathcurveto{\pgfqpoint{3.357809in}{3.211633in}}{\pgfqpoint{3.347210in}{3.216023in}}{\pgfqpoint{3.336160in}{3.216023in}}%
\pgfpathcurveto{\pgfqpoint{3.325110in}{3.216023in}}{\pgfqpoint{3.314511in}{3.211633in}}{\pgfqpoint{3.306697in}{3.203819in}}%
\pgfpathcurveto{\pgfqpoint{3.298884in}{3.196005in}}{\pgfqpoint{3.294493in}{3.185406in}}{\pgfqpoint{3.294493in}{3.174356in}}%
\pgfpathcurveto{\pgfqpoint{3.294493in}{3.163306in}}{\pgfqpoint{3.298884in}{3.152707in}}{\pgfqpoint{3.306697in}{3.144893in}}%
\pgfpathcurveto{\pgfqpoint{3.314511in}{3.137080in}}{\pgfqpoint{3.325110in}{3.132690in}}{\pgfqpoint{3.336160in}{3.132690in}}%
\pgfpathclose%
\pgfusepath{stroke,fill}%
\end{pgfscope}%
\begin{pgfscope}%
\pgfpathrectangle{\pgfqpoint{0.648703in}{0.548769in}}{\pgfqpoint{5.112893in}{3.102590in}}%
\pgfusepath{clip}%
\pgfsetbuttcap%
\pgfsetroundjoin%
\definecolor{currentfill}{rgb}{1.000000,0.498039,0.054902}%
\pgfsetfillcolor{currentfill}%
\pgfsetlinewidth{1.003750pt}%
\definecolor{currentstroke}{rgb}{1.000000,0.498039,0.054902}%
\pgfsetstrokecolor{currentstroke}%
\pgfsetdash{}{0pt}%
\pgfpathmoveto{\pgfqpoint{4.094322in}{3.140985in}}%
\pgfpathcurveto{\pgfqpoint{4.105373in}{3.140985in}}{\pgfqpoint{4.115972in}{3.145375in}}{\pgfqpoint{4.123785in}{3.153189in}}%
\pgfpathcurveto{\pgfqpoint{4.131599in}{3.161003in}}{\pgfqpoint{4.135989in}{3.171602in}}{\pgfqpoint{4.135989in}{3.182652in}}%
\pgfpathcurveto{\pgfqpoint{4.135989in}{3.193702in}}{\pgfqpoint{4.131599in}{3.204301in}}{\pgfqpoint{4.123785in}{3.212115in}}%
\pgfpathcurveto{\pgfqpoint{4.115972in}{3.219928in}}{\pgfqpoint{4.105373in}{3.224319in}}{\pgfqpoint{4.094322in}{3.224319in}}%
\pgfpathcurveto{\pgfqpoint{4.083272in}{3.224319in}}{\pgfqpoint{4.072673in}{3.219928in}}{\pgfqpoint{4.064860in}{3.212115in}}%
\pgfpathcurveto{\pgfqpoint{4.057046in}{3.204301in}}{\pgfqpoint{4.052656in}{3.193702in}}{\pgfqpoint{4.052656in}{3.182652in}}%
\pgfpathcurveto{\pgfqpoint{4.052656in}{3.171602in}}{\pgfqpoint{4.057046in}{3.161003in}}{\pgfqpoint{4.064860in}{3.153189in}}%
\pgfpathcurveto{\pgfqpoint{4.072673in}{3.145375in}}{\pgfqpoint{4.083272in}{3.140985in}}{\pgfqpoint{4.094322in}{3.140985in}}%
\pgfpathclose%
\pgfusepath{stroke,fill}%
\end{pgfscope}%
\begin{pgfscope}%
\pgfpathrectangle{\pgfqpoint{0.648703in}{0.548769in}}{\pgfqpoint{5.112893in}{3.102590in}}%
\pgfusepath{clip}%
\pgfsetbuttcap%
\pgfsetroundjoin%
\definecolor{currentfill}{rgb}{0.121569,0.466667,0.705882}%
\pgfsetfillcolor{currentfill}%
\pgfsetlinewidth{1.003750pt}%
\definecolor{currentstroke}{rgb}{0.121569,0.466667,0.705882}%
\pgfsetstrokecolor{currentstroke}%
\pgfsetdash{}{0pt}%
\pgfpathmoveto{\pgfqpoint{1.709003in}{3.132690in}}%
\pgfpathcurveto{\pgfqpoint{1.720053in}{3.132690in}}{\pgfqpoint{1.730652in}{3.137080in}}{\pgfqpoint{1.738466in}{3.144893in}}%
\pgfpathcurveto{\pgfqpoint{1.746279in}{3.152707in}}{\pgfqpoint{1.750670in}{3.163306in}}{\pgfqpoint{1.750670in}{3.174356in}}%
\pgfpathcurveto{\pgfqpoint{1.750670in}{3.185406in}}{\pgfqpoint{1.746279in}{3.196005in}}{\pgfqpoint{1.738466in}{3.203819in}}%
\pgfpathcurveto{\pgfqpoint{1.730652in}{3.211633in}}{\pgfqpoint{1.720053in}{3.216023in}}{\pgfqpoint{1.709003in}{3.216023in}}%
\pgfpathcurveto{\pgfqpoint{1.697953in}{3.216023in}}{\pgfqpoint{1.687354in}{3.211633in}}{\pgfqpoint{1.679540in}{3.203819in}}%
\pgfpathcurveto{\pgfqpoint{1.671726in}{3.196005in}}{\pgfqpoint{1.667336in}{3.185406in}}{\pgfqpoint{1.667336in}{3.174356in}}%
\pgfpathcurveto{\pgfqpoint{1.667336in}{3.163306in}}{\pgfqpoint{1.671726in}{3.152707in}}{\pgfqpoint{1.679540in}{3.144893in}}%
\pgfpathcurveto{\pgfqpoint{1.687354in}{3.137080in}}{\pgfqpoint{1.697953in}{3.132690in}}{\pgfqpoint{1.709003in}{3.132690in}}%
\pgfpathclose%
\pgfusepath{stroke,fill}%
\end{pgfscope}%
\begin{pgfscope}%
\pgfpathrectangle{\pgfqpoint{0.648703in}{0.548769in}}{\pgfqpoint{5.112893in}{3.102590in}}%
\pgfusepath{clip}%
\pgfsetbuttcap%
\pgfsetroundjoin%
\definecolor{currentfill}{rgb}{1.000000,0.498039,0.054902}%
\pgfsetfillcolor{currentfill}%
\pgfsetlinewidth{1.003750pt}%
\definecolor{currentstroke}{rgb}{1.000000,0.498039,0.054902}%
\pgfsetstrokecolor{currentstroke}%
\pgfsetdash{}{0pt}%
\pgfpathmoveto{\pgfqpoint{3.390280in}{3.140985in}}%
\pgfpathcurveto{\pgfqpoint{3.401330in}{3.140985in}}{\pgfqpoint{3.411929in}{3.145375in}}{\pgfqpoint{3.419743in}{3.153189in}}%
\pgfpathcurveto{\pgfqpoint{3.427557in}{3.161003in}}{\pgfqpoint{3.431947in}{3.171602in}}{\pgfqpoint{3.431947in}{3.182652in}}%
\pgfpathcurveto{\pgfqpoint{3.431947in}{3.193702in}}{\pgfqpoint{3.427557in}{3.204301in}}{\pgfqpoint{3.419743in}{3.212115in}}%
\pgfpathcurveto{\pgfqpoint{3.411929in}{3.219928in}}{\pgfqpoint{3.401330in}{3.224319in}}{\pgfqpoint{3.390280in}{3.224319in}}%
\pgfpathcurveto{\pgfqpoint{3.379230in}{3.224319in}}{\pgfqpoint{3.368631in}{3.219928in}}{\pgfqpoint{3.360818in}{3.212115in}}%
\pgfpathcurveto{\pgfqpoint{3.353004in}{3.204301in}}{\pgfqpoint{3.348614in}{3.193702in}}{\pgfqpoint{3.348614in}{3.182652in}}%
\pgfpathcurveto{\pgfqpoint{3.348614in}{3.171602in}}{\pgfqpoint{3.353004in}{3.161003in}}{\pgfqpoint{3.360818in}{3.153189in}}%
\pgfpathcurveto{\pgfqpoint{3.368631in}{3.145375in}}{\pgfqpoint{3.379230in}{3.140985in}}{\pgfqpoint{3.390280in}{3.140985in}}%
\pgfpathclose%
\pgfusepath{stroke,fill}%
\end{pgfscope}%
\begin{pgfscope}%
\pgfpathrectangle{\pgfqpoint{0.648703in}{0.548769in}}{\pgfqpoint{5.112893in}{3.102590in}}%
\pgfusepath{clip}%
\pgfsetbuttcap%
\pgfsetroundjoin%
\definecolor{currentfill}{rgb}{1.000000,0.498039,0.054902}%
\pgfsetfillcolor{currentfill}%
\pgfsetlinewidth{1.003750pt}%
\definecolor{currentstroke}{rgb}{1.000000,0.498039,0.054902}%
\pgfsetstrokecolor{currentstroke}%
\pgfsetdash{}{0pt}%
\pgfpathmoveto{\pgfqpoint{4.071548in}{3.145133in}}%
\pgfpathcurveto{\pgfqpoint{4.082598in}{3.145133in}}{\pgfqpoint{4.093197in}{3.149523in}}{\pgfqpoint{4.101011in}{3.157337in}}%
\pgfpathcurveto{\pgfqpoint{4.108824in}{3.165151in}}{\pgfqpoint{4.113215in}{3.175750in}}{\pgfqpoint{4.113215in}{3.186800in}}%
\pgfpathcurveto{\pgfqpoint{4.113215in}{3.197850in}}{\pgfqpoint{4.108824in}{3.208449in}}{\pgfqpoint{4.101011in}{3.216262in}}%
\pgfpathcurveto{\pgfqpoint{4.093197in}{3.224076in}}{\pgfqpoint{4.082598in}{3.228466in}}{\pgfqpoint{4.071548in}{3.228466in}}%
\pgfpathcurveto{\pgfqpoint{4.060498in}{3.228466in}}{\pgfqpoint{4.049899in}{3.224076in}}{\pgfqpoint{4.042085in}{3.216262in}}%
\pgfpathcurveto{\pgfqpoint{4.034272in}{3.208449in}}{\pgfqpoint{4.029881in}{3.197850in}}{\pgfqpoint{4.029881in}{3.186800in}}%
\pgfpathcurveto{\pgfqpoint{4.029881in}{3.175750in}}{\pgfqpoint{4.034272in}{3.165151in}}{\pgfqpoint{4.042085in}{3.157337in}}%
\pgfpathcurveto{\pgfqpoint{4.049899in}{3.149523in}}{\pgfqpoint{4.060498in}{3.145133in}}{\pgfqpoint{4.071548in}{3.145133in}}%
\pgfpathclose%
\pgfusepath{stroke,fill}%
\end{pgfscope}%
\begin{pgfscope}%
\pgfpathrectangle{\pgfqpoint{0.648703in}{0.548769in}}{\pgfqpoint{5.112893in}{3.102590in}}%
\pgfusepath{clip}%
\pgfsetbuttcap%
\pgfsetroundjoin%
\definecolor{currentfill}{rgb}{1.000000,0.498039,0.054902}%
\pgfsetfillcolor{currentfill}%
\pgfsetlinewidth{1.003750pt}%
\definecolor{currentstroke}{rgb}{1.000000,0.498039,0.054902}%
\pgfsetstrokecolor{currentstroke}%
\pgfsetdash{}{0pt}%
\pgfpathmoveto{\pgfqpoint{2.739559in}{3.145133in}}%
\pgfpathcurveto{\pgfqpoint{2.750609in}{3.145133in}}{\pgfqpoint{2.761208in}{3.149523in}}{\pgfqpoint{2.769022in}{3.157337in}}%
\pgfpathcurveto{\pgfqpoint{2.776835in}{3.165151in}}{\pgfqpoint{2.781226in}{3.175750in}}{\pgfqpoint{2.781226in}{3.186800in}}%
\pgfpathcurveto{\pgfqpoint{2.781226in}{3.197850in}}{\pgfqpoint{2.776835in}{3.208449in}}{\pgfqpoint{2.769022in}{3.216262in}}%
\pgfpathcurveto{\pgfqpoint{2.761208in}{3.224076in}}{\pgfqpoint{2.750609in}{3.228466in}}{\pgfqpoint{2.739559in}{3.228466in}}%
\pgfpathcurveto{\pgfqpoint{2.728509in}{3.228466in}}{\pgfqpoint{2.717910in}{3.224076in}}{\pgfqpoint{2.710096in}{3.216262in}}%
\pgfpathcurveto{\pgfqpoint{2.702283in}{3.208449in}}{\pgfqpoint{2.697892in}{3.197850in}}{\pgfqpoint{2.697892in}{3.186800in}}%
\pgfpathcurveto{\pgfqpoint{2.697892in}{3.175750in}}{\pgfqpoint{2.702283in}{3.165151in}}{\pgfqpoint{2.710096in}{3.157337in}}%
\pgfpathcurveto{\pgfqpoint{2.717910in}{3.149523in}}{\pgfqpoint{2.728509in}{3.145133in}}{\pgfqpoint{2.739559in}{3.145133in}}%
\pgfpathclose%
\pgfusepath{stroke,fill}%
\end{pgfscope}%
\begin{pgfscope}%
\pgfpathrectangle{\pgfqpoint{0.648703in}{0.548769in}}{\pgfqpoint{5.112893in}{3.102590in}}%
\pgfusepath{clip}%
\pgfsetbuttcap%
\pgfsetroundjoin%
\definecolor{currentfill}{rgb}{1.000000,0.498039,0.054902}%
\pgfsetfillcolor{currentfill}%
\pgfsetlinewidth{1.003750pt}%
\definecolor{currentstroke}{rgb}{1.000000,0.498039,0.054902}%
\pgfsetstrokecolor{currentstroke}%
\pgfsetdash{}{0pt}%
\pgfpathmoveto{\pgfqpoint{2.345098in}{3.136837in}}%
\pgfpathcurveto{\pgfqpoint{2.356148in}{3.136837in}}{\pgfqpoint{2.366747in}{3.141228in}}{\pgfqpoint{2.374561in}{3.149041in}}%
\pgfpathcurveto{\pgfqpoint{2.382374in}{3.156855in}}{\pgfqpoint{2.386765in}{3.167454in}}{\pgfqpoint{2.386765in}{3.178504in}}%
\pgfpathcurveto{\pgfqpoint{2.386765in}{3.189554in}}{\pgfqpoint{2.382374in}{3.200153in}}{\pgfqpoint{2.374561in}{3.207967in}}%
\pgfpathcurveto{\pgfqpoint{2.366747in}{3.215780in}}{\pgfqpoint{2.356148in}{3.220171in}}{\pgfqpoint{2.345098in}{3.220171in}}%
\pgfpathcurveto{\pgfqpoint{2.334048in}{3.220171in}}{\pgfqpoint{2.323449in}{3.215780in}}{\pgfqpoint{2.315635in}{3.207967in}}%
\pgfpathcurveto{\pgfqpoint{2.307822in}{3.200153in}}{\pgfqpoint{2.303431in}{3.189554in}}{\pgfqpoint{2.303431in}{3.178504in}}%
\pgfpathcurveto{\pgfqpoint{2.303431in}{3.167454in}}{\pgfqpoint{2.307822in}{3.156855in}}{\pgfqpoint{2.315635in}{3.149041in}}%
\pgfpathcurveto{\pgfqpoint{2.323449in}{3.141228in}}{\pgfqpoint{2.334048in}{3.136837in}}{\pgfqpoint{2.345098in}{3.136837in}}%
\pgfpathclose%
\pgfusepath{stroke,fill}%
\end{pgfscope}%
\begin{pgfscope}%
\pgfpathrectangle{\pgfqpoint{0.648703in}{0.548769in}}{\pgfqpoint{5.112893in}{3.102590in}}%
\pgfusepath{clip}%
\pgfsetbuttcap%
\pgfsetroundjoin%
\definecolor{currentfill}{rgb}{1.000000,0.498039,0.054902}%
\pgfsetfillcolor{currentfill}%
\pgfsetlinewidth{1.003750pt}%
\definecolor{currentstroke}{rgb}{1.000000,0.498039,0.054902}%
\pgfsetstrokecolor{currentstroke}%
\pgfsetdash{}{0pt}%
\pgfpathmoveto{\pgfqpoint{4.259251in}{3.468665in}}%
\pgfpathcurveto{\pgfqpoint{4.270302in}{3.468665in}}{\pgfqpoint{4.280901in}{3.473055in}}{\pgfqpoint{4.288714in}{3.480869in}}%
\pgfpathcurveto{\pgfqpoint{4.296528in}{3.488683in}}{\pgfqpoint{4.300918in}{3.499282in}}{\pgfqpoint{4.300918in}{3.510332in}}%
\pgfpathcurveto{\pgfqpoint{4.300918in}{3.521382in}}{\pgfqpoint{4.296528in}{3.531981in}}{\pgfqpoint{4.288714in}{3.539795in}}%
\pgfpathcurveto{\pgfqpoint{4.280901in}{3.547608in}}{\pgfqpoint{4.270302in}{3.551998in}}{\pgfqpoint{4.259251in}{3.551998in}}%
\pgfpathcurveto{\pgfqpoint{4.248201in}{3.551998in}}{\pgfqpoint{4.237602in}{3.547608in}}{\pgfqpoint{4.229789in}{3.539795in}}%
\pgfpathcurveto{\pgfqpoint{4.221975in}{3.531981in}}{\pgfqpoint{4.217585in}{3.521382in}}{\pgfqpoint{4.217585in}{3.510332in}}%
\pgfpathcurveto{\pgfqpoint{4.217585in}{3.499282in}}{\pgfqpoint{4.221975in}{3.488683in}}{\pgfqpoint{4.229789in}{3.480869in}}%
\pgfpathcurveto{\pgfqpoint{4.237602in}{3.473055in}}{\pgfqpoint{4.248201in}{3.468665in}}{\pgfqpoint{4.259251in}{3.468665in}}%
\pgfpathclose%
\pgfusepath{stroke,fill}%
\end{pgfscope}%
\begin{pgfscope}%
\pgfpathrectangle{\pgfqpoint{0.648703in}{0.548769in}}{\pgfqpoint{5.112893in}{3.102590in}}%
\pgfusepath{clip}%
\pgfsetbuttcap%
\pgfsetroundjoin%
\definecolor{currentfill}{rgb}{1.000000,0.498039,0.054902}%
\pgfsetfillcolor{currentfill}%
\pgfsetlinewidth{1.003750pt}%
\definecolor{currentstroke}{rgb}{1.000000,0.498039,0.054902}%
\pgfsetstrokecolor{currentstroke}%
\pgfsetdash{}{0pt}%
\pgfpathmoveto{\pgfqpoint{2.813613in}{3.136837in}}%
\pgfpathcurveto{\pgfqpoint{2.824663in}{3.136837in}}{\pgfqpoint{2.835262in}{3.141228in}}{\pgfqpoint{2.843076in}{3.149041in}}%
\pgfpathcurveto{\pgfqpoint{2.850889in}{3.156855in}}{\pgfqpoint{2.855280in}{3.167454in}}{\pgfqpoint{2.855280in}{3.178504in}}%
\pgfpathcurveto{\pgfqpoint{2.855280in}{3.189554in}}{\pgfqpoint{2.850889in}{3.200153in}}{\pgfqpoint{2.843076in}{3.207967in}}%
\pgfpathcurveto{\pgfqpoint{2.835262in}{3.215780in}}{\pgfqpoint{2.824663in}{3.220171in}}{\pgfqpoint{2.813613in}{3.220171in}}%
\pgfpathcurveto{\pgfqpoint{2.802563in}{3.220171in}}{\pgfqpoint{2.791964in}{3.215780in}}{\pgfqpoint{2.784150in}{3.207967in}}%
\pgfpathcurveto{\pgfqpoint{2.776337in}{3.200153in}}{\pgfqpoint{2.771946in}{3.189554in}}{\pgfqpoint{2.771946in}{3.178504in}}%
\pgfpathcurveto{\pgfqpoint{2.771946in}{3.167454in}}{\pgfqpoint{2.776337in}{3.156855in}}{\pgfqpoint{2.784150in}{3.149041in}}%
\pgfpathcurveto{\pgfqpoint{2.791964in}{3.141228in}}{\pgfqpoint{2.802563in}{3.136837in}}{\pgfqpoint{2.813613in}{3.136837in}}%
\pgfpathclose%
\pgfusepath{stroke,fill}%
\end{pgfscope}%
\begin{pgfscope}%
\pgfpathrectangle{\pgfqpoint{0.648703in}{0.548769in}}{\pgfqpoint{5.112893in}{3.102590in}}%
\pgfusepath{clip}%
\pgfsetbuttcap%
\pgfsetroundjoin%
\definecolor{currentfill}{rgb}{1.000000,0.498039,0.054902}%
\pgfsetfillcolor{currentfill}%
\pgfsetlinewidth{1.003750pt}%
\definecolor{currentstroke}{rgb}{1.000000,0.498039,0.054902}%
\pgfsetstrokecolor{currentstroke}%
\pgfsetdash{}{0pt}%
\pgfpathmoveto{\pgfqpoint{2.614761in}{3.136837in}}%
\pgfpathcurveto{\pgfqpoint{2.625811in}{3.136837in}}{\pgfqpoint{2.636410in}{3.141228in}}{\pgfqpoint{2.644224in}{3.149041in}}%
\pgfpathcurveto{\pgfqpoint{2.652037in}{3.156855in}}{\pgfqpoint{2.656428in}{3.167454in}}{\pgfqpoint{2.656428in}{3.178504in}}%
\pgfpathcurveto{\pgfqpoint{2.656428in}{3.189554in}}{\pgfqpoint{2.652037in}{3.200153in}}{\pgfqpoint{2.644224in}{3.207967in}}%
\pgfpathcurveto{\pgfqpoint{2.636410in}{3.215780in}}{\pgfqpoint{2.625811in}{3.220171in}}{\pgfqpoint{2.614761in}{3.220171in}}%
\pgfpathcurveto{\pgfqpoint{2.603711in}{3.220171in}}{\pgfqpoint{2.593112in}{3.215780in}}{\pgfqpoint{2.585298in}{3.207967in}}%
\pgfpathcurveto{\pgfqpoint{2.577485in}{3.200153in}}{\pgfqpoint{2.573094in}{3.189554in}}{\pgfqpoint{2.573094in}{3.178504in}}%
\pgfpathcurveto{\pgfqpoint{2.573094in}{3.167454in}}{\pgfqpoint{2.577485in}{3.156855in}}{\pgfqpoint{2.585298in}{3.149041in}}%
\pgfpathcurveto{\pgfqpoint{2.593112in}{3.141228in}}{\pgfqpoint{2.603711in}{3.136837in}}{\pgfqpoint{2.614761in}{3.136837in}}%
\pgfpathclose%
\pgfusepath{stroke,fill}%
\end{pgfscope}%
\begin{pgfscope}%
\pgfpathrectangle{\pgfqpoint{0.648703in}{0.548769in}}{\pgfqpoint{5.112893in}{3.102590in}}%
\pgfusepath{clip}%
\pgfsetbuttcap%
\pgfsetroundjoin%
\definecolor{currentfill}{rgb}{1.000000,0.498039,0.054902}%
\pgfsetfillcolor{currentfill}%
\pgfsetlinewidth{1.003750pt}%
\definecolor{currentstroke}{rgb}{1.000000,0.498039,0.054902}%
\pgfsetstrokecolor{currentstroke}%
\pgfsetdash{}{0pt}%
\pgfpathmoveto{\pgfqpoint{2.788515in}{3.136837in}}%
\pgfpathcurveto{\pgfqpoint{2.799565in}{3.136837in}}{\pgfqpoint{2.810164in}{3.141228in}}{\pgfqpoint{2.817978in}{3.149041in}}%
\pgfpathcurveto{\pgfqpoint{2.825792in}{3.156855in}}{\pgfqpoint{2.830182in}{3.167454in}}{\pgfqpoint{2.830182in}{3.178504in}}%
\pgfpathcurveto{\pgfqpoint{2.830182in}{3.189554in}}{\pgfqpoint{2.825792in}{3.200153in}}{\pgfqpoint{2.817978in}{3.207967in}}%
\pgfpathcurveto{\pgfqpoint{2.810164in}{3.215780in}}{\pgfqpoint{2.799565in}{3.220171in}}{\pgfqpoint{2.788515in}{3.220171in}}%
\pgfpathcurveto{\pgfqpoint{2.777465in}{3.220171in}}{\pgfqpoint{2.766866in}{3.215780in}}{\pgfqpoint{2.759052in}{3.207967in}}%
\pgfpathcurveto{\pgfqpoint{2.751239in}{3.200153in}}{\pgfqpoint{2.746848in}{3.189554in}}{\pgfqpoint{2.746848in}{3.178504in}}%
\pgfpathcurveto{\pgfqpoint{2.746848in}{3.167454in}}{\pgfqpoint{2.751239in}{3.156855in}}{\pgfqpoint{2.759052in}{3.149041in}}%
\pgfpathcurveto{\pgfqpoint{2.766866in}{3.141228in}}{\pgfqpoint{2.777465in}{3.136837in}}{\pgfqpoint{2.788515in}{3.136837in}}%
\pgfpathclose%
\pgfusepath{stroke,fill}%
\end{pgfscope}%
\begin{pgfscope}%
\pgfpathrectangle{\pgfqpoint{0.648703in}{0.548769in}}{\pgfqpoint{5.112893in}{3.102590in}}%
\pgfusepath{clip}%
\pgfsetbuttcap%
\pgfsetroundjoin%
\definecolor{currentfill}{rgb}{0.839216,0.152941,0.156863}%
\pgfsetfillcolor{currentfill}%
\pgfsetlinewidth{1.003750pt}%
\definecolor{currentstroke}{rgb}{0.839216,0.152941,0.156863}%
\pgfsetstrokecolor{currentstroke}%
\pgfsetdash{}{0pt}%
\pgfpathmoveto{\pgfqpoint{5.048681in}{3.136837in}}%
\pgfpathcurveto{\pgfqpoint{5.059731in}{3.136837in}}{\pgfqpoint{5.070330in}{3.141228in}}{\pgfqpoint{5.078143in}{3.149041in}}%
\pgfpathcurveto{\pgfqpoint{5.085957in}{3.156855in}}{\pgfqpoint{5.090347in}{3.167454in}}{\pgfqpoint{5.090347in}{3.178504in}}%
\pgfpathcurveto{\pgfqpoint{5.090347in}{3.189554in}}{\pgfqpoint{5.085957in}{3.200153in}}{\pgfqpoint{5.078143in}{3.207967in}}%
\pgfpathcurveto{\pgfqpoint{5.070330in}{3.215780in}}{\pgfqpoint{5.059731in}{3.220171in}}{\pgfqpoint{5.048681in}{3.220171in}}%
\pgfpathcurveto{\pgfqpoint{5.037630in}{3.220171in}}{\pgfqpoint{5.027031in}{3.215780in}}{\pgfqpoint{5.019218in}{3.207967in}}%
\pgfpathcurveto{\pgfqpoint{5.011404in}{3.200153in}}{\pgfqpoint{5.007014in}{3.189554in}}{\pgfqpoint{5.007014in}{3.178504in}}%
\pgfpathcurveto{\pgfqpoint{5.007014in}{3.167454in}}{\pgfqpoint{5.011404in}{3.156855in}}{\pgfqpoint{5.019218in}{3.149041in}}%
\pgfpathcurveto{\pgfqpoint{5.027031in}{3.141228in}}{\pgfqpoint{5.037630in}{3.136837in}}{\pgfqpoint{5.048681in}{3.136837in}}%
\pgfpathclose%
\pgfusepath{stroke,fill}%
\end{pgfscope}%
\begin{pgfscope}%
\pgfpathrectangle{\pgfqpoint{0.648703in}{0.548769in}}{\pgfqpoint{5.112893in}{3.102590in}}%
\pgfusepath{clip}%
\pgfsetbuttcap%
\pgfsetroundjoin%
\definecolor{currentfill}{rgb}{1.000000,0.498039,0.054902}%
\pgfsetfillcolor{currentfill}%
\pgfsetlinewidth{1.003750pt}%
\definecolor{currentstroke}{rgb}{1.000000,0.498039,0.054902}%
\pgfsetstrokecolor{currentstroke}%
\pgfsetdash{}{0pt}%
\pgfpathmoveto{\pgfqpoint{4.845856in}{3.149281in}}%
\pgfpathcurveto{\pgfqpoint{4.856906in}{3.149281in}}{\pgfqpoint{4.867505in}{3.153671in}}{\pgfqpoint{4.875319in}{3.161485in}}%
\pgfpathcurveto{\pgfqpoint{4.883132in}{3.169298in}}{\pgfqpoint{4.887523in}{3.179897in}}{\pgfqpoint{4.887523in}{3.190948in}}%
\pgfpathcurveto{\pgfqpoint{4.887523in}{3.201998in}}{\pgfqpoint{4.883132in}{3.212597in}}{\pgfqpoint{4.875319in}{3.220410in}}%
\pgfpathcurveto{\pgfqpoint{4.867505in}{3.228224in}}{\pgfqpoint{4.856906in}{3.232614in}}{\pgfqpoint{4.845856in}{3.232614in}}%
\pgfpathcurveto{\pgfqpoint{4.834806in}{3.232614in}}{\pgfqpoint{4.824207in}{3.228224in}}{\pgfqpoint{4.816393in}{3.220410in}}%
\pgfpathcurveto{\pgfqpoint{4.808579in}{3.212597in}}{\pgfqpoint{4.804189in}{3.201998in}}{\pgfqpoint{4.804189in}{3.190948in}}%
\pgfpathcurveto{\pgfqpoint{4.804189in}{3.179897in}}{\pgfqpoint{4.808579in}{3.169298in}}{\pgfqpoint{4.816393in}{3.161485in}}%
\pgfpathcurveto{\pgfqpoint{4.824207in}{3.153671in}}{\pgfqpoint{4.834806in}{3.149281in}}{\pgfqpoint{4.845856in}{3.149281in}}%
\pgfpathclose%
\pgfusepath{stroke,fill}%
\end{pgfscope}%
\begin{pgfscope}%
\pgfpathrectangle{\pgfqpoint{0.648703in}{0.548769in}}{\pgfqpoint{5.112893in}{3.102590in}}%
\pgfusepath{clip}%
\pgfsetbuttcap%
\pgfsetroundjoin%
\definecolor{currentfill}{rgb}{1.000000,0.498039,0.054902}%
\pgfsetfillcolor{currentfill}%
\pgfsetlinewidth{1.003750pt}%
\definecolor{currentstroke}{rgb}{1.000000,0.498039,0.054902}%
\pgfsetstrokecolor{currentstroke}%
\pgfsetdash{}{0pt}%
\pgfpathmoveto{\pgfqpoint{4.420059in}{3.286160in}}%
\pgfpathcurveto{\pgfqpoint{4.431109in}{3.286160in}}{\pgfqpoint{4.441708in}{3.290550in}}{\pgfqpoint{4.449522in}{3.298364in}}%
\pgfpathcurveto{\pgfqpoint{4.457336in}{3.306177in}}{\pgfqpoint{4.461726in}{3.316776in}}{\pgfqpoint{4.461726in}{3.327827in}}%
\pgfpathcurveto{\pgfqpoint{4.461726in}{3.338877in}}{\pgfqpoint{4.457336in}{3.349476in}}{\pgfqpoint{4.449522in}{3.357289in}}%
\pgfpathcurveto{\pgfqpoint{4.441708in}{3.365103in}}{\pgfqpoint{4.431109in}{3.369493in}}{\pgfqpoint{4.420059in}{3.369493in}}%
\pgfpathcurveto{\pgfqpoint{4.409009in}{3.369493in}}{\pgfqpoint{4.398410in}{3.365103in}}{\pgfqpoint{4.390596in}{3.357289in}}%
\pgfpathcurveto{\pgfqpoint{4.382783in}{3.349476in}}{\pgfqpoint{4.378392in}{3.338877in}}{\pgfqpoint{4.378392in}{3.327827in}}%
\pgfpathcurveto{\pgfqpoint{4.378392in}{3.316776in}}{\pgfqpoint{4.382783in}{3.306177in}}{\pgfqpoint{4.390596in}{3.298364in}}%
\pgfpathcurveto{\pgfqpoint{4.398410in}{3.290550in}}{\pgfqpoint{4.409009in}{3.286160in}}{\pgfqpoint{4.420059in}{3.286160in}}%
\pgfpathclose%
\pgfusepath{stroke,fill}%
\end{pgfscope}%
\begin{pgfscope}%
\pgfpathrectangle{\pgfqpoint{0.648703in}{0.548769in}}{\pgfqpoint{5.112893in}{3.102590in}}%
\pgfusepath{clip}%
\pgfsetbuttcap%
\pgfsetroundjoin%
\definecolor{currentfill}{rgb}{1.000000,0.498039,0.054902}%
\pgfsetfillcolor{currentfill}%
\pgfsetlinewidth{1.003750pt}%
\definecolor{currentstroke}{rgb}{1.000000,0.498039,0.054902}%
\pgfsetstrokecolor{currentstroke}%
\pgfsetdash{}{0pt}%
\pgfpathmoveto{\pgfqpoint{3.056811in}{3.136837in}}%
\pgfpathcurveto{\pgfqpoint{3.067861in}{3.136837in}}{\pgfqpoint{3.078460in}{3.141228in}}{\pgfqpoint{3.086274in}{3.149041in}}%
\pgfpathcurveto{\pgfqpoint{3.094087in}{3.156855in}}{\pgfqpoint{3.098478in}{3.167454in}}{\pgfqpoint{3.098478in}{3.178504in}}%
\pgfpathcurveto{\pgfqpoint{3.098478in}{3.189554in}}{\pgfqpoint{3.094087in}{3.200153in}}{\pgfqpoint{3.086274in}{3.207967in}}%
\pgfpathcurveto{\pgfqpoint{3.078460in}{3.215780in}}{\pgfqpoint{3.067861in}{3.220171in}}{\pgfqpoint{3.056811in}{3.220171in}}%
\pgfpathcurveto{\pgfqpoint{3.045761in}{3.220171in}}{\pgfqpoint{3.035162in}{3.215780in}}{\pgfqpoint{3.027348in}{3.207967in}}%
\pgfpathcurveto{\pgfqpoint{3.019535in}{3.200153in}}{\pgfqpoint{3.015144in}{3.189554in}}{\pgfqpoint{3.015144in}{3.178504in}}%
\pgfpathcurveto{\pgfqpoint{3.015144in}{3.167454in}}{\pgfqpoint{3.019535in}{3.156855in}}{\pgfqpoint{3.027348in}{3.149041in}}%
\pgfpathcurveto{\pgfqpoint{3.035162in}{3.141228in}}{\pgfqpoint{3.045761in}{3.136837in}}{\pgfqpoint{3.056811in}{3.136837in}}%
\pgfpathclose%
\pgfusepath{stroke,fill}%
\end{pgfscope}%
\begin{pgfscope}%
\pgfpathrectangle{\pgfqpoint{0.648703in}{0.548769in}}{\pgfqpoint{5.112893in}{3.102590in}}%
\pgfusepath{clip}%
\pgfsetbuttcap%
\pgfsetroundjoin%
\definecolor{currentfill}{rgb}{1.000000,0.498039,0.054902}%
\pgfsetfillcolor{currentfill}%
\pgfsetlinewidth{1.003750pt}%
\definecolor{currentstroke}{rgb}{1.000000,0.498039,0.054902}%
\pgfsetstrokecolor{currentstroke}%
\pgfsetdash{}{0pt}%
\pgfpathmoveto{\pgfqpoint{3.635851in}{3.149281in}}%
\pgfpathcurveto{\pgfqpoint{3.646901in}{3.149281in}}{\pgfqpoint{3.657500in}{3.153671in}}{\pgfqpoint{3.665313in}{3.161485in}}%
\pgfpathcurveto{\pgfqpoint{3.673127in}{3.169298in}}{\pgfqpoint{3.677517in}{3.179897in}}{\pgfqpoint{3.677517in}{3.190948in}}%
\pgfpathcurveto{\pgfqpoint{3.677517in}{3.201998in}}{\pgfqpoint{3.673127in}{3.212597in}}{\pgfqpoint{3.665313in}{3.220410in}}%
\pgfpathcurveto{\pgfqpoint{3.657500in}{3.228224in}}{\pgfqpoint{3.646901in}{3.232614in}}{\pgfqpoint{3.635851in}{3.232614in}}%
\pgfpathcurveto{\pgfqpoint{3.624801in}{3.232614in}}{\pgfqpoint{3.614202in}{3.228224in}}{\pgfqpoint{3.606388in}{3.220410in}}%
\pgfpathcurveto{\pgfqpoint{3.598574in}{3.212597in}}{\pgfqpoint{3.594184in}{3.201998in}}{\pgfqpoint{3.594184in}{3.190948in}}%
\pgfpathcurveto{\pgfqpoint{3.594184in}{3.179897in}}{\pgfqpoint{3.598574in}{3.169298in}}{\pgfqpoint{3.606388in}{3.161485in}}%
\pgfpathcurveto{\pgfqpoint{3.614202in}{3.153671in}}{\pgfqpoint{3.624801in}{3.149281in}}{\pgfqpoint{3.635851in}{3.149281in}}%
\pgfpathclose%
\pgfusepath{stroke,fill}%
\end{pgfscope}%
\begin{pgfscope}%
\pgfpathrectangle{\pgfqpoint{0.648703in}{0.548769in}}{\pgfqpoint{5.112893in}{3.102590in}}%
\pgfusepath{clip}%
\pgfsetbuttcap%
\pgfsetroundjoin%
\definecolor{currentfill}{rgb}{0.121569,0.466667,0.705882}%
\pgfsetfillcolor{currentfill}%
\pgfsetlinewidth{1.003750pt}%
\definecolor{currentstroke}{rgb}{0.121569,0.466667,0.705882}%
\pgfsetstrokecolor{currentstroke}%
\pgfsetdash{}{0pt}%
\pgfpathmoveto{\pgfqpoint{3.685451in}{3.128542in}}%
\pgfpathcurveto{\pgfqpoint{3.696501in}{3.128542in}}{\pgfqpoint{3.707100in}{3.132932in}}{\pgfqpoint{3.714913in}{3.140746in}}%
\pgfpathcurveto{\pgfqpoint{3.722727in}{3.148559in}}{\pgfqpoint{3.727117in}{3.159158in}}{\pgfqpoint{3.727117in}{3.170208in}}%
\pgfpathcurveto{\pgfqpoint{3.727117in}{3.181258in}}{\pgfqpoint{3.722727in}{3.191857in}}{\pgfqpoint{3.714913in}{3.199671in}}%
\pgfpathcurveto{\pgfqpoint{3.707100in}{3.207485in}}{\pgfqpoint{3.696501in}{3.211875in}}{\pgfqpoint{3.685451in}{3.211875in}}%
\pgfpathcurveto{\pgfqpoint{3.674400in}{3.211875in}}{\pgfqpoint{3.663801in}{3.207485in}}{\pgfqpoint{3.655988in}{3.199671in}}%
\pgfpathcurveto{\pgfqpoint{3.648174in}{3.191857in}}{\pgfqpoint{3.643784in}{3.181258in}}{\pgfqpoint{3.643784in}{3.170208in}}%
\pgfpathcurveto{\pgfqpoint{3.643784in}{3.159158in}}{\pgfqpoint{3.648174in}{3.148559in}}{\pgfqpoint{3.655988in}{3.140746in}}%
\pgfpathcurveto{\pgfqpoint{3.663801in}{3.132932in}}{\pgfqpoint{3.674400in}{3.128542in}}{\pgfqpoint{3.685451in}{3.128542in}}%
\pgfpathclose%
\pgfusepath{stroke,fill}%
\end{pgfscope}%
\begin{pgfscope}%
\pgfpathrectangle{\pgfqpoint{0.648703in}{0.548769in}}{\pgfqpoint{5.112893in}{3.102590in}}%
\pgfusepath{clip}%
\pgfsetbuttcap%
\pgfsetroundjoin%
\definecolor{currentfill}{rgb}{1.000000,0.498039,0.054902}%
\pgfsetfillcolor{currentfill}%
\pgfsetlinewidth{1.003750pt}%
\definecolor{currentstroke}{rgb}{1.000000,0.498039,0.054902}%
\pgfsetstrokecolor{currentstroke}%
\pgfsetdash{}{0pt}%
\pgfpathmoveto{\pgfqpoint{4.201424in}{3.232238in}}%
\pgfpathcurveto{\pgfqpoint{4.212474in}{3.232238in}}{\pgfqpoint{4.223073in}{3.236628in}}{\pgfqpoint{4.230887in}{3.244442in}}%
\pgfpathcurveto{\pgfqpoint{4.238700in}{3.252255in}}{\pgfqpoint{4.243091in}{3.262854in}}{\pgfqpoint{4.243091in}{3.273905in}}%
\pgfpathcurveto{\pgfqpoint{4.243091in}{3.284955in}}{\pgfqpoint{4.238700in}{3.295554in}}{\pgfqpoint{4.230887in}{3.303367in}}%
\pgfpathcurveto{\pgfqpoint{4.223073in}{3.311181in}}{\pgfqpoint{4.212474in}{3.315571in}}{\pgfqpoint{4.201424in}{3.315571in}}%
\pgfpathcurveto{\pgfqpoint{4.190374in}{3.315571in}}{\pgfqpoint{4.179775in}{3.311181in}}{\pgfqpoint{4.171961in}{3.303367in}}%
\pgfpathcurveto{\pgfqpoint{4.164148in}{3.295554in}}{\pgfqpoint{4.159757in}{3.284955in}}{\pgfqpoint{4.159757in}{3.273905in}}%
\pgfpathcurveto{\pgfqpoint{4.159757in}{3.262854in}}{\pgfqpoint{4.164148in}{3.252255in}}{\pgfqpoint{4.171961in}{3.244442in}}%
\pgfpathcurveto{\pgfqpoint{4.179775in}{3.236628in}}{\pgfqpoint{4.190374in}{3.232238in}}{\pgfqpoint{4.201424in}{3.232238in}}%
\pgfpathclose%
\pgfusepath{stroke,fill}%
\end{pgfscope}%
\begin{pgfscope}%
\pgfpathrectangle{\pgfqpoint{0.648703in}{0.548769in}}{\pgfqpoint{5.112893in}{3.102590in}}%
\pgfusepath{clip}%
\pgfsetbuttcap%
\pgfsetroundjoin%
\definecolor{currentfill}{rgb}{0.121569,0.466667,0.705882}%
\pgfsetfillcolor{currentfill}%
\pgfsetlinewidth{1.003750pt}%
\definecolor{currentstroke}{rgb}{0.121569,0.466667,0.705882}%
\pgfsetstrokecolor{currentstroke}%
\pgfsetdash{}{0pt}%
\pgfpathmoveto{\pgfqpoint{0.873115in}{0.648129in}}%
\pgfpathcurveto{\pgfqpoint{0.884166in}{0.648129in}}{\pgfqpoint{0.894765in}{0.652519in}}{\pgfqpoint{0.902578in}{0.660333in}}%
\pgfpathcurveto{\pgfqpoint{0.910392in}{0.668146in}}{\pgfqpoint{0.914782in}{0.678745in}}{\pgfqpoint{0.914782in}{0.689796in}}%
\pgfpathcurveto{\pgfqpoint{0.914782in}{0.700846in}}{\pgfqpoint{0.910392in}{0.711445in}}{\pgfqpoint{0.902578in}{0.719258in}}%
\pgfpathcurveto{\pgfqpoint{0.894765in}{0.727072in}}{\pgfqpoint{0.884166in}{0.731462in}}{\pgfqpoint{0.873115in}{0.731462in}}%
\pgfpathcurveto{\pgfqpoint{0.862065in}{0.731462in}}{\pgfqpoint{0.851466in}{0.727072in}}{\pgfqpoint{0.843653in}{0.719258in}}%
\pgfpathcurveto{\pgfqpoint{0.835839in}{0.711445in}}{\pgfqpoint{0.831449in}{0.700846in}}{\pgfqpoint{0.831449in}{0.689796in}}%
\pgfpathcurveto{\pgfqpoint{0.831449in}{0.678745in}}{\pgfqpoint{0.835839in}{0.668146in}}{\pgfqpoint{0.843653in}{0.660333in}}%
\pgfpathcurveto{\pgfqpoint{0.851466in}{0.652519in}}{\pgfqpoint{0.862065in}{0.648129in}}{\pgfqpoint{0.873115in}{0.648129in}}%
\pgfpathclose%
\pgfusepath{stroke,fill}%
\end{pgfscope}%
\begin{pgfscope}%
\pgfpathrectangle{\pgfqpoint{0.648703in}{0.548769in}}{\pgfqpoint{5.112893in}{3.102590in}}%
\pgfusepath{clip}%
\pgfsetbuttcap%
\pgfsetroundjoin%
\definecolor{currentfill}{rgb}{1.000000,0.498039,0.054902}%
\pgfsetfillcolor{currentfill}%
\pgfsetlinewidth{1.003750pt}%
\definecolor{currentstroke}{rgb}{1.000000,0.498039,0.054902}%
\pgfsetstrokecolor{currentstroke}%
\pgfsetdash{}{0pt}%
\pgfpathmoveto{\pgfqpoint{2.155893in}{3.136837in}}%
\pgfpathcurveto{\pgfqpoint{2.166943in}{3.136837in}}{\pgfqpoint{2.177542in}{3.141228in}}{\pgfqpoint{2.185355in}{3.149041in}}%
\pgfpathcurveto{\pgfqpoint{2.193169in}{3.156855in}}{\pgfqpoint{2.197559in}{3.167454in}}{\pgfqpoint{2.197559in}{3.178504in}}%
\pgfpathcurveto{\pgfqpoint{2.197559in}{3.189554in}}{\pgfqpoint{2.193169in}{3.200153in}}{\pgfqpoint{2.185355in}{3.207967in}}%
\pgfpathcurveto{\pgfqpoint{2.177542in}{3.215780in}}{\pgfqpoint{2.166943in}{3.220171in}}{\pgfqpoint{2.155893in}{3.220171in}}%
\pgfpathcurveto{\pgfqpoint{2.144843in}{3.220171in}}{\pgfqpoint{2.134244in}{3.215780in}}{\pgfqpoint{2.126430in}{3.207967in}}%
\pgfpathcurveto{\pgfqpoint{2.118616in}{3.200153in}}{\pgfqpoint{2.114226in}{3.189554in}}{\pgfqpoint{2.114226in}{3.178504in}}%
\pgfpathcurveto{\pgfqpoint{2.114226in}{3.167454in}}{\pgfqpoint{2.118616in}{3.156855in}}{\pgfqpoint{2.126430in}{3.149041in}}%
\pgfpathcurveto{\pgfqpoint{2.134244in}{3.141228in}}{\pgfqpoint{2.144843in}{3.136837in}}{\pgfqpoint{2.155893in}{3.136837in}}%
\pgfpathclose%
\pgfusepath{stroke,fill}%
\end{pgfscope}%
\begin{pgfscope}%
\pgfpathrectangle{\pgfqpoint{0.648703in}{0.548769in}}{\pgfqpoint{5.112893in}{3.102590in}}%
\pgfusepath{clip}%
\pgfsetbuttcap%
\pgfsetroundjoin%
\definecolor{currentfill}{rgb}{1.000000,0.498039,0.054902}%
\pgfsetfillcolor{currentfill}%
\pgfsetlinewidth{1.003750pt}%
\definecolor{currentstroke}{rgb}{1.000000,0.498039,0.054902}%
\pgfsetstrokecolor{currentstroke}%
\pgfsetdash{}{0pt}%
\pgfpathmoveto{\pgfqpoint{4.407537in}{3.140985in}}%
\pgfpathcurveto{\pgfqpoint{4.418587in}{3.140985in}}{\pgfqpoint{4.429186in}{3.145375in}}{\pgfqpoint{4.436999in}{3.153189in}}%
\pgfpathcurveto{\pgfqpoint{4.444813in}{3.161003in}}{\pgfqpoint{4.449203in}{3.171602in}}{\pgfqpoint{4.449203in}{3.182652in}}%
\pgfpathcurveto{\pgfqpoint{4.449203in}{3.193702in}}{\pgfqpoint{4.444813in}{3.204301in}}{\pgfqpoint{4.436999in}{3.212115in}}%
\pgfpathcurveto{\pgfqpoint{4.429186in}{3.219928in}}{\pgfqpoint{4.418587in}{3.224319in}}{\pgfqpoint{4.407537in}{3.224319in}}%
\pgfpathcurveto{\pgfqpoint{4.396486in}{3.224319in}}{\pgfqpoint{4.385887in}{3.219928in}}{\pgfqpoint{4.378074in}{3.212115in}}%
\pgfpathcurveto{\pgfqpoint{4.370260in}{3.204301in}}{\pgfqpoint{4.365870in}{3.193702in}}{\pgfqpoint{4.365870in}{3.182652in}}%
\pgfpathcurveto{\pgfqpoint{4.365870in}{3.171602in}}{\pgfqpoint{4.370260in}{3.161003in}}{\pgfqpoint{4.378074in}{3.153189in}}%
\pgfpathcurveto{\pgfqpoint{4.385887in}{3.145375in}}{\pgfqpoint{4.396486in}{3.140985in}}{\pgfqpoint{4.407537in}{3.140985in}}%
\pgfpathclose%
\pgfusepath{stroke,fill}%
\end{pgfscope}%
\begin{pgfscope}%
\pgfpathrectangle{\pgfqpoint{0.648703in}{0.548769in}}{\pgfqpoint{5.112893in}{3.102590in}}%
\pgfusepath{clip}%
\pgfsetbuttcap%
\pgfsetroundjoin%
\definecolor{currentfill}{rgb}{1.000000,0.498039,0.054902}%
\pgfsetfillcolor{currentfill}%
\pgfsetlinewidth{1.003750pt}%
\definecolor{currentstroke}{rgb}{1.000000,0.498039,0.054902}%
\pgfsetstrokecolor{currentstroke}%
\pgfsetdash{}{0pt}%
\pgfpathmoveto{\pgfqpoint{3.849314in}{3.149281in}}%
\pgfpathcurveto{\pgfqpoint{3.860364in}{3.149281in}}{\pgfqpoint{3.870963in}{3.153671in}}{\pgfqpoint{3.878776in}{3.161485in}}%
\pgfpathcurveto{\pgfqpoint{3.886590in}{3.169298in}}{\pgfqpoint{3.890980in}{3.179897in}}{\pgfqpoint{3.890980in}{3.190948in}}%
\pgfpathcurveto{\pgfqpoint{3.890980in}{3.201998in}}{\pgfqpoint{3.886590in}{3.212597in}}{\pgfqpoint{3.878776in}{3.220410in}}%
\pgfpathcurveto{\pgfqpoint{3.870963in}{3.228224in}}{\pgfqpoint{3.860364in}{3.232614in}}{\pgfqpoint{3.849314in}{3.232614in}}%
\pgfpathcurveto{\pgfqpoint{3.838263in}{3.232614in}}{\pgfqpoint{3.827664in}{3.228224in}}{\pgfqpoint{3.819851in}{3.220410in}}%
\pgfpathcurveto{\pgfqpoint{3.812037in}{3.212597in}}{\pgfqpoint{3.807647in}{3.201998in}}{\pgfqpoint{3.807647in}{3.190948in}}%
\pgfpathcurveto{\pgfqpoint{3.807647in}{3.179897in}}{\pgfqpoint{3.812037in}{3.169298in}}{\pgfqpoint{3.819851in}{3.161485in}}%
\pgfpathcurveto{\pgfqpoint{3.827664in}{3.153671in}}{\pgfqpoint{3.838263in}{3.149281in}}{\pgfqpoint{3.849314in}{3.149281in}}%
\pgfpathclose%
\pgfusepath{stroke,fill}%
\end{pgfscope}%
\begin{pgfscope}%
\pgfpathrectangle{\pgfqpoint{0.648703in}{0.548769in}}{\pgfqpoint{5.112893in}{3.102590in}}%
\pgfusepath{clip}%
\pgfsetbuttcap%
\pgfsetroundjoin%
\definecolor{currentfill}{rgb}{1.000000,0.498039,0.054902}%
\pgfsetfillcolor{currentfill}%
\pgfsetlinewidth{1.003750pt}%
\definecolor{currentstroke}{rgb}{1.000000,0.498039,0.054902}%
\pgfsetstrokecolor{currentstroke}%
\pgfsetdash{}{0pt}%
\pgfpathmoveto{\pgfqpoint{1.996441in}{3.140985in}}%
\pgfpathcurveto{\pgfqpoint{2.007492in}{3.140985in}}{\pgfqpoint{2.018091in}{3.145375in}}{\pgfqpoint{2.025904in}{3.153189in}}%
\pgfpathcurveto{\pgfqpoint{2.033718in}{3.161003in}}{\pgfqpoint{2.038108in}{3.171602in}}{\pgfqpoint{2.038108in}{3.182652in}}%
\pgfpathcurveto{\pgfqpoint{2.038108in}{3.193702in}}{\pgfqpoint{2.033718in}{3.204301in}}{\pgfqpoint{2.025904in}{3.212115in}}%
\pgfpathcurveto{\pgfqpoint{2.018091in}{3.219928in}}{\pgfqpoint{2.007492in}{3.224319in}}{\pgfqpoint{1.996441in}{3.224319in}}%
\pgfpathcurveto{\pgfqpoint{1.985391in}{3.224319in}}{\pgfqpoint{1.974792in}{3.219928in}}{\pgfqpoint{1.966979in}{3.212115in}}%
\pgfpathcurveto{\pgfqpoint{1.959165in}{3.204301in}}{\pgfqpoint{1.954775in}{3.193702in}}{\pgfqpoint{1.954775in}{3.182652in}}%
\pgfpathcurveto{\pgfqpoint{1.954775in}{3.171602in}}{\pgfqpoint{1.959165in}{3.161003in}}{\pgfqpoint{1.966979in}{3.153189in}}%
\pgfpathcurveto{\pgfqpoint{1.974792in}{3.145375in}}{\pgfqpoint{1.985391in}{3.140985in}}{\pgfqpoint{1.996441in}{3.140985in}}%
\pgfpathclose%
\pgfusepath{stroke,fill}%
\end{pgfscope}%
\begin{pgfscope}%
\pgfpathrectangle{\pgfqpoint{0.648703in}{0.548769in}}{\pgfqpoint{5.112893in}{3.102590in}}%
\pgfusepath{clip}%
\pgfsetbuttcap%
\pgfsetroundjoin%
\definecolor{currentfill}{rgb}{1.000000,0.498039,0.054902}%
\pgfsetfillcolor{currentfill}%
\pgfsetlinewidth{1.003750pt}%
\definecolor{currentstroke}{rgb}{1.000000,0.498039,0.054902}%
\pgfsetstrokecolor{currentstroke}%
\pgfsetdash{}{0pt}%
\pgfpathmoveto{\pgfqpoint{2.217712in}{3.136837in}}%
\pgfpathcurveto{\pgfqpoint{2.228762in}{3.136837in}}{\pgfqpoint{2.239361in}{3.141228in}}{\pgfqpoint{2.247175in}{3.149041in}}%
\pgfpathcurveto{\pgfqpoint{2.254988in}{3.156855in}}{\pgfqpoint{2.259379in}{3.167454in}}{\pgfqpoint{2.259379in}{3.178504in}}%
\pgfpathcurveto{\pgfqpoint{2.259379in}{3.189554in}}{\pgfqpoint{2.254988in}{3.200153in}}{\pgfqpoint{2.247175in}{3.207967in}}%
\pgfpathcurveto{\pgfqpoint{2.239361in}{3.215780in}}{\pgfqpoint{2.228762in}{3.220171in}}{\pgfqpoint{2.217712in}{3.220171in}}%
\pgfpathcurveto{\pgfqpoint{2.206662in}{3.220171in}}{\pgfqpoint{2.196063in}{3.215780in}}{\pgfqpoint{2.188249in}{3.207967in}}%
\pgfpathcurveto{\pgfqpoint{2.180436in}{3.200153in}}{\pgfqpoint{2.176045in}{3.189554in}}{\pgfqpoint{2.176045in}{3.178504in}}%
\pgfpathcurveto{\pgfqpoint{2.176045in}{3.167454in}}{\pgfqpoint{2.180436in}{3.156855in}}{\pgfqpoint{2.188249in}{3.149041in}}%
\pgfpathcurveto{\pgfqpoint{2.196063in}{3.141228in}}{\pgfqpoint{2.206662in}{3.136837in}}{\pgfqpoint{2.217712in}{3.136837in}}%
\pgfpathclose%
\pgfusepath{stroke,fill}%
\end{pgfscope}%
\begin{pgfscope}%
\pgfpathrectangle{\pgfqpoint{0.648703in}{0.548769in}}{\pgfqpoint{5.112893in}{3.102590in}}%
\pgfusepath{clip}%
\pgfsetbuttcap%
\pgfsetroundjoin%
\definecolor{currentfill}{rgb}{1.000000,0.498039,0.054902}%
\pgfsetfillcolor{currentfill}%
\pgfsetlinewidth{1.003750pt}%
\definecolor{currentstroke}{rgb}{1.000000,0.498039,0.054902}%
\pgfsetstrokecolor{currentstroke}%
\pgfsetdash{}{0pt}%
\pgfpathmoveto{\pgfqpoint{2.648744in}{3.145133in}}%
\pgfpathcurveto{\pgfqpoint{2.659794in}{3.145133in}}{\pgfqpoint{2.670393in}{3.149523in}}{\pgfqpoint{2.678207in}{3.157337in}}%
\pgfpathcurveto{\pgfqpoint{2.686020in}{3.165151in}}{\pgfqpoint{2.690411in}{3.175750in}}{\pgfqpoint{2.690411in}{3.186800in}}%
\pgfpathcurveto{\pgfqpoint{2.690411in}{3.197850in}}{\pgfqpoint{2.686020in}{3.208449in}}{\pgfqpoint{2.678207in}{3.216262in}}%
\pgfpathcurveto{\pgfqpoint{2.670393in}{3.224076in}}{\pgfqpoint{2.659794in}{3.228466in}}{\pgfqpoint{2.648744in}{3.228466in}}%
\pgfpathcurveto{\pgfqpoint{2.637694in}{3.228466in}}{\pgfqpoint{2.627095in}{3.224076in}}{\pgfqpoint{2.619281in}{3.216262in}}%
\pgfpathcurveto{\pgfqpoint{2.611468in}{3.208449in}}{\pgfqpoint{2.607077in}{3.197850in}}{\pgfqpoint{2.607077in}{3.186800in}}%
\pgfpathcurveto{\pgfqpoint{2.607077in}{3.175750in}}{\pgfqpoint{2.611468in}{3.165151in}}{\pgfqpoint{2.619281in}{3.157337in}}%
\pgfpathcurveto{\pgfqpoint{2.627095in}{3.149523in}}{\pgfqpoint{2.637694in}{3.145133in}}{\pgfqpoint{2.648744in}{3.145133in}}%
\pgfpathclose%
\pgfusepath{stroke,fill}%
\end{pgfscope}%
\begin{pgfscope}%
\pgfpathrectangle{\pgfqpoint{0.648703in}{0.548769in}}{\pgfqpoint{5.112893in}{3.102590in}}%
\pgfusepath{clip}%
\pgfsetbuttcap%
\pgfsetroundjoin%
\definecolor{currentfill}{rgb}{0.121569,0.466667,0.705882}%
\pgfsetfillcolor{currentfill}%
\pgfsetlinewidth{1.003750pt}%
\definecolor{currentstroke}{rgb}{0.121569,0.466667,0.705882}%
\pgfsetstrokecolor{currentstroke}%
\pgfsetdash{}{0pt}%
\pgfpathmoveto{\pgfqpoint{3.979402in}{3.132690in}}%
\pgfpathcurveto{\pgfqpoint{3.990452in}{3.132690in}}{\pgfqpoint{4.001051in}{3.137080in}}{\pgfqpoint{4.008864in}{3.144893in}}%
\pgfpathcurveto{\pgfqpoint{4.016678in}{3.152707in}}{\pgfqpoint{4.021068in}{3.163306in}}{\pgfqpoint{4.021068in}{3.174356in}}%
\pgfpathcurveto{\pgfqpoint{4.021068in}{3.185406in}}{\pgfqpoint{4.016678in}{3.196005in}}{\pgfqpoint{4.008864in}{3.203819in}}%
\pgfpathcurveto{\pgfqpoint{4.001051in}{3.211633in}}{\pgfqpoint{3.990452in}{3.216023in}}{\pgfqpoint{3.979402in}{3.216023in}}%
\pgfpathcurveto{\pgfqpoint{3.968351in}{3.216023in}}{\pgfqpoint{3.957752in}{3.211633in}}{\pgfqpoint{3.949939in}{3.203819in}}%
\pgfpathcurveto{\pgfqpoint{3.942125in}{3.196005in}}{\pgfqpoint{3.937735in}{3.185406in}}{\pgfqpoint{3.937735in}{3.174356in}}%
\pgfpathcurveto{\pgfqpoint{3.937735in}{3.163306in}}{\pgfqpoint{3.942125in}{3.152707in}}{\pgfqpoint{3.949939in}{3.144893in}}%
\pgfpathcurveto{\pgfqpoint{3.957752in}{3.137080in}}{\pgfqpoint{3.968351in}{3.132690in}}{\pgfqpoint{3.979402in}{3.132690in}}%
\pgfpathclose%
\pgfusepath{stroke,fill}%
\end{pgfscope}%
\begin{pgfscope}%
\pgfpathrectangle{\pgfqpoint{0.648703in}{0.548769in}}{\pgfqpoint{5.112893in}{3.102590in}}%
\pgfusepath{clip}%
\pgfsetbuttcap%
\pgfsetroundjoin%
\definecolor{currentfill}{rgb}{1.000000,0.498039,0.054902}%
\pgfsetfillcolor{currentfill}%
\pgfsetlinewidth{1.003750pt}%
\definecolor{currentstroke}{rgb}{1.000000,0.498039,0.054902}%
\pgfsetstrokecolor{currentstroke}%
\pgfsetdash{}{0pt}%
\pgfpathmoveto{\pgfqpoint{4.361247in}{3.348378in}}%
\pgfpathcurveto{\pgfqpoint{4.372298in}{3.348378in}}{\pgfqpoint{4.382897in}{3.352768in}}{\pgfqpoint{4.390710in}{3.360581in}}%
\pgfpathcurveto{\pgfqpoint{4.398524in}{3.368395in}}{\pgfqpoint{4.402914in}{3.378994in}}{\pgfqpoint{4.402914in}{3.390044in}}%
\pgfpathcurveto{\pgfqpoint{4.402914in}{3.401094in}}{\pgfqpoint{4.398524in}{3.411693in}}{\pgfqpoint{4.390710in}{3.419507in}}%
\pgfpathcurveto{\pgfqpoint{4.382897in}{3.427321in}}{\pgfqpoint{4.372298in}{3.431711in}}{\pgfqpoint{4.361247in}{3.431711in}}%
\pgfpathcurveto{\pgfqpoint{4.350197in}{3.431711in}}{\pgfqpoint{4.339598in}{3.427321in}}{\pgfqpoint{4.331785in}{3.419507in}}%
\pgfpathcurveto{\pgfqpoint{4.323971in}{3.411693in}}{\pgfqpoint{4.319581in}{3.401094in}}{\pgfqpoint{4.319581in}{3.390044in}}%
\pgfpathcurveto{\pgfqpoint{4.319581in}{3.378994in}}{\pgfqpoint{4.323971in}{3.368395in}}{\pgfqpoint{4.331785in}{3.360581in}}%
\pgfpathcurveto{\pgfqpoint{4.339598in}{3.352768in}}{\pgfqpoint{4.350197in}{3.348378in}}{\pgfqpoint{4.361247in}{3.348378in}}%
\pgfpathclose%
\pgfusepath{stroke,fill}%
\end{pgfscope}%
\begin{pgfscope}%
\pgfsetbuttcap%
\pgfsetroundjoin%
\definecolor{currentfill}{rgb}{0.000000,0.000000,0.000000}%
\pgfsetfillcolor{currentfill}%
\pgfsetlinewidth{0.803000pt}%
\definecolor{currentstroke}{rgb}{0.000000,0.000000,0.000000}%
\pgfsetstrokecolor{currentstroke}%
\pgfsetdash{}{0pt}%
\pgfsys@defobject{currentmarker}{\pgfqpoint{0.000000in}{-0.048611in}}{\pgfqpoint{0.000000in}{0.000000in}}{%
\pgfpathmoveto{\pgfqpoint{0.000000in}{0.000000in}}%
\pgfpathlineto{\pgfqpoint{0.000000in}{-0.048611in}}%
\pgfusepath{stroke,fill}%
}%
\begin{pgfscope}%
\pgfsys@transformshift{0.873110in}{0.548769in}%
\pgfsys@useobject{currentmarker}{}%
\end{pgfscope}%
\end{pgfscope}%
\begin{pgfscope}%
\definecolor{textcolor}{rgb}{0.000000,0.000000,0.000000}%
\pgfsetstrokecolor{textcolor}%
\pgfsetfillcolor{textcolor}%
\pgftext[x=0.873110in,y=0.451547in,,top]{\color{textcolor}\sffamily\fontsize{10.000000}{12.000000}\selectfont \(\displaystyle {0.0}\)}%
\end{pgfscope}%
\begin{pgfscope}%
\pgfsetbuttcap%
\pgfsetroundjoin%
\definecolor{currentfill}{rgb}{0.000000,0.000000,0.000000}%
\pgfsetfillcolor{currentfill}%
\pgfsetlinewidth{0.803000pt}%
\definecolor{currentstroke}{rgb}{0.000000,0.000000,0.000000}%
\pgfsetstrokecolor{currentstroke}%
\pgfsetdash{}{0pt}%
\pgfsys@defobject{currentmarker}{\pgfqpoint{0.000000in}{-0.048611in}}{\pgfqpoint{0.000000in}{0.000000in}}{%
\pgfpathmoveto{\pgfqpoint{0.000000in}{0.000000in}}%
\pgfpathlineto{\pgfqpoint{0.000000in}{-0.048611in}}%
\pgfusepath{stroke,fill}%
}%
\begin{pgfscope}%
\pgfsys@transformshift{1.361959in}{0.548769in}%
\pgfsys@useobject{currentmarker}{}%
\end{pgfscope}%
\end{pgfscope}%
\begin{pgfscope}%
\definecolor{textcolor}{rgb}{0.000000,0.000000,0.000000}%
\pgfsetstrokecolor{textcolor}%
\pgfsetfillcolor{textcolor}%
\pgftext[x=1.361959in,y=0.451547in,,top]{\color{textcolor}\sffamily\fontsize{10.000000}{12.000000}\selectfont \(\displaystyle {0.1}\)}%
\end{pgfscope}%
\begin{pgfscope}%
\pgfsetbuttcap%
\pgfsetroundjoin%
\definecolor{currentfill}{rgb}{0.000000,0.000000,0.000000}%
\pgfsetfillcolor{currentfill}%
\pgfsetlinewidth{0.803000pt}%
\definecolor{currentstroke}{rgb}{0.000000,0.000000,0.000000}%
\pgfsetstrokecolor{currentstroke}%
\pgfsetdash{}{0pt}%
\pgfsys@defobject{currentmarker}{\pgfqpoint{0.000000in}{-0.048611in}}{\pgfqpoint{0.000000in}{0.000000in}}{%
\pgfpathmoveto{\pgfqpoint{0.000000in}{0.000000in}}%
\pgfpathlineto{\pgfqpoint{0.000000in}{-0.048611in}}%
\pgfusepath{stroke,fill}%
}%
\begin{pgfscope}%
\pgfsys@transformshift{1.850808in}{0.548769in}%
\pgfsys@useobject{currentmarker}{}%
\end{pgfscope}%
\end{pgfscope}%
\begin{pgfscope}%
\definecolor{textcolor}{rgb}{0.000000,0.000000,0.000000}%
\pgfsetstrokecolor{textcolor}%
\pgfsetfillcolor{textcolor}%
\pgftext[x=1.850808in,y=0.451547in,,top]{\color{textcolor}\sffamily\fontsize{10.000000}{12.000000}\selectfont \(\displaystyle {0.2}\)}%
\end{pgfscope}%
\begin{pgfscope}%
\pgfsetbuttcap%
\pgfsetroundjoin%
\definecolor{currentfill}{rgb}{0.000000,0.000000,0.000000}%
\pgfsetfillcolor{currentfill}%
\pgfsetlinewidth{0.803000pt}%
\definecolor{currentstroke}{rgb}{0.000000,0.000000,0.000000}%
\pgfsetstrokecolor{currentstroke}%
\pgfsetdash{}{0pt}%
\pgfsys@defobject{currentmarker}{\pgfqpoint{0.000000in}{-0.048611in}}{\pgfqpoint{0.000000in}{0.000000in}}{%
\pgfpathmoveto{\pgfqpoint{0.000000in}{0.000000in}}%
\pgfpathlineto{\pgfqpoint{0.000000in}{-0.048611in}}%
\pgfusepath{stroke,fill}%
}%
\begin{pgfscope}%
\pgfsys@transformshift{2.339656in}{0.548769in}%
\pgfsys@useobject{currentmarker}{}%
\end{pgfscope}%
\end{pgfscope}%
\begin{pgfscope}%
\definecolor{textcolor}{rgb}{0.000000,0.000000,0.000000}%
\pgfsetstrokecolor{textcolor}%
\pgfsetfillcolor{textcolor}%
\pgftext[x=2.339656in,y=0.451547in,,top]{\color{textcolor}\sffamily\fontsize{10.000000}{12.000000}\selectfont \(\displaystyle {0.3}\)}%
\end{pgfscope}%
\begin{pgfscope}%
\pgfsetbuttcap%
\pgfsetroundjoin%
\definecolor{currentfill}{rgb}{0.000000,0.000000,0.000000}%
\pgfsetfillcolor{currentfill}%
\pgfsetlinewidth{0.803000pt}%
\definecolor{currentstroke}{rgb}{0.000000,0.000000,0.000000}%
\pgfsetstrokecolor{currentstroke}%
\pgfsetdash{}{0pt}%
\pgfsys@defobject{currentmarker}{\pgfqpoint{0.000000in}{-0.048611in}}{\pgfqpoint{0.000000in}{0.000000in}}{%
\pgfpathmoveto{\pgfqpoint{0.000000in}{0.000000in}}%
\pgfpathlineto{\pgfqpoint{0.000000in}{-0.048611in}}%
\pgfusepath{stroke,fill}%
}%
\begin{pgfscope}%
\pgfsys@transformshift{2.828505in}{0.548769in}%
\pgfsys@useobject{currentmarker}{}%
\end{pgfscope}%
\end{pgfscope}%
\begin{pgfscope}%
\definecolor{textcolor}{rgb}{0.000000,0.000000,0.000000}%
\pgfsetstrokecolor{textcolor}%
\pgfsetfillcolor{textcolor}%
\pgftext[x=2.828505in,y=0.451547in,,top]{\color{textcolor}\sffamily\fontsize{10.000000}{12.000000}\selectfont \(\displaystyle {0.4}\)}%
\end{pgfscope}%
\begin{pgfscope}%
\pgfsetbuttcap%
\pgfsetroundjoin%
\definecolor{currentfill}{rgb}{0.000000,0.000000,0.000000}%
\pgfsetfillcolor{currentfill}%
\pgfsetlinewidth{0.803000pt}%
\definecolor{currentstroke}{rgb}{0.000000,0.000000,0.000000}%
\pgfsetstrokecolor{currentstroke}%
\pgfsetdash{}{0pt}%
\pgfsys@defobject{currentmarker}{\pgfqpoint{0.000000in}{-0.048611in}}{\pgfqpoint{0.000000in}{0.000000in}}{%
\pgfpathmoveto{\pgfqpoint{0.000000in}{0.000000in}}%
\pgfpathlineto{\pgfqpoint{0.000000in}{-0.048611in}}%
\pgfusepath{stroke,fill}%
}%
\begin{pgfscope}%
\pgfsys@transformshift{3.317354in}{0.548769in}%
\pgfsys@useobject{currentmarker}{}%
\end{pgfscope}%
\end{pgfscope}%
\begin{pgfscope}%
\definecolor{textcolor}{rgb}{0.000000,0.000000,0.000000}%
\pgfsetstrokecolor{textcolor}%
\pgfsetfillcolor{textcolor}%
\pgftext[x=3.317354in,y=0.451547in,,top]{\color{textcolor}\sffamily\fontsize{10.000000}{12.000000}\selectfont \(\displaystyle {0.5}\)}%
\end{pgfscope}%
\begin{pgfscope}%
\pgfsetbuttcap%
\pgfsetroundjoin%
\definecolor{currentfill}{rgb}{0.000000,0.000000,0.000000}%
\pgfsetfillcolor{currentfill}%
\pgfsetlinewidth{0.803000pt}%
\definecolor{currentstroke}{rgb}{0.000000,0.000000,0.000000}%
\pgfsetstrokecolor{currentstroke}%
\pgfsetdash{}{0pt}%
\pgfsys@defobject{currentmarker}{\pgfqpoint{0.000000in}{-0.048611in}}{\pgfqpoint{0.000000in}{0.000000in}}{%
\pgfpathmoveto{\pgfqpoint{0.000000in}{0.000000in}}%
\pgfpathlineto{\pgfqpoint{0.000000in}{-0.048611in}}%
\pgfusepath{stroke,fill}%
}%
\begin{pgfscope}%
\pgfsys@transformshift{3.806202in}{0.548769in}%
\pgfsys@useobject{currentmarker}{}%
\end{pgfscope}%
\end{pgfscope}%
\begin{pgfscope}%
\definecolor{textcolor}{rgb}{0.000000,0.000000,0.000000}%
\pgfsetstrokecolor{textcolor}%
\pgfsetfillcolor{textcolor}%
\pgftext[x=3.806202in,y=0.451547in,,top]{\color{textcolor}\sffamily\fontsize{10.000000}{12.000000}\selectfont \(\displaystyle {0.6}\)}%
\end{pgfscope}%
\begin{pgfscope}%
\pgfsetbuttcap%
\pgfsetroundjoin%
\definecolor{currentfill}{rgb}{0.000000,0.000000,0.000000}%
\pgfsetfillcolor{currentfill}%
\pgfsetlinewidth{0.803000pt}%
\definecolor{currentstroke}{rgb}{0.000000,0.000000,0.000000}%
\pgfsetstrokecolor{currentstroke}%
\pgfsetdash{}{0pt}%
\pgfsys@defobject{currentmarker}{\pgfqpoint{0.000000in}{-0.048611in}}{\pgfqpoint{0.000000in}{0.000000in}}{%
\pgfpathmoveto{\pgfqpoint{0.000000in}{0.000000in}}%
\pgfpathlineto{\pgfqpoint{0.000000in}{-0.048611in}}%
\pgfusepath{stroke,fill}%
}%
\begin{pgfscope}%
\pgfsys@transformshift{4.295051in}{0.548769in}%
\pgfsys@useobject{currentmarker}{}%
\end{pgfscope}%
\end{pgfscope}%
\begin{pgfscope}%
\definecolor{textcolor}{rgb}{0.000000,0.000000,0.000000}%
\pgfsetstrokecolor{textcolor}%
\pgfsetfillcolor{textcolor}%
\pgftext[x=4.295051in,y=0.451547in,,top]{\color{textcolor}\sffamily\fontsize{10.000000}{12.000000}\selectfont \(\displaystyle {0.7}\)}%
\end{pgfscope}%
\begin{pgfscope}%
\pgfsetbuttcap%
\pgfsetroundjoin%
\definecolor{currentfill}{rgb}{0.000000,0.000000,0.000000}%
\pgfsetfillcolor{currentfill}%
\pgfsetlinewidth{0.803000pt}%
\definecolor{currentstroke}{rgb}{0.000000,0.000000,0.000000}%
\pgfsetstrokecolor{currentstroke}%
\pgfsetdash{}{0pt}%
\pgfsys@defobject{currentmarker}{\pgfqpoint{0.000000in}{-0.048611in}}{\pgfqpoint{0.000000in}{0.000000in}}{%
\pgfpathmoveto{\pgfqpoint{0.000000in}{0.000000in}}%
\pgfpathlineto{\pgfqpoint{0.000000in}{-0.048611in}}%
\pgfusepath{stroke,fill}%
}%
\begin{pgfscope}%
\pgfsys@transformshift{4.783899in}{0.548769in}%
\pgfsys@useobject{currentmarker}{}%
\end{pgfscope}%
\end{pgfscope}%
\begin{pgfscope}%
\definecolor{textcolor}{rgb}{0.000000,0.000000,0.000000}%
\pgfsetstrokecolor{textcolor}%
\pgfsetfillcolor{textcolor}%
\pgftext[x=4.783899in,y=0.451547in,,top]{\color{textcolor}\sffamily\fontsize{10.000000}{12.000000}\selectfont \(\displaystyle {0.8}\)}%
\end{pgfscope}%
\begin{pgfscope}%
\pgfsetbuttcap%
\pgfsetroundjoin%
\definecolor{currentfill}{rgb}{0.000000,0.000000,0.000000}%
\pgfsetfillcolor{currentfill}%
\pgfsetlinewidth{0.803000pt}%
\definecolor{currentstroke}{rgb}{0.000000,0.000000,0.000000}%
\pgfsetstrokecolor{currentstroke}%
\pgfsetdash{}{0pt}%
\pgfsys@defobject{currentmarker}{\pgfqpoint{0.000000in}{-0.048611in}}{\pgfqpoint{0.000000in}{0.000000in}}{%
\pgfpathmoveto{\pgfqpoint{0.000000in}{0.000000in}}%
\pgfpathlineto{\pgfqpoint{0.000000in}{-0.048611in}}%
\pgfusepath{stroke,fill}%
}%
\begin{pgfscope}%
\pgfsys@transformshift{5.272748in}{0.548769in}%
\pgfsys@useobject{currentmarker}{}%
\end{pgfscope}%
\end{pgfscope}%
\begin{pgfscope}%
\definecolor{textcolor}{rgb}{0.000000,0.000000,0.000000}%
\pgfsetstrokecolor{textcolor}%
\pgfsetfillcolor{textcolor}%
\pgftext[x=5.272748in,y=0.451547in,,top]{\color{textcolor}\sffamily\fontsize{10.000000}{12.000000}\selectfont \(\displaystyle {0.9}\)}%
\end{pgfscope}%
\begin{pgfscope}%
\pgfsetbuttcap%
\pgfsetroundjoin%
\definecolor{currentfill}{rgb}{0.000000,0.000000,0.000000}%
\pgfsetfillcolor{currentfill}%
\pgfsetlinewidth{0.803000pt}%
\definecolor{currentstroke}{rgb}{0.000000,0.000000,0.000000}%
\pgfsetstrokecolor{currentstroke}%
\pgfsetdash{}{0pt}%
\pgfsys@defobject{currentmarker}{\pgfqpoint{0.000000in}{-0.048611in}}{\pgfqpoint{0.000000in}{0.000000in}}{%
\pgfpathmoveto{\pgfqpoint{0.000000in}{0.000000in}}%
\pgfpathlineto{\pgfqpoint{0.000000in}{-0.048611in}}%
\pgfusepath{stroke,fill}%
}%
\begin{pgfscope}%
\pgfsys@transformshift{5.761597in}{0.548769in}%
\pgfsys@useobject{currentmarker}{}%
\end{pgfscope}%
\end{pgfscope}%
\begin{pgfscope}%
\definecolor{textcolor}{rgb}{0.000000,0.000000,0.000000}%
\pgfsetstrokecolor{textcolor}%
\pgfsetfillcolor{textcolor}%
\pgftext[x=5.761597in,y=0.451547in,,top]{\color{textcolor}\sffamily\fontsize{10.000000}{12.000000}\selectfont \(\displaystyle {1.0}\)}%
\end{pgfscope}%
\begin{pgfscope}%
\definecolor{textcolor}{rgb}{0.000000,0.000000,0.000000}%
\pgfsetstrokecolor{textcolor}%
\pgfsetfillcolor{textcolor}%
\pgftext[x=3.205150in,y=0.272658in,,top]{\color{textcolor}\sffamily\fontsize{10.000000}{12.000000}\selectfont Edge Count}%
\end{pgfscope}%
\begin{pgfscope}%
\definecolor{textcolor}{rgb}{0.000000,0.000000,0.000000}%
\pgfsetstrokecolor{textcolor}%
\pgfsetfillcolor{textcolor}%
\pgftext[x=5.761597in,y=0.286547in,right,top]{\color{textcolor}\sffamily\fontsize{10.000000}{12.000000}\selectfont \(\displaystyle \times{10^{8}}{}\)}%
\end{pgfscope}%
\begin{pgfscope}%
\pgfsetbuttcap%
\pgfsetroundjoin%
\definecolor{currentfill}{rgb}{0.000000,0.000000,0.000000}%
\pgfsetfillcolor{currentfill}%
\pgfsetlinewidth{0.803000pt}%
\definecolor{currentstroke}{rgb}{0.000000,0.000000,0.000000}%
\pgfsetstrokecolor{currentstroke}%
\pgfsetdash{}{0pt}%
\pgfsys@defobject{currentmarker}{\pgfqpoint{-0.048611in}{0.000000in}}{\pgfqpoint{0.000000in}{0.000000in}}{%
\pgfpathmoveto{\pgfqpoint{0.000000in}{0.000000in}}%
\pgfpathlineto{\pgfqpoint{-0.048611in}{0.000000in}}%
\pgfusepath{stroke,fill}%
}%
\begin{pgfscope}%
\pgfsys@transformshift{0.648703in}{0.689796in}%
\pgfsys@useobject{currentmarker}{}%
\end{pgfscope}%
\end{pgfscope}%
\begin{pgfscope}%
\definecolor{textcolor}{rgb}{0.000000,0.000000,0.000000}%
\pgfsetstrokecolor{textcolor}%
\pgfsetfillcolor{textcolor}%
\pgftext[x=0.482036in, y=0.641601in, left, base]{\color{textcolor}\sffamily\fontsize{10.000000}{12.000000}\selectfont \(\displaystyle {0}\)}%
\end{pgfscope}%
\begin{pgfscope}%
\pgfsetbuttcap%
\pgfsetroundjoin%
\definecolor{currentfill}{rgb}{0.000000,0.000000,0.000000}%
\pgfsetfillcolor{currentfill}%
\pgfsetlinewidth{0.803000pt}%
\definecolor{currentstroke}{rgb}{0.000000,0.000000,0.000000}%
\pgfsetstrokecolor{currentstroke}%
\pgfsetdash{}{0pt}%
\pgfsys@defobject{currentmarker}{\pgfqpoint{-0.048611in}{0.000000in}}{\pgfqpoint{0.000000in}{0.000000in}}{%
\pgfpathmoveto{\pgfqpoint{0.000000in}{0.000000in}}%
\pgfpathlineto{\pgfqpoint{-0.048611in}{0.000000in}}%
\pgfusepath{stroke,fill}%
}%
\begin{pgfscope}%
\pgfsys@transformshift{0.648703in}{1.104580in}%
\pgfsys@useobject{currentmarker}{}%
\end{pgfscope}%
\end{pgfscope}%
\begin{pgfscope}%
\definecolor{textcolor}{rgb}{0.000000,0.000000,0.000000}%
\pgfsetstrokecolor{textcolor}%
\pgfsetfillcolor{textcolor}%
\pgftext[x=0.343147in, y=1.056386in, left, base]{\color{textcolor}\sffamily\fontsize{10.000000}{12.000000}\selectfont \(\displaystyle {100}\)}%
\end{pgfscope}%
\begin{pgfscope}%
\pgfsetbuttcap%
\pgfsetroundjoin%
\definecolor{currentfill}{rgb}{0.000000,0.000000,0.000000}%
\pgfsetfillcolor{currentfill}%
\pgfsetlinewidth{0.803000pt}%
\definecolor{currentstroke}{rgb}{0.000000,0.000000,0.000000}%
\pgfsetstrokecolor{currentstroke}%
\pgfsetdash{}{0pt}%
\pgfsys@defobject{currentmarker}{\pgfqpoint{-0.048611in}{0.000000in}}{\pgfqpoint{0.000000in}{0.000000in}}{%
\pgfpathmoveto{\pgfqpoint{0.000000in}{0.000000in}}%
\pgfpathlineto{\pgfqpoint{-0.048611in}{0.000000in}}%
\pgfusepath{stroke,fill}%
}%
\begin{pgfscope}%
\pgfsys@transformshift{0.648703in}{1.519365in}%
\pgfsys@useobject{currentmarker}{}%
\end{pgfscope}%
\end{pgfscope}%
\begin{pgfscope}%
\definecolor{textcolor}{rgb}{0.000000,0.000000,0.000000}%
\pgfsetstrokecolor{textcolor}%
\pgfsetfillcolor{textcolor}%
\pgftext[x=0.343147in, y=1.471171in, left, base]{\color{textcolor}\sffamily\fontsize{10.000000}{12.000000}\selectfont \(\displaystyle {200}\)}%
\end{pgfscope}%
\begin{pgfscope}%
\pgfsetbuttcap%
\pgfsetroundjoin%
\definecolor{currentfill}{rgb}{0.000000,0.000000,0.000000}%
\pgfsetfillcolor{currentfill}%
\pgfsetlinewidth{0.803000pt}%
\definecolor{currentstroke}{rgb}{0.000000,0.000000,0.000000}%
\pgfsetstrokecolor{currentstroke}%
\pgfsetdash{}{0pt}%
\pgfsys@defobject{currentmarker}{\pgfqpoint{-0.048611in}{0.000000in}}{\pgfqpoint{0.000000in}{0.000000in}}{%
\pgfpathmoveto{\pgfqpoint{0.000000in}{0.000000in}}%
\pgfpathlineto{\pgfqpoint{-0.048611in}{0.000000in}}%
\pgfusepath{stroke,fill}%
}%
\begin{pgfscope}%
\pgfsys@transformshift{0.648703in}{1.934150in}%
\pgfsys@useobject{currentmarker}{}%
\end{pgfscope}%
\end{pgfscope}%
\begin{pgfscope}%
\definecolor{textcolor}{rgb}{0.000000,0.000000,0.000000}%
\pgfsetstrokecolor{textcolor}%
\pgfsetfillcolor{textcolor}%
\pgftext[x=0.343147in, y=1.885955in, left, base]{\color{textcolor}\sffamily\fontsize{10.000000}{12.000000}\selectfont \(\displaystyle {300}\)}%
\end{pgfscope}%
\begin{pgfscope}%
\pgfsetbuttcap%
\pgfsetroundjoin%
\definecolor{currentfill}{rgb}{0.000000,0.000000,0.000000}%
\pgfsetfillcolor{currentfill}%
\pgfsetlinewidth{0.803000pt}%
\definecolor{currentstroke}{rgb}{0.000000,0.000000,0.000000}%
\pgfsetstrokecolor{currentstroke}%
\pgfsetdash{}{0pt}%
\pgfsys@defobject{currentmarker}{\pgfqpoint{-0.048611in}{0.000000in}}{\pgfqpoint{0.000000in}{0.000000in}}{%
\pgfpathmoveto{\pgfqpoint{0.000000in}{0.000000in}}%
\pgfpathlineto{\pgfqpoint{-0.048611in}{0.000000in}}%
\pgfusepath{stroke,fill}%
}%
\begin{pgfscope}%
\pgfsys@transformshift{0.648703in}{2.348935in}%
\pgfsys@useobject{currentmarker}{}%
\end{pgfscope}%
\end{pgfscope}%
\begin{pgfscope}%
\definecolor{textcolor}{rgb}{0.000000,0.000000,0.000000}%
\pgfsetstrokecolor{textcolor}%
\pgfsetfillcolor{textcolor}%
\pgftext[x=0.343147in, y=2.300740in, left, base]{\color{textcolor}\sffamily\fontsize{10.000000}{12.000000}\selectfont \(\displaystyle {400}\)}%
\end{pgfscope}%
\begin{pgfscope}%
\pgfsetbuttcap%
\pgfsetroundjoin%
\definecolor{currentfill}{rgb}{0.000000,0.000000,0.000000}%
\pgfsetfillcolor{currentfill}%
\pgfsetlinewidth{0.803000pt}%
\definecolor{currentstroke}{rgb}{0.000000,0.000000,0.000000}%
\pgfsetstrokecolor{currentstroke}%
\pgfsetdash{}{0pt}%
\pgfsys@defobject{currentmarker}{\pgfqpoint{-0.048611in}{0.000000in}}{\pgfqpoint{0.000000in}{0.000000in}}{%
\pgfpathmoveto{\pgfqpoint{0.000000in}{0.000000in}}%
\pgfpathlineto{\pgfqpoint{-0.048611in}{0.000000in}}%
\pgfusepath{stroke,fill}%
}%
\begin{pgfscope}%
\pgfsys@transformshift{0.648703in}{2.763719in}%
\pgfsys@useobject{currentmarker}{}%
\end{pgfscope}%
\end{pgfscope}%
\begin{pgfscope}%
\definecolor{textcolor}{rgb}{0.000000,0.000000,0.000000}%
\pgfsetstrokecolor{textcolor}%
\pgfsetfillcolor{textcolor}%
\pgftext[x=0.343147in, y=2.715525in, left, base]{\color{textcolor}\sffamily\fontsize{10.000000}{12.000000}\selectfont \(\displaystyle {500}\)}%
\end{pgfscope}%
\begin{pgfscope}%
\pgfsetbuttcap%
\pgfsetroundjoin%
\definecolor{currentfill}{rgb}{0.000000,0.000000,0.000000}%
\pgfsetfillcolor{currentfill}%
\pgfsetlinewidth{0.803000pt}%
\definecolor{currentstroke}{rgb}{0.000000,0.000000,0.000000}%
\pgfsetstrokecolor{currentstroke}%
\pgfsetdash{}{0pt}%
\pgfsys@defobject{currentmarker}{\pgfqpoint{-0.048611in}{0.000000in}}{\pgfqpoint{0.000000in}{0.000000in}}{%
\pgfpathmoveto{\pgfqpoint{0.000000in}{0.000000in}}%
\pgfpathlineto{\pgfqpoint{-0.048611in}{0.000000in}}%
\pgfusepath{stroke,fill}%
}%
\begin{pgfscope}%
\pgfsys@transformshift{0.648703in}{3.178504in}%
\pgfsys@useobject{currentmarker}{}%
\end{pgfscope}%
\end{pgfscope}%
\begin{pgfscope}%
\definecolor{textcolor}{rgb}{0.000000,0.000000,0.000000}%
\pgfsetstrokecolor{textcolor}%
\pgfsetfillcolor{textcolor}%
\pgftext[x=0.343147in, y=3.130310in, left, base]{\color{textcolor}\sffamily\fontsize{10.000000}{12.000000}\selectfont \(\displaystyle {600}\)}%
\end{pgfscope}%
\begin{pgfscope}%
\pgfsetbuttcap%
\pgfsetroundjoin%
\definecolor{currentfill}{rgb}{0.000000,0.000000,0.000000}%
\pgfsetfillcolor{currentfill}%
\pgfsetlinewidth{0.803000pt}%
\definecolor{currentstroke}{rgb}{0.000000,0.000000,0.000000}%
\pgfsetstrokecolor{currentstroke}%
\pgfsetdash{}{0pt}%
\pgfsys@defobject{currentmarker}{\pgfqpoint{-0.048611in}{0.000000in}}{\pgfqpoint{0.000000in}{0.000000in}}{%
\pgfpathmoveto{\pgfqpoint{0.000000in}{0.000000in}}%
\pgfpathlineto{\pgfqpoint{-0.048611in}{0.000000in}}%
\pgfusepath{stroke,fill}%
}%
\begin{pgfscope}%
\pgfsys@transformshift{0.648703in}{3.593289in}%
\pgfsys@useobject{currentmarker}{}%
\end{pgfscope}%
\end{pgfscope}%
\begin{pgfscope}%
\definecolor{textcolor}{rgb}{0.000000,0.000000,0.000000}%
\pgfsetstrokecolor{textcolor}%
\pgfsetfillcolor{textcolor}%
\pgftext[x=0.343147in, y=3.545094in, left, base]{\color{textcolor}\sffamily\fontsize{10.000000}{12.000000}\selectfont \(\displaystyle {700}\)}%
\end{pgfscope}%
\begin{pgfscope}%
\definecolor{textcolor}{rgb}{0.000000,0.000000,0.000000}%
\pgfsetstrokecolor{textcolor}%
\pgfsetfillcolor{textcolor}%
\pgftext[x=0.287592in,y=2.100064in,,bottom,rotate=90.000000]{\color{textcolor}\sffamily\fontsize{10.000000}{12.000000}\selectfont Data Flow Time (s)}%
\end{pgfscope}%
\begin{pgfscope}%
\pgfsetrectcap%
\pgfsetmiterjoin%
\pgfsetlinewidth{0.803000pt}%
\definecolor{currentstroke}{rgb}{0.000000,0.000000,0.000000}%
\pgfsetstrokecolor{currentstroke}%
\pgfsetdash{}{0pt}%
\pgfpathmoveto{\pgfqpoint{0.648703in}{0.548769in}}%
\pgfpathlineto{\pgfqpoint{0.648703in}{3.651359in}}%
\pgfusepath{stroke}%
\end{pgfscope}%
\begin{pgfscope}%
\pgfsetrectcap%
\pgfsetmiterjoin%
\pgfsetlinewidth{0.803000pt}%
\definecolor{currentstroke}{rgb}{0.000000,0.000000,0.000000}%
\pgfsetstrokecolor{currentstroke}%
\pgfsetdash{}{0pt}%
\pgfpathmoveto{\pgfqpoint{5.761597in}{0.548769in}}%
\pgfpathlineto{\pgfqpoint{5.761597in}{3.651359in}}%
\pgfusepath{stroke}%
\end{pgfscope}%
\begin{pgfscope}%
\pgfsetrectcap%
\pgfsetmiterjoin%
\pgfsetlinewidth{0.803000pt}%
\definecolor{currentstroke}{rgb}{0.000000,0.000000,0.000000}%
\pgfsetstrokecolor{currentstroke}%
\pgfsetdash{}{0pt}%
\pgfpathmoveto{\pgfqpoint{0.648703in}{0.548769in}}%
\pgfpathlineto{\pgfqpoint{5.761597in}{0.548769in}}%
\pgfusepath{stroke}%
\end{pgfscope}%
\begin{pgfscope}%
\pgfsetrectcap%
\pgfsetmiterjoin%
\pgfsetlinewidth{0.803000pt}%
\definecolor{currentstroke}{rgb}{0.000000,0.000000,0.000000}%
\pgfsetstrokecolor{currentstroke}%
\pgfsetdash{}{0pt}%
\pgfpathmoveto{\pgfqpoint{0.648703in}{3.651359in}}%
\pgfpathlineto{\pgfqpoint{5.761597in}{3.651359in}}%
\pgfusepath{stroke}%
\end{pgfscope}%
\begin{pgfscope}%
\definecolor{textcolor}{rgb}{0.000000,0.000000,0.000000}%
\pgfsetstrokecolor{textcolor}%
\pgfsetfillcolor{textcolor}%
\pgftext[x=3.205150in,y=3.734692in,,base]{\color{textcolor}\sffamily\fontsize{12.000000}{14.400000}\selectfont Forward}%
\end{pgfscope}%
\begin{pgfscope}%
\pgfsetbuttcap%
\pgfsetmiterjoin%
\definecolor{currentfill}{rgb}{1.000000,1.000000,1.000000}%
\pgfsetfillcolor{currentfill}%
\pgfsetfillopacity{0.800000}%
\pgfsetlinewidth{1.003750pt}%
\definecolor{currentstroke}{rgb}{0.800000,0.800000,0.800000}%
\pgfsetstrokecolor{currentstroke}%
\pgfsetstrokeopacity{0.800000}%
\pgfsetdash{}{0pt}%
\pgfpathmoveto{\pgfqpoint{4.212013in}{0.618213in}}%
\pgfpathlineto{\pgfqpoint{5.664374in}{0.618213in}}%
\pgfpathquadraticcurveto{\pgfqpoint{5.692152in}{0.618213in}}{\pgfqpoint{5.692152in}{0.645991in}}%
\pgfpathlineto{\pgfqpoint{5.692152in}{1.214463in}}%
\pgfpathquadraticcurveto{\pgfqpoint{5.692152in}{1.242241in}}{\pgfqpoint{5.664374in}{1.242241in}}%
\pgfpathlineto{\pgfqpoint{4.212013in}{1.242241in}}%
\pgfpathquadraticcurveto{\pgfqpoint{4.184236in}{1.242241in}}{\pgfqpoint{4.184236in}{1.214463in}}%
\pgfpathlineto{\pgfqpoint{4.184236in}{0.645991in}}%
\pgfpathquadraticcurveto{\pgfqpoint{4.184236in}{0.618213in}}{\pgfqpoint{4.212013in}{0.618213in}}%
\pgfpathclose%
\pgfusepath{stroke,fill}%
\end{pgfscope}%
\begin{pgfscope}%
\pgfsetbuttcap%
\pgfsetroundjoin%
\definecolor{currentfill}{rgb}{0.121569,0.466667,0.705882}%
\pgfsetfillcolor{currentfill}%
\pgfsetlinewidth{1.003750pt}%
\definecolor{currentstroke}{rgb}{0.121569,0.466667,0.705882}%
\pgfsetstrokecolor{currentstroke}%
\pgfsetdash{}{0pt}%
\pgfsys@defobject{currentmarker}{\pgfqpoint{-0.034722in}{-0.034722in}}{\pgfqpoint{0.034722in}{0.034722in}}{%
\pgfpathmoveto{\pgfqpoint{0.000000in}{-0.034722in}}%
\pgfpathcurveto{\pgfqpoint{0.009208in}{-0.034722in}}{\pgfqpoint{0.018041in}{-0.031064in}}{\pgfqpoint{0.024552in}{-0.024552in}}%
\pgfpathcurveto{\pgfqpoint{0.031064in}{-0.018041in}}{\pgfqpoint{0.034722in}{-0.009208in}}{\pgfqpoint{0.034722in}{0.000000in}}%
\pgfpathcurveto{\pgfqpoint{0.034722in}{0.009208in}}{\pgfqpoint{0.031064in}{0.018041in}}{\pgfqpoint{0.024552in}{0.024552in}}%
\pgfpathcurveto{\pgfqpoint{0.018041in}{0.031064in}}{\pgfqpoint{0.009208in}{0.034722in}}{\pgfqpoint{0.000000in}{0.034722in}}%
\pgfpathcurveto{\pgfqpoint{-0.009208in}{0.034722in}}{\pgfqpoint{-0.018041in}{0.031064in}}{\pgfqpoint{-0.024552in}{0.024552in}}%
\pgfpathcurveto{\pgfqpoint{-0.031064in}{0.018041in}}{\pgfqpoint{-0.034722in}{0.009208in}}{\pgfqpoint{-0.034722in}{0.000000in}}%
\pgfpathcurveto{\pgfqpoint{-0.034722in}{-0.009208in}}{\pgfqpoint{-0.031064in}{-0.018041in}}{\pgfqpoint{-0.024552in}{-0.024552in}}%
\pgfpathcurveto{\pgfqpoint{-0.018041in}{-0.031064in}}{\pgfqpoint{-0.009208in}{-0.034722in}}{\pgfqpoint{0.000000in}{-0.034722in}}%
\pgfpathclose%
\pgfusepath{stroke,fill}%
}%
\begin{pgfscope}%
\pgfsys@transformshift{4.378680in}{1.138074in}%
\pgfsys@useobject{currentmarker}{}%
\end{pgfscope}%
\end{pgfscope}%
\begin{pgfscope}%
\definecolor{textcolor}{rgb}{0.000000,0.000000,0.000000}%
\pgfsetstrokecolor{textcolor}%
\pgfsetfillcolor{textcolor}%
\pgftext[x=4.628680in,y=1.089463in,left,base]{\color{textcolor}\sffamily\fontsize{10.000000}{12.000000}\selectfont No Timeout}%
\end{pgfscope}%
\begin{pgfscope}%
\pgfsetbuttcap%
\pgfsetroundjoin%
\definecolor{currentfill}{rgb}{1.000000,0.498039,0.054902}%
\pgfsetfillcolor{currentfill}%
\pgfsetlinewidth{1.003750pt}%
\definecolor{currentstroke}{rgb}{1.000000,0.498039,0.054902}%
\pgfsetstrokecolor{currentstroke}%
\pgfsetdash{}{0pt}%
\pgfsys@defobject{currentmarker}{\pgfqpoint{-0.034722in}{-0.034722in}}{\pgfqpoint{0.034722in}{0.034722in}}{%
\pgfpathmoveto{\pgfqpoint{0.000000in}{-0.034722in}}%
\pgfpathcurveto{\pgfqpoint{0.009208in}{-0.034722in}}{\pgfqpoint{0.018041in}{-0.031064in}}{\pgfqpoint{0.024552in}{-0.024552in}}%
\pgfpathcurveto{\pgfqpoint{0.031064in}{-0.018041in}}{\pgfqpoint{0.034722in}{-0.009208in}}{\pgfqpoint{0.034722in}{0.000000in}}%
\pgfpathcurveto{\pgfqpoint{0.034722in}{0.009208in}}{\pgfqpoint{0.031064in}{0.018041in}}{\pgfqpoint{0.024552in}{0.024552in}}%
\pgfpathcurveto{\pgfqpoint{0.018041in}{0.031064in}}{\pgfqpoint{0.009208in}{0.034722in}}{\pgfqpoint{0.000000in}{0.034722in}}%
\pgfpathcurveto{\pgfqpoint{-0.009208in}{0.034722in}}{\pgfqpoint{-0.018041in}{0.031064in}}{\pgfqpoint{-0.024552in}{0.024552in}}%
\pgfpathcurveto{\pgfqpoint{-0.031064in}{0.018041in}}{\pgfqpoint{-0.034722in}{0.009208in}}{\pgfqpoint{-0.034722in}{0.000000in}}%
\pgfpathcurveto{\pgfqpoint{-0.034722in}{-0.009208in}}{\pgfqpoint{-0.031064in}{-0.018041in}}{\pgfqpoint{-0.024552in}{-0.024552in}}%
\pgfpathcurveto{\pgfqpoint{-0.018041in}{-0.031064in}}{\pgfqpoint{-0.009208in}{-0.034722in}}{\pgfqpoint{0.000000in}{-0.034722in}}%
\pgfpathclose%
\pgfusepath{stroke,fill}%
}%
\begin{pgfscope}%
\pgfsys@transformshift{4.378680in}{0.944463in}%
\pgfsys@useobject{currentmarker}{}%
\end{pgfscope}%
\end{pgfscope}%
\begin{pgfscope}%
\definecolor{textcolor}{rgb}{0.000000,0.000000,0.000000}%
\pgfsetstrokecolor{textcolor}%
\pgfsetfillcolor{textcolor}%
\pgftext[x=4.628680in,y=0.895852in,left,base]{\color{textcolor}\sffamily\fontsize{10.000000}{12.000000}\selectfont Time Timeout}%
\end{pgfscope}%
\begin{pgfscope}%
\pgfsetbuttcap%
\pgfsetroundjoin%
\definecolor{currentfill}{rgb}{0.839216,0.152941,0.156863}%
\pgfsetfillcolor{currentfill}%
\pgfsetlinewidth{1.003750pt}%
\definecolor{currentstroke}{rgb}{0.839216,0.152941,0.156863}%
\pgfsetstrokecolor{currentstroke}%
\pgfsetdash{}{0pt}%
\pgfsys@defobject{currentmarker}{\pgfqpoint{-0.034722in}{-0.034722in}}{\pgfqpoint{0.034722in}{0.034722in}}{%
\pgfpathmoveto{\pgfqpoint{0.000000in}{-0.034722in}}%
\pgfpathcurveto{\pgfqpoint{0.009208in}{-0.034722in}}{\pgfqpoint{0.018041in}{-0.031064in}}{\pgfqpoint{0.024552in}{-0.024552in}}%
\pgfpathcurveto{\pgfqpoint{0.031064in}{-0.018041in}}{\pgfqpoint{0.034722in}{-0.009208in}}{\pgfqpoint{0.034722in}{0.000000in}}%
\pgfpathcurveto{\pgfqpoint{0.034722in}{0.009208in}}{\pgfqpoint{0.031064in}{0.018041in}}{\pgfqpoint{0.024552in}{0.024552in}}%
\pgfpathcurveto{\pgfqpoint{0.018041in}{0.031064in}}{\pgfqpoint{0.009208in}{0.034722in}}{\pgfqpoint{0.000000in}{0.034722in}}%
\pgfpathcurveto{\pgfqpoint{-0.009208in}{0.034722in}}{\pgfqpoint{-0.018041in}{0.031064in}}{\pgfqpoint{-0.024552in}{0.024552in}}%
\pgfpathcurveto{\pgfqpoint{-0.031064in}{0.018041in}}{\pgfqpoint{-0.034722in}{0.009208in}}{\pgfqpoint{-0.034722in}{0.000000in}}%
\pgfpathcurveto{\pgfqpoint{-0.034722in}{-0.009208in}}{\pgfqpoint{-0.031064in}{-0.018041in}}{\pgfqpoint{-0.024552in}{-0.024552in}}%
\pgfpathcurveto{\pgfqpoint{-0.018041in}{-0.031064in}}{\pgfqpoint{-0.009208in}{-0.034722in}}{\pgfqpoint{0.000000in}{-0.034722in}}%
\pgfpathclose%
\pgfusepath{stroke,fill}%
}%
\begin{pgfscope}%
\pgfsys@transformshift{4.378680in}{0.750852in}%
\pgfsys@useobject{currentmarker}{}%
\end{pgfscope}%
\end{pgfscope}%
\begin{pgfscope}%
\definecolor{textcolor}{rgb}{0.000000,0.000000,0.000000}%
\pgfsetstrokecolor{textcolor}%
\pgfsetfillcolor{textcolor}%
\pgftext[x=4.628680in,y=0.702241in,left,base]{\color{textcolor}\sffamily\fontsize{10.000000}{12.000000}\selectfont Memory Timeout}%
\end{pgfscope}%
\end{pgfpicture}%
\makeatother%
\endgroup%

                }
            \end{subfigure}
            \qquad
            \begin{subfigure}[]{0.45\textwidth}
                \centering
                \resizebox{\columnwidth}{!}{
                    %% Creator: Matplotlib, PGF backend
%%
%% To include the figure in your LaTeX document, write
%%   \input{<filename>.pgf}
%%
%% Make sure the required packages are loaded in your preamble
%%   \usepackage{pgf}
%%
%% and, on pdftex
%%   \usepackage[utf8]{inputenc}\DeclareUnicodeCharacter{2212}{-}
%%
%% or, on luatex and xetex
%%   \usepackage{unicode-math}
%%
%% Figures using additional raster images can only be included by \input if
%% they are in the same directory as the main LaTeX file. For loading figures
%% from other directories you can use the `import` package
%%   \usepackage{import}
%%
%% and then include the figures with
%%   \import{<path to file>}{<filename>.pgf}
%%
%% Matplotlib used the following preamble
%%   \usepackage{amsmath}
%%   \usepackage{fontspec}
%%
\begingroup%
\makeatletter%
\begin{pgfpicture}%
\pgfpathrectangle{\pgfpointorigin}{\pgfqpoint{6.000000in}{4.000000in}}%
\pgfusepath{use as bounding box, clip}%
\begin{pgfscope}%
\pgfsetbuttcap%
\pgfsetmiterjoin%
\definecolor{currentfill}{rgb}{1.000000,1.000000,1.000000}%
\pgfsetfillcolor{currentfill}%
\pgfsetlinewidth{0.000000pt}%
\definecolor{currentstroke}{rgb}{1.000000,1.000000,1.000000}%
\pgfsetstrokecolor{currentstroke}%
\pgfsetdash{}{0pt}%
\pgfpathmoveto{\pgfqpoint{0.000000in}{0.000000in}}%
\pgfpathlineto{\pgfqpoint{6.000000in}{0.000000in}}%
\pgfpathlineto{\pgfqpoint{6.000000in}{4.000000in}}%
\pgfpathlineto{\pgfqpoint{0.000000in}{4.000000in}}%
\pgfpathclose%
\pgfusepath{fill}%
\end{pgfscope}%
\begin{pgfscope}%
\pgfsetbuttcap%
\pgfsetmiterjoin%
\definecolor{currentfill}{rgb}{1.000000,1.000000,1.000000}%
\pgfsetfillcolor{currentfill}%
\pgfsetlinewidth{0.000000pt}%
\definecolor{currentstroke}{rgb}{0.000000,0.000000,0.000000}%
\pgfsetstrokecolor{currentstroke}%
\pgfsetstrokeopacity{0.000000}%
\pgfsetdash{}{0pt}%
\pgfpathmoveto{\pgfqpoint{0.648703in}{0.548769in}}%
\pgfpathlineto{\pgfqpoint{5.850000in}{0.548769in}}%
\pgfpathlineto{\pgfqpoint{5.850000in}{3.651359in}}%
\pgfpathlineto{\pgfqpoint{0.648703in}{3.651359in}}%
\pgfpathclose%
\pgfusepath{fill}%
\end{pgfscope}%
\begin{pgfscope}%
\pgfpathrectangle{\pgfqpoint{0.648703in}{0.548769in}}{\pgfqpoint{5.201297in}{3.102590in}}%
\pgfusepath{clip}%
\pgfsetbuttcap%
\pgfsetroundjoin%
\definecolor{currentfill}{rgb}{0.121569,0.466667,0.705882}%
\pgfsetfillcolor{currentfill}%
\pgfsetlinewidth{1.003750pt}%
\definecolor{currentstroke}{rgb}{0.121569,0.466667,0.705882}%
\pgfsetstrokecolor{currentstroke}%
\pgfsetdash{}{0pt}%
\pgfpathmoveto{\pgfqpoint{0.942234in}{0.673501in}}%
\pgfpathcurveto{\pgfqpoint{0.953284in}{0.673501in}}{\pgfqpoint{0.963883in}{0.677891in}}{\pgfqpoint{0.971697in}{0.685705in}}%
\pgfpathcurveto{\pgfqpoint{0.979510in}{0.693519in}}{\pgfqpoint{0.983901in}{0.704118in}}{\pgfqpoint{0.983901in}{0.715168in}}%
\pgfpathcurveto{\pgfqpoint{0.983901in}{0.726218in}}{\pgfqpoint{0.979510in}{0.736817in}}{\pgfqpoint{0.971697in}{0.744631in}}%
\pgfpathcurveto{\pgfqpoint{0.963883in}{0.752444in}}{\pgfqpoint{0.953284in}{0.756834in}}{\pgfqpoint{0.942234in}{0.756834in}}%
\pgfpathcurveto{\pgfqpoint{0.931184in}{0.756834in}}{\pgfqpoint{0.920585in}{0.752444in}}{\pgfqpoint{0.912771in}{0.744631in}}%
\pgfpathcurveto{\pgfqpoint{0.904958in}{0.736817in}}{\pgfqpoint{0.900567in}{0.726218in}}{\pgfqpoint{0.900567in}{0.715168in}}%
\pgfpathcurveto{\pgfqpoint{0.900567in}{0.704118in}}{\pgfqpoint{0.904958in}{0.693519in}}{\pgfqpoint{0.912771in}{0.685705in}}%
\pgfpathcurveto{\pgfqpoint{0.920585in}{0.677891in}}{\pgfqpoint{0.931184in}{0.673501in}}{\pgfqpoint{0.942234in}{0.673501in}}%
\pgfpathclose%
\pgfusepath{stroke,fill}%
\end{pgfscope}%
\begin{pgfscope}%
\pgfpathrectangle{\pgfqpoint{0.648703in}{0.548769in}}{\pgfqpoint{5.201297in}{3.102590in}}%
\pgfusepath{clip}%
\pgfsetbuttcap%
\pgfsetroundjoin%
\definecolor{currentfill}{rgb}{0.121569,0.466667,0.705882}%
\pgfsetfillcolor{currentfill}%
\pgfsetlinewidth{1.003750pt}%
\definecolor{currentstroke}{rgb}{0.121569,0.466667,0.705882}%
\pgfsetstrokecolor{currentstroke}%
\pgfsetdash{}{0pt}%
\pgfpathmoveto{\pgfqpoint{2.476879in}{3.176886in}}%
\pgfpathcurveto{\pgfqpoint{2.487929in}{3.176886in}}{\pgfqpoint{2.498528in}{3.181276in}}{\pgfqpoint{2.506342in}{3.189089in}}%
\pgfpathcurveto{\pgfqpoint{2.514155in}{3.196903in}}{\pgfqpoint{2.518546in}{3.207502in}}{\pgfqpoint{2.518546in}{3.218552in}}%
\pgfpathcurveto{\pgfqpoint{2.518546in}{3.229602in}}{\pgfqpoint{2.514155in}{3.240201in}}{\pgfqpoint{2.506342in}{3.248015in}}%
\pgfpathcurveto{\pgfqpoint{2.498528in}{3.255829in}}{\pgfqpoint{2.487929in}{3.260219in}}{\pgfqpoint{2.476879in}{3.260219in}}%
\pgfpathcurveto{\pgfqpoint{2.465829in}{3.260219in}}{\pgfqpoint{2.455230in}{3.255829in}}{\pgfqpoint{2.447416in}{3.248015in}}%
\pgfpathcurveto{\pgfqpoint{2.439602in}{3.240201in}}{\pgfqpoint{2.435212in}{3.229602in}}{\pgfqpoint{2.435212in}{3.218552in}}%
\pgfpathcurveto{\pgfqpoint{2.435212in}{3.207502in}}{\pgfqpoint{2.439602in}{3.196903in}}{\pgfqpoint{2.447416in}{3.189089in}}%
\pgfpathcurveto{\pgfqpoint{2.455230in}{3.181276in}}{\pgfqpoint{2.465829in}{3.176886in}}{\pgfqpoint{2.476879in}{3.176886in}}%
\pgfpathclose%
\pgfusepath{stroke,fill}%
\end{pgfscope}%
\begin{pgfscope}%
\pgfpathrectangle{\pgfqpoint{0.648703in}{0.548769in}}{\pgfqpoint{5.201297in}{3.102590in}}%
\pgfusepath{clip}%
\pgfsetbuttcap%
\pgfsetroundjoin%
\definecolor{currentfill}{rgb}{1.000000,0.498039,0.054902}%
\pgfsetfillcolor{currentfill}%
\pgfsetlinewidth{1.003750pt}%
\definecolor{currentstroke}{rgb}{1.000000,0.498039,0.054902}%
\pgfsetstrokecolor{currentstroke}%
\pgfsetdash{}{0pt}%
\pgfpathmoveto{\pgfqpoint{3.771734in}{3.198029in}}%
\pgfpathcurveto{\pgfqpoint{3.782784in}{3.198029in}}{\pgfqpoint{3.793383in}{3.202419in}}{\pgfqpoint{3.801197in}{3.210233in}}%
\pgfpathcurveto{\pgfqpoint{3.809010in}{3.218046in}}{\pgfqpoint{3.813400in}{3.228646in}}{\pgfqpoint{3.813400in}{3.239696in}}%
\pgfpathcurveto{\pgfqpoint{3.813400in}{3.250746in}}{\pgfqpoint{3.809010in}{3.261345in}}{\pgfqpoint{3.801197in}{3.269158in}}%
\pgfpathcurveto{\pgfqpoint{3.793383in}{3.276972in}}{\pgfqpoint{3.782784in}{3.281362in}}{\pgfqpoint{3.771734in}{3.281362in}}%
\pgfpathcurveto{\pgfqpoint{3.760684in}{3.281362in}}{\pgfqpoint{3.750085in}{3.276972in}}{\pgfqpoint{3.742271in}{3.269158in}}%
\pgfpathcurveto{\pgfqpoint{3.734457in}{3.261345in}}{\pgfqpoint{3.730067in}{3.250746in}}{\pgfqpoint{3.730067in}{3.239696in}}%
\pgfpathcurveto{\pgfqpoint{3.730067in}{3.228646in}}{\pgfqpoint{3.734457in}{3.218046in}}{\pgfqpoint{3.742271in}{3.210233in}}%
\pgfpathcurveto{\pgfqpoint{3.750085in}{3.202419in}}{\pgfqpoint{3.760684in}{3.198029in}}{\pgfqpoint{3.771734in}{3.198029in}}%
\pgfpathclose%
\pgfusepath{stroke,fill}%
\end{pgfscope}%
\begin{pgfscope}%
\pgfpathrectangle{\pgfqpoint{0.648703in}{0.548769in}}{\pgfqpoint{5.201297in}{3.102590in}}%
\pgfusepath{clip}%
\pgfsetbuttcap%
\pgfsetroundjoin%
\definecolor{currentfill}{rgb}{1.000000,0.498039,0.054902}%
\pgfsetfillcolor{currentfill}%
\pgfsetlinewidth{1.003750pt}%
\definecolor{currentstroke}{rgb}{1.000000,0.498039,0.054902}%
\pgfsetstrokecolor{currentstroke}%
\pgfsetdash{}{0pt}%
\pgfpathmoveto{\pgfqpoint{3.266726in}{3.185343in}}%
\pgfpathcurveto{\pgfqpoint{3.277776in}{3.185343in}}{\pgfqpoint{3.288375in}{3.189733in}}{\pgfqpoint{3.296189in}{3.197547in}}%
\pgfpathcurveto{\pgfqpoint{3.304002in}{3.205360in}}{\pgfqpoint{3.308393in}{3.215959in}}{\pgfqpoint{3.308393in}{3.227010in}}%
\pgfpathcurveto{\pgfqpoint{3.308393in}{3.238060in}}{\pgfqpoint{3.304002in}{3.248659in}}{\pgfqpoint{3.296189in}{3.256472in}}%
\pgfpathcurveto{\pgfqpoint{3.288375in}{3.264286in}}{\pgfqpoint{3.277776in}{3.268676in}}{\pgfqpoint{3.266726in}{3.268676in}}%
\pgfpathcurveto{\pgfqpoint{3.255676in}{3.268676in}}{\pgfqpoint{3.245077in}{3.264286in}}{\pgfqpoint{3.237263in}{3.256472in}}%
\pgfpathcurveto{\pgfqpoint{3.229450in}{3.248659in}}{\pgfqpoint{3.225059in}{3.238060in}}{\pgfqpoint{3.225059in}{3.227010in}}%
\pgfpathcurveto{\pgfqpoint{3.225059in}{3.215959in}}{\pgfqpoint{3.229450in}{3.205360in}}{\pgfqpoint{3.237263in}{3.197547in}}%
\pgfpathcurveto{\pgfqpoint{3.245077in}{3.189733in}}{\pgfqpoint{3.255676in}{3.185343in}}{\pgfqpoint{3.266726in}{3.185343in}}%
\pgfpathclose%
\pgfusepath{stroke,fill}%
\end{pgfscope}%
\begin{pgfscope}%
\pgfpathrectangle{\pgfqpoint{0.648703in}{0.548769in}}{\pgfqpoint{5.201297in}{3.102590in}}%
\pgfusepath{clip}%
\pgfsetbuttcap%
\pgfsetroundjoin%
\definecolor{currentfill}{rgb}{0.121569,0.466667,0.705882}%
\pgfsetfillcolor{currentfill}%
\pgfsetlinewidth{1.003750pt}%
\definecolor{currentstroke}{rgb}{0.121569,0.466667,0.705882}%
\pgfsetstrokecolor{currentstroke}%
\pgfsetdash{}{0pt}%
\pgfpathmoveto{\pgfqpoint{0.888473in}{0.652358in}}%
\pgfpathcurveto{\pgfqpoint{0.899523in}{0.652358in}}{\pgfqpoint{0.910122in}{0.656748in}}{\pgfqpoint{0.917936in}{0.664562in}}%
\pgfpathcurveto{\pgfqpoint{0.925749in}{0.672375in}}{\pgfqpoint{0.930139in}{0.682974in}}{\pgfqpoint{0.930139in}{0.694024in}}%
\pgfpathcurveto{\pgfqpoint{0.930139in}{0.705074in}}{\pgfqpoint{0.925749in}{0.715673in}}{\pgfqpoint{0.917936in}{0.723487in}}%
\pgfpathcurveto{\pgfqpoint{0.910122in}{0.731301in}}{\pgfqpoint{0.899523in}{0.735691in}}{\pgfqpoint{0.888473in}{0.735691in}}%
\pgfpathcurveto{\pgfqpoint{0.877423in}{0.735691in}}{\pgfqpoint{0.866824in}{0.731301in}}{\pgfqpoint{0.859010in}{0.723487in}}%
\pgfpathcurveto{\pgfqpoint{0.851196in}{0.715673in}}{\pgfqpoint{0.846806in}{0.705074in}}{\pgfqpoint{0.846806in}{0.694024in}}%
\pgfpathcurveto{\pgfqpoint{0.846806in}{0.682974in}}{\pgfqpoint{0.851196in}{0.672375in}}{\pgfqpoint{0.859010in}{0.664562in}}%
\pgfpathcurveto{\pgfqpoint{0.866824in}{0.656748in}}{\pgfqpoint{0.877423in}{0.652358in}}{\pgfqpoint{0.888473in}{0.652358in}}%
\pgfpathclose%
\pgfusepath{stroke,fill}%
\end{pgfscope}%
\begin{pgfscope}%
\pgfpathrectangle{\pgfqpoint{0.648703in}{0.548769in}}{\pgfqpoint{5.201297in}{3.102590in}}%
\pgfusepath{clip}%
\pgfsetbuttcap%
\pgfsetroundjoin%
\definecolor{currentfill}{rgb}{0.121569,0.466667,0.705882}%
\pgfsetfillcolor{currentfill}%
\pgfsetlinewidth{1.003750pt}%
\definecolor{currentstroke}{rgb}{0.121569,0.466667,0.705882}%
\pgfsetstrokecolor{currentstroke}%
\pgfsetdash{}{0pt}%
\pgfpathmoveto{\pgfqpoint{4.160102in}{3.181114in}}%
\pgfpathcurveto{\pgfqpoint{4.171152in}{3.181114in}}{\pgfqpoint{4.181751in}{3.185504in}}{\pgfqpoint{4.189565in}{3.193318in}}%
\pgfpathcurveto{\pgfqpoint{4.197378in}{3.201132in}}{\pgfqpoint{4.201769in}{3.211731in}}{\pgfqpoint{4.201769in}{3.222781in}}%
\pgfpathcurveto{\pgfqpoint{4.201769in}{3.233831in}}{\pgfqpoint{4.197378in}{3.244430in}}{\pgfqpoint{4.189565in}{3.252244in}}%
\pgfpathcurveto{\pgfqpoint{4.181751in}{3.260057in}}{\pgfqpoint{4.171152in}{3.264448in}}{\pgfqpoint{4.160102in}{3.264448in}}%
\pgfpathcurveto{\pgfqpoint{4.149052in}{3.264448in}}{\pgfqpoint{4.138453in}{3.260057in}}{\pgfqpoint{4.130639in}{3.252244in}}%
\pgfpathcurveto{\pgfqpoint{4.122826in}{3.244430in}}{\pgfqpoint{4.118435in}{3.233831in}}{\pgfqpoint{4.118435in}{3.222781in}}%
\pgfpathcurveto{\pgfqpoint{4.118435in}{3.211731in}}{\pgfqpoint{4.122826in}{3.201132in}}{\pgfqpoint{4.130639in}{3.193318in}}%
\pgfpathcurveto{\pgfqpoint{4.138453in}{3.185504in}}{\pgfqpoint{4.149052in}{3.181114in}}{\pgfqpoint{4.160102in}{3.181114in}}%
\pgfpathclose%
\pgfusepath{stroke,fill}%
\end{pgfscope}%
\begin{pgfscope}%
\pgfpathrectangle{\pgfqpoint{0.648703in}{0.548769in}}{\pgfqpoint{5.201297in}{3.102590in}}%
\pgfusepath{clip}%
\pgfsetbuttcap%
\pgfsetroundjoin%
\definecolor{currentfill}{rgb}{1.000000,0.498039,0.054902}%
\pgfsetfillcolor{currentfill}%
\pgfsetlinewidth{1.003750pt}%
\definecolor{currentstroke}{rgb}{1.000000,0.498039,0.054902}%
\pgfsetstrokecolor{currentstroke}%
\pgfsetdash{}{0pt}%
\pgfpathmoveto{\pgfqpoint{4.104316in}{3.189572in}}%
\pgfpathcurveto{\pgfqpoint{4.115366in}{3.189572in}}{\pgfqpoint{4.125965in}{3.193962in}}{\pgfqpoint{4.133778in}{3.201775in}}%
\pgfpathcurveto{\pgfqpoint{4.141592in}{3.209589in}}{\pgfqpoint{4.145982in}{3.220188in}}{\pgfqpoint{4.145982in}{3.231238in}}%
\pgfpathcurveto{\pgfqpoint{4.145982in}{3.242288in}}{\pgfqpoint{4.141592in}{3.252887in}}{\pgfqpoint{4.133778in}{3.260701in}}%
\pgfpathcurveto{\pgfqpoint{4.125965in}{3.268515in}}{\pgfqpoint{4.115366in}{3.272905in}}{\pgfqpoint{4.104316in}{3.272905in}}%
\pgfpathcurveto{\pgfqpoint{4.093265in}{3.272905in}}{\pgfqpoint{4.082666in}{3.268515in}}{\pgfqpoint{4.074853in}{3.260701in}}%
\pgfpathcurveto{\pgfqpoint{4.067039in}{3.252887in}}{\pgfqpoint{4.062649in}{3.242288in}}{\pgfqpoint{4.062649in}{3.231238in}}%
\pgfpathcurveto{\pgfqpoint{4.062649in}{3.220188in}}{\pgfqpoint{4.067039in}{3.209589in}}{\pgfqpoint{4.074853in}{3.201775in}}%
\pgfpathcurveto{\pgfqpoint{4.082666in}{3.193962in}}{\pgfqpoint{4.093265in}{3.189572in}}{\pgfqpoint{4.104316in}{3.189572in}}%
\pgfpathclose%
\pgfusepath{stroke,fill}%
\end{pgfscope}%
\begin{pgfscope}%
\pgfpathrectangle{\pgfqpoint{0.648703in}{0.548769in}}{\pgfqpoint{5.201297in}{3.102590in}}%
\pgfusepath{clip}%
\pgfsetbuttcap%
\pgfsetroundjoin%
\definecolor{currentfill}{rgb}{0.121569,0.466667,0.705882}%
\pgfsetfillcolor{currentfill}%
\pgfsetlinewidth{1.003750pt}%
\definecolor{currentstroke}{rgb}{0.121569,0.466667,0.705882}%
\pgfsetstrokecolor{currentstroke}%
\pgfsetdash{}{0pt}%
\pgfpathmoveto{\pgfqpoint{0.885185in}{0.648129in}}%
\pgfpathcurveto{\pgfqpoint{0.896235in}{0.648129in}}{\pgfqpoint{0.906834in}{0.652519in}}{\pgfqpoint{0.914648in}{0.660333in}}%
\pgfpathcurveto{\pgfqpoint{0.922462in}{0.668146in}}{\pgfqpoint{0.926852in}{0.678745in}}{\pgfqpoint{0.926852in}{0.689796in}}%
\pgfpathcurveto{\pgfqpoint{0.926852in}{0.700846in}}{\pgfqpoint{0.922462in}{0.711445in}}{\pgfqpoint{0.914648in}{0.719258in}}%
\pgfpathcurveto{\pgfqpoint{0.906834in}{0.727072in}}{\pgfqpoint{0.896235in}{0.731462in}}{\pgfqpoint{0.885185in}{0.731462in}}%
\pgfpathcurveto{\pgfqpoint{0.874135in}{0.731462in}}{\pgfqpoint{0.863536in}{0.727072in}}{\pgfqpoint{0.855722in}{0.719258in}}%
\pgfpathcurveto{\pgfqpoint{0.847909in}{0.711445in}}{\pgfqpoint{0.843518in}{0.700846in}}{\pgfqpoint{0.843518in}{0.689796in}}%
\pgfpathcurveto{\pgfqpoint{0.843518in}{0.678745in}}{\pgfqpoint{0.847909in}{0.668146in}}{\pgfqpoint{0.855722in}{0.660333in}}%
\pgfpathcurveto{\pgfqpoint{0.863536in}{0.652519in}}{\pgfqpoint{0.874135in}{0.648129in}}{\pgfqpoint{0.885185in}{0.648129in}}%
\pgfpathclose%
\pgfusepath{stroke,fill}%
\end{pgfscope}%
\begin{pgfscope}%
\pgfpathrectangle{\pgfqpoint{0.648703in}{0.548769in}}{\pgfqpoint{5.201297in}{3.102590in}}%
\pgfusepath{clip}%
\pgfsetbuttcap%
\pgfsetroundjoin%
\definecolor{currentfill}{rgb}{0.121569,0.466667,0.705882}%
\pgfsetfillcolor{currentfill}%
\pgfsetlinewidth{1.003750pt}%
\definecolor{currentstroke}{rgb}{0.121569,0.466667,0.705882}%
\pgfsetstrokecolor{currentstroke}%
\pgfsetdash{}{0pt}%
\pgfpathmoveto{\pgfqpoint{1.290950in}{0.774990in}}%
\pgfpathcurveto{\pgfqpoint{1.302000in}{0.774990in}}{\pgfqpoint{1.312599in}{0.779380in}}{\pgfqpoint{1.320412in}{0.787194in}}%
\pgfpathcurveto{\pgfqpoint{1.328226in}{0.795007in}}{\pgfqpoint{1.332616in}{0.805606in}}{\pgfqpoint{1.332616in}{0.816656in}}%
\pgfpathcurveto{\pgfqpoint{1.332616in}{0.827706in}}{\pgfqpoint{1.328226in}{0.838305in}}{\pgfqpoint{1.320412in}{0.846119in}}%
\pgfpathcurveto{\pgfqpoint{1.312599in}{0.853933in}}{\pgfqpoint{1.302000in}{0.858323in}}{\pgfqpoint{1.290950in}{0.858323in}}%
\pgfpathcurveto{\pgfqpoint{1.279899in}{0.858323in}}{\pgfqpoint{1.269300in}{0.853933in}}{\pgfqpoint{1.261487in}{0.846119in}}%
\pgfpathcurveto{\pgfqpoint{1.253673in}{0.838305in}}{\pgfqpoint{1.249283in}{0.827706in}}{\pgfqpoint{1.249283in}{0.816656in}}%
\pgfpathcurveto{\pgfqpoint{1.249283in}{0.805606in}}{\pgfqpoint{1.253673in}{0.795007in}}{\pgfqpoint{1.261487in}{0.787194in}}%
\pgfpathcurveto{\pgfqpoint{1.269300in}{0.779380in}}{\pgfqpoint{1.279899in}{0.774990in}}{\pgfqpoint{1.290950in}{0.774990in}}%
\pgfpathclose%
\pgfusepath{stroke,fill}%
\end{pgfscope}%
\begin{pgfscope}%
\pgfpathrectangle{\pgfqpoint{0.648703in}{0.548769in}}{\pgfqpoint{5.201297in}{3.102590in}}%
\pgfusepath{clip}%
\pgfsetbuttcap%
\pgfsetroundjoin%
\definecolor{currentfill}{rgb}{0.121569,0.466667,0.705882}%
\pgfsetfillcolor{currentfill}%
\pgfsetlinewidth{1.003750pt}%
\definecolor{currentstroke}{rgb}{0.121569,0.466667,0.705882}%
\pgfsetstrokecolor{currentstroke}%
\pgfsetdash{}{0pt}%
\pgfpathmoveto{\pgfqpoint{1.186639in}{0.783447in}}%
\pgfpathcurveto{\pgfqpoint{1.197689in}{0.783447in}}{\pgfqpoint{1.208288in}{0.787837in}}{\pgfqpoint{1.216102in}{0.795651in}}%
\pgfpathcurveto{\pgfqpoint{1.223915in}{0.803465in}}{\pgfqpoint{1.228305in}{0.814064in}}{\pgfqpoint{1.228305in}{0.825114in}}%
\pgfpathcurveto{\pgfqpoint{1.228305in}{0.836164in}}{\pgfqpoint{1.223915in}{0.846763in}}{\pgfqpoint{1.216102in}{0.854576in}}%
\pgfpathcurveto{\pgfqpoint{1.208288in}{0.862390in}}{\pgfqpoint{1.197689in}{0.866780in}}{\pgfqpoint{1.186639in}{0.866780in}}%
\pgfpathcurveto{\pgfqpoint{1.175589in}{0.866780in}}{\pgfqpoint{1.164990in}{0.862390in}}{\pgfqpoint{1.157176in}{0.854576in}}%
\pgfpathcurveto{\pgfqpoint{1.149362in}{0.846763in}}{\pgfqpoint{1.144972in}{0.836164in}}{\pgfqpoint{1.144972in}{0.825114in}}%
\pgfpathcurveto{\pgfqpoint{1.144972in}{0.814064in}}{\pgfqpoint{1.149362in}{0.803465in}}{\pgfqpoint{1.157176in}{0.795651in}}%
\pgfpathcurveto{\pgfqpoint{1.164990in}{0.787837in}}{\pgfqpoint{1.175589in}{0.783447in}}{\pgfqpoint{1.186639in}{0.783447in}}%
\pgfpathclose%
\pgfusepath{stroke,fill}%
\end{pgfscope}%
\begin{pgfscope}%
\pgfpathrectangle{\pgfqpoint{0.648703in}{0.548769in}}{\pgfqpoint{5.201297in}{3.102590in}}%
\pgfusepath{clip}%
\pgfsetbuttcap%
\pgfsetroundjoin%
\definecolor{currentfill}{rgb}{0.121569,0.466667,0.705882}%
\pgfsetfillcolor{currentfill}%
\pgfsetlinewidth{1.003750pt}%
\definecolor{currentstroke}{rgb}{0.121569,0.466667,0.705882}%
\pgfsetstrokecolor{currentstroke}%
\pgfsetdash{}{0pt}%
\pgfpathmoveto{\pgfqpoint{0.906149in}{0.652358in}}%
\pgfpathcurveto{\pgfqpoint{0.917199in}{0.652358in}}{\pgfqpoint{0.927798in}{0.656748in}}{\pgfqpoint{0.935611in}{0.664562in}}%
\pgfpathcurveto{\pgfqpoint{0.943425in}{0.672375in}}{\pgfqpoint{0.947815in}{0.682974in}}{\pgfqpoint{0.947815in}{0.694024in}}%
\pgfpathcurveto{\pgfqpoint{0.947815in}{0.705074in}}{\pgfqpoint{0.943425in}{0.715673in}}{\pgfqpoint{0.935611in}{0.723487in}}%
\pgfpathcurveto{\pgfqpoint{0.927798in}{0.731301in}}{\pgfqpoint{0.917199in}{0.735691in}}{\pgfqpoint{0.906149in}{0.735691in}}%
\pgfpathcurveto{\pgfqpoint{0.895098in}{0.735691in}}{\pgfqpoint{0.884499in}{0.731301in}}{\pgfqpoint{0.876686in}{0.723487in}}%
\pgfpathcurveto{\pgfqpoint{0.868872in}{0.715673in}}{\pgfqpoint{0.864482in}{0.705074in}}{\pgfqpoint{0.864482in}{0.694024in}}%
\pgfpathcurveto{\pgfqpoint{0.864482in}{0.682974in}}{\pgfqpoint{0.868872in}{0.672375in}}{\pgfqpoint{0.876686in}{0.664562in}}%
\pgfpathcurveto{\pgfqpoint{0.884499in}{0.656748in}}{\pgfqpoint{0.895098in}{0.652358in}}{\pgfqpoint{0.906149in}{0.652358in}}%
\pgfpathclose%
\pgfusepath{stroke,fill}%
\end{pgfscope}%
\begin{pgfscope}%
\pgfpathrectangle{\pgfqpoint{0.648703in}{0.548769in}}{\pgfqpoint{5.201297in}{3.102590in}}%
\pgfusepath{clip}%
\pgfsetbuttcap%
\pgfsetroundjoin%
\definecolor{currentfill}{rgb}{0.121569,0.466667,0.705882}%
\pgfsetfillcolor{currentfill}%
\pgfsetlinewidth{1.003750pt}%
\definecolor{currentstroke}{rgb}{0.121569,0.466667,0.705882}%
\pgfsetstrokecolor{currentstroke}%
\pgfsetdash{}{0pt}%
\pgfpathmoveto{\pgfqpoint{0.885165in}{0.648129in}}%
\pgfpathcurveto{\pgfqpoint{0.896216in}{0.648129in}}{\pgfqpoint{0.906815in}{0.652519in}}{\pgfqpoint{0.914628in}{0.660333in}}%
\pgfpathcurveto{\pgfqpoint{0.922442in}{0.668146in}}{\pgfqpoint{0.926832in}{0.678745in}}{\pgfqpoint{0.926832in}{0.689796in}}%
\pgfpathcurveto{\pgfqpoint{0.926832in}{0.700846in}}{\pgfqpoint{0.922442in}{0.711445in}}{\pgfqpoint{0.914628in}{0.719258in}}%
\pgfpathcurveto{\pgfqpoint{0.906815in}{0.727072in}}{\pgfqpoint{0.896216in}{0.731462in}}{\pgfqpoint{0.885165in}{0.731462in}}%
\pgfpathcurveto{\pgfqpoint{0.874115in}{0.731462in}}{\pgfqpoint{0.863516in}{0.727072in}}{\pgfqpoint{0.855703in}{0.719258in}}%
\pgfpathcurveto{\pgfqpoint{0.847889in}{0.711445in}}{\pgfqpoint{0.843499in}{0.700846in}}{\pgfqpoint{0.843499in}{0.689796in}}%
\pgfpathcurveto{\pgfqpoint{0.843499in}{0.678745in}}{\pgfqpoint{0.847889in}{0.668146in}}{\pgfqpoint{0.855703in}{0.660333in}}%
\pgfpathcurveto{\pgfqpoint{0.863516in}{0.652519in}}{\pgfqpoint{0.874115in}{0.648129in}}{\pgfqpoint{0.885165in}{0.648129in}}%
\pgfpathclose%
\pgfusepath{stroke,fill}%
\end{pgfscope}%
\begin{pgfscope}%
\pgfpathrectangle{\pgfqpoint{0.648703in}{0.548769in}}{\pgfqpoint{5.201297in}{3.102590in}}%
\pgfusepath{clip}%
\pgfsetbuttcap%
\pgfsetroundjoin%
\definecolor{currentfill}{rgb}{0.121569,0.466667,0.705882}%
\pgfsetfillcolor{currentfill}%
\pgfsetlinewidth{1.003750pt}%
\definecolor{currentstroke}{rgb}{0.121569,0.466667,0.705882}%
\pgfsetstrokecolor{currentstroke}%
\pgfsetdash{}{0pt}%
\pgfpathmoveto{\pgfqpoint{0.885153in}{0.648129in}}%
\pgfpathcurveto{\pgfqpoint{0.896203in}{0.648129in}}{\pgfqpoint{0.906802in}{0.652519in}}{\pgfqpoint{0.914616in}{0.660333in}}%
\pgfpathcurveto{\pgfqpoint{0.922429in}{0.668146in}}{\pgfqpoint{0.926820in}{0.678745in}}{\pgfqpoint{0.926820in}{0.689796in}}%
\pgfpathcurveto{\pgfqpoint{0.926820in}{0.700846in}}{\pgfqpoint{0.922429in}{0.711445in}}{\pgfqpoint{0.914616in}{0.719258in}}%
\pgfpathcurveto{\pgfqpoint{0.906802in}{0.727072in}}{\pgfqpoint{0.896203in}{0.731462in}}{\pgfqpoint{0.885153in}{0.731462in}}%
\pgfpathcurveto{\pgfqpoint{0.874103in}{0.731462in}}{\pgfqpoint{0.863504in}{0.727072in}}{\pgfqpoint{0.855690in}{0.719258in}}%
\pgfpathcurveto{\pgfqpoint{0.847876in}{0.711445in}}{\pgfqpoint{0.843486in}{0.700846in}}{\pgfqpoint{0.843486in}{0.689796in}}%
\pgfpathcurveto{\pgfqpoint{0.843486in}{0.678745in}}{\pgfqpoint{0.847876in}{0.668146in}}{\pgfqpoint{0.855690in}{0.660333in}}%
\pgfpathcurveto{\pgfqpoint{0.863504in}{0.652519in}}{\pgfqpoint{0.874103in}{0.648129in}}{\pgfqpoint{0.885153in}{0.648129in}}%
\pgfpathclose%
\pgfusepath{stroke,fill}%
\end{pgfscope}%
\begin{pgfscope}%
\pgfpathrectangle{\pgfqpoint{0.648703in}{0.548769in}}{\pgfqpoint{5.201297in}{3.102590in}}%
\pgfusepath{clip}%
\pgfsetbuttcap%
\pgfsetroundjoin%
\definecolor{currentfill}{rgb}{1.000000,0.498039,0.054902}%
\pgfsetfillcolor{currentfill}%
\pgfsetlinewidth{1.003750pt}%
\definecolor{currentstroke}{rgb}{1.000000,0.498039,0.054902}%
\pgfsetstrokecolor{currentstroke}%
\pgfsetdash{}{0pt}%
\pgfpathmoveto{\pgfqpoint{4.895053in}{3.193800in}}%
\pgfpathcurveto{\pgfqpoint{4.906103in}{3.193800in}}{\pgfqpoint{4.916702in}{3.198191in}}{\pgfqpoint{4.924516in}{3.206004in}}%
\pgfpathcurveto{\pgfqpoint{4.932330in}{3.213818in}}{\pgfqpoint{4.936720in}{3.224417in}}{\pgfqpoint{4.936720in}{3.235467in}}%
\pgfpathcurveto{\pgfqpoint{4.936720in}{3.246517in}}{\pgfqpoint{4.932330in}{3.257116in}}{\pgfqpoint{4.924516in}{3.264930in}}%
\pgfpathcurveto{\pgfqpoint{4.916702in}{3.272743in}}{\pgfqpoint{4.906103in}{3.277134in}}{\pgfqpoint{4.895053in}{3.277134in}}%
\pgfpathcurveto{\pgfqpoint{4.884003in}{3.277134in}}{\pgfqpoint{4.873404in}{3.272743in}}{\pgfqpoint{4.865590in}{3.264930in}}%
\pgfpathcurveto{\pgfqpoint{4.857777in}{3.257116in}}{\pgfqpoint{4.853387in}{3.246517in}}{\pgfqpoint{4.853387in}{3.235467in}}%
\pgfpathcurveto{\pgfqpoint{4.853387in}{3.224417in}}{\pgfqpoint{4.857777in}{3.213818in}}{\pgfqpoint{4.865590in}{3.206004in}}%
\pgfpathcurveto{\pgfqpoint{4.873404in}{3.198191in}}{\pgfqpoint{4.884003in}{3.193800in}}{\pgfqpoint{4.895053in}{3.193800in}}%
\pgfpathclose%
\pgfusepath{stroke,fill}%
\end{pgfscope}%
\begin{pgfscope}%
\pgfpathrectangle{\pgfqpoint{0.648703in}{0.548769in}}{\pgfqpoint{5.201297in}{3.102590in}}%
\pgfusepath{clip}%
\pgfsetbuttcap%
\pgfsetroundjoin%
\definecolor{currentfill}{rgb}{1.000000,0.498039,0.054902}%
\pgfsetfillcolor{currentfill}%
\pgfsetlinewidth{1.003750pt}%
\definecolor{currentstroke}{rgb}{1.000000,0.498039,0.054902}%
\pgfsetstrokecolor{currentstroke}%
\pgfsetdash{}{0pt}%
\pgfpathmoveto{\pgfqpoint{5.199508in}{3.231859in}}%
\pgfpathcurveto{\pgfqpoint{5.210558in}{3.231859in}}{\pgfqpoint{5.221158in}{3.236249in}}{\pgfqpoint{5.228971in}{3.244062in}}%
\pgfpathcurveto{\pgfqpoint{5.236785in}{3.251876in}}{\pgfqpoint{5.241175in}{3.262475in}}{\pgfqpoint{5.241175in}{3.273525in}}%
\pgfpathcurveto{\pgfqpoint{5.241175in}{3.284575in}}{\pgfqpoint{5.236785in}{3.295174in}}{\pgfqpoint{5.228971in}{3.302988in}}%
\pgfpathcurveto{\pgfqpoint{5.221158in}{3.310802in}}{\pgfqpoint{5.210558in}{3.315192in}}{\pgfqpoint{5.199508in}{3.315192in}}%
\pgfpathcurveto{\pgfqpoint{5.188458in}{3.315192in}}{\pgfqpoint{5.177859in}{3.310802in}}{\pgfqpoint{5.170046in}{3.302988in}}%
\pgfpathcurveto{\pgfqpoint{5.162232in}{3.295174in}}{\pgfqpoint{5.157842in}{3.284575in}}{\pgfqpoint{5.157842in}{3.273525in}}%
\pgfpathcurveto{\pgfqpoint{5.157842in}{3.262475in}}{\pgfqpoint{5.162232in}{3.251876in}}{\pgfqpoint{5.170046in}{3.244062in}}%
\pgfpathcurveto{\pgfqpoint{5.177859in}{3.236249in}}{\pgfqpoint{5.188458in}{3.231859in}}{\pgfqpoint{5.199508in}{3.231859in}}%
\pgfpathclose%
\pgfusepath{stroke,fill}%
\end{pgfscope}%
\begin{pgfscope}%
\pgfpathrectangle{\pgfqpoint{0.648703in}{0.548769in}}{\pgfqpoint{5.201297in}{3.102590in}}%
\pgfusepath{clip}%
\pgfsetbuttcap%
\pgfsetroundjoin%
\definecolor{currentfill}{rgb}{0.121569,0.466667,0.705882}%
\pgfsetfillcolor{currentfill}%
\pgfsetlinewidth{1.003750pt}%
\definecolor{currentstroke}{rgb}{0.121569,0.466667,0.705882}%
\pgfsetstrokecolor{currentstroke}%
\pgfsetdash{}{0pt}%
\pgfpathmoveto{\pgfqpoint{1.253129in}{0.758075in}}%
\pgfpathcurveto{\pgfqpoint{1.264179in}{0.758075in}}{\pgfqpoint{1.274778in}{0.762465in}}{\pgfqpoint{1.282591in}{0.770279in}}%
\pgfpathcurveto{\pgfqpoint{1.290405in}{0.778092in}}{\pgfqpoint{1.294795in}{0.788691in}}{\pgfqpoint{1.294795in}{0.799742in}}%
\pgfpathcurveto{\pgfqpoint{1.294795in}{0.810792in}}{\pgfqpoint{1.290405in}{0.821391in}}{\pgfqpoint{1.282591in}{0.829204in}}%
\pgfpathcurveto{\pgfqpoint{1.274778in}{0.837018in}}{\pgfqpoint{1.264179in}{0.841408in}}{\pgfqpoint{1.253129in}{0.841408in}}%
\pgfpathcurveto{\pgfqpoint{1.242078in}{0.841408in}}{\pgfqpoint{1.231479in}{0.837018in}}{\pgfqpoint{1.223666in}{0.829204in}}%
\pgfpathcurveto{\pgfqpoint{1.215852in}{0.821391in}}{\pgfqpoint{1.211462in}{0.810792in}}{\pgfqpoint{1.211462in}{0.799742in}}%
\pgfpathcurveto{\pgfqpoint{1.211462in}{0.788691in}}{\pgfqpoint{1.215852in}{0.778092in}}{\pgfqpoint{1.223666in}{0.770279in}}%
\pgfpathcurveto{\pgfqpoint{1.231479in}{0.762465in}}{\pgfqpoint{1.242078in}{0.758075in}}{\pgfqpoint{1.253129in}{0.758075in}}%
\pgfpathclose%
\pgfusepath{stroke,fill}%
\end{pgfscope}%
\begin{pgfscope}%
\pgfpathrectangle{\pgfqpoint{0.648703in}{0.548769in}}{\pgfqpoint{5.201297in}{3.102590in}}%
\pgfusepath{clip}%
\pgfsetbuttcap%
\pgfsetroundjoin%
\definecolor{currentfill}{rgb}{1.000000,0.498039,0.054902}%
\pgfsetfillcolor{currentfill}%
\pgfsetlinewidth{1.003750pt}%
\definecolor{currentstroke}{rgb}{1.000000,0.498039,0.054902}%
\pgfsetstrokecolor{currentstroke}%
\pgfsetdash{}{0pt}%
\pgfpathmoveto{\pgfqpoint{2.937889in}{3.185343in}}%
\pgfpathcurveto{\pgfqpoint{2.948939in}{3.185343in}}{\pgfqpoint{2.959538in}{3.189733in}}{\pgfqpoint{2.967352in}{3.197547in}}%
\pgfpathcurveto{\pgfqpoint{2.975166in}{3.205360in}}{\pgfqpoint{2.979556in}{3.215959in}}{\pgfqpoint{2.979556in}{3.227010in}}%
\pgfpathcurveto{\pgfqpoint{2.979556in}{3.238060in}}{\pgfqpoint{2.975166in}{3.248659in}}{\pgfqpoint{2.967352in}{3.256472in}}%
\pgfpathcurveto{\pgfqpoint{2.959538in}{3.264286in}}{\pgfqpoint{2.948939in}{3.268676in}}{\pgfqpoint{2.937889in}{3.268676in}}%
\pgfpathcurveto{\pgfqpoint{2.926839in}{3.268676in}}{\pgfqpoint{2.916240in}{3.264286in}}{\pgfqpoint{2.908427in}{3.256472in}}%
\pgfpathcurveto{\pgfqpoint{2.900613in}{3.248659in}}{\pgfqpoint{2.896223in}{3.238060in}}{\pgfqpoint{2.896223in}{3.227010in}}%
\pgfpathcurveto{\pgfqpoint{2.896223in}{3.215959in}}{\pgfqpoint{2.900613in}{3.205360in}}{\pgfqpoint{2.908427in}{3.197547in}}%
\pgfpathcurveto{\pgfqpoint{2.916240in}{3.189733in}}{\pgfqpoint{2.926839in}{3.185343in}}{\pgfqpoint{2.937889in}{3.185343in}}%
\pgfpathclose%
\pgfusepath{stroke,fill}%
\end{pgfscope}%
\begin{pgfscope}%
\pgfpathrectangle{\pgfqpoint{0.648703in}{0.548769in}}{\pgfqpoint{5.201297in}{3.102590in}}%
\pgfusepath{clip}%
\pgfsetbuttcap%
\pgfsetroundjoin%
\definecolor{currentfill}{rgb}{1.000000,0.498039,0.054902}%
\pgfsetfillcolor{currentfill}%
\pgfsetlinewidth{1.003750pt}%
\definecolor{currentstroke}{rgb}{1.000000,0.498039,0.054902}%
\pgfsetstrokecolor{currentstroke}%
\pgfsetdash{}{0pt}%
\pgfpathmoveto{\pgfqpoint{4.455856in}{3.202258in}}%
\pgfpathcurveto{\pgfqpoint{4.466906in}{3.202258in}}{\pgfqpoint{4.477505in}{3.206648in}}{\pgfqpoint{4.485319in}{3.214462in}}%
\pgfpathcurveto{\pgfqpoint{4.493132in}{3.222275in}}{\pgfqpoint{4.497523in}{3.232874in}}{\pgfqpoint{4.497523in}{3.243924in}}%
\pgfpathcurveto{\pgfqpoint{4.497523in}{3.254974in}}{\pgfqpoint{4.493132in}{3.265573in}}{\pgfqpoint{4.485319in}{3.273387in}}%
\pgfpathcurveto{\pgfqpoint{4.477505in}{3.281201in}}{\pgfqpoint{4.466906in}{3.285591in}}{\pgfqpoint{4.455856in}{3.285591in}}%
\pgfpathcurveto{\pgfqpoint{4.444806in}{3.285591in}}{\pgfqpoint{4.434207in}{3.281201in}}{\pgfqpoint{4.426393in}{3.273387in}}%
\pgfpathcurveto{\pgfqpoint{4.418580in}{3.265573in}}{\pgfqpoint{4.414189in}{3.254974in}}{\pgfqpoint{4.414189in}{3.243924in}}%
\pgfpathcurveto{\pgfqpoint{4.414189in}{3.232874in}}{\pgfqpoint{4.418580in}{3.222275in}}{\pgfqpoint{4.426393in}{3.214462in}}%
\pgfpathcurveto{\pgfqpoint{4.434207in}{3.206648in}}{\pgfqpoint{4.444806in}{3.202258in}}{\pgfqpoint{4.455856in}{3.202258in}}%
\pgfpathclose%
\pgfusepath{stroke,fill}%
\end{pgfscope}%
\begin{pgfscope}%
\pgfpathrectangle{\pgfqpoint{0.648703in}{0.548769in}}{\pgfqpoint{5.201297in}{3.102590in}}%
\pgfusepath{clip}%
\pgfsetbuttcap%
\pgfsetroundjoin%
\definecolor{currentfill}{rgb}{0.121569,0.466667,0.705882}%
\pgfsetfillcolor{currentfill}%
\pgfsetlinewidth{1.003750pt}%
\definecolor{currentstroke}{rgb}{0.121569,0.466667,0.705882}%
\pgfsetstrokecolor{currentstroke}%
\pgfsetdash{}{0pt}%
\pgfpathmoveto{\pgfqpoint{0.885156in}{0.648129in}}%
\pgfpathcurveto{\pgfqpoint{0.896206in}{0.648129in}}{\pgfqpoint{0.906805in}{0.652519in}}{\pgfqpoint{0.914619in}{0.660333in}}%
\pgfpathcurveto{\pgfqpoint{0.922432in}{0.668146in}}{\pgfqpoint{0.926823in}{0.678745in}}{\pgfqpoint{0.926823in}{0.689796in}}%
\pgfpathcurveto{\pgfqpoint{0.926823in}{0.700846in}}{\pgfqpoint{0.922432in}{0.711445in}}{\pgfqpoint{0.914619in}{0.719258in}}%
\pgfpathcurveto{\pgfqpoint{0.906805in}{0.727072in}}{\pgfqpoint{0.896206in}{0.731462in}}{\pgfqpoint{0.885156in}{0.731462in}}%
\pgfpathcurveto{\pgfqpoint{0.874106in}{0.731462in}}{\pgfqpoint{0.863507in}{0.727072in}}{\pgfqpoint{0.855693in}{0.719258in}}%
\pgfpathcurveto{\pgfqpoint{0.847879in}{0.711445in}}{\pgfqpoint{0.843489in}{0.700846in}}{\pgfqpoint{0.843489in}{0.689796in}}%
\pgfpathcurveto{\pgfqpoint{0.843489in}{0.678745in}}{\pgfqpoint{0.847879in}{0.668146in}}{\pgfqpoint{0.855693in}{0.660333in}}%
\pgfpathcurveto{\pgfqpoint{0.863507in}{0.652519in}}{\pgfqpoint{0.874106in}{0.648129in}}{\pgfqpoint{0.885156in}{0.648129in}}%
\pgfpathclose%
\pgfusepath{stroke,fill}%
\end{pgfscope}%
\begin{pgfscope}%
\pgfpathrectangle{\pgfqpoint{0.648703in}{0.548769in}}{\pgfqpoint{5.201297in}{3.102590in}}%
\pgfusepath{clip}%
\pgfsetbuttcap%
\pgfsetroundjoin%
\definecolor{currentfill}{rgb}{1.000000,0.498039,0.054902}%
\pgfsetfillcolor{currentfill}%
\pgfsetlinewidth{1.003750pt}%
\definecolor{currentstroke}{rgb}{1.000000,0.498039,0.054902}%
\pgfsetstrokecolor{currentstroke}%
\pgfsetdash{}{0pt}%
\pgfpathmoveto{\pgfqpoint{4.458495in}{3.206486in}}%
\pgfpathcurveto{\pgfqpoint{4.469545in}{3.206486in}}{\pgfqpoint{4.480144in}{3.210877in}}{\pgfqpoint{4.487958in}{3.218690in}}%
\pgfpathcurveto{\pgfqpoint{4.495771in}{3.226504in}}{\pgfqpoint{4.500162in}{3.237103in}}{\pgfqpoint{4.500162in}{3.248153in}}%
\pgfpathcurveto{\pgfqpoint{4.500162in}{3.259203in}}{\pgfqpoint{4.495771in}{3.269802in}}{\pgfqpoint{4.487958in}{3.277616in}}%
\pgfpathcurveto{\pgfqpoint{4.480144in}{3.285429in}}{\pgfqpoint{4.469545in}{3.289820in}}{\pgfqpoint{4.458495in}{3.289820in}}%
\pgfpathcurveto{\pgfqpoint{4.447445in}{3.289820in}}{\pgfqpoint{4.436846in}{3.285429in}}{\pgfqpoint{4.429032in}{3.277616in}}%
\pgfpathcurveto{\pgfqpoint{4.421219in}{3.269802in}}{\pgfqpoint{4.416828in}{3.259203in}}{\pgfqpoint{4.416828in}{3.248153in}}%
\pgfpathcurveto{\pgfqpoint{4.416828in}{3.237103in}}{\pgfqpoint{4.421219in}{3.226504in}}{\pgfqpoint{4.429032in}{3.218690in}}%
\pgfpathcurveto{\pgfqpoint{4.436846in}{3.210877in}}{\pgfqpoint{4.447445in}{3.206486in}}{\pgfqpoint{4.458495in}{3.206486in}}%
\pgfpathclose%
\pgfusepath{stroke,fill}%
\end{pgfscope}%
\begin{pgfscope}%
\pgfpathrectangle{\pgfqpoint{0.648703in}{0.548769in}}{\pgfqpoint{5.201297in}{3.102590in}}%
\pgfusepath{clip}%
\pgfsetbuttcap%
\pgfsetroundjoin%
\definecolor{currentfill}{rgb}{0.121569,0.466667,0.705882}%
\pgfsetfillcolor{currentfill}%
\pgfsetlinewidth{1.003750pt}%
\definecolor{currentstroke}{rgb}{0.121569,0.466667,0.705882}%
\pgfsetstrokecolor{currentstroke}%
\pgfsetdash{}{0pt}%
\pgfpathmoveto{\pgfqpoint{0.885153in}{0.648129in}}%
\pgfpathcurveto{\pgfqpoint{0.896203in}{0.648129in}}{\pgfqpoint{0.906802in}{0.652519in}}{\pgfqpoint{0.914616in}{0.660333in}}%
\pgfpathcurveto{\pgfqpoint{0.922429in}{0.668146in}}{\pgfqpoint{0.926820in}{0.678745in}}{\pgfqpoint{0.926820in}{0.689796in}}%
\pgfpathcurveto{\pgfqpoint{0.926820in}{0.700846in}}{\pgfqpoint{0.922429in}{0.711445in}}{\pgfqpoint{0.914616in}{0.719258in}}%
\pgfpathcurveto{\pgfqpoint{0.906802in}{0.727072in}}{\pgfqpoint{0.896203in}{0.731462in}}{\pgfqpoint{0.885153in}{0.731462in}}%
\pgfpathcurveto{\pgfqpoint{0.874103in}{0.731462in}}{\pgfqpoint{0.863504in}{0.727072in}}{\pgfqpoint{0.855690in}{0.719258in}}%
\pgfpathcurveto{\pgfqpoint{0.847876in}{0.711445in}}{\pgfqpoint{0.843486in}{0.700846in}}{\pgfqpoint{0.843486in}{0.689796in}}%
\pgfpathcurveto{\pgfqpoint{0.843486in}{0.678745in}}{\pgfqpoint{0.847876in}{0.668146in}}{\pgfqpoint{0.855690in}{0.660333in}}%
\pgfpathcurveto{\pgfqpoint{0.863504in}{0.652519in}}{\pgfqpoint{0.874103in}{0.648129in}}{\pgfqpoint{0.885153in}{0.648129in}}%
\pgfpathclose%
\pgfusepath{stroke,fill}%
\end{pgfscope}%
\begin{pgfscope}%
\pgfpathrectangle{\pgfqpoint{0.648703in}{0.548769in}}{\pgfqpoint{5.201297in}{3.102590in}}%
\pgfusepath{clip}%
\pgfsetbuttcap%
\pgfsetroundjoin%
\definecolor{currentfill}{rgb}{0.121569,0.466667,0.705882}%
\pgfsetfillcolor{currentfill}%
\pgfsetlinewidth{1.003750pt}%
\definecolor{currentstroke}{rgb}{0.121569,0.466667,0.705882}%
\pgfsetstrokecolor{currentstroke}%
\pgfsetdash{}{0pt}%
\pgfpathmoveto{\pgfqpoint{0.885167in}{0.648129in}}%
\pgfpathcurveto{\pgfqpoint{0.896217in}{0.648129in}}{\pgfqpoint{0.906816in}{0.652519in}}{\pgfqpoint{0.914630in}{0.660333in}}%
\pgfpathcurveto{\pgfqpoint{0.922444in}{0.668146in}}{\pgfqpoint{0.926834in}{0.678745in}}{\pgfqpoint{0.926834in}{0.689796in}}%
\pgfpathcurveto{\pgfqpoint{0.926834in}{0.700846in}}{\pgfqpoint{0.922444in}{0.711445in}}{\pgfqpoint{0.914630in}{0.719258in}}%
\pgfpathcurveto{\pgfqpoint{0.906816in}{0.727072in}}{\pgfqpoint{0.896217in}{0.731462in}}{\pgfqpoint{0.885167in}{0.731462in}}%
\pgfpathcurveto{\pgfqpoint{0.874117in}{0.731462in}}{\pgfqpoint{0.863518in}{0.727072in}}{\pgfqpoint{0.855704in}{0.719258in}}%
\pgfpathcurveto{\pgfqpoint{0.847891in}{0.711445in}}{\pgfqpoint{0.843501in}{0.700846in}}{\pgfqpoint{0.843501in}{0.689796in}}%
\pgfpathcurveto{\pgfqpoint{0.843501in}{0.678745in}}{\pgfqpoint{0.847891in}{0.668146in}}{\pgfqpoint{0.855704in}{0.660333in}}%
\pgfpathcurveto{\pgfqpoint{0.863518in}{0.652519in}}{\pgfqpoint{0.874117in}{0.648129in}}{\pgfqpoint{0.885167in}{0.648129in}}%
\pgfpathclose%
\pgfusepath{stroke,fill}%
\end{pgfscope}%
\begin{pgfscope}%
\pgfpathrectangle{\pgfqpoint{0.648703in}{0.548769in}}{\pgfqpoint{5.201297in}{3.102590in}}%
\pgfusepath{clip}%
\pgfsetbuttcap%
\pgfsetroundjoin%
\definecolor{currentfill}{rgb}{0.121569,0.466667,0.705882}%
\pgfsetfillcolor{currentfill}%
\pgfsetlinewidth{1.003750pt}%
\definecolor{currentstroke}{rgb}{0.121569,0.466667,0.705882}%
\pgfsetstrokecolor{currentstroke}%
\pgfsetdash{}{0pt}%
\pgfpathmoveto{\pgfqpoint{1.295113in}{0.817277in}}%
\pgfpathcurveto{\pgfqpoint{1.306163in}{0.817277in}}{\pgfqpoint{1.316762in}{0.821667in}}{\pgfqpoint{1.324576in}{0.829480in}}%
\pgfpathcurveto{\pgfqpoint{1.332389in}{0.837294in}}{\pgfqpoint{1.336780in}{0.847893in}}{\pgfqpoint{1.336780in}{0.858943in}}%
\pgfpathcurveto{\pgfqpoint{1.336780in}{0.869993in}}{\pgfqpoint{1.332389in}{0.880592in}}{\pgfqpoint{1.324576in}{0.888406in}}%
\pgfpathcurveto{\pgfqpoint{1.316762in}{0.896220in}}{\pgfqpoint{1.306163in}{0.900610in}}{\pgfqpoint{1.295113in}{0.900610in}}%
\pgfpathcurveto{\pgfqpoint{1.284063in}{0.900610in}}{\pgfqpoint{1.273464in}{0.896220in}}{\pgfqpoint{1.265650in}{0.888406in}}%
\pgfpathcurveto{\pgfqpoint{1.257836in}{0.880592in}}{\pgfqpoint{1.253446in}{0.869993in}}{\pgfqpoint{1.253446in}{0.858943in}}%
\pgfpathcurveto{\pgfqpoint{1.253446in}{0.847893in}}{\pgfqpoint{1.257836in}{0.837294in}}{\pgfqpoint{1.265650in}{0.829480in}}%
\pgfpathcurveto{\pgfqpoint{1.273464in}{0.821667in}}{\pgfqpoint{1.284063in}{0.817277in}}{\pgfqpoint{1.295113in}{0.817277in}}%
\pgfpathclose%
\pgfusepath{stroke,fill}%
\end{pgfscope}%
\begin{pgfscope}%
\pgfpathrectangle{\pgfqpoint{0.648703in}{0.548769in}}{\pgfqpoint{5.201297in}{3.102590in}}%
\pgfusepath{clip}%
\pgfsetbuttcap%
\pgfsetroundjoin%
\definecolor{currentfill}{rgb}{1.000000,0.498039,0.054902}%
\pgfsetfillcolor{currentfill}%
\pgfsetlinewidth{1.003750pt}%
\definecolor{currentstroke}{rgb}{1.000000,0.498039,0.054902}%
\pgfsetstrokecolor{currentstroke}%
\pgfsetdash{}{0pt}%
\pgfpathmoveto{\pgfqpoint{3.892902in}{3.189572in}}%
\pgfpathcurveto{\pgfqpoint{3.903952in}{3.189572in}}{\pgfqpoint{3.914551in}{3.193962in}}{\pgfqpoint{3.922365in}{3.201775in}}%
\pgfpathcurveto{\pgfqpoint{3.930178in}{3.209589in}}{\pgfqpoint{3.934568in}{3.220188in}}{\pgfqpoint{3.934568in}{3.231238in}}%
\pgfpathcurveto{\pgfqpoint{3.934568in}{3.242288in}}{\pgfqpoint{3.930178in}{3.252887in}}{\pgfqpoint{3.922365in}{3.260701in}}%
\pgfpathcurveto{\pgfqpoint{3.914551in}{3.268515in}}{\pgfqpoint{3.903952in}{3.272905in}}{\pgfqpoint{3.892902in}{3.272905in}}%
\pgfpathcurveto{\pgfqpoint{3.881852in}{3.272905in}}{\pgfqpoint{3.871253in}{3.268515in}}{\pgfqpoint{3.863439in}{3.260701in}}%
\pgfpathcurveto{\pgfqpoint{3.855625in}{3.252887in}}{\pgfqpoint{3.851235in}{3.242288in}}{\pgfqpoint{3.851235in}{3.231238in}}%
\pgfpathcurveto{\pgfqpoint{3.851235in}{3.220188in}}{\pgfqpoint{3.855625in}{3.209589in}}{\pgfqpoint{3.863439in}{3.201775in}}%
\pgfpathcurveto{\pgfqpoint{3.871253in}{3.193962in}}{\pgfqpoint{3.881852in}{3.189572in}}{\pgfqpoint{3.892902in}{3.189572in}}%
\pgfpathclose%
\pgfusepath{stroke,fill}%
\end{pgfscope}%
\begin{pgfscope}%
\pgfpathrectangle{\pgfqpoint{0.648703in}{0.548769in}}{\pgfqpoint{5.201297in}{3.102590in}}%
\pgfusepath{clip}%
\pgfsetbuttcap%
\pgfsetroundjoin%
\definecolor{currentfill}{rgb}{1.000000,0.498039,0.054902}%
\pgfsetfillcolor{currentfill}%
\pgfsetlinewidth{1.003750pt}%
\definecolor{currentstroke}{rgb}{1.000000,0.498039,0.054902}%
\pgfsetstrokecolor{currentstroke}%
\pgfsetdash{}{0pt}%
\pgfpathmoveto{\pgfqpoint{2.573522in}{3.214944in}}%
\pgfpathcurveto{\pgfqpoint{2.584572in}{3.214944in}}{\pgfqpoint{2.595171in}{3.219334in}}{\pgfqpoint{2.602985in}{3.227148in}}%
\pgfpathcurveto{\pgfqpoint{2.610798in}{3.234961in}}{\pgfqpoint{2.615189in}{3.245560in}}{\pgfqpoint{2.615189in}{3.256610in}}%
\pgfpathcurveto{\pgfqpoint{2.615189in}{3.267661in}}{\pgfqpoint{2.610798in}{3.278260in}}{\pgfqpoint{2.602985in}{3.286073in}}%
\pgfpathcurveto{\pgfqpoint{2.595171in}{3.293887in}}{\pgfqpoint{2.584572in}{3.298277in}}{\pgfqpoint{2.573522in}{3.298277in}}%
\pgfpathcurveto{\pgfqpoint{2.562472in}{3.298277in}}{\pgfqpoint{2.551873in}{3.293887in}}{\pgfqpoint{2.544059in}{3.286073in}}%
\pgfpathcurveto{\pgfqpoint{2.536245in}{3.278260in}}{\pgfqpoint{2.531855in}{3.267661in}}{\pgfqpoint{2.531855in}{3.256610in}}%
\pgfpathcurveto{\pgfqpoint{2.531855in}{3.245560in}}{\pgfqpoint{2.536245in}{3.234961in}}{\pgfqpoint{2.544059in}{3.227148in}}%
\pgfpathcurveto{\pgfqpoint{2.551873in}{3.219334in}}{\pgfqpoint{2.562472in}{3.214944in}}{\pgfqpoint{2.573522in}{3.214944in}}%
\pgfpathclose%
\pgfusepath{stroke,fill}%
\end{pgfscope}%
\begin{pgfscope}%
\pgfpathrectangle{\pgfqpoint{0.648703in}{0.548769in}}{\pgfqpoint{5.201297in}{3.102590in}}%
\pgfusepath{clip}%
\pgfsetbuttcap%
\pgfsetroundjoin%
\definecolor{currentfill}{rgb}{0.121569,0.466667,0.705882}%
\pgfsetfillcolor{currentfill}%
\pgfsetlinewidth{1.003750pt}%
\definecolor{currentstroke}{rgb}{0.121569,0.466667,0.705882}%
\pgfsetstrokecolor{currentstroke}%
\pgfsetdash{}{0pt}%
\pgfpathmoveto{\pgfqpoint{0.885144in}{0.648129in}}%
\pgfpathcurveto{\pgfqpoint{0.896194in}{0.648129in}}{\pgfqpoint{0.906793in}{0.652519in}}{\pgfqpoint{0.914607in}{0.660333in}}%
\pgfpathcurveto{\pgfqpoint{0.922420in}{0.668146in}}{\pgfqpoint{0.926811in}{0.678745in}}{\pgfqpoint{0.926811in}{0.689796in}}%
\pgfpathcurveto{\pgfqpoint{0.926811in}{0.700846in}}{\pgfqpoint{0.922420in}{0.711445in}}{\pgfqpoint{0.914607in}{0.719258in}}%
\pgfpathcurveto{\pgfqpoint{0.906793in}{0.727072in}}{\pgfqpoint{0.896194in}{0.731462in}}{\pgfqpoint{0.885144in}{0.731462in}}%
\pgfpathcurveto{\pgfqpoint{0.874094in}{0.731462in}}{\pgfqpoint{0.863495in}{0.727072in}}{\pgfqpoint{0.855681in}{0.719258in}}%
\pgfpathcurveto{\pgfqpoint{0.847868in}{0.711445in}}{\pgfqpoint{0.843477in}{0.700846in}}{\pgfqpoint{0.843477in}{0.689796in}}%
\pgfpathcurveto{\pgfqpoint{0.843477in}{0.678745in}}{\pgfqpoint{0.847868in}{0.668146in}}{\pgfqpoint{0.855681in}{0.660333in}}%
\pgfpathcurveto{\pgfqpoint{0.863495in}{0.652519in}}{\pgfqpoint{0.874094in}{0.648129in}}{\pgfqpoint{0.885144in}{0.648129in}}%
\pgfpathclose%
\pgfusepath{stroke,fill}%
\end{pgfscope}%
\begin{pgfscope}%
\pgfpathrectangle{\pgfqpoint{0.648703in}{0.548769in}}{\pgfqpoint{5.201297in}{3.102590in}}%
\pgfusepath{clip}%
\pgfsetbuttcap%
\pgfsetroundjoin%
\definecolor{currentfill}{rgb}{0.121569,0.466667,0.705882}%
\pgfsetfillcolor{currentfill}%
\pgfsetlinewidth{1.003750pt}%
\definecolor{currentstroke}{rgb}{0.121569,0.466667,0.705882}%
\pgfsetstrokecolor{currentstroke}%
\pgfsetdash{}{0pt}%
\pgfpathmoveto{\pgfqpoint{1.186633in}{0.758075in}}%
\pgfpathcurveto{\pgfqpoint{1.197683in}{0.758075in}}{\pgfqpoint{1.208282in}{0.762465in}}{\pgfqpoint{1.216096in}{0.770279in}}%
\pgfpathcurveto{\pgfqpoint{1.223909in}{0.778092in}}{\pgfqpoint{1.228300in}{0.788691in}}{\pgfqpoint{1.228300in}{0.799742in}}%
\pgfpathcurveto{\pgfqpoint{1.228300in}{0.810792in}}{\pgfqpoint{1.223909in}{0.821391in}}{\pgfqpoint{1.216096in}{0.829204in}}%
\pgfpathcurveto{\pgfqpoint{1.208282in}{0.837018in}}{\pgfqpoint{1.197683in}{0.841408in}}{\pgfqpoint{1.186633in}{0.841408in}}%
\pgfpathcurveto{\pgfqpoint{1.175583in}{0.841408in}}{\pgfqpoint{1.164984in}{0.837018in}}{\pgfqpoint{1.157170in}{0.829204in}}%
\pgfpathcurveto{\pgfqpoint{1.149357in}{0.821391in}}{\pgfqpoint{1.144966in}{0.810792in}}{\pgfqpoint{1.144966in}{0.799742in}}%
\pgfpathcurveto{\pgfqpoint{1.144966in}{0.788691in}}{\pgfqpoint{1.149357in}{0.778092in}}{\pgfqpoint{1.157170in}{0.770279in}}%
\pgfpathcurveto{\pgfqpoint{1.164984in}{0.762465in}}{\pgfqpoint{1.175583in}{0.758075in}}{\pgfqpoint{1.186633in}{0.758075in}}%
\pgfpathclose%
\pgfusepath{stroke,fill}%
\end{pgfscope}%
\begin{pgfscope}%
\pgfpathrectangle{\pgfqpoint{0.648703in}{0.548769in}}{\pgfqpoint{5.201297in}{3.102590in}}%
\pgfusepath{clip}%
\pgfsetbuttcap%
\pgfsetroundjoin%
\definecolor{currentfill}{rgb}{1.000000,0.498039,0.054902}%
\pgfsetfillcolor{currentfill}%
\pgfsetlinewidth{1.003750pt}%
\definecolor{currentstroke}{rgb}{1.000000,0.498039,0.054902}%
\pgfsetstrokecolor{currentstroke}%
\pgfsetdash{}{0pt}%
\pgfpathmoveto{\pgfqpoint{4.341781in}{3.193800in}}%
\pgfpathcurveto{\pgfqpoint{4.352831in}{3.193800in}}{\pgfqpoint{4.363430in}{3.198191in}}{\pgfqpoint{4.371244in}{3.206004in}}%
\pgfpathcurveto{\pgfqpoint{4.379057in}{3.213818in}}{\pgfqpoint{4.383448in}{3.224417in}}{\pgfqpoint{4.383448in}{3.235467in}}%
\pgfpathcurveto{\pgfqpoint{4.383448in}{3.246517in}}{\pgfqpoint{4.379057in}{3.257116in}}{\pgfqpoint{4.371244in}{3.264930in}}%
\pgfpathcurveto{\pgfqpoint{4.363430in}{3.272743in}}{\pgfqpoint{4.352831in}{3.277134in}}{\pgfqpoint{4.341781in}{3.277134in}}%
\pgfpathcurveto{\pgfqpoint{4.330731in}{3.277134in}}{\pgfqpoint{4.320132in}{3.272743in}}{\pgfqpoint{4.312318in}{3.264930in}}%
\pgfpathcurveto{\pgfqpoint{4.304505in}{3.257116in}}{\pgfqpoint{4.300114in}{3.246517in}}{\pgfqpoint{4.300114in}{3.235467in}}%
\pgfpathcurveto{\pgfqpoint{4.300114in}{3.224417in}}{\pgfqpoint{4.304505in}{3.213818in}}{\pgfqpoint{4.312318in}{3.206004in}}%
\pgfpathcurveto{\pgfqpoint{4.320132in}{3.198191in}}{\pgfqpoint{4.330731in}{3.193800in}}{\pgfqpoint{4.341781in}{3.193800in}}%
\pgfpathclose%
\pgfusepath{stroke,fill}%
\end{pgfscope}%
\begin{pgfscope}%
\pgfpathrectangle{\pgfqpoint{0.648703in}{0.548769in}}{\pgfqpoint{5.201297in}{3.102590in}}%
\pgfusepath{clip}%
\pgfsetbuttcap%
\pgfsetroundjoin%
\definecolor{currentfill}{rgb}{0.121569,0.466667,0.705882}%
\pgfsetfillcolor{currentfill}%
\pgfsetlinewidth{1.003750pt}%
\definecolor{currentstroke}{rgb}{0.121569,0.466667,0.705882}%
\pgfsetstrokecolor{currentstroke}%
\pgfsetdash{}{0pt}%
\pgfpathmoveto{\pgfqpoint{0.885126in}{0.648129in}}%
\pgfpathcurveto{\pgfqpoint{0.896176in}{0.648129in}}{\pgfqpoint{0.906775in}{0.652519in}}{\pgfqpoint{0.914589in}{0.660333in}}%
\pgfpathcurveto{\pgfqpoint{0.922402in}{0.668146in}}{\pgfqpoint{0.926793in}{0.678745in}}{\pgfqpoint{0.926793in}{0.689796in}}%
\pgfpathcurveto{\pgfqpoint{0.926793in}{0.700846in}}{\pgfqpoint{0.922402in}{0.711445in}}{\pgfqpoint{0.914589in}{0.719258in}}%
\pgfpathcurveto{\pgfqpoint{0.906775in}{0.727072in}}{\pgfqpoint{0.896176in}{0.731462in}}{\pgfqpoint{0.885126in}{0.731462in}}%
\pgfpathcurveto{\pgfqpoint{0.874076in}{0.731462in}}{\pgfqpoint{0.863477in}{0.727072in}}{\pgfqpoint{0.855663in}{0.719258in}}%
\pgfpathcurveto{\pgfqpoint{0.847850in}{0.711445in}}{\pgfqpoint{0.843459in}{0.700846in}}{\pgfqpoint{0.843459in}{0.689796in}}%
\pgfpathcurveto{\pgfqpoint{0.843459in}{0.678745in}}{\pgfqpoint{0.847850in}{0.668146in}}{\pgfqpoint{0.855663in}{0.660333in}}%
\pgfpathcurveto{\pgfqpoint{0.863477in}{0.652519in}}{\pgfqpoint{0.874076in}{0.648129in}}{\pgfqpoint{0.885126in}{0.648129in}}%
\pgfpathclose%
\pgfusepath{stroke,fill}%
\end{pgfscope}%
\begin{pgfscope}%
\pgfpathrectangle{\pgfqpoint{0.648703in}{0.548769in}}{\pgfqpoint{5.201297in}{3.102590in}}%
\pgfusepath{clip}%
\pgfsetbuttcap%
\pgfsetroundjoin%
\definecolor{currentfill}{rgb}{1.000000,0.498039,0.054902}%
\pgfsetfillcolor{currentfill}%
\pgfsetlinewidth{1.003750pt}%
\definecolor{currentstroke}{rgb}{1.000000,0.498039,0.054902}%
\pgfsetstrokecolor{currentstroke}%
\pgfsetdash{}{0pt}%
\pgfpathmoveto{\pgfqpoint{3.443721in}{3.198029in}}%
\pgfpathcurveto{\pgfqpoint{3.454771in}{3.198029in}}{\pgfqpoint{3.465370in}{3.202419in}}{\pgfqpoint{3.473183in}{3.210233in}}%
\pgfpathcurveto{\pgfqpoint{3.480997in}{3.218046in}}{\pgfqpoint{3.485387in}{3.228646in}}{\pgfqpoint{3.485387in}{3.239696in}}%
\pgfpathcurveto{\pgfqpoint{3.485387in}{3.250746in}}{\pgfqpoint{3.480997in}{3.261345in}}{\pgfqpoint{3.473183in}{3.269158in}}%
\pgfpathcurveto{\pgfqpoint{3.465370in}{3.276972in}}{\pgfqpoint{3.454771in}{3.281362in}}{\pgfqpoint{3.443721in}{3.281362in}}%
\pgfpathcurveto{\pgfqpoint{3.432671in}{3.281362in}}{\pgfqpoint{3.422072in}{3.276972in}}{\pgfqpoint{3.414258in}{3.269158in}}%
\pgfpathcurveto{\pgfqpoint{3.406444in}{3.261345in}}{\pgfqpoint{3.402054in}{3.250746in}}{\pgfqpoint{3.402054in}{3.239696in}}%
\pgfpathcurveto{\pgfqpoint{3.402054in}{3.228646in}}{\pgfqpoint{3.406444in}{3.218046in}}{\pgfqpoint{3.414258in}{3.210233in}}%
\pgfpathcurveto{\pgfqpoint{3.422072in}{3.202419in}}{\pgfqpoint{3.432671in}{3.198029in}}{\pgfqpoint{3.443721in}{3.198029in}}%
\pgfpathclose%
\pgfusepath{stroke,fill}%
\end{pgfscope}%
\begin{pgfscope}%
\pgfpathrectangle{\pgfqpoint{0.648703in}{0.548769in}}{\pgfqpoint{5.201297in}{3.102590in}}%
\pgfusepath{clip}%
\pgfsetbuttcap%
\pgfsetroundjoin%
\definecolor{currentfill}{rgb}{1.000000,0.498039,0.054902}%
\pgfsetfillcolor{currentfill}%
\pgfsetlinewidth{1.003750pt}%
\definecolor{currentstroke}{rgb}{1.000000,0.498039,0.054902}%
\pgfsetstrokecolor{currentstroke}%
\pgfsetdash{}{0pt}%
\pgfpathmoveto{\pgfqpoint{4.363817in}{3.193800in}}%
\pgfpathcurveto{\pgfqpoint{4.374867in}{3.193800in}}{\pgfqpoint{4.385466in}{3.198191in}}{\pgfqpoint{4.393280in}{3.206004in}}%
\pgfpathcurveto{\pgfqpoint{4.401094in}{3.213818in}}{\pgfqpoint{4.405484in}{3.224417in}}{\pgfqpoint{4.405484in}{3.235467in}}%
\pgfpathcurveto{\pgfqpoint{4.405484in}{3.246517in}}{\pgfqpoint{4.401094in}{3.257116in}}{\pgfqpoint{4.393280in}{3.264930in}}%
\pgfpathcurveto{\pgfqpoint{4.385466in}{3.272743in}}{\pgfqpoint{4.374867in}{3.277134in}}{\pgfqpoint{4.363817in}{3.277134in}}%
\pgfpathcurveto{\pgfqpoint{4.352767in}{3.277134in}}{\pgfqpoint{4.342168in}{3.272743in}}{\pgfqpoint{4.334354in}{3.264930in}}%
\pgfpathcurveto{\pgfqpoint{4.326541in}{3.257116in}}{\pgfqpoint{4.322150in}{3.246517in}}{\pgfqpoint{4.322150in}{3.235467in}}%
\pgfpathcurveto{\pgfqpoint{4.322150in}{3.224417in}}{\pgfqpoint{4.326541in}{3.213818in}}{\pgfqpoint{4.334354in}{3.206004in}}%
\pgfpathcurveto{\pgfqpoint{4.342168in}{3.198191in}}{\pgfqpoint{4.352767in}{3.193800in}}{\pgfqpoint{4.363817in}{3.193800in}}%
\pgfpathclose%
\pgfusepath{stroke,fill}%
\end{pgfscope}%
\begin{pgfscope}%
\pgfpathrectangle{\pgfqpoint{0.648703in}{0.548769in}}{\pgfqpoint{5.201297in}{3.102590in}}%
\pgfusepath{clip}%
\pgfsetbuttcap%
\pgfsetroundjoin%
\definecolor{currentfill}{rgb}{0.121569,0.466667,0.705882}%
\pgfsetfillcolor{currentfill}%
\pgfsetlinewidth{1.003750pt}%
\definecolor{currentstroke}{rgb}{0.121569,0.466667,0.705882}%
\pgfsetstrokecolor{currentstroke}%
\pgfsetdash{}{0pt}%
\pgfpathmoveto{\pgfqpoint{0.893762in}{0.652358in}}%
\pgfpathcurveto{\pgfqpoint{0.904812in}{0.652358in}}{\pgfqpoint{0.915411in}{0.656748in}}{\pgfqpoint{0.923225in}{0.664562in}}%
\pgfpathcurveto{\pgfqpoint{0.931038in}{0.672375in}}{\pgfqpoint{0.935429in}{0.682974in}}{\pgfqpoint{0.935429in}{0.694024in}}%
\pgfpathcurveto{\pgfqpoint{0.935429in}{0.705074in}}{\pgfqpoint{0.931038in}{0.715673in}}{\pgfqpoint{0.923225in}{0.723487in}}%
\pgfpathcurveto{\pgfqpoint{0.915411in}{0.731301in}}{\pgfqpoint{0.904812in}{0.735691in}}{\pgfqpoint{0.893762in}{0.735691in}}%
\pgfpathcurveto{\pgfqpoint{0.882712in}{0.735691in}}{\pgfqpoint{0.872113in}{0.731301in}}{\pgfqpoint{0.864299in}{0.723487in}}%
\pgfpathcurveto{\pgfqpoint{0.856486in}{0.715673in}}{\pgfqpoint{0.852095in}{0.705074in}}{\pgfqpoint{0.852095in}{0.694024in}}%
\pgfpathcurveto{\pgfqpoint{0.852095in}{0.682974in}}{\pgfqpoint{0.856486in}{0.672375in}}{\pgfqpoint{0.864299in}{0.664562in}}%
\pgfpathcurveto{\pgfqpoint{0.872113in}{0.656748in}}{\pgfqpoint{0.882712in}{0.652358in}}{\pgfqpoint{0.893762in}{0.652358in}}%
\pgfpathclose%
\pgfusepath{stroke,fill}%
\end{pgfscope}%
\begin{pgfscope}%
\pgfpathrectangle{\pgfqpoint{0.648703in}{0.548769in}}{\pgfqpoint{5.201297in}{3.102590in}}%
\pgfusepath{clip}%
\pgfsetbuttcap%
\pgfsetroundjoin%
\definecolor{currentfill}{rgb}{0.121569,0.466667,0.705882}%
\pgfsetfillcolor{currentfill}%
\pgfsetlinewidth{1.003750pt}%
\definecolor{currentstroke}{rgb}{0.121569,0.466667,0.705882}%
\pgfsetstrokecolor{currentstroke}%
\pgfsetdash{}{0pt}%
\pgfpathmoveto{\pgfqpoint{0.885189in}{0.648129in}}%
\pgfpathcurveto{\pgfqpoint{0.896239in}{0.648129in}}{\pgfqpoint{0.906838in}{0.652519in}}{\pgfqpoint{0.914651in}{0.660333in}}%
\pgfpathcurveto{\pgfqpoint{0.922465in}{0.668146in}}{\pgfqpoint{0.926855in}{0.678745in}}{\pgfqpoint{0.926855in}{0.689796in}}%
\pgfpathcurveto{\pgfqpoint{0.926855in}{0.700846in}}{\pgfqpoint{0.922465in}{0.711445in}}{\pgfqpoint{0.914651in}{0.719258in}}%
\pgfpathcurveto{\pgfqpoint{0.906838in}{0.727072in}}{\pgfqpoint{0.896239in}{0.731462in}}{\pgfqpoint{0.885189in}{0.731462in}}%
\pgfpathcurveto{\pgfqpoint{0.874139in}{0.731462in}}{\pgfqpoint{0.863539in}{0.727072in}}{\pgfqpoint{0.855726in}{0.719258in}}%
\pgfpathcurveto{\pgfqpoint{0.847912in}{0.711445in}}{\pgfqpoint{0.843522in}{0.700846in}}{\pgfqpoint{0.843522in}{0.689796in}}%
\pgfpathcurveto{\pgfqpoint{0.843522in}{0.678745in}}{\pgfqpoint{0.847912in}{0.668146in}}{\pgfqpoint{0.855726in}{0.660333in}}%
\pgfpathcurveto{\pgfqpoint{0.863539in}{0.652519in}}{\pgfqpoint{0.874139in}{0.648129in}}{\pgfqpoint{0.885189in}{0.648129in}}%
\pgfpathclose%
\pgfusepath{stroke,fill}%
\end{pgfscope}%
\begin{pgfscope}%
\pgfpathrectangle{\pgfqpoint{0.648703in}{0.548769in}}{\pgfqpoint{5.201297in}{3.102590in}}%
\pgfusepath{clip}%
\pgfsetbuttcap%
\pgfsetroundjoin%
\definecolor{currentfill}{rgb}{0.121569,0.466667,0.705882}%
\pgfsetfillcolor{currentfill}%
\pgfsetlinewidth{1.003750pt}%
\definecolor{currentstroke}{rgb}{0.121569,0.466667,0.705882}%
\pgfsetstrokecolor{currentstroke}%
\pgfsetdash{}{0pt}%
\pgfpathmoveto{\pgfqpoint{0.885173in}{0.648129in}}%
\pgfpathcurveto{\pgfqpoint{0.896223in}{0.648129in}}{\pgfqpoint{0.906822in}{0.652519in}}{\pgfqpoint{0.914635in}{0.660333in}}%
\pgfpathcurveto{\pgfqpoint{0.922449in}{0.668146in}}{\pgfqpoint{0.926839in}{0.678745in}}{\pgfqpoint{0.926839in}{0.689796in}}%
\pgfpathcurveto{\pgfqpoint{0.926839in}{0.700846in}}{\pgfqpoint{0.922449in}{0.711445in}}{\pgfqpoint{0.914635in}{0.719258in}}%
\pgfpathcurveto{\pgfqpoint{0.906822in}{0.727072in}}{\pgfqpoint{0.896223in}{0.731462in}}{\pgfqpoint{0.885173in}{0.731462in}}%
\pgfpathcurveto{\pgfqpoint{0.874123in}{0.731462in}}{\pgfqpoint{0.863523in}{0.727072in}}{\pgfqpoint{0.855710in}{0.719258in}}%
\pgfpathcurveto{\pgfqpoint{0.847896in}{0.711445in}}{\pgfqpoint{0.843506in}{0.700846in}}{\pgfqpoint{0.843506in}{0.689796in}}%
\pgfpathcurveto{\pgfqpoint{0.843506in}{0.678745in}}{\pgfqpoint{0.847896in}{0.668146in}}{\pgfqpoint{0.855710in}{0.660333in}}%
\pgfpathcurveto{\pgfqpoint{0.863523in}{0.652519in}}{\pgfqpoint{0.874123in}{0.648129in}}{\pgfqpoint{0.885173in}{0.648129in}}%
\pgfpathclose%
\pgfusepath{stroke,fill}%
\end{pgfscope}%
\begin{pgfscope}%
\pgfpathrectangle{\pgfqpoint{0.648703in}{0.548769in}}{\pgfqpoint{5.201297in}{3.102590in}}%
\pgfusepath{clip}%
\pgfsetbuttcap%
\pgfsetroundjoin%
\definecolor{currentfill}{rgb}{0.121569,0.466667,0.705882}%
\pgfsetfillcolor{currentfill}%
\pgfsetlinewidth{1.003750pt}%
\definecolor{currentstroke}{rgb}{0.121569,0.466667,0.705882}%
\pgfsetstrokecolor{currentstroke}%
\pgfsetdash{}{0pt}%
\pgfpathmoveto{\pgfqpoint{0.885757in}{0.648129in}}%
\pgfpathcurveto{\pgfqpoint{0.896807in}{0.648129in}}{\pgfqpoint{0.907406in}{0.652519in}}{\pgfqpoint{0.915220in}{0.660333in}}%
\pgfpathcurveto{\pgfqpoint{0.923033in}{0.668146in}}{\pgfqpoint{0.927424in}{0.678745in}}{\pgfqpoint{0.927424in}{0.689796in}}%
\pgfpathcurveto{\pgfqpoint{0.927424in}{0.700846in}}{\pgfqpoint{0.923033in}{0.711445in}}{\pgfqpoint{0.915220in}{0.719258in}}%
\pgfpathcurveto{\pgfqpoint{0.907406in}{0.727072in}}{\pgfqpoint{0.896807in}{0.731462in}}{\pgfqpoint{0.885757in}{0.731462in}}%
\pgfpathcurveto{\pgfqpoint{0.874707in}{0.731462in}}{\pgfqpoint{0.864108in}{0.727072in}}{\pgfqpoint{0.856294in}{0.719258in}}%
\pgfpathcurveto{\pgfqpoint{0.848481in}{0.711445in}}{\pgfqpoint{0.844090in}{0.700846in}}{\pgfqpoint{0.844090in}{0.689796in}}%
\pgfpathcurveto{\pgfqpoint{0.844090in}{0.678745in}}{\pgfqpoint{0.848481in}{0.668146in}}{\pgfqpoint{0.856294in}{0.660333in}}%
\pgfpathcurveto{\pgfqpoint{0.864108in}{0.652519in}}{\pgfqpoint{0.874707in}{0.648129in}}{\pgfqpoint{0.885757in}{0.648129in}}%
\pgfpathclose%
\pgfusepath{stroke,fill}%
\end{pgfscope}%
\begin{pgfscope}%
\pgfpathrectangle{\pgfqpoint{0.648703in}{0.548769in}}{\pgfqpoint{5.201297in}{3.102590in}}%
\pgfusepath{clip}%
\pgfsetbuttcap%
\pgfsetroundjoin%
\definecolor{currentfill}{rgb}{0.121569,0.466667,0.705882}%
\pgfsetfillcolor{currentfill}%
\pgfsetlinewidth{1.003750pt}%
\definecolor{currentstroke}{rgb}{0.121569,0.466667,0.705882}%
\pgfsetstrokecolor{currentstroke}%
\pgfsetdash{}{0pt}%
\pgfpathmoveto{\pgfqpoint{0.885249in}{0.648129in}}%
\pgfpathcurveto{\pgfqpoint{0.896299in}{0.648129in}}{\pgfqpoint{0.906898in}{0.652519in}}{\pgfqpoint{0.914711in}{0.660333in}}%
\pgfpathcurveto{\pgfqpoint{0.922525in}{0.668146in}}{\pgfqpoint{0.926915in}{0.678745in}}{\pgfqpoint{0.926915in}{0.689796in}}%
\pgfpathcurveto{\pgfqpoint{0.926915in}{0.700846in}}{\pgfqpoint{0.922525in}{0.711445in}}{\pgfqpoint{0.914711in}{0.719258in}}%
\pgfpathcurveto{\pgfqpoint{0.906898in}{0.727072in}}{\pgfqpoint{0.896299in}{0.731462in}}{\pgfqpoint{0.885249in}{0.731462in}}%
\pgfpathcurveto{\pgfqpoint{0.874199in}{0.731462in}}{\pgfqpoint{0.863599in}{0.727072in}}{\pgfqpoint{0.855786in}{0.719258in}}%
\pgfpathcurveto{\pgfqpoint{0.847972in}{0.711445in}}{\pgfqpoint{0.843582in}{0.700846in}}{\pgfqpoint{0.843582in}{0.689796in}}%
\pgfpathcurveto{\pgfqpoint{0.843582in}{0.678745in}}{\pgfqpoint{0.847972in}{0.668146in}}{\pgfqpoint{0.855786in}{0.660333in}}%
\pgfpathcurveto{\pgfqpoint{0.863599in}{0.652519in}}{\pgfqpoint{0.874199in}{0.648129in}}{\pgfqpoint{0.885249in}{0.648129in}}%
\pgfpathclose%
\pgfusepath{stroke,fill}%
\end{pgfscope}%
\begin{pgfscope}%
\pgfpathrectangle{\pgfqpoint{0.648703in}{0.548769in}}{\pgfqpoint{5.201297in}{3.102590in}}%
\pgfusepath{clip}%
\pgfsetbuttcap%
\pgfsetroundjoin%
\definecolor{currentfill}{rgb}{0.121569,0.466667,0.705882}%
\pgfsetfillcolor{currentfill}%
\pgfsetlinewidth{1.003750pt}%
\definecolor{currentstroke}{rgb}{0.121569,0.466667,0.705882}%
\pgfsetstrokecolor{currentstroke}%
\pgfsetdash{}{0pt}%
\pgfpathmoveto{\pgfqpoint{0.885156in}{0.648129in}}%
\pgfpathcurveto{\pgfqpoint{0.896206in}{0.648129in}}{\pgfqpoint{0.906805in}{0.652519in}}{\pgfqpoint{0.914619in}{0.660333in}}%
\pgfpathcurveto{\pgfqpoint{0.922432in}{0.668146in}}{\pgfqpoint{0.926823in}{0.678745in}}{\pgfqpoint{0.926823in}{0.689796in}}%
\pgfpathcurveto{\pgfqpoint{0.926823in}{0.700846in}}{\pgfqpoint{0.922432in}{0.711445in}}{\pgfqpoint{0.914619in}{0.719258in}}%
\pgfpathcurveto{\pgfqpoint{0.906805in}{0.727072in}}{\pgfqpoint{0.896206in}{0.731462in}}{\pgfqpoint{0.885156in}{0.731462in}}%
\pgfpathcurveto{\pgfqpoint{0.874106in}{0.731462in}}{\pgfqpoint{0.863507in}{0.727072in}}{\pgfqpoint{0.855693in}{0.719258in}}%
\pgfpathcurveto{\pgfqpoint{0.847879in}{0.711445in}}{\pgfqpoint{0.843489in}{0.700846in}}{\pgfqpoint{0.843489in}{0.689796in}}%
\pgfpathcurveto{\pgfqpoint{0.843489in}{0.678745in}}{\pgfqpoint{0.847879in}{0.668146in}}{\pgfqpoint{0.855693in}{0.660333in}}%
\pgfpathcurveto{\pgfqpoint{0.863507in}{0.652519in}}{\pgfqpoint{0.874106in}{0.648129in}}{\pgfqpoint{0.885156in}{0.648129in}}%
\pgfpathclose%
\pgfusepath{stroke,fill}%
\end{pgfscope}%
\begin{pgfscope}%
\pgfpathrectangle{\pgfqpoint{0.648703in}{0.548769in}}{\pgfqpoint{5.201297in}{3.102590in}}%
\pgfusepath{clip}%
\pgfsetbuttcap%
\pgfsetroundjoin%
\definecolor{currentfill}{rgb}{0.121569,0.466667,0.705882}%
\pgfsetfillcolor{currentfill}%
\pgfsetlinewidth{1.003750pt}%
\definecolor{currentstroke}{rgb}{0.121569,0.466667,0.705882}%
\pgfsetstrokecolor{currentstroke}%
\pgfsetdash{}{0pt}%
\pgfpathmoveto{\pgfqpoint{4.327493in}{3.181114in}}%
\pgfpathcurveto{\pgfqpoint{4.338544in}{3.181114in}}{\pgfqpoint{4.349143in}{3.185504in}}{\pgfqpoint{4.356956in}{3.193318in}}%
\pgfpathcurveto{\pgfqpoint{4.364770in}{3.201132in}}{\pgfqpoint{4.369160in}{3.211731in}}{\pgfqpoint{4.369160in}{3.222781in}}%
\pgfpathcurveto{\pgfqpoint{4.369160in}{3.233831in}}{\pgfqpoint{4.364770in}{3.244430in}}{\pgfqpoint{4.356956in}{3.252244in}}%
\pgfpathcurveto{\pgfqpoint{4.349143in}{3.260057in}}{\pgfqpoint{4.338544in}{3.264448in}}{\pgfqpoint{4.327493in}{3.264448in}}%
\pgfpathcurveto{\pgfqpoint{4.316443in}{3.264448in}}{\pgfqpoint{4.305844in}{3.260057in}}{\pgfqpoint{4.298031in}{3.252244in}}%
\pgfpathcurveto{\pgfqpoint{4.290217in}{3.244430in}}{\pgfqpoint{4.285827in}{3.233831in}}{\pgfqpoint{4.285827in}{3.222781in}}%
\pgfpathcurveto{\pgfqpoint{4.285827in}{3.211731in}}{\pgfqpoint{4.290217in}{3.201132in}}{\pgfqpoint{4.298031in}{3.193318in}}%
\pgfpathcurveto{\pgfqpoint{4.305844in}{3.185504in}}{\pgfqpoint{4.316443in}{3.181114in}}{\pgfqpoint{4.327493in}{3.181114in}}%
\pgfpathclose%
\pgfusepath{stroke,fill}%
\end{pgfscope}%
\begin{pgfscope}%
\pgfpathrectangle{\pgfqpoint{0.648703in}{0.548769in}}{\pgfqpoint{5.201297in}{3.102590in}}%
\pgfusepath{clip}%
\pgfsetbuttcap%
\pgfsetroundjoin%
\definecolor{currentfill}{rgb}{1.000000,0.498039,0.054902}%
\pgfsetfillcolor{currentfill}%
\pgfsetlinewidth{1.003750pt}%
\definecolor{currentstroke}{rgb}{1.000000,0.498039,0.054902}%
\pgfsetstrokecolor{currentstroke}%
\pgfsetdash{}{0pt}%
\pgfpathmoveto{\pgfqpoint{3.760146in}{3.193800in}}%
\pgfpathcurveto{\pgfqpoint{3.771196in}{3.193800in}}{\pgfqpoint{3.781795in}{3.198191in}}{\pgfqpoint{3.789609in}{3.206004in}}%
\pgfpathcurveto{\pgfqpoint{3.797422in}{3.213818in}}{\pgfqpoint{3.801813in}{3.224417in}}{\pgfqpoint{3.801813in}{3.235467in}}%
\pgfpathcurveto{\pgfqpoint{3.801813in}{3.246517in}}{\pgfqpoint{3.797422in}{3.257116in}}{\pgfqpoint{3.789609in}{3.264930in}}%
\pgfpathcurveto{\pgfqpoint{3.781795in}{3.272743in}}{\pgfqpoint{3.771196in}{3.277134in}}{\pgfqpoint{3.760146in}{3.277134in}}%
\pgfpathcurveto{\pgfqpoint{3.749096in}{3.277134in}}{\pgfqpoint{3.738497in}{3.272743in}}{\pgfqpoint{3.730683in}{3.264930in}}%
\pgfpathcurveto{\pgfqpoint{3.722869in}{3.257116in}}{\pgfqpoint{3.718479in}{3.246517in}}{\pgfqpoint{3.718479in}{3.235467in}}%
\pgfpathcurveto{\pgfqpoint{3.718479in}{3.224417in}}{\pgfqpoint{3.722869in}{3.213818in}}{\pgfqpoint{3.730683in}{3.206004in}}%
\pgfpathcurveto{\pgfqpoint{3.738497in}{3.198191in}}{\pgfqpoint{3.749096in}{3.193800in}}{\pgfqpoint{3.760146in}{3.193800in}}%
\pgfpathclose%
\pgfusepath{stroke,fill}%
\end{pgfscope}%
\begin{pgfscope}%
\pgfpathrectangle{\pgfqpoint{0.648703in}{0.548769in}}{\pgfqpoint{5.201297in}{3.102590in}}%
\pgfusepath{clip}%
\pgfsetbuttcap%
\pgfsetroundjoin%
\definecolor{currentfill}{rgb}{1.000000,0.498039,0.054902}%
\pgfsetfillcolor{currentfill}%
\pgfsetlinewidth{1.003750pt}%
\definecolor{currentstroke}{rgb}{1.000000,0.498039,0.054902}%
\pgfsetstrokecolor{currentstroke}%
\pgfsetdash{}{0pt}%
\pgfpathmoveto{\pgfqpoint{3.937147in}{3.193800in}}%
\pgfpathcurveto{\pgfqpoint{3.948197in}{3.193800in}}{\pgfqpoint{3.958796in}{3.198191in}}{\pgfqpoint{3.966609in}{3.206004in}}%
\pgfpathcurveto{\pgfqpoint{3.974423in}{3.213818in}}{\pgfqpoint{3.978813in}{3.224417in}}{\pgfqpoint{3.978813in}{3.235467in}}%
\pgfpathcurveto{\pgfqpoint{3.978813in}{3.246517in}}{\pgfqpoint{3.974423in}{3.257116in}}{\pgfqpoint{3.966609in}{3.264930in}}%
\pgfpathcurveto{\pgfqpoint{3.958796in}{3.272743in}}{\pgfqpoint{3.948197in}{3.277134in}}{\pgfqpoint{3.937147in}{3.277134in}}%
\pgfpathcurveto{\pgfqpoint{3.926096in}{3.277134in}}{\pgfqpoint{3.915497in}{3.272743in}}{\pgfqpoint{3.907684in}{3.264930in}}%
\pgfpathcurveto{\pgfqpoint{3.899870in}{3.257116in}}{\pgfqpoint{3.895480in}{3.246517in}}{\pgfqpoint{3.895480in}{3.235467in}}%
\pgfpathcurveto{\pgfqpoint{3.895480in}{3.224417in}}{\pgfqpoint{3.899870in}{3.213818in}}{\pgfqpoint{3.907684in}{3.206004in}}%
\pgfpathcurveto{\pgfqpoint{3.915497in}{3.198191in}}{\pgfqpoint{3.926096in}{3.193800in}}{\pgfqpoint{3.937147in}{3.193800in}}%
\pgfpathclose%
\pgfusepath{stroke,fill}%
\end{pgfscope}%
\begin{pgfscope}%
\pgfpathrectangle{\pgfqpoint{0.648703in}{0.548769in}}{\pgfqpoint{5.201297in}{3.102590in}}%
\pgfusepath{clip}%
\pgfsetbuttcap%
\pgfsetroundjoin%
\definecolor{currentfill}{rgb}{0.121569,0.466667,0.705882}%
\pgfsetfillcolor{currentfill}%
\pgfsetlinewidth{1.003750pt}%
\definecolor{currentstroke}{rgb}{0.121569,0.466667,0.705882}%
\pgfsetstrokecolor{currentstroke}%
\pgfsetdash{}{0pt}%
\pgfpathmoveto{\pgfqpoint{0.885133in}{0.648129in}}%
\pgfpathcurveto{\pgfqpoint{0.896183in}{0.648129in}}{\pgfqpoint{0.906782in}{0.652519in}}{\pgfqpoint{0.914596in}{0.660333in}}%
\pgfpathcurveto{\pgfqpoint{0.922409in}{0.668146in}}{\pgfqpoint{0.926799in}{0.678745in}}{\pgfqpoint{0.926799in}{0.689796in}}%
\pgfpathcurveto{\pgfqpoint{0.926799in}{0.700846in}}{\pgfqpoint{0.922409in}{0.711445in}}{\pgfqpoint{0.914596in}{0.719258in}}%
\pgfpathcurveto{\pgfqpoint{0.906782in}{0.727072in}}{\pgfqpoint{0.896183in}{0.731462in}}{\pgfqpoint{0.885133in}{0.731462in}}%
\pgfpathcurveto{\pgfqpoint{0.874083in}{0.731462in}}{\pgfqpoint{0.863484in}{0.727072in}}{\pgfqpoint{0.855670in}{0.719258in}}%
\pgfpathcurveto{\pgfqpoint{0.847856in}{0.711445in}}{\pgfqpoint{0.843466in}{0.700846in}}{\pgfqpoint{0.843466in}{0.689796in}}%
\pgfpathcurveto{\pgfqpoint{0.843466in}{0.678745in}}{\pgfqpoint{0.847856in}{0.668146in}}{\pgfqpoint{0.855670in}{0.660333in}}%
\pgfpathcurveto{\pgfqpoint{0.863484in}{0.652519in}}{\pgfqpoint{0.874083in}{0.648129in}}{\pgfqpoint{0.885133in}{0.648129in}}%
\pgfpathclose%
\pgfusepath{stroke,fill}%
\end{pgfscope}%
\begin{pgfscope}%
\pgfpathrectangle{\pgfqpoint{0.648703in}{0.548769in}}{\pgfqpoint{5.201297in}{3.102590in}}%
\pgfusepath{clip}%
\pgfsetbuttcap%
\pgfsetroundjoin%
\definecolor{currentfill}{rgb}{0.121569,0.466667,0.705882}%
\pgfsetfillcolor{currentfill}%
\pgfsetlinewidth{1.003750pt}%
\definecolor{currentstroke}{rgb}{0.121569,0.466667,0.705882}%
\pgfsetstrokecolor{currentstroke}%
\pgfsetdash{}{0pt}%
\pgfpathmoveto{\pgfqpoint{0.937349in}{0.665044in}}%
\pgfpathcurveto{\pgfqpoint{0.948399in}{0.665044in}}{\pgfqpoint{0.958998in}{0.669434in}}{\pgfqpoint{0.966811in}{0.677248in}}%
\pgfpathcurveto{\pgfqpoint{0.974625in}{0.685061in}}{\pgfqpoint{0.979015in}{0.695660in}}{\pgfqpoint{0.979015in}{0.706710in}}%
\pgfpathcurveto{\pgfqpoint{0.979015in}{0.717760in}}{\pgfqpoint{0.974625in}{0.728360in}}{\pgfqpoint{0.966811in}{0.736173in}}%
\pgfpathcurveto{\pgfqpoint{0.958998in}{0.743987in}}{\pgfqpoint{0.948399in}{0.748377in}}{\pgfqpoint{0.937349in}{0.748377in}}%
\pgfpathcurveto{\pgfqpoint{0.926298in}{0.748377in}}{\pgfqpoint{0.915699in}{0.743987in}}{\pgfqpoint{0.907886in}{0.736173in}}%
\pgfpathcurveto{\pgfqpoint{0.900072in}{0.728360in}}{\pgfqpoint{0.895682in}{0.717760in}}{\pgfqpoint{0.895682in}{0.706710in}}%
\pgfpathcurveto{\pgfqpoint{0.895682in}{0.695660in}}{\pgfqpoint{0.900072in}{0.685061in}}{\pgfqpoint{0.907886in}{0.677248in}}%
\pgfpathcurveto{\pgfqpoint{0.915699in}{0.669434in}}{\pgfqpoint{0.926298in}{0.665044in}}{\pgfqpoint{0.937349in}{0.665044in}}%
\pgfpathclose%
\pgfusepath{stroke,fill}%
\end{pgfscope}%
\begin{pgfscope}%
\pgfpathrectangle{\pgfqpoint{0.648703in}{0.548769in}}{\pgfqpoint{5.201297in}{3.102590in}}%
\pgfusepath{clip}%
\pgfsetbuttcap%
\pgfsetroundjoin%
\definecolor{currentfill}{rgb}{1.000000,0.498039,0.054902}%
\pgfsetfillcolor{currentfill}%
\pgfsetlinewidth{1.003750pt}%
\definecolor{currentstroke}{rgb}{1.000000,0.498039,0.054902}%
\pgfsetstrokecolor{currentstroke}%
\pgfsetdash{}{0pt}%
\pgfpathmoveto{\pgfqpoint{4.290427in}{3.189572in}}%
\pgfpathcurveto{\pgfqpoint{4.301478in}{3.189572in}}{\pgfqpoint{4.312077in}{3.193962in}}{\pgfqpoint{4.319890in}{3.201775in}}%
\pgfpathcurveto{\pgfqpoint{4.327704in}{3.209589in}}{\pgfqpoint{4.332094in}{3.220188in}}{\pgfqpoint{4.332094in}{3.231238in}}%
\pgfpathcurveto{\pgfqpoint{4.332094in}{3.242288in}}{\pgfqpoint{4.327704in}{3.252887in}}{\pgfqpoint{4.319890in}{3.260701in}}%
\pgfpathcurveto{\pgfqpoint{4.312077in}{3.268515in}}{\pgfqpoint{4.301478in}{3.272905in}}{\pgfqpoint{4.290427in}{3.272905in}}%
\pgfpathcurveto{\pgfqpoint{4.279377in}{3.272905in}}{\pgfqpoint{4.268778in}{3.268515in}}{\pgfqpoint{4.260965in}{3.260701in}}%
\pgfpathcurveto{\pgfqpoint{4.253151in}{3.252887in}}{\pgfqpoint{4.248761in}{3.242288in}}{\pgfqpoint{4.248761in}{3.231238in}}%
\pgfpathcurveto{\pgfqpoint{4.248761in}{3.220188in}}{\pgfqpoint{4.253151in}{3.209589in}}{\pgfqpoint{4.260965in}{3.201775in}}%
\pgfpathcurveto{\pgfqpoint{4.268778in}{3.193962in}}{\pgfqpoint{4.279377in}{3.189572in}}{\pgfqpoint{4.290427in}{3.189572in}}%
\pgfpathclose%
\pgfusepath{stroke,fill}%
\end{pgfscope}%
\begin{pgfscope}%
\pgfpathrectangle{\pgfqpoint{0.648703in}{0.548769in}}{\pgfqpoint{5.201297in}{3.102590in}}%
\pgfusepath{clip}%
\pgfsetbuttcap%
\pgfsetroundjoin%
\definecolor{currentfill}{rgb}{0.121569,0.466667,0.705882}%
\pgfsetfillcolor{currentfill}%
\pgfsetlinewidth{1.003750pt}%
\definecolor{currentstroke}{rgb}{0.121569,0.466667,0.705882}%
\pgfsetstrokecolor{currentstroke}%
\pgfsetdash{}{0pt}%
\pgfpathmoveto{\pgfqpoint{1.188038in}{0.808819in}}%
\pgfpathcurveto{\pgfqpoint{1.199088in}{0.808819in}}{\pgfqpoint{1.209687in}{0.813209in}}{\pgfqpoint{1.217501in}{0.821023in}}%
\pgfpathcurveto{\pgfqpoint{1.225314in}{0.828837in}}{\pgfqpoint{1.229704in}{0.839436in}}{\pgfqpoint{1.229704in}{0.850486in}}%
\pgfpathcurveto{\pgfqpoint{1.229704in}{0.861536in}}{\pgfqpoint{1.225314in}{0.872135in}}{\pgfqpoint{1.217501in}{0.879949in}}%
\pgfpathcurveto{\pgfqpoint{1.209687in}{0.887762in}}{\pgfqpoint{1.199088in}{0.892152in}}{\pgfqpoint{1.188038in}{0.892152in}}%
\pgfpathcurveto{\pgfqpoint{1.176988in}{0.892152in}}{\pgfqpoint{1.166389in}{0.887762in}}{\pgfqpoint{1.158575in}{0.879949in}}%
\pgfpathcurveto{\pgfqpoint{1.150761in}{0.872135in}}{\pgfqpoint{1.146371in}{0.861536in}}{\pgfqpoint{1.146371in}{0.850486in}}%
\pgfpathcurveto{\pgfqpoint{1.146371in}{0.839436in}}{\pgfqpoint{1.150761in}{0.828837in}}{\pgfqpoint{1.158575in}{0.821023in}}%
\pgfpathcurveto{\pgfqpoint{1.166389in}{0.813209in}}{\pgfqpoint{1.176988in}{0.808819in}}{\pgfqpoint{1.188038in}{0.808819in}}%
\pgfpathclose%
\pgfusepath{stroke,fill}%
\end{pgfscope}%
\begin{pgfscope}%
\pgfpathrectangle{\pgfqpoint{0.648703in}{0.548769in}}{\pgfqpoint{5.201297in}{3.102590in}}%
\pgfusepath{clip}%
\pgfsetbuttcap%
\pgfsetroundjoin%
\definecolor{currentfill}{rgb}{0.839216,0.152941,0.156863}%
\pgfsetfillcolor{currentfill}%
\pgfsetlinewidth{1.003750pt}%
\definecolor{currentstroke}{rgb}{0.839216,0.152941,0.156863}%
\pgfsetstrokecolor{currentstroke}%
\pgfsetdash{}{0pt}%
\pgfpathmoveto{\pgfqpoint{4.527085in}{3.198029in}}%
\pgfpathcurveto{\pgfqpoint{4.538135in}{3.198029in}}{\pgfqpoint{4.548734in}{3.202419in}}{\pgfqpoint{4.556548in}{3.210233in}}%
\pgfpathcurveto{\pgfqpoint{4.564361in}{3.218046in}}{\pgfqpoint{4.568751in}{3.228646in}}{\pgfqpoint{4.568751in}{3.239696in}}%
\pgfpathcurveto{\pgfqpoint{4.568751in}{3.250746in}}{\pgfqpoint{4.564361in}{3.261345in}}{\pgfqpoint{4.556548in}{3.269158in}}%
\pgfpathcurveto{\pgfqpoint{4.548734in}{3.276972in}}{\pgfqpoint{4.538135in}{3.281362in}}{\pgfqpoint{4.527085in}{3.281362in}}%
\pgfpathcurveto{\pgfqpoint{4.516035in}{3.281362in}}{\pgfqpoint{4.505436in}{3.276972in}}{\pgfqpoint{4.497622in}{3.269158in}}%
\pgfpathcurveto{\pgfqpoint{4.489808in}{3.261345in}}{\pgfqpoint{4.485418in}{3.250746in}}{\pgfqpoint{4.485418in}{3.239696in}}%
\pgfpathcurveto{\pgfqpoint{4.485418in}{3.228646in}}{\pgfqpoint{4.489808in}{3.218046in}}{\pgfqpoint{4.497622in}{3.210233in}}%
\pgfpathcurveto{\pgfqpoint{4.505436in}{3.202419in}}{\pgfqpoint{4.516035in}{3.198029in}}{\pgfqpoint{4.527085in}{3.198029in}}%
\pgfpathclose%
\pgfusepath{stroke,fill}%
\end{pgfscope}%
\begin{pgfscope}%
\pgfpathrectangle{\pgfqpoint{0.648703in}{0.548769in}}{\pgfqpoint{5.201297in}{3.102590in}}%
\pgfusepath{clip}%
\pgfsetbuttcap%
\pgfsetroundjoin%
\definecolor{currentfill}{rgb}{0.121569,0.466667,0.705882}%
\pgfsetfillcolor{currentfill}%
\pgfsetlinewidth{1.003750pt}%
\definecolor{currentstroke}{rgb}{0.121569,0.466667,0.705882}%
\pgfsetstrokecolor{currentstroke}%
\pgfsetdash{}{0pt}%
\pgfpathmoveto{\pgfqpoint{0.888288in}{0.648129in}}%
\pgfpathcurveto{\pgfqpoint{0.899338in}{0.648129in}}{\pgfqpoint{0.909937in}{0.652519in}}{\pgfqpoint{0.917751in}{0.660333in}}%
\pgfpathcurveto{\pgfqpoint{0.925565in}{0.668146in}}{\pgfqpoint{0.929955in}{0.678745in}}{\pgfqpoint{0.929955in}{0.689796in}}%
\pgfpathcurveto{\pgfqpoint{0.929955in}{0.700846in}}{\pgfqpoint{0.925565in}{0.711445in}}{\pgfqpoint{0.917751in}{0.719258in}}%
\pgfpathcurveto{\pgfqpoint{0.909937in}{0.727072in}}{\pgfqpoint{0.899338in}{0.731462in}}{\pgfqpoint{0.888288in}{0.731462in}}%
\pgfpathcurveto{\pgfqpoint{0.877238in}{0.731462in}}{\pgfqpoint{0.866639in}{0.727072in}}{\pgfqpoint{0.858826in}{0.719258in}}%
\pgfpathcurveto{\pgfqpoint{0.851012in}{0.711445in}}{\pgfqpoint{0.846622in}{0.700846in}}{\pgfqpoint{0.846622in}{0.689796in}}%
\pgfpathcurveto{\pgfqpoint{0.846622in}{0.678745in}}{\pgfqpoint{0.851012in}{0.668146in}}{\pgfqpoint{0.858826in}{0.660333in}}%
\pgfpathcurveto{\pgfqpoint{0.866639in}{0.652519in}}{\pgfqpoint{0.877238in}{0.648129in}}{\pgfqpoint{0.888288in}{0.648129in}}%
\pgfpathclose%
\pgfusepath{stroke,fill}%
\end{pgfscope}%
\begin{pgfscope}%
\pgfpathrectangle{\pgfqpoint{0.648703in}{0.548769in}}{\pgfqpoint{5.201297in}{3.102590in}}%
\pgfusepath{clip}%
\pgfsetbuttcap%
\pgfsetroundjoin%
\definecolor{currentfill}{rgb}{1.000000,0.498039,0.054902}%
\pgfsetfillcolor{currentfill}%
\pgfsetlinewidth{1.003750pt}%
\definecolor{currentstroke}{rgb}{1.000000,0.498039,0.054902}%
\pgfsetstrokecolor{currentstroke}%
\pgfsetdash{}{0pt}%
\pgfpathmoveto{\pgfqpoint{3.574228in}{3.185343in}}%
\pgfpathcurveto{\pgfqpoint{3.585279in}{3.185343in}}{\pgfqpoint{3.595878in}{3.189733in}}{\pgfqpoint{3.603691in}{3.197547in}}%
\pgfpathcurveto{\pgfqpoint{3.611505in}{3.205360in}}{\pgfqpoint{3.615895in}{3.215959in}}{\pgfqpoint{3.615895in}{3.227010in}}%
\pgfpathcurveto{\pgfqpoint{3.615895in}{3.238060in}}{\pgfqpoint{3.611505in}{3.248659in}}{\pgfqpoint{3.603691in}{3.256472in}}%
\pgfpathcurveto{\pgfqpoint{3.595878in}{3.264286in}}{\pgfqpoint{3.585279in}{3.268676in}}{\pgfqpoint{3.574228in}{3.268676in}}%
\pgfpathcurveto{\pgfqpoint{3.563178in}{3.268676in}}{\pgfqpoint{3.552579in}{3.264286in}}{\pgfqpoint{3.544766in}{3.256472in}}%
\pgfpathcurveto{\pgfqpoint{3.536952in}{3.248659in}}{\pgfqpoint{3.532562in}{3.238060in}}{\pgfqpoint{3.532562in}{3.227010in}}%
\pgfpathcurveto{\pgfqpoint{3.532562in}{3.215959in}}{\pgfqpoint{3.536952in}{3.205360in}}{\pgfqpoint{3.544766in}{3.197547in}}%
\pgfpathcurveto{\pgfqpoint{3.552579in}{3.189733in}}{\pgfqpoint{3.563178in}{3.185343in}}{\pgfqpoint{3.574228in}{3.185343in}}%
\pgfpathclose%
\pgfusepath{stroke,fill}%
\end{pgfscope}%
\begin{pgfscope}%
\pgfpathrectangle{\pgfqpoint{0.648703in}{0.548769in}}{\pgfqpoint{5.201297in}{3.102590in}}%
\pgfusepath{clip}%
\pgfsetbuttcap%
\pgfsetroundjoin%
\definecolor{currentfill}{rgb}{0.121569,0.466667,0.705882}%
\pgfsetfillcolor{currentfill}%
\pgfsetlinewidth{1.003750pt}%
\definecolor{currentstroke}{rgb}{0.121569,0.466667,0.705882}%
\pgfsetstrokecolor{currentstroke}%
\pgfsetdash{}{0pt}%
\pgfpathmoveto{\pgfqpoint{0.890288in}{0.648129in}}%
\pgfpathcurveto{\pgfqpoint{0.901338in}{0.648129in}}{\pgfqpoint{0.911937in}{0.652519in}}{\pgfqpoint{0.919751in}{0.660333in}}%
\pgfpathcurveto{\pgfqpoint{0.927564in}{0.668146in}}{\pgfqpoint{0.931955in}{0.678745in}}{\pgfqpoint{0.931955in}{0.689796in}}%
\pgfpathcurveto{\pgfqpoint{0.931955in}{0.700846in}}{\pgfqpoint{0.927564in}{0.711445in}}{\pgfqpoint{0.919751in}{0.719258in}}%
\pgfpathcurveto{\pgfqpoint{0.911937in}{0.727072in}}{\pgfqpoint{0.901338in}{0.731462in}}{\pgfqpoint{0.890288in}{0.731462in}}%
\pgfpathcurveto{\pgfqpoint{0.879238in}{0.731462in}}{\pgfqpoint{0.868639in}{0.727072in}}{\pgfqpoint{0.860825in}{0.719258in}}%
\pgfpathcurveto{\pgfqpoint{0.853012in}{0.711445in}}{\pgfqpoint{0.848621in}{0.700846in}}{\pgfqpoint{0.848621in}{0.689796in}}%
\pgfpathcurveto{\pgfqpoint{0.848621in}{0.678745in}}{\pgfqpoint{0.853012in}{0.668146in}}{\pgfqpoint{0.860825in}{0.660333in}}%
\pgfpathcurveto{\pgfqpoint{0.868639in}{0.652519in}}{\pgfqpoint{0.879238in}{0.648129in}}{\pgfqpoint{0.890288in}{0.648129in}}%
\pgfpathclose%
\pgfusepath{stroke,fill}%
\end{pgfscope}%
\begin{pgfscope}%
\pgfpathrectangle{\pgfqpoint{0.648703in}{0.548769in}}{\pgfqpoint{5.201297in}{3.102590in}}%
\pgfusepath{clip}%
\pgfsetbuttcap%
\pgfsetroundjoin%
\definecolor{currentfill}{rgb}{1.000000,0.498039,0.054902}%
\pgfsetfillcolor{currentfill}%
\pgfsetlinewidth{1.003750pt}%
\definecolor{currentstroke}{rgb}{1.000000,0.498039,0.054902}%
\pgfsetstrokecolor{currentstroke}%
\pgfsetdash{}{0pt}%
\pgfpathmoveto{\pgfqpoint{4.224344in}{3.185343in}}%
\pgfpathcurveto{\pgfqpoint{4.235394in}{3.185343in}}{\pgfqpoint{4.245993in}{3.189733in}}{\pgfqpoint{4.253807in}{3.197547in}}%
\pgfpathcurveto{\pgfqpoint{4.261620in}{3.205360in}}{\pgfqpoint{4.266010in}{3.215959in}}{\pgfqpoint{4.266010in}{3.227010in}}%
\pgfpathcurveto{\pgfqpoint{4.266010in}{3.238060in}}{\pgfqpoint{4.261620in}{3.248659in}}{\pgfqpoint{4.253807in}{3.256472in}}%
\pgfpathcurveto{\pgfqpoint{4.245993in}{3.264286in}}{\pgfqpoint{4.235394in}{3.268676in}}{\pgfqpoint{4.224344in}{3.268676in}}%
\pgfpathcurveto{\pgfqpoint{4.213294in}{3.268676in}}{\pgfqpoint{4.202695in}{3.264286in}}{\pgfqpoint{4.194881in}{3.256472in}}%
\pgfpathcurveto{\pgfqpoint{4.187067in}{3.248659in}}{\pgfqpoint{4.182677in}{3.238060in}}{\pgfqpoint{4.182677in}{3.227010in}}%
\pgfpathcurveto{\pgfqpoint{4.182677in}{3.215959in}}{\pgfqpoint{4.187067in}{3.205360in}}{\pgfqpoint{4.194881in}{3.197547in}}%
\pgfpathcurveto{\pgfqpoint{4.202695in}{3.189733in}}{\pgfqpoint{4.213294in}{3.185343in}}{\pgfqpoint{4.224344in}{3.185343in}}%
\pgfpathclose%
\pgfusepath{stroke,fill}%
\end{pgfscope}%
\begin{pgfscope}%
\pgfpathrectangle{\pgfqpoint{0.648703in}{0.548769in}}{\pgfqpoint{5.201297in}{3.102590in}}%
\pgfusepath{clip}%
\pgfsetbuttcap%
\pgfsetroundjoin%
\definecolor{currentfill}{rgb}{0.121569,0.466667,0.705882}%
\pgfsetfillcolor{currentfill}%
\pgfsetlinewidth{1.003750pt}%
\definecolor{currentstroke}{rgb}{0.121569,0.466667,0.705882}%
\pgfsetstrokecolor{currentstroke}%
\pgfsetdash{}{0pt}%
\pgfpathmoveto{\pgfqpoint{0.885169in}{0.648129in}}%
\pgfpathcurveto{\pgfqpoint{0.896219in}{0.648129in}}{\pgfqpoint{0.906818in}{0.652519in}}{\pgfqpoint{0.914632in}{0.660333in}}%
\pgfpathcurveto{\pgfqpoint{0.922445in}{0.668146in}}{\pgfqpoint{0.926836in}{0.678745in}}{\pgfqpoint{0.926836in}{0.689796in}}%
\pgfpathcurveto{\pgfqpoint{0.926836in}{0.700846in}}{\pgfqpoint{0.922445in}{0.711445in}}{\pgfqpoint{0.914632in}{0.719258in}}%
\pgfpathcurveto{\pgfqpoint{0.906818in}{0.727072in}}{\pgfqpoint{0.896219in}{0.731462in}}{\pgfqpoint{0.885169in}{0.731462in}}%
\pgfpathcurveto{\pgfqpoint{0.874119in}{0.731462in}}{\pgfqpoint{0.863520in}{0.727072in}}{\pgfqpoint{0.855706in}{0.719258in}}%
\pgfpathcurveto{\pgfqpoint{0.847893in}{0.711445in}}{\pgfqpoint{0.843502in}{0.700846in}}{\pgfqpoint{0.843502in}{0.689796in}}%
\pgfpathcurveto{\pgfqpoint{0.843502in}{0.678745in}}{\pgfqpoint{0.847893in}{0.668146in}}{\pgfqpoint{0.855706in}{0.660333in}}%
\pgfpathcurveto{\pgfqpoint{0.863520in}{0.652519in}}{\pgfqpoint{0.874119in}{0.648129in}}{\pgfqpoint{0.885169in}{0.648129in}}%
\pgfpathclose%
\pgfusepath{stroke,fill}%
\end{pgfscope}%
\begin{pgfscope}%
\pgfpathrectangle{\pgfqpoint{0.648703in}{0.548769in}}{\pgfqpoint{5.201297in}{3.102590in}}%
\pgfusepath{clip}%
\pgfsetbuttcap%
\pgfsetroundjoin%
\definecolor{currentfill}{rgb}{1.000000,0.498039,0.054902}%
\pgfsetfillcolor{currentfill}%
\pgfsetlinewidth{1.003750pt}%
\definecolor{currentstroke}{rgb}{1.000000,0.498039,0.054902}%
\pgfsetstrokecolor{currentstroke}%
\pgfsetdash{}{0pt}%
\pgfpathmoveto{\pgfqpoint{4.032373in}{3.278374in}}%
\pgfpathcurveto{\pgfqpoint{4.043423in}{3.278374in}}{\pgfqpoint{4.054022in}{3.282764in}}{\pgfqpoint{4.061836in}{3.290578in}}%
\pgfpathcurveto{\pgfqpoint{4.069649in}{3.298392in}}{\pgfqpoint{4.074040in}{3.308991in}}{\pgfqpoint{4.074040in}{3.320041in}}%
\pgfpathcurveto{\pgfqpoint{4.074040in}{3.331091in}}{\pgfqpoint{4.069649in}{3.341690in}}{\pgfqpoint{4.061836in}{3.349504in}}%
\pgfpathcurveto{\pgfqpoint{4.054022in}{3.357317in}}{\pgfqpoint{4.043423in}{3.361707in}}{\pgfqpoint{4.032373in}{3.361707in}}%
\pgfpathcurveto{\pgfqpoint{4.021323in}{3.361707in}}{\pgfqpoint{4.010724in}{3.357317in}}{\pgfqpoint{4.002910in}{3.349504in}}%
\pgfpathcurveto{\pgfqpoint{3.995096in}{3.341690in}}{\pgfqpoint{3.990706in}{3.331091in}}{\pgfqpoint{3.990706in}{3.320041in}}%
\pgfpathcurveto{\pgfqpoint{3.990706in}{3.308991in}}{\pgfqpoint{3.995096in}{3.298392in}}{\pgfqpoint{4.002910in}{3.290578in}}%
\pgfpathcurveto{\pgfqpoint{4.010724in}{3.282764in}}{\pgfqpoint{4.021323in}{3.278374in}}{\pgfqpoint{4.032373in}{3.278374in}}%
\pgfpathclose%
\pgfusepath{stroke,fill}%
\end{pgfscope}%
\begin{pgfscope}%
\pgfpathrectangle{\pgfqpoint{0.648703in}{0.548769in}}{\pgfqpoint{5.201297in}{3.102590in}}%
\pgfusepath{clip}%
\pgfsetbuttcap%
\pgfsetroundjoin%
\definecolor{currentfill}{rgb}{0.121569,0.466667,0.705882}%
\pgfsetfillcolor{currentfill}%
\pgfsetlinewidth{1.003750pt}%
\definecolor{currentstroke}{rgb}{0.121569,0.466667,0.705882}%
\pgfsetstrokecolor{currentstroke}%
\pgfsetdash{}{0pt}%
\pgfpathmoveto{\pgfqpoint{0.885130in}{0.648129in}}%
\pgfpathcurveto{\pgfqpoint{0.896180in}{0.648129in}}{\pgfqpoint{0.906779in}{0.652519in}}{\pgfqpoint{0.914593in}{0.660333in}}%
\pgfpathcurveto{\pgfqpoint{0.922406in}{0.668146in}}{\pgfqpoint{0.926797in}{0.678745in}}{\pgfqpoint{0.926797in}{0.689796in}}%
\pgfpathcurveto{\pgfqpoint{0.926797in}{0.700846in}}{\pgfqpoint{0.922406in}{0.711445in}}{\pgfqpoint{0.914593in}{0.719258in}}%
\pgfpathcurveto{\pgfqpoint{0.906779in}{0.727072in}}{\pgfqpoint{0.896180in}{0.731462in}}{\pgfqpoint{0.885130in}{0.731462in}}%
\pgfpathcurveto{\pgfqpoint{0.874080in}{0.731462in}}{\pgfqpoint{0.863481in}{0.727072in}}{\pgfqpoint{0.855667in}{0.719258in}}%
\pgfpathcurveto{\pgfqpoint{0.847854in}{0.711445in}}{\pgfqpoint{0.843463in}{0.700846in}}{\pgfqpoint{0.843463in}{0.689796in}}%
\pgfpathcurveto{\pgfqpoint{0.843463in}{0.678745in}}{\pgfqpoint{0.847854in}{0.668146in}}{\pgfqpoint{0.855667in}{0.660333in}}%
\pgfpathcurveto{\pgfqpoint{0.863481in}{0.652519in}}{\pgfqpoint{0.874080in}{0.648129in}}{\pgfqpoint{0.885130in}{0.648129in}}%
\pgfpathclose%
\pgfusepath{stroke,fill}%
\end{pgfscope}%
\begin{pgfscope}%
\pgfpathrectangle{\pgfqpoint{0.648703in}{0.548769in}}{\pgfqpoint{5.201297in}{3.102590in}}%
\pgfusepath{clip}%
\pgfsetbuttcap%
\pgfsetroundjoin%
\definecolor{currentfill}{rgb}{1.000000,0.498039,0.054902}%
\pgfsetfillcolor{currentfill}%
\pgfsetlinewidth{1.003750pt}%
\definecolor{currentstroke}{rgb}{1.000000,0.498039,0.054902}%
\pgfsetstrokecolor{currentstroke}%
\pgfsetdash{}{0pt}%
\pgfpathmoveto{\pgfqpoint{3.966867in}{3.202258in}}%
\pgfpathcurveto{\pgfqpoint{3.977917in}{3.202258in}}{\pgfqpoint{3.988516in}{3.206648in}}{\pgfqpoint{3.996329in}{3.214462in}}%
\pgfpathcurveto{\pgfqpoint{4.004143in}{3.222275in}}{\pgfqpoint{4.008533in}{3.232874in}}{\pgfqpoint{4.008533in}{3.243924in}}%
\pgfpathcurveto{\pgfqpoint{4.008533in}{3.254974in}}{\pgfqpoint{4.004143in}{3.265573in}}{\pgfqpoint{3.996329in}{3.273387in}}%
\pgfpathcurveto{\pgfqpoint{3.988516in}{3.281201in}}{\pgfqpoint{3.977917in}{3.285591in}}{\pgfqpoint{3.966867in}{3.285591in}}%
\pgfpathcurveto{\pgfqpoint{3.955817in}{3.285591in}}{\pgfqpoint{3.945218in}{3.281201in}}{\pgfqpoint{3.937404in}{3.273387in}}%
\pgfpathcurveto{\pgfqpoint{3.929590in}{3.265573in}}{\pgfqpoint{3.925200in}{3.254974in}}{\pgfqpoint{3.925200in}{3.243924in}}%
\pgfpathcurveto{\pgfqpoint{3.925200in}{3.232874in}}{\pgfqpoint{3.929590in}{3.222275in}}{\pgfqpoint{3.937404in}{3.214462in}}%
\pgfpathcurveto{\pgfqpoint{3.945218in}{3.206648in}}{\pgfqpoint{3.955817in}{3.202258in}}{\pgfqpoint{3.966867in}{3.202258in}}%
\pgfpathclose%
\pgfusepath{stroke,fill}%
\end{pgfscope}%
\begin{pgfscope}%
\pgfpathrectangle{\pgfqpoint{0.648703in}{0.548769in}}{\pgfqpoint{5.201297in}{3.102590in}}%
\pgfusepath{clip}%
\pgfsetbuttcap%
\pgfsetroundjoin%
\definecolor{currentfill}{rgb}{1.000000,0.498039,0.054902}%
\pgfsetfillcolor{currentfill}%
\pgfsetlinewidth{1.003750pt}%
\definecolor{currentstroke}{rgb}{1.000000,0.498039,0.054902}%
\pgfsetstrokecolor{currentstroke}%
\pgfsetdash{}{0pt}%
\pgfpathmoveto{\pgfqpoint{3.730132in}{3.185343in}}%
\pgfpathcurveto{\pgfqpoint{3.741182in}{3.185343in}}{\pgfqpoint{3.751781in}{3.189733in}}{\pgfqpoint{3.759595in}{3.197547in}}%
\pgfpathcurveto{\pgfqpoint{3.767408in}{3.205360in}}{\pgfqpoint{3.771799in}{3.215959in}}{\pgfqpoint{3.771799in}{3.227010in}}%
\pgfpathcurveto{\pgfqpoint{3.771799in}{3.238060in}}{\pgfqpoint{3.767408in}{3.248659in}}{\pgfqpoint{3.759595in}{3.256472in}}%
\pgfpathcurveto{\pgfqpoint{3.751781in}{3.264286in}}{\pgfqpoint{3.741182in}{3.268676in}}{\pgfqpoint{3.730132in}{3.268676in}}%
\pgfpathcurveto{\pgfqpoint{3.719082in}{3.268676in}}{\pgfqpoint{3.708483in}{3.264286in}}{\pgfqpoint{3.700669in}{3.256472in}}%
\pgfpathcurveto{\pgfqpoint{3.692855in}{3.248659in}}{\pgfqpoint{3.688465in}{3.238060in}}{\pgfqpoint{3.688465in}{3.227010in}}%
\pgfpathcurveto{\pgfqpoint{3.688465in}{3.215959in}}{\pgfqpoint{3.692855in}{3.205360in}}{\pgfqpoint{3.700669in}{3.197547in}}%
\pgfpathcurveto{\pgfqpoint{3.708483in}{3.189733in}}{\pgfqpoint{3.719082in}{3.185343in}}{\pgfqpoint{3.730132in}{3.185343in}}%
\pgfpathclose%
\pgfusepath{stroke,fill}%
\end{pgfscope}%
\begin{pgfscope}%
\pgfpathrectangle{\pgfqpoint{0.648703in}{0.548769in}}{\pgfqpoint{5.201297in}{3.102590in}}%
\pgfusepath{clip}%
\pgfsetbuttcap%
\pgfsetroundjoin%
\definecolor{currentfill}{rgb}{0.121569,0.466667,0.705882}%
\pgfsetfillcolor{currentfill}%
\pgfsetlinewidth{1.003750pt}%
\definecolor{currentstroke}{rgb}{0.121569,0.466667,0.705882}%
\pgfsetstrokecolor{currentstroke}%
\pgfsetdash{}{0pt}%
\pgfpathmoveto{\pgfqpoint{0.885152in}{0.648129in}}%
\pgfpathcurveto{\pgfqpoint{0.896202in}{0.648129in}}{\pgfqpoint{0.906801in}{0.652519in}}{\pgfqpoint{0.914615in}{0.660333in}}%
\pgfpathcurveto{\pgfqpoint{0.922429in}{0.668146in}}{\pgfqpoint{0.926819in}{0.678745in}}{\pgfqpoint{0.926819in}{0.689796in}}%
\pgfpathcurveto{\pgfqpoint{0.926819in}{0.700846in}}{\pgfqpoint{0.922429in}{0.711445in}}{\pgfqpoint{0.914615in}{0.719258in}}%
\pgfpathcurveto{\pgfqpoint{0.906801in}{0.727072in}}{\pgfqpoint{0.896202in}{0.731462in}}{\pgfqpoint{0.885152in}{0.731462in}}%
\pgfpathcurveto{\pgfqpoint{0.874102in}{0.731462in}}{\pgfqpoint{0.863503in}{0.727072in}}{\pgfqpoint{0.855689in}{0.719258in}}%
\pgfpathcurveto{\pgfqpoint{0.847876in}{0.711445in}}{\pgfqpoint{0.843485in}{0.700846in}}{\pgfqpoint{0.843485in}{0.689796in}}%
\pgfpathcurveto{\pgfqpoint{0.843485in}{0.678745in}}{\pgfqpoint{0.847876in}{0.668146in}}{\pgfqpoint{0.855689in}{0.660333in}}%
\pgfpathcurveto{\pgfqpoint{0.863503in}{0.652519in}}{\pgfqpoint{0.874102in}{0.648129in}}{\pgfqpoint{0.885152in}{0.648129in}}%
\pgfpathclose%
\pgfusepath{stroke,fill}%
\end{pgfscope}%
\begin{pgfscope}%
\pgfpathrectangle{\pgfqpoint{0.648703in}{0.548769in}}{\pgfqpoint{5.201297in}{3.102590in}}%
\pgfusepath{clip}%
\pgfsetbuttcap%
\pgfsetroundjoin%
\definecolor{currentfill}{rgb}{1.000000,0.498039,0.054902}%
\pgfsetfillcolor{currentfill}%
\pgfsetlinewidth{1.003750pt}%
\definecolor{currentstroke}{rgb}{1.000000,0.498039,0.054902}%
\pgfsetstrokecolor{currentstroke}%
\pgfsetdash{}{0pt}%
\pgfpathmoveto{\pgfqpoint{5.086885in}{3.185343in}}%
\pgfpathcurveto{\pgfqpoint{5.097935in}{3.185343in}}{\pgfqpoint{5.108534in}{3.189733in}}{\pgfqpoint{5.116347in}{3.197547in}}%
\pgfpathcurveto{\pgfqpoint{5.124161in}{3.205360in}}{\pgfqpoint{5.128551in}{3.215959in}}{\pgfqpoint{5.128551in}{3.227010in}}%
\pgfpathcurveto{\pgfqpoint{5.128551in}{3.238060in}}{\pgfqpoint{5.124161in}{3.248659in}}{\pgfqpoint{5.116347in}{3.256472in}}%
\pgfpathcurveto{\pgfqpoint{5.108534in}{3.264286in}}{\pgfqpoint{5.097935in}{3.268676in}}{\pgfqpoint{5.086885in}{3.268676in}}%
\pgfpathcurveto{\pgfqpoint{5.075835in}{3.268676in}}{\pgfqpoint{5.065235in}{3.264286in}}{\pgfqpoint{5.057422in}{3.256472in}}%
\pgfpathcurveto{\pgfqpoint{5.049608in}{3.248659in}}{\pgfqpoint{5.045218in}{3.238060in}}{\pgfqpoint{5.045218in}{3.227010in}}%
\pgfpathcurveto{\pgfqpoint{5.045218in}{3.215959in}}{\pgfqpoint{5.049608in}{3.205360in}}{\pgfqpoint{5.057422in}{3.197547in}}%
\pgfpathcurveto{\pgfqpoint{5.065235in}{3.189733in}}{\pgfqpoint{5.075835in}{3.185343in}}{\pgfqpoint{5.086885in}{3.185343in}}%
\pgfpathclose%
\pgfusepath{stroke,fill}%
\end{pgfscope}%
\begin{pgfscope}%
\pgfpathrectangle{\pgfqpoint{0.648703in}{0.548769in}}{\pgfqpoint{5.201297in}{3.102590in}}%
\pgfusepath{clip}%
\pgfsetbuttcap%
\pgfsetroundjoin%
\definecolor{currentfill}{rgb}{1.000000,0.498039,0.054902}%
\pgfsetfillcolor{currentfill}%
\pgfsetlinewidth{1.003750pt}%
\definecolor{currentstroke}{rgb}{1.000000,0.498039,0.054902}%
\pgfsetstrokecolor{currentstroke}%
\pgfsetdash{}{0pt}%
\pgfpathmoveto{\pgfqpoint{5.051376in}{3.193800in}}%
\pgfpathcurveto{\pgfqpoint{5.062426in}{3.193800in}}{\pgfqpoint{5.073025in}{3.198191in}}{\pgfqpoint{5.080838in}{3.206004in}}%
\pgfpathcurveto{\pgfqpoint{5.088652in}{3.213818in}}{\pgfqpoint{5.093042in}{3.224417in}}{\pgfqpoint{5.093042in}{3.235467in}}%
\pgfpathcurveto{\pgfqpoint{5.093042in}{3.246517in}}{\pgfqpoint{5.088652in}{3.257116in}}{\pgfqpoint{5.080838in}{3.264930in}}%
\pgfpathcurveto{\pgfqpoint{5.073025in}{3.272743in}}{\pgfqpoint{5.062426in}{3.277134in}}{\pgfqpoint{5.051376in}{3.277134in}}%
\pgfpathcurveto{\pgfqpoint{5.040325in}{3.277134in}}{\pgfqpoint{5.029726in}{3.272743in}}{\pgfqpoint{5.021913in}{3.264930in}}%
\pgfpathcurveto{\pgfqpoint{5.014099in}{3.257116in}}{\pgfqpoint{5.009709in}{3.246517in}}{\pgfqpoint{5.009709in}{3.235467in}}%
\pgfpathcurveto{\pgfqpoint{5.009709in}{3.224417in}}{\pgfqpoint{5.014099in}{3.213818in}}{\pgfqpoint{5.021913in}{3.206004in}}%
\pgfpathcurveto{\pgfqpoint{5.029726in}{3.198191in}}{\pgfqpoint{5.040325in}{3.193800in}}{\pgfqpoint{5.051376in}{3.193800in}}%
\pgfpathclose%
\pgfusepath{stroke,fill}%
\end{pgfscope}%
\begin{pgfscope}%
\pgfpathrectangle{\pgfqpoint{0.648703in}{0.548769in}}{\pgfqpoint{5.201297in}{3.102590in}}%
\pgfusepath{clip}%
\pgfsetbuttcap%
\pgfsetroundjoin%
\definecolor{currentfill}{rgb}{0.121569,0.466667,0.705882}%
\pgfsetfillcolor{currentfill}%
\pgfsetlinewidth{1.003750pt}%
\definecolor{currentstroke}{rgb}{0.121569,0.466667,0.705882}%
\pgfsetstrokecolor{currentstroke}%
\pgfsetdash{}{0pt}%
\pgfpathmoveto{\pgfqpoint{0.906983in}{0.656586in}}%
\pgfpathcurveto{\pgfqpoint{0.918033in}{0.656586in}}{\pgfqpoint{0.928632in}{0.660977in}}{\pgfqpoint{0.936446in}{0.668790in}}%
\pgfpathcurveto{\pgfqpoint{0.944260in}{0.676604in}}{\pgfqpoint{0.948650in}{0.687203in}}{\pgfqpoint{0.948650in}{0.698253in}}%
\pgfpathcurveto{\pgfqpoint{0.948650in}{0.709303in}}{\pgfqpoint{0.944260in}{0.719902in}}{\pgfqpoint{0.936446in}{0.727716in}}%
\pgfpathcurveto{\pgfqpoint{0.928632in}{0.735529in}}{\pgfqpoint{0.918033in}{0.739920in}}{\pgfqpoint{0.906983in}{0.739920in}}%
\pgfpathcurveto{\pgfqpoint{0.895933in}{0.739920in}}{\pgfqpoint{0.885334in}{0.735529in}}{\pgfqpoint{0.877520in}{0.727716in}}%
\pgfpathcurveto{\pgfqpoint{0.869707in}{0.719902in}}{\pgfqpoint{0.865316in}{0.709303in}}{\pgfqpoint{0.865316in}{0.698253in}}%
\pgfpathcurveto{\pgfqpoint{0.865316in}{0.687203in}}{\pgfqpoint{0.869707in}{0.676604in}}{\pgfqpoint{0.877520in}{0.668790in}}%
\pgfpathcurveto{\pgfqpoint{0.885334in}{0.660977in}}{\pgfqpoint{0.895933in}{0.656586in}}{\pgfqpoint{0.906983in}{0.656586in}}%
\pgfpathclose%
\pgfusepath{stroke,fill}%
\end{pgfscope}%
\begin{pgfscope}%
\pgfpathrectangle{\pgfqpoint{0.648703in}{0.548769in}}{\pgfqpoint{5.201297in}{3.102590in}}%
\pgfusepath{clip}%
\pgfsetbuttcap%
\pgfsetroundjoin%
\definecolor{currentfill}{rgb}{1.000000,0.498039,0.054902}%
\pgfsetfillcolor{currentfill}%
\pgfsetlinewidth{1.003750pt}%
\definecolor{currentstroke}{rgb}{1.000000,0.498039,0.054902}%
\pgfsetstrokecolor{currentstroke}%
\pgfsetdash{}{0pt}%
\pgfpathmoveto{\pgfqpoint{4.520000in}{3.214944in}}%
\pgfpathcurveto{\pgfqpoint{4.531050in}{3.214944in}}{\pgfqpoint{4.541649in}{3.219334in}}{\pgfqpoint{4.549463in}{3.227148in}}%
\pgfpathcurveto{\pgfqpoint{4.557276in}{3.234961in}}{\pgfqpoint{4.561666in}{3.245560in}}{\pgfqpoint{4.561666in}{3.256610in}}%
\pgfpathcurveto{\pgfqpoint{4.561666in}{3.267661in}}{\pgfqpoint{4.557276in}{3.278260in}}{\pgfqpoint{4.549463in}{3.286073in}}%
\pgfpathcurveto{\pgfqpoint{4.541649in}{3.293887in}}{\pgfqpoint{4.531050in}{3.298277in}}{\pgfqpoint{4.520000in}{3.298277in}}%
\pgfpathcurveto{\pgfqpoint{4.508950in}{3.298277in}}{\pgfqpoint{4.498351in}{3.293887in}}{\pgfqpoint{4.490537in}{3.286073in}}%
\pgfpathcurveto{\pgfqpoint{4.482723in}{3.278260in}}{\pgfqpoint{4.478333in}{3.267661in}}{\pgfqpoint{4.478333in}{3.256610in}}%
\pgfpathcurveto{\pgfqpoint{4.478333in}{3.245560in}}{\pgfqpoint{4.482723in}{3.234961in}}{\pgfqpoint{4.490537in}{3.227148in}}%
\pgfpathcurveto{\pgfqpoint{4.498351in}{3.219334in}}{\pgfqpoint{4.508950in}{3.214944in}}{\pgfqpoint{4.520000in}{3.214944in}}%
\pgfpathclose%
\pgfusepath{stroke,fill}%
\end{pgfscope}%
\begin{pgfscope}%
\pgfpathrectangle{\pgfqpoint{0.648703in}{0.548769in}}{\pgfqpoint{5.201297in}{3.102590in}}%
\pgfusepath{clip}%
\pgfsetbuttcap%
\pgfsetroundjoin%
\definecolor{currentfill}{rgb}{1.000000,0.498039,0.054902}%
\pgfsetfillcolor{currentfill}%
\pgfsetlinewidth{1.003750pt}%
\definecolor{currentstroke}{rgb}{1.000000,0.498039,0.054902}%
\pgfsetstrokecolor{currentstroke}%
\pgfsetdash{}{0pt}%
\pgfpathmoveto{\pgfqpoint{4.583676in}{3.202258in}}%
\pgfpathcurveto{\pgfqpoint{4.594726in}{3.202258in}}{\pgfqpoint{4.605325in}{3.206648in}}{\pgfqpoint{4.613138in}{3.214462in}}%
\pgfpathcurveto{\pgfqpoint{4.620952in}{3.222275in}}{\pgfqpoint{4.625342in}{3.232874in}}{\pgfqpoint{4.625342in}{3.243924in}}%
\pgfpathcurveto{\pgfqpoint{4.625342in}{3.254974in}}{\pgfqpoint{4.620952in}{3.265573in}}{\pgfqpoint{4.613138in}{3.273387in}}%
\pgfpathcurveto{\pgfqpoint{4.605325in}{3.281201in}}{\pgfqpoint{4.594726in}{3.285591in}}{\pgfqpoint{4.583676in}{3.285591in}}%
\pgfpathcurveto{\pgfqpoint{4.572626in}{3.285591in}}{\pgfqpoint{4.562027in}{3.281201in}}{\pgfqpoint{4.554213in}{3.273387in}}%
\pgfpathcurveto{\pgfqpoint{4.546399in}{3.265573in}}{\pgfqpoint{4.542009in}{3.254974in}}{\pgfqpoint{4.542009in}{3.243924in}}%
\pgfpathcurveto{\pgfqpoint{4.542009in}{3.232874in}}{\pgfqpoint{4.546399in}{3.222275in}}{\pgfqpoint{4.554213in}{3.214462in}}%
\pgfpathcurveto{\pgfqpoint{4.562027in}{3.206648in}}{\pgfqpoint{4.572626in}{3.202258in}}{\pgfqpoint{4.583676in}{3.202258in}}%
\pgfpathclose%
\pgfusepath{stroke,fill}%
\end{pgfscope}%
\begin{pgfscope}%
\pgfpathrectangle{\pgfqpoint{0.648703in}{0.548769in}}{\pgfqpoint{5.201297in}{3.102590in}}%
\pgfusepath{clip}%
\pgfsetbuttcap%
\pgfsetroundjoin%
\definecolor{currentfill}{rgb}{1.000000,0.498039,0.054902}%
\pgfsetfillcolor{currentfill}%
\pgfsetlinewidth{1.003750pt}%
\definecolor{currentstroke}{rgb}{1.000000,0.498039,0.054902}%
\pgfsetstrokecolor{currentstroke}%
\pgfsetdash{}{0pt}%
\pgfpathmoveto{\pgfqpoint{3.996244in}{3.244545in}}%
\pgfpathcurveto{\pgfqpoint{4.007295in}{3.244545in}}{\pgfqpoint{4.017894in}{3.248935in}}{\pgfqpoint{4.025707in}{3.256748in}}%
\pgfpathcurveto{\pgfqpoint{4.033521in}{3.264562in}}{\pgfqpoint{4.037911in}{3.275161in}}{\pgfqpoint{4.037911in}{3.286211in}}%
\pgfpathcurveto{\pgfqpoint{4.037911in}{3.297261in}}{\pgfqpoint{4.033521in}{3.307860in}}{\pgfqpoint{4.025707in}{3.315674in}}%
\pgfpathcurveto{\pgfqpoint{4.017894in}{3.323488in}}{\pgfqpoint{4.007295in}{3.327878in}}{\pgfqpoint{3.996244in}{3.327878in}}%
\pgfpathcurveto{\pgfqpoint{3.985194in}{3.327878in}}{\pgfqpoint{3.974595in}{3.323488in}}{\pgfqpoint{3.966782in}{3.315674in}}%
\pgfpathcurveto{\pgfqpoint{3.958968in}{3.307860in}}{\pgfqpoint{3.954578in}{3.297261in}}{\pgfqpoint{3.954578in}{3.286211in}}%
\pgfpathcurveto{\pgfqpoint{3.954578in}{3.275161in}}{\pgfqpoint{3.958968in}{3.264562in}}{\pgfqpoint{3.966782in}{3.256748in}}%
\pgfpathcurveto{\pgfqpoint{3.974595in}{3.248935in}}{\pgfqpoint{3.985194in}{3.244545in}}{\pgfqpoint{3.996244in}{3.244545in}}%
\pgfpathclose%
\pgfusepath{stroke,fill}%
\end{pgfscope}%
\begin{pgfscope}%
\pgfpathrectangle{\pgfqpoint{0.648703in}{0.548769in}}{\pgfqpoint{5.201297in}{3.102590in}}%
\pgfusepath{clip}%
\pgfsetbuttcap%
\pgfsetroundjoin%
\definecolor{currentfill}{rgb}{1.000000,0.498039,0.054902}%
\pgfsetfillcolor{currentfill}%
\pgfsetlinewidth{1.003750pt}%
\definecolor{currentstroke}{rgb}{1.000000,0.498039,0.054902}%
\pgfsetstrokecolor{currentstroke}%
\pgfsetdash{}{0pt}%
\pgfpathmoveto{\pgfqpoint{4.211434in}{3.312204in}}%
\pgfpathcurveto{\pgfqpoint{4.222484in}{3.312204in}}{\pgfqpoint{4.233083in}{3.316594in}}{\pgfqpoint{4.240896in}{3.324407in}}%
\pgfpathcurveto{\pgfqpoint{4.248710in}{3.332221in}}{\pgfqpoint{4.253100in}{3.342820in}}{\pgfqpoint{4.253100in}{3.353870in}}%
\pgfpathcurveto{\pgfqpoint{4.253100in}{3.364920in}}{\pgfqpoint{4.248710in}{3.375519in}}{\pgfqpoint{4.240896in}{3.383333in}}%
\pgfpathcurveto{\pgfqpoint{4.233083in}{3.391147in}}{\pgfqpoint{4.222484in}{3.395537in}}{\pgfqpoint{4.211434in}{3.395537in}}%
\pgfpathcurveto{\pgfqpoint{4.200384in}{3.395537in}}{\pgfqpoint{4.189785in}{3.391147in}}{\pgfqpoint{4.181971in}{3.383333in}}%
\pgfpathcurveto{\pgfqpoint{4.174157in}{3.375519in}}{\pgfqpoint{4.169767in}{3.364920in}}{\pgfqpoint{4.169767in}{3.353870in}}%
\pgfpathcurveto{\pgfqpoint{4.169767in}{3.342820in}}{\pgfqpoint{4.174157in}{3.332221in}}{\pgfqpoint{4.181971in}{3.324407in}}%
\pgfpathcurveto{\pgfqpoint{4.189785in}{3.316594in}}{\pgfqpoint{4.200384in}{3.312204in}}{\pgfqpoint{4.211434in}{3.312204in}}%
\pgfpathclose%
\pgfusepath{stroke,fill}%
\end{pgfscope}%
\begin{pgfscope}%
\pgfpathrectangle{\pgfqpoint{0.648703in}{0.548769in}}{\pgfqpoint{5.201297in}{3.102590in}}%
\pgfusepath{clip}%
\pgfsetbuttcap%
\pgfsetroundjoin%
\definecolor{currentfill}{rgb}{1.000000,0.498039,0.054902}%
\pgfsetfillcolor{currentfill}%
\pgfsetlinewidth{1.003750pt}%
\definecolor{currentstroke}{rgb}{1.000000,0.498039,0.054902}%
\pgfsetstrokecolor{currentstroke}%
\pgfsetdash{}{0pt}%
\pgfpathmoveto{\pgfqpoint{4.089988in}{3.210715in}}%
\pgfpathcurveto{\pgfqpoint{4.101038in}{3.210715in}}{\pgfqpoint{4.111637in}{3.215105in}}{\pgfqpoint{4.119450in}{3.222919in}}%
\pgfpathcurveto{\pgfqpoint{4.127264in}{3.230733in}}{\pgfqpoint{4.131654in}{3.241332in}}{\pgfqpoint{4.131654in}{3.252382in}}%
\pgfpathcurveto{\pgfqpoint{4.131654in}{3.263432in}}{\pgfqpoint{4.127264in}{3.274031in}}{\pgfqpoint{4.119450in}{3.281844in}}%
\pgfpathcurveto{\pgfqpoint{4.111637in}{3.289658in}}{\pgfqpoint{4.101038in}{3.294048in}}{\pgfqpoint{4.089988in}{3.294048in}}%
\pgfpathcurveto{\pgfqpoint{4.078937in}{3.294048in}}{\pgfqpoint{4.068338in}{3.289658in}}{\pgfqpoint{4.060525in}{3.281844in}}%
\pgfpathcurveto{\pgfqpoint{4.052711in}{3.274031in}}{\pgfqpoint{4.048321in}{3.263432in}}{\pgfqpoint{4.048321in}{3.252382in}}%
\pgfpathcurveto{\pgfqpoint{4.048321in}{3.241332in}}{\pgfqpoint{4.052711in}{3.230733in}}{\pgfqpoint{4.060525in}{3.222919in}}%
\pgfpathcurveto{\pgfqpoint{4.068338in}{3.215105in}}{\pgfqpoint{4.078937in}{3.210715in}}{\pgfqpoint{4.089988in}{3.210715in}}%
\pgfpathclose%
\pgfusepath{stroke,fill}%
\end{pgfscope}%
\begin{pgfscope}%
\pgfpathrectangle{\pgfqpoint{0.648703in}{0.548769in}}{\pgfqpoint{5.201297in}{3.102590in}}%
\pgfusepath{clip}%
\pgfsetbuttcap%
\pgfsetroundjoin%
\definecolor{currentfill}{rgb}{1.000000,0.498039,0.054902}%
\pgfsetfillcolor{currentfill}%
\pgfsetlinewidth{1.003750pt}%
\definecolor{currentstroke}{rgb}{1.000000,0.498039,0.054902}%
\pgfsetstrokecolor{currentstroke}%
\pgfsetdash{}{0pt}%
\pgfpathmoveto{\pgfqpoint{4.476608in}{3.236087in}}%
\pgfpathcurveto{\pgfqpoint{4.487658in}{3.236087in}}{\pgfqpoint{4.498257in}{3.240477in}}{\pgfqpoint{4.506071in}{3.248291in}}%
\pgfpathcurveto{\pgfqpoint{4.513885in}{3.256105in}}{\pgfqpoint{4.518275in}{3.266704in}}{\pgfqpoint{4.518275in}{3.277754in}}%
\pgfpathcurveto{\pgfqpoint{4.518275in}{3.288804in}}{\pgfqpoint{4.513885in}{3.299403in}}{\pgfqpoint{4.506071in}{3.307217in}}%
\pgfpathcurveto{\pgfqpoint{4.498257in}{3.315030in}}{\pgfqpoint{4.487658in}{3.319421in}}{\pgfqpoint{4.476608in}{3.319421in}}%
\pgfpathcurveto{\pgfqpoint{4.465558in}{3.319421in}}{\pgfqpoint{4.454959in}{3.315030in}}{\pgfqpoint{4.447145in}{3.307217in}}%
\pgfpathcurveto{\pgfqpoint{4.439332in}{3.299403in}}{\pgfqpoint{4.434942in}{3.288804in}}{\pgfqpoint{4.434942in}{3.277754in}}%
\pgfpathcurveto{\pgfqpoint{4.434942in}{3.266704in}}{\pgfqpoint{4.439332in}{3.256105in}}{\pgfqpoint{4.447145in}{3.248291in}}%
\pgfpathcurveto{\pgfqpoint{4.454959in}{3.240477in}}{\pgfqpoint{4.465558in}{3.236087in}}{\pgfqpoint{4.476608in}{3.236087in}}%
\pgfpathclose%
\pgfusepath{stroke,fill}%
\end{pgfscope}%
\begin{pgfscope}%
\pgfpathrectangle{\pgfqpoint{0.648703in}{0.548769in}}{\pgfqpoint{5.201297in}{3.102590in}}%
\pgfusepath{clip}%
\pgfsetbuttcap%
\pgfsetroundjoin%
\definecolor{currentfill}{rgb}{0.121569,0.466667,0.705882}%
\pgfsetfillcolor{currentfill}%
\pgfsetlinewidth{1.003750pt}%
\definecolor{currentstroke}{rgb}{0.121569,0.466667,0.705882}%
\pgfsetstrokecolor{currentstroke}%
\pgfsetdash{}{0pt}%
\pgfpathmoveto{\pgfqpoint{0.885163in}{0.648129in}}%
\pgfpathcurveto{\pgfqpoint{0.896213in}{0.648129in}}{\pgfqpoint{0.906812in}{0.652519in}}{\pgfqpoint{0.914626in}{0.660333in}}%
\pgfpathcurveto{\pgfqpoint{0.922440in}{0.668146in}}{\pgfqpoint{0.926830in}{0.678745in}}{\pgfqpoint{0.926830in}{0.689796in}}%
\pgfpathcurveto{\pgfqpoint{0.926830in}{0.700846in}}{\pgfqpoint{0.922440in}{0.711445in}}{\pgfqpoint{0.914626in}{0.719258in}}%
\pgfpathcurveto{\pgfqpoint{0.906812in}{0.727072in}}{\pgfqpoint{0.896213in}{0.731462in}}{\pgfqpoint{0.885163in}{0.731462in}}%
\pgfpathcurveto{\pgfqpoint{0.874113in}{0.731462in}}{\pgfqpoint{0.863514in}{0.727072in}}{\pgfqpoint{0.855701in}{0.719258in}}%
\pgfpathcurveto{\pgfqpoint{0.847887in}{0.711445in}}{\pgfqpoint{0.843497in}{0.700846in}}{\pgfqpoint{0.843497in}{0.689796in}}%
\pgfpathcurveto{\pgfqpoint{0.843497in}{0.678745in}}{\pgfqpoint{0.847887in}{0.668146in}}{\pgfqpoint{0.855701in}{0.660333in}}%
\pgfpathcurveto{\pgfqpoint{0.863514in}{0.652519in}}{\pgfqpoint{0.874113in}{0.648129in}}{\pgfqpoint{0.885163in}{0.648129in}}%
\pgfpathclose%
\pgfusepath{stroke,fill}%
\end{pgfscope}%
\begin{pgfscope}%
\pgfpathrectangle{\pgfqpoint{0.648703in}{0.548769in}}{\pgfqpoint{5.201297in}{3.102590in}}%
\pgfusepath{clip}%
\pgfsetbuttcap%
\pgfsetroundjoin%
\definecolor{currentfill}{rgb}{0.121569,0.466667,0.705882}%
\pgfsetfillcolor{currentfill}%
\pgfsetlinewidth{1.003750pt}%
\definecolor{currentstroke}{rgb}{0.121569,0.466667,0.705882}%
\pgfsetstrokecolor{currentstroke}%
\pgfsetdash{}{0pt}%
\pgfpathmoveto{\pgfqpoint{0.885172in}{0.648129in}}%
\pgfpathcurveto{\pgfqpoint{0.896222in}{0.648129in}}{\pgfqpoint{0.906821in}{0.652519in}}{\pgfqpoint{0.914634in}{0.660333in}}%
\pgfpathcurveto{\pgfqpoint{0.922448in}{0.668146in}}{\pgfqpoint{0.926838in}{0.678745in}}{\pgfqpoint{0.926838in}{0.689796in}}%
\pgfpathcurveto{\pgfqpoint{0.926838in}{0.700846in}}{\pgfqpoint{0.922448in}{0.711445in}}{\pgfqpoint{0.914634in}{0.719258in}}%
\pgfpathcurveto{\pgfqpoint{0.906821in}{0.727072in}}{\pgfqpoint{0.896222in}{0.731462in}}{\pgfqpoint{0.885172in}{0.731462in}}%
\pgfpathcurveto{\pgfqpoint{0.874121in}{0.731462in}}{\pgfqpoint{0.863522in}{0.727072in}}{\pgfqpoint{0.855709in}{0.719258in}}%
\pgfpathcurveto{\pgfqpoint{0.847895in}{0.711445in}}{\pgfqpoint{0.843505in}{0.700846in}}{\pgfqpoint{0.843505in}{0.689796in}}%
\pgfpathcurveto{\pgfqpoint{0.843505in}{0.678745in}}{\pgfqpoint{0.847895in}{0.668146in}}{\pgfqpoint{0.855709in}{0.660333in}}%
\pgfpathcurveto{\pgfqpoint{0.863522in}{0.652519in}}{\pgfqpoint{0.874121in}{0.648129in}}{\pgfqpoint{0.885172in}{0.648129in}}%
\pgfpathclose%
\pgfusepath{stroke,fill}%
\end{pgfscope}%
\begin{pgfscope}%
\pgfpathrectangle{\pgfqpoint{0.648703in}{0.548769in}}{\pgfqpoint{5.201297in}{3.102590in}}%
\pgfusepath{clip}%
\pgfsetbuttcap%
\pgfsetroundjoin%
\definecolor{currentfill}{rgb}{0.121569,0.466667,0.705882}%
\pgfsetfillcolor{currentfill}%
\pgfsetlinewidth{1.003750pt}%
\definecolor{currentstroke}{rgb}{0.121569,0.466667,0.705882}%
\pgfsetstrokecolor{currentstroke}%
\pgfsetdash{}{0pt}%
\pgfpathmoveto{\pgfqpoint{0.885158in}{0.648129in}}%
\pgfpathcurveto{\pgfqpoint{0.896208in}{0.648129in}}{\pgfqpoint{0.906807in}{0.652519in}}{\pgfqpoint{0.914621in}{0.660333in}}%
\pgfpathcurveto{\pgfqpoint{0.922434in}{0.668146in}}{\pgfqpoint{0.926825in}{0.678745in}}{\pgfqpoint{0.926825in}{0.689796in}}%
\pgfpathcurveto{\pgfqpoint{0.926825in}{0.700846in}}{\pgfqpoint{0.922434in}{0.711445in}}{\pgfqpoint{0.914621in}{0.719258in}}%
\pgfpathcurveto{\pgfqpoint{0.906807in}{0.727072in}}{\pgfqpoint{0.896208in}{0.731462in}}{\pgfqpoint{0.885158in}{0.731462in}}%
\pgfpathcurveto{\pgfqpoint{0.874108in}{0.731462in}}{\pgfqpoint{0.863509in}{0.727072in}}{\pgfqpoint{0.855695in}{0.719258in}}%
\pgfpathcurveto{\pgfqpoint{0.847882in}{0.711445in}}{\pgfqpoint{0.843491in}{0.700846in}}{\pgfqpoint{0.843491in}{0.689796in}}%
\pgfpathcurveto{\pgfqpoint{0.843491in}{0.678745in}}{\pgfqpoint{0.847882in}{0.668146in}}{\pgfqpoint{0.855695in}{0.660333in}}%
\pgfpathcurveto{\pgfqpoint{0.863509in}{0.652519in}}{\pgfqpoint{0.874108in}{0.648129in}}{\pgfqpoint{0.885158in}{0.648129in}}%
\pgfpathclose%
\pgfusepath{stroke,fill}%
\end{pgfscope}%
\begin{pgfscope}%
\pgfpathrectangle{\pgfqpoint{0.648703in}{0.548769in}}{\pgfqpoint{5.201297in}{3.102590in}}%
\pgfusepath{clip}%
\pgfsetbuttcap%
\pgfsetroundjoin%
\definecolor{currentfill}{rgb}{1.000000,0.498039,0.054902}%
\pgfsetfillcolor{currentfill}%
\pgfsetlinewidth{1.003750pt}%
\definecolor{currentstroke}{rgb}{1.000000,0.498039,0.054902}%
\pgfsetstrokecolor{currentstroke}%
\pgfsetdash{}{0pt}%
\pgfpathmoveto{\pgfqpoint{4.409622in}{3.405235in}}%
\pgfpathcurveto{\pgfqpoint{4.420672in}{3.405235in}}{\pgfqpoint{4.431271in}{3.409625in}}{\pgfqpoint{4.439084in}{3.417439in}}%
\pgfpathcurveto{\pgfqpoint{4.446898in}{3.425252in}}{\pgfqpoint{4.451288in}{3.435851in}}{\pgfqpoint{4.451288in}{3.446901in}}%
\pgfpathcurveto{\pgfqpoint{4.451288in}{3.457952in}}{\pgfqpoint{4.446898in}{3.468551in}}{\pgfqpoint{4.439084in}{3.476364in}}%
\pgfpathcurveto{\pgfqpoint{4.431271in}{3.484178in}}{\pgfqpoint{4.420672in}{3.488568in}}{\pgfqpoint{4.409622in}{3.488568in}}%
\pgfpathcurveto{\pgfqpoint{4.398571in}{3.488568in}}{\pgfqpoint{4.387972in}{3.484178in}}{\pgfqpoint{4.380159in}{3.476364in}}%
\pgfpathcurveto{\pgfqpoint{4.372345in}{3.468551in}}{\pgfqpoint{4.367955in}{3.457952in}}{\pgfqpoint{4.367955in}{3.446901in}}%
\pgfpathcurveto{\pgfqpoint{4.367955in}{3.435851in}}{\pgfqpoint{4.372345in}{3.425252in}}{\pgfqpoint{4.380159in}{3.417439in}}%
\pgfpathcurveto{\pgfqpoint{4.387972in}{3.409625in}}{\pgfqpoint{4.398571in}{3.405235in}}{\pgfqpoint{4.409622in}{3.405235in}}%
\pgfpathclose%
\pgfusepath{stroke,fill}%
\end{pgfscope}%
\begin{pgfscope}%
\pgfpathrectangle{\pgfqpoint{0.648703in}{0.548769in}}{\pgfqpoint{5.201297in}{3.102590in}}%
\pgfusepath{clip}%
\pgfsetbuttcap%
\pgfsetroundjoin%
\definecolor{currentfill}{rgb}{1.000000,0.498039,0.054902}%
\pgfsetfillcolor{currentfill}%
\pgfsetlinewidth{1.003750pt}%
\definecolor{currentstroke}{rgb}{1.000000,0.498039,0.054902}%
\pgfsetstrokecolor{currentstroke}%
\pgfsetdash{}{0pt}%
\pgfpathmoveto{\pgfqpoint{4.555762in}{3.193800in}}%
\pgfpathcurveto{\pgfqpoint{4.566813in}{3.193800in}}{\pgfqpoint{4.577412in}{3.198191in}}{\pgfqpoint{4.585225in}{3.206004in}}%
\pgfpathcurveto{\pgfqpoint{4.593039in}{3.213818in}}{\pgfqpoint{4.597429in}{3.224417in}}{\pgfqpoint{4.597429in}{3.235467in}}%
\pgfpathcurveto{\pgfqpoint{4.597429in}{3.246517in}}{\pgfqpoint{4.593039in}{3.257116in}}{\pgfqpoint{4.585225in}{3.264930in}}%
\pgfpathcurveto{\pgfqpoint{4.577412in}{3.272743in}}{\pgfqpoint{4.566813in}{3.277134in}}{\pgfqpoint{4.555762in}{3.277134in}}%
\pgfpathcurveto{\pgfqpoint{4.544712in}{3.277134in}}{\pgfqpoint{4.534113in}{3.272743in}}{\pgfqpoint{4.526300in}{3.264930in}}%
\pgfpathcurveto{\pgfqpoint{4.518486in}{3.257116in}}{\pgfqpoint{4.514096in}{3.246517in}}{\pgfqpoint{4.514096in}{3.235467in}}%
\pgfpathcurveto{\pgfqpoint{4.514096in}{3.224417in}}{\pgfqpoint{4.518486in}{3.213818in}}{\pgfqpoint{4.526300in}{3.206004in}}%
\pgfpathcurveto{\pgfqpoint{4.534113in}{3.198191in}}{\pgfqpoint{4.544712in}{3.193800in}}{\pgfqpoint{4.555762in}{3.193800in}}%
\pgfpathclose%
\pgfusepath{stroke,fill}%
\end{pgfscope}%
\begin{pgfscope}%
\pgfpathrectangle{\pgfqpoint{0.648703in}{0.548769in}}{\pgfqpoint{5.201297in}{3.102590in}}%
\pgfusepath{clip}%
\pgfsetbuttcap%
\pgfsetroundjoin%
\definecolor{currentfill}{rgb}{1.000000,0.498039,0.054902}%
\pgfsetfillcolor{currentfill}%
\pgfsetlinewidth{1.003750pt}%
\definecolor{currentstroke}{rgb}{1.000000,0.498039,0.054902}%
\pgfsetstrokecolor{currentstroke}%
\pgfsetdash{}{0pt}%
\pgfpathmoveto{\pgfqpoint{4.048129in}{3.358719in}}%
\pgfpathcurveto{\pgfqpoint{4.059179in}{3.358719in}}{\pgfqpoint{4.069778in}{3.363109in}}{\pgfqpoint{4.077591in}{3.370923in}}%
\pgfpathcurveto{\pgfqpoint{4.085405in}{3.378737in}}{\pgfqpoint{4.089795in}{3.389336in}}{\pgfqpoint{4.089795in}{3.400386in}}%
\pgfpathcurveto{\pgfqpoint{4.089795in}{3.411436in}}{\pgfqpoint{4.085405in}{3.422035in}}{\pgfqpoint{4.077591in}{3.429849in}}%
\pgfpathcurveto{\pgfqpoint{4.069778in}{3.437662in}}{\pgfqpoint{4.059179in}{3.442053in}}{\pgfqpoint{4.048129in}{3.442053in}}%
\pgfpathcurveto{\pgfqpoint{4.037079in}{3.442053in}}{\pgfqpoint{4.026479in}{3.437662in}}{\pgfqpoint{4.018666in}{3.429849in}}%
\pgfpathcurveto{\pgfqpoint{4.010852in}{3.422035in}}{\pgfqpoint{4.006462in}{3.411436in}}{\pgfqpoint{4.006462in}{3.400386in}}%
\pgfpathcurveto{\pgfqpoint{4.006462in}{3.389336in}}{\pgfqpoint{4.010852in}{3.378737in}}{\pgfqpoint{4.018666in}{3.370923in}}%
\pgfpathcurveto{\pgfqpoint{4.026479in}{3.363109in}}{\pgfqpoint{4.037079in}{3.358719in}}{\pgfqpoint{4.048129in}{3.358719in}}%
\pgfpathclose%
\pgfusepath{stroke,fill}%
\end{pgfscope}%
\begin{pgfscope}%
\pgfpathrectangle{\pgfqpoint{0.648703in}{0.548769in}}{\pgfqpoint{5.201297in}{3.102590in}}%
\pgfusepath{clip}%
\pgfsetbuttcap%
\pgfsetroundjoin%
\definecolor{currentfill}{rgb}{0.121569,0.466667,0.705882}%
\pgfsetfillcolor{currentfill}%
\pgfsetlinewidth{1.003750pt}%
\definecolor{currentstroke}{rgb}{0.121569,0.466667,0.705882}%
\pgfsetstrokecolor{currentstroke}%
\pgfsetdash{}{0pt}%
\pgfpathmoveto{\pgfqpoint{0.885130in}{0.648129in}}%
\pgfpathcurveto{\pgfqpoint{0.896180in}{0.648129in}}{\pgfqpoint{0.906779in}{0.652519in}}{\pgfqpoint{0.914593in}{0.660333in}}%
\pgfpathcurveto{\pgfqpoint{0.922407in}{0.668146in}}{\pgfqpoint{0.926797in}{0.678745in}}{\pgfqpoint{0.926797in}{0.689796in}}%
\pgfpathcurveto{\pgfqpoint{0.926797in}{0.700846in}}{\pgfqpoint{0.922407in}{0.711445in}}{\pgfqpoint{0.914593in}{0.719258in}}%
\pgfpathcurveto{\pgfqpoint{0.906779in}{0.727072in}}{\pgfqpoint{0.896180in}{0.731462in}}{\pgfqpoint{0.885130in}{0.731462in}}%
\pgfpathcurveto{\pgfqpoint{0.874080in}{0.731462in}}{\pgfqpoint{0.863481in}{0.727072in}}{\pgfqpoint{0.855667in}{0.719258in}}%
\pgfpathcurveto{\pgfqpoint{0.847854in}{0.711445in}}{\pgfqpoint{0.843463in}{0.700846in}}{\pgfqpoint{0.843463in}{0.689796in}}%
\pgfpathcurveto{\pgfqpoint{0.843463in}{0.678745in}}{\pgfqpoint{0.847854in}{0.668146in}}{\pgfqpoint{0.855667in}{0.660333in}}%
\pgfpathcurveto{\pgfqpoint{0.863481in}{0.652519in}}{\pgfqpoint{0.874080in}{0.648129in}}{\pgfqpoint{0.885130in}{0.648129in}}%
\pgfpathclose%
\pgfusepath{stroke,fill}%
\end{pgfscope}%
\begin{pgfscope}%
\pgfpathrectangle{\pgfqpoint{0.648703in}{0.548769in}}{\pgfqpoint{5.201297in}{3.102590in}}%
\pgfusepath{clip}%
\pgfsetbuttcap%
\pgfsetroundjoin%
\definecolor{currentfill}{rgb}{0.121569,0.466667,0.705882}%
\pgfsetfillcolor{currentfill}%
\pgfsetlinewidth{1.003750pt}%
\definecolor{currentstroke}{rgb}{0.121569,0.466667,0.705882}%
\pgfsetstrokecolor{currentstroke}%
\pgfsetdash{}{0pt}%
\pgfpathmoveto{\pgfqpoint{1.182333in}{0.796133in}}%
\pgfpathcurveto{\pgfqpoint{1.193384in}{0.796133in}}{\pgfqpoint{1.203983in}{0.800523in}}{\pgfqpoint{1.211796in}{0.808337in}}%
\pgfpathcurveto{\pgfqpoint{1.219610in}{0.816151in}}{\pgfqpoint{1.224000in}{0.826750in}}{\pgfqpoint{1.224000in}{0.837800in}}%
\pgfpathcurveto{\pgfqpoint{1.224000in}{0.848850in}}{\pgfqpoint{1.219610in}{0.859449in}}{\pgfqpoint{1.211796in}{0.867263in}}%
\pgfpathcurveto{\pgfqpoint{1.203983in}{0.875076in}}{\pgfqpoint{1.193384in}{0.879466in}}{\pgfqpoint{1.182333in}{0.879466in}}%
\pgfpathcurveto{\pgfqpoint{1.171283in}{0.879466in}}{\pgfqpoint{1.160684in}{0.875076in}}{\pgfqpoint{1.152871in}{0.867263in}}%
\pgfpathcurveto{\pgfqpoint{1.145057in}{0.859449in}}{\pgfqpoint{1.140667in}{0.848850in}}{\pgfqpoint{1.140667in}{0.837800in}}%
\pgfpathcurveto{\pgfqpoint{1.140667in}{0.826750in}}{\pgfqpoint{1.145057in}{0.816151in}}{\pgfqpoint{1.152871in}{0.808337in}}%
\pgfpathcurveto{\pgfqpoint{1.160684in}{0.800523in}}{\pgfqpoint{1.171283in}{0.796133in}}{\pgfqpoint{1.182333in}{0.796133in}}%
\pgfpathclose%
\pgfusepath{stroke,fill}%
\end{pgfscope}%
\begin{pgfscope}%
\pgfpathrectangle{\pgfqpoint{0.648703in}{0.548769in}}{\pgfqpoint{5.201297in}{3.102590in}}%
\pgfusepath{clip}%
\pgfsetbuttcap%
\pgfsetroundjoin%
\definecolor{currentfill}{rgb}{0.121569,0.466667,0.705882}%
\pgfsetfillcolor{currentfill}%
\pgfsetlinewidth{1.003750pt}%
\definecolor{currentstroke}{rgb}{0.121569,0.466667,0.705882}%
\pgfsetstrokecolor{currentstroke}%
\pgfsetdash{}{0pt}%
\pgfpathmoveto{\pgfqpoint{4.961385in}{3.155742in}}%
\pgfpathcurveto{\pgfqpoint{4.972436in}{3.155742in}}{\pgfqpoint{4.983035in}{3.160132in}}{\pgfqpoint{4.990848in}{3.167946in}}%
\pgfpathcurveto{\pgfqpoint{4.998662in}{3.175760in}}{\pgfqpoint{5.003052in}{3.186359in}}{\pgfqpoint{5.003052in}{3.197409in}}%
\pgfpathcurveto{\pgfqpoint{5.003052in}{3.208459in}}{\pgfqpoint{4.998662in}{3.219058in}}{\pgfqpoint{4.990848in}{3.226872in}}%
\pgfpathcurveto{\pgfqpoint{4.983035in}{3.234685in}}{\pgfqpoint{4.972436in}{3.239075in}}{\pgfqpoint{4.961385in}{3.239075in}}%
\pgfpathcurveto{\pgfqpoint{4.950335in}{3.239075in}}{\pgfqpoint{4.939736in}{3.234685in}}{\pgfqpoint{4.931923in}{3.226872in}}%
\pgfpathcurveto{\pgfqpoint{4.924109in}{3.219058in}}{\pgfqpoint{4.919719in}{3.208459in}}{\pgfqpoint{4.919719in}{3.197409in}}%
\pgfpathcurveto{\pgfqpoint{4.919719in}{3.186359in}}{\pgfqpoint{4.924109in}{3.175760in}}{\pgfqpoint{4.931923in}{3.167946in}}%
\pgfpathcurveto{\pgfqpoint{4.939736in}{3.160132in}}{\pgfqpoint{4.950335in}{3.155742in}}{\pgfqpoint{4.961385in}{3.155742in}}%
\pgfpathclose%
\pgfusepath{stroke,fill}%
\end{pgfscope}%
\begin{pgfscope}%
\pgfpathrectangle{\pgfqpoint{0.648703in}{0.548769in}}{\pgfqpoint{5.201297in}{3.102590in}}%
\pgfusepath{clip}%
\pgfsetbuttcap%
\pgfsetroundjoin%
\definecolor{currentfill}{rgb}{0.121569,0.466667,0.705882}%
\pgfsetfillcolor{currentfill}%
\pgfsetlinewidth{1.003750pt}%
\definecolor{currentstroke}{rgb}{0.121569,0.466667,0.705882}%
\pgfsetstrokecolor{currentstroke}%
\pgfsetdash{}{0pt}%
\pgfpathmoveto{\pgfqpoint{0.885153in}{0.648129in}}%
\pgfpathcurveto{\pgfqpoint{0.896203in}{0.648129in}}{\pgfqpoint{0.906802in}{0.652519in}}{\pgfqpoint{0.914616in}{0.660333in}}%
\pgfpathcurveto{\pgfqpoint{0.922429in}{0.668146in}}{\pgfqpoint{0.926820in}{0.678745in}}{\pgfqpoint{0.926820in}{0.689796in}}%
\pgfpathcurveto{\pgfqpoint{0.926820in}{0.700846in}}{\pgfqpoint{0.922429in}{0.711445in}}{\pgfqpoint{0.914616in}{0.719258in}}%
\pgfpathcurveto{\pgfqpoint{0.906802in}{0.727072in}}{\pgfqpoint{0.896203in}{0.731462in}}{\pgfqpoint{0.885153in}{0.731462in}}%
\pgfpathcurveto{\pgfqpoint{0.874103in}{0.731462in}}{\pgfqpoint{0.863504in}{0.727072in}}{\pgfqpoint{0.855690in}{0.719258in}}%
\pgfpathcurveto{\pgfqpoint{0.847876in}{0.711445in}}{\pgfqpoint{0.843486in}{0.700846in}}{\pgfqpoint{0.843486in}{0.689796in}}%
\pgfpathcurveto{\pgfqpoint{0.843486in}{0.678745in}}{\pgfqpoint{0.847876in}{0.668146in}}{\pgfqpoint{0.855690in}{0.660333in}}%
\pgfpathcurveto{\pgfqpoint{0.863504in}{0.652519in}}{\pgfqpoint{0.874103in}{0.648129in}}{\pgfqpoint{0.885153in}{0.648129in}}%
\pgfpathclose%
\pgfusepath{stroke,fill}%
\end{pgfscope}%
\begin{pgfscope}%
\pgfpathrectangle{\pgfqpoint{0.648703in}{0.548769in}}{\pgfqpoint{5.201297in}{3.102590in}}%
\pgfusepath{clip}%
\pgfsetbuttcap%
\pgfsetroundjoin%
\definecolor{currentfill}{rgb}{0.121569,0.466667,0.705882}%
\pgfsetfillcolor{currentfill}%
\pgfsetlinewidth{1.003750pt}%
\definecolor{currentstroke}{rgb}{0.121569,0.466667,0.705882}%
\pgfsetstrokecolor{currentstroke}%
\pgfsetdash{}{0pt}%
\pgfpathmoveto{\pgfqpoint{0.885156in}{0.648129in}}%
\pgfpathcurveto{\pgfqpoint{0.896206in}{0.648129in}}{\pgfqpoint{0.906805in}{0.652519in}}{\pgfqpoint{0.914619in}{0.660333in}}%
\pgfpathcurveto{\pgfqpoint{0.922432in}{0.668146in}}{\pgfqpoint{0.926823in}{0.678745in}}{\pgfqpoint{0.926823in}{0.689796in}}%
\pgfpathcurveto{\pgfqpoint{0.926823in}{0.700846in}}{\pgfqpoint{0.922432in}{0.711445in}}{\pgfqpoint{0.914619in}{0.719258in}}%
\pgfpathcurveto{\pgfqpoint{0.906805in}{0.727072in}}{\pgfqpoint{0.896206in}{0.731462in}}{\pgfqpoint{0.885156in}{0.731462in}}%
\pgfpathcurveto{\pgfqpoint{0.874106in}{0.731462in}}{\pgfqpoint{0.863507in}{0.727072in}}{\pgfqpoint{0.855693in}{0.719258in}}%
\pgfpathcurveto{\pgfqpoint{0.847879in}{0.711445in}}{\pgfqpoint{0.843489in}{0.700846in}}{\pgfqpoint{0.843489in}{0.689796in}}%
\pgfpathcurveto{\pgfqpoint{0.843489in}{0.678745in}}{\pgfqpoint{0.847879in}{0.668146in}}{\pgfqpoint{0.855693in}{0.660333in}}%
\pgfpathcurveto{\pgfqpoint{0.863507in}{0.652519in}}{\pgfqpoint{0.874106in}{0.648129in}}{\pgfqpoint{0.885156in}{0.648129in}}%
\pgfpathclose%
\pgfusepath{stroke,fill}%
\end{pgfscope}%
\begin{pgfscope}%
\pgfpathrectangle{\pgfqpoint{0.648703in}{0.548769in}}{\pgfqpoint{5.201297in}{3.102590in}}%
\pgfusepath{clip}%
\pgfsetbuttcap%
\pgfsetroundjoin%
\definecolor{currentfill}{rgb}{0.121569,0.466667,0.705882}%
\pgfsetfillcolor{currentfill}%
\pgfsetlinewidth{1.003750pt}%
\definecolor{currentstroke}{rgb}{0.121569,0.466667,0.705882}%
\pgfsetstrokecolor{currentstroke}%
\pgfsetdash{}{0pt}%
\pgfpathmoveto{\pgfqpoint{4.062166in}{2.512981in}}%
\pgfpathcurveto{\pgfqpoint{4.073216in}{2.512981in}}{\pgfqpoint{4.083815in}{2.517371in}}{\pgfqpoint{4.091629in}{2.525185in}}%
\pgfpathcurveto{\pgfqpoint{4.099443in}{2.532999in}}{\pgfqpoint{4.103833in}{2.543598in}}{\pgfqpoint{4.103833in}{2.554648in}}%
\pgfpathcurveto{\pgfqpoint{4.103833in}{2.565698in}}{\pgfqpoint{4.099443in}{2.576297in}}{\pgfqpoint{4.091629in}{2.584111in}}%
\pgfpathcurveto{\pgfqpoint{4.083815in}{2.591924in}}{\pgfqpoint{4.073216in}{2.596315in}}{\pgfqpoint{4.062166in}{2.596315in}}%
\pgfpathcurveto{\pgfqpoint{4.051116in}{2.596315in}}{\pgfqpoint{4.040517in}{2.591924in}}{\pgfqpoint{4.032704in}{2.584111in}}%
\pgfpathcurveto{\pgfqpoint{4.024890in}{2.576297in}}{\pgfqpoint{4.020500in}{2.565698in}}{\pgfqpoint{4.020500in}{2.554648in}}%
\pgfpathcurveto{\pgfqpoint{4.020500in}{2.543598in}}{\pgfqpoint{4.024890in}{2.532999in}}{\pgfqpoint{4.032704in}{2.525185in}}%
\pgfpathcurveto{\pgfqpoint{4.040517in}{2.517371in}}{\pgfqpoint{4.051116in}{2.512981in}}{\pgfqpoint{4.062166in}{2.512981in}}%
\pgfpathclose%
\pgfusepath{stroke,fill}%
\end{pgfscope}%
\begin{pgfscope}%
\pgfpathrectangle{\pgfqpoint{0.648703in}{0.548769in}}{\pgfqpoint{5.201297in}{3.102590in}}%
\pgfusepath{clip}%
\pgfsetbuttcap%
\pgfsetroundjoin%
\definecolor{currentfill}{rgb}{1.000000,0.498039,0.054902}%
\pgfsetfillcolor{currentfill}%
\pgfsetlinewidth{1.003750pt}%
\definecolor{currentstroke}{rgb}{1.000000,0.498039,0.054902}%
\pgfsetstrokecolor{currentstroke}%
\pgfsetdash{}{0pt}%
\pgfpathmoveto{\pgfqpoint{4.899286in}{3.206486in}}%
\pgfpathcurveto{\pgfqpoint{4.910336in}{3.206486in}}{\pgfqpoint{4.920935in}{3.210877in}}{\pgfqpoint{4.928749in}{3.218690in}}%
\pgfpathcurveto{\pgfqpoint{4.936563in}{3.226504in}}{\pgfqpoint{4.940953in}{3.237103in}}{\pgfqpoint{4.940953in}{3.248153in}}%
\pgfpathcurveto{\pgfqpoint{4.940953in}{3.259203in}}{\pgfqpoint{4.936563in}{3.269802in}}{\pgfqpoint{4.928749in}{3.277616in}}%
\pgfpathcurveto{\pgfqpoint{4.920935in}{3.285429in}}{\pgfqpoint{4.910336in}{3.289820in}}{\pgfqpoint{4.899286in}{3.289820in}}%
\pgfpathcurveto{\pgfqpoint{4.888236in}{3.289820in}}{\pgfqpoint{4.877637in}{3.285429in}}{\pgfqpoint{4.869823in}{3.277616in}}%
\pgfpathcurveto{\pgfqpoint{4.862010in}{3.269802in}}{\pgfqpoint{4.857619in}{3.259203in}}{\pgfqpoint{4.857619in}{3.248153in}}%
\pgfpathcurveto{\pgfqpoint{4.857619in}{3.237103in}}{\pgfqpoint{4.862010in}{3.226504in}}{\pgfqpoint{4.869823in}{3.218690in}}%
\pgfpathcurveto{\pgfqpoint{4.877637in}{3.210877in}}{\pgfqpoint{4.888236in}{3.206486in}}{\pgfqpoint{4.899286in}{3.206486in}}%
\pgfpathclose%
\pgfusepath{stroke,fill}%
\end{pgfscope}%
\begin{pgfscope}%
\pgfpathrectangle{\pgfqpoint{0.648703in}{0.548769in}}{\pgfqpoint{5.201297in}{3.102590in}}%
\pgfusepath{clip}%
\pgfsetbuttcap%
\pgfsetroundjoin%
\definecolor{currentfill}{rgb}{1.000000,0.498039,0.054902}%
\pgfsetfillcolor{currentfill}%
\pgfsetlinewidth{1.003750pt}%
\definecolor{currentstroke}{rgb}{1.000000,0.498039,0.054902}%
\pgfsetstrokecolor{currentstroke}%
\pgfsetdash{}{0pt}%
\pgfpathmoveto{\pgfqpoint{3.427525in}{3.198029in}}%
\pgfpathcurveto{\pgfqpoint{3.438575in}{3.198029in}}{\pgfqpoint{3.449174in}{3.202419in}}{\pgfqpoint{3.456987in}{3.210233in}}%
\pgfpathcurveto{\pgfqpoint{3.464801in}{3.218046in}}{\pgfqpoint{3.469191in}{3.228646in}}{\pgfqpoint{3.469191in}{3.239696in}}%
\pgfpathcurveto{\pgfqpoint{3.469191in}{3.250746in}}{\pgfqpoint{3.464801in}{3.261345in}}{\pgfqpoint{3.456987in}{3.269158in}}%
\pgfpathcurveto{\pgfqpoint{3.449174in}{3.276972in}}{\pgfqpoint{3.438575in}{3.281362in}}{\pgfqpoint{3.427525in}{3.281362in}}%
\pgfpathcurveto{\pgfqpoint{3.416475in}{3.281362in}}{\pgfqpoint{3.405876in}{3.276972in}}{\pgfqpoint{3.398062in}{3.269158in}}%
\pgfpathcurveto{\pgfqpoint{3.390248in}{3.261345in}}{\pgfqpoint{3.385858in}{3.250746in}}{\pgfqpoint{3.385858in}{3.239696in}}%
\pgfpathcurveto{\pgfqpoint{3.385858in}{3.228646in}}{\pgfqpoint{3.390248in}{3.218046in}}{\pgfqpoint{3.398062in}{3.210233in}}%
\pgfpathcurveto{\pgfqpoint{3.405876in}{3.202419in}}{\pgfqpoint{3.416475in}{3.198029in}}{\pgfqpoint{3.427525in}{3.198029in}}%
\pgfpathclose%
\pgfusepath{stroke,fill}%
\end{pgfscope}%
\begin{pgfscope}%
\pgfpathrectangle{\pgfqpoint{0.648703in}{0.548769in}}{\pgfqpoint{5.201297in}{3.102590in}}%
\pgfusepath{clip}%
\pgfsetbuttcap%
\pgfsetroundjoin%
\definecolor{currentfill}{rgb}{1.000000,0.498039,0.054902}%
\pgfsetfillcolor{currentfill}%
\pgfsetlinewidth{1.003750pt}%
\definecolor{currentstroke}{rgb}{1.000000,0.498039,0.054902}%
\pgfsetstrokecolor{currentstroke}%
\pgfsetdash{}{0pt}%
\pgfpathmoveto{\pgfqpoint{4.461215in}{3.248773in}}%
\pgfpathcurveto{\pgfqpoint{4.472266in}{3.248773in}}{\pgfqpoint{4.482865in}{3.253164in}}{\pgfqpoint{4.490678in}{3.260977in}}%
\pgfpathcurveto{\pgfqpoint{4.498492in}{3.268791in}}{\pgfqpoint{4.502882in}{3.279390in}}{\pgfqpoint{4.502882in}{3.290440in}}%
\pgfpathcurveto{\pgfqpoint{4.502882in}{3.301490in}}{\pgfqpoint{4.498492in}{3.312089in}}{\pgfqpoint{4.490678in}{3.319903in}}%
\pgfpathcurveto{\pgfqpoint{4.482865in}{3.327716in}}{\pgfqpoint{4.472266in}{3.332107in}}{\pgfqpoint{4.461215in}{3.332107in}}%
\pgfpathcurveto{\pgfqpoint{4.450165in}{3.332107in}}{\pgfqpoint{4.439566in}{3.327716in}}{\pgfqpoint{4.431753in}{3.319903in}}%
\pgfpathcurveto{\pgfqpoint{4.423939in}{3.312089in}}{\pgfqpoint{4.419549in}{3.301490in}}{\pgfqpoint{4.419549in}{3.290440in}}%
\pgfpathcurveto{\pgfqpoint{4.419549in}{3.279390in}}{\pgfqpoint{4.423939in}{3.268791in}}{\pgfqpoint{4.431753in}{3.260977in}}%
\pgfpathcurveto{\pgfqpoint{4.439566in}{3.253164in}}{\pgfqpoint{4.450165in}{3.248773in}}{\pgfqpoint{4.461215in}{3.248773in}}%
\pgfpathclose%
\pgfusepath{stroke,fill}%
\end{pgfscope}%
\begin{pgfscope}%
\pgfpathrectangle{\pgfqpoint{0.648703in}{0.548769in}}{\pgfqpoint{5.201297in}{3.102590in}}%
\pgfusepath{clip}%
\pgfsetbuttcap%
\pgfsetroundjoin%
\definecolor{currentfill}{rgb}{1.000000,0.498039,0.054902}%
\pgfsetfillcolor{currentfill}%
\pgfsetlinewidth{1.003750pt}%
\definecolor{currentstroke}{rgb}{1.000000,0.498039,0.054902}%
\pgfsetstrokecolor{currentstroke}%
\pgfsetdash{}{0pt}%
\pgfpathmoveto{\pgfqpoint{3.737781in}{3.189572in}}%
\pgfpathcurveto{\pgfqpoint{3.748831in}{3.189572in}}{\pgfqpoint{3.759430in}{3.193962in}}{\pgfqpoint{3.767244in}{3.201775in}}%
\pgfpathcurveto{\pgfqpoint{3.775058in}{3.209589in}}{\pgfqpoint{3.779448in}{3.220188in}}{\pgfqpoint{3.779448in}{3.231238in}}%
\pgfpathcurveto{\pgfqpoint{3.779448in}{3.242288in}}{\pgfqpoint{3.775058in}{3.252887in}}{\pgfqpoint{3.767244in}{3.260701in}}%
\pgfpathcurveto{\pgfqpoint{3.759430in}{3.268515in}}{\pgfqpoint{3.748831in}{3.272905in}}{\pgfqpoint{3.737781in}{3.272905in}}%
\pgfpathcurveto{\pgfqpoint{3.726731in}{3.272905in}}{\pgfqpoint{3.716132in}{3.268515in}}{\pgfqpoint{3.708318in}{3.260701in}}%
\pgfpathcurveto{\pgfqpoint{3.700505in}{3.252887in}}{\pgfqpoint{3.696114in}{3.242288in}}{\pgfqpoint{3.696114in}{3.231238in}}%
\pgfpathcurveto{\pgfqpoint{3.696114in}{3.220188in}}{\pgfqpoint{3.700505in}{3.209589in}}{\pgfqpoint{3.708318in}{3.201775in}}%
\pgfpathcurveto{\pgfqpoint{3.716132in}{3.193962in}}{\pgfqpoint{3.726731in}{3.189572in}}{\pgfqpoint{3.737781in}{3.189572in}}%
\pgfpathclose%
\pgfusepath{stroke,fill}%
\end{pgfscope}%
\begin{pgfscope}%
\pgfpathrectangle{\pgfqpoint{0.648703in}{0.548769in}}{\pgfqpoint{5.201297in}{3.102590in}}%
\pgfusepath{clip}%
\pgfsetbuttcap%
\pgfsetroundjoin%
\definecolor{currentfill}{rgb}{0.121569,0.466667,0.705882}%
\pgfsetfillcolor{currentfill}%
\pgfsetlinewidth{1.003750pt}%
\definecolor{currentstroke}{rgb}{0.121569,0.466667,0.705882}%
\pgfsetstrokecolor{currentstroke}%
\pgfsetdash{}{0pt}%
\pgfpathmoveto{\pgfqpoint{4.991606in}{3.181114in}}%
\pgfpathcurveto{\pgfqpoint{5.002656in}{3.181114in}}{\pgfqpoint{5.013255in}{3.185504in}}{\pgfqpoint{5.021068in}{3.193318in}}%
\pgfpathcurveto{\pgfqpoint{5.028882in}{3.201132in}}{\pgfqpoint{5.033272in}{3.211731in}}{\pgfqpoint{5.033272in}{3.222781in}}%
\pgfpathcurveto{\pgfqpoint{5.033272in}{3.233831in}}{\pgfqpoint{5.028882in}{3.244430in}}{\pgfqpoint{5.021068in}{3.252244in}}%
\pgfpathcurveto{\pgfqpoint{5.013255in}{3.260057in}}{\pgfqpoint{5.002656in}{3.264448in}}{\pgfqpoint{4.991606in}{3.264448in}}%
\pgfpathcurveto{\pgfqpoint{4.980555in}{3.264448in}}{\pgfqpoint{4.969956in}{3.260057in}}{\pgfqpoint{4.962143in}{3.252244in}}%
\pgfpathcurveto{\pgfqpoint{4.954329in}{3.244430in}}{\pgfqpoint{4.949939in}{3.233831in}}{\pgfqpoint{4.949939in}{3.222781in}}%
\pgfpathcurveto{\pgfqpoint{4.949939in}{3.211731in}}{\pgfqpoint{4.954329in}{3.201132in}}{\pgfqpoint{4.962143in}{3.193318in}}%
\pgfpathcurveto{\pgfqpoint{4.969956in}{3.185504in}}{\pgfqpoint{4.980555in}{3.181114in}}{\pgfqpoint{4.991606in}{3.181114in}}%
\pgfpathclose%
\pgfusepath{stroke,fill}%
\end{pgfscope}%
\begin{pgfscope}%
\pgfpathrectangle{\pgfqpoint{0.648703in}{0.548769in}}{\pgfqpoint{5.201297in}{3.102590in}}%
\pgfusepath{clip}%
\pgfsetbuttcap%
\pgfsetroundjoin%
\definecolor{currentfill}{rgb}{0.839216,0.152941,0.156863}%
\pgfsetfillcolor{currentfill}%
\pgfsetlinewidth{1.003750pt}%
\definecolor{currentstroke}{rgb}{0.839216,0.152941,0.156863}%
\pgfsetstrokecolor{currentstroke}%
\pgfsetdash{}{0pt}%
\pgfpathmoveto{\pgfqpoint{3.423505in}{3.202258in}}%
\pgfpathcurveto{\pgfqpoint{3.434555in}{3.202258in}}{\pgfqpoint{3.445154in}{3.206648in}}{\pgfqpoint{3.452968in}{3.214462in}}%
\pgfpathcurveto{\pgfqpoint{3.460781in}{3.222275in}}{\pgfqpoint{3.465171in}{3.232874in}}{\pgfqpoint{3.465171in}{3.243924in}}%
\pgfpathcurveto{\pgfqpoint{3.465171in}{3.254974in}}{\pgfqpoint{3.460781in}{3.265573in}}{\pgfqpoint{3.452968in}{3.273387in}}%
\pgfpathcurveto{\pgfqpoint{3.445154in}{3.281201in}}{\pgfqpoint{3.434555in}{3.285591in}}{\pgfqpoint{3.423505in}{3.285591in}}%
\pgfpathcurveto{\pgfqpoint{3.412455in}{3.285591in}}{\pgfqpoint{3.401856in}{3.281201in}}{\pgfqpoint{3.394042in}{3.273387in}}%
\pgfpathcurveto{\pgfqpoint{3.386228in}{3.265573in}}{\pgfqpoint{3.381838in}{3.254974in}}{\pgfqpoint{3.381838in}{3.243924in}}%
\pgfpathcurveto{\pgfqpoint{3.381838in}{3.232874in}}{\pgfqpoint{3.386228in}{3.222275in}}{\pgfqpoint{3.394042in}{3.214462in}}%
\pgfpathcurveto{\pgfqpoint{3.401856in}{3.206648in}}{\pgfqpoint{3.412455in}{3.202258in}}{\pgfqpoint{3.423505in}{3.202258in}}%
\pgfpathclose%
\pgfusepath{stroke,fill}%
\end{pgfscope}%
\begin{pgfscope}%
\pgfpathrectangle{\pgfqpoint{0.648703in}{0.548769in}}{\pgfqpoint{5.201297in}{3.102590in}}%
\pgfusepath{clip}%
\pgfsetbuttcap%
\pgfsetroundjoin%
\definecolor{currentfill}{rgb}{1.000000,0.498039,0.054902}%
\pgfsetfillcolor{currentfill}%
\pgfsetlinewidth{1.003750pt}%
\definecolor{currentstroke}{rgb}{1.000000,0.498039,0.054902}%
\pgfsetstrokecolor{currentstroke}%
\pgfsetdash{}{0pt}%
\pgfpathmoveto{\pgfqpoint{4.676266in}{3.185343in}}%
\pgfpathcurveto{\pgfqpoint{4.687316in}{3.185343in}}{\pgfqpoint{4.697915in}{3.189733in}}{\pgfqpoint{4.705729in}{3.197547in}}%
\pgfpathcurveto{\pgfqpoint{4.713542in}{3.205360in}}{\pgfqpoint{4.717932in}{3.215959in}}{\pgfqpoint{4.717932in}{3.227010in}}%
\pgfpathcurveto{\pgfqpoint{4.717932in}{3.238060in}}{\pgfqpoint{4.713542in}{3.248659in}}{\pgfqpoint{4.705729in}{3.256472in}}%
\pgfpathcurveto{\pgfqpoint{4.697915in}{3.264286in}}{\pgfqpoint{4.687316in}{3.268676in}}{\pgfqpoint{4.676266in}{3.268676in}}%
\pgfpathcurveto{\pgfqpoint{4.665216in}{3.268676in}}{\pgfqpoint{4.654617in}{3.264286in}}{\pgfqpoint{4.646803in}{3.256472in}}%
\pgfpathcurveto{\pgfqpoint{4.638989in}{3.248659in}}{\pgfqpoint{4.634599in}{3.238060in}}{\pgfqpoint{4.634599in}{3.227010in}}%
\pgfpathcurveto{\pgfqpoint{4.634599in}{3.215959in}}{\pgfqpoint{4.638989in}{3.205360in}}{\pgfqpoint{4.646803in}{3.197547in}}%
\pgfpathcurveto{\pgfqpoint{4.654617in}{3.189733in}}{\pgfqpoint{4.665216in}{3.185343in}}{\pgfqpoint{4.676266in}{3.185343in}}%
\pgfpathclose%
\pgfusepath{stroke,fill}%
\end{pgfscope}%
\begin{pgfscope}%
\pgfpathrectangle{\pgfqpoint{0.648703in}{0.548769in}}{\pgfqpoint{5.201297in}{3.102590in}}%
\pgfusepath{clip}%
\pgfsetbuttcap%
\pgfsetroundjoin%
\definecolor{currentfill}{rgb}{1.000000,0.498039,0.054902}%
\pgfsetfillcolor{currentfill}%
\pgfsetlinewidth{1.003750pt}%
\definecolor{currentstroke}{rgb}{1.000000,0.498039,0.054902}%
\pgfsetstrokecolor{currentstroke}%
\pgfsetdash{}{0pt}%
\pgfpathmoveto{\pgfqpoint{3.928384in}{3.278374in}}%
\pgfpathcurveto{\pgfqpoint{3.939435in}{3.278374in}}{\pgfqpoint{3.950034in}{3.282764in}}{\pgfqpoint{3.957847in}{3.290578in}}%
\pgfpathcurveto{\pgfqpoint{3.965661in}{3.298392in}}{\pgfqpoint{3.970051in}{3.308991in}}{\pgfqpoint{3.970051in}{3.320041in}}%
\pgfpathcurveto{\pgfqpoint{3.970051in}{3.331091in}}{\pgfqpoint{3.965661in}{3.341690in}}{\pgfqpoint{3.957847in}{3.349504in}}%
\pgfpathcurveto{\pgfqpoint{3.950034in}{3.357317in}}{\pgfqpoint{3.939435in}{3.361707in}}{\pgfqpoint{3.928384in}{3.361707in}}%
\pgfpathcurveto{\pgfqpoint{3.917334in}{3.361707in}}{\pgfqpoint{3.906735in}{3.357317in}}{\pgfqpoint{3.898922in}{3.349504in}}%
\pgfpathcurveto{\pgfqpoint{3.891108in}{3.341690in}}{\pgfqpoint{3.886718in}{3.331091in}}{\pgfqpoint{3.886718in}{3.320041in}}%
\pgfpathcurveto{\pgfqpoint{3.886718in}{3.308991in}}{\pgfqpoint{3.891108in}{3.298392in}}{\pgfqpoint{3.898922in}{3.290578in}}%
\pgfpathcurveto{\pgfqpoint{3.906735in}{3.282764in}}{\pgfqpoint{3.917334in}{3.278374in}}{\pgfqpoint{3.928384in}{3.278374in}}%
\pgfpathclose%
\pgfusepath{stroke,fill}%
\end{pgfscope}%
\begin{pgfscope}%
\pgfpathrectangle{\pgfqpoint{0.648703in}{0.548769in}}{\pgfqpoint{5.201297in}{3.102590in}}%
\pgfusepath{clip}%
\pgfsetbuttcap%
\pgfsetroundjoin%
\definecolor{currentfill}{rgb}{0.121569,0.466667,0.705882}%
\pgfsetfillcolor{currentfill}%
\pgfsetlinewidth{1.003750pt}%
\definecolor{currentstroke}{rgb}{0.121569,0.466667,0.705882}%
\pgfsetstrokecolor{currentstroke}%
\pgfsetdash{}{0pt}%
\pgfpathmoveto{\pgfqpoint{2.858941in}{3.181114in}}%
\pgfpathcurveto{\pgfqpoint{2.869991in}{3.181114in}}{\pgfqpoint{2.880590in}{3.185504in}}{\pgfqpoint{2.888404in}{3.193318in}}%
\pgfpathcurveto{\pgfqpoint{2.896217in}{3.201132in}}{\pgfqpoint{2.900608in}{3.211731in}}{\pgfqpoint{2.900608in}{3.222781in}}%
\pgfpathcurveto{\pgfqpoint{2.900608in}{3.233831in}}{\pgfqpoint{2.896217in}{3.244430in}}{\pgfqpoint{2.888404in}{3.252244in}}%
\pgfpathcurveto{\pgfqpoint{2.880590in}{3.260057in}}{\pgfqpoint{2.869991in}{3.264448in}}{\pgfqpoint{2.858941in}{3.264448in}}%
\pgfpathcurveto{\pgfqpoint{2.847891in}{3.264448in}}{\pgfqpoint{2.837292in}{3.260057in}}{\pgfqpoint{2.829478in}{3.252244in}}%
\pgfpathcurveto{\pgfqpoint{2.821665in}{3.244430in}}{\pgfqpoint{2.817274in}{3.233831in}}{\pgfqpoint{2.817274in}{3.222781in}}%
\pgfpathcurveto{\pgfqpoint{2.817274in}{3.211731in}}{\pgfqpoint{2.821665in}{3.201132in}}{\pgfqpoint{2.829478in}{3.193318in}}%
\pgfpathcurveto{\pgfqpoint{2.837292in}{3.185504in}}{\pgfqpoint{2.847891in}{3.181114in}}{\pgfqpoint{2.858941in}{3.181114in}}%
\pgfpathclose%
\pgfusepath{stroke,fill}%
\end{pgfscope}%
\begin{pgfscope}%
\pgfpathrectangle{\pgfqpoint{0.648703in}{0.548769in}}{\pgfqpoint{5.201297in}{3.102590in}}%
\pgfusepath{clip}%
\pgfsetbuttcap%
\pgfsetroundjoin%
\definecolor{currentfill}{rgb}{1.000000,0.498039,0.054902}%
\pgfsetfillcolor{currentfill}%
\pgfsetlinewidth{1.003750pt}%
\definecolor{currentstroke}{rgb}{1.000000,0.498039,0.054902}%
\pgfsetstrokecolor{currentstroke}%
\pgfsetdash{}{0pt}%
\pgfpathmoveto{\pgfqpoint{3.923636in}{3.219172in}}%
\pgfpathcurveto{\pgfqpoint{3.934686in}{3.219172in}}{\pgfqpoint{3.945285in}{3.223563in}}{\pgfqpoint{3.953099in}{3.231376in}}%
\pgfpathcurveto{\pgfqpoint{3.960913in}{3.239190in}}{\pgfqpoint{3.965303in}{3.249789in}}{\pgfqpoint{3.965303in}{3.260839in}}%
\pgfpathcurveto{\pgfqpoint{3.965303in}{3.271889in}}{\pgfqpoint{3.960913in}{3.282488in}}{\pgfqpoint{3.953099in}{3.290302in}}%
\pgfpathcurveto{\pgfqpoint{3.945285in}{3.298116in}}{\pgfqpoint{3.934686in}{3.302506in}}{\pgfqpoint{3.923636in}{3.302506in}}%
\pgfpathcurveto{\pgfqpoint{3.912586in}{3.302506in}}{\pgfqpoint{3.901987in}{3.298116in}}{\pgfqpoint{3.894173in}{3.290302in}}%
\pgfpathcurveto{\pgfqpoint{3.886360in}{3.282488in}}{\pgfqpoint{3.881970in}{3.271889in}}{\pgfqpoint{3.881970in}{3.260839in}}%
\pgfpathcurveto{\pgfqpoint{3.881970in}{3.249789in}}{\pgfqpoint{3.886360in}{3.239190in}}{\pgfqpoint{3.894173in}{3.231376in}}%
\pgfpathcurveto{\pgfqpoint{3.901987in}{3.223563in}}{\pgfqpoint{3.912586in}{3.219172in}}{\pgfqpoint{3.923636in}{3.219172in}}%
\pgfpathclose%
\pgfusepath{stroke,fill}%
\end{pgfscope}%
\begin{pgfscope}%
\pgfpathrectangle{\pgfqpoint{0.648703in}{0.548769in}}{\pgfqpoint{5.201297in}{3.102590in}}%
\pgfusepath{clip}%
\pgfsetbuttcap%
\pgfsetroundjoin%
\definecolor{currentfill}{rgb}{1.000000,0.498039,0.054902}%
\pgfsetfillcolor{currentfill}%
\pgfsetlinewidth{1.003750pt}%
\definecolor{currentstroke}{rgb}{1.000000,0.498039,0.054902}%
\pgfsetstrokecolor{currentstroke}%
\pgfsetdash{}{0pt}%
\pgfpathmoveto{\pgfqpoint{5.578438in}{3.468665in}}%
\pgfpathcurveto{\pgfqpoint{5.589488in}{3.468665in}}{\pgfqpoint{5.600087in}{3.473055in}}{\pgfqpoint{5.607900in}{3.480869in}}%
\pgfpathcurveto{\pgfqpoint{5.615714in}{3.488683in}}{\pgfqpoint{5.620104in}{3.499282in}}{\pgfqpoint{5.620104in}{3.510332in}}%
\pgfpathcurveto{\pgfqpoint{5.620104in}{3.521382in}}{\pgfqpoint{5.615714in}{3.531981in}}{\pgfqpoint{5.607900in}{3.539795in}}%
\pgfpathcurveto{\pgfqpoint{5.600087in}{3.547608in}}{\pgfqpoint{5.589488in}{3.551998in}}{\pgfqpoint{5.578438in}{3.551998in}}%
\pgfpathcurveto{\pgfqpoint{5.567388in}{3.551998in}}{\pgfqpoint{5.556788in}{3.547608in}}{\pgfqpoint{5.548975in}{3.539795in}}%
\pgfpathcurveto{\pgfqpoint{5.541161in}{3.531981in}}{\pgfqpoint{5.536771in}{3.521382in}}{\pgfqpoint{5.536771in}{3.510332in}}%
\pgfpathcurveto{\pgfqpoint{5.536771in}{3.499282in}}{\pgfqpoint{5.541161in}{3.488683in}}{\pgfqpoint{5.548975in}{3.480869in}}%
\pgfpathcurveto{\pgfqpoint{5.556788in}{3.473055in}}{\pgfqpoint{5.567388in}{3.468665in}}{\pgfqpoint{5.578438in}{3.468665in}}%
\pgfpathclose%
\pgfusepath{stroke,fill}%
\end{pgfscope}%
\begin{pgfscope}%
\pgfpathrectangle{\pgfqpoint{0.648703in}{0.548769in}}{\pgfqpoint{5.201297in}{3.102590in}}%
\pgfusepath{clip}%
\pgfsetbuttcap%
\pgfsetroundjoin%
\definecolor{currentfill}{rgb}{1.000000,0.498039,0.054902}%
\pgfsetfillcolor{currentfill}%
\pgfsetlinewidth{1.003750pt}%
\definecolor{currentstroke}{rgb}{1.000000,0.498039,0.054902}%
\pgfsetstrokecolor{currentstroke}%
\pgfsetdash{}{0pt}%
\pgfpathmoveto{\pgfqpoint{4.302468in}{3.189572in}}%
\pgfpathcurveto{\pgfqpoint{4.313518in}{3.189572in}}{\pgfqpoint{4.324117in}{3.193962in}}{\pgfqpoint{4.331930in}{3.201775in}}%
\pgfpathcurveto{\pgfqpoint{4.339744in}{3.209589in}}{\pgfqpoint{4.344134in}{3.220188in}}{\pgfqpoint{4.344134in}{3.231238in}}%
\pgfpathcurveto{\pgfqpoint{4.344134in}{3.242288in}}{\pgfqpoint{4.339744in}{3.252887in}}{\pgfqpoint{4.331930in}{3.260701in}}%
\pgfpathcurveto{\pgfqpoint{4.324117in}{3.268515in}}{\pgfqpoint{4.313518in}{3.272905in}}{\pgfqpoint{4.302468in}{3.272905in}}%
\pgfpathcurveto{\pgfqpoint{4.291418in}{3.272905in}}{\pgfqpoint{4.280819in}{3.268515in}}{\pgfqpoint{4.273005in}{3.260701in}}%
\pgfpathcurveto{\pgfqpoint{4.265191in}{3.252887in}}{\pgfqpoint{4.260801in}{3.242288in}}{\pgfqpoint{4.260801in}{3.231238in}}%
\pgfpathcurveto{\pgfqpoint{4.260801in}{3.220188in}}{\pgfqpoint{4.265191in}{3.209589in}}{\pgfqpoint{4.273005in}{3.201775in}}%
\pgfpathcurveto{\pgfqpoint{4.280819in}{3.193962in}}{\pgfqpoint{4.291418in}{3.189572in}}{\pgfqpoint{4.302468in}{3.189572in}}%
\pgfpathclose%
\pgfusepath{stroke,fill}%
\end{pgfscope}%
\begin{pgfscope}%
\pgfpathrectangle{\pgfqpoint{0.648703in}{0.548769in}}{\pgfqpoint{5.201297in}{3.102590in}}%
\pgfusepath{clip}%
\pgfsetbuttcap%
\pgfsetroundjoin%
\definecolor{currentfill}{rgb}{1.000000,0.498039,0.054902}%
\pgfsetfillcolor{currentfill}%
\pgfsetlinewidth{1.003750pt}%
\definecolor{currentstroke}{rgb}{1.000000,0.498039,0.054902}%
\pgfsetstrokecolor{currentstroke}%
\pgfsetdash{}{0pt}%
\pgfpathmoveto{\pgfqpoint{2.985027in}{3.189572in}}%
\pgfpathcurveto{\pgfqpoint{2.996077in}{3.189572in}}{\pgfqpoint{3.006676in}{3.193962in}}{\pgfqpoint{3.014490in}{3.201775in}}%
\pgfpathcurveto{\pgfqpoint{3.022303in}{3.209589in}}{\pgfqpoint{3.026693in}{3.220188in}}{\pgfqpoint{3.026693in}{3.231238in}}%
\pgfpathcurveto{\pgfqpoint{3.026693in}{3.242288in}}{\pgfqpoint{3.022303in}{3.252887in}}{\pgfqpoint{3.014490in}{3.260701in}}%
\pgfpathcurveto{\pgfqpoint{3.006676in}{3.268515in}}{\pgfqpoint{2.996077in}{3.272905in}}{\pgfqpoint{2.985027in}{3.272905in}}%
\pgfpathcurveto{\pgfqpoint{2.973977in}{3.272905in}}{\pgfqpoint{2.963378in}{3.268515in}}{\pgfqpoint{2.955564in}{3.260701in}}%
\pgfpathcurveto{\pgfqpoint{2.947750in}{3.252887in}}{\pgfqpoint{2.943360in}{3.242288in}}{\pgfqpoint{2.943360in}{3.231238in}}%
\pgfpathcurveto{\pgfqpoint{2.943360in}{3.220188in}}{\pgfqpoint{2.947750in}{3.209589in}}{\pgfqpoint{2.955564in}{3.201775in}}%
\pgfpathcurveto{\pgfqpoint{2.963378in}{3.193962in}}{\pgfqpoint{2.973977in}{3.189572in}}{\pgfqpoint{2.985027in}{3.189572in}}%
\pgfpathclose%
\pgfusepath{stroke,fill}%
\end{pgfscope}%
\begin{pgfscope}%
\pgfpathrectangle{\pgfqpoint{0.648703in}{0.548769in}}{\pgfqpoint{5.201297in}{3.102590in}}%
\pgfusepath{clip}%
\pgfsetbuttcap%
\pgfsetroundjoin%
\definecolor{currentfill}{rgb}{0.121569,0.466667,0.705882}%
\pgfsetfillcolor{currentfill}%
\pgfsetlinewidth{1.003750pt}%
\definecolor{currentstroke}{rgb}{0.121569,0.466667,0.705882}%
\pgfsetstrokecolor{currentstroke}%
\pgfsetdash{}{0pt}%
\pgfpathmoveto{\pgfqpoint{4.277449in}{3.181114in}}%
\pgfpathcurveto{\pgfqpoint{4.288499in}{3.181114in}}{\pgfqpoint{4.299098in}{3.185504in}}{\pgfqpoint{4.306912in}{3.193318in}}%
\pgfpathcurveto{\pgfqpoint{4.314725in}{3.201132in}}{\pgfqpoint{4.319116in}{3.211731in}}{\pgfqpoint{4.319116in}{3.222781in}}%
\pgfpathcurveto{\pgfqpoint{4.319116in}{3.233831in}}{\pgfqpoint{4.314725in}{3.244430in}}{\pgfqpoint{4.306912in}{3.252244in}}%
\pgfpathcurveto{\pgfqpoint{4.299098in}{3.260057in}}{\pgfqpoint{4.288499in}{3.264448in}}{\pgfqpoint{4.277449in}{3.264448in}}%
\pgfpathcurveto{\pgfqpoint{4.266399in}{3.264448in}}{\pgfqpoint{4.255800in}{3.260057in}}{\pgfqpoint{4.247986in}{3.252244in}}%
\pgfpathcurveto{\pgfqpoint{4.240172in}{3.244430in}}{\pgfqpoint{4.235782in}{3.233831in}}{\pgfqpoint{4.235782in}{3.222781in}}%
\pgfpathcurveto{\pgfqpoint{4.235782in}{3.211731in}}{\pgfqpoint{4.240172in}{3.201132in}}{\pgfqpoint{4.247986in}{3.193318in}}%
\pgfpathcurveto{\pgfqpoint{4.255800in}{3.185504in}}{\pgfqpoint{4.266399in}{3.181114in}}{\pgfqpoint{4.277449in}{3.181114in}}%
\pgfpathclose%
\pgfusepath{stroke,fill}%
\end{pgfscope}%
\begin{pgfscope}%
\pgfpathrectangle{\pgfqpoint{0.648703in}{0.548769in}}{\pgfqpoint{5.201297in}{3.102590in}}%
\pgfusepath{clip}%
\pgfsetbuttcap%
\pgfsetroundjoin%
\definecolor{currentfill}{rgb}{1.000000,0.498039,0.054902}%
\pgfsetfillcolor{currentfill}%
\pgfsetlinewidth{1.003750pt}%
\definecolor{currentstroke}{rgb}{1.000000,0.498039,0.054902}%
\pgfsetstrokecolor{currentstroke}%
\pgfsetdash{}{0pt}%
\pgfpathmoveto{\pgfqpoint{4.912758in}{3.185343in}}%
\pgfpathcurveto{\pgfqpoint{4.923808in}{3.185343in}}{\pgfqpoint{4.934407in}{3.189733in}}{\pgfqpoint{4.942221in}{3.197547in}}%
\pgfpathcurveto{\pgfqpoint{4.950034in}{3.205360in}}{\pgfqpoint{4.954425in}{3.215959in}}{\pgfqpoint{4.954425in}{3.227010in}}%
\pgfpathcurveto{\pgfqpoint{4.954425in}{3.238060in}}{\pgfqpoint{4.950034in}{3.248659in}}{\pgfqpoint{4.942221in}{3.256472in}}%
\pgfpathcurveto{\pgfqpoint{4.934407in}{3.264286in}}{\pgfqpoint{4.923808in}{3.268676in}}{\pgfqpoint{4.912758in}{3.268676in}}%
\pgfpathcurveto{\pgfqpoint{4.901708in}{3.268676in}}{\pgfqpoint{4.891109in}{3.264286in}}{\pgfqpoint{4.883295in}{3.256472in}}%
\pgfpathcurveto{\pgfqpoint{4.875482in}{3.248659in}}{\pgfqpoint{4.871091in}{3.238060in}}{\pgfqpoint{4.871091in}{3.227010in}}%
\pgfpathcurveto{\pgfqpoint{4.871091in}{3.215959in}}{\pgfqpoint{4.875482in}{3.205360in}}{\pgfqpoint{4.883295in}{3.197547in}}%
\pgfpathcurveto{\pgfqpoint{4.891109in}{3.189733in}}{\pgfqpoint{4.901708in}{3.185343in}}{\pgfqpoint{4.912758in}{3.185343in}}%
\pgfpathclose%
\pgfusepath{stroke,fill}%
\end{pgfscope}%
\begin{pgfscope}%
\pgfpathrectangle{\pgfqpoint{0.648703in}{0.548769in}}{\pgfqpoint{5.201297in}{3.102590in}}%
\pgfusepath{clip}%
\pgfsetbuttcap%
\pgfsetroundjoin%
\definecolor{currentfill}{rgb}{1.000000,0.498039,0.054902}%
\pgfsetfillcolor{currentfill}%
\pgfsetlinewidth{1.003750pt}%
\definecolor{currentstroke}{rgb}{1.000000,0.498039,0.054902}%
\pgfsetstrokecolor{currentstroke}%
\pgfsetdash{}{0pt}%
\pgfpathmoveto{\pgfqpoint{4.695740in}{3.358719in}}%
\pgfpathcurveto{\pgfqpoint{4.706790in}{3.358719in}}{\pgfqpoint{4.717389in}{3.363109in}}{\pgfqpoint{4.725202in}{3.370923in}}%
\pgfpathcurveto{\pgfqpoint{4.733016in}{3.378737in}}{\pgfqpoint{4.737406in}{3.389336in}}{\pgfqpoint{4.737406in}{3.400386in}}%
\pgfpathcurveto{\pgfqpoint{4.737406in}{3.411436in}}{\pgfqpoint{4.733016in}{3.422035in}}{\pgfqpoint{4.725202in}{3.429849in}}%
\pgfpathcurveto{\pgfqpoint{4.717389in}{3.437662in}}{\pgfqpoint{4.706790in}{3.442053in}}{\pgfqpoint{4.695740in}{3.442053in}}%
\pgfpathcurveto{\pgfqpoint{4.684689in}{3.442053in}}{\pgfqpoint{4.674090in}{3.437662in}}{\pgfqpoint{4.666277in}{3.429849in}}%
\pgfpathcurveto{\pgfqpoint{4.658463in}{3.422035in}}{\pgfqpoint{4.654073in}{3.411436in}}{\pgfqpoint{4.654073in}{3.400386in}}%
\pgfpathcurveto{\pgfqpoint{4.654073in}{3.389336in}}{\pgfqpoint{4.658463in}{3.378737in}}{\pgfqpoint{4.666277in}{3.370923in}}%
\pgfpathcurveto{\pgfqpoint{4.674090in}{3.363109in}}{\pgfqpoint{4.684689in}{3.358719in}}{\pgfqpoint{4.695740in}{3.358719in}}%
\pgfpathclose%
\pgfusepath{stroke,fill}%
\end{pgfscope}%
\begin{pgfscope}%
\pgfpathrectangle{\pgfqpoint{0.648703in}{0.548769in}}{\pgfqpoint{5.201297in}{3.102590in}}%
\pgfusepath{clip}%
\pgfsetbuttcap%
\pgfsetroundjoin%
\definecolor{currentfill}{rgb}{1.000000,0.498039,0.054902}%
\pgfsetfillcolor{currentfill}%
\pgfsetlinewidth{1.003750pt}%
\definecolor{currentstroke}{rgb}{1.000000,0.498039,0.054902}%
\pgfsetstrokecolor{currentstroke}%
\pgfsetdash{}{0pt}%
\pgfpathmoveto{\pgfqpoint{4.209057in}{3.198029in}}%
\pgfpathcurveto{\pgfqpoint{4.220107in}{3.198029in}}{\pgfqpoint{4.230706in}{3.202419in}}{\pgfqpoint{4.238520in}{3.210233in}}%
\pgfpathcurveto{\pgfqpoint{4.246333in}{3.218046in}}{\pgfqpoint{4.250724in}{3.228646in}}{\pgfqpoint{4.250724in}{3.239696in}}%
\pgfpathcurveto{\pgfqpoint{4.250724in}{3.250746in}}{\pgfqpoint{4.246333in}{3.261345in}}{\pgfqpoint{4.238520in}{3.269158in}}%
\pgfpathcurveto{\pgfqpoint{4.230706in}{3.276972in}}{\pgfqpoint{4.220107in}{3.281362in}}{\pgfqpoint{4.209057in}{3.281362in}}%
\pgfpathcurveto{\pgfqpoint{4.198007in}{3.281362in}}{\pgfqpoint{4.187408in}{3.276972in}}{\pgfqpoint{4.179594in}{3.269158in}}%
\pgfpathcurveto{\pgfqpoint{4.171781in}{3.261345in}}{\pgfqpoint{4.167390in}{3.250746in}}{\pgfqpoint{4.167390in}{3.239696in}}%
\pgfpathcurveto{\pgfqpoint{4.167390in}{3.228646in}}{\pgfqpoint{4.171781in}{3.218046in}}{\pgfqpoint{4.179594in}{3.210233in}}%
\pgfpathcurveto{\pgfqpoint{4.187408in}{3.202419in}}{\pgfqpoint{4.198007in}{3.198029in}}{\pgfqpoint{4.209057in}{3.198029in}}%
\pgfpathclose%
\pgfusepath{stroke,fill}%
\end{pgfscope}%
\begin{pgfscope}%
\pgfpathrectangle{\pgfqpoint{0.648703in}{0.548769in}}{\pgfqpoint{5.201297in}{3.102590in}}%
\pgfusepath{clip}%
\pgfsetbuttcap%
\pgfsetroundjoin%
\definecolor{currentfill}{rgb}{0.839216,0.152941,0.156863}%
\pgfsetfillcolor{currentfill}%
\pgfsetlinewidth{1.003750pt}%
\definecolor{currentstroke}{rgb}{0.839216,0.152941,0.156863}%
\pgfsetstrokecolor{currentstroke}%
\pgfsetdash{}{0pt}%
\pgfpathmoveto{\pgfqpoint{4.569686in}{3.193800in}}%
\pgfpathcurveto{\pgfqpoint{4.580736in}{3.193800in}}{\pgfqpoint{4.591335in}{3.198191in}}{\pgfqpoint{4.599148in}{3.206004in}}%
\pgfpathcurveto{\pgfqpoint{4.606962in}{3.213818in}}{\pgfqpoint{4.611352in}{3.224417in}}{\pgfqpoint{4.611352in}{3.235467in}}%
\pgfpathcurveto{\pgfqpoint{4.611352in}{3.246517in}}{\pgfqpoint{4.606962in}{3.257116in}}{\pgfqpoint{4.599148in}{3.264930in}}%
\pgfpathcurveto{\pgfqpoint{4.591335in}{3.272743in}}{\pgfqpoint{4.580736in}{3.277134in}}{\pgfqpoint{4.569686in}{3.277134in}}%
\pgfpathcurveto{\pgfqpoint{4.558636in}{3.277134in}}{\pgfqpoint{4.548036in}{3.272743in}}{\pgfqpoint{4.540223in}{3.264930in}}%
\pgfpathcurveto{\pgfqpoint{4.532409in}{3.257116in}}{\pgfqpoint{4.528019in}{3.246517in}}{\pgfqpoint{4.528019in}{3.235467in}}%
\pgfpathcurveto{\pgfqpoint{4.528019in}{3.224417in}}{\pgfqpoint{4.532409in}{3.213818in}}{\pgfqpoint{4.540223in}{3.206004in}}%
\pgfpathcurveto{\pgfqpoint{4.548036in}{3.198191in}}{\pgfqpoint{4.558636in}{3.193800in}}{\pgfqpoint{4.569686in}{3.193800in}}%
\pgfpathclose%
\pgfusepath{stroke,fill}%
\end{pgfscope}%
\begin{pgfscope}%
\pgfpathrectangle{\pgfqpoint{0.648703in}{0.548769in}}{\pgfqpoint{5.201297in}{3.102590in}}%
\pgfusepath{clip}%
\pgfsetbuttcap%
\pgfsetroundjoin%
\definecolor{currentfill}{rgb}{1.000000,0.498039,0.054902}%
\pgfsetfillcolor{currentfill}%
\pgfsetlinewidth{1.003750pt}%
\definecolor{currentstroke}{rgb}{1.000000,0.498039,0.054902}%
\pgfsetstrokecolor{currentstroke}%
\pgfsetdash{}{0pt}%
\pgfpathmoveto{\pgfqpoint{2.977627in}{3.193800in}}%
\pgfpathcurveto{\pgfqpoint{2.988677in}{3.193800in}}{\pgfqpoint{2.999276in}{3.198191in}}{\pgfqpoint{3.007089in}{3.206004in}}%
\pgfpathcurveto{\pgfqpoint{3.014903in}{3.213818in}}{\pgfqpoint{3.019293in}{3.224417in}}{\pgfqpoint{3.019293in}{3.235467in}}%
\pgfpathcurveto{\pgfqpoint{3.019293in}{3.246517in}}{\pgfqpoint{3.014903in}{3.257116in}}{\pgfqpoint{3.007089in}{3.264930in}}%
\pgfpathcurveto{\pgfqpoint{2.999276in}{3.272743in}}{\pgfqpoint{2.988677in}{3.277134in}}{\pgfqpoint{2.977627in}{3.277134in}}%
\pgfpathcurveto{\pgfqpoint{2.966576in}{3.277134in}}{\pgfqpoint{2.955977in}{3.272743in}}{\pgfqpoint{2.948164in}{3.264930in}}%
\pgfpathcurveto{\pgfqpoint{2.940350in}{3.257116in}}{\pgfqpoint{2.935960in}{3.246517in}}{\pgfqpoint{2.935960in}{3.235467in}}%
\pgfpathcurveto{\pgfqpoint{2.935960in}{3.224417in}}{\pgfqpoint{2.940350in}{3.213818in}}{\pgfqpoint{2.948164in}{3.206004in}}%
\pgfpathcurveto{\pgfqpoint{2.955977in}{3.198191in}}{\pgfqpoint{2.966576in}{3.193800in}}{\pgfqpoint{2.977627in}{3.193800in}}%
\pgfpathclose%
\pgfusepath{stroke,fill}%
\end{pgfscope}%
\begin{pgfscope}%
\pgfpathrectangle{\pgfqpoint{0.648703in}{0.548769in}}{\pgfqpoint{5.201297in}{3.102590in}}%
\pgfusepath{clip}%
\pgfsetbuttcap%
\pgfsetroundjoin%
\definecolor{currentfill}{rgb}{1.000000,0.498039,0.054902}%
\pgfsetfillcolor{currentfill}%
\pgfsetlinewidth{1.003750pt}%
\definecolor{currentstroke}{rgb}{1.000000,0.498039,0.054902}%
\pgfsetstrokecolor{currentstroke}%
\pgfsetdash{}{0pt}%
\pgfpathmoveto{\pgfqpoint{4.982588in}{3.185343in}}%
\pgfpathcurveto{\pgfqpoint{4.993638in}{3.185343in}}{\pgfqpoint{5.004237in}{3.189733in}}{\pgfqpoint{5.012051in}{3.197547in}}%
\pgfpathcurveto{\pgfqpoint{5.019865in}{3.205360in}}{\pgfqpoint{5.024255in}{3.215959in}}{\pgfqpoint{5.024255in}{3.227010in}}%
\pgfpathcurveto{\pgfqpoint{5.024255in}{3.238060in}}{\pgfqpoint{5.019865in}{3.248659in}}{\pgfqpoint{5.012051in}{3.256472in}}%
\pgfpathcurveto{\pgfqpoint{5.004237in}{3.264286in}}{\pgfqpoint{4.993638in}{3.268676in}}{\pgfqpoint{4.982588in}{3.268676in}}%
\pgfpathcurveto{\pgfqpoint{4.971538in}{3.268676in}}{\pgfqpoint{4.960939in}{3.264286in}}{\pgfqpoint{4.953125in}{3.256472in}}%
\pgfpathcurveto{\pgfqpoint{4.945312in}{3.248659in}}{\pgfqpoint{4.940922in}{3.238060in}}{\pgfqpoint{4.940922in}{3.227010in}}%
\pgfpathcurveto{\pgfqpoint{4.940922in}{3.215959in}}{\pgfqpoint{4.945312in}{3.205360in}}{\pgfqpoint{4.953125in}{3.197547in}}%
\pgfpathcurveto{\pgfqpoint{4.960939in}{3.189733in}}{\pgfqpoint{4.971538in}{3.185343in}}{\pgfqpoint{4.982588in}{3.185343in}}%
\pgfpathclose%
\pgfusepath{stroke,fill}%
\end{pgfscope}%
\begin{pgfscope}%
\pgfpathrectangle{\pgfqpoint{0.648703in}{0.548769in}}{\pgfqpoint{5.201297in}{3.102590in}}%
\pgfusepath{clip}%
\pgfsetbuttcap%
\pgfsetroundjoin%
\definecolor{currentfill}{rgb}{1.000000,0.498039,0.054902}%
\pgfsetfillcolor{currentfill}%
\pgfsetlinewidth{1.003750pt}%
\definecolor{currentstroke}{rgb}{1.000000,0.498039,0.054902}%
\pgfsetstrokecolor{currentstroke}%
\pgfsetdash{}{0pt}%
\pgfpathmoveto{\pgfqpoint{4.950044in}{3.185343in}}%
\pgfpathcurveto{\pgfqpoint{4.961095in}{3.185343in}}{\pgfqpoint{4.971694in}{3.189733in}}{\pgfqpoint{4.979507in}{3.197547in}}%
\pgfpathcurveto{\pgfqpoint{4.987321in}{3.205360in}}{\pgfqpoint{4.991711in}{3.215959in}}{\pgfqpoint{4.991711in}{3.227010in}}%
\pgfpathcurveto{\pgfqpoint{4.991711in}{3.238060in}}{\pgfqpoint{4.987321in}{3.248659in}}{\pgfqpoint{4.979507in}{3.256472in}}%
\pgfpathcurveto{\pgfqpoint{4.971694in}{3.264286in}}{\pgfqpoint{4.961095in}{3.268676in}}{\pgfqpoint{4.950044in}{3.268676in}}%
\pgfpathcurveto{\pgfqpoint{4.938994in}{3.268676in}}{\pgfqpoint{4.928395in}{3.264286in}}{\pgfqpoint{4.920582in}{3.256472in}}%
\pgfpathcurveto{\pgfqpoint{4.912768in}{3.248659in}}{\pgfqpoint{4.908378in}{3.238060in}}{\pgfqpoint{4.908378in}{3.227010in}}%
\pgfpathcurveto{\pgfqpoint{4.908378in}{3.215959in}}{\pgfqpoint{4.912768in}{3.205360in}}{\pgfqpoint{4.920582in}{3.197547in}}%
\pgfpathcurveto{\pgfqpoint{4.928395in}{3.189733in}}{\pgfqpoint{4.938994in}{3.185343in}}{\pgfqpoint{4.950044in}{3.185343in}}%
\pgfpathclose%
\pgfusepath{stroke,fill}%
\end{pgfscope}%
\begin{pgfscope}%
\pgfpathrectangle{\pgfqpoint{0.648703in}{0.548769in}}{\pgfqpoint{5.201297in}{3.102590in}}%
\pgfusepath{clip}%
\pgfsetbuttcap%
\pgfsetroundjoin%
\definecolor{currentfill}{rgb}{1.000000,0.498039,0.054902}%
\pgfsetfillcolor{currentfill}%
\pgfsetlinewidth{1.003750pt}%
\definecolor{currentstroke}{rgb}{1.000000,0.498039,0.054902}%
\pgfsetstrokecolor{currentstroke}%
\pgfsetdash{}{0pt}%
\pgfpathmoveto{\pgfqpoint{3.927520in}{3.193800in}}%
\pgfpathcurveto{\pgfqpoint{3.938570in}{3.193800in}}{\pgfqpoint{3.949169in}{3.198191in}}{\pgfqpoint{3.956983in}{3.206004in}}%
\pgfpathcurveto{\pgfqpoint{3.964796in}{3.213818in}}{\pgfqpoint{3.969186in}{3.224417in}}{\pgfqpoint{3.969186in}{3.235467in}}%
\pgfpathcurveto{\pgfqpoint{3.969186in}{3.246517in}}{\pgfqpoint{3.964796in}{3.257116in}}{\pgfqpoint{3.956983in}{3.264930in}}%
\pgfpathcurveto{\pgfqpoint{3.949169in}{3.272743in}}{\pgfqpoint{3.938570in}{3.277134in}}{\pgfqpoint{3.927520in}{3.277134in}}%
\pgfpathcurveto{\pgfqpoint{3.916470in}{3.277134in}}{\pgfqpoint{3.905871in}{3.272743in}}{\pgfqpoint{3.898057in}{3.264930in}}%
\pgfpathcurveto{\pgfqpoint{3.890243in}{3.257116in}}{\pgfqpoint{3.885853in}{3.246517in}}{\pgfqpoint{3.885853in}{3.235467in}}%
\pgfpathcurveto{\pgfqpoint{3.885853in}{3.224417in}}{\pgfqpoint{3.890243in}{3.213818in}}{\pgfqpoint{3.898057in}{3.206004in}}%
\pgfpathcurveto{\pgfqpoint{3.905871in}{3.198191in}}{\pgfqpoint{3.916470in}{3.193800in}}{\pgfqpoint{3.927520in}{3.193800in}}%
\pgfpathclose%
\pgfusepath{stroke,fill}%
\end{pgfscope}%
\begin{pgfscope}%
\pgfpathrectangle{\pgfqpoint{0.648703in}{0.548769in}}{\pgfqpoint{5.201297in}{3.102590in}}%
\pgfusepath{clip}%
\pgfsetbuttcap%
\pgfsetroundjoin%
\definecolor{currentfill}{rgb}{0.839216,0.152941,0.156863}%
\pgfsetfillcolor{currentfill}%
\pgfsetlinewidth{1.003750pt}%
\definecolor{currentstroke}{rgb}{0.839216,0.152941,0.156863}%
\pgfsetstrokecolor{currentstroke}%
\pgfsetdash{}{0pt}%
\pgfpathmoveto{\pgfqpoint{5.613577in}{3.189572in}}%
\pgfpathcurveto{\pgfqpoint{5.624628in}{3.189572in}}{\pgfqpoint{5.635227in}{3.193962in}}{\pgfqpoint{5.643040in}{3.201775in}}%
\pgfpathcurveto{\pgfqpoint{5.650854in}{3.209589in}}{\pgfqpoint{5.655244in}{3.220188in}}{\pgfqpoint{5.655244in}{3.231238in}}%
\pgfpathcurveto{\pgfqpoint{5.655244in}{3.242288in}}{\pgfqpoint{5.650854in}{3.252887in}}{\pgfqpoint{5.643040in}{3.260701in}}%
\pgfpathcurveto{\pgfqpoint{5.635227in}{3.268515in}}{\pgfqpoint{5.624628in}{3.272905in}}{\pgfqpoint{5.613577in}{3.272905in}}%
\pgfpathcurveto{\pgfqpoint{5.602527in}{3.272905in}}{\pgfqpoint{5.591928in}{3.268515in}}{\pgfqpoint{5.584115in}{3.260701in}}%
\pgfpathcurveto{\pgfqpoint{5.576301in}{3.252887in}}{\pgfqpoint{5.571911in}{3.242288in}}{\pgfqpoint{5.571911in}{3.231238in}}%
\pgfpathcurveto{\pgfqpoint{5.571911in}{3.220188in}}{\pgfqpoint{5.576301in}{3.209589in}}{\pgfqpoint{5.584115in}{3.201775in}}%
\pgfpathcurveto{\pgfqpoint{5.591928in}{3.193962in}}{\pgfqpoint{5.602527in}{3.189572in}}{\pgfqpoint{5.613577in}{3.189572in}}%
\pgfpathclose%
\pgfusepath{stroke,fill}%
\end{pgfscope}%
\begin{pgfscope}%
\pgfpathrectangle{\pgfqpoint{0.648703in}{0.548769in}}{\pgfqpoint{5.201297in}{3.102590in}}%
\pgfusepath{clip}%
\pgfsetbuttcap%
\pgfsetroundjoin%
\definecolor{currentfill}{rgb}{1.000000,0.498039,0.054902}%
\pgfsetfillcolor{currentfill}%
\pgfsetlinewidth{1.003750pt}%
\definecolor{currentstroke}{rgb}{1.000000,0.498039,0.054902}%
\pgfsetstrokecolor{currentstroke}%
\pgfsetdash{}{0pt}%
\pgfpathmoveto{\pgfqpoint{4.701406in}{3.198029in}}%
\pgfpathcurveto{\pgfqpoint{4.712456in}{3.198029in}}{\pgfqpoint{4.723055in}{3.202419in}}{\pgfqpoint{4.730868in}{3.210233in}}%
\pgfpathcurveto{\pgfqpoint{4.738682in}{3.218046in}}{\pgfqpoint{4.743072in}{3.228646in}}{\pgfqpoint{4.743072in}{3.239696in}}%
\pgfpathcurveto{\pgfqpoint{4.743072in}{3.250746in}}{\pgfqpoint{4.738682in}{3.261345in}}{\pgfqpoint{4.730868in}{3.269158in}}%
\pgfpathcurveto{\pgfqpoint{4.723055in}{3.276972in}}{\pgfqpoint{4.712456in}{3.281362in}}{\pgfqpoint{4.701406in}{3.281362in}}%
\pgfpathcurveto{\pgfqpoint{4.690356in}{3.281362in}}{\pgfqpoint{4.679756in}{3.276972in}}{\pgfqpoint{4.671943in}{3.269158in}}%
\pgfpathcurveto{\pgfqpoint{4.664129in}{3.261345in}}{\pgfqpoint{4.659739in}{3.250746in}}{\pgfqpoint{4.659739in}{3.239696in}}%
\pgfpathcurveto{\pgfqpoint{4.659739in}{3.228646in}}{\pgfqpoint{4.664129in}{3.218046in}}{\pgfqpoint{4.671943in}{3.210233in}}%
\pgfpathcurveto{\pgfqpoint{4.679756in}{3.202419in}}{\pgfqpoint{4.690356in}{3.198029in}}{\pgfqpoint{4.701406in}{3.198029in}}%
\pgfpathclose%
\pgfusepath{stroke,fill}%
\end{pgfscope}%
\begin{pgfscope}%
\pgfpathrectangle{\pgfqpoint{0.648703in}{0.548769in}}{\pgfqpoint{5.201297in}{3.102590in}}%
\pgfusepath{clip}%
\pgfsetbuttcap%
\pgfsetroundjoin%
\definecolor{currentfill}{rgb}{0.121569,0.466667,0.705882}%
\pgfsetfillcolor{currentfill}%
\pgfsetlinewidth{1.003750pt}%
\definecolor{currentstroke}{rgb}{0.121569,0.466667,0.705882}%
\pgfsetstrokecolor{currentstroke}%
\pgfsetdash{}{0pt}%
\pgfpathmoveto{\pgfqpoint{3.564566in}{3.181114in}}%
\pgfpathcurveto{\pgfqpoint{3.575617in}{3.181114in}}{\pgfqpoint{3.586216in}{3.185504in}}{\pgfqpoint{3.594029in}{3.193318in}}%
\pgfpathcurveto{\pgfqpoint{3.601843in}{3.201132in}}{\pgfqpoint{3.606233in}{3.211731in}}{\pgfqpoint{3.606233in}{3.222781in}}%
\pgfpathcurveto{\pgfqpoint{3.606233in}{3.233831in}}{\pgfqpoint{3.601843in}{3.244430in}}{\pgfqpoint{3.594029in}{3.252244in}}%
\pgfpathcurveto{\pgfqpoint{3.586216in}{3.260057in}}{\pgfqpoint{3.575617in}{3.264448in}}{\pgfqpoint{3.564566in}{3.264448in}}%
\pgfpathcurveto{\pgfqpoint{3.553516in}{3.264448in}}{\pgfqpoint{3.542917in}{3.260057in}}{\pgfqpoint{3.535104in}{3.252244in}}%
\pgfpathcurveto{\pgfqpoint{3.527290in}{3.244430in}}{\pgfqpoint{3.522900in}{3.233831in}}{\pgfqpoint{3.522900in}{3.222781in}}%
\pgfpathcurveto{\pgfqpoint{3.522900in}{3.211731in}}{\pgfqpoint{3.527290in}{3.201132in}}{\pgfqpoint{3.535104in}{3.193318in}}%
\pgfpathcurveto{\pgfqpoint{3.542917in}{3.185504in}}{\pgfqpoint{3.553516in}{3.181114in}}{\pgfqpoint{3.564566in}{3.181114in}}%
\pgfpathclose%
\pgfusepath{stroke,fill}%
\end{pgfscope}%
\begin{pgfscope}%
\pgfpathrectangle{\pgfqpoint{0.648703in}{0.548769in}}{\pgfqpoint{5.201297in}{3.102590in}}%
\pgfusepath{clip}%
\pgfsetbuttcap%
\pgfsetroundjoin%
\definecolor{currentfill}{rgb}{0.121569,0.466667,0.705882}%
\pgfsetfillcolor{currentfill}%
\pgfsetlinewidth{1.003750pt}%
\definecolor{currentstroke}{rgb}{0.121569,0.466667,0.705882}%
\pgfsetstrokecolor{currentstroke}%
\pgfsetdash{}{0pt}%
\pgfpathmoveto{\pgfqpoint{1.305911in}{0.939909in}}%
\pgfpathcurveto{\pgfqpoint{1.316961in}{0.939909in}}{\pgfqpoint{1.327560in}{0.944299in}}{\pgfqpoint{1.335373in}{0.952112in}}%
\pgfpathcurveto{\pgfqpoint{1.343187in}{0.959926in}}{\pgfqpoint{1.347577in}{0.970525in}}{\pgfqpoint{1.347577in}{0.981575in}}%
\pgfpathcurveto{\pgfqpoint{1.347577in}{0.992625in}}{\pgfqpoint{1.343187in}{1.003224in}}{\pgfqpoint{1.335373in}{1.011038in}}%
\pgfpathcurveto{\pgfqpoint{1.327560in}{1.018852in}}{\pgfqpoint{1.316961in}{1.023242in}}{\pgfqpoint{1.305911in}{1.023242in}}%
\pgfpathcurveto{\pgfqpoint{1.294861in}{1.023242in}}{\pgfqpoint{1.284262in}{1.018852in}}{\pgfqpoint{1.276448in}{1.011038in}}%
\pgfpathcurveto{\pgfqpoint{1.268634in}{1.003224in}}{\pgfqpoint{1.264244in}{0.992625in}}{\pgfqpoint{1.264244in}{0.981575in}}%
\pgfpathcurveto{\pgfqpoint{1.264244in}{0.970525in}}{\pgfqpoint{1.268634in}{0.959926in}}{\pgfqpoint{1.276448in}{0.952112in}}%
\pgfpathcurveto{\pgfqpoint{1.284262in}{0.944299in}}{\pgfqpoint{1.294861in}{0.939909in}}{\pgfqpoint{1.305911in}{0.939909in}}%
\pgfpathclose%
\pgfusepath{stroke,fill}%
\end{pgfscope}%
\begin{pgfscope}%
\pgfpathrectangle{\pgfqpoint{0.648703in}{0.548769in}}{\pgfqpoint{5.201297in}{3.102590in}}%
\pgfusepath{clip}%
\pgfsetbuttcap%
\pgfsetroundjoin%
\definecolor{currentfill}{rgb}{0.839216,0.152941,0.156863}%
\pgfsetfillcolor{currentfill}%
\pgfsetlinewidth{1.003750pt}%
\definecolor{currentstroke}{rgb}{0.839216,0.152941,0.156863}%
\pgfsetstrokecolor{currentstroke}%
\pgfsetdash{}{0pt}%
\pgfpathmoveto{\pgfqpoint{5.194064in}{3.210715in}}%
\pgfpathcurveto{\pgfqpoint{5.205114in}{3.210715in}}{\pgfqpoint{5.215713in}{3.215105in}}{\pgfqpoint{5.223527in}{3.222919in}}%
\pgfpathcurveto{\pgfqpoint{5.231341in}{3.230733in}}{\pgfqpoint{5.235731in}{3.241332in}}{\pgfqpoint{5.235731in}{3.252382in}}%
\pgfpathcurveto{\pgfqpoint{5.235731in}{3.263432in}}{\pgfqpoint{5.231341in}{3.274031in}}{\pgfqpoint{5.223527in}{3.281844in}}%
\pgfpathcurveto{\pgfqpoint{5.215713in}{3.289658in}}{\pgfqpoint{5.205114in}{3.294048in}}{\pgfqpoint{5.194064in}{3.294048in}}%
\pgfpathcurveto{\pgfqpoint{5.183014in}{3.294048in}}{\pgfqpoint{5.172415in}{3.289658in}}{\pgfqpoint{5.164601in}{3.281844in}}%
\pgfpathcurveto{\pgfqpoint{5.156788in}{3.274031in}}{\pgfqpoint{5.152397in}{3.263432in}}{\pgfqpoint{5.152397in}{3.252382in}}%
\pgfpathcurveto{\pgfqpoint{5.152397in}{3.241332in}}{\pgfqpoint{5.156788in}{3.230733in}}{\pgfqpoint{5.164601in}{3.222919in}}%
\pgfpathcurveto{\pgfqpoint{5.172415in}{3.215105in}}{\pgfqpoint{5.183014in}{3.210715in}}{\pgfqpoint{5.194064in}{3.210715in}}%
\pgfpathclose%
\pgfusepath{stroke,fill}%
\end{pgfscope}%
\begin{pgfscope}%
\pgfpathrectangle{\pgfqpoint{0.648703in}{0.548769in}}{\pgfqpoint{5.201297in}{3.102590in}}%
\pgfusepath{clip}%
\pgfsetbuttcap%
\pgfsetroundjoin%
\definecolor{currentfill}{rgb}{1.000000,0.498039,0.054902}%
\pgfsetfillcolor{currentfill}%
\pgfsetlinewidth{1.003750pt}%
\definecolor{currentstroke}{rgb}{1.000000,0.498039,0.054902}%
\pgfsetstrokecolor{currentstroke}%
\pgfsetdash{}{0pt}%
\pgfpathmoveto{\pgfqpoint{4.149869in}{3.202258in}}%
\pgfpathcurveto{\pgfqpoint{4.160919in}{3.202258in}}{\pgfqpoint{4.171518in}{3.206648in}}{\pgfqpoint{4.179332in}{3.214462in}}%
\pgfpathcurveto{\pgfqpoint{4.187145in}{3.222275in}}{\pgfqpoint{4.191536in}{3.232874in}}{\pgfqpoint{4.191536in}{3.243924in}}%
\pgfpathcurveto{\pgfqpoint{4.191536in}{3.254974in}}{\pgfqpoint{4.187145in}{3.265573in}}{\pgfqpoint{4.179332in}{3.273387in}}%
\pgfpathcurveto{\pgfqpoint{4.171518in}{3.281201in}}{\pgfqpoint{4.160919in}{3.285591in}}{\pgfqpoint{4.149869in}{3.285591in}}%
\pgfpathcurveto{\pgfqpoint{4.138819in}{3.285591in}}{\pgfqpoint{4.128220in}{3.281201in}}{\pgfqpoint{4.120406in}{3.273387in}}%
\pgfpathcurveto{\pgfqpoint{4.112593in}{3.265573in}}{\pgfqpoint{4.108202in}{3.254974in}}{\pgfqpoint{4.108202in}{3.243924in}}%
\pgfpathcurveto{\pgfqpoint{4.108202in}{3.232874in}}{\pgfqpoint{4.112593in}{3.222275in}}{\pgfqpoint{4.120406in}{3.214462in}}%
\pgfpathcurveto{\pgfqpoint{4.128220in}{3.206648in}}{\pgfqpoint{4.138819in}{3.202258in}}{\pgfqpoint{4.149869in}{3.202258in}}%
\pgfpathclose%
\pgfusepath{stroke,fill}%
\end{pgfscope}%
\begin{pgfscope}%
\pgfpathrectangle{\pgfqpoint{0.648703in}{0.548769in}}{\pgfqpoint{5.201297in}{3.102590in}}%
\pgfusepath{clip}%
\pgfsetbuttcap%
\pgfsetroundjoin%
\definecolor{currentfill}{rgb}{1.000000,0.498039,0.054902}%
\pgfsetfillcolor{currentfill}%
\pgfsetlinewidth{1.003750pt}%
\definecolor{currentstroke}{rgb}{1.000000,0.498039,0.054902}%
\pgfsetstrokecolor{currentstroke}%
\pgfsetdash{}{0pt}%
\pgfpathmoveto{\pgfqpoint{3.675696in}{3.206486in}}%
\pgfpathcurveto{\pgfqpoint{3.686747in}{3.206486in}}{\pgfqpoint{3.697346in}{3.210877in}}{\pgfqpoint{3.705159in}{3.218690in}}%
\pgfpathcurveto{\pgfqpoint{3.712973in}{3.226504in}}{\pgfqpoint{3.717363in}{3.237103in}}{\pgfqpoint{3.717363in}{3.248153in}}%
\pgfpathcurveto{\pgfqpoint{3.717363in}{3.259203in}}{\pgfqpoint{3.712973in}{3.269802in}}{\pgfqpoint{3.705159in}{3.277616in}}%
\pgfpathcurveto{\pgfqpoint{3.697346in}{3.285429in}}{\pgfqpoint{3.686747in}{3.289820in}}{\pgfqpoint{3.675696in}{3.289820in}}%
\pgfpathcurveto{\pgfqpoint{3.664646in}{3.289820in}}{\pgfqpoint{3.654047in}{3.285429in}}{\pgfqpoint{3.646234in}{3.277616in}}%
\pgfpathcurveto{\pgfqpoint{3.638420in}{3.269802in}}{\pgfqpoint{3.634030in}{3.259203in}}{\pgfqpoint{3.634030in}{3.248153in}}%
\pgfpathcurveto{\pgfqpoint{3.634030in}{3.237103in}}{\pgfqpoint{3.638420in}{3.226504in}}{\pgfqpoint{3.646234in}{3.218690in}}%
\pgfpathcurveto{\pgfqpoint{3.654047in}{3.210877in}}{\pgfqpoint{3.664646in}{3.206486in}}{\pgfqpoint{3.675696in}{3.206486in}}%
\pgfpathclose%
\pgfusepath{stroke,fill}%
\end{pgfscope}%
\begin{pgfscope}%
\pgfpathrectangle{\pgfqpoint{0.648703in}{0.548769in}}{\pgfqpoint{5.201297in}{3.102590in}}%
\pgfusepath{clip}%
\pgfsetbuttcap%
\pgfsetroundjoin%
\definecolor{currentfill}{rgb}{1.000000,0.498039,0.054902}%
\pgfsetfillcolor{currentfill}%
\pgfsetlinewidth{1.003750pt}%
\definecolor{currentstroke}{rgb}{1.000000,0.498039,0.054902}%
\pgfsetstrokecolor{currentstroke}%
\pgfsetdash{}{0pt}%
\pgfpathmoveto{\pgfqpoint{4.366692in}{3.198029in}}%
\pgfpathcurveto{\pgfqpoint{4.377742in}{3.198029in}}{\pgfqpoint{4.388341in}{3.202419in}}{\pgfqpoint{4.396155in}{3.210233in}}%
\pgfpathcurveto{\pgfqpoint{4.403968in}{3.218046in}}{\pgfqpoint{4.408359in}{3.228646in}}{\pgfqpoint{4.408359in}{3.239696in}}%
\pgfpathcurveto{\pgfqpoint{4.408359in}{3.250746in}}{\pgfqpoint{4.403968in}{3.261345in}}{\pgfqpoint{4.396155in}{3.269158in}}%
\pgfpathcurveto{\pgfqpoint{4.388341in}{3.276972in}}{\pgfqpoint{4.377742in}{3.281362in}}{\pgfqpoint{4.366692in}{3.281362in}}%
\pgfpathcurveto{\pgfqpoint{4.355642in}{3.281362in}}{\pgfqpoint{4.345043in}{3.276972in}}{\pgfqpoint{4.337229in}{3.269158in}}%
\pgfpathcurveto{\pgfqpoint{4.329415in}{3.261345in}}{\pgfqpoint{4.325025in}{3.250746in}}{\pgfqpoint{4.325025in}{3.239696in}}%
\pgfpathcurveto{\pgfqpoint{4.325025in}{3.228646in}}{\pgfqpoint{4.329415in}{3.218046in}}{\pgfqpoint{4.337229in}{3.210233in}}%
\pgfpathcurveto{\pgfqpoint{4.345043in}{3.202419in}}{\pgfqpoint{4.355642in}{3.198029in}}{\pgfqpoint{4.366692in}{3.198029in}}%
\pgfpathclose%
\pgfusepath{stroke,fill}%
\end{pgfscope}%
\begin{pgfscope}%
\pgfpathrectangle{\pgfqpoint{0.648703in}{0.548769in}}{\pgfqpoint{5.201297in}{3.102590in}}%
\pgfusepath{clip}%
\pgfsetbuttcap%
\pgfsetroundjoin%
\definecolor{currentfill}{rgb}{1.000000,0.498039,0.054902}%
\pgfsetfillcolor{currentfill}%
\pgfsetlinewidth{1.003750pt}%
\definecolor{currentstroke}{rgb}{1.000000,0.498039,0.054902}%
\pgfsetstrokecolor{currentstroke}%
\pgfsetdash{}{0pt}%
\pgfpathmoveto{\pgfqpoint{4.773162in}{3.202258in}}%
\pgfpathcurveto{\pgfqpoint{4.784212in}{3.202258in}}{\pgfqpoint{4.794811in}{3.206648in}}{\pgfqpoint{4.802624in}{3.214462in}}%
\pgfpathcurveto{\pgfqpoint{4.810438in}{3.222275in}}{\pgfqpoint{4.814828in}{3.232874in}}{\pgfqpoint{4.814828in}{3.243924in}}%
\pgfpathcurveto{\pgfqpoint{4.814828in}{3.254974in}}{\pgfqpoint{4.810438in}{3.265573in}}{\pgfqpoint{4.802624in}{3.273387in}}%
\pgfpathcurveto{\pgfqpoint{4.794811in}{3.281201in}}{\pgfqpoint{4.784212in}{3.285591in}}{\pgfqpoint{4.773162in}{3.285591in}}%
\pgfpathcurveto{\pgfqpoint{4.762111in}{3.285591in}}{\pgfqpoint{4.751512in}{3.281201in}}{\pgfqpoint{4.743699in}{3.273387in}}%
\pgfpathcurveto{\pgfqpoint{4.735885in}{3.265573in}}{\pgfqpoint{4.731495in}{3.254974in}}{\pgfqpoint{4.731495in}{3.243924in}}%
\pgfpathcurveto{\pgfqpoint{4.731495in}{3.232874in}}{\pgfqpoint{4.735885in}{3.222275in}}{\pgfqpoint{4.743699in}{3.214462in}}%
\pgfpathcurveto{\pgfqpoint{4.751512in}{3.206648in}}{\pgfqpoint{4.762111in}{3.202258in}}{\pgfqpoint{4.773162in}{3.202258in}}%
\pgfpathclose%
\pgfusepath{stroke,fill}%
\end{pgfscope}%
\begin{pgfscope}%
\pgfpathrectangle{\pgfqpoint{0.648703in}{0.548769in}}{\pgfqpoint{5.201297in}{3.102590in}}%
\pgfusepath{clip}%
\pgfsetbuttcap%
\pgfsetroundjoin%
\definecolor{currentfill}{rgb}{1.000000,0.498039,0.054902}%
\pgfsetfillcolor{currentfill}%
\pgfsetlinewidth{1.003750pt}%
\definecolor{currentstroke}{rgb}{1.000000,0.498039,0.054902}%
\pgfsetstrokecolor{currentstroke}%
\pgfsetdash{}{0pt}%
\pgfpathmoveto{\pgfqpoint{3.332037in}{3.189572in}}%
\pgfpathcurveto{\pgfqpoint{3.343088in}{3.189572in}}{\pgfqpoint{3.353687in}{3.193962in}}{\pgfqpoint{3.361500in}{3.201775in}}%
\pgfpathcurveto{\pgfqpoint{3.369314in}{3.209589in}}{\pgfqpoint{3.373704in}{3.220188in}}{\pgfqpoint{3.373704in}{3.231238in}}%
\pgfpathcurveto{\pgfqpoint{3.373704in}{3.242288in}}{\pgfqpoint{3.369314in}{3.252887in}}{\pgfqpoint{3.361500in}{3.260701in}}%
\pgfpathcurveto{\pgfqpoint{3.353687in}{3.268515in}}{\pgfqpoint{3.343088in}{3.272905in}}{\pgfqpoint{3.332037in}{3.272905in}}%
\pgfpathcurveto{\pgfqpoint{3.320987in}{3.272905in}}{\pgfqpoint{3.310388in}{3.268515in}}{\pgfqpoint{3.302575in}{3.260701in}}%
\pgfpathcurveto{\pgfqpoint{3.294761in}{3.252887in}}{\pgfqpoint{3.290371in}{3.242288in}}{\pgfqpoint{3.290371in}{3.231238in}}%
\pgfpathcurveto{\pgfqpoint{3.290371in}{3.220188in}}{\pgfqpoint{3.294761in}{3.209589in}}{\pgfqpoint{3.302575in}{3.201775in}}%
\pgfpathcurveto{\pgfqpoint{3.310388in}{3.193962in}}{\pgfqpoint{3.320987in}{3.189572in}}{\pgfqpoint{3.332037in}{3.189572in}}%
\pgfpathclose%
\pgfusepath{stroke,fill}%
\end{pgfscope}%
\begin{pgfscope}%
\pgfpathrectangle{\pgfqpoint{0.648703in}{0.548769in}}{\pgfqpoint{5.201297in}{3.102590in}}%
\pgfusepath{clip}%
\pgfsetbuttcap%
\pgfsetroundjoin%
\definecolor{currentfill}{rgb}{0.121569,0.466667,0.705882}%
\pgfsetfillcolor{currentfill}%
\pgfsetlinewidth{1.003750pt}%
\definecolor{currentstroke}{rgb}{0.121569,0.466667,0.705882}%
\pgfsetstrokecolor{currentstroke}%
\pgfsetdash{}{0pt}%
\pgfpathmoveto{\pgfqpoint{4.143016in}{3.181114in}}%
\pgfpathcurveto{\pgfqpoint{4.154067in}{3.181114in}}{\pgfqpoint{4.164666in}{3.185504in}}{\pgfqpoint{4.172479in}{3.193318in}}%
\pgfpathcurveto{\pgfqpoint{4.180293in}{3.201132in}}{\pgfqpoint{4.184683in}{3.211731in}}{\pgfqpoint{4.184683in}{3.222781in}}%
\pgfpathcurveto{\pgfqpoint{4.184683in}{3.233831in}}{\pgfqpoint{4.180293in}{3.244430in}}{\pgfqpoint{4.172479in}{3.252244in}}%
\pgfpathcurveto{\pgfqpoint{4.164666in}{3.260057in}}{\pgfqpoint{4.154067in}{3.264448in}}{\pgfqpoint{4.143016in}{3.264448in}}%
\pgfpathcurveto{\pgfqpoint{4.131966in}{3.264448in}}{\pgfqpoint{4.121367in}{3.260057in}}{\pgfqpoint{4.113554in}{3.252244in}}%
\pgfpathcurveto{\pgfqpoint{4.105740in}{3.244430in}}{\pgfqpoint{4.101350in}{3.233831in}}{\pgfqpoint{4.101350in}{3.222781in}}%
\pgfpathcurveto{\pgfqpoint{4.101350in}{3.211731in}}{\pgfqpoint{4.105740in}{3.201132in}}{\pgfqpoint{4.113554in}{3.193318in}}%
\pgfpathcurveto{\pgfqpoint{4.121367in}{3.185504in}}{\pgfqpoint{4.131966in}{3.181114in}}{\pgfqpoint{4.143016in}{3.181114in}}%
\pgfpathclose%
\pgfusepath{stroke,fill}%
\end{pgfscope}%
\begin{pgfscope}%
\pgfpathrectangle{\pgfqpoint{0.648703in}{0.548769in}}{\pgfqpoint{5.201297in}{3.102590in}}%
\pgfusepath{clip}%
\pgfsetbuttcap%
\pgfsetroundjoin%
\definecolor{currentfill}{rgb}{1.000000,0.498039,0.054902}%
\pgfsetfillcolor{currentfill}%
\pgfsetlinewidth{1.003750pt}%
\definecolor{currentstroke}{rgb}{1.000000,0.498039,0.054902}%
\pgfsetstrokecolor{currentstroke}%
\pgfsetdash{}{0pt}%
\pgfpathmoveto{\pgfqpoint{3.822035in}{3.244545in}}%
\pgfpathcurveto{\pgfqpoint{3.833085in}{3.244545in}}{\pgfqpoint{3.843684in}{3.248935in}}{\pgfqpoint{3.851497in}{3.256748in}}%
\pgfpathcurveto{\pgfqpoint{3.859311in}{3.264562in}}{\pgfqpoint{3.863701in}{3.275161in}}{\pgfqpoint{3.863701in}{3.286211in}}%
\pgfpathcurveto{\pgfqpoint{3.863701in}{3.297261in}}{\pgfqpoint{3.859311in}{3.307860in}}{\pgfqpoint{3.851497in}{3.315674in}}%
\pgfpathcurveto{\pgfqpoint{3.843684in}{3.323488in}}{\pgfqpoint{3.833085in}{3.327878in}}{\pgfqpoint{3.822035in}{3.327878in}}%
\pgfpathcurveto{\pgfqpoint{3.810985in}{3.327878in}}{\pgfqpoint{3.800385in}{3.323488in}}{\pgfqpoint{3.792572in}{3.315674in}}%
\pgfpathcurveto{\pgfqpoint{3.784758in}{3.307860in}}{\pgfqpoint{3.780368in}{3.297261in}}{\pgfqpoint{3.780368in}{3.286211in}}%
\pgfpathcurveto{\pgfqpoint{3.780368in}{3.275161in}}{\pgfqpoint{3.784758in}{3.264562in}}{\pgfqpoint{3.792572in}{3.256748in}}%
\pgfpathcurveto{\pgfqpoint{3.800385in}{3.248935in}}{\pgfqpoint{3.810985in}{3.244545in}}{\pgfqpoint{3.822035in}{3.244545in}}%
\pgfpathclose%
\pgfusepath{stroke,fill}%
\end{pgfscope}%
\begin{pgfscope}%
\pgfpathrectangle{\pgfqpoint{0.648703in}{0.548769in}}{\pgfqpoint{5.201297in}{3.102590in}}%
\pgfusepath{clip}%
\pgfsetbuttcap%
\pgfsetroundjoin%
\definecolor{currentfill}{rgb}{1.000000,0.498039,0.054902}%
\pgfsetfillcolor{currentfill}%
\pgfsetlinewidth{1.003750pt}%
\definecolor{currentstroke}{rgb}{1.000000,0.498039,0.054902}%
\pgfsetstrokecolor{currentstroke}%
\pgfsetdash{}{0pt}%
\pgfpathmoveto{\pgfqpoint{3.516729in}{3.189572in}}%
\pgfpathcurveto{\pgfqpoint{3.527779in}{3.189572in}}{\pgfqpoint{3.538378in}{3.193962in}}{\pgfqpoint{3.546192in}{3.201775in}}%
\pgfpathcurveto{\pgfqpoint{3.554005in}{3.209589in}}{\pgfqpoint{3.558395in}{3.220188in}}{\pgfqpoint{3.558395in}{3.231238in}}%
\pgfpathcurveto{\pgfqpoint{3.558395in}{3.242288in}}{\pgfqpoint{3.554005in}{3.252887in}}{\pgfqpoint{3.546192in}{3.260701in}}%
\pgfpathcurveto{\pgfqpoint{3.538378in}{3.268515in}}{\pgfqpoint{3.527779in}{3.272905in}}{\pgfqpoint{3.516729in}{3.272905in}}%
\pgfpathcurveto{\pgfqpoint{3.505679in}{3.272905in}}{\pgfqpoint{3.495080in}{3.268515in}}{\pgfqpoint{3.487266in}{3.260701in}}%
\pgfpathcurveto{\pgfqpoint{3.479452in}{3.252887in}}{\pgfqpoint{3.475062in}{3.242288in}}{\pgfqpoint{3.475062in}{3.231238in}}%
\pgfpathcurveto{\pgfqpoint{3.475062in}{3.220188in}}{\pgfqpoint{3.479452in}{3.209589in}}{\pgfqpoint{3.487266in}{3.201775in}}%
\pgfpathcurveto{\pgfqpoint{3.495080in}{3.193962in}}{\pgfqpoint{3.505679in}{3.189572in}}{\pgfqpoint{3.516729in}{3.189572in}}%
\pgfpathclose%
\pgfusepath{stroke,fill}%
\end{pgfscope}%
\begin{pgfscope}%
\pgfpathrectangle{\pgfqpoint{0.648703in}{0.548769in}}{\pgfqpoint{5.201297in}{3.102590in}}%
\pgfusepath{clip}%
\pgfsetbuttcap%
\pgfsetroundjoin%
\definecolor{currentfill}{rgb}{1.000000,0.498039,0.054902}%
\pgfsetfillcolor{currentfill}%
\pgfsetlinewidth{1.003750pt}%
\definecolor{currentstroke}{rgb}{1.000000,0.498039,0.054902}%
\pgfsetstrokecolor{currentstroke}%
\pgfsetdash{}{0pt}%
\pgfpathmoveto{\pgfqpoint{5.018868in}{3.210715in}}%
\pgfpathcurveto{\pgfqpoint{5.029918in}{3.210715in}}{\pgfqpoint{5.040517in}{3.215105in}}{\pgfqpoint{5.048330in}{3.222919in}}%
\pgfpathcurveto{\pgfqpoint{5.056144in}{3.230733in}}{\pgfqpoint{5.060534in}{3.241332in}}{\pgfqpoint{5.060534in}{3.252382in}}%
\pgfpathcurveto{\pgfqpoint{5.060534in}{3.263432in}}{\pgfqpoint{5.056144in}{3.274031in}}{\pgfqpoint{5.048330in}{3.281844in}}%
\pgfpathcurveto{\pgfqpoint{5.040517in}{3.289658in}}{\pgfqpoint{5.029918in}{3.294048in}}{\pgfqpoint{5.018868in}{3.294048in}}%
\pgfpathcurveto{\pgfqpoint{5.007818in}{3.294048in}}{\pgfqpoint{4.997219in}{3.289658in}}{\pgfqpoint{4.989405in}{3.281844in}}%
\pgfpathcurveto{\pgfqpoint{4.981591in}{3.274031in}}{\pgfqpoint{4.977201in}{3.263432in}}{\pgfqpoint{4.977201in}{3.252382in}}%
\pgfpathcurveto{\pgfqpoint{4.977201in}{3.241332in}}{\pgfqpoint{4.981591in}{3.230733in}}{\pgfqpoint{4.989405in}{3.222919in}}%
\pgfpathcurveto{\pgfqpoint{4.997219in}{3.215105in}}{\pgfqpoint{5.007818in}{3.210715in}}{\pgfqpoint{5.018868in}{3.210715in}}%
\pgfpathclose%
\pgfusepath{stroke,fill}%
\end{pgfscope}%
\begin{pgfscope}%
\pgfpathrectangle{\pgfqpoint{0.648703in}{0.548769in}}{\pgfqpoint{5.201297in}{3.102590in}}%
\pgfusepath{clip}%
\pgfsetbuttcap%
\pgfsetroundjoin%
\definecolor{currentfill}{rgb}{1.000000,0.498039,0.054902}%
\pgfsetfillcolor{currentfill}%
\pgfsetlinewidth{1.003750pt}%
\definecolor{currentstroke}{rgb}{1.000000,0.498039,0.054902}%
\pgfsetstrokecolor{currentstroke}%
\pgfsetdash{}{0pt}%
\pgfpathmoveto{\pgfqpoint{3.927367in}{3.189572in}}%
\pgfpathcurveto{\pgfqpoint{3.938417in}{3.189572in}}{\pgfqpoint{3.949016in}{3.193962in}}{\pgfqpoint{3.956829in}{3.201775in}}%
\pgfpathcurveto{\pgfqpoint{3.964643in}{3.209589in}}{\pgfqpoint{3.969033in}{3.220188in}}{\pgfqpoint{3.969033in}{3.231238in}}%
\pgfpathcurveto{\pgfqpoint{3.969033in}{3.242288in}}{\pgfqpoint{3.964643in}{3.252887in}}{\pgfqpoint{3.956829in}{3.260701in}}%
\pgfpathcurveto{\pgfqpoint{3.949016in}{3.268515in}}{\pgfqpoint{3.938417in}{3.272905in}}{\pgfqpoint{3.927367in}{3.272905in}}%
\pgfpathcurveto{\pgfqpoint{3.916316in}{3.272905in}}{\pgfqpoint{3.905717in}{3.268515in}}{\pgfqpoint{3.897904in}{3.260701in}}%
\pgfpathcurveto{\pgfqpoint{3.890090in}{3.252887in}}{\pgfqpoint{3.885700in}{3.242288in}}{\pgfqpoint{3.885700in}{3.231238in}}%
\pgfpathcurveto{\pgfqpoint{3.885700in}{3.220188in}}{\pgfqpoint{3.890090in}{3.209589in}}{\pgfqpoint{3.897904in}{3.201775in}}%
\pgfpathcurveto{\pgfqpoint{3.905717in}{3.193962in}}{\pgfqpoint{3.916316in}{3.189572in}}{\pgfqpoint{3.927367in}{3.189572in}}%
\pgfpathclose%
\pgfusepath{stroke,fill}%
\end{pgfscope}%
\begin{pgfscope}%
\pgfpathrectangle{\pgfqpoint{0.648703in}{0.548769in}}{\pgfqpoint{5.201297in}{3.102590in}}%
\pgfusepath{clip}%
\pgfsetbuttcap%
\pgfsetroundjoin%
\definecolor{currentfill}{rgb}{0.839216,0.152941,0.156863}%
\pgfsetfillcolor{currentfill}%
\pgfsetlinewidth{1.003750pt}%
\definecolor{currentstroke}{rgb}{0.839216,0.152941,0.156863}%
\pgfsetstrokecolor{currentstroke}%
\pgfsetdash{}{0pt}%
\pgfpathmoveto{\pgfqpoint{3.776605in}{3.181114in}}%
\pgfpathcurveto{\pgfqpoint{3.787655in}{3.181114in}}{\pgfqpoint{3.798254in}{3.185504in}}{\pgfqpoint{3.806067in}{3.193318in}}%
\pgfpathcurveto{\pgfqpoint{3.813881in}{3.201132in}}{\pgfqpoint{3.818271in}{3.211731in}}{\pgfqpoint{3.818271in}{3.222781in}}%
\pgfpathcurveto{\pgfqpoint{3.818271in}{3.233831in}}{\pgfqpoint{3.813881in}{3.244430in}}{\pgfqpoint{3.806067in}{3.252244in}}%
\pgfpathcurveto{\pgfqpoint{3.798254in}{3.260057in}}{\pgfqpoint{3.787655in}{3.264448in}}{\pgfqpoint{3.776605in}{3.264448in}}%
\pgfpathcurveto{\pgfqpoint{3.765555in}{3.264448in}}{\pgfqpoint{3.754955in}{3.260057in}}{\pgfqpoint{3.747142in}{3.252244in}}%
\pgfpathcurveto{\pgfqpoint{3.739328in}{3.244430in}}{\pgfqpoint{3.734938in}{3.233831in}}{\pgfqpoint{3.734938in}{3.222781in}}%
\pgfpathcurveto{\pgfqpoint{3.734938in}{3.211731in}}{\pgfqpoint{3.739328in}{3.201132in}}{\pgfqpoint{3.747142in}{3.193318in}}%
\pgfpathcurveto{\pgfqpoint{3.754955in}{3.185504in}}{\pgfqpoint{3.765555in}{3.181114in}}{\pgfqpoint{3.776605in}{3.181114in}}%
\pgfpathclose%
\pgfusepath{stroke,fill}%
\end{pgfscope}%
\begin{pgfscope}%
\pgfpathrectangle{\pgfqpoint{0.648703in}{0.548769in}}{\pgfqpoint{5.201297in}{3.102590in}}%
\pgfusepath{clip}%
\pgfsetbuttcap%
\pgfsetroundjoin%
\definecolor{currentfill}{rgb}{1.000000,0.498039,0.054902}%
\pgfsetfillcolor{currentfill}%
\pgfsetlinewidth{1.003750pt}%
\definecolor{currentstroke}{rgb}{1.000000,0.498039,0.054902}%
\pgfsetstrokecolor{currentstroke}%
\pgfsetdash{}{0pt}%
\pgfpathmoveto{\pgfqpoint{4.041627in}{3.202258in}}%
\pgfpathcurveto{\pgfqpoint{4.052677in}{3.202258in}}{\pgfqpoint{4.063276in}{3.206648in}}{\pgfqpoint{4.071090in}{3.214462in}}%
\pgfpathcurveto{\pgfqpoint{4.078904in}{3.222275in}}{\pgfqpoint{4.083294in}{3.232874in}}{\pgfqpoint{4.083294in}{3.243924in}}%
\pgfpathcurveto{\pgfqpoint{4.083294in}{3.254974in}}{\pgfqpoint{4.078904in}{3.265573in}}{\pgfqpoint{4.071090in}{3.273387in}}%
\pgfpathcurveto{\pgfqpoint{4.063276in}{3.281201in}}{\pgfqpoint{4.052677in}{3.285591in}}{\pgfqpoint{4.041627in}{3.285591in}}%
\pgfpathcurveto{\pgfqpoint{4.030577in}{3.285591in}}{\pgfqpoint{4.019978in}{3.281201in}}{\pgfqpoint{4.012164in}{3.273387in}}%
\pgfpathcurveto{\pgfqpoint{4.004351in}{3.265573in}}{\pgfqpoint{3.999961in}{3.254974in}}{\pgfqpoint{3.999961in}{3.243924in}}%
\pgfpathcurveto{\pgfqpoint{3.999961in}{3.232874in}}{\pgfqpoint{4.004351in}{3.222275in}}{\pgfqpoint{4.012164in}{3.214462in}}%
\pgfpathcurveto{\pgfqpoint{4.019978in}{3.206648in}}{\pgfqpoint{4.030577in}{3.202258in}}{\pgfqpoint{4.041627in}{3.202258in}}%
\pgfpathclose%
\pgfusepath{stroke,fill}%
\end{pgfscope}%
\begin{pgfscope}%
\pgfpathrectangle{\pgfqpoint{0.648703in}{0.548769in}}{\pgfqpoint{5.201297in}{3.102590in}}%
\pgfusepath{clip}%
\pgfsetbuttcap%
\pgfsetroundjoin%
\definecolor{currentfill}{rgb}{0.839216,0.152941,0.156863}%
\pgfsetfillcolor{currentfill}%
\pgfsetlinewidth{1.003750pt}%
\definecolor{currentstroke}{rgb}{0.839216,0.152941,0.156863}%
\pgfsetstrokecolor{currentstroke}%
\pgfsetdash{}{0pt}%
\pgfpathmoveto{\pgfqpoint{4.302894in}{3.265688in}}%
\pgfpathcurveto{\pgfqpoint{4.313944in}{3.265688in}}{\pgfqpoint{4.324543in}{3.270078in}}{\pgfqpoint{4.332357in}{3.277892in}}%
\pgfpathcurveto{\pgfqpoint{4.340171in}{3.285706in}}{\pgfqpoint{4.344561in}{3.296305in}}{\pgfqpoint{4.344561in}{3.307355in}}%
\pgfpathcurveto{\pgfqpoint{4.344561in}{3.318405in}}{\pgfqpoint{4.340171in}{3.329004in}}{\pgfqpoint{4.332357in}{3.336817in}}%
\pgfpathcurveto{\pgfqpoint{4.324543in}{3.344631in}}{\pgfqpoint{4.313944in}{3.349021in}}{\pgfqpoint{4.302894in}{3.349021in}}%
\pgfpathcurveto{\pgfqpoint{4.291844in}{3.349021in}}{\pgfqpoint{4.281245in}{3.344631in}}{\pgfqpoint{4.273431in}{3.336817in}}%
\pgfpathcurveto{\pgfqpoint{4.265618in}{3.329004in}}{\pgfqpoint{4.261228in}{3.318405in}}{\pgfqpoint{4.261228in}{3.307355in}}%
\pgfpathcurveto{\pgfqpoint{4.261228in}{3.296305in}}{\pgfqpoint{4.265618in}{3.285706in}}{\pgfqpoint{4.273431in}{3.277892in}}%
\pgfpathcurveto{\pgfqpoint{4.281245in}{3.270078in}}{\pgfqpoint{4.291844in}{3.265688in}}{\pgfqpoint{4.302894in}{3.265688in}}%
\pgfpathclose%
\pgfusepath{stroke,fill}%
\end{pgfscope}%
\begin{pgfscope}%
\pgfpathrectangle{\pgfqpoint{0.648703in}{0.548769in}}{\pgfqpoint{5.201297in}{3.102590in}}%
\pgfusepath{clip}%
\pgfsetbuttcap%
\pgfsetroundjoin%
\definecolor{currentfill}{rgb}{0.121569,0.466667,0.705882}%
\pgfsetfillcolor{currentfill}%
\pgfsetlinewidth{1.003750pt}%
\definecolor{currentstroke}{rgb}{0.121569,0.466667,0.705882}%
\pgfsetstrokecolor{currentstroke}%
\pgfsetdash{}{0pt}%
\pgfpathmoveto{\pgfqpoint{4.408499in}{2.808990in}}%
\pgfpathcurveto{\pgfqpoint{4.419549in}{2.808990in}}{\pgfqpoint{4.430148in}{2.813380in}}{\pgfqpoint{4.437962in}{2.821193in}}%
\pgfpathcurveto{\pgfqpoint{4.445776in}{2.829007in}}{\pgfqpoint{4.450166in}{2.839606in}}{\pgfqpoint{4.450166in}{2.850656in}}%
\pgfpathcurveto{\pgfqpoint{4.450166in}{2.861706in}}{\pgfqpoint{4.445776in}{2.872305in}}{\pgfqpoint{4.437962in}{2.880119in}}%
\pgfpathcurveto{\pgfqpoint{4.430148in}{2.887933in}}{\pgfqpoint{4.419549in}{2.892323in}}{\pgfqpoint{4.408499in}{2.892323in}}%
\pgfpathcurveto{\pgfqpoint{4.397449in}{2.892323in}}{\pgfqpoint{4.386850in}{2.887933in}}{\pgfqpoint{4.379036in}{2.880119in}}%
\pgfpathcurveto{\pgfqpoint{4.371223in}{2.872305in}}{\pgfqpoint{4.366833in}{2.861706in}}{\pgfqpoint{4.366833in}{2.850656in}}%
\pgfpathcurveto{\pgfqpoint{4.366833in}{2.839606in}}{\pgfqpoint{4.371223in}{2.829007in}}{\pgfqpoint{4.379036in}{2.821193in}}%
\pgfpathcurveto{\pgfqpoint{4.386850in}{2.813380in}}{\pgfqpoint{4.397449in}{2.808990in}}{\pgfqpoint{4.408499in}{2.808990in}}%
\pgfpathclose%
\pgfusepath{stroke,fill}%
\end{pgfscope}%
\begin{pgfscope}%
\pgfpathrectangle{\pgfqpoint{0.648703in}{0.548769in}}{\pgfqpoint{5.201297in}{3.102590in}}%
\pgfusepath{clip}%
\pgfsetbuttcap%
\pgfsetroundjoin%
\definecolor{currentfill}{rgb}{1.000000,0.498039,0.054902}%
\pgfsetfillcolor{currentfill}%
\pgfsetlinewidth{1.003750pt}%
\definecolor{currentstroke}{rgb}{1.000000,0.498039,0.054902}%
\pgfsetstrokecolor{currentstroke}%
\pgfsetdash{}{0pt}%
\pgfpathmoveto{\pgfqpoint{3.870401in}{3.185343in}}%
\pgfpathcurveto{\pgfqpoint{3.881451in}{3.185343in}}{\pgfqpoint{3.892050in}{3.189733in}}{\pgfqpoint{3.899864in}{3.197547in}}%
\pgfpathcurveto{\pgfqpoint{3.907677in}{3.205360in}}{\pgfqpoint{3.912068in}{3.215959in}}{\pgfqpoint{3.912068in}{3.227010in}}%
\pgfpathcurveto{\pgfqpoint{3.912068in}{3.238060in}}{\pgfqpoint{3.907677in}{3.248659in}}{\pgfqpoint{3.899864in}{3.256472in}}%
\pgfpathcurveto{\pgfqpoint{3.892050in}{3.264286in}}{\pgfqpoint{3.881451in}{3.268676in}}{\pgfqpoint{3.870401in}{3.268676in}}%
\pgfpathcurveto{\pgfqpoint{3.859351in}{3.268676in}}{\pgfqpoint{3.848752in}{3.264286in}}{\pgfqpoint{3.840938in}{3.256472in}}%
\pgfpathcurveto{\pgfqpoint{3.833125in}{3.248659in}}{\pgfqpoint{3.828734in}{3.238060in}}{\pgfqpoint{3.828734in}{3.227010in}}%
\pgfpathcurveto{\pgfqpoint{3.828734in}{3.215959in}}{\pgfqpoint{3.833125in}{3.205360in}}{\pgfqpoint{3.840938in}{3.197547in}}%
\pgfpathcurveto{\pgfqpoint{3.848752in}{3.189733in}}{\pgfqpoint{3.859351in}{3.185343in}}{\pgfqpoint{3.870401in}{3.185343in}}%
\pgfpathclose%
\pgfusepath{stroke,fill}%
\end{pgfscope}%
\begin{pgfscope}%
\pgfpathrectangle{\pgfqpoint{0.648703in}{0.548769in}}{\pgfqpoint{5.201297in}{3.102590in}}%
\pgfusepath{clip}%
\pgfsetbuttcap%
\pgfsetroundjoin%
\definecolor{currentfill}{rgb}{1.000000,0.498039,0.054902}%
\pgfsetfillcolor{currentfill}%
\pgfsetlinewidth{1.003750pt}%
\definecolor{currentstroke}{rgb}{1.000000,0.498039,0.054902}%
\pgfsetstrokecolor{currentstroke}%
\pgfsetdash{}{0pt}%
\pgfpathmoveto{\pgfqpoint{4.590373in}{3.193800in}}%
\pgfpathcurveto{\pgfqpoint{4.601423in}{3.193800in}}{\pgfqpoint{4.612022in}{3.198191in}}{\pgfqpoint{4.619836in}{3.206004in}}%
\pgfpathcurveto{\pgfqpoint{4.627650in}{3.213818in}}{\pgfqpoint{4.632040in}{3.224417in}}{\pgfqpoint{4.632040in}{3.235467in}}%
\pgfpathcurveto{\pgfqpoint{4.632040in}{3.246517in}}{\pgfqpoint{4.627650in}{3.257116in}}{\pgfqpoint{4.619836in}{3.264930in}}%
\pgfpathcurveto{\pgfqpoint{4.612022in}{3.272743in}}{\pgfqpoint{4.601423in}{3.277134in}}{\pgfqpoint{4.590373in}{3.277134in}}%
\pgfpathcurveto{\pgfqpoint{4.579323in}{3.277134in}}{\pgfqpoint{4.568724in}{3.272743in}}{\pgfqpoint{4.560910in}{3.264930in}}%
\pgfpathcurveto{\pgfqpoint{4.553097in}{3.257116in}}{\pgfqpoint{4.548706in}{3.246517in}}{\pgfqpoint{4.548706in}{3.235467in}}%
\pgfpathcurveto{\pgfqpoint{4.548706in}{3.224417in}}{\pgfqpoint{4.553097in}{3.213818in}}{\pgfqpoint{4.560910in}{3.206004in}}%
\pgfpathcurveto{\pgfqpoint{4.568724in}{3.198191in}}{\pgfqpoint{4.579323in}{3.193800in}}{\pgfqpoint{4.590373in}{3.193800in}}%
\pgfpathclose%
\pgfusepath{stroke,fill}%
\end{pgfscope}%
\begin{pgfscope}%
\pgfpathrectangle{\pgfqpoint{0.648703in}{0.548769in}}{\pgfqpoint{5.201297in}{3.102590in}}%
\pgfusepath{clip}%
\pgfsetbuttcap%
\pgfsetroundjoin%
\definecolor{currentfill}{rgb}{1.000000,0.498039,0.054902}%
\pgfsetfillcolor{currentfill}%
\pgfsetlinewidth{1.003750pt}%
\definecolor{currentstroke}{rgb}{1.000000,0.498039,0.054902}%
\pgfsetstrokecolor{currentstroke}%
\pgfsetdash{}{0pt}%
\pgfpathmoveto{\pgfqpoint{4.221591in}{3.362948in}}%
\pgfpathcurveto{\pgfqpoint{4.232641in}{3.362948in}}{\pgfqpoint{4.243240in}{3.367338in}}{\pgfqpoint{4.251053in}{3.375152in}}%
\pgfpathcurveto{\pgfqpoint{4.258867in}{3.382965in}}{\pgfqpoint{4.263257in}{3.393564in}}{\pgfqpoint{4.263257in}{3.404615in}}%
\pgfpathcurveto{\pgfqpoint{4.263257in}{3.415665in}}{\pgfqpoint{4.258867in}{3.426264in}}{\pgfqpoint{4.251053in}{3.434077in}}%
\pgfpathcurveto{\pgfqpoint{4.243240in}{3.441891in}}{\pgfqpoint{4.232641in}{3.446281in}}{\pgfqpoint{4.221591in}{3.446281in}}%
\pgfpathcurveto{\pgfqpoint{4.210540in}{3.446281in}}{\pgfqpoint{4.199941in}{3.441891in}}{\pgfqpoint{4.192128in}{3.434077in}}%
\pgfpathcurveto{\pgfqpoint{4.184314in}{3.426264in}}{\pgfqpoint{4.179924in}{3.415665in}}{\pgfqpoint{4.179924in}{3.404615in}}%
\pgfpathcurveto{\pgfqpoint{4.179924in}{3.393564in}}{\pgfqpoint{4.184314in}{3.382965in}}{\pgfqpoint{4.192128in}{3.375152in}}%
\pgfpathcurveto{\pgfqpoint{4.199941in}{3.367338in}}{\pgfqpoint{4.210540in}{3.362948in}}{\pgfqpoint{4.221591in}{3.362948in}}%
\pgfpathclose%
\pgfusepath{stroke,fill}%
\end{pgfscope}%
\begin{pgfscope}%
\pgfpathrectangle{\pgfqpoint{0.648703in}{0.548769in}}{\pgfqpoint{5.201297in}{3.102590in}}%
\pgfusepath{clip}%
\pgfsetbuttcap%
\pgfsetroundjoin%
\definecolor{currentfill}{rgb}{1.000000,0.498039,0.054902}%
\pgfsetfillcolor{currentfill}%
\pgfsetlinewidth{1.003750pt}%
\definecolor{currentstroke}{rgb}{1.000000,0.498039,0.054902}%
\pgfsetstrokecolor{currentstroke}%
\pgfsetdash{}{0pt}%
\pgfpathmoveto{\pgfqpoint{2.733192in}{3.257231in}}%
\pgfpathcurveto{\pgfqpoint{2.744242in}{3.257231in}}{\pgfqpoint{2.754841in}{3.261621in}}{\pgfqpoint{2.762655in}{3.269435in}}%
\pgfpathcurveto{\pgfqpoint{2.770468in}{3.277248in}}{\pgfqpoint{2.774859in}{3.287847in}}{\pgfqpoint{2.774859in}{3.298897in}}%
\pgfpathcurveto{\pgfqpoint{2.774859in}{3.309947in}}{\pgfqpoint{2.770468in}{3.320546in}}{\pgfqpoint{2.762655in}{3.328360in}}%
\pgfpathcurveto{\pgfqpoint{2.754841in}{3.336174in}}{\pgfqpoint{2.744242in}{3.340564in}}{\pgfqpoint{2.733192in}{3.340564in}}%
\pgfpathcurveto{\pgfqpoint{2.722142in}{3.340564in}}{\pgfqpoint{2.711543in}{3.336174in}}{\pgfqpoint{2.703729in}{3.328360in}}%
\pgfpathcurveto{\pgfqpoint{2.695916in}{3.320546in}}{\pgfqpoint{2.691525in}{3.309947in}}{\pgfqpoint{2.691525in}{3.298897in}}%
\pgfpathcurveto{\pgfqpoint{2.691525in}{3.287847in}}{\pgfqpoint{2.695916in}{3.277248in}}{\pgfqpoint{2.703729in}{3.269435in}}%
\pgfpathcurveto{\pgfqpoint{2.711543in}{3.261621in}}{\pgfqpoint{2.722142in}{3.257231in}}{\pgfqpoint{2.733192in}{3.257231in}}%
\pgfpathclose%
\pgfusepath{stroke,fill}%
\end{pgfscope}%
\begin{pgfscope}%
\pgfpathrectangle{\pgfqpoint{0.648703in}{0.548769in}}{\pgfqpoint{5.201297in}{3.102590in}}%
\pgfusepath{clip}%
\pgfsetbuttcap%
\pgfsetroundjoin%
\definecolor{currentfill}{rgb}{1.000000,0.498039,0.054902}%
\pgfsetfillcolor{currentfill}%
\pgfsetlinewidth{1.003750pt}%
\definecolor{currentstroke}{rgb}{1.000000,0.498039,0.054902}%
\pgfsetstrokecolor{currentstroke}%
\pgfsetdash{}{0pt}%
\pgfpathmoveto{\pgfqpoint{3.067417in}{3.189572in}}%
\pgfpathcurveto{\pgfqpoint{3.078467in}{3.189572in}}{\pgfqpoint{3.089066in}{3.193962in}}{\pgfqpoint{3.096880in}{3.201775in}}%
\pgfpathcurveto{\pgfqpoint{3.104693in}{3.209589in}}{\pgfqpoint{3.109084in}{3.220188in}}{\pgfqpoint{3.109084in}{3.231238in}}%
\pgfpathcurveto{\pgfqpoint{3.109084in}{3.242288in}}{\pgfqpoint{3.104693in}{3.252887in}}{\pgfqpoint{3.096880in}{3.260701in}}%
\pgfpathcurveto{\pgfqpoint{3.089066in}{3.268515in}}{\pgfqpoint{3.078467in}{3.272905in}}{\pgfqpoint{3.067417in}{3.272905in}}%
\pgfpathcurveto{\pgfqpoint{3.056367in}{3.272905in}}{\pgfqpoint{3.045768in}{3.268515in}}{\pgfqpoint{3.037954in}{3.260701in}}%
\pgfpathcurveto{\pgfqpoint{3.030140in}{3.252887in}}{\pgfqpoint{3.025750in}{3.242288in}}{\pgfqpoint{3.025750in}{3.231238in}}%
\pgfpathcurveto{\pgfqpoint{3.025750in}{3.220188in}}{\pgfqpoint{3.030140in}{3.209589in}}{\pgfqpoint{3.037954in}{3.201775in}}%
\pgfpathcurveto{\pgfqpoint{3.045768in}{3.193962in}}{\pgfqpoint{3.056367in}{3.189572in}}{\pgfqpoint{3.067417in}{3.189572in}}%
\pgfpathclose%
\pgfusepath{stroke,fill}%
\end{pgfscope}%
\begin{pgfscope}%
\pgfpathrectangle{\pgfqpoint{0.648703in}{0.548769in}}{\pgfqpoint{5.201297in}{3.102590in}}%
\pgfusepath{clip}%
\pgfsetbuttcap%
\pgfsetroundjoin%
\definecolor{currentfill}{rgb}{1.000000,0.498039,0.054902}%
\pgfsetfillcolor{currentfill}%
\pgfsetlinewidth{1.003750pt}%
\definecolor{currentstroke}{rgb}{1.000000,0.498039,0.054902}%
\pgfsetstrokecolor{currentstroke}%
\pgfsetdash{}{0pt}%
\pgfpathmoveto{\pgfqpoint{4.235991in}{3.189572in}}%
\pgfpathcurveto{\pgfqpoint{4.247042in}{3.189572in}}{\pgfqpoint{4.257641in}{3.193962in}}{\pgfqpoint{4.265454in}{3.201775in}}%
\pgfpathcurveto{\pgfqpoint{4.273268in}{3.209589in}}{\pgfqpoint{4.277658in}{3.220188in}}{\pgfqpoint{4.277658in}{3.231238in}}%
\pgfpathcurveto{\pgfqpoint{4.277658in}{3.242288in}}{\pgfqpoint{4.273268in}{3.252887in}}{\pgfqpoint{4.265454in}{3.260701in}}%
\pgfpathcurveto{\pgfqpoint{4.257641in}{3.268515in}}{\pgfqpoint{4.247042in}{3.272905in}}{\pgfqpoint{4.235991in}{3.272905in}}%
\pgfpathcurveto{\pgfqpoint{4.224941in}{3.272905in}}{\pgfqpoint{4.214342in}{3.268515in}}{\pgfqpoint{4.206529in}{3.260701in}}%
\pgfpathcurveto{\pgfqpoint{4.198715in}{3.252887in}}{\pgfqpoint{4.194325in}{3.242288in}}{\pgfqpoint{4.194325in}{3.231238in}}%
\pgfpathcurveto{\pgfqpoint{4.194325in}{3.220188in}}{\pgfqpoint{4.198715in}{3.209589in}}{\pgfqpoint{4.206529in}{3.201775in}}%
\pgfpathcurveto{\pgfqpoint{4.214342in}{3.193962in}}{\pgfqpoint{4.224941in}{3.189572in}}{\pgfqpoint{4.235991in}{3.189572in}}%
\pgfpathclose%
\pgfusepath{stroke,fill}%
\end{pgfscope}%
\begin{pgfscope}%
\pgfpathrectangle{\pgfqpoint{0.648703in}{0.548769in}}{\pgfqpoint{5.201297in}{3.102590in}}%
\pgfusepath{clip}%
\pgfsetbuttcap%
\pgfsetroundjoin%
\definecolor{currentfill}{rgb}{0.121569,0.466667,0.705882}%
\pgfsetfillcolor{currentfill}%
\pgfsetlinewidth{1.003750pt}%
\definecolor{currentstroke}{rgb}{0.121569,0.466667,0.705882}%
\pgfsetstrokecolor{currentstroke}%
\pgfsetdash{}{0pt}%
\pgfpathmoveto{\pgfqpoint{3.916490in}{3.181114in}}%
\pgfpathcurveto{\pgfqpoint{3.927540in}{3.181114in}}{\pgfqpoint{3.938139in}{3.185504in}}{\pgfqpoint{3.945953in}{3.193318in}}%
\pgfpathcurveto{\pgfqpoint{3.953766in}{3.201132in}}{\pgfqpoint{3.958157in}{3.211731in}}{\pgfqpoint{3.958157in}{3.222781in}}%
\pgfpathcurveto{\pgfqpoint{3.958157in}{3.233831in}}{\pgfqpoint{3.953766in}{3.244430in}}{\pgfqpoint{3.945953in}{3.252244in}}%
\pgfpathcurveto{\pgfqpoint{3.938139in}{3.260057in}}{\pgfqpoint{3.927540in}{3.264448in}}{\pgfqpoint{3.916490in}{3.264448in}}%
\pgfpathcurveto{\pgfqpoint{3.905440in}{3.264448in}}{\pgfqpoint{3.894841in}{3.260057in}}{\pgfqpoint{3.887027in}{3.252244in}}%
\pgfpathcurveto{\pgfqpoint{3.879214in}{3.244430in}}{\pgfqpoint{3.874823in}{3.233831in}}{\pgfqpoint{3.874823in}{3.222781in}}%
\pgfpathcurveto{\pgfqpoint{3.874823in}{3.211731in}}{\pgfqpoint{3.879214in}{3.201132in}}{\pgfqpoint{3.887027in}{3.193318in}}%
\pgfpathcurveto{\pgfqpoint{3.894841in}{3.185504in}}{\pgfqpoint{3.905440in}{3.181114in}}{\pgfqpoint{3.916490in}{3.181114in}}%
\pgfpathclose%
\pgfusepath{stroke,fill}%
\end{pgfscope}%
\begin{pgfscope}%
\pgfpathrectangle{\pgfqpoint{0.648703in}{0.548769in}}{\pgfqpoint{5.201297in}{3.102590in}}%
\pgfusepath{clip}%
\pgfsetbuttcap%
\pgfsetroundjoin%
\definecolor{currentfill}{rgb}{1.000000,0.498039,0.054902}%
\pgfsetfillcolor{currentfill}%
\pgfsetlinewidth{1.003750pt}%
\definecolor{currentstroke}{rgb}{1.000000,0.498039,0.054902}%
\pgfsetstrokecolor{currentstroke}%
\pgfsetdash{}{0pt}%
\pgfpathmoveto{\pgfqpoint{4.908730in}{3.185343in}}%
\pgfpathcurveto{\pgfqpoint{4.919780in}{3.185343in}}{\pgfqpoint{4.930379in}{3.189733in}}{\pgfqpoint{4.938193in}{3.197547in}}%
\pgfpathcurveto{\pgfqpoint{4.946006in}{3.205360in}}{\pgfqpoint{4.950396in}{3.215959in}}{\pgfqpoint{4.950396in}{3.227010in}}%
\pgfpathcurveto{\pgfqpoint{4.950396in}{3.238060in}}{\pgfqpoint{4.946006in}{3.248659in}}{\pgfqpoint{4.938193in}{3.256472in}}%
\pgfpathcurveto{\pgfqpoint{4.930379in}{3.264286in}}{\pgfqpoint{4.919780in}{3.268676in}}{\pgfqpoint{4.908730in}{3.268676in}}%
\pgfpathcurveto{\pgfqpoint{4.897680in}{3.268676in}}{\pgfqpoint{4.887081in}{3.264286in}}{\pgfqpoint{4.879267in}{3.256472in}}%
\pgfpathcurveto{\pgfqpoint{4.871453in}{3.248659in}}{\pgfqpoint{4.867063in}{3.238060in}}{\pgfqpoint{4.867063in}{3.227010in}}%
\pgfpathcurveto{\pgfqpoint{4.867063in}{3.215959in}}{\pgfqpoint{4.871453in}{3.205360in}}{\pgfqpoint{4.879267in}{3.197547in}}%
\pgfpathcurveto{\pgfqpoint{4.887081in}{3.189733in}}{\pgfqpoint{4.897680in}{3.185343in}}{\pgfqpoint{4.908730in}{3.185343in}}%
\pgfpathclose%
\pgfusepath{stroke,fill}%
\end{pgfscope}%
\begin{pgfscope}%
\pgfpathrectangle{\pgfqpoint{0.648703in}{0.548769in}}{\pgfqpoint{5.201297in}{3.102590in}}%
\pgfusepath{clip}%
\pgfsetbuttcap%
\pgfsetroundjoin%
\definecolor{currentfill}{rgb}{0.121569,0.466667,0.705882}%
\pgfsetfillcolor{currentfill}%
\pgfsetlinewidth{1.003750pt}%
\definecolor{currentstroke}{rgb}{0.121569,0.466667,0.705882}%
\pgfsetstrokecolor{currentstroke}%
\pgfsetdash{}{0pt}%
\pgfpathmoveto{\pgfqpoint{0.947025in}{0.681958in}}%
\pgfpathcurveto{\pgfqpoint{0.958076in}{0.681958in}}{\pgfqpoint{0.968675in}{0.686349in}}{\pgfqpoint{0.976488in}{0.694162in}}%
\pgfpathcurveto{\pgfqpoint{0.984302in}{0.701976in}}{\pgfqpoint{0.988692in}{0.712575in}}{\pgfqpoint{0.988692in}{0.723625in}}%
\pgfpathcurveto{\pgfqpoint{0.988692in}{0.734675in}}{\pgfqpoint{0.984302in}{0.745274in}}{\pgfqpoint{0.976488in}{0.753088in}}%
\pgfpathcurveto{\pgfqpoint{0.968675in}{0.760902in}}{\pgfqpoint{0.958076in}{0.765292in}}{\pgfqpoint{0.947025in}{0.765292in}}%
\pgfpathcurveto{\pgfqpoint{0.935975in}{0.765292in}}{\pgfqpoint{0.925376in}{0.760902in}}{\pgfqpoint{0.917563in}{0.753088in}}%
\pgfpathcurveto{\pgfqpoint{0.909749in}{0.745274in}}{\pgfqpoint{0.905359in}{0.734675in}}{\pgfqpoint{0.905359in}{0.723625in}}%
\pgfpathcurveto{\pgfqpoint{0.905359in}{0.712575in}}{\pgfqpoint{0.909749in}{0.701976in}}{\pgfqpoint{0.917563in}{0.694162in}}%
\pgfpathcurveto{\pgfqpoint{0.925376in}{0.686349in}}{\pgfqpoint{0.935975in}{0.681958in}}{\pgfqpoint{0.947025in}{0.681958in}}%
\pgfpathclose%
\pgfusepath{stroke,fill}%
\end{pgfscope}%
\begin{pgfscope}%
\pgfpathrectangle{\pgfqpoint{0.648703in}{0.548769in}}{\pgfqpoint{5.201297in}{3.102590in}}%
\pgfusepath{clip}%
\pgfsetbuttcap%
\pgfsetroundjoin%
\definecolor{currentfill}{rgb}{0.839216,0.152941,0.156863}%
\pgfsetfillcolor{currentfill}%
\pgfsetlinewidth{1.003750pt}%
\definecolor{currentstroke}{rgb}{0.839216,0.152941,0.156863}%
\pgfsetstrokecolor{currentstroke}%
\pgfsetdash{}{0pt}%
\pgfpathmoveto{\pgfqpoint{3.056841in}{3.214944in}}%
\pgfpathcurveto{\pgfqpoint{3.067891in}{3.214944in}}{\pgfqpoint{3.078490in}{3.219334in}}{\pgfqpoint{3.086304in}{3.227148in}}%
\pgfpathcurveto{\pgfqpoint{3.094117in}{3.234961in}}{\pgfqpoint{3.098508in}{3.245560in}}{\pgfqpoint{3.098508in}{3.256610in}}%
\pgfpathcurveto{\pgfqpoint{3.098508in}{3.267661in}}{\pgfqpoint{3.094117in}{3.278260in}}{\pgfqpoint{3.086304in}{3.286073in}}%
\pgfpathcurveto{\pgfqpoint{3.078490in}{3.293887in}}{\pgfqpoint{3.067891in}{3.298277in}}{\pgfqpoint{3.056841in}{3.298277in}}%
\pgfpathcurveto{\pgfqpoint{3.045791in}{3.298277in}}{\pgfqpoint{3.035192in}{3.293887in}}{\pgfqpoint{3.027378in}{3.286073in}}%
\pgfpathcurveto{\pgfqpoint{3.019564in}{3.278260in}}{\pgfqpoint{3.015174in}{3.267661in}}{\pgfqpoint{3.015174in}{3.256610in}}%
\pgfpathcurveto{\pgfqpoint{3.015174in}{3.245560in}}{\pgfqpoint{3.019564in}{3.234961in}}{\pgfqpoint{3.027378in}{3.227148in}}%
\pgfpathcurveto{\pgfqpoint{3.035192in}{3.219334in}}{\pgfqpoint{3.045791in}{3.214944in}}{\pgfqpoint{3.056841in}{3.214944in}}%
\pgfpathclose%
\pgfusepath{stroke,fill}%
\end{pgfscope}%
\begin{pgfscope}%
\pgfpathrectangle{\pgfqpoint{0.648703in}{0.548769in}}{\pgfqpoint{5.201297in}{3.102590in}}%
\pgfusepath{clip}%
\pgfsetbuttcap%
\pgfsetroundjoin%
\definecolor{currentfill}{rgb}{1.000000,0.498039,0.054902}%
\pgfsetfillcolor{currentfill}%
\pgfsetlinewidth{1.003750pt}%
\definecolor{currentstroke}{rgb}{1.000000,0.498039,0.054902}%
\pgfsetstrokecolor{currentstroke}%
\pgfsetdash{}{0pt}%
\pgfpathmoveto{\pgfqpoint{4.989029in}{3.185343in}}%
\pgfpathcurveto{\pgfqpoint{5.000080in}{3.185343in}}{\pgfqpoint{5.010679in}{3.189733in}}{\pgfqpoint{5.018492in}{3.197547in}}%
\pgfpathcurveto{\pgfqpoint{5.026306in}{3.205360in}}{\pgfqpoint{5.030696in}{3.215959in}}{\pgfqpoint{5.030696in}{3.227010in}}%
\pgfpathcurveto{\pgfqpoint{5.030696in}{3.238060in}}{\pgfqpoint{5.026306in}{3.248659in}}{\pgfqpoint{5.018492in}{3.256472in}}%
\pgfpathcurveto{\pgfqpoint{5.010679in}{3.264286in}}{\pgfqpoint{5.000080in}{3.268676in}}{\pgfqpoint{4.989029in}{3.268676in}}%
\pgfpathcurveto{\pgfqpoint{4.977979in}{3.268676in}}{\pgfqpoint{4.967380in}{3.264286in}}{\pgfqpoint{4.959567in}{3.256472in}}%
\pgfpathcurveto{\pgfqpoint{4.951753in}{3.248659in}}{\pgfqpoint{4.947363in}{3.238060in}}{\pgfqpoint{4.947363in}{3.227010in}}%
\pgfpathcurveto{\pgfqpoint{4.947363in}{3.215959in}}{\pgfqpoint{4.951753in}{3.205360in}}{\pgfqpoint{4.959567in}{3.197547in}}%
\pgfpathcurveto{\pgfqpoint{4.967380in}{3.189733in}}{\pgfqpoint{4.977979in}{3.185343in}}{\pgfqpoint{4.989029in}{3.185343in}}%
\pgfpathclose%
\pgfusepath{stroke,fill}%
\end{pgfscope}%
\begin{pgfscope}%
\pgfpathrectangle{\pgfqpoint{0.648703in}{0.548769in}}{\pgfqpoint{5.201297in}{3.102590in}}%
\pgfusepath{clip}%
\pgfsetbuttcap%
\pgfsetroundjoin%
\definecolor{currentfill}{rgb}{0.121569,0.466667,0.705882}%
\pgfsetfillcolor{currentfill}%
\pgfsetlinewidth{1.003750pt}%
\definecolor{currentstroke}{rgb}{0.121569,0.466667,0.705882}%
\pgfsetstrokecolor{currentstroke}%
\pgfsetdash{}{0pt}%
\pgfpathmoveto{\pgfqpoint{1.765095in}{1.138657in}}%
\pgfpathcurveto{\pgfqpoint{1.776145in}{1.138657in}}{\pgfqpoint{1.786744in}{1.143047in}}{\pgfqpoint{1.794558in}{1.150861in}}%
\pgfpathcurveto{\pgfqpoint{1.802371in}{1.158674in}}{\pgfqpoint{1.806762in}{1.169274in}}{\pgfqpoint{1.806762in}{1.180324in}}%
\pgfpathcurveto{\pgfqpoint{1.806762in}{1.191374in}}{\pgfqpoint{1.802371in}{1.201973in}}{\pgfqpoint{1.794558in}{1.209786in}}%
\pgfpathcurveto{\pgfqpoint{1.786744in}{1.217600in}}{\pgfqpoint{1.776145in}{1.221990in}}{\pgfqpoint{1.765095in}{1.221990in}}%
\pgfpathcurveto{\pgfqpoint{1.754045in}{1.221990in}}{\pgfqpoint{1.743446in}{1.217600in}}{\pgfqpoint{1.735632in}{1.209786in}}%
\pgfpathcurveto{\pgfqpoint{1.727818in}{1.201973in}}{\pgfqpoint{1.723428in}{1.191374in}}{\pgfqpoint{1.723428in}{1.180324in}}%
\pgfpathcurveto{\pgfqpoint{1.723428in}{1.169274in}}{\pgfqpoint{1.727818in}{1.158674in}}{\pgfqpoint{1.735632in}{1.150861in}}%
\pgfpathcurveto{\pgfqpoint{1.743446in}{1.143047in}}{\pgfqpoint{1.754045in}{1.138657in}}{\pgfqpoint{1.765095in}{1.138657in}}%
\pgfpathclose%
\pgfusepath{stroke,fill}%
\end{pgfscope}%
\begin{pgfscope}%
\pgfpathrectangle{\pgfqpoint{0.648703in}{0.548769in}}{\pgfqpoint{5.201297in}{3.102590in}}%
\pgfusepath{clip}%
\pgfsetbuttcap%
\pgfsetroundjoin%
\definecolor{currentfill}{rgb}{0.121569,0.466667,0.705882}%
\pgfsetfillcolor{currentfill}%
\pgfsetlinewidth{1.003750pt}%
\definecolor{currentstroke}{rgb}{0.121569,0.466667,0.705882}%
\pgfsetstrokecolor{currentstroke}%
\pgfsetdash{}{0pt}%
\pgfpathmoveto{\pgfqpoint{2.253792in}{1.087913in}}%
\pgfpathcurveto{\pgfqpoint{2.264842in}{1.087913in}}{\pgfqpoint{2.275441in}{1.092303in}}{\pgfqpoint{2.283255in}{1.100117in}}%
\pgfpathcurveto{\pgfqpoint{2.291068in}{1.107930in}}{\pgfqpoint{2.295459in}{1.118529in}}{\pgfqpoint{2.295459in}{1.129579in}}%
\pgfpathcurveto{\pgfqpoint{2.295459in}{1.140629in}}{\pgfqpoint{2.291068in}{1.151229in}}{\pgfqpoint{2.283255in}{1.159042in}}%
\pgfpathcurveto{\pgfqpoint{2.275441in}{1.166856in}}{\pgfqpoint{2.264842in}{1.171246in}}{\pgfqpoint{2.253792in}{1.171246in}}%
\pgfpathcurveto{\pgfqpoint{2.242742in}{1.171246in}}{\pgfqpoint{2.232143in}{1.166856in}}{\pgfqpoint{2.224329in}{1.159042in}}%
\pgfpathcurveto{\pgfqpoint{2.216516in}{1.151229in}}{\pgfqpoint{2.212125in}{1.140629in}}{\pgfqpoint{2.212125in}{1.129579in}}%
\pgfpathcurveto{\pgfqpoint{2.212125in}{1.118529in}}{\pgfqpoint{2.216516in}{1.107930in}}{\pgfqpoint{2.224329in}{1.100117in}}%
\pgfpathcurveto{\pgfqpoint{2.232143in}{1.092303in}}{\pgfqpoint{2.242742in}{1.087913in}}{\pgfqpoint{2.253792in}{1.087913in}}%
\pgfpathclose%
\pgfusepath{stroke,fill}%
\end{pgfscope}%
\begin{pgfscope}%
\pgfpathrectangle{\pgfqpoint{0.648703in}{0.548769in}}{\pgfqpoint{5.201297in}{3.102590in}}%
\pgfusepath{clip}%
\pgfsetbuttcap%
\pgfsetroundjoin%
\definecolor{currentfill}{rgb}{1.000000,0.498039,0.054902}%
\pgfsetfillcolor{currentfill}%
\pgfsetlinewidth{1.003750pt}%
\definecolor{currentstroke}{rgb}{1.000000,0.498039,0.054902}%
\pgfsetstrokecolor{currentstroke}%
\pgfsetdash{}{0pt}%
\pgfpathmoveto{\pgfqpoint{2.798986in}{3.189572in}}%
\pgfpathcurveto{\pgfqpoint{2.810036in}{3.189572in}}{\pgfqpoint{2.820635in}{3.193962in}}{\pgfqpoint{2.828449in}{3.201775in}}%
\pgfpathcurveto{\pgfqpoint{2.836262in}{3.209589in}}{\pgfqpoint{2.840652in}{3.220188in}}{\pgfqpoint{2.840652in}{3.231238in}}%
\pgfpathcurveto{\pgfqpoint{2.840652in}{3.242288in}}{\pgfqpoint{2.836262in}{3.252887in}}{\pgfqpoint{2.828449in}{3.260701in}}%
\pgfpathcurveto{\pgfqpoint{2.820635in}{3.268515in}}{\pgfqpoint{2.810036in}{3.272905in}}{\pgfqpoint{2.798986in}{3.272905in}}%
\pgfpathcurveto{\pgfqpoint{2.787936in}{3.272905in}}{\pgfqpoint{2.777337in}{3.268515in}}{\pgfqpoint{2.769523in}{3.260701in}}%
\pgfpathcurveto{\pgfqpoint{2.761709in}{3.252887in}}{\pgfqpoint{2.757319in}{3.242288in}}{\pgfqpoint{2.757319in}{3.231238in}}%
\pgfpathcurveto{\pgfqpoint{2.757319in}{3.220188in}}{\pgfqpoint{2.761709in}{3.209589in}}{\pgfqpoint{2.769523in}{3.201775in}}%
\pgfpathcurveto{\pgfqpoint{2.777337in}{3.193962in}}{\pgfqpoint{2.787936in}{3.189572in}}{\pgfqpoint{2.798986in}{3.189572in}}%
\pgfpathclose%
\pgfusepath{stroke,fill}%
\end{pgfscope}%
\begin{pgfscope}%
\pgfpathrectangle{\pgfqpoint{0.648703in}{0.548769in}}{\pgfqpoint{5.201297in}{3.102590in}}%
\pgfusepath{clip}%
\pgfsetbuttcap%
\pgfsetroundjoin%
\definecolor{currentfill}{rgb}{0.121569,0.466667,0.705882}%
\pgfsetfillcolor{currentfill}%
\pgfsetlinewidth{1.003750pt}%
\definecolor{currentstroke}{rgb}{0.121569,0.466667,0.705882}%
\pgfsetstrokecolor{currentstroke}%
\pgfsetdash{}{0pt}%
\pgfpathmoveto{\pgfqpoint{1.812522in}{1.024482in}}%
\pgfpathcurveto{\pgfqpoint{1.823572in}{1.024482in}}{\pgfqpoint{1.834171in}{1.028873in}}{\pgfqpoint{1.841984in}{1.036686in}}%
\pgfpathcurveto{\pgfqpoint{1.849798in}{1.044500in}}{\pgfqpoint{1.854188in}{1.055099in}}{\pgfqpoint{1.854188in}{1.066149in}}%
\pgfpathcurveto{\pgfqpoint{1.854188in}{1.077199in}}{\pgfqpoint{1.849798in}{1.087798in}}{\pgfqpoint{1.841984in}{1.095612in}}%
\pgfpathcurveto{\pgfqpoint{1.834171in}{1.103425in}}{\pgfqpoint{1.823572in}{1.107816in}}{\pgfqpoint{1.812522in}{1.107816in}}%
\pgfpathcurveto{\pgfqpoint{1.801471in}{1.107816in}}{\pgfqpoint{1.790872in}{1.103425in}}{\pgfqpoint{1.783059in}{1.095612in}}%
\pgfpathcurveto{\pgfqpoint{1.775245in}{1.087798in}}{\pgfqpoint{1.770855in}{1.077199in}}{\pgfqpoint{1.770855in}{1.066149in}}%
\pgfpathcurveto{\pgfqpoint{1.770855in}{1.055099in}}{\pgfqpoint{1.775245in}{1.044500in}}{\pgfqpoint{1.783059in}{1.036686in}}%
\pgfpathcurveto{\pgfqpoint{1.790872in}{1.028873in}}{\pgfqpoint{1.801471in}{1.024482in}}{\pgfqpoint{1.812522in}{1.024482in}}%
\pgfpathclose%
\pgfusepath{stroke,fill}%
\end{pgfscope}%
\begin{pgfscope}%
\pgfpathrectangle{\pgfqpoint{0.648703in}{0.548769in}}{\pgfqpoint{5.201297in}{3.102590in}}%
\pgfusepath{clip}%
\pgfsetbuttcap%
\pgfsetroundjoin%
\definecolor{currentfill}{rgb}{0.121569,0.466667,0.705882}%
\pgfsetfillcolor{currentfill}%
\pgfsetlinewidth{1.003750pt}%
\definecolor{currentstroke}{rgb}{0.121569,0.466667,0.705882}%
\pgfsetstrokecolor{currentstroke}%
\pgfsetdash{}{0pt}%
\pgfpathmoveto{\pgfqpoint{1.392941in}{0.813048in}}%
\pgfpathcurveto{\pgfqpoint{1.403991in}{0.813048in}}{\pgfqpoint{1.414590in}{0.817438in}}{\pgfqpoint{1.422404in}{0.825252in}}%
\pgfpathcurveto{\pgfqpoint{1.430217in}{0.833065in}}{\pgfqpoint{1.434608in}{0.843664in}}{\pgfqpoint{1.434608in}{0.854715in}}%
\pgfpathcurveto{\pgfqpoint{1.434608in}{0.865765in}}{\pgfqpoint{1.430217in}{0.876364in}}{\pgfqpoint{1.422404in}{0.884177in}}%
\pgfpathcurveto{\pgfqpoint{1.414590in}{0.891991in}}{\pgfqpoint{1.403991in}{0.896381in}}{\pgfqpoint{1.392941in}{0.896381in}}%
\pgfpathcurveto{\pgfqpoint{1.381891in}{0.896381in}}{\pgfqpoint{1.371292in}{0.891991in}}{\pgfqpoint{1.363478in}{0.884177in}}%
\pgfpathcurveto{\pgfqpoint{1.355665in}{0.876364in}}{\pgfqpoint{1.351274in}{0.865765in}}{\pgfqpoint{1.351274in}{0.854715in}}%
\pgfpathcurveto{\pgfqpoint{1.351274in}{0.843664in}}{\pgfqpoint{1.355665in}{0.833065in}}{\pgfqpoint{1.363478in}{0.825252in}}%
\pgfpathcurveto{\pgfqpoint{1.371292in}{0.817438in}}{\pgfqpoint{1.381891in}{0.813048in}}{\pgfqpoint{1.392941in}{0.813048in}}%
\pgfpathclose%
\pgfusepath{stroke,fill}%
\end{pgfscope}%
\begin{pgfscope}%
\pgfpathrectangle{\pgfqpoint{0.648703in}{0.548769in}}{\pgfqpoint{5.201297in}{3.102590in}}%
\pgfusepath{clip}%
\pgfsetbuttcap%
\pgfsetroundjoin%
\definecolor{currentfill}{rgb}{0.839216,0.152941,0.156863}%
\pgfsetfillcolor{currentfill}%
\pgfsetlinewidth{1.003750pt}%
\definecolor{currentstroke}{rgb}{0.839216,0.152941,0.156863}%
\pgfsetstrokecolor{currentstroke}%
\pgfsetdash{}{0pt}%
\pgfpathmoveto{\pgfqpoint{4.991468in}{3.202258in}}%
\pgfpathcurveto{\pgfqpoint{5.002518in}{3.202258in}}{\pgfqpoint{5.013117in}{3.206648in}}{\pgfqpoint{5.020931in}{3.214462in}}%
\pgfpathcurveto{\pgfqpoint{5.028745in}{3.222275in}}{\pgfqpoint{5.033135in}{3.232874in}}{\pgfqpoint{5.033135in}{3.243924in}}%
\pgfpathcurveto{\pgfqpoint{5.033135in}{3.254974in}}{\pgfqpoint{5.028745in}{3.265573in}}{\pgfqpoint{5.020931in}{3.273387in}}%
\pgfpathcurveto{\pgfqpoint{5.013117in}{3.281201in}}{\pgfqpoint{5.002518in}{3.285591in}}{\pgfqpoint{4.991468in}{3.285591in}}%
\pgfpathcurveto{\pgfqpoint{4.980418in}{3.285591in}}{\pgfqpoint{4.969819in}{3.281201in}}{\pgfqpoint{4.962006in}{3.273387in}}%
\pgfpathcurveto{\pgfqpoint{4.954192in}{3.265573in}}{\pgfqpoint{4.949802in}{3.254974in}}{\pgfqpoint{4.949802in}{3.243924in}}%
\pgfpathcurveto{\pgfqpoint{4.949802in}{3.232874in}}{\pgfqpoint{4.954192in}{3.222275in}}{\pgfqpoint{4.962006in}{3.214462in}}%
\pgfpathcurveto{\pgfqpoint{4.969819in}{3.206648in}}{\pgfqpoint{4.980418in}{3.202258in}}{\pgfqpoint{4.991468in}{3.202258in}}%
\pgfpathclose%
\pgfusepath{stroke,fill}%
\end{pgfscope}%
\begin{pgfscope}%
\pgfsetbuttcap%
\pgfsetroundjoin%
\definecolor{currentfill}{rgb}{0.000000,0.000000,0.000000}%
\pgfsetfillcolor{currentfill}%
\pgfsetlinewidth{0.803000pt}%
\definecolor{currentstroke}{rgb}{0.000000,0.000000,0.000000}%
\pgfsetstrokecolor{currentstroke}%
\pgfsetdash{}{0pt}%
\pgfsys@defobject{currentmarker}{\pgfqpoint{0.000000in}{-0.048611in}}{\pgfqpoint{0.000000in}{0.000000in}}{%
\pgfpathmoveto{\pgfqpoint{0.000000in}{0.000000in}}%
\pgfpathlineto{\pgfqpoint{0.000000in}{-0.048611in}}%
\pgfusepath{stroke,fill}%
}%
\begin{pgfscope}%
\pgfsys@transformshift{0.885126in}{0.548769in}%
\pgfsys@useobject{currentmarker}{}%
\end{pgfscope}%
\end{pgfscope}%
\begin{pgfscope}%
\definecolor{textcolor}{rgb}{0.000000,0.000000,0.000000}%
\pgfsetstrokecolor{textcolor}%
\pgfsetfillcolor{textcolor}%
\pgftext[x=0.885126in,y=0.451547in,,top]{\color{textcolor}\sffamily\fontsize{10.000000}{12.000000}\selectfont \(\displaystyle {0.0}\)}%
\end{pgfscope}%
\begin{pgfscope}%
\pgfsetbuttcap%
\pgfsetroundjoin%
\definecolor{currentfill}{rgb}{0.000000,0.000000,0.000000}%
\pgfsetfillcolor{currentfill}%
\pgfsetlinewidth{0.803000pt}%
\definecolor{currentstroke}{rgb}{0.000000,0.000000,0.000000}%
\pgfsetstrokecolor{currentstroke}%
\pgfsetdash{}{0pt}%
\pgfsys@defobject{currentmarker}{\pgfqpoint{0.000000in}{-0.048611in}}{\pgfqpoint{0.000000in}{0.000000in}}{%
\pgfpathmoveto{\pgfqpoint{0.000000in}{0.000000in}}%
\pgfpathlineto{\pgfqpoint{0.000000in}{-0.048611in}}%
\pgfusepath{stroke,fill}%
}%
\begin{pgfscope}%
\pgfsys@transformshift{1.359861in}{0.548769in}%
\pgfsys@useobject{currentmarker}{}%
\end{pgfscope}%
\end{pgfscope}%
\begin{pgfscope}%
\definecolor{textcolor}{rgb}{0.000000,0.000000,0.000000}%
\pgfsetstrokecolor{textcolor}%
\pgfsetfillcolor{textcolor}%
\pgftext[x=1.359861in,y=0.451547in,,top]{\color{textcolor}\sffamily\fontsize{10.000000}{12.000000}\selectfont \(\displaystyle {0.1}\)}%
\end{pgfscope}%
\begin{pgfscope}%
\pgfsetbuttcap%
\pgfsetroundjoin%
\definecolor{currentfill}{rgb}{0.000000,0.000000,0.000000}%
\pgfsetfillcolor{currentfill}%
\pgfsetlinewidth{0.803000pt}%
\definecolor{currentstroke}{rgb}{0.000000,0.000000,0.000000}%
\pgfsetstrokecolor{currentstroke}%
\pgfsetdash{}{0pt}%
\pgfsys@defobject{currentmarker}{\pgfqpoint{0.000000in}{-0.048611in}}{\pgfqpoint{0.000000in}{0.000000in}}{%
\pgfpathmoveto{\pgfqpoint{0.000000in}{0.000000in}}%
\pgfpathlineto{\pgfqpoint{0.000000in}{-0.048611in}}%
\pgfusepath{stroke,fill}%
}%
\begin{pgfscope}%
\pgfsys@transformshift{1.834596in}{0.548769in}%
\pgfsys@useobject{currentmarker}{}%
\end{pgfscope}%
\end{pgfscope}%
\begin{pgfscope}%
\definecolor{textcolor}{rgb}{0.000000,0.000000,0.000000}%
\pgfsetstrokecolor{textcolor}%
\pgfsetfillcolor{textcolor}%
\pgftext[x=1.834596in,y=0.451547in,,top]{\color{textcolor}\sffamily\fontsize{10.000000}{12.000000}\selectfont \(\displaystyle {0.2}\)}%
\end{pgfscope}%
\begin{pgfscope}%
\pgfsetbuttcap%
\pgfsetroundjoin%
\definecolor{currentfill}{rgb}{0.000000,0.000000,0.000000}%
\pgfsetfillcolor{currentfill}%
\pgfsetlinewidth{0.803000pt}%
\definecolor{currentstroke}{rgb}{0.000000,0.000000,0.000000}%
\pgfsetstrokecolor{currentstroke}%
\pgfsetdash{}{0pt}%
\pgfsys@defobject{currentmarker}{\pgfqpoint{0.000000in}{-0.048611in}}{\pgfqpoint{0.000000in}{0.000000in}}{%
\pgfpathmoveto{\pgfqpoint{0.000000in}{0.000000in}}%
\pgfpathlineto{\pgfqpoint{0.000000in}{-0.048611in}}%
\pgfusepath{stroke,fill}%
}%
\begin{pgfscope}%
\pgfsys@transformshift{2.309330in}{0.548769in}%
\pgfsys@useobject{currentmarker}{}%
\end{pgfscope}%
\end{pgfscope}%
\begin{pgfscope}%
\definecolor{textcolor}{rgb}{0.000000,0.000000,0.000000}%
\pgfsetstrokecolor{textcolor}%
\pgfsetfillcolor{textcolor}%
\pgftext[x=2.309330in,y=0.451547in,,top]{\color{textcolor}\sffamily\fontsize{10.000000}{12.000000}\selectfont \(\displaystyle {0.3}\)}%
\end{pgfscope}%
\begin{pgfscope}%
\pgfsetbuttcap%
\pgfsetroundjoin%
\definecolor{currentfill}{rgb}{0.000000,0.000000,0.000000}%
\pgfsetfillcolor{currentfill}%
\pgfsetlinewidth{0.803000pt}%
\definecolor{currentstroke}{rgb}{0.000000,0.000000,0.000000}%
\pgfsetstrokecolor{currentstroke}%
\pgfsetdash{}{0pt}%
\pgfsys@defobject{currentmarker}{\pgfqpoint{0.000000in}{-0.048611in}}{\pgfqpoint{0.000000in}{0.000000in}}{%
\pgfpathmoveto{\pgfqpoint{0.000000in}{0.000000in}}%
\pgfpathlineto{\pgfqpoint{0.000000in}{-0.048611in}}%
\pgfusepath{stroke,fill}%
}%
\begin{pgfscope}%
\pgfsys@transformshift{2.784065in}{0.548769in}%
\pgfsys@useobject{currentmarker}{}%
\end{pgfscope}%
\end{pgfscope}%
\begin{pgfscope}%
\definecolor{textcolor}{rgb}{0.000000,0.000000,0.000000}%
\pgfsetstrokecolor{textcolor}%
\pgfsetfillcolor{textcolor}%
\pgftext[x=2.784065in,y=0.451547in,,top]{\color{textcolor}\sffamily\fontsize{10.000000}{12.000000}\selectfont \(\displaystyle {0.4}\)}%
\end{pgfscope}%
\begin{pgfscope}%
\pgfsetbuttcap%
\pgfsetroundjoin%
\definecolor{currentfill}{rgb}{0.000000,0.000000,0.000000}%
\pgfsetfillcolor{currentfill}%
\pgfsetlinewidth{0.803000pt}%
\definecolor{currentstroke}{rgb}{0.000000,0.000000,0.000000}%
\pgfsetstrokecolor{currentstroke}%
\pgfsetdash{}{0pt}%
\pgfsys@defobject{currentmarker}{\pgfqpoint{0.000000in}{-0.048611in}}{\pgfqpoint{0.000000in}{0.000000in}}{%
\pgfpathmoveto{\pgfqpoint{0.000000in}{0.000000in}}%
\pgfpathlineto{\pgfqpoint{0.000000in}{-0.048611in}}%
\pgfusepath{stroke,fill}%
}%
\begin{pgfscope}%
\pgfsys@transformshift{3.258800in}{0.548769in}%
\pgfsys@useobject{currentmarker}{}%
\end{pgfscope}%
\end{pgfscope}%
\begin{pgfscope}%
\definecolor{textcolor}{rgb}{0.000000,0.000000,0.000000}%
\pgfsetstrokecolor{textcolor}%
\pgfsetfillcolor{textcolor}%
\pgftext[x=3.258800in,y=0.451547in,,top]{\color{textcolor}\sffamily\fontsize{10.000000}{12.000000}\selectfont \(\displaystyle {0.5}\)}%
\end{pgfscope}%
\begin{pgfscope}%
\pgfsetbuttcap%
\pgfsetroundjoin%
\definecolor{currentfill}{rgb}{0.000000,0.000000,0.000000}%
\pgfsetfillcolor{currentfill}%
\pgfsetlinewidth{0.803000pt}%
\definecolor{currentstroke}{rgb}{0.000000,0.000000,0.000000}%
\pgfsetstrokecolor{currentstroke}%
\pgfsetdash{}{0pt}%
\pgfsys@defobject{currentmarker}{\pgfqpoint{0.000000in}{-0.048611in}}{\pgfqpoint{0.000000in}{0.000000in}}{%
\pgfpathmoveto{\pgfqpoint{0.000000in}{0.000000in}}%
\pgfpathlineto{\pgfqpoint{0.000000in}{-0.048611in}}%
\pgfusepath{stroke,fill}%
}%
\begin{pgfscope}%
\pgfsys@transformshift{3.733535in}{0.548769in}%
\pgfsys@useobject{currentmarker}{}%
\end{pgfscope}%
\end{pgfscope}%
\begin{pgfscope}%
\definecolor{textcolor}{rgb}{0.000000,0.000000,0.000000}%
\pgfsetstrokecolor{textcolor}%
\pgfsetfillcolor{textcolor}%
\pgftext[x=3.733535in,y=0.451547in,,top]{\color{textcolor}\sffamily\fontsize{10.000000}{12.000000}\selectfont \(\displaystyle {0.6}\)}%
\end{pgfscope}%
\begin{pgfscope}%
\pgfsetbuttcap%
\pgfsetroundjoin%
\definecolor{currentfill}{rgb}{0.000000,0.000000,0.000000}%
\pgfsetfillcolor{currentfill}%
\pgfsetlinewidth{0.803000pt}%
\definecolor{currentstroke}{rgb}{0.000000,0.000000,0.000000}%
\pgfsetstrokecolor{currentstroke}%
\pgfsetdash{}{0pt}%
\pgfsys@defobject{currentmarker}{\pgfqpoint{0.000000in}{-0.048611in}}{\pgfqpoint{0.000000in}{0.000000in}}{%
\pgfpathmoveto{\pgfqpoint{0.000000in}{0.000000in}}%
\pgfpathlineto{\pgfqpoint{0.000000in}{-0.048611in}}%
\pgfusepath{stroke,fill}%
}%
\begin{pgfscope}%
\pgfsys@transformshift{4.208270in}{0.548769in}%
\pgfsys@useobject{currentmarker}{}%
\end{pgfscope}%
\end{pgfscope}%
\begin{pgfscope}%
\definecolor{textcolor}{rgb}{0.000000,0.000000,0.000000}%
\pgfsetstrokecolor{textcolor}%
\pgfsetfillcolor{textcolor}%
\pgftext[x=4.208270in,y=0.451547in,,top]{\color{textcolor}\sffamily\fontsize{10.000000}{12.000000}\selectfont \(\displaystyle {0.7}\)}%
\end{pgfscope}%
\begin{pgfscope}%
\pgfsetbuttcap%
\pgfsetroundjoin%
\definecolor{currentfill}{rgb}{0.000000,0.000000,0.000000}%
\pgfsetfillcolor{currentfill}%
\pgfsetlinewidth{0.803000pt}%
\definecolor{currentstroke}{rgb}{0.000000,0.000000,0.000000}%
\pgfsetstrokecolor{currentstroke}%
\pgfsetdash{}{0pt}%
\pgfsys@defobject{currentmarker}{\pgfqpoint{0.000000in}{-0.048611in}}{\pgfqpoint{0.000000in}{0.000000in}}{%
\pgfpathmoveto{\pgfqpoint{0.000000in}{0.000000in}}%
\pgfpathlineto{\pgfqpoint{0.000000in}{-0.048611in}}%
\pgfusepath{stroke,fill}%
}%
\begin{pgfscope}%
\pgfsys@transformshift{4.683005in}{0.548769in}%
\pgfsys@useobject{currentmarker}{}%
\end{pgfscope}%
\end{pgfscope}%
\begin{pgfscope}%
\definecolor{textcolor}{rgb}{0.000000,0.000000,0.000000}%
\pgfsetstrokecolor{textcolor}%
\pgfsetfillcolor{textcolor}%
\pgftext[x=4.683005in,y=0.451547in,,top]{\color{textcolor}\sffamily\fontsize{10.000000}{12.000000}\selectfont \(\displaystyle {0.8}\)}%
\end{pgfscope}%
\begin{pgfscope}%
\pgfsetbuttcap%
\pgfsetroundjoin%
\definecolor{currentfill}{rgb}{0.000000,0.000000,0.000000}%
\pgfsetfillcolor{currentfill}%
\pgfsetlinewidth{0.803000pt}%
\definecolor{currentstroke}{rgb}{0.000000,0.000000,0.000000}%
\pgfsetstrokecolor{currentstroke}%
\pgfsetdash{}{0pt}%
\pgfsys@defobject{currentmarker}{\pgfqpoint{0.000000in}{-0.048611in}}{\pgfqpoint{0.000000in}{0.000000in}}{%
\pgfpathmoveto{\pgfqpoint{0.000000in}{0.000000in}}%
\pgfpathlineto{\pgfqpoint{0.000000in}{-0.048611in}}%
\pgfusepath{stroke,fill}%
}%
\begin{pgfscope}%
\pgfsys@transformshift{5.157740in}{0.548769in}%
\pgfsys@useobject{currentmarker}{}%
\end{pgfscope}%
\end{pgfscope}%
\begin{pgfscope}%
\definecolor{textcolor}{rgb}{0.000000,0.000000,0.000000}%
\pgfsetstrokecolor{textcolor}%
\pgfsetfillcolor{textcolor}%
\pgftext[x=5.157740in,y=0.451547in,,top]{\color{textcolor}\sffamily\fontsize{10.000000}{12.000000}\selectfont \(\displaystyle {0.9}\)}%
\end{pgfscope}%
\begin{pgfscope}%
\pgfsetbuttcap%
\pgfsetroundjoin%
\definecolor{currentfill}{rgb}{0.000000,0.000000,0.000000}%
\pgfsetfillcolor{currentfill}%
\pgfsetlinewidth{0.803000pt}%
\definecolor{currentstroke}{rgb}{0.000000,0.000000,0.000000}%
\pgfsetstrokecolor{currentstroke}%
\pgfsetdash{}{0pt}%
\pgfsys@defobject{currentmarker}{\pgfqpoint{0.000000in}{-0.048611in}}{\pgfqpoint{0.000000in}{0.000000in}}{%
\pgfpathmoveto{\pgfqpoint{0.000000in}{0.000000in}}%
\pgfpathlineto{\pgfqpoint{0.000000in}{-0.048611in}}%
\pgfusepath{stroke,fill}%
}%
\begin{pgfscope}%
\pgfsys@transformshift{5.632474in}{0.548769in}%
\pgfsys@useobject{currentmarker}{}%
\end{pgfscope}%
\end{pgfscope}%
\begin{pgfscope}%
\definecolor{textcolor}{rgb}{0.000000,0.000000,0.000000}%
\pgfsetstrokecolor{textcolor}%
\pgfsetfillcolor{textcolor}%
\pgftext[x=5.632474in,y=0.451547in,,top]{\color{textcolor}\sffamily\fontsize{10.000000}{12.000000}\selectfont \(\displaystyle {1.0}\)}%
\end{pgfscope}%
\begin{pgfscope}%
\definecolor{textcolor}{rgb}{0.000000,0.000000,0.000000}%
\pgfsetstrokecolor{textcolor}%
\pgfsetfillcolor{textcolor}%
\pgftext[x=3.249352in,y=0.272658in,,top]{\color{textcolor}\sffamily\fontsize{10.000000}{12.000000}\selectfont Edge Count}%
\end{pgfscope}%
\begin{pgfscope}%
\definecolor{textcolor}{rgb}{0.000000,0.000000,0.000000}%
\pgfsetstrokecolor{textcolor}%
\pgfsetfillcolor{textcolor}%
\pgftext[x=5.850000in,y=0.286547in,right,top]{\color{textcolor}\sffamily\fontsize{10.000000}{12.000000}\selectfont \(\displaystyle \times{10^{8}}{}\)}%
\end{pgfscope}%
\begin{pgfscope}%
\pgfsetbuttcap%
\pgfsetroundjoin%
\definecolor{currentfill}{rgb}{0.000000,0.000000,0.000000}%
\pgfsetfillcolor{currentfill}%
\pgfsetlinewidth{0.803000pt}%
\definecolor{currentstroke}{rgb}{0.000000,0.000000,0.000000}%
\pgfsetstrokecolor{currentstroke}%
\pgfsetdash{}{0pt}%
\pgfsys@defobject{currentmarker}{\pgfqpoint{-0.048611in}{0.000000in}}{\pgfqpoint{0.000000in}{0.000000in}}{%
\pgfpathmoveto{\pgfqpoint{0.000000in}{0.000000in}}%
\pgfpathlineto{\pgfqpoint{-0.048611in}{0.000000in}}%
\pgfusepath{stroke,fill}%
}%
\begin{pgfscope}%
\pgfsys@transformshift{0.648703in}{0.689796in}%
\pgfsys@useobject{currentmarker}{}%
\end{pgfscope}%
\end{pgfscope}%
\begin{pgfscope}%
\definecolor{textcolor}{rgb}{0.000000,0.000000,0.000000}%
\pgfsetstrokecolor{textcolor}%
\pgfsetfillcolor{textcolor}%
\pgftext[x=0.482036in, y=0.641601in, left, base]{\color{textcolor}\sffamily\fontsize{10.000000}{12.000000}\selectfont \(\displaystyle {0}\)}%
\end{pgfscope}%
\begin{pgfscope}%
\pgfsetbuttcap%
\pgfsetroundjoin%
\definecolor{currentfill}{rgb}{0.000000,0.000000,0.000000}%
\pgfsetfillcolor{currentfill}%
\pgfsetlinewidth{0.803000pt}%
\definecolor{currentstroke}{rgb}{0.000000,0.000000,0.000000}%
\pgfsetstrokecolor{currentstroke}%
\pgfsetdash{}{0pt}%
\pgfsys@defobject{currentmarker}{\pgfqpoint{-0.048611in}{0.000000in}}{\pgfqpoint{0.000000in}{0.000000in}}{%
\pgfpathmoveto{\pgfqpoint{0.000000in}{0.000000in}}%
\pgfpathlineto{\pgfqpoint{-0.048611in}{0.000000in}}%
\pgfusepath{stroke,fill}%
}%
\begin{pgfscope}%
\pgfsys@transformshift{0.648703in}{1.112665in}%
\pgfsys@useobject{currentmarker}{}%
\end{pgfscope}%
\end{pgfscope}%
\begin{pgfscope}%
\definecolor{textcolor}{rgb}{0.000000,0.000000,0.000000}%
\pgfsetstrokecolor{textcolor}%
\pgfsetfillcolor{textcolor}%
\pgftext[x=0.343147in, y=1.064470in, left, base]{\color{textcolor}\sffamily\fontsize{10.000000}{12.000000}\selectfont \(\displaystyle {100}\)}%
\end{pgfscope}%
\begin{pgfscope}%
\pgfsetbuttcap%
\pgfsetroundjoin%
\definecolor{currentfill}{rgb}{0.000000,0.000000,0.000000}%
\pgfsetfillcolor{currentfill}%
\pgfsetlinewidth{0.803000pt}%
\definecolor{currentstroke}{rgb}{0.000000,0.000000,0.000000}%
\pgfsetstrokecolor{currentstroke}%
\pgfsetdash{}{0pt}%
\pgfsys@defobject{currentmarker}{\pgfqpoint{-0.048611in}{0.000000in}}{\pgfqpoint{0.000000in}{0.000000in}}{%
\pgfpathmoveto{\pgfqpoint{0.000000in}{0.000000in}}%
\pgfpathlineto{\pgfqpoint{-0.048611in}{0.000000in}}%
\pgfusepath{stroke,fill}%
}%
\begin{pgfscope}%
\pgfsys@transformshift{0.648703in}{1.535534in}%
\pgfsys@useobject{currentmarker}{}%
\end{pgfscope}%
\end{pgfscope}%
\begin{pgfscope}%
\definecolor{textcolor}{rgb}{0.000000,0.000000,0.000000}%
\pgfsetstrokecolor{textcolor}%
\pgfsetfillcolor{textcolor}%
\pgftext[x=0.343147in, y=1.487339in, left, base]{\color{textcolor}\sffamily\fontsize{10.000000}{12.000000}\selectfont \(\displaystyle {200}\)}%
\end{pgfscope}%
\begin{pgfscope}%
\pgfsetbuttcap%
\pgfsetroundjoin%
\definecolor{currentfill}{rgb}{0.000000,0.000000,0.000000}%
\pgfsetfillcolor{currentfill}%
\pgfsetlinewidth{0.803000pt}%
\definecolor{currentstroke}{rgb}{0.000000,0.000000,0.000000}%
\pgfsetstrokecolor{currentstroke}%
\pgfsetdash{}{0pt}%
\pgfsys@defobject{currentmarker}{\pgfqpoint{-0.048611in}{0.000000in}}{\pgfqpoint{0.000000in}{0.000000in}}{%
\pgfpathmoveto{\pgfqpoint{0.000000in}{0.000000in}}%
\pgfpathlineto{\pgfqpoint{-0.048611in}{0.000000in}}%
\pgfusepath{stroke,fill}%
}%
\begin{pgfscope}%
\pgfsys@transformshift{0.648703in}{1.958403in}%
\pgfsys@useobject{currentmarker}{}%
\end{pgfscope}%
\end{pgfscope}%
\begin{pgfscope}%
\definecolor{textcolor}{rgb}{0.000000,0.000000,0.000000}%
\pgfsetstrokecolor{textcolor}%
\pgfsetfillcolor{textcolor}%
\pgftext[x=0.343147in, y=1.910208in, left, base]{\color{textcolor}\sffamily\fontsize{10.000000}{12.000000}\selectfont \(\displaystyle {300}\)}%
\end{pgfscope}%
\begin{pgfscope}%
\pgfsetbuttcap%
\pgfsetroundjoin%
\definecolor{currentfill}{rgb}{0.000000,0.000000,0.000000}%
\pgfsetfillcolor{currentfill}%
\pgfsetlinewidth{0.803000pt}%
\definecolor{currentstroke}{rgb}{0.000000,0.000000,0.000000}%
\pgfsetstrokecolor{currentstroke}%
\pgfsetdash{}{0pt}%
\pgfsys@defobject{currentmarker}{\pgfqpoint{-0.048611in}{0.000000in}}{\pgfqpoint{0.000000in}{0.000000in}}{%
\pgfpathmoveto{\pgfqpoint{0.000000in}{0.000000in}}%
\pgfpathlineto{\pgfqpoint{-0.048611in}{0.000000in}}%
\pgfusepath{stroke,fill}%
}%
\begin{pgfscope}%
\pgfsys@transformshift{0.648703in}{2.381272in}%
\pgfsys@useobject{currentmarker}{}%
\end{pgfscope}%
\end{pgfscope}%
\begin{pgfscope}%
\definecolor{textcolor}{rgb}{0.000000,0.000000,0.000000}%
\pgfsetstrokecolor{textcolor}%
\pgfsetfillcolor{textcolor}%
\pgftext[x=0.343147in, y=2.333077in, left, base]{\color{textcolor}\sffamily\fontsize{10.000000}{12.000000}\selectfont \(\displaystyle {400}\)}%
\end{pgfscope}%
\begin{pgfscope}%
\pgfsetbuttcap%
\pgfsetroundjoin%
\definecolor{currentfill}{rgb}{0.000000,0.000000,0.000000}%
\pgfsetfillcolor{currentfill}%
\pgfsetlinewidth{0.803000pt}%
\definecolor{currentstroke}{rgb}{0.000000,0.000000,0.000000}%
\pgfsetstrokecolor{currentstroke}%
\pgfsetdash{}{0pt}%
\pgfsys@defobject{currentmarker}{\pgfqpoint{-0.048611in}{0.000000in}}{\pgfqpoint{0.000000in}{0.000000in}}{%
\pgfpathmoveto{\pgfqpoint{0.000000in}{0.000000in}}%
\pgfpathlineto{\pgfqpoint{-0.048611in}{0.000000in}}%
\pgfusepath{stroke,fill}%
}%
\begin{pgfscope}%
\pgfsys@transformshift{0.648703in}{2.804141in}%
\pgfsys@useobject{currentmarker}{}%
\end{pgfscope}%
\end{pgfscope}%
\begin{pgfscope}%
\definecolor{textcolor}{rgb}{0.000000,0.000000,0.000000}%
\pgfsetstrokecolor{textcolor}%
\pgfsetfillcolor{textcolor}%
\pgftext[x=0.343147in, y=2.755946in, left, base]{\color{textcolor}\sffamily\fontsize{10.000000}{12.000000}\selectfont \(\displaystyle {500}\)}%
\end{pgfscope}%
\begin{pgfscope}%
\pgfsetbuttcap%
\pgfsetroundjoin%
\definecolor{currentfill}{rgb}{0.000000,0.000000,0.000000}%
\pgfsetfillcolor{currentfill}%
\pgfsetlinewidth{0.803000pt}%
\definecolor{currentstroke}{rgb}{0.000000,0.000000,0.000000}%
\pgfsetstrokecolor{currentstroke}%
\pgfsetdash{}{0pt}%
\pgfsys@defobject{currentmarker}{\pgfqpoint{-0.048611in}{0.000000in}}{\pgfqpoint{0.000000in}{0.000000in}}{%
\pgfpathmoveto{\pgfqpoint{0.000000in}{0.000000in}}%
\pgfpathlineto{\pgfqpoint{-0.048611in}{0.000000in}}%
\pgfusepath{stroke,fill}%
}%
\begin{pgfscope}%
\pgfsys@transformshift{0.648703in}{3.227010in}%
\pgfsys@useobject{currentmarker}{}%
\end{pgfscope}%
\end{pgfscope}%
\begin{pgfscope}%
\definecolor{textcolor}{rgb}{0.000000,0.000000,0.000000}%
\pgfsetstrokecolor{textcolor}%
\pgfsetfillcolor{textcolor}%
\pgftext[x=0.343147in, y=3.178815in, left, base]{\color{textcolor}\sffamily\fontsize{10.000000}{12.000000}\selectfont \(\displaystyle {600}\)}%
\end{pgfscope}%
\begin{pgfscope}%
\pgfsetbuttcap%
\pgfsetroundjoin%
\definecolor{currentfill}{rgb}{0.000000,0.000000,0.000000}%
\pgfsetfillcolor{currentfill}%
\pgfsetlinewidth{0.803000pt}%
\definecolor{currentstroke}{rgb}{0.000000,0.000000,0.000000}%
\pgfsetstrokecolor{currentstroke}%
\pgfsetdash{}{0pt}%
\pgfsys@defobject{currentmarker}{\pgfqpoint{-0.048611in}{0.000000in}}{\pgfqpoint{0.000000in}{0.000000in}}{%
\pgfpathmoveto{\pgfqpoint{0.000000in}{0.000000in}}%
\pgfpathlineto{\pgfqpoint{-0.048611in}{0.000000in}}%
\pgfusepath{stroke,fill}%
}%
\begin{pgfscope}%
\pgfsys@transformshift{0.648703in}{3.649879in}%
\pgfsys@useobject{currentmarker}{}%
\end{pgfscope}%
\end{pgfscope}%
\begin{pgfscope}%
\definecolor{textcolor}{rgb}{0.000000,0.000000,0.000000}%
\pgfsetstrokecolor{textcolor}%
\pgfsetfillcolor{textcolor}%
\pgftext[x=0.343147in, y=3.601684in, left, base]{\color{textcolor}\sffamily\fontsize{10.000000}{12.000000}\selectfont \(\displaystyle {700}\)}%
\end{pgfscope}%
\begin{pgfscope}%
\definecolor{textcolor}{rgb}{0.000000,0.000000,0.000000}%
\pgfsetstrokecolor{textcolor}%
\pgfsetfillcolor{textcolor}%
\pgftext[x=0.287592in,y=2.100064in,,bottom,rotate=90.000000]{\color{textcolor}\sffamily\fontsize{10.000000}{12.000000}\selectfont Data Flow Time (s)}%
\end{pgfscope}%
\begin{pgfscope}%
\pgfsetrectcap%
\pgfsetmiterjoin%
\pgfsetlinewidth{0.803000pt}%
\definecolor{currentstroke}{rgb}{0.000000,0.000000,0.000000}%
\pgfsetstrokecolor{currentstroke}%
\pgfsetdash{}{0pt}%
\pgfpathmoveto{\pgfqpoint{0.648703in}{0.548769in}}%
\pgfpathlineto{\pgfqpoint{0.648703in}{3.651359in}}%
\pgfusepath{stroke}%
\end{pgfscope}%
\begin{pgfscope}%
\pgfsetrectcap%
\pgfsetmiterjoin%
\pgfsetlinewidth{0.803000pt}%
\definecolor{currentstroke}{rgb}{0.000000,0.000000,0.000000}%
\pgfsetstrokecolor{currentstroke}%
\pgfsetdash{}{0pt}%
\pgfpathmoveto{\pgfqpoint{5.850000in}{0.548769in}}%
\pgfpathlineto{\pgfqpoint{5.850000in}{3.651359in}}%
\pgfusepath{stroke}%
\end{pgfscope}%
\begin{pgfscope}%
\pgfsetrectcap%
\pgfsetmiterjoin%
\pgfsetlinewidth{0.803000pt}%
\definecolor{currentstroke}{rgb}{0.000000,0.000000,0.000000}%
\pgfsetstrokecolor{currentstroke}%
\pgfsetdash{}{0pt}%
\pgfpathmoveto{\pgfqpoint{0.648703in}{0.548769in}}%
\pgfpathlineto{\pgfqpoint{5.850000in}{0.548769in}}%
\pgfusepath{stroke}%
\end{pgfscope}%
\begin{pgfscope}%
\pgfsetrectcap%
\pgfsetmiterjoin%
\pgfsetlinewidth{0.803000pt}%
\definecolor{currentstroke}{rgb}{0.000000,0.000000,0.000000}%
\pgfsetstrokecolor{currentstroke}%
\pgfsetdash{}{0pt}%
\pgfpathmoveto{\pgfqpoint{0.648703in}{3.651359in}}%
\pgfpathlineto{\pgfqpoint{5.850000in}{3.651359in}}%
\pgfusepath{stroke}%
\end{pgfscope}%
\begin{pgfscope}%
\definecolor{textcolor}{rgb}{0.000000,0.000000,0.000000}%
\pgfsetstrokecolor{textcolor}%
\pgfsetfillcolor{textcolor}%
\pgftext[x=3.249352in,y=3.734692in,,base]{\color{textcolor}\sffamily\fontsize{12.000000}{14.400000}\selectfont Backward}%
\end{pgfscope}%
\begin{pgfscope}%
\pgfsetbuttcap%
\pgfsetmiterjoin%
\definecolor{currentfill}{rgb}{1.000000,1.000000,1.000000}%
\pgfsetfillcolor{currentfill}%
\pgfsetfillopacity{0.800000}%
\pgfsetlinewidth{1.003750pt}%
\definecolor{currentstroke}{rgb}{0.800000,0.800000,0.800000}%
\pgfsetstrokecolor{currentstroke}%
\pgfsetstrokeopacity{0.800000}%
\pgfsetdash{}{0pt}%
\pgfpathmoveto{\pgfqpoint{0.745926in}{2.957886in}}%
\pgfpathlineto{\pgfqpoint{2.198287in}{2.957886in}}%
\pgfpathquadraticcurveto{\pgfqpoint{2.226065in}{2.957886in}}{\pgfqpoint{2.226065in}{2.985664in}}%
\pgfpathlineto{\pgfqpoint{2.226065in}{3.554136in}}%
\pgfpathquadraticcurveto{\pgfqpoint{2.226065in}{3.581914in}}{\pgfqpoint{2.198287in}{3.581914in}}%
\pgfpathlineto{\pgfqpoint{0.745926in}{3.581914in}}%
\pgfpathquadraticcurveto{\pgfqpoint{0.718148in}{3.581914in}}{\pgfqpoint{0.718148in}{3.554136in}}%
\pgfpathlineto{\pgfqpoint{0.718148in}{2.985664in}}%
\pgfpathquadraticcurveto{\pgfqpoint{0.718148in}{2.957886in}}{\pgfqpoint{0.745926in}{2.957886in}}%
\pgfpathclose%
\pgfusepath{stroke,fill}%
\end{pgfscope}%
\begin{pgfscope}%
\pgfsetbuttcap%
\pgfsetroundjoin%
\definecolor{currentfill}{rgb}{0.121569,0.466667,0.705882}%
\pgfsetfillcolor{currentfill}%
\pgfsetlinewidth{1.003750pt}%
\definecolor{currentstroke}{rgb}{0.121569,0.466667,0.705882}%
\pgfsetstrokecolor{currentstroke}%
\pgfsetdash{}{0pt}%
\pgfsys@defobject{currentmarker}{\pgfqpoint{-0.034722in}{-0.034722in}}{\pgfqpoint{0.034722in}{0.034722in}}{%
\pgfpathmoveto{\pgfqpoint{0.000000in}{-0.034722in}}%
\pgfpathcurveto{\pgfqpoint{0.009208in}{-0.034722in}}{\pgfqpoint{0.018041in}{-0.031064in}}{\pgfqpoint{0.024552in}{-0.024552in}}%
\pgfpathcurveto{\pgfqpoint{0.031064in}{-0.018041in}}{\pgfqpoint{0.034722in}{-0.009208in}}{\pgfqpoint{0.034722in}{0.000000in}}%
\pgfpathcurveto{\pgfqpoint{0.034722in}{0.009208in}}{\pgfqpoint{0.031064in}{0.018041in}}{\pgfqpoint{0.024552in}{0.024552in}}%
\pgfpathcurveto{\pgfqpoint{0.018041in}{0.031064in}}{\pgfqpoint{0.009208in}{0.034722in}}{\pgfqpoint{0.000000in}{0.034722in}}%
\pgfpathcurveto{\pgfqpoint{-0.009208in}{0.034722in}}{\pgfqpoint{-0.018041in}{0.031064in}}{\pgfqpoint{-0.024552in}{0.024552in}}%
\pgfpathcurveto{\pgfqpoint{-0.031064in}{0.018041in}}{\pgfqpoint{-0.034722in}{0.009208in}}{\pgfqpoint{-0.034722in}{0.000000in}}%
\pgfpathcurveto{\pgfqpoint{-0.034722in}{-0.009208in}}{\pgfqpoint{-0.031064in}{-0.018041in}}{\pgfqpoint{-0.024552in}{-0.024552in}}%
\pgfpathcurveto{\pgfqpoint{-0.018041in}{-0.031064in}}{\pgfqpoint{-0.009208in}{-0.034722in}}{\pgfqpoint{0.000000in}{-0.034722in}}%
\pgfpathclose%
\pgfusepath{stroke,fill}%
}%
\begin{pgfscope}%
\pgfsys@transformshift{0.912592in}{3.477748in}%
\pgfsys@useobject{currentmarker}{}%
\end{pgfscope}%
\end{pgfscope}%
\begin{pgfscope}%
\definecolor{textcolor}{rgb}{0.000000,0.000000,0.000000}%
\pgfsetstrokecolor{textcolor}%
\pgfsetfillcolor{textcolor}%
\pgftext[x=1.162592in,y=3.429136in,left,base]{\color{textcolor}\sffamily\fontsize{10.000000}{12.000000}\selectfont No Timeout}%
\end{pgfscope}%
\begin{pgfscope}%
\pgfsetbuttcap%
\pgfsetroundjoin%
\definecolor{currentfill}{rgb}{1.000000,0.498039,0.054902}%
\pgfsetfillcolor{currentfill}%
\pgfsetlinewidth{1.003750pt}%
\definecolor{currentstroke}{rgb}{1.000000,0.498039,0.054902}%
\pgfsetstrokecolor{currentstroke}%
\pgfsetdash{}{0pt}%
\pgfsys@defobject{currentmarker}{\pgfqpoint{-0.034722in}{-0.034722in}}{\pgfqpoint{0.034722in}{0.034722in}}{%
\pgfpathmoveto{\pgfqpoint{0.000000in}{-0.034722in}}%
\pgfpathcurveto{\pgfqpoint{0.009208in}{-0.034722in}}{\pgfqpoint{0.018041in}{-0.031064in}}{\pgfqpoint{0.024552in}{-0.024552in}}%
\pgfpathcurveto{\pgfqpoint{0.031064in}{-0.018041in}}{\pgfqpoint{0.034722in}{-0.009208in}}{\pgfqpoint{0.034722in}{0.000000in}}%
\pgfpathcurveto{\pgfqpoint{0.034722in}{0.009208in}}{\pgfqpoint{0.031064in}{0.018041in}}{\pgfqpoint{0.024552in}{0.024552in}}%
\pgfpathcurveto{\pgfqpoint{0.018041in}{0.031064in}}{\pgfqpoint{0.009208in}{0.034722in}}{\pgfqpoint{0.000000in}{0.034722in}}%
\pgfpathcurveto{\pgfqpoint{-0.009208in}{0.034722in}}{\pgfqpoint{-0.018041in}{0.031064in}}{\pgfqpoint{-0.024552in}{0.024552in}}%
\pgfpathcurveto{\pgfqpoint{-0.031064in}{0.018041in}}{\pgfqpoint{-0.034722in}{0.009208in}}{\pgfqpoint{-0.034722in}{0.000000in}}%
\pgfpathcurveto{\pgfqpoint{-0.034722in}{-0.009208in}}{\pgfqpoint{-0.031064in}{-0.018041in}}{\pgfqpoint{-0.024552in}{-0.024552in}}%
\pgfpathcurveto{\pgfqpoint{-0.018041in}{-0.031064in}}{\pgfqpoint{-0.009208in}{-0.034722in}}{\pgfqpoint{0.000000in}{-0.034722in}}%
\pgfpathclose%
\pgfusepath{stroke,fill}%
}%
\begin{pgfscope}%
\pgfsys@transformshift{0.912592in}{3.284136in}%
\pgfsys@useobject{currentmarker}{}%
\end{pgfscope}%
\end{pgfscope}%
\begin{pgfscope}%
\definecolor{textcolor}{rgb}{0.000000,0.000000,0.000000}%
\pgfsetstrokecolor{textcolor}%
\pgfsetfillcolor{textcolor}%
\pgftext[x=1.162592in,y=3.235525in,left,base]{\color{textcolor}\sffamily\fontsize{10.000000}{12.000000}\selectfont Time Timeout}%
\end{pgfscope}%
\begin{pgfscope}%
\pgfsetbuttcap%
\pgfsetroundjoin%
\definecolor{currentfill}{rgb}{0.839216,0.152941,0.156863}%
\pgfsetfillcolor{currentfill}%
\pgfsetlinewidth{1.003750pt}%
\definecolor{currentstroke}{rgb}{0.839216,0.152941,0.156863}%
\pgfsetstrokecolor{currentstroke}%
\pgfsetdash{}{0pt}%
\pgfsys@defobject{currentmarker}{\pgfqpoint{-0.034722in}{-0.034722in}}{\pgfqpoint{0.034722in}{0.034722in}}{%
\pgfpathmoveto{\pgfqpoint{0.000000in}{-0.034722in}}%
\pgfpathcurveto{\pgfqpoint{0.009208in}{-0.034722in}}{\pgfqpoint{0.018041in}{-0.031064in}}{\pgfqpoint{0.024552in}{-0.024552in}}%
\pgfpathcurveto{\pgfqpoint{0.031064in}{-0.018041in}}{\pgfqpoint{0.034722in}{-0.009208in}}{\pgfqpoint{0.034722in}{0.000000in}}%
\pgfpathcurveto{\pgfqpoint{0.034722in}{0.009208in}}{\pgfqpoint{0.031064in}{0.018041in}}{\pgfqpoint{0.024552in}{0.024552in}}%
\pgfpathcurveto{\pgfqpoint{0.018041in}{0.031064in}}{\pgfqpoint{0.009208in}{0.034722in}}{\pgfqpoint{0.000000in}{0.034722in}}%
\pgfpathcurveto{\pgfqpoint{-0.009208in}{0.034722in}}{\pgfqpoint{-0.018041in}{0.031064in}}{\pgfqpoint{-0.024552in}{0.024552in}}%
\pgfpathcurveto{\pgfqpoint{-0.031064in}{0.018041in}}{\pgfqpoint{-0.034722in}{0.009208in}}{\pgfqpoint{-0.034722in}{0.000000in}}%
\pgfpathcurveto{\pgfqpoint{-0.034722in}{-0.009208in}}{\pgfqpoint{-0.031064in}{-0.018041in}}{\pgfqpoint{-0.024552in}{-0.024552in}}%
\pgfpathcurveto{\pgfqpoint{-0.018041in}{-0.031064in}}{\pgfqpoint{-0.009208in}{-0.034722in}}{\pgfqpoint{0.000000in}{-0.034722in}}%
\pgfpathclose%
\pgfusepath{stroke,fill}%
}%
\begin{pgfscope}%
\pgfsys@transformshift{0.912592in}{3.090525in}%
\pgfsys@useobject{currentmarker}{}%
\end{pgfscope}%
\end{pgfscope}%
\begin{pgfscope}%
\definecolor{textcolor}{rgb}{0.000000,0.000000,0.000000}%
\pgfsetstrokecolor{textcolor}%
\pgfsetfillcolor{textcolor}%
\pgftext[x=1.162592in,y=3.041914in,left,base]{\color{textcolor}\sffamily\fontsize{10.000000}{12.000000}\selectfont Memory Timeout}%
\end{pgfscope}%
\end{pgfpicture}%
\makeatother%
\endgroup%

                }
            \end{subfigure}
            \caption{Total Edges}
            \label{f:dfedgestotal}
        \end{subfigure}
        \bigbreak
        \begin{subfigure}[b]{\textwidth}
            \centering
            \begin{subfigure}[]{0.45\textwidth}
                \centering
                \resizebox{\columnwidth}{!}{
                    %% Creator: Matplotlib, PGF backend
%%
%% To include the figure in your LaTeX document, write
%%   \input{<filename>.pgf}
%%
%% Make sure the required packages are loaded in your preamble
%%   \usepackage{pgf}
%%
%% and, on pdftex
%%   \usepackage[utf8]{inputenc}\DeclareUnicodeCharacter{2212}{-}
%%
%% or, on luatex and xetex
%%   \usepackage{unicode-math}
%%
%% Figures using additional raster images can only be included by \input if
%% they are in the same directory as the main LaTeX file. For loading figures
%% from other directories you can use the `import` package
%%   \usepackage{import}
%%
%% and then include the figures with
%%   \import{<path to file>}{<filename>.pgf}
%%
%% Matplotlib used the following preamble
%%   \usepackage{amsmath}
%%   \usepackage{fontspec}
%%
\begingroup%
\makeatletter%
\begin{pgfpicture}%
\pgfpathrectangle{\pgfpointorigin}{\pgfqpoint{6.000000in}{4.000000in}}%
\pgfusepath{use as bounding box, clip}%
\begin{pgfscope}%
\pgfsetbuttcap%
\pgfsetmiterjoin%
\definecolor{currentfill}{rgb}{1.000000,1.000000,1.000000}%
\pgfsetfillcolor{currentfill}%
\pgfsetlinewidth{0.000000pt}%
\definecolor{currentstroke}{rgb}{1.000000,1.000000,1.000000}%
\pgfsetstrokecolor{currentstroke}%
\pgfsetdash{}{0pt}%
\pgfpathmoveto{\pgfqpoint{0.000000in}{0.000000in}}%
\pgfpathlineto{\pgfqpoint{6.000000in}{0.000000in}}%
\pgfpathlineto{\pgfqpoint{6.000000in}{4.000000in}}%
\pgfpathlineto{\pgfqpoint{0.000000in}{4.000000in}}%
\pgfpathclose%
\pgfusepath{fill}%
\end{pgfscope}%
\begin{pgfscope}%
\pgfsetbuttcap%
\pgfsetmiterjoin%
\definecolor{currentfill}{rgb}{1.000000,1.000000,1.000000}%
\pgfsetfillcolor{currentfill}%
\pgfsetlinewidth{0.000000pt}%
\definecolor{currentstroke}{rgb}{0.000000,0.000000,0.000000}%
\pgfsetstrokecolor{currentstroke}%
\pgfsetstrokeopacity{0.000000}%
\pgfsetdash{}{0pt}%
\pgfpathmoveto{\pgfqpoint{0.648703in}{0.548769in}}%
\pgfpathlineto{\pgfqpoint{5.761597in}{0.548769in}}%
\pgfpathlineto{\pgfqpoint{5.761597in}{3.651359in}}%
\pgfpathlineto{\pgfqpoint{0.648703in}{3.651359in}}%
\pgfpathclose%
\pgfusepath{fill}%
\end{pgfscope}%
\begin{pgfscope}%
\pgfpathrectangle{\pgfqpoint{0.648703in}{0.548769in}}{\pgfqpoint{5.112893in}{3.102590in}}%
\pgfusepath{clip}%
\pgfsetbuttcap%
\pgfsetroundjoin%
\definecolor{currentfill}{rgb}{0.121569,0.466667,0.705882}%
\pgfsetfillcolor{currentfill}%
\pgfsetlinewidth{1.003750pt}%
\definecolor{currentstroke}{rgb}{0.121569,0.466667,0.705882}%
\pgfsetstrokecolor{currentstroke}%
\pgfsetdash{}{0pt}%
\pgfpathmoveto{\pgfqpoint{0.814247in}{0.648129in}}%
\pgfpathcurveto{\pgfqpoint{0.825297in}{0.648129in}}{\pgfqpoint{0.835896in}{0.652519in}}{\pgfqpoint{0.843710in}{0.660333in}}%
\pgfpathcurveto{\pgfqpoint{0.851524in}{0.668146in}}{\pgfqpoint{0.855914in}{0.678745in}}{\pgfqpoint{0.855914in}{0.689796in}}%
\pgfpathcurveto{\pgfqpoint{0.855914in}{0.700846in}}{\pgfqpoint{0.851524in}{0.711445in}}{\pgfqpoint{0.843710in}{0.719258in}}%
\pgfpathcurveto{\pgfqpoint{0.835896in}{0.727072in}}{\pgfqpoint{0.825297in}{0.731462in}}{\pgfqpoint{0.814247in}{0.731462in}}%
\pgfpathcurveto{\pgfqpoint{0.803197in}{0.731462in}}{\pgfqpoint{0.792598in}{0.727072in}}{\pgfqpoint{0.784784in}{0.719258in}}%
\pgfpathcurveto{\pgfqpoint{0.776971in}{0.711445in}}{\pgfqpoint{0.772581in}{0.700846in}}{\pgfqpoint{0.772581in}{0.689796in}}%
\pgfpathcurveto{\pgfqpoint{0.772581in}{0.678745in}}{\pgfqpoint{0.776971in}{0.668146in}}{\pgfqpoint{0.784784in}{0.660333in}}%
\pgfpathcurveto{\pgfqpoint{0.792598in}{0.652519in}}{\pgfqpoint{0.803197in}{0.648129in}}{\pgfqpoint{0.814247in}{0.648129in}}%
\pgfpathclose%
\pgfusepath{stroke,fill}%
\end{pgfscope}%
\begin{pgfscope}%
\pgfpathrectangle{\pgfqpoint{0.648703in}{0.548769in}}{\pgfqpoint{5.112893in}{3.102590in}}%
\pgfusepath{clip}%
\pgfsetbuttcap%
\pgfsetroundjoin%
\definecolor{currentfill}{rgb}{1.000000,0.498039,0.054902}%
\pgfsetfillcolor{currentfill}%
\pgfsetlinewidth{1.003750pt}%
\definecolor{currentstroke}{rgb}{1.000000,0.498039,0.054902}%
\pgfsetstrokecolor{currentstroke}%
\pgfsetdash{}{0pt}%
\pgfpathmoveto{\pgfqpoint{1.717543in}{3.124394in}}%
\pgfpathcurveto{\pgfqpoint{1.728593in}{3.124394in}}{\pgfqpoint{1.739192in}{3.128784in}}{\pgfqpoint{1.747006in}{3.136598in}}%
\pgfpathcurveto{\pgfqpoint{1.754819in}{3.144411in}}{\pgfqpoint{1.759210in}{3.155010in}}{\pgfqpoint{1.759210in}{3.166060in}}%
\pgfpathcurveto{\pgfqpoint{1.759210in}{3.177111in}}{\pgfqpoint{1.754819in}{3.187710in}}{\pgfqpoint{1.747006in}{3.195523in}}%
\pgfpathcurveto{\pgfqpoint{1.739192in}{3.203337in}}{\pgfqpoint{1.728593in}{3.207727in}}{\pgfqpoint{1.717543in}{3.207727in}}%
\pgfpathcurveto{\pgfqpoint{1.706493in}{3.207727in}}{\pgfqpoint{1.695894in}{3.203337in}}{\pgfqpoint{1.688080in}{3.195523in}}%
\pgfpathcurveto{\pgfqpoint{1.680266in}{3.187710in}}{\pgfqpoint{1.675876in}{3.177111in}}{\pgfqpoint{1.675876in}{3.166060in}}%
\pgfpathcurveto{\pgfqpoint{1.675876in}{3.155010in}}{\pgfqpoint{1.680266in}{3.144411in}}{\pgfqpoint{1.688080in}{3.136598in}}%
\pgfpathcurveto{\pgfqpoint{1.695894in}{3.128784in}}{\pgfqpoint{1.706493in}{3.124394in}}{\pgfqpoint{1.717543in}{3.124394in}}%
\pgfpathclose%
\pgfusepath{stroke,fill}%
\end{pgfscope}%
\begin{pgfscope}%
\pgfpathrectangle{\pgfqpoint{0.648703in}{0.548769in}}{\pgfqpoint{5.112893in}{3.102590in}}%
\pgfusepath{clip}%
\pgfsetbuttcap%
\pgfsetroundjoin%
\definecolor{currentfill}{rgb}{1.000000,0.498039,0.054902}%
\pgfsetfillcolor{currentfill}%
\pgfsetlinewidth{1.003750pt}%
\definecolor{currentstroke}{rgb}{1.000000,0.498039,0.054902}%
\pgfsetstrokecolor{currentstroke}%
\pgfsetdash{}{0pt}%
\pgfpathmoveto{\pgfqpoint{3.100680in}{3.140985in}}%
\pgfpathcurveto{\pgfqpoint{3.111730in}{3.140985in}}{\pgfqpoint{3.122329in}{3.145375in}}{\pgfqpoint{3.130142in}{3.153189in}}%
\pgfpathcurveto{\pgfqpoint{3.137956in}{3.161003in}}{\pgfqpoint{3.142346in}{3.171602in}}{\pgfqpoint{3.142346in}{3.182652in}}%
\pgfpathcurveto{\pgfqpoint{3.142346in}{3.193702in}}{\pgfqpoint{3.137956in}{3.204301in}}{\pgfqpoint{3.130142in}{3.212115in}}%
\pgfpathcurveto{\pgfqpoint{3.122329in}{3.219928in}}{\pgfqpoint{3.111730in}{3.224319in}}{\pgfqpoint{3.100680in}{3.224319in}}%
\pgfpathcurveto{\pgfqpoint{3.089629in}{3.224319in}}{\pgfqpoint{3.079030in}{3.219928in}}{\pgfqpoint{3.071217in}{3.212115in}}%
\pgfpathcurveto{\pgfqpoint{3.063403in}{3.204301in}}{\pgfqpoint{3.059013in}{3.193702in}}{\pgfqpoint{3.059013in}{3.182652in}}%
\pgfpathcurveto{\pgfqpoint{3.059013in}{3.171602in}}{\pgfqpoint{3.063403in}{3.161003in}}{\pgfqpoint{3.071217in}{3.153189in}}%
\pgfpathcurveto{\pgfqpoint{3.079030in}{3.145375in}}{\pgfqpoint{3.089629in}{3.140985in}}{\pgfqpoint{3.100680in}{3.140985in}}%
\pgfpathclose%
\pgfusepath{stroke,fill}%
\end{pgfscope}%
\begin{pgfscope}%
\pgfpathrectangle{\pgfqpoint{0.648703in}{0.548769in}}{\pgfqpoint{5.112893in}{3.102590in}}%
\pgfusepath{clip}%
\pgfsetbuttcap%
\pgfsetroundjoin%
\definecolor{currentfill}{rgb}{1.000000,0.498039,0.054902}%
\pgfsetfillcolor{currentfill}%
\pgfsetlinewidth{1.003750pt}%
\definecolor{currentstroke}{rgb}{1.000000,0.498039,0.054902}%
\pgfsetstrokecolor{currentstroke}%
\pgfsetdash{}{0pt}%
\pgfpathmoveto{\pgfqpoint{2.798309in}{3.132690in}}%
\pgfpathcurveto{\pgfqpoint{2.809359in}{3.132690in}}{\pgfqpoint{2.819958in}{3.137080in}}{\pgfqpoint{2.827772in}{3.144893in}}%
\pgfpathcurveto{\pgfqpoint{2.835585in}{3.152707in}}{\pgfqpoint{2.839975in}{3.163306in}}{\pgfqpoint{2.839975in}{3.174356in}}%
\pgfpathcurveto{\pgfqpoint{2.839975in}{3.185406in}}{\pgfqpoint{2.835585in}{3.196005in}}{\pgfqpoint{2.827772in}{3.203819in}}%
\pgfpathcurveto{\pgfqpoint{2.819958in}{3.211633in}}{\pgfqpoint{2.809359in}{3.216023in}}{\pgfqpoint{2.798309in}{3.216023in}}%
\pgfpathcurveto{\pgfqpoint{2.787259in}{3.216023in}}{\pgfqpoint{2.776660in}{3.211633in}}{\pgfqpoint{2.768846in}{3.203819in}}%
\pgfpathcurveto{\pgfqpoint{2.761032in}{3.196005in}}{\pgfqpoint{2.756642in}{3.185406in}}{\pgfqpoint{2.756642in}{3.174356in}}%
\pgfpathcurveto{\pgfqpoint{2.756642in}{3.163306in}}{\pgfqpoint{2.761032in}{3.152707in}}{\pgfqpoint{2.768846in}{3.144893in}}%
\pgfpathcurveto{\pgfqpoint{2.776660in}{3.137080in}}{\pgfqpoint{2.787259in}{3.132690in}}{\pgfqpoint{2.798309in}{3.132690in}}%
\pgfpathclose%
\pgfusepath{stroke,fill}%
\end{pgfscope}%
\begin{pgfscope}%
\pgfpathrectangle{\pgfqpoint{0.648703in}{0.548769in}}{\pgfqpoint{5.112893in}{3.102590in}}%
\pgfusepath{clip}%
\pgfsetbuttcap%
\pgfsetroundjoin%
\definecolor{currentfill}{rgb}{1.000000,0.498039,0.054902}%
\pgfsetfillcolor{currentfill}%
\pgfsetlinewidth{1.003750pt}%
\definecolor{currentstroke}{rgb}{1.000000,0.498039,0.054902}%
\pgfsetstrokecolor{currentstroke}%
\pgfsetdash{}{0pt}%
\pgfpathmoveto{\pgfqpoint{3.537556in}{3.136837in}}%
\pgfpathcurveto{\pgfqpoint{3.548606in}{3.136837in}}{\pgfqpoint{3.559206in}{3.141228in}}{\pgfqpoint{3.567019in}{3.149041in}}%
\pgfpathcurveto{\pgfqpoint{3.574833in}{3.156855in}}{\pgfqpoint{3.579223in}{3.167454in}}{\pgfqpoint{3.579223in}{3.178504in}}%
\pgfpathcurveto{\pgfqpoint{3.579223in}{3.189554in}}{\pgfqpoint{3.574833in}{3.200153in}}{\pgfqpoint{3.567019in}{3.207967in}}%
\pgfpathcurveto{\pgfqpoint{3.559206in}{3.215780in}}{\pgfqpoint{3.548606in}{3.220171in}}{\pgfqpoint{3.537556in}{3.220171in}}%
\pgfpathcurveto{\pgfqpoint{3.526506in}{3.220171in}}{\pgfqpoint{3.515907in}{3.215780in}}{\pgfqpoint{3.508094in}{3.207967in}}%
\pgfpathcurveto{\pgfqpoint{3.500280in}{3.200153in}}{\pgfqpoint{3.495890in}{3.189554in}}{\pgfqpoint{3.495890in}{3.178504in}}%
\pgfpathcurveto{\pgfqpoint{3.495890in}{3.167454in}}{\pgfqpoint{3.500280in}{3.156855in}}{\pgfqpoint{3.508094in}{3.149041in}}%
\pgfpathcurveto{\pgfqpoint{3.515907in}{3.141228in}}{\pgfqpoint{3.526506in}{3.136837in}}{\pgfqpoint{3.537556in}{3.136837in}}%
\pgfpathclose%
\pgfusepath{stroke,fill}%
\end{pgfscope}%
\begin{pgfscope}%
\pgfpathrectangle{\pgfqpoint{0.648703in}{0.548769in}}{\pgfqpoint{5.112893in}{3.102590in}}%
\pgfusepath{clip}%
\pgfsetbuttcap%
\pgfsetroundjoin%
\definecolor{currentfill}{rgb}{1.000000,0.498039,0.054902}%
\pgfsetfillcolor{currentfill}%
\pgfsetlinewidth{1.003750pt}%
\definecolor{currentstroke}{rgb}{1.000000,0.498039,0.054902}%
\pgfsetstrokecolor{currentstroke}%
\pgfsetdash{}{0pt}%
\pgfpathmoveto{\pgfqpoint{2.320558in}{3.132690in}}%
\pgfpathcurveto{\pgfqpoint{2.331608in}{3.132690in}}{\pgfqpoint{2.342207in}{3.137080in}}{\pgfqpoint{2.350021in}{3.144893in}}%
\pgfpathcurveto{\pgfqpoint{2.357834in}{3.152707in}}{\pgfqpoint{2.362224in}{3.163306in}}{\pgfqpoint{2.362224in}{3.174356in}}%
\pgfpathcurveto{\pgfqpoint{2.362224in}{3.185406in}}{\pgfqpoint{2.357834in}{3.196005in}}{\pgfqpoint{2.350021in}{3.203819in}}%
\pgfpathcurveto{\pgfqpoint{2.342207in}{3.211633in}}{\pgfqpoint{2.331608in}{3.216023in}}{\pgfqpoint{2.320558in}{3.216023in}}%
\pgfpathcurveto{\pgfqpoint{2.309508in}{3.216023in}}{\pgfqpoint{2.298909in}{3.211633in}}{\pgfqpoint{2.291095in}{3.203819in}}%
\pgfpathcurveto{\pgfqpoint{2.283281in}{3.196005in}}{\pgfqpoint{2.278891in}{3.185406in}}{\pgfqpoint{2.278891in}{3.174356in}}%
\pgfpathcurveto{\pgfqpoint{2.278891in}{3.163306in}}{\pgfqpoint{2.283281in}{3.152707in}}{\pgfqpoint{2.291095in}{3.144893in}}%
\pgfpathcurveto{\pgfqpoint{2.298909in}{3.137080in}}{\pgfqpoint{2.309508in}{3.132690in}}{\pgfqpoint{2.320558in}{3.132690in}}%
\pgfpathclose%
\pgfusepath{stroke,fill}%
\end{pgfscope}%
\begin{pgfscope}%
\pgfpathrectangle{\pgfqpoint{0.648703in}{0.548769in}}{\pgfqpoint{5.112893in}{3.102590in}}%
\pgfusepath{clip}%
\pgfsetbuttcap%
\pgfsetroundjoin%
\definecolor{currentfill}{rgb}{1.000000,0.498039,0.054902}%
\pgfsetfillcolor{currentfill}%
\pgfsetlinewidth{1.003750pt}%
\definecolor{currentstroke}{rgb}{1.000000,0.498039,0.054902}%
\pgfsetstrokecolor{currentstroke}%
\pgfsetdash{}{0pt}%
\pgfpathmoveto{\pgfqpoint{2.820088in}{3.128542in}}%
\pgfpathcurveto{\pgfqpoint{2.831138in}{3.128542in}}{\pgfqpoint{2.841737in}{3.132932in}}{\pgfqpoint{2.849550in}{3.140746in}}%
\pgfpathcurveto{\pgfqpoint{2.857364in}{3.148559in}}{\pgfqpoint{2.861754in}{3.159158in}}{\pgfqpoint{2.861754in}{3.170208in}}%
\pgfpathcurveto{\pgfqpoint{2.861754in}{3.181258in}}{\pgfqpoint{2.857364in}{3.191857in}}{\pgfqpoint{2.849550in}{3.199671in}}%
\pgfpathcurveto{\pgfqpoint{2.841737in}{3.207485in}}{\pgfqpoint{2.831138in}{3.211875in}}{\pgfqpoint{2.820088in}{3.211875in}}%
\pgfpathcurveto{\pgfqpoint{2.809037in}{3.211875in}}{\pgfqpoint{2.798438in}{3.207485in}}{\pgfqpoint{2.790625in}{3.199671in}}%
\pgfpathcurveto{\pgfqpoint{2.782811in}{3.191857in}}{\pgfqpoint{2.778421in}{3.181258in}}{\pgfqpoint{2.778421in}{3.170208in}}%
\pgfpathcurveto{\pgfqpoint{2.778421in}{3.159158in}}{\pgfqpoint{2.782811in}{3.148559in}}{\pgfqpoint{2.790625in}{3.140746in}}%
\pgfpathcurveto{\pgfqpoint{2.798438in}{3.132932in}}{\pgfqpoint{2.809037in}{3.128542in}}{\pgfqpoint{2.820088in}{3.128542in}}%
\pgfpathclose%
\pgfusepath{stroke,fill}%
\end{pgfscope}%
\begin{pgfscope}%
\pgfpathrectangle{\pgfqpoint{0.648703in}{0.548769in}}{\pgfqpoint{5.112893in}{3.102590in}}%
\pgfusepath{clip}%
\pgfsetbuttcap%
\pgfsetroundjoin%
\definecolor{currentfill}{rgb}{1.000000,0.498039,0.054902}%
\pgfsetfillcolor{currentfill}%
\pgfsetlinewidth{1.003750pt}%
\definecolor{currentstroke}{rgb}{1.000000,0.498039,0.054902}%
\pgfsetstrokecolor{currentstroke}%
\pgfsetdash{}{0pt}%
\pgfpathmoveto{\pgfqpoint{3.321379in}{3.149281in}}%
\pgfpathcurveto{\pgfqpoint{3.332429in}{3.149281in}}{\pgfqpoint{3.343028in}{3.153671in}}{\pgfqpoint{3.350841in}{3.161485in}}%
\pgfpathcurveto{\pgfqpoint{3.358655in}{3.169298in}}{\pgfqpoint{3.363045in}{3.179897in}}{\pgfqpoint{3.363045in}{3.190948in}}%
\pgfpathcurveto{\pgfqpoint{3.363045in}{3.201998in}}{\pgfqpoint{3.358655in}{3.212597in}}{\pgfqpoint{3.350841in}{3.220410in}}%
\pgfpathcurveto{\pgfqpoint{3.343028in}{3.228224in}}{\pgfqpoint{3.332429in}{3.232614in}}{\pgfqpoint{3.321379in}{3.232614in}}%
\pgfpathcurveto{\pgfqpoint{3.310328in}{3.232614in}}{\pgfqpoint{3.299729in}{3.228224in}}{\pgfqpoint{3.291916in}{3.220410in}}%
\pgfpathcurveto{\pgfqpoint{3.284102in}{3.212597in}}{\pgfqpoint{3.279712in}{3.201998in}}{\pgfqpoint{3.279712in}{3.190948in}}%
\pgfpathcurveto{\pgfqpoint{3.279712in}{3.179897in}}{\pgfqpoint{3.284102in}{3.169298in}}{\pgfqpoint{3.291916in}{3.161485in}}%
\pgfpathcurveto{\pgfqpoint{3.299729in}{3.153671in}}{\pgfqpoint{3.310328in}{3.149281in}}{\pgfqpoint{3.321379in}{3.149281in}}%
\pgfpathclose%
\pgfusepath{stroke,fill}%
\end{pgfscope}%
\begin{pgfscope}%
\pgfpathrectangle{\pgfqpoint{0.648703in}{0.548769in}}{\pgfqpoint{5.112893in}{3.102590in}}%
\pgfusepath{clip}%
\pgfsetbuttcap%
\pgfsetroundjoin%
\definecolor{currentfill}{rgb}{1.000000,0.498039,0.054902}%
\pgfsetfillcolor{currentfill}%
\pgfsetlinewidth{1.003750pt}%
\definecolor{currentstroke}{rgb}{1.000000,0.498039,0.054902}%
\pgfsetstrokecolor{currentstroke}%
\pgfsetdash{}{0pt}%
\pgfpathmoveto{\pgfqpoint{3.355309in}{3.240534in}}%
\pgfpathcurveto{\pgfqpoint{3.366359in}{3.240534in}}{\pgfqpoint{3.376958in}{3.244924in}}{\pgfqpoint{3.384772in}{3.252737in}}%
\pgfpathcurveto{\pgfqpoint{3.392586in}{3.260551in}}{\pgfqpoint{3.396976in}{3.271150in}}{\pgfqpoint{3.396976in}{3.282200in}}%
\pgfpathcurveto{\pgfqpoint{3.396976in}{3.293250in}}{\pgfqpoint{3.392586in}{3.303849in}}{\pgfqpoint{3.384772in}{3.311663in}}%
\pgfpathcurveto{\pgfqpoint{3.376958in}{3.319477in}}{\pgfqpoint{3.366359in}{3.323867in}}{\pgfqpoint{3.355309in}{3.323867in}}%
\pgfpathcurveto{\pgfqpoint{3.344259in}{3.323867in}}{\pgfqpoint{3.333660in}{3.319477in}}{\pgfqpoint{3.325846in}{3.311663in}}%
\pgfpathcurveto{\pgfqpoint{3.318033in}{3.303849in}}{\pgfqpoint{3.313642in}{3.293250in}}{\pgfqpoint{3.313642in}{3.282200in}}%
\pgfpathcurveto{\pgfqpoint{3.313642in}{3.271150in}}{\pgfqpoint{3.318033in}{3.260551in}}{\pgfqpoint{3.325846in}{3.252737in}}%
\pgfpathcurveto{\pgfqpoint{3.333660in}{3.244924in}}{\pgfqpoint{3.344259in}{3.240534in}}{\pgfqpoint{3.355309in}{3.240534in}}%
\pgfpathclose%
\pgfusepath{stroke,fill}%
\end{pgfscope}%
\begin{pgfscope}%
\pgfpathrectangle{\pgfqpoint{0.648703in}{0.548769in}}{\pgfqpoint{5.112893in}{3.102590in}}%
\pgfusepath{clip}%
\pgfsetbuttcap%
\pgfsetroundjoin%
\definecolor{currentfill}{rgb}{0.121569,0.466667,0.705882}%
\pgfsetfillcolor{currentfill}%
\pgfsetlinewidth{1.003750pt}%
\definecolor{currentstroke}{rgb}{0.121569,0.466667,0.705882}%
\pgfsetstrokecolor{currentstroke}%
\pgfsetdash{}{0pt}%
\pgfpathmoveto{\pgfqpoint{0.843973in}{0.664720in}}%
\pgfpathcurveto{\pgfqpoint{0.855023in}{0.664720in}}{\pgfqpoint{0.865622in}{0.669111in}}{\pgfqpoint{0.873435in}{0.676924in}}%
\pgfpathcurveto{\pgfqpoint{0.881249in}{0.684738in}}{\pgfqpoint{0.885639in}{0.695337in}}{\pgfqpoint{0.885639in}{0.706387in}}%
\pgfpathcurveto{\pgfqpoint{0.885639in}{0.717437in}}{\pgfqpoint{0.881249in}{0.728036in}}{\pgfqpoint{0.873435in}{0.735850in}}%
\pgfpathcurveto{\pgfqpoint{0.865622in}{0.743663in}}{\pgfqpoint{0.855023in}{0.748054in}}{\pgfqpoint{0.843973in}{0.748054in}}%
\pgfpathcurveto{\pgfqpoint{0.832922in}{0.748054in}}{\pgfqpoint{0.822323in}{0.743663in}}{\pgfqpoint{0.814510in}{0.735850in}}%
\pgfpathcurveto{\pgfqpoint{0.806696in}{0.728036in}}{\pgfqpoint{0.802306in}{0.717437in}}{\pgfqpoint{0.802306in}{0.706387in}}%
\pgfpathcurveto{\pgfqpoint{0.802306in}{0.695337in}}{\pgfqpoint{0.806696in}{0.684738in}}{\pgfqpoint{0.814510in}{0.676924in}}%
\pgfpathcurveto{\pgfqpoint{0.822323in}{0.669111in}}{\pgfqpoint{0.832922in}{0.664720in}}{\pgfqpoint{0.843973in}{0.664720in}}%
\pgfpathclose%
\pgfusepath{stroke,fill}%
\end{pgfscope}%
\begin{pgfscope}%
\pgfpathrectangle{\pgfqpoint{0.648703in}{0.548769in}}{\pgfqpoint{5.112893in}{3.102590in}}%
\pgfusepath{clip}%
\pgfsetbuttcap%
\pgfsetroundjoin%
\definecolor{currentfill}{rgb}{1.000000,0.498039,0.054902}%
\pgfsetfillcolor{currentfill}%
\pgfsetlinewidth{1.003750pt}%
\definecolor{currentstroke}{rgb}{1.000000,0.498039,0.054902}%
\pgfsetstrokecolor{currentstroke}%
\pgfsetdash{}{0pt}%
\pgfpathmoveto{\pgfqpoint{3.546616in}{3.157577in}}%
\pgfpathcurveto{\pgfqpoint{3.557666in}{3.157577in}}{\pgfqpoint{3.568265in}{3.161967in}}{\pgfqpoint{3.576079in}{3.169780in}}%
\pgfpathcurveto{\pgfqpoint{3.583893in}{3.177594in}}{\pgfqpoint{3.588283in}{3.188193in}}{\pgfqpoint{3.588283in}{3.199243in}}%
\pgfpathcurveto{\pgfqpoint{3.588283in}{3.210293in}}{\pgfqpoint{3.583893in}{3.220892in}}{\pgfqpoint{3.576079in}{3.228706in}}%
\pgfpathcurveto{\pgfqpoint{3.568265in}{3.236520in}}{\pgfqpoint{3.557666in}{3.240910in}}{\pgfqpoint{3.546616in}{3.240910in}}%
\pgfpathcurveto{\pgfqpoint{3.535566in}{3.240910in}}{\pgfqpoint{3.524967in}{3.236520in}}{\pgfqpoint{3.517153in}{3.228706in}}%
\pgfpathcurveto{\pgfqpoint{3.509340in}{3.220892in}}{\pgfqpoint{3.504950in}{3.210293in}}{\pgfqpoint{3.504950in}{3.199243in}}%
\pgfpathcurveto{\pgfqpoint{3.504950in}{3.188193in}}{\pgfqpoint{3.509340in}{3.177594in}}{\pgfqpoint{3.517153in}{3.169780in}}%
\pgfpathcurveto{\pgfqpoint{3.524967in}{3.161967in}}{\pgfqpoint{3.535566in}{3.157577in}}{\pgfqpoint{3.546616in}{3.157577in}}%
\pgfpathclose%
\pgfusepath{stroke,fill}%
\end{pgfscope}%
\begin{pgfscope}%
\pgfpathrectangle{\pgfqpoint{0.648703in}{0.548769in}}{\pgfqpoint{5.112893in}{3.102590in}}%
\pgfusepath{clip}%
\pgfsetbuttcap%
\pgfsetroundjoin%
\definecolor{currentfill}{rgb}{1.000000,0.498039,0.054902}%
\pgfsetfillcolor{currentfill}%
\pgfsetlinewidth{1.003750pt}%
\definecolor{currentstroke}{rgb}{1.000000,0.498039,0.054902}%
\pgfsetstrokecolor{currentstroke}%
\pgfsetdash{}{0pt}%
\pgfpathmoveto{\pgfqpoint{3.132889in}{3.140985in}}%
\pgfpathcurveto{\pgfqpoint{3.143939in}{3.140985in}}{\pgfqpoint{3.154538in}{3.145375in}}{\pgfqpoint{3.162351in}{3.153189in}}%
\pgfpathcurveto{\pgfqpoint{3.170165in}{3.161003in}}{\pgfqpoint{3.174555in}{3.171602in}}{\pgfqpoint{3.174555in}{3.182652in}}%
\pgfpathcurveto{\pgfqpoint{3.174555in}{3.193702in}}{\pgfqpoint{3.170165in}{3.204301in}}{\pgfqpoint{3.162351in}{3.212115in}}%
\pgfpathcurveto{\pgfqpoint{3.154538in}{3.219928in}}{\pgfqpoint{3.143939in}{3.224319in}}{\pgfqpoint{3.132889in}{3.224319in}}%
\pgfpathcurveto{\pgfqpoint{3.121839in}{3.224319in}}{\pgfqpoint{3.111240in}{3.219928in}}{\pgfqpoint{3.103426in}{3.212115in}}%
\pgfpathcurveto{\pgfqpoint{3.095612in}{3.204301in}}{\pgfqpoint{3.091222in}{3.193702in}}{\pgfqpoint{3.091222in}{3.182652in}}%
\pgfpathcurveto{\pgfqpoint{3.091222in}{3.171602in}}{\pgfqpoint{3.095612in}{3.161003in}}{\pgfqpoint{3.103426in}{3.153189in}}%
\pgfpathcurveto{\pgfqpoint{3.111240in}{3.145375in}}{\pgfqpoint{3.121839in}{3.140985in}}{\pgfqpoint{3.132889in}{3.140985in}}%
\pgfpathclose%
\pgfusepath{stroke,fill}%
\end{pgfscope}%
\begin{pgfscope}%
\pgfpathrectangle{\pgfqpoint{0.648703in}{0.548769in}}{\pgfqpoint{5.112893in}{3.102590in}}%
\pgfusepath{clip}%
\pgfsetbuttcap%
\pgfsetroundjoin%
\definecolor{currentfill}{rgb}{1.000000,0.498039,0.054902}%
\pgfsetfillcolor{currentfill}%
\pgfsetlinewidth{1.003750pt}%
\definecolor{currentstroke}{rgb}{1.000000,0.498039,0.054902}%
\pgfsetstrokecolor{currentstroke}%
\pgfsetdash{}{0pt}%
\pgfpathmoveto{\pgfqpoint{2.982199in}{3.136837in}}%
\pgfpathcurveto{\pgfqpoint{2.993249in}{3.136837in}}{\pgfqpoint{3.003848in}{3.141228in}}{\pgfqpoint{3.011661in}{3.149041in}}%
\pgfpathcurveto{\pgfqpoint{3.019475in}{3.156855in}}{\pgfqpoint{3.023865in}{3.167454in}}{\pgfqpoint{3.023865in}{3.178504in}}%
\pgfpathcurveto{\pgfqpoint{3.023865in}{3.189554in}}{\pgfqpoint{3.019475in}{3.200153in}}{\pgfqpoint{3.011661in}{3.207967in}}%
\pgfpathcurveto{\pgfqpoint{3.003848in}{3.215780in}}{\pgfqpoint{2.993249in}{3.220171in}}{\pgfqpoint{2.982199in}{3.220171in}}%
\pgfpathcurveto{\pgfqpoint{2.971148in}{3.220171in}}{\pgfqpoint{2.960549in}{3.215780in}}{\pgfqpoint{2.952736in}{3.207967in}}%
\pgfpathcurveto{\pgfqpoint{2.944922in}{3.200153in}}{\pgfqpoint{2.940532in}{3.189554in}}{\pgfqpoint{2.940532in}{3.178504in}}%
\pgfpathcurveto{\pgfqpoint{2.940532in}{3.167454in}}{\pgfqpoint{2.944922in}{3.156855in}}{\pgfqpoint{2.952736in}{3.149041in}}%
\pgfpathcurveto{\pgfqpoint{2.960549in}{3.141228in}}{\pgfqpoint{2.971148in}{3.136837in}}{\pgfqpoint{2.982199in}{3.136837in}}%
\pgfpathclose%
\pgfusepath{stroke,fill}%
\end{pgfscope}%
\begin{pgfscope}%
\pgfpathrectangle{\pgfqpoint{0.648703in}{0.548769in}}{\pgfqpoint{5.112893in}{3.102590in}}%
\pgfusepath{clip}%
\pgfsetbuttcap%
\pgfsetroundjoin%
\definecolor{currentfill}{rgb}{1.000000,0.498039,0.054902}%
\pgfsetfillcolor{currentfill}%
\pgfsetlinewidth{1.003750pt}%
\definecolor{currentstroke}{rgb}{1.000000,0.498039,0.054902}%
\pgfsetstrokecolor{currentstroke}%
\pgfsetdash{}{0pt}%
\pgfpathmoveto{\pgfqpoint{3.312311in}{3.136837in}}%
\pgfpathcurveto{\pgfqpoint{3.323361in}{3.136837in}}{\pgfqpoint{3.333960in}{3.141228in}}{\pgfqpoint{3.341774in}{3.149041in}}%
\pgfpathcurveto{\pgfqpoint{3.349588in}{3.156855in}}{\pgfqpoint{3.353978in}{3.167454in}}{\pgfqpoint{3.353978in}{3.178504in}}%
\pgfpathcurveto{\pgfqpoint{3.353978in}{3.189554in}}{\pgfqpoint{3.349588in}{3.200153in}}{\pgfqpoint{3.341774in}{3.207967in}}%
\pgfpathcurveto{\pgfqpoint{3.333960in}{3.215780in}}{\pgfqpoint{3.323361in}{3.220171in}}{\pgfqpoint{3.312311in}{3.220171in}}%
\pgfpathcurveto{\pgfqpoint{3.301261in}{3.220171in}}{\pgfqpoint{3.290662in}{3.215780in}}{\pgfqpoint{3.282849in}{3.207967in}}%
\pgfpathcurveto{\pgfqpoint{3.275035in}{3.200153in}}{\pgfqpoint{3.270645in}{3.189554in}}{\pgfqpoint{3.270645in}{3.178504in}}%
\pgfpathcurveto{\pgfqpoint{3.270645in}{3.167454in}}{\pgfqpoint{3.275035in}{3.156855in}}{\pgfqpoint{3.282849in}{3.149041in}}%
\pgfpathcurveto{\pgfqpoint{3.290662in}{3.141228in}}{\pgfqpoint{3.301261in}{3.136837in}}{\pgfqpoint{3.312311in}{3.136837in}}%
\pgfpathclose%
\pgfusepath{stroke,fill}%
\end{pgfscope}%
\begin{pgfscope}%
\pgfpathrectangle{\pgfqpoint{0.648703in}{0.548769in}}{\pgfqpoint{5.112893in}{3.102590in}}%
\pgfusepath{clip}%
\pgfsetbuttcap%
\pgfsetroundjoin%
\definecolor{currentfill}{rgb}{1.000000,0.498039,0.054902}%
\pgfsetfillcolor{currentfill}%
\pgfsetlinewidth{1.003750pt}%
\definecolor{currentstroke}{rgb}{1.000000,0.498039,0.054902}%
\pgfsetstrokecolor{currentstroke}%
\pgfsetdash{}{0pt}%
\pgfpathmoveto{\pgfqpoint{1.450783in}{3.136837in}}%
\pgfpathcurveto{\pgfqpoint{1.461833in}{3.136837in}}{\pgfqpoint{1.472432in}{3.141228in}}{\pgfqpoint{1.480246in}{3.149041in}}%
\pgfpathcurveto{\pgfqpoint{1.488059in}{3.156855in}}{\pgfqpoint{1.492450in}{3.167454in}}{\pgfqpoint{1.492450in}{3.178504in}}%
\pgfpathcurveto{\pgfqpoint{1.492450in}{3.189554in}}{\pgfqpoint{1.488059in}{3.200153in}}{\pgfqpoint{1.480246in}{3.207967in}}%
\pgfpathcurveto{\pgfqpoint{1.472432in}{3.215780in}}{\pgfqpoint{1.461833in}{3.220171in}}{\pgfqpoint{1.450783in}{3.220171in}}%
\pgfpathcurveto{\pgfqpoint{1.439733in}{3.220171in}}{\pgfqpoint{1.429134in}{3.215780in}}{\pgfqpoint{1.421320in}{3.207967in}}%
\pgfpathcurveto{\pgfqpoint{1.413507in}{3.200153in}}{\pgfqpoint{1.409116in}{3.189554in}}{\pgfqpoint{1.409116in}{3.178504in}}%
\pgfpathcurveto{\pgfqpoint{1.409116in}{3.167454in}}{\pgfqpoint{1.413507in}{3.156855in}}{\pgfqpoint{1.421320in}{3.149041in}}%
\pgfpathcurveto{\pgfqpoint{1.429134in}{3.141228in}}{\pgfqpoint{1.439733in}{3.136837in}}{\pgfqpoint{1.450783in}{3.136837in}}%
\pgfpathclose%
\pgfusepath{stroke,fill}%
\end{pgfscope}%
\begin{pgfscope}%
\pgfpathrectangle{\pgfqpoint{0.648703in}{0.548769in}}{\pgfqpoint{5.112893in}{3.102590in}}%
\pgfusepath{clip}%
\pgfsetbuttcap%
\pgfsetroundjoin%
\definecolor{currentfill}{rgb}{1.000000,0.498039,0.054902}%
\pgfsetfillcolor{currentfill}%
\pgfsetlinewidth{1.003750pt}%
\definecolor{currentstroke}{rgb}{1.000000,0.498039,0.054902}%
\pgfsetstrokecolor{currentstroke}%
\pgfsetdash{}{0pt}%
\pgfpathmoveto{\pgfqpoint{2.457815in}{3.136837in}}%
\pgfpathcurveto{\pgfqpoint{2.468865in}{3.136837in}}{\pgfqpoint{2.479464in}{3.141228in}}{\pgfqpoint{2.487277in}{3.149041in}}%
\pgfpathcurveto{\pgfqpoint{2.495091in}{3.156855in}}{\pgfqpoint{2.499481in}{3.167454in}}{\pgfqpoint{2.499481in}{3.178504in}}%
\pgfpathcurveto{\pgfqpoint{2.499481in}{3.189554in}}{\pgfqpoint{2.495091in}{3.200153in}}{\pgfqpoint{2.487277in}{3.207967in}}%
\pgfpathcurveto{\pgfqpoint{2.479464in}{3.215780in}}{\pgfqpoint{2.468865in}{3.220171in}}{\pgfqpoint{2.457815in}{3.220171in}}%
\pgfpathcurveto{\pgfqpoint{2.446764in}{3.220171in}}{\pgfqpoint{2.436165in}{3.215780in}}{\pgfqpoint{2.428352in}{3.207967in}}%
\pgfpathcurveto{\pgfqpoint{2.420538in}{3.200153in}}{\pgfqpoint{2.416148in}{3.189554in}}{\pgfqpoint{2.416148in}{3.178504in}}%
\pgfpathcurveto{\pgfqpoint{2.416148in}{3.167454in}}{\pgfqpoint{2.420538in}{3.156855in}}{\pgfqpoint{2.428352in}{3.149041in}}%
\pgfpathcurveto{\pgfqpoint{2.436165in}{3.141228in}}{\pgfqpoint{2.446764in}{3.136837in}}{\pgfqpoint{2.457815in}{3.136837in}}%
\pgfpathclose%
\pgfusepath{stroke,fill}%
\end{pgfscope}%
\begin{pgfscope}%
\pgfpathrectangle{\pgfqpoint{0.648703in}{0.548769in}}{\pgfqpoint{5.112893in}{3.102590in}}%
\pgfusepath{clip}%
\pgfsetbuttcap%
\pgfsetroundjoin%
\definecolor{currentfill}{rgb}{1.000000,0.498039,0.054902}%
\pgfsetfillcolor{currentfill}%
\pgfsetlinewidth{1.003750pt}%
\definecolor{currentstroke}{rgb}{1.000000,0.498039,0.054902}%
\pgfsetstrokecolor{currentstroke}%
\pgfsetdash{}{0pt}%
\pgfpathmoveto{\pgfqpoint{3.559101in}{3.136837in}}%
\pgfpathcurveto{\pgfqpoint{3.570152in}{3.136837in}}{\pgfqpoint{3.580751in}{3.141228in}}{\pgfqpoint{3.588564in}{3.149041in}}%
\pgfpathcurveto{\pgfqpoint{3.596378in}{3.156855in}}{\pgfqpoint{3.600768in}{3.167454in}}{\pgfqpoint{3.600768in}{3.178504in}}%
\pgfpathcurveto{\pgfqpoint{3.600768in}{3.189554in}}{\pgfqpoint{3.596378in}{3.200153in}}{\pgfqpoint{3.588564in}{3.207967in}}%
\pgfpathcurveto{\pgfqpoint{3.580751in}{3.215780in}}{\pgfqpoint{3.570152in}{3.220171in}}{\pgfqpoint{3.559101in}{3.220171in}}%
\pgfpathcurveto{\pgfqpoint{3.548051in}{3.220171in}}{\pgfqpoint{3.537452in}{3.215780in}}{\pgfqpoint{3.529639in}{3.207967in}}%
\pgfpathcurveto{\pgfqpoint{3.521825in}{3.200153in}}{\pgfqpoint{3.517435in}{3.189554in}}{\pgfqpoint{3.517435in}{3.178504in}}%
\pgfpathcurveto{\pgfqpoint{3.517435in}{3.167454in}}{\pgfqpoint{3.521825in}{3.156855in}}{\pgfqpoint{3.529639in}{3.149041in}}%
\pgfpathcurveto{\pgfqpoint{3.537452in}{3.141228in}}{\pgfqpoint{3.548051in}{3.136837in}}{\pgfqpoint{3.559101in}{3.136837in}}%
\pgfpathclose%
\pgfusepath{stroke,fill}%
\end{pgfscope}%
\begin{pgfscope}%
\pgfpathrectangle{\pgfqpoint{0.648703in}{0.548769in}}{\pgfqpoint{5.112893in}{3.102590in}}%
\pgfusepath{clip}%
\pgfsetbuttcap%
\pgfsetroundjoin%
\definecolor{currentfill}{rgb}{0.121569,0.466667,0.705882}%
\pgfsetfillcolor{currentfill}%
\pgfsetlinewidth{1.003750pt}%
\definecolor{currentstroke}{rgb}{0.121569,0.466667,0.705882}%
\pgfsetstrokecolor{currentstroke}%
\pgfsetdash{}{0pt}%
\pgfpathmoveto{\pgfqpoint{3.470921in}{2.846488in}}%
\pgfpathcurveto{\pgfqpoint{3.481971in}{2.846488in}}{\pgfqpoint{3.492570in}{2.850878in}}{\pgfqpoint{3.500384in}{2.858692in}}%
\pgfpathcurveto{\pgfqpoint{3.508197in}{2.866506in}}{\pgfqpoint{3.512587in}{2.877105in}}{\pgfqpoint{3.512587in}{2.888155in}}%
\pgfpathcurveto{\pgfqpoint{3.512587in}{2.899205in}}{\pgfqpoint{3.508197in}{2.909804in}}{\pgfqpoint{3.500384in}{2.917617in}}%
\pgfpathcurveto{\pgfqpoint{3.492570in}{2.925431in}}{\pgfqpoint{3.481971in}{2.929821in}}{\pgfqpoint{3.470921in}{2.929821in}}%
\pgfpathcurveto{\pgfqpoint{3.459871in}{2.929821in}}{\pgfqpoint{3.449272in}{2.925431in}}{\pgfqpoint{3.441458in}{2.917617in}}%
\pgfpathcurveto{\pgfqpoint{3.433644in}{2.909804in}}{\pgfqpoint{3.429254in}{2.899205in}}{\pgfqpoint{3.429254in}{2.888155in}}%
\pgfpathcurveto{\pgfqpoint{3.429254in}{2.877105in}}{\pgfqpoint{3.433644in}{2.866506in}}{\pgfqpoint{3.441458in}{2.858692in}}%
\pgfpathcurveto{\pgfqpoint{3.449272in}{2.850878in}}{\pgfqpoint{3.459871in}{2.846488in}}{\pgfqpoint{3.470921in}{2.846488in}}%
\pgfpathclose%
\pgfusepath{stroke,fill}%
\end{pgfscope}%
\begin{pgfscope}%
\pgfpathrectangle{\pgfqpoint{0.648703in}{0.548769in}}{\pgfqpoint{5.112893in}{3.102590in}}%
\pgfusepath{clip}%
\pgfsetbuttcap%
\pgfsetroundjoin%
\definecolor{currentfill}{rgb}{0.121569,0.466667,0.705882}%
\pgfsetfillcolor{currentfill}%
\pgfsetlinewidth{1.003750pt}%
\definecolor{currentstroke}{rgb}{0.121569,0.466667,0.705882}%
\pgfsetstrokecolor{currentstroke}%
\pgfsetdash{}{0pt}%
\pgfpathmoveto{\pgfqpoint{1.211704in}{3.074620in}}%
\pgfpathcurveto{\pgfqpoint{1.222754in}{3.074620in}}{\pgfqpoint{1.233353in}{3.079010in}}{\pgfqpoint{1.241166in}{3.086824in}}%
\pgfpathcurveto{\pgfqpoint{1.248980in}{3.094637in}}{\pgfqpoint{1.253370in}{3.105236in}}{\pgfqpoint{1.253370in}{3.116286in}}%
\pgfpathcurveto{\pgfqpoint{1.253370in}{3.127336in}}{\pgfqpoint{1.248980in}{3.137935in}}{\pgfqpoint{1.241166in}{3.145749in}}%
\pgfpathcurveto{\pgfqpoint{1.233353in}{3.153563in}}{\pgfqpoint{1.222754in}{3.157953in}}{\pgfqpoint{1.211704in}{3.157953in}}%
\pgfpathcurveto{\pgfqpoint{1.200654in}{3.157953in}}{\pgfqpoint{1.190055in}{3.153563in}}{\pgfqpoint{1.182241in}{3.145749in}}%
\pgfpathcurveto{\pgfqpoint{1.174427in}{3.137935in}}{\pgfqpoint{1.170037in}{3.127336in}}{\pgfqpoint{1.170037in}{3.116286in}}%
\pgfpathcurveto{\pgfqpoint{1.170037in}{3.105236in}}{\pgfqpoint{1.174427in}{3.094637in}}{\pgfqpoint{1.182241in}{3.086824in}}%
\pgfpathcurveto{\pgfqpoint{1.190055in}{3.079010in}}{\pgfqpoint{1.200654in}{3.074620in}}{\pgfqpoint{1.211704in}{3.074620in}}%
\pgfpathclose%
\pgfusepath{stroke,fill}%
\end{pgfscope}%
\begin{pgfscope}%
\pgfpathrectangle{\pgfqpoint{0.648703in}{0.548769in}}{\pgfqpoint{5.112893in}{3.102590in}}%
\pgfusepath{clip}%
\pgfsetbuttcap%
\pgfsetroundjoin%
\definecolor{currentfill}{rgb}{1.000000,0.498039,0.054902}%
\pgfsetfillcolor{currentfill}%
\pgfsetlinewidth{1.003750pt}%
\definecolor{currentstroke}{rgb}{1.000000,0.498039,0.054902}%
\pgfsetstrokecolor{currentstroke}%
\pgfsetdash{}{0pt}%
\pgfpathmoveto{\pgfqpoint{3.375721in}{3.315195in}}%
\pgfpathcurveto{\pgfqpoint{3.386771in}{3.315195in}}{\pgfqpoint{3.397370in}{3.319585in}}{\pgfqpoint{3.405184in}{3.327399in}}%
\pgfpathcurveto{\pgfqpoint{3.412997in}{3.335212in}}{\pgfqpoint{3.417388in}{3.345811in}}{\pgfqpoint{3.417388in}{3.356861in}}%
\pgfpathcurveto{\pgfqpoint{3.417388in}{3.367912in}}{\pgfqpoint{3.412997in}{3.378511in}}{\pgfqpoint{3.405184in}{3.386324in}}%
\pgfpathcurveto{\pgfqpoint{3.397370in}{3.394138in}}{\pgfqpoint{3.386771in}{3.398528in}}{\pgfqpoint{3.375721in}{3.398528in}}%
\pgfpathcurveto{\pgfqpoint{3.364671in}{3.398528in}}{\pgfqpoint{3.354072in}{3.394138in}}{\pgfqpoint{3.346258in}{3.386324in}}%
\pgfpathcurveto{\pgfqpoint{3.338444in}{3.378511in}}{\pgfqpoint{3.334054in}{3.367912in}}{\pgfqpoint{3.334054in}{3.356861in}}%
\pgfpathcurveto{\pgfqpoint{3.334054in}{3.345811in}}{\pgfqpoint{3.338444in}{3.335212in}}{\pgfqpoint{3.346258in}{3.327399in}}%
\pgfpathcurveto{\pgfqpoint{3.354072in}{3.319585in}}{\pgfqpoint{3.364671in}{3.315195in}}{\pgfqpoint{3.375721in}{3.315195in}}%
\pgfpathclose%
\pgfusepath{stroke,fill}%
\end{pgfscope}%
\begin{pgfscope}%
\pgfpathrectangle{\pgfqpoint{0.648703in}{0.548769in}}{\pgfqpoint{5.112893in}{3.102590in}}%
\pgfusepath{clip}%
\pgfsetbuttcap%
\pgfsetroundjoin%
\definecolor{currentfill}{rgb}{0.121569,0.466667,0.705882}%
\pgfsetfillcolor{currentfill}%
\pgfsetlinewidth{1.003750pt}%
\definecolor{currentstroke}{rgb}{0.121569,0.466667,0.705882}%
\pgfsetstrokecolor{currentstroke}%
\pgfsetdash{}{0pt}%
\pgfpathmoveto{\pgfqpoint{0.814449in}{0.648129in}}%
\pgfpathcurveto{\pgfqpoint{0.825499in}{0.648129in}}{\pgfqpoint{0.836098in}{0.652519in}}{\pgfqpoint{0.843912in}{0.660333in}}%
\pgfpathcurveto{\pgfqpoint{0.851726in}{0.668146in}}{\pgfqpoint{0.856116in}{0.678745in}}{\pgfqpoint{0.856116in}{0.689796in}}%
\pgfpathcurveto{\pgfqpoint{0.856116in}{0.700846in}}{\pgfqpoint{0.851726in}{0.711445in}}{\pgfqpoint{0.843912in}{0.719258in}}%
\pgfpathcurveto{\pgfqpoint{0.836098in}{0.727072in}}{\pgfqpoint{0.825499in}{0.731462in}}{\pgfqpoint{0.814449in}{0.731462in}}%
\pgfpathcurveto{\pgfqpoint{0.803399in}{0.731462in}}{\pgfqpoint{0.792800in}{0.727072in}}{\pgfqpoint{0.784987in}{0.719258in}}%
\pgfpathcurveto{\pgfqpoint{0.777173in}{0.711445in}}{\pgfqpoint{0.772783in}{0.700846in}}{\pgfqpoint{0.772783in}{0.689796in}}%
\pgfpathcurveto{\pgfqpoint{0.772783in}{0.678745in}}{\pgfqpoint{0.777173in}{0.668146in}}{\pgfqpoint{0.784987in}{0.660333in}}%
\pgfpathcurveto{\pgfqpoint{0.792800in}{0.652519in}}{\pgfqpoint{0.803399in}{0.648129in}}{\pgfqpoint{0.814449in}{0.648129in}}%
\pgfpathclose%
\pgfusepath{stroke,fill}%
\end{pgfscope}%
\begin{pgfscope}%
\pgfpathrectangle{\pgfqpoint{0.648703in}{0.548769in}}{\pgfqpoint{5.112893in}{3.102590in}}%
\pgfusepath{clip}%
\pgfsetbuttcap%
\pgfsetroundjoin%
\definecolor{currentfill}{rgb}{0.121569,0.466667,0.705882}%
\pgfsetfillcolor{currentfill}%
\pgfsetlinewidth{1.003750pt}%
\definecolor{currentstroke}{rgb}{0.121569,0.466667,0.705882}%
\pgfsetstrokecolor{currentstroke}%
\pgfsetdash{}{0pt}%
\pgfpathmoveto{\pgfqpoint{1.661759in}{1.166610in}}%
\pgfpathcurveto{\pgfqpoint{1.672809in}{1.166610in}}{\pgfqpoint{1.683408in}{1.171000in}}{\pgfqpoint{1.691222in}{1.178814in}}%
\pgfpathcurveto{\pgfqpoint{1.699036in}{1.186627in}}{\pgfqpoint{1.703426in}{1.197226in}}{\pgfqpoint{1.703426in}{1.208277in}}%
\pgfpathcurveto{\pgfqpoint{1.703426in}{1.219327in}}{\pgfqpoint{1.699036in}{1.229926in}}{\pgfqpoint{1.691222in}{1.237739in}}%
\pgfpathcurveto{\pgfqpoint{1.683408in}{1.245553in}}{\pgfqpoint{1.672809in}{1.249943in}}{\pgfqpoint{1.661759in}{1.249943in}}%
\pgfpathcurveto{\pgfqpoint{1.650709in}{1.249943in}}{\pgfqpoint{1.640110in}{1.245553in}}{\pgfqpoint{1.632296in}{1.237739in}}%
\pgfpathcurveto{\pgfqpoint{1.624483in}{1.229926in}}{\pgfqpoint{1.620092in}{1.219327in}}{\pgfqpoint{1.620092in}{1.208277in}}%
\pgfpathcurveto{\pgfqpoint{1.620092in}{1.197226in}}{\pgfqpoint{1.624483in}{1.186627in}}{\pgfqpoint{1.632296in}{1.178814in}}%
\pgfpathcurveto{\pgfqpoint{1.640110in}{1.171000in}}{\pgfqpoint{1.650709in}{1.166610in}}{\pgfqpoint{1.661759in}{1.166610in}}%
\pgfpathclose%
\pgfusepath{stroke,fill}%
\end{pgfscope}%
\begin{pgfscope}%
\pgfpathrectangle{\pgfqpoint{0.648703in}{0.548769in}}{\pgfqpoint{5.112893in}{3.102590in}}%
\pgfusepath{clip}%
\pgfsetbuttcap%
\pgfsetroundjoin%
\definecolor{currentfill}{rgb}{1.000000,0.498039,0.054902}%
\pgfsetfillcolor{currentfill}%
\pgfsetlinewidth{1.003750pt}%
\definecolor{currentstroke}{rgb}{1.000000,0.498039,0.054902}%
\pgfsetstrokecolor{currentstroke}%
\pgfsetdash{}{0pt}%
\pgfpathmoveto{\pgfqpoint{3.206408in}{3.136837in}}%
\pgfpathcurveto{\pgfqpoint{3.217458in}{3.136837in}}{\pgfqpoint{3.228057in}{3.141228in}}{\pgfqpoint{3.235870in}{3.149041in}}%
\pgfpathcurveto{\pgfqpoint{3.243684in}{3.156855in}}{\pgfqpoint{3.248074in}{3.167454in}}{\pgfqpoint{3.248074in}{3.178504in}}%
\pgfpathcurveto{\pgfqpoint{3.248074in}{3.189554in}}{\pgfqpoint{3.243684in}{3.200153in}}{\pgfqpoint{3.235870in}{3.207967in}}%
\pgfpathcurveto{\pgfqpoint{3.228057in}{3.215780in}}{\pgfqpoint{3.217458in}{3.220171in}}{\pgfqpoint{3.206408in}{3.220171in}}%
\pgfpathcurveto{\pgfqpoint{3.195357in}{3.220171in}}{\pgfqpoint{3.184758in}{3.215780in}}{\pgfqpoint{3.176945in}{3.207967in}}%
\pgfpathcurveto{\pgfqpoint{3.169131in}{3.200153in}}{\pgfqpoint{3.164741in}{3.189554in}}{\pgfqpoint{3.164741in}{3.178504in}}%
\pgfpathcurveto{\pgfqpoint{3.164741in}{3.167454in}}{\pgfqpoint{3.169131in}{3.156855in}}{\pgfqpoint{3.176945in}{3.149041in}}%
\pgfpathcurveto{\pgfqpoint{3.184758in}{3.141228in}}{\pgfqpoint{3.195357in}{3.136837in}}{\pgfqpoint{3.206408in}{3.136837in}}%
\pgfpathclose%
\pgfusepath{stroke,fill}%
\end{pgfscope}%
\begin{pgfscope}%
\pgfpathrectangle{\pgfqpoint{0.648703in}{0.548769in}}{\pgfqpoint{5.112893in}{3.102590in}}%
\pgfusepath{clip}%
\pgfsetbuttcap%
\pgfsetroundjoin%
\definecolor{currentfill}{rgb}{1.000000,0.498039,0.054902}%
\pgfsetfillcolor{currentfill}%
\pgfsetlinewidth{1.003750pt}%
\definecolor{currentstroke}{rgb}{1.000000,0.498039,0.054902}%
\pgfsetstrokecolor{currentstroke}%
\pgfsetdash{}{0pt}%
\pgfpathmoveto{\pgfqpoint{1.920326in}{3.132690in}}%
\pgfpathcurveto{\pgfqpoint{1.931376in}{3.132690in}}{\pgfqpoint{1.941975in}{3.137080in}}{\pgfqpoint{1.949788in}{3.144893in}}%
\pgfpathcurveto{\pgfqpoint{1.957602in}{3.152707in}}{\pgfqpoint{1.961992in}{3.163306in}}{\pgfqpoint{1.961992in}{3.174356in}}%
\pgfpathcurveto{\pgfqpoint{1.961992in}{3.185406in}}{\pgfqpoint{1.957602in}{3.196005in}}{\pgfqpoint{1.949788in}{3.203819in}}%
\pgfpathcurveto{\pgfqpoint{1.941975in}{3.211633in}}{\pgfqpoint{1.931376in}{3.216023in}}{\pgfqpoint{1.920326in}{3.216023in}}%
\pgfpathcurveto{\pgfqpoint{1.909276in}{3.216023in}}{\pgfqpoint{1.898676in}{3.211633in}}{\pgfqpoint{1.890863in}{3.203819in}}%
\pgfpathcurveto{\pgfqpoint{1.883049in}{3.196005in}}{\pgfqpoint{1.878659in}{3.185406in}}{\pgfqpoint{1.878659in}{3.174356in}}%
\pgfpathcurveto{\pgfqpoint{1.878659in}{3.163306in}}{\pgfqpoint{1.883049in}{3.152707in}}{\pgfqpoint{1.890863in}{3.144893in}}%
\pgfpathcurveto{\pgfqpoint{1.898676in}{3.137080in}}{\pgfqpoint{1.909276in}{3.132690in}}{\pgfqpoint{1.920326in}{3.132690in}}%
\pgfpathclose%
\pgfusepath{stroke,fill}%
\end{pgfscope}%
\begin{pgfscope}%
\pgfpathrectangle{\pgfqpoint{0.648703in}{0.548769in}}{\pgfqpoint{5.112893in}{3.102590in}}%
\pgfusepath{clip}%
\pgfsetbuttcap%
\pgfsetroundjoin%
\definecolor{currentfill}{rgb}{0.839216,0.152941,0.156863}%
\pgfsetfillcolor{currentfill}%
\pgfsetlinewidth{1.003750pt}%
\definecolor{currentstroke}{rgb}{0.839216,0.152941,0.156863}%
\pgfsetstrokecolor{currentstroke}%
\pgfsetdash{}{0pt}%
\pgfpathmoveto{\pgfqpoint{3.829641in}{3.149281in}}%
\pgfpathcurveto{\pgfqpoint{3.840691in}{3.149281in}}{\pgfqpoint{3.851290in}{3.153671in}}{\pgfqpoint{3.859103in}{3.161485in}}%
\pgfpathcurveto{\pgfqpoint{3.866917in}{3.169298in}}{\pgfqpoint{3.871307in}{3.179897in}}{\pgfqpoint{3.871307in}{3.190948in}}%
\pgfpathcurveto{\pgfqpoint{3.871307in}{3.201998in}}{\pgfqpoint{3.866917in}{3.212597in}}{\pgfqpoint{3.859103in}{3.220410in}}%
\pgfpathcurveto{\pgfqpoint{3.851290in}{3.228224in}}{\pgfqpoint{3.840691in}{3.232614in}}{\pgfqpoint{3.829641in}{3.232614in}}%
\pgfpathcurveto{\pgfqpoint{3.818591in}{3.232614in}}{\pgfqpoint{3.807992in}{3.228224in}}{\pgfqpoint{3.800178in}{3.220410in}}%
\pgfpathcurveto{\pgfqpoint{3.792364in}{3.212597in}}{\pgfqpoint{3.787974in}{3.201998in}}{\pgfqpoint{3.787974in}{3.190948in}}%
\pgfpathcurveto{\pgfqpoint{3.787974in}{3.179897in}}{\pgfqpoint{3.792364in}{3.169298in}}{\pgfqpoint{3.800178in}{3.161485in}}%
\pgfpathcurveto{\pgfqpoint{3.807992in}{3.153671in}}{\pgfqpoint{3.818591in}{3.149281in}}{\pgfqpoint{3.829641in}{3.149281in}}%
\pgfpathclose%
\pgfusepath{stroke,fill}%
\end{pgfscope}%
\begin{pgfscope}%
\pgfpathrectangle{\pgfqpoint{0.648703in}{0.548769in}}{\pgfqpoint{5.112893in}{3.102590in}}%
\pgfusepath{clip}%
\pgfsetbuttcap%
\pgfsetroundjoin%
\definecolor{currentfill}{rgb}{1.000000,0.498039,0.054902}%
\pgfsetfillcolor{currentfill}%
\pgfsetlinewidth{1.003750pt}%
\definecolor{currentstroke}{rgb}{1.000000,0.498039,0.054902}%
\pgfsetstrokecolor{currentstroke}%
\pgfsetdash{}{0pt}%
\pgfpathmoveto{\pgfqpoint{3.104449in}{3.174168in}}%
\pgfpathcurveto{\pgfqpoint{3.115499in}{3.174168in}}{\pgfqpoint{3.126098in}{3.178558in}}{\pgfqpoint{3.133911in}{3.186372in}}%
\pgfpathcurveto{\pgfqpoint{3.141725in}{3.194185in}}{\pgfqpoint{3.146115in}{3.204785in}}{\pgfqpoint{3.146115in}{3.215835in}}%
\pgfpathcurveto{\pgfqpoint{3.146115in}{3.226885in}}{\pgfqpoint{3.141725in}{3.237484in}}{\pgfqpoint{3.133911in}{3.245297in}}%
\pgfpathcurveto{\pgfqpoint{3.126098in}{3.253111in}}{\pgfqpoint{3.115499in}{3.257501in}}{\pgfqpoint{3.104449in}{3.257501in}}%
\pgfpathcurveto{\pgfqpoint{3.093398in}{3.257501in}}{\pgfqpoint{3.082799in}{3.253111in}}{\pgfqpoint{3.074986in}{3.245297in}}%
\pgfpathcurveto{\pgfqpoint{3.067172in}{3.237484in}}{\pgfqpoint{3.062782in}{3.226885in}}{\pgfqpoint{3.062782in}{3.215835in}}%
\pgfpathcurveto{\pgfqpoint{3.062782in}{3.204785in}}{\pgfqpoint{3.067172in}{3.194185in}}{\pgfqpoint{3.074986in}{3.186372in}}%
\pgfpathcurveto{\pgfqpoint{3.082799in}{3.178558in}}{\pgfqpoint{3.093398in}{3.174168in}}{\pgfqpoint{3.104449in}{3.174168in}}%
\pgfpathclose%
\pgfusepath{stroke,fill}%
\end{pgfscope}%
\begin{pgfscope}%
\pgfpathrectangle{\pgfqpoint{0.648703in}{0.548769in}}{\pgfqpoint{5.112893in}{3.102590in}}%
\pgfusepath{clip}%
\pgfsetbuttcap%
\pgfsetroundjoin%
\definecolor{currentfill}{rgb}{0.121569,0.466667,0.705882}%
\pgfsetfillcolor{currentfill}%
\pgfsetlinewidth{1.003750pt}%
\definecolor{currentstroke}{rgb}{0.121569,0.466667,0.705882}%
\pgfsetstrokecolor{currentstroke}%
\pgfsetdash{}{0pt}%
\pgfpathmoveto{\pgfqpoint{0.846845in}{0.656425in}}%
\pgfpathcurveto{\pgfqpoint{0.857895in}{0.656425in}}{\pgfqpoint{0.868494in}{0.660815in}}{\pgfqpoint{0.876308in}{0.668629in}}%
\pgfpathcurveto{\pgfqpoint{0.884122in}{0.676442in}}{\pgfqpoint{0.888512in}{0.687041in}}{\pgfqpoint{0.888512in}{0.698091in}}%
\pgfpathcurveto{\pgfqpoint{0.888512in}{0.709141in}}{\pgfqpoint{0.884122in}{0.719740in}}{\pgfqpoint{0.876308in}{0.727554in}}%
\pgfpathcurveto{\pgfqpoint{0.868494in}{0.735368in}}{\pgfqpoint{0.857895in}{0.739758in}}{\pgfqpoint{0.846845in}{0.739758in}}%
\pgfpathcurveto{\pgfqpoint{0.835795in}{0.739758in}}{\pgfqpoint{0.825196in}{0.735368in}}{\pgfqpoint{0.817382in}{0.727554in}}%
\pgfpathcurveto{\pgfqpoint{0.809569in}{0.719740in}}{\pgfqpoint{0.805179in}{0.709141in}}{\pgfqpoint{0.805179in}{0.698091in}}%
\pgfpathcurveto{\pgfqpoint{0.805179in}{0.687041in}}{\pgfqpoint{0.809569in}{0.676442in}}{\pgfqpoint{0.817382in}{0.668629in}}%
\pgfpathcurveto{\pgfqpoint{0.825196in}{0.660815in}}{\pgfqpoint{0.835795in}{0.656425in}}{\pgfqpoint{0.846845in}{0.656425in}}%
\pgfpathclose%
\pgfusepath{stroke,fill}%
\end{pgfscope}%
\begin{pgfscope}%
\pgfpathrectangle{\pgfqpoint{0.648703in}{0.548769in}}{\pgfqpoint{5.112893in}{3.102590in}}%
\pgfusepath{clip}%
\pgfsetbuttcap%
\pgfsetroundjoin%
\definecolor{currentfill}{rgb}{1.000000,0.498039,0.054902}%
\pgfsetfillcolor{currentfill}%
\pgfsetlinewidth{1.003750pt}%
\definecolor{currentstroke}{rgb}{1.000000,0.498039,0.054902}%
\pgfsetstrokecolor{currentstroke}%
\pgfsetdash{}{0pt}%
\pgfpathmoveto{\pgfqpoint{1.518468in}{3.124394in}}%
\pgfpathcurveto{\pgfqpoint{1.529518in}{3.124394in}}{\pgfqpoint{1.540117in}{3.128784in}}{\pgfqpoint{1.547931in}{3.136598in}}%
\pgfpathcurveto{\pgfqpoint{1.555745in}{3.144411in}}{\pgfqpoint{1.560135in}{3.155010in}}{\pgfqpoint{1.560135in}{3.166060in}}%
\pgfpathcurveto{\pgfqpoint{1.560135in}{3.177111in}}{\pgfqpoint{1.555745in}{3.187710in}}{\pgfqpoint{1.547931in}{3.195523in}}%
\pgfpathcurveto{\pgfqpoint{1.540117in}{3.203337in}}{\pgfqpoint{1.529518in}{3.207727in}}{\pgfqpoint{1.518468in}{3.207727in}}%
\pgfpathcurveto{\pgfqpoint{1.507418in}{3.207727in}}{\pgfqpoint{1.496819in}{3.203337in}}{\pgfqpoint{1.489005in}{3.195523in}}%
\pgfpathcurveto{\pgfqpoint{1.481192in}{3.187710in}}{\pgfqpoint{1.476801in}{3.177111in}}{\pgfqpoint{1.476801in}{3.166060in}}%
\pgfpathcurveto{\pgfqpoint{1.476801in}{3.155010in}}{\pgfqpoint{1.481192in}{3.144411in}}{\pgfqpoint{1.489005in}{3.136598in}}%
\pgfpathcurveto{\pgfqpoint{1.496819in}{3.128784in}}{\pgfqpoint{1.507418in}{3.124394in}}{\pgfqpoint{1.518468in}{3.124394in}}%
\pgfpathclose%
\pgfusepath{stroke,fill}%
\end{pgfscope}%
\begin{pgfscope}%
\pgfpathrectangle{\pgfqpoint{0.648703in}{0.548769in}}{\pgfqpoint{5.112893in}{3.102590in}}%
\pgfusepath{clip}%
\pgfsetbuttcap%
\pgfsetroundjoin%
\definecolor{currentfill}{rgb}{0.121569,0.466667,0.705882}%
\pgfsetfillcolor{currentfill}%
\pgfsetlinewidth{1.003750pt}%
\definecolor{currentstroke}{rgb}{0.121569,0.466667,0.705882}%
\pgfsetstrokecolor{currentstroke}%
\pgfsetdash{}{0pt}%
\pgfpathmoveto{\pgfqpoint{0.814442in}{0.648129in}}%
\pgfpathcurveto{\pgfqpoint{0.825492in}{0.648129in}}{\pgfqpoint{0.836091in}{0.652519in}}{\pgfqpoint{0.843905in}{0.660333in}}%
\pgfpathcurveto{\pgfqpoint{0.851719in}{0.668146in}}{\pgfqpoint{0.856109in}{0.678745in}}{\pgfqpoint{0.856109in}{0.689796in}}%
\pgfpathcurveto{\pgfqpoint{0.856109in}{0.700846in}}{\pgfqpoint{0.851719in}{0.711445in}}{\pgfqpoint{0.843905in}{0.719258in}}%
\pgfpathcurveto{\pgfqpoint{0.836091in}{0.727072in}}{\pgfqpoint{0.825492in}{0.731462in}}{\pgfqpoint{0.814442in}{0.731462in}}%
\pgfpathcurveto{\pgfqpoint{0.803392in}{0.731462in}}{\pgfqpoint{0.792793in}{0.727072in}}{\pgfqpoint{0.784979in}{0.719258in}}%
\pgfpathcurveto{\pgfqpoint{0.777166in}{0.711445in}}{\pgfqpoint{0.772776in}{0.700846in}}{\pgfqpoint{0.772776in}{0.689796in}}%
\pgfpathcurveto{\pgfqpoint{0.772776in}{0.678745in}}{\pgfqpoint{0.777166in}{0.668146in}}{\pgfqpoint{0.784979in}{0.660333in}}%
\pgfpathcurveto{\pgfqpoint{0.792793in}{0.652519in}}{\pgfqpoint{0.803392in}{0.648129in}}{\pgfqpoint{0.814442in}{0.648129in}}%
\pgfpathclose%
\pgfusepath{stroke,fill}%
\end{pgfscope}%
\begin{pgfscope}%
\pgfpathrectangle{\pgfqpoint{0.648703in}{0.548769in}}{\pgfqpoint{5.112893in}{3.102590in}}%
\pgfusepath{clip}%
\pgfsetbuttcap%
\pgfsetroundjoin%
\definecolor{currentfill}{rgb}{1.000000,0.498039,0.054902}%
\pgfsetfillcolor{currentfill}%
\pgfsetlinewidth{1.003750pt}%
\definecolor{currentstroke}{rgb}{1.000000,0.498039,0.054902}%
\pgfsetstrokecolor{currentstroke}%
\pgfsetdash{}{0pt}%
\pgfpathmoveto{\pgfqpoint{1.730792in}{3.136837in}}%
\pgfpathcurveto{\pgfqpoint{1.741842in}{3.136837in}}{\pgfqpoint{1.752441in}{3.141228in}}{\pgfqpoint{1.760254in}{3.149041in}}%
\pgfpathcurveto{\pgfqpoint{1.768068in}{3.156855in}}{\pgfqpoint{1.772458in}{3.167454in}}{\pgfqpoint{1.772458in}{3.178504in}}%
\pgfpathcurveto{\pgfqpoint{1.772458in}{3.189554in}}{\pgfqpoint{1.768068in}{3.200153in}}{\pgfqpoint{1.760254in}{3.207967in}}%
\pgfpathcurveto{\pgfqpoint{1.752441in}{3.215780in}}{\pgfqpoint{1.741842in}{3.220171in}}{\pgfqpoint{1.730792in}{3.220171in}}%
\pgfpathcurveto{\pgfqpoint{1.719741in}{3.220171in}}{\pgfqpoint{1.709142in}{3.215780in}}{\pgfqpoint{1.701329in}{3.207967in}}%
\pgfpathcurveto{\pgfqpoint{1.693515in}{3.200153in}}{\pgfqpoint{1.689125in}{3.189554in}}{\pgfqpoint{1.689125in}{3.178504in}}%
\pgfpathcurveto{\pgfqpoint{1.689125in}{3.167454in}}{\pgfqpoint{1.693515in}{3.156855in}}{\pgfqpoint{1.701329in}{3.149041in}}%
\pgfpathcurveto{\pgfqpoint{1.709142in}{3.141228in}}{\pgfqpoint{1.719741in}{3.136837in}}{\pgfqpoint{1.730792in}{3.136837in}}%
\pgfpathclose%
\pgfusepath{stroke,fill}%
\end{pgfscope}%
\begin{pgfscope}%
\pgfpathrectangle{\pgfqpoint{0.648703in}{0.548769in}}{\pgfqpoint{5.112893in}{3.102590in}}%
\pgfusepath{clip}%
\pgfsetbuttcap%
\pgfsetroundjoin%
\definecolor{currentfill}{rgb}{1.000000,0.498039,0.054902}%
\pgfsetfillcolor{currentfill}%
\pgfsetlinewidth{1.003750pt}%
\definecolor{currentstroke}{rgb}{1.000000,0.498039,0.054902}%
\pgfsetstrokecolor{currentstroke}%
\pgfsetdash{}{0pt}%
\pgfpathmoveto{\pgfqpoint{3.353345in}{3.140985in}}%
\pgfpathcurveto{\pgfqpoint{3.364396in}{3.140985in}}{\pgfqpoint{3.374995in}{3.145375in}}{\pgfqpoint{3.382808in}{3.153189in}}%
\pgfpathcurveto{\pgfqpoint{3.390622in}{3.161003in}}{\pgfqpoint{3.395012in}{3.171602in}}{\pgfqpoint{3.395012in}{3.182652in}}%
\pgfpathcurveto{\pgfqpoint{3.395012in}{3.193702in}}{\pgfqpoint{3.390622in}{3.204301in}}{\pgfqpoint{3.382808in}{3.212115in}}%
\pgfpathcurveto{\pgfqpoint{3.374995in}{3.219928in}}{\pgfqpoint{3.364396in}{3.224319in}}{\pgfqpoint{3.353345in}{3.224319in}}%
\pgfpathcurveto{\pgfqpoint{3.342295in}{3.224319in}}{\pgfqpoint{3.331696in}{3.219928in}}{\pgfqpoint{3.323883in}{3.212115in}}%
\pgfpathcurveto{\pgfqpoint{3.316069in}{3.204301in}}{\pgfqpoint{3.311679in}{3.193702in}}{\pgfqpoint{3.311679in}{3.182652in}}%
\pgfpathcurveto{\pgfqpoint{3.311679in}{3.171602in}}{\pgfqpoint{3.316069in}{3.161003in}}{\pgfqpoint{3.323883in}{3.153189in}}%
\pgfpathcurveto{\pgfqpoint{3.331696in}{3.145375in}}{\pgfqpoint{3.342295in}{3.140985in}}{\pgfqpoint{3.353345in}{3.140985in}}%
\pgfpathclose%
\pgfusepath{stroke,fill}%
\end{pgfscope}%
\begin{pgfscope}%
\pgfpathrectangle{\pgfqpoint{0.648703in}{0.548769in}}{\pgfqpoint{5.112893in}{3.102590in}}%
\pgfusepath{clip}%
\pgfsetbuttcap%
\pgfsetroundjoin%
\definecolor{currentfill}{rgb}{1.000000,0.498039,0.054902}%
\pgfsetfillcolor{currentfill}%
\pgfsetlinewidth{1.003750pt}%
\definecolor{currentstroke}{rgb}{1.000000,0.498039,0.054902}%
\pgfsetstrokecolor{currentstroke}%
\pgfsetdash{}{0pt}%
\pgfpathmoveto{\pgfqpoint{1.915844in}{3.165872in}}%
\pgfpathcurveto{\pgfqpoint{1.926894in}{3.165872in}}{\pgfqpoint{1.937493in}{3.170263in}}{\pgfqpoint{1.945307in}{3.178076in}}%
\pgfpathcurveto{\pgfqpoint{1.953121in}{3.185890in}}{\pgfqpoint{1.957511in}{3.196489in}}{\pgfqpoint{1.957511in}{3.207539in}}%
\pgfpathcurveto{\pgfqpoint{1.957511in}{3.218589in}}{\pgfqpoint{1.953121in}{3.229188in}}{\pgfqpoint{1.945307in}{3.237002in}}%
\pgfpathcurveto{\pgfqpoint{1.937493in}{3.244815in}}{\pgfqpoint{1.926894in}{3.249206in}}{\pgfqpoint{1.915844in}{3.249206in}}%
\pgfpathcurveto{\pgfqpoint{1.904794in}{3.249206in}}{\pgfqpoint{1.894195in}{3.244815in}}{\pgfqpoint{1.886382in}{3.237002in}}%
\pgfpathcurveto{\pgfqpoint{1.878568in}{3.229188in}}{\pgfqpoint{1.874178in}{3.218589in}}{\pgfqpoint{1.874178in}{3.207539in}}%
\pgfpathcurveto{\pgfqpoint{1.874178in}{3.196489in}}{\pgfqpoint{1.878568in}{3.185890in}}{\pgfqpoint{1.886382in}{3.178076in}}%
\pgfpathcurveto{\pgfqpoint{1.894195in}{3.170263in}}{\pgfqpoint{1.904794in}{3.165872in}}{\pgfqpoint{1.915844in}{3.165872in}}%
\pgfpathclose%
\pgfusepath{stroke,fill}%
\end{pgfscope}%
\begin{pgfscope}%
\pgfpathrectangle{\pgfqpoint{0.648703in}{0.548769in}}{\pgfqpoint{5.112893in}{3.102590in}}%
\pgfusepath{clip}%
\pgfsetbuttcap%
\pgfsetroundjoin%
\definecolor{currentfill}{rgb}{1.000000,0.498039,0.054902}%
\pgfsetfillcolor{currentfill}%
\pgfsetlinewidth{1.003750pt}%
\definecolor{currentstroke}{rgb}{1.000000,0.498039,0.054902}%
\pgfsetstrokecolor{currentstroke}%
\pgfsetdash{}{0pt}%
\pgfpathmoveto{\pgfqpoint{3.254298in}{3.145133in}}%
\pgfpathcurveto{\pgfqpoint{3.265348in}{3.145133in}}{\pgfqpoint{3.275947in}{3.149523in}}{\pgfqpoint{3.283760in}{3.157337in}}%
\pgfpathcurveto{\pgfqpoint{3.291574in}{3.165151in}}{\pgfqpoint{3.295964in}{3.175750in}}{\pgfqpoint{3.295964in}{3.186800in}}%
\pgfpathcurveto{\pgfqpoint{3.295964in}{3.197850in}}{\pgfqpoint{3.291574in}{3.208449in}}{\pgfqpoint{3.283760in}{3.216262in}}%
\pgfpathcurveto{\pgfqpoint{3.275947in}{3.224076in}}{\pgfqpoint{3.265348in}{3.228466in}}{\pgfqpoint{3.254298in}{3.228466in}}%
\pgfpathcurveto{\pgfqpoint{3.243248in}{3.228466in}}{\pgfqpoint{3.232648in}{3.224076in}}{\pgfqpoint{3.224835in}{3.216262in}}%
\pgfpathcurveto{\pgfqpoint{3.217021in}{3.208449in}}{\pgfqpoint{3.212631in}{3.197850in}}{\pgfqpoint{3.212631in}{3.186800in}}%
\pgfpathcurveto{\pgfqpoint{3.212631in}{3.175750in}}{\pgfqpoint{3.217021in}{3.165151in}}{\pgfqpoint{3.224835in}{3.157337in}}%
\pgfpathcurveto{\pgfqpoint{3.232648in}{3.149523in}}{\pgfqpoint{3.243248in}{3.145133in}}{\pgfqpoint{3.254298in}{3.145133in}}%
\pgfpathclose%
\pgfusepath{stroke,fill}%
\end{pgfscope}%
\begin{pgfscope}%
\pgfpathrectangle{\pgfqpoint{0.648703in}{0.548769in}}{\pgfqpoint{5.112893in}{3.102590in}}%
\pgfusepath{clip}%
\pgfsetbuttcap%
\pgfsetroundjoin%
\definecolor{currentfill}{rgb}{1.000000,0.498039,0.054902}%
\pgfsetfillcolor{currentfill}%
\pgfsetlinewidth{1.003750pt}%
\definecolor{currentstroke}{rgb}{1.000000,0.498039,0.054902}%
\pgfsetstrokecolor{currentstroke}%
\pgfsetdash{}{0pt}%
\pgfpathmoveto{\pgfqpoint{3.532669in}{3.157577in}}%
\pgfpathcurveto{\pgfqpoint{3.543720in}{3.157577in}}{\pgfqpoint{3.554319in}{3.161967in}}{\pgfqpoint{3.562132in}{3.169780in}}%
\pgfpathcurveto{\pgfqpoint{3.569946in}{3.177594in}}{\pgfqpoint{3.574336in}{3.188193in}}{\pgfqpoint{3.574336in}{3.199243in}}%
\pgfpathcurveto{\pgfqpoint{3.574336in}{3.210293in}}{\pgfqpoint{3.569946in}{3.220892in}}{\pgfqpoint{3.562132in}{3.228706in}}%
\pgfpathcurveto{\pgfqpoint{3.554319in}{3.236520in}}{\pgfqpoint{3.543720in}{3.240910in}}{\pgfqpoint{3.532669in}{3.240910in}}%
\pgfpathcurveto{\pgfqpoint{3.521619in}{3.240910in}}{\pgfqpoint{3.511020in}{3.236520in}}{\pgfqpoint{3.503207in}{3.228706in}}%
\pgfpathcurveto{\pgfqpoint{3.495393in}{3.220892in}}{\pgfqpoint{3.491003in}{3.210293in}}{\pgfqpoint{3.491003in}{3.199243in}}%
\pgfpathcurveto{\pgfqpoint{3.491003in}{3.188193in}}{\pgfqpoint{3.495393in}{3.177594in}}{\pgfqpoint{3.503207in}{3.169780in}}%
\pgfpathcurveto{\pgfqpoint{3.511020in}{3.161967in}}{\pgfqpoint{3.521619in}{3.157577in}}{\pgfqpoint{3.532669in}{3.157577in}}%
\pgfpathclose%
\pgfusepath{stroke,fill}%
\end{pgfscope}%
\begin{pgfscope}%
\pgfpathrectangle{\pgfqpoint{0.648703in}{0.548769in}}{\pgfqpoint{5.112893in}{3.102590in}}%
\pgfusepath{clip}%
\pgfsetbuttcap%
\pgfsetroundjoin%
\definecolor{currentfill}{rgb}{1.000000,0.498039,0.054902}%
\pgfsetfillcolor{currentfill}%
\pgfsetlinewidth{1.003750pt}%
\definecolor{currentstroke}{rgb}{1.000000,0.498039,0.054902}%
\pgfsetstrokecolor{currentstroke}%
\pgfsetdash{}{0pt}%
\pgfpathmoveto{\pgfqpoint{3.490038in}{3.323490in}}%
\pgfpathcurveto{\pgfqpoint{3.501088in}{3.323490in}}{\pgfqpoint{3.511688in}{3.327881in}}{\pgfqpoint{3.519501in}{3.335694in}}%
\pgfpathcurveto{\pgfqpoint{3.527315in}{3.343508in}}{\pgfqpoint{3.531705in}{3.354107in}}{\pgfqpoint{3.531705in}{3.365157in}}%
\pgfpathcurveto{\pgfqpoint{3.531705in}{3.376207in}}{\pgfqpoint{3.527315in}{3.386806in}}{\pgfqpoint{3.519501in}{3.394620in}}%
\pgfpathcurveto{\pgfqpoint{3.511688in}{3.402434in}}{\pgfqpoint{3.501088in}{3.406824in}}{\pgfqpoint{3.490038in}{3.406824in}}%
\pgfpathcurveto{\pgfqpoint{3.478988in}{3.406824in}}{\pgfqpoint{3.468389in}{3.402434in}}{\pgfqpoint{3.460576in}{3.394620in}}%
\pgfpathcurveto{\pgfqpoint{3.452762in}{3.386806in}}{\pgfqpoint{3.448372in}{3.376207in}}{\pgfqpoint{3.448372in}{3.365157in}}%
\pgfpathcurveto{\pgfqpoint{3.448372in}{3.354107in}}{\pgfqpoint{3.452762in}{3.343508in}}{\pgfqpoint{3.460576in}{3.335694in}}%
\pgfpathcurveto{\pgfqpoint{3.468389in}{3.327881in}}{\pgfqpoint{3.478988in}{3.323490in}}{\pgfqpoint{3.490038in}{3.323490in}}%
\pgfpathclose%
\pgfusepath{stroke,fill}%
\end{pgfscope}%
\begin{pgfscope}%
\pgfpathrectangle{\pgfqpoint{0.648703in}{0.548769in}}{\pgfqpoint{5.112893in}{3.102590in}}%
\pgfusepath{clip}%
\pgfsetbuttcap%
\pgfsetroundjoin%
\definecolor{currentfill}{rgb}{0.121569,0.466667,0.705882}%
\pgfsetfillcolor{currentfill}%
\pgfsetlinewidth{1.003750pt}%
\definecolor{currentstroke}{rgb}{0.121569,0.466667,0.705882}%
\pgfsetstrokecolor{currentstroke}%
\pgfsetdash{}{0pt}%
\pgfpathmoveto{\pgfqpoint{2.802377in}{2.410964in}}%
\pgfpathcurveto{\pgfqpoint{2.813427in}{2.410964in}}{\pgfqpoint{2.824026in}{2.415354in}}{\pgfqpoint{2.831840in}{2.423168in}}%
\pgfpathcurveto{\pgfqpoint{2.839653in}{2.430982in}}{\pgfqpoint{2.844044in}{2.441581in}}{\pgfqpoint{2.844044in}{2.452631in}}%
\pgfpathcurveto{\pgfqpoint{2.844044in}{2.463681in}}{\pgfqpoint{2.839653in}{2.474280in}}{\pgfqpoint{2.831840in}{2.482094in}}%
\pgfpathcurveto{\pgfqpoint{2.824026in}{2.489907in}}{\pgfqpoint{2.813427in}{2.494297in}}{\pgfqpoint{2.802377in}{2.494297in}}%
\pgfpathcurveto{\pgfqpoint{2.791327in}{2.494297in}}{\pgfqpoint{2.780728in}{2.489907in}}{\pgfqpoint{2.772914in}{2.482094in}}%
\pgfpathcurveto{\pgfqpoint{2.765101in}{2.474280in}}{\pgfqpoint{2.760710in}{2.463681in}}{\pgfqpoint{2.760710in}{2.452631in}}%
\pgfpathcurveto{\pgfqpoint{2.760710in}{2.441581in}}{\pgfqpoint{2.765101in}{2.430982in}}{\pgfqpoint{2.772914in}{2.423168in}}%
\pgfpathcurveto{\pgfqpoint{2.780728in}{2.415354in}}{\pgfqpoint{2.791327in}{2.410964in}}{\pgfqpoint{2.802377in}{2.410964in}}%
\pgfpathclose%
\pgfusepath{stroke,fill}%
\end{pgfscope}%
\begin{pgfscope}%
\pgfpathrectangle{\pgfqpoint{0.648703in}{0.548769in}}{\pgfqpoint{5.112893in}{3.102590in}}%
\pgfusepath{clip}%
\pgfsetbuttcap%
\pgfsetroundjoin%
\definecolor{currentfill}{rgb}{1.000000,0.498039,0.054902}%
\pgfsetfillcolor{currentfill}%
\pgfsetlinewidth{1.003750pt}%
\definecolor{currentstroke}{rgb}{1.000000,0.498039,0.054902}%
\pgfsetstrokecolor{currentstroke}%
\pgfsetdash{}{0pt}%
\pgfpathmoveto{\pgfqpoint{3.436829in}{3.199055in}}%
\pgfpathcurveto{\pgfqpoint{3.447879in}{3.199055in}}{\pgfqpoint{3.458478in}{3.203445in}}{\pgfqpoint{3.466292in}{3.211259in}}%
\pgfpathcurveto{\pgfqpoint{3.474106in}{3.219073in}}{\pgfqpoint{3.478496in}{3.229672in}}{\pgfqpoint{3.478496in}{3.240722in}}%
\pgfpathcurveto{\pgfqpoint{3.478496in}{3.251772in}}{\pgfqpoint{3.474106in}{3.262371in}}{\pgfqpoint{3.466292in}{3.270185in}}%
\pgfpathcurveto{\pgfqpoint{3.458478in}{3.277998in}}{\pgfqpoint{3.447879in}{3.282388in}}{\pgfqpoint{3.436829in}{3.282388in}}%
\pgfpathcurveto{\pgfqpoint{3.425779in}{3.282388in}}{\pgfqpoint{3.415180in}{3.277998in}}{\pgfqpoint{3.407366in}{3.270185in}}%
\pgfpathcurveto{\pgfqpoint{3.399553in}{3.262371in}}{\pgfqpoint{3.395163in}{3.251772in}}{\pgfqpoint{3.395163in}{3.240722in}}%
\pgfpathcurveto{\pgfqpoint{3.395163in}{3.229672in}}{\pgfqpoint{3.399553in}{3.219073in}}{\pgfqpoint{3.407366in}{3.211259in}}%
\pgfpathcurveto{\pgfqpoint{3.415180in}{3.203445in}}{\pgfqpoint{3.425779in}{3.199055in}}{\pgfqpoint{3.436829in}{3.199055in}}%
\pgfpathclose%
\pgfusepath{stroke,fill}%
\end{pgfscope}%
\begin{pgfscope}%
\pgfpathrectangle{\pgfqpoint{0.648703in}{0.548769in}}{\pgfqpoint{5.112893in}{3.102590in}}%
\pgfusepath{clip}%
\pgfsetbuttcap%
\pgfsetroundjoin%
\definecolor{currentfill}{rgb}{1.000000,0.498039,0.054902}%
\pgfsetfillcolor{currentfill}%
\pgfsetlinewidth{1.003750pt}%
\definecolor{currentstroke}{rgb}{1.000000,0.498039,0.054902}%
\pgfsetstrokecolor{currentstroke}%
\pgfsetdash{}{0pt}%
\pgfpathmoveto{\pgfqpoint{1.597714in}{3.124394in}}%
\pgfpathcurveto{\pgfqpoint{1.608764in}{3.124394in}}{\pgfqpoint{1.619363in}{3.128784in}}{\pgfqpoint{1.627177in}{3.136598in}}%
\pgfpathcurveto{\pgfqpoint{1.634990in}{3.144411in}}{\pgfqpoint{1.639381in}{3.155010in}}{\pgfqpoint{1.639381in}{3.166060in}}%
\pgfpathcurveto{\pgfqpoint{1.639381in}{3.177111in}}{\pgfqpoint{1.634990in}{3.187710in}}{\pgfqpoint{1.627177in}{3.195523in}}%
\pgfpathcurveto{\pgfqpoint{1.619363in}{3.203337in}}{\pgfqpoint{1.608764in}{3.207727in}}{\pgfqpoint{1.597714in}{3.207727in}}%
\pgfpathcurveto{\pgfqpoint{1.586664in}{3.207727in}}{\pgfqpoint{1.576065in}{3.203337in}}{\pgfqpoint{1.568251in}{3.195523in}}%
\pgfpathcurveto{\pgfqpoint{1.560438in}{3.187710in}}{\pgfqpoint{1.556047in}{3.177111in}}{\pgfqpoint{1.556047in}{3.166060in}}%
\pgfpathcurveto{\pgfqpoint{1.556047in}{3.155010in}}{\pgfqpoint{1.560438in}{3.144411in}}{\pgfqpoint{1.568251in}{3.136598in}}%
\pgfpathcurveto{\pgfqpoint{1.576065in}{3.128784in}}{\pgfqpoint{1.586664in}{3.124394in}}{\pgfqpoint{1.597714in}{3.124394in}}%
\pgfpathclose%
\pgfusepath{stroke,fill}%
\end{pgfscope}%
\begin{pgfscope}%
\pgfpathrectangle{\pgfqpoint{0.648703in}{0.548769in}}{\pgfqpoint{5.112893in}{3.102590in}}%
\pgfusepath{clip}%
\pgfsetbuttcap%
\pgfsetroundjoin%
\definecolor{currentfill}{rgb}{1.000000,0.498039,0.054902}%
\pgfsetfillcolor{currentfill}%
\pgfsetlinewidth{1.003750pt}%
\definecolor{currentstroke}{rgb}{1.000000,0.498039,0.054902}%
\pgfsetstrokecolor{currentstroke}%
\pgfsetdash{}{0pt}%
\pgfpathmoveto{\pgfqpoint{2.400571in}{3.132690in}}%
\pgfpathcurveto{\pgfqpoint{2.411622in}{3.132690in}}{\pgfqpoint{2.422221in}{3.137080in}}{\pgfqpoint{2.430034in}{3.144893in}}%
\pgfpathcurveto{\pgfqpoint{2.437848in}{3.152707in}}{\pgfqpoint{2.442238in}{3.163306in}}{\pgfqpoint{2.442238in}{3.174356in}}%
\pgfpathcurveto{\pgfqpoint{2.442238in}{3.185406in}}{\pgfqpoint{2.437848in}{3.196005in}}{\pgfqpoint{2.430034in}{3.203819in}}%
\pgfpathcurveto{\pgfqpoint{2.422221in}{3.211633in}}{\pgfqpoint{2.411622in}{3.216023in}}{\pgfqpoint{2.400571in}{3.216023in}}%
\pgfpathcurveto{\pgfqpoint{2.389521in}{3.216023in}}{\pgfqpoint{2.378922in}{3.211633in}}{\pgfqpoint{2.371109in}{3.203819in}}%
\pgfpathcurveto{\pgfqpoint{2.363295in}{3.196005in}}{\pgfqpoint{2.358905in}{3.185406in}}{\pgfqpoint{2.358905in}{3.174356in}}%
\pgfpathcurveto{\pgfqpoint{2.358905in}{3.163306in}}{\pgfqpoint{2.363295in}{3.152707in}}{\pgfqpoint{2.371109in}{3.144893in}}%
\pgfpathcurveto{\pgfqpoint{2.378922in}{3.137080in}}{\pgfqpoint{2.389521in}{3.132690in}}{\pgfqpoint{2.400571in}{3.132690in}}%
\pgfpathclose%
\pgfusepath{stroke,fill}%
\end{pgfscope}%
\begin{pgfscope}%
\pgfpathrectangle{\pgfqpoint{0.648703in}{0.548769in}}{\pgfqpoint{5.112893in}{3.102590in}}%
\pgfusepath{clip}%
\pgfsetbuttcap%
\pgfsetroundjoin%
\definecolor{currentfill}{rgb}{1.000000,0.498039,0.054902}%
\pgfsetfillcolor{currentfill}%
\pgfsetlinewidth{1.003750pt}%
\definecolor{currentstroke}{rgb}{1.000000,0.498039,0.054902}%
\pgfsetstrokecolor{currentstroke}%
\pgfsetdash{}{0pt}%
\pgfpathmoveto{\pgfqpoint{1.516042in}{3.132690in}}%
\pgfpathcurveto{\pgfqpoint{1.527093in}{3.132690in}}{\pgfqpoint{1.537692in}{3.137080in}}{\pgfqpoint{1.545505in}{3.144893in}}%
\pgfpathcurveto{\pgfqpoint{1.553319in}{3.152707in}}{\pgfqpoint{1.557709in}{3.163306in}}{\pgfqpoint{1.557709in}{3.174356in}}%
\pgfpathcurveto{\pgfqpoint{1.557709in}{3.185406in}}{\pgfqpoint{1.553319in}{3.196005in}}{\pgfqpoint{1.545505in}{3.203819in}}%
\pgfpathcurveto{\pgfqpoint{1.537692in}{3.211633in}}{\pgfqpoint{1.527093in}{3.216023in}}{\pgfqpoint{1.516042in}{3.216023in}}%
\pgfpathcurveto{\pgfqpoint{1.504992in}{3.216023in}}{\pgfqpoint{1.494393in}{3.211633in}}{\pgfqpoint{1.486580in}{3.203819in}}%
\pgfpathcurveto{\pgfqpoint{1.478766in}{3.196005in}}{\pgfqpoint{1.474376in}{3.185406in}}{\pgfqpoint{1.474376in}{3.174356in}}%
\pgfpathcurveto{\pgfqpoint{1.474376in}{3.163306in}}{\pgfqpoint{1.478766in}{3.152707in}}{\pgfqpoint{1.486580in}{3.144893in}}%
\pgfpathcurveto{\pgfqpoint{1.494393in}{3.137080in}}{\pgfqpoint{1.504992in}{3.132690in}}{\pgfqpoint{1.516042in}{3.132690in}}%
\pgfpathclose%
\pgfusepath{stroke,fill}%
\end{pgfscope}%
\begin{pgfscope}%
\pgfpathrectangle{\pgfqpoint{0.648703in}{0.548769in}}{\pgfqpoint{5.112893in}{3.102590in}}%
\pgfusepath{clip}%
\pgfsetbuttcap%
\pgfsetroundjoin%
\definecolor{currentfill}{rgb}{1.000000,0.498039,0.054902}%
\pgfsetfillcolor{currentfill}%
\pgfsetlinewidth{1.003750pt}%
\definecolor{currentstroke}{rgb}{1.000000,0.498039,0.054902}%
\pgfsetstrokecolor{currentstroke}%
\pgfsetdash{}{0pt}%
\pgfpathmoveto{\pgfqpoint{2.864111in}{3.136837in}}%
\pgfpathcurveto{\pgfqpoint{2.875161in}{3.136837in}}{\pgfqpoint{2.885760in}{3.141228in}}{\pgfqpoint{2.893574in}{3.149041in}}%
\pgfpathcurveto{\pgfqpoint{2.901388in}{3.156855in}}{\pgfqpoint{2.905778in}{3.167454in}}{\pgfqpoint{2.905778in}{3.178504in}}%
\pgfpathcurveto{\pgfqpoint{2.905778in}{3.189554in}}{\pgfqpoint{2.901388in}{3.200153in}}{\pgfqpoint{2.893574in}{3.207967in}}%
\pgfpathcurveto{\pgfqpoint{2.885760in}{3.215780in}}{\pgfqpoint{2.875161in}{3.220171in}}{\pgfqpoint{2.864111in}{3.220171in}}%
\pgfpathcurveto{\pgfqpoint{2.853061in}{3.220171in}}{\pgfqpoint{2.842462in}{3.215780in}}{\pgfqpoint{2.834649in}{3.207967in}}%
\pgfpathcurveto{\pgfqpoint{2.826835in}{3.200153in}}{\pgfqpoint{2.822445in}{3.189554in}}{\pgfqpoint{2.822445in}{3.178504in}}%
\pgfpathcurveto{\pgfqpoint{2.822445in}{3.167454in}}{\pgfqpoint{2.826835in}{3.156855in}}{\pgfqpoint{2.834649in}{3.149041in}}%
\pgfpathcurveto{\pgfqpoint{2.842462in}{3.141228in}}{\pgfqpoint{2.853061in}{3.136837in}}{\pgfqpoint{2.864111in}{3.136837in}}%
\pgfpathclose%
\pgfusepath{stroke,fill}%
\end{pgfscope}%
\begin{pgfscope}%
\pgfpathrectangle{\pgfqpoint{0.648703in}{0.548769in}}{\pgfqpoint{5.112893in}{3.102590in}}%
\pgfusepath{clip}%
\pgfsetbuttcap%
\pgfsetroundjoin%
\definecolor{currentfill}{rgb}{1.000000,0.498039,0.054902}%
\pgfsetfillcolor{currentfill}%
\pgfsetlinewidth{1.003750pt}%
\definecolor{currentstroke}{rgb}{1.000000,0.498039,0.054902}%
\pgfsetstrokecolor{currentstroke}%
\pgfsetdash{}{0pt}%
\pgfpathmoveto{\pgfqpoint{3.475070in}{3.244681in}}%
\pgfpathcurveto{\pgfqpoint{3.486120in}{3.244681in}}{\pgfqpoint{3.496719in}{3.249072in}}{\pgfqpoint{3.504533in}{3.256885in}}%
\pgfpathcurveto{\pgfqpoint{3.512346in}{3.264699in}}{\pgfqpoint{3.516737in}{3.275298in}}{\pgfqpoint{3.516737in}{3.286348in}}%
\pgfpathcurveto{\pgfqpoint{3.516737in}{3.297398in}}{\pgfqpoint{3.512346in}{3.307997in}}{\pgfqpoint{3.504533in}{3.315811in}}%
\pgfpathcurveto{\pgfqpoint{3.496719in}{3.323624in}}{\pgfqpoint{3.486120in}{3.328015in}}{\pgfqpoint{3.475070in}{3.328015in}}%
\pgfpathcurveto{\pgfqpoint{3.464020in}{3.328015in}}{\pgfqpoint{3.453421in}{3.323624in}}{\pgfqpoint{3.445607in}{3.315811in}}%
\pgfpathcurveto{\pgfqpoint{3.437794in}{3.307997in}}{\pgfqpoint{3.433403in}{3.297398in}}{\pgfqpoint{3.433403in}{3.286348in}}%
\pgfpathcurveto{\pgfqpoint{3.433403in}{3.275298in}}{\pgfqpoint{3.437794in}{3.264699in}}{\pgfqpoint{3.445607in}{3.256885in}}%
\pgfpathcurveto{\pgfqpoint{3.453421in}{3.249072in}}{\pgfqpoint{3.464020in}{3.244681in}}{\pgfqpoint{3.475070in}{3.244681in}}%
\pgfpathclose%
\pgfusepath{stroke,fill}%
\end{pgfscope}%
\begin{pgfscope}%
\pgfpathrectangle{\pgfqpoint{0.648703in}{0.548769in}}{\pgfqpoint{5.112893in}{3.102590in}}%
\pgfusepath{clip}%
\pgfsetbuttcap%
\pgfsetroundjoin%
\definecolor{currentfill}{rgb}{1.000000,0.498039,0.054902}%
\pgfsetfillcolor{currentfill}%
\pgfsetlinewidth{1.003750pt}%
\definecolor{currentstroke}{rgb}{1.000000,0.498039,0.054902}%
\pgfsetstrokecolor{currentstroke}%
\pgfsetdash{}{0pt}%
\pgfpathmoveto{\pgfqpoint{3.384493in}{3.190759in}}%
\pgfpathcurveto{\pgfqpoint{3.395543in}{3.190759in}}{\pgfqpoint{3.406142in}{3.195150in}}{\pgfqpoint{3.413956in}{3.202963in}}%
\pgfpathcurveto{\pgfqpoint{3.421769in}{3.210777in}}{\pgfqpoint{3.426159in}{3.221376in}}{\pgfqpoint{3.426159in}{3.232426in}}%
\pgfpathcurveto{\pgfqpoint{3.426159in}{3.243476in}}{\pgfqpoint{3.421769in}{3.254075in}}{\pgfqpoint{3.413956in}{3.261889in}}%
\pgfpathcurveto{\pgfqpoint{3.406142in}{3.269702in}}{\pgfqpoint{3.395543in}{3.274093in}}{\pgfqpoint{3.384493in}{3.274093in}}%
\pgfpathcurveto{\pgfqpoint{3.373443in}{3.274093in}}{\pgfqpoint{3.362844in}{3.269702in}}{\pgfqpoint{3.355030in}{3.261889in}}%
\pgfpathcurveto{\pgfqpoint{3.347216in}{3.254075in}}{\pgfqpoint{3.342826in}{3.243476in}}{\pgfqpoint{3.342826in}{3.232426in}}%
\pgfpathcurveto{\pgfqpoint{3.342826in}{3.221376in}}{\pgfqpoint{3.347216in}{3.210777in}}{\pgfqpoint{3.355030in}{3.202963in}}%
\pgfpathcurveto{\pgfqpoint{3.362844in}{3.195150in}}{\pgfqpoint{3.373443in}{3.190759in}}{\pgfqpoint{3.384493in}{3.190759in}}%
\pgfpathclose%
\pgfusepath{stroke,fill}%
\end{pgfscope}%
\begin{pgfscope}%
\pgfpathrectangle{\pgfqpoint{0.648703in}{0.548769in}}{\pgfqpoint{5.112893in}{3.102590in}}%
\pgfusepath{clip}%
\pgfsetbuttcap%
\pgfsetroundjoin%
\definecolor{currentfill}{rgb}{0.121569,0.466667,0.705882}%
\pgfsetfillcolor{currentfill}%
\pgfsetlinewidth{1.003750pt}%
\definecolor{currentstroke}{rgb}{0.121569,0.466667,0.705882}%
\pgfsetstrokecolor{currentstroke}%
\pgfsetdash{}{0pt}%
\pgfpathmoveto{\pgfqpoint{0.840359in}{0.664720in}}%
\pgfpathcurveto{\pgfqpoint{0.851409in}{0.664720in}}{\pgfqpoint{0.862008in}{0.669111in}}{\pgfqpoint{0.869822in}{0.676924in}}%
\pgfpathcurveto{\pgfqpoint{0.877635in}{0.684738in}}{\pgfqpoint{0.882026in}{0.695337in}}{\pgfqpoint{0.882026in}{0.706387in}}%
\pgfpathcurveto{\pgfqpoint{0.882026in}{0.717437in}}{\pgfqpoint{0.877635in}{0.728036in}}{\pgfqpoint{0.869822in}{0.735850in}}%
\pgfpathcurveto{\pgfqpoint{0.862008in}{0.743663in}}{\pgfqpoint{0.851409in}{0.748054in}}{\pgfqpoint{0.840359in}{0.748054in}}%
\pgfpathcurveto{\pgfqpoint{0.829309in}{0.748054in}}{\pgfqpoint{0.818710in}{0.743663in}}{\pgfqpoint{0.810896in}{0.735850in}}%
\pgfpathcurveto{\pgfqpoint{0.803083in}{0.728036in}}{\pgfqpoint{0.798692in}{0.717437in}}{\pgfqpoint{0.798692in}{0.706387in}}%
\pgfpathcurveto{\pgfqpoint{0.798692in}{0.695337in}}{\pgfqpoint{0.803083in}{0.684738in}}{\pgfqpoint{0.810896in}{0.676924in}}%
\pgfpathcurveto{\pgfqpoint{0.818710in}{0.669111in}}{\pgfqpoint{0.829309in}{0.664720in}}{\pgfqpoint{0.840359in}{0.664720in}}%
\pgfpathclose%
\pgfusepath{stroke,fill}%
\end{pgfscope}%
\begin{pgfscope}%
\pgfpathrectangle{\pgfqpoint{0.648703in}{0.548769in}}{\pgfqpoint{5.112893in}{3.102590in}}%
\pgfusepath{clip}%
\pgfsetbuttcap%
\pgfsetroundjoin%
\definecolor{currentfill}{rgb}{1.000000,0.498039,0.054902}%
\pgfsetfillcolor{currentfill}%
\pgfsetlinewidth{1.003750pt}%
\definecolor{currentstroke}{rgb}{1.000000,0.498039,0.054902}%
\pgfsetstrokecolor{currentstroke}%
\pgfsetdash{}{0pt}%
\pgfpathmoveto{\pgfqpoint{3.564494in}{3.145133in}}%
\pgfpathcurveto{\pgfqpoint{3.575545in}{3.145133in}}{\pgfqpoint{3.586144in}{3.149523in}}{\pgfqpoint{3.593957in}{3.157337in}}%
\pgfpathcurveto{\pgfqpoint{3.601771in}{3.165151in}}{\pgfqpoint{3.606161in}{3.175750in}}{\pgfqpoint{3.606161in}{3.186800in}}%
\pgfpathcurveto{\pgfqpoint{3.606161in}{3.197850in}}{\pgfqpoint{3.601771in}{3.208449in}}{\pgfqpoint{3.593957in}{3.216262in}}%
\pgfpathcurveto{\pgfqpoint{3.586144in}{3.224076in}}{\pgfqpoint{3.575545in}{3.228466in}}{\pgfqpoint{3.564494in}{3.228466in}}%
\pgfpathcurveto{\pgfqpoint{3.553444in}{3.228466in}}{\pgfqpoint{3.542845in}{3.224076in}}{\pgfqpoint{3.535032in}{3.216262in}}%
\pgfpathcurveto{\pgfqpoint{3.527218in}{3.208449in}}{\pgfqpoint{3.522828in}{3.197850in}}{\pgfqpoint{3.522828in}{3.186800in}}%
\pgfpathcurveto{\pgfqpoint{3.522828in}{3.175750in}}{\pgfqpoint{3.527218in}{3.165151in}}{\pgfqpoint{3.535032in}{3.157337in}}%
\pgfpathcurveto{\pgfqpoint{3.542845in}{3.149523in}}{\pgfqpoint{3.553444in}{3.145133in}}{\pgfqpoint{3.564494in}{3.145133in}}%
\pgfpathclose%
\pgfusepath{stroke,fill}%
\end{pgfscope}%
\begin{pgfscope}%
\pgfpathrectangle{\pgfqpoint{0.648703in}{0.548769in}}{\pgfqpoint{5.112893in}{3.102590in}}%
\pgfusepath{clip}%
\pgfsetbuttcap%
\pgfsetroundjoin%
\definecolor{currentfill}{rgb}{0.121569,0.466667,0.705882}%
\pgfsetfillcolor{currentfill}%
\pgfsetlinewidth{1.003750pt}%
\definecolor{currentstroke}{rgb}{0.121569,0.466667,0.705882}%
\pgfsetstrokecolor{currentstroke}%
\pgfsetdash{}{0pt}%
\pgfpathmoveto{\pgfqpoint{0.814347in}{0.648129in}}%
\pgfpathcurveto{\pgfqpoint{0.825397in}{0.648129in}}{\pgfqpoint{0.835996in}{0.652519in}}{\pgfqpoint{0.843810in}{0.660333in}}%
\pgfpathcurveto{\pgfqpoint{0.851623in}{0.668146in}}{\pgfqpoint{0.856013in}{0.678745in}}{\pgfqpoint{0.856013in}{0.689796in}}%
\pgfpathcurveto{\pgfqpoint{0.856013in}{0.700846in}}{\pgfqpoint{0.851623in}{0.711445in}}{\pgfqpoint{0.843810in}{0.719258in}}%
\pgfpathcurveto{\pgfqpoint{0.835996in}{0.727072in}}{\pgfqpoint{0.825397in}{0.731462in}}{\pgfqpoint{0.814347in}{0.731462in}}%
\pgfpathcurveto{\pgfqpoint{0.803297in}{0.731462in}}{\pgfqpoint{0.792698in}{0.727072in}}{\pgfqpoint{0.784884in}{0.719258in}}%
\pgfpathcurveto{\pgfqpoint{0.777070in}{0.711445in}}{\pgfqpoint{0.772680in}{0.700846in}}{\pgfqpoint{0.772680in}{0.689796in}}%
\pgfpathcurveto{\pgfqpoint{0.772680in}{0.678745in}}{\pgfqpoint{0.777070in}{0.668146in}}{\pgfqpoint{0.784884in}{0.660333in}}%
\pgfpathcurveto{\pgfqpoint{0.792698in}{0.652519in}}{\pgfqpoint{0.803297in}{0.648129in}}{\pgfqpoint{0.814347in}{0.648129in}}%
\pgfpathclose%
\pgfusepath{stroke,fill}%
\end{pgfscope}%
\begin{pgfscope}%
\pgfpathrectangle{\pgfqpoint{0.648703in}{0.548769in}}{\pgfqpoint{5.112893in}{3.102590in}}%
\pgfusepath{clip}%
\pgfsetbuttcap%
\pgfsetroundjoin%
\definecolor{currentfill}{rgb}{1.000000,0.498039,0.054902}%
\pgfsetfillcolor{currentfill}%
\pgfsetlinewidth{1.003750pt}%
\definecolor{currentstroke}{rgb}{1.000000,0.498039,0.054902}%
\pgfsetstrokecolor{currentstroke}%
\pgfsetdash{}{0pt}%
\pgfpathmoveto{\pgfqpoint{3.304295in}{3.149281in}}%
\pgfpathcurveto{\pgfqpoint{3.315345in}{3.149281in}}{\pgfqpoint{3.325944in}{3.153671in}}{\pgfqpoint{3.333757in}{3.161485in}}%
\pgfpathcurveto{\pgfqpoint{3.341571in}{3.169298in}}{\pgfqpoint{3.345961in}{3.179897in}}{\pgfqpoint{3.345961in}{3.190948in}}%
\pgfpathcurveto{\pgfqpoint{3.345961in}{3.201998in}}{\pgfqpoint{3.341571in}{3.212597in}}{\pgfqpoint{3.333757in}{3.220410in}}%
\pgfpathcurveto{\pgfqpoint{3.325944in}{3.228224in}}{\pgfqpoint{3.315345in}{3.232614in}}{\pgfqpoint{3.304295in}{3.232614in}}%
\pgfpathcurveto{\pgfqpoint{3.293245in}{3.232614in}}{\pgfqpoint{3.282646in}{3.228224in}}{\pgfqpoint{3.274832in}{3.220410in}}%
\pgfpathcurveto{\pgfqpoint{3.267018in}{3.212597in}}{\pgfqpoint{3.262628in}{3.201998in}}{\pgfqpoint{3.262628in}{3.190948in}}%
\pgfpathcurveto{\pgfqpoint{3.262628in}{3.179897in}}{\pgfqpoint{3.267018in}{3.169298in}}{\pgfqpoint{3.274832in}{3.161485in}}%
\pgfpathcurveto{\pgfqpoint{3.282646in}{3.153671in}}{\pgfqpoint{3.293245in}{3.149281in}}{\pgfqpoint{3.304295in}{3.149281in}}%
\pgfpathclose%
\pgfusepath{stroke,fill}%
\end{pgfscope}%
\begin{pgfscope}%
\pgfpathrectangle{\pgfqpoint{0.648703in}{0.548769in}}{\pgfqpoint{5.112893in}{3.102590in}}%
\pgfusepath{clip}%
\pgfsetbuttcap%
\pgfsetroundjoin%
\definecolor{currentfill}{rgb}{1.000000,0.498039,0.054902}%
\pgfsetfillcolor{currentfill}%
\pgfsetlinewidth{1.003750pt}%
\definecolor{currentstroke}{rgb}{1.000000,0.498039,0.054902}%
\pgfsetstrokecolor{currentstroke}%
\pgfsetdash{}{0pt}%
\pgfpathmoveto{\pgfqpoint{3.361246in}{3.124394in}}%
\pgfpathcurveto{\pgfqpoint{3.372296in}{3.124394in}}{\pgfqpoint{3.382895in}{3.128784in}}{\pgfqpoint{3.390709in}{3.136598in}}%
\pgfpathcurveto{\pgfqpoint{3.398522in}{3.144411in}}{\pgfqpoint{3.402913in}{3.155010in}}{\pgfqpoint{3.402913in}{3.166060in}}%
\pgfpathcurveto{\pgfqpoint{3.402913in}{3.177111in}}{\pgfqpoint{3.398522in}{3.187710in}}{\pgfqpoint{3.390709in}{3.195523in}}%
\pgfpathcurveto{\pgfqpoint{3.382895in}{3.203337in}}{\pgfqpoint{3.372296in}{3.207727in}}{\pgfqpoint{3.361246in}{3.207727in}}%
\pgfpathcurveto{\pgfqpoint{3.350196in}{3.207727in}}{\pgfqpoint{3.339597in}{3.203337in}}{\pgfqpoint{3.331783in}{3.195523in}}%
\pgfpathcurveto{\pgfqpoint{3.323969in}{3.187710in}}{\pgfqpoint{3.319579in}{3.177111in}}{\pgfqpoint{3.319579in}{3.166060in}}%
\pgfpathcurveto{\pgfqpoint{3.319579in}{3.155010in}}{\pgfqpoint{3.323969in}{3.144411in}}{\pgfqpoint{3.331783in}{3.136598in}}%
\pgfpathcurveto{\pgfqpoint{3.339597in}{3.128784in}}{\pgfqpoint{3.350196in}{3.124394in}}{\pgfqpoint{3.361246in}{3.124394in}}%
\pgfpathclose%
\pgfusepath{stroke,fill}%
\end{pgfscope}%
\begin{pgfscope}%
\pgfpathrectangle{\pgfqpoint{0.648703in}{0.548769in}}{\pgfqpoint{5.112893in}{3.102590in}}%
\pgfusepath{clip}%
\pgfsetbuttcap%
\pgfsetroundjoin%
\definecolor{currentfill}{rgb}{1.000000,0.498039,0.054902}%
\pgfsetfillcolor{currentfill}%
\pgfsetlinewidth{1.003750pt}%
\definecolor{currentstroke}{rgb}{1.000000,0.498039,0.054902}%
\pgfsetstrokecolor{currentstroke}%
\pgfsetdash{}{0pt}%
\pgfpathmoveto{\pgfqpoint{3.382972in}{3.136837in}}%
\pgfpathcurveto{\pgfqpoint{3.394022in}{3.136837in}}{\pgfqpoint{3.404621in}{3.141228in}}{\pgfqpoint{3.412435in}{3.149041in}}%
\pgfpathcurveto{\pgfqpoint{3.420249in}{3.156855in}}{\pgfqpoint{3.424639in}{3.167454in}}{\pgfqpoint{3.424639in}{3.178504in}}%
\pgfpathcurveto{\pgfqpoint{3.424639in}{3.189554in}}{\pgfqpoint{3.420249in}{3.200153in}}{\pgfqpoint{3.412435in}{3.207967in}}%
\pgfpathcurveto{\pgfqpoint{3.404621in}{3.215780in}}{\pgfqpoint{3.394022in}{3.220171in}}{\pgfqpoint{3.382972in}{3.220171in}}%
\pgfpathcurveto{\pgfqpoint{3.371922in}{3.220171in}}{\pgfqpoint{3.361323in}{3.215780in}}{\pgfqpoint{3.353509in}{3.207967in}}%
\pgfpathcurveto{\pgfqpoint{3.345696in}{3.200153in}}{\pgfqpoint{3.341306in}{3.189554in}}{\pgfqpoint{3.341306in}{3.178504in}}%
\pgfpathcurveto{\pgfqpoint{3.341306in}{3.167454in}}{\pgfqpoint{3.345696in}{3.156855in}}{\pgfqpoint{3.353509in}{3.149041in}}%
\pgfpathcurveto{\pgfqpoint{3.361323in}{3.141228in}}{\pgfqpoint{3.371922in}{3.136837in}}{\pgfqpoint{3.382972in}{3.136837in}}%
\pgfpathclose%
\pgfusepath{stroke,fill}%
\end{pgfscope}%
\begin{pgfscope}%
\pgfpathrectangle{\pgfqpoint{0.648703in}{0.548769in}}{\pgfqpoint{5.112893in}{3.102590in}}%
\pgfusepath{clip}%
\pgfsetbuttcap%
\pgfsetroundjoin%
\definecolor{currentfill}{rgb}{1.000000,0.498039,0.054902}%
\pgfsetfillcolor{currentfill}%
\pgfsetlinewidth{1.003750pt}%
\definecolor{currentstroke}{rgb}{1.000000,0.498039,0.054902}%
\pgfsetstrokecolor{currentstroke}%
\pgfsetdash{}{0pt}%
\pgfpathmoveto{\pgfqpoint{2.093506in}{3.140985in}}%
\pgfpathcurveto{\pgfqpoint{2.104556in}{3.140985in}}{\pgfqpoint{2.115155in}{3.145375in}}{\pgfqpoint{2.122969in}{3.153189in}}%
\pgfpathcurveto{\pgfqpoint{2.130782in}{3.161003in}}{\pgfqpoint{2.135173in}{3.171602in}}{\pgfqpoint{2.135173in}{3.182652in}}%
\pgfpathcurveto{\pgfqpoint{2.135173in}{3.193702in}}{\pgfqpoint{2.130782in}{3.204301in}}{\pgfqpoint{2.122969in}{3.212115in}}%
\pgfpathcurveto{\pgfqpoint{2.115155in}{3.219928in}}{\pgfqpoint{2.104556in}{3.224319in}}{\pgfqpoint{2.093506in}{3.224319in}}%
\pgfpathcurveto{\pgfqpoint{2.082456in}{3.224319in}}{\pgfqpoint{2.071857in}{3.219928in}}{\pgfqpoint{2.064043in}{3.212115in}}%
\pgfpathcurveto{\pgfqpoint{2.056230in}{3.204301in}}{\pgfqpoint{2.051839in}{3.193702in}}{\pgfqpoint{2.051839in}{3.182652in}}%
\pgfpathcurveto{\pgfqpoint{2.051839in}{3.171602in}}{\pgfqpoint{2.056230in}{3.161003in}}{\pgfqpoint{2.064043in}{3.153189in}}%
\pgfpathcurveto{\pgfqpoint{2.071857in}{3.145375in}}{\pgfqpoint{2.082456in}{3.140985in}}{\pgfqpoint{2.093506in}{3.140985in}}%
\pgfpathclose%
\pgfusepath{stroke,fill}%
\end{pgfscope}%
\begin{pgfscope}%
\pgfpathrectangle{\pgfqpoint{0.648703in}{0.548769in}}{\pgfqpoint{5.112893in}{3.102590in}}%
\pgfusepath{clip}%
\pgfsetbuttcap%
\pgfsetroundjoin%
\definecolor{currentfill}{rgb}{1.000000,0.498039,0.054902}%
\pgfsetfillcolor{currentfill}%
\pgfsetlinewidth{1.003750pt}%
\definecolor{currentstroke}{rgb}{1.000000,0.498039,0.054902}%
\pgfsetstrokecolor{currentstroke}%
\pgfsetdash{}{0pt}%
\pgfpathmoveto{\pgfqpoint{3.099764in}{3.145133in}}%
\pgfpathcurveto{\pgfqpoint{3.110814in}{3.145133in}}{\pgfqpoint{3.121413in}{3.149523in}}{\pgfqpoint{3.129226in}{3.157337in}}%
\pgfpathcurveto{\pgfqpoint{3.137040in}{3.165151in}}{\pgfqpoint{3.141430in}{3.175750in}}{\pgfqpoint{3.141430in}{3.186800in}}%
\pgfpathcurveto{\pgfqpoint{3.141430in}{3.197850in}}{\pgfqpoint{3.137040in}{3.208449in}}{\pgfqpoint{3.129226in}{3.216262in}}%
\pgfpathcurveto{\pgfqpoint{3.121413in}{3.224076in}}{\pgfqpoint{3.110814in}{3.228466in}}{\pgfqpoint{3.099764in}{3.228466in}}%
\pgfpathcurveto{\pgfqpoint{3.088713in}{3.228466in}}{\pgfqpoint{3.078114in}{3.224076in}}{\pgfqpoint{3.070301in}{3.216262in}}%
\pgfpathcurveto{\pgfqpoint{3.062487in}{3.208449in}}{\pgfqpoint{3.058097in}{3.197850in}}{\pgfqpoint{3.058097in}{3.186800in}}%
\pgfpathcurveto{\pgfqpoint{3.058097in}{3.175750in}}{\pgfqpoint{3.062487in}{3.165151in}}{\pgfqpoint{3.070301in}{3.157337in}}%
\pgfpathcurveto{\pgfqpoint{3.078114in}{3.149523in}}{\pgfqpoint{3.088713in}{3.145133in}}{\pgfqpoint{3.099764in}{3.145133in}}%
\pgfpathclose%
\pgfusepath{stroke,fill}%
\end{pgfscope}%
\begin{pgfscope}%
\pgfpathrectangle{\pgfqpoint{0.648703in}{0.548769in}}{\pgfqpoint{5.112893in}{3.102590in}}%
\pgfusepath{clip}%
\pgfsetbuttcap%
\pgfsetroundjoin%
\definecolor{currentfill}{rgb}{1.000000,0.498039,0.054902}%
\pgfsetfillcolor{currentfill}%
\pgfsetlinewidth{1.003750pt}%
\definecolor{currentstroke}{rgb}{1.000000,0.498039,0.054902}%
\pgfsetstrokecolor{currentstroke}%
\pgfsetdash{}{0pt}%
\pgfpathmoveto{\pgfqpoint{2.350613in}{3.132690in}}%
\pgfpathcurveto{\pgfqpoint{2.361663in}{3.132690in}}{\pgfqpoint{2.372262in}{3.137080in}}{\pgfqpoint{2.380076in}{3.144893in}}%
\pgfpathcurveto{\pgfqpoint{2.387889in}{3.152707in}}{\pgfqpoint{2.392279in}{3.163306in}}{\pgfqpoint{2.392279in}{3.174356in}}%
\pgfpathcurveto{\pgfqpoint{2.392279in}{3.185406in}}{\pgfqpoint{2.387889in}{3.196005in}}{\pgfqpoint{2.380076in}{3.203819in}}%
\pgfpathcurveto{\pgfqpoint{2.372262in}{3.211633in}}{\pgfqpoint{2.361663in}{3.216023in}}{\pgfqpoint{2.350613in}{3.216023in}}%
\pgfpathcurveto{\pgfqpoint{2.339563in}{3.216023in}}{\pgfqpoint{2.328964in}{3.211633in}}{\pgfqpoint{2.321150in}{3.203819in}}%
\pgfpathcurveto{\pgfqpoint{2.313336in}{3.196005in}}{\pgfqpoint{2.308946in}{3.185406in}}{\pgfqpoint{2.308946in}{3.174356in}}%
\pgfpathcurveto{\pgfqpoint{2.308946in}{3.163306in}}{\pgfqpoint{2.313336in}{3.152707in}}{\pgfqpoint{2.321150in}{3.144893in}}%
\pgfpathcurveto{\pgfqpoint{2.328964in}{3.137080in}}{\pgfqpoint{2.339563in}{3.132690in}}{\pgfqpoint{2.350613in}{3.132690in}}%
\pgfpathclose%
\pgfusepath{stroke,fill}%
\end{pgfscope}%
\begin{pgfscope}%
\pgfpathrectangle{\pgfqpoint{0.648703in}{0.548769in}}{\pgfqpoint{5.112893in}{3.102590in}}%
\pgfusepath{clip}%
\pgfsetbuttcap%
\pgfsetroundjoin%
\definecolor{currentfill}{rgb}{0.121569,0.466667,0.705882}%
\pgfsetfillcolor{currentfill}%
\pgfsetlinewidth{1.003750pt}%
\definecolor{currentstroke}{rgb}{0.121569,0.466667,0.705882}%
\pgfsetstrokecolor{currentstroke}%
\pgfsetdash{}{0pt}%
\pgfpathmoveto{\pgfqpoint{0.848896in}{0.660572in}}%
\pgfpathcurveto{\pgfqpoint{0.859946in}{0.660572in}}{\pgfqpoint{0.870545in}{0.664963in}}{\pgfqpoint{0.878359in}{0.672776in}}%
\pgfpathcurveto{\pgfqpoint{0.886173in}{0.680590in}}{\pgfqpoint{0.890563in}{0.691189in}}{\pgfqpoint{0.890563in}{0.702239in}}%
\pgfpathcurveto{\pgfqpoint{0.890563in}{0.713289in}}{\pgfqpoint{0.886173in}{0.723888in}}{\pgfqpoint{0.878359in}{0.731702in}}%
\pgfpathcurveto{\pgfqpoint{0.870545in}{0.739516in}}{\pgfqpoint{0.859946in}{0.743906in}}{\pgfqpoint{0.848896in}{0.743906in}}%
\pgfpathcurveto{\pgfqpoint{0.837846in}{0.743906in}}{\pgfqpoint{0.827247in}{0.739516in}}{\pgfqpoint{0.819433in}{0.731702in}}%
\pgfpathcurveto{\pgfqpoint{0.811620in}{0.723888in}}{\pgfqpoint{0.807229in}{0.713289in}}{\pgfqpoint{0.807229in}{0.702239in}}%
\pgfpathcurveto{\pgfqpoint{0.807229in}{0.691189in}}{\pgfqpoint{0.811620in}{0.680590in}}{\pgfqpoint{0.819433in}{0.672776in}}%
\pgfpathcurveto{\pgfqpoint{0.827247in}{0.664963in}}{\pgfqpoint{0.837846in}{0.660572in}}{\pgfqpoint{0.848896in}{0.660572in}}%
\pgfpathclose%
\pgfusepath{stroke,fill}%
\end{pgfscope}%
\begin{pgfscope}%
\pgfpathrectangle{\pgfqpoint{0.648703in}{0.548769in}}{\pgfqpoint{5.112893in}{3.102590in}}%
\pgfusepath{clip}%
\pgfsetbuttcap%
\pgfsetroundjoin%
\definecolor{currentfill}{rgb}{1.000000,0.498039,0.054902}%
\pgfsetfillcolor{currentfill}%
\pgfsetlinewidth{1.003750pt}%
\definecolor{currentstroke}{rgb}{1.000000,0.498039,0.054902}%
\pgfsetstrokecolor{currentstroke}%
\pgfsetdash{}{0pt}%
\pgfpathmoveto{\pgfqpoint{3.227365in}{3.136837in}}%
\pgfpathcurveto{\pgfqpoint{3.238415in}{3.136837in}}{\pgfqpoint{3.249014in}{3.141228in}}{\pgfqpoint{3.256827in}{3.149041in}}%
\pgfpathcurveto{\pgfqpoint{3.264641in}{3.156855in}}{\pgfqpoint{3.269031in}{3.167454in}}{\pgfqpoint{3.269031in}{3.178504in}}%
\pgfpathcurveto{\pgfqpoint{3.269031in}{3.189554in}}{\pgfqpoint{3.264641in}{3.200153in}}{\pgfqpoint{3.256827in}{3.207967in}}%
\pgfpathcurveto{\pgfqpoint{3.249014in}{3.215780in}}{\pgfqpoint{3.238415in}{3.220171in}}{\pgfqpoint{3.227365in}{3.220171in}}%
\pgfpathcurveto{\pgfqpoint{3.216315in}{3.220171in}}{\pgfqpoint{3.205716in}{3.215780in}}{\pgfqpoint{3.197902in}{3.207967in}}%
\pgfpathcurveto{\pgfqpoint{3.190088in}{3.200153in}}{\pgfqpoint{3.185698in}{3.189554in}}{\pgfqpoint{3.185698in}{3.178504in}}%
\pgfpathcurveto{\pgfqpoint{3.185698in}{3.167454in}}{\pgfqpoint{3.190088in}{3.156855in}}{\pgfqpoint{3.197902in}{3.149041in}}%
\pgfpathcurveto{\pgfqpoint{3.205716in}{3.141228in}}{\pgfqpoint{3.216315in}{3.136837in}}{\pgfqpoint{3.227365in}{3.136837in}}%
\pgfpathclose%
\pgfusepath{stroke,fill}%
\end{pgfscope}%
\begin{pgfscope}%
\pgfpathrectangle{\pgfqpoint{0.648703in}{0.548769in}}{\pgfqpoint{5.112893in}{3.102590in}}%
\pgfusepath{clip}%
\pgfsetbuttcap%
\pgfsetroundjoin%
\definecolor{currentfill}{rgb}{1.000000,0.498039,0.054902}%
\pgfsetfillcolor{currentfill}%
\pgfsetlinewidth{1.003750pt}%
\definecolor{currentstroke}{rgb}{1.000000,0.498039,0.054902}%
\pgfsetstrokecolor{currentstroke}%
\pgfsetdash{}{0pt}%
\pgfpathmoveto{\pgfqpoint{3.398272in}{3.157577in}}%
\pgfpathcurveto{\pgfqpoint{3.409322in}{3.157577in}}{\pgfqpoint{3.419921in}{3.161967in}}{\pgfqpoint{3.427735in}{3.169780in}}%
\pgfpathcurveto{\pgfqpoint{3.435548in}{3.177594in}}{\pgfqpoint{3.439938in}{3.188193in}}{\pgfqpoint{3.439938in}{3.199243in}}%
\pgfpathcurveto{\pgfqpoint{3.439938in}{3.210293in}}{\pgfqpoint{3.435548in}{3.220892in}}{\pgfqpoint{3.427735in}{3.228706in}}%
\pgfpathcurveto{\pgfqpoint{3.419921in}{3.236520in}}{\pgfqpoint{3.409322in}{3.240910in}}{\pgfqpoint{3.398272in}{3.240910in}}%
\pgfpathcurveto{\pgfqpoint{3.387222in}{3.240910in}}{\pgfqpoint{3.376623in}{3.236520in}}{\pgfqpoint{3.368809in}{3.228706in}}%
\pgfpathcurveto{\pgfqpoint{3.360995in}{3.220892in}}{\pgfqpoint{3.356605in}{3.210293in}}{\pgfqpoint{3.356605in}{3.199243in}}%
\pgfpathcurveto{\pgfqpoint{3.356605in}{3.188193in}}{\pgfqpoint{3.360995in}{3.177594in}}{\pgfqpoint{3.368809in}{3.169780in}}%
\pgfpathcurveto{\pgfqpoint{3.376623in}{3.161967in}}{\pgfqpoint{3.387222in}{3.157577in}}{\pgfqpoint{3.398272in}{3.157577in}}%
\pgfpathclose%
\pgfusepath{stroke,fill}%
\end{pgfscope}%
\begin{pgfscope}%
\pgfpathrectangle{\pgfqpoint{0.648703in}{0.548769in}}{\pgfqpoint{5.112893in}{3.102590in}}%
\pgfusepath{clip}%
\pgfsetbuttcap%
\pgfsetroundjoin%
\definecolor{currentfill}{rgb}{0.121569,0.466667,0.705882}%
\pgfsetfillcolor{currentfill}%
\pgfsetlinewidth{1.003750pt}%
\definecolor{currentstroke}{rgb}{0.121569,0.466667,0.705882}%
\pgfsetstrokecolor{currentstroke}%
\pgfsetdash{}{0pt}%
\pgfpathmoveto{\pgfqpoint{0.814194in}{0.648129in}}%
\pgfpathcurveto{\pgfqpoint{0.825244in}{0.648129in}}{\pgfqpoint{0.835843in}{0.652519in}}{\pgfqpoint{0.843657in}{0.660333in}}%
\pgfpathcurveto{\pgfqpoint{0.851470in}{0.668146in}}{\pgfqpoint{0.855860in}{0.678745in}}{\pgfqpoint{0.855860in}{0.689796in}}%
\pgfpathcurveto{\pgfqpoint{0.855860in}{0.700846in}}{\pgfqpoint{0.851470in}{0.711445in}}{\pgfqpoint{0.843657in}{0.719258in}}%
\pgfpathcurveto{\pgfqpoint{0.835843in}{0.727072in}}{\pgfqpoint{0.825244in}{0.731462in}}{\pgfqpoint{0.814194in}{0.731462in}}%
\pgfpathcurveto{\pgfqpoint{0.803144in}{0.731462in}}{\pgfqpoint{0.792545in}{0.727072in}}{\pgfqpoint{0.784731in}{0.719258in}}%
\pgfpathcurveto{\pgfqpoint{0.776917in}{0.711445in}}{\pgfqpoint{0.772527in}{0.700846in}}{\pgfqpoint{0.772527in}{0.689796in}}%
\pgfpathcurveto{\pgfqpoint{0.772527in}{0.678745in}}{\pgfqpoint{0.776917in}{0.668146in}}{\pgfqpoint{0.784731in}{0.660333in}}%
\pgfpathcurveto{\pgfqpoint{0.792545in}{0.652519in}}{\pgfqpoint{0.803144in}{0.648129in}}{\pgfqpoint{0.814194in}{0.648129in}}%
\pgfpathclose%
\pgfusepath{stroke,fill}%
\end{pgfscope}%
\begin{pgfscope}%
\pgfpathrectangle{\pgfqpoint{0.648703in}{0.548769in}}{\pgfqpoint{5.112893in}{3.102590in}}%
\pgfusepath{clip}%
\pgfsetbuttcap%
\pgfsetroundjoin%
\definecolor{currentfill}{rgb}{0.121569,0.466667,0.705882}%
\pgfsetfillcolor{currentfill}%
\pgfsetlinewidth{1.003750pt}%
\definecolor{currentstroke}{rgb}{0.121569,0.466667,0.705882}%
\pgfsetstrokecolor{currentstroke}%
\pgfsetdash{}{0pt}%
\pgfpathmoveto{\pgfqpoint{0.816739in}{0.648129in}}%
\pgfpathcurveto{\pgfqpoint{0.827789in}{0.648129in}}{\pgfqpoint{0.838388in}{0.652519in}}{\pgfqpoint{0.846202in}{0.660333in}}%
\pgfpathcurveto{\pgfqpoint{0.854016in}{0.668146in}}{\pgfqpoint{0.858406in}{0.678745in}}{\pgfqpoint{0.858406in}{0.689796in}}%
\pgfpathcurveto{\pgfqpoint{0.858406in}{0.700846in}}{\pgfqpoint{0.854016in}{0.711445in}}{\pgfqpoint{0.846202in}{0.719258in}}%
\pgfpathcurveto{\pgfqpoint{0.838388in}{0.727072in}}{\pgfqpoint{0.827789in}{0.731462in}}{\pgfqpoint{0.816739in}{0.731462in}}%
\pgfpathcurveto{\pgfqpoint{0.805689in}{0.731462in}}{\pgfqpoint{0.795090in}{0.727072in}}{\pgfqpoint{0.787276in}{0.719258in}}%
\pgfpathcurveto{\pgfqpoint{0.779463in}{0.711445in}}{\pgfqpoint{0.775073in}{0.700846in}}{\pgfqpoint{0.775073in}{0.689796in}}%
\pgfpathcurveto{\pgfqpoint{0.775073in}{0.678745in}}{\pgfqpoint{0.779463in}{0.668146in}}{\pgfqpoint{0.787276in}{0.660333in}}%
\pgfpathcurveto{\pgfqpoint{0.795090in}{0.652519in}}{\pgfqpoint{0.805689in}{0.648129in}}{\pgfqpoint{0.816739in}{0.648129in}}%
\pgfpathclose%
\pgfusepath{stroke,fill}%
\end{pgfscope}%
\begin{pgfscope}%
\pgfpathrectangle{\pgfqpoint{0.648703in}{0.548769in}}{\pgfqpoint{5.112893in}{3.102590in}}%
\pgfusepath{clip}%
\pgfsetbuttcap%
\pgfsetroundjoin%
\definecolor{currentfill}{rgb}{1.000000,0.498039,0.054902}%
\pgfsetfillcolor{currentfill}%
\pgfsetlinewidth{1.003750pt}%
\definecolor{currentstroke}{rgb}{1.000000,0.498039,0.054902}%
\pgfsetstrokecolor{currentstroke}%
\pgfsetdash{}{0pt}%
\pgfpathmoveto{\pgfqpoint{3.197497in}{3.132690in}}%
\pgfpathcurveto{\pgfqpoint{3.208547in}{3.132690in}}{\pgfqpoint{3.219146in}{3.137080in}}{\pgfqpoint{3.226959in}{3.144893in}}%
\pgfpathcurveto{\pgfqpoint{3.234773in}{3.152707in}}{\pgfqpoint{3.239163in}{3.163306in}}{\pgfqpoint{3.239163in}{3.174356in}}%
\pgfpathcurveto{\pgfqpoint{3.239163in}{3.185406in}}{\pgfqpoint{3.234773in}{3.196005in}}{\pgfqpoint{3.226959in}{3.203819in}}%
\pgfpathcurveto{\pgfqpoint{3.219146in}{3.211633in}}{\pgfqpoint{3.208547in}{3.216023in}}{\pgfqpoint{3.197497in}{3.216023in}}%
\pgfpathcurveto{\pgfqpoint{3.186447in}{3.216023in}}{\pgfqpoint{3.175848in}{3.211633in}}{\pgfqpoint{3.168034in}{3.203819in}}%
\pgfpathcurveto{\pgfqpoint{3.160220in}{3.196005in}}{\pgfqpoint{3.155830in}{3.185406in}}{\pgfqpoint{3.155830in}{3.174356in}}%
\pgfpathcurveto{\pgfqpoint{3.155830in}{3.163306in}}{\pgfqpoint{3.160220in}{3.152707in}}{\pgfqpoint{3.168034in}{3.144893in}}%
\pgfpathcurveto{\pgfqpoint{3.175848in}{3.137080in}}{\pgfqpoint{3.186447in}{3.132690in}}{\pgfqpoint{3.197497in}{3.132690in}}%
\pgfpathclose%
\pgfusepath{stroke,fill}%
\end{pgfscope}%
\begin{pgfscope}%
\pgfpathrectangle{\pgfqpoint{0.648703in}{0.548769in}}{\pgfqpoint{5.112893in}{3.102590in}}%
\pgfusepath{clip}%
\pgfsetbuttcap%
\pgfsetroundjoin%
\definecolor{currentfill}{rgb}{1.000000,0.498039,0.054902}%
\pgfsetfillcolor{currentfill}%
\pgfsetlinewidth{1.003750pt}%
\definecolor{currentstroke}{rgb}{1.000000,0.498039,0.054902}%
\pgfsetstrokecolor{currentstroke}%
\pgfsetdash{}{0pt}%
\pgfpathmoveto{\pgfqpoint{3.320672in}{3.219794in}}%
\pgfpathcurveto{\pgfqpoint{3.331723in}{3.219794in}}{\pgfqpoint{3.342322in}{3.224185in}}{\pgfqpoint{3.350135in}{3.231998in}}%
\pgfpathcurveto{\pgfqpoint{3.357949in}{3.239812in}}{\pgfqpoint{3.362339in}{3.250411in}}{\pgfqpoint{3.362339in}{3.261461in}}%
\pgfpathcurveto{\pgfqpoint{3.362339in}{3.272511in}}{\pgfqpoint{3.357949in}{3.283110in}}{\pgfqpoint{3.350135in}{3.290924in}}%
\pgfpathcurveto{\pgfqpoint{3.342322in}{3.298737in}}{\pgfqpoint{3.331723in}{3.303128in}}{\pgfqpoint{3.320672in}{3.303128in}}%
\pgfpathcurveto{\pgfqpoint{3.309622in}{3.303128in}}{\pgfqpoint{3.299023in}{3.298737in}}{\pgfqpoint{3.291210in}{3.290924in}}%
\pgfpathcurveto{\pgfqpoint{3.283396in}{3.283110in}}{\pgfqpoint{3.279006in}{3.272511in}}{\pgfqpoint{3.279006in}{3.261461in}}%
\pgfpathcurveto{\pgfqpoint{3.279006in}{3.250411in}}{\pgfqpoint{3.283396in}{3.239812in}}{\pgfqpoint{3.291210in}{3.231998in}}%
\pgfpathcurveto{\pgfqpoint{3.299023in}{3.224185in}}{\pgfqpoint{3.309622in}{3.219794in}}{\pgfqpoint{3.320672in}{3.219794in}}%
\pgfpathclose%
\pgfusepath{stroke,fill}%
\end{pgfscope}%
\begin{pgfscope}%
\pgfpathrectangle{\pgfqpoint{0.648703in}{0.548769in}}{\pgfqpoint{5.112893in}{3.102590in}}%
\pgfusepath{clip}%
\pgfsetbuttcap%
\pgfsetroundjoin%
\definecolor{currentfill}{rgb}{1.000000,0.498039,0.054902}%
\pgfsetfillcolor{currentfill}%
\pgfsetlinewidth{1.003750pt}%
\definecolor{currentstroke}{rgb}{1.000000,0.498039,0.054902}%
\pgfsetstrokecolor{currentstroke}%
\pgfsetdash{}{0pt}%
\pgfpathmoveto{\pgfqpoint{2.703859in}{3.140985in}}%
\pgfpathcurveto{\pgfqpoint{2.714909in}{3.140985in}}{\pgfqpoint{2.725508in}{3.145375in}}{\pgfqpoint{2.733322in}{3.153189in}}%
\pgfpathcurveto{\pgfqpoint{2.741135in}{3.161003in}}{\pgfqpoint{2.745525in}{3.171602in}}{\pgfqpoint{2.745525in}{3.182652in}}%
\pgfpathcurveto{\pgfqpoint{2.745525in}{3.193702in}}{\pgfqpoint{2.741135in}{3.204301in}}{\pgfqpoint{2.733322in}{3.212115in}}%
\pgfpathcurveto{\pgfqpoint{2.725508in}{3.219928in}}{\pgfqpoint{2.714909in}{3.224319in}}{\pgfqpoint{2.703859in}{3.224319in}}%
\pgfpathcurveto{\pgfqpoint{2.692809in}{3.224319in}}{\pgfqpoint{2.682210in}{3.219928in}}{\pgfqpoint{2.674396in}{3.212115in}}%
\pgfpathcurveto{\pgfqpoint{2.666582in}{3.204301in}}{\pgfqpoint{2.662192in}{3.193702in}}{\pgfqpoint{2.662192in}{3.182652in}}%
\pgfpathcurveto{\pgfqpoint{2.662192in}{3.171602in}}{\pgfqpoint{2.666582in}{3.161003in}}{\pgfqpoint{2.674396in}{3.153189in}}%
\pgfpathcurveto{\pgfqpoint{2.682210in}{3.145375in}}{\pgfqpoint{2.692809in}{3.140985in}}{\pgfqpoint{2.703859in}{3.140985in}}%
\pgfpathclose%
\pgfusepath{stroke,fill}%
\end{pgfscope}%
\begin{pgfscope}%
\pgfpathrectangle{\pgfqpoint{0.648703in}{0.548769in}}{\pgfqpoint{5.112893in}{3.102590in}}%
\pgfusepath{clip}%
\pgfsetbuttcap%
\pgfsetroundjoin%
\definecolor{currentfill}{rgb}{1.000000,0.498039,0.054902}%
\pgfsetfillcolor{currentfill}%
\pgfsetlinewidth{1.003750pt}%
\definecolor{currentstroke}{rgb}{1.000000,0.498039,0.054902}%
\pgfsetstrokecolor{currentstroke}%
\pgfsetdash{}{0pt}%
\pgfpathmoveto{\pgfqpoint{2.032686in}{3.145133in}}%
\pgfpathcurveto{\pgfqpoint{2.043736in}{3.145133in}}{\pgfqpoint{2.054336in}{3.149523in}}{\pgfqpoint{2.062149in}{3.157337in}}%
\pgfpathcurveto{\pgfqpoint{2.069963in}{3.165151in}}{\pgfqpoint{2.074353in}{3.175750in}}{\pgfqpoint{2.074353in}{3.186800in}}%
\pgfpathcurveto{\pgfqpoint{2.074353in}{3.197850in}}{\pgfqpoint{2.069963in}{3.208449in}}{\pgfqpoint{2.062149in}{3.216262in}}%
\pgfpathcurveto{\pgfqpoint{2.054336in}{3.224076in}}{\pgfqpoint{2.043736in}{3.228466in}}{\pgfqpoint{2.032686in}{3.228466in}}%
\pgfpathcurveto{\pgfqpoint{2.021636in}{3.228466in}}{\pgfqpoint{2.011037in}{3.224076in}}{\pgfqpoint{2.003224in}{3.216262in}}%
\pgfpathcurveto{\pgfqpoint{1.995410in}{3.208449in}}{\pgfqpoint{1.991020in}{3.197850in}}{\pgfqpoint{1.991020in}{3.186800in}}%
\pgfpathcurveto{\pgfqpoint{1.991020in}{3.175750in}}{\pgfqpoint{1.995410in}{3.165151in}}{\pgfqpoint{2.003224in}{3.157337in}}%
\pgfpathcurveto{\pgfqpoint{2.011037in}{3.149523in}}{\pgfqpoint{2.021636in}{3.145133in}}{\pgfqpoint{2.032686in}{3.145133in}}%
\pgfpathclose%
\pgfusepath{stroke,fill}%
\end{pgfscope}%
\begin{pgfscope}%
\pgfpathrectangle{\pgfqpoint{0.648703in}{0.548769in}}{\pgfqpoint{5.112893in}{3.102590in}}%
\pgfusepath{clip}%
\pgfsetbuttcap%
\pgfsetroundjoin%
\definecolor{currentfill}{rgb}{1.000000,0.498039,0.054902}%
\pgfsetfillcolor{currentfill}%
\pgfsetlinewidth{1.003750pt}%
\definecolor{currentstroke}{rgb}{1.000000,0.498039,0.054902}%
\pgfsetstrokecolor{currentstroke}%
\pgfsetdash{}{0pt}%
\pgfpathmoveto{\pgfqpoint{3.332783in}{3.136837in}}%
\pgfpathcurveto{\pgfqpoint{3.343833in}{3.136837in}}{\pgfqpoint{3.354432in}{3.141228in}}{\pgfqpoint{3.362246in}{3.149041in}}%
\pgfpathcurveto{\pgfqpoint{3.370060in}{3.156855in}}{\pgfqpoint{3.374450in}{3.167454in}}{\pgfqpoint{3.374450in}{3.178504in}}%
\pgfpathcurveto{\pgfqpoint{3.374450in}{3.189554in}}{\pgfqpoint{3.370060in}{3.200153in}}{\pgfqpoint{3.362246in}{3.207967in}}%
\pgfpathcurveto{\pgfqpoint{3.354432in}{3.215780in}}{\pgfqpoint{3.343833in}{3.220171in}}{\pgfqpoint{3.332783in}{3.220171in}}%
\pgfpathcurveto{\pgfqpoint{3.321733in}{3.220171in}}{\pgfqpoint{3.311134in}{3.215780in}}{\pgfqpoint{3.303320in}{3.207967in}}%
\pgfpathcurveto{\pgfqpoint{3.295507in}{3.200153in}}{\pgfqpoint{3.291116in}{3.189554in}}{\pgfqpoint{3.291116in}{3.178504in}}%
\pgfpathcurveto{\pgfqpoint{3.291116in}{3.167454in}}{\pgfqpoint{3.295507in}{3.156855in}}{\pgfqpoint{3.303320in}{3.149041in}}%
\pgfpathcurveto{\pgfqpoint{3.311134in}{3.141228in}}{\pgfqpoint{3.321733in}{3.136837in}}{\pgfqpoint{3.332783in}{3.136837in}}%
\pgfpathclose%
\pgfusepath{stroke,fill}%
\end{pgfscope}%
\begin{pgfscope}%
\pgfpathrectangle{\pgfqpoint{0.648703in}{0.548769in}}{\pgfqpoint{5.112893in}{3.102590in}}%
\pgfusepath{clip}%
\pgfsetbuttcap%
\pgfsetroundjoin%
\definecolor{currentfill}{rgb}{0.121569,0.466667,0.705882}%
\pgfsetfillcolor{currentfill}%
\pgfsetlinewidth{1.003750pt}%
\definecolor{currentstroke}{rgb}{0.121569,0.466667,0.705882}%
\pgfsetstrokecolor{currentstroke}%
\pgfsetdash{}{0pt}%
\pgfpathmoveto{\pgfqpoint{0.814190in}{0.648129in}}%
\pgfpathcurveto{\pgfqpoint{0.825240in}{0.648129in}}{\pgfqpoint{0.835839in}{0.652519in}}{\pgfqpoint{0.843652in}{0.660333in}}%
\pgfpathcurveto{\pgfqpoint{0.851466in}{0.668146in}}{\pgfqpoint{0.855856in}{0.678745in}}{\pgfqpoint{0.855856in}{0.689796in}}%
\pgfpathcurveto{\pgfqpoint{0.855856in}{0.700846in}}{\pgfqpoint{0.851466in}{0.711445in}}{\pgfqpoint{0.843652in}{0.719258in}}%
\pgfpathcurveto{\pgfqpoint{0.835839in}{0.727072in}}{\pgfqpoint{0.825240in}{0.731462in}}{\pgfqpoint{0.814190in}{0.731462in}}%
\pgfpathcurveto{\pgfqpoint{0.803140in}{0.731462in}}{\pgfqpoint{0.792541in}{0.727072in}}{\pgfqpoint{0.784727in}{0.719258in}}%
\pgfpathcurveto{\pgfqpoint{0.776913in}{0.711445in}}{\pgfqpoint{0.772523in}{0.700846in}}{\pgfqpoint{0.772523in}{0.689796in}}%
\pgfpathcurveto{\pgfqpoint{0.772523in}{0.678745in}}{\pgfqpoint{0.776913in}{0.668146in}}{\pgfqpoint{0.784727in}{0.660333in}}%
\pgfpathcurveto{\pgfqpoint{0.792541in}{0.652519in}}{\pgfqpoint{0.803140in}{0.648129in}}{\pgfqpoint{0.814190in}{0.648129in}}%
\pgfpathclose%
\pgfusepath{stroke,fill}%
\end{pgfscope}%
\begin{pgfscope}%
\pgfpathrectangle{\pgfqpoint{0.648703in}{0.548769in}}{\pgfqpoint{5.112893in}{3.102590in}}%
\pgfusepath{clip}%
\pgfsetbuttcap%
\pgfsetroundjoin%
\definecolor{currentfill}{rgb}{1.000000,0.498039,0.054902}%
\pgfsetfillcolor{currentfill}%
\pgfsetlinewidth{1.003750pt}%
\definecolor{currentstroke}{rgb}{1.000000,0.498039,0.054902}%
\pgfsetstrokecolor{currentstroke}%
\pgfsetdash{}{0pt}%
\pgfpathmoveto{\pgfqpoint{3.414907in}{3.132690in}}%
\pgfpathcurveto{\pgfqpoint{3.425957in}{3.132690in}}{\pgfqpoint{3.436556in}{3.137080in}}{\pgfqpoint{3.444370in}{3.144893in}}%
\pgfpathcurveto{\pgfqpoint{3.452184in}{3.152707in}}{\pgfqpoint{3.456574in}{3.163306in}}{\pgfqpoint{3.456574in}{3.174356in}}%
\pgfpathcurveto{\pgfqpoint{3.456574in}{3.185406in}}{\pgfqpoint{3.452184in}{3.196005in}}{\pgfqpoint{3.444370in}{3.203819in}}%
\pgfpathcurveto{\pgfqpoint{3.436556in}{3.211633in}}{\pgfqpoint{3.425957in}{3.216023in}}{\pgfqpoint{3.414907in}{3.216023in}}%
\pgfpathcurveto{\pgfqpoint{3.403857in}{3.216023in}}{\pgfqpoint{3.393258in}{3.211633in}}{\pgfqpoint{3.385444in}{3.203819in}}%
\pgfpathcurveto{\pgfqpoint{3.377631in}{3.196005in}}{\pgfqpoint{3.373241in}{3.185406in}}{\pgfqpoint{3.373241in}{3.174356in}}%
\pgfpathcurveto{\pgfqpoint{3.373241in}{3.163306in}}{\pgfqpoint{3.377631in}{3.152707in}}{\pgfqpoint{3.385444in}{3.144893in}}%
\pgfpathcurveto{\pgfqpoint{3.393258in}{3.137080in}}{\pgfqpoint{3.403857in}{3.132690in}}{\pgfqpoint{3.414907in}{3.132690in}}%
\pgfpathclose%
\pgfusepath{stroke,fill}%
\end{pgfscope}%
\begin{pgfscope}%
\pgfpathrectangle{\pgfqpoint{0.648703in}{0.548769in}}{\pgfqpoint{5.112893in}{3.102590in}}%
\pgfusepath{clip}%
\pgfsetbuttcap%
\pgfsetroundjoin%
\definecolor{currentfill}{rgb}{1.000000,0.498039,0.054902}%
\pgfsetfillcolor{currentfill}%
\pgfsetlinewidth{1.003750pt}%
\definecolor{currentstroke}{rgb}{1.000000,0.498039,0.054902}%
\pgfsetstrokecolor{currentstroke}%
\pgfsetdash{}{0pt}%
\pgfpathmoveto{\pgfqpoint{3.399613in}{3.145133in}}%
\pgfpathcurveto{\pgfqpoint{3.410663in}{3.145133in}}{\pgfqpoint{3.421262in}{3.149523in}}{\pgfqpoint{3.429076in}{3.157337in}}%
\pgfpathcurveto{\pgfqpoint{3.436889in}{3.165151in}}{\pgfqpoint{3.441279in}{3.175750in}}{\pgfqpoint{3.441279in}{3.186800in}}%
\pgfpathcurveto{\pgfqpoint{3.441279in}{3.197850in}}{\pgfqpoint{3.436889in}{3.208449in}}{\pgfqpoint{3.429076in}{3.216262in}}%
\pgfpathcurveto{\pgfqpoint{3.421262in}{3.224076in}}{\pgfqpoint{3.410663in}{3.228466in}}{\pgfqpoint{3.399613in}{3.228466in}}%
\pgfpathcurveto{\pgfqpoint{3.388563in}{3.228466in}}{\pgfqpoint{3.377964in}{3.224076in}}{\pgfqpoint{3.370150in}{3.216262in}}%
\pgfpathcurveto{\pgfqpoint{3.362336in}{3.208449in}}{\pgfqpoint{3.357946in}{3.197850in}}{\pgfqpoint{3.357946in}{3.186800in}}%
\pgfpathcurveto{\pgfqpoint{3.357946in}{3.175750in}}{\pgfqpoint{3.362336in}{3.165151in}}{\pgfqpoint{3.370150in}{3.157337in}}%
\pgfpathcurveto{\pgfqpoint{3.377964in}{3.149523in}}{\pgfqpoint{3.388563in}{3.145133in}}{\pgfqpoint{3.399613in}{3.145133in}}%
\pgfpathclose%
\pgfusepath{stroke,fill}%
\end{pgfscope}%
\begin{pgfscope}%
\pgfpathrectangle{\pgfqpoint{0.648703in}{0.548769in}}{\pgfqpoint{5.112893in}{3.102590in}}%
\pgfusepath{clip}%
\pgfsetbuttcap%
\pgfsetroundjoin%
\definecolor{currentfill}{rgb}{1.000000,0.498039,0.054902}%
\pgfsetfillcolor{currentfill}%
\pgfsetlinewidth{1.003750pt}%
\definecolor{currentstroke}{rgb}{1.000000,0.498039,0.054902}%
\pgfsetstrokecolor{currentstroke}%
\pgfsetdash{}{0pt}%
\pgfpathmoveto{\pgfqpoint{2.102653in}{3.132690in}}%
\pgfpathcurveto{\pgfqpoint{2.113703in}{3.132690in}}{\pgfqpoint{2.124302in}{3.137080in}}{\pgfqpoint{2.132115in}{3.144893in}}%
\pgfpathcurveto{\pgfqpoint{2.139929in}{3.152707in}}{\pgfqpoint{2.144319in}{3.163306in}}{\pgfqpoint{2.144319in}{3.174356in}}%
\pgfpathcurveto{\pgfqpoint{2.144319in}{3.185406in}}{\pgfqpoint{2.139929in}{3.196005in}}{\pgfqpoint{2.132115in}{3.203819in}}%
\pgfpathcurveto{\pgfqpoint{2.124302in}{3.211633in}}{\pgfqpoint{2.113703in}{3.216023in}}{\pgfqpoint{2.102653in}{3.216023in}}%
\pgfpathcurveto{\pgfqpoint{2.091602in}{3.216023in}}{\pgfqpoint{2.081003in}{3.211633in}}{\pgfqpoint{2.073190in}{3.203819in}}%
\pgfpathcurveto{\pgfqpoint{2.065376in}{3.196005in}}{\pgfqpoint{2.060986in}{3.185406in}}{\pgfqpoint{2.060986in}{3.174356in}}%
\pgfpathcurveto{\pgfqpoint{2.060986in}{3.163306in}}{\pgfqpoint{2.065376in}{3.152707in}}{\pgfqpoint{2.073190in}{3.144893in}}%
\pgfpathcurveto{\pgfqpoint{2.081003in}{3.137080in}}{\pgfqpoint{2.091602in}{3.132690in}}{\pgfqpoint{2.102653in}{3.132690in}}%
\pgfpathclose%
\pgfusepath{stroke,fill}%
\end{pgfscope}%
\begin{pgfscope}%
\pgfpathrectangle{\pgfqpoint{0.648703in}{0.548769in}}{\pgfqpoint{5.112893in}{3.102590in}}%
\pgfusepath{clip}%
\pgfsetbuttcap%
\pgfsetroundjoin%
\definecolor{currentfill}{rgb}{1.000000,0.498039,0.054902}%
\pgfsetfillcolor{currentfill}%
\pgfsetlinewidth{1.003750pt}%
\definecolor{currentstroke}{rgb}{1.000000,0.498039,0.054902}%
\pgfsetstrokecolor{currentstroke}%
\pgfsetdash{}{0pt}%
\pgfpathmoveto{\pgfqpoint{4.123607in}{3.157577in}}%
\pgfpathcurveto{\pgfqpoint{4.134657in}{3.157577in}}{\pgfqpoint{4.145256in}{3.161967in}}{\pgfqpoint{4.153070in}{3.169780in}}%
\pgfpathcurveto{\pgfqpoint{4.160884in}{3.177594in}}{\pgfqpoint{4.165274in}{3.188193in}}{\pgfqpoint{4.165274in}{3.199243in}}%
\pgfpathcurveto{\pgfqpoint{4.165274in}{3.210293in}}{\pgfqpoint{4.160884in}{3.220892in}}{\pgfqpoint{4.153070in}{3.228706in}}%
\pgfpathcurveto{\pgfqpoint{4.145256in}{3.236520in}}{\pgfqpoint{4.134657in}{3.240910in}}{\pgfqpoint{4.123607in}{3.240910in}}%
\pgfpathcurveto{\pgfqpoint{4.112557in}{3.240910in}}{\pgfqpoint{4.101958in}{3.236520in}}{\pgfqpoint{4.094144in}{3.228706in}}%
\pgfpathcurveto{\pgfqpoint{4.086331in}{3.220892in}}{\pgfqpoint{4.081941in}{3.210293in}}{\pgfqpoint{4.081941in}{3.199243in}}%
\pgfpathcurveto{\pgfqpoint{4.081941in}{3.188193in}}{\pgfqpoint{4.086331in}{3.177594in}}{\pgfqpoint{4.094144in}{3.169780in}}%
\pgfpathcurveto{\pgfqpoint{4.101958in}{3.161967in}}{\pgfqpoint{4.112557in}{3.157577in}}{\pgfqpoint{4.123607in}{3.157577in}}%
\pgfpathclose%
\pgfusepath{stroke,fill}%
\end{pgfscope}%
\begin{pgfscope}%
\pgfpathrectangle{\pgfqpoint{0.648703in}{0.548769in}}{\pgfqpoint{5.112893in}{3.102590in}}%
\pgfusepath{clip}%
\pgfsetbuttcap%
\pgfsetroundjoin%
\definecolor{currentfill}{rgb}{1.000000,0.498039,0.054902}%
\pgfsetfillcolor{currentfill}%
\pgfsetlinewidth{1.003750pt}%
\definecolor{currentstroke}{rgb}{1.000000,0.498039,0.054902}%
\pgfsetstrokecolor{currentstroke}%
\pgfsetdash{}{0pt}%
\pgfpathmoveto{\pgfqpoint{3.375217in}{3.145133in}}%
\pgfpathcurveto{\pgfqpoint{3.386267in}{3.145133in}}{\pgfqpoint{3.396866in}{3.149523in}}{\pgfqpoint{3.404680in}{3.157337in}}%
\pgfpathcurveto{\pgfqpoint{3.412493in}{3.165151in}}{\pgfqpoint{3.416883in}{3.175750in}}{\pgfqpoint{3.416883in}{3.186800in}}%
\pgfpathcurveto{\pgfqpoint{3.416883in}{3.197850in}}{\pgfqpoint{3.412493in}{3.208449in}}{\pgfqpoint{3.404680in}{3.216262in}}%
\pgfpathcurveto{\pgfqpoint{3.396866in}{3.224076in}}{\pgfqpoint{3.386267in}{3.228466in}}{\pgfqpoint{3.375217in}{3.228466in}}%
\pgfpathcurveto{\pgfqpoint{3.364167in}{3.228466in}}{\pgfqpoint{3.353568in}{3.224076in}}{\pgfqpoint{3.345754in}{3.216262in}}%
\pgfpathcurveto{\pgfqpoint{3.337940in}{3.208449in}}{\pgfqpoint{3.333550in}{3.197850in}}{\pgfqpoint{3.333550in}{3.186800in}}%
\pgfpathcurveto{\pgfqpoint{3.333550in}{3.175750in}}{\pgfqpoint{3.337940in}{3.165151in}}{\pgfqpoint{3.345754in}{3.157337in}}%
\pgfpathcurveto{\pgfqpoint{3.353568in}{3.149523in}}{\pgfqpoint{3.364167in}{3.145133in}}{\pgfqpoint{3.375217in}{3.145133in}}%
\pgfpathclose%
\pgfusepath{stroke,fill}%
\end{pgfscope}%
\begin{pgfscope}%
\pgfpathrectangle{\pgfqpoint{0.648703in}{0.548769in}}{\pgfqpoint{5.112893in}{3.102590in}}%
\pgfusepath{clip}%
\pgfsetbuttcap%
\pgfsetroundjoin%
\definecolor{currentfill}{rgb}{1.000000,0.498039,0.054902}%
\pgfsetfillcolor{currentfill}%
\pgfsetlinewidth{1.003750pt}%
\definecolor{currentstroke}{rgb}{1.000000,0.498039,0.054902}%
\pgfsetstrokecolor{currentstroke}%
\pgfsetdash{}{0pt}%
\pgfpathmoveto{\pgfqpoint{3.387595in}{3.190759in}}%
\pgfpathcurveto{\pgfqpoint{3.398645in}{3.190759in}}{\pgfqpoint{3.409244in}{3.195150in}}{\pgfqpoint{3.417057in}{3.202963in}}%
\pgfpathcurveto{\pgfqpoint{3.424871in}{3.210777in}}{\pgfqpoint{3.429261in}{3.221376in}}{\pgfqpoint{3.429261in}{3.232426in}}%
\pgfpathcurveto{\pgfqpoint{3.429261in}{3.243476in}}{\pgfqpoint{3.424871in}{3.254075in}}{\pgfqpoint{3.417057in}{3.261889in}}%
\pgfpathcurveto{\pgfqpoint{3.409244in}{3.269702in}}{\pgfqpoint{3.398645in}{3.274093in}}{\pgfqpoint{3.387595in}{3.274093in}}%
\pgfpathcurveto{\pgfqpoint{3.376545in}{3.274093in}}{\pgfqpoint{3.365945in}{3.269702in}}{\pgfqpoint{3.358132in}{3.261889in}}%
\pgfpathcurveto{\pgfqpoint{3.350318in}{3.254075in}}{\pgfqpoint{3.345928in}{3.243476in}}{\pgfqpoint{3.345928in}{3.232426in}}%
\pgfpathcurveto{\pgfqpoint{3.345928in}{3.221376in}}{\pgfqpoint{3.350318in}{3.210777in}}{\pgfqpoint{3.358132in}{3.202963in}}%
\pgfpathcurveto{\pgfqpoint{3.365945in}{3.195150in}}{\pgfqpoint{3.376545in}{3.190759in}}{\pgfqpoint{3.387595in}{3.190759in}}%
\pgfpathclose%
\pgfusepath{stroke,fill}%
\end{pgfscope}%
\begin{pgfscope}%
\pgfpathrectangle{\pgfqpoint{0.648703in}{0.548769in}}{\pgfqpoint{5.112893in}{3.102590in}}%
\pgfusepath{clip}%
\pgfsetbuttcap%
\pgfsetroundjoin%
\definecolor{currentfill}{rgb}{1.000000,0.498039,0.054902}%
\pgfsetfillcolor{currentfill}%
\pgfsetlinewidth{1.003750pt}%
\definecolor{currentstroke}{rgb}{1.000000,0.498039,0.054902}%
\pgfsetstrokecolor{currentstroke}%
\pgfsetdash{}{0pt}%
\pgfpathmoveto{\pgfqpoint{3.311245in}{3.207351in}}%
\pgfpathcurveto{\pgfqpoint{3.322296in}{3.207351in}}{\pgfqpoint{3.332895in}{3.211741in}}{\pgfqpoint{3.340708in}{3.219555in}}%
\pgfpathcurveto{\pgfqpoint{3.348522in}{3.227368in}}{\pgfqpoint{3.352912in}{3.237967in}}{\pgfqpoint{3.352912in}{3.249017in}}%
\pgfpathcurveto{\pgfqpoint{3.352912in}{3.260068in}}{\pgfqpoint{3.348522in}{3.270667in}}{\pgfqpoint{3.340708in}{3.278480in}}%
\pgfpathcurveto{\pgfqpoint{3.332895in}{3.286294in}}{\pgfqpoint{3.322296in}{3.290684in}}{\pgfqpoint{3.311245in}{3.290684in}}%
\pgfpathcurveto{\pgfqpoint{3.300195in}{3.290684in}}{\pgfqpoint{3.289596in}{3.286294in}}{\pgfqpoint{3.281783in}{3.278480in}}%
\pgfpathcurveto{\pgfqpoint{3.273969in}{3.270667in}}{\pgfqpoint{3.269579in}{3.260068in}}{\pgfqpoint{3.269579in}{3.249017in}}%
\pgfpathcurveto{\pgfqpoint{3.269579in}{3.237967in}}{\pgfqpoint{3.273969in}{3.227368in}}{\pgfqpoint{3.281783in}{3.219555in}}%
\pgfpathcurveto{\pgfqpoint{3.289596in}{3.211741in}}{\pgfqpoint{3.300195in}{3.207351in}}{\pgfqpoint{3.311245in}{3.207351in}}%
\pgfpathclose%
\pgfusepath{stroke,fill}%
\end{pgfscope}%
\begin{pgfscope}%
\pgfpathrectangle{\pgfqpoint{0.648703in}{0.548769in}}{\pgfqpoint{5.112893in}{3.102590in}}%
\pgfusepath{clip}%
\pgfsetbuttcap%
\pgfsetroundjoin%
\definecolor{currentfill}{rgb}{0.121569,0.466667,0.705882}%
\pgfsetfillcolor{currentfill}%
\pgfsetlinewidth{1.003750pt}%
\definecolor{currentstroke}{rgb}{0.121569,0.466667,0.705882}%
\pgfsetstrokecolor{currentstroke}%
\pgfsetdash{}{0pt}%
\pgfpathmoveto{\pgfqpoint{0.814175in}{0.648129in}}%
\pgfpathcurveto{\pgfqpoint{0.825225in}{0.648129in}}{\pgfqpoint{0.835824in}{0.652519in}}{\pgfqpoint{0.843638in}{0.660333in}}%
\pgfpathcurveto{\pgfqpoint{0.851451in}{0.668146in}}{\pgfqpoint{0.855842in}{0.678745in}}{\pgfqpoint{0.855842in}{0.689796in}}%
\pgfpathcurveto{\pgfqpoint{0.855842in}{0.700846in}}{\pgfqpoint{0.851451in}{0.711445in}}{\pgfqpoint{0.843638in}{0.719258in}}%
\pgfpathcurveto{\pgfqpoint{0.835824in}{0.727072in}}{\pgfqpoint{0.825225in}{0.731462in}}{\pgfqpoint{0.814175in}{0.731462in}}%
\pgfpathcurveto{\pgfqpoint{0.803125in}{0.731462in}}{\pgfqpoint{0.792526in}{0.727072in}}{\pgfqpoint{0.784712in}{0.719258in}}%
\pgfpathcurveto{\pgfqpoint{0.776899in}{0.711445in}}{\pgfqpoint{0.772508in}{0.700846in}}{\pgfqpoint{0.772508in}{0.689796in}}%
\pgfpathcurveto{\pgfqpoint{0.772508in}{0.678745in}}{\pgfqpoint{0.776899in}{0.668146in}}{\pgfqpoint{0.784712in}{0.660333in}}%
\pgfpathcurveto{\pgfqpoint{0.792526in}{0.652519in}}{\pgfqpoint{0.803125in}{0.648129in}}{\pgfqpoint{0.814175in}{0.648129in}}%
\pgfpathclose%
\pgfusepath{stroke,fill}%
\end{pgfscope}%
\begin{pgfscope}%
\pgfpathrectangle{\pgfqpoint{0.648703in}{0.548769in}}{\pgfqpoint{5.112893in}{3.102590in}}%
\pgfusepath{clip}%
\pgfsetbuttcap%
\pgfsetroundjoin%
\definecolor{currentfill}{rgb}{0.121569,0.466667,0.705882}%
\pgfsetfillcolor{currentfill}%
\pgfsetlinewidth{1.003750pt}%
\definecolor{currentstroke}{rgb}{0.121569,0.466667,0.705882}%
\pgfsetstrokecolor{currentstroke}%
\pgfsetdash{}{0pt}%
\pgfpathmoveto{\pgfqpoint{0.838268in}{0.656425in}}%
\pgfpathcurveto{\pgfqpoint{0.849318in}{0.656425in}}{\pgfqpoint{0.859917in}{0.660815in}}{\pgfqpoint{0.867731in}{0.668629in}}%
\pgfpathcurveto{\pgfqpoint{0.875545in}{0.676442in}}{\pgfqpoint{0.879935in}{0.687041in}}{\pgfqpoint{0.879935in}{0.698091in}}%
\pgfpathcurveto{\pgfqpoint{0.879935in}{0.709141in}}{\pgfqpoint{0.875545in}{0.719740in}}{\pgfqpoint{0.867731in}{0.727554in}}%
\pgfpathcurveto{\pgfqpoint{0.859917in}{0.735368in}}{\pgfqpoint{0.849318in}{0.739758in}}{\pgfqpoint{0.838268in}{0.739758in}}%
\pgfpathcurveto{\pgfqpoint{0.827218in}{0.739758in}}{\pgfqpoint{0.816619in}{0.735368in}}{\pgfqpoint{0.808806in}{0.727554in}}%
\pgfpathcurveto{\pgfqpoint{0.800992in}{0.719740in}}{\pgfqpoint{0.796602in}{0.709141in}}{\pgfqpoint{0.796602in}{0.698091in}}%
\pgfpathcurveto{\pgfqpoint{0.796602in}{0.687041in}}{\pgfqpoint{0.800992in}{0.676442in}}{\pgfqpoint{0.808806in}{0.668629in}}%
\pgfpathcurveto{\pgfqpoint{0.816619in}{0.660815in}}{\pgfqpoint{0.827218in}{0.656425in}}{\pgfqpoint{0.838268in}{0.656425in}}%
\pgfpathclose%
\pgfusepath{stroke,fill}%
\end{pgfscope}%
\begin{pgfscope}%
\pgfpathrectangle{\pgfqpoint{0.648703in}{0.548769in}}{\pgfqpoint{5.112893in}{3.102590in}}%
\pgfusepath{clip}%
\pgfsetbuttcap%
\pgfsetroundjoin%
\definecolor{currentfill}{rgb}{1.000000,0.498039,0.054902}%
\pgfsetfillcolor{currentfill}%
\pgfsetlinewidth{1.003750pt}%
\definecolor{currentstroke}{rgb}{1.000000,0.498039,0.054902}%
\pgfsetstrokecolor{currentstroke}%
\pgfsetdash{}{0pt}%
\pgfpathmoveto{\pgfqpoint{1.049739in}{3.099507in}}%
\pgfpathcurveto{\pgfqpoint{1.060789in}{3.099507in}}{\pgfqpoint{1.071388in}{3.103897in}}{\pgfqpoint{1.079201in}{3.111711in}}%
\pgfpathcurveto{\pgfqpoint{1.087015in}{3.119524in}}{\pgfqpoint{1.091405in}{3.130123in}}{\pgfqpoint{1.091405in}{3.141173in}}%
\pgfpathcurveto{\pgfqpoint{1.091405in}{3.152224in}}{\pgfqpoint{1.087015in}{3.162823in}}{\pgfqpoint{1.079201in}{3.170636in}}%
\pgfpathcurveto{\pgfqpoint{1.071388in}{3.178450in}}{\pgfqpoint{1.060789in}{3.182840in}}{\pgfqpoint{1.049739in}{3.182840in}}%
\pgfpathcurveto{\pgfqpoint{1.038689in}{3.182840in}}{\pgfqpoint{1.028089in}{3.178450in}}{\pgfqpoint{1.020276in}{3.170636in}}%
\pgfpathcurveto{\pgfqpoint{1.012462in}{3.162823in}}{\pgfqpoint{1.008072in}{3.152224in}}{\pgfqpoint{1.008072in}{3.141173in}}%
\pgfpathcurveto{\pgfqpoint{1.008072in}{3.130123in}}{\pgfqpoint{1.012462in}{3.119524in}}{\pgfqpoint{1.020276in}{3.111711in}}%
\pgfpathcurveto{\pgfqpoint{1.028089in}{3.103897in}}{\pgfqpoint{1.038689in}{3.099507in}}{\pgfqpoint{1.049739in}{3.099507in}}%
\pgfpathclose%
\pgfusepath{stroke,fill}%
\end{pgfscope}%
\begin{pgfscope}%
\pgfpathrectangle{\pgfqpoint{0.648703in}{0.548769in}}{\pgfqpoint{5.112893in}{3.102590in}}%
\pgfusepath{clip}%
\pgfsetbuttcap%
\pgfsetroundjoin%
\definecolor{currentfill}{rgb}{1.000000,0.498039,0.054902}%
\pgfsetfillcolor{currentfill}%
\pgfsetlinewidth{1.003750pt}%
\definecolor{currentstroke}{rgb}{1.000000,0.498039,0.054902}%
\pgfsetstrokecolor{currentstroke}%
\pgfsetdash{}{0pt}%
\pgfpathmoveto{\pgfqpoint{3.755641in}{3.136837in}}%
\pgfpathcurveto{\pgfqpoint{3.766691in}{3.136837in}}{\pgfqpoint{3.777290in}{3.141228in}}{\pgfqpoint{3.785104in}{3.149041in}}%
\pgfpathcurveto{\pgfqpoint{3.792917in}{3.156855in}}{\pgfqpoint{3.797308in}{3.167454in}}{\pgfqpoint{3.797308in}{3.178504in}}%
\pgfpathcurveto{\pgfqpoint{3.797308in}{3.189554in}}{\pgfqpoint{3.792917in}{3.200153in}}{\pgfqpoint{3.785104in}{3.207967in}}%
\pgfpathcurveto{\pgfqpoint{3.777290in}{3.215780in}}{\pgfqpoint{3.766691in}{3.220171in}}{\pgfqpoint{3.755641in}{3.220171in}}%
\pgfpathcurveto{\pgfqpoint{3.744591in}{3.220171in}}{\pgfqpoint{3.733992in}{3.215780in}}{\pgfqpoint{3.726178in}{3.207967in}}%
\pgfpathcurveto{\pgfqpoint{3.718365in}{3.200153in}}{\pgfqpoint{3.713974in}{3.189554in}}{\pgfqpoint{3.713974in}{3.178504in}}%
\pgfpathcurveto{\pgfqpoint{3.713974in}{3.167454in}}{\pgfqpoint{3.718365in}{3.156855in}}{\pgfqpoint{3.726178in}{3.149041in}}%
\pgfpathcurveto{\pgfqpoint{3.733992in}{3.141228in}}{\pgfqpoint{3.744591in}{3.136837in}}{\pgfqpoint{3.755641in}{3.136837in}}%
\pgfpathclose%
\pgfusepath{stroke,fill}%
\end{pgfscope}%
\begin{pgfscope}%
\pgfpathrectangle{\pgfqpoint{0.648703in}{0.548769in}}{\pgfqpoint{5.112893in}{3.102590in}}%
\pgfusepath{clip}%
\pgfsetbuttcap%
\pgfsetroundjoin%
\definecolor{currentfill}{rgb}{0.121569,0.466667,0.705882}%
\pgfsetfillcolor{currentfill}%
\pgfsetlinewidth{1.003750pt}%
\definecolor{currentstroke}{rgb}{0.121569,0.466667,0.705882}%
\pgfsetstrokecolor{currentstroke}%
\pgfsetdash{}{0pt}%
\pgfpathmoveto{\pgfqpoint{0.814178in}{0.648129in}}%
\pgfpathcurveto{\pgfqpoint{0.825229in}{0.648129in}}{\pgfqpoint{0.835828in}{0.652519in}}{\pgfqpoint{0.843641in}{0.660333in}}%
\pgfpathcurveto{\pgfqpoint{0.851455in}{0.668146in}}{\pgfqpoint{0.855845in}{0.678745in}}{\pgfqpoint{0.855845in}{0.689796in}}%
\pgfpathcurveto{\pgfqpoint{0.855845in}{0.700846in}}{\pgfqpoint{0.851455in}{0.711445in}}{\pgfqpoint{0.843641in}{0.719258in}}%
\pgfpathcurveto{\pgfqpoint{0.835828in}{0.727072in}}{\pgfqpoint{0.825229in}{0.731462in}}{\pgfqpoint{0.814178in}{0.731462in}}%
\pgfpathcurveto{\pgfqpoint{0.803128in}{0.731462in}}{\pgfqpoint{0.792529in}{0.727072in}}{\pgfqpoint{0.784716in}{0.719258in}}%
\pgfpathcurveto{\pgfqpoint{0.776902in}{0.711445in}}{\pgfqpoint{0.772512in}{0.700846in}}{\pgfqpoint{0.772512in}{0.689796in}}%
\pgfpathcurveto{\pgfqpoint{0.772512in}{0.678745in}}{\pgfqpoint{0.776902in}{0.668146in}}{\pgfqpoint{0.784716in}{0.660333in}}%
\pgfpathcurveto{\pgfqpoint{0.792529in}{0.652519in}}{\pgfqpoint{0.803128in}{0.648129in}}{\pgfqpoint{0.814178in}{0.648129in}}%
\pgfpathclose%
\pgfusepath{stroke,fill}%
\end{pgfscope}%
\begin{pgfscope}%
\pgfpathrectangle{\pgfqpoint{0.648703in}{0.548769in}}{\pgfqpoint{5.112893in}{3.102590in}}%
\pgfusepath{clip}%
\pgfsetbuttcap%
\pgfsetroundjoin%
\definecolor{currentfill}{rgb}{0.121569,0.466667,0.705882}%
\pgfsetfillcolor{currentfill}%
\pgfsetlinewidth{1.003750pt}%
\definecolor{currentstroke}{rgb}{0.121569,0.466667,0.705882}%
\pgfsetstrokecolor{currentstroke}%
\pgfsetdash{}{0pt}%
\pgfpathmoveto{\pgfqpoint{0.814190in}{0.648129in}}%
\pgfpathcurveto{\pgfqpoint{0.825240in}{0.648129in}}{\pgfqpoint{0.835839in}{0.652519in}}{\pgfqpoint{0.843652in}{0.660333in}}%
\pgfpathcurveto{\pgfqpoint{0.851466in}{0.668146in}}{\pgfqpoint{0.855856in}{0.678745in}}{\pgfqpoint{0.855856in}{0.689796in}}%
\pgfpathcurveto{\pgfqpoint{0.855856in}{0.700846in}}{\pgfqpoint{0.851466in}{0.711445in}}{\pgfqpoint{0.843652in}{0.719258in}}%
\pgfpathcurveto{\pgfqpoint{0.835839in}{0.727072in}}{\pgfqpoint{0.825240in}{0.731462in}}{\pgfqpoint{0.814190in}{0.731462in}}%
\pgfpathcurveto{\pgfqpoint{0.803140in}{0.731462in}}{\pgfqpoint{0.792541in}{0.727072in}}{\pgfqpoint{0.784727in}{0.719258in}}%
\pgfpathcurveto{\pgfqpoint{0.776913in}{0.711445in}}{\pgfqpoint{0.772523in}{0.700846in}}{\pgfqpoint{0.772523in}{0.689796in}}%
\pgfpathcurveto{\pgfqpoint{0.772523in}{0.678745in}}{\pgfqpoint{0.776913in}{0.668146in}}{\pgfqpoint{0.784727in}{0.660333in}}%
\pgfpathcurveto{\pgfqpoint{0.792541in}{0.652519in}}{\pgfqpoint{0.803140in}{0.648129in}}{\pgfqpoint{0.814190in}{0.648129in}}%
\pgfpathclose%
\pgfusepath{stroke,fill}%
\end{pgfscope}%
\begin{pgfscope}%
\pgfpathrectangle{\pgfqpoint{0.648703in}{0.548769in}}{\pgfqpoint{5.112893in}{3.102590in}}%
\pgfusepath{clip}%
\pgfsetbuttcap%
\pgfsetroundjoin%
\definecolor{currentfill}{rgb}{0.839216,0.152941,0.156863}%
\pgfsetfillcolor{currentfill}%
\pgfsetlinewidth{1.003750pt}%
\definecolor{currentstroke}{rgb}{0.839216,0.152941,0.156863}%
\pgfsetstrokecolor{currentstroke}%
\pgfsetdash{}{0pt}%
\pgfpathmoveto{\pgfqpoint{3.319531in}{3.410595in}}%
\pgfpathcurveto{\pgfqpoint{3.330582in}{3.410595in}}{\pgfqpoint{3.341181in}{3.414986in}}{\pgfqpoint{3.348994in}{3.422799in}}%
\pgfpathcurveto{\pgfqpoint{3.356808in}{3.430613in}}{\pgfqpoint{3.361198in}{3.441212in}}{\pgfqpoint{3.361198in}{3.452262in}}%
\pgfpathcurveto{\pgfqpoint{3.361198in}{3.463312in}}{\pgfqpoint{3.356808in}{3.473911in}}{\pgfqpoint{3.348994in}{3.481725in}}%
\pgfpathcurveto{\pgfqpoint{3.341181in}{3.489538in}}{\pgfqpoint{3.330582in}{3.493929in}}{\pgfqpoint{3.319531in}{3.493929in}}%
\pgfpathcurveto{\pgfqpoint{3.308481in}{3.493929in}}{\pgfqpoint{3.297882in}{3.489538in}}{\pgfqpoint{3.290069in}{3.481725in}}%
\pgfpathcurveto{\pgfqpoint{3.282255in}{3.473911in}}{\pgfqpoint{3.277865in}{3.463312in}}{\pgfqpoint{3.277865in}{3.452262in}}%
\pgfpathcurveto{\pgfqpoint{3.277865in}{3.441212in}}{\pgfqpoint{3.282255in}{3.430613in}}{\pgfqpoint{3.290069in}{3.422799in}}%
\pgfpathcurveto{\pgfqpoint{3.297882in}{3.414986in}}{\pgfqpoint{3.308481in}{3.410595in}}{\pgfqpoint{3.319531in}{3.410595in}}%
\pgfpathclose%
\pgfusepath{stroke,fill}%
\end{pgfscope}%
\begin{pgfscope}%
\pgfpathrectangle{\pgfqpoint{0.648703in}{0.548769in}}{\pgfqpoint{5.112893in}{3.102590in}}%
\pgfusepath{clip}%
\pgfsetbuttcap%
\pgfsetroundjoin%
\definecolor{currentfill}{rgb}{1.000000,0.498039,0.054902}%
\pgfsetfillcolor{currentfill}%
\pgfsetlinewidth{1.003750pt}%
\definecolor{currentstroke}{rgb}{1.000000,0.498039,0.054902}%
\pgfsetstrokecolor{currentstroke}%
\pgfsetdash{}{0pt}%
\pgfpathmoveto{\pgfqpoint{3.242417in}{3.128542in}}%
\pgfpathcurveto{\pgfqpoint{3.253467in}{3.128542in}}{\pgfqpoint{3.264066in}{3.132932in}}{\pgfqpoint{3.271880in}{3.140746in}}%
\pgfpathcurveto{\pgfqpoint{3.279694in}{3.148559in}}{\pgfqpoint{3.284084in}{3.159158in}}{\pgfqpoint{3.284084in}{3.170208in}}%
\pgfpathcurveto{\pgfqpoint{3.284084in}{3.181258in}}{\pgfqpoint{3.279694in}{3.191857in}}{\pgfqpoint{3.271880in}{3.199671in}}%
\pgfpathcurveto{\pgfqpoint{3.264066in}{3.207485in}}{\pgfqpoint{3.253467in}{3.211875in}}{\pgfqpoint{3.242417in}{3.211875in}}%
\pgfpathcurveto{\pgfqpoint{3.231367in}{3.211875in}}{\pgfqpoint{3.220768in}{3.207485in}}{\pgfqpoint{3.212955in}{3.199671in}}%
\pgfpathcurveto{\pgfqpoint{3.205141in}{3.191857in}}{\pgfqpoint{3.200751in}{3.181258in}}{\pgfqpoint{3.200751in}{3.170208in}}%
\pgfpathcurveto{\pgfqpoint{3.200751in}{3.159158in}}{\pgfqpoint{3.205141in}{3.148559in}}{\pgfqpoint{3.212955in}{3.140746in}}%
\pgfpathcurveto{\pgfqpoint{3.220768in}{3.132932in}}{\pgfqpoint{3.231367in}{3.128542in}}{\pgfqpoint{3.242417in}{3.128542in}}%
\pgfpathclose%
\pgfusepath{stroke,fill}%
\end{pgfscope}%
\begin{pgfscope}%
\pgfpathrectangle{\pgfqpoint{0.648703in}{0.548769in}}{\pgfqpoint{5.112893in}{3.102590in}}%
\pgfusepath{clip}%
\pgfsetbuttcap%
\pgfsetroundjoin%
\definecolor{currentfill}{rgb}{1.000000,0.498039,0.054902}%
\pgfsetfillcolor{currentfill}%
\pgfsetlinewidth{1.003750pt}%
\definecolor{currentstroke}{rgb}{1.000000,0.498039,0.054902}%
\pgfsetstrokecolor{currentstroke}%
\pgfsetdash{}{0pt}%
\pgfpathmoveto{\pgfqpoint{3.312788in}{3.315195in}}%
\pgfpathcurveto{\pgfqpoint{3.323838in}{3.315195in}}{\pgfqpoint{3.334437in}{3.319585in}}{\pgfqpoint{3.342251in}{3.327399in}}%
\pgfpathcurveto{\pgfqpoint{3.350065in}{3.335212in}}{\pgfqpoint{3.354455in}{3.345811in}}{\pgfqpoint{3.354455in}{3.356861in}}%
\pgfpathcurveto{\pgfqpoint{3.354455in}{3.367912in}}{\pgfqpoint{3.350065in}{3.378511in}}{\pgfqpoint{3.342251in}{3.386324in}}%
\pgfpathcurveto{\pgfqpoint{3.334437in}{3.394138in}}{\pgfqpoint{3.323838in}{3.398528in}}{\pgfqpoint{3.312788in}{3.398528in}}%
\pgfpathcurveto{\pgfqpoint{3.301738in}{3.398528in}}{\pgfqpoint{3.291139in}{3.394138in}}{\pgfqpoint{3.283325in}{3.386324in}}%
\pgfpathcurveto{\pgfqpoint{3.275512in}{3.378511in}}{\pgfqpoint{3.271122in}{3.367912in}}{\pgfqpoint{3.271122in}{3.356861in}}%
\pgfpathcurveto{\pgfqpoint{3.271122in}{3.345811in}}{\pgfqpoint{3.275512in}{3.335212in}}{\pgfqpoint{3.283325in}{3.327399in}}%
\pgfpathcurveto{\pgfqpoint{3.291139in}{3.319585in}}{\pgfqpoint{3.301738in}{3.315195in}}{\pgfqpoint{3.312788in}{3.315195in}}%
\pgfpathclose%
\pgfusepath{stroke,fill}%
\end{pgfscope}%
\begin{pgfscope}%
\pgfpathrectangle{\pgfqpoint{0.648703in}{0.548769in}}{\pgfqpoint{5.112893in}{3.102590in}}%
\pgfusepath{clip}%
\pgfsetbuttcap%
\pgfsetroundjoin%
\definecolor{currentfill}{rgb}{1.000000,0.498039,0.054902}%
\pgfsetfillcolor{currentfill}%
\pgfsetlinewidth{1.003750pt}%
\definecolor{currentstroke}{rgb}{1.000000,0.498039,0.054902}%
\pgfsetstrokecolor{currentstroke}%
\pgfsetdash{}{0pt}%
\pgfpathmoveto{\pgfqpoint{3.124939in}{3.136837in}}%
\pgfpathcurveto{\pgfqpoint{3.135989in}{3.136837in}}{\pgfqpoint{3.146588in}{3.141228in}}{\pgfqpoint{3.154402in}{3.149041in}}%
\pgfpathcurveto{\pgfqpoint{3.162215in}{3.156855in}}{\pgfqpoint{3.166606in}{3.167454in}}{\pgfqpoint{3.166606in}{3.178504in}}%
\pgfpathcurveto{\pgfqpoint{3.166606in}{3.189554in}}{\pgfqpoint{3.162215in}{3.200153in}}{\pgfqpoint{3.154402in}{3.207967in}}%
\pgfpathcurveto{\pgfqpoint{3.146588in}{3.215780in}}{\pgfqpoint{3.135989in}{3.220171in}}{\pgfqpoint{3.124939in}{3.220171in}}%
\pgfpathcurveto{\pgfqpoint{3.113889in}{3.220171in}}{\pgfqpoint{3.103290in}{3.215780in}}{\pgfqpoint{3.095476in}{3.207967in}}%
\pgfpathcurveto{\pgfqpoint{3.087663in}{3.200153in}}{\pgfqpoint{3.083272in}{3.189554in}}{\pgfqpoint{3.083272in}{3.178504in}}%
\pgfpathcurveto{\pgfqpoint{3.083272in}{3.167454in}}{\pgfqpoint{3.087663in}{3.156855in}}{\pgfqpoint{3.095476in}{3.149041in}}%
\pgfpathcurveto{\pgfqpoint{3.103290in}{3.141228in}}{\pgfqpoint{3.113889in}{3.136837in}}{\pgfqpoint{3.124939in}{3.136837in}}%
\pgfpathclose%
\pgfusepath{stroke,fill}%
\end{pgfscope}%
\begin{pgfscope}%
\pgfpathrectangle{\pgfqpoint{0.648703in}{0.548769in}}{\pgfqpoint{5.112893in}{3.102590in}}%
\pgfusepath{clip}%
\pgfsetbuttcap%
\pgfsetroundjoin%
\definecolor{currentfill}{rgb}{1.000000,0.498039,0.054902}%
\pgfsetfillcolor{currentfill}%
\pgfsetlinewidth{1.003750pt}%
\definecolor{currentstroke}{rgb}{1.000000,0.498039,0.054902}%
\pgfsetstrokecolor{currentstroke}%
\pgfsetdash{}{0pt}%
\pgfpathmoveto{\pgfqpoint{3.230957in}{3.136837in}}%
\pgfpathcurveto{\pgfqpoint{3.242008in}{3.136837in}}{\pgfqpoint{3.252607in}{3.141228in}}{\pgfqpoint{3.260420in}{3.149041in}}%
\pgfpathcurveto{\pgfqpoint{3.268234in}{3.156855in}}{\pgfqpoint{3.272624in}{3.167454in}}{\pgfqpoint{3.272624in}{3.178504in}}%
\pgfpathcurveto{\pgfqpoint{3.272624in}{3.189554in}}{\pgfqpoint{3.268234in}{3.200153in}}{\pgfqpoint{3.260420in}{3.207967in}}%
\pgfpathcurveto{\pgfqpoint{3.252607in}{3.215780in}}{\pgfqpoint{3.242008in}{3.220171in}}{\pgfqpoint{3.230957in}{3.220171in}}%
\pgfpathcurveto{\pgfqpoint{3.219907in}{3.220171in}}{\pgfqpoint{3.209308in}{3.215780in}}{\pgfqpoint{3.201495in}{3.207967in}}%
\pgfpathcurveto{\pgfqpoint{3.193681in}{3.200153in}}{\pgfqpoint{3.189291in}{3.189554in}}{\pgfqpoint{3.189291in}{3.178504in}}%
\pgfpathcurveto{\pgfqpoint{3.189291in}{3.167454in}}{\pgfqpoint{3.193681in}{3.156855in}}{\pgfqpoint{3.201495in}{3.149041in}}%
\pgfpathcurveto{\pgfqpoint{3.209308in}{3.141228in}}{\pgfqpoint{3.219907in}{3.136837in}}{\pgfqpoint{3.230957in}{3.136837in}}%
\pgfpathclose%
\pgfusepath{stroke,fill}%
\end{pgfscope}%
\begin{pgfscope}%
\pgfpathrectangle{\pgfqpoint{0.648703in}{0.548769in}}{\pgfqpoint{5.112893in}{3.102590in}}%
\pgfusepath{clip}%
\pgfsetbuttcap%
\pgfsetroundjoin%
\definecolor{currentfill}{rgb}{1.000000,0.498039,0.054902}%
\pgfsetfillcolor{currentfill}%
\pgfsetlinewidth{1.003750pt}%
\definecolor{currentstroke}{rgb}{1.000000,0.498039,0.054902}%
\pgfsetstrokecolor{currentstroke}%
\pgfsetdash{}{0pt}%
\pgfpathmoveto{\pgfqpoint{1.694651in}{3.140985in}}%
\pgfpathcurveto{\pgfqpoint{1.705701in}{3.140985in}}{\pgfqpoint{1.716300in}{3.145375in}}{\pgfqpoint{1.724114in}{3.153189in}}%
\pgfpathcurveto{\pgfqpoint{1.731927in}{3.161003in}}{\pgfqpoint{1.736318in}{3.171602in}}{\pgfqpoint{1.736318in}{3.182652in}}%
\pgfpathcurveto{\pgfqpoint{1.736318in}{3.193702in}}{\pgfqpoint{1.731927in}{3.204301in}}{\pgfqpoint{1.724114in}{3.212115in}}%
\pgfpathcurveto{\pgfqpoint{1.716300in}{3.219928in}}{\pgfqpoint{1.705701in}{3.224319in}}{\pgfqpoint{1.694651in}{3.224319in}}%
\pgfpathcurveto{\pgfqpoint{1.683601in}{3.224319in}}{\pgfqpoint{1.673002in}{3.219928in}}{\pgfqpoint{1.665188in}{3.212115in}}%
\pgfpathcurveto{\pgfqpoint{1.657375in}{3.204301in}}{\pgfqpoint{1.652984in}{3.193702in}}{\pgfqpoint{1.652984in}{3.182652in}}%
\pgfpathcurveto{\pgfqpoint{1.652984in}{3.171602in}}{\pgfqpoint{1.657375in}{3.161003in}}{\pgfqpoint{1.665188in}{3.153189in}}%
\pgfpathcurveto{\pgfqpoint{1.673002in}{3.145375in}}{\pgfqpoint{1.683601in}{3.140985in}}{\pgfqpoint{1.694651in}{3.140985in}}%
\pgfpathclose%
\pgfusepath{stroke,fill}%
\end{pgfscope}%
\begin{pgfscope}%
\pgfpathrectangle{\pgfqpoint{0.648703in}{0.548769in}}{\pgfqpoint{5.112893in}{3.102590in}}%
\pgfusepath{clip}%
\pgfsetbuttcap%
\pgfsetroundjoin%
\definecolor{currentfill}{rgb}{1.000000,0.498039,0.054902}%
\pgfsetfillcolor{currentfill}%
\pgfsetlinewidth{1.003750pt}%
\definecolor{currentstroke}{rgb}{1.000000,0.498039,0.054902}%
\pgfsetstrokecolor{currentstroke}%
\pgfsetdash{}{0pt}%
\pgfpathmoveto{\pgfqpoint{3.286316in}{3.136837in}}%
\pgfpathcurveto{\pgfqpoint{3.297366in}{3.136837in}}{\pgfqpoint{3.307965in}{3.141228in}}{\pgfqpoint{3.315778in}{3.149041in}}%
\pgfpathcurveto{\pgfqpoint{3.323592in}{3.156855in}}{\pgfqpoint{3.327982in}{3.167454in}}{\pgfqpoint{3.327982in}{3.178504in}}%
\pgfpathcurveto{\pgfqpoint{3.327982in}{3.189554in}}{\pgfqpoint{3.323592in}{3.200153in}}{\pgfqpoint{3.315778in}{3.207967in}}%
\pgfpathcurveto{\pgfqpoint{3.307965in}{3.215780in}}{\pgfqpoint{3.297366in}{3.220171in}}{\pgfqpoint{3.286316in}{3.220171in}}%
\pgfpathcurveto{\pgfqpoint{3.275265in}{3.220171in}}{\pgfqpoint{3.264666in}{3.215780in}}{\pgfqpoint{3.256853in}{3.207967in}}%
\pgfpathcurveto{\pgfqpoint{3.249039in}{3.200153in}}{\pgfqpoint{3.244649in}{3.189554in}}{\pgfqpoint{3.244649in}{3.178504in}}%
\pgfpathcurveto{\pgfqpoint{3.244649in}{3.167454in}}{\pgfqpoint{3.249039in}{3.156855in}}{\pgfqpoint{3.256853in}{3.149041in}}%
\pgfpathcurveto{\pgfqpoint{3.264666in}{3.141228in}}{\pgfqpoint{3.275265in}{3.136837in}}{\pgfqpoint{3.286316in}{3.136837in}}%
\pgfpathclose%
\pgfusepath{stroke,fill}%
\end{pgfscope}%
\begin{pgfscope}%
\pgfpathrectangle{\pgfqpoint{0.648703in}{0.548769in}}{\pgfqpoint{5.112893in}{3.102590in}}%
\pgfusepath{clip}%
\pgfsetbuttcap%
\pgfsetroundjoin%
\definecolor{currentfill}{rgb}{1.000000,0.498039,0.054902}%
\pgfsetfillcolor{currentfill}%
\pgfsetlinewidth{1.003750pt}%
\definecolor{currentstroke}{rgb}{1.000000,0.498039,0.054902}%
\pgfsetstrokecolor{currentstroke}%
\pgfsetdash{}{0pt}%
\pgfpathmoveto{\pgfqpoint{3.358941in}{3.356673in}}%
\pgfpathcurveto{\pgfqpoint{3.369991in}{3.356673in}}{\pgfqpoint{3.380590in}{3.361064in}}{\pgfqpoint{3.388404in}{3.368877in}}%
\pgfpathcurveto{\pgfqpoint{3.396217in}{3.376691in}}{\pgfqpoint{3.400607in}{3.387290in}}{\pgfqpoint{3.400607in}{3.398340in}}%
\pgfpathcurveto{\pgfqpoint{3.400607in}{3.409390in}}{\pgfqpoint{3.396217in}{3.419989in}}{\pgfqpoint{3.388404in}{3.427803in}}%
\pgfpathcurveto{\pgfqpoint{3.380590in}{3.435616in}}{\pgfqpoint{3.369991in}{3.440007in}}{\pgfqpoint{3.358941in}{3.440007in}}%
\pgfpathcurveto{\pgfqpoint{3.347891in}{3.440007in}}{\pgfqpoint{3.337292in}{3.435616in}}{\pgfqpoint{3.329478in}{3.427803in}}%
\pgfpathcurveto{\pgfqpoint{3.321664in}{3.419989in}}{\pgfqpoint{3.317274in}{3.409390in}}{\pgfqpoint{3.317274in}{3.398340in}}%
\pgfpathcurveto{\pgfqpoint{3.317274in}{3.387290in}}{\pgfqpoint{3.321664in}{3.376691in}}{\pgfqpoint{3.329478in}{3.368877in}}%
\pgfpathcurveto{\pgfqpoint{3.337292in}{3.361064in}}{\pgfqpoint{3.347891in}{3.356673in}}{\pgfqpoint{3.358941in}{3.356673in}}%
\pgfpathclose%
\pgfusepath{stroke,fill}%
\end{pgfscope}%
\begin{pgfscope}%
\pgfpathrectangle{\pgfqpoint{0.648703in}{0.548769in}}{\pgfqpoint{5.112893in}{3.102590in}}%
\pgfusepath{clip}%
\pgfsetbuttcap%
\pgfsetroundjoin%
\definecolor{currentfill}{rgb}{1.000000,0.498039,0.054902}%
\pgfsetfillcolor{currentfill}%
\pgfsetlinewidth{1.003750pt}%
\definecolor{currentstroke}{rgb}{1.000000,0.498039,0.054902}%
\pgfsetstrokecolor{currentstroke}%
\pgfsetdash{}{0pt}%
\pgfpathmoveto{\pgfqpoint{1.099201in}{3.219794in}}%
\pgfpathcurveto{\pgfqpoint{1.110251in}{3.219794in}}{\pgfqpoint{1.120851in}{3.224185in}}{\pgfqpoint{1.128664in}{3.231998in}}%
\pgfpathcurveto{\pgfqpoint{1.136478in}{3.239812in}}{\pgfqpoint{1.140868in}{3.250411in}}{\pgfqpoint{1.140868in}{3.261461in}}%
\pgfpathcurveto{\pgfqpoint{1.140868in}{3.272511in}}{\pgfqpoint{1.136478in}{3.283110in}}{\pgfqpoint{1.128664in}{3.290924in}}%
\pgfpathcurveto{\pgfqpoint{1.120851in}{3.298737in}}{\pgfqpoint{1.110251in}{3.303128in}}{\pgfqpoint{1.099201in}{3.303128in}}%
\pgfpathcurveto{\pgfqpoint{1.088151in}{3.303128in}}{\pgfqpoint{1.077552in}{3.298737in}}{\pgfqpoint{1.069739in}{3.290924in}}%
\pgfpathcurveto{\pgfqpoint{1.061925in}{3.283110in}}{\pgfqpoint{1.057535in}{3.272511in}}{\pgfqpoint{1.057535in}{3.261461in}}%
\pgfpathcurveto{\pgfqpoint{1.057535in}{3.250411in}}{\pgfqpoint{1.061925in}{3.239812in}}{\pgfqpoint{1.069739in}{3.231998in}}%
\pgfpathcurveto{\pgfqpoint{1.077552in}{3.224185in}}{\pgfqpoint{1.088151in}{3.219794in}}{\pgfqpoint{1.099201in}{3.219794in}}%
\pgfpathclose%
\pgfusepath{stroke,fill}%
\end{pgfscope}%
\begin{pgfscope}%
\pgfpathrectangle{\pgfqpoint{0.648703in}{0.548769in}}{\pgfqpoint{5.112893in}{3.102590in}}%
\pgfusepath{clip}%
\pgfsetbuttcap%
\pgfsetroundjoin%
\definecolor{currentfill}{rgb}{1.000000,0.498039,0.054902}%
\pgfsetfillcolor{currentfill}%
\pgfsetlinewidth{1.003750pt}%
\definecolor{currentstroke}{rgb}{1.000000,0.498039,0.054902}%
\pgfsetstrokecolor{currentstroke}%
\pgfsetdash{}{0pt}%
\pgfpathmoveto{\pgfqpoint{2.617044in}{3.120246in}}%
\pgfpathcurveto{\pgfqpoint{2.628095in}{3.120246in}}{\pgfqpoint{2.638694in}{3.124636in}}{\pgfqpoint{2.646507in}{3.132450in}}%
\pgfpathcurveto{\pgfqpoint{2.654321in}{3.140263in}}{\pgfqpoint{2.658711in}{3.150862in}}{\pgfqpoint{2.658711in}{3.161913in}}%
\pgfpathcurveto{\pgfqpoint{2.658711in}{3.172963in}}{\pgfqpoint{2.654321in}{3.183562in}}{\pgfqpoint{2.646507in}{3.191375in}}%
\pgfpathcurveto{\pgfqpoint{2.638694in}{3.199189in}}{\pgfqpoint{2.628095in}{3.203579in}}{\pgfqpoint{2.617044in}{3.203579in}}%
\pgfpathcurveto{\pgfqpoint{2.605994in}{3.203579in}}{\pgfqpoint{2.595395in}{3.199189in}}{\pgfqpoint{2.587582in}{3.191375in}}%
\pgfpathcurveto{\pgfqpoint{2.579768in}{3.183562in}}{\pgfqpoint{2.575378in}{3.172963in}}{\pgfqpoint{2.575378in}{3.161913in}}%
\pgfpathcurveto{\pgfqpoint{2.575378in}{3.150862in}}{\pgfqpoint{2.579768in}{3.140263in}}{\pgfqpoint{2.587582in}{3.132450in}}%
\pgfpathcurveto{\pgfqpoint{2.595395in}{3.124636in}}{\pgfqpoint{2.605994in}{3.120246in}}{\pgfqpoint{2.617044in}{3.120246in}}%
\pgfpathclose%
\pgfusepath{stroke,fill}%
\end{pgfscope}%
\begin{pgfscope}%
\pgfpathrectangle{\pgfqpoint{0.648703in}{0.548769in}}{\pgfqpoint{5.112893in}{3.102590in}}%
\pgfusepath{clip}%
\pgfsetbuttcap%
\pgfsetroundjoin%
\definecolor{currentfill}{rgb}{1.000000,0.498039,0.054902}%
\pgfsetfillcolor{currentfill}%
\pgfsetlinewidth{1.003750pt}%
\definecolor{currentstroke}{rgb}{1.000000,0.498039,0.054902}%
\pgfsetstrokecolor{currentstroke}%
\pgfsetdash{}{0pt}%
\pgfpathmoveto{\pgfqpoint{3.232208in}{3.140985in}}%
\pgfpathcurveto{\pgfqpoint{3.243258in}{3.140985in}}{\pgfqpoint{3.253857in}{3.145375in}}{\pgfqpoint{3.261671in}{3.153189in}}%
\pgfpathcurveto{\pgfqpoint{3.269484in}{3.161003in}}{\pgfqpoint{3.273874in}{3.171602in}}{\pgfqpoint{3.273874in}{3.182652in}}%
\pgfpathcurveto{\pgfqpoint{3.273874in}{3.193702in}}{\pgfqpoint{3.269484in}{3.204301in}}{\pgfqpoint{3.261671in}{3.212115in}}%
\pgfpathcurveto{\pgfqpoint{3.253857in}{3.219928in}}{\pgfqpoint{3.243258in}{3.224319in}}{\pgfqpoint{3.232208in}{3.224319in}}%
\pgfpathcurveto{\pgfqpoint{3.221158in}{3.224319in}}{\pgfqpoint{3.210559in}{3.219928in}}{\pgfqpoint{3.202745in}{3.212115in}}%
\pgfpathcurveto{\pgfqpoint{3.194931in}{3.204301in}}{\pgfqpoint{3.190541in}{3.193702in}}{\pgfqpoint{3.190541in}{3.182652in}}%
\pgfpathcurveto{\pgfqpoint{3.190541in}{3.171602in}}{\pgfqpoint{3.194931in}{3.161003in}}{\pgfqpoint{3.202745in}{3.153189in}}%
\pgfpathcurveto{\pgfqpoint{3.210559in}{3.145375in}}{\pgfqpoint{3.221158in}{3.140985in}}{\pgfqpoint{3.232208in}{3.140985in}}%
\pgfpathclose%
\pgfusepath{stroke,fill}%
\end{pgfscope}%
\begin{pgfscope}%
\pgfpathrectangle{\pgfqpoint{0.648703in}{0.548769in}}{\pgfqpoint{5.112893in}{3.102590in}}%
\pgfusepath{clip}%
\pgfsetbuttcap%
\pgfsetroundjoin%
\definecolor{currentfill}{rgb}{1.000000,0.498039,0.054902}%
\pgfsetfillcolor{currentfill}%
\pgfsetlinewidth{1.003750pt}%
\definecolor{currentstroke}{rgb}{1.000000,0.498039,0.054902}%
\pgfsetstrokecolor{currentstroke}%
\pgfsetdash{}{0pt}%
\pgfpathmoveto{\pgfqpoint{3.394771in}{3.145133in}}%
\pgfpathcurveto{\pgfqpoint{3.405822in}{3.145133in}}{\pgfqpoint{3.416421in}{3.149523in}}{\pgfqpoint{3.424234in}{3.157337in}}%
\pgfpathcurveto{\pgfqpoint{3.432048in}{3.165151in}}{\pgfqpoint{3.436438in}{3.175750in}}{\pgfqpoint{3.436438in}{3.186800in}}%
\pgfpathcurveto{\pgfqpoint{3.436438in}{3.197850in}}{\pgfqpoint{3.432048in}{3.208449in}}{\pgfqpoint{3.424234in}{3.216262in}}%
\pgfpathcurveto{\pgfqpoint{3.416421in}{3.224076in}}{\pgfqpoint{3.405822in}{3.228466in}}{\pgfqpoint{3.394771in}{3.228466in}}%
\pgfpathcurveto{\pgfqpoint{3.383721in}{3.228466in}}{\pgfqpoint{3.373122in}{3.224076in}}{\pgfqpoint{3.365309in}{3.216262in}}%
\pgfpathcurveto{\pgfqpoint{3.357495in}{3.208449in}}{\pgfqpoint{3.353105in}{3.197850in}}{\pgfqpoint{3.353105in}{3.186800in}}%
\pgfpathcurveto{\pgfqpoint{3.353105in}{3.175750in}}{\pgfqpoint{3.357495in}{3.165151in}}{\pgfqpoint{3.365309in}{3.157337in}}%
\pgfpathcurveto{\pgfqpoint{3.373122in}{3.149523in}}{\pgfqpoint{3.383721in}{3.145133in}}{\pgfqpoint{3.394771in}{3.145133in}}%
\pgfpathclose%
\pgfusepath{stroke,fill}%
\end{pgfscope}%
\begin{pgfscope}%
\pgfpathrectangle{\pgfqpoint{0.648703in}{0.548769in}}{\pgfqpoint{5.112893in}{3.102590in}}%
\pgfusepath{clip}%
\pgfsetbuttcap%
\pgfsetroundjoin%
\definecolor{currentfill}{rgb}{1.000000,0.498039,0.054902}%
\pgfsetfillcolor{currentfill}%
\pgfsetlinewidth{1.003750pt}%
\definecolor{currentstroke}{rgb}{1.000000,0.498039,0.054902}%
\pgfsetstrokecolor{currentstroke}%
\pgfsetdash{}{0pt}%
\pgfpathmoveto{\pgfqpoint{1.368001in}{3.136837in}}%
\pgfpathcurveto{\pgfqpoint{1.379051in}{3.136837in}}{\pgfqpoint{1.389650in}{3.141228in}}{\pgfqpoint{1.397464in}{3.149041in}}%
\pgfpathcurveto{\pgfqpoint{1.405278in}{3.156855in}}{\pgfqpoint{1.409668in}{3.167454in}}{\pgfqpoint{1.409668in}{3.178504in}}%
\pgfpathcurveto{\pgfqpoint{1.409668in}{3.189554in}}{\pgfqpoint{1.405278in}{3.200153in}}{\pgfqpoint{1.397464in}{3.207967in}}%
\pgfpathcurveto{\pgfqpoint{1.389650in}{3.215780in}}{\pgfqpoint{1.379051in}{3.220171in}}{\pgfqpoint{1.368001in}{3.220171in}}%
\pgfpathcurveto{\pgfqpoint{1.356951in}{3.220171in}}{\pgfqpoint{1.346352in}{3.215780in}}{\pgfqpoint{1.338538in}{3.207967in}}%
\pgfpathcurveto{\pgfqpoint{1.330725in}{3.200153in}}{\pgfqpoint{1.326335in}{3.189554in}}{\pgfqpoint{1.326335in}{3.178504in}}%
\pgfpathcurveto{\pgfqpoint{1.326335in}{3.167454in}}{\pgfqpoint{1.330725in}{3.156855in}}{\pgfqpoint{1.338538in}{3.149041in}}%
\pgfpathcurveto{\pgfqpoint{1.346352in}{3.141228in}}{\pgfqpoint{1.356951in}{3.136837in}}{\pgfqpoint{1.368001in}{3.136837in}}%
\pgfpathclose%
\pgfusepath{stroke,fill}%
\end{pgfscope}%
\begin{pgfscope}%
\pgfpathrectangle{\pgfqpoint{0.648703in}{0.548769in}}{\pgfqpoint{5.112893in}{3.102590in}}%
\pgfusepath{clip}%
\pgfsetbuttcap%
\pgfsetroundjoin%
\definecolor{currentfill}{rgb}{1.000000,0.498039,0.054902}%
\pgfsetfillcolor{currentfill}%
\pgfsetlinewidth{1.003750pt}%
\definecolor{currentstroke}{rgb}{1.000000,0.498039,0.054902}%
\pgfsetstrokecolor{currentstroke}%
\pgfsetdash{}{0pt}%
\pgfpathmoveto{\pgfqpoint{1.759012in}{3.132690in}}%
\pgfpathcurveto{\pgfqpoint{1.770062in}{3.132690in}}{\pgfqpoint{1.780661in}{3.137080in}}{\pgfqpoint{1.788475in}{3.144893in}}%
\pgfpathcurveto{\pgfqpoint{1.796288in}{3.152707in}}{\pgfqpoint{1.800679in}{3.163306in}}{\pgfqpoint{1.800679in}{3.174356in}}%
\pgfpathcurveto{\pgfqpoint{1.800679in}{3.185406in}}{\pgfqpoint{1.796288in}{3.196005in}}{\pgfqpoint{1.788475in}{3.203819in}}%
\pgfpathcurveto{\pgfqpoint{1.780661in}{3.211633in}}{\pgfqpoint{1.770062in}{3.216023in}}{\pgfqpoint{1.759012in}{3.216023in}}%
\pgfpathcurveto{\pgfqpoint{1.747962in}{3.216023in}}{\pgfqpoint{1.737363in}{3.211633in}}{\pgfqpoint{1.729549in}{3.203819in}}%
\pgfpathcurveto{\pgfqpoint{1.721736in}{3.196005in}}{\pgfqpoint{1.717345in}{3.185406in}}{\pgfqpoint{1.717345in}{3.174356in}}%
\pgfpathcurveto{\pgfqpoint{1.717345in}{3.163306in}}{\pgfqpoint{1.721736in}{3.152707in}}{\pgfqpoint{1.729549in}{3.144893in}}%
\pgfpathcurveto{\pgfqpoint{1.737363in}{3.137080in}}{\pgfqpoint{1.747962in}{3.132690in}}{\pgfqpoint{1.759012in}{3.132690in}}%
\pgfpathclose%
\pgfusepath{stroke,fill}%
\end{pgfscope}%
\begin{pgfscope}%
\pgfpathrectangle{\pgfqpoint{0.648703in}{0.548769in}}{\pgfqpoint{5.112893in}{3.102590in}}%
\pgfusepath{clip}%
\pgfsetbuttcap%
\pgfsetroundjoin%
\definecolor{currentfill}{rgb}{1.000000,0.498039,0.054902}%
\pgfsetfillcolor{currentfill}%
\pgfsetlinewidth{1.003750pt}%
\definecolor{currentstroke}{rgb}{1.000000,0.498039,0.054902}%
\pgfsetstrokecolor{currentstroke}%
\pgfsetdash{}{0pt}%
\pgfpathmoveto{\pgfqpoint{2.873386in}{3.132690in}}%
\pgfpathcurveto{\pgfqpoint{2.884436in}{3.132690in}}{\pgfqpoint{2.895035in}{3.137080in}}{\pgfqpoint{2.902849in}{3.144893in}}%
\pgfpathcurveto{\pgfqpoint{2.910663in}{3.152707in}}{\pgfqpoint{2.915053in}{3.163306in}}{\pgfqpoint{2.915053in}{3.174356in}}%
\pgfpathcurveto{\pgfqpoint{2.915053in}{3.185406in}}{\pgfqpoint{2.910663in}{3.196005in}}{\pgfqpoint{2.902849in}{3.203819in}}%
\pgfpathcurveto{\pgfqpoint{2.895035in}{3.211633in}}{\pgfqpoint{2.884436in}{3.216023in}}{\pgfqpoint{2.873386in}{3.216023in}}%
\pgfpathcurveto{\pgfqpoint{2.862336in}{3.216023in}}{\pgfqpoint{2.851737in}{3.211633in}}{\pgfqpoint{2.843924in}{3.203819in}}%
\pgfpathcurveto{\pgfqpoint{2.836110in}{3.196005in}}{\pgfqpoint{2.831720in}{3.185406in}}{\pgfqpoint{2.831720in}{3.174356in}}%
\pgfpathcurveto{\pgfqpoint{2.831720in}{3.163306in}}{\pgfqpoint{2.836110in}{3.152707in}}{\pgfqpoint{2.843924in}{3.144893in}}%
\pgfpathcurveto{\pgfqpoint{2.851737in}{3.137080in}}{\pgfqpoint{2.862336in}{3.132690in}}{\pgfqpoint{2.873386in}{3.132690in}}%
\pgfpathclose%
\pgfusepath{stroke,fill}%
\end{pgfscope}%
\begin{pgfscope}%
\pgfpathrectangle{\pgfqpoint{0.648703in}{0.548769in}}{\pgfqpoint{5.112893in}{3.102590in}}%
\pgfusepath{clip}%
\pgfsetbuttcap%
\pgfsetroundjoin%
\definecolor{currentfill}{rgb}{1.000000,0.498039,0.054902}%
\pgfsetfillcolor{currentfill}%
\pgfsetlinewidth{1.003750pt}%
\definecolor{currentstroke}{rgb}{1.000000,0.498039,0.054902}%
\pgfsetstrokecolor{currentstroke}%
\pgfsetdash{}{0pt}%
\pgfpathmoveto{\pgfqpoint{2.759826in}{3.136837in}}%
\pgfpathcurveto{\pgfqpoint{2.770876in}{3.136837in}}{\pgfqpoint{2.781475in}{3.141228in}}{\pgfqpoint{2.789289in}{3.149041in}}%
\pgfpathcurveto{\pgfqpoint{2.797102in}{3.156855in}}{\pgfqpoint{2.801492in}{3.167454in}}{\pgfqpoint{2.801492in}{3.178504in}}%
\pgfpathcurveto{\pgfqpoint{2.801492in}{3.189554in}}{\pgfqpoint{2.797102in}{3.200153in}}{\pgfqpoint{2.789289in}{3.207967in}}%
\pgfpathcurveto{\pgfqpoint{2.781475in}{3.215780in}}{\pgfqpoint{2.770876in}{3.220171in}}{\pgfqpoint{2.759826in}{3.220171in}}%
\pgfpathcurveto{\pgfqpoint{2.748776in}{3.220171in}}{\pgfqpoint{2.738177in}{3.215780in}}{\pgfqpoint{2.730363in}{3.207967in}}%
\pgfpathcurveto{\pgfqpoint{2.722549in}{3.200153in}}{\pgfqpoint{2.718159in}{3.189554in}}{\pgfqpoint{2.718159in}{3.178504in}}%
\pgfpathcurveto{\pgfqpoint{2.718159in}{3.167454in}}{\pgfqpoint{2.722549in}{3.156855in}}{\pgfqpoint{2.730363in}{3.149041in}}%
\pgfpathcurveto{\pgfqpoint{2.738177in}{3.141228in}}{\pgfqpoint{2.748776in}{3.136837in}}{\pgfqpoint{2.759826in}{3.136837in}}%
\pgfpathclose%
\pgfusepath{stroke,fill}%
\end{pgfscope}%
\begin{pgfscope}%
\pgfpathrectangle{\pgfqpoint{0.648703in}{0.548769in}}{\pgfqpoint{5.112893in}{3.102590in}}%
\pgfusepath{clip}%
\pgfsetbuttcap%
\pgfsetroundjoin%
\definecolor{currentfill}{rgb}{0.839216,0.152941,0.156863}%
\pgfsetfillcolor{currentfill}%
\pgfsetlinewidth{1.003750pt}%
\definecolor{currentstroke}{rgb}{0.839216,0.152941,0.156863}%
\pgfsetstrokecolor{currentstroke}%
\pgfsetdash{}{0pt}%
\pgfpathmoveto{\pgfqpoint{2.623831in}{3.120246in}}%
\pgfpathcurveto{\pgfqpoint{2.634882in}{3.120246in}}{\pgfqpoint{2.645481in}{3.124636in}}{\pgfqpoint{2.653294in}{3.132450in}}%
\pgfpathcurveto{\pgfqpoint{2.661108in}{3.140263in}}{\pgfqpoint{2.665498in}{3.150862in}}{\pgfqpoint{2.665498in}{3.161913in}}%
\pgfpathcurveto{\pgfqpoint{2.665498in}{3.172963in}}{\pgfqpoint{2.661108in}{3.183562in}}{\pgfqpoint{2.653294in}{3.191375in}}%
\pgfpathcurveto{\pgfqpoint{2.645481in}{3.199189in}}{\pgfqpoint{2.634882in}{3.203579in}}{\pgfqpoint{2.623831in}{3.203579in}}%
\pgfpathcurveto{\pgfqpoint{2.612781in}{3.203579in}}{\pgfqpoint{2.602182in}{3.199189in}}{\pgfqpoint{2.594369in}{3.191375in}}%
\pgfpathcurveto{\pgfqpoint{2.586555in}{3.183562in}}{\pgfqpoint{2.582165in}{3.172963in}}{\pgfqpoint{2.582165in}{3.161913in}}%
\pgfpathcurveto{\pgfqpoint{2.582165in}{3.150862in}}{\pgfqpoint{2.586555in}{3.140263in}}{\pgfqpoint{2.594369in}{3.132450in}}%
\pgfpathcurveto{\pgfqpoint{2.602182in}{3.124636in}}{\pgfqpoint{2.612781in}{3.120246in}}{\pgfqpoint{2.623831in}{3.120246in}}%
\pgfpathclose%
\pgfusepath{stroke,fill}%
\end{pgfscope}%
\begin{pgfscope}%
\pgfpathrectangle{\pgfqpoint{0.648703in}{0.548769in}}{\pgfqpoint{5.112893in}{3.102590in}}%
\pgfusepath{clip}%
\pgfsetbuttcap%
\pgfsetroundjoin%
\definecolor{currentfill}{rgb}{1.000000,0.498039,0.054902}%
\pgfsetfillcolor{currentfill}%
\pgfsetlinewidth{1.003750pt}%
\definecolor{currentstroke}{rgb}{1.000000,0.498039,0.054902}%
\pgfsetstrokecolor{currentstroke}%
\pgfsetdash{}{0pt}%
\pgfpathmoveto{\pgfqpoint{3.370013in}{3.244681in}}%
\pgfpathcurveto{\pgfqpoint{3.381063in}{3.244681in}}{\pgfqpoint{3.391662in}{3.249072in}}{\pgfqpoint{3.399475in}{3.256885in}}%
\pgfpathcurveto{\pgfqpoint{3.407289in}{3.264699in}}{\pgfqpoint{3.411679in}{3.275298in}}{\pgfqpoint{3.411679in}{3.286348in}}%
\pgfpathcurveto{\pgfqpoint{3.411679in}{3.297398in}}{\pgfqpoint{3.407289in}{3.307997in}}{\pgfqpoint{3.399475in}{3.315811in}}%
\pgfpathcurveto{\pgfqpoint{3.391662in}{3.323624in}}{\pgfqpoint{3.381063in}{3.328015in}}{\pgfqpoint{3.370013in}{3.328015in}}%
\pgfpathcurveto{\pgfqpoint{3.358963in}{3.328015in}}{\pgfqpoint{3.348364in}{3.323624in}}{\pgfqpoint{3.340550in}{3.315811in}}%
\pgfpathcurveto{\pgfqpoint{3.332736in}{3.307997in}}{\pgfqpoint{3.328346in}{3.297398in}}{\pgfqpoint{3.328346in}{3.286348in}}%
\pgfpathcurveto{\pgfqpoint{3.328346in}{3.275298in}}{\pgfqpoint{3.332736in}{3.264699in}}{\pgfqpoint{3.340550in}{3.256885in}}%
\pgfpathcurveto{\pgfqpoint{3.348364in}{3.249072in}}{\pgfqpoint{3.358963in}{3.244681in}}{\pgfqpoint{3.370013in}{3.244681in}}%
\pgfpathclose%
\pgfusepath{stroke,fill}%
\end{pgfscope}%
\begin{pgfscope}%
\pgfpathrectangle{\pgfqpoint{0.648703in}{0.548769in}}{\pgfqpoint{5.112893in}{3.102590in}}%
\pgfusepath{clip}%
\pgfsetbuttcap%
\pgfsetroundjoin%
\definecolor{currentfill}{rgb}{1.000000,0.498039,0.054902}%
\pgfsetfillcolor{currentfill}%
\pgfsetlinewidth{1.003750pt}%
\definecolor{currentstroke}{rgb}{1.000000,0.498039,0.054902}%
\pgfsetstrokecolor{currentstroke}%
\pgfsetdash{}{0pt}%
\pgfpathmoveto{\pgfqpoint{3.103462in}{3.128542in}}%
\pgfpathcurveto{\pgfqpoint{3.114512in}{3.128542in}}{\pgfqpoint{3.125111in}{3.132932in}}{\pgfqpoint{3.132924in}{3.140746in}}%
\pgfpathcurveto{\pgfqpoint{3.140738in}{3.148559in}}{\pgfqpoint{3.145128in}{3.159158in}}{\pgfqpoint{3.145128in}{3.170208in}}%
\pgfpathcurveto{\pgfqpoint{3.145128in}{3.181258in}}{\pgfqpoint{3.140738in}{3.191857in}}{\pgfqpoint{3.132924in}{3.199671in}}%
\pgfpathcurveto{\pgfqpoint{3.125111in}{3.207485in}}{\pgfqpoint{3.114512in}{3.211875in}}{\pgfqpoint{3.103462in}{3.211875in}}%
\pgfpathcurveto{\pgfqpoint{3.092412in}{3.211875in}}{\pgfqpoint{3.081812in}{3.207485in}}{\pgfqpoint{3.073999in}{3.199671in}}%
\pgfpathcurveto{\pgfqpoint{3.066185in}{3.191857in}}{\pgfqpoint{3.061795in}{3.181258in}}{\pgfqpoint{3.061795in}{3.170208in}}%
\pgfpathcurveto{\pgfqpoint{3.061795in}{3.159158in}}{\pgfqpoint{3.066185in}{3.148559in}}{\pgfqpoint{3.073999in}{3.140746in}}%
\pgfpathcurveto{\pgfqpoint{3.081812in}{3.132932in}}{\pgfqpoint{3.092412in}{3.128542in}}{\pgfqpoint{3.103462in}{3.128542in}}%
\pgfpathclose%
\pgfusepath{stroke,fill}%
\end{pgfscope}%
\begin{pgfscope}%
\pgfpathrectangle{\pgfqpoint{0.648703in}{0.548769in}}{\pgfqpoint{5.112893in}{3.102590in}}%
\pgfusepath{clip}%
\pgfsetbuttcap%
\pgfsetroundjoin%
\definecolor{currentfill}{rgb}{1.000000,0.498039,0.054902}%
\pgfsetfillcolor{currentfill}%
\pgfsetlinewidth{1.003750pt}%
\definecolor{currentstroke}{rgb}{1.000000,0.498039,0.054902}%
\pgfsetstrokecolor{currentstroke}%
\pgfsetdash{}{0pt}%
\pgfpathmoveto{\pgfqpoint{2.223498in}{3.136837in}}%
\pgfpathcurveto{\pgfqpoint{2.234549in}{3.136837in}}{\pgfqpoint{2.245148in}{3.141228in}}{\pgfqpoint{2.252961in}{3.149041in}}%
\pgfpathcurveto{\pgfqpoint{2.260775in}{3.156855in}}{\pgfqpoint{2.265165in}{3.167454in}}{\pgfqpoint{2.265165in}{3.178504in}}%
\pgfpathcurveto{\pgfqpoint{2.265165in}{3.189554in}}{\pgfqpoint{2.260775in}{3.200153in}}{\pgfqpoint{2.252961in}{3.207967in}}%
\pgfpathcurveto{\pgfqpoint{2.245148in}{3.215780in}}{\pgfqpoint{2.234549in}{3.220171in}}{\pgfqpoint{2.223498in}{3.220171in}}%
\pgfpathcurveto{\pgfqpoint{2.212448in}{3.220171in}}{\pgfqpoint{2.201849in}{3.215780in}}{\pgfqpoint{2.194036in}{3.207967in}}%
\pgfpathcurveto{\pgfqpoint{2.186222in}{3.200153in}}{\pgfqpoint{2.181832in}{3.189554in}}{\pgfqpoint{2.181832in}{3.178504in}}%
\pgfpathcurveto{\pgfqpoint{2.181832in}{3.167454in}}{\pgfqpoint{2.186222in}{3.156855in}}{\pgfqpoint{2.194036in}{3.149041in}}%
\pgfpathcurveto{\pgfqpoint{2.201849in}{3.141228in}}{\pgfqpoint{2.212448in}{3.136837in}}{\pgfqpoint{2.223498in}{3.136837in}}%
\pgfpathclose%
\pgfusepath{stroke,fill}%
\end{pgfscope}%
\begin{pgfscope}%
\pgfpathrectangle{\pgfqpoint{0.648703in}{0.548769in}}{\pgfqpoint{5.112893in}{3.102590in}}%
\pgfusepath{clip}%
\pgfsetbuttcap%
\pgfsetroundjoin%
\definecolor{currentfill}{rgb}{1.000000,0.498039,0.054902}%
\pgfsetfillcolor{currentfill}%
\pgfsetlinewidth{1.003750pt}%
\definecolor{currentstroke}{rgb}{1.000000,0.498039,0.054902}%
\pgfsetstrokecolor{currentstroke}%
\pgfsetdash{}{0pt}%
\pgfpathmoveto{\pgfqpoint{1.613511in}{3.136837in}}%
\pgfpathcurveto{\pgfqpoint{1.624561in}{3.136837in}}{\pgfqpoint{1.635160in}{3.141228in}}{\pgfqpoint{1.642973in}{3.149041in}}%
\pgfpathcurveto{\pgfqpoint{1.650787in}{3.156855in}}{\pgfqpoint{1.655177in}{3.167454in}}{\pgfqpoint{1.655177in}{3.178504in}}%
\pgfpathcurveto{\pgfqpoint{1.655177in}{3.189554in}}{\pgfqpoint{1.650787in}{3.200153in}}{\pgfqpoint{1.642973in}{3.207967in}}%
\pgfpathcurveto{\pgfqpoint{1.635160in}{3.215780in}}{\pgfqpoint{1.624561in}{3.220171in}}{\pgfqpoint{1.613511in}{3.220171in}}%
\pgfpathcurveto{\pgfqpoint{1.602461in}{3.220171in}}{\pgfqpoint{1.591861in}{3.215780in}}{\pgfqpoint{1.584048in}{3.207967in}}%
\pgfpathcurveto{\pgfqpoint{1.576234in}{3.200153in}}{\pgfqpoint{1.571844in}{3.189554in}}{\pgfqpoint{1.571844in}{3.178504in}}%
\pgfpathcurveto{\pgfqpoint{1.571844in}{3.167454in}}{\pgfqpoint{1.576234in}{3.156855in}}{\pgfqpoint{1.584048in}{3.149041in}}%
\pgfpathcurveto{\pgfqpoint{1.591861in}{3.141228in}}{\pgfqpoint{1.602461in}{3.136837in}}{\pgfqpoint{1.613511in}{3.136837in}}%
\pgfpathclose%
\pgfusepath{stroke,fill}%
\end{pgfscope}%
\begin{pgfscope}%
\pgfpathrectangle{\pgfqpoint{0.648703in}{0.548769in}}{\pgfqpoint{5.112893in}{3.102590in}}%
\pgfusepath{clip}%
\pgfsetbuttcap%
\pgfsetroundjoin%
\definecolor{currentfill}{rgb}{1.000000,0.498039,0.054902}%
\pgfsetfillcolor{currentfill}%
\pgfsetlinewidth{1.003750pt}%
\definecolor{currentstroke}{rgb}{1.000000,0.498039,0.054902}%
\pgfsetstrokecolor{currentstroke}%
\pgfsetdash{}{0pt}%
\pgfpathmoveto{\pgfqpoint{1.785536in}{3.136837in}}%
\pgfpathcurveto{\pgfqpoint{1.796586in}{3.136837in}}{\pgfqpoint{1.807185in}{3.141228in}}{\pgfqpoint{1.814999in}{3.149041in}}%
\pgfpathcurveto{\pgfqpoint{1.822812in}{3.156855in}}{\pgfqpoint{1.827202in}{3.167454in}}{\pgfqpoint{1.827202in}{3.178504in}}%
\pgfpathcurveto{\pgfqpoint{1.827202in}{3.189554in}}{\pgfqpoint{1.822812in}{3.200153in}}{\pgfqpoint{1.814999in}{3.207967in}}%
\pgfpathcurveto{\pgfqpoint{1.807185in}{3.215780in}}{\pgfqpoint{1.796586in}{3.220171in}}{\pgfqpoint{1.785536in}{3.220171in}}%
\pgfpathcurveto{\pgfqpoint{1.774486in}{3.220171in}}{\pgfqpoint{1.763887in}{3.215780in}}{\pgfqpoint{1.756073in}{3.207967in}}%
\pgfpathcurveto{\pgfqpoint{1.748259in}{3.200153in}}{\pgfqpoint{1.743869in}{3.189554in}}{\pgfqpoint{1.743869in}{3.178504in}}%
\pgfpathcurveto{\pgfqpoint{1.743869in}{3.167454in}}{\pgfqpoint{1.748259in}{3.156855in}}{\pgfqpoint{1.756073in}{3.149041in}}%
\pgfpathcurveto{\pgfqpoint{1.763887in}{3.141228in}}{\pgfqpoint{1.774486in}{3.136837in}}{\pgfqpoint{1.785536in}{3.136837in}}%
\pgfpathclose%
\pgfusepath{stroke,fill}%
\end{pgfscope}%
\begin{pgfscope}%
\pgfpathrectangle{\pgfqpoint{0.648703in}{0.548769in}}{\pgfqpoint{5.112893in}{3.102590in}}%
\pgfusepath{clip}%
\pgfsetbuttcap%
\pgfsetroundjoin%
\definecolor{currentfill}{rgb}{1.000000,0.498039,0.054902}%
\pgfsetfillcolor{currentfill}%
\pgfsetlinewidth{1.003750pt}%
\definecolor{currentstroke}{rgb}{1.000000,0.498039,0.054902}%
\pgfsetstrokecolor{currentstroke}%
\pgfsetdash{}{0pt}%
\pgfpathmoveto{\pgfqpoint{2.885377in}{3.132690in}}%
\pgfpathcurveto{\pgfqpoint{2.896427in}{3.132690in}}{\pgfqpoint{2.907026in}{3.137080in}}{\pgfqpoint{2.914839in}{3.144893in}}%
\pgfpathcurveto{\pgfqpoint{2.922653in}{3.152707in}}{\pgfqpoint{2.927043in}{3.163306in}}{\pgfqpoint{2.927043in}{3.174356in}}%
\pgfpathcurveto{\pgfqpoint{2.927043in}{3.185406in}}{\pgfqpoint{2.922653in}{3.196005in}}{\pgfqpoint{2.914839in}{3.203819in}}%
\pgfpathcurveto{\pgfqpoint{2.907026in}{3.211633in}}{\pgfqpoint{2.896427in}{3.216023in}}{\pgfqpoint{2.885377in}{3.216023in}}%
\pgfpathcurveto{\pgfqpoint{2.874326in}{3.216023in}}{\pgfqpoint{2.863727in}{3.211633in}}{\pgfqpoint{2.855914in}{3.203819in}}%
\pgfpathcurveto{\pgfqpoint{2.848100in}{3.196005in}}{\pgfqpoint{2.843710in}{3.185406in}}{\pgfqpoint{2.843710in}{3.174356in}}%
\pgfpathcurveto{\pgfqpoint{2.843710in}{3.163306in}}{\pgfqpoint{2.848100in}{3.152707in}}{\pgfqpoint{2.855914in}{3.144893in}}%
\pgfpathcurveto{\pgfqpoint{2.863727in}{3.137080in}}{\pgfqpoint{2.874326in}{3.132690in}}{\pgfqpoint{2.885377in}{3.132690in}}%
\pgfpathclose%
\pgfusepath{stroke,fill}%
\end{pgfscope}%
\begin{pgfscope}%
\pgfpathrectangle{\pgfqpoint{0.648703in}{0.548769in}}{\pgfqpoint{5.112893in}{3.102590in}}%
\pgfusepath{clip}%
\pgfsetbuttcap%
\pgfsetroundjoin%
\definecolor{currentfill}{rgb}{1.000000,0.498039,0.054902}%
\pgfsetfillcolor{currentfill}%
\pgfsetlinewidth{1.003750pt}%
\definecolor{currentstroke}{rgb}{1.000000,0.498039,0.054902}%
\pgfsetstrokecolor{currentstroke}%
\pgfsetdash{}{0pt}%
\pgfpathmoveto{\pgfqpoint{3.173908in}{3.107802in}}%
\pgfpathcurveto{\pgfqpoint{3.184958in}{3.107802in}}{\pgfqpoint{3.195557in}{3.112193in}}{\pgfqpoint{3.203371in}{3.120006in}}%
\pgfpathcurveto{\pgfqpoint{3.211185in}{3.127820in}}{\pgfqpoint{3.215575in}{3.138419in}}{\pgfqpoint{3.215575in}{3.149469in}}%
\pgfpathcurveto{\pgfqpoint{3.215575in}{3.160519in}}{\pgfqpoint{3.211185in}{3.171118in}}{\pgfqpoint{3.203371in}{3.178932in}}%
\pgfpathcurveto{\pgfqpoint{3.195557in}{3.186745in}}{\pgfqpoint{3.184958in}{3.191136in}}{\pgfqpoint{3.173908in}{3.191136in}}%
\pgfpathcurveto{\pgfqpoint{3.162858in}{3.191136in}}{\pgfqpoint{3.152259in}{3.186745in}}{\pgfqpoint{3.144445in}{3.178932in}}%
\pgfpathcurveto{\pgfqpoint{3.136632in}{3.171118in}}{\pgfqpoint{3.132242in}{3.160519in}}{\pgfqpoint{3.132242in}{3.149469in}}%
\pgfpathcurveto{\pgfqpoint{3.132242in}{3.138419in}}{\pgfqpoint{3.136632in}{3.127820in}}{\pgfqpoint{3.144445in}{3.120006in}}%
\pgfpathcurveto{\pgfqpoint{3.152259in}{3.112193in}}{\pgfqpoint{3.162858in}{3.107802in}}{\pgfqpoint{3.173908in}{3.107802in}}%
\pgfpathclose%
\pgfusepath{stroke,fill}%
\end{pgfscope}%
\begin{pgfscope}%
\pgfpathrectangle{\pgfqpoint{0.648703in}{0.548769in}}{\pgfqpoint{5.112893in}{3.102590in}}%
\pgfusepath{clip}%
\pgfsetbuttcap%
\pgfsetroundjoin%
\definecolor{currentfill}{rgb}{1.000000,0.498039,0.054902}%
\pgfsetfillcolor{currentfill}%
\pgfsetlinewidth{1.003750pt}%
\definecolor{currentstroke}{rgb}{1.000000,0.498039,0.054902}%
\pgfsetstrokecolor{currentstroke}%
\pgfsetdash{}{0pt}%
\pgfpathmoveto{\pgfqpoint{3.183645in}{3.132690in}}%
\pgfpathcurveto{\pgfqpoint{3.194695in}{3.132690in}}{\pgfqpoint{3.205294in}{3.137080in}}{\pgfqpoint{3.213108in}{3.144893in}}%
\pgfpathcurveto{\pgfqpoint{3.220922in}{3.152707in}}{\pgfqpoint{3.225312in}{3.163306in}}{\pgfqpoint{3.225312in}{3.174356in}}%
\pgfpathcurveto{\pgfqpoint{3.225312in}{3.185406in}}{\pgfqpoint{3.220922in}{3.196005in}}{\pgfqpoint{3.213108in}{3.203819in}}%
\pgfpathcurveto{\pgfqpoint{3.205294in}{3.211633in}}{\pgfqpoint{3.194695in}{3.216023in}}{\pgfqpoint{3.183645in}{3.216023in}}%
\pgfpathcurveto{\pgfqpoint{3.172595in}{3.216023in}}{\pgfqpoint{3.161996in}{3.211633in}}{\pgfqpoint{3.154182in}{3.203819in}}%
\pgfpathcurveto{\pgfqpoint{3.146369in}{3.196005in}}{\pgfqpoint{3.141978in}{3.185406in}}{\pgfqpoint{3.141978in}{3.174356in}}%
\pgfpathcurveto{\pgfqpoint{3.141978in}{3.163306in}}{\pgfqpoint{3.146369in}{3.152707in}}{\pgfqpoint{3.154182in}{3.144893in}}%
\pgfpathcurveto{\pgfqpoint{3.161996in}{3.137080in}}{\pgfqpoint{3.172595in}{3.132690in}}{\pgfqpoint{3.183645in}{3.132690in}}%
\pgfpathclose%
\pgfusepath{stroke,fill}%
\end{pgfscope}%
\begin{pgfscope}%
\pgfpathrectangle{\pgfqpoint{0.648703in}{0.548769in}}{\pgfqpoint{5.112893in}{3.102590in}}%
\pgfusepath{clip}%
\pgfsetbuttcap%
\pgfsetroundjoin%
\definecolor{currentfill}{rgb}{0.121569,0.466667,0.705882}%
\pgfsetfillcolor{currentfill}%
\pgfsetlinewidth{1.003750pt}%
\definecolor{currentstroke}{rgb}{0.121569,0.466667,0.705882}%
\pgfsetstrokecolor{currentstroke}%
\pgfsetdash{}{0pt}%
\pgfpathmoveto{\pgfqpoint{0.814186in}{0.648129in}}%
\pgfpathcurveto{\pgfqpoint{0.825236in}{0.648129in}}{\pgfqpoint{0.835835in}{0.652519in}}{\pgfqpoint{0.843649in}{0.660333in}}%
\pgfpathcurveto{\pgfqpoint{0.851462in}{0.668146in}}{\pgfqpoint{0.855853in}{0.678745in}}{\pgfqpoint{0.855853in}{0.689796in}}%
\pgfpathcurveto{\pgfqpoint{0.855853in}{0.700846in}}{\pgfqpoint{0.851462in}{0.711445in}}{\pgfqpoint{0.843649in}{0.719258in}}%
\pgfpathcurveto{\pgfqpoint{0.835835in}{0.727072in}}{\pgfqpoint{0.825236in}{0.731462in}}{\pgfqpoint{0.814186in}{0.731462in}}%
\pgfpathcurveto{\pgfqpoint{0.803136in}{0.731462in}}{\pgfqpoint{0.792537in}{0.727072in}}{\pgfqpoint{0.784723in}{0.719258in}}%
\pgfpathcurveto{\pgfqpoint{0.776909in}{0.711445in}}{\pgfqpoint{0.772519in}{0.700846in}}{\pgfqpoint{0.772519in}{0.689796in}}%
\pgfpathcurveto{\pgfqpoint{0.772519in}{0.678745in}}{\pgfqpoint{0.776909in}{0.668146in}}{\pgfqpoint{0.784723in}{0.660333in}}%
\pgfpathcurveto{\pgfqpoint{0.792537in}{0.652519in}}{\pgfqpoint{0.803136in}{0.648129in}}{\pgfqpoint{0.814186in}{0.648129in}}%
\pgfpathclose%
\pgfusepath{stroke,fill}%
\end{pgfscope}%
\begin{pgfscope}%
\pgfpathrectangle{\pgfqpoint{0.648703in}{0.548769in}}{\pgfqpoint{5.112893in}{3.102590in}}%
\pgfusepath{clip}%
\pgfsetbuttcap%
\pgfsetroundjoin%
\definecolor{currentfill}{rgb}{1.000000,0.498039,0.054902}%
\pgfsetfillcolor{currentfill}%
\pgfsetlinewidth{1.003750pt}%
\definecolor{currentstroke}{rgb}{1.000000,0.498039,0.054902}%
\pgfsetstrokecolor{currentstroke}%
\pgfsetdash{}{0pt}%
\pgfpathmoveto{\pgfqpoint{1.416513in}{3.140985in}}%
\pgfpathcurveto{\pgfqpoint{1.427563in}{3.140985in}}{\pgfqpoint{1.438163in}{3.145375in}}{\pgfqpoint{1.445976in}{3.153189in}}%
\pgfpathcurveto{\pgfqpoint{1.453790in}{3.161003in}}{\pgfqpoint{1.458180in}{3.171602in}}{\pgfqpoint{1.458180in}{3.182652in}}%
\pgfpathcurveto{\pgfqpoint{1.458180in}{3.193702in}}{\pgfqpoint{1.453790in}{3.204301in}}{\pgfqpoint{1.445976in}{3.212115in}}%
\pgfpathcurveto{\pgfqpoint{1.438163in}{3.219928in}}{\pgfqpoint{1.427563in}{3.224319in}}{\pgfqpoint{1.416513in}{3.224319in}}%
\pgfpathcurveto{\pgfqpoint{1.405463in}{3.224319in}}{\pgfqpoint{1.394864in}{3.219928in}}{\pgfqpoint{1.387051in}{3.212115in}}%
\pgfpathcurveto{\pgfqpoint{1.379237in}{3.204301in}}{\pgfqpoint{1.374847in}{3.193702in}}{\pgfqpoint{1.374847in}{3.182652in}}%
\pgfpathcurveto{\pgfqpoint{1.374847in}{3.171602in}}{\pgfqpoint{1.379237in}{3.161003in}}{\pgfqpoint{1.387051in}{3.153189in}}%
\pgfpathcurveto{\pgfqpoint{1.394864in}{3.145375in}}{\pgfqpoint{1.405463in}{3.140985in}}{\pgfqpoint{1.416513in}{3.140985in}}%
\pgfpathclose%
\pgfusepath{stroke,fill}%
\end{pgfscope}%
\begin{pgfscope}%
\pgfpathrectangle{\pgfqpoint{0.648703in}{0.548769in}}{\pgfqpoint{5.112893in}{3.102590in}}%
\pgfusepath{clip}%
\pgfsetbuttcap%
\pgfsetroundjoin%
\definecolor{currentfill}{rgb}{1.000000,0.498039,0.054902}%
\pgfsetfillcolor{currentfill}%
\pgfsetlinewidth{1.003750pt}%
\definecolor{currentstroke}{rgb}{1.000000,0.498039,0.054902}%
\pgfsetstrokecolor{currentstroke}%
\pgfsetdash{}{0pt}%
\pgfpathmoveto{\pgfqpoint{3.374747in}{3.145133in}}%
\pgfpathcurveto{\pgfqpoint{3.385798in}{3.145133in}}{\pgfqpoint{3.396397in}{3.149523in}}{\pgfqpoint{3.404210in}{3.157337in}}%
\pgfpathcurveto{\pgfqpoint{3.412024in}{3.165151in}}{\pgfqpoint{3.416414in}{3.175750in}}{\pgfqpoint{3.416414in}{3.186800in}}%
\pgfpathcurveto{\pgfqpoint{3.416414in}{3.197850in}}{\pgfqpoint{3.412024in}{3.208449in}}{\pgfqpoint{3.404210in}{3.216262in}}%
\pgfpathcurveto{\pgfqpoint{3.396397in}{3.224076in}}{\pgfqpoint{3.385798in}{3.228466in}}{\pgfqpoint{3.374747in}{3.228466in}}%
\pgfpathcurveto{\pgfqpoint{3.363697in}{3.228466in}}{\pgfqpoint{3.353098in}{3.224076in}}{\pgfqpoint{3.345285in}{3.216262in}}%
\pgfpathcurveto{\pgfqpoint{3.337471in}{3.208449in}}{\pgfqpoint{3.333081in}{3.197850in}}{\pgfqpoint{3.333081in}{3.186800in}}%
\pgfpathcurveto{\pgfqpoint{3.333081in}{3.175750in}}{\pgfqpoint{3.337471in}{3.165151in}}{\pgfqpoint{3.345285in}{3.157337in}}%
\pgfpathcurveto{\pgfqpoint{3.353098in}{3.149523in}}{\pgfqpoint{3.363697in}{3.145133in}}{\pgfqpoint{3.374747in}{3.145133in}}%
\pgfpathclose%
\pgfusepath{stroke,fill}%
\end{pgfscope}%
\begin{pgfscope}%
\pgfpathrectangle{\pgfqpoint{0.648703in}{0.548769in}}{\pgfqpoint{5.112893in}{3.102590in}}%
\pgfusepath{clip}%
\pgfsetbuttcap%
\pgfsetroundjoin%
\definecolor{currentfill}{rgb}{1.000000,0.498039,0.054902}%
\pgfsetfillcolor{currentfill}%
\pgfsetlinewidth{1.003750pt}%
\definecolor{currentstroke}{rgb}{1.000000,0.498039,0.054902}%
\pgfsetstrokecolor{currentstroke}%
\pgfsetdash{}{0pt}%
\pgfpathmoveto{\pgfqpoint{3.484139in}{3.140985in}}%
\pgfpathcurveto{\pgfqpoint{3.495189in}{3.140985in}}{\pgfqpoint{3.505788in}{3.145375in}}{\pgfqpoint{3.513602in}{3.153189in}}%
\pgfpathcurveto{\pgfqpoint{3.521415in}{3.161003in}}{\pgfqpoint{3.525806in}{3.171602in}}{\pgfqpoint{3.525806in}{3.182652in}}%
\pgfpathcurveto{\pgfqpoint{3.525806in}{3.193702in}}{\pgfqpoint{3.521415in}{3.204301in}}{\pgfqpoint{3.513602in}{3.212115in}}%
\pgfpathcurveto{\pgfqpoint{3.505788in}{3.219928in}}{\pgfqpoint{3.495189in}{3.224319in}}{\pgfqpoint{3.484139in}{3.224319in}}%
\pgfpathcurveto{\pgfqpoint{3.473089in}{3.224319in}}{\pgfqpoint{3.462490in}{3.219928in}}{\pgfqpoint{3.454676in}{3.212115in}}%
\pgfpathcurveto{\pgfqpoint{3.446863in}{3.204301in}}{\pgfqpoint{3.442472in}{3.193702in}}{\pgfqpoint{3.442472in}{3.182652in}}%
\pgfpathcurveto{\pgfqpoint{3.442472in}{3.171602in}}{\pgfqpoint{3.446863in}{3.161003in}}{\pgfqpoint{3.454676in}{3.153189in}}%
\pgfpathcurveto{\pgfqpoint{3.462490in}{3.145375in}}{\pgfqpoint{3.473089in}{3.140985in}}{\pgfqpoint{3.484139in}{3.140985in}}%
\pgfpathclose%
\pgfusepath{stroke,fill}%
\end{pgfscope}%
\begin{pgfscope}%
\pgfpathrectangle{\pgfqpoint{0.648703in}{0.548769in}}{\pgfqpoint{5.112893in}{3.102590in}}%
\pgfusepath{clip}%
\pgfsetbuttcap%
\pgfsetroundjoin%
\definecolor{currentfill}{rgb}{0.121569,0.466667,0.705882}%
\pgfsetfillcolor{currentfill}%
\pgfsetlinewidth{1.003750pt}%
\definecolor{currentstroke}{rgb}{0.121569,0.466667,0.705882}%
\pgfsetstrokecolor{currentstroke}%
\pgfsetdash{}{0pt}%
\pgfpathmoveto{\pgfqpoint{0.924582in}{0.697903in}}%
\pgfpathcurveto{\pgfqpoint{0.935632in}{0.697903in}}{\pgfqpoint{0.946231in}{0.702293in}}{\pgfqpoint{0.954045in}{0.710107in}}%
\pgfpathcurveto{\pgfqpoint{0.961859in}{0.717921in}}{\pgfqpoint{0.966249in}{0.728520in}}{\pgfqpoint{0.966249in}{0.739570in}}%
\pgfpathcurveto{\pgfqpoint{0.966249in}{0.750620in}}{\pgfqpoint{0.961859in}{0.761219in}}{\pgfqpoint{0.954045in}{0.769033in}}%
\pgfpathcurveto{\pgfqpoint{0.946231in}{0.776846in}}{\pgfqpoint{0.935632in}{0.781236in}}{\pgfqpoint{0.924582in}{0.781236in}}%
\pgfpathcurveto{\pgfqpoint{0.913532in}{0.781236in}}{\pgfqpoint{0.902933in}{0.776846in}}{\pgfqpoint{0.895119in}{0.769033in}}%
\pgfpathcurveto{\pgfqpoint{0.887306in}{0.761219in}}{\pgfqpoint{0.882915in}{0.750620in}}{\pgfqpoint{0.882915in}{0.739570in}}%
\pgfpathcurveto{\pgfqpoint{0.882915in}{0.728520in}}{\pgfqpoint{0.887306in}{0.717921in}}{\pgfqpoint{0.895119in}{0.710107in}}%
\pgfpathcurveto{\pgfqpoint{0.902933in}{0.702293in}}{\pgfqpoint{0.913532in}{0.697903in}}{\pgfqpoint{0.924582in}{0.697903in}}%
\pgfpathclose%
\pgfusepath{stroke,fill}%
\end{pgfscope}%
\begin{pgfscope}%
\pgfpathrectangle{\pgfqpoint{0.648703in}{0.548769in}}{\pgfqpoint{5.112893in}{3.102590in}}%
\pgfusepath{clip}%
\pgfsetbuttcap%
\pgfsetroundjoin%
\definecolor{currentfill}{rgb}{1.000000,0.498039,0.054902}%
\pgfsetfillcolor{currentfill}%
\pgfsetlinewidth{1.003750pt}%
\definecolor{currentstroke}{rgb}{1.000000,0.498039,0.054902}%
\pgfsetstrokecolor{currentstroke}%
\pgfsetdash{}{0pt}%
\pgfpathmoveto{\pgfqpoint{1.606355in}{3.140985in}}%
\pgfpathcurveto{\pgfqpoint{1.617405in}{3.140985in}}{\pgfqpoint{1.628004in}{3.145375in}}{\pgfqpoint{1.635817in}{3.153189in}}%
\pgfpathcurveto{\pgfqpoint{1.643631in}{3.161003in}}{\pgfqpoint{1.648021in}{3.171602in}}{\pgfqpoint{1.648021in}{3.182652in}}%
\pgfpathcurveto{\pgfqpoint{1.648021in}{3.193702in}}{\pgfqpoint{1.643631in}{3.204301in}}{\pgfqpoint{1.635817in}{3.212115in}}%
\pgfpathcurveto{\pgfqpoint{1.628004in}{3.219928in}}{\pgfqpoint{1.617405in}{3.224319in}}{\pgfqpoint{1.606355in}{3.224319in}}%
\pgfpathcurveto{\pgfqpoint{1.595305in}{3.224319in}}{\pgfqpoint{1.584705in}{3.219928in}}{\pgfqpoint{1.576892in}{3.212115in}}%
\pgfpathcurveto{\pgfqpoint{1.569078in}{3.204301in}}{\pgfqpoint{1.564688in}{3.193702in}}{\pgfqpoint{1.564688in}{3.182652in}}%
\pgfpathcurveto{\pgfqpoint{1.564688in}{3.171602in}}{\pgfqpoint{1.569078in}{3.161003in}}{\pgfqpoint{1.576892in}{3.153189in}}%
\pgfpathcurveto{\pgfqpoint{1.584705in}{3.145375in}}{\pgfqpoint{1.595305in}{3.140985in}}{\pgfqpoint{1.606355in}{3.140985in}}%
\pgfpathclose%
\pgfusepath{stroke,fill}%
\end{pgfscope}%
\begin{pgfscope}%
\pgfpathrectangle{\pgfqpoint{0.648703in}{0.548769in}}{\pgfqpoint{5.112893in}{3.102590in}}%
\pgfusepath{clip}%
\pgfsetbuttcap%
\pgfsetroundjoin%
\definecolor{currentfill}{rgb}{1.000000,0.498039,0.054902}%
\pgfsetfillcolor{currentfill}%
\pgfsetlinewidth{1.003750pt}%
\definecolor{currentstroke}{rgb}{1.000000,0.498039,0.054902}%
\pgfsetstrokecolor{currentstroke}%
\pgfsetdash{}{0pt}%
\pgfpathmoveto{\pgfqpoint{2.544034in}{3.132690in}}%
\pgfpathcurveto{\pgfqpoint{2.555085in}{3.132690in}}{\pgfqpoint{2.565684in}{3.137080in}}{\pgfqpoint{2.573497in}{3.144893in}}%
\pgfpathcurveto{\pgfqpoint{2.581311in}{3.152707in}}{\pgfqpoint{2.585701in}{3.163306in}}{\pgfqpoint{2.585701in}{3.174356in}}%
\pgfpathcurveto{\pgfqpoint{2.585701in}{3.185406in}}{\pgfqpoint{2.581311in}{3.196005in}}{\pgfqpoint{2.573497in}{3.203819in}}%
\pgfpathcurveto{\pgfqpoint{2.565684in}{3.211633in}}{\pgfqpoint{2.555085in}{3.216023in}}{\pgfqpoint{2.544034in}{3.216023in}}%
\pgfpathcurveto{\pgfqpoint{2.532984in}{3.216023in}}{\pgfqpoint{2.522385in}{3.211633in}}{\pgfqpoint{2.514572in}{3.203819in}}%
\pgfpathcurveto{\pgfqpoint{2.506758in}{3.196005in}}{\pgfqpoint{2.502368in}{3.185406in}}{\pgfqpoint{2.502368in}{3.174356in}}%
\pgfpathcurveto{\pgfqpoint{2.502368in}{3.163306in}}{\pgfqpoint{2.506758in}{3.152707in}}{\pgfqpoint{2.514572in}{3.144893in}}%
\pgfpathcurveto{\pgfqpoint{2.522385in}{3.137080in}}{\pgfqpoint{2.532984in}{3.132690in}}{\pgfqpoint{2.544034in}{3.132690in}}%
\pgfpathclose%
\pgfusepath{stroke,fill}%
\end{pgfscope}%
\begin{pgfscope}%
\pgfpathrectangle{\pgfqpoint{0.648703in}{0.548769in}}{\pgfqpoint{5.112893in}{3.102590in}}%
\pgfusepath{clip}%
\pgfsetbuttcap%
\pgfsetroundjoin%
\definecolor{currentfill}{rgb}{1.000000,0.498039,0.054902}%
\pgfsetfillcolor{currentfill}%
\pgfsetlinewidth{1.003750pt}%
\definecolor{currentstroke}{rgb}{1.000000,0.498039,0.054902}%
\pgfsetstrokecolor{currentstroke}%
\pgfsetdash{}{0pt}%
\pgfpathmoveto{\pgfqpoint{3.340238in}{3.140985in}}%
\pgfpathcurveto{\pgfqpoint{3.351288in}{3.140985in}}{\pgfqpoint{3.361887in}{3.145375in}}{\pgfqpoint{3.369701in}{3.153189in}}%
\pgfpathcurveto{\pgfqpoint{3.377515in}{3.161003in}}{\pgfqpoint{3.381905in}{3.171602in}}{\pgfqpoint{3.381905in}{3.182652in}}%
\pgfpathcurveto{\pgfqpoint{3.381905in}{3.193702in}}{\pgfqpoint{3.377515in}{3.204301in}}{\pgfqpoint{3.369701in}{3.212115in}}%
\pgfpathcurveto{\pgfqpoint{3.361887in}{3.219928in}}{\pgfqpoint{3.351288in}{3.224319in}}{\pgfqpoint{3.340238in}{3.224319in}}%
\pgfpathcurveto{\pgfqpoint{3.329188in}{3.224319in}}{\pgfqpoint{3.318589in}{3.219928in}}{\pgfqpoint{3.310776in}{3.212115in}}%
\pgfpathcurveto{\pgfqpoint{3.302962in}{3.204301in}}{\pgfqpoint{3.298572in}{3.193702in}}{\pgfqpoint{3.298572in}{3.182652in}}%
\pgfpathcurveto{\pgfqpoint{3.298572in}{3.171602in}}{\pgfqpoint{3.302962in}{3.161003in}}{\pgfqpoint{3.310776in}{3.153189in}}%
\pgfpathcurveto{\pgfqpoint{3.318589in}{3.145375in}}{\pgfqpoint{3.329188in}{3.140985in}}{\pgfqpoint{3.340238in}{3.140985in}}%
\pgfpathclose%
\pgfusepath{stroke,fill}%
\end{pgfscope}%
\begin{pgfscope}%
\pgfpathrectangle{\pgfqpoint{0.648703in}{0.548769in}}{\pgfqpoint{5.112893in}{3.102590in}}%
\pgfusepath{clip}%
\pgfsetbuttcap%
\pgfsetroundjoin%
\definecolor{currentfill}{rgb}{1.000000,0.498039,0.054902}%
\pgfsetfillcolor{currentfill}%
\pgfsetlinewidth{1.003750pt}%
\definecolor{currentstroke}{rgb}{1.000000,0.498039,0.054902}%
\pgfsetstrokecolor{currentstroke}%
\pgfsetdash{}{0pt}%
\pgfpathmoveto{\pgfqpoint{1.460542in}{3.132690in}}%
\pgfpathcurveto{\pgfqpoint{1.471592in}{3.132690in}}{\pgfqpoint{1.482191in}{3.137080in}}{\pgfqpoint{1.490005in}{3.144893in}}%
\pgfpathcurveto{\pgfqpoint{1.497819in}{3.152707in}}{\pgfqpoint{1.502209in}{3.163306in}}{\pgfqpoint{1.502209in}{3.174356in}}%
\pgfpathcurveto{\pgfqpoint{1.502209in}{3.185406in}}{\pgfqpoint{1.497819in}{3.196005in}}{\pgfqpoint{1.490005in}{3.203819in}}%
\pgfpathcurveto{\pgfqpoint{1.482191in}{3.211633in}}{\pgfqpoint{1.471592in}{3.216023in}}{\pgfqpoint{1.460542in}{3.216023in}}%
\pgfpathcurveto{\pgfqpoint{1.449492in}{3.216023in}}{\pgfqpoint{1.438893in}{3.211633in}}{\pgfqpoint{1.431079in}{3.203819in}}%
\pgfpathcurveto{\pgfqpoint{1.423266in}{3.196005in}}{\pgfqpoint{1.418876in}{3.185406in}}{\pgfqpoint{1.418876in}{3.174356in}}%
\pgfpathcurveto{\pgfqpoint{1.418876in}{3.163306in}}{\pgfqpoint{1.423266in}{3.152707in}}{\pgfqpoint{1.431079in}{3.144893in}}%
\pgfpathcurveto{\pgfqpoint{1.438893in}{3.137080in}}{\pgfqpoint{1.449492in}{3.132690in}}{\pgfqpoint{1.460542in}{3.132690in}}%
\pgfpathclose%
\pgfusepath{stroke,fill}%
\end{pgfscope}%
\begin{pgfscope}%
\pgfpathrectangle{\pgfqpoint{0.648703in}{0.548769in}}{\pgfqpoint{5.112893in}{3.102590in}}%
\pgfusepath{clip}%
\pgfsetbuttcap%
\pgfsetroundjoin%
\definecolor{currentfill}{rgb}{1.000000,0.498039,0.054902}%
\pgfsetfillcolor{currentfill}%
\pgfsetlinewidth{1.003750pt}%
\definecolor{currentstroke}{rgb}{1.000000,0.498039,0.054902}%
\pgfsetstrokecolor{currentstroke}%
\pgfsetdash{}{0pt}%
\pgfpathmoveto{\pgfqpoint{2.462963in}{3.140985in}}%
\pgfpathcurveto{\pgfqpoint{2.474013in}{3.140985in}}{\pgfqpoint{2.484612in}{3.145375in}}{\pgfqpoint{2.492426in}{3.153189in}}%
\pgfpathcurveto{\pgfqpoint{2.500239in}{3.161003in}}{\pgfqpoint{2.504630in}{3.171602in}}{\pgfqpoint{2.504630in}{3.182652in}}%
\pgfpathcurveto{\pgfqpoint{2.504630in}{3.193702in}}{\pgfqpoint{2.500239in}{3.204301in}}{\pgfqpoint{2.492426in}{3.212115in}}%
\pgfpathcurveto{\pgfqpoint{2.484612in}{3.219928in}}{\pgfqpoint{2.474013in}{3.224319in}}{\pgfqpoint{2.462963in}{3.224319in}}%
\pgfpathcurveto{\pgfqpoint{2.451913in}{3.224319in}}{\pgfqpoint{2.441314in}{3.219928in}}{\pgfqpoint{2.433500in}{3.212115in}}%
\pgfpathcurveto{\pgfqpoint{2.425687in}{3.204301in}}{\pgfqpoint{2.421296in}{3.193702in}}{\pgfqpoint{2.421296in}{3.182652in}}%
\pgfpathcurveto{\pgfqpoint{2.421296in}{3.171602in}}{\pgfqpoint{2.425687in}{3.161003in}}{\pgfqpoint{2.433500in}{3.153189in}}%
\pgfpathcurveto{\pgfqpoint{2.441314in}{3.145375in}}{\pgfqpoint{2.451913in}{3.140985in}}{\pgfqpoint{2.462963in}{3.140985in}}%
\pgfpathclose%
\pgfusepath{stroke,fill}%
\end{pgfscope}%
\begin{pgfscope}%
\pgfpathrectangle{\pgfqpoint{0.648703in}{0.548769in}}{\pgfqpoint{5.112893in}{3.102590in}}%
\pgfusepath{clip}%
\pgfsetbuttcap%
\pgfsetroundjoin%
\definecolor{currentfill}{rgb}{1.000000,0.498039,0.054902}%
\pgfsetfillcolor{currentfill}%
\pgfsetlinewidth{1.003750pt}%
\definecolor{currentstroke}{rgb}{1.000000,0.498039,0.054902}%
\pgfsetstrokecolor{currentstroke}%
\pgfsetdash{}{0pt}%
\pgfpathmoveto{\pgfqpoint{3.202448in}{3.145133in}}%
\pgfpathcurveto{\pgfqpoint{3.213498in}{3.145133in}}{\pgfqpoint{3.224097in}{3.149523in}}{\pgfqpoint{3.231911in}{3.157337in}}%
\pgfpathcurveto{\pgfqpoint{3.239724in}{3.165151in}}{\pgfqpoint{3.244115in}{3.175750in}}{\pgfqpoint{3.244115in}{3.186800in}}%
\pgfpathcurveto{\pgfqpoint{3.244115in}{3.197850in}}{\pgfqpoint{3.239724in}{3.208449in}}{\pgfqpoint{3.231911in}{3.216262in}}%
\pgfpathcurveto{\pgfqpoint{3.224097in}{3.224076in}}{\pgfqpoint{3.213498in}{3.228466in}}{\pgfqpoint{3.202448in}{3.228466in}}%
\pgfpathcurveto{\pgfqpoint{3.191398in}{3.228466in}}{\pgfqpoint{3.180799in}{3.224076in}}{\pgfqpoint{3.172985in}{3.216262in}}%
\pgfpathcurveto{\pgfqpoint{3.165171in}{3.208449in}}{\pgfqpoint{3.160781in}{3.197850in}}{\pgfqpoint{3.160781in}{3.186800in}}%
\pgfpathcurveto{\pgfqpoint{3.160781in}{3.175750in}}{\pgfqpoint{3.165171in}{3.165151in}}{\pgfqpoint{3.172985in}{3.157337in}}%
\pgfpathcurveto{\pgfqpoint{3.180799in}{3.149523in}}{\pgfqpoint{3.191398in}{3.145133in}}{\pgfqpoint{3.202448in}{3.145133in}}%
\pgfpathclose%
\pgfusepath{stroke,fill}%
\end{pgfscope}%
\begin{pgfscope}%
\pgfpathrectangle{\pgfqpoint{0.648703in}{0.548769in}}{\pgfqpoint{5.112893in}{3.102590in}}%
\pgfusepath{clip}%
\pgfsetbuttcap%
\pgfsetroundjoin%
\definecolor{currentfill}{rgb}{1.000000,0.498039,0.054902}%
\pgfsetfillcolor{currentfill}%
\pgfsetlinewidth{1.003750pt}%
\definecolor{currentstroke}{rgb}{1.000000,0.498039,0.054902}%
\pgfsetstrokecolor{currentstroke}%
\pgfsetdash{}{0pt}%
\pgfpathmoveto{\pgfqpoint{2.094170in}{3.145133in}}%
\pgfpathcurveto{\pgfqpoint{2.105220in}{3.145133in}}{\pgfqpoint{2.115819in}{3.149523in}}{\pgfqpoint{2.123633in}{3.157337in}}%
\pgfpathcurveto{\pgfqpoint{2.131446in}{3.165151in}}{\pgfqpoint{2.135837in}{3.175750in}}{\pgfqpoint{2.135837in}{3.186800in}}%
\pgfpathcurveto{\pgfqpoint{2.135837in}{3.197850in}}{\pgfqpoint{2.131446in}{3.208449in}}{\pgfqpoint{2.123633in}{3.216262in}}%
\pgfpathcurveto{\pgfqpoint{2.115819in}{3.224076in}}{\pgfqpoint{2.105220in}{3.228466in}}{\pgfqpoint{2.094170in}{3.228466in}}%
\pgfpathcurveto{\pgfqpoint{2.083120in}{3.228466in}}{\pgfqpoint{2.072521in}{3.224076in}}{\pgfqpoint{2.064707in}{3.216262in}}%
\pgfpathcurveto{\pgfqpoint{2.056894in}{3.208449in}}{\pgfqpoint{2.052503in}{3.197850in}}{\pgfqpoint{2.052503in}{3.186800in}}%
\pgfpathcurveto{\pgfqpoint{2.052503in}{3.175750in}}{\pgfqpoint{2.056894in}{3.165151in}}{\pgfqpoint{2.064707in}{3.157337in}}%
\pgfpathcurveto{\pgfqpoint{2.072521in}{3.149523in}}{\pgfqpoint{2.083120in}{3.145133in}}{\pgfqpoint{2.094170in}{3.145133in}}%
\pgfpathclose%
\pgfusepath{stroke,fill}%
\end{pgfscope}%
\begin{pgfscope}%
\pgfpathrectangle{\pgfqpoint{0.648703in}{0.548769in}}{\pgfqpoint{5.112893in}{3.102590in}}%
\pgfusepath{clip}%
\pgfsetbuttcap%
\pgfsetroundjoin%
\definecolor{currentfill}{rgb}{1.000000,0.498039,0.054902}%
\pgfsetfillcolor{currentfill}%
\pgfsetlinewidth{1.003750pt}%
\definecolor{currentstroke}{rgb}{1.000000,0.498039,0.054902}%
\pgfsetstrokecolor{currentstroke}%
\pgfsetdash{}{0pt}%
\pgfpathmoveto{\pgfqpoint{2.055810in}{3.136837in}}%
\pgfpathcurveto{\pgfqpoint{2.066860in}{3.136837in}}{\pgfqpoint{2.077459in}{3.141228in}}{\pgfqpoint{2.085273in}{3.149041in}}%
\pgfpathcurveto{\pgfqpoint{2.093086in}{3.156855in}}{\pgfqpoint{2.097477in}{3.167454in}}{\pgfqpoint{2.097477in}{3.178504in}}%
\pgfpathcurveto{\pgfqpoint{2.097477in}{3.189554in}}{\pgfqpoint{2.093086in}{3.200153in}}{\pgfqpoint{2.085273in}{3.207967in}}%
\pgfpathcurveto{\pgfqpoint{2.077459in}{3.215780in}}{\pgfqpoint{2.066860in}{3.220171in}}{\pgfqpoint{2.055810in}{3.220171in}}%
\pgfpathcurveto{\pgfqpoint{2.044760in}{3.220171in}}{\pgfqpoint{2.034161in}{3.215780in}}{\pgfqpoint{2.026347in}{3.207967in}}%
\pgfpathcurveto{\pgfqpoint{2.018534in}{3.200153in}}{\pgfqpoint{2.014143in}{3.189554in}}{\pgfqpoint{2.014143in}{3.178504in}}%
\pgfpathcurveto{\pgfqpoint{2.014143in}{3.167454in}}{\pgfqpoint{2.018534in}{3.156855in}}{\pgfqpoint{2.026347in}{3.149041in}}%
\pgfpathcurveto{\pgfqpoint{2.034161in}{3.141228in}}{\pgfqpoint{2.044760in}{3.136837in}}{\pgfqpoint{2.055810in}{3.136837in}}%
\pgfpathclose%
\pgfusepath{stroke,fill}%
\end{pgfscope}%
\begin{pgfscope}%
\pgfpathrectangle{\pgfqpoint{0.648703in}{0.548769in}}{\pgfqpoint{5.112893in}{3.102590in}}%
\pgfusepath{clip}%
\pgfsetbuttcap%
\pgfsetroundjoin%
\definecolor{currentfill}{rgb}{1.000000,0.498039,0.054902}%
\pgfsetfillcolor{currentfill}%
\pgfsetlinewidth{1.003750pt}%
\definecolor{currentstroke}{rgb}{1.000000,0.498039,0.054902}%
\pgfsetstrokecolor{currentstroke}%
\pgfsetdash{}{0pt}%
\pgfpathmoveto{\pgfqpoint{3.404674in}{3.468665in}}%
\pgfpathcurveto{\pgfqpoint{3.415724in}{3.468665in}}{\pgfqpoint{3.426323in}{3.473055in}}{\pgfqpoint{3.434137in}{3.480869in}}%
\pgfpathcurveto{\pgfqpoint{3.441950in}{3.488683in}}{\pgfqpoint{3.446340in}{3.499282in}}{\pgfqpoint{3.446340in}{3.510332in}}%
\pgfpathcurveto{\pgfqpoint{3.446340in}{3.521382in}}{\pgfqpoint{3.441950in}{3.531981in}}{\pgfqpoint{3.434137in}{3.539795in}}%
\pgfpathcurveto{\pgfqpoint{3.426323in}{3.547608in}}{\pgfqpoint{3.415724in}{3.551998in}}{\pgfqpoint{3.404674in}{3.551998in}}%
\pgfpathcurveto{\pgfqpoint{3.393624in}{3.551998in}}{\pgfqpoint{3.383025in}{3.547608in}}{\pgfqpoint{3.375211in}{3.539795in}}%
\pgfpathcurveto{\pgfqpoint{3.367397in}{3.531981in}}{\pgfqpoint{3.363007in}{3.521382in}}{\pgfqpoint{3.363007in}{3.510332in}}%
\pgfpathcurveto{\pgfqpoint{3.363007in}{3.499282in}}{\pgfqpoint{3.367397in}{3.488683in}}{\pgfqpoint{3.375211in}{3.480869in}}%
\pgfpathcurveto{\pgfqpoint{3.383025in}{3.473055in}}{\pgfqpoint{3.393624in}{3.468665in}}{\pgfqpoint{3.404674in}{3.468665in}}%
\pgfpathclose%
\pgfusepath{stroke,fill}%
\end{pgfscope}%
\begin{pgfscope}%
\pgfpathrectangle{\pgfqpoint{0.648703in}{0.548769in}}{\pgfqpoint{5.112893in}{3.102590in}}%
\pgfusepath{clip}%
\pgfsetbuttcap%
\pgfsetroundjoin%
\definecolor{currentfill}{rgb}{1.000000,0.498039,0.054902}%
\pgfsetfillcolor{currentfill}%
\pgfsetlinewidth{1.003750pt}%
\definecolor{currentstroke}{rgb}{1.000000,0.498039,0.054902}%
\pgfsetstrokecolor{currentstroke}%
\pgfsetdash{}{0pt}%
\pgfpathmoveto{\pgfqpoint{2.177204in}{3.136837in}}%
\pgfpathcurveto{\pgfqpoint{2.188254in}{3.136837in}}{\pgfqpoint{2.198853in}{3.141228in}}{\pgfqpoint{2.206667in}{3.149041in}}%
\pgfpathcurveto{\pgfqpoint{2.214481in}{3.156855in}}{\pgfqpoint{2.218871in}{3.167454in}}{\pgfqpoint{2.218871in}{3.178504in}}%
\pgfpathcurveto{\pgfqpoint{2.218871in}{3.189554in}}{\pgfqpoint{2.214481in}{3.200153in}}{\pgfqpoint{2.206667in}{3.207967in}}%
\pgfpathcurveto{\pgfqpoint{2.198853in}{3.215780in}}{\pgfqpoint{2.188254in}{3.220171in}}{\pgfqpoint{2.177204in}{3.220171in}}%
\pgfpathcurveto{\pgfqpoint{2.166154in}{3.220171in}}{\pgfqpoint{2.155555in}{3.215780in}}{\pgfqpoint{2.147741in}{3.207967in}}%
\pgfpathcurveto{\pgfqpoint{2.139928in}{3.200153in}}{\pgfqpoint{2.135538in}{3.189554in}}{\pgfqpoint{2.135538in}{3.178504in}}%
\pgfpathcurveto{\pgfqpoint{2.135538in}{3.167454in}}{\pgfqpoint{2.139928in}{3.156855in}}{\pgfqpoint{2.147741in}{3.149041in}}%
\pgfpathcurveto{\pgfqpoint{2.155555in}{3.141228in}}{\pgfqpoint{2.166154in}{3.136837in}}{\pgfqpoint{2.177204in}{3.136837in}}%
\pgfpathclose%
\pgfusepath{stroke,fill}%
\end{pgfscope}%
\begin{pgfscope}%
\pgfpathrectangle{\pgfqpoint{0.648703in}{0.548769in}}{\pgfqpoint{5.112893in}{3.102590in}}%
\pgfusepath{clip}%
\pgfsetbuttcap%
\pgfsetroundjoin%
\definecolor{currentfill}{rgb}{1.000000,0.498039,0.054902}%
\pgfsetfillcolor{currentfill}%
\pgfsetlinewidth{1.003750pt}%
\definecolor{currentstroke}{rgb}{1.000000,0.498039,0.054902}%
\pgfsetstrokecolor{currentstroke}%
\pgfsetdash{}{0pt}%
\pgfpathmoveto{\pgfqpoint{1.981462in}{3.136837in}}%
\pgfpathcurveto{\pgfqpoint{1.992512in}{3.136837in}}{\pgfqpoint{2.003111in}{3.141228in}}{\pgfqpoint{2.010925in}{3.149041in}}%
\pgfpathcurveto{\pgfqpoint{2.018739in}{3.156855in}}{\pgfqpoint{2.023129in}{3.167454in}}{\pgfqpoint{2.023129in}{3.178504in}}%
\pgfpathcurveto{\pgfqpoint{2.023129in}{3.189554in}}{\pgfqpoint{2.018739in}{3.200153in}}{\pgfqpoint{2.010925in}{3.207967in}}%
\pgfpathcurveto{\pgfqpoint{2.003111in}{3.215780in}}{\pgfqpoint{1.992512in}{3.220171in}}{\pgfqpoint{1.981462in}{3.220171in}}%
\pgfpathcurveto{\pgfqpoint{1.970412in}{3.220171in}}{\pgfqpoint{1.959813in}{3.215780in}}{\pgfqpoint{1.952000in}{3.207967in}}%
\pgfpathcurveto{\pgfqpoint{1.944186in}{3.200153in}}{\pgfqpoint{1.939796in}{3.189554in}}{\pgfqpoint{1.939796in}{3.178504in}}%
\pgfpathcurveto{\pgfqpoint{1.939796in}{3.167454in}}{\pgfqpoint{1.944186in}{3.156855in}}{\pgfqpoint{1.952000in}{3.149041in}}%
\pgfpathcurveto{\pgfqpoint{1.959813in}{3.141228in}}{\pgfqpoint{1.970412in}{3.136837in}}{\pgfqpoint{1.981462in}{3.136837in}}%
\pgfpathclose%
\pgfusepath{stroke,fill}%
\end{pgfscope}%
\begin{pgfscope}%
\pgfpathrectangle{\pgfqpoint{0.648703in}{0.548769in}}{\pgfqpoint{5.112893in}{3.102590in}}%
\pgfusepath{clip}%
\pgfsetbuttcap%
\pgfsetroundjoin%
\definecolor{currentfill}{rgb}{1.000000,0.498039,0.054902}%
\pgfsetfillcolor{currentfill}%
\pgfsetlinewidth{1.003750pt}%
\definecolor{currentstroke}{rgb}{1.000000,0.498039,0.054902}%
\pgfsetstrokecolor{currentstroke}%
\pgfsetdash{}{0pt}%
\pgfpathmoveto{\pgfqpoint{2.269066in}{3.136837in}}%
\pgfpathcurveto{\pgfqpoint{2.280116in}{3.136837in}}{\pgfqpoint{2.290716in}{3.141228in}}{\pgfqpoint{2.298529in}{3.149041in}}%
\pgfpathcurveto{\pgfqpoint{2.306343in}{3.156855in}}{\pgfqpoint{2.310733in}{3.167454in}}{\pgfqpoint{2.310733in}{3.178504in}}%
\pgfpathcurveto{\pgfqpoint{2.310733in}{3.189554in}}{\pgfqpoint{2.306343in}{3.200153in}}{\pgfqpoint{2.298529in}{3.207967in}}%
\pgfpathcurveto{\pgfqpoint{2.290716in}{3.215780in}}{\pgfqpoint{2.280116in}{3.220171in}}{\pgfqpoint{2.269066in}{3.220171in}}%
\pgfpathcurveto{\pgfqpoint{2.258016in}{3.220171in}}{\pgfqpoint{2.247417in}{3.215780in}}{\pgfqpoint{2.239604in}{3.207967in}}%
\pgfpathcurveto{\pgfqpoint{2.231790in}{3.200153in}}{\pgfqpoint{2.227400in}{3.189554in}}{\pgfqpoint{2.227400in}{3.178504in}}%
\pgfpathcurveto{\pgfqpoint{2.227400in}{3.167454in}}{\pgfqpoint{2.231790in}{3.156855in}}{\pgfqpoint{2.239604in}{3.149041in}}%
\pgfpathcurveto{\pgfqpoint{2.247417in}{3.141228in}}{\pgfqpoint{2.258016in}{3.136837in}}{\pgfqpoint{2.269066in}{3.136837in}}%
\pgfpathclose%
\pgfusepath{stroke,fill}%
\end{pgfscope}%
\begin{pgfscope}%
\pgfpathrectangle{\pgfqpoint{0.648703in}{0.548769in}}{\pgfqpoint{5.112893in}{3.102590in}}%
\pgfusepath{clip}%
\pgfsetbuttcap%
\pgfsetroundjoin%
\definecolor{currentfill}{rgb}{0.839216,0.152941,0.156863}%
\pgfsetfillcolor{currentfill}%
\pgfsetlinewidth{1.003750pt}%
\definecolor{currentstroke}{rgb}{0.839216,0.152941,0.156863}%
\pgfsetstrokecolor{currentstroke}%
\pgfsetdash{}{0pt}%
\pgfpathmoveto{\pgfqpoint{3.991390in}{3.136837in}}%
\pgfpathcurveto{\pgfqpoint{4.002441in}{3.136837in}}{\pgfqpoint{4.013040in}{3.141228in}}{\pgfqpoint{4.020853in}{3.149041in}}%
\pgfpathcurveto{\pgfqpoint{4.028667in}{3.156855in}}{\pgfqpoint{4.033057in}{3.167454in}}{\pgfqpoint{4.033057in}{3.178504in}}%
\pgfpathcurveto{\pgfqpoint{4.033057in}{3.189554in}}{\pgfqpoint{4.028667in}{3.200153in}}{\pgfqpoint{4.020853in}{3.207967in}}%
\pgfpathcurveto{\pgfqpoint{4.013040in}{3.215780in}}{\pgfqpoint{4.002441in}{3.220171in}}{\pgfqpoint{3.991390in}{3.220171in}}%
\pgfpathcurveto{\pgfqpoint{3.980340in}{3.220171in}}{\pgfqpoint{3.969741in}{3.215780in}}{\pgfqpoint{3.961928in}{3.207967in}}%
\pgfpathcurveto{\pgfqpoint{3.954114in}{3.200153in}}{\pgfqpoint{3.949724in}{3.189554in}}{\pgfqpoint{3.949724in}{3.178504in}}%
\pgfpathcurveto{\pgfqpoint{3.949724in}{3.167454in}}{\pgfqpoint{3.954114in}{3.156855in}}{\pgfqpoint{3.961928in}{3.149041in}}%
\pgfpathcurveto{\pgfqpoint{3.969741in}{3.141228in}}{\pgfqpoint{3.980340in}{3.136837in}}{\pgfqpoint{3.991390in}{3.136837in}}%
\pgfpathclose%
\pgfusepath{stroke,fill}%
\end{pgfscope}%
\begin{pgfscope}%
\pgfpathrectangle{\pgfqpoint{0.648703in}{0.548769in}}{\pgfqpoint{5.112893in}{3.102590in}}%
\pgfusepath{clip}%
\pgfsetbuttcap%
\pgfsetroundjoin%
\definecolor{currentfill}{rgb}{1.000000,0.498039,0.054902}%
\pgfsetfillcolor{currentfill}%
\pgfsetlinewidth{1.003750pt}%
\definecolor{currentstroke}{rgb}{1.000000,0.498039,0.054902}%
\pgfsetstrokecolor{currentstroke}%
\pgfsetdash{}{0pt}%
\pgfpathmoveto{\pgfqpoint{3.585556in}{3.149281in}}%
\pgfpathcurveto{\pgfqpoint{3.596606in}{3.149281in}}{\pgfqpoint{3.607205in}{3.153671in}}{\pgfqpoint{3.615019in}{3.161485in}}%
\pgfpathcurveto{\pgfqpoint{3.622832in}{3.169298in}}{\pgfqpoint{3.627223in}{3.179897in}}{\pgfqpoint{3.627223in}{3.190948in}}%
\pgfpathcurveto{\pgfqpoint{3.627223in}{3.201998in}}{\pgfqpoint{3.622832in}{3.212597in}}{\pgfqpoint{3.615019in}{3.220410in}}%
\pgfpathcurveto{\pgfqpoint{3.607205in}{3.228224in}}{\pgfqpoint{3.596606in}{3.232614in}}{\pgfqpoint{3.585556in}{3.232614in}}%
\pgfpathcurveto{\pgfqpoint{3.574506in}{3.232614in}}{\pgfqpoint{3.563907in}{3.228224in}}{\pgfqpoint{3.556093in}{3.220410in}}%
\pgfpathcurveto{\pgfqpoint{3.548279in}{3.212597in}}{\pgfqpoint{3.543889in}{3.201998in}}{\pgfqpoint{3.543889in}{3.190948in}}%
\pgfpathcurveto{\pgfqpoint{3.543889in}{3.179897in}}{\pgfqpoint{3.548279in}{3.169298in}}{\pgfqpoint{3.556093in}{3.161485in}}%
\pgfpathcurveto{\pgfqpoint{3.563907in}{3.153671in}}{\pgfqpoint{3.574506in}{3.149281in}}{\pgfqpoint{3.585556in}{3.149281in}}%
\pgfpathclose%
\pgfusepath{stroke,fill}%
\end{pgfscope}%
\begin{pgfscope}%
\pgfpathrectangle{\pgfqpoint{0.648703in}{0.548769in}}{\pgfqpoint{5.112893in}{3.102590in}}%
\pgfusepath{clip}%
\pgfsetbuttcap%
\pgfsetroundjoin%
\definecolor{currentfill}{rgb}{1.000000,0.498039,0.054902}%
\pgfsetfillcolor{currentfill}%
\pgfsetlinewidth{1.003750pt}%
\definecolor{currentstroke}{rgb}{1.000000,0.498039,0.054902}%
\pgfsetstrokecolor{currentstroke}%
\pgfsetdash{}{0pt}%
\pgfpathmoveto{\pgfqpoint{3.306916in}{3.286160in}}%
\pgfpathcurveto{\pgfqpoint{3.317966in}{3.286160in}}{\pgfqpoint{3.328565in}{3.290550in}}{\pgfqpoint{3.336379in}{3.298364in}}%
\pgfpathcurveto{\pgfqpoint{3.344193in}{3.306177in}}{\pgfqpoint{3.348583in}{3.316776in}}{\pgfqpoint{3.348583in}{3.327827in}}%
\pgfpathcurveto{\pgfqpoint{3.348583in}{3.338877in}}{\pgfqpoint{3.344193in}{3.349476in}}{\pgfqpoint{3.336379in}{3.357289in}}%
\pgfpathcurveto{\pgfqpoint{3.328565in}{3.365103in}}{\pgfqpoint{3.317966in}{3.369493in}}{\pgfqpoint{3.306916in}{3.369493in}}%
\pgfpathcurveto{\pgfqpoint{3.295866in}{3.369493in}}{\pgfqpoint{3.285267in}{3.365103in}}{\pgfqpoint{3.277453in}{3.357289in}}%
\pgfpathcurveto{\pgfqpoint{3.269640in}{3.349476in}}{\pgfqpoint{3.265250in}{3.338877in}}{\pgfqpoint{3.265250in}{3.327827in}}%
\pgfpathcurveto{\pgfqpoint{3.265250in}{3.316776in}}{\pgfqpoint{3.269640in}{3.306177in}}{\pgfqpoint{3.277453in}{3.298364in}}%
\pgfpathcurveto{\pgfqpoint{3.285267in}{3.290550in}}{\pgfqpoint{3.295866in}{3.286160in}}{\pgfqpoint{3.306916in}{3.286160in}}%
\pgfpathclose%
\pgfusepath{stroke,fill}%
\end{pgfscope}%
\begin{pgfscope}%
\pgfpathrectangle{\pgfqpoint{0.648703in}{0.548769in}}{\pgfqpoint{5.112893in}{3.102590in}}%
\pgfusepath{clip}%
\pgfsetbuttcap%
\pgfsetroundjoin%
\definecolor{currentfill}{rgb}{1.000000,0.498039,0.054902}%
\pgfsetfillcolor{currentfill}%
\pgfsetlinewidth{1.003750pt}%
\definecolor{currentstroke}{rgb}{1.000000,0.498039,0.054902}%
\pgfsetstrokecolor{currentstroke}%
\pgfsetdash{}{0pt}%
\pgfpathmoveto{\pgfqpoint{2.441391in}{3.136837in}}%
\pgfpathcurveto{\pgfqpoint{2.452441in}{3.136837in}}{\pgfqpoint{2.463040in}{3.141228in}}{\pgfqpoint{2.470853in}{3.149041in}}%
\pgfpathcurveto{\pgfqpoint{2.478667in}{3.156855in}}{\pgfqpoint{2.483057in}{3.167454in}}{\pgfqpoint{2.483057in}{3.178504in}}%
\pgfpathcurveto{\pgfqpoint{2.483057in}{3.189554in}}{\pgfqpoint{2.478667in}{3.200153in}}{\pgfqpoint{2.470853in}{3.207967in}}%
\pgfpathcurveto{\pgfqpoint{2.463040in}{3.215780in}}{\pgfqpoint{2.452441in}{3.220171in}}{\pgfqpoint{2.441391in}{3.220171in}}%
\pgfpathcurveto{\pgfqpoint{2.430340in}{3.220171in}}{\pgfqpoint{2.419741in}{3.215780in}}{\pgfqpoint{2.411928in}{3.207967in}}%
\pgfpathcurveto{\pgfqpoint{2.404114in}{3.200153in}}{\pgfqpoint{2.399724in}{3.189554in}}{\pgfqpoint{2.399724in}{3.178504in}}%
\pgfpathcurveto{\pgfqpoint{2.399724in}{3.167454in}}{\pgfqpoint{2.404114in}{3.156855in}}{\pgfqpoint{2.411928in}{3.149041in}}%
\pgfpathcurveto{\pgfqpoint{2.419741in}{3.141228in}}{\pgfqpoint{2.430340in}{3.136837in}}{\pgfqpoint{2.441391in}{3.136837in}}%
\pgfpathclose%
\pgfusepath{stroke,fill}%
\end{pgfscope}%
\begin{pgfscope}%
\pgfpathrectangle{\pgfqpoint{0.648703in}{0.548769in}}{\pgfqpoint{5.112893in}{3.102590in}}%
\pgfusepath{clip}%
\pgfsetbuttcap%
\pgfsetroundjoin%
\definecolor{currentfill}{rgb}{1.000000,0.498039,0.054902}%
\pgfsetfillcolor{currentfill}%
\pgfsetlinewidth{1.003750pt}%
\definecolor{currentstroke}{rgb}{1.000000,0.498039,0.054902}%
\pgfsetstrokecolor{currentstroke}%
\pgfsetdash{}{0pt}%
\pgfpathmoveto{\pgfqpoint{2.958339in}{3.149281in}}%
\pgfpathcurveto{\pgfqpoint{2.969389in}{3.149281in}}{\pgfqpoint{2.979988in}{3.153671in}}{\pgfqpoint{2.987802in}{3.161485in}}%
\pgfpathcurveto{\pgfqpoint{2.995616in}{3.169298in}}{\pgfqpoint{3.000006in}{3.179897in}}{\pgfqpoint{3.000006in}{3.190948in}}%
\pgfpathcurveto{\pgfqpoint{3.000006in}{3.201998in}}{\pgfqpoint{2.995616in}{3.212597in}}{\pgfqpoint{2.987802in}{3.220410in}}%
\pgfpathcurveto{\pgfqpoint{2.979988in}{3.228224in}}{\pgfqpoint{2.969389in}{3.232614in}}{\pgfqpoint{2.958339in}{3.232614in}}%
\pgfpathcurveto{\pgfqpoint{2.947289in}{3.232614in}}{\pgfqpoint{2.936690in}{3.228224in}}{\pgfqpoint{2.928876in}{3.220410in}}%
\pgfpathcurveto{\pgfqpoint{2.921063in}{3.212597in}}{\pgfqpoint{2.916672in}{3.201998in}}{\pgfqpoint{2.916672in}{3.190948in}}%
\pgfpathcurveto{\pgfqpoint{2.916672in}{3.179897in}}{\pgfqpoint{2.921063in}{3.169298in}}{\pgfqpoint{2.928876in}{3.161485in}}%
\pgfpathcurveto{\pgfqpoint{2.936690in}{3.153671in}}{\pgfqpoint{2.947289in}{3.149281in}}{\pgfqpoint{2.958339in}{3.149281in}}%
\pgfpathclose%
\pgfusepath{stroke,fill}%
\end{pgfscope}%
\begin{pgfscope}%
\pgfpathrectangle{\pgfqpoint{0.648703in}{0.548769in}}{\pgfqpoint{5.112893in}{3.102590in}}%
\pgfusepath{clip}%
\pgfsetbuttcap%
\pgfsetroundjoin%
\definecolor{currentfill}{rgb}{1.000000,0.498039,0.054902}%
\pgfsetfillcolor{currentfill}%
\pgfsetlinewidth{1.003750pt}%
\definecolor{currentstroke}{rgb}{1.000000,0.498039,0.054902}%
\pgfsetstrokecolor{currentstroke}%
\pgfsetdash{}{0pt}%
\pgfpathmoveto{\pgfqpoint{2.978048in}{3.128542in}}%
\pgfpathcurveto{\pgfqpoint{2.989098in}{3.128542in}}{\pgfqpoint{2.999697in}{3.132932in}}{\pgfqpoint{3.007511in}{3.140746in}}%
\pgfpathcurveto{\pgfqpoint{3.015324in}{3.148559in}}{\pgfqpoint{3.019715in}{3.159158in}}{\pgfqpoint{3.019715in}{3.170208in}}%
\pgfpathcurveto{\pgfqpoint{3.019715in}{3.181258in}}{\pgfqpoint{3.015324in}{3.191857in}}{\pgfqpoint{3.007511in}{3.199671in}}%
\pgfpathcurveto{\pgfqpoint{2.999697in}{3.207485in}}{\pgfqpoint{2.989098in}{3.211875in}}{\pgfqpoint{2.978048in}{3.211875in}}%
\pgfpathcurveto{\pgfqpoint{2.966998in}{3.211875in}}{\pgfqpoint{2.956399in}{3.207485in}}{\pgfqpoint{2.948585in}{3.199671in}}%
\pgfpathcurveto{\pgfqpoint{2.940771in}{3.191857in}}{\pgfqpoint{2.936381in}{3.181258in}}{\pgfqpoint{2.936381in}{3.170208in}}%
\pgfpathcurveto{\pgfqpoint{2.936381in}{3.159158in}}{\pgfqpoint{2.940771in}{3.148559in}}{\pgfqpoint{2.948585in}{3.140746in}}%
\pgfpathcurveto{\pgfqpoint{2.956399in}{3.132932in}}{\pgfqpoint{2.966998in}{3.128542in}}{\pgfqpoint{2.978048in}{3.128542in}}%
\pgfpathclose%
\pgfusepath{stroke,fill}%
\end{pgfscope}%
\begin{pgfscope}%
\pgfpathrectangle{\pgfqpoint{0.648703in}{0.548769in}}{\pgfqpoint{5.112893in}{3.102590in}}%
\pgfusepath{clip}%
\pgfsetbuttcap%
\pgfsetroundjoin%
\definecolor{currentfill}{rgb}{1.000000,0.498039,0.054902}%
\pgfsetfillcolor{currentfill}%
\pgfsetlinewidth{1.003750pt}%
\definecolor{currentstroke}{rgb}{1.000000,0.498039,0.054902}%
\pgfsetstrokecolor{currentstroke}%
\pgfsetdash{}{0pt}%
\pgfpathmoveto{\pgfqpoint{3.314969in}{3.232238in}}%
\pgfpathcurveto{\pgfqpoint{3.326019in}{3.232238in}}{\pgfqpoint{3.336618in}{3.236628in}}{\pgfqpoint{3.344432in}{3.244442in}}%
\pgfpathcurveto{\pgfqpoint{3.352246in}{3.252255in}}{\pgfqpoint{3.356636in}{3.262854in}}{\pgfqpoint{3.356636in}{3.273905in}}%
\pgfpathcurveto{\pgfqpoint{3.356636in}{3.284955in}}{\pgfqpoint{3.352246in}{3.295554in}}{\pgfqpoint{3.344432in}{3.303367in}}%
\pgfpathcurveto{\pgfqpoint{3.336618in}{3.311181in}}{\pgfqpoint{3.326019in}{3.315571in}}{\pgfqpoint{3.314969in}{3.315571in}}%
\pgfpathcurveto{\pgfqpoint{3.303919in}{3.315571in}}{\pgfqpoint{3.293320in}{3.311181in}}{\pgfqpoint{3.285506in}{3.303367in}}%
\pgfpathcurveto{\pgfqpoint{3.277693in}{3.295554in}}{\pgfqpoint{3.273303in}{3.284955in}}{\pgfqpoint{3.273303in}{3.273905in}}%
\pgfpathcurveto{\pgfqpoint{3.273303in}{3.262854in}}{\pgfqpoint{3.277693in}{3.252255in}}{\pgfqpoint{3.285506in}{3.244442in}}%
\pgfpathcurveto{\pgfqpoint{3.293320in}{3.236628in}}{\pgfqpoint{3.303919in}{3.232238in}}{\pgfqpoint{3.314969in}{3.232238in}}%
\pgfpathclose%
\pgfusepath{stroke,fill}%
\end{pgfscope}%
\begin{pgfscope}%
\pgfpathrectangle{\pgfqpoint{0.648703in}{0.548769in}}{\pgfqpoint{5.112893in}{3.102590in}}%
\pgfusepath{clip}%
\pgfsetbuttcap%
\pgfsetroundjoin%
\definecolor{currentfill}{rgb}{0.121569,0.466667,0.705882}%
\pgfsetfillcolor{currentfill}%
\pgfsetlinewidth{1.003750pt}%
\definecolor{currentstroke}{rgb}{0.121569,0.466667,0.705882}%
\pgfsetstrokecolor{currentstroke}%
\pgfsetdash{}{0pt}%
\pgfpathmoveto{\pgfqpoint{0.814180in}{0.648129in}}%
\pgfpathcurveto{\pgfqpoint{0.825230in}{0.648129in}}{\pgfqpoint{0.835829in}{0.652519in}}{\pgfqpoint{0.843643in}{0.660333in}}%
\pgfpathcurveto{\pgfqpoint{0.851456in}{0.668146in}}{\pgfqpoint{0.855847in}{0.678745in}}{\pgfqpoint{0.855847in}{0.689796in}}%
\pgfpathcurveto{\pgfqpoint{0.855847in}{0.700846in}}{\pgfqpoint{0.851456in}{0.711445in}}{\pgfqpoint{0.843643in}{0.719258in}}%
\pgfpathcurveto{\pgfqpoint{0.835829in}{0.727072in}}{\pgfqpoint{0.825230in}{0.731462in}}{\pgfqpoint{0.814180in}{0.731462in}}%
\pgfpathcurveto{\pgfqpoint{0.803130in}{0.731462in}}{\pgfqpoint{0.792531in}{0.727072in}}{\pgfqpoint{0.784717in}{0.719258in}}%
\pgfpathcurveto{\pgfqpoint{0.776904in}{0.711445in}}{\pgfqpoint{0.772513in}{0.700846in}}{\pgfqpoint{0.772513in}{0.689796in}}%
\pgfpathcurveto{\pgfqpoint{0.772513in}{0.678745in}}{\pgfqpoint{0.776904in}{0.668146in}}{\pgfqpoint{0.784717in}{0.660333in}}%
\pgfpathcurveto{\pgfqpoint{0.792531in}{0.652519in}}{\pgfqpoint{0.803130in}{0.648129in}}{\pgfqpoint{0.814180in}{0.648129in}}%
\pgfpathclose%
\pgfusepath{stroke,fill}%
\end{pgfscope}%
\begin{pgfscope}%
\pgfpathrectangle{\pgfqpoint{0.648703in}{0.548769in}}{\pgfqpoint{5.112893in}{3.102590in}}%
\pgfusepath{clip}%
\pgfsetbuttcap%
\pgfsetroundjoin%
\definecolor{currentfill}{rgb}{1.000000,0.498039,0.054902}%
\pgfsetfillcolor{currentfill}%
\pgfsetlinewidth{1.003750pt}%
\definecolor{currentstroke}{rgb}{1.000000,0.498039,0.054902}%
\pgfsetstrokecolor{currentstroke}%
\pgfsetdash{}{0pt}%
\pgfpathmoveto{\pgfqpoint{1.725111in}{3.136837in}}%
\pgfpathcurveto{\pgfqpoint{1.736161in}{3.136837in}}{\pgfqpoint{1.746760in}{3.141228in}}{\pgfqpoint{1.754574in}{3.149041in}}%
\pgfpathcurveto{\pgfqpoint{1.762387in}{3.156855in}}{\pgfqpoint{1.766777in}{3.167454in}}{\pgfqpoint{1.766777in}{3.178504in}}%
\pgfpathcurveto{\pgfqpoint{1.766777in}{3.189554in}}{\pgfqpoint{1.762387in}{3.200153in}}{\pgfqpoint{1.754574in}{3.207967in}}%
\pgfpathcurveto{\pgfqpoint{1.746760in}{3.215780in}}{\pgfqpoint{1.736161in}{3.220171in}}{\pgfqpoint{1.725111in}{3.220171in}}%
\pgfpathcurveto{\pgfqpoint{1.714061in}{3.220171in}}{\pgfqpoint{1.703462in}{3.215780in}}{\pgfqpoint{1.695648in}{3.207967in}}%
\pgfpathcurveto{\pgfqpoint{1.687834in}{3.200153in}}{\pgfqpoint{1.683444in}{3.189554in}}{\pgfqpoint{1.683444in}{3.178504in}}%
\pgfpathcurveto{\pgfqpoint{1.683444in}{3.167454in}}{\pgfqpoint{1.687834in}{3.156855in}}{\pgfqpoint{1.695648in}{3.149041in}}%
\pgfpathcurveto{\pgfqpoint{1.703462in}{3.141228in}}{\pgfqpoint{1.714061in}{3.136837in}}{\pgfqpoint{1.725111in}{3.136837in}}%
\pgfpathclose%
\pgfusepath{stroke,fill}%
\end{pgfscope}%
\begin{pgfscope}%
\pgfpathrectangle{\pgfqpoint{0.648703in}{0.548769in}}{\pgfqpoint{5.112893in}{3.102590in}}%
\pgfusepath{clip}%
\pgfsetbuttcap%
\pgfsetroundjoin%
\definecolor{currentfill}{rgb}{1.000000,0.498039,0.054902}%
\pgfsetfillcolor{currentfill}%
\pgfsetlinewidth{1.003750pt}%
\definecolor{currentstroke}{rgb}{1.000000,0.498039,0.054902}%
\pgfsetstrokecolor{currentstroke}%
\pgfsetdash{}{0pt}%
\pgfpathmoveto{\pgfqpoint{3.535442in}{3.140985in}}%
\pgfpathcurveto{\pgfqpoint{3.546492in}{3.140985in}}{\pgfqpoint{3.557091in}{3.145375in}}{\pgfqpoint{3.564905in}{3.153189in}}%
\pgfpathcurveto{\pgfqpoint{3.572718in}{3.161003in}}{\pgfqpoint{3.577108in}{3.171602in}}{\pgfqpoint{3.577108in}{3.182652in}}%
\pgfpathcurveto{\pgfqpoint{3.577108in}{3.193702in}}{\pgfqpoint{3.572718in}{3.204301in}}{\pgfqpoint{3.564905in}{3.212115in}}%
\pgfpathcurveto{\pgfqpoint{3.557091in}{3.219928in}}{\pgfqpoint{3.546492in}{3.224319in}}{\pgfqpoint{3.535442in}{3.224319in}}%
\pgfpathcurveto{\pgfqpoint{3.524392in}{3.224319in}}{\pgfqpoint{3.513793in}{3.219928in}}{\pgfqpoint{3.505979in}{3.212115in}}%
\pgfpathcurveto{\pgfqpoint{3.498165in}{3.204301in}}{\pgfqpoint{3.493775in}{3.193702in}}{\pgfqpoint{3.493775in}{3.182652in}}%
\pgfpathcurveto{\pgfqpoint{3.493775in}{3.171602in}}{\pgfqpoint{3.498165in}{3.161003in}}{\pgfqpoint{3.505979in}{3.153189in}}%
\pgfpathcurveto{\pgfqpoint{3.513793in}{3.145375in}}{\pgfqpoint{3.524392in}{3.140985in}}{\pgfqpoint{3.535442in}{3.140985in}}%
\pgfpathclose%
\pgfusepath{stroke,fill}%
\end{pgfscope}%
\begin{pgfscope}%
\pgfpathrectangle{\pgfqpoint{0.648703in}{0.548769in}}{\pgfqpoint{5.112893in}{3.102590in}}%
\pgfusepath{clip}%
\pgfsetbuttcap%
\pgfsetroundjoin%
\definecolor{currentfill}{rgb}{1.000000,0.498039,0.054902}%
\pgfsetfillcolor{currentfill}%
\pgfsetlinewidth{1.003750pt}%
\definecolor{currentstroke}{rgb}{1.000000,0.498039,0.054902}%
\pgfsetstrokecolor{currentstroke}%
\pgfsetdash{}{0pt}%
\pgfpathmoveto{\pgfqpoint{3.065089in}{3.149281in}}%
\pgfpathcurveto{\pgfqpoint{3.076139in}{3.149281in}}{\pgfqpoint{3.086738in}{3.153671in}}{\pgfqpoint{3.094551in}{3.161485in}}%
\pgfpathcurveto{\pgfqpoint{3.102365in}{3.169298in}}{\pgfqpoint{3.106755in}{3.179897in}}{\pgfqpoint{3.106755in}{3.190948in}}%
\pgfpathcurveto{\pgfqpoint{3.106755in}{3.201998in}}{\pgfqpoint{3.102365in}{3.212597in}}{\pgfqpoint{3.094551in}{3.220410in}}%
\pgfpathcurveto{\pgfqpoint{3.086738in}{3.228224in}}{\pgfqpoint{3.076139in}{3.232614in}}{\pgfqpoint{3.065089in}{3.232614in}}%
\pgfpathcurveto{\pgfqpoint{3.054039in}{3.232614in}}{\pgfqpoint{3.043439in}{3.228224in}}{\pgfqpoint{3.035626in}{3.220410in}}%
\pgfpathcurveto{\pgfqpoint{3.027812in}{3.212597in}}{\pgfqpoint{3.023422in}{3.201998in}}{\pgfqpoint{3.023422in}{3.190948in}}%
\pgfpathcurveto{\pgfqpoint{3.023422in}{3.179897in}}{\pgfqpoint{3.027812in}{3.169298in}}{\pgfqpoint{3.035626in}{3.161485in}}%
\pgfpathcurveto{\pgfqpoint{3.043439in}{3.153671in}}{\pgfqpoint{3.054039in}{3.149281in}}{\pgfqpoint{3.065089in}{3.149281in}}%
\pgfpathclose%
\pgfusepath{stroke,fill}%
\end{pgfscope}%
\begin{pgfscope}%
\pgfpathrectangle{\pgfqpoint{0.648703in}{0.548769in}}{\pgfqpoint{5.112893in}{3.102590in}}%
\pgfusepath{clip}%
\pgfsetbuttcap%
\pgfsetroundjoin%
\definecolor{currentfill}{rgb}{1.000000,0.498039,0.054902}%
\pgfsetfillcolor{currentfill}%
\pgfsetlinewidth{1.003750pt}%
\definecolor{currentstroke}{rgb}{1.000000,0.498039,0.054902}%
\pgfsetstrokecolor{currentstroke}%
\pgfsetdash{}{0pt}%
\pgfpathmoveto{\pgfqpoint{1.515383in}{3.140985in}}%
\pgfpathcurveto{\pgfqpoint{1.526434in}{3.140985in}}{\pgfqpoint{1.537033in}{3.145375in}}{\pgfqpoint{1.544846in}{3.153189in}}%
\pgfpathcurveto{\pgfqpoint{1.552660in}{3.161003in}}{\pgfqpoint{1.557050in}{3.171602in}}{\pgfqpoint{1.557050in}{3.182652in}}%
\pgfpathcurveto{\pgfqpoint{1.557050in}{3.193702in}}{\pgfqpoint{1.552660in}{3.204301in}}{\pgfqpoint{1.544846in}{3.212115in}}%
\pgfpathcurveto{\pgfqpoint{1.537033in}{3.219928in}}{\pgfqpoint{1.526434in}{3.224319in}}{\pgfqpoint{1.515383in}{3.224319in}}%
\pgfpathcurveto{\pgfqpoint{1.504333in}{3.224319in}}{\pgfqpoint{1.493734in}{3.219928in}}{\pgfqpoint{1.485921in}{3.212115in}}%
\pgfpathcurveto{\pgfqpoint{1.478107in}{3.204301in}}{\pgfqpoint{1.473717in}{3.193702in}}{\pgfqpoint{1.473717in}{3.182652in}}%
\pgfpathcurveto{\pgfqpoint{1.473717in}{3.171602in}}{\pgfqpoint{1.478107in}{3.161003in}}{\pgfqpoint{1.485921in}{3.153189in}}%
\pgfpathcurveto{\pgfqpoint{1.493734in}{3.145375in}}{\pgfqpoint{1.504333in}{3.140985in}}{\pgfqpoint{1.515383in}{3.140985in}}%
\pgfpathclose%
\pgfusepath{stroke,fill}%
\end{pgfscope}%
\begin{pgfscope}%
\pgfpathrectangle{\pgfqpoint{0.648703in}{0.548769in}}{\pgfqpoint{5.112893in}{3.102590in}}%
\pgfusepath{clip}%
\pgfsetbuttcap%
\pgfsetroundjoin%
\definecolor{currentfill}{rgb}{1.000000,0.498039,0.054902}%
\pgfsetfillcolor{currentfill}%
\pgfsetlinewidth{1.003750pt}%
\definecolor{currentstroke}{rgb}{1.000000,0.498039,0.054902}%
\pgfsetstrokecolor{currentstroke}%
\pgfsetdash{}{0pt}%
\pgfpathmoveto{\pgfqpoint{1.766882in}{3.136837in}}%
\pgfpathcurveto{\pgfqpoint{1.777932in}{3.136837in}}{\pgfqpoint{1.788531in}{3.141228in}}{\pgfqpoint{1.796345in}{3.149041in}}%
\pgfpathcurveto{\pgfqpoint{1.804159in}{3.156855in}}{\pgfqpoint{1.808549in}{3.167454in}}{\pgfqpoint{1.808549in}{3.178504in}}%
\pgfpathcurveto{\pgfqpoint{1.808549in}{3.189554in}}{\pgfqpoint{1.804159in}{3.200153in}}{\pgfqpoint{1.796345in}{3.207967in}}%
\pgfpathcurveto{\pgfqpoint{1.788531in}{3.215780in}}{\pgfqpoint{1.777932in}{3.220171in}}{\pgfqpoint{1.766882in}{3.220171in}}%
\pgfpathcurveto{\pgfqpoint{1.755832in}{3.220171in}}{\pgfqpoint{1.745233in}{3.215780in}}{\pgfqpoint{1.737419in}{3.207967in}}%
\pgfpathcurveto{\pgfqpoint{1.729606in}{3.200153in}}{\pgfqpoint{1.725215in}{3.189554in}}{\pgfqpoint{1.725215in}{3.178504in}}%
\pgfpathcurveto{\pgfqpoint{1.725215in}{3.167454in}}{\pgfqpoint{1.729606in}{3.156855in}}{\pgfqpoint{1.737419in}{3.149041in}}%
\pgfpathcurveto{\pgfqpoint{1.745233in}{3.141228in}}{\pgfqpoint{1.755832in}{3.136837in}}{\pgfqpoint{1.766882in}{3.136837in}}%
\pgfpathclose%
\pgfusepath{stroke,fill}%
\end{pgfscope}%
\begin{pgfscope}%
\pgfpathrectangle{\pgfqpoint{0.648703in}{0.548769in}}{\pgfqpoint{5.112893in}{3.102590in}}%
\pgfusepath{clip}%
\pgfsetbuttcap%
\pgfsetroundjoin%
\definecolor{currentfill}{rgb}{1.000000,0.498039,0.054902}%
\pgfsetfillcolor{currentfill}%
\pgfsetlinewidth{1.003750pt}%
\definecolor{currentstroke}{rgb}{1.000000,0.498039,0.054902}%
\pgfsetstrokecolor{currentstroke}%
\pgfsetdash{}{0pt}%
\pgfpathmoveto{\pgfqpoint{2.215237in}{3.145133in}}%
\pgfpathcurveto{\pgfqpoint{2.226287in}{3.145133in}}{\pgfqpoint{2.236886in}{3.149523in}}{\pgfqpoint{2.244700in}{3.157337in}}%
\pgfpathcurveto{\pgfqpoint{2.252513in}{3.165151in}}{\pgfqpoint{2.256903in}{3.175750in}}{\pgfqpoint{2.256903in}{3.186800in}}%
\pgfpathcurveto{\pgfqpoint{2.256903in}{3.197850in}}{\pgfqpoint{2.252513in}{3.208449in}}{\pgfqpoint{2.244700in}{3.216262in}}%
\pgfpathcurveto{\pgfqpoint{2.236886in}{3.224076in}}{\pgfqpoint{2.226287in}{3.228466in}}{\pgfqpoint{2.215237in}{3.228466in}}%
\pgfpathcurveto{\pgfqpoint{2.204187in}{3.228466in}}{\pgfqpoint{2.193588in}{3.224076in}}{\pgfqpoint{2.185774in}{3.216262in}}%
\pgfpathcurveto{\pgfqpoint{2.177960in}{3.208449in}}{\pgfqpoint{2.173570in}{3.197850in}}{\pgfqpoint{2.173570in}{3.186800in}}%
\pgfpathcurveto{\pgfqpoint{2.173570in}{3.175750in}}{\pgfqpoint{2.177960in}{3.165151in}}{\pgfqpoint{2.185774in}{3.157337in}}%
\pgfpathcurveto{\pgfqpoint{2.193588in}{3.149523in}}{\pgfqpoint{2.204187in}{3.145133in}}{\pgfqpoint{2.215237in}{3.145133in}}%
\pgfpathclose%
\pgfusepath{stroke,fill}%
\end{pgfscope}%
\begin{pgfscope}%
\pgfpathrectangle{\pgfqpoint{0.648703in}{0.548769in}}{\pgfqpoint{5.112893in}{3.102590in}}%
\pgfusepath{clip}%
\pgfsetbuttcap%
\pgfsetroundjoin%
\definecolor{currentfill}{rgb}{1.000000,0.498039,0.054902}%
\pgfsetfillcolor{currentfill}%
\pgfsetlinewidth{1.003750pt}%
\definecolor{currentstroke}{rgb}{1.000000,0.498039,0.054902}%
\pgfsetstrokecolor{currentstroke}%
\pgfsetdash{}{0pt}%
\pgfpathmoveto{\pgfqpoint{3.268214in}{3.132690in}}%
\pgfpathcurveto{\pgfqpoint{3.279264in}{3.132690in}}{\pgfqpoint{3.289863in}{3.137080in}}{\pgfqpoint{3.297676in}{3.144893in}}%
\pgfpathcurveto{\pgfqpoint{3.305490in}{3.152707in}}{\pgfqpoint{3.309880in}{3.163306in}}{\pgfqpoint{3.309880in}{3.174356in}}%
\pgfpathcurveto{\pgfqpoint{3.309880in}{3.185406in}}{\pgfqpoint{3.305490in}{3.196005in}}{\pgfqpoint{3.297676in}{3.203819in}}%
\pgfpathcurveto{\pgfqpoint{3.289863in}{3.211633in}}{\pgfqpoint{3.279264in}{3.216023in}}{\pgfqpoint{3.268214in}{3.216023in}}%
\pgfpathcurveto{\pgfqpoint{3.257163in}{3.216023in}}{\pgfqpoint{3.246564in}{3.211633in}}{\pgfqpoint{3.238751in}{3.203819in}}%
\pgfpathcurveto{\pgfqpoint{3.230937in}{3.196005in}}{\pgfqpoint{3.226547in}{3.185406in}}{\pgfqpoint{3.226547in}{3.174356in}}%
\pgfpathcurveto{\pgfqpoint{3.226547in}{3.163306in}}{\pgfqpoint{3.230937in}{3.152707in}}{\pgfqpoint{3.238751in}{3.144893in}}%
\pgfpathcurveto{\pgfqpoint{3.246564in}{3.137080in}}{\pgfqpoint{3.257163in}{3.132690in}}{\pgfqpoint{3.268214in}{3.132690in}}%
\pgfpathclose%
\pgfusepath{stroke,fill}%
\end{pgfscope}%
\begin{pgfscope}%
\pgfpathrectangle{\pgfqpoint{0.648703in}{0.548769in}}{\pgfqpoint{5.112893in}{3.102590in}}%
\pgfusepath{clip}%
\pgfsetbuttcap%
\pgfsetroundjoin%
\definecolor{currentfill}{rgb}{1.000000,0.498039,0.054902}%
\pgfsetfillcolor{currentfill}%
\pgfsetlinewidth{1.003750pt}%
\definecolor{currentstroke}{rgb}{1.000000,0.498039,0.054902}%
\pgfsetstrokecolor{currentstroke}%
\pgfsetdash{}{0pt}%
\pgfpathmoveto{\pgfqpoint{3.333013in}{3.348378in}}%
\pgfpathcurveto{\pgfqpoint{3.344063in}{3.348378in}}{\pgfqpoint{3.354662in}{3.352768in}}{\pgfqpoint{3.362476in}{3.360581in}}%
\pgfpathcurveto{\pgfqpoint{3.370289in}{3.368395in}}{\pgfqpoint{3.374680in}{3.378994in}}{\pgfqpoint{3.374680in}{3.390044in}}%
\pgfpathcurveto{\pgfqpoint{3.374680in}{3.401094in}}{\pgfqpoint{3.370289in}{3.411693in}}{\pgfqpoint{3.362476in}{3.419507in}}%
\pgfpathcurveto{\pgfqpoint{3.354662in}{3.427321in}}{\pgfqpoint{3.344063in}{3.431711in}}{\pgfqpoint{3.333013in}{3.431711in}}%
\pgfpathcurveto{\pgfqpoint{3.321963in}{3.431711in}}{\pgfqpoint{3.311364in}{3.427321in}}{\pgfqpoint{3.303550in}{3.419507in}}%
\pgfpathcurveto{\pgfqpoint{3.295737in}{3.411693in}}{\pgfqpoint{3.291346in}{3.401094in}}{\pgfqpoint{3.291346in}{3.390044in}}%
\pgfpathcurveto{\pgfqpoint{3.291346in}{3.378994in}}{\pgfqpoint{3.295737in}{3.368395in}}{\pgfqpoint{3.303550in}{3.360581in}}%
\pgfpathcurveto{\pgfqpoint{3.311364in}{3.352768in}}{\pgfqpoint{3.321963in}{3.348378in}}{\pgfqpoint{3.333013in}{3.348378in}}%
\pgfpathclose%
\pgfusepath{stroke,fill}%
\end{pgfscope}%
\begin{pgfscope}%
\pgfsetbuttcap%
\pgfsetroundjoin%
\definecolor{currentfill}{rgb}{0.000000,0.000000,0.000000}%
\pgfsetfillcolor{currentfill}%
\pgfsetlinewidth{0.803000pt}%
\definecolor{currentstroke}{rgb}{0.000000,0.000000,0.000000}%
\pgfsetstrokecolor{currentstroke}%
\pgfsetdash{}{0pt}%
\pgfsys@defobject{currentmarker}{\pgfqpoint{0.000000in}{-0.048611in}}{\pgfqpoint{0.000000in}{0.000000in}}{%
\pgfpathmoveto{\pgfqpoint{0.000000in}{0.000000in}}%
\pgfpathlineto{\pgfqpoint{0.000000in}{-0.048611in}}%
\pgfusepath{stroke,fill}%
}%
\begin{pgfscope}%
\pgfsys@transformshift{0.814175in}{0.548769in}%
\pgfsys@useobject{currentmarker}{}%
\end{pgfscope}%
\end{pgfscope}%
\begin{pgfscope}%
\definecolor{textcolor}{rgb}{0.000000,0.000000,0.000000}%
\pgfsetstrokecolor{textcolor}%
\pgfsetfillcolor{textcolor}%
\pgftext[x=0.814175in,y=0.451547in,,top]{\color{textcolor}\sffamily\fontsize{10.000000}{12.000000}\selectfont \(\displaystyle {0.0}\)}%
\end{pgfscope}%
\begin{pgfscope}%
\pgfsetbuttcap%
\pgfsetroundjoin%
\definecolor{currentfill}{rgb}{0.000000,0.000000,0.000000}%
\pgfsetfillcolor{currentfill}%
\pgfsetlinewidth{0.803000pt}%
\definecolor{currentstroke}{rgb}{0.000000,0.000000,0.000000}%
\pgfsetstrokecolor{currentstroke}%
\pgfsetdash{}{0pt}%
\pgfsys@defobject{currentmarker}{\pgfqpoint{0.000000in}{-0.048611in}}{\pgfqpoint{0.000000in}{0.000000in}}{%
\pgfpathmoveto{\pgfqpoint{0.000000in}{0.000000in}}%
\pgfpathlineto{\pgfqpoint{0.000000in}{-0.048611in}}%
\pgfusepath{stroke,fill}%
}%
\begin{pgfscope}%
\pgfsys@transformshift{1.308917in}{0.548769in}%
\pgfsys@useobject{currentmarker}{}%
\end{pgfscope}%
\end{pgfscope}%
\begin{pgfscope}%
\definecolor{textcolor}{rgb}{0.000000,0.000000,0.000000}%
\pgfsetstrokecolor{textcolor}%
\pgfsetfillcolor{textcolor}%
\pgftext[x=1.308917in,y=0.451547in,,top]{\color{textcolor}\sffamily\fontsize{10.000000}{12.000000}\selectfont \(\displaystyle {0.1}\)}%
\end{pgfscope}%
\begin{pgfscope}%
\pgfsetbuttcap%
\pgfsetroundjoin%
\definecolor{currentfill}{rgb}{0.000000,0.000000,0.000000}%
\pgfsetfillcolor{currentfill}%
\pgfsetlinewidth{0.803000pt}%
\definecolor{currentstroke}{rgb}{0.000000,0.000000,0.000000}%
\pgfsetstrokecolor{currentstroke}%
\pgfsetdash{}{0pt}%
\pgfsys@defobject{currentmarker}{\pgfqpoint{0.000000in}{-0.048611in}}{\pgfqpoint{0.000000in}{0.000000in}}{%
\pgfpathmoveto{\pgfqpoint{0.000000in}{0.000000in}}%
\pgfpathlineto{\pgfqpoint{0.000000in}{-0.048611in}}%
\pgfusepath{stroke,fill}%
}%
\begin{pgfscope}%
\pgfsys@transformshift{1.803659in}{0.548769in}%
\pgfsys@useobject{currentmarker}{}%
\end{pgfscope}%
\end{pgfscope}%
\begin{pgfscope}%
\definecolor{textcolor}{rgb}{0.000000,0.000000,0.000000}%
\pgfsetstrokecolor{textcolor}%
\pgfsetfillcolor{textcolor}%
\pgftext[x=1.803659in,y=0.451547in,,top]{\color{textcolor}\sffamily\fontsize{10.000000}{12.000000}\selectfont \(\displaystyle {0.2}\)}%
\end{pgfscope}%
\begin{pgfscope}%
\pgfsetbuttcap%
\pgfsetroundjoin%
\definecolor{currentfill}{rgb}{0.000000,0.000000,0.000000}%
\pgfsetfillcolor{currentfill}%
\pgfsetlinewidth{0.803000pt}%
\definecolor{currentstroke}{rgb}{0.000000,0.000000,0.000000}%
\pgfsetstrokecolor{currentstroke}%
\pgfsetdash{}{0pt}%
\pgfsys@defobject{currentmarker}{\pgfqpoint{0.000000in}{-0.048611in}}{\pgfqpoint{0.000000in}{0.000000in}}{%
\pgfpathmoveto{\pgfqpoint{0.000000in}{0.000000in}}%
\pgfpathlineto{\pgfqpoint{0.000000in}{-0.048611in}}%
\pgfusepath{stroke,fill}%
}%
\begin{pgfscope}%
\pgfsys@transformshift{2.298401in}{0.548769in}%
\pgfsys@useobject{currentmarker}{}%
\end{pgfscope}%
\end{pgfscope}%
\begin{pgfscope}%
\definecolor{textcolor}{rgb}{0.000000,0.000000,0.000000}%
\pgfsetstrokecolor{textcolor}%
\pgfsetfillcolor{textcolor}%
\pgftext[x=2.298401in,y=0.451547in,,top]{\color{textcolor}\sffamily\fontsize{10.000000}{12.000000}\selectfont \(\displaystyle {0.3}\)}%
\end{pgfscope}%
\begin{pgfscope}%
\pgfsetbuttcap%
\pgfsetroundjoin%
\definecolor{currentfill}{rgb}{0.000000,0.000000,0.000000}%
\pgfsetfillcolor{currentfill}%
\pgfsetlinewidth{0.803000pt}%
\definecolor{currentstroke}{rgb}{0.000000,0.000000,0.000000}%
\pgfsetstrokecolor{currentstroke}%
\pgfsetdash{}{0pt}%
\pgfsys@defobject{currentmarker}{\pgfqpoint{0.000000in}{-0.048611in}}{\pgfqpoint{0.000000in}{0.000000in}}{%
\pgfpathmoveto{\pgfqpoint{0.000000in}{0.000000in}}%
\pgfpathlineto{\pgfqpoint{0.000000in}{-0.048611in}}%
\pgfusepath{stroke,fill}%
}%
\begin{pgfscope}%
\pgfsys@transformshift{2.793144in}{0.548769in}%
\pgfsys@useobject{currentmarker}{}%
\end{pgfscope}%
\end{pgfscope}%
\begin{pgfscope}%
\definecolor{textcolor}{rgb}{0.000000,0.000000,0.000000}%
\pgfsetstrokecolor{textcolor}%
\pgfsetfillcolor{textcolor}%
\pgftext[x=2.793144in,y=0.451547in,,top]{\color{textcolor}\sffamily\fontsize{10.000000}{12.000000}\selectfont \(\displaystyle {0.4}\)}%
\end{pgfscope}%
\begin{pgfscope}%
\pgfsetbuttcap%
\pgfsetroundjoin%
\definecolor{currentfill}{rgb}{0.000000,0.000000,0.000000}%
\pgfsetfillcolor{currentfill}%
\pgfsetlinewidth{0.803000pt}%
\definecolor{currentstroke}{rgb}{0.000000,0.000000,0.000000}%
\pgfsetstrokecolor{currentstroke}%
\pgfsetdash{}{0pt}%
\pgfsys@defobject{currentmarker}{\pgfqpoint{0.000000in}{-0.048611in}}{\pgfqpoint{0.000000in}{0.000000in}}{%
\pgfpathmoveto{\pgfqpoint{0.000000in}{0.000000in}}%
\pgfpathlineto{\pgfqpoint{0.000000in}{-0.048611in}}%
\pgfusepath{stroke,fill}%
}%
\begin{pgfscope}%
\pgfsys@transformshift{3.287886in}{0.548769in}%
\pgfsys@useobject{currentmarker}{}%
\end{pgfscope}%
\end{pgfscope}%
\begin{pgfscope}%
\definecolor{textcolor}{rgb}{0.000000,0.000000,0.000000}%
\pgfsetstrokecolor{textcolor}%
\pgfsetfillcolor{textcolor}%
\pgftext[x=3.287886in,y=0.451547in,,top]{\color{textcolor}\sffamily\fontsize{10.000000}{12.000000}\selectfont \(\displaystyle {0.5}\)}%
\end{pgfscope}%
\begin{pgfscope}%
\pgfsetbuttcap%
\pgfsetroundjoin%
\definecolor{currentfill}{rgb}{0.000000,0.000000,0.000000}%
\pgfsetfillcolor{currentfill}%
\pgfsetlinewidth{0.803000pt}%
\definecolor{currentstroke}{rgb}{0.000000,0.000000,0.000000}%
\pgfsetstrokecolor{currentstroke}%
\pgfsetdash{}{0pt}%
\pgfsys@defobject{currentmarker}{\pgfqpoint{0.000000in}{-0.048611in}}{\pgfqpoint{0.000000in}{0.000000in}}{%
\pgfpathmoveto{\pgfqpoint{0.000000in}{0.000000in}}%
\pgfpathlineto{\pgfqpoint{0.000000in}{-0.048611in}}%
\pgfusepath{stroke,fill}%
}%
\begin{pgfscope}%
\pgfsys@transformshift{3.782628in}{0.548769in}%
\pgfsys@useobject{currentmarker}{}%
\end{pgfscope}%
\end{pgfscope}%
\begin{pgfscope}%
\definecolor{textcolor}{rgb}{0.000000,0.000000,0.000000}%
\pgfsetstrokecolor{textcolor}%
\pgfsetfillcolor{textcolor}%
\pgftext[x=3.782628in,y=0.451547in,,top]{\color{textcolor}\sffamily\fontsize{10.000000}{12.000000}\selectfont \(\displaystyle {0.6}\)}%
\end{pgfscope}%
\begin{pgfscope}%
\pgfsetbuttcap%
\pgfsetroundjoin%
\definecolor{currentfill}{rgb}{0.000000,0.000000,0.000000}%
\pgfsetfillcolor{currentfill}%
\pgfsetlinewidth{0.803000pt}%
\definecolor{currentstroke}{rgb}{0.000000,0.000000,0.000000}%
\pgfsetstrokecolor{currentstroke}%
\pgfsetdash{}{0pt}%
\pgfsys@defobject{currentmarker}{\pgfqpoint{0.000000in}{-0.048611in}}{\pgfqpoint{0.000000in}{0.000000in}}{%
\pgfpathmoveto{\pgfqpoint{0.000000in}{0.000000in}}%
\pgfpathlineto{\pgfqpoint{0.000000in}{-0.048611in}}%
\pgfusepath{stroke,fill}%
}%
\begin{pgfscope}%
\pgfsys@transformshift{4.277370in}{0.548769in}%
\pgfsys@useobject{currentmarker}{}%
\end{pgfscope}%
\end{pgfscope}%
\begin{pgfscope}%
\definecolor{textcolor}{rgb}{0.000000,0.000000,0.000000}%
\pgfsetstrokecolor{textcolor}%
\pgfsetfillcolor{textcolor}%
\pgftext[x=4.277370in,y=0.451547in,,top]{\color{textcolor}\sffamily\fontsize{10.000000}{12.000000}\selectfont \(\displaystyle {0.7}\)}%
\end{pgfscope}%
\begin{pgfscope}%
\pgfsetbuttcap%
\pgfsetroundjoin%
\definecolor{currentfill}{rgb}{0.000000,0.000000,0.000000}%
\pgfsetfillcolor{currentfill}%
\pgfsetlinewidth{0.803000pt}%
\definecolor{currentstroke}{rgb}{0.000000,0.000000,0.000000}%
\pgfsetstrokecolor{currentstroke}%
\pgfsetdash{}{0pt}%
\pgfsys@defobject{currentmarker}{\pgfqpoint{0.000000in}{-0.048611in}}{\pgfqpoint{0.000000in}{0.000000in}}{%
\pgfpathmoveto{\pgfqpoint{0.000000in}{0.000000in}}%
\pgfpathlineto{\pgfqpoint{0.000000in}{-0.048611in}}%
\pgfusepath{stroke,fill}%
}%
\begin{pgfscope}%
\pgfsys@transformshift{4.772112in}{0.548769in}%
\pgfsys@useobject{currentmarker}{}%
\end{pgfscope}%
\end{pgfscope}%
\begin{pgfscope}%
\definecolor{textcolor}{rgb}{0.000000,0.000000,0.000000}%
\pgfsetstrokecolor{textcolor}%
\pgfsetfillcolor{textcolor}%
\pgftext[x=4.772112in,y=0.451547in,,top]{\color{textcolor}\sffamily\fontsize{10.000000}{12.000000}\selectfont \(\displaystyle {0.8}\)}%
\end{pgfscope}%
\begin{pgfscope}%
\pgfsetbuttcap%
\pgfsetroundjoin%
\definecolor{currentfill}{rgb}{0.000000,0.000000,0.000000}%
\pgfsetfillcolor{currentfill}%
\pgfsetlinewidth{0.803000pt}%
\definecolor{currentstroke}{rgb}{0.000000,0.000000,0.000000}%
\pgfsetstrokecolor{currentstroke}%
\pgfsetdash{}{0pt}%
\pgfsys@defobject{currentmarker}{\pgfqpoint{0.000000in}{-0.048611in}}{\pgfqpoint{0.000000in}{0.000000in}}{%
\pgfpathmoveto{\pgfqpoint{0.000000in}{0.000000in}}%
\pgfpathlineto{\pgfqpoint{0.000000in}{-0.048611in}}%
\pgfusepath{stroke,fill}%
}%
\begin{pgfscope}%
\pgfsys@transformshift{5.266854in}{0.548769in}%
\pgfsys@useobject{currentmarker}{}%
\end{pgfscope}%
\end{pgfscope}%
\begin{pgfscope}%
\definecolor{textcolor}{rgb}{0.000000,0.000000,0.000000}%
\pgfsetstrokecolor{textcolor}%
\pgfsetfillcolor{textcolor}%
\pgftext[x=5.266854in,y=0.451547in,,top]{\color{textcolor}\sffamily\fontsize{10.000000}{12.000000}\selectfont \(\displaystyle {0.9}\)}%
\end{pgfscope}%
\begin{pgfscope}%
\pgfsetbuttcap%
\pgfsetroundjoin%
\definecolor{currentfill}{rgb}{0.000000,0.000000,0.000000}%
\pgfsetfillcolor{currentfill}%
\pgfsetlinewidth{0.803000pt}%
\definecolor{currentstroke}{rgb}{0.000000,0.000000,0.000000}%
\pgfsetstrokecolor{currentstroke}%
\pgfsetdash{}{0pt}%
\pgfsys@defobject{currentmarker}{\pgfqpoint{0.000000in}{-0.048611in}}{\pgfqpoint{0.000000in}{0.000000in}}{%
\pgfpathmoveto{\pgfqpoint{0.000000in}{0.000000in}}%
\pgfpathlineto{\pgfqpoint{0.000000in}{-0.048611in}}%
\pgfusepath{stroke,fill}%
}%
\begin{pgfscope}%
\pgfsys@transformshift{5.761597in}{0.548769in}%
\pgfsys@useobject{currentmarker}{}%
\end{pgfscope}%
\end{pgfscope}%
\begin{pgfscope}%
\definecolor{textcolor}{rgb}{0.000000,0.000000,0.000000}%
\pgfsetstrokecolor{textcolor}%
\pgfsetfillcolor{textcolor}%
\pgftext[x=5.761597in,y=0.451547in,,top]{\color{textcolor}\sffamily\fontsize{10.000000}{12.000000}\selectfont \(\displaystyle {1.0}\)}%
\end{pgfscope}%
\begin{pgfscope}%
\definecolor{textcolor}{rgb}{0.000000,0.000000,0.000000}%
\pgfsetstrokecolor{textcolor}%
\pgfsetfillcolor{textcolor}%
\pgftext[x=3.205150in,y=0.272658in,,top]{\color{textcolor}\sffamily\fontsize{10.000000}{12.000000}\selectfont Infoflow Edge Count}%
\end{pgfscope}%
\begin{pgfscope}%
\definecolor{textcolor}{rgb}{0.000000,0.000000,0.000000}%
\pgfsetstrokecolor{textcolor}%
\pgfsetfillcolor{textcolor}%
\pgftext[x=5.761597in,y=0.286547in,right,top]{\color{textcolor}\sffamily\fontsize{10.000000}{12.000000}\selectfont \(\displaystyle \times{10^{8}}{}\)}%
\end{pgfscope}%
\begin{pgfscope}%
\pgfsetbuttcap%
\pgfsetroundjoin%
\definecolor{currentfill}{rgb}{0.000000,0.000000,0.000000}%
\pgfsetfillcolor{currentfill}%
\pgfsetlinewidth{0.803000pt}%
\definecolor{currentstroke}{rgb}{0.000000,0.000000,0.000000}%
\pgfsetstrokecolor{currentstroke}%
\pgfsetdash{}{0pt}%
\pgfsys@defobject{currentmarker}{\pgfqpoint{-0.048611in}{0.000000in}}{\pgfqpoint{0.000000in}{0.000000in}}{%
\pgfpathmoveto{\pgfqpoint{0.000000in}{0.000000in}}%
\pgfpathlineto{\pgfqpoint{-0.048611in}{0.000000in}}%
\pgfusepath{stroke,fill}%
}%
\begin{pgfscope}%
\pgfsys@transformshift{0.648703in}{0.689796in}%
\pgfsys@useobject{currentmarker}{}%
\end{pgfscope}%
\end{pgfscope}%
\begin{pgfscope}%
\definecolor{textcolor}{rgb}{0.000000,0.000000,0.000000}%
\pgfsetstrokecolor{textcolor}%
\pgfsetfillcolor{textcolor}%
\pgftext[x=0.482036in, y=0.641601in, left, base]{\color{textcolor}\sffamily\fontsize{10.000000}{12.000000}\selectfont \(\displaystyle {0}\)}%
\end{pgfscope}%
\begin{pgfscope}%
\pgfsetbuttcap%
\pgfsetroundjoin%
\definecolor{currentfill}{rgb}{0.000000,0.000000,0.000000}%
\pgfsetfillcolor{currentfill}%
\pgfsetlinewidth{0.803000pt}%
\definecolor{currentstroke}{rgb}{0.000000,0.000000,0.000000}%
\pgfsetstrokecolor{currentstroke}%
\pgfsetdash{}{0pt}%
\pgfsys@defobject{currentmarker}{\pgfqpoint{-0.048611in}{0.000000in}}{\pgfqpoint{0.000000in}{0.000000in}}{%
\pgfpathmoveto{\pgfqpoint{0.000000in}{0.000000in}}%
\pgfpathlineto{\pgfqpoint{-0.048611in}{0.000000in}}%
\pgfusepath{stroke,fill}%
}%
\begin{pgfscope}%
\pgfsys@transformshift{0.648703in}{1.104580in}%
\pgfsys@useobject{currentmarker}{}%
\end{pgfscope}%
\end{pgfscope}%
\begin{pgfscope}%
\definecolor{textcolor}{rgb}{0.000000,0.000000,0.000000}%
\pgfsetstrokecolor{textcolor}%
\pgfsetfillcolor{textcolor}%
\pgftext[x=0.343147in, y=1.056386in, left, base]{\color{textcolor}\sffamily\fontsize{10.000000}{12.000000}\selectfont \(\displaystyle {100}\)}%
\end{pgfscope}%
\begin{pgfscope}%
\pgfsetbuttcap%
\pgfsetroundjoin%
\definecolor{currentfill}{rgb}{0.000000,0.000000,0.000000}%
\pgfsetfillcolor{currentfill}%
\pgfsetlinewidth{0.803000pt}%
\definecolor{currentstroke}{rgb}{0.000000,0.000000,0.000000}%
\pgfsetstrokecolor{currentstroke}%
\pgfsetdash{}{0pt}%
\pgfsys@defobject{currentmarker}{\pgfqpoint{-0.048611in}{0.000000in}}{\pgfqpoint{0.000000in}{0.000000in}}{%
\pgfpathmoveto{\pgfqpoint{0.000000in}{0.000000in}}%
\pgfpathlineto{\pgfqpoint{-0.048611in}{0.000000in}}%
\pgfusepath{stroke,fill}%
}%
\begin{pgfscope}%
\pgfsys@transformshift{0.648703in}{1.519365in}%
\pgfsys@useobject{currentmarker}{}%
\end{pgfscope}%
\end{pgfscope}%
\begin{pgfscope}%
\definecolor{textcolor}{rgb}{0.000000,0.000000,0.000000}%
\pgfsetstrokecolor{textcolor}%
\pgfsetfillcolor{textcolor}%
\pgftext[x=0.343147in, y=1.471171in, left, base]{\color{textcolor}\sffamily\fontsize{10.000000}{12.000000}\selectfont \(\displaystyle {200}\)}%
\end{pgfscope}%
\begin{pgfscope}%
\pgfsetbuttcap%
\pgfsetroundjoin%
\definecolor{currentfill}{rgb}{0.000000,0.000000,0.000000}%
\pgfsetfillcolor{currentfill}%
\pgfsetlinewidth{0.803000pt}%
\definecolor{currentstroke}{rgb}{0.000000,0.000000,0.000000}%
\pgfsetstrokecolor{currentstroke}%
\pgfsetdash{}{0pt}%
\pgfsys@defobject{currentmarker}{\pgfqpoint{-0.048611in}{0.000000in}}{\pgfqpoint{0.000000in}{0.000000in}}{%
\pgfpathmoveto{\pgfqpoint{0.000000in}{0.000000in}}%
\pgfpathlineto{\pgfqpoint{-0.048611in}{0.000000in}}%
\pgfusepath{stroke,fill}%
}%
\begin{pgfscope}%
\pgfsys@transformshift{0.648703in}{1.934150in}%
\pgfsys@useobject{currentmarker}{}%
\end{pgfscope}%
\end{pgfscope}%
\begin{pgfscope}%
\definecolor{textcolor}{rgb}{0.000000,0.000000,0.000000}%
\pgfsetstrokecolor{textcolor}%
\pgfsetfillcolor{textcolor}%
\pgftext[x=0.343147in, y=1.885955in, left, base]{\color{textcolor}\sffamily\fontsize{10.000000}{12.000000}\selectfont \(\displaystyle {300}\)}%
\end{pgfscope}%
\begin{pgfscope}%
\pgfsetbuttcap%
\pgfsetroundjoin%
\definecolor{currentfill}{rgb}{0.000000,0.000000,0.000000}%
\pgfsetfillcolor{currentfill}%
\pgfsetlinewidth{0.803000pt}%
\definecolor{currentstroke}{rgb}{0.000000,0.000000,0.000000}%
\pgfsetstrokecolor{currentstroke}%
\pgfsetdash{}{0pt}%
\pgfsys@defobject{currentmarker}{\pgfqpoint{-0.048611in}{0.000000in}}{\pgfqpoint{0.000000in}{0.000000in}}{%
\pgfpathmoveto{\pgfqpoint{0.000000in}{0.000000in}}%
\pgfpathlineto{\pgfqpoint{-0.048611in}{0.000000in}}%
\pgfusepath{stroke,fill}%
}%
\begin{pgfscope}%
\pgfsys@transformshift{0.648703in}{2.348935in}%
\pgfsys@useobject{currentmarker}{}%
\end{pgfscope}%
\end{pgfscope}%
\begin{pgfscope}%
\definecolor{textcolor}{rgb}{0.000000,0.000000,0.000000}%
\pgfsetstrokecolor{textcolor}%
\pgfsetfillcolor{textcolor}%
\pgftext[x=0.343147in, y=2.300740in, left, base]{\color{textcolor}\sffamily\fontsize{10.000000}{12.000000}\selectfont \(\displaystyle {400}\)}%
\end{pgfscope}%
\begin{pgfscope}%
\pgfsetbuttcap%
\pgfsetroundjoin%
\definecolor{currentfill}{rgb}{0.000000,0.000000,0.000000}%
\pgfsetfillcolor{currentfill}%
\pgfsetlinewidth{0.803000pt}%
\definecolor{currentstroke}{rgb}{0.000000,0.000000,0.000000}%
\pgfsetstrokecolor{currentstroke}%
\pgfsetdash{}{0pt}%
\pgfsys@defobject{currentmarker}{\pgfqpoint{-0.048611in}{0.000000in}}{\pgfqpoint{0.000000in}{0.000000in}}{%
\pgfpathmoveto{\pgfqpoint{0.000000in}{0.000000in}}%
\pgfpathlineto{\pgfqpoint{-0.048611in}{0.000000in}}%
\pgfusepath{stroke,fill}%
}%
\begin{pgfscope}%
\pgfsys@transformshift{0.648703in}{2.763719in}%
\pgfsys@useobject{currentmarker}{}%
\end{pgfscope}%
\end{pgfscope}%
\begin{pgfscope}%
\definecolor{textcolor}{rgb}{0.000000,0.000000,0.000000}%
\pgfsetstrokecolor{textcolor}%
\pgfsetfillcolor{textcolor}%
\pgftext[x=0.343147in, y=2.715525in, left, base]{\color{textcolor}\sffamily\fontsize{10.000000}{12.000000}\selectfont \(\displaystyle {500}\)}%
\end{pgfscope}%
\begin{pgfscope}%
\pgfsetbuttcap%
\pgfsetroundjoin%
\definecolor{currentfill}{rgb}{0.000000,0.000000,0.000000}%
\pgfsetfillcolor{currentfill}%
\pgfsetlinewidth{0.803000pt}%
\definecolor{currentstroke}{rgb}{0.000000,0.000000,0.000000}%
\pgfsetstrokecolor{currentstroke}%
\pgfsetdash{}{0pt}%
\pgfsys@defobject{currentmarker}{\pgfqpoint{-0.048611in}{0.000000in}}{\pgfqpoint{0.000000in}{0.000000in}}{%
\pgfpathmoveto{\pgfqpoint{0.000000in}{0.000000in}}%
\pgfpathlineto{\pgfqpoint{-0.048611in}{0.000000in}}%
\pgfusepath{stroke,fill}%
}%
\begin{pgfscope}%
\pgfsys@transformshift{0.648703in}{3.178504in}%
\pgfsys@useobject{currentmarker}{}%
\end{pgfscope}%
\end{pgfscope}%
\begin{pgfscope}%
\definecolor{textcolor}{rgb}{0.000000,0.000000,0.000000}%
\pgfsetstrokecolor{textcolor}%
\pgfsetfillcolor{textcolor}%
\pgftext[x=0.343147in, y=3.130310in, left, base]{\color{textcolor}\sffamily\fontsize{10.000000}{12.000000}\selectfont \(\displaystyle {600}\)}%
\end{pgfscope}%
\begin{pgfscope}%
\pgfsetbuttcap%
\pgfsetroundjoin%
\definecolor{currentfill}{rgb}{0.000000,0.000000,0.000000}%
\pgfsetfillcolor{currentfill}%
\pgfsetlinewidth{0.803000pt}%
\definecolor{currentstroke}{rgb}{0.000000,0.000000,0.000000}%
\pgfsetstrokecolor{currentstroke}%
\pgfsetdash{}{0pt}%
\pgfsys@defobject{currentmarker}{\pgfqpoint{-0.048611in}{0.000000in}}{\pgfqpoint{0.000000in}{0.000000in}}{%
\pgfpathmoveto{\pgfqpoint{0.000000in}{0.000000in}}%
\pgfpathlineto{\pgfqpoint{-0.048611in}{0.000000in}}%
\pgfusepath{stroke,fill}%
}%
\begin{pgfscope}%
\pgfsys@transformshift{0.648703in}{3.593289in}%
\pgfsys@useobject{currentmarker}{}%
\end{pgfscope}%
\end{pgfscope}%
\begin{pgfscope}%
\definecolor{textcolor}{rgb}{0.000000,0.000000,0.000000}%
\pgfsetstrokecolor{textcolor}%
\pgfsetfillcolor{textcolor}%
\pgftext[x=0.343147in, y=3.545094in, left, base]{\color{textcolor}\sffamily\fontsize{10.000000}{12.000000}\selectfont \(\displaystyle {700}\)}%
\end{pgfscope}%
\begin{pgfscope}%
\definecolor{textcolor}{rgb}{0.000000,0.000000,0.000000}%
\pgfsetstrokecolor{textcolor}%
\pgfsetfillcolor{textcolor}%
\pgftext[x=0.287592in,y=2.100064in,,bottom,rotate=90.000000]{\color{textcolor}\sffamily\fontsize{10.000000}{12.000000}\selectfont Data Flow Time (s)}%
\end{pgfscope}%
\begin{pgfscope}%
\pgfpathrectangle{\pgfqpoint{0.648703in}{0.548769in}}{\pgfqpoint{5.112893in}{3.102590in}}%
\pgfusepath{clip}%
\pgfsetrectcap%
\pgfsetroundjoin%
\pgfsetlinewidth{1.505625pt}%
\definecolor{currentstroke}{rgb}{0.000000,0.500000,0.000000}%
\pgfsetstrokecolor{currentstroke}%
\pgfsetdash{}{0pt}%
\pgfpathmoveto{\pgfqpoint{0.814175in}{0.689876in}}%
\pgfpathlineto{\pgfqpoint{0.847604in}{0.915819in}}%
\pgfpathlineto{\pgfqpoint{0.881032in}{1.127625in}}%
\pgfpathlineto{\pgfqpoint{0.914461in}{1.325838in}}%
\pgfpathlineto{\pgfqpoint{0.947889in}{1.510993in}}%
\pgfpathlineto{\pgfqpoint{0.981318in}{1.683614in}}%
\pgfpathlineto{\pgfqpoint{1.014747in}{1.844217in}}%
\pgfpathlineto{\pgfqpoint{1.048175in}{1.993306in}}%
\pgfpathlineto{\pgfqpoint{1.081604in}{2.131376in}}%
\pgfpathlineto{\pgfqpoint{1.115032in}{2.258912in}}%
\pgfpathlineto{\pgfqpoint{1.148461in}{2.376391in}}%
\pgfpathlineto{\pgfqpoint{1.181890in}{2.484278in}}%
\pgfpathlineto{\pgfqpoint{1.215318in}{2.583029in}}%
\pgfpathlineto{\pgfqpoint{1.248747in}{2.673089in}}%
\pgfpathlineto{\pgfqpoint{1.282176in}{2.754896in}}%
\pgfpathlineto{\pgfqpoint{1.315604in}{2.828876in}}%
\pgfpathlineto{\pgfqpoint{1.349033in}{2.895444in}}%
\pgfpathlineto{\pgfqpoint{1.382461in}{2.955009in}}%
\pgfpathlineto{\pgfqpoint{1.415890in}{3.007967in}}%
\pgfpathlineto{\pgfqpoint{1.449319in}{3.054706in}}%
\pgfpathlineto{\pgfqpoint{1.482747in}{3.095602in}}%
\pgfpathlineto{\pgfqpoint{1.516176in}{3.131023in}}%
\pgfpathlineto{\pgfqpoint{1.549604in}{3.161328in}}%
\pgfpathlineto{\pgfqpoint{1.583033in}{3.186864in}}%
\pgfpathlineto{\pgfqpoint{1.616462in}{3.207968in}}%
\pgfpathlineto{\pgfqpoint{1.649890in}{3.224970in}}%
\pgfpathlineto{\pgfqpoint{1.683319in}{3.238189in}}%
\pgfpathlineto{\pgfqpoint{1.716747in}{3.247931in}}%
\pgfpathlineto{\pgfqpoint{1.750176in}{3.254498in}}%
\pgfpathlineto{\pgfqpoint{1.783605in}{3.258177in}}%
\pgfpathlineto{\pgfqpoint{1.817033in}{3.259247in}}%
\pgfpathlineto{\pgfqpoint{1.850462in}{3.257979in}}%
\pgfpathlineto{\pgfqpoint{1.883890in}{3.254632in}}%
\pgfpathlineto{\pgfqpoint{1.917319in}{3.249455in}}%
\pgfpathlineto{\pgfqpoint{1.950748in}{3.242690in}}%
\pgfpathlineto{\pgfqpoint{1.984176in}{3.234564in}}%
\pgfpathlineto{\pgfqpoint{2.017605in}{3.225301in}}%
\pgfpathlineto{\pgfqpoint{2.051034in}{3.215108in}}%
\pgfpathlineto{\pgfqpoint{2.084462in}{3.204188in}}%
\pgfpathlineto{\pgfqpoint{2.117891in}{3.192732in}}%
\pgfpathlineto{\pgfqpoint{2.151319in}{3.180920in}}%
\pgfpathlineto{\pgfqpoint{2.184748in}{3.168924in}}%
\pgfpathlineto{\pgfqpoint{2.218177in}{3.156905in}}%
\pgfpathlineto{\pgfqpoint{2.251605in}{3.145014in}}%
\pgfpathlineto{\pgfqpoint{2.285034in}{3.133395in}}%
\pgfpathlineto{\pgfqpoint{2.318462in}{3.122178in}}%
\pgfpathlineto{\pgfqpoint{2.351891in}{3.111486in}}%
\pgfpathlineto{\pgfqpoint{2.385320in}{3.101431in}}%
\pgfpathlineto{\pgfqpoint{2.418748in}{3.092116in}}%
\pgfpathlineto{\pgfqpoint{2.452177in}{3.083634in}}%
\pgfpathlineto{\pgfqpoint{2.485605in}{3.076067in}}%
\pgfpathlineto{\pgfqpoint{2.519034in}{3.069489in}}%
\pgfpathlineto{\pgfqpoint{2.552463in}{3.063964in}}%
\pgfpathlineto{\pgfqpoint{2.585891in}{3.059543in}}%
\pgfpathlineto{\pgfqpoint{2.619320in}{3.056272in}}%
\pgfpathlineto{\pgfqpoint{2.652748in}{3.054184in}}%
\pgfpathlineto{\pgfqpoint{2.686177in}{3.053304in}}%
\pgfpathlineto{\pgfqpoint{2.719606in}{3.053644in}}%
\pgfpathlineto{\pgfqpoint{2.753034in}{3.055211in}}%
\pgfpathlineto{\pgfqpoint{2.786463in}{3.057998in}}%
\pgfpathlineto{\pgfqpoint{2.819891in}{3.061990in}}%
\pgfpathlineto{\pgfqpoint{2.853320in}{3.067163in}}%
\pgfpathlineto{\pgfqpoint{2.886749in}{3.073481in}}%
\pgfpathlineto{\pgfqpoint{2.920177in}{3.080900in}}%
\pgfpathlineto{\pgfqpoint{2.953606in}{3.089365in}}%
\pgfpathlineto{\pgfqpoint{2.987035in}{3.098813in}}%
\pgfpathlineto{\pgfqpoint{3.020463in}{3.109168in}}%
\pgfpathlineto{\pgfqpoint{3.053892in}{3.120348in}}%
\pgfpathlineto{\pgfqpoint{3.087320in}{3.132258in}}%
\pgfpathlineto{\pgfqpoint{3.120749in}{3.144794in}}%
\pgfpathlineto{\pgfqpoint{3.154178in}{3.157845in}}%
\pgfpathlineto{\pgfqpoint{3.187606in}{3.171285in}}%
\pgfpathlineto{\pgfqpoint{3.221035in}{3.184983in}}%
\pgfpathlineto{\pgfqpoint{3.254463in}{3.198796in}}%
\pgfpathlineto{\pgfqpoint{3.287892in}{3.212571in}}%
\pgfpathlineto{\pgfqpoint{3.321321in}{3.226145in}}%
\pgfpathlineto{\pgfqpoint{3.354749in}{3.239347in}}%
\pgfpathlineto{\pgfqpoint{3.388178in}{3.251994in}}%
\pgfpathlineto{\pgfqpoint{3.421606in}{3.263895in}}%
\pgfpathlineto{\pgfqpoint{3.455035in}{3.274847in}}%
\pgfpathlineto{\pgfqpoint{3.488464in}{3.284640in}}%
\pgfpathlineto{\pgfqpoint{3.521892in}{3.293051in}}%
\pgfpathlineto{\pgfqpoint{3.555321in}{3.299851in}}%
\pgfpathlineto{\pgfqpoint{3.588749in}{3.304798in}}%
\pgfpathlineto{\pgfqpoint{3.622178in}{3.307640in}}%
\pgfpathlineto{\pgfqpoint{3.655607in}{3.308119in}}%
\pgfpathlineto{\pgfqpoint{3.689035in}{3.305962in}}%
\pgfpathlineto{\pgfqpoint{3.722464in}{3.300891in}}%
\pgfpathlineto{\pgfqpoint{3.755892in}{3.292615in}}%
\pgfpathlineto{\pgfqpoint{3.789321in}{3.280834in}}%
\pgfpathlineto{\pgfqpoint{3.822750in}{3.265239in}}%
\pgfpathlineto{\pgfqpoint{3.856178in}{3.245510in}}%
\pgfpathlineto{\pgfqpoint{3.889607in}{3.221318in}}%
\pgfpathlineto{\pgfqpoint{3.923036in}{3.192324in}}%
\pgfpathlineto{\pgfqpoint{3.956464in}{3.158179in}}%
\pgfpathlineto{\pgfqpoint{3.989893in}{3.118524in}}%
\pgfpathlineto{\pgfqpoint{4.023321in}{3.072991in}}%
\pgfpathlineto{\pgfqpoint{4.056750in}{3.021201in}}%
\pgfpathlineto{\pgfqpoint{4.090179in}{2.962766in}}%
\pgfpathlineto{\pgfqpoint{4.123607in}{2.897289in}}%
\pgfusepath{stroke}%
\end{pgfscope}%
\begin{pgfscope}%
\pgfsetrectcap%
\pgfsetmiterjoin%
\pgfsetlinewidth{0.803000pt}%
\definecolor{currentstroke}{rgb}{0.000000,0.000000,0.000000}%
\pgfsetstrokecolor{currentstroke}%
\pgfsetdash{}{0pt}%
\pgfpathmoveto{\pgfqpoint{0.648703in}{0.548769in}}%
\pgfpathlineto{\pgfqpoint{0.648703in}{3.651359in}}%
\pgfusepath{stroke}%
\end{pgfscope}%
\begin{pgfscope}%
\pgfsetrectcap%
\pgfsetmiterjoin%
\pgfsetlinewidth{0.803000pt}%
\definecolor{currentstroke}{rgb}{0.000000,0.000000,0.000000}%
\pgfsetstrokecolor{currentstroke}%
\pgfsetdash{}{0pt}%
\pgfpathmoveto{\pgfqpoint{5.761597in}{0.548769in}}%
\pgfpathlineto{\pgfqpoint{5.761597in}{3.651359in}}%
\pgfusepath{stroke}%
\end{pgfscope}%
\begin{pgfscope}%
\pgfsetrectcap%
\pgfsetmiterjoin%
\pgfsetlinewidth{0.803000pt}%
\definecolor{currentstroke}{rgb}{0.000000,0.000000,0.000000}%
\pgfsetstrokecolor{currentstroke}%
\pgfsetdash{}{0pt}%
\pgfpathmoveto{\pgfqpoint{0.648703in}{0.548769in}}%
\pgfpathlineto{\pgfqpoint{5.761597in}{0.548769in}}%
\pgfusepath{stroke}%
\end{pgfscope}%
\begin{pgfscope}%
\pgfsetrectcap%
\pgfsetmiterjoin%
\pgfsetlinewidth{0.803000pt}%
\definecolor{currentstroke}{rgb}{0.000000,0.000000,0.000000}%
\pgfsetstrokecolor{currentstroke}%
\pgfsetdash{}{0pt}%
\pgfpathmoveto{\pgfqpoint{0.648703in}{3.651359in}}%
\pgfpathlineto{\pgfqpoint{5.761597in}{3.651359in}}%
\pgfusepath{stroke}%
\end{pgfscope}%
\begin{pgfscope}%
\definecolor{textcolor}{rgb}{0.000000,0.000000,0.000000}%
\pgfsetstrokecolor{textcolor}%
\pgfsetfillcolor{textcolor}%
\pgftext[x=3.205150in,y=3.734692in,,base]{\color{textcolor}\sffamily\fontsize{12.000000}{14.400000}\selectfont Forward}%
\end{pgfscope}%
\begin{pgfscope}%
\pgfsetbuttcap%
\pgfsetmiterjoin%
\definecolor{currentfill}{rgb}{1.000000,1.000000,1.000000}%
\pgfsetfillcolor{currentfill}%
\pgfsetfillopacity{0.800000}%
\pgfsetlinewidth{1.003750pt}%
\definecolor{currentstroke}{rgb}{0.800000,0.800000,0.800000}%
\pgfsetstrokecolor{currentstroke}%
\pgfsetstrokeopacity{0.800000}%
\pgfsetdash{}{0pt}%
\pgfpathmoveto{\pgfqpoint{4.212013in}{2.762053in}}%
\pgfpathlineto{\pgfqpoint{5.664374in}{2.762053in}}%
\pgfpathquadraticcurveto{\pgfqpoint{5.692152in}{2.762053in}}{\pgfqpoint{5.692152in}{2.789831in}}%
\pgfpathlineto{\pgfqpoint{5.692152in}{3.554136in}}%
\pgfpathquadraticcurveto{\pgfqpoint{5.692152in}{3.581914in}}{\pgfqpoint{5.664374in}{3.581914in}}%
\pgfpathlineto{\pgfqpoint{4.212013in}{3.581914in}}%
\pgfpathquadraticcurveto{\pgfqpoint{4.184236in}{3.581914in}}{\pgfqpoint{4.184236in}{3.554136in}}%
\pgfpathlineto{\pgfqpoint{4.184236in}{2.789831in}}%
\pgfpathquadraticcurveto{\pgfqpoint{4.184236in}{2.762053in}}{\pgfqpoint{4.212013in}{2.762053in}}%
\pgfpathclose%
\pgfusepath{stroke,fill}%
\end{pgfscope}%
\begin{pgfscope}%
\pgfsetbuttcap%
\pgfsetroundjoin%
\definecolor{currentfill}{rgb}{0.121569,0.466667,0.705882}%
\pgfsetfillcolor{currentfill}%
\pgfsetlinewidth{1.003750pt}%
\definecolor{currentstroke}{rgb}{0.121569,0.466667,0.705882}%
\pgfsetstrokecolor{currentstroke}%
\pgfsetdash{}{0pt}%
\pgfsys@defobject{currentmarker}{\pgfqpoint{-0.034722in}{-0.034722in}}{\pgfqpoint{0.034722in}{0.034722in}}{%
\pgfpathmoveto{\pgfqpoint{0.000000in}{-0.034722in}}%
\pgfpathcurveto{\pgfqpoint{0.009208in}{-0.034722in}}{\pgfqpoint{0.018041in}{-0.031064in}}{\pgfqpoint{0.024552in}{-0.024552in}}%
\pgfpathcurveto{\pgfqpoint{0.031064in}{-0.018041in}}{\pgfqpoint{0.034722in}{-0.009208in}}{\pgfqpoint{0.034722in}{0.000000in}}%
\pgfpathcurveto{\pgfqpoint{0.034722in}{0.009208in}}{\pgfqpoint{0.031064in}{0.018041in}}{\pgfqpoint{0.024552in}{0.024552in}}%
\pgfpathcurveto{\pgfqpoint{0.018041in}{0.031064in}}{\pgfqpoint{0.009208in}{0.034722in}}{\pgfqpoint{0.000000in}{0.034722in}}%
\pgfpathcurveto{\pgfqpoint{-0.009208in}{0.034722in}}{\pgfqpoint{-0.018041in}{0.031064in}}{\pgfqpoint{-0.024552in}{0.024552in}}%
\pgfpathcurveto{\pgfqpoint{-0.031064in}{0.018041in}}{\pgfqpoint{-0.034722in}{0.009208in}}{\pgfqpoint{-0.034722in}{0.000000in}}%
\pgfpathcurveto{\pgfqpoint{-0.034722in}{-0.009208in}}{\pgfqpoint{-0.031064in}{-0.018041in}}{\pgfqpoint{-0.024552in}{-0.024552in}}%
\pgfpathcurveto{\pgfqpoint{-0.018041in}{-0.031064in}}{\pgfqpoint{-0.009208in}{-0.034722in}}{\pgfqpoint{0.000000in}{-0.034722in}}%
\pgfpathclose%
\pgfusepath{stroke,fill}%
}%
\begin{pgfscope}%
\pgfsys@transformshift{4.378680in}{3.477748in}%
\pgfsys@useobject{currentmarker}{}%
\end{pgfscope}%
\end{pgfscope}%
\begin{pgfscope}%
\definecolor{textcolor}{rgb}{0.000000,0.000000,0.000000}%
\pgfsetstrokecolor{textcolor}%
\pgfsetfillcolor{textcolor}%
\pgftext[x=4.628680in,y=3.429136in,left,base]{\color{textcolor}\sffamily\fontsize{10.000000}{12.000000}\selectfont No Timeout}%
\end{pgfscope}%
\begin{pgfscope}%
\pgfsetbuttcap%
\pgfsetroundjoin%
\definecolor{currentfill}{rgb}{1.000000,0.498039,0.054902}%
\pgfsetfillcolor{currentfill}%
\pgfsetlinewidth{1.003750pt}%
\definecolor{currentstroke}{rgb}{1.000000,0.498039,0.054902}%
\pgfsetstrokecolor{currentstroke}%
\pgfsetdash{}{0pt}%
\pgfsys@defobject{currentmarker}{\pgfqpoint{-0.034722in}{-0.034722in}}{\pgfqpoint{0.034722in}{0.034722in}}{%
\pgfpathmoveto{\pgfqpoint{0.000000in}{-0.034722in}}%
\pgfpathcurveto{\pgfqpoint{0.009208in}{-0.034722in}}{\pgfqpoint{0.018041in}{-0.031064in}}{\pgfqpoint{0.024552in}{-0.024552in}}%
\pgfpathcurveto{\pgfqpoint{0.031064in}{-0.018041in}}{\pgfqpoint{0.034722in}{-0.009208in}}{\pgfqpoint{0.034722in}{0.000000in}}%
\pgfpathcurveto{\pgfqpoint{0.034722in}{0.009208in}}{\pgfqpoint{0.031064in}{0.018041in}}{\pgfqpoint{0.024552in}{0.024552in}}%
\pgfpathcurveto{\pgfqpoint{0.018041in}{0.031064in}}{\pgfqpoint{0.009208in}{0.034722in}}{\pgfqpoint{0.000000in}{0.034722in}}%
\pgfpathcurveto{\pgfqpoint{-0.009208in}{0.034722in}}{\pgfqpoint{-0.018041in}{0.031064in}}{\pgfqpoint{-0.024552in}{0.024552in}}%
\pgfpathcurveto{\pgfqpoint{-0.031064in}{0.018041in}}{\pgfqpoint{-0.034722in}{0.009208in}}{\pgfqpoint{-0.034722in}{0.000000in}}%
\pgfpathcurveto{\pgfqpoint{-0.034722in}{-0.009208in}}{\pgfqpoint{-0.031064in}{-0.018041in}}{\pgfqpoint{-0.024552in}{-0.024552in}}%
\pgfpathcurveto{\pgfqpoint{-0.018041in}{-0.031064in}}{\pgfqpoint{-0.009208in}{-0.034722in}}{\pgfqpoint{0.000000in}{-0.034722in}}%
\pgfpathclose%
\pgfusepath{stroke,fill}%
}%
\begin{pgfscope}%
\pgfsys@transformshift{4.378680in}{3.284136in}%
\pgfsys@useobject{currentmarker}{}%
\end{pgfscope}%
\end{pgfscope}%
\begin{pgfscope}%
\definecolor{textcolor}{rgb}{0.000000,0.000000,0.000000}%
\pgfsetstrokecolor{textcolor}%
\pgfsetfillcolor{textcolor}%
\pgftext[x=4.628680in,y=3.235525in,left,base]{\color{textcolor}\sffamily\fontsize{10.000000}{12.000000}\selectfont Time Timeout}%
\end{pgfscope}%
\begin{pgfscope}%
\pgfsetbuttcap%
\pgfsetroundjoin%
\definecolor{currentfill}{rgb}{0.839216,0.152941,0.156863}%
\pgfsetfillcolor{currentfill}%
\pgfsetlinewidth{1.003750pt}%
\definecolor{currentstroke}{rgb}{0.839216,0.152941,0.156863}%
\pgfsetstrokecolor{currentstroke}%
\pgfsetdash{}{0pt}%
\pgfsys@defobject{currentmarker}{\pgfqpoint{-0.034722in}{-0.034722in}}{\pgfqpoint{0.034722in}{0.034722in}}{%
\pgfpathmoveto{\pgfqpoint{0.000000in}{-0.034722in}}%
\pgfpathcurveto{\pgfqpoint{0.009208in}{-0.034722in}}{\pgfqpoint{0.018041in}{-0.031064in}}{\pgfqpoint{0.024552in}{-0.024552in}}%
\pgfpathcurveto{\pgfqpoint{0.031064in}{-0.018041in}}{\pgfqpoint{0.034722in}{-0.009208in}}{\pgfqpoint{0.034722in}{0.000000in}}%
\pgfpathcurveto{\pgfqpoint{0.034722in}{0.009208in}}{\pgfqpoint{0.031064in}{0.018041in}}{\pgfqpoint{0.024552in}{0.024552in}}%
\pgfpathcurveto{\pgfqpoint{0.018041in}{0.031064in}}{\pgfqpoint{0.009208in}{0.034722in}}{\pgfqpoint{0.000000in}{0.034722in}}%
\pgfpathcurveto{\pgfqpoint{-0.009208in}{0.034722in}}{\pgfqpoint{-0.018041in}{0.031064in}}{\pgfqpoint{-0.024552in}{0.024552in}}%
\pgfpathcurveto{\pgfqpoint{-0.031064in}{0.018041in}}{\pgfqpoint{-0.034722in}{0.009208in}}{\pgfqpoint{-0.034722in}{0.000000in}}%
\pgfpathcurveto{\pgfqpoint{-0.034722in}{-0.009208in}}{\pgfqpoint{-0.031064in}{-0.018041in}}{\pgfqpoint{-0.024552in}{-0.024552in}}%
\pgfpathcurveto{\pgfqpoint{-0.018041in}{-0.031064in}}{\pgfqpoint{-0.009208in}{-0.034722in}}{\pgfqpoint{0.000000in}{-0.034722in}}%
\pgfpathclose%
\pgfusepath{stroke,fill}%
}%
\begin{pgfscope}%
\pgfsys@transformshift{4.378680in}{3.090525in}%
\pgfsys@useobject{currentmarker}{}%
\end{pgfscope}%
\end{pgfscope}%
\begin{pgfscope}%
\definecolor{textcolor}{rgb}{0.000000,0.000000,0.000000}%
\pgfsetstrokecolor{textcolor}%
\pgfsetfillcolor{textcolor}%
\pgftext[x=4.628680in,y=3.041914in,left,base]{\color{textcolor}\sffamily\fontsize{10.000000}{12.000000}\selectfont Memory Timeout}%
\end{pgfscope}%
\begin{pgfscope}%
\pgfsetrectcap%
\pgfsetroundjoin%
\pgfsetlinewidth{1.505625pt}%
\definecolor{currentstroke}{rgb}{0.000000,0.500000,0.000000}%
\pgfsetstrokecolor{currentstroke}%
\pgfsetdash{}{0pt}%
\pgfpathmoveto{\pgfqpoint{4.239791in}{2.894692in}}%
\pgfpathlineto{\pgfqpoint{4.517569in}{2.894692in}}%
\pgfusepath{stroke}%
\end{pgfscope}%
\begin{pgfscope}%
\definecolor{textcolor}{rgb}{0.000000,0.000000,0.000000}%
\pgfsetstrokecolor{textcolor}%
\pgfsetfillcolor{textcolor}%
\pgftext[x=4.628680in,y=2.846081in,left,base]{\color{textcolor}\sffamily\fontsize{10.000000}{12.000000}\selectfont Polyfit}%
\end{pgfscope}%
\end{pgfpicture}%
\makeatother%
\endgroup%

                }
            \end{subfigure}
            \qquad
            \begin{subfigure}[]{0.45\textwidth}
                \centering
                \resizebox{\columnwidth}{!}{
                    %% Creator: Matplotlib, PGF backend
%%
%% To include the figure in your LaTeX document, write
%%   \input{<filename>.pgf}
%%
%% Make sure the required packages are loaded in your preamble
%%   \usepackage{pgf}
%%
%% and, on pdftex
%%   \usepackage[utf8]{inputenc}\DeclareUnicodeCharacter{2212}{-}
%%
%% or, on luatex and xetex
%%   \usepackage{unicode-math}
%%
%% Figures using additional raster images can only be included by \input if
%% they are in the same directory as the main LaTeX file. For loading figures
%% from other directories you can use the `import` package
%%   \usepackage{import}
%%
%% and then include the figures with
%%   \import{<path to file>}{<filename>.pgf}
%%
%% Matplotlib used the following preamble
%%   \usepackage{amsmath}
%%   \usepackage{fontspec}
%%
\begingroup%
\makeatletter%
\begin{pgfpicture}%
\pgfpathrectangle{\pgfpointorigin}{\pgfqpoint{6.000000in}{4.000000in}}%
\pgfusepath{use as bounding box, clip}%
\begin{pgfscope}%
\pgfsetbuttcap%
\pgfsetmiterjoin%
\definecolor{currentfill}{rgb}{1.000000,1.000000,1.000000}%
\pgfsetfillcolor{currentfill}%
\pgfsetlinewidth{0.000000pt}%
\definecolor{currentstroke}{rgb}{1.000000,1.000000,1.000000}%
\pgfsetstrokecolor{currentstroke}%
\pgfsetdash{}{0pt}%
\pgfpathmoveto{\pgfqpoint{0.000000in}{0.000000in}}%
\pgfpathlineto{\pgfqpoint{6.000000in}{0.000000in}}%
\pgfpathlineto{\pgfqpoint{6.000000in}{4.000000in}}%
\pgfpathlineto{\pgfqpoint{0.000000in}{4.000000in}}%
\pgfpathclose%
\pgfusepath{fill}%
\end{pgfscope}%
\begin{pgfscope}%
\pgfsetbuttcap%
\pgfsetmiterjoin%
\definecolor{currentfill}{rgb}{1.000000,1.000000,1.000000}%
\pgfsetfillcolor{currentfill}%
\pgfsetlinewidth{0.000000pt}%
\definecolor{currentstroke}{rgb}{0.000000,0.000000,0.000000}%
\pgfsetstrokecolor{currentstroke}%
\pgfsetstrokeopacity{0.000000}%
\pgfsetdash{}{0pt}%
\pgfpathmoveto{\pgfqpoint{0.648703in}{0.548769in}}%
\pgfpathlineto{\pgfqpoint{5.761597in}{0.548769in}}%
\pgfpathlineto{\pgfqpoint{5.761597in}{3.651359in}}%
\pgfpathlineto{\pgfqpoint{0.648703in}{3.651359in}}%
\pgfpathclose%
\pgfusepath{fill}%
\end{pgfscope}%
\begin{pgfscope}%
\pgfpathrectangle{\pgfqpoint{0.648703in}{0.548769in}}{\pgfqpoint{5.112893in}{3.102590in}}%
\pgfusepath{clip}%
\pgfsetbuttcap%
\pgfsetroundjoin%
\definecolor{currentfill}{rgb}{0.121569,0.466667,0.705882}%
\pgfsetfillcolor{currentfill}%
\pgfsetlinewidth{1.003750pt}%
\definecolor{currentstroke}{rgb}{0.121569,0.466667,0.705882}%
\pgfsetstrokecolor{currentstroke}%
\pgfsetdash{}{0pt}%
\pgfpathmoveto{\pgfqpoint{0.801536in}{0.742043in}}%
\pgfpathcurveto{\pgfqpoint{0.812586in}{0.742043in}}{\pgfqpoint{0.823185in}{0.746433in}}{\pgfqpoint{0.830999in}{0.754247in}}%
\pgfpathcurveto{\pgfqpoint{0.838813in}{0.762061in}}{\pgfqpoint{0.843203in}{0.772660in}}{\pgfqpoint{0.843203in}{0.783710in}}%
\pgfpathcurveto{\pgfqpoint{0.843203in}{0.794760in}}{\pgfqpoint{0.838813in}{0.805359in}}{\pgfqpoint{0.830999in}{0.813172in}}%
\pgfpathcurveto{\pgfqpoint{0.823185in}{0.820986in}}{\pgfqpoint{0.812586in}{0.825376in}}{\pgfqpoint{0.801536in}{0.825376in}}%
\pgfpathcurveto{\pgfqpoint{0.790486in}{0.825376in}}{\pgfqpoint{0.779887in}{0.820986in}}{\pgfqpoint{0.772074in}{0.813172in}}%
\pgfpathcurveto{\pgfqpoint{0.764260in}{0.805359in}}{\pgfqpoint{0.759870in}{0.794760in}}{\pgfqpoint{0.759870in}{0.783710in}}%
\pgfpathcurveto{\pgfqpoint{0.759870in}{0.772660in}}{\pgfqpoint{0.764260in}{0.762061in}}{\pgfqpoint{0.772074in}{0.754247in}}%
\pgfpathcurveto{\pgfqpoint{0.779887in}{0.746433in}}{\pgfqpoint{0.790486in}{0.742043in}}{\pgfqpoint{0.801536in}{0.742043in}}%
\pgfpathclose%
\pgfusepath{stroke,fill}%
\end{pgfscope}%
\begin{pgfscope}%
\pgfpathrectangle{\pgfqpoint{0.648703in}{0.548769in}}{\pgfqpoint{5.112893in}{3.102590in}}%
\pgfusepath{clip}%
\pgfsetbuttcap%
\pgfsetroundjoin%
\definecolor{currentfill}{rgb}{0.121569,0.466667,0.705882}%
\pgfsetfillcolor{currentfill}%
\pgfsetlinewidth{1.003750pt}%
\definecolor{currentstroke}{rgb}{0.121569,0.466667,0.705882}%
\pgfsetstrokecolor{currentstroke}%
\pgfsetdash{}{0pt}%
\pgfpathmoveto{\pgfqpoint{1.256560in}{3.184040in}}%
\pgfpathcurveto{\pgfqpoint{1.267611in}{3.184040in}}{\pgfqpoint{1.278210in}{3.188431in}}{\pgfqpoint{1.286023in}{3.196244in}}%
\pgfpathcurveto{\pgfqpoint{1.293837in}{3.204058in}}{\pgfqpoint{1.298227in}{3.214657in}}{\pgfqpoint{1.298227in}{3.225707in}}%
\pgfpathcurveto{\pgfqpoint{1.298227in}{3.236757in}}{\pgfqpoint{1.293837in}{3.247356in}}{\pgfqpoint{1.286023in}{3.255170in}}%
\pgfpathcurveto{\pgfqpoint{1.278210in}{3.262984in}}{\pgfqpoint{1.267611in}{3.267374in}}{\pgfqpoint{1.256560in}{3.267374in}}%
\pgfpathcurveto{\pgfqpoint{1.245510in}{3.267374in}}{\pgfqpoint{1.234911in}{3.262984in}}{\pgfqpoint{1.227098in}{3.255170in}}%
\pgfpathcurveto{\pgfqpoint{1.219284in}{3.247356in}}{\pgfqpoint{1.214894in}{3.236757in}}{\pgfqpoint{1.214894in}{3.225707in}}%
\pgfpathcurveto{\pgfqpoint{1.214894in}{3.214657in}}{\pgfqpoint{1.219284in}{3.204058in}}{\pgfqpoint{1.227098in}{3.196244in}}%
\pgfpathcurveto{\pgfqpoint{1.234911in}{3.188431in}}{\pgfqpoint{1.245510in}{3.184040in}}{\pgfqpoint{1.256560in}{3.184040in}}%
\pgfpathclose%
\pgfusepath{stroke,fill}%
\end{pgfscope}%
\begin{pgfscope}%
\pgfpathrectangle{\pgfqpoint{0.648703in}{0.548769in}}{\pgfqpoint{5.112893in}{3.102590in}}%
\pgfusepath{clip}%
\pgfsetbuttcap%
\pgfsetroundjoin%
\definecolor{currentfill}{rgb}{1.000000,0.498039,0.054902}%
\pgfsetfillcolor{currentfill}%
\pgfsetlinewidth{1.003750pt}%
\definecolor{currentstroke}{rgb}{1.000000,0.498039,0.054902}%
\pgfsetstrokecolor{currentstroke}%
\pgfsetdash{}{0pt}%
\pgfpathmoveto{\pgfqpoint{1.415379in}{3.204665in}}%
\pgfpathcurveto{\pgfqpoint{1.426429in}{3.204665in}}{\pgfqpoint{1.437028in}{3.209056in}}{\pgfqpoint{1.444842in}{3.216869in}}%
\pgfpathcurveto{\pgfqpoint{1.452655in}{3.224683in}}{\pgfqpoint{1.457046in}{3.235282in}}{\pgfqpoint{1.457046in}{3.246332in}}%
\pgfpathcurveto{\pgfqpoint{1.457046in}{3.257382in}}{\pgfqpoint{1.452655in}{3.267981in}}{\pgfqpoint{1.444842in}{3.275795in}}%
\pgfpathcurveto{\pgfqpoint{1.437028in}{3.283608in}}{\pgfqpoint{1.426429in}{3.287999in}}{\pgfqpoint{1.415379in}{3.287999in}}%
\pgfpathcurveto{\pgfqpoint{1.404329in}{3.287999in}}{\pgfqpoint{1.393730in}{3.283608in}}{\pgfqpoint{1.385916in}{3.275795in}}%
\pgfpathcurveto{\pgfqpoint{1.378103in}{3.267981in}}{\pgfqpoint{1.373712in}{3.257382in}}{\pgfqpoint{1.373712in}{3.246332in}}%
\pgfpathcurveto{\pgfqpoint{1.373712in}{3.235282in}}{\pgfqpoint{1.378103in}{3.224683in}}{\pgfqpoint{1.385916in}{3.216869in}}%
\pgfpathcurveto{\pgfqpoint{1.393730in}{3.209056in}}{\pgfqpoint{1.404329in}{3.204665in}}{\pgfqpoint{1.415379in}{3.204665in}}%
\pgfpathclose%
\pgfusepath{stroke,fill}%
\end{pgfscope}%
\begin{pgfscope}%
\pgfpathrectangle{\pgfqpoint{0.648703in}{0.548769in}}{\pgfqpoint{5.112893in}{3.102590in}}%
\pgfusepath{clip}%
\pgfsetbuttcap%
\pgfsetroundjoin%
\definecolor{currentfill}{rgb}{1.000000,0.498039,0.054902}%
\pgfsetfillcolor{currentfill}%
\pgfsetlinewidth{1.003750pt}%
\definecolor{currentstroke}{rgb}{1.000000,0.498039,0.054902}%
\pgfsetstrokecolor{currentstroke}%
\pgfsetdash{}{0pt}%
\pgfpathmoveto{\pgfqpoint{1.313351in}{3.192290in}}%
\pgfpathcurveto{\pgfqpoint{1.324401in}{3.192290in}}{\pgfqpoint{1.335000in}{3.196681in}}{\pgfqpoint{1.342813in}{3.204494in}}%
\pgfpathcurveto{\pgfqpoint{1.350627in}{3.212308in}}{\pgfqpoint{1.355017in}{3.222907in}}{\pgfqpoint{1.355017in}{3.233957in}}%
\pgfpathcurveto{\pgfqpoint{1.355017in}{3.245007in}}{\pgfqpoint{1.350627in}{3.255606in}}{\pgfqpoint{1.342813in}{3.263420in}}%
\pgfpathcurveto{\pgfqpoint{1.335000in}{3.271234in}}{\pgfqpoint{1.324401in}{3.275624in}}{\pgfqpoint{1.313351in}{3.275624in}}%
\pgfpathcurveto{\pgfqpoint{1.302300in}{3.275624in}}{\pgfqpoint{1.291701in}{3.271234in}}{\pgfqpoint{1.283888in}{3.263420in}}%
\pgfpathcurveto{\pgfqpoint{1.276074in}{3.255606in}}{\pgfqpoint{1.271684in}{3.245007in}}{\pgfqpoint{1.271684in}{3.233957in}}%
\pgfpathcurveto{\pgfqpoint{1.271684in}{3.222907in}}{\pgfqpoint{1.276074in}{3.212308in}}{\pgfqpoint{1.283888in}{3.204494in}}%
\pgfpathcurveto{\pgfqpoint{1.291701in}{3.196681in}}{\pgfqpoint{1.302300in}{3.192290in}}{\pgfqpoint{1.313351in}{3.192290in}}%
\pgfpathclose%
\pgfusepath{stroke,fill}%
\end{pgfscope}%
\begin{pgfscope}%
\pgfpathrectangle{\pgfqpoint{0.648703in}{0.548769in}}{\pgfqpoint{5.112893in}{3.102590in}}%
\pgfusepath{clip}%
\pgfsetbuttcap%
\pgfsetroundjoin%
\definecolor{currentfill}{rgb}{0.121569,0.466667,0.705882}%
\pgfsetfillcolor{currentfill}%
\pgfsetlinewidth{1.003750pt}%
\definecolor{currentstroke}{rgb}{0.121569,0.466667,0.705882}%
\pgfsetstrokecolor{currentstroke}%
\pgfsetdash{}{0pt}%
\pgfpathmoveto{\pgfqpoint{0.769540in}{0.721418in}}%
\pgfpathcurveto{\pgfqpoint{0.780590in}{0.721418in}}{\pgfqpoint{0.791189in}{0.725808in}}{\pgfqpoint{0.799003in}{0.733622in}}%
\pgfpathcurveto{\pgfqpoint{0.806816in}{0.741436in}}{\pgfqpoint{0.811207in}{0.752035in}}{\pgfqpoint{0.811207in}{0.763085in}}%
\pgfpathcurveto{\pgfqpoint{0.811207in}{0.774135in}}{\pgfqpoint{0.806816in}{0.784734in}}{\pgfqpoint{0.799003in}{0.792548in}}%
\pgfpathcurveto{\pgfqpoint{0.791189in}{0.800361in}}{\pgfqpoint{0.780590in}{0.804751in}}{\pgfqpoint{0.769540in}{0.804751in}}%
\pgfpathcurveto{\pgfqpoint{0.758490in}{0.804751in}}{\pgfqpoint{0.747891in}{0.800361in}}{\pgfqpoint{0.740077in}{0.792548in}}%
\pgfpathcurveto{\pgfqpoint{0.732264in}{0.784734in}}{\pgfqpoint{0.727873in}{0.774135in}}{\pgfqpoint{0.727873in}{0.763085in}}%
\pgfpathcurveto{\pgfqpoint{0.727873in}{0.752035in}}{\pgfqpoint{0.732264in}{0.741436in}}{\pgfqpoint{0.740077in}{0.733622in}}%
\pgfpathcurveto{\pgfqpoint{0.747891in}{0.725808in}}{\pgfqpoint{0.758490in}{0.721418in}}{\pgfqpoint{0.769540in}{0.721418in}}%
\pgfpathclose%
\pgfusepath{stroke,fill}%
\end{pgfscope}%
\begin{pgfscope}%
\pgfpathrectangle{\pgfqpoint{0.648703in}{0.548769in}}{\pgfqpoint{5.112893in}{3.102590in}}%
\pgfusepath{clip}%
\pgfsetbuttcap%
\pgfsetroundjoin%
\definecolor{currentfill}{rgb}{0.121569,0.466667,0.705882}%
\pgfsetfillcolor{currentfill}%
\pgfsetlinewidth{1.003750pt}%
\definecolor{currentstroke}{rgb}{0.121569,0.466667,0.705882}%
\pgfsetstrokecolor{currentstroke}%
\pgfsetdash{}{0pt}%
\pgfpathmoveto{\pgfqpoint{2.421202in}{3.188165in}}%
\pgfpathcurveto{\pgfqpoint{2.432252in}{3.188165in}}{\pgfqpoint{2.442851in}{3.192556in}}{\pgfqpoint{2.450665in}{3.200369in}}%
\pgfpathcurveto{\pgfqpoint{2.458479in}{3.208183in}}{\pgfqpoint{2.462869in}{3.218782in}}{\pgfqpoint{2.462869in}{3.229832in}}%
\pgfpathcurveto{\pgfqpoint{2.462869in}{3.240882in}}{\pgfqpoint{2.458479in}{3.251481in}}{\pgfqpoint{2.450665in}{3.259295in}}%
\pgfpathcurveto{\pgfqpoint{2.442851in}{3.267109in}}{\pgfqpoint{2.432252in}{3.271499in}}{\pgfqpoint{2.421202in}{3.271499in}}%
\pgfpathcurveto{\pgfqpoint{2.410152in}{3.271499in}}{\pgfqpoint{2.399553in}{3.267109in}}{\pgfqpoint{2.391740in}{3.259295in}}%
\pgfpathcurveto{\pgfqpoint{2.383926in}{3.251481in}}{\pgfqpoint{2.379536in}{3.240882in}}{\pgfqpoint{2.379536in}{3.229832in}}%
\pgfpathcurveto{\pgfqpoint{2.379536in}{3.218782in}}{\pgfqpoint{2.383926in}{3.208183in}}{\pgfqpoint{2.391740in}{3.200369in}}%
\pgfpathcurveto{\pgfqpoint{2.399553in}{3.192556in}}{\pgfqpoint{2.410152in}{3.188165in}}{\pgfqpoint{2.421202in}{3.188165in}}%
\pgfpathclose%
\pgfusepath{stroke,fill}%
\end{pgfscope}%
\begin{pgfscope}%
\pgfpathrectangle{\pgfqpoint{0.648703in}{0.548769in}}{\pgfqpoint{5.112893in}{3.102590in}}%
\pgfusepath{clip}%
\pgfsetbuttcap%
\pgfsetroundjoin%
\definecolor{currentfill}{rgb}{1.000000,0.498039,0.054902}%
\pgfsetfillcolor{currentfill}%
\pgfsetlinewidth{1.003750pt}%
\definecolor{currentstroke}{rgb}{1.000000,0.498039,0.054902}%
\pgfsetstrokecolor{currentstroke}%
\pgfsetdash{}{0pt}%
\pgfpathmoveto{\pgfqpoint{1.567576in}{3.196415in}}%
\pgfpathcurveto{\pgfqpoint{1.578626in}{3.196415in}}{\pgfqpoint{1.589225in}{3.200806in}}{\pgfqpoint{1.597039in}{3.208619in}}%
\pgfpathcurveto{\pgfqpoint{1.604852in}{3.216433in}}{\pgfqpoint{1.609243in}{3.227032in}}{\pgfqpoint{1.609243in}{3.238082in}}%
\pgfpathcurveto{\pgfqpoint{1.609243in}{3.249132in}}{\pgfqpoint{1.604852in}{3.259731in}}{\pgfqpoint{1.597039in}{3.267545in}}%
\pgfpathcurveto{\pgfqpoint{1.589225in}{3.275358in}}{\pgfqpoint{1.578626in}{3.279749in}}{\pgfqpoint{1.567576in}{3.279749in}}%
\pgfpathcurveto{\pgfqpoint{1.556526in}{3.279749in}}{\pgfqpoint{1.545927in}{3.275358in}}{\pgfqpoint{1.538113in}{3.267545in}}%
\pgfpathcurveto{\pgfqpoint{1.530300in}{3.259731in}}{\pgfqpoint{1.525909in}{3.249132in}}{\pgfqpoint{1.525909in}{3.238082in}}%
\pgfpathcurveto{\pgfqpoint{1.525909in}{3.227032in}}{\pgfqpoint{1.530300in}{3.216433in}}{\pgfqpoint{1.538113in}{3.208619in}}%
\pgfpathcurveto{\pgfqpoint{1.545927in}{3.200806in}}{\pgfqpoint{1.556526in}{3.196415in}}{\pgfqpoint{1.567576in}{3.196415in}}%
\pgfpathclose%
\pgfusepath{stroke,fill}%
\end{pgfscope}%
\begin{pgfscope}%
\pgfpathrectangle{\pgfqpoint{0.648703in}{0.548769in}}{\pgfqpoint{5.112893in}{3.102590in}}%
\pgfusepath{clip}%
\pgfsetbuttcap%
\pgfsetroundjoin%
\definecolor{currentfill}{rgb}{0.121569,0.466667,0.705882}%
\pgfsetfillcolor{currentfill}%
\pgfsetlinewidth{1.003750pt}%
\definecolor{currentstroke}{rgb}{0.121569,0.466667,0.705882}%
\pgfsetstrokecolor{currentstroke}%
\pgfsetdash{}{0pt}%
\pgfpathmoveto{\pgfqpoint{0.767695in}{0.717293in}}%
\pgfpathcurveto{\pgfqpoint{0.778745in}{0.717293in}}{\pgfqpoint{0.789344in}{0.721683in}}{\pgfqpoint{0.797157in}{0.729497in}}%
\pgfpathcurveto{\pgfqpoint{0.804971in}{0.737311in}}{\pgfqpoint{0.809361in}{0.747910in}}{\pgfqpoint{0.809361in}{0.758960in}}%
\pgfpathcurveto{\pgfqpoint{0.809361in}{0.770010in}}{\pgfqpoint{0.804971in}{0.780609in}}{\pgfqpoint{0.797157in}{0.788423in}}%
\pgfpathcurveto{\pgfqpoint{0.789344in}{0.796236in}}{\pgfqpoint{0.778745in}{0.800626in}}{\pgfqpoint{0.767695in}{0.800626in}}%
\pgfpathcurveto{\pgfqpoint{0.756644in}{0.800626in}}{\pgfqpoint{0.746045in}{0.796236in}}{\pgfqpoint{0.738232in}{0.788423in}}%
\pgfpathcurveto{\pgfqpoint{0.730418in}{0.780609in}}{\pgfqpoint{0.726028in}{0.770010in}}{\pgfqpoint{0.726028in}{0.758960in}}%
\pgfpathcurveto{\pgfqpoint{0.726028in}{0.747910in}}{\pgfqpoint{0.730418in}{0.737311in}}{\pgfqpoint{0.738232in}{0.729497in}}%
\pgfpathcurveto{\pgfqpoint{0.746045in}{0.721683in}}{\pgfqpoint{0.756644in}{0.717293in}}{\pgfqpoint{0.767695in}{0.717293in}}%
\pgfpathclose%
\pgfusepath{stroke,fill}%
\end{pgfscope}%
\begin{pgfscope}%
\pgfpathrectangle{\pgfqpoint{0.648703in}{0.548769in}}{\pgfqpoint{5.112893in}{3.102590in}}%
\pgfusepath{clip}%
\pgfsetbuttcap%
\pgfsetroundjoin%
\definecolor{currentfill}{rgb}{0.121569,0.466667,0.705882}%
\pgfsetfillcolor{currentfill}%
\pgfsetlinewidth{1.003750pt}%
\definecolor{currentstroke}{rgb}{0.121569,0.466667,0.705882}%
\pgfsetstrokecolor{currentstroke}%
\pgfsetdash{}{0pt}%
\pgfpathmoveto{\pgfqpoint{0.880295in}{0.841043in}}%
\pgfpathcurveto{\pgfqpoint{0.891346in}{0.841043in}}{\pgfqpoint{0.901945in}{0.845433in}}{\pgfqpoint{0.909758in}{0.853247in}}%
\pgfpathcurveto{\pgfqpoint{0.917572in}{0.861060in}}{\pgfqpoint{0.921962in}{0.871659in}}{\pgfqpoint{0.921962in}{0.882710in}}%
\pgfpathcurveto{\pgfqpoint{0.921962in}{0.893760in}}{\pgfqpoint{0.917572in}{0.904359in}}{\pgfqpoint{0.909758in}{0.912172in}}%
\pgfpathcurveto{\pgfqpoint{0.901945in}{0.919986in}}{\pgfqpoint{0.891346in}{0.924376in}}{\pgfqpoint{0.880295in}{0.924376in}}%
\pgfpathcurveto{\pgfqpoint{0.869245in}{0.924376in}}{\pgfqpoint{0.858646in}{0.919986in}}{\pgfqpoint{0.850833in}{0.912172in}}%
\pgfpathcurveto{\pgfqpoint{0.843019in}{0.904359in}}{\pgfqpoint{0.838629in}{0.893760in}}{\pgfqpoint{0.838629in}{0.882710in}}%
\pgfpathcurveto{\pgfqpoint{0.838629in}{0.871659in}}{\pgfqpoint{0.843019in}{0.861060in}}{\pgfqpoint{0.850833in}{0.853247in}}%
\pgfpathcurveto{\pgfqpoint{0.858646in}{0.845433in}}{\pgfqpoint{0.869245in}{0.841043in}}{\pgfqpoint{0.880295in}{0.841043in}}%
\pgfpathclose%
\pgfusepath{stroke,fill}%
\end{pgfscope}%
\begin{pgfscope}%
\pgfpathrectangle{\pgfqpoint{0.648703in}{0.548769in}}{\pgfqpoint{5.112893in}{3.102590in}}%
\pgfusepath{clip}%
\pgfsetbuttcap%
\pgfsetroundjoin%
\definecolor{currentfill}{rgb}{0.121569,0.466667,0.705882}%
\pgfsetfillcolor{currentfill}%
\pgfsetlinewidth{1.003750pt}%
\definecolor{currentstroke}{rgb}{0.121569,0.466667,0.705882}%
\pgfsetstrokecolor{currentstroke}%
\pgfsetdash{}{0pt}%
\pgfpathmoveto{\pgfqpoint{0.826549in}{0.849293in}}%
\pgfpathcurveto{\pgfqpoint{0.837599in}{0.849293in}}{\pgfqpoint{0.848198in}{0.853683in}}{\pgfqpoint{0.856012in}{0.861497in}}%
\pgfpathcurveto{\pgfqpoint{0.863825in}{0.869310in}}{\pgfqpoint{0.868216in}{0.879909in}}{\pgfqpoint{0.868216in}{0.890960in}}%
\pgfpathcurveto{\pgfqpoint{0.868216in}{0.902010in}}{\pgfqpoint{0.863825in}{0.912609in}}{\pgfqpoint{0.856012in}{0.920422in}}%
\pgfpathcurveto{\pgfqpoint{0.848198in}{0.928236in}}{\pgfqpoint{0.837599in}{0.932626in}}{\pgfqpoint{0.826549in}{0.932626in}}%
\pgfpathcurveto{\pgfqpoint{0.815499in}{0.932626in}}{\pgfqpoint{0.804900in}{0.928236in}}{\pgfqpoint{0.797086in}{0.920422in}}%
\pgfpathcurveto{\pgfqpoint{0.789273in}{0.912609in}}{\pgfqpoint{0.784882in}{0.902010in}}{\pgfqpoint{0.784882in}{0.890960in}}%
\pgfpathcurveto{\pgfqpoint{0.784882in}{0.879909in}}{\pgfqpoint{0.789273in}{0.869310in}}{\pgfqpoint{0.797086in}{0.861497in}}%
\pgfpathcurveto{\pgfqpoint{0.804900in}{0.853683in}}{\pgfqpoint{0.815499in}{0.849293in}}{\pgfqpoint{0.826549in}{0.849293in}}%
\pgfpathclose%
\pgfusepath{stroke,fill}%
\end{pgfscope}%
\begin{pgfscope}%
\pgfpathrectangle{\pgfqpoint{0.648703in}{0.548769in}}{\pgfqpoint{5.112893in}{3.102590in}}%
\pgfusepath{clip}%
\pgfsetbuttcap%
\pgfsetroundjoin%
\definecolor{currentfill}{rgb}{0.121569,0.466667,0.705882}%
\pgfsetfillcolor{currentfill}%
\pgfsetlinewidth{1.003750pt}%
\definecolor{currentstroke}{rgb}{0.121569,0.466667,0.705882}%
\pgfsetstrokecolor{currentstroke}%
\pgfsetdash{}{0pt}%
\pgfpathmoveto{\pgfqpoint{0.774728in}{0.721418in}}%
\pgfpathcurveto{\pgfqpoint{0.785778in}{0.721418in}}{\pgfqpoint{0.796377in}{0.725808in}}{\pgfqpoint{0.804190in}{0.733622in}}%
\pgfpathcurveto{\pgfqpoint{0.812004in}{0.741436in}}{\pgfqpoint{0.816394in}{0.752035in}}{\pgfqpoint{0.816394in}{0.763085in}}%
\pgfpathcurveto{\pgfqpoint{0.816394in}{0.774135in}}{\pgfqpoint{0.812004in}{0.784734in}}{\pgfqpoint{0.804190in}{0.792548in}}%
\pgfpathcurveto{\pgfqpoint{0.796377in}{0.800361in}}{\pgfqpoint{0.785778in}{0.804751in}}{\pgfqpoint{0.774728in}{0.804751in}}%
\pgfpathcurveto{\pgfqpoint{0.763677in}{0.804751in}}{\pgfqpoint{0.753078in}{0.800361in}}{\pgfqpoint{0.745265in}{0.792548in}}%
\pgfpathcurveto{\pgfqpoint{0.737451in}{0.784734in}}{\pgfqpoint{0.733061in}{0.774135in}}{\pgfqpoint{0.733061in}{0.763085in}}%
\pgfpathcurveto{\pgfqpoint{0.733061in}{0.752035in}}{\pgfqpoint{0.737451in}{0.741436in}}{\pgfqpoint{0.745265in}{0.733622in}}%
\pgfpathcurveto{\pgfqpoint{0.753078in}{0.725808in}}{\pgfqpoint{0.763677in}{0.721418in}}{\pgfqpoint{0.774728in}{0.721418in}}%
\pgfpathclose%
\pgfusepath{stroke,fill}%
\end{pgfscope}%
\begin{pgfscope}%
\pgfpathrectangle{\pgfqpoint{0.648703in}{0.548769in}}{\pgfqpoint{5.112893in}{3.102590in}}%
\pgfusepath{clip}%
\pgfsetbuttcap%
\pgfsetroundjoin%
\definecolor{currentfill}{rgb}{0.121569,0.466667,0.705882}%
\pgfsetfillcolor{currentfill}%
\pgfsetlinewidth{1.003750pt}%
\definecolor{currentstroke}{rgb}{0.121569,0.466667,0.705882}%
\pgfsetstrokecolor{currentstroke}%
\pgfsetdash{}{0pt}%
\pgfpathmoveto{\pgfqpoint{0.767699in}{0.717293in}}%
\pgfpathcurveto{\pgfqpoint{0.778749in}{0.717293in}}{\pgfqpoint{0.789348in}{0.721683in}}{\pgfqpoint{0.797162in}{0.729497in}}%
\pgfpathcurveto{\pgfqpoint{0.804975in}{0.737311in}}{\pgfqpoint{0.809365in}{0.747910in}}{\pgfqpoint{0.809365in}{0.758960in}}%
\pgfpathcurveto{\pgfqpoint{0.809365in}{0.770010in}}{\pgfqpoint{0.804975in}{0.780609in}}{\pgfqpoint{0.797162in}{0.788423in}}%
\pgfpathcurveto{\pgfqpoint{0.789348in}{0.796236in}}{\pgfqpoint{0.778749in}{0.800626in}}{\pgfqpoint{0.767699in}{0.800626in}}%
\pgfpathcurveto{\pgfqpoint{0.756649in}{0.800626in}}{\pgfqpoint{0.746050in}{0.796236in}}{\pgfqpoint{0.738236in}{0.788423in}}%
\pgfpathcurveto{\pgfqpoint{0.730422in}{0.780609in}}{\pgfqpoint{0.726032in}{0.770010in}}{\pgfqpoint{0.726032in}{0.758960in}}%
\pgfpathcurveto{\pgfqpoint{0.726032in}{0.747910in}}{\pgfqpoint{0.730422in}{0.737311in}}{\pgfqpoint{0.738236in}{0.729497in}}%
\pgfpathcurveto{\pgfqpoint{0.746050in}{0.721683in}}{\pgfqpoint{0.756649in}{0.717293in}}{\pgfqpoint{0.767699in}{0.717293in}}%
\pgfpathclose%
\pgfusepath{stroke,fill}%
\end{pgfscope}%
\begin{pgfscope}%
\pgfpathrectangle{\pgfqpoint{0.648703in}{0.548769in}}{\pgfqpoint{5.112893in}{3.102590in}}%
\pgfusepath{clip}%
\pgfsetbuttcap%
\pgfsetroundjoin%
\definecolor{currentfill}{rgb}{0.121569,0.466667,0.705882}%
\pgfsetfillcolor{currentfill}%
\pgfsetlinewidth{1.003750pt}%
\definecolor{currentstroke}{rgb}{0.121569,0.466667,0.705882}%
\pgfsetstrokecolor{currentstroke}%
\pgfsetdash{}{0pt}%
\pgfpathmoveto{\pgfqpoint{0.767688in}{0.717293in}}%
\pgfpathcurveto{\pgfqpoint{0.778738in}{0.717293in}}{\pgfqpoint{0.789337in}{0.721683in}}{\pgfqpoint{0.797151in}{0.729497in}}%
\pgfpathcurveto{\pgfqpoint{0.804964in}{0.737311in}}{\pgfqpoint{0.809355in}{0.747910in}}{\pgfqpoint{0.809355in}{0.758960in}}%
\pgfpathcurveto{\pgfqpoint{0.809355in}{0.770010in}}{\pgfqpoint{0.804964in}{0.780609in}}{\pgfqpoint{0.797151in}{0.788423in}}%
\pgfpathcurveto{\pgfqpoint{0.789337in}{0.796236in}}{\pgfqpoint{0.778738in}{0.800626in}}{\pgfqpoint{0.767688in}{0.800626in}}%
\pgfpathcurveto{\pgfqpoint{0.756638in}{0.800626in}}{\pgfqpoint{0.746039in}{0.796236in}}{\pgfqpoint{0.738225in}{0.788423in}}%
\pgfpathcurveto{\pgfqpoint{0.730412in}{0.780609in}}{\pgfqpoint{0.726021in}{0.770010in}}{\pgfqpoint{0.726021in}{0.758960in}}%
\pgfpathcurveto{\pgfqpoint{0.726021in}{0.747910in}}{\pgfqpoint{0.730412in}{0.737311in}}{\pgfqpoint{0.738225in}{0.729497in}}%
\pgfpathcurveto{\pgfqpoint{0.746039in}{0.721683in}}{\pgfqpoint{0.756638in}{0.717293in}}{\pgfqpoint{0.767688in}{0.717293in}}%
\pgfpathclose%
\pgfusepath{stroke,fill}%
\end{pgfscope}%
\begin{pgfscope}%
\pgfpathrectangle{\pgfqpoint{0.648703in}{0.548769in}}{\pgfqpoint{5.112893in}{3.102590in}}%
\pgfusepath{clip}%
\pgfsetbuttcap%
\pgfsetroundjoin%
\definecolor{currentfill}{rgb}{1.000000,0.498039,0.054902}%
\pgfsetfillcolor{currentfill}%
\pgfsetlinewidth{1.003750pt}%
\definecolor{currentstroke}{rgb}{1.000000,0.498039,0.054902}%
\pgfsetstrokecolor{currentstroke}%
\pgfsetdash{}{0pt}%
\pgfpathmoveto{\pgfqpoint{2.787879in}{3.200540in}}%
\pgfpathcurveto{\pgfqpoint{2.798929in}{3.200540in}}{\pgfqpoint{2.809528in}{3.204931in}}{\pgfqpoint{2.817342in}{3.212744in}}%
\pgfpathcurveto{\pgfqpoint{2.825156in}{3.220558in}}{\pgfqpoint{2.829546in}{3.231157in}}{\pgfqpoint{2.829546in}{3.242207in}}%
\pgfpathcurveto{\pgfqpoint{2.829546in}{3.253257in}}{\pgfqpoint{2.825156in}{3.263856in}}{\pgfqpoint{2.817342in}{3.271670in}}%
\pgfpathcurveto{\pgfqpoint{2.809528in}{3.279483in}}{\pgfqpoint{2.798929in}{3.283874in}}{\pgfqpoint{2.787879in}{3.283874in}}%
\pgfpathcurveto{\pgfqpoint{2.776829in}{3.283874in}}{\pgfqpoint{2.766230in}{3.279483in}}{\pgfqpoint{2.758416in}{3.271670in}}%
\pgfpathcurveto{\pgfqpoint{2.750603in}{3.263856in}}{\pgfqpoint{2.746213in}{3.253257in}}{\pgfqpoint{2.746213in}{3.242207in}}%
\pgfpathcurveto{\pgfqpoint{2.746213in}{3.231157in}}{\pgfqpoint{2.750603in}{3.220558in}}{\pgfqpoint{2.758416in}{3.212744in}}%
\pgfpathcurveto{\pgfqpoint{2.766230in}{3.204931in}}{\pgfqpoint{2.776829in}{3.200540in}}{\pgfqpoint{2.787879in}{3.200540in}}%
\pgfpathclose%
\pgfusepath{stroke,fill}%
\end{pgfscope}%
\begin{pgfscope}%
\pgfpathrectangle{\pgfqpoint{0.648703in}{0.548769in}}{\pgfqpoint{5.112893in}{3.102590in}}%
\pgfusepath{clip}%
\pgfsetbuttcap%
\pgfsetroundjoin%
\definecolor{currentfill}{rgb}{1.000000,0.498039,0.054902}%
\pgfsetfillcolor{currentfill}%
\pgfsetlinewidth{1.003750pt}%
\definecolor{currentstroke}{rgb}{1.000000,0.498039,0.054902}%
\pgfsetstrokecolor{currentstroke}%
\pgfsetdash{}{0pt}%
\pgfpathmoveto{\pgfqpoint{2.087159in}{3.237665in}}%
\pgfpathcurveto{\pgfqpoint{2.098209in}{3.237665in}}{\pgfqpoint{2.108808in}{3.242056in}}{\pgfqpoint{2.116622in}{3.249869in}}%
\pgfpathcurveto{\pgfqpoint{2.124436in}{3.257683in}}{\pgfqpoint{2.128826in}{3.268282in}}{\pgfqpoint{2.128826in}{3.279332in}}%
\pgfpathcurveto{\pgfqpoint{2.128826in}{3.290382in}}{\pgfqpoint{2.124436in}{3.300981in}}{\pgfqpoint{2.116622in}{3.308795in}}%
\pgfpathcurveto{\pgfqpoint{2.108808in}{3.316608in}}{\pgfqpoint{2.098209in}{3.320999in}}{\pgfqpoint{2.087159in}{3.320999in}}%
\pgfpathcurveto{\pgfqpoint{2.076109in}{3.320999in}}{\pgfqpoint{2.065510in}{3.316608in}}{\pgfqpoint{2.057696in}{3.308795in}}%
\pgfpathcurveto{\pgfqpoint{2.049883in}{3.300981in}}{\pgfqpoint{2.045493in}{3.290382in}}{\pgfqpoint{2.045493in}{3.279332in}}%
\pgfpathcurveto{\pgfqpoint{2.045493in}{3.268282in}}{\pgfqpoint{2.049883in}{3.257683in}}{\pgfqpoint{2.057696in}{3.249869in}}%
\pgfpathcurveto{\pgfqpoint{2.065510in}{3.242056in}}{\pgfqpoint{2.076109in}{3.237665in}}{\pgfqpoint{2.087159in}{3.237665in}}%
\pgfpathclose%
\pgfusepath{stroke,fill}%
\end{pgfscope}%
\begin{pgfscope}%
\pgfpathrectangle{\pgfqpoint{0.648703in}{0.548769in}}{\pgfqpoint{5.112893in}{3.102590in}}%
\pgfusepath{clip}%
\pgfsetbuttcap%
\pgfsetroundjoin%
\definecolor{currentfill}{rgb}{0.121569,0.466667,0.705882}%
\pgfsetfillcolor{currentfill}%
\pgfsetlinewidth{1.003750pt}%
\definecolor{currentstroke}{rgb}{0.121569,0.466667,0.705882}%
\pgfsetstrokecolor{currentstroke}%
\pgfsetdash{}{0pt}%
\pgfpathmoveto{\pgfqpoint{0.834317in}{0.824543in}}%
\pgfpathcurveto{\pgfqpoint{0.845367in}{0.824543in}}{\pgfqpoint{0.855966in}{0.828933in}}{\pgfqpoint{0.863780in}{0.836747in}}%
\pgfpathcurveto{\pgfqpoint{0.871593in}{0.844560in}}{\pgfqpoint{0.875983in}{0.855159in}}{\pgfqpoint{0.875983in}{0.866210in}}%
\pgfpathcurveto{\pgfqpoint{0.875983in}{0.877260in}}{\pgfqpoint{0.871593in}{0.887859in}}{\pgfqpoint{0.863780in}{0.895672in}}%
\pgfpathcurveto{\pgfqpoint{0.855966in}{0.903486in}}{\pgfqpoint{0.845367in}{0.907876in}}{\pgfqpoint{0.834317in}{0.907876in}}%
\pgfpathcurveto{\pgfqpoint{0.823267in}{0.907876in}}{\pgfqpoint{0.812668in}{0.903486in}}{\pgfqpoint{0.804854in}{0.895672in}}%
\pgfpathcurveto{\pgfqpoint{0.797040in}{0.887859in}}{\pgfqpoint{0.792650in}{0.877260in}}{\pgfqpoint{0.792650in}{0.866210in}}%
\pgfpathcurveto{\pgfqpoint{0.792650in}{0.855159in}}{\pgfqpoint{0.797040in}{0.844560in}}{\pgfqpoint{0.804854in}{0.836747in}}%
\pgfpathcurveto{\pgfqpoint{0.812668in}{0.828933in}}{\pgfqpoint{0.823267in}{0.824543in}}{\pgfqpoint{0.834317in}{0.824543in}}%
\pgfpathclose%
\pgfusepath{stroke,fill}%
\end{pgfscope}%
\begin{pgfscope}%
\pgfpathrectangle{\pgfqpoint{0.648703in}{0.548769in}}{\pgfqpoint{5.112893in}{3.102590in}}%
\pgfusepath{clip}%
\pgfsetbuttcap%
\pgfsetroundjoin%
\definecolor{currentfill}{rgb}{1.000000,0.498039,0.054902}%
\pgfsetfillcolor{currentfill}%
\pgfsetlinewidth{1.003750pt}%
\definecolor{currentstroke}{rgb}{1.000000,0.498039,0.054902}%
\pgfsetstrokecolor{currentstroke}%
\pgfsetdash{}{0pt}%
\pgfpathmoveto{\pgfqpoint{1.072379in}{3.192290in}}%
\pgfpathcurveto{\pgfqpoint{1.083429in}{3.192290in}}{\pgfqpoint{1.094028in}{3.196681in}}{\pgfqpoint{1.101842in}{3.204494in}}%
\pgfpathcurveto{\pgfqpoint{1.109655in}{3.212308in}}{\pgfqpoint{1.114046in}{3.222907in}}{\pgfqpoint{1.114046in}{3.233957in}}%
\pgfpathcurveto{\pgfqpoint{1.114046in}{3.245007in}}{\pgfqpoint{1.109655in}{3.255606in}}{\pgfqpoint{1.101842in}{3.263420in}}%
\pgfpathcurveto{\pgfqpoint{1.094028in}{3.271234in}}{\pgfqpoint{1.083429in}{3.275624in}}{\pgfqpoint{1.072379in}{3.275624in}}%
\pgfpathcurveto{\pgfqpoint{1.061329in}{3.275624in}}{\pgfqpoint{1.050730in}{3.271234in}}{\pgfqpoint{1.042916in}{3.263420in}}%
\pgfpathcurveto{\pgfqpoint{1.035103in}{3.255606in}}{\pgfqpoint{1.030712in}{3.245007in}}{\pgfqpoint{1.030712in}{3.233957in}}%
\pgfpathcurveto{\pgfqpoint{1.030712in}{3.222907in}}{\pgfqpoint{1.035103in}{3.212308in}}{\pgfqpoint{1.042916in}{3.204494in}}%
\pgfpathcurveto{\pgfqpoint{1.050730in}{3.196681in}}{\pgfqpoint{1.061329in}{3.192290in}}{\pgfqpoint{1.072379in}{3.192290in}}%
\pgfpathclose%
\pgfusepath{stroke,fill}%
\end{pgfscope}%
\begin{pgfscope}%
\pgfpathrectangle{\pgfqpoint{0.648703in}{0.548769in}}{\pgfqpoint{5.112893in}{3.102590in}}%
\pgfusepath{clip}%
\pgfsetbuttcap%
\pgfsetroundjoin%
\definecolor{currentfill}{rgb}{1.000000,0.498039,0.054902}%
\pgfsetfillcolor{currentfill}%
\pgfsetlinewidth{1.003750pt}%
\definecolor{currentstroke}{rgb}{1.000000,0.498039,0.054902}%
\pgfsetstrokecolor{currentstroke}%
\pgfsetdash{}{0pt}%
\pgfpathmoveto{\pgfqpoint{1.780793in}{3.208790in}}%
\pgfpathcurveto{\pgfqpoint{1.791844in}{3.208790in}}{\pgfqpoint{1.802443in}{3.213181in}}{\pgfqpoint{1.810256in}{3.220994in}}%
\pgfpathcurveto{\pgfqpoint{1.818070in}{3.228808in}}{\pgfqpoint{1.822460in}{3.239407in}}{\pgfqpoint{1.822460in}{3.250457in}}%
\pgfpathcurveto{\pgfqpoint{1.822460in}{3.261507in}}{\pgfqpoint{1.818070in}{3.272106in}}{\pgfqpoint{1.810256in}{3.279920in}}%
\pgfpathcurveto{\pgfqpoint{1.802443in}{3.287733in}}{\pgfqpoint{1.791844in}{3.292124in}}{\pgfqpoint{1.780793in}{3.292124in}}%
\pgfpathcurveto{\pgfqpoint{1.769743in}{3.292124in}}{\pgfqpoint{1.759144in}{3.287733in}}{\pgfqpoint{1.751331in}{3.279920in}}%
\pgfpathcurveto{\pgfqpoint{1.743517in}{3.272106in}}{\pgfqpoint{1.739127in}{3.261507in}}{\pgfqpoint{1.739127in}{3.250457in}}%
\pgfpathcurveto{\pgfqpoint{1.739127in}{3.239407in}}{\pgfqpoint{1.743517in}{3.228808in}}{\pgfqpoint{1.751331in}{3.220994in}}%
\pgfpathcurveto{\pgfqpoint{1.759144in}{3.213181in}}{\pgfqpoint{1.769743in}{3.208790in}}{\pgfqpoint{1.780793in}{3.208790in}}%
\pgfpathclose%
\pgfusepath{stroke,fill}%
\end{pgfscope}%
\begin{pgfscope}%
\pgfpathrectangle{\pgfqpoint{0.648703in}{0.548769in}}{\pgfqpoint{5.112893in}{3.102590in}}%
\pgfusepath{clip}%
\pgfsetbuttcap%
\pgfsetroundjoin%
\definecolor{currentfill}{rgb}{0.121569,0.466667,0.705882}%
\pgfsetfillcolor{currentfill}%
\pgfsetlinewidth{1.003750pt}%
\definecolor{currentstroke}{rgb}{0.121569,0.466667,0.705882}%
\pgfsetstrokecolor{currentstroke}%
\pgfsetdash{}{0pt}%
\pgfpathmoveto{\pgfqpoint{0.767691in}{0.717293in}}%
\pgfpathcurveto{\pgfqpoint{0.778741in}{0.717293in}}{\pgfqpoint{0.789340in}{0.721683in}}{\pgfqpoint{0.797154in}{0.729497in}}%
\pgfpathcurveto{\pgfqpoint{0.804968in}{0.737311in}}{\pgfqpoint{0.809358in}{0.747910in}}{\pgfqpoint{0.809358in}{0.758960in}}%
\pgfpathcurveto{\pgfqpoint{0.809358in}{0.770010in}}{\pgfqpoint{0.804968in}{0.780609in}}{\pgfqpoint{0.797154in}{0.788423in}}%
\pgfpathcurveto{\pgfqpoint{0.789340in}{0.796236in}}{\pgfqpoint{0.778741in}{0.800626in}}{\pgfqpoint{0.767691in}{0.800626in}}%
\pgfpathcurveto{\pgfqpoint{0.756641in}{0.800626in}}{\pgfqpoint{0.746042in}{0.796236in}}{\pgfqpoint{0.738228in}{0.788423in}}%
\pgfpathcurveto{\pgfqpoint{0.730415in}{0.780609in}}{\pgfqpoint{0.726024in}{0.770010in}}{\pgfqpoint{0.726024in}{0.758960in}}%
\pgfpathcurveto{\pgfqpoint{0.726024in}{0.747910in}}{\pgfqpoint{0.730415in}{0.737311in}}{\pgfqpoint{0.738228in}{0.729497in}}%
\pgfpathcurveto{\pgfqpoint{0.746042in}{0.721683in}}{\pgfqpoint{0.756641in}{0.717293in}}{\pgfqpoint{0.767691in}{0.717293in}}%
\pgfpathclose%
\pgfusepath{stroke,fill}%
\end{pgfscope}%
\begin{pgfscope}%
\pgfpathrectangle{\pgfqpoint{0.648703in}{0.548769in}}{\pgfqpoint{5.112893in}{3.102590in}}%
\pgfusepath{clip}%
\pgfsetbuttcap%
\pgfsetroundjoin%
\definecolor{currentfill}{rgb}{1.000000,0.498039,0.054902}%
\pgfsetfillcolor{currentfill}%
\pgfsetlinewidth{1.003750pt}%
\definecolor{currentstroke}{rgb}{1.000000,0.498039,0.054902}%
\pgfsetstrokecolor{currentstroke}%
\pgfsetdash{}{0pt}%
\pgfpathmoveto{\pgfqpoint{1.953918in}{3.212915in}}%
\pgfpathcurveto{\pgfqpoint{1.964969in}{3.212915in}}{\pgfqpoint{1.975568in}{3.217306in}}{\pgfqpoint{1.983381in}{3.225119in}}%
\pgfpathcurveto{\pgfqpoint{1.991195in}{3.232933in}}{\pgfqpoint{1.995585in}{3.243532in}}{\pgfqpoint{1.995585in}{3.254582in}}%
\pgfpathcurveto{\pgfqpoint{1.995585in}{3.265632in}}{\pgfqpoint{1.991195in}{3.276231in}}{\pgfqpoint{1.983381in}{3.284045in}}%
\pgfpathcurveto{\pgfqpoint{1.975568in}{3.291858in}}{\pgfqpoint{1.964969in}{3.296249in}}{\pgfqpoint{1.953918in}{3.296249in}}%
\pgfpathcurveto{\pgfqpoint{1.942868in}{3.296249in}}{\pgfqpoint{1.932269in}{3.291858in}}{\pgfqpoint{1.924456in}{3.284045in}}%
\pgfpathcurveto{\pgfqpoint{1.916642in}{3.276231in}}{\pgfqpoint{1.912252in}{3.265632in}}{\pgfqpoint{1.912252in}{3.254582in}}%
\pgfpathcurveto{\pgfqpoint{1.912252in}{3.243532in}}{\pgfqpoint{1.916642in}{3.232933in}}{\pgfqpoint{1.924456in}{3.225119in}}%
\pgfpathcurveto{\pgfqpoint{1.932269in}{3.217306in}}{\pgfqpoint{1.942868in}{3.212915in}}{\pgfqpoint{1.953918in}{3.212915in}}%
\pgfpathclose%
\pgfusepath{stroke,fill}%
\end{pgfscope}%
\begin{pgfscope}%
\pgfpathrectangle{\pgfqpoint{0.648703in}{0.548769in}}{\pgfqpoint{5.112893in}{3.102590in}}%
\pgfusepath{clip}%
\pgfsetbuttcap%
\pgfsetroundjoin%
\definecolor{currentfill}{rgb}{0.121569,0.466667,0.705882}%
\pgfsetfillcolor{currentfill}%
\pgfsetlinewidth{1.003750pt}%
\definecolor{currentstroke}{rgb}{0.121569,0.466667,0.705882}%
\pgfsetstrokecolor{currentstroke}%
\pgfsetdash{}{0pt}%
\pgfpathmoveto{\pgfqpoint{0.767688in}{0.717293in}}%
\pgfpathcurveto{\pgfqpoint{0.778738in}{0.717293in}}{\pgfqpoint{0.789337in}{0.721683in}}{\pgfqpoint{0.797151in}{0.729497in}}%
\pgfpathcurveto{\pgfqpoint{0.804964in}{0.737311in}}{\pgfqpoint{0.809355in}{0.747910in}}{\pgfqpoint{0.809355in}{0.758960in}}%
\pgfpathcurveto{\pgfqpoint{0.809355in}{0.770010in}}{\pgfqpoint{0.804964in}{0.780609in}}{\pgfqpoint{0.797151in}{0.788423in}}%
\pgfpathcurveto{\pgfqpoint{0.789337in}{0.796236in}}{\pgfqpoint{0.778738in}{0.800626in}}{\pgfqpoint{0.767688in}{0.800626in}}%
\pgfpathcurveto{\pgfqpoint{0.756638in}{0.800626in}}{\pgfqpoint{0.746039in}{0.796236in}}{\pgfqpoint{0.738225in}{0.788423in}}%
\pgfpathcurveto{\pgfqpoint{0.730412in}{0.780609in}}{\pgfqpoint{0.726021in}{0.770010in}}{\pgfqpoint{0.726021in}{0.758960in}}%
\pgfpathcurveto{\pgfqpoint{0.726021in}{0.747910in}}{\pgfqpoint{0.730412in}{0.737311in}}{\pgfqpoint{0.738225in}{0.729497in}}%
\pgfpathcurveto{\pgfqpoint{0.746039in}{0.721683in}}{\pgfqpoint{0.756638in}{0.717293in}}{\pgfqpoint{0.767688in}{0.717293in}}%
\pgfpathclose%
\pgfusepath{stroke,fill}%
\end{pgfscope}%
\begin{pgfscope}%
\pgfpathrectangle{\pgfqpoint{0.648703in}{0.548769in}}{\pgfqpoint{5.112893in}{3.102590in}}%
\pgfusepath{clip}%
\pgfsetbuttcap%
\pgfsetroundjoin%
\definecolor{currentfill}{rgb}{0.121569,0.466667,0.705882}%
\pgfsetfillcolor{currentfill}%
\pgfsetlinewidth{1.003750pt}%
\definecolor{currentstroke}{rgb}{0.121569,0.466667,0.705882}%
\pgfsetstrokecolor{currentstroke}%
\pgfsetdash{}{0pt}%
\pgfpathmoveto{\pgfqpoint{0.767690in}{0.717293in}}%
\pgfpathcurveto{\pgfqpoint{0.778740in}{0.717293in}}{\pgfqpoint{0.789339in}{0.721683in}}{\pgfqpoint{0.797153in}{0.729497in}}%
\pgfpathcurveto{\pgfqpoint{0.804967in}{0.737311in}}{\pgfqpoint{0.809357in}{0.747910in}}{\pgfqpoint{0.809357in}{0.758960in}}%
\pgfpathcurveto{\pgfqpoint{0.809357in}{0.770010in}}{\pgfqpoint{0.804967in}{0.780609in}}{\pgfqpoint{0.797153in}{0.788423in}}%
\pgfpathcurveto{\pgfqpoint{0.789339in}{0.796236in}}{\pgfqpoint{0.778740in}{0.800626in}}{\pgfqpoint{0.767690in}{0.800626in}}%
\pgfpathcurveto{\pgfqpoint{0.756640in}{0.800626in}}{\pgfqpoint{0.746041in}{0.796236in}}{\pgfqpoint{0.738227in}{0.788423in}}%
\pgfpathcurveto{\pgfqpoint{0.730414in}{0.780609in}}{\pgfqpoint{0.726023in}{0.770010in}}{\pgfqpoint{0.726023in}{0.758960in}}%
\pgfpathcurveto{\pgfqpoint{0.726023in}{0.747910in}}{\pgfqpoint{0.730414in}{0.737311in}}{\pgfqpoint{0.738227in}{0.729497in}}%
\pgfpathcurveto{\pgfqpoint{0.746041in}{0.721683in}}{\pgfqpoint{0.756640in}{0.717293in}}{\pgfqpoint{0.767690in}{0.717293in}}%
\pgfpathclose%
\pgfusepath{stroke,fill}%
\end{pgfscope}%
\begin{pgfscope}%
\pgfpathrectangle{\pgfqpoint{0.648703in}{0.548769in}}{\pgfqpoint{5.112893in}{3.102590in}}%
\pgfusepath{clip}%
\pgfsetbuttcap%
\pgfsetroundjoin%
\definecolor{currentfill}{rgb}{0.121569,0.466667,0.705882}%
\pgfsetfillcolor{currentfill}%
\pgfsetlinewidth{1.003750pt}%
\definecolor{currentstroke}{rgb}{0.121569,0.466667,0.705882}%
\pgfsetstrokecolor{currentstroke}%
\pgfsetdash{}{0pt}%
\pgfpathmoveto{\pgfqpoint{0.849513in}{0.882293in}}%
\pgfpathcurveto{\pgfqpoint{0.860563in}{0.882293in}}{\pgfqpoint{0.871162in}{0.886683in}}{\pgfqpoint{0.878975in}{0.894497in}}%
\pgfpathcurveto{\pgfqpoint{0.886789in}{0.902310in}}{\pgfqpoint{0.891179in}{0.912909in}}{\pgfqpoint{0.891179in}{0.923960in}}%
\pgfpathcurveto{\pgfqpoint{0.891179in}{0.935010in}}{\pgfqpoint{0.886789in}{0.945609in}}{\pgfqpoint{0.878975in}{0.953422in}}%
\pgfpathcurveto{\pgfqpoint{0.871162in}{0.961236in}}{\pgfqpoint{0.860563in}{0.965626in}}{\pgfqpoint{0.849513in}{0.965626in}}%
\pgfpathcurveto{\pgfqpoint{0.838463in}{0.965626in}}{\pgfqpoint{0.827864in}{0.961236in}}{\pgfqpoint{0.820050in}{0.953422in}}%
\pgfpathcurveto{\pgfqpoint{0.812236in}{0.945609in}}{\pgfqpoint{0.807846in}{0.935010in}}{\pgfqpoint{0.807846in}{0.923960in}}%
\pgfpathcurveto{\pgfqpoint{0.807846in}{0.912909in}}{\pgfqpoint{0.812236in}{0.902310in}}{\pgfqpoint{0.820050in}{0.894497in}}%
\pgfpathcurveto{\pgfqpoint{0.827864in}{0.886683in}}{\pgfqpoint{0.838463in}{0.882293in}}{\pgfqpoint{0.849513in}{0.882293in}}%
\pgfpathclose%
\pgfusepath{stroke,fill}%
\end{pgfscope}%
\begin{pgfscope}%
\pgfpathrectangle{\pgfqpoint{0.648703in}{0.548769in}}{\pgfqpoint{5.112893in}{3.102590in}}%
\pgfusepath{clip}%
\pgfsetbuttcap%
\pgfsetroundjoin%
\definecolor{currentfill}{rgb}{1.000000,0.498039,0.054902}%
\pgfsetfillcolor{currentfill}%
\pgfsetlinewidth{1.003750pt}%
\definecolor{currentstroke}{rgb}{1.000000,0.498039,0.054902}%
\pgfsetstrokecolor{currentstroke}%
\pgfsetdash{}{0pt}%
\pgfpathmoveto{\pgfqpoint{1.387980in}{3.196415in}}%
\pgfpathcurveto{\pgfqpoint{1.399030in}{3.196415in}}{\pgfqpoint{1.409629in}{3.200806in}}{\pgfqpoint{1.417443in}{3.208619in}}%
\pgfpathcurveto{\pgfqpoint{1.425256in}{3.216433in}}{\pgfqpoint{1.429647in}{3.227032in}}{\pgfqpoint{1.429647in}{3.238082in}}%
\pgfpathcurveto{\pgfqpoint{1.429647in}{3.249132in}}{\pgfqpoint{1.425256in}{3.259731in}}{\pgfqpoint{1.417443in}{3.267545in}}%
\pgfpathcurveto{\pgfqpoint{1.409629in}{3.275358in}}{\pgfqpoint{1.399030in}{3.279749in}}{\pgfqpoint{1.387980in}{3.279749in}}%
\pgfpathcurveto{\pgfqpoint{1.376930in}{3.279749in}}{\pgfqpoint{1.366331in}{3.275358in}}{\pgfqpoint{1.358517in}{3.267545in}}%
\pgfpathcurveto{\pgfqpoint{1.350704in}{3.259731in}}{\pgfqpoint{1.346313in}{3.249132in}}{\pgfqpoint{1.346313in}{3.238082in}}%
\pgfpathcurveto{\pgfqpoint{1.346313in}{3.227032in}}{\pgfqpoint{1.350704in}{3.216433in}}{\pgfqpoint{1.358517in}{3.208619in}}%
\pgfpathcurveto{\pgfqpoint{1.366331in}{3.200806in}}{\pgfqpoint{1.376930in}{3.196415in}}{\pgfqpoint{1.387980in}{3.196415in}}%
\pgfpathclose%
\pgfusepath{stroke,fill}%
\end{pgfscope}%
\begin{pgfscope}%
\pgfpathrectangle{\pgfqpoint{0.648703in}{0.548769in}}{\pgfqpoint{5.112893in}{3.102590in}}%
\pgfusepath{clip}%
\pgfsetbuttcap%
\pgfsetroundjoin%
\definecolor{currentfill}{rgb}{1.000000,0.498039,0.054902}%
\pgfsetfillcolor{currentfill}%
\pgfsetlinewidth{1.003750pt}%
\definecolor{currentstroke}{rgb}{1.000000,0.498039,0.054902}%
\pgfsetstrokecolor{currentstroke}%
\pgfsetdash{}{0pt}%
\pgfpathmoveto{\pgfqpoint{1.045860in}{3.221165in}}%
\pgfpathcurveto{\pgfqpoint{1.056910in}{3.221165in}}{\pgfqpoint{1.067509in}{3.225556in}}{\pgfqpoint{1.075322in}{3.233369in}}%
\pgfpathcurveto{\pgfqpoint{1.083136in}{3.241183in}}{\pgfqpoint{1.087526in}{3.251782in}}{\pgfqpoint{1.087526in}{3.262832in}}%
\pgfpathcurveto{\pgfqpoint{1.087526in}{3.273882in}}{\pgfqpoint{1.083136in}{3.284481in}}{\pgfqpoint{1.075322in}{3.292295in}}%
\pgfpathcurveto{\pgfqpoint{1.067509in}{3.300108in}}{\pgfqpoint{1.056910in}{3.304499in}}{\pgfqpoint{1.045860in}{3.304499in}}%
\pgfpathcurveto{\pgfqpoint{1.034809in}{3.304499in}}{\pgfqpoint{1.024210in}{3.300108in}}{\pgfqpoint{1.016397in}{3.292295in}}%
\pgfpathcurveto{\pgfqpoint{1.008583in}{3.284481in}}{\pgfqpoint{1.004193in}{3.273882in}}{\pgfqpoint{1.004193in}{3.262832in}}%
\pgfpathcurveto{\pgfqpoint{1.004193in}{3.251782in}}{\pgfqpoint{1.008583in}{3.241183in}}{\pgfqpoint{1.016397in}{3.233369in}}%
\pgfpathcurveto{\pgfqpoint{1.024210in}{3.225556in}}{\pgfqpoint{1.034809in}{3.221165in}}{\pgfqpoint{1.045860in}{3.221165in}}%
\pgfpathclose%
\pgfusepath{stroke,fill}%
\end{pgfscope}%
\begin{pgfscope}%
\pgfpathrectangle{\pgfqpoint{0.648703in}{0.548769in}}{\pgfqpoint{5.112893in}{3.102590in}}%
\pgfusepath{clip}%
\pgfsetbuttcap%
\pgfsetroundjoin%
\definecolor{currentfill}{rgb}{0.121569,0.466667,0.705882}%
\pgfsetfillcolor{currentfill}%
\pgfsetlinewidth{1.003750pt}%
\definecolor{currentstroke}{rgb}{0.121569,0.466667,0.705882}%
\pgfsetstrokecolor{currentstroke}%
\pgfsetdash{}{0pt}%
\pgfpathmoveto{\pgfqpoint{0.767683in}{0.717293in}}%
\pgfpathcurveto{\pgfqpoint{0.778734in}{0.717293in}}{\pgfqpoint{0.789333in}{0.721683in}}{\pgfqpoint{0.797146in}{0.729497in}}%
\pgfpathcurveto{\pgfqpoint{0.804960in}{0.737311in}}{\pgfqpoint{0.809350in}{0.747910in}}{\pgfqpoint{0.809350in}{0.758960in}}%
\pgfpathcurveto{\pgfqpoint{0.809350in}{0.770010in}}{\pgfqpoint{0.804960in}{0.780609in}}{\pgfqpoint{0.797146in}{0.788423in}}%
\pgfpathcurveto{\pgfqpoint{0.789333in}{0.796236in}}{\pgfqpoint{0.778734in}{0.800626in}}{\pgfqpoint{0.767683in}{0.800626in}}%
\pgfpathcurveto{\pgfqpoint{0.756633in}{0.800626in}}{\pgfqpoint{0.746034in}{0.796236in}}{\pgfqpoint{0.738221in}{0.788423in}}%
\pgfpathcurveto{\pgfqpoint{0.730407in}{0.780609in}}{\pgfqpoint{0.726017in}{0.770010in}}{\pgfqpoint{0.726017in}{0.758960in}}%
\pgfpathcurveto{\pgfqpoint{0.726017in}{0.747910in}}{\pgfqpoint{0.730407in}{0.737311in}}{\pgfqpoint{0.738221in}{0.729497in}}%
\pgfpathcurveto{\pgfqpoint{0.746034in}{0.721683in}}{\pgfqpoint{0.756633in}{0.717293in}}{\pgfqpoint{0.767683in}{0.717293in}}%
\pgfpathclose%
\pgfusepath{stroke,fill}%
\end{pgfscope}%
\begin{pgfscope}%
\pgfpathrectangle{\pgfqpoint{0.648703in}{0.548769in}}{\pgfqpoint{5.112893in}{3.102590in}}%
\pgfusepath{clip}%
\pgfsetbuttcap%
\pgfsetroundjoin%
\definecolor{currentfill}{rgb}{0.121569,0.466667,0.705882}%
\pgfsetfillcolor{currentfill}%
\pgfsetlinewidth{1.003750pt}%
\definecolor{currentstroke}{rgb}{0.121569,0.466667,0.705882}%
\pgfsetstrokecolor{currentstroke}%
\pgfsetdash{}{0pt}%
\pgfpathmoveto{\pgfqpoint{0.826549in}{0.824543in}}%
\pgfpathcurveto{\pgfqpoint{0.837599in}{0.824543in}}{\pgfqpoint{0.848198in}{0.828933in}}{\pgfqpoint{0.856012in}{0.836747in}}%
\pgfpathcurveto{\pgfqpoint{0.863825in}{0.844560in}}{\pgfqpoint{0.868216in}{0.855159in}}{\pgfqpoint{0.868216in}{0.866210in}}%
\pgfpathcurveto{\pgfqpoint{0.868216in}{0.877260in}}{\pgfqpoint{0.863825in}{0.887859in}}{\pgfqpoint{0.856012in}{0.895672in}}%
\pgfpathcurveto{\pgfqpoint{0.848198in}{0.903486in}}{\pgfqpoint{0.837599in}{0.907876in}}{\pgfqpoint{0.826549in}{0.907876in}}%
\pgfpathcurveto{\pgfqpoint{0.815499in}{0.907876in}}{\pgfqpoint{0.804900in}{0.903486in}}{\pgfqpoint{0.797086in}{0.895672in}}%
\pgfpathcurveto{\pgfqpoint{0.789273in}{0.887859in}}{\pgfqpoint{0.784882in}{0.877260in}}{\pgfqpoint{0.784882in}{0.866210in}}%
\pgfpathcurveto{\pgfqpoint{0.784882in}{0.855159in}}{\pgfqpoint{0.789273in}{0.844560in}}{\pgfqpoint{0.797086in}{0.836747in}}%
\pgfpathcurveto{\pgfqpoint{0.804900in}{0.828933in}}{\pgfqpoint{0.815499in}{0.824543in}}{\pgfqpoint{0.826549in}{0.824543in}}%
\pgfpathclose%
\pgfusepath{stroke,fill}%
\end{pgfscope}%
\begin{pgfscope}%
\pgfpathrectangle{\pgfqpoint{0.648703in}{0.548769in}}{\pgfqpoint{5.112893in}{3.102590in}}%
\pgfusepath{clip}%
\pgfsetbuttcap%
\pgfsetroundjoin%
\definecolor{currentfill}{rgb}{1.000000,0.498039,0.054902}%
\pgfsetfillcolor{currentfill}%
\pgfsetlinewidth{1.003750pt}%
\definecolor{currentstroke}{rgb}{1.000000,0.498039,0.054902}%
\pgfsetstrokecolor{currentstroke}%
\pgfsetdash{}{0pt}%
\pgfpathmoveto{\pgfqpoint{1.630663in}{3.200540in}}%
\pgfpathcurveto{\pgfqpoint{1.641713in}{3.200540in}}{\pgfqpoint{1.652312in}{3.204931in}}{\pgfqpoint{1.660126in}{3.212744in}}%
\pgfpathcurveto{\pgfqpoint{1.667939in}{3.220558in}}{\pgfqpoint{1.672329in}{3.231157in}}{\pgfqpoint{1.672329in}{3.242207in}}%
\pgfpathcurveto{\pgfqpoint{1.672329in}{3.253257in}}{\pgfqpoint{1.667939in}{3.263856in}}{\pgfqpoint{1.660126in}{3.271670in}}%
\pgfpathcurveto{\pgfqpoint{1.652312in}{3.279483in}}{\pgfqpoint{1.641713in}{3.283874in}}{\pgfqpoint{1.630663in}{3.283874in}}%
\pgfpathcurveto{\pgfqpoint{1.619613in}{3.283874in}}{\pgfqpoint{1.609014in}{3.279483in}}{\pgfqpoint{1.601200in}{3.271670in}}%
\pgfpathcurveto{\pgfqpoint{1.593386in}{3.263856in}}{\pgfqpoint{1.588996in}{3.253257in}}{\pgfqpoint{1.588996in}{3.242207in}}%
\pgfpathcurveto{\pgfqpoint{1.588996in}{3.231157in}}{\pgfqpoint{1.593386in}{3.220558in}}{\pgfqpoint{1.601200in}{3.212744in}}%
\pgfpathcurveto{\pgfqpoint{1.609014in}{3.204931in}}{\pgfqpoint{1.619613in}{3.200540in}}{\pgfqpoint{1.630663in}{3.200540in}}%
\pgfpathclose%
\pgfusepath{stroke,fill}%
\end{pgfscope}%
\begin{pgfscope}%
\pgfpathrectangle{\pgfqpoint{0.648703in}{0.548769in}}{\pgfqpoint{5.112893in}{3.102590in}}%
\pgfusepath{clip}%
\pgfsetbuttcap%
\pgfsetroundjoin%
\definecolor{currentfill}{rgb}{0.121569,0.466667,0.705882}%
\pgfsetfillcolor{currentfill}%
\pgfsetlinewidth{1.003750pt}%
\definecolor{currentstroke}{rgb}{0.121569,0.466667,0.705882}%
\pgfsetstrokecolor{currentstroke}%
\pgfsetdash{}{0pt}%
\pgfpathmoveto{\pgfqpoint{0.767673in}{0.717293in}}%
\pgfpathcurveto{\pgfqpoint{0.778723in}{0.717293in}}{\pgfqpoint{0.789322in}{0.721683in}}{\pgfqpoint{0.797136in}{0.729497in}}%
\pgfpathcurveto{\pgfqpoint{0.804949in}{0.737311in}}{\pgfqpoint{0.809340in}{0.747910in}}{\pgfqpoint{0.809340in}{0.758960in}}%
\pgfpathcurveto{\pgfqpoint{0.809340in}{0.770010in}}{\pgfqpoint{0.804949in}{0.780609in}}{\pgfqpoint{0.797136in}{0.788423in}}%
\pgfpathcurveto{\pgfqpoint{0.789322in}{0.796236in}}{\pgfqpoint{0.778723in}{0.800626in}}{\pgfqpoint{0.767673in}{0.800626in}}%
\pgfpathcurveto{\pgfqpoint{0.756623in}{0.800626in}}{\pgfqpoint{0.746024in}{0.796236in}}{\pgfqpoint{0.738210in}{0.788423in}}%
\pgfpathcurveto{\pgfqpoint{0.730396in}{0.780609in}}{\pgfqpoint{0.726006in}{0.770010in}}{\pgfqpoint{0.726006in}{0.758960in}}%
\pgfpathcurveto{\pgfqpoint{0.726006in}{0.747910in}}{\pgfqpoint{0.730396in}{0.737311in}}{\pgfqpoint{0.738210in}{0.729497in}}%
\pgfpathcurveto{\pgfqpoint{0.746024in}{0.721683in}}{\pgfqpoint{0.756623in}{0.717293in}}{\pgfqpoint{0.767673in}{0.717293in}}%
\pgfpathclose%
\pgfusepath{stroke,fill}%
\end{pgfscope}%
\begin{pgfscope}%
\pgfpathrectangle{\pgfqpoint{0.648703in}{0.548769in}}{\pgfqpoint{5.112893in}{3.102590in}}%
\pgfusepath{clip}%
\pgfsetbuttcap%
\pgfsetroundjoin%
\definecolor{currentfill}{rgb}{1.000000,0.498039,0.054902}%
\pgfsetfillcolor{currentfill}%
\pgfsetlinewidth{1.003750pt}%
\definecolor{currentstroke}{rgb}{1.000000,0.498039,0.054902}%
\pgfsetstrokecolor{currentstroke}%
\pgfsetdash{}{0pt}%
\pgfpathmoveto{\pgfqpoint{1.756416in}{3.204665in}}%
\pgfpathcurveto{\pgfqpoint{1.767467in}{3.204665in}}{\pgfqpoint{1.778066in}{3.209056in}}{\pgfqpoint{1.785879in}{3.216869in}}%
\pgfpathcurveto{\pgfqpoint{1.793693in}{3.224683in}}{\pgfqpoint{1.798083in}{3.235282in}}{\pgfqpoint{1.798083in}{3.246332in}}%
\pgfpathcurveto{\pgfqpoint{1.798083in}{3.257382in}}{\pgfqpoint{1.793693in}{3.267981in}}{\pgfqpoint{1.785879in}{3.275795in}}%
\pgfpathcurveto{\pgfqpoint{1.778066in}{3.283608in}}{\pgfqpoint{1.767467in}{3.287999in}}{\pgfqpoint{1.756416in}{3.287999in}}%
\pgfpathcurveto{\pgfqpoint{1.745366in}{3.287999in}}{\pgfqpoint{1.734767in}{3.283608in}}{\pgfqpoint{1.726954in}{3.275795in}}%
\pgfpathcurveto{\pgfqpoint{1.719140in}{3.267981in}}{\pgfqpoint{1.714750in}{3.257382in}}{\pgfqpoint{1.714750in}{3.246332in}}%
\pgfpathcurveto{\pgfqpoint{1.714750in}{3.235282in}}{\pgfqpoint{1.719140in}{3.224683in}}{\pgfqpoint{1.726954in}{3.216869in}}%
\pgfpathcurveto{\pgfqpoint{1.734767in}{3.209056in}}{\pgfqpoint{1.745366in}{3.204665in}}{\pgfqpoint{1.756416in}{3.204665in}}%
\pgfpathclose%
\pgfusepath{stroke,fill}%
\end{pgfscope}%
\begin{pgfscope}%
\pgfpathrectangle{\pgfqpoint{0.648703in}{0.548769in}}{\pgfqpoint{5.112893in}{3.102590in}}%
\pgfusepath{clip}%
\pgfsetbuttcap%
\pgfsetroundjoin%
\definecolor{currentfill}{rgb}{1.000000,0.498039,0.054902}%
\pgfsetfillcolor{currentfill}%
\pgfsetlinewidth{1.003750pt}%
\definecolor{currentstroke}{rgb}{1.000000,0.498039,0.054902}%
\pgfsetstrokecolor{currentstroke}%
\pgfsetdash{}{0pt}%
\pgfpathmoveto{\pgfqpoint{1.653849in}{3.200540in}}%
\pgfpathcurveto{\pgfqpoint{1.664899in}{3.200540in}}{\pgfqpoint{1.675498in}{3.204931in}}{\pgfqpoint{1.683312in}{3.212744in}}%
\pgfpathcurveto{\pgfqpoint{1.691125in}{3.220558in}}{\pgfqpoint{1.695515in}{3.231157in}}{\pgfqpoint{1.695515in}{3.242207in}}%
\pgfpathcurveto{\pgfqpoint{1.695515in}{3.253257in}}{\pgfqpoint{1.691125in}{3.263856in}}{\pgfqpoint{1.683312in}{3.271670in}}%
\pgfpathcurveto{\pgfqpoint{1.675498in}{3.279483in}}{\pgfqpoint{1.664899in}{3.283874in}}{\pgfqpoint{1.653849in}{3.283874in}}%
\pgfpathcurveto{\pgfqpoint{1.642799in}{3.283874in}}{\pgfqpoint{1.632200in}{3.279483in}}{\pgfqpoint{1.624386in}{3.271670in}}%
\pgfpathcurveto{\pgfqpoint{1.616572in}{3.263856in}}{\pgfqpoint{1.612182in}{3.253257in}}{\pgfqpoint{1.612182in}{3.242207in}}%
\pgfpathcurveto{\pgfqpoint{1.612182in}{3.231157in}}{\pgfqpoint{1.616572in}{3.220558in}}{\pgfqpoint{1.624386in}{3.212744in}}%
\pgfpathcurveto{\pgfqpoint{1.632200in}{3.204931in}}{\pgfqpoint{1.642799in}{3.200540in}}{\pgfqpoint{1.653849in}{3.200540in}}%
\pgfpathclose%
\pgfusepath{stroke,fill}%
\end{pgfscope}%
\begin{pgfscope}%
\pgfpathrectangle{\pgfqpoint{0.648703in}{0.548769in}}{\pgfqpoint{5.112893in}{3.102590in}}%
\pgfusepath{clip}%
\pgfsetbuttcap%
\pgfsetroundjoin%
\definecolor{currentfill}{rgb}{0.121569,0.466667,0.705882}%
\pgfsetfillcolor{currentfill}%
\pgfsetlinewidth{1.003750pt}%
\definecolor{currentstroke}{rgb}{0.121569,0.466667,0.705882}%
\pgfsetstrokecolor{currentstroke}%
\pgfsetdash{}{0pt}%
\pgfpathmoveto{\pgfqpoint{0.772206in}{0.721418in}}%
\pgfpathcurveto{\pgfqpoint{0.783256in}{0.721418in}}{\pgfqpoint{0.793855in}{0.725808in}}{\pgfqpoint{0.801669in}{0.733622in}}%
\pgfpathcurveto{\pgfqpoint{0.809482in}{0.741436in}}{\pgfqpoint{0.813873in}{0.752035in}}{\pgfqpoint{0.813873in}{0.763085in}}%
\pgfpathcurveto{\pgfqpoint{0.813873in}{0.774135in}}{\pgfqpoint{0.809482in}{0.784734in}}{\pgfqpoint{0.801669in}{0.792548in}}%
\pgfpathcurveto{\pgfqpoint{0.793855in}{0.800361in}}{\pgfqpoint{0.783256in}{0.804751in}}{\pgfqpoint{0.772206in}{0.804751in}}%
\pgfpathcurveto{\pgfqpoint{0.761156in}{0.804751in}}{\pgfqpoint{0.750557in}{0.800361in}}{\pgfqpoint{0.742743in}{0.792548in}}%
\pgfpathcurveto{\pgfqpoint{0.734930in}{0.784734in}}{\pgfqpoint{0.730539in}{0.774135in}}{\pgfqpoint{0.730539in}{0.763085in}}%
\pgfpathcurveto{\pgfqpoint{0.730539in}{0.752035in}}{\pgfqpoint{0.734930in}{0.741436in}}{\pgfqpoint{0.742743in}{0.733622in}}%
\pgfpathcurveto{\pgfqpoint{0.750557in}{0.725808in}}{\pgfqpoint{0.761156in}{0.721418in}}{\pgfqpoint{0.772206in}{0.721418in}}%
\pgfpathclose%
\pgfusepath{stroke,fill}%
\end{pgfscope}%
\begin{pgfscope}%
\pgfpathrectangle{\pgfqpoint{0.648703in}{0.548769in}}{\pgfqpoint{5.112893in}{3.102590in}}%
\pgfusepath{clip}%
\pgfsetbuttcap%
\pgfsetroundjoin%
\definecolor{currentfill}{rgb}{0.121569,0.466667,0.705882}%
\pgfsetfillcolor{currentfill}%
\pgfsetlinewidth{1.003750pt}%
\definecolor{currentstroke}{rgb}{0.121569,0.466667,0.705882}%
\pgfsetstrokecolor{currentstroke}%
\pgfsetdash{}{0pt}%
\pgfpathmoveto{\pgfqpoint{0.767708in}{0.717293in}}%
\pgfpathcurveto{\pgfqpoint{0.778758in}{0.717293in}}{\pgfqpoint{0.789357in}{0.721683in}}{\pgfqpoint{0.797170in}{0.729497in}}%
\pgfpathcurveto{\pgfqpoint{0.804984in}{0.737311in}}{\pgfqpoint{0.809374in}{0.747910in}}{\pgfqpoint{0.809374in}{0.758960in}}%
\pgfpathcurveto{\pgfqpoint{0.809374in}{0.770010in}}{\pgfqpoint{0.804984in}{0.780609in}}{\pgfqpoint{0.797170in}{0.788423in}}%
\pgfpathcurveto{\pgfqpoint{0.789357in}{0.796236in}}{\pgfqpoint{0.778758in}{0.800626in}}{\pgfqpoint{0.767708in}{0.800626in}}%
\pgfpathcurveto{\pgfqpoint{0.756658in}{0.800626in}}{\pgfqpoint{0.746058in}{0.796236in}}{\pgfqpoint{0.738245in}{0.788423in}}%
\pgfpathcurveto{\pgfqpoint{0.730431in}{0.780609in}}{\pgfqpoint{0.726041in}{0.770010in}}{\pgfqpoint{0.726041in}{0.758960in}}%
\pgfpathcurveto{\pgfqpoint{0.726041in}{0.747910in}}{\pgfqpoint{0.730431in}{0.737311in}}{\pgfqpoint{0.738245in}{0.729497in}}%
\pgfpathcurveto{\pgfqpoint{0.746058in}{0.721683in}}{\pgfqpoint{0.756658in}{0.717293in}}{\pgfqpoint{0.767708in}{0.717293in}}%
\pgfpathclose%
\pgfusepath{stroke,fill}%
\end{pgfscope}%
\begin{pgfscope}%
\pgfpathrectangle{\pgfqpoint{0.648703in}{0.548769in}}{\pgfqpoint{5.112893in}{3.102590in}}%
\pgfusepath{clip}%
\pgfsetbuttcap%
\pgfsetroundjoin%
\definecolor{currentfill}{rgb}{0.121569,0.466667,0.705882}%
\pgfsetfillcolor{currentfill}%
\pgfsetlinewidth{1.003750pt}%
\definecolor{currentstroke}{rgb}{0.121569,0.466667,0.705882}%
\pgfsetstrokecolor{currentstroke}%
\pgfsetdash{}{0pt}%
\pgfpathmoveto{\pgfqpoint{0.767702in}{0.717293in}}%
\pgfpathcurveto{\pgfqpoint{0.778752in}{0.717293in}}{\pgfqpoint{0.789351in}{0.721683in}}{\pgfqpoint{0.797165in}{0.729497in}}%
\pgfpathcurveto{\pgfqpoint{0.804979in}{0.737311in}}{\pgfqpoint{0.809369in}{0.747910in}}{\pgfqpoint{0.809369in}{0.758960in}}%
\pgfpathcurveto{\pgfqpoint{0.809369in}{0.770010in}}{\pgfqpoint{0.804979in}{0.780609in}}{\pgfqpoint{0.797165in}{0.788423in}}%
\pgfpathcurveto{\pgfqpoint{0.789351in}{0.796236in}}{\pgfqpoint{0.778752in}{0.800626in}}{\pgfqpoint{0.767702in}{0.800626in}}%
\pgfpathcurveto{\pgfqpoint{0.756652in}{0.800626in}}{\pgfqpoint{0.746053in}{0.796236in}}{\pgfqpoint{0.738239in}{0.788423in}}%
\pgfpathcurveto{\pgfqpoint{0.730426in}{0.780609in}}{\pgfqpoint{0.726036in}{0.770010in}}{\pgfqpoint{0.726036in}{0.758960in}}%
\pgfpathcurveto{\pgfqpoint{0.726036in}{0.747910in}}{\pgfqpoint{0.730426in}{0.737311in}}{\pgfqpoint{0.738239in}{0.729497in}}%
\pgfpathcurveto{\pgfqpoint{0.746053in}{0.721683in}}{\pgfqpoint{0.756652in}{0.717293in}}{\pgfqpoint{0.767702in}{0.717293in}}%
\pgfpathclose%
\pgfusepath{stroke,fill}%
\end{pgfscope}%
\begin{pgfscope}%
\pgfpathrectangle{\pgfqpoint{0.648703in}{0.548769in}}{\pgfqpoint{5.112893in}{3.102590in}}%
\pgfusepath{clip}%
\pgfsetbuttcap%
\pgfsetroundjoin%
\definecolor{currentfill}{rgb}{0.121569,0.466667,0.705882}%
\pgfsetfillcolor{currentfill}%
\pgfsetlinewidth{1.003750pt}%
\definecolor{currentstroke}{rgb}{0.121569,0.466667,0.705882}%
\pgfsetstrokecolor{currentstroke}%
\pgfsetdash{}{0pt}%
\pgfpathmoveto{\pgfqpoint{0.767914in}{0.717293in}}%
\pgfpathcurveto{\pgfqpoint{0.778965in}{0.717293in}}{\pgfqpoint{0.789564in}{0.721683in}}{\pgfqpoint{0.797377in}{0.729497in}}%
\pgfpathcurveto{\pgfqpoint{0.805191in}{0.737311in}}{\pgfqpoint{0.809581in}{0.747910in}}{\pgfqpoint{0.809581in}{0.758960in}}%
\pgfpathcurveto{\pgfqpoint{0.809581in}{0.770010in}}{\pgfqpoint{0.805191in}{0.780609in}}{\pgfqpoint{0.797377in}{0.788423in}}%
\pgfpathcurveto{\pgfqpoint{0.789564in}{0.796236in}}{\pgfqpoint{0.778965in}{0.800626in}}{\pgfqpoint{0.767914in}{0.800626in}}%
\pgfpathcurveto{\pgfqpoint{0.756864in}{0.800626in}}{\pgfqpoint{0.746265in}{0.796236in}}{\pgfqpoint{0.738452in}{0.788423in}}%
\pgfpathcurveto{\pgfqpoint{0.730638in}{0.780609in}}{\pgfqpoint{0.726248in}{0.770010in}}{\pgfqpoint{0.726248in}{0.758960in}}%
\pgfpathcurveto{\pgfqpoint{0.726248in}{0.747910in}}{\pgfqpoint{0.730638in}{0.737311in}}{\pgfqpoint{0.738452in}{0.729497in}}%
\pgfpathcurveto{\pgfqpoint{0.746265in}{0.721683in}}{\pgfqpoint{0.756864in}{0.717293in}}{\pgfqpoint{0.767914in}{0.717293in}}%
\pgfpathclose%
\pgfusepath{stroke,fill}%
\end{pgfscope}%
\begin{pgfscope}%
\pgfpathrectangle{\pgfqpoint{0.648703in}{0.548769in}}{\pgfqpoint{5.112893in}{3.102590in}}%
\pgfusepath{clip}%
\pgfsetbuttcap%
\pgfsetroundjoin%
\definecolor{currentfill}{rgb}{0.121569,0.466667,0.705882}%
\pgfsetfillcolor{currentfill}%
\pgfsetlinewidth{1.003750pt}%
\definecolor{currentstroke}{rgb}{0.121569,0.466667,0.705882}%
\pgfsetstrokecolor{currentstroke}%
\pgfsetdash{}{0pt}%
\pgfpathmoveto{\pgfqpoint{0.767766in}{0.717293in}}%
\pgfpathcurveto{\pgfqpoint{0.778816in}{0.717293in}}{\pgfqpoint{0.789415in}{0.721683in}}{\pgfqpoint{0.797228in}{0.729497in}}%
\pgfpathcurveto{\pgfqpoint{0.805042in}{0.737311in}}{\pgfqpoint{0.809432in}{0.747910in}}{\pgfqpoint{0.809432in}{0.758960in}}%
\pgfpathcurveto{\pgfqpoint{0.809432in}{0.770010in}}{\pgfqpoint{0.805042in}{0.780609in}}{\pgfqpoint{0.797228in}{0.788423in}}%
\pgfpathcurveto{\pgfqpoint{0.789415in}{0.796236in}}{\pgfqpoint{0.778816in}{0.800626in}}{\pgfqpoint{0.767766in}{0.800626in}}%
\pgfpathcurveto{\pgfqpoint{0.756716in}{0.800626in}}{\pgfqpoint{0.746117in}{0.796236in}}{\pgfqpoint{0.738303in}{0.788423in}}%
\pgfpathcurveto{\pgfqpoint{0.730489in}{0.780609in}}{\pgfqpoint{0.726099in}{0.770010in}}{\pgfqpoint{0.726099in}{0.758960in}}%
\pgfpathcurveto{\pgfqpoint{0.726099in}{0.747910in}}{\pgfqpoint{0.730489in}{0.737311in}}{\pgfqpoint{0.738303in}{0.729497in}}%
\pgfpathcurveto{\pgfqpoint{0.746117in}{0.721683in}}{\pgfqpoint{0.756716in}{0.717293in}}{\pgfqpoint{0.767766in}{0.717293in}}%
\pgfpathclose%
\pgfusepath{stroke,fill}%
\end{pgfscope}%
\begin{pgfscope}%
\pgfpathrectangle{\pgfqpoint{0.648703in}{0.548769in}}{\pgfqpoint{5.112893in}{3.102590in}}%
\pgfusepath{clip}%
\pgfsetbuttcap%
\pgfsetroundjoin%
\definecolor{currentfill}{rgb}{0.121569,0.466667,0.705882}%
\pgfsetfillcolor{currentfill}%
\pgfsetlinewidth{1.003750pt}%
\definecolor{currentstroke}{rgb}{0.121569,0.466667,0.705882}%
\pgfsetstrokecolor{currentstroke}%
\pgfsetdash{}{0pt}%
\pgfpathmoveto{\pgfqpoint{0.767691in}{0.717293in}}%
\pgfpathcurveto{\pgfqpoint{0.778741in}{0.717293in}}{\pgfqpoint{0.789340in}{0.721683in}}{\pgfqpoint{0.797154in}{0.729497in}}%
\pgfpathcurveto{\pgfqpoint{0.804968in}{0.737311in}}{\pgfqpoint{0.809358in}{0.747910in}}{\pgfqpoint{0.809358in}{0.758960in}}%
\pgfpathcurveto{\pgfqpoint{0.809358in}{0.770010in}}{\pgfqpoint{0.804968in}{0.780609in}}{\pgfqpoint{0.797154in}{0.788423in}}%
\pgfpathcurveto{\pgfqpoint{0.789340in}{0.796236in}}{\pgfqpoint{0.778741in}{0.800626in}}{\pgfqpoint{0.767691in}{0.800626in}}%
\pgfpathcurveto{\pgfqpoint{0.756641in}{0.800626in}}{\pgfqpoint{0.746042in}{0.796236in}}{\pgfqpoint{0.738228in}{0.788423in}}%
\pgfpathcurveto{\pgfqpoint{0.730415in}{0.780609in}}{\pgfqpoint{0.726024in}{0.770010in}}{\pgfqpoint{0.726024in}{0.758960in}}%
\pgfpathcurveto{\pgfqpoint{0.726024in}{0.747910in}}{\pgfqpoint{0.730415in}{0.737311in}}{\pgfqpoint{0.738228in}{0.729497in}}%
\pgfpathcurveto{\pgfqpoint{0.746042in}{0.721683in}}{\pgfqpoint{0.756641in}{0.717293in}}{\pgfqpoint{0.767691in}{0.717293in}}%
\pgfpathclose%
\pgfusepath{stroke,fill}%
\end{pgfscope}%
\begin{pgfscope}%
\pgfpathrectangle{\pgfqpoint{0.648703in}{0.548769in}}{\pgfqpoint{5.112893in}{3.102590in}}%
\pgfusepath{clip}%
\pgfsetbuttcap%
\pgfsetroundjoin%
\definecolor{currentfill}{rgb}{0.121569,0.466667,0.705882}%
\pgfsetfillcolor{currentfill}%
\pgfsetlinewidth{1.003750pt}%
\definecolor{currentstroke}{rgb}{0.121569,0.466667,0.705882}%
\pgfsetstrokecolor{currentstroke}%
\pgfsetdash{}{0pt}%
\pgfpathmoveto{\pgfqpoint{1.331548in}{3.188165in}}%
\pgfpathcurveto{\pgfqpoint{1.342599in}{3.188165in}}{\pgfqpoint{1.353198in}{3.192556in}}{\pgfqpoint{1.361011in}{3.200369in}}%
\pgfpathcurveto{\pgfqpoint{1.368825in}{3.208183in}}{\pgfqpoint{1.373215in}{3.218782in}}{\pgfqpoint{1.373215in}{3.229832in}}%
\pgfpathcurveto{\pgfqpoint{1.373215in}{3.240882in}}{\pgfqpoint{1.368825in}{3.251481in}}{\pgfqpoint{1.361011in}{3.259295in}}%
\pgfpathcurveto{\pgfqpoint{1.353198in}{3.267109in}}{\pgfqpoint{1.342599in}{3.271499in}}{\pgfqpoint{1.331548in}{3.271499in}}%
\pgfpathcurveto{\pgfqpoint{1.320498in}{3.271499in}}{\pgfqpoint{1.309899in}{3.267109in}}{\pgfqpoint{1.302086in}{3.259295in}}%
\pgfpathcurveto{\pgfqpoint{1.294272in}{3.251481in}}{\pgfqpoint{1.289882in}{3.240882in}}{\pgfqpoint{1.289882in}{3.229832in}}%
\pgfpathcurveto{\pgfqpoint{1.289882in}{3.218782in}}{\pgfqpoint{1.294272in}{3.208183in}}{\pgfqpoint{1.302086in}{3.200369in}}%
\pgfpathcurveto{\pgfqpoint{1.309899in}{3.192556in}}{\pgfqpoint{1.320498in}{3.188165in}}{\pgfqpoint{1.331548in}{3.188165in}}%
\pgfpathclose%
\pgfusepath{stroke,fill}%
\end{pgfscope}%
\begin{pgfscope}%
\pgfpathrectangle{\pgfqpoint{0.648703in}{0.548769in}}{\pgfqpoint{5.112893in}{3.102590in}}%
\pgfusepath{clip}%
\pgfsetbuttcap%
\pgfsetroundjoin%
\definecolor{currentfill}{rgb}{1.000000,0.498039,0.054902}%
\pgfsetfillcolor{currentfill}%
\pgfsetlinewidth{1.003750pt}%
\definecolor{currentstroke}{rgb}{1.000000,0.498039,0.054902}%
\pgfsetstrokecolor{currentstroke}%
\pgfsetdash{}{0pt}%
\pgfpathmoveto{\pgfqpoint{1.215438in}{3.200540in}}%
\pgfpathcurveto{\pgfqpoint{1.226488in}{3.200540in}}{\pgfqpoint{1.237087in}{3.204931in}}{\pgfqpoint{1.244901in}{3.212744in}}%
\pgfpathcurveto{\pgfqpoint{1.252714in}{3.220558in}}{\pgfqpoint{1.257105in}{3.231157in}}{\pgfqpoint{1.257105in}{3.242207in}}%
\pgfpathcurveto{\pgfqpoint{1.257105in}{3.253257in}}{\pgfqpoint{1.252714in}{3.263856in}}{\pgfqpoint{1.244901in}{3.271670in}}%
\pgfpathcurveto{\pgfqpoint{1.237087in}{3.279483in}}{\pgfqpoint{1.226488in}{3.283874in}}{\pgfqpoint{1.215438in}{3.283874in}}%
\pgfpathcurveto{\pgfqpoint{1.204388in}{3.283874in}}{\pgfqpoint{1.193789in}{3.279483in}}{\pgfqpoint{1.185975in}{3.271670in}}%
\pgfpathcurveto{\pgfqpoint{1.178162in}{3.263856in}}{\pgfqpoint{1.173771in}{3.253257in}}{\pgfqpoint{1.173771in}{3.242207in}}%
\pgfpathcurveto{\pgfqpoint{1.173771in}{3.231157in}}{\pgfqpoint{1.178162in}{3.220558in}}{\pgfqpoint{1.185975in}{3.212744in}}%
\pgfpathcurveto{\pgfqpoint{1.193789in}{3.204931in}}{\pgfqpoint{1.204388in}{3.200540in}}{\pgfqpoint{1.215438in}{3.200540in}}%
\pgfpathclose%
\pgfusepath{stroke,fill}%
\end{pgfscope}%
\begin{pgfscope}%
\pgfpathrectangle{\pgfqpoint{0.648703in}{0.548769in}}{\pgfqpoint{5.112893in}{3.102590in}}%
\pgfusepath{clip}%
\pgfsetbuttcap%
\pgfsetroundjoin%
\definecolor{currentfill}{rgb}{1.000000,0.498039,0.054902}%
\pgfsetfillcolor{currentfill}%
\pgfsetlinewidth{1.003750pt}%
\definecolor{currentstroke}{rgb}{1.000000,0.498039,0.054902}%
\pgfsetstrokecolor{currentstroke}%
\pgfsetdash{}{0pt}%
\pgfpathmoveto{\pgfqpoint{1.477135in}{3.200540in}}%
\pgfpathcurveto{\pgfqpoint{1.488185in}{3.200540in}}{\pgfqpoint{1.498784in}{3.204931in}}{\pgfqpoint{1.506598in}{3.212744in}}%
\pgfpathcurveto{\pgfqpoint{1.514411in}{3.220558in}}{\pgfqpoint{1.518801in}{3.231157in}}{\pgfqpoint{1.518801in}{3.242207in}}%
\pgfpathcurveto{\pgfqpoint{1.518801in}{3.253257in}}{\pgfqpoint{1.514411in}{3.263856in}}{\pgfqpoint{1.506598in}{3.271670in}}%
\pgfpathcurveto{\pgfqpoint{1.498784in}{3.279483in}}{\pgfqpoint{1.488185in}{3.283874in}}{\pgfqpoint{1.477135in}{3.283874in}}%
\pgfpathcurveto{\pgfqpoint{1.466085in}{3.283874in}}{\pgfqpoint{1.455486in}{3.279483in}}{\pgfqpoint{1.447672in}{3.271670in}}%
\pgfpathcurveto{\pgfqpoint{1.439858in}{3.263856in}}{\pgfqpoint{1.435468in}{3.253257in}}{\pgfqpoint{1.435468in}{3.242207in}}%
\pgfpathcurveto{\pgfqpoint{1.435468in}{3.231157in}}{\pgfqpoint{1.439858in}{3.220558in}}{\pgfqpoint{1.447672in}{3.212744in}}%
\pgfpathcurveto{\pgfqpoint{1.455486in}{3.204931in}}{\pgfqpoint{1.466085in}{3.200540in}}{\pgfqpoint{1.477135in}{3.200540in}}%
\pgfpathclose%
\pgfusepath{stroke,fill}%
\end{pgfscope}%
\begin{pgfscope}%
\pgfpathrectangle{\pgfqpoint{0.648703in}{0.548769in}}{\pgfqpoint{5.112893in}{3.102590in}}%
\pgfusepath{clip}%
\pgfsetbuttcap%
\pgfsetroundjoin%
\definecolor{currentfill}{rgb}{0.121569,0.466667,0.705882}%
\pgfsetfillcolor{currentfill}%
\pgfsetlinewidth{1.003750pt}%
\definecolor{currentstroke}{rgb}{0.121569,0.466667,0.705882}%
\pgfsetstrokecolor{currentstroke}%
\pgfsetdash{}{0pt}%
\pgfpathmoveto{\pgfqpoint{0.767680in}{0.717293in}}%
\pgfpathcurveto{\pgfqpoint{0.778730in}{0.717293in}}{\pgfqpoint{0.789329in}{0.721683in}}{\pgfqpoint{0.797143in}{0.729497in}}%
\pgfpathcurveto{\pgfqpoint{0.804956in}{0.737311in}}{\pgfqpoint{0.809346in}{0.747910in}}{\pgfqpoint{0.809346in}{0.758960in}}%
\pgfpathcurveto{\pgfqpoint{0.809346in}{0.770010in}}{\pgfqpoint{0.804956in}{0.780609in}}{\pgfqpoint{0.797143in}{0.788423in}}%
\pgfpathcurveto{\pgfqpoint{0.789329in}{0.796236in}}{\pgfqpoint{0.778730in}{0.800626in}}{\pgfqpoint{0.767680in}{0.800626in}}%
\pgfpathcurveto{\pgfqpoint{0.756630in}{0.800626in}}{\pgfqpoint{0.746031in}{0.796236in}}{\pgfqpoint{0.738217in}{0.788423in}}%
\pgfpathcurveto{\pgfqpoint{0.730403in}{0.780609in}}{\pgfqpoint{0.726013in}{0.770010in}}{\pgfqpoint{0.726013in}{0.758960in}}%
\pgfpathcurveto{\pgfqpoint{0.726013in}{0.747910in}}{\pgfqpoint{0.730403in}{0.737311in}}{\pgfqpoint{0.738217in}{0.729497in}}%
\pgfpathcurveto{\pgfqpoint{0.746031in}{0.721683in}}{\pgfqpoint{0.756630in}{0.717293in}}{\pgfqpoint{0.767680in}{0.717293in}}%
\pgfpathclose%
\pgfusepath{stroke,fill}%
\end{pgfscope}%
\begin{pgfscope}%
\pgfpathrectangle{\pgfqpoint{0.648703in}{0.548769in}}{\pgfqpoint{5.112893in}{3.102590in}}%
\pgfusepath{clip}%
\pgfsetbuttcap%
\pgfsetroundjoin%
\definecolor{currentfill}{rgb}{0.121569,0.466667,0.705882}%
\pgfsetfillcolor{currentfill}%
\pgfsetlinewidth{1.003750pt}%
\definecolor{currentstroke}{rgb}{0.121569,0.466667,0.705882}%
\pgfsetstrokecolor{currentstroke}%
\pgfsetdash{}{0pt}%
\pgfpathmoveto{\pgfqpoint{0.787143in}{0.733793in}}%
\pgfpathcurveto{\pgfqpoint{0.798193in}{0.733793in}}{\pgfqpoint{0.808792in}{0.738183in}}{\pgfqpoint{0.816605in}{0.745997in}}%
\pgfpathcurveto{\pgfqpoint{0.824419in}{0.753811in}}{\pgfqpoint{0.828809in}{0.764410in}}{\pgfqpoint{0.828809in}{0.775460in}}%
\pgfpathcurveto{\pgfqpoint{0.828809in}{0.786510in}}{\pgfqpoint{0.824419in}{0.797109in}}{\pgfqpoint{0.816605in}{0.804923in}}%
\pgfpathcurveto{\pgfqpoint{0.808792in}{0.812736in}}{\pgfqpoint{0.798193in}{0.817126in}}{\pgfqpoint{0.787143in}{0.817126in}}%
\pgfpathcurveto{\pgfqpoint{0.776092in}{0.817126in}}{\pgfqpoint{0.765493in}{0.812736in}}{\pgfqpoint{0.757680in}{0.804923in}}%
\pgfpathcurveto{\pgfqpoint{0.749866in}{0.797109in}}{\pgfqpoint{0.745476in}{0.786510in}}{\pgfqpoint{0.745476in}{0.775460in}}%
\pgfpathcurveto{\pgfqpoint{0.745476in}{0.764410in}}{\pgfqpoint{0.749866in}{0.753811in}}{\pgfqpoint{0.757680in}{0.745997in}}%
\pgfpathcurveto{\pgfqpoint{0.765493in}{0.738183in}}{\pgfqpoint{0.776092in}{0.733793in}}{\pgfqpoint{0.787143in}{0.733793in}}%
\pgfpathclose%
\pgfusepath{stroke,fill}%
\end{pgfscope}%
\begin{pgfscope}%
\pgfpathrectangle{\pgfqpoint{0.648703in}{0.548769in}}{\pgfqpoint{5.112893in}{3.102590in}}%
\pgfusepath{clip}%
\pgfsetbuttcap%
\pgfsetroundjoin%
\definecolor{currentfill}{rgb}{1.000000,0.498039,0.054902}%
\pgfsetfillcolor{currentfill}%
\pgfsetlinewidth{1.003750pt}%
\definecolor{currentstroke}{rgb}{1.000000,0.498039,0.054902}%
\pgfsetstrokecolor{currentstroke}%
\pgfsetdash{}{0pt}%
\pgfpathmoveto{\pgfqpoint{1.628110in}{3.196415in}}%
\pgfpathcurveto{\pgfqpoint{1.639160in}{3.196415in}}{\pgfqpoint{1.649759in}{3.200806in}}{\pgfqpoint{1.657572in}{3.208619in}}%
\pgfpathcurveto{\pgfqpoint{1.665386in}{3.216433in}}{\pgfqpoint{1.669776in}{3.227032in}}{\pgfqpoint{1.669776in}{3.238082in}}%
\pgfpathcurveto{\pgfqpoint{1.669776in}{3.249132in}}{\pgfqpoint{1.665386in}{3.259731in}}{\pgfqpoint{1.657572in}{3.267545in}}%
\pgfpathcurveto{\pgfqpoint{1.649759in}{3.275358in}}{\pgfqpoint{1.639160in}{3.279749in}}{\pgfqpoint{1.628110in}{3.279749in}}%
\pgfpathcurveto{\pgfqpoint{1.617060in}{3.279749in}}{\pgfqpoint{1.606460in}{3.275358in}}{\pgfqpoint{1.598647in}{3.267545in}}%
\pgfpathcurveto{\pgfqpoint{1.590833in}{3.259731in}}{\pgfqpoint{1.586443in}{3.249132in}}{\pgfqpoint{1.586443in}{3.238082in}}%
\pgfpathcurveto{\pgfqpoint{1.586443in}{3.227032in}}{\pgfqpoint{1.590833in}{3.216433in}}{\pgfqpoint{1.598647in}{3.208619in}}%
\pgfpathcurveto{\pgfqpoint{1.606460in}{3.200806in}}{\pgfqpoint{1.617060in}{3.196415in}}{\pgfqpoint{1.628110in}{3.196415in}}%
\pgfpathclose%
\pgfusepath{stroke,fill}%
\end{pgfscope}%
\begin{pgfscope}%
\pgfpathrectangle{\pgfqpoint{0.648703in}{0.548769in}}{\pgfqpoint{5.112893in}{3.102590in}}%
\pgfusepath{clip}%
\pgfsetbuttcap%
\pgfsetroundjoin%
\definecolor{currentfill}{rgb}{0.121569,0.466667,0.705882}%
\pgfsetfillcolor{currentfill}%
\pgfsetlinewidth{1.003750pt}%
\definecolor{currentstroke}{rgb}{0.121569,0.466667,0.705882}%
\pgfsetstrokecolor{currentstroke}%
\pgfsetdash{}{0pt}%
\pgfpathmoveto{\pgfqpoint{0.826789in}{0.874043in}}%
\pgfpathcurveto{\pgfqpoint{0.837839in}{0.874043in}}{\pgfqpoint{0.848438in}{0.878433in}}{\pgfqpoint{0.856252in}{0.886247in}}%
\pgfpathcurveto{\pgfqpoint{0.864065in}{0.894060in}}{\pgfqpoint{0.868456in}{0.904659in}}{\pgfqpoint{0.868456in}{0.915710in}}%
\pgfpathcurveto{\pgfqpoint{0.868456in}{0.926760in}}{\pgfqpoint{0.864065in}{0.937359in}}{\pgfqpoint{0.856252in}{0.945172in}}%
\pgfpathcurveto{\pgfqpoint{0.848438in}{0.952986in}}{\pgfqpoint{0.837839in}{0.957376in}}{\pgfqpoint{0.826789in}{0.957376in}}%
\pgfpathcurveto{\pgfqpoint{0.815739in}{0.957376in}}{\pgfqpoint{0.805140in}{0.952986in}}{\pgfqpoint{0.797326in}{0.945172in}}%
\pgfpathcurveto{\pgfqpoint{0.789512in}{0.937359in}}{\pgfqpoint{0.785122in}{0.926760in}}{\pgfqpoint{0.785122in}{0.915710in}}%
\pgfpathcurveto{\pgfqpoint{0.785122in}{0.904659in}}{\pgfqpoint{0.789512in}{0.894060in}}{\pgfqpoint{0.797326in}{0.886247in}}%
\pgfpathcurveto{\pgfqpoint{0.805140in}{0.878433in}}{\pgfqpoint{0.815739in}{0.874043in}}{\pgfqpoint{0.826789in}{0.874043in}}%
\pgfpathclose%
\pgfusepath{stroke,fill}%
\end{pgfscope}%
\begin{pgfscope}%
\pgfpathrectangle{\pgfqpoint{0.648703in}{0.548769in}}{\pgfqpoint{5.112893in}{3.102590in}}%
\pgfusepath{clip}%
\pgfsetbuttcap%
\pgfsetroundjoin%
\definecolor{currentfill}{rgb}{0.839216,0.152941,0.156863}%
\pgfsetfillcolor{currentfill}%
\pgfsetlinewidth{1.003750pt}%
\definecolor{currentstroke}{rgb}{0.839216,0.152941,0.156863}%
\pgfsetstrokecolor{currentstroke}%
\pgfsetdash{}{0pt}%
\pgfpathmoveto{\pgfqpoint{1.621951in}{3.204665in}}%
\pgfpathcurveto{\pgfqpoint{1.633001in}{3.204665in}}{\pgfqpoint{1.643600in}{3.209056in}}{\pgfqpoint{1.651414in}{3.216869in}}%
\pgfpathcurveto{\pgfqpoint{1.659227in}{3.224683in}}{\pgfqpoint{1.663618in}{3.235282in}}{\pgfqpoint{1.663618in}{3.246332in}}%
\pgfpathcurveto{\pgfqpoint{1.663618in}{3.257382in}}{\pgfqpoint{1.659227in}{3.267981in}}{\pgfqpoint{1.651414in}{3.275795in}}%
\pgfpathcurveto{\pgfqpoint{1.643600in}{3.283608in}}{\pgfqpoint{1.633001in}{3.287999in}}{\pgfqpoint{1.621951in}{3.287999in}}%
\pgfpathcurveto{\pgfqpoint{1.610901in}{3.287999in}}{\pgfqpoint{1.600302in}{3.283608in}}{\pgfqpoint{1.592488in}{3.275795in}}%
\pgfpathcurveto{\pgfqpoint{1.584674in}{3.267981in}}{\pgfqpoint{1.580284in}{3.257382in}}{\pgfqpoint{1.580284in}{3.246332in}}%
\pgfpathcurveto{\pgfqpoint{1.580284in}{3.235282in}}{\pgfqpoint{1.584674in}{3.224683in}}{\pgfqpoint{1.592488in}{3.216869in}}%
\pgfpathcurveto{\pgfqpoint{1.600302in}{3.209056in}}{\pgfqpoint{1.610901in}{3.204665in}}{\pgfqpoint{1.621951in}{3.204665in}}%
\pgfpathclose%
\pgfusepath{stroke,fill}%
\end{pgfscope}%
\begin{pgfscope}%
\pgfpathrectangle{\pgfqpoint{0.648703in}{0.548769in}}{\pgfqpoint{5.112893in}{3.102590in}}%
\pgfusepath{clip}%
\pgfsetbuttcap%
\pgfsetroundjoin%
\definecolor{currentfill}{rgb}{0.121569,0.466667,0.705882}%
\pgfsetfillcolor{currentfill}%
\pgfsetlinewidth{1.003750pt}%
\definecolor{currentstroke}{rgb}{0.121569,0.466667,0.705882}%
\pgfsetstrokecolor{currentstroke}%
\pgfsetdash{}{0pt}%
\pgfpathmoveto{\pgfqpoint{0.769423in}{0.717293in}}%
\pgfpathcurveto{\pgfqpoint{0.780473in}{0.717293in}}{\pgfqpoint{0.791072in}{0.721683in}}{\pgfqpoint{0.798885in}{0.729497in}}%
\pgfpathcurveto{\pgfqpoint{0.806699in}{0.737311in}}{\pgfqpoint{0.811089in}{0.747910in}}{\pgfqpoint{0.811089in}{0.758960in}}%
\pgfpathcurveto{\pgfqpoint{0.811089in}{0.770010in}}{\pgfqpoint{0.806699in}{0.780609in}}{\pgfqpoint{0.798885in}{0.788423in}}%
\pgfpathcurveto{\pgfqpoint{0.791072in}{0.796236in}}{\pgfqpoint{0.780473in}{0.800626in}}{\pgfqpoint{0.769423in}{0.800626in}}%
\pgfpathcurveto{\pgfqpoint{0.758372in}{0.800626in}}{\pgfqpoint{0.747773in}{0.796236in}}{\pgfqpoint{0.739960in}{0.788423in}}%
\pgfpathcurveto{\pgfqpoint{0.732146in}{0.780609in}}{\pgfqpoint{0.727756in}{0.770010in}}{\pgfqpoint{0.727756in}{0.758960in}}%
\pgfpathcurveto{\pgfqpoint{0.727756in}{0.747910in}}{\pgfqpoint{0.732146in}{0.737311in}}{\pgfqpoint{0.739960in}{0.729497in}}%
\pgfpathcurveto{\pgfqpoint{0.747773in}{0.721683in}}{\pgfqpoint{0.758372in}{0.717293in}}{\pgfqpoint{0.769423in}{0.717293in}}%
\pgfpathclose%
\pgfusepath{stroke,fill}%
\end{pgfscope}%
\begin{pgfscope}%
\pgfpathrectangle{\pgfqpoint{0.648703in}{0.548769in}}{\pgfqpoint{5.112893in}{3.102590in}}%
\pgfusepath{clip}%
\pgfsetbuttcap%
\pgfsetroundjoin%
\definecolor{currentfill}{rgb}{1.000000,0.498039,0.054902}%
\pgfsetfillcolor{currentfill}%
\pgfsetlinewidth{1.003750pt}%
\definecolor{currentstroke}{rgb}{1.000000,0.498039,0.054902}%
\pgfsetstrokecolor{currentstroke}%
\pgfsetdash{}{0pt}%
\pgfpathmoveto{\pgfqpoint{1.428157in}{3.192290in}}%
\pgfpathcurveto{\pgfqpoint{1.439207in}{3.192290in}}{\pgfqpoint{1.449806in}{3.196681in}}{\pgfqpoint{1.457619in}{3.204494in}}%
\pgfpathcurveto{\pgfqpoint{1.465433in}{3.212308in}}{\pgfqpoint{1.469823in}{3.222907in}}{\pgfqpoint{1.469823in}{3.233957in}}%
\pgfpathcurveto{\pgfqpoint{1.469823in}{3.245007in}}{\pgfqpoint{1.465433in}{3.255606in}}{\pgfqpoint{1.457619in}{3.263420in}}%
\pgfpathcurveto{\pgfqpoint{1.449806in}{3.271234in}}{\pgfqpoint{1.439207in}{3.275624in}}{\pgfqpoint{1.428157in}{3.275624in}}%
\pgfpathcurveto{\pgfqpoint{1.417106in}{3.275624in}}{\pgfqpoint{1.406507in}{3.271234in}}{\pgfqpoint{1.398694in}{3.263420in}}%
\pgfpathcurveto{\pgfqpoint{1.390880in}{3.255606in}}{\pgfqpoint{1.386490in}{3.245007in}}{\pgfqpoint{1.386490in}{3.233957in}}%
\pgfpathcurveto{\pgfqpoint{1.386490in}{3.222907in}}{\pgfqpoint{1.390880in}{3.212308in}}{\pgfqpoint{1.398694in}{3.204494in}}%
\pgfpathcurveto{\pgfqpoint{1.406507in}{3.196681in}}{\pgfqpoint{1.417106in}{3.192290in}}{\pgfqpoint{1.428157in}{3.192290in}}%
\pgfpathclose%
\pgfusepath{stroke,fill}%
\end{pgfscope}%
\begin{pgfscope}%
\pgfpathrectangle{\pgfqpoint{0.648703in}{0.548769in}}{\pgfqpoint{5.112893in}{3.102590in}}%
\pgfusepath{clip}%
\pgfsetbuttcap%
\pgfsetroundjoin%
\definecolor{currentfill}{rgb}{0.121569,0.466667,0.705882}%
\pgfsetfillcolor{currentfill}%
\pgfsetlinewidth{1.003750pt}%
\definecolor{currentstroke}{rgb}{0.121569,0.466667,0.705882}%
\pgfsetstrokecolor{currentstroke}%
\pgfsetdash{}{0pt}%
\pgfpathmoveto{\pgfqpoint{0.770222in}{0.717293in}}%
\pgfpathcurveto{\pgfqpoint{0.781272in}{0.717293in}}{\pgfqpoint{0.791871in}{0.721683in}}{\pgfqpoint{0.799685in}{0.729497in}}%
\pgfpathcurveto{\pgfqpoint{0.807498in}{0.737311in}}{\pgfqpoint{0.811889in}{0.747910in}}{\pgfqpoint{0.811889in}{0.758960in}}%
\pgfpathcurveto{\pgfqpoint{0.811889in}{0.770010in}}{\pgfqpoint{0.807498in}{0.780609in}}{\pgfqpoint{0.799685in}{0.788423in}}%
\pgfpathcurveto{\pgfqpoint{0.791871in}{0.796236in}}{\pgfqpoint{0.781272in}{0.800626in}}{\pgfqpoint{0.770222in}{0.800626in}}%
\pgfpathcurveto{\pgfqpoint{0.759172in}{0.800626in}}{\pgfqpoint{0.748573in}{0.796236in}}{\pgfqpoint{0.740759in}{0.788423in}}%
\pgfpathcurveto{\pgfqpoint{0.732946in}{0.780609in}}{\pgfqpoint{0.728555in}{0.770010in}}{\pgfqpoint{0.728555in}{0.758960in}}%
\pgfpathcurveto{\pgfqpoint{0.728555in}{0.747910in}}{\pgfqpoint{0.732946in}{0.737311in}}{\pgfqpoint{0.740759in}{0.729497in}}%
\pgfpathcurveto{\pgfqpoint{0.748573in}{0.721683in}}{\pgfqpoint{0.759172in}{0.717293in}}{\pgfqpoint{0.770222in}{0.717293in}}%
\pgfpathclose%
\pgfusepath{stroke,fill}%
\end{pgfscope}%
\begin{pgfscope}%
\pgfpathrectangle{\pgfqpoint{0.648703in}{0.548769in}}{\pgfqpoint{5.112893in}{3.102590in}}%
\pgfusepath{clip}%
\pgfsetbuttcap%
\pgfsetroundjoin%
\definecolor{currentfill}{rgb}{1.000000,0.498039,0.054902}%
\pgfsetfillcolor{currentfill}%
\pgfsetlinewidth{1.003750pt}%
\definecolor{currentstroke}{rgb}{1.000000,0.498039,0.054902}%
\pgfsetstrokecolor{currentstroke}%
\pgfsetdash{}{0pt}%
\pgfpathmoveto{\pgfqpoint{2.490748in}{3.192290in}}%
\pgfpathcurveto{\pgfqpoint{2.501798in}{3.192290in}}{\pgfqpoint{2.512397in}{3.196681in}}{\pgfqpoint{2.520210in}{3.204494in}}%
\pgfpathcurveto{\pgfqpoint{2.528024in}{3.212308in}}{\pgfqpoint{2.532414in}{3.222907in}}{\pgfqpoint{2.532414in}{3.233957in}}%
\pgfpathcurveto{\pgfqpoint{2.532414in}{3.245007in}}{\pgfqpoint{2.528024in}{3.255606in}}{\pgfqpoint{2.520210in}{3.263420in}}%
\pgfpathcurveto{\pgfqpoint{2.512397in}{3.271234in}}{\pgfqpoint{2.501798in}{3.275624in}}{\pgfqpoint{2.490748in}{3.275624in}}%
\pgfpathcurveto{\pgfqpoint{2.479698in}{3.275624in}}{\pgfqpoint{2.469099in}{3.271234in}}{\pgfqpoint{2.461285in}{3.263420in}}%
\pgfpathcurveto{\pgfqpoint{2.453471in}{3.255606in}}{\pgfqpoint{2.449081in}{3.245007in}}{\pgfqpoint{2.449081in}{3.233957in}}%
\pgfpathcurveto{\pgfqpoint{2.449081in}{3.222907in}}{\pgfqpoint{2.453471in}{3.212308in}}{\pgfqpoint{2.461285in}{3.204494in}}%
\pgfpathcurveto{\pgfqpoint{2.469099in}{3.196681in}}{\pgfqpoint{2.479698in}{3.192290in}}{\pgfqpoint{2.490748in}{3.192290in}}%
\pgfpathclose%
\pgfusepath{stroke,fill}%
\end{pgfscope}%
\begin{pgfscope}%
\pgfpathrectangle{\pgfqpoint{0.648703in}{0.548769in}}{\pgfqpoint{5.112893in}{3.102590in}}%
\pgfusepath{clip}%
\pgfsetbuttcap%
\pgfsetroundjoin%
\definecolor{currentfill}{rgb}{0.121569,0.466667,0.705882}%
\pgfsetfillcolor{currentfill}%
\pgfsetlinewidth{1.003750pt}%
\definecolor{currentstroke}{rgb}{0.121569,0.466667,0.705882}%
\pgfsetstrokecolor{currentstroke}%
\pgfsetdash{}{0pt}%
\pgfpathmoveto{\pgfqpoint{0.767696in}{0.717293in}}%
\pgfpathcurveto{\pgfqpoint{0.778746in}{0.717293in}}{\pgfqpoint{0.789345in}{0.721683in}}{\pgfqpoint{0.797159in}{0.729497in}}%
\pgfpathcurveto{\pgfqpoint{0.804972in}{0.737311in}}{\pgfqpoint{0.809363in}{0.747910in}}{\pgfqpoint{0.809363in}{0.758960in}}%
\pgfpathcurveto{\pgfqpoint{0.809363in}{0.770010in}}{\pgfqpoint{0.804972in}{0.780609in}}{\pgfqpoint{0.797159in}{0.788423in}}%
\pgfpathcurveto{\pgfqpoint{0.789345in}{0.796236in}}{\pgfqpoint{0.778746in}{0.800626in}}{\pgfqpoint{0.767696in}{0.800626in}}%
\pgfpathcurveto{\pgfqpoint{0.756646in}{0.800626in}}{\pgfqpoint{0.746047in}{0.796236in}}{\pgfqpoint{0.738233in}{0.788423in}}%
\pgfpathcurveto{\pgfqpoint{0.730419in}{0.780609in}}{\pgfqpoint{0.726029in}{0.770010in}}{\pgfqpoint{0.726029in}{0.758960in}}%
\pgfpathcurveto{\pgfqpoint{0.726029in}{0.747910in}}{\pgfqpoint{0.730419in}{0.737311in}}{\pgfqpoint{0.738233in}{0.729497in}}%
\pgfpathcurveto{\pgfqpoint{0.746047in}{0.721683in}}{\pgfqpoint{0.756646in}{0.717293in}}{\pgfqpoint{0.767696in}{0.717293in}}%
\pgfpathclose%
\pgfusepath{stroke,fill}%
\end{pgfscope}%
\begin{pgfscope}%
\pgfpathrectangle{\pgfqpoint{0.648703in}{0.548769in}}{\pgfqpoint{5.112893in}{3.102590in}}%
\pgfusepath{clip}%
\pgfsetbuttcap%
\pgfsetroundjoin%
\definecolor{currentfill}{rgb}{1.000000,0.498039,0.054902}%
\pgfsetfillcolor{currentfill}%
\pgfsetlinewidth{1.003750pt}%
\definecolor{currentstroke}{rgb}{1.000000,0.498039,0.054902}%
\pgfsetstrokecolor{currentstroke}%
\pgfsetdash{}{0pt}%
\pgfpathmoveto{\pgfqpoint{1.557925in}{3.283040in}}%
\pgfpathcurveto{\pgfqpoint{1.568975in}{3.283040in}}{\pgfqpoint{1.579574in}{3.287431in}}{\pgfqpoint{1.587387in}{3.295244in}}%
\pgfpathcurveto{\pgfqpoint{1.595201in}{3.303058in}}{\pgfqpoint{1.599591in}{3.313657in}}{\pgfqpoint{1.599591in}{3.324707in}}%
\pgfpathcurveto{\pgfqpoint{1.599591in}{3.335757in}}{\pgfqpoint{1.595201in}{3.346356in}}{\pgfqpoint{1.587387in}{3.354170in}}%
\pgfpathcurveto{\pgfqpoint{1.579574in}{3.361983in}}{\pgfqpoint{1.568975in}{3.366374in}}{\pgfqpoint{1.557925in}{3.366374in}}%
\pgfpathcurveto{\pgfqpoint{1.546875in}{3.366374in}}{\pgfqpoint{1.536276in}{3.361983in}}{\pgfqpoint{1.528462in}{3.354170in}}%
\pgfpathcurveto{\pgfqpoint{1.520648in}{3.346356in}}{\pgfqpoint{1.516258in}{3.335757in}}{\pgfqpoint{1.516258in}{3.324707in}}%
\pgfpathcurveto{\pgfqpoint{1.516258in}{3.313657in}}{\pgfqpoint{1.520648in}{3.303058in}}{\pgfqpoint{1.528462in}{3.295244in}}%
\pgfpathcurveto{\pgfqpoint{1.536276in}{3.287431in}}{\pgfqpoint{1.546875in}{3.283040in}}{\pgfqpoint{1.557925in}{3.283040in}}%
\pgfpathclose%
\pgfusepath{stroke,fill}%
\end{pgfscope}%
\begin{pgfscope}%
\pgfpathrectangle{\pgfqpoint{0.648703in}{0.548769in}}{\pgfqpoint{5.112893in}{3.102590in}}%
\pgfusepath{clip}%
\pgfsetbuttcap%
\pgfsetroundjoin%
\definecolor{currentfill}{rgb}{0.121569,0.466667,0.705882}%
\pgfsetfillcolor{currentfill}%
\pgfsetlinewidth{1.003750pt}%
\definecolor{currentstroke}{rgb}{0.121569,0.466667,0.705882}%
\pgfsetstrokecolor{currentstroke}%
\pgfsetdash{}{0pt}%
\pgfpathmoveto{\pgfqpoint{0.767677in}{0.717293in}}%
\pgfpathcurveto{\pgfqpoint{0.778727in}{0.717293in}}{\pgfqpoint{0.789326in}{0.721683in}}{\pgfqpoint{0.797140in}{0.729497in}}%
\pgfpathcurveto{\pgfqpoint{0.804953in}{0.737311in}}{\pgfqpoint{0.809343in}{0.747910in}}{\pgfqpoint{0.809343in}{0.758960in}}%
\pgfpathcurveto{\pgfqpoint{0.809343in}{0.770010in}}{\pgfqpoint{0.804953in}{0.780609in}}{\pgfqpoint{0.797140in}{0.788423in}}%
\pgfpathcurveto{\pgfqpoint{0.789326in}{0.796236in}}{\pgfqpoint{0.778727in}{0.800626in}}{\pgfqpoint{0.767677in}{0.800626in}}%
\pgfpathcurveto{\pgfqpoint{0.756627in}{0.800626in}}{\pgfqpoint{0.746028in}{0.796236in}}{\pgfqpoint{0.738214in}{0.788423in}}%
\pgfpathcurveto{\pgfqpoint{0.730400in}{0.780609in}}{\pgfqpoint{0.726010in}{0.770010in}}{\pgfqpoint{0.726010in}{0.758960in}}%
\pgfpathcurveto{\pgfqpoint{0.726010in}{0.747910in}}{\pgfqpoint{0.730400in}{0.737311in}}{\pgfqpoint{0.738214in}{0.729497in}}%
\pgfpathcurveto{\pgfqpoint{0.746028in}{0.721683in}}{\pgfqpoint{0.756627in}{0.717293in}}{\pgfqpoint{0.767677in}{0.717293in}}%
\pgfpathclose%
\pgfusepath{stroke,fill}%
\end{pgfscope}%
\begin{pgfscope}%
\pgfpathrectangle{\pgfqpoint{0.648703in}{0.548769in}}{\pgfqpoint{5.112893in}{3.102590in}}%
\pgfusepath{clip}%
\pgfsetbuttcap%
\pgfsetroundjoin%
\definecolor{currentfill}{rgb}{1.000000,0.498039,0.054902}%
\pgfsetfillcolor{currentfill}%
\pgfsetlinewidth{1.003750pt}%
\definecolor{currentstroke}{rgb}{1.000000,0.498039,0.054902}%
\pgfsetstrokecolor{currentstroke}%
\pgfsetdash{}{0pt}%
\pgfpathmoveto{\pgfqpoint{1.359071in}{3.208790in}}%
\pgfpathcurveto{\pgfqpoint{1.370121in}{3.208790in}}{\pgfqpoint{1.380720in}{3.213181in}}{\pgfqpoint{1.388534in}{3.220994in}}%
\pgfpathcurveto{\pgfqpoint{1.396347in}{3.228808in}}{\pgfqpoint{1.400738in}{3.239407in}}{\pgfqpoint{1.400738in}{3.250457in}}%
\pgfpathcurveto{\pgfqpoint{1.400738in}{3.261507in}}{\pgfqpoint{1.396347in}{3.272106in}}{\pgfqpoint{1.388534in}{3.279920in}}%
\pgfpathcurveto{\pgfqpoint{1.380720in}{3.287733in}}{\pgfqpoint{1.370121in}{3.292124in}}{\pgfqpoint{1.359071in}{3.292124in}}%
\pgfpathcurveto{\pgfqpoint{1.348021in}{3.292124in}}{\pgfqpoint{1.337422in}{3.287733in}}{\pgfqpoint{1.329608in}{3.279920in}}%
\pgfpathcurveto{\pgfqpoint{1.321794in}{3.272106in}}{\pgfqpoint{1.317404in}{3.261507in}}{\pgfqpoint{1.317404in}{3.250457in}}%
\pgfpathcurveto{\pgfqpoint{1.317404in}{3.239407in}}{\pgfqpoint{1.321794in}{3.228808in}}{\pgfqpoint{1.329608in}{3.220994in}}%
\pgfpathcurveto{\pgfqpoint{1.337422in}{3.213181in}}{\pgfqpoint{1.348021in}{3.208790in}}{\pgfqpoint{1.359071in}{3.208790in}}%
\pgfpathclose%
\pgfusepath{stroke,fill}%
\end{pgfscope}%
\begin{pgfscope}%
\pgfpathrectangle{\pgfqpoint{0.648703in}{0.548769in}}{\pgfqpoint{5.112893in}{3.102590in}}%
\pgfusepath{clip}%
\pgfsetbuttcap%
\pgfsetroundjoin%
\definecolor{currentfill}{rgb}{1.000000,0.498039,0.054902}%
\pgfsetfillcolor{currentfill}%
\pgfsetlinewidth{1.003750pt}%
\definecolor{currentstroke}{rgb}{1.000000,0.498039,0.054902}%
\pgfsetstrokecolor{currentstroke}%
\pgfsetdash{}{0pt}%
\pgfpathmoveto{\pgfqpoint{1.469120in}{3.192290in}}%
\pgfpathcurveto{\pgfqpoint{1.480170in}{3.192290in}}{\pgfqpoint{1.490769in}{3.196681in}}{\pgfqpoint{1.498583in}{3.204494in}}%
\pgfpathcurveto{\pgfqpoint{1.506396in}{3.212308in}}{\pgfqpoint{1.510787in}{3.222907in}}{\pgfqpoint{1.510787in}{3.233957in}}%
\pgfpathcurveto{\pgfqpoint{1.510787in}{3.245007in}}{\pgfqpoint{1.506396in}{3.255606in}}{\pgfqpoint{1.498583in}{3.263420in}}%
\pgfpathcurveto{\pgfqpoint{1.490769in}{3.271234in}}{\pgfqpoint{1.480170in}{3.275624in}}{\pgfqpoint{1.469120in}{3.275624in}}%
\pgfpathcurveto{\pgfqpoint{1.458070in}{3.275624in}}{\pgfqpoint{1.447471in}{3.271234in}}{\pgfqpoint{1.439657in}{3.263420in}}%
\pgfpathcurveto{\pgfqpoint{1.431843in}{3.255606in}}{\pgfqpoint{1.427453in}{3.245007in}}{\pgfqpoint{1.427453in}{3.233957in}}%
\pgfpathcurveto{\pgfqpoint{1.427453in}{3.222907in}}{\pgfqpoint{1.431843in}{3.212308in}}{\pgfqpoint{1.439657in}{3.204494in}}%
\pgfpathcurveto{\pgfqpoint{1.447471in}{3.196681in}}{\pgfqpoint{1.458070in}{3.192290in}}{\pgfqpoint{1.469120in}{3.192290in}}%
\pgfpathclose%
\pgfusepath{stroke,fill}%
\end{pgfscope}%
\begin{pgfscope}%
\pgfpathrectangle{\pgfqpoint{0.648703in}{0.548769in}}{\pgfqpoint{5.112893in}{3.102590in}}%
\pgfusepath{clip}%
\pgfsetbuttcap%
\pgfsetroundjoin%
\definecolor{currentfill}{rgb}{0.121569,0.466667,0.705882}%
\pgfsetfillcolor{currentfill}%
\pgfsetlinewidth{1.003750pt}%
\definecolor{currentstroke}{rgb}{0.121569,0.466667,0.705882}%
\pgfsetstrokecolor{currentstroke}%
\pgfsetdash{}{0pt}%
\pgfpathmoveto{\pgfqpoint{0.767688in}{0.717293in}}%
\pgfpathcurveto{\pgfqpoint{0.778738in}{0.717293in}}{\pgfqpoint{0.789337in}{0.721683in}}{\pgfqpoint{0.797150in}{0.729497in}}%
\pgfpathcurveto{\pgfqpoint{0.804964in}{0.737311in}}{\pgfqpoint{0.809354in}{0.747910in}}{\pgfqpoint{0.809354in}{0.758960in}}%
\pgfpathcurveto{\pgfqpoint{0.809354in}{0.770010in}}{\pgfqpoint{0.804964in}{0.780609in}}{\pgfqpoint{0.797150in}{0.788423in}}%
\pgfpathcurveto{\pgfqpoint{0.789337in}{0.796236in}}{\pgfqpoint{0.778738in}{0.800626in}}{\pgfqpoint{0.767688in}{0.800626in}}%
\pgfpathcurveto{\pgfqpoint{0.756637in}{0.800626in}}{\pgfqpoint{0.746038in}{0.796236in}}{\pgfqpoint{0.738225in}{0.788423in}}%
\pgfpathcurveto{\pgfqpoint{0.730411in}{0.780609in}}{\pgfqpoint{0.726021in}{0.770010in}}{\pgfqpoint{0.726021in}{0.758960in}}%
\pgfpathcurveto{\pgfqpoint{0.726021in}{0.747910in}}{\pgfqpoint{0.730411in}{0.737311in}}{\pgfqpoint{0.738225in}{0.729497in}}%
\pgfpathcurveto{\pgfqpoint{0.746038in}{0.721683in}}{\pgfqpoint{0.756637in}{0.717293in}}{\pgfqpoint{0.767688in}{0.717293in}}%
\pgfpathclose%
\pgfusepath{stroke,fill}%
\end{pgfscope}%
\begin{pgfscope}%
\pgfpathrectangle{\pgfqpoint{0.648703in}{0.548769in}}{\pgfqpoint{5.112893in}{3.102590in}}%
\pgfusepath{clip}%
\pgfsetbuttcap%
\pgfsetroundjoin%
\definecolor{currentfill}{rgb}{1.000000,0.498039,0.054902}%
\pgfsetfillcolor{currentfill}%
\pgfsetlinewidth{1.003750pt}%
\definecolor{currentstroke}{rgb}{1.000000,0.498039,0.054902}%
\pgfsetstrokecolor{currentstroke}%
\pgfsetdash{}{0pt}%
\pgfpathmoveto{\pgfqpoint{1.793222in}{3.192290in}}%
\pgfpathcurveto{\pgfqpoint{1.804272in}{3.192290in}}{\pgfqpoint{1.814872in}{3.196681in}}{\pgfqpoint{1.822685in}{3.204494in}}%
\pgfpathcurveto{\pgfqpoint{1.830499in}{3.212308in}}{\pgfqpoint{1.834889in}{3.222907in}}{\pgfqpoint{1.834889in}{3.233957in}}%
\pgfpathcurveto{\pgfqpoint{1.834889in}{3.245007in}}{\pgfqpoint{1.830499in}{3.255606in}}{\pgfqpoint{1.822685in}{3.263420in}}%
\pgfpathcurveto{\pgfqpoint{1.814872in}{3.271234in}}{\pgfqpoint{1.804272in}{3.275624in}}{\pgfqpoint{1.793222in}{3.275624in}}%
\pgfpathcurveto{\pgfqpoint{1.782172in}{3.275624in}}{\pgfqpoint{1.771573in}{3.271234in}}{\pgfqpoint{1.763760in}{3.263420in}}%
\pgfpathcurveto{\pgfqpoint{1.755946in}{3.255606in}}{\pgfqpoint{1.751556in}{3.245007in}}{\pgfqpoint{1.751556in}{3.233957in}}%
\pgfpathcurveto{\pgfqpoint{1.751556in}{3.222907in}}{\pgfqpoint{1.755946in}{3.212308in}}{\pgfqpoint{1.763760in}{3.204494in}}%
\pgfpathcurveto{\pgfqpoint{1.771573in}{3.196681in}}{\pgfqpoint{1.782172in}{3.192290in}}{\pgfqpoint{1.793222in}{3.192290in}}%
\pgfpathclose%
\pgfusepath{stroke,fill}%
\end{pgfscope}%
\begin{pgfscope}%
\pgfpathrectangle{\pgfqpoint{0.648703in}{0.548769in}}{\pgfqpoint{5.112893in}{3.102590in}}%
\pgfusepath{clip}%
\pgfsetbuttcap%
\pgfsetroundjoin%
\definecolor{currentfill}{rgb}{1.000000,0.498039,0.054902}%
\pgfsetfillcolor{currentfill}%
\pgfsetlinewidth{1.003750pt}%
\definecolor{currentstroke}{rgb}{1.000000,0.498039,0.054902}%
\pgfsetstrokecolor{currentstroke}%
\pgfsetdash{}{0pt}%
\pgfpathmoveto{\pgfqpoint{1.889975in}{3.200540in}}%
\pgfpathcurveto{\pgfqpoint{1.901025in}{3.200540in}}{\pgfqpoint{1.911624in}{3.204931in}}{\pgfqpoint{1.919437in}{3.212744in}}%
\pgfpathcurveto{\pgfqpoint{1.927251in}{3.220558in}}{\pgfqpoint{1.931641in}{3.231157in}}{\pgfqpoint{1.931641in}{3.242207in}}%
\pgfpathcurveto{\pgfqpoint{1.931641in}{3.253257in}}{\pgfqpoint{1.927251in}{3.263856in}}{\pgfqpoint{1.919437in}{3.271670in}}%
\pgfpathcurveto{\pgfqpoint{1.911624in}{3.279483in}}{\pgfqpoint{1.901025in}{3.283874in}}{\pgfqpoint{1.889975in}{3.283874in}}%
\pgfpathcurveto{\pgfqpoint{1.878925in}{3.283874in}}{\pgfqpoint{1.868325in}{3.279483in}}{\pgfqpoint{1.860512in}{3.271670in}}%
\pgfpathcurveto{\pgfqpoint{1.852698in}{3.263856in}}{\pgfqpoint{1.848308in}{3.253257in}}{\pgfqpoint{1.848308in}{3.242207in}}%
\pgfpathcurveto{\pgfqpoint{1.848308in}{3.231157in}}{\pgfqpoint{1.852698in}{3.220558in}}{\pgfqpoint{1.860512in}{3.212744in}}%
\pgfpathcurveto{\pgfqpoint{1.868325in}{3.204931in}}{\pgfqpoint{1.878925in}{3.200540in}}{\pgfqpoint{1.889975in}{3.200540in}}%
\pgfpathclose%
\pgfusepath{stroke,fill}%
\end{pgfscope}%
\begin{pgfscope}%
\pgfpathrectangle{\pgfqpoint{0.648703in}{0.548769in}}{\pgfqpoint{5.112893in}{3.102590in}}%
\pgfusepath{clip}%
\pgfsetbuttcap%
\pgfsetroundjoin%
\definecolor{currentfill}{rgb}{0.121569,0.466667,0.705882}%
\pgfsetfillcolor{currentfill}%
\pgfsetlinewidth{1.003750pt}%
\definecolor{currentstroke}{rgb}{0.121569,0.466667,0.705882}%
\pgfsetstrokecolor{currentstroke}%
\pgfsetdash{}{0pt}%
\pgfpathmoveto{\pgfqpoint{0.779873in}{0.725543in}}%
\pgfpathcurveto{\pgfqpoint{0.790923in}{0.725543in}}{\pgfqpoint{0.801522in}{0.729933in}}{\pgfqpoint{0.809336in}{0.737747in}}%
\pgfpathcurveto{\pgfqpoint{0.817150in}{0.745561in}}{\pgfqpoint{0.821540in}{0.756160in}}{\pgfqpoint{0.821540in}{0.767210in}}%
\pgfpathcurveto{\pgfqpoint{0.821540in}{0.778260in}}{\pgfqpoint{0.817150in}{0.788859in}}{\pgfqpoint{0.809336in}{0.796673in}}%
\pgfpathcurveto{\pgfqpoint{0.801522in}{0.804486in}}{\pgfqpoint{0.790923in}{0.808876in}}{\pgfqpoint{0.779873in}{0.808876in}}%
\pgfpathcurveto{\pgfqpoint{0.768823in}{0.808876in}}{\pgfqpoint{0.758224in}{0.804486in}}{\pgfqpoint{0.750410in}{0.796673in}}%
\pgfpathcurveto{\pgfqpoint{0.742597in}{0.788859in}}{\pgfqpoint{0.738207in}{0.778260in}}{\pgfqpoint{0.738207in}{0.767210in}}%
\pgfpathcurveto{\pgfqpoint{0.738207in}{0.756160in}}{\pgfqpoint{0.742597in}{0.745561in}}{\pgfqpoint{0.750410in}{0.737747in}}%
\pgfpathcurveto{\pgfqpoint{0.758224in}{0.729933in}}{\pgfqpoint{0.768823in}{0.725543in}}{\pgfqpoint{0.779873in}{0.725543in}}%
\pgfpathclose%
\pgfusepath{stroke,fill}%
\end{pgfscope}%
\begin{pgfscope}%
\pgfpathrectangle{\pgfqpoint{0.648703in}{0.548769in}}{\pgfqpoint{5.112893in}{3.102590in}}%
\pgfusepath{clip}%
\pgfsetbuttcap%
\pgfsetroundjoin%
\definecolor{currentfill}{rgb}{1.000000,0.498039,0.054902}%
\pgfsetfillcolor{currentfill}%
\pgfsetlinewidth{1.003750pt}%
\definecolor{currentstroke}{rgb}{1.000000,0.498039,0.054902}%
\pgfsetstrokecolor{currentstroke}%
\pgfsetdash{}{0pt}%
\pgfpathmoveto{\pgfqpoint{2.016983in}{3.221165in}}%
\pgfpathcurveto{\pgfqpoint{2.028033in}{3.221165in}}{\pgfqpoint{2.038632in}{3.225556in}}{\pgfqpoint{2.046445in}{3.233369in}}%
\pgfpathcurveto{\pgfqpoint{2.054259in}{3.241183in}}{\pgfqpoint{2.058649in}{3.251782in}}{\pgfqpoint{2.058649in}{3.262832in}}%
\pgfpathcurveto{\pgfqpoint{2.058649in}{3.273882in}}{\pgfqpoint{2.054259in}{3.284481in}}{\pgfqpoint{2.046445in}{3.292295in}}%
\pgfpathcurveto{\pgfqpoint{2.038632in}{3.300108in}}{\pgfqpoint{2.028033in}{3.304499in}}{\pgfqpoint{2.016983in}{3.304499in}}%
\pgfpathcurveto{\pgfqpoint{2.005933in}{3.304499in}}{\pgfqpoint{1.995334in}{3.300108in}}{\pgfqpoint{1.987520in}{3.292295in}}%
\pgfpathcurveto{\pgfqpoint{1.979706in}{3.284481in}}{\pgfqpoint{1.975316in}{3.273882in}}{\pgfqpoint{1.975316in}{3.262832in}}%
\pgfpathcurveto{\pgfqpoint{1.975316in}{3.251782in}}{\pgfqpoint{1.979706in}{3.241183in}}{\pgfqpoint{1.987520in}{3.233369in}}%
\pgfpathcurveto{\pgfqpoint{1.995334in}{3.225556in}}{\pgfqpoint{2.005933in}{3.221165in}}{\pgfqpoint{2.016983in}{3.221165in}}%
\pgfpathclose%
\pgfusepath{stroke,fill}%
\end{pgfscope}%
\begin{pgfscope}%
\pgfpathrectangle{\pgfqpoint{0.648703in}{0.548769in}}{\pgfqpoint{5.112893in}{3.102590in}}%
\pgfusepath{clip}%
\pgfsetbuttcap%
\pgfsetroundjoin%
\definecolor{currentfill}{rgb}{1.000000,0.498039,0.054902}%
\pgfsetfillcolor{currentfill}%
\pgfsetlinewidth{1.003750pt}%
\definecolor{currentstroke}{rgb}{1.000000,0.498039,0.054902}%
\pgfsetstrokecolor{currentstroke}%
\pgfsetdash{}{0pt}%
\pgfpathmoveto{\pgfqpoint{1.578083in}{3.208790in}}%
\pgfpathcurveto{\pgfqpoint{1.589134in}{3.208790in}}{\pgfqpoint{1.599733in}{3.213181in}}{\pgfqpoint{1.607546in}{3.220994in}}%
\pgfpathcurveto{\pgfqpoint{1.615360in}{3.228808in}}{\pgfqpoint{1.619750in}{3.239407in}}{\pgfqpoint{1.619750in}{3.250457in}}%
\pgfpathcurveto{\pgfqpoint{1.619750in}{3.261507in}}{\pgfqpoint{1.615360in}{3.272106in}}{\pgfqpoint{1.607546in}{3.279920in}}%
\pgfpathcurveto{\pgfqpoint{1.599733in}{3.287733in}}{\pgfqpoint{1.589134in}{3.292124in}}{\pgfqpoint{1.578083in}{3.292124in}}%
\pgfpathcurveto{\pgfqpoint{1.567033in}{3.292124in}}{\pgfqpoint{1.556434in}{3.287733in}}{\pgfqpoint{1.548621in}{3.279920in}}%
\pgfpathcurveto{\pgfqpoint{1.540807in}{3.272106in}}{\pgfqpoint{1.536417in}{3.261507in}}{\pgfqpoint{1.536417in}{3.250457in}}%
\pgfpathcurveto{\pgfqpoint{1.536417in}{3.239407in}}{\pgfqpoint{1.540807in}{3.228808in}}{\pgfqpoint{1.548621in}{3.220994in}}%
\pgfpathcurveto{\pgfqpoint{1.556434in}{3.213181in}}{\pgfqpoint{1.567033in}{3.208790in}}{\pgfqpoint{1.578083in}{3.208790in}}%
\pgfpathclose%
\pgfusepath{stroke,fill}%
\end{pgfscope}%
\begin{pgfscope}%
\pgfpathrectangle{\pgfqpoint{0.648703in}{0.548769in}}{\pgfqpoint{5.112893in}{3.102590in}}%
\pgfusepath{clip}%
\pgfsetbuttcap%
\pgfsetroundjoin%
\definecolor{currentfill}{rgb}{1.000000,0.498039,0.054902}%
\pgfsetfillcolor{currentfill}%
\pgfsetlinewidth{1.003750pt}%
\definecolor{currentstroke}{rgb}{1.000000,0.498039,0.054902}%
\pgfsetstrokecolor{currentstroke}%
\pgfsetdash{}{0pt}%
\pgfpathmoveto{\pgfqpoint{1.524248in}{3.250040in}}%
\pgfpathcurveto{\pgfqpoint{1.535299in}{3.250040in}}{\pgfqpoint{1.545898in}{3.254431in}}{\pgfqpoint{1.553711in}{3.262244in}}%
\pgfpathcurveto{\pgfqpoint{1.561525in}{3.270058in}}{\pgfqpoint{1.565915in}{3.280657in}}{\pgfqpoint{1.565915in}{3.291707in}}%
\pgfpathcurveto{\pgfqpoint{1.565915in}{3.302757in}}{\pgfqpoint{1.561525in}{3.313356in}}{\pgfqpoint{1.553711in}{3.321170in}}%
\pgfpathcurveto{\pgfqpoint{1.545898in}{3.328983in}}{\pgfqpoint{1.535299in}{3.333374in}}{\pgfqpoint{1.524248in}{3.333374in}}%
\pgfpathcurveto{\pgfqpoint{1.513198in}{3.333374in}}{\pgfqpoint{1.502599in}{3.328983in}}{\pgfqpoint{1.494786in}{3.321170in}}%
\pgfpathcurveto{\pgfqpoint{1.486972in}{3.313356in}}{\pgfqpoint{1.482582in}{3.302757in}}{\pgfqpoint{1.482582in}{3.291707in}}%
\pgfpathcurveto{\pgfqpoint{1.482582in}{3.280657in}}{\pgfqpoint{1.486972in}{3.270058in}}{\pgfqpoint{1.494786in}{3.262244in}}%
\pgfpathcurveto{\pgfqpoint{1.502599in}{3.254431in}}{\pgfqpoint{1.513198in}{3.250040in}}{\pgfqpoint{1.524248in}{3.250040in}}%
\pgfpathclose%
\pgfusepath{stroke,fill}%
\end{pgfscope}%
\begin{pgfscope}%
\pgfpathrectangle{\pgfqpoint{0.648703in}{0.548769in}}{\pgfqpoint{5.112893in}{3.102590in}}%
\pgfusepath{clip}%
\pgfsetbuttcap%
\pgfsetroundjoin%
\definecolor{currentfill}{rgb}{1.000000,0.498039,0.054902}%
\pgfsetfillcolor{currentfill}%
\pgfsetlinewidth{1.003750pt}%
\definecolor{currentstroke}{rgb}{1.000000,0.498039,0.054902}%
\pgfsetstrokecolor{currentstroke}%
\pgfsetdash{}{0pt}%
\pgfpathmoveto{\pgfqpoint{1.728455in}{3.316040in}}%
\pgfpathcurveto{\pgfqpoint{1.739505in}{3.316040in}}{\pgfqpoint{1.750105in}{3.320431in}}{\pgfqpoint{1.757918in}{3.328244in}}%
\pgfpathcurveto{\pgfqpoint{1.765732in}{3.336058in}}{\pgfqpoint{1.770122in}{3.346657in}}{\pgfqpoint{1.770122in}{3.357707in}}%
\pgfpathcurveto{\pgfqpoint{1.770122in}{3.368757in}}{\pgfqpoint{1.765732in}{3.379356in}}{\pgfqpoint{1.757918in}{3.387170in}}%
\pgfpathcurveto{\pgfqpoint{1.750105in}{3.394983in}}{\pgfqpoint{1.739505in}{3.399374in}}{\pgfqpoint{1.728455in}{3.399374in}}%
\pgfpathcurveto{\pgfqpoint{1.717405in}{3.399374in}}{\pgfqpoint{1.706806in}{3.394983in}}{\pgfqpoint{1.698993in}{3.387170in}}%
\pgfpathcurveto{\pgfqpoint{1.691179in}{3.379356in}}{\pgfqpoint{1.686789in}{3.368757in}}{\pgfqpoint{1.686789in}{3.357707in}}%
\pgfpathcurveto{\pgfqpoint{1.686789in}{3.346657in}}{\pgfqpoint{1.691179in}{3.336058in}}{\pgfqpoint{1.698993in}{3.328244in}}%
\pgfpathcurveto{\pgfqpoint{1.706806in}{3.320431in}}{\pgfqpoint{1.717405in}{3.316040in}}{\pgfqpoint{1.728455in}{3.316040in}}%
\pgfpathclose%
\pgfusepath{stroke,fill}%
\end{pgfscope}%
\begin{pgfscope}%
\pgfpathrectangle{\pgfqpoint{0.648703in}{0.548769in}}{\pgfqpoint{5.112893in}{3.102590in}}%
\pgfusepath{clip}%
\pgfsetbuttcap%
\pgfsetroundjoin%
\definecolor{currentfill}{rgb}{1.000000,0.498039,0.054902}%
\pgfsetfillcolor{currentfill}%
\pgfsetlinewidth{1.003750pt}%
\definecolor{currentstroke}{rgb}{1.000000,0.498039,0.054902}%
\pgfsetstrokecolor{currentstroke}%
\pgfsetdash{}{0pt}%
\pgfpathmoveto{\pgfqpoint{1.545440in}{3.217040in}}%
\pgfpathcurveto{\pgfqpoint{1.556490in}{3.217040in}}{\pgfqpoint{1.567089in}{3.221431in}}{\pgfqpoint{1.574903in}{3.229244in}}%
\pgfpathcurveto{\pgfqpoint{1.582716in}{3.237058in}}{\pgfqpoint{1.587107in}{3.247657in}}{\pgfqpoint{1.587107in}{3.258707in}}%
\pgfpathcurveto{\pgfqpoint{1.587107in}{3.269757in}}{\pgfqpoint{1.582716in}{3.280356in}}{\pgfqpoint{1.574903in}{3.288170in}}%
\pgfpathcurveto{\pgfqpoint{1.567089in}{3.295983in}}{\pgfqpoint{1.556490in}{3.300374in}}{\pgfqpoint{1.545440in}{3.300374in}}%
\pgfpathcurveto{\pgfqpoint{1.534390in}{3.300374in}}{\pgfqpoint{1.523791in}{3.295983in}}{\pgfqpoint{1.515977in}{3.288170in}}%
\pgfpathcurveto{\pgfqpoint{1.508164in}{3.280356in}}{\pgfqpoint{1.503773in}{3.269757in}}{\pgfqpoint{1.503773in}{3.258707in}}%
\pgfpathcurveto{\pgfqpoint{1.503773in}{3.247657in}}{\pgfqpoint{1.508164in}{3.237058in}}{\pgfqpoint{1.515977in}{3.229244in}}%
\pgfpathcurveto{\pgfqpoint{1.523791in}{3.221431in}}{\pgfqpoint{1.534390in}{3.217040in}}{\pgfqpoint{1.545440in}{3.217040in}}%
\pgfpathclose%
\pgfusepath{stroke,fill}%
\end{pgfscope}%
\begin{pgfscope}%
\pgfpathrectangle{\pgfqpoint{0.648703in}{0.548769in}}{\pgfqpoint{5.112893in}{3.102590in}}%
\pgfusepath{clip}%
\pgfsetbuttcap%
\pgfsetroundjoin%
\definecolor{currentfill}{rgb}{1.000000,0.498039,0.054902}%
\pgfsetfillcolor{currentfill}%
\pgfsetlinewidth{1.003750pt}%
\definecolor{currentstroke}{rgb}{1.000000,0.498039,0.054902}%
\pgfsetstrokecolor{currentstroke}%
\pgfsetdash{}{0pt}%
\pgfpathmoveto{\pgfqpoint{2.032170in}{3.241790in}}%
\pgfpathcurveto{\pgfqpoint{2.043220in}{3.241790in}}{\pgfqpoint{2.053819in}{3.246181in}}{\pgfqpoint{2.061632in}{3.253994in}}%
\pgfpathcurveto{\pgfqpoint{2.069446in}{3.261808in}}{\pgfqpoint{2.073836in}{3.272407in}}{\pgfqpoint{2.073836in}{3.283457in}}%
\pgfpathcurveto{\pgfqpoint{2.073836in}{3.294507in}}{\pgfqpoint{2.069446in}{3.305106in}}{\pgfqpoint{2.061632in}{3.312920in}}%
\pgfpathcurveto{\pgfqpoint{2.053819in}{3.320733in}}{\pgfqpoint{2.043220in}{3.325124in}}{\pgfqpoint{2.032170in}{3.325124in}}%
\pgfpathcurveto{\pgfqpoint{2.021120in}{3.325124in}}{\pgfqpoint{2.010520in}{3.320733in}}{\pgfqpoint{2.002707in}{3.312920in}}%
\pgfpathcurveto{\pgfqpoint{1.994893in}{3.305106in}}{\pgfqpoint{1.990503in}{3.294507in}}{\pgfqpoint{1.990503in}{3.283457in}}%
\pgfpathcurveto{\pgfqpoint{1.990503in}{3.272407in}}{\pgfqpoint{1.994893in}{3.261808in}}{\pgfqpoint{2.002707in}{3.253994in}}%
\pgfpathcurveto{\pgfqpoint{2.010520in}{3.246181in}}{\pgfqpoint{2.021120in}{3.241790in}}{\pgfqpoint{2.032170in}{3.241790in}}%
\pgfpathclose%
\pgfusepath{stroke,fill}%
\end{pgfscope}%
\begin{pgfscope}%
\pgfpathrectangle{\pgfqpoint{0.648703in}{0.548769in}}{\pgfqpoint{5.112893in}{3.102590in}}%
\pgfusepath{clip}%
\pgfsetbuttcap%
\pgfsetroundjoin%
\definecolor{currentfill}{rgb}{0.121569,0.466667,0.705882}%
\pgfsetfillcolor{currentfill}%
\pgfsetlinewidth{1.003750pt}%
\definecolor{currentstroke}{rgb}{0.121569,0.466667,0.705882}%
\pgfsetstrokecolor{currentstroke}%
\pgfsetdash{}{0pt}%
\pgfpathmoveto{\pgfqpoint{0.767676in}{0.717293in}}%
\pgfpathcurveto{\pgfqpoint{0.778726in}{0.717293in}}{\pgfqpoint{0.789325in}{0.721683in}}{\pgfqpoint{0.797139in}{0.729497in}}%
\pgfpathcurveto{\pgfqpoint{0.804952in}{0.737311in}}{\pgfqpoint{0.809343in}{0.747910in}}{\pgfqpoint{0.809343in}{0.758960in}}%
\pgfpathcurveto{\pgfqpoint{0.809343in}{0.770010in}}{\pgfqpoint{0.804952in}{0.780609in}}{\pgfqpoint{0.797139in}{0.788423in}}%
\pgfpathcurveto{\pgfqpoint{0.789325in}{0.796236in}}{\pgfqpoint{0.778726in}{0.800626in}}{\pgfqpoint{0.767676in}{0.800626in}}%
\pgfpathcurveto{\pgfqpoint{0.756626in}{0.800626in}}{\pgfqpoint{0.746027in}{0.796236in}}{\pgfqpoint{0.738213in}{0.788423in}}%
\pgfpathcurveto{\pgfqpoint{0.730400in}{0.780609in}}{\pgfqpoint{0.726009in}{0.770010in}}{\pgfqpoint{0.726009in}{0.758960in}}%
\pgfpathcurveto{\pgfqpoint{0.726009in}{0.747910in}}{\pgfqpoint{0.730400in}{0.737311in}}{\pgfqpoint{0.738213in}{0.729497in}}%
\pgfpathcurveto{\pgfqpoint{0.746027in}{0.721683in}}{\pgfqpoint{0.756626in}{0.717293in}}{\pgfqpoint{0.767676in}{0.717293in}}%
\pgfpathclose%
\pgfusepath{stroke,fill}%
\end{pgfscope}%
\begin{pgfscope}%
\pgfpathrectangle{\pgfqpoint{0.648703in}{0.548769in}}{\pgfqpoint{5.112893in}{3.102590in}}%
\pgfusepath{clip}%
\pgfsetbuttcap%
\pgfsetroundjoin%
\definecolor{currentfill}{rgb}{0.121569,0.466667,0.705882}%
\pgfsetfillcolor{currentfill}%
\pgfsetlinewidth{1.003750pt}%
\definecolor{currentstroke}{rgb}{0.121569,0.466667,0.705882}%
\pgfsetstrokecolor{currentstroke}%
\pgfsetdash{}{0pt}%
\pgfpathmoveto{\pgfqpoint{0.767698in}{0.717293in}}%
\pgfpathcurveto{\pgfqpoint{0.778749in}{0.717293in}}{\pgfqpoint{0.789348in}{0.721683in}}{\pgfqpoint{0.797161in}{0.729497in}}%
\pgfpathcurveto{\pgfqpoint{0.804975in}{0.737311in}}{\pgfqpoint{0.809365in}{0.747910in}}{\pgfqpoint{0.809365in}{0.758960in}}%
\pgfpathcurveto{\pgfqpoint{0.809365in}{0.770010in}}{\pgfqpoint{0.804975in}{0.780609in}}{\pgfqpoint{0.797161in}{0.788423in}}%
\pgfpathcurveto{\pgfqpoint{0.789348in}{0.796236in}}{\pgfqpoint{0.778749in}{0.800626in}}{\pgfqpoint{0.767698in}{0.800626in}}%
\pgfpathcurveto{\pgfqpoint{0.756648in}{0.800626in}}{\pgfqpoint{0.746049in}{0.796236in}}{\pgfqpoint{0.738236in}{0.788423in}}%
\pgfpathcurveto{\pgfqpoint{0.730422in}{0.780609in}}{\pgfqpoint{0.726032in}{0.770010in}}{\pgfqpoint{0.726032in}{0.758960in}}%
\pgfpathcurveto{\pgfqpoint{0.726032in}{0.747910in}}{\pgfqpoint{0.730422in}{0.737311in}}{\pgfqpoint{0.738236in}{0.729497in}}%
\pgfpathcurveto{\pgfqpoint{0.746049in}{0.721683in}}{\pgfqpoint{0.756648in}{0.717293in}}{\pgfqpoint{0.767698in}{0.717293in}}%
\pgfpathclose%
\pgfusepath{stroke,fill}%
\end{pgfscope}%
\begin{pgfscope}%
\pgfpathrectangle{\pgfqpoint{0.648703in}{0.548769in}}{\pgfqpoint{5.112893in}{3.102590in}}%
\pgfusepath{clip}%
\pgfsetbuttcap%
\pgfsetroundjoin%
\definecolor{currentfill}{rgb}{0.121569,0.466667,0.705882}%
\pgfsetfillcolor{currentfill}%
\pgfsetlinewidth{1.003750pt}%
\definecolor{currentstroke}{rgb}{0.121569,0.466667,0.705882}%
\pgfsetstrokecolor{currentstroke}%
\pgfsetdash{}{0pt}%
\pgfpathmoveto{\pgfqpoint{0.767693in}{0.717293in}}%
\pgfpathcurveto{\pgfqpoint{0.778743in}{0.717293in}}{\pgfqpoint{0.789342in}{0.721683in}}{\pgfqpoint{0.797156in}{0.729497in}}%
\pgfpathcurveto{\pgfqpoint{0.804970in}{0.737311in}}{\pgfqpoint{0.809360in}{0.747910in}}{\pgfqpoint{0.809360in}{0.758960in}}%
\pgfpathcurveto{\pgfqpoint{0.809360in}{0.770010in}}{\pgfqpoint{0.804970in}{0.780609in}}{\pgfqpoint{0.797156in}{0.788423in}}%
\pgfpathcurveto{\pgfqpoint{0.789342in}{0.796236in}}{\pgfqpoint{0.778743in}{0.800626in}}{\pgfqpoint{0.767693in}{0.800626in}}%
\pgfpathcurveto{\pgfqpoint{0.756643in}{0.800626in}}{\pgfqpoint{0.746044in}{0.796236in}}{\pgfqpoint{0.738230in}{0.788423in}}%
\pgfpathcurveto{\pgfqpoint{0.730417in}{0.780609in}}{\pgfqpoint{0.726027in}{0.770010in}}{\pgfqpoint{0.726027in}{0.758960in}}%
\pgfpathcurveto{\pgfqpoint{0.726027in}{0.747910in}}{\pgfqpoint{0.730417in}{0.737311in}}{\pgfqpoint{0.738230in}{0.729497in}}%
\pgfpathcurveto{\pgfqpoint{0.746044in}{0.721683in}}{\pgfqpoint{0.756643in}{0.717293in}}{\pgfqpoint{0.767693in}{0.717293in}}%
\pgfpathclose%
\pgfusepath{stroke,fill}%
\end{pgfscope}%
\begin{pgfscope}%
\pgfpathrectangle{\pgfqpoint{0.648703in}{0.548769in}}{\pgfqpoint{5.112893in}{3.102590in}}%
\pgfusepath{clip}%
\pgfsetbuttcap%
\pgfsetroundjoin%
\definecolor{currentfill}{rgb}{1.000000,0.498039,0.054902}%
\pgfsetfillcolor{currentfill}%
\pgfsetlinewidth{1.003750pt}%
\definecolor{currentstroke}{rgb}{1.000000,0.498039,0.054902}%
\pgfsetstrokecolor{currentstroke}%
\pgfsetdash{}{0pt}%
\pgfpathmoveto{\pgfqpoint{1.944325in}{3.406790in}}%
\pgfpathcurveto{\pgfqpoint{1.955375in}{3.406790in}}{\pgfqpoint{1.965974in}{3.411180in}}{\pgfqpoint{1.973788in}{3.418994in}}%
\pgfpathcurveto{\pgfqpoint{1.981602in}{3.426808in}}{\pgfqpoint{1.985992in}{3.437407in}}{\pgfqpoint{1.985992in}{3.448457in}}%
\pgfpathcurveto{\pgfqpoint{1.985992in}{3.459507in}}{\pgfqpoint{1.981602in}{3.470106in}}{\pgfqpoint{1.973788in}{3.477920in}}%
\pgfpathcurveto{\pgfqpoint{1.965974in}{3.485733in}}{\pgfqpoint{1.955375in}{3.490124in}}{\pgfqpoint{1.944325in}{3.490124in}}%
\pgfpathcurveto{\pgfqpoint{1.933275in}{3.490124in}}{\pgfqpoint{1.922676in}{3.485733in}}{\pgfqpoint{1.914863in}{3.477920in}}%
\pgfpathcurveto{\pgfqpoint{1.907049in}{3.470106in}}{\pgfqpoint{1.902659in}{3.459507in}}{\pgfqpoint{1.902659in}{3.448457in}}%
\pgfpathcurveto{\pgfqpoint{1.902659in}{3.437407in}}{\pgfqpoint{1.907049in}{3.426808in}}{\pgfqpoint{1.914863in}{3.418994in}}%
\pgfpathcurveto{\pgfqpoint{1.922676in}{3.411180in}}{\pgfqpoint{1.933275in}{3.406790in}}{\pgfqpoint{1.944325in}{3.406790in}}%
\pgfpathclose%
\pgfusepath{stroke,fill}%
\end{pgfscope}%
\begin{pgfscope}%
\pgfpathrectangle{\pgfqpoint{0.648703in}{0.548769in}}{\pgfqpoint{5.112893in}{3.102590in}}%
\pgfusepath{clip}%
\pgfsetbuttcap%
\pgfsetroundjoin%
\definecolor{currentfill}{rgb}{1.000000,0.498039,0.054902}%
\pgfsetfillcolor{currentfill}%
\pgfsetlinewidth{1.003750pt}%
\definecolor{currentstroke}{rgb}{1.000000,0.498039,0.054902}%
\pgfsetstrokecolor{currentstroke}%
\pgfsetdash{}{0pt}%
\pgfpathmoveto{\pgfqpoint{1.669308in}{3.200540in}}%
\pgfpathcurveto{\pgfqpoint{1.680358in}{3.200540in}}{\pgfqpoint{1.690957in}{3.204931in}}{\pgfqpoint{1.698771in}{3.212744in}}%
\pgfpathcurveto{\pgfqpoint{1.706584in}{3.220558in}}{\pgfqpoint{1.710974in}{3.231157in}}{\pgfqpoint{1.710974in}{3.242207in}}%
\pgfpathcurveto{\pgfqpoint{1.710974in}{3.253257in}}{\pgfqpoint{1.706584in}{3.263856in}}{\pgfqpoint{1.698771in}{3.271670in}}%
\pgfpathcurveto{\pgfqpoint{1.690957in}{3.279483in}}{\pgfqpoint{1.680358in}{3.283874in}}{\pgfqpoint{1.669308in}{3.283874in}}%
\pgfpathcurveto{\pgfqpoint{1.658258in}{3.283874in}}{\pgfqpoint{1.647659in}{3.279483in}}{\pgfqpoint{1.639845in}{3.271670in}}%
\pgfpathcurveto{\pgfqpoint{1.632031in}{3.263856in}}{\pgfqpoint{1.627641in}{3.253257in}}{\pgfqpoint{1.627641in}{3.242207in}}%
\pgfpathcurveto{\pgfqpoint{1.627641in}{3.231157in}}{\pgfqpoint{1.632031in}{3.220558in}}{\pgfqpoint{1.639845in}{3.212744in}}%
\pgfpathcurveto{\pgfqpoint{1.647659in}{3.204931in}}{\pgfqpoint{1.658258in}{3.200540in}}{\pgfqpoint{1.669308in}{3.200540in}}%
\pgfpathclose%
\pgfusepath{stroke,fill}%
\end{pgfscope}%
\begin{pgfscope}%
\pgfpathrectangle{\pgfqpoint{0.648703in}{0.548769in}}{\pgfqpoint{5.112893in}{3.102590in}}%
\pgfusepath{clip}%
\pgfsetbuttcap%
\pgfsetroundjoin%
\definecolor{currentfill}{rgb}{1.000000,0.498039,0.054902}%
\pgfsetfillcolor{currentfill}%
\pgfsetlinewidth{1.003750pt}%
\definecolor{currentstroke}{rgb}{1.000000,0.498039,0.054902}%
\pgfsetstrokecolor{currentstroke}%
\pgfsetdash{}{0pt}%
\pgfpathmoveto{\pgfqpoint{1.579457in}{3.361415in}}%
\pgfpathcurveto{\pgfqpoint{1.590507in}{3.361415in}}{\pgfqpoint{1.601106in}{3.365806in}}{\pgfqpoint{1.608920in}{3.373619in}}%
\pgfpathcurveto{\pgfqpoint{1.616733in}{3.381433in}}{\pgfqpoint{1.621123in}{3.392032in}}{\pgfqpoint{1.621123in}{3.403082in}}%
\pgfpathcurveto{\pgfqpoint{1.621123in}{3.414132in}}{\pgfqpoint{1.616733in}{3.424731in}}{\pgfqpoint{1.608920in}{3.432545in}}%
\pgfpathcurveto{\pgfqpoint{1.601106in}{3.440358in}}{\pgfqpoint{1.590507in}{3.444749in}}{\pgfqpoint{1.579457in}{3.444749in}}%
\pgfpathcurveto{\pgfqpoint{1.568407in}{3.444749in}}{\pgfqpoint{1.557808in}{3.440358in}}{\pgfqpoint{1.549994in}{3.432545in}}%
\pgfpathcurveto{\pgfqpoint{1.542180in}{3.424731in}}{\pgfqpoint{1.537790in}{3.414132in}}{\pgfqpoint{1.537790in}{3.403082in}}%
\pgfpathcurveto{\pgfqpoint{1.537790in}{3.392032in}}{\pgfqpoint{1.542180in}{3.381433in}}{\pgfqpoint{1.549994in}{3.373619in}}%
\pgfpathcurveto{\pgfqpoint{1.557808in}{3.365806in}}{\pgfqpoint{1.568407in}{3.361415in}}{\pgfqpoint{1.579457in}{3.361415in}}%
\pgfpathclose%
\pgfusepath{stroke,fill}%
\end{pgfscope}%
\begin{pgfscope}%
\pgfpathrectangle{\pgfqpoint{0.648703in}{0.548769in}}{\pgfqpoint{5.112893in}{3.102590in}}%
\pgfusepath{clip}%
\pgfsetbuttcap%
\pgfsetroundjoin%
\definecolor{currentfill}{rgb}{0.121569,0.466667,0.705882}%
\pgfsetfillcolor{currentfill}%
\pgfsetlinewidth{1.003750pt}%
\definecolor{currentstroke}{rgb}{0.121569,0.466667,0.705882}%
\pgfsetstrokecolor{currentstroke}%
\pgfsetdash{}{0pt}%
\pgfpathmoveto{\pgfqpoint{0.767676in}{0.717293in}}%
\pgfpathcurveto{\pgfqpoint{0.778726in}{0.717293in}}{\pgfqpoint{0.789325in}{0.721683in}}{\pgfqpoint{0.797139in}{0.729497in}}%
\pgfpathcurveto{\pgfqpoint{0.804952in}{0.737311in}}{\pgfqpoint{0.809342in}{0.747910in}}{\pgfqpoint{0.809342in}{0.758960in}}%
\pgfpathcurveto{\pgfqpoint{0.809342in}{0.770010in}}{\pgfqpoint{0.804952in}{0.780609in}}{\pgfqpoint{0.797139in}{0.788423in}}%
\pgfpathcurveto{\pgfqpoint{0.789325in}{0.796236in}}{\pgfqpoint{0.778726in}{0.800626in}}{\pgfqpoint{0.767676in}{0.800626in}}%
\pgfpathcurveto{\pgfqpoint{0.756626in}{0.800626in}}{\pgfqpoint{0.746027in}{0.796236in}}{\pgfqpoint{0.738213in}{0.788423in}}%
\pgfpathcurveto{\pgfqpoint{0.730399in}{0.780609in}}{\pgfqpoint{0.726009in}{0.770010in}}{\pgfqpoint{0.726009in}{0.758960in}}%
\pgfpathcurveto{\pgfqpoint{0.726009in}{0.747910in}}{\pgfqpoint{0.730399in}{0.737311in}}{\pgfqpoint{0.738213in}{0.729497in}}%
\pgfpathcurveto{\pgfqpoint{0.746027in}{0.721683in}}{\pgfqpoint{0.756626in}{0.717293in}}{\pgfqpoint{0.767676in}{0.717293in}}%
\pgfpathclose%
\pgfusepath{stroke,fill}%
\end{pgfscope}%
\begin{pgfscope}%
\pgfpathrectangle{\pgfqpoint{0.648703in}{0.548769in}}{\pgfqpoint{5.112893in}{3.102590in}}%
\pgfusepath{clip}%
\pgfsetbuttcap%
\pgfsetroundjoin%
\definecolor{currentfill}{rgb}{0.121569,0.466667,0.705882}%
\pgfsetfillcolor{currentfill}%
\pgfsetlinewidth{1.003750pt}%
\definecolor{currentstroke}{rgb}{0.121569,0.466667,0.705882}%
\pgfsetstrokecolor{currentstroke}%
\pgfsetdash{}{0pt}%
\pgfpathmoveto{\pgfqpoint{0.826549in}{0.861668in}}%
\pgfpathcurveto{\pgfqpoint{0.837599in}{0.861668in}}{\pgfqpoint{0.848198in}{0.866058in}}{\pgfqpoint{0.856012in}{0.873872in}}%
\pgfpathcurveto{\pgfqpoint{0.863825in}{0.881685in}}{\pgfqpoint{0.868216in}{0.892284in}}{\pgfqpoint{0.868216in}{0.903335in}}%
\pgfpathcurveto{\pgfqpoint{0.868216in}{0.914385in}}{\pgfqpoint{0.863825in}{0.924984in}}{\pgfqpoint{0.856012in}{0.932797in}}%
\pgfpathcurveto{\pgfqpoint{0.848198in}{0.940611in}}{\pgfqpoint{0.837599in}{0.945001in}}{\pgfqpoint{0.826549in}{0.945001in}}%
\pgfpathcurveto{\pgfqpoint{0.815499in}{0.945001in}}{\pgfqpoint{0.804900in}{0.940611in}}{\pgfqpoint{0.797086in}{0.932797in}}%
\pgfpathcurveto{\pgfqpoint{0.789272in}{0.924984in}}{\pgfqpoint{0.784882in}{0.914385in}}{\pgfqpoint{0.784882in}{0.903335in}}%
\pgfpathcurveto{\pgfqpoint{0.784882in}{0.892284in}}{\pgfqpoint{0.789272in}{0.881685in}}{\pgfqpoint{0.797086in}{0.873872in}}%
\pgfpathcurveto{\pgfqpoint{0.804900in}{0.866058in}}{\pgfqpoint{0.815499in}{0.861668in}}{\pgfqpoint{0.826549in}{0.861668in}}%
\pgfpathclose%
\pgfusepath{stroke,fill}%
\end{pgfscope}%
\begin{pgfscope}%
\pgfpathrectangle{\pgfqpoint{0.648703in}{0.548769in}}{\pgfqpoint{5.112893in}{3.102590in}}%
\pgfusepath{clip}%
\pgfsetbuttcap%
\pgfsetroundjoin%
\definecolor{currentfill}{rgb}{0.121569,0.466667,0.705882}%
\pgfsetfillcolor{currentfill}%
\pgfsetlinewidth{1.003750pt}%
\definecolor{currentstroke}{rgb}{0.121569,0.466667,0.705882}%
\pgfsetstrokecolor{currentstroke}%
\pgfsetdash{}{0pt}%
\pgfpathmoveto{\pgfqpoint{1.764736in}{3.163415in}}%
\pgfpathcurveto{\pgfqpoint{1.775786in}{3.163415in}}{\pgfqpoint{1.786385in}{3.167806in}}{\pgfqpoint{1.794199in}{3.175619in}}%
\pgfpathcurveto{\pgfqpoint{1.802013in}{3.183433in}}{\pgfqpoint{1.806403in}{3.194032in}}{\pgfqpoint{1.806403in}{3.205082in}}%
\pgfpathcurveto{\pgfqpoint{1.806403in}{3.216132in}}{\pgfqpoint{1.802013in}{3.226731in}}{\pgfqpoint{1.794199in}{3.234545in}}%
\pgfpathcurveto{\pgfqpoint{1.786385in}{3.242359in}}{\pgfqpoint{1.775786in}{3.246749in}}{\pgfqpoint{1.764736in}{3.246749in}}%
\pgfpathcurveto{\pgfqpoint{1.753686in}{3.246749in}}{\pgfqpoint{1.743087in}{3.242359in}}{\pgfqpoint{1.735273in}{3.234545in}}%
\pgfpathcurveto{\pgfqpoint{1.727460in}{3.226731in}}{\pgfqpoint{1.723070in}{3.216132in}}{\pgfqpoint{1.723070in}{3.205082in}}%
\pgfpathcurveto{\pgfqpoint{1.723070in}{3.194032in}}{\pgfqpoint{1.727460in}{3.183433in}}{\pgfqpoint{1.735273in}{3.175619in}}%
\pgfpathcurveto{\pgfqpoint{1.743087in}{3.167806in}}{\pgfqpoint{1.753686in}{3.163415in}}{\pgfqpoint{1.764736in}{3.163415in}}%
\pgfpathclose%
\pgfusepath{stroke,fill}%
\end{pgfscope}%
\begin{pgfscope}%
\pgfpathrectangle{\pgfqpoint{0.648703in}{0.548769in}}{\pgfqpoint{5.112893in}{3.102590in}}%
\pgfusepath{clip}%
\pgfsetbuttcap%
\pgfsetroundjoin%
\definecolor{currentfill}{rgb}{0.121569,0.466667,0.705882}%
\pgfsetfillcolor{currentfill}%
\pgfsetlinewidth{1.003750pt}%
\definecolor{currentstroke}{rgb}{0.121569,0.466667,0.705882}%
\pgfsetstrokecolor{currentstroke}%
\pgfsetdash{}{0pt}%
\pgfpathmoveto{\pgfqpoint{0.767688in}{0.717293in}}%
\pgfpathcurveto{\pgfqpoint{0.778738in}{0.717293in}}{\pgfqpoint{0.789337in}{0.721683in}}{\pgfqpoint{0.797151in}{0.729497in}}%
\pgfpathcurveto{\pgfqpoint{0.804964in}{0.737311in}}{\pgfqpoint{0.809355in}{0.747910in}}{\pgfqpoint{0.809355in}{0.758960in}}%
\pgfpathcurveto{\pgfqpoint{0.809355in}{0.770010in}}{\pgfqpoint{0.804964in}{0.780609in}}{\pgfqpoint{0.797151in}{0.788423in}}%
\pgfpathcurveto{\pgfqpoint{0.789337in}{0.796236in}}{\pgfqpoint{0.778738in}{0.800626in}}{\pgfqpoint{0.767688in}{0.800626in}}%
\pgfpathcurveto{\pgfqpoint{0.756638in}{0.800626in}}{\pgfqpoint{0.746039in}{0.796236in}}{\pgfqpoint{0.738225in}{0.788423in}}%
\pgfpathcurveto{\pgfqpoint{0.730412in}{0.780609in}}{\pgfqpoint{0.726021in}{0.770010in}}{\pgfqpoint{0.726021in}{0.758960in}}%
\pgfpathcurveto{\pgfqpoint{0.726021in}{0.747910in}}{\pgfqpoint{0.730412in}{0.737311in}}{\pgfqpoint{0.738225in}{0.729497in}}%
\pgfpathcurveto{\pgfqpoint{0.746039in}{0.721683in}}{\pgfqpoint{0.756638in}{0.717293in}}{\pgfqpoint{0.767688in}{0.717293in}}%
\pgfpathclose%
\pgfusepath{stroke,fill}%
\end{pgfscope}%
\begin{pgfscope}%
\pgfpathrectangle{\pgfqpoint{0.648703in}{0.548769in}}{\pgfqpoint{5.112893in}{3.102590in}}%
\pgfusepath{clip}%
\pgfsetbuttcap%
\pgfsetroundjoin%
\definecolor{currentfill}{rgb}{0.121569,0.466667,0.705882}%
\pgfsetfillcolor{currentfill}%
\pgfsetlinewidth{1.003750pt}%
\definecolor{currentstroke}{rgb}{0.121569,0.466667,0.705882}%
\pgfsetstrokecolor{currentstroke}%
\pgfsetdash{}{0pt}%
\pgfpathmoveto{\pgfqpoint{0.767691in}{0.717293in}}%
\pgfpathcurveto{\pgfqpoint{0.778741in}{0.717293in}}{\pgfqpoint{0.789340in}{0.721683in}}{\pgfqpoint{0.797154in}{0.729497in}}%
\pgfpathcurveto{\pgfqpoint{0.804968in}{0.737311in}}{\pgfqpoint{0.809358in}{0.747910in}}{\pgfqpoint{0.809358in}{0.758960in}}%
\pgfpathcurveto{\pgfqpoint{0.809358in}{0.770010in}}{\pgfqpoint{0.804968in}{0.780609in}}{\pgfqpoint{0.797154in}{0.788423in}}%
\pgfpathcurveto{\pgfqpoint{0.789340in}{0.796236in}}{\pgfqpoint{0.778741in}{0.800626in}}{\pgfqpoint{0.767691in}{0.800626in}}%
\pgfpathcurveto{\pgfqpoint{0.756641in}{0.800626in}}{\pgfqpoint{0.746042in}{0.796236in}}{\pgfqpoint{0.738228in}{0.788423in}}%
\pgfpathcurveto{\pgfqpoint{0.730415in}{0.780609in}}{\pgfqpoint{0.726024in}{0.770010in}}{\pgfqpoint{0.726024in}{0.758960in}}%
\pgfpathcurveto{\pgfqpoint{0.726024in}{0.747910in}}{\pgfqpoint{0.730415in}{0.737311in}}{\pgfqpoint{0.738228in}{0.729497in}}%
\pgfpathcurveto{\pgfqpoint{0.746042in}{0.721683in}}{\pgfqpoint{0.756641in}{0.717293in}}{\pgfqpoint{0.767691in}{0.717293in}}%
\pgfpathclose%
\pgfusepath{stroke,fill}%
\end{pgfscope}%
\begin{pgfscope}%
\pgfpathrectangle{\pgfqpoint{0.648703in}{0.548769in}}{\pgfqpoint{5.112893in}{3.102590in}}%
\pgfusepath{clip}%
\pgfsetbuttcap%
\pgfsetroundjoin%
\definecolor{currentfill}{rgb}{0.121569,0.466667,0.705882}%
\pgfsetfillcolor{currentfill}%
\pgfsetlinewidth{1.003750pt}%
\definecolor{currentstroke}{rgb}{0.121569,0.466667,0.705882}%
\pgfsetstrokecolor{currentstroke}%
\pgfsetdash{}{0pt}%
\pgfpathmoveto{\pgfqpoint{1.785014in}{2.536416in}}%
\pgfpathcurveto{\pgfqpoint{1.796064in}{2.536416in}}{\pgfqpoint{1.806663in}{2.540806in}}{\pgfqpoint{1.814477in}{2.548620in}}%
\pgfpathcurveto{\pgfqpoint{1.822291in}{2.556434in}}{\pgfqpoint{1.826681in}{2.567033in}}{\pgfqpoint{1.826681in}{2.578083in}}%
\pgfpathcurveto{\pgfqpoint{1.826681in}{2.589133in}}{\pgfqpoint{1.822291in}{2.599732in}}{\pgfqpoint{1.814477in}{2.607546in}}%
\pgfpathcurveto{\pgfqpoint{1.806663in}{2.615359in}}{\pgfqpoint{1.796064in}{2.619749in}}{\pgfqpoint{1.785014in}{2.619749in}}%
\pgfpathcurveto{\pgfqpoint{1.773964in}{2.619749in}}{\pgfqpoint{1.763365in}{2.615359in}}{\pgfqpoint{1.755551in}{2.607546in}}%
\pgfpathcurveto{\pgfqpoint{1.747738in}{2.599732in}}{\pgfqpoint{1.743348in}{2.589133in}}{\pgfqpoint{1.743348in}{2.578083in}}%
\pgfpathcurveto{\pgfqpoint{1.743348in}{2.567033in}}{\pgfqpoint{1.747738in}{2.556434in}}{\pgfqpoint{1.755551in}{2.548620in}}%
\pgfpathcurveto{\pgfqpoint{1.763365in}{2.540806in}}{\pgfqpoint{1.773964in}{2.536416in}}{\pgfqpoint{1.785014in}{2.536416in}}%
\pgfpathclose%
\pgfusepath{stroke,fill}%
\end{pgfscope}%
\begin{pgfscope}%
\pgfpathrectangle{\pgfqpoint{0.648703in}{0.548769in}}{\pgfqpoint{5.112893in}{3.102590in}}%
\pgfusepath{clip}%
\pgfsetbuttcap%
\pgfsetroundjoin%
\definecolor{currentfill}{rgb}{1.000000,0.498039,0.054902}%
\pgfsetfillcolor{currentfill}%
\pgfsetlinewidth{1.003750pt}%
\definecolor{currentstroke}{rgb}{1.000000,0.498039,0.054902}%
\pgfsetstrokecolor{currentstroke}%
\pgfsetdash{}{0pt}%
\pgfpathmoveto{\pgfqpoint{1.533698in}{3.212915in}}%
\pgfpathcurveto{\pgfqpoint{1.544748in}{3.212915in}}{\pgfqpoint{1.555347in}{3.217306in}}{\pgfqpoint{1.563161in}{3.225119in}}%
\pgfpathcurveto{\pgfqpoint{1.570974in}{3.232933in}}{\pgfqpoint{1.575364in}{3.243532in}}{\pgfqpoint{1.575364in}{3.254582in}}%
\pgfpathcurveto{\pgfqpoint{1.575364in}{3.265632in}}{\pgfqpoint{1.570974in}{3.276231in}}{\pgfqpoint{1.563161in}{3.284045in}}%
\pgfpathcurveto{\pgfqpoint{1.555347in}{3.291858in}}{\pgfqpoint{1.544748in}{3.296249in}}{\pgfqpoint{1.533698in}{3.296249in}}%
\pgfpathcurveto{\pgfqpoint{1.522648in}{3.296249in}}{\pgfqpoint{1.512049in}{3.291858in}}{\pgfqpoint{1.504235in}{3.284045in}}%
\pgfpathcurveto{\pgfqpoint{1.496421in}{3.276231in}}{\pgfqpoint{1.492031in}{3.265632in}}{\pgfqpoint{1.492031in}{3.254582in}}%
\pgfpathcurveto{\pgfqpoint{1.492031in}{3.243532in}}{\pgfqpoint{1.496421in}{3.232933in}}{\pgfqpoint{1.504235in}{3.225119in}}%
\pgfpathcurveto{\pgfqpoint{1.512049in}{3.217306in}}{\pgfqpoint{1.522648in}{3.212915in}}{\pgfqpoint{1.533698in}{3.212915in}}%
\pgfpathclose%
\pgfusepath{stroke,fill}%
\end{pgfscope}%
\begin{pgfscope}%
\pgfpathrectangle{\pgfqpoint{0.648703in}{0.548769in}}{\pgfqpoint{5.112893in}{3.102590in}}%
\pgfusepath{clip}%
\pgfsetbuttcap%
\pgfsetroundjoin%
\definecolor{currentfill}{rgb}{1.000000,0.498039,0.054902}%
\pgfsetfillcolor{currentfill}%
\pgfsetlinewidth{1.003750pt}%
\definecolor{currentstroke}{rgb}{1.000000,0.498039,0.054902}%
\pgfsetstrokecolor{currentstroke}%
\pgfsetdash{}{0pt}%
\pgfpathmoveto{\pgfqpoint{1.618869in}{3.204665in}}%
\pgfpathcurveto{\pgfqpoint{1.629919in}{3.204665in}}{\pgfqpoint{1.640518in}{3.209056in}}{\pgfqpoint{1.648332in}{3.216869in}}%
\pgfpathcurveto{\pgfqpoint{1.656146in}{3.224683in}}{\pgfqpoint{1.660536in}{3.235282in}}{\pgfqpoint{1.660536in}{3.246332in}}%
\pgfpathcurveto{\pgfqpoint{1.660536in}{3.257382in}}{\pgfqpoint{1.656146in}{3.267981in}}{\pgfqpoint{1.648332in}{3.275795in}}%
\pgfpathcurveto{\pgfqpoint{1.640518in}{3.283608in}}{\pgfqpoint{1.629919in}{3.287999in}}{\pgfqpoint{1.618869in}{3.287999in}}%
\pgfpathcurveto{\pgfqpoint{1.607819in}{3.287999in}}{\pgfqpoint{1.597220in}{3.283608in}}{\pgfqpoint{1.589407in}{3.275795in}}%
\pgfpathcurveto{\pgfqpoint{1.581593in}{3.267981in}}{\pgfqpoint{1.577203in}{3.257382in}}{\pgfqpoint{1.577203in}{3.246332in}}%
\pgfpathcurveto{\pgfqpoint{1.577203in}{3.235282in}}{\pgfqpoint{1.581593in}{3.224683in}}{\pgfqpoint{1.589407in}{3.216869in}}%
\pgfpathcurveto{\pgfqpoint{1.597220in}{3.209056in}}{\pgfqpoint{1.607819in}{3.204665in}}{\pgfqpoint{1.618869in}{3.204665in}}%
\pgfpathclose%
\pgfusepath{stroke,fill}%
\end{pgfscope}%
\begin{pgfscope}%
\pgfpathrectangle{\pgfqpoint{0.648703in}{0.548769in}}{\pgfqpoint{5.112893in}{3.102590in}}%
\pgfusepath{clip}%
\pgfsetbuttcap%
\pgfsetroundjoin%
\definecolor{currentfill}{rgb}{1.000000,0.498039,0.054902}%
\pgfsetfillcolor{currentfill}%
\pgfsetlinewidth{1.003750pt}%
\definecolor{currentstroke}{rgb}{1.000000,0.498039,0.054902}%
\pgfsetstrokecolor{currentstroke}%
\pgfsetdash{}{0pt}%
\pgfpathmoveto{\pgfqpoint{1.928786in}{3.254165in}}%
\pgfpathcurveto{\pgfqpoint{1.939836in}{3.254165in}}{\pgfqpoint{1.950435in}{3.258556in}}{\pgfqpoint{1.958249in}{3.266369in}}%
\pgfpathcurveto{\pgfqpoint{1.966062in}{3.274183in}}{\pgfqpoint{1.970452in}{3.284782in}}{\pgfqpoint{1.970452in}{3.295832in}}%
\pgfpathcurveto{\pgfqpoint{1.970452in}{3.306882in}}{\pgfqpoint{1.966062in}{3.317481in}}{\pgfqpoint{1.958249in}{3.325295in}}%
\pgfpathcurveto{\pgfqpoint{1.950435in}{3.333108in}}{\pgfqpoint{1.939836in}{3.337499in}}{\pgfqpoint{1.928786in}{3.337499in}}%
\pgfpathcurveto{\pgfqpoint{1.917736in}{3.337499in}}{\pgfqpoint{1.907137in}{3.333108in}}{\pgfqpoint{1.899323in}{3.325295in}}%
\pgfpathcurveto{\pgfqpoint{1.891509in}{3.317481in}}{\pgfqpoint{1.887119in}{3.306882in}}{\pgfqpoint{1.887119in}{3.295832in}}%
\pgfpathcurveto{\pgfqpoint{1.887119in}{3.284782in}}{\pgfqpoint{1.891509in}{3.274183in}}{\pgfqpoint{1.899323in}{3.266369in}}%
\pgfpathcurveto{\pgfqpoint{1.907137in}{3.258556in}}{\pgfqpoint{1.917736in}{3.254165in}}{\pgfqpoint{1.928786in}{3.254165in}}%
\pgfpathclose%
\pgfusepath{stroke,fill}%
\end{pgfscope}%
\begin{pgfscope}%
\pgfpathrectangle{\pgfqpoint{0.648703in}{0.548769in}}{\pgfqpoint{5.112893in}{3.102590in}}%
\pgfusepath{clip}%
\pgfsetbuttcap%
\pgfsetroundjoin%
\definecolor{currentfill}{rgb}{1.000000,0.498039,0.054902}%
\pgfsetfillcolor{currentfill}%
\pgfsetlinewidth{1.003750pt}%
\definecolor{currentstroke}{rgb}{1.000000,0.498039,0.054902}%
\pgfsetstrokecolor{currentstroke}%
\pgfsetdash{}{0pt}%
\pgfpathmoveto{\pgfqpoint{1.486688in}{3.196415in}}%
\pgfpathcurveto{\pgfqpoint{1.497738in}{3.196415in}}{\pgfqpoint{1.508337in}{3.200806in}}{\pgfqpoint{1.516151in}{3.208619in}}%
\pgfpathcurveto{\pgfqpoint{1.523964in}{3.216433in}}{\pgfqpoint{1.528355in}{3.227032in}}{\pgfqpoint{1.528355in}{3.238082in}}%
\pgfpathcurveto{\pgfqpoint{1.528355in}{3.249132in}}{\pgfqpoint{1.523964in}{3.259731in}}{\pgfqpoint{1.516151in}{3.267545in}}%
\pgfpathcurveto{\pgfqpoint{1.508337in}{3.275358in}}{\pgfqpoint{1.497738in}{3.279749in}}{\pgfqpoint{1.486688in}{3.279749in}}%
\pgfpathcurveto{\pgfqpoint{1.475638in}{3.279749in}}{\pgfqpoint{1.465039in}{3.275358in}}{\pgfqpoint{1.457225in}{3.267545in}}%
\pgfpathcurveto{\pgfqpoint{1.449412in}{3.259731in}}{\pgfqpoint{1.445021in}{3.249132in}}{\pgfqpoint{1.445021in}{3.238082in}}%
\pgfpathcurveto{\pgfqpoint{1.445021in}{3.227032in}}{\pgfqpoint{1.449412in}{3.216433in}}{\pgfqpoint{1.457225in}{3.208619in}}%
\pgfpathcurveto{\pgfqpoint{1.465039in}{3.200806in}}{\pgfqpoint{1.475638in}{3.196415in}}{\pgfqpoint{1.486688in}{3.196415in}}%
\pgfpathclose%
\pgfusepath{stroke,fill}%
\end{pgfscope}%
\begin{pgfscope}%
\pgfpathrectangle{\pgfqpoint{0.648703in}{0.548769in}}{\pgfqpoint{5.112893in}{3.102590in}}%
\pgfusepath{clip}%
\pgfsetbuttcap%
\pgfsetroundjoin%
\definecolor{currentfill}{rgb}{0.121569,0.466667,0.705882}%
\pgfsetfillcolor{currentfill}%
\pgfsetlinewidth{1.003750pt}%
\definecolor{currentstroke}{rgb}{0.121569,0.466667,0.705882}%
\pgfsetstrokecolor{currentstroke}%
\pgfsetdash{}{0pt}%
\pgfpathmoveto{\pgfqpoint{2.293344in}{3.188165in}}%
\pgfpathcurveto{\pgfqpoint{2.304394in}{3.188165in}}{\pgfqpoint{2.314993in}{3.192556in}}{\pgfqpoint{2.322806in}{3.200369in}}%
\pgfpathcurveto{\pgfqpoint{2.330620in}{3.208183in}}{\pgfqpoint{2.335010in}{3.218782in}}{\pgfqpoint{2.335010in}{3.229832in}}%
\pgfpathcurveto{\pgfqpoint{2.335010in}{3.240882in}}{\pgfqpoint{2.330620in}{3.251481in}}{\pgfqpoint{2.322806in}{3.259295in}}%
\pgfpathcurveto{\pgfqpoint{2.314993in}{3.267109in}}{\pgfqpoint{2.304394in}{3.271499in}}{\pgfqpoint{2.293344in}{3.271499in}}%
\pgfpathcurveto{\pgfqpoint{2.282294in}{3.271499in}}{\pgfqpoint{2.271695in}{3.267109in}}{\pgfqpoint{2.263881in}{3.259295in}}%
\pgfpathcurveto{\pgfqpoint{2.256067in}{3.251481in}}{\pgfqpoint{2.251677in}{3.240882in}}{\pgfqpoint{2.251677in}{3.229832in}}%
\pgfpathcurveto{\pgfqpoint{2.251677in}{3.218782in}}{\pgfqpoint{2.256067in}{3.208183in}}{\pgfqpoint{2.263881in}{3.200369in}}%
\pgfpathcurveto{\pgfqpoint{2.271695in}{3.192556in}}{\pgfqpoint{2.282294in}{3.188165in}}{\pgfqpoint{2.293344in}{3.188165in}}%
\pgfpathclose%
\pgfusepath{stroke,fill}%
\end{pgfscope}%
\begin{pgfscope}%
\pgfpathrectangle{\pgfqpoint{0.648703in}{0.548769in}}{\pgfqpoint{5.112893in}{3.102590in}}%
\pgfusepath{clip}%
\pgfsetbuttcap%
\pgfsetroundjoin%
\definecolor{currentfill}{rgb}{0.839216,0.152941,0.156863}%
\pgfsetfillcolor{currentfill}%
\pgfsetlinewidth{1.003750pt}%
\definecolor{currentstroke}{rgb}{0.839216,0.152941,0.156863}%
\pgfsetstrokecolor{currentstroke}%
\pgfsetdash{}{0pt}%
\pgfpathmoveto{\pgfqpoint{1.727451in}{3.208790in}}%
\pgfpathcurveto{\pgfqpoint{1.738501in}{3.208790in}}{\pgfqpoint{1.749100in}{3.213181in}}{\pgfqpoint{1.756913in}{3.220994in}}%
\pgfpathcurveto{\pgfqpoint{1.764727in}{3.228808in}}{\pgfqpoint{1.769117in}{3.239407in}}{\pgfqpoint{1.769117in}{3.250457in}}%
\pgfpathcurveto{\pgfqpoint{1.769117in}{3.261507in}}{\pgfqpoint{1.764727in}{3.272106in}}{\pgfqpoint{1.756913in}{3.279920in}}%
\pgfpathcurveto{\pgfqpoint{1.749100in}{3.287733in}}{\pgfqpoint{1.738501in}{3.292124in}}{\pgfqpoint{1.727451in}{3.292124in}}%
\pgfpathcurveto{\pgfqpoint{1.716401in}{3.292124in}}{\pgfqpoint{1.705802in}{3.287733in}}{\pgfqpoint{1.697988in}{3.279920in}}%
\pgfpathcurveto{\pgfqpoint{1.690174in}{3.272106in}}{\pgfqpoint{1.685784in}{3.261507in}}{\pgfqpoint{1.685784in}{3.250457in}}%
\pgfpathcurveto{\pgfqpoint{1.685784in}{3.239407in}}{\pgfqpoint{1.690174in}{3.228808in}}{\pgfqpoint{1.697988in}{3.220994in}}%
\pgfpathcurveto{\pgfqpoint{1.705802in}{3.213181in}}{\pgfqpoint{1.716401in}{3.208790in}}{\pgfqpoint{1.727451in}{3.208790in}}%
\pgfpathclose%
\pgfusepath{stroke,fill}%
\end{pgfscope}%
\begin{pgfscope}%
\pgfpathrectangle{\pgfqpoint{0.648703in}{0.548769in}}{\pgfqpoint{5.112893in}{3.102590in}}%
\pgfusepath{clip}%
\pgfsetbuttcap%
\pgfsetroundjoin%
\definecolor{currentfill}{rgb}{1.000000,0.498039,0.054902}%
\pgfsetfillcolor{currentfill}%
\pgfsetlinewidth{1.003750pt}%
\definecolor{currentstroke}{rgb}{1.000000,0.498039,0.054902}%
\pgfsetstrokecolor{currentstroke}%
\pgfsetdash{}{0pt}%
\pgfpathmoveto{\pgfqpoint{1.923384in}{3.192290in}}%
\pgfpathcurveto{\pgfqpoint{1.934434in}{3.192290in}}{\pgfqpoint{1.945033in}{3.196681in}}{\pgfqpoint{1.952847in}{3.204494in}}%
\pgfpathcurveto{\pgfqpoint{1.960660in}{3.212308in}}{\pgfqpoint{1.965051in}{3.222907in}}{\pgfqpoint{1.965051in}{3.233957in}}%
\pgfpathcurveto{\pgfqpoint{1.965051in}{3.245007in}}{\pgfqpoint{1.960660in}{3.255606in}}{\pgfqpoint{1.952847in}{3.263420in}}%
\pgfpathcurveto{\pgfqpoint{1.945033in}{3.271234in}}{\pgfqpoint{1.934434in}{3.275624in}}{\pgfqpoint{1.923384in}{3.275624in}}%
\pgfpathcurveto{\pgfqpoint{1.912334in}{3.275624in}}{\pgfqpoint{1.901735in}{3.271234in}}{\pgfqpoint{1.893921in}{3.263420in}}%
\pgfpathcurveto{\pgfqpoint{1.886108in}{3.255606in}}{\pgfqpoint{1.881717in}{3.245007in}}{\pgfqpoint{1.881717in}{3.233957in}}%
\pgfpathcurveto{\pgfqpoint{1.881717in}{3.222907in}}{\pgfqpoint{1.886108in}{3.212308in}}{\pgfqpoint{1.893921in}{3.204494in}}%
\pgfpathcurveto{\pgfqpoint{1.901735in}{3.196681in}}{\pgfqpoint{1.912334in}{3.192290in}}{\pgfqpoint{1.923384in}{3.192290in}}%
\pgfpathclose%
\pgfusepath{stroke,fill}%
\end{pgfscope}%
\begin{pgfscope}%
\pgfpathrectangle{\pgfqpoint{0.648703in}{0.548769in}}{\pgfqpoint{5.112893in}{3.102590in}}%
\pgfusepath{clip}%
\pgfsetbuttcap%
\pgfsetroundjoin%
\definecolor{currentfill}{rgb}{1.000000,0.498039,0.054902}%
\pgfsetfillcolor{currentfill}%
\pgfsetlinewidth{1.003750pt}%
\definecolor{currentstroke}{rgb}{1.000000,0.498039,0.054902}%
\pgfsetstrokecolor{currentstroke}%
\pgfsetdash{}{0pt}%
\pgfpathmoveto{\pgfqpoint{1.452726in}{3.283040in}}%
\pgfpathcurveto{\pgfqpoint{1.463776in}{3.283040in}}{\pgfqpoint{1.474375in}{3.287431in}}{\pgfqpoint{1.482189in}{3.295244in}}%
\pgfpathcurveto{\pgfqpoint{1.490002in}{3.303058in}}{\pgfqpoint{1.494393in}{3.313657in}}{\pgfqpoint{1.494393in}{3.324707in}}%
\pgfpathcurveto{\pgfqpoint{1.494393in}{3.335757in}}{\pgfqpoint{1.490002in}{3.346356in}}{\pgfqpoint{1.482189in}{3.354170in}}%
\pgfpathcurveto{\pgfqpoint{1.474375in}{3.361983in}}{\pgfqpoint{1.463776in}{3.366374in}}{\pgfqpoint{1.452726in}{3.366374in}}%
\pgfpathcurveto{\pgfqpoint{1.441676in}{3.366374in}}{\pgfqpoint{1.431077in}{3.361983in}}{\pgfqpoint{1.423263in}{3.354170in}}%
\pgfpathcurveto{\pgfqpoint{1.415450in}{3.346356in}}{\pgfqpoint{1.411059in}{3.335757in}}{\pgfqpoint{1.411059in}{3.324707in}}%
\pgfpathcurveto{\pgfqpoint{1.411059in}{3.313657in}}{\pgfqpoint{1.415450in}{3.303058in}}{\pgfqpoint{1.423263in}{3.295244in}}%
\pgfpathcurveto{\pgfqpoint{1.431077in}{3.287431in}}{\pgfqpoint{1.441676in}{3.283040in}}{\pgfqpoint{1.452726in}{3.283040in}}%
\pgfpathclose%
\pgfusepath{stroke,fill}%
\end{pgfscope}%
\begin{pgfscope}%
\pgfpathrectangle{\pgfqpoint{0.648703in}{0.548769in}}{\pgfqpoint{5.112893in}{3.102590in}}%
\pgfusepath{clip}%
\pgfsetbuttcap%
\pgfsetroundjoin%
\definecolor{currentfill}{rgb}{0.121569,0.466667,0.705882}%
\pgfsetfillcolor{currentfill}%
\pgfsetlinewidth{1.003750pt}%
\definecolor{currentstroke}{rgb}{0.121569,0.466667,0.705882}%
\pgfsetstrokecolor{currentstroke}%
\pgfsetdash{}{0pt}%
\pgfpathmoveto{\pgfqpoint{1.454791in}{3.188165in}}%
\pgfpathcurveto{\pgfqpoint{1.465841in}{3.188165in}}{\pgfqpoint{1.476440in}{3.192556in}}{\pgfqpoint{1.484253in}{3.200369in}}%
\pgfpathcurveto{\pgfqpoint{1.492067in}{3.208183in}}{\pgfqpoint{1.496457in}{3.218782in}}{\pgfqpoint{1.496457in}{3.229832in}}%
\pgfpathcurveto{\pgfqpoint{1.496457in}{3.240882in}}{\pgfqpoint{1.492067in}{3.251481in}}{\pgfqpoint{1.484253in}{3.259295in}}%
\pgfpathcurveto{\pgfqpoint{1.476440in}{3.267109in}}{\pgfqpoint{1.465841in}{3.271499in}}{\pgfqpoint{1.454791in}{3.271499in}}%
\pgfpathcurveto{\pgfqpoint{1.443740in}{3.271499in}}{\pgfqpoint{1.433141in}{3.267109in}}{\pgfqpoint{1.425328in}{3.259295in}}%
\pgfpathcurveto{\pgfqpoint{1.417514in}{3.251481in}}{\pgfqpoint{1.413124in}{3.240882in}}{\pgfqpoint{1.413124in}{3.229832in}}%
\pgfpathcurveto{\pgfqpoint{1.413124in}{3.218782in}}{\pgfqpoint{1.417514in}{3.208183in}}{\pgfqpoint{1.425328in}{3.200369in}}%
\pgfpathcurveto{\pgfqpoint{1.433141in}{3.192556in}}{\pgfqpoint{1.443740in}{3.188165in}}{\pgfqpoint{1.454791in}{3.188165in}}%
\pgfpathclose%
\pgfusepath{stroke,fill}%
\end{pgfscope}%
\begin{pgfscope}%
\pgfpathrectangle{\pgfqpoint{0.648703in}{0.548769in}}{\pgfqpoint{5.112893in}{3.102590in}}%
\pgfusepath{clip}%
\pgfsetbuttcap%
\pgfsetroundjoin%
\definecolor{currentfill}{rgb}{1.000000,0.498039,0.054902}%
\pgfsetfillcolor{currentfill}%
\pgfsetlinewidth{1.003750pt}%
\definecolor{currentstroke}{rgb}{1.000000,0.498039,0.054902}%
\pgfsetstrokecolor{currentstroke}%
\pgfsetdash{}{0pt}%
\pgfpathmoveto{\pgfqpoint{1.357281in}{3.225290in}}%
\pgfpathcurveto{\pgfqpoint{1.368331in}{3.225290in}}{\pgfqpoint{1.378930in}{3.229681in}}{\pgfqpoint{1.386743in}{3.237494in}}%
\pgfpathcurveto{\pgfqpoint{1.394557in}{3.245308in}}{\pgfqpoint{1.398947in}{3.255907in}}{\pgfqpoint{1.398947in}{3.266957in}}%
\pgfpathcurveto{\pgfqpoint{1.398947in}{3.278007in}}{\pgfqpoint{1.394557in}{3.288606in}}{\pgfqpoint{1.386743in}{3.296420in}}%
\pgfpathcurveto{\pgfqpoint{1.378930in}{3.304233in}}{\pgfqpoint{1.368331in}{3.308624in}}{\pgfqpoint{1.357281in}{3.308624in}}%
\pgfpathcurveto{\pgfqpoint{1.346231in}{3.308624in}}{\pgfqpoint{1.335632in}{3.304233in}}{\pgfqpoint{1.327818in}{3.296420in}}%
\pgfpathcurveto{\pgfqpoint{1.320004in}{3.288606in}}{\pgfqpoint{1.315614in}{3.278007in}}{\pgfqpoint{1.315614in}{3.266957in}}%
\pgfpathcurveto{\pgfqpoint{1.315614in}{3.255907in}}{\pgfqpoint{1.320004in}{3.245308in}}{\pgfqpoint{1.327818in}{3.237494in}}%
\pgfpathcurveto{\pgfqpoint{1.335632in}{3.229681in}}{\pgfqpoint{1.346231in}{3.225290in}}{\pgfqpoint{1.357281in}{3.225290in}}%
\pgfpathclose%
\pgfusepath{stroke,fill}%
\end{pgfscope}%
\begin{pgfscope}%
\pgfpathrectangle{\pgfqpoint{0.648703in}{0.548769in}}{\pgfqpoint{5.112893in}{3.102590in}}%
\pgfusepath{clip}%
\pgfsetbuttcap%
\pgfsetroundjoin%
\definecolor{currentfill}{rgb}{1.000000,0.498039,0.054902}%
\pgfsetfillcolor{currentfill}%
\pgfsetlinewidth{1.003750pt}%
\definecolor{currentstroke}{rgb}{1.000000,0.498039,0.054902}%
\pgfsetstrokecolor{currentstroke}%
\pgfsetdash{}{0pt}%
\pgfpathmoveto{\pgfqpoint{3.147062in}{3.468665in}}%
\pgfpathcurveto{\pgfqpoint{3.158112in}{3.468665in}}{\pgfqpoint{3.168712in}{3.473055in}}{\pgfqpoint{3.176525in}{3.480869in}}%
\pgfpathcurveto{\pgfqpoint{3.184339in}{3.488683in}}{\pgfqpoint{3.188729in}{3.499282in}}{\pgfqpoint{3.188729in}{3.510332in}}%
\pgfpathcurveto{\pgfqpoint{3.188729in}{3.521382in}}{\pgfqpoint{3.184339in}{3.531981in}}{\pgfqpoint{3.176525in}{3.539795in}}%
\pgfpathcurveto{\pgfqpoint{3.168712in}{3.547608in}}{\pgfqpoint{3.158112in}{3.551998in}}{\pgfqpoint{3.147062in}{3.551998in}}%
\pgfpathcurveto{\pgfqpoint{3.136012in}{3.551998in}}{\pgfqpoint{3.125413in}{3.547608in}}{\pgfqpoint{3.117600in}{3.539795in}}%
\pgfpathcurveto{\pgfqpoint{3.109786in}{3.531981in}}{\pgfqpoint{3.105396in}{3.521382in}}{\pgfqpoint{3.105396in}{3.510332in}}%
\pgfpathcurveto{\pgfqpoint{3.105396in}{3.499282in}}{\pgfqpoint{3.109786in}{3.488683in}}{\pgfqpoint{3.117600in}{3.480869in}}%
\pgfpathcurveto{\pgfqpoint{3.125413in}{3.473055in}}{\pgfqpoint{3.136012in}{3.468665in}}{\pgfqpoint{3.147062in}{3.468665in}}%
\pgfpathclose%
\pgfusepath{stroke,fill}%
\end{pgfscope}%
\begin{pgfscope}%
\pgfpathrectangle{\pgfqpoint{0.648703in}{0.548769in}}{\pgfqpoint{5.112893in}{3.102590in}}%
\pgfusepath{clip}%
\pgfsetbuttcap%
\pgfsetroundjoin%
\definecolor{currentfill}{rgb}{1.000000,0.498039,0.054902}%
\pgfsetfillcolor{currentfill}%
\pgfsetlinewidth{1.003750pt}%
\definecolor{currentstroke}{rgb}{1.000000,0.498039,0.054902}%
\pgfsetstrokecolor{currentstroke}%
\pgfsetdash{}{0pt}%
\pgfpathmoveto{\pgfqpoint{1.742713in}{3.196415in}}%
\pgfpathcurveto{\pgfqpoint{1.753763in}{3.196415in}}{\pgfqpoint{1.764362in}{3.200806in}}{\pgfqpoint{1.772176in}{3.208619in}}%
\pgfpathcurveto{\pgfqpoint{1.779989in}{3.216433in}}{\pgfqpoint{1.784379in}{3.227032in}}{\pgfqpoint{1.784379in}{3.238082in}}%
\pgfpathcurveto{\pgfqpoint{1.784379in}{3.249132in}}{\pgfqpoint{1.779989in}{3.259731in}}{\pgfqpoint{1.772176in}{3.267545in}}%
\pgfpathcurveto{\pgfqpoint{1.764362in}{3.275358in}}{\pgfqpoint{1.753763in}{3.279749in}}{\pgfqpoint{1.742713in}{3.279749in}}%
\pgfpathcurveto{\pgfqpoint{1.731663in}{3.279749in}}{\pgfqpoint{1.721064in}{3.275358in}}{\pgfqpoint{1.713250in}{3.267545in}}%
\pgfpathcurveto{\pgfqpoint{1.705436in}{3.259731in}}{\pgfqpoint{1.701046in}{3.249132in}}{\pgfqpoint{1.701046in}{3.238082in}}%
\pgfpathcurveto{\pgfqpoint{1.701046in}{3.227032in}}{\pgfqpoint{1.705436in}{3.216433in}}{\pgfqpoint{1.713250in}{3.208619in}}%
\pgfpathcurveto{\pgfqpoint{1.721064in}{3.200806in}}{\pgfqpoint{1.731663in}{3.196415in}}{\pgfqpoint{1.742713in}{3.196415in}}%
\pgfpathclose%
\pgfusepath{stroke,fill}%
\end{pgfscope}%
\begin{pgfscope}%
\pgfpathrectangle{\pgfqpoint{0.648703in}{0.548769in}}{\pgfqpoint{5.112893in}{3.102590in}}%
\pgfusepath{clip}%
\pgfsetbuttcap%
\pgfsetroundjoin%
\definecolor{currentfill}{rgb}{1.000000,0.498039,0.054902}%
\pgfsetfillcolor{currentfill}%
\pgfsetlinewidth{1.003750pt}%
\definecolor{currentstroke}{rgb}{1.000000,0.498039,0.054902}%
\pgfsetstrokecolor{currentstroke}%
\pgfsetdash{}{0pt}%
\pgfpathmoveto{\pgfqpoint{1.547282in}{3.196415in}}%
\pgfpathcurveto{\pgfqpoint{1.558332in}{3.196415in}}{\pgfqpoint{1.568931in}{3.200806in}}{\pgfqpoint{1.576745in}{3.208619in}}%
\pgfpathcurveto{\pgfqpoint{1.584559in}{3.216433in}}{\pgfqpoint{1.588949in}{3.227032in}}{\pgfqpoint{1.588949in}{3.238082in}}%
\pgfpathcurveto{\pgfqpoint{1.588949in}{3.249132in}}{\pgfqpoint{1.584559in}{3.259731in}}{\pgfqpoint{1.576745in}{3.267545in}}%
\pgfpathcurveto{\pgfqpoint{1.568931in}{3.275358in}}{\pgfqpoint{1.558332in}{3.279749in}}{\pgfqpoint{1.547282in}{3.279749in}}%
\pgfpathcurveto{\pgfqpoint{1.536232in}{3.279749in}}{\pgfqpoint{1.525633in}{3.275358in}}{\pgfqpoint{1.517819in}{3.267545in}}%
\pgfpathcurveto{\pgfqpoint{1.510006in}{3.259731in}}{\pgfqpoint{1.505615in}{3.249132in}}{\pgfqpoint{1.505615in}{3.238082in}}%
\pgfpathcurveto{\pgfqpoint{1.505615in}{3.227032in}}{\pgfqpoint{1.510006in}{3.216433in}}{\pgfqpoint{1.517819in}{3.208619in}}%
\pgfpathcurveto{\pgfqpoint{1.525633in}{3.200806in}}{\pgfqpoint{1.536232in}{3.196415in}}{\pgfqpoint{1.547282in}{3.196415in}}%
\pgfpathclose%
\pgfusepath{stroke,fill}%
\end{pgfscope}%
\begin{pgfscope}%
\pgfpathrectangle{\pgfqpoint{0.648703in}{0.548769in}}{\pgfqpoint{5.112893in}{3.102590in}}%
\pgfusepath{clip}%
\pgfsetbuttcap%
\pgfsetroundjoin%
\definecolor{currentfill}{rgb}{0.121569,0.466667,0.705882}%
\pgfsetfillcolor{currentfill}%
\pgfsetlinewidth{1.003750pt}%
\definecolor{currentstroke}{rgb}{0.121569,0.466667,0.705882}%
\pgfsetstrokecolor{currentstroke}%
\pgfsetdash{}{0pt}%
\pgfpathmoveto{\pgfqpoint{1.500028in}{3.188165in}}%
\pgfpathcurveto{\pgfqpoint{1.511078in}{3.188165in}}{\pgfqpoint{1.521677in}{3.192556in}}{\pgfqpoint{1.529491in}{3.200369in}}%
\pgfpathcurveto{\pgfqpoint{1.537304in}{3.208183in}}{\pgfqpoint{1.541695in}{3.218782in}}{\pgfqpoint{1.541695in}{3.229832in}}%
\pgfpathcurveto{\pgfqpoint{1.541695in}{3.240882in}}{\pgfqpoint{1.537304in}{3.251481in}}{\pgfqpoint{1.529491in}{3.259295in}}%
\pgfpathcurveto{\pgfqpoint{1.521677in}{3.267109in}}{\pgfqpoint{1.511078in}{3.271499in}}{\pgfqpoint{1.500028in}{3.271499in}}%
\pgfpathcurveto{\pgfqpoint{1.488978in}{3.271499in}}{\pgfqpoint{1.478379in}{3.267109in}}{\pgfqpoint{1.470565in}{3.259295in}}%
\pgfpathcurveto{\pgfqpoint{1.462752in}{3.251481in}}{\pgfqpoint{1.458361in}{3.240882in}}{\pgfqpoint{1.458361in}{3.229832in}}%
\pgfpathcurveto{\pgfqpoint{1.458361in}{3.218782in}}{\pgfqpoint{1.462752in}{3.208183in}}{\pgfqpoint{1.470565in}{3.200369in}}%
\pgfpathcurveto{\pgfqpoint{1.478379in}{3.192556in}}{\pgfqpoint{1.488978in}{3.188165in}}{\pgfqpoint{1.500028in}{3.188165in}}%
\pgfpathclose%
\pgfusepath{stroke,fill}%
\end{pgfscope}%
\begin{pgfscope}%
\pgfpathrectangle{\pgfqpoint{0.648703in}{0.548769in}}{\pgfqpoint{5.112893in}{3.102590in}}%
\pgfusepath{clip}%
\pgfsetbuttcap%
\pgfsetroundjoin%
\definecolor{currentfill}{rgb}{1.000000,0.498039,0.054902}%
\pgfsetfillcolor{currentfill}%
\pgfsetlinewidth{1.003750pt}%
\definecolor{currentstroke}{rgb}{1.000000,0.498039,0.054902}%
\pgfsetstrokecolor{currentstroke}%
\pgfsetdash{}{0pt}%
\pgfpathmoveto{\pgfqpoint{2.433702in}{3.192290in}}%
\pgfpathcurveto{\pgfqpoint{2.444752in}{3.192290in}}{\pgfqpoint{2.455351in}{3.196681in}}{\pgfqpoint{2.463165in}{3.204494in}}%
\pgfpathcurveto{\pgfqpoint{2.470979in}{3.212308in}}{\pgfqpoint{2.475369in}{3.222907in}}{\pgfqpoint{2.475369in}{3.233957in}}%
\pgfpathcurveto{\pgfqpoint{2.475369in}{3.245007in}}{\pgfqpoint{2.470979in}{3.255606in}}{\pgfqpoint{2.463165in}{3.263420in}}%
\pgfpathcurveto{\pgfqpoint{2.455351in}{3.271234in}}{\pgfqpoint{2.444752in}{3.275624in}}{\pgfqpoint{2.433702in}{3.275624in}}%
\pgfpathcurveto{\pgfqpoint{2.422652in}{3.275624in}}{\pgfqpoint{2.412053in}{3.271234in}}{\pgfqpoint{2.404240in}{3.263420in}}%
\pgfpathcurveto{\pgfqpoint{2.396426in}{3.255606in}}{\pgfqpoint{2.392036in}{3.245007in}}{\pgfqpoint{2.392036in}{3.233957in}}%
\pgfpathcurveto{\pgfqpoint{2.392036in}{3.222907in}}{\pgfqpoint{2.396426in}{3.212308in}}{\pgfqpoint{2.404240in}{3.204494in}}%
\pgfpathcurveto{\pgfqpoint{2.412053in}{3.196681in}}{\pgfqpoint{2.422652in}{3.192290in}}{\pgfqpoint{2.433702in}{3.192290in}}%
\pgfpathclose%
\pgfusepath{stroke,fill}%
\end{pgfscope}%
\begin{pgfscope}%
\pgfpathrectangle{\pgfqpoint{0.648703in}{0.548769in}}{\pgfqpoint{5.112893in}{3.102590in}}%
\pgfusepath{clip}%
\pgfsetbuttcap%
\pgfsetroundjoin%
\definecolor{currentfill}{rgb}{1.000000,0.498039,0.054902}%
\pgfsetfillcolor{currentfill}%
\pgfsetlinewidth{1.003750pt}%
\definecolor{currentstroke}{rgb}{1.000000,0.498039,0.054902}%
\pgfsetstrokecolor{currentstroke}%
\pgfsetdash{}{0pt}%
\pgfpathmoveto{\pgfqpoint{2.185379in}{3.361415in}}%
\pgfpathcurveto{\pgfqpoint{2.196430in}{3.361415in}}{\pgfqpoint{2.207029in}{3.365806in}}{\pgfqpoint{2.214842in}{3.373619in}}%
\pgfpathcurveto{\pgfqpoint{2.222656in}{3.381433in}}{\pgfqpoint{2.227046in}{3.392032in}}{\pgfqpoint{2.227046in}{3.403082in}}%
\pgfpathcurveto{\pgfqpoint{2.227046in}{3.414132in}}{\pgfqpoint{2.222656in}{3.424731in}}{\pgfqpoint{2.214842in}{3.432545in}}%
\pgfpathcurveto{\pgfqpoint{2.207029in}{3.440358in}}{\pgfqpoint{2.196430in}{3.444749in}}{\pgfqpoint{2.185379in}{3.444749in}}%
\pgfpathcurveto{\pgfqpoint{2.174329in}{3.444749in}}{\pgfqpoint{2.163730in}{3.440358in}}{\pgfqpoint{2.155917in}{3.432545in}}%
\pgfpathcurveto{\pgfqpoint{2.148103in}{3.424731in}}{\pgfqpoint{2.143713in}{3.414132in}}{\pgfqpoint{2.143713in}{3.403082in}}%
\pgfpathcurveto{\pgfqpoint{2.143713in}{3.392032in}}{\pgfqpoint{2.148103in}{3.381433in}}{\pgfqpoint{2.155917in}{3.373619in}}%
\pgfpathcurveto{\pgfqpoint{2.163730in}{3.365806in}}{\pgfqpoint{2.174329in}{3.361415in}}{\pgfqpoint{2.185379in}{3.361415in}}%
\pgfpathclose%
\pgfusepath{stroke,fill}%
\end{pgfscope}%
\begin{pgfscope}%
\pgfpathrectangle{\pgfqpoint{0.648703in}{0.548769in}}{\pgfqpoint{5.112893in}{3.102590in}}%
\pgfusepath{clip}%
\pgfsetbuttcap%
\pgfsetroundjoin%
\definecolor{currentfill}{rgb}{1.000000,0.498039,0.054902}%
\pgfsetfillcolor{currentfill}%
\pgfsetlinewidth{1.003750pt}%
\definecolor{currentstroke}{rgb}{1.000000,0.498039,0.054902}%
\pgfsetstrokecolor{currentstroke}%
\pgfsetdash{}{0pt}%
\pgfpathmoveto{\pgfqpoint{1.883549in}{3.204665in}}%
\pgfpathcurveto{\pgfqpoint{1.894599in}{3.204665in}}{\pgfqpoint{1.905199in}{3.209056in}}{\pgfqpoint{1.913012in}{3.216869in}}%
\pgfpathcurveto{\pgfqpoint{1.920826in}{3.224683in}}{\pgfqpoint{1.925216in}{3.235282in}}{\pgfqpoint{1.925216in}{3.246332in}}%
\pgfpathcurveto{\pgfqpoint{1.925216in}{3.257382in}}{\pgfqpoint{1.920826in}{3.267981in}}{\pgfqpoint{1.913012in}{3.275795in}}%
\pgfpathcurveto{\pgfqpoint{1.905199in}{3.283608in}}{\pgfqpoint{1.894599in}{3.287999in}}{\pgfqpoint{1.883549in}{3.287999in}}%
\pgfpathcurveto{\pgfqpoint{1.872499in}{3.287999in}}{\pgfqpoint{1.861900in}{3.283608in}}{\pgfqpoint{1.854087in}{3.275795in}}%
\pgfpathcurveto{\pgfqpoint{1.846273in}{3.267981in}}{\pgfqpoint{1.841883in}{3.257382in}}{\pgfqpoint{1.841883in}{3.246332in}}%
\pgfpathcurveto{\pgfqpoint{1.841883in}{3.235282in}}{\pgfqpoint{1.846273in}{3.224683in}}{\pgfqpoint{1.854087in}{3.216869in}}%
\pgfpathcurveto{\pgfqpoint{1.861900in}{3.209056in}}{\pgfqpoint{1.872499in}{3.204665in}}{\pgfqpoint{1.883549in}{3.204665in}}%
\pgfpathclose%
\pgfusepath{stroke,fill}%
\end{pgfscope}%
\begin{pgfscope}%
\pgfpathrectangle{\pgfqpoint{0.648703in}{0.548769in}}{\pgfqpoint{5.112893in}{3.102590in}}%
\pgfusepath{clip}%
\pgfsetbuttcap%
\pgfsetroundjoin%
\definecolor{currentfill}{rgb}{0.839216,0.152941,0.156863}%
\pgfsetfillcolor{currentfill}%
\pgfsetlinewidth{1.003750pt}%
\definecolor{currentstroke}{rgb}{0.839216,0.152941,0.156863}%
\pgfsetstrokecolor{currentstroke}%
\pgfsetdash{}{0pt}%
\pgfpathmoveto{\pgfqpoint{1.484844in}{3.200540in}}%
\pgfpathcurveto{\pgfqpoint{1.495894in}{3.200540in}}{\pgfqpoint{1.506493in}{3.204931in}}{\pgfqpoint{1.514306in}{3.212744in}}%
\pgfpathcurveto{\pgfqpoint{1.522120in}{3.220558in}}{\pgfqpoint{1.526510in}{3.231157in}}{\pgfqpoint{1.526510in}{3.242207in}}%
\pgfpathcurveto{\pgfqpoint{1.526510in}{3.253257in}}{\pgfqpoint{1.522120in}{3.263856in}}{\pgfqpoint{1.514306in}{3.271670in}}%
\pgfpathcurveto{\pgfqpoint{1.506493in}{3.279483in}}{\pgfqpoint{1.495894in}{3.283874in}}{\pgfqpoint{1.484844in}{3.283874in}}%
\pgfpathcurveto{\pgfqpoint{1.473793in}{3.283874in}}{\pgfqpoint{1.463194in}{3.279483in}}{\pgfqpoint{1.455381in}{3.271670in}}%
\pgfpathcurveto{\pgfqpoint{1.447567in}{3.263856in}}{\pgfqpoint{1.443177in}{3.253257in}}{\pgfqpoint{1.443177in}{3.242207in}}%
\pgfpathcurveto{\pgfqpoint{1.443177in}{3.231157in}}{\pgfqpoint{1.447567in}{3.220558in}}{\pgfqpoint{1.455381in}{3.212744in}}%
\pgfpathcurveto{\pgfqpoint{1.463194in}{3.204931in}}{\pgfqpoint{1.473793in}{3.200540in}}{\pgfqpoint{1.484844in}{3.200540in}}%
\pgfpathclose%
\pgfusepath{stroke,fill}%
\end{pgfscope}%
\begin{pgfscope}%
\pgfpathrectangle{\pgfqpoint{0.648703in}{0.548769in}}{\pgfqpoint{5.112893in}{3.102590in}}%
\pgfusepath{clip}%
\pgfsetbuttcap%
\pgfsetroundjoin%
\definecolor{currentfill}{rgb}{1.000000,0.498039,0.054902}%
\pgfsetfillcolor{currentfill}%
\pgfsetlinewidth{1.003750pt}%
\definecolor{currentstroke}{rgb}{1.000000,0.498039,0.054902}%
\pgfsetstrokecolor{currentstroke}%
\pgfsetdash{}{0pt}%
\pgfpathmoveto{\pgfqpoint{1.373864in}{3.200540in}}%
\pgfpathcurveto{\pgfqpoint{1.384915in}{3.200540in}}{\pgfqpoint{1.395514in}{3.204931in}}{\pgfqpoint{1.403327in}{3.212744in}}%
\pgfpathcurveto{\pgfqpoint{1.411141in}{3.220558in}}{\pgfqpoint{1.415531in}{3.231157in}}{\pgfqpoint{1.415531in}{3.242207in}}%
\pgfpathcurveto{\pgfqpoint{1.415531in}{3.253257in}}{\pgfqpoint{1.411141in}{3.263856in}}{\pgfqpoint{1.403327in}{3.271670in}}%
\pgfpathcurveto{\pgfqpoint{1.395514in}{3.279483in}}{\pgfqpoint{1.384915in}{3.283874in}}{\pgfqpoint{1.373864in}{3.283874in}}%
\pgfpathcurveto{\pgfqpoint{1.362814in}{3.283874in}}{\pgfqpoint{1.352215in}{3.279483in}}{\pgfqpoint{1.344402in}{3.271670in}}%
\pgfpathcurveto{\pgfqpoint{1.336588in}{3.263856in}}{\pgfqpoint{1.332198in}{3.253257in}}{\pgfqpoint{1.332198in}{3.242207in}}%
\pgfpathcurveto{\pgfqpoint{1.332198in}{3.231157in}}{\pgfqpoint{1.336588in}{3.220558in}}{\pgfqpoint{1.344402in}{3.212744in}}%
\pgfpathcurveto{\pgfqpoint{1.352215in}{3.204931in}}{\pgfqpoint{1.362814in}{3.200540in}}{\pgfqpoint{1.373864in}{3.200540in}}%
\pgfpathclose%
\pgfusepath{stroke,fill}%
\end{pgfscope}%
\begin{pgfscope}%
\pgfpathrectangle{\pgfqpoint{0.648703in}{0.548769in}}{\pgfqpoint{5.112893in}{3.102590in}}%
\pgfusepath{clip}%
\pgfsetbuttcap%
\pgfsetroundjoin%
\definecolor{currentfill}{rgb}{1.000000,0.498039,0.054902}%
\pgfsetfillcolor{currentfill}%
\pgfsetlinewidth{1.003750pt}%
\definecolor{currentstroke}{rgb}{1.000000,0.498039,0.054902}%
\pgfsetstrokecolor{currentstroke}%
\pgfsetdash{}{0pt}%
\pgfpathmoveto{\pgfqpoint{2.206047in}{3.192290in}}%
\pgfpathcurveto{\pgfqpoint{2.217098in}{3.192290in}}{\pgfqpoint{2.227697in}{3.196681in}}{\pgfqpoint{2.235510in}{3.204494in}}%
\pgfpathcurveto{\pgfqpoint{2.243324in}{3.212308in}}{\pgfqpoint{2.247714in}{3.222907in}}{\pgfqpoint{2.247714in}{3.233957in}}%
\pgfpathcurveto{\pgfqpoint{2.247714in}{3.245007in}}{\pgfqpoint{2.243324in}{3.255606in}}{\pgfqpoint{2.235510in}{3.263420in}}%
\pgfpathcurveto{\pgfqpoint{2.227697in}{3.271234in}}{\pgfqpoint{2.217098in}{3.275624in}}{\pgfqpoint{2.206047in}{3.275624in}}%
\pgfpathcurveto{\pgfqpoint{2.194997in}{3.275624in}}{\pgfqpoint{2.184398in}{3.271234in}}{\pgfqpoint{2.176585in}{3.263420in}}%
\pgfpathcurveto{\pgfqpoint{2.168771in}{3.255606in}}{\pgfqpoint{2.164381in}{3.245007in}}{\pgfqpoint{2.164381in}{3.233957in}}%
\pgfpathcurveto{\pgfqpoint{2.164381in}{3.222907in}}{\pgfqpoint{2.168771in}{3.212308in}}{\pgfqpoint{2.176585in}{3.204494in}}%
\pgfpathcurveto{\pgfqpoint{2.184398in}{3.196681in}}{\pgfqpoint{2.194997in}{3.192290in}}{\pgfqpoint{2.206047in}{3.192290in}}%
\pgfpathclose%
\pgfusepath{stroke,fill}%
\end{pgfscope}%
\begin{pgfscope}%
\pgfpathrectangle{\pgfqpoint{0.648703in}{0.548769in}}{\pgfqpoint{5.112893in}{3.102590in}}%
\pgfusepath{clip}%
\pgfsetbuttcap%
\pgfsetroundjoin%
\definecolor{currentfill}{rgb}{1.000000,0.498039,0.054902}%
\pgfsetfillcolor{currentfill}%
\pgfsetlinewidth{1.003750pt}%
\definecolor{currentstroke}{rgb}{1.000000,0.498039,0.054902}%
\pgfsetstrokecolor{currentstroke}%
\pgfsetdash{}{0pt}%
\pgfpathmoveto{\pgfqpoint{2.008244in}{3.192290in}}%
\pgfpathcurveto{\pgfqpoint{2.019294in}{3.192290in}}{\pgfqpoint{2.029893in}{3.196681in}}{\pgfqpoint{2.037707in}{3.204494in}}%
\pgfpathcurveto{\pgfqpoint{2.045520in}{3.212308in}}{\pgfqpoint{2.049911in}{3.222907in}}{\pgfqpoint{2.049911in}{3.233957in}}%
\pgfpathcurveto{\pgfqpoint{2.049911in}{3.245007in}}{\pgfqpoint{2.045520in}{3.255606in}}{\pgfqpoint{2.037707in}{3.263420in}}%
\pgfpathcurveto{\pgfqpoint{2.029893in}{3.271234in}}{\pgfqpoint{2.019294in}{3.275624in}}{\pgfqpoint{2.008244in}{3.275624in}}%
\pgfpathcurveto{\pgfqpoint{1.997194in}{3.275624in}}{\pgfqpoint{1.986595in}{3.271234in}}{\pgfqpoint{1.978781in}{3.263420in}}%
\pgfpathcurveto{\pgfqpoint{1.970968in}{3.255606in}}{\pgfqpoint{1.966577in}{3.245007in}}{\pgfqpoint{1.966577in}{3.233957in}}%
\pgfpathcurveto{\pgfqpoint{1.966577in}{3.222907in}}{\pgfqpoint{1.970968in}{3.212308in}}{\pgfqpoint{1.978781in}{3.204494in}}%
\pgfpathcurveto{\pgfqpoint{1.986595in}{3.196681in}}{\pgfqpoint{1.997194in}{3.192290in}}{\pgfqpoint{2.008244in}{3.192290in}}%
\pgfpathclose%
\pgfusepath{stroke,fill}%
\end{pgfscope}%
\begin{pgfscope}%
\pgfpathrectangle{\pgfqpoint{0.648703in}{0.548769in}}{\pgfqpoint{5.112893in}{3.102590in}}%
\pgfusepath{clip}%
\pgfsetbuttcap%
\pgfsetroundjoin%
\definecolor{currentfill}{rgb}{1.000000,0.498039,0.054902}%
\pgfsetfillcolor{currentfill}%
\pgfsetlinewidth{1.003750pt}%
\definecolor{currentstroke}{rgb}{1.000000,0.498039,0.054902}%
\pgfsetstrokecolor{currentstroke}%
\pgfsetdash{}{0pt}%
\pgfpathmoveto{\pgfqpoint{1.715926in}{3.200540in}}%
\pgfpathcurveto{\pgfqpoint{1.726976in}{3.200540in}}{\pgfqpoint{1.737575in}{3.204931in}}{\pgfqpoint{1.745389in}{3.212744in}}%
\pgfpathcurveto{\pgfqpoint{1.753203in}{3.220558in}}{\pgfqpoint{1.757593in}{3.231157in}}{\pgfqpoint{1.757593in}{3.242207in}}%
\pgfpathcurveto{\pgfqpoint{1.757593in}{3.253257in}}{\pgfqpoint{1.753203in}{3.263856in}}{\pgfqpoint{1.745389in}{3.271670in}}%
\pgfpathcurveto{\pgfqpoint{1.737575in}{3.279483in}}{\pgfqpoint{1.726976in}{3.283874in}}{\pgfqpoint{1.715926in}{3.283874in}}%
\pgfpathcurveto{\pgfqpoint{1.704876in}{3.283874in}}{\pgfqpoint{1.694277in}{3.279483in}}{\pgfqpoint{1.686464in}{3.271670in}}%
\pgfpathcurveto{\pgfqpoint{1.678650in}{3.263856in}}{\pgfqpoint{1.674260in}{3.253257in}}{\pgfqpoint{1.674260in}{3.242207in}}%
\pgfpathcurveto{\pgfqpoint{1.674260in}{3.231157in}}{\pgfqpoint{1.678650in}{3.220558in}}{\pgfqpoint{1.686464in}{3.212744in}}%
\pgfpathcurveto{\pgfqpoint{1.694277in}{3.204931in}}{\pgfqpoint{1.704876in}{3.200540in}}{\pgfqpoint{1.715926in}{3.200540in}}%
\pgfpathclose%
\pgfusepath{stroke,fill}%
\end{pgfscope}%
\begin{pgfscope}%
\pgfpathrectangle{\pgfqpoint{0.648703in}{0.548769in}}{\pgfqpoint{5.112893in}{3.102590in}}%
\pgfusepath{clip}%
\pgfsetbuttcap%
\pgfsetroundjoin%
\definecolor{currentfill}{rgb}{0.839216,0.152941,0.156863}%
\pgfsetfillcolor{currentfill}%
\pgfsetlinewidth{1.003750pt}%
\definecolor{currentstroke}{rgb}{0.839216,0.152941,0.156863}%
\pgfsetstrokecolor{currentstroke}%
\pgfsetdash{}{0pt}%
\pgfpathmoveto{\pgfqpoint{2.293943in}{3.196415in}}%
\pgfpathcurveto{\pgfqpoint{2.304993in}{3.196415in}}{\pgfqpoint{2.315592in}{3.200806in}}{\pgfqpoint{2.323405in}{3.208619in}}%
\pgfpathcurveto{\pgfqpoint{2.331219in}{3.216433in}}{\pgfqpoint{2.335609in}{3.227032in}}{\pgfqpoint{2.335609in}{3.238082in}}%
\pgfpathcurveto{\pgfqpoint{2.335609in}{3.249132in}}{\pgfqpoint{2.331219in}{3.259731in}}{\pgfqpoint{2.323405in}{3.267545in}}%
\pgfpathcurveto{\pgfqpoint{2.315592in}{3.275358in}}{\pgfqpoint{2.304993in}{3.279749in}}{\pgfqpoint{2.293943in}{3.279749in}}%
\pgfpathcurveto{\pgfqpoint{2.282893in}{3.279749in}}{\pgfqpoint{2.272293in}{3.275358in}}{\pgfqpoint{2.264480in}{3.267545in}}%
\pgfpathcurveto{\pgfqpoint{2.256666in}{3.259731in}}{\pgfqpoint{2.252276in}{3.249132in}}{\pgfqpoint{2.252276in}{3.238082in}}%
\pgfpathcurveto{\pgfqpoint{2.252276in}{3.227032in}}{\pgfqpoint{2.256666in}{3.216433in}}{\pgfqpoint{2.264480in}{3.208619in}}%
\pgfpathcurveto{\pgfqpoint{2.272293in}{3.200806in}}{\pgfqpoint{2.282893in}{3.196415in}}{\pgfqpoint{2.293943in}{3.196415in}}%
\pgfpathclose%
\pgfusepath{stroke,fill}%
\end{pgfscope}%
\begin{pgfscope}%
\pgfpathrectangle{\pgfqpoint{0.648703in}{0.548769in}}{\pgfqpoint{5.112893in}{3.102590in}}%
\pgfusepath{clip}%
\pgfsetbuttcap%
\pgfsetroundjoin%
\definecolor{currentfill}{rgb}{1.000000,0.498039,0.054902}%
\pgfsetfillcolor{currentfill}%
\pgfsetlinewidth{1.003750pt}%
\definecolor{currentstroke}{rgb}{1.000000,0.498039,0.054902}%
\pgfsetstrokecolor{currentstroke}%
\pgfsetdash{}{0pt}%
\pgfpathmoveto{\pgfqpoint{2.212521in}{3.204665in}}%
\pgfpathcurveto{\pgfqpoint{2.223571in}{3.204665in}}{\pgfqpoint{2.234170in}{3.209056in}}{\pgfqpoint{2.241983in}{3.216869in}}%
\pgfpathcurveto{\pgfqpoint{2.249797in}{3.224683in}}{\pgfqpoint{2.254187in}{3.235282in}}{\pgfqpoint{2.254187in}{3.246332in}}%
\pgfpathcurveto{\pgfqpoint{2.254187in}{3.257382in}}{\pgfqpoint{2.249797in}{3.267981in}}{\pgfqpoint{2.241983in}{3.275795in}}%
\pgfpathcurveto{\pgfqpoint{2.234170in}{3.283608in}}{\pgfqpoint{2.223571in}{3.287999in}}{\pgfqpoint{2.212521in}{3.287999in}}%
\pgfpathcurveto{\pgfqpoint{2.201470in}{3.287999in}}{\pgfqpoint{2.190871in}{3.283608in}}{\pgfqpoint{2.183058in}{3.275795in}}%
\pgfpathcurveto{\pgfqpoint{2.175244in}{3.267981in}}{\pgfqpoint{2.170854in}{3.257382in}}{\pgfqpoint{2.170854in}{3.246332in}}%
\pgfpathcurveto{\pgfqpoint{2.170854in}{3.235282in}}{\pgfqpoint{2.175244in}{3.224683in}}{\pgfqpoint{2.183058in}{3.216869in}}%
\pgfpathcurveto{\pgfqpoint{2.190871in}{3.209056in}}{\pgfqpoint{2.201470in}{3.204665in}}{\pgfqpoint{2.212521in}{3.204665in}}%
\pgfpathclose%
\pgfusepath{stroke,fill}%
\end{pgfscope}%
\begin{pgfscope}%
\pgfpathrectangle{\pgfqpoint{0.648703in}{0.548769in}}{\pgfqpoint{5.112893in}{3.102590in}}%
\pgfusepath{clip}%
\pgfsetbuttcap%
\pgfsetroundjoin%
\definecolor{currentfill}{rgb}{0.121569,0.466667,0.705882}%
\pgfsetfillcolor{currentfill}%
\pgfsetlinewidth{1.003750pt}%
\definecolor{currentstroke}{rgb}{0.121569,0.466667,0.705882}%
\pgfsetstrokecolor{currentstroke}%
\pgfsetdash{}{0pt}%
\pgfpathmoveto{\pgfqpoint{1.578730in}{3.188165in}}%
\pgfpathcurveto{\pgfqpoint{1.589780in}{3.188165in}}{\pgfqpoint{1.600379in}{3.192556in}}{\pgfqpoint{1.608192in}{3.200369in}}%
\pgfpathcurveto{\pgfqpoint{1.616006in}{3.208183in}}{\pgfqpoint{1.620396in}{3.218782in}}{\pgfqpoint{1.620396in}{3.229832in}}%
\pgfpathcurveto{\pgfqpoint{1.620396in}{3.240882in}}{\pgfqpoint{1.616006in}{3.251481in}}{\pgfqpoint{1.608192in}{3.259295in}}%
\pgfpathcurveto{\pgfqpoint{1.600379in}{3.267109in}}{\pgfqpoint{1.589780in}{3.271499in}}{\pgfqpoint{1.578730in}{3.271499in}}%
\pgfpathcurveto{\pgfqpoint{1.567680in}{3.271499in}}{\pgfqpoint{1.557081in}{3.267109in}}{\pgfqpoint{1.549267in}{3.259295in}}%
\pgfpathcurveto{\pgfqpoint{1.541453in}{3.251481in}}{\pgfqpoint{1.537063in}{3.240882in}}{\pgfqpoint{1.537063in}{3.229832in}}%
\pgfpathcurveto{\pgfqpoint{1.537063in}{3.218782in}}{\pgfqpoint{1.541453in}{3.208183in}}{\pgfqpoint{1.549267in}{3.200369in}}%
\pgfpathcurveto{\pgfqpoint{1.557081in}{3.192556in}}{\pgfqpoint{1.567680in}{3.188165in}}{\pgfqpoint{1.578730in}{3.188165in}}%
\pgfpathclose%
\pgfusepath{stroke,fill}%
\end{pgfscope}%
\begin{pgfscope}%
\pgfpathrectangle{\pgfqpoint{0.648703in}{0.548769in}}{\pgfqpoint{5.112893in}{3.102590in}}%
\pgfusepath{clip}%
\pgfsetbuttcap%
\pgfsetroundjoin%
\definecolor{currentfill}{rgb}{0.121569,0.466667,0.705882}%
\pgfsetfillcolor{currentfill}%
\pgfsetlinewidth{1.003750pt}%
\definecolor{currentstroke}{rgb}{0.121569,0.466667,0.705882}%
\pgfsetstrokecolor{currentstroke}%
\pgfsetdash{}{0pt}%
\pgfpathmoveto{\pgfqpoint{0.868436in}{1.001918in}}%
\pgfpathcurveto{\pgfqpoint{0.879486in}{1.001918in}}{\pgfqpoint{0.890085in}{1.006308in}}{\pgfqpoint{0.897898in}{1.014122in}}%
\pgfpathcurveto{\pgfqpoint{0.905712in}{1.021935in}}{\pgfqpoint{0.910102in}{1.032534in}}{\pgfqpoint{0.910102in}{1.043584in}}%
\pgfpathcurveto{\pgfqpoint{0.910102in}{1.054635in}}{\pgfqpoint{0.905712in}{1.065234in}}{\pgfqpoint{0.897898in}{1.073047in}}%
\pgfpathcurveto{\pgfqpoint{0.890085in}{1.080861in}}{\pgfqpoint{0.879486in}{1.085251in}}{\pgfqpoint{0.868436in}{1.085251in}}%
\pgfpathcurveto{\pgfqpoint{0.857385in}{1.085251in}}{\pgfqpoint{0.846786in}{1.080861in}}{\pgfqpoint{0.838973in}{1.073047in}}%
\pgfpathcurveto{\pgfqpoint{0.831159in}{1.065234in}}{\pgfqpoint{0.826769in}{1.054635in}}{\pgfqpoint{0.826769in}{1.043584in}}%
\pgfpathcurveto{\pgfqpoint{0.826769in}{1.032534in}}{\pgfqpoint{0.831159in}{1.021935in}}{\pgfqpoint{0.838973in}{1.014122in}}%
\pgfpathcurveto{\pgfqpoint{0.846786in}{1.006308in}}{\pgfqpoint{0.857385in}{1.001918in}}{\pgfqpoint{0.868436in}{1.001918in}}%
\pgfpathclose%
\pgfusepath{stroke,fill}%
\end{pgfscope}%
\begin{pgfscope}%
\pgfpathrectangle{\pgfqpoint{0.648703in}{0.548769in}}{\pgfqpoint{5.112893in}{3.102590in}}%
\pgfusepath{clip}%
\pgfsetbuttcap%
\pgfsetroundjoin%
\definecolor{currentfill}{rgb}{0.839216,0.152941,0.156863}%
\pgfsetfillcolor{currentfill}%
\pgfsetlinewidth{1.003750pt}%
\definecolor{currentstroke}{rgb}{0.839216,0.152941,0.156863}%
\pgfsetstrokecolor{currentstroke}%
\pgfsetdash{}{0pt}%
\pgfpathmoveto{\pgfqpoint{1.602803in}{3.217040in}}%
\pgfpathcurveto{\pgfqpoint{1.613853in}{3.217040in}}{\pgfqpoint{1.624452in}{3.221431in}}{\pgfqpoint{1.632266in}{3.229244in}}%
\pgfpathcurveto{\pgfqpoint{1.640080in}{3.237058in}}{\pgfqpoint{1.644470in}{3.247657in}}{\pgfqpoint{1.644470in}{3.258707in}}%
\pgfpathcurveto{\pgfqpoint{1.644470in}{3.269757in}}{\pgfqpoint{1.640080in}{3.280356in}}{\pgfqpoint{1.632266in}{3.288170in}}%
\pgfpathcurveto{\pgfqpoint{1.624452in}{3.295983in}}{\pgfqpoint{1.613853in}{3.300374in}}{\pgfqpoint{1.602803in}{3.300374in}}%
\pgfpathcurveto{\pgfqpoint{1.591753in}{3.300374in}}{\pgfqpoint{1.581154in}{3.295983in}}{\pgfqpoint{1.573340in}{3.288170in}}%
\pgfpathcurveto{\pgfqpoint{1.565527in}{3.280356in}}{\pgfqpoint{1.561137in}{3.269757in}}{\pgfqpoint{1.561137in}{3.258707in}}%
\pgfpathcurveto{\pgfqpoint{1.561137in}{3.247657in}}{\pgfqpoint{1.565527in}{3.237058in}}{\pgfqpoint{1.573340in}{3.229244in}}%
\pgfpathcurveto{\pgfqpoint{1.581154in}{3.221431in}}{\pgfqpoint{1.591753in}{3.217040in}}{\pgfqpoint{1.602803in}{3.217040in}}%
\pgfpathclose%
\pgfusepath{stroke,fill}%
\end{pgfscope}%
\begin{pgfscope}%
\pgfpathrectangle{\pgfqpoint{0.648703in}{0.548769in}}{\pgfqpoint{5.112893in}{3.102590in}}%
\pgfusepath{clip}%
\pgfsetbuttcap%
\pgfsetroundjoin%
\definecolor{currentfill}{rgb}{1.000000,0.498039,0.054902}%
\pgfsetfillcolor{currentfill}%
\pgfsetlinewidth{1.003750pt}%
\definecolor{currentstroke}{rgb}{1.000000,0.498039,0.054902}%
\pgfsetstrokecolor{currentstroke}%
\pgfsetdash{}{0pt}%
\pgfpathmoveto{\pgfqpoint{1.802287in}{3.208790in}}%
\pgfpathcurveto{\pgfqpoint{1.813337in}{3.208790in}}{\pgfqpoint{1.823936in}{3.213181in}}{\pgfqpoint{1.831750in}{3.220994in}}%
\pgfpathcurveto{\pgfqpoint{1.839564in}{3.228808in}}{\pgfqpoint{1.843954in}{3.239407in}}{\pgfqpoint{1.843954in}{3.250457in}}%
\pgfpathcurveto{\pgfqpoint{1.843954in}{3.261507in}}{\pgfqpoint{1.839564in}{3.272106in}}{\pgfqpoint{1.831750in}{3.279920in}}%
\pgfpathcurveto{\pgfqpoint{1.823936in}{3.287733in}}{\pgfqpoint{1.813337in}{3.292124in}}{\pgfqpoint{1.802287in}{3.292124in}}%
\pgfpathcurveto{\pgfqpoint{1.791237in}{3.292124in}}{\pgfqpoint{1.780638in}{3.287733in}}{\pgfqpoint{1.772824in}{3.279920in}}%
\pgfpathcurveto{\pgfqpoint{1.765011in}{3.272106in}}{\pgfqpoint{1.760620in}{3.261507in}}{\pgfqpoint{1.760620in}{3.250457in}}%
\pgfpathcurveto{\pgfqpoint{1.760620in}{3.239407in}}{\pgfqpoint{1.765011in}{3.228808in}}{\pgfqpoint{1.772824in}{3.220994in}}%
\pgfpathcurveto{\pgfqpoint{1.780638in}{3.213181in}}{\pgfqpoint{1.791237in}{3.208790in}}{\pgfqpoint{1.802287in}{3.208790in}}%
\pgfpathclose%
\pgfusepath{stroke,fill}%
\end{pgfscope}%
\begin{pgfscope}%
\pgfpathrectangle{\pgfqpoint{0.648703in}{0.548769in}}{\pgfqpoint{5.112893in}{3.102590in}}%
\pgfusepath{clip}%
\pgfsetbuttcap%
\pgfsetroundjoin%
\definecolor{currentfill}{rgb}{1.000000,0.498039,0.054902}%
\pgfsetfillcolor{currentfill}%
\pgfsetlinewidth{1.003750pt}%
\definecolor{currentstroke}{rgb}{1.000000,0.498039,0.054902}%
\pgfsetstrokecolor{currentstroke}%
\pgfsetdash{}{0pt}%
\pgfpathmoveto{\pgfqpoint{1.532348in}{3.212915in}}%
\pgfpathcurveto{\pgfqpoint{1.543398in}{3.212915in}}{\pgfqpoint{1.553997in}{3.217306in}}{\pgfqpoint{1.561810in}{3.225119in}}%
\pgfpathcurveto{\pgfqpoint{1.569624in}{3.232933in}}{\pgfqpoint{1.574014in}{3.243532in}}{\pgfqpoint{1.574014in}{3.254582in}}%
\pgfpathcurveto{\pgfqpoint{1.574014in}{3.265632in}}{\pgfqpoint{1.569624in}{3.276231in}}{\pgfqpoint{1.561810in}{3.284045in}}%
\pgfpathcurveto{\pgfqpoint{1.553997in}{3.291858in}}{\pgfqpoint{1.543398in}{3.296249in}}{\pgfqpoint{1.532348in}{3.296249in}}%
\pgfpathcurveto{\pgfqpoint{1.521297in}{3.296249in}}{\pgfqpoint{1.510698in}{3.291858in}}{\pgfqpoint{1.502885in}{3.284045in}}%
\pgfpathcurveto{\pgfqpoint{1.495071in}{3.276231in}}{\pgfqpoint{1.490681in}{3.265632in}}{\pgfqpoint{1.490681in}{3.254582in}}%
\pgfpathcurveto{\pgfqpoint{1.490681in}{3.243532in}}{\pgfqpoint{1.495071in}{3.232933in}}{\pgfqpoint{1.502885in}{3.225119in}}%
\pgfpathcurveto{\pgfqpoint{1.510698in}{3.217306in}}{\pgfqpoint{1.521297in}{3.212915in}}{\pgfqpoint{1.532348in}{3.212915in}}%
\pgfpathclose%
\pgfusepath{stroke,fill}%
\end{pgfscope}%
\begin{pgfscope}%
\pgfpathrectangle{\pgfqpoint{0.648703in}{0.548769in}}{\pgfqpoint{5.112893in}{3.102590in}}%
\pgfusepath{clip}%
\pgfsetbuttcap%
\pgfsetroundjoin%
\definecolor{currentfill}{rgb}{1.000000,0.498039,0.054902}%
\pgfsetfillcolor{currentfill}%
\pgfsetlinewidth{1.003750pt}%
\definecolor{currentstroke}{rgb}{1.000000,0.498039,0.054902}%
\pgfsetstrokecolor{currentstroke}%
\pgfsetdash{}{0pt}%
\pgfpathmoveto{\pgfqpoint{1.525921in}{3.204665in}}%
\pgfpathcurveto{\pgfqpoint{1.536971in}{3.204665in}}{\pgfqpoint{1.547570in}{3.209056in}}{\pgfqpoint{1.555384in}{3.216869in}}%
\pgfpathcurveto{\pgfqpoint{1.563198in}{3.224683in}}{\pgfqpoint{1.567588in}{3.235282in}}{\pgfqpoint{1.567588in}{3.246332in}}%
\pgfpathcurveto{\pgfqpoint{1.567588in}{3.257382in}}{\pgfqpoint{1.563198in}{3.267981in}}{\pgfqpoint{1.555384in}{3.275795in}}%
\pgfpathcurveto{\pgfqpoint{1.547570in}{3.283608in}}{\pgfqpoint{1.536971in}{3.287999in}}{\pgfqpoint{1.525921in}{3.287999in}}%
\pgfpathcurveto{\pgfqpoint{1.514871in}{3.287999in}}{\pgfqpoint{1.504272in}{3.283608in}}{\pgfqpoint{1.496459in}{3.275795in}}%
\pgfpathcurveto{\pgfqpoint{1.488645in}{3.267981in}}{\pgfqpoint{1.484255in}{3.257382in}}{\pgfqpoint{1.484255in}{3.246332in}}%
\pgfpathcurveto{\pgfqpoint{1.484255in}{3.235282in}}{\pgfqpoint{1.488645in}{3.224683in}}{\pgfqpoint{1.496459in}{3.216869in}}%
\pgfpathcurveto{\pgfqpoint{1.504272in}{3.209056in}}{\pgfqpoint{1.514871in}{3.204665in}}{\pgfqpoint{1.525921in}{3.204665in}}%
\pgfpathclose%
\pgfusepath{stroke,fill}%
\end{pgfscope}%
\begin{pgfscope}%
\pgfpathrectangle{\pgfqpoint{0.648703in}{0.548769in}}{\pgfqpoint{5.112893in}{3.102590in}}%
\pgfusepath{clip}%
\pgfsetbuttcap%
\pgfsetroundjoin%
\definecolor{currentfill}{rgb}{1.000000,0.498039,0.054902}%
\pgfsetfillcolor{currentfill}%
\pgfsetlinewidth{1.003750pt}%
\definecolor{currentstroke}{rgb}{1.000000,0.498039,0.054902}%
\pgfsetstrokecolor{currentstroke}%
\pgfsetdash{}{0pt}%
\pgfpathmoveto{\pgfqpoint{1.945733in}{3.208790in}}%
\pgfpathcurveto{\pgfqpoint{1.956783in}{3.208790in}}{\pgfqpoint{1.967382in}{3.213181in}}{\pgfqpoint{1.975196in}{3.220994in}}%
\pgfpathcurveto{\pgfqpoint{1.983009in}{3.228808in}}{\pgfqpoint{1.987399in}{3.239407in}}{\pgfqpoint{1.987399in}{3.250457in}}%
\pgfpathcurveto{\pgfqpoint{1.987399in}{3.261507in}}{\pgfqpoint{1.983009in}{3.272106in}}{\pgfqpoint{1.975196in}{3.279920in}}%
\pgfpathcurveto{\pgfqpoint{1.967382in}{3.287733in}}{\pgfqpoint{1.956783in}{3.292124in}}{\pgfqpoint{1.945733in}{3.292124in}}%
\pgfpathcurveto{\pgfqpoint{1.934683in}{3.292124in}}{\pgfqpoint{1.924084in}{3.287733in}}{\pgfqpoint{1.916270in}{3.279920in}}%
\pgfpathcurveto{\pgfqpoint{1.908456in}{3.272106in}}{\pgfqpoint{1.904066in}{3.261507in}}{\pgfqpoint{1.904066in}{3.250457in}}%
\pgfpathcurveto{\pgfqpoint{1.904066in}{3.239407in}}{\pgfqpoint{1.908456in}{3.228808in}}{\pgfqpoint{1.916270in}{3.220994in}}%
\pgfpathcurveto{\pgfqpoint{1.924084in}{3.213181in}}{\pgfqpoint{1.934683in}{3.208790in}}{\pgfqpoint{1.945733in}{3.208790in}}%
\pgfpathclose%
\pgfusepath{stroke,fill}%
\end{pgfscope}%
\begin{pgfscope}%
\pgfpathrectangle{\pgfqpoint{0.648703in}{0.548769in}}{\pgfqpoint{5.112893in}{3.102590in}}%
\pgfusepath{clip}%
\pgfsetbuttcap%
\pgfsetroundjoin%
\definecolor{currentfill}{rgb}{1.000000,0.498039,0.054902}%
\pgfsetfillcolor{currentfill}%
\pgfsetlinewidth{1.003750pt}%
\definecolor{currentstroke}{rgb}{1.000000,0.498039,0.054902}%
\pgfsetstrokecolor{currentstroke}%
\pgfsetdash{}{0pt}%
\pgfpathmoveto{\pgfqpoint{1.650322in}{3.196415in}}%
\pgfpathcurveto{\pgfqpoint{1.661372in}{3.196415in}}{\pgfqpoint{1.671971in}{3.200806in}}{\pgfqpoint{1.679785in}{3.208619in}}%
\pgfpathcurveto{\pgfqpoint{1.687598in}{3.216433in}}{\pgfqpoint{1.691988in}{3.227032in}}{\pgfqpoint{1.691988in}{3.238082in}}%
\pgfpathcurveto{\pgfqpoint{1.691988in}{3.249132in}}{\pgfqpoint{1.687598in}{3.259731in}}{\pgfqpoint{1.679785in}{3.267545in}}%
\pgfpathcurveto{\pgfqpoint{1.671971in}{3.275358in}}{\pgfqpoint{1.661372in}{3.279749in}}{\pgfqpoint{1.650322in}{3.279749in}}%
\pgfpathcurveto{\pgfqpoint{1.639272in}{3.279749in}}{\pgfqpoint{1.628673in}{3.275358in}}{\pgfqpoint{1.620859in}{3.267545in}}%
\pgfpathcurveto{\pgfqpoint{1.613045in}{3.259731in}}{\pgfqpoint{1.608655in}{3.249132in}}{\pgfqpoint{1.608655in}{3.238082in}}%
\pgfpathcurveto{\pgfqpoint{1.608655in}{3.227032in}}{\pgfqpoint{1.613045in}{3.216433in}}{\pgfqpoint{1.620859in}{3.208619in}}%
\pgfpathcurveto{\pgfqpoint{1.628673in}{3.200806in}}{\pgfqpoint{1.639272in}{3.196415in}}{\pgfqpoint{1.650322in}{3.196415in}}%
\pgfpathclose%
\pgfusepath{stroke,fill}%
\end{pgfscope}%
\begin{pgfscope}%
\pgfpathrectangle{\pgfqpoint{0.648703in}{0.548769in}}{\pgfqpoint{5.112893in}{3.102590in}}%
\pgfusepath{clip}%
\pgfsetbuttcap%
\pgfsetroundjoin%
\definecolor{currentfill}{rgb}{0.121569,0.466667,0.705882}%
\pgfsetfillcolor{currentfill}%
\pgfsetlinewidth{1.003750pt}%
\definecolor{currentstroke}{rgb}{0.121569,0.466667,0.705882}%
\pgfsetstrokecolor{currentstroke}%
\pgfsetdash{}{0pt}%
\pgfpathmoveto{\pgfqpoint{1.503930in}{3.188165in}}%
\pgfpathcurveto{\pgfqpoint{1.514980in}{3.188165in}}{\pgfqpoint{1.525579in}{3.192556in}}{\pgfqpoint{1.533392in}{3.200369in}}%
\pgfpathcurveto{\pgfqpoint{1.541206in}{3.208183in}}{\pgfqpoint{1.545596in}{3.218782in}}{\pgfqpoint{1.545596in}{3.229832in}}%
\pgfpathcurveto{\pgfqpoint{1.545596in}{3.240882in}}{\pgfqpoint{1.541206in}{3.251481in}}{\pgfqpoint{1.533392in}{3.259295in}}%
\pgfpathcurveto{\pgfqpoint{1.525579in}{3.267109in}}{\pgfqpoint{1.514980in}{3.271499in}}{\pgfqpoint{1.503930in}{3.271499in}}%
\pgfpathcurveto{\pgfqpoint{1.492879in}{3.271499in}}{\pgfqpoint{1.482280in}{3.267109in}}{\pgfqpoint{1.474467in}{3.259295in}}%
\pgfpathcurveto{\pgfqpoint{1.466653in}{3.251481in}}{\pgfqpoint{1.462263in}{3.240882in}}{\pgfqpoint{1.462263in}{3.229832in}}%
\pgfpathcurveto{\pgfqpoint{1.462263in}{3.218782in}}{\pgfqpoint{1.466653in}{3.208183in}}{\pgfqpoint{1.474467in}{3.200369in}}%
\pgfpathcurveto{\pgfqpoint{1.482280in}{3.192556in}}{\pgfqpoint{1.492879in}{3.188165in}}{\pgfqpoint{1.503930in}{3.188165in}}%
\pgfpathclose%
\pgfusepath{stroke,fill}%
\end{pgfscope}%
\begin{pgfscope}%
\pgfpathrectangle{\pgfqpoint{0.648703in}{0.548769in}}{\pgfqpoint{5.112893in}{3.102590in}}%
\pgfusepath{clip}%
\pgfsetbuttcap%
\pgfsetroundjoin%
\definecolor{currentfill}{rgb}{1.000000,0.498039,0.054902}%
\pgfsetfillcolor{currentfill}%
\pgfsetlinewidth{1.003750pt}%
\definecolor{currentstroke}{rgb}{1.000000,0.498039,0.054902}%
\pgfsetstrokecolor{currentstroke}%
\pgfsetdash{}{0pt}%
\pgfpathmoveto{\pgfqpoint{1.497860in}{3.250040in}}%
\pgfpathcurveto{\pgfqpoint{1.508910in}{3.250040in}}{\pgfqpoint{1.519509in}{3.254431in}}{\pgfqpoint{1.527322in}{3.262244in}}%
\pgfpathcurveto{\pgfqpoint{1.535136in}{3.270058in}}{\pgfqpoint{1.539526in}{3.280657in}}{\pgfqpoint{1.539526in}{3.291707in}}%
\pgfpathcurveto{\pgfqpoint{1.539526in}{3.302757in}}{\pgfqpoint{1.535136in}{3.313356in}}{\pgfqpoint{1.527322in}{3.321170in}}%
\pgfpathcurveto{\pgfqpoint{1.519509in}{3.328983in}}{\pgfqpoint{1.508910in}{3.333374in}}{\pgfqpoint{1.497860in}{3.333374in}}%
\pgfpathcurveto{\pgfqpoint{1.486810in}{3.333374in}}{\pgfqpoint{1.476211in}{3.328983in}}{\pgfqpoint{1.468397in}{3.321170in}}%
\pgfpathcurveto{\pgfqpoint{1.460583in}{3.313356in}}{\pgfqpoint{1.456193in}{3.302757in}}{\pgfqpoint{1.456193in}{3.291707in}}%
\pgfpathcurveto{\pgfqpoint{1.456193in}{3.280657in}}{\pgfqpoint{1.460583in}{3.270058in}}{\pgfqpoint{1.468397in}{3.262244in}}%
\pgfpathcurveto{\pgfqpoint{1.476211in}{3.254431in}}{\pgfqpoint{1.486810in}{3.250040in}}{\pgfqpoint{1.497860in}{3.250040in}}%
\pgfpathclose%
\pgfusepath{stroke,fill}%
\end{pgfscope}%
\begin{pgfscope}%
\pgfpathrectangle{\pgfqpoint{0.648703in}{0.548769in}}{\pgfqpoint{5.112893in}{3.102590in}}%
\pgfusepath{clip}%
\pgfsetbuttcap%
\pgfsetroundjoin%
\definecolor{currentfill}{rgb}{1.000000,0.498039,0.054902}%
\pgfsetfillcolor{currentfill}%
\pgfsetlinewidth{1.003750pt}%
\definecolor{currentstroke}{rgb}{1.000000,0.498039,0.054902}%
\pgfsetstrokecolor{currentstroke}%
\pgfsetdash{}{0pt}%
\pgfpathmoveto{\pgfqpoint{1.507387in}{3.196415in}}%
\pgfpathcurveto{\pgfqpoint{1.518437in}{3.196415in}}{\pgfqpoint{1.529036in}{3.200806in}}{\pgfqpoint{1.536850in}{3.208619in}}%
\pgfpathcurveto{\pgfqpoint{1.544664in}{3.216433in}}{\pgfqpoint{1.549054in}{3.227032in}}{\pgfqpoint{1.549054in}{3.238082in}}%
\pgfpathcurveto{\pgfqpoint{1.549054in}{3.249132in}}{\pgfqpoint{1.544664in}{3.259731in}}{\pgfqpoint{1.536850in}{3.267545in}}%
\pgfpathcurveto{\pgfqpoint{1.529036in}{3.275358in}}{\pgfqpoint{1.518437in}{3.279749in}}{\pgfqpoint{1.507387in}{3.279749in}}%
\pgfpathcurveto{\pgfqpoint{1.496337in}{3.279749in}}{\pgfqpoint{1.485738in}{3.275358in}}{\pgfqpoint{1.477925in}{3.267545in}}%
\pgfpathcurveto{\pgfqpoint{1.470111in}{3.259731in}}{\pgfqpoint{1.465721in}{3.249132in}}{\pgfqpoint{1.465721in}{3.238082in}}%
\pgfpathcurveto{\pgfqpoint{1.465721in}{3.227032in}}{\pgfqpoint{1.470111in}{3.216433in}}{\pgfqpoint{1.477925in}{3.208619in}}%
\pgfpathcurveto{\pgfqpoint{1.485738in}{3.200806in}}{\pgfqpoint{1.496337in}{3.196415in}}{\pgfqpoint{1.507387in}{3.196415in}}%
\pgfpathclose%
\pgfusepath{stroke,fill}%
\end{pgfscope}%
\begin{pgfscope}%
\pgfpathrectangle{\pgfqpoint{0.648703in}{0.548769in}}{\pgfqpoint{5.112893in}{3.102590in}}%
\pgfusepath{clip}%
\pgfsetbuttcap%
\pgfsetroundjoin%
\definecolor{currentfill}{rgb}{1.000000,0.498039,0.054902}%
\pgfsetfillcolor{currentfill}%
\pgfsetlinewidth{1.003750pt}%
\definecolor{currentstroke}{rgb}{1.000000,0.498039,0.054902}%
\pgfsetstrokecolor{currentstroke}%
\pgfsetdash{}{0pt}%
\pgfpathmoveto{\pgfqpoint{2.637762in}{3.217040in}}%
\pgfpathcurveto{\pgfqpoint{2.648812in}{3.217040in}}{\pgfqpoint{2.659411in}{3.221431in}}{\pgfqpoint{2.667224in}{3.229244in}}%
\pgfpathcurveto{\pgfqpoint{2.675038in}{3.237058in}}{\pgfqpoint{2.679428in}{3.247657in}}{\pgfqpoint{2.679428in}{3.258707in}}%
\pgfpathcurveto{\pgfqpoint{2.679428in}{3.269757in}}{\pgfqpoint{2.675038in}{3.280356in}}{\pgfqpoint{2.667224in}{3.288170in}}%
\pgfpathcurveto{\pgfqpoint{2.659411in}{3.295983in}}{\pgfqpoint{2.648812in}{3.300374in}}{\pgfqpoint{2.637762in}{3.300374in}}%
\pgfpathcurveto{\pgfqpoint{2.626711in}{3.300374in}}{\pgfqpoint{2.616112in}{3.295983in}}{\pgfqpoint{2.608299in}{3.288170in}}%
\pgfpathcurveto{\pgfqpoint{2.600485in}{3.280356in}}{\pgfqpoint{2.596095in}{3.269757in}}{\pgfqpoint{2.596095in}{3.258707in}}%
\pgfpathcurveto{\pgfqpoint{2.596095in}{3.247657in}}{\pgfqpoint{2.600485in}{3.237058in}}{\pgfqpoint{2.608299in}{3.229244in}}%
\pgfpathcurveto{\pgfqpoint{2.616112in}{3.221431in}}{\pgfqpoint{2.626711in}{3.217040in}}{\pgfqpoint{2.637762in}{3.217040in}}%
\pgfpathclose%
\pgfusepath{stroke,fill}%
\end{pgfscope}%
\begin{pgfscope}%
\pgfpathrectangle{\pgfqpoint{0.648703in}{0.548769in}}{\pgfqpoint{5.112893in}{3.102590in}}%
\pgfusepath{clip}%
\pgfsetbuttcap%
\pgfsetroundjoin%
\definecolor{currentfill}{rgb}{1.000000,0.498039,0.054902}%
\pgfsetfillcolor{currentfill}%
\pgfsetlinewidth{1.003750pt}%
\definecolor{currentstroke}{rgb}{1.000000,0.498039,0.054902}%
\pgfsetstrokecolor{currentstroke}%
\pgfsetdash{}{0pt}%
\pgfpathmoveto{\pgfqpoint{1.845807in}{3.196415in}}%
\pgfpathcurveto{\pgfqpoint{1.856858in}{3.196415in}}{\pgfqpoint{1.867457in}{3.200806in}}{\pgfqpoint{1.875270in}{3.208619in}}%
\pgfpathcurveto{\pgfqpoint{1.883084in}{3.216433in}}{\pgfqpoint{1.887474in}{3.227032in}}{\pgfqpoint{1.887474in}{3.238082in}}%
\pgfpathcurveto{\pgfqpoint{1.887474in}{3.249132in}}{\pgfqpoint{1.883084in}{3.259731in}}{\pgfqpoint{1.875270in}{3.267545in}}%
\pgfpathcurveto{\pgfqpoint{1.867457in}{3.275358in}}{\pgfqpoint{1.856858in}{3.279749in}}{\pgfqpoint{1.845807in}{3.279749in}}%
\pgfpathcurveto{\pgfqpoint{1.834757in}{3.279749in}}{\pgfqpoint{1.824158in}{3.275358in}}{\pgfqpoint{1.816345in}{3.267545in}}%
\pgfpathcurveto{\pgfqpoint{1.808531in}{3.259731in}}{\pgfqpoint{1.804141in}{3.249132in}}{\pgfqpoint{1.804141in}{3.238082in}}%
\pgfpathcurveto{\pgfqpoint{1.804141in}{3.227032in}}{\pgfqpoint{1.808531in}{3.216433in}}{\pgfqpoint{1.816345in}{3.208619in}}%
\pgfpathcurveto{\pgfqpoint{1.824158in}{3.200806in}}{\pgfqpoint{1.834757in}{3.196415in}}{\pgfqpoint{1.845807in}{3.196415in}}%
\pgfpathclose%
\pgfusepath{stroke,fill}%
\end{pgfscope}%
\begin{pgfscope}%
\pgfpathrectangle{\pgfqpoint{0.648703in}{0.548769in}}{\pgfqpoint{5.112893in}{3.102590in}}%
\pgfusepath{clip}%
\pgfsetbuttcap%
\pgfsetroundjoin%
\definecolor{currentfill}{rgb}{0.839216,0.152941,0.156863}%
\pgfsetfillcolor{currentfill}%
\pgfsetlinewidth{1.003750pt}%
\definecolor{currentstroke}{rgb}{0.839216,0.152941,0.156863}%
\pgfsetstrokecolor{currentstroke}%
\pgfsetdash{}{0pt}%
\pgfpathmoveto{\pgfqpoint{1.647043in}{3.188165in}}%
\pgfpathcurveto{\pgfqpoint{1.658093in}{3.188165in}}{\pgfqpoint{1.668692in}{3.192556in}}{\pgfqpoint{1.676506in}{3.200369in}}%
\pgfpathcurveto{\pgfqpoint{1.684319in}{3.208183in}}{\pgfqpoint{1.688710in}{3.218782in}}{\pgfqpoint{1.688710in}{3.229832in}}%
\pgfpathcurveto{\pgfqpoint{1.688710in}{3.240882in}}{\pgfqpoint{1.684319in}{3.251481in}}{\pgfqpoint{1.676506in}{3.259295in}}%
\pgfpathcurveto{\pgfqpoint{1.668692in}{3.267109in}}{\pgfqpoint{1.658093in}{3.271499in}}{\pgfqpoint{1.647043in}{3.271499in}}%
\pgfpathcurveto{\pgfqpoint{1.635993in}{3.271499in}}{\pgfqpoint{1.625394in}{3.267109in}}{\pgfqpoint{1.617580in}{3.259295in}}%
\pgfpathcurveto{\pgfqpoint{1.609767in}{3.251481in}}{\pgfqpoint{1.605376in}{3.240882in}}{\pgfqpoint{1.605376in}{3.229832in}}%
\pgfpathcurveto{\pgfqpoint{1.605376in}{3.218782in}}{\pgfqpoint{1.609767in}{3.208183in}}{\pgfqpoint{1.617580in}{3.200369in}}%
\pgfpathcurveto{\pgfqpoint{1.625394in}{3.192556in}}{\pgfqpoint{1.635993in}{3.188165in}}{\pgfqpoint{1.647043in}{3.188165in}}%
\pgfpathclose%
\pgfusepath{stroke,fill}%
\end{pgfscope}%
\begin{pgfscope}%
\pgfpathrectangle{\pgfqpoint{0.648703in}{0.548769in}}{\pgfqpoint{5.112893in}{3.102590in}}%
\pgfusepath{clip}%
\pgfsetbuttcap%
\pgfsetroundjoin%
\definecolor{currentfill}{rgb}{1.000000,0.498039,0.054902}%
\pgfsetfillcolor{currentfill}%
\pgfsetlinewidth{1.003750pt}%
\definecolor{currentstroke}{rgb}{1.000000,0.498039,0.054902}%
\pgfsetstrokecolor{currentstroke}%
\pgfsetdash{}{0pt}%
\pgfpathmoveto{\pgfqpoint{1.555911in}{3.208790in}}%
\pgfpathcurveto{\pgfqpoint{1.566961in}{3.208790in}}{\pgfqpoint{1.577560in}{3.213181in}}{\pgfqpoint{1.585373in}{3.220994in}}%
\pgfpathcurveto{\pgfqpoint{1.593187in}{3.228808in}}{\pgfqpoint{1.597577in}{3.239407in}}{\pgfqpoint{1.597577in}{3.250457in}}%
\pgfpathcurveto{\pgfqpoint{1.597577in}{3.261507in}}{\pgfqpoint{1.593187in}{3.272106in}}{\pgfqpoint{1.585373in}{3.279920in}}%
\pgfpathcurveto{\pgfqpoint{1.577560in}{3.287733in}}{\pgfqpoint{1.566961in}{3.292124in}}{\pgfqpoint{1.555911in}{3.292124in}}%
\pgfpathcurveto{\pgfqpoint{1.544861in}{3.292124in}}{\pgfqpoint{1.534261in}{3.287733in}}{\pgfqpoint{1.526448in}{3.279920in}}%
\pgfpathcurveto{\pgfqpoint{1.518634in}{3.272106in}}{\pgfqpoint{1.514244in}{3.261507in}}{\pgfqpoint{1.514244in}{3.250457in}}%
\pgfpathcurveto{\pgfqpoint{1.514244in}{3.239407in}}{\pgfqpoint{1.518634in}{3.228808in}}{\pgfqpoint{1.526448in}{3.220994in}}%
\pgfpathcurveto{\pgfqpoint{1.534261in}{3.213181in}}{\pgfqpoint{1.544861in}{3.208790in}}{\pgfqpoint{1.555911in}{3.208790in}}%
\pgfpathclose%
\pgfusepath{stroke,fill}%
\end{pgfscope}%
\begin{pgfscope}%
\pgfpathrectangle{\pgfqpoint{0.648703in}{0.548769in}}{\pgfqpoint{5.112893in}{3.102590in}}%
\pgfusepath{clip}%
\pgfsetbuttcap%
\pgfsetroundjoin%
\definecolor{currentfill}{rgb}{0.839216,0.152941,0.156863}%
\pgfsetfillcolor{currentfill}%
\pgfsetlinewidth{1.003750pt}%
\definecolor{currentstroke}{rgb}{0.839216,0.152941,0.156863}%
\pgfsetstrokecolor{currentstroke}%
\pgfsetdash{}{0pt}%
\pgfpathmoveto{\pgfqpoint{1.809362in}{3.270665in}}%
\pgfpathcurveto{\pgfqpoint{1.820412in}{3.270665in}}{\pgfqpoint{1.831011in}{3.275056in}}{\pgfqpoint{1.838825in}{3.282869in}}%
\pgfpathcurveto{\pgfqpoint{1.846639in}{3.290683in}}{\pgfqpoint{1.851029in}{3.301282in}}{\pgfqpoint{1.851029in}{3.312332in}}%
\pgfpathcurveto{\pgfqpoint{1.851029in}{3.323382in}}{\pgfqpoint{1.846639in}{3.333981in}}{\pgfqpoint{1.838825in}{3.341795in}}%
\pgfpathcurveto{\pgfqpoint{1.831011in}{3.349608in}}{\pgfqpoint{1.820412in}{3.353999in}}{\pgfqpoint{1.809362in}{3.353999in}}%
\pgfpathcurveto{\pgfqpoint{1.798312in}{3.353999in}}{\pgfqpoint{1.787713in}{3.349608in}}{\pgfqpoint{1.779899in}{3.341795in}}%
\pgfpathcurveto{\pgfqpoint{1.772086in}{3.333981in}}{\pgfqpoint{1.767695in}{3.323382in}}{\pgfqpoint{1.767695in}{3.312332in}}%
\pgfpathcurveto{\pgfqpoint{1.767695in}{3.301282in}}{\pgfqpoint{1.772086in}{3.290683in}}{\pgfqpoint{1.779899in}{3.282869in}}%
\pgfpathcurveto{\pgfqpoint{1.787713in}{3.275056in}}{\pgfqpoint{1.798312in}{3.270665in}}{\pgfqpoint{1.809362in}{3.270665in}}%
\pgfpathclose%
\pgfusepath{stroke,fill}%
\end{pgfscope}%
\begin{pgfscope}%
\pgfpathrectangle{\pgfqpoint{0.648703in}{0.548769in}}{\pgfqpoint{5.112893in}{3.102590in}}%
\pgfusepath{clip}%
\pgfsetbuttcap%
\pgfsetroundjoin%
\definecolor{currentfill}{rgb}{0.121569,0.466667,0.705882}%
\pgfsetfillcolor{currentfill}%
\pgfsetlinewidth{1.003750pt}%
\definecolor{currentstroke}{rgb}{0.121569,0.466667,0.705882}%
\pgfsetstrokecolor{currentstroke}%
\pgfsetdash{}{0pt}%
\pgfpathmoveto{\pgfqpoint{3.129461in}{2.825166in}}%
\pgfpathcurveto{\pgfqpoint{3.140511in}{2.825166in}}{\pgfqpoint{3.151110in}{2.829556in}}{\pgfqpoint{3.158924in}{2.837370in}}%
\pgfpathcurveto{\pgfqpoint{3.166737in}{2.845183in}}{\pgfqpoint{3.171127in}{2.855782in}}{\pgfqpoint{3.171127in}{2.866832in}}%
\pgfpathcurveto{\pgfqpoint{3.171127in}{2.877883in}}{\pgfqpoint{3.166737in}{2.888482in}}{\pgfqpoint{3.158924in}{2.896295in}}%
\pgfpathcurveto{\pgfqpoint{3.151110in}{2.904109in}}{\pgfqpoint{3.140511in}{2.908499in}}{\pgfqpoint{3.129461in}{2.908499in}}%
\pgfpathcurveto{\pgfqpoint{3.118411in}{2.908499in}}{\pgfqpoint{3.107812in}{2.904109in}}{\pgfqpoint{3.099998in}{2.896295in}}%
\pgfpathcurveto{\pgfqpoint{3.092184in}{2.888482in}}{\pgfqpoint{3.087794in}{2.877883in}}{\pgfqpoint{3.087794in}{2.866832in}}%
\pgfpathcurveto{\pgfqpoint{3.087794in}{2.855782in}}{\pgfqpoint{3.092184in}{2.845183in}}{\pgfqpoint{3.099998in}{2.837370in}}%
\pgfpathcurveto{\pgfqpoint{3.107812in}{2.829556in}}{\pgfqpoint{3.118411in}{2.825166in}}{\pgfqpoint{3.129461in}{2.825166in}}%
\pgfpathclose%
\pgfusepath{stroke,fill}%
\end{pgfscope}%
\begin{pgfscope}%
\pgfpathrectangle{\pgfqpoint{0.648703in}{0.548769in}}{\pgfqpoint{5.112893in}{3.102590in}}%
\pgfusepath{clip}%
\pgfsetbuttcap%
\pgfsetroundjoin%
\definecolor{currentfill}{rgb}{1.000000,0.498039,0.054902}%
\pgfsetfillcolor{currentfill}%
\pgfsetlinewidth{1.003750pt}%
\definecolor{currentstroke}{rgb}{1.000000,0.498039,0.054902}%
\pgfsetstrokecolor{currentstroke}%
\pgfsetdash{}{0pt}%
\pgfpathmoveto{\pgfqpoint{1.716889in}{3.192290in}}%
\pgfpathcurveto{\pgfqpoint{1.727939in}{3.192290in}}{\pgfqpoint{1.738538in}{3.196681in}}{\pgfqpoint{1.746351in}{3.204494in}}%
\pgfpathcurveto{\pgfqpoint{1.754165in}{3.212308in}}{\pgfqpoint{1.758555in}{3.222907in}}{\pgfqpoint{1.758555in}{3.233957in}}%
\pgfpathcurveto{\pgfqpoint{1.758555in}{3.245007in}}{\pgfqpoint{1.754165in}{3.255606in}}{\pgfqpoint{1.746351in}{3.263420in}}%
\pgfpathcurveto{\pgfqpoint{1.738538in}{3.271234in}}{\pgfqpoint{1.727939in}{3.275624in}}{\pgfqpoint{1.716889in}{3.275624in}}%
\pgfpathcurveto{\pgfqpoint{1.705839in}{3.275624in}}{\pgfqpoint{1.695239in}{3.271234in}}{\pgfqpoint{1.687426in}{3.263420in}}%
\pgfpathcurveto{\pgfqpoint{1.679612in}{3.255606in}}{\pgfqpoint{1.675222in}{3.245007in}}{\pgfqpoint{1.675222in}{3.233957in}}%
\pgfpathcurveto{\pgfqpoint{1.675222in}{3.222907in}}{\pgfqpoint{1.679612in}{3.212308in}}{\pgfqpoint{1.687426in}{3.204494in}}%
\pgfpathcurveto{\pgfqpoint{1.695239in}{3.196681in}}{\pgfqpoint{1.705839in}{3.192290in}}{\pgfqpoint{1.716889in}{3.192290in}}%
\pgfpathclose%
\pgfusepath{stroke,fill}%
\end{pgfscope}%
\begin{pgfscope}%
\pgfpathrectangle{\pgfqpoint{0.648703in}{0.548769in}}{\pgfqpoint{5.112893in}{3.102590in}}%
\pgfusepath{clip}%
\pgfsetbuttcap%
\pgfsetroundjoin%
\definecolor{currentfill}{rgb}{1.000000,0.498039,0.054902}%
\pgfsetfillcolor{currentfill}%
\pgfsetlinewidth{1.003750pt}%
\definecolor{currentstroke}{rgb}{1.000000,0.498039,0.054902}%
\pgfsetstrokecolor{currentstroke}%
\pgfsetdash{}{0pt}%
\pgfpathmoveto{\pgfqpoint{1.674065in}{3.200540in}}%
\pgfpathcurveto{\pgfqpoint{1.685115in}{3.200540in}}{\pgfqpoint{1.695714in}{3.204931in}}{\pgfqpoint{1.703528in}{3.212744in}}%
\pgfpathcurveto{\pgfqpoint{1.711342in}{3.220558in}}{\pgfqpoint{1.715732in}{3.231157in}}{\pgfqpoint{1.715732in}{3.242207in}}%
\pgfpathcurveto{\pgfqpoint{1.715732in}{3.253257in}}{\pgfqpoint{1.711342in}{3.263856in}}{\pgfqpoint{1.703528in}{3.271670in}}%
\pgfpathcurveto{\pgfqpoint{1.695714in}{3.279483in}}{\pgfqpoint{1.685115in}{3.283874in}}{\pgfqpoint{1.674065in}{3.283874in}}%
\pgfpathcurveto{\pgfqpoint{1.663015in}{3.283874in}}{\pgfqpoint{1.652416in}{3.279483in}}{\pgfqpoint{1.644603in}{3.271670in}}%
\pgfpathcurveto{\pgfqpoint{1.636789in}{3.263856in}}{\pgfqpoint{1.632399in}{3.253257in}}{\pgfqpoint{1.632399in}{3.242207in}}%
\pgfpathcurveto{\pgfqpoint{1.632399in}{3.231157in}}{\pgfqpoint{1.636789in}{3.220558in}}{\pgfqpoint{1.644603in}{3.212744in}}%
\pgfpathcurveto{\pgfqpoint{1.652416in}{3.204931in}}{\pgfqpoint{1.663015in}{3.200540in}}{\pgfqpoint{1.674065in}{3.200540in}}%
\pgfpathclose%
\pgfusepath{stroke,fill}%
\end{pgfscope}%
\begin{pgfscope}%
\pgfpathrectangle{\pgfqpoint{0.648703in}{0.548769in}}{\pgfqpoint{5.112893in}{3.102590in}}%
\pgfusepath{clip}%
\pgfsetbuttcap%
\pgfsetroundjoin%
\definecolor{currentfill}{rgb}{1.000000,0.498039,0.054902}%
\pgfsetfillcolor{currentfill}%
\pgfsetlinewidth{1.003750pt}%
\definecolor{currentstroke}{rgb}{1.000000,0.498039,0.054902}%
\pgfsetstrokecolor{currentstroke}%
\pgfsetdash{}{0pt}%
\pgfpathmoveto{\pgfqpoint{2.031062in}{3.365540in}}%
\pgfpathcurveto{\pgfqpoint{2.042112in}{3.365540in}}{\pgfqpoint{2.052711in}{3.369931in}}{\pgfqpoint{2.060525in}{3.377744in}}%
\pgfpathcurveto{\pgfqpoint{2.068338in}{3.385558in}}{\pgfqpoint{2.072728in}{3.396157in}}{\pgfqpoint{2.072728in}{3.407207in}}%
\pgfpathcurveto{\pgfqpoint{2.072728in}{3.418257in}}{\pgfqpoint{2.068338in}{3.428856in}}{\pgfqpoint{2.060525in}{3.436670in}}%
\pgfpathcurveto{\pgfqpoint{2.052711in}{3.444483in}}{\pgfqpoint{2.042112in}{3.448874in}}{\pgfqpoint{2.031062in}{3.448874in}}%
\pgfpathcurveto{\pgfqpoint{2.020012in}{3.448874in}}{\pgfqpoint{2.009413in}{3.444483in}}{\pgfqpoint{2.001599in}{3.436670in}}%
\pgfpathcurveto{\pgfqpoint{1.993785in}{3.428856in}}{\pgfqpoint{1.989395in}{3.418257in}}{\pgfqpoint{1.989395in}{3.407207in}}%
\pgfpathcurveto{\pgfqpoint{1.989395in}{3.396157in}}{\pgfqpoint{1.993785in}{3.385558in}}{\pgfqpoint{2.001599in}{3.377744in}}%
\pgfpathcurveto{\pgfqpoint{2.009413in}{3.369931in}}{\pgfqpoint{2.020012in}{3.365540in}}{\pgfqpoint{2.031062in}{3.365540in}}%
\pgfpathclose%
\pgfusepath{stroke,fill}%
\end{pgfscope}%
\begin{pgfscope}%
\pgfpathrectangle{\pgfqpoint{0.648703in}{0.548769in}}{\pgfqpoint{5.112893in}{3.102590in}}%
\pgfusepath{clip}%
\pgfsetbuttcap%
\pgfsetroundjoin%
\definecolor{currentfill}{rgb}{1.000000,0.498039,0.054902}%
\pgfsetfillcolor{currentfill}%
\pgfsetlinewidth{1.003750pt}%
\definecolor{currentstroke}{rgb}{1.000000,0.498039,0.054902}%
\pgfsetstrokecolor{currentstroke}%
\pgfsetdash{}{0pt}%
\pgfpathmoveto{\pgfqpoint{2.467167in}{3.262415in}}%
\pgfpathcurveto{\pgfqpoint{2.478217in}{3.262415in}}{\pgfqpoint{2.488816in}{3.266806in}}{\pgfqpoint{2.496630in}{3.274619in}}%
\pgfpathcurveto{\pgfqpoint{2.504444in}{3.282433in}}{\pgfqpoint{2.508834in}{3.293032in}}{\pgfqpoint{2.508834in}{3.304082in}}%
\pgfpathcurveto{\pgfqpoint{2.508834in}{3.315132in}}{\pgfqpoint{2.504444in}{3.325731in}}{\pgfqpoint{2.496630in}{3.333545in}}%
\pgfpathcurveto{\pgfqpoint{2.488816in}{3.341358in}}{\pgfqpoint{2.478217in}{3.345749in}}{\pgfqpoint{2.467167in}{3.345749in}}%
\pgfpathcurveto{\pgfqpoint{2.456117in}{3.345749in}}{\pgfqpoint{2.445518in}{3.341358in}}{\pgfqpoint{2.437704in}{3.333545in}}%
\pgfpathcurveto{\pgfqpoint{2.429891in}{3.325731in}}{\pgfqpoint{2.425501in}{3.315132in}}{\pgfqpoint{2.425501in}{3.304082in}}%
\pgfpathcurveto{\pgfqpoint{2.425501in}{3.293032in}}{\pgfqpoint{2.429891in}{3.282433in}}{\pgfqpoint{2.437704in}{3.274619in}}%
\pgfpathcurveto{\pgfqpoint{2.445518in}{3.266806in}}{\pgfqpoint{2.456117in}{3.262415in}}{\pgfqpoint{2.467167in}{3.262415in}}%
\pgfpathclose%
\pgfusepath{stroke,fill}%
\end{pgfscope}%
\begin{pgfscope}%
\pgfpathrectangle{\pgfqpoint{0.648703in}{0.548769in}}{\pgfqpoint{5.112893in}{3.102590in}}%
\pgfusepath{clip}%
\pgfsetbuttcap%
\pgfsetroundjoin%
\definecolor{currentfill}{rgb}{1.000000,0.498039,0.054902}%
\pgfsetfillcolor{currentfill}%
\pgfsetlinewidth{1.003750pt}%
\definecolor{currentstroke}{rgb}{1.000000,0.498039,0.054902}%
\pgfsetstrokecolor{currentstroke}%
\pgfsetdash{}{0pt}%
\pgfpathmoveto{\pgfqpoint{1.222331in}{3.196415in}}%
\pgfpathcurveto{\pgfqpoint{1.233381in}{3.196415in}}{\pgfqpoint{1.243981in}{3.200806in}}{\pgfqpoint{1.251794in}{3.208619in}}%
\pgfpathcurveto{\pgfqpoint{1.259608in}{3.216433in}}{\pgfqpoint{1.263998in}{3.227032in}}{\pgfqpoint{1.263998in}{3.238082in}}%
\pgfpathcurveto{\pgfqpoint{1.263998in}{3.249132in}}{\pgfqpoint{1.259608in}{3.259731in}}{\pgfqpoint{1.251794in}{3.267545in}}%
\pgfpathcurveto{\pgfqpoint{1.243981in}{3.275358in}}{\pgfqpoint{1.233381in}{3.279749in}}{\pgfqpoint{1.222331in}{3.279749in}}%
\pgfpathcurveto{\pgfqpoint{1.211281in}{3.279749in}}{\pgfqpoint{1.200682in}{3.275358in}}{\pgfqpoint{1.192869in}{3.267545in}}%
\pgfpathcurveto{\pgfqpoint{1.185055in}{3.259731in}}{\pgfqpoint{1.180665in}{3.249132in}}{\pgfqpoint{1.180665in}{3.238082in}}%
\pgfpathcurveto{\pgfqpoint{1.180665in}{3.227032in}}{\pgfqpoint{1.185055in}{3.216433in}}{\pgfqpoint{1.192869in}{3.208619in}}%
\pgfpathcurveto{\pgfqpoint{1.200682in}{3.200806in}}{\pgfqpoint{1.211281in}{3.196415in}}{\pgfqpoint{1.222331in}{3.196415in}}%
\pgfpathclose%
\pgfusepath{stroke,fill}%
\end{pgfscope}%
\begin{pgfscope}%
\pgfpathrectangle{\pgfqpoint{0.648703in}{0.548769in}}{\pgfqpoint{5.112893in}{3.102590in}}%
\pgfusepath{clip}%
\pgfsetbuttcap%
\pgfsetroundjoin%
\definecolor{currentfill}{rgb}{1.000000,0.498039,0.054902}%
\pgfsetfillcolor{currentfill}%
\pgfsetlinewidth{1.003750pt}%
\definecolor{currentstroke}{rgb}{1.000000,0.498039,0.054902}%
\pgfsetstrokecolor{currentstroke}%
\pgfsetdash{}{0pt}%
\pgfpathmoveto{\pgfqpoint{1.903322in}{3.196415in}}%
\pgfpathcurveto{\pgfqpoint{1.914372in}{3.196415in}}{\pgfqpoint{1.924971in}{3.200806in}}{\pgfqpoint{1.932785in}{3.208619in}}%
\pgfpathcurveto{\pgfqpoint{1.940599in}{3.216433in}}{\pgfqpoint{1.944989in}{3.227032in}}{\pgfqpoint{1.944989in}{3.238082in}}%
\pgfpathcurveto{\pgfqpoint{1.944989in}{3.249132in}}{\pgfqpoint{1.940599in}{3.259731in}}{\pgfqpoint{1.932785in}{3.267545in}}%
\pgfpathcurveto{\pgfqpoint{1.924971in}{3.275358in}}{\pgfqpoint{1.914372in}{3.279749in}}{\pgfqpoint{1.903322in}{3.279749in}}%
\pgfpathcurveto{\pgfqpoint{1.892272in}{3.279749in}}{\pgfqpoint{1.881673in}{3.275358in}}{\pgfqpoint{1.873859in}{3.267545in}}%
\pgfpathcurveto{\pgfqpoint{1.866046in}{3.259731in}}{\pgfqpoint{1.861655in}{3.249132in}}{\pgfqpoint{1.861655in}{3.238082in}}%
\pgfpathcurveto{\pgfqpoint{1.861655in}{3.227032in}}{\pgfqpoint{1.866046in}{3.216433in}}{\pgfqpoint{1.873859in}{3.208619in}}%
\pgfpathcurveto{\pgfqpoint{1.881673in}{3.200806in}}{\pgfqpoint{1.892272in}{3.196415in}}{\pgfqpoint{1.903322in}{3.196415in}}%
\pgfpathclose%
\pgfusepath{stroke,fill}%
\end{pgfscope}%
\begin{pgfscope}%
\pgfpathrectangle{\pgfqpoint{0.648703in}{0.548769in}}{\pgfqpoint{5.112893in}{3.102590in}}%
\pgfusepath{clip}%
\pgfsetbuttcap%
\pgfsetroundjoin%
\definecolor{currentfill}{rgb}{0.121569,0.466667,0.705882}%
\pgfsetfillcolor{currentfill}%
\pgfsetlinewidth{1.003750pt}%
\definecolor{currentstroke}{rgb}{0.121569,0.466667,0.705882}%
\pgfsetstrokecolor{currentstroke}%
\pgfsetdash{}{0pt}%
\pgfpathmoveto{\pgfqpoint{1.670740in}{3.188165in}}%
\pgfpathcurveto{\pgfqpoint{1.681791in}{3.188165in}}{\pgfqpoint{1.692390in}{3.192556in}}{\pgfqpoint{1.700203in}{3.200369in}}%
\pgfpathcurveto{\pgfqpoint{1.708017in}{3.208183in}}{\pgfqpoint{1.712407in}{3.218782in}}{\pgfqpoint{1.712407in}{3.229832in}}%
\pgfpathcurveto{\pgfqpoint{1.712407in}{3.240882in}}{\pgfqpoint{1.708017in}{3.251481in}}{\pgfqpoint{1.700203in}{3.259295in}}%
\pgfpathcurveto{\pgfqpoint{1.692390in}{3.267109in}}{\pgfqpoint{1.681791in}{3.271499in}}{\pgfqpoint{1.670740in}{3.271499in}}%
\pgfpathcurveto{\pgfqpoint{1.659690in}{3.271499in}}{\pgfqpoint{1.649091in}{3.267109in}}{\pgfqpoint{1.641278in}{3.259295in}}%
\pgfpathcurveto{\pgfqpoint{1.633464in}{3.251481in}}{\pgfqpoint{1.629074in}{3.240882in}}{\pgfqpoint{1.629074in}{3.229832in}}%
\pgfpathcurveto{\pgfqpoint{1.629074in}{3.218782in}}{\pgfqpoint{1.633464in}{3.208183in}}{\pgfqpoint{1.641278in}{3.200369in}}%
\pgfpathcurveto{\pgfqpoint{1.649091in}{3.192556in}}{\pgfqpoint{1.659690in}{3.188165in}}{\pgfqpoint{1.670740in}{3.188165in}}%
\pgfpathclose%
\pgfusepath{stroke,fill}%
\end{pgfscope}%
\begin{pgfscope}%
\pgfpathrectangle{\pgfqpoint{0.648703in}{0.548769in}}{\pgfqpoint{5.112893in}{3.102590in}}%
\pgfusepath{clip}%
\pgfsetbuttcap%
\pgfsetroundjoin%
\definecolor{currentfill}{rgb}{1.000000,0.498039,0.054902}%
\pgfsetfillcolor{currentfill}%
\pgfsetlinewidth{1.003750pt}%
\definecolor{currentstroke}{rgb}{1.000000,0.498039,0.054902}%
\pgfsetstrokecolor{currentstroke}%
\pgfsetdash{}{0pt}%
\pgfpathmoveto{\pgfqpoint{1.893203in}{3.192290in}}%
\pgfpathcurveto{\pgfqpoint{1.904253in}{3.192290in}}{\pgfqpoint{1.914852in}{3.196681in}}{\pgfqpoint{1.922666in}{3.204494in}}%
\pgfpathcurveto{\pgfqpoint{1.930480in}{3.212308in}}{\pgfqpoint{1.934870in}{3.222907in}}{\pgfqpoint{1.934870in}{3.233957in}}%
\pgfpathcurveto{\pgfqpoint{1.934870in}{3.245007in}}{\pgfqpoint{1.930480in}{3.255606in}}{\pgfqpoint{1.922666in}{3.263420in}}%
\pgfpathcurveto{\pgfqpoint{1.914852in}{3.271234in}}{\pgfqpoint{1.904253in}{3.275624in}}{\pgfqpoint{1.893203in}{3.275624in}}%
\pgfpathcurveto{\pgfqpoint{1.882153in}{3.275624in}}{\pgfqpoint{1.871554in}{3.271234in}}{\pgfqpoint{1.863741in}{3.263420in}}%
\pgfpathcurveto{\pgfqpoint{1.855927in}{3.255606in}}{\pgfqpoint{1.851537in}{3.245007in}}{\pgfqpoint{1.851537in}{3.233957in}}%
\pgfpathcurveto{\pgfqpoint{1.851537in}{3.222907in}}{\pgfqpoint{1.855927in}{3.212308in}}{\pgfqpoint{1.863741in}{3.204494in}}%
\pgfpathcurveto{\pgfqpoint{1.871554in}{3.196681in}}{\pgfqpoint{1.882153in}{3.192290in}}{\pgfqpoint{1.893203in}{3.192290in}}%
\pgfpathclose%
\pgfusepath{stroke,fill}%
\end{pgfscope}%
\begin{pgfscope}%
\pgfpathrectangle{\pgfqpoint{0.648703in}{0.548769in}}{\pgfqpoint{5.112893in}{3.102590in}}%
\pgfusepath{clip}%
\pgfsetbuttcap%
\pgfsetroundjoin%
\definecolor{currentfill}{rgb}{0.121569,0.466667,0.705882}%
\pgfsetfillcolor{currentfill}%
\pgfsetlinewidth{1.003750pt}%
\definecolor{currentstroke}{rgb}{0.121569,0.466667,0.705882}%
\pgfsetstrokecolor{currentstroke}%
\pgfsetdash{}{0pt}%
\pgfpathmoveto{\pgfqpoint{0.779911in}{0.750293in}}%
\pgfpathcurveto{\pgfqpoint{0.790962in}{0.750293in}}{\pgfqpoint{0.801561in}{0.754683in}}{\pgfqpoint{0.809374in}{0.762497in}}%
\pgfpathcurveto{\pgfqpoint{0.817188in}{0.770311in}}{\pgfqpoint{0.821578in}{0.780910in}}{\pgfqpoint{0.821578in}{0.791960in}}%
\pgfpathcurveto{\pgfqpoint{0.821578in}{0.803010in}}{\pgfqpoint{0.817188in}{0.813609in}}{\pgfqpoint{0.809374in}{0.821422in}}%
\pgfpathcurveto{\pgfqpoint{0.801561in}{0.829236in}}{\pgfqpoint{0.790962in}{0.833626in}}{\pgfqpoint{0.779911in}{0.833626in}}%
\pgfpathcurveto{\pgfqpoint{0.768861in}{0.833626in}}{\pgfqpoint{0.758262in}{0.829236in}}{\pgfqpoint{0.750449in}{0.821422in}}%
\pgfpathcurveto{\pgfqpoint{0.742635in}{0.813609in}}{\pgfqpoint{0.738245in}{0.803010in}}{\pgfqpoint{0.738245in}{0.791960in}}%
\pgfpathcurveto{\pgfqpoint{0.738245in}{0.780910in}}{\pgfqpoint{0.742635in}{0.770311in}}{\pgfqpoint{0.750449in}{0.762497in}}%
\pgfpathcurveto{\pgfqpoint{0.758262in}{0.754683in}}{\pgfqpoint{0.768861in}{0.750293in}}{\pgfqpoint{0.779911in}{0.750293in}}%
\pgfpathclose%
\pgfusepath{stroke,fill}%
\end{pgfscope}%
\begin{pgfscope}%
\pgfpathrectangle{\pgfqpoint{0.648703in}{0.548769in}}{\pgfqpoint{5.112893in}{3.102590in}}%
\pgfusepath{clip}%
\pgfsetbuttcap%
\pgfsetroundjoin%
\definecolor{currentfill}{rgb}{0.839216,0.152941,0.156863}%
\pgfsetfillcolor{currentfill}%
\pgfsetlinewidth{1.003750pt}%
\definecolor{currentstroke}{rgb}{0.839216,0.152941,0.156863}%
\pgfsetstrokecolor{currentstroke}%
\pgfsetdash{}{0pt}%
\pgfpathmoveto{\pgfqpoint{2.393762in}{3.221165in}}%
\pgfpathcurveto{\pgfqpoint{2.404812in}{3.221165in}}{\pgfqpoint{2.415411in}{3.225556in}}{\pgfqpoint{2.423225in}{3.233369in}}%
\pgfpathcurveto{\pgfqpoint{2.431039in}{3.241183in}}{\pgfqpoint{2.435429in}{3.251782in}}{\pgfqpoint{2.435429in}{3.262832in}}%
\pgfpathcurveto{\pgfqpoint{2.435429in}{3.273882in}}{\pgfqpoint{2.431039in}{3.284481in}}{\pgfqpoint{2.423225in}{3.292295in}}%
\pgfpathcurveto{\pgfqpoint{2.415411in}{3.300108in}}{\pgfqpoint{2.404812in}{3.304499in}}{\pgfqpoint{2.393762in}{3.304499in}}%
\pgfpathcurveto{\pgfqpoint{2.382712in}{3.304499in}}{\pgfqpoint{2.372113in}{3.300108in}}{\pgfqpoint{2.364300in}{3.292295in}}%
\pgfpathcurveto{\pgfqpoint{2.356486in}{3.284481in}}{\pgfqpoint{2.352096in}{3.273882in}}{\pgfqpoint{2.352096in}{3.262832in}}%
\pgfpathcurveto{\pgfqpoint{2.352096in}{3.251782in}}{\pgfqpoint{2.356486in}{3.241183in}}{\pgfqpoint{2.364300in}{3.233369in}}%
\pgfpathcurveto{\pgfqpoint{2.372113in}{3.225556in}}{\pgfqpoint{2.382712in}{3.221165in}}{\pgfqpoint{2.393762in}{3.221165in}}%
\pgfpathclose%
\pgfusepath{stroke,fill}%
\end{pgfscope}%
\begin{pgfscope}%
\pgfpathrectangle{\pgfqpoint{0.648703in}{0.548769in}}{\pgfqpoint{5.112893in}{3.102590in}}%
\pgfusepath{clip}%
\pgfsetbuttcap%
\pgfsetroundjoin%
\definecolor{currentfill}{rgb}{1.000000,0.498039,0.054902}%
\pgfsetfillcolor{currentfill}%
\pgfsetlinewidth{1.003750pt}%
\definecolor{currentstroke}{rgb}{1.000000,0.498039,0.054902}%
\pgfsetstrokecolor{currentstroke}%
\pgfsetdash{}{0pt}%
\pgfpathmoveto{\pgfqpoint{2.073568in}{3.192290in}}%
\pgfpathcurveto{\pgfqpoint{2.084618in}{3.192290in}}{\pgfqpoint{2.095217in}{3.196681in}}{\pgfqpoint{2.103031in}{3.204494in}}%
\pgfpathcurveto{\pgfqpoint{2.110844in}{3.212308in}}{\pgfqpoint{2.115235in}{3.222907in}}{\pgfqpoint{2.115235in}{3.233957in}}%
\pgfpathcurveto{\pgfqpoint{2.115235in}{3.245007in}}{\pgfqpoint{2.110844in}{3.255606in}}{\pgfqpoint{2.103031in}{3.263420in}}%
\pgfpathcurveto{\pgfqpoint{2.095217in}{3.271234in}}{\pgfqpoint{2.084618in}{3.275624in}}{\pgfqpoint{2.073568in}{3.275624in}}%
\pgfpathcurveto{\pgfqpoint{2.062518in}{3.275624in}}{\pgfqpoint{2.051919in}{3.271234in}}{\pgfqpoint{2.044105in}{3.263420in}}%
\pgfpathcurveto{\pgfqpoint{2.036291in}{3.255606in}}{\pgfqpoint{2.031901in}{3.245007in}}{\pgfqpoint{2.031901in}{3.233957in}}%
\pgfpathcurveto{\pgfqpoint{2.031901in}{3.222907in}}{\pgfqpoint{2.036291in}{3.212308in}}{\pgfqpoint{2.044105in}{3.204494in}}%
\pgfpathcurveto{\pgfqpoint{2.051919in}{3.196681in}}{\pgfqpoint{2.062518in}{3.192290in}}{\pgfqpoint{2.073568in}{3.192290in}}%
\pgfpathclose%
\pgfusepath{stroke,fill}%
\end{pgfscope}%
\begin{pgfscope}%
\pgfpathrectangle{\pgfqpoint{0.648703in}{0.548769in}}{\pgfqpoint{5.112893in}{3.102590in}}%
\pgfusepath{clip}%
\pgfsetbuttcap%
\pgfsetroundjoin%
\definecolor{currentfill}{rgb}{0.121569,0.466667,0.705882}%
\pgfsetfillcolor{currentfill}%
\pgfsetlinewidth{1.003750pt}%
\definecolor{currentstroke}{rgb}{0.121569,0.466667,0.705882}%
\pgfsetstrokecolor{currentstroke}%
\pgfsetdash{}{0pt}%
\pgfpathmoveto{\pgfqpoint{1.058071in}{1.195793in}}%
\pgfpathcurveto{\pgfqpoint{1.069121in}{1.195793in}}{\pgfqpoint{1.079720in}{1.200183in}}{\pgfqpoint{1.087533in}{1.207996in}}%
\pgfpathcurveto{\pgfqpoint{1.095347in}{1.215810in}}{\pgfqpoint{1.099737in}{1.226409in}}{\pgfqpoint{1.099737in}{1.237459in}}%
\pgfpathcurveto{\pgfqpoint{1.099737in}{1.248509in}}{\pgfqpoint{1.095347in}{1.259108in}}{\pgfqpoint{1.087533in}{1.266922in}}%
\pgfpathcurveto{\pgfqpoint{1.079720in}{1.274736in}}{\pgfqpoint{1.069121in}{1.279126in}}{\pgfqpoint{1.058071in}{1.279126in}}%
\pgfpathcurveto{\pgfqpoint{1.047020in}{1.279126in}}{\pgfqpoint{1.036421in}{1.274736in}}{\pgfqpoint{1.028608in}{1.266922in}}%
\pgfpathcurveto{\pgfqpoint{1.020794in}{1.259108in}}{\pgfqpoint{1.016404in}{1.248509in}}{\pgfqpoint{1.016404in}{1.237459in}}%
\pgfpathcurveto{\pgfqpoint{1.016404in}{1.226409in}}{\pgfqpoint{1.020794in}{1.215810in}}{\pgfqpoint{1.028608in}{1.207996in}}%
\pgfpathcurveto{\pgfqpoint{1.036421in}{1.200183in}}{\pgfqpoint{1.047020in}{1.195793in}}{\pgfqpoint{1.058071in}{1.195793in}}%
\pgfpathclose%
\pgfusepath{stroke,fill}%
\end{pgfscope}%
\begin{pgfscope}%
\pgfpathrectangle{\pgfqpoint{0.648703in}{0.548769in}}{\pgfqpoint{5.112893in}{3.102590in}}%
\pgfusepath{clip}%
\pgfsetbuttcap%
\pgfsetroundjoin%
\definecolor{currentfill}{rgb}{0.121569,0.466667,0.705882}%
\pgfsetfillcolor{currentfill}%
\pgfsetlinewidth{1.003750pt}%
\definecolor{currentstroke}{rgb}{0.121569,0.466667,0.705882}%
\pgfsetstrokecolor{currentstroke}%
\pgfsetdash{}{0pt}%
\pgfpathmoveto{\pgfqpoint{1.346253in}{1.146293in}}%
\pgfpathcurveto{\pgfqpoint{1.357303in}{1.146293in}}{\pgfqpoint{1.367902in}{1.150683in}}{\pgfqpoint{1.375716in}{1.158496in}}%
\pgfpathcurveto{\pgfqpoint{1.383529in}{1.166310in}}{\pgfqpoint{1.387920in}{1.176909in}}{\pgfqpoint{1.387920in}{1.187959in}}%
\pgfpathcurveto{\pgfqpoint{1.387920in}{1.199009in}}{\pgfqpoint{1.383529in}{1.209608in}}{\pgfqpoint{1.375716in}{1.217422in}}%
\pgfpathcurveto{\pgfqpoint{1.367902in}{1.225236in}}{\pgfqpoint{1.357303in}{1.229626in}}{\pgfqpoint{1.346253in}{1.229626in}}%
\pgfpathcurveto{\pgfqpoint{1.335203in}{1.229626in}}{\pgfqpoint{1.324604in}{1.225236in}}{\pgfqpoint{1.316790in}{1.217422in}}%
\pgfpathcurveto{\pgfqpoint{1.308977in}{1.209608in}}{\pgfqpoint{1.304586in}{1.199009in}}{\pgfqpoint{1.304586in}{1.187959in}}%
\pgfpathcurveto{\pgfqpoint{1.304586in}{1.176909in}}{\pgfqpoint{1.308977in}{1.166310in}}{\pgfqpoint{1.316790in}{1.158496in}}%
\pgfpathcurveto{\pgfqpoint{1.324604in}{1.150683in}}{\pgfqpoint{1.335203in}{1.146293in}}{\pgfqpoint{1.346253in}{1.146293in}}%
\pgfpathclose%
\pgfusepath{stroke,fill}%
\end{pgfscope}%
\begin{pgfscope}%
\pgfpathrectangle{\pgfqpoint{0.648703in}{0.548769in}}{\pgfqpoint{5.112893in}{3.102590in}}%
\pgfusepath{clip}%
\pgfsetbuttcap%
\pgfsetroundjoin%
\definecolor{currentfill}{rgb}{1.000000,0.498039,0.054902}%
\pgfsetfillcolor{currentfill}%
\pgfsetlinewidth{1.003750pt}%
\definecolor{currentstroke}{rgb}{1.000000,0.498039,0.054902}%
\pgfsetstrokecolor{currentstroke}%
\pgfsetdash{}{0pt}%
\pgfpathmoveto{\pgfqpoint{1.626967in}{3.196415in}}%
\pgfpathcurveto{\pgfqpoint{1.638017in}{3.196415in}}{\pgfqpoint{1.648616in}{3.200806in}}{\pgfqpoint{1.656430in}{3.208619in}}%
\pgfpathcurveto{\pgfqpoint{1.664244in}{3.216433in}}{\pgfqpoint{1.668634in}{3.227032in}}{\pgfqpoint{1.668634in}{3.238082in}}%
\pgfpathcurveto{\pgfqpoint{1.668634in}{3.249132in}}{\pgfqpoint{1.664244in}{3.259731in}}{\pgfqpoint{1.656430in}{3.267545in}}%
\pgfpathcurveto{\pgfqpoint{1.648616in}{3.275358in}}{\pgfqpoint{1.638017in}{3.279749in}}{\pgfqpoint{1.626967in}{3.279749in}}%
\pgfpathcurveto{\pgfqpoint{1.615917in}{3.279749in}}{\pgfqpoint{1.605318in}{3.275358in}}{\pgfqpoint{1.597504in}{3.267545in}}%
\pgfpathcurveto{\pgfqpoint{1.589691in}{3.259731in}}{\pgfqpoint{1.585301in}{3.249132in}}{\pgfqpoint{1.585301in}{3.238082in}}%
\pgfpathcurveto{\pgfqpoint{1.585301in}{3.227032in}}{\pgfqpoint{1.589691in}{3.216433in}}{\pgfqpoint{1.597504in}{3.208619in}}%
\pgfpathcurveto{\pgfqpoint{1.605318in}{3.200806in}}{\pgfqpoint{1.615917in}{3.196415in}}{\pgfqpoint{1.626967in}{3.196415in}}%
\pgfpathclose%
\pgfusepath{stroke,fill}%
\end{pgfscope}%
\begin{pgfscope}%
\pgfpathrectangle{\pgfqpoint{0.648703in}{0.548769in}}{\pgfqpoint{5.112893in}{3.102590in}}%
\pgfusepath{clip}%
\pgfsetbuttcap%
\pgfsetroundjoin%
\definecolor{currentfill}{rgb}{0.121569,0.466667,0.705882}%
\pgfsetfillcolor{currentfill}%
\pgfsetlinewidth{1.003750pt}%
\definecolor{currentstroke}{rgb}{0.121569,0.466667,0.705882}%
\pgfsetstrokecolor{currentstroke}%
\pgfsetdash{}{0pt}%
\pgfpathmoveto{\pgfqpoint{1.173188in}{1.084418in}}%
\pgfpathcurveto{\pgfqpoint{1.184238in}{1.084418in}}{\pgfqpoint{1.194837in}{1.088808in}}{\pgfqpoint{1.202651in}{1.096622in}}%
\pgfpathcurveto{\pgfqpoint{1.210464in}{1.104435in}}{\pgfqpoint{1.214855in}{1.115034in}}{\pgfqpoint{1.214855in}{1.126084in}}%
\pgfpathcurveto{\pgfqpoint{1.214855in}{1.137134in}}{\pgfqpoint{1.210464in}{1.147734in}}{\pgfqpoint{1.202651in}{1.155547in}}%
\pgfpathcurveto{\pgfqpoint{1.194837in}{1.163361in}}{\pgfqpoint{1.184238in}{1.167751in}}{\pgfqpoint{1.173188in}{1.167751in}}%
\pgfpathcurveto{\pgfqpoint{1.162138in}{1.167751in}}{\pgfqpoint{1.151539in}{1.163361in}}{\pgfqpoint{1.143725in}{1.155547in}}%
\pgfpathcurveto{\pgfqpoint{1.135911in}{1.147734in}}{\pgfqpoint{1.131521in}{1.137134in}}{\pgfqpoint{1.131521in}{1.126084in}}%
\pgfpathcurveto{\pgfqpoint{1.131521in}{1.115034in}}{\pgfqpoint{1.135911in}{1.104435in}}{\pgfqpoint{1.143725in}{1.096622in}}%
\pgfpathcurveto{\pgfqpoint{1.151539in}{1.088808in}}{\pgfqpoint{1.162138in}{1.084418in}}{\pgfqpoint{1.173188in}{1.084418in}}%
\pgfpathclose%
\pgfusepath{stroke,fill}%
\end{pgfscope}%
\begin{pgfscope}%
\pgfpathrectangle{\pgfqpoint{0.648703in}{0.548769in}}{\pgfqpoint{5.112893in}{3.102590in}}%
\pgfusepath{clip}%
\pgfsetbuttcap%
\pgfsetroundjoin%
\definecolor{currentfill}{rgb}{0.121569,0.466667,0.705882}%
\pgfsetfillcolor{currentfill}%
\pgfsetlinewidth{1.003750pt}%
\definecolor{currentstroke}{rgb}{0.121569,0.466667,0.705882}%
\pgfsetstrokecolor{currentstroke}%
\pgfsetdash{}{0pt}%
\pgfpathmoveto{\pgfqpoint{0.827216in}{0.878168in}}%
\pgfpathcurveto{\pgfqpoint{0.838267in}{0.878168in}}{\pgfqpoint{0.848866in}{0.882558in}}{\pgfqpoint{0.856679in}{0.890372in}}%
\pgfpathcurveto{\pgfqpoint{0.864493in}{0.898185in}}{\pgfqpoint{0.868883in}{0.908784in}}{\pgfqpoint{0.868883in}{0.919835in}}%
\pgfpathcurveto{\pgfqpoint{0.868883in}{0.930885in}}{\pgfqpoint{0.864493in}{0.941484in}}{\pgfqpoint{0.856679in}{0.949297in}}%
\pgfpathcurveto{\pgfqpoint{0.848866in}{0.957111in}}{\pgfqpoint{0.838267in}{0.961501in}}{\pgfqpoint{0.827216in}{0.961501in}}%
\pgfpathcurveto{\pgfqpoint{0.816166in}{0.961501in}}{\pgfqpoint{0.805567in}{0.957111in}}{\pgfqpoint{0.797754in}{0.949297in}}%
\pgfpathcurveto{\pgfqpoint{0.789940in}{0.941484in}}{\pgfqpoint{0.785550in}{0.930885in}}{\pgfqpoint{0.785550in}{0.919835in}}%
\pgfpathcurveto{\pgfqpoint{0.785550in}{0.908784in}}{\pgfqpoint{0.789940in}{0.898185in}}{\pgfqpoint{0.797754in}{0.890372in}}%
\pgfpathcurveto{\pgfqpoint{0.805567in}{0.882558in}}{\pgfqpoint{0.816166in}{0.878168in}}{\pgfqpoint{0.827216in}{0.878168in}}%
\pgfpathclose%
\pgfusepath{stroke,fill}%
\end{pgfscope}%
\begin{pgfscope}%
\pgfpathrectangle{\pgfqpoint{0.648703in}{0.548769in}}{\pgfqpoint{5.112893in}{3.102590in}}%
\pgfusepath{clip}%
\pgfsetbuttcap%
\pgfsetroundjoin%
\definecolor{currentfill}{rgb}{0.839216,0.152941,0.156863}%
\pgfsetfillcolor{currentfill}%
\pgfsetlinewidth{1.003750pt}%
\definecolor{currentstroke}{rgb}{0.839216,0.152941,0.156863}%
\pgfsetstrokecolor{currentstroke}%
\pgfsetdash{}{0pt}%
\pgfpathmoveto{\pgfqpoint{1.571940in}{3.208790in}}%
\pgfpathcurveto{\pgfqpoint{1.582990in}{3.208790in}}{\pgfqpoint{1.593589in}{3.213181in}}{\pgfqpoint{1.601402in}{3.220994in}}%
\pgfpathcurveto{\pgfqpoint{1.609216in}{3.228808in}}{\pgfqpoint{1.613606in}{3.239407in}}{\pgfqpoint{1.613606in}{3.250457in}}%
\pgfpathcurveto{\pgfqpoint{1.613606in}{3.261507in}}{\pgfqpoint{1.609216in}{3.272106in}}{\pgfqpoint{1.601402in}{3.279920in}}%
\pgfpathcurveto{\pgfqpoint{1.593589in}{3.287733in}}{\pgfqpoint{1.582990in}{3.292124in}}{\pgfqpoint{1.571940in}{3.292124in}}%
\pgfpathcurveto{\pgfqpoint{1.560890in}{3.292124in}}{\pgfqpoint{1.550290in}{3.287733in}}{\pgfqpoint{1.542477in}{3.279920in}}%
\pgfpathcurveto{\pgfqpoint{1.534663in}{3.272106in}}{\pgfqpoint{1.530273in}{3.261507in}}{\pgfqpoint{1.530273in}{3.250457in}}%
\pgfpathcurveto{\pgfqpoint{1.530273in}{3.239407in}}{\pgfqpoint{1.534663in}{3.228808in}}{\pgfqpoint{1.542477in}{3.220994in}}%
\pgfpathcurveto{\pgfqpoint{1.550290in}{3.213181in}}{\pgfqpoint{1.560890in}{3.208790in}}{\pgfqpoint{1.571940in}{3.208790in}}%
\pgfpathclose%
\pgfusepath{stroke,fill}%
\end{pgfscope}%
\begin{pgfscope}%
\pgfsetbuttcap%
\pgfsetroundjoin%
\definecolor{currentfill}{rgb}{0.000000,0.000000,0.000000}%
\pgfsetfillcolor{currentfill}%
\pgfsetlinewidth{0.803000pt}%
\definecolor{currentstroke}{rgb}{0.000000,0.000000,0.000000}%
\pgfsetstrokecolor{currentstroke}%
\pgfsetdash{}{0pt}%
\pgfsys@defobject{currentmarker}{\pgfqpoint{0.000000in}{-0.048611in}}{\pgfqpoint{0.000000in}{0.000000in}}{%
\pgfpathmoveto{\pgfqpoint{0.000000in}{0.000000in}}%
\pgfpathlineto{\pgfqpoint{0.000000in}{-0.048611in}}%
\pgfusepath{stroke,fill}%
}%
\begin{pgfscope}%
\pgfsys@transformshift{0.767673in}{0.548769in}%
\pgfsys@useobject{currentmarker}{}%
\end{pgfscope}%
\end{pgfscope}%
\begin{pgfscope}%
\definecolor{textcolor}{rgb}{0.000000,0.000000,0.000000}%
\pgfsetstrokecolor{textcolor}%
\pgfsetfillcolor{textcolor}%
\pgftext[x=0.767673in,y=0.451547in,,top]{\color{textcolor}\sffamily\fontsize{10.000000}{12.000000}\selectfont \(\displaystyle {0.0}\)}%
\end{pgfscope}%
\begin{pgfscope}%
\pgfsetbuttcap%
\pgfsetroundjoin%
\definecolor{currentfill}{rgb}{0.000000,0.000000,0.000000}%
\pgfsetfillcolor{currentfill}%
\pgfsetlinewidth{0.803000pt}%
\definecolor{currentstroke}{rgb}{0.000000,0.000000,0.000000}%
\pgfsetstrokecolor{currentstroke}%
\pgfsetdash{}{0pt}%
\pgfsys@defobject{currentmarker}{\pgfqpoint{0.000000in}{-0.048611in}}{\pgfqpoint{0.000000in}{0.000000in}}{%
\pgfpathmoveto{\pgfqpoint{0.000000in}{0.000000in}}%
\pgfpathlineto{\pgfqpoint{0.000000in}{-0.048611in}}%
\pgfusepath{stroke,fill}%
}%
\begin{pgfscope}%
\pgfsys@transformshift{1.267065in}{0.548769in}%
\pgfsys@useobject{currentmarker}{}%
\end{pgfscope}%
\end{pgfscope}%
\begin{pgfscope}%
\definecolor{textcolor}{rgb}{0.000000,0.000000,0.000000}%
\pgfsetstrokecolor{textcolor}%
\pgfsetfillcolor{textcolor}%
\pgftext[x=1.267065in,y=0.451547in,,top]{\color{textcolor}\sffamily\fontsize{10.000000}{12.000000}\selectfont \(\displaystyle {0.1}\)}%
\end{pgfscope}%
\begin{pgfscope}%
\pgfsetbuttcap%
\pgfsetroundjoin%
\definecolor{currentfill}{rgb}{0.000000,0.000000,0.000000}%
\pgfsetfillcolor{currentfill}%
\pgfsetlinewidth{0.803000pt}%
\definecolor{currentstroke}{rgb}{0.000000,0.000000,0.000000}%
\pgfsetstrokecolor{currentstroke}%
\pgfsetdash{}{0pt}%
\pgfsys@defobject{currentmarker}{\pgfqpoint{0.000000in}{-0.048611in}}{\pgfqpoint{0.000000in}{0.000000in}}{%
\pgfpathmoveto{\pgfqpoint{0.000000in}{0.000000in}}%
\pgfpathlineto{\pgfqpoint{0.000000in}{-0.048611in}}%
\pgfusepath{stroke,fill}%
}%
\begin{pgfscope}%
\pgfsys@transformshift{1.766458in}{0.548769in}%
\pgfsys@useobject{currentmarker}{}%
\end{pgfscope}%
\end{pgfscope}%
\begin{pgfscope}%
\definecolor{textcolor}{rgb}{0.000000,0.000000,0.000000}%
\pgfsetstrokecolor{textcolor}%
\pgfsetfillcolor{textcolor}%
\pgftext[x=1.766458in,y=0.451547in,,top]{\color{textcolor}\sffamily\fontsize{10.000000}{12.000000}\selectfont \(\displaystyle {0.2}\)}%
\end{pgfscope}%
\begin{pgfscope}%
\pgfsetbuttcap%
\pgfsetroundjoin%
\definecolor{currentfill}{rgb}{0.000000,0.000000,0.000000}%
\pgfsetfillcolor{currentfill}%
\pgfsetlinewidth{0.803000pt}%
\definecolor{currentstroke}{rgb}{0.000000,0.000000,0.000000}%
\pgfsetstrokecolor{currentstroke}%
\pgfsetdash{}{0pt}%
\pgfsys@defobject{currentmarker}{\pgfqpoint{0.000000in}{-0.048611in}}{\pgfqpoint{0.000000in}{0.000000in}}{%
\pgfpathmoveto{\pgfqpoint{0.000000in}{0.000000in}}%
\pgfpathlineto{\pgfqpoint{0.000000in}{-0.048611in}}%
\pgfusepath{stroke,fill}%
}%
\begin{pgfscope}%
\pgfsys@transformshift{2.265850in}{0.548769in}%
\pgfsys@useobject{currentmarker}{}%
\end{pgfscope}%
\end{pgfscope}%
\begin{pgfscope}%
\definecolor{textcolor}{rgb}{0.000000,0.000000,0.000000}%
\pgfsetstrokecolor{textcolor}%
\pgfsetfillcolor{textcolor}%
\pgftext[x=2.265850in,y=0.451547in,,top]{\color{textcolor}\sffamily\fontsize{10.000000}{12.000000}\selectfont \(\displaystyle {0.3}\)}%
\end{pgfscope}%
\begin{pgfscope}%
\pgfsetbuttcap%
\pgfsetroundjoin%
\definecolor{currentfill}{rgb}{0.000000,0.000000,0.000000}%
\pgfsetfillcolor{currentfill}%
\pgfsetlinewidth{0.803000pt}%
\definecolor{currentstroke}{rgb}{0.000000,0.000000,0.000000}%
\pgfsetstrokecolor{currentstroke}%
\pgfsetdash{}{0pt}%
\pgfsys@defobject{currentmarker}{\pgfqpoint{0.000000in}{-0.048611in}}{\pgfqpoint{0.000000in}{0.000000in}}{%
\pgfpathmoveto{\pgfqpoint{0.000000in}{0.000000in}}%
\pgfpathlineto{\pgfqpoint{0.000000in}{-0.048611in}}%
\pgfusepath{stroke,fill}%
}%
\begin{pgfscope}%
\pgfsys@transformshift{2.765242in}{0.548769in}%
\pgfsys@useobject{currentmarker}{}%
\end{pgfscope}%
\end{pgfscope}%
\begin{pgfscope}%
\definecolor{textcolor}{rgb}{0.000000,0.000000,0.000000}%
\pgfsetstrokecolor{textcolor}%
\pgfsetfillcolor{textcolor}%
\pgftext[x=2.765242in,y=0.451547in,,top]{\color{textcolor}\sffamily\fontsize{10.000000}{12.000000}\selectfont \(\displaystyle {0.4}\)}%
\end{pgfscope}%
\begin{pgfscope}%
\pgfsetbuttcap%
\pgfsetroundjoin%
\definecolor{currentfill}{rgb}{0.000000,0.000000,0.000000}%
\pgfsetfillcolor{currentfill}%
\pgfsetlinewidth{0.803000pt}%
\definecolor{currentstroke}{rgb}{0.000000,0.000000,0.000000}%
\pgfsetstrokecolor{currentstroke}%
\pgfsetdash{}{0pt}%
\pgfsys@defobject{currentmarker}{\pgfqpoint{0.000000in}{-0.048611in}}{\pgfqpoint{0.000000in}{0.000000in}}{%
\pgfpathmoveto{\pgfqpoint{0.000000in}{0.000000in}}%
\pgfpathlineto{\pgfqpoint{0.000000in}{-0.048611in}}%
\pgfusepath{stroke,fill}%
}%
\begin{pgfscope}%
\pgfsys@transformshift{3.264635in}{0.548769in}%
\pgfsys@useobject{currentmarker}{}%
\end{pgfscope}%
\end{pgfscope}%
\begin{pgfscope}%
\definecolor{textcolor}{rgb}{0.000000,0.000000,0.000000}%
\pgfsetstrokecolor{textcolor}%
\pgfsetfillcolor{textcolor}%
\pgftext[x=3.264635in,y=0.451547in,,top]{\color{textcolor}\sffamily\fontsize{10.000000}{12.000000}\selectfont \(\displaystyle {0.5}\)}%
\end{pgfscope}%
\begin{pgfscope}%
\pgfsetbuttcap%
\pgfsetroundjoin%
\definecolor{currentfill}{rgb}{0.000000,0.000000,0.000000}%
\pgfsetfillcolor{currentfill}%
\pgfsetlinewidth{0.803000pt}%
\definecolor{currentstroke}{rgb}{0.000000,0.000000,0.000000}%
\pgfsetstrokecolor{currentstroke}%
\pgfsetdash{}{0pt}%
\pgfsys@defobject{currentmarker}{\pgfqpoint{0.000000in}{-0.048611in}}{\pgfqpoint{0.000000in}{0.000000in}}{%
\pgfpathmoveto{\pgfqpoint{0.000000in}{0.000000in}}%
\pgfpathlineto{\pgfqpoint{0.000000in}{-0.048611in}}%
\pgfusepath{stroke,fill}%
}%
\begin{pgfscope}%
\pgfsys@transformshift{3.764027in}{0.548769in}%
\pgfsys@useobject{currentmarker}{}%
\end{pgfscope}%
\end{pgfscope}%
\begin{pgfscope}%
\definecolor{textcolor}{rgb}{0.000000,0.000000,0.000000}%
\pgfsetstrokecolor{textcolor}%
\pgfsetfillcolor{textcolor}%
\pgftext[x=3.764027in,y=0.451547in,,top]{\color{textcolor}\sffamily\fontsize{10.000000}{12.000000}\selectfont \(\displaystyle {0.6}\)}%
\end{pgfscope}%
\begin{pgfscope}%
\pgfsetbuttcap%
\pgfsetroundjoin%
\definecolor{currentfill}{rgb}{0.000000,0.000000,0.000000}%
\pgfsetfillcolor{currentfill}%
\pgfsetlinewidth{0.803000pt}%
\definecolor{currentstroke}{rgb}{0.000000,0.000000,0.000000}%
\pgfsetstrokecolor{currentstroke}%
\pgfsetdash{}{0pt}%
\pgfsys@defobject{currentmarker}{\pgfqpoint{0.000000in}{-0.048611in}}{\pgfqpoint{0.000000in}{0.000000in}}{%
\pgfpathmoveto{\pgfqpoint{0.000000in}{0.000000in}}%
\pgfpathlineto{\pgfqpoint{0.000000in}{-0.048611in}}%
\pgfusepath{stroke,fill}%
}%
\begin{pgfscope}%
\pgfsys@transformshift{4.263419in}{0.548769in}%
\pgfsys@useobject{currentmarker}{}%
\end{pgfscope}%
\end{pgfscope}%
\begin{pgfscope}%
\definecolor{textcolor}{rgb}{0.000000,0.000000,0.000000}%
\pgfsetstrokecolor{textcolor}%
\pgfsetfillcolor{textcolor}%
\pgftext[x=4.263419in,y=0.451547in,,top]{\color{textcolor}\sffamily\fontsize{10.000000}{12.000000}\selectfont \(\displaystyle {0.7}\)}%
\end{pgfscope}%
\begin{pgfscope}%
\pgfsetbuttcap%
\pgfsetroundjoin%
\definecolor{currentfill}{rgb}{0.000000,0.000000,0.000000}%
\pgfsetfillcolor{currentfill}%
\pgfsetlinewidth{0.803000pt}%
\definecolor{currentstroke}{rgb}{0.000000,0.000000,0.000000}%
\pgfsetstrokecolor{currentstroke}%
\pgfsetdash{}{0pt}%
\pgfsys@defobject{currentmarker}{\pgfqpoint{0.000000in}{-0.048611in}}{\pgfqpoint{0.000000in}{0.000000in}}{%
\pgfpathmoveto{\pgfqpoint{0.000000in}{0.000000in}}%
\pgfpathlineto{\pgfqpoint{0.000000in}{-0.048611in}}%
\pgfusepath{stroke,fill}%
}%
\begin{pgfscope}%
\pgfsys@transformshift{4.762812in}{0.548769in}%
\pgfsys@useobject{currentmarker}{}%
\end{pgfscope}%
\end{pgfscope}%
\begin{pgfscope}%
\definecolor{textcolor}{rgb}{0.000000,0.000000,0.000000}%
\pgfsetstrokecolor{textcolor}%
\pgfsetfillcolor{textcolor}%
\pgftext[x=4.762812in,y=0.451547in,,top]{\color{textcolor}\sffamily\fontsize{10.000000}{12.000000}\selectfont \(\displaystyle {0.8}\)}%
\end{pgfscope}%
\begin{pgfscope}%
\pgfsetbuttcap%
\pgfsetroundjoin%
\definecolor{currentfill}{rgb}{0.000000,0.000000,0.000000}%
\pgfsetfillcolor{currentfill}%
\pgfsetlinewidth{0.803000pt}%
\definecolor{currentstroke}{rgb}{0.000000,0.000000,0.000000}%
\pgfsetstrokecolor{currentstroke}%
\pgfsetdash{}{0pt}%
\pgfsys@defobject{currentmarker}{\pgfqpoint{0.000000in}{-0.048611in}}{\pgfqpoint{0.000000in}{0.000000in}}{%
\pgfpathmoveto{\pgfqpoint{0.000000in}{0.000000in}}%
\pgfpathlineto{\pgfqpoint{0.000000in}{-0.048611in}}%
\pgfusepath{stroke,fill}%
}%
\begin{pgfscope}%
\pgfsys@transformshift{5.262204in}{0.548769in}%
\pgfsys@useobject{currentmarker}{}%
\end{pgfscope}%
\end{pgfscope}%
\begin{pgfscope}%
\definecolor{textcolor}{rgb}{0.000000,0.000000,0.000000}%
\pgfsetstrokecolor{textcolor}%
\pgfsetfillcolor{textcolor}%
\pgftext[x=5.262204in,y=0.451547in,,top]{\color{textcolor}\sffamily\fontsize{10.000000}{12.000000}\selectfont \(\displaystyle {0.9}\)}%
\end{pgfscope}%
\begin{pgfscope}%
\pgfsetbuttcap%
\pgfsetroundjoin%
\definecolor{currentfill}{rgb}{0.000000,0.000000,0.000000}%
\pgfsetfillcolor{currentfill}%
\pgfsetlinewidth{0.803000pt}%
\definecolor{currentstroke}{rgb}{0.000000,0.000000,0.000000}%
\pgfsetstrokecolor{currentstroke}%
\pgfsetdash{}{0pt}%
\pgfsys@defobject{currentmarker}{\pgfqpoint{0.000000in}{-0.048611in}}{\pgfqpoint{0.000000in}{0.000000in}}{%
\pgfpathmoveto{\pgfqpoint{0.000000in}{0.000000in}}%
\pgfpathlineto{\pgfqpoint{0.000000in}{-0.048611in}}%
\pgfusepath{stroke,fill}%
}%
\begin{pgfscope}%
\pgfsys@transformshift{5.761597in}{0.548769in}%
\pgfsys@useobject{currentmarker}{}%
\end{pgfscope}%
\end{pgfscope}%
\begin{pgfscope}%
\definecolor{textcolor}{rgb}{0.000000,0.000000,0.000000}%
\pgfsetstrokecolor{textcolor}%
\pgfsetfillcolor{textcolor}%
\pgftext[x=5.761597in,y=0.451547in,,top]{\color{textcolor}\sffamily\fontsize{10.000000}{12.000000}\selectfont \(\displaystyle {1.0}\)}%
\end{pgfscope}%
\begin{pgfscope}%
\definecolor{textcolor}{rgb}{0.000000,0.000000,0.000000}%
\pgfsetstrokecolor{textcolor}%
\pgfsetfillcolor{textcolor}%
\pgftext[x=3.205150in,y=0.272658in,,top]{\color{textcolor}\sffamily\fontsize{10.000000}{12.000000}\selectfont Infoflow Edge Count}%
\end{pgfscope}%
\begin{pgfscope}%
\definecolor{textcolor}{rgb}{0.000000,0.000000,0.000000}%
\pgfsetstrokecolor{textcolor}%
\pgfsetfillcolor{textcolor}%
\pgftext[x=5.761597in,y=0.286547in,right,top]{\color{textcolor}\sffamily\fontsize{10.000000}{12.000000}\selectfont \(\displaystyle \times{10^{8}}{}\)}%
\end{pgfscope}%
\begin{pgfscope}%
\pgfsetbuttcap%
\pgfsetroundjoin%
\definecolor{currentfill}{rgb}{0.000000,0.000000,0.000000}%
\pgfsetfillcolor{currentfill}%
\pgfsetlinewidth{0.803000pt}%
\definecolor{currentstroke}{rgb}{0.000000,0.000000,0.000000}%
\pgfsetstrokecolor{currentstroke}%
\pgfsetdash{}{0pt}%
\pgfsys@defobject{currentmarker}{\pgfqpoint{-0.048611in}{0.000000in}}{\pgfqpoint{0.000000in}{0.000000in}}{%
\pgfpathmoveto{\pgfqpoint{0.000000in}{0.000000in}}%
\pgfpathlineto{\pgfqpoint{-0.048611in}{0.000000in}}%
\pgfusepath{stroke,fill}%
}%
\begin{pgfscope}%
\pgfsys@transformshift{0.648703in}{0.758960in}%
\pgfsys@useobject{currentmarker}{}%
\end{pgfscope}%
\end{pgfscope}%
\begin{pgfscope}%
\definecolor{textcolor}{rgb}{0.000000,0.000000,0.000000}%
\pgfsetstrokecolor{textcolor}%
\pgfsetfillcolor{textcolor}%
\pgftext[x=0.482036in, y=0.710765in, left, base]{\color{textcolor}\sffamily\fontsize{10.000000}{12.000000}\selectfont \(\displaystyle {0}\)}%
\end{pgfscope}%
\begin{pgfscope}%
\pgfsetbuttcap%
\pgfsetroundjoin%
\definecolor{currentfill}{rgb}{0.000000,0.000000,0.000000}%
\pgfsetfillcolor{currentfill}%
\pgfsetlinewidth{0.803000pt}%
\definecolor{currentstroke}{rgb}{0.000000,0.000000,0.000000}%
\pgfsetstrokecolor{currentstroke}%
\pgfsetdash{}{0pt}%
\pgfsys@defobject{currentmarker}{\pgfqpoint{-0.048611in}{0.000000in}}{\pgfqpoint{0.000000in}{0.000000in}}{%
\pgfpathmoveto{\pgfqpoint{0.000000in}{0.000000in}}%
\pgfpathlineto{\pgfqpoint{-0.048611in}{0.000000in}}%
\pgfusepath{stroke,fill}%
}%
\begin{pgfscope}%
\pgfsys@transformshift{0.648703in}{1.171459in}%
\pgfsys@useobject{currentmarker}{}%
\end{pgfscope}%
\end{pgfscope}%
\begin{pgfscope}%
\definecolor{textcolor}{rgb}{0.000000,0.000000,0.000000}%
\pgfsetstrokecolor{textcolor}%
\pgfsetfillcolor{textcolor}%
\pgftext[x=0.343147in, y=1.123265in, left, base]{\color{textcolor}\sffamily\fontsize{10.000000}{12.000000}\selectfont \(\displaystyle {100}\)}%
\end{pgfscope}%
\begin{pgfscope}%
\pgfsetbuttcap%
\pgfsetroundjoin%
\definecolor{currentfill}{rgb}{0.000000,0.000000,0.000000}%
\pgfsetfillcolor{currentfill}%
\pgfsetlinewidth{0.803000pt}%
\definecolor{currentstroke}{rgb}{0.000000,0.000000,0.000000}%
\pgfsetstrokecolor{currentstroke}%
\pgfsetdash{}{0pt}%
\pgfsys@defobject{currentmarker}{\pgfqpoint{-0.048611in}{0.000000in}}{\pgfqpoint{0.000000in}{0.000000in}}{%
\pgfpathmoveto{\pgfqpoint{0.000000in}{0.000000in}}%
\pgfpathlineto{\pgfqpoint{-0.048611in}{0.000000in}}%
\pgfusepath{stroke,fill}%
}%
\begin{pgfscope}%
\pgfsys@transformshift{0.648703in}{1.583959in}%
\pgfsys@useobject{currentmarker}{}%
\end{pgfscope}%
\end{pgfscope}%
\begin{pgfscope}%
\definecolor{textcolor}{rgb}{0.000000,0.000000,0.000000}%
\pgfsetstrokecolor{textcolor}%
\pgfsetfillcolor{textcolor}%
\pgftext[x=0.343147in, y=1.535764in, left, base]{\color{textcolor}\sffamily\fontsize{10.000000}{12.000000}\selectfont \(\displaystyle {200}\)}%
\end{pgfscope}%
\begin{pgfscope}%
\pgfsetbuttcap%
\pgfsetroundjoin%
\definecolor{currentfill}{rgb}{0.000000,0.000000,0.000000}%
\pgfsetfillcolor{currentfill}%
\pgfsetlinewidth{0.803000pt}%
\definecolor{currentstroke}{rgb}{0.000000,0.000000,0.000000}%
\pgfsetstrokecolor{currentstroke}%
\pgfsetdash{}{0pt}%
\pgfsys@defobject{currentmarker}{\pgfqpoint{-0.048611in}{0.000000in}}{\pgfqpoint{0.000000in}{0.000000in}}{%
\pgfpathmoveto{\pgfqpoint{0.000000in}{0.000000in}}%
\pgfpathlineto{\pgfqpoint{-0.048611in}{0.000000in}}%
\pgfusepath{stroke,fill}%
}%
\begin{pgfscope}%
\pgfsys@transformshift{0.648703in}{1.996458in}%
\pgfsys@useobject{currentmarker}{}%
\end{pgfscope}%
\end{pgfscope}%
\begin{pgfscope}%
\definecolor{textcolor}{rgb}{0.000000,0.000000,0.000000}%
\pgfsetstrokecolor{textcolor}%
\pgfsetfillcolor{textcolor}%
\pgftext[x=0.343147in, y=1.948264in, left, base]{\color{textcolor}\sffamily\fontsize{10.000000}{12.000000}\selectfont \(\displaystyle {300}\)}%
\end{pgfscope}%
\begin{pgfscope}%
\pgfsetbuttcap%
\pgfsetroundjoin%
\definecolor{currentfill}{rgb}{0.000000,0.000000,0.000000}%
\pgfsetfillcolor{currentfill}%
\pgfsetlinewidth{0.803000pt}%
\definecolor{currentstroke}{rgb}{0.000000,0.000000,0.000000}%
\pgfsetstrokecolor{currentstroke}%
\pgfsetdash{}{0pt}%
\pgfsys@defobject{currentmarker}{\pgfqpoint{-0.048611in}{0.000000in}}{\pgfqpoint{0.000000in}{0.000000in}}{%
\pgfpathmoveto{\pgfqpoint{0.000000in}{0.000000in}}%
\pgfpathlineto{\pgfqpoint{-0.048611in}{0.000000in}}%
\pgfusepath{stroke,fill}%
}%
\begin{pgfscope}%
\pgfsys@transformshift{0.648703in}{2.408958in}%
\pgfsys@useobject{currentmarker}{}%
\end{pgfscope}%
\end{pgfscope}%
\begin{pgfscope}%
\definecolor{textcolor}{rgb}{0.000000,0.000000,0.000000}%
\pgfsetstrokecolor{textcolor}%
\pgfsetfillcolor{textcolor}%
\pgftext[x=0.343147in, y=2.360764in, left, base]{\color{textcolor}\sffamily\fontsize{10.000000}{12.000000}\selectfont \(\displaystyle {400}\)}%
\end{pgfscope}%
\begin{pgfscope}%
\pgfsetbuttcap%
\pgfsetroundjoin%
\definecolor{currentfill}{rgb}{0.000000,0.000000,0.000000}%
\pgfsetfillcolor{currentfill}%
\pgfsetlinewidth{0.803000pt}%
\definecolor{currentstroke}{rgb}{0.000000,0.000000,0.000000}%
\pgfsetstrokecolor{currentstroke}%
\pgfsetdash{}{0pt}%
\pgfsys@defobject{currentmarker}{\pgfqpoint{-0.048611in}{0.000000in}}{\pgfqpoint{0.000000in}{0.000000in}}{%
\pgfpathmoveto{\pgfqpoint{0.000000in}{0.000000in}}%
\pgfpathlineto{\pgfqpoint{-0.048611in}{0.000000in}}%
\pgfusepath{stroke,fill}%
}%
\begin{pgfscope}%
\pgfsys@transformshift{0.648703in}{2.821458in}%
\pgfsys@useobject{currentmarker}{}%
\end{pgfscope}%
\end{pgfscope}%
\begin{pgfscope}%
\definecolor{textcolor}{rgb}{0.000000,0.000000,0.000000}%
\pgfsetstrokecolor{textcolor}%
\pgfsetfillcolor{textcolor}%
\pgftext[x=0.343147in, y=2.773263in, left, base]{\color{textcolor}\sffamily\fontsize{10.000000}{12.000000}\selectfont \(\displaystyle {500}\)}%
\end{pgfscope}%
\begin{pgfscope}%
\pgfsetbuttcap%
\pgfsetroundjoin%
\definecolor{currentfill}{rgb}{0.000000,0.000000,0.000000}%
\pgfsetfillcolor{currentfill}%
\pgfsetlinewidth{0.803000pt}%
\definecolor{currentstroke}{rgb}{0.000000,0.000000,0.000000}%
\pgfsetstrokecolor{currentstroke}%
\pgfsetdash{}{0pt}%
\pgfsys@defobject{currentmarker}{\pgfqpoint{-0.048611in}{0.000000in}}{\pgfqpoint{0.000000in}{0.000000in}}{%
\pgfpathmoveto{\pgfqpoint{0.000000in}{0.000000in}}%
\pgfpathlineto{\pgfqpoint{-0.048611in}{0.000000in}}%
\pgfusepath{stroke,fill}%
}%
\begin{pgfscope}%
\pgfsys@transformshift{0.648703in}{3.233957in}%
\pgfsys@useobject{currentmarker}{}%
\end{pgfscope}%
\end{pgfscope}%
\begin{pgfscope}%
\definecolor{textcolor}{rgb}{0.000000,0.000000,0.000000}%
\pgfsetstrokecolor{textcolor}%
\pgfsetfillcolor{textcolor}%
\pgftext[x=0.343147in, y=3.185763in, left, base]{\color{textcolor}\sffamily\fontsize{10.000000}{12.000000}\selectfont \(\displaystyle {600}\)}%
\end{pgfscope}%
\begin{pgfscope}%
\pgfsetbuttcap%
\pgfsetroundjoin%
\definecolor{currentfill}{rgb}{0.000000,0.000000,0.000000}%
\pgfsetfillcolor{currentfill}%
\pgfsetlinewidth{0.803000pt}%
\definecolor{currentstroke}{rgb}{0.000000,0.000000,0.000000}%
\pgfsetstrokecolor{currentstroke}%
\pgfsetdash{}{0pt}%
\pgfsys@defobject{currentmarker}{\pgfqpoint{-0.048611in}{0.000000in}}{\pgfqpoint{0.000000in}{0.000000in}}{%
\pgfpathmoveto{\pgfqpoint{0.000000in}{0.000000in}}%
\pgfpathlineto{\pgfqpoint{-0.048611in}{0.000000in}}%
\pgfusepath{stroke,fill}%
}%
\begin{pgfscope}%
\pgfsys@transformshift{0.648703in}{3.646457in}%
\pgfsys@useobject{currentmarker}{}%
\end{pgfscope}%
\end{pgfscope}%
\begin{pgfscope}%
\definecolor{textcolor}{rgb}{0.000000,0.000000,0.000000}%
\pgfsetstrokecolor{textcolor}%
\pgfsetfillcolor{textcolor}%
\pgftext[x=0.343147in, y=3.598262in, left, base]{\color{textcolor}\sffamily\fontsize{10.000000}{12.000000}\selectfont \(\displaystyle {700}\)}%
\end{pgfscope}%
\begin{pgfscope}%
\definecolor{textcolor}{rgb}{0.000000,0.000000,0.000000}%
\pgfsetstrokecolor{textcolor}%
\pgfsetfillcolor{textcolor}%
\pgftext[x=0.287592in,y=2.100064in,,bottom,rotate=90.000000]{\color{textcolor}\sffamily\fontsize{10.000000}{12.000000}\selectfont Data Flow Time (s)}%
\end{pgfscope}%
\begin{pgfscope}%
\pgfpathrectangle{\pgfqpoint{0.648703in}{0.548769in}}{\pgfqpoint{5.112893in}{3.102590in}}%
\pgfusepath{clip}%
\pgfsetrectcap%
\pgfsetroundjoin%
\pgfsetlinewidth{1.505625pt}%
\definecolor{currentstroke}{rgb}{0.000000,0.500000,0.000000}%
\pgfsetstrokecolor{currentstroke}%
\pgfsetdash{}{0pt}%
\pgfpathmoveto{\pgfqpoint{0.767673in}{0.689796in}}%
\pgfpathlineto{\pgfqpoint{0.791707in}{0.860732in}}%
\pgfpathlineto{\pgfqpoint{0.815741in}{1.023474in}}%
\pgfpathlineto{\pgfqpoint{0.839776in}{1.178274in}}%
\pgfpathlineto{\pgfqpoint{0.863810in}{1.325383in}}%
\pgfpathlineto{\pgfqpoint{0.887844in}{1.465045in}}%
\pgfpathlineto{\pgfqpoint{0.911878in}{1.597501in}}%
\pgfpathlineto{\pgfqpoint{0.935913in}{1.722990in}}%
\pgfpathlineto{\pgfqpoint{0.959947in}{1.841746in}}%
\pgfpathlineto{\pgfqpoint{0.983981in}{1.953998in}}%
\pgfpathlineto{\pgfqpoint{1.008015in}{2.059973in}}%
\pgfpathlineto{\pgfqpoint{1.032049in}{2.159894in}}%
\pgfpathlineto{\pgfqpoint{1.056084in}{2.253978in}}%
\pgfpathlineto{\pgfqpoint{1.080118in}{2.342442in}}%
\pgfpathlineto{\pgfqpoint{1.104152in}{2.425496in}}%
\pgfpathlineto{\pgfqpoint{1.128186in}{2.503347in}}%
\pgfpathlineto{\pgfqpoint{1.152221in}{2.576199in}}%
\pgfpathlineto{\pgfqpoint{1.176255in}{2.644252in}}%
\pgfpathlineto{\pgfqpoint{1.200289in}{2.707701in}}%
\pgfpathlineto{\pgfqpoint{1.224323in}{2.766739in}}%
\pgfpathlineto{\pgfqpoint{1.248358in}{2.821554in}}%
\pgfpathlineto{\pgfqpoint{1.272392in}{2.872329in}}%
\pgfpathlineto{\pgfqpoint{1.296426in}{2.919247in}}%
\pgfpathlineto{\pgfqpoint{1.320460in}{2.962483in}}%
\pgfpathlineto{\pgfqpoint{1.344495in}{3.002211in}}%
\pgfpathlineto{\pgfqpoint{1.368529in}{3.038600in}}%
\pgfpathlineto{\pgfqpoint{1.392563in}{3.071815in}}%
\pgfpathlineto{\pgfqpoint{1.416597in}{3.102018in}}%
\pgfpathlineto{\pgfqpoint{1.440632in}{3.129367in}}%
\pgfpathlineto{\pgfqpoint{1.464666in}{3.154015in}}%
\pgfpathlineto{\pgfqpoint{1.488700in}{3.176114in}}%
\pgfpathlineto{\pgfqpoint{1.512734in}{3.195809in}}%
\pgfpathlineto{\pgfqpoint{1.536768in}{3.213242in}}%
\pgfpathlineto{\pgfqpoint{1.560803in}{3.228554in}}%
\pgfpathlineto{\pgfqpoint{1.584837in}{3.241878in}}%
\pgfpathlineto{\pgfqpoint{1.608871in}{3.253345in}}%
\pgfpathlineto{\pgfqpoint{1.632905in}{3.263084in}}%
\pgfpathlineto{\pgfqpoint{1.656940in}{3.271217in}}%
\pgfpathlineto{\pgfqpoint{1.680974in}{3.277865in}}%
\pgfpathlineto{\pgfqpoint{1.705008in}{3.283143in}}%
\pgfpathlineto{\pgfqpoint{1.729042in}{3.287164in}}%
\pgfpathlineto{\pgfqpoint{1.753077in}{3.290035in}}%
\pgfpathlineto{\pgfqpoint{1.777111in}{3.291862in}}%
\pgfpathlineto{\pgfqpoint{1.801145in}{3.292744in}}%
\pgfpathlineto{\pgfqpoint{1.825179in}{3.292780in}}%
\pgfpathlineto{\pgfqpoint{1.849214in}{3.292061in}}%
\pgfpathlineto{\pgfqpoint{1.873248in}{3.290677in}}%
\pgfpathlineto{\pgfqpoint{1.897282in}{3.288715in}}%
\pgfpathlineto{\pgfqpoint{1.921316in}{3.286254in}}%
\pgfpathlineto{\pgfqpoint{1.945351in}{3.283374in}}%
\pgfpathlineto{\pgfqpoint{1.969385in}{3.280148in}}%
\pgfpathlineto{\pgfqpoint{1.993419in}{3.276646in}}%
\pgfpathlineto{\pgfqpoint{2.017453in}{3.272936in}}%
\pgfpathlineto{\pgfqpoint{2.041487in}{3.269079in}}%
\pgfpathlineto{\pgfqpoint{2.065522in}{3.265134in}}%
\pgfpathlineto{\pgfqpoint{2.089556in}{3.261157in}}%
\pgfpathlineto{\pgfqpoint{2.113590in}{3.257199in}}%
\pgfpathlineto{\pgfqpoint{2.137624in}{3.253306in}}%
\pgfpathlineto{\pgfqpoint{2.161659in}{3.249524in}}%
\pgfpathlineto{\pgfqpoint{2.185693in}{3.245890in}}%
\pgfpathlineto{\pgfqpoint{2.209727in}{3.242442in}}%
\pgfpathlineto{\pgfqpoint{2.233761in}{3.239212in}}%
\pgfpathlineto{\pgfqpoint{2.257796in}{3.236228in}}%
\pgfpathlineto{\pgfqpoint{2.281830in}{3.233514in}}%
\pgfpathlineto{\pgfqpoint{2.305864in}{3.231092in}}%
\pgfpathlineto{\pgfqpoint{2.329898in}{3.228977in}}%
\pgfpathlineto{\pgfqpoint{2.353933in}{3.227185in}}%
\pgfpathlineto{\pgfqpoint{2.377967in}{3.225723in}}%
\pgfpathlineto{\pgfqpoint{2.402001in}{3.224597in}}%
\pgfpathlineto{\pgfqpoint{2.426035in}{3.223809in}}%
\pgfpathlineto{\pgfqpoint{2.450069in}{3.223358in}}%
\pgfpathlineto{\pgfqpoint{2.474104in}{3.223236in}}%
\pgfpathlineto{\pgfqpoint{2.498138in}{3.223434in}}%
\pgfpathlineto{\pgfqpoint{2.522172in}{3.223940in}}%
\pgfpathlineto{\pgfqpoint{2.546206in}{3.224735in}}%
\pgfpathlineto{\pgfqpoint{2.570241in}{3.225799in}}%
\pgfpathlineto{\pgfqpoint{2.594275in}{3.227106in}}%
\pgfpathlineto{\pgfqpoint{2.618309in}{3.228627in}}%
\pgfpathlineto{\pgfqpoint{2.642343in}{3.230331in}}%
\pgfpathlineto{\pgfqpoint{2.666378in}{3.232181in}}%
\pgfpathlineto{\pgfqpoint{2.690412in}{3.234137in}}%
\pgfpathlineto{\pgfqpoint{2.714446in}{3.236155in}}%
\pgfpathlineto{\pgfqpoint{2.738480in}{3.238186in}}%
\pgfpathlineto{\pgfqpoint{2.762515in}{3.240181in}}%
\pgfpathlineto{\pgfqpoint{2.786549in}{3.242082in}}%
\pgfpathlineto{\pgfqpoint{2.810583in}{3.243831in}}%
\pgfpathlineto{\pgfqpoint{2.834617in}{3.245365in}}%
\pgfpathlineto{\pgfqpoint{2.858652in}{3.246618in}}%
\pgfpathlineto{\pgfqpoint{2.882686in}{3.247517in}}%
\pgfpathlineto{\pgfqpoint{2.906720in}{3.247990in}}%
\pgfpathlineto{\pgfqpoint{2.930754in}{3.247958in}}%
\pgfpathlineto{\pgfqpoint{2.954788in}{3.247339in}}%
\pgfpathlineto{\pgfqpoint{2.978823in}{3.246047in}}%
\pgfpathlineto{\pgfqpoint{3.002857in}{3.243993in}}%
\pgfpathlineto{\pgfqpoint{3.026891in}{3.241082in}}%
\pgfpathlineto{\pgfqpoint{3.050925in}{3.237219in}}%
\pgfpathlineto{\pgfqpoint{3.074960in}{3.232301in}}%
\pgfpathlineto{\pgfqpoint{3.098994in}{3.226223in}}%
\pgfpathlineto{\pgfqpoint{3.123028in}{3.218878in}}%
\pgfpathlineto{\pgfqpoint{3.147062in}{3.210153in}}%
\pgfusepath{stroke}%
\end{pgfscope}%
\begin{pgfscope}%
\pgfsetrectcap%
\pgfsetmiterjoin%
\pgfsetlinewidth{0.803000pt}%
\definecolor{currentstroke}{rgb}{0.000000,0.000000,0.000000}%
\pgfsetstrokecolor{currentstroke}%
\pgfsetdash{}{0pt}%
\pgfpathmoveto{\pgfqpoint{0.648703in}{0.548769in}}%
\pgfpathlineto{\pgfqpoint{0.648703in}{3.651359in}}%
\pgfusepath{stroke}%
\end{pgfscope}%
\begin{pgfscope}%
\pgfsetrectcap%
\pgfsetmiterjoin%
\pgfsetlinewidth{0.803000pt}%
\definecolor{currentstroke}{rgb}{0.000000,0.000000,0.000000}%
\pgfsetstrokecolor{currentstroke}%
\pgfsetdash{}{0pt}%
\pgfpathmoveto{\pgfqpoint{5.761597in}{0.548769in}}%
\pgfpathlineto{\pgfqpoint{5.761597in}{3.651359in}}%
\pgfusepath{stroke}%
\end{pgfscope}%
\begin{pgfscope}%
\pgfsetrectcap%
\pgfsetmiterjoin%
\pgfsetlinewidth{0.803000pt}%
\definecolor{currentstroke}{rgb}{0.000000,0.000000,0.000000}%
\pgfsetstrokecolor{currentstroke}%
\pgfsetdash{}{0pt}%
\pgfpathmoveto{\pgfqpoint{0.648703in}{0.548769in}}%
\pgfpathlineto{\pgfqpoint{5.761597in}{0.548769in}}%
\pgfusepath{stroke}%
\end{pgfscope}%
\begin{pgfscope}%
\pgfsetrectcap%
\pgfsetmiterjoin%
\pgfsetlinewidth{0.803000pt}%
\definecolor{currentstroke}{rgb}{0.000000,0.000000,0.000000}%
\pgfsetstrokecolor{currentstroke}%
\pgfsetdash{}{0pt}%
\pgfpathmoveto{\pgfqpoint{0.648703in}{3.651359in}}%
\pgfpathlineto{\pgfqpoint{5.761597in}{3.651359in}}%
\pgfusepath{stroke}%
\end{pgfscope}%
\begin{pgfscope}%
\definecolor{textcolor}{rgb}{0.000000,0.000000,0.000000}%
\pgfsetstrokecolor{textcolor}%
\pgfsetfillcolor{textcolor}%
\pgftext[x=3.205150in,y=3.734692in,,base]{\color{textcolor}\sffamily\fontsize{12.000000}{14.400000}\selectfont BW}%
\end{pgfscope}%
\begin{pgfscope}%
\pgfsetbuttcap%
\pgfsetmiterjoin%
\definecolor{currentfill}{rgb}{1.000000,1.000000,1.000000}%
\pgfsetfillcolor{currentfill}%
\pgfsetfillopacity{0.800000}%
\pgfsetlinewidth{1.003750pt}%
\definecolor{currentstroke}{rgb}{0.800000,0.800000,0.800000}%
\pgfsetstrokecolor{currentstroke}%
\pgfsetstrokeopacity{0.800000}%
\pgfsetdash{}{0pt}%
\pgfpathmoveto{\pgfqpoint{4.212013in}{2.762053in}}%
\pgfpathlineto{\pgfqpoint{5.664374in}{2.762053in}}%
\pgfpathquadraticcurveto{\pgfqpoint{5.692152in}{2.762053in}}{\pgfqpoint{5.692152in}{2.789831in}}%
\pgfpathlineto{\pgfqpoint{5.692152in}{3.554136in}}%
\pgfpathquadraticcurveto{\pgfqpoint{5.692152in}{3.581914in}}{\pgfqpoint{5.664374in}{3.581914in}}%
\pgfpathlineto{\pgfqpoint{4.212013in}{3.581914in}}%
\pgfpathquadraticcurveto{\pgfqpoint{4.184236in}{3.581914in}}{\pgfqpoint{4.184236in}{3.554136in}}%
\pgfpathlineto{\pgfqpoint{4.184236in}{2.789831in}}%
\pgfpathquadraticcurveto{\pgfqpoint{4.184236in}{2.762053in}}{\pgfqpoint{4.212013in}{2.762053in}}%
\pgfpathclose%
\pgfusepath{stroke,fill}%
\end{pgfscope}%
\begin{pgfscope}%
\pgfsetbuttcap%
\pgfsetroundjoin%
\definecolor{currentfill}{rgb}{0.121569,0.466667,0.705882}%
\pgfsetfillcolor{currentfill}%
\pgfsetlinewidth{1.003750pt}%
\definecolor{currentstroke}{rgb}{0.121569,0.466667,0.705882}%
\pgfsetstrokecolor{currentstroke}%
\pgfsetdash{}{0pt}%
\pgfsys@defobject{currentmarker}{\pgfqpoint{-0.034722in}{-0.034722in}}{\pgfqpoint{0.034722in}{0.034722in}}{%
\pgfpathmoveto{\pgfqpoint{0.000000in}{-0.034722in}}%
\pgfpathcurveto{\pgfqpoint{0.009208in}{-0.034722in}}{\pgfqpoint{0.018041in}{-0.031064in}}{\pgfqpoint{0.024552in}{-0.024552in}}%
\pgfpathcurveto{\pgfqpoint{0.031064in}{-0.018041in}}{\pgfqpoint{0.034722in}{-0.009208in}}{\pgfqpoint{0.034722in}{0.000000in}}%
\pgfpathcurveto{\pgfqpoint{0.034722in}{0.009208in}}{\pgfqpoint{0.031064in}{0.018041in}}{\pgfqpoint{0.024552in}{0.024552in}}%
\pgfpathcurveto{\pgfqpoint{0.018041in}{0.031064in}}{\pgfqpoint{0.009208in}{0.034722in}}{\pgfqpoint{0.000000in}{0.034722in}}%
\pgfpathcurveto{\pgfqpoint{-0.009208in}{0.034722in}}{\pgfqpoint{-0.018041in}{0.031064in}}{\pgfqpoint{-0.024552in}{0.024552in}}%
\pgfpathcurveto{\pgfqpoint{-0.031064in}{0.018041in}}{\pgfqpoint{-0.034722in}{0.009208in}}{\pgfqpoint{-0.034722in}{0.000000in}}%
\pgfpathcurveto{\pgfqpoint{-0.034722in}{-0.009208in}}{\pgfqpoint{-0.031064in}{-0.018041in}}{\pgfqpoint{-0.024552in}{-0.024552in}}%
\pgfpathcurveto{\pgfqpoint{-0.018041in}{-0.031064in}}{\pgfqpoint{-0.009208in}{-0.034722in}}{\pgfqpoint{0.000000in}{-0.034722in}}%
\pgfpathclose%
\pgfusepath{stroke,fill}%
}%
\begin{pgfscope}%
\pgfsys@transformshift{4.378680in}{3.477748in}%
\pgfsys@useobject{currentmarker}{}%
\end{pgfscope}%
\end{pgfscope}%
\begin{pgfscope}%
\definecolor{textcolor}{rgb}{0.000000,0.000000,0.000000}%
\pgfsetstrokecolor{textcolor}%
\pgfsetfillcolor{textcolor}%
\pgftext[x=4.628680in,y=3.429136in,left,base]{\color{textcolor}\sffamily\fontsize{10.000000}{12.000000}\selectfont No Timeout}%
\end{pgfscope}%
\begin{pgfscope}%
\pgfsetbuttcap%
\pgfsetroundjoin%
\definecolor{currentfill}{rgb}{1.000000,0.498039,0.054902}%
\pgfsetfillcolor{currentfill}%
\pgfsetlinewidth{1.003750pt}%
\definecolor{currentstroke}{rgb}{1.000000,0.498039,0.054902}%
\pgfsetstrokecolor{currentstroke}%
\pgfsetdash{}{0pt}%
\pgfsys@defobject{currentmarker}{\pgfqpoint{-0.034722in}{-0.034722in}}{\pgfqpoint{0.034722in}{0.034722in}}{%
\pgfpathmoveto{\pgfqpoint{0.000000in}{-0.034722in}}%
\pgfpathcurveto{\pgfqpoint{0.009208in}{-0.034722in}}{\pgfqpoint{0.018041in}{-0.031064in}}{\pgfqpoint{0.024552in}{-0.024552in}}%
\pgfpathcurveto{\pgfqpoint{0.031064in}{-0.018041in}}{\pgfqpoint{0.034722in}{-0.009208in}}{\pgfqpoint{0.034722in}{0.000000in}}%
\pgfpathcurveto{\pgfqpoint{0.034722in}{0.009208in}}{\pgfqpoint{0.031064in}{0.018041in}}{\pgfqpoint{0.024552in}{0.024552in}}%
\pgfpathcurveto{\pgfqpoint{0.018041in}{0.031064in}}{\pgfqpoint{0.009208in}{0.034722in}}{\pgfqpoint{0.000000in}{0.034722in}}%
\pgfpathcurveto{\pgfqpoint{-0.009208in}{0.034722in}}{\pgfqpoint{-0.018041in}{0.031064in}}{\pgfqpoint{-0.024552in}{0.024552in}}%
\pgfpathcurveto{\pgfqpoint{-0.031064in}{0.018041in}}{\pgfqpoint{-0.034722in}{0.009208in}}{\pgfqpoint{-0.034722in}{0.000000in}}%
\pgfpathcurveto{\pgfqpoint{-0.034722in}{-0.009208in}}{\pgfqpoint{-0.031064in}{-0.018041in}}{\pgfqpoint{-0.024552in}{-0.024552in}}%
\pgfpathcurveto{\pgfqpoint{-0.018041in}{-0.031064in}}{\pgfqpoint{-0.009208in}{-0.034722in}}{\pgfqpoint{0.000000in}{-0.034722in}}%
\pgfpathclose%
\pgfusepath{stroke,fill}%
}%
\begin{pgfscope}%
\pgfsys@transformshift{4.378680in}{3.284136in}%
\pgfsys@useobject{currentmarker}{}%
\end{pgfscope}%
\end{pgfscope}%
\begin{pgfscope}%
\definecolor{textcolor}{rgb}{0.000000,0.000000,0.000000}%
\pgfsetstrokecolor{textcolor}%
\pgfsetfillcolor{textcolor}%
\pgftext[x=4.628680in,y=3.235525in,left,base]{\color{textcolor}\sffamily\fontsize{10.000000}{12.000000}\selectfont Time Timeout}%
\end{pgfscope}%
\begin{pgfscope}%
\pgfsetbuttcap%
\pgfsetroundjoin%
\definecolor{currentfill}{rgb}{0.839216,0.152941,0.156863}%
\pgfsetfillcolor{currentfill}%
\pgfsetlinewidth{1.003750pt}%
\definecolor{currentstroke}{rgb}{0.839216,0.152941,0.156863}%
\pgfsetstrokecolor{currentstroke}%
\pgfsetdash{}{0pt}%
\pgfsys@defobject{currentmarker}{\pgfqpoint{-0.034722in}{-0.034722in}}{\pgfqpoint{0.034722in}{0.034722in}}{%
\pgfpathmoveto{\pgfqpoint{0.000000in}{-0.034722in}}%
\pgfpathcurveto{\pgfqpoint{0.009208in}{-0.034722in}}{\pgfqpoint{0.018041in}{-0.031064in}}{\pgfqpoint{0.024552in}{-0.024552in}}%
\pgfpathcurveto{\pgfqpoint{0.031064in}{-0.018041in}}{\pgfqpoint{0.034722in}{-0.009208in}}{\pgfqpoint{0.034722in}{0.000000in}}%
\pgfpathcurveto{\pgfqpoint{0.034722in}{0.009208in}}{\pgfqpoint{0.031064in}{0.018041in}}{\pgfqpoint{0.024552in}{0.024552in}}%
\pgfpathcurveto{\pgfqpoint{0.018041in}{0.031064in}}{\pgfqpoint{0.009208in}{0.034722in}}{\pgfqpoint{0.000000in}{0.034722in}}%
\pgfpathcurveto{\pgfqpoint{-0.009208in}{0.034722in}}{\pgfqpoint{-0.018041in}{0.031064in}}{\pgfqpoint{-0.024552in}{0.024552in}}%
\pgfpathcurveto{\pgfqpoint{-0.031064in}{0.018041in}}{\pgfqpoint{-0.034722in}{0.009208in}}{\pgfqpoint{-0.034722in}{0.000000in}}%
\pgfpathcurveto{\pgfqpoint{-0.034722in}{-0.009208in}}{\pgfqpoint{-0.031064in}{-0.018041in}}{\pgfqpoint{-0.024552in}{-0.024552in}}%
\pgfpathcurveto{\pgfqpoint{-0.018041in}{-0.031064in}}{\pgfqpoint{-0.009208in}{-0.034722in}}{\pgfqpoint{0.000000in}{-0.034722in}}%
\pgfpathclose%
\pgfusepath{stroke,fill}%
}%
\begin{pgfscope}%
\pgfsys@transformshift{4.378680in}{3.090525in}%
\pgfsys@useobject{currentmarker}{}%
\end{pgfscope}%
\end{pgfscope}%
\begin{pgfscope}%
\definecolor{textcolor}{rgb}{0.000000,0.000000,0.000000}%
\pgfsetstrokecolor{textcolor}%
\pgfsetfillcolor{textcolor}%
\pgftext[x=4.628680in,y=3.041914in,left,base]{\color{textcolor}\sffamily\fontsize{10.000000}{12.000000}\selectfont Memory Timeout}%
\end{pgfscope}%
\begin{pgfscope}%
\pgfsetrectcap%
\pgfsetroundjoin%
\pgfsetlinewidth{1.505625pt}%
\definecolor{currentstroke}{rgb}{0.000000,0.500000,0.000000}%
\pgfsetstrokecolor{currentstroke}%
\pgfsetdash{}{0pt}%
\pgfpathmoveto{\pgfqpoint{4.239791in}{2.894692in}}%
\pgfpathlineto{\pgfqpoint{4.517569in}{2.894692in}}%
\pgfusepath{stroke}%
\end{pgfscope}%
\begin{pgfscope}%
\definecolor{textcolor}{rgb}{0.000000,0.000000,0.000000}%
\pgfsetstrokecolor{textcolor}%
\pgfsetfillcolor{textcolor}%
\pgftext[x=4.628680in,y=2.846081in,left,base]{\color{textcolor}\sffamily\fontsize{10.000000}{12.000000}\selectfont Polyfit}%
\end{pgfscope}%
\end{pgfpicture}%
\makeatother%
\endgroup%

                }
            \end{subfigure}
            \caption{Infoflow Edges}
            \label{f:dfedgesi}
        \end{subfigure}
        \bigbreak
        \begin{subfigure}[b]{\textwidth}
            \centering
            \begin{subfigure}[]{0.45\textwidth}
                \centering
                \resizebox{\columnwidth}{!}{
                    %% Creator: Matplotlib, PGF backend
%%
%% To include the figure in your LaTeX document, write
%%   \input{<filename>.pgf}
%%
%% Make sure the required packages are loaded in your preamble
%%   \usepackage{pgf}
%%
%% and, on pdftex
%%   \usepackage[utf8]{inputenc}\DeclareUnicodeCharacter{2212}{-}
%%
%% or, on luatex and xetex
%%   \usepackage{unicode-math}
%%
%% Figures using additional raster images can only be included by \input if
%% they are in the same directory as the main LaTeX file. For loading figures
%% from other directories you can use the `import` package
%%   \usepackage{import}
%%
%% and then include the figures with
%%   \import{<path to file>}{<filename>.pgf}
%%
%% Matplotlib used the following preamble
%%   \usepackage{amsmath}
%%   \usepackage{fontspec}
%%
\begingroup%
\makeatletter%
\begin{pgfpicture}%
\pgfpathrectangle{\pgfpointorigin}{\pgfqpoint{6.000000in}{4.000000in}}%
\pgfusepath{use as bounding box, clip}%
\begin{pgfscope}%
\pgfsetbuttcap%
\pgfsetmiterjoin%
\definecolor{currentfill}{rgb}{1.000000,1.000000,1.000000}%
\pgfsetfillcolor{currentfill}%
\pgfsetlinewidth{0.000000pt}%
\definecolor{currentstroke}{rgb}{1.000000,1.000000,1.000000}%
\pgfsetstrokecolor{currentstroke}%
\pgfsetdash{}{0pt}%
\pgfpathmoveto{\pgfqpoint{0.000000in}{0.000000in}}%
\pgfpathlineto{\pgfqpoint{6.000000in}{0.000000in}}%
\pgfpathlineto{\pgfqpoint{6.000000in}{4.000000in}}%
\pgfpathlineto{\pgfqpoint{0.000000in}{4.000000in}}%
\pgfpathclose%
\pgfusepath{fill}%
\end{pgfscope}%
\begin{pgfscope}%
\pgfsetbuttcap%
\pgfsetmiterjoin%
\definecolor{currentfill}{rgb}{1.000000,1.000000,1.000000}%
\pgfsetfillcolor{currentfill}%
\pgfsetlinewidth{0.000000pt}%
\definecolor{currentstroke}{rgb}{0.000000,0.000000,0.000000}%
\pgfsetstrokecolor{currentstroke}%
\pgfsetstrokeopacity{0.000000}%
\pgfsetdash{}{0pt}%
\pgfpathmoveto{\pgfqpoint{0.648703in}{0.548769in}}%
\pgfpathlineto{\pgfqpoint{5.761597in}{0.548769in}}%
\pgfpathlineto{\pgfqpoint{5.761597in}{3.651359in}}%
\pgfpathlineto{\pgfqpoint{0.648703in}{3.651359in}}%
\pgfpathclose%
\pgfusepath{fill}%
\end{pgfscope}%
\begin{pgfscope}%
\pgfpathrectangle{\pgfqpoint{0.648703in}{0.548769in}}{\pgfqpoint{5.112893in}{3.102590in}}%
\pgfusepath{clip}%
\pgfsetbuttcap%
\pgfsetroundjoin%
\definecolor{currentfill}{rgb}{0.121569,0.466667,0.705882}%
\pgfsetfillcolor{currentfill}%
\pgfsetlinewidth{1.003750pt}%
\definecolor{currentstroke}{rgb}{0.121569,0.466667,0.705882}%
\pgfsetstrokecolor{currentstroke}%
\pgfsetdash{}{0pt}%
\pgfpathmoveto{\pgfqpoint{0.719388in}{0.663642in}}%
\pgfpathcurveto{\pgfqpoint{0.730438in}{0.663642in}}{\pgfqpoint{0.741037in}{0.668032in}}{\pgfqpoint{0.748851in}{0.675846in}}%
\pgfpathcurveto{\pgfqpoint{0.756665in}{0.683659in}}{\pgfqpoint{0.761055in}{0.694258in}}{\pgfqpoint{0.761055in}{0.705309in}}%
\pgfpathcurveto{\pgfqpoint{0.761055in}{0.716359in}}{\pgfqpoint{0.756665in}{0.726958in}}{\pgfqpoint{0.748851in}{0.734771in}}%
\pgfpathcurveto{\pgfqpoint{0.741037in}{0.742585in}}{\pgfqpoint{0.730438in}{0.746975in}}{\pgfqpoint{0.719388in}{0.746975in}}%
\pgfpathcurveto{\pgfqpoint{0.708338in}{0.746975in}}{\pgfqpoint{0.697739in}{0.742585in}}{\pgfqpoint{0.689926in}{0.734771in}}%
\pgfpathcurveto{\pgfqpoint{0.682112in}{0.726958in}}{\pgfqpoint{0.677722in}{0.716359in}}{\pgfqpoint{0.677722in}{0.705309in}}%
\pgfpathcurveto{\pgfqpoint{0.677722in}{0.694258in}}{\pgfqpoint{0.682112in}{0.683659in}}{\pgfqpoint{0.689926in}{0.675846in}}%
\pgfpathcurveto{\pgfqpoint{0.697739in}{0.668032in}}{\pgfqpoint{0.708338in}{0.663642in}}{\pgfqpoint{0.719388in}{0.663642in}}%
\pgfpathclose%
\pgfusepath{stroke,fill}%
\end{pgfscope}%
\begin{pgfscope}%
\pgfpathrectangle{\pgfqpoint{0.648703in}{0.548769in}}{\pgfqpoint{5.112893in}{3.102590in}}%
\pgfusepath{clip}%
\pgfsetbuttcap%
\pgfsetroundjoin%
\definecolor{currentfill}{rgb}{0.121569,0.466667,0.705882}%
\pgfsetfillcolor{currentfill}%
\pgfsetlinewidth{1.003750pt}%
\definecolor{currentstroke}{rgb}{0.121569,0.466667,0.705882}%
\pgfsetstrokecolor{currentstroke}%
\pgfsetdash{}{0pt}%
\pgfpathmoveto{\pgfqpoint{1.056670in}{3.126287in}}%
\pgfpathcurveto{\pgfqpoint{1.067720in}{3.126287in}}{\pgfqpoint{1.078319in}{3.130678in}}{\pgfqpoint{1.086133in}{3.138491in}}%
\pgfpathcurveto{\pgfqpoint{1.093947in}{3.146305in}}{\pgfqpoint{1.098337in}{3.156904in}}{\pgfqpoint{1.098337in}{3.167954in}}%
\pgfpathcurveto{\pgfqpoint{1.098337in}{3.179004in}}{\pgfqpoint{1.093947in}{3.189603in}}{\pgfqpoint{1.086133in}{3.197417in}}%
\pgfpathcurveto{\pgfqpoint{1.078319in}{3.205230in}}{\pgfqpoint{1.067720in}{3.209621in}}{\pgfqpoint{1.056670in}{3.209621in}}%
\pgfpathcurveto{\pgfqpoint{1.045620in}{3.209621in}}{\pgfqpoint{1.035021in}{3.205230in}}{\pgfqpoint{1.027208in}{3.197417in}}%
\pgfpathcurveto{\pgfqpoint{1.019394in}{3.189603in}}{\pgfqpoint{1.015004in}{3.179004in}}{\pgfqpoint{1.015004in}{3.167954in}}%
\pgfpathcurveto{\pgfqpoint{1.015004in}{3.156904in}}{\pgfqpoint{1.019394in}{3.146305in}}{\pgfqpoint{1.027208in}{3.138491in}}%
\pgfpathcurveto{\pgfqpoint{1.035021in}{3.130678in}}{\pgfqpoint{1.045620in}{3.126287in}}{\pgfqpoint{1.056670in}{3.126287in}}%
\pgfpathclose%
\pgfusepath{stroke,fill}%
\end{pgfscope}%
\begin{pgfscope}%
\pgfpathrectangle{\pgfqpoint{0.648703in}{0.548769in}}{\pgfqpoint{5.112893in}{3.102590in}}%
\pgfusepath{clip}%
\pgfsetbuttcap%
\pgfsetroundjoin%
\definecolor{currentfill}{rgb}{1.000000,0.498039,0.054902}%
\pgfsetfillcolor{currentfill}%
\pgfsetlinewidth{1.003750pt}%
\definecolor{currentstroke}{rgb}{1.000000,0.498039,0.054902}%
\pgfsetstrokecolor{currentstroke}%
\pgfsetdash{}{0pt}%
\pgfpathmoveto{\pgfqpoint{1.736886in}{3.142787in}}%
\pgfpathcurveto{\pgfqpoint{1.747936in}{3.142787in}}{\pgfqpoint{1.758535in}{3.147178in}}{\pgfqpoint{1.766349in}{3.154991in}}%
\pgfpathcurveto{\pgfqpoint{1.774163in}{3.162805in}}{\pgfqpoint{1.778553in}{3.173404in}}{\pgfqpoint{1.778553in}{3.184454in}}%
\pgfpathcurveto{\pgfqpoint{1.778553in}{3.195504in}}{\pgfqpoint{1.774163in}{3.206103in}}{\pgfqpoint{1.766349in}{3.213917in}}%
\pgfpathcurveto{\pgfqpoint{1.758535in}{3.221731in}}{\pgfqpoint{1.747936in}{3.226121in}}{\pgfqpoint{1.736886in}{3.226121in}}%
\pgfpathcurveto{\pgfqpoint{1.725836in}{3.226121in}}{\pgfqpoint{1.715237in}{3.221731in}}{\pgfqpoint{1.707423in}{3.213917in}}%
\pgfpathcurveto{\pgfqpoint{1.699610in}{3.206103in}}{\pgfqpoint{1.695220in}{3.195504in}}{\pgfqpoint{1.695220in}{3.184454in}}%
\pgfpathcurveto{\pgfqpoint{1.695220in}{3.173404in}}{\pgfqpoint{1.699610in}{3.162805in}}{\pgfqpoint{1.707423in}{3.154991in}}%
\pgfpathcurveto{\pgfqpoint{1.715237in}{3.147178in}}{\pgfqpoint{1.725836in}{3.142787in}}{\pgfqpoint{1.736886in}{3.142787in}}%
\pgfpathclose%
\pgfusepath{stroke,fill}%
\end{pgfscope}%
\begin{pgfscope}%
\pgfpathrectangle{\pgfqpoint{0.648703in}{0.548769in}}{\pgfqpoint{5.112893in}{3.102590in}}%
\pgfusepath{clip}%
\pgfsetbuttcap%
\pgfsetroundjoin%
\definecolor{currentfill}{rgb}{0.121569,0.466667,0.705882}%
\pgfsetfillcolor{currentfill}%
\pgfsetlinewidth{1.003750pt}%
\definecolor{currentstroke}{rgb}{0.121569,0.466667,0.705882}%
\pgfsetstrokecolor{currentstroke}%
\pgfsetdash{}{0pt}%
\pgfpathmoveto{\pgfqpoint{1.614862in}{3.134537in}}%
\pgfpathcurveto{\pgfqpoint{1.625913in}{3.134537in}}{\pgfqpoint{1.636512in}{3.138928in}}{\pgfqpoint{1.644325in}{3.146741in}}%
\pgfpathcurveto{\pgfqpoint{1.652139in}{3.154555in}}{\pgfqpoint{1.656529in}{3.165154in}}{\pgfqpoint{1.656529in}{3.176204in}}%
\pgfpathcurveto{\pgfqpoint{1.656529in}{3.187254in}}{\pgfqpoint{1.652139in}{3.197853in}}{\pgfqpoint{1.644325in}{3.205667in}}%
\pgfpathcurveto{\pgfqpoint{1.636512in}{3.213480in}}{\pgfqpoint{1.625913in}{3.217871in}}{\pgfqpoint{1.614862in}{3.217871in}}%
\pgfpathcurveto{\pgfqpoint{1.603812in}{3.217871in}}{\pgfqpoint{1.593213in}{3.213480in}}{\pgfqpoint{1.585400in}{3.205667in}}%
\pgfpathcurveto{\pgfqpoint{1.577586in}{3.197853in}}{\pgfqpoint{1.573196in}{3.187254in}}{\pgfqpoint{1.573196in}{3.176204in}}%
\pgfpathcurveto{\pgfqpoint{1.573196in}{3.165154in}}{\pgfqpoint{1.577586in}{3.154555in}}{\pgfqpoint{1.585400in}{3.146741in}}%
\pgfpathcurveto{\pgfqpoint{1.593213in}{3.138928in}}{\pgfqpoint{1.603812in}{3.134537in}}{\pgfqpoint{1.614862in}{3.134537in}}%
\pgfpathclose%
\pgfusepath{stroke,fill}%
\end{pgfscope}%
\begin{pgfscope}%
\pgfpathrectangle{\pgfqpoint{0.648703in}{0.548769in}}{\pgfqpoint{5.112893in}{3.102590in}}%
\pgfusepath{clip}%
\pgfsetbuttcap%
\pgfsetroundjoin%
\definecolor{currentfill}{rgb}{1.000000,0.498039,0.054902}%
\pgfsetfillcolor{currentfill}%
\pgfsetlinewidth{1.003750pt}%
\definecolor{currentstroke}{rgb}{1.000000,0.498039,0.054902}%
\pgfsetstrokecolor{currentstroke}%
\pgfsetdash{}{0pt}%
\pgfpathmoveto{\pgfqpoint{1.741118in}{3.138662in}}%
\pgfpathcurveto{\pgfqpoint{1.752168in}{3.138662in}}{\pgfqpoint{1.762767in}{3.143053in}}{\pgfqpoint{1.770581in}{3.150866in}}%
\pgfpathcurveto{\pgfqpoint{1.778395in}{3.158680in}}{\pgfqpoint{1.782785in}{3.169279in}}{\pgfqpoint{1.782785in}{3.180329in}}%
\pgfpathcurveto{\pgfqpoint{1.782785in}{3.191379in}}{\pgfqpoint{1.778395in}{3.201978in}}{\pgfqpoint{1.770581in}{3.209792in}}%
\pgfpathcurveto{\pgfqpoint{1.762767in}{3.217605in}}{\pgfqpoint{1.752168in}{3.221996in}}{\pgfqpoint{1.741118in}{3.221996in}}%
\pgfpathcurveto{\pgfqpoint{1.730068in}{3.221996in}}{\pgfqpoint{1.719469in}{3.217605in}}{\pgfqpoint{1.711655in}{3.209792in}}%
\pgfpathcurveto{\pgfqpoint{1.703842in}{3.201978in}}{\pgfqpoint{1.699452in}{3.191379in}}{\pgfqpoint{1.699452in}{3.180329in}}%
\pgfpathcurveto{\pgfqpoint{1.699452in}{3.169279in}}{\pgfqpoint{1.703842in}{3.158680in}}{\pgfqpoint{1.711655in}{3.150866in}}%
\pgfpathcurveto{\pgfqpoint{1.719469in}{3.143053in}}{\pgfqpoint{1.730068in}{3.138662in}}{\pgfqpoint{1.741118in}{3.138662in}}%
\pgfpathclose%
\pgfusepath{stroke,fill}%
\end{pgfscope}%
\begin{pgfscope}%
\pgfpathrectangle{\pgfqpoint{0.648703in}{0.548769in}}{\pgfqpoint{5.112893in}{3.102590in}}%
\pgfusepath{clip}%
\pgfsetbuttcap%
\pgfsetroundjoin%
\definecolor{currentfill}{rgb}{0.121569,0.466667,0.705882}%
\pgfsetfillcolor{currentfill}%
\pgfsetlinewidth{1.003750pt}%
\definecolor{currentstroke}{rgb}{0.121569,0.466667,0.705882}%
\pgfsetstrokecolor{currentstroke}%
\pgfsetdash{}{0pt}%
\pgfpathmoveto{\pgfqpoint{1.280694in}{3.134537in}}%
\pgfpathcurveto{\pgfqpoint{1.291744in}{3.134537in}}{\pgfqpoint{1.302343in}{3.138928in}}{\pgfqpoint{1.310156in}{3.146741in}}%
\pgfpathcurveto{\pgfqpoint{1.317970in}{3.154555in}}{\pgfqpoint{1.322360in}{3.165154in}}{\pgfqpoint{1.322360in}{3.176204in}}%
\pgfpathcurveto{\pgfqpoint{1.322360in}{3.187254in}}{\pgfqpoint{1.317970in}{3.197853in}}{\pgfqpoint{1.310156in}{3.205667in}}%
\pgfpathcurveto{\pgfqpoint{1.302343in}{3.213480in}}{\pgfqpoint{1.291744in}{3.217871in}}{\pgfqpoint{1.280694in}{3.217871in}}%
\pgfpathcurveto{\pgfqpoint{1.269643in}{3.217871in}}{\pgfqpoint{1.259044in}{3.213480in}}{\pgfqpoint{1.251231in}{3.205667in}}%
\pgfpathcurveto{\pgfqpoint{1.243417in}{3.197853in}}{\pgfqpoint{1.239027in}{3.187254in}}{\pgfqpoint{1.239027in}{3.176204in}}%
\pgfpathcurveto{\pgfqpoint{1.239027in}{3.165154in}}{\pgfqpoint{1.243417in}{3.154555in}}{\pgfqpoint{1.251231in}{3.146741in}}%
\pgfpathcurveto{\pgfqpoint{1.259044in}{3.138928in}}{\pgfqpoint{1.269643in}{3.134537in}}{\pgfqpoint{1.280694in}{3.134537in}}%
\pgfpathclose%
\pgfusepath{stroke,fill}%
\end{pgfscope}%
\begin{pgfscope}%
\pgfpathrectangle{\pgfqpoint{0.648703in}{0.548769in}}{\pgfqpoint{5.112893in}{3.102590in}}%
\pgfusepath{clip}%
\pgfsetbuttcap%
\pgfsetroundjoin%
\definecolor{currentfill}{rgb}{0.121569,0.466667,0.705882}%
\pgfsetfillcolor{currentfill}%
\pgfsetlinewidth{1.003750pt}%
\definecolor{currentstroke}{rgb}{0.121569,0.466667,0.705882}%
\pgfsetstrokecolor{currentstroke}%
\pgfsetdash{}{0pt}%
\pgfpathmoveto{\pgfqpoint{1.399474in}{3.130412in}}%
\pgfpathcurveto{\pgfqpoint{1.410524in}{3.130412in}}{\pgfqpoint{1.421123in}{3.134803in}}{\pgfqpoint{1.428937in}{3.142616in}}%
\pgfpathcurveto{\pgfqpoint{1.436751in}{3.150430in}}{\pgfqpoint{1.441141in}{3.161029in}}{\pgfqpoint{1.441141in}{3.172079in}}%
\pgfpathcurveto{\pgfqpoint{1.441141in}{3.183129in}}{\pgfqpoint{1.436751in}{3.193728in}}{\pgfqpoint{1.428937in}{3.201542in}}%
\pgfpathcurveto{\pgfqpoint{1.421123in}{3.209355in}}{\pgfqpoint{1.410524in}{3.213746in}}{\pgfqpoint{1.399474in}{3.213746in}}%
\pgfpathcurveto{\pgfqpoint{1.388424in}{3.213746in}}{\pgfqpoint{1.377825in}{3.209355in}}{\pgfqpoint{1.370012in}{3.201542in}}%
\pgfpathcurveto{\pgfqpoint{1.362198in}{3.193728in}}{\pgfqpoint{1.357808in}{3.183129in}}{\pgfqpoint{1.357808in}{3.172079in}}%
\pgfpathcurveto{\pgfqpoint{1.357808in}{3.161029in}}{\pgfqpoint{1.362198in}{3.150430in}}{\pgfqpoint{1.370012in}{3.142616in}}%
\pgfpathcurveto{\pgfqpoint{1.377825in}{3.134803in}}{\pgfqpoint{1.388424in}{3.130412in}}{\pgfqpoint{1.399474in}{3.130412in}}%
\pgfpathclose%
\pgfusepath{stroke,fill}%
\end{pgfscope}%
\begin{pgfscope}%
\pgfpathrectangle{\pgfqpoint{0.648703in}{0.548769in}}{\pgfqpoint{5.112893in}{3.102590in}}%
\pgfusepath{clip}%
\pgfsetbuttcap%
\pgfsetroundjoin%
\definecolor{currentfill}{rgb}{1.000000,0.498039,0.054902}%
\pgfsetfillcolor{currentfill}%
\pgfsetlinewidth{1.003750pt}%
\definecolor{currentstroke}{rgb}{1.000000,0.498039,0.054902}%
\pgfsetstrokecolor{currentstroke}%
\pgfsetdash{}{0pt}%
\pgfpathmoveto{\pgfqpoint{1.615320in}{3.151038in}}%
\pgfpathcurveto{\pgfqpoint{1.626370in}{3.151038in}}{\pgfqpoint{1.636969in}{3.155428in}}{\pgfqpoint{1.644783in}{3.163241in}}%
\pgfpathcurveto{\pgfqpoint{1.652596in}{3.171055in}}{\pgfqpoint{1.656986in}{3.181654in}}{\pgfqpoint{1.656986in}{3.192704in}}%
\pgfpathcurveto{\pgfqpoint{1.656986in}{3.203754in}}{\pgfqpoint{1.652596in}{3.214353in}}{\pgfqpoint{1.644783in}{3.222167in}}%
\pgfpathcurveto{\pgfqpoint{1.636969in}{3.229981in}}{\pgfqpoint{1.626370in}{3.234371in}}{\pgfqpoint{1.615320in}{3.234371in}}%
\pgfpathcurveto{\pgfqpoint{1.604270in}{3.234371in}}{\pgfqpoint{1.593671in}{3.229981in}}{\pgfqpoint{1.585857in}{3.222167in}}%
\pgfpathcurveto{\pgfqpoint{1.578043in}{3.214353in}}{\pgfqpoint{1.573653in}{3.203754in}}{\pgfqpoint{1.573653in}{3.192704in}}%
\pgfpathcurveto{\pgfqpoint{1.573653in}{3.181654in}}{\pgfqpoint{1.578043in}{3.171055in}}{\pgfqpoint{1.585857in}{3.163241in}}%
\pgfpathcurveto{\pgfqpoint{1.593671in}{3.155428in}}{\pgfqpoint{1.604270in}{3.151038in}}{\pgfqpoint{1.615320in}{3.151038in}}%
\pgfpathclose%
\pgfusepath{stroke,fill}%
\end{pgfscope}%
\begin{pgfscope}%
\pgfpathrectangle{\pgfqpoint{0.648703in}{0.548769in}}{\pgfqpoint{5.112893in}{3.102590in}}%
\pgfusepath{clip}%
\pgfsetbuttcap%
\pgfsetroundjoin%
\definecolor{currentfill}{rgb}{1.000000,0.498039,0.054902}%
\pgfsetfillcolor{currentfill}%
\pgfsetlinewidth{1.003750pt}%
\definecolor{currentstroke}{rgb}{1.000000,0.498039,0.054902}%
\pgfsetstrokecolor{currentstroke}%
\pgfsetdash{}{0pt}%
\pgfpathmoveto{\pgfqpoint{1.538548in}{3.241788in}}%
\pgfpathcurveto{\pgfqpoint{1.549598in}{3.241788in}}{\pgfqpoint{1.560197in}{3.246179in}}{\pgfqpoint{1.568010in}{3.253992in}}%
\pgfpathcurveto{\pgfqpoint{1.575824in}{3.261806in}}{\pgfqpoint{1.580214in}{3.272405in}}{\pgfqpoint{1.580214in}{3.283455in}}%
\pgfpathcurveto{\pgfqpoint{1.580214in}{3.294505in}}{\pgfqpoint{1.575824in}{3.305104in}}{\pgfqpoint{1.568010in}{3.312918in}}%
\pgfpathcurveto{\pgfqpoint{1.560197in}{3.320731in}}{\pgfqpoint{1.549598in}{3.325122in}}{\pgfqpoint{1.538548in}{3.325122in}}%
\pgfpathcurveto{\pgfqpoint{1.527498in}{3.325122in}}{\pgfqpoint{1.516899in}{3.320731in}}{\pgfqpoint{1.509085in}{3.312918in}}%
\pgfpathcurveto{\pgfqpoint{1.501271in}{3.305104in}}{\pgfqpoint{1.496881in}{3.294505in}}{\pgfqpoint{1.496881in}{3.283455in}}%
\pgfpathcurveto{\pgfqpoint{1.496881in}{3.272405in}}{\pgfqpoint{1.501271in}{3.261806in}}{\pgfqpoint{1.509085in}{3.253992in}}%
\pgfpathcurveto{\pgfqpoint{1.516899in}{3.246179in}}{\pgfqpoint{1.527498in}{3.241788in}}{\pgfqpoint{1.538548in}{3.241788in}}%
\pgfpathclose%
\pgfusepath{stroke,fill}%
\end{pgfscope}%
\begin{pgfscope}%
\pgfpathrectangle{\pgfqpoint{0.648703in}{0.548769in}}{\pgfqpoint{5.112893in}{3.102590in}}%
\pgfusepath{clip}%
\pgfsetbuttcap%
\pgfsetroundjoin%
\definecolor{currentfill}{rgb}{0.121569,0.466667,0.705882}%
\pgfsetfillcolor{currentfill}%
\pgfsetlinewidth{1.003750pt}%
\definecolor{currentstroke}{rgb}{0.121569,0.466667,0.705882}%
\pgfsetstrokecolor{currentstroke}%
\pgfsetdash{}{0pt}%
\pgfpathmoveto{\pgfqpoint{0.725464in}{0.680142in}}%
\pgfpathcurveto{\pgfqpoint{0.736514in}{0.680142in}}{\pgfqpoint{0.747113in}{0.684532in}}{\pgfqpoint{0.754927in}{0.692346in}}%
\pgfpathcurveto{\pgfqpoint{0.762740in}{0.700160in}}{\pgfqpoint{0.767131in}{0.710759in}}{\pgfqpoint{0.767131in}{0.721809in}}%
\pgfpathcurveto{\pgfqpoint{0.767131in}{0.732859in}}{\pgfqpoint{0.762740in}{0.743458in}}{\pgfqpoint{0.754927in}{0.751272in}}%
\pgfpathcurveto{\pgfqpoint{0.747113in}{0.759085in}}{\pgfqpoint{0.736514in}{0.763475in}}{\pgfqpoint{0.725464in}{0.763475in}}%
\pgfpathcurveto{\pgfqpoint{0.714414in}{0.763475in}}{\pgfqpoint{0.703815in}{0.759085in}}{\pgfqpoint{0.696001in}{0.751272in}}%
\pgfpathcurveto{\pgfqpoint{0.688187in}{0.743458in}}{\pgfqpoint{0.683797in}{0.732859in}}{\pgfqpoint{0.683797in}{0.721809in}}%
\pgfpathcurveto{\pgfqpoint{0.683797in}{0.710759in}}{\pgfqpoint{0.688187in}{0.700160in}}{\pgfqpoint{0.696001in}{0.692346in}}%
\pgfpathcurveto{\pgfqpoint{0.703815in}{0.684532in}}{\pgfqpoint{0.714414in}{0.680142in}}{\pgfqpoint{0.725464in}{0.680142in}}%
\pgfpathclose%
\pgfusepath{stroke,fill}%
\end{pgfscope}%
\begin{pgfscope}%
\pgfpathrectangle{\pgfqpoint{0.648703in}{0.548769in}}{\pgfqpoint{5.112893in}{3.102590in}}%
\pgfusepath{clip}%
\pgfsetbuttcap%
\pgfsetroundjoin%
\definecolor{currentfill}{rgb}{1.000000,0.498039,0.054902}%
\pgfsetfillcolor{currentfill}%
\pgfsetlinewidth{1.003750pt}%
\definecolor{currentstroke}{rgb}{1.000000,0.498039,0.054902}%
\pgfsetstrokecolor{currentstroke}%
\pgfsetdash{}{0pt}%
\pgfpathmoveto{\pgfqpoint{1.577691in}{3.159288in}}%
\pgfpathcurveto{\pgfqpoint{1.588741in}{3.159288in}}{\pgfqpoint{1.599340in}{3.163678in}}{\pgfqpoint{1.607154in}{3.171491in}}%
\pgfpathcurveto{\pgfqpoint{1.614967in}{3.179305in}}{\pgfqpoint{1.619358in}{3.189904in}}{\pgfqpoint{1.619358in}{3.200954in}}%
\pgfpathcurveto{\pgfqpoint{1.619358in}{3.212004in}}{\pgfqpoint{1.614967in}{3.222603in}}{\pgfqpoint{1.607154in}{3.230417in}}%
\pgfpathcurveto{\pgfqpoint{1.599340in}{3.238231in}}{\pgfqpoint{1.588741in}{3.242621in}}{\pgfqpoint{1.577691in}{3.242621in}}%
\pgfpathcurveto{\pgfqpoint{1.566641in}{3.242621in}}{\pgfqpoint{1.556042in}{3.238231in}}{\pgfqpoint{1.548228in}{3.230417in}}%
\pgfpathcurveto{\pgfqpoint{1.540415in}{3.222603in}}{\pgfqpoint{1.536024in}{3.212004in}}{\pgfqpoint{1.536024in}{3.200954in}}%
\pgfpathcurveto{\pgfqpoint{1.536024in}{3.189904in}}{\pgfqpoint{1.540415in}{3.179305in}}{\pgfqpoint{1.548228in}{3.171491in}}%
\pgfpathcurveto{\pgfqpoint{1.556042in}{3.163678in}}{\pgfqpoint{1.566641in}{3.159288in}}{\pgfqpoint{1.577691in}{3.159288in}}%
\pgfpathclose%
\pgfusepath{stroke,fill}%
\end{pgfscope}%
\begin{pgfscope}%
\pgfpathrectangle{\pgfqpoint{0.648703in}{0.548769in}}{\pgfqpoint{5.112893in}{3.102590in}}%
\pgfusepath{clip}%
\pgfsetbuttcap%
\pgfsetroundjoin%
\definecolor{currentfill}{rgb}{1.000000,0.498039,0.054902}%
\pgfsetfillcolor{currentfill}%
\pgfsetlinewidth{1.003750pt}%
\definecolor{currentstroke}{rgb}{1.000000,0.498039,0.054902}%
\pgfsetstrokecolor{currentstroke}%
\pgfsetdash{}{0pt}%
\pgfpathmoveto{\pgfqpoint{1.485329in}{3.142787in}}%
\pgfpathcurveto{\pgfqpoint{1.496379in}{3.142787in}}{\pgfqpoint{1.506978in}{3.147178in}}{\pgfqpoint{1.514792in}{3.154991in}}%
\pgfpathcurveto{\pgfqpoint{1.522605in}{3.162805in}}{\pgfqpoint{1.526996in}{3.173404in}}{\pgfqpoint{1.526996in}{3.184454in}}%
\pgfpathcurveto{\pgfqpoint{1.526996in}{3.195504in}}{\pgfqpoint{1.522605in}{3.206103in}}{\pgfqpoint{1.514792in}{3.213917in}}%
\pgfpathcurveto{\pgfqpoint{1.506978in}{3.221731in}}{\pgfqpoint{1.496379in}{3.226121in}}{\pgfqpoint{1.485329in}{3.226121in}}%
\pgfpathcurveto{\pgfqpoint{1.474279in}{3.226121in}}{\pgfqpoint{1.463680in}{3.221731in}}{\pgfqpoint{1.455866in}{3.213917in}}%
\pgfpathcurveto{\pgfqpoint{1.448052in}{3.206103in}}{\pgfqpoint{1.443662in}{3.195504in}}{\pgfqpoint{1.443662in}{3.184454in}}%
\pgfpathcurveto{\pgfqpoint{1.443662in}{3.173404in}}{\pgfqpoint{1.448052in}{3.162805in}}{\pgfqpoint{1.455866in}{3.154991in}}%
\pgfpathcurveto{\pgfqpoint{1.463680in}{3.147178in}}{\pgfqpoint{1.474279in}{3.142787in}}{\pgfqpoint{1.485329in}{3.142787in}}%
\pgfpathclose%
\pgfusepath{stroke,fill}%
\end{pgfscope}%
\begin{pgfscope}%
\pgfpathrectangle{\pgfqpoint{0.648703in}{0.548769in}}{\pgfqpoint{5.112893in}{3.102590in}}%
\pgfusepath{clip}%
\pgfsetbuttcap%
\pgfsetroundjoin%
\definecolor{currentfill}{rgb}{1.000000,0.498039,0.054902}%
\pgfsetfillcolor{currentfill}%
\pgfsetlinewidth{1.003750pt}%
\definecolor{currentstroke}{rgb}{1.000000,0.498039,0.054902}%
\pgfsetstrokecolor{currentstroke}%
\pgfsetdash{}{0pt}%
\pgfpathmoveto{\pgfqpoint{1.659009in}{3.138662in}}%
\pgfpathcurveto{\pgfqpoint{1.670059in}{3.138662in}}{\pgfqpoint{1.680658in}{3.143053in}}{\pgfqpoint{1.688472in}{3.150866in}}%
\pgfpathcurveto{\pgfqpoint{1.696285in}{3.158680in}}{\pgfqpoint{1.700676in}{3.169279in}}{\pgfqpoint{1.700676in}{3.180329in}}%
\pgfpathcurveto{\pgfqpoint{1.700676in}{3.191379in}}{\pgfqpoint{1.696285in}{3.201978in}}{\pgfqpoint{1.688472in}{3.209792in}}%
\pgfpathcurveto{\pgfqpoint{1.680658in}{3.217605in}}{\pgfqpoint{1.670059in}{3.221996in}}{\pgfqpoint{1.659009in}{3.221996in}}%
\pgfpathcurveto{\pgfqpoint{1.647959in}{3.221996in}}{\pgfqpoint{1.637360in}{3.217605in}}{\pgfqpoint{1.629546in}{3.209792in}}%
\pgfpathcurveto{\pgfqpoint{1.621733in}{3.201978in}}{\pgfqpoint{1.617342in}{3.191379in}}{\pgfqpoint{1.617342in}{3.180329in}}%
\pgfpathcurveto{\pgfqpoint{1.617342in}{3.169279in}}{\pgfqpoint{1.621733in}{3.158680in}}{\pgfqpoint{1.629546in}{3.150866in}}%
\pgfpathcurveto{\pgfqpoint{1.637360in}{3.143053in}}{\pgfqpoint{1.647959in}{3.138662in}}{\pgfqpoint{1.659009in}{3.138662in}}%
\pgfpathclose%
\pgfusepath{stroke,fill}%
\end{pgfscope}%
\begin{pgfscope}%
\pgfpathrectangle{\pgfqpoint{0.648703in}{0.548769in}}{\pgfqpoint{5.112893in}{3.102590in}}%
\pgfusepath{clip}%
\pgfsetbuttcap%
\pgfsetroundjoin%
\definecolor{currentfill}{rgb}{1.000000,0.498039,0.054902}%
\pgfsetfillcolor{currentfill}%
\pgfsetlinewidth{1.003750pt}%
\definecolor{currentstroke}{rgb}{1.000000,0.498039,0.054902}%
\pgfsetstrokecolor{currentstroke}%
\pgfsetdash{}{0pt}%
\pgfpathmoveto{\pgfqpoint{1.573350in}{3.138662in}}%
\pgfpathcurveto{\pgfqpoint{1.584401in}{3.138662in}}{\pgfqpoint{1.595000in}{3.143053in}}{\pgfqpoint{1.602813in}{3.150866in}}%
\pgfpathcurveto{\pgfqpoint{1.610627in}{3.158680in}}{\pgfqpoint{1.615017in}{3.169279in}}{\pgfqpoint{1.615017in}{3.180329in}}%
\pgfpathcurveto{\pgfqpoint{1.615017in}{3.191379in}}{\pgfqpoint{1.610627in}{3.201978in}}{\pgfqpoint{1.602813in}{3.209792in}}%
\pgfpathcurveto{\pgfqpoint{1.595000in}{3.217605in}}{\pgfqpoint{1.584401in}{3.221996in}}{\pgfqpoint{1.573350in}{3.221996in}}%
\pgfpathcurveto{\pgfqpoint{1.562300in}{3.221996in}}{\pgfqpoint{1.551701in}{3.217605in}}{\pgfqpoint{1.543888in}{3.209792in}}%
\pgfpathcurveto{\pgfqpoint{1.536074in}{3.201978in}}{\pgfqpoint{1.531684in}{3.191379in}}{\pgfqpoint{1.531684in}{3.180329in}}%
\pgfpathcurveto{\pgfqpoint{1.531684in}{3.169279in}}{\pgfqpoint{1.536074in}{3.158680in}}{\pgfqpoint{1.543888in}{3.150866in}}%
\pgfpathcurveto{\pgfqpoint{1.551701in}{3.143053in}}{\pgfqpoint{1.562300in}{3.138662in}}{\pgfqpoint{1.573350in}{3.138662in}}%
\pgfpathclose%
\pgfusepath{stroke,fill}%
\end{pgfscope}%
\begin{pgfscope}%
\pgfpathrectangle{\pgfqpoint{0.648703in}{0.548769in}}{\pgfqpoint{5.112893in}{3.102590in}}%
\pgfusepath{clip}%
\pgfsetbuttcap%
\pgfsetroundjoin%
\definecolor{currentfill}{rgb}{1.000000,0.498039,0.054902}%
\pgfsetfillcolor{currentfill}%
\pgfsetlinewidth{1.003750pt}%
\definecolor{currentstroke}{rgb}{1.000000,0.498039,0.054902}%
\pgfsetstrokecolor{currentstroke}%
\pgfsetdash{}{0pt}%
\pgfpathmoveto{\pgfqpoint{1.114589in}{3.138662in}}%
\pgfpathcurveto{\pgfqpoint{1.125640in}{3.138662in}}{\pgfqpoint{1.136239in}{3.143053in}}{\pgfqpoint{1.144052in}{3.150866in}}%
\pgfpathcurveto{\pgfqpoint{1.151866in}{3.158680in}}{\pgfqpoint{1.156256in}{3.169279in}}{\pgfqpoint{1.156256in}{3.180329in}}%
\pgfpathcurveto{\pgfqpoint{1.156256in}{3.191379in}}{\pgfqpoint{1.151866in}{3.201978in}}{\pgfqpoint{1.144052in}{3.209792in}}%
\pgfpathcurveto{\pgfqpoint{1.136239in}{3.217605in}}{\pgfqpoint{1.125640in}{3.221996in}}{\pgfqpoint{1.114589in}{3.221996in}}%
\pgfpathcurveto{\pgfqpoint{1.103539in}{3.221996in}}{\pgfqpoint{1.092940in}{3.217605in}}{\pgfqpoint{1.085127in}{3.209792in}}%
\pgfpathcurveto{\pgfqpoint{1.077313in}{3.201978in}}{\pgfqpoint{1.072923in}{3.191379in}}{\pgfqpoint{1.072923in}{3.180329in}}%
\pgfpathcurveto{\pgfqpoint{1.072923in}{3.169279in}}{\pgfqpoint{1.077313in}{3.158680in}}{\pgfqpoint{1.085127in}{3.150866in}}%
\pgfpathcurveto{\pgfqpoint{1.092940in}{3.143053in}}{\pgfqpoint{1.103539in}{3.138662in}}{\pgfqpoint{1.114589in}{3.138662in}}%
\pgfpathclose%
\pgfusepath{stroke,fill}%
\end{pgfscope}%
\begin{pgfscope}%
\pgfpathrectangle{\pgfqpoint{0.648703in}{0.548769in}}{\pgfqpoint{5.112893in}{3.102590in}}%
\pgfusepath{clip}%
\pgfsetbuttcap%
\pgfsetroundjoin%
\definecolor{currentfill}{rgb}{1.000000,0.498039,0.054902}%
\pgfsetfillcolor{currentfill}%
\pgfsetlinewidth{1.003750pt}%
\definecolor{currentstroke}{rgb}{1.000000,0.498039,0.054902}%
\pgfsetstrokecolor{currentstroke}%
\pgfsetdash{}{0pt}%
\pgfpathmoveto{\pgfqpoint{1.024028in}{3.138662in}}%
\pgfpathcurveto{\pgfqpoint{1.035079in}{3.138662in}}{\pgfqpoint{1.045678in}{3.143053in}}{\pgfqpoint{1.053491in}{3.150866in}}%
\pgfpathcurveto{\pgfqpoint{1.061305in}{3.158680in}}{\pgfqpoint{1.065695in}{3.169279in}}{\pgfqpoint{1.065695in}{3.180329in}}%
\pgfpathcurveto{\pgfqpoint{1.065695in}{3.191379in}}{\pgfqpoint{1.061305in}{3.201978in}}{\pgfqpoint{1.053491in}{3.209792in}}%
\pgfpathcurveto{\pgfqpoint{1.045678in}{3.217605in}}{\pgfqpoint{1.035079in}{3.221996in}}{\pgfqpoint{1.024028in}{3.221996in}}%
\pgfpathcurveto{\pgfqpoint{1.012978in}{3.221996in}}{\pgfqpoint{1.002379in}{3.217605in}}{\pgfqpoint{0.994566in}{3.209792in}}%
\pgfpathcurveto{\pgfqpoint{0.986752in}{3.201978in}}{\pgfqpoint{0.982362in}{3.191379in}}{\pgfqpoint{0.982362in}{3.180329in}}%
\pgfpathcurveto{\pgfqpoint{0.982362in}{3.169279in}}{\pgfqpoint{0.986752in}{3.158680in}}{\pgfqpoint{0.994566in}{3.150866in}}%
\pgfpathcurveto{\pgfqpoint{1.002379in}{3.143053in}}{\pgfqpoint{1.012978in}{3.138662in}}{\pgfqpoint{1.024028in}{3.138662in}}%
\pgfpathclose%
\pgfusepath{stroke,fill}%
\end{pgfscope}%
\begin{pgfscope}%
\pgfpathrectangle{\pgfqpoint{0.648703in}{0.548769in}}{\pgfqpoint{5.112893in}{3.102590in}}%
\pgfusepath{clip}%
\pgfsetbuttcap%
\pgfsetroundjoin%
\definecolor{currentfill}{rgb}{1.000000,0.498039,0.054902}%
\pgfsetfillcolor{currentfill}%
\pgfsetlinewidth{1.003750pt}%
\definecolor{currentstroke}{rgb}{1.000000,0.498039,0.054902}%
\pgfsetstrokecolor{currentstroke}%
\pgfsetdash{}{0pt}%
\pgfpathmoveto{\pgfqpoint{1.507191in}{3.138662in}}%
\pgfpathcurveto{\pgfqpoint{1.518241in}{3.138662in}}{\pgfqpoint{1.528840in}{3.143053in}}{\pgfqpoint{1.536653in}{3.150866in}}%
\pgfpathcurveto{\pgfqpoint{1.544467in}{3.158680in}}{\pgfqpoint{1.548857in}{3.169279in}}{\pgfqpoint{1.548857in}{3.180329in}}%
\pgfpathcurveto{\pgfqpoint{1.548857in}{3.191379in}}{\pgfqpoint{1.544467in}{3.201978in}}{\pgfqpoint{1.536653in}{3.209792in}}%
\pgfpathcurveto{\pgfqpoint{1.528840in}{3.217605in}}{\pgfqpoint{1.518241in}{3.221996in}}{\pgfqpoint{1.507191in}{3.221996in}}%
\pgfpathcurveto{\pgfqpoint{1.496140in}{3.221996in}}{\pgfqpoint{1.485541in}{3.217605in}}{\pgfqpoint{1.477728in}{3.209792in}}%
\pgfpathcurveto{\pgfqpoint{1.469914in}{3.201978in}}{\pgfqpoint{1.465524in}{3.191379in}}{\pgfqpoint{1.465524in}{3.180329in}}%
\pgfpathcurveto{\pgfqpoint{1.465524in}{3.169279in}}{\pgfqpoint{1.469914in}{3.158680in}}{\pgfqpoint{1.477728in}{3.150866in}}%
\pgfpathcurveto{\pgfqpoint{1.485541in}{3.143053in}}{\pgfqpoint{1.496140in}{3.138662in}}{\pgfqpoint{1.507191in}{3.138662in}}%
\pgfpathclose%
\pgfusepath{stroke,fill}%
\end{pgfscope}%
\begin{pgfscope}%
\pgfpathrectangle{\pgfqpoint{0.648703in}{0.548769in}}{\pgfqpoint{5.112893in}{3.102590in}}%
\pgfusepath{clip}%
\pgfsetbuttcap%
\pgfsetroundjoin%
\definecolor{currentfill}{rgb}{0.121569,0.466667,0.705882}%
\pgfsetfillcolor{currentfill}%
\pgfsetlinewidth{1.003750pt}%
\definecolor{currentstroke}{rgb}{0.121569,0.466667,0.705882}%
\pgfsetstrokecolor{currentstroke}%
\pgfsetdash{}{0pt}%
\pgfpathmoveto{\pgfqpoint{1.456631in}{2.849910in}}%
\pgfpathcurveto{\pgfqpoint{1.467681in}{2.849910in}}{\pgfqpoint{1.478280in}{2.854300in}}{\pgfqpoint{1.486094in}{2.862114in}}%
\pgfpathcurveto{\pgfqpoint{1.493908in}{2.869928in}}{\pgfqpoint{1.498298in}{2.880527in}}{\pgfqpoint{1.498298in}{2.891577in}}%
\pgfpathcurveto{\pgfqpoint{1.498298in}{2.902627in}}{\pgfqpoint{1.493908in}{2.913226in}}{\pgfqpoint{1.486094in}{2.921039in}}%
\pgfpathcurveto{\pgfqpoint{1.478280in}{2.928853in}}{\pgfqpoint{1.467681in}{2.933243in}}{\pgfqpoint{1.456631in}{2.933243in}}%
\pgfpathcurveto{\pgfqpoint{1.445581in}{2.933243in}}{\pgfqpoint{1.434982in}{2.928853in}}{\pgfqpoint{1.427169in}{2.921039in}}%
\pgfpathcurveto{\pgfqpoint{1.419355in}{2.913226in}}{\pgfqpoint{1.414965in}{2.902627in}}{\pgfqpoint{1.414965in}{2.891577in}}%
\pgfpathcurveto{\pgfqpoint{1.414965in}{2.880527in}}{\pgfqpoint{1.419355in}{2.869928in}}{\pgfqpoint{1.427169in}{2.862114in}}%
\pgfpathcurveto{\pgfqpoint{1.434982in}{2.854300in}}{\pgfqpoint{1.445581in}{2.849910in}}{\pgfqpoint{1.456631in}{2.849910in}}%
\pgfpathclose%
\pgfusepath{stroke,fill}%
\end{pgfscope}%
\begin{pgfscope}%
\pgfpathrectangle{\pgfqpoint{0.648703in}{0.548769in}}{\pgfqpoint{5.112893in}{3.102590in}}%
\pgfusepath{clip}%
\pgfsetbuttcap%
\pgfsetroundjoin%
\definecolor{currentfill}{rgb}{0.121569,0.466667,0.705882}%
\pgfsetfillcolor{currentfill}%
\pgfsetlinewidth{1.003750pt}%
\definecolor{currentstroke}{rgb}{0.121569,0.466667,0.705882}%
\pgfsetstrokecolor{currentstroke}%
\pgfsetdash{}{0pt}%
\pgfpathmoveto{\pgfqpoint{0.961166in}{3.076787in}}%
\pgfpathcurveto{\pgfqpoint{0.972216in}{3.076787in}}{\pgfqpoint{0.982815in}{3.081177in}}{\pgfqpoint{0.990629in}{3.088991in}}%
\pgfpathcurveto{\pgfqpoint{0.998443in}{3.096804in}}{\pgfqpoint{1.002833in}{3.107403in}}{\pgfqpoint{1.002833in}{3.118454in}}%
\pgfpathcurveto{\pgfqpoint{1.002833in}{3.129504in}}{\pgfqpoint{0.998443in}{3.140103in}}{\pgfqpoint{0.990629in}{3.147916in}}%
\pgfpathcurveto{\pgfqpoint{0.982815in}{3.155730in}}{\pgfqpoint{0.972216in}{3.160120in}}{\pgfqpoint{0.961166in}{3.160120in}}%
\pgfpathcurveto{\pgfqpoint{0.950116in}{3.160120in}}{\pgfqpoint{0.939517in}{3.155730in}}{\pgfqpoint{0.931704in}{3.147916in}}%
\pgfpathcurveto{\pgfqpoint{0.923890in}{3.140103in}}{\pgfqpoint{0.919500in}{3.129504in}}{\pgfqpoint{0.919500in}{3.118454in}}%
\pgfpathcurveto{\pgfqpoint{0.919500in}{3.107403in}}{\pgfqpoint{0.923890in}{3.096804in}}{\pgfqpoint{0.931704in}{3.088991in}}%
\pgfpathcurveto{\pgfqpoint{0.939517in}{3.081177in}}{\pgfqpoint{0.950116in}{3.076787in}}{\pgfqpoint{0.961166in}{3.076787in}}%
\pgfpathclose%
\pgfusepath{stroke,fill}%
\end{pgfscope}%
\begin{pgfscope}%
\pgfpathrectangle{\pgfqpoint{0.648703in}{0.548769in}}{\pgfqpoint{5.112893in}{3.102590in}}%
\pgfusepath{clip}%
\pgfsetbuttcap%
\pgfsetroundjoin%
\definecolor{currentfill}{rgb}{1.000000,0.498039,0.054902}%
\pgfsetfillcolor{currentfill}%
\pgfsetlinewidth{1.003750pt}%
\definecolor{currentstroke}{rgb}{1.000000,0.498039,0.054902}%
\pgfsetstrokecolor{currentstroke}%
\pgfsetdash{}{0pt}%
\pgfpathmoveto{\pgfqpoint{1.412435in}{3.316039in}}%
\pgfpathcurveto{\pgfqpoint{1.423485in}{3.316039in}}{\pgfqpoint{1.434084in}{3.320429in}}{\pgfqpoint{1.441898in}{3.328243in}}%
\pgfpathcurveto{\pgfqpoint{1.449711in}{3.336056in}}{\pgfqpoint{1.454102in}{3.346655in}}{\pgfqpoint{1.454102in}{3.357706in}}%
\pgfpathcurveto{\pgfqpoint{1.454102in}{3.368756in}}{\pgfqpoint{1.449711in}{3.379355in}}{\pgfqpoint{1.441898in}{3.387168in}}%
\pgfpathcurveto{\pgfqpoint{1.434084in}{3.394982in}}{\pgfqpoint{1.423485in}{3.399372in}}{\pgfqpoint{1.412435in}{3.399372in}}%
\pgfpathcurveto{\pgfqpoint{1.401385in}{3.399372in}}{\pgfqpoint{1.390786in}{3.394982in}}{\pgfqpoint{1.382972in}{3.387168in}}%
\pgfpathcurveto{\pgfqpoint{1.375159in}{3.379355in}}{\pgfqpoint{1.370768in}{3.368756in}}{\pgfqpoint{1.370768in}{3.357706in}}%
\pgfpathcurveto{\pgfqpoint{1.370768in}{3.346655in}}{\pgfqpoint{1.375159in}{3.336056in}}{\pgfqpoint{1.382972in}{3.328243in}}%
\pgfpathcurveto{\pgfqpoint{1.390786in}{3.320429in}}{\pgfqpoint{1.401385in}{3.316039in}}{\pgfqpoint{1.412435in}{3.316039in}}%
\pgfpathclose%
\pgfusepath{stroke,fill}%
\end{pgfscope}%
\begin{pgfscope}%
\pgfpathrectangle{\pgfqpoint{0.648703in}{0.548769in}}{\pgfqpoint{5.112893in}{3.102590in}}%
\pgfusepath{clip}%
\pgfsetbuttcap%
\pgfsetroundjoin%
\definecolor{currentfill}{rgb}{0.121569,0.466667,0.705882}%
\pgfsetfillcolor{currentfill}%
\pgfsetlinewidth{1.003750pt}%
\definecolor{currentstroke}{rgb}{0.121569,0.466667,0.705882}%
\pgfsetstrokecolor{currentstroke}%
\pgfsetdash{}{0pt}%
\pgfpathmoveto{\pgfqpoint{0.719455in}{0.663642in}}%
\pgfpathcurveto{\pgfqpoint{0.730505in}{0.663642in}}{\pgfqpoint{0.741104in}{0.668032in}}{\pgfqpoint{0.748918in}{0.675846in}}%
\pgfpathcurveto{\pgfqpoint{0.756732in}{0.683659in}}{\pgfqpoint{0.761122in}{0.694258in}}{\pgfqpoint{0.761122in}{0.705309in}}%
\pgfpathcurveto{\pgfqpoint{0.761122in}{0.716359in}}{\pgfqpoint{0.756732in}{0.726958in}}{\pgfqpoint{0.748918in}{0.734771in}}%
\pgfpathcurveto{\pgfqpoint{0.741104in}{0.742585in}}{\pgfqpoint{0.730505in}{0.746975in}}{\pgfqpoint{0.719455in}{0.746975in}}%
\pgfpathcurveto{\pgfqpoint{0.708405in}{0.746975in}}{\pgfqpoint{0.697806in}{0.742585in}}{\pgfqpoint{0.689993in}{0.734771in}}%
\pgfpathcurveto{\pgfqpoint{0.682179in}{0.726958in}}{\pgfqpoint{0.677789in}{0.716359in}}{\pgfqpoint{0.677789in}{0.705309in}}%
\pgfpathcurveto{\pgfqpoint{0.677789in}{0.694258in}}{\pgfqpoint{0.682179in}{0.683659in}}{\pgfqpoint{0.689993in}{0.675846in}}%
\pgfpathcurveto{\pgfqpoint{0.697806in}{0.668032in}}{\pgfqpoint{0.708405in}{0.663642in}}{\pgfqpoint{0.719455in}{0.663642in}}%
\pgfpathclose%
\pgfusepath{stroke,fill}%
\end{pgfscope}%
\begin{pgfscope}%
\pgfpathrectangle{\pgfqpoint{0.648703in}{0.548769in}}{\pgfqpoint{5.112893in}{3.102590in}}%
\pgfusepath{clip}%
\pgfsetbuttcap%
\pgfsetroundjoin%
\definecolor{currentfill}{rgb}{0.121569,0.466667,0.705882}%
\pgfsetfillcolor{currentfill}%
\pgfsetlinewidth{1.003750pt}%
\definecolor{currentstroke}{rgb}{0.121569,0.466667,0.705882}%
\pgfsetstrokecolor{currentstroke}%
\pgfsetdash{}{0pt}%
\pgfpathmoveto{\pgfqpoint{0.985857in}{1.179271in}}%
\pgfpathcurveto{\pgfqpoint{0.996907in}{1.179271in}}{\pgfqpoint{1.007506in}{1.183661in}}{\pgfqpoint{1.015320in}{1.191475in}}%
\pgfpathcurveto{\pgfqpoint{1.023133in}{1.199289in}}{\pgfqpoint{1.027524in}{1.209888in}}{\pgfqpoint{1.027524in}{1.220938in}}%
\pgfpathcurveto{\pgfqpoint{1.027524in}{1.231988in}}{\pgfqpoint{1.023133in}{1.242587in}}{\pgfqpoint{1.015320in}{1.250401in}}%
\pgfpathcurveto{\pgfqpoint{1.007506in}{1.258214in}}{\pgfqpoint{0.996907in}{1.262605in}}{\pgfqpoint{0.985857in}{1.262605in}}%
\pgfpathcurveto{\pgfqpoint{0.974807in}{1.262605in}}{\pgfqpoint{0.964208in}{1.258214in}}{\pgfqpoint{0.956394in}{1.250401in}}%
\pgfpathcurveto{\pgfqpoint{0.948581in}{1.242587in}}{\pgfqpoint{0.944190in}{1.231988in}}{\pgfqpoint{0.944190in}{1.220938in}}%
\pgfpathcurveto{\pgfqpoint{0.944190in}{1.209888in}}{\pgfqpoint{0.948581in}{1.199289in}}{\pgfqpoint{0.956394in}{1.191475in}}%
\pgfpathcurveto{\pgfqpoint{0.964208in}{1.183661in}}{\pgfqpoint{0.974807in}{1.179271in}}{\pgfqpoint{0.985857in}{1.179271in}}%
\pgfpathclose%
\pgfusepath{stroke,fill}%
\end{pgfscope}%
\begin{pgfscope}%
\pgfpathrectangle{\pgfqpoint{0.648703in}{0.548769in}}{\pgfqpoint{5.112893in}{3.102590in}}%
\pgfusepath{clip}%
\pgfsetbuttcap%
\pgfsetroundjoin%
\definecolor{currentfill}{rgb}{1.000000,0.498039,0.054902}%
\pgfsetfillcolor{currentfill}%
\pgfsetlinewidth{1.003750pt}%
\definecolor{currentstroke}{rgb}{1.000000,0.498039,0.054902}%
\pgfsetstrokecolor{currentstroke}%
\pgfsetdash{}{0pt}%
\pgfpathmoveto{\pgfqpoint{1.519565in}{3.138662in}}%
\pgfpathcurveto{\pgfqpoint{1.530616in}{3.138662in}}{\pgfqpoint{1.541215in}{3.143053in}}{\pgfqpoint{1.549028in}{3.150866in}}%
\pgfpathcurveto{\pgfqpoint{1.556842in}{3.158680in}}{\pgfqpoint{1.561232in}{3.169279in}}{\pgfqpoint{1.561232in}{3.180329in}}%
\pgfpathcurveto{\pgfqpoint{1.561232in}{3.191379in}}{\pgfqpoint{1.556842in}{3.201978in}}{\pgfqpoint{1.549028in}{3.209792in}}%
\pgfpathcurveto{\pgfqpoint{1.541215in}{3.217605in}}{\pgfqpoint{1.530616in}{3.221996in}}{\pgfqpoint{1.519565in}{3.221996in}}%
\pgfpathcurveto{\pgfqpoint{1.508515in}{3.221996in}}{\pgfqpoint{1.497916in}{3.217605in}}{\pgfqpoint{1.490103in}{3.209792in}}%
\pgfpathcurveto{\pgfqpoint{1.482289in}{3.201978in}}{\pgfqpoint{1.477899in}{3.191379in}}{\pgfqpoint{1.477899in}{3.180329in}}%
\pgfpathcurveto{\pgfqpoint{1.477899in}{3.169279in}}{\pgfqpoint{1.482289in}{3.158680in}}{\pgfqpoint{1.490103in}{3.150866in}}%
\pgfpathcurveto{\pgfqpoint{1.497916in}{3.143053in}}{\pgfqpoint{1.508515in}{3.138662in}}{\pgfqpoint{1.519565in}{3.138662in}}%
\pgfpathclose%
\pgfusepath{stroke,fill}%
\end{pgfscope}%
\begin{pgfscope}%
\pgfpathrectangle{\pgfqpoint{0.648703in}{0.548769in}}{\pgfqpoint{5.112893in}{3.102590in}}%
\pgfusepath{clip}%
\pgfsetbuttcap%
\pgfsetroundjoin%
\definecolor{currentfill}{rgb}{0.121569,0.466667,0.705882}%
\pgfsetfillcolor{currentfill}%
\pgfsetlinewidth{1.003750pt}%
\definecolor{currentstroke}{rgb}{0.121569,0.466667,0.705882}%
\pgfsetstrokecolor{currentstroke}%
\pgfsetdash{}{0pt}%
\pgfpathmoveto{\pgfqpoint{1.156089in}{3.134537in}}%
\pgfpathcurveto{\pgfqpoint{1.167139in}{3.134537in}}{\pgfqpoint{1.177739in}{3.138928in}}{\pgfqpoint{1.185552in}{3.146741in}}%
\pgfpathcurveto{\pgfqpoint{1.193366in}{3.154555in}}{\pgfqpoint{1.197756in}{3.165154in}}{\pgfqpoint{1.197756in}{3.176204in}}%
\pgfpathcurveto{\pgfqpoint{1.197756in}{3.187254in}}{\pgfqpoint{1.193366in}{3.197853in}}{\pgfqpoint{1.185552in}{3.205667in}}%
\pgfpathcurveto{\pgfqpoint{1.177739in}{3.213480in}}{\pgfqpoint{1.167139in}{3.217871in}}{\pgfqpoint{1.156089in}{3.217871in}}%
\pgfpathcurveto{\pgfqpoint{1.145039in}{3.217871in}}{\pgfqpoint{1.134440in}{3.213480in}}{\pgfqpoint{1.126627in}{3.205667in}}%
\pgfpathcurveto{\pgfqpoint{1.118813in}{3.197853in}}{\pgfqpoint{1.114423in}{3.187254in}}{\pgfqpoint{1.114423in}{3.176204in}}%
\pgfpathcurveto{\pgfqpoint{1.114423in}{3.165154in}}{\pgfqpoint{1.118813in}{3.154555in}}{\pgfqpoint{1.126627in}{3.146741in}}%
\pgfpathcurveto{\pgfqpoint{1.134440in}{3.138928in}}{\pgfqpoint{1.145039in}{3.134537in}}{\pgfqpoint{1.156089in}{3.134537in}}%
\pgfpathclose%
\pgfusepath{stroke,fill}%
\end{pgfscope}%
\begin{pgfscope}%
\pgfpathrectangle{\pgfqpoint{0.648703in}{0.548769in}}{\pgfqpoint{5.112893in}{3.102590in}}%
\pgfusepath{clip}%
\pgfsetbuttcap%
\pgfsetroundjoin%
\definecolor{currentfill}{rgb}{0.839216,0.152941,0.156863}%
\pgfsetfillcolor{currentfill}%
\pgfsetlinewidth{1.003750pt}%
\definecolor{currentstroke}{rgb}{0.839216,0.152941,0.156863}%
\pgfsetstrokecolor{currentstroke}%
\pgfsetdash{}{0pt}%
\pgfpathmoveto{\pgfqpoint{1.626662in}{3.151038in}}%
\pgfpathcurveto{\pgfqpoint{1.637712in}{3.151038in}}{\pgfqpoint{1.648311in}{3.155428in}}{\pgfqpoint{1.656125in}{3.163241in}}%
\pgfpathcurveto{\pgfqpoint{1.663938in}{3.171055in}}{\pgfqpoint{1.668329in}{3.181654in}}{\pgfqpoint{1.668329in}{3.192704in}}%
\pgfpathcurveto{\pgfqpoint{1.668329in}{3.203754in}}{\pgfqpoint{1.663938in}{3.214353in}}{\pgfqpoint{1.656125in}{3.222167in}}%
\pgfpathcurveto{\pgfqpoint{1.648311in}{3.229981in}}{\pgfqpoint{1.637712in}{3.234371in}}{\pgfqpoint{1.626662in}{3.234371in}}%
\pgfpathcurveto{\pgfqpoint{1.615612in}{3.234371in}}{\pgfqpoint{1.605013in}{3.229981in}}{\pgfqpoint{1.597199in}{3.222167in}}%
\pgfpathcurveto{\pgfqpoint{1.589386in}{3.214353in}}{\pgfqpoint{1.584995in}{3.203754in}}{\pgfqpoint{1.584995in}{3.192704in}}%
\pgfpathcurveto{\pgfqpoint{1.584995in}{3.181654in}}{\pgfqpoint{1.589386in}{3.171055in}}{\pgfqpoint{1.597199in}{3.163241in}}%
\pgfpathcurveto{\pgfqpoint{1.605013in}{3.155428in}}{\pgfqpoint{1.615612in}{3.151038in}}{\pgfqpoint{1.626662in}{3.151038in}}%
\pgfpathclose%
\pgfusepath{stroke,fill}%
\end{pgfscope}%
\begin{pgfscope}%
\pgfpathrectangle{\pgfqpoint{0.648703in}{0.548769in}}{\pgfqpoint{5.112893in}{3.102590in}}%
\pgfusepath{clip}%
\pgfsetbuttcap%
\pgfsetroundjoin%
\definecolor{currentfill}{rgb}{1.000000,0.498039,0.054902}%
\pgfsetfillcolor{currentfill}%
\pgfsetlinewidth{1.003750pt}%
\definecolor{currentstroke}{rgb}{1.000000,0.498039,0.054902}%
\pgfsetstrokecolor{currentstroke}%
\pgfsetdash{}{0pt}%
\pgfpathmoveto{\pgfqpoint{1.662629in}{3.175788in}}%
\pgfpathcurveto{\pgfqpoint{1.673680in}{3.175788in}}{\pgfqpoint{1.684279in}{3.180178in}}{\pgfqpoint{1.692092in}{3.187992in}}%
\pgfpathcurveto{\pgfqpoint{1.699906in}{3.195805in}}{\pgfqpoint{1.704296in}{3.206404in}}{\pgfqpoint{1.704296in}{3.217454in}}%
\pgfpathcurveto{\pgfqpoint{1.704296in}{3.228505in}}{\pgfqpoint{1.699906in}{3.239104in}}{\pgfqpoint{1.692092in}{3.246917in}}%
\pgfpathcurveto{\pgfqpoint{1.684279in}{3.254731in}}{\pgfqpoint{1.673680in}{3.259121in}}{\pgfqpoint{1.662629in}{3.259121in}}%
\pgfpathcurveto{\pgfqpoint{1.651579in}{3.259121in}}{\pgfqpoint{1.640980in}{3.254731in}}{\pgfqpoint{1.633167in}{3.246917in}}%
\pgfpathcurveto{\pgfqpoint{1.625353in}{3.239104in}}{\pgfqpoint{1.620963in}{3.228505in}}{\pgfqpoint{1.620963in}{3.217454in}}%
\pgfpathcurveto{\pgfqpoint{1.620963in}{3.206404in}}{\pgfqpoint{1.625353in}{3.195805in}}{\pgfqpoint{1.633167in}{3.187992in}}%
\pgfpathcurveto{\pgfqpoint{1.640980in}{3.180178in}}{\pgfqpoint{1.651579in}{3.175788in}}{\pgfqpoint{1.662629in}{3.175788in}}%
\pgfpathclose%
\pgfusepath{stroke,fill}%
\end{pgfscope}%
\begin{pgfscope}%
\pgfpathrectangle{\pgfqpoint{0.648703in}{0.548769in}}{\pgfqpoint{5.112893in}{3.102590in}}%
\pgfusepath{clip}%
\pgfsetbuttcap%
\pgfsetroundjoin%
\definecolor{currentfill}{rgb}{0.121569,0.466667,0.705882}%
\pgfsetfillcolor{currentfill}%
\pgfsetlinewidth{1.003750pt}%
\definecolor{currentstroke}{rgb}{0.121569,0.466667,0.705882}%
\pgfsetstrokecolor{currentstroke}%
\pgfsetdash{}{0pt}%
\pgfpathmoveto{\pgfqpoint{0.725775in}{0.671892in}}%
\pgfpathcurveto{\pgfqpoint{0.736825in}{0.671892in}}{\pgfqpoint{0.747424in}{0.676282in}}{\pgfqpoint{0.755238in}{0.684096in}}%
\pgfpathcurveto{\pgfqpoint{0.763052in}{0.691909in}}{\pgfqpoint{0.767442in}{0.702509in}}{\pgfqpoint{0.767442in}{0.713559in}}%
\pgfpathcurveto{\pgfqpoint{0.767442in}{0.724609in}}{\pgfqpoint{0.763052in}{0.735208in}}{\pgfqpoint{0.755238in}{0.743021in}}%
\pgfpathcurveto{\pgfqpoint{0.747424in}{0.750835in}}{\pgfqpoint{0.736825in}{0.755225in}}{\pgfqpoint{0.725775in}{0.755225in}}%
\pgfpathcurveto{\pgfqpoint{0.714725in}{0.755225in}}{\pgfqpoint{0.704126in}{0.750835in}}{\pgfqpoint{0.696312in}{0.743021in}}%
\pgfpathcurveto{\pgfqpoint{0.688499in}{0.735208in}}{\pgfqpoint{0.684108in}{0.724609in}}{\pgfqpoint{0.684108in}{0.713559in}}%
\pgfpathcurveto{\pgfqpoint{0.684108in}{0.702509in}}{\pgfqpoint{0.688499in}{0.691909in}}{\pgfqpoint{0.696312in}{0.684096in}}%
\pgfpathcurveto{\pgfqpoint{0.704126in}{0.676282in}}{\pgfqpoint{0.714725in}{0.671892in}}{\pgfqpoint{0.725775in}{0.671892in}}%
\pgfpathclose%
\pgfusepath{stroke,fill}%
\end{pgfscope}%
\begin{pgfscope}%
\pgfpathrectangle{\pgfqpoint{0.648703in}{0.548769in}}{\pgfqpoint{5.112893in}{3.102590in}}%
\pgfusepath{clip}%
\pgfsetbuttcap%
\pgfsetroundjoin%
\definecolor{currentfill}{rgb}{0.121569,0.466667,0.705882}%
\pgfsetfillcolor{currentfill}%
\pgfsetlinewidth{1.003750pt}%
\definecolor{currentstroke}{rgb}{0.121569,0.466667,0.705882}%
\pgfsetstrokecolor{currentstroke}%
\pgfsetdash{}{0pt}%
\pgfpathmoveto{\pgfqpoint{1.070710in}{3.126287in}}%
\pgfpathcurveto{\pgfqpoint{1.081760in}{3.126287in}}{\pgfqpoint{1.092359in}{3.130678in}}{\pgfqpoint{1.100172in}{3.138491in}}%
\pgfpathcurveto{\pgfqpoint{1.107986in}{3.146305in}}{\pgfqpoint{1.112376in}{3.156904in}}{\pgfqpoint{1.112376in}{3.167954in}}%
\pgfpathcurveto{\pgfqpoint{1.112376in}{3.179004in}}{\pgfqpoint{1.107986in}{3.189603in}}{\pgfqpoint{1.100172in}{3.197417in}}%
\pgfpathcurveto{\pgfqpoint{1.092359in}{3.205230in}}{\pgfqpoint{1.081760in}{3.209621in}}{\pgfqpoint{1.070710in}{3.209621in}}%
\pgfpathcurveto{\pgfqpoint{1.059659in}{3.209621in}}{\pgfqpoint{1.049060in}{3.205230in}}{\pgfqpoint{1.041247in}{3.197417in}}%
\pgfpathcurveto{\pgfqpoint{1.033433in}{3.189603in}}{\pgfqpoint{1.029043in}{3.179004in}}{\pgfqpoint{1.029043in}{3.167954in}}%
\pgfpathcurveto{\pgfqpoint{1.029043in}{3.156904in}}{\pgfqpoint{1.033433in}{3.146305in}}{\pgfqpoint{1.041247in}{3.138491in}}%
\pgfpathcurveto{\pgfqpoint{1.049060in}{3.130678in}}{\pgfqpoint{1.059659in}{3.126287in}}{\pgfqpoint{1.070710in}{3.126287in}}%
\pgfpathclose%
\pgfusepath{stroke,fill}%
\end{pgfscope}%
\begin{pgfscope}%
\pgfpathrectangle{\pgfqpoint{0.648703in}{0.548769in}}{\pgfqpoint{5.112893in}{3.102590in}}%
\pgfusepath{clip}%
\pgfsetbuttcap%
\pgfsetroundjoin%
\definecolor{currentfill}{rgb}{0.121569,0.466667,0.705882}%
\pgfsetfillcolor{currentfill}%
\pgfsetlinewidth{1.003750pt}%
\definecolor{currentstroke}{rgb}{0.121569,0.466667,0.705882}%
\pgfsetstrokecolor{currentstroke}%
\pgfsetdash{}{0pt}%
\pgfpathmoveto{\pgfqpoint{0.719480in}{0.663642in}}%
\pgfpathcurveto{\pgfqpoint{0.730530in}{0.663642in}}{\pgfqpoint{0.741129in}{0.668032in}}{\pgfqpoint{0.748943in}{0.675846in}}%
\pgfpathcurveto{\pgfqpoint{0.756757in}{0.683659in}}{\pgfqpoint{0.761147in}{0.694258in}}{\pgfqpoint{0.761147in}{0.705309in}}%
\pgfpathcurveto{\pgfqpoint{0.761147in}{0.716359in}}{\pgfqpoint{0.756757in}{0.726958in}}{\pgfqpoint{0.748943in}{0.734771in}}%
\pgfpathcurveto{\pgfqpoint{0.741129in}{0.742585in}}{\pgfqpoint{0.730530in}{0.746975in}}{\pgfqpoint{0.719480in}{0.746975in}}%
\pgfpathcurveto{\pgfqpoint{0.708430in}{0.746975in}}{\pgfqpoint{0.697831in}{0.742585in}}{\pgfqpoint{0.690017in}{0.734771in}}%
\pgfpathcurveto{\pgfqpoint{0.682204in}{0.726958in}}{\pgfqpoint{0.677814in}{0.716359in}}{\pgfqpoint{0.677814in}{0.705309in}}%
\pgfpathcurveto{\pgfqpoint{0.677814in}{0.694258in}}{\pgfqpoint{0.682204in}{0.683659in}}{\pgfqpoint{0.690017in}{0.675846in}}%
\pgfpathcurveto{\pgfqpoint{0.697831in}{0.668032in}}{\pgfqpoint{0.708430in}{0.663642in}}{\pgfqpoint{0.719480in}{0.663642in}}%
\pgfpathclose%
\pgfusepath{stroke,fill}%
\end{pgfscope}%
\begin{pgfscope}%
\pgfpathrectangle{\pgfqpoint{0.648703in}{0.548769in}}{\pgfqpoint{5.112893in}{3.102590in}}%
\pgfusepath{clip}%
\pgfsetbuttcap%
\pgfsetroundjoin%
\definecolor{currentfill}{rgb}{1.000000,0.498039,0.054902}%
\pgfsetfillcolor{currentfill}%
\pgfsetlinewidth{1.003750pt}%
\definecolor{currentstroke}{rgb}{1.000000,0.498039,0.054902}%
\pgfsetstrokecolor{currentstroke}%
\pgfsetdash{}{0pt}%
\pgfpathmoveto{\pgfqpoint{1.198510in}{3.138662in}}%
\pgfpathcurveto{\pgfqpoint{1.209560in}{3.138662in}}{\pgfqpoint{1.220159in}{3.143053in}}{\pgfqpoint{1.227973in}{3.150866in}}%
\pgfpathcurveto{\pgfqpoint{1.235786in}{3.158680in}}{\pgfqpoint{1.240177in}{3.169279in}}{\pgfqpoint{1.240177in}{3.180329in}}%
\pgfpathcurveto{\pgfqpoint{1.240177in}{3.191379in}}{\pgfqpoint{1.235786in}{3.201978in}}{\pgfqpoint{1.227973in}{3.209792in}}%
\pgfpathcurveto{\pgfqpoint{1.220159in}{3.217605in}}{\pgfqpoint{1.209560in}{3.221996in}}{\pgfqpoint{1.198510in}{3.221996in}}%
\pgfpathcurveto{\pgfqpoint{1.187460in}{3.221996in}}{\pgfqpoint{1.176861in}{3.217605in}}{\pgfqpoint{1.169047in}{3.209792in}}%
\pgfpathcurveto{\pgfqpoint{1.161234in}{3.201978in}}{\pgfqpoint{1.156843in}{3.191379in}}{\pgfqpoint{1.156843in}{3.180329in}}%
\pgfpathcurveto{\pgfqpoint{1.156843in}{3.169279in}}{\pgfqpoint{1.161234in}{3.158680in}}{\pgfqpoint{1.169047in}{3.150866in}}%
\pgfpathcurveto{\pgfqpoint{1.176861in}{3.143053in}}{\pgfqpoint{1.187460in}{3.138662in}}{\pgfqpoint{1.198510in}{3.138662in}}%
\pgfpathclose%
\pgfusepath{stroke,fill}%
\end{pgfscope}%
\begin{pgfscope}%
\pgfpathrectangle{\pgfqpoint{0.648703in}{0.548769in}}{\pgfqpoint{5.112893in}{3.102590in}}%
\pgfusepath{clip}%
\pgfsetbuttcap%
\pgfsetroundjoin%
\definecolor{currentfill}{rgb}{1.000000,0.498039,0.054902}%
\pgfsetfillcolor{currentfill}%
\pgfsetlinewidth{1.003750pt}%
\definecolor{currentstroke}{rgb}{1.000000,0.498039,0.054902}%
\pgfsetstrokecolor{currentstroke}%
\pgfsetdash{}{0pt}%
\pgfpathmoveto{\pgfqpoint{1.485062in}{3.142787in}}%
\pgfpathcurveto{\pgfqpoint{1.496112in}{3.142787in}}{\pgfqpoint{1.506711in}{3.147178in}}{\pgfqpoint{1.514525in}{3.154991in}}%
\pgfpathcurveto{\pgfqpoint{1.522339in}{3.162805in}}{\pgfqpoint{1.526729in}{3.173404in}}{\pgfqpoint{1.526729in}{3.184454in}}%
\pgfpathcurveto{\pgfqpoint{1.526729in}{3.195504in}}{\pgfqpoint{1.522339in}{3.206103in}}{\pgfqpoint{1.514525in}{3.213917in}}%
\pgfpathcurveto{\pgfqpoint{1.506711in}{3.221731in}}{\pgfqpoint{1.496112in}{3.226121in}}{\pgfqpoint{1.485062in}{3.226121in}}%
\pgfpathcurveto{\pgfqpoint{1.474012in}{3.226121in}}{\pgfqpoint{1.463413in}{3.221731in}}{\pgfqpoint{1.455599in}{3.213917in}}%
\pgfpathcurveto{\pgfqpoint{1.447786in}{3.206103in}}{\pgfqpoint{1.443395in}{3.195504in}}{\pgfqpoint{1.443395in}{3.184454in}}%
\pgfpathcurveto{\pgfqpoint{1.443395in}{3.173404in}}{\pgfqpoint{1.447786in}{3.162805in}}{\pgfqpoint{1.455599in}{3.154991in}}%
\pgfpathcurveto{\pgfqpoint{1.463413in}{3.147178in}}{\pgfqpoint{1.474012in}{3.142787in}}{\pgfqpoint{1.485062in}{3.142787in}}%
\pgfpathclose%
\pgfusepath{stroke,fill}%
\end{pgfscope}%
\begin{pgfscope}%
\pgfpathrectangle{\pgfqpoint{0.648703in}{0.548769in}}{\pgfqpoint{5.112893in}{3.102590in}}%
\pgfusepath{clip}%
\pgfsetbuttcap%
\pgfsetroundjoin%
\definecolor{currentfill}{rgb}{1.000000,0.498039,0.054902}%
\pgfsetfillcolor{currentfill}%
\pgfsetlinewidth{1.003750pt}%
\definecolor{currentstroke}{rgb}{1.000000,0.498039,0.054902}%
\pgfsetstrokecolor{currentstroke}%
\pgfsetdash{}{0pt}%
\pgfpathmoveto{\pgfqpoint{1.217061in}{3.167538in}}%
\pgfpathcurveto{\pgfqpoint{1.228111in}{3.167538in}}{\pgfqpoint{1.238710in}{3.171928in}}{\pgfqpoint{1.246523in}{3.179742in}}%
\pgfpathcurveto{\pgfqpoint{1.254337in}{3.187555in}}{\pgfqpoint{1.258727in}{3.198154in}}{\pgfqpoint{1.258727in}{3.209204in}}%
\pgfpathcurveto{\pgfqpoint{1.258727in}{3.220254in}}{\pgfqpoint{1.254337in}{3.230853in}}{\pgfqpoint{1.246523in}{3.238667in}}%
\pgfpathcurveto{\pgfqpoint{1.238710in}{3.246481in}}{\pgfqpoint{1.228111in}{3.250871in}}{\pgfqpoint{1.217061in}{3.250871in}}%
\pgfpathcurveto{\pgfqpoint{1.206010in}{3.250871in}}{\pgfqpoint{1.195411in}{3.246481in}}{\pgfqpoint{1.187598in}{3.238667in}}%
\pgfpathcurveto{\pgfqpoint{1.179784in}{3.230853in}}{\pgfqpoint{1.175394in}{3.220254in}}{\pgfqpoint{1.175394in}{3.209204in}}%
\pgfpathcurveto{\pgfqpoint{1.175394in}{3.198154in}}{\pgfqpoint{1.179784in}{3.187555in}}{\pgfqpoint{1.187598in}{3.179742in}}%
\pgfpathcurveto{\pgfqpoint{1.195411in}{3.171928in}}{\pgfqpoint{1.206010in}{3.167538in}}{\pgfqpoint{1.217061in}{3.167538in}}%
\pgfpathclose%
\pgfusepath{stroke,fill}%
\end{pgfscope}%
\begin{pgfscope}%
\pgfpathrectangle{\pgfqpoint{0.648703in}{0.548769in}}{\pgfqpoint{5.112893in}{3.102590in}}%
\pgfusepath{clip}%
\pgfsetbuttcap%
\pgfsetroundjoin%
\definecolor{currentfill}{rgb}{1.000000,0.498039,0.054902}%
\pgfsetfillcolor{currentfill}%
\pgfsetlinewidth{1.003750pt}%
\definecolor{currentstroke}{rgb}{1.000000,0.498039,0.054902}%
\pgfsetstrokecolor{currentstroke}%
\pgfsetdash{}{0pt}%
\pgfpathmoveto{\pgfqpoint{1.577255in}{3.146912in}}%
\pgfpathcurveto{\pgfqpoint{1.588305in}{3.146912in}}{\pgfqpoint{1.598904in}{3.151303in}}{\pgfqpoint{1.606718in}{3.159116in}}%
\pgfpathcurveto{\pgfqpoint{1.614531in}{3.166930in}}{\pgfqpoint{1.618922in}{3.177529in}}{\pgfqpoint{1.618922in}{3.188579in}}%
\pgfpathcurveto{\pgfqpoint{1.618922in}{3.199629in}}{\pgfqpoint{1.614531in}{3.210228in}}{\pgfqpoint{1.606718in}{3.218042in}}%
\pgfpathcurveto{\pgfqpoint{1.598904in}{3.225856in}}{\pgfqpoint{1.588305in}{3.230246in}}{\pgfqpoint{1.577255in}{3.230246in}}%
\pgfpathcurveto{\pgfqpoint{1.566205in}{3.230246in}}{\pgfqpoint{1.555606in}{3.225856in}}{\pgfqpoint{1.547792in}{3.218042in}}%
\pgfpathcurveto{\pgfqpoint{1.539979in}{3.210228in}}{\pgfqpoint{1.535588in}{3.199629in}}{\pgfqpoint{1.535588in}{3.188579in}}%
\pgfpathcurveto{\pgfqpoint{1.535588in}{3.177529in}}{\pgfqpoint{1.539979in}{3.166930in}}{\pgfqpoint{1.547792in}{3.159116in}}%
\pgfpathcurveto{\pgfqpoint{1.555606in}{3.151303in}}{\pgfqpoint{1.566205in}{3.146912in}}{\pgfqpoint{1.577255in}{3.146912in}}%
\pgfpathclose%
\pgfusepath{stroke,fill}%
\end{pgfscope}%
\begin{pgfscope}%
\pgfpathrectangle{\pgfqpoint{0.648703in}{0.548769in}}{\pgfqpoint{5.112893in}{3.102590in}}%
\pgfusepath{clip}%
\pgfsetbuttcap%
\pgfsetroundjoin%
\definecolor{currentfill}{rgb}{1.000000,0.498039,0.054902}%
\pgfsetfillcolor{currentfill}%
\pgfsetlinewidth{1.003750pt}%
\definecolor{currentstroke}{rgb}{1.000000,0.498039,0.054902}%
\pgfsetstrokecolor{currentstroke}%
\pgfsetdash{}{0pt}%
\pgfpathmoveto{\pgfqpoint{1.551576in}{3.159288in}}%
\pgfpathcurveto{\pgfqpoint{1.562626in}{3.159288in}}{\pgfqpoint{1.573226in}{3.163678in}}{\pgfqpoint{1.581039in}{3.171491in}}%
\pgfpathcurveto{\pgfqpoint{1.588853in}{3.179305in}}{\pgfqpoint{1.593243in}{3.189904in}}{\pgfqpoint{1.593243in}{3.200954in}}%
\pgfpathcurveto{\pgfqpoint{1.593243in}{3.212004in}}{\pgfqpoint{1.588853in}{3.222603in}}{\pgfqpoint{1.581039in}{3.230417in}}%
\pgfpathcurveto{\pgfqpoint{1.573226in}{3.238231in}}{\pgfqpoint{1.562626in}{3.242621in}}{\pgfqpoint{1.551576in}{3.242621in}}%
\pgfpathcurveto{\pgfqpoint{1.540526in}{3.242621in}}{\pgfqpoint{1.529927in}{3.238231in}}{\pgfqpoint{1.522114in}{3.230417in}}%
\pgfpathcurveto{\pgfqpoint{1.514300in}{3.222603in}}{\pgfqpoint{1.509910in}{3.212004in}}{\pgfqpoint{1.509910in}{3.200954in}}%
\pgfpathcurveto{\pgfqpoint{1.509910in}{3.189904in}}{\pgfqpoint{1.514300in}{3.179305in}}{\pgfqpoint{1.522114in}{3.171491in}}%
\pgfpathcurveto{\pgfqpoint{1.529927in}{3.163678in}}{\pgfqpoint{1.540526in}{3.159288in}}{\pgfqpoint{1.551576in}{3.159288in}}%
\pgfpathclose%
\pgfusepath{stroke,fill}%
\end{pgfscope}%
\begin{pgfscope}%
\pgfpathrectangle{\pgfqpoint{0.648703in}{0.548769in}}{\pgfqpoint{5.112893in}{3.102590in}}%
\pgfusepath{clip}%
\pgfsetbuttcap%
\pgfsetroundjoin%
\definecolor{currentfill}{rgb}{1.000000,0.498039,0.054902}%
\pgfsetfillcolor{currentfill}%
\pgfsetlinewidth{1.003750pt}%
\definecolor{currentstroke}{rgb}{1.000000,0.498039,0.054902}%
\pgfsetstrokecolor{currentstroke}%
\pgfsetdash{}{0pt}%
\pgfpathmoveto{\pgfqpoint{1.557896in}{3.324289in}}%
\pgfpathcurveto{\pgfqpoint{1.568946in}{3.324289in}}{\pgfqpoint{1.579545in}{3.328679in}}{\pgfqpoint{1.587359in}{3.336493in}}%
\pgfpathcurveto{\pgfqpoint{1.595172in}{3.344306in}}{\pgfqpoint{1.599563in}{3.354905in}}{\pgfqpoint{1.599563in}{3.365956in}}%
\pgfpathcurveto{\pgfqpoint{1.599563in}{3.377006in}}{\pgfqpoint{1.595172in}{3.387605in}}{\pgfqpoint{1.587359in}{3.395418in}}%
\pgfpathcurveto{\pgfqpoint{1.579545in}{3.403232in}}{\pgfqpoint{1.568946in}{3.407622in}}{\pgfqpoint{1.557896in}{3.407622in}}%
\pgfpathcurveto{\pgfqpoint{1.546846in}{3.407622in}}{\pgfqpoint{1.536247in}{3.403232in}}{\pgfqpoint{1.528433in}{3.395418in}}%
\pgfpathcurveto{\pgfqpoint{1.520619in}{3.387605in}}{\pgfqpoint{1.516229in}{3.377006in}}{\pgfqpoint{1.516229in}{3.365956in}}%
\pgfpathcurveto{\pgfqpoint{1.516229in}{3.354905in}}{\pgfqpoint{1.520619in}{3.344306in}}{\pgfqpoint{1.528433in}{3.336493in}}%
\pgfpathcurveto{\pgfqpoint{1.536247in}{3.328679in}}{\pgfqpoint{1.546846in}{3.324289in}}{\pgfqpoint{1.557896in}{3.324289in}}%
\pgfpathclose%
\pgfusepath{stroke,fill}%
\end{pgfscope}%
\begin{pgfscope}%
\pgfpathrectangle{\pgfqpoint{0.648703in}{0.548769in}}{\pgfqpoint{5.112893in}{3.102590in}}%
\pgfusepath{clip}%
\pgfsetbuttcap%
\pgfsetroundjoin%
\definecolor{currentfill}{rgb}{0.121569,0.466667,0.705882}%
\pgfsetfillcolor{currentfill}%
\pgfsetlinewidth{1.003750pt}%
\definecolor{currentstroke}{rgb}{0.121569,0.466667,0.705882}%
\pgfsetstrokecolor{currentstroke}%
\pgfsetdash{}{0pt}%
\pgfpathmoveto{\pgfqpoint{1.454304in}{2.416781in}}%
\pgfpathcurveto{\pgfqpoint{1.465354in}{2.416781in}}{\pgfqpoint{1.475953in}{2.421172in}}{\pgfqpoint{1.483767in}{2.428985in}}%
\pgfpathcurveto{\pgfqpoint{1.491581in}{2.436799in}}{\pgfqpoint{1.495971in}{2.447398in}}{\pgfqpoint{1.495971in}{2.458448in}}%
\pgfpathcurveto{\pgfqpoint{1.495971in}{2.469498in}}{\pgfqpoint{1.491581in}{2.480097in}}{\pgfqpoint{1.483767in}{2.487911in}}%
\pgfpathcurveto{\pgfqpoint{1.475953in}{2.495725in}}{\pgfqpoint{1.465354in}{2.500115in}}{\pgfqpoint{1.454304in}{2.500115in}}%
\pgfpathcurveto{\pgfqpoint{1.443254in}{2.500115in}}{\pgfqpoint{1.432655in}{2.495725in}}{\pgfqpoint{1.424841in}{2.487911in}}%
\pgfpathcurveto{\pgfqpoint{1.417028in}{2.480097in}}{\pgfqpoint{1.412638in}{2.469498in}}{\pgfqpoint{1.412638in}{2.458448in}}%
\pgfpathcurveto{\pgfqpoint{1.412638in}{2.447398in}}{\pgfqpoint{1.417028in}{2.436799in}}{\pgfqpoint{1.424841in}{2.428985in}}%
\pgfpathcurveto{\pgfqpoint{1.432655in}{2.421172in}}{\pgfqpoint{1.443254in}{2.416781in}}{\pgfqpoint{1.454304in}{2.416781in}}%
\pgfpathclose%
\pgfusepath{stroke,fill}%
\end{pgfscope}%
\begin{pgfscope}%
\pgfpathrectangle{\pgfqpoint{0.648703in}{0.548769in}}{\pgfqpoint{5.112893in}{3.102590in}}%
\pgfusepath{clip}%
\pgfsetbuttcap%
\pgfsetroundjoin%
\definecolor{currentfill}{rgb}{1.000000,0.498039,0.054902}%
\pgfsetfillcolor{currentfill}%
\pgfsetlinewidth{1.003750pt}%
\definecolor{currentstroke}{rgb}{1.000000,0.498039,0.054902}%
\pgfsetstrokecolor{currentstroke}%
\pgfsetdash{}{0pt}%
\pgfpathmoveto{\pgfqpoint{1.606099in}{3.200538in}}%
\pgfpathcurveto{\pgfqpoint{1.617149in}{3.200538in}}{\pgfqpoint{1.627748in}{3.204928in}}{\pgfqpoint{1.635562in}{3.212742in}}%
\pgfpathcurveto{\pgfqpoint{1.643375in}{3.220555in}}{\pgfqpoint{1.647765in}{3.231154in}}{\pgfqpoint{1.647765in}{3.242205in}}%
\pgfpathcurveto{\pgfqpoint{1.647765in}{3.253255in}}{\pgfqpoint{1.643375in}{3.263854in}}{\pgfqpoint{1.635562in}{3.271667in}}%
\pgfpathcurveto{\pgfqpoint{1.627748in}{3.279481in}}{\pgfqpoint{1.617149in}{3.283871in}}{\pgfqpoint{1.606099in}{3.283871in}}%
\pgfpathcurveto{\pgfqpoint{1.595049in}{3.283871in}}{\pgfqpoint{1.584450in}{3.279481in}}{\pgfqpoint{1.576636in}{3.271667in}}%
\pgfpathcurveto{\pgfqpoint{1.568822in}{3.263854in}}{\pgfqpoint{1.564432in}{3.253255in}}{\pgfqpoint{1.564432in}{3.242205in}}%
\pgfpathcurveto{\pgfqpoint{1.564432in}{3.231154in}}{\pgfqpoint{1.568822in}{3.220555in}}{\pgfqpoint{1.576636in}{3.212742in}}%
\pgfpathcurveto{\pgfqpoint{1.584450in}{3.204928in}}{\pgfqpoint{1.595049in}{3.200538in}}{\pgfqpoint{1.606099in}{3.200538in}}%
\pgfpathclose%
\pgfusepath{stroke,fill}%
\end{pgfscope}%
\begin{pgfscope}%
\pgfpathrectangle{\pgfqpoint{0.648703in}{0.548769in}}{\pgfqpoint{5.112893in}{3.102590in}}%
\pgfusepath{clip}%
\pgfsetbuttcap%
\pgfsetroundjoin%
\definecolor{currentfill}{rgb}{0.121569,0.466667,0.705882}%
\pgfsetfillcolor{currentfill}%
\pgfsetlinewidth{1.003750pt}%
\definecolor{currentstroke}{rgb}{0.121569,0.466667,0.705882}%
\pgfsetstrokecolor{currentstroke}%
\pgfsetdash{}{0pt}%
\pgfpathmoveto{\pgfqpoint{1.006456in}{3.126287in}}%
\pgfpathcurveto{\pgfqpoint{1.017506in}{3.126287in}}{\pgfqpoint{1.028105in}{3.130678in}}{\pgfqpoint{1.035918in}{3.138491in}}%
\pgfpathcurveto{\pgfqpoint{1.043732in}{3.146305in}}{\pgfqpoint{1.048122in}{3.156904in}}{\pgfqpoint{1.048122in}{3.167954in}}%
\pgfpathcurveto{\pgfqpoint{1.048122in}{3.179004in}}{\pgfqpoint{1.043732in}{3.189603in}}{\pgfqpoint{1.035918in}{3.197417in}}%
\pgfpathcurveto{\pgfqpoint{1.028105in}{3.205230in}}{\pgfqpoint{1.017506in}{3.209621in}}{\pgfqpoint{1.006456in}{3.209621in}}%
\pgfpathcurveto{\pgfqpoint{0.995405in}{3.209621in}}{\pgfqpoint{0.984806in}{3.205230in}}{\pgfqpoint{0.976993in}{3.197417in}}%
\pgfpathcurveto{\pgfqpoint{0.969179in}{3.189603in}}{\pgfqpoint{0.964789in}{3.179004in}}{\pgfqpoint{0.964789in}{3.167954in}}%
\pgfpathcurveto{\pgfqpoint{0.964789in}{3.156904in}}{\pgfqpoint{0.969179in}{3.146305in}}{\pgfqpoint{0.976993in}{3.138491in}}%
\pgfpathcurveto{\pgfqpoint{0.984806in}{3.130678in}}{\pgfqpoint{0.995405in}{3.126287in}}{\pgfqpoint{1.006456in}{3.126287in}}%
\pgfpathclose%
\pgfusepath{stroke,fill}%
\end{pgfscope}%
\begin{pgfscope}%
\pgfpathrectangle{\pgfqpoint{0.648703in}{0.548769in}}{\pgfqpoint{5.112893in}{3.102590in}}%
\pgfusepath{clip}%
\pgfsetbuttcap%
\pgfsetroundjoin%
\definecolor{currentfill}{rgb}{0.121569,0.466667,0.705882}%
\pgfsetfillcolor{currentfill}%
\pgfsetlinewidth{1.003750pt}%
\definecolor{currentstroke}{rgb}{0.121569,0.466667,0.705882}%
\pgfsetstrokecolor{currentstroke}%
\pgfsetdash{}{0pt}%
\pgfpathmoveto{\pgfqpoint{1.306917in}{3.134537in}}%
\pgfpathcurveto{\pgfqpoint{1.317967in}{3.134537in}}{\pgfqpoint{1.328566in}{3.138928in}}{\pgfqpoint{1.336380in}{3.146741in}}%
\pgfpathcurveto{\pgfqpoint{1.344194in}{3.154555in}}{\pgfqpoint{1.348584in}{3.165154in}}{\pgfqpoint{1.348584in}{3.176204in}}%
\pgfpathcurveto{\pgfqpoint{1.348584in}{3.187254in}}{\pgfqpoint{1.344194in}{3.197853in}}{\pgfqpoint{1.336380in}{3.205667in}}%
\pgfpathcurveto{\pgfqpoint{1.328566in}{3.213480in}}{\pgfqpoint{1.317967in}{3.217871in}}{\pgfqpoint{1.306917in}{3.217871in}}%
\pgfpathcurveto{\pgfqpoint{1.295867in}{3.217871in}}{\pgfqpoint{1.285268in}{3.213480in}}{\pgfqpoint{1.277454in}{3.205667in}}%
\pgfpathcurveto{\pgfqpoint{1.269641in}{3.197853in}}{\pgfqpoint{1.265250in}{3.187254in}}{\pgfqpoint{1.265250in}{3.176204in}}%
\pgfpathcurveto{\pgfqpoint{1.265250in}{3.165154in}}{\pgfqpoint{1.269641in}{3.154555in}}{\pgfqpoint{1.277454in}{3.146741in}}%
\pgfpathcurveto{\pgfqpoint{1.285268in}{3.138928in}}{\pgfqpoint{1.295867in}{3.134537in}}{\pgfqpoint{1.306917in}{3.134537in}}%
\pgfpathclose%
\pgfusepath{stroke,fill}%
\end{pgfscope}%
\begin{pgfscope}%
\pgfpathrectangle{\pgfqpoint{0.648703in}{0.548769in}}{\pgfqpoint{5.112893in}{3.102590in}}%
\pgfusepath{clip}%
\pgfsetbuttcap%
\pgfsetroundjoin%
\definecolor{currentfill}{rgb}{0.121569,0.466667,0.705882}%
\pgfsetfillcolor{currentfill}%
\pgfsetlinewidth{1.003750pt}%
\definecolor{currentstroke}{rgb}{0.121569,0.466667,0.705882}%
\pgfsetstrokecolor{currentstroke}%
\pgfsetdash{}{0pt}%
\pgfpathmoveto{\pgfqpoint{1.050042in}{3.134537in}}%
\pgfpathcurveto{\pgfqpoint{1.061092in}{3.134537in}}{\pgfqpoint{1.071691in}{3.138928in}}{\pgfqpoint{1.079505in}{3.146741in}}%
\pgfpathcurveto{\pgfqpoint{1.087318in}{3.154555in}}{\pgfqpoint{1.091709in}{3.165154in}}{\pgfqpoint{1.091709in}{3.176204in}}%
\pgfpathcurveto{\pgfqpoint{1.091709in}{3.187254in}}{\pgfqpoint{1.087318in}{3.197853in}}{\pgfqpoint{1.079505in}{3.205667in}}%
\pgfpathcurveto{\pgfqpoint{1.071691in}{3.213480in}}{\pgfqpoint{1.061092in}{3.217871in}}{\pgfqpoint{1.050042in}{3.217871in}}%
\pgfpathcurveto{\pgfqpoint{1.038992in}{3.217871in}}{\pgfqpoint{1.028393in}{3.213480in}}{\pgfqpoint{1.020579in}{3.205667in}}%
\pgfpathcurveto{\pgfqpoint{1.012765in}{3.197853in}}{\pgfqpoint{1.008375in}{3.187254in}}{\pgfqpoint{1.008375in}{3.176204in}}%
\pgfpathcurveto{\pgfqpoint{1.008375in}{3.165154in}}{\pgfqpoint{1.012765in}{3.154555in}}{\pgfqpoint{1.020579in}{3.146741in}}%
\pgfpathcurveto{\pgfqpoint{1.028393in}{3.138928in}}{\pgfqpoint{1.038992in}{3.134537in}}{\pgfqpoint{1.050042in}{3.134537in}}%
\pgfpathclose%
\pgfusepath{stroke,fill}%
\end{pgfscope}%
\begin{pgfscope}%
\pgfpathrectangle{\pgfqpoint{0.648703in}{0.548769in}}{\pgfqpoint{5.112893in}{3.102590in}}%
\pgfusepath{clip}%
\pgfsetbuttcap%
\pgfsetroundjoin%
\definecolor{currentfill}{rgb}{1.000000,0.498039,0.054902}%
\pgfsetfillcolor{currentfill}%
\pgfsetlinewidth{1.003750pt}%
\definecolor{currentstroke}{rgb}{1.000000,0.498039,0.054902}%
\pgfsetstrokecolor{currentstroke}%
\pgfsetdash{}{0pt}%
\pgfpathmoveto{\pgfqpoint{1.556369in}{3.138662in}}%
\pgfpathcurveto{\pgfqpoint{1.567419in}{3.138662in}}{\pgfqpoint{1.578018in}{3.143053in}}{\pgfqpoint{1.585832in}{3.150866in}}%
\pgfpathcurveto{\pgfqpoint{1.593646in}{3.158680in}}{\pgfqpoint{1.598036in}{3.169279in}}{\pgfqpoint{1.598036in}{3.180329in}}%
\pgfpathcurveto{\pgfqpoint{1.598036in}{3.191379in}}{\pgfqpoint{1.593646in}{3.201978in}}{\pgfqpoint{1.585832in}{3.209792in}}%
\pgfpathcurveto{\pgfqpoint{1.578018in}{3.217605in}}{\pgfqpoint{1.567419in}{3.221996in}}{\pgfqpoint{1.556369in}{3.221996in}}%
\pgfpathcurveto{\pgfqpoint{1.545319in}{3.221996in}}{\pgfqpoint{1.534720in}{3.217605in}}{\pgfqpoint{1.526906in}{3.209792in}}%
\pgfpathcurveto{\pgfqpoint{1.519093in}{3.201978in}}{\pgfqpoint{1.514703in}{3.191379in}}{\pgfqpoint{1.514703in}{3.180329in}}%
\pgfpathcurveto{\pgfqpoint{1.514703in}{3.169279in}}{\pgfqpoint{1.519093in}{3.158680in}}{\pgfqpoint{1.526906in}{3.150866in}}%
\pgfpathcurveto{\pgfqpoint{1.534720in}{3.143053in}}{\pgfqpoint{1.545319in}{3.138662in}}{\pgfqpoint{1.556369in}{3.138662in}}%
\pgfpathclose%
\pgfusepath{stroke,fill}%
\end{pgfscope}%
\begin{pgfscope}%
\pgfpathrectangle{\pgfqpoint{0.648703in}{0.548769in}}{\pgfqpoint{5.112893in}{3.102590in}}%
\pgfusepath{clip}%
\pgfsetbuttcap%
\pgfsetroundjoin%
\definecolor{currentfill}{rgb}{1.000000,0.498039,0.054902}%
\pgfsetfillcolor{currentfill}%
\pgfsetlinewidth{1.003750pt}%
\definecolor{currentstroke}{rgb}{1.000000,0.498039,0.054902}%
\pgfsetstrokecolor{currentstroke}%
\pgfsetdash{}{0pt}%
\pgfpathmoveto{\pgfqpoint{1.759169in}{3.245913in}}%
\pgfpathcurveto{\pgfqpoint{1.770219in}{3.245913in}}{\pgfqpoint{1.780818in}{3.250304in}}{\pgfqpoint{1.788632in}{3.258117in}}%
\pgfpathcurveto{\pgfqpoint{1.796446in}{3.265931in}}{\pgfqpoint{1.800836in}{3.276530in}}{\pgfqpoint{1.800836in}{3.287580in}}%
\pgfpathcurveto{\pgfqpoint{1.800836in}{3.298630in}}{\pgfqpoint{1.796446in}{3.309229in}}{\pgfqpoint{1.788632in}{3.317043in}}%
\pgfpathcurveto{\pgfqpoint{1.780818in}{3.324856in}}{\pgfqpoint{1.770219in}{3.329247in}}{\pgfqpoint{1.759169in}{3.329247in}}%
\pgfpathcurveto{\pgfqpoint{1.748119in}{3.329247in}}{\pgfqpoint{1.737520in}{3.324856in}}{\pgfqpoint{1.729706in}{3.317043in}}%
\pgfpathcurveto{\pgfqpoint{1.721893in}{3.309229in}}{\pgfqpoint{1.717502in}{3.298630in}}{\pgfqpoint{1.717502in}{3.287580in}}%
\pgfpathcurveto{\pgfqpoint{1.717502in}{3.276530in}}{\pgfqpoint{1.721893in}{3.265931in}}{\pgfqpoint{1.729706in}{3.258117in}}%
\pgfpathcurveto{\pgfqpoint{1.737520in}{3.250304in}}{\pgfqpoint{1.748119in}{3.245913in}}{\pgfqpoint{1.759169in}{3.245913in}}%
\pgfpathclose%
\pgfusepath{stroke,fill}%
\end{pgfscope}%
\begin{pgfscope}%
\pgfpathrectangle{\pgfqpoint{0.648703in}{0.548769in}}{\pgfqpoint{5.112893in}{3.102590in}}%
\pgfusepath{clip}%
\pgfsetbuttcap%
\pgfsetroundjoin%
\definecolor{currentfill}{rgb}{1.000000,0.498039,0.054902}%
\pgfsetfillcolor{currentfill}%
\pgfsetlinewidth{1.003750pt}%
\definecolor{currentstroke}{rgb}{1.000000,0.498039,0.054902}%
\pgfsetstrokecolor{currentstroke}%
\pgfsetdash{}{0pt}%
\pgfpathmoveto{\pgfqpoint{1.536725in}{3.192288in}}%
\pgfpathcurveto{\pgfqpoint{1.547775in}{3.192288in}}{\pgfqpoint{1.558374in}{3.196678in}}{\pgfqpoint{1.566188in}{3.204492in}}%
\pgfpathcurveto{\pgfqpoint{1.574002in}{3.212305in}}{\pgfqpoint{1.578392in}{3.222904in}}{\pgfqpoint{1.578392in}{3.233955in}}%
\pgfpathcurveto{\pgfqpoint{1.578392in}{3.245005in}}{\pgfqpoint{1.574002in}{3.255604in}}{\pgfqpoint{1.566188in}{3.263417in}}%
\pgfpathcurveto{\pgfqpoint{1.558374in}{3.271231in}}{\pgfqpoint{1.547775in}{3.275621in}}{\pgfqpoint{1.536725in}{3.275621in}}%
\pgfpathcurveto{\pgfqpoint{1.525675in}{3.275621in}}{\pgfqpoint{1.515076in}{3.271231in}}{\pgfqpoint{1.507262in}{3.263417in}}%
\pgfpathcurveto{\pgfqpoint{1.499449in}{3.255604in}}{\pgfqpoint{1.495058in}{3.245005in}}{\pgfqpoint{1.495058in}{3.233955in}}%
\pgfpathcurveto{\pgfqpoint{1.495058in}{3.222904in}}{\pgfqpoint{1.499449in}{3.212305in}}{\pgfqpoint{1.507262in}{3.204492in}}%
\pgfpathcurveto{\pgfqpoint{1.515076in}{3.196678in}}{\pgfqpoint{1.525675in}{3.192288in}}{\pgfqpoint{1.536725in}{3.192288in}}%
\pgfpathclose%
\pgfusepath{stroke,fill}%
\end{pgfscope}%
\begin{pgfscope}%
\pgfpathrectangle{\pgfqpoint{0.648703in}{0.548769in}}{\pgfqpoint{5.112893in}{3.102590in}}%
\pgfusepath{clip}%
\pgfsetbuttcap%
\pgfsetroundjoin%
\definecolor{currentfill}{rgb}{0.121569,0.466667,0.705882}%
\pgfsetfillcolor{currentfill}%
\pgfsetlinewidth{1.003750pt}%
\definecolor{currentstroke}{rgb}{0.121569,0.466667,0.705882}%
\pgfsetstrokecolor{currentstroke}%
\pgfsetdash{}{0pt}%
\pgfpathmoveto{\pgfqpoint{0.724981in}{0.680142in}}%
\pgfpathcurveto{\pgfqpoint{0.736032in}{0.680142in}}{\pgfqpoint{0.746631in}{0.684532in}}{\pgfqpoint{0.754444in}{0.692346in}}%
\pgfpathcurveto{\pgfqpoint{0.762258in}{0.700160in}}{\pgfqpoint{0.766648in}{0.710759in}}{\pgfqpoint{0.766648in}{0.721809in}}%
\pgfpathcurveto{\pgfqpoint{0.766648in}{0.732859in}}{\pgfqpoint{0.762258in}{0.743458in}}{\pgfqpoint{0.754444in}{0.751272in}}%
\pgfpathcurveto{\pgfqpoint{0.746631in}{0.759085in}}{\pgfqpoint{0.736032in}{0.763475in}}{\pgfqpoint{0.724981in}{0.763475in}}%
\pgfpathcurveto{\pgfqpoint{0.713931in}{0.763475in}}{\pgfqpoint{0.703332in}{0.759085in}}{\pgfqpoint{0.695519in}{0.751272in}}%
\pgfpathcurveto{\pgfqpoint{0.687705in}{0.743458in}}{\pgfqpoint{0.683315in}{0.732859in}}{\pgfqpoint{0.683315in}{0.721809in}}%
\pgfpathcurveto{\pgfqpoint{0.683315in}{0.710759in}}{\pgfqpoint{0.687705in}{0.700160in}}{\pgfqpoint{0.695519in}{0.692346in}}%
\pgfpathcurveto{\pgfqpoint{0.703332in}{0.684532in}}{\pgfqpoint{0.713931in}{0.680142in}}{\pgfqpoint{0.724981in}{0.680142in}}%
\pgfpathclose%
\pgfusepath{stroke,fill}%
\end{pgfscope}%
\begin{pgfscope}%
\pgfpathrectangle{\pgfqpoint{0.648703in}{0.548769in}}{\pgfqpoint{5.112893in}{3.102590in}}%
\pgfusepath{clip}%
\pgfsetbuttcap%
\pgfsetroundjoin%
\definecolor{currentfill}{rgb}{1.000000,0.498039,0.054902}%
\pgfsetfillcolor{currentfill}%
\pgfsetlinewidth{1.003750pt}%
\definecolor{currentstroke}{rgb}{1.000000,0.498039,0.054902}%
\pgfsetstrokecolor{currentstroke}%
\pgfsetdash{}{0pt}%
\pgfpathmoveto{\pgfqpoint{1.622589in}{3.146912in}}%
\pgfpathcurveto{\pgfqpoint{1.633640in}{3.146912in}}{\pgfqpoint{1.644239in}{3.151303in}}{\pgfqpoint{1.652052in}{3.159116in}}%
\pgfpathcurveto{\pgfqpoint{1.659866in}{3.166930in}}{\pgfqpoint{1.664256in}{3.177529in}}{\pgfqpoint{1.664256in}{3.188579in}}%
\pgfpathcurveto{\pgfqpoint{1.664256in}{3.199629in}}{\pgfqpoint{1.659866in}{3.210228in}}{\pgfqpoint{1.652052in}{3.218042in}}%
\pgfpathcurveto{\pgfqpoint{1.644239in}{3.225856in}}{\pgfqpoint{1.633640in}{3.230246in}}{\pgfqpoint{1.622589in}{3.230246in}}%
\pgfpathcurveto{\pgfqpoint{1.611539in}{3.230246in}}{\pgfqpoint{1.600940in}{3.225856in}}{\pgfqpoint{1.593127in}{3.218042in}}%
\pgfpathcurveto{\pgfqpoint{1.585313in}{3.210228in}}{\pgfqpoint{1.580923in}{3.199629in}}{\pgfqpoint{1.580923in}{3.188579in}}%
\pgfpathcurveto{\pgfqpoint{1.580923in}{3.177529in}}{\pgfqpoint{1.585313in}{3.166930in}}{\pgfqpoint{1.593127in}{3.159116in}}%
\pgfpathcurveto{\pgfqpoint{1.600940in}{3.151303in}}{\pgfqpoint{1.611539in}{3.146912in}}{\pgfqpoint{1.622589in}{3.146912in}}%
\pgfpathclose%
\pgfusepath{stroke,fill}%
\end{pgfscope}%
\begin{pgfscope}%
\pgfpathrectangle{\pgfqpoint{0.648703in}{0.548769in}}{\pgfqpoint{5.112893in}{3.102590in}}%
\pgfusepath{clip}%
\pgfsetbuttcap%
\pgfsetroundjoin%
\definecolor{currentfill}{rgb}{0.121569,0.466667,0.705882}%
\pgfsetfillcolor{currentfill}%
\pgfsetlinewidth{1.003750pt}%
\definecolor{currentstroke}{rgb}{0.121569,0.466667,0.705882}%
\pgfsetstrokecolor{currentstroke}%
\pgfsetdash{}{0pt}%
\pgfpathmoveto{\pgfqpoint{0.719403in}{0.663642in}}%
\pgfpathcurveto{\pgfqpoint{0.730454in}{0.663642in}}{\pgfqpoint{0.741053in}{0.668032in}}{\pgfqpoint{0.748866in}{0.675846in}}%
\pgfpathcurveto{\pgfqpoint{0.756680in}{0.683659in}}{\pgfqpoint{0.761070in}{0.694258in}}{\pgfqpoint{0.761070in}{0.705309in}}%
\pgfpathcurveto{\pgfqpoint{0.761070in}{0.716359in}}{\pgfqpoint{0.756680in}{0.726958in}}{\pgfqpoint{0.748866in}{0.734771in}}%
\pgfpathcurveto{\pgfqpoint{0.741053in}{0.742585in}}{\pgfqpoint{0.730454in}{0.746975in}}{\pgfqpoint{0.719403in}{0.746975in}}%
\pgfpathcurveto{\pgfqpoint{0.708353in}{0.746975in}}{\pgfqpoint{0.697754in}{0.742585in}}{\pgfqpoint{0.689941in}{0.734771in}}%
\pgfpathcurveto{\pgfqpoint{0.682127in}{0.726958in}}{\pgfqpoint{0.677737in}{0.716359in}}{\pgfqpoint{0.677737in}{0.705309in}}%
\pgfpathcurveto{\pgfqpoint{0.677737in}{0.694258in}}{\pgfqpoint{0.682127in}{0.683659in}}{\pgfqpoint{0.689941in}{0.675846in}}%
\pgfpathcurveto{\pgfqpoint{0.697754in}{0.668032in}}{\pgfqpoint{0.708353in}{0.663642in}}{\pgfqpoint{0.719403in}{0.663642in}}%
\pgfpathclose%
\pgfusepath{stroke,fill}%
\end{pgfscope}%
\begin{pgfscope}%
\pgfpathrectangle{\pgfqpoint{0.648703in}{0.548769in}}{\pgfqpoint{5.112893in}{3.102590in}}%
\pgfusepath{clip}%
\pgfsetbuttcap%
\pgfsetroundjoin%
\definecolor{currentfill}{rgb}{1.000000,0.498039,0.054902}%
\pgfsetfillcolor{currentfill}%
\pgfsetlinewidth{1.003750pt}%
\definecolor{currentstroke}{rgb}{1.000000,0.498039,0.054902}%
\pgfsetstrokecolor{currentstroke}%
\pgfsetdash{}{0pt}%
\pgfpathmoveto{\pgfqpoint{1.677521in}{3.151038in}}%
\pgfpathcurveto{\pgfqpoint{1.688571in}{3.151038in}}{\pgfqpoint{1.699170in}{3.155428in}}{\pgfqpoint{1.706984in}{3.163241in}}%
\pgfpathcurveto{\pgfqpoint{1.714798in}{3.171055in}}{\pgfqpoint{1.719188in}{3.181654in}}{\pgfqpoint{1.719188in}{3.192704in}}%
\pgfpathcurveto{\pgfqpoint{1.719188in}{3.203754in}}{\pgfqpoint{1.714798in}{3.214353in}}{\pgfqpoint{1.706984in}{3.222167in}}%
\pgfpathcurveto{\pgfqpoint{1.699170in}{3.229981in}}{\pgfqpoint{1.688571in}{3.234371in}}{\pgfqpoint{1.677521in}{3.234371in}}%
\pgfpathcurveto{\pgfqpoint{1.666471in}{3.234371in}}{\pgfqpoint{1.655872in}{3.229981in}}{\pgfqpoint{1.648058in}{3.222167in}}%
\pgfpathcurveto{\pgfqpoint{1.640245in}{3.214353in}}{\pgfqpoint{1.635854in}{3.203754in}}{\pgfqpoint{1.635854in}{3.192704in}}%
\pgfpathcurveto{\pgfqpoint{1.635854in}{3.181654in}}{\pgfqpoint{1.640245in}{3.171055in}}{\pgfqpoint{1.648058in}{3.163241in}}%
\pgfpathcurveto{\pgfqpoint{1.655872in}{3.155428in}}{\pgfqpoint{1.666471in}{3.151038in}}{\pgfqpoint{1.677521in}{3.151038in}}%
\pgfpathclose%
\pgfusepath{stroke,fill}%
\end{pgfscope}%
\begin{pgfscope}%
\pgfpathrectangle{\pgfqpoint{0.648703in}{0.548769in}}{\pgfqpoint{5.112893in}{3.102590in}}%
\pgfusepath{clip}%
\pgfsetbuttcap%
\pgfsetroundjoin%
\definecolor{currentfill}{rgb}{0.121569,0.466667,0.705882}%
\pgfsetfillcolor{currentfill}%
\pgfsetlinewidth{1.003750pt}%
\definecolor{currentstroke}{rgb}{0.121569,0.466667,0.705882}%
\pgfsetstrokecolor{currentstroke}%
\pgfsetdash{}{0pt}%
\pgfpathmoveto{\pgfqpoint{1.591844in}{3.126287in}}%
\pgfpathcurveto{\pgfqpoint{1.602894in}{3.126287in}}{\pgfqpoint{1.613494in}{3.130678in}}{\pgfqpoint{1.621307in}{3.138491in}}%
\pgfpathcurveto{\pgfqpoint{1.629121in}{3.146305in}}{\pgfqpoint{1.633511in}{3.156904in}}{\pgfqpoint{1.633511in}{3.167954in}}%
\pgfpathcurveto{\pgfqpoint{1.633511in}{3.179004in}}{\pgfqpoint{1.629121in}{3.189603in}}{\pgfqpoint{1.621307in}{3.197417in}}%
\pgfpathcurveto{\pgfqpoint{1.613494in}{3.205230in}}{\pgfqpoint{1.602894in}{3.209621in}}{\pgfqpoint{1.591844in}{3.209621in}}%
\pgfpathcurveto{\pgfqpoint{1.580794in}{3.209621in}}{\pgfqpoint{1.570195in}{3.205230in}}{\pgfqpoint{1.562382in}{3.197417in}}%
\pgfpathcurveto{\pgfqpoint{1.554568in}{3.189603in}}{\pgfqpoint{1.550178in}{3.179004in}}{\pgfqpoint{1.550178in}{3.167954in}}%
\pgfpathcurveto{\pgfqpoint{1.550178in}{3.156904in}}{\pgfqpoint{1.554568in}{3.146305in}}{\pgfqpoint{1.562382in}{3.138491in}}%
\pgfpathcurveto{\pgfqpoint{1.570195in}{3.130678in}}{\pgfqpoint{1.580794in}{3.126287in}}{\pgfqpoint{1.591844in}{3.126287in}}%
\pgfpathclose%
\pgfusepath{stroke,fill}%
\end{pgfscope}%
\begin{pgfscope}%
\pgfpathrectangle{\pgfqpoint{0.648703in}{0.548769in}}{\pgfqpoint{5.112893in}{3.102590in}}%
\pgfusepath{clip}%
\pgfsetbuttcap%
\pgfsetroundjoin%
\definecolor{currentfill}{rgb}{1.000000,0.498039,0.054902}%
\pgfsetfillcolor{currentfill}%
\pgfsetlinewidth{1.003750pt}%
\definecolor{currentstroke}{rgb}{1.000000,0.498039,0.054902}%
\pgfsetstrokecolor{currentstroke}%
\pgfsetdash{}{0pt}%
\pgfpathmoveto{\pgfqpoint{1.436761in}{3.138662in}}%
\pgfpathcurveto{\pgfqpoint{1.447812in}{3.138662in}}{\pgfqpoint{1.458411in}{3.143053in}}{\pgfqpoint{1.466224in}{3.150866in}}%
\pgfpathcurveto{\pgfqpoint{1.474038in}{3.158680in}}{\pgfqpoint{1.478428in}{3.169279in}}{\pgfqpoint{1.478428in}{3.180329in}}%
\pgfpathcurveto{\pgfqpoint{1.478428in}{3.191379in}}{\pgfqpoint{1.474038in}{3.201978in}}{\pgfqpoint{1.466224in}{3.209792in}}%
\pgfpathcurveto{\pgfqpoint{1.458411in}{3.217605in}}{\pgfqpoint{1.447812in}{3.221996in}}{\pgfqpoint{1.436761in}{3.221996in}}%
\pgfpathcurveto{\pgfqpoint{1.425711in}{3.221996in}}{\pgfqpoint{1.415112in}{3.217605in}}{\pgfqpoint{1.407299in}{3.209792in}}%
\pgfpathcurveto{\pgfqpoint{1.399485in}{3.201978in}}{\pgfqpoint{1.395095in}{3.191379in}}{\pgfqpoint{1.395095in}{3.180329in}}%
\pgfpathcurveto{\pgfqpoint{1.395095in}{3.169279in}}{\pgfqpoint{1.399485in}{3.158680in}}{\pgfqpoint{1.407299in}{3.150866in}}%
\pgfpathcurveto{\pgfqpoint{1.415112in}{3.143053in}}{\pgfqpoint{1.425711in}{3.138662in}}{\pgfqpoint{1.436761in}{3.138662in}}%
\pgfpathclose%
\pgfusepath{stroke,fill}%
\end{pgfscope}%
\begin{pgfscope}%
\pgfpathrectangle{\pgfqpoint{0.648703in}{0.548769in}}{\pgfqpoint{5.112893in}{3.102590in}}%
\pgfusepath{clip}%
\pgfsetbuttcap%
\pgfsetroundjoin%
\definecolor{currentfill}{rgb}{1.000000,0.498039,0.054902}%
\pgfsetfillcolor{currentfill}%
\pgfsetlinewidth{1.003750pt}%
\definecolor{currentstroke}{rgb}{1.000000,0.498039,0.054902}%
\pgfsetstrokecolor{currentstroke}%
\pgfsetdash{}{0pt}%
\pgfpathmoveto{\pgfqpoint{1.202351in}{3.142787in}}%
\pgfpathcurveto{\pgfqpoint{1.213401in}{3.142787in}}{\pgfqpoint{1.224000in}{3.147178in}}{\pgfqpoint{1.231814in}{3.154991in}}%
\pgfpathcurveto{\pgfqpoint{1.239628in}{3.162805in}}{\pgfqpoint{1.244018in}{3.173404in}}{\pgfqpoint{1.244018in}{3.184454in}}%
\pgfpathcurveto{\pgfqpoint{1.244018in}{3.195504in}}{\pgfqpoint{1.239628in}{3.206103in}}{\pgfqpoint{1.231814in}{3.213917in}}%
\pgfpathcurveto{\pgfqpoint{1.224000in}{3.221731in}}{\pgfqpoint{1.213401in}{3.226121in}}{\pgfqpoint{1.202351in}{3.226121in}}%
\pgfpathcurveto{\pgfqpoint{1.191301in}{3.226121in}}{\pgfqpoint{1.180702in}{3.221731in}}{\pgfqpoint{1.172888in}{3.213917in}}%
\pgfpathcurveto{\pgfqpoint{1.165075in}{3.206103in}}{\pgfqpoint{1.160685in}{3.195504in}}{\pgfqpoint{1.160685in}{3.184454in}}%
\pgfpathcurveto{\pgfqpoint{1.160685in}{3.173404in}}{\pgfqpoint{1.165075in}{3.162805in}}{\pgfqpoint{1.172888in}{3.154991in}}%
\pgfpathcurveto{\pgfqpoint{1.180702in}{3.147178in}}{\pgfqpoint{1.191301in}{3.142787in}}{\pgfqpoint{1.202351in}{3.142787in}}%
\pgfpathclose%
\pgfusepath{stroke,fill}%
\end{pgfscope}%
\begin{pgfscope}%
\pgfpathrectangle{\pgfqpoint{0.648703in}{0.548769in}}{\pgfqpoint{5.112893in}{3.102590in}}%
\pgfusepath{clip}%
\pgfsetbuttcap%
\pgfsetroundjoin%
\definecolor{currentfill}{rgb}{1.000000,0.498039,0.054902}%
\pgfsetfillcolor{currentfill}%
\pgfsetlinewidth{1.003750pt}%
\definecolor{currentstroke}{rgb}{1.000000,0.498039,0.054902}%
\pgfsetstrokecolor{currentstroke}%
\pgfsetdash{}{0pt}%
\pgfpathmoveto{\pgfqpoint{1.582892in}{3.146912in}}%
\pgfpathcurveto{\pgfqpoint{1.593943in}{3.146912in}}{\pgfqpoint{1.604542in}{3.151303in}}{\pgfqpoint{1.612355in}{3.159116in}}%
\pgfpathcurveto{\pgfqpoint{1.620169in}{3.166930in}}{\pgfqpoint{1.624559in}{3.177529in}}{\pgfqpoint{1.624559in}{3.188579in}}%
\pgfpathcurveto{\pgfqpoint{1.624559in}{3.199629in}}{\pgfqpoint{1.620169in}{3.210228in}}{\pgfqpoint{1.612355in}{3.218042in}}%
\pgfpathcurveto{\pgfqpoint{1.604542in}{3.225856in}}{\pgfqpoint{1.593943in}{3.230246in}}{\pgfqpoint{1.582892in}{3.230246in}}%
\pgfpathcurveto{\pgfqpoint{1.571842in}{3.230246in}}{\pgfqpoint{1.561243in}{3.225856in}}{\pgfqpoint{1.553430in}{3.218042in}}%
\pgfpathcurveto{\pgfqpoint{1.545616in}{3.210228in}}{\pgfqpoint{1.541226in}{3.199629in}}{\pgfqpoint{1.541226in}{3.188579in}}%
\pgfpathcurveto{\pgfqpoint{1.541226in}{3.177529in}}{\pgfqpoint{1.545616in}{3.166930in}}{\pgfqpoint{1.553430in}{3.159116in}}%
\pgfpathcurveto{\pgfqpoint{1.561243in}{3.151303in}}{\pgfqpoint{1.571842in}{3.146912in}}{\pgfqpoint{1.582892in}{3.146912in}}%
\pgfpathclose%
\pgfusepath{stroke,fill}%
\end{pgfscope}%
\begin{pgfscope}%
\pgfpathrectangle{\pgfqpoint{0.648703in}{0.548769in}}{\pgfqpoint{5.112893in}{3.102590in}}%
\pgfusepath{clip}%
\pgfsetbuttcap%
\pgfsetroundjoin%
\definecolor{currentfill}{rgb}{0.121569,0.466667,0.705882}%
\pgfsetfillcolor{currentfill}%
\pgfsetlinewidth{1.003750pt}%
\definecolor{currentstroke}{rgb}{0.121569,0.466667,0.705882}%
\pgfsetstrokecolor{currentstroke}%
\pgfsetdash{}{0pt}%
\pgfpathmoveto{\pgfqpoint{1.306736in}{3.134537in}}%
\pgfpathcurveto{\pgfqpoint{1.317786in}{3.134537in}}{\pgfqpoint{1.328385in}{3.138928in}}{\pgfqpoint{1.336199in}{3.146741in}}%
\pgfpathcurveto{\pgfqpoint{1.344013in}{3.154555in}}{\pgfqpoint{1.348403in}{3.165154in}}{\pgfqpoint{1.348403in}{3.176204in}}%
\pgfpathcurveto{\pgfqpoint{1.348403in}{3.187254in}}{\pgfqpoint{1.344013in}{3.197853in}}{\pgfqpoint{1.336199in}{3.205667in}}%
\pgfpathcurveto{\pgfqpoint{1.328385in}{3.213480in}}{\pgfqpoint{1.317786in}{3.217871in}}{\pgfqpoint{1.306736in}{3.217871in}}%
\pgfpathcurveto{\pgfqpoint{1.295686in}{3.217871in}}{\pgfqpoint{1.285087in}{3.213480in}}{\pgfqpoint{1.277273in}{3.205667in}}%
\pgfpathcurveto{\pgfqpoint{1.269460in}{3.197853in}}{\pgfqpoint{1.265069in}{3.187254in}}{\pgfqpoint{1.265069in}{3.176204in}}%
\pgfpathcurveto{\pgfqpoint{1.265069in}{3.165154in}}{\pgfqpoint{1.269460in}{3.154555in}}{\pgfqpoint{1.277273in}{3.146741in}}%
\pgfpathcurveto{\pgfqpoint{1.285087in}{3.138928in}}{\pgfqpoint{1.295686in}{3.134537in}}{\pgfqpoint{1.306736in}{3.134537in}}%
\pgfpathclose%
\pgfusepath{stroke,fill}%
\end{pgfscope}%
\begin{pgfscope}%
\pgfpathrectangle{\pgfqpoint{0.648703in}{0.548769in}}{\pgfqpoint{5.112893in}{3.102590in}}%
\pgfusepath{clip}%
\pgfsetbuttcap%
\pgfsetroundjoin%
\definecolor{currentfill}{rgb}{0.121569,0.466667,0.705882}%
\pgfsetfillcolor{currentfill}%
\pgfsetlinewidth{1.003750pt}%
\definecolor{currentstroke}{rgb}{0.121569,0.466667,0.705882}%
\pgfsetstrokecolor{currentstroke}%
\pgfsetdash{}{0pt}%
\pgfpathmoveto{\pgfqpoint{0.736532in}{0.676017in}}%
\pgfpathcurveto{\pgfqpoint{0.747582in}{0.676017in}}{\pgfqpoint{0.758181in}{0.680407in}}{\pgfqpoint{0.765995in}{0.688221in}}%
\pgfpathcurveto{\pgfqpoint{0.773808in}{0.696035in}}{\pgfqpoint{0.778199in}{0.706634in}}{\pgfqpoint{0.778199in}{0.717684in}}%
\pgfpathcurveto{\pgfqpoint{0.778199in}{0.728734in}}{\pgfqpoint{0.773808in}{0.739333in}}{\pgfqpoint{0.765995in}{0.747146in}}%
\pgfpathcurveto{\pgfqpoint{0.758181in}{0.754960in}}{\pgfqpoint{0.747582in}{0.759350in}}{\pgfqpoint{0.736532in}{0.759350in}}%
\pgfpathcurveto{\pgfqpoint{0.725482in}{0.759350in}}{\pgfqpoint{0.714883in}{0.754960in}}{\pgfqpoint{0.707069in}{0.747146in}}%
\pgfpathcurveto{\pgfqpoint{0.699256in}{0.739333in}}{\pgfqpoint{0.694865in}{0.728734in}}{\pgfqpoint{0.694865in}{0.717684in}}%
\pgfpathcurveto{\pgfqpoint{0.694865in}{0.706634in}}{\pgfqpoint{0.699256in}{0.696035in}}{\pgfqpoint{0.707069in}{0.688221in}}%
\pgfpathcurveto{\pgfqpoint{0.714883in}{0.680407in}}{\pgfqpoint{0.725482in}{0.676017in}}{\pgfqpoint{0.736532in}{0.676017in}}%
\pgfpathclose%
\pgfusepath{stroke,fill}%
\end{pgfscope}%
\begin{pgfscope}%
\pgfpathrectangle{\pgfqpoint{0.648703in}{0.548769in}}{\pgfqpoint{5.112893in}{3.102590in}}%
\pgfusepath{clip}%
\pgfsetbuttcap%
\pgfsetroundjoin%
\definecolor{currentfill}{rgb}{1.000000,0.498039,0.054902}%
\pgfsetfillcolor{currentfill}%
\pgfsetlinewidth{1.003750pt}%
\definecolor{currentstroke}{rgb}{1.000000,0.498039,0.054902}%
\pgfsetstrokecolor{currentstroke}%
\pgfsetdash{}{0pt}%
\pgfpathmoveto{\pgfqpoint{1.788814in}{3.138662in}}%
\pgfpathcurveto{\pgfqpoint{1.799864in}{3.138662in}}{\pgfqpoint{1.810463in}{3.143053in}}{\pgfqpoint{1.818277in}{3.150866in}}%
\pgfpathcurveto{\pgfqpoint{1.826090in}{3.158680in}}{\pgfqpoint{1.830481in}{3.169279in}}{\pgfqpoint{1.830481in}{3.180329in}}%
\pgfpathcurveto{\pgfqpoint{1.830481in}{3.191379in}}{\pgfqpoint{1.826090in}{3.201978in}}{\pgfqpoint{1.818277in}{3.209792in}}%
\pgfpathcurveto{\pgfqpoint{1.810463in}{3.217605in}}{\pgfqpoint{1.799864in}{3.221996in}}{\pgfqpoint{1.788814in}{3.221996in}}%
\pgfpathcurveto{\pgfqpoint{1.777764in}{3.221996in}}{\pgfqpoint{1.767165in}{3.217605in}}{\pgfqpoint{1.759351in}{3.209792in}}%
\pgfpathcurveto{\pgfqpoint{1.751538in}{3.201978in}}{\pgfqpoint{1.747147in}{3.191379in}}{\pgfqpoint{1.747147in}{3.180329in}}%
\pgfpathcurveto{\pgfqpoint{1.747147in}{3.169279in}}{\pgfqpoint{1.751538in}{3.158680in}}{\pgfqpoint{1.759351in}{3.150866in}}%
\pgfpathcurveto{\pgfqpoint{1.767165in}{3.143053in}}{\pgfqpoint{1.777764in}{3.138662in}}{\pgfqpoint{1.788814in}{3.138662in}}%
\pgfpathclose%
\pgfusepath{stroke,fill}%
\end{pgfscope}%
\begin{pgfscope}%
\pgfpathrectangle{\pgfqpoint{0.648703in}{0.548769in}}{\pgfqpoint{5.112893in}{3.102590in}}%
\pgfusepath{clip}%
\pgfsetbuttcap%
\pgfsetroundjoin%
\definecolor{currentfill}{rgb}{1.000000,0.498039,0.054902}%
\pgfsetfillcolor{currentfill}%
\pgfsetlinewidth{1.003750pt}%
\definecolor{currentstroke}{rgb}{1.000000,0.498039,0.054902}%
\pgfsetstrokecolor{currentstroke}%
\pgfsetdash{}{0pt}%
\pgfpathmoveto{\pgfqpoint{1.564242in}{3.159288in}}%
\pgfpathcurveto{\pgfqpoint{1.575292in}{3.159288in}}{\pgfqpoint{1.585891in}{3.163678in}}{\pgfqpoint{1.593704in}{3.171491in}}%
\pgfpathcurveto{\pgfqpoint{1.601518in}{3.179305in}}{\pgfqpoint{1.605908in}{3.189904in}}{\pgfqpoint{1.605908in}{3.200954in}}%
\pgfpathcurveto{\pgfqpoint{1.605908in}{3.212004in}}{\pgfqpoint{1.601518in}{3.222603in}}{\pgfqpoint{1.593704in}{3.230417in}}%
\pgfpathcurveto{\pgfqpoint{1.585891in}{3.238231in}}{\pgfqpoint{1.575292in}{3.242621in}}{\pgfqpoint{1.564242in}{3.242621in}}%
\pgfpathcurveto{\pgfqpoint{1.553191in}{3.242621in}}{\pgfqpoint{1.542592in}{3.238231in}}{\pgfqpoint{1.534779in}{3.230417in}}%
\pgfpathcurveto{\pgfqpoint{1.526965in}{3.222603in}}{\pgfqpoint{1.522575in}{3.212004in}}{\pgfqpoint{1.522575in}{3.200954in}}%
\pgfpathcurveto{\pgfqpoint{1.522575in}{3.189904in}}{\pgfqpoint{1.526965in}{3.179305in}}{\pgfqpoint{1.534779in}{3.171491in}}%
\pgfpathcurveto{\pgfqpoint{1.542592in}{3.163678in}}{\pgfqpoint{1.553191in}{3.159288in}}{\pgfqpoint{1.564242in}{3.159288in}}%
\pgfpathclose%
\pgfusepath{stroke,fill}%
\end{pgfscope}%
\begin{pgfscope}%
\pgfpathrectangle{\pgfqpoint{0.648703in}{0.548769in}}{\pgfqpoint{5.112893in}{3.102590in}}%
\pgfusepath{clip}%
\pgfsetbuttcap%
\pgfsetroundjoin%
\definecolor{currentfill}{rgb}{0.121569,0.466667,0.705882}%
\pgfsetfillcolor{currentfill}%
\pgfsetlinewidth{1.003750pt}%
\definecolor{currentstroke}{rgb}{0.121569,0.466667,0.705882}%
\pgfsetstrokecolor{currentstroke}%
\pgfsetdash{}{0pt}%
\pgfpathmoveto{\pgfqpoint{0.719376in}{0.663642in}}%
\pgfpathcurveto{\pgfqpoint{0.730426in}{0.663642in}}{\pgfqpoint{0.741025in}{0.668032in}}{\pgfqpoint{0.748839in}{0.675846in}}%
\pgfpathcurveto{\pgfqpoint{0.756652in}{0.683659in}}{\pgfqpoint{0.761043in}{0.694258in}}{\pgfqpoint{0.761043in}{0.705309in}}%
\pgfpathcurveto{\pgfqpoint{0.761043in}{0.716359in}}{\pgfqpoint{0.756652in}{0.726958in}}{\pgfqpoint{0.748839in}{0.734771in}}%
\pgfpathcurveto{\pgfqpoint{0.741025in}{0.742585in}}{\pgfqpoint{0.730426in}{0.746975in}}{\pgfqpoint{0.719376in}{0.746975in}}%
\pgfpathcurveto{\pgfqpoint{0.708326in}{0.746975in}}{\pgfqpoint{0.697727in}{0.742585in}}{\pgfqpoint{0.689913in}{0.734771in}}%
\pgfpathcurveto{\pgfqpoint{0.682099in}{0.726958in}}{\pgfqpoint{0.677709in}{0.716359in}}{\pgfqpoint{0.677709in}{0.705309in}}%
\pgfpathcurveto{\pgfqpoint{0.677709in}{0.694258in}}{\pgfqpoint{0.682099in}{0.683659in}}{\pgfqpoint{0.689913in}{0.675846in}}%
\pgfpathcurveto{\pgfqpoint{0.697727in}{0.668032in}}{\pgfqpoint{0.708326in}{0.663642in}}{\pgfqpoint{0.719376in}{0.663642in}}%
\pgfpathclose%
\pgfusepath{stroke,fill}%
\end{pgfscope}%
\begin{pgfscope}%
\pgfpathrectangle{\pgfqpoint{0.648703in}{0.548769in}}{\pgfqpoint{5.112893in}{3.102590in}}%
\pgfusepath{clip}%
\pgfsetbuttcap%
\pgfsetroundjoin%
\definecolor{currentfill}{rgb}{0.121569,0.466667,0.705882}%
\pgfsetfillcolor{currentfill}%
\pgfsetlinewidth{1.003750pt}%
\definecolor{currentstroke}{rgb}{0.121569,0.466667,0.705882}%
\pgfsetstrokecolor{currentstroke}%
\pgfsetdash{}{0pt}%
\pgfpathmoveto{\pgfqpoint{0.720147in}{0.663642in}}%
\pgfpathcurveto{\pgfqpoint{0.731197in}{0.663642in}}{\pgfqpoint{0.741796in}{0.668032in}}{\pgfqpoint{0.749610in}{0.675846in}}%
\pgfpathcurveto{\pgfqpoint{0.757424in}{0.683659in}}{\pgfqpoint{0.761814in}{0.694258in}}{\pgfqpoint{0.761814in}{0.705309in}}%
\pgfpathcurveto{\pgfqpoint{0.761814in}{0.716359in}}{\pgfqpoint{0.757424in}{0.726958in}}{\pgfqpoint{0.749610in}{0.734771in}}%
\pgfpathcurveto{\pgfqpoint{0.741796in}{0.742585in}}{\pgfqpoint{0.731197in}{0.746975in}}{\pgfqpoint{0.720147in}{0.746975in}}%
\pgfpathcurveto{\pgfqpoint{0.709097in}{0.746975in}}{\pgfqpoint{0.698498in}{0.742585in}}{\pgfqpoint{0.690684in}{0.734771in}}%
\pgfpathcurveto{\pgfqpoint{0.682871in}{0.726958in}}{\pgfqpoint{0.678480in}{0.716359in}}{\pgfqpoint{0.678480in}{0.705309in}}%
\pgfpathcurveto{\pgfqpoint{0.678480in}{0.694258in}}{\pgfqpoint{0.682871in}{0.683659in}}{\pgfqpoint{0.690684in}{0.675846in}}%
\pgfpathcurveto{\pgfqpoint{0.698498in}{0.668032in}}{\pgfqpoint{0.709097in}{0.663642in}}{\pgfqpoint{0.720147in}{0.663642in}}%
\pgfpathclose%
\pgfusepath{stroke,fill}%
\end{pgfscope}%
\begin{pgfscope}%
\pgfpathrectangle{\pgfqpoint{0.648703in}{0.548769in}}{\pgfqpoint{5.112893in}{3.102590in}}%
\pgfusepath{clip}%
\pgfsetbuttcap%
\pgfsetroundjoin%
\definecolor{currentfill}{rgb}{0.121569,0.466667,0.705882}%
\pgfsetfillcolor{currentfill}%
\pgfsetlinewidth{1.003750pt}%
\definecolor{currentstroke}{rgb}{0.121569,0.466667,0.705882}%
\pgfsetstrokecolor{currentstroke}%
\pgfsetdash{}{0pt}%
\pgfpathmoveto{\pgfqpoint{1.565201in}{3.134537in}}%
\pgfpathcurveto{\pgfqpoint{1.576251in}{3.134537in}}{\pgfqpoint{1.586850in}{3.138928in}}{\pgfqpoint{1.594664in}{3.146741in}}%
\pgfpathcurveto{\pgfqpoint{1.602477in}{3.154555in}}{\pgfqpoint{1.606868in}{3.165154in}}{\pgfqpoint{1.606868in}{3.176204in}}%
\pgfpathcurveto{\pgfqpoint{1.606868in}{3.187254in}}{\pgfqpoint{1.602477in}{3.197853in}}{\pgfqpoint{1.594664in}{3.205667in}}%
\pgfpathcurveto{\pgfqpoint{1.586850in}{3.213480in}}{\pgfqpoint{1.576251in}{3.217871in}}{\pgfqpoint{1.565201in}{3.217871in}}%
\pgfpathcurveto{\pgfqpoint{1.554151in}{3.217871in}}{\pgfqpoint{1.543552in}{3.213480in}}{\pgfqpoint{1.535738in}{3.205667in}}%
\pgfpathcurveto{\pgfqpoint{1.527925in}{3.197853in}}{\pgfqpoint{1.523534in}{3.187254in}}{\pgfqpoint{1.523534in}{3.176204in}}%
\pgfpathcurveto{\pgfqpoint{1.523534in}{3.165154in}}{\pgfqpoint{1.527925in}{3.154555in}}{\pgfqpoint{1.535738in}{3.146741in}}%
\pgfpathcurveto{\pgfqpoint{1.543552in}{3.138928in}}{\pgfqpoint{1.554151in}{3.134537in}}{\pgfqpoint{1.565201in}{3.134537in}}%
\pgfpathclose%
\pgfusepath{stroke,fill}%
\end{pgfscope}%
\begin{pgfscope}%
\pgfpathrectangle{\pgfqpoint{0.648703in}{0.548769in}}{\pgfqpoint{5.112893in}{3.102590in}}%
\pgfusepath{clip}%
\pgfsetbuttcap%
\pgfsetroundjoin%
\definecolor{currentfill}{rgb}{1.000000,0.498039,0.054902}%
\pgfsetfillcolor{currentfill}%
\pgfsetlinewidth{1.003750pt}%
\definecolor{currentstroke}{rgb}{1.000000,0.498039,0.054902}%
\pgfsetstrokecolor{currentstroke}%
\pgfsetdash{}{0pt}%
\pgfpathmoveto{\pgfqpoint{1.492415in}{3.221163in}}%
\pgfpathcurveto{\pgfqpoint{1.503465in}{3.221163in}}{\pgfqpoint{1.514064in}{3.225553in}}{\pgfqpoint{1.521878in}{3.233367in}}%
\pgfpathcurveto{\pgfqpoint{1.529691in}{3.241181in}}{\pgfqpoint{1.534082in}{3.251780in}}{\pgfqpoint{1.534082in}{3.262830in}}%
\pgfpathcurveto{\pgfqpoint{1.534082in}{3.273880in}}{\pgfqpoint{1.529691in}{3.284479in}}{\pgfqpoint{1.521878in}{3.292293in}}%
\pgfpathcurveto{\pgfqpoint{1.514064in}{3.300106in}}{\pgfqpoint{1.503465in}{3.304496in}}{\pgfqpoint{1.492415in}{3.304496in}}%
\pgfpathcurveto{\pgfqpoint{1.481365in}{3.304496in}}{\pgfqpoint{1.470766in}{3.300106in}}{\pgfqpoint{1.462952in}{3.292293in}}%
\pgfpathcurveto{\pgfqpoint{1.455139in}{3.284479in}}{\pgfqpoint{1.450748in}{3.273880in}}{\pgfqpoint{1.450748in}{3.262830in}}%
\pgfpathcurveto{\pgfqpoint{1.450748in}{3.251780in}}{\pgfqpoint{1.455139in}{3.241181in}}{\pgfqpoint{1.462952in}{3.233367in}}%
\pgfpathcurveto{\pgfqpoint{1.470766in}{3.225553in}}{\pgfqpoint{1.481365in}{3.221163in}}{\pgfqpoint{1.492415in}{3.221163in}}%
\pgfpathclose%
\pgfusepath{stroke,fill}%
\end{pgfscope}%
\begin{pgfscope}%
\pgfpathrectangle{\pgfqpoint{0.648703in}{0.548769in}}{\pgfqpoint{5.112893in}{3.102590in}}%
\pgfusepath{clip}%
\pgfsetbuttcap%
\pgfsetroundjoin%
\definecolor{currentfill}{rgb}{1.000000,0.498039,0.054902}%
\pgfsetfillcolor{currentfill}%
\pgfsetlinewidth{1.003750pt}%
\definecolor{currentstroke}{rgb}{1.000000,0.498039,0.054902}%
\pgfsetstrokecolor{currentstroke}%
\pgfsetdash{}{0pt}%
\pgfpathmoveto{\pgfqpoint{1.433257in}{3.142787in}}%
\pgfpathcurveto{\pgfqpoint{1.444307in}{3.142787in}}{\pgfqpoint{1.454906in}{3.147178in}}{\pgfqpoint{1.462719in}{3.154991in}}%
\pgfpathcurveto{\pgfqpoint{1.470533in}{3.162805in}}{\pgfqpoint{1.474923in}{3.173404in}}{\pgfqpoint{1.474923in}{3.184454in}}%
\pgfpathcurveto{\pgfqpoint{1.474923in}{3.195504in}}{\pgfqpoint{1.470533in}{3.206103in}}{\pgfqpoint{1.462719in}{3.213917in}}%
\pgfpathcurveto{\pgfqpoint{1.454906in}{3.221731in}}{\pgfqpoint{1.444307in}{3.226121in}}{\pgfqpoint{1.433257in}{3.226121in}}%
\pgfpathcurveto{\pgfqpoint{1.422207in}{3.226121in}}{\pgfqpoint{1.411608in}{3.221731in}}{\pgfqpoint{1.403794in}{3.213917in}}%
\pgfpathcurveto{\pgfqpoint{1.395980in}{3.206103in}}{\pgfqpoint{1.391590in}{3.195504in}}{\pgfqpoint{1.391590in}{3.184454in}}%
\pgfpathcurveto{\pgfqpoint{1.391590in}{3.173404in}}{\pgfqpoint{1.395980in}{3.162805in}}{\pgfqpoint{1.403794in}{3.154991in}}%
\pgfpathcurveto{\pgfqpoint{1.411608in}{3.147178in}}{\pgfqpoint{1.422207in}{3.142787in}}{\pgfqpoint{1.433257in}{3.142787in}}%
\pgfpathclose%
\pgfusepath{stroke,fill}%
\end{pgfscope}%
\begin{pgfscope}%
\pgfpathrectangle{\pgfqpoint{0.648703in}{0.548769in}}{\pgfqpoint{5.112893in}{3.102590in}}%
\pgfusepath{clip}%
\pgfsetbuttcap%
\pgfsetroundjoin%
\definecolor{currentfill}{rgb}{1.000000,0.498039,0.054902}%
\pgfsetfillcolor{currentfill}%
\pgfsetlinewidth{1.003750pt}%
\definecolor{currentstroke}{rgb}{1.000000,0.498039,0.054902}%
\pgfsetstrokecolor{currentstroke}%
\pgfsetdash{}{0pt}%
\pgfpathmoveto{\pgfqpoint{1.252819in}{3.146912in}}%
\pgfpathcurveto{\pgfqpoint{1.263869in}{3.146912in}}{\pgfqpoint{1.274468in}{3.151303in}}{\pgfqpoint{1.282282in}{3.159116in}}%
\pgfpathcurveto{\pgfqpoint{1.290095in}{3.166930in}}{\pgfqpoint{1.294486in}{3.177529in}}{\pgfqpoint{1.294486in}{3.188579in}}%
\pgfpathcurveto{\pgfqpoint{1.294486in}{3.199629in}}{\pgfqpoint{1.290095in}{3.210228in}}{\pgfqpoint{1.282282in}{3.218042in}}%
\pgfpathcurveto{\pgfqpoint{1.274468in}{3.225856in}}{\pgfqpoint{1.263869in}{3.230246in}}{\pgfqpoint{1.252819in}{3.230246in}}%
\pgfpathcurveto{\pgfqpoint{1.241769in}{3.230246in}}{\pgfqpoint{1.231170in}{3.225856in}}{\pgfqpoint{1.223356in}{3.218042in}}%
\pgfpathcurveto{\pgfqpoint{1.215543in}{3.210228in}}{\pgfqpoint{1.211152in}{3.199629in}}{\pgfqpoint{1.211152in}{3.188579in}}%
\pgfpathcurveto{\pgfqpoint{1.211152in}{3.177529in}}{\pgfqpoint{1.215543in}{3.166930in}}{\pgfqpoint{1.223356in}{3.159116in}}%
\pgfpathcurveto{\pgfqpoint{1.231170in}{3.151303in}}{\pgfqpoint{1.241769in}{3.146912in}}{\pgfqpoint{1.252819in}{3.146912in}}%
\pgfpathclose%
\pgfusepath{stroke,fill}%
\end{pgfscope}%
\begin{pgfscope}%
\pgfpathrectangle{\pgfqpoint{0.648703in}{0.548769in}}{\pgfqpoint{5.112893in}{3.102590in}}%
\pgfusepath{clip}%
\pgfsetbuttcap%
\pgfsetroundjoin%
\definecolor{currentfill}{rgb}{1.000000,0.498039,0.054902}%
\pgfsetfillcolor{currentfill}%
\pgfsetlinewidth{1.003750pt}%
\definecolor{currentstroke}{rgb}{1.000000,0.498039,0.054902}%
\pgfsetstrokecolor{currentstroke}%
\pgfsetdash{}{0pt}%
\pgfpathmoveto{\pgfqpoint{1.557650in}{3.138662in}}%
\pgfpathcurveto{\pgfqpoint{1.568700in}{3.138662in}}{\pgfqpoint{1.579299in}{3.143053in}}{\pgfqpoint{1.587113in}{3.150866in}}%
\pgfpathcurveto{\pgfqpoint{1.594926in}{3.158680in}}{\pgfqpoint{1.599316in}{3.169279in}}{\pgfqpoint{1.599316in}{3.180329in}}%
\pgfpathcurveto{\pgfqpoint{1.599316in}{3.191379in}}{\pgfqpoint{1.594926in}{3.201978in}}{\pgfqpoint{1.587113in}{3.209792in}}%
\pgfpathcurveto{\pgfqpoint{1.579299in}{3.217605in}}{\pgfqpoint{1.568700in}{3.221996in}}{\pgfqpoint{1.557650in}{3.221996in}}%
\pgfpathcurveto{\pgfqpoint{1.546600in}{3.221996in}}{\pgfqpoint{1.536001in}{3.217605in}}{\pgfqpoint{1.528187in}{3.209792in}}%
\pgfpathcurveto{\pgfqpoint{1.520373in}{3.201978in}}{\pgfqpoint{1.515983in}{3.191379in}}{\pgfqpoint{1.515983in}{3.180329in}}%
\pgfpathcurveto{\pgfqpoint{1.515983in}{3.169279in}}{\pgfqpoint{1.520373in}{3.158680in}}{\pgfqpoint{1.528187in}{3.150866in}}%
\pgfpathcurveto{\pgfqpoint{1.536001in}{3.143053in}}{\pgfqpoint{1.546600in}{3.138662in}}{\pgfqpoint{1.557650in}{3.138662in}}%
\pgfpathclose%
\pgfusepath{stroke,fill}%
\end{pgfscope}%
\begin{pgfscope}%
\pgfpathrectangle{\pgfqpoint{0.648703in}{0.548769in}}{\pgfqpoint{5.112893in}{3.102590in}}%
\pgfusepath{clip}%
\pgfsetbuttcap%
\pgfsetroundjoin%
\definecolor{currentfill}{rgb}{0.121569,0.466667,0.705882}%
\pgfsetfillcolor{currentfill}%
\pgfsetlinewidth{1.003750pt}%
\definecolor{currentstroke}{rgb}{0.121569,0.466667,0.705882}%
\pgfsetstrokecolor{currentstroke}%
\pgfsetdash{}{0pt}%
\pgfpathmoveto{\pgfqpoint{0.719376in}{0.663642in}}%
\pgfpathcurveto{\pgfqpoint{0.730426in}{0.663642in}}{\pgfqpoint{0.741025in}{0.668032in}}{\pgfqpoint{0.748838in}{0.675846in}}%
\pgfpathcurveto{\pgfqpoint{0.756652in}{0.683659in}}{\pgfqpoint{0.761042in}{0.694258in}}{\pgfqpoint{0.761042in}{0.705309in}}%
\pgfpathcurveto{\pgfqpoint{0.761042in}{0.716359in}}{\pgfqpoint{0.756652in}{0.726958in}}{\pgfqpoint{0.748838in}{0.734771in}}%
\pgfpathcurveto{\pgfqpoint{0.741025in}{0.742585in}}{\pgfqpoint{0.730426in}{0.746975in}}{\pgfqpoint{0.719376in}{0.746975in}}%
\pgfpathcurveto{\pgfqpoint{0.708325in}{0.746975in}}{\pgfqpoint{0.697726in}{0.742585in}}{\pgfqpoint{0.689913in}{0.734771in}}%
\pgfpathcurveto{\pgfqpoint{0.682099in}{0.726958in}}{\pgfqpoint{0.677709in}{0.716359in}}{\pgfqpoint{0.677709in}{0.705309in}}%
\pgfpathcurveto{\pgfqpoint{0.677709in}{0.694258in}}{\pgfqpoint{0.682099in}{0.683659in}}{\pgfqpoint{0.689913in}{0.675846in}}%
\pgfpathcurveto{\pgfqpoint{0.697726in}{0.668032in}}{\pgfqpoint{0.708325in}{0.663642in}}{\pgfqpoint{0.719376in}{0.663642in}}%
\pgfpathclose%
\pgfusepath{stroke,fill}%
\end{pgfscope}%
\begin{pgfscope}%
\pgfpathrectangle{\pgfqpoint{0.648703in}{0.548769in}}{\pgfqpoint{5.112893in}{3.102590in}}%
\pgfusepath{clip}%
\pgfsetbuttcap%
\pgfsetroundjoin%
\definecolor{currentfill}{rgb}{0.121569,0.466667,0.705882}%
\pgfsetfillcolor{currentfill}%
\pgfsetlinewidth{1.003750pt}%
\definecolor{currentstroke}{rgb}{0.121569,0.466667,0.705882}%
\pgfsetstrokecolor{currentstroke}%
\pgfsetdash{}{0pt}%
\pgfpathmoveto{\pgfqpoint{1.578369in}{3.134537in}}%
\pgfpathcurveto{\pgfqpoint{1.589419in}{3.134537in}}{\pgfqpoint{1.600018in}{3.138928in}}{\pgfqpoint{1.607831in}{3.146741in}}%
\pgfpathcurveto{\pgfqpoint{1.615645in}{3.154555in}}{\pgfqpoint{1.620035in}{3.165154in}}{\pgfqpoint{1.620035in}{3.176204in}}%
\pgfpathcurveto{\pgfqpoint{1.620035in}{3.187254in}}{\pgfqpoint{1.615645in}{3.197853in}}{\pgfqpoint{1.607831in}{3.205667in}}%
\pgfpathcurveto{\pgfqpoint{1.600018in}{3.213480in}}{\pgfqpoint{1.589419in}{3.217871in}}{\pgfqpoint{1.578369in}{3.217871in}}%
\pgfpathcurveto{\pgfqpoint{1.567318in}{3.217871in}}{\pgfqpoint{1.556719in}{3.213480in}}{\pgfqpoint{1.548906in}{3.205667in}}%
\pgfpathcurveto{\pgfqpoint{1.541092in}{3.197853in}}{\pgfqpoint{1.536702in}{3.187254in}}{\pgfqpoint{1.536702in}{3.176204in}}%
\pgfpathcurveto{\pgfqpoint{1.536702in}{3.165154in}}{\pgfqpoint{1.541092in}{3.154555in}}{\pgfqpoint{1.548906in}{3.146741in}}%
\pgfpathcurveto{\pgfqpoint{1.556719in}{3.138928in}}{\pgfqpoint{1.567318in}{3.134537in}}{\pgfqpoint{1.578369in}{3.134537in}}%
\pgfpathclose%
\pgfusepath{stroke,fill}%
\end{pgfscope}%
\begin{pgfscope}%
\pgfpathrectangle{\pgfqpoint{0.648703in}{0.548769in}}{\pgfqpoint{5.112893in}{3.102590in}}%
\pgfusepath{clip}%
\pgfsetbuttcap%
\pgfsetroundjoin%
\definecolor{currentfill}{rgb}{1.000000,0.498039,0.054902}%
\pgfsetfillcolor{currentfill}%
\pgfsetlinewidth{1.003750pt}%
\definecolor{currentstroke}{rgb}{1.000000,0.498039,0.054902}%
\pgfsetstrokecolor{currentstroke}%
\pgfsetdash{}{0pt}%
\pgfpathmoveto{\pgfqpoint{1.686323in}{3.146912in}}%
\pgfpathcurveto{\pgfqpoint{1.697373in}{3.146912in}}{\pgfqpoint{1.707972in}{3.151303in}}{\pgfqpoint{1.715786in}{3.159116in}}%
\pgfpathcurveto{\pgfqpoint{1.723600in}{3.166930in}}{\pgfqpoint{1.727990in}{3.177529in}}{\pgfqpoint{1.727990in}{3.188579in}}%
\pgfpathcurveto{\pgfqpoint{1.727990in}{3.199629in}}{\pgfqpoint{1.723600in}{3.210228in}}{\pgfqpoint{1.715786in}{3.218042in}}%
\pgfpathcurveto{\pgfqpoint{1.707972in}{3.225856in}}{\pgfqpoint{1.697373in}{3.230246in}}{\pgfqpoint{1.686323in}{3.230246in}}%
\pgfpathcurveto{\pgfqpoint{1.675273in}{3.230246in}}{\pgfqpoint{1.664674in}{3.225856in}}{\pgfqpoint{1.656860in}{3.218042in}}%
\pgfpathcurveto{\pgfqpoint{1.649047in}{3.210228in}}{\pgfqpoint{1.644657in}{3.199629in}}{\pgfqpoint{1.644657in}{3.188579in}}%
\pgfpathcurveto{\pgfqpoint{1.644657in}{3.177529in}}{\pgfqpoint{1.649047in}{3.166930in}}{\pgfqpoint{1.656860in}{3.159116in}}%
\pgfpathcurveto{\pgfqpoint{1.664674in}{3.151303in}}{\pgfqpoint{1.675273in}{3.146912in}}{\pgfqpoint{1.686323in}{3.146912in}}%
\pgfpathclose%
\pgfusepath{stroke,fill}%
\end{pgfscope}%
\begin{pgfscope}%
\pgfpathrectangle{\pgfqpoint{0.648703in}{0.548769in}}{\pgfqpoint{5.112893in}{3.102590in}}%
\pgfusepath{clip}%
\pgfsetbuttcap%
\pgfsetroundjoin%
\definecolor{currentfill}{rgb}{0.121569,0.466667,0.705882}%
\pgfsetfillcolor{currentfill}%
\pgfsetlinewidth{1.003750pt}%
\definecolor{currentstroke}{rgb}{0.121569,0.466667,0.705882}%
\pgfsetstrokecolor{currentstroke}%
\pgfsetdash{}{0pt}%
\pgfpathmoveto{\pgfqpoint{1.195401in}{3.134537in}}%
\pgfpathcurveto{\pgfqpoint{1.206451in}{3.134537in}}{\pgfqpoint{1.217050in}{3.138928in}}{\pgfqpoint{1.224864in}{3.146741in}}%
\pgfpathcurveto{\pgfqpoint{1.232677in}{3.154555in}}{\pgfqpoint{1.237067in}{3.165154in}}{\pgfqpoint{1.237067in}{3.176204in}}%
\pgfpathcurveto{\pgfqpoint{1.237067in}{3.187254in}}{\pgfqpoint{1.232677in}{3.197853in}}{\pgfqpoint{1.224864in}{3.205667in}}%
\pgfpathcurveto{\pgfqpoint{1.217050in}{3.213480in}}{\pgfqpoint{1.206451in}{3.217871in}}{\pgfqpoint{1.195401in}{3.217871in}}%
\pgfpathcurveto{\pgfqpoint{1.184351in}{3.217871in}}{\pgfqpoint{1.173752in}{3.213480in}}{\pgfqpoint{1.165938in}{3.205667in}}%
\pgfpathcurveto{\pgfqpoint{1.158124in}{3.197853in}}{\pgfqpoint{1.153734in}{3.187254in}}{\pgfqpoint{1.153734in}{3.176204in}}%
\pgfpathcurveto{\pgfqpoint{1.153734in}{3.165154in}}{\pgfqpoint{1.158124in}{3.154555in}}{\pgfqpoint{1.165938in}{3.146741in}}%
\pgfpathcurveto{\pgfqpoint{1.173752in}{3.138928in}}{\pgfqpoint{1.184351in}{3.134537in}}{\pgfqpoint{1.195401in}{3.134537in}}%
\pgfpathclose%
\pgfusepath{stroke,fill}%
\end{pgfscope}%
\begin{pgfscope}%
\pgfpathrectangle{\pgfqpoint{0.648703in}{0.548769in}}{\pgfqpoint{5.112893in}{3.102590in}}%
\pgfusepath{clip}%
\pgfsetbuttcap%
\pgfsetroundjoin%
\definecolor{currentfill}{rgb}{1.000000,0.498039,0.054902}%
\pgfsetfillcolor{currentfill}%
\pgfsetlinewidth{1.003750pt}%
\definecolor{currentstroke}{rgb}{1.000000,0.498039,0.054902}%
\pgfsetstrokecolor{currentstroke}%
\pgfsetdash{}{0pt}%
\pgfpathmoveto{\pgfqpoint{1.975819in}{3.159288in}}%
\pgfpathcurveto{\pgfqpoint{1.986869in}{3.159288in}}{\pgfqpoint{1.997468in}{3.163678in}}{\pgfqpoint{2.005282in}{3.171491in}}%
\pgfpathcurveto{\pgfqpoint{2.013096in}{3.179305in}}{\pgfqpoint{2.017486in}{3.189904in}}{\pgfqpoint{2.017486in}{3.200954in}}%
\pgfpathcurveto{\pgfqpoint{2.017486in}{3.212004in}}{\pgfqpoint{2.013096in}{3.222603in}}{\pgfqpoint{2.005282in}{3.230417in}}%
\pgfpathcurveto{\pgfqpoint{1.997468in}{3.238231in}}{\pgfqpoint{1.986869in}{3.242621in}}{\pgfqpoint{1.975819in}{3.242621in}}%
\pgfpathcurveto{\pgfqpoint{1.964769in}{3.242621in}}{\pgfqpoint{1.954170in}{3.238231in}}{\pgfqpoint{1.946356in}{3.230417in}}%
\pgfpathcurveto{\pgfqpoint{1.938543in}{3.222603in}}{\pgfqpoint{1.934153in}{3.212004in}}{\pgfqpoint{1.934153in}{3.200954in}}%
\pgfpathcurveto{\pgfqpoint{1.934153in}{3.189904in}}{\pgfqpoint{1.938543in}{3.179305in}}{\pgfqpoint{1.946356in}{3.171491in}}%
\pgfpathcurveto{\pgfqpoint{1.954170in}{3.163678in}}{\pgfqpoint{1.964769in}{3.159288in}}{\pgfqpoint{1.975819in}{3.159288in}}%
\pgfpathclose%
\pgfusepath{stroke,fill}%
\end{pgfscope}%
\begin{pgfscope}%
\pgfpathrectangle{\pgfqpoint{0.648703in}{0.548769in}}{\pgfqpoint{5.112893in}{3.102590in}}%
\pgfusepath{clip}%
\pgfsetbuttcap%
\pgfsetroundjoin%
\definecolor{currentfill}{rgb}{1.000000,0.498039,0.054902}%
\pgfsetfillcolor{currentfill}%
\pgfsetlinewidth{1.003750pt}%
\definecolor{currentstroke}{rgb}{1.000000,0.498039,0.054902}%
\pgfsetstrokecolor{currentstroke}%
\pgfsetdash{}{0pt}%
\pgfpathmoveto{\pgfqpoint{1.491313in}{3.146912in}}%
\pgfpathcurveto{\pgfqpoint{1.502363in}{3.146912in}}{\pgfqpoint{1.512962in}{3.151303in}}{\pgfqpoint{1.520776in}{3.159116in}}%
\pgfpathcurveto{\pgfqpoint{1.528590in}{3.166930in}}{\pgfqpoint{1.532980in}{3.177529in}}{\pgfqpoint{1.532980in}{3.188579in}}%
\pgfpathcurveto{\pgfqpoint{1.532980in}{3.199629in}}{\pgfqpoint{1.528590in}{3.210228in}}{\pgfqpoint{1.520776in}{3.218042in}}%
\pgfpathcurveto{\pgfqpoint{1.512962in}{3.225856in}}{\pgfqpoint{1.502363in}{3.230246in}}{\pgfqpoint{1.491313in}{3.230246in}}%
\pgfpathcurveto{\pgfqpoint{1.480263in}{3.230246in}}{\pgfqpoint{1.469664in}{3.225856in}}{\pgfqpoint{1.461850in}{3.218042in}}%
\pgfpathcurveto{\pgfqpoint{1.454037in}{3.210228in}}{\pgfqpoint{1.449646in}{3.199629in}}{\pgfqpoint{1.449646in}{3.188579in}}%
\pgfpathcurveto{\pgfqpoint{1.449646in}{3.177529in}}{\pgfqpoint{1.454037in}{3.166930in}}{\pgfqpoint{1.461850in}{3.159116in}}%
\pgfpathcurveto{\pgfqpoint{1.469664in}{3.151303in}}{\pgfqpoint{1.480263in}{3.146912in}}{\pgfqpoint{1.491313in}{3.146912in}}%
\pgfpathclose%
\pgfusepath{stroke,fill}%
\end{pgfscope}%
\begin{pgfscope}%
\pgfpathrectangle{\pgfqpoint{0.648703in}{0.548769in}}{\pgfqpoint{5.112893in}{3.102590in}}%
\pgfusepath{clip}%
\pgfsetbuttcap%
\pgfsetroundjoin%
\definecolor{currentfill}{rgb}{1.000000,0.498039,0.054902}%
\pgfsetfillcolor{currentfill}%
\pgfsetlinewidth{1.003750pt}%
\definecolor{currentstroke}{rgb}{1.000000,0.498039,0.054902}%
\pgfsetstrokecolor{currentstroke}%
\pgfsetdash{}{0pt}%
\pgfpathmoveto{\pgfqpoint{1.633582in}{3.192288in}}%
\pgfpathcurveto{\pgfqpoint{1.644632in}{3.192288in}}{\pgfqpoint{1.655231in}{3.196678in}}{\pgfqpoint{1.663045in}{3.204492in}}%
\pgfpathcurveto{\pgfqpoint{1.670858in}{3.212305in}}{\pgfqpoint{1.675248in}{3.222904in}}{\pgfqpoint{1.675248in}{3.233955in}}%
\pgfpathcurveto{\pgfqpoint{1.675248in}{3.245005in}}{\pgfqpoint{1.670858in}{3.255604in}}{\pgfqpoint{1.663045in}{3.263417in}}%
\pgfpathcurveto{\pgfqpoint{1.655231in}{3.271231in}}{\pgfqpoint{1.644632in}{3.275621in}}{\pgfqpoint{1.633582in}{3.275621in}}%
\pgfpathcurveto{\pgfqpoint{1.622532in}{3.275621in}}{\pgfqpoint{1.611933in}{3.271231in}}{\pgfqpoint{1.604119in}{3.263417in}}%
\pgfpathcurveto{\pgfqpoint{1.596305in}{3.255604in}}{\pgfqpoint{1.591915in}{3.245005in}}{\pgfqpoint{1.591915in}{3.233955in}}%
\pgfpathcurveto{\pgfqpoint{1.591915in}{3.222904in}}{\pgfqpoint{1.596305in}{3.212305in}}{\pgfqpoint{1.604119in}{3.204492in}}%
\pgfpathcurveto{\pgfqpoint{1.611933in}{3.196678in}}{\pgfqpoint{1.622532in}{3.192288in}}{\pgfqpoint{1.633582in}{3.192288in}}%
\pgfpathclose%
\pgfusepath{stroke,fill}%
\end{pgfscope}%
\begin{pgfscope}%
\pgfpathrectangle{\pgfqpoint{0.648703in}{0.548769in}}{\pgfqpoint{5.112893in}{3.102590in}}%
\pgfusepath{clip}%
\pgfsetbuttcap%
\pgfsetroundjoin%
\definecolor{currentfill}{rgb}{1.000000,0.498039,0.054902}%
\pgfsetfillcolor{currentfill}%
\pgfsetlinewidth{1.003750pt}%
\definecolor{currentstroke}{rgb}{1.000000,0.498039,0.054902}%
\pgfsetstrokecolor{currentstroke}%
\pgfsetdash{}{0pt}%
\pgfpathmoveto{\pgfqpoint{1.447096in}{3.208788in}}%
\pgfpathcurveto{\pgfqpoint{1.458146in}{3.208788in}}{\pgfqpoint{1.468745in}{3.213178in}}{\pgfqpoint{1.476559in}{3.220992in}}%
\pgfpathcurveto{\pgfqpoint{1.484372in}{3.228805in}}{\pgfqpoint{1.488763in}{3.239405in}}{\pgfqpoint{1.488763in}{3.250455in}}%
\pgfpathcurveto{\pgfqpoint{1.488763in}{3.261505in}}{\pgfqpoint{1.484372in}{3.272104in}}{\pgfqpoint{1.476559in}{3.279917in}}%
\pgfpathcurveto{\pgfqpoint{1.468745in}{3.287731in}}{\pgfqpoint{1.458146in}{3.292121in}}{\pgfqpoint{1.447096in}{3.292121in}}%
\pgfpathcurveto{\pgfqpoint{1.436046in}{3.292121in}}{\pgfqpoint{1.425447in}{3.287731in}}{\pgfqpoint{1.417633in}{3.279917in}}%
\pgfpathcurveto{\pgfqpoint{1.409820in}{3.272104in}}{\pgfqpoint{1.405429in}{3.261505in}}{\pgfqpoint{1.405429in}{3.250455in}}%
\pgfpathcurveto{\pgfqpoint{1.405429in}{3.239405in}}{\pgfqpoint{1.409820in}{3.228805in}}{\pgfqpoint{1.417633in}{3.220992in}}%
\pgfpathcurveto{\pgfqpoint{1.425447in}{3.213178in}}{\pgfqpoint{1.436046in}{3.208788in}}{\pgfqpoint{1.447096in}{3.208788in}}%
\pgfpathclose%
\pgfusepath{stroke,fill}%
\end{pgfscope}%
\begin{pgfscope}%
\pgfpathrectangle{\pgfqpoint{0.648703in}{0.548769in}}{\pgfqpoint{5.112893in}{3.102590in}}%
\pgfusepath{clip}%
\pgfsetbuttcap%
\pgfsetroundjoin%
\definecolor{currentfill}{rgb}{0.121569,0.466667,0.705882}%
\pgfsetfillcolor{currentfill}%
\pgfsetlinewidth{1.003750pt}%
\definecolor{currentstroke}{rgb}{0.121569,0.466667,0.705882}%
\pgfsetstrokecolor{currentstroke}%
\pgfsetdash{}{0pt}%
\pgfpathmoveto{\pgfqpoint{0.719376in}{0.663642in}}%
\pgfpathcurveto{\pgfqpoint{0.730426in}{0.663642in}}{\pgfqpoint{0.741025in}{0.668032in}}{\pgfqpoint{0.748838in}{0.675846in}}%
\pgfpathcurveto{\pgfqpoint{0.756652in}{0.683659in}}{\pgfqpoint{0.761042in}{0.694258in}}{\pgfqpoint{0.761042in}{0.705309in}}%
\pgfpathcurveto{\pgfqpoint{0.761042in}{0.716359in}}{\pgfqpoint{0.756652in}{0.726958in}}{\pgfqpoint{0.748838in}{0.734771in}}%
\pgfpathcurveto{\pgfqpoint{0.741025in}{0.742585in}}{\pgfqpoint{0.730426in}{0.746975in}}{\pgfqpoint{0.719376in}{0.746975in}}%
\pgfpathcurveto{\pgfqpoint{0.708325in}{0.746975in}}{\pgfqpoint{0.697726in}{0.742585in}}{\pgfqpoint{0.689913in}{0.734771in}}%
\pgfpathcurveto{\pgfqpoint{0.682099in}{0.726958in}}{\pgfqpoint{0.677709in}{0.716359in}}{\pgfqpoint{0.677709in}{0.705309in}}%
\pgfpathcurveto{\pgfqpoint{0.677709in}{0.694258in}}{\pgfqpoint{0.682099in}{0.683659in}}{\pgfqpoint{0.689913in}{0.675846in}}%
\pgfpathcurveto{\pgfqpoint{0.697726in}{0.668032in}}{\pgfqpoint{0.708325in}{0.663642in}}{\pgfqpoint{0.719376in}{0.663642in}}%
\pgfpathclose%
\pgfusepath{stroke,fill}%
\end{pgfscope}%
\begin{pgfscope}%
\pgfpathrectangle{\pgfqpoint{0.648703in}{0.548769in}}{\pgfqpoint{5.112893in}{3.102590in}}%
\pgfusepath{clip}%
\pgfsetbuttcap%
\pgfsetroundjoin%
\definecolor{currentfill}{rgb}{0.121569,0.466667,0.705882}%
\pgfsetfillcolor{currentfill}%
\pgfsetlinewidth{1.003750pt}%
\definecolor{currentstroke}{rgb}{0.121569,0.466667,0.705882}%
\pgfsetstrokecolor{currentstroke}%
\pgfsetdash{}{0pt}%
\pgfpathmoveto{\pgfqpoint{0.724724in}{0.671892in}}%
\pgfpathcurveto{\pgfqpoint{0.735774in}{0.671892in}}{\pgfqpoint{0.746373in}{0.676282in}}{\pgfqpoint{0.754187in}{0.684096in}}%
\pgfpathcurveto{\pgfqpoint{0.762001in}{0.691909in}}{\pgfqpoint{0.766391in}{0.702509in}}{\pgfqpoint{0.766391in}{0.713559in}}%
\pgfpathcurveto{\pgfqpoint{0.766391in}{0.724609in}}{\pgfqpoint{0.762001in}{0.735208in}}{\pgfqpoint{0.754187in}{0.743021in}}%
\pgfpathcurveto{\pgfqpoint{0.746373in}{0.750835in}}{\pgfqpoint{0.735774in}{0.755225in}}{\pgfqpoint{0.724724in}{0.755225in}}%
\pgfpathcurveto{\pgfqpoint{0.713674in}{0.755225in}}{\pgfqpoint{0.703075in}{0.750835in}}{\pgfqpoint{0.695261in}{0.743021in}}%
\pgfpathcurveto{\pgfqpoint{0.687448in}{0.735208in}}{\pgfqpoint{0.683057in}{0.724609in}}{\pgfqpoint{0.683057in}{0.713559in}}%
\pgfpathcurveto{\pgfqpoint{0.683057in}{0.702509in}}{\pgfqpoint{0.687448in}{0.691909in}}{\pgfqpoint{0.695261in}{0.684096in}}%
\pgfpathcurveto{\pgfqpoint{0.703075in}{0.676282in}}{\pgfqpoint{0.713674in}{0.671892in}}{\pgfqpoint{0.724724in}{0.671892in}}%
\pgfpathclose%
\pgfusepath{stroke,fill}%
\end{pgfscope}%
\begin{pgfscope}%
\pgfpathrectangle{\pgfqpoint{0.648703in}{0.548769in}}{\pgfqpoint{5.112893in}{3.102590in}}%
\pgfusepath{clip}%
\pgfsetbuttcap%
\pgfsetroundjoin%
\definecolor{currentfill}{rgb}{0.121569,0.466667,0.705882}%
\pgfsetfillcolor{currentfill}%
\pgfsetlinewidth{1.003750pt}%
\definecolor{currentstroke}{rgb}{0.121569,0.466667,0.705882}%
\pgfsetstrokecolor{currentstroke}%
\pgfsetdash{}{0pt}%
\pgfpathmoveto{\pgfqpoint{0.841876in}{3.101537in}}%
\pgfpathcurveto{\pgfqpoint{0.852926in}{3.101537in}}{\pgfqpoint{0.863525in}{3.105927in}}{\pgfqpoint{0.871339in}{3.113741in}}%
\pgfpathcurveto{\pgfqpoint{0.879152in}{3.121555in}}{\pgfqpoint{0.883542in}{3.132154in}}{\pgfqpoint{0.883542in}{3.143204in}}%
\pgfpathcurveto{\pgfqpoint{0.883542in}{3.154254in}}{\pgfqpoint{0.879152in}{3.164853in}}{\pgfqpoint{0.871339in}{3.172667in}}%
\pgfpathcurveto{\pgfqpoint{0.863525in}{3.180480in}}{\pgfqpoint{0.852926in}{3.184870in}}{\pgfqpoint{0.841876in}{3.184870in}}%
\pgfpathcurveto{\pgfqpoint{0.830826in}{3.184870in}}{\pgfqpoint{0.820227in}{3.180480in}}{\pgfqpoint{0.812413in}{3.172667in}}%
\pgfpathcurveto{\pgfqpoint{0.804599in}{3.164853in}}{\pgfqpoint{0.800209in}{3.154254in}}{\pgfqpoint{0.800209in}{3.143204in}}%
\pgfpathcurveto{\pgfqpoint{0.800209in}{3.132154in}}{\pgfqpoint{0.804599in}{3.121555in}}{\pgfqpoint{0.812413in}{3.113741in}}%
\pgfpathcurveto{\pgfqpoint{0.820227in}{3.105927in}}{\pgfqpoint{0.830826in}{3.101537in}}{\pgfqpoint{0.841876in}{3.101537in}}%
\pgfpathclose%
\pgfusepath{stroke,fill}%
\end{pgfscope}%
\begin{pgfscope}%
\pgfpathrectangle{\pgfqpoint{0.648703in}{0.548769in}}{\pgfqpoint{5.112893in}{3.102590in}}%
\pgfusepath{clip}%
\pgfsetbuttcap%
\pgfsetroundjoin%
\definecolor{currentfill}{rgb}{1.000000,0.498039,0.054902}%
\pgfsetfillcolor{currentfill}%
\pgfsetlinewidth{1.003750pt}%
\definecolor{currentstroke}{rgb}{1.000000,0.498039,0.054902}%
\pgfsetstrokecolor{currentstroke}%
\pgfsetdash{}{0pt}%
\pgfpathmoveto{\pgfqpoint{1.613874in}{3.138662in}}%
\pgfpathcurveto{\pgfqpoint{1.624924in}{3.138662in}}{\pgfqpoint{1.635523in}{3.143053in}}{\pgfqpoint{1.643337in}{3.150866in}}%
\pgfpathcurveto{\pgfqpoint{1.651151in}{3.158680in}}{\pgfqpoint{1.655541in}{3.169279in}}{\pgfqpoint{1.655541in}{3.180329in}}%
\pgfpathcurveto{\pgfqpoint{1.655541in}{3.191379in}}{\pgfqpoint{1.651151in}{3.201978in}}{\pgfqpoint{1.643337in}{3.209792in}}%
\pgfpathcurveto{\pgfqpoint{1.635523in}{3.217605in}}{\pgfqpoint{1.624924in}{3.221996in}}{\pgfqpoint{1.613874in}{3.221996in}}%
\pgfpathcurveto{\pgfqpoint{1.602824in}{3.221996in}}{\pgfqpoint{1.592225in}{3.217605in}}{\pgfqpoint{1.584411in}{3.209792in}}%
\pgfpathcurveto{\pgfqpoint{1.576598in}{3.201978in}}{\pgfqpoint{1.572208in}{3.191379in}}{\pgfqpoint{1.572208in}{3.180329in}}%
\pgfpathcurveto{\pgfqpoint{1.572208in}{3.169279in}}{\pgfqpoint{1.576598in}{3.158680in}}{\pgfqpoint{1.584411in}{3.150866in}}%
\pgfpathcurveto{\pgfqpoint{1.592225in}{3.143053in}}{\pgfqpoint{1.602824in}{3.138662in}}{\pgfqpoint{1.613874in}{3.138662in}}%
\pgfpathclose%
\pgfusepath{stroke,fill}%
\end{pgfscope}%
\begin{pgfscope}%
\pgfpathrectangle{\pgfqpoint{0.648703in}{0.548769in}}{\pgfqpoint{5.112893in}{3.102590in}}%
\pgfusepath{clip}%
\pgfsetbuttcap%
\pgfsetroundjoin%
\definecolor{currentfill}{rgb}{0.121569,0.466667,0.705882}%
\pgfsetfillcolor{currentfill}%
\pgfsetlinewidth{1.003750pt}%
\definecolor{currentstroke}{rgb}{0.121569,0.466667,0.705882}%
\pgfsetstrokecolor{currentstroke}%
\pgfsetdash{}{0pt}%
\pgfpathmoveto{\pgfqpoint{0.719376in}{0.663642in}}%
\pgfpathcurveto{\pgfqpoint{0.730426in}{0.663642in}}{\pgfqpoint{0.741025in}{0.668032in}}{\pgfqpoint{0.748838in}{0.675846in}}%
\pgfpathcurveto{\pgfqpoint{0.756652in}{0.683659in}}{\pgfqpoint{0.761042in}{0.694258in}}{\pgfqpoint{0.761042in}{0.705309in}}%
\pgfpathcurveto{\pgfqpoint{0.761042in}{0.716359in}}{\pgfqpoint{0.756652in}{0.726958in}}{\pgfqpoint{0.748838in}{0.734771in}}%
\pgfpathcurveto{\pgfqpoint{0.741025in}{0.742585in}}{\pgfqpoint{0.730426in}{0.746975in}}{\pgfqpoint{0.719376in}{0.746975in}}%
\pgfpathcurveto{\pgfqpoint{0.708325in}{0.746975in}}{\pgfqpoint{0.697726in}{0.742585in}}{\pgfqpoint{0.689913in}{0.734771in}}%
\pgfpathcurveto{\pgfqpoint{0.682099in}{0.726958in}}{\pgfqpoint{0.677709in}{0.716359in}}{\pgfqpoint{0.677709in}{0.705309in}}%
\pgfpathcurveto{\pgfqpoint{0.677709in}{0.694258in}}{\pgfqpoint{0.682099in}{0.683659in}}{\pgfqpoint{0.689913in}{0.675846in}}%
\pgfpathcurveto{\pgfqpoint{0.697726in}{0.668032in}}{\pgfqpoint{0.708325in}{0.663642in}}{\pgfqpoint{0.719376in}{0.663642in}}%
\pgfpathclose%
\pgfusepath{stroke,fill}%
\end{pgfscope}%
\begin{pgfscope}%
\pgfpathrectangle{\pgfqpoint{0.648703in}{0.548769in}}{\pgfqpoint{5.112893in}{3.102590in}}%
\pgfusepath{clip}%
\pgfsetbuttcap%
\pgfsetroundjoin%
\definecolor{currentfill}{rgb}{0.121569,0.466667,0.705882}%
\pgfsetfillcolor{currentfill}%
\pgfsetlinewidth{1.003750pt}%
\definecolor{currentstroke}{rgb}{0.121569,0.466667,0.705882}%
\pgfsetstrokecolor{currentstroke}%
\pgfsetdash{}{0pt}%
\pgfpathmoveto{\pgfqpoint{0.719376in}{0.663642in}}%
\pgfpathcurveto{\pgfqpoint{0.730426in}{0.663642in}}{\pgfqpoint{0.741025in}{0.668032in}}{\pgfqpoint{0.748838in}{0.675846in}}%
\pgfpathcurveto{\pgfqpoint{0.756652in}{0.683659in}}{\pgfqpoint{0.761042in}{0.694258in}}{\pgfqpoint{0.761042in}{0.705309in}}%
\pgfpathcurveto{\pgfqpoint{0.761042in}{0.716359in}}{\pgfqpoint{0.756652in}{0.726958in}}{\pgfqpoint{0.748838in}{0.734771in}}%
\pgfpathcurveto{\pgfqpoint{0.741025in}{0.742585in}}{\pgfqpoint{0.730426in}{0.746975in}}{\pgfqpoint{0.719376in}{0.746975in}}%
\pgfpathcurveto{\pgfqpoint{0.708325in}{0.746975in}}{\pgfqpoint{0.697726in}{0.742585in}}{\pgfqpoint{0.689913in}{0.734771in}}%
\pgfpathcurveto{\pgfqpoint{0.682099in}{0.726958in}}{\pgfqpoint{0.677709in}{0.716359in}}{\pgfqpoint{0.677709in}{0.705309in}}%
\pgfpathcurveto{\pgfqpoint{0.677709in}{0.694258in}}{\pgfqpoint{0.682099in}{0.683659in}}{\pgfqpoint{0.689913in}{0.675846in}}%
\pgfpathcurveto{\pgfqpoint{0.697726in}{0.668032in}}{\pgfqpoint{0.708325in}{0.663642in}}{\pgfqpoint{0.719376in}{0.663642in}}%
\pgfpathclose%
\pgfusepath{stroke,fill}%
\end{pgfscope}%
\begin{pgfscope}%
\pgfpathrectangle{\pgfqpoint{0.648703in}{0.548769in}}{\pgfqpoint{5.112893in}{3.102590in}}%
\pgfusepath{clip}%
\pgfsetbuttcap%
\pgfsetroundjoin%
\definecolor{currentfill}{rgb}{0.839216,0.152941,0.156863}%
\pgfsetfillcolor{currentfill}%
\pgfsetlinewidth{1.003750pt}%
\definecolor{currentstroke}{rgb}{0.839216,0.152941,0.156863}%
\pgfsetstrokecolor{currentstroke}%
\pgfsetdash{}{0pt}%
\pgfpathmoveto{\pgfqpoint{1.677595in}{3.410915in}}%
\pgfpathcurveto{\pgfqpoint{1.688645in}{3.410915in}}{\pgfqpoint{1.699244in}{3.415305in}}{\pgfqpoint{1.707058in}{3.423119in}}%
\pgfpathcurveto{\pgfqpoint{1.714871in}{3.430932in}}{\pgfqpoint{1.719262in}{3.441531in}}{\pgfqpoint{1.719262in}{3.452581in}}%
\pgfpathcurveto{\pgfqpoint{1.719262in}{3.463631in}}{\pgfqpoint{1.714871in}{3.474230in}}{\pgfqpoint{1.707058in}{3.482044in}}%
\pgfpathcurveto{\pgfqpoint{1.699244in}{3.489858in}}{\pgfqpoint{1.688645in}{3.494248in}}{\pgfqpoint{1.677595in}{3.494248in}}%
\pgfpathcurveto{\pgfqpoint{1.666545in}{3.494248in}}{\pgfqpoint{1.655946in}{3.489858in}}{\pgfqpoint{1.648132in}{3.482044in}}%
\pgfpathcurveto{\pgfqpoint{1.640319in}{3.474230in}}{\pgfqpoint{1.635928in}{3.463631in}}{\pgfqpoint{1.635928in}{3.452581in}}%
\pgfpathcurveto{\pgfqpoint{1.635928in}{3.441531in}}{\pgfqpoint{1.640319in}{3.430932in}}{\pgfqpoint{1.648132in}{3.423119in}}%
\pgfpathcurveto{\pgfqpoint{1.655946in}{3.415305in}}{\pgfqpoint{1.666545in}{3.410915in}}{\pgfqpoint{1.677595in}{3.410915in}}%
\pgfpathclose%
\pgfusepath{stroke,fill}%
\end{pgfscope}%
\begin{pgfscope}%
\pgfpathrectangle{\pgfqpoint{0.648703in}{0.548769in}}{\pgfqpoint{5.112893in}{3.102590in}}%
\pgfusepath{clip}%
\pgfsetbuttcap%
\pgfsetroundjoin%
\definecolor{currentfill}{rgb}{0.121569,0.466667,0.705882}%
\pgfsetfillcolor{currentfill}%
\pgfsetlinewidth{1.003750pt}%
\definecolor{currentstroke}{rgb}{0.121569,0.466667,0.705882}%
\pgfsetstrokecolor{currentstroke}%
\pgfsetdash{}{0pt}%
\pgfpathmoveto{\pgfqpoint{1.578898in}{3.130412in}}%
\pgfpathcurveto{\pgfqpoint{1.589948in}{3.130412in}}{\pgfqpoint{1.600547in}{3.134803in}}{\pgfqpoint{1.608360in}{3.142616in}}%
\pgfpathcurveto{\pgfqpoint{1.616174in}{3.150430in}}{\pgfqpoint{1.620564in}{3.161029in}}{\pgfqpoint{1.620564in}{3.172079in}}%
\pgfpathcurveto{\pgfqpoint{1.620564in}{3.183129in}}{\pgfqpoint{1.616174in}{3.193728in}}{\pgfqpoint{1.608360in}{3.201542in}}%
\pgfpathcurveto{\pgfqpoint{1.600547in}{3.209355in}}{\pgfqpoint{1.589948in}{3.213746in}}{\pgfqpoint{1.578898in}{3.213746in}}%
\pgfpathcurveto{\pgfqpoint{1.567847in}{3.213746in}}{\pgfqpoint{1.557248in}{3.209355in}}{\pgfqpoint{1.549435in}{3.201542in}}%
\pgfpathcurveto{\pgfqpoint{1.541621in}{3.193728in}}{\pgfqpoint{1.537231in}{3.183129in}}{\pgfqpoint{1.537231in}{3.172079in}}%
\pgfpathcurveto{\pgfqpoint{1.537231in}{3.161029in}}{\pgfqpoint{1.541621in}{3.150430in}}{\pgfqpoint{1.549435in}{3.142616in}}%
\pgfpathcurveto{\pgfqpoint{1.557248in}{3.134803in}}{\pgfqpoint{1.567847in}{3.130412in}}{\pgfqpoint{1.578898in}{3.130412in}}%
\pgfpathclose%
\pgfusepath{stroke,fill}%
\end{pgfscope}%
\begin{pgfscope}%
\pgfpathrectangle{\pgfqpoint{0.648703in}{0.548769in}}{\pgfqpoint{5.112893in}{3.102590in}}%
\pgfusepath{clip}%
\pgfsetbuttcap%
\pgfsetroundjoin%
\definecolor{currentfill}{rgb}{1.000000,0.498039,0.054902}%
\pgfsetfillcolor{currentfill}%
\pgfsetlinewidth{1.003750pt}%
\definecolor{currentstroke}{rgb}{1.000000,0.498039,0.054902}%
\pgfsetstrokecolor{currentstroke}%
\pgfsetdash{}{0pt}%
\pgfpathmoveto{\pgfqpoint{1.623586in}{3.316039in}}%
\pgfpathcurveto{\pgfqpoint{1.634636in}{3.316039in}}{\pgfqpoint{1.645235in}{3.320429in}}{\pgfqpoint{1.653049in}{3.328243in}}%
\pgfpathcurveto{\pgfqpoint{1.660863in}{3.336056in}}{\pgfqpoint{1.665253in}{3.346655in}}{\pgfqpoint{1.665253in}{3.357706in}}%
\pgfpathcurveto{\pgfqpoint{1.665253in}{3.368756in}}{\pgfqpoint{1.660863in}{3.379355in}}{\pgfqpoint{1.653049in}{3.387168in}}%
\pgfpathcurveto{\pgfqpoint{1.645235in}{3.394982in}}{\pgfqpoint{1.634636in}{3.399372in}}{\pgfqpoint{1.623586in}{3.399372in}}%
\pgfpathcurveto{\pgfqpoint{1.612536in}{3.399372in}}{\pgfqpoint{1.601937in}{3.394982in}}{\pgfqpoint{1.594123in}{3.387168in}}%
\pgfpathcurveto{\pgfqpoint{1.586310in}{3.379355in}}{\pgfqpoint{1.581920in}{3.368756in}}{\pgfqpoint{1.581920in}{3.357706in}}%
\pgfpathcurveto{\pgfqpoint{1.581920in}{3.346655in}}{\pgfqpoint{1.586310in}{3.336056in}}{\pgfqpoint{1.594123in}{3.328243in}}%
\pgfpathcurveto{\pgfqpoint{1.601937in}{3.320429in}}{\pgfqpoint{1.612536in}{3.316039in}}{\pgfqpoint{1.623586in}{3.316039in}}%
\pgfpathclose%
\pgfusepath{stroke,fill}%
\end{pgfscope}%
\begin{pgfscope}%
\pgfpathrectangle{\pgfqpoint{0.648703in}{0.548769in}}{\pgfqpoint{5.112893in}{3.102590in}}%
\pgfusepath{clip}%
\pgfsetbuttcap%
\pgfsetroundjoin%
\definecolor{currentfill}{rgb}{1.000000,0.498039,0.054902}%
\pgfsetfillcolor{currentfill}%
\pgfsetlinewidth{1.003750pt}%
\definecolor{currentstroke}{rgb}{1.000000,0.498039,0.054902}%
\pgfsetstrokecolor{currentstroke}%
\pgfsetdash{}{0pt}%
\pgfpathmoveto{\pgfqpoint{1.520788in}{3.138662in}}%
\pgfpathcurveto{\pgfqpoint{1.531838in}{3.138662in}}{\pgfqpoint{1.542437in}{3.143053in}}{\pgfqpoint{1.550250in}{3.150866in}}%
\pgfpathcurveto{\pgfqpoint{1.558064in}{3.158680in}}{\pgfqpoint{1.562454in}{3.169279in}}{\pgfqpoint{1.562454in}{3.180329in}}%
\pgfpathcurveto{\pgfqpoint{1.562454in}{3.191379in}}{\pgfqpoint{1.558064in}{3.201978in}}{\pgfqpoint{1.550250in}{3.209792in}}%
\pgfpathcurveto{\pgfqpoint{1.542437in}{3.217605in}}{\pgfqpoint{1.531838in}{3.221996in}}{\pgfqpoint{1.520788in}{3.221996in}}%
\pgfpathcurveto{\pgfqpoint{1.509738in}{3.221996in}}{\pgfqpoint{1.499138in}{3.217605in}}{\pgfqpoint{1.491325in}{3.209792in}}%
\pgfpathcurveto{\pgfqpoint{1.483511in}{3.201978in}}{\pgfqpoint{1.479121in}{3.191379in}}{\pgfqpoint{1.479121in}{3.180329in}}%
\pgfpathcurveto{\pgfqpoint{1.479121in}{3.169279in}}{\pgfqpoint{1.483511in}{3.158680in}}{\pgfqpoint{1.491325in}{3.150866in}}%
\pgfpathcurveto{\pgfqpoint{1.499138in}{3.143053in}}{\pgfqpoint{1.509738in}{3.138662in}}{\pgfqpoint{1.520788in}{3.138662in}}%
\pgfpathclose%
\pgfusepath{stroke,fill}%
\end{pgfscope}%
\begin{pgfscope}%
\pgfpathrectangle{\pgfqpoint{0.648703in}{0.548769in}}{\pgfqpoint{5.112893in}{3.102590in}}%
\pgfusepath{clip}%
\pgfsetbuttcap%
\pgfsetroundjoin%
\definecolor{currentfill}{rgb}{1.000000,0.498039,0.054902}%
\pgfsetfillcolor{currentfill}%
\pgfsetlinewidth{1.003750pt}%
\definecolor{currentstroke}{rgb}{1.000000,0.498039,0.054902}%
\pgfsetstrokecolor{currentstroke}%
\pgfsetdash{}{0pt}%
\pgfpathmoveto{\pgfqpoint{1.622838in}{3.138662in}}%
\pgfpathcurveto{\pgfqpoint{1.633888in}{3.138662in}}{\pgfqpoint{1.644487in}{3.143053in}}{\pgfqpoint{1.652301in}{3.150866in}}%
\pgfpathcurveto{\pgfqpoint{1.660114in}{3.158680in}}{\pgfqpoint{1.664505in}{3.169279in}}{\pgfqpoint{1.664505in}{3.180329in}}%
\pgfpathcurveto{\pgfqpoint{1.664505in}{3.191379in}}{\pgfqpoint{1.660114in}{3.201978in}}{\pgfqpoint{1.652301in}{3.209792in}}%
\pgfpathcurveto{\pgfqpoint{1.644487in}{3.217605in}}{\pgfqpoint{1.633888in}{3.221996in}}{\pgfqpoint{1.622838in}{3.221996in}}%
\pgfpathcurveto{\pgfqpoint{1.611788in}{3.221996in}}{\pgfqpoint{1.601189in}{3.217605in}}{\pgfqpoint{1.593375in}{3.209792in}}%
\pgfpathcurveto{\pgfqpoint{1.585561in}{3.201978in}}{\pgfqpoint{1.581171in}{3.191379in}}{\pgfqpoint{1.581171in}{3.180329in}}%
\pgfpathcurveto{\pgfqpoint{1.581171in}{3.169279in}}{\pgfqpoint{1.585561in}{3.158680in}}{\pgfqpoint{1.593375in}{3.150866in}}%
\pgfpathcurveto{\pgfqpoint{1.601189in}{3.143053in}}{\pgfqpoint{1.611788in}{3.138662in}}{\pgfqpoint{1.622838in}{3.138662in}}%
\pgfpathclose%
\pgfusepath{stroke,fill}%
\end{pgfscope}%
\begin{pgfscope}%
\pgfpathrectangle{\pgfqpoint{0.648703in}{0.548769in}}{\pgfqpoint{5.112893in}{3.102590in}}%
\pgfusepath{clip}%
\pgfsetbuttcap%
\pgfsetroundjoin%
\definecolor{currentfill}{rgb}{1.000000,0.498039,0.054902}%
\pgfsetfillcolor{currentfill}%
\pgfsetlinewidth{1.003750pt}%
\definecolor{currentstroke}{rgb}{1.000000,0.498039,0.054902}%
\pgfsetstrokecolor{currentstroke}%
\pgfsetdash{}{0pt}%
\pgfpathmoveto{\pgfqpoint{1.124501in}{3.142787in}}%
\pgfpathcurveto{\pgfqpoint{1.135551in}{3.142787in}}{\pgfqpoint{1.146150in}{3.147178in}}{\pgfqpoint{1.153964in}{3.154991in}}%
\pgfpathcurveto{\pgfqpoint{1.161777in}{3.162805in}}{\pgfqpoint{1.166167in}{3.173404in}}{\pgfqpoint{1.166167in}{3.184454in}}%
\pgfpathcurveto{\pgfqpoint{1.166167in}{3.195504in}}{\pgfqpoint{1.161777in}{3.206103in}}{\pgfqpoint{1.153964in}{3.213917in}}%
\pgfpathcurveto{\pgfqpoint{1.146150in}{3.221731in}}{\pgfqpoint{1.135551in}{3.226121in}}{\pgfqpoint{1.124501in}{3.226121in}}%
\pgfpathcurveto{\pgfqpoint{1.113451in}{3.226121in}}{\pgfqpoint{1.102852in}{3.221731in}}{\pgfqpoint{1.095038in}{3.213917in}}%
\pgfpathcurveto{\pgfqpoint{1.087224in}{3.206103in}}{\pgfqpoint{1.082834in}{3.195504in}}{\pgfqpoint{1.082834in}{3.184454in}}%
\pgfpathcurveto{\pgfqpoint{1.082834in}{3.173404in}}{\pgfqpoint{1.087224in}{3.162805in}}{\pgfqpoint{1.095038in}{3.154991in}}%
\pgfpathcurveto{\pgfqpoint{1.102852in}{3.147178in}}{\pgfqpoint{1.113451in}{3.142787in}}{\pgfqpoint{1.124501in}{3.142787in}}%
\pgfpathclose%
\pgfusepath{stroke,fill}%
\end{pgfscope}%
\begin{pgfscope}%
\pgfpathrectangle{\pgfqpoint{0.648703in}{0.548769in}}{\pgfqpoint{5.112893in}{3.102590in}}%
\pgfusepath{clip}%
\pgfsetbuttcap%
\pgfsetroundjoin%
\definecolor{currentfill}{rgb}{1.000000,0.498039,0.054902}%
\pgfsetfillcolor{currentfill}%
\pgfsetlinewidth{1.003750pt}%
\definecolor{currentstroke}{rgb}{1.000000,0.498039,0.054902}%
\pgfsetstrokecolor{currentstroke}%
\pgfsetdash{}{0pt}%
\pgfpathmoveto{\pgfqpoint{1.589149in}{3.138662in}}%
\pgfpathcurveto{\pgfqpoint{1.600199in}{3.138662in}}{\pgfqpoint{1.610798in}{3.143053in}}{\pgfqpoint{1.618612in}{3.150866in}}%
\pgfpathcurveto{\pgfqpoint{1.626425in}{3.158680in}}{\pgfqpoint{1.630816in}{3.169279in}}{\pgfqpoint{1.630816in}{3.180329in}}%
\pgfpathcurveto{\pgfqpoint{1.630816in}{3.191379in}}{\pgfqpoint{1.626425in}{3.201978in}}{\pgfqpoint{1.618612in}{3.209792in}}%
\pgfpathcurveto{\pgfqpoint{1.610798in}{3.217605in}}{\pgfqpoint{1.600199in}{3.221996in}}{\pgfqpoint{1.589149in}{3.221996in}}%
\pgfpathcurveto{\pgfqpoint{1.578099in}{3.221996in}}{\pgfqpoint{1.567500in}{3.217605in}}{\pgfqpoint{1.559686in}{3.209792in}}%
\pgfpathcurveto{\pgfqpoint{1.551873in}{3.201978in}}{\pgfqpoint{1.547482in}{3.191379in}}{\pgfqpoint{1.547482in}{3.180329in}}%
\pgfpathcurveto{\pgfqpoint{1.547482in}{3.169279in}}{\pgfqpoint{1.551873in}{3.158680in}}{\pgfqpoint{1.559686in}{3.150866in}}%
\pgfpathcurveto{\pgfqpoint{1.567500in}{3.143053in}}{\pgfqpoint{1.578099in}{3.138662in}}{\pgfqpoint{1.589149in}{3.138662in}}%
\pgfpathclose%
\pgfusepath{stroke,fill}%
\end{pgfscope}%
\begin{pgfscope}%
\pgfpathrectangle{\pgfqpoint{0.648703in}{0.548769in}}{\pgfqpoint{5.112893in}{3.102590in}}%
\pgfusepath{clip}%
\pgfsetbuttcap%
\pgfsetroundjoin%
\definecolor{currentfill}{rgb}{1.000000,0.498039,0.054902}%
\pgfsetfillcolor{currentfill}%
\pgfsetlinewidth{1.003750pt}%
\definecolor{currentstroke}{rgb}{1.000000,0.498039,0.054902}%
\pgfsetstrokecolor{currentstroke}%
\pgfsetdash{}{0pt}%
\pgfpathmoveto{\pgfqpoint{1.708431in}{3.357289in}}%
\pgfpathcurveto{\pgfqpoint{1.719482in}{3.357289in}}{\pgfqpoint{1.730081in}{3.361679in}}{\pgfqpoint{1.737894in}{3.369493in}}%
\pgfpathcurveto{\pgfqpoint{1.745708in}{3.377307in}}{\pgfqpoint{1.750098in}{3.387906in}}{\pgfqpoint{1.750098in}{3.398956in}}%
\pgfpathcurveto{\pgfqpoint{1.750098in}{3.410006in}}{\pgfqpoint{1.745708in}{3.420605in}}{\pgfqpoint{1.737894in}{3.428419in}}%
\pgfpathcurveto{\pgfqpoint{1.730081in}{3.436232in}}{\pgfqpoint{1.719482in}{3.440623in}}{\pgfqpoint{1.708431in}{3.440623in}}%
\pgfpathcurveto{\pgfqpoint{1.697381in}{3.440623in}}{\pgfqpoint{1.686782in}{3.436232in}}{\pgfqpoint{1.678969in}{3.428419in}}%
\pgfpathcurveto{\pgfqpoint{1.671155in}{3.420605in}}{\pgfqpoint{1.666765in}{3.410006in}}{\pgfqpoint{1.666765in}{3.398956in}}%
\pgfpathcurveto{\pgfqpoint{1.666765in}{3.387906in}}{\pgfqpoint{1.671155in}{3.377307in}}{\pgfqpoint{1.678969in}{3.369493in}}%
\pgfpathcurveto{\pgfqpoint{1.686782in}{3.361679in}}{\pgfqpoint{1.697381in}{3.357289in}}{\pgfqpoint{1.708431in}{3.357289in}}%
\pgfpathclose%
\pgfusepath{stroke,fill}%
\end{pgfscope}%
\begin{pgfscope}%
\pgfpathrectangle{\pgfqpoint{0.648703in}{0.548769in}}{\pgfqpoint{5.112893in}{3.102590in}}%
\pgfusepath{clip}%
\pgfsetbuttcap%
\pgfsetroundjoin%
\definecolor{currentfill}{rgb}{1.000000,0.498039,0.054902}%
\pgfsetfillcolor{currentfill}%
\pgfsetlinewidth{1.003750pt}%
\definecolor{currentstroke}{rgb}{1.000000,0.498039,0.054902}%
\pgfsetstrokecolor{currentstroke}%
\pgfsetdash{}{0pt}%
\pgfpathmoveto{\pgfqpoint{2.132817in}{3.221163in}}%
\pgfpathcurveto{\pgfqpoint{2.143868in}{3.221163in}}{\pgfqpoint{2.154467in}{3.225553in}}{\pgfqpoint{2.162280in}{3.233367in}}%
\pgfpathcurveto{\pgfqpoint{2.170094in}{3.241181in}}{\pgfqpoint{2.174484in}{3.251780in}}{\pgfqpoint{2.174484in}{3.262830in}}%
\pgfpathcurveto{\pgfqpoint{2.174484in}{3.273880in}}{\pgfqpoint{2.170094in}{3.284479in}}{\pgfqpoint{2.162280in}{3.292293in}}%
\pgfpathcurveto{\pgfqpoint{2.154467in}{3.300106in}}{\pgfqpoint{2.143868in}{3.304496in}}{\pgfqpoint{2.132817in}{3.304496in}}%
\pgfpathcurveto{\pgfqpoint{2.121767in}{3.304496in}}{\pgfqpoint{2.111168in}{3.300106in}}{\pgfqpoint{2.103355in}{3.292293in}}%
\pgfpathcurveto{\pgfqpoint{2.095541in}{3.284479in}}{\pgfqpoint{2.091151in}{3.273880in}}{\pgfqpoint{2.091151in}{3.262830in}}%
\pgfpathcurveto{\pgfqpoint{2.091151in}{3.251780in}}{\pgfqpoint{2.095541in}{3.241181in}}{\pgfqpoint{2.103355in}{3.233367in}}%
\pgfpathcurveto{\pgfqpoint{2.111168in}{3.225553in}}{\pgfqpoint{2.121767in}{3.221163in}}{\pgfqpoint{2.132817in}{3.221163in}}%
\pgfpathclose%
\pgfusepath{stroke,fill}%
\end{pgfscope}%
\begin{pgfscope}%
\pgfpathrectangle{\pgfqpoint{0.648703in}{0.548769in}}{\pgfqpoint{5.112893in}{3.102590in}}%
\pgfusepath{clip}%
\pgfsetbuttcap%
\pgfsetroundjoin%
\definecolor{currentfill}{rgb}{0.121569,0.466667,0.705882}%
\pgfsetfillcolor{currentfill}%
\pgfsetlinewidth{1.003750pt}%
\definecolor{currentstroke}{rgb}{0.121569,0.466667,0.705882}%
\pgfsetstrokecolor{currentstroke}%
\pgfsetdash{}{0pt}%
\pgfpathmoveto{\pgfqpoint{1.573933in}{3.122162in}}%
\pgfpathcurveto{\pgfqpoint{1.584983in}{3.122162in}}{\pgfqpoint{1.595582in}{3.126553in}}{\pgfqpoint{1.603396in}{3.134366in}}%
\pgfpathcurveto{\pgfqpoint{1.611209in}{3.142180in}}{\pgfqpoint{1.615600in}{3.152779in}}{\pgfqpoint{1.615600in}{3.163829in}}%
\pgfpathcurveto{\pgfqpoint{1.615600in}{3.174879in}}{\pgfqpoint{1.611209in}{3.185478in}}{\pgfqpoint{1.603396in}{3.193292in}}%
\pgfpathcurveto{\pgfqpoint{1.595582in}{3.201105in}}{\pgfqpoint{1.584983in}{3.205496in}}{\pgfqpoint{1.573933in}{3.205496in}}%
\pgfpathcurveto{\pgfqpoint{1.562883in}{3.205496in}}{\pgfqpoint{1.552284in}{3.201105in}}{\pgfqpoint{1.544470in}{3.193292in}}%
\pgfpathcurveto{\pgfqpoint{1.536657in}{3.185478in}}{\pgfqpoint{1.532266in}{3.174879in}}{\pgfqpoint{1.532266in}{3.163829in}}%
\pgfpathcurveto{\pgfqpoint{1.532266in}{3.152779in}}{\pgfqpoint{1.536657in}{3.142180in}}{\pgfqpoint{1.544470in}{3.134366in}}%
\pgfpathcurveto{\pgfqpoint{1.552284in}{3.126553in}}{\pgfqpoint{1.562883in}{3.122162in}}{\pgfqpoint{1.573933in}{3.122162in}}%
\pgfpathclose%
\pgfusepath{stroke,fill}%
\end{pgfscope}%
\begin{pgfscope}%
\pgfpathrectangle{\pgfqpoint{0.648703in}{0.548769in}}{\pgfqpoint{5.112893in}{3.102590in}}%
\pgfusepath{clip}%
\pgfsetbuttcap%
\pgfsetroundjoin%
\definecolor{currentfill}{rgb}{1.000000,0.498039,0.054902}%
\pgfsetfillcolor{currentfill}%
\pgfsetlinewidth{1.003750pt}%
\definecolor{currentstroke}{rgb}{1.000000,0.498039,0.054902}%
\pgfsetstrokecolor{currentstroke}%
\pgfsetdash{}{0pt}%
\pgfpathmoveto{\pgfqpoint{1.724749in}{3.142787in}}%
\pgfpathcurveto{\pgfqpoint{1.735799in}{3.142787in}}{\pgfqpoint{1.746398in}{3.147178in}}{\pgfqpoint{1.754211in}{3.154991in}}%
\pgfpathcurveto{\pgfqpoint{1.762025in}{3.162805in}}{\pgfqpoint{1.766415in}{3.173404in}}{\pgfqpoint{1.766415in}{3.184454in}}%
\pgfpathcurveto{\pgfqpoint{1.766415in}{3.195504in}}{\pgfqpoint{1.762025in}{3.206103in}}{\pgfqpoint{1.754211in}{3.213917in}}%
\pgfpathcurveto{\pgfqpoint{1.746398in}{3.221731in}}{\pgfqpoint{1.735799in}{3.226121in}}{\pgfqpoint{1.724749in}{3.226121in}}%
\pgfpathcurveto{\pgfqpoint{1.713698in}{3.226121in}}{\pgfqpoint{1.703099in}{3.221731in}}{\pgfqpoint{1.695286in}{3.213917in}}%
\pgfpathcurveto{\pgfqpoint{1.687472in}{3.206103in}}{\pgfqpoint{1.683082in}{3.195504in}}{\pgfqpoint{1.683082in}{3.184454in}}%
\pgfpathcurveto{\pgfqpoint{1.683082in}{3.173404in}}{\pgfqpoint{1.687472in}{3.162805in}}{\pgfqpoint{1.695286in}{3.154991in}}%
\pgfpathcurveto{\pgfqpoint{1.703099in}{3.147178in}}{\pgfqpoint{1.713698in}{3.142787in}}{\pgfqpoint{1.724749in}{3.142787in}}%
\pgfpathclose%
\pgfusepath{stroke,fill}%
\end{pgfscope}%
\begin{pgfscope}%
\pgfpathrectangle{\pgfqpoint{0.648703in}{0.548769in}}{\pgfqpoint{5.112893in}{3.102590in}}%
\pgfusepath{clip}%
\pgfsetbuttcap%
\pgfsetroundjoin%
\definecolor{currentfill}{rgb}{1.000000,0.498039,0.054902}%
\pgfsetfillcolor{currentfill}%
\pgfsetlinewidth{1.003750pt}%
\definecolor{currentstroke}{rgb}{1.000000,0.498039,0.054902}%
\pgfsetstrokecolor{currentstroke}%
\pgfsetdash{}{0pt}%
\pgfpathmoveto{\pgfqpoint{1.745488in}{3.146912in}}%
\pgfpathcurveto{\pgfqpoint{1.756538in}{3.146912in}}{\pgfqpoint{1.767137in}{3.151303in}}{\pgfqpoint{1.774951in}{3.159116in}}%
\pgfpathcurveto{\pgfqpoint{1.782764in}{3.166930in}}{\pgfqpoint{1.787155in}{3.177529in}}{\pgfqpoint{1.787155in}{3.188579in}}%
\pgfpathcurveto{\pgfqpoint{1.787155in}{3.199629in}}{\pgfqpoint{1.782764in}{3.210228in}}{\pgfqpoint{1.774951in}{3.218042in}}%
\pgfpathcurveto{\pgfqpoint{1.767137in}{3.225856in}}{\pgfqpoint{1.756538in}{3.230246in}}{\pgfqpoint{1.745488in}{3.230246in}}%
\pgfpathcurveto{\pgfqpoint{1.734438in}{3.230246in}}{\pgfqpoint{1.723839in}{3.225856in}}{\pgfqpoint{1.716025in}{3.218042in}}%
\pgfpathcurveto{\pgfqpoint{1.708211in}{3.210228in}}{\pgfqpoint{1.703821in}{3.199629in}}{\pgfqpoint{1.703821in}{3.188579in}}%
\pgfpathcurveto{\pgfqpoint{1.703821in}{3.177529in}}{\pgfqpoint{1.708211in}{3.166930in}}{\pgfqpoint{1.716025in}{3.159116in}}%
\pgfpathcurveto{\pgfqpoint{1.723839in}{3.151303in}}{\pgfqpoint{1.734438in}{3.146912in}}{\pgfqpoint{1.745488in}{3.146912in}}%
\pgfpathclose%
\pgfusepath{stroke,fill}%
\end{pgfscope}%
\begin{pgfscope}%
\pgfpathrectangle{\pgfqpoint{0.648703in}{0.548769in}}{\pgfqpoint{5.112893in}{3.102590in}}%
\pgfusepath{clip}%
\pgfsetbuttcap%
\pgfsetroundjoin%
\definecolor{currentfill}{rgb}{1.000000,0.498039,0.054902}%
\pgfsetfillcolor{currentfill}%
\pgfsetlinewidth{1.003750pt}%
\definecolor{currentstroke}{rgb}{1.000000,0.498039,0.054902}%
\pgfsetstrokecolor{currentstroke}%
\pgfsetdash{}{0pt}%
\pgfpathmoveto{\pgfqpoint{0.914586in}{3.138662in}}%
\pgfpathcurveto{\pgfqpoint{0.925636in}{3.138662in}}{\pgfqpoint{0.936235in}{3.143053in}}{\pgfqpoint{0.944049in}{3.150866in}}%
\pgfpathcurveto{\pgfqpoint{0.951862in}{3.158680in}}{\pgfqpoint{0.956253in}{3.169279in}}{\pgfqpoint{0.956253in}{3.180329in}}%
\pgfpathcurveto{\pgfqpoint{0.956253in}{3.191379in}}{\pgfqpoint{0.951862in}{3.201978in}}{\pgfqpoint{0.944049in}{3.209792in}}%
\pgfpathcurveto{\pgfqpoint{0.936235in}{3.217605in}}{\pgfqpoint{0.925636in}{3.221996in}}{\pgfqpoint{0.914586in}{3.221996in}}%
\pgfpathcurveto{\pgfqpoint{0.903536in}{3.221996in}}{\pgfqpoint{0.892937in}{3.217605in}}{\pgfqpoint{0.885123in}{3.209792in}}%
\pgfpathcurveto{\pgfqpoint{0.877310in}{3.201978in}}{\pgfqpoint{0.872919in}{3.191379in}}{\pgfqpoint{0.872919in}{3.180329in}}%
\pgfpathcurveto{\pgfqpoint{0.872919in}{3.169279in}}{\pgfqpoint{0.877310in}{3.158680in}}{\pgfqpoint{0.885123in}{3.150866in}}%
\pgfpathcurveto{\pgfqpoint{0.892937in}{3.143053in}}{\pgfqpoint{0.903536in}{3.138662in}}{\pgfqpoint{0.914586in}{3.138662in}}%
\pgfpathclose%
\pgfusepath{stroke,fill}%
\end{pgfscope}%
\begin{pgfscope}%
\pgfpathrectangle{\pgfqpoint{0.648703in}{0.548769in}}{\pgfqpoint{5.112893in}{3.102590in}}%
\pgfusepath{clip}%
\pgfsetbuttcap%
\pgfsetroundjoin%
\definecolor{currentfill}{rgb}{0.121569,0.466667,0.705882}%
\pgfsetfillcolor{currentfill}%
\pgfsetlinewidth{1.003750pt}%
\definecolor{currentstroke}{rgb}{0.121569,0.466667,0.705882}%
\pgfsetstrokecolor{currentstroke}%
\pgfsetdash{}{0pt}%
\pgfpathmoveto{\pgfqpoint{0.949186in}{3.134537in}}%
\pgfpathcurveto{\pgfqpoint{0.960236in}{3.134537in}}{\pgfqpoint{0.970835in}{3.138928in}}{\pgfqpoint{0.978649in}{3.146741in}}%
\pgfpathcurveto{\pgfqpoint{0.986463in}{3.154555in}}{\pgfqpoint{0.990853in}{3.165154in}}{\pgfqpoint{0.990853in}{3.176204in}}%
\pgfpathcurveto{\pgfqpoint{0.990853in}{3.187254in}}{\pgfqpoint{0.986463in}{3.197853in}}{\pgfqpoint{0.978649in}{3.205667in}}%
\pgfpathcurveto{\pgfqpoint{0.970835in}{3.213480in}}{\pgfqpoint{0.960236in}{3.217871in}}{\pgfqpoint{0.949186in}{3.217871in}}%
\pgfpathcurveto{\pgfqpoint{0.938136in}{3.217871in}}{\pgfqpoint{0.927537in}{3.213480in}}{\pgfqpoint{0.919723in}{3.205667in}}%
\pgfpathcurveto{\pgfqpoint{0.911910in}{3.197853in}}{\pgfqpoint{0.907520in}{3.187254in}}{\pgfqpoint{0.907520in}{3.176204in}}%
\pgfpathcurveto{\pgfqpoint{0.907520in}{3.165154in}}{\pgfqpoint{0.911910in}{3.154555in}}{\pgfqpoint{0.919723in}{3.146741in}}%
\pgfpathcurveto{\pgfqpoint{0.927537in}{3.138928in}}{\pgfqpoint{0.938136in}{3.134537in}}{\pgfqpoint{0.949186in}{3.134537in}}%
\pgfpathclose%
\pgfusepath{stroke,fill}%
\end{pgfscope}%
\begin{pgfscope}%
\pgfpathrectangle{\pgfqpoint{0.648703in}{0.548769in}}{\pgfqpoint{5.112893in}{3.102590in}}%
\pgfusepath{clip}%
\pgfsetbuttcap%
\pgfsetroundjoin%
\definecolor{currentfill}{rgb}{0.121569,0.466667,0.705882}%
\pgfsetfillcolor{currentfill}%
\pgfsetlinewidth{1.003750pt}%
\definecolor{currentstroke}{rgb}{0.121569,0.466667,0.705882}%
\pgfsetstrokecolor{currentstroke}%
\pgfsetdash{}{0pt}%
\pgfpathmoveto{\pgfqpoint{1.778121in}{3.134537in}}%
\pgfpathcurveto{\pgfqpoint{1.789171in}{3.134537in}}{\pgfqpoint{1.799771in}{3.138928in}}{\pgfqpoint{1.807584in}{3.146741in}}%
\pgfpathcurveto{\pgfqpoint{1.815398in}{3.154555in}}{\pgfqpoint{1.819788in}{3.165154in}}{\pgfqpoint{1.819788in}{3.176204in}}%
\pgfpathcurveto{\pgfqpoint{1.819788in}{3.187254in}}{\pgfqpoint{1.815398in}{3.197853in}}{\pgfqpoint{1.807584in}{3.205667in}}%
\pgfpathcurveto{\pgfqpoint{1.799771in}{3.213480in}}{\pgfqpoint{1.789171in}{3.217871in}}{\pgfqpoint{1.778121in}{3.217871in}}%
\pgfpathcurveto{\pgfqpoint{1.767071in}{3.217871in}}{\pgfqpoint{1.756472in}{3.213480in}}{\pgfqpoint{1.748659in}{3.205667in}}%
\pgfpathcurveto{\pgfqpoint{1.740845in}{3.197853in}}{\pgfqpoint{1.736455in}{3.187254in}}{\pgfqpoint{1.736455in}{3.176204in}}%
\pgfpathcurveto{\pgfqpoint{1.736455in}{3.165154in}}{\pgfqpoint{1.740845in}{3.154555in}}{\pgfqpoint{1.748659in}{3.146741in}}%
\pgfpathcurveto{\pgfqpoint{1.756472in}{3.138928in}}{\pgfqpoint{1.767071in}{3.134537in}}{\pgfqpoint{1.778121in}{3.134537in}}%
\pgfpathclose%
\pgfusepath{stroke,fill}%
\end{pgfscope}%
\begin{pgfscope}%
\pgfpathrectangle{\pgfqpoint{0.648703in}{0.548769in}}{\pgfqpoint{5.112893in}{3.102590in}}%
\pgfusepath{clip}%
\pgfsetbuttcap%
\pgfsetroundjoin%
\definecolor{currentfill}{rgb}{1.000000,0.498039,0.054902}%
\pgfsetfillcolor{currentfill}%
\pgfsetlinewidth{1.003750pt}%
\definecolor{currentstroke}{rgb}{1.000000,0.498039,0.054902}%
\pgfsetstrokecolor{currentstroke}%
\pgfsetdash{}{0pt}%
\pgfpathmoveto{\pgfqpoint{1.453450in}{3.138662in}}%
\pgfpathcurveto{\pgfqpoint{1.464500in}{3.138662in}}{\pgfqpoint{1.475099in}{3.143053in}}{\pgfqpoint{1.482912in}{3.150866in}}%
\pgfpathcurveto{\pgfqpoint{1.490726in}{3.158680in}}{\pgfqpoint{1.495116in}{3.169279in}}{\pgfqpoint{1.495116in}{3.180329in}}%
\pgfpathcurveto{\pgfqpoint{1.495116in}{3.191379in}}{\pgfqpoint{1.490726in}{3.201978in}}{\pgfqpoint{1.482912in}{3.209792in}}%
\pgfpathcurveto{\pgfqpoint{1.475099in}{3.217605in}}{\pgfqpoint{1.464500in}{3.221996in}}{\pgfqpoint{1.453450in}{3.221996in}}%
\pgfpathcurveto{\pgfqpoint{1.442399in}{3.221996in}}{\pgfqpoint{1.431800in}{3.217605in}}{\pgfqpoint{1.423987in}{3.209792in}}%
\pgfpathcurveto{\pgfqpoint{1.416173in}{3.201978in}}{\pgfqpoint{1.411783in}{3.191379in}}{\pgfqpoint{1.411783in}{3.180329in}}%
\pgfpathcurveto{\pgfqpoint{1.411783in}{3.169279in}}{\pgfqpoint{1.416173in}{3.158680in}}{\pgfqpoint{1.423987in}{3.150866in}}%
\pgfpathcurveto{\pgfqpoint{1.431800in}{3.143053in}}{\pgfqpoint{1.442399in}{3.138662in}}{\pgfqpoint{1.453450in}{3.138662in}}%
\pgfpathclose%
\pgfusepath{stroke,fill}%
\end{pgfscope}%
\begin{pgfscope}%
\pgfpathrectangle{\pgfqpoint{0.648703in}{0.548769in}}{\pgfqpoint{5.112893in}{3.102590in}}%
\pgfusepath{clip}%
\pgfsetbuttcap%
\pgfsetroundjoin%
\definecolor{currentfill}{rgb}{0.839216,0.152941,0.156863}%
\pgfsetfillcolor{currentfill}%
\pgfsetlinewidth{1.003750pt}%
\definecolor{currentstroke}{rgb}{0.839216,0.152941,0.156863}%
\pgfsetstrokecolor{currentstroke}%
\pgfsetdash{}{0pt}%
\pgfpathmoveto{\pgfqpoint{1.338626in}{3.122162in}}%
\pgfpathcurveto{\pgfqpoint{1.349677in}{3.122162in}}{\pgfqpoint{1.360276in}{3.126553in}}{\pgfqpoint{1.368089in}{3.134366in}}%
\pgfpathcurveto{\pgfqpoint{1.375903in}{3.142180in}}{\pgfqpoint{1.380293in}{3.152779in}}{\pgfqpoint{1.380293in}{3.163829in}}%
\pgfpathcurveto{\pgfqpoint{1.380293in}{3.174879in}}{\pgfqpoint{1.375903in}{3.185478in}}{\pgfqpoint{1.368089in}{3.193292in}}%
\pgfpathcurveto{\pgfqpoint{1.360276in}{3.201105in}}{\pgfqpoint{1.349677in}{3.205496in}}{\pgfqpoint{1.338626in}{3.205496in}}%
\pgfpathcurveto{\pgfqpoint{1.327576in}{3.205496in}}{\pgfqpoint{1.316977in}{3.201105in}}{\pgfqpoint{1.309164in}{3.193292in}}%
\pgfpathcurveto{\pgfqpoint{1.301350in}{3.185478in}}{\pgfqpoint{1.296960in}{3.174879in}}{\pgfqpoint{1.296960in}{3.163829in}}%
\pgfpathcurveto{\pgfqpoint{1.296960in}{3.152779in}}{\pgfqpoint{1.301350in}{3.142180in}}{\pgfqpoint{1.309164in}{3.134366in}}%
\pgfpathcurveto{\pgfqpoint{1.316977in}{3.126553in}}{\pgfqpoint{1.327576in}{3.122162in}}{\pgfqpoint{1.338626in}{3.122162in}}%
\pgfpathclose%
\pgfusepath{stroke,fill}%
\end{pgfscope}%
\begin{pgfscope}%
\pgfpathrectangle{\pgfqpoint{0.648703in}{0.548769in}}{\pgfqpoint{5.112893in}{3.102590in}}%
\pgfusepath{clip}%
\pgfsetbuttcap%
\pgfsetroundjoin%
\definecolor{currentfill}{rgb}{1.000000,0.498039,0.054902}%
\pgfsetfillcolor{currentfill}%
\pgfsetlinewidth{1.003750pt}%
\definecolor{currentstroke}{rgb}{1.000000,0.498039,0.054902}%
\pgfsetstrokecolor{currentstroke}%
\pgfsetdash{}{0pt}%
\pgfpathmoveto{\pgfqpoint{1.961613in}{3.245913in}}%
\pgfpathcurveto{\pgfqpoint{1.972663in}{3.245913in}}{\pgfqpoint{1.983262in}{3.250304in}}{\pgfqpoint{1.991076in}{3.258117in}}%
\pgfpathcurveto{\pgfqpoint{1.998890in}{3.265931in}}{\pgfqpoint{2.003280in}{3.276530in}}{\pgfqpoint{2.003280in}{3.287580in}}%
\pgfpathcurveto{\pgfqpoint{2.003280in}{3.298630in}}{\pgfqpoint{1.998890in}{3.309229in}}{\pgfqpoint{1.991076in}{3.317043in}}%
\pgfpathcurveto{\pgfqpoint{1.983262in}{3.324856in}}{\pgfqpoint{1.972663in}{3.329247in}}{\pgfqpoint{1.961613in}{3.329247in}}%
\pgfpathcurveto{\pgfqpoint{1.950563in}{3.329247in}}{\pgfqpoint{1.939964in}{3.324856in}}{\pgfqpoint{1.932150in}{3.317043in}}%
\pgfpathcurveto{\pgfqpoint{1.924337in}{3.309229in}}{\pgfqpoint{1.919947in}{3.298630in}}{\pgfqpoint{1.919947in}{3.287580in}}%
\pgfpathcurveto{\pgfqpoint{1.919947in}{3.276530in}}{\pgfqpoint{1.924337in}{3.265931in}}{\pgfqpoint{1.932150in}{3.258117in}}%
\pgfpathcurveto{\pgfqpoint{1.939964in}{3.250304in}}{\pgfqpoint{1.950563in}{3.245913in}}{\pgfqpoint{1.961613in}{3.245913in}}%
\pgfpathclose%
\pgfusepath{stroke,fill}%
\end{pgfscope}%
\begin{pgfscope}%
\pgfpathrectangle{\pgfqpoint{0.648703in}{0.548769in}}{\pgfqpoint{5.112893in}{3.102590in}}%
\pgfusepath{clip}%
\pgfsetbuttcap%
\pgfsetroundjoin%
\definecolor{currentfill}{rgb}{0.121569,0.466667,0.705882}%
\pgfsetfillcolor{currentfill}%
\pgfsetlinewidth{1.003750pt}%
\definecolor{currentstroke}{rgb}{0.121569,0.466667,0.705882}%
\pgfsetstrokecolor{currentstroke}%
\pgfsetdash{}{0pt}%
\pgfpathmoveto{\pgfqpoint{1.437292in}{3.130412in}}%
\pgfpathcurveto{\pgfqpoint{1.448343in}{3.130412in}}{\pgfqpoint{1.458942in}{3.134803in}}{\pgfqpoint{1.466755in}{3.142616in}}%
\pgfpathcurveto{\pgfqpoint{1.474569in}{3.150430in}}{\pgfqpoint{1.478959in}{3.161029in}}{\pgfqpoint{1.478959in}{3.172079in}}%
\pgfpathcurveto{\pgfqpoint{1.478959in}{3.183129in}}{\pgfqpoint{1.474569in}{3.193728in}}{\pgfqpoint{1.466755in}{3.201542in}}%
\pgfpathcurveto{\pgfqpoint{1.458942in}{3.209355in}}{\pgfqpoint{1.448343in}{3.213746in}}{\pgfqpoint{1.437292in}{3.213746in}}%
\pgfpathcurveto{\pgfqpoint{1.426242in}{3.213746in}}{\pgfqpoint{1.415643in}{3.209355in}}{\pgfqpoint{1.407830in}{3.201542in}}%
\pgfpathcurveto{\pgfqpoint{1.400016in}{3.193728in}}{\pgfqpoint{1.395626in}{3.183129in}}{\pgfqpoint{1.395626in}{3.172079in}}%
\pgfpathcurveto{\pgfqpoint{1.395626in}{3.161029in}}{\pgfqpoint{1.400016in}{3.150430in}}{\pgfqpoint{1.407830in}{3.142616in}}%
\pgfpathcurveto{\pgfqpoint{1.415643in}{3.134803in}}{\pgfqpoint{1.426242in}{3.130412in}}{\pgfqpoint{1.437292in}{3.130412in}}%
\pgfpathclose%
\pgfusepath{stroke,fill}%
\end{pgfscope}%
\begin{pgfscope}%
\pgfpathrectangle{\pgfqpoint{0.648703in}{0.548769in}}{\pgfqpoint{5.112893in}{3.102590in}}%
\pgfusepath{clip}%
\pgfsetbuttcap%
\pgfsetroundjoin%
\definecolor{currentfill}{rgb}{1.000000,0.498039,0.054902}%
\pgfsetfillcolor{currentfill}%
\pgfsetlinewidth{1.003750pt}%
\definecolor{currentstroke}{rgb}{1.000000,0.498039,0.054902}%
\pgfsetstrokecolor{currentstroke}%
\pgfsetdash{}{0pt}%
\pgfpathmoveto{\pgfqpoint{1.229063in}{3.138662in}}%
\pgfpathcurveto{\pgfqpoint{1.240114in}{3.138662in}}{\pgfqpoint{1.250713in}{3.143053in}}{\pgfqpoint{1.258526in}{3.150866in}}%
\pgfpathcurveto{\pgfqpoint{1.266340in}{3.158680in}}{\pgfqpoint{1.270730in}{3.169279in}}{\pgfqpoint{1.270730in}{3.180329in}}%
\pgfpathcurveto{\pgfqpoint{1.270730in}{3.191379in}}{\pgfqpoint{1.266340in}{3.201978in}}{\pgfqpoint{1.258526in}{3.209792in}}%
\pgfpathcurveto{\pgfqpoint{1.250713in}{3.217605in}}{\pgfqpoint{1.240114in}{3.221996in}}{\pgfqpoint{1.229063in}{3.221996in}}%
\pgfpathcurveto{\pgfqpoint{1.218013in}{3.221996in}}{\pgfqpoint{1.207414in}{3.217605in}}{\pgfqpoint{1.199601in}{3.209792in}}%
\pgfpathcurveto{\pgfqpoint{1.191787in}{3.201978in}}{\pgfqpoint{1.187397in}{3.191379in}}{\pgfqpoint{1.187397in}{3.180329in}}%
\pgfpathcurveto{\pgfqpoint{1.187397in}{3.169279in}}{\pgfqpoint{1.191787in}{3.158680in}}{\pgfqpoint{1.199601in}{3.150866in}}%
\pgfpathcurveto{\pgfqpoint{1.207414in}{3.143053in}}{\pgfqpoint{1.218013in}{3.138662in}}{\pgfqpoint{1.229063in}{3.138662in}}%
\pgfpathclose%
\pgfusepath{stroke,fill}%
\end{pgfscope}%
\begin{pgfscope}%
\pgfpathrectangle{\pgfqpoint{0.648703in}{0.548769in}}{\pgfqpoint{5.112893in}{3.102590in}}%
\pgfusepath{clip}%
\pgfsetbuttcap%
\pgfsetroundjoin%
\definecolor{currentfill}{rgb}{1.000000,0.498039,0.054902}%
\pgfsetfillcolor{currentfill}%
\pgfsetlinewidth{1.003750pt}%
\definecolor{currentstroke}{rgb}{1.000000,0.498039,0.054902}%
\pgfsetstrokecolor{currentstroke}%
\pgfsetdash{}{0pt}%
\pgfpathmoveto{\pgfqpoint{1.398884in}{3.138662in}}%
\pgfpathcurveto{\pgfqpoint{1.409935in}{3.138662in}}{\pgfqpoint{1.420534in}{3.143053in}}{\pgfqpoint{1.428347in}{3.150866in}}%
\pgfpathcurveto{\pgfqpoint{1.436161in}{3.158680in}}{\pgfqpoint{1.440551in}{3.169279in}}{\pgfqpoint{1.440551in}{3.180329in}}%
\pgfpathcurveto{\pgfqpoint{1.440551in}{3.191379in}}{\pgfqpoint{1.436161in}{3.201978in}}{\pgfqpoint{1.428347in}{3.209792in}}%
\pgfpathcurveto{\pgfqpoint{1.420534in}{3.217605in}}{\pgfqpoint{1.409935in}{3.221996in}}{\pgfqpoint{1.398884in}{3.221996in}}%
\pgfpathcurveto{\pgfqpoint{1.387834in}{3.221996in}}{\pgfqpoint{1.377235in}{3.217605in}}{\pgfqpoint{1.369422in}{3.209792in}}%
\pgfpathcurveto{\pgfqpoint{1.361608in}{3.201978in}}{\pgfqpoint{1.357218in}{3.191379in}}{\pgfqpoint{1.357218in}{3.180329in}}%
\pgfpathcurveto{\pgfqpoint{1.357218in}{3.169279in}}{\pgfqpoint{1.361608in}{3.158680in}}{\pgfqpoint{1.369422in}{3.150866in}}%
\pgfpathcurveto{\pgfqpoint{1.377235in}{3.143053in}}{\pgfqpoint{1.387834in}{3.138662in}}{\pgfqpoint{1.398884in}{3.138662in}}%
\pgfpathclose%
\pgfusepath{stroke,fill}%
\end{pgfscope}%
\begin{pgfscope}%
\pgfpathrectangle{\pgfqpoint{0.648703in}{0.548769in}}{\pgfqpoint{5.112893in}{3.102590in}}%
\pgfusepath{clip}%
\pgfsetbuttcap%
\pgfsetroundjoin%
\definecolor{currentfill}{rgb}{1.000000,0.498039,0.054902}%
\pgfsetfillcolor{currentfill}%
\pgfsetlinewidth{1.003750pt}%
\definecolor{currentstroke}{rgb}{1.000000,0.498039,0.054902}%
\pgfsetstrokecolor{currentstroke}%
\pgfsetdash{}{0pt}%
\pgfpathmoveto{\pgfqpoint{1.126269in}{3.138662in}}%
\pgfpathcurveto{\pgfqpoint{1.137319in}{3.138662in}}{\pgfqpoint{1.147918in}{3.143053in}}{\pgfqpoint{1.155732in}{3.150866in}}%
\pgfpathcurveto{\pgfqpoint{1.163546in}{3.158680in}}{\pgfqpoint{1.167936in}{3.169279in}}{\pgfqpoint{1.167936in}{3.180329in}}%
\pgfpathcurveto{\pgfqpoint{1.167936in}{3.191379in}}{\pgfqpoint{1.163546in}{3.201978in}}{\pgfqpoint{1.155732in}{3.209792in}}%
\pgfpathcurveto{\pgfqpoint{1.147918in}{3.217605in}}{\pgfqpoint{1.137319in}{3.221996in}}{\pgfqpoint{1.126269in}{3.221996in}}%
\pgfpathcurveto{\pgfqpoint{1.115219in}{3.221996in}}{\pgfqpoint{1.104620in}{3.217605in}}{\pgfqpoint{1.096806in}{3.209792in}}%
\pgfpathcurveto{\pgfqpoint{1.088993in}{3.201978in}}{\pgfqpoint{1.084603in}{3.191379in}}{\pgfqpoint{1.084603in}{3.180329in}}%
\pgfpathcurveto{\pgfqpoint{1.084603in}{3.169279in}}{\pgfqpoint{1.088993in}{3.158680in}}{\pgfqpoint{1.096806in}{3.150866in}}%
\pgfpathcurveto{\pgfqpoint{1.104620in}{3.143053in}}{\pgfqpoint{1.115219in}{3.138662in}}{\pgfqpoint{1.126269in}{3.138662in}}%
\pgfpathclose%
\pgfusepath{stroke,fill}%
\end{pgfscope}%
\begin{pgfscope}%
\pgfpathrectangle{\pgfqpoint{0.648703in}{0.548769in}}{\pgfqpoint{5.112893in}{3.102590in}}%
\pgfusepath{clip}%
\pgfsetbuttcap%
\pgfsetroundjoin%
\definecolor{currentfill}{rgb}{0.121569,0.466667,0.705882}%
\pgfsetfillcolor{currentfill}%
\pgfsetlinewidth{1.003750pt}%
\definecolor{currentstroke}{rgb}{0.121569,0.466667,0.705882}%
\pgfsetstrokecolor{currentstroke}%
\pgfsetdash{}{0pt}%
\pgfpathmoveto{\pgfqpoint{1.477687in}{3.134537in}}%
\pgfpathcurveto{\pgfqpoint{1.488737in}{3.134537in}}{\pgfqpoint{1.499336in}{3.138928in}}{\pgfqpoint{1.507150in}{3.146741in}}%
\pgfpathcurveto{\pgfqpoint{1.514964in}{3.154555in}}{\pgfqpoint{1.519354in}{3.165154in}}{\pgfqpoint{1.519354in}{3.176204in}}%
\pgfpathcurveto{\pgfqpoint{1.519354in}{3.187254in}}{\pgfqpoint{1.514964in}{3.197853in}}{\pgfqpoint{1.507150in}{3.205667in}}%
\pgfpathcurveto{\pgfqpoint{1.499336in}{3.213480in}}{\pgfqpoint{1.488737in}{3.217871in}}{\pgfqpoint{1.477687in}{3.217871in}}%
\pgfpathcurveto{\pgfqpoint{1.466637in}{3.217871in}}{\pgfqpoint{1.456038in}{3.213480in}}{\pgfqpoint{1.448224in}{3.205667in}}%
\pgfpathcurveto{\pgfqpoint{1.440411in}{3.197853in}}{\pgfqpoint{1.436020in}{3.187254in}}{\pgfqpoint{1.436020in}{3.176204in}}%
\pgfpathcurveto{\pgfqpoint{1.436020in}{3.165154in}}{\pgfqpoint{1.440411in}{3.154555in}}{\pgfqpoint{1.448224in}{3.146741in}}%
\pgfpathcurveto{\pgfqpoint{1.456038in}{3.138928in}}{\pgfqpoint{1.466637in}{3.134537in}}{\pgfqpoint{1.477687in}{3.134537in}}%
\pgfpathclose%
\pgfusepath{stroke,fill}%
\end{pgfscope}%
\begin{pgfscope}%
\pgfpathrectangle{\pgfqpoint{0.648703in}{0.548769in}}{\pgfqpoint{5.112893in}{3.102590in}}%
\pgfusepath{clip}%
\pgfsetbuttcap%
\pgfsetroundjoin%
\definecolor{currentfill}{rgb}{0.121569,0.466667,0.705882}%
\pgfsetfillcolor{currentfill}%
\pgfsetlinewidth{1.003750pt}%
\definecolor{currentstroke}{rgb}{0.121569,0.466667,0.705882}%
\pgfsetstrokecolor{currentstroke}%
\pgfsetdash{}{0pt}%
\pgfpathmoveto{\pgfqpoint{1.503593in}{3.109787in}}%
\pgfpathcurveto{\pgfqpoint{1.514643in}{3.109787in}}{\pgfqpoint{1.525242in}{3.114177in}}{\pgfqpoint{1.533056in}{3.121991in}}%
\pgfpathcurveto{\pgfqpoint{1.540869in}{3.129805in}}{\pgfqpoint{1.545259in}{3.140404in}}{\pgfqpoint{1.545259in}{3.151454in}}%
\pgfpathcurveto{\pgfqpoint{1.545259in}{3.162504in}}{\pgfqpoint{1.540869in}{3.173103in}}{\pgfqpoint{1.533056in}{3.180917in}}%
\pgfpathcurveto{\pgfqpoint{1.525242in}{3.188730in}}{\pgfqpoint{1.514643in}{3.193121in}}{\pgfqpoint{1.503593in}{3.193121in}}%
\pgfpathcurveto{\pgfqpoint{1.492543in}{3.193121in}}{\pgfqpoint{1.481944in}{3.188730in}}{\pgfqpoint{1.474130in}{3.180917in}}%
\pgfpathcurveto{\pgfqpoint{1.466316in}{3.173103in}}{\pgfqpoint{1.461926in}{3.162504in}}{\pgfqpoint{1.461926in}{3.151454in}}%
\pgfpathcurveto{\pgfqpoint{1.461926in}{3.140404in}}{\pgfqpoint{1.466316in}{3.129805in}}{\pgfqpoint{1.474130in}{3.121991in}}%
\pgfpathcurveto{\pgfqpoint{1.481944in}{3.114177in}}{\pgfqpoint{1.492543in}{3.109787in}}{\pgfqpoint{1.503593in}{3.109787in}}%
\pgfpathclose%
\pgfusepath{stroke,fill}%
\end{pgfscope}%
\begin{pgfscope}%
\pgfpathrectangle{\pgfqpoint{0.648703in}{0.548769in}}{\pgfqpoint{5.112893in}{3.102590in}}%
\pgfusepath{clip}%
\pgfsetbuttcap%
\pgfsetroundjoin%
\definecolor{currentfill}{rgb}{0.121569,0.466667,0.705882}%
\pgfsetfillcolor{currentfill}%
\pgfsetlinewidth{1.003750pt}%
\definecolor{currentstroke}{rgb}{0.121569,0.466667,0.705882}%
\pgfsetstrokecolor{currentstroke}%
\pgfsetdash{}{0pt}%
\pgfpathmoveto{\pgfqpoint{1.589307in}{3.134537in}}%
\pgfpathcurveto{\pgfqpoint{1.600357in}{3.134537in}}{\pgfqpoint{1.610956in}{3.138928in}}{\pgfqpoint{1.618770in}{3.146741in}}%
\pgfpathcurveto{\pgfqpoint{1.626584in}{3.154555in}}{\pgfqpoint{1.630974in}{3.165154in}}{\pgfqpoint{1.630974in}{3.176204in}}%
\pgfpathcurveto{\pgfqpoint{1.630974in}{3.187254in}}{\pgfqpoint{1.626584in}{3.197853in}}{\pgfqpoint{1.618770in}{3.205667in}}%
\pgfpathcurveto{\pgfqpoint{1.610956in}{3.213480in}}{\pgfqpoint{1.600357in}{3.217871in}}{\pgfqpoint{1.589307in}{3.217871in}}%
\pgfpathcurveto{\pgfqpoint{1.578257in}{3.217871in}}{\pgfqpoint{1.567658in}{3.213480in}}{\pgfqpoint{1.559844in}{3.205667in}}%
\pgfpathcurveto{\pgfqpoint{1.552031in}{3.197853in}}{\pgfqpoint{1.547640in}{3.187254in}}{\pgfqpoint{1.547640in}{3.176204in}}%
\pgfpathcurveto{\pgfqpoint{1.547640in}{3.165154in}}{\pgfqpoint{1.552031in}{3.154555in}}{\pgfqpoint{1.559844in}{3.146741in}}%
\pgfpathcurveto{\pgfqpoint{1.567658in}{3.138928in}}{\pgfqpoint{1.578257in}{3.134537in}}{\pgfqpoint{1.589307in}{3.134537in}}%
\pgfpathclose%
\pgfusepath{stroke,fill}%
\end{pgfscope}%
\begin{pgfscope}%
\pgfpathrectangle{\pgfqpoint{0.648703in}{0.548769in}}{\pgfqpoint{5.112893in}{3.102590in}}%
\pgfusepath{clip}%
\pgfsetbuttcap%
\pgfsetroundjoin%
\definecolor{currentfill}{rgb}{0.121569,0.466667,0.705882}%
\pgfsetfillcolor{currentfill}%
\pgfsetlinewidth{1.003750pt}%
\definecolor{currentstroke}{rgb}{0.121569,0.466667,0.705882}%
\pgfsetstrokecolor{currentstroke}%
\pgfsetdash{}{0pt}%
\pgfpathmoveto{\pgfqpoint{0.719376in}{0.663642in}}%
\pgfpathcurveto{\pgfqpoint{0.730426in}{0.663642in}}{\pgfqpoint{0.741025in}{0.668032in}}{\pgfqpoint{0.748838in}{0.675846in}}%
\pgfpathcurveto{\pgfqpoint{0.756652in}{0.683659in}}{\pgfqpoint{0.761042in}{0.694258in}}{\pgfqpoint{0.761042in}{0.705309in}}%
\pgfpathcurveto{\pgfqpoint{0.761042in}{0.716359in}}{\pgfqpoint{0.756652in}{0.726958in}}{\pgfqpoint{0.748838in}{0.734771in}}%
\pgfpathcurveto{\pgfqpoint{0.741025in}{0.742585in}}{\pgfqpoint{0.730426in}{0.746975in}}{\pgfqpoint{0.719376in}{0.746975in}}%
\pgfpathcurveto{\pgfqpoint{0.708326in}{0.746975in}}{\pgfqpoint{0.697727in}{0.742585in}}{\pgfqpoint{0.689913in}{0.734771in}}%
\pgfpathcurveto{\pgfqpoint{0.682099in}{0.726958in}}{\pgfqpoint{0.677709in}{0.716359in}}{\pgfqpoint{0.677709in}{0.705309in}}%
\pgfpathcurveto{\pgfqpoint{0.677709in}{0.694258in}}{\pgfqpoint{0.682099in}{0.683659in}}{\pgfqpoint{0.689913in}{0.675846in}}%
\pgfpathcurveto{\pgfqpoint{0.697727in}{0.668032in}}{\pgfqpoint{0.708326in}{0.663642in}}{\pgfqpoint{0.719376in}{0.663642in}}%
\pgfpathclose%
\pgfusepath{stroke,fill}%
\end{pgfscope}%
\begin{pgfscope}%
\pgfpathrectangle{\pgfqpoint{0.648703in}{0.548769in}}{\pgfqpoint{5.112893in}{3.102590in}}%
\pgfusepath{clip}%
\pgfsetbuttcap%
\pgfsetroundjoin%
\definecolor{currentfill}{rgb}{1.000000,0.498039,0.054902}%
\pgfsetfillcolor{currentfill}%
\pgfsetlinewidth{1.003750pt}%
\definecolor{currentstroke}{rgb}{1.000000,0.498039,0.054902}%
\pgfsetstrokecolor{currentstroke}%
\pgfsetdash{}{0pt}%
\pgfpathmoveto{\pgfqpoint{0.977891in}{3.142787in}}%
\pgfpathcurveto{\pgfqpoint{0.988941in}{3.142787in}}{\pgfqpoint{0.999540in}{3.147178in}}{\pgfqpoint{1.007354in}{3.154991in}}%
\pgfpathcurveto{\pgfqpoint{1.015167in}{3.162805in}}{\pgfqpoint{1.019558in}{3.173404in}}{\pgfqpoint{1.019558in}{3.184454in}}%
\pgfpathcurveto{\pgfqpoint{1.019558in}{3.195504in}}{\pgfqpoint{1.015167in}{3.206103in}}{\pgfqpoint{1.007354in}{3.213917in}}%
\pgfpathcurveto{\pgfqpoint{0.999540in}{3.221731in}}{\pgfqpoint{0.988941in}{3.226121in}}{\pgfqpoint{0.977891in}{3.226121in}}%
\pgfpathcurveto{\pgfqpoint{0.966841in}{3.226121in}}{\pgfqpoint{0.956242in}{3.221731in}}{\pgfqpoint{0.948428in}{3.213917in}}%
\pgfpathcurveto{\pgfqpoint{0.940614in}{3.206103in}}{\pgfqpoint{0.936224in}{3.195504in}}{\pgfqpoint{0.936224in}{3.184454in}}%
\pgfpathcurveto{\pgfqpoint{0.936224in}{3.173404in}}{\pgfqpoint{0.940614in}{3.162805in}}{\pgfqpoint{0.948428in}{3.154991in}}%
\pgfpathcurveto{\pgfqpoint{0.956242in}{3.147178in}}{\pgfqpoint{0.966841in}{3.142787in}}{\pgfqpoint{0.977891in}{3.142787in}}%
\pgfpathclose%
\pgfusepath{stroke,fill}%
\end{pgfscope}%
\begin{pgfscope}%
\pgfpathrectangle{\pgfqpoint{0.648703in}{0.548769in}}{\pgfqpoint{5.112893in}{3.102590in}}%
\pgfusepath{clip}%
\pgfsetbuttcap%
\pgfsetroundjoin%
\definecolor{currentfill}{rgb}{1.000000,0.498039,0.054902}%
\pgfsetfillcolor{currentfill}%
\pgfsetlinewidth{1.003750pt}%
\definecolor{currentstroke}{rgb}{1.000000,0.498039,0.054902}%
\pgfsetstrokecolor{currentstroke}%
\pgfsetdash{}{0pt}%
\pgfpathmoveto{\pgfqpoint{1.539087in}{3.146912in}}%
\pgfpathcurveto{\pgfqpoint{1.550137in}{3.146912in}}{\pgfqpoint{1.560736in}{3.151303in}}{\pgfqpoint{1.568549in}{3.159116in}}%
\pgfpathcurveto{\pgfqpoint{1.576363in}{3.166930in}}{\pgfqpoint{1.580753in}{3.177529in}}{\pgfqpoint{1.580753in}{3.188579in}}%
\pgfpathcurveto{\pgfqpoint{1.580753in}{3.199629in}}{\pgfqpoint{1.576363in}{3.210228in}}{\pgfqpoint{1.568549in}{3.218042in}}%
\pgfpathcurveto{\pgfqpoint{1.560736in}{3.225856in}}{\pgfqpoint{1.550137in}{3.230246in}}{\pgfqpoint{1.539087in}{3.230246in}}%
\pgfpathcurveto{\pgfqpoint{1.528036in}{3.230246in}}{\pgfqpoint{1.517437in}{3.225856in}}{\pgfqpoint{1.509624in}{3.218042in}}%
\pgfpathcurveto{\pgfqpoint{1.501810in}{3.210228in}}{\pgfqpoint{1.497420in}{3.199629in}}{\pgfqpoint{1.497420in}{3.188579in}}%
\pgfpathcurveto{\pgfqpoint{1.497420in}{3.177529in}}{\pgfqpoint{1.501810in}{3.166930in}}{\pgfqpoint{1.509624in}{3.159116in}}%
\pgfpathcurveto{\pgfqpoint{1.517437in}{3.151303in}}{\pgfqpoint{1.528036in}{3.146912in}}{\pgfqpoint{1.539087in}{3.146912in}}%
\pgfpathclose%
\pgfusepath{stroke,fill}%
\end{pgfscope}%
\begin{pgfscope}%
\pgfpathrectangle{\pgfqpoint{0.648703in}{0.548769in}}{\pgfqpoint{5.112893in}{3.102590in}}%
\pgfusepath{clip}%
\pgfsetbuttcap%
\pgfsetroundjoin%
\definecolor{currentfill}{rgb}{1.000000,0.498039,0.054902}%
\pgfsetfillcolor{currentfill}%
\pgfsetlinewidth{1.003750pt}%
\definecolor{currentstroke}{rgb}{1.000000,0.498039,0.054902}%
\pgfsetstrokecolor{currentstroke}%
\pgfsetdash{}{0pt}%
\pgfpathmoveto{\pgfqpoint{1.731351in}{3.142787in}}%
\pgfpathcurveto{\pgfqpoint{1.742402in}{3.142787in}}{\pgfqpoint{1.753001in}{3.147178in}}{\pgfqpoint{1.760814in}{3.154991in}}%
\pgfpathcurveto{\pgfqpoint{1.768628in}{3.162805in}}{\pgfqpoint{1.773018in}{3.173404in}}{\pgfqpoint{1.773018in}{3.184454in}}%
\pgfpathcurveto{\pgfqpoint{1.773018in}{3.195504in}}{\pgfqpoint{1.768628in}{3.206103in}}{\pgfqpoint{1.760814in}{3.213917in}}%
\pgfpathcurveto{\pgfqpoint{1.753001in}{3.221731in}}{\pgfqpoint{1.742402in}{3.226121in}}{\pgfqpoint{1.731351in}{3.226121in}}%
\pgfpathcurveto{\pgfqpoint{1.720301in}{3.226121in}}{\pgfqpoint{1.709702in}{3.221731in}}{\pgfqpoint{1.701889in}{3.213917in}}%
\pgfpathcurveto{\pgfqpoint{1.694075in}{3.206103in}}{\pgfqpoint{1.689685in}{3.195504in}}{\pgfqpoint{1.689685in}{3.184454in}}%
\pgfpathcurveto{\pgfqpoint{1.689685in}{3.173404in}}{\pgfqpoint{1.694075in}{3.162805in}}{\pgfqpoint{1.701889in}{3.154991in}}%
\pgfpathcurveto{\pgfqpoint{1.709702in}{3.147178in}}{\pgfqpoint{1.720301in}{3.142787in}}{\pgfqpoint{1.731351in}{3.142787in}}%
\pgfpathclose%
\pgfusepath{stroke,fill}%
\end{pgfscope}%
\begin{pgfscope}%
\pgfpathrectangle{\pgfqpoint{0.648703in}{0.548769in}}{\pgfqpoint{5.112893in}{3.102590in}}%
\pgfusepath{clip}%
\pgfsetbuttcap%
\pgfsetroundjoin%
\definecolor{currentfill}{rgb}{0.121569,0.466667,0.705882}%
\pgfsetfillcolor{currentfill}%
\pgfsetlinewidth{1.003750pt}%
\definecolor{currentstroke}{rgb}{0.121569,0.466667,0.705882}%
\pgfsetstrokecolor{currentstroke}%
\pgfsetdash{}{0pt}%
\pgfpathmoveto{\pgfqpoint{0.765377in}{0.713142in}}%
\pgfpathcurveto{\pgfqpoint{0.776427in}{0.713142in}}{\pgfqpoint{0.787026in}{0.717533in}}{\pgfqpoint{0.794839in}{0.725346in}}%
\pgfpathcurveto{\pgfqpoint{0.802653in}{0.733160in}}{\pgfqpoint{0.807043in}{0.743759in}}{\pgfqpoint{0.807043in}{0.754809in}}%
\pgfpathcurveto{\pgfqpoint{0.807043in}{0.765859in}}{\pgfqpoint{0.802653in}{0.776458in}}{\pgfqpoint{0.794839in}{0.784272in}}%
\pgfpathcurveto{\pgfqpoint{0.787026in}{0.792085in}}{\pgfqpoint{0.776427in}{0.796476in}}{\pgfqpoint{0.765377in}{0.796476in}}%
\pgfpathcurveto{\pgfqpoint{0.754327in}{0.796476in}}{\pgfqpoint{0.743727in}{0.792085in}}{\pgfqpoint{0.735914in}{0.784272in}}%
\pgfpathcurveto{\pgfqpoint{0.728100in}{0.776458in}}{\pgfqpoint{0.723710in}{0.765859in}}{\pgfqpoint{0.723710in}{0.754809in}}%
\pgfpathcurveto{\pgfqpoint{0.723710in}{0.743759in}}{\pgfqpoint{0.728100in}{0.733160in}}{\pgfqpoint{0.735914in}{0.725346in}}%
\pgfpathcurveto{\pgfqpoint{0.743727in}{0.717533in}}{\pgfqpoint{0.754327in}{0.713142in}}{\pgfqpoint{0.765377in}{0.713142in}}%
\pgfpathclose%
\pgfusepath{stroke,fill}%
\end{pgfscope}%
\begin{pgfscope}%
\pgfpathrectangle{\pgfqpoint{0.648703in}{0.548769in}}{\pgfqpoint{5.112893in}{3.102590in}}%
\pgfusepath{clip}%
\pgfsetbuttcap%
\pgfsetroundjoin%
\definecolor{currentfill}{rgb}{1.000000,0.498039,0.054902}%
\pgfsetfillcolor{currentfill}%
\pgfsetlinewidth{1.003750pt}%
\definecolor{currentstroke}{rgb}{1.000000,0.498039,0.054902}%
\pgfsetstrokecolor{currentstroke}%
\pgfsetdash{}{0pt}%
\pgfpathmoveto{\pgfqpoint{1.042313in}{3.142787in}}%
\pgfpathcurveto{\pgfqpoint{1.053364in}{3.142787in}}{\pgfqpoint{1.063963in}{3.147178in}}{\pgfqpoint{1.071776in}{3.154991in}}%
\pgfpathcurveto{\pgfqpoint{1.079590in}{3.162805in}}{\pgfqpoint{1.083980in}{3.173404in}}{\pgfqpoint{1.083980in}{3.184454in}}%
\pgfpathcurveto{\pgfqpoint{1.083980in}{3.195504in}}{\pgfqpoint{1.079590in}{3.206103in}}{\pgfqpoint{1.071776in}{3.213917in}}%
\pgfpathcurveto{\pgfqpoint{1.063963in}{3.221731in}}{\pgfqpoint{1.053364in}{3.226121in}}{\pgfqpoint{1.042313in}{3.226121in}}%
\pgfpathcurveto{\pgfqpoint{1.031263in}{3.226121in}}{\pgfqpoint{1.020664in}{3.221731in}}{\pgfqpoint{1.012851in}{3.213917in}}%
\pgfpathcurveto{\pgfqpoint{1.005037in}{3.206103in}}{\pgfqpoint{1.000647in}{3.195504in}}{\pgfqpoint{1.000647in}{3.184454in}}%
\pgfpathcurveto{\pgfqpoint{1.000647in}{3.173404in}}{\pgfqpoint{1.005037in}{3.162805in}}{\pgfqpoint{1.012851in}{3.154991in}}%
\pgfpathcurveto{\pgfqpoint{1.020664in}{3.147178in}}{\pgfqpoint{1.031263in}{3.142787in}}{\pgfqpoint{1.042313in}{3.142787in}}%
\pgfpathclose%
\pgfusepath{stroke,fill}%
\end{pgfscope}%
\begin{pgfscope}%
\pgfpathrectangle{\pgfqpoint{0.648703in}{0.548769in}}{\pgfqpoint{5.112893in}{3.102590in}}%
\pgfusepath{clip}%
\pgfsetbuttcap%
\pgfsetroundjoin%
\definecolor{currentfill}{rgb}{0.121569,0.466667,0.705882}%
\pgfsetfillcolor{currentfill}%
\pgfsetlinewidth{1.003750pt}%
\definecolor{currentstroke}{rgb}{0.121569,0.466667,0.705882}%
\pgfsetstrokecolor{currentstroke}%
\pgfsetdash{}{0pt}%
\pgfpathmoveto{\pgfqpoint{1.496878in}{3.134537in}}%
\pgfpathcurveto{\pgfqpoint{1.507928in}{3.134537in}}{\pgfqpoint{1.518527in}{3.138928in}}{\pgfqpoint{1.526341in}{3.146741in}}%
\pgfpathcurveto{\pgfqpoint{1.534154in}{3.154555in}}{\pgfqpoint{1.538544in}{3.165154in}}{\pgfqpoint{1.538544in}{3.176204in}}%
\pgfpathcurveto{\pgfqpoint{1.538544in}{3.187254in}}{\pgfqpoint{1.534154in}{3.197853in}}{\pgfqpoint{1.526341in}{3.205667in}}%
\pgfpathcurveto{\pgfqpoint{1.518527in}{3.213480in}}{\pgfqpoint{1.507928in}{3.217871in}}{\pgfqpoint{1.496878in}{3.217871in}}%
\pgfpathcurveto{\pgfqpoint{1.485828in}{3.217871in}}{\pgfqpoint{1.475229in}{3.213480in}}{\pgfqpoint{1.467415in}{3.205667in}}%
\pgfpathcurveto{\pgfqpoint{1.459601in}{3.197853in}}{\pgfqpoint{1.455211in}{3.187254in}}{\pgfqpoint{1.455211in}{3.176204in}}%
\pgfpathcurveto{\pgfqpoint{1.455211in}{3.165154in}}{\pgfqpoint{1.459601in}{3.154555in}}{\pgfqpoint{1.467415in}{3.146741in}}%
\pgfpathcurveto{\pgfqpoint{1.475229in}{3.138928in}}{\pgfqpoint{1.485828in}{3.134537in}}{\pgfqpoint{1.496878in}{3.134537in}}%
\pgfpathclose%
\pgfusepath{stroke,fill}%
\end{pgfscope}%
\begin{pgfscope}%
\pgfpathrectangle{\pgfqpoint{0.648703in}{0.548769in}}{\pgfqpoint{5.112893in}{3.102590in}}%
\pgfusepath{clip}%
\pgfsetbuttcap%
\pgfsetroundjoin%
\definecolor{currentfill}{rgb}{1.000000,0.498039,0.054902}%
\pgfsetfillcolor{currentfill}%
\pgfsetlinewidth{1.003750pt}%
\definecolor{currentstroke}{rgb}{1.000000,0.498039,0.054902}%
\pgfsetstrokecolor{currentstroke}%
\pgfsetdash{}{0pt}%
\pgfpathmoveto{\pgfqpoint{1.467423in}{3.142787in}}%
\pgfpathcurveto{\pgfqpoint{1.478473in}{3.142787in}}{\pgfqpoint{1.489072in}{3.147178in}}{\pgfqpoint{1.496886in}{3.154991in}}%
\pgfpathcurveto{\pgfqpoint{1.504699in}{3.162805in}}{\pgfqpoint{1.509090in}{3.173404in}}{\pgfqpoint{1.509090in}{3.184454in}}%
\pgfpathcurveto{\pgfqpoint{1.509090in}{3.195504in}}{\pgfqpoint{1.504699in}{3.206103in}}{\pgfqpoint{1.496886in}{3.213917in}}%
\pgfpathcurveto{\pgfqpoint{1.489072in}{3.221731in}}{\pgfqpoint{1.478473in}{3.226121in}}{\pgfqpoint{1.467423in}{3.226121in}}%
\pgfpathcurveto{\pgfqpoint{1.456373in}{3.226121in}}{\pgfqpoint{1.445774in}{3.221731in}}{\pgfqpoint{1.437960in}{3.213917in}}%
\pgfpathcurveto{\pgfqpoint{1.430147in}{3.206103in}}{\pgfqpoint{1.425756in}{3.195504in}}{\pgfqpoint{1.425756in}{3.184454in}}%
\pgfpathcurveto{\pgfqpoint{1.425756in}{3.173404in}}{\pgfqpoint{1.430147in}{3.162805in}}{\pgfqpoint{1.437960in}{3.154991in}}%
\pgfpathcurveto{\pgfqpoint{1.445774in}{3.147178in}}{\pgfqpoint{1.456373in}{3.142787in}}{\pgfqpoint{1.467423in}{3.142787in}}%
\pgfpathclose%
\pgfusepath{stroke,fill}%
\end{pgfscope}%
\begin{pgfscope}%
\pgfpathrectangle{\pgfqpoint{0.648703in}{0.548769in}}{\pgfqpoint{5.112893in}{3.102590in}}%
\pgfusepath{clip}%
\pgfsetbuttcap%
\pgfsetroundjoin%
\definecolor{currentfill}{rgb}{0.121569,0.466667,0.705882}%
\pgfsetfillcolor{currentfill}%
\pgfsetlinewidth{1.003750pt}%
\definecolor{currentstroke}{rgb}{0.121569,0.466667,0.705882}%
\pgfsetstrokecolor{currentstroke}%
\pgfsetdash{}{0pt}%
\pgfpathmoveto{\pgfqpoint{0.922803in}{3.134537in}}%
\pgfpathcurveto{\pgfqpoint{0.933853in}{3.134537in}}{\pgfqpoint{0.944452in}{3.138928in}}{\pgfqpoint{0.952265in}{3.146741in}}%
\pgfpathcurveto{\pgfqpoint{0.960079in}{3.154555in}}{\pgfqpoint{0.964469in}{3.165154in}}{\pgfqpoint{0.964469in}{3.176204in}}%
\pgfpathcurveto{\pgfqpoint{0.964469in}{3.187254in}}{\pgfqpoint{0.960079in}{3.197853in}}{\pgfqpoint{0.952265in}{3.205667in}}%
\pgfpathcurveto{\pgfqpoint{0.944452in}{3.213480in}}{\pgfqpoint{0.933853in}{3.217871in}}{\pgfqpoint{0.922803in}{3.217871in}}%
\pgfpathcurveto{\pgfqpoint{0.911753in}{3.217871in}}{\pgfqpoint{0.901154in}{3.213480in}}{\pgfqpoint{0.893340in}{3.205667in}}%
\pgfpathcurveto{\pgfqpoint{0.885526in}{3.197853in}}{\pgfqpoint{0.881136in}{3.187254in}}{\pgfqpoint{0.881136in}{3.176204in}}%
\pgfpathcurveto{\pgfqpoint{0.881136in}{3.165154in}}{\pgfqpoint{0.885526in}{3.154555in}}{\pgfqpoint{0.893340in}{3.146741in}}%
\pgfpathcurveto{\pgfqpoint{0.901154in}{3.138928in}}{\pgfqpoint{0.911753in}{3.134537in}}{\pgfqpoint{0.922803in}{3.134537in}}%
\pgfpathclose%
\pgfusepath{stroke,fill}%
\end{pgfscope}%
\begin{pgfscope}%
\pgfpathrectangle{\pgfqpoint{0.648703in}{0.548769in}}{\pgfqpoint{5.112893in}{3.102590in}}%
\pgfusepath{clip}%
\pgfsetbuttcap%
\pgfsetroundjoin%
\definecolor{currentfill}{rgb}{1.000000,0.498039,0.054902}%
\pgfsetfillcolor{currentfill}%
\pgfsetlinewidth{1.003750pt}%
\definecolor{currentstroke}{rgb}{1.000000,0.498039,0.054902}%
\pgfsetstrokecolor{currentstroke}%
\pgfsetdash{}{0pt}%
\pgfpathmoveto{\pgfqpoint{1.635325in}{3.142787in}}%
\pgfpathcurveto{\pgfqpoint{1.646375in}{3.142787in}}{\pgfqpoint{1.656974in}{3.147178in}}{\pgfqpoint{1.664788in}{3.154991in}}%
\pgfpathcurveto{\pgfqpoint{1.672601in}{3.162805in}}{\pgfqpoint{1.676992in}{3.173404in}}{\pgfqpoint{1.676992in}{3.184454in}}%
\pgfpathcurveto{\pgfqpoint{1.676992in}{3.195504in}}{\pgfqpoint{1.672601in}{3.206103in}}{\pgfqpoint{1.664788in}{3.213917in}}%
\pgfpathcurveto{\pgfqpoint{1.656974in}{3.221731in}}{\pgfqpoint{1.646375in}{3.226121in}}{\pgfqpoint{1.635325in}{3.226121in}}%
\pgfpathcurveto{\pgfqpoint{1.624275in}{3.226121in}}{\pgfqpoint{1.613676in}{3.221731in}}{\pgfqpoint{1.605862in}{3.213917in}}%
\pgfpathcurveto{\pgfqpoint{1.598049in}{3.206103in}}{\pgfqpoint{1.593658in}{3.195504in}}{\pgfqpoint{1.593658in}{3.184454in}}%
\pgfpathcurveto{\pgfqpoint{1.593658in}{3.173404in}}{\pgfqpoint{1.598049in}{3.162805in}}{\pgfqpoint{1.605862in}{3.154991in}}%
\pgfpathcurveto{\pgfqpoint{1.613676in}{3.147178in}}{\pgfqpoint{1.624275in}{3.142787in}}{\pgfqpoint{1.635325in}{3.142787in}}%
\pgfpathclose%
\pgfusepath{stroke,fill}%
\end{pgfscope}%
\begin{pgfscope}%
\pgfpathrectangle{\pgfqpoint{0.648703in}{0.548769in}}{\pgfqpoint{5.112893in}{3.102590in}}%
\pgfusepath{clip}%
\pgfsetbuttcap%
\pgfsetroundjoin%
\definecolor{currentfill}{rgb}{1.000000,0.498039,0.054902}%
\pgfsetfillcolor{currentfill}%
\pgfsetlinewidth{1.003750pt}%
\definecolor{currentstroke}{rgb}{1.000000,0.498039,0.054902}%
\pgfsetstrokecolor{currentstroke}%
\pgfsetdash{}{0pt}%
\pgfpathmoveto{\pgfqpoint{1.584363in}{3.146912in}}%
\pgfpathcurveto{\pgfqpoint{1.595413in}{3.146912in}}{\pgfqpoint{1.606012in}{3.151303in}}{\pgfqpoint{1.613826in}{3.159116in}}%
\pgfpathcurveto{\pgfqpoint{1.621639in}{3.166930in}}{\pgfqpoint{1.626030in}{3.177529in}}{\pgfqpoint{1.626030in}{3.188579in}}%
\pgfpathcurveto{\pgfqpoint{1.626030in}{3.199629in}}{\pgfqpoint{1.621639in}{3.210228in}}{\pgfqpoint{1.613826in}{3.218042in}}%
\pgfpathcurveto{\pgfqpoint{1.606012in}{3.225856in}}{\pgfqpoint{1.595413in}{3.230246in}}{\pgfqpoint{1.584363in}{3.230246in}}%
\pgfpathcurveto{\pgfqpoint{1.573313in}{3.230246in}}{\pgfqpoint{1.562714in}{3.225856in}}{\pgfqpoint{1.554900in}{3.218042in}}%
\pgfpathcurveto{\pgfqpoint{1.547087in}{3.210228in}}{\pgfqpoint{1.542696in}{3.199629in}}{\pgfqpoint{1.542696in}{3.188579in}}%
\pgfpathcurveto{\pgfqpoint{1.542696in}{3.177529in}}{\pgfqpoint{1.547087in}{3.166930in}}{\pgfqpoint{1.554900in}{3.159116in}}%
\pgfpathcurveto{\pgfqpoint{1.562714in}{3.151303in}}{\pgfqpoint{1.573313in}{3.146912in}}{\pgfqpoint{1.584363in}{3.146912in}}%
\pgfpathclose%
\pgfusepath{stroke,fill}%
\end{pgfscope}%
\begin{pgfscope}%
\pgfpathrectangle{\pgfqpoint{0.648703in}{0.548769in}}{\pgfqpoint{5.112893in}{3.102590in}}%
\pgfusepath{clip}%
\pgfsetbuttcap%
\pgfsetroundjoin%
\definecolor{currentfill}{rgb}{1.000000,0.498039,0.054902}%
\pgfsetfillcolor{currentfill}%
\pgfsetlinewidth{1.003750pt}%
\definecolor{currentstroke}{rgb}{1.000000,0.498039,0.054902}%
\pgfsetstrokecolor{currentstroke}%
\pgfsetdash{}{0pt}%
\pgfpathmoveto{\pgfqpoint{1.339999in}{3.146912in}}%
\pgfpathcurveto{\pgfqpoint{1.351049in}{3.146912in}}{\pgfqpoint{1.361648in}{3.151303in}}{\pgfqpoint{1.369462in}{3.159116in}}%
\pgfpathcurveto{\pgfqpoint{1.377276in}{3.166930in}}{\pgfqpoint{1.381666in}{3.177529in}}{\pgfqpoint{1.381666in}{3.188579in}}%
\pgfpathcurveto{\pgfqpoint{1.381666in}{3.199629in}}{\pgfqpoint{1.377276in}{3.210228in}}{\pgfqpoint{1.369462in}{3.218042in}}%
\pgfpathcurveto{\pgfqpoint{1.361648in}{3.225856in}}{\pgfqpoint{1.351049in}{3.230246in}}{\pgfqpoint{1.339999in}{3.230246in}}%
\pgfpathcurveto{\pgfqpoint{1.328949in}{3.230246in}}{\pgfqpoint{1.318350in}{3.225856in}}{\pgfqpoint{1.310536in}{3.218042in}}%
\pgfpathcurveto{\pgfqpoint{1.302723in}{3.210228in}}{\pgfqpoint{1.298332in}{3.199629in}}{\pgfqpoint{1.298332in}{3.188579in}}%
\pgfpathcurveto{\pgfqpoint{1.298332in}{3.177529in}}{\pgfqpoint{1.302723in}{3.166930in}}{\pgfqpoint{1.310536in}{3.159116in}}%
\pgfpathcurveto{\pgfqpoint{1.318350in}{3.151303in}}{\pgfqpoint{1.328949in}{3.146912in}}{\pgfqpoint{1.339999in}{3.146912in}}%
\pgfpathclose%
\pgfusepath{stroke,fill}%
\end{pgfscope}%
\begin{pgfscope}%
\pgfpathrectangle{\pgfqpoint{0.648703in}{0.548769in}}{\pgfqpoint{5.112893in}{3.102590in}}%
\pgfusepath{clip}%
\pgfsetbuttcap%
\pgfsetroundjoin%
\definecolor{currentfill}{rgb}{1.000000,0.498039,0.054902}%
\pgfsetfillcolor{currentfill}%
\pgfsetlinewidth{1.003750pt}%
\definecolor{currentstroke}{rgb}{1.000000,0.498039,0.054902}%
\pgfsetstrokecolor{currentstroke}%
\pgfsetdash{}{0pt}%
\pgfpathmoveto{\pgfqpoint{0.972228in}{3.138662in}}%
\pgfpathcurveto{\pgfqpoint{0.983278in}{3.138662in}}{\pgfqpoint{0.993877in}{3.143053in}}{\pgfqpoint{1.001691in}{3.150866in}}%
\pgfpathcurveto{\pgfqpoint{1.009504in}{3.158680in}}{\pgfqpoint{1.013895in}{3.169279in}}{\pgfqpoint{1.013895in}{3.180329in}}%
\pgfpathcurveto{\pgfqpoint{1.013895in}{3.191379in}}{\pgfqpoint{1.009504in}{3.201978in}}{\pgfqpoint{1.001691in}{3.209792in}}%
\pgfpathcurveto{\pgfqpoint{0.993877in}{3.217605in}}{\pgfqpoint{0.983278in}{3.221996in}}{\pgfqpoint{0.972228in}{3.221996in}}%
\pgfpathcurveto{\pgfqpoint{0.961178in}{3.221996in}}{\pgfqpoint{0.950579in}{3.217605in}}{\pgfqpoint{0.942765in}{3.209792in}}%
\pgfpathcurveto{\pgfqpoint{0.934952in}{3.201978in}}{\pgfqpoint{0.930561in}{3.191379in}}{\pgfqpoint{0.930561in}{3.180329in}}%
\pgfpathcurveto{\pgfqpoint{0.930561in}{3.169279in}}{\pgfqpoint{0.934952in}{3.158680in}}{\pgfqpoint{0.942765in}{3.150866in}}%
\pgfpathcurveto{\pgfqpoint{0.950579in}{3.143053in}}{\pgfqpoint{0.961178in}{3.138662in}}{\pgfqpoint{0.972228in}{3.138662in}}%
\pgfpathclose%
\pgfusepath{stroke,fill}%
\end{pgfscope}%
\begin{pgfscope}%
\pgfpathrectangle{\pgfqpoint{0.648703in}{0.548769in}}{\pgfqpoint{5.112893in}{3.102590in}}%
\pgfusepath{clip}%
\pgfsetbuttcap%
\pgfsetroundjoin%
\definecolor{currentfill}{rgb}{1.000000,0.498039,0.054902}%
\pgfsetfillcolor{currentfill}%
\pgfsetlinewidth{1.003750pt}%
\definecolor{currentstroke}{rgb}{1.000000,0.498039,0.054902}%
\pgfsetstrokecolor{currentstroke}%
\pgfsetdash{}{0pt}%
\pgfpathmoveto{\pgfqpoint{1.571869in}{3.468665in}}%
\pgfpathcurveto{\pgfqpoint{1.582919in}{3.468665in}}{\pgfqpoint{1.593518in}{3.473055in}}{\pgfqpoint{1.601331in}{3.480869in}}%
\pgfpathcurveto{\pgfqpoint{1.609145in}{3.488683in}}{\pgfqpoint{1.613535in}{3.499282in}}{\pgfqpoint{1.613535in}{3.510332in}}%
\pgfpathcurveto{\pgfqpoint{1.613535in}{3.521382in}}{\pgfqpoint{1.609145in}{3.531981in}}{\pgfqpoint{1.601331in}{3.539795in}}%
\pgfpathcurveto{\pgfqpoint{1.593518in}{3.547608in}}{\pgfqpoint{1.582919in}{3.551998in}}{\pgfqpoint{1.571869in}{3.551998in}}%
\pgfpathcurveto{\pgfqpoint{1.560819in}{3.551998in}}{\pgfqpoint{1.550220in}{3.547608in}}{\pgfqpoint{1.542406in}{3.539795in}}%
\pgfpathcurveto{\pgfqpoint{1.534592in}{3.531981in}}{\pgfqpoint{1.530202in}{3.521382in}}{\pgfqpoint{1.530202in}{3.510332in}}%
\pgfpathcurveto{\pgfqpoint{1.530202in}{3.499282in}}{\pgfqpoint{1.534592in}{3.488683in}}{\pgfqpoint{1.542406in}{3.480869in}}%
\pgfpathcurveto{\pgfqpoint{1.550220in}{3.473055in}}{\pgfqpoint{1.560819in}{3.468665in}}{\pgfqpoint{1.571869in}{3.468665in}}%
\pgfpathclose%
\pgfusepath{stroke,fill}%
\end{pgfscope}%
\begin{pgfscope}%
\pgfpathrectangle{\pgfqpoint{0.648703in}{0.548769in}}{\pgfqpoint{5.112893in}{3.102590in}}%
\pgfusepath{clip}%
\pgfsetbuttcap%
\pgfsetroundjoin%
\definecolor{currentfill}{rgb}{1.000000,0.498039,0.054902}%
\pgfsetfillcolor{currentfill}%
\pgfsetlinewidth{1.003750pt}%
\definecolor{currentstroke}{rgb}{1.000000,0.498039,0.054902}%
\pgfsetstrokecolor{currentstroke}%
\pgfsetdash{}{0pt}%
\pgfpathmoveto{\pgfqpoint{1.331757in}{3.138662in}}%
\pgfpathcurveto{\pgfqpoint{1.342807in}{3.138662in}}{\pgfqpoint{1.353406in}{3.143053in}}{\pgfqpoint{1.361220in}{3.150866in}}%
\pgfpathcurveto{\pgfqpoint{1.369033in}{3.158680in}}{\pgfqpoint{1.373424in}{3.169279in}}{\pgfqpoint{1.373424in}{3.180329in}}%
\pgfpathcurveto{\pgfqpoint{1.373424in}{3.191379in}}{\pgfqpoint{1.369033in}{3.201978in}}{\pgfqpoint{1.361220in}{3.209792in}}%
\pgfpathcurveto{\pgfqpoint{1.353406in}{3.217605in}}{\pgfqpoint{1.342807in}{3.221996in}}{\pgfqpoint{1.331757in}{3.221996in}}%
\pgfpathcurveto{\pgfqpoint{1.320707in}{3.221996in}}{\pgfqpoint{1.310108in}{3.217605in}}{\pgfqpoint{1.302294in}{3.209792in}}%
\pgfpathcurveto{\pgfqpoint{1.294480in}{3.201978in}}{\pgfqpoint{1.290090in}{3.191379in}}{\pgfqpoint{1.290090in}{3.180329in}}%
\pgfpathcurveto{\pgfqpoint{1.290090in}{3.169279in}}{\pgfqpoint{1.294480in}{3.158680in}}{\pgfqpoint{1.302294in}{3.150866in}}%
\pgfpathcurveto{\pgfqpoint{1.310108in}{3.143053in}}{\pgfqpoint{1.320707in}{3.138662in}}{\pgfqpoint{1.331757in}{3.138662in}}%
\pgfpathclose%
\pgfusepath{stroke,fill}%
\end{pgfscope}%
\begin{pgfscope}%
\pgfpathrectangle{\pgfqpoint{0.648703in}{0.548769in}}{\pgfqpoint{5.112893in}{3.102590in}}%
\pgfusepath{clip}%
\pgfsetbuttcap%
\pgfsetroundjoin%
\definecolor{currentfill}{rgb}{1.000000,0.498039,0.054902}%
\pgfsetfillcolor{currentfill}%
\pgfsetlinewidth{1.003750pt}%
\definecolor{currentstroke}{rgb}{1.000000,0.498039,0.054902}%
\pgfsetstrokecolor{currentstroke}%
\pgfsetdash{}{0pt}%
\pgfpathmoveto{\pgfqpoint{1.326144in}{3.138662in}}%
\pgfpathcurveto{\pgfqpoint{1.337194in}{3.138662in}}{\pgfqpoint{1.347793in}{3.143053in}}{\pgfqpoint{1.355607in}{3.150866in}}%
\pgfpathcurveto{\pgfqpoint{1.363420in}{3.158680in}}{\pgfqpoint{1.367811in}{3.169279in}}{\pgfqpoint{1.367811in}{3.180329in}}%
\pgfpathcurveto{\pgfqpoint{1.367811in}{3.191379in}}{\pgfqpoint{1.363420in}{3.201978in}}{\pgfqpoint{1.355607in}{3.209792in}}%
\pgfpathcurveto{\pgfqpoint{1.347793in}{3.217605in}}{\pgfqpoint{1.337194in}{3.221996in}}{\pgfqpoint{1.326144in}{3.221996in}}%
\pgfpathcurveto{\pgfqpoint{1.315094in}{3.221996in}}{\pgfqpoint{1.304495in}{3.217605in}}{\pgfqpoint{1.296681in}{3.209792in}}%
\pgfpathcurveto{\pgfqpoint{1.288867in}{3.201978in}}{\pgfqpoint{1.284477in}{3.191379in}}{\pgfqpoint{1.284477in}{3.180329in}}%
\pgfpathcurveto{\pgfqpoint{1.284477in}{3.169279in}}{\pgfqpoint{1.288867in}{3.158680in}}{\pgfqpoint{1.296681in}{3.150866in}}%
\pgfpathcurveto{\pgfqpoint{1.304495in}{3.143053in}}{\pgfqpoint{1.315094in}{3.138662in}}{\pgfqpoint{1.326144in}{3.138662in}}%
\pgfpathclose%
\pgfusepath{stroke,fill}%
\end{pgfscope}%
\begin{pgfscope}%
\pgfpathrectangle{\pgfqpoint{0.648703in}{0.548769in}}{\pgfqpoint{5.112893in}{3.102590in}}%
\pgfusepath{clip}%
\pgfsetbuttcap%
\pgfsetroundjoin%
\definecolor{currentfill}{rgb}{1.000000,0.498039,0.054902}%
\pgfsetfillcolor{currentfill}%
\pgfsetlinewidth{1.003750pt}%
\definecolor{currentstroke}{rgb}{1.000000,0.498039,0.054902}%
\pgfsetstrokecolor{currentstroke}%
\pgfsetdash{}{0pt}%
\pgfpathmoveto{\pgfqpoint{1.212247in}{3.138662in}}%
\pgfpathcurveto{\pgfqpoint{1.223297in}{3.138662in}}{\pgfqpoint{1.233896in}{3.143053in}}{\pgfqpoint{1.241710in}{3.150866in}}%
\pgfpathcurveto{\pgfqpoint{1.249524in}{3.158680in}}{\pgfqpoint{1.253914in}{3.169279in}}{\pgfqpoint{1.253914in}{3.180329in}}%
\pgfpathcurveto{\pgfqpoint{1.253914in}{3.191379in}}{\pgfqpoint{1.249524in}{3.201978in}}{\pgfqpoint{1.241710in}{3.209792in}}%
\pgfpathcurveto{\pgfqpoint{1.233896in}{3.217605in}}{\pgfqpoint{1.223297in}{3.221996in}}{\pgfqpoint{1.212247in}{3.221996in}}%
\pgfpathcurveto{\pgfqpoint{1.201197in}{3.221996in}}{\pgfqpoint{1.190598in}{3.217605in}}{\pgfqpoint{1.182785in}{3.209792in}}%
\pgfpathcurveto{\pgfqpoint{1.174971in}{3.201978in}}{\pgfqpoint{1.170581in}{3.191379in}}{\pgfqpoint{1.170581in}{3.180329in}}%
\pgfpathcurveto{\pgfqpoint{1.170581in}{3.169279in}}{\pgfqpoint{1.174971in}{3.158680in}}{\pgfqpoint{1.182785in}{3.150866in}}%
\pgfpathcurveto{\pgfqpoint{1.190598in}{3.143053in}}{\pgfqpoint{1.201197in}{3.138662in}}{\pgfqpoint{1.212247in}{3.138662in}}%
\pgfpathclose%
\pgfusepath{stroke,fill}%
\end{pgfscope}%
\begin{pgfscope}%
\pgfpathrectangle{\pgfqpoint{0.648703in}{0.548769in}}{\pgfqpoint{5.112893in}{3.102590in}}%
\pgfusepath{clip}%
\pgfsetbuttcap%
\pgfsetroundjoin%
\definecolor{currentfill}{rgb}{0.839216,0.152941,0.156863}%
\pgfsetfillcolor{currentfill}%
\pgfsetlinewidth{1.003750pt}%
\definecolor{currentstroke}{rgb}{0.839216,0.152941,0.156863}%
\pgfsetstrokecolor{currentstroke}%
\pgfsetdash{}{0pt}%
\pgfpathmoveto{\pgfqpoint{1.788165in}{3.138662in}}%
\pgfpathcurveto{\pgfqpoint{1.799215in}{3.138662in}}{\pgfqpoint{1.809814in}{3.143053in}}{\pgfqpoint{1.817628in}{3.150866in}}%
\pgfpathcurveto{\pgfqpoint{1.825441in}{3.158680in}}{\pgfqpoint{1.829832in}{3.169279in}}{\pgfqpoint{1.829832in}{3.180329in}}%
\pgfpathcurveto{\pgfqpoint{1.829832in}{3.191379in}}{\pgfqpoint{1.825441in}{3.201978in}}{\pgfqpoint{1.817628in}{3.209792in}}%
\pgfpathcurveto{\pgfqpoint{1.809814in}{3.217605in}}{\pgfqpoint{1.799215in}{3.221996in}}{\pgfqpoint{1.788165in}{3.221996in}}%
\pgfpathcurveto{\pgfqpoint{1.777115in}{3.221996in}}{\pgfqpoint{1.766516in}{3.217605in}}{\pgfqpoint{1.758702in}{3.209792in}}%
\pgfpathcurveto{\pgfqpoint{1.750889in}{3.201978in}}{\pgfqpoint{1.746498in}{3.191379in}}{\pgfqpoint{1.746498in}{3.180329in}}%
\pgfpathcurveto{\pgfqpoint{1.746498in}{3.169279in}}{\pgfqpoint{1.750889in}{3.158680in}}{\pgfqpoint{1.758702in}{3.150866in}}%
\pgfpathcurveto{\pgfqpoint{1.766516in}{3.143053in}}{\pgfqpoint{1.777115in}{3.138662in}}{\pgfqpoint{1.788165in}{3.138662in}}%
\pgfpathclose%
\pgfusepath{stroke,fill}%
\end{pgfscope}%
\begin{pgfscope}%
\pgfpathrectangle{\pgfqpoint{0.648703in}{0.548769in}}{\pgfqpoint{5.112893in}{3.102590in}}%
\pgfusepath{clip}%
\pgfsetbuttcap%
\pgfsetroundjoin%
\definecolor{currentfill}{rgb}{1.000000,0.498039,0.054902}%
\pgfsetfillcolor{currentfill}%
\pgfsetlinewidth{1.003750pt}%
\definecolor{currentstroke}{rgb}{1.000000,0.498039,0.054902}%
\pgfsetstrokecolor{currentstroke}%
\pgfsetdash{}{0pt}%
\pgfpathmoveto{\pgfqpoint{1.992573in}{3.151038in}}%
\pgfpathcurveto{\pgfqpoint{2.003623in}{3.151038in}}{\pgfqpoint{2.014222in}{3.155428in}}{\pgfqpoint{2.022036in}{3.163241in}}%
\pgfpathcurveto{\pgfqpoint{2.029849in}{3.171055in}}{\pgfqpoint{2.034240in}{3.181654in}}{\pgfqpoint{2.034240in}{3.192704in}}%
\pgfpathcurveto{\pgfqpoint{2.034240in}{3.203754in}}{\pgfqpoint{2.029849in}{3.214353in}}{\pgfqpoint{2.022036in}{3.222167in}}%
\pgfpathcurveto{\pgfqpoint{2.014222in}{3.229981in}}{\pgfqpoint{2.003623in}{3.234371in}}{\pgfqpoint{1.992573in}{3.234371in}}%
\pgfpathcurveto{\pgfqpoint{1.981523in}{3.234371in}}{\pgfqpoint{1.970924in}{3.229981in}}{\pgfqpoint{1.963110in}{3.222167in}}%
\pgfpathcurveto{\pgfqpoint{1.955296in}{3.214353in}}{\pgfqpoint{1.950906in}{3.203754in}}{\pgfqpoint{1.950906in}{3.192704in}}%
\pgfpathcurveto{\pgfqpoint{1.950906in}{3.181654in}}{\pgfqpoint{1.955296in}{3.171055in}}{\pgfqpoint{1.963110in}{3.163241in}}%
\pgfpathcurveto{\pgfqpoint{1.970924in}{3.155428in}}{\pgfqpoint{1.981523in}{3.151038in}}{\pgfqpoint{1.992573in}{3.151038in}}%
\pgfpathclose%
\pgfusepath{stroke,fill}%
\end{pgfscope}%
\begin{pgfscope}%
\pgfpathrectangle{\pgfqpoint{0.648703in}{0.548769in}}{\pgfqpoint{5.112893in}{3.102590in}}%
\pgfusepath{clip}%
\pgfsetbuttcap%
\pgfsetroundjoin%
\definecolor{currentfill}{rgb}{1.000000,0.498039,0.054902}%
\pgfsetfillcolor{currentfill}%
\pgfsetlinewidth{1.003750pt}%
\definecolor{currentstroke}{rgb}{1.000000,0.498039,0.054902}%
\pgfsetstrokecolor{currentstroke}%
\pgfsetdash{}{0pt}%
\pgfpathmoveto{\pgfqpoint{1.837364in}{3.287164in}}%
\pgfpathcurveto{\pgfqpoint{1.848414in}{3.287164in}}{\pgfqpoint{1.859013in}{3.291554in}}{\pgfqpoint{1.866827in}{3.299368in}}%
\pgfpathcurveto{\pgfqpoint{1.874641in}{3.307181in}}{\pgfqpoint{1.879031in}{3.317780in}}{\pgfqpoint{1.879031in}{3.328830in}}%
\pgfpathcurveto{\pgfqpoint{1.879031in}{3.339880in}}{\pgfqpoint{1.874641in}{3.350479in}}{\pgfqpoint{1.866827in}{3.358293in}}%
\pgfpathcurveto{\pgfqpoint{1.859013in}{3.366107in}}{\pgfqpoint{1.848414in}{3.370497in}}{\pgfqpoint{1.837364in}{3.370497in}}%
\pgfpathcurveto{\pgfqpoint{1.826314in}{3.370497in}}{\pgfqpoint{1.815715in}{3.366107in}}{\pgfqpoint{1.807901in}{3.358293in}}%
\pgfpathcurveto{\pgfqpoint{1.800088in}{3.350479in}}{\pgfqpoint{1.795698in}{3.339880in}}{\pgfqpoint{1.795698in}{3.328830in}}%
\pgfpathcurveto{\pgfqpoint{1.795698in}{3.317780in}}{\pgfqpoint{1.800088in}{3.307181in}}{\pgfqpoint{1.807901in}{3.299368in}}%
\pgfpathcurveto{\pgfqpoint{1.815715in}{3.291554in}}{\pgfqpoint{1.826314in}{3.287164in}}{\pgfqpoint{1.837364in}{3.287164in}}%
\pgfpathclose%
\pgfusepath{stroke,fill}%
\end{pgfscope}%
\begin{pgfscope}%
\pgfpathrectangle{\pgfqpoint{0.648703in}{0.548769in}}{\pgfqpoint{5.112893in}{3.102590in}}%
\pgfusepath{clip}%
\pgfsetbuttcap%
\pgfsetroundjoin%
\definecolor{currentfill}{rgb}{1.000000,0.498039,0.054902}%
\pgfsetfillcolor{currentfill}%
\pgfsetlinewidth{1.003750pt}%
\definecolor{currentstroke}{rgb}{1.000000,0.498039,0.054902}%
\pgfsetstrokecolor{currentstroke}%
\pgfsetdash{}{0pt}%
\pgfpathmoveto{\pgfqpoint{1.313355in}{3.138662in}}%
\pgfpathcurveto{\pgfqpoint{1.324405in}{3.138662in}}{\pgfqpoint{1.335004in}{3.143053in}}{\pgfqpoint{1.342817in}{3.150866in}}%
\pgfpathcurveto{\pgfqpoint{1.350631in}{3.158680in}}{\pgfqpoint{1.355021in}{3.169279in}}{\pgfqpoint{1.355021in}{3.180329in}}%
\pgfpathcurveto{\pgfqpoint{1.355021in}{3.191379in}}{\pgfqpoint{1.350631in}{3.201978in}}{\pgfqpoint{1.342817in}{3.209792in}}%
\pgfpathcurveto{\pgfqpoint{1.335004in}{3.217605in}}{\pgfqpoint{1.324405in}{3.221996in}}{\pgfqpoint{1.313355in}{3.221996in}}%
\pgfpathcurveto{\pgfqpoint{1.302304in}{3.221996in}}{\pgfqpoint{1.291705in}{3.217605in}}{\pgfqpoint{1.283892in}{3.209792in}}%
\pgfpathcurveto{\pgfqpoint{1.276078in}{3.201978in}}{\pgfqpoint{1.271688in}{3.191379in}}{\pgfqpoint{1.271688in}{3.180329in}}%
\pgfpathcurveto{\pgfqpoint{1.271688in}{3.169279in}}{\pgfqpoint{1.276078in}{3.158680in}}{\pgfqpoint{1.283892in}{3.150866in}}%
\pgfpathcurveto{\pgfqpoint{1.291705in}{3.143053in}}{\pgfqpoint{1.302304in}{3.138662in}}{\pgfqpoint{1.313355in}{3.138662in}}%
\pgfpathclose%
\pgfusepath{stroke,fill}%
\end{pgfscope}%
\begin{pgfscope}%
\pgfpathrectangle{\pgfqpoint{0.648703in}{0.548769in}}{\pgfqpoint{5.112893in}{3.102590in}}%
\pgfusepath{clip}%
\pgfsetbuttcap%
\pgfsetroundjoin%
\definecolor{currentfill}{rgb}{1.000000,0.498039,0.054902}%
\pgfsetfillcolor{currentfill}%
\pgfsetlinewidth{1.003750pt}%
\definecolor{currentstroke}{rgb}{1.000000,0.498039,0.054902}%
\pgfsetstrokecolor{currentstroke}%
\pgfsetdash{}{0pt}%
\pgfpathmoveto{\pgfqpoint{1.383750in}{3.151038in}}%
\pgfpathcurveto{\pgfqpoint{1.394800in}{3.151038in}}{\pgfqpoint{1.405399in}{3.155428in}}{\pgfqpoint{1.413213in}{3.163241in}}%
\pgfpathcurveto{\pgfqpoint{1.421026in}{3.171055in}}{\pgfqpoint{1.425417in}{3.181654in}}{\pgfqpoint{1.425417in}{3.192704in}}%
\pgfpathcurveto{\pgfqpoint{1.425417in}{3.203754in}}{\pgfqpoint{1.421026in}{3.214353in}}{\pgfqpoint{1.413213in}{3.222167in}}%
\pgfpathcurveto{\pgfqpoint{1.405399in}{3.229981in}}{\pgfqpoint{1.394800in}{3.234371in}}{\pgfqpoint{1.383750in}{3.234371in}}%
\pgfpathcurveto{\pgfqpoint{1.372700in}{3.234371in}}{\pgfqpoint{1.362101in}{3.229981in}}{\pgfqpoint{1.354287in}{3.222167in}}%
\pgfpathcurveto{\pgfqpoint{1.346474in}{3.214353in}}{\pgfqpoint{1.342083in}{3.203754in}}{\pgfqpoint{1.342083in}{3.192704in}}%
\pgfpathcurveto{\pgfqpoint{1.342083in}{3.181654in}}{\pgfqpoint{1.346474in}{3.171055in}}{\pgfqpoint{1.354287in}{3.163241in}}%
\pgfpathcurveto{\pgfqpoint{1.362101in}{3.155428in}}{\pgfqpoint{1.372700in}{3.151038in}}{\pgfqpoint{1.383750in}{3.151038in}}%
\pgfpathclose%
\pgfusepath{stroke,fill}%
\end{pgfscope}%
\begin{pgfscope}%
\pgfpathrectangle{\pgfqpoint{0.648703in}{0.548769in}}{\pgfqpoint{5.112893in}{3.102590in}}%
\pgfusepath{clip}%
\pgfsetbuttcap%
\pgfsetroundjoin%
\definecolor{currentfill}{rgb}{0.121569,0.466667,0.705882}%
\pgfsetfillcolor{currentfill}%
\pgfsetlinewidth{1.003750pt}%
\definecolor{currentstroke}{rgb}{0.121569,0.466667,0.705882}%
\pgfsetstrokecolor{currentstroke}%
\pgfsetdash{}{0pt}%
\pgfpathmoveto{\pgfqpoint{1.414823in}{3.130412in}}%
\pgfpathcurveto{\pgfqpoint{1.425873in}{3.130412in}}{\pgfqpoint{1.436473in}{3.134803in}}{\pgfqpoint{1.444286in}{3.142616in}}%
\pgfpathcurveto{\pgfqpoint{1.452100in}{3.150430in}}{\pgfqpoint{1.456490in}{3.161029in}}{\pgfqpoint{1.456490in}{3.172079in}}%
\pgfpathcurveto{\pgfqpoint{1.456490in}{3.183129in}}{\pgfqpoint{1.452100in}{3.193728in}}{\pgfqpoint{1.444286in}{3.201542in}}%
\pgfpathcurveto{\pgfqpoint{1.436473in}{3.209355in}}{\pgfqpoint{1.425873in}{3.213746in}}{\pgfqpoint{1.414823in}{3.213746in}}%
\pgfpathcurveto{\pgfqpoint{1.403773in}{3.213746in}}{\pgfqpoint{1.393174in}{3.209355in}}{\pgfqpoint{1.385361in}{3.201542in}}%
\pgfpathcurveto{\pgfqpoint{1.377547in}{3.193728in}}{\pgfqpoint{1.373157in}{3.183129in}}{\pgfqpoint{1.373157in}{3.172079in}}%
\pgfpathcurveto{\pgfqpoint{1.373157in}{3.161029in}}{\pgfqpoint{1.377547in}{3.150430in}}{\pgfqpoint{1.385361in}{3.142616in}}%
\pgfpathcurveto{\pgfqpoint{1.393174in}{3.134803in}}{\pgfqpoint{1.403773in}{3.130412in}}{\pgfqpoint{1.414823in}{3.130412in}}%
\pgfpathclose%
\pgfusepath{stroke,fill}%
\end{pgfscope}%
\begin{pgfscope}%
\pgfpathrectangle{\pgfqpoint{0.648703in}{0.548769in}}{\pgfqpoint{5.112893in}{3.102590in}}%
\pgfusepath{clip}%
\pgfsetbuttcap%
\pgfsetroundjoin%
\definecolor{currentfill}{rgb}{1.000000,0.498039,0.054902}%
\pgfsetfillcolor{currentfill}%
\pgfsetlinewidth{1.003750pt}%
\definecolor{currentstroke}{rgb}{1.000000,0.498039,0.054902}%
\pgfsetstrokecolor{currentstroke}%
\pgfsetdash{}{0pt}%
\pgfpathmoveto{\pgfqpoint{1.603646in}{3.233538in}}%
\pgfpathcurveto{\pgfqpoint{1.614696in}{3.233538in}}{\pgfqpoint{1.625295in}{3.237928in}}{\pgfqpoint{1.633109in}{3.245742in}}%
\pgfpathcurveto{\pgfqpoint{1.640922in}{3.253556in}}{\pgfqpoint{1.645313in}{3.264155in}}{\pgfqpoint{1.645313in}{3.275205in}}%
\pgfpathcurveto{\pgfqpoint{1.645313in}{3.286255in}}{\pgfqpoint{1.640922in}{3.296854in}}{\pgfqpoint{1.633109in}{3.304668in}}%
\pgfpathcurveto{\pgfqpoint{1.625295in}{3.312481in}}{\pgfqpoint{1.614696in}{3.316872in}}{\pgfqpoint{1.603646in}{3.316872in}}%
\pgfpathcurveto{\pgfqpoint{1.592596in}{3.316872in}}{\pgfqpoint{1.581997in}{3.312481in}}{\pgfqpoint{1.574183in}{3.304668in}}%
\pgfpathcurveto{\pgfqpoint{1.566370in}{3.296854in}}{\pgfqpoint{1.561979in}{3.286255in}}{\pgfqpoint{1.561979in}{3.275205in}}%
\pgfpathcurveto{\pgfqpoint{1.561979in}{3.264155in}}{\pgfqpoint{1.566370in}{3.253556in}}{\pgfqpoint{1.574183in}{3.245742in}}%
\pgfpathcurveto{\pgfqpoint{1.581997in}{3.237928in}}{\pgfqpoint{1.592596in}{3.233538in}}{\pgfqpoint{1.603646in}{3.233538in}}%
\pgfpathclose%
\pgfusepath{stroke,fill}%
\end{pgfscope}%
\begin{pgfscope}%
\pgfpathrectangle{\pgfqpoint{0.648703in}{0.548769in}}{\pgfqpoint{5.112893in}{3.102590in}}%
\pgfusepath{clip}%
\pgfsetbuttcap%
\pgfsetroundjoin%
\definecolor{currentfill}{rgb}{0.121569,0.466667,0.705882}%
\pgfsetfillcolor{currentfill}%
\pgfsetlinewidth{1.003750pt}%
\definecolor{currentstroke}{rgb}{0.121569,0.466667,0.705882}%
\pgfsetstrokecolor{currentstroke}%
\pgfsetdash{}{0pt}%
\pgfpathmoveto{\pgfqpoint{0.719376in}{0.663642in}}%
\pgfpathcurveto{\pgfqpoint{0.730426in}{0.663642in}}{\pgfqpoint{0.741025in}{0.668032in}}{\pgfqpoint{0.748838in}{0.675846in}}%
\pgfpathcurveto{\pgfqpoint{0.756652in}{0.683659in}}{\pgfqpoint{0.761042in}{0.694258in}}{\pgfqpoint{0.761042in}{0.705309in}}%
\pgfpathcurveto{\pgfqpoint{0.761042in}{0.716359in}}{\pgfqpoint{0.756652in}{0.726958in}}{\pgfqpoint{0.748838in}{0.734771in}}%
\pgfpathcurveto{\pgfqpoint{0.741025in}{0.742585in}}{\pgfqpoint{0.730426in}{0.746975in}}{\pgfqpoint{0.719376in}{0.746975in}}%
\pgfpathcurveto{\pgfqpoint{0.708325in}{0.746975in}}{\pgfqpoint{0.697726in}{0.742585in}}{\pgfqpoint{0.689913in}{0.734771in}}%
\pgfpathcurveto{\pgfqpoint{0.682099in}{0.726958in}}{\pgfqpoint{0.677709in}{0.716359in}}{\pgfqpoint{0.677709in}{0.705309in}}%
\pgfpathcurveto{\pgfqpoint{0.677709in}{0.694258in}}{\pgfqpoint{0.682099in}{0.683659in}}{\pgfqpoint{0.689913in}{0.675846in}}%
\pgfpathcurveto{\pgfqpoint{0.697726in}{0.668032in}}{\pgfqpoint{0.708325in}{0.663642in}}{\pgfqpoint{0.719376in}{0.663642in}}%
\pgfpathclose%
\pgfusepath{stroke,fill}%
\end{pgfscope}%
\begin{pgfscope}%
\pgfpathrectangle{\pgfqpoint{0.648703in}{0.548769in}}{\pgfqpoint{5.112893in}{3.102590in}}%
\pgfusepath{clip}%
\pgfsetbuttcap%
\pgfsetroundjoin%
\definecolor{currentfill}{rgb}{1.000000,0.498039,0.054902}%
\pgfsetfillcolor{currentfill}%
\pgfsetlinewidth{1.003750pt}%
\definecolor{currentstroke}{rgb}{1.000000,0.498039,0.054902}%
\pgfsetstrokecolor{currentstroke}%
\pgfsetdash{}{0pt}%
\pgfpathmoveto{\pgfqpoint{1.114108in}{3.138662in}}%
\pgfpathcurveto{\pgfqpoint{1.125159in}{3.138662in}}{\pgfqpoint{1.135758in}{3.143053in}}{\pgfqpoint{1.143571in}{3.150866in}}%
\pgfpathcurveto{\pgfqpoint{1.151385in}{3.158680in}}{\pgfqpoint{1.155775in}{3.169279in}}{\pgfqpoint{1.155775in}{3.180329in}}%
\pgfpathcurveto{\pgfqpoint{1.155775in}{3.191379in}}{\pgfqpoint{1.151385in}{3.201978in}}{\pgfqpoint{1.143571in}{3.209792in}}%
\pgfpathcurveto{\pgfqpoint{1.135758in}{3.217605in}}{\pgfqpoint{1.125159in}{3.221996in}}{\pgfqpoint{1.114108in}{3.221996in}}%
\pgfpathcurveto{\pgfqpoint{1.103058in}{3.221996in}}{\pgfqpoint{1.092459in}{3.217605in}}{\pgfqpoint{1.084646in}{3.209792in}}%
\pgfpathcurveto{\pgfqpoint{1.076832in}{3.201978in}}{\pgfqpoint{1.072442in}{3.191379in}}{\pgfqpoint{1.072442in}{3.180329in}}%
\pgfpathcurveto{\pgfqpoint{1.072442in}{3.169279in}}{\pgfqpoint{1.076832in}{3.158680in}}{\pgfqpoint{1.084646in}{3.150866in}}%
\pgfpathcurveto{\pgfqpoint{1.092459in}{3.143053in}}{\pgfqpoint{1.103058in}{3.138662in}}{\pgfqpoint{1.114108in}{3.138662in}}%
\pgfpathclose%
\pgfusepath{stroke,fill}%
\end{pgfscope}%
\begin{pgfscope}%
\pgfpathrectangle{\pgfqpoint{0.648703in}{0.548769in}}{\pgfqpoint{5.112893in}{3.102590in}}%
\pgfusepath{clip}%
\pgfsetbuttcap%
\pgfsetroundjoin%
\definecolor{currentfill}{rgb}{1.000000,0.498039,0.054902}%
\pgfsetfillcolor{currentfill}%
\pgfsetlinewidth{1.003750pt}%
\definecolor{currentstroke}{rgb}{1.000000,0.498039,0.054902}%
\pgfsetstrokecolor{currentstroke}%
\pgfsetdash{}{0pt}%
\pgfpathmoveto{\pgfqpoint{1.591543in}{3.142787in}}%
\pgfpathcurveto{\pgfqpoint{1.602594in}{3.142787in}}{\pgfqpoint{1.613193in}{3.147178in}}{\pgfqpoint{1.621006in}{3.154991in}}%
\pgfpathcurveto{\pgfqpoint{1.628820in}{3.162805in}}{\pgfqpoint{1.633210in}{3.173404in}}{\pgfqpoint{1.633210in}{3.184454in}}%
\pgfpathcurveto{\pgfqpoint{1.633210in}{3.195504in}}{\pgfqpoint{1.628820in}{3.206103in}}{\pgfqpoint{1.621006in}{3.213917in}}%
\pgfpathcurveto{\pgfqpoint{1.613193in}{3.221731in}}{\pgfqpoint{1.602594in}{3.226121in}}{\pgfqpoint{1.591543in}{3.226121in}}%
\pgfpathcurveto{\pgfqpoint{1.580493in}{3.226121in}}{\pgfqpoint{1.569894in}{3.221731in}}{\pgfqpoint{1.562081in}{3.213917in}}%
\pgfpathcurveto{\pgfqpoint{1.554267in}{3.206103in}}{\pgfqpoint{1.549877in}{3.195504in}}{\pgfqpoint{1.549877in}{3.184454in}}%
\pgfpathcurveto{\pgfqpoint{1.549877in}{3.173404in}}{\pgfqpoint{1.554267in}{3.162805in}}{\pgfqpoint{1.562081in}{3.154991in}}%
\pgfpathcurveto{\pgfqpoint{1.569894in}{3.147178in}}{\pgfqpoint{1.580493in}{3.142787in}}{\pgfqpoint{1.591543in}{3.142787in}}%
\pgfpathclose%
\pgfusepath{stroke,fill}%
\end{pgfscope}%
\begin{pgfscope}%
\pgfpathrectangle{\pgfqpoint{0.648703in}{0.548769in}}{\pgfqpoint{5.112893in}{3.102590in}}%
\pgfusepath{clip}%
\pgfsetbuttcap%
\pgfsetroundjoin%
\definecolor{currentfill}{rgb}{1.000000,0.498039,0.054902}%
\pgfsetfillcolor{currentfill}%
\pgfsetlinewidth{1.003750pt}%
\definecolor{currentstroke}{rgb}{1.000000,0.498039,0.054902}%
\pgfsetstrokecolor{currentstroke}%
\pgfsetdash{}{0pt}%
\pgfpathmoveto{\pgfqpoint{1.495131in}{3.151038in}}%
\pgfpathcurveto{\pgfqpoint{1.506181in}{3.151038in}}{\pgfqpoint{1.516780in}{3.155428in}}{\pgfqpoint{1.524594in}{3.163241in}}%
\pgfpathcurveto{\pgfqpoint{1.532407in}{3.171055in}}{\pgfqpoint{1.536798in}{3.181654in}}{\pgfqpoint{1.536798in}{3.192704in}}%
\pgfpathcurveto{\pgfqpoint{1.536798in}{3.203754in}}{\pgfqpoint{1.532407in}{3.214353in}}{\pgfqpoint{1.524594in}{3.222167in}}%
\pgfpathcurveto{\pgfqpoint{1.516780in}{3.229981in}}{\pgfqpoint{1.506181in}{3.234371in}}{\pgfqpoint{1.495131in}{3.234371in}}%
\pgfpathcurveto{\pgfqpoint{1.484081in}{3.234371in}}{\pgfqpoint{1.473482in}{3.229981in}}{\pgfqpoint{1.465668in}{3.222167in}}%
\pgfpathcurveto{\pgfqpoint{1.457855in}{3.214353in}}{\pgfqpoint{1.453464in}{3.203754in}}{\pgfqpoint{1.453464in}{3.192704in}}%
\pgfpathcurveto{\pgfqpoint{1.453464in}{3.181654in}}{\pgfqpoint{1.457855in}{3.171055in}}{\pgfqpoint{1.465668in}{3.163241in}}%
\pgfpathcurveto{\pgfqpoint{1.473482in}{3.155428in}}{\pgfqpoint{1.484081in}{3.151038in}}{\pgfqpoint{1.495131in}{3.151038in}}%
\pgfpathclose%
\pgfusepath{stroke,fill}%
\end{pgfscope}%
\begin{pgfscope}%
\pgfpathrectangle{\pgfqpoint{0.648703in}{0.548769in}}{\pgfqpoint{5.112893in}{3.102590in}}%
\pgfusepath{clip}%
\pgfsetbuttcap%
\pgfsetroundjoin%
\definecolor{currentfill}{rgb}{1.000000,0.498039,0.054902}%
\pgfsetfillcolor{currentfill}%
\pgfsetlinewidth{1.003750pt}%
\definecolor{currentstroke}{rgb}{1.000000,0.498039,0.054902}%
\pgfsetstrokecolor{currentstroke}%
\pgfsetdash{}{0pt}%
\pgfpathmoveto{\pgfqpoint{1.163389in}{3.142787in}}%
\pgfpathcurveto{\pgfqpoint{1.174439in}{3.142787in}}{\pgfqpoint{1.185038in}{3.147178in}}{\pgfqpoint{1.192852in}{3.154991in}}%
\pgfpathcurveto{\pgfqpoint{1.200665in}{3.162805in}}{\pgfqpoint{1.205055in}{3.173404in}}{\pgfqpoint{1.205055in}{3.184454in}}%
\pgfpathcurveto{\pgfqpoint{1.205055in}{3.195504in}}{\pgfqpoint{1.200665in}{3.206103in}}{\pgfqpoint{1.192852in}{3.213917in}}%
\pgfpathcurveto{\pgfqpoint{1.185038in}{3.221731in}}{\pgfqpoint{1.174439in}{3.226121in}}{\pgfqpoint{1.163389in}{3.226121in}}%
\pgfpathcurveto{\pgfqpoint{1.152339in}{3.226121in}}{\pgfqpoint{1.141740in}{3.221731in}}{\pgfqpoint{1.133926in}{3.213917in}}%
\pgfpathcurveto{\pgfqpoint{1.126112in}{3.206103in}}{\pgfqpoint{1.121722in}{3.195504in}}{\pgfqpoint{1.121722in}{3.184454in}}%
\pgfpathcurveto{\pgfqpoint{1.121722in}{3.173404in}}{\pgfqpoint{1.126112in}{3.162805in}}{\pgfqpoint{1.133926in}{3.154991in}}%
\pgfpathcurveto{\pgfqpoint{1.141740in}{3.147178in}}{\pgfqpoint{1.152339in}{3.142787in}}{\pgfqpoint{1.163389in}{3.142787in}}%
\pgfpathclose%
\pgfusepath{stroke,fill}%
\end{pgfscope}%
\begin{pgfscope}%
\pgfpathrectangle{\pgfqpoint{0.648703in}{0.548769in}}{\pgfqpoint{5.112893in}{3.102590in}}%
\pgfusepath{clip}%
\pgfsetbuttcap%
\pgfsetroundjoin%
\definecolor{currentfill}{rgb}{1.000000,0.498039,0.054902}%
\pgfsetfillcolor{currentfill}%
\pgfsetlinewidth{1.003750pt}%
\definecolor{currentstroke}{rgb}{1.000000,0.498039,0.054902}%
\pgfsetstrokecolor{currentstroke}%
\pgfsetdash{}{0pt}%
\pgfpathmoveto{\pgfqpoint{1.135300in}{3.138662in}}%
\pgfpathcurveto{\pgfqpoint{1.146350in}{3.138662in}}{\pgfqpoint{1.156949in}{3.143053in}}{\pgfqpoint{1.164763in}{3.150866in}}%
\pgfpathcurveto{\pgfqpoint{1.172577in}{3.158680in}}{\pgfqpoint{1.176967in}{3.169279in}}{\pgfqpoint{1.176967in}{3.180329in}}%
\pgfpathcurveto{\pgfqpoint{1.176967in}{3.191379in}}{\pgfqpoint{1.172577in}{3.201978in}}{\pgfqpoint{1.164763in}{3.209792in}}%
\pgfpathcurveto{\pgfqpoint{1.156949in}{3.217605in}}{\pgfqpoint{1.146350in}{3.221996in}}{\pgfqpoint{1.135300in}{3.221996in}}%
\pgfpathcurveto{\pgfqpoint{1.124250in}{3.221996in}}{\pgfqpoint{1.113651in}{3.217605in}}{\pgfqpoint{1.105837in}{3.209792in}}%
\pgfpathcurveto{\pgfqpoint{1.098024in}{3.201978in}}{\pgfqpoint{1.093633in}{3.191379in}}{\pgfqpoint{1.093633in}{3.180329in}}%
\pgfpathcurveto{\pgfqpoint{1.093633in}{3.169279in}}{\pgfqpoint{1.098024in}{3.158680in}}{\pgfqpoint{1.105837in}{3.150866in}}%
\pgfpathcurveto{\pgfqpoint{1.113651in}{3.143053in}}{\pgfqpoint{1.124250in}{3.138662in}}{\pgfqpoint{1.135300in}{3.138662in}}%
\pgfpathclose%
\pgfusepath{stroke,fill}%
\end{pgfscope}%
\begin{pgfscope}%
\pgfpathrectangle{\pgfqpoint{0.648703in}{0.548769in}}{\pgfqpoint{5.112893in}{3.102590in}}%
\pgfusepath{clip}%
\pgfsetbuttcap%
\pgfsetroundjoin%
\definecolor{currentfill}{rgb}{1.000000,0.498039,0.054902}%
\pgfsetfillcolor{currentfill}%
\pgfsetlinewidth{1.003750pt}%
\definecolor{currentstroke}{rgb}{1.000000,0.498039,0.054902}%
\pgfsetstrokecolor{currentstroke}%
\pgfsetdash{}{0pt}%
\pgfpathmoveto{\pgfqpoint{1.122942in}{3.146912in}}%
\pgfpathcurveto{\pgfqpoint{1.133992in}{3.146912in}}{\pgfqpoint{1.144591in}{3.151303in}}{\pgfqpoint{1.152404in}{3.159116in}}%
\pgfpathcurveto{\pgfqpoint{1.160218in}{3.166930in}}{\pgfqpoint{1.164608in}{3.177529in}}{\pgfqpoint{1.164608in}{3.188579in}}%
\pgfpathcurveto{\pgfqpoint{1.164608in}{3.199629in}}{\pgfqpoint{1.160218in}{3.210228in}}{\pgfqpoint{1.152404in}{3.218042in}}%
\pgfpathcurveto{\pgfqpoint{1.144591in}{3.225856in}}{\pgfqpoint{1.133992in}{3.230246in}}{\pgfqpoint{1.122942in}{3.230246in}}%
\pgfpathcurveto{\pgfqpoint{1.111892in}{3.230246in}}{\pgfqpoint{1.101293in}{3.225856in}}{\pgfqpoint{1.093479in}{3.218042in}}%
\pgfpathcurveto{\pgfqpoint{1.085665in}{3.210228in}}{\pgfqpoint{1.081275in}{3.199629in}}{\pgfqpoint{1.081275in}{3.188579in}}%
\pgfpathcurveto{\pgfqpoint{1.081275in}{3.177529in}}{\pgfqpoint{1.085665in}{3.166930in}}{\pgfqpoint{1.093479in}{3.159116in}}%
\pgfpathcurveto{\pgfqpoint{1.101293in}{3.151303in}}{\pgfqpoint{1.111892in}{3.146912in}}{\pgfqpoint{1.122942in}{3.146912in}}%
\pgfpathclose%
\pgfusepath{stroke,fill}%
\end{pgfscope}%
\begin{pgfscope}%
\pgfpathrectangle{\pgfqpoint{0.648703in}{0.548769in}}{\pgfqpoint{5.112893in}{3.102590in}}%
\pgfusepath{clip}%
\pgfsetbuttcap%
\pgfsetroundjoin%
\definecolor{currentfill}{rgb}{0.121569,0.466667,0.705882}%
\pgfsetfillcolor{currentfill}%
\pgfsetlinewidth{1.003750pt}%
\definecolor{currentstroke}{rgb}{0.121569,0.466667,0.705882}%
\pgfsetstrokecolor{currentstroke}%
\pgfsetdash{}{0pt}%
\pgfpathmoveto{\pgfqpoint{1.422293in}{3.134537in}}%
\pgfpathcurveto{\pgfqpoint{1.433343in}{3.134537in}}{\pgfqpoint{1.443942in}{3.138928in}}{\pgfqpoint{1.451756in}{3.146741in}}%
\pgfpathcurveto{\pgfqpoint{1.459569in}{3.154555in}}{\pgfqpoint{1.463960in}{3.165154in}}{\pgfqpoint{1.463960in}{3.176204in}}%
\pgfpathcurveto{\pgfqpoint{1.463960in}{3.187254in}}{\pgfqpoint{1.459569in}{3.197853in}}{\pgfqpoint{1.451756in}{3.205667in}}%
\pgfpathcurveto{\pgfqpoint{1.443942in}{3.213480in}}{\pgfqpoint{1.433343in}{3.217871in}}{\pgfqpoint{1.422293in}{3.217871in}}%
\pgfpathcurveto{\pgfqpoint{1.411243in}{3.217871in}}{\pgfqpoint{1.400644in}{3.213480in}}{\pgfqpoint{1.392830in}{3.205667in}}%
\pgfpathcurveto{\pgfqpoint{1.385017in}{3.197853in}}{\pgfqpoint{1.380626in}{3.187254in}}{\pgfqpoint{1.380626in}{3.176204in}}%
\pgfpathcurveto{\pgfqpoint{1.380626in}{3.165154in}}{\pgfqpoint{1.385017in}{3.154555in}}{\pgfqpoint{1.392830in}{3.146741in}}%
\pgfpathcurveto{\pgfqpoint{1.400644in}{3.138928in}}{\pgfqpoint{1.411243in}{3.134537in}}{\pgfqpoint{1.422293in}{3.134537in}}%
\pgfpathclose%
\pgfusepath{stroke,fill}%
\end{pgfscope}%
\begin{pgfscope}%
\pgfpathrectangle{\pgfqpoint{0.648703in}{0.548769in}}{\pgfqpoint{5.112893in}{3.102590in}}%
\pgfusepath{clip}%
\pgfsetbuttcap%
\pgfsetroundjoin%
\definecolor{currentfill}{rgb}{1.000000,0.498039,0.054902}%
\pgfsetfillcolor{currentfill}%
\pgfsetlinewidth{1.003750pt}%
\definecolor{currentstroke}{rgb}{1.000000,0.498039,0.054902}%
\pgfsetstrokecolor{currentstroke}%
\pgfsetdash{}{0pt}%
\pgfpathmoveto{\pgfqpoint{1.750106in}{3.349039in}}%
\pgfpathcurveto{\pgfqpoint{1.761156in}{3.349039in}}{\pgfqpoint{1.771755in}{3.353429in}}{\pgfqpoint{1.779569in}{3.361243in}}%
\pgfpathcurveto{\pgfqpoint{1.787383in}{3.369057in}}{\pgfqpoint{1.791773in}{3.379656in}}{\pgfqpoint{1.791773in}{3.390706in}}%
\pgfpathcurveto{\pgfqpoint{1.791773in}{3.401756in}}{\pgfqpoint{1.787383in}{3.412355in}}{\pgfqpoint{1.779569in}{3.420169in}}%
\pgfpathcurveto{\pgfqpoint{1.771755in}{3.427982in}}{\pgfqpoint{1.761156in}{3.432372in}}{\pgfqpoint{1.750106in}{3.432372in}}%
\pgfpathcurveto{\pgfqpoint{1.739056in}{3.432372in}}{\pgfqpoint{1.728457in}{3.427982in}}{\pgfqpoint{1.720643in}{3.420169in}}%
\pgfpathcurveto{\pgfqpoint{1.712830in}{3.412355in}}{\pgfqpoint{1.708440in}{3.401756in}}{\pgfqpoint{1.708440in}{3.390706in}}%
\pgfpathcurveto{\pgfqpoint{1.708440in}{3.379656in}}{\pgfqpoint{1.712830in}{3.369057in}}{\pgfqpoint{1.720643in}{3.361243in}}%
\pgfpathcurveto{\pgfqpoint{1.728457in}{3.353429in}}{\pgfqpoint{1.739056in}{3.349039in}}{\pgfqpoint{1.750106in}{3.349039in}}%
\pgfpathclose%
\pgfusepath{stroke,fill}%
\end{pgfscope}%
\begin{pgfscope}%
\pgfsetbuttcap%
\pgfsetroundjoin%
\definecolor{currentfill}{rgb}{0.000000,0.000000,0.000000}%
\pgfsetfillcolor{currentfill}%
\pgfsetlinewidth{0.803000pt}%
\definecolor{currentstroke}{rgb}{0.000000,0.000000,0.000000}%
\pgfsetstrokecolor{currentstroke}%
\pgfsetdash{}{0pt}%
\pgfsys@defobject{currentmarker}{\pgfqpoint{0.000000in}{-0.048611in}}{\pgfqpoint{0.000000in}{0.000000in}}{%
\pgfpathmoveto{\pgfqpoint{0.000000in}{0.000000in}}%
\pgfpathlineto{\pgfqpoint{0.000000in}{-0.048611in}}%
\pgfusepath{stroke,fill}%
}%
\begin{pgfscope}%
\pgfsys@transformshift{0.719376in}{0.548769in}%
\pgfsys@useobject{currentmarker}{}%
\end{pgfscope}%
\end{pgfscope}%
\begin{pgfscope}%
\definecolor{textcolor}{rgb}{0.000000,0.000000,0.000000}%
\pgfsetstrokecolor{textcolor}%
\pgfsetfillcolor{textcolor}%
\pgftext[x=0.719376in,y=0.451547in,,top]{\color{textcolor}\sffamily\fontsize{10.000000}{12.000000}\selectfont \(\displaystyle {0.0}\)}%
\end{pgfscope}%
\begin{pgfscope}%
\pgfsetbuttcap%
\pgfsetroundjoin%
\definecolor{currentfill}{rgb}{0.000000,0.000000,0.000000}%
\pgfsetfillcolor{currentfill}%
\pgfsetlinewidth{0.803000pt}%
\definecolor{currentstroke}{rgb}{0.000000,0.000000,0.000000}%
\pgfsetstrokecolor{currentstroke}%
\pgfsetdash{}{0pt}%
\pgfsys@defobject{currentmarker}{\pgfqpoint{0.000000in}{-0.048611in}}{\pgfqpoint{0.000000in}{0.000000in}}{%
\pgfpathmoveto{\pgfqpoint{0.000000in}{0.000000in}}%
\pgfpathlineto{\pgfqpoint{0.000000in}{-0.048611in}}%
\pgfusepath{stroke,fill}%
}%
\begin{pgfscope}%
\pgfsys@transformshift{1.223598in}{0.548769in}%
\pgfsys@useobject{currentmarker}{}%
\end{pgfscope}%
\end{pgfscope}%
\begin{pgfscope}%
\definecolor{textcolor}{rgb}{0.000000,0.000000,0.000000}%
\pgfsetstrokecolor{textcolor}%
\pgfsetfillcolor{textcolor}%
\pgftext[x=1.223598in,y=0.451547in,,top]{\color{textcolor}\sffamily\fontsize{10.000000}{12.000000}\selectfont \(\displaystyle {0.1}\)}%
\end{pgfscope}%
\begin{pgfscope}%
\pgfsetbuttcap%
\pgfsetroundjoin%
\definecolor{currentfill}{rgb}{0.000000,0.000000,0.000000}%
\pgfsetfillcolor{currentfill}%
\pgfsetlinewidth{0.803000pt}%
\definecolor{currentstroke}{rgb}{0.000000,0.000000,0.000000}%
\pgfsetstrokecolor{currentstroke}%
\pgfsetdash{}{0pt}%
\pgfsys@defobject{currentmarker}{\pgfqpoint{0.000000in}{-0.048611in}}{\pgfqpoint{0.000000in}{0.000000in}}{%
\pgfpathmoveto{\pgfqpoint{0.000000in}{0.000000in}}%
\pgfpathlineto{\pgfqpoint{0.000000in}{-0.048611in}}%
\pgfusepath{stroke,fill}%
}%
\begin{pgfscope}%
\pgfsys@transformshift{1.727820in}{0.548769in}%
\pgfsys@useobject{currentmarker}{}%
\end{pgfscope}%
\end{pgfscope}%
\begin{pgfscope}%
\definecolor{textcolor}{rgb}{0.000000,0.000000,0.000000}%
\pgfsetstrokecolor{textcolor}%
\pgfsetfillcolor{textcolor}%
\pgftext[x=1.727820in,y=0.451547in,,top]{\color{textcolor}\sffamily\fontsize{10.000000}{12.000000}\selectfont \(\displaystyle {0.2}\)}%
\end{pgfscope}%
\begin{pgfscope}%
\pgfsetbuttcap%
\pgfsetroundjoin%
\definecolor{currentfill}{rgb}{0.000000,0.000000,0.000000}%
\pgfsetfillcolor{currentfill}%
\pgfsetlinewidth{0.803000pt}%
\definecolor{currentstroke}{rgb}{0.000000,0.000000,0.000000}%
\pgfsetstrokecolor{currentstroke}%
\pgfsetdash{}{0pt}%
\pgfsys@defobject{currentmarker}{\pgfqpoint{0.000000in}{-0.048611in}}{\pgfqpoint{0.000000in}{0.000000in}}{%
\pgfpathmoveto{\pgfqpoint{0.000000in}{0.000000in}}%
\pgfpathlineto{\pgfqpoint{0.000000in}{-0.048611in}}%
\pgfusepath{stroke,fill}%
}%
\begin{pgfscope}%
\pgfsys@transformshift{2.232042in}{0.548769in}%
\pgfsys@useobject{currentmarker}{}%
\end{pgfscope}%
\end{pgfscope}%
\begin{pgfscope}%
\definecolor{textcolor}{rgb}{0.000000,0.000000,0.000000}%
\pgfsetstrokecolor{textcolor}%
\pgfsetfillcolor{textcolor}%
\pgftext[x=2.232042in,y=0.451547in,,top]{\color{textcolor}\sffamily\fontsize{10.000000}{12.000000}\selectfont \(\displaystyle {0.3}\)}%
\end{pgfscope}%
\begin{pgfscope}%
\pgfsetbuttcap%
\pgfsetroundjoin%
\definecolor{currentfill}{rgb}{0.000000,0.000000,0.000000}%
\pgfsetfillcolor{currentfill}%
\pgfsetlinewidth{0.803000pt}%
\definecolor{currentstroke}{rgb}{0.000000,0.000000,0.000000}%
\pgfsetstrokecolor{currentstroke}%
\pgfsetdash{}{0pt}%
\pgfsys@defobject{currentmarker}{\pgfqpoint{0.000000in}{-0.048611in}}{\pgfqpoint{0.000000in}{0.000000in}}{%
\pgfpathmoveto{\pgfqpoint{0.000000in}{0.000000in}}%
\pgfpathlineto{\pgfqpoint{0.000000in}{-0.048611in}}%
\pgfusepath{stroke,fill}%
}%
\begin{pgfscope}%
\pgfsys@transformshift{2.736264in}{0.548769in}%
\pgfsys@useobject{currentmarker}{}%
\end{pgfscope}%
\end{pgfscope}%
\begin{pgfscope}%
\definecolor{textcolor}{rgb}{0.000000,0.000000,0.000000}%
\pgfsetstrokecolor{textcolor}%
\pgfsetfillcolor{textcolor}%
\pgftext[x=2.736264in,y=0.451547in,,top]{\color{textcolor}\sffamily\fontsize{10.000000}{12.000000}\selectfont \(\displaystyle {0.4}\)}%
\end{pgfscope}%
\begin{pgfscope}%
\pgfsetbuttcap%
\pgfsetroundjoin%
\definecolor{currentfill}{rgb}{0.000000,0.000000,0.000000}%
\pgfsetfillcolor{currentfill}%
\pgfsetlinewidth{0.803000pt}%
\definecolor{currentstroke}{rgb}{0.000000,0.000000,0.000000}%
\pgfsetstrokecolor{currentstroke}%
\pgfsetdash{}{0pt}%
\pgfsys@defobject{currentmarker}{\pgfqpoint{0.000000in}{-0.048611in}}{\pgfqpoint{0.000000in}{0.000000in}}{%
\pgfpathmoveto{\pgfqpoint{0.000000in}{0.000000in}}%
\pgfpathlineto{\pgfqpoint{0.000000in}{-0.048611in}}%
\pgfusepath{stroke,fill}%
}%
\begin{pgfscope}%
\pgfsys@transformshift{3.240486in}{0.548769in}%
\pgfsys@useobject{currentmarker}{}%
\end{pgfscope}%
\end{pgfscope}%
\begin{pgfscope}%
\definecolor{textcolor}{rgb}{0.000000,0.000000,0.000000}%
\pgfsetstrokecolor{textcolor}%
\pgfsetfillcolor{textcolor}%
\pgftext[x=3.240486in,y=0.451547in,,top]{\color{textcolor}\sffamily\fontsize{10.000000}{12.000000}\selectfont \(\displaystyle {0.5}\)}%
\end{pgfscope}%
\begin{pgfscope}%
\pgfsetbuttcap%
\pgfsetroundjoin%
\definecolor{currentfill}{rgb}{0.000000,0.000000,0.000000}%
\pgfsetfillcolor{currentfill}%
\pgfsetlinewidth{0.803000pt}%
\definecolor{currentstroke}{rgb}{0.000000,0.000000,0.000000}%
\pgfsetstrokecolor{currentstroke}%
\pgfsetdash{}{0pt}%
\pgfsys@defobject{currentmarker}{\pgfqpoint{0.000000in}{-0.048611in}}{\pgfqpoint{0.000000in}{0.000000in}}{%
\pgfpathmoveto{\pgfqpoint{0.000000in}{0.000000in}}%
\pgfpathlineto{\pgfqpoint{0.000000in}{-0.048611in}}%
\pgfusepath{stroke,fill}%
}%
\begin{pgfscope}%
\pgfsys@transformshift{3.744708in}{0.548769in}%
\pgfsys@useobject{currentmarker}{}%
\end{pgfscope}%
\end{pgfscope}%
\begin{pgfscope}%
\definecolor{textcolor}{rgb}{0.000000,0.000000,0.000000}%
\pgfsetstrokecolor{textcolor}%
\pgfsetfillcolor{textcolor}%
\pgftext[x=3.744708in,y=0.451547in,,top]{\color{textcolor}\sffamily\fontsize{10.000000}{12.000000}\selectfont \(\displaystyle {0.6}\)}%
\end{pgfscope}%
\begin{pgfscope}%
\pgfsetbuttcap%
\pgfsetroundjoin%
\definecolor{currentfill}{rgb}{0.000000,0.000000,0.000000}%
\pgfsetfillcolor{currentfill}%
\pgfsetlinewidth{0.803000pt}%
\definecolor{currentstroke}{rgb}{0.000000,0.000000,0.000000}%
\pgfsetstrokecolor{currentstroke}%
\pgfsetdash{}{0pt}%
\pgfsys@defobject{currentmarker}{\pgfqpoint{0.000000in}{-0.048611in}}{\pgfqpoint{0.000000in}{0.000000in}}{%
\pgfpathmoveto{\pgfqpoint{0.000000in}{0.000000in}}%
\pgfpathlineto{\pgfqpoint{0.000000in}{-0.048611in}}%
\pgfusepath{stroke,fill}%
}%
\begin{pgfscope}%
\pgfsys@transformshift{4.248930in}{0.548769in}%
\pgfsys@useobject{currentmarker}{}%
\end{pgfscope}%
\end{pgfscope}%
\begin{pgfscope}%
\definecolor{textcolor}{rgb}{0.000000,0.000000,0.000000}%
\pgfsetstrokecolor{textcolor}%
\pgfsetfillcolor{textcolor}%
\pgftext[x=4.248930in,y=0.451547in,,top]{\color{textcolor}\sffamily\fontsize{10.000000}{12.000000}\selectfont \(\displaystyle {0.7}\)}%
\end{pgfscope}%
\begin{pgfscope}%
\pgfsetbuttcap%
\pgfsetroundjoin%
\definecolor{currentfill}{rgb}{0.000000,0.000000,0.000000}%
\pgfsetfillcolor{currentfill}%
\pgfsetlinewidth{0.803000pt}%
\definecolor{currentstroke}{rgb}{0.000000,0.000000,0.000000}%
\pgfsetstrokecolor{currentstroke}%
\pgfsetdash{}{0pt}%
\pgfsys@defobject{currentmarker}{\pgfqpoint{0.000000in}{-0.048611in}}{\pgfqpoint{0.000000in}{0.000000in}}{%
\pgfpathmoveto{\pgfqpoint{0.000000in}{0.000000in}}%
\pgfpathlineto{\pgfqpoint{0.000000in}{-0.048611in}}%
\pgfusepath{stroke,fill}%
}%
\begin{pgfscope}%
\pgfsys@transformshift{4.753152in}{0.548769in}%
\pgfsys@useobject{currentmarker}{}%
\end{pgfscope}%
\end{pgfscope}%
\begin{pgfscope}%
\definecolor{textcolor}{rgb}{0.000000,0.000000,0.000000}%
\pgfsetstrokecolor{textcolor}%
\pgfsetfillcolor{textcolor}%
\pgftext[x=4.753152in,y=0.451547in,,top]{\color{textcolor}\sffamily\fontsize{10.000000}{12.000000}\selectfont \(\displaystyle {0.8}\)}%
\end{pgfscope}%
\begin{pgfscope}%
\pgfsetbuttcap%
\pgfsetroundjoin%
\definecolor{currentfill}{rgb}{0.000000,0.000000,0.000000}%
\pgfsetfillcolor{currentfill}%
\pgfsetlinewidth{0.803000pt}%
\definecolor{currentstroke}{rgb}{0.000000,0.000000,0.000000}%
\pgfsetstrokecolor{currentstroke}%
\pgfsetdash{}{0pt}%
\pgfsys@defobject{currentmarker}{\pgfqpoint{0.000000in}{-0.048611in}}{\pgfqpoint{0.000000in}{0.000000in}}{%
\pgfpathmoveto{\pgfqpoint{0.000000in}{0.000000in}}%
\pgfpathlineto{\pgfqpoint{0.000000in}{-0.048611in}}%
\pgfusepath{stroke,fill}%
}%
\begin{pgfscope}%
\pgfsys@transformshift{5.257375in}{0.548769in}%
\pgfsys@useobject{currentmarker}{}%
\end{pgfscope}%
\end{pgfscope}%
\begin{pgfscope}%
\definecolor{textcolor}{rgb}{0.000000,0.000000,0.000000}%
\pgfsetstrokecolor{textcolor}%
\pgfsetfillcolor{textcolor}%
\pgftext[x=5.257375in,y=0.451547in,,top]{\color{textcolor}\sffamily\fontsize{10.000000}{12.000000}\selectfont \(\displaystyle {0.9}\)}%
\end{pgfscope}%
\begin{pgfscope}%
\pgfsetbuttcap%
\pgfsetroundjoin%
\definecolor{currentfill}{rgb}{0.000000,0.000000,0.000000}%
\pgfsetfillcolor{currentfill}%
\pgfsetlinewidth{0.803000pt}%
\definecolor{currentstroke}{rgb}{0.000000,0.000000,0.000000}%
\pgfsetstrokecolor{currentstroke}%
\pgfsetdash{}{0pt}%
\pgfsys@defobject{currentmarker}{\pgfqpoint{0.000000in}{-0.048611in}}{\pgfqpoint{0.000000in}{0.000000in}}{%
\pgfpathmoveto{\pgfqpoint{0.000000in}{0.000000in}}%
\pgfpathlineto{\pgfqpoint{0.000000in}{-0.048611in}}%
\pgfusepath{stroke,fill}%
}%
\begin{pgfscope}%
\pgfsys@transformshift{5.761597in}{0.548769in}%
\pgfsys@useobject{currentmarker}{}%
\end{pgfscope}%
\end{pgfscope}%
\begin{pgfscope}%
\definecolor{textcolor}{rgb}{0.000000,0.000000,0.000000}%
\pgfsetstrokecolor{textcolor}%
\pgfsetfillcolor{textcolor}%
\pgftext[x=5.761597in,y=0.451547in,,top]{\color{textcolor}\sffamily\fontsize{10.000000}{12.000000}\selectfont \(\displaystyle {1.0}\)}%
\end{pgfscope}%
\begin{pgfscope}%
\definecolor{textcolor}{rgb}{0.000000,0.000000,0.000000}%
\pgfsetstrokecolor{textcolor}%
\pgfsetfillcolor{textcolor}%
\pgftext[x=3.205150in,y=0.272658in,,top]{\color{textcolor}\sffamily\fontsize{10.000000}{12.000000}\selectfont Alias Edge Count}%
\end{pgfscope}%
\begin{pgfscope}%
\definecolor{textcolor}{rgb}{0.000000,0.000000,0.000000}%
\pgfsetstrokecolor{textcolor}%
\pgfsetfillcolor{textcolor}%
\pgftext[x=5.761597in,y=0.286547in,right,top]{\color{textcolor}\sffamily\fontsize{10.000000}{12.000000}\selectfont \(\displaystyle \times{10^{8}}{}\)}%
\end{pgfscope}%
\begin{pgfscope}%
\pgfsetbuttcap%
\pgfsetroundjoin%
\definecolor{currentfill}{rgb}{0.000000,0.000000,0.000000}%
\pgfsetfillcolor{currentfill}%
\pgfsetlinewidth{0.803000pt}%
\definecolor{currentstroke}{rgb}{0.000000,0.000000,0.000000}%
\pgfsetstrokecolor{currentstroke}%
\pgfsetdash{}{0pt}%
\pgfsys@defobject{currentmarker}{\pgfqpoint{-0.048611in}{0.000000in}}{\pgfqpoint{0.000000in}{0.000000in}}{%
\pgfpathmoveto{\pgfqpoint{0.000000in}{0.000000in}}%
\pgfpathlineto{\pgfqpoint{-0.048611in}{0.000000in}}%
\pgfusepath{stroke,fill}%
}%
\begin{pgfscope}%
\pgfsys@transformshift{0.648703in}{0.705309in}%
\pgfsys@useobject{currentmarker}{}%
\end{pgfscope}%
\end{pgfscope}%
\begin{pgfscope}%
\definecolor{textcolor}{rgb}{0.000000,0.000000,0.000000}%
\pgfsetstrokecolor{textcolor}%
\pgfsetfillcolor{textcolor}%
\pgftext[x=0.482036in, y=0.657114in, left, base]{\color{textcolor}\sffamily\fontsize{10.000000}{12.000000}\selectfont \(\displaystyle {0}\)}%
\end{pgfscope}%
\begin{pgfscope}%
\pgfsetbuttcap%
\pgfsetroundjoin%
\definecolor{currentfill}{rgb}{0.000000,0.000000,0.000000}%
\pgfsetfillcolor{currentfill}%
\pgfsetlinewidth{0.803000pt}%
\definecolor{currentstroke}{rgb}{0.000000,0.000000,0.000000}%
\pgfsetstrokecolor{currentstroke}%
\pgfsetdash{}{0pt}%
\pgfsys@defobject{currentmarker}{\pgfqpoint{-0.048611in}{0.000000in}}{\pgfqpoint{0.000000in}{0.000000in}}{%
\pgfpathmoveto{\pgfqpoint{0.000000in}{0.000000in}}%
\pgfpathlineto{\pgfqpoint{-0.048611in}{0.000000in}}%
\pgfusepath{stroke,fill}%
}%
\begin{pgfscope}%
\pgfsys@transformshift{0.648703in}{1.117812in}%
\pgfsys@useobject{currentmarker}{}%
\end{pgfscope}%
\end{pgfscope}%
\begin{pgfscope}%
\definecolor{textcolor}{rgb}{0.000000,0.000000,0.000000}%
\pgfsetstrokecolor{textcolor}%
\pgfsetfillcolor{textcolor}%
\pgftext[x=0.343147in, y=1.069618in, left, base]{\color{textcolor}\sffamily\fontsize{10.000000}{12.000000}\selectfont \(\displaystyle {100}\)}%
\end{pgfscope}%
\begin{pgfscope}%
\pgfsetbuttcap%
\pgfsetroundjoin%
\definecolor{currentfill}{rgb}{0.000000,0.000000,0.000000}%
\pgfsetfillcolor{currentfill}%
\pgfsetlinewidth{0.803000pt}%
\definecolor{currentstroke}{rgb}{0.000000,0.000000,0.000000}%
\pgfsetstrokecolor{currentstroke}%
\pgfsetdash{}{0pt}%
\pgfsys@defobject{currentmarker}{\pgfqpoint{-0.048611in}{0.000000in}}{\pgfqpoint{0.000000in}{0.000000in}}{%
\pgfpathmoveto{\pgfqpoint{0.000000in}{0.000000in}}%
\pgfpathlineto{\pgfqpoint{-0.048611in}{0.000000in}}%
\pgfusepath{stroke,fill}%
}%
\begin{pgfscope}%
\pgfsys@transformshift{0.648703in}{1.530315in}%
\pgfsys@useobject{currentmarker}{}%
\end{pgfscope}%
\end{pgfscope}%
\begin{pgfscope}%
\definecolor{textcolor}{rgb}{0.000000,0.000000,0.000000}%
\pgfsetstrokecolor{textcolor}%
\pgfsetfillcolor{textcolor}%
\pgftext[x=0.343147in, y=1.482121in, left, base]{\color{textcolor}\sffamily\fontsize{10.000000}{12.000000}\selectfont \(\displaystyle {200}\)}%
\end{pgfscope}%
\begin{pgfscope}%
\pgfsetbuttcap%
\pgfsetroundjoin%
\definecolor{currentfill}{rgb}{0.000000,0.000000,0.000000}%
\pgfsetfillcolor{currentfill}%
\pgfsetlinewidth{0.803000pt}%
\definecolor{currentstroke}{rgb}{0.000000,0.000000,0.000000}%
\pgfsetstrokecolor{currentstroke}%
\pgfsetdash{}{0pt}%
\pgfsys@defobject{currentmarker}{\pgfqpoint{-0.048611in}{0.000000in}}{\pgfqpoint{0.000000in}{0.000000in}}{%
\pgfpathmoveto{\pgfqpoint{0.000000in}{0.000000in}}%
\pgfpathlineto{\pgfqpoint{-0.048611in}{0.000000in}}%
\pgfusepath{stroke,fill}%
}%
\begin{pgfscope}%
\pgfsys@transformshift{0.648703in}{1.942819in}%
\pgfsys@useobject{currentmarker}{}%
\end{pgfscope}%
\end{pgfscope}%
\begin{pgfscope}%
\definecolor{textcolor}{rgb}{0.000000,0.000000,0.000000}%
\pgfsetstrokecolor{textcolor}%
\pgfsetfillcolor{textcolor}%
\pgftext[x=0.343147in, y=1.894624in, left, base]{\color{textcolor}\sffamily\fontsize{10.000000}{12.000000}\selectfont \(\displaystyle {300}\)}%
\end{pgfscope}%
\begin{pgfscope}%
\pgfsetbuttcap%
\pgfsetroundjoin%
\definecolor{currentfill}{rgb}{0.000000,0.000000,0.000000}%
\pgfsetfillcolor{currentfill}%
\pgfsetlinewidth{0.803000pt}%
\definecolor{currentstroke}{rgb}{0.000000,0.000000,0.000000}%
\pgfsetstrokecolor{currentstroke}%
\pgfsetdash{}{0pt}%
\pgfsys@defobject{currentmarker}{\pgfqpoint{-0.048611in}{0.000000in}}{\pgfqpoint{0.000000in}{0.000000in}}{%
\pgfpathmoveto{\pgfqpoint{0.000000in}{0.000000in}}%
\pgfpathlineto{\pgfqpoint{-0.048611in}{0.000000in}}%
\pgfusepath{stroke,fill}%
}%
\begin{pgfscope}%
\pgfsys@transformshift{0.648703in}{2.355322in}%
\pgfsys@useobject{currentmarker}{}%
\end{pgfscope}%
\end{pgfscope}%
\begin{pgfscope}%
\definecolor{textcolor}{rgb}{0.000000,0.000000,0.000000}%
\pgfsetstrokecolor{textcolor}%
\pgfsetfillcolor{textcolor}%
\pgftext[x=0.343147in, y=2.307128in, left, base]{\color{textcolor}\sffamily\fontsize{10.000000}{12.000000}\selectfont \(\displaystyle {400}\)}%
\end{pgfscope}%
\begin{pgfscope}%
\pgfsetbuttcap%
\pgfsetroundjoin%
\definecolor{currentfill}{rgb}{0.000000,0.000000,0.000000}%
\pgfsetfillcolor{currentfill}%
\pgfsetlinewidth{0.803000pt}%
\definecolor{currentstroke}{rgb}{0.000000,0.000000,0.000000}%
\pgfsetstrokecolor{currentstroke}%
\pgfsetdash{}{0pt}%
\pgfsys@defobject{currentmarker}{\pgfqpoint{-0.048611in}{0.000000in}}{\pgfqpoint{0.000000in}{0.000000in}}{%
\pgfpathmoveto{\pgfqpoint{0.000000in}{0.000000in}}%
\pgfpathlineto{\pgfqpoint{-0.048611in}{0.000000in}}%
\pgfusepath{stroke,fill}%
}%
\begin{pgfscope}%
\pgfsys@transformshift{0.648703in}{2.767826in}%
\pgfsys@useobject{currentmarker}{}%
\end{pgfscope}%
\end{pgfscope}%
\begin{pgfscope}%
\definecolor{textcolor}{rgb}{0.000000,0.000000,0.000000}%
\pgfsetstrokecolor{textcolor}%
\pgfsetfillcolor{textcolor}%
\pgftext[x=0.343147in, y=2.719631in, left, base]{\color{textcolor}\sffamily\fontsize{10.000000}{12.000000}\selectfont \(\displaystyle {500}\)}%
\end{pgfscope}%
\begin{pgfscope}%
\pgfsetbuttcap%
\pgfsetroundjoin%
\definecolor{currentfill}{rgb}{0.000000,0.000000,0.000000}%
\pgfsetfillcolor{currentfill}%
\pgfsetlinewidth{0.803000pt}%
\definecolor{currentstroke}{rgb}{0.000000,0.000000,0.000000}%
\pgfsetstrokecolor{currentstroke}%
\pgfsetdash{}{0pt}%
\pgfsys@defobject{currentmarker}{\pgfqpoint{-0.048611in}{0.000000in}}{\pgfqpoint{0.000000in}{0.000000in}}{%
\pgfpathmoveto{\pgfqpoint{0.000000in}{0.000000in}}%
\pgfpathlineto{\pgfqpoint{-0.048611in}{0.000000in}}%
\pgfusepath{stroke,fill}%
}%
\begin{pgfscope}%
\pgfsys@transformshift{0.648703in}{3.180329in}%
\pgfsys@useobject{currentmarker}{}%
\end{pgfscope}%
\end{pgfscope}%
\begin{pgfscope}%
\definecolor{textcolor}{rgb}{0.000000,0.000000,0.000000}%
\pgfsetstrokecolor{textcolor}%
\pgfsetfillcolor{textcolor}%
\pgftext[x=0.343147in, y=3.132135in, left, base]{\color{textcolor}\sffamily\fontsize{10.000000}{12.000000}\selectfont \(\displaystyle {600}\)}%
\end{pgfscope}%
\begin{pgfscope}%
\pgfsetbuttcap%
\pgfsetroundjoin%
\definecolor{currentfill}{rgb}{0.000000,0.000000,0.000000}%
\pgfsetfillcolor{currentfill}%
\pgfsetlinewidth{0.803000pt}%
\definecolor{currentstroke}{rgb}{0.000000,0.000000,0.000000}%
\pgfsetstrokecolor{currentstroke}%
\pgfsetdash{}{0pt}%
\pgfsys@defobject{currentmarker}{\pgfqpoint{-0.048611in}{0.000000in}}{\pgfqpoint{0.000000in}{0.000000in}}{%
\pgfpathmoveto{\pgfqpoint{0.000000in}{0.000000in}}%
\pgfpathlineto{\pgfqpoint{-0.048611in}{0.000000in}}%
\pgfusepath{stroke,fill}%
}%
\begin{pgfscope}%
\pgfsys@transformshift{0.648703in}{3.592832in}%
\pgfsys@useobject{currentmarker}{}%
\end{pgfscope}%
\end{pgfscope}%
\begin{pgfscope}%
\definecolor{textcolor}{rgb}{0.000000,0.000000,0.000000}%
\pgfsetstrokecolor{textcolor}%
\pgfsetfillcolor{textcolor}%
\pgftext[x=0.343147in, y=3.544638in, left, base]{\color{textcolor}\sffamily\fontsize{10.000000}{12.000000}\selectfont \(\displaystyle {700}\)}%
\end{pgfscope}%
\begin{pgfscope}%
\definecolor{textcolor}{rgb}{0.000000,0.000000,0.000000}%
\pgfsetstrokecolor{textcolor}%
\pgfsetfillcolor{textcolor}%
\pgftext[x=0.287592in,y=2.100064in,,bottom,rotate=90.000000]{\color{textcolor}\sffamily\fontsize{10.000000}{12.000000}\selectfont Data Flow Time (s)}%
\end{pgfscope}%
\begin{pgfscope}%
\pgfpathrectangle{\pgfqpoint{0.648703in}{0.548769in}}{\pgfqpoint{5.112893in}{3.102590in}}%
\pgfusepath{clip}%
\pgfsetrectcap%
\pgfsetroundjoin%
\pgfsetlinewidth{1.505625pt}%
\definecolor{currentstroke}{rgb}{0.000000,0.500000,0.000000}%
\pgfsetstrokecolor{currentstroke}%
\pgfsetdash{}{0pt}%
\pgfpathmoveto{\pgfqpoint{0.719376in}{0.689796in}}%
\pgfpathlineto{\pgfqpoint{0.733653in}{0.905941in}}%
\pgfpathlineto{\pgfqpoint{0.747930in}{1.108984in}}%
\pgfpathlineto{\pgfqpoint{0.762207in}{1.299422in}}%
\pgfpathlineto{\pgfqpoint{0.776484in}{1.477746in}}%
\pgfpathlineto{\pgfqpoint{0.790761in}{1.644438in}}%
\pgfpathlineto{\pgfqpoint{0.805039in}{1.799968in}}%
\pgfpathlineto{\pgfqpoint{0.819316in}{1.944801in}}%
\pgfpathlineto{\pgfqpoint{0.833593in}{2.079391in}}%
\pgfpathlineto{\pgfqpoint{0.847870in}{2.204182in}}%
\pgfpathlineto{\pgfqpoint{0.862147in}{2.319612in}}%
\pgfpathlineto{\pgfqpoint{0.876425in}{2.426107in}}%
\pgfpathlineto{\pgfqpoint{0.890702in}{2.524085in}}%
\pgfpathlineto{\pgfqpoint{0.904979in}{2.613956in}}%
\pgfpathlineto{\pgfqpoint{0.919256in}{2.696120in}}%
\pgfpathlineto{\pgfqpoint{0.933533in}{2.770968in}}%
\pgfpathlineto{\pgfqpoint{0.947811in}{2.838883in}}%
\pgfpathlineto{\pgfqpoint{0.962088in}{2.900238in}}%
\pgfpathlineto{\pgfqpoint{0.976365in}{2.955397in}}%
\pgfpathlineto{\pgfqpoint{0.990642in}{3.004716in}}%
\pgfpathlineto{\pgfqpoint{1.004919in}{3.048541in}}%
\pgfpathlineto{\pgfqpoint{1.019197in}{3.087209in}}%
\pgfpathlineto{\pgfqpoint{1.033474in}{3.121049in}}%
\pgfpathlineto{\pgfqpoint{1.047751in}{3.150381in}}%
\pgfpathlineto{\pgfqpoint{1.062028in}{3.175514in}}%
\pgfpathlineto{\pgfqpoint{1.076305in}{3.196750in}}%
\pgfpathlineto{\pgfqpoint{1.090582in}{3.214383in}}%
\pgfpathlineto{\pgfqpoint{1.104860in}{3.228694in}}%
\pgfpathlineto{\pgfqpoint{1.119137in}{3.239959in}}%
\pgfpathlineto{\pgfqpoint{1.133414in}{3.248443in}}%
\pgfpathlineto{\pgfqpoint{1.147691in}{3.254403in}}%
\pgfpathlineto{\pgfqpoint{1.161968in}{3.258086in}}%
\pgfpathlineto{\pgfqpoint{1.176246in}{3.259731in}}%
\pgfpathlineto{\pgfqpoint{1.190523in}{3.259568in}}%
\pgfpathlineto{\pgfqpoint{1.204800in}{3.257816in}}%
\pgfpathlineto{\pgfqpoint{1.219077in}{3.254689in}}%
\pgfpathlineto{\pgfqpoint{1.233354in}{3.250387in}}%
\pgfpathlineto{\pgfqpoint{1.247632in}{3.245106in}}%
\pgfpathlineto{\pgfqpoint{1.261909in}{3.239029in}}%
\pgfpathlineto{\pgfqpoint{1.276186in}{3.232333in}}%
\pgfpathlineto{\pgfqpoint{1.290463in}{3.225184in}}%
\pgfpathlineto{\pgfqpoint{1.304740in}{3.217740in}}%
\pgfpathlineto{\pgfqpoint{1.319018in}{3.210149in}}%
\pgfpathlineto{\pgfqpoint{1.333295in}{3.202552in}}%
\pgfpathlineto{\pgfqpoint{1.347572in}{3.195079in}}%
\pgfpathlineto{\pgfqpoint{1.361849in}{3.187851in}}%
\pgfpathlineto{\pgfqpoint{1.376126in}{3.180982in}}%
\pgfpathlineto{\pgfqpoint{1.390403in}{3.174576in}}%
\pgfpathlineto{\pgfqpoint{1.404681in}{3.168728in}}%
\pgfpathlineto{\pgfqpoint{1.418958in}{3.163522in}}%
\pgfpathlineto{\pgfqpoint{1.433235in}{3.159036in}}%
\pgfpathlineto{\pgfqpoint{1.447512in}{3.155339in}}%
\pgfpathlineto{\pgfqpoint{1.461789in}{3.152488in}}%
\pgfpathlineto{\pgfqpoint{1.476067in}{3.150535in}}%
\pgfpathlineto{\pgfqpoint{1.490344in}{3.149519in}}%
\pgfpathlineto{\pgfqpoint{1.504621in}{3.149472in}}%
\pgfpathlineto{\pgfqpoint{1.518898in}{3.150419in}}%
\pgfpathlineto{\pgfqpoint{1.533175in}{3.152372in}}%
\pgfpathlineto{\pgfqpoint{1.547453in}{3.155337in}}%
\pgfpathlineto{\pgfqpoint{1.561730in}{3.159309in}}%
\pgfpathlineto{\pgfqpoint{1.576007in}{3.164276in}}%
\pgfpathlineto{\pgfqpoint{1.590284in}{3.170216in}}%
\pgfpathlineto{\pgfqpoint{1.604561in}{3.177098in}}%
\pgfpathlineto{\pgfqpoint{1.618839in}{3.184882in}}%
\pgfpathlineto{\pgfqpoint{1.633116in}{3.193519in}}%
\pgfpathlineto{\pgfqpoint{1.647393in}{3.202951in}}%
\pgfpathlineto{\pgfqpoint{1.661670in}{3.213112in}}%
\pgfpathlineto{\pgfqpoint{1.675947in}{3.223925in}}%
\pgfpathlineto{\pgfqpoint{1.690224in}{3.235306in}}%
\pgfpathlineto{\pgfqpoint{1.704502in}{3.247160in}}%
\pgfpathlineto{\pgfqpoint{1.718779in}{3.259386in}}%
\pgfpathlineto{\pgfqpoint{1.733056in}{3.271871in}}%
\pgfpathlineto{\pgfqpoint{1.747333in}{3.284494in}}%
\pgfpathlineto{\pgfqpoint{1.761610in}{3.297126in}}%
\pgfpathlineto{\pgfqpoint{1.775888in}{3.309629in}}%
\pgfpathlineto{\pgfqpoint{1.790165in}{3.321853in}}%
\pgfpathlineto{\pgfqpoint{1.804442in}{3.333643in}}%
\pgfpathlineto{\pgfqpoint{1.818719in}{3.344834in}}%
\pgfpathlineto{\pgfqpoint{1.832996in}{3.355249in}}%
\pgfpathlineto{\pgfqpoint{1.847274in}{3.364706in}}%
\pgfpathlineto{\pgfqpoint{1.861551in}{3.373012in}}%
\pgfpathlineto{\pgfqpoint{1.875828in}{3.379966in}}%
\pgfpathlineto{\pgfqpoint{1.890105in}{3.385356in}}%
\pgfpathlineto{\pgfqpoint{1.904382in}{3.388964in}}%
\pgfpathlineto{\pgfqpoint{1.918660in}{3.390560in}}%
\pgfpathlineto{\pgfqpoint{1.932937in}{3.389908in}}%
\pgfpathlineto{\pgfqpoint{1.947214in}{3.386760in}}%
\pgfpathlineto{\pgfqpoint{1.961491in}{3.380861in}}%
\pgfpathlineto{\pgfqpoint{1.975768in}{3.371947in}}%
\pgfpathlineto{\pgfqpoint{1.990045in}{3.359743in}}%
\pgfpathlineto{\pgfqpoint{2.004323in}{3.343969in}}%
\pgfpathlineto{\pgfqpoint{2.018600in}{3.324331in}}%
\pgfpathlineto{\pgfqpoint{2.032877in}{3.300531in}}%
\pgfpathlineto{\pgfqpoint{2.047154in}{3.272258in}}%
\pgfpathlineto{\pgfqpoint{2.061431in}{3.239193in}}%
\pgfpathlineto{\pgfqpoint{2.075709in}{3.201010in}}%
\pgfpathlineto{\pgfqpoint{2.089986in}{3.157373in}}%
\pgfpathlineto{\pgfqpoint{2.104263in}{3.107935in}}%
\pgfpathlineto{\pgfqpoint{2.118540in}{3.052343in}}%
\pgfpathlineto{\pgfqpoint{2.132817in}{2.990233in}}%
\pgfusepath{stroke}%
\end{pgfscope}%
\begin{pgfscope}%
\pgfsetrectcap%
\pgfsetmiterjoin%
\pgfsetlinewidth{0.803000pt}%
\definecolor{currentstroke}{rgb}{0.000000,0.000000,0.000000}%
\pgfsetstrokecolor{currentstroke}%
\pgfsetdash{}{0pt}%
\pgfpathmoveto{\pgfqpoint{0.648703in}{0.548769in}}%
\pgfpathlineto{\pgfqpoint{0.648703in}{3.651359in}}%
\pgfusepath{stroke}%
\end{pgfscope}%
\begin{pgfscope}%
\pgfsetrectcap%
\pgfsetmiterjoin%
\pgfsetlinewidth{0.803000pt}%
\definecolor{currentstroke}{rgb}{0.000000,0.000000,0.000000}%
\pgfsetstrokecolor{currentstroke}%
\pgfsetdash{}{0pt}%
\pgfpathmoveto{\pgfqpoint{5.761597in}{0.548769in}}%
\pgfpathlineto{\pgfqpoint{5.761597in}{3.651359in}}%
\pgfusepath{stroke}%
\end{pgfscope}%
\begin{pgfscope}%
\pgfsetrectcap%
\pgfsetmiterjoin%
\pgfsetlinewidth{0.803000pt}%
\definecolor{currentstroke}{rgb}{0.000000,0.000000,0.000000}%
\pgfsetstrokecolor{currentstroke}%
\pgfsetdash{}{0pt}%
\pgfpathmoveto{\pgfqpoint{0.648703in}{0.548769in}}%
\pgfpathlineto{\pgfqpoint{5.761597in}{0.548769in}}%
\pgfusepath{stroke}%
\end{pgfscope}%
\begin{pgfscope}%
\pgfsetrectcap%
\pgfsetmiterjoin%
\pgfsetlinewidth{0.803000pt}%
\definecolor{currentstroke}{rgb}{0.000000,0.000000,0.000000}%
\pgfsetstrokecolor{currentstroke}%
\pgfsetdash{}{0pt}%
\pgfpathmoveto{\pgfqpoint{0.648703in}{3.651359in}}%
\pgfpathlineto{\pgfqpoint{5.761597in}{3.651359in}}%
\pgfusepath{stroke}%
\end{pgfscope}%
\begin{pgfscope}%
\definecolor{textcolor}{rgb}{0.000000,0.000000,0.000000}%
\pgfsetstrokecolor{textcolor}%
\pgfsetfillcolor{textcolor}%
\pgftext[x=3.205150in,y=3.734692in,,base]{\color{textcolor}\sffamily\fontsize{12.000000}{14.400000}\selectfont Forward}%
\end{pgfscope}%
\begin{pgfscope}%
\pgfsetbuttcap%
\pgfsetmiterjoin%
\definecolor{currentfill}{rgb}{1.000000,1.000000,1.000000}%
\pgfsetfillcolor{currentfill}%
\pgfsetfillopacity{0.800000}%
\pgfsetlinewidth{1.003750pt}%
\definecolor{currentstroke}{rgb}{0.800000,0.800000,0.800000}%
\pgfsetstrokecolor{currentstroke}%
\pgfsetstrokeopacity{0.800000}%
\pgfsetdash{}{0pt}%
\pgfpathmoveto{\pgfqpoint{4.212013in}{2.762053in}}%
\pgfpathlineto{\pgfqpoint{5.664374in}{2.762053in}}%
\pgfpathquadraticcurveto{\pgfqpoint{5.692152in}{2.762053in}}{\pgfqpoint{5.692152in}{2.789831in}}%
\pgfpathlineto{\pgfqpoint{5.692152in}{3.554136in}}%
\pgfpathquadraticcurveto{\pgfqpoint{5.692152in}{3.581914in}}{\pgfqpoint{5.664374in}{3.581914in}}%
\pgfpathlineto{\pgfqpoint{4.212013in}{3.581914in}}%
\pgfpathquadraticcurveto{\pgfqpoint{4.184236in}{3.581914in}}{\pgfqpoint{4.184236in}{3.554136in}}%
\pgfpathlineto{\pgfqpoint{4.184236in}{2.789831in}}%
\pgfpathquadraticcurveto{\pgfqpoint{4.184236in}{2.762053in}}{\pgfqpoint{4.212013in}{2.762053in}}%
\pgfpathclose%
\pgfusepath{stroke,fill}%
\end{pgfscope}%
\begin{pgfscope}%
\pgfsetbuttcap%
\pgfsetroundjoin%
\definecolor{currentfill}{rgb}{0.121569,0.466667,0.705882}%
\pgfsetfillcolor{currentfill}%
\pgfsetlinewidth{1.003750pt}%
\definecolor{currentstroke}{rgb}{0.121569,0.466667,0.705882}%
\pgfsetstrokecolor{currentstroke}%
\pgfsetdash{}{0pt}%
\pgfsys@defobject{currentmarker}{\pgfqpoint{-0.034722in}{-0.034722in}}{\pgfqpoint{0.034722in}{0.034722in}}{%
\pgfpathmoveto{\pgfqpoint{0.000000in}{-0.034722in}}%
\pgfpathcurveto{\pgfqpoint{0.009208in}{-0.034722in}}{\pgfqpoint{0.018041in}{-0.031064in}}{\pgfqpoint{0.024552in}{-0.024552in}}%
\pgfpathcurveto{\pgfqpoint{0.031064in}{-0.018041in}}{\pgfqpoint{0.034722in}{-0.009208in}}{\pgfqpoint{0.034722in}{0.000000in}}%
\pgfpathcurveto{\pgfqpoint{0.034722in}{0.009208in}}{\pgfqpoint{0.031064in}{0.018041in}}{\pgfqpoint{0.024552in}{0.024552in}}%
\pgfpathcurveto{\pgfqpoint{0.018041in}{0.031064in}}{\pgfqpoint{0.009208in}{0.034722in}}{\pgfqpoint{0.000000in}{0.034722in}}%
\pgfpathcurveto{\pgfqpoint{-0.009208in}{0.034722in}}{\pgfqpoint{-0.018041in}{0.031064in}}{\pgfqpoint{-0.024552in}{0.024552in}}%
\pgfpathcurveto{\pgfqpoint{-0.031064in}{0.018041in}}{\pgfqpoint{-0.034722in}{0.009208in}}{\pgfqpoint{-0.034722in}{0.000000in}}%
\pgfpathcurveto{\pgfqpoint{-0.034722in}{-0.009208in}}{\pgfqpoint{-0.031064in}{-0.018041in}}{\pgfqpoint{-0.024552in}{-0.024552in}}%
\pgfpathcurveto{\pgfqpoint{-0.018041in}{-0.031064in}}{\pgfqpoint{-0.009208in}{-0.034722in}}{\pgfqpoint{0.000000in}{-0.034722in}}%
\pgfpathclose%
\pgfusepath{stroke,fill}%
}%
\begin{pgfscope}%
\pgfsys@transformshift{4.378680in}{3.477748in}%
\pgfsys@useobject{currentmarker}{}%
\end{pgfscope}%
\end{pgfscope}%
\begin{pgfscope}%
\definecolor{textcolor}{rgb}{0.000000,0.000000,0.000000}%
\pgfsetstrokecolor{textcolor}%
\pgfsetfillcolor{textcolor}%
\pgftext[x=4.628680in,y=3.429136in,left,base]{\color{textcolor}\sffamily\fontsize{10.000000}{12.000000}\selectfont No Timeout}%
\end{pgfscope}%
\begin{pgfscope}%
\pgfsetbuttcap%
\pgfsetroundjoin%
\definecolor{currentfill}{rgb}{1.000000,0.498039,0.054902}%
\pgfsetfillcolor{currentfill}%
\pgfsetlinewidth{1.003750pt}%
\definecolor{currentstroke}{rgb}{1.000000,0.498039,0.054902}%
\pgfsetstrokecolor{currentstroke}%
\pgfsetdash{}{0pt}%
\pgfsys@defobject{currentmarker}{\pgfqpoint{-0.034722in}{-0.034722in}}{\pgfqpoint{0.034722in}{0.034722in}}{%
\pgfpathmoveto{\pgfqpoint{0.000000in}{-0.034722in}}%
\pgfpathcurveto{\pgfqpoint{0.009208in}{-0.034722in}}{\pgfqpoint{0.018041in}{-0.031064in}}{\pgfqpoint{0.024552in}{-0.024552in}}%
\pgfpathcurveto{\pgfqpoint{0.031064in}{-0.018041in}}{\pgfqpoint{0.034722in}{-0.009208in}}{\pgfqpoint{0.034722in}{0.000000in}}%
\pgfpathcurveto{\pgfqpoint{0.034722in}{0.009208in}}{\pgfqpoint{0.031064in}{0.018041in}}{\pgfqpoint{0.024552in}{0.024552in}}%
\pgfpathcurveto{\pgfqpoint{0.018041in}{0.031064in}}{\pgfqpoint{0.009208in}{0.034722in}}{\pgfqpoint{0.000000in}{0.034722in}}%
\pgfpathcurveto{\pgfqpoint{-0.009208in}{0.034722in}}{\pgfqpoint{-0.018041in}{0.031064in}}{\pgfqpoint{-0.024552in}{0.024552in}}%
\pgfpathcurveto{\pgfqpoint{-0.031064in}{0.018041in}}{\pgfqpoint{-0.034722in}{0.009208in}}{\pgfqpoint{-0.034722in}{0.000000in}}%
\pgfpathcurveto{\pgfqpoint{-0.034722in}{-0.009208in}}{\pgfqpoint{-0.031064in}{-0.018041in}}{\pgfqpoint{-0.024552in}{-0.024552in}}%
\pgfpathcurveto{\pgfqpoint{-0.018041in}{-0.031064in}}{\pgfqpoint{-0.009208in}{-0.034722in}}{\pgfqpoint{0.000000in}{-0.034722in}}%
\pgfpathclose%
\pgfusepath{stroke,fill}%
}%
\begin{pgfscope}%
\pgfsys@transformshift{4.378680in}{3.284136in}%
\pgfsys@useobject{currentmarker}{}%
\end{pgfscope}%
\end{pgfscope}%
\begin{pgfscope}%
\definecolor{textcolor}{rgb}{0.000000,0.000000,0.000000}%
\pgfsetstrokecolor{textcolor}%
\pgfsetfillcolor{textcolor}%
\pgftext[x=4.628680in,y=3.235525in,left,base]{\color{textcolor}\sffamily\fontsize{10.000000}{12.000000}\selectfont Time Timeout}%
\end{pgfscope}%
\begin{pgfscope}%
\pgfsetbuttcap%
\pgfsetroundjoin%
\definecolor{currentfill}{rgb}{0.839216,0.152941,0.156863}%
\pgfsetfillcolor{currentfill}%
\pgfsetlinewidth{1.003750pt}%
\definecolor{currentstroke}{rgb}{0.839216,0.152941,0.156863}%
\pgfsetstrokecolor{currentstroke}%
\pgfsetdash{}{0pt}%
\pgfsys@defobject{currentmarker}{\pgfqpoint{-0.034722in}{-0.034722in}}{\pgfqpoint{0.034722in}{0.034722in}}{%
\pgfpathmoveto{\pgfqpoint{0.000000in}{-0.034722in}}%
\pgfpathcurveto{\pgfqpoint{0.009208in}{-0.034722in}}{\pgfqpoint{0.018041in}{-0.031064in}}{\pgfqpoint{0.024552in}{-0.024552in}}%
\pgfpathcurveto{\pgfqpoint{0.031064in}{-0.018041in}}{\pgfqpoint{0.034722in}{-0.009208in}}{\pgfqpoint{0.034722in}{0.000000in}}%
\pgfpathcurveto{\pgfqpoint{0.034722in}{0.009208in}}{\pgfqpoint{0.031064in}{0.018041in}}{\pgfqpoint{0.024552in}{0.024552in}}%
\pgfpathcurveto{\pgfqpoint{0.018041in}{0.031064in}}{\pgfqpoint{0.009208in}{0.034722in}}{\pgfqpoint{0.000000in}{0.034722in}}%
\pgfpathcurveto{\pgfqpoint{-0.009208in}{0.034722in}}{\pgfqpoint{-0.018041in}{0.031064in}}{\pgfqpoint{-0.024552in}{0.024552in}}%
\pgfpathcurveto{\pgfqpoint{-0.031064in}{0.018041in}}{\pgfqpoint{-0.034722in}{0.009208in}}{\pgfqpoint{-0.034722in}{0.000000in}}%
\pgfpathcurveto{\pgfqpoint{-0.034722in}{-0.009208in}}{\pgfqpoint{-0.031064in}{-0.018041in}}{\pgfqpoint{-0.024552in}{-0.024552in}}%
\pgfpathcurveto{\pgfqpoint{-0.018041in}{-0.031064in}}{\pgfqpoint{-0.009208in}{-0.034722in}}{\pgfqpoint{0.000000in}{-0.034722in}}%
\pgfpathclose%
\pgfusepath{stroke,fill}%
}%
\begin{pgfscope}%
\pgfsys@transformshift{4.378680in}{3.090525in}%
\pgfsys@useobject{currentmarker}{}%
\end{pgfscope}%
\end{pgfscope}%
\begin{pgfscope}%
\definecolor{textcolor}{rgb}{0.000000,0.000000,0.000000}%
\pgfsetstrokecolor{textcolor}%
\pgfsetfillcolor{textcolor}%
\pgftext[x=4.628680in,y=3.041914in,left,base]{\color{textcolor}\sffamily\fontsize{10.000000}{12.000000}\selectfont Memory Timeout}%
\end{pgfscope}%
\begin{pgfscope}%
\pgfsetrectcap%
\pgfsetroundjoin%
\pgfsetlinewidth{1.505625pt}%
\definecolor{currentstroke}{rgb}{0.000000,0.500000,0.000000}%
\pgfsetstrokecolor{currentstroke}%
\pgfsetdash{}{0pt}%
\pgfpathmoveto{\pgfqpoint{4.239791in}{2.894692in}}%
\pgfpathlineto{\pgfqpoint{4.517569in}{2.894692in}}%
\pgfusepath{stroke}%
\end{pgfscope}%
\begin{pgfscope}%
\definecolor{textcolor}{rgb}{0.000000,0.000000,0.000000}%
\pgfsetstrokecolor{textcolor}%
\pgfsetfillcolor{textcolor}%
\pgftext[x=4.628680in,y=2.846081in,left,base]{\color{textcolor}\sffamily\fontsize{10.000000}{12.000000}\selectfont Polyfit}%
\end{pgfscope}%
\end{pgfpicture}%
\makeatother%
\endgroup%

                }
            \end{subfigure}
            \qquad
            \begin{subfigure}[]{0.45\textwidth}
                \centering
                \resizebox{\columnwidth}{!}{
                    %% Creator: Matplotlib, PGF backend
%%
%% To include the figure in your LaTeX document, write
%%   \input{<filename>.pgf}
%%
%% Make sure the required packages are loaded in your preamble
%%   \usepackage{pgf}
%%
%% and, on pdftex
%%   \usepackage[utf8]{inputenc}\DeclareUnicodeCharacter{2212}{-}
%%
%% or, on luatex and xetex
%%   \usepackage{unicode-math}
%%
%% Figures using additional raster images can only be included by \input if
%% they are in the same directory as the main LaTeX file. For loading figures
%% from other directories you can use the `import` package
%%   \usepackage{import}
%%
%% and then include the figures with
%%   \import{<path to file>}{<filename>.pgf}
%%
%% Matplotlib used the following preamble
%%   \usepackage{amsmath}
%%   \usepackage{fontspec}
%%
\begingroup%
\makeatletter%
\begin{pgfpicture}%
\pgfpathrectangle{\pgfpointorigin}{\pgfqpoint{6.000000in}{4.000000in}}%
\pgfusepath{use as bounding box, clip}%
\begin{pgfscope}%
\pgfsetbuttcap%
\pgfsetmiterjoin%
\definecolor{currentfill}{rgb}{1.000000,1.000000,1.000000}%
\pgfsetfillcolor{currentfill}%
\pgfsetlinewidth{0.000000pt}%
\definecolor{currentstroke}{rgb}{1.000000,1.000000,1.000000}%
\pgfsetstrokecolor{currentstroke}%
\pgfsetdash{}{0pt}%
\pgfpathmoveto{\pgfqpoint{0.000000in}{0.000000in}}%
\pgfpathlineto{\pgfqpoint{6.000000in}{0.000000in}}%
\pgfpathlineto{\pgfqpoint{6.000000in}{4.000000in}}%
\pgfpathlineto{\pgfqpoint{0.000000in}{4.000000in}}%
\pgfpathclose%
\pgfusepath{fill}%
\end{pgfscope}%
\begin{pgfscope}%
\pgfsetbuttcap%
\pgfsetmiterjoin%
\definecolor{currentfill}{rgb}{1.000000,1.000000,1.000000}%
\pgfsetfillcolor{currentfill}%
\pgfsetlinewidth{0.000000pt}%
\definecolor{currentstroke}{rgb}{0.000000,0.000000,0.000000}%
\pgfsetstrokecolor{currentstroke}%
\pgfsetstrokeopacity{0.000000}%
\pgfsetdash{}{0pt}%
\pgfpathmoveto{\pgfqpoint{0.648703in}{0.548769in}}%
\pgfpathlineto{\pgfqpoint{5.761597in}{0.548769in}}%
\pgfpathlineto{\pgfqpoint{5.761597in}{3.651359in}}%
\pgfpathlineto{\pgfqpoint{0.648703in}{3.651359in}}%
\pgfpathclose%
\pgfusepath{fill}%
\end{pgfscope}%
\begin{pgfscope}%
\pgfpathrectangle{\pgfqpoint{0.648703in}{0.548769in}}{\pgfqpoint{5.112893in}{3.102590in}}%
\pgfusepath{clip}%
\pgfsetbuttcap%
\pgfsetroundjoin%
\definecolor{currentfill}{rgb}{0.121569,0.466667,0.705882}%
\pgfsetfillcolor{currentfill}%
\pgfsetlinewidth{1.003750pt}%
\definecolor{currentstroke}{rgb}{0.121569,0.466667,0.705882}%
\pgfsetstrokecolor{currentstroke}%
\pgfsetdash{}{0pt}%
\pgfpathmoveto{\pgfqpoint{0.857108in}{0.683395in}}%
\pgfpathcurveto{\pgfqpoint{0.868158in}{0.683395in}}{\pgfqpoint{0.878757in}{0.687785in}}{\pgfqpoint{0.886571in}{0.695599in}}%
\pgfpathcurveto{\pgfqpoint{0.894385in}{0.703413in}}{\pgfqpoint{0.898775in}{0.714012in}}{\pgfqpoint{0.898775in}{0.725062in}}%
\pgfpathcurveto{\pgfqpoint{0.898775in}{0.736112in}}{\pgfqpoint{0.894385in}{0.746711in}}{\pgfqpoint{0.886571in}{0.754525in}}%
\pgfpathcurveto{\pgfqpoint{0.878757in}{0.762338in}}{\pgfqpoint{0.868158in}{0.766728in}}{\pgfqpoint{0.857108in}{0.766728in}}%
\pgfpathcurveto{\pgfqpoint{0.846058in}{0.766728in}}{\pgfqpoint{0.835459in}{0.762338in}}{\pgfqpoint{0.827645in}{0.754525in}}%
\pgfpathcurveto{\pgfqpoint{0.819832in}{0.746711in}}{\pgfqpoint{0.815442in}{0.736112in}}{\pgfqpoint{0.815442in}{0.725062in}}%
\pgfpathcurveto{\pgfqpoint{0.815442in}{0.714012in}}{\pgfqpoint{0.819832in}{0.703413in}}{\pgfqpoint{0.827645in}{0.695599in}}%
\pgfpathcurveto{\pgfqpoint{0.835459in}{0.687785in}}{\pgfqpoint{0.846058in}{0.683395in}}{\pgfqpoint{0.857108in}{0.683395in}}%
\pgfpathclose%
\pgfusepath{stroke,fill}%
\end{pgfscope}%
\begin{pgfscope}%
\pgfpathrectangle{\pgfqpoint{0.648703in}{0.548769in}}{\pgfqpoint{5.112893in}{3.102590in}}%
\pgfusepath{clip}%
\pgfsetbuttcap%
\pgfsetroundjoin%
\definecolor{currentfill}{rgb}{1.000000,0.498039,0.054902}%
\pgfsetfillcolor{currentfill}%
\pgfsetlinewidth{1.003750pt}%
\definecolor{currentstroke}{rgb}{1.000000,0.498039,0.054902}%
\pgfsetstrokecolor{currentstroke}%
\pgfsetdash{}{0pt}%
\pgfpathmoveto{\pgfqpoint{2.001683in}{3.177918in}}%
\pgfpathcurveto{\pgfqpoint{2.012733in}{3.177918in}}{\pgfqpoint{2.023332in}{3.182309in}}{\pgfqpoint{2.031146in}{3.190122in}}%
\pgfpathcurveto{\pgfqpoint{2.038959in}{3.197936in}}{\pgfqpoint{2.043350in}{3.208535in}}{\pgfqpoint{2.043350in}{3.219585in}}%
\pgfpathcurveto{\pgfqpoint{2.043350in}{3.230635in}}{\pgfqpoint{2.038959in}{3.241234in}}{\pgfqpoint{2.031146in}{3.249048in}}%
\pgfpathcurveto{\pgfqpoint{2.023332in}{3.256861in}}{\pgfqpoint{2.012733in}{3.261252in}}{\pgfqpoint{2.001683in}{3.261252in}}%
\pgfpathcurveto{\pgfqpoint{1.990633in}{3.261252in}}{\pgfqpoint{1.980034in}{3.256861in}}{\pgfqpoint{1.972220in}{3.249048in}}%
\pgfpathcurveto{\pgfqpoint{1.964406in}{3.241234in}}{\pgfqpoint{1.960016in}{3.230635in}}{\pgfqpoint{1.960016in}{3.219585in}}%
\pgfpathcurveto{\pgfqpoint{1.960016in}{3.208535in}}{\pgfqpoint{1.964406in}{3.197936in}}{\pgfqpoint{1.972220in}{3.190122in}}%
\pgfpathcurveto{\pgfqpoint{1.980034in}{3.182309in}}{\pgfqpoint{1.990633in}{3.177918in}}{\pgfqpoint{2.001683in}{3.177918in}}%
\pgfpathclose%
\pgfusepath{stroke,fill}%
\end{pgfscope}%
\begin{pgfscope}%
\pgfpathrectangle{\pgfqpoint{0.648703in}{0.548769in}}{\pgfqpoint{5.112893in}{3.102590in}}%
\pgfusepath{clip}%
\pgfsetbuttcap%
\pgfsetroundjoin%
\definecolor{currentfill}{rgb}{1.000000,0.498039,0.054902}%
\pgfsetfillcolor{currentfill}%
\pgfsetlinewidth{1.003750pt}%
\definecolor{currentstroke}{rgb}{1.000000,0.498039,0.054902}%
\pgfsetstrokecolor{currentstroke}%
\pgfsetdash{}{0pt}%
\pgfpathmoveto{\pgfqpoint{3.189659in}{3.198987in}}%
\pgfpathcurveto{\pgfqpoint{3.200709in}{3.198987in}}{\pgfqpoint{3.211308in}{3.203377in}}{\pgfqpoint{3.219122in}{3.211191in}}%
\pgfpathcurveto{\pgfqpoint{3.226935in}{3.219004in}}{\pgfqpoint{3.231326in}{3.229603in}}{\pgfqpoint{3.231326in}{3.240654in}}%
\pgfpathcurveto{\pgfqpoint{3.231326in}{3.251704in}}{\pgfqpoint{3.226935in}{3.262303in}}{\pgfqpoint{3.219122in}{3.270116in}}%
\pgfpathcurveto{\pgfqpoint{3.211308in}{3.277930in}}{\pgfqpoint{3.200709in}{3.282320in}}{\pgfqpoint{3.189659in}{3.282320in}}%
\pgfpathcurveto{\pgfqpoint{3.178609in}{3.282320in}}{\pgfqpoint{3.168010in}{3.277930in}}{\pgfqpoint{3.160196in}{3.270116in}}%
\pgfpathcurveto{\pgfqpoint{3.152383in}{3.262303in}}{\pgfqpoint{3.147992in}{3.251704in}}{\pgfqpoint{3.147992in}{3.240654in}}%
\pgfpathcurveto{\pgfqpoint{3.147992in}{3.229603in}}{\pgfqpoint{3.152383in}{3.219004in}}{\pgfqpoint{3.160196in}{3.211191in}}%
\pgfpathcurveto{\pgfqpoint{3.168010in}{3.203377in}}{\pgfqpoint{3.178609in}{3.198987in}}{\pgfqpoint{3.189659in}{3.198987in}}%
\pgfpathclose%
\pgfusepath{stroke,fill}%
\end{pgfscope}%
\begin{pgfscope}%
\pgfpathrectangle{\pgfqpoint{0.648703in}{0.548769in}}{\pgfqpoint{5.112893in}{3.102590in}}%
\pgfusepath{clip}%
\pgfsetbuttcap%
\pgfsetroundjoin%
\definecolor{currentfill}{rgb}{1.000000,0.498039,0.054902}%
\pgfsetfillcolor{currentfill}%
\pgfsetlinewidth{1.003750pt}%
\definecolor{currentstroke}{rgb}{1.000000,0.498039,0.054902}%
\pgfsetstrokecolor{currentstroke}%
\pgfsetdash{}{0pt}%
\pgfpathmoveto{\pgfqpoint{2.765912in}{3.186346in}}%
\pgfpathcurveto{\pgfqpoint{2.776963in}{3.186346in}}{\pgfqpoint{2.787562in}{3.190736in}}{\pgfqpoint{2.795375in}{3.198550in}}%
\pgfpathcurveto{\pgfqpoint{2.803189in}{3.206363in}}{\pgfqpoint{2.807579in}{3.216962in}}{\pgfqpoint{2.807579in}{3.228012in}}%
\pgfpathcurveto{\pgfqpoint{2.807579in}{3.239063in}}{\pgfqpoint{2.803189in}{3.249662in}}{\pgfqpoint{2.795375in}{3.257475in}}%
\pgfpathcurveto{\pgfqpoint{2.787562in}{3.265289in}}{\pgfqpoint{2.776963in}{3.269679in}}{\pgfqpoint{2.765912in}{3.269679in}}%
\pgfpathcurveto{\pgfqpoint{2.754862in}{3.269679in}}{\pgfqpoint{2.744263in}{3.265289in}}{\pgfqpoint{2.736450in}{3.257475in}}%
\pgfpathcurveto{\pgfqpoint{2.728636in}{3.249662in}}{\pgfqpoint{2.724246in}{3.239063in}}{\pgfqpoint{2.724246in}{3.228012in}}%
\pgfpathcurveto{\pgfqpoint{2.724246in}{3.216962in}}{\pgfqpoint{2.728636in}{3.206363in}}{\pgfqpoint{2.736450in}{3.198550in}}%
\pgfpathcurveto{\pgfqpoint{2.744263in}{3.190736in}}{\pgfqpoint{2.754862in}{3.186346in}}{\pgfqpoint{2.765912in}{3.186346in}}%
\pgfpathclose%
\pgfusepath{stroke,fill}%
\end{pgfscope}%
\begin{pgfscope}%
\pgfpathrectangle{\pgfqpoint{0.648703in}{0.548769in}}{\pgfqpoint{5.112893in}{3.102590in}}%
\pgfusepath{clip}%
\pgfsetbuttcap%
\pgfsetroundjoin%
\definecolor{currentfill}{rgb}{0.121569,0.466667,0.705882}%
\pgfsetfillcolor{currentfill}%
\pgfsetlinewidth{1.003750pt}%
\definecolor{currentstroke}{rgb}{0.121569,0.466667,0.705882}%
\pgfsetstrokecolor{currentstroke}%
\pgfsetdash{}{0pt}%
\pgfpathmoveto{\pgfqpoint{0.832863in}{0.662326in}}%
\pgfpathcurveto{\pgfqpoint{0.843914in}{0.662326in}}{\pgfqpoint{0.854513in}{0.666717in}}{\pgfqpoint{0.862326in}{0.674530in}}%
\pgfpathcurveto{\pgfqpoint{0.870140in}{0.682344in}}{\pgfqpoint{0.874530in}{0.692943in}}{\pgfqpoint{0.874530in}{0.703993in}}%
\pgfpathcurveto{\pgfqpoint{0.874530in}{0.715043in}}{\pgfqpoint{0.870140in}{0.725642in}}{\pgfqpoint{0.862326in}{0.733456in}}%
\pgfpathcurveto{\pgfqpoint{0.854513in}{0.741270in}}{\pgfqpoint{0.843914in}{0.745660in}}{\pgfqpoint{0.832863in}{0.745660in}}%
\pgfpathcurveto{\pgfqpoint{0.821813in}{0.745660in}}{\pgfqpoint{0.811214in}{0.741270in}}{\pgfqpoint{0.803401in}{0.733456in}}%
\pgfpathcurveto{\pgfqpoint{0.795587in}{0.725642in}}{\pgfqpoint{0.791197in}{0.715043in}}{\pgfqpoint{0.791197in}{0.703993in}}%
\pgfpathcurveto{\pgfqpoint{0.791197in}{0.692943in}}{\pgfqpoint{0.795587in}{0.682344in}}{\pgfqpoint{0.803401in}{0.674530in}}%
\pgfpathcurveto{\pgfqpoint{0.811214in}{0.666717in}}{\pgfqpoint{0.821813in}{0.662326in}}{\pgfqpoint{0.832863in}{0.662326in}}%
\pgfpathclose%
\pgfusepath{stroke,fill}%
\end{pgfscope}%
\begin{pgfscope}%
\pgfpathrectangle{\pgfqpoint{0.648703in}{0.548769in}}{\pgfqpoint{5.112893in}{3.102590in}}%
\pgfusepath{clip}%
\pgfsetbuttcap%
\pgfsetroundjoin%
\definecolor{currentfill}{rgb}{1.000000,0.498039,0.054902}%
\pgfsetfillcolor{currentfill}%
\pgfsetlinewidth{1.003750pt}%
\definecolor{currentstroke}{rgb}{1.000000,0.498039,0.054902}%
\pgfsetstrokecolor{currentstroke}%
\pgfsetdash{}{0pt}%
\pgfpathmoveto{\pgfqpoint{2.599977in}{3.182132in}}%
\pgfpathcurveto{\pgfqpoint{2.611027in}{3.182132in}}{\pgfqpoint{2.621626in}{3.186522in}}{\pgfqpoint{2.629440in}{3.194336in}}%
\pgfpathcurveto{\pgfqpoint{2.637254in}{3.202150in}}{\pgfqpoint{2.641644in}{3.212749in}}{\pgfqpoint{2.641644in}{3.223799in}}%
\pgfpathcurveto{\pgfqpoint{2.641644in}{3.234849in}}{\pgfqpoint{2.637254in}{3.245448in}}{\pgfqpoint{2.629440in}{3.253262in}}%
\pgfpathcurveto{\pgfqpoint{2.621626in}{3.261075in}}{\pgfqpoint{2.611027in}{3.265465in}}{\pgfqpoint{2.599977in}{3.265465in}}%
\pgfpathcurveto{\pgfqpoint{2.588927in}{3.265465in}}{\pgfqpoint{2.578328in}{3.261075in}}{\pgfqpoint{2.570514in}{3.253262in}}%
\pgfpathcurveto{\pgfqpoint{2.562701in}{3.245448in}}{\pgfqpoint{2.558311in}{3.234849in}}{\pgfqpoint{2.558311in}{3.223799in}}%
\pgfpathcurveto{\pgfqpoint{2.558311in}{3.212749in}}{\pgfqpoint{2.562701in}{3.202150in}}{\pgfqpoint{2.570514in}{3.194336in}}%
\pgfpathcurveto{\pgfqpoint{2.578328in}{3.186522in}}{\pgfqpoint{2.588927in}{3.182132in}}{\pgfqpoint{2.599977in}{3.182132in}}%
\pgfpathclose%
\pgfusepath{stroke,fill}%
\end{pgfscope}%
\begin{pgfscope}%
\pgfpathrectangle{\pgfqpoint{0.648703in}{0.548769in}}{\pgfqpoint{5.112893in}{3.102590in}}%
\pgfusepath{clip}%
\pgfsetbuttcap%
\pgfsetroundjoin%
\definecolor{currentfill}{rgb}{1.000000,0.498039,0.054902}%
\pgfsetfillcolor{currentfill}%
\pgfsetlinewidth{1.003750pt}%
\definecolor{currentstroke}{rgb}{1.000000,0.498039,0.054902}%
\pgfsetstrokecolor{currentstroke}%
\pgfsetdash{}{0pt}%
\pgfpathmoveto{\pgfqpoint{3.384802in}{3.190560in}}%
\pgfpathcurveto{\pgfqpoint{3.395852in}{3.190560in}}{\pgfqpoint{3.406451in}{3.194950in}}{\pgfqpoint{3.414265in}{3.202763in}}%
\pgfpathcurveto{\pgfqpoint{3.422079in}{3.210577in}}{\pgfqpoint{3.426469in}{3.221176in}}{\pgfqpoint{3.426469in}{3.232226in}}%
\pgfpathcurveto{\pgfqpoint{3.426469in}{3.243276in}}{\pgfqpoint{3.422079in}{3.253875in}}{\pgfqpoint{3.414265in}{3.261689in}}%
\pgfpathcurveto{\pgfqpoint{3.406451in}{3.269503in}}{\pgfqpoint{3.395852in}{3.273893in}}{\pgfqpoint{3.384802in}{3.273893in}}%
\pgfpathcurveto{\pgfqpoint{3.373752in}{3.273893in}}{\pgfqpoint{3.363153in}{3.269503in}}{\pgfqpoint{3.355340in}{3.261689in}}%
\pgfpathcurveto{\pgfqpoint{3.347526in}{3.253875in}}{\pgfqpoint{3.343136in}{3.243276in}}{\pgfqpoint{3.343136in}{3.232226in}}%
\pgfpathcurveto{\pgfqpoint{3.343136in}{3.221176in}}{\pgfqpoint{3.347526in}{3.210577in}}{\pgfqpoint{3.355340in}{3.202763in}}%
\pgfpathcurveto{\pgfqpoint{3.363153in}{3.194950in}}{\pgfqpoint{3.373752in}{3.190560in}}{\pgfqpoint{3.384802in}{3.190560in}}%
\pgfpathclose%
\pgfusepath{stroke,fill}%
\end{pgfscope}%
\begin{pgfscope}%
\pgfpathrectangle{\pgfqpoint{0.648703in}{0.548769in}}{\pgfqpoint{5.112893in}{3.102590in}}%
\pgfusepath{clip}%
\pgfsetbuttcap%
\pgfsetroundjoin%
\definecolor{currentfill}{rgb}{0.121569,0.466667,0.705882}%
\pgfsetfillcolor{currentfill}%
\pgfsetlinewidth{1.003750pt}%
\definecolor{currentstroke}{rgb}{0.121569,0.466667,0.705882}%
\pgfsetstrokecolor{currentstroke}%
\pgfsetdash{}{0pt}%
\pgfpathmoveto{\pgfqpoint{0.831271in}{0.658113in}}%
\pgfpathcurveto{\pgfqpoint{0.842321in}{0.658113in}}{\pgfqpoint{0.852920in}{0.662503in}}{\pgfqpoint{0.860734in}{0.670317in}}%
\pgfpathcurveto{\pgfqpoint{0.868548in}{0.678130in}}{\pgfqpoint{0.872938in}{0.688729in}}{\pgfqpoint{0.872938in}{0.699779in}}%
\pgfpathcurveto{\pgfqpoint{0.872938in}{0.710830in}}{\pgfqpoint{0.868548in}{0.721429in}}{\pgfqpoint{0.860734in}{0.729242in}}%
\pgfpathcurveto{\pgfqpoint{0.852920in}{0.737056in}}{\pgfqpoint{0.842321in}{0.741446in}}{\pgfqpoint{0.831271in}{0.741446in}}%
\pgfpathcurveto{\pgfqpoint{0.820221in}{0.741446in}}{\pgfqpoint{0.809622in}{0.737056in}}{\pgfqpoint{0.801808in}{0.729242in}}%
\pgfpathcurveto{\pgfqpoint{0.793995in}{0.721429in}}{\pgfqpoint{0.789604in}{0.710830in}}{\pgfqpoint{0.789604in}{0.699779in}}%
\pgfpathcurveto{\pgfqpoint{0.789604in}{0.688729in}}{\pgfqpoint{0.793995in}{0.678130in}}{\pgfqpoint{0.801808in}{0.670317in}}%
\pgfpathcurveto{\pgfqpoint{0.809622in}{0.662503in}}{\pgfqpoint{0.820221in}{0.658113in}}{\pgfqpoint{0.831271in}{0.658113in}}%
\pgfpathclose%
\pgfusepath{stroke,fill}%
\end{pgfscope}%
\begin{pgfscope}%
\pgfpathrectangle{\pgfqpoint{0.648703in}{0.548769in}}{\pgfqpoint{5.112893in}{3.102590in}}%
\pgfusepath{clip}%
\pgfsetbuttcap%
\pgfsetroundjoin%
\definecolor{currentfill}{rgb}{0.121569,0.466667,0.705882}%
\pgfsetfillcolor{currentfill}%
\pgfsetlinewidth{1.003750pt}%
\definecolor{currentstroke}{rgb}{0.121569,0.466667,0.705882}%
\pgfsetstrokecolor{currentstroke}%
\pgfsetdash{}{0pt}%
\pgfpathmoveto{\pgfqpoint{1.141511in}{0.784524in}}%
\pgfpathcurveto{\pgfqpoint{1.152561in}{0.784524in}}{\pgfqpoint{1.163160in}{0.788915in}}{\pgfqpoint{1.170973in}{0.796728in}}%
\pgfpathcurveto{\pgfqpoint{1.178787in}{0.804542in}}{\pgfqpoint{1.183177in}{0.815141in}}{\pgfqpoint{1.183177in}{0.826191in}}%
\pgfpathcurveto{\pgfqpoint{1.183177in}{0.837241in}}{\pgfqpoint{1.178787in}{0.847840in}}{\pgfqpoint{1.170973in}{0.855654in}}%
\pgfpathcurveto{\pgfqpoint{1.163160in}{0.863467in}}{\pgfqpoint{1.152561in}{0.867858in}}{\pgfqpoint{1.141511in}{0.867858in}}%
\pgfpathcurveto{\pgfqpoint{1.130460in}{0.867858in}}{\pgfqpoint{1.119861in}{0.863467in}}{\pgfqpoint{1.112048in}{0.855654in}}%
\pgfpathcurveto{\pgfqpoint{1.104234in}{0.847840in}}{\pgfqpoint{1.099844in}{0.837241in}}{\pgfqpoint{1.099844in}{0.826191in}}%
\pgfpathcurveto{\pgfqpoint{1.099844in}{0.815141in}}{\pgfqpoint{1.104234in}{0.804542in}}{\pgfqpoint{1.112048in}{0.796728in}}%
\pgfpathcurveto{\pgfqpoint{1.119861in}{0.788915in}}{\pgfqpoint{1.130460in}{0.784524in}}{\pgfqpoint{1.141511in}{0.784524in}}%
\pgfpathclose%
\pgfusepath{stroke,fill}%
\end{pgfscope}%
\begin{pgfscope}%
\pgfpathrectangle{\pgfqpoint{0.648703in}{0.548769in}}{\pgfqpoint{5.112893in}{3.102590in}}%
\pgfusepath{clip}%
\pgfsetbuttcap%
\pgfsetroundjoin%
\definecolor{currentfill}{rgb}{0.121569,0.466667,0.705882}%
\pgfsetfillcolor{currentfill}%
\pgfsetlinewidth{1.003750pt}%
\definecolor{currentstroke}{rgb}{0.121569,0.466667,0.705882}%
\pgfsetstrokecolor{currentstroke}%
\pgfsetdash{}{0pt}%
\pgfpathmoveto{\pgfqpoint{1.086241in}{0.792952in}}%
\pgfpathcurveto{\pgfqpoint{1.097291in}{0.792952in}}{\pgfqpoint{1.107890in}{0.797342in}}{\pgfqpoint{1.115704in}{0.805156in}}%
\pgfpathcurveto{\pgfqpoint{1.123517in}{0.812969in}}{\pgfqpoint{1.127907in}{0.823568in}}{\pgfqpoint{1.127907in}{0.834618in}}%
\pgfpathcurveto{\pgfqpoint{1.127907in}{0.845669in}}{\pgfqpoint{1.123517in}{0.856268in}}{\pgfqpoint{1.115704in}{0.864081in}}%
\pgfpathcurveto{\pgfqpoint{1.107890in}{0.871895in}}{\pgfqpoint{1.097291in}{0.876285in}}{\pgfqpoint{1.086241in}{0.876285in}}%
\pgfpathcurveto{\pgfqpoint{1.075191in}{0.876285in}}{\pgfqpoint{1.064592in}{0.871895in}}{\pgfqpoint{1.056778in}{0.864081in}}%
\pgfpathcurveto{\pgfqpoint{1.048964in}{0.856268in}}{\pgfqpoint{1.044574in}{0.845669in}}{\pgfqpoint{1.044574in}{0.834618in}}%
\pgfpathcurveto{\pgfqpoint{1.044574in}{0.823568in}}{\pgfqpoint{1.048964in}{0.812969in}}{\pgfqpoint{1.056778in}{0.805156in}}%
\pgfpathcurveto{\pgfqpoint{1.064592in}{0.797342in}}{\pgfqpoint{1.075191in}{0.792952in}}{\pgfqpoint{1.086241in}{0.792952in}}%
\pgfpathclose%
\pgfusepath{stroke,fill}%
\end{pgfscope}%
\begin{pgfscope}%
\pgfpathrectangle{\pgfqpoint{0.648703in}{0.548769in}}{\pgfqpoint{5.112893in}{3.102590in}}%
\pgfusepath{clip}%
\pgfsetbuttcap%
\pgfsetroundjoin%
\definecolor{currentfill}{rgb}{0.121569,0.466667,0.705882}%
\pgfsetfillcolor{currentfill}%
\pgfsetlinewidth{1.003750pt}%
\definecolor{currentstroke}{rgb}{0.121569,0.466667,0.705882}%
\pgfsetstrokecolor{currentstroke}%
\pgfsetdash{}{0pt}%
\pgfpathmoveto{\pgfqpoint{0.846099in}{0.662326in}}%
\pgfpathcurveto{\pgfqpoint{0.857149in}{0.662326in}}{\pgfqpoint{0.867748in}{0.666717in}}{\pgfqpoint{0.875562in}{0.674530in}}%
\pgfpathcurveto{\pgfqpoint{0.883376in}{0.682344in}}{\pgfqpoint{0.887766in}{0.692943in}}{\pgfqpoint{0.887766in}{0.703993in}}%
\pgfpathcurveto{\pgfqpoint{0.887766in}{0.715043in}}{\pgfqpoint{0.883376in}{0.725642in}}{\pgfqpoint{0.875562in}{0.733456in}}%
\pgfpathcurveto{\pgfqpoint{0.867748in}{0.741270in}}{\pgfqpoint{0.857149in}{0.745660in}}{\pgfqpoint{0.846099in}{0.745660in}}%
\pgfpathcurveto{\pgfqpoint{0.835049in}{0.745660in}}{\pgfqpoint{0.824450in}{0.741270in}}{\pgfqpoint{0.816637in}{0.733456in}}%
\pgfpathcurveto{\pgfqpoint{0.808823in}{0.725642in}}{\pgfqpoint{0.804433in}{0.715043in}}{\pgfqpoint{0.804433in}{0.703993in}}%
\pgfpathcurveto{\pgfqpoint{0.804433in}{0.692943in}}{\pgfqpoint{0.808823in}{0.682344in}}{\pgfqpoint{0.816637in}{0.674530in}}%
\pgfpathcurveto{\pgfqpoint{0.824450in}{0.666717in}}{\pgfqpoint{0.835049in}{0.662326in}}{\pgfqpoint{0.846099in}{0.662326in}}%
\pgfpathclose%
\pgfusepath{stroke,fill}%
\end{pgfscope}%
\begin{pgfscope}%
\pgfpathrectangle{\pgfqpoint{0.648703in}{0.548769in}}{\pgfqpoint{5.112893in}{3.102590in}}%
\pgfusepath{clip}%
\pgfsetbuttcap%
\pgfsetroundjoin%
\definecolor{currentfill}{rgb}{0.121569,0.466667,0.705882}%
\pgfsetfillcolor{currentfill}%
\pgfsetlinewidth{1.003750pt}%
\definecolor{currentstroke}{rgb}{0.121569,0.466667,0.705882}%
\pgfsetstrokecolor{currentstroke}%
\pgfsetdash{}{0pt}%
\pgfpathmoveto{\pgfqpoint{0.831246in}{0.658113in}}%
\pgfpathcurveto{\pgfqpoint{0.842296in}{0.658113in}}{\pgfqpoint{0.852896in}{0.662503in}}{\pgfqpoint{0.860709in}{0.670317in}}%
\pgfpathcurveto{\pgfqpoint{0.868523in}{0.678130in}}{\pgfqpoint{0.872913in}{0.688729in}}{\pgfqpoint{0.872913in}{0.699779in}}%
\pgfpathcurveto{\pgfqpoint{0.872913in}{0.710830in}}{\pgfqpoint{0.868523in}{0.721429in}}{\pgfqpoint{0.860709in}{0.729242in}}%
\pgfpathcurveto{\pgfqpoint{0.852896in}{0.737056in}}{\pgfqpoint{0.842296in}{0.741446in}}{\pgfqpoint{0.831246in}{0.741446in}}%
\pgfpathcurveto{\pgfqpoint{0.820196in}{0.741446in}}{\pgfqpoint{0.809597in}{0.737056in}}{\pgfqpoint{0.801784in}{0.729242in}}%
\pgfpathcurveto{\pgfqpoint{0.793970in}{0.721429in}}{\pgfqpoint{0.789580in}{0.710830in}}{\pgfqpoint{0.789580in}{0.699779in}}%
\pgfpathcurveto{\pgfqpoint{0.789580in}{0.688729in}}{\pgfqpoint{0.793970in}{0.678130in}}{\pgfqpoint{0.801784in}{0.670317in}}%
\pgfpathcurveto{\pgfqpoint{0.809597in}{0.662503in}}{\pgfqpoint{0.820196in}{0.658113in}}{\pgfqpoint{0.831246in}{0.658113in}}%
\pgfpathclose%
\pgfusepath{stroke,fill}%
\end{pgfscope}%
\begin{pgfscope}%
\pgfpathrectangle{\pgfqpoint{0.648703in}{0.548769in}}{\pgfqpoint{5.112893in}{3.102590in}}%
\pgfusepath{clip}%
\pgfsetbuttcap%
\pgfsetroundjoin%
\definecolor{currentfill}{rgb}{0.121569,0.466667,0.705882}%
\pgfsetfillcolor{currentfill}%
\pgfsetlinewidth{1.003750pt}%
\definecolor{currentstroke}{rgb}{0.121569,0.466667,0.705882}%
\pgfsetstrokecolor{currentstroke}%
\pgfsetdash{}{0pt}%
\pgfpathmoveto{\pgfqpoint{0.831244in}{0.658113in}}%
\pgfpathcurveto{\pgfqpoint{0.842294in}{0.658113in}}{\pgfqpoint{0.852893in}{0.662503in}}{\pgfqpoint{0.860707in}{0.670317in}}%
\pgfpathcurveto{\pgfqpoint{0.868520in}{0.678130in}}{\pgfqpoint{0.872911in}{0.688729in}}{\pgfqpoint{0.872911in}{0.699779in}}%
\pgfpathcurveto{\pgfqpoint{0.872911in}{0.710830in}}{\pgfqpoint{0.868520in}{0.721429in}}{\pgfqpoint{0.860707in}{0.729242in}}%
\pgfpathcurveto{\pgfqpoint{0.852893in}{0.737056in}}{\pgfqpoint{0.842294in}{0.741446in}}{\pgfqpoint{0.831244in}{0.741446in}}%
\pgfpathcurveto{\pgfqpoint{0.820194in}{0.741446in}}{\pgfqpoint{0.809595in}{0.737056in}}{\pgfqpoint{0.801781in}{0.729242in}}%
\pgfpathcurveto{\pgfqpoint{0.793968in}{0.721429in}}{\pgfqpoint{0.789577in}{0.710830in}}{\pgfqpoint{0.789577in}{0.699779in}}%
\pgfpathcurveto{\pgfqpoint{0.789577in}{0.688729in}}{\pgfqpoint{0.793968in}{0.678130in}}{\pgfqpoint{0.801781in}{0.670317in}}%
\pgfpathcurveto{\pgfqpoint{0.809595in}{0.662503in}}{\pgfqpoint{0.820194in}{0.658113in}}{\pgfqpoint{0.831244in}{0.658113in}}%
\pgfpathclose%
\pgfusepath{stroke,fill}%
\end{pgfscope}%
\begin{pgfscope}%
\pgfpathrectangle{\pgfqpoint{0.648703in}{0.548769in}}{\pgfqpoint{5.112893in}{3.102590in}}%
\pgfusepath{clip}%
\pgfsetbuttcap%
\pgfsetroundjoin%
\definecolor{currentfill}{rgb}{1.000000,0.498039,0.054902}%
\pgfsetfillcolor{currentfill}%
\pgfsetlinewidth{1.003750pt}%
\definecolor{currentstroke}{rgb}{1.000000,0.498039,0.054902}%
\pgfsetstrokecolor{currentstroke}%
\pgfsetdash{}{0pt}%
\pgfpathmoveto{\pgfqpoint{3.001252in}{3.194773in}}%
\pgfpathcurveto{\pgfqpoint{3.012302in}{3.194773in}}{\pgfqpoint{3.022901in}{3.199163in}}{\pgfqpoint{3.030715in}{3.206977in}}%
\pgfpathcurveto{\pgfqpoint{3.038528in}{3.214791in}}{\pgfqpoint{3.042918in}{3.225390in}}{\pgfqpoint{3.042918in}{3.236440in}}%
\pgfpathcurveto{\pgfqpoint{3.042918in}{3.247490in}}{\pgfqpoint{3.038528in}{3.258089in}}{\pgfqpoint{3.030715in}{3.265903in}}%
\pgfpathcurveto{\pgfqpoint{3.022901in}{3.273716in}}{\pgfqpoint{3.012302in}{3.278107in}}{\pgfqpoint{3.001252in}{3.278107in}}%
\pgfpathcurveto{\pgfqpoint{2.990202in}{3.278107in}}{\pgfqpoint{2.979603in}{3.273716in}}{\pgfqpoint{2.971789in}{3.265903in}}%
\pgfpathcurveto{\pgfqpoint{2.963975in}{3.258089in}}{\pgfqpoint{2.959585in}{3.247490in}}{\pgfqpoint{2.959585in}{3.236440in}}%
\pgfpathcurveto{\pgfqpoint{2.959585in}{3.225390in}}{\pgfqpoint{2.963975in}{3.214791in}}{\pgfqpoint{2.971789in}{3.206977in}}%
\pgfpathcurveto{\pgfqpoint{2.979603in}{3.199163in}}{\pgfqpoint{2.990202in}{3.194773in}}{\pgfqpoint{3.001252in}{3.194773in}}%
\pgfpathclose%
\pgfusepath{stroke,fill}%
\end{pgfscope}%
\begin{pgfscope}%
\pgfpathrectangle{\pgfqpoint{0.648703in}{0.548769in}}{\pgfqpoint{5.112893in}{3.102590in}}%
\pgfusepath{clip}%
\pgfsetbuttcap%
\pgfsetroundjoin%
\definecolor{currentfill}{rgb}{1.000000,0.498039,0.054902}%
\pgfsetfillcolor{currentfill}%
\pgfsetlinewidth{1.003750pt}%
\definecolor{currentstroke}{rgb}{1.000000,0.498039,0.054902}%
\pgfsetstrokecolor{currentstroke}%
\pgfsetdash{}{0pt}%
\pgfpathmoveto{\pgfqpoint{4.009246in}{3.232697in}}%
\pgfpathcurveto{\pgfqpoint{4.020296in}{3.232697in}}{\pgfqpoint{4.030895in}{3.237087in}}{\pgfqpoint{4.038709in}{3.244901in}}%
\pgfpathcurveto{\pgfqpoint{4.046522in}{3.252714in}}{\pgfqpoint{4.050913in}{3.263313in}}{\pgfqpoint{4.050913in}{3.274363in}}%
\pgfpathcurveto{\pgfqpoint{4.050913in}{3.285414in}}{\pgfqpoint{4.046522in}{3.296013in}}{\pgfqpoint{4.038709in}{3.303826in}}%
\pgfpathcurveto{\pgfqpoint{4.030895in}{3.311640in}}{\pgfqpoint{4.020296in}{3.316030in}}{\pgfqpoint{4.009246in}{3.316030in}}%
\pgfpathcurveto{\pgfqpoint{3.998196in}{3.316030in}}{\pgfqpoint{3.987597in}{3.311640in}}{\pgfqpoint{3.979783in}{3.303826in}}%
\pgfpathcurveto{\pgfqpoint{3.971970in}{3.296013in}}{\pgfqpoint{3.967579in}{3.285414in}}{\pgfqpoint{3.967579in}{3.274363in}}%
\pgfpathcurveto{\pgfqpoint{3.967579in}{3.263313in}}{\pgfqpoint{3.971970in}{3.252714in}}{\pgfqpoint{3.979783in}{3.244901in}}%
\pgfpathcurveto{\pgfqpoint{3.987597in}{3.237087in}}{\pgfqpoint{3.998196in}{3.232697in}}{\pgfqpoint{4.009246in}{3.232697in}}%
\pgfpathclose%
\pgfusepath{stroke,fill}%
\end{pgfscope}%
\begin{pgfscope}%
\pgfpathrectangle{\pgfqpoint{0.648703in}{0.548769in}}{\pgfqpoint{5.112893in}{3.102590in}}%
\pgfusepath{clip}%
\pgfsetbuttcap%
\pgfsetroundjoin%
\definecolor{currentfill}{rgb}{0.121569,0.466667,0.705882}%
\pgfsetfillcolor{currentfill}%
\pgfsetlinewidth{1.003750pt}%
\definecolor{currentstroke}{rgb}{0.121569,0.466667,0.705882}%
\pgfsetstrokecolor{currentstroke}%
\pgfsetdash{}{0pt}%
\pgfpathmoveto{\pgfqpoint{1.147625in}{0.767669in}}%
\pgfpathcurveto{\pgfqpoint{1.158675in}{0.767669in}}{\pgfqpoint{1.169274in}{0.772060in}}{\pgfqpoint{1.177088in}{0.779873in}}%
\pgfpathcurveto{\pgfqpoint{1.184901in}{0.787687in}}{\pgfqpoint{1.189292in}{0.798286in}}{\pgfqpoint{1.189292in}{0.809336in}}%
\pgfpathcurveto{\pgfqpoint{1.189292in}{0.820386in}}{\pgfqpoint{1.184901in}{0.830985in}}{\pgfqpoint{1.177088in}{0.838799in}}%
\pgfpathcurveto{\pgfqpoint{1.169274in}{0.846613in}}{\pgfqpoint{1.158675in}{0.851003in}}{\pgfqpoint{1.147625in}{0.851003in}}%
\pgfpathcurveto{\pgfqpoint{1.136575in}{0.851003in}}{\pgfqpoint{1.125976in}{0.846613in}}{\pgfqpoint{1.118162in}{0.838799in}}%
\pgfpathcurveto{\pgfqpoint{1.110348in}{0.830985in}}{\pgfqpoint{1.105958in}{0.820386in}}{\pgfqpoint{1.105958in}{0.809336in}}%
\pgfpathcurveto{\pgfqpoint{1.105958in}{0.798286in}}{\pgfqpoint{1.110348in}{0.787687in}}{\pgfqpoint{1.118162in}{0.779873in}}%
\pgfpathcurveto{\pgfqpoint{1.125976in}{0.772060in}}{\pgfqpoint{1.136575in}{0.767669in}}{\pgfqpoint{1.147625in}{0.767669in}}%
\pgfpathclose%
\pgfusepath{stroke,fill}%
\end{pgfscope}%
\begin{pgfscope}%
\pgfpathrectangle{\pgfqpoint{0.648703in}{0.548769in}}{\pgfqpoint{5.112893in}{3.102590in}}%
\pgfusepath{clip}%
\pgfsetbuttcap%
\pgfsetroundjoin%
\definecolor{currentfill}{rgb}{1.000000,0.498039,0.054902}%
\pgfsetfillcolor{currentfill}%
\pgfsetlinewidth{1.003750pt}%
\definecolor{currentstroke}{rgb}{1.000000,0.498039,0.054902}%
\pgfsetstrokecolor{currentstroke}%
\pgfsetdash{}{0pt}%
\pgfpathmoveto{\pgfqpoint{2.662303in}{3.186346in}}%
\pgfpathcurveto{\pgfqpoint{2.673353in}{3.186346in}}{\pgfqpoint{2.683952in}{3.190736in}}{\pgfqpoint{2.691766in}{3.198550in}}%
\pgfpathcurveto{\pgfqpoint{2.699580in}{3.206363in}}{\pgfqpoint{2.703970in}{3.216962in}}{\pgfqpoint{2.703970in}{3.228012in}}%
\pgfpathcurveto{\pgfqpoint{2.703970in}{3.239063in}}{\pgfqpoint{2.699580in}{3.249662in}}{\pgfqpoint{2.691766in}{3.257475in}}%
\pgfpathcurveto{\pgfqpoint{2.683952in}{3.265289in}}{\pgfqpoint{2.673353in}{3.269679in}}{\pgfqpoint{2.662303in}{3.269679in}}%
\pgfpathcurveto{\pgfqpoint{2.651253in}{3.269679in}}{\pgfqpoint{2.640654in}{3.265289in}}{\pgfqpoint{2.632840in}{3.257475in}}%
\pgfpathcurveto{\pgfqpoint{2.625027in}{3.249662in}}{\pgfqpoint{2.620636in}{3.239063in}}{\pgfqpoint{2.620636in}{3.228012in}}%
\pgfpathcurveto{\pgfqpoint{2.620636in}{3.216962in}}{\pgfqpoint{2.625027in}{3.206363in}}{\pgfqpoint{2.632840in}{3.198550in}}%
\pgfpathcurveto{\pgfqpoint{2.640654in}{3.190736in}}{\pgfqpoint{2.651253in}{3.186346in}}{\pgfqpoint{2.662303in}{3.186346in}}%
\pgfpathclose%
\pgfusepath{stroke,fill}%
\end{pgfscope}%
\begin{pgfscope}%
\pgfpathrectangle{\pgfqpoint{0.648703in}{0.548769in}}{\pgfqpoint{5.112893in}{3.102590in}}%
\pgfusepath{clip}%
\pgfsetbuttcap%
\pgfsetroundjoin%
\definecolor{currentfill}{rgb}{1.000000,0.498039,0.054902}%
\pgfsetfillcolor{currentfill}%
\pgfsetlinewidth{1.003750pt}%
\definecolor{currentstroke}{rgb}{1.000000,0.498039,0.054902}%
\pgfsetstrokecolor{currentstroke}%
\pgfsetdash{}{0pt}%
\pgfpathmoveto{\pgfqpoint{3.539391in}{3.203201in}}%
\pgfpathcurveto{\pgfqpoint{3.550441in}{3.203201in}}{\pgfqpoint{3.561041in}{3.207591in}}{\pgfqpoint{3.568854in}{3.215405in}}%
\pgfpathcurveto{\pgfqpoint{3.576668in}{3.223218in}}{\pgfqpoint{3.581058in}{3.233817in}}{\pgfqpoint{3.581058in}{3.244867in}}%
\pgfpathcurveto{\pgfqpoint{3.581058in}{3.255917in}}{\pgfqpoint{3.576668in}{3.266516in}}{\pgfqpoint{3.568854in}{3.274330in}}%
\pgfpathcurveto{\pgfqpoint{3.561041in}{3.282144in}}{\pgfqpoint{3.550441in}{3.286534in}}{\pgfqpoint{3.539391in}{3.286534in}}%
\pgfpathcurveto{\pgfqpoint{3.528341in}{3.286534in}}{\pgfqpoint{3.517742in}{3.282144in}}{\pgfqpoint{3.509929in}{3.274330in}}%
\pgfpathcurveto{\pgfqpoint{3.502115in}{3.266516in}}{\pgfqpoint{3.497725in}{3.255917in}}{\pgfqpoint{3.497725in}{3.244867in}}%
\pgfpathcurveto{\pgfqpoint{3.497725in}{3.233817in}}{\pgfqpoint{3.502115in}{3.223218in}}{\pgfqpoint{3.509929in}{3.215405in}}%
\pgfpathcurveto{\pgfqpoint{3.517742in}{3.207591in}}{\pgfqpoint{3.528341in}{3.203201in}}{\pgfqpoint{3.539391in}{3.203201in}}%
\pgfpathclose%
\pgfusepath{stroke,fill}%
\end{pgfscope}%
\begin{pgfscope}%
\pgfpathrectangle{\pgfqpoint{0.648703in}{0.548769in}}{\pgfqpoint{5.112893in}{3.102590in}}%
\pgfusepath{clip}%
\pgfsetbuttcap%
\pgfsetroundjoin%
\definecolor{currentfill}{rgb}{0.121569,0.466667,0.705882}%
\pgfsetfillcolor{currentfill}%
\pgfsetlinewidth{1.003750pt}%
\definecolor{currentstroke}{rgb}{0.121569,0.466667,0.705882}%
\pgfsetstrokecolor{currentstroke}%
\pgfsetdash{}{0pt}%
\pgfpathmoveto{\pgfqpoint{0.831244in}{0.658113in}}%
\pgfpathcurveto{\pgfqpoint{0.842294in}{0.658113in}}{\pgfqpoint{0.852893in}{0.662503in}}{\pgfqpoint{0.860707in}{0.670317in}}%
\pgfpathcurveto{\pgfqpoint{0.868520in}{0.678130in}}{\pgfqpoint{0.872911in}{0.688729in}}{\pgfqpoint{0.872911in}{0.699779in}}%
\pgfpathcurveto{\pgfqpoint{0.872911in}{0.710830in}}{\pgfqpoint{0.868520in}{0.721429in}}{\pgfqpoint{0.860707in}{0.729242in}}%
\pgfpathcurveto{\pgfqpoint{0.852893in}{0.737056in}}{\pgfqpoint{0.842294in}{0.741446in}}{\pgfqpoint{0.831244in}{0.741446in}}%
\pgfpathcurveto{\pgfqpoint{0.820194in}{0.741446in}}{\pgfqpoint{0.809595in}{0.737056in}}{\pgfqpoint{0.801781in}{0.729242in}}%
\pgfpathcurveto{\pgfqpoint{0.793968in}{0.721429in}}{\pgfqpoint{0.789577in}{0.710830in}}{\pgfqpoint{0.789577in}{0.699779in}}%
\pgfpathcurveto{\pgfqpoint{0.789577in}{0.688729in}}{\pgfqpoint{0.793968in}{0.678130in}}{\pgfqpoint{0.801781in}{0.670317in}}%
\pgfpathcurveto{\pgfqpoint{0.809595in}{0.662503in}}{\pgfqpoint{0.820194in}{0.658113in}}{\pgfqpoint{0.831244in}{0.658113in}}%
\pgfpathclose%
\pgfusepath{stroke,fill}%
\end{pgfscope}%
\begin{pgfscope}%
\pgfpathrectangle{\pgfqpoint{0.648703in}{0.548769in}}{\pgfqpoint{5.112893in}{3.102590in}}%
\pgfusepath{clip}%
\pgfsetbuttcap%
\pgfsetroundjoin%
\definecolor{currentfill}{rgb}{1.000000,0.498039,0.054902}%
\pgfsetfillcolor{currentfill}%
\pgfsetlinewidth{1.003750pt}%
\definecolor{currentstroke}{rgb}{1.000000,0.498039,0.054902}%
\pgfsetstrokecolor{currentstroke}%
\pgfsetdash{}{0pt}%
\pgfpathmoveto{\pgfqpoint{3.371210in}{3.207414in}}%
\pgfpathcurveto{\pgfqpoint{3.382261in}{3.207414in}}{\pgfqpoint{3.392860in}{3.211805in}}{\pgfqpoint{3.400673in}{3.219618in}}%
\pgfpathcurveto{\pgfqpoint{3.408487in}{3.227432in}}{\pgfqpoint{3.412877in}{3.238031in}}{\pgfqpoint{3.412877in}{3.249081in}}%
\pgfpathcurveto{\pgfqpoint{3.412877in}{3.260131in}}{\pgfqpoint{3.408487in}{3.270730in}}{\pgfqpoint{3.400673in}{3.278544in}}%
\pgfpathcurveto{\pgfqpoint{3.392860in}{3.286357in}}{\pgfqpoint{3.382261in}{3.290748in}}{\pgfqpoint{3.371210in}{3.290748in}}%
\pgfpathcurveto{\pgfqpoint{3.360160in}{3.290748in}}{\pgfqpoint{3.349561in}{3.286357in}}{\pgfqpoint{3.341748in}{3.278544in}}%
\pgfpathcurveto{\pgfqpoint{3.333934in}{3.270730in}}{\pgfqpoint{3.329544in}{3.260131in}}{\pgfqpoint{3.329544in}{3.249081in}}%
\pgfpathcurveto{\pgfqpoint{3.329544in}{3.238031in}}{\pgfqpoint{3.333934in}{3.227432in}}{\pgfqpoint{3.341748in}{3.219618in}}%
\pgfpathcurveto{\pgfqpoint{3.349561in}{3.211805in}}{\pgfqpoint{3.360160in}{3.207414in}}{\pgfqpoint{3.371210in}{3.207414in}}%
\pgfpathclose%
\pgfusepath{stroke,fill}%
\end{pgfscope}%
\begin{pgfscope}%
\pgfpathrectangle{\pgfqpoint{0.648703in}{0.548769in}}{\pgfqpoint{5.112893in}{3.102590in}}%
\pgfusepath{clip}%
\pgfsetbuttcap%
\pgfsetroundjoin%
\definecolor{currentfill}{rgb}{0.121569,0.466667,0.705882}%
\pgfsetfillcolor{currentfill}%
\pgfsetlinewidth{1.003750pt}%
\definecolor{currentstroke}{rgb}{0.121569,0.466667,0.705882}%
\pgfsetstrokecolor{currentstroke}%
\pgfsetdash{}{0pt}%
\pgfpathmoveto{\pgfqpoint{0.831244in}{0.658113in}}%
\pgfpathcurveto{\pgfqpoint{0.842294in}{0.658113in}}{\pgfqpoint{0.852893in}{0.662503in}}{\pgfqpoint{0.860707in}{0.670317in}}%
\pgfpathcurveto{\pgfqpoint{0.868520in}{0.678130in}}{\pgfqpoint{0.872911in}{0.688729in}}{\pgfqpoint{0.872911in}{0.699779in}}%
\pgfpathcurveto{\pgfqpoint{0.872911in}{0.710830in}}{\pgfqpoint{0.868520in}{0.721429in}}{\pgfqpoint{0.860707in}{0.729242in}}%
\pgfpathcurveto{\pgfqpoint{0.852893in}{0.737056in}}{\pgfqpoint{0.842294in}{0.741446in}}{\pgfqpoint{0.831244in}{0.741446in}}%
\pgfpathcurveto{\pgfqpoint{0.820194in}{0.741446in}}{\pgfqpoint{0.809595in}{0.737056in}}{\pgfqpoint{0.801781in}{0.729242in}}%
\pgfpathcurveto{\pgfqpoint{0.793968in}{0.721429in}}{\pgfqpoint{0.789577in}{0.710830in}}{\pgfqpoint{0.789577in}{0.699779in}}%
\pgfpathcurveto{\pgfqpoint{0.789577in}{0.688729in}}{\pgfqpoint{0.793968in}{0.678130in}}{\pgfqpoint{0.801781in}{0.670317in}}%
\pgfpathcurveto{\pgfqpoint{0.809595in}{0.662503in}}{\pgfqpoint{0.820194in}{0.658113in}}{\pgfqpoint{0.831244in}{0.658113in}}%
\pgfpathclose%
\pgfusepath{stroke,fill}%
\end{pgfscope}%
\begin{pgfscope}%
\pgfpathrectangle{\pgfqpoint{0.648703in}{0.548769in}}{\pgfqpoint{5.112893in}{3.102590in}}%
\pgfusepath{clip}%
\pgfsetbuttcap%
\pgfsetroundjoin%
\definecolor{currentfill}{rgb}{0.121569,0.466667,0.705882}%
\pgfsetfillcolor{currentfill}%
\pgfsetlinewidth{1.003750pt}%
\definecolor{currentstroke}{rgb}{0.121569,0.466667,0.705882}%
\pgfsetstrokecolor{currentstroke}%
\pgfsetdash{}{0pt}%
\pgfpathmoveto{\pgfqpoint{0.831257in}{0.658113in}}%
\pgfpathcurveto{\pgfqpoint{0.842307in}{0.658113in}}{\pgfqpoint{0.852906in}{0.662503in}}{\pgfqpoint{0.860720in}{0.670317in}}%
\pgfpathcurveto{\pgfqpoint{0.868533in}{0.678130in}}{\pgfqpoint{0.872924in}{0.688729in}}{\pgfqpoint{0.872924in}{0.699779in}}%
\pgfpathcurveto{\pgfqpoint{0.872924in}{0.710830in}}{\pgfqpoint{0.868533in}{0.721429in}}{\pgfqpoint{0.860720in}{0.729242in}}%
\pgfpathcurveto{\pgfqpoint{0.852906in}{0.737056in}}{\pgfqpoint{0.842307in}{0.741446in}}{\pgfqpoint{0.831257in}{0.741446in}}%
\pgfpathcurveto{\pgfqpoint{0.820207in}{0.741446in}}{\pgfqpoint{0.809608in}{0.737056in}}{\pgfqpoint{0.801794in}{0.729242in}}%
\pgfpathcurveto{\pgfqpoint{0.793980in}{0.721429in}}{\pgfqpoint{0.789590in}{0.710830in}}{\pgfqpoint{0.789590in}{0.699779in}}%
\pgfpathcurveto{\pgfqpoint{0.789590in}{0.688729in}}{\pgfqpoint{0.793980in}{0.678130in}}{\pgfqpoint{0.801794in}{0.670317in}}%
\pgfpathcurveto{\pgfqpoint{0.809608in}{0.662503in}}{\pgfqpoint{0.820207in}{0.658113in}}{\pgfqpoint{0.831257in}{0.658113in}}%
\pgfpathclose%
\pgfusepath{stroke,fill}%
\end{pgfscope}%
\begin{pgfscope}%
\pgfpathrectangle{\pgfqpoint{0.648703in}{0.548769in}}{\pgfqpoint{5.112893in}{3.102590in}}%
\pgfusepath{clip}%
\pgfsetbuttcap%
\pgfsetroundjoin%
\definecolor{currentfill}{rgb}{0.121569,0.466667,0.705882}%
\pgfsetfillcolor{currentfill}%
\pgfsetlinewidth{1.003750pt}%
\definecolor{currentstroke}{rgb}{0.121569,0.466667,0.705882}%
\pgfsetstrokecolor{currentstroke}%
\pgfsetdash{}{0pt}%
\pgfpathmoveto{\pgfqpoint{1.176225in}{0.826662in}}%
\pgfpathcurveto{\pgfqpoint{1.187275in}{0.826662in}}{\pgfqpoint{1.197874in}{0.831052in}}{\pgfqpoint{1.205688in}{0.838865in}}%
\pgfpathcurveto{\pgfqpoint{1.213502in}{0.846679in}}{\pgfqpoint{1.217892in}{0.857278in}}{\pgfqpoint{1.217892in}{0.868328in}}%
\pgfpathcurveto{\pgfqpoint{1.217892in}{0.879378in}}{\pgfqpoint{1.213502in}{0.889977in}}{\pgfqpoint{1.205688in}{0.897791in}}%
\pgfpathcurveto{\pgfqpoint{1.197874in}{0.905605in}}{\pgfqpoint{1.187275in}{0.909995in}}{\pgfqpoint{1.176225in}{0.909995in}}%
\pgfpathcurveto{\pgfqpoint{1.165175in}{0.909995in}}{\pgfqpoint{1.154576in}{0.905605in}}{\pgfqpoint{1.146763in}{0.897791in}}%
\pgfpathcurveto{\pgfqpoint{1.138949in}{0.889977in}}{\pgfqpoint{1.134559in}{0.879378in}}{\pgfqpoint{1.134559in}{0.868328in}}%
\pgfpathcurveto{\pgfqpoint{1.134559in}{0.857278in}}{\pgfqpoint{1.138949in}{0.846679in}}{\pgfqpoint{1.146763in}{0.838865in}}%
\pgfpathcurveto{\pgfqpoint{1.154576in}{0.831052in}}{\pgfqpoint{1.165175in}{0.826662in}}{\pgfqpoint{1.176225in}{0.826662in}}%
\pgfpathclose%
\pgfusepath{stroke,fill}%
\end{pgfscope}%
\begin{pgfscope}%
\pgfpathrectangle{\pgfqpoint{0.648703in}{0.548769in}}{\pgfqpoint{5.112893in}{3.102590in}}%
\pgfusepath{clip}%
\pgfsetbuttcap%
\pgfsetroundjoin%
\definecolor{currentfill}{rgb}{1.000000,0.498039,0.054902}%
\pgfsetfillcolor{currentfill}%
\pgfsetlinewidth{1.003750pt}%
\definecolor{currentstroke}{rgb}{1.000000,0.498039,0.054902}%
\pgfsetstrokecolor{currentstroke}%
\pgfsetdash{}{0pt}%
\pgfpathmoveto{\pgfqpoint{3.342548in}{3.190560in}}%
\pgfpathcurveto{\pgfqpoint{3.353599in}{3.190560in}}{\pgfqpoint{3.364198in}{3.194950in}}{\pgfqpoint{3.372011in}{3.202763in}}%
\pgfpathcurveto{\pgfqpoint{3.379825in}{3.210577in}}{\pgfqpoint{3.384215in}{3.221176in}}{\pgfqpoint{3.384215in}{3.232226in}}%
\pgfpathcurveto{\pgfqpoint{3.384215in}{3.243276in}}{\pgfqpoint{3.379825in}{3.253875in}}{\pgfqpoint{3.372011in}{3.261689in}}%
\pgfpathcurveto{\pgfqpoint{3.364198in}{3.269503in}}{\pgfqpoint{3.353599in}{3.273893in}}{\pgfqpoint{3.342548in}{3.273893in}}%
\pgfpathcurveto{\pgfqpoint{3.331498in}{3.273893in}}{\pgfqpoint{3.320899in}{3.269503in}}{\pgfqpoint{3.313086in}{3.261689in}}%
\pgfpathcurveto{\pgfqpoint{3.305272in}{3.253875in}}{\pgfqpoint{3.300882in}{3.243276in}}{\pgfqpoint{3.300882in}{3.232226in}}%
\pgfpathcurveto{\pgfqpoint{3.300882in}{3.221176in}}{\pgfqpoint{3.305272in}{3.210577in}}{\pgfqpoint{3.313086in}{3.202763in}}%
\pgfpathcurveto{\pgfqpoint{3.320899in}{3.194950in}}{\pgfqpoint{3.331498in}{3.190560in}}{\pgfqpoint{3.342548in}{3.190560in}}%
\pgfpathclose%
\pgfusepath{stroke,fill}%
\end{pgfscope}%
\begin{pgfscope}%
\pgfpathrectangle{\pgfqpoint{0.648703in}{0.548769in}}{\pgfqpoint{5.112893in}{3.102590in}}%
\pgfusepath{clip}%
\pgfsetbuttcap%
\pgfsetroundjoin%
\definecolor{currentfill}{rgb}{1.000000,0.498039,0.054902}%
\pgfsetfillcolor{currentfill}%
\pgfsetlinewidth{1.003750pt}%
\definecolor{currentstroke}{rgb}{1.000000,0.498039,0.054902}%
\pgfsetstrokecolor{currentstroke}%
\pgfsetdash{}{0pt}%
\pgfpathmoveto{\pgfqpoint{2.310071in}{3.215842in}}%
\pgfpathcurveto{\pgfqpoint{2.321121in}{3.215842in}}{\pgfqpoint{2.331720in}{3.220232in}}{\pgfqpoint{2.339534in}{3.228046in}}%
\pgfpathcurveto{\pgfqpoint{2.347347in}{3.235859in}}{\pgfqpoint{2.351737in}{3.246458in}}{\pgfqpoint{2.351737in}{3.257508in}}%
\pgfpathcurveto{\pgfqpoint{2.351737in}{3.268559in}}{\pgfqpoint{2.347347in}{3.279158in}}{\pgfqpoint{2.339534in}{3.286971in}}%
\pgfpathcurveto{\pgfqpoint{2.331720in}{3.294785in}}{\pgfqpoint{2.321121in}{3.299175in}}{\pgfqpoint{2.310071in}{3.299175in}}%
\pgfpathcurveto{\pgfqpoint{2.299021in}{3.299175in}}{\pgfqpoint{2.288422in}{3.294785in}}{\pgfqpoint{2.280608in}{3.286971in}}%
\pgfpathcurveto{\pgfqpoint{2.272794in}{3.279158in}}{\pgfqpoint{2.268404in}{3.268559in}}{\pgfqpoint{2.268404in}{3.257508in}}%
\pgfpathcurveto{\pgfqpoint{2.268404in}{3.246458in}}{\pgfqpoint{2.272794in}{3.235859in}}{\pgfqpoint{2.280608in}{3.228046in}}%
\pgfpathcurveto{\pgfqpoint{2.288422in}{3.220232in}}{\pgfqpoint{2.299021in}{3.215842in}}{\pgfqpoint{2.310071in}{3.215842in}}%
\pgfpathclose%
\pgfusepath{stroke,fill}%
\end{pgfscope}%
\begin{pgfscope}%
\pgfpathrectangle{\pgfqpoint{0.648703in}{0.548769in}}{\pgfqpoint{5.112893in}{3.102590in}}%
\pgfusepath{clip}%
\pgfsetbuttcap%
\pgfsetroundjoin%
\definecolor{currentfill}{rgb}{0.121569,0.466667,0.705882}%
\pgfsetfillcolor{currentfill}%
\pgfsetlinewidth{1.003750pt}%
\definecolor{currentstroke}{rgb}{0.121569,0.466667,0.705882}%
\pgfsetstrokecolor{currentstroke}%
\pgfsetdash{}{0pt}%
\pgfpathmoveto{\pgfqpoint{0.831239in}{0.658113in}}%
\pgfpathcurveto{\pgfqpoint{0.842289in}{0.658113in}}{\pgfqpoint{0.852889in}{0.662503in}}{\pgfqpoint{0.860702in}{0.670317in}}%
\pgfpathcurveto{\pgfqpoint{0.868516in}{0.678130in}}{\pgfqpoint{0.872906in}{0.688729in}}{\pgfqpoint{0.872906in}{0.699779in}}%
\pgfpathcurveto{\pgfqpoint{0.872906in}{0.710830in}}{\pgfqpoint{0.868516in}{0.721429in}}{\pgfqpoint{0.860702in}{0.729242in}}%
\pgfpathcurveto{\pgfqpoint{0.852889in}{0.737056in}}{\pgfqpoint{0.842289in}{0.741446in}}{\pgfqpoint{0.831239in}{0.741446in}}%
\pgfpathcurveto{\pgfqpoint{0.820189in}{0.741446in}}{\pgfqpoint{0.809590in}{0.737056in}}{\pgfqpoint{0.801777in}{0.729242in}}%
\pgfpathcurveto{\pgfqpoint{0.793963in}{0.721429in}}{\pgfqpoint{0.789573in}{0.710830in}}{\pgfqpoint{0.789573in}{0.699779in}}%
\pgfpathcurveto{\pgfqpoint{0.789573in}{0.688729in}}{\pgfqpoint{0.793963in}{0.678130in}}{\pgfqpoint{0.801777in}{0.670317in}}%
\pgfpathcurveto{\pgfqpoint{0.809590in}{0.662503in}}{\pgfqpoint{0.820189in}{0.658113in}}{\pgfqpoint{0.831239in}{0.658113in}}%
\pgfpathclose%
\pgfusepath{stroke,fill}%
\end{pgfscope}%
\begin{pgfscope}%
\pgfpathrectangle{\pgfqpoint{0.648703in}{0.548769in}}{\pgfqpoint{5.112893in}{3.102590in}}%
\pgfusepath{clip}%
\pgfsetbuttcap%
\pgfsetroundjoin%
\definecolor{currentfill}{rgb}{0.121569,0.466667,0.705882}%
\pgfsetfillcolor{currentfill}%
\pgfsetlinewidth{1.003750pt}%
\definecolor{currentstroke}{rgb}{0.121569,0.466667,0.705882}%
\pgfsetstrokecolor{currentstroke}%
\pgfsetdash{}{0pt}%
\pgfpathmoveto{\pgfqpoint{1.086235in}{0.767669in}}%
\pgfpathcurveto{\pgfqpoint{1.097285in}{0.767669in}}{\pgfqpoint{1.107884in}{0.772060in}}{\pgfqpoint{1.115698in}{0.779873in}}%
\pgfpathcurveto{\pgfqpoint{1.123511in}{0.787687in}}{\pgfqpoint{1.127901in}{0.798286in}}{\pgfqpoint{1.127901in}{0.809336in}}%
\pgfpathcurveto{\pgfqpoint{1.127901in}{0.820386in}}{\pgfqpoint{1.123511in}{0.830985in}}{\pgfqpoint{1.115698in}{0.838799in}}%
\pgfpathcurveto{\pgfqpoint{1.107884in}{0.846613in}}{\pgfqpoint{1.097285in}{0.851003in}}{\pgfqpoint{1.086235in}{0.851003in}}%
\pgfpathcurveto{\pgfqpoint{1.075185in}{0.851003in}}{\pgfqpoint{1.064586in}{0.846613in}}{\pgfqpoint{1.056772in}{0.838799in}}%
\pgfpathcurveto{\pgfqpoint{1.048958in}{0.830985in}}{\pgfqpoint{1.044568in}{0.820386in}}{\pgfqpoint{1.044568in}{0.809336in}}%
\pgfpathcurveto{\pgfqpoint{1.044568in}{0.798286in}}{\pgfqpoint{1.048958in}{0.787687in}}{\pgfqpoint{1.056772in}{0.779873in}}%
\pgfpathcurveto{\pgfqpoint{1.064586in}{0.772060in}}{\pgfqpoint{1.075185in}{0.767669in}}{\pgfqpoint{1.086235in}{0.767669in}}%
\pgfpathclose%
\pgfusepath{stroke,fill}%
\end{pgfscope}%
\begin{pgfscope}%
\pgfpathrectangle{\pgfqpoint{0.648703in}{0.548769in}}{\pgfqpoint{5.112893in}{3.102590in}}%
\pgfusepath{clip}%
\pgfsetbuttcap%
\pgfsetroundjoin%
\definecolor{currentfill}{rgb}{1.000000,0.498039,0.054902}%
\pgfsetfillcolor{currentfill}%
\pgfsetlinewidth{1.003750pt}%
\definecolor{currentstroke}{rgb}{1.000000,0.498039,0.054902}%
\pgfsetstrokecolor{currentstroke}%
\pgfsetdash{}{0pt}%
\pgfpathmoveto{\pgfqpoint{3.569139in}{3.194773in}}%
\pgfpathcurveto{\pgfqpoint{3.580189in}{3.194773in}}{\pgfqpoint{3.590788in}{3.199163in}}{\pgfqpoint{3.598601in}{3.206977in}}%
\pgfpathcurveto{\pgfqpoint{3.606415in}{3.214791in}}{\pgfqpoint{3.610805in}{3.225390in}}{\pgfqpoint{3.610805in}{3.236440in}}%
\pgfpathcurveto{\pgfqpoint{3.610805in}{3.247490in}}{\pgfqpoint{3.606415in}{3.258089in}}{\pgfqpoint{3.598601in}{3.265903in}}%
\pgfpathcurveto{\pgfqpoint{3.590788in}{3.273716in}}{\pgfqpoint{3.580189in}{3.278107in}}{\pgfqpoint{3.569139in}{3.278107in}}%
\pgfpathcurveto{\pgfqpoint{3.558088in}{3.278107in}}{\pgfqpoint{3.547489in}{3.273716in}}{\pgfqpoint{3.539676in}{3.265903in}}%
\pgfpathcurveto{\pgfqpoint{3.531862in}{3.258089in}}{\pgfqpoint{3.527472in}{3.247490in}}{\pgfqpoint{3.527472in}{3.236440in}}%
\pgfpathcurveto{\pgfqpoint{3.527472in}{3.225390in}}{\pgfqpoint{3.531862in}{3.214791in}}{\pgfqpoint{3.539676in}{3.206977in}}%
\pgfpathcurveto{\pgfqpoint{3.547489in}{3.199163in}}{\pgfqpoint{3.558088in}{3.194773in}}{\pgfqpoint{3.569139in}{3.194773in}}%
\pgfpathclose%
\pgfusepath{stroke,fill}%
\end{pgfscope}%
\begin{pgfscope}%
\pgfpathrectangle{\pgfqpoint{0.648703in}{0.548769in}}{\pgfqpoint{5.112893in}{3.102590in}}%
\pgfusepath{clip}%
\pgfsetbuttcap%
\pgfsetroundjoin%
\definecolor{currentfill}{rgb}{0.121569,0.466667,0.705882}%
\pgfsetfillcolor{currentfill}%
\pgfsetlinewidth{1.003750pt}%
\definecolor{currentstroke}{rgb}{0.121569,0.466667,0.705882}%
\pgfsetstrokecolor{currentstroke}%
\pgfsetdash{}{0pt}%
\pgfpathmoveto{\pgfqpoint{0.831231in}{0.658113in}}%
\pgfpathcurveto{\pgfqpoint{0.842281in}{0.658113in}}{\pgfqpoint{0.852880in}{0.662503in}}{\pgfqpoint{0.860694in}{0.670317in}}%
\pgfpathcurveto{\pgfqpoint{0.868507in}{0.678130in}}{\pgfqpoint{0.872898in}{0.688729in}}{\pgfqpoint{0.872898in}{0.699779in}}%
\pgfpathcurveto{\pgfqpoint{0.872898in}{0.710830in}}{\pgfqpoint{0.868507in}{0.721429in}}{\pgfqpoint{0.860694in}{0.729242in}}%
\pgfpathcurveto{\pgfqpoint{0.852880in}{0.737056in}}{\pgfqpoint{0.842281in}{0.741446in}}{\pgfqpoint{0.831231in}{0.741446in}}%
\pgfpathcurveto{\pgfqpoint{0.820181in}{0.741446in}}{\pgfqpoint{0.809582in}{0.737056in}}{\pgfqpoint{0.801768in}{0.729242in}}%
\pgfpathcurveto{\pgfqpoint{0.793955in}{0.721429in}}{\pgfqpoint{0.789564in}{0.710830in}}{\pgfqpoint{0.789564in}{0.699779in}}%
\pgfpathcurveto{\pgfqpoint{0.789564in}{0.688729in}}{\pgfqpoint{0.793955in}{0.678130in}}{\pgfqpoint{0.801768in}{0.670317in}}%
\pgfpathcurveto{\pgfqpoint{0.809582in}{0.662503in}}{\pgfqpoint{0.820181in}{0.658113in}}{\pgfqpoint{0.831231in}{0.658113in}}%
\pgfpathclose%
\pgfusepath{stroke,fill}%
\end{pgfscope}%
\begin{pgfscope}%
\pgfpathrectangle{\pgfqpoint{0.648703in}{0.548769in}}{\pgfqpoint{5.112893in}{3.102590in}}%
\pgfusepath{clip}%
\pgfsetbuttcap%
\pgfsetroundjoin%
\definecolor{currentfill}{rgb}{1.000000,0.498039,0.054902}%
\pgfsetfillcolor{currentfill}%
\pgfsetlinewidth{1.003750pt}%
\definecolor{currentstroke}{rgb}{1.000000,0.498039,0.054902}%
\pgfsetstrokecolor{currentstroke}%
\pgfsetdash{}{0pt}%
\pgfpathmoveto{\pgfqpoint{2.512303in}{3.198987in}}%
\pgfpathcurveto{\pgfqpoint{2.523354in}{3.198987in}}{\pgfqpoint{2.533953in}{3.203377in}}{\pgfqpoint{2.541766in}{3.211191in}}%
\pgfpathcurveto{\pgfqpoint{2.549580in}{3.219004in}}{\pgfqpoint{2.553970in}{3.229603in}}{\pgfqpoint{2.553970in}{3.240654in}}%
\pgfpathcurveto{\pgfqpoint{2.553970in}{3.251704in}}{\pgfqpoint{2.549580in}{3.262303in}}{\pgfqpoint{2.541766in}{3.270116in}}%
\pgfpathcurveto{\pgfqpoint{2.533953in}{3.277930in}}{\pgfqpoint{2.523354in}{3.282320in}}{\pgfqpoint{2.512303in}{3.282320in}}%
\pgfpathcurveto{\pgfqpoint{2.501253in}{3.282320in}}{\pgfqpoint{2.490654in}{3.277930in}}{\pgfqpoint{2.482841in}{3.270116in}}%
\pgfpathcurveto{\pgfqpoint{2.475027in}{3.262303in}}{\pgfqpoint{2.470637in}{3.251704in}}{\pgfqpoint{2.470637in}{3.240654in}}%
\pgfpathcurveto{\pgfqpoint{2.470637in}{3.229603in}}{\pgfqpoint{2.475027in}{3.219004in}}{\pgfqpoint{2.482841in}{3.211191in}}%
\pgfpathcurveto{\pgfqpoint{2.490654in}{3.203377in}}{\pgfqpoint{2.501253in}{3.198987in}}{\pgfqpoint{2.512303in}{3.198987in}}%
\pgfpathclose%
\pgfusepath{stroke,fill}%
\end{pgfscope}%
\begin{pgfscope}%
\pgfpathrectangle{\pgfqpoint{0.648703in}{0.548769in}}{\pgfqpoint{5.112893in}{3.102590in}}%
\pgfusepath{clip}%
\pgfsetbuttcap%
\pgfsetroundjoin%
\definecolor{currentfill}{rgb}{1.000000,0.498039,0.054902}%
\pgfsetfillcolor{currentfill}%
\pgfsetlinewidth{1.003750pt}%
\definecolor{currentstroke}{rgb}{1.000000,0.498039,0.054902}%
\pgfsetstrokecolor{currentstroke}%
\pgfsetdash{}{0pt}%
\pgfpathmoveto{\pgfqpoint{3.569133in}{3.194773in}}%
\pgfpathcurveto{\pgfqpoint{3.580183in}{3.194773in}}{\pgfqpoint{3.590782in}{3.199163in}}{\pgfqpoint{3.598596in}{3.206977in}}%
\pgfpathcurveto{\pgfqpoint{3.606410in}{3.214791in}}{\pgfqpoint{3.610800in}{3.225390in}}{\pgfqpoint{3.610800in}{3.236440in}}%
\pgfpathcurveto{\pgfqpoint{3.610800in}{3.247490in}}{\pgfqpoint{3.606410in}{3.258089in}}{\pgfqpoint{3.598596in}{3.265903in}}%
\pgfpathcurveto{\pgfqpoint{3.590782in}{3.273716in}}{\pgfqpoint{3.580183in}{3.278107in}}{\pgfqpoint{3.569133in}{3.278107in}}%
\pgfpathcurveto{\pgfqpoint{3.558083in}{3.278107in}}{\pgfqpoint{3.547484in}{3.273716in}}{\pgfqpoint{3.539670in}{3.265903in}}%
\pgfpathcurveto{\pgfqpoint{3.531857in}{3.258089in}}{\pgfqpoint{3.527467in}{3.247490in}}{\pgfqpoint{3.527467in}{3.236440in}}%
\pgfpathcurveto{\pgfqpoint{3.527467in}{3.225390in}}{\pgfqpoint{3.531857in}{3.214791in}}{\pgfqpoint{3.539670in}{3.206977in}}%
\pgfpathcurveto{\pgfqpoint{3.547484in}{3.199163in}}{\pgfqpoint{3.558083in}{3.194773in}}{\pgfqpoint{3.569133in}{3.194773in}}%
\pgfpathclose%
\pgfusepath{stroke,fill}%
\end{pgfscope}%
\begin{pgfscope}%
\pgfpathrectangle{\pgfqpoint{0.648703in}{0.548769in}}{\pgfqpoint{5.112893in}{3.102590in}}%
\pgfusepath{clip}%
\pgfsetbuttcap%
\pgfsetroundjoin%
\definecolor{currentfill}{rgb}{0.121569,0.466667,0.705882}%
\pgfsetfillcolor{currentfill}%
\pgfsetlinewidth{1.003750pt}%
\definecolor{currentstroke}{rgb}{0.121569,0.466667,0.705882}%
\pgfsetstrokecolor{currentstroke}%
\pgfsetdash{}{0pt}%
\pgfpathmoveto{\pgfqpoint{0.835725in}{0.662326in}}%
\pgfpathcurveto{\pgfqpoint{0.846775in}{0.662326in}}{\pgfqpoint{0.857374in}{0.666717in}}{\pgfqpoint{0.865187in}{0.674530in}}%
\pgfpathcurveto{\pgfqpoint{0.873001in}{0.682344in}}{\pgfqpoint{0.877391in}{0.692943in}}{\pgfqpoint{0.877391in}{0.703993in}}%
\pgfpathcurveto{\pgfqpoint{0.877391in}{0.715043in}}{\pgfqpoint{0.873001in}{0.725642in}}{\pgfqpoint{0.865187in}{0.733456in}}%
\pgfpathcurveto{\pgfqpoint{0.857374in}{0.741270in}}{\pgfqpoint{0.846775in}{0.745660in}}{\pgfqpoint{0.835725in}{0.745660in}}%
\pgfpathcurveto{\pgfqpoint{0.824674in}{0.745660in}}{\pgfqpoint{0.814075in}{0.741270in}}{\pgfqpoint{0.806262in}{0.733456in}}%
\pgfpathcurveto{\pgfqpoint{0.798448in}{0.725642in}}{\pgfqpoint{0.794058in}{0.715043in}}{\pgfqpoint{0.794058in}{0.703993in}}%
\pgfpathcurveto{\pgfqpoint{0.794058in}{0.692943in}}{\pgfqpoint{0.798448in}{0.682344in}}{\pgfqpoint{0.806262in}{0.674530in}}%
\pgfpathcurveto{\pgfqpoint{0.814075in}{0.666717in}}{\pgfqpoint{0.824674in}{0.662326in}}{\pgfqpoint{0.835725in}{0.662326in}}%
\pgfpathclose%
\pgfusepath{stroke,fill}%
\end{pgfscope}%
\begin{pgfscope}%
\pgfpathrectangle{\pgfqpoint{0.648703in}{0.548769in}}{\pgfqpoint{5.112893in}{3.102590in}}%
\pgfusepath{clip}%
\pgfsetbuttcap%
\pgfsetroundjoin%
\definecolor{currentfill}{rgb}{0.121569,0.466667,0.705882}%
\pgfsetfillcolor{currentfill}%
\pgfsetlinewidth{1.003750pt}%
\definecolor{currentstroke}{rgb}{0.121569,0.466667,0.705882}%
\pgfsetstrokecolor{currentstroke}%
\pgfsetdash{}{0pt}%
\pgfpathmoveto{\pgfqpoint{0.831262in}{0.658113in}}%
\pgfpathcurveto{\pgfqpoint{0.842312in}{0.658113in}}{\pgfqpoint{0.852911in}{0.662503in}}{\pgfqpoint{0.860725in}{0.670317in}}%
\pgfpathcurveto{\pgfqpoint{0.868538in}{0.678130in}}{\pgfqpoint{0.872928in}{0.688729in}}{\pgfqpoint{0.872928in}{0.699779in}}%
\pgfpathcurveto{\pgfqpoint{0.872928in}{0.710830in}}{\pgfqpoint{0.868538in}{0.721429in}}{\pgfqpoint{0.860725in}{0.729242in}}%
\pgfpathcurveto{\pgfqpoint{0.852911in}{0.737056in}}{\pgfqpoint{0.842312in}{0.741446in}}{\pgfqpoint{0.831262in}{0.741446in}}%
\pgfpathcurveto{\pgfqpoint{0.820212in}{0.741446in}}{\pgfqpoint{0.809613in}{0.737056in}}{\pgfqpoint{0.801799in}{0.729242in}}%
\pgfpathcurveto{\pgfqpoint{0.793985in}{0.721429in}}{\pgfqpoint{0.789595in}{0.710830in}}{\pgfqpoint{0.789595in}{0.699779in}}%
\pgfpathcurveto{\pgfqpoint{0.789595in}{0.688729in}}{\pgfqpoint{0.793985in}{0.678130in}}{\pgfqpoint{0.801799in}{0.670317in}}%
\pgfpathcurveto{\pgfqpoint{0.809613in}{0.662503in}}{\pgfqpoint{0.820212in}{0.658113in}}{\pgfqpoint{0.831262in}{0.658113in}}%
\pgfpathclose%
\pgfusepath{stroke,fill}%
\end{pgfscope}%
\begin{pgfscope}%
\pgfpathrectangle{\pgfqpoint{0.648703in}{0.548769in}}{\pgfqpoint{5.112893in}{3.102590in}}%
\pgfusepath{clip}%
\pgfsetbuttcap%
\pgfsetroundjoin%
\definecolor{currentfill}{rgb}{0.121569,0.466667,0.705882}%
\pgfsetfillcolor{currentfill}%
\pgfsetlinewidth{1.003750pt}%
\definecolor{currentstroke}{rgb}{0.121569,0.466667,0.705882}%
\pgfsetstrokecolor{currentstroke}%
\pgfsetdash{}{0pt}%
\pgfpathmoveto{\pgfqpoint{0.831251in}{0.658113in}}%
\pgfpathcurveto{\pgfqpoint{0.842301in}{0.658113in}}{\pgfqpoint{0.852900in}{0.662503in}}{\pgfqpoint{0.860713in}{0.670317in}}%
\pgfpathcurveto{\pgfqpoint{0.868527in}{0.678130in}}{\pgfqpoint{0.872917in}{0.688729in}}{\pgfqpoint{0.872917in}{0.699779in}}%
\pgfpathcurveto{\pgfqpoint{0.872917in}{0.710830in}}{\pgfqpoint{0.868527in}{0.721429in}}{\pgfqpoint{0.860713in}{0.729242in}}%
\pgfpathcurveto{\pgfqpoint{0.852900in}{0.737056in}}{\pgfqpoint{0.842301in}{0.741446in}}{\pgfqpoint{0.831251in}{0.741446in}}%
\pgfpathcurveto{\pgfqpoint{0.820200in}{0.741446in}}{\pgfqpoint{0.809601in}{0.737056in}}{\pgfqpoint{0.801788in}{0.729242in}}%
\pgfpathcurveto{\pgfqpoint{0.793974in}{0.721429in}}{\pgfqpoint{0.789584in}{0.710830in}}{\pgfqpoint{0.789584in}{0.699779in}}%
\pgfpathcurveto{\pgfqpoint{0.789584in}{0.688729in}}{\pgfqpoint{0.793974in}{0.678130in}}{\pgfqpoint{0.801788in}{0.670317in}}%
\pgfpathcurveto{\pgfqpoint{0.809601in}{0.662503in}}{\pgfqpoint{0.820200in}{0.658113in}}{\pgfqpoint{0.831251in}{0.658113in}}%
\pgfpathclose%
\pgfusepath{stroke,fill}%
\end{pgfscope}%
\begin{pgfscope}%
\pgfpathrectangle{\pgfqpoint{0.648703in}{0.548769in}}{\pgfqpoint{5.112893in}{3.102590in}}%
\pgfusepath{clip}%
\pgfsetbuttcap%
\pgfsetroundjoin%
\definecolor{currentfill}{rgb}{0.121569,0.466667,0.705882}%
\pgfsetfillcolor{currentfill}%
\pgfsetlinewidth{1.003750pt}%
\definecolor{currentstroke}{rgb}{0.121569,0.466667,0.705882}%
\pgfsetstrokecolor{currentstroke}%
\pgfsetdash{}{0pt}%
\pgfpathmoveto{\pgfqpoint{0.831648in}{0.658113in}}%
\pgfpathcurveto{\pgfqpoint{0.842698in}{0.658113in}}{\pgfqpoint{0.853297in}{0.662503in}}{\pgfqpoint{0.861111in}{0.670317in}}%
\pgfpathcurveto{\pgfqpoint{0.868924in}{0.678130in}}{\pgfqpoint{0.873314in}{0.688729in}}{\pgfqpoint{0.873314in}{0.699779in}}%
\pgfpathcurveto{\pgfqpoint{0.873314in}{0.710830in}}{\pgfqpoint{0.868924in}{0.721429in}}{\pgfqpoint{0.861111in}{0.729242in}}%
\pgfpathcurveto{\pgfqpoint{0.853297in}{0.737056in}}{\pgfqpoint{0.842698in}{0.741446in}}{\pgfqpoint{0.831648in}{0.741446in}}%
\pgfpathcurveto{\pgfqpoint{0.820598in}{0.741446in}}{\pgfqpoint{0.809999in}{0.737056in}}{\pgfqpoint{0.802185in}{0.729242in}}%
\pgfpathcurveto{\pgfqpoint{0.794371in}{0.721429in}}{\pgfqpoint{0.789981in}{0.710830in}}{\pgfqpoint{0.789981in}{0.699779in}}%
\pgfpathcurveto{\pgfqpoint{0.789981in}{0.688729in}}{\pgfqpoint{0.794371in}{0.678130in}}{\pgfqpoint{0.802185in}{0.670317in}}%
\pgfpathcurveto{\pgfqpoint{0.809999in}{0.662503in}}{\pgfqpoint{0.820598in}{0.658113in}}{\pgfqpoint{0.831648in}{0.658113in}}%
\pgfpathclose%
\pgfusepath{stroke,fill}%
\end{pgfscope}%
\begin{pgfscope}%
\pgfpathrectangle{\pgfqpoint{0.648703in}{0.548769in}}{\pgfqpoint{5.112893in}{3.102590in}}%
\pgfusepath{clip}%
\pgfsetbuttcap%
\pgfsetroundjoin%
\definecolor{currentfill}{rgb}{0.121569,0.466667,0.705882}%
\pgfsetfillcolor{currentfill}%
\pgfsetlinewidth{1.003750pt}%
\definecolor{currentstroke}{rgb}{0.121569,0.466667,0.705882}%
\pgfsetstrokecolor{currentstroke}%
\pgfsetdash{}{0pt}%
\pgfpathmoveto{\pgfqpoint{0.831267in}{0.658113in}}%
\pgfpathcurveto{\pgfqpoint{0.842317in}{0.658113in}}{\pgfqpoint{0.852916in}{0.662503in}}{\pgfqpoint{0.860730in}{0.670317in}}%
\pgfpathcurveto{\pgfqpoint{0.868543in}{0.678130in}}{\pgfqpoint{0.872933in}{0.688729in}}{\pgfqpoint{0.872933in}{0.699779in}}%
\pgfpathcurveto{\pgfqpoint{0.872933in}{0.710830in}}{\pgfqpoint{0.868543in}{0.721429in}}{\pgfqpoint{0.860730in}{0.729242in}}%
\pgfpathcurveto{\pgfqpoint{0.852916in}{0.737056in}}{\pgfqpoint{0.842317in}{0.741446in}}{\pgfqpoint{0.831267in}{0.741446in}}%
\pgfpathcurveto{\pgfqpoint{0.820217in}{0.741446in}}{\pgfqpoint{0.809618in}{0.737056in}}{\pgfqpoint{0.801804in}{0.729242in}}%
\pgfpathcurveto{\pgfqpoint{0.793990in}{0.721429in}}{\pgfqpoint{0.789600in}{0.710830in}}{\pgfqpoint{0.789600in}{0.699779in}}%
\pgfpathcurveto{\pgfqpoint{0.789600in}{0.688729in}}{\pgfqpoint{0.793990in}{0.678130in}}{\pgfqpoint{0.801804in}{0.670317in}}%
\pgfpathcurveto{\pgfqpoint{0.809618in}{0.662503in}}{\pgfqpoint{0.820217in}{0.658113in}}{\pgfqpoint{0.831267in}{0.658113in}}%
\pgfpathclose%
\pgfusepath{stroke,fill}%
\end{pgfscope}%
\begin{pgfscope}%
\pgfpathrectangle{\pgfqpoint{0.648703in}{0.548769in}}{\pgfqpoint{5.112893in}{3.102590in}}%
\pgfusepath{clip}%
\pgfsetbuttcap%
\pgfsetroundjoin%
\definecolor{currentfill}{rgb}{0.121569,0.466667,0.705882}%
\pgfsetfillcolor{currentfill}%
\pgfsetlinewidth{1.003750pt}%
\definecolor{currentstroke}{rgb}{0.121569,0.466667,0.705882}%
\pgfsetstrokecolor{currentstroke}%
\pgfsetdash{}{0pt}%
\pgfpathmoveto{\pgfqpoint{0.831244in}{0.658113in}}%
\pgfpathcurveto{\pgfqpoint{0.842294in}{0.658113in}}{\pgfqpoint{0.852893in}{0.662503in}}{\pgfqpoint{0.860707in}{0.670317in}}%
\pgfpathcurveto{\pgfqpoint{0.868520in}{0.678130in}}{\pgfqpoint{0.872911in}{0.688729in}}{\pgfqpoint{0.872911in}{0.699779in}}%
\pgfpathcurveto{\pgfqpoint{0.872911in}{0.710830in}}{\pgfqpoint{0.868520in}{0.721429in}}{\pgfqpoint{0.860707in}{0.729242in}}%
\pgfpathcurveto{\pgfqpoint{0.852893in}{0.737056in}}{\pgfqpoint{0.842294in}{0.741446in}}{\pgfqpoint{0.831244in}{0.741446in}}%
\pgfpathcurveto{\pgfqpoint{0.820194in}{0.741446in}}{\pgfqpoint{0.809595in}{0.737056in}}{\pgfqpoint{0.801781in}{0.729242in}}%
\pgfpathcurveto{\pgfqpoint{0.793968in}{0.721429in}}{\pgfqpoint{0.789577in}{0.710830in}}{\pgfqpoint{0.789577in}{0.699779in}}%
\pgfpathcurveto{\pgfqpoint{0.789577in}{0.688729in}}{\pgfqpoint{0.793968in}{0.678130in}}{\pgfqpoint{0.801781in}{0.670317in}}%
\pgfpathcurveto{\pgfqpoint{0.809595in}{0.662503in}}{\pgfqpoint{0.820194in}{0.658113in}}{\pgfqpoint{0.831244in}{0.658113in}}%
\pgfpathclose%
\pgfusepath{stroke,fill}%
\end{pgfscope}%
\begin{pgfscope}%
\pgfpathrectangle{\pgfqpoint{0.648703in}{0.548769in}}{\pgfqpoint{5.112893in}{3.102590in}}%
\pgfusepath{clip}%
\pgfsetbuttcap%
\pgfsetroundjoin%
\definecolor{currentfill}{rgb}{1.000000,0.498039,0.054902}%
\pgfsetfillcolor{currentfill}%
\pgfsetlinewidth{1.003750pt}%
\definecolor{currentstroke}{rgb}{1.000000,0.498039,0.054902}%
\pgfsetstrokecolor{currentstroke}%
\pgfsetdash{}{0pt}%
\pgfpathmoveto{\pgfqpoint{3.849608in}{3.182132in}}%
\pgfpathcurveto{\pgfqpoint{3.860658in}{3.182132in}}{\pgfqpoint{3.871257in}{3.186522in}}{\pgfqpoint{3.879070in}{3.194336in}}%
\pgfpathcurveto{\pgfqpoint{3.886884in}{3.202150in}}{\pgfqpoint{3.891274in}{3.212749in}}{\pgfqpoint{3.891274in}{3.223799in}}%
\pgfpathcurveto{\pgfqpoint{3.891274in}{3.234849in}}{\pgfqpoint{3.886884in}{3.245448in}}{\pgfqpoint{3.879070in}{3.253262in}}%
\pgfpathcurveto{\pgfqpoint{3.871257in}{3.261075in}}{\pgfqpoint{3.860658in}{3.265465in}}{\pgfqpoint{3.849608in}{3.265465in}}%
\pgfpathcurveto{\pgfqpoint{3.838557in}{3.265465in}}{\pgfqpoint{3.827958in}{3.261075in}}{\pgfqpoint{3.820145in}{3.253262in}}%
\pgfpathcurveto{\pgfqpoint{3.812331in}{3.245448in}}{\pgfqpoint{3.807941in}{3.234849in}}{\pgfqpoint{3.807941in}{3.223799in}}%
\pgfpathcurveto{\pgfqpoint{3.807941in}{3.212749in}}{\pgfqpoint{3.812331in}{3.202150in}}{\pgfqpoint{3.820145in}{3.194336in}}%
\pgfpathcurveto{\pgfqpoint{3.827958in}{3.186522in}}{\pgfqpoint{3.838557in}{3.182132in}}{\pgfqpoint{3.849608in}{3.182132in}}%
\pgfpathclose%
\pgfusepath{stroke,fill}%
\end{pgfscope}%
\begin{pgfscope}%
\pgfpathrectangle{\pgfqpoint{0.648703in}{0.548769in}}{\pgfqpoint{5.112893in}{3.102590in}}%
\pgfusepath{clip}%
\pgfsetbuttcap%
\pgfsetroundjoin%
\definecolor{currentfill}{rgb}{1.000000,0.498039,0.054902}%
\pgfsetfillcolor{currentfill}%
\pgfsetlinewidth{1.003750pt}%
\definecolor{currentstroke}{rgb}{1.000000,0.498039,0.054902}%
\pgfsetstrokecolor{currentstroke}%
\pgfsetdash{}{0pt}%
\pgfpathmoveto{\pgfqpoint{3.375021in}{3.194773in}}%
\pgfpathcurveto{\pgfqpoint{3.386071in}{3.194773in}}{\pgfqpoint{3.396670in}{3.199163in}}{\pgfqpoint{3.404483in}{3.206977in}}%
\pgfpathcurveto{\pgfqpoint{3.412297in}{3.214791in}}{\pgfqpoint{3.416687in}{3.225390in}}{\pgfqpoint{3.416687in}{3.236440in}}%
\pgfpathcurveto{\pgfqpoint{3.416687in}{3.247490in}}{\pgfqpoint{3.412297in}{3.258089in}}{\pgfqpoint{3.404483in}{3.265903in}}%
\pgfpathcurveto{\pgfqpoint{3.396670in}{3.273716in}}{\pgfqpoint{3.386071in}{3.278107in}}{\pgfqpoint{3.375021in}{3.278107in}}%
\pgfpathcurveto{\pgfqpoint{3.363971in}{3.278107in}}{\pgfqpoint{3.353371in}{3.273716in}}{\pgfqpoint{3.345558in}{3.265903in}}%
\pgfpathcurveto{\pgfqpoint{3.337744in}{3.258089in}}{\pgfqpoint{3.333354in}{3.247490in}}{\pgfqpoint{3.333354in}{3.236440in}}%
\pgfpathcurveto{\pgfqpoint{3.333354in}{3.225390in}}{\pgfqpoint{3.337744in}{3.214791in}}{\pgfqpoint{3.345558in}{3.206977in}}%
\pgfpathcurveto{\pgfqpoint{3.353371in}{3.199163in}}{\pgfqpoint{3.363971in}{3.194773in}}{\pgfqpoint{3.375021in}{3.194773in}}%
\pgfpathclose%
\pgfusepath{stroke,fill}%
\end{pgfscope}%
\begin{pgfscope}%
\pgfpathrectangle{\pgfqpoint{0.648703in}{0.548769in}}{\pgfqpoint{5.112893in}{3.102590in}}%
\pgfusepath{clip}%
\pgfsetbuttcap%
\pgfsetroundjoin%
\definecolor{currentfill}{rgb}{1.000000,0.498039,0.054902}%
\pgfsetfillcolor{currentfill}%
\pgfsetlinewidth{1.003750pt}%
\definecolor{currentstroke}{rgb}{1.000000,0.498039,0.054902}%
\pgfsetstrokecolor{currentstroke}%
\pgfsetdash{}{0pt}%
\pgfpathmoveto{\pgfqpoint{3.300479in}{3.194773in}}%
\pgfpathcurveto{\pgfqpoint{3.311529in}{3.194773in}}{\pgfqpoint{3.322128in}{3.199163in}}{\pgfqpoint{3.329942in}{3.206977in}}%
\pgfpathcurveto{\pgfqpoint{3.337755in}{3.214791in}}{\pgfqpoint{3.342146in}{3.225390in}}{\pgfqpoint{3.342146in}{3.236440in}}%
\pgfpathcurveto{\pgfqpoint{3.342146in}{3.247490in}}{\pgfqpoint{3.337755in}{3.258089in}}{\pgfqpoint{3.329942in}{3.265903in}}%
\pgfpathcurveto{\pgfqpoint{3.322128in}{3.273716in}}{\pgfqpoint{3.311529in}{3.278107in}}{\pgfqpoint{3.300479in}{3.278107in}}%
\pgfpathcurveto{\pgfqpoint{3.289429in}{3.278107in}}{\pgfqpoint{3.278830in}{3.273716in}}{\pgfqpoint{3.271016in}{3.265903in}}%
\pgfpathcurveto{\pgfqpoint{3.263202in}{3.258089in}}{\pgfqpoint{3.258812in}{3.247490in}}{\pgfqpoint{3.258812in}{3.236440in}}%
\pgfpathcurveto{\pgfqpoint{3.258812in}{3.225390in}}{\pgfqpoint{3.263202in}{3.214791in}}{\pgfqpoint{3.271016in}{3.206977in}}%
\pgfpathcurveto{\pgfqpoint{3.278830in}{3.199163in}}{\pgfqpoint{3.289429in}{3.194773in}}{\pgfqpoint{3.300479in}{3.194773in}}%
\pgfpathclose%
\pgfusepath{stroke,fill}%
\end{pgfscope}%
\begin{pgfscope}%
\pgfpathrectangle{\pgfqpoint{0.648703in}{0.548769in}}{\pgfqpoint{5.112893in}{3.102590in}}%
\pgfusepath{clip}%
\pgfsetbuttcap%
\pgfsetroundjoin%
\definecolor{currentfill}{rgb}{0.121569,0.466667,0.705882}%
\pgfsetfillcolor{currentfill}%
\pgfsetlinewidth{1.003750pt}%
\definecolor{currentstroke}{rgb}{0.121569,0.466667,0.705882}%
\pgfsetstrokecolor{currentstroke}%
\pgfsetdash{}{0pt}%
\pgfpathmoveto{\pgfqpoint{0.831231in}{0.658113in}}%
\pgfpathcurveto{\pgfqpoint{0.842281in}{0.658113in}}{\pgfqpoint{0.852880in}{0.662503in}}{\pgfqpoint{0.860694in}{0.670317in}}%
\pgfpathcurveto{\pgfqpoint{0.868508in}{0.678130in}}{\pgfqpoint{0.872898in}{0.688729in}}{\pgfqpoint{0.872898in}{0.699779in}}%
\pgfpathcurveto{\pgfqpoint{0.872898in}{0.710830in}}{\pgfqpoint{0.868508in}{0.721429in}}{\pgfqpoint{0.860694in}{0.729242in}}%
\pgfpathcurveto{\pgfqpoint{0.852880in}{0.737056in}}{\pgfqpoint{0.842281in}{0.741446in}}{\pgfqpoint{0.831231in}{0.741446in}}%
\pgfpathcurveto{\pgfqpoint{0.820181in}{0.741446in}}{\pgfqpoint{0.809582in}{0.737056in}}{\pgfqpoint{0.801768in}{0.729242in}}%
\pgfpathcurveto{\pgfqpoint{0.793955in}{0.721429in}}{\pgfqpoint{0.789565in}{0.710830in}}{\pgfqpoint{0.789565in}{0.699779in}}%
\pgfpathcurveto{\pgfqpoint{0.789565in}{0.688729in}}{\pgfqpoint{0.793955in}{0.678130in}}{\pgfqpoint{0.801768in}{0.670317in}}%
\pgfpathcurveto{\pgfqpoint{0.809582in}{0.662503in}}{\pgfqpoint{0.820181in}{0.658113in}}{\pgfqpoint{0.831231in}{0.658113in}}%
\pgfpathclose%
\pgfusepath{stroke,fill}%
\end{pgfscope}%
\begin{pgfscope}%
\pgfpathrectangle{\pgfqpoint{0.648703in}{0.548769in}}{\pgfqpoint{5.112893in}{3.102590in}}%
\pgfusepath{clip}%
\pgfsetbuttcap%
\pgfsetroundjoin%
\definecolor{currentfill}{rgb}{0.121569,0.466667,0.705882}%
\pgfsetfillcolor{currentfill}%
\pgfsetlinewidth{1.003750pt}%
\definecolor{currentstroke}{rgb}{0.121569,0.466667,0.705882}%
\pgfsetstrokecolor{currentstroke}%
\pgfsetdash{}{0pt}%
\pgfpathmoveto{\pgfqpoint{0.866245in}{0.674968in}}%
\pgfpathcurveto{\pgfqpoint{0.877295in}{0.674968in}}{\pgfqpoint{0.887894in}{0.679358in}}{\pgfqpoint{0.895708in}{0.687172in}}%
\pgfpathcurveto{\pgfqpoint{0.903521in}{0.694985in}}{\pgfqpoint{0.907912in}{0.705584in}}{\pgfqpoint{0.907912in}{0.716634in}}%
\pgfpathcurveto{\pgfqpoint{0.907912in}{0.727684in}}{\pgfqpoint{0.903521in}{0.738283in}}{\pgfqpoint{0.895708in}{0.746097in}}%
\pgfpathcurveto{\pgfqpoint{0.887894in}{0.753911in}}{\pgfqpoint{0.877295in}{0.758301in}}{\pgfqpoint{0.866245in}{0.758301in}}%
\pgfpathcurveto{\pgfqpoint{0.855195in}{0.758301in}}{\pgfqpoint{0.844596in}{0.753911in}}{\pgfqpoint{0.836782in}{0.746097in}}%
\pgfpathcurveto{\pgfqpoint{0.828969in}{0.738283in}}{\pgfqpoint{0.824578in}{0.727684in}}{\pgfqpoint{0.824578in}{0.716634in}}%
\pgfpathcurveto{\pgfqpoint{0.824578in}{0.705584in}}{\pgfqpoint{0.828969in}{0.694985in}}{\pgfqpoint{0.836782in}{0.687172in}}%
\pgfpathcurveto{\pgfqpoint{0.844596in}{0.679358in}}{\pgfqpoint{0.855195in}{0.674968in}}{\pgfqpoint{0.866245in}{0.674968in}}%
\pgfpathclose%
\pgfusepath{stroke,fill}%
\end{pgfscope}%
\begin{pgfscope}%
\pgfpathrectangle{\pgfqpoint{0.648703in}{0.548769in}}{\pgfqpoint{5.112893in}{3.102590in}}%
\pgfusepath{clip}%
\pgfsetbuttcap%
\pgfsetroundjoin%
\definecolor{currentfill}{rgb}{1.000000,0.498039,0.054902}%
\pgfsetfillcolor{currentfill}%
\pgfsetlinewidth{1.003750pt}%
\definecolor{currentstroke}{rgb}{1.000000,0.498039,0.054902}%
\pgfsetstrokecolor{currentstroke}%
\pgfsetdash{}{0pt}%
\pgfpathmoveto{\pgfqpoint{3.518326in}{3.190560in}}%
\pgfpathcurveto{\pgfqpoint{3.529376in}{3.190560in}}{\pgfqpoint{3.539975in}{3.194950in}}{\pgfqpoint{3.547789in}{3.202763in}}%
\pgfpathcurveto{\pgfqpoint{3.555602in}{3.210577in}}{\pgfqpoint{3.559993in}{3.221176in}}{\pgfqpoint{3.559993in}{3.232226in}}%
\pgfpathcurveto{\pgfqpoint{3.559993in}{3.243276in}}{\pgfqpoint{3.555602in}{3.253875in}}{\pgfqpoint{3.547789in}{3.261689in}}%
\pgfpathcurveto{\pgfqpoint{3.539975in}{3.269503in}}{\pgfqpoint{3.529376in}{3.273893in}}{\pgfqpoint{3.518326in}{3.273893in}}%
\pgfpathcurveto{\pgfqpoint{3.507276in}{3.273893in}}{\pgfqpoint{3.496677in}{3.269503in}}{\pgfqpoint{3.488863in}{3.261689in}}%
\pgfpathcurveto{\pgfqpoint{3.481049in}{3.253875in}}{\pgfqpoint{3.476659in}{3.243276in}}{\pgfqpoint{3.476659in}{3.232226in}}%
\pgfpathcurveto{\pgfqpoint{3.476659in}{3.221176in}}{\pgfqpoint{3.481049in}{3.210577in}}{\pgfqpoint{3.488863in}{3.202763in}}%
\pgfpathcurveto{\pgfqpoint{3.496677in}{3.194950in}}{\pgfqpoint{3.507276in}{3.190560in}}{\pgfqpoint{3.518326in}{3.190560in}}%
\pgfpathclose%
\pgfusepath{stroke,fill}%
\end{pgfscope}%
\begin{pgfscope}%
\pgfpathrectangle{\pgfqpoint{0.648703in}{0.548769in}}{\pgfqpoint{5.112893in}{3.102590in}}%
\pgfusepath{clip}%
\pgfsetbuttcap%
\pgfsetroundjoin%
\definecolor{currentfill}{rgb}{0.121569,0.466667,0.705882}%
\pgfsetfillcolor{currentfill}%
\pgfsetlinewidth{1.003750pt}%
\definecolor{currentstroke}{rgb}{0.121569,0.466667,0.705882}%
\pgfsetstrokecolor{currentstroke}%
\pgfsetdash{}{0pt}%
\pgfpathmoveto{\pgfqpoint{1.087457in}{0.818234in}}%
\pgfpathcurveto{\pgfqpoint{1.098507in}{0.818234in}}{\pgfqpoint{1.109106in}{0.822624in}}{\pgfqpoint{1.116920in}{0.830438in}}%
\pgfpathcurveto{\pgfqpoint{1.124733in}{0.838252in}}{\pgfqpoint{1.129124in}{0.848851in}}{\pgfqpoint{1.129124in}{0.859901in}}%
\pgfpathcurveto{\pgfqpoint{1.129124in}{0.870951in}}{\pgfqpoint{1.124733in}{0.881550in}}{\pgfqpoint{1.116920in}{0.889364in}}%
\pgfpathcurveto{\pgfqpoint{1.109106in}{0.897177in}}{\pgfqpoint{1.098507in}{0.901567in}}{\pgfqpoint{1.087457in}{0.901567in}}%
\pgfpathcurveto{\pgfqpoint{1.076407in}{0.901567in}}{\pgfqpoint{1.065808in}{0.897177in}}{\pgfqpoint{1.057994in}{0.889364in}}%
\pgfpathcurveto{\pgfqpoint{1.050180in}{0.881550in}}{\pgfqpoint{1.045790in}{0.870951in}}{\pgfqpoint{1.045790in}{0.859901in}}%
\pgfpathcurveto{\pgfqpoint{1.045790in}{0.848851in}}{\pgfqpoint{1.050180in}{0.838252in}}{\pgfqpoint{1.057994in}{0.830438in}}%
\pgfpathcurveto{\pgfqpoint{1.065808in}{0.822624in}}{\pgfqpoint{1.076407in}{0.818234in}}{\pgfqpoint{1.087457in}{0.818234in}}%
\pgfpathclose%
\pgfusepath{stroke,fill}%
\end{pgfscope}%
\begin{pgfscope}%
\pgfpathrectangle{\pgfqpoint{0.648703in}{0.548769in}}{\pgfqpoint{5.112893in}{3.102590in}}%
\pgfusepath{clip}%
\pgfsetbuttcap%
\pgfsetroundjoin%
\definecolor{currentfill}{rgb}{0.839216,0.152941,0.156863}%
\pgfsetfillcolor{currentfill}%
\pgfsetlinewidth{1.003750pt}%
\definecolor{currentstroke}{rgb}{0.839216,0.152941,0.156863}%
\pgfsetstrokecolor{currentstroke}%
\pgfsetdash{}{0pt}%
\pgfpathmoveto{\pgfqpoint{3.770187in}{3.198987in}}%
\pgfpathcurveto{\pgfqpoint{3.781237in}{3.198987in}}{\pgfqpoint{3.791836in}{3.203377in}}{\pgfqpoint{3.799650in}{3.211191in}}%
\pgfpathcurveto{\pgfqpoint{3.807463in}{3.219004in}}{\pgfqpoint{3.811854in}{3.229603in}}{\pgfqpoint{3.811854in}{3.240654in}}%
\pgfpathcurveto{\pgfqpoint{3.811854in}{3.251704in}}{\pgfqpoint{3.807463in}{3.262303in}}{\pgfqpoint{3.799650in}{3.270116in}}%
\pgfpathcurveto{\pgfqpoint{3.791836in}{3.277930in}}{\pgfqpoint{3.781237in}{3.282320in}}{\pgfqpoint{3.770187in}{3.282320in}}%
\pgfpathcurveto{\pgfqpoint{3.759137in}{3.282320in}}{\pgfqpoint{3.748538in}{3.277930in}}{\pgfqpoint{3.740724in}{3.270116in}}%
\pgfpathcurveto{\pgfqpoint{3.732911in}{3.262303in}}{\pgfqpoint{3.728520in}{3.251704in}}{\pgfqpoint{3.728520in}{3.240654in}}%
\pgfpathcurveto{\pgfqpoint{3.728520in}{3.229603in}}{\pgfqpoint{3.732911in}{3.219004in}}{\pgfqpoint{3.740724in}{3.211191in}}%
\pgfpathcurveto{\pgfqpoint{3.748538in}{3.203377in}}{\pgfqpoint{3.759137in}{3.198987in}}{\pgfqpoint{3.770187in}{3.198987in}}%
\pgfpathclose%
\pgfusepath{stroke,fill}%
\end{pgfscope}%
\begin{pgfscope}%
\pgfpathrectangle{\pgfqpoint{0.648703in}{0.548769in}}{\pgfqpoint{5.112893in}{3.102590in}}%
\pgfusepath{clip}%
\pgfsetbuttcap%
\pgfsetroundjoin%
\definecolor{currentfill}{rgb}{0.121569,0.466667,0.705882}%
\pgfsetfillcolor{currentfill}%
\pgfsetlinewidth{1.003750pt}%
\definecolor{currentstroke}{rgb}{0.121569,0.466667,0.705882}%
\pgfsetstrokecolor{currentstroke}%
\pgfsetdash{}{0pt}%
\pgfpathmoveto{\pgfqpoint{0.832788in}{0.658113in}}%
\pgfpathcurveto{\pgfqpoint{0.843838in}{0.658113in}}{\pgfqpoint{0.854437in}{0.662503in}}{\pgfqpoint{0.862251in}{0.670317in}}%
\pgfpathcurveto{\pgfqpoint{0.870064in}{0.678130in}}{\pgfqpoint{0.874454in}{0.688729in}}{\pgfqpoint{0.874454in}{0.699779in}}%
\pgfpathcurveto{\pgfqpoint{0.874454in}{0.710830in}}{\pgfqpoint{0.870064in}{0.721429in}}{\pgfqpoint{0.862251in}{0.729242in}}%
\pgfpathcurveto{\pgfqpoint{0.854437in}{0.737056in}}{\pgfqpoint{0.843838in}{0.741446in}}{\pgfqpoint{0.832788in}{0.741446in}}%
\pgfpathcurveto{\pgfqpoint{0.821738in}{0.741446in}}{\pgfqpoint{0.811139in}{0.737056in}}{\pgfqpoint{0.803325in}{0.729242in}}%
\pgfpathcurveto{\pgfqpoint{0.795511in}{0.721429in}}{\pgfqpoint{0.791121in}{0.710830in}}{\pgfqpoint{0.791121in}{0.699779in}}%
\pgfpathcurveto{\pgfqpoint{0.791121in}{0.688729in}}{\pgfqpoint{0.795511in}{0.678130in}}{\pgfqpoint{0.803325in}{0.670317in}}%
\pgfpathcurveto{\pgfqpoint{0.811139in}{0.662503in}}{\pgfqpoint{0.821738in}{0.658113in}}{\pgfqpoint{0.832788in}{0.658113in}}%
\pgfpathclose%
\pgfusepath{stroke,fill}%
\end{pgfscope}%
\begin{pgfscope}%
\pgfpathrectangle{\pgfqpoint{0.648703in}{0.548769in}}{\pgfqpoint{5.112893in}{3.102590in}}%
\pgfusepath{clip}%
\pgfsetbuttcap%
\pgfsetroundjoin%
\definecolor{currentfill}{rgb}{1.000000,0.498039,0.054902}%
\pgfsetfillcolor{currentfill}%
\pgfsetlinewidth{1.003750pt}%
\definecolor{currentstroke}{rgb}{1.000000,0.498039,0.054902}%
\pgfsetstrokecolor{currentstroke}%
\pgfsetdash{}{0pt}%
\pgfpathmoveto{\pgfqpoint{2.971925in}{3.186346in}}%
\pgfpathcurveto{\pgfqpoint{2.982975in}{3.186346in}}{\pgfqpoint{2.993574in}{3.190736in}}{\pgfqpoint{3.001387in}{3.198550in}}%
\pgfpathcurveto{\pgfqpoint{3.009201in}{3.206363in}}{\pgfqpoint{3.013591in}{3.216962in}}{\pgfqpoint{3.013591in}{3.228012in}}%
\pgfpathcurveto{\pgfqpoint{3.013591in}{3.239063in}}{\pgfqpoint{3.009201in}{3.249662in}}{\pgfqpoint{3.001387in}{3.257475in}}%
\pgfpathcurveto{\pgfqpoint{2.993574in}{3.265289in}}{\pgfqpoint{2.982975in}{3.269679in}}{\pgfqpoint{2.971925in}{3.269679in}}%
\pgfpathcurveto{\pgfqpoint{2.960874in}{3.269679in}}{\pgfqpoint{2.950275in}{3.265289in}}{\pgfqpoint{2.942462in}{3.257475in}}%
\pgfpathcurveto{\pgfqpoint{2.934648in}{3.249662in}}{\pgfqpoint{2.930258in}{3.239063in}}{\pgfqpoint{2.930258in}{3.228012in}}%
\pgfpathcurveto{\pgfqpoint{2.930258in}{3.216962in}}{\pgfqpoint{2.934648in}{3.206363in}}{\pgfqpoint{2.942462in}{3.198550in}}%
\pgfpathcurveto{\pgfqpoint{2.950275in}{3.190736in}}{\pgfqpoint{2.960874in}{3.186346in}}{\pgfqpoint{2.971925in}{3.186346in}}%
\pgfpathclose%
\pgfusepath{stroke,fill}%
\end{pgfscope}%
\begin{pgfscope}%
\pgfpathrectangle{\pgfqpoint{0.648703in}{0.548769in}}{\pgfqpoint{5.112893in}{3.102590in}}%
\pgfusepath{clip}%
\pgfsetbuttcap%
\pgfsetroundjoin%
\definecolor{currentfill}{rgb}{0.121569,0.466667,0.705882}%
\pgfsetfillcolor{currentfill}%
\pgfsetlinewidth{1.003750pt}%
\definecolor{currentstroke}{rgb}{0.121569,0.466667,0.705882}%
\pgfsetstrokecolor{currentstroke}%
\pgfsetdash{}{0pt}%
\pgfpathmoveto{\pgfqpoint{0.834075in}{0.658113in}}%
\pgfpathcurveto{\pgfqpoint{0.845125in}{0.658113in}}{\pgfqpoint{0.855724in}{0.662503in}}{\pgfqpoint{0.863538in}{0.670317in}}%
\pgfpathcurveto{\pgfqpoint{0.871352in}{0.678130in}}{\pgfqpoint{0.875742in}{0.688729in}}{\pgfqpoint{0.875742in}{0.699779in}}%
\pgfpathcurveto{\pgfqpoint{0.875742in}{0.710830in}}{\pgfqpoint{0.871352in}{0.721429in}}{\pgfqpoint{0.863538in}{0.729242in}}%
\pgfpathcurveto{\pgfqpoint{0.855724in}{0.737056in}}{\pgfqpoint{0.845125in}{0.741446in}}{\pgfqpoint{0.834075in}{0.741446in}}%
\pgfpathcurveto{\pgfqpoint{0.823025in}{0.741446in}}{\pgfqpoint{0.812426in}{0.737056in}}{\pgfqpoint{0.804612in}{0.729242in}}%
\pgfpathcurveto{\pgfqpoint{0.796799in}{0.721429in}}{\pgfqpoint{0.792409in}{0.710830in}}{\pgfqpoint{0.792409in}{0.699779in}}%
\pgfpathcurveto{\pgfqpoint{0.792409in}{0.688729in}}{\pgfqpoint{0.796799in}{0.678130in}}{\pgfqpoint{0.804612in}{0.670317in}}%
\pgfpathcurveto{\pgfqpoint{0.812426in}{0.662503in}}{\pgfqpoint{0.823025in}{0.658113in}}{\pgfqpoint{0.834075in}{0.658113in}}%
\pgfpathclose%
\pgfusepath{stroke,fill}%
\end{pgfscope}%
\begin{pgfscope}%
\pgfpathrectangle{\pgfqpoint{0.648703in}{0.548769in}}{\pgfqpoint{5.112893in}{3.102590in}}%
\pgfusepath{clip}%
\pgfsetbuttcap%
\pgfsetroundjoin%
\definecolor{currentfill}{rgb}{1.000000,0.498039,0.054902}%
\pgfsetfillcolor{currentfill}%
\pgfsetlinewidth{1.003750pt}%
\definecolor{currentstroke}{rgb}{1.000000,0.498039,0.054902}%
\pgfsetstrokecolor{currentstroke}%
\pgfsetdash{}{0pt}%
\pgfpathmoveto{\pgfqpoint{2.598035in}{3.186346in}}%
\pgfpathcurveto{\pgfqpoint{2.609086in}{3.186346in}}{\pgfqpoint{2.619685in}{3.190736in}}{\pgfqpoint{2.627498in}{3.198550in}}%
\pgfpathcurveto{\pgfqpoint{2.635312in}{3.206363in}}{\pgfqpoint{2.639702in}{3.216962in}}{\pgfqpoint{2.639702in}{3.228012in}}%
\pgfpathcurveto{\pgfqpoint{2.639702in}{3.239063in}}{\pgfqpoint{2.635312in}{3.249662in}}{\pgfqpoint{2.627498in}{3.257475in}}%
\pgfpathcurveto{\pgfqpoint{2.619685in}{3.265289in}}{\pgfqpoint{2.609086in}{3.269679in}}{\pgfqpoint{2.598035in}{3.269679in}}%
\pgfpathcurveto{\pgfqpoint{2.586985in}{3.269679in}}{\pgfqpoint{2.576386in}{3.265289in}}{\pgfqpoint{2.568573in}{3.257475in}}%
\pgfpathcurveto{\pgfqpoint{2.560759in}{3.249662in}}{\pgfqpoint{2.556369in}{3.239063in}}{\pgfqpoint{2.556369in}{3.228012in}}%
\pgfpathcurveto{\pgfqpoint{2.556369in}{3.216962in}}{\pgfqpoint{2.560759in}{3.206363in}}{\pgfqpoint{2.568573in}{3.198550in}}%
\pgfpathcurveto{\pgfqpoint{2.576386in}{3.190736in}}{\pgfqpoint{2.586985in}{3.186346in}}{\pgfqpoint{2.598035in}{3.186346in}}%
\pgfpathclose%
\pgfusepath{stroke,fill}%
\end{pgfscope}%
\begin{pgfscope}%
\pgfpathrectangle{\pgfqpoint{0.648703in}{0.548769in}}{\pgfqpoint{5.112893in}{3.102590in}}%
\pgfusepath{clip}%
\pgfsetbuttcap%
\pgfsetroundjoin%
\definecolor{currentfill}{rgb}{0.121569,0.466667,0.705882}%
\pgfsetfillcolor{currentfill}%
\pgfsetlinewidth{1.003750pt}%
\definecolor{currentstroke}{rgb}{0.121569,0.466667,0.705882}%
\pgfsetstrokecolor{currentstroke}%
\pgfsetdash{}{0pt}%
\pgfpathmoveto{\pgfqpoint{0.831253in}{0.658113in}}%
\pgfpathcurveto{\pgfqpoint{0.842303in}{0.658113in}}{\pgfqpoint{0.852902in}{0.662503in}}{\pgfqpoint{0.860716in}{0.670317in}}%
\pgfpathcurveto{\pgfqpoint{0.868529in}{0.678130in}}{\pgfqpoint{0.872920in}{0.688729in}}{\pgfqpoint{0.872920in}{0.699779in}}%
\pgfpathcurveto{\pgfqpoint{0.872920in}{0.710830in}}{\pgfqpoint{0.868529in}{0.721429in}}{\pgfqpoint{0.860716in}{0.729242in}}%
\pgfpathcurveto{\pgfqpoint{0.852902in}{0.737056in}}{\pgfqpoint{0.842303in}{0.741446in}}{\pgfqpoint{0.831253in}{0.741446in}}%
\pgfpathcurveto{\pgfqpoint{0.820203in}{0.741446in}}{\pgfqpoint{0.809604in}{0.737056in}}{\pgfqpoint{0.801790in}{0.729242in}}%
\pgfpathcurveto{\pgfqpoint{0.793977in}{0.721429in}}{\pgfqpoint{0.789586in}{0.710830in}}{\pgfqpoint{0.789586in}{0.699779in}}%
\pgfpathcurveto{\pgfqpoint{0.789586in}{0.688729in}}{\pgfqpoint{0.793977in}{0.678130in}}{\pgfqpoint{0.801790in}{0.670317in}}%
\pgfpathcurveto{\pgfqpoint{0.809604in}{0.662503in}}{\pgfqpoint{0.820203in}{0.658113in}}{\pgfqpoint{0.831253in}{0.658113in}}%
\pgfpathclose%
\pgfusepath{stroke,fill}%
\end{pgfscope}%
\begin{pgfscope}%
\pgfpathrectangle{\pgfqpoint{0.648703in}{0.548769in}}{\pgfqpoint{5.112893in}{3.102590in}}%
\pgfusepath{clip}%
\pgfsetbuttcap%
\pgfsetroundjoin%
\definecolor{currentfill}{rgb}{1.000000,0.498039,0.054902}%
\pgfsetfillcolor{currentfill}%
\pgfsetlinewidth{1.003750pt}%
\definecolor{currentstroke}{rgb}{1.000000,0.498039,0.054902}%
\pgfsetstrokecolor{currentstroke}%
\pgfsetdash{}{0pt}%
\pgfpathmoveto{\pgfqpoint{3.319615in}{3.279048in}}%
\pgfpathcurveto{\pgfqpoint{3.330665in}{3.279048in}}{\pgfqpoint{3.341264in}{3.283438in}}{\pgfqpoint{3.349077in}{3.291252in}}%
\pgfpathcurveto{\pgfqpoint{3.356891in}{3.299065in}}{\pgfqpoint{3.361281in}{3.309664in}}{\pgfqpoint{3.361281in}{3.320714in}}%
\pgfpathcurveto{\pgfqpoint{3.361281in}{3.331764in}}{\pgfqpoint{3.356891in}{3.342363in}}{\pgfqpoint{3.349077in}{3.350177in}}%
\pgfpathcurveto{\pgfqpoint{3.341264in}{3.357991in}}{\pgfqpoint{3.330665in}{3.362381in}}{\pgfqpoint{3.319615in}{3.362381in}}%
\pgfpathcurveto{\pgfqpoint{3.308564in}{3.362381in}}{\pgfqpoint{3.297965in}{3.357991in}}{\pgfqpoint{3.290152in}{3.350177in}}%
\pgfpathcurveto{\pgfqpoint{3.282338in}{3.342363in}}{\pgfqpoint{3.277948in}{3.331764in}}{\pgfqpoint{3.277948in}{3.320714in}}%
\pgfpathcurveto{\pgfqpoint{3.277948in}{3.309664in}}{\pgfqpoint{3.282338in}{3.299065in}}{\pgfqpoint{3.290152in}{3.291252in}}%
\pgfpathcurveto{\pgfqpoint{3.297965in}{3.283438in}}{\pgfqpoint{3.308564in}{3.279048in}}{\pgfqpoint{3.319615in}{3.279048in}}%
\pgfpathclose%
\pgfusepath{stroke,fill}%
\end{pgfscope}%
\begin{pgfscope}%
\pgfpathrectangle{\pgfqpoint{0.648703in}{0.548769in}}{\pgfqpoint{5.112893in}{3.102590in}}%
\pgfusepath{clip}%
\pgfsetbuttcap%
\pgfsetroundjoin%
\definecolor{currentfill}{rgb}{0.121569,0.466667,0.705882}%
\pgfsetfillcolor{currentfill}%
\pgfsetlinewidth{1.003750pt}%
\definecolor{currentstroke}{rgb}{0.121569,0.466667,0.705882}%
\pgfsetstrokecolor{currentstroke}%
\pgfsetdash{}{0pt}%
\pgfpathmoveto{\pgfqpoint{0.831231in}{0.658113in}}%
\pgfpathcurveto{\pgfqpoint{0.842282in}{0.658113in}}{\pgfqpoint{0.852881in}{0.662503in}}{\pgfqpoint{0.860694in}{0.670317in}}%
\pgfpathcurveto{\pgfqpoint{0.868508in}{0.678130in}}{\pgfqpoint{0.872898in}{0.688729in}}{\pgfqpoint{0.872898in}{0.699779in}}%
\pgfpathcurveto{\pgfqpoint{0.872898in}{0.710830in}}{\pgfqpoint{0.868508in}{0.721429in}}{\pgfqpoint{0.860694in}{0.729242in}}%
\pgfpathcurveto{\pgfqpoint{0.852881in}{0.737056in}}{\pgfqpoint{0.842282in}{0.741446in}}{\pgfqpoint{0.831231in}{0.741446in}}%
\pgfpathcurveto{\pgfqpoint{0.820181in}{0.741446in}}{\pgfqpoint{0.809582in}{0.737056in}}{\pgfqpoint{0.801769in}{0.729242in}}%
\pgfpathcurveto{\pgfqpoint{0.793955in}{0.721429in}}{\pgfqpoint{0.789565in}{0.710830in}}{\pgfqpoint{0.789565in}{0.699779in}}%
\pgfpathcurveto{\pgfqpoint{0.789565in}{0.688729in}}{\pgfqpoint{0.793955in}{0.678130in}}{\pgfqpoint{0.801769in}{0.670317in}}%
\pgfpathcurveto{\pgfqpoint{0.809582in}{0.662503in}}{\pgfqpoint{0.820181in}{0.658113in}}{\pgfqpoint{0.831231in}{0.658113in}}%
\pgfpathclose%
\pgfusepath{stroke,fill}%
\end{pgfscope}%
\begin{pgfscope}%
\pgfpathrectangle{\pgfqpoint{0.648703in}{0.548769in}}{\pgfqpoint{5.112893in}{3.102590in}}%
\pgfusepath{clip}%
\pgfsetbuttcap%
\pgfsetroundjoin%
\definecolor{currentfill}{rgb}{1.000000,0.498039,0.054902}%
\pgfsetfillcolor{currentfill}%
\pgfsetlinewidth{1.003750pt}%
\definecolor{currentstroke}{rgb}{1.000000,0.498039,0.054902}%
\pgfsetstrokecolor{currentstroke}%
\pgfsetdash{}{0pt}%
\pgfpathmoveto{\pgfqpoint{3.447906in}{3.203201in}}%
\pgfpathcurveto{\pgfqpoint{3.458956in}{3.203201in}}{\pgfqpoint{3.469555in}{3.207591in}}{\pgfqpoint{3.477369in}{3.215405in}}%
\pgfpathcurveto{\pgfqpoint{3.485182in}{3.223218in}}{\pgfqpoint{3.489573in}{3.233817in}}{\pgfqpoint{3.489573in}{3.244867in}}%
\pgfpathcurveto{\pgfqpoint{3.489573in}{3.255917in}}{\pgfqpoint{3.485182in}{3.266516in}}{\pgfqpoint{3.477369in}{3.274330in}}%
\pgfpathcurveto{\pgfqpoint{3.469555in}{3.282144in}}{\pgfqpoint{3.458956in}{3.286534in}}{\pgfqpoint{3.447906in}{3.286534in}}%
\pgfpathcurveto{\pgfqpoint{3.436856in}{3.286534in}}{\pgfqpoint{3.426257in}{3.282144in}}{\pgfqpoint{3.418443in}{3.274330in}}%
\pgfpathcurveto{\pgfqpoint{3.410630in}{3.266516in}}{\pgfqpoint{3.406239in}{3.255917in}}{\pgfqpoint{3.406239in}{3.244867in}}%
\pgfpathcurveto{\pgfqpoint{3.406239in}{3.233817in}}{\pgfqpoint{3.410630in}{3.223218in}}{\pgfqpoint{3.418443in}{3.215405in}}%
\pgfpathcurveto{\pgfqpoint{3.426257in}{3.207591in}}{\pgfqpoint{3.436856in}{3.203201in}}{\pgfqpoint{3.447906in}{3.203201in}}%
\pgfpathclose%
\pgfusepath{stroke,fill}%
\end{pgfscope}%
\begin{pgfscope}%
\pgfpathrectangle{\pgfqpoint{0.648703in}{0.548769in}}{\pgfqpoint{5.112893in}{3.102590in}}%
\pgfusepath{clip}%
\pgfsetbuttcap%
\pgfsetroundjoin%
\definecolor{currentfill}{rgb}{1.000000,0.498039,0.054902}%
\pgfsetfillcolor{currentfill}%
\pgfsetlinewidth{1.003750pt}%
\definecolor{currentstroke}{rgb}{1.000000,0.498039,0.054902}%
\pgfsetstrokecolor{currentstroke}%
\pgfsetdash{}{0pt}%
\pgfpathmoveto{\pgfqpoint{3.093396in}{3.186346in}}%
\pgfpathcurveto{\pgfqpoint{3.104447in}{3.186346in}}{\pgfqpoint{3.115046in}{3.190736in}}{\pgfqpoint{3.122859in}{3.198550in}}%
\pgfpathcurveto{\pgfqpoint{3.130673in}{3.206363in}}{\pgfqpoint{3.135063in}{3.216962in}}{\pgfqpoint{3.135063in}{3.228012in}}%
\pgfpathcurveto{\pgfqpoint{3.135063in}{3.239063in}}{\pgfqpoint{3.130673in}{3.249662in}}{\pgfqpoint{3.122859in}{3.257475in}}%
\pgfpathcurveto{\pgfqpoint{3.115046in}{3.265289in}}{\pgfqpoint{3.104447in}{3.269679in}}{\pgfqpoint{3.093396in}{3.269679in}}%
\pgfpathcurveto{\pgfqpoint{3.082346in}{3.269679in}}{\pgfqpoint{3.071747in}{3.265289in}}{\pgfqpoint{3.063934in}{3.257475in}}%
\pgfpathcurveto{\pgfqpoint{3.056120in}{3.249662in}}{\pgfqpoint{3.051730in}{3.239063in}}{\pgfqpoint{3.051730in}{3.228012in}}%
\pgfpathcurveto{\pgfqpoint{3.051730in}{3.216962in}}{\pgfqpoint{3.056120in}{3.206363in}}{\pgfqpoint{3.063934in}{3.198550in}}%
\pgfpathcurveto{\pgfqpoint{3.071747in}{3.190736in}}{\pgfqpoint{3.082346in}{3.186346in}}{\pgfqpoint{3.093396in}{3.186346in}}%
\pgfpathclose%
\pgfusepath{stroke,fill}%
\end{pgfscope}%
\begin{pgfscope}%
\pgfpathrectangle{\pgfqpoint{0.648703in}{0.548769in}}{\pgfqpoint{5.112893in}{3.102590in}}%
\pgfusepath{clip}%
\pgfsetbuttcap%
\pgfsetroundjoin%
\definecolor{currentfill}{rgb}{0.121569,0.466667,0.705882}%
\pgfsetfillcolor{currentfill}%
\pgfsetlinewidth{1.003750pt}%
\definecolor{currentstroke}{rgb}{0.121569,0.466667,0.705882}%
\pgfsetstrokecolor{currentstroke}%
\pgfsetdash{}{0pt}%
\pgfpathmoveto{\pgfqpoint{0.831244in}{0.658113in}}%
\pgfpathcurveto{\pgfqpoint{0.842294in}{0.658113in}}{\pgfqpoint{0.852893in}{0.662503in}}{\pgfqpoint{0.860707in}{0.670317in}}%
\pgfpathcurveto{\pgfqpoint{0.868520in}{0.678130in}}{\pgfqpoint{0.872910in}{0.688729in}}{\pgfqpoint{0.872910in}{0.699779in}}%
\pgfpathcurveto{\pgfqpoint{0.872910in}{0.710830in}}{\pgfqpoint{0.868520in}{0.721429in}}{\pgfqpoint{0.860707in}{0.729242in}}%
\pgfpathcurveto{\pgfqpoint{0.852893in}{0.737056in}}{\pgfqpoint{0.842294in}{0.741446in}}{\pgfqpoint{0.831244in}{0.741446in}}%
\pgfpathcurveto{\pgfqpoint{0.820194in}{0.741446in}}{\pgfqpoint{0.809595in}{0.737056in}}{\pgfqpoint{0.801781in}{0.729242in}}%
\pgfpathcurveto{\pgfqpoint{0.793967in}{0.721429in}}{\pgfqpoint{0.789577in}{0.710830in}}{\pgfqpoint{0.789577in}{0.699779in}}%
\pgfpathcurveto{\pgfqpoint{0.789577in}{0.688729in}}{\pgfqpoint{0.793967in}{0.678130in}}{\pgfqpoint{0.801781in}{0.670317in}}%
\pgfpathcurveto{\pgfqpoint{0.809595in}{0.662503in}}{\pgfqpoint{0.820194in}{0.658113in}}{\pgfqpoint{0.831244in}{0.658113in}}%
\pgfpathclose%
\pgfusepath{stroke,fill}%
\end{pgfscope}%
\begin{pgfscope}%
\pgfpathrectangle{\pgfqpoint{0.648703in}{0.548769in}}{\pgfqpoint{5.112893in}{3.102590in}}%
\pgfusepath{clip}%
\pgfsetbuttcap%
\pgfsetroundjoin%
\definecolor{currentfill}{rgb}{1.000000,0.498039,0.054902}%
\pgfsetfillcolor{currentfill}%
\pgfsetlinewidth{1.003750pt}%
\definecolor{currentstroke}{rgb}{1.000000,0.498039,0.054902}%
\pgfsetstrokecolor{currentstroke}%
\pgfsetdash{}{0pt}%
\pgfpathmoveto{\pgfqpoint{4.182476in}{3.186346in}}%
\pgfpathcurveto{\pgfqpoint{4.193526in}{3.186346in}}{\pgfqpoint{4.204125in}{3.190736in}}{\pgfqpoint{4.211939in}{3.198550in}}%
\pgfpathcurveto{\pgfqpoint{4.219753in}{3.206363in}}{\pgfqpoint{4.224143in}{3.216962in}}{\pgfqpoint{4.224143in}{3.228012in}}%
\pgfpathcurveto{\pgfqpoint{4.224143in}{3.239063in}}{\pgfqpoint{4.219753in}{3.249662in}}{\pgfqpoint{4.211939in}{3.257475in}}%
\pgfpathcurveto{\pgfqpoint{4.204125in}{3.265289in}}{\pgfqpoint{4.193526in}{3.269679in}}{\pgfqpoint{4.182476in}{3.269679in}}%
\pgfpathcurveto{\pgfqpoint{4.171426in}{3.269679in}}{\pgfqpoint{4.160827in}{3.265289in}}{\pgfqpoint{4.153014in}{3.257475in}}%
\pgfpathcurveto{\pgfqpoint{4.145200in}{3.249662in}}{\pgfqpoint{4.140810in}{3.239063in}}{\pgfqpoint{4.140810in}{3.228012in}}%
\pgfpathcurveto{\pgfqpoint{4.140810in}{3.216962in}}{\pgfqpoint{4.145200in}{3.206363in}}{\pgfqpoint{4.153014in}{3.198550in}}%
\pgfpathcurveto{\pgfqpoint{4.160827in}{3.190736in}}{\pgfqpoint{4.171426in}{3.186346in}}{\pgfqpoint{4.182476in}{3.186346in}}%
\pgfpathclose%
\pgfusepath{stroke,fill}%
\end{pgfscope}%
\begin{pgfscope}%
\pgfpathrectangle{\pgfqpoint{0.648703in}{0.548769in}}{\pgfqpoint{5.112893in}{3.102590in}}%
\pgfusepath{clip}%
\pgfsetbuttcap%
\pgfsetroundjoin%
\definecolor{currentfill}{rgb}{1.000000,0.498039,0.054902}%
\pgfsetfillcolor{currentfill}%
\pgfsetlinewidth{1.003750pt}%
\definecolor{currentstroke}{rgb}{1.000000,0.498039,0.054902}%
\pgfsetstrokecolor{currentstroke}%
\pgfsetdash{}{0pt}%
\pgfpathmoveto{\pgfqpoint{4.050077in}{3.194773in}}%
\pgfpathcurveto{\pgfqpoint{4.061128in}{3.194773in}}{\pgfqpoint{4.071727in}{3.199163in}}{\pgfqpoint{4.079540in}{3.206977in}}%
\pgfpathcurveto{\pgfqpoint{4.087354in}{3.214791in}}{\pgfqpoint{4.091744in}{3.225390in}}{\pgfqpoint{4.091744in}{3.236440in}}%
\pgfpathcurveto{\pgfqpoint{4.091744in}{3.247490in}}{\pgfqpoint{4.087354in}{3.258089in}}{\pgfqpoint{4.079540in}{3.265903in}}%
\pgfpathcurveto{\pgfqpoint{4.071727in}{3.273716in}}{\pgfqpoint{4.061128in}{3.278107in}}{\pgfqpoint{4.050077in}{3.278107in}}%
\pgfpathcurveto{\pgfqpoint{4.039027in}{3.278107in}}{\pgfqpoint{4.028428in}{3.273716in}}{\pgfqpoint{4.020615in}{3.265903in}}%
\pgfpathcurveto{\pgfqpoint{4.012801in}{3.258089in}}{\pgfqpoint{4.008411in}{3.247490in}}{\pgfqpoint{4.008411in}{3.236440in}}%
\pgfpathcurveto{\pgfqpoint{4.008411in}{3.225390in}}{\pgfqpoint{4.012801in}{3.214791in}}{\pgfqpoint{4.020615in}{3.206977in}}%
\pgfpathcurveto{\pgfqpoint{4.028428in}{3.199163in}}{\pgfqpoint{4.039027in}{3.194773in}}{\pgfqpoint{4.050077in}{3.194773in}}%
\pgfpathclose%
\pgfusepath{stroke,fill}%
\end{pgfscope}%
\begin{pgfscope}%
\pgfpathrectangle{\pgfqpoint{0.648703in}{0.548769in}}{\pgfqpoint{5.112893in}{3.102590in}}%
\pgfusepath{clip}%
\pgfsetbuttcap%
\pgfsetroundjoin%
\definecolor{currentfill}{rgb}{0.121569,0.466667,0.705882}%
\pgfsetfillcolor{currentfill}%
\pgfsetlinewidth{1.003750pt}%
\definecolor{currentstroke}{rgb}{0.121569,0.466667,0.705882}%
\pgfsetstrokecolor{currentstroke}%
\pgfsetdash{}{0pt}%
\pgfpathmoveto{\pgfqpoint{0.841886in}{0.666540in}}%
\pgfpathcurveto{\pgfqpoint{0.852936in}{0.666540in}}{\pgfqpoint{0.863535in}{0.670930in}}{\pgfqpoint{0.871349in}{0.678744in}}%
\pgfpathcurveto{\pgfqpoint{0.879162in}{0.686558in}}{\pgfqpoint{0.883552in}{0.697157in}}{\pgfqpoint{0.883552in}{0.708207in}}%
\pgfpathcurveto{\pgfqpoint{0.883552in}{0.719257in}}{\pgfqpoint{0.879162in}{0.729856in}}{\pgfqpoint{0.871349in}{0.737670in}}%
\pgfpathcurveto{\pgfqpoint{0.863535in}{0.745483in}}{\pgfqpoint{0.852936in}{0.749874in}}{\pgfqpoint{0.841886in}{0.749874in}}%
\pgfpathcurveto{\pgfqpoint{0.830836in}{0.749874in}}{\pgfqpoint{0.820237in}{0.745483in}}{\pgfqpoint{0.812423in}{0.737670in}}%
\pgfpathcurveto{\pgfqpoint{0.804609in}{0.729856in}}{\pgfqpoint{0.800219in}{0.719257in}}{\pgfqpoint{0.800219in}{0.708207in}}%
\pgfpathcurveto{\pgfqpoint{0.800219in}{0.697157in}}{\pgfqpoint{0.804609in}{0.686558in}}{\pgfqpoint{0.812423in}{0.678744in}}%
\pgfpathcurveto{\pgfqpoint{0.820237in}{0.670930in}}{\pgfqpoint{0.830836in}{0.666540in}}{\pgfqpoint{0.841886in}{0.666540in}}%
\pgfpathclose%
\pgfusepath{stroke,fill}%
\end{pgfscope}%
\begin{pgfscope}%
\pgfpathrectangle{\pgfqpoint{0.648703in}{0.548769in}}{\pgfqpoint{5.112893in}{3.102590in}}%
\pgfusepath{clip}%
\pgfsetbuttcap%
\pgfsetroundjoin%
\definecolor{currentfill}{rgb}{1.000000,0.498039,0.054902}%
\pgfsetfillcolor{currentfill}%
\pgfsetlinewidth{1.003750pt}%
\definecolor{currentstroke}{rgb}{1.000000,0.498039,0.054902}%
\pgfsetstrokecolor{currentstroke}%
\pgfsetdash{}{0pt}%
\pgfpathmoveto{\pgfqpoint{3.372825in}{3.215842in}}%
\pgfpathcurveto{\pgfqpoint{3.383875in}{3.215842in}}{\pgfqpoint{3.394474in}{3.220232in}}{\pgfqpoint{3.402287in}{3.228046in}}%
\pgfpathcurveto{\pgfqpoint{3.410101in}{3.235859in}}{\pgfqpoint{3.414491in}{3.246458in}}{\pgfqpoint{3.414491in}{3.257508in}}%
\pgfpathcurveto{\pgfqpoint{3.414491in}{3.268559in}}{\pgfqpoint{3.410101in}{3.279158in}}{\pgfqpoint{3.402287in}{3.286971in}}%
\pgfpathcurveto{\pgfqpoint{3.394474in}{3.294785in}}{\pgfqpoint{3.383875in}{3.299175in}}{\pgfqpoint{3.372825in}{3.299175in}}%
\pgfpathcurveto{\pgfqpoint{3.361775in}{3.299175in}}{\pgfqpoint{3.351175in}{3.294785in}}{\pgfqpoint{3.343362in}{3.286971in}}%
\pgfpathcurveto{\pgfqpoint{3.335548in}{3.279158in}}{\pgfqpoint{3.331158in}{3.268559in}}{\pgfqpoint{3.331158in}{3.257508in}}%
\pgfpathcurveto{\pgfqpoint{3.331158in}{3.246458in}}{\pgfqpoint{3.335548in}{3.235859in}}{\pgfqpoint{3.343362in}{3.228046in}}%
\pgfpathcurveto{\pgfqpoint{3.351175in}{3.220232in}}{\pgfqpoint{3.361775in}{3.215842in}}{\pgfqpoint{3.372825in}{3.215842in}}%
\pgfpathclose%
\pgfusepath{stroke,fill}%
\end{pgfscope}%
\begin{pgfscope}%
\pgfpathrectangle{\pgfqpoint{0.648703in}{0.548769in}}{\pgfqpoint{5.112893in}{3.102590in}}%
\pgfusepath{clip}%
\pgfsetbuttcap%
\pgfsetroundjoin%
\definecolor{currentfill}{rgb}{1.000000,0.498039,0.054902}%
\pgfsetfillcolor{currentfill}%
\pgfsetlinewidth{1.003750pt}%
\definecolor{currentstroke}{rgb}{1.000000,0.498039,0.054902}%
\pgfsetstrokecolor{currentstroke}%
\pgfsetdash{}{0pt}%
\pgfpathmoveto{\pgfqpoint{3.872269in}{3.203201in}}%
\pgfpathcurveto{\pgfqpoint{3.883319in}{3.203201in}}{\pgfqpoint{3.893918in}{3.207591in}}{\pgfqpoint{3.901732in}{3.215405in}}%
\pgfpathcurveto{\pgfqpoint{3.909545in}{3.223218in}}{\pgfqpoint{3.913935in}{3.233817in}}{\pgfqpoint{3.913935in}{3.244867in}}%
\pgfpathcurveto{\pgfqpoint{3.913935in}{3.255917in}}{\pgfqpoint{3.909545in}{3.266516in}}{\pgfqpoint{3.901732in}{3.274330in}}%
\pgfpathcurveto{\pgfqpoint{3.893918in}{3.282144in}}{\pgfqpoint{3.883319in}{3.286534in}}{\pgfqpoint{3.872269in}{3.286534in}}%
\pgfpathcurveto{\pgfqpoint{3.861219in}{3.286534in}}{\pgfqpoint{3.850620in}{3.282144in}}{\pgfqpoint{3.842806in}{3.274330in}}%
\pgfpathcurveto{\pgfqpoint{3.834992in}{3.266516in}}{\pgfqpoint{3.830602in}{3.255917in}}{\pgfqpoint{3.830602in}{3.244867in}}%
\pgfpathcurveto{\pgfqpoint{3.830602in}{3.233817in}}{\pgfqpoint{3.834992in}{3.223218in}}{\pgfqpoint{3.842806in}{3.215405in}}%
\pgfpathcurveto{\pgfqpoint{3.850620in}{3.207591in}}{\pgfqpoint{3.861219in}{3.203201in}}{\pgfqpoint{3.872269in}{3.203201in}}%
\pgfpathclose%
\pgfusepath{stroke,fill}%
\end{pgfscope}%
\begin{pgfscope}%
\pgfpathrectangle{\pgfqpoint{0.648703in}{0.548769in}}{\pgfqpoint{5.112893in}{3.102590in}}%
\pgfusepath{clip}%
\pgfsetbuttcap%
\pgfsetroundjoin%
\definecolor{currentfill}{rgb}{1.000000,0.498039,0.054902}%
\pgfsetfillcolor{currentfill}%
\pgfsetlinewidth{1.003750pt}%
\definecolor{currentstroke}{rgb}{1.000000,0.498039,0.054902}%
\pgfsetstrokecolor{currentstroke}%
\pgfsetdash{}{0pt}%
\pgfpathmoveto{\pgfqpoint{3.315341in}{3.245338in}}%
\pgfpathcurveto{\pgfqpoint{3.326391in}{3.245338in}}{\pgfqpoint{3.336990in}{3.249728in}}{\pgfqpoint{3.344804in}{3.257542in}}%
\pgfpathcurveto{\pgfqpoint{3.352617in}{3.265355in}}{\pgfqpoint{3.357008in}{3.275954in}}{\pgfqpoint{3.357008in}{3.287005in}}%
\pgfpathcurveto{\pgfqpoint{3.357008in}{3.298055in}}{\pgfqpoint{3.352617in}{3.308654in}}{\pgfqpoint{3.344804in}{3.316467in}}%
\pgfpathcurveto{\pgfqpoint{3.336990in}{3.324281in}}{\pgfqpoint{3.326391in}{3.328671in}}{\pgfqpoint{3.315341in}{3.328671in}}%
\pgfpathcurveto{\pgfqpoint{3.304291in}{3.328671in}}{\pgfqpoint{3.293692in}{3.324281in}}{\pgfqpoint{3.285878in}{3.316467in}}%
\pgfpathcurveto{\pgfqpoint{3.278065in}{3.308654in}}{\pgfqpoint{3.273674in}{3.298055in}}{\pgfqpoint{3.273674in}{3.287005in}}%
\pgfpathcurveto{\pgfqpoint{3.273674in}{3.275954in}}{\pgfqpoint{3.278065in}{3.265355in}}{\pgfqpoint{3.285878in}{3.257542in}}%
\pgfpathcurveto{\pgfqpoint{3.293692in}{3.249728in}}{\pgfqpoint{3.304291in}{3.245338in}}{\pgfqpoint{3.315341in}{3.245338in}}%
\pgfpathclose%
\pgfusepath{stroke,fill}%
\end{pgfscope}%
\begin{pgfscope}%
\pgfpathrectangle{\pgfqpoint{0.648703in}{0.548769in}}{\pgfqpoint{5.112893in}{3.102590in}}%
\pgfusepath{clip}%
\pgfsetbuttcap%
\pgfsetroundjoin%
\definecolor{currentfill}{rgb}{1.000000,0.498039,0.054902}%
\pgfsetfillcolor{currentfill}%
\pgfsetlinewidth{1.003750pt}%
\definecolor{currentstroke}{rgb}{1.000000,0.498039,0.054902}%
\pgfsetstrokecolor{currentstroke}%
\pgfsetdash{}{0pt}%
\pgfpathmoveto{\pgfqpoint{3.337218in}{3.312757in}}%
\pgfpathcurveto{\pgfqpoint{3.348268in}{3.312757in}}{\pgfqpoint{3.358867in}{3.317148in}}{\pgfqpoint{3.366681in}{3.324961in}}%
\pgfpathcurveto{\pgfqpoint{3.374495in}{3.332775in}}{\pgfqpoint{3.378885in}{3.343374in}}{\pgfqpoint{3.378885in}{3.354424in}}%
\pgfpathcurveto{\pgfqpoint{3.378885in}{3.365474in}}{\pgfqpoint{3.374495in}{3.376073in}}{\pgfqpoint{3.366681in}{3.383887in}}%
\pgfpathcurveto{\pgfqpoint{3.358867in}{3.391701in}}{\pgfqpoint{3.348268in}{3.396091in}}{\pgfqpoint{3.337218in}{3.396091in}}%
\pgfpathcurveto{\pgfqpoint{3.326168in}{3.396091in}}{\pgfqpoint{3.315569in}{3.391701in}}{\pgfqpoint{3.307755in}{3.383887in}}%
\pgfpathcurveto{\pgfqpoint{3.299942in}{3.376073in}}{\pgfqpoint{3.295551in}{3.365474in}}{\pgfqpoint{3.295551in}{3.354424in}}%
\pgfpathcurveto{\pgfqpoint{3.295551in}{3.343374in}}{\pgfqpoint{3.299942in}{3.332775in}}{\pgfqpoint{3.307755in}{3.324961in}}%
\pgfpathcurveto{\pgfqpoint{3.315569in}{3.317148in}}{\pgfqpoint{3.326168in}{3.312757in}}{\pgfqpoint{3.337218in}{3.312757in}}%
\pgfpathclose%
\pgfusepath{stroke,fill}%
\end{pgfscope}%
\begin{pgfscope}%
\pgfpathrectangle{\pgfqpoint{0.648703in}{0.548769in}}{\pgfqpoint{5.112893in}{3.102590in}}%
\pgfusepath{clip}%
\pgfsetbuttcap%
\pgfsetroundjoin%
\definecolor{currentfill}{rgb}{1.000000,0.498039,0.054902}%
\pgfsetfillcolor{currentfill}%
\pgfsetlinewidth{1.003750pt}%
\definecolor{currentstroke}{rgb}{1.000000,0.498039,0.054902}%
\pgfsetstrokecolor{currentstroke}%
\pgfsetdash{}{0pt}%
\pgfpathmoveto{\pgfqpoint{3.391776in}{3.211628in}}%
\pgfpathcurveto{\pgfqpoint{3.402826in}{3.211628in}}{\pgfqpoint{3.413425in}{3.216018in}}{\pgfqpoint{3.421239in}{3.223832in}}%
\pgfpathcurveto{\pgfqpoint{3.429053in}{3.231646in}}{\pgfqpoint{3.433443in}{3.242245in}}{\pgfqpoint{3.433443in}{3.253295in}}%
\pgfpathcurveto{\pgfqpoint{3.433443in}{3.264345in}}{\pgfqpoint{3.429053in}{3.274944in}}{\pgfqpoint{3.421239in}{3.282758in}}%
\pgfpathcurveto{\pgfqpoint{3.413425in}{3.290571in}}{\pgfqpoint{3.402826in}{3.294961in}}{\pgfqpoint{3.391776in}{3.294961in}}%
\pgfpathcurveto{\pgfqpoint{3.380726in}{3.294961in}}{\pgfqpoint{3.370127in}{3.290571in}}{\pgfqpoint{3.362313in}{3.282758in}}%
\pgfpathcurveto{\pgfqpoint{3.354500in}{3.274944in}}{\pgfqpoint{3.350110in}{3.264345in}}{\pgfqpoint{3.350110in}{3.253295in}}%
\pgfpathcurveto{\pgfqpoint{3.350110in}{3.242245in}}{\pgfqpoint{3.354500in}{3.231646in}}{\pgfqpoint{3.362313in}{3.223832in}}%
\pgfpathcurveto{\pgfqpoint{3.370127in}{3.216018in}}{\pgfqpoint{3.380726in}{3.211628in}}{\pgfqpoint{3.391776in}{3.211628in}}%
\pgfpathclose%
\pgfusepath{stroke,fill}%
\end{pgfscope}%
\begin{pgfscope}%
\pgfpathrectangle{\pgfqpoint{0.648703in}{0.548769in}}{\pgfqpoint{5.112893in}{3.102590in}}%
\pgfusepath{clip}%
\pgfsetbuttcap%
\pgfsetroundjoin%
\definecolor{currentfill}{rgb}{1.000000,0.498039,0.054902}%
\pgfsetfillcolor{currentfill}%
\pgfsetlinewidth{1.003750pt}%
\definecolor{currentstroke}{rgb}{1.000000,0.498039,0.054902}%
\pgfsetstrokecolor{currentstroke}%
\pgfsetdash{}{0pt}%
\pgfpathmoveto{\pgfqpoint{3.312767in}{3.236910in}}%
\pgfpathcurveto{\pgfqpoint{3.323817in}{3.236910in}}{\pgfqpoint{3.334416in}{3.241301in}}{\pgfqpoint{3.342229in}{3.249114in}}%
\pgfpathcurveto{\pgfqpoint{3.350043in}{3.256928in}}{\pgfqpoint{3.354433in}{3.267527in}}{\pgfqpoint{3.354433in}{3.278577in}}%
\pgfpathcurveto{\pgfqpoint{3.354433in}{3.289627in}}{\pgfqpoint{3.350043in}{3.300226in}}{\pgfqpoint{3.342229in}{3.308040in}}%
\pgfpathcurveto{\pgfqpoint{3.334416in}{3.315854in}}{\pgfqpoint{3.323817in}{3.320244in}}{\pgfqpoint{3.312767in}{3.320244in}}%
\pgfpathcurveto{\pgfqpoint{3.301717in}{3.320244in}}{\pgfqpoint{3.291118in}{3.315854in}}{\pgfqpoint{3.283304in}{3.308040in}}%
\pgfpathcurveto{\pgfqpoint{3.275490in}{3.300226in}}{\pgfqpoint{3.271100in}{3.289627in}}{\pgfqpoint{3.271100in}{3.278577in}}%
\pgfpathcurveto{\pgfqpoint{3.271100in}{3.267527in}}{\pgfqpoint{3.275490in}{3.256928in}}{\pgfqpoint{3.283304in}{3.249114in}}%
\pgfpathcurveto{\pgfqpoint{3.291118in}{3.241301in}}{\pgfqpoint{3.301717in}{3.236910in}}{\pgfqpoint{3.312767in}{3.236910in}}%
\pgfpathclose%
\pgfusepath{stroke,fill}%
\end{pgfscope}%
\begin{pgfscope}%
\pgfpathrectangle{\pgfqpoint{0.648703in}{0.548769in}}{\pgfqpoint{5.112893in}{3.102590in}}%
\pgfusepath{clip}%
\pgfsetbuttcap%
\pgfsetroundjoin%
\definecolor{currentfill}{rgb}{0.121569,0.466667,0.705882}%
\pgfsetfillcolor{currentfill}%
\pgfsetlinewidth{1.003750pt}%
\definecolor{currentstroke}{rgb}{0.121569,0.466667,0.705882}%
\pgfsetstrokecolor{currentstroke}%
\pgfsetdash{}{0pt}%
\pgfpathmoveto{\pgfqpoint{0.831267in}{0.658113in}}%
\pgfpathcurveto{\pgfqpoint{0.842317in}{0.658113in}}{\pgfqpoint{0.852916in}{0.662503in}}{\pgfqpoint{0.860730in}{0.670317in}}%
\pgfpathcurveto{\pgfqpoint{0.868543in}{0.678130in}}{\pgfqpoint{0.872933in}{0.688729in}}{\pgfqpoint{0.872933in}{0.699779in}}%
\pgfpathcurveto{\pgfqpoint{0.872933in}{0.710830in}}{\pgfqpoint{0.868543in}{0.721429in}}{\pgfqpoint{0.860730in}{0.729242in}}%
\pgfpathcurveto{\pgfqpoint{0.852916in}{0.737056in}}{\pgfqpoint{0.842317in}{0.741446in}}{\pgfqpoint{0.831267in}{0.741446in}}%
\pgfpathcurveto{\pgfqpoint{0.820217in}{0.741446in}}{\pgfqpoint{0.809618in}{0.737056in}}{\pgfqpoint{0.801804in}{0.729242in}}%
\pgfpathcurveto{\pgfqpoint{0.793990in}{0.721429in}}{\pgfqpoint{0.789600in}{0.710830in}}{\pgfqpoint{0.789600in}{0.699779in}}%
\pgfpathcurveto{\pgfqpoint{0.789600in}{0.688729in}}{\pgfqpoint{0.793990in}{0.678130in}}{\pgfqpoint{0.801804in}{0.670317in}}%
\pgfpathcurveto{\pgfqpoint{0.809618in}{0.662503in}}{\pgfqpoint{0.820217in}{0.658113in}}{\pgfqpoint{0.831267in}{0.658113in}}%
\pgfpathclose%
\pgfusepath{stroke,fill}%
\end{pgfscope}%
\begin{pgfscope}%
\pgfpathrectangle{\pgfqpoint{0.648703in}{0.548769in}}{\pgfqpoint{5.112893in}{3.102590in}}%
\pgfusepath{clip}%
\pgfsetbuttcap%
\pgfsetroundjoin%
\definecolor{currentfill}{rgb}{0.121569,0.466667,0.705882}%
\pgfsetfillcolor{currentfill}%
\pgfsetlinewidth{1.003750pt}%
\definecolor{currentstroke}{rgb}{0.121569,0.466667,0.705882}%
\pgfsetstrokecolor{currentstroke}%
\pgfsetdash{}{0pt}%
\pgfpathmoveto{\pgfqpoint{0.831253in}{0.658113in}}%
\pgfpathcurveto{\pgfqpoint{0.842303in}{0.658113in}}{\pgfqpoint{0.852902in}{0.662503in}}{\pgfqpoint{0.860716in}{0.670317in}}%
\pgfpathcurveto{\pgfqpoint{0.868530in}{0.678130in}}{\pgfqpoint{0.872920in}{0.688729in}}{\pgfqpoint{0.872920in}{0.699779in}}%
\pgfpathcurveto{\pgfqpoint{0.872920in}{0.710830in}}{\pgfqpoint{0.868530in}{0.721429in}}{\pgfqpoint{0.860716in}{0.729242in}}%
\pgfpathcurveto{\pgfqpoint{0.852902in}{0.737056in}}{\pgfqpoint{0.842303in}{0.741446in}}{\pgfqpoint{0.831253in}{0.741446in}}%
\pgfpathcurveto{\pgfqpoint{0.820203in}{0.741446in}}{\pgfqpoint{0.809604in}{0.737056in}}{\pgfqpoint{0.801790in}{0.729242in}}%
\pgfpathcurveto{\pgfqpoint{0.793977in}{0.721429in}}{\pgfqpoint{0.789586in}{0.710830in}}{\pgfqpoint{0.789586in}{0.699779in}}%
\pgfpathcurveto{\pgfqpoint{0.789586in}{0.688729in}}{\pgfqpoint{0.793977in}{0.678130in}}{\pgfqpoint{0.801790in}{0.670317in}}%
\pgfpathcurveto{\pgfqpoint{0.809604in}{0.662503in}}{\pgfqpoint{0.820203in}{0.658113in}}{\pgfqpoint{0.831253in}{0.658113in}}%
\pgfpathclose%
\pgfusepath{stroke,fill}%
\end{pgfscope}%
\begin{pgfscope}%
\pgfpathrectangle{\pgfqpoint{0.648703in}{0.548769in}}{\pgfqpoint{5.112893in}{3.102590in}}%
\pgfusepath{clip}%
\pgfsetbuttcap%
\pgfsetroundjoin%
\definecolor{currentfill}{rgb}{0.121569,0.466667,0.705882}%
\pgfsetfillcolor{currentfill}%
\pgfsetlinewidth{1.003750pt}%
\definecolor{currentstroke}{rgb}{0.121569,0.466667,0.705882}%
\pgfsetstrokecolor{currentstroke}%
\pgfsetdash{}{0pt}%
\pgfpathmoveto{\pgfqpoint{0.831244in}{0.658113in}}%
\pgfpathcurveto{\pgfqpoint{0.842294in}{0.658113in}}{\pgfqpoint{0.852893in}{0.662503in}}{\pgfqpoint{0.860707in}{0.670317in}}%
\pgfpathcurveto{\pgfqpoint{0.868521in}{0.678130in}}{\pgfqpoint{0.872911in}{0.688729in}}{\pgfqpoint{0.872911in}{0.699779in}}%
\pgfpathcurveto{\pgfqpoint{0.872911in}{0.710830in}}{\pgfqpoint{0.868521in}{0.721429in}}{\pgfqpoint{0.860707in}{0.729242in}}%
\pgfpathcurveto{\pgfqpoint{0.852893in}{0.737056in}}{\pgfqpoint{0.842294in}{0.741446in}}{\pgfqpoint{0.831244in}{0.741446in}}%
\pgfpathcurveto{\pgfqpoint{0.820194in}{0.741446in}}{\pgfqpoint{0.809595in}{0.737056in}}{\pgfqpoint{0.801781in}{0.729242in}}%
\pgfpathcurveto{\pgfqpoint{0.793968in}{0.721429in}}{\pgfqpoint{0.789578in}{0.710830in}}{\pgfqpoint{0.789578in}{0.699779in}}%
\pgfpathcurveto{\pgfqpoint{0.789578in}{0.688729in}}{\pgfqpoint{0.793968in}{0.678130in}}{\pgfqpoint{0.801781in}{0.670317in}}%
\pgfpathcurveto{\pgfqpoint{0.809595in}{0.662503in}}{\pgfqpoint{0.820194in}{0.658113in}}{\pgfqpoint{0.831244in}{0.658113in}}%
\pgfpathclose%
\pgfusepath{stroke,fill}%
\end{pgfscope}%
\begin{pgfscope}%
\pgfpathrectangle{\pgfqpoint{0.648703in}{0.548769in}}{\pgfqpoint{5.112893in}{3.102590in}}%
\pgfusepath{clip}%
\pgfsetbuttcap%
\pgfsetroundjoin%
\definecolor{currentfill}{rgb}{1.000000,0.498039,0.054902}%
\pgfsetfillcolor{currentfill}%
\pgfsetlinewidth{1.003750pt}%
\definecolor{currentstroke}{rgb}{1.000000,0.498039,0.054902}%
\pgfsetstrokecolor{currentstroke}%
\pgfsetdash{}{0pt}%
\pgfpathmoveto{\pgfqpoint{3.329924in}{3.405459in}}%
\pgfpathcurveto{\pgfqpoint{3.340974in}{3.405459in}}{\pgfqpoint{3.351573in}{3.409850in}}{\pgfqpoint{3.359387in}{3.417663in}}%
\pgfpathcurveto{\pgfqpoint{3.367200in}{3.425477in}}{\pgfqpoint{3.371591in}{3.436076in}}{\pgfqpoint{3.371591in}{3.447126in}}%
\pgfpathcurveto{\pgfqpoint{3.371591in}{3.458176in}}{\pgfqpoint{3.367200in}{3.468775in}}{\pgfqpoint{3.359387in}{3.476589in}}%
\pgfpathcurveto{\pgfqpoint{3.351573in}{3.484402in}}{\pgfqpoint{3.340974in}{3.488793in}}{\pgfqpoint{3.329924in}{3.488793in}}%
\pgfpathcurveto{\pgfqpoint{3.318874in}{3.488793in}}{\pgfqpoint{3.308275in}{3.484402in}}{\pgfqpoint{3.300461in}{3.476589in}}%
\pgfpathcurveto{\pgfqpoint{3.292648in}{3.468775in}}{\pgfqpoint{3.288257in}{3.458176in}}{\pgfqpoint{3.288257in}{3.447126in}}%
\pgfpathcurveto{\pgfqpoint{3.288257in}{3.436076in}}{\pgfqpoint{3.292648in}{3.425477in}}{\pgfqpoint{3.300461in}{3.417663in}}%
\pgfpathcurveto{\pgfqpoint{3.308275in}{3.409850in}}{\pgfqpoint{3.318874in}{3.405459in}}{\pgfqpoint{3.329924in}{3.405459in}}%
\pgfpathclose%
\pgfusepath{stroke,fill}%
\end{pgfscope}%
\begin{pgfscope}%
\pgfpathrectangle{\pgfqpoint{0.648703in}{0.548769in}}{\pgfqpoint{5.112893in}{3.102590in}}%
\pgfusepath{clip}%
\pgfsetbuttcap%
\pgfsetroundjoin%
\definecolor{currentfill}{rgb}{1.000000,0.498039,0.054902}%
\pgfsetfillcolor{currentfill}%
\pgfsetlinewidth{1.003750pt}%
\definecolor{currentstroke}{rgb}{1.000000,0.498039,0.054902}%
\pgfsetstrokecolor{currentstroke}%
\pgfsetdash{}{0pt}%
\pgfpathmoveto{\pgfqpoint{3.753216in}{3.194773in}}%
\pgfpathcurveto{\pgfqpoint{3.764266in}{3.194773in}}{\pgfqpoint{3.774865in}{3.199163in}}{\pgfqpoint{3.782679in}{3.206977in}}%
\pgfpathcurveto{\pgfqpoint{3.790492in}{3.214791in}}{\pgfqpoint{3.794883in}{3.225390in}}{\pgfqpoint{3.794883in}{3.236440in}}%
\pgfpathcurveto{\pgfqpoint{3.794883in}{3.247490in}}{\pgfqpoint{3.790492in}{3.258089in}}{\pgfqpoint{3.782679in}{3.265903in}}%
\pgfpathcurveto{\pgfqpoint{3.774865in}{3.273716in}}{\pgfqpoint{3.764266in}{3.278107in}}{\pgfqpoint{3.753216in}{3.278107in}}%
\pgfpathcurveto{\pgfqpoint{3.742166in}{3.278107in}}{\pgfqpoint{3.731567in}{3.273716in}}{\pgfqpoint{3.723753in}{3.265903in}}%
\pgfpathcurveto{\pgfqpoint{3.715940in}{3.258089in}}{\pgfqpoint{3.711549in}{3.247490in}}{\pgfqpoint{3.711549in}{3.236440in}}%
\pgfpathcurveto{\pgfqpoint{3.711549in}{3.225390in}}{\pgfqpoint{3.715940in}{3.214791in}}{\pgfqpoint{3.723753in}{3.206977in}}%
\pgfpathcurveto{\pgfqpoint{3.731567in}{3.199163in}}{\pgfqpoint{3.742166in}{3.194773in}}{\pgfqpoint{3.753216in}{3.194773in}}%
\pgfpathclose%
\pgfusepath{stroke,fill}%
\end{pgfscope}%
\begin{pgfscope}%
\pgfpathrectangle{\pgfqpoint{0.648703in}{0.548769in}}{\pgfqpoint{5.112893in}{3.102590in}}%
\pgfusepath{clip}%
\pgfsetbuttcap%
\pgfsetroundjoin%
\definecolor{currentfill}{rgb}{1.000000,0.498039,0.054902}%
\pgfsetfillcolor{currentfill}%
\pgfsetlinewidth{1.003750pt}%
\definecolor{currentstroke}{rgb}{1.000000,0.498039,0.054902}%
\pgfsetstrokecolor{currentstroke}%
\pgfsetdash{}{0pt}%
\pgfpathmoveto{\pgfqpoint{3.314720in}{3.359108in}}%
\pgfpathcurveto{\pgfqpoint{3.325770in}{3.359108in}}{\pgfqpoint{3.336369in}{3.363499in}}{\pgfqpoint{3.344182in}{3.371312in}}%
\pgfpathcurveto{\pgfqpoint{3.351996in}{3.379126in}}{\pgfqpoint{3.356386in}{3.389725in}}{\pgfqpoint{3.356386in}{3.400775in}}%
\pgfpathcurveto{\pgfqpoint{3.356386in}{3.411825in}}{\pgfqpoint{3.351996in}{3.422424in}}{\pgfqpoint{3.344182in}{3.430238in}}%
\pgfpathcurveto{\pgfqpoint{3.336369in}{3.438051in}}{\pgfqpoint{3.325770in}{3.442442in}}{\pgfqpoint{3.314720in}{3.442442in}}%
\pgfpathcurveto{\pgfqpoint{3.303670in}{3.442442in}}{\pgfqpoint{3.293071in}{3.438051in}}{\pgfqpoint{3.285257in}{3.430238in}}%
\pgfpathcurveto{\pgfqpoint{3.277443in}{3.422424in}}{\pgfqpoint{3.273053in}{3.411825in}}{\pgfqpoint{3.273053in}{3.400775in}}%
\pgfpathcurveto{\pgfqpoint{3.273053in}{3.389725in}}{\pgfqpoint{3.277443in}{3.379126in}}{\pgfqpoint{3.285257in}{3.371312in}}%
\pgfpathcurveto{\pgfqpoint{3.293071in}{3.363499in}}{\pgfqpoint{3.303670in}{3.359108in}}{\pgfqpoint{3.314720in}{3.359108in}}%
\pgfpathclose%
\pgfusepath{stroke,fill}%
\end{pgfscope}%
\begin{pgfscope}%
\pgfpathrectangle{\pgfqpoint{0.648703in}{0.548769in}}{\pgfqpoint{5.112893in}{3.102590in}}%
\pgfusepath{clip}%
\pgfsetbuttcap%
\pgfsetroundjoin%
\definecolor{currentfill}{rgb}{0.121569,0.466667,0.705882}%
\pgfsetfillcolor{currentfill}%
\pgfsetlinewidth{1.003750pt}%
\definecolor{currentstroke}{rgb}{0.121569,0.466667,0.705882}%
\pgfsetstrokecolor{currentstroke}%
\pgfsetdash{}{0pt}%
\pgfpathmoveto{\pgfqpoint{0.831233in}{0.658113in}}%
\pgfpathcurveto{\pgfqpoint{0.842283in}{0.658113in}}{\pgfqpoint{0.852882in}{0.662503in}}{\pgfqpoint{0.860695in}{0.670317in}}%
\pgfpathcurveto{\pgfqpoint{0.868509in}{0.678130in}}{\pgfqpoint{0.872899in}{0.688729in}}{\pgfqpoint{0.872899in}{0.699779in}}%
\pgfpathcurveto{\pgfqpoint{0.872899in}{0.710830in}}{\pgfqpoint{0.868509in}{0.721429in}}{\pgfqpoint{0.860695in}{0.729242in}}%
\pgfpathcurveto{\pgfqpoint{0.852882in}{0.737056in}}{\pgfqpoint{0.842283in}{0.741446in}}{\pgfqpoint{0.831233in}{0.741446in}}%
\pgfpathcurveto{\pgfqpoint{0.820182in}{0.741446in}}{\pgfqpoint{0.809583in}{0.737056in}}{\pgfqpoint{0.801770in}{0.729242in}}%
\pgfpathcurveto{\pgfqpoint{0.793956in}{0.721429in}}{\pgfqpoint{0.789566in}{0.710830in}}{\pgfqpoint{0.789566in}{0.699779in}}%
\pgfpathcurveto{\pgfqpoint{0.789566in}{0.688729in}}{\pgfqpoint{0.793956in}{0.678130in}}{\pgfqpoint{0.801770in}{0.670317in}}%
\pgfpathcurveto{\pgfqpoint{0.809583in}{0.662503in}}{\pgfqpoint{0.820182in}{0.658113in}}{\pgfqpoint{0.831233in}{0.658113in}}%
\pgfpathclose%
\pgfusepath{stroke,fill}%
\end{pgfscope}%
\begin{pgfscope}%
\pgfpathrectangle{\pgfqpoint{0.648703in}{0.548769in}}{\pgfqpoint{5.112893in}{3.102590in}}%
\pgfusepath{clip}%
\pgfsetbuttcap%
\pgfsetroundjoin%
\definecolor{currentfill}{rgb}{0.121569,0.466667,0.705882}%
\pgfsetfillcolor{currentfill}%
\pgfsetlinewidth{1.003750pt}%
\definecolor{currentstroke}{rgb}{0.121569,0.466667,0.705882}%
\pgfsetstrokecolor{currentstroke}%
\pgfsetdash{}{0pt}%
\pgfpathmoveto{\pgfqpoint{1.081770in}{0.805593in}}%
\pgfpathcurveto{\pgfqpoint{1.092820in}{0.805593in}}{\pgfqpoint{1.103419in}{0.809983in}}{\pgfqpoint{1.111232in}{0.817797in}}%
\pgfpathcurveto{\pgfqpoint{1.119046in}{0.825610in}}{\pgfqpoint{1.123436in}{0.836210in}}{\pgfqpoint{1.123436in}{0.847260in}}%
\pgfpathcurveto{\pgfqpoint{1.123436in}{0.858310in}}{\pgfqpoint{1.119046in}{0.868909in}}{\pgfqpoint{1.111232in}{0.876722in}}%
\pgfpathcurveto{\pgfqpoint{1.103419in}{0.884536in}}{\pgfqpoint{1.092820in}{0.888926in}}{\pgfqpoint{1.081770in}{0.888926in}}%
\pgfpathcurveto{\pgfqpoint{1.070720in}{0.888926in}}{\pgfqpoint{1.060120in}{0.884536in}}{\pgfqpoint{1.052307in}{0.876722in}}%
\pgfpathcurveto{\pgfqpoint{1.044493in}{0.868909in}}{\pgfqpoint{1.040103in}{0.858310in}}{\pgfqpoint{1.040103in}{0.847260in}}%
\pgfpathcurveto{\pgfqpoint{1.040103in}{0.836210in}}{\pgfqpoint{1.044493in}{0.825610in}}{\pgfqpoint{1.052307in}{0.817797in}}%
\pgfpathcurveto{\pgfqpoint{1.060120in}{0.809983in}}{\pgfqpoint{1.070720in}{0.805593in}}{\pgfqpoint{1.081770in}{0.805593in}}%
\pgfpathclose%
\pgfusepath{stroke,fill}%
\end{pgfscope}%
\begin{pgfscope}%
\pgfpathrectangle{\pgfqpoint{0.648703in}{0.548769in}}{\pgfqpoint{5.112893in}{3.102590in}}%
\pgfusepath{clip}%
\pgfsetbuttcap%
\pgfsetroundjoin%
\definecolor{currentfill}{rgb}{1.000000,0.498039,0.054902}%
\pgfsetfillcolor{currentfill}%
\pgfsetlinewidth{1.003750pt}%
\definecolor{currentstroke}{rgb}{1.000000,0.498039,0.054902}%
\pgfsetstrokecolor{currentstroke}%
\pgfsetdash{}{0pt}%
\pgfpathmoveto{\pgfqpoint{4.080263in}{3.156850in}}%
\pgfpathcurveto{\pgfqpoint{4.091313in}{3.156850in}}{\pgfqpoint{4.101912in}{3.161240in}}{\pgfqpoint{4.109725in}{3.169054in}}%
\pgfpathcurveto{\pgfqpoint{4.117539in}{3.176867in}}{\pgfqpoint{4.121929in}{3.187466in}}{\pgfqpoint{4.121929in}{3.198516in}}%
\pgfpathcurveto{\pgfqpoint{4.121929in}{3.209567in}}{\pgfqpoint{4.117539in}{3.220166in}}{\pgfqpoint{4.109725in}{3.227979in}}%
\pgfpathcurveto{\pgfqpoint{4.101912in}{3.235793in}}{\pgfqpoint{4.091313in}{3.240183in}}{\pgfqpoint{4.080263in}{3.240183in}}%
\pgfpathcurveto{\pgfqpoint{4.069212in}{3.240183in}}{\pgfqpoint{4.058613in}{3.235793in}}{\pgfqpoint{4.050800in}{3.227979in}}%
\pgfpathcurveto{\pgfqpoint{4.042986in}{3.220166in}}{\pgfqpoint{4.038596in}{3.209567in}}{\pgfqpoint{4.038596in}{3.198516in}}%
\pgfpathcurveto{\pgfqpoint{4.038596in}{3.187466in}}{\pgfqpoint{4.042986in}{3.176867in}}{\pgfqpoint{4.050800in}{3.169054in}}%
\pgfpathcurveto{\pgfqpoint{4.058613in}{3.161240in}}{\pgfqpoint{4.069212in}{3.156850in}}{\pgfqpoint{4.080263in}{3.156850in}}%
\pgfpathclose%
\pgfusepath{stroke,fill}%
\end{pgfscope}%
\begin{pgfscope}%
\pgfpathrectangle{\pgfqpoint{0.648703in}{0.548769in}}{\pgfqpoint{5.112893in}{3.102590in}}%
\pgfusepath{clip}%
\pgfsetbuttcap%
\pgfsetroundjoin%
\definecolor{currentfill}{rgb}{0.121569,0.466667,0.705882}%
\pgfsetfillcolor{currentfill}%
\pgfsetlinewidth{1.003750pt}%
\definecolor{currentstroke}{rgb}{0.121569,0.466667,0.705882}%
\pgfsetstrokecolor{currentstroke}%
\pgfsetdash{}{0pt}%
\pgfpathmoveto{\pgfqpoint{0.831244in}{0.658113in}}%
\pgfpathcurveto{\pgfqpoint{0.842294in}{0.658113in}}{\pgfqpoint{0.852893in}{0.662503in}}{\pgfqpoint{0.860707in}{0.670317in}}%
\pgfpathcurveto{\pgfqpoint{0.868520in}{0.678130in}}{\pgfqpoint{0.872911in}{0.688729in}}{\pgfqpoint{0.872911in}{0.699779in}}%
\pgfpathcurveto{\pgfqpoint{0.872911in}{0.710830in}}{\pgfqpoint{0.868520in}{0.721429in}}{\pgfqpoint{0.860707in}{0.729242in}}%
\pgfpathcurveto{\pgfqpoint{0.852893in}{0.737056in}}{\pgfqpoint{0.842294in}{0.741446in}}{\pgfqpoint{0.831244in}{0.741446in}}%
\pgfpathcurveto{\pgfqpoint{0.820194in}{0.741446in}}{\pgfqpoint{0.809595in}{0.737056in}}{\pgfqpoint{0.801781in}{0.729242in}}%
\pgfpathcurveto{\pgfqpoint{0.793968in}{0.721429in}}{\pgfqpoint{0.789577in}{0.710830in}}{\pgfqpoint{0.789577in}{0.699779in}}%
\pgfpathcurveto{\pgfqpoint{0.789577in}{0.688729in}}{\pgfqpoint{0.793968in}{0.678130in}}{\pgfqpoint{0.801781in}{0.670317in}}%
\pgfpathcurveto{\pgfqpoint{0.809595in}{0.662503in}}{\pgfqpoint{0.820194in}{0.658113in}}{\pgfqpoint{0.831244in}{0.658113in}}%
\pgfpathclose%
\pgfusepath{stroke,fill}%
\end{pgfscope}%
\begin{pgfscope}%
\pgfpathrectangle{\pgfqpoint{0.648703in}{0.548769in}}{\pgfqpoint{5.112893in}{3.102590in}}%
\pgfusepath{clip}%
\pgfsetbuttcap%
\pgfsetroundjoin%
\definecolor{currentfill}{rgb}{0.121569,0.466667,0.705882}%
\pgfsetfillcolor{currentfill}%
\pgfsetlinewidth{1.003750pt}%
\definecolor{currentstroke}{rgb}{0.121569,0.466667,0.705882}%
\pgfsetstrokecolor{currentstroke}%
\pgfsetdash{}{0pt}%
\pgfpathmoveto{\pgfqpoint{0.831244in}{0.658113in}}%
\pgfpathcurveto{\pgfqpoint{0.842294in}{0.658113in}}{\pgfqpoint{0.852893in}{0.662503in}}{\pgfqpoint{0.860707in}{0.670317in}}%
\pgfpathcurveto{\pgfqpoint{0.868520in}{0.678130in}}{\pgfqpoint{0.872911in}{0.688729in}}{\pgfqpoint{0.872911in}{0.699779in}}%
\pgfpathcurveto{\pgfqpoint{0.872911in}{0.710830in}}{\pgfqpoint{0.868520in}{0.721429in}}{\pgfqpoint{0.860707in}{0.729242in}}%
\pgfpathcurveto{\pgfqpoint{0.852893in}{0.737056in}}{\pgfqpoint{0.842294in}{0.741446in}}{\pgfqpoint{0.831244in}{0.741446in}}%
\pgfpathcurveto{\pgfqpoint{0.820194in}{0.741446in}}{\pgfqpoint{0.809595in}{0.737056in}}{\pgfqpoint{0.801781in}{0.729242in}}%
\pgfpathcurveto{\pgfqpoint{0.793968in}{0.721429in}}{\pgfqpoint{0.789577in}{0.710830in}}{\pgfqpoint{0.789577in}{0.699779in}}%
\pgfpathcurveto{\pgfqpoint{0.789577in}{0.688729in}}{\pgfqpoint{0.793968in}{0.678130in}}{\pgfqpoint{0.801781in}{0.670317in}}%
\pgfpathcurveto{\pgfqpoint{0.809595in}{0.662503in}}{\pgfqpoint{0.820194in}{0.658113in}}{\pgfqpoint{0.831244in}{0.658113in}}%
\pgfpathclose%
\pgfusepath{stroke,fill}%
\end{pgfscope}%
\begin{pgfscope}%
\pgfpathrectangle{\pgfqpoint{0.648703in}{0.548769in}}{\pgfqpoint{5.112893in}{3.102590in}}%
\pgfusepath{clip}%
\pgfsetbuttcap%
\pgfsetroundjoin%
\definecolor{currentfill}{rgb}{0.121569,0.466667,0.705882}%
\pgfsetfillcolor{currentfill}%
\pgfsetlinewidth{1.003750pt}%
\definecolor{currentstroke}{rgb}{0.121569,0.466667,0.705882}%
\pgfsetstrokecolor{currentstroke}%
\pgfsetdash{}{0pt}%
\pgfpathmoveto{\pgfqpoint{3.126357in}{2.516364in}}%
\pgfpathcurveto{\pgfqpoint{3.137407in}{2.516364in}}{\pgfqpoint{3.148006in}{2.520754in}}{\pgfqpoint{3.155820in}{2.528568in}}%
\pgfpathcurveto{\pgfqpoint{3.163634in}{2.536382in}}{\pgfqpoint{3.168024in}{2.546981in}}{\pgfqpoint{3.168024in}{2.558031in}}%
\pgfpathcurveto{\pgfqpoint{3.168024in}{2.569081in}}{\pgfqpoint{3.163634in}{2.579680in}}{\pgfqpoint{3.155820in}{2.587493in}}%
\pgfpathcurveto{\pgfqpoint{3.148006in}{2.595307in}}{\pgfqpoint{3.137407in}{2.599697in}}{\pgfqpoint{3.126357in}{2.599697in}}%
\pgfpathcurveto{\pgfqpoint{3.115307in}{2.599697in}}{\pgfqpoint{3.104708in}{2.595307in}}{\pgfqpoint{3.096894in}{2.587493in}}%
\pgfpathcurveto{\pgfqpoint{3.089081in}{2.579680in}}{\pgfqpoint{3.084691in}{2.569081in}}{\pgfqpoint{3.084691in}{2.558031in}}%
\pgfpathcurveto{\pgfqpoint{3.084691in}{2.546981in}}{\pgfqpoint{3.089081in}{2.536382in}}{\pgfqpoint{3.096894in}{2.528568in}}%
\pgfpathcurveto{\pgfqpoint{3.104708in}{2.520754in}}{\pgfqpoint{3.115307in}{2.516364in}}{\pgfqpoint{3.126357in}{2.516364in}}%
\pgfpathclose%
\pgfusepath{stroke,fill}%
\end{pgfscope}%
\begin{pgfscope}%
\pgfpathrectangle{\pgfqpoint{0.648703in}{0.548769in}}{\pgfqpoint{5.112893in}{3.102590in}}%
\pgfusepath{clip}%
\pgfsetbuttcap%
\pgfsetroundjoin%
\definecolor{currentfill}{rgb}{1.000000,0.498039,0.054902}%
\pgfsetfillcolor{currentfill}%
\pgfsetlinewidth{1.003750pt}%
\definecolor{currentstroke}{rgb}{1.000000,0.498039,0.054902}%
\pgfsetstrokecolor{currentstroke}%
\pgfsetdash{}{0pt}%
\pgfpathmoveto{\pgfqpoint{4.243867in}{3.207414in}}%
\pgfpathcurveto{\pgfqpoint{4.254917in}{3.207414in}}{\pgfqpoint{4.265516in}{3.211805in}}{\pgfqpoint{4.273330in}{3.219618in}}%
\pgfpathcurveto{\pgfqpoint{4.281144in}{3.227432in}}{\pgfqpoint{4.285534in}{3.238031in}}{\pgfqpoint{4.285534in}{3.249081in}}%
\pgfpathcurveto{\pgfqpoint{4.285534in}{3.260131in}}{\pgfqpoint{4.281144in}{3.270730in}}{\pgfqpoint{4.273330in}{3.278544in}}%
\pgfpathcurveto{\pgfqpoint{4.265516in}{3.286357in}}{\pgfqpoint{4.254917in}{3.290748in}}{\pgfqpoint{4.243867in}{3.290748in}}%
\pgfpathcurveto{\pgfqpoint{4.232817in}{3.290748in}}{\pgfqpoint{4.222218in}{3.286357in}}{\pgfqpoint{4.214404in}{3.278544in}}%
\pgfpathcurveto{\pgfqpoint{4.206591in}{3.270730in}}{\pgfqpoint{4.202201in}{3.260131in}}{\pgfqpoint{4.202201in}{3.249081in}}%
\pgfpathcurveto{\pgfqpoint{4.202201in}{3.238031in}}{\pgfqpoint{4.206591in}{3.227432in}}{\pgfqpoint{4.214404in}{3.219618in}}%
\pgfpathcurveto{\pgfqpoint{4.222218in}{3.211805in}}{\pgfqpoint{4.232817in}{3.207414in}}{\pgfqpoint{4.243867in}{3.207414in}}%
\pgfpathclose%
\pgfusepath{stroke,fill}%
\end{pgfscope}%
\begin{pgfscope}%
\pgfpathrectangle{\pgfqpoint{0.648703in}{0.548769in}}{\pgfqpoint{5.112893in}{3.102590in}}%
\pgfusepath{clip}%
\pgfsetbuttcap%
\pgfsetroundjoin%
\definecolor{currentfill}{rgb}{1.000000,0.498039,0.054902}%
\pgfsetfillcolor{currentfill}%
\pgfsetlinewidth{1.003750pt}%
\definecolor{currentstroke}{rgb}{1.000000,0.498039,0.054902}%
\pgfsetstrokecolor{currentstroke}%
\pgfsetdash{}{0pt}%
\pgfpathmoveto{\pgfqpoint{2.631280in}{3.198987in}}%
\pgfpathcurveto{\pgfqpoint{2.642330in}{3.198987in}}{\pgfqpoint{2.652929in}{3.203377in}}{\pgfqpoint{2.660743in}{3.211191in}}%
\pgfpathcurveto{\pgfqpoint{2.668556in}{3.219004in}}{\pgfqpoint{2.672946in}{3.229603in}}{\pgfqpoint{2.672946in}{3.240654in}}%
\pgfpathcurveto{\pgfqpoint{2.672946in}{3.251704in}}{\pgfqpoint{2.668556in}{3.262303in}}{\pgfqpoint{2.660743in}{3.270116in}}%
\pgfpathcurveto{\pgfqpoint{2.652929in}{3.277930in}}{\pgfqpoint{2.642330in}{3.282320in}}{\pgfqpoint{2.631280in}{3.282320in}}%
\pgfpathcurveto{\pgfqpoint{2.620230in}{3.282320in}}{\pgfqpoint{2.609631in}{3.277930in}}{\pgfqpoint{2.601817in}{3.270116in}}%
\pgfpathcurveto{\pgfqpoint{2.594003in}{3.262303in}}{\pgfqpoint{2.589613in}{3.251704in}}{\pgfqpoint{2.589613in}{3.240654in}}%
\pgfpathcurveto{\pgfqpoint{2.589613in}{3.229603in}}{\pgfqpoint{2.594003in}{3.219004in}}{\pgfqpoint{2.601817in}{3.211191in}}%
\pgfpathcurveto{\pgfqpoint{2.609631in}{3.203377in}}{\pgfqpoint{2.620230in}{3.198987in}}{\pgfqpoint{2.631280in}{3.198987in}}%
\pgfpathclose%
\pgfusepath{stroke,fill}%
\end{pgfscope}%
\begin{pgfscope}%
\pgfpathrectangle{\pgfqpoint{0.648703in}{0.548769in}}{\pgfqpoint{5.112893in}{3.102590in}}%
\pgfusepath{clip}%
\pgfsetbuttcap%
\pgfsetroundjoin%
\definecolor{currentfill}{rgb}{1.000000,0.498039,0.054902}%
\pgfsetfillcolor{currentfill}%
\pgfsetlinewidth{1.003750pt}%
\definecolor{currentstroke}{rgb}{1.000000,0.498039,0.054902}%
\pgfsetstrokecolor{currentstroke}%
\pgfsetdash{}{0pt}%
\pgfpathmoveto{\pgfqpoint{3.398849in}{3.249552in}}%
\pgfpathcurveto{\pgfqpoint{3.409899in}{3.249552in}}{\pgfqpoint{3.420498in}{3.253942in}}{\pgfqpoint{3.428311in}{3.261755in}}%
\pgfpathcurveto{\pgfqpoint{3.436125in}{3.269569in}}{\pgfqpoint{3.440515in}{3.280168in}}{\pgfqpoint{3.440515in}{3.291218in}}%
\pgfpathcurveto{\pgfqpoint{3.440515in}{3.302268in}}{\pgfqpoint{3.436125in}{3.312867in}}{\pgfqpoint{3.428311in}{3.320681in}}%
\pgfpathcurveto{\pgfqpoint{3.420498in}{3.328495in}}{\pgfqpoint{3.409899in}{3.332885in}}{\pgfqpoint{3.398849in}{3.332885in}}%
\pgfpathcurveto{\pgfqpoint{3.387798in}{3.332885in}}{\pgfqpoint{3.377199in}{3.328495in}}{\pgfqpoint{3.369386in}{3.320681in}}%
\pgfpathcurveto{\pgfqpoint{3.361572in}{3.312867in}}{\pgfqpoint{3.357182in}{3.302268in}}{\pgfqpoint{3.357182in}{3.291218in}}%
\pgfpathcurveto{\pgfqpoint{3.357182in}{3.280168in}}{\pgfqpoint{3.361572in}{3.269569in}}{\pgfqpoint{3.369386in}{3.261755in}}%
\pgfpathcurveto{\pgfqpoint{3.377199in}{3.253942in}}{\pgfqpoint{3.387798in}{3.249552in}}{\pgfqpoint{3.398849in}{3.249552in}}%
\pgfpathclose%
\pgfusepath{stroke,fill}%
\end{pgfscope}%
\begin{pgfscope}%
\pgfpathrectangle{\pgfqpoint{0.648703in}{0.548769in}}{\pgfqpoint{5.112893in}{3.102590in}}%
\pgfusepath{clip}%
\pgfsetbuttcap%
\pgfsetroundjoin%
\definecolor{currentfill}{rgb}{1.000000,0.498039,0.054902}%
\pgfsetfillcolor{currentfill}%
\pgfsetlinewidth{1.003750pt}%
\definecolor{currentstroke}{rgb}{1.000000,0.498039,0.054902}%
\pgfsetstrokecolor{currentstroke}%
\pgfsetdash{}{0pt}%
\pgfpathmoveto{\pgfqpoint{3.083996in}{3.190560in}}%
\pgfpathcurveto{\pgfqpoint{3.095046in}{3.190560in}}{\pgfqpoint{3.105645in}{3.194950in}}{\pgfqpoint{3.113459in}{3.202763in}}%
\pgfpathcurveto{\pgfqpoint{3.121272in}{3.210577in}}{\pgfqpoint{3.125663in}{3.221176in}}{\pgfqpoint{3.125663in}{3.232226in}}%
\pgfpathcurveto{\pgfqpoint{3.125663in}{3.243276in}}{\pgfqpoint{3.121272in}{3.253875in}}{\pgfqpoint{3.113459in}{3.261689in}}%
\pgfpathcurveto{\pgfqpoint{3.105645in}{3.269503in}}{\pgfqpoint{3.095046in}{3.273893in}}{\pgfqpoint{3.083996in}{3.273893in}}%
\pgfpathcurveto{\pgfqpoint{3.072946in}{3.273893in}}{\pgfqpoint{3.062347in}{3.269503in}}{\pgfqpoint{3.054533in}{3.261689in}}%
\pgfpathcurveto{\pgfqpoint{3.046720in}{3.253875in}}{\pgfqpoint{3.042329in}{3.243276in}}{\pgfqpoint{3.042329in}{3.232226in}}%
\pgfpathcurveto{\pgfqpoint{3.042329in}{3.221176in}}{\pgfqpoint{3.046720in}{3.210577in}}{\pgfqpoint{3.054533in}{3.202763in}}%
\pgfpathcurveto{\pgfqpoint{3.062347in}{3.194950in}}{\pgfqpoint{3.072946in}{3.190560in}}{\pgfqpoint{3.083996in}{3.190560in}}%
\pgfpathclose%
\pgfusepath{stroke,fill}%
\end{pgfscope}%
\begin{pgfscope}%
\pgfpathrectangle{\pgfqpoint{0.648703in}{0.548769in}}{\pgfqpoint{5.112893in}{3.102590in}}%
\pgfusepath{clip}%
\pgfsetbuttcap%
\pgfsetroundjoin%
\definecolor{currentfill}{rgb}{1.000000,0.498039,0.054902}%
\pgfsetfillcolor{currentfill}%
\pgfsetlinewidth{1.003750pt}%
\definecolor{currentstroke}{rgb}{1.000000,0.498039,0.054902}%
\pgfsetstrokecolor{currentstroke}%
\pgfsetdash{}{0pt}%
\pgfpathmoveto{\pgfqpoint{3.589768in}{3.182132in}}%
\pgfpathcurveto{\pgfqpoint{3.600818in}{3.182132in}}{\pgfqpoint{3.611417in}{3.186522in}}{\pgfqpoint{3.619231in}{3.194336in}}%
\pgfpathcurveto{\pgfqpoint{3.627044in}{3.202150in}}{\pgfqpoint{3.631435in}{3.212749in}}{\pgfqpoint{3.631435in}{3.223799in}}%
\pgfpathcurveto{\pgfqpoint{3.631435in}{3.234849in}}{\pgfqpoint{3.627044in}{3.245448in}}{\pgfqpoint{3.619231in}{3.253262in}}%
\pgfpathcurveto{\pgfqpoint{3.611417in}{3.261075in}}{\pgfqpoint{3.600818in}{3.265465in}}{\pgfqpoint{3.589768in}{3.265465in}}%
\pgfpathcurveto{\pgfqpoint{3.578718in}{3.265465in}}{\pgfqpoint{3.568119in}{3.261075in}}{\pgfqpoint{3.560305in}{3.253262in}}%
\pgfpathcurveto{\pgfqpoint{3.552491in}{3.245448in}}{\pgfqpoint{3.548101in}{3.234849in}}{\pgfqpoint{3.548101in}{3.223799in}}%
\pgfpathcurveto{\pgfqpoint{3.548101in}{3.212749in}}{\pgfqpoint{3.552491in}{3.202150in}}{\pgfqpoint{3.560305in}{3.194336in}}%
\pgfpathcurveto{\pgfqpoint{3.568119in}{3.186522in}}{\pgfqpoint{3.578718in}{3.182132in}}{\pgfqpoint{3.589768in}{3.182132in}}%
\pgfpathclose%
\pgfusepath{stroke,fill}%
\end{pgfscope}%
\begin{pgfscope}%
\pgfpathrectangle{\pgfqpoint{0.648703in}{0.548769in}}{\pgfqpoint{5.112893in}{3.102590in}}%
\pgfusepath{clip}%
\pgfsetbuttcap%
\pgfsetroundjoin%
\definecolor{currentfill}{rgb}{0.839216,0.152941,0.156863}%
\pgfsetfillcolor{currentfill}%
\pgfsetlinewidth{1.003750pt}%
\definecolor{currentstroke}{rgb}{0.839216,0.152941,0.156863}%
\pgfsetstrokecolor{currentstroke}%
\pgfsetdash{}{0pt}%
\pgfpathmoveto{\pgfqpoint{2.519905in}{3.203201in}}%
\pgfpathcurveto{\pgfqpoint{2.530955in}{3.203201in}}{\pgfqpoint{2.541554in}{3.207591in}}{\pgfqpoint{2.549368in}{3.215405in}}%
\pgfpathcurveto{\pgfqpoint{2.557182in}{3.223218in}}{\pgfqpoint{2.561572in}{3.233817in}}{\pgfqpoint{2.561572in}{3.244867in}}%
\pgfpathcurveto{\pgfqpoint{2.561572in}{3.255917in}}{\pgfqpoint{2.557182in}{3.266516in}}{\pgfqpoint{2.549368in}{3.274330in}}%
\pgfpathcurveto{\pgfqpoint{2.541554in}{3.282144in}}{\pgfqpoint{2.530955in}{3.286534in}}{\pgfqpoint{2.519905in}{3.286534in}}%
\pgfpathcurveto{\pgfqpoint{2.508855in}{3.286534in}}{\pgfqpoint{2.498256in}{3.282144in}}{\pgfqpoint{2.490443in}{3.274330in}}%
\pgfpathcurveto{\pgfqpoint{2.482629in}{3.266516in}}{\pgfqpoint{2.478239in}{3.255917in}}{\pgfqpoint{2.478239in}{3.244867in}}%
\pgfpathcurveto{\pgfqpoint{2.478239in}{3.233817in}}{\pgfqpoint{2.482629in}{3.223218in}}{\pgfqpoint{2.490443in}{3.215405in}}%
\pgfpathcurveto{\pgfqpoint{2.498256in}{3.207591in}}{\pgfqpoint{2.508855in}{3.203201in}}{\pgfqpoint{2.519905in}{3.203201in}}%
\pgfpathclose%
\pgfusepath{stroke,fill}%
\end{pgfscope}%
\begin{pgfscope}%
\pgfpathrectangle{\pgfqpoint{0.648703in}{0.548769in}}{\pgfqpoint{5.112893in}{3.102590in}}%
\pgfusepath{clip}%
\pgfsetbuttcap%
\pgfsetroundjoin%
\definecolor{currentfill}{rgb}{1.000000,0.498039,0.054902}%
\pgfsetfillcolor{currentfill}%
\pgfsetlinewidth{1.003750pt}%
\definecolor{currentstroke}{rgb}{1.000000,0.498039,0.054902}%
\pgfsetstrokecolor{currentstroke}%
\pgfsetdash{}{0pt}%
\pgfpathmoveto{\pgfqpoint{3.627522in}{3.186346in}}%
\pgfpathcurveto{\pgfqpoint{3.638573in}{3.186346in}}{\pgfqpoint{3.649172in}{3.190736in}}{\pgfqpoint{3.656985in}{3.198550in}}%
\pgfpathcurveto{\pgfqpoint{3.664799in}{3.206363in}}{\pgfqpoint{3.669189in}{3.216962in}}{\pgfqpoint{3.669189in}{3.228012in}}%
\pgfpathcurveto{\pgfqpoint{3.669189in}{3.239063in}}{\pgfqpoint{3.664799in}{3.249662in}}{\pgfqpoint{3.656985in}{3.257475in}}%
\pgfpathcurveto{\pgfqpoint{3.649172in}{3.265289in}}{\pgfqpoint{3.638573in}{3.269679in}}{\pgfqpoint{3.627522in}{3.269679in}}%
\pgfpathcurveto{\pgfqpoint{3.616472in}{3.269679in}}{\pgfqpoint{3.605873in}{3.265289in}}{\pgfqpoint{3.598060in}{3.257475in}}%
\pgfpathcurveto{\pgfqpoint{3.590246in}{3.249662in}}{\pgfqpoint{3.585856in}{3.239063in}}{\pgfqpoint{3.585856in}{3.228012in}}%
\pgfpathcurveto{\pgfqpoint{3.585856in}{3.216962in}}{\pgfqpoint{3.590246in}{3.206363in}}{\pgfqpoint{3.598060in}{3.198550in}}%
\pgfpathcurveto{\pgfqpoint{3.605873in}{3.190736in}}{\pgfqpoint{3.616472in}{3.186346in}}{\pgfqpoint{3.627522in}{3.186346in}}%
\pgfpathclose%
\pgfusepath{stroke,fill}%
\end{pgfscope}%
\begin{pgfscope}%
\pgfpathrectangle{\pgfqpoint{0.648703in}{0.548769in}}{\pgfqpoint{5.112893in}{3.102590in}}%
\pgfusepath{clip}%
\pgfsetbuttcap%
\pgfsetroundjoin%
\definecolor{currentfill}{rgb}{1.000000,0.498039,0.054902}%
\pgfsetfillcolor{currentfill}%
\pgfsetlinewidth{1.003750pt}%
\definecolor{currentstroke}{rgb}{1.000000,0.498039,0.054902}%
\pgfsetstrokecolor{currentstroke}%
\pgfsetdash{}{0pt}%
\pgfpathmoveto{\pgfqpoint{3.315477in}{3.279048in}}%
\pgfpathcurveto{\pgfqpoint{3.326527in}{3.279048in}}{\pgfqpoint{3.337126in}{3.283438in}}{\pgfqpoint{3.344940in}{3.291252in}}%
\pgfpathcurveto{\pgfqpoint{3.352753in}{3.299065in}}{\pgfqpoint{3.357144in}{3.309664in}}{\pgfqpoint{3.357144in}{3.320714in}}%
\pgfpathcurveto{\pgfqpoint{3.357144in}{3.331764in}}{\pgfqpoint{3.352753in}{3.342363in}}{\pgfqpoint{3.344940in}{3.350177in}}%
\pgfpathcurveto{\pgfqpoint{3.337126in}{3.357991in}}{\pgfqpoint{3.326527in}{3.362381in}}{\pgfqpoint{3.315477in}{3.362381in}}%
\pgfpathcurveto{\pgfqpoint{3.304427in}{3.362381in}}{\pgfqpoint{3.293828in}{3.357991in}}{\pgfqpoint{3.286014in}{3.350177in}}%
\pgfpathcurveto{\pgfqpoint{3.278201in}{3.342363in}}{\pgfqpoint{3.273810in}{3.331764in}}{\pgfqpoint{3.273810in}{3.320714in}}%
\pgfpathcurveto{\pgfqpoint{3.273810in}{3.309664in}}{\pgfqpoint{3.278201in}{3.299065in}}{\pgfqpoint{3.286014in}{3.291252in}}%
\pgfpathcurveto{\pgfqpoint{3.293828in}{3.283438in}}{\pgfqpoint{3.304427in}{3.279048in}}{\pgfqpoint{3.315477in}{3.279048in}}%
\pgfpathclose%
\pgfusepath{stroke,fill}%
\end{pgfscope}%
\begin{pgfscope}%
\pgfpathrectangle{\pgfqpoint{0.648703in}{0.548769in}}{\pgfqpoint{5.112893in}{3.102590in}}%
\pgfusepath{clip}%
\pgfsetbuttcap%
\pgfsetroundjoin%
\definecolor{currentfill}{rgb}{1.000000,0.498039,0.054902}%
\pgfsetfillcolor{currentfill}%
\pgfsetlinewidth{1.003750pt}%
\definecolor{currentstroke}{rgb}{1.000000,0.498039,0.054902}%
\pgfsetstrokecolor{currentstroke}%
\pgfsetdash{}{0pt}%
\pgfpathmoveto{\pgfqpoint{2.202767in}{3.182132in}}%
\pgfpathcurveto{\pgfqpoint{2.213817in}{3.182132in}}{\pgfqpoint{2.224416in}{3.186522in}}{\pgfqpoint{2.232230in}{3.194336in}}%
\pgfpathcurveto{\pgfqpoint{2.240043in}{3.202150in}}{\pgfqpoint{2.244433in}{3.212749in}}{\pgfqpoint{2.244433in}{3.223799in}}%
\pgfpathcurveto{\pgfqpoint{2.244433in}{3.234849in}}{\pgfqpoint{2.240043in}{3.245448in}}{\pgfqpoint{2.232230in}{3.253262in}}%
\pgfpathcurveto{\pgfqpoint{2.224416in}{3.261075in}}{\pgfqpoint{2.213817in}{3.265465in}}{\pgfqpoint{2.202767in}{3.265465in}}%
\pgfpathcurveto{\pgfqpoint{2.191717in}{3.265465in}}{\pgfqpoint{2.181118in}{3.261075in}}{\pgfqpoint{2.173304in}{3.253262in}}%
\pgfpathcurveto{\pgfqpoint{2.165490in}{3.245448in}}{\pgfqpoint{2.161100in}{3.234849in}}{\pgfqpoint{2.161100in}{3.223799in}}%
\pgfpathcurveto{\pgfqpoint{2.161100in}{3.212749in}}{\pgfqpoint{2.165490in}{3.202150in}}{\pgfqpoint{2.173304in}{3.194336in}}%
\pgfpathcurveto{\pgfqpoint{2.181118in}{3.186522in}}{\pgfqpoint{2.191717in}{3.182132in}}{\pgfqpoint{2.202767in}{3.182132in}}%
\pgfpathclose%
\pgfusepath{stroke,fill}%
\end{pgfscope}%
\begin{pgfscope}%
\pgfpathrectangle{\pgfqpoint{0.648703in}{0.548769in}}{\pgfqpoint{5.112893in}{3.102590in}}%
\pgfusepath{clip}%
\pgfsetbuttcap%
\pgfsetroundjoin%
\definecolor{currentfill}{rgb}{1.000000,0.498039,0.054902}%
\pgfsetfillcolor{currentfill}%
\pgfsetlinewidth{1.003750pt}%
\definecolor{currentstroke}{rgb}{1.000000,0.498039,0.054902}%
\pgfsetstrokecolor{currentstroke}%
\pgfsetdash{}{0pt}%
\pgfpathmoveto{\pgfqpoint{3.404776in}{3.220056in}}%
\pgfpathcurveto{\pgfqpoint{3.415827in}{3.220056in}}{\pgfqpoint{3.426426in}{3.224446in}}{\pgfqpoint{3.434239in}{3.232259in}}%
\pgfpathcurveto{\pgfqpoint{3.442053in}{3.240073in}}{\pgfqpoint{3.446443in}{3.250672in}}{\pgfqpoint{3.446443in}{3.261722in}}%
\pgfpathcurveto{\pgfqpoint{3.446443in}{3.272772in}}{\pgfqpoint{3.442053in}{3.283371in}}{\pgfqpoint{3.434239in}{3.291185in}}%
\pgfpathcurveto{\pgfqpoint{3.426426in}{3.298999in}}{\pgfqpoint{3.415827in}{3.303389in}}{\pgfqpoint{3.404776in}{3.303389in}}%
\pgfpathcurveto{\pgfqpoint{3.393726in}{3.303389in}}{\pgfqpoint{3.383127in}{3.298999in}}{\pgfqpoint{3.375314in}{3.291185in}}%
\pgfpathcurveto{\pgfqpoint{3.367500in}{3.283371in}}{\pgfqpoint{3.363110in}{3.272772in}}{\pgfqpoint{3.363110in}{3.261722in}}%
\pgfpathcurveto{\pgfqpoint{3.363110in}{3.250672in}}{\pgfqpoint{3.367500in}{3.240073in}}{\pgfqpoint{3.375314in}{3.232259in}}%
\pgfpathcurveto{\pgfqpoint{3.383127in}{3.224446in}}{\pgfqpoint{3.393726in}{3.220056in}}{\pgfqpoint{3.404776in}{3.220056in}}%
\pgfpathclose%
\pgfusepath{stroke,fill}%
\end{pgfscope}%
\begin{pgfscope}%
\pgfpathrectangle{\pgfqpoint{0.648703in}{0.548769in}}{\pgfqpoint{5.112893in}{3.102590in}}%
\pgfusepath{clip}%
\pgfsetbuttcap%
\pgfsetroundjoin%
\definecolor{currentfill}{rgb}{1.000000,0.498039,0.054902}%
\pgfsetfillcolor{currentfill}%
\pgfsetlinewidth{1.003750pt}%
\definecolor{currentstroke}{rgb}{1.000000,0.498039,0.054902}%
\pgfsetstrokecolor{currentstroke}%
\pgfsetdash{}{0pt}%
\pgfpathmoveto{\pgfqpoint{3.356370in}{3.468665in}}%
\pgfpathcurveto{\pgfqpoint{3.367420in}{3.468665in}}{\pgfqpoint{3.378019in}{3.473055in}}{\pgfqpoint{3.385833in}{3.480869in}}%
\pgfpathcurveto{\pgfqpoint{3.393646in}{3.488683in}}{\pgfqpoint{3.398037in}{3.499282in}}{\pgfqpoint{3.398037in}{3.510332in}}%
\pgfpathcurveto{\pgfqpoint{3.398037in}{3.521382in}}{\pgfqpoint{3.393646in}{3.531981in}}{\pgfqpoint{3.385833in}{3.539795in}}%
\pgfpathcurveto{\pgfqpoint{3.378019in}{3.547608in}}{\pgfqpoint{3.367420in}{3.551998in}}{\pgfqpoint{3.356370in}{3.551998in}}%
\pgfpathcurveto{\pgfqpoint{3.345320in}{3.551998in}}{\pgfqpoint{3.334721in}{3.547608in}}{\pgfqpoint{3.326907in}{3.539795in}}%
\pgfpathcurveto{\pgfqpoint{3.319093in}{3.531981in}}{\pgfqpoint{3.314703in}{3.521382in}}{\pgfqpoint{3.314703in}{3.510332in}}%
\pgfpathcurveto{\pgfqpoint{3.314703in}{3.499282in}}{\pgfqpoint{3.319093in}{3.488683in}}{\pgfqpoint{3.326907in}{3.480869in}}%
\pgfpathcurveto{\pgfqpoint{3.334721in}{3.473055in}}{\pgfqpoint{3.345320in}{3.468665in}}{\pgfqpoint{3.356370in}{3.468665in}}%
\pgfpathclose%
\pgfusepath{stroke,fill}%
\end{pgfscope}%
\begin{pgfscope}%
\pgfpathrectangle{\pgfqpoint{0.648703in}{0.548769in}}{\pgfqpoint{5.112893in}{3.102590in}}%
\pgfusepath{clip}%
\pgfsetbuttcap%
\pgfsetroundjoin%
\definecolor{currentfill}{rgb}{1.000000,0.498039,0.054902}%
\pgfsetfillcolor{currentfill}%
\pgfsetlinewidth{1.003750pt}%
\definecolor{currentstroke}{rgb}{1.000000,0.498039,0.054902}%
\pgfsetstrokecolor{currentstroke}%
\pgfsetdash{}{0pt}%
\pgfpathmoveto{\pgfqpoint{3.417686in}{3.190560in}}%
\pgfpathcurveto{\pgfqpoint{3.428736in}{3.190560in}}{\pgfqpoint{3.439335in}{3.194950in}}{\pgfqpoint{3.447148in}{3.202763in}}%
\pgfpathcurveto{\pgfqpoint{3.454962in}{3.210577in}}{\pgfqpoint{3.459352in}{3.221176in}}{\pgfqpoint{3.459352in}{3.232226in}}%
\pgfpathcurveto{\pgfqpoint{3.459352in}{3.243276in}}{\pgfqpoint{3.454962in}{3.253875in}}{\pgfqpoint{3.447148in}{3.261689in}}%
\pgfpathcurveto{\pgfqpoint{3.439335in}{3.269503in}}{\pgfqpoint{3.428736in}{3.273893in}}{\pgfqpoint{3.417686in}{3.273893in}}%
\pgfpathcurveto{\pgfqpoint{3.406636in}{3.273893in}}{\pgfqpoint{3.396037in}{3.269503in}}{\pgfqpoint{3.388223in}{3.261689in}}%
\pgfpathcurveto{\pgfqpoint{3.380409in}{3.253875in}}{\pgfqpoint{3.376019in}{3.243276in}}{\pgfqpoint{3.376019in}{3.232226in}}%
\pgfpathcurveto{\pgfqpoint{3.376019in}{3.221176in}}{\pgfqpoint{3.380409in}{3.210577in}}{\pgfqpoint{3.388223in}{3.202763in}}%
\pgfpathcurveto{\pgfqpoint{3.396037in}{3.194950in}}{\pgfqpoint{3.406636in}{3.190560in}}{\pgfqpoint{3.417686in}{3.190560in}}%
\pgfpathclose%
\pgfusepath{stroke,fill}%
\end{pgfscope}%
\begin{pgfscope}%
\pgfpathrectangle{\pgfqpoint{0.648703in}{0.548769in}}{\pgfqpoint{5.112893in}{3.102590in}}%
\pgfusepath{clip}%
\pgfsetbuttcap%
\pgfsetroundjoin%
\definecolor{currentfill}{rgb}{1.000000,0.498039,0.054902}%
\pgfsetfillcolor{currentfill}%
\pgfsetlinewidth{1.003750pt}%
\definecolor{currentstroke}{rgb}{1.000000,0.498039,0.054902}%
\pgfsetstrokecolor{currentstroke}%
\pgfsetdash{}{0pt}%
\pgfpathmoveto{\pgfqpoint{2.242399in}{3.190560in}}%
\pgfpathcurveto{\pgfqpoint{2.253449in}{3.190560in}}{\pgfqpoint{2.264048in}{3.194950in}}{\pgfqpoint{2.271862in}{3.202763in}}%
\pgfpathcurveto{\pgfqpoint{2.279675in}{3.210577in}}{\pgfqpoint{2.284066in}{3.221176in}}{\pgfqpoint{2.284066in}{3.232226in}}%
\pgfpathcurveto{\pgfqpoint{2.284066in}{3.243276in}}{\pgfqpoint{2.279675in}{3.253875in}}{\pgfqpoint{2.271862in}{3.261689in}}%
\pgfpathcurveto{\pgfqpoint{2.264048in}{3.269503in}}{\pgfqpoint{2.253449in}{3.273893in}}{\pgfqpoint{2.242399in}{3.273893in}}%
\pgfpathcurveto{\pgfqpoint{2.231349in}{3.273893in}}{\pgfqpoint{2.220750in}{3.269503in}}{\pgfqpoint{2.212936in}{3.261689in}}%
\pgfpathcurveto{\pgfqpoint{2.205123in}{3.253875in}}{\pgfqpoint{2.200732in}{3.243276in}}{\pgfqpoint{2.200732in}{3.232226in}}%
\pgfpathcurveto{\pgfqpoint{2.200732in}{3.221176in}}{\pgfqpoint{2.205123in}{3.210577in}}{\pgfqpoint{2.212936in}{3.202763in}}%
\pgfpathcurveto{\pgfqpoint{2.220750in}{3.194950in}}{\pgfqpoint{2.231349in}{3.190560in}}{\pgfqpoint{2.242399in}{3.190560in}}%
\pgfpathclose%
\pgfusepath{stroke,fill}%
\end{pgfscope}%
\begin{pgfscope}%
\pgfpathrectangle{\pgfqpoint{0.648703in}{0.548769in}}{\pgfqpoint{5.112893in}{3.102590in}}%
\pgfusepath{clip}%
\pgfsetbuttcap%
\pgfsetroundjoin%
\definecolor{currentfill}{rgb}{1.000000,0.498039,0.054902}%
\pgfsetfillcolor{currentfill}%
\pgfsetlinewidth{1.003750pt}%
\definecolor{currentstroke}{rgb}{1.000000,0.498039,0.054902}%
\pgfsetstrokecolor{currentstroke}%
\pgfsetdash{}{0pt}%
\pgfpathmoveto{\pgfqpoint{3.631298in}{3.182132in}}%
\pgfpathcurveto{\pgfqpoint{3.642349in}{3.182132in}}{\pgfqpoint{3.652948in}{3.186522in}}{\pgfqpoint{3.660761in}{3.194336in}}%
\pgfpathcurveto{\pgfqpoint{3.668575in}{3.202150in}}{\pgfqpoint{3.672965in}{3.212749in}}{\pgfqpoint{3.672965in}{3.223799in}}%
\pgfpathcurveto{\pgfqpoint{3.672965in}{3.234849in}}{\pgfqpoint{3.668575in}{3.245448in}}{\pgfqpoint{3.660761in}{3.253262in}}%
\pgfpathcurveto{\pgfqpoint{3.652948in}{3.261075in}}{\pgfqpoint{3.642349in}{3.265465in}}{\pgfqpoint{3.631298in}{3.265465in}}%
\pgfpathcurveto{\pgfqpoint{3.620248in}{3.265465in}}{\pgfqpoint{3.609649in}{3.261075in}}{\pgfqpoint{3.601836in}{3.253262in}}%
\pgfpathcurveto{\pgfqpoint{3.594022in}{3.245448in}}{\pgfqpoint{3.589632in}{3.234849in}}{\pgfqpoint{3.589632in}{3.223799in}}%
\pgfpathcurveto{\pgfqpoint{3.589632in}{3.212749in}}{\pgfqpoint{3.594022in}{3.202150in}}{\pgfqpoint{3.601836in}{3.194336in}}%
\pgfpathcurveto{\pgfqpoint{3.609649in}{3.186522in}}{\pgfqpoint{3.620248in}{3.182132in}}{\pgfqpoint{3.631298in}{3.182132in}}%
\pgfpathclose%
\pgfusepath{stroke,fill}%
\end{pgfscope}%
\begin{pgfscope}%
\pgfpathrectangle{\pgfqpoint{0.648703in}{0.548769in}}{\pgfqpoint{5.112893in}{3.102590in}}%
\pgfusepath{clip}%
\pgfsetbuttcap%
\pgfsetroundjoin%
\definecolor{currentfill}{rgb}{1.000000,0.498039,0.054902}%
\pgfsetfillcolor{currentfill}%
\pgfsetlinewidth{1.003750pt}%
\definecolor{currentstroke}{rgb}{1.000000,0.498039,0.054902}%
\pgfsetstrokecolor{currentstroke}%
\pgfsetdash{}{0pt}%
\pgfpathmoveto{\pgfqpoint{3.369308in}{3.186346in}}%
\pgfpathcurveto{\pgfqpoint{3.380358in}{3.186346in}}{\pgfqpoint{3.390957in}{3.190736in}}{\pgfqpoint{3.398771in}{3.198550in}}%
\pgfpathcurveto{\pgfqpoint{3.406585in}{3.206363in}}{\pgfqpoint{3.410975in}{3.216962in}}{\pgfqpoint{3.410975in}{3.228012in}}%
\pgfpathcurveto{\pgfqpoint{3.410975in}{3.239063in}}{\pgfqpoint{3.406585in}{3.249662in}}{\pgfqpoint{3.398771in}{3.257475in}}%
\pgfpathcurveto{\pgfqpoint{3.390957in}{3.265289in}}{\pgfqpoint{3.380358in}{3.269679in}}{\pgfqpoint{3.369308in}{3.269679in}}%
\pgfpathcurveto{\pgfqpoint{3.358258in}{3.269679in}}{\pgfqpoint{3.347659in}{3.265289in}}{\pgfqpoint{3.339846in}{3.257475in}}%
\pgfpathcurveto{\pgfqpoint{3.332032in}{3.249662in}}{\pgfqpoint{3.327642in}{3.239063in}}{\pgfqpoint{3.327642in}{3.228012in}}%
\pgfpathcurveto{\pgfqpoint{3.327642in}{3.216962in}}{\pgfqpoint{3.332032in}{3.206363in}}{\pgfqpoint{3.339846in}{3.198550in}}%
\pgfpathcurveto{\pgfqpoint{3.347659in}{3.190736in}}{\pgfqpoint{3.358258in}{3.186346in}}{\pgfqpoint{3.369308in}{3.186346in}}%
\pgfpathclose%
\pgfusepath{stroke,fill}%
\end{pgfscope}%
\begin{pgfscope}%
\pgfpathrectangle{\pgfqpoint{0.648703in}{0.548769in}}{\pgfqpoint{5.112893in}{3.102590in}}%
\pgfusepath{clip}%
\pgfsetbuttcap%
\pgfsetroundjoin%
\definecolor{currentfill}{rgb}{1.000000,0.498039,0.054902}%
\pgfsetfillcolor{currentfill}%
\pgfsetlinewidth{1.003750pt}%
\definecolor{currentstroke}{rgb}{1.000000,0.498039,0.054902}%
\pgfsetstrokecolor{currentstroke}%
\pgfsetdash{}{0pt}%
\pgfpathmoveto{\pgfqpoint{3.389086in}{3.359108in}}%
\pgfpathcurveto{\pgfqpoint{3.400136in}{3.359108in}}{\pgfqpoint{3.410735in}{3.363499in}}{\pgfqpoint{3.418549in}{3.371312in}}%
\pgfpathcurveto{\pgfqpoint{3.426362in}{3.379126in}}{\pgfqpoint{3.430753in}{3.389725in}}{\pgfqpoint{3.430753in}{3.400775in}}%
\pgfpathcurveto{\pgfqpoint{3.430753in}{3.411825in}}{\pgfqpoint{3.426362in}{3.422424in}}{\pgfqpoint{3.418549in}{3.430238in}}%
\pgfpathcurveto{\pgfqpoint{3.410735in}{3.438051in}}{\pgfqpoint{3.400136in}{3.442442in}}{\pgfqpoint{3.389086in}{3.442442in}}%
\pgfpathcurveto{\pgfqpoint{3.378036in}{3.442442in}}{\pgfqpoint{3.367437in}{3.438051in}}{\pgfqpoint{3.359623in}{3.430238in}}%
\pgfpathcurveto{\pgfqpoint{3.351810in}{3.422424in}}{\pgfqpoint{3.347419in}{3.411825in}}{\pgfqpoint{3.347419in}{3.400775in}}%
\pgfpathcurveto{\pgfqpoint{3.347419in}{3.389725in}}{\pgfqpoint{3.351810in}{3.379126in}}{\pgfqpoint{3.359623in}{3.371312in}}%
\pgfpathcurveto{\pgfqpoint{3.367437in}{3.363499in}}{\pgfqpoint{3.378036in}{3.359108in}}{\pgfqpoint{3.389086in}{3.359108in}}%
\pgfpathclose%
\pgfusepath{stroke,fill}%
\end{pgfscope}%
\begin{pgfscope}%
\pgfpathrectangle{\pgfqpoint{0.648703in}{0.548769in}}{\pgfqpoint{5.112893in}{3.102590in}}%
\pgfusepath{clip}%
\pgfsetbuttcap%
\pgfsetroundjoin%
\definecolor{currentfill}{rgb}{1.000000,0.498039,0.054902}%
\pgfsetfillcolor{currentfill}%
\pgfsetlinewidth{1.003750pt}%
\definecolor{currentstroke}{rgb}{1.000000,0.498039,0.054902}%
\pgfsetstrokecolor{currentstroke}%
\pgfsetdash{}{0pt}%
\pgfpathmoveto{\pgfqpoint{3.181630in}{3.198987in}}%
\pgfpathcurveto{\pgfqpoint{3.192680in}{3.198987in}}{\pgfqpoint{3.203279in}{3.203377in}}{\pgfqpoint{3.211093in}{3.211191in}}%
\pgfpathcurveto{\pgfqpoint{3.218906in}{3.219004in}}{\pgfqpoint{3.223296in}{3.229603in}}{\pgfqpoint{3.223296in}{3.240654in}}%
\pgfpathcurveto{\pgfqpoint{3.223296in}{3.251704in}}{\pgfqpoint{3.218906in}{3.262303in}}{\pgfqpoint{3.211093in}{3.270116in}}%
\pgfpathcurveto{\pgfqpoint{3.203279in}{3.277930in}}{\pgfqpoint{3.192680in}{3.282320in}}{\pgfqpoint{3.181630in}{3.282320in}}%
\pgfpathcurveto{\pgfqpoint{3.170580in}{3.282320in}}{\pgfqpoint{3.159981in}{3.277930in}}{\pgfqpoint{3.152167in}{3.270116in}}%
\pgfpathcurveto{\pgfqpoint{3.144353in}{3.262303in}}{\pgfqpoint{3.139963in}{3.251704in}}{\pgfqpoint{3.139963in}{3.240654in}}%
\pgfpathcurveto{\pgfqpoint{3.139963in}{3.229603in}}{\pgfqpoint{3.144353in}{3.219004in}}{\pgfqpoint{3.152167in}{3.211191in}}%
\pgfpathcurveto{\pgfqpoint{3.159981in}{3.203377in}}{\pgfqpoint{3.170580in}{3.198987in}}{\pgfqpoint{3.181630in}{3.198987in}}%
\pgfpathclose%
\pgfusepath{stroke,fill}%
\end{pgfscope}%
\begin{pgfscope}%
\pgfpathrectangle{\pgfqpoint{0.648703in}{0.548769in}}{\pgfqpoint{5.112893in}{3.102590in}}%
\pgfusepath{clip}%
\pgfsetbuttcap%
\pgfsetroundjoin%
\definecolor{currentfill}{rgb}{0.839216,0.152941,0.156863}%
\pgfsetfillcolor{currentfill}%
\pgfsetlinewidth{1.003750pt}%
\definecolor{currentstroke}{rgb}{0.839216,0.152941,0.156863}%
\pgfsetstrokecolor{currentstroke}%
\pgfsetdash{}{0pt}%
\pgfpathmoveto{\pgfqpoint{3.949793in}{3.194773in}}%
\pgfpathcurveto{\pgfqpoint{3.960843in}{3.194773in}}{\pgfqpoint{3.971442in}{3.199163in}}{\pgfqpoint{3.979255in}{3.206977in}}%
\pgfpathcurveto{\pgfqpoint{3.987069in}{3.214791in}}{\pgfqpoint{3.991459in}{3.225390in}}{\pgfqpoint{3.991459in}{3.236440in}}%
\pgfpathcurveto{\pgfqpoint{3.991459in}{3.247490in}}{\pgfqpoint{3.987069in}{3.258089in}}{\pgfqpoint{3.979255in}{3.265903in}}%
\pgfpathcurveto{\pgfqpoint{3.971442in}{3.273716in}}{\pgfqpoint{3.960843in}{3.278107in}}{\pgfqpoint{3.949793in}{3.278107in}}%
\pgfpathcurveto{\pgfqpoint{3.938742in}{3.278107in}}{\pgfqpoint{3.928143in}{3.273716in}}{\pgfqpoint{3.920330in}{3.265903in}}%
\pgfpathcurveto{\pgfqpoint{3.912516in}{3.258089in}}{\pgfqpoint{3.908126in}{3.247490in}}{\pgfqpoint{3.908126in}{3.236440in}}%
\pgfpathcurveto{\pgfqpoint{3.908126in}{3.225390in}}{\pgfqpoint{3.912516in}{3.214791in}}{\pgfqpoint{3.920330in}{3.206977in}}%
\pgfpathcurveto{\pgfqpoint{3.928143in}{3.199163in}}{\pgfqpoint{3.938742in}{3.194773in}}{\pgfqpoint{3.949793in}{3.194773in}}%
\pgfpathclose%
\pgfusepath{stroke,fill}%
\end{pgfscope}%
\begin{pgfscope}%
\pgfpathrectangle{\pgfqpoint{0.648703in}{0.548769in}}{\pgfqpoint{5.112893in}{3.102590in}}%
\pgfusepath{clip}%
\pgfsetbuttcap%
\pgfsetroundjoin%
\definecolor{currentfill}{rgb}{1.000000,0.498039,0.054902}%
\pgfsetfillcolor{currentfill}%
\pgfsetlinewidth{1.003750pt}%
\definecolor{currentstroke}{rgb}{1.000000,0.498039,0.054902}%
\pgfsetstrokecolor{currentstroke}%
\pgfsetdash{}{0pt}%
\pgfpathmoveto{\pgfqpoint{2.405924in}{3.194773in}}%
\pgfpathcurveto{\pgfqpoint{2.416974in}{3.194773in}}{\pgfqpoint{2.427573in}{3.199163in}}{\pgfqpoint{2.435387in}{3.206977in}}%
\pgfpathcurveto{\pgfqpoint{2.443200in}{3.214791in}}{\pgfqpoint{2.447591in}{3.225390in}}{\pgfqpoint{2.447591in}{3.236440in}}%
\pgfpathcurveto{\pgfqpoint{2.447591in}{3.247490in}}{\pgfqpoint{2.443200in}{3.258089in}}{\pgfqpoint{2.435387in}{3.265903in}}%
\pgfpathcurveto{\pgfqpoint{2.427573in}{3.273716in}}{\pgfqpoint{2.416974in}{3.278107in}}{\pgfqpoint{2.405924in}{3.278107in}}%
\pgfpathcurveto{\pgfqpoint{2.394874in}{3.278107in}}{\pgfqpoint{2.384275in}{3.273716in}}{\pgfqpoint{2.376461in}{3.265903in}}%
\pgfpathcurveto{\pgfqpoint{2.368648in}{3.258089in}}{\pgfqpoint{2.364257in}{3.247490in}}{\pgfqpoint{2.364257in}{3.236440in}}%
\pgfpathcurveto{\pgfqpoint{2.364257in}{3.225390in}}{\pgfqpoint{2.368648in}{3.214791in}}{\pgfqpoint{2.376461in}{3.206977in}}%
\pgfpathcurveto{\pgfqpoint{2.384275in}{3.199163in}}{\pgfqpoint{2.394874in}{3.194773in}}{\pgfqpoint{2.405924in}{3.194773in}}%
\pgfpathclose%
\pgfusepath{stroke,fill}%
\end{pgfscope}%
\begin{pgfscope}%
\pgfpathrectangle{\pgfqpoint{0.648703in}{0.548769in}}{\pgfqpoint{5.112893in}{3.102590in}}%
\pgfusepath{clip}%
\pgfsetbuttcap%
\pgfsetroundjoin%
\definecolor{currentfill}{rgb}{1.000000,0.498039,0.054902}%
\pgfsetfillcolor{currentfill}%
\pgfsetlinewidth{1.003750pt}%
\definecolor{currentstroke}{rgb}{1.000000,0.498039,0.054902}%
\pgfsetstrokecolor{currentstroke}%
\pgfsetdash{}{0pt}%
\pgfpathmoveto{\pgfqpoint{3.666588in}{3.186346in}}%
\pgfpathcurveto{\pgfqpoint{3.677638in}{3.186346in}}{\pgfqpoint{3.688237in}{3.190736in}}{\pgfqpoint{3.696051in}{3.198550in}}%
\pgfpathcurveto{\pgfqpoint{3.703864in}{3.206363in}}{\pgfqpoint{3.708255in}{3.216962in}}{\pgfqpoint{3.708255in}{3.228012in}}%
\pgfpathcurveto{\pgfqpoint{3.708255in}{3.239063in}}{\pgfqpoint{3.703864in}{3.249662in}}{\pgfqpoint{3.696051in}{3.257475in}}%
\pgfpathcurveto{\pgfqpoint{3.688237in}{3.265289in}}{\pgfqpoint{3.677638in}{3.269679in}}{\pgfqpoint{3.666588in}{3.269679in}}%
\pgfpathcurveto{\pgfqpoint{3.655538in}{3.269679in}}{\pgfqpoint{3.644939in}{3.265289in}}{\pgfqpoint{3.637125in}{3.257475in}}%
\pgfpathcurveto{\pgfqpoint{3.629312in}{3.249662in}}{\pgfqpoint{3.624921in}{3.239063in}}{\pgfqpoint{3.624921in}{3.228012in}}%
\pgfpathcurveto{\pgfqpoint{3.624921in}{3.216962in}}{\pgfqpoint{3.629312in}{3.206363in}}{\pgfqpoint{3.637125in}{3.198550in}}%
\pgfpathcurveto{\pgfqpoint{3.644939in}{3.190736in}}{\pgfqpoint{3.655538in}{3.186346in}}{\pgfqpoint{3.666588in}{3.186346in}}%
\pgfpathclose%
\pgfusepath{stroke,fill}%
\end{pgfscope}%
\begin{pgfscope}%
\pgfpathrectangle{\pgfqpoint{0.648703in}{0.548769in}}{\pgfqpoint{5.112893in}{3.102590in}}%
\pgfusepath{clip}%
\pgfsetbuttcap%
\pgfsetroundjoin%
\definecolor{currentfill}{rgb}{1.000000,0.498039,0.054902}%
\pgfsetfillcolor{currentfill}%
\pgfsetlinewidth{1.003750pt}%
\definecolor{currentstroke}{rgb}{1.000000,0.498039,0.054902}%
\pgfsetstrokecolor{currentstroke}%
\pgfsetdash{}{0pt}%
\pgfpathmoveto{\pgfqpoint{3.828076in}{3.186346in}}%
\pgfpathcurveto{\pgfqpoint{3.839126in}{3.186346in}}{\pgfqpoint{3.849725in}{3.190736in}}{\pgfqpoint{3.857538in}{3.198550in}}%
\pgfpathcurveto{\pgfqpoint{3.865352in}{3.206363in}}{\pgfqpoint{3.869742in}{3.216962in}}{\pgfqpoint{3.869742in}{3.228012in}}%
\pgfpathcurveto{\pgfqpoint{3.869742in}{3.239063in}}{\pgfqpoint{3.865352in}{3.249662in}}{\pgfqpoint{3.857538in}{3.257475in}}%
\pgfpathcurveto{\pgfqpoint{3.849725in}{3.265289in}}{\pgfqpoint{3.839126in}{3.269679in}}{\pgfqpoint{3.828076in}{3.269679in}}%
\pgfpathcurveto{\pgfqpoint{3.817026in}{3.269679in}}{\pgfqpoint{3.806427in}{3.265289in}}{\pgfqpoint{3.798613in}{3.257475in}}%
\pgfpathcurveto{\pgfqpoint{3.790799in}{3.249662in}}{\pgfqpoint{3.786409in}{3.239063in}}{\pgfqpoint{3.786409in}{3.228012in}}%
\pgfpathcurveto{\pgfqpoint{3.786409in}{3.216962in}}{\pgfqpoint{3.790799in}{3.206363in}}{\pgfqpoint{3.798613in}{3.198550in}}%
\pgfpathcurveto{\pgfqpoint{3.806427in}{3.190736in}}{\pgfqpoint{3.817026in}{3.186346in}}{\pgfqpoint{3.828076in}{3.186346in}}%
\pgfpathclose%
\pgfusepath{stroke,fill}%
\end{pgfscope}%
\begin{pgfscope}%
\pgfpathrectangle{\pgfqpoint{0.648703in}{0.548769in}}{\pgfqpoint{5.112893in}{3.102590in}}%
\pgfusepath{clip}%
\pgfsetbuttcap%
\pgfsetroundjoin%
\definecolor{currentfill}{rgb}{1.000000,0.498039,0.054902}%
\pgfsetfillcolor{currentfill}%
\pgfsetlinewidth{1.003750pt}%
\definecolor{currentstroke}{rgb}{1.000000,0.498039,0.054902}%
\pgfsetstrokecolor{currentstroke}%
\pgfsetdash{}{0pt}%
\pgfpathmoveto{\pgfqpoint{3.054729in}{3.194773in}}%
\pgfpathcurveto{\pgfqpoint{3.065779in}{3.194773in}}{\pgfqpoint{3.076378in}{3.199163in}}{\pgfqpoint{3.084191in}{3.206977in}}%
\pgfpathcurveto{\pgfqpoint{3.092005in}{3.214791in}}{\pgfqpoint{3.096395in}{3.225390in}}{\pgfqpoint{3.096395in}{3.236440in}}%
\pgfpathcurveto{\pgfqpoint{3.096395in}{3.247490in}}{\pgfqpoint{3.092005in}{3.258089in}}{\pgfqpoint{3.084191in}{3.265903in}}%
\pgfpathcurveto{\pgfqpoint{3.076378in}{3.273716in}}{\pgfqpoint{3.065779in}{3.278107in}}{\pgfqpoint{3.054729in}{3.278107in}}%
\pgfpathcurveto{\pgfqpoint{3.043678in}{3.278107in}}{\pgfqpoint{3.033079in}{3.273716in}}{\pgfqpoint{3.025266in}{3.265903in}}%
\pgfpathcurveto{\pgfqpoint{3.017452in}{3.258089in}}{\pgfqpoint{3.013062in}{3.247490in}}{\pgfqpoint{3.013062in}{3.236440in}}%
\pgfpathcurveto{\pgfqpoint{3.013062in}{3.225390in}}{\pgfqpoint{3.017452in}{3.214791in}}{\pgfqpoint{3.025266in}{3.206977in}}%
\pgfpathcurveto{\pgfqpoint{3.033079in}{3.199163in}}{\pgfqpoint{3.043678in}{3.194773in}}{\pgfqpoint{3.054729in}{3.194773in}}%
\pgfpathclose%
\pgfusepath{stroke,fill}%
\end{pgfscope}%
\begin{pgfscope}%
\pgfpathrectangle{\pgfqpoint{0.648703in}{0.548769in}}{\pgfqpoint{5.112893in}{3.102590in}}%
\pgfusepath{clip}%
\pgfsetbuttcap%
\pgfsetroundjoin%
\definecolor{currentfill}{rgb}{0.839216,0.152941,0.156863}%
\pgfsetfillcolor{currentfill}%
\pgfsetlinewidth{1.003750pt}%
\definecolor{currentstroke}{rgb}{0.839216,0.152941,0.156863}%
\pgfsetstrokecolor{currentstroke}%
\pgfsetdash{}{0pt}%
\pgfpathmoveto{\pgfqpoint{4.235126in}{3.190560in}}%
\pgfpathcurveto{\pgfqpoint{4.246176in}{3.190560in}}{\pgfqpoint{4.256775in}{3.194950in}}{\pgfqpoint{4.264589in}{3.202763in}}%
\pgfpathcurveto{\pgfqpoint{4.272403in}{3.210577in}}{\pgfqpoint{4.276793in}{3.221176in}}{\pgfqpoint{4.276793in}{3.232226in}}%
\pgfpathcurveto{\pgfqpoint{4.276793in}{3.243276in}}{\pgfqpoint{4.272403in}{3.253875in}}{\pgfqpoint{4.264589in}{3.261689in}}%
\pgfpathcurveto{\pgfqpoint{4.256775in}{3.269503in}}{\pgfqpoint{4.246176in}{3.273893in}}{\pgfqpoint{4.235126in}{3.273893in}}%
\pgfpathcurveto{\pgfqpoint{4.224076in}{3.273893in}}{\pgfqpoint{4.213477in}{3.269503in}}{\pgfqpoint{4.205664in}{3.261689in}}%
\pgfpathcurveto{\pgfqpoint{4.197850in}{3.253875in}}{\pgfqpoint{4.193460in}{3.243276in}}{\pgfqpoint{4.193460in}{3.232226in}}%
\pgfpathcurveto{\pgfqpoint{4.193460in}{3.221176in}}{\pgfqpoint{4.197850in}{3.210577in}}{\pgfqpoint{4.205664in}{3.202763in}}%
\pgfpathcurveto{\pgfqpoint{4.213477in}{3.194950in}}{\pgfqpoint{4.224076in}{3.190560in}}{\pgfqpoint{4.235126in}{3.190560in}}%
\pgfpathclose%
\pgfusepath{stroke,fill}%
\end{pgfscope}%
\begin{pgfscope}%
\pgfpathrectangle{\pgfqpoint{0.648703in}{0.548769in}}{\pgfqpoint{5.112893in}{3.102590in}}%
\pgfusepath{clip}%
\pgfsetbuttcap%
\pgfsetroundjoin%
\definecolor{currentfill}{rgb}{1.000000,0.498039,0.054902}%
\pgfsetfillcolor{currentfill}%
\pgfsetlinewidth{1.003750pt}%
\definecolor{currentstroke}{rgb}{1.000000,0.498039,0.054902}%
\pgfsetstrokecolor{currentstroke}%
\pgfsetdash{}{0pt}%
\pgfpathmoveto{\pgfqpoint{3.368175in}{3.198987in}}%
\pgfpathcurveto{\pgfqpoint{3.379225in}{3.198987in}}{\pgfqpoint{3.389824in}{3.203377in}}{\pgfqpoint{3.397638in}{3.211191in}}%
\pgfpathcurveto{\pgfqpoint{3.405451in}{3.219004in}}{\pgfqpoint{3.409841in}{3.229603in}}{\pgfqpoint{3.409841in}{3.240654in}}%
\pgfpathcurveto{\pgfqpoint{3.409841in}{3.251704in}}{\pgfqpoint{3.405451in}{3.262303in}}{\pgfqpoint{3.397638in}{3.270116in}}%
\pgfpathcurveto{\pgfqpoint{3.389824in}{3.277930in}}{\pgfqpoint{3.379225in}{3.282320in}}{\pgfqpoint{3.368175in}{3.282320in}}%
\pgfpathcurveto{\pgfqpoint{3.357125in}{3.282320in}}{\pgfqpoint{3.346526in}{3.277930in}}{\pgfqpoint{3.338712in}{3.270116in}}%
\pgfpathcurveto{\pgfqpoint{3.330898in}{3.262303in}}{\pgfqpoint{3.326508in}{3.251704in}}{\pgfqpoint{3.326508in}{3.240654in}}%
\pgfpathcurveto{\pgfqpoint{3.326508in}{3.229603in}}{\pgfqpoint{3.330898in}{3.219004in}}{\pgfqpoint{3.338712in}{3.211191in}}%
\pgfpathcurveto{\pgfqpoint{3.346526in}{3.203377in}}{\pgfqpoint{3.357125in}{3.198987in}}{\pgfqpoint{3.368175in}{3.198987in}}%
\pgfpathclose%
\pgfusepath{stroke,fill}%
\end{pgfscope}%
\begin{pgfscope}%
\pgfpathrectangle{\pgfqpoint{0.648703in}{0.548769in}}{\pgfqpoint{5.112893in}{3.102590in}}%
\pgfusepath{clip}%
\pgfsetbuttcap%
\pgfsetroundjoin%
\definecolor{currentfill}{rgb}{1.000000,0.498039,0.054902}%
\pgfsetfillcolor{currentfill}%
\pgfsetlinewidth{1.003750pt}%
\definecolor{currentstroke}{rgb}{1.000000,0.498039,0.054902}%
\pgfsetstrokecolor{currentstroke}%
\pgfsetdash{}{0pt}%
\pgfpathmoveto{\pgfqpoint{2.813233in}{3.182132in}}%
\pgfpathcurveto{\pgfqpoint{2.824283in}{3.182132in}}{\pgfqpoint{2.834882in}{3.186522in}}{\pgfqpoint{2.842696in}{3.194336in}}%
\pgfpathcurveto{\pgfqpoint{2.850510in}{3.202150in}}{\pgfqpoint{2.854900in}{3.212749in}}{\pgfqpoint{2.854900in}{3.223799in}}%
\pgfpathcurveto{\pgfqpoint{2.854900in}{3.234849in}}{\pgfqpoint{2.850510in}{3.245448in}}{\pgfqpoint{2.842696in}{3.253262in}}%
\pgfpathcurveto{\pgfqpoint{2.834882in}{3.261075in}}{\pgfqpoint{2.824283in}{3.265465in}}{\pgfqpoint{2.813233in}{3.265465in}}%
\pgfpathcurveto{\pgfqpoint{2.802183in}{3.265465in}}{\pgfqpoint{2.791584in}{3.261075in}}{\pgfqpoint{2.783771in}{3.253262in}}%
\pgfpathcurveto{\pgfqpoint{2.775957in}{3.245448in}}{\pgfqpoint{2.771567in}{3.234849in}}{\pgfqpoint{2.771567in}{3.223799in}}%
\pgfpathcurveto{\pgfqpoint{2.771567in}{3.212749in}}{\pgfqpoint{2.775957in}{3.202150in}}{\pgfqpoint{2.783771in}{3.194336in}}%
\pgfpathcurveto{\pgfqpoint{2.791584in}{3.186522in}}{\pgfqpoint{2.802183in}{3.182132in}}{\pgfqpoint{2.813233in}{3.182132in}}%
\pgfpathclose%
\pgfusepath{stroke,fill}%
\end{pgfscope}%
\begin{pgfscope}%
\pgfpathrectangle{\pgfqpoint{0.648703in}{0.548769in}}{\pgfqpoint{5.112893in}{3.102590in}}%
\pgfusepath{clip}%
\pgfsetbuttcap%
\pgfsetroundjoin%
\definecolor{currentfill}{rgb}{0.121569,0.466667,0.705882}%
\pgfsetfillcolor{currentfill}%
\pgfsetlinewidth{1.003750pt}%
\definecolor{currentstroke}{rgb}{0.121569,0.466667,0.705882}%
\pgfsetstrokecolor{currentstroke}%
\pgfsetdash{}{0pt}%
\pgfpathmoveto{\pgfqpoint{1.168757in}{0.948860in}}%
\pgfpathcurveto{\pgfqpoint{1.179807in}{0.948860in}}{\pgfqpoint{1.190406in}{0.953250in}}{\pgfqpoint{1.198220in}{0.961063in}}%
\pgfpathcurveto{\pgfqpoint{1.206034in}{0.968877in}}{\pgfqpoint{1.210424in}{0.979476in}}{\pgfqpoint{1.210424in}{0.990526in}}%
\pgfpathcurveto{\pgfqpoint{1.210424in}{1.001576in}}{\pgfqpoint{1.206034in}{1.012175in}}{\pgfqpoint{1.198220in}{1.019989in}}%
\pgfpathcurveto{\pgfqpoint{1.190406in}{1.027803in}}{\pgfqpoint{1.179807in}{1.032193in}}{\pgfqpoint{1.168757in}{1.032193in}}%
\pgfpathcurveto{\pgfqpoint{1.157707in}{1.032193in}}{\pgfqpoint{1.147108in}{1.027803in}}{\pgfqpoint{1.139295in}{1.019989in}}%
\pgfpathcurveto{\pgfqpoint{1.131481in}{1.012175in}}{\pgfqpoint{1.127091in}{1.001576in}}{\pgfqpoint{1.127091in}{0.990526in}}%
\pgfpathcurveto{\pgfqpoint{1.127091in}{0.979476in}}{\pgfqpoint{1.131481in}{0.968877in}}{\pgfqpoint{1.139295in}{0.961063in}}%
\pgfpathcurveto{\pgfqpoint{1.147108in}{0.953250in}}{\pgfqpoint{1.157707in}{0.948860in}}{\pgfqpoint{1.168757in}{0.948860in}}%
\pgfpathclose%
\pgfusepath{stroke,fill}%
\end{pgfscope}%
\begin{pgfscope}%
\pgfpathrectangle{\pgfqpoint{0.648703in}{0.548769in}}{\pgfqpoint{5.112893in}{3.102590in}}%
\pgfusepath{clip}%
\pgfsetbuttcap%
\pgfsetroundjoin%
\definecolor{currentfill}{rgb}{0.839216,0.152941,0.156863}%
\pgfsetfillcolor{currentfill}%
\pgfsetlinewidth{1.003750pt}%
\definecolor{currentstroke}{rgb}{0.839216,0.152941,0.156863}%
\pgfsetstrokecolor{currentstroke}%
\pgfsetdash{}{0pt}%
\pgfpathmoveto{\pgfqpoint{4.481783in}{3.211628in}}%
\pgfpathcurveto{\pgfqpoint{4.492834in}{3.211628in}}{\pgfqpoint{4.503433in}{3.216018in}}{\pgfqpoint{4.511246in}{3.223832in}}%
\pgfpathcurveto{\pgfqpoint{4.519060in}{3.231646in}}{\pgfqpoint{4.523450in}{3.242245in}}{\pgfqpoint{4.523450in}{3.253295in}}%
\pgfpathcurveto{\pgfqpoint{4.523450in}{3.264345in}}{\pgfqpoint{4.519060in}{3.274944in}}{\pgfqpoint{4.511246in}{3.282758in}}%
\pgfpathcurveto{\pgfqpoint{4.503433in}{3.290571in}}{\pgfqpoint{4.492834in}{3.294961in}}{\pgfqpoint{4.481783in}{3.294961in}}%
\pgfpathcurveto{\pgfqpoint{4.470733in}{3.294961in}}{\pgfqpoint{4.460134in}{3.290571in}}{\pgfqpoint{4.452321in}{3.282758in}}%
\pgfpathcurveto{\pgfqpoint{4.444507in}{3.274944in}}{\pgfqpoint{4.440117in}{3.264345in}}{\pgfqpoint{4.440117in}{3.253295in}}%
\pgfpathcurveto{\pgfqpoint{4.440117in}{3.242245in}}{\pgfqpoint{4.444507in}{3.231646in}}{\pgfqpoint{4.452321in}{3.223832in}}%
\pgfpathcurveto{\pgfqpoint{4.460134in}{3.216018in}}{\pgfqpoint{4.470733in}{3.211628in}}{\pgfqpoint{4.481783in}{3.211628in}}%
\pgfpathclose%
\pgfusepath{stroke,fill}%
\end{pgfscope}%
\begin{pgfscope}%
\pgfpathrectangle{\pgfqpoint{0.648703in}{0.548769in}}{\pgfqpoint{5.112893in}{3.102590in}}%
\pgfusepath{clip}%
\pgfsetbuttcap%
\pgfsetroundjoin%
\definecolor{currentfill}{rgb}{1.000000,0.498039,0.054902}%
\pgfsetfillcolor{currentfill}%
\pgfsetlinewidth{1.003750pt}%
\definecolor{currentstroke}{rgb}{1.000000,0.498039,0.054902}%
\pgfsetstrokecolor{currentstroke}%
\pgfsetdash{}{0pt}%
\pgfpathmoveto{\pgfqpoint{3.200388in}{3.203201in}}%
\pgfpathcurveto{\pgfqpoint{3.211438in}{3.203201in}}{\pgfqpoint{3.222037in}{3.207591in}}{\pgfqpoint{3.229851in}{3.215405in}}%
\pgfpathcurveto{\pgfqpoint{3.237664in}{3.223218in}}{\pgfqpoint{3.242055in}{3.233817in}}{\pgfqpoint{3.242055in}{3.244867in}}%
\pgfpathcurveto{\pgfqpoint{3.242055in}{3.255917in}}{\pgfqpoint{3.237664in}{3.266516in}}{\pgfqpoint{3.229851in}{3.274330in}}%
\pgfpathcurveto{\pgfqpoint{3.222037in}{3.282144in}}{\pgfqpoint{3.211438in}{3.286534in}}{\pgfqpoint{3.200388in}{3.286534in}}%
\pgfpathcurveto{\pgfqpoint{3.189338in}{3.286534in}}{\pgfqpoint{3.178739in}{3.282144in}}{\pgfqpoint{3.170925in}{3.274330in}}%
\pgfpathcurveto{\pgfqpoint{3.163111in}{3.266516in}}{\pgfqpoint{3.158721in}{3.255917in}}{\pgfqpoint{3.158721in}{3.244867in}}%
\pgfpathcurveto{\pgfqpoint{3.158721in}{3.233817in}}{\pgfqpoint{3.163111in}{3.223218in}}{\pgfqpoint{3.170925in}{3.215405in}}%
\pgfpathcurveto{\pgfqpoint{3.178739in}{3.207591in}}{\pgfqpoint{3.189338in}{3.203201in}}{\pgfqpoint{3.200388in}{3.203201in}}%
\pgfpathclose%
\pgfusepath{stroke,fill}%
\end{pgfscope}%
\begin{pgfscope}%
\pgfpathrectangle{\pgfqpoint{0.648703in}{0.548769in}}{\pgfqpoint{5.112893in}{3.102590in}}%
\pgfusepath{clip}%
\pgfsetbuttcap%
\pgfsetroundjoin%
\definecolor{currentfill}{rgb}{1.000000,0.498039,0.054902}%
\pgfsetfillcolor{currentfill}%
\pgfsetlinewidth{1.003750pt}%
\definecolor{currentstroke}{rgb}{1.000000,0.498039,0.054902}%
\pgfsetstrokecolor{currentstroke}%
\pgfsetdash{}{0pt}%
\pgfpathmoveto{\pgfqpoint{2.974439in}{3.207414in}}%
\pgfpathcurveto{\pgfqpoint{2.985490in}{3.207414in}}{\pgfqpoint{2.996089in}{3.211805in}}{\pgfqpoint{3.003902in}{3.219618in}}%
\pgfpathcurveto{\pgfqpoint{3.011716in}{3.227432in}}{\pgfqpoint{3.016106in}{3.238031in}}{\pgfqpoint{3.016106in}{3.249081in}}%
\pgfpathcurveto{\pgfqpoint{3.016106in}{3.260131in}}{\pgfqpoint{3.011716in}{3.270730in}}{\pgfqpoint{3.003902in}{3.278544in}}%
\pgfpathcurveto{\pgfqpoint{2.996089in}{3.286357in}}{\pgfqpoint{2.985490in}{3.290748in}}{\pgfqpoint{2.974439in}{3.290748in}}%
\pgfpathcurveto{\pgfqpoint{2.963389in}{3.290748in}}{\pgfqpoint{2.952790in}{3.286357in}}{\pgfqpoint{2.944977in}{3.278544in}}%
\pgfpathcurveto{\pgfqpoint{2.937163in}{3.270730in}}{\pgfqpoint{2.932773in}{3.260131in}}{\pgfqpoint{2.932773in}{3.249081in}}%
\pgfpathcurveto{\pgfqpoint{2.932773in}{3.238031in}}{\pgfqpoint{2.937163in}{3.227432in}}{\pgfqpoint{2.944977in}{3.219618in}}%
\pgfpathcurveto{\pgfqpoint{2.952790in}{3.211805in}}{\pgfqpoint{2.963389in}{3.207414in}}{\pgfqpoint{2.974439in}{3.207414in}}%
\pgfpathclose%
\pgfusepath{stroke,fill}%
\end{pgfscope}%
\begin{pgfscope}%
\pgfpathrectangle{\pgfqpoint{0.648703in}{0.548769in}}{\pgfqpoint{5.112893in}{3.102590in}}%
\pgfusepath{clip}%
\pgfsetbuttcap%
\pgfsetroundjoin%
\definecolor{currentfill}{rgb}{1.000000,0.498039,0.054902}%
\pgfsetfillcolor{currentfill}%
\pgfsetlinewidth{1.003750pt}%
\definecolor{currentstroke}{rgb}{1.000000,0.498039,0.054902}%
\pgfsetstrokecolor{currentstroke}%
\pgfsetdash{}{0pt}%
\pgfpathmoveto{\pgfqpoint{3.698418in}{3.198987in}}%
\pgfpathcurveto{\pgfqpoint{3.709468in}{3.198987in}}{\pgfqpoint{3.720067in}{3.203377in}}{\pgfqpoint{3.727881in}{3.211191in}}%
\pgfpathcurveto{\pgfqpoint{3.735695in}{3.219004in}}{\pgfqpoint{3.740085in}{3.229603in}}{\pgfqpoint{3.740085in}{3.240654in}}%
\pgfpathcurveto{\pgfqpoint{3.740085in}{3.251704in}}{\pgfqpoint{3.735695in}{3.262303in}}{\pgfqpoint{3.727881in}{3.270116in}}%
\pgfpathcurveto{\pgfqpoint{3.720067in}{3.277930in}}{\pgfqpoint{3.709468in}{3.282320in}}{\pgfqpoint{3.698418in}{3.282320in}}%
\pgfpathcurveto{\pgfqpoint{3.687368in}{3.282320in}}{\pgfqpoint{3.676769in}{3.277930in}}{\pgfqpoint{3.668955in}{3.270116in}}%
\pgfpathcurveto{\pgfqpoint{3.661142in}{3.262303in}}{\pgfqpoint{3.656751in}{3.251704in}}{\pgfqpoint{3.656751in}{3.240654in}}%
\pgfpathcurveto{\pgfqpoint{3.656751in}{3.229603in}}{\pgfqpoint{3.661142in}{3.219004in}}{\pgfqpoint{3.668955in}{3.211191in}}%
\pgfpathcurveto{\pgfqpoint{3.676769in}{3.203377in}}{\pgfqpoint{3.687368in}{3.198987in}}{\pgfqpoint{3.698418in}{3.198987in}}%
\pgfpathclose%
\pgfusepath{stroke,fill}%
\end{pgfscope}%
\begin{pgfscope}%
\pgfpathrectangle{\pgfqpoint{0.648703in}{0.548769in}}{\pgfqpoint{5.112893in}{3.102590in}}%
\pgfusepath{clip}%
\pgfsetbuttcap%
\pgfsetroundjoin%
\definecolor{currentfill}{rgb}{1.000000,0.498039,0.054902}%
\pgfsetfillcolor{currentfill}%
\pgfsetlinewidth{1.003750pt}%
\definecolor{currentstroke}{rgb}{1.000000,0.498039,0.054902}%
\pgfsetstrokecolor{currentstroke}%
\pgfsetdash{}{0pt}%
\pgfpathmoveto{\pgfqpoint{3.706089in}{3.203201in}}%
\pgfpathcurveto{\pgfqpoint{3.717140in}{3.203201in}}{\pgfqpoint{3.727739in}{3.207591in}}{\pgfqpoint{3.735552in}{3.215405in}}%
\pgfpathcurveto{\pgfqpoint{3.743366in}{3.223218in}}{\pgfqpoint{3.747756in}{3.233817in}}{\pgfqpoint{3.747756in}{3.244867in}}%
\pgfpathcurveto{\pgfqpoint{3.747756in}{3.255917in}}{\pgfqpoint{3.743366in}{3.266516in}}{\pgfqpoint{3.735552in}{3.274330in}}%
\pgfpathcurveto{\pgfqpoint{3.727739in}{3.282144in}}{\pgfqpoint{3.717140in}{3.286534in}}{\pgfqpoint{3.706089in}{3.286534in}}%
\pgfpathcurveto{\pgfqpoint{3.695039in}{3.286534in}}{\pgfqpoint{3.684440in}{3.282144in}}{\pgfqpoint{3.676627in}{3.274330in}}%
\pgfpathcurveto{\pgfqpoint{3.668813in}{3.266516in}}{\pgfqpoint{3.664423in}{3.255917in}}{\pgfqpoint{3.664423in}{3.244867in}}%
\pgfpathcurveto{\pgfqpoint{3.664423in}{3.233817in}}{\pgfqpoint{3.668813in}{3.223218in}}{\pgfqpoint{3.676627in}{3.215405in}}%
\pgfpathcurveto{\pgfqpoint{3.684440in}{3.207591in}}{\pgfqpoint{3.695039in}{3.203201in}}{\pgfqpoint{3.706089in}{3.203201in}}%
\pgfpathclose%
\pgfusepath{stroke,fill}%
\end{pgfscope}%
\begin{pgfscope}%
\pgfpathrectangle{\pgfqpoint{0.648703in}{0.548769in}}{\pgfqpoint{5.112893in}{3.102590in}}%
\pgfusepath{clip}%
\pgfsetbuttcap%
\pgfsetroundjoin%
\definecolor{currentfill}{rgb}{1.000000,0.498039,0.054902}%
\pgfsetfillcolor{currentfill}%
\pgfsetlinewidth{1.003750pt}%
\definecolor{currentstroke}{rgb}{1.000000,0.498039,0.054902}%
\pgfsetstrokecolor{currentstroke}%
\pgfsetdash{}{0pt}%
\pgfpathmoveto{\pgfqpoint{2.501059in}{3.190560in}}%
\pgfpathcurveto{\pgfqpoint{2.512109in}{3.190560in}}{\pgfqpoint{2.522708in}{3.194950in}}{\pgfqpoint{2.530522in}{3.202763in}}%
\pgfpathcurveto{\pgfqpoint{2.538336in}{3.210577in}}{\pgfqpoint{2.542726in}{3.221176in}}{\pgfqpoint{2.542726in}{3.232226in}}%
\pgfpathcurveto{\pgfqpoint{2.542726in}{3.243276in}}{\pgfqpoint{2.538336in}{3.253875in}}{\pgfqpoint{2.530522in}{3.261689in}}%
\pgfpathcurveto{\pgfqpoint{2.522708in}{3.269503in}}{\pgfqpoint{2.512109in}{3.273893in}}{\pgfqpoint{2.501059in}{3.273893in}}%
\pgfpathcurveto{\pgfqpoint{2.490009in}{3.273893in}}{\pgfqpoint{2.479410in}{3.269503in}}{\pgfqpoint{2.471596in}{3.261689in}}%
\pgfpathcurveto{\pgfqpoint{2.463783in}{3.253875in}}{\pgfqpoint{2.459392in}{3.243276in}}{\pgfqpoint{2.459392in}{3.232226in}}%
\pgfpathcurveto{\pgfqpoint{2.459392in}{3.221176in}}{\pgfqpoint{2.463783in}{3.210577in}}{\pgfqpoint{2.471596in}{3.202763in}}%
\pgfpathcurveto{\pgfqpoint{2.479410in}{3.194950in}}{\pgfqpoint{2.490009in}{3.190560in}}{\pgfqpoint{2.501059in}{3.190560in}}%
\pgfpathclose%
\pgfusepath{stroke,fill}%
\end{pgfscope}%
\begin{pgfscope}%
\pgfpathrectangle{\pgfqpoint{0.648703in}{0.548769in}}{\pgfqpoint{5.112893in}{3.102590in}}%
\pgfusepath{clip}%
\pgfsetbuttcap%
\pgfsetroundjoin%
\definecolor{currentfill}{rgb}{1.000000,0.498039,0.054902}%
\pgfsetfillcolor{currentfill}%
\pgfsetlinewidth{1.003750pt}%
\definecolor{currentstroke}{rgb}{1.000000,0.498039,0.054902}%
\pgfsetstrokecolor{currentstroke}%
\pgfsetdash{}{0pt}%
\pgfpathmoveto{\pgfqpoint{3.487832in}{3.182132in}}%
\pgfpathcurveto{\pgfqpoint{3.498882in}{3.182132in}}{\pgfqpoint{3.509481in}{3.186522in}}{\pgfqpoint{3.517294in}{3.194336in}}%
\pgfpathcurveto{\pgfqpoint{3.525108in}{3.202150in}}{\pgfqpoint{3.529498in}{3.212749in}}{\pgfqpoint{3.529498in}{3.223799in}}%
\pgfpathcurveto{\pgfqpoint{3.529498in}{3.234849in}}{\pgfqpoint{3.525108in}{3.245448in}}{\pgfqpoint{3.517294in}{3.253262in}}%
\pgfpathcurveto{\pgfqpoint{3.509481in}{3.261075in}}{\pgfqpoint{3.498882in}{3.265465in}}{\pgfqpoint{3.487832in}{3.265465in}}%
\pgfpathcurveto{\pgfqpoint{3.476781in}{3.265465in}}{\pgfqpoint{3.466182in}{3.261075in}}{\pgfqpoint{3.458369in}{3.253262in}}%
\pgfpathcurveto{\pgfqpoint{3.450555in}{3.245448in}}{\pgfqpoint{3.446165in}{3.234849in}}{\pgfqpoint{3.446165in}{3.223799in}}%
\pgfpathcurveto{\pgfqpoint{3.446165in}{3.212749in}}{\pgfqpoint{3.450555in}{3.202150in}}{\pgfqpoint{3.458369in}{3.194336in}}%
\pgfpathcurveto{\pgfqpoint{3.466182in}{3.186522in}}{\pgfqpoint{3.476781in}{3.182132in}}{\pgfqpoint{3.487832in}{3.182132in}}%
\pgfpathclose%
\pgfusepath{stroke,fill}%
\end{pgfscope}%
\begin{pgfscope}%
\pgfpathrectangle{\pgfqpoint{0.648703in}{0.548769in}}{\pgfqpoint{5.112893in}{3.102590in}}%
\pgfusepath{clip}%
\pgfsetbuttcap%
\pgfsetroundjoin%
\definecolor{currentfill}{rgb}{1.000000,0.498039,0.054902}%
\pgfsetfillcolor{currentfill}%
\pgfsetlinewidth{1.003750pt}%
\definecolor{currentstroke}{rgb}{1.000000,0.498039,0.054902}%
\pgfsetstrokecolor{currentstroke}%
\pgfsetdash{}{0pt}%
\pgfpathmoveto{\pgfqpoint{3.160468in}{3.245338in}}%
\pgfpathcurveto{\pgfqpoint{3.171518in}{3.245338in}}{\pgfqpoint{3.182117in}{3.249728in}}{\pgfqpoint{3.189931in}{3.257542in}}%
\pgfpathcurveto{\pgfqpoint{3.197744in}{3.265355in}}{\pgfqpoint{3.202135in}{3.275954in}}{\pgfqpoint{3.202135in}{3.287005in}}%
\pgfpathcurveto{\pgfqpoint{3.202135in}{3.298055in}}{\pgfqpoint{3.197744in}{3.308654in}}{\pgfqpoint{3.189931in}{3.316467in}}%
\pgfpathcurveto{\pgfqpoint{3.182117in}{3.324281in}}{\pgfqpoint{3.171518in}{3.328671in}}{\pgfqpoint{3.160468in}{3.328671in}}%
\pgfpathcurveto{\pgfqpoint{3.149418in}{3.328671in}}{\pgfqpoint{3.138819in}{3.324281in}}{\pgfqpoint{3.131005in}{3.316467in}}%
\pgfpathcurveto{\pgfqpoint{3.123192in}{3.308654in}}{\pgfqpoint{3.118801in}{3.298055in}}{\pgfqpoint{3.118801in}{3.287005in}}%
\pgfpathcurveto{\pgfqpoint{3.118801in}{3.275954in}}{\pgfqpoint{3.123192in}{3.265355in}}{\pgfqpoint{3.131005in}{3.257542in}}%
\pgfpathcurveto{\pgfqpoint{3.138819in}{3.249728in}}{\pgfqpoint{3.149418in}{3.245338in}}{\pgfqpoint{3.160468in}{3.245338in}}%
\pgfpathclose%
\pgfusepath{stroke,fill}%
\end{pgfscope}%
\begin{pgfscope}%
\pgfpathrectangle{\pgfqpoint{0.648703in}{0.548769in}}{\pgfqpoint{5.112893in}{3.102590in}}%
\pgfusepath{clip}%
\pgfsetbuttcap%
\pgfsetroundjoin%
\definecolor{currentfill}{rgb}{1.000000,0.498039,0.054902}%
\pgfsetfillcolor{currentfill}%
\pgfsetlinewidth{1.003750pt}%
\definecolor{currentstroke}{rgb}{1.000000,0.498039,0.054902}%
\pgfsetstrokecolor{currentstroke}%
\pgfsetdash{}{0pt}%
\pgfpathmoveto{\pgfqpoint{2.833986in}{3.190560in}}%
\pgfpathcurveto{\pgfqpoint{2.845036in}{3.190560in}}{\pgfqpoint{2.855635in}{3.194950in}}{\pgfqpoint{2.863449in}{3.202763in}}%
\pgfpathcurveto{\pgfqpoint{2.871262in}{3.210577in}}{\pgfqpoint{2.875652in}{3.221176in}}{\pgfqpoint{2.875652in}{3.232226in}}%
\pgfpathcurveto{\pgfqpoint{2.875652in}{3.243276in}}{\pgfqpoint{2.871262in}{3.253875in}}{\pgfqpoint{2.863449in}{3.261689in}}%
\pgfpathcurveto{\pgfqpoint{2.855635in}{3.269503in}}{\pgfqpoint{2.845036in}{3.273893in}}{\pgfqpoint{2.833986in}{3.273893in}}%
\pgfpathcurveto{\pgfqpoint{2.822936in}{3.273893in}}{\pgfqpoint{2.812337in}{3.269503in}}{\pgfqpoint{2.804523in}{3.261689in}}%
\pgfpathcurveto{\pgfqpoint{2.796709in}{3.253875in}}{\pgfqpoint{2.792319in}{3.243276in}}{\pgfqpoint{2.792319in}{3.232226in}}%
\pgfpathcurveto{\pgfqpoint{2.792319in}{3.221176in}}{\pgfqpoint{2.796709in}{3.210577in}}{\pgfqpoint{2.804523in}{3.202763in}}%
\pgfpathcurveto{\pgfqpoint{2.812337in}{3.194950in}}{\pgfqpoint{2.822936in}{3.190560in}}{\pgfqpoint{2.833986in}{3.190560in}}%
\pgfpathclose%
\pgfusepath{stroke,fill}%
\end{pgfscope}%
\begin{pgfscope}%
\pgfpathrectangle{\pgfqpoint{0.648703in}{0.548769in}}{\pgfqpoint{5.112893in}{3.102590in}}%
\pgfusepath{clip}%
\pgfsetbuttcap%
\pgfsetroundjoin%
\definecolor{currentfill}{rgb}{1.000000,0.498039,0.054902}%
\pgfsetfillcolor{currentfill}%
\pgfsetlinewidth{1.003750pt}%
\definecolor{currentstroke}{rgb}{1.000000,0.498039,0.054902}%
\pgfsetstrokecolor{currentstroke}%
\pgfsetdash{}{0pt}%
\pgfpathmoveto{\pgfqpoint{3.278046in}{3.211628in}}%
\pgfpathcurveto{\pgfqpoint{3.289097in}{3.211628in}}{\pgfqpoint{3.299696in}{3.216018in}}{\pgfqpoint{3.307509in}{3.223832in}}%
\pgfpathcurveto{\pgfqpoint{3.315323in}{3.231646in}}{\pgfqpoint{3.319713in}{3.242245in}}{\pgfqpoint{3.319713in}{3.253295in}}%
\pgfpathcurveto{\pgfqpoint{3.319713in}{3.264345in}}{\pgfqpoint{3.315323in}{3.274944in}}{\pgfqpoint{3.307509in}{3.282758in}}%
\pgfpathcurveto{\pgfqpoint{3.299696in}{3.290571in}}{\pgfqpoint{3.289097in}{3.294961in}}{\pgfqpoint{3.278046in}{3.294961in}}%
\pgfpathcurveto{\pgfqpoint{3.266996in}{3.294961in}}{\pgfqpoint{3.256397in}{3.290571in}}{\pgfqpoint{3.248584in}{3.282758in}}%
\pgfpathcurveto{\pgfqpoint{3.240770in}{3.274944in}}{\pgfqpoint{3.236380in}{3.264345in}}{\pgfqpoint{3.236380in}{3.253295in}}%
\pgfpathcurveto{\pgfqpoint{3.236380in}{3.242245in}}{\pgfqpoint{3.240770in}{3.231646in}}{\pgfqpoint{3.248584in}{3.223832in}}%
\pgfpathcurveto{\pgfqpoint{3.256397in}{3.216018in}}{\pgfqpoint{3.266996in}{3.211628in}}{\pgfqpoint{3.278046in}{3.211628in}}%
\pgfpathclose%
\pgfusepath{stroke,fill}%
\end{pgfscope}%
\begin{pgfscope}%
\pgfpathrectangle{\pgfqpoint{0.648703in}{0.548769in}}{\pgfqpoint{5.112893in}{3.102590in}}%
\pgfusepath{clip}%
\pgfsetbuttcap%
\pgfsetroundjoin%
\definecolor{currentfill}{rgb}{1.000000,0.498039,0.054902}%
\pgfsetfillcolor{currentfill}%
\pgfsetlinewidth{1.003750pt}%
\definecolor{currentstroke}{rgb}{1.000000,0.498039,0.054902}%
\pgfsetstrokecolor{currentstroke}%
\pgfsetdash{}{0pt}%
\pgfpathmoveto{\pgfqpoint{2.926341in}{3.190560in}}%
\pgfpathcurveto{\pgfqpoint{2.937392in}{3.190560in}}{\pgfqpoint{2.947991in}{3.194950in}}{\pgfqpoint{2.955804in}{3.202763in}}%
\pgfpathcurveto{\pgfqpoint{2.963618in}{3.210577in}}{\pgfqpoint{2.968008in}{3.221176in}}{\pgfqpoint{2.968008in}{3.232226in}}%
\pgfpathcurveto{\pgfqpoint{2.968008in}{3.243276in}}{\pgfqpoint{2.963618in}{3.253875in}}{\pgfqpoint{2.955804in}{3.261689in}}%
\pgfpathcurveto{\pgfqpoint{2.947991in}{3.269503in}}{\pgfqpoint{2.937392in}{3.273893in}}{\pgfqpoint{2.926341in}{3.273893in}}%
\pgfpathcurveto{\pgfqpoint{2.915291in}{3.273893in}}{\pgfqpoint{2.904692in}{3.269503in}}{\pgfqpoint{2.896879in}{3.261689in}}%
\pgfpathcurveto{\pgfqpoint{2.889065in}{3.253875in}}{\pgfqpoint{2.884675in}{3.243276in}}{\pgfqpoint{2.884675in}{3.232226in}}%
\pgfpathcurveto{\pgfqpoint{2.884675in}{3.221176in}}{\pgfqpoint{2.889065in}{3.210577in}}{\pgfqpoint{2.896879in}{3.202763in}}%
\pgfpathcurveto{\pgfqpoint{2.904692in}{3.194950in}}{\pgfqpoint{2.915291in}{3.190560in}}{\pgfqpoint{2.926341in}{3.190560in}}%
\pgfpathclose%
\pgfusepath{stroke,fill}%
\end{pgfscope}%
\begin{pgfscope}%
\pgfpathrectangle{\pgfqpoint{0.648703in}{0.548769in}}{\pgfqpoint{5.112893in}{3.102590in}}%
\pgfusepath{clip}%
\pgfsetbuttcap%
\pgfsetroundjoin%
\definecolor{currentfill}{rgb}{0.839216,0.152941,0.156863}%
\pgfsetfillcolor{currentfill}%
\pgfsetlinewidth{1.003750pt}%
\definecolor{currentstroke}{rgb}{0.839216,0.152941,0.156863}%
\pgfsetstrokecolor{currentstroke}%
\pgfsetdash{}{0pt}%
\pgfpathmoveto{\pgfqpoint{2.966002in}{3.182132in}}%
\pgfpathcurveto{\pgfqpoint{2.977052in}{3.182132in}}{\pgfqpoint{2.987651in}{3.186522in}}{\pgfqpoint{2.995465in}{3.194336in}}%
\pgfpathcurveto{\pgfqpoint{3.003278in}{3.202150in}}{\pgfqpoint{3.007669in}{3.212749in}}{\pgfqpoint{3.007669in}{3.223799in}}%
\pgfpathcurveto{\pgfqpoint{3.007669in}{3.234849in}}{\pgfqpoint{3.003278in}{3.245448in}}{\pgfqpoint{2.995465in}{3.253262in}}%
\pgfpathcurveto{\pgfqpoint{2.987651in}{3.261075in}}{\pgfqpoint{2.977052in}{3.265465in}}{\pgfqpoint{2.966002in}{3.265465in}}%
\pgfpathcurveto{\pgfqpoint{2.954952in}{3.265465in}}{\pgfqpoint{2.944353in}{3.261075in}}{\pgfqpoint{2.936539in}{3.253262in}}%
\pgfpathcurveto{\pgfqpoint{2.928726in}{3.245448in}}{\pgfqpoint{2.924335in}{3.234849in}}{\pgfqpoint{2.924335in}{3.223799in}}%
\pgfpathcurveto{\pgfqpoint{2.924335in}{3.212749in}}{\pgfqpoint{2.928726in}{3.202150in}}{\pgfqpoint{2.936539in}{3.194336in}}%
\pgfpathcurveto{\pgfqpoint{2.944353in}{3.186522in}}{\pgfqpoint{2.954952in}{3.182132in}}{\pgfqpoint{2.966002in}{3.182132in}}%
\pgfpathclose%
\pgfusepath{stroke,fill}%
\end{pgfscope}%
\begin{pgfscope}%
\pgfpathrectangle{\pgfqpoint{0.648703in}{0.548769in}}{\pgfqpoint{5.112893in}{3.102590in}}%
\pgfusepath{clip}%
\pgfsetbuttcap%
\pgfsetroundjoin%
\definecolor{currentfill}{rgb}{1.000000,0.498039,0.054902}%
\pgfsetfillcolor{currentfill}%
\pgfsetlinewidth{1.003750pt}%
\definecolor{currentstroke}{rgb}{1.000000,0.498039,0.054902}%
\pgfsetstrokecolor{currentstroke}%
\pgfsetdash{}{0pt}%
\pgfpathmoveto{\pgfqpoint{3.331214in}{3.203201in}}%
\pgfpathcurveto{\pgfqpoint{3.342264in}{3.203201in}}{\pgfqpoint{3.352863in}{3.207591in}}{\pgfqpoint{3.360677in}{3.215405in}}%
\pgfpathcurveto{\pgfqpoint{3.368491in}{3.223218in}}{\pgfqpoint{3.372881in}{3.233817in}}{\pgfqpoint{3.372881in}{3.244867in}}%
\pgfpathcurveto{\pgfqpoint{3.372881in}{3.255917in}}{\pgfqpoint{3.368491in}{3.266516in}}{\pgfqpoint{3.360677in}{3.274330in}}%
\pgfpathcurveto{\pgfqpoint{3.352863in}{3.282144in}}{\pgfqpoint{3.342264in}{3.286534in}}{\pgfqpoint{3.331214in}{3.286534in}}%
\pgfpathcurveto{\pgfqpoint{3.320164in}{3.286534in}}{\pgfqpoint{3.309565in}{3.282144in}}{\pgfqpoint{3.301751in}{3.274330in}}%
\pgfpathcurveto{\pgfqpoint{3.293938in}{3.266516in}}{\pgfqpoint{3.289547in}{3.255917in}}{\pgfqpoint{3.289547in}{3.244867in}}%
\pgfpathcurveto{\pgfqpoint{3.289547in}{3.233817in}}{\pgfqpoint{3.293938in}{3.223218in}}{\pgfqpoint{3.301751in}{3.215405in}}%
\pgfpathcurveto{\pgfqpoint{3.309565in}{3.207591in}}{\pgfqpoint{3.320164in}{3.203201in}}{\pgfqpoint{3.331214in}{3.203201in}}%
\pgfpathclose%
\pgfusepath{stroke,fill}%
\end{pgfscope}%
\begin{pgfscope}%
\pgfpathrectangle{\pgfqpoint{0.648703in}{0.548769in}}{\pgfqpoint{5.112893in}{3.102590in}}%
\pgfusepath{clip}%
\pgfsetbuttcap%
\pgfsetroundjoin%
\definecolor{currentfill}{rgb}{0.839216,0.152941,0.156863}%
\pgfsetfillcolor{currentfill}%
\pgfsetlinewidth{1.003750pt}%
\definecolor{currentstroke}{rgb}{0.839216,0.152941,0.156863}%
\pgfsetstrokecolor{currentstroke}%
\pgfsetdash{}{0pt}%
\pgfpathmoveto{\pgfqpoint{3.352328in}{3.266406in}}%
\pgfpathcurveto{\pgfqpoint{3.363378in}{3.266406in}}{\pgfqpoint{3.373977in}{3.270797in}}{\pgfqpoint{3.381790in}{3.278610in}}%
\pgfpathcurveto{\pgfqpoint{3.389604in}{3.286424in}}{\pgfqpoint{3.393994in}{3.297023in}}{\pgfqpoint{3.393994in}{3.308073in}}%
\pgfpathcurveto{\pgfqpoint{3.393994in}{3.319123in}}{\pgfqpoint{3.389604in}{3.329722in}}{\pgfqpoint{3.381790in}{3.337536in}}%
\pgfpathcurveto{\pgfqpoint{3.373977in}{3.345350in}}{\pgfqpoint{3.363378in}{3.349740in}}{\pgfqpoint{3.352328in}{3.349740in}}%
\pgfpathcurveto{\pgfqpoint{3.341277in}{3.349740in}}{\pgfqpoint{3.330678in}{3.345350in}}{\pgfqpoint{3.322865in}{3.337536in}}%
\pgfpathcurveto{\pgfqpoint{3.315051in}{3.329722in}}{\pgfqpoint{3.310661in}{3.319123in}}{\pgfqpoint{3.310661in}{3.308073in}}%
\pgfpathcurveto{\pgfqpoint{3.310661in}{3.297023in}}{\pgfqpoint{3.315051in}{3.286424in}}{\pgfqpoint{3.322865in}{3.278610in}}%
\pgfpathcurveto{\pgfqpoint{3.330678in}{3.270797in}}{\pgfqpoint{3.341277in}{3.266406in}}{\pgfqpoint{3.352328in}{3.266406in}}%
\pgfpathclose%
\pgfusepath{stroke,fill}%
\end{pgfscope}%
\begin{pgfscope}%
\pgfpathrectangle{\pgfqpoint{0.648703in}{0.548769in}}{\pgfqpoint{5.112893in}{3.102590in}}%
\pgfusepath{clip}%
\pgfsetbuttcap%
\pgfsetroundjoin%
\definecolor{currentfill}{rgb}{0.121569,0.466667,0.705882}%
\pgfsetfillcolor{currentfill}%
\pgfsetlinewidth{1.003750pt}%
\definecolor{currentstroke}{rgb}{0.121569,0.466667,0.705882}%
\pgfsetstrokecolor{currentstroke}%
\pgfsetdash{}{0pt}%
\pgfpathmoveto{\pgfqpoint{2.158706in}{2.811325in}}%
\pgfpathcurveto{\pgfqpoint{2.169756in}{2.811325in}}{\pgfqpoint{2.180355in}{2.815715in}}{\pgfqpoint{2.188169in}{2.823528in}}%
\pgfpathcurveto{\pgfqpoint{2.195983in}{2.831342in}}{\pgfqpoint{2.200373in}{2.841941in}}{\pgfqpoint{2.200373in}{2.852991in}}%
\pgfpathcurveto{\pgfqpoint{2.200373in}{2.864041in}}{\pgfqpoint{2.195983in}{2.874640in}}{\pgfqpoint{2.188169in}{2.882454in}}%
\pgfpathcurveto{\pgfqpoint{2.180355in}{2.890268in}}{\pgfqpoint{2.169756in}{2.894658in}}{\pgfqpoint{2.158706in}{2.894658in}}%
\pgfpathcurveto{\pgfqpoint{2.147656in}{2.894658in}}{\pgfqpoint{2.137057in}{2.890268in}}{\pgfqpoint{2.129243in}{2.882454in}}%
\pgfpathcurveto{\pgfqpoint{2.121430in}{2.874640in}}{\pgfqpoint{2.117040in}{2.864041in}}{\pgfqpoint{2.117040in}{2.852991in}}%
\pgfpathcurveto{\pgfqpoint{2.117040in}{2.841941in}}{\pgfqpoint{2.121430in}{2.831342in}}{\pgfqpoint{2.129243in}{2.823528in}}%
\pgfpathcurveto{\pgfqpoint{2.137057in}{2.815715in}}{\pgfqpoint{2.147656in}{2.811325in}}{\pgfqpoint{2.158706in}{2.811325in}}%
\pgfpathclose%
\pgfusepath{stroke,fill}%
\end{pgfscope}%
\begin{pgfscope}%
\pgfpathrectangle{\pgfqpoint{0.648703in}{0.548769in}}{\pgfqpoint{5.112893in}{3.102590in}}%
\pgfusepath{clip}%
\pgfsetbuttcap%
\pgfsetroundjoin%
\definecolor{currentfill}{rgb}{1.000000,0.498039,0.054902}%
\pgfsetfillcolor{currentfill}%
\pgfsetlinewidth{1.003750pt}%
\definecolor{currentstroke}{rgb}{1.000000,0.498039,0.054902}%
\pgfsetstrokecolor{currentstroke}%
\pgfsetdash{}{0pt}%
\pgfpathmoveto{\pgfqpoint{2.994458in}{3.186346in}}%
\pgfpathcurveto{\pgfqpoint{3.005508in}{3.186346in}}{\pgfqpoint{3.016107in}{3.190736in}}{\pgfqpoint{3.023921in}{3.198550in}}%
\pgfpathcurveto{\pgfqpoint{3.031734in}{3.206363in}}{\pgfqpoint{3.036124in}{3.216962in}}{\pgfqpoint{3.036124in}{3.228012in}}%
\pgfpathcurveto{\pgfqpoint{3.036124in}{3.239063in}}{\pgfqpoint{3.031734in}{3.249662in}}{\pgfqpoint{3.023921in}{3.257475in}}%
\pgfpathcurveto{\pgfqpoint{3.016107in}{3.265289in}}{\pgfqpoint{3.005508in}{3.269679in}}{\pgfqpoint{2.994458in}{3.269679in}}%
\pgfpathcurveto{\pgfqpoint{2.983408in}{3.269679in}}{\pgfqpoint{2.972809in}{3.265289in}}{\pgfqpoint{2.964995in}{3.257475in}}%
\pgfpathcurveto{\pgfqpoint{2.957181in}{3.249662in}}{\pgfqpoint{2.952791in}{3.239063in}}{\pgfqpoint{2.952791in}{3.228012in}}%
\pgfpathcurveto{\pgfqpoint{2.952791in}{3.216962in}}{\pgfqpoint{2.957181in}{3.206363in}}{\pgfqpoint{2.964995in}{3.198550in}}%
\pgfpathcurveto{\pgfqpoint{2.972809in}{3.190736in}}{\pgfqpoint{2.983408in}{3.186346in}}{\pgfqpoint{2.994458in}{3.186346in}}%
\pgfpathclose%
\pgfusepath{stroke,fill}%
\end{pgfscope}%
\begin{pgfscope}%
\pgfpathrectangle{\pgfqpoint{0.648703in}{0.548769in}}{\pgfqpoint{5.112893in}{3.102590in}}%
\pgfusepath{clip}%
\pgfsetbuttcap%
\pgfsetroundjoin%
\definecolor{currentfill}{rgb}{1.000000,0.498039,0.054902}%
\pgfsetfillcolor{currentfill}%
\pgfsetlinewidth{1.003750pt}%
\definecolor{currentstroke}{rgb}{1.000000,0.498039,0.054902}%
\pgfsetstrokecolor{currentstroke}%
\pgfsetdash{}{0pt}%
\pgfpathmoveto{\pgfqpoint{3.784464in}{3.194773in}}%
\pgfpathcurveto{\pgfqpoint{3.795514in}{3.194773in}}{\pgfqpoint{3.806113in}{3.199163in}}{\pgfqpoint{3.813927in}{3.206977in}}%
\pgfpathcurveto{\pgfqpoint{3.821740in}{3.214791in}}{\pgfqpoint{3.826131in}{3.225390in}}{\pgfqpoint{3.826131in}{3.236440in}}%
\pgfpathcurveto{\pgfqpoint{3.826131in}{3.247490in}}{\pgfqpoint{3.821740in}{3.258089in}}{\pgfqpoint{3.813927in}{3.265903in}}%
\pgfpathcurveto{\pgfqpoint{3.806113in}{3.273716in}}{\pgfqpoint{3.795514in}{3.278107in}}{\pgfqpoint{3.784464in}{3.278107in}}%
\pgfpathcurveto{\pgfqpoint{3.773414in}{3.278107in}}{\pgfqpoint{3.762815in}{3.273716in}}{\pgfqpoint{3.755001in}{3.265903in}}%
\pgfpathcurveto{\pgfqpoint{3.747188in}{3.258089in}}{\pgfqpoint{3.742797in}{3.247490in}}{\pgfqpoint{3.742797in}{3.236440in}}%
\pgfpathcurveto{\pgfqpoint{3.742797in}{3.225390in}}{\pgfqpoint{3.747188in}{3.214791in}}{\pgfqpoint{3.755001in}{3.206977in}}%
\pgfpathcurveto{\pgfqpoint{3.762815in}{3.199163in}}{\pgfqpoint{3.773414in}{3.194773in}}{\pgfqpoint{3.784464in}{3.194773in}}%
\pgfpathclose%
\pgfusepath{stroke,fill}%
\end{pgfscope}%
\begin{pgfscope}%
\pgfpathrectangle{\pgfqpoint{0.648703in}{0.548769in}}{\pgfqpoint{5.112893in}{3.102590in}}%
\pgfusepath{clip}%
\pgfsetbuttcap%
\pgfsetroundjoin%
\definecolor{currentfill}{rgb}{1.000000,0.498039,0.054902}%
\pgfsetfillcolor{currentfill}%
\pgfsetlinewidth{1.003750pt}%
\definecolor{currentstroke}{rgb}{1.000000,0.498039,0.054902}%
\pgfsetstrokecolor{currentstroke}%
\pgfsetdash{}{0pt}%
\pgfpathmoveto{\pgfqpoint{3.049011in}{3.363322in}}%
\pgfpathcurveto{\pgfqpoint{3.060062in}{3.363322in}}{\pgfqpoint{3.070661in}{3.367712in}}{\pgfqpoint{3.078474in}{3.375526in}}%
\pgfpathcurveto{\pgfqpoint{3.086288in}{3.383340in}}{\pgfqpoint{3.090678in}{3.393939in}}{\pgfqpoint{3.090678in}{3.404989in}}%
\pgfpathcurveto{\pgfqpoint{3.090678in}{3.416039in}}{\pgfqpoint{3.086288in}{3.426638in}}{\pgfqpoint{3.078474in}{3.434452in}}%
\pgfpathcurveto{\pgfqpoint{3.070661in}{3.442265in}}{\pgfqpoint{3.060062in}{3.446655in}}{\pgfqpoint{3.049011in}{3.446655in}}%
\pgfpathcurveto{\pgfqpoint{3.037961in}{3.446655in}}{\pgfqpoint{3.027362in}{3.442265in}}{\pgfqpoint{3.019549in}{3.434452in}}%
\pgfpathcurveto{\pgfqpoint{3.011735in}{3.426638in}}{\pgfqpoint{3.007345in}{3.416039in}}{\pgfqpoint{3.007345in}{3.404989in}}%
\pgfpathcurveto{\pgfqpoint{3.007345in}{3.393939in}}{\pgfqpoint{3.011735in}{3.383340in}}{\pgfqpoint{3.019549in}{3.375526in}}%
\pgfpathcurveto{\pgfqpoint{3.027362in}{3.367712in}}{\pgfqpoint{3.037961in}{3.363322in}}{\pgfqpoint{3.049011in}{3.363322in}}%
\pgfpathclose%
\pgfusepath{stroke,fill}%
\end{pgfscope}%
\begin{pgfscope}%
\pgfpathrectangle{\pgfqpoint{0.648703in}{0.548769in}}{\pgfqpoint{5.112893in}{3.102590in}}%
\pgfusepath{clip}%
\pgfsetbuttcap%
\pgfsetroundjoin%
\definecolor{currentfill}{rgb}{1.000000,0.498039,0.054902}%
\pgfsetfillcolor{currentfill}%
\pgfsetlinewidth{1.003750pt}%
\definecolor{currentstroke}{rgb}{1.000000,0.498039,0.054902}%
\pgfsetstrokecolor{currentstroke}%
\pgfsetdash{}{0pt}%
\pgfpathmoveto{\pgfqpoint{1.072678in}{3.257979in}}%
\pgfpathcurveto{\pgfqpoint{1.083728in}{3.257979in}}{\pgfqpoint{1.094327in}{3.262369in}}{\pgfqpoint{1.102141in}{3.270183in}}%
\pgfpathcurveto{\pgfqpoint{1.109954in}{3.277997in}}{\pgfqpoint{1.114345in}{3.288596in}}{\pgfqpoint{1.114345in}{3.299646in}}%
\pgfpathcurveto{\pgfqpoint{1.114345in}{3.310696in}}{\pgfqpoint{1.109954in}{3.321295in}}{\pgfqpoint{1.102141in}{3.329108in}}%
\pgfpathcurveto{\pgfqpoint{1.094327in}{3.336922in}}{\pgfqpoint{1.083728in}{3.341312in}}{\pgfqpoint{1.072678in}{3.341312in}}%
\pgfpathcurveto{\pgfqpoint{1.061628in}{3.341312in}}{\pgfqpoint{1.051029in}{3.336922in}}{\pgfqpoint{1.043215in}{3.329108in}}%
\pgfpathcurveto{\pgfqpoint{1.035402in}{3.321295in}}{\pgfqpoint{1.031011in}{3.310696in}}{\pgfqpoint{1.031011in}{3.299646in}}%
\pgfpathcurveto{\pgfqpoint{1.031011in}{3.288596in}}{\pgfqpoint{1.035402in}{3.277997in}}{\pgfqpoint{1.043215in}{3.270183in}}%
\pgfpathcurveto{\pgfqpoint{1.051029in}{3.262369in}}{\pgfqpoint{1.061628in}{3.257979in}}{\pgfqpoint{1.072678in}{3.257979in}}%
\pgfpathclose%
\pgfusepath{stroke,fill}%
\end{pgfscope}%
\begin{pgfscope}%
\pgfpathrectangle{\pgfqpoint{0.648703in}{0.548769in}}{\pgfqpoint{5.112893in}{3.102590in}}%
\pgfusepath{clip}%
\pgfsetbuttcap%
\pgfsetroundjoin%
\definecolor{currentfill}{rgb}{1.000000,0.498039,0.054902}%
\pgfsetfillcolor{currentfill}%
\pgfsetlinewidth{1.003750pt}%
\definecolor{currentstroke}{rgb}{1.000000,0.498039,0.054902}%
\pgfsetstrokecolor{currentstroke}%
\pgfsetdash{}{0pt}%
\pgfpathmoveto{\pgfqpoint{2.648780in}{3.190560in}}%
\pgfpathcurveto{\pgfqpoint{2.659830in}{3.190560in}}{\pgfqpoint{2.670430in}{3.194950in}}{\pgfqpoint{2.678243in}{3.202763in}}%
\pgfpathcurveto{\pgfqpoint{2.686057in}{3.210577in}}{\pgfqpoint{2.690447in}{3.221176in}}{\pgfqpoint{2.690447in}{3.232226in}}%
\pgfpathcurveto{\pgfqpoint{2.690447in}{3.243276in}}{\pgfqpoint{2.686057in}{3.253875in}}{\pgfqpoint{2.678243in}{3.261689in}}%
\pgfpathcurveto{\pgfqpoint{2.670430in}{3.269503in}}{\pgfqpoint{2.659830in}{3.273893in}}{\pgfqpoint{2.648780in}{3.273893in}}%
\pgfpathcurveto{\pgfqpoint{2.637730in}{3.273893in}}{\pgfqpoint{2.627131in}{3.269503in}}{\pgfqpoint{2.619318in}{3.261689in}}%
\pgfpathcurveto{\pgfqpoint{2.611504in}{3.253875in}}{\pgfqpoint{2.607114in}{3.243276in}}{\pgfqpoint{2.607114in}{3.232226in}}%
\pgfpathcurveto{\pgfqpoint{2.607114in}{3.221176in}}{\pgfqpoint{2.611504in}{3.210577in}}{\pgfqpoint{2.619318in}{3.202763in}}%
\pgfpathcurveto{\pgfqpoint{2.627131in}{3.194950in}}{\pgfqpoint{2.637730in}{3.190560in}}{\pgfqpoint{2.648780in}{3.190560in}}%
\pgfpathclose%
\pgfusepath{stroke,fill}%
\end{pgfscope}%
\begin{pgfscope}%
\pgfpathrectangle{\pgfqpoint{0.648703in}{0.548769in}}{\pgfqpoint{5.112893in}{3.102590in}}%
\pgfusepath{clip}%
\pgfsetbuttcap%
\pgfsetroundjoin%
\definecolor{currentfill}{rgb}{1.000000,0.498039,0.054902}%
\pgfsetfillcolor{currentfill}%
\pgfsetlinewidth{1.003750pt}%
\definecolor{currentstroke}{rgb}{1.000000,0.498039,0.054902}%
\pgfsetstrokecolor{currentstroke}%
\pgfsetdash{}{0pt}%
\pgfpathmoveto{\pgfqpoint{3.190081in}{3.190560in}}%
\pgfpathcurveto{\pgfqpoint{3.201132in}{3.190560in}}{\pgfqpoint{3.211731in}{3.194950in}}{\pgfqpoint{3.219544in}{3.202763in}}%
\pgfpathcurveto{\pgfqpoint{3.227358in}{3.210577in}}{\pgfqpoint{3.231748in}{3.221176in}}{\pgfqpoint{3.231748in}{3.232226in}}%
\pgfpathcurveto{\pgfqpoint{3.231748in}{3.243276in}}{\pgfqpoint{3.227358in}{3.253875in}}{\pgfqpoint{3.219544in}{3.261689in}}%
\pgfpathcurveto{\pgfqpoint{3.211731in}{3.269503in}}{\pgfqpoint{3.201132in}{3.273893in}}{\pgfqpoint{3.190081in}{3.273893in}}%
\pgfpathcurveto{\pgfqpoint{3.179031in}{3.273893in}}{\pgfqpoint{3.168432in}{3.269503in}}{\pgfqpoint{3.160619in}{3.261689in}}%
\pgfpathcurveto{\pgfqpoint{3.152805in}{3.253875in}}{\pgfqpoint{3.148415in}{3.243276in}}{\pgfqpoint{3.148415in}{3.232226in}}%
\pgfpathcurveto{\pgfqpoint{3.148415in}{3.221176in}}{\pgfqpoint{3.152805in}{3.210577in}}{\pgfqpoint{3.160619in}{3.202763in}}%
\pgfpathcurveto{\pgfqpoint{3.168432in}{3.194950in}}{\pgfqpoint{3.179031in}{3.190560in}}{\pgfqpoint{3.190081in}{3.190560in}}%
\pgfpathclose%
\pgfusepath{stroke,fill}%
\end{pgfscope}%
\begin{pgfscope}%
\pgfpathrectangle{\pgfqpoint{0.648703in}{0.548769in}}{\pgfqpoint{5.112893in}{3.102590in}}%
\pgfusepath{clip}%
\pgfsetbuttcap%
\pgfsetroundjoin%
\definecolor{currentfill}{rgb}{1.000000,0.498039,0.054902}%
\pgfsetfillcolor{currentfill}%
\pgfsetlinewidth{1.003750pt}%
\definecolor{currentstroke}{rgb}{1.000000,0.498039,0.054902}%
\pgfsetstrokecolor{currentstroke}%
\pgfsetdash{}{0pt}%
\pgfpathmoveto{\pgfqpoint{3.087884in}{3.182132in}}%
\pgfpathcurveto{\pgfqpoint{3.098935in}{3.182132in}}{\pgfqpoint{3.109534in}{3.186522in}}{\pgfqpoint{3.117347in}{3.194336in}}%
\pgfpathcurveto{\pgfqpoint{3.125161in}{3.202150in}}{\pgfqpoint{3.129551in}{3.212749in}}{\pgfqpoint{3.129551in}{3.223799in}}%
\pgfpathcurveto{\pgfqpoint{3.129551in}{3.234849in}}{\pgfqpoint{3.125161in}{3.245448in}}{\pgfqpoint{3.117347in}{3.253262in}}%
\pgfpathcurveto{\pgfqpoint{3.109534in}{3.261075in}}{\pgfqpoint{3.098935in}{3.265465in}}{\pgfqpoint{3.087884in}{3.265465in}}%
\pgfpathcurveto{\pgfqpoint{3.076834in}{3.265465in}}{\pgfqpoint{3.066235in}{3.261075in}}{\pgfqpoint{3.058422in}{3.253262in}}%
\pgfpathcurveto{\pgfqpoint{3.050608in}{3.245448in}}{\pgfqpoint{3.046218in}{3.234849in}}{\pgfqpoint{3.046218in}{3.223799in}}%
\pgfpathcurveto{\pgfqpoint{3.046218in}{3.212749in}}{\pgfqpoint{3.050608in}{3.202150in}}{\pgfqpoint{3.058422in}{3.194336in}}%
\pgfpathcurveto{\pgfqpoint{3.066235in}{3.186522in}}{\pgfqpoint{3.076834in}{3.182132in}}{\pgfqpoint{3.087884in}{3.182132in}}%
\pgfpathclose%
\pgfusepath{stroke,fill}%
\end{pgfscope}%
\begin{pgfscope}%
\pgfpathrectangle{\pgfqpoint{0.648703in}{0.548769in}}{\pgfqpoint{5.112893in}{3.102590in}}%
\pgfusepath{clip}%
\pgfsetbuttcap%
\pgfsetroundjoin%
\definecolor{currentfill}{rgb}{1.000000,0.498039,0.054902}%
\pgfsetfillcolor{currentfill}%
\pgfsetlinewidth{1.003750pt}%
\definecolor{currentstroke}{rgb}{1.000000,0.498039,0.054902}%
\pgfsetstrokecolor{currentstroke}%
\pgfsetdash{}{0pt}%
\pgfpathmoveto{\pgfqpoint{3.898745in}{3.186346in}}%
\pgfpathcurveto{\pgfqpoint{3.909795in}{3.186346in}}{\pgfqpoint{3.920394in}{3.190736in}}{\pgfqpoint{3.928208in}{3.198550in}}%
\pgfpathcurveto{\pgfqpoint{3.936021in}{3.206363in}}{\pgfqpoint{3.940411in}{3.216962in}}{\pgfqpoint{3.940411in}{3.228012in}}%
\pgfpathcurveto{\pgfqpoint{3.940411in}{3.239063in}}{\pgfqpoint{3.936021in}{3.249662in}}{\pgfqpoint{3.928208in}{3.257475in}}%
\pgfpathcurveto{\pgfqpoint{3.920394in}{3.265289in}}{\pgfqpoint{3.909795in}{3.269679in}}{\pgfqpoint{3.898745in}{3.269679in}}%
\pgfpathcurveto{\pgfqpoint{3.887695in}{3.269679in}}{\pgfqpoint{3.877096in}{3.265289in}}{\pgfqpoint{3.869282in}{3.257475in}}%
\pgfpathcurveto{\pgfqpoint{3.861468in}{3.249662in}}{\pgfqpoint{3.857078in}{3.239063in}}{\pgfqpoint{3.857078in}{3.228012in}}%
\pgfpathcurveto{\pgfqpoint{3.857078in}{3.216962in}}{\pgfqpoint{3.861468in}{3.206363in}}{\pgfqpoint{3.869282in}{3.198550in}}%
\pgfpathcurveto{\pgfqpoint{3.877096in}{3.190736in}}{\pgfqpoint{3.887695in}{3.186346in}}{\pgfqpoint{3.898745in}{3.186346in}}%
\pgfpathclose%
\pgfusepath{stroke,fill}%
\end{pgfscope}%
\begin{pgfscope}%
\pgfpathrectangle{\pgfqpoint{0.648703in}{0.548769in}}{\pgfqpoint{5.112893in}{3.102590in}}%
\pgfusepath{clip}%
\pgfsetbuttcap%
\pgfsetroundjoin%
\definecolor{currentfill}{rgb}{0.121569,0.466667,0.705882}%
\pgfsetfillcolor{currentfill}%
\pgfsetlinewidth{1.003750pt}%
\definecolor{currentstroke}{rgb}{0.121569,0.466667,0.705882}%
\pgfsetstrokecolor{currentstroke}%
\pgfsetdash{}{0pt}%
\pgfpathmoveto{\pgfqpoint{0.883434in}{0.691823in}}%
\pgfpathcurveto{\pgfqpoint{0.894484in}{0.691823in}}{\pgfqpoint{0.905083in}{0.696213in}}{\pgfqpoint{0.912897in}{0.704026in}}%
\pgfpathcurveto{\pgfqpoint{0.920710in}{0.711840in}}{\pgfqpoint{0.925101in}{0.722439in}}{\pgfqpoint{0.925101in}{0.733489in}}%
\pgfpathcurveto{\pgfqpoint{0.925101in}{0.744539in}}{\pgfqpoint{0.920710in}{0.755138in}}{\pgfqpoint{0.912897in}{0.762952in}}%
\pgfpathcurveto{\pgfqpoint{0.905083in}{0.770766in}}{\pgfqpoint{0.894484in}{0.775156in}}{\pgfqpoint{0.883434in}{0.775156in}}%
\pgfpathcurveto{\pgfqpoint{0.872384in}{0.775156in}}{\pgfqpoint{0.861785in}{0.770766in}}{\pgfqpoint{0.853971in}{0.762952in}}%
\pgfpathcurveto{\pgfqpoint{0.846158in}{0.755138in}}{\pgfqpoint{0.841767in}{0.744539in}}{\pgfqpoint{0.841767in}{0.733489in}}%
\pgfpathcurveto{\pgfqpoint{0.841767in}{0.722439in}}{\pgfqpoint{0.846158in}{0.711840in}}{\pgfqpoint{0.853971in}{0.704026in}}%
\pgfpathcurveto{\pgfqpoint{0.861785in}{0.696213in}}{\pgfqpoint{0.872384in}{0.691823in}}{\pgfqpoint{0.883434in}{0.691823in}}%
\pgfpathclose%
\pgfusepath{stroke,fill}%
\end{pgfscope}%
\begin{pgfscope}%
\pgfpathrectangle{\pgfqpoint{0.648703in}{0.548769in}}{\pgfqpoint{5.112893in}{3.102590in}}%
\pgfusepath{clip}%
\pgfsetbuttcap%
\pgfsetroundjoin%
\definecolor{currentfill}{rgb}{0.839216,0.152941,0.156863}%
\pgfsetfillcolor{currentfill}%
\pgfsetlinewidth{1.003750pt}%
\definecolor{currentstroke}{rgb}{0.839216,0.152941,0.156863}%
\pgfsetstrokecolor{currentstroke}%
\pgfsetdash{}{0pt}%
\pgfpathmoveto{\pgfqpoint{1.481275in}{3.215842in}}%
\pgfpathcurveto{\pgfqpoint{1.492325in}{3.215842in}}{\pgfqpoint{1.502924in}{3.220232in}}{\pgfqpoint{1.510737in}{3.228046in}}%
\pgfpathcurveto{\pgfqpoint{1.518551in}{3.235859in}}{\pgfqpoint{1.522941in}{3.246458in}}{\pgfqpoint{1.522941in}{3.257508in}}%
\pgfpathcurveto{\pgfqpoint{1.522941in}{3.268559in}}{\pgfqpoint{1.518551in}{3.279158in}}{\pgfqpoint{1.510737in}{3.286971in}}%
\pgfpathcurveto{\pgfqpoint{1.502924in}{3.294785in}}{\pgfqpoint{1.492325in}{3.299175in}}{\pgfqpoint{1.481275in}{3.299175in}}%
\pgfpathcurveto{\pgfqpoint{1.470225in}{3.299175in}}{\pgfqpoint{1.459625in}{3.294785in}}{\pgfqpoint{1.451812in}{3.286971in}}%
\pgfpathcurveto{\pgfqpoint{1.443998in}{3.279158in}}{\pgfqpoint{1.439608in}{3.268559in}}{\pgfqpoint{1.439608in}{3.257508in}}%
\pgfpathcurveto{\pgfqpoint{1.439608in}{3.246458in}}{\pgfqpoint{1.443998in}{3.235859in}}{\pgfqpoint{1.451812in}{3.228046in}}%
\pgfpathcurveto{\pgfqpoint{1.459625in}{3.220232in}}{\pgfqpoint{1.470225in}{3.215842in}}{\pgfqpoint{1.481275in}{3.215842in}}%
\pgfpathclose%
\pgfusepath{stroke,fill}%
\end{pgfscope}%
\begin{pgfscope}%
\pgfpathrectangle{\pgfqpoint{0.648703in}{0.548769in}}{\pgfqpoint{5.112893in}{3.102590in}}%
\pgfusepath{clip}%
\pgfsetbuttcap%
\pgfsetroundjoin%
\definecolor{currentfill}{rgb}{1.000000,0.498039,0.054902}%
\pgfsetfillcolor{currentfill}%
\pgfsetlinewidth{1.003750pt}%
\definecolor{currentstroke}{rgb}{1.000000,0.498039,0.054902}%
\pgfsetstrokecolor{currentstroke}%
\pgfsetdash{}{0pt}%
\pgfpathmoveto{\pgfqpoint{3.804071in}{3.186346in}}%
\pgfpathcurveto{\pgfqpoint{3.815121in}{3.186346in}}{\pgfqpoint{3.825720in}{3.190736in}}{\pgfqpoint{3.833534in}{3.198550in}}%
\pgfpathcurveto{\pgfqpoint{3.841348in}{3.206363in}}{\pgfqpoint{3.845738in}{3.216962in}}{\pgfqpoint{3.845738in}{3.228012in}}%
\pgfpathcurveto{\pgfqpoint{3.845738in}{3.239063in}}{\pgfqpoint{3.841348in}{3.249662in}}{\pgfqpoint{3.833534in}{3.257475in}}%
\pgfpathcurveto{\pgfqpoint{3.825720in}{3.265289in}}{\pgfqpoint{3.815121in}{3.269679in}}{\pgfqpoint{3.804071in}{3.269679in}}%
\pgfpathcurveto{\pgfqpoint{3.793021in}{3.269679in}}{\pgfqpoint{3.782422in}{3.265289in}}{\pgfqpoint{3.774608in}{3.257475in}}%
\pgfpathcurveto{\pgfqpoint{3.766795in}{3.249662in}}{\pgfqpoint{3.762404in}{3.239063in}}{\pgfqpoint{3.762404in}{3.228012in}}%
\pgfpathcurveto{\pgfqpoint{3.762404in}{3.216962in}}{\pgfqpoint{3.766795in}{3.206363in}}{\pgfqpoint{3.774608in}{3.198550in}}%
\pgfpathcurveto{\pgfqpoint{3.782422in}{3.190736in}}{\pgfqpoint{3.793021in}{3.186346in}}{\pgfqpoint{3.804071in}{3.186346in}}%
\pgfpathclose%
\pgfusepath{stroke,fill}%
\end{pgfscope}%
\begin{pgfscope}%
\pgfpathrectangle{\pgfqpoint{0.648703in}{0.548769in}}{\pgfqpoint{5.112893in}{3.102590in}}%
\pgfusepath{clip}%
\pgfsetbuttcap%
\pgfsetroundjoin%
\definecolor{currentfill}{rgb}{0.121569,0.466667,0.705882}%
\pgfsetfillcolor{currentfill}%
\pgfsetlinewidth{1.003750pt}%
\definecolor{currentstroke}{rgb}{0.121569,0.466667,0.705882}%
\pgfsetstrokecolor{currentstroke}%
\pgfsetdash{}{0pt}%
\pgfpathmoveto{\pgfqpoint{1.458422in}{1.146904in}}%
\pgfpathcurveto{\pgfqpoint{1.469472in}{1.146904in}}{\pgfqpoint{1.480071in}{1.151295in}}{\pgfqpoint{1.487885in}{1.159108in}}%
\pgfpathcurveto{\pgfqpoint{1.495699in}{1.166922in}}{\pgfqpoint{1.500089in}{1.177521in}}{\pgfqpoint{1.500089in}{1.188571in}}%
\pgfpathcurveto{\pgfqpoint{1.500089in}{1.199621in}}{\pgfqpoint{1.495699in}{1.210220in}}{\pgfqpoint{1.487885in}{1.218034in}}%
\pgfpathcurveto{\pgfqpoint{1.480071in}{1.225848in}}{\pgfqpoint{1.469472in}{1.230238in}}{\pgfqpoint{1.458422in}{1.230238in}}%
\pgfpathcurveto{\pgfqpoint{1.447372in}{1.230238in}}{\pgfqpoint{1.436773in}{1.225848in}}{\pgfqpoint{1.428959in}{1.218034in}}%
\pgfpathcurveto{\pgfqpoint{1.421146in}{1.210220in}}{\pgfqpoint{1.416756in}{1.199621in}}{\pgfqpoint{1.416756in}{1.188571in}}%
\pgfpathcurveto{\pgfqpoint{1.416756in}{1.177521in}}{\pgfqpoint{1.421146in}{1.166922in}}{\pgfqpoint{1.428959in}{1.159108in}}%
\pgfpathcurveto{\pgfqpoint{1.436773in}{1.151295in}}{\pgfqpoint{1.447372in}{1.146904in}}{\pgfqpoint{1.458422in}{1.146904in}}%
\pgfpathclose%
\pgfusepath{stroke,fill}%
\end{pgfscope}%
\begin{pgfscope}%
\pgfpathrectangle{\pgfqpoint{0.648703in}{0.548769in}}{\pgfqpoint{5.112893in}{3.102590in}}%
\pgfusepath{clip}%
\pgfsetbuttcap%
\pgfsetroundjoin%
\definecolor{currentfill}{rgb}{0.121569,0.466667,0.705882}%
\pgfsetfillcolor{currentfill}%
\pgfsetlinewidth{1.003750pt}%
\definecolor{currentstroke}{rgb}{0.121569,0.466667,0.705882}%
\pgfsetstrokecolor{currentstroke}%
\pgfsetdash{}{0pt}%
\pgfpathmoveto{\pgfqpoint{1.681445in}{1.096340in}}%
\pgfpathcurveto{\pgfqpoint{1.692495in}{1.096340in}}{\pgfqpoint{1.703094in}{1.100730in}}{\pgfqpoint{1.710907in}{1.108544in}}%
\pgfpathcurveto{\pgfqpoint{1.718721in}{1.116357in}}{\pgfqpoint{1.723111in}{1.126956in}}{\pgfqpoint{1.723111in}{1.138006in}}%
\pgfpathcurveto{\pgfqpoint{1.723111in}{1.149057in}}{\pgfqpoint{1.718721in}{1.159656in}}{\pgfqpoint{1.710907in}{1.167469in}}%
\pgfpathcurveto{\pgfqpoint{1.703094in}{1.175283in}}{\pgfqpoint{1.692495in}{1.179673in}}{\pgfqpoint{1.681445in}{1.179673in}}%
\pgfpathcurveto{\pgfqpoint{1.670394in}{1.179673in}}{\pgfqpoint{1.659795in}{1.175283in}}{\pgfqpoint{1.651982in}{1.167469in}}%
\pgfpathcurveto{\pgfqpoint{1.644168in}{1.159656in}}{\pgfqpoint{1.639778in}{1.149057in}}{\pgfqpoint{1.639778in}{1.138006in}}%
\pgfpathcurveto{\pgfqpoint{1.639778in}{1.126956in}}{\pgfqpoint{1.644168in}{1.116357in}}{\pgfqpoint{1.651982in}{1.108544in}}%
\pgfpathcurveto{\pgfqpoint{1.659795in}{1.100730in}}{\pgfqpoint{1.670394in}{1.096340in}}{\pgfqpoint{1.681445in}{1.096340in}}%
\pgfpathclose%
\pgfusepath{stroke,fill}%
\end{pgfscope}%
\begin{pgfscope}%
\pgfpathrectangle{\pgfqpoint{0.648703in}{0.548769in}}{\pgfqpoint{5.112893in}{3.102590in}}%
\pgfusepath{clip}%
\pgfsetbuttcap%
\pgfsetroundjoin%
\definecolor{currentfill}{rgb}{1.000000,0.498039,0.054902}%
\pgfsetfillcolor{currentfill}%
\pgfsetlinewidth{1.003750pt}%
\definecolor{currentstroke}{rgb}{1.000000,0.498039,0.054902}%
\pgfsetstrokecolor{currentstroke}%
\pgfsetdash{}{0pt}%
\pgfpathmoveto{\pgfqpoint{1.970515in}{3.190560in}}%
\pgfpathcurveto{\pgfqpoint{1.981565in}{3.190560in}}{\pgfqpoint{1.992164in}{3.194950in}}{\pgfqpoint{1.999978in}{3.202763in}}%
\pgfpathcurveto{\pgfqpoint{2.007791in}{3.210577in}}{\pgfqpoint{2.012182in}{3.221176in}}{\pgfqpoint{2.012182in}{3.232226in}}%
\pgfpathcurveto{\pgfqpoint{2.012182in}{3.243276in}}{\pgfqpoint{2.007791in}{3.253875in}}{\pgfqpoint{1.999978in}{3.261689in}}%
\pgfpathcurveto{\pgfqpoint{1.992164in}{3.269503in}}{\pgfqpoint{1.981565in}{3.273893in}}{\pgfqpoint{1.970515in}{3.273893in}}%
\pgfpathcurveto{\pgfqpoint{1.959465in}{3.273893in}}{\pgfqpoint{1.948866in}{3.269503in}}{\pgfqpoint{1.941052in}{3.261689in}}%
\pgfpathcurveto{\pgfqpoint{1.933238in}{3.253875in}}{\pgfqpoint{1.928848in}{3.243276in}}{\pgfqpoint{1.928848in}{3.232226in}}%
\pgfpathcurveto{\pgfqpoint{1.928848in}{3.221176in}}{\pgfqpoint{1.933238in}{3.210577in}}{\pgfqpoint{1.941052in}{3.202763in}}%
\pgfpathcurveto{\pgfqpoint{1.948866in}{3.194950in}}{\pgfqpoint{1.959465in}{3.190560in}}{\pgfqpoint{1.970515in}{3.190560in}}%
\pgfpathclose%
\pgfusepath{stroke,fill}%
\end{pgfscope}%
\begin{pgfscope}%
\pgfpathrectangle{\pgfqpoint{0.648703in}{0.548769in}}{\pgfqpoint{5.112893in}{3.102590in}}%
\pgfusepath{clip}%
\pgfsetbuttcap%
\pgfsetroundjoin%
\definecolor{currentfill}{rgb}{0.121569,0.466667,0.705882}%
\pgfsetfillcolor{currentfill}%
\pgfsetlinewidth{1.003750pt}%
\definecolor{currentstroke}{rgb}{0.121569,0.466667,0.705882}%
\pgfsetstrokecolor{currentstroke}%
\pgfsetdash{}{0pt}%
\pgfpathmoveto{\pgfqpoint{1.394025in}{1.033134in}}%
\pgfpathcurveto{\pgfqpoint{1.405075in}{1.033134in}}{\pgfqpoint{1.415674in}{1.037524in}}{\pgfqpoint{1.423488in}{1.045338in}}%
\pgfpathcurveto{\pgfqpoint{1.431302in}{1.053151in}}{\pgfqpoint{1.435692in}{1.063751in}}{\pgfqpoint{1.435692in}{1.074801in}}%
\pgfpathcurveto{\pgfqpoint{1.435692in}{1.085851in}}{\pgfqpoint{1.431302in}{1.096450in}}{\pgfqpoint{1.423488in}{1.104263in}}%
\pgfpathcurveto{\pgfqpoint{1.415674in}{1.112077in}}{\pgfqpoint{1.405075in}{1.116467in}}{\pgfqpoint{1.394025in}{1.116467in}}%
\pgfpathcurveto{\pgfqpoint{1.382975in}{1.116467in}}{\pgfqpoint{1.372376in}{1.112077in}}{\pgfqpoint{1.364562in}{1.104263in}}%
\pgfpathcurveto{\pgfqpoint{1.356749in}{1.096450in}}{\pgfqpoint{1.352358in}{1.085851in}}{\pgfqpoint{1.352358in}{1.074801in}}%
\pgfpathcurveto{\pgfqpoint{1.352358in}{1.063751in}}{\pgfqpoint{1.356749in}{1.053151in}}{\pgfqpoint{1.364562in}{1.045338in}}%
\pgfpathcurveto{\pgfqpoint{1.372376in}{1.037524in}}{\pgfqpoint{1.382975in}{1.033134in}}{\pgfqpoint{1.394025in}{1.033134in}}%
\pgfpathclose%
\pgfusepath{stroke,fill}%
\end{pgfscope}%
\begin{pgfscope}%
\pgfpathrectangle{\pgfqpoint{0.648703in}{0.548769in}}{\pgfqpoint{5.112893in}{3.102590in}}%
\pgfusepath{clip}%
\pgfsetbuttcap%
\pgfsetroundjoin%
\definecolor{currentfill}{rgb}{0.121569,0.466667,0.705882}%
\pgfsetfillcolor{currentfill}%
\pgfsetlinewidth{1.003750pt}%
\definecolor{currentstroke}{rgb}{0.121569,0.466667,0.705882}%
\pgfsetstrokecolor{currentstroke}%
\pgfsetdash{}{0pt}%
\pgfpathmoveto{\pgfqpoint{1.299837in}{0.822448in}}%
\pgfpathcurveto{\pgfqpoint{1.310887in}{0.822448in}}{\pgfqpoint{1.321487in}{0.826838in}}{\pgfqpoint{1.329300in}{0.834652in}}%
\pgfpathcurveto{\pgfqpoint{1.337114in}{0.842465in}}{\pgfqpoint{1.341504in}{0.853064in}}{\pgfqpoint{1.341504in}{0.864115in}}%
\pgfpathcurveto{\pgfqpoint{1.341504in}{0.875165in}}{\pgfqpoint{1.337114in}{0.885764in}}{\pgfqpoint{1.329300in}{0.893577in}}%
\pgfpathcurveto{\pgfqpoint{1.321487in}{0.901391in}}{\pgfqpoint{1.310887in}{0.905781in}}{\pgfqpoint{1.299837in}{0.905781in}}%
\pgfpathcurveto{\pgfqpoint{1.288787in}{0.905781in}}{\pgfqpoint{1.278188in}{0.901391in}}{\pgfqpoint{1.270375in}{0.893577in}}%
\pgfpathcurveto{\pgfqpoint{1.262561in}{0.885764in}}{\pgfqpoint{1.258171in}{0.875165in}}{\pgfqpoint{1.258171in}{0.864115in}}%
\pgfpathcurveto{\pgfqpoint{1.258171in}{0.853064in}}{\pgfqpoint{1.262561in}{0.842465in}}{\pgfqpoint{1.270375in}{0.834652in}}%
\pgfpathcurveto{\pgfqpoint{1.278188in}{0.826838in}}{\pgfqpoint{1.288787in}{0.822448in}}{\pgfqpoint{1.299837in}{0.822448in}}%
\pgfpathclose%
\pgfusepath{stroke,fill}%
\end{pgfscope}%
\begin{pgfscope}%
\pgfpathrectangle{\pgfqpoint{0.648703in}{0.548769in}}{\pgfqpoint{5.112893in}{3.102590in}}%
\pgfusepath{clip}%
\pgfsetbuttcap%
\pgfsetroundjoin%
\definecolor{currentfill}{rgb}{0.839216,0.152941,0.156863}%
\pgfsetfillcolor{currentfill}%
\pgfsetlinewidth{1.003750pt}%
\definecolor{currentstroke}{rgb}{0.839216,0.152941,0.156863}%
\pgfsetstrokecolor{currentstroke}%
\pgfsetdash{}{0pt}%
\pgfpathmoveto{\pgfqpoint{4.301848in}{3.203201in}}%
\pgfpathcurveto{\pgfqpoint{4.312898in}{3.203201in}}{\pgfqpoint{4.323497in}{3.207591in}}{\pgfqpoint{4.331311in}{3.215405in}}%
\pgfpathcurveto{\pgfqpoint{4.339124in}{3.223218in}}{\pgfqpoint{4.343515in}{3.233817in}}{\pgfqpoint{4.343515in}{3.244867in}}%
\pgfpathcurveto{\pgfqpoint{4.343515in}{3.255917in}}{\pgfqpoint{4.339124in}{3.266516in}}{\pgfqpoint{4.331311in}{3.274330in}}%
\pgfpathcurveto{\pgfqpoint{4.323497in}{3.282144in}}{\pgfqpoint{4.312898in}{3.286534in}}{\pgfqpoint{4.301848in}{3.286534in}}%
\pgfpathcurveto{\pgfqpoint{4.290798in}{3.286534in}}{\pgfqpoint{4.280199in}{3.282144in}}{\pgfqpoint{4.272385in}{3.274330in}}%
\pgfpathcurveto{\pgfqpoint{4.264572in}{3.266516in}}{\pgfqpoint{4.260181in}{3.255917in}}{\pgfqpoint{4.260181in}{3.244867in}}%
\pgfpathcurveto{\pgfqpoint{4.260181in}{3.233817in}}{\pgfqpoint{4.264572in}{3.223218in}}{\pgfqpoint{4.272385in}{3.215405in}}%
\pgfpathcurveto{\pgfqpoint{4.280199in}{3.207591in}}{\pgfqpoint{4.290798in}{3.203201in}}{\pgfqpoint{4.301848in}{3.203201in}}%
\pgfpathclose%
\pgfusepath{stroke,fill}%
\end{pgfscope}%
\begin{pgfscope}%
\pgfsetbuttcap%
\pgfsetroundjoin%
\definecolor{currentfill}{rgb}{0.000000,0.000000,0.000000}%
\pgfsetfillcolor{currentfill}%
\pgfsetlinewidth{0.803000pt}%
\definecolor{currentstroke}{rgb}{0.000000,0.000000,0.000000}%
\pgfsetstrokecolor{currentstroke}%
\pgfsetdash{}{0pt}%
\pgfsys@defobject{currentmarker}{\pgfqpoint{0.000000in}{-0.048611in}}{\pgfqpoint{0.000000in}{0.000000in}}{%
\pgfpathmoveto{\pgfqpoint{0.000000in}{0.000000in}}%
\pgfpathlineto{\pgfqpoint{0.000000in}{-0.048611in}}%
\pgfusepath{stroke,fill}%
}%
\begin{pgfscope}%
\pgfsys@transformshift{0.831231in}{0.548769in}%
\pgfsys@useobject{currentmarker}{}%
\end{pgfscope}%
\end{pgfscope}%
\begin{pgfscope}%
\definecolor{textcolor}{rgb}{0.000000,0.000000,0.000000}%
\pgfsetstrokecolor{textcolor}%
\pgfsetfillcolor{textcolor}%
\pgftext[x=0.831231in,y=0.451547in,,top]{\color{textcolor}\sffamily\fontsize{10.000000}{12.000000}\selectfont \(\displaystyle {0.0}\)}%
\end{pgfscope}%
\begin{pgfscope}%
\pgfsetbuttcap%
\pgfsetroundjoin%
\definecolor{currentfill}{rgb}{0.000000,0.000000,0.000000}%
\pgfsetfillcolor{currentfill}%
\pgfsetlinewidth{0.803000pt}%
\definecolor{currentstroke}{rgb}{0.000000,0.000000,0.000000}%
\pgfsetstrokecolor{currentstroke}%
\pgfsetdash{}{0pt}%
\pgfsys@defobject{currentmarker}{\pgfqpoint{0.000000in}{-0.048611in}}{\pgfqpoint{0.000000in}{0.000000in}}{%
\pgfpathmoveto{\pgfqpoint{0.000000in}{0.000000in}}%
\pgfpathlineto{\pgfqpoint{0.000000in}{-0.048611in}}%
\pgfusepath{stroke,fill}%
}%
\begin{pgfscope}%
\pgfsys@transformshift{1.324268in}{0.548769in}%
\pgfsys@useobject{currentmarker}{}%
\end{pgfscope}%
\end{pgfscope}%
\begin{pgfscope}%
\definecolor{textcolor}{rgb}{0.000000,0.000000,0.000000}%
\pgfsetstrokecolor{textcolor}%
\pgfsetfillcolor{textcolor}%
\pgftext[x=1.324268in,y=0.451547in,,top]{\color{textcolor}\sffamily\fontsize{10.000000}{12.000000}\selectfont \(\displaystyle {0.1}\)}%
\end{pgfscope}%
\begin{pgfscope}%
\pgfsetbuttcap%
\pgfsetroundjoin%
\definecolor{currentfill}{rgb}{0.000000,0.000000,0.000000}%
\pgfsetfillcolor{currentfill}%
\pgfsetlinewidth{0.803000pt}%
\definecolor{currentstroke}{rgb}{0.000000,0.000000,0.000000}%
\pgfsetstrokecolor{currentstroke}%
\pgfsetdash{}{0pt}%
\pgfsys@defobject{currentmarker}{\pgfqpoint{0.000000in}{-0.048611in}}{\pgfqpoint{0.000000in}{0.000000in}}{%
\pgfpathmoveto{\pgfqpoint{0.000000in}{0.000000in}}%
\pgfpathlineto{\pgfqpoint{0.000000in}{-0.048611in}}%
\pgfusepath{stroke,fill}%
}%
\begin{pgfscope}%
\pgfsys@transformshift{1.817304in}{0.548769in}%
\pgfsys@useobject{currentmarker}{}%
\end{pgfscope}%
\end{pgfscope}%
\begin{pgfscope}%
\definecolor{textcolor}{rgb}{0.000000,0.000000,0.000000}%
\pgfsetstrokecolor{textcolor}%
\pgfsetfillcolor{textcolor}%
\pgftext[x=1.817304in,y=0.451547in,,top]{\color{textcolor}\sffamily\fontsize{10.000000}{12.000000}\selectfont \(\displaystyle {0.2}\)}%
\end{pgfscope}%
\begin{pgfscope}%
\pgfsetbuttcap%
\pgfsetroundjoin%
\definecolor{currentfill}{rgb}{0.000000,0.000000,0.000000}%
\pgfsetfillcolor{currentfill}%
\pgfsetlinewidth{0.803000pt}%
\definecolor{currentstroke}{rgb}{0.000000,0.000000,0.000000}%
\pgfsetstrokecolor{currentstroke}%
\pgfsetdash{}{0pt}%
\pgfsys@defobject{currentmarker}{\pgfqpoint{0.000000in}{-0.048611in}}{\pgfqpoint{0.000000in}{0.000000in}}{%
\pgfpathmoveto{\pgfqpoint{0.000000in}{0.000000in}}%
\pgfpathlineto{\pgfqpoint{0.000000in}{-0.048611in}}%
\pgfusepath{stroke,fill}%
}%
\begin{pgfscope}%
\pgfsys@transformshift{2.310341in}{0.548769in}%
\pgfsys@useobject{currentmarker}{}%
\end{pgfscope}%
\end{pgfscope}%
\begin{pgfscope}%
\definecolor{textcolor}{rgb}{0.000000,0.000000,0.000000}%
\pgfsetstrokecolor{textcolor}%
\pgfsetfillcolor{textcolor}%
\pgftext[x=2.310341in,y=0.451547in,,top]{\color{textcolor}\sffamily\fontsize{10.000000}{12.000000}\selectfont \(\displaystyle {0.3}\)}%
\end{pgfscope}%
\begin{pgfscope}%
\pgfsetbuttcap%
\pgfsetroundjoin%
\definecolor{currentfill}{rgb}{0.000000,0.000000,0.000000}%
\pgfsetfillcolor{currentfill}%
\pgfsetlinewidth{0.803000pt}%
\definecolor{currentstroke}{rgb}{0.000000,0.000000,0.000000}%
\pgfsetstrokecolor{currentstroke}%
\pgfsetdash{}{0pt}%
\pgfsys@defobject{currentmarker}{\pgfqpoint{0.000000in}{-0.048611in}}{\pgfqpoint{0.000000in}{0.000000in}}{%
\pgfpathmoveto{\pgfqpoint{0.000000in}{0.000000in}}%
\pgfpathlineto{\pgfqpoint{0.000000in}{-0.048611in}}%
\pgfusepath{stroke,fill}%
}%
\begin{pgfscope}%
\pgfsys@transformshift{2.803377in}{0.548769in}%
\pgfsys@useobject{currentmarker}{}%
\end{pgfscope}%
\end{pgfscope}%
\begin{pgfscope}%
\definecolor{textcolor}{rgb}{0.000000,0.000000,0.000000}%
\pgfsetstrokecolor{textcolor}%
\pgfsetfillcolor{textcolor}%
\pgftext[x=2.803377in,y=0.451547in,,top]{\color{textcolor}\sffamily\fontsize{10.000000}{12.000000}\selectfont \(\displaystyle {0.4}\)}%
\end{pgfscope}%
\begin{pgfscope}%
\pgfsetbuttcap%
\pgfsetroundjoin%
\definecolor{currentfill}{rgb}{0.000000,0.000000,0.000000}%
\pgfsetfillcolor{currentfill}%
\pgfsetlinewidth{0.803000pt}%
\definecolor{currentstroke}{rgb}{0.000000,0.000000,0.000000}%
\pgfsetstrokecolor{currentstroke}%
\pgfsetdash{}{0pt}%
\pgfsys@defobject{currentmarker}{\pgfqpoint{0.000000in}{-0.048611in}}{\pgfqpoint{0.000000in}{0.000000in}}{%
\pgfpathmoveto{\pgfqpoint{0.000000in}{0.000000in}}%
\pgfpathlineto{\pgfqpoint{0.000000in}{-0.048611in}}%
\pgfusepath{stroke,fill}%
}%
\begin{pgfscope}%
\pgfsys@transformshift{3.296414in}{0.548769in}%
\pgfsys@useobject{currentmarker}{}%
\end{pgfscope}%
\end{pgfscope}%
\begin{pgfscope}%
\definecolor{textcolor}{rgb}{0.000000,0.000000,0.000000}%
\pgfsetstrokecolor{textcolor}%
\pgfsetfillcolor{textcolor}%
\pgftext[x=3.296414in,y=0.451547in,,top]{\color{textcolor}\sffamily\fontsize{10.000000}{12.000000}\selectfont \(\displaystyle {0.5}\)}%
\end{pgfscope}%
\begin{pgfscope}%
\pgfsetbuttcap%
\pgfsetroundjoin%
\definecolor{currentfill}{rgb}{0.000000,0.000000,0.000000}%
\pgfsetfillcolor{currentfill}%
\pgfsetlinewidth{0.803000pt}%
\definecolor{currentstroke}{rgb}{0.000000,0.000000,0.000000}%
\pgfsetstrokecolor{currentstroke}%
\pgfsetdash{}{0pt}%
\pgfsys@defobject{currentmarker}{\pgfqpoint{0.000000in}{-0.048611in}}{\pgfqpoint{0.000000in}{0.000000in}}{%
\pgfpathmoveto{\pgfqpoint{0.000000in}{0.000000in}}%
\pgfpathlineto{\pgfqpoint{0.000000in}{-0.048611in}}%
\pgfusepath{stroke,fill}%
}%
\begin{pgfscope}%
\pgfsys@transformshift{3.789450in}{0.548769in}%
\pgfsys@useobject{currentmarker}{}%
\end{pgfscope}%
\end{pgfscope}%
\begin{pgfscope}%
\definecolor{textcolor}{rgb}{0.000000,0.000000,0.000000}%
\pgfsetstrokecolor{textcolor}%
\pgfsetfillcolor{textcolor}%
\pgftext[x=3.789450in,y=0.451547in,,top]{\color{textcolor}\sffamily\fontsize{10.000000}{12.000000}\selectfont \(\displaystyle {0.6}\)}%
\end{pgfscope}%
\begin{pgfscope}%
\pgfsetbuttcap%
\pgfsetroundjoin%
\definecolor{currentfill}{rgb}{0.000000,0.000000,0.000000}%
\pgfsetfillcolor{currentfill}%
\pgfsetlinewidth{0.803000pt}%
\definecolor{currentstroke}{rgb}{0.000000,0.000000,0.000000}%
\pgfsetstrokecolor{currentstroke}%
\pgfsetdash{}{0pt}%
\pgfsys@defobject{currentmarker}{\pgfqpoint{0.000000in}{-0.048611in}}{\pgfqpoint{0.000000in}{0.000000in}}{%
\pgfpathmoveto{\pgfqpoint{0.000000in}{0.000000in}}%
\pgfpathlineto{\pgfqpoint{0.000000in}{-0.048611in}}%
\pgfusepath{stroke,fill}%
}%
\begin{pgfscope}%
\pgfsys@transformshift{4.282487in}{0.548769in}%
\pgfsys@useobject{currentmarker}{}%
\end{pgfscope}%
\end{pgfscope}%
\begin{pgfscope}%
\definecolor{textcolor}{rgb}{0.000000,0.000000,0.000000}%
\pgfsetstrokecolor{textcolor}%
\pgfsetfillcolor{textcolor}%
\pgftext[x=4.282487in,y=0.451547in,,top]{\color{textcolor}\sffamily\fontsize{10.000000}{12.000000}\selectfont \(\displaystyle {0.7}\)}%
\end{pgfscope}%
\begin{pgfscope}%
\pgfsetbuttcap%
\pgfsetroundjoin%
\definecolor{currentfill}{rgb}{0.000000,0.000000,0.000000}%
\pgfsetfillcolor{currentfill}%
\pgfsetlinewidth{0.803000pt}%
\definecolor{currentstroke}{rgb}{0.000000,0.000000,0.000000}%
\pgfsetstrokecolor{currentstroke}%
\pgfsetdash{}{0pt}%
\pgfsys@defobject{currentmarker}{\pgfqpoint{0.000000in}{-0.048611in}}{\pgfqpoint{0.000000in}{0.000000in}}{%
\pgfpathmoveto{\pgfqpoint{0.000000in}{0.000000in}}%
\pgfpathlineto{\pgfqpoint{0.000000in}{-0.048611in}}%
\pgfusepath{stroke,fill}%
}%
\begin{pgfscope}%
\pgfsys@transformshift{4.775524in}{0.548769in}%
\pgfsys@useobject{currentmarker}{}%
\end{pgfscope}%
\end{pgfscope}%
\begin{pgfscope}%
\definecolor{textcolor}{rgb}{0.000000,0.000000,0.000000}%
\pgfsetstrokecolor{textcolor}%
\pgfsetfillcolor{textcolor}%
\pgftext[x=4.775524in,y=0.451547in,,top]{\color{textcolor}\sffamily\fontsize{10.000000}{12.000000}\selectfont \(\displaystyle {0.8}\)}%
\end{pgfscope}%
\begin{pgfscope}%
\pgfsetbuttcap%
\pgfsetroundjoin%
\definecolor{currentfill}{rgb}{0.000000,0.000000,0.000000}%
\pgfsetfillcolor{currentfill}%
\pgfsetlinewidth{0.803000pt}%
\definecolor{currentstroke}{rgb}{0.000000,0.000000,0.000000}%
\pgfsetstrokecolor{currentstroke}%
\pgfsetdash{}{0pt}%
\pgfsys@defobject{currentmarker}{\pgfqpoint{0.000000in}{-0.048611in}}{\pgfqpoint{0.000000in}{0.000000in}}{%
\pgfpathmoveto{\pgfqpoint{0.000000in}{0.000000in}}%
\pgfpathlineto{\pgfqpoint{0.000000in}{-0.048611in}}%
\pgfusepath{stroke,fill}%
}%
\begin{pgfscope}%
\pgfsys@transformshift{5.268560in}{0.548769in}%
\pgfsys@useobject{currentmarker}{}%
\end{pgfscope}%
\end{pgfscope}%
\begin{pgfscope}%
\definecolor{textcolor}{rgb}{0.000000,0.000000,0.000000}%
\pgfsetstrokecolor{textcolor}%
\pgfsetfillcolor{textcolor}%
\pgftext[x=5.268560in,y=0.451547in,,top]{\color{textcolor}\sffamily\fontsize{10.000000}{12.000000}\selectfont \(\displaystyle {0.9}\)}%
\end{pgfscope}%
\begin{pgfscope}%
\pgfsetbuttcap%
\pgfsetroundjoin%
\definecolor{currentfill}{rgb}{0.000000,0.000000,0.000000}%
\pgfsetfillcolor{currentfill}%
\pgfsetlinewidth{0.803000pt}%
\definecolor{currentstroke}{rgb}{0.000000,0.000000,0.000000}%
\pgfsetstrokecolor{currentstroke}%
\pgfsetdash{}{0pt}%
\pgfsys@defobject{currentmarker}{\pgfqpoint{0.000000in}{-0.048611in}}{\pgfqpoint{0.000000in}{0.000000in}}{%
\pgfpathmoveto{\pgfqpoint{0.000000in}{0.000000in}}%
\pgfpathlineto{\pgfqpoint{0.000000in}{-0.048611in}}%
\pgfusepath{stroke,fill}%
}%
\begin{pgfscope}%
\pgfsys@transformshift{5.761597in}{0.548769in}%
\pgfsys@useobject{currentmarker}{}%
\end{pgfscope}%
\end{pgfscope}%
\begin{pgfscope}%
\definecolor{textcolor}{rgb}{0.000000,0.000000,0.000000}%
\pgfsetstrokecolor{textcolor}%
\pgfsetfillcolor{textcolor}%
\pgftext[x=5.761597in,y=0.451547in,,top]{\color{textcolor}\sffamily\fontsize{10.000000}{12.000000}\selectfont \(\displaystyle {1.0}\)}%
\end{pgfscope}%
\begin{pgfscope}%
\definecolor{textcolor}{rgb}{0.000000,0.000000,0.000000}%
\pgfsetstrokecolor{textcolor}%
\pgfsetfillcolor{textcolor}%
\pgftext[x=3.205150in,y=0.272658in,,top]{\color{textcolor}\sffamily\fontsize{10.000000}{12.000000}\selectfont Alias Edge Count}%
\end{pgfscope}%
\begin{pgfscope}%
\definecolor{textcolor}{rgb}{0.000000,0.000000,0.000000}%
\pgfsetstrokecolor{textcolor}%
\pgfsetfillcolor{textcolor}%
\pgftext[x=5.761597in,y=0.286547in,right,top]{\color{textcolor}\sffamily\fontsize{10.000000}{12.000000}\selectfont \(\displaystyle \times{10^{8}}{}\)}%
\end{pgfscope}%
\begin{pgfscope}%
\pgfsetbuttcap%
\pgfsetroundjoin%
\definecolor{currentfill}{rgb}{0.000000,0.000000,0.000000}%
\pgfsetfillcolor{currentfill}%
\pgfsetlinewidth{0.803000pt}%
\definecolor{currentstroke}{rgb}{0.000000,0.000000,0.000000}%
\pgfsetstrokecolor{currentstroke}%
\pgfsetdash{}{0pt}%
\pgfsys@defobject{currentmarker}{\pgfqpoint{-0.048611in}{0.000000in}}{\pgfqpoint{0.000000in}{0.000000in}}{%
\pgfpathmoveto{\pgfqpoint{0.000000in}{0.000000in}}%
\pgfpathlineto{\pgfqpoint{-0.048611in}{0.000000in}}%
\pgfusepath{stroke,fill}%
}%
\begin{pgfscope}%
\pgfsys@transformshift{0.648703in}{0.699779in}%
\pgfsys@useobject{currentmarker}{}%
\end{pgfscope}%
\end{pgfscope}%
\begin{pgfscope}%
\definecolor{textcolor}{rgb}{0.000000,0.000000,0.000000}%
\pgfsetstrokecolor{textcolor}%
\pgfsetfillcolor{textcolor}%
\pgftext[x=0.482036in, y=0.651585in, left, base]{\color{textcolor}\sffamily\fontsize{10.000000}{12.000000}\selectfont \(\displaystyle {0}\)}%
\end{pgfscope}%
\begin{pgfscope}%
\pgfsetbuttcap%
\pgfsetroundjoin%
\definecolor{currentfill}{rgb}{0.000000,0.000000,0.000000}%
\pgfsetfillcolor{currentfill}%
\pgfsetlinewidth{0.803000pt}%
\definecolor{currentstroke}{rgb}{0.000000,0.000000,0.000000}%
\pgfsetstrokecolor{currentstroke}%
\pgfsetdash{}{0pt}%
\pgfsys@defobject{currentmarker}{\pgfqpoint{-0.048611in}{0.000000in}}{\pgfqpoint{0.000000in}{0.000000in}}{%
\pgfpathmoveto{\pgfqpoint{0.000000in}{0.000000in}}%
\pgfpathlineto{\pgfqpoint{-0.048611in}{0.000000in}}%
\pgfusepath{stroke,fill}%
}%
\begin{pgfscope}%
\pgfsys@transformshift{0.648703in}{1.121152in}%
\pgfsys@useobject{currentmarker}{}%
\end{pgfscope}%
\end{pgfscope}%
\begin{pgfscope}%
\definecolor{textcolor}{rgb}{0.000000,0.000000,0.000000}%
\pgfsetstrokecolor{textcolor}%
\pgfsetfillcolor{textcolor}%
\pgftext[x=0.343147in, y=1.072957in, left, base]{\color{textcolor}\sffamily\fontsize{10.000000}{12.000000}\selectfont \(\displaystyle {100}\)}%
\end{pgfscope}%
\begin{pgfscope}%
\pgfsetbuttcap%
\pgfsetroundjoin%
\definecolor{currentfill}{rgb}{0.000000,0.000000,0.000000}%
\pgfsetfillcolor{currentfill}%
\pgfsetlinewidth{0.803000pt}%
\definecolor{currentstroke}{rgb}{0.000000,0.000000,0.000000}%
\pgfsetstrokecolor{currentstroke}%
\pgfsetdash{}{0pt}%
\pgfsys@defobject{currentmarker}{\pgfqpoint{-0.048611in}{0.000000in}}{\pgfqpoint{0.000000in}{0.000000in}}{%
\pgfpathmoveto{\pgfqpoint{0.000000in}{0.000000in}}%
\pgfpathlineto{\pgfqpoint{-0.048611in}{0.000000in}}%
\pgfusepath{stroke,fill}%
}%
\begin{pgfscope}%
\pgfsys@transformshift{0.648703in}{1.542524in}%
\pgfsys@useobject{currentmarker}{}%
\end{pgfscope}%
\end{pgfscope}%
\begin{pgfscope}%
\definecolor{textcolor}{rgb}{0.000000,0.000000,0.000000}%
\pgfsetstrokecolor{textcolor}%
\pgfsetfillcolor{textcolor}%
\pgftext[x=0.343147in, y=1.494329in, left, base]{\color{textcolor}\sffamily\fontsize{10.000000}{12.000000}\selectfont \(\displaystyle {200}\)}%
\end{pgfscope}%
\begin{pgfscope}%
\pgfsetbuttcap%
\pgfsetroundjoin%
\definecolor{currentfill}{rgb}{0.000000,0.000000,0.000000}%
\pgfsetfillcolor{currentfill}%
\pgfsetlinewidth{0.803000pt}%
\definecolor{currentstroke}{rgb}{0.000000,0.000000,0.000000}%
\pgfsetstrokecolor{currentstroke}%
\pgfsetdash{}{0pt}%
\pgfsys@defobject{currentmarker}{\pgfqpoint{-0.048611in}{0.000000in}}{\pgfqpoint{0.000000in}{0.000000in}}{%
\pgfpathmoveto{\pgfqpoint{0.000000in}{0.000000in}}%
\pgfpathlineto{\pgfqpoint{-0.048611in}{0.000000in}}%
\pgfusepath{stroke,fill}%
}%
\begin{pgfscope}%
\pgfsys@transformshift{0.648703in}{1.963896in}%
\pgfsys@useobject{currentmarker}{}%
\end{pgfscope}%
\end{pgfscope}%
\begin{pgfscope}%
\definecolor{textcolor}{rgb}{0.000000,0.000000,0.000000}%
\pgfsetstrokecolor{textcolor}%
\pgfsetfillcolor{textcolor}%
\pgftext[x=0.343147in, y=1.915701in, left, base]{\color{textcolor}\sffamily\fontsize{10.000000}{12.000000}\selectfont \(\displaystyle {300}\)}%
\end{pgfscope}%
\begin{pgfscope}%
\pgfsetbuttcap%
\pgfsetroundjoin%
\definecolor{currentfill}{rgb}{0.000000,0.000000,0.000000}%
\pgfsetfillcolor{currentfill}%
\pgfsetlinewidth{0.803000pt}%
\definecolor{currentstroke}{rgb}{0.000000,0.000000,0.000000}%
\pgfsetstrokecolor{currentstroke}%
\pgfsetdash{}{0pt}%
\pgfsys@defobject{currentmarker}{\pgfqpoint{-0.048611in}{0.000000in}}{\pgfqpoint{0.000000in}{0.000000in}}{%
\pgfpathmoveto{\pgfqpoint{0.000000in}{0.000000in}}%
\pgfpathlineto{\pgfqpoint{-0.048611in}{0.000000in}}%
\pgfusepath{stroke,fill}%
}%
\begin{pgfscope}%
\pgfsys@transformshift{0.648703in}{2.385268in}%
\pgfsys@useobject{currentmarker}{}%
\end{pgfscope}%
\end{pgfscope}%
\begin{pgfscope}%
\definecolor{textcolor}{rgb}{0.000000,0.000000,0.000000}%
\pgfsetstrokecolor{textcolor}%
\pgfsetfillcolor{textcolor}%
\pgftext[x=0.343147in, y=2.337074in, left, base]{\color{textcolor}\sffamily\fontsize{10.000000}{12.000000}\selectfont \(\displaystyle {400}\)}%
\end{pgfscope}%
\begin{pgfscope}%
\pgfsetbuttcap%
\pgfsetroundjoin%
\definecolor{currentfill}{rgb}{0.000000,0.000000,0.000000}%
\pgfsetfillcolor{currentfill}%
\pgfsetlinewidth{0.803000pt}%
\definecolor{currentstroke}{rgb}{0.000000,0.000000,0.000000}%
\pgfsetstrokecolor{currentstroke}%
\pgfsetdash{}{0pt}%
\pgfsys@defobject{currentmarker}{\pgfqpoint{-0.048611in}{0.000000in}}{\pgfqpoint{0.000000in}{0.000000in}}{%
\pgfpathmoveto{\pgfqpoint{0.000000in}{0.000000in}}%
\pgfpathlineto{\pgfqpoint{-0.048611in}{0.000000in}}%
\pgfusepath{stroke,fill}%
}%
\begin{pgfscope}%
\pgfsys@transformshift{0.648703in}{2.806640in}%
\pgfsys@useobject{currentmarker}{}%
\end{pgfscope}%
\end{pgfscope}%
\begin{pgfscope}%
\definecolor{textcolor}{rgb}{0.000000,0.000000,0.000000}%
\pgfsetstrokecolor{textcolor}%
\pgfsetfillcolor{textcolor}%
\pgftext[x=0.343147in, y=2.758446in, left, base]{\color{textcolor}\sffamily\fontsize{10.000000}{12.000000}\selectfont \(\displaystyle {500}\)}%
\end{pgfscope}%
\begin{pgfscope}%
\pgfsetbuttcap%
\pgfsetroundjoin%
\definecolor{currentfill}{rgb}{0.000000,0.000000,0.000000}%
\pgfsetfillcolor{currentfill}%
\pgfsetlinewidth{0.803000pt}%
\definecolor{currentstroke}{rgb}{0.000000,0.000000,0.000000}%
\pgfsetstrokecolor{currentstroke}%
\pgfsetdash{}{0pt}%
\pgfsys@defobject{currentmarker}{\pgfqpoint{-0.048611in}{0.000000in}}{\pgfqpoint{0.000000in}{0.000000in}}{%
\pgfpathmoveto{\pgfqpoint{0.000000in}{0.000000in}}%
\pgfpathlineto{\pgfqpoint{-0.048611in}{0.000000in}}%
\pgfusepath{stroke,fill}%
}%
\begin{pgfscope}%
\pgfsys@transformshift{0.648703in}{3.228012in}%
\pgfsys@useobject{currentmarker}{}%
\end{pgfscope}%
\end{pgfscope}%
\begin{pgfscope}%
\definecolor{textcolor}{rgb}{0.000000,0.000000,0.000000}%
\pgfsetstrokecolor{textcolor}%
\pgfsetfillcolor{textcolor}%
\pgftext[x=0.343147in, y=3.179818in, left, base]{\color{textcolor}\sffamily\fontsize{10.000000}{12.000000}\selectfont \(\displaystyle {600}\)}%
\end{pgfscope}%
\begin{pgfscope}%
\pgfsetbuttcap%
\pgfsetroundjoin%
\definecolor{currentfill}{rgb}{0.000000,0.000000,0.000000}%
\pgfsetfillcolor{currentfill}%
\pgfsetlinewidth{0.803000pt}%
\definecolor{currentstroke}{rgb}{0.000000,0.000000,0.000000}%
\pgfsetstrokecolor{currentstroke}%
\pgfsetdash{}{0pt}%
\pgfsys@defobject{currentmarker}{\pgfqpoint{-0.048611in}{0.000000in}}{\pgfqpoint{0.000000in}{0.000000in}}{%
\pgfpathmoveto{\pgfqpoint{0.000000in}{0.000000in}}%
\pgfpathlineto{\pgfqpoint{-0.048611in}{0.000000in}}%
\pgfusepath{stroke,fill}%
}%
\begin{pgfscope}%
\pgfsys@transformshift{0.648703in}{3.649385in}%
\pgfsys@useobject{currentmarker}{}%
\end{pgfscope}%
\end{pgfscope}%
\begin{pgfscope}%
\definecolor{textcolor}{rgb}{0.000000,0.000000,0.000000}%
\pgfsetstrokecolor{textcolor}%
\pgfsetfillcolor{textcolor}%
\pgftext[x=0.343147in, y=3.601190in, left, base]{\color{textcolor}\sffamily\fontsize{10.000000}{12.000000}\selectfont \(\displaystyle {700}\)}%
\end{pgfscope}%
\begin{pgfscope}%
\definecolor{textcolor}{rgb}{0.000000,0.000000,0.000000}%
\pgfsetstrokecolor{textcolor}%
\pgfsetfillcolor{textcolor}%
\pgftext[x=0.287592in,y=2.100064in,,bottom,rotate=90.000000]{\color{textcolor}\sffamily\fontsize{10.000000}{12.000000}\selectfont Data Flow Time (s)}%
\end{pgfscope}%
\begin{pgfscope}%
\pgfpathrectangle{\pgfqpoint{0.648703in}{0.548769in}}{\pgfqpoint{5.112893in}{3.102590in}}%
\pgfusepath{clip}%
\pgfsetrectcap%
\pgfsetroundjoin%
\pgfsetlinewidth{1.505625pt}%
\definecolor{currentstroke}{rgb}{0.000000,0.500000,0.000000}%
\pgfsetstrokecolor{currentstroke}%
\pgfsetdash{}{0pt}%
\pgfpathmoveto{\pgfqpoint{0.831231in}{0.689796in}}%
\pgfpathlineto{\pgfqpoint{0.868105in}{0.741360in}}%
\pgfpathlineto{\pgfqpoint{0.904980in}{0.795254in}}%
\pgfpathlineto{\pgfqpoint{0.941854in}{0.851275in}}%
\pgfpathlineto{\pgfqpoint{0.978728in}{0.909226in}}%
\pgfpathlineto{\pgfqpoint{1.015602in}{0.968914in}}%
\pgfpathlineto{\pgfqpoint{1.052477in}{1.030150in}}%
\pgfpathlineto{\pgfqpoint{1.089351in}{1.092752in}}%
\pgfpathlineto{\pgfqpoint{1.126225in}{1.156540in}}%
\pgfpathlineto{\pgfqpoint{1.163099in}{1.221341in}}%
\pgfpathlineto{\pgfqpoint{1.199974in}{1.286987in}}%
\pgfpathlineto{\pgfqpoint{1.236848in}{1.353312in}}%
\pgfpathlineto{\pgfqpoint{1.273722in}{1.420159in}}%
\pgfpathlineto{\pgfqpoint{1.310596in}{1.487373in}}%
\pgfpathlineto{\pgfqpoint{1.347471in}{1.554804in}}%
\pgfpathlineto{\pgfqpoint{1.384345in}{1.622308in}}%
\pgfpathlineto{\pgfqpoint{1.421219in}{1.689744in}}%
\pgfpathlineto{\pgfqpoint{1.458094in}{1.756979in}}%
\pgfpathlineto{\pgfqpoint{1.494968in}{1.823881in}}%
\pgfpathlineto{\pgfqpoint{1.531842in}{1.890325in}}%
\pgfpathlineto{\pgfqpoint{1.568716in}{1.956192in}}%
\pgfpathlineto{\pgfqpoint{1.605591in}{2.021364in}}%
\pgfpathlineto{\pgfqpoint{1.642465in}{2.085732in}}%
\pgfpathlineto{\pgfqpoint{1.679339in}{2.149190in}}%
\pgfpathlineto{\pgfqpoint{1.716213in}{2.211635in}}%
\pgfpathlineto{\pgfqpoint{1.753088in}{2.272972in}}%
\pgfpathlineto{\pgfqpoint{1.789962in}{2.333110in}}%
\pgfpathlineto{\pgfqpoint{1.826836in}{2.391961in}}%
\pgfpathlineto{\pgfqpoint{1.863710in}{2.449444in}}%
\pgfpathlineto{\pgfqpoint{1.900585in}{2.505481in}}%
\pgfpathlineto{\pgfqpoint{1.937459in}{2.560001in}}%
\pgfpathlineto{\pgfqpoint{1.974333in}{2.612937in}}%
\pgfpathlineto{\pgfqpoint{2.011208in}{2.664225in}}%
\pgfpathlineto{\pgfqpoint{2.048082in}{2.713807in}}%
\pgfpathlineto{\pgfqpoint{2.084956in}{2.761632in}}%
\pgfpathlineto{\pgfqpoint{2.121830in}{2.807651in}}%
\pgfpathlineto{\pgfqpoint{2.158705in}{2.851821in}}%
\pgfpathlineto{\pgfqpoint{2.195579in}{2.894103in}}%
\pgfpathlineto{\pgfqpoint{2.232453in}{2.934464in}}%
\pgfpathlineto{\pgfqpoint{2.269327in}{2.972876in}}%
\pgfpathlineto{\pgfqpoint{2.306202in}{3.009313in}}%
\pgfpathlineto{\pgfqpoint{2.343076in}{3.043758in}}%
\pgfpathlineto{\pgfqpoint{2.379950in}{3.076196in}}%
\pgfpathlineto{\pgfqpoint{2.416824in}{3.106617in}}%
\pgfpathlineto{\pgfqpoint{2.453699in}{3.135018in}}%
\pgfpathlineto{\pgfqpoint{2.490573in}{3.161398in}}%
\pgfpathlineto{\pgfqpoint{2.527447in}{3.185762in}}%
\pgfpathlineto{\pgfqpoint{2.564322in}{3.208120in}}%
\pgfpathlineto{\pgfqpoint{2.601196in}{3.228488in}}%
\pgfpathlineto{\pgfqpoint{2.638070in}{3.246884in}}%
\pgfpathlineto{\pgfqpoint{2.674944in}{3.263334in}}%
\pgfpathlineto{\pgfqpoint{2.711819in}{3.277866in}}%
\pgfpathlineto{\pgfqpoint{2.748693in}{3.290514in}}%
\pgfpathlineto{\pgfqpoint{2.785567in}{3.301318in}}%
\pgfpathlineto{\pgfqpoint{2.822441in}{3.310320in}}%
\pgfpathlineto{\pgfqpoint{2.859316in}{3.317571in}}%
\pgfpathlineto{\pgfqpoint{2.896190in}{3.323122in}}%
\pgfpathlineto{\pgfqpoint{2.933064in}{3.327033in}}%
\pgfpathlineto{\pgfqpoint{2.969938in}{3.329366in}}%
\pgfpathlineto{\pgfqpoint{3.006813in}{3.330190in}}%
\pgfpathlineto{\pgfqpoint{3.043687in}{3.329577in}}%
\pgfpathlineto{\pgfqpoint{3.080561in}{3.327604in}}%
\pgfpathlineto{\pgfqpoint{3.117436in}{3.324354in}}%
\pgfpathlineto{\pgfqpoint{3.154310in}{3.319915in}}%
\pgfpathlineto{\pgfqpoint{3.191184in}{3.314377in}}%
\pgfpathlineto{\pgfqpoint{3.228058in}{3.307839in}}%
\pgfpathlineto{\pgfqpoint{3.264933in}{3.300402in}}%
\pgfpathlineto{\pgfqpoint{3.301807in}{3.292172in}}%
\pgfpathlineto{\pgfqpoint{3.338681in}{3.283261in}}%
\pgfpathlineto{\pgfqpoint{3.375555in}{3.273785in}}%
\pgfpathlineto{\pgfqpoint{3.412430in}{3.263864in}}%
\pgfpathlineto{\pgfqpoint{3.449304in}{3.253626in}}%
\pgfpathlineto{\pgfqpoint{3.486178in}{3.243200in}}%
\pgfpathlineto{\pgfqpoint{3.523052in}{3.232723in}}%
\pgfpathlineto{\pgfqpoint{3.559927in}{3.222334in}}%
\pgfpathlineto{\pgfqpoint{3.596801in}{3.212179in}}%
\pgfpathlineto{\pgfqpoint{3.633675in}{3.202407in}}%
\pgfpathlineto{\pgfqpoint{3.670550in}{3.193174in}}%
\pgfpathlineto{\pgfqpoint{3.707424in}{3.184639in}}%
\pgfpathlineto{\pgfqpoint{3.744298in}{3.176967in}}%
\pgfpathlineto{\pgfqpoint{3.781172in}{3.170327in}}%
\pgfpathlineto{\pgfqpoint{3.818047in}{3.164893in}}%
\pgfpathlineto{\pgfqpoint{3.854921in}{3.160845in}}%
\pgfpathlineto{\pgfqpoint{3.891795in}{3.158365in}}%
\pgfpathlineto{\pgfqpoint{3.928669in}{3.157644in}}%
\pgfpathlineto{\pgfqpoint{3.965544in}{3.158873in}}%
\pgfpathlineto{\pgfqpoint{4.002418in}{3.162253in}}%
\pgfpathlineto{\pgfqpoint{4.039292in}{3.167985in}}%
\pgfpathlineto{\pgfqpoint{4.076166in}{3.176278in}}%
\pgfpathlineto{\pgfqpoint{4.113041in}{3.187344in}}%
\pgfpathlineto{\pgfqpoint{4.149915in}{3.201402in}}%
\pgfpathlineto{\pgfqpoint{4.186789in}{3.218674in}}%
\pgfpathlineto{\pgfqpoint{4.223664in}{3.239386in}}%
\pgfpathlineto{\pgfqpoint{4.260538in}{3.263772in}}%
\pgfpathlineto{\pgfqpoint{4.297412in}{3.292068in}}%
\pgfpathlineto{\pgfqpoint{4.334286in}{3.324516in}}%
\pgfpathlineto{\pgfqpoint{4.371161in}{3.361362in}}%
\pgfpathlineto{\pgfqpoint{4.408035in}{3.402859in}}%
\pgfpathlineto{\pgfqpoint{4.444909in}{3.449262in}}%
\pgfpathlineto{\pgfqpoint{4.481783in}{3.500832in}}%
\pgfusepath{stroke}%
\end{pgfscope}%
\begin{pgfscope}%
\pgfsetrectcap%
\pgfsetmiterjoin%
\pgfsetlinewidth{0.803000pt}%
\definecolor{currentstroke}{rgb}{0.000000,0.000000,0.000000}%
\pgfsetstrokecolor{currentstroke}%
\pgfsetdash{}{0pt}%
\pgfpathmoveto{\pgfqpoint{0.648703in}{0.548769in}}%
\pgfpathlineto{\pgfqpoint{0.648703in}{3.651359in}}%
\pgfusepath{stroke}%
\end{pgfscope}%
\begin{pgfscope}%
\pgfsetrectcap%
\pgfsetmiterjoin%
\pgfsetlinewidth{0.803000pt}%
\definecolor{currentstroke}{rgb}{0.000000,0.000000,0.000000}%
\pgfsetstrokecolor{currentstroke}%
\pgfsetdash{}{0pt}%
\pgfpathmoveto{\pgfqpoint{5.761597in}{0.548769in}}%
\pgfpathlineto{\pgfqpoint{5.761597in}{3.651359in}}%
\pgfusepath{stroke}%
\end{pgfscope}%
\begin{pgfscope}%
\pgfsetrectcap%
\pgfsetmiterjoin%
\pgfsetlinewidth{0.803000pt}%
\definecolor{currentstroke}{rgb}{0.000000,0.000000,0.000000}%
\pgfsetstrokecolor{currentstroke}%
\pgfsetdash{}{0pt}%
\pgfpathmoveto{\pgfqpoint{0.648703in}{0.548769in}}%
\pgfpathlineto{\pgfqpoint{5.761597in}{0.548769in}}%
\pgfusepath{stroke}%
\end{pgfscope}%
\begin{pgfscope}%
\pgfsetrectcap%
\pgfsetmiterjoin%
\pgfsetlinewidth{0.803000pt}%
\definecolor{currentstroke}{rgb}{0.000000,0.000000,0.000000}%
\pgfsetstrokecolor{currentstroke}%
\pgfsetdash{}{0pt}%
\pgfpathmoveto{\pgfqpoint{0.648703in}{3.651359in}}%
\pgfpathlineto{\pgfqpoint{5.761597in}{3.651359in}}%
\pgfusepath{stroke}%
\end{pgfscope}%
\begin{pgfscope}%
\definecolor{textcolor}{rgb}{0.000000,0.000000,0.000000}%
\pgfsetstrokecolor{textcolor}%
\pgfsetfillcolor{textcolor}%
\pgftext[x=3.205150in,y=3.734692in,,base]{\color{textcolor}\sffamily\fontsize{12.000000}{14.400000}\selectfont Backward}%
\end{pgfscope}%
\begin{pgfscope}%
\pgfsetbuttcap%
\pgfsetmiterjoin%
\definecolor{currentfill}{rgb}{1.000000,1.000000,1.000000}%
\pgfsetfillcolor{currentfill}%
\pgfsetfillopacity{0.800000}%
\pgfsetlinewidth{1.003750pt}%
\definecolor{currentstroke}{rgb}{0.800000,0.800000,0.800000}%
\pgfsetstrokecolor{currentstroke}%
\pgfsetstrokeopacity{0.800000}%
\pgfsetdash{}{0pt}%
\pgfpathmoveto{\pgfqpoint{4.212013in}{0.618213in}}%
\pgfpathlineto{\pgfqpoint{5.664374in}{0.618213in}}%
\pgfpathquadraticcurveto{\pgfqpoint{5.692152in}{0.618213in}}{\pgfqpoint{5.692152in}{0.645991in}}%
\pgfpathlineto{\pgfqpoint{5.692152in}{1.410297in}}%
\pgfpathquadraticcurveto{\pgfqpoint{5.692152in}{1.438074in}}{\pgfqpoint{5.664374in}{1.438074in}}%
\pgfpathlineto{\pgfqpoint{4.212013in}{1.438074in}}%
\pgfpathquadraticcurveto{\pgfqpoint{4.184236in}{1.438074in}}{\pgfqpoint{4.184236in}{1.410297in}}%
\pgfpathlineto{\pgfqpoint{4.184236in}{0.645991in}}%
\pgfpathquadraticcurveto{\pgfqpoint{4.184236in}{0.618213in}}{\pgfqpoint{4.212013in}{0.618213in}}%
\pgfpathclose%
\pgfusepath{stroke,fill}%
\end{pgfscope}%
\begin{pgfscope}%
\pgfsetbuttcap%
\pgfsetroundjoin%
\definecolor{currentfill}{rgb}{0.121569,0.466667,0.705882}%
\pgfsetfillcolor{currentfill}%
\pgfsetlinewidth{1.003750pt}%
\definecolor{currentstroke}{rgb}{0.121569,0.466667,0.705882}%
\pgfsetstrokecolor{currentstroke}%
\pgfsetdash{}{0pt}%
\pgfsys@defobject{currentmarker}{\pgfqpoint{-0.034722in}{-0.034722in}}{\pgfqpoint{0.034722in}{0.034722in}}{%
\pgfpathmoveto{\pgfqpoint{0.000000in}{-0.034722in}}%
\pgfpathcurveto{\pgfqpoint{0.009208in}{-0.034722in}}{\pgfqpoint{0.018041in}{-0.031064in}}{\pgfqpoint{0.024552in}{-0.024552in}}%
\pgfpathcurveto{\pgfqpoint{0.031064in}{-0.018041in}}{\pgfqpoint{0.034722in}{-0.009208in}}{\pgfqpoint{0.034722in}{0.000000in}}%
\pgfpathcurveto{\pgfqpoint{0.034722in}{0.009208in}}{\pgfqpoint{0.031064in}{0.018041in}}{\pgfqpoint{0.024552in}{0.024552in}}%
\pgfpathcurveto{\pgfqpoint{0.018041in}{0.031064in}}{\pgfqpoint{0.009208in}{0.034722in}}{\pgfqpoint{0.000000in}{0.034722in}}%
\pgfpathcurveto{\pgfqpoint{-0.009208in}{0.034722in}}{\pgfqpoint{-0.018041in}{0.031064in}}{\pgfqpoint{-0.024552in}{0.024552in}}%
\pgfpathcurveto{\pgfqpoint{-0.031064in}{0.018041in}}{\pgfqpoint{-0.034722in}{0.009208in}}{\pgfqpoint{-0.034722in}{0.000000in}}%
\pgfpathcurveto{\pgfqpoint{-0.034722in}{-0.009208in}}{\pgfqpoint{-0.031064in}{-0.018041in}}{\pgfqpoint{-0.024552in}{-0.024552in}}%
\pgfpathcurveto{\pgfqpoint{-0.018041in}{-0.031064in}}{\pgfqpoint{-0.009208in}{-0.034722in}}{\pgfqpoint{0.000000in}{-0.034722in}}%
\pgfpathclose%
\pgfusepath{stroke,fill}%
}%
\begin{pgfscope}%
\pgfsys@transformshift{4.378680in}{1.333908in}%
\pgfsys@useobject{currentmarker}{}%
\end{pgfscope}%
\end{pgfscope}%
\begin{pgfscope}%
\definecolor{textcolor}{rgb}{0.000000,0.000000,0.000000}%
\pgfsetstrokecolor{textcolor}%
\pgfsetfillcolor{textcolor}%
\pgftext[x=4.628680in,y=1.285297in,left,base]{\color{textcolor}\sffamily\fontsize{10.000000}{12.000000}\selectfont No Timeout}%
\end{pgfscope}%
\begin{pgfscope}%
\pgfsetbuttcap%
\pgfsetroundjoin%
\definecolor{currentfill}{rgb}{1.000000,0.498039,0.054902}%
\pgfsetfillcolor{currentfill}%
\pgfsetlinewidth{1.003750pt}%
\definecolor{currentstroke}{rgb}{1.000000,0.498039,0.054902}%
\pgfsetstrokecolor{currentstroke}%
\pgfsetdash{}{0pt}%
\pgfsys@defobject{currentmarker}{\pgfqpoint{-0.034722in}{-0.034722in}}{\pgfqpoint{0.034722in}{0.034722in}}{%
\pgfpathmoveto{\pgfqpoint{0.000000in}{-0.034722in}}%
\pgfpathcurveto{\pgfqpoint{0.009208in}{-0.034722in}}{\pgfqpoint{0.018041in}{-0.031064in}}{\pgfqpoint{0.024552in}{-0.024552in}}%
\pgfpathcurveto{\pgfqpoint{0.031064in}{-0.018041in}}{\pgfqpoint{0.034722in}{-0.009208in}}{\pgfqpoint{0.034722in}{0.000000in}}%
\pgfpathcurveto{\pgfqpoint{0.034722in}{0.009208in}}{\pgfqpoint{0.031064in}{0.018041in}}{\pgfqpoint{0.024552in}{0.024552in}}%
\pgfpathcurveto{\pgfqpoint{0.018041in}{0.031064in}}{\pgfqpoint{0.009208in}{0.034722in}}{\pgfqpoint{0.000000in}{0.034722in}}%
\pgfpathcurveto{\pgfqpoint{-0.009208in}{0.034722in}}{\pgfqpoint{-0.018041in}{0.031064in}}{\pgfqpoint{-0.024552in}{0.024552in}}%
\pgfpathcurveto{\pgfqpoint{-0.031064in}{0.018041in}}{\pgfqpoint{-0.034722in}{0.009208in}}{\pgfqpoint{-0.034722in}{0.000000in}}%
\pgfpathcurveto{\pgfqpoint{-0.034722in}{-0.009208in}}{\pgfqpoint{-0.031064in}{-0.018041in}}{\pgfqpoint{-0.024552in}{-0.024552in}}%
\pgfpathcurveto{\pgfqpoint{-0.018041in}{-0.031064in}}{\pgfqpoint{-0.009208in}{-0.034722in}}{\pgfqpoint{0.000000in}{-0.034722in}}%
\pgfpathclose%
\pgfusepath{stroke,fill}%
}%
\begin{pgfscope}%
\pgfsys@transformshift{4.378680in}{1.140297in}%
\pgfsys@useobject{currentmarker}{}%
\end{pgfscope}%
\end{pgfscope}%
\begin{pgfscope}%
\definecolor{textcolor}{rgb}{0.000000,0.000000,0.000000}%
\pgfsetstrokecolor{textcolor}%
\pgfsetfillcolor{textcolor}%
\pgftext[x=4.628680in,y=1.091685in,left,base]{\color{textcolor}\sffamily\fontsize{10.000000}{12.000000}\selectfont Time Timeout}%
\end{pgfscope}%
\begin{pgfscope}%
\pgfsetbuttcap%
\pgfsetroundjoin%
\definecolor{currentfill}{rgb}{0.839216,0.152941,0.156863}%
\pgfsetfillcolor{currentfill}%
\pgfsetlinewidth{1.003750pt}%
\definecolor{currentstroke}{rgb}{0.839216,0.152941,0.156863}%
\pgfsetstrokecolor{currentstroke}%
\pgfsetdash{}{0pt}%
\pgfsys@defobject{currentmarker}{\pgfqpoint{-0.034722in}{-0.034722in}}{\pgfqpoint{0.034722in}{0.034722in}}{%
\pgfpathmoveto{\pgfqpoint{0.000000in}{-0.034722in}}%
\pgfpathcurveto{\pgfqpoint{0.009208in}{-0.034722in}}{\pgfqpoint{0.018041in}{-0.031064in}}{\pgfqpoint{0.024552in}{-0.024552in}}%
\pgfpathcurveto{\pgfqpoint{0.031064in}{-0.018041in}}{\pgfqpoint{0.034722in}{-0.009208in}}{\pgfqpoint{0.034722in}{0.000000in}}%
\pgfpathcurveto{\pgfqpoint{0.034722in}{0.009208in}}{\pgfqpoint{0.031064in}{0.018041in}}{\pgfqpoint{0.024552in}{0.024552in}}%
\pgfpathcurveto{\pgfqpoint{0.018041in}{0.031064in}}{\pgfqpoint{0.009208in}{0.034722in}}{\pgfqpoint{0.000000in}{0.034722in}}%
\pgfpathcurveto{\pgfqpoint{-0.009208in}{0.034722in}}{\pgfqpoint{-0.018041in}{0.031064in}}{\pgfqpoint{-0.024552in}{0.024552in}}%
\pgfpathcurveto{\pgfqpoint{-0.031064in}{0.018041in}}{\pgfqpoint{-0.034722in}{0.009208in}}{\pgfqpoint{-0.034722in}{0.000000in}}%
\pgfpathcurveto{\pgfqpoint{-0.034722in}{-0.009208in}}{\pgfqpoint{-0.031064in}{-0.018041in}}{\pgfqpoint{-0.024552in}{-0.024552in}}%
\pgfpathcurveto{\pgfqpoint{-0.018041in}{-0.031064in}}{\pgfqpoint{-0.009208in}{-0.034722in}}{\pgfqpoint{0.000000in}{-0.034722in}}%
\pgfpathclose%
\pgfusepath{stroke,fill}%
}%
\begin{pgfscope}%
\pgfsys@transformshift{4.378680in}{0.946685in}%
\pgfsys@useobject{currentmarker}{}%
\end{pgfscope}%
\end{pgfscope}%
\begin{pgfscope}%
\definecolor{textcolor}{rgb}{0.000000,0.000000,0.000000}%
\pgfsetstrokecolor{textcolor}%
\pgfsetfillcolor{textcolor}%
\pgftext[x=4.628680in,y=0.898074in,left,base]{\color{textcolor}\sffamily\fontsize{10.000000}{12.000000}\selectfont Memory Timeout}%
\end{pgfscope}%
\begin{pgfscope}%
\pgfsetrectcap%
\pgfsetroundjoin%
\pgfsetlinewidth{1.505625pt}%
\definecolor{currentstroke}{rgb}{0.000000,0.500000,0.000000}%
\pgfsetstrokecolor{currentstroke}%
\pgfsetdash{}{0pt}%
\pgfpathmoveto{\pgfqpoint{4.239791in}{0.750852in}}%
\pgfpathlineto{\pgfqpoint{4.517569in}{0.750852in}}%
\pgfusepath{stroke}%
\end{pgfscope}%
\begin{pgfscope}%
\definecolor{textcolor}{rgb}{0.000000,0.000000,0.000000}%
\pgfsetstrokecolor{textcolor}%
\pgfsetfillcolor{textcolor}%
\pgftext[x=4.628680in,y=0.702241in,left,base]{\color{textcolor}\sffamily\fontsize{10.000000}{12.000000}\selectfont Polyfit}%
\end{pgfscope}%
\end{pgfpicture}%
\makeatother%
\endgroup%

                }
            \end{subfigure}
            \caption{Alias Edges}
            \label{f:dfedgesa}
        \end{subfigure}
        \caption{Data Flow Time in Comparison to Edge Count}
        \label{f:dfedges}
    \end{figure}

    To conclude, our backward analysis is efficient enough to be an alternative to the existing implementation.
    We even found that it performed slightly better on our app set.
    Our evaluation shows that there is no correlation between an apriori known parameter and the runtime of \textsc{FlowDroid} - in both directions.
    Furthermore, we did not find any apriori known parameter to decide the favorable direction either.
    The edge propagations have shown that our implementation can analyze roughly $10^7$ more edges than the existing implementation in ten minutes. Though, the sample size of 200 apps is too small to generalize statements and our data was rather challenging to interpret with a large standard deviation.

    \FloatBarrier
    \subsection{Memory Evaluation}\label{s:memex}

    \begin{table}[tbp]
        \centering
        \begin{tabular}{l | r | r | r}
            & \multicolumn{3}{c}{\textbf{Forward}}\\
            \textbf{Metric} & \textbf{Avg} & \textbf{Median} & $\mathbf{P_{85}}$\\
            \hline\hline
            Maximum Memory Consumption & $10005.68MB$ & $10459.48MB$ & $15482.98MB$\\
            \hline
            Maximum Memory Consumption & \multirow{2}{*}{$1535.20MB$} & \multirow{2}{*}{$6.98MB$} & \multirow{2}{*}{$4952.20MB$}\\
            Without Timeouts & & &\\            
            \multicolumn{4}{c}{}\\
            & \multicolumn{3}{c}{\textbf{Backward}}\\
            \textbf{Metric} & \textbf{Avg} & \textbf{Median} & $\mathbf{P_{85}}$\\
            \hline\hline
            Maximum Memory Consumption & $8326.27MB$ & $10008.52MB$ & $14539.64MB$\\
            \hline
            Maximum Memory Consumption & \multirow{2}{*}{$1473.21MB$} & \multirow{2}{*}{$7.33MB$} & \multirow{2}{*}{$1566.61MB$}\\
            Without Timeouts & & &\\
        \end{tabular}
        \caption{Memory Results}
        \label{t:memres}
    \end{table}

    \autoref{t:memres} shows an overview of the results from the memory evaluation.
    Note that we only measured the memory usage of the edges in the exploded supergraph and not of the full program.
    Also, unlike the time measurements, the memory consumption is much more distributed distributed across the range.
    The measurements with timeouts show similar values for both directions with a bias toward our implementation.
    Though, maximum measurements with timeouts are not really meaningful because of the cut-off at ten minutes.
    Without timeouts the gap gets a bit bigger.
    The average maximum memory consumption of our implementation is around $65MB$ lower than the existing one. 
    Especially in the 85\textsuperscript{th} percentile our implementation shines where it needs $3.3GB$ less memory.

    Next, we look at the memory consumption difference per app in \autoref{f:memHist}.
    The x-axis shows the delta maximum memory consumption in megabytes and the y-axis the frequency.
    Each bin is $1GB$ wide.
    The delta is calculated with forward as the refrence: $m_{\mathit{Backward}}-m_{\mathit{Forward}}$.
    Again, we see a gathering around $0$. 
    Otherwise, the histogram has a more uniform distribution than its \hyperref[f:deltaHist]{time counterpart}.
    Just as in the overview, there is a slight bias towards the backward analysis.
    We argue this bias is related to the faster backward analysis in the app set. 
    A faster analysis means less edges\footnotemark{} and the edges should correlate linearly with the memory usage of the exploded supergraph.
    \footnotetext{This was at least true for apps without timeout.} 
    We looked at this by comparing the sign of the delta data flow time with the sign of the delta memory consumption. 
    48 apps had different signs, with 23 being negligibly close to 0. 
    Hence, the claim is true for 109 of 134 apps.
    We also calculated the Pearson correlation coefficient for the data flow time and the maximum memory consumption.
    For the forward analysis the coefficient is $0.75$ and $0.87$ for the backward implementation.
    Both values indicate a correlation.

    \begin{figure}[tbp]
        \centering
        \resizebox{0.75\columnwidth}{!}{
            %% Creator: Matplotlib, PGF backend
%%
%% To include the figure in your LaTeX document, write
%%   \input{<filename>.pgf}
%%
%% Make sure the required packages are loaded in your preamble
%%   \usepackage{pgf}
%%
%% and, on pdftex
%%   \usepackage[utf8]{inputenc}\DeclareUnicodeCharacter{2212}{-}
%%
%% or, on luatex and xetex
%%   \usepackage{unicode-math}
%%
%% Figures using additional raster images can only be included by \input if
%% they are in the same directory as the main LaTeX file. For loading figures
%% from other directories you can use the `import` package
%%   \usepackage{import}
%%
%% and then include the figures with
%%   \import{<path to file>}{<filename>.pgf}
%%
%% Matplotlib used the following preamble
%%   \usepackage{amsmath}
%%   \usepackage{fontspec}
%%
\begingroup%
\makeatletter%
\begin{pgfpicture}%
\pgfpathrectangle{\pgfpointorigin}{\pgfqpoint{6.000000in}{4.000000in}}%
\pgfusepath{use as bounding box, clip}%
\begin{pgfscope}%
\pgfsetbuttcap%
\pgfsetmiterjoin%
\definecolor{currentfill}{rgb}{1.000000,1.000000,1.000000}%
\pgfsetfillcolor{currentfill}%
\pgfsetlinewidth{0.000000pt}%
\definecolor{currentstroke}{rgb}{1.000000,1.000000,1.000000}%
\pgfsetstrokecolor{currentstroke}%
\pgfsetdash{}{0pt}%
\pgfpathmoveto{\pgfqpoint{0.000000in}{0.000000in}}%
\pgfpathlineto{\pgfqpoint{6.000000in}{0.000000in}}%
\pgfpathlineto{\pgfqpoint{6.000000in}{4.000000in}}%
\pgfpathlineto{\pgfqpoint{0.000000in}{4.000000in}}%
\pgfpathclose%
\pgfusepath{fill}%
\end{pgfscope}%
\begin{pgfscope}%
\pgfsetbuttcap%
\pgfsetmiterjoin%
\definecolor{currentfill}{rgb}{1.000000,1.000000,1.000000}%
\pgfsetfillcolor{currentfill}%
\pgfsetlinewidth{0.000000pt}%
\definecolor{currentstroke}{rgb}{0.000000,0.000000,0.000000}%
\pgfsetstrokecolor{currentstroke}%
\pgfsetstrokeopacity{0.000000}%
\pgfsetdash{}{0pt}%
\pgfpathmoveto{\pgfqpoint{0.750000in}{0.500000in}}%
\pgfpathlineto{\pgfqpoint{5.400000in}{0.500000in}}%
\pgfpathlineto{\pgfqpoint{5.400000in}{3.520000in}}%
\pgfpathlineto{\pgfqpoint{0.750000in}{3.520000in}}%
\pgfpathclose%
\pgfusepath{fill}%
\end{pgfscope}%
\begin{pgfscope}%
\pgfpathrectangle{\pgfqpoint{0.750000in}{0.500000in}}{\pgfqpoint{4.650000in}{3.020000in}}%
\pgfusepath{clip}%
\pgfsetbuttcap%
\pgfsetmiterjoin%
\definecolor{currentfill}{rgb}{0.121569,0.466667,0.705882}%
\pgfsetfillcolor{currentfill}%
\pgfsetlinewidth{1.003750pt}%
\definecolor{currentstroke}{rgb}{0.000000,0.000000,0.000000}%
\pgfsetstrokecolor{currentstroke}%
\pgfsetdash{}{0pt}%
\pgfpathmoveto{\pgfqpoint{0.961364in}{0.500000in}}%
\pgfpathlineto{\pgfqpoint{1.069755in}{0.500000in}}%
\pgfpathlineto{\pgfqpoint{1.069755in}{0.659788in}}%
\pgfpathlineto{\pgfqpoint{0.961364in}{0.659788in}}%
\pgfpathclose%
\pgfusepath{stroke,fill}%
\end{pgfscope}%
\begin{pgfscope}%
\pgfpathrectangle{\pgfqpoint{0.750000in}{0.500000in}}{\pgfqpoint{4.650000in}{3.020000in}}%
\pgfusepath{clip}%
\pgfsetbuttcap%
\pgfsetmiterjoin%
\definecolor{currentfill}{rgb}{0.121569,0.466667,0.705882}%
\pgfsetfillcolor{currentfill}%
\pgfsetlinewidth{1.003750pt}%
\definecolor{currentstroke}{rgb}{0.000000,0.000000,0.000000}%
\pgfsetstrokecolor{currentstroke}%
\pgfsetdash{}{0pt}%
\pgfpathmoveto{\pgfqpoint{1.069755in}{0.500000in}}%
\pgfpathlineto{\pgfqpoint{1.178147in}{0.500000in}}%
\pgfpathlineto{\pgfqpoint{1.178147in}{0.500000in}}%
\pgfpathlineto{\pgfqpoint{1.069755in}{0.500000in}}%
\pgfpathclose%
\pgfusepath{stroke,fill}%
\end{pgfscope}%
\begin{pgfscope}%
\pgfpathrectangle{\pgfqpoint{0.750000in}{0.500000in}}{\pgfqpoint{4.650000in}{3.020000in}}%
\pgfusepath{clip}%
\pgfsetbuttcap%
\pgfsetmiterjoin%
\definecolor{currentfill}{rgb}{0.121569,0.466667,0.705882}%
\pgfsetfillcolor{currentfill}%
\pgfsetlinewidth{1.003750pt}%
\definecolor{currentstroke}{rgb}{0.000000,0.000000,0.000000}%
\pgfsetstrokecolor{currentstroke}%
\pgfsetdash{}{0pt}%
\pgfpathmoveto{\pgfqpoint{1.178147in}{0.500000in}}%
\pgfpathlineto{\pgfqpoint{1.286538in}{0.500000in}}%
\pgfpathlineto{\pgfqpoint{1.286538in}{0.659788in}}%
\pgfpathlineto{\pgfqpoint{1.178147in}{0.659788in}}%
\pgfpathclose%
\pgfusepath{stroke,fill}%
\end{pgfscope}%
\begin{pgfscope}%
\pgfpathrectangle{\pgfqpoint{0.750000in}{0.500000in}}{\pgfqpoint{4.650000in}{3.020000in}}%
\pgfusepath{clip}%
\pgfsetbuttcap%
\pgfsetmiterjoin%
\definecolor{currentfill}{rgb}{0.121569,0.466667,0.705882}%
\pgfsetfillcolor{currentfill}%
\pgfsetlinewidth{1.003750pt}%
\definecolor{currentstroke}{rgb}{0.000000,0.000000,0.000000}%
\pgfsetstrokecolor{currentstroke}%
\pgfsetdash{}{0pt}%
\pgfpathmoveto{\pgfqpoint{1.286538in}{0.500000in}}%
\pgfpathlineto{\pgfqpoint{1.394930in}{0.500000in}}%
\pgfpathlineto{\pgfqpoint{1.394930in}{0.819577in}}%
\pgfpathlineto{\pgfqpoint{1.286538in}{0.819577in}}%
\pgfpathclose%
\pgfusepath{stroke,fill}%
\end{pgfscope}%
\begin{pgfscope}%
\pgfpathrectangle{\pgfqpoint{0.750000in}{0.500000in}}{\pgfqpoint{4.650000in}{3.020000in}}%
\pgfusepath{clip}%
\pgfsetbuttcap%
\pgfsetmiterjoin%
\definecolor{currentfill}{rgb}{0.121569,0.466667,0.705882}%
\pgfsetfillcolor{currentfill}%
\pgfsetlinewidth{1.003750pt}%
\definecolor{currentstroke}{rgb}{0.000000,0.000000,0.000000}%
\pgfsetstrokecolor{currentstroke}%
\pgfsetdash{}{0pt}%
\pgfpathmoveto{\pgfqpoint{1.394930in}{0.500000in}}%
\pgfpathlineto{\pgfqpoint{1.503322in}{0.500000in}}%
\pgfpathlineto{\pgfqpoint{1.503322in}{0.819577in}}%
\pgfpathlineto{\pgfqpoint{1.394930in}{0.819577in}}%
\pgfpathclose%
\pgfusepath{stroke,fill}%
\end{pgfscope}%
\begin{pgfscope}%
\pgfpathrectangle{\pgfqpoint{0.750000in}{0.500000in}}{\pgfqpoint{4.650000in}{3.020000in}}%
\pgfusepath{clip}%
\pgfsetbuttcap%
\pgfsetmiterjoin%
\definecolor{currentfill}{rgb}{0.121569,0.466667,0.705882}%
\pgfsetfillcolor{currentfill}%
\pgfsetlinewidth{1.003750pt}%
\definecolor{currentstroke}{rgb}{0.000000,0.000000,0.000000}%
\pgfsetstrokecolor{currentstroke}%
\pgfsetdash{}{0pt}%
\pgfpathmoveto{\pgfqpoint{1.503322in}{0.500000in}}%
\pgfpathlineto{\pgfqpoint{1.611713in}{0.500000in}}%
\pgfpathlineto{\pgfqpoint{1.611713in}{0.979365in}}%
\pgfpathlineto{\pgfqpoint{1.503322in}{0.979365in}}%
\pgfpathclose%
\pgfusepath{stroke,fill}%
\end{pgfscope}%
\begin{pgfscope}%
\pgfpathrectangle{\pgfqpoint{0.750000in}{0.500000in}}{\pgfqpoint{4.650000in}{3.020000in}}%
\pgfusepath{clip}%
\pgfsetbuttcap%
\pgfsetmiterjoin%
\definecolor{currentfill}{rgb}{0.121569,0.466667,0.705882}%
\pgfsetfillcolor{currentfill}%
\pgfsetlinewidth{1.003750pt}%
\definecolor{currentstroke}{rgb}{0.000000,0.000000,0.000000}%
\pgfsetstrokecolor{currentstroke}%
\pgfsetdash{}{0pt}%
\pgfpathmoveto{\pgfqpoint{1.611713in}{0.500000in}}%
\pgfpathlineto{\pgfqpoint{1.720105in}{0.500000in}}%
\pgfpathlineto{\pgfqpoint{1.720105in}{1.139153in}}%
\pgfpathlineto{\pgfqpoint{1.611713in}{1.139153in}}%
\pgfpathclose%
\pgfusepath{stroke,fill}%
\end{pgfscope}%
\begin{pgfscope}%
\pgfpathrectangle{\pgfqpoint{0.750000in}{0.500000in}}{\pgfqpoint{4.650000in}{3.020000in}}%
\pgfusepath{clip}%
\pgfsetbuttcap%
\pgfsetmiterjoin%
\definecolor{currentfill}{rgb}{0.121569,0.466667,0.705882}%
\pgfsetfillcolor{currentfill}%
\pgfsetlinewidth{1.003750pt}%
\definecolor{currentstroke}{rgb}{0.000000,0.000000,0.000000}%
\pgfsetstrokecolor{currentstroke}%
\pgfsetdash{}{0pt}%
\pgfpathmoveto{\pgfqpoint{1.720105in}{0.500000in}}%
\pgfpathlineto{\pgfqpoint{1.828497in}{0.500000in}}%
\pgfpathlineto{\pgfqpoint{1.828497in}{0.819577in}}%
\pgfpathlineto{\pgfqpoint{1.720105in}{0.819577in}}%
\pgfpathclose%
\pgfusepath{stroke,fill}%
\end{pgfscope}%
\begin{pgfscope}%
\pgfpathrectangle{\pgfqpoint{0.750000in}{0.500000in}}{\pgfqpoint{4.650000in}{3.020000in}}%
\pgfusepath{clip}%
\pgfsetbuttcap%
\pgfsetmiterjoin%
\definecolor{currentfill}{rgb}{0.121569,0.466667,0.705882}%
\pgfsetfillcolor{currentfill}%
\pgfsetlinewidth{1.003750pt}%
\definecolor{currentstroke}{rgb}{0.000000,0.000000,0.000000}%
\pgfsetstrokecolor{currentstroke}%
\pgfsetdash{}{0pt}%
\pgfpathmoveto{\pgfqpoint{1.828497in}{0.500000in}}%
\pgfpathlineto{\pgfqpoint{1.936888in}{0.500000in}}%
\pgfpathlineto{\pgfqpoint{1.936888in}{0.819577in}}%
\pgfpathlineto{\pgfqpoint{1.828497in}{0.819577in}}%
\pgfpathclose%
\pgfusepath{stroke,fill}%
\end{pgfscope}%
\begin{pgfscope}%
\pgfpathrectangle{\pgfqpoint{0.750000in}{0.500000in}}{\pgfqpoint{4.650000in}{3.020000in}}%
\pgfusepath{clip}%
\pgfsetbuttcap%
\pgfsetmiterjoin%
\definecolor{currentfill}{rgb}{0.121569,0.466667,0.705882}%
\pgfsetfillcolor{currentfill}%
\pgfsetlinewidth{1.003750pt}%
\definecolor{currentstroke}{rgb}{0.000000,0.000000,0.000000}%
\pgfsetstrokecolor{currentstroke}%
\pgfsetdash{}{0pt}%
\pgfpathmoveto{\pgfqpoint{1.936888in}{0.500000in}}%
\pgfpathlineto{\pgfqpoint{2.045280in}{0.500000in}}%
\pgfpathlineto{\pgfqpoint{2.045280in}{0.979365in}}%
\pgfpathlineto{\pgfqpoint{1.936888in}{0.979365in}}%
\pgfpathclose%
\pgfusepath{stroke,fill}%
\end{pgfscope}%
\begin{pgfscope}%
\pgfpathrectangle{\pgfqpoint{0.750000in}{0.500000in}}{\pgfqpoint{4.650000in}{3.020000in}}%
\pgfusepath{clip}%
\pgfsetbuttcap%
\pgfsetmiterjoin%
\definecolor{currentfill}{rgb}{0.121569,0.466667,0.705882}%
\pgfsetfillcolor{currentfill}%
\pgfsetlinewidth{1.003750pt}%
\definecolor{currentstroke}{rgb}{0.000000,0.000000,0.000000}%
\pgfsetstrokecolor{currentstroke}%
\pgfsetdash{}{0pt}%
\pgfpathmoveto{\pgfqpoint{2.045280in}{0.500000in}}%
\pgfpathlineto{\pgfqpoint{2.153671in}{0.500000in}}%
\pgfpathlineto{\pgfqpoint{2.153671in}{0.819577in}}%
\pgfpathlineto{\pgfqpoint{2.045280in}{0.819577in}}%
\pgfpathclose%
\pgfusepath{stroke,fill}%
\end{pgfscope}%
\begin{pgfscope}%
\pgfpathrectangle{\pgfqpoint{0.750000in}{0.500000in}}{\pgfqpoint{4.650000in}{3.020000in}}%
\pgfusepath{clip}%
\pgfsetbuttcap%
\pgfsetmiterjoin%
\definecolor{currentfill}{rgb}{0.121569,0.466667,0.705882}%
\pgfsetfillcolor{currentfill}%
\pgfsetlinewidth{1.003750pt}%
\definecolor{currentstroke}{rgb}{0.000000,0.000000,0.000000}%
\pgfsetstrokecolor{currentstroke}%
\pgfsetdash{}{0pt}%
\pgfpathmoveto{\pgfqpoint{2.153671in}{0.500000in}}%
\pgfpathlineto{\pgfqpoint{2.262063in}{0.500000in}}%
\pgfpathlineto{\pgfqpoint{2.262063in}{1.139153in}}%
\pgfpathlineto{\pgfqpoint{2.153671in}{1.139153in}}%
\pgfpathclose%
\pgfusepath{stroke,fill}%
\end{pgfscope}%
\begin{pgfscope}%
\pgfpathrectangle{\pgfqpoint{0.750000in}{0.500000in}}{\pgfqpoint{4.650000in}{3.020000in}}%
\pgfusepath{clip}%
\pgfsetbuttcap%
\pgfsetmiterjoin%
\definecolor{currentfill}{rgb}{0.121569,0.466667,0.705882}%
\pgfsetfillcolor{currentfill}%
\pgfsetlinewidth{1.003750pt}%
\definecolor{currentstroke}{rgb}{0.000000,0.000000,0.000000}%
\pgfsetstrokecolor{currentstroke}%
\pgfsetdash{}{0pt}%
\pgfpathmoveto{\pgfqpoint{2.262063in}{0.500000in}}%
\pgfpathlineto{\pgfqpoint{2.370455in}{0.500000in}}%
\pgfpathlineto{\pgfqpoint{2.370455in}{1.298942in}}%
\pgfpathlineto{\pgfqpoint{2.262063in}{1.298942in}}%
\pgfpathclose%
\pgfusepath{stroke,fill}%
\end{pgfscope}%
\begin{pgfscope}%
\pgfpathrectangle{\pgfqpoint{0.750000in}{0.500000in}}{\pgfqpoint{4.650000in}{3.020000in}}%
\pgfusepath{clip}%
\pgfsetbuttcap%
\pgfsetmiterjoin%
\definecolor{currentfill}{rgb}{0.121569,0.466667,0.705882}%
\pgfsetfillcolor{currentfill}%
\pgfsetlinewidth{1.003750pt}%
\definecolor{currentstroke}{rgb}{0.000000,0.000000,0.000000}%
\pgfsetstrokecolor{currentstroke}%
\pgfsetdash{}{0pt}%
\pgfpathmoveto{\pgfqpoint{2.370455in}{0.500000in}}%
\pgfpathlineto{\pgfqpoint{2.478846in}{0.500000in}}%
\pgfpathlineto{\pgfqpoint{2.478846in}{1.298942in}}%
\pgfpathlineto{\pgfqpoint{2.370455in}{1.298942in}}%
\pgfpathclose%
\pgfusepath{stroke,fill}%
\end{pgfscope}%
\begin{pgfscope}%
\pgfpathrectangle{\pgfqpoint{0.750000in}{0.500000in}}{\pgfqpoint{4.650000in}{3.020000in}}%
\pgfusepath{clip}%
\pgfsetbuttcap%
\pgfsetmiterjoin%
\definecolor{currentfill}{rgb}{0.121569,0.466667,0.705882}%
\pgfsetfillcolor{currentfill}%
\pgfsetlinewidth{1.003750pt}%
\definecolor{currentstroke}{rgb}{0.000000,0.000000,0.000000}%
\pgfsetstrokecolor{currentstroke}%
\pgfsetdash{}{0pt}%
\pgfpathmoveto{\pgfqpoint{2.478846in}{0.500000in}}%
\pgfpathlineto{\pgfqpoint{2.587238in}{0.500000in}}%
\pgfpathlineto{\pgfqpoint{2.587238in}{0.979365in}}%
\pgfpathlineto{\pgfqpoint{2.478846in}{0.979365in}}%
\pgfpathclose%
\pgfusepath{stroke,fill}%
\end{pgfscope}%
\begin{pgfscope}%
\pgfpathrectangle{\pgfqpoint{0.750000in}{0.500000in}}{\pgfqpoint{4.650000in}{3.020000in}}%
\pgfusepath{clip}%
\pgfsetbuttcap%
\pgfsetmiterjoin%
\definecolor{currentfill}{rgb}{0.121569,0.466667,0.705882}%
\pgfsetfillcolor{currentfill}%
\pgfsetlinewidth{1.003750pt}%
\definecolor{currentstroke}{rgb}{0.000000,0.000000,0.000000}%
\pgfsetstrokecolor{currentstroke}%
\pgfsetdash{}{0pt}%
\pgfpathmoveto{\pgfqpoint{2.587238in}{0.500000in}}%
\pgfpathlineto{\pgfqpoint{2.695629in}{0.500000in}}%
\pgfpathlineto{\pgfqpoint{2.695629in}{1.618519in}}%
\pgfpathlineto{\pgfqpoint{2.587238in}{1.618519in}}%
\pgfpathclose%
\pgfusepath{stroke,fill}%
\end{pgfscope}%
\begin{pgfscope}%
\pgfpathrectangle{\pgfqpoint{0.750000in}{0.500000in}}{\pgfqpoint{4.650000in}{3.020000in}}%
\pgfusepath{clip}%
\pgfsetbuttcap%
\pgfsetmiterjoin%
\definecolor{currentfill}{rgb}{0.121569,0.466667,0.705882}%
\pgfsetfillcolor{currentfill}%
\pgfsetlinewidth{1.003750pt}%
\definecolor{currentstroke}{rgb}{0.000000,0.000000,0.000000}%
\pgfsetstrokecolor{currentstroke}%
\pgfsetdash{}{0pt}%
\pgfpathmoveto{\pgfqpoint{2.695629in}{0.500000in}}%
\pgfpathlineto{\pgfqpoint{2.804021in}{0.500000in}}%
\pgfpathlineto{\pgfqpoint{2.804021in}{1.458730in}}%
\pgfpathlineto{\pgfqpoint{2.695629in}{1.458730in}}%
\pgfpathclose%
\pgfusepath{stroke,fill}%
\end{pgfscope}%
\begin{pgfscope}%
\pgfpathrectangle{\pgfqpoint{0.750000in}{0.500000in}}{\pgfqpoint{4.650000in}{3.020000in}}%
\pgfusepath{clip}%
\pgfsetbuttcap%
\pgfsetmiterjoin%
\definecolor{currentfill}{rgb}{0.121569,0.466667,0.705882}%
\pgfsetfillcolor{currentfill}%
\pgfsetlinewidth{1.003750pt}%
\definecolor{currentstroke}{rgb}{0.000000,0.000000,0.000000}%
\pgfsetstrokecolor{currentstroke}%
\pgfsetdash{}{0pt}%
\pgfpathmoveto{\pgfqpoint{2.804021in}{0.500000in}}%
\pgfpathlineto{\pgfqpoint{2.912413in}{0.500000in}}%
\pgfpathlineto{\pgfqpoint{2.912413in}{1.618519in}}%
\pgfpathlineto{\pgfqpoint{2.804021in}{1.618519in}}%
\pgfpathclose%
\pgfusepath{stroke,fill}%
\end{pgfscope}%
\begin{pgfscope}%
\pgfpathrectangle{\pgfqpoint{0.750000in}{0.500000in}}{\pgfqpoint{4.650000in}{3.020000in}}%
\pgfusepath{clip}%
\pgfsetbuttcap%
\pgfsetmiterjoin%
\definecolor{currentfill}{rgb}{0.121569,0.466667,0.705882}%
\pgfsetfillcolor{currentfill}%
\pgfsetlinewidth{1.003750pt}%
\definecolor{currentstroke}{rgb}{0.000000,0.000000,0.000000}%
\pgfsetstrokecolor{currentstroke}%
\pgfsetdash{}{0pt}%
\pgfpathmoveto{\pgfqpoint{2.912413in}{0.500000in}}%
\pgfpathlineto{\pgfqpoint{3.020804in}{0.500000in}}%
\pgfpathlineto{\pgfqpoint{3.020804in}{1.778307in}}%
\pgfpathlineto{\pgfqpoint{2.912413in}{1.778307in}}%
\pgfpathclose%
\pgfusepath{stroke,fill}%
\end{pgfscope}%
\begin{pgfscope}%
\pgfpathrectangle{\pgfqpoint{0.750000in}{0.500000in}}{\pgfqpoint{4.650000in}{3.020000in}}%
\pgfusepath{clip}%
\pgfsetbuttcap%
\pgfsetmiterjoin%
\definecolor{currentfill}{rgb}{0.121569,0.466667,0.705882}%
\pgfsetfillcolor{currentfill}%
\pgfsetlinewidth{1.003750pt}%
\definecolor{currentstroke}{rgb}{0.000000,0.000000,0.000000}%
\pgfsetstrokecolor{currentstroke}%
\pgfsetdash{}{0pt}%
\pgfpathmoveto{\pgfqpoint{3.020804in}{0.500000in}}%
\pgfpathlineto{\pgfqpoint{3.129196in}{0.500000in}}%
\pgfpathlineto{\pgfqpoint{3.129196in}{1.298942in}}%
\pgfpathlineto{\pgfqpoint{3.020804in}{1.298942in}}%
\pgfpathclose%
\pgfusepath{stroke,fill}%
\end{pgfscope}%
\begin{pgfscope}%
\pgfpathrectangle{\pgfqpoint{0.750000in}{0.500000in}}{\pgfqpoint{4.650000in}{3.020000in}}%
\pgfusepath{clip}%
\pgfsetbuttcap%
\pgfsetmiterjoin%
\definecolor{currentfill}{rgb}{0.121569,0.466667,0.705882}%
\pgfsetfillcolor{currentfill}%
\pgfsetlinewidth{1.003750pt}%
\definecolor{currentstroke}{rgb}{0.000000,0.000000,0.000000}%
\pgfsetstrokecolor{currentstroke}%
\pgfsetdash{}{0pt}%
\pgfpathmoveto{\pgfqpoint{3.129196in}{0.500000in}}%
\pgfpathlineto{\pgfqpoint{3.237587in}{0.500000in}}%
\pgfpathlineto{\pgfqpoint{3.237587in}{3.376190in}}%
\pgfpathlineto{\pgfqpoint{3.129196in}{3.376190in}}%
\pgfpathclose%
\pgfusepath{stroke,fill}%
\end{pgfscope}%
\begin{pgfscope}%
\pgfpathrectangle{\pgfqpoint{0.750000in}{0.500000in}}{\pgfqpoint{4.650000in}{3.020000in}}%
\pgfusepath{clip}%
\pgfsetbuttcap%
\pgfsetmiterjoin%
\definecolor{currentfill}{rgb}{0.121569,0.466667,0.705882}%
\pgfsetfillcolor{currentfill}%
\pgfsetlinewidth{1.003750pt}%
\definecolor{currentstroke}{rgb}{0.000000,0.000000,0.000000}%
\pgfsetstrokecolor{currentstroke}%
\pgfsetdash{}{0pt}%
\pgfpathmoveto{\pgfqpoint{3.237587in}{0.500000in}}%
\pgfpathlineto{\pgfqpoint{3.345979in}{0.500000in}}%
\pgfpathlineto{\pgfqpoint{3.345979in}{1.778307in}}%
\pgfpathlineto{\pgfqpoint{3.237587in}{1.778307in}}%
\pgfpathclose%
\pgfusepath{stroke,fill}%
\end{pgfscope}%
\begin{pgfscope}%
\pgfpathrectangle{\pgfqpoint{0.750000in}{0.500000in}}{\pgfqpoint{4.650000in}{3.020000in}}%
\pgfusepath{clip}%
\pgfsetbuttcap%
\pgfsetmiterjoin%
\definecolor{currentfill}{rgb}{0.121569,0.466667,0.705882}%
\pgfsetfillcolor{currentfill}%
\pgfsetlinewidth{1.003750pt}%
\definecolor{currentstroke}{rgb}{0.000000,0.000000,0.000000}%
\pgfsetstrokecolor{currentstroke}%
\pgfsetdash{}{0pt}%
\pgfpathmoveto{\pgfqpoint{3.345979in}{0.500000in}}%
\pgfpathlineto{\pgfqpoint{3.454371in}{0.500000in}}%
\pgfpathlineto{\pgfqpoint{3.454371in}{1.618519in}}%
\pgfpathlineto{\pgfqpoint{3.345979in}{1.618519in}}%
\pgfpathclose%
\pgfusepath{stroke,fill}%
\end{pgfscope}%
\begin{pgfscope}%
\pgfpathrectangle{\pgfqpoint{0.750000in}{0.500000in}}{\pgfqpoint{4.650000in}{3.020000in}}%
\pgfusepath{clip}%
\pgfsetbuttcap%
\pgfsetmiterjoin%
\definecolor{currentfill}{rgb}{0.121569,0.466667,0.705882}%
\pgfsetfillcolor{currentfill}%
\pgfsetlinewidth{1.003750pt}%
\definecolor{currentstroke}{rgb}{0.000000,0.000000,0.000000}%
\pgfsetstrokecolor{currentstroke}%
\pgfsetdash{}{0pt}%
\pgfpathmoveto{\pgfqpoint{3.454371in}{0.500000in}}%
\pgfpathlineto{\pgfqpoint{3.562762in}{0.500000in}}%
\pgfpathlineto{\pgfqpoint{3.562762in}{1.139153in}}%
\pgfpathlineto{\pgfqpoint{3.454371in}{1.139153in}}%
\pgfpathclose%
\pgfusepath{stroke,fill}%
\end{pgfscope}%
\begin{pgfscope}%
\pgfpathrectangle{\pgfqpoint{0.750000in}{0.500000in}}{\pgfqpoint{4.650000in}{3.020000in}}%
\pgfusepath{clip}%
\pgfsetbuttcap%
\pgfsetmiterjoin%
\definecolor{currentfill}{rgb}{0.121569,0.466667,0.705882}%
\pgfsetfillcolor{currentfill}%
\pgfsetlinewidth{1.003750pt}%
\definecolor{currentstroke}{rgb}{0.000000,0.000000,0.000000}%
\pgfsetstrokecolor{currentstroke}%
\pgfsetdash{}{0pt}%
\pgfpathmoveto{\pgfqpoint{3.562762in}{0.500000in}}%
\pgfpathlineto{\pgfqpoint{3.671154in}{0.500000in}}%
\pgfpathlineto{\pgfqpoint{3.671154in}{0.979365in}}%
\pgfpathlineto{\pgfqpoint{3.562762in}{0.979365in}}%
\pgfpathclose%
\pgfusepath{stroke,fill}%
\end{pgfscope}%
\begin{pgfscope}%
\pgfpathrectangle{\pgfqpoint{0.750000in}{0.500000in}}{\pgfqpoint{4.650000in}{3.020000in}}%
\pgfusepath{clip}%
\pgfsetbuttcap%
\pgfsetmiterjoin%
\definecolor{currentfill}{rgb}{0.121569,0.466667,0.705882}%
\pgfsetfillcolor{currentfill}%
\pgfsetlinewidth{1.003750pt}%
\definecolor{currentstroke}{rgb}{0.000000,0.000000,0.000000}%
\pgfsetstrokecolor{currentstroke}%
\pgfsetdash{}{0pt}%
\pgfpathmoveto{\pgfqpoint{3.671154in}{0.500000in}}%
\pgfpathlineto{\pgfqpoint{3.779545in}{0.500000in}}%
\pgfpathlineto{\pgfqpoint{3.779545in}{1.458730in}}%
\pgfpathlineto{\pgfqpoint{3.671154in}{1.458730in}}%
\pgfpathclose%
\pgfusepath{stroke,fill}%
\end{pgfscope}%
\begin{pgfscope}%
\pgfpathrectangle{\pgfqpoint{0.750000in}{0.500000in}}{\pgfqpoint{4.650000in}{3.020000in}}%
\pgfusepath{clip}%
\pgfsetbuttcap%
\pgfsetmiterjoin%
\definecolor{currentfill}{rgb}{0.121569,0.466667,0.705882}%
\pgfsetfillcolor{currentfill}%
\pgfsetlinewidth{1.003750pt}%
\definecolor{currentstroke}{rgb}{0.000000,0.000000,0.000000}%
\pgfsetstrokecolor{currentstroke}%
\pgfsetdash{}{0pt}%
\pgfpathmoveto{\pgfqpoint{3.779545in}{0.500000in}}%
\pgfpathlineto{\pgfqpoint{3.887937in}{0.500000in}}%
\pgfpathlineto{\pgfqpoint{3.887937in}{0.659788in}}%
\pgfpathlineto{\pgfqpoint{3.779545in}{0.659788in}}%
\pgfpathclose%
\pgfusepath{stroke,fill}%
\end{pgfscope}%
\begin{pgfscope}%
\pgfpathrectangle{\pgfqpoint{0.750000in}{0.500000in}}{\pgfqpoint{4.650000in}{3.020000in}}%
\pgfusepath{clip}%
\pgfsetbuttcap%
\pgfsetmiterjoin%
\definecolor{currentfill}{rgb}{0.121569,0.466667,0.705882}%
\pgfsetfillcolor{currentfill}%
\pgfsetlinewidth{1.003750pt}%
\definecolor{currentstroke}{rgb}{0.000000,0.000000,0.000000}%
\pgfsetstrokecolor{currentstroke}%
\pgfsetdash{}{0pt}%
\pgfpathmoveto{\pgfqpoint{3.887937in}{0.500000in}}%
\pgfpathlineto{\pgfqpoint{3.996329in}{0.500000in}}%
\pgfpathlineto{\pgfqpoint{3.996329in}{1.139153in}}%
\pgfpathlineto{\pgfqpoint{3.887937in}{1.139153in}}%
\pgfpathclose%
\pgfusepath{stroke,fill}%
\end{pgfscope}%
\begin{pgfscope}%
\pgfpathrectangle{\pgfqpoint{0.750000in}{0.500000in}}{\pgfqpoint{4.650000in}{3.020000in}}%
\pgfusepath{clip}%
\pgfsetbuttcap%
\pgfsetmiterjoin%
\definecolor{currentfill}{rgb}{0.121569,0.466667,0.705882}%
\pgfsetfillcolor{currentfill}%
\pgfsetlinewidth{1.003750pt}%
\definecolor{currentstroke}{rgb}{0.000000,0.000000,0.000000}%
\pgfsetstrokecolor{currentstroke}%
\pgfsetdash{}{0pt}%
\pgfpathmoveto{\pgfqpoint{3.996329in}{0.500000in}}%
\pgfpathlineto{\pgfqpoint{4.104720in}{0.500000in}}%
\pgfpathlineto{\pgfqpoint{4.104720in}{0.979365in}}%
\pgfpathlineto{\pgfqpoint{3.996329in}{0.979365in}}%
\pgfpathclose%
\pgfusepath{stroke,fill}%
\end{pgfscope}%
\begin{pgfscope}%
\pgfpathrectangle{\pgfqpoint{0.750000in}{0.500000in}}{\pgfqpoint{4.650000in}{3.020000in}}%
\pgfusepath{clip}%
\pgfsetbuttcap%
\pgfsetmiterjoin%
\definecolor{currentfill}{rgb}{0.121569,0.466667,0.705882}%
\pgfsetfillcolor{currentfill}%
\pgfsetlinewidth{1.003750pt}%
\definecolor{currentstroke}{rgb}{0.000000,0.000000,0.000000}%
\pgfsetstrokecolor{currentstroke}%
\pgfsetdash{}{0pt}%
\pgfpathmoveto{\pgfqpoint{4.104720in}{0.500000in}}%
\pgfpathlineto{\pgfqpoint{4.213112in}{0.500000in}}%
\pgfpathlineto{\pgfqpoint{4.213112in}{0.979365in}}%
\pgfpathlineto{\pgfqpoint{4.104720in}{0.979365in}}%
\pgfpathclose%
\pgfusepath{stroke,fill}%
\end{pgfscope}%
\begin{pgfscope}%
\pgfpathrectangle{\pgfqpoint{0.750000in}{0.500000in}}{\pgfqpoint{4.650000in}{3.020000in}}%
\pgfusepath{clip}%
\pgfsetbuttcap%
\pgfsetmiterjoin%
\definecolor{currentfill}{rgb}{0.121569,0.466667,0.705882}%
\pgfsetfillcolor{currentfill}%
\pgfsetlinewidth{1.003750pt}%
\definecolor{currentstroke}{rgb}{0.000000,0.000000,0.000000}%
\pgfsetstrokecolor{currentstroke}%
\pgfsetdash{}{0pt}%
\pgfpathmoveto{\pgfqpoint{4.213112in}{0.500000in}}%
\pgfpathlineto{\pgfqpoint{4.321503in}{0.500000in}}%
\pgfpathlineto{\pgfqpoint{4.321503in}{0.659788in}}%
\pgfpathlineto{\pgfqpoint{4.213112in}{0.659788in}}%
\pgfpathclose%
\pgfusepath{stroke,fill}%
\end{pgfscope}%
\begin{pgfscope}%
\pgfpathrectangle{\pgfqpoint{0.750000in}{0.500000in}}{\pgfqpoint{4.650000in}{3.020000in}}%
\pgfusepath{clip}%
\pgfsetbuttcap%
\pgfsetmiterjoin%
\definecolor{currentfill}{rgb}{0.121569,0.466667,0.705882}%
\pgfsetfillcolor{currentfill}%
\pgfsetlinewidth{1.003750pt}%
\definecolor{currentstroke}{rgb}{0.000000,0.000000,0.000000}%
\pgfsetstrokecolor{currentstroke}%
\pgfsetdash{}{0pt}%
\pgfpathmoveto{\pgfqpoint{4.321503in}{0.500000in}}%
\pgfpathlineto{\pgfqpoint{4.429895in}{0.500000in}}%
\pgfpathlineto{\pgfqpoint{4.429895in}{0.819577in}}%
\pgfpathlineto{\pgfqpoint{4.321503in}{0.819577in}}%
\pgfpathclose%
\pgfusepath{stroke,fill}%
\end{pgfscope}%
\begin{pgfscope}%
\pgfpathrectangle{\pgfqpoint{0.750000in}{0.500000in}}{\pgfqpoint{4.650000in}{3.020000in}}%
\pgfusepath{clip}%
\pgfsetbuttcap%
\pgfsetmiterjoin%
\definecolor{currentfill}{rgb}{0.121569,0.466667,0.705882}%
\pgfsetfillcolor{currentfill}%
\pgfsetlinewidth{1.003750pt}%
\definecolor{currentstroke}{rgb}{0.000000,0.000000,0.000000}%
\pgfsetstrokecolor{currentstroke}%
\pgfsetdash{}{0pt}%
\pgfpathmoveto{\pgfqpoint{4.429895in}{0.500000in}}%
\pgfpathlineto{\pgfqpoint{4.538287in}{0.500000in}}%
\pgfpathlineto{\pgfqpoint{4.538287in}{0.659788in}}%
\pgfpathlineto{\pgfqpoint{4.429895in}{0.659788in}}%
\pgfpathclose%
\pgfusepath{stroke,fill}%
\end{pgfscope}%
\begin{pgfscope}%
\pgfpathrectangle{\pgfqpoint{0.750000in}{0.500000in}}{\pgfqpoint{4.650000in}{3.020000in}}%
\pgfusepath{clip}%
\pgfsetbuttcap%
\pgfsetmiterjoin%
\definecolor{currentfill}{rgb}{0.121569,0.466667,0.705882}%
\pgfsetfillcolor{currentfill}%
\pgfsetlinewidth{1.003750pt}%
\definecolor{currentstroke}{rgb}{0.000000,0.000000,0.000000}%
\pgfsetstrokecolor{currentstroke}%
\pgfsetdash{}{0pt}%
\pgfpathmoveto{\pgfqpoint{4.538287in}{0.500000in}}%
\pgfpathlineto{\pgfqpoint{4.646678in}{0.500000in}}%
\pgfpathlineto{\pgfqpoint{4.646678in}{0.659788in}}%
\pgfpathlineto{\pgfqpoint{4.538287in}{0.659788in}}%
\pgfpathclose%
\pgfusepath{stroke,fill}%
\end{pgfscope}%
\begin{pgfscope}%
\pgfpathrectangle{\pgfqpoint{0.750000in}{0.500000in}}{\pgfqpoint{4.650000in}{3.020000in}}%
\pgfusepath{clip}%
\pgfsetbuttcap%
\pgfsetmiterjoin%
\definecolor{currentfill}{rgb}{0.121569,0.466667,0.705882}%
\pgfsetfillcolor{currentfill}%
\pgfsetlinewidth{1.003750pt}%
\definecolor{currentstroke}{rgb}{0.000000,0.000000,0.000000}%
\pgfsetstrokecolor{currentstroke}%
\pgfsetdash{}{0pt}%
\pgfpathmoveto{\pgfqpoint{4.646678in}{0.500000in}}%
\pgfpathlineto{\pgfqpoint{4.755070in}{0.500000in}}%
\pgfpathlineto{\pgfqpoint{4.755070in}{0.500000in}}%
\pgfpathlineto{\pgfqpoint{4.646678in}{0.500000in}}%
\pgfpathclose%
\pgfusepath{stroke,fill}%
\end{pgfscope}%
\begin{pgfscope}%
\pgfpathrectangle{\pgfqpoint{0.750000in}{0.500000in}}{\pgfqpoint{4.650000in}{3.020000in}}%
\pgfusepath{clip}%
\pgfsetbuttcap%
\pgfsetmiterjoin%
\definecolor{currentfill}{rgb}{0.121569,0.466667,0.705882}%
\pgfsetfillcolor{currentfill}%
\pgfsetlinewidth{1.003750pt}%
\definecolor{currentstroke}{rgb}{0.000000,0.000000,0.000000}%
\pgfsetstrokecolor{currentstroke}%
\pgfsetdash{}{0pt}%
\pgfpathmoveto{\pgfqpoint{4.755070in}{0.500000in}}%
\pgfpathlineto{\pgfqpoint{4.863462in}{0.500000in}}%
\pgfpathlineto{\pgfqpoint{4.863462in}{0.500000in}}%
\pgfpathlineto{\pgfqpoint{4.755070in}{0.500000in}}%
\pgfpathclose%
\pgfusepath{stroke,fill}%
\end{pgfscope}%
\begin{pgfscope}%
\pgfpathrectangle{\pgfqpoint{0.750000in}{0.500000in}}{\pgfqpoint{4.650000in}{3.020000in}}%
\pgfusepath{clip}%
\pgfsetbuttcap%
\pgfsetmiterjoin%
\definecolor{currentfill}{rgb}{0.121569,0.466667,0.705882}%
\pgfsetfillcolor{currentfill}%
\pgfsetlinewidth{1.003750pt}%
\definecolor{currentstroke}{rgb}{0.000000,0.000000,0.000000}%
\pgfsetstrokecolor{currentstroke}%
\pgfsetdash{}{0pt}%
\pgfpathmoveto{\pgfqpoint{4.863462in}{0.500000in}}%
\pgfpathlineto{\pgfqpoint{4.971853in}{0.500000in}}%
\pgfpathlineto{\pgfqpoint{4.971853in}{0.500000in}}%
\pgfpathlineto{\pgfqpoint{4.863462in}{0.500000in}}%
\pgfpathclose%
\pgfusepath{stroke,fill}%
\end{pgfscope}%
\begin{pgfscope}%
\pgfpathrectangle{\pgfqpoint{0.750000in}{0.500000in}}{\pgfqpoint{4.650000in}{3.020000in}}%
\pgfusepath{clip}%
\pgfsetbuttcap%
\pgfsetmiterjoin%
\definecolor{currentfill}{rgb}{0.121569,0.466667,0.705882}%
\pgfsetfillcolor{currentfill}%
\pgfsetlinewidth{1.003750pt}%
\definecolor{currentstroke}{rgb}{0.000000,0.000000,0.000000}%
\pgfsetstrokecolor{currentstroke}%
\pgfsetdash{}{0pt}%
\pgfpathmoveto{\pgfqpoint{4.971853in}{0.500000in}}%
\pgfpathlineto{\pgfqpoint{5.080245in}{0.500000in}}%
\pgfpathlineto{\pgfqpoint{5.080245in}{0.500000in}}%
\pgfpathlineto{\pgfqpoint{4.971853in}{0.500000in}}%
\pgfpathclose%
\pgfusepath{stroke,fill}%
\end{pgfscope}%
\begin{pgfscope}%
\pgfpathrectangle{\pgfqpoint{0.750000in}{0.500000in}}{\pgfqpoint{4.650000in}{3.020000in}}%
\pgfusepath{clip}%
\pgfsetbuttcap%
\pgfsetmiterjoin%
\definecolor{currentfill}{rgb}{0.121569,0.466667,0.705882}%
\pgfsetfillcolor{currentfill}%
\pgfsetlinewidth{1.003750pt}%
\definecolor{currentstroke}{rgb}{0.000000,0.000000,0.000000}%
\pgfsetstrokecolor{currentstroke}%
\pgfsetdash{}{0pt}%
\pgfpathmoveto{\pgfqpoint{5.080245in}{0.500000in}}%
\pgfpathlineto{\pgfqpoint{5.188636in}{0.500000in}}%
\pgfpathlineto{\pgfqpoint{5.188636in}{0.659788in}}%
\pgfpathlineto{\pgfqpoint{5.080245in}{0.659788in}}%
\pgfpathclose%
\pgfusepath{stroke,fill}%
\end{pgfscope}%
\begin{pgfscope}%
\pgfsetbuttcap%
\pgfsetroundjoin%
\definecolor{currentfill}{rgb}{0.000000,0.000000,0.000000}%
\pgfsetfillcolor{currentfill}%
\pgfsetlinewidth{0.803000pt}%
\definecolor{currentstroke}{rgb}{0.000000,0.000000,0.000000}%
\pgfsetstrokecolor{currentstroke}%
\pgfsetdash{}{0pt}%
\pgfsys@defobject{currentmarker}{\pgfqpoint{0.000000in}{-0.048611in}}{\pgfqpoint{0.000000in}{0.000000in}}{%
\pgfpathmoveto{\pgfqpoint{0.000000in}{0.000000in}}%
\pgfpathlineto{\pgfqpoint{0.000000in}{-0.048611in}}%
\pgfusepath{stroke,fill}%
}%
\begin{pgfscope}%
\pgfsys@transformshift{0.961364in}{0.500000in}%
\pgfsys@useobject{currentmarker}{}%
\end{pgfscope}%
\end{pgfscope}%
\begin{pgfscope}%
\definecolor{textcolor}{rgb}{0.000000,0.000000,0.000000}%
\pgfsetstrokecolor{textcolor}%
\pgfsetfillcolor{textcolor}%
\pgftext[x=0.961364in,y=0.402778in,,top]{\color{textcolor}\sffamily\fontsize{10.000000}{12.000000}\selectfont \(\displaystyle {-20000}\)}%
\end{pgfscope}%
\begin{pgfscope}%
\pgfsetbuttcap%
\pgfsetroundjoin%
\definecolor{currentfill}{rgb}{0.000000,0.000000,0.000000}%
\pgfsetfillcolor{currentfill}%
\pgfsetlinewidth{0.803000pt}%
\definecolor{currentstroke}{rgb}{0.000000,0.000000,0.000000}%
\pgfsetstrokecolor{currentstroke}%
\pgfsetdash{}{0pt}%
\pgfsys@defobject{currentmarker}{\pgfqpoint{0.000000in}{-0.048611in}}{\pgfqpoint{0.000000in}{0.000000in}}{%
\pgfpathmoveto{\pgfqpoint{0.000000in}{0.000000in}}%
\pgfpathlineto{\pgfqpoint{0.000000in}{-0.048611in}}%
\pgfusepath{stroke,fill}%
}%
\begin{pgfscope}%
\pgfsys@transformshift{1.503322in}{0.500000in}%
\pgfsys@useobject{currentmarker}{}%
\end{pgfscope}%
\end{pgfscope}%
\begin{pgfscope}%
\definecolor{textcolor}{rgb}{0.000000,0.000000,0.000000}%
\pgfsetstrokecolor{textcolor}%
\pgfsetfillcolor{textcolor}%
\pgftext[x=1.503322in,y=0.402778in,,top]{\color{textcolor}\sffamily\fontsize{10.000000}{12.000000}\selectfont \(\displaystyle {-15000}\)}%
\end{pgfscope}%
\begin{pgfscope}%
\pgfsetbuttcap%
\pgfsetroundjoin%
\definecolor{currentfill}{rgb}{0.000000,0.000000,0.000000}%
\pgfsetfillcolor{currentfill}%
\pgfsetlinewidth{0.803000pt}%
\definecolor{currentstroke}{rgb}{0.000000,0.000000,0.000000}%
\pgfsetstrokecolor{currentstroke}%
\pgfsetdash{}{0pt}%
\pgfsys@defobject{currentmarker}{\pgfqpoint{0.000000in}{-0.048611in}}{\pgfqpoint{0.000000in}{0.000000in}}{%
\pgfpathmoveto{\pgfqpoint{0.000000in}{0.000000in}}%
\pgfpathlineto{\pgfqpoint{0.000000in}{-0.048611in}}%
\pgfusepath{stroke,fill}%
}%
\begin{pgfscope}%
\pgfsys@transformshift{2.045280in}{0.500000in}%
\pgfsys@useobject{currentmarker}{}%
\end{pgfscope}%
\end{pgfscope}%
\begin{pgfscope}%
\definecolor{textcolor}{rgb}{0.000000,0.000000,0.000000}%
\pgfsetstrokecolor{textcolor}%
\pgfsetfillcolor{textcolor}%
\pgftext[x=2.045280in,y=0.402778in,,top]{\color{textcolor}\sffamily\fontsize{10.000000}{12.000000}\selectfont \(\displaystyle {-10000}\)}%
\end{pgfscope}%
\begin{pgfscope}%
\pgfsetbuttcap%
\pgfsetroundjoin%
\definecolor{currentfill}{rgb}{0.000000,0.000000,0.000000}%
\pgfsetfillcolor{currentfill}%
\pgfsetlinewidth{0.803000pt}%
\definecolor{currentstroke}{rgb}{0.000000,0.000000,0.000000}%
\pgfsetstrokecolor{currentstroke}%
\pgfsetdash{}{0pt}%
\pgfsys@defobject{currentmarker}{\pgfqpoint{0.000000in}{-0.048611in}}{\pgfqpoint{0.000000in}{0.000000in}}{%
\pgfpathmoveto{\pgfqpoint{0.000000in}{0.000000in}}%
\pgfpathlineto{\pgfqpoint{0.000000in}{-0.048611in}}%
\pgfusepath{stroke,fill}%
}%
\begin{pgfscope}%
\pgfsys@transformshift{2.587238in}{0.500000in}%
\pgfsys@useobject{currentmarker}{}%
\end{pgfscope}%
\end{pgfscope}%
\begin{pgfscope}%
\definecolor{textcolor}{rgb}{0.000000,0.000000,0.000000}%
\pgfsetstrokecolor{textcolor}%
\pgfsetfillcolor{textcolor}%
\pgftext[x=2.587238in,y=0.402778in,,top]{\color{textcolor}\sffamily\fontsize{10.000000}{12.000000}\selectfont \(\displaystyle {-5000}\)}%
\end{pgfscope}%
\begin{pgfscope}%
\pgfsetbuttcap%
\pgfsetroundjoin%
\definecolor{currentfill}{rgb}{0.000000,0.000000,0.000000}%
\pgfsetfillcolor{currentfill}%
\pgfsetlinewidth{0.803000pt}%
\definecolor{currentstroke}{rgb}{0.000000,0.000000,0.000000}%
\pgfsetstrokecolor{currentstroke}%
\pgfsetdash{}{0pt}%
\pgfsys@defobject{currentmarker}{\pgfqpoint{0.000000in}{-0.048611in}}{\pgfqpoint{0.000000in}{0.000000in}}{%
\pgfpathmoveto{\pgfqpoint{0.000000in}{0.000000in}}%
\pgfpathlineto{\pgfqpoint{0.000000in}{-0.048611in}}%
\pgfusepath{stroke,fill}%
}%
\begin{pgfscope}%
\pgfsys@transformshift{3.129196in}{0.500000in}%
\pgfsys@useobject{currentmarker}{}%
\end{pgfscope}%
\end{pgfscope}%
\begin{pgfscope}%
\definecolor{textcolor}{rgb}{0.000000,0.000000,0.000000}%
\pgfsetstrokecolor{textcolor}%
\pgfsetfillcolor{textcolor}%
\pgftext[x=3.129196in,y=0.402778in,,top]{\color{textcolor}\sffamily\fontsize{10.000000}{12.000000}\selectfont \(\displaystyle {0}\)}%
\end{pgfscope}%
\begin{pgfscope}%
\pgfsetbuttcap%
\pgfsetroundjoin%
\definecolor{currentfill}{rgb}{0.000000,0.000000,0.000000}%
\pgfsetfillcolor{currentfill}%
\pgfsetlinewidth{0.803000pt}%
\definecolor{currentstroke}{rgb}{0.000000,0.000000,0.000000}%
\pgfsetstrokecolor{currentstroke}%
\pgfsetdash{}{0pt}%
\pgfsys@defobject{currentmarker}{\pgfqpoint{0.000000in}{-0.048611in}}{\pgfqpoint{0.000000in}{0.000000in}}{%
\pgfpathmoveto{\pgfqpoint{0.000000in}{0.000000in}}%
\pgfpathlineto{\pgfqpoint{0.000000in}{-0.048611in}}%
\pgfusepath{stroke,fill}%
}%
\begin{pgfscope}%
\pgfsys@transformshift{3.671154in}{0.500000in}%
\pgfsys@useobject{currentmarker}{}%
\end{pgfscope}%
\end{pgfscope}%
\begin{pgfscope}%
\definecolor{textcolor}{rgb}{0.000000,0.000000,0.000000}%
\pgfsetstrokecolor{textcolor}%
\pgfsetfillcolor{textcolor}%
\pgftext[x=3.671154in,y=0.402778in,,top]{\color{textcolor}\sffamily\fontsize{10.000000}{12.000000}\selectfont \(\displaystyle {5000}\)}%
\end{pgfscope}%
\begin{pgfscope}%
\pgfsetbuttcap%
\pgfsetroundjoin%
\definecolor{currentfill}{rgb}{0.000000,0.000000,0.000000}%
\pgfsetfillcolor{currentfill}%
\pgfsetlinewidth{0.803000pt}%
\definecolor{currentstroke}{rgb}{0.000000,0.000000,0.000000}%
\pgfsetstrokecolor{currentstroke}%
\pgfsetdash{}{0pt}%
\pgfsys@defobject{currentmarker}{\pgfqpoint{0.000000in}{-0.048611in}}{\pgfqpoint{0.000000in}{0.000000in}}{%
\pgfpathmoveto{\pgfqpoint{0.000000in}{0.000000in}}%
\pgfpathlineto{\pgfqpoint{0.000000in}{-0.048611in}}%
\pgfusepath{stroke,fill}%
}%
\begin{pgfscope}%
\pgfsys@transformshift{4.213112in}{0.500000in}%
\pgfsys@useobject{currentmarker}{}%
\end{pgfscope}%
\end{pgfscope}%
\begin{pgfscope}%
\definecolor{textcolor}{rgb}{0.000000,0.000000,0.000000}%
\pgfsetstrokecolor{textcolor}%
\pgfsetfillcolor{textcolor}%
\pgftext[x=4.213112in,y=0.402778in,,top]{\color{textcolor}\sffamily\fontsize{10.000000}{12.000000}\selectfont \(\displaystyle {10000}\)}%
\end{pgfscope}%
\begin{pgfscope}%
\pgfsetbuttcap%
\pgfsetroundjoin%
\definecolor{currentfill}{rgb}{0.000000,0.000000,0.000000}%
\pgfsetfillcolor{currentfill}%
\pgfsetlinewidth{0.803000pt}%
\definecolor{currentstroke}{rgb}{0.000000,0.000000,0.000000}%
\pgfsetstrokecolor{currentstroke}%
\pgfsetdash{}{0pt}%
\pgfsys@defobject{currentmarker}{\pgfqpoint{0.000000in}{-0.048611in}}{\pgfqpoint{0.000000in}{0.000000in}}{%
\pgfpathmoveto{\pgfqpoint{0.000000in}{0.000000in}}%
\pgfpathlineto{\pgfqpoint{0.000000in}{-0.048611in}}%
\pgfusepath{stroke,fill}%
}%
\begin{pgfscope}%
\pgfsys@transformshift{4.755070in}{0.500000in}%
\pgfsys@useobject{currentmarker}{}%
\end{pgfscope}%
\end{pgfscope}%
\begin{pgfscope}%
\definecolor{textcolor}{rgb}{0.000000,0.000000,0.000000}%
\pgfsetstrokecolor{textcolor}%
\pgfsetfillcolor{textcolor}%
\pgftext[x=4.755070in,y=0.402778in,,top]{\color{textcolor}\sffamily\fontsize{10.000000}{12.000000}\selectfont \(\displaystyle {15000}\)}%
\end{pgfscope}%
\begin{pgfscope}%
\pgfsetbuttcap%
\pgfsetroundjoin%
\definecolor{currentfill}{rgb}{0.000000,0.000000,0.000000}%
\pgfsetfillcolor{currentfill}%
\pgfsetlinewidth{0.803000pt}%
\definecolor{currentstroke}{rgb}{0.000000,0.000000,0.000000}%
\pgfsetstrokecolor{currentstroke}%
\pgfsetdash{}{0pt}%
\pgfsys@defobject{currentmarker}{\pgfqpoint{0.000000in}{-0.048611in}}{\pgfqpoint{0.000000in}{0.000000in}}{%
\pgfpathmoveto{\pgfqpoint{0.000000in}{0.000000in}}%
\pgfpathlineto{\pgfqpoint{0.000000in}{-0.048611in}}%
\pgfusepath{stroke,fill}%
}%
\begin{pgfscope}%
\pgfsys@transformshift{5.297028in}{0.500000in}%
\pgfsys@useobject{currentmarker}{}%
\end{pgfscope}%
\end{pgfscope}%
\begin{pgfscope}%
\definecolor{textcolor}{rgb}{0.000000,0.000000,0.000000}%
\pgfsetstrokecolor{textcolor}%
\pgfsetfillcolor{textcolor}%
\pgftext[x=5.297028in,y=0.402778in,,top]{\color{textcolor}\sffamily\fontsize{10.000000}{12.000000}\selectfont \(\displaystyle {20000}\)}%
\end{pgfscope}%
\begin{pgfscope}%
\definecolor{textcolor}{rgb}{0.000000,0.000000,0.000000}%
\pgfsetstrokecolor{textcolor}%
\pgfsetfillcolor{textcolor}%
\pgftext[x=3.075000in,y=0.223889in,,top]{\color{textcolor}\sffamily\fontsize{10.000000}{12.000000}\selectfont \(\displaystyle \Delta\) Maximum Memory Consumption}%
\end{pgfscope}%
\begin{pgfscope}%
\pgfsetbuttcap%
\pgfsetroundjoin%
\definecolor{currentfill}{rgb}{0.000000,0.000000,0.000000}%
\pgfsetfillcolor{currentfill}%
\pgfsetlinewidth{0.803000pt}%
\definecolor{currentstroke}{rgb}{0.000000,0.000000,0.000000}%
\pgfsetstrokecolor{currentstroke}%
\pgfsetdash{}{0pt}%
\pgfsys@defobject{currentmarker}{\pgfqpoint{-0.048611in}{0.000000in}}{\pgfqpoint{0.000000in}{0.000000in}}{%
\pgfpathmoveto{\pgfqpoint{0.000000in}{0.000000in}}%
\pgfpathlineto{\pgfqpoint{-0.048611in}{0.000000in}}%
\pgfusepath{stroke,fill}%
}%
\begin{pgfscope}%
\pgfsys@transformshift{0.750000in}{0.500000in}%
\pgfsys@useobject{currentmarker}{}%
\end{pgfscope}%
\end{pgfscope}%
\begin{pgfscope}%
\definecolor{textcolor}{rgb}{0.000000,0.000000,0.000000}%
\pgfsetstrokecolor{textcolor}%
\pgfsetfillcolor{textcolor}%
\pgftext[x=0.583333in, y=0.451806in, left, base]{\color{textcolor}\sffamily\fontsize{10.000000}{12.000000}\selectfont \(\displaystyle {0}\)}%
\end{pgfscope}%
\begin{pgfscope}%
\pgfsetbuttcap%
\pgfsetroundjoin%
\definecolor{currentfill}{rgb}{0.000000,0.000000,0.000000}%
\pgfsetfillcolor{currentfill}%
\pgfsetlinewidth{0.803000pt}%
\definecolor{currentstroke}{rgb}{0.000000,0.000000,0.000000}%
\pgfsetstrokecolor{currentstroke}%
\pgfsetdash{}{0pt}%
\pgfsys@defobject{currentmarker}{\pgfqpoint{-0.048611in}{0.000000in}}{\pgfqpoint{0.000000in}{0.000000in}}{%
\pgfpathmoveto{\pgfqpoint{0.000000in}{0.000000in}}%
\pgfpathlineto{\pgfqpoint{-0.048611in}{0.000000in}}%
\pgfusepath{stroke,fill}%
}%
\begin{pgfscope}%
\pgfsys@transformshift{0.750000in}{0.819577in}%
\pgfsys@useobject{currentmarker}{}%
\end{pgfscope}%
\end{pgfscope}%
\begin{pgfscope}%
\definecolor{textcolor}{rgb}{0.000000,0.000000,0.000000}%
\pgfsetstrokecolor{textcolor}%
\pgfsetfillcolor{textcolor}%
\pgftext[x=0.583333in, y=0.771382in, left, base]{\color{textcolor}\sffamily\fontsize{10.000000}{12.000000}\selectfont \(\displaystyle {2}\)}%
\end{pgfscope}%
\begin{pgfscope}%
\pgfsetbuttcap%
\pgfsetroundjoin%
\definecolor{currentfill}{rgb}{0.000000,0.000000,0.000000}%
\pgfsetfillcolor{currentfill}%
\pgfsetlinewidth{0.803000pt}%
\definecolor{currentstroke}{rgb}{0.000000,0.000000,0.000000}%
\pgfsetstrokecolor{currentstroke}%
\pgfsetdash{}{0pt}%
\pgfsys@defobject{currentmarker}{\pgfqpoint{-0.048611in}{0.000000in}}{\pgfqpoint{0.000000in}{0.000000in}}{%
\pgfpathmoveto{\pgfqpoint{0.000000in}{0.000000in}}%
\pgfpathlineto{\pgfqpoint{-0.048611in}{0.000000in}}%
\pgfusepath{stroke,fill}%
}%
\begin{pgfscope}%
\pgfsys@transformshift{0.750000in}{1.139153in}%
\pgfsys@useobject{currentmarker}{}%
\end{pgfscope}%
\end{pgfscope}%
\begin{pgfscope}%
\definecolor{textcolor}{rgb}{0.000000,0.000000,0.000000}%
\pgfsetstrokecolor{textcolor}%
\pgfsetfillcolor{textcolor}%
\pgftext[x=0.583333in, y=1.090959in, left, base]{\color{textcolor}\sffamily\fontsize{10.000000}{12.000000}\selectfont \(\displaystyle {4}\)}%
\end{pgfscope}%
\begin{pgfscope}%
\pgfsetbuttcap%
\pgfsetroundjoin%
\definecolor{currentfill}{rgb}{0.000000,0.000000,0.000000}%
\pgfsetfillcolor{currentfill}%
\pgfsetlinewidth{0.803000pt}%
\definecolor{currentstroke}{rgb}{0.000000,0.000000,0.000000}%
\pgfsetstrokecolor{currentstroke}%
\pgfsetdash{}{0pt}%
\pgfsys@defobject{currentmarker}{\pgfqpoint{-0.048611in}{0.000000in}}{\pgfqpoint{0.000000in}{0.000000in}}{%
\pgfpathmoveto{\pgfqpoint{0.000000in}{0.000000in}}%
\pgfpathlineto{\pgfqpoint{-0.048611in}{0.000000in}}%
\pgfusepath{stroke,fill}%
}%
\begin{pgfscope}%
\pgfsys@transformshift{0.750000in}{1.458730in}%
\pgfsys@useobject{currentmarker}{}%
\end{pgfscope}%
\end{pgfscope}%
\begin{pgfscope}%
\definecolor{textcolor}{rgb}{0.000000,0.000000,0.000000}%
\pgfsetstrokecolor{textcolor}%
\pgfsetfillcolor{textcolor}%
\pgftext[x=0.583333in, y=1.410536in, left, base]{\color{textcolor}\sffamily\fontsize{10.000000}{12.000000}\selectfont \(\displaystyle {6}\)}%
\end{pgfscope}%
\begin{pgfscope}%
\pgfsetbuttcap%
\pgfsetroundjoin%
\definecolor{currentfill}{rgb}{0.000000,0.000000,0.000000}%
\pgfsetfillcolor{currentfill}%
\pgfsetlinewidth{0.803000pt}%
\definecolor{currentstroke}{rgb}{0.000000,0.000000,0.000000}%
\pgfsetstrokecolor{currentstroke}%
\pgfsetdash{}{0pt}%
\pgfsys@defobject{currentmarker}{\pgfqpoint{-0.048611in}{0.000000in}}{\pgfqpoint{0.000000in}{0.000000in}}{%
\pgfpathmoveto{\pgfqpoint{0.000000in}{0.000000in}}%
\pgfpathlineto{\pgfqpoint{-0.048611in}{0.000000in}}%
\pgfusepath{stroke,fill}%
}%
\begin{pgfscope}%
\pgfsys@transformshift{0.750000in}{1.778307in}%
\pgfsys@useobject{currentmarker}{}%
\end{pgfscope}%
\end{pgfscope}%
\begin{pgfscope}%
\definecolor{textcolor}{rgb}{0.000000,0.000000,0.000000}%
\pgfsetstrokecolor{textcolor}%
\pgfsetfillcolor{textcolor}%
\pgftext[x=0.583333in, y=1.730112in, left, base]{\color{textcolor}\sffamily\fontsize{10.000000}{12.000000}\selectfont \(\displaystyle {8}\)}%
\end{pgfscope}%
\begin{pgfscope}%
\pgfsetbuttcap%
\pgfsetroundjoin%
\definecolor{currentfill}{rgb}{0.000000,0.000000,0.000000}%
\pgfsetfillcolor{currentfill}%
\pgfsetlinewidth{0.803000pt}%
\definecolor{currentstroke}{rgb}{0.000000,0.000000,0.000000}%
\pgfsetstrokecolor{currentstroke}%
\pgfsetdash{}{0pt}%
\pgfsys@defobject{currentmarker}{\pgfqpoint{-0.048611in}{0.000000in}}{\pgfqpoint{0.000000in}{0.000000in}}{%
\pgfpathmoveto{\pgfqpoint{0.000000in}{0.000000in}}%
\pgfpathlineto{\pgfqpoint{-0.048611in}{0.000000in}}%
\pgfusepath{stroke,fill}%
}%
\begin{pgfscope}%
\pgfsys@transformshift{0.750000in}{2.097884in}%
\pgfsys@useobject{currentmarker}{}%
\end{pgfscope}%
\end{pgfscope}%
\begin{pgfscope}%
\definecolor{textcolor}{rgb}{0.000000,0.000000,0.000000}%
\pgfsetstrokecolor{textcolor}%
\pgfsetfillcolor{textcolor}%
\pgftext[x=0.513888in, y=2.049689in, left, base]{\color{textcolor}\sffamily\fontsize{10.000000}{12.000000}\selectfont \(\displaystyle {10}\)}%
\end{pgfscope}%
\begin{pgfscope}%
\pgfsetbuttcap%
\pgfsetroundjoin%
\definecolor{currentfill}{rgb}{0.000000,0.000000,0.000000}%
\pgfsetfillcolor{currentfill}%
\pgfsetlinewidth{0.803000pt}%
\definecolor{currentstroke}{rgb}{0.000000,0.000000,0.000000}%
\pgfsetstrokecolor{currentstroke}%
\pgfsetdash{}{0pt}%
\pgfsys@defobject{currentmarker}{\pgfqpoint{-0.048611in}{0.000000in}}{\pgfqpoint{0.000000in}{0.000000in}}{%
\pgfpathmoveto{\pgfqpoint{0.000000in}{0.000000in}}%
\pgfpathlineto{\pgfqpoint{-0.048611in}{0.000000in}}%
\pgfusepath{stroke,fill}%
}%
\begin{pgfscope}%
\pgfsys@transformshift{0.750000in}{2.417460in}%
\pgfsys@useobject{currentmarker}{}%
\end{pgfscope}%
\end{pgfscope}%
\begin{pgfscope}%
\definecolor{textcolor}{rgb}{0.000000,0.000000,0.000000}%
\pgfsetstrokecolor{textcolor}%
\pgfsetfillcolor{textcolor}%
\pgftext[x=0.513888in, y=2.369266in, left, base]{\color{textcolor}\sffamily\fontsize{10.000000}{12.000000}\selectfont \(\displaystyle {12}\)}%
\end{pgfscope}%
\begin{pgfscope}%
\pgfsetbuttcap%
\pgfsetroundjoin%
\definecolor{currentfill}{rgb}{0.000000,0.000000,0.000000}%
\pgfsetfillcolor{currentfill}%
\pgfsetlinewidth{0.803000pt}%
\definecolor{currentstroke}{rgb}{0.000000,0.000000,0.000000}%
\pgfsetstrokecolor{currentstroke}%
\pgfsetdash{}{0pt}%
\pgfsys@defobject{currentmarker}{\pgfqpoint{-0.048611in}{0.000000in}}{\pgfqpoint{0.000000in}{0.000000in}}{%
\pgfpathmoveto{\pgfqpoint{0.000000in}{0.000000in}}%
\pgfpathlineto{\pgfqpoint{-0.048611in}{0.000000in}}%
\pgfusepath{stroke,fill}%
}%
\begin{pgfscope}%
\pgfsys@transformshift{0.750000in}{2.737037in}%
\pgfsys@useobject{currentmarker}{}%
\end{pgfscope}%
\end{pgfscope}%
\begin{pgfscope}%
\definecolor{textcolor}{rgb}{0.000000,0.000000,0.000000}%
\pgfsetstrokecolor{textcolor}%
\pgfsetfillcolor{textcolor}%
\pgftext[x=0.513888in, y=2.688843in, left, base]{\color{textcolor}\sffamily\fontsize{10.000000}{12.000000}\selectfont \(\displaystyle {14}\)}%
\end{pgfscope}%
\begin{pgfscope}%
\pgfsetbuttcap%
\pgfsetroundjoin%
\definecolor{currentfill}{rgb}{0.000000,0.000000,0.000000}%
\pgfsetfillcolor{currentfill}%
\pgfsetlinewidth{0.803000pt}%
\definecolor{currentstroke}{rgb}{0.000000,0.000000,0.000000}%
\pgfsetstrokecolor{currentstroke}%
\pgfsetdash{}{0pt}%
\pgfsys@defobject{currentmarker}{\pgfqpoint{-0.048611in}{0.000000in}}{\pgfqpoint{0.000000in}{0.000000in}}{%
\pgfpathmoveto{\pgfqpoint{0.000000in}{0.000000in}}%
\pgfpathlineto{\pgfqpoint{-0.048611in}{0.000000in}}%
\pgfusepath{stroke,fill}%
}%
\begin{pgfscope}%
\pgfsys@transformshift{0.750000in}{3.056614in}%
\pgfsys@useobject{currentmarker}{}%
\end{pgfscope}%
\end{pgfscope}%
\begin{pgfscope}%
\definecolor{textcolor}{rgb}{0.000000,0.000000,0.000000}%
\pgfsetstrokecolor{textcolor}%
\pgfsetfillcolor{textcolor}%
\pgftext[x=0.513888in, y=3.008419in, left, base]{\color{textcolor}\sffamily\fontsize{10.000000}{12.000000}\selectfont \(\displaystyle {16}\)}%
\end{pgfscope}%
\begin{pgfscope}%
\pgfsetbuttcap%
\pgfsetroundjoin%
\definecolor{currentfill}{rgb}{0.000000,0.000000,0.000000}%
\pgfsetfillcolor{currentfill}%
\pgfsetlinewidth{0.803000pt}%
\definecolor{currentstroke}{rgb}{0.000000,0.000000,0.000000}%
\pgfsetstrokecolor{currentstroke}%
\pgfsetdash{}{0pt}%
\pgfsys@defobject{currentmarker}{\pgfqpoint{-0.048611in}{0.000000in}}{\pgfqpoint{0.000000in}{0.000000in}}{%
\pgfpathmoveto{\pgfqpoint{0.000000in}{0.000000in}}%
\pgfpathlineto{\pgfqpoint{-0.048611in}{0.000000in}}%
\pgfusepath{stroke,fill}%
}%
\begin{pgfscope}%
\pgfsys@transformshift{0.750000in}{3.376190in}%
\pgfsys@useobject{currentmarker}{}%
\end{pgfscope}%
\end{pgfscope}%
\begin{pgfscope}%
\definecolor{textcolor}{rgb}{0.000000,0.000000,0.000000}%
\pgfsetstrokecolor{textcolor}%
\pgfsetfillcolor{textcolor}%
\pgftext[x=0.513888in, y=3.327996in, left, base]{\color{textcolor}\sffamily\fontsize{10.000000}{12.000000}\selectfont \(\displaystyle {18}\)}%
\end{pgfscope}%
\begin{pgfscope}%
\definecolor{textcolor}{rgb}{0.000000,0.000000,0.000000}%
\pgfsetstrokecolor{textcolor}%
\pgfsetfillcolor{textcolor}%
\pgftext[x=0.458333in,y=2.010000in,,bottom,rotate=90.000000]{\color{textcolor}\sffamily\fontsize{10.000000}{12.000000}\selectfont Frequency}%
\end{pgfscope}%
\begin{pgfscope}%
\pgfsetrectcap%
\pgfsetmiterjoin%
\pgfsetlinewidth{0.803000pt}%
\definecolor{currentstroke}{rgb}{0.000000,0.000000,0.000000}%
\pgfsetstrokecolor{currentstroke}%
\pgfsetdash{}{0pt}%
\pgfpathmoveto{\pgfqpoint{0.750000in}{0.500000in}}%
\pgfpathlineto{\pgfqpoint{0.750000in}{3.520000in}}%
\pgfusepath{stroke}%
\end{pgfscope}%
\begin{pgfscope}%
\pgfsetrectcap%
\pgfsetmiterjoin%
\pgfsetlinewidth{0.803000pt}%
\definecolor{currentstroke}{rgb}{0.000000,0.000000,0.000000}%
\pgfsetstrokecolor{currentstroke}%
\pgfsetdash{}{0pt}%
\pgfpathmoveto{\pgfqpoint{5.400000in}{0.500000in}}%
\pgfpathlineto{\pgfqpoint{5.400000in}{3.520000in}}%
\pgfusepath{stroke}%
\end{pgfscope}%
\begin{pgfscope}%
\pgfsetrectcap%
\pgfsetmiterjoin%
\pgfsetlinewidth{0.803000pt}%
\definecolor{currentstroke}{rgb}{0.000000,0.000000,0.000000}%
\pgfsetstrokecolor{currentstroke}%
\pgfsetdash{}{0pt}%
\pgfpathmoveto{\pgfqpoint{0.750000in}{0.500000in}}%
\pgfpathlineto{\pgfqpoint{5.400000in}{0.500000in}}%
\pgfusepath{stroke}%
\end{pgfscope}%
\begin{pgfscope}%
\pgfsetrectcap%
\pgfsetmiterjoin%
\pgfsetlinewidth{0.803000pt}%
\definecolor{currentstroke}{rgb}{0.000000,0.000000,0.000000}%
\pgfsetstrokecolor{currentstroke}%
\pgfsetdash{}{0pt}%
\pgfpathmoveto{\pgfqpoint{0.750000in}{3.520000in}}%
\pgfpathlineto{\pgfqpoint{5.400000in}{3.520000in}}%
\pgfusepath{stroke}%
\end{pgfscope}%
\end{pgfpicture}%
\makeatother%
\endgroup%

        }
        \caption{Histogram of the Delta Maximum Memory Consumption}
        \label{f:memHist}
    \end{figure}

    For completeness, we now validate our claim that the edges in the exploded supergraph correlate linearly with the memory consumption.
    We expect a linear correlation because the exploded supergraph is represented as a \code{HashMap} of edges and taints. 
    Consider \autoref{f:maxmemedges} where a clear, linear correlation is visible.
    We confirmed our claim.

    \begin{figure}[tbp]
        \begin{subfigure}[b]{\textwidth}
            \centering
            \begin{subfigure}[]{0.45\textwidth}
                \centering
                \resizebox{\columnwidth}{!}{
                    %% Creator: Matplotlib, PGF backend
%%
%% To include the figure in your LaTeX document, write
%%   \input{<filename>.pgf}
%%
%% Make sure the required packages are loaded in your preamble
%%   \usepackage{pgf}
%%
%% and, on pdftex
%%   \usepackage[utf8]{inputenc}\DeclareUnicodeCharacter{2212}{-}
%%
%% or, on luatex and xetex
%%   \usepackage{unicode-math}
%%
%% Figures using additional raster images can only be included by \input if
%% they are in the same directory as the main LaTeX file. For loading figures
%% from other directories you can use the `import` package
%%   \usepackage{import}
%%
%% and then include the figures with
%%   \import{<path to file>}{<filename>.pgf}
%%
%% Matplotlib used the following preamble
%%   \usepackage{amsmath}
%%   \usepackage{fontspec}
%%
\begingroup%
\makeatletter%
\begin{pgfpicture}%
\pgfpathrectangle{\pgfpointorigin}{\pgfqpoint{6.000000in}{4.000000in}}%
\pgfusepath{use as bounding box, clip}%
\begin{pgfscope}%
\pgfsetbuttcap%
\pgfsetmiterjoin%
\definecolor{currentfill}{rgb}{1.000000,1.000000,1.000000}%
\pgfsetfillcolor{currentfill}%
\pgfsetlinewidth{0.000000pt}%
\definecolor{currentstroke}{rgb}{1.000000,1.000000,1.000000}%
\pgfsetstrokecolor{currentstroke}%
\pgfsetdash{}{0pt}%
\pgfpathmoveto{\pgfqpoint{0.000000in}{0.000000in}}%
\pgfpathlineto{\pgfqpoint{6.000000in}{0.000000in}}%
\pgfpathlineto{\pgfqpoint{6.000000in}{4.000000in}}%
\pgfpathlineto{\pgfqpoint{0.000000in}{4.000000in}}%
\pgfpathclose%
\pgfusepath{fill}%
\end{pgfscope}%
\begin{pgfscope}%
\pgfsetbuttcap%
\pgfsetmiterjoin%
\definecolor{currentfill}{rgb}{1.000000,1.000000,1.000000}%
\pgfsetfillcolor{currentfill}%
\pgfsetlinewidth{0.000000pt}%
\definecolor{currentstroke}{rgb}{0.000000,0.000000,0.000000}%
\pgfsetstrokecolor{currentstroke}%
\pgfsetstrokeopacity{0.000000}%
\pgfsetdash{}{0pt}%
\pgfpathmoveto{\pgfqpoint{0.787074in}{0.548769in}}%
\pgfpathlineto{\pgfqpoint{5.761597in}{0.548769in}}%
\pgfpathlineto{\pgfqpoint{5.761597in}{3.651359in}}%
\pgfpathlineto{\pgfqpoint{0.787074in}{3.651359in}}%
\pgfpathclose%
\pgfusepath{fill}%
\end{pgfscope}%
\begin{pgfscope}%
\pgfpathrectangle{\pgfqpoint{0.787074in}{0.548769in}}{\pgfqpoint{4.974523in}{3.102590in}}%
\pgfusepath{clip}%
\pgfsetbuttcap%
\pgfsetroundjoin%
\definecolor{currentfill}{rgb}{0.121569,0.466667,0.705882}%
\pgfsetfillcolor{currentfill}%
\pgfsetlinewidth{1.003750pt}%
\definecolor{currentstroke}{rgb}{0.121569,0.466667,0.705882}%
\pgfsetstrokecolor{currentstroke}%
\pgfsetdash{}{0pt}%
\pgfpathmoveto{\pgfqpoint{1.005489in}{0.648198in}}%
\pgfpathcurveto{\pgfqpoint{1.016540in}{0.648198in}}{\pgfqpoint{1.027139in}{0.652588in}}{\pgfqpoint{1.034952in}{0.660402in}}%
\pgfpathcurveto{\pgfqpoint{1.042766in}{0.668215in}}{\pgfqpoint{1.047156in}{0.678814in}}{\pgfqpoint{1.047156in}{0.689865in}}%
\pgfpathcurveto{\pgfqpoint{1.047156in}{0.700915in}}{\pgfqpoint{1.042766in}{0.711514in}}{\pgfqpoint{1.034952in}{0.719327in}}%
\pgfpathcurveto{\pgfqpoint{1.027139in}{0.727141in}}{\pgfqpoint{1.016540in}{0.731531in}}{\pgfqpoint{1.005489in}{0.731531in}}%
\pgfpathcurveto{\pgfqpoint{0.994439in}{0.731531in}}{\pgfqpoint{0.983840in}{0.727141in}}{\pgfqpoint{0.976027in}{0.719327in}}%
\pgfpathcurveto{\pgfqpoint{0.968213in}{0.711514in}}{\pgfqpoint{0.963823in}{0.700915in}}{\pgfqpoint{0.963823in}{0.689865in}}%
\pgfpathcurveto{\pgfqpoint{0.963823in}{0.678814in}}{\pgfqpoint{0.968213in}{0.668215in}}{\pgfqpoint{0.976027in}{0.660402in}}%
\pgfpathcurveto{\pgfqpoint{0.983840in}{0.652588in}}{\pgfqpoint{0.994439in}{0.648198in}}{\pgfqpoint{1.005489in}{0.648198in}}%
\pgfpathclose%
\pgfusepath{stroke,fill}%
\end{pgfscope}%
\begin{pgfscope}%
\pgfpathrectangle{\pgfqpoint{0.787074in}{0.548769in}}{\pgfqpoint{4.974523in}{3.102590in}}%
\pgfusepath{clip}%
\pgfsetbuttcap%
\pgfsetroundjoin%
\definecolor{currentfill}{rgb}{0.121569,0.466667,0.705882}%
\pgfsetfillcolor{currentfill}%
\pgfsetlinewidth{1.003750pt}%
\definecolor{currentstroke}{rgb}{0.121569,0.466667,0.705882}%
\pgfsetstrokecolor{currentstroke}%
\pgfsetdash{}{0pt}%
\pgfpathmoveto{\pgfqpoint{2.192019in}{1.887135in}}%
\pgfpathcurveto{\pgfqpoint{2.203069in}{1.887135in}}{\pgfqpoint{2.213668in}{1.891526in}}{\pgfqpoint{2.221482in}{1.899339in}}%
\pgfpathcurveto{\pgfqpoint{2.229295in}{1.907153in}}{\pgfqpoint{2.233685in}{1.917752in}}{\pgfqpoint{2.233685in}{1.928802in}}%
\pgfpathcurveto{\pgfqpoint{2.233685in}{1.939852in}}{\pgfqpoint{2.229295in}{1.950451in}}{\pgfqpoint{2.221482in}{1.958265in}}%
\pgfpathcurveto{\pgfqpoint{2.213668in}{1.966079in}}{\pgfqpoint{2.203069in}{1.970469in}}{\pgfqpoint{2.192019in}{1.970469in}}%
\pgfpathcurveto{\pgfqpoint{2.180969in}{1.970469in}}{\pgfqpoint{2.170370in}{1.966079in}}{\pgfqpoint{2.162556in}{1.958265in}}%
\pgfpathcurveto{\pgfqpoint{2.154742in}{1.950451in}}{\pgfqpoint{2.150352in}{1.939852in}}{\pgfqpoint{2.150352in}{1.928802in}}%
\pgfpathcurveto{\pgfqpoint{2.150352in}{1.917752in}}{\pgfqpoint{2.154742in}{1.907153in}}{\pgfqpoint{2.162556in}{1.899339in}}%
\pgfpathcurveto{\pgfqpoint{2.170370in}{1.891526in}}{\pgfqpoint{2.180969in}{1.887135in}}{\pgfqpoint{2.192019in}{1.887135in}}%
\pgfpathclose%
\pgfusepath{stroke,fill}%
\end{pgfscope}%
\begin{pgfscope}%
\pgfpathrectangle{\pgfqpoint{0.787074in}{0.548769in}}{\pgfqpoint{4.974523in}{3.102590in}}%
\pgfusepath{clip}%
\pgfsetbuttcap%
\pgfsetroundjoin%
\definecolor{currentfill}{rgb}{1.000000,0.498039,0.054902}%
\pgfsetfillcolor{currentfill}%
\pgfsetlinewidth{1.003750pt}%
\definecolor{currentstroke}{rgb}{1.000000,0.498039,0.054902}%
\pgfsetstrokecolor{currentstroke}%
\pgfsetdash{}{0pt}%
\pgfpathmoveto{\pgfqpoint{4.163322in}{2.790510in}}%
\pgfpathcurveto{\pgfqpoint{4.174372in}{2.790510in}}{\pgfqpoint{4.184971in}{2.794900in}}{\pgfqpoint{4.192785in}{2.802714in}}%
\pgfpathcurveto{\pgfqpoint{4.200599in}{2.810527in}}{\pgfqpoint{4.204989in}{2.821126in}}{\pgfqpoint{4.204989in}{2.832176in}}%
\pgfpathcurveto{\pgfqpoint{4.204989in}{2.843227in}}{\pgfqpoint{4.200599in}{2.853826in}}{\pgfqpoint{4.192785in}{2.861639in}}%
\pgfpathcurveto{\pgfqpoint{4.184971in}{2.869453in}}{\pgfqpoint{4.174372in}{2.873843in}}{\pgfqpoint{4.163322in}{2.873843in}}%
\pgfpathcurveto{\pgfqpoint{4.152272in}{2.873843in}}{\pgfqpoint{4.141673in}{2.869453in}}{\pgfqpoint{4.133859in}{2.861639in}}%
\pgfpathcurveto{\pgfqpoint{4.126046in}{2.853826in}}{\pgfqpoint{4.121655in}{2.843227in}}{\pgfqpoint{4.121655in}{2.832176in}}%
\pgfpathcurveto{\pgfqpoint{4.121655in}{2.821126in}}{\pgfqpoint{4.126046in}{2.810527in}}{\pgfqpoint{4.133859in}{2.802714in}}%
\pgfpathcurveto{\pgfqpoint{4.141673in}{2.794900in}}{\pgfqpoint{4.152272in}{2.790510in}}{\pgfqpoint{4.163322in}{2.790510in}}%
\pgfpathclose%
\pgfusepath{stroke,fill}%
\end{pgfscope}%
\begin{pgfscope}%
\pgfpathrectangle{\pgfqpoint{0.787074in}{0.548769in}}{\pgfqpoint{4.974523in}{3.102590in}}%
\pgfusepath{clip}%
\pgfsetbuttcap%
\pgfsetroundjoin%
\definecolor{currentfill}{rgb}{0.121569,0.466667,0.705882}%
\pgfsetfillcolor{currentfill}%
\pgfsetlinewidth{1.003750pt}%
\definecolor{currentstroke}{rgb}{0.121569,0.466667,0.705882}%
\pgfsetstrokecolor{currentstroke}%
\pgfsetdash{}{0pt}%
\pgfpathmoveto{\pgfqpoint{3.757537in}{2.680271in}}%
\pgfpathcurveto{\pgfqpoint{3.768587in}{2.680271in}}{\pgfqpoint{3.779186in}{2.684661in}}{\pgfqpoint{3.787000in}{2.692475in}}%
\pgfpathcurveto{\pgfqpoint{3.794814in}{2.700289in}}{\pgfqpoint{3.799204in}{2.710888in}}{\pgfqpoint{3.799204in}{2.721938in}}%
\pgfpathcurveto{\pgfqpoint{3.799204in}{2.732988in}}{\pgfqpoint{3.794814in}{2.743587in}}{\pgfqpoint{3.787000in}{2.751401in}}%
\pgfpathcurveto{\pgfqpoint{3.779186in}{2.759214in}}{\pgfqpoint{3.768587in}{2.763604in}}{\pgfqpoint{3.757537in}{2.763604in}}%
\pgfpathcurveto{\pgfqpoint{3.746487in}{2.763604in}}{\pgfqpoint{3.735888in}{2.759214in}}{\pgfqpoint{3.728074in}{2.751401in}}%
\pgfpathcurveto{\pgfqpoint{3.720261in}{2.743587in}}{\pgfqpoint{3.715871in}{2.732988in}}{\pgfqpoint{3.715871in}{2.721938in}}%
\pgfpathcurveto{\pgfqpoint{3.715871in}{2.710888in}}{\pgfqpoint{3.720261in}{2.700289in}}{\pgfqpoint{3.728074in}{2.692475in}}%
\pgfpathcurveto{\pgfqpoint{3.735888in}{2.684661in}}{\pgfqpoint{3.746487in}{2.680271in}}{\pgfqpoint{3.757537in}{2.680271in}}%
\pgfpathclose%
\pgfusepath{stroke,fill}%
\end{pgfscope}%
\begin{pgfscope}%
\pgfpathrectangle{\pgfqpoint{0.787074in}{0.548769in}}{\pgfqpoint{4.974523in}{3.102590in}}%
\pgfusepath{clip}%
\pgfsetbuttcap%
\pgfsetroundjoin%
\definecolor{currentfill}{rgb}{1.000000,0.498039,0.054902}%
\pgfsetfillcolor{currentfill}%
\pgfsetlinewidth{1.003750pt}%
\definecolor{currentstroke}{rgb}{1.000000,0.498039,0.054902}%
\pgfsetstrokecolor{currentstroke}%
\pgfsetdash{}{0pt}%
\pgfpathmoveto{\pgfqpoint{4.587304in}{2.647277in}}%
\pgfpathcurveto{\pgfqpoint{4.598354in}{2.647277in}}{\pgfqpoint{4.608953in}{2.651668in}}{\pgfqpoint{4.616767in}{2.659481in}}%
\pgfpathcurveto{\pgfqpoint{4.624581in}{2.667295in}}{\pgfqpoint{4.628971in}{2.677894in}}{\pgfqpoint{4.628971in}{2.688944in}}%
\pgfpathcurveto{\pgfqpoint{4.628971in}{2.699994in}}{\pgfqpoint{4.624581in}{2.710593in}}{\pgfqpoint{4.616767in}{2.718407in}}%
\pgfpathcurveto{\pgfqpoint{4.608953in}{2.726221in}}{\pgfqpoint{4.598354in}{2.730611in}}{\pgfqpoint{4.587304in}{2.730611in}}%
\pgfpathcurveto{\pgfqpoint{4.576254in}{2.730611in}}{\pgfqpoint{4.565655in}{2.726221in}}{\pgfqpoint{4.557841in}{2.718407in}}%
\pgfpathcurveto{\pgfqpoint{4.550028in}{2.710593in}}{\pgfqpoint{4.545637in}{2.699994in}}{\pgfqpoint{4.545637in}{2.688944in}}%
\pgfpathcurveto{\pgfqpoint{4.545637in}{2.677894in}}{\pgfqpoint{4.550028in}{2.667295in}}{\pgfqpoint{4.557841in}{2.659481in}}%
\pgfpathcurveto{\pgfqpoint{4.565655in}{2.651668in}}{\pgfqpoint{4.576254in}{2.647277in}}{\pgfqpoint{4.587304in}{2.647277in}}%
\pgfpathclose%
\pgfusepath{stroke,fill}%
\end{pgfscope}%
\begin{pgfscope}%
\pgfpathrectangle{\pgfqpoint{0.787074in}{0.548769in}}{\pgfqpoint{4.974523in}{3.102590in}}%
\pgfusepath{clip}%
\pgfsetbuttcap%
\pgfsetroundjoin%
\definecolor{currentfill}{rgb}{0.121569,0.466667,0.705882}%
\pgfsetfillcolor{currentfill}%
\pgfsetlinewidth{1.003750pt}%
\definecolor{currentstroke}{rgb}{0.121569,0.466667,0.705882}%
\pgfsetstrokecolor{currentstroke}%
\pgfsetdash{}{0pt}%
\pgfpathmoveto{\pgfqpoint{2.983040in}{2.057905in}}%
\pgfpathcurveto{\pgfqpoint{2.994090in}{2.057905in}}{\pgfqpoint{3.004689in}{2.062295in}}{\pgfqpoint{3.012503in}{2.070109in}}%
\pgfpathcurveto{\pgfqpoint{3.020317in}{2.077922in}}{\pgfqpoint{3.024707in}{2.088521in}}{\pgfqpoint{3.024707in}{2.099572in}}%
\pgfpathcurveto{\pgfqpoint{3.024707in}{2.110622in}}{\pgfqpoint{3.020317in}{2.121221in}}{\pgfqpoint{3.012503in}{2.129034in}}%
\pgfpathcurveto{\pgfqpoint{3.004689in}{2.136848in}}{\pgfqpoint{2.994090in}{2.141238in}}{\pgfqpoint{2.983040in}{2.141238in}}%
\pgfpathcurveto{\pgfqpoint{2.971990in}{2.141238in}}{\pgfqpoint{2.961391in}{2.136848in}}{\pgfqpoint{2.953577in}{2.129034in}}%
\pgfpathcurveto{\pgfqpoint{2.945764in}{2.121221in}}{\pgfqpoint{2.941374in}{2.110622in}}{\pgfqpoint{2.941374in}{2.099572in}}%
\pgfpathcurveto{\pgfqpoint{2.941374in}{2.088521in}}{\pgfqpoint{2.945764in}{2.077922in}}{\pgfqpoint{2.953577in}{2.070109in}}%
\pgfpathcurveto{\pgfqpoint{2.961391in}{2.062295in}}{\pgfqpoint{2.971990in}{2.057905in}}{\pgfqpoint{2.983040in}{2.057905in}}%
\pgfpathclose%
\pgfusepath{stroke,fill}%
\end{pgfscope}%
\begin{pgfscope}%
\pgfpathrectangle{\pgfqpoint{0.787074in}{0.548769in}}{\pgfqpoint{4.974523in}{3.102590in}}%
\pgfusepath{clip}%
\pgfsetbuttcap%
\pgfsetroundjoin%
\definecolor{currentfill}{rgb}{0.121569,0.466667,0.705882}%
\pgfsetfillcolor{currentfill}%
\pgfsetlinewidth{1.003750pt}%
\definecolor{currentstroke}{rgb}{0.121569,0.466667,0.705882}%
\pgfsetstrokecolor{currentstroke}%
\pgfsetdash{}{0pt}%
\pgfpathmoveto{\pgfqpoint{3.575304in}{2.153691in}}%
\pgfpathcurveto{\pgfqpoint{3.586355in}{2.153691in}}{\pgfqpoint{3.596954in}{2.158081in}}{\pgfqpoint{3.604767in}{2.165895in}}%
\pgfpathcurveto{\pgfqpoint{3.612581in}{2.173708in}}{\pgfqpoint{3.616971in}{2.184307in}}{\pgfqpoint{3.616971in}{2.195357in}}%
\pgfpathcurveto{\pgfqpoint{3.616971in}{2.206408in}}{\pgfqpoint{3.612581in}{2.217007in}}{\pgfqpoint{3.604767in}{2.224820in}}%
\pgfpathcurveto{\pgfqpoint{3.596954in}{2.232634in}}{\pgfqpoint{3.586355in}{2.237024in}}{\pgfqpoint{3.575304in}{2.237024in}}%
\pgfpathcurveto{\pgfqpoint{3.564254in}{2.237024in}}{\pgfqpoint{3.553655in}{2.232634in}}{\pgfqpoint{3.545842in}{2.224820in}}%
\pgfpathcurveto{\pgfqpoint{3.538028in}{2.217007in}}{\pgfqpoint{3.533638in}{2.206408in}}{\pgfqpoint{3.533638in}{2.195357in}}%
\pgfpathcurveto{\pgfqpoint{3.533638in}{2.184307in}}{\pgfqpoint{3.538028in}{2.173708in}}{\pgfqpoint{3.545842in}{2.165895in}}%
\pgfpathcurveto{\pgfqpoint{3.553655in}{2.158081in}}{\pgfqpoint{3.564254in}{2.153691in}}{\pgfqpoint{3.575304in}{2.153691in}}%
\pgfpathclose%
\pgfusepath{stroke,fill}%
\end{pgfscope}%
\begin{pgfscope}%
\pgfpathrectangle{\pgfqpoint{0.787074in}{0.548769in}}{\pgfqpoint{4.974523in}{3.102590in}}%
\pgfusepath{clip}%
\pgfsetbuttcap%
\pgfsetroundjoin%
\definecolor{currentfill}{rgb}{1.000000,0.498039,0.054902}%
\pgfsetfillcolor{currentfill}%
\pgfsetlinewidth{1.003750pt}%
\definecolor{currentstroke}{rgb}{1.000000,0.498039,0.054902}%
\pgfsetstrokecolor{currentstroke}%
\pgfsetdash{}{0pt}%
\pgfpathmoveto{\pgfqpoint{4.260820in}{2.714270in}}%
\pgfpathcurveto{\pgfqpoint{4.271870in}{2.714270in}}{\pgfqpoint{4.282469in}{2.718660in}}{\pgfqpoint{4.290283in}{2.726474in}}%
\pgfpathcurveto{\pgfqpoint{4.298096in}{2.734288in}}{\pgfqpoint{4.302487in}{2.744887in}}{\pgfqpoint{4.302487in}{2.755937in}}%
\pgfpathcurveto{\pgfqpoint{4.302487in}{2.766987in}}{\pgfqpoint{4.298096in}{2.777586in}}{\pgfqpoint{4.290283in}{2.785399in}}%
\pgfpathcurveto{\pgfqpoint{4.282469in}{2.793213in}}{\pgfqpoint{4.271870in}{2.797603in}}{\pgfqpoint{4.260820in}{2.797603in}}%
\pgfpathcurveto{\pgfqpoint{4.249770in}{2.797603in}}{\pgfqpoint{4.239171in}{2.793213in}}{\pgfqpoint{4.231357in}{2.785399in}}%
\pgfpathcurveto{\pgfqpoint{4.223544in}{2.777586in}}{\pgfqpoint{4.219153in}{2.766987in}}{\pgfqpoint{4.219153in}{2.755937in}}%
\pgfpathcurveto{\pgfqpoint{4.219153in}{2.744887in}}{\pgfqpoint{4.223544in}{2.734288in}}{\pgfqpoint{4.231357in}{2.726474in}}%
\pgfpathcurveto{\pgfqpoint{4.239171in}{2.718660in}}{\pgfqpoint{4.249770in}{2.714270in}}{\pgfqpoint{4.260820in}{2.714270in}}%
\pgfpathclose%
\pgfusepath{stroke,fill}%
\end{pgfscope}%
\begin{pgfscope}%
\pgfpathrectangle{\pgfqpoint{0.787074in}{0.548769in}}{\pgfqpoint{4.974523in}{3.102590in}}%
\pgfusepath{clip}%
\pgfsetbuttcap%
\pgfsetroundjoin%
\definecolor{currentfill}{rgb}{1.000000,0.498039,0.054902}%
\pgfsetfillcolor{currentfill}%
\pgfsetlinewidth{1.003750pt}%
\definecolor{currentstroke}{rgb}{1.000000,0.498039,0.054902}%
\pgfsetstrokecolor{currentstroke}%
\pgfsetdash{}{0pt}%
\pgfpathmoveto{\pgfqpoint{4.221022in}{2.861842in}}%
\pgfpathcurveto{\pgfqpoint{4.232072in}{2.861842in}}{\pgfqpoint{4.242671in}{2.866233in}}{\pgfqpoint{4.250485in}{2.874046in}}%
\pgfpathcurveto{\pgfqpoint{4.258299in}{2.881860in}}{\pgfqpoint{4.262689in}{2.892459in}}{\pgfqpoint{4.262689in}{2.903509in}}%
\pgfpathcurveto{\pgfqpoint{4.262689in}{2.914559in}}{\pgfqpoint{4.258299in}{2.925158in}}{\pgfqpoint{4.250485in}{2.932972in}}%
\pgfpathcurveto{\pgfqpoint{4.242671in}{2.940786in}}{\pgfqpoint{4.232072in}{2.945176in}}{\pgfqpoint{4.221022in}{2.945176in}}%
\pgfpathcurveto{\pgfqpoint{4.209972in}{2.945176in}}{\pgfqpoint{4.199373in}{2.940786in}}{\pgfqpoint{4.191559in}{2.932972in}}%
\pgfpathcurveto{\pgfqpoint{4.183746in}{2.925158in}}{\pgfqpoint{4.179355in}{2.914559in}}{\pgfqpoint{4.179355in}{2.903509in}}%
\pgfpathcurveto{\pgfqpoint{4.179355in}{2.892459in}}{\pgfqpoint{4.183746in}{2.881860in}}{\pgfqpoint{4.191559in}{2.874046in}}%
\pgfpathcurveto{\pgfqpoint{4.199373in}{2.866233in}}{\pgfqpoint{4.209972in}{2.861842in}}{\pgfqpoint{4.221022in}{2.861842in}}%
\pgfpathclose%
\pgfusepath{stroke,fill}%
\end{pgfscope}%
\begin{pgfscope}%
\pgfpathrectangle{\pgfqpoint{0.787074in}{0.548769in}}{\pgfqpoint{4.974523in}{3.102590in}}%
\pgfusepath{clip}%
\pgfsetbuttcap%
\pgfsetroundjoin%
\definecolor{currentfill}{rgb}{0.121569,0.466667,0.705882}%
\pgfsetfillcolor{currentfill}%
\pgfsetlinewidth{1.003750pt}%
\definecolor{currentstroke}{rgb}{0.121569,0.466667,0.705882}%
\pgfsetstrokecolor{currentstroke}%
\pgfsetdash{}{0pt}%
\pgfpathmoveto{\pgfqpoint{1.039797in}{0.668329in}}%
\pgfpathcurveto{\pgfqpoint{1.050847in}{0.668329in}}{\pgfqpoint{1.061446in}{0.672720in}}{\pgfqpoint{1.069259in}{0.680533in}}%
\pgfpathcurveto{\pgfqpoint{1.077073in}{0.688347in}}{\pgfqpoint{1.081463in}{0.698946in}}{\pgfqpoint{1.081463in}{0.709996in}}%
\pgfpathcurveto{\pgfqpoint{1.081463in}{0.721046in}}{\pgfqpoint{1.077073in}{0.731645in}}{\pgfqpoint{1.069259in}{0.739459in}}%
\pgfpathcurveto{\pgfqpoint{1.061446in}{0.747272in}}{\pgfqpoint{1.050847in}{0.751663in}}{\pgfqpoint{1.039797in}{0.751663in}}%
\pgfpathcurveto{\pgfqpoint{1.028747in}{0.751663in}}{\pgfqpoint{1.018147in}{0.747272in}}{\pgfqpoint{1.010334in}{0.739459in}}%
\pgfpathcurveto{\pgfqpoint{1.002520in}{0.731645in}}{\pgfqpoint{0.998130in}{0.721046in}}{\pgfqpoint{0.998130in}{0.709996in}}%
\pgfpathcurveto{\pgfqpoint{0.998130in}{0.698946in}}{\pgfqpoint{1.002520in}{0.688347in}}{\pgfqpoint{1.010334in}{0.680533in}}%
\pgfpathcurveto{\pgfqpoint{1.018147in}{0.672720in}}{\pgfqpoint{1.028747in}{0.668329in}}{\pgfqpoint{1.039797in}{0.668329in}}%
\pgfpathclose%
\pgfusepath{stroke,fill}%
\end{pgfscope}%
\begin{pgfscope}%
\pgfpathrectangle{\pgfqpoint{0.787074in}{0.548769in}}{\pgfqpoint{4.974523in}{3.102590in}}%
\pgfusepath{clip}%
\pgfsetbuttcap%
\pgfsetroundjoin%
\definecolor{currentfill}{rgb}{1.000000,0.498039,0.054902}%
\pgfsetfillcolor{currentfill}%
\pgfsetlinewidth{1.003750pt}%
\definecolor{currentstroke}{rgb}{1.000000,0.498039,0.054902}%
\pgfsetstrokecolor{currentstroke}%
\pgfsetdash{}{0pt}%
\pgfpathmoveto{\pgfqpoint{4.441858in}{3.008532in}}%
\pgfpathcurveto{\pgfqpoint{4.452908in}{3.008532in}}{\pgfqpoint{4.463507in}{3.012923in}}{\pgfqpoint{4.471320in}{3.020736in}}%
\pgfpathcurveto{\pgfqpoint{4.479134in}{3.028550in}}{\pgfqpoint{4.483524in}{3.039149in}}{\pgfqpoint{4.483524in}{3.050199in}}%
\pgfpathcurveto{\pgfqpoint{4.483524in}{3.061249in}}{\pgfqpoint{4.479134in}{3.071848in}}{\pgfqpoint{4.471320in}{3.079662in}}%
\pgfpathcurveto{\pgfqpoint{4.463507in}{3.087476in}}{\pgfqpoint{4.452908in}{3.091866in}}{\pgfqpoint{4.441858in}{3.091866in}}%
\pgfpathcurveto{\pgfqpoint{4.430807in}{3.091866in}}{\pgfqpoint{4.420208in}{3.087476in}}{\pgfqpoint{4.412395in}{3.079662in}}%
\pgfpathcurveto{\pgfqpoint{4.404581in}{3.071848in}}{\pgfqpoint{4.400191in}{3.061249in}}{\pgfqpoint{4.400191in}{3.050199in}}%
\pgfpathcurveto{\pgfqpoint{4.400191in}{3.039149in}}{\pgfqpoint{4.404581in}{3.028550in}}{\pgfqpoint{4.412395in}{3.020736in}}%
\pgfpathcurveto{\pgfqpoint{4.420208in}{3.012923in}}{\pgfqpoint{4.430807in}{3.008532in}}{\pgfqpoint{4.441858in}{3.008532in}}%
\pgfpathclose%
\pgfusepath{stroke,fill}%
\end{pgfscope}%
\begin{pgfscope}%
\pgfpathrectangle{\pgfqpoint{0.787074in}{0.548769in}}{\pgfqpoint{4.974523in}{3.102590in}}%
\pgfusepath{clip}%
\pgfsetbuttcap%
\pgfsetroundjoin%
\definecolor{currentfill}{rgb}{1.000000,0.498039,0.054902}%
\pgfsetfillcolor{currentfill}%
\pgfsetlinewidth{1.003750pt}%
\definecolor{currentstroke}{rgb}{1.000000,0.498039,0.054902}%
\pgfsetstrokecolor{currentstroke}%
\pgfsetdash{}{0pt}%
\pgfpathmoveto{\pgfqpoint{3.956999in}{2.645063in}}%
\pgfpathcurveto{\pgfqpoint{3.968049in}{2.645063in}}{\pgfqpoint{3.978648in}{2.649453in}}{\pgfqpoint{3.986462in}{2.657266in}}%
\pgfpathcurveto{\pgfqpoint{3.994276in}{2.665080in}}{\pgfqpoint{3.998666in}{2.675679in}}{\pgfqpoint{3.998666in}{2.686729in}}%
\pgfpathcurveto{\pgfqpoint{3.998666in}{2.697779in}}{\pgfqpoint{3.994276in}{2.708378in}}{\pgfqpoint{3.986462in}{2.716192in}}%
\pgfpathcurveto{\pgfqpoint{3.978648in}{2.724006in}}{\pgfqpoint{3.968049in}{2.728396in}}{\pgfqpoint{3.956999in}{2.728396in}}%
\pgfpathcurveto{\pgfqpoint{3.945949in}{2.728396in}}{\pgfqpoint{3.935350in}{2.724006in}}{\pgfqpoint{3.927536in}{2.716192in}}%
\pgfpathcurveto{\pgfqpoint{3.919723in}{2.708378in}}{\pgfqpoint{3.915332in}{2.697779in}}{\pgfqpoint{3.915332in}{2.686729in}}%
\pgfpathcurveto{\pgfqpoint{3.915332in}{2.675679in}}{\pgfqpoint{3.919723in}{2.665080in}}{\pgfqpoint{3.927536in}{2.657266in}}%
\pgfpathcurveto{\pgfqpoint{3.935350in}{2.649453in}}{\pgfqpoint{3.945949in}{2.645063in}}{\pgfqpoint{3.956999in}{2.645063in}}%
\pgfpathclose%
\pgfusepath{stroke,fill}%
\end{pgfscope}%
\begin{pgfscope}%
\pgfpathrectangle{\pgfqpoint{0.787074in}{0.548769in}}{\pgfqpoint{4.974523in}{3.102590in}}%
\pgfusepath{clip}%
\pgfsetbuttcap%
\pgfsetroundjoin%
\definecolor{currentfill}{rgb}{1.000000,0.498039,0.054902}%
\pgfsetfillcolor{currentfill}%
\pgfsetlinewidth{1.003750pt}%
\definecolor{currentstroke}{rgb}{1.000000,0.498039,0.054902}%
\pgfsetstrokecolor{currentstroke}%
\pgfsetdash{}{0pt}%
\pgfpathmoveto{\pgfqpoint{3.975961in}{3.140866in}}%
\pgfpathcurveto{\pgfqpoint{3.987011in}{3.140866in}}{\pgfqpoint{3.997610in}{3.145257in}}{\pgfqpoint{4.005424in}{3.153070in}}%
\pgfpathcurveto{\pgfqpoint{4.013238in}{3.160884in}}{\pgfqpoint{4.017628in}{3.171483in}}{\pgfqpoint{4.017628in}{3.182533in}}%
\pgfpathcurveto{\pgfqpoint{4.017628in}{3.193583in}}{\pgfqpoint{4.013238in}{3.204182in}}{\pgfqpoint{4.005424in}{3.211996in}}%
\pgfpathcurveto{\pgfqpoint{3.997610in}{3.219809in}}{\pgfqpoint{3.987011in}{3.224200in}}{\pgfqpoint{3.975961in}{3.224200in}}%
\pgfpathcurveto{\pgfqpoint{3.964911in}{3.224200in}}{\pgfqpoint{3.954312in}{3.219809in}}{\pgfqpoint{3.946499in}{3.211996in}}%
\pgfpathcurveto{\pgfqpoint{3.938685in}{3.204182in}}{\pgfqpoint{3.934295in}{3.193583in}}{\pgfqpoint{3.934295in}{3.182533in}}%
\pgfpathcurveto{\pgfqpoint{3.934295in}{3.171483in}}{\pgfqpoint{3.938685in}{3.160884in}}{\pgfqpoint{3.946499in}{3.153070in}}%
\pgfpathcurveto{\pgfqpoint{3.954312in}{3.145257in}}{\pgfqpoint{3.964911in}{3.140866in}}{\pgfqpoint{3.975961in}{3.140866in}}%
\pgfpathclose%
\pgfusepath{stroke,fill}%
\end{pgfscope}%
\begin{pgfscope}%
\pgfpathrectangle{\pgfqpoint{0.787074in}{0.548769in}}{\pgfqpoint{4.974523in}{3.102590in}}%
\pgfusepath{clip}%
\pgfsetbuttcap%
\pgfsetroundjoin%
\definecolor{currentfill}{rgb}{1.000000,0.498039,0.054902}%
\pgfsetfillcolor{currentfill}%
\pgfsetlinewidth{1.003750pt}%
\definecolor{currentstroke}{rgb}{1.000000,0.498039,0.054902}%
\pgfsetstrokecolor{currentstroke}%
\pgfsetdash{}{0pt}%
\pgfpathmoveto{\pgfqpoint{4.212515in}{2.974499in}}%
\pgfpathcurveto{\pgfqpoint{4.223565in}{2.974499in}}{\pgfqpoint{4.234164in}{2.978889in}}{\pgfqpoint{4.241978in}{2.986703in}}%
\pgfpathcurveto{\pgfqpoint{4.249791in}{2.994517in}}{\pgfqpoint{4.254182in}{3.005116in}}{\pgfqpoint{4.254182in}{3.016166in}}%
\pgfpathcurveto{\pgfqpoint{4.254182in}{3.027216in}}{\pgfqpoint{4.249791in}{3.037815in}}{\pgfqpoint{4.241978in}{3.045629in}}%
\pgfpathcurveto{\pgfqpoint{4.234164in}{3.053442in}}{\pgfqpoint{4.223565in}{3.057833in}}{\pgfqpoint{4.212515in}{3.057833in}}%
\pgfpathcurveto{\pgfqpoint{4.201465in}{3.057833in}}{\pgfqpoint{4.190866in}{3.053442in}}{\pgfqpoint{4.183052in}{3.045629in}}%
\pgfpathcurveto{\pgfqpoint{4.175238in}{3.037815in}}{\pgfqpoint{4.170848in}{3.027216in}}{\pgfqpoint{4.170848in}{3.016166in}}%
\pgfpathcurveto{\pgfqpoint{4.170848in}{3.005116in}}{\pgfqpoint{4.175238in}{2.994517in}}{\pgfqpoint{4.183052in}{2.986703in}}%
\pgfpathcurveto{\pgfqpoint{4.190866in}{2.978889in}}{\pgfqpoint{4.201465in}{2.974499in}}{\pgfqpoint{4.212515in}{2.974499in}}%
\pgfpathclose%
\pgfusepath{stroke,fill}%
\end{pgfscope}%
\begin{pgfscope}%
\pgfpathrectangle{\pgfqpoint{0.787074in}{0.548769in}}{\pgfqpoint{4.974523in}{3.102590in}}%
\pgfusepath{clip}%
\pgfsetbuttcap%
\pgfsetroundjoin%
\definecolor{currentfill}{rgb}{1.000000,0.498039,0.054902}%
\pgfsetfillcolor{currentfill}%
\pgfsetlinewidth{1.003750pt}%
\definecolor{currentstroke}{rgb}{1.000000,0.498039,0.054902}%
\pgfsetstrokecolor{currentstroke}%
\pgfsetdash{}{0pt}%
\pgfpathmoveto{\pgfqpoint{1.990203in}{1.902652in}}%
\pgfpathcurveto{\pgfqpoint{2.001254in}{1.902652in}}{\pgfqpoint{2.011853in}{1.907042in}}{\pgfqpoint{2.019666in}{1.914856in}}%
\pgfpathcurveto{\pgfqpoint{2.027480in}{1.922669in}}{\pgfqpoint{2.031870in}{1.933268in}}{\pgfqpoint{2.031870in}{1.944318in}}%
\pgfpathcurveto{\pgfqpoint{2.031870in}{1.955368in}}{\pgfqpoint{2.027480in}{1.965967in}}{\pgfqpoint{2.019666in}{1.973781in}}%
\pgfpathcurveto{\pgfqpoint{2.011853in}{1.981595in}}{\pgfqpoint{2.001254in}{1.985985in}}{\pgfqpoint{1.990203in}{1.985985in}}%
\pgfpathcurveto{\pgfqpoint{1.979153in}{1.985985in}}{\pgfqpoint{1.968554in}{1.981595in}}{\pgfqpoint{1.960741in}{1.973781in}}%
\pgfpathcurveto{\pgfqpoint{1.952927in}{1.965967in}}{\pgfqpoint{1.948537in}{1.955368in}}{\pgfqpoint{1.948537in}{1.944318in}}%
\pgfpathcurveto{\pgfqpoint{1.948537in}{1.933268in}}{\pgfqpoint{1.952927in}{1.922669in}}{\pgfqpoint{1.960741in}{1.914856in}}%
\pgfpathcurveto{\pgfqpoint{1.968554in}{1.907042in}}{\pgfqpoint{1.979153in}{1.902652in}}{\pgfqpoint{1.990203in}{1.902652in}}%
\pgfpathclose%
\pgfusepath{stroke,fill}%
\end{pgfscope}%
\begin{pgfscope}%
\pgfpathrectangle{\pgfqpoint{0.787074in}{0.548769in}}{\pgfqpoint{4.974523in}{3.102590in}}%
\pgfusepath{clip}%
\pgfsetbuttcap%
\pgfsetroundjoin%
\definecolor{currentfill}{rgb}{1.000000,0.498039,0.054902}%
\pgfsetfillcolor{currentfill}%
\pgfsetlinewidth{1.003750pt}%
\definecolor{currentstroke}{rgb}{1.000000,0.498039,0.054902}%
\pgfsetstrokecolor{currentstroke}%
\pgfsetdash{}{0pt}%
\pgfpathmoveto{\pgfqpoint{2.872887in}{1.942448in}}%
\pgfpathcurveto{\pgfqpoint{2.883937in}{1.942448in}}{\pgfqpoint{2.894536in}{1.946838in}}{\pgfqpoint{2.902349in}{1.954652in}}%
\pgfpathcurveto{\pgfqpoint{2.910163in}{1.962466in}}{\pgfqpoint{2.914553in}{1.973065in}}{\pgfqpoint{2.914553in}{1.984115in}}%
\pgfpathcurveto{\pgfqpoint{2.914553in}{1.995165in}}{\pgfqpoint{2.910163in}{2.005764in}}{\pgfqpoint{2.902349in}{2.013578in}}%
\pgfpathcurveto{\pgfqpoint{2.894536in}{2.021391in}}{\pgfqpoint{2.883937in}{2.025781in}}{\pgfqpoint{2.872887in}{2.025781in}}%
\pgfpathcurveto{\pgfqpoint{2.861836in}{2.025781in}}{\pgfqpoint{2.851237in}{2.021391in}}{\pgfqpoint{2.843424in}{2.013578in}}%
\pgfpathcurveto{\pgfqpoint{2.835610in}{2.005764in}}{\pgfqpoint{2.831220in}{1.995165in}}{\pgfqpoint{2.831220in}{1.984115in}}%
\pgfpathcurveto{\pgfqpoint{2.831220in}{1.973065in}}{\pgfqpoint{2.835610in}{1.962466in}}{\pgfqpoint{2.843424in}{1.954652in}}%
\pgfpathcurveto{\pgfqpoint{2.851237in}{1.946838in}}{\pgfqpoint{2.861836in}{1.942448in}}{\pgfqpoint{2.872887in}{1.942448in}}%
\pgfpathclose%
\pgfusepath{stroke,fill}%
\end{pgfscope}%
\begin{pgfscope}%
\pgfpathrectangle{\pgfqpoint{0.787074in}{0.548769in}}{\pgfqpoint{4.974523in}{3.102590in}}%
\pgfusepath{clip}%
\pgfsetbuttcap%
\pgfsetroundjoin%
\definecolor{currentfill}{rgb}{1.000000,0.498039,0.054902}%
\pgfsetfillcolor{currentfill}%
\pgfsetlinewidth{1.003750pt}%
\definecolor{currentstroke}{rgb}{1.000000,0.498039,0.054902}%
\pgfsetstrokecolor{currentstroke}%
\pgfsetdash{}{0pt}%
\pgfpathmoveto{\pgfqpoint{4.387359in}{2.924069in}}%
\pgfpathcurveto{\pgfqpoint{4.398409in}{2.924069in}}{\pgfqpoint{4.409008in}{2.928459in}}{\pgfqpoint{4.416822in}{2.936273in}}%
\pgfpathcurveto{\pgfqpoint{4.424635in}{2.944086in}}{\pgfqpoint{4.429026in}{2.954685in}}{\pgfqpoint{4.429026in}{2.965736in}}%
\pgfpathcurveto{\pgfqpoint{4.429026in}{2.976786in}}{\pgfqpoint{4.424635in}{2.987385in}}{\pgfqpoint{4.416822in}{2.995198in}}%
\pgfpathcurveto{\pgfqpoint{4.409008in}{3.003012in}}{\pgfqpoint{4.398409in}{3.007402in}}{\pgfqpoint{4.387359in}{3.007402in}}%
\pgfpathcurveto{\pgfqpoint{4.376309in}{3.007402in}}{\pgfqpoint{4.365710in}{3.003012in}}{\pgfqpoint{4.357896in}{2.995198in}}%
\pgfpathcurveto{\pgfqpoint{4.350082in}{2.987385in}}{\pgfqpoint{4.345692in}{2.976786in}}{\pgfqpoint{4.345692in}{2.965736in}}%
\pgfpathcurveto{\pgfqpoint{4.345692in}{2.954685in}}{\pgfqpoint{4.350082in}{2.944086in}}{\pgfqpoint{4.357896in}{2.936273in}}%
\pgfpathcurveto{\pgfqpoint{4.365710in}{2.928459in}}{\pgfqpoint{4.376309in}{2.924069in}}{\pgfqpoint{4.387359in}{2.924069in}}%
\pgfpathclose%
\pgfusepath{stroke,fill}%
\end{pgfscope}%
\begin{pgfscope}%
\pgfpathrectangle{\pgfqpoint{0.787074in}{0.548769in}}{\pgfqpoint{4.974523in}{3.102590in}}%
\pgfusepath{clip}%
\pgfsetbuttcap%
\pgfsetroundjoin%
\definecolor{currentfill}{rgb}{0.121569,0.466667,0.705882}%
\pgfsetfillcolor{currentfill}%
\pgfsetlinewidth{1.003750pt}%
\definecolor{currentstroke}{rgb}{0.121569,0.466667,0.705882}%
\pgfsetstrokecolor{currentstroke}%
\pgfsetdash{}{0pt}%
\pgfpathmoveto{\pgfqpoint{4.254895in}{2.510157in}}%
\pgfpathcurveto{\pgfqpoint{4.265946in}{2.510157in}}{\pgfqpoint{4.276545in}{2.514547in}}{\pgfqpoint{4.284358in}{2.522361in}}%
\pgfpathcurveto{\pgfqpoint{4.292172in}{2.530174in}}{\pgfqpoint{4.296562in}{2.540773in}}{\pgfqpoint{4.296562in}{2.551824in}}%
\pgfpathcurveto{\pgfqpoint{4.296562in}{2.562874in}}{\pgfqpoint{4.292172in}{2.573473in}}{\pgfqpoint{4.284358in}{2.581286in}}%
\pgfpathcurveto{\pgfqpoint{4.276545in}{2.589100in}}{\pgfqpoint{4.265946in}{2.593490in}}{\pgfqpoint{4.254895in}{2.593490in}}%
\pgfpathcurveto{\pgfqpoint{4.243845in}{2.593490in}}{\pgfqpoint{4.233246in}{2.589100in}}{\pgfqpoint{4.225433in}{2.581286in}}%
\pgfpathcurveto{\pgfqpoint{4.217619in}{2.573473in}}{\pgfqpoint{4.213229in}{2.562874in}}{\pgfqpoint{4.213229in}{2.551824in}}%
\pgfpathcurveto{\pgfqpoint{4.213229in}{2.540773in}}{\pgfqpoint{4.217619in}{2.530174in}}{\pgfqpoint{4.225433in}{2.522361in}}%
\pgfpathcurveto{\pgfqpoint{4.233246in}{2.514547in}}{\pgfqpoint{4.243845in}{2.510157in}}{\pgfqpoint{4.254895in}{2.510157in}}%
\pgfpathclose%
\pgfusepath{stroke,fill}%
\end{pgfscope}%
\begin{pgfscope}%
\pgfpathrectangle{\pgfqpoint{0.787074in}{0.548769in}}{\pgfqpoint{4.974523in}{3.102590in}}%
\pgfusepath{clip}%
\pgfsetbuttcap%
\pgfsetroundjoin%
\definecolor{currentfill}{rgb}{0.121569,0.466667,0.705882}%
\pgfsetfillcolor{currentfill}%
\pgfsetlinewidth{1.003750pt}%
\definecolor{currentstroke}{rgb}{0.121569,0.466667,0.705882}%
\pgfsetstrokecolor{currentstroke}%
\pgfsetdash{}{0pt}%
\pgfpathmoveto{\pgfqpoint{1.615646in}{1.380136in}}%
\pgfpathcurveto{\pgfqpoint{1.626696in}{1.380136in}}{\pgfqpoint{1.637295in}{1.384527in}}{\pgfqpoint{1.645108in}{1.392340in}}%
\pgfpathcurveto{\pgfqpoint{1.652922in}{1.400154in}}{\pgfqpoint{1.657312in}{1.410753in}}{\pgfqpoint{1.657312in}{1.421803in}}%
\pgfpathcurveto{\pgfqpoint{1.657312in}{1.432853in}}{\pgfqpoint{1.652922in}{1.443452in}}{\pgfqpoint{1.645108in}{1.451266in}}%
\pgfpathcurveto{\pgfqpoint{1.637295in}{1.459079in}}{\pgfqpoint{1.626696in}{1.463470in}}{\pgfqpoint{1.615646in}{1.463470in}}%
\pgfpathcurveto{\pgfqpoint{1.604595in}{1.463470in}}{\pgfqpoint{1.593996in}{1.459079in}}{\pgfqpoint{1.586183in}{1.451266in}}%
\pgfpathcurveto{\pgfqpoint{1.578369in}{1.443452in}}{\pgfqpoint{1.573979in}{1.432853in}}{\pgfqpoint{1.573979in}{1.421803in}}%
\pgfpathcurveto{\pgfqpoint{1.573979in}{1.410753in}}{\pgfqpoint{1.578369in}{1.400154in}}{\pgfqpoint{1.586183in}{1.392340in}}%
\pgfpathcurveto{\pgfqpoint{1.593996in}{1.384527in}}{\pgfqpoint{1.604595in}{1.380136in}}{\pgfqpoint{1.615646in}{1.380136in}}%
\pgfpathclose%
\pgfusepath{stroke,fill}%
\end{pgfscope}%
\begin{pgfscope}%
\pgfpathrectangle{\pgfqpoint{0.787074in}{0.548769in}}{\pgfqpoint{4.974523in}{3.102590in}}%
\pgfusepath{clip}%
\pgfsetbuttcap%
\pgfsetroundjoin%
\definecolor{currentfill}{rgb}{1.000000,0.498039,0.054902}%
\pgfsetfillcolor{currentfill}%
\pgfsetlinewidth{1.003750pt}%
\definecolor{currentstroke}{rgb}{1.000000,0.498039,0.054902}%
\pgfsetstrokecolor{currentstroke}%
\pgfsetdash{}{0pt}%
\pgfpathmoveto{\pgfqpoint{4.121686in}{2.854158in}}%
\pgfpathcurveto{\pgfqpoint{4.132736in}{2.854158in}}{\pgfqpoint{4.143335in}{2.858548in}}{\pgfqpoint{4.151149in}{2.866362in}}%
\pgfpathcurveto{\pgfqpoint{4.158963in}{2.874175in}}{\pgfqpoint{4.163353in}{2.884774in}}{\pgfqpoint{4.163353in}{2.895825in}}%
\pgfpathcurveto{\pgfqpoint{4.163353in}{2.906875in}}{\pgfqpoint{4.158963in}{2.917474in}}{\pgfqpoint{4.151149in}{2.925287in}}%
\pgfpathcurveto{\pgfqpoint{4.143335in}{2.933101in}}{\pgfqpoint{4.132736in}{2.937491in}}{\pgfqpoint{4.121686in}{2.937491in}}%
\pgfpathcurveto{\pgfqpoint{4.110636in}{2.937491in}}{\pgfqpoint{4.100037in}{2.933101in}}{\pgfqpoint{4.092223in}{2.925287in}}%
\pgfpathcurveto{\pgfqpoint{4.084410in}{2.917474in}}{\pgfqpoint{4.080020in}{2.906875in}}{\pgfqpoint{4.080020in}{2.895825in}}%
\pgfpathcurveto{\pgfqpoint{4.080020in}{2.884774in}}{\pgfqpoint{4.084410in}{2.874175in}}{\pgfqpoint{4.092223in}{2.866362in}}%
\pgfpathcurveto{\pgfqpoint{4.100037in}{2.858548in}}{\pgfqpoint{4.110636in}{2.854158in}}{\pgfqpoint{4.121686in}{2.854158in}}%
\pgfpathclose%
\pgfusepath{stroke,fill}%
\end{pgfscope}%
\begin{pgfscope}%
\pgfpathrectangle{\pgfqpoint{0.787074in}{0.548769in}}{\pgfqpoint{4.974523in}{3.102590in}}%
\pgfusepath{clip}%
\pgfsetbuttcap%
\pgfsetroundjoin%
\definecolor{currentfill}{rgb}{0.121569,0.466667,0.705882}%
\pgfsetfillcolor{currentfill}%
\pgfsetlinewidth{1.003750pt}%
\definecolor{currentstroke}{rgb}{0.121569,0.466667,0.705882}%
\pgfsetstrokecolor{currentstroke}%
\pgfsetdash{}{0pt}%
\pgfpathmoveto{\pgfqpoint{1.005747in}{0.648318in}}%
\pgfpathcurveto{\pgfqpoint{1.016797in}{0.648318in}}{\pgfqpoint{1.027396in}{0.652709in}}{\pgfqpoint{1.035210in}{0.660522in}}%
\pgfpathcurveto{\pgfqpoint{1.043023in}{0.668336in}}{\pgfqpoint{1.047414in}{0.678935in}}{\pgfqpoint{1.047414in}{0.689985in}}%
\pgfpathcurveto{\pgfqpoint{1.047414in}{0.701035in}}{\pgfqpoint{1.043023in}{0.711634in}}{\pgfqpoint{1.035210in}{0.719448in}}%
\pgfpathcurveto{\pgfqpoint{1.027396in}{0.727262in}}{\pgfqpoint{1.016797in}{0.731652in}}{\pgfqpoint{1.005747in}{0.731652in}}%
\pgfpathcurveto{\pgfqpoint{0.994697in}{0.731652in}}{\pgfqpoint{0.984098in}{0.727262in}}{\pgfqpoint{0.976284in}{0.719448in}}%
\pgfpathcurveto{\pgfqpoint{0.968470in}{0.711634in}}{\pgfqpoint{0.964080in}{0.701035in}}{\pgfqpoint{0.964080in}{0.689985in}}%
\pgfpathcurveto{\pgfqpoint{0.964080in}{0.678935in}}{\pgfqpoint{0.968470in}{0.668336in}}{\pgfqpoint{0.976284in}{0.660522in}}%
\pgfpathcurveto{\pgfqpoint{0.984098in}{0.652709in}}{\pgfqpoint{0.994697in}{0.648318in}}{\pgfqpoint{1.005747in}{0.648318in}}%
\pgfpathclose%
\pgfusepath{stroke,fill}%
\end{pgfscope}%
\begin{pgfscope}%
\pgfpathrectangle{\pgfqpoint{0.787074in}{0.548769in}}{\pgfqpoint{4.974523in}{3.102590in}}%
\pgfusepath{clip}%
\pgfsetbuttcap%
\pgfsetroundjoin%
\definecolor{currentfill}{rgb}{0.121569,0.466667,0.705882}%
\pgfsetfillcolor{currentfill}%
\pgfsetlinewidth{1.003750pt}%
\definecolor{currentstroke}{rgb}{0.121569,0.466667,0.705882}%
\pgfsetstrokecolor{currentstroke}%
\pgfsetdash{}{0pt}%
\pgfpathmoveto{\pgfqpoint{2.071595in}{1.283193in}}%
\pgfpathcurveto{\pgfqpoint{2.082645in}{1.283193in}}{\pgfqpoint{2.093244in}{1.287584in}}{\pgfqpoint{2.101058in}{1.295397in}}%
\pgfpathcurveto{\pgfqpoint{2.108871in}{1.303211in}}{\pgfqpoint{2.113262in}{1.313810in}}{\pgfqpoint{2.113262in}{1.324860in}}%
\pgfpathcurveto{\pgfqpoint{2.113262in}{1.335910in}}{\pgfqpoint{2.108871in}{1.346509in}}{\pgfqpoint{2.101058in}{1.354323in}}%
\pgfpathcurveto{\pgfqpoint{2.093244in}{1.362137in}}{\pgfqpoint{2.082645in}{1.366527in}}{\pgfqpoint{2.071595in}{1.366527in}}%
\pgfpathcurveto{\pgfqpoint{2.060545in}{1.366527in}}{\pgfqpoint{2.049946in}{1.362137in}}{\pgfqpoint{2.042132in}{1.354323in}}%
\pgfpathcurveto{\pgfqpoint{2.034319in}{1.346509in}}{\pgfqpoint{2.029928in}{1.335910in}}{\pgfqpoint{2.029928in}{1.324860in}}%
\pgfpathcurveto{\pgfqpoint{2.029928in}{1.313810in}}{\pgfqpoint{2.034319in}{1.303211in}}{\pgfqpoint{2.042132in}{1.295397in}}%
\pgfpathcurveto{\pgfqpoint{2.049946in}{1.287584in}}{\pgfqpoint{2.060545in}{1.283193in}}{\pgfqpoint{2.071595in}{1.283193in}}%
\pgfpathclose%
\pgfusepath{stroke,fill}%
\end{pgfscope}%
\begin{pgfscope}%
\pgfpathrectangle{\pgfqpoint{0.787074in}{0.548769in}}{\pgfqpoint{4.974523in}{3.102590in}}%
\pgfusepath{clip}%
\pgfsetbuttcap%
\pgfsetroundjoin%
\definecolor{currentfill}{rgb}{1.000000,0.498039,0.054902}%
\pgfsetfillcolor{currentfill}%
\pgfsetlinewidth{1.003750pt}%
\definecolor{currentstroke}{rgb}{1.000000,0.498039,0.054902}%
\pgfsetstrokecolor{currentstroke}%
\pgfsetdash{}{0pt}%
\pgfpathmoveto{\pgfqpoint{4.059971in}{2.443414in}}%
\pgfpathcurveto{\pgfqpoint{4.071021in}{2.443414in}}{\pgfqpoint{4.081620in}{2.447804in}}{\pgfqpoint{4.089433in}{2.455618in}}%
\pgfpathcurveto{\pgfqpoint{4.097247in}{2.463432in}}{\pgfqpoint{4.101637in}{2.474031in}}{\pgfqpoint{4.101637in}{2.485081in}}%
\pgfpathcurveto{\pgfqpoint{4.101637in}{2.496131in}}{\pgfqpoint{4.097247in}{2.506730in}}{\pgfqpoint{4.089433in}{2.514544in}}%
\pgfpathcurveto{\pgfqpoint{4.081620in}{2.522357in}}{\pgfqpoint{4.071021in}{2.526747in}}{\pgfqpoint{4.059971in}{2.526747in}}%
\pgfpathcurveto{\pgfqpoint{4.048921in}{2.526747in}}{\pgfqpoint{4.038322in}{2.522357in}}{\pgfqpoint{4.030508in}{2.514544in}}%
\pgfpathcurveto{\pgfqpoint{4.022694in}{2.506730in}}{\pgfqpoint{4.018304in}{2.496131in}}{\pgfqpoint{4.018304in}{2.485081in}}%
\pgfpathcurveto{\pgfqpoint{4.018304in}{2.474031in}}{\pgfqpoint{4.022694in}{2.463432in}}{\pgfqpoint{4.030508in}{2.455618in}}%
\pgfpathcurveto{\pgfqpoint{4.038322in}{2.447804in}}{\pgfqpoint{4.048921in}{2.443414in}}{\pgfqpoint{4.059971in}{2.443414in}}%
\pgfpathclose%
\pgfusepath{stroke,fill}%
\end{pgfscope}%
\begin{pgfscope}%
\pgfpathrectangle{\pgfqpoint{0.787074in}{0.548769in}}{\pgfqpoint{4.974523in}{3.102590in}}%
\pgfusepath{clip}%
\pgfsetbuttcap%
\pgfsetroundjoin%
\definecolor{currentfill}{rgb}{0.121569,0.466667,0.705882}%
\pgfsetfillcolor{currentfill}%
\pgfsetlinewidth{1.003750pt}%
\definecolor{currentstroke}{rgb}{0.121569,0.466667,0.705882}%
\pgfsetstrokecolor{currentstroke}%
\pgfsetdash{}{0pt}%
\pgfpathmoveto{\pgfqpoint{2.480743in}{1.745269in}}%
\pgfpathcurveto{\pgfqpoint{2.491793in}{1.745269in}}{\pgfqpoint{2.502392in}{1.749660in}}{\pgfqpoint{2.510205in}{1.757473in}}%
\pgfpathcurveto{\pgfqpoint{2.518019in}{1.765287in}}{\pgfqpoint{2.522409in}{1.775886in}}{\pgfqpoint{2.522409in}{1.786936in}}%
\pgfpathcurveto{\pgfqpoint{2.522409in}{1.797986in}}{\pgfqpoint{2.518019in}{1.808585in}}{\pgfqpoint{2.510205in}{1.816399in}}%
\pgfpathcurveto{\pgfqpoint{2.502392in}{1.824212in}}{\pgfqpoint{2.491793in}{1.828603in}}{\pgfqpoint{2.480743in}{1.828603in}}%
\pgfpathcurveto{\pgfqpoint{2.469693in}{1.828603in}}{\pgfqpoint{2.459093in}{1.824212in}}{\pgfqpoint{2.451280in}{1.816399in}}%
\pgfpathcurveto{\pgfqpoint{2.443466in}{1.808585in}}{\pgfqpoint{2.439076in}{1.797986in}}{\pgfqpoint{2.439076in}{1.786936in}}%
\pgfpathcurveto{\pgfqpoint{2.439076in}{1.775886in}}{\pgfqpoint{2.443466in}{1.765287in}}{\pgfqpoint{2.451280in}{1.757473in}}%
\pgfpathcurveto{\pgfqpoint{2.459093in}{1.749660in}}{\pgfqpoint{2.469693in}{1.745269in}}{\pgfqpoint{2.480743in}{1.745269in}}%
\pgfpathclose%
\pgfusepath{stroke,fill}%
\end{pgfscope}%
\begin{pgfscope}%
\pgfpathrectangle{\pgfqpoint{0.787074in}{0.548769in}}{\pgfqpoint{4.974523in}{3.102590in}}%
\pgfusepath{clip}%
\pgfsetbuttcap%
\pgfsetroundjoin%
\definecolor{currentfill}{rgb}{0.839216,0.152941,0.156863}%
\pgfsetfillcolor{currentfill}%
\pgfsetlinewidth{1.003750pt}%
\definecolor{currentstroke}{rgb}{0.839216,0.152941,0.156863}%
\pgfsetstrokecolor{currentstroke}%
\pgfsetdash{}{0pt}%
\pgfpathmoveto{\pgfqpoint{4.760135in}{3.066498in}}%
\pgfpathcurveto{\pgfqpoint{4.771185in}{3.066498in}}{\pgfqpoint{4.781784in}{3.070888in}}{\pgfqpoint{4.789598in}{3.078701in}}%
\pgfpathcurveto{\pgfqpoint{4.797412in}{3.086515in}}{\pgfqpoint{4.801802in}{3.097114in}}{\pgfqpoint{4.801802in}{3.108164in}}%
\pgfpathcurveto{\pgfqpoint{4.801802in}{3.119214in}}{\pgfqpoint{4.797412in}{3.129813in}}{\pgfqpoint{4.789598in}{3.137627in}}%
\pgfpathcurveto{\pgfqpoint{4.781784in}{3.145441in}}{\pgfqpoint{4.771185in}{3.149831in}}{\pgfqpoint{4.760135in}{3.149831in}}%
\pgfpathcurveto{\pgfqpoint{4.749085in}{3.149831in}}{\pgfqpoint{4.738486in}{3.145441in}}{\pgfqpoint{4.730672in}{3.137627in}}%
\pgfpathcurveto{\pgfqpoint{4.722859in}{3.129813in}}{\pgfqpoint{4.718469in}{3.119214in}}{\pgfqpoint{4.718469in}{3.108164in}}%
\pgfpathcurveto{\pgfqpoint{4.718469in}{3.097114in}}{\pgfqpoint{4.722859in}{3.086515in}}{\pgfqpoint{4.730672in}{3.078701in}}%
\pgfpathcurveto{\pgfqpoint{4.738486in}{3.070888in}}{\pgfqpoint{4.749085in}{3.066498in}}{\pgfqpoint{4.760135in}{3.066498in}}%
\pgfpathclose%
\pgfusepath{stroke,fill}%
\end{pgfscope}%
\begin{pgfscope}%
\pgfpathrectangle{\pgfqpoint{0.787074in}{0.548769in}}{\pgfqpoint{4.974523in}{3.102590in}}%
\pgfusepath{clip}%
\pgfsetbuttcap%
\pgfsetroundjoin%
\definecolor{currentfill}{rgb}{1.000000,0.498039,0.054902}%
\pgfsetfillcolor{currentfill}%
\pgfsetlinewidth{1.003750pt}%
\definecolor{currentstroke}{rgb}{1.000000,0.498039,0.054902}%
\pgfsetstrokecolor{currentstroke}%
\pgfsetdash{}{0pt}%
\pgfpathmoveto{\pgfqpoint{4.096901in}{3.012840in}}%
\pgfpathcurveto{\pgfqpoint{4.107951in}{3.012840in}}{\pgfqpoint{4.118550in}{3.017231in}}{\pgfqpoint{4.126364in}{3.025044in}}%
\pgfpathcurveto{\pgfqpoint{4.134177in}{3.032858in}}{\pgfqpoint{4.138568in}{3.043457in}}{\pgfqpoint{4.138568in}{3.054507in}}%
\pgfpathcurveto{\pgfqpoint{4.138568in}{3.065557in}}{\pgfqpoint{4.134177in}{3.076156in}}{\pgfqpoint{4.126364in}{3.083970in}}%
\pgfpathcurveto{\pgfqpoint{4.118550in}{3.091783in}}{\pgfqpoint{4.107951in}{3.096174in}}{\pgfqpoint{4.096901in}{3.096174in}}%
\pgfpathcurveto{\pgfqpoint{4.085851in}{3.096174in}}{\pgfqpoint{4.075252in}{3.091783in}}{\pgfqpoint{4.067438in}{3.083970in}}%
\pgfpathcurveto{\pgfqpoint{4.059625in}{3.076156in}}{\pgfqpoint{4.055234in}{3.065557in}}{\pgfqpoint{4.055234in}{3.054507in}}%
\pgfpathcurveto{\pgfqpoint{4.055234in}{3.043457in}}{\pgfqpoint{4.059625in}{3.032858in}}{\pgfqpoint{4.067438in}{3.025044in}}%
\pgfpathcurveto{\pgfqpoint{4.075252in}{3.017231in}}{\pgfqpoint{4.085851in}{3.012840in}}{\pgfqpoint{4.096901in}{3.012840in}}%
\pgfpathclose%
\pgfusepath{stroke,fill}%
\end{pgfscope}%
\begin{pgfscope}%
\pgfpathrectangle{\pgfqpoint{0.787074in}{0.548769in}}{\pgfqpoint{4.974523in}{3.102590in}}%
\pgfusepath{clip}%
\pgfsetbuttcap%
\pgfsetroundjoin%
\definecolor{currentfill}{rgb}{0.121569,0.466667,0.705882}%
\pgfsetfillcolor{currentfill}%
\pgfsetlinewidth{1.003750pt}%
\definecolor{currentstroke}{rgb}{0.121569,0.466667,0.705882}%
\pgfsetstrokecolor{currentstroke}%
\pgfsetdash{}{0pt}%
\pgfpathmoveto{\pgfqpoint{1.042852in}{0.670241in}}%
\pgfpathcurveto{\pgfqpoint{1.053902in}{0.670241in}}{\pgfqpoint{1.064501in}{0.674631in}}{\pgfqpoint{1.072315in}{0.682445in}}%
\pgfpathcurveto{\pgfqpoint{1.080128in}{0.690258in}}{\pgfqpoint{1.084518in}{0.700857in}}{\pgfqpoint{1.084518in}{0.711907in}}%
\pgfpathcurveto{\pgfqpoint{1.084518in}{0.722958in}}{\pgfqpoint{1.080128in}{0.733557in}}{\pgfqpoint{1.072315in}{0.741370in}}%
\pgfpathcurveto{\pgfqpoint{1.064501in}{0.749184in}}{\pgfqpoint{1.053902in}{0.753574in}}{\pgfqpoint{1.042852in}{0.753574in}}%
\pgfpathcurveto{\pgfqpoint{1.031802in}{0.753574in}}{\pgfqpoint{1.021203in}{0.749184in}}{\pgfqpoint{1.013389in}{0.741370in}}%
\pgfpathcurveto{\pgfqpoint{1.005575in}{0.733557in}}{\pgfqpoint{1.001185in}{0.722958in}}{\pgfqpoint{1.001185in}{0.711907in}}%
\pgfpathcurveto{\pgfqpoint{1.001185in}{0.700857in}}{\pgfqpoint{1.005575in}{0.690258in}}{\pgfqpoint{1.013389in}{0.682445in}}%
\pgfpathcurveto{\pgfqpoint{1.021203in}{0.674631in}}{\pgfqpoint{1.031802in}{0.670241in}}{\pgfqpoint{1.042852in}{0.670241in}}%
\pgfpathclose%
\pgfusepath{stroke,fill}%
\end{pgfscope}%
\begin{pgfscope}%
\pgfpathrectangle{\pgfqpoint{0.787074in}{0.548769in}}{\pgfqpoint{4.974523in}{3.102590in}}%
\pgfusepath{clip}%
\pgfsetbuttcap%
\pgfsetroundjoin%
\definecolor{currentfill}{rgb}{0.121569,0.466667,0.705882}%
\pgfsetfillcolor{currentfill}%
\pgfsetlinewidth{1.003750pt}%
\definecolor{currentstroke}{rgb}{0.121569,0.466667,0.705882}%
\pgfsetstrokecolor{currentstroke}%
\pgfsetdash{}{0pt}%
\pgfpathmoveto{\pgfqpoint{2.013882in}{1.395275in}}%
\pgfpathcurveto{\pgfqpoint{2.024932in}{1.395275in}}{\pgfqpoint{2.035531in}{1.399666in}}{\pgfqpoint{2.043345in}{1.407479in}}%
\pgfpathcurveto{\pgfqpoint{2.051158in}{1.415293in}}{\pgfqpoint{2.055548in}{1.425892in}}{\pgfqpoint{2.055548in}{1.436942in}}%
\pgfpathcurveto{\pgfqpoint{2.055548in}{1.447992in}}{\pgfqpoint{2.051158in}{1.458591in}}{\pgfqpoint{2.043345in}{1.466405in}}%
\pgfpathcurveto{\pgfqpoint{2.035531in}{1.474219in}}{\pgfqpoint{2.024932in}{1.478609in}}{\pgfqpoint{2.013882in}{1.478609in}}%
\pgfpathcurveto{\pgfqpoint{2.002832in}{1.478609in}}{\pgfqpoint{1.992233in}{1.474219in}}{\pgfqpoint{1.984419in}{1.466405in}}%
\pgfpathcurveto{\pgfqpoint{1.976605in}{1.458591in}}{\pgfqpoint{1.972215in}{1.447992in}}{\pgfqpoint{1.972215in}{1.436942in}}%
\pgfpathcurveto{\pgfqpoint{1.972215in}{1.425892in}}{\pgfqpoint{1.976605in}{1.415293in}}{\pgfqpoint{1.984419in}{1.407479in}}%
\pgfpathcurveto{\pgfqpoint{1.992233in}{1.399666in}}{\pgfqpoint{2.002832in}{1.395275in}}{\pgfqpoint{2.013882in}{1.395275in}}%
\pgfpathclose%
\pgfusepath{stroke,fill}%
\end{pgfscope}%
\begin{pgfscope}%
\pgfpathrectangle{\pgfqpoint{0.787074in}{0.548769in}}{\pgfqpoint{4.974523in}{3.102590in}}%
\pgfusepath{clip}%
\pgfsetbuttcap%
\pgfsetroundjoin%
\definecolor{currentfill}{rgb}{0.121569,0.466667,0.705882}%
\pgfsetfillcolor{currentfill}%
\pgfsetlinewidth{1.003750pt}%
\definecolor{currentstroke}{rgb}{0.121569,0.466667,0.705882}%
\pgfsetstrokecolor{currentstroke}%
\pgfsetdash{}{0pt}%
\pgfpathmoveto{\pgfqpoint{1.005764in}{0.648350in}}%
\pgfpathcurveto{\pgfqpoint{1.016814in}{0.648350in}}{\pgfqpoint{1.027413in}{0.652741in}}{\pgfqpoint{1.035226in}{0.660554in}}%
\pgfpathcurveto{\pgfqpoint{1.043040in}{0.668368in}}{\pgfqpoint{1.047430in}{0.678967in}}{\pgfqpoint{1.047430in}{0.690017in}}%
\pgfpathcurveto{\pgfqpoint{1.047430in}{0.701067in}}{\pgfqpoint{1.043040in}{0.711666in}}{\pgfqpoint{1.035226in}{0.719480in}}%
\pgfpathcurveto{\pgfqpoint{1.027413in}{0.727293in}}{\pgfqpoint{1.016814in}{0.731684in}}{\pgfqpoint{1.005764in}{0.731684in}}%
\pgfpathcurveto{\pgfqpoint{0.994713in}{0.731684in}}{\pgfqpoint{0.984114in}{0.727293in}}{\pgfqpoint{0.976301in}{0.719480in}}%
\pgfpathcurveto{\pgfqpoint{0.968487in}{0.711666in}}{\pgfqpoint{0.964097in}{0.701067in}}{\pgfqpoint{0.964097in}{0.690017in}}%
\pgfpathcurveto{\pgfqpoint{0.964097in}{0.678967in}}{\pgfqpoint{0.968487in}{0.668368in}}{\pgfqpoint{0.976301in}{0.660554in}}%
\pgfpathcurveto{\pgfqpoint{0.984114in}{0.652741in}}{\pgfqpoint{0.994713in}{0.648350in}}{\pgfqpoint{1.005764in}{0.648350in}}%
\pgfpathclose%
\pgfusepath{stroke,fill}%
\end{pgfscope}%
\begin{pgfscope}%
\pgfpathrectangle{\pgfqpoint{0.787074in}{0.548769in}}{\pgfqpoint{4.974523in}{3.102590in}}%
\pgfusepath{clip}%
\pgfsetbuttcap%
\pgfsetroundjoin%
\definecolor{currentfill}{rgb}{1.000000,0.498039,0.054902}%
\pgfsetfillcolor{currentfill}%
\pgfsetlinewidth{1.003750pt}%
\definecolor{currentstroke}{rgb}{1.000000,0.498039,0.054902}%
\pgfsetstrokecolor{currentstroke}%
\pgfsetdash{}{0pt}%
\pgfpathmoveto{\pgfqpoint{2.338549in}{1.744372in}}%
\pgfpathcurveto{\pgfqpoint{2.349599in}{1.744372in}}{\pgfqpoint{2.360198in}{1.748762in}}{\pgfqpoint{2.368012in}{1.756576in}}%
\pgfpathcurveto{\pgfqpoint{2.375825in}{1.764389in}}{\pgfqpoint{2.380216in}{1.774988in}}{\pgfqpoint{2.380216in}{1.786038in}}%
\pgfpathcurveto{\pgfqpoint{2.380216in}{1.797089in}}{\pgfqpoint{2.375825in}{1.807688in}}{\pgfqpoint{2.368012in}{1.815501in}}%
\pgfpathcurveto{\pgfqpoint{2.360198in}{1.823315in}}{\pgfqpoint{2.349599in}{1.827705in}}{\pgfqpoint{2.338549in}{1.827705in}}%
\pgfpathcurveto{\pgfqpoint{2.327499in}{1.827705in}}{\pgfqpoint{2.316900in}{1.823315in}}{\pgfqpoint{2.309086in}{1.815501in}}%
\pgfpathcurveto{\pgfqpoint{2.301272in}{1.807688in}}{\pgfqpoint{2.296882in}{1.797089in}}{\pgfqpoint{2.296882in}{1.786038in}}%
\pgfpathcurveto{\pgfqpoint{2.296882in}{1.774988in}}{\pgfqpoint{2.301272in}{1.764389in}}{\pgfqpoint{2.309086in}{1.756576in}}%
\pgfpathcurveto{\pgfqpoint{2.316900in}{1.748762in}}{\pgfqpoint{2.327499in}{1.744372in}}{\pgfqpoint{2.338549in}{1.744372in}}%
\pgfpathclose%
\pgfusepath{stroke,fill}%
\end{pgfscope}%
\begin{pgfscope}%
\pgfpathrectangle{\pgfqpoint{0.787074in}{0.548769in}}{\pgfqpoint{4.974523in}{3.102590in}}%
\pgfusepath{clip}%
\pgfsetbuttcap%
\pgfsetroundjoin%
\definecolor{currentfill}{rgb}{1.000000,0.498039,0.054902}%
\pgfsetfillcolor{currentfill}%
\pgfsetlinewidth{1.003750pt}%
\definecolor{currentstroke}{rgb}{1.000000,0.498039,0.054902}%
\pgfsetstrokecolor{currentstroke}%
\pgfsetdash{}{0pt}%
\pgfpathmoveto{\pgfqpoint{4.168683in}{2.684286in}}%
\pgfpathcurveto{\pgfqpoint{4.179733in}{2.684286in}}{\pgfqpoint{4.190332in}{2.688676in}}{\pgfqpoint{4.198146in}{2.696490in}}%
\pgfpathcurveto{\pgfqpoint{4.205959in}{2.704304in}}{\pgfqpoint{4.210350in}{2.714903in}}{\pgfqpoint{4.210350in}{2.725953in}}%
\pgfpathcurveto{\pgfqpoint{4.210350in}{2.737003in}}{\pgfqpoint{4.205959in}{2.747602in}}{\pgfqpoint{4.198146in}{2.755416in}}%
\pgfpathcurveto{\pgfqpoint{4.190332in}{2.763229in}}{\pgfqpoint{4.179733in}{2.767619in}}{\pgfqpoint{4.168683in}{2.767619in}}%
\pgfpathcurveto{\pgfqpoint{4.157633in}{2.767619in}}{\pgfqpoint{4.147034in}{2.763229in}}{\pgfqpoint{4.139220in}{2.755416in}}%
\pgfpathcurveto{\pgfqpoint{4.131406in}{2.747602in}}{\pgfqpoint{4.127016in}{2.737003in}}{\pgfqpoint{4.127016in}{2.725953in}}%
\pgfpathcurveto{\pgfqpoint{4.127016in}{2.714903in}}{\pgfqpoint{4.131406in}{2.704304in}}{\pgfqpoint{4.139220in}{2.696490in}}%
\pgfpathcurveto{\pgfqpoint{4.147034in}{2.688676in}}{\pgfqpoint{4.157633in}{2.684286in}}{\pgfqpoint{4.168683in}{2.684286in}}%
\pgfpathclose%
\pgfusepath{stroke,fill}%
\end{pgfscope}%
\begin{pgfscope}%
\pgfpathrectangle{\pgfqpoint{0.787074in}{0.548769in}}{\pgfqpoint{4.974523in}{3.102590in}}%
\pgfusepath{clip}%
\pgfsetbuttcap%
\pgfsetroundjoin%
\definecolor{currentfill}{rgb}{1.000000,0.498039,0.054902}%
\pgfsetfillcolor{currentfill}%
\pgfsetlinewidth{1.003750pt}%
\definecolor{currentstroke}{rgb}{1.000000,0.498039,0.054902}%
\pgfsetstrokecolor{currentstroke}%
\pgfsetdash{}{0pt}%
\pgfpathmoveto{\pgfqpoint{2.533947in}{2.189411in}}%
\pgfpathcurveto{\pgfqpoint{2.544997in}{2.189411in}}{\pgfqpoint{2.555596in}{2.193801in}}{\pgfqpoint{2.563410in}{2.201614in}}%
\pgfpathcurveto{\pgfqpoint{2.571223in}{2.209428in}}{\pgfqpoint{2.575614in}{2.220027in}}{\pgfqpoint{2.575614in}{2.231077in}}%
\pgfpathcurveto{\pgfqpoint{2.575614in}{2.242127in}}{\pgfqpoint{2.571223in}{2.252726in}}{\pgfqpoint{2.563410in}{2.260540in}}%
\pgfpathcurveto{\pgfqpoint{2.555596in}{2.268354in}}{\pgfqpoint{2.544997in}{2.272744in}}{\pgfqpoint{2.533947in}{2.272744in}}%
\pgfpathcurveto{\pgfqpoint{2.522897in}{2.272744in}}{\pgfqpoint{2.512298in}{2.268354in}}{\pgfqpoint{2.504484in}{2.260540in}}%
\pgfpathcurveto{\pgfqpoint{2.496671in}{2.252726in}}{\pgfqpoint{2.492280in}{2.242127in}}{\pgfqpoint{2.492280in}{2.231077in}}%
\pgfpathcurveto{\pgfqpoint{2.492280in}{2.220027in}}{\pgfqpoint{2.496671in}{2.209428in}}{\pgfqpoint{2.504484in}{2.201614in}}%
\pgfpathcurveto{\pgfqpoint{2.512298in}{2.193801in}}{\pgfqpoint{2.522897in}{2.189411in}}{\pgfqpoint{2.533947in}{2.189411in}}%
\pgfpathclose%
\pgfusepath{stroke,fill}%
\end{pgfscope}%
\begin{pgfscope}%
\pgfpathrectangle{\pgfqpoint{0.787074in}{0.548769in}}{\pgfqpoint{4.974523in}{3.102590in}}%
\pgfusepath{clip}%
\pgfsetbuttcap%
\pgfsetroundjoin%
\definecolor{currentfill}{rgb}{1.000000,0.498039,0.054902}%
\pgfsetfillcolor{currentfill}%
\pgfsetlinewidth{1.003750pt}%
\definecolor{currentstroke}{rgb}{1.000000,0.498039,0.054902}%
\pgfsetstrokecolor{currentstroke}%
\pgfsetdash{}{0pt}%
\pgfpathmoveto{\pgfqpoint{4.160427in}{2.951689in}}%
\pgfpathcurveto{\pgfqpoint{4.171477in}{2.951689in}}{\pgfqpoint{4.182076in}{2.956079in}}{\pgfqpoint{4.189889in}{2.963893in}}%
\pgfpathcurveto{\pgfqpoint{4.197703in}{2.971707in}}{\pgfqpoint{4.202093in}{2.982306in}}{\pgfqpoint{4.202093in}{2.993356in}}%
\pgfpathcurveto{\pgfqpoint{4.202093in}{3.004406in}}{\pgfqpoint{4.197703in}{3.015005in}}{\pgfqpoint{4.189889in}{3.022818in}}%
\pgfpathcurveto{\pgfqpoint{4.182076in}{3.030632in}}{\pgfqpoint{4.171477in}{3.035022in}}{\pgfqpoint{4.160427in}{3.035022in}}%
\pgfpathcurveto{\pgfqpoint{4.149376in}{3.035022in}}{\pgfqpoint{4.138777in}{3.030632in}}{\pgfqpoint{4.130964in}{3.022818in}}%
\pgfpathcurveto{\pgfqpoint{4.123150in}{3.015005in}}{\pgfqpoint{4.118760in}{3.004406in}}{\pgfqpoint{4.118760in}{2.993356in}}%
\pgfpathcurveto{\pgfqpoint{4.118760in}{2.982306in}}{\pgfqpoint{4.123150in}{2.971707in}}{\pgfqpoint{4.130964in}{2.963893in}}%
\pgfpathcurveto{\pgfqpoint{4.138777in}{2.956079in}}{\pgfqpoint{4.149376in}{2.951689in}}{\pgfqpoint{4.160427in}{2.951689in}}%
\pgfpathclose%
\pgfusepath{stroke,fill}%
\end{pgfscope}%
\begin{pgfscope}%
\pgfpathrectangle{\pgfqpoint{0.787074in}{0.548769in}}{\pgfqpoint{4.974523in}{3.102590in}}%
\pgfusepath{clip}%
\pgfsetbuttcap%
\pgfsetroundjoin%
\definecolor{currentfill}{rgb}{1.000000,0.498039,0.054902}%
\pgfsetfillcolor{currentfill}%
\pgfsetlinewidth{1.003750pt}%
\definecolor{currentstroke}{rgb}{1.000000,0.498039,0.054902}%
\pgfsetstrokecolor{currentstroke}%
\pgfsetdash{}{0pt}%
\pgfpathmoveto{\pgfqpoint{4.403817in}{2.956805in}}%
\pgfpathcurveto{\pgfqpoint{4.414867in}{2.956805in}}{\pgfqpoint{4.425466in}{2.961195in}}{\pgfqpoint{4.433279in}{2.969009in}}%
\pgfpathcurveto{\pgfqpoint{4.441093in}{2.976822in}}{\pgfqpoint{4.445483in}{2.987421in}}{\pgfqpoint{4.445483in}{2.998472in}}%
\pgfpathcurveto{\pgfqpoint{4.445483in}{3.009522in}}{\pgfqpoint{4.441093in}{3.020121in}}{\pgfqpoint{4.433279in}{3.027934in}}%
\pgfpathcurveto{\pgfqpoint{4.425466in}{3.035748in}}{\pgfqpoint{4.414867in}{3.040138in}}{\pgfqpoint{4.403817in}{3.040138in}}%
\pgfpathcurveto{\pgfqpoint{4.392766in}{3.040138in}}{\pgfqpoint{4.382167in}{3.035748in}}{\pgfqpoint{4.374354in}{3.027934in}}%
\pgfpathcurveto{\pgfqpoint{4.366540in}{3.020121in}}{\pgfqpoint{4.362150in}{3.009522in}}{\pgfqpoint{4.362150in}{2.998472in}}%
\pgfpathcurveto{\pgfqpoint{4.362150in}{2.987421in}}{\pgfqpoint{4.366540in}{2.976822in}}{\pgfqpoint{4.374354in}{2.969009in}}%
\pgfpathcurveto{\pgfqpoint{4.382167in}{2.961195in}}{\pgfqpoint{4.392766in}{2.956805in}}{\pgfqpoint{4.403817in}{2.956805in}}%
\pgfpathclose%
\pgfusepath{stroke,fill}%
\end{pgfscope}%
\begin{pgfscope}%
\pgfpathrectangle{\pgfqpoint{0.787074in}{0.548769in}}{\pgfqpoint{4.974523in}{3.102590in}}%
\pgfusepath{clip}%
\pgfsetbuttcap%
\pgfsetroundjoin%
\definecolor{currentfill}{rgb}{1.000000,0.498039,0.054902}%
\pgfsetfillcolor{currentfill}%
\pgfsetlinewidth{1.003750pt}%
\definecolor{currentstroke}{rgb}{1.000000,0.498039,0.054902}%
\pgfsetstrokecolor{currentstroke}%
\pgfsetdash{}{0pt}%
\pgfpathmoveto{\pgfqpoint{4.368794in}{2.890755in}}%
\pgfpathcurveto{\pgfqpoint{4.379844in}{2.890755in}}{\pgfqpoint{4.390443in}{2.895145in}}{\pgfqpoint{4.398257in}{2.902959in}}%
\pgfpathcurveto{\pgfqpoint{4.406071in}{2.910773in}}{\pgfqpoint{4.410461in}{2.921372in}}{\pgfqpoint{4.410461in}{2.932422in}}%
\pgfpathcurveto{\pgfqpoint{4.410461in}{2.943472in}}{\pgfqpoint{4.406071in}{2.954071in}}{\pgfqpoint{4.398257in}{2.961885in}}%
\pgfpathcurveto{\pgfqpoint{4.390443in}{2.969698in}}{\pgfqpoint{4.379844in}{2.974089in}}{\pgfqpoint{4.368794in}{2.974089in}}%
\pgfpathcurveto{\pgfqpoint{4.357744in}{2.974089in}}{\pgfqpoint{4.347145in}{2.969698in}}{\pgfqpoint{4.339331in}{2.961885in}}%
\pgfpathcurveto{\pgfqpoint{4.331518in}{2.954071in}}{\pgfqpoint{4.327128in}{2.943472in}}{\pgfqpoint{4.327128in}{2.932422in}}%
\pgfpathcurveto{\pgfqpoint{4.327128in}{2.921372in}}{\pgfqpoint{4.331518in}{2.910773in}}{\pgfqpoint{4.339331in}{2.902959in}}%
\pgfpathcurveto{\pgfqpoint{4.347145in}{2.895145in}}{\pgfqpoint{4.357744in}{2.890755in}}{\pgfqpoint{4.368794in}{2.890755in}}%
\pgfpathclose%
\pgfusepath{stroke,fill}%
\end{pgfscope}%
\begin{pgfscope}%
\pgfpathrectangle{\pgfqpoint{0.787074in}{0.548769in}}{\pgfqpoint{4.974523in}{3.102590in}}%
\pgfusepath{clip}%
\pgfsetbuttcap%
\pgfsetroundjoin%
\definecolor{currentfill}{rgb}{0.121569,0.466667,0.705882}%
\pgfsetfillcolor{currentfill}%
\pgfsetlinewidth{1.003750pt}%
\definecolor{currentstroke}{rgb}{0.121569,0.466667,0.705882}%
\pgfsetstrokecolor{currentstroke}%
\pgfsetdash{}{0pt}%
\pgfpathmoveto{\pgfqpoint{3.609998in}{2.074998in}}%
\pgfpathcurveto{\pgfqpoint{3.621048in}{2.074998in}}{\pgfqpoint{3.631647in}{2.079388in}}{\pgfqpoint{3.639461in}{2.087202in}}%
\pgfpathcurveto{\pgfqpoint{3.647274in}{2.095016in}}{\pgfqpoint{3.651665in}{2.105615in}}{\pgfqpoint{3.651665in}{2.116665in}}%
\pgfpathcurveto{\pgfqpoint{3.651665in}{2.127715in}}{\pgfqpoint{3.647274in}{2.138314in}}{\pgfqpoint{3.639461in}{2.146127in}}%
\pgfpathcurveto{\pgfqpoint{3.631647in}{2.153941in}}{\pgfqpoint{3.621048in}{2.158331in}}{\pgfqpoint{3.609998in}{2.158331in}}%
\pgfpathcurveto{\pgfqpoint{3.598948in}{2.158331in}}{\pgfqpoint{3.588349in}{2.153941in}}{\pgfqpoint{3.580535in}{2.146127in}}%
\pgfpathcurveto{\pgfqpoint{3.572722in}{2.138314in}}{\pgfqpoint{3.568331in}{2.127715in}}{\pgfqpoint{3.568331in}{2.116665in}}%
\pgfpathcurveto{\pgfqpoint{3.568331in}{2.105615in}}{\pgfqpoint{3.572722in}{2.095016in}}{\pgfqpoint{3.580535in}{2.087202in}}%
\pgfpathcurveto{\pgfqpoint{3.588349in}{2.079388in}}{\pgfqpoint{3.598948in}{2.074998in}}{\pgfqpoint{3.609998in}{2.074998in}}%
\pgfpathclose%
\pgfusepath{stroke,fill}%
\end{pgfscope}%
\begin{pgfscope}%
\pgfpathrectangle{\pgfqpoint{0.787074in}{0.548769in}}{\pgfqpoint{4.974523in}{3.102590in}}%
\pgfusepath{clip}%
\pgfsetbuttcap%
\pgfsetroundjoin%
\definecolor{currentfill}{rgb}{1.000000,0.498039,0.054902}%
\pgfsetfillcolor{currentfill}%
\pgfsetlinewidth{1.003750pt}%
\definecolor{currentstroke}{rgb}{1.000000,0.498039,0.054902}%
\pgfsetstrokecolor{currentstroke}%
\pgfsetdash{}{0pt}%
\pgfpathmoveto{\pgfqpoint{4.363110in}{2.746524in}}%
\pgfpathcurveto{\pgfqpoint{4.374160in}{2.746524in}}{\pgfqpoint{4.384760in}{2.750915in}}{\pgfqpoint{4.392573in}{2.758728in}}%
\pgfpathcurveto{\pgfqpoint{4.400387in}{2.766542in}}{\pgfqpoint{4.404777in}{2.777141in}}{\pgfqpoint{4.404777in}{2.788191in}}%
\pgfpathcurveto{\pgfqpoint{4.404777in}{2.799241in}}{\pgfqpoint{4.400387in}{2.809840in}}{\pgfqpoint{4.392573in}{2.817654in}}%
\pgfpathcurveto{\pgfqpoint{4.384760in}{2.825468in}}{\pgfqpoint{4.374160in}{2.829858in}}{\pgfqpoint{4.363110in}{2.829858in}}%
\pgfpathcurveto{\pgfqpoint{4.352060in}{2.829858in}}{\pgfqpoint{4.341461in}{2.825468in}}{\pgfqpoint{4.333648in}{2.817654in}}%
\pgfpathcurveto{\pgfqpoint{4.325834in}{2.809840in}}{\pgfqpoint{4.321444in}{2.799241in}}{\pgfqpoint{4.321444in}{2.788191in}}%
\pgfpathcurveto{\pgfqpoint{4.321444in}{2.777141in}}{\pgfqpoint{4.325834in}{2.766542in}}{\pgfqpoint{4.333648in}{2.758728in}}%
\pgfpathcurveto{\pgfqpoint{4.341461in}{2.750915in}}{\pgfqpoint{4.352060in}{2.746524in}}{\pgfqpoint{4.363110in}{2.746524in}}%
\pgfpathclose%
\pgfusepath{stroke,fill}%
\end{pgfscope}%
\begin{pgfscope}%
\pgfpathrectangle{\pgfqpoint{0.787074in}{0.548769in}}{\pgfqpoint{4.974523in}{3.102590in}}%
\pgfusepath{clip}%
\pgfsetbuttcap%
\pgfsetroundjoin%
\definecolor{currentfill}{rgb}{0.121569,0.466667,0.705882}%
\pgfsetfillcolor{currentfill}%
\pgfsetlinewidth{1.003750pt}%
\definecolor{currentstroke}{rgb}{0.121569,0.466667,0.705882}%
\pgfsetstrokecolor{currentstroke}%
\pgfsetdash{}{0pt}%
\pgfpathmoveto{\pgfqpoint{2.029455in}{1.732307in}}%
\pgfpathcurveto{\pgfqpoint{2.040506in}{1.732307in}}{\pgfqpoint{2.051105in}{1.736697in}}{\pgfqpoint{2.058918in}{1.744511in}}%
\pgfpathcurveto{\pgfqpoint{2.066732in}{1.752325in}}{\pgfqpoint{2.071122in}{1.762924in}}{\pgfqpoint{2.071122in}{1.773974in}}%
\pgfpathcurveto{\pgfqpoint{2.071122in}{1.785024in}}{\pgfqpoint{2.066732in}{1.795623in}}{\pgfqpoint{2.058918in}{1.803437in}}%
\pgfpathcurveto{\pgfqpoint{2.051105in}{1.811250in}}{\pgfqpoint{2.040506in}{1.815640in}}{\pgfqpoint{2.029455in}{1.815640in}}%
\pgfpathcurveto{\pgfqpoint{2.018405in}{1.815640in}}{\pgfqpoint{2.007806in}{1.811250in}}{\pgfqpoint{1.999993in}{1.803437in}}%
\pgfpathcurveto{\pgfqpoint{1.992179in}{1.795623in}}{\pgfqpoint{1.987789in}{1.785024in}}{\pgfqpoint{1.987789in}{1.773974in}}%
\pgfpathcurveto{\pgfqpoint{1.987789in}{1.762924in}}{\pgfqpoint{1.992179in}{1.752325in}}{\pgfqpoint{1.999993in}{1.744511in}}%
\pgfpathcurveto{\pgfqpoint{2.007806in}{1.736697in}}{\pgfqpoint{2.018405in}{1.732307in}}{\pgfqpoint{2.029455in}{1.732307in}}%
\pgfpathclose%
\pgfusepath{stroke,fill}%
\end{pgfscope}%
\begin{pgfscope}%
\pgfpathrectangle{\pgfqpoint{0.787074in}{0.548769in}}{\pgfqpoint{4.974523in}{3.102590in}}%
\pgfusepath{clip}%
\pgfsetbuttcap%
\pgfsetroundjoin%
\definecolor{currentfill}{rgb}{0.121569,0.466667,0.705882}%
\pgfsetfillcolor{currentfill}%
\pgfsetlinewidth{1.003750pt}%
\definecolor{currentstroke}{rgb}{0.121569,0.466667,0.705882}%
\pgfsetstrokecolor{currentstroke}%
\pgfsetdash{}{0pt}%
\pgfpathmoveto{\pgfqpoint{3.084697in}{2.443186in}}%
\pgfpathcurveto{\pgfqpoint{3.095747in}{2.443186in}}{\pgfqpoint{3.106346in}{2.447576in}}{\pgfqpoint{3.114160in}{2.455390in}}%
\pgfpathcurveto{\pgfqpoint{3.121974in}{2.463203in}}{\pgfqpoint{3.126364in}{2.473802in}}{\pgfqpoint{3.126364in}{2.484852in}}%
\pgfpathcurveto{\pgfqpoint{3.126364in}{2.495903in}}{\pgfqpoint{3.121974in}{2.506502in}}{\pgfqpoint{3.114160in}{2.514315in}}%
\pgfpathcurveto{\pgfqpoint{3.106346in}{2.522129in}}{\pgfqpoint{3.095747in}{2.526519in}}{\pgfqpoint{3.084697in}{2.526519in}}%
\pgfpathcurveto{\pgfqpoint{3.073647in}{2.526519in}}{\pgfqpoint{3.063048in}{2.522129in}}{\pgfqpoint{3.055234in}{2.514315in}}%
\pgfpathcurveto{\pgfqpoint{3.047421in}{2.506502in}}{\pgfqpoint{3.043031in}{2.495903in}}{\pgfqpoint{3.043031in}{2.484852in}}%
\pgfpathcurveto{\pgfqpoint{3.043031in}{2.473802in}}{\pgfqpoint{3.047421in}{2.463203in}}{\pgfqpoint{3.055234in}{2.455390in}}%
\pgfpathcurveto{\pgfqpoint{3.063048in}{2.447576in}}{\pgfqpoint{3.073647in}{2.443186in}}{\pgfqpoint{3.084697in}{2.443186in}}%
\pgfpathclose%
\pgfusepath{stroke,fill}%
\end{pgfscope}%
\begin{pgfscope}%
\pgfpathrectangle{\pgfqpoint{0.787074in}{0.548769in}}{\pgfqpoint{4.974523in}{3.102590in}}%
\pgfusepath{clip}%
\pgfsetbuttcap%
\pgfsetroundjoin%
\definecolor{currentfill}{rgb}{0.121569,0.466667,0.705882}%
\pgfsetfillcolor{currentfill}%
\pgfsetlinewidth{1.003750pt}%
\definecolor{currentstroke}{rgb}{0.121569,0.466667,0.705882}%
\pgfsetstrokecolor{currentstroke}%
\pgfsetdash{}{0pt}%
\pgfpathmoveto{\pgfqpoint{1.992055in}{1.419741in}}%
\pgfpathcurveto{\pgfqpoint{2.003105in}{1.419741in}}{\pgfqpoint{2.013704in}{1.424131in}}{\pgfqpoint{2.021517in}{1.431945in}}%
\pgfpathcurveto{\pgfqpoint{2.029331in}{1.439759in}}{\pgfqpoint{2.033721in}{1.450358in}}{\pgfqpoint{2.033721in}{1.461408in}}%
\pgfpathcurveto{\pgfqpoint{2.033721in}{1.472458in}}{\pgfqpoint{2.029331in}{1.483057in}}{\pgfqpoint{2.021517in}{1.490870in}}%
\pgfpathcurveto{\pgfqpoint{2.013704in}{1.498684in}}{\pgfqpoint{2.003105in}{1.503074in}}{\pgfqpoint{1.992055in}{1.503074in}}%
\pgfpathcurveto{\pgfqpoint{1.981004in}{1.503074in}}{\pgfqpoint{1.970405in}{1.498684in}}{\pgfqpoint{1.962592in}{1.490870in}}%
\pgfpathcurveto{\pgfqpoint{1.954778in}{1.483057in}}{\pgfqpoint{1.950388in}{1.472458in}}{\pgfqpoint{1.950388in}{1.461408in}}%
\pgfpathcurveto{\pgfqpoint{1.950388in}{1.450358in}}{\pgfqpoint{1.954778in}{1.439759in}}{\pgfqpoint{1.962592in}{1.431945in}}%
\pgfpathcurveto{\pgfqpoint{1.970405in}{1.424131in}}{\pgfqpoint{1.981004in}{1.419741in}}{\pgfqpoint{1.992055in}{1.419741in}}%
\pgfpathclose%
\pgfusepath{stroke,fill}%
\end{pgfscope}%
\begin{pgfscope}%
\pgfpathrectangle{\pgfqpoint{0.787074in}{0.548769in}}{\pgfqpoint{4.974523in}{3.102590in}}%
\pgfusepath{clip}%
\pgfsetbuttcap%
\pgfsetroundjoin%
\definecolor{currentfill}{rgb}{1.000000,0.498039,0.054902}%
\pgfsetfillcolor{currentfill}%
\pgfsetlinewidth{1.003750pt}%
\definecolor{currentstroke}{rgb}{1.000000,0.498039,0.054902}%
\pgfsetstrokecolor{currentstroke}%
\pgfsetdash{}{0pt}%
\pgfpathmoveto{\pgfqpoint{3.765621in}{2.589074in}}%
\pgfpathcurveto{\pgfqpoint{3.776671in}{2.589074in}}{\pgfqpoint{3.787270in}{2.593464in}}{\pgfqpoint{3.795084in}{2.601278in}}%
\pgfpathcurveto{\pgfqpoint{3.802898in}{2.609092in}}{\pgfqpoint{3.807288in}{2.619691in}}{\pgfqpoint{3.807288in}{2.630741in}}%
\pgfpathcurveto{\pgfqpoint{3.807288in}{2.641791in}}{\pgfqpoint{3.802898in}{2.652390in}}{\pgfqpoint{3.795084in}{2.660204in}}%
\pgfpathcurveto{\pgfqpoint{3.787270in}{2.668017in}}{\pgfqpoint{3.776671in}{2.672407in}}{\pgfqpoint{3.765621in}{2.672407in}}%
\pgfpathcurveto{\pgfqpoint{3.754571in}{2.672407in}}{\pgfqpoint{3.743972in}{2.668017in}}{\pgfqpoint{3.736158in}{2.660204in}}%
\pgfpathcurveto{\pgfqpoint{3.728345in}{2.652390in}}{\pgfqpoint{3.723955in}{2.641791in}}{\pgfqpoint{3.723955in}{2.630741in}}%
\pgfpathcurveto{\pgfqpoint{3.723955in}{2.619691in}}{\pgfqpoint{3.728345in}{2.609092in}}{\pgfqpoint{3.736158in}{2.601278in}}%
\pgfpathcurveto{\pgfqpoint{3.743972in}{2.593464in}}{\pgfqpoint{3.754571in}{2.589074in}}{\pgfqpoint{3.765621in}{2.589074in}}%
\pgfpathclose%
\pgfusepath{stroke,fill}%
\end{pgfscope}%
\begin{pgfscope}%
\pgfpathrectangle{\pgfqpoint{0.787074in}{0.548769in}}{\pgfqpoint{4.974523in}{3.102590in}}%
\pgfusepath{clip}%
\pgfsetbuttcap%
\pgfsetroundjoin%
\definecolor{currentfill}{rgb}{1.000000,0.498039,0.054902}%
\pgfsetfillcolor{currentfill}%
\pgfsetlinewidth{1.003750pt}%
\definecolor{currentstroke}{rgb}{1.000000,0.498039,0.054902}%
\pgfsetstrokecolor{currentstroke}%
\pgfsetdash{}{0pt}%
\pgfpathmoveto{\pgfqpoint{4.544260in}{2.590352in}}%
\pgfpathcurveto{\pgfqpoint{4.555310in}{2.590352in}}{\pgfqpoint{4.565909in}{2.594742in}}{\pgfqpoint{4.573723in}{2.602556in}}%
\pgfpathcurveto{\pgfqpoint{4.581536in}{2.610369in}}{\pgfqpoint{4.585927in}{2.620968in}}{\pgfqpoint{4.585927in}{2.632018in}}%
\pgfpathcurveto{\pgfqpoint{4.585927in}{2.643068in}}{\pgfqpoint{4.581536in}{2.653668in}}{\pgfqpoint{4.573723in}{2.661481in}}%
\pgfpathcurveto{\pgfqpoint{4.565909in}{2.669295in}}{\pgfqpoint{4.555310in}{2.673685in}}{\pgfqpoint{4.544260in}{2.673685in}}%
\pgfpathcurveto{\pgfqpoint{4.533210in}{2.673685in}}{\pgfqpoint{4.522611in}{2.669295in}}{\pgfqpoint{4.514797in}{2.661481in}}%
\pgfpathcurveto{\pgfqpoint{4.506984in}{2.653668in}}{\pgfqpoint{4.502593in}{2.643068in}}{\pgfqpoint{4.502593in}{2.632018in}}%
\pgfpathcurveto{\pgfqpoint{4.502593in}{2.620968in}}{\pgfqpoint{4.506984in}{2.610369in}}{\pgfqpoint{4.514797in}{2.602556in}}%
\pgfpathcurveto{\pgfqpoint{4.522611in}{2.594742in}}{\pgfqpoint{4.533210in}{2.590352in}}{\pgfqpoint{4.544260in}{2.590352in}}%
\pgfpathclose%
\pgfusepath{stroke,fill}%
\end{pgfscope}%
\begin{pgfscope}%
\pgfpathrectangle{\pgfqpoint{0.787074in}{0.548769in}}{\pgfqpoint{4.974523in}{3.102590in}}%
\pgfusepath{clip}%
\pgfsetbuttcap%
\pgfsetroundjoin%
\definecolor{currentfill}{rgb}{1.000000,0.498039,0.054902}%
\pgfsetfillcolor{currentfill}%
\pgfsetlinewidth{1.003750pt}%
\definecolor{currentstroke}{rgb}{1.000000,0.498039,0.054902}%
\pgfsetstrokecolor{currentstroke}%
\pgfsetdash{}{0pt}%
\pgfpathmoveto{\pgfqpoint{4.247359in}{2.993235in}}%
\pgfpathcurveto{\pgfqpoint{4.258409in}{2.993235in}}{\pgfqpoint{4.269008in}{2.997625in}}{\pgfqpoint{4.276821in}{3.005439in}}%
\pgfpathcurveto{\pgfqpoint{4.284635in}{3.013252in}}{\pgfqpoint{4.289025in}{3.023851in}}{\pgfqpoint{4.289025in}{3.034902in}}%
\pgfpathcurveto{\pgfqpoint{4.289025in}{3.045952in}}{\pgfqpoint{4.284635in}{3.056551in}}{\pgfqpoint{4.276821in}{3.064364in}}%
\pgfpathcurveto{\pgfqpoint{4.269008in}{3.072178in}}{\pgfqpoint{4.258409in}{3.076568in}}{\pgfqpoint{4.247359in}{3.076568in}}%
\pgfpathcurveto{\pgfqpoint{4.236308in}{3.076568in}}{\pgfqpoint{4.225709in}{3.072178in}}{\pgfqpoint{4.217896in}{3.064364in}}%
\pgfpathcurveto{\pgfqpoint{4.210082in}{3.056551in}}{\pgfqpoint{4.205692in}{3.045952in}}{\pgfqpoint{4.205692in}{3.034902in}}%
\pgfpathcurveto{\pgfqpoint{4.205692in}{3.023851in}}{\pgfqpoint{4.210082in}{3.013252in}}{\pgfqpoint{4.217896in}{3.005439in}}%
\pgfpathcurveto{\pgfqpoint{4.225709in}{2.997625in}}{\pgfqpoint{4.236308in}{2.993235in}}{\pgfqpoint{4.247359in}{2.993235in}}%
\pgfpathclose%
\pgfusepath{stroke,fill}%
\end{pgfscope}%
\begin{pgfscope}%
\pgfpathrectangle{\pgfqpoint{0.787074in}{0.548769in}}{\pgfqpoint{4.974523in}{3.102590in}}%
\pgfusepath{clip}%
\pgfsetbuttcap%
\pgfsetroundjoin%
\definecolor{currentfill}{rgb}{0.121569,0.466667,0.705882}%
\pgfsetfillcolor{currentfill}%
\pgfsetlinewidth{1.003750pt}%
\definecolor{currentstroke}{rgb}{0.121569,0.466667,0.705882}%
\pgfsetstrokecolor{currentstroke}%
\pgfsetdash{}{0pt}%
\pgfpathmoveto{\pgfqpoint{1.035868in}{0.666848in}}%
\pgfpathcurveto{\pgfqpoint{1.046918in}{0.666848in}}{\pgfqpoint{1.057517in}{0.671238in}}{\pgfqpoint{1.065330in}{0.679052in}}%
\pgfpathcurveto{\pgfqpoint{1.073144in}{0.686866in}}{\pgfqpoint{1.077534in}{0.697465in}}{\pgfqpoint{1.077534in}{0.708515in}}%
\pgfpathcurveto{\pgfqpoint{1.077534in}{0.719565in}}{\pgfqpoint{1.073144in}{0.730164in}}{\pgfqpoint{1.065330in}{0.737978in}}%
\pgfpathcurveto{\pgfqpoint{1.057517in}{0.745791in}}{\pgfqpoint{1.046918in}{0.750182in}}{\pgfqpoint{1.035868in}{0.750182in}}%
\pgfpathcurveto{\pgfqpoint{1.024817in}{0.750182in}}{\pgfqpoint{1.014218in}{0.745791in}}{\pgfqpoint{1.006405in}{0.737978in}}%
\pgfpathcurveto{\pgfqpoint{0.998591in}{0.730164in}}{\pgfqpoint{0.994201in}{0.719565in}}{\pgfqpoint{0.994201in}{0.708515in}}%
\pgfpathcurveto{\pgfqpoint{0.994201in}{0.697465in}}{\pgfqpoint{0.998591in}{0.686866in}}{\pgfqpoint{1.006405in}{0.679052in}}%
\pgfpathcurveto{\pgfqpoint{1.014218in}{0.671238in}}{\pgfqpoint{1.024817in}{0.666848in}}{\pgfqpoint{1.035868in}{0.666848in}}%
\pgfpathclose%
\pgfusepath{stroke,fill}%
\end{pgfscope}%
\begin{pgfscope}%
\pgfpathrectangle{\pgfqpoint{0.787074in}{0.548769in}}{\pgfqpoint{4.974523in}{3.102590in}}%
\pgfusepath{clip}%
\pgfsetbuttcap%
\pgfsetroundjoin%
\definecolor{currentfill}{rgb}{1.000000,0.498039,0.054902}%
\pgfsetfillcolor{currentfill}%
\pgfsetlinewidth{1.003750pt}%
\definecolor{currentstroke}{rgb}{1.000000,0.498039,0.054902}%
\pgfsetstrokecolor{currentstroke}%
\pgfsetdash{}{0pt}%
\pgfpathmoveto{\pgfqpoint{4.501396in}{2.802044in}}%
\pgfpathcurveto{\pgfqpoint{4.512446in}{2.802044in}}{\pgfqpoint{4.523045in}{2.806435in}}{\pgfqpoint{4.530859in}{2.814248in}}%
\pgfpathcurveto{\pgfqpoint{4.538672in}{2.822062in}}{\pgfqpoint{4.543063in}{2.832661in}}{\pgfqpoint{4.543063in}{2.843711in}}%
\pgfpathcurveto{\pgfqpoint{4.543063in}{2.854761in}}{\pgfqpoint{4.538672in}{2.865360in}}{\pgfqpoint{4.530859in}{2.873174in}}%
\pgfpathcurveto{\pgfqpoint{4.523045in}{2.880987in}}{\pgfqpoint{4.512446in}{2.885378in}}{\pgfqpoint{4.501396in}{2.885378in}}%
\pgfpathcurveto{\pgfqpoint{4.490346in}{2.885378in}}{\pgfqpoint{4.479747in}{2.880987in}}{\pgfqpoint{4.471933in}{2.873174in}}%
\pgfpathcurveto{\pgfqpoint{4.464120in}{2.865360in}}{\pgfqpoint{4.459729in}{2.854761in}}{\pgfqpoint{4.459729in}{2.843711in}}%
\pgfpathcurveto{\pgfqpoint{4.459729in}{2.832661in}}{\pgfqpoint{4.464120in}{2.822062in}}{\pgfqpoint{4.471933in}{2.814248in}}%
\pgfpathcurveto{\pgfqpoint{4.479747in}{2.806435in}}{\pgfqpoint{4.490346in}{2.802044in}}{\pgfqpoint{4.501396in}{2.802044in}}%
\pgfpathclose%
\pgfusepath{stroke,fill}%
\end{pgfscope}%
\begin{pgfscope}%
\pgfpathrectangle{\pgfqpoint{0.787074in}{0.548769in}}{\pgfqpoint{4.974523in}{3.102590in}}%
\pgfusepath{clip}%
\pgfsetbuttcap%
\pgfsetroundjoin%
\definecolor{currentfill}{rgb}{0.121569,0.466667,0.705882}%
\pgfsetfillcolor{currentfill}%
\pgfsetlinewidth{1.003750pt}%
\definecolor{currentstroke}{rgb}{0.121569,0.466667,0.705882}%
\pgfsetstrokecolor{currentstroke}%
\pgfsetdash{}{0pt}%
\pgfpathmoveto{\pgfqpoint{1.005599in}{0.648244in}}%
\pgfpathcurveto{\pgfqpoint{1.016649in}{0.648244in}}{\pgfqpoint{1.027249in}{0.652634in}}{\pgfqpoint{1.035062in}{0.660448in}}%
\pgfpathcurveto{\pgfqpoint{1.042876in}{0.668261in}}{\pgfqpoint{1.047266in}{0.678860in}}{\pgfqpoint{1.047266in}{0.689911in}}%
\pgfpathcurveto{\pgfqpoint{1.047266in}{0.700961in}}{\pgfqpoint{1.042876in}{0.711560in}}{\pgfqpoint{1.035062in}{0.719373in}}%
\pgfpathcurveto{\pgfqpoint{1.027249in}{0.727187in}}{\pgfqpoint{1.016649in}{0.731577in}}{\pgfqpoint{1.005599in}{0.731577in}}%
\pgfpathcurveto{\pgfqpoint{0.994549in}{0.731577in}}{\pgfqpoint{0.983950in}{0.727187in}}{\pgfqpoint{0.976137in}{0.719373in}}%
\pgfpathcurveto{\pgfqpoint{0.968323in}{0.711560in}}{\pgfqpoint{0.963933in}{0.700961in}}{\pgfqpoint{0.963933in}{0.689911in}}%
\pgfpathcurveto{\pgfqpoint{0.963933in}{0.678860in}}{\pgfqpoint{0.968323in}{0.668261in}}{\pgfqpoint{0.976137in}{0.660448in}}%
\pgfpathcurveto{\pgfqpoint{0.983950in}{0.652634in}}{\pgfqpoint{0.994549in}{0.648244in}}{\pgfqpoint{1.005599in}{0.648244in}}%
\pgfpathclose%
\pgfusepath{stroke,fill}%
\end{pgfscope}%
\begin{pgfscope}%
\pgfpathrectangle{\pgfqpoint{0.787074in}{0.548769in}}{\pgfqpoint{4.974523in}{3.102590in}}%
\pgfusepath{clip}%
\pgfsetbuttcap%
\pgfsetroundjoin%
\definecolor{currentfill}{rgb}{1.000000,0.498039,0.054902}%
\pgfsetfillcolor{currentfill}%
\pgfsetlinewidth{1.003750pt}%
\definecolor{currentstroke}{rgb}{1.000000,0.498039,0.054902}%
\pgfsetstrokecolor{currentstroke}%
\pgfsetdash{}{0pt}%
\pgfpathmoveto{\pgfqpoint{4.303069in}{3.229318in}}%
\pgfpathcurveto{\pgfqpoint{4.314120in}{3.229318in}}{\pgfqpoint{4.324719in}{3.233708in}}{\pgfqpoint{4.332532in}{3.241522in}}%
\pgfpathcurveto{\pgfqpoint{4.340346in}{3.249335in}}{\pgfqpoint{4.344736in}{3.259934in}}{\pgfqpoint{4.344736in}{3.270984in}}%
\pgfpathcurveto{\pgfqpoint{4.344736in}{3.282035in}}{\pgfqpoint{4.340346in}{3.292634in}}{\pgfqpoint{4.332532in}{3.300447in}}%
\pgfpathcurveto{\pgfqpoint{4.324719in}{3.308261in}}{\pgfqpoint{4.314120in}{3.312651in}}{\pgfqpoint{4.303069in}{3.312651in}}%
\pgfpathcurveto{\pgfqpoint{4.292019in}{3.312651in}}{\pgfqpoint{4.281420in}{3.308261in}}{\pgfqpoint{4.273607in}{3.300447in}}%
\pgfpathcurveto{\pgfqpoint{4.265793in}{3.292634in}}{\pgfqpoint{4.261403in}{3.282035in}}{\pgfqpoint{4.261403in}{3.270984in}}%
\pgfpathcurveto{\pgfqpoint{4.261403in}{3.259934in}}{\pgfqpoint{4.265793in}{3.249335in}}{\pgfqpoint{4.273607in}{3.241522in}}%
\pgfpathcurveto{\pgfqpoint{4.281420in}{3.233708in}}{\pgfqpoint{4.292019in}{3.229318in}}{\pgfqpoint{4.303069in}{3.229318in}}%
\pgfpathclose%
\pgfusepath{stroke,fill}%
\end{pgfscope}%
\begin{pgfscope}%
\pgfpathrectangle{\pgfqpoint{0.787074in}{0.548769in}}{\pgfqpoint{4.974523in}{3.102590in}}%
\pgfusepath{clip}%
\pgfsetbuttcap%
\pgfsetroundjoin%
\definecolor{currentfill}{rgb}{0.121569,0.466667,0.705882}%
\pgfsetfillcolor{currentfill}%
\pgfsetlinewidth{1.003750pt}%
\definecolor{currentstroke}{rgb}{0.121569,0.466667,0.705882}%
\pgfsetstrokecolor{currentstroke}%
\pgfsetdash{}{0pt}%
\pgfpathmoveto{\pgfqpoint{4.277003in}{2.961491in}}%
\pgfpathcurveto{\pgfqpoint{4.288053in}{2.961491in}}{\pgfqpoint{4.298652in}{2.965882in}}{\pgfqpoint{4.306465in}{2.973695in}}%
\pgfpathcurveto{\pgfqpoint{4.314279in}{2.981509in}}{\pgfqpoint{4.318669in}{2.992108in}}{\pgfqpoint{4.318669in}{3.003158in}}%
\pgfpathcurveto{\pgfqpoint{4.318669in}{3.014208in}}{\pgfqpoint{4.314279in}{3.024807in}}{\pgfqpoint{4.306465in}{3.032621in}}%
\pgfpathcurveto{\pgfqpoint{4.298652in}{3.040434in}}{\pgfqpoint{4.288053in}{3.044825in}}{\pgfqpoint{4.277003in}{3.044825in}}%
\pgfpathcurveto{\pgfqpoint{4.265953in}{3.044825in}}{\pgfqpoint{4.255354in}{3.040434in}}{\pgfqpoint{4.247540in}{3.032621in}}%
\pgfpathcurveto{\pgfqpoint{4.239726in}{3.024807in}}{\pgfqpoint{4.235336in}{3.014208in}}{\pgfqpoint{4.235336in}{3.003158in}}%
\pgfpathcurveto{\pgfqpoint{4.235336in}{2.992108in}}{\pgfqpoint{4.239726in}{2.981509in}}{\pgfqpoint{4.247540in}{2.973695in}}%
\pgfpathcurveto{\pgfqpoint{4.255354in}{2.965882in}}{\pgfqpoint{4.265953in}{2.961491in}}{\pgfqpoint{4.277003in}{2.961491in}}%
\pgfpathclose%
\pgfusepath{stroke,fill}%
\end{pgfscope}%
\begin{pgfscope}%
\pgfpathrectangle{\pgfqpoint{0.787074in}{0.548769in}}{\pgfqpoint{4.974523in}{3.102590in}}%
\pgfusepath{clip}%
\pgfsetbuttcap%
\pgfsetroundjoin%
\definecolor{currentfill}{rgb}{1.000000,0.498039,0.054902}%
\pgfsetfillcolor{currentfill}%
\pgfsetlinewidth{1.003750pt}%
\definecolor{currentstroke}{rgb}{1.000000,0.498039,0.054902}%
\pgfsetstrokecolor{currentstroke}%
\pgfsetdash{}{0pt}%
\pgfpathmoveto{\pgfqpoint{4.151604in}{2.689616in}}%
\pgfpathcurveto{\pgfqpoint{4.162654in}{2.689616in}}{\pgfqpoint{4.173253in}{2.694006in}}{\pgfqpoint{4.181067in}{2.701820in}}%
\pgfpathcurveto{\pgfqpoint{4.188880in}{2.709634in}}{\pgfqpoint{4.193270in}{2.720233in}}{\pgfqpoint{4.193270in}{2.731283in}}%
\pgfpathcurveto{\pgfqpoint{4.193270in}{2.742333in}}{\pgfqpoint{4.188880in}{2.752932in}}{\pgfqpoint{4.181067in}{2.760746in}}%
\pgfpathcurveto{\pgfqpoint{4.173253in}{2.768559in}}{\pgfqpoint{4.162654in}{2.772950in}}{\pgfqpoint{4.151604in}{2.772950in}}%
\pgfpathcurveto{\pgfqpoint{4.140554in}{2.772950in}}{\pgfqpoint{4.129955in}{2.768559in}}{\pgfqpoint{4.122141in}{2.760746in}}%
\pgfpathcurveto{\pgfqpoint{4.114327in}{2.752932in}}{\pgfqpoint{4.109937in}{2.742333in}}{\pgfqpoint{4.109937in}{2.731283in}}%
\pgfpathcurveto{\pgfqpoint{4.109937in}{2.720233in}}{\pgfqpoint{4.114327in}{2.709634in}}{\pgfqpoint{4.122141in}{2.701820in}}%
\pgfpathcurveto{\pgfqpoint{4.129955in}{2.694006in}}{\pgfqpoint{4.140554in}{2.689616in}}{\pgfqpoint{4.151604in}{2.689616in}}%
\pgfpathclose%
\pgfusepath{stroke,fill}%
\end{pgfscope}%
\begin{pgfscope}%
\pgfpathrectangle{\pgfqpoint{0.787074in}{0.548769in}}{\pgfqpoint{4.974523in}{3.102590in}}%
\pgfusepath{clip}%
\pgfsetbuttcap%
\pgfsetroundjoin%
\definecolor{currentfill}{rgb}{1.000000,0.498039,0.054902}%
\pgfsetfillcolor{currentfill}%
\pgfsetlinewidth{1.003750pt}%
\definecolor{currentstroke}{rgb}{1.000000,0.498039,0.054902}%
\pgfsetstrokecolor{currentstroke}%
\pgfsetdash{}{0pt}%
\pgfpathmoveto{\pgfqpoint{2.690867in}{1.853687in}}%
\pgfpathcurveto{\pgfqpoint{2.701917in}{1.853687in}}{\pgfqpoint{2.712516in}{1.858077in}}{\pgfqpoint{2.720329in}{1.865891in}}%
\pgfpathcurveto{\pgfqpoint{2.728143in}{1.873705in}}{\pgfqpoint{2.732533in}{1.884304in}}{\pgfqpoint{2.732533in}{1.895354in}}%
\pgfpathcurveto{\pgfqpoint{2.732533in}{1.906404in}}{\pgfqpoint{2.728143in}{1.917003in}}{\pgfqpoint{2.720329in}{1.924817in}}%
\pgfpathcurveto{\pgfqpoint{2.712516in}{1.932630in}}{\pgfqpoint{2.701917in}{1.937021in}}{\pgfqpoint{2.690867in}{1.937021in}}%
\pgfpathcurveto{\pgfqpoint{2.679817in}{1.937021in}}{\pgfqpoint{2.669218in}{1.932630in}}{\pgfqpoint{2.661404in}{1.924817in}}%
\pgfpathcurveto{\pgfqpoint{2.653590in}{1.917003in}}{\pgfqpoint{2.649200in}{1.906404in}}{\pgfqpoint{2.649200in}{1.895354in}}%
\pgfpathcurveto{\pgfqpoint{2.649200in}{1.884304in}}{\pgfqpoint{2.653590in}{1.873705in}}{\pgfqpoint{2.661404in}{1.865891in}}%
\pgfpathcurveto{\pgfqpoint{2.669218in}{1.858077in}}{\pgfqpoint{2.679817in}{1.853687in}}{\pgfqpoint{2.690867in}{1.853687in}}%
\pgfpathclose%
\pgfusepath{stroke,fill}%
\end{pgfscope}%
\begin{pgfscope}%
\pgfpathrectangle{\pgfqpoint{0.787074in}{0.548769in}}{\pgfqpoint{4.974523in}{3.102590in}}%
\pgfusepath{clip}%
\pgfsetbuttcap%
\pgfsetroundjoin%
\definecolor{currentfill}{rgb}{1.000000,0.498039,0.054902}%
\pgfsetfillcolor{currentfill}%
\pgfsetlinewidth{1.003750pt}%
\definecolor{currentstroke}{rgb}{1.000000,0.498039,0.054902}%
\pgfsetstrokecolor{currentstroke}%
\pgfsetdash{}{0pt}%
\pgfpathmoveto{\pgfqpoint{4.017183in}{2.416725in}}%
\pgfpathcurveto{\pgfqpoint{4.028233in}{2.416725in}}{\pgfqpoint{4.038832in}{2.421116in}}{\pgfqpoint{4.046646in}{2.428929in}}%
\pgfpathcurveto{\pgfqpoint{4.054460in}{2.436743in}}{\pgfqpoint{4.058850in}{2.447342in}}{\pgfqpoint{4.058850in}{2.458392in}}%
\pgfpathcurveto{\pgfqpoint{4.058850in}{2.469442in}}{\pgfqpoint{4.054460in}{2.480041in}}{\pgfqpoint{4.046646in}{2.487855in}}%
\pgfpathcurveto{\pgfqpoint{4.038832in}{2.495669in}}{\pgfqpoint{4.028233in}{2.500059in}}{\pgfqpoint{4.017183in}{2.500059in}}%
\pgfpathcurveto{\pgfqpoint{4.006133in}{2.500059in}}{\pgfqpoint{3.995534in}{2.495669in}}{\pgfqpoint{3.987721in}{2.487855in}}%
\pgfpathcurveto{\pgfqpoint{3.979907in}{2.480041in}}{\pgfqpoint{3.975517in}{2.469442in}}{\pgfqpoint{3.975517in}{2.458392in}}%
\pgfpathcurveto{\pgfqpoint{3.975517in}{2.447342in}}{\pgfqpoint{3.979907in}{2.436743in}}{\pgfqpoint{3.987721in}{2.428929in}}%
\pgfpathcurveto{\pgfqpoint{3.995534in}{2.421116in}}{\pgfqpoint{4.006133in}{2.416725in}}{\pgfqpoint{4.017183in}{2.416725in}}%
\pgfpathclose%
\pgfusepath{stroke,fill}%
\end{pgfscope}%
\begin{pgfscope}%
\pgfpathrectangle{\pgfqpoint{0.787074in}{0.548769in}}{\pgfqpoint{4.974523in}{3.102590in}}%
\pgfusepath{clip}%
\pgfsetbuttcap%
\pgfsetroundjoin%
\definecolor{currentfill}{rgb}{0.121569,0.466667,0.705882}%
\pgfsetfillcolor{currentfill}%
\pgfsetlinewidth{1.003750pt}%
\definecolor{currentstroke}{rgb}{0.121569,0.466667,0.705882}%
\pgfsetstrokecolor{currentstroke}%
\pgfsetdash{}{0pt}%
\pgfpathmoveto{\pgfqpoint{3.036499in}{2.410505in}}%
\pgfpathcurveto{\pgfqpoint{3.047549in}{2.410505in}}{\pgfqpoint{3.058148in}{2.414896in}}{\pgfqpoint{3.065962in}{2.422709in}}%
\pgfpathcurveto{\pgfqpoint{3.073775in}{2.430523in}}{\pgfqpoint{3.078165in}{2.441122in}}{\pgfqpoint{3.078165in}{2.452172in}}%
\pgfpathcurveto{\pgfqpoint{3.078165in}{2.463222in}}{\pgfqpoint{3.073775in}{2.473821in}}{\pgfqpoint{3.065962in}{2.481635in}}%
\pgfpathcurveto{\pgfqpoint{3.058148in}{2.489449in}}{\pgfqpoint{3.047549in}{2.493839in}}{\pgfqpoint{3.036499in}{2.493839in}}%
\pgfpathcurveto{\pgfqpoint{3.025449in}{2.493839in}}{\pgfqpoint{3.014850in}{2.489449in}}{\pgfqpoint{3.007036in}{2.481635in}}%
\pgfpathcurveto{\pgfqpoint{2.999222in}{2.473821in}}{\pgfqpoint{2.994832in}{2.463222in}}{\pgfqpoint{2.994832in}{2.452172in}}%
\pgfpathcurveto{\pgfqpoint{2.994832in}{2.441122in}}{\pgfqpoint{2.999222in}{2.430523in}}{\pgfqpoint{3.007036in}{2.422709in}}%
\pgfpathcurveto{\pgfqpoint{3.014850in}{2.414896in}}{\pgfqpoint{3.025449in}{2.410505in}}{\pgfqpoint{3.036499in}{2.410505in}}%
\pgfpathclose%
\pgfusepath{stroke,fill}%
\end{pgfscope}%
\begin{pgfscope}%
\pgfpathrectangle{\pgfqpoint{0.787074in}{0.548769in}}{\pgfqpoint{4.974523in}{3.102590in}}%
\pgfusepath{clip}%
\pgfsetbuttcap%
\pgfsetroundjoin%
\definecolor{currentfill}{rgb}{0.121569,0.466667,0.705882}%
\pgfsetfillcolor{currentfill}%
\pgfsetlinewidth{1.003750pt}%
\definecolor{currentstroke}{rgb}{0.121569,0.466667,0.705882}%
\pgfsetstrokecolor{currentstroke}%
\pgfsetdash{}{0pt}%
\pgfpathmoveto{\pgfqpoint{1.054970in}{0.678421in}}%
\pgfpathcurveto{\pgfqpoint{1.066020in}{0.678421in}}{\pgfqpoint{1.076619in}{0.682812in}}{\pgfqpoint{1.084433in}{0.690625in}}%
\pgfpathcurveto{\pgfqpoint{1.092247in}{0.698439in}}{\pgfqpoint{1.096637in}{0.709038in}}{\pgfqpoint{1.096637in}{0.720088in}}%
\pgfpathcurveto{\pgfqpoint{1.096637in}{0.731138in}}{\pgfqpoint{1.092247in}{0.741737in}}{\pgfqpoint{1.084433in}{0.749551in}}%
\pgfpathcurveto{\pgfqpoint{1.076619in}{0.757364in}}{\pgfqpoint{1.066020in}{0.761755in}}{\pgfqpoint{1.054970in}{0.761755in}}%
\pgfpathcurveto{\pgfqpoint{1.043920in}{0.761755in}}{\pgfqpoint{1.033321in}{0.757364in}}{\pgfqpoint{1.025507in}{0.749551in}}%
\pgfpathcurveto{\pgfqpoint{1.017694in}{0.741737in}}{\pgfqpoint{1.013304in}{0.731138in}}{\pgfqpoint{1.013304in}{0.720088in}}%
\pgfpathcurveto{\pgfqpoint{1.013304in}{0.709038in}}{\pgfqpoint{1.017694in}{0.698439in}}{\pgfqpoint{1.025507in}{0.690625in}}%
\pgfpathcurveto{\pgfqpoint{1.033321in}{0.682812in}}{\pgfqpoint{1.043920in}{0.678421in}}{\pgfqpoint{1.054970in}{0.678421in}}%
\pgfpathclose%
\pgfusepath{stroke,fill}%
\end{pgfscope}%
\begin{pgfscope}%
\pgfpathrectangle{\pgfqpoint{0.787074in}{0.548769in}}{\pgfqpoint{4.974523in}{3.102590in}}%
\pgfusepath{clip}%
\pgfsetbuttcap%
\pgfsetroundjoin%
\definecolor{currentfill}{rgb}{1.000000,0.498039,0.054902}%
\pgfsetfillcolor{currentfill}%
\pgfsetlinewidth{1.003750pt}%
\definecolor{currentstroke}{rgb}{1.000000,0.498039,0.054902}%
\pgfsetstrokecolor{currentstroke}%
\pgfsetdash{}{0pt}%
\pgfpathmoveto{\pgfqpoint{4.334093in}{2.587414in}}%
\pgfpathcurveto{\pgfqpoint{4.345143in}{2.587414in}}{\pgfqpoint{4.355742in}{2.591804in}}{\pgfqpoint{4.363555in}{2.599618in}}%
\pgfpathcurveto{\pgfqpoint{4.371369in}{2.607432in}}{\pgfqpoint{4.375759in}{2.618031in}}{\pgfqpoint{4.375759in}{2.629081in}}%
\pgfpathcurveto{\pgfqpoint{4.375759in}{2.640131in}}{\pgfqpoint{4.371369in}{2.650730in}}{\pgfqpoint{4.363555in}{2.658544in}}%
\pgfpathcurveto{\pgfqpoint{4.355742in}{2.666357in}}{\pgfqpoint{4.345143in}{2.670747in}}{\pgfqpoint{4.334093in}{2.670747in}}%
\pgfpathcurveto{\pgfqpoint{4.323042in}{2.670747in}}{\pgfqpoint{4.312443in}{2.666357in}}{\pgfqpoint{4.304630in}{2.658544in}}%
\pgfpathcurveto{\pgfqpoint{4.296816in}{2.650730in}}{\pgfqpoint{4.292426in}{2.640131in}}{\pgfqpoint{4.292426in}{2.629081in}}%
\pgfpathcurveto{\pgfqpoint{4.292426in}{2.618031in}}{\pgfqpoint{4.296816in}{2.607432in}}{\pgfqpoint{4.304630in}{2.599618in}}%
\pgfpathcurveto{\pgfqpoint{4.312443in}{2.591804in}}{\pgfqpoint{4.323042in}{2.587414in}}{\pgfqpoint{4.334093in}{2.587414in}}%
\pgfpathclose%
\pgfusepath{stroke,fill}%
\end{pgfscope}%
\begin{pgfscope}%
\pgfpathrectangle{\pgfqpoint{0.787074in}{0.548769in}}{\pgfqpoint{4.974523in}{3.102590in}}%
\pgfusepath{clip}%
\pgfsetbuttcap%
\pgfsetroundjoin%
\definecolor{currentfill}{rgb}{1.000000,0.498039,0.054902}%
\pgfsetfillcolor{currentfill}%
\pgfsetlinewidth{1.003750pt}%
\definecolor{currentstroke}{rgb}{1.000000,0.498039,0.054902}%
\pgfsetstrokecolor{currentstroke}%
\pgfsetdash{}{0pt}%
\pgfpathmoveto{\pgfqpoint{4.286560in}{3.097620in}}%
\pgfpathcurveto{\pgfqpoint{4.297611in}{3.097620in}}{\pgfqpoint{4.308210in}{3.102011in}}{\pgfqpoint{4.316023in}{3.109824in}}%
\pgfpathcurveto{\pgfqpoint{4.323837in}{3.117638in}}{\pgfqpoint{4.328227in}{3.128237in}}{\pgfqpoint{4.328227in}{3.139287in}}%
\pgfpathcurveto{\pgfqpoint{4.328227in}{3.150337in}}{\pgfqpoint{4.323837in}{3.160936in}}{\pgfqpoint{4.316023in}{3.168750in}}%
\pgfpathcurveto{\pgfqpoint{4.308210in}{3.176564in}}{\pgfqpoint{4.297611in}{3.180954in}}{\pgfqpoint{4.286560in}{3.180954in}}%
\pgfpathcurveto{\pgfqpoint{4.275510in}{3.180954in}}{\pgfqpoint{4.264911in}{3.176564in}}{\pgfqpoint{4.257098in}{3.168750in}}%
\pgfpathcurveto{\pgfqpoint{4.249284in}{3.160936in}}{\pgfqpoint{4.244894in}{3.150337in}}{\pgfqpoint{4.244894in}{3.139287in}}%
\pgfpathcurveto{\pgfqpoint{4.244894in}{3.128237in}}{\pgfqpoint{4.249284in}{3.117638in}}{\pgfqpoint{4.257098in}{3.109824in}}%
\pgfpathcurveto{\pgfqpoint{4.264911in}{3.102011in}}{\pgfqpoint{4.275510in}{3.097620in}}{\pgfqpoint{4.286560in}{3.097620in}}%
\pgfpathclose%
\pgfusepath{stroke,fill}%
\end{pgfscope}%
\begin{pgfscope}%
\pgfpathrectangle{\pgfqpoint{0.787074in}{0.548769in}}{\pgfqpoint{4.974523in}{3.102590in}}%
\pgfusepath{clip}%
\pgfsetbuttcap%
\pgfsetroundjoin%
\definecolor{currentfill}{rgb}{0.121569,0.466667,0.705882}%
\pgfsetfillcolor{currentfill}%
\pgfsetlinewidth{1.003750pt}%
\definecolor{currentstroke}{rgb}{0.121569,0.466667,0.705882}%
\pgfsetstrokecolor{currentstroke}%
\pgfsetdash{}{0pt}%
\pgfpathmoveto{\pgfqpoint{1.005426in}{0.648139in}}%
\pgfpathcurveto{\pgfqpoint{1.016476in}{0.648139in}}{\pgfqpoint{1.027075in}{0.652529in}}{\pgfqpoint{1.034889in}{0.660343in}}%
\pgfpathcurveto{\pgfqpoint{1.042703in}{0.668156in}}{\pgfqpoint{1.047093in}{0.678755in}}{\pgfqpoint{1.047093in}{0.689806in}}%
\pgfpathcurveto{\pgfqpoint{1.047093in}{0.700856in}}{\pgfqpoint{1.042703in}{0.711455in}}{\pgfqpoint{1.034889in}{0.719268in}}%
\pgfpathcurveto{\pgfqpoint{1.027075in}{0.727082in}}{\pgfqpoint{1.016476in}{0.731472in}}{\pgfqpoint{1.005426in}{0.731472in}}%
\pgfpathcurveto{\pgfqpoint{0.994376in}{0.731472in}}{\pgfqpoint{0.983777in}{0.727082in}}{\pgfqpoint{0.975963in}{0.719268in}}%
\pgfpathcurveto{\pgfqpoint{0.968150in}{0.711455in}}{\pgfqpoint{0.963760in}{0.700856in}}{\pgfqpoint{0.963760in}{0.689806in}}%
\pgfpathcurveto{\pgfqpoint{0.963760in}{0.678755in}}{\pgfqpoint{0.968150in}{0.668156in}}{\pgfqpoint{0.975963in}{0.660343in}}%
\pgfpathcurveto{\pgfqpoint{0.983777in}{0.652529in}}{\pgfqpoint{0.994376in}{0.648139in}}{\pgfqpoint{1.005426in}{0.648139in}}%
\pgfpathclose%
\pgfusepath{stroke,fill}%
\end{pgfscope}%
\begin{pgfscope}%
\pgfpathrectangle{\pgfqpoint{0.787074in}{0.548769in}}{\pgfqpoint{4.974523in}{3.102590in}}%
\pgfusepath{clip}%
\pgfsetbuttcap%
\pgfsetroundjoin%
\definecolor{currentfill}{rgb}{0.121569,0.466667,0.705882}%
\pgfsetfillcolor{currentfill}%
\pgfsetlinewidth{1.003750pt}%
\definecolor{currentstroke}{rgb}{0.121569,0.466667,0.705882}%
\pgfsetstrokecolor{currentstroke}%
\pgfsetdash{}{0pt}%
\pgfpathmoveto{\pgfqpoint{1.008601in}{0.649969in}}%
\pgfpathcurveto{\pgfqpoint{1.019651in}{0.649969in}}{\pgfqpoint{1.030250in}{0.654360in}}{\pgfqpoint{1.038064in}{0.662173in}}%
\pgfpathcurveto{\pgfqpoint{1.045877in}{0.669987in}}{\pgfqpoint{1.050267in}{0.680586in}}{\pgfqpoint{1.050267in}{0.691636in}}%
\pgfpathcurveto{\pgfqpoint{1.050267in}{0.702686in}}{\pgfqpoint{1.045877in}{0.713285in}}{\pgfqpoint{1.038064in}{0.721099in}}%
\pgfpathcurveto{\pgfqpoint{1.030250in}{0.728913in}}{\pgfqpoint{1.019651in}{0.733303in}}{\pgfqpoint{1.008601in}{0.733303in}}%
\pgfpathcurveto{\pgfqpoint{0.997551in}{0.733303in}}{\pgfqpoint{0.986952in}{0.728913in}}{\pgfqpoint{0.979138in}{0.721099in}}%
\pgfpathcurveto{\pgfqpoint{0.971324in}{0.713285in}}{\pgfqpoint{0.966934in}{0.702686in}}{\pgfqpoint{0.966934in}{0.691636in}}%
\pgfpathcurveto{\pgfqpoint{0.966934in}{0.680586in}}{\pgfqpoint{0.971324in}{0.669987in}}{\pgfqpoint{0.979138in}{0.662173in}}%
\pgfpathcurveto{\pgfqpoint{0.986952in}{0.654360in}}{\pgfqpoint{0.997551in}{0.649969in}}{\pgfqpoint{1.008601in}{0.649969in}}%
\pgfpathclose%
\pgfusepath{stroke,fill}%
\end{pgfscope}%
\begin{pgfscope}%
\pgfpathrectangle{\pgfqpoint{0.787074in}{0.548769in}}{\pgfqpoint{4.974523in}{3.102590in}}%
\pgfusepath{clip}%
\pgfsetbuttcap%
\pgfsetroundjoin%
\definecolor{currentfill}{rgb}{0.121569,0.466667,0.705882}%
\pgfsetfillcolor{currentfill}%
\pgfsetlinewidth{1.003750pt}%
\definecolor{currentstroke}{rgb}{0.121569,0.466667,0.705882}%
\pgfsetstrokecolor{currentstroke}%
\pgfsetdash{}{0pt}%
\pgfpathmoveto{\pgfqpoint{4.094451in}{2.881001in}}%
\pgfpathcurveto{\pgfqpoint{4.105501in}{2.881001in}}{\pgfqpoint{4.116100in}{2.885391in}}{\pgfqpoint{4.123914in}{2.893205in}}%
\pgfpathcurveto{\pgfqpoint{4.131727in}{2.901018in}}{\pgfqpoint{4.136118in}{2.911617in}}{\pgfqpoint{4.136118in}{2.922668in}}%
\pgfpathcurveto{\pgfqpoint{4.136118in}{2.933718in}}{\pgfqpoint{4.131727in}{2.944317in}}{\pgfqpoint{4.123914in}{2.952130in}}%
\pgfpathcurveto{\pgfqpoint{4.116100in}{2.959944in}}{\pgfqpoint{4.105501in}{2.964334in}}{\pgfqpoint{4.094451in}{2.964334in}}%
\pgfpathcurveto{\pgfqpoint{4.083401in}{2.964334in}}{\pgfqpoint{4.072802in}{2.959944in}}{\pgfqpoint{4.064988in}{2.952130in}}%
\pgfpathcurveto{\pgfqpoint{4.057175in}{2.944317in}}{\pgfqpoint{4.052784in}{2.933718in}}{\pgfqpoint{4.052784in}{2.922668in}}%
\pgfpathcurveto{\pgfqpoint{4.052784in}{2.911617in}}{\pgfqpoint{4.057175in}{2.901018in}}{\pgfqpoint{4.064988in}{2.893205in}}%
\pgfpathcurveto{\pgfqpoint{4.072802in}{2.885391in}}{\pgfqpoint{4.083401in}{2.881001in}}{\pgfqpoint{4.094451in}{2.881001in}}%
\pgfpathclose%
\pgfusepath{stroke,fill}%
\end{pgfscope}%
\begin{pgfscope}%
\pgfpathrectangle{\pgfqpoint{0.787074in}{0.548769in}}{\pgfqpoint{4.974523in}{3.102590in}}%
\pgfusepath{clip}%
\pgfsetbuttcap%
\pgfsetroundjoin%
\definecolor{currentfill}{rgb}{1.000000,0.498039,0.054902}%
\pgfsetfillcolor{currentfill}%
\pgfsetlinewidth{1.003750pt}%
\definecolor{currentstroke}{rgb}{1.000000,0.498039,0.054902}%
\pgfsetstrokecolor{currentstroke}%
\pgfsetdash{}{0pt}%
\pgfpathmoveto{\pgfqpoint{4.144209in}{3.082877in}}%
\pgfpathcurveto{\pgfqpoint{4.155259in}{3.082877in}}{\pgfqpoint{4.165858in}{3.087267in}}{\pgfqpoint{4.173671in}{3.095080in}}%
\pgfpathcurveto{\pgfqpoint{4.181485in}{3.102894in}}{\pgfqpoint{4.185875in}{3.113493in}}{\pgfqpoint{4.185875in}{3.124543in}}%
\pgfpathcurveto{\pgfqpoint{4.185875in}{3.135593in}}{\pgfqpoint{4.181485in}{3.146192in}}{\pgfqpoint{4.173671in}{3.154006in}}%
\pgfpathcurveto{\pgfqpoint{4.165858in}{3.161820in}}{\pgfqpoint{4.155259in}{3.166210in}}{\pgfqpoint{4.144209in}{3.166210in}}%
\pgfpathcurveto{\pgfqpoint{4.133158in}{3.166210in}}{\pgfqpoint{4.122559in}{3.161820in}}{\pgfqpoint{4.114746in}{3.154006in}}%
\pgfpathcurveto{\pgfqpoint{4.106932in}{3.146192in}}{\pgfqpoint{4.102542in}{3.135593in}}{\pgfqpoint{4.102542in}{3.124543in}}%
\pgfpathcurveto{\pgfqpoint{4.102542in}{3.113493in}}{\pgfqpoint{4.106932in}{3.102894in}}{\pgfqpoint{4.114746in}{3.095080in}}%
\pgfpathcurveto{\pgfqpoint{4.122559in}{3.087267in}}{\pgfqpoint{4.133158in}{3.082877in}}{\pgfqpoint{4.144209in}{3.082877in}}%
\pgfpathclose%
\pgfusepath{stroke,fill}%
\end{pgfscope}%
\begin{pgfscope}%
\pgfpathrectangle{\pgfqpoint{0.787074in}{0.548769in}}{\pgfqpoint{4.974523in}{3.102590in}}%
\pgfusepath{clip}%
\pgfsetbuttcap%
\pgfsetroundjoin%
\definecolor{currentfill}{rgb}{1.000000,0.498039,0.054902}%
\pgfsetfillcolor{currentfill}%
\pgfsetlinewidth{1.003750pt}%
\definecolor{currentstroke}{rgb}{1.000000,0.498039,0.054902}%
\pgfsetstrokecolor{currentstroke}%
\pgfsetdash{}{0pt}%
\pgfpathmoveto{\pgfqpoint{3.495434in}{2.563602in}}%
\pgfpathcurveto{\pgfqpoint{3.506484in}{2.563602in}}{\pgfqpoint{3.517083in}{2.567992in}}{\pgfqpoint{3.524897in}{2.575806in}}%
\pgfpathcurveto{\pgfqpoint{3.532711in}{2.583619in}}{\pgfqpoint{3.537101in}{2.594218in}}{\pgfqpoint{3.537101in}{2.605268in}}%
\pgfpathcurveto{\pgfqpoint{3.537101in}{2.616318in}}{\pgfqpoint{3.532711in}{2.626917in}}{\pgfqpoint{3.524897in}{2.634731in}}%
\pgfpathcurveto{\pgfqpoint{3.517083in}{2.642545in}}{\pgfqpoint{3.506484in}{2.646935in}}{\pgfqpoint{3.495434in}{2.646935in}}%
\pgfpathcurveto{\pgfqpoint{3.484384in}{2.646935in}}{\pgfqpoint{3.473785in}{2.642545in}}{\pgfqpoint{3.465971in}{2.634731in}}%
\pgfpathcurveto{\pgfqpoint{3.458158in}{2.626917in}}{\pgfqpoint{3.453768in}{2.616318in}}{\pgfqpoint{3.453768in}{2.605268in}}%
\pgfpathcurveto{\pgfqpoint{3.453768in}{2.594218in}}{\pgfqpoint{3.458158in}{2.583619in}}{\pgfqpoint{3.465971in}{2.575806in}}%
\pgfpathcurveto{\pgfqpoint{3.473785in}{2.567992in}}{\pgfqpoint{3.484384in}{2.563602in}}{\pgfqpoint{3.495434in}{2.563602in}}%
\pgfpathclose%
\pgfusepath{stroke,fill}%
\end{pgfscope}%
\begin{pgfscope}%
\pgfpathrectangle{\pgfqpoint{0.787074in}{0.548769in}}{\pgfqpoint{4.974523in}{3.102590in}}%
\pgfusepath{clip}%
\pgfsetbuttcap%
\pgfsetroundjoin%
\definecolor{currentfill}{rgb}{1.000000,0.498039,0.054902}%
\pgfsetfillcolor{currentfill}%
\pgfsetlinewidth{1.003750pt}%
\definecolor{currentstroke}{rgb}{1.000000,0.498039,0.054902}%
\pgfsetstrokecolor{currentstroke}%
\pgfsetdash{}{0pt}%
\pgfpathmoveto{\pgfqpoint{2.680003in}{2.277203in}}%
\pgfpathcurveto{\pgfqpoint{2.691053in}{2.277203in}}{\pgfqpoint{2.701652in}{2.281594in}}{\pgfqpoint{2.709465in}{2.289407in}}%
\pgfpathcurveto{\pgfqpoint{2.717279in}{2.297221in}}{\pgfqpoint{2.721669in}{2.307820in}}{\pgfqpoint{2.721669in}{2.318870in}}%
\pgfpathcurveto{\pgfqpoint{2.721669in}{2.329920in}}{\pgfqpoint{2.717279in}{2.340519in}}{\pgfqpoint{2.709465in}{2.348333in}}%
\pgfpathcurveto{\pgfqpoint{2.701652in}{2.356146in}}{\pgfqpoint{2.691053in}{2.360537in}}{\pgfqpoint{2.680003in}{2.360537in}}%
\pgfpathcurveto{\pgfqpoint{2.668953in}{2.360537in}}{\pgfqpoint{2.658354in}{2.356146in}}{\pgfqpoint{2.650540in}{2.348333in}}%
\pgfpathcurveto{\pgfqpoint{2.642726in}{2.340519in}}{\pgfqpoint{2.638336in}{2.329920in}}{\pgfqpoint{2.638336in}{2.318870in}}%
\pgfpathcurveto{\pgfqpoint{2.638336in}{2.307820in}}{\pgfqpoint{2.642726in}{2.297221in}}{\pgfqpoint{2.650540in}{2.289407in}}%
\pgfpathcurveto{\pgfqpoint{2.658354in}{2.281594in}}{\pgfqpoint{2.668953in}{2.277203in}}{\pgfqpoint{2.680003in}{2.277203in}}%
\pgfpathclose%
\pgfusepath{stroke,fill}%
\end{pgfscope}%
\begin{pgfscope}%
\pgfpathrectangle{\pgfqpoint{0.787074in}{0.548769in}}{\pgfqpoint{4.974523in}{3.102590in}}%
\pgfusepath{clip}%
\pgfsetbuttcap%
\pgfsetroundjoin%
\definecolor{currentfill}{rgb}{1.000000,0.498039,0.054902}%
\pgfsetfillcolor{currentfill}%
\pgfsetlinewidth{1.003750pt}%
\definecolor{currentstroke}{rgb}{1.000000,0.498039,0.054902}%
\pgfsetstrokecolor{currentstroke}%
\pgfsetdash{}{0pt}%
\pgfpathmoveto{\pgfqpoint{4.217385in}{2.871516in}}%
\pgfpathcurveto{\pgfqpoint{4.228435in}{2.871516in}}{\pgfqpoint{4.239034in}{2.875907in}}{\pgfqpoint{4.246848in}{2.883720in}}%
\pgfpathcurveto{\pgfqpoint{4.254662in}{2.891534in}}{\pgfqpoint{4.259052in}{2.902133in}}{\pgfqpoint{4.259052in}{2.913183in}}%
\pgfpathcurveto{\pgfqpoint{4.259052in}{2.924233in}}{\pgfqpoint{4.254662in}{2.934832in}}{\pgfqpoint{4.246848in}{2.942646in}}%
\pgfpathcurveto{\pgfqpoint{4.239034in}{2.950459in}}{\pgfqpoint{4.228435in}{2.954850in}}{\pgfqpoint{4.217385in}{2.954850in}}%
\pgfpathcurveto{\pgfqpoint{4.206335in}{2.954850in}}{\pgfqpoint{4.195736in}{2.950459in}}{\pgfqpoint{4.187923in}{2.942646in}}%
\pgfpathcurveto{\pgfqpoint{4.180109in}{2.934832in}}{\pgfqpoint{4.175719in}{2.924233in}}{\pgfqpoint{4.175719in}{2.913183in}}%
\pgfpathcurveto{\pgfqpoint{4.175719in}{2.902133in}}{\pgfqpoint{4.180109in}{2.891534in}}{\pgfqpoint{4.187923in}{2.883720in}}%
\pgfpathcurveto{\pgfqpoint{4.195736in}{2.875907in}}{\pgfqpoint{4.206335in}{2.871516in}}{\pgfqpoint{4.217385in}{2.871516in}}%
\pgfpathclose%
\pgfusepath{stroke,fill}%
\end{pgfscope}%
\begin{pgfscope}%
\pgfpathrectangle{\pgfqpoint{0.787074in}{0.548769in}}{\pgfqpoint{4.974523in}{3.102590in}}%
\pgfusepath{clip}%
\pgfsetbuttcap%
\pgfsetroundjoin%
\definecolor{currentfill}{rgb}{0.121569,0.466667,0.705882}%
\pgfsetfillcolor{currentfill}%
\pgfsetlinewidth{1.003750pt}%
\definecolor{currentstroke}{rgb}{0.121569,0.466667,0.705882}%
\pgfsetstrokecolor{currentstroke}%
\pgfsetdash{}{0pt}%
\pgfpathmoveto{\pgfqpoint{1.005422in}{0.648134in}}%
\pgfpathcurveto{\pgfqpoint{1.016472in}{0.648134in}}{\pgfqpoint{1.027071in}{0.652524in}}{\pgfqpoint{1.034885in}{0.660338in}}%
\pgfpathcurveto{\pgfqpoint{1.042698in}{0.668152in}}{\pgfqpoint{1.047089in}{0.678751in}}{\pgfqpoint{1.047089in}{0.689801in}}%
\pgfpathcurveto{\pgfqpoint{1.047089in}{0.700851in}}{\pgfqpoint{1.042698in}{0.711450in}}{\pgfqpoint{1.034885in}{0.719264in}}%
\pgfpathcurveto{\pgfqpoint{1.027071in}{0.727077in}}{\pgfqpoint{1.016472in}{0.731467in}}{\pgfqpoint{1.005422in}{0.731467in}}%
\pgfpathcurveto{\pgfqpoint{0.994372in}{0.731467in}}{\pgfqpoint{0.983773in}{0.727077in}}{\pgfqpoint{0.975959in}{0.719264in}}%
\pgfpathcurveto{\pgfqpoint{0.968146in}{0.711450in}}{\pgfqpoint{0.963755in}{0.700851in}}{\pgfqpoint{0.963755in}{0.689801in}}%
\pgfpathcurveto{\pgfqpoint{0.963755in}{0.678751in}}{\pgfqpoint{0.968146in}{0.668152in}}{\pgfqpoint{0.975959in}{0.660338in}}%
\pgfpathcurveto{\pgfqpoint{0.983773in}{0.652524in}}{\pgfqpoint{0.994372in}{0.648134in}}{\pgfqpoint{1.005422in}{0.648134in}}%
\pgfpathclose%
\pgfusepath{stroke,fill}%
\end{pgfscope}%
\begin{pgfscope}%
\pgfpathrectangle{\pgfqpoint{0.787074in}{0.548769in}}{\pgfqpoint{4.974523in}{3.102590in}}%
\pgfusepath{clip}%
\pgfsetbuttcap%
\pgfsetroundjoin%
\definecolor{currentfill}{rgb}{0.121569,0.466667,0.705882}%
\pgfsetfillcolor{currentfill}%
\pgfsetlinewidth{1.003750pt}%
\definecolor{currentstroke}{rgb}{0.121569,0.466667,0.705882}%
\pgfsetstrokecolor{currentstroke}%
\pgfsetdash{}{0pt}%
\pgfpathmoveto{\pgfqpoint{4.315878in}{3.215961in}}%
\pgfpathcurveto{\pgfqpoint{4.326929in}{3.215961in}}{\pgfqpoint{4.337528in}{3.220351in}}{\pgfqpoint{4.345341in}{3.228165in}}%
\pgfpathcurveto{\pgfqpoint{4.353155in}{3.235978in}}{\pgfqpoint{4.357545in}{3.246577in}}{\pgfqpoint{4.357545in}{3.257627in}}%
\pgfpathcurveto{\pgfqpoint{4.357545in}{3.268677in}}{\pgfqpoint{4.353155in}{3.279276in}}{\pgfqpoint{4.345341in}{3.287090in}}%
\pgfpathcurveto{\pgfqpoint{4.337528in}{3.294904in}}{\pgfqpoint{4.326929in}{3.299294in}}{\pgfqpoint{4.315878in}{3.299294in}}%
\pgfpathcurveto{\pgfqpoint{4.304828in}{3.299294in}}{\pgfqpoint{4.294229in}{3.294904in}}{\pgfqpoint{4.286416in}{3.287090in}}%
\pgfpathcurveto{\pgfqpoint{4.278602in}{3.279276in}}{\pgfqpoint{4.274212in}{3.268677in}}{\pgfqpoint{4.274212in}{3.257627in}}%
\pgfpathcurveto{\pgfqpoint{4.274212in}{3.246577in}}{\pgfqpoint{4.278602in}{3.235978in}}{\pgfqpoint{4.286416in}{3.228165in}}%
\pgfpathcurveto{\pgfqpoint{4.294229in}{3.220351in}}{\pgfqpoint{4.304828in}{3.215961in}}{\pgfqpoint{4.315878in}{3.215961in}}%
\pgfpathclose%
\pgfusepath{stroke,fill}%
\end{pgfscope}%
\begin{pgfscope}%
\pgfpathrectangle{\pgfqpoint{0.787074in}{0.548769in}}{\pgfqpoint{4.974523in}{3.102590in}}%
\pgfusepath{clip}%
\pgfsetbuttcap%
\pgfsetroundjoin%
\definecolor{currentfill}{rgb}{1.000000,0.498039,0.054902}%
\pgfsetfillcolor{currentfill}%
\pgfsetlinewidth{1.003750pt}%
\definecolor{currentstroke}{rgb}{1.000000,0.498039,0.054902}%
\pgfsetstrokecolor{currentstroke}%
\pgfsetdash{}{0pt}%
\pgfpathmoveto{\pgfqpoint{4.403006in}{3.082101in}}%
\pgfpathcurveto{\pgfqpoint{4.414056in}{3.082101in}}{\pgfqpoint{4.424655in}{3.086492in}}{\pgfqpoint{4.432469in}{3.094305in}}%
\pgfpathcurveto{\pgfqpoint{4.440282in}{3.102119in}}{\pgfqpoint{4.444673in}{3.112718in}}{\pgfqpoint{4.444673in}{3.123768in}}%
\pgfpathcurveto{\pgfqpoint{4.444673in}{3.134818in}}{\pgfqpoint{4.440282in}{3.145417in}}{\pgfqpoint{4.432469in}{3.153231in}}%
\pgfpathcurveto{\pgfqpoint{4.424655in}{3.161044in}}{\pgfqpoint{4.414056in}{3.165435in}}{\pgfqpoint{4.403006in}{3.165435in}}%
\pgfpathcurveto{\pgfqpoint{4.391956in}{3.165435in}}{\pgfqpoint{4.381357in}{3.161044in}}{\pgfqpoint{4.373543in}{3.153231in}}%
\pgfpathcurveto{\pgfqpoint{4.365730in}{3.145417in}}{\pgfqpoint{4.361339in}{3.134818in}}{\pgfqpoint{4.361339in}{3.123768in}}%
\pgfpathcurveto{\pgfqpoint{4.361339in}{3.112718in}}{\pgfqpoint{4.365730in}{3.102119in}}{\pgfqpoint{4.373543in}{3.094305in}}%
\pgfpathcurveto{\pgfqpoint{4.381357in}{3.086492in}}{\pgfqpoint{4.391956in}{3.082101in}}{\pgfqpoint{4.403006in}{3.082101in}}%
\pgfpathclose%
\pgfusepath{stroke,fill}%
\end{pgfscope}%
\begin{pgfscope}%
\pgfpathrectangle{\pgfqpoint{0.787074in}{0.548769in}}{\pgfqpoint{4.974523in}{3.102590in}}%
\pgfusepath{clip}%
\pgfsetbuttcap%
\pgfsetroundjoin%
\definecolor{currentfill}{rgb}{0.121569,0.466667,0.705882}%
\pgfsetfillcolor{currentfill}%
\pgfsetlinewidth{1.003750pt}%
\definecolor{currentstroke}{rgb}{0.121569,0.466667,0.705882}%
\pgfsetstrokecolor{currentstroke}%
\pgfsetdash{}{0pt}%
\pgfpathmoveto{\pgfqpoint{2.693103in}{1.896734in}}%
\pgfpathcurveto{\pgfqpoint{2.704154in}{1.896734in}}{\pgfqpoint{2.714753in}{1.901124in}}{\pgfqpoint{2.722566in}{1.908938in}}%
\pgfpathcurveto{\pgfqpoint{2.730380in}{1.916751in}}{\pgfqpoint{2.734770in}{1.927350in}}{\pgfqpoint{2.734770in}{1.938401in}}%
\pgfpathcurveto{\pgfqpoint{2.734770in}{1.949451in}}{\pgfqpoint{2.730380in}{1.960050in}}{\pgfqpoint{2.722566in}{1.967863in}}%
\pgfpathcurveto{\pgfqpoint{2.714753in}{1.975677in}}{\pgfqpoint{2.704154in}{1.980067in}}{\pgfqpoint{2.693103in}{1.980067in}}%
\pgfpathcurveto{\pgfqpoint{2.682053in}{1.980067in}}{\pgfqpoint{2.671454in}{1.975677in}}{\pgfqpoint{2.663641in}{1.967863in}}%
\pgfpathcurveto{\pgfqpoint{2.655827in}{1.960050in}}{\pgfqpoint{2.651437in}{1.949451in}}{\pgfqpoint{2.651437in}{1.938401in}}%
\pgfpathcurveto{\pgfqpoint{2.651437in}{1.927350in}}{\pgfqpoint{2.655827in}{1.916751in}}{\pgfqpoint{2.663641in}{1.908938in}}%
\pgfpathcurveto{\pgfqpoint{2.671454in}{1.901124in}}{\pgfqpoint{2.682053in}{1.896734in}}{\pgfqpoint{2.693103in}{1.896734in}}%
\pgfpathclose%
\pgfusepath{stroke,fill}%
\end{pgfscope}%
\begin{pgfscope}%
\pgfpathrectangle{\pgfqpoint{0.787074in}{0.548769in}}{\pgfqpoint{4.974523in}{3.102590in}}%
\pgfusepath{clip}%
\pgfsetbuttcap%
\pgfsetroundjoin%
\definecolor{currentfill}{rgb}{1.000000,0.498039,0.054902}%
\pgfsetfillcolor{currentfill}%
\pgfsetlinewidth{1.003750pt}%
\definecolor{currentstroke}{rgb}{1.000000,0.498039,0.054902}%
\pgfsetstrokecolor{currentstroke}%
\pgfsetdash{}{0pt}%
\pgfpathmoveto{\pgfqpoint{5.372089in}{3.048752in}}%
\pgfpathcurveto{\pgfqpoint{5.383140in}{3.048752in}}{\pgfqpoint{5.393739in}{3.053142in}}{\pgfqpoint{5.401552in}{3.060956in}}%
\pgfpathcurveto{\pgfqpoint{5.409366in}{3.068769in}}{\pgfqpoint{5.413756in}{3.079368in}}{\pgfqpoint{5.413756in}{3.090419in}}%
\pgfpathcurveto{\pgfqpoint{5.413756in}{3.101469in}}{\pgfqpoint{5.409366in}{3.112068in}}{\pgfqpoint{5.401552in}{3.119881in}}%
\pgfpathcurveto{\pgfqpoint{5.393739in}{3.127695in}}{\pgfqpoint{5.383140in}{3.132085in}}{\pgfqpoint{5.372089in}{3.132085in}}%
\pgfpathcurveto{\pgfqpoint{5.361039in}{3.132085in}}{\pgfqpoint{5.350440in}{3.127695in}}{\pgfqpoint{5.342627in}{3.119881in}}%
\pgfpathcurveto{\pgfqpoint{5.334813in}{3.112068in}}{\pgfqpoint{5.330423in}{3.101469in}}{\pgfqpoint{5.330423in}{3.090419in}}%
\pgfpathcurveto{\pgfqpoint{5.330423in}{3.079368in}}{\pgfqpoint{5.334813in}{3.068769in}}{\pgfqpoint{5.342627in}{3.060956in}}%
\pgfpathcurveto{\pgfqpoint{5.350440in}{3.053142in}}{\pgfqpoint{5.361039in}{3.048752in}}{\pgfqpoint{5.372089in}{3.048752in}}%
\pgfpathclose%
\pgfusepath{stroke,fill}%
\end{pgfscope}%
\begin{pgfscope}%
\pgfpathrectangle{\pgfqpoint{0.787074in}{0.548769in}}{\pgfqpoint{4.974523in}{3.102590in}}%
\pgfusepath{clip}%
\pgfsetbuttcap%
\pgfsetroundjoin%
\definecolor{currentfill}{rgb}{1.000000,0.498039,0.054902}%
\pgfsetfillcolor{currentfill}%
\pgfsetlinewidth{1.003750pt}%
\definecolor{currentstroke}{rgb}{1.000000,0.498039,0.054902}%
\pgfsetstrokecolor{currentstroke}%
\pgfsetdash{}{0pt}%
\pgfpathmoveto{\pgfqpoint{4.195605in}{2.964483in}}%
\pgfpathcurveto{\pgfqpoint{4.206655in}{2.964483in}}{\pgfqpoint{4.217254in}{2.968874in}}{\pgfqpoint{4.225068in}{2.976687in}}%
\pgfpathcurveto{\pgfqpoint{4.232882in}{2.984501in}}{\pgfqpoint{4.237272in}{2.995100in}}{\pgfqpoint{4.237272in}{3.006150in}}%
\pgfpathcurveto{\pgfqpoint{4.237272in}{3.017200in}}{\pgfqpoint{4.232882in}{3.027799in}}{\pgfqpoint{4.225068in}{3.035613in}}%
\pgfpathcurveto{\pgfqpoint{4.217254in}{3.043426in}}{\pgfqpoint{4.206655in}{3.047817in}}{\pgfqpoint{4.195605in}{3.047817in}}%
\pgfpathcurveto{\pgfqpoint{4.184555in}{3.047817in}}{\pgfqpoint{4.173956in}{3.043426in}}{\pgfqpoint{4.166142in}{3.035613in}}%
\pgfpathcurveto{\pgfqpoint{4.158329in}{3.027799in}}{\pgfqpoint{4.153939in}{3.017200in}}{\pgfqpoint{4.153939in}{3.006150in}}%
\pgfpathcurveto{\pgfqpoint{4.153939in}{2.995100in}}{\pgfqpoint{4.158329in}{2.984501in}}{\pgfqpoint{4.166142in}{2.976687in}}%
\pgfpathcurveto{\pgfqpoint{4.173956in}{2.968874in}}{\pgfqpoint{4.184555in}{2.964483in}}{\pgfqpoint{4.195605in}{2.964483in}}%
\pgfpathclose%
\pgfusepath{stroke,fill}%
\end{pgfscope}%
\begin{pgfscope}%
\pgfpathrectangle{\pgfqpoint{0.787074in}{0.548769in}}{\pgfqpoint{4.974523in}{3.102590in}}%
\pgfusepath{clip}%
\pgfsetbuttcap%
\pgfsetroundjoin%
\definecolor{currentfill}{rgb}{1.000000,0.498039,0.054902}%
\pgfsetfillcolor{currentfill}%
\pgfsetlinewidth{1.003750pt}%
\definecolor{currentstroke}{rgb}{1.000000,0.498039,0.054902}%
\pgfsetstrokecolor{currentstroke}%
\pgfsetdash{}{0pt}%
\pgfpathmoveto{\pgfqpoint{4.341703in}{2.843196in}}%
\pgfpathcurveto{\pgfqpoint{4.352753in}{2.843196in}}{\pgfqpoint{4.363352in}{2.847587in}}{\pgfqpoint{4.371166in}{2.855400in}}%
\pgfpathcurveto{\pgfqpoint{4.378979in}{2.863214in}}{\pgfqpoint{4.383369in}{2.873813in}}{\pgfqpoint{4.383369in}{2.884863in}}%
\pgfpathcurveto{\pgfqpoint{4.383369in}{2.895913in}}{\pgfqpoint{4.378979in}{2.906512in}}{\pgfqpoint{4.371166in}{2.914326in}}%
\pgfpathcurveto{\pgfqpoint{4.363352in}{2.922140in}}{\pgfqpoint{4.352753in}{2.926530in}}{\pgfqpoint{4.341703in}{2.926530in}}%
\pgfpathcurveto{\pgfqpoint{4.330653in}{2.926530in}}{\pgfqpoint{4.320054in}{2.922140in}}{\pgfqpoint{4.312240in}{2.914326in}}%
\pgfpathcurveto{\pgfqpoint{4.304426in}{2.906512in}}{\pgfqpoint{4.300036in}{2.895913in}}{\pgfqpoint{4.300036in}{2.884863in}}%
\pgfpathcurveto{\pgfqpoint{4.300036in}{2.873813in}}{\pgfqpoint{4.304426in}{2.863214in}}{\pgfqpoint{4.312240in}{2.855400in}}%
\pgfpathcurveto{\pgfqpoint{4.320054in}{2.847587in}}{\pgfqpoint{4.330653in}{2.843196in}}{\pgfqpoint{4.341703in}{2.843196in}}%
\pgfpathclose%
\pgfusepath{stroke,fill}%
\end{pgfscope}%
\begin{pgfscope}%
\pgfpathrectangle{\pgfqpoint{0.787074in}{0.548769in}}{\pgfqpoint{4.974523in}{3.102590in}}%
\pgfusepath{clip}%
\pgfsetbuttcap%
\pgfsetroundjoin%
\definecolor{currentfill}{rgb}{1.000000,0.498039,0.054902}%
\pgfsetfillcolor{currentfill}%
\pgfsetlinewidth{1.003750pt}%
\definecolor{currentstroke}{rgb}{1.000000,0.498039,0.054902}%
\pgfsetstrokecolor{currentstroke}%
\pgfsetdash{}{0pt}%
\pgfpathmoveto{\pgfqpoint{4.092398in}{2.715747in}}%
\pgfpathcurveto{\pgfqpoint{4.103448in}{2.715747in}}{\pgfqpoint{4.114047in}{2.720137in}}{\pgfqpoint{4.121860in}{2.727951in}}%
\pgfpathcurveto{\pgfqpoint{4.129674in}{2.735764in}}{\pgfqpoint{4.134064in}{2.746363in}}{\pgfqpoint{4.134064in}{2.757413in}}%
\pgfpathcurveto{\pgfqpoint{4.134064in}{2.768463in}}{\pgfqpoint{4.129674in}{2.779062in}}{\pgfqpoint{4.121860in}{2.786876in}}%
\pgfpathcurveto{\pgfqpoint{4.114047in}{2.794690in}}{\pgfqpoint{4.103448in}{2.799080in}}{\pgfqpoint{4.092398in}{2.799080in}}%
\pgfpathcurveto{\pgfqpoint{4.081348in}{2.799080in}}{\pgfqpoint{4.070749in}{2.794690in}}{\pgfqpoint{4.062935in}{2.786876in}}%
\pgfpathcurveto{\pgfqpoint{4.055121in}{2.779062in}}{\pgfqpoint{4.050731in}{2.768463in}}{\pgfqpoint{4.050731in}{2.757413in}}%
\pgfpathcurveto{\pgfqpoint{4.050731in}{2.746363in}}{\pgfqpoint{4.055121in}{2.735764in}}{\pgfqpoint{4.062935in}{2.727951in}}%
\pgfpathcurveto{\pgfqpoint{4.070749in}{2.720137in}}{\pgfqpoint{4.081348in}{2.715747in}}{\pgfqpoint{4.092398in}{2.715747in}}%
\pgfpathclose%
\pgfusepath{stroke,fill}%
\end{pgfscope}%
\begin{pgfscope}%
\pgfpathrectangle{\pgfqpoint{0.787074in}{0.548769in}}{\pgfqpoint{4.974523in}{3.102590in}}%
\pgfusepath{clip}%
\pgfsetbuttcap%
\pgfsetroundjoin%
\definecolor{currentfill}{rgb}{0.121569,0.466667,0.705882}%
\pgfsetfillcolor{currentfill}%
\pgfsetlinewidth{1.003750pt}%
\definecolor{currentstroke}{rgb}{0.121569,0.466667,0.705882}%
\pgfsetstrokecolor{currentstroke}%
\pgfsetdash{}{0pt}%
\pgfpathmoveto{\pgfqpoint{1.005408in}{0.648129in}}%
\pgfpathcurveto{\pgfqpoint{1.016458in}{0.648129in}}{\pgfqpoint{1.027057in}{0.652519in}}{\pgfqpoint{1.034871in}{0.660333in}}%
\pgfpathcurveto{\pgfqpoint{1.042684in}{0.668146in}}{\pgfqpoint{1.047075in}{0.678745in}}{\pgfqpoint{1.047075in}{0.689796in}}%
\pgfpathcurveto{\pgfqpoint{1.047075in}{0.700846in}}{\pgfqpoint{1.042684in}{0.711445in}}{\pgfqpoint{1.034871in}{0.719258in}}%
\pgfpathcurveto{\pgfqpoint{1.027057in}{0.727072in}}{\pgfqpoint{1.016458in}{0.731462in}}{\pgfqpoint{1.005408in}{0.731462in}}%
\pgfpathcurveto{\pgfqpoint{0.994358in}{0.731462in}}{\pgfqpoint{0.983759in}{0.727072in}}{\pgfqpoint{0.975945in}{0.719258in}}%
\pgfpathcurveto{\pgfqpoint{0.968132in}{0.711445in}}{\pgfqpoint{0.963741in}{0.700846in}}{\pgfqpoint{0.963741in}{0.689796in}}%
\pgfpathcurveto{\pgfqpoint{0.963741in}{0.678745in}}{\pgfqpoint{0.968132in}{0.668146in}}{\pgfqpoint{0.975945in}{0.660333in}}%
\pgfpathcurveto{\pgfqpoint{0.983759in}{0.652519in}}{\pgfqpoint{0.994358in}{0.648129in}}{\pgfqpoint{1.005408in}{0.648129in}}%
\pgfpathclose%
\pgfusepath{stroke,fill}%
\end{pgfscope}%
\begin{pgfscope}%
\pgfpathrectangle{\pgfqpoint{0.787074in}{0.548769in}}{\pgfqpoint{4.974523in}{3.102590in}}%
\pgfusepath{clip}%
\pgfsetbuttcap%
\pgfsetroundjoin%
\definecolor{currentfill}{rgb}{0.121569,0.466667,0.705882}%
\pgfsetfillcolor{currentfill}%
\pgfsetlinewidth{1.003750pt}%
\definecolor{currentstroke}{rgb}{0.121569,0.466667,0.705882}%
\pgfsetstrokecolor{currentstroke}%
\pgfsetdash{}{0pt}%
\pgfpathmoveto{\pgfqpoint{1.033615in}{0.665742in}}%
\pgfpathcurveto{\pgfqpoint{1.044665in}{0.665742in}}{\pgfqpoint{1.055264in}{0.670132in}}{\pgfqpoint{1.063078in}{0.677946in}}%
\pgfpathcurveto{\pgfqpoint{1.070892in}{0.685760in}}{\pgfqpoint{1.075282in}{0.696359in}}{\pgfqpoint{1.075282in}{0.707409in}}%
\pgfpathcurveto{\pgfqpoint{1.075282in}{0.718459in}}{\pgfqpoint{1.070892in}{0.729058in}}{\pgfqpoint{1.063078in}{0.736872in}}%
\pgfpathcurveto{\pgfqpoint{1.055264in}{0.744685in}}{\pgfqpoint{1.044665in}{0.749076in}}{\pgfqpoint{1.033615in}{0.749076in}}%
\pgfpathcurveto{\pgfqpoint{1.022565in}{0.749076in}}{\pgfqpoint{1.011966in}{0.744685in}}{\pgfqpoint{1.004152in}{0.736872in}}%
\pgfpathcurveto{\pgfqpoint{0.996339in}{0.729058in}}{\pgfqpoint{0.991948in}{0.718459in}}{\pgfqpoint{0.991948in}{0.707409in}}%
\pgfpathcurveto{\pgfqpoint{0.991948in}{0.696359in}}{\pgfqpoint{0.996339in}{0.685760in}}{\pgfqpoint{1.004152in}{0.677946in}}%
\pgfpathcurveto{\pgfqpoint{1.011966in}{0.670132in}}{\pgfqpoint{1.022565in}{0.665742in}}{\pgfqpoint{1.033615in}{0.665742in}}%
\pgfpathclose%
\pgfusepath{stroke,fill}%
\end{pgfscope}%
\begin{pgfscope}%
\pgfpathrectangle{\pgfqpoint{0.787074in}{0.548769in}}{\pgfqpoint{4.974523in}{3.102590in}}%
\pgfusepath{clip}%
\pgfsetbuttcap%
\pgfsetroundjoin%
\definecolor{currentfill}{rgb}{0.121569,0.466667,0.705882}%
\pgfsetfillcolor{currentfill}%
\pgfsetlinewidth{1.003750pt}%
\definecolor{currentstroke}{rgb}{0.121569,0.466667,0.705882}%
\pgfsetstrokecolor{currentstroke}%
\pgfsetdash{}{0pt}%
\pgfpathmoveto{\pgfqpoint{1.347417in}{1.038852in}}%
\pgfpathcurveto{\pgfqpoint{1.358468in}{1.038852in}}{\pgfqpoint{1.369067in}{1.043242in}}{\pgfqpoint{1.376880in}{1.051055in}}%
\pgfpathcurveto{\pgfqpoint{1.384694in}{1.058869in}}{\pgfqpoint{1.389084in}{1.069468in}}{\pgfqpoint{1.389084in}{1.080518in}}%
\pgfpathcurveto{\pgfqpoint{1.389084in}{1.091568in}}{\pgfqpoint{1.384694in}{1.102167in}}{\pgfqpoint{1.376880in}{1.109981in}}%
\pgfpathcurveto{\pgfqpoint{1.369067in}{1.117795in}}{\pgfqpoint{1.358468in}{1.122185in}}{\pgfqpoint{1.347417in}{1.122185in}}%
\pgfpathcurveto{\pgfqpoint{1.336367in}{1.122185in}}{\pgfqpoint{1.325768in}{1.117795in}}{\pgfqpoint{1.317955in}{1.109981in}}%
\pgfpathcurveto{\pgfqpoint{1.310141in}{1.102167in}}{\pgfqpoint{1.305751in}{1.091568in}}{\pgfqpoint{1.305751in}{1.080518in}}%
\pgfpathcurveto{\pgfqpoint{1.305751in}{1.069468in}}{\pgfqpoint{1.310141in}{1.058869in}}{\pgfqpoint{1.317955in}{1.051055in}}%
\pgfpathcurveto{\pgfqpoint{1.325768in}{1.043242in}}{\pgfqpoint{1.336367in}{1.038852in}}{\pgfqpoint{1.347417in}{1.038852in}}%
\pgfpathclose%
\pgfusepath{stroke,fill}%
\end{pgfscope}%
\begin{pgfscope}%
\pgfpathrectangle{\pgfqpoint{0.787074in}{0.548769in}}{\pgfqpoint{4.974523in}{3.102590in}}%
\pgfusepath{clip}%
\pgfsetbuttcap%
\pgfsetroundjoin%
\definecolor{currentfill}{rgb}{1.000000,0.498039,0.054902}%
\pgfsetfillcolor{currentfill}%
\pgfsetlinewidth{1.003750pt}%
\definecolor{currentstroke}{rgb}{1.000000,0.498039,0.054902}%
\pgfsetstrokecolor{currentstroke}%
\pgfsetdash{}{0pt}%
\pgfpathmoveto{\pgfqpoint{4.676933in}{2.996726in}}%
\pgfpathcurveto{\pgfqpoint{4.687984in}{2.996726in}}{\pgfqpoint{4.698583in}{3.001116in}}{\pgfqpoint{4.706396in}{3.008930in}}%
\pgfpathcurveto{\pgfqpoint{4.714210in}{3.016743in}}{\pgfqpoint{4.718600in}{3.027342in}}{\pgfqpoint{4.718600in}{3.038392in}}%
\pgfpathcurveto{\pgfqpoint{4.718600in}{3.049443in}}{\pgfqpoint{4.714210in}{3.060042in}}{\pgfqpoint{4.706396in}{3.067855in}}%
\pgfpathcurveto{\pgfqpoint{4.698583in}{3.075669in}}{\pgfqpoint{4.687984in}{3.080059in}}{\pgfqpoint{4.676933in}{3.080059in}}%
\pgfpathcurveto{\pgfqpoint{4.665883in}{3.080059in}}{\pgfqpoint{4.655284in}{3.075669in}}{\pgfqpoint{4.647471in}{3.067855in}}%
\pgfpathcurveto{\pgfqpoint{4.639657in}{3.060042in}}{\pgfqpoint{4.635267in}{3.049443in}}{\pgfqpoint{4.635267in}{3.038392in}}%
\pgfpathcurveto{\pgfqpoint{4.635267in}{3.027342in}}{\pgfqpoint{4.639657in}{3.016743in}}{\pgfqpoint{4.647471in}{3.008930in}}%
\pgfpathcurveto{\pgfqpoint{4.655284in}{3.001116in}}{\pgfqpoint{4.665883in}{2.996726in}}{\pgfqpoint{4.676933in}{2.996726in}}%
\pgfpathclose%
\pgfusepath{stroke,fill}%
\end{pgfscope}%
\begin{pgfscope}%
\pgfpathrectangle{\pgfqpoint{0.787074in}{0.548769in}}{\pgfqpoint{4.974523in}{3.102590in}}%
\pgfusepath{clip}%
\pgfsetbuttcap%
\pgfsetroundjoin%
\definecolor{currentfill}{rgb}{0.121569,0.466667,0.705882}%
\pgfsetfillcolor{currentfill}%
\pgfsetlinewidth{1.003750pt}%
\definecolor{currentstroke}{rgb}{0.121569,0.466667,0.705882}%
\pgfsetstrokecolor{currentstroke}%
\pgfsetdash{}{0pt}%
\pgfpathmoveto{\pgfqpoint{1.005411in}{0.648130in}}%
\pgfpathcurveto{\pgfqpoint{1.016461in}{0.648130in}}{\pgfqpoint{1.027060in}{0.652521in}}{\pgfqpoint{1.034874in}{0.660334in}}%
\pgfpathcurveto{\pgfqpoint{1.042688in}{0.668148in}}{\pgfqpoint{1.047078in}{0.678747in}}{\pgfqpoint{1.047078in}{0.689797in}}%
\pgfpathcurveto{\pgfqpoint{1.047078in}{0.700847in}}{\pgfqpoint{1.042688in}{0.711446in}}{\pgfqpoint{1.034874in}{0.719260in}}%
\pgfpathcurveto{\pgfqpoint{1.027060in}{0.727073in}}{\pgfqpoint{1.016461in}{0.731464in}}{\pgfqpoint{1.005411in}{0.731464in}}%
\pgfpathcurveto{\pgfqpoint{0.994361in}{0.731464in}}{\pgfqpoint{0.983762in}{0.727073in}}{\pgfqpoint{0.975948in}{0.719260in}}%
\pgfpathcurveto{\pgfqpoint{0.968135in}{0.711446in}}{\pgfqpoint{0.963744in}{0.700847in}}{\pgfqpoint{0.963744in}{0.689797in}}%
\pgfpathcurveto{\pgfqpoint{0.963744in}{0.678747in}}{\pgfqpoint{0.968135in}{0.668148in}}{\pgfqpoint{0.975948in}{0.660334in}}%
\pgfpathcurveto{\pgfqpoint{0.983762in}{0.652521in}}{\pgfqpoint{0.994361in}{0.648130in}}{\pgfqpoint{1.005411in}{0.648130in}}%
\pgfpathclose%
\pgfusepath{stroke,fill}%
\end{pgfscope}%
\begin{pgfscope}%
\pgfpathrectangle{\pgfqpoint{0.787074in}{0.548769in}}{\pgfqpoint{4.974523in}{3.102590in}}%
\pgfusepath{clip}%
\pgfsetbuttcap%
\pgfsetroundjoin%
\definecolor{currentfill}{rgb}{0.121569,0.466667,0.705882}%
\pgfsetfillcolor{currentfill}%
\pgfsetlinewidth{1.003750pt}%
\definecolor{currentstroke}{rgb}{0.121569,0.466667,0.705882}%
\pgfsetstrokecolor{currentstroke}%
\pgfsetdash{}{0pt}%
\pgfpathmoveto{\pgfqpoint{1.005422in}{0.648134in}}%
\pgfpathcurveto{\pgfqpoint{1.016472in}{0.648134in}}{\pgfqpoint{1.027071in}{0.652524in}}{\pgfqpoint{1.034885in}{0.660338in}}%
\pgfpathcurveto{\pgfqpoint{1.042698in}{0.668152in}}{\pgfqpoint{1.047089in}{0.678751in}}{\pgfqpoint{1.047089in}{0.689801in}}%
\pgfpathcurveto{\pgfqpoint{1.047089in}{0.700851in}}{\pgfqpoint{1.042698in}{0.711450in}}{\pgfqpoint{1.034885in}{0.719264in}}%
\pgfpathcurveto{\pgfqpoint{1.027071in}{0.727077in}}{\pgfqpoint{1.016472in}{0.731467in}}{\pgfqpoint{1.005422in}{0.731467in}}%
\pgfpathcurveto{\pgfqpoint{0.994372in}{0.731467in}}{\pgfqpoint{0.983773in}{0.727077in}}{\pgfqpoint{0.975959in}{0.719264in}}%
\pgfpathcurveto{\pgfqpoint{0.968146in}{0.711450in}}{\pgfqpoint{0.963755in}{0.700851in}}{\pgfqpoint{0.963755in}{0.689801in}}%
\pgfpathcurveto{\pgfqpoint{0.963755in}{0.678751in}}{\pgfqpoint{0.968146in}{0.668152in}}{\pgfqpoint{0.975959in}{0.660338in}}%
\pgfpathcurveto{\pgfqpoint{0.983773in}{0.652524in}}{\pgfqpoint{0.994372in}{0.648134in}}{\pgfqpoint{1.005422in}{0.648134in}}%
\pgfpathclose%
\pgfusepath{stroke,fill}%
\end{pgfscope}%
\begin{pgfscope}%
\pgfpathrectangle{\pgfqpoint{0.787074in}{0.548769in}}{\pgfqpoint{4.974523in}{3.102590in}}%
\pgfusepath{clip}%
\pgfsetbuttcap%
\pgfsetroundjoin%
\definecolor{currentfill}{rgb}{0.839216,0.152941,0.156863}%
\pgfsetfillcolor{currentfill}%
\pgfsetlinewidth{1.003750pt}%
\definecolor{currentstroke}{rgb}{0.839216,0.152941,0.156863}%
\pgfsetstrokecolor{currentstroke}%
\pgfsetdash{}{0pt}%
\pgfpathmoveto{\pgfqpoint{4.317787in}{3.468665in}}%
\pgfpathcurveto{\pgfqpoint{4.328837in}{3.468665in}}{\pgfqpoint{4.339436in}{3.473055in}}{\pgfqpoint{4.347250in}{3.480869in}}%
\pgfpathcurveto{\pgfqpoint{4.355063in}{3.488683in}}{\pgfqpoint{4.359453in}{3.499282in}}{\pgfqpoint{4.359453in}{3.510332in}}%
\pgfpathcurveto{\pgfqpoint{4.359453in}{3.521382in}}{\pgfqpoint{4.355063in}{3.531981in}}{\pgfqpoint{4.347250in}{3.539795in}}%
\pgfpathcurveto{\pgfqpoint{4.339436in}{3.547608in}}{\pgfqpoint{4.328837in}{3.551998in}}{\pgfqpoint{4.317787in}{3.551998in}}%
\pgfpathcurveto{\pgfqpoint{4.306737in}{3.551998in}}{\pgfqpoint{4.296138in}{3.547608in}}{\pgfqpoint{4.288324in}{3.539795in}}%
\pgfpathcurveto{\pgfqpoint{4.280510in}{3.531981in}}{\pgfqpoint{4.276120in}{3.521382in}}{\pgfqpoint{4.276120in}{3.510332in}}%
\pgfpathcurveto{\pgfqpoint{4.276120in}{3.499282in}}{\pgfqpoint{4.280510in}{3.488683in}}{\pgfqpoint{4.288324in}{3.480869in}}%
\pgfpathcurveto{\pgfqpoint{4.296138in}{3.473055in}}{\pgfqpoint{4.306737in}{3.468665in}}{\pgfqpoint{4.317787in}{3.468665in}}%
\pgfpathclose%
\pgfusepath{stroke,fill}%
\end{pgfscope}%
\begin{pgfscope}%
\pgfpathrectangle{\pgfqpoint{0.787074in}{0.548769in}}{\pgfqpoint{4.974523in}{3.102590in}}%
\pgfusepath{clip}%
\pgfsetbuttcap%
\pgfsetroundjoin%
\definecolor{currentfill}{rgb}{0.121569,0.466667,0.705882}%
\pgfsetfillcolor{currentfill}%
\pgfsetlinewidth{1.003750pt}%
\definecolor{currentstroke}{rgb}{0.121569,0.466667,0.705882}%
\pgfsetstrokecolor{currentstroke}%
\pgfsetdash{}{0pt}%
\pgfpathmoveto{\pgfqpoint{4.150555in}{3.009766in}}%
\pgfpathcurveto{\pgfqpoint{4.161605in}{3.009766in}}{\pgfqpoint{4.172204in}{3.014156in}}{\pgfqpoint{4.180018in}{3.021970in}}%
\pgfpathcurveto{\pgfqpoint{4.187831in}{3.029783in}}{\pgfqpoint{4.192221in}{3.040383in}}{\pgfqpoint{4.192221in}{3.051433in}}%
\pgfpathcurveto{\pgfqpoint{4.192221in}{3.062483in}}{\pgfqpoint{4.187831in}{3.073082in}}{\pgfqpoint{4.180018in}{3.080895in}}%
\pgfpathcurveto{\pgfqpoint{4.172204in}{3.088709in}}{\pgfqpoint{4.161605in}{3.093099in}}{\pgfqpoint{4.150555in}{3.093099in}}%
\pgfpathcurveto{\pgfqpoint{4.139505in}{3.093099in}}{\pgfqpoint{4.128906in}{3.088709in}}{\pgfqpoint{4.121092in}{3.080895in}}%
\pgfpathcurveto{\pgfqpoint{4.113278in}{3.073082in}}{\pgfqpoint{4.108888in}{3.062483in}}{\pgfqpoint{4.108888in}{3.051433in}}%
\pgfpathcurveto{\pgfqpoint{4.108888in}{3.040383in}}{\pgfqpoint{4.113278in}{3.029783in}}{\pgfqpoint{4.121092in}{3.021970in}}%
\pgfpathcurveto{\pgfqpoint{4.128906in}{3.014156in}}{\pgfqpoint{4.139505in}{3.009766in}}{\pgfqpoint{4.150555in}{3.009766in}}%
\pgfpathclose%
\pgfusepath{stroke,fill}%
\end{pgfscope}%
\begin{pgfscope}%
\pgfpathrectangle{\pgfqpoint{0.787074in}{0.548769in}}{\pgfqpoint{4.974523in}{3.102590in}}%
\pgfusepath{clip}%
\pgfsetbuttcap%
\pgfsetroundjoin%
\definecolor{currentfill}{rgb}{1.000000,0.498039,0.054902}%
\pgfsetfillcolor{currentfill}%
\pgfsetlinewidth{1.003750pt}%
\definecolor{currentstroke}{rgb}{1.000000,0.498039,0.054902}%
\pgfsetstrokecolor{currentstroke}%
\pgfsetdash{}{0pt}%
\pgfpathmoveto{\pgfqpoint{4.260359in}{3.140942in}}%
\pgfpathcurveto{\pgfqpoint{4.271409in}{3.140942in}}{\pgfqpoint{4.282009in}{3.145332in}}{\pgfqpoint{4.289822in}{3.153146in}}%
\pgfpathcurveto{\pgfqpoint{4.297636in}{3.160959in}}{\pgfqpoint{4.302026in}{3.171558in}}{\pgfqpoint{4.302026in}{3.182609in}}%
\pgfpathcurveto{\pgfqpoint{4.302026in}{3.193659in}}{\pgfqpoint{4.297636in}{3.204258in}}{\pgfqpoint{4.289822in}{3.212071in}}%
\pgfpathcurveto{\pgfqpoint{4.282009in}{3.219885in}}{\pgfqpoint{4.271409in}{3.224275in}}{\pgfqpoint{4.260359in}{3.224275in}}%
\pgfpathcurveto{\pgfqpoint{4.249309in}{3.224275in}}{\pgfqpoint{4.238710in}{3.219885in}}{\pgfqpoint{4.230897in}{3.212071in}}%
\pgfpathcurveto{\pgfqpoint{4.223083in}{3.204258in}}{\pgfqpoint{4.218693in}{3.193659in}}{\pgfqpoint{4.218693in}{3.182609in}}%
\pgfpathcurveto{\pgfqpoint{4.218693in}{3.171558in}}{\pgfqpoint{4.223083in}{3.160959in}}{\pgfqpoint{4.230897in}{3.153146in}}%
\pgfpathcurveto{\pgfqpoint{4.238710in}{3.145332in}}{\pgfqpoint{4.249309in}{3.140942in}}{\pgfqpoint{4.260359in}{3.140942in}}%
\pgfpathclose%
\pgfusepath{stroke,fill}%
\end{pgfscope}%
\begin{pgfscope}%
\pgfpathrectangle{\pgfqpoint{0.787074in}{0.548769in}}{\pgfqpoint{4.974523in}{3.102590in}}%
\pgfusepath{clip}%
\pgfsetbuttcap%
\pgfsetroundjoin%
\definecolor{currentfill}{rgb}{1.000000,0.498039,0.054902}%
\pgfsetfillcolor{currentfill}%
\pgfsetlinewidth{1.003750pt}%
\definecolor{currentstroke}{rgb}{1.000000,0.498039,0.054902}%
\pgfsetstrokecolor{currentstroke}%
\pgfsetdash{}{0pt}%
\pgfpathmoveto{\pgfqpoint{3.982804in}{2.861427in}}%
\pgfpathcurveto{\pgfqpoint{3.993854in}{2.861427in}}{\pgfqpoint{4.004453in}{2.865817in}}{\pgfqpoint{4.012267in}{2.873631in}}%
\pgfpathcurveto{\pgfqpoint{4.020080in}{2.881444in}}{\pgfqpoint{4.024471in}{2.892043in}}{\pgfqpoint{4.024471in}{2.903094in}}%
\pgfpathcurveto{\pgfqpoint{4.024471in}{2.914144in}}{\pgfqpoint{4.020080in}{2.924743in}}{\pgfqpoint{4.012267in}{2.932556in}}%
\pgfpathcurveto{\pgfqpoint{4.004453in}{2.940370in}}{\pgfqpoint{3.993854in}{2.944760in}}{\pgfqpoint{3.982804in}{2.944760in}}%
\pgfpathcurveto{\pgfqpoint{3.971754in}{2.944760in}}{\pgfqpoint{3.961155in}{2.940370in}}{\pgfqpoint{3.953341in}{2.932556in}}%
\pgfpathcurveto{\pgfqpoint{3.945528in}{2.924743in}}{\pgfqpoint{3.941137in}{2.914144in}}{\pgfqpoint{3.941137in}{2.903094in}}%
\pgfpathcurveto{\pgfqpoint{3.941137in}{2.892043in}}{\pgfqpoint{3.945528in}{2.881444in}}{\pgfqpoint{3.953341in}{2.873631in}}%
\pgfpathcurveto{\pgfqpoint{3.961155in}{2.865817in}}{\pgfqpoint{3.971754in}{2.861427in}}{\pgfqpoint{3.982804in}{2.861427in}}%
\pgfpathclose%
\pgfusepath{stroke,fill}%
\end{pgfscope}%
\begin{pgfscope}%
\pgfpathrectangle{\pgfqpoint{0.787074in}{0.548769in}}{\pgfqpoint{4.974523in}{3.102590in}}%
\pgfusepath{clip}%
\pgfsetbuttcap%
\pgfsetroundjoin%
\definecolor{currentfill}{rgb}{1.000000,0.498039,0.054902}%
\pgfsetfillcolor{currentfill}%
\pgfsetlinewidth{1.003750pt}%
\definecolor{currentstroke}{rgb}{1.000000,0.498039,0.054902}%
\pgfsetstrokecolor{currentstroke}%
\pgfsetdash{}{0pt}%
\pgfpathmoveto{\pgfqpoint{4.180986in}{3.035300in}}%
\pgfpathcurveto{\pgfqpoint{4.192036in}{3.035300in}}{\pgfqpoint{4.202635in}{3.039690in}}{\pgfqpoint{4.210448in}{3.047504in}}%
\pgfpathcurveto{\pgfqpoint{4.218262in}{3.055317in}}{\pgfqpoint{4.222652in}{3.065916in}}{\pgfqpoint{4.222652in}{3.076967in}}%
\pgfpathcurveto{\pgfqpoint{4.222652in}{3.088017in}}{\pgfqpoint{4.218262in}{3.098616in}}{\pgfqpoint{4.210448in}{3.106429in}}%
\pgfpathcurveto{\pgfqpoint{4.202635in}{3.114243in}}{\pgfqpoint{4.192036in}{3.118633in}}{\pgfqpoint{4.180986in}{3.118633in}}%
\pgfpathcurveto{\pgfqpoint{4.169936in}{3.118633in}}{\pgfqpoint{4.159337in}{3.114243in}}{\pgfqpoint{4.151523in}{3.106429in}}%
\pgfpathcurveto{\pgfqpoint{4.143709in}{3.098616in}}{\pgfqpoint{4.139319in}{3.088017in}}{\pgfqpoint{4.139319in}{3.076967in}}%
\pgfpathcurveto{\pgfqpoint{4.139319in}{3.065916in}}{\pgfqpoint{4.143709in}{3.055317in}}{\pgfqpoint{4.151523in}{3.047504in}}%
\pgfpathcurveto{\pgfqpoint{4.159337in}{3.039690in}}{\pgfqpoint{4.169936in}{3.035300in}}{\pgfqpoint{4.180986in}{3.035300in}}%
\pgfpathclose%
\pgfusepath{stroke,fill}%
\end{pgfscope}%
\begin{pgfscope}%
\pgfpathrectangle{\pgfqpoint{0.787074in}{0.548769in}}{\pgfqpoint{4.974523in}{3.102590in}}%
\pgfusepath{clip}%
\pgfsetbuttcap%
\pgfsetroundjoin%
\definecolor{currentfill}{rgb}{1.000000,0.498039,0.054902}%
\pgfsetfillcolor{currentfill}%
\pgfsetlinewidth{1.003750pt}%
\definecolor{currentstroke}{rgb}{1.000000,0.498039,0.054902}%
\pgfsetstrokecolor{currentstroke}%
\pgfsetdash{}{0pt}%
\pgfpathmoveto{\pgfqpoint{2.233994in}{2.247058in}}%
\pgfpathcurveto{\pgfqpoint{2.245045in}{2.247058in}}{\pgfqpoint{2.255644in}{2.251449in}}{\pgfqpoint{2.263457in}{2.259262in}}%
\pgfpathcurveto{\pgfqpoint{2.271271in}{2.267076in}}{\pgfqpoint{2.275661in}{2.277675in}}{\pgfqpoint{2.275661in}{2.288725in}}%
\pgfpathcurveto{\pgfqpoint{2.275661in}{2.299775in}}{\pgfqpoint{2.271271in}{2.310374in}}{\pgfqpoint{2.263457in}{2.318188in}}%
\pgfpathcurveto{\pgfqpoint{2.255644in}{2.326002in}}{\pgfqpoint{2.245045in}{2.330392in}}{\pgfqpoint{2.233994in}{2.330392in}}%
\pgfpathcurveto{\pgfqpoint{2.222944in}{2.330392in}}{\pgfqpoint{2.212345in}{2.326002in}}{\pgfqpoint{2.204532in}{2.318188in}}%
\pgfpathcurveto{\pgfqpoint{2.196718in}{2.310374in}}{\pgfqpoint{2.192328in}{2.299775in}}{\pgfqpoint{2.192328in}{2.288725in}}%
\pgfpathcurveto{\pgfqpoint{2.192328in}{2.277675in}}{\pgfqpoint{2.196718in}{2.267076in}}{\pgfqpoint{2.204532in}{2.259262in}}%
\pgfpathcurveto{\pgfqpoint{2.212345in}{2.251449in}}{\pgfqpoint{2.222944in}{2.247058in}}{\pgfqpoint{2.233994in}{2.247058in}}%
\pgfpathclose%
\pgfusepath{stroke,fill}%
\end{pgfscope}%
\begin{pgfscope}%
\pgfpathrectangle{\pgfqpoint{0.787074in}{0.548769in}}{\pgfqpoint{4.974523in}{3.102590in}}%
\pgfusepath{clip}%
\pgfsetbuttcap%
\pgfsetroundjoin%
\definecolor{currentfill}{rgb}{1.000000,0.498039,0.054902}%
\pgfsetfillcolor{currentfill}%
\pgfsetlinewidth{1.003750pt}%
\definecolor{currentstroke}{rgb}{1.000000,0.498039,0.054902}%
\pgfsetstrokecolor{currentstroke}%
\pgfsetdash{}{0pt}%
\pgfpathmoveto{\pgfqpoint{4.202426in}{2.894377in}}%
\pgfpathcurveto{\pgfqpoint{4.213476in}{2.894377in}}{\pgfqpoint{4.224075in}{2.898767in}}{\pgfqpoint{4.231889in}{2.906581in}}%
\pgfpathcurveto{\pgfqpoint{4.239703in}{2.914394in}}{\pgfqpoint{4.244093in}{2.924993in}}{\pgfqpoint{4.244093in}{2.936044in}}%
\pgfpathcurveto{\pgfqpoint{4.244093in}{2.947094in}}{\pgfqpoint{4.239703in}{2.957693in}}{\pgfqpoint{4.231889in}{2.965506in}}%
\pgfpathcurveto{\pgfqpoint{4.224075in}{2.973320in}}{\pgfqpoint{4.213476in}{2.977710in}}{\pgfqpoint{4.202426in}{2.977710in}}%
\pgfpathcurveto{\pgfqpoint{4.191376in}{2.977710in}}{\pgfqpoint{4.180777in}{2.973320in}}{\pgfqpoint{4.172963in}{2.965506in}}%
\pgfpathcurveto{\pgfqpoint{4.165150in}{2.957693in}}{\pgfqpoint{4.160760in}{2.947094in}}{\pgfqpoint{4.160760in}{2.936044in}}%
\pgfpathcurveto{\pgfqpoint{4.160760in}{2.924993in}}{\pgfqpoint{4.165150in}{2.914394in}}{\pgfqpoint{4.172963in}{2.906581in}}%
\pgfpathcurveto{\pgfqpoint{4.180777in}{2.898767in}}{\pgfqpoint{4.191376in}{2.894377in}}{\pgfqpoint{4.202426in}{2.894377in}}%
\pgfpathclose%
\pgfusepath{stroke,fill}%
\end{pgfscope}%
\begin{pgfscope}%
\pgfpathrectangle{\pgfqpoint{0.787074in}{0.548769in}}{\pgfqpoint{4.974523in}{3.102590in}}%
\pgfusepath{clip}%
\pgfsetbuttcap%
\pgfsetroundjoin%
\definecolor{currentfill}{rgb}{1.000000,0.498039,0.054902}%
\pgfsetfillcolor{currentfill}%
\pgfsetlinewidth{1.003750pt}%
\definecolor{currentstroke}{rgb}{1.000000,0.498039,0.054902}%
\pgfsetstrokecolor{currentstroke}%
\pgfsetdash{}{0pt}%
\pgfpathmoveto{\pgfqpoint{4.384760in}{3.298118in}}%
\pgfpathcurveto{\pgfqpoint{4.395810in}{3.298118in}}{\pgfqpoint{4.406409in}{3.302508in}}{\pgfqpoint{4.414223in}{3.310321in}}%
\pgfpathcurveto{\pgfqpoint{4.422036in}{3.318135in}}{\pgfqpoint{4.426427in}{3.328734in}}{\pgfqpoint{4.426427in}{3.339784in}}%
\pgfpathcurveto{\pgfqpoint{4.426427in}{3.350834in}}{\pgfqpoint{4.422036in}{3.361433in}}{\pgfqpoint{4.414223in}{3.369247in}}%
\pgfpathcurveto{\pgfqpoint{4.406409in}{3.377061in}}{\pgfqpoint{4.395810in}{3.381451in}}{\pgfqpoint{4.384760in}{3.381451in}}%
\pgfpathcurveto{\pgfqpoint{4.373710in}{3.381451in}}{\pgfqpoint{4.363111in}{3.377061in}}{\pgfqpoint{4.355297in}{3.369247in}}%
\pgfpathcurveto{\pgfqpoint{4.347484in}{3.361433in}}{\pgfqpoint{4.343093in}{3.350834in}}{\pgfqpoint{4.343093in}{3.339784in}}%
\pgfpathcurveto{\pgfqpoint{4.343093in}{3.328734in}}{\pgfqpoint{4.347484in}{3.318135in}}{\pgfqpoint{4.355297in}{3.310321in}}%
\pgfpathcurveto{\pgfqpoint{4.363111in}{3.302508in}}{\pgfqpoint{4.373710in}{3.298118in}}{\pgfqpoint{4.384760in}{3.298118in}}%
\pgfpathclose%
\pgfusepath{stroke,fill}%
\end{pgfscope}%
\begin{pgfscope}%
\pgfpathrectangle{\pgfqpoint{0.787074in}{0.548769in}}{\pgfqpoint{4.974523in}{3.102590in}}%
\pgfusepath{clip}%
\pgfsetbuttcap%
\pgfsetroundjoin%
\definecolor{currentfill}{rgb}{1.000000,0.498039,0.054902}%
\pgfsetfillcolor{currentfill}%
\pgfsetlinewidth{1.003750pt}%
\definecolor{currentstroke}{rgb}{1.000000,0.498039,0.054902}%
\pgfsetstrokecolor{currentstroke}%
\pgfsetdash{}{0pt}%
\pgfpathmoveto{\pgfqpoint{2.612678in}{2.566130in}}%
\pgfpathcurveto{\pgfqpoint{2.623728in}{2.566130in}}{\pgfqpoint{2.634327in}{2.570520in}}{\pgfqpoint{2.642141in}{2.578334in}}%
\pgfpathcurveto{\pgfqpoint{2.649955in}{2.586148in}}{\pgfqpoint{2.654345in}{2.596747in}}{\pgfqpoint{2.654345in}{2.607797in}}%
\pgfpathcurveto{\pgfqpoint{2.654345in}{2.618847in}}{\pgfqpoint{2.649955in}{2.629446in}}{\pgfqpoint{2.642141in}{2.637260in}}%
\pgfpathcurveto{\pgfqpoint{2.634327in}{2.645073in}}{\pgfqpoint{2.623728in}{2.649463in}}{\pgfqpoint{2.612678in}{2.649463in}}%
\pgfpathcurveto{\pgfqpoint{2.601628in}{2.649463in}}{\pgfqpoint{2.591029in}{2.645073in}}{\pgfqpoint{2.583215in}{2.637260in}}%
\pgfpathcurveto{\pgfqpoint{2.575402in}{2.629446in}}{\pgfqpoint{2.571011in}{2.618847in}}{\pgfqpoint{2.571011in}{2.607797in}}%
\pgfpathcurveto{\pgfqpoint{2.571011in}{2.596747in}}{\pgfqpoint{2.575402in}{2.586148in}}{\pgfqpoint{2.583215in}{2.578334in}}%
\pgfpathcurveto{\pgfqpoint{2.591029in}{2.570520in}}{\pgfqpoint{2.601628in}{2.566130in}}{\pgfqpoint{2.612678in}{2.566130in}}%
\pgfpathclose%
\pgfusepath{stroke,fill}%
\end{pgfscope}%
\begin{pgfscope}%
\pgfpathrectangle{\pgfqpoint{0.787074in}{0.548769in}}{\pgfqpoint{4.974523in}{3.102590in}}%
\pgfusepath{clip}%
\pgfsetbuttcap%
\pgfsetroundjoin%
\definecolor{currentfill}{rgb}{0.121569,0.466667,0.705882}%
\pgfsetfillcolor{currentfill}%
\pgfsetlinewidth{1.003750pt}%
\definecolor{currentstroke}{rgb}{0.121569,0.466667,0.705882}%
\pgfsetstrokecolor{currentstroke}%
\pgfsetdash{}{0pt}%
\pgfpathmoveto{\pgfqpoint{3.544672in}{2.795473in}}%
\pgfpathcurveto{\pgfqpoint{3.555722in}{2.795473in}}{\pgfqpoint{3.566321in}{2.799863in}}{\pgfqpoint{3.574134in}{2.807676in}}%
\pgfpathcurveto{\pgfqpoint{3.581948in}{2.815490in}}{\pgfqpoint{3.586338in}{2.826089in}}{\pgfqpoint{3.586338in}{2.837139in}}%
\pgfpathcurveto{\pgfqpoint{3.586338in}{2.848189in}}{\pgfqpoint{3.581948in}{2.858788in}}{\pgfqpoint{3.574134in}{2.866602in}}%
\pgfpathcurveto{\pgfqpoint{3.566321in}{2.874416in}}{\pgfqpoint{3.555722in}{2.878806in}}{\pgfqpoint{3.544672in}{2.878806in}}%
\pgfpathcurveto{\pgfqpoint{3.533621in}{2.878806in}}{\pgfqpoint{3.523022in}{2.874416in}}{\pgfqpoint{3.515209in}{2.866602in}}%
\pgfpathcurveto{\pgfqpoint{3.507395in}{2.858788in}}{\pgfqpoint{3.503005in}{2.848189in}}{\pgfqpoint{3.503005in}{2.837139in}}%
\pgfpathcurveto{\pgfqpoint{3.503005in}{2.826089in}}{\pgfqpoint{3.507395in}{2.815490in}}{\pgfqpoint{3.515209in}{2.807676in}}%
\pgfpathcurveto{\pgfqpoint{3.523022in}{2.799863in}}{\pgfqpoint{3.533621in}{2.795473in}}{\pgfqpoint{3.544672in}{2.795473in}}%
\pgfpathclose%
\pgfusepath{stroke,fill}%
\end{pgfscope}%
\begin{pgfscope}%
\pgfpathrectangle{\pgfqpoint{0.787074in}{0.548769in}}{\pgfqpoint{4.974523in}{3.102590in}}%
\pgfusepath{clip}%
\pgfsetbuttcap%
\pgfsetroundjoin%
\definecolor{currentfill}{rgb}{1.000000,0.498039,0.054902}%
\pgfsetfillcolor{currentfill}%
\pgfsetlinewidth{1.003750pt}%
\definecolor{currentstroke}{rgb}{1.000000,0.498039,0.054902}%
\pgfsetstrokecolor{currentstroke}%
\pgfsetdash{}{0pt}%
\pgfpathmoveto{\pgfqpoint{4.278317in}{3.027648in}}%
\pgfpathcurveto{\pgfqpoint{4.289367in}{3.027648in}}{\pgfqpoint{4.299966in}{3.032039in}}{\pgfqpoint{4.307780in}{3.039852in}}%
\pgfpathcurveto{\pgfqpoint{4.315594in}{3.047666in}}{\pgfqpoint{4.319984in}{3.058265in}}{\pgfqpoint{4.319984in}{3.069315in}}%
\pgfpathcurveto{\pgfqpoint{4.319984in}{3.080365in}}{\pgfqpoint{4.315594in}{3.090964in}}{\pgfqpoint{4.307780in}{3.098778in}}%
\pgfpathcurveto{\pgfqpoint{4.299966in}{3.106591in}}{\pgfqpoint{4.289367in}{3.110982in}}{\pgfqpoint{4.278317in}{3.110982in}}%
\pgfpathcurveto{\pgfqpoint{4.267267in}{3.110982in}}{\pgfqpoint{4.256668in}{3.106591in}}{\pgfqpoint{4.248855in}{3.098778in}}%
\pgfpathcurveto{\pgfqpoint{4.241041in}{3.090964in}}{\pgfqpoint{4.236651in}{3.080365in}}{\pgfqpoint{4.236651in}{3.069315in}}%
\pgfpathcurveto{\pgfqpoint{4.236651in}{3.058265in}}{\pgfqpoint{4.241041in}{3.047666in}}{\pgfqpoint{4.248855in}{3.039852in}}%
\pgfpathcurveto{\pgfqpoint{4.256668in}{3.032039in}}{\pgfqpoint{4.267267in}{3.027648in}}{\pgfqpoint{4.278317in}{3.027648in}}%
\pgfpathclose%
\pgfusepath{stroke,fill}%
\end{pgfscope}%
\begin{pgfscope}%
\pgfpathrectangle{\pgfqpoint{0.787074in}{0.548769in}}{\pgfqpoint{4.974523in}{3.102590in}}%
\pgfusepath{clip}%
\pgfsetbuttcap%
\pgfsetroundjoin%
\definecolor{currentfill}{rgb}{1.000000,0.498039,0.054902}%
\pgfsetfillcolor{currentfill}%
\pgfsetlinewidth{1.003750pt}%
\definecolor{currentstroke}{rgb}{1.000000,0.498039,0.054902}%
\pgfsetstrokecolor{currentstroke}%
\pgfsetdash{}{0pt}%
\pgfpathmoveto{\pgfqpoint{4.454160in}{2.815221in}}%
\pgfpathcurveto{\pgfqpoint{4.465210in}{2.815221in}}{\pgfqpoint{4.475809in}{2.819611in}}{\pgfqpoint{4.483623in}{2.827425in}}%
\pgfpathcurveto{\pgfqpoint{4.491436in}{2.835238in}}{\pgfqpoint{4.495827in}{2.845837in}}{\pgfqpoint{4.495827in}{2.856887in}}%
\pgfpathcurveto{\pgfqpoint{4.495827in}{2.867937in}}{\pgfqpoint{4.491436in}{2.878537in}}{\pgfqpoint{4.483623in}{2.886350in}}%
\pgfpathcurveto{\pgfqpoint{4.475809in}{2.894164in}}{\pgfqpoint{4.465210in}{2.898554in}}{\pgfqpoint{4.454160in}{2.898554in}}%
\pgfpathcurveto{\pgfqpoint{4.443110in}{2.898554in}}{\pgfqpoint{4.432511in}{2.894164in}}{\pgfqpoint{4.424697in}{2.886350in}}%
\pgfpathcurveto{\pgfqpoint{4.416884in}{2.878537in}}{\pgfqpoint{4.412493in}{2.867937in}}{\pgfqpoint{4.412493in}{2.856887in}}%
\pgfpathcurveto{\pgfqpoint{4.412493in}{2.845837in}}{\pgfqpoint{4.416884in}{2.835238in}}{\pgfqpoint{4.424697in}{2.827425in}}%
\pgfpathcurveto{\pgfqpoint{4.432511in}{2.819611in}}{\pgfqpoint{4.443110in}{2.815221in}}{\pgfqpoint{4.454160in}{2.815221in}}%
\pgfpathclose%
\pgfusepath{stroke,fill}%
\end{pgfscope}%
\begin{pgfscope}%
\pgfpathrectangle{\pgfqpoint{0.787074in}{0.548769in}}{\pgfqpoint{4.974523in}{3.102590in}}%
\pgfusepath{clip}%
\pgfsetbuttcap%
\pgfsetroundjoin%
\definecolor{currentfill}{rgb}{1.000000,0.498039,0.054902}%
\pgfsetfillcolor{currentfill}%
\pgfsetlinewidth{1.003750pt}%
\definecolor{currentstroke}{rgb}{1.000000,0.498039,0.054902}%
\pgfsetstrokecolor{currentstroke}%
\pgfsetdash{}{0pt}%
\pgfpathmoveto{\pgfqpoint{1.721964in}{1.272567in}}%
\pgfpathcurveto{\pgfqpoint{1.733014in}{1.272567in}}{\pgfqpoint{1.743613in}{1.276958in}}{\pgfqpoint{1.751426in}{1.284771in}}%
\pgfpathcurveto{\pgfqpoint{1.759240in}{1.292585in}}{\pgfqpoint{1.763630in}{1.303184in}}{\pgfqpoint{1.763630in}{1.314234in}}%
\pgfpathcurveto{\pgfqpoint{1.763630in}{1.325284in}}{\pgfqpoint{1.759240in}{1.335883in}}{\pgfqpoint{1.751426in}{1.343697in}}%
\pgfpathcurveto{\pgfqpoint{1.743613in}{1.351510in}}{\pgfqpoint{1.733014in}{1.355901in}}{\pgfqpoint{1.721964in}{1.355901in}}%
\pgfpathcurveto{\pgfqpoint{1.710914in}{1.355901in}}{\pgfqpoint{1.700315in}{1.351510in}}{\pgfqpoint{1.692501in}{1.343697in}}%
\pgfpathcurveto{\pgfqpoint{1.684687in}{1.335883in}}{\pgfqpoint{1.680297in}{1.325284in}}{\pgfqpoint{1.680297in}{1.314234in}}%
\pgfpathcurveto{\pgfqpoint{1.680297in}{1.303184in}}{\pgfqpoint{1.684687in}{1.292585in}}{\pgfqpoint{1.692501in}{1.284771in}}%
\pgfpathcurveto{\pgfqpoint{1.700315in}{1.276958in}}{\pgfqpoint{1.710914in}{1.272567in}}{\pgfqpoint{1.721964in}{1.272567in}}%
\pgfpathclose%
\pgfusepath{stroke,fill}%
\end{pgfscope}%
\begin{pgfscope}%
\pgfpathrectangle{\pgfqpoint{0.787074in}{0.548769in}}{\pgfqpoint{4.974523in}{3.102590in}}%
\pgfusepath{clip}%
\pgfsetbuttcap%
\pgfsetroundjoin%
\definecolor{currentfill}{rgb}{0.121569,0.466667,0.705882}%
\pgfsetfillcolor{currentfill}%
\pgfsetlinewidth{1.003750pt}%
\definecolor{currentstroke}{rgb}{0.121569,0.466667,0.705882}%
\pgfsetstrokecolor{currentstroke}%
\pgfsetdash{}{0pt}%
\pgfpathmoveto{\pgfqpoint{2.130498in}{1.853887in}}%
\pgfpathcurveto{\pgfqpoint{2.141548in}{1.853887in}}{\pgfqpoint{2.152147in}{1.858277in}}{\pgfqpoint{2.159961in}{1.866090in}}%
\pgfpathcurveto{\pgfqpoint{2.167775in}{1.873904in}}{\pgfqpoint{2.172165in}{1.884503in}}{\pgfqpoint{2.172165in}{1.895553in}}%
\pgfpathcurveto{\pgfqpoint{2.172165in}{1.906603in}}{\pgfqpoint{2.167775in}{1.917202in}}{\pgfqpoint{2.159961in}{1.925016in}}%
\pgfpathcurveto{\pgfqpoint{2.152147in}{1.932830in}}{\pgfqpoint{2.141548in}{1.937220in}}{\pgfqpoint{2.130498in}{1.937220in}}%
\pgfpathcurveto{\pgfqpoint{2.119448in}{1.937220in}}{\pgfqpoint{2.108849in}{1.932830in}}{\pgfqpoint{2.101035in}{1.925016in}}%
\pgfpathcurveto{\pgfqpoint{2.093222in}{1.917202in}}{\pgfqpoint{2.088832in}{1.906603in}}{\pgfqpoint{2.088832in}{1.895553in}}%
\pgfpathcurveto{\pgfqpoint{2.088832in}{1.884503in}}{\pgfqpoint{2.093222in}{1.873904in}}{\pgfqpoint{2.101035in}{1.866090in}}%
\pgfpathcurveto{\pgfqpoint{2.108849in}{1.858277in}}{\pgfqpoint{2.119448in}{1.853887in}}{\pgfqpoint{2.130498in}{1.853887in}}%
\pgfpathclose%
\pgfusepath{stroke,fill}%
\end{pgfscope}%
\begin{pgfscope}%
\pgfpathrectangle{\pgfqpoint{0.787074in}{0.548769in}}{\pgfqpoint{4.974523in}{3.102590in}}%
\pgfusepath{clip}%
\pgfsetbuttcap%
\pgfsetroundjoin%
\definecolor{currentfill}{rgb}{0.121569,0.466667,0.705882}%
\pgfsetfillcolor{currentfill}%
\pgfsetlinewidth{1.003750pt}%
\definecolor{currentstroke}{rgb}{0.121569,0.466667,0.705882}%
\pgfsetstrokecolor{currentstroke}%
\pgfsetdash{}{0pt}%
\pgfpathmoveto{\pgfqpoint{3.983710in}{2.914142in}}%
\pgfpathcurveto{\pgfqpoint{3.994761in}{2.914142in}}{\pgfqpoint{4.005360in}{2.918533in}}{\pgfqpoint{4.013173in}{2.926346in}}%
\pgfpathcurveto{\pgfqpoint{4.020987in}{2.934160in}}{\pgfqpoint{4.025377in}{2.944759in}}{\pgfqpoint{4.025377in}{2.955809in}}%
\pgfpathcurveto{\pgfqpoint{4.025377in}{2.966859in}}{\pgfqpoint{4.020987in}{2.977458in}}{\pgfqpoint{4.013173in}{2.985272in}}%
\pgfpathcurveto{\pgfqpoint{4.005360in}{2.993085in}}{\pgfqpoint{3.994761in}{2.997476in}}{\pgfqpoint{3.983710in}{2.997476in}}%
\pgfpathcurveto{\pgfqpoint{3.972660in}{2.997476in}}{\pgfqpoint{3.962061in}{2.993085in}}{\pgfqpoint{3.954248in}{2.985272in}}%
\pgfpathcurveto{\pgfqpoint{3.946434in}{2.977458in}}{\pgfqpoint{3.942044in}{2.966859in}}{\pgfqpoint{3.942044in}{2.955809in}}%
\pgfpathcurveto{\pgfqpoint{3.942044in}{2.944759in}}{\pgfqpoint{3.946434in}{2.934160in}}{\pgfqpoint{3.954248in}{2.926346in}}%
\pgfpathcurveto{\pgfqpoint{3.962061in}{2.918533in}}{\pgfqpoint{3.972660in}{2.914142in}}{\pgfqpoint{3.983710in}{2.914142in}}%
\pgfpathclose%
\pgfusepath{stroke,fill}%
\end{pgfscope}%
\begin{pgfscope}%
\pgfpathrectangle{\pgfqpoint{0.787074in}{0.548769in}}{\pgfqpoint{4.974523in}{3.102590in}}%
\pgfusepath{clip}%
\pgfsetbuttcap%
\pgfsetroundjoin%
\definecolor{currentfill}{rgb}{1.000000,0.498039,0.054902}%
\pgfsetfillcolor{currentfill}%
\pgfsetlinewidth{1.003750pt}%
\definecolor{currentstroke}{rgb}{1.000000,0.498039,0.054902}%
\pgfsetstrokecolor{currentstroke}%
\pgfsetdash{}{0pt}%
\pgfpathmoveto{\pgfqpoint{3.568285in}{2.606637in}}%
\pgfpathcurveto{\pgfqpoint{3.579335in}{2.606637in}}{\pgfqpoint{3.589934in}{2.611027in}}{\pgfqpoint{3.597748in}{2.618841in}}%
\pgfpathcurveto{\pgfqpoint{3.605562in}{2.626654in}}{\pgfqpoint{3.609952in}{2.637253in}}{\pgfqpoint{3.609952in}{2.648303in}}%
\pgfpathcurveto{\pgfqpoint{3.609952in}{2.659353in}}{\pgfqpoint{3.605562in}{2.669952in}}{\pgfqpoint{3.597748in}{2.677766in}}%
\pgfpathcurveto{\pgfqpoint{3.589934in}{2.685580in}}{\pgfqpoint{3.579335in}{2.689970in}}{\pgfqpoint{3.568285in}{2.689970in}}%
\pgfpathcurveto{\pgfqpoint{3.557235in}{2.689970in}}{\pgfqpoint{3.546636in}{2.685580in}}{\pgfqpoint{3.538822in}{2.677766in}}%
\pgfpathcurveto{\pgfqpoint{3.531009in}{2.669952in}}{\pgfqpoint{3.526619in}{2.659353in}}{\pgfqpoint{3.526619in}{2.648303in}}%
\pgfpathcurveto{\pgfqpoint{3.526619in}{2.637253in}}{\pgfqpoint{3.531009in}{2.626654in}}{\pgfqpoint{3.538822in}{2.618841in}}%
\pgfpathcurveto{\pgfqpoint{3.546636in}{2.611027in}}{\pgfqpoint{3.557235in}{2.606637in}}{\pgfqpoint{3.568285in}{2.606637in}}%
\pgfpathclose%
\pgfusepath{stroke,fill}%
\end{pgfscope}%
\begin{pgfscope}%
\pgfpathrectangle{\pgfqpoint{0.787074in}{0.548769in}}{\pgfqpoint{4.974523in}{3.102590in}}%
\pgfusepath{clip}%
\pgfsetbuttcap%
\pgfsetroundjoin%
\definecolor{currentfill}{rgb}{0.839216,0.152941,0.156863}%
\pgfsetfillcolor{currentfill}%
\pgfsetlinewidth{1.003750pt}%
\definecolor{currentstroke}{rgb}{0.839216,0.152941,0.156863}%
\pgfsetstrokecolor{currentstroke}%
\pgfsetdash{}{0pt}%
\pgfpathmoveto{\pgfqpoint{3.329238in}{2.324482in}}%
\pgfpathcurveto{\pgfqpoint{3.340288in}{2.324482in}}{\pgfqpoint{3.350887in}{2.328873in}}{\pgfqpoint{3.358701in}{2.336686in}}%
\pgfpathcurveto{\pgfqpoint{3.366514in}{2.344500in}}{\pgfqpoint{3.370905in}{2.355099in}}{\pgfqpoint{3.370905in}{2.366149in}}%
\pgfpathcurveto{\pgfqpoint{3.370905in}{2.377199in}}{\pgfqpoint{3.366514in}{2.387798in}}{\pgfqpoint{3.358701in}{2.395612in}}%
\pgfpathcurveto{\pgfqpoint{3.350887in}{2.403426in}}{\pgfqpoint{3.340288in}{2.407816in}}{\pgfqpoint{3.329238in}{2.407816in}}%
\pgfpathcurveto{\pgfqpoint{3.318188in}{2.407816in}}{\pgfqpoint{3.307589in}{2.403426in}}{\pgfqpoint{3.299775in}{2.395612in}}%
\pgfpathcurveto{\pgfqpoint{3.291962in}{2.387798in}}{\pgfqpoint{3.287571in}{2.377199in}}{\pgfqpoint{3.287571in}{2.366149in}}%
\pgfpathcurveto{\pgfqpoint{3.287571in}{2.355099in}}{\pgfqpoint{3.291962in}{2.344500in}}{\pgfqpoint{3.299775in}{2.336686in}}%
\pgfpathcurveto{\pgfqpoint{3.307589in}{2.328873in}}{\pgfqpoint{3.318188in}{2.324482in}}{\pgfqpoint{3.329238in}{2.324482in}}%
\pgfpathclose%
\pgfusepath{stroke,fill}%
\end{pgfscope}%
\begin{pgfscope}%
\pgfpathrectangle{\pgfqpoint{0.787074in}{0.548769in}}{\pgfqpoint{4.974523in}{3.102590in}}%
\pgfusepath{clip}%
\pgfsetbuttcap%
\pgfsetroundjoin%
\definecolor{currentfill}{rgb}{1.000000,0.498039,0.054902}%
\pgfsetfillcolor{currentfill}%
\pgfsetlinewidth{1.003750pt}%
\definecolor{currentstroke}{rgb}{1.000000,0.498039,0.054902}%
\pgfsetstrokecolor{currentstroke}%
\pgfsetdash{}{0pt}%
\pgfpathmoveto{\pgfqpoint{4.634223in}{2.951278in}}%
\pgfpathcurveto{\pgfqpoint{4.645274in}{2.951278in}}{\pgfqpoint{4.655873in}{2.955668in}}{\pgfqpoint{4.663686in}{2.963482in}}%
\pgfpathcurveto{\pgfqpoint{4.671500in}{2.971295in}}{\pgfqpoint{4.675890in}{2.981894in}}{\pgfqpoint{4.675890in}{2.992945in}}%
\pgfpathcurveto{\pgfqpoint{4.675890in}{3.003995in}}{\pgfqpoint{4.671500in}{3.014594in}}{\pgfqpoint{4.663686in}{3.022407in}}%
\pgfpathcurveto{\pgfqpoint{4.655873in}{3.030221in}}{\pgfqpoint{4.645274in}{3.034611in}}{\pgfqpoint{4.634223in}{3.034611in}}%
\pgfpathcurveto{\pgfqpoint{4.623173in}{3.034611in}}{\pgfqpoint{4.612574in}{3.030221in}}{\pgfqpoint{4.604761in}{3.022407in}}%
\pgfpathcurveto{\pgfqpoint{4.596947in}{3.014594in}}{\pgfqpoint{4.592557in}{3.003995in}}{\pgfqpoint{4.592557in}{2.992945in}}%
\pgfpathcurveto{\pgfqpoint{4.592557in}{2.981894in}}{\pgfqpoint{4.596947in}{2.971295in}}{\pgfqpoint{4.604761in}{2.963482in}}%
\pgfpathcurveto{\pgfqpoint{4.612574in}{2.955668in}}{\pgfqpoint{4.623173in}{2.951278in}}{\pgfqpoint{4.634223in}{2.951278in}}%
\pgfpathclose%
\pgfusepath{stroke,fill}%
\end{pgfscope}%
\begin{pgfscope}%
\pgfpathrectangle{\pgfqpoint{0.787074in}{0.548769in}}{\pgfqpoint{4.974523in}{3.102590in}}%
\pgfusepath{clip}%
\pgfsetbuttcap%
\pgfsetroundjoin%
\definecolor{currentfill}{rgb}{0.121569,0.466667,0.705882}%
\pgfsetfillcolor{currentfill}%
\pgfsetlinewidth{1.003750pt}%
\definecolor{currentstroke}{rgb}{0.121569,0.466667,0.705882}%
\pgfsetstrokecolor{currentstroke}%
\pgfsetdash{}{0pt}%
\pgfpathmoveto{\pgfqpoint{3.883398in}{2.536597in}}%
\pgfpathcurveto{\pgfqpoint{3.894448in}{2.536597in}}{\pgfqpoint{3.905047in}{2.540987in}}{\pgfqpoint{3.912861in}{2.548801in}}%
\pgfpathcurveto{\pgfqpoint{3.920674in}{2.556614in}}{\pgfqpoint{3.925065in}{2.567213in}}{\pgfqpoint{3.925065in}{2.578263in}}%
\pgfpathcurveto{\pgfqpoint{3.925065in}{2.589314in}}{\pgfqpoint{3.920674in}{2.599913in}}{\pgfqpoint{3.912861in}{2.607726in}}%
\pgfpathcurveto{\pgfqpoint{3.905047in}{2.615540in}}{\pgfqpoint{3.894448in}{2.619930in}}{\pgfqpoint{3.883398in}{2.619930in}}%
\pgfpathcurveto{\pgfqpoint{3.872348in}{2.619930in}}{\pgfqpoint{3.861749in}{2.615540in}}{\pgfqpoint{3.853935in}{2.607726in}}%
\pgfpathcurveto{\pgfqpoint{3.846122in}{2.599913in}}{\pgfqpoint{3.841731in}{2.589314in}}{\pgfqpoint{3.841731in}{2.578263in}}%
\pgfpathcurveto{\pgfqpoint{3.841731in}{2.567213in}}{\pgfqpoint{3.846122in}{2.556614in}}{\pgfqpoint{3.853935in}{2.548801in}}%
\pgfpathcurveto{\pgfqpoint{3.861749in}{2.540987in}}{\pgfqpoint{3.872348in}{2.536597in}}{\pgfqpoint{3.883398in}{2.536597in}}%
\pgfpathclose%
\pgfusepath{stroke,fill}%
\end{pgfscope}%
\begin{pgfscope}%
\pgfpathrectangle{\pgfqpoint{0.787074in}{0.548769in}}{\pgfqpoint{4.974523in}{3.102590in}}%
\pgfusepath{clip}%
\pgfsetbuttcap%
\pgfsetroundjoin%
\definecolor{currentfill}{rgb}{1.000000,0.498039,0.054902}%
\pgfsetfillcolor{currentfill}%
\pgfsetlinewidth{1.003750pt}%
\definecolor{currentstroke}{rgb}{1.000000,0.498039,0.054902}%
\pgfsetstrokecolor{currentstroke}%
\pgfsetdash{}{0pt}%
\pgfpathmoveto{\pgfqpoint{2.841031in}{2.094657in}}%
\pgfpathcurveto{\pgfqpoint{2.852081in}{2.094657in}}{\pgfqpoint{2.862680in}{2.099047in}}{\pgfqpoint{2.870494in}{2.106861in}}%
\pgfpathcurveto{\pgfqpoint{2.878308in}{2.114675in}}{\pgfqpoint{2.882698in}{2.125274in}}{\pgfqpoint{2.882698in}{2.136324in}}%
\pgfpathcurveto{\pgfqpoint{2.882698in}{2.147374in}}{\pgfqpoint{2.878308in}{2.157973in}}{\pgfqpoint{2.870494in}{2.165787in}}%
\pgfpathcurveto{\pgfqpoint{2.862680in}{2.173600in}}{\pgfqpoint{2.852081in}{2.177990in}}{\pgfqpoint{2.841031in}{2.177990in}}%
\pgfpathcurveto{\pgfqpoint{2.829981in}{2.177990in}}{\pgfqpoint{2.819382in}{2.173600in}}{\pgfqpoint{2.811569in}{2.165787in}}%
\pgfpathcurveto{\pgfqpoint{2.803755in}{2.157973in}}{\pgfqpoint{2.799365in}{2.147374in}}{\pgfqpoint{2.799365in}{2.136324in}}%
\pgfpathcurveto{\pgfqpoint{2.799365in}{2.125274in}}{\pgfqpoint{2.803755in}{2.114675in}}{\pgfqpoint{2.811569in}{2.106861in}}%
\pgfpathcurveto{\pgfqpoint{2.819382in}{2.099047in}}{\pgfqpoint{2.829981in}{2.094657in}}{\pgfqpoint{2.841031in}{2.094657in}}%
\pgfpathclose%
\pgfusepath{stroke,fill}%
\end{pgfscope}%
\begin{pgfscope}%
\pgfpathrectangle{\pgfqpoint{0.787074in}{0.548769in}}{\pgfqpoint{4.974523in}{3.102590in}}%
\pgfusepath{clip}%
\pgfsetbuttcap%
\pgfsetroundjoin%
\definecolor{currentfill}{rgb}{1.000000,0.498039,0.054902}%
\pgfsetfillcolor{currentfill}%
\pgfsetlinewidth{1.003750pt}%
\definecolor{currentstroke}{rgb}{1.000000,0.498039,0.054902}%
\pgfsetstrokecolor{currentstroke}%
\pgfsetdash{}{0pt}%
\pgfpathmoveto{\pgfqpoint{2.414809in}{2.224129in}}%
\pgfpathcurveto{\pgfqpoint{2.425859in}{2.224129in}}{\pgfqpoint{2.436458in}{2.228519in}}{\pgfqpoint{2.444272in}{2.236332in}}%
\pgfpathcurveto{\pgfqpoint{2.452085in}{2.244146in}}{\pgfqpoint{2.456476in}{2.254745in}}{\pgfqpoint{2.456476in}{2.265795in}}%
\pgfpathcurveto{\pgfqpoint{2.456476in}{2.276845in}}{\pgfqpoint{2.452085in}{2.287444in}}{\pgfqpoint{2.444272in}{2.295258in}}%
\pgfpathcurveto{\pgfqpoint{2.436458in}{2.303072in}}{\pgfqpoint{2.425859in}{2.307462in}}{\pgfqpoint{2.414809in}{2.307462in}}%
\pgfpathcurveto{\pgfqpoint{2.403759in}{2.307462in}}{\pgfqpoint{2.393160in}{2.303072in}}{\pgfqpoint{2.385346in}{2.295258in}}%
\pgfpathcurveto{\pgfqpoint{2.377533in}{2.287444in}}{\pgfqpoint{2.373142in}{2.276845in}}{\pgfqpoint{2.373142in}{2.265795in}}%
\pgfpathcurveto{\pgfqpoint{2.373142in}{2.254745in}}{\pgfqpoint{2.377533in}{2.244146in}}{\pgfqpoint{2.385346in}{2.236332in}}%
\pgfpathcurveto{\pgfqpoint{2.393160in}{2.228519in}}{\pgfqpoint{2.403759in}{2.224129in}}{\pgfqpoint{2.414809in}{2.224129in}}%
\pgfpathclose%
\pgfusepath{stroke,fill}%
\end{pgfscope}%
\begin{pgfscope}%
\pgfpathrectangle{\pgfqpoint{0.787074in}{0.548769in}}{\pgfqpoint{4.974523in}{3.102590in}}%
\pgfusepath{clip}%
\pgfsetbuttcap%
\pgfsetroundjoin%
\definecolor{currentfill}{rgb}{1.000000,0.498039,0.054902}%
\pgfsetfillcolor{currentfill}%
\pgfsetlinewidth{1.003750pt}%
\definecolor{currentstroke}{rgb}{1.000000,0.498039,0.054902}%
\pgfsetstrokecolor{currentstroke}%
\pgfsetdash{}{0pt}%
\pgfpathmoveto{\pgfqpoint{2.323034in}{2.050679in}}%
\pgfpathcurveto{\pgfqpoint{2.334084in}{2.050679in}}{\pgfqpoint{2.344684in}{2.055070in}}{\pgfqpoint{2.352497in}{2.062883in}}%
\pgfpathcurveto{\pgfqpoint{2.360311in}{2.070697in}}{\pgfqpoint{2.364701in}{2.081296in}}{\pgfqpoint{2.364701in}{2.092346in}}%
\pgfpathcurveto{\pgfqpoint{2.364701in}{2.103396in}}{\pgfqpoint{2.360311in}{2.113995in}}{\pgfqpoint{2.352497in}{2.121809in}}%
\pgfpathcurveto{\pgfqpoint{2.344684in}{2.129622in}}{\pgfqpoint{2.334084in}{2.134013in}}{\pgfqpoint{2.323034in}{2.134013in}}%
\pgfpathcurveto{\pgfqpoint{2.311984in}{2.134013in}}{\pgfqpoint{2.301385in}{2.129622in}}{\pgfqpoint{2.293572in}{2.121809in}}%
\pgfpathcurveto{\pgfqpoint{2.285758in}{2.113995in}}{\pgfqpoint{2.281368in}{2.103396in}}{\pgfqpoint{2.281368in}{2.092346in}}%
\pgfpathcurveto{\pgfqpoint{2.281368in}{2.081296in}}{\pgfqpoint{2.285758in}{2.070697in}}{\pgfqpoint{2.293572in}{2.062883in}}%
\pgfpathcurveto{\pgfqpoint{2.301385in}{2.055070in}}{\pgfqpoint{2.311984in}{2.050679in}}{\pgfqpoint{2.323034in}{2.050679in}}%
\pgfpathclose%
\pgfusepath{stroke,fill}%
\end{pgfscope}%
\begin{pgfscope}%
\pgfpathrectangle{\pgfqpoint{0.787074in}{0.548769in}}{\pgfqpoint{4.974523in}{3.102590in}}%
\pgfusepath{clip}%
\pgfsetbuttcap%
\pgfsetroundjoin%
\definecolor{currentfill}{rgb}{0.121569,0.466667,0.705882}%
\pgfsetfillcolor{currentfill}%
\pgfsetlinewidth{1.003750pt}%
\definecolor{currentstroke}{rgb}{0.121569,0.466667,0.705882}%
\pgfsetstrokecolor{currentstroke}%
\pgfsetdash{}{0pt}%
\pgfpathmoveto{\pgfqpoint{3.711846in}{2.613502in}}%
\pgfpathcurveto{\pgfqpoint{3.722896in}{2.613502in}}{\pgfqpoint{3.733495in}{2.617892in}}{\pgfqpoint{3.741308in}{2.625706in}}%
\pgfpathcurveto{\pgfqpoint{3.749122in}{2.633519in}}{\pgfqpoint{3.753512in}{2.644118in}}{\pgfqpoint{3.753512in}{2.655169in}}%
\pgfpathcurveto{\pgfqpoint{3.753512in}{2.666219in}}{\pgfqpoint{3.749122in}{2.676818in}}{\pgfqpoint{3.741308in}{2.684631in}}%
\pgfpathcurveto{\pgfqpoint{3.733495in}{2.692445in}}{\pgfqpoint{3.722896in}{2.696835in}}{\pgfqpoint{3.711846in}{2.696835in}}%
\pgfpathcurveto{\pgfqpoint{3.700796in}{2.696835in}}{\pgfqpoint{3.690197in}{2.692445in}}{\pgfqpoint{3.682383in}{2.684631in}}%
\pgfpathcurveto{\pgfqpoint{3.674569in}{2.676818in}}{\pgfqpoint{3.670179in}{2.666219in}}{\pgfqpoint{3.670179in}{2.655169in}}%
\pgfpathcurveto{\pgfqpoint{3.670179in}{2.644118in}}{\pgfqpoint{3.674569in}{2.633519in}}{\pgfqpoint{3.682383in}{2.625706in}}%
\pgfpathcurveto{\pgfqpoint{3.690197in}{2.617892in}}{\pgfqpoint{3.700796in}{2.613502in}}{\pgfqpoint{3.711846in}{2.613502in}}%
\pgfpathclose%
\pgfusepath{stroke,fill}%
\end{pgfscope}%
\begin{pgfscope}%
\pgfpathrectangle{\pgfqpoint{0.787074in}{0.548769in}}{\pgfqpoint{4.974523in}{3.102590in}}%
\pgfusepath{clip}%
\pgfsetbuttcap%
\pgfsetroundjoin%
\definecolor{currentfill}{rgb}{0.121569,0.466667,0.705882}%
\pgfsetfillcolor{currentfill}%
\pgfsetlinewidth{1.003750pt}%
\definecolor{currentstroke}{rgb}{0.121569,0.466667,0.705882}%
\pgfsetstrokecolor{currentstroke}%
\pgfsetdash{}{0pt}%
\pgfpathmoveto{\pgfqpoint{4.013661in}{2.701999in}}%
\pgfpathcurveto{\pgfqpoint{4.024711in}{2.701999in}}{\pgfqpoint{4.035310in}{2.706389in}}{\pgfqpoint{4.043124in}{2.714203in}}%
\pgfpathcurveto{\pgfqpoint{4.050937in}{2.722016in}}{\pgfqpoint{4.055328in}{2.732616in}}{\pgfqpoint{4.055328in}{2.743666in}}%
\pgfpathcurveto{\pgfqpoint{4.055328in}{2.754716in}}{\pgfqpoint{4.050937in}{2.765315in}}{\pgfqpoint{4.043124in}{2.773128in}}%
\pgfpathcurveto{\pgfqpoint{4.035310in}{2.780942in}}{\pgfqpoint{4.024711in}{2.785332in}}{\pgfqpoint{4.013661in}{2.785332in}}%
\pgfpathcurveto{\pgfqpoint{4.002611in}{2.785332in}}{\pgfqpoint{3.992012in}{2.780942in}}{\pgfqpoint{3.984198in}{2.773128in}}%
\pgfpathcurveto{\pgfqpoint{3.976384in}{2.765315in}}{\pgfqpoint{3.971994in}{2.754716in}}{\pgfqpoint{3.971994in}{2.743666in}}%
\pgfpathcurveto{\pgfqpoint{3.971994in}{2.732616in}}{\pgfqpoint{3.976384in}{2.722016in}}{\pgfqpoint{3.984198in}{2.714203in}}%
\pgfpathcurveto{\pgfqpoint{3.992012in}{2.706389in}}{\pgfqpoint{4.002611in}{2.701999in}}{\pgfqpoint{4.013661in}{2.701999in}}%
\pgfpathclose%
\pgfusepath{stroke,fill}%
\end{pgfscope}%
\begin{pgfscope}%
\pgfpathrectangle{\pgfqpoint{0.787074in}{0.548769in}}{\pgfqpoint{4.974523in}{3.102590in}}%
\pgfusepath{clip}%
\pgfsetbuttcap%
\pgfsetroundjoin%
\definecolor{currentfill}{rgb}{0.121569,0.466667,0.705882}%
\pgfsetfillcolor{currentfill}%
\pgfsetlinewidth{1.003750pt}%
\definecolor{currentstroke}{rgb}{0.121569,0.466667,0.705882}%
\pgfsetstrokecolor{currentstroke}%
\pgfsetdash{}{0pt}%
\pgfpathmoveto{\pgfqpoint{4.103873in}{2.930649in}}%
\pgfpathcurveto{\pgfqpoint{4.114924in}{2.930649in}}{\pgfqpoint{4.125523in}{2.935039in}}{\pgfqpoint{4.133336in}{2.942853in}}%
\pgfpathcurveto{\pgfqpoint{4.141150in}{2.950666in}}{\pgfqpoint{4.145540in}{2.961265in}}{\pgfqpoint{4.145540in}{2.972316in}}%
\pgfpathcurveto{\pgfqpoint{4.145540in}{2.983366in}}{\pgfqpoint{4.141150in}{2.993965in}}{\pgfqpoint{4.133336in}{3.001778in}}%
\pgfpathcurveto{\pgfqpoint{4.125523in}{3.009592in}}{\pgfqpoint{4.114924in}{3.013982in}}{\pgfqpoint{4.103873in}{3.013982in}}%
\pgfpathcurveto{\pgfqpoint{4.092823in}{3.013982in}}{\pgfqpoint{4.082224in}{3.009592in}}{\pgfqpoint{4.074411in}{3.001778in}}%
\pgfpathcurveto{\pgfqpoint{4.066597in}{2.993965in}}{\pgfqpoint{4.062207in}{2.983366in}}{\pgfqpoint{4.062207in}{2.972316in}}%
\pgfpathcurveto{\pgfqpoint{4.062207in}{2.961265in}}{\pgfqpoint{4.066597in}{2.950666in}}{\pgfqpoint{4.074411in}{2.942853in}}%
\pgfpathcurveto{\pgfqpoint{4.082224in}{2.935039in}}{\pgfqpoint{4.092823in}{2.930649in}}{\pgfqpoint{4.103873in}{2.930649in}}%
\pgfpathclose%
\pgfusepath{stroke,fill}%
\end{pgfscope}%
\begin{pgfscope}%
\pgfpathrectangle{\pgfqpoint{0.787074in}{0.548769in}}{\pgfqpoint{4.974523in}{3.102590in}}%
\pgfusepath{clip}%
\pgfsetbuttcap%
\pgfsetroundjoin%
\definecolor{currentfill}{rgb}{0.121569,0.466667,0.705882}%
\pgfsetfillcolor{currentfill}%
\pgfsetlinewidth{1.003750pt}%
\definecolor{currentstroke}{rgb}{0.121569,0.466667,0.705882}%
\pgfsetstrokecolor{currentstroke}%
\pgfsetdash{}{0pt}%
\pgfpathmoveto{\pgfqpoint{1.005419in}{0.648134in}}%
\pgfpathcurveto{\pgfqpoint{1.016469in}{0.648134in}}{\pgfqpoint{1.027068in}{0.652525in}}{\pgfqpoint{1.034881in}{0.660338in}}%
\pgfpathcurveto{\pgfqpoint{1.042695in}{0.668152in}}{\pgfqpoint{1.047085in}{0.678751in}}{\pgfqpoint{1.047085in}{0.689801in}}%
\pgfpathcurveto{\pgfqpoint{1.047085in}{0.700851in}}{\pgfqpoint{1.042695in}{0.711450in}}{\pgfqpoint{1.034881in}{0.719264in}}%
\pgfpathcurveto{\pgfqpoint{1.027068in}{0.727077in}}{\pgfqpoint{1.016469in}{0.731468in}}{\pgfqpoint{1.005419in}{0.731468in}}%
\pgfpathcurveto{\pgfqpoint{0.994368in}{0.731468in}}{\pgfqpoint{0.983769in}{0.727077in}}{\pgfqpoint{0.975956in}{0.719264in}}%
\pgfpathcurveto{\pgfqpoint{0.968142in}{0.711450in}}{\pgfqpoint{0.963752in}{0.700851in}}{\pgfqpoint{0.963752in}{0.689801in}}%
\pgfpathcurveto{\pgfqpoint{0.963752in}{0.678751in}}{\pgfqpoint{0.968142in}{0.668152in}}{\pgfqpoint{0.975956in}{0.660338in}}%
\pgfpathcurveto{\pgfqpoint{0.983769in}{0.652525in}}{\pgfqpoint{0.994368in}{0.648134in}}{\pgfqpoint{1.005419in}{0.648134in}}%
\pgfpathclose%
\pgfusepath{stroke,fill}%
\end{pgfscope}%
\begin{pgfscope}%
\pgfpathrectangle{\pgfqpoint{0.787074in}{0.548769in}}{\pgfqpoint{4.974523in}{3.102590in}}%
\pgfusepath{clip}%
\pgfsetbuttcap%
\pgfsetroundjoin%
\definecolor{currentfill}{rgb}{1.000000,0.498039,0.054902}%
\pgfsetfillcolor{currentfill}%
\pgfsetlinewidth{1.003750pt}%
\definecolor{currentstroke}{rgb}{1.000000,0.498039,0.054902}%
\pgfsetstrokecolor{currentstroke}%
\pgfsetdash{}{0pt}%
\pgfpathmoveto{\pgfqpoint{1.828314in}{1.583814in}}%
\pgfpathcurveto{\pgfqpoint{1.839365in}{1.583814in}}{\pgfqpoint{1.849964in}{1.588205in}}{\pgfqpoint{1.857777in}{1.596018in}}%
\pgfpathcurveto{\pgfqpoint{1.865591in}{1.603832in}}{\pgfqpoint{1.869981in}{1.614431in}}{\pgfqpoint{1.869981in}{1.625481in}}%
\pgfpathcurveto{\pgfqpoint{1.869981in}{1.636531in}}{\pgfqpoint{1.865591in}{1.647130in}}{\pgfqpoint{1.857777in}{1.654944in}}%
\pgfpathcurveto{\pgfqpoint{1.849964in}{1.662758in}}{\pgfqpoint{1.839365in}{1.667148in}}{\pgfqpoint{1.828314in}{1.667148in}}%
\pgfpathcurveto{\pgfqpoint{1.817264in}{1.667148in}}{\pgfqpoint{1.806665in}{1.662758in}}{\pgfqpoint{1.798852in}{1.654944in}}%
\pgfpathcurveto{\pgfqpoint{1.791038in}{1.647130in}}{\pgfqpoint{1.786648in}{1.636531in}}{\pgfqpoint{1.786648in}{1.625481in}}%
\pgfpathcurveto{\pgfqpoint{1.786648in}{1.614431in}}{\pgfqpoint{1.791038in}{1.603832in}}{\pgfqpoint{1.798852in}{1.596018in}}%
\pgfpathcurveto{\pgfqpoint{1.806665in}{1.588205in}}{\pgfqpoint{1.817264in}{1.583814in}}{\pgfqpoint{1.828314in}{1.583814in}}%
\pgfpathclose%
\pgfusepath{stroke,fill}%
\end{pgfscope}%
\begin{pgfscope}%
\pgfpathrectangle{\pgfqpoint{0.787074in}{0.548769in}}{\pgfqpoint{4.974523in}{3.102590in}}%
\pgfusepath{clip}%
\pgfsetbuttcap%
\pgfsetroundjoin%
\definecolor{currentfill}{rgb}{1.000000,0.498039,0.054902}%
\pgfsetfillcolor{currentfill}%
\pgfsetlinewidth{1.003750pt}%
\definecolor{currentstroke}{rgb}{1.000000,0.498039,0.054902}%
\pgfsetstrokecolor{currentstroke}%
\pgfsetdash{}{0pt}%
\pgfpathmoveto{\pgfqpoint{4.240217in}{3.152602in}}%
\pgfpathcurveto{\pgfqpoint{4.251267in}{3.152602in}}{\pgfqpoint{4.261867in}{3.156992in}}{\pgfqpoint{4.269680in}{3.164806in}}%
\pgfpathcurveto{\pgfqpoint{4.277494in}{3.172619in}}{\pgfqpoint{4.281884in}{3.183218in}}{\pgfqpoint{4.281884in}{3.194268in}}%
\pgfpathcurveto{\pgfqpoint{4.281884in}{3.205319in}}{\pgfqpoint{4.277494in}{3.215918in}}{\pgfqpoint{4.269680in}{3.223731in}}%
\pgfpathcurveto{\pgfqpoint{4.261867in}{3.231545in}}{\pgfqpoint{4.251267in}{3.235935in}}{\pgfqpoint{4.240217in}{3.235935in}}%
\pgfpathcurveto{\pgfqpoint{4.229167in}{3.235935in}}{\pgfqpoint{4.218568in}{3.231545in}}{\pgfqpoint{4.210755in}{3.223731in}}%
\pgfpathcurveto{\pgfqpoint{4.202941in}{3.215918in}}{\pgfqpoint{4.198551in}{3.205319in}}{\pgfqpoint{4.198551in}{3.194268in}}%
\pgfpathcurveto{\pgfqpoint{4.198551in}{3.183218in}}{\pgfqpoint{4.202941in}{3.172619in}}{\pgfqpoint{4.210755in}{3.164806in}}%
\pgfpathcurveto{\pgfqpoint{4.218568in}{3.156992in}}{\pgfqpoint{4.229167in}{3.152602in}}{\pgfqpoint{4.240217in}{3.152602in}}%
\pgfpathclose%
\pgfusepath{stroke,fill}%
\end{pgfscope}%
\begin{pgfscope}%
\pgfpathrectangle{\pgfqpoint{0.787074in}{0.548769in}}{\pgfqpoint{4.974523in}{3.102590in}}%
\pgfusepath{clip}%
\pgfsetbuttcap%
\pgfsetroundjoin%
\definecolor{currentfill}{rgb}{1.000000,0.498039,0.054902}%
\pgfsetfillcolor{currentfill}%
\pgfsetlinewidth{1.003750pt}%
\definecolor{currentstroke}{rgb}{1.000000,0.498039,0.054902}%
\pgfsetstrokecolor{currentstroke}%
\pgfsetdash{}{0pt}%
\pgfpathmoveto{\pgfqpoint{4.526739in}{2.882713in}}%
\pgfpathcurveto{\pgfqpoint{4.537789in}{2.882713in}}{\pgfqpoint{4.548388in}{2.887104in}}{\pgfqpoint{4.556202in}{2.894917in}}%
\pgfpathcurveto{\pgfqpoint{4.564015in}{2.902731in}}{\pgfqpoint{4.568406in}{2.913330in}}{\pgfqpoint{4.568406in}{2.924380in}}%
\pgfpathcurveto{\pgfqpoint{4.568406in}{2.935430in}}{\pgfqpoint{4.564015in}{2.946029in}}{\pgfqpoint{4.556202in}{2.953843in}}%
\pgfpathcurveto{\pgfqpoint{4.548388in}{2.961656in}}{\pgfqpoint{4.537789in}{2.966047in}}{\pgfqpoint{4.526739in}{2.966047in}}%
\pgfpathcurveto{\pgfqpoint{4.515689in}{2.966047in}}{\pgfqpoint{4.505090in}{2.961656in}}{\pgfqpoint{4.497276in}{2.953843in}}%
\pgfpathcurveto{\pgfqpoint{4.489462in}{2.946029in}}{\pgfqpoint{4.485072in}{2.935430in}}{\pgfqpoint{4.485072in}{2.924380in}}%
\pgfpathcurveto{\pgfqpoint{4.485072in}{2.913330in}}{\pgfqpoint{4.489462in}{2.902731in}}{\pgfqpoint{4.497276in}{2.894917in}}%
\pgfpathcurveto{\pgfqpoint{4.505090in}{2.887104in}}{\pgfqpoint{4.515689in}{2.882713in}}{\pgfqpoint{4.526739in}{2.882713in}}%
\pgfpathclose%
\pgfusepath{stroke,fill}%
\end{pgfscope}%
\begin{pgfscope}%
\pgfpathrectangle{\pgfqpoint{0.787074in}{0.548769in}}{\pgfqpoint{4.974523in}{3.102590in}}%
\pgfusepath{clip}%
\pgfsetbuttcap%
\pgfsetroundjoin%
\definecolor{currentfill}{rgb}{0.121569,0.466667,0.705882}%
\pgfsetfillcolor{currentfill}%
\pgfsetlinewidth{1.003750pt}%
\definecolor{currentstroke}{rgb}{0.121569,0.466667,0.705882}%
\pgfsetstrokecolor{currentstroke}%
\pgfsetdash{}{0pt}%
\pgfpathmoveto{\pgfqpoint{1.154939in}{0.749337in}}%
\pgfpathcurveto{\pgfqpoint{1.165989in}{0.749337in}}{\pgfqpoint{1.176588in}{0.753727in}}{\pgfqpoint{1.184402in}{0.761541in}}%
\pgfpathcurveto{\pgfqpoint{1.192215in}{0.769355in}}{\pgfqpoint{1.196606in}{0.779954in}}{\pgfqpoint{1.196606in}{0.791004in}}%
\pgfpathcurveto{\pgfqpoint{1.196606in}{0.802054in}}{\pgfqpoint{1.192215in}{0.812653in}}{\pgfqpoint{1.184402in}{0.820467in}}%
\pgfpathcurveto{\pgfqpoint{1.176588in}{0.828280in}}{\pgfqpoint{1.165989in}{0.832670in}}{\pgfqpoint{1.154939in}{0.832670in}}%
\pgfpathcurveto{\pgfqpoint{1.143889in}{0.832670in}}{\pgfqpoint{1.133290in}{0.828280in}}{\pgfqpoint{1.125476in}{0.820467in}}%
\pgfpathcurveto{\pgfqpoint{1.117663in}{0.812653in}}{\pgfqpoint{1.113272in}{0.802054in}}{\pgfqpoint{1.113272in}{0.791004in}}%
\pgfpathcurveto{\pgfqpoint{1.113272in}{0.779954in}}{\pgfqpoint{1.117663in}{0.769355in}}{\pgfqpoint{1.125476in}{0.761541in}}%
\pgfpathcurveto{\pgfqpoint{1.133290in}{0.753727in}}{\pgfqpoint{1.143889in}{0.749337in}}{\pgfqpoint{1.154939in}{0.749337in}}%
\pgfpathclose%
\pgfusepath{stroke,fill}%
\end{pgfscope}%
\begin{pgfscope}%
\pgfpathrectangle{\pgfqpoint{0.787074in}{0.548769in}}{\pgfqpoint{4.974523in}{3.102590in}}%
\pgfusepath{clip}%
\pgfsetbuttcap%
\pgfsetroundjoin%
\definecolor{currentfill}{rgb}{1.000000,0.498039,0.054902}%
\pgfsetfillcolor{currentfill}%
\pgfsetlinewidth{1.003750pt}%
\definecolor{currentstroke}{rgb}{1.000000,0.498039,0.054902}%
\pgfsetstrokecolor{currentstroke}%
\pgfsetdash{}{0pt}%
\pgfpathmoveto{\pgfqpoint{2.071586in}{1.712229in}}%
\pgfpathcurveto{\pgfqpoint{2.082636in}{1.712229in}}{\pgfqpoint{2.093235in}{1.716619in}}{\pgfqpoint{2.101049in}{1.724432in}}%
\pgfpathcurveto{\pgfqpoint{2.108862in}{1.732246in}}{\pgfqpoint{2.113252in}{1.742845in}}{\pgfqpoint{2.113252in}{1.753895in}}%
\pgfpathcurveto{\pgfqpoint{2.113252in}{1.764945in}}{\pgfqpoint{2.108862in}{1.775544in}}{\pgfqpoint{2.101049in}{1.783358in}}%
\pgfpathcurveto{\pgfqpoint{2.093235in}{1.791172in}}{\pgfqpoint{2.082636in}{1.795562in}}{\pgfqpoint{2.071586in}{1.795562in}}%
\pgfpathcurveto{\pgfqpoint{2.060536in}{1.795562in}}{\pgfqpoint{2.049937in}{1.791172in}}{\pgfqpoint{2.042123in}{1.783358in}}%
\pgfpathcurveto{\pgfqpoint{2.034309in}{1.775544in}}{\pgfqpoint{2.029919in}{1.764945in}}{\pgfqpoint{2.029919in}{1.753895in}}%
\pgfpathcurveto{\pgfqpoint{2.029919in}{1.742845in}}{\pgfqpoint{2.034309in}{1.732246in}}{\pgfqpoint{2.042123in}{1.724432in}}%
\pgfpathcurveto{\pgfqpoint{2.049937in}{1.716619in}}{\pgfqpoint{2.060536in}{1.712229in}}{\pgfqpoint{2.071586in}{1.712229in}}%
\pgfpathclose%
\pgfusepath{stroke,fill}%
\end{pgfscope}%
\begin{pgfscope}%
\pgfpathrectangle{\pgfqpoint{0.787074in}{0.548769in}}{\pgfqpoint{4.974523in}{3.102590in}}%
\pgfusepath{clip}%
\pgfsetbuttcap%
\pgfsetroundjoin%
\definecolor{currentfill}{rgb}{0.121569,0.466667,0.705882}%
\pgfsetfillcolor{currentfill}%
\pgfsetlinewidth{1.003750pt}%
\definecolor{currentstroke}{rgb}{0.121569,0.466667,0.705882}%
\pgfsetstrokecolor{currentstroke}%
\pgfsetdash{}{0pt}%
\pgfpathmoveto{\pgfqpoint{3.401800in}{2.635102in}}%
\pgfpathcurveto{\pgfqpoint{3.412850in}{2.635102in}}{\pgfqpoint{3.423449in}{2.639493in}}{\pgfqpoint{3.431262in}{2.647306in}}%
\pgfpathcurveto{\pgfqpoint{3.439076in}{2.655120in}}{\pgfqpoint{3.443466in}{2.665719in}}{\pgfqpoint{3.443466in}{2.676769in}}%
\pgfpathcurveto{\pgfqpoint{3.443466in}{2.687819in}}{\pgfqpoint{3.439076in}{2.698418in}}{\pgfqpoint{3.431262in}{2.706232in}}%
\pgfpathcurveto{\pgfqpoint{3.423449in}{2.714045in}}{\pgfqpoint{3.412850in}{2.718436in}}{\pgfqpoint{3.401800in}{2.718436in}}%
\pgfpathcurveto{\pgfqpoint{3.390749in}{2.718436in}}{\pgfqpoint{3.380150in}{2.714045in}}{\pgfqpoint{3.372337in}{2.706232in}}%
\pgfpathcurveto{\pgfqpoint{3.364523in}{2.698418in}}{\pgfqpoint{3.360133in}{2.687819in}}{\pgfqpoint{3.360133in}{2.676769in}}%
\pgfpathcurveto{\pgfqpoint{3.360133in}{2.665719in}}{\pgfqpoint{3.364523in}{2.655120in}}{\pgfqpoint{3.372337in}{2.647306in}}%
\pgfpathcurveto{\pgfqpoint{3.380150in}{2.639493in}}{\pgfqpoint{3.390749in}{2.635102in}}{\pgfqpoint{3.401800in}{2.635102in}}%
\pgfpathclose%
\pgfusepath{stroke,fill}%
\end{pgfscope}%
\begin{pgfscope}%
\pgfpathrectangle{\pgfqpoint{0.787074in}{0.548769in}}{\pgfqpoint{4.974523in}{3.102590in}}%
\pgfusepath{clip}%
\pgfsetbuttcap%
\pgfsetroundjoin%
\definecolor{currentfill}{rgb}{1.000000,0.498039,0.054902}%
\pgfsetfillcolor{currentfill}%
\pgfsetlinewidth{1.003750pt}%
\definecolor{currentstroke}{rgb}{1.000000,0.498039,0.054902}%
\pgfsetstrokecolor{currentstroke}%
\pgfsetdash{}{0pt}%
\pgfpathmoveto{\pgfqpoint{4.139444in}{2.957127in}}%
\pgfpathcurveto{\pgfqpoint{4.150494in}{2.957127in}}{\pgfqpoint{4.161093in}{2.961517in}}{\pgfqpoint{4.168907in}{2.969331in}}%
\pgfpathcurveto{\pgfqpoint{4.176720in}{2.977144in}}{\pgfqpoint{4.181111in}{2.987743in}}{\pgfqpoint{4.181111in}{2.998793in}}%
\pgfpathcurveto{\pgfqpoint{4.181111in}{3.009844in}}{\pgfqpoint{4.176720in}{3.020443in}}{\pgfqpoint{4.168907in}{3.028256in}}%
\pgfpathcurveto{\pgfqpoint{4.161093in}{3.036070in}}{\pgfqpoint{4.150494in}{3.040460in}}{\pgfqpoint{4.139444in}{3.040460in}}%
\pgfpathcurveto{\pgfqpoint{4.128394in}{3.040460in}}{\pgfqpoint{4.117795in}{3.036070in}}{\pgfqpoint{4.109981in}{3.028256in}}%
\pgfpathcurveto{\pgfqpoint{4.102167in}{3.020443in}}{\pgfqpoint{4.097777in}{3.009844in}}{\pgfqpoint{4.097777in}{2.998793in}}%
\pgfpathcurveto{\pgfqpoint{4.097777in}{2.987743in}}{\pgfqpoint{4.102167in}{2.977144in}}{\pgfqpoint{4.109981in}{2.969331in}}%
\pgfpathcurveto{\pgfqpoint{4.117795in}{2.961517in}}{\pgfqpoint{4.128394in}{2.957127in}}{\pgfqpoint{4.139444in}{2.957127in}}%
\pgfpathclose%
\pgfusepath{stroke,fill}%
\end{pgfscope}%
\begin{pgfscope}%
\pgfpathrectangle{\pgfqpoint{0.787074in}{0.548769in}}{\pgfqpoint{4.974523in}{3.102590in}}%
\pgfusepath{clip}%
\pgfsetbuttcap%
\pgfsetroundjoin%
\definecolor{currentfill}{rgb}{0.121569,0.466667,0.705882}%
\pgfsetfillcolor{currentfill}%
\pgfsetlinewidth{1.003750pt}%
\definecolor{currentstroke}{rgb}{0.121569,0.466667,0.705882}%
\pgfsetstrokecolor{currentstroke}%
\pgfsetdash{}{0pt}%
\pgfpathmoveto{\pgfqpoint{1.818678in}{1.266126in}}%
\pgfpathcurveto{\pgfqpoint{1.829729in}{1.266126in}}{\pgfqpoint{1.840328in}{1.270516in}}{\pgfqpoint{1.848141in}{1.278330in}}%
\pgfpathcurveto{\pgfqpoint{1.855955in}{1.286143in}}{\pgfqpoint{1.860345in}{1.296742in}}{\pgfqpoint{1.860345in}{1.307792in}}%
\pgfpathcurveto{\pgfqpoint{1.860345in}{1.318842in}}{\pgfqpoint{1.855955in}{1.329441in}}{\pgfqpoint{1.848141in}{1.337255in}}%
\pgfpathcurveto{\pgfqpoint{1.840328in}{1.345069in}}{\pgfqpoint{1.829729in}{1.349459in}}{\pgfqpoint{1.818678in}{1.349459in}}%
\pgfpathcurveto{\pgfqpoint{1.807628in}{1.349459in}}{\pgfqpoint{1.797029in}{1.345069in}}{\pgfqpoint{1.789216in}{1.337255in}}%
\pgfpathcurveto{\pgfqpoint{1.781402in}{1.329441in}}{\pgfqpoint{1.777012in}{1.318842in}}{\pgfqpoint{1.777012in}{1.307792in}}%
\pgfpathcurveto{\pgfqpoint{1.777012in}{1.296742in}}{\pgfqpoint{1.781402in}{1.286143in}}{\pgfqpoint{1.789216in}{1.278330in}}%
\pgfpathcurveto{\pgfqpoint{1.797029in}{1.270516in}}{\pgfqpoint{1.807628in}{1.266126in}}{\pgfqpoint{1.818678in}{1.266126in}}%
\pgfpathclose%
\pgfusepath{stroke,fill}%
\end{pgfscope}%
\begin{pgfscope}%
\pgfpathrectangle{\pgfqpoint{0.787074in}{0.548769in}}{\pgfqpoint{4.974523in}{3.102590in}}%
\pgfusepath{clip}%
\pgfsetbuttcap%
\pgfsetroundjoin%
\definecolor{currentfill}{rgb}{1.000000,0.498039,0.054902}%
\pgfsetfillcolor{currentfill}%
\pgfsetlinewidth{1.003750pt}%
\definecolor{currentstroke}{rgb}{1.000000,0.498039,0.054902}%
\pgfsetstrokecolor{currentstroke}%
\pgfsetdash{}{0pt}%
\pgfpathmoveto{\pgfqpoint{3.454455in}{2.376418in}}%
\pgfpathcurveto{\pgfqpoint{3.465505in}{2.376418in}}{\pgfqpoint{3.476105in}{2.380808in}}{\pgfqpoint{3.483918in}{2.388622in}}%
\pgfpathcurveto{\pgfqpoint{3.491732in}{2.396436in}}{\pgfqpoint{3.496122in}{2.407035in}}{\pgfqpoint{3.496122in}{2.418085in}}%
\pgfpathcurveto{\pgfqpoint{3.496122in}{2.429135in}}{\pgfqpoint{3.491732in}{2.439734in}}{\pgfqpoint{3.483918in}{2.447548in}}%
\pgfpathcurveto{\pgfqpoint{3.476105in}{2.455361in}}{\pgfqpoint{3.465505in}{2.459751in}}{\pgfqpoint{3.454455in}{2.459751in}}%
\pgfpathcurveto{\pgfqpoint{3.443405in}{2.459751in}}{\pgfqpoint{3.432806in}{2.455361in}}{\pgfqpoint{3.424993in}{2.447548in}}%
\pgfpathcurveto{\pgfqpoint{3.417179in}{2.439734in}}{\pgfqpoint{3.412789in}{2.429135in}}{\pgfqpoint{3.412789in}{2.418085in}}%
\pgfpathcurveto{\pgfqpoint{3.412789in}{2.407035in}}{\pgfqpoint{3.417179in}{2.396436in}}{\pgfqpoint{3.424993in}{2.388622in}}%
\pgfpathcurveto{\pgfqpoint{3.432806in}{2.380808in}}{\pgfqpoint{3.443405in}{2.376418in}}{\pgfqpoint{3.454455in}{2.376418in}}%
\pgfpathclose%
\pgfusepath{stroke,fill}%
\end{pgfscope}%
\begin{pgfscope}%
\pgfpathrectangle{\pgfqpoint{0.787074in}{0.548769in}}{\pgfqpoint{4.974523in}{3.102590in}}%
\pgfusepath{clip}%
\pgfsetbuttcap%
\pgfsetroundjoin%
\definecolor{currentfill}{rgb}{1.000000,0.498039,0.054902}%
\pgfsetfillcolor{currentfill}%
\pgfsetlinewidth{1.003750pt}%
\definecolor{currentstroke}{rgb}{1.000000,0.498039,0.054902}%
\pgfsetstrokecolor{currentstroke}%
\pgfsetdash{}{0pt}%
\pgfpathmoveto{\pgfqpoint{4.117286in}{3.006227in}}%
\pgfpathcurveto{\pgfqpoint{4.128336in}{3.006227in}}{\pgfqpoint{4.138935in}{3.010617in}}{\pgfqpoint{4.146749in}{3.018431in}}%
\pgfpathcurveto{\pgfqpoint{4.154562in}{3.026244in}}{\pgfqpoint{4.158952in}{3.036844in}}{\pgfqpoint{4.158952in}{3.047894in}}%
\pgfpathcurveto{\pgfqpoint{4.158952in}{3.058944in}}{\pgfqpoint{4.154562in}{3.069543in}}{\pgfqpoint{4.146749in}{3.077356in}}%
\pgfpathcurveto{\pgfqpoint{4.138935in}{3.085170in}}{\pgfqpoint{4.128336in}{3.089560in}}{\pgfqpoint{4.117286in}{3.089560in}}%
\pgfpathcurveto{\pgfqpoint{4.106236in}{3.089560in}}{\pgfqpoint{4.095637in}{3.085170in}}{\pgfqpoint{4.087823in}{3.077356in}}%
\pgfpathcurveto{\pgfqpoint{4.080009in}{3.069543in}}{\pgfqpoint{4.075619in}{3.058944in}}{\pgfqpoint{4.075619in}{3.047894in}}%
\pgfpathcurveto{\pgfqpoint{4.075619in}{3.036844in}}{\pgfqpoint{4.080009in}{3.026244in}}{\pgfqpoint{4.087823in}{3.018431in}}%
\pgfpathcurveto{\pgfqpoint{4.095637in}{3.010617in}}{\pgfqpoint{4.106236in}{3.006227in}}{\pgfqpoint{4.117286in}{3.006227in}}%
\pgfpathclose%
\pgfusepath{stroke,fill}%
\end{pgfscope}%
\begin{pgfscope}%
\pgfpathrectangle{\pgfqpoint{0.787074in}{0.548769in}}{\pgfqpoint{4.974523in}{3.102590in}}%
\pgfusepath{clip}%
\pgfsetbuttcap%
\pgfsetroundjoin%
\definecolor{currentfill}{rgb}{1.000000,0.498039,0.054902}%
\pgfsetfillcolor{currentfill}%
\pgfsetlinewidth{1.003750pt}%
\definecolor{currentstroke}{rgb}{1.000000,0.498039,0.054902}%
\pgfsetstrokecolor{currentstroke}%
\pgfsetdash{}{0pt}%
\pgfpathmoveto{\pgfqpoint{2.821344in}{2.381259in}}%
\pgfpathcurveto{\pgfqpoint{2.832395in}{2.381259in}}{\pgfqpoint{2.842994in}{2.385649in}}{\pgfqpoint{2.850807in}{2.393463in}}%
\pgfpathcurveto{\pgfqpoint{2.858621in}{2.401277in}}{\pgfqpoint{2.863011in}{2.411876in}}{\pgfqpoint{2.863011in}{2.422926in}}%
\pgfpathcurveto{\pgfqpoint{2.863011in}{2.433976in}}{\pgfqpoint{2.858621in}{2.444575in}}{\pgfqpoint{2.850807in}{2.452388in}}%
\pgfpathcurveto{\pgfqpoint{2.842994in}{2.460202in}}{\pgfqpoint{2.832395in}{2.464592in}}{\pgfqpoint{2.821344in}{2.464592in}}%
\pgfpathcurveto{\pgfqpoint{2.810294in}{2.464592in}}{\pgfqpoint{2.799695in}{2.460202in}}{\pgfqpoint{2.791882in}{2.452388in}}%
\pgfpathcurveto{\pgfqpoint{2.784068in}{2.444575in}}{\pgfqpoint{2.779678in}{2.433976in}}{\pgfqpoint{2.779678in}{2.422926in}}%
\pgfpathcurveto{\pgfqpoint{2.779678in}{2.411876in}}{\pgfqpoint{2.784068in}{2.401277in}}{\pgfqpoint{2.791882in}{2.393463in}}%
\pgfpathcurveto{\pgfqpoint{2.799695in}{2.385649in}}{\pgfqpoint{2.810294in}{2.381259in}}{\pgfqpoint{2.821344in}{2.381259in}}%
\pgfpathclose%
\pgfusepath{stroke,fill}%
\end{pgfscope}%
\begin{pgfscope}%
\pgfpathrectangle{\pgfqpoint{0.787074in}{0.548769in}}{\pgfqpoint{4.974523in}{3.102590in}}%
\pgfusepath{clip}%
\pgfsetbuttcap%
\pgfsetroundjoin%
\definecolor{currentfill}{rgb}{1.000000,0.498039,0.054902}%
\pgfsetfillcolor{currentfill}%
\pgfsetlinewidth{1.003750pt}%
\definecolor{currentstroke}{rgb}{1.000000,0.498039,0.054902}%
\pgfsetstrokecolor{currentstroke}%
\pgfsetdash{}{0pt}%
\pgfpathmoveto{\pgfqpoint{2.437559in}{1.619069in}}%
\pgfpathcurveto{\pgfqpoint{2.448609in}{1.619069in}}{\pgfqpoint{2.459208in}{1.623460in}}{\pgfqpoint{2.467022in}{1.631273in}}%
\pgfpathcurveto{\pgfqpoint{2.474835in}{1.639087in}}{\pgfqpoint{2.479225in}{1.649686in}}{\pgfqpoint{2.479225in}{1.660736in}}%
\pgfpathcurveto{\pgfqpoint{2.479225in}{1.671786in}}{\pgfqpoint{2.474835in}{1.682385in}}{\pgfqpoint{2.467022in}{1.690199in}}%
\pgfpathcurveto{\pgfqpoint{2.459208in}{1.698012in}}{\pgfqpoint{2.448609in}{1.702403in}}{\pgfqpoint{2.437559in}{1.702403in}}%
\pgfpathcurveto{\pgfqpoint{2.426509in}{1.702403in}}{\pgfqpoint{2.415910in}{1.698012in}}{\pgfqpoint{2.408096in}{1.690199in}}%
\pgfpathcurveto{\pgfqpoint{2.400282in}{1.682385in}}{\pgfqpoint{2.395892in}{1.671786in}}{\pgfqpoint{2.395892in}{1.660736in}}%
\pgfpathcurveto{\pgfqpoint{2.395892in}{1.649686in}}{\pgfqpoint{2.400282in}{1.639087in}}{\pgfqpoint{2.408096in}{1.631273in}}%
\pgfpathcurveto{\pgfqpoint{2.415910in}{1.623460in}}{\pgfqpoint{2.426509in}{1.619069in}}{\pgfqpoint{2.437559in}{1.619069in}}%
\pgfpathclose%
\pgfusepath{stroke,fill}%
\end{pgfscope}%
\begin{pgfscope}%
\pgfpathrectangle{\pgfqpoint{0.787074in}{0.548769in}}{\pgfqpoint{4.974523in}{3.102590in}}%
\pgfusepath{clip}%
\pgfsetbuttcap%
\pgfsetroundjoin%
\definecolor{currentfill}{rgb}{1.000000,0.498039,0.054902}%
\pgfsetfillcolor{currentfill}%
\pgfsetlinewidth{1.003750pt}%
\definecolor{currentstroke}{rgb}{1.000000,0.498039,0.054902}%
\pgfsetstrokecolor{currentstroke}%
\pgfsetdash{}{0pt}%
\pgfpathmoveto{\pgfqpoint{4.299909in}{2.847775in}}%
\pgfpathcurveto{\pgfqpoint{4.310960in}{2.847775in}}{\pgfqpoint{4.321559in}{2.852165in}}{\pgfqpoint{4.329372in}{2.859979in}}%
\pgfpathcurveto{\pgfqpoint{4.337186in}{2.867793in}}{\pgfqpoint{4.341576in}{2.878392in}}{\pgfqpoint{4.341576in}{2.889442in}}%
\pgfpathcurveto{\pgfqpoint{4.341576in}{2.900492in}}{\pgfqpoint{4.337186in}{2.911091in}}{\pgfqpoint{4.329372in}{2.918904in}}%
\pgfpathcurveto{\pgfqpoint{4.321559in}{2.926718in}}{\pgfqpoint{4.310960in}{2.931108in}}{\pgfqpoint{4.299909in}{2.931108in}}%
\pgfpathcurveto{\pgfqpoint{4.288859in}{2.931108in}}{\pgfqpoint{4.278260in}{2.926718in}}{\pgfqpoint{4.270447in}{2.918904in}}%
\pgfpathcurveto{\pgfqpoint{4.262633in}{2.911091in}}{\pgfqpoint{4.258243in}{2.900492in}}{\pgfqpoint{4.258243in}{2.889442in}}%
\pgfpathcurveto{\pgfqpoint{4.258243in}{2.878392in}}{\pgfqpoint{4.262633in}{2.867793in}}{\pgfqpoint{4.270447in}{2.859979in}}%
\pgfpathcurveto{\pgfqpoint{4.278260in}{2.852165in}}{\pgfqpoint{4.288859in}{2.847775in}}{\pgfqpoint{4.299909in}{2.847775in}}%
\pgfpathclose%
\pgfusepath{stroke,fill}%
\end{pgfscope}%
\begin{pgfscope}%
\pgfpathrectangle{\pgfqpoint{0.787074in}{0.548769in}}{\pgfqpoint{4.974523in}{3.102590in}}%
\pgfusepath{clip}%
\pgfsetbuttcap%
\pgfsetroundjoin%
\definecolor{currentfill}{rgb}{1.000000,0.498039,0.054902}%
\pgfsetfillcolor{currentfill}%
\pgfsetlinewidth{1.003750pt}%
\definecolor{currentstroke}{rgb}{1.000000,0.498039,0.054902}%
\pgfsetstrokecolor{currentstroke}%
\pgfsetdash{}{0pt}%
\pgfpathmoveto{\pgfqpoint{2.893394in}{2.241897in}}%
\pgfpathcurveto{\pgfqpoint{2.904445in}{2.241897in}}{\pgfqpoint{2.915044in}{2.246287in}}{\pgfqpoint{2.922857in}{2.254101in}}%
\pgfpathcurveto{\pgfqpoint{2.930671in}{2.261914in}}{\pgfqpoint{2.935061in}{2.272513in}}{\pgfqpoint{2.935061in}{2.283563in}}%
\pgfpathcurveto{\pgfqpoint{2.935061in}{2.294613in}}{\pgfqpoint{2.930671in}{2.305212in}}{\pgfqpoint{2.922857in}{2.313026in}}%
\pgfpathcurveto{\pgfqpoint{2.915044in}{2.320840in}}{\pgfqpoint{2.904445in}{2.325230in}}{\pgfqpoint{2.893394in}{2.325230in}}%
\pgfpathcurveto{\pgfqpoint{2.882344in}{2.325230in}}{\pgfqpoint{2.871745in}{2.320840in}}{\pgfqpoint{2.863932in}{2.313026in}}%
\pgfpathcurveto{\pgfqpoint{2.856118in}{2.305212in}}{\pgfqpoint{2.851728in}{2.294613in}}{\pgfqpoint{2.851728in}{2.283563in}}%
\pgfpathcurveto{\pgfqpoint{2.851728in}{2.272513in}}{\pgfqpoint{2.856118in}{2.261914in}}{\pgfqpoint{2.863932in}{2.254101in}}%
\pgfpathcurveto{\pgfqpoint{2.871745in}{2.246287in}}{\pgfqpoint{2.882344in}{2.241897in}}{\pgfqpoint{2.893394in}{2.241897in}}%
\pgfpathclose%
\pgfusepath{stroke,fill}%
\end{pgfscope}%
\begin{pgfscope}%
\pgfpathrectangle{\pgfqpoint{0.787074in}{0.548769in}}{\pgfqpoint{4.974523in}{3.102590in}}%
\pgfusepath{clip}%
\pgfsetbuttcap%
\pgfsetroundjoin%
\definecolor{currentfill}{rgb}{1.000000,0.498039,0.054902}%
\pgfsetfillcolor{currentfill}%
\pgfsetlinewidth{1.003750pt}%
\definecolor{currentstroke}{rgb}{1.000000,0.498039,0.054902}%
\pgfsetstrokecolor{currentstroke}%
\pgfsetdash{}{0pt}%
\pgfpathmoveto{\pgfqpoint{2.699924in}{1.903634in}}%
\pgfpathcurveto{\pgfqpoint{2.710974in}{1.903634in}}{\pgfqpoint{2.721573in}{1.908024in}}{\pgfqpoint{2.729387in}{1.915838in}}%
\pgfpathcurveto{\pgfqpoint{2.737200in}{1.923652in}}{\pgfqpoint{2.741591in}{1.934251in}}{\pgfqpoint{2.741591in}{1.945301in}}%
\pgfpathcurveto{\pgfqpoint{2.741591in}{1.956351in}}{\pgfqpoint{2.737200in}{1.966950in}}{\pgfqpoint{2.729387in}{1.974764in}}%
\pgfpathcurveto{\pgfqpoint{2.721573in}{1.982577in}}{\pgfqpoint{2.710974in}{1.986968in}}{\pgfqpoint{2.699924in}{1.986968in}}%
\pgfpathcurveto{\pgfqpoint{2.688874in}{1.986968in}}{\pgfqpoint{2.678275in}{1.982577in}}{\pgfqpoint{2.670461in}{1.974764in}}%
\pgfpathcurveto{\pgfqpoint{2.662648in}{1.966950in}}{\pgfqpoint{2.658257in}{1.956351in}}{\pgfqpoint{2.658257in}{1.945301in}}%
\pgfpathcurveto{\pgfqpoint{2.658257in}{1.934251in}}{\pgfqpoint{2.662648in}{1.923652in}}{\pgfqpoint{2.670461in}{1.915838in}}%
\pgfpathcurveto{\pgfqpoint{2.678275in}{1.908024in}}{\pgfqpoint{2.688874in}{1.903634in}}{\pgfqpoint{2.699924in}{1.903634in}}%
\pgfpathclose%
\pgfusepath{stroke,fill}%
\end{pgfscope}%
\begin{pgfscope}%
\pgfpathrectangle{\pgfqpoint{0.787074in}{0.548769in}}{\pgfqpoint{4.974523in}{3.102590in}}%
\pgfusepath{clip}%
\pgfsetbuttcap%
\pgfsetroundjoin%
\definecolor{currentfill}{rgb}{1.000000,0.498039,0.054902}%
\pgfsetfillcolor{currentfill}%
\pgfsetlinewidth{1.003750pt}%
\definecolor{currentstroke}{rgb}{1.000000,0.498039,0.054902}%
\pgfsetstrokecolor{currentstroke}%
\pgfsetdash{}{0pt}%
\pgfpathmoveto{\pgfqpoint{2.868976in}{2.451696in}}%
\pgfpathcurveto{\pgfqpoint{2.880026in}{2.451696in}}{\pgfqpoint{2.890625in}{2.456086in}}{\pgfqpoint{2.898439in}{2.463900in}}%
\pgfpathcurveto{\pgfqpoint{2.906252in}{2.471713in}}{\pgfqpoint{2.910642in}{2.482312in}}{\pgfqpoint{2.910642in}{2.493362in}}%
\pgfpathcurveto{\pgfqpoint{2.910642in}{2.504412in}}{\pgfqpoint{2.906252in}{2.515012in}}{\pgfqpoint{2.898439in}{2.522825in}}%
\pgfpathcurveto{\pgfqpoint{2.890625in}{2.530639in}}{\pgfqpoint{2.880026in}{2.535029in}}{\pgfqpoint{2.868976in}{2.535029in}}%
\pgfpathcurveto{\pgfqpoint{2.857926in}{2.535029in}}{\pgfqpoint{2.847327in}{2.530639in}}{\pgfqpoint{2.839513in}{2.522825in}}%
\pgfpathcurveto{\pgfqpoint{2.831699in}{2.515012in}}{\pgfqpoint{2.827309in}{2.504412in}}{\pgfqpoint{2.827309in}{2.493362in}}%
\pgfpathcurveto{\pgfqpoint{2.827309in}{2.482312in}}{\pgfqpoint{2.831699in}{2.471713in}}{\pgfqpoint{2.839513in}{2.463900in}}%
\pgfpathcurveto{\pgfqpoint{2.847327in}{2.456086in}}{\pgfqpoint{2.857926in}{2.451696in}}{\pgfqpoint{2.868976in}{2.451696in}}%
\pgfpathclose%
\pgfusepath{stroke,fill}%
\end{pgfscope}%
\begin{pgfscope}%
\pgfpathrectangle{\pgfqpoint{0.787074in}{0.548769in}}{\pgfqpoint{4.974523in}{3.102590in}}%
\pgfusepath{clip}%
\pgfsetbuttcap%
\pgfsetroundjoin%
\definecolor{currentfill}{rgb}{0.839216,0.152941,0.156863}%
\pgfsetfillcolor{currentfill}%
\pgfsetlinewidth{1.003750pt}%
\definecolor{currentstroke}{rgb}{0.839216,0.152941,0.156863}%
\pgfsetstrokecolor{currentstroke}%
\pgfsetdash{}{0pt}%
\pgfpathmoveto{\pgfqpoint{5.067974in}{3.176910in}}%
\pgfpathcurveto{\pgfqpoint{5.079024in}{3.176910in}}{\pgfqpoint{5.089623in}{3.181301in}}{\pgfqpoint{5.097437in}{3.189114in}}%
\pgfpathcurveto{\pgfqpoint{5.105251in}{3.196928in}}{\pgfqpoint{5.109641in}{3.207527in}}{\pgfqpoint{5.109641in}{3.218577in}}%
\pgfpathcurveto{\pgfqpoint{5.109641in}{3.229627in}}{\pgfqpoint{5.105251in}{3.240226in}}{\pgfqpoint{5.097437in}{3.248040in}}%
\pgfpathcurveto{\pgfqpoint{5.089623in}{3.255853in}}{\pgfqpoint{5.079024in}{3.260244in}}{\pgfqpoint{5.067974in}{3.260244in}}%
\pgfpathcurveto{\pgfqpoint{5.056924in}{3.260244in}}{\pgfqpoint{5.046325in}{3.255853in}}{\pgfqpoint{5.038511in}{3.248040in}}%
\pgfpathcurveto{\pgfqpoint{5.030698in}{3.240226in}}{\pgfqpoint{5.026308in}{3.229627in}}{\pgfqpoint{5.026308in}{3.218577in}}%
\pgfpathcurveto{\pgfqpoint{5.026308in}{3.207527in}}{\pgfqpoint{5.030698in}{3.196928in}}{\pgfqpoint{5.038511in}{3.189114in}}%
\pgfpathcurveto{\pgfqpoint{5.046325in}{3.181301in}}{\pgfqpoint{5.056924in}{3.176910in}}{\pgfqpoint{5.067974in}{3.176910in}}%
\pgfpathclose%
\pgfusepath{stroke,fill}%
\end{pgfscope}%
\begin{pgfscope}%
\pgfpathrectangle{\pgfqpoint{0.787074in}{0.548769in}}{\pgfqpoint{4.974523in}{3.102590in}}%
\pgfusepath{clip}%
\pgfsetbuttcap%
\pgfsetroundjoin%
\definecolor{currentfill}{rgb}{1.000000,0.498039,0.054902}%
\pgfsetfillcolor{currentfill}%
\pgfsetlinewidth{1.003750pt}%
\definecolor{currentstroke}{rgb}{1.000000,0.498039,0.054902}%
\pgfsetstrokecolor{currentstroke}%
\pgfsetdash{}{0pt}%
\pgfpathmoveto{\pgfqpoint{4.870639in}{2.972026in}}%
\pgfpathcurveto{\pgfqpoint{4.881689in}{2.972026in}}{\pgfqpoint{4.892288in}{2.976417in}}{\pgfqpoint{4.900101in}{2.984230in}}%
\pgfpathcurveto{\pgfqpoint{4.907915in}{2.992044in}}{\pgfqpoint{4.912305in}{3.002643in}}{\pgfqpoint{4.912305in}{3.013693in}}%
\pgfpathcurveto{\pgfqpoint{4.912305in}{3.024743in}}{\pgfqpoint{4.907915in}{3.035342in}}{\pgfqpoint{4.900101in}{3.043156in}}%
\pgfpathcurveto{\pgfqpoint{4.892288in}{3.050969in}}{\pgfqpoint{4.881689in}{3.055360in}}{\pgfqpoint{4.870639in}{3.055360in}}%
\pgfpathcurveto{\pgfqpoint{4.859588in}{3.055360in}}{\pgfqpoint{4.848989in}{3.050969in}}{\pgfqpoint{4.841176in}{3.043156in}}%
\pgfpathcurveto{\pgfqpoint{4.833362in}{3.035342in}}{\pgfqpoint{4.828972in}{3.024743in}}{\pgfqpoint{4.828972in}{3.013693in}}%
\pgfpathcurveto{\pgfqpoint{4.828972in}{3.002643in}}{\pgfqpoint{4.833362in}{2.992044in}}{\pgfqpoint{4.841176in}{2.984230in}}%
\pgfpathcurveto{\pgfqpoint{4.848989in}{2.976417in}}{\pgfqpoint{4.859588in}{2.972026in}}{\pgfqpoint{4.870639in}{2.972026in}}%
\pgfpathclose%
\pgfusepath{stroke,fill}%
\end{pgfscope}%
\begin{pgfscope}%
\pgfpathrectangle{\pgfqpoint{0.787074in}{0.548769in}}{\pgfqpoint{4.974523in}{3.102590in}}%
\pgfusepath{clip}%
\pgfsetbuttcap%
\pgfsetroundjoin%
\definecolor{currentfill}{rgb}{1.000000,0.498039,0.054902}%
\pgfsetfillcolor{currentfill}%
\pgfsetlinewidth{1.003750pt}%
\definecolor{currentstroke}{rgb}{1.000000,0.498039,0.054902}%
\pgfsetstrokecolor{currentstroke}%
\pgfsetdash{}{0pt}%
\pgfpathmoveto{\pgfqpoint{4.456365in}{3.386124in}}%
\pgfpathcurveto{\pgfqpoint{4.467415in}{3.386124in}}{\pgfqpoint{4.478014in}{3.390514in}}{\pgfqpoint{4.485828in}{3.398328in}}%
\pgfpathcurveto{\pgfqpoint{4.493642in}{3.406142in}}{\pgfqpoint{4.498032in}{3.416741in}}{\pgfqpoint{4.498032in}{3.427791in}}%
\pgfpathcurveto{\pgfqpoint{4.498032in}{3.438841in}}{\pgfqpoint{4.493642in}{3.449440in}}{\pgfqpoint{4.485828in}{3.457254in}}%
\pgfpathcurveto{\pgfqpoint{4.478014in}{3.465067in}}{\pgfqpoint{4.467415in}{3.469458in}}{\pgfqpoint{4.456365in}{3.469458in}}%
\pgfpathcurveto{\pgfqpoint{4.445315in}{3.469458in}}{\pgfqpoint{4.434716in}{3.465067in}}{\pgfqpoint{4.426902in}{3.457254in}}%
\pgfpathcurveto{\pgfqpoint{4.419089in}{3.449440in}}{\pgfqpoint{4.414699in}{3.438841in}}{\pgfqpoint{4.414699in}{3.427791in}}%
\pgfpathcurveto{\pgfqpoint{4.414699in}{3.416741in}}{\pgfqpoint{4.419089in}{3.406142in}}{\pgfqpoint{4.426902in}{3.398328in}}%
\pgfpathcurveto{\pgfqpoint{4.434716in}{3.390514in}}{\pgfqpoint{4.445315in}{3.386124in}}{\pgfqpoint{4.456365in}{3.386124in}}%
\pgfpathclose%
\pgfusepath{stroke,fill}%
\end{pgfscope}%
\begin{pgfscope}%
\pgfpathrectangle{\pgfqpoint{0.787074in}{0.548769in}}{\pgfqpoint{4.974523in}{3.102590in}}%
\pgfusepath{clip}%
\pgfsetbuttcap%
\pgfsetroundjoin%
\definecolor{currentfill}{rgb}{1.000000,0.498039,0.054902}%
\pgfsetfillcolor{currentfill}%
\pgfsetlinewidth{1.003750pt}%
\definecolor{currentstroke}{rgb}{1.000000,0.498039,0.054902}%
\pgfsetstrokecolor{currentstroke}%
\pgfsetdash{}{0pt}%
\pgfpathmoveto{\pgfqpoint{3.130011in}{2.328792in}}%
\pgfpathcurveto{\pgfqpoint{3.141061in}{2.328792in}}{\pgfqpoint{3.151660in}{2.333182in}}{\pgfqpoint{3.159473in}{2.340995in}}%
\pgfpathcurveto{\pgfqpoint{3.167287in}{2.348809in}}{\pgfqpoint{3.171677in}{2.359408in}}{\pgfqpoint{3.171677in}{2.370458in}}%
\pgfpathcurveto{\pgfqpoint{3.171677in}{2.381508in}}{\pgfqpoint{3.167287in}{2.392107in}}{\pgfqpoint{3.159473in}{2.399921in}}%
\pgfpathcurveto{\pgfqpoint{3.151660in}{2.407735in}}{\pgfqpoint{3.141061in}{2.412125in}}{\pgfqpoint{3.130011in}{2.412125in}}%
\pgfpathcurveto{\pgfqpoint{3.118961in}{2.412125in}}{\pgfqpoint{3.108362in}{2.407735in}}{\pgfqpoint{3.100548in}{2.399921in}}%
\pgfpathcurveto{\pgfqpoint{3.092734in}{2.392107in}}{\pgfqpoint{3.088344in}{2.381508in}}{\pgfqpoint{3.088344in}{2.370458in}}%
\pgfpathcurveto{\pgfqpoint{3.088344in}{2.359408in}}{\pgfqpoint{3.092734in}{2.348809in}}{\pgfqpoint{3.100548in}{2.340995in}}%
\pgfpathcurveto{\pgfqpoint{3.108362in}{2.333182in}}{\pgfqpoint{3.118961in}{2.328792in}}{\pgfqpoint{3.130011in}{2.328792in}}%
\pgfpathclose%
\pgfusepath{stroke,fill}%
\end{pgfscope}%
\begin{pgfscope}%
\pgfpathrectangle{\pgfqpoint{0.787074in}{0.548769in}}{\pgfqpoint{4.974523in}{3.102590in}}%
\pgfusepath{clip}%
\pgfsetbuttcap%
\pgfsetroundjoin%
\definecolor{currentfill}{rgb}{1.000000,0.498039,0.054902}%
\pgfsetfillcolor{currentfill}%
\pgfsetlinewidth{1.003750pt}%
\definecolor{currentstroke}{rgb}{1.000000,0.498039,0.054902}%
\pgfsetstrokecolor{currentstroke}%
\pgfsetdash{}{0pt}%
\pgfpathmoveto{\pgfqpoint{4.243647in}{3.073267in}}%
\pgfpathcurveto{\pgfqpoint{4.254697in}{3.073267in}}{\pgfqpoint{4.265296in}{3.077657in}}{\pgfqpoint{4.273110in}{3.085471in}}%
\pgfpathcurveto{\pgfqpoint{4.280923in}{3.093285in}}{\pgfqpoint{4.285314in}{3.103884in}}{\pgfqpoint{4.285314in}{3.114934in}}%
\pgfpathcurveto{\pgfqpoint{4.285314in}{3.125984in}}{\pgfqpoint{4.280923in}{3.136583in}}{\pgfqpoint{4.273110in}{3.144396in}}%
\pgfpathcurveto{\pgfqpoint{4.265296in}{3.152210in}}{\pgfqpoint{4.254697in}{3.156600in}}{\pgfqpoint{4.243647in}{3.156600in}}%
\pgfpathcurveto{\pgfqpoint{4.232597in}{3.156600in}}{\pgfqpoint{4.221998in}{3.152210in}}{\pgfqpoint{4.214184in}{3.144396in}}%
\pgfpathcurveto{\pgfqpoint{4.206371in}{3.136583in}}{\pgfqpoint{4.201980in}{3.125984in}}{\pgfqpoint{4.201980in}{3.114934in}}%
\pgfpathcurveto{\pgfqpoint{4.201980in}{3.103884in}}{\pgfqpoint{4.206371in}{3.093285in}}{\pgfqpoint{4.214184in}{3.085471in}}%
\pgfpathcurveto{\pgfqpoint{4.221998in}{3.077657in}}{\pgfqpoint{4.232597in}{3.073267in}}{\pgfqpoint{4.243647in}{3.073267in}}%
\pgfpathclose%
\pgfusepath{stroke,fill}%
\end{pgfscope}%
\begin{pgfscope}%
\pgfpathrectangle{\pgfqpoint{0.787074in}{0.548769in}}{\pgfqpoint{4.974523in}{3.102590in}}%
\pgfusepath{clip}%
\pgfsetbuttcap%
\pgfsetroundjoin%
\definecolor{currentfill}{rgb}{0.121569,0.466667,0.705882}%
\pgfsetfillcolor{currentfill}%
\pgfsetlinewidth{1.003750pt}%
\definecolor{currentstroke}{rgb}{0.121569,0.466667,0.705882}%
\pgfsetstrokecolor{currentstroke}%
\pgfsetdash{}{0pt}%
\pgfpathmoveto{\pgfqpoint{1.005413in}{0.648131in}}%
\pgfpathcurveto{\pgfqpoint{1.016463in}{0.648131in}}{\pgfqpoint{1.027062in}{0.652521in}}{\pgfqpoint{1.034875in}{0.660335in}}%
\pgfpathcurveto{\pgfqpoint{1.042689in}{0.668148in}}{\pgfqpoint{1.047079in}{0.678748in}}{\pgfqpoint{1.047079in}{0.689798in}}%
\pgfpathcurveto{\pgfqpoint{1.047079in}{0.700848in}}{\pgfqpoint{1.042689in}{0.711447in}}{\pgfqpoint{1.034875in}{0.719260in}}%
\pgfpathcurveto{\pgfqpoint{1.027062in}{0.727074in}}{\pgfqpoint{1.016463in}{0.731464in}}{\pgfqpoint{1.005413in}{0.731464in}}%
\pgfpathcurveto{\pgfqpoint{0.994362in}{0.731464in}}{\pgfqpoint{0.983763in}{0.727074in}}{\pgfqpoint{0.975950in}{0.719260in}}%
\pgfpathcurveto{\pgfqpoint{0.968136in}{0.711447in}}{\pgfqpoint{0.963746in}{0.700848in}}{\pgfqpoint{0.963746in}{0.689798in}}%
\pgfpathcurveto{\pgfqpoint{0.963746in}{0.678748in}}{\pgfqpoint{0.968136in}{0.668148in}}{\pgfqpoint{0.975950in}{0.660335in}}%
\pgfpathcurveto{\pgfqpoint{0.983763in}{0.652521in}}{\pgfqpoint{0.994362in}{0.648131in}}{\pgfqpoint{1.005413in}{0.648131in}}%
\pgfpathclose%
\pgfusepath{stroke,fill}%
\end{pgfscope}%
\begin{pgfscope}%
\pgfpathrectangle{\pgfqpoint{0.787074in}{0.548769in}}{\pgfqpoint{4.974523in}{3.102590in}}%
\pgfusepath{clip}%
\pgfsetbuttcap%
\pgfsetroundjoin%
\definecolor{currentfill}{rgb}{1.000000,0.498039,0.054902}%
\pgfsetfillcolor{currentfill}%
\pgfsetlinewidth{1.003750pt}%
\definecolor{currentstroke}{rgb}{1.000000,0.498039,0.054902}%
\pgfsetstrokecolor{currentstroke}%
\pgfsetdash{}{0pt}%
\pgfpathmoveto{\pgfqpoint{2.253474in}{1.987192in}}%
\pgfpathcurveto{\pgfqpoint{2.264524in}{1.987192in}}{\pgfqpoint{2.275123in}{1.991582in}}{\pgfqpoint{2.282937in}{1.999395in}}%
\pgfpathcurveto{\pgfqpoint{2.290750in}{2.007209in}}{\pgfqpoint{2.295141in}{2.017808in}}{\pgfqpoint{2.295141in}{2.028858in}}%
\pgfpathcurveto{\pgfqpoint{2.295141in}{2.039908in}}{\pgfqpoint{2.290750in}{2.050507in}}{\pgfqpoint{2.282937in}{2.058321in}}%
\pgfpathcurveto{\pgfqpoint{2.275123in}{2.066135in}}{\pgfqpoint{2.264524in}{2.070525in}}{\pgfqpoint{2.253474in}{2.070525in}}%
\pgfpathcurveto{\pgfqpoint{2.242424in}{2.070525in}}{\pgfqpoint{2.231825in}{2.066135in}}{\pgfqpoint{2.224011in}{2.058321in}}%
\pgfpathcurveto{\pgfqpoint{2.216198in}{2.050507in}}{\pgfqpoint{2.211807in}{2.039908in}}{\pgfqpoint{2.211807in}{2.028858in}}%
\pgfpathcurveto{\pgfqpoint{2.211807in}{2.017808in}}{\pgfqpoint{2.216198in}{2.007209in}}{\pgfqpoint{2.224011in}{1.999395in}}%
\pgfpathcurveto{\pgfqpoint{2.231825in}{1.991582in}}{\pgfqpoint{2.242424in}{1.987192in}}{\pgfqpoint{2.253474in}{1.987192in}}%
\pgfpathclose%
\pgfusepath{stroke,fill}%
\end{pgfscope}%
\begin{pgfscope}%
\pgfpathrectangle{\pgfqpoint{0.787074in}{0.548769in}}{\pgfqpoint{4.974523in}{3.102590in}}%
\pgfusepath{clip}%
\pgfsetbuttcap%
\pgfsetroundjoin%
\definecolor{currentfill}{rgb}{1.000000,0.498039,0.054902}%
\pgfsetfillcolor{currentfill}%
\pgfsetlinewidth{1.003750pt}%
\definecolor{currentstroke}{rgb}{1.000000,0.498039,0.054902}%
\pgfsetstrokecolor{currentstroke}%
\pgfsetdash{}{0pt}%
\pgfpathmoveto{\pgfqpoint{4.444182in}{2.820143in}}%
\pgfpathcurveto{\pgfqpoint{4.455232in}{2.820143in}}{\pgfqpoint{4.465831in}{2.824533in}}{\pgfqpoint{4.473644in}{2.832347in}}%
\pgfpathcurveto{\pgfqpoint{4.481458in}{2.840160in}}{\pgfqpoint{4.485848in}{2.850759in}}{\pgfqpoint{4.485848in}{2.861810in}}%
\pgfpathcurveto{\pgfqpoint{4.485848in}{2.872860in}}{\pgfqpoint{4.481458in}{2.883459in}}{\pgfqpoint{4.473644in}{2.891272in}}%
\pgfpathcurveto{\pgfqpoint{4.465831in}{2.899086in}}{\pgfqpoint{4.455232in}{2.903476in}}{\pgfqpoint{4.444182in}{2.903476in}}%
\pgfpathcurveto{\pgfqpoint{4.433131in}{2.903476in}}{\pgfqpoint{4.422532in}{2.899086in}}{\pgfqpoint{4.414719in}{2.891272in}}%
\pgfpathcurveto{\pgfqpoint{4.406905in}{2.883459in}}{\pgfqpoint{4.402515in}{2.872860in}}{\pgfqpoint{4.402515in}{2.861810in}}%
\pgfpathcurveto{\pgfqpoint{4.402515in}{2.850759in}}{\pgfqpoint{4.406905in}{2.840160in}}{\pgfqpoint{4.414719in}{2.832347in}}%
\pgfpathcurveto{\pgfqpoint{4.422532in}{2.824533in}}{\pgfqpoint{4.433131in}{2.820143in}}{\pgfqpoint{4.444182in}{2.820143in}}%
\pgfpathclose%
\pgfusepath{stroke,fill}%
\end{pgfscope}%
\begin{pgfscope}%
\pgfpathrectangle{\pgfqpoint{0.787074in}{0.548769in}}{\pgfqpoint{4.974523in}{3.102590in}}%
\pgfusepath{clip}%
\pgfsetbuttcap%
\pgfsetroundjoin%
\definecolor{currentfill}{rgb}{1.000000,0.498039,0.054902}%
\pgfsetfillcolor{currentfill}%
\pgfsetlinewidth{1.003750pt}%
\definecolor{currentstroke}{rgb}{1.000000,0.498039,0.054902}%
\pgfsetstrokecolor{currentstroke}%
\pgfsetdash{}{0pt}%
\pgfpathmoveto{\pgfqpoint{3.901066in}{2.523468in}}%
\pgfpathcurveto{\pgfqpoint{3.912116in}{2.523468in}}{\pgfqpoint{3.922715in}{2.527858in}}{\pgfqpoint{3.930528in}{2.535672in}}%
\pgfpathcurveto{\pgfqpoint{3.938342in}{2.543485in}}{\pgfqpoint{3.942732in}{2.554084in}}{\pgfqpoint{3.942732in}{2.565134in}}%
\pgfpathcurveto{\pgfqpoint{3.942732in}{2.576185in}}{\pgfqpoint{3.938342in}{2.586784in}}{\pgfqpoint{3.930528in}{2.594597in}}%
\pgfpathcurveto{\pgfqpoint{3.922715in}{2.602411in}}{\pgfqpoint{3.912116in}{2.606801in}}{\pgfqpoint{3.901066in}{2.606801in}}%
\pgfpathcurveto{\pgfqpoint{3.890016in}{2.606801in}}{\pgfqpoint{3.879417in}{2.602411in}}{\pgfqpoint{3.871603in}{2.594597in}}%
\pgfpathcurveto{\pgfqpoint{3.863789in}{2.586784in}}{\pgfqpoint{3.859399in}{2.576185in}}{\pgfqpoint{3.859399in}{2.565134in}}%
\pgfpathcurveto{\pgfqpoint{3.859399in}{2.554084in}}{\pgfqpoint{3.863789in}{2.543485in}}{\pgfqpoint{3.871603in}{2.535672in}}%
\pgfpathcurveto{\pgfqpoint{3.879417in}{2.527858in}}{\pgfqpoint{3.890016in}{2.523468in}}{\pgfqpoint{3.901066in}{2.523468in}}%
\pgfpathclose%
\pgfusepath{stroke,fill}%
\end{pgfscope}%
\begin{pgfscope}%
\pgfpathrectangle{\pgfqpoint{0.787074in}{0.548769in}}{\pgfqpoint{4.974523in}{3.102590in}}%
\pgfusepath{clip}%
\pgfsetbuttcap%
\pgfsetroundjoin%
\definecolor{currentfill}{rgb}{1.000000,0.498039,0.054902}%
\pgfsetfillcolor{currentfill}%
\pgfsetlinewidth{1.003750pt}%
\definecolor{currentstroke}{rgb}{1.000000,0.498039,0.054902}%
\pgfsetstrokecolor{currentstroke}%
\pgfsetdash{}{0pt}%
\pgfpathmoveto{\pgfqpoint{2.098338in}{1.569832in}}%
\pgfpathcurveto{\pgfqpoint{2.109388in}{1.569832in}}{\pgfqpoint{2.119987in}{1.574223in}}{\pgfqpoint{2.127801in}{1.582036in}}%
\pgfpathcurveto{\pgfqpoint{2.135614in}{1.589850in}}{\pgfqpoint{2.140005in}{1.600449in}}{\pgfqpoint{2.140005in}{1.611499in}}%
\pgfpathcurveto{\pgfqpoint{2.140005in}{1.622549in}}{\pgfqpoint{2.135614in}{1.633148in}}{\pgfqpoint{2.127801in}{1.640962in}}%
\pgfpathcurveto{\pgfqpoint{2.119987in}{1.648775in}}{\pgfqpoint{2.109388in}{1.653166in}}{\pgfqpoint{2.098338in}{1.653166in}}%
\pgfpathcurveto{\pgfqpoint{2.087288in}{1.653166in}}{\pgfqpoint{2.076689in}{1.648775in}}{\pgfqpoint{2.068875in}{1.640962in}}%
\pgfpathcurveto{\pgfqpoint{2.061062in}{1.633148in}}{\pgfqpoint{2.056671in}{1.622549in}}{\pgfqpoint{2.056671in}{1.611499in}}%
\pgfpathcurveto{\pgfqpoint{2.056671in}{1.600449in}}{\pgfqpoint{2.061062in}{1.589850in}}{\pgfqpoint{2.068875in}{1.582036in}}%
\pgfpathcurveto{\pgfqpoint{2.076689in}{1.574223in}}{\pgfqpoint{2.087288in}{1.569832in}}{\pgfqpoint{2.098338in}{1.569832in}}%
\pgfpathclose%
\pgfusepath{stroke,fill}%
\end{pgfscope}%
\begin{pgfscope}%
\pgfpathrectangle{\pgfqpoint{0.787074in}{0.548769in}}{\pgfqpoint{4.974523in}{3.102590in}}%
\pgfusepath{clip}%
\pgfsetbuttcap%
\pgfsetroundjoin%
\definecolor{currentfill}{rgb}{1.000000,0.498039,0.054902}%
\pgfsetfillcolor{currentfill}%
\pgfsetlinewidth{1.003750pt}%
\definecolor{currentstroke}{rgb}{1.000000,0.498039,0.054902}%
\pgfsetstrokecolor{currentstroke}%
\pgfsetdash{}{0pt}%
\pgfpathmoveto{\pgfqpoint{2.732987in}{2.228984in}}%
\pgfpathcurveto{\pgfqpoint{2.744037in}{2.228984in}}{\pgfqpoint{2.754636in}{2.233374in}}{\pgfqpoint{2.762450in}{2.241188in}}%
\pgfpathcurveto{\pgfqpoint{2.770264in}{2.249002in}}{\pgfqpoint{2.774654in}{2.259601in}}{\pgfqpoint{2.774654in}{2.270651in}}%
\pgfpathcurveto{\pgfqpoint{2.774654in}{2.281701in}}{\pgfqpoint{2.770264in}{2.292300in}}{\pgfqpoint{2.762450in}{2.300114in}}%
\pgfpathcurveto{\pgfqpoint{2.754636in}{2.307927in}}{\pgfqpoint{2.744037in}{2.312318in}}{\pgfqpoint{2.732987in}{2.312318in}}%
\pgfpathcurveto{\pgfqpoint{2.721937in}{2.312318in}}{\pgfqpoint{2.711338in}{2.307927in}}{\pgfqpoint{2.703524in}{2.300114in}}%
\pgfpathcurveto{\pgfqpoint{2.695711in}{2.292300in}}{\pgfqpoint{2.691321in}{2.281701in}}{\pgfqpoint{2.691321in}{2.270651in}}%
\pgfpathcurveto{\pgfqpoint{2.691321in}{2.259601in}}{\pgfqpoint{2.695711in}{2.249002in}}{\pgfqpoint{2.703524in}{2.241188in}}%
\pgfpathcurveto{\pgfqpoint{2.711338in}{2.233374in}}{\pgfqpoint{2.721937in}{2.228984in}}{\pgfqpoint{2.732987in}{2.228984in}}%
\pgfpathclose%
\pgfusepath{stroke,fill}%
\end{pgfscope}%
\begin{pgfscope}%
\pgfpathrectangle{\pgfqpoint{0.787074in}{0.548769in}}{\pgfqpoint{4.974523in}{3.102590in}}%
\pgfusepath{clip}%
\pgfsetbuttcap%
\pgfsetroundjoin%
\definecolor{currentfill}{rgb}{0.121569,0.466667,0.705882}%
\pgfsetfillcolor{currentfill}%
\pgfsetlinewidth{1.003750pt}%
\definecolor{currentstroke}{rgb}{0.121569,0.466667,0.705882}%
\pgfsetstrokecolor{currentstroke}%
\pgfsetdash{}{0pt}%
\pgfpathmoveto{\pgfqpoint{4.027633in}{2.843750in}}%
\pgfpathcurveto{\pgfqpoint{4.038683in}{2.843750in}}{\pgfqpoint{4.049282in}{2.848141in}}{\pgfqpoint{4.057096in}{2.855954in}}%
\pgfpathcurveto{\pgfqpoint{4.064910in}{2.863768in}}{\pgfqpoint{4.069300in}{2.874367in}}{\pgfqpoint{4.069300in}{2.885417in}}%
\pgfpathcurveto{\pgfqpoint{4.069300in}{2.896467in}}{\pgfqpoint{4.064910in}{2.907066in}}{\pgfqpoint{4.057096in}{2.914880in}}%
\pgfpathcurveto{\pgfqpoint{4.049282in}{2.922693in}}{\pgfqpoint{4.038683in}{2.927084in}}{\pgfqpoint{4.027633in}{2.927084in}}%
\pgfpathcurveto{\pgfqpoint{4.016583in}{2.927084in}}{\pgfqpoint{4.005984in}{2.922693in}}{\pgfqpoint{3.998170in}{2.914880in}}%
\pgfpathcurveto{\pgfqpoint{3.990357in}{2.907066in}}{\pgfqpoint{3.985967in}{2.896467in}}{\pgfqpoint{3.985967in}{2.885417in}}%
\pgfpathcurveto{\pgfqpoint{3.985967in}{2.874367in}}{\pgfqpoint{3.990357in}{2.863768in}}{\pgfqpoint{3.998170in}{2.855954in}}%
\pgfpathcurveto{\pgfqpoint{4.005984in}{2.848141in}}{\pgfqpoint{4.016583in}{2.843750in}}{\pgfqpoint{4.027633in}{2.843750in}}%
\pgfpathclose%
\pgfusepath{stroke,fill}%
\end{pgfscope}%
\begin{pgfscope}%
\pgfpathrectangle{\pgfqpoint{0.787074in}{0.548769in}}{\pgfqpoint{4.974523in}{3.102590in}}%
\pgfusepath{clip}%
\pgfsetbuttcap%
\pgfsetroundjoin%
\definecolor{currentfill}{rgb}{1.000000,0.498039,0.054902}%
\pgfsetfillcolor{currentfill}%
\pgfsetlinewidth{1.003750pt}%
\definecolor{currentstroke}{rgb}{1.000000,0.498039,0.054902}%
\pgfsetstrokecolor{currentstroke}%
\pgfsetdash{}{0pt}%
\pgfpathmoveto{\pgfqpoint{4.399145in}{3.155929in}}%
\pgfpathcurveto{\pgfqpoint{4.410195in}{3.155929in}}{\pgfqpoint{4.420794in}{3.160320in}}{\pgfqpoint{4.428608in}{3.168133in}}%
\pgfpathcurveto{\pgfqpoint{4.436422in}{3.175947in}}{\pgfqpoint{4.440812in}{3.186546in}}{\pgfqpoint{4.440812in}{3.197596in}}%
\pgfpathcurveto{\pgfqpoint{4.440812in}{3.208646in}}{\pgfqpoint{4.436422in}{3.219245in}}{\pgfqpoint{4.428608in}{3.227059in}}%
\pgfpathcurveto{\pgfqpoint{4.420794in}{3.234872in}}{\pgfqpoint{4.410195in}{3.239263in}}{\pgfqpoint{4.399145in}{3.239263in}}%
\pgfpathcurveto{\pgfqpoint{4.388095in}{3.239263in}}{\pgfqpoint{4.377496in}{3.234872in}}{\pgfqpoint{4.369682in}{3.227059in}}%
\pgfpathcurveto{\pgfqpoint{4.361869in}{3.219245in}}{\pgfqpoint{4.357478in}{3.208646in}}{\pgfqpoint{4.357478in}{3.197596in}}%
\pgfpathcurveto{\pgfqpoint{4.357478in}{3.186546in}}{\pgfqpoint{4.361869in}{3.175947in}}{\pgfqpoint{4.369682in}{3.168133in}}%
\pgfpathcurveto{\pgfqpoint{4.377496in}{3.160320in}}{\pgfqpoint{4.388095in}{3.155929in}}{\pgfqpoint{4.399145in}{3.155929in}}%
\pgfpathclose%
\pgfusepath{stroke,fill}%
\end{pgfscope}%
\begin{pgfscope}%
\pgfsetbuttcap%
\pgfsetroundjoin%
\definecolor{currentfill}{rgb}{0.000000,0.000000,0.000000}%
\pgfsetfillcolor{currentfill}%
\pgfsetlinewidth{0.803000pt}%
\definecolor{currentstroke}{rgb}{0.000000,0.000000,0.000000}%
\pgfsetstrokecolor{currentstroke}%
\pgfsetdash{}{0pt}%
\pgfsys@defobject{currentmarker}{\pgfqpoint{0.000000in}{-0.048611in}}{\pgfqpoint{0.000000in}{0.000000in}}{%
\pgfpathmoveto{\pgfqpoint{0.000000in}{0.000000in}}%
\pgfpathlineto{\pgfqpoint{0.000000in}{-0.048611in}}%
\pgfusepath{stroke,fill}%
}%
\begin{pgfscope}%
\pgfsys@transformshift{1.005408in}{0.548769in}%
\pgfsys@useobject{currentmarker}{}%
\end{pgfscope}%
\end{pgfscope}%
\begin{pgfscope}%
\definecolor{textcolor}{rgb}{0.000000,0.000000,0.000000}%
\pgfsetstrokecolor{textcolor}%
\pgfsetfillcolor{textcolor}%
\pgftext[x=1.005408in,y=0.451547in,,top]{\color{textcolor}\sffamily\fontsize{10.000000}{12.000000}\selectfont \(\displaystyle {0.0}\)}%
\end{pgfscope}%
\begin{pgfscope}%
\pgfsetbuttcap%
\pgfsetroundjoin%
\definecolor{currentfill}{rgb}{0.000000,0.000000,0.000000}%
\pgfsetfillcolor{currentfill}%
\pgfsetlinewidth{0.803000pt}%
\definecolor{currentstroke}{rgb}{0.000000,0.000000,0.000000}%
\pgfsetstrokecolor{currentstroke}%
\pgfsetdash{}{0pt}%
\pgfsys@defobject{currentmarker}{\pgfqpoint{0.000000in}{-0.048611in}}{\pgfqpoint{0.000000in}{0.000000in}}{%
\pgfpathmoveto{\pgfqpoint{0.000000in}{0.000000in}}%
\pgfpathlineto{\pgfqpoint{0.000000in}{-0.048611in}}%
\pgfusepath{stroke,fill}%
}%
\begin{pgfscope}%
\pgfsys@transformshift{1.481027in}{0.548769in}%
\pgfsys@useobject{currentmarker}{}%
\end{pgfscope}%
\end{pgfscope}%
\begin{pgfscope}%
\definecolor{textcolor}{rgb}{0.000000,0.000000,0.000000}%
\pgfsetstrokecolor{textcolor}%
\pgfsetfillcolor{textcolor}%
\pgftext[x=1.481027in,y=0.451547in,,top]{\color{textcolor}\sffamily\fontsize{10.000000}{12.000000}\selectfont \(\displaystyle {0.1}\)}%
\end{pgfscope}%
\begin{pgfscope}%
\pgfsetbuttcap%
\pgfsetroundjoin%
\definecolor{currentfill}{rgb}{0.000000,0.000000,0.000000}%
\pgfsetfillcolor{currentfill}%
\pgfsetlinewidth{0.803000pt}%
\definecolor{currentstroke}{rgb}{0.000000,0.000000,0.000000}%
\pgfsetstrokecolor{currentstroke}%
\pgfsetdash{}{0pt}%
\pgfsys@defobject{currentmarker}{\pgfqpoint{0.000000in}{-0.048611in}}{\pgfqpoint{0.000000in}{0.000000in}}{%
\pgfpathmoveto{\pgfqpoint{0.000000in}{0.000000in}}%
\pgfpathlineto{\pgfqpoint{0.000000in}{-0.048611in}}%
\pgfusepath{stroke,fill}%
}%
\begin{pgfscope}%
\pgfsys@transformshift{1.956645in}{0.548769in}%
\pgfsys@useobject{currentmarker}{}%
\end{pgfscope}%
\end{pgfscope}%
\begin{pgfscope}%
\definecolor{textcolor}{rgb}{0.000000,0.000000,0.000000}%
\pgfsetstrokecolor{textcolor}%
\pgfsetfillcolor{textcolor}%
\pgftext[x=1.956645in,y=0.451547in,,top]{\color{textcolor}\sffamily\fontsize{10.000000}{12.000000}\selectfont \(\displaystyle {0.2}\)}%
\end{pgfscope}%
\begin{pgfscope}%
\pgfsetbuttcap%
\pgfsetroundjoin%
\definecolor{currentfill}{rgb}{0.000000,0.000000,0.000000}%
\pgfsetfillcolor{currentfill}%
\pgfsetlinewidth{0.803000pt}%
\definecolor{currentstroke}{rgb}{0.000000,0.000000,0.000000}%
\pgfsetstrokecolor{currentstroke}%
\pgfsetdash{}{0pt}%
\pgfsys@defobject{currentmarker}{\pgfqpoint{0.000000in}{-0.048611in}}{\pgfqpoint{0.000000in}{0.000000in}}{%
\pgfpathmoveto{\pgfqpoint{0.000000in}{0.000000in}}%
\pgfpathlineto{\pgfqpoint{0.000000in}{-0.048611in}}%
\pgfusepath{stroke,fill}%
}%
\begin{pgfscope}%
\pgfsys@transformshift{2.432264in}{0.548769in}%
\pgfsys@useobject{currentmarker}{}%
\end{pgfscope}%
\end{pgfscope}%
\begin{pgfscope}%
\definecolor{textcolor}{rgb}{0.000000,0.000000,0.000000}%
\pgfsetstrokecolor{textcolor}%
\pgfsetfillcolor{textcolor}%
\pgftext[x=2.432264in,y=0.451547in,,top]{\color{textcolor}\sffamily\fontsize{10.000000}{12.000000}\selectfont \(\displaystyle {0.3}\)}%
\end{pgfscope}%
\begin{pgfscope}%
\pgfsetbuttcap%
\pgfsetroundjoin%
\definecolor{currentfill}{rgb}{0.000000,0.000000,0.000000}%
\pgfsetfillcolor{currentfill}%
\pgfsetlinewidth{0.803000pt}%
\definecolor{currentstroke}{rgb}{0.000000,0.000000,0.000000}%
\pgfsetstrokecolor{currentstroke}%
\pgfsetdash{}{0pt}%
\pgfsys@defobject{currentmarker}{\pgfqpoint{0.000000in}{-0.048611in}}{\pgfqpoint{0.000000in}{0.000000in}}{%
\pgfpathmoveto{\pgfqpoint{0.000000in}{0.000000in}}%
\pgfpathlineto{\pgfqpoint{0.000000in}{-0.048611in}}%
\pgfusepath{stroke,fill}%
}%
\begin{pgfscope}%
\pgfsys@transformshift{2.907883in}{0.548769in}%
\pgfsys@useobject{currentmarker}{}%
\end{pgfscope}%
\end{pgfscope}%
\begin{pgfscope}%
\definecolor{textcolor}{rgb}{0.000000,0.000000,0.000000}%
\pgfsetstrokecolor{textcolor}%
\pgfsetfillcolor{textcolor}%
\pgftext[x=2.907883in,y=0.451547in,,top]{\color{textcolor}\sffamily\fontsize{10.000000}{12.000000}\selectfont \(\displaystyle {0.4}\)}%
\end{pgfscope}%
\begin{pgfscope}%
\pgfsetbuttcap%
\pgfsetroundjoin%
\definecolor{currentfill}{rgb}{0.000000,0.000000,0.000000}%
\pgfsetfillcolor{currentfill}%
\pgfsetlinewidth{0.803000pt}%
\definecolor{currentstroke}{rgb}{0.000000,0.000000,0.000000}%
\pgfsetstrokecolor{currentstroke}%
\pgfsetdash{}{0pt}%
\pgfsys@defobject{currentmarker}{\pgfqpoint{0.000000in}{-0.048611in}}{\pgfqpoint{0.000000in}{0.000000in}}{%
\pgfpathmoveto{\pgfqpoint{0.000000in}{0.000000in}}%
\pgfpathlineto{\pgfqpoint{0.000000in}{-0.048611in}}%
\pgfusepath{stroke,fill}%
}%
\begin{pgfscope}%
\pgfsys@transformshift{3.383502in}{0.548769in}%
\pgfsys@useobject{currentmarker}{}%
\end{pgfscope}%
\end{pgfscope}%
\begin{pgfscope}%
\definecolor{textcolor}{rgb}{0.000000,0.000000,0.000000}%
\pgfsetstrokecolor{textcolor}%
\pgfsetfillcolor{textcolor}%
\pgftext[x=3.383502in,y=0.451547in,,top]{\color{textcolor}\sffamily\fontsize{10.000000}{12.000000}\selectfont \(\displaystyle {0.5}\)}%
\end{pgfscope}%
\begin{pgfscope}%
\pgfsetbuttcap%
\pgfsetroundjoin%
\definecolor{currentfill}{rgb}{0.000000,0.000000,0.000000}%
\pgfsetfillcolor{currentfill}%
\pgfsetlinewidth{0.803000pt}%
\definecolor{currentstroke}{rgb}{0.000000,0.000000,0.000000}%
\pgfsetstrokecolor{currentstroke}%
\pgfsetdash{}{0pt}%
\pgfsys@defobject{currentmarker}{\pgfqpoint{0.000000in}{-0.048611in}}{\pgfqpoint{0.000000in}{0.000000in}}{%
\pgfpathmoveto{\pgfqpoint{0.000000in}{0.000000in}}%
\pgfpathlineto{\pgfqpoint{0.000000in}{-0.048611in}}%
\pgfusepath{stroke,fill}%
}%
\begin{pgfscope}%
\pgfsys@transformshift{3.859121in}{0.548769in}%
\pgfsys@useobject{currentmarker}{}%
\end{pgfscope}%
\end{pgfscope}%
\begin{pgfscope}%
\definecolor{textcolor}{rgb}{0.000000,0.000000,0.000000}%
\pgfsetstrokecolor{textcolor}%
\pgfsetfillcolor{textcolor}%
\pgftext[x=3.859121in,y=0.451547in,,top]{\color{textcolor}\sffamily\fontsize{10.000000}{12.000000}\selectfont \(\displaystyle {0.6}\)}%
\end{pgfscope}%
\begin{pgfscope}%
\pgfsetbuttcap%
\pgfsetroundjoin%
\definecolor{currentfill}{rgb}{0.000000,0.000000,0.000000}%
\pgfsetfillcolor{currentfill}%
\pgfsetlinewidth{0.803000pt}%
\definecolor{currentstroke}{rgb}{0.000000,0.000000,0.000000}%
\pgfsetstrokecolor{currentstroke}%
\pgfsetdash{}{0pt}%
\pgfsys@defobject{currentmarker}{\pgfqpoint{0.000000in}{-0.048611in}}{\pgfqpoint{0.000000in}{0.000000in}}{%
\pgfpathmoveto{\pgfqpoint{0.000000in}{0.000000in}}%
\pgfpathlineto{\pgfqpoint{0.000000in}{-0.048611in}}%
\pgfusepath{stroke,fill}%
}%
\begin{pgfscope}%
\pgfsys@transformshift{4.334740in}{0.548769in}%
\pgfsys@useobject{currentmarker}{}%
\end{pgfscope}%
\end{pgfscope}%
\begin{pgfscope}%
\definecolor{textcolor}{rgb}{0.000000,0.000000,0.000000}%
\pgfsetstrokecolor{textcolor}%
\pgfsetfillcolor{textcolor}%
\pgftext[x=4.334740in,y=0.451547in,,top]{\color{textcolor}\sffamily\fontsize{10.000000}{12.000000}\selectfont \(\displaystyle {0.7}\)}%
\end{pgfscope}%
\begin{pgfscope}%
\pgfsetbuttcap%
\pgfsetroundjoin%
\definecolor{currentfill}{rgb}{0.000000,0.000000,0.000000}%
\pgfsetfillcolor{currentfill}%
\pgfsetlinewidth{0.803000pt}%
\definecolor{currentstroke}{rgb}{0.000000,0.000000,0.000000}%
\pgfsetstrokecolor{currentstroke}%
\pgfsetdash{}{0pt}%
\pgfsys@defobject{currentmarker}{\pgfqpoint{0.000000in}{-0.048611in}}{\pgfqpoint{0.000000in}{0.000000in}}{%
\pgfpathmoveto{\pgfqpoint{0.000000in}{0.000000in}}%
\pgfpathlineto{\pgfqpoint{0.000000in}{-0.048611in}}%
\pgfusepath{stroke,fill}%
}%
\begin{pgfscope}%
\pgfsys@transformshift{4.810359in}{0.548769in}%
\pgfsys@useobject{currentmarker}{}%
\end{pgfscope}%
\end{pgfscope}%
\begin{pgfscope}%
\definecolor{textcolor}{rgb}{0.000000,0.000000,0.000000}%
\pgfsetstrokecolor{textcolor}%
\pgfsetfillcolor{textcolor}%
\pgftext[x=4.810359in,y=0.451547in,,top]{\color{textcolor}\sffamily\fontsize{10.000000}{12.000000}\selectfont \(\displaystyle {0.8}\)}%
\end{pgfscope}%
\begin{pgfscope}%
\pgfsetbuttcap%
\pgfsetroundjoin%
\definecolor{currentfill}{rgb}{0.000000,0.000000,0.000000}%
\pgfsetfillcolor{currentfill}%
\pgfsetlinewidth{0.803000pt}%
\definecolor{currentstroke}{rgb}{0.000000,0.000000,0.000000}%
\pgfsetstrokecolor{currentstroke}%
\pgfsetdash{}{0pt}%
\pgfsys@defobject{currentmarker}{\pgfqpoint{0.000000in}{-0.048611in}}{\pgfqpoint{0.000000in}{0.000000in}}{%
\pgfpathmoveto{\pgfqpoint{0.000000in}{0.000000in}}%
\pgfpathlineto{\pgfqpoint{0.000000in}{-0.048611in}}%
\pgfusepath{stroke,fill}%
}%
\begin{pgfscope}%
\pgfsys@transformshift{5.285978in}{0.548769in}%
\pgfsys@useobject{currentmarker}{}%
\end{pgfscope}%
\end{pgfscope}%
\begin{pgfscope}%
\definecolor{textcolor}{rgb}{0.000000,0.000000,0.000000}%
\pgfsetstrokecolor{textcolor}%
\pgfsetfillcolor{textcolor}%
\pgftext[x=5.285978in,y=0.451547in,,top]{\color{textcolor}\sffamily\fontsize{10.000000}{12.000000}\selectfont \(\displaystyle {0.9}\)}%
\end{pgfscope}%
\begin{pgfscope}%
\pgfsetbuttcap%
\pgfsetroundjoin%
\definecolor{currentfill}{rgb}{0.000000,0.000000,0.000000}%
\pgfsetfillcolor{currentfill}%
\pgfsetlinewidth{0.803000pt}%
\definecolor{currentstroke}{rgb}{0.000000,0.000000,0.000000}%
\pgfsetstrokecolor{currentstroke}%
\pgfsetdash{}{0pt}%
\pgfsys@defobject{currentmarker}{\pgfqpoint{0.000000in}{-0.048611in}}{\pgfqpoint{0.000000in}{0.000000in}}{%
\pgfpathmoveto{\pgfqpoint{0.000000in}{0.000000in}}%
\pgfpathlineto{\pgfqpoint{0.000000in}{-0.048611in}}%
\pgfusepath{stroke,fill}%
}%
\begin{pgfscope}%
\pgfsys@transformshift{5.761597in}{0.548769in}%
\pgfsys@useobject{currentmarker}{}%
\end{pgfscope}%
\end{pgfscope}%
\begin{pgfscope}%
\definecolor{textcolor}{rgb}{0.000000,0.000000,0.000000}%
\pgfsetstrokecolor{textcolor}%
\pgfsetfillcolor{textcolor}%
\pgftext[x=5.761597in,y=0.451547in,,top]{\color{textcolor}\sffamily\fontsize{10.000000}{12.000000}\selectfont \(\displaystyle {1.0}\)}%
\end{pgfscope}%
\begin{pgfscope}%
\definecolor{textcolor}{rgb}{0.000000,0.000000,0.000000}%
\pgfsetstrokecolor{textcolor}%
\pgfsetfillcolor{textcolor}%
\pgftext[x=3.274335in,y=0.272658in,,top]{\color{textcolor}\sffamily\fontsize{10.000000}{12.000000}\selectfont Edge Count}%
\end{pgfscope}%
\begin{pgfscope}%
\definecolor{textcolor}{rgb}{0.000000,0.000000,0.000000}%
\pgfsetstrokecolor{textcolor}%
\pgfsetfillcolor{textcolor}%
\pgftext[x=5.761597in,y=0.286547in,right,top]{\color{textcolor}\sffamily\fontsize{10.000000}{12.000000}\selectfont \(\displaystyle \times{10^{8}}{}\)}%
\end{pgfscope}%
\begin{pgfscope}%
\pgfsetbuttcap%
\pgfsetroundjoin%
\definecolor{currentfill}{rgb}{0.000000,0.000000,0.000000}%
\pgfsetfillcolor{currentfill}%
\pgfsetlinewidth{0.803000pt}%
\definecolor{currentstroke}{rgb}{0.000000,0.000000,0.000000}%
\pgfsetstrokecolor{currentstroke}%
\pgfsetdash{}{0pt}%
\pgfsys@defobject{currentmarker}{\pgfqpoint{-0.048611in}{0.000000in}}{\pgfqpoint{0.000000in}{0.000000in}}{%
\pgfpathmoveto{\pgfqpoint{0.000000in}{0.000000in}}%
\pgfpathlineto{\pgfqpoint{-0.048611in}{0.000000in}}%
\pgfusepath{stroke,fill}%
}%
\begin{pgfscope}%
\pgfsys@transformshift{0.787074in}{0.689795in}%
\pgfsys@useobject{currentmarker}{}%
\end{pgfscope}%
\end{pgfscope}%
\begin{pgfscope}%
\definecolor{textcolor}{rgb}{0.000000,0.000000,0.000000}%
\pgfsetstrokecolor{textcolor}%
\pgfsetfillcolor{textcolor}%
\pgftext[x=0.620407in, y=0.641601in, left, base]{\color{textcolor}\sffamily\fontsize{10.000000}{12.000000}\selectfont \(\displaystyle {0}\)}%
\end{pgfscope}%
\begin{pgfscope}%
\pgfsetbuttcap%
\pgfsetroundjoin%
\definecolor{currentfill}{rgb}{0.000000,0.000000,0.000000}%
\pgfsetfillcolor{currentfill}%
\pgfsetlinewidth{0.803000pt}%
\definecolor{currentstroke}{rgb}{0.000000,0.000000,0.000000}%
\pgfsetstrokecolor{currentstroke}%
\pgfsetdash{}{0pt}%
\pgfsys@defobject{currentmarker}{\pgfqpoint{-0.048611in}{0.000000in}}{\pgfqpoint{0.000000in}{0.000000in}}{%
\pgfpathmoveto{\pgfqpoint{0.000000in}{0.000000in}}%
\pgfpathlineto{\pgfqpoint{-0.048611in}{0.000000in}}%
\pgfusepath{stroke,fill}%
}%
\begin{pgfscope}%
\pgfsys@transformshift{0.787074in}{1.059666in}%
\pgfsys@useobject{currentmarker}{}%
\end{pgfscope}%
\end{pgfscope}%
\begin{pgfscope}%
\definecolor{textcolor}{rgb}{0.000000,0.000000,0.000000}%
\pgfsetstrokecolor{textcolor}%
\pgfsetfillcolor{textcolor}%
\pgftext[x=0.412073in, y=1.011471in, left, base]{\color{textcolor}\sffamily\fontsize{10.000000}{12.000000}\selectfont \(\displaystyle {2500}\)}%
\end{pgfscope}%
\begin{pgfscope}%
\pgfsetbuttcap%
\pgfsetroundjoin%
\definecolor{currentfill}{rgb}{0.000000,0.000000,0.000000}%
\pgfsetfillcolor{currentfill}%
\pgfsetlinewidth{0.803000pt}%
\definecolor{currentstroke}{rgb}{0.000000,0.000000,0.000000}%
\pgfsetstrokecolor{currentstroke}%
\pgfsetdash{}{0pt}%
\pgfsys@defobject{currentmarker}{\pgfqpoint{-0.048611in}{0.000000in}}{\pgfqpoint{0.000000in}{0.000000in}}{%
\pgfpathmoveto{\pgfqpoint{0.000000in}{0.000000in}}%
\pgfpathlineto{\pgfqpoint{-0.048611in}{0.000000in}}%
\pgfusepath{stroke,fill}%
}%
\begin{pgfscope}%
\pgfsys@transformshift{0.787074in}{1.429536in}%
\pgfsys@useobject{currentmarker}{}%
\end{pgfscope}%
\end{pgfscope}%
\begin{pgfscope}%
\definecolor{textcolor}{rgb}{0.000000,0.000000,0.000000}%
\pgfsetstrokecolor{textcolor}%
\pgfsetfillcolor{textcolor}%
\pgftext[x=0.412073in, y=1.381342in, left, base]{\color{textcolor}\sffamily\fontsize{10.000000}{12.000000}\selectfont \(\displaystyle {5000}\)}%
\end{pgfscope}%
\begin{pgfscope}%
\pgfsetbuttcap%
\pgfsetroundjoin%
\definecolor{currentfill}{rgb}{0.000000,0.000000,0.000000}%
\pgfsetfillcolor{currentfill}%
\pgfsetlinewidth{0.803000pt}%
\definecolor{currentstroke}{rgb}{0.000000,0.000000,0.000000}%
\pgfsetstrokecolor{currentstroke}%
\pgfsetdash{}{0pt}%
\pgfsys@defobject{currentmarker}{\pgfqpoint{-0.048611in}{0.000000in}}{\pgfqpoint{0.000000in}{0.000000in}}{%
\pgfpathmoveto{\pgfqpoint{0.000000in}{0.000000in}}%
\pgfpathlineto{\pgfqpoint{-0.048611in}{0.000000in}}%
\pgfusepath{stroke,fill}%
}%
\begin{pgfscope}%
\pgfsys@transformshift{0.787074in}{1.799407in}%
\pgfsys@useobject{currentmarker}{}%
\end{pgfscope}%
\end{pgfscope}%
\begin{pgfscope}%
\definecolor{textcolor}{rgb}{0.000000,0.000000,0.000000}%
\pgfsetstrokecolor{textcolor}%
\pgfsetfillcolor{textcolor}%
\pgftext[x=0.412073in, y=1.751213in, left, base]{\color{textcolor}\sffamily\fontsize{10.000000}{12.000000}\selectfont \(\displaystyle {7500}\)}%
\end{pgfscope}%
\begin{pgfscope}%
\pgfsetbuttcap%
\pgfsetroundjoin%
\definecolor{currentfill}{rgb}{0.000000,0.000000,0.000000}%
\pgfsetfillcolor{currentfill}%
\pgfsetlinewidth{0.803000pt}%
\definecolor{currentstroke}{rgb}{0.000000,0.000000,0.000000}%
\pgfsetstrokecolor{currentstroke}%
\pgfsetdash{}{0pt}%
\pgfsys@defobject{currentmarker}{\pgfqpoint{-0.048611in}{0.000000in}}{\pgfqpoint{0.000000in}{0.000000in}}{%
\pgfpathmoveto{\pgfqpoint{0.000000in}{0.000000in}}%
\pgfpathlineto{\pgfqpoint{-0.048611in}{0.000000in}}%
\pgfusepath{stroke,fill}%
}%
\begin{pgfscope}%
\pgfsys@transformshift{0.787074in}{2.169278in}%
\pgfsys@useobject{currentmarker}{}%
\end{pgfscope}%
\end{pgfscope}%
\begin{pgfscope}%
\definecolor{textcolor}{rgb}{0.000000,0.000000,0.000000}%
\pgfsetstrokecolor{textcolor}%
\pgfsetfillcolor{textcolor}%
\pgftext[x=0.342628in, y=2.121083in, left, base]{\color{textcolor}\sffamily\fontsize{10.000000}{12.000000}\selectfont \(\displaystyle {10000}\)}%
\end{pgfscope}%
\begin{pgfscope}%
\pgfsetbuttcap%
\pgfsetroundjoin%
\definecolor{currentfill}{rgb}{0.000000,0.000000,0.000000}%
\pgfsetfillcolor{currentfill}%
\pgfsetlinewidth{0.803000pt}%
\definecolor{currentstroke}{rgb}{0.000000,0.000000,0.000000}%
\pgfsetstrokecolor{currentstroke}%
\pgfsetdash{}{0pt}%
\pgfsys@defobject{currentmarker}{\pgfqpoint{-0.048611in}{0.000000in}}{\pgfqpoint{0.000000in}{0.000000in}}{%
\pgfpathmoveto{\pgfqpoint{0.000000in}{0.000000in}}%
\pgfpathlineto{\pgfqpoint{-0.048611in}{0.000000in}}%
\pgfusepath{stroke,fill}%
}%
\begin{pgfscope}%
\pgfsys@transformshift{0.787074in}{2.539148in}%
\pgfsys@useobject{currentmarker}{}%
\end{pgfscope}%
\end{pgfscope}%
\begin{pgfscope}%
\definecolor{textcolor}{rgb}{0.000000,0.000000,0.000000}%
\pgfsetstrokecolor{textcolor}%
\pgfsetfillcolor{textcolor}%
\pgftext[x=0.342628in, y=2.490954in, left, base]{\color{textcolor}\sffamily\fontsize{10.000000}{12.000000}\selectfont \(\displaystyle {12500}\)}%
\end{pgfscope}%
\begin{pgfscope}%
\pgfsetbuttcap%
\pgfsetroundjoin%
\definecolor{currentfill}{rgb}{0.000000,0.000000,0.000000}%
\pgfsetfillcolor{currentfill}%
\pgfsetlinewidth{0.803000pt}%
\definecolor{currentstroke}{rgb}{0.000000,0.000000,0.000000}%
\pgfsetstrokecolor{currentstroke}%
\pgfsetdash{}{0pt}%
\pgfsys@defobject{currentmarker}{\pgfqpoint{-0.048611in}{0.000000in}}{\pgfqpoint{0.000000in}{0.000000in}}{%
\pgfpathmoveto{\pgfqpoint{0.000000in}{0.000000in}}%
\pgfpathlineto{\pgfqpoint{-0.048611in}{0.000000in}}%
\pgfusepath{stroke,fill}%
}%
\begin{pgfscope}%
\pgfsys@transformshift{0.787074in}{2.909019in}%
\pgfsys@useobject{currentmarker}{}%
\end{pgfscope}%
\end{pgfscope}%
\begin{pgfscope}%
\definecolor{textcolor}{rgb}{0.000000,0.000000,0.000000}%
\pgfsetstrokecolor{textcolor}%
\pgfsetfillcolor{textcolor}%
\pgftext[x=0.342628in, y=2.860824in, left, base]{\color{textcolor}\sffamily\fontsize{10.000000}{12.000000}\selectfont \(\displaystyle {15000}\)}%
\end{pgfscope}%
\begin{pgfscope}%
\pgfsetbuttcap%
\pgfsetroundjoin%
\definecolor{currentfill}{rgb}{0.000000,0.000000,0.000000}%
\pgfsetfillcolor{currentfill}%
\pgfsetlinewidth{0.803000pt}%
\definecolor{currentstroke}{rgb}{0.000000,0.000000,0.000000}%
\pgfsetstrokecolor{currentstroke}%
\pgfsetdash{}{0pt}%
\pgfsys@defobject{currentmarker}{\pgfqpoint{-0.048611in}{0.000000in}}{\pgfqpoint{0.000000in}{0.000000in}}{%
\pgfpathmoveto{\pgfqpoint{0.000000in}{0.000000in}}%
\pgfpathlineto{\pgfqpoint{-0.048611in}{0.000000in}}%
\pgfusepath{stroke,fill}%
}%
\begin{pgfscope}%
\pgfsys@transformshift{0.787074in}{3.278889in}%
\pgfsys@useobject{currentmarker}{}%
\end{pgfscope}%
\end{pgfscope}%
\begin{pgfscope}%
\definecolor{textcolor}{rgb}{0.000000,0.000000,0.000000}%
\pgfsetstrokecolor{textcolor}%
\pgfsetfillcolor{textcolor}%
\pgftext[x=0.342628in, y=3.230695in, left, base]{\color{textcolor}\sffamily\fontsize{10.000000}{12.000000}\selectfont \(\displaystyle {17500}\)}%
\end{pgfscope}%
\begin{pgfscope}%
\pgfsetbuttcap%
\pgfsetroundjoin%
\definecolor{currentfill}{rgb}{0.000000,0.000000,0.000000}%
\pgfsetfillcolor{currentfill}%
\pgfsetlinewidth{0.803000pt}%
\definecolor{currentstroke}{rgb}{0.000000,0.000000,0.000000}%
\pgfsetstrokecolor{currentstroke}%
\pgfsetdash{}{0pt}%
\pgfsys@defobject{currentmarker}{\pgfqpoint{-0.048611in}{0.000000in}}{\pgfqpoint{0.000000in}{0.000000in}}{%
\pgfpathmoveto{\pgfqpoint{0.000000in}{0.000000in}}%
\pgfpathlineto{\pgfqpoint{-0.048611in}{0.000000in}}%
\pgfusepath{stroke,fill}%
}%
\begin{pgfscope}%
\pgfsys@transformshift{0.787074in}{3.648760in}%
\pgfsys@useobject{currentmarker}{}%
\end{pgfscope}%
\end{pgfscope}%
\begin{pgfscope}%
\definecolor{textcolor}{rgb}{0.000000,0.000000,0.000000}%
\pgfsetstrokecolor{textcolor}%
\pgfsetfillcolor{textcolor}%
\pgftext[x=0.342628in, y=3.600565in, left, base]{\color{textcolor}\sffamily\fontsize{10.000000}{12.000000}\selectfont \(\displaystyle {20000}\)}%
\end{pgfscope}%
\begin{pgfscope}%
\definecolor{textcolor}{rgb}{0.000000,0.000000,0.000000}%
\pgfsetstrokecolor{textcolor}%
\pgfsetfillcolor{textcolor}%
\pgftext[x=0.287073in,y=2.100064in,,bottom,rotate=90.000000]{\color{textcolor}\sffamily\fontsize{10.000000}{12.000000}\selectfont Maximum Memory Usage (MB)}%
\end{pgfscope}%
\begin{pgfscope}%
\pgfsetrectcap%
\pgfsetmiterjoin%
\pgfsetlinewidth{0.803000pt}%
\definecolor{currentstroke}{rgb}{0.000000,0.000000,0.000000}%
\pgfsetstrokecolor{currentstroke}%
\pgfsetdash{}{0pt}%
\pgfpathmoveto{\pgfqpoint{0.787074in}{0.548769in}}%
\pgfpathlineto{\pgfqpoint{0.787074in}{3.651359in}}%
\pgfusepath{stroke}%
\end{pgfscope}%
\begin{pgfscope}%
\pgfsetrectcap%
\pgfsetmiterjoin%
\pgfsetlinewidth{0.803000pt}%
\definecolor{currentstroke}{rgb}{0.000000,0.000000,0.000000}%
\pgfsetstrokecolor{currentstroke}%
\pgfsetdash{}{0pt}%
\pgfpathmoveto{\pgfqpoint{5.761597in}{0.548769in}}%
\pgfpathlineto{\pgfqpoint{5.761597in}{3.651359in}}%
\pgfusepath{stroke}%
\end{pgfscope}%
\begin{pgfscope}%
\pgfsetrectcap%
\pgfsetmiterjoin%
\pgfsetlinewidth{0.803000pt}%
\definecolor{currentstroke}{rgb}{0.000000,0.000000,0.000000}%
\pgfsetstrokecolor{currentstroke}%
\pgfsetdash{}{0pt}%
\pgfpathmoveto{\pgfqpoint{0.787074in}{0.548769in}}%
\pgfpathlineto{\pgfqpoint{5.761597in}{0.548769in}}%
\pgfusepath{stroke}%
\end{pgfscope}%
\begin{pgfscope}%
\pgfsetrectcap%
\pgfsetmiterjoin%
\pgfsetlinewidth{0.803000pt}%
\definecolor{currentstroke}{rgb}{0.000000,0.000000,0.000000}%
\pgfsetstrokecolor{currentstroke}%
\pgfsetdash{}{0pt}%
\pgfpathmoveto{\pgfqpoint{0.787074in}{3.651359in}}%
\pgfpathlineto{\pgfqpoint{5.761597in}{3.651359in}}%
\pgfusepath{stroke}%
\end{pgfscope}%
\begin{pgfscope}%
\definecolor{textcolor}{rgb}{0.000000,0.000000,0.000000}%
\pgfsetstrokecolor{textcolor}%
\pgfsetfillcolor{textcolor}%
\pgftext[x=3.274335in,y=3.734692in,,base]{\color{textcolor}\sffamily\fontsize{12.000000}{14.400000}\selectfont Forwards}%
\end{pgfscope}%
\begin{pgfscope}%
\pgfsetbuttcap%
\pgfsetmiterjoin%
\definecolor{currentfill}{rgb}{1.000000,1.000000,1.000000}%
\pgfsetfillcolor{currentfill}%
\pgfsetfillopacity{0.800000}%
\pgfsetlinewidth{1.003750pt}%
\definecolor{currentstroke}{rgb}{0.800000,0.800000,0.800000}%
\pgfsetstrokecolor{currentstroke}%
\pgfsetstrokeopacity{0.800000}%
\pgfsetdash{}{0pt}%
\pgfpathmoveto{\pgfqpoint{0.884296in}{2.957886in}}%
\pgfpathlineto{\pgfqpoint{2.336657in}{2.957886in}}%
\pgfpathquadraticcurveto{\pgfqpoint{2.364435in}{2.957886in}}{\pgfqpoint{2.364435in}{2.985664in}}%
\pgfpathlineto{\pgfqpoint{2.364435in}{3.554136in}}%
\pgfpathquadraticcurveto{\pgfqpoint{2.364435in}{3.581914in}}{\pgfqpoint{2.336657in}{3.581914in}}%
\pgfpathlineto{\pgfqpoint{0.884296in}{3.581914in}}%
\pgfpathquadraticcurveto{\pgfqpoint{0.856518in}{3.581914in}}{\pgfqpoint{0.856518in}{3.554136in}}%
\pgfpathlineto{\pgfqpoint{0.856518in}{2.985664in}}%
\pgfpathquadraticcurveto{\pgfqpoint{0.856518in}{2.957886in}}{\pgfqpoint{0.884296in}{2.957886in}}%
\pgfpathclose%
\pgfusepath{stroke,fill}%
\end{pgfscope}%
\begin{pgfscope}%
\pgfsetbuttcap%
\pgfsetroundjoin%
\definecolor{currentfill}{rgb}{0.121569,0.466667,0.705882}%
\pgfsetfillcolor{currentfill}%
\pgfsetlinewidth{1.003750pt}%
\definecolor{currentstroke}{rgb}{0.121569,0.466667,0.705882}%
\pgfsetstrokecolor{currentstroke}%
\pgfsetdash{}{0pt}%
\pgfsys@defobject{currentmarker}{\pgfqpoint{-0.034722in}{-0.034722in}}{\pgfqpoint{0.034722in}{0.034722in}}{%
\pgfpathmoveto{\pgfqpoint{0.000000in}{-0.034722in}}%
\pgfpathcurveto{\pgfqpoint{0.009208in}{-0.034722in}}{\pgfqpoint{0.018041in}{-0.031064in}}{\pgfqpoint{0.024552in}{-0.024552in}}%
\pgfpathcurveto{\pgfqpoint{0.031064in}{-0.018041in}}{\pgfqpoint{0.034722in}{-0.009208in}}{\pgfqpoint{0.034722in}{0.000000in}}%
\pgfpathcurveto{\pgfqpoint{0.034722in}{0.009208in}}{\pgfqpoint{0.031064in}{0.018041in}}{\pgfqpoint{0.024552in}{0.024552in}}%
\pgfpathcurveto{\pgfqpoint{0.018041in}{0.031064in}}{\pgfqpoint{0.009208in}{0.034722in}}{\pgfqpoint{0.000000in}{0.034722in}}%
\pgfpathcurveto{\pgfqpoint{-0.009208in}{0.034722in}}{\pgfqpoint{-0.018041in}{0.031064in}}{\pgfqpoint{-0.024552in}{0.024552in}}%
\pgfpathcurveto{\pgfqpoint{-0.031064in}{0.018041in}}{\pgfqpoint{-0.034722in}{0.009208in}}{\pgfqpoint{-0.034722in}{0.000000in}}%
\pgfpathcurveto{\pgfqpoint{-0.034722in}{-0.009208in}}{\pgfqpoint{-0.031064in}{-0.018041in}}{\pgfqpoint{-0.024552in}{-0.024552in}}%
\pgfpathcurveto{\pgfqpoint{-0.018041in}{-0.031064in}}{\pgfqpoint{-0.009208in}{-0.034722in}}{\pgfqpoint{0.000000in}{-0.034722in}}%
\pgfpathclose%
\pgfusepath{stroke,fill}%
}%
\begin{pgfscope}%
\pgfsys@transformshift{1.050963in}{3.477748in}%
\pgfsys@useobject{currentmarker}{}%
\end{pgfscope}%
\end{pgfscope}%
\begin{pgfscope}%
\definecolor{textcolor}{rgb}{0.000000,0.000000,0.000000}%
\pgfsetstrokecolor{textcolor}%
\pgfsetfillcolor{textcolor}%
\pgftext[x=1.300963in,y=3.429136in,left,base]{\color{textcolor}\sffamily\fontsize{10.000000}{12.000000}\selectfont No Timeout}%
\end{pgfscope}%
\begin{pgfscope}%
\pgfsetbuttcap%
\pgfsetroundjoin%
\definecolor{currentfill}{rgb}{1.000000,0.498039,0.054902}%
\pgfsetfillcolor{currentfill}%
\pgfsetlinewidth{1.003750pt}%
\definecolor{currentstroke}{rgb}{1.000000,0.498039,0.054902}%
\pgfsetstrokecolor{currentstroke}%
\pgfsetdash{}{0pt}%
\pgfsys@defobject{currentmarker}{\pgfqpoint{-0.034722in}{-0.034722in}}{\pgfqpoint{0.034722in}{0.034722in}}{%
\pgfpathmoveto{\pgfqpoint{0.000000in}{-0.034722in}}%
\pgfpathcurveto{\pgfqpoint{0.009208in}{-0.034722in}}{\pgfqpoint{0.018041in}{-0.031064in}}{\pgfqpoint{0.024552in}{-0.024552in}}%
\pgfpathcurveto{\pgfqpoint{0.031064in}{-0.018041in}}{\pgfqpoint{0.034722in}{-0.009208in}}{\pgfqpoint{0.034722in}{0.000000in}}%
\pgfpathcurveto{\pgfqpoint{0.034722in}{0.009208in}}{\pgfqpoint{0.031064in}{0.018041in}}{\pgfqpoint{0.024552in}{0.024552in}}%
\pgfpathcurveto{\pgfqpoint{0.018041in}{0.031064in}}{\pgfqpoint{0.009208in}{0.034722in}}{\pgfqpoint{0.000000in}{0.034722in}}%
\pgfpathcurveto{\pgfqpoint{-0.009208in}{0.034722in}}{\pgfqpoint{-0.018041in}{0.031064in}}{\pgfqpoint{-0.024552in}{0.024552in}}%
\pgfpathcurveto{\pgfqpoint{-0.031064in}{0.018041in}}{\pgfqpoint{-0.034722in}{0.009208in}}{\pgfqpoint{-0.034722in}{0.000000in}}%
\pgfpathcurveto{\pgfqpoint{-0.034722in}{-0.009208in}}{\pgfqpoint{-0.031064in}{-0.018041in}}{\pgfqpoint{-0.024552in}{-0.024552in}}%
\pgfpathcurveto{\pgfqpoint{-0.018041in}{-0.031064in}}{\pgfqpoint{-0.009208in}{-0.034722in}}{\pgfqpoint{0.000000in}{-0.034722in}}%
\pgfpathclose%
\pgfusepath{stroke,fill}%
}%
\begin{pgfscope}%
\pgfsys@transformshift{1.050963in}{3.284136in}%
\pgfsys@useobject{currentmarker}{}%
\end{pgfscope}%
\end{pgfscope}%
\begin{pgfscope}%
\definecolor{textcolor}{rgb}{0.000000,0.000000,0.000000}%
\pgfsetstrokecolor{textcolor}%
\pgfsetfillcolor{textcolor}%
\pgftext[x=1.300963in,y=3.235525in,left,base]{\color{textcolor}\sffamily\fontsize{10.000000}{12.000000}\selectfont Time Timeout}%
\end{pgfscope}%
\begin{pgfscope}%
\pgfsetbuttcap%
\pgfsetroundjoin%
\definecolor{currentfill}{rgb}{0.839216,0.152941,0.156863}%
\pgfsetfillcolor{currentfill}%
\pgfsetlinewidth{1.003750pt}%
\definecolor{currentstroke}{rgb}{0.839216,0.152941,0.156863}%
\pgfsetstrokecolor{currentstroke}%
\pgfsetdash{}{0pt}%
\pgfsys@defobject{currentmarker}{\pgfqpoint{-0.034722in}{-0.034722in}}{\pgfqpoint{0.034722in}{0.034722in}}{%
\pgfpathmoveto{\pgfqpoint{0.000000in}{-0.034722in}}%
\pgfpathcurveto{\pgfqpoint{0.009208in}{-0.034722in}}{\pgfqpoint{0.018041in}{-0.031064in}}{\pgfqpoint{0.024552in}{-0.024552in}}%
\pgfpathcurveto{\pgfqpoint{0.031064in}{-0.018041in}}{\pgfqpoint{0.034722in}{-0.009208in}}{\pgfqpoint{0.034722in}{0.000000in}}%
\pgfpathcurveto{\pgfqpoint{0.034722in}{0.009208in}}{\pgfqpoint{0.031064in}{0.018041in}}{\pgfqpoint{0.024552in}{0.024552in}}%
\pgfpathcurveto{\pgfqpoint{0.018041in}{0.031064in}}{\pgfqpoint{0.009208in}{0.034722in}}{\pgfqpoint{0.000000in}{0.034722in}}%
\pgfpathcurveto{\pgfqpoint{-0.009208in}{0.034722in}}{\pgfqpoint{-0.018041in}{0.031064in}}{\pgfqpoint{-0.024552in}{0.024552in}}%
\pgfpathcurveto{\pgfqpoint{-0.031064in}{0.018041in}}{\pgfqpoint{-0.034722in}{0.009208in}}{\pgfqpoint{-0.034722in}{0.000000in}}%
\pgfpathcurveto{\pgfqpoint{-0.034722in}{-0.009208in}}{\pgfqpoint{-0.031064in}{-0.018041in}}{\pgfqpoint{-0.024552in}{-0.024552in}}%
\pgfpathcurveto{\pgfqpoint{-0.018041in}{-0.031064in}}{\pgfqpoint{-0.009208in}{-0.034722in}}{\pgfqpoint{0.000000in}{-0.034722in}}%
\pgfpathclose%
\pgfusepath{stroke,fill}%
}%
\begin{pgfscope}%
\pgfsys@transformshift{1.050963in}{3.090525in}%
\pgfsys@useobject{currentmarker}{}%
\end{pgfscope}%
\end{pgfscope}%
\begin{pgfscope}%
\definecolor{textcolor}{rgb}{0.000000,0.000000,0.000000}%
\pgfsetstrokecolor{textcolor}%
\pgfsetfillcolor{textcolor}%
\pgftext[x=1.300963in,y=3.041914in,left,base]{\color{textcolor}\sffamily\fontsize{10.000000}{12.000000}\selectfont Memory Timeout}%
\end{pgfscope}%
\end{pgfpicture}%
\makeatother%
\endgroup%

                }
            \end{subfigure}
            \qquad
            \begin{subfigure}[]{0.45\textwidth}
                \centering
                \resizebox{\columnwidth}{!}{
                    %% Creator: Matplotlib, PGF backend
%%
%% To include the figure in your LaTeX document, write
%%   \input{<filename>.pgf}
%%
%% Make sure the required packages are loaded in your preamble
%%   \usepackage{pgf}
%%
%% and, on pdftex
%%   \usepackage[utf8]{inputenc}\DeclareUnicodeCharacter{2212}{-}
%%
%% or, on luatex and xetex
%%   \usepackage{unicode-math}
%%
%% Figures using additional raster images can only be included by \input if
%% they are in the same directory as the main LaTeX file. For loading figures
%% from other directories you can use the `import` package
%%   \usepackage{import}
%%
%% and then include the figures with
%%   \import{<path to file>}{<filename>.pgf}
%%
%% Matplotlib used the following preamble
%%   \usepackage{amsmath}
%%   \usepackage{fontspec}
%%
\begingroup%
\makeatletter%
\begin{pgfpicture}%
\pgfpathrectangle{\pgfpointorigin}{\pgfqpoint{6.000000in}{4.000000in}}%
\pgfusepath{use as bounding box, clip}%
\begin{pgfscope}%
\pgfsetbuttcap%
\pgfsetmiterjoin%
\definecolor{currentfill}{rgb}{1.000000,1.000000,1.000000}%
\pgfsetfillcolor{currentfill}%
\pgfsetlinewidth{0.000000pt}%
\definecolor{currentstroke}{rgb}{1.000000,1.000000,1.000000}%
\pgfsetstrokecolor{currentstroke}%
\pgfsetdash{}{0pt}%
\pgfpathmoveto{\pgfqpoint{0.000000in}{0.000000in}}%
\pgfpathlineto{\pgfqpoint{6.000000in}{0.000000in}}%
\pgfpathlineto{\pgfqpoint{6.000000in}{4.000000in}}%
\pgfpathlineto{\pgfqpoint{0.000000in}{4.000000in}}%
\pgfpathclose%
\pgfusepath{fill}%
\end{pgfscope}%
\begin{pgfscope}%
\pgfsetbuttcap%
\pgfsetmiterjoin%
\definecolor{currentfill}{rgb}{1.000000,1.000000,1.000000}%
\pgfsetfillcolor{currentfill}%
\pgfsetlinewidth{0.000000pt}%
\definecolor{currentstroke}{rgb}{0.000000,0.000000,0.000000}%
\pgfsetstrokecolor{currentstroke}%
\pgfsetstrokeopacity{0.000000}%
\pgfsetdash{}{0pt}%
\pgfpathmoveto{\pgfqpoint{0.787074in}{0.548769in}}%
\pgfpathlineto{\pgfqpoint{5.850000in}{0.548769in}}%
\pgfpathlineto{\pgfqpoint{5.850000in}{3.651359in}}%
\pgfpathlineto{\pgfqpoint{0.787074in}{3.651359in}}%
\pgfpathclose%
\pgfusepath{fill}%
\end{pgfscope}%
\begin{pgfscope}%
\pgfpathrectangle{\pgfqpoint{0.787074in}{0.548769in}}{\pgfqpoint{5.062926in}{3.102590in}}%
\pgfusepath{clip}%
\pgfsetbuttcap%
\pgfsetroundjoin%
\definecolor{currentfill}{rgb}{0.121569,0.466667,0.705882}%
\pgfsetfillcolor{currentfill}%
\pgfsetlinewidth{1.003750pt}%
\definecolor{currentstroke}{rgb}{0.121569,0.466667,0.705882}%
\pgfsetstrokecolor{currentstroke}%
\pgfsetdash{}{0pt}%
\pgfpathmoveto{\pgfqpoint{1.072796in}{0.676616in}}%
\pgfpathcurveto{\pgfqpoint{1.083846in}{0.676616in}}{\pgfqpoint{1.094445in}{0.681006in}}{\pgfqpoint{1.102258in}{0.688820in}}%
\pgfpathcurveto{\pgfqpoint{1.110072in}{0.696633in}}{\pgfqpoint{1.114462in}{0.707232in}}{\pgfqpoint{1.114462in}{0.718283in}}%
\pgfpathcurveto{\pgfqpoint{1.114462in}{0.729333in}}{\pgfqpoint{1.110072in}{0.739932in}}{\pgfqpoint{1.102258in}{0.747745in}}%
\pgfpathcurveto{\pgfqpoint{1.094445in}{0.755559in}}{\pgfqpoint{1.083846in}{0.759949in}}{\pgfqpoint{1.072796in}{0.759949in}}%
\pgfpathcurveto{\pgfqpoint{1.061745in}{0.759949in}}{\pgfqpoint{1.051146in}{0.755559in}}{\pgfqpoint{1.043333in}{0.747745in}}%
\pgfpathcurveto{\pgfqpoint{1.035519in}{0.739932in}}{\pgfqpoint{1.031129in}{0.729333in}}{\pgfqpoint{1.031129in}{0.718283in}}%
\pgfpathcurveto{\pgfqpoint{1.031129in}{0.707232in}}{\pgfqpoint{1.035519in}{0.696633in}}{\pgfqpoint{1.043333in}{0.688820in}}%
\pgfpathcurveto{\pgfqpoint{1.051146in}{0.681006in}}{\pgfqpoint{1.061745in}{0.676616in}}{\pgfqpoint{1.072796in}{0.676616in}}%
\pgfpathclose%
\pgfusepath{stroke,fill}%
\end{pgfscope}%
\begin{pgfscope}%
\pgfpathrectangle{\pgfqpoint{0.787074in}{0.548769in}}{\pgfqpoint{5.062926in}{3.102590in}}%
\pgfusepath{clip}%
\pgfsetbuttcap%
\pgfsetroundjoin%
\definecolor{currentfill}{rgb}{0.121569,0.466667,0.705882}%
\pgfsetfillcolor{currentfill}%
\pgfsetlinewidth{1.003750pt}%
\definecolor{currentstroke}{rgb}{0.121569,0.466667,0.705882}%
\pgfsetstrokecolor{currentstroke}%
\pgfsetdash{}{0pt}%
\pgfpathmoveto{\pgfqpoint{2.566614in}{2.052247in}}%
\pgfpathcurveto{\pgfqpoint{2.577664in}{2.052247in}}{\pgfqpoint{2.588263in}{2.056638in}}{\pgfqpoint{2.596077in}{2.064451in}}%
\pgfpathcurveto{\pgfqpoint{2.603891in}{2.072265in}}{\pgfqpoint{2.608281in}{2.082864in}}{\pgfqpoint{2.608281in}{2.093914in}}%
\pgfpathcurveto{\pgfqpoint{2.608281in}{2.104964in}}{\pgfqpoint{2.603891in}{2.115563in}}{\pgfqpoint{2.596077in}{2.123377in}}%
\pgfpathcurveto{\pgfqpoint{2.588263in}{2.131190in}}{\pgfqpoint{2.577664in}{2.135581in}}{\pgfqpoint{2.566614in}{2.135581in}}%
\pgfpathcurveto{\pgfqpoint{2.555564in}{2.135581in}}{\pgfqpoint{2.544965in}{2.131190in}}{\pgfqpoint{2.537151in}{2.123377in}}%
\pgfpathcurveto{\pgfqpoint{2.529338in}{2.115563in}}{\pgfqpoint{2.524948in}{2.104964in}}{\pgfqpoint{2.524948in}{2.093914in}}%
\pgfpathcurveto{\pgfqpoint{2.524948in}{2.082864in}}{\pgfqpoint{2.529338in}{2.072265in}}{\pgfqpoint{2.537151in}{2.064451in}}%
\pgfpathcurveto{\pgfqpoint{2.544965in}{2.056638in}}{\pgfqpoint{2.555564in}{2.052247in}}{\pgfqpoint{2.566614in}{2.052247in}}%
\pgfpathclose%
\pgfusepath{stroke,fill}%
\end{pgfscope}%
\begin{pgfscope}%
\pgfpathrectangle{\pgfqpoint{0.787074in}{0.548769in}}{\pgfqpoint{5.062926in}{3.102590in}}%
\pgfusepath{clip}%
\pgfsetbuttcap%
\pgfsetroundjoin%
\definecolor{currentfill}{rgb}{1.000000,0.498039,0.054902}%
\pgfsetfillcolor{currentfill}%
\pgfsetlinewidth{1.003750pt}%
\definecolor{currentstroke}{rgb}{1.000000,0.498039,0.054902}%
\pgfsetstrokecolor{currentstroke}%
\pgfsetdash{}{0pt}%
\pgfpathmoveto{\pgfqpoint{3.827022in}{2.417739in}}%
\pgfpathcurveto{\pgfqpoint{3.838072in}{2.417739in}}{\pgfqpoint{3.848671in}{2.422129in}}{\pgfqpoint{3.856485in}{2.429943in}}%
\pgfpathcurveto{\pgfqpoint{3.864298in}{2.437757in}}{\pgfqpoint{3.868689in}{2.448356in}}{\pgfqpoint{3.868689in}{2.459406in}}%
\pgfpathcurveto{\pgfqpoint{3.868689in}{2.470456in}}{\pgfqpoint{3.864298in}{2.481055in}}{\pgfqpoint{3.856485in}{2.488869in}}%
\pgfpathcurveto{\pgfqpoint{3.848671in}{2.496682in}}{\pgfqpoint{3.838072in}{2.501072in}}{\pgfqpoint{3.827022in}{2.501072in}}%
\pgfpathcurveto{\pgfqpoint{3.815972in}{2.501072in}}{\pgfqpoint{3.805373in}{2.496682in}}{\pgfqpoint{3.797559in}{2.488869in}}%
\pgfpathcurveto{\pgfqpoint{3.789746in}{2.481055in}}{\pgfqpoint{3.785355in}{2.470456in}}{\pgfqpoint{3.785355in}{2.459406in}}%
\pgfpathcurveto{\pgfqpoint{3.785355in}{2.448356in}}{\pgfqpoint{3.789746in}{2.437757in}}{\pgfqpoint{3.797559in}{2.429943in}}%
\pgfpathcurveto{\pgfqpoint{3.805373in}{2.422129in}}{\pgfqpoint{3.815972in}{2.417739in}}{\pgfqpoint{3.827022in}{2.417739in}}%
\pgfpathclose%
\pgfusepath{stroke,fill}%
\end{pgfscope}%
\begin{pgfscope}%
\pgfpathrectangle{\pgfqpoint{0.787074in}{0.548769in}}{\pgfqpoint{5.062926in}{3.102590in}}%
\pgfusepath{clip}%
\pgfsetbuttcap%
\pgfsetroundjoin%
\definecolor{currentfill}{rgb}{1.000000,0.498039,0.054902}%
\pgfsetfillcolor{currentfill}%
\pgfsetlinewidth{1.003750pt}%
\definecolor{currentstroke}{rgb}{1.000000,0.498039,0.054902}%
\pgfsetstrokecolor{currentstroke}%
\pgfsetdash{}{0pt}%
\pgfpathmoveto{\pgfqpoint{3.335449in}{2.242926in}}%
\pgfpathcurveto{\pgfqpoint{3.346499in}{2.242926in}}{\pgfqpoint{3.357098in}{2.247317in}}{\pgfqpoint{3.364912in}{2.255130in}}%
\pgfpathcurveto{\pgfqpoint{3.372725in}{2.262944in}}{\pgfqpoint{3.377116in}{2.273543in}}{\pgfqpoint{3.377116in}{2.284593in}}%
\pgfpathcurveto{\pgfqpoint{3.377116in}{2.295643in}}{\pgfqpoint{3.372725in}{2.306242in}}{\pgfqpoint{3.364912in}{2.314056in}}%
\pgfpathcurveto{\pgfqpoint{3.357098in}{2.321869in}}{\pgfqpoint{3.346499in}{2.326260in}}{\pgfqpoint{3.335449in}{2.326260in}}%
\pgfpathcurveto{\pgfqpoint{3.324399in}{2.326260in}}{\pgfqpoint{3.313800in}{2.321869in}}{\pgfqpoint{3.305986in}{2.314056in}}%
\pgfpathcurveto{\pgfqpoint{3.298173in}{2.306242in}}{\pgfqpoint{3.293782in}{2.295643in}}{\pgfqpoint{3.293782in}{2.284593in}}%
\pgfpathcurveto{\pgfqpoint{3.293782in}{2.273543in}}{\pgfqpoint{3.298173in}{2.262944in}}{\pgfqpoint{3.305986in}{2.255130in}}%
\pgfpathcurveto{\pgfqpoint{3.313800in}{2.247317in}}{\pgfqpoint{3.324399in}{2.242926in}}{\pgfqpoint{3.335449in}{2.242926in}}%
\pgfpathclose%
\pgfusepath{stroke,fill}%
\end{pgfscope}%
\begin{pgfscope}%
\pgfpathrectangle{\pgfqpoint{0.787074in}{0.548769in}}{\pgfqpoint{5.062926in}{3.102590in}}%
\pgfusepath{clip}%
\pgfsetbuttcap%
\pgfsetroundjoin%
\definecolor{currentfill}{rgb}{0.121569,0.466667,0.705882}%
\pgfsetfillcolor{currentfill}%
\pgfsetlinewidth{1.003750pt}%
\definecolor{currentstroke}{rgb}{0.121569,0.466667,0.705882}%
\pgfsetstrokecolor{currentstroke}%
\pgfsetdash{}{0pt}%
\pgfpathmoveto{\pgfqpoint{1.020465in}{0.650065in}}%
\pgfpathcurveto{\pgfqpoint{1.031515in}{0.650065in}}{\pgfqpoint{1.042114in}{0.654455in}}{\pgfqpoint{1.049927in}{0.662269in}}%
\pgfpathcurveto{\pgfqpoint{1.057741in}{0.670082in}}{\pgfqpoint{1.062131in}{0.680681in}}{\pgfqpoint{1.062131in}{0.691731in}}%
\pgfpathcurveto{\pgfqpoint{1.062131in}{0.702781in}}{\pgfqpoint{1.057741in}{0.713380in}}{\pgfqpoint{1.049927in}{0.721194in}}%
\pgfpathcurveto{\pgfqpoint{1.042114in}{0.729008in}}{\pgfqpoint{1.031515in}{0.733398in}}{\pgfqpoint{1.020465in}{0.733398in}}%
\pgfpathcurveto{\pgfqpoint{1.009414in}{0.733398in}}{\pgfqpoint{0.998815in}{0.729008in}}{\pgfqpoint{0.991002in}{0.721194in}}%
\pgfpathcurveto{\pgfqpoint{0.983188in}{0.713380in}}{\pgfqpoint{0.978798in}{0.702781in}}{\pgfqpoint{0.978798in}{0.691731in}}%
\pgfpathcurveto{\pgfqpoint{0.978798in}{0.680681in}}{\pgfqpoint{0.983188in}{0.670082in}}{\pgfqpoint{0.991002in}{0.662269in}}%
\pgfpathcurveto{\pgfqpoint{0.998815in}{0.654455in}}{\pgfqpoint{1.009414in}{0.650065in}}{\pgfqpoint{1.020465in}{0.650065in}}%
\pgfpathclose%
\pgfusepath{stroke,fill}%
\end{pgfscope}%
\begin{pgfscope}%
\pgfpathrectangle{\pgfqpoint{0.787074in}{0.548769in}}{\pgfqpoint{5.062926in}{3.102590in}}%
\pgfusepath{clip}%
\pgfsetbuttcap%
\pgfsetroundjoin%
\definecolor{currentfill}{rgb}{0.121569,0.466667,0.705882}%
\pgfsetfillcolor{currentfill}%
\pgfsetlinewidth{1.003750pt}%
\definecolor{currentstroke}{rgb}{0.121569,0.466667,0.705882}%
\pgfsetstrokecolor{currentstroke}%
\pgfsetdash{}{0pt}%
\pgfpathmoveto{\pgfqpoint{4.205058in}{2.492055in}}%
\pgfpathcurveto{\pgfqpoint{4.216109in}{2.492055in}}{\pgfqpoint{4.226708in}{2.496445in}}{\pgfqpoint{4.234521in}{2.504258in}}%
\pgfpathcurveto{\pgfqpoint{4.242335in}{2.512072in}}{\pgfqpoint{4.246725in}{2.522671in}}{\pgfqpoint{4.246725in}{2.533721in}}%
\pgfpathcurveto{\pgfqpoint{4.246725in}{2.544771in}}{\pgfqpoint{4.242335in}{2.555370in}}{\pgfqpoint{4.234521in}{2.563184in}}%
\pgfpathcurveto{\pgfqpoint{4.226708in}{2.570998in}}{\pgfqpoint{4.216109in}{2.575388in}}{\pgfqpoint{4.205058in}{2.575388in}}%
\pgfpathcurveto{\pgfqpoint{4.194008in}{2.575388in}}{\pgfqpoint{4.183409in}{2.570998in}}{\pgfqpoint{4.175596in}{2.563184in}}%
\pgfpathcurveto{\pgfqpoint{4.167782in}{2.555370in}}{\pgfqpoint{4.163392in}{2.544771in}}{\pgfqpoint{4.163392in}{2.533721in}}%
\pgfpathcurveto{\pgfqpoint{4.163392in}{2.522671in}}{\pgfqpoint{4.167782in}{2.512072in}}{\pgfqpoint{4.175596in}{2.504258in}}%
\pgfpathcurveto{\pgfqpoint{4.183409in}{2.496445in}}{\pgfqpoint{4.194008in}{2.492055in}}{\pgfqpoint{4.205058in}{2.492055in}}%
\pgfpathclose%
\pgfusepath{stroke,fill}%
\end{pgfscope}%
\begin{pgfscope}%
\pgfpathrectangle{\pgfqpoint{0.787074in}{0.548769in}}{\pgfqpoint{5.062926in}{3.102590in}}%
\pgfusepath{clip}%
\pgfsetbuttcap%
\pgfsetroundjoin%
\definecolor{currentfill}{rgb}{1.000000,0.498039,0.054902}%
\pgfsetfillcolor{currentfill}%
\pgfsetlinewidth{1.003750pt}%
\definecolor{currentstroke}{rgb}{1.000000,0.498039,0.054902}%
\pgfsetstrokecolor{currentstroke}%
\pgfsetdash{}{0pt}%
\pgfpathmoveto{\pgfqpoint{4.150756in}{2.495853in}}%
\pgfpathcurveto{\pgfqpoint{4.161806in}{2.495853in}}{\pgfqpoint{4.172405in}{2.500244in}}{\pgfqpoint{4.180219in}{2.508057in}}%
\pgfpathcurveto{\pgfqpoint{4.188033in}{2.515871in}}{\pgfqpoint{4.192423in}{2.526470in}}{\pgfqpoint{4.192423in}{2.537520in}}%
\pgfpathcurveto{\pgfqpoint{4.192423in}{2.548570in}}{\pgfqpoint{4.188033in}{2.559169in}}{\pgfqpoint{4.180219in}{2.566983in}}%
\pgfpathcurveto{\pgfqpoint{4.172405in}{2.574797in}}{\pgfqpoint{4.161806in}{2.579187in}}{\pgfqpoint{4.150756in}{2.579187in}}%
\pgfpathcurveto{\pgfqpoint{4.139706in}{2.579187in}}{\pgfqpoint{4.129107in}{2.574797in}}{\pgfqpoint{4.121293in}{2.566983in}}%
\pgfpathcurveto{\pgfqpoint{4.113480in}{2.559169in}}{\pgfqpoint{4.109089in}{2.548570in}}{\pgfqpoint{4.109089in}{2.537520in}}%
\pgfpathcurveto{\pgfqpoint{4.109089in}{2.526470in}}{\pgfqpoint{4.113480in}{2.515871in}}{\pgfqpoint{4.121293in}{2.508057in}}%
\pgfpathcurveto{\pgfqpoint{4.129107in}{2.500244in}}{\pgfqpoint{4.139706in}{2.495853in}}{\pgfqpoint{4.150756in}{2.495853in}}%
\pgfpathclose%
\pgfusepath{stroke,fill}%
\end{pgfscope}%
\begin{pgfscope}%
\pgfpathrectangle{\pgfqpoint{0.787074in}{0.548769in}}{\pgfqpoint{5.062926in}{3.102590in}}%
\pgfusepath{clip}%
\pgfsetbuttcap%
\pgfsetroundjoin%
\definecolor{currentfill}{rgb}{0.121569,0.466667,0.705882}%
\pgfsetfillcolor{currentfill}%
\pgfsetlinewidth{1.003750pt}%
\definecolor{currentstroke}{rgb}{0.121569,0.466667,0.705882}%
\pgfsetstrokecolor{currentstroke}%
\pgfsetdash{}{0pt}%
\pgfpathmoveto{\pgfqpoint{1.017264in}{0.648161in}}%
\pgfpathcurveto{\pgfqpoint{1.028315in}{0.648161in}}{\pgfqpoint{1.038914in}{0.652552in}}{\pgfqpoint{1.046727in}{0.660365in}}%
\pgfpathcurveto{\pgfqpoint{1.054541in}{0.668179in}}{\pgfqpoint{1.058931in}{0.678778in}}{\pgfqpoint{1.058931in}{0.689828in}}%
\pgfpathcurveto{\pgfqpoint{1.058931in}{0.700878in}}{\pgfqpoint{1.054541in}{0.711477in}}{\pgfqpoint{1.046727in}{0.719291in}}%
\pgfpathcurveto{\pgfqpoint{1.038914in}{0.727104in}}{\pgfqpoint{1.028315in}{0.731495in}}{\pgfqpoint{1.017264in}{0.731495in}}%
\pgfpathcurveto{\pgfqpoint{1.006214in}{0.731495in}}{\pgfqpoint{0.995615in}{0.727104in}}{\pgfqpoint{0.987802in}{0.719291in}}%
\pgfpathcurveto{\pgfqpoint{0.979988in}{0.711477in}}{\pgfqpoint{0.975598in}{0.700878in}}{\pgfqpoint{0.975598in}{0.689828in}}%
\pgfpathcurveto{\pgfqpoint{0.975598in}{0.678778in}}{\pgfqpoint{0.979988in}{0.668179in}}{\pgfqpoint{0.987802in}{0.660365in}}%
\pgfpathcurveto{\pgfqpoint{0.995615in}{0.652552in}}{\pgfqpoint{1.006214in}{0.648161in}}{\pgfqpoint{1.017264in}{0.648161in}}%
\pgfpathclose%
\pgfusepath{stroke,fill}%
\end{pgfscope}%
\begin{pgfscope}%
\pgfpathrectangle{\pgfqpoint{0.787074in}{0.548769in}}{\pgfqpoint{5.062926in}{3.102590in}}%
\pgfusepath{clip}%
\pgfsetbuttcap%
\pgfsetroundjoin%
\definecolor{currentfill}{rgb}{0.121569,0.466667,0.705882}%
\pgfsetfillcolor{currentfill}%
\pgfsetlinewidth{1.003750pt}%
\definecolor{currentstroke}{rgb}{0.121569,0.466667,0.705882}%
\pgfsetstrokecolor{currentstroke}%
\pgfsetdash{}{0pt}%
\pgfpathmoveto{\pgfqpoint{1.412234in}{0.839743in}}%
\pgfpathcurveto{\pgfqpoint{1.423284in}{0.839743in}}{\pgfqpoint{1.433883in}{0.844134in}}{\pgfqpoint{1.441697in}{0.851947in}}%
\pgfpathcurveto{\pgfqpoint{1.449511in}{0.859761in}}{\pgfqpoint{1.453901in}{0.870360in}}{\pgfqpoint{1.453901in}{0.881410in}}%
\pgfpathcurveto{\pgfqpoint{1.453901in}{0.892460in}}{\pgfqpoint{1.449511in}{0.903059in}}{\pgfqpoint{1.441697in}{0.910873in}}%
\pgfpathcurveto{\pgfqpoint{1.433883in}{0.918687in}}{\pgfqpoint{1.423284in}{0.923077in}}{\pgfqpoint{1.412234in}{0.923077in}}%
\pgfpathcurveto{\pgfqpoint{1.401184in}{0.923077in}}{\pgfqpoint{1.390585in}{0.918687in}}{\pgfqpoint{1.382772in}{0.910873in}}%
\pgfpathcurveto{\pgfqpoint{1.374958in}{0.903059in}}{\pgfqpoint{1.370568in}{0.892460in}}{\pgfqpoint{1.370568in}{0.881410in}}%
\pgfpathcurveto{\pgfqpoint{1.370568in}{0.870360in}}{\pgfqpoint{1.374958in}{0.859761in}}{\pgfqpoint{1.382772in}{0.851947in}}%
\pgfpathcurveto{\pgfqpoint{1.390585in}{0.844134in}}{\pgfqpoint{1.401184in}{0.839743in}}{\pgfqpoint{1.412234in}{0.839743in}}%
\pgfpathclose%
\pgfusepath{stroke,fill}%
\end{pgfscope}%
\begin{pgfscope}%
\pgfpathrectangle{\pgfqpoint{0.787074in}{0.548769in}}{\pgfqpoint{5.062926in}{3.102590in}}%
\pgfusepath{clip}%
\pgfsetbuttcap%
\pgfsetroundjoin%
\definecolor{currentfill}{rgb}{0.121569,0.466667,0.705882}%
\pgfsetfillcolor{currentfill}%
\pgfsetlinewidth{1.003750pt}%
\definecolor{currentstroke}{rgb}{0.121569,0.466667,0.705882}%
\pgfsetstrokecolor{currentstroke}%
\pgfsetdash{}{0pt}%
\pgfpathmoveto{\pgfqpoint{1.310698in}{0.786120in}}%
\pgfpathcurveto{\pgfqpoint{1.321749in}{0.786120in}}{\pgfqpoint{1.332348in}{0.790510in}}{\pgfqpoint{1.340161in}{0.798324in}}%
\pgfpathcurveto{\pgfqpoint{1.347975in}{0.806137in}}{\pgfqpoint{1.352365in}{0.816736in}}{\pgfqpoint{1.352365in}{0.827787in}}%
\pgfpathcurveto{\pgfqpoint{1.352365in}{0.838837in}}{\pgfqpoint{1.347975in}{0.849436in}}{\pgfqpoint{1.340161in}{0.857249in}}%
\pgfpathcurveto{\pgfqpoint{1.332348in}{0.865063in}}{\pgfqpoint{1.321749in}{0.869453in}}{\pgfqpoint{1.310698in}{0.869453in}}%
\pgfpathcurveto{\pgfqpoint{1.299648in}{0.869453in}}{\pgfqpoint{1.289049in}{0.865063in}}{\pgfqpoint{1.281236in}{0.857249in}}%
\pgfpathcurveto{\pgfqpoint{1.273422in}{0.849436in}}{\pgfqpoint{1.269032in}{0.838837in}}{\pgfqpoint{1.269032in}{0.827787in}}%
\pgfpathcurveto{\pgfqpoint{1.269032in}{0.816736in}}{\pgfqpoint{1.273422in}{0.806137in}}{\pgfqpoint{1.281236in}{0.798324in}}%
\pgfpathcurveto{\pgfqpoint{1.289049in}{0.790510in}}{\pgfqpoint{1.299648in}{0.786120in}}{\pgfqpoint{1.310698in}{0.786120in}}%
\pgfpathclose%
\pgfusepath{stroke,fill}%
\end{pgfscope}%
\begin{pgfscope}%
\pgfpathrectangle{\pgfqpoint{0.787074in}{0.548769in}}{\pgfqpoint{5.062926in}{3.102590in}}%
\pgfusepath{clip}%
\pgfsetbuttcap%
\pgfsetroundjoin%
\definecolor{currentfill}{rgb}{0.121569,0.466667,0.705882}%
\pgfsetfillcolor{currentfill}%
\pgfsetlinewidth{1.003750pt}%
\definecolor{currentstroke}{rgb}{0.121569,0.466667,0.705882}%
\pgfsetstrokecolor{currentstroke}%
\pgfsetdash{}{0pt}%
\pgfpathmoveto{\pgfqpoint{1.037670in}{0.660513in}}%
\pgfpathcurveto{\pgfqpoint{1.048720in}{0.660513in}}{\pgfqpoint{1.059319in}{0.664903in}}{\pgfqpoint{1.067133in}{0.672717in}}%
\pgfpathcurveto{\pgfqpoint{1.074947in}{0.680531in}}{\pgfqpoint{1.079337in}{0.691130in}}{\pgfqpoint{1.079337in}{0.702180in}}%
\pgfpathcurveto{\pgfqpoint{1.079337in}{0.713230in}}{\pgfqpoint{1.074947in}{0.723829in}}{\pgfqpoint{1.067133in}{0.731643in}}%
\pgfpathcurveto{\pgfqpoint{1.059319in}{0.739456in}}{\pgfqpoint{1.048720in}{0.743847in}}{\pgfqpoint{1.037670in}{0.743847in}}%
\pgfpathcurveto{\pgfqpoint{1.026620in}{0.743847in}}{\pgfqpoint{1.016021in}{0.739456in}}{\pgfqpoint{1.008207in}{0.731643in}}%
\pgfpathcurveto{\pgfqpoint{1.000394in}{0.723829in}}{\pgfqpoint{0.996004in}{0.713230in}}{\pgfqpoint{0.996004in}{0.702180in}}%
\pgfpathcurveto{\pgfqpoint{0.996004in}{0.691130in}}{\pgfqpoint{1.000394in}{0.680531in}}{\pgfqpoint{1.008207in}{0.672717in}}%
\pgfpathcurveto{\pgfqpoint{1.016021in}{0.664903in}}{\pgfqpoint{1.026620in}{0.660513in}}{\pgfqpoint{1.037670in}{0.660513in}}%
\pgfpathclose%
\pgfusepath{stroke,fill}%
\end{pgfscope}%
\begin{pgfscope}%
\pgfpathrectangle{\pgfqpoint{0.787074in}{0.548769in}}{\pgfqpoint{5.062926in}{3.102590in}}%
\pgfusepath{clip}%
\pgfsetbuttcap%
\pgfsetroundjoin%
\definecolor{currentfill}{rgb}{0.121569,0.466667,0.705882}%
\pgfsetfillcolor{currentfill}%
\pgfsetlinewidth{1.003750pt}%
\definecolor{currentstroke}{rgb}{0.121569,0.466667,0.705882}%
\pgfsetstrokecolor{currentstroke}%
\pgfsetdash{}{0pt}%
\pgfpathmoveto{\pgfqpoint{1.017245in}{0.648155in}}%
\pgfpathcurveto{\pgfqpoint{1.028295in}{0.648155in}}{\pgfqpoint{1.038894in}{0.652545in}}{\pgfqpoint{1.046708in}{0.660359in}}%
\pgfpathcurveto{\pgfqpoint{1.054522in}{0.668173in}}{\pgfqpoint{1.058912in}{0.678772in}}{\pgfqpoint{1.058912in}{0.689822in}}%
\pgfpathcurveto{\pgfqpoint{1.058912in}{0.700872in}}{\pgfqpoint{1.054522in}{0.711471in}}{\pgfqpoint{1.046708in}{0.719285in}}%
\pgfpathcurveto{\pgfqpoint{1.038894in}{0.727098in}}{\pgfqpoint{1.028295in}{0.731489in}}{\pgfqpoint{1.017245in}{0.731489in}}%
\pgfpathcurveto{\pgfqpoint{1.006195in}{0.731489in}}{\pgfqpoint{0.995596in}{0.727098in}}{\pgfqpoint{0.987782in}{0.719285in}}%
\pgfpathcurveto{\pgfqpoint{0.979969in}{0.711471in}}{\pgfqpoint{0.975579in}{0.700872in}}{\pgfqpoint{0.975579in}{0.689822in}}%
\pgfpathcurveto{\pgfqpoint{0.975579in}{0.678772in}}{\pgfqpoint{0.979969in}{0.668173in}}{\pgfqpoint{0.987782in}{0.660359in}}%
\pgfpathcurveto{\pgfqpoint{0.995596in}{0.652545in}}{\pgfqpoint{1.006195in}{0.648155in}}{\pgfqpoint{1.017245in}{0.648155in}}%
\pgfpathclose%
\pgfusepath{stroke,fill}%
\end{pgfscope}%
\begin{pgfscope}%
\pgfpathrectangle{\pgfqpoint{0.787074in}{0.548769in}}{\pgfqpoint{5.062926in}{3.102590in}}%
\pgfusepath{clip}%
\pgfsetbuttcap%
\pgfsetroundjoin%
\definecolor{currentfill}{rgb}{0.121569,0.466667,0.705882}%
\pgfsetfillcolor{currentfill}%
\pgfsetlinewidth{1.003750pt}%
\definecolor{currentstroke}{rgb}{0.121569,0.466667,0.705882}%
\pgfsetstrokecolor{currentstroke}%
\pgfsetdash{}{0pt}%
\pgfpathmoveto{\pgfqpoint{1.017233in}{0.648148in}}%
\pgfpathcurveto{\pgfqpoint{1.028283in}{0.648148in}}{\pgfqpoint{1.038882in}{0.652538in}}{\pgfqpoint{1.046696in}{0.660352in}}%
\pgfpathcurveto{\pgfqpoint{1.054509in}{0.668166in}}{\pgfqpoint{1.058900in}{0.678765in}}{\pgfqpoint{1.058900in}{0.689815in}}%
\pgfpathcurveto{\pgfqpoint{1.058900in}{0.700865in}}{\pgfqpoint{1.054509in}{0.711464in}}{\pgfqpoint{1.046696in}{0.719278in}}%
\pgfpathcurveto{\pgfqpoint{1.038882in}{0.727091in}}{\pgfqpoint{1.028283in}{0.731482in}}{\pgfqpoint{1.017233in}{0.731482in}}%
\pgfpathcurveto{\pgfqpoint{1.006183in}{0.731482in}}{\pgfqpoint{0.995584in}{0.727091in}}{\pgfqpoint{0.987770in}{0.719278in}}%
\pgfpathcurveto{\pgfqpoint{0.979957in}{0.711464in}}{\pgfqpoint{0.975566in}{0.700865in}}{\pgfqpoint{0.975566in}{0.689815in}}%
\pgfpathcurveto{\pgfqpoint{0.975566in}{0.678765in}}{\pgfqpoint{0.979957in}{0.668166in}}{\pgfqpoint{0.987770in}{0.660352in}}%
\pgfpathcurveto{\pgfqpoint{0.995584in}{0.652538in}}{\pgfqpoint{1.006183in}{0.648148in}}{\pgfqpoint{1.017233in}{0.648148in}}%
\pgfpathclose%
\pgfusepath{stroke,fill}%
\end{pgfscope}%
\begin{pgfscope}%
\pgfpathrectangle{\pgfqpoint{0.787074in}{0.548769in}}{\pgfqpoint{5.062926in}{3.102590in}}%
\pgfusepath{clip}%
\pgfsetbuttcap%
\pgfsetroundjoin%
\definecolor{currentfill}{rgb}{1.000000,0.498039,0.054902}%
\pgfsetfillcolor{currentfill}%
\pgfsetlinewidth{1.003750pt}%
\definecolor{currentstroke}{rgb}{1.000000,0.498039,0.054902}%
\pgfsetstrokecolor{currentstroke}%
\pgfsetdash{}{0pt}%
\pgfpathmoveto{\pgfqpoint{4.920458in}{2.337155in}}%
\pgfpathcurveto{\pgfqpoint{4.931508in}{2.337155in}}{\pgfqpoint{4.942107in}{2.341545in}}{\pgfqpoint{4.949921in}{2.349358in}}%
\pgfpathcurveto{\pgfqpoint{4.957734in}{2.357172in}}{\pgfqpoint{4.962124in}{2.367771in}}{\pgfqpoint{4.962124in}{2.378821in}}%
\pgfpathcurveto{\pgfqpoint{4.962124in}{2.389871in}}{\pgfqpoint{4.957734in}{2.400470in}}{\pgfqpoint{4.949921in}{2.408284in}}%
\pgfpathcurveto{\pgfqpoint{4.942107in}{2.416098in}}{\pgfqpoint{4.931508in}{2.420488in}}{\pgfqpoint{4.920458in}{2.420488in}}%
\pgfpathcurveto{\pgfqpoint{4.909408in}{2.420488in}}{\pgfqpoint{4.898809in}{2.416098in}}{\pgfqpoint{4.890995in}{2.408284in}}%
\pgfpathcurveto{\pgfqpoint{4.883181in}{2.400470in}}{\pgfqpoint{4.878791in}{2.389871in}}{\pgfqpoint{4.878791in}{2.378821in}}%
\pgfpathcurveto{\pgfqpoint{4.878791in}{2.367771in}}{\pgfqpoint{4.883181in}{2.357172in}}{\pgfqpoint{4.890995in}{2.349358in}}%
\pgfpathcurveto{\pgfqpoint{4.898809in}{2.341545in}}{\pgfqpoint{4.909408in}{2.337155in}}{\pgfqpoint{4.920458in}{2.337155in}}%
\pgfpathclose%
\pgfusepath{stroke,fill}%
\end{pgfscope}%
\begin{pgfscope}%
\pgfpathrectangle{\pgfqpoint{0.787074in}{0.548769in}}{\pgfqpoint{5.062926in}{3.102590in}}%
\pgfusepath{clip}%
\pgfsetbuttcap%
\pgfsetroundjoin%
\definecolor{currentfill}{rgb}{1.000000,0.498039,0.054902}%
\pgfsetfillcolor{currentfill}%
\pgfsetlinewidth{1.003750pt}%
\definecolor{currentstroke}{rgb}{1.000000,0.498039,0.054902}%
\pgfsetstrokecolor{currentstroke}%
\pgfsetdash{}{0pt}%
\pgfpathmoveto{\pgfqpoint{5.216813in}{2.913319in}}%
\pgfpathcurveto{\pgfqpoint{5.227864in}{2.913319in}}{\pgfqpoint{5.238463in}{2.917709in}}{\pgfqpoint{5.246276in}{2.925523in}}%
\pgfpathcurveto{\pgfqpoint{5.254090in}{2.933336in}}{\pgfqpoint{5.258480in}{2.943935in}}{\pgfqpoint{5.258480in}{2.954985in}}%
\pgfpathcurveto{\pgfqpoint{5.258480in}{2.966036in}}{\pgfqpoint{5.254090in}{2.976635in}}{\pgfqpoint{5.246276in}{2.984448in}}%
\pgfpathcurveto{\pgfqpoint{5.238463in}{2.992262in}}{\pgfqpoint{5.227864in}{2.996652in}}{\pgfqpoint{5.216813in}{2.996652in}}%
\pgfpathcurveto{\pgfqpoint{5.205763in}{2.996652in}}{\pgfqpoint{5.195164in}{2.992262in}}{\pgfqpoint{5.187351in}{2.984448in}}%
\pgfpathcurveto{\pgfqpoint{5.179537in}{2.976635in}}{\pgfqpoint{5.175147in}{2.966036in}}{\pgfqpoint{5.175147in}{2.954985in}}%
\pgfpathcurveto{\pgfqpoint{5.175147in}{2.943935in}}{\pgfqpoint{5.179537in}{2.933336in}}{\pgfqpoint{5.187351in}{2.925523in}}%
\pgfpathcurveto{\pgfqpoint{5.195164in}{2.917709in}}{\pgfqpoint{5.205763in}{2.913319in}}{\pgfqpoint{5.216813in}{2.913319in}}%
\pgfpathclose%
\pgfusepath{stroke,fill}%
\end{pgfscope}%
\begin{pgfscope}%
\pgfpathrectangle{\pgfqpoint{0.787074in}{0.548769in}}{\pgfqpoint{5.062926in}{3.102590in}}%
\pgfusepath{clip}%
\pgfsetbuttcap%
\pgfsetroundjoin%
\definecolor{currentfill}{rgb}{0.121569,0.466667,0.705882}%
\pgfsetfillcolor{currentfill}%
\pgfsetlinewidth{1.003750pt}%
\definecolor{currentstroke}{rgb}{0.121569,0.466667,0.705882}%
\pgfsetstrokecolor{currentstroke}%
\pgfsetdash{}{0pt}%
\pgfpathmoveto{\pgfqpoint{1.375419in}{0.825123in}}%
\pgfpathcurveto{\pgfqpoint{1.386470in}{0.825123in}}{\pgfqpoint{1.397069in}{0.829514in}}{\pgfqpoint{1.404882in}{0.837327in}}%
\pgfpathcurveto{\pgfqpoint{1.412696in}{0.845141in}}{\pgfqpoint{1.417086in}{0.855740in}}{\pgfqpoint{1.417086in}{0.866790in}}%
\pgfpathcurveto{\pgfqpoint{1.417086in}{0.877840in}}{\pgfqpoint{1.412696in}{0.888439in}}{\pgfqpoint{1.404882in}{0.896253in}}%
\pgfpathcurveto{\pgfqpoint{1.397069in}{0.904067in}}{\pgfqpoint{1.386470in}{0.908457in}}{\pgfqpoint{1.375419in}{0.908457in}}%
\pgfpathcurveto{\pgfqpoint{1.364369in}{0.908457in}}{\pgfqpoint{1.353770in}{0.904067in}}{\pgfqpoint{1.345957in}{0.896253in}}%
\pgfpathcurveto{\pgfqpoint{1.338143in}{0.888439in}}{\pgfqpoint{1.333753in}{0.877840in}}{\pgfqpoint{1.333753in}{0.866790in}}%
\pgfpathcurveto{\pgfqpoint{1.333753in}{0.855740in}}{\pgfqpoint{1.338143in}{0.845141in}}{\pgfqpoint{1.345957in}{0.837327in}}%
\pgfpathcurveto{\pgfqpoint{1.353770in}{0.829514in}}{\pgfqpoint{1.364369in}{0.825123in}}{\pgfqpoint{1.375419in}{0.825123in}}%
\pgfpathclose%
\pgfusepath{stroke,fill}%
\end{pgfscope}%
\begin{pgfscope}%
\pgfpathrectangle{\pgfqpoint{0.787074in}{0.548769in}}{\pgfqpoint{5.062926in}{3.102590in}}%
\pgfusepath{clip}%
\pgfsetbuttcap%
\pgfsetroundjoin%
\definecolor{currentfill}{rgb}{1.000000,0.498039,0.054902}%
\pgfsetfillcolor{currentfill}%
\pgfsetlinewidth{1.003750pt}%
\definecolor{currentstroke}{rgb}{1.000000,0.498039,0.054902}%
\pgfsetstrokecolor{currentstroke}%
\pgfsetdash{}{0pt}%
\pgfpathmoveto{\pgfqpoint{3.015360in}{1.776870in}}%
\pgfpathcurveto{\pgfqpoint{3.026410in}{1.776870in}}{\pgfqpoint{3.037010in}{1.781260in}}{\pgfqpoint{3.044823in}{1.789074in}}%
\pgfpathcurveto{\pgfqpoint{3.052637in}{1.796887in}}{\pgfqpoint{3.057027in}{1.807486in}}{\pgfqpoint{3.057027in}{1.818537in}}%
\pgfpathcurveto{\pgfqpoint{3.057027in}{1.829587in}}{\pgfqpoint{3.052637in}{1.840186in}}{\pgfqpoint{3.044823in}{1.847999in}}%
\pgfpathcurveto{\pgfqpoint{3.037010in}{1.855813in}}{\pgfqpoint{3.026410in}{1.860203in}}{\pgfqpoint{3.015360in}{1.860203in}}%
\pgfpathcurveto{\pgfqpoint{3.004310in}{1.860203in}}{\pgfqpoint{2.993711in}{1.855813in}}{\pgfqpoint{2.985898in}{1.847999in}}%
\pgfpathcurveto{\pgfqpoint{2.978084in}{1.840186in}}{\pgfqpoint{2.973694in}{1.829587in}}{\pgfqpoint{2.973694in}{1.818537in}}%
\pgfpathcurveto{\pgfqpoint{2.973694in}{1.807486in}}{\pgfqpoint{2.978084in}{1.796887in}}{\pgfqpoint{2.985898in}{1.789074in}}%
\pgfpathcurveto{\pgfqpoint{2.993711in}{1.781260in}}{\pgfqpoint{3.004310in}{1.776870in}}{\pgfqpoint{3.015360in}{1.776870in}}%
\pgfpathclose%
\pgfusepath{stroke,fill}%
\end{pgfscope}%
\begin{pgfscope}%
\pgfpathrectangle{\pgfqpoint{0.787074in}{0.548769in}}{\pgfqpoint{5.062926in}{3.102590in}}%
\pgfusepath{clip}%
\pgfsetbuttcap%
\pgfsetroundjoin%
\definecolor{currentfill}{rgb}{1.000000,0.498039,0.054902}%
\pgfsetfillcolor{currentfill}%
\pgfsetlinewidth{1.003750pt}%
\definecolor{currentstroke}{rgb}{1.000000,0.498039,0.054902}%
\pgfsetstrokecolor{currentstroke}%
\pgfsetdash{}{0pt}%
\pgfpathmoveto{\pgfqpoint{4.492945in}{2.789062in}}%
\pgfpathcurveto{\pgfqpoint{4.503995in}{2.789062in}}{\pgfqpoint{4.514594in}{2.793453in}}{\pgfqpoint{4.522407in}{2.801266in}}%
\pgfpathcurveto{\pgfqpoint{4.530221in}{2.809080in}}{\pgfqpoint{4.534611in}{2.819679in}}{\pgfqpoint{4.534611in}{2.830729in}}%
\pgfpathcurveto{\pgfqpoint{4.534611in}{2.841779in}}{\pgfqpoint{4.530221in}{2.852378in}}{\pgfqpoint{4.522407in}{2.860192in}}%
\pgfpathcurveto{\pgfqpoint{4.514594in}{2.868005in}}{\pgfqpoint{4.503995in}{2.872396in}}{\pgfqpoint{4.492945in}{2.872396in}}%
\pgfpathcurveto{\pgfqpoint{4.481894in}{2.872396in}}{\pgfqpoint{4.471295in}{2.868005in}}{\pgfqpoint{4.463482in}{2.860192in}}%
\pgfpathcurveto{\pgfqpoint{4.455668in}{2.852378in}}{\pgfqpoint{4.451278in}{2.841779in}}{\pgfqpoint{4.451278in}{2.830729in}}%
\pgfpathcurveto{\pgfqpoint{4.451278in}{2.819679in}}{\pgfqpoint{4.455668in}{2.809080in}}{\pgfqpoint{4.463482in}{2.801266in}}%
\pgfpathcurveto{\pgfqpoint{4.471295in}{2.793453in}}{\pgfqpoint{4.481894in}{2.789062in}}{\pgfqpoint{4.492945in}{2.789062in}}%
\pgfpathclose%
\pgfusepath{stroke,fill}%
\end{pgfscope}%
\begin{pgfscope}%
\pgfpathrectangle{\pgfqpoint{0.787074in}{0.548769in}}{\pgfqpoint{5.062926in}{3.102590in}}%
\pgfusepath{clip}%
\pgfsetbuttcap%
\pgfsetroundjoin%
\definecolor{currentfill}{rgb}{0.121569,0.466667,0.705882}%
\pgfsetfillcolor{currentfill}%
\pgfsetlinewidth{1.003750pt}%
\definecolor{currentstroke}{rgb}{0.121569,0.466667,0.705882}%
\pgfsetstrokecolor{currentstroke}%
\pgfsetdash{}{0pt}%
\pgfpathmoveto{\pgfqpoint{1.017236in}{0.648149in}}%
\pgfpathcurveto{\pgfqpoint{1.028286in}{0.648149in}}{\pgfqpoint{1.038885in}{0.652540in}}{\pgfqpoint{1.046699in}{0.660353in}}%
\pgfpathcurveto{\pgfqpoint{1.054512in}{0.668167in}}{\pgfqpoint{1.058903in}{0.678766in}}{\pgfqpoint{1.058903in}{0.689816in}}%
\pgfpathcurveto{\pgfqpoint{1.058903in}{0.700866in}}{\pgfqpoint{1.054512in}{0.711465in}}{\pgfqpoint{1.046699in}{0.719279in}}%
\pgfpathcurveto{\pgfqpoint{1.038885in}{0.727093in}}{\pgfqpoint{1.028286in}{0.731483in}}{\pgfqpoint{1.017236in}{0.731483in}}%
\pgfpathcurveto{\pgfqpoint{1.006186in}{0.731483in}}{\pgfqpoint{0.995587in}{0.727093in}}{\pgfqpoint{0.987773in}{0.719279in}}%
\pgfpathcurveto{\pgfqpoint{0.979960in}{0.711465in}}{\pgfqpoint{0.975569in}{0.700866in}}{\pgfqpoint{0.975569in}{0.689816in}}%
\pgfpathcurveto{\pgfqpoint{0.975569in}{0.678766in}}{\pgfqpoint{0.979960in}{0.668167in}}{\pgfqpoint{0.987773in}{0.660353in}}%
\pgfpathcurveto{\pgfqpoint{0.995587in}{0.652540in}}{\pgfqpoint{1.006186in}{0.648149in}}{\pgfqpoint{1.017236in}{0.648149in}}%
\pgfpathclose%
\pgfusepath{stroke,fill}%
\end{pgfscope}%
\begin{pgfscope}%
\pgfpathrectangle{\pgfqpoint{0.787074in}{0.548769in}}{\pgfqpoint{5.062926in}{3.102590in}}%
\pgfusepath{clip}%
\pgfsetbuttcap%
\pgfsetroundjoin%
\definecolor{currentfill}{rgb}{1.000000,0.498039,0.054902}%
\pgfsetfillcolor{currentfill}%
\pgfsetlinewidth{1.003750pt}%
\definecolor{currentstroke}{rgb}{1.000000,0.498039,0.054902}%
\pgfsetstrokecolor{currentstroke}%
\pgfsetdash{}{0pt}%
\pgfpathmoveto{\pgfqpoint{4.495513in}{2.789836in}}%
\pgfpathcurveto{\pgfqpoint{4.506563in}{2.789836in}}{\pgfqpoint{4.517162in}{2.794226in}}{\pgfqpoint{4.524976in}{2.802040in}}%
\pgfpathcurveto{\pgfqpoint{4.532790in}{2.809854in}}{\pgfqpoint{4.537180in}{2.820453in}}{\pgfqpoint{4.537180in}{2.831503in}}%
\pgfpathcurveto{\pgfqpoint{4.537180in}{2.842553in}}{\pgfqpoint{4.532790in}{2.853152in}}{\pgfqpoint{4.524976in}{2.860966in}}%
\pgfpathcurveto{\pgfqpoint{4.517162in}{2.868779in}}{\pgfqpoint{4.506563in}{2.873169in}}{\pgfqpoint{4.495513in}{2.873169in}}%
\pgfpathcurveto{\pgfqpoint{4.484463in}{2.873169in}}{\pgfqpoint{4.473864in}{2.868779in}}{\pgfqpoint{4.466051in}{2.860966in}}%
\pgfpathcurveto{\pgfqpoint{4.458237in}{2.853152in}}{\pgfqpoint{4.453847in}{2.842553in}}{\pgfqpoint{4.453847in}{2.831503in}}%
\pgfpathcurveto{\pgfqpoint{4.453847in}{2.820453in}}{\pgfqpoint{4.458237in}{2.809854in}}{\pgfqpoint{4.466051in}{2.802040in}}%
\pgfpathcurveto{\pgfqpoint{4.473864in}{2.794226in}}{\pgfqpoint{4.484463in}{2.789836in}}{\pgfqpoint{4.495513in}{2.789836in}}%
\pgfpathclose%
\pgfusepath{stroke,fill}%
\end{pgfscope}%
\begin{pgfscope}%
\pgfpathrectangle{\pgfqpoint{0.787074in}{0.548769in}}{\pgfqpoint{5.062926in}{3.102590in}}%
\pgfusepath{clip}%
\pgfsetbuttcap%
\pgfsetroundjoin%
\definecolor{currentfill}{rgb}{0.121569,0.466667,0.705882}%
\pgfsetfillcolor{currentfill}%
\pgfsetlinewidth{1.003750pt}%
\definecolor{currentstroke}{rgb}{0.121569,0.466667,0.705882}%
\pgfsetstrokecolor{currentstroke}%
\pgfsetdash{}{0pt}%
\pgfpathmoveto{\pgfqpoint{1.017233in}{0.648148in}}%
\pgfpathcurveto{\pgfqpoint{1.028283in}{0.648148in}}{\pgfqpoint{1.038882in}{0.652539in}}{\pgfqpoint{1.046696in}{0.660352in}}%
\pgfpathcurveto{\pgfqpoint{1.054509in}{0.668166in}}{\pgfqpoint{1.058900in}{0.678765in}}{\pgfqpoint{1.058900in}{0.689815in}}%
\pgfpathcurveto{\pgfqpoint{1.058900in}{0.700865in}}{\pgfqpoint{1.054509in}{0.711464in}}{\pgfqpoint{1.046696in}{0.719278in}}%
\pgfpathcurveto{\pgfqpoint{1.038882in}{0.727091in}}{\pgfqpoint{1.028283in}{0.731482in}}{\pgfqpoint{1.017233in}{0.731482in}}%
\pgfpathcurveto{\pgfqpoint{1.006183in}{0.731482in}}{\pgfqpoint{0.995584in}{0.727091in}}{\pgfqpoint{0.987770in}{0.719278in}}%
\pgfpathcurveto{\pgfqpoint{0.979957in}{0.711464in}}{\pgfqpoint{0.975566in}{0.700865in}}{\pgfqpoint{0.975566in}{0.689815in}}%
\pgfpathcurveto{\pgfqpoint{0.975566in}{0.678765in}}{\pgfqpoint{0.979957in}{0.668166in}}{\pgfqpoint{0.987770in}{0.660352in}}%
\pgfpathcurveto{\pgfqpoint{0.995584in}{0.652539in}}{\pgfqpoint{1.006183in}{0.648148in}}{\pgfqpoint{1.017233in}{0.648148in}}%
\pgfpathclose%
\pgfusepath{stroke,fill}%
\end{pgfscope}%
\begin{pgfscope}%
\pgfpathrectangle{\pgfqpoint{0.787074in}{0.548769in}}{\pgfqpoint{5.062926in}{3.102590in}}%
\pgfusepath{clip}%
\pgfsetbuttcap%
\pgfsetroundjoin%
\definecolor{currentfill}{rgb}{0.121569,0.466667,0.705882}%
\pgfsetfillcolor{currentfill}%
\pgfsetlinewidth{1.003750pt}%
\definecolor{currentstroke}{rgb}{0.121569,0.466667,0.705882}%
\pgfsetstrokecolor{currentstroke}%
\pgfsetdash{}{0pt}%
\pgfpathmoveto{\pgfqpoint{1.017247in}{0.648168in}}%
\pgfpathcurveto{\pgfqpoint{1.028297in}{0.648168in}}{\pgfqpoint{1.038896in}{0.652558in}}{\pgfqpoint{1.046710in}{0.660371in}}%
\pgfpathcurveto{\pgfqpoint{1.054523in}{0.668185in}}{\pgfqpoint{1.058914in}{0.678784in}}{\pgfqpoint{1.058914in}{0.689834in}}%
\pgfpathcurveto{\pgfqpoint{1.058914in}{0.700884in}}{\pgfqpoint{1.054523in}{0.711483in}}{\pgfqpoint{1.046710in}{0.719297in}}%
\pgfpathcurveto{\pgfqpoint{1.038896in}{0.727111in}}{\pgfqpoint{1.028297in}{0.731501in}}{\pgfqpoint{1.017247in}{0.731501in}}%
\pgfpathcurveto{\pgfqpoint{1.006197in}{0.731501in}}{\pgfqpoint{0.995598in}{0.727111in}}{\pgfqpoint{0.987784in}{0.719297in}}%
\pgfpathcurveto{\pgfqpoint{0.979971in}{0.711483in}}{\pgfqpoint{0.975580in}{0.700884in}}{\pgfqpoint{0.975580in}{0.689834in}}%
\pgfpathcurveto{\pgfqpoint{0.975580in}{0.678784in}}{\pgfqpoint{0.979971in}{0.668185in}}{\pgfqpoint{0.987784in}{0.660371in}}%
\pgfpathcurveto{\pgfqpoint{0.995598in}{0.652558in}}{\pgfqpoint{1.006197in}{0.648168in}}{\pgfqpoint{1.017247in}{0.648168in}}%
\pgfpathclose%
\pgfusepath{stroke,fill}%
\end{pgfscope}%
\begin{pgfscope}%
\pgfpathrectangle{\pgfqpoint{0.787074in}{0.548769in}}{\pgfqpoint{5.062926in}{3.102590in}}%
\pgfusepath{clip}%
\pgfsetbuttcap%
\pgfsetroundjoin%
\definecolor{currentfill}{rgb}{0.121569,0.466667,0.705882}%
\pgfsetfillcolor{currentfill}%
\pgfsetlinewidth{1.003750pt}%
\definecolor{currentstroke}{rgb}{0.121569,0.466667,0.705882}%
\pgfsetstrokecolor{currentstroke}%
\pgfsetdash{}{0pt}%
\pgfpathmoveto{\pgfqpoint{1.416287in}{0.858168in}}%
\pgfpathcurveto{\pgfqpoint{1.427337in}{0.858168in}}{\pgfqpoint{1.437936in}{0.862558in}}{\pgfqpoint{1.445750in}{0.870372in}}%
\pgfpathcurveto{\pgfqpoint{1.453563in}{0.878185in}}{\pgfqpoint{1.457953in}{0.888784in}}{\pgfqpoint{1.457953in}{0.899835in}}%
\pgfpathcurveto{\pgfqpoint{1.457953in}{0.910885in}}{\pgfqpoint{1.453563in}{0.921484in}}{\pgfqpoint{1.445750in}{0.929297in}}%
\pgfpathcurveto{\pgfqpoint{1.437936in}{0.937111in}}{\pgfqpoint{1.427337in}{0.941501in}}{\pgfqpoint{1.416287in}{0.941501in}}%
\pgfpathcurveto{\pgfqpoint{1.405237in}{0.941501in}}{\pgfqpoint{1.394638in}{0.937111in}}{\pgfqpoint{1.386824in}{0.929297in}}%
\pgfpathcurveto{\pgfqpoint{1.379010in}{0.921484in}}{\pgfqpoint{1.374620in}{0.910885in}}{\pgfqpoint{1.374620in}{0.899835in}}%
\pgfpathcurveto{\pgfqpoint{1.374620in}{0.888784in}}{\pgfqpoint{1.379010in}{0.878185in}}{\pgfqpoint{1.386824in}{0.870372in}}%
\pgfpathcurveto{\pgfqpoint{1.394638in}{0.862558in}}{\pgfqpoint{1.405237in}{0.858168in}}{\pgfqpoint{1.416287in}{0.858168in}}%
\pgfpathclose%
\pgfusepath{stroke,fill}%
\end{pgfscope}%
\begin{pgfscope}%
\pgfpathrectangle{\pgfqpoint{0.787074in}{0.548769in}}{\pgfqpoint{5.062926in}{3.102590in}}%
\pgfusepath{clip}%
\pgfsetbuttcap%
\pgfsetroundjoin%
\definecolor{currentfill}{rgb}{1.000000,0.498039,0.054902}%
\pgfsetfillcolor{currentfill}%
\pgfsetlinewidth{1.003750pt}%
\definecolor{currentstroke}{rgb}{1.000000,0.498039,0.054902}%
\pgfsetstrokecolor{currentstroke}%
\pgfsetdash{}{0pt}%
\pgfpathmoveto{\pgfqpoint{3.944967in}{2.478569in}}%
\pgfpathcurveto{\pgfqpoint{3.956017in}{2.478569in}}{\pgfqpoint{3.966616in}{2.482959in}}{\pgfqpoint{3.974429in}{2.490773in}}%
\pgfpathcurveto{\pgfqpoint{3.982243in}{2.498586in}}{\pgfqpoint{3.986633in}{2.509185in}}{\pgfqpoint{3.986633in}{2.520235in}}%
\pgfpathcurveto{\pgfqpoint{3.986633in}{2.531286in}}{\pgfqpoint{3.982243in}{2.541885in}}{\pgfqpoint{3.974429in}{2.549698in}}%
\pgfpathcurveto{\pgfqpoint{3.966616in}{2.557512in}}{\pgfqpoint{3.956017in}{2.561902in}}{\pgfqpoint{3.944967in}{2.561902in}}%
\pgfpathcurveto{\pgfqpoint{3.933917in}{2.561902in}}{\pgfqpoint{3.923317in}{2.557512in}}{\pgfqpoint{3.915504in}{2.549698in}}%
\pgfpathcurveto{\pgfqpoint{3.907690in}{2.541885in}}{\pgfqpoint{3.903300in}{2.531286in}}{\pgfqpoint{3.903300in}{2.520235in}}%
\pgfpathcurveto{\pgfqpoint{3.903300in}{2.509185in}}{\pgfqpoint{3.907690in}{2.498586in}}{\pgfqpoint{3.915504in}{2.490773in}}%
\pgfpathcurveto{\pgfqpoint{3.923317in}{2.482959in}}{\pgfqpoint{3.933917in}{2.478569in}}{\pgfqpoint{3.944967in}{2.478569in}}%
\pgfpathclose%
\pgfusepath{stroke,fill}%
\end{pgfscope}%
\begin{pgfscope}%
\pgfpathrectangle{\pgfqpoint{0.787074in}{0.548769in}}{\pgfqpoint{5.062926in}{3.102590in}}%
\pgfusepath{clip}%
\pgfsetbuttcap%
\pgfsetroundjoin%
\definecolor{currentfill}{rgb}{1.000000,0.498039,0.054902}%
\pgfsetfillcolor{currentfill}%
\pgfsetlinewidth{1.003750pt}%
\definecolor{currentstroke}{rgb}{1.000000,0.498039,0.054902}%
\pgfsetstrokecolor{currentstroke}%
\pgfsetdash{}{0pt}%
\pgfpathmoveto{\pgfqpoint{2.660686in}{1.627903in}}%
\pgfpathcurveto{\pgfqpoint{2.671736in}{1.627903in}}{\pgfqpoint{2.682335in}{1.632294in}}{\pgfqpoint{2.690149in}{1.640107in}}%
\pgfpathcurveto{\pgfqpoint{2.697963in}{1.647921in}}{\pgfqpoint{2.702353in}{1.658520in}}{\pgfqpoint{2.702353in}{1.669570in}}%
\pgfpathcurveto{\pgfqpoint{2.702353in}{1.680620in}}{\pgfqpoint{2.697963in}{1.691219in}}{\pgfqpoint{2.690149in}{1.699033in}}%
\pgfpathcurveto{\pgfqpoint{2.682335in}{1.706846in}}{\pgfqpoint{2.671736in}{1.711237in}}{\pgfqpoint{2.660686in}{1.711237in}}%
\pgfpathcurveto{\pgfqpoint{2.649636in}{1.711237in}}{\pgfqpoint{2.639037in}{1.706846in}}{\pgfqpoint{2.631223in}{1.699033in}}%
\pgfpathcurveto{\pgfqpoint{2.623410in}{1.691219in}}{\pgfqpoint{2.619020in}{1.680620in}}{\pgfqpoint{2.619020in}{1.669570in}}%
\pgfpathcurveto{\pgfqpoint{2.619020in}{1.658520in}}{\pgfqpoint{2.623410in}{1.647921in}}{\pgfqpoint{2.631223in}{1.640107in}}%
\pgfpathcurveto{\pgfqpoint{2.639037in}{1.632294in}}{\pgfqpoint{2.649636in}{1.627903in}}{\pgfqpoint{2.660686in}{1.627903in}}%
\pgfpathclose%
\pgfusepath{stroke,fill}%
\end{pgfscope}%
\begin{pgfscope}%
\pgfpathrectangle{\pgfqpoint{0.787074in}{0.548769in}}{\pgfqpoint{5.062926in}{3.102590in}}%
\pgfusepath{clip}%
\pgfsetbuttcap%
\pgfsetroundjoin%
\definecolor{currentfill}{rgb}{0.121569,0.466667,0.705882}%
\pgfsetfillcolor{currentfill}%
\pgfsetlinewidth{1.003750pt}%
\definecolor{currentstroke}{rgb}{0.121569,0.466667,0.705882}%
\pgfsetstrokecolor{currentstroke}%
\pgfsetdash{}{0pt}%
\pgfpathmoveto{\pgfqpoint{1.017224in}{0.648140in}}%
\pgfpathcurveto{\pgfqpoint{1.028275in}{0.648140in}}{\pgfqpoint{1.038874in}{0.652531in}}{\pgfqpoint{1.046687in}{0.660344in}}%
\pgfpathcurveto{\pgfqpoint{1.054501in}{0.668158in}}{\pgfqpoint{1.058891in}{0.678757in}}{\pgfqpoint{1.058891in}{0.689807in}}%
\pgfpathcurveto{\pgfqpoint{1.058891in}{0.700857in}}{\pgfqpoint{1.054501in}{0.711456in}}{\pgfqpoint{1.046687in}{0.719270in}}%
\pgfpathcurveto{\pgfqpoint{1.038874in}{0.727084in}}{\pgfqpoint{1.028275in}{0.731474in}}{\pgfqpoint{1.017224in}{0.731474in}}%
\pgfpathcurveto{\pgfqpoint{1.006174in}{0.731474in}}{\pgfqpoint{0.995575in}{0.727084in}}{\pgfqpoint{0.987762in}{0.719270in}}%
\pgfpathcurveto{\pgfqpoint{0.979948in}{0.711456in}}{\pgfqpoint{0.975558in}{0.700857in}}{\pgfqpoint{0.975558in}{0.689807in}}%
\pgfpathcurveto{\pgfqpoint{0.975558in}{0.678757in}}{\pgfqpoint{0.979948in}{0.668158in}}{\pgfqpoint{0.987762in}{0.660344in}}%
\pgfpathcurveto{\pgfqpoint{0.995575in}{0.652531in}}{\pgfqpoint{1.006174in}{0.648140in}}{\pgfqpoint{1.017224in}{0.648140in}}%
\pgfpathclose%
\pgfusepath{stroke,fill}%
\end{pgfscope}%
\begin{pgfscope}%
\pgfpathrectangle{\pgfqpoint{0.787074in}{0.548769in}}{\pgfqpoint{5.062926in}{3.102590in}}%
\pgfusepath{clip}%
\pgfsetbuttcap%
\pgfsetroundjoin%
\definecolor{currentfill}{rgb}{0.121569,0.466667,0.705882}%
\pgfsetfillcolor{currentfill}%
\pgfsetlinewidth{1.003750pt}%
\definecolor{currentstroke}{rgb}{0.121569,0.466667,0.705882}%
\pgfsetstrokecolor{currentstroke}%
\pgfsetdash{}{0pt}%
\pgfpathmoveto{\pgfqpoint{1.310693in}{0.786141in}}%
\pgfpathcurveto{\pgfqpoint{1.321743in}{0.786141in}}{\pgfqpoint{1.332342in}{0.790531in}}{\pgfqpoint{1.340156in}{0.798345in}}%
\pgfpathcurveto{\pgfqpoint{1.347969in}{0.806159in}}{\pgfqpoint{1.352360in}{0.816758in}}{\pgfqpoint{1.352360in}{0.827808in}}%
\pgfpathcurveto{\pgfqpoint{1.352360in}{0.838858in}}{\pgfqpoint{1.347969in}{0.849457in}}{\pgfqpoint{1.340156in}{0.857271in}}%
\pgfpathcurveto{\pgfqpoint{1.332342in}{0.865084in}}{\pgfqpoint{1.321743in}{0.869474in}}{\pgfqpoint{1.310693in}{0.869474in}}%
\pgfpathcurveto{\pgfqpoint{1.299643in}{0.869474in}}{\pgfqpoint{1.289044in}{0.865084in}}{\pgfqpoint{1.281230in}{0.857271in}}%
\pgfpathcurveto{\pgfqpoint{1.273416in}{0.849457in}}{\pgfqpoint{1.269026in}{0.838858in}}{\pgfqpoint{1.269026in}{0.827808in}}%
\pgfpathcurveto{\pgfqpoint{1.269026in}{0.816758in}}{\pgfqpoint{1.273416in}{0.806159in}}{\pgfqpoint{1.281230in}{0.798345in}}%
\pgfpathcurveto{\pgfqpoint{1.289044in}{0.790531in}}{\pgfqpoint{1.299643in}{0.786141in}}{\pgfqpoint{1.310693in}{0.786141in}}%
\pgfpathclose%
\pgfusepath{stroke,fill}%
\end{pgfscope}%
\begin{pgfscope}%
\pgfpathrectangle{\pgfqpoint{0.787074in}{0.548769in}}{\pgfqpoint{5.062926in}{3.102590in}}%
\pgfusepath{clip}%
\pgfsetbuttcap%
\pgfsetroundjoin%
\definecolor{currentfill}{rgb}{1.000000,0.498039,0.054902}%
\pgfsetfillcolor{currentfill}%
\pgfsetlinewidth{1.003750pt}%
\definecolor{currentstroke}{rgb}{1.000000,0.498039,0.054902}%
\pgfsetstrokecolor{currentstroke}%
\pgfsetdash{}{0pt}%
\pgfpathmoveto{\pgfqpoint{4.381904in}{2.612758in}}%
\pgfpathcurveto{\pgfqpoint{4.392954in}{2.612758in}}{\pgfqpoint{4.403553in}{2.617148in}}{\pgfqpoint{4.411367in}{2.624962in}}%
\pgfpathcurveto{\pgfqpoint{4.419181in}{2.632775in}}{\pgfqpoint{4.423571in}{2.643374in}}{\pgfqpoint{4.423571in}{2.654424in}}%
\pgfpathcurveto{\pgfqpoint{4.423571in}{2.665474in}}{\pgfqpoint{4.419181in}{2.676074in}}{\pgfqpoint{4.411367in}{2.683887in}}%
\pgfpathcurveto{\pgfqpoint{4.403553in}{2.691701in}}{\pgfqpoint{4.392954in}{2.696091in}}{\pgfqpoint{4.381904in}{2.696091in}}%
\pgfpathcurveto{\pgfqpoint{4.370854in}{2.696091in}}{\pgfqpoint{4.360255in}{2.691701in}}{\pgfqpoint{4.352442in}{2.683887in}}%
\pgfpathcurveto{\pgfqpoint{4.344628in}{2.676074in}}{\pgfqpoint{4.340238in}{2.665474in}}{\pgfqpoint{4.340238in}{2.654424in}}%
\pgfpathcurveto{\pgfqpoint{4.340238in}{2.643374in}}{\pgfqpoint{4.344628in}{2.632775in}}{\pgfqpoint{4.352442in}{2.624962in}}%
\pgfpathcurveto{\pgfqpoint{4.360255in}{2.617148in}}{\pgfqpoint{4.370854in}{2.612758in}}{\pgfqpoint{4.381904in}{2.612758in}}%
\pgfpathclose%
\pgfusepath{stroke,fill}%
\end{pgfscope}%
\begin{pgfscope}%
\pgfpathrectangle{\pgfqpoint{0.787074in}{0.548769in}}{\pgfqpoint{5.062926in}{3.102590in}}%
\pgfusepath{clip}%
\pgfsetbuttcap%
\pgfsetroundjoin%
\definecolor{currentfill}{rgb}{0.121569,0.466667,0.705882}%
\pgfsetfillcolor{currentfill}%
\pgfsetlinewidth{1.003750pt}%
\definecolor{currentstroke}{rgb}{0.121569,0.466667,0.705882}%
\pgfsetstrokecolor{currentstroke}%
\pgfsetdash{}{0pt}%
\pgfpathmoveto{\pgfqpoint{1.017207in}{0.648129in}}%
\pgfpathcurveto{\pgfqpoint{1.028257in}{0.648129in}}{\pgfqpoint{1.038856in}{0.652519in}}{\pgfqpoint{1.046670in}{0.660333in}}%
\pgfpathcurveto{\pgfqpoint{1.054483in}{0.668146in}}{\pgfqpoint{1.058874in}{0.678745in}}{\pgfqpoint{1.058874in}{0.689796in}}%
\pgfpathcurveto{\pgfqpoint{1.058874in}{0.700846in}}{\pgfqpoint{1.054483in}{0.711445in}}{\pgfqpoint{1.046670in}{0.719258in}}%
\pgfpathcurveto{\pgfqpoint{1.038856in}{0.727072in}}{\pgfqpoint{1.028257in}{0.731462in}}{\pgfqpoint{1.017207in}{0.731462in}}%
\pgfpathcurveto{\pgfqpoint{1.006157in}{0.731462in}}{\pgfqpoint{0.995558in}{0.727072in}}{\pgfqpoint{0.987744in}{0.719258in}}%
\pgfpathcurveto{\pgfqpoint{0.979930in}{0.711445in}}{\pgfqpoint{0.975540in}{0.700846in}}{\pgfqpoint{0.975540in}{0.689796in}}%
\pgfpathcurveto{\pgfqpoint{0.975540in}{0.678745in}}{\pgfqpoint{0.979930in}{0.668146in}}{\pgfqpoint{0.987744in}{0.660333in}}%
\pgfpathcurveto{\pgfqpoint{0.995558in}{0.652519in}}{\pgfqpoint{1.006157in}{0.648129in}}{\pgfqpoint{1.017207in}{0.648129in}}%
\pgfpathclose%
\pgfusepath{stroke,fill}%
\end{pgfscope}%
\begin{pgfscope}%
\pgfpathrectangle{\pgfqpoint{0.787074in}{0.548769in}}{\pgfqpoint{5.062926in}{3.102590in}}%
\pgfusepath{clip}%
\pgfsetbuttcap%
\pgfsetroundjoin%
\definecolor{currentfill}{rgb}{1.000000,0.498039,0.054902}%
\pgfsetfillcolor{currentfill}%
\pgfsetlinewidth{1.003750pt}%
\definecolor{currentstroke}{rgb}{1.000000,0.498039,0.054902}%
\pgfsetstrokecolor{currentstroke}%
\pgfsetdash{}{0pt}%
\pgfpathmoveto{\pgfqpoint{3.507735in}{2.514801in}}%
\pgfpathcurveto{\pgfqpoint{3.518785in}{2.514801in}}{\pgfqpoint{3.529384in}{2.519191in}}{\pgfqpoint{3.537198in}{2.527004in}}%
\pgfpathcurveto{\pgfqpoint{3.545011in}{2.534818in}}{\pgfqpoint{3.549402in}{2.545417in}}{\pgfqpoint{3.549402in}{2.556467in}}%
\pgfpathcurveto{\pgfqpoint{3.549402in}{2.567517in}}{\pgfqpoint{3.545011in}{2.578116in}}{\pgfqpoint{3.537198in}{2.585930in}}%
\pgfpathcurveto{\pgfqpoint{3.529384in}{2.593744in}}{\pgfqpoint{3.518785in}{2.598134in}}{\pgfqpoint{3.507735in}{2.598134in}}%
\pgfpathcurveto{\pgfqpoint{3.496685in}{2.598134in}}{\pgfqpoint{3.486086in}{2.593744in}}{\pgfqpoint{3.478272in}{2.585930in}}%
\pgfpathcurveto{\pgfqpoint{3.470459in}{2.578116in}}{\pgfqpoint{3.466068in}{2.567517in}}{\pgfqpoint{3.466068in}{2.556467in}}%
\pgfpathcurveto{\pgfqpoint{3.466068in}{2.545417in}}{\pgfqpoint{3.470459in}{2.534818in}}{\pgfqpoint{3.478272in}{2.527004in}}%
\pgfpathcurveto{\pgfqpoint{3.486086in}{2.519191in}}{\pgfqpoint{3.496685in}{2.514801in}}{\pgfqpoint{3.507735in}{2.514801in}}%
\pgfpathclose%
\pgfusepath{stroke,fill}%
\end{pgfscope}%
\begin{pgfscope}%
\pgfpathrectangle{\pgfqpoint{0.787074in}{0.548769in}}{\pgfqpoint{5.062926in}{3.102590in}}%
\pgfusepath{clip}%
\pgfsetbuttcap%
\pgfsetroundjoin%
\definecolor{currentfill}{rgb}{1.000000,0.498039,0.054902}%
\pgfsetfillcolor{currentfill}%
\pgfsetlinewidth{1.003750pt}%
\definecolor{currentstroke}{rgb}{1.000000,0.498039,0.054902}%
\pgfsetstrokecolor{currentstroke}%
\pgfsetdash{}{0pt}%
\pgfpathmoveto{\pgfqpoint{4.403354in}{2.647988in}}%
\pgfpathcurveto{\pgfqpoint{4.414404in}{2.647988in}}{\pgfqpoint{4.425003in}{2.652378in}}{\pgfqpoint{4.432817in}{2.660191in}}%
\pgfpathcurveto{\pgfqpoint{4.440631in}{2.668005in}}{\pgfqpoint{4.445021in}{2.678604in}}{\pgfqpoint{4.445021in}{2.689654in}}%
\pgfpathcurveto{\pgfqpoint{4.445021in}{2.700704in}}{\pgfqpoint{4.440631in}{2.711303in}}{\pgfqpoint{4.432817in}{2.719117in}}%
\pgfpathcurveto{\pgfqpoint{4.425003in}{2.726931in}}{\pgfqpoint{4.414404in}{2.731321in}}{\pgfqpoint{4.403354in}{2.731321in}}%
\pgfpathcurveto{\pgfqpoint{4.392304in}{2.731321in}}{\pgfqpoint{4.381705in}{2.726931in}}{\pgfqpoint{4.373891in}{2.719117in}}%
\pgfpathcurveto{\pgfqpoint{4.366078in}{2.711303in}}{\pgfqpoint{4.361687in}{2.700704in}}{\pgfqpoint{4.361687in}{2.689654in}}%
\pgfpathcurveto{\pgfqpoint{4.361687in}{2.678604in}}{\pgfqpoint{4.366078in}{2.668005in}}{\pgfqpoint{4.373891in}{2.660191in}}%
\pgfpathcurveto{\pgfqpoint{4.381705in}{2.652378in}}{\pgfqpoint{4.392304in}{2.647988in}}{\pgfqpoint{4.403354in}{2.647988in}}%
\pgfpathclose%
\pgfusepath{stroke,fill}%
\end{pgfscope}%
\begin{pgfscope}%
\pgfpathrectangle{\pgfqpoint{0.787074in}{0.548769in}}{\pgfqpoint{5.062926in}{3.102590in}}%
\pgfusepath{clip}%
\pgfsetbuttcap%
\pgfsetroundjoin%
\definecolor{currentfill}{rgb}{0.121569,0.466667,0.705882}%
\pgfsetfillcolor{currentfill}%
\pgfsetlinewidth{1.003750pt}%
\definecolor{currentstroke}{rgb}{0.121569,0.466667,0.705882}%
\pgfsetstrokecolor{currentstroke}%
\pgfsetdash{}{0pt}%
\pgfpathmoveto{\pgfqpoint{1.025613in}{0.652182in}}%
\pgfpathcurveto{\pgfqpoint{1.036663in}{0.652182in}}{\pgfqpoint{1.047262in}{0.656572in}}{\pgfqpoint{1.055076in}{0.664386in}}%
\pgfpathcurveto{\pgfqpoint{1.062889in}{0.672199in}}{\pgfqpoint{1.067280in}{0.682798in}}{\pgfqpoint{1.067280in}{0.693848in}}%
\pgfpathcurveto{\pgfqpoint{1.067280in}{0.704899in}}{\pgfqpoint{1.062889in}{0.715498in}}{\pgfqpoint{1.055076in}{0.723311in}}%
\pgfpathcurveto{\pgfqpoint{1.047262in}{0.731125in}}{\pgfqpoint{1.036663in}{0.735515in}}{\pgfqpoint{1.025613in}{0.735515in}}%
\pgfpathcurveto{\pgfqpoint{1.014563in}{0.735515in}}{\pgfqpoint{1.003964in}{0.731125in}}{\pgfqpoint{0.996150in}{0.723311in}}%
\pgfpathcurveto{\pgfqpoint{0.988337in}{0.715498in}}{\pgfqpoint{0.983946in}{0.704899in}}{\pgfqpoint{0.983946in}{0.693848in}}%
\pgfpathcurveto{\pgfqpoint{0.983946in}{0.682798in}}{\pgfqpoint{0.988337in}{0.672199in}}{\pgfqpoint{0.996150in}{0.664386in}}%
\pgfpathcurveto{\pgfqpoint{1.003964in}{0.656572in}}{\pgfqpoint{1.014563in}{0.652182in}}{\pgfqpoint{1.025613in}{0.652182in}}%
\pgfpathclose%
\pgfusepath{stroke,fill}%
\end{pgfscope}%
\begin{pgfscope}%
\pgfpathrectangle{\pgfqpoint{0.787074in}{0.548769in}}{\pgfqpoint{5.062926in}{3.102590in}}%
\pgfusepath{clip}%
\pgfsetbuttcap%
\pgfsetroundjoin%
\definecolor{currentfill}{rgb}{0.121569,0.466667,0.705882}%
\pgfsetfillcolor{currentfill}%
\pgfsetlinewidth{1.003750pt}%
\definecolor{currentstroke}{rgb}{0.121569,0.466667,0.705882}%
\pgfsetstrokecolor{currentstroke}%
\pgfsetdash{}{0pt}%
\pgfpathmoveto{\pgfqpoint{1.017268in}{0.648180in}}%
\pgfpathcurveto{\pgfqpoint{1.028318in}{0.648180in}}{\pgfqpoint{1.038917in}{0.652570in}}{\pgfqpoint{1.046731in}{0.660384in}}%
\pgfpathcurveto{\pgfqpoint{1.054544in}{0.668197in}}{\pgfqpoint{1.058935in}{0.678797in}}{\pgfqpoint{1.058935in}{0.689847in}}%
\pgfpathcurveto{\pgfqpoint{1.058935in}{0.700897in}}{\pgfqpoint{1.054544in}{0.711496in}}{\pgfqpoint{1.046731in}{0.719309in}}%
\pgfpathcurveto{\pgfqpoint{1.038917in}{0.727123in}}{\pgfqpoint{1.028318in}{0.731513in}}{\pgfqpoint{1.017268in}{0.731513in}}%
\pgfpathcurveto{\pgfqpoint{1.006218in}{0.731513in}}{\pgfqpoint{0.995619in}{0.727123in}}{\pgfqpoint{0.987805in}{0.719309in}}%
\pgfpathcurveto{\pgfqpoint{0.979991in}{0.711496in}}{\pgfqpoint{0.975601in}{0.700897in}}{\pgfqpoint{0.975601in}{0.689847in}}%
\pgfpathcurveto{\pgfqpoint{0.975601in}{0.678797in}}{\pgfqpoint{0.979991in}{0.668197in}}{\pgfqpoint{0.987805in}{0.660384in}}%
\pgfpathcurveto{\pgfqpoint{0.995619in}{0.652570in}}{\pgfqpoint{1.006218in}{0.648180in}}{\pgfqpoint{1.017268in}{0.648180in}}%
\pgfpathclose%
\pgfusepath{stroke,fill}%
\end{pgfscope}%
\begin{pgfscope}%
\pgfpathrectangle{\pgfqpoint{0.787074in}{0.548769in}}{\pgfqpoint{5.062926in}{3.102590in}}%
\pgfusepath{clip}%
\pgfsetbuttcap%
\pgfsetroundjoin%
\definecolor{currentfill}{rgb}{0.121569,0.466667,0.705882}%
\pgfsetfillcolor{currentfill}%
\pgfsetlinewidth{1.003750pt}%
\definecolor{currentstroke}{rgb}{0.121569,0.466667,0.705882}%
\pgfsetstrokecolor{currentstroke}%
\pgfsetdash{}{0pt}%
\pgfpathmoveto{\pgfqpoint{1.017252in}{0.648153in}}%
\pgfpathcurveto{\pgfqpoint{1.028302in}{0.648153in}}{\pgfqpoint{1.038901in}{0.652543in}}{\pgfqpoint{1.046715in}{0.660356in}}%
\pgfpathcurveto{\pgfqpoint{1.054529in}{0.668170in}}{\pgfqpoint{1.058919in}{0.678769in}}{\pgfqpoint{1.058919in}{0.689819in}}%
\pgfpathcurveto{\pgfqpoint{1.058919in}{0.700869in}}{\pgfqpoint{1.054529in}{0.711468in}}{\pgfqpoint{1.046715in}{0.719282in}}%
\pgfpathcurveto{\pgfqpoint{1.038901in}{0.727096in}}{\pgfqpoint{1.028302in}{0.731486in}}{\pgfqpoint{1.017252in}{0.731486in}}%
\pgfpathcurveto{\pgfqpoint{1.006202in}{0.731486in}}{\pgfqpoint{0.995603in}{0.727096in}}{\pgfqpoint{0.987789in}{0.719282in}}%
\pgfpathcurveto{\pgfqpoint{0.979976in}{0.711468in}}{\pgfqpoint{0.975586in}{0.700869in}}{\pgfqpoint{0.975586in}{0.689819in}}%
\pgfpathcurveto{\pgfqpoint{0.975586in}{0.678769in}}{\pgfqpoint{0.979976in}{0.668170in}}{\pgfqpoint{0.987789in}{0.660356in}}%
\pgfpathcurveto{\pgfqpoint{0.995603in}{0.652543in}}{\pgfqpoint{1.006202in}{0.648153in}}{\pgfqpoint{1.017252in}{0.648153in}}%
\pgfpathclose%
\pgfusepath{stroke,fill}%
\end{pgfscope}%
\begin{pgfscope}%
\pgfpathrectangle{\pgfqpoint{0.787074in}{0.548769in}}{\pgfqpoint{5.062926in}{3.102590in}}%
\pgfusepath{clip}%
\pgfsetbuttcap%
\pgfsetroundjoin%
\definecolor{currentfill}{rgb}{0.121569,0.466667,0.705882}%
\pgfsetfillcolor{currentfill}%
\pgfsetlinewidth{1.003750pt}%
\definecolor{currentstroke}{rgb}{0.121569,0.466667,0.705882}%
\pgfsetstrokecolor{currentstroke}%
\pgfsetdash{}{0pt}%
\pgfpathmoveto{\pgfqpoint{1.017821in}{0.648454in}}%
\pgfpathcurveto{\pgfqpoint{1.028871in}{0.648454in}}{\pgfqpoint{1.039470in}{0.652844in}}{\pgfqpoint{1.047284in}{0.660658in}}%
\pgfpathcurveto{\pgfqpoint{1.055097in}{0.668471in}}{\pgfqpoint{1.059488in}{0.679070in}}{\pgfqpoint{1.059488in}{0.690121in}}%
\pgfpathcurveto{\pgfqpoint{1.059488in}{0.701171in}}{\pgfqpoint{1.055097in}{0.711770in}}{\pgfqpoint{1.047284in}{0.719583in}}%
\pgfpathcurveto{\pgfqpoint{1.039470in}{0.727397in}}{\pgfqpoint{1.028871in}{0.731787in}}{\pgfqpoint{1.017821in}{0.731787in}}%
\pgfpathcurveto{\pgfqpoint{1.006771in}{0.731787in}}{\pgfqpoint{0.996172in}{0.727397in}}{\pgfqpoint{0.988358in}{0.719583in}}%
\pgfpathcurveto{\pgfqpoint{0.980545in}{0.711770in}}{\pgfqpoint{0.976154in}{0.701171in}}{\pgfqpoint{0.976154in}{0.690121in}}%
\pgfpathcurveto{\pgfqpoint{0.976154in}{0.679070in}}{\pgfqpoint{0.980545in}{0.668471in}}{\pgfqpoint{0.988358in}{0.660658in}}%
\pgfpathcurveto{\pgfqpoint{0.996172in}{0.652844in}}{\pgfqpoint{1.006771in}{0.648454in}}{\pgfqpoint{1.017821in}{0.648454in}}%
\pgfpathclose%
\pgfusepath{stroke,fill}%
\end{pgfscope}%
\begin{pgfscope}%
\pgfpathrectangle{\pgfqpoint{0.787074in}{0.548769in}}{\pgfqpoint{5.062926in}{3.102590in}}%
\pgfusepath{clip}%
\pgfsetbuttcap%
\pgfsetroundjoin%
\definecolor{currentfill}{rgb}{0.121569,0.466667,0.705882}%
\pgfsetfillcolor{currentfill}%
\pgfsetlinewidth{1.003750pt}%
\definecolor{currentstroke}{rgb}{0.121569,0.466667,0.705882}%
\pgfsetstrokecolor{currentstroke}%
\pgfsetdash{}{0pt}%
\pgfpathmoveto{\pgfqpoint{1.017326in}{0.648199in}}%
\pgfpathcurveto{\pgfqpoint{1.028376in}{0.648199in}}{\pgfqpoint{1.038975in}{0.652589in}}{\pgfqpoint{1.046789in}{0.660403in}}%
\pgfpathcurveto{\pgfqpoint{1.054603in}{0.668217in}}{\pgfqpoint{1.058993in}{0.678816in}}{\pgfqpoint{1.058993in}{0.689866in}}%
\pgfpathcurveto{\pgfqpoint{1.058993in}{0.700916in}}{\pgfqpoint{1.054603in}{0.711515in}}{\pgfqpoint{1.046789in}{0.719329in}}%
\pgfpathcurveto{\pgfqpoint{1.038975in}{0.727142in}}{\pgfqpoint{1.028376in}{0.731533in}}{\pgfqpoint{1.017326in}{0.731533in}}%
\pgfpathcurveto{\pgfqpoint{1.006276in}{0.731533in}}{\pgfqpoint{0.995677in}{0.727142in}}{\pgfqpoint{0.987863in}{0.719329in}}%
\pgfpathcurveto{\pgfqpoint{0.980050in}{0.711515in}}{\pgfqpoint{0.975660in}{0.700916in}}{\pgfqpoint{0.975660in}{0.689866in}}%
\pgfpathcurveto{\pgfqpoint{0.975660in}{0.678816in}}{\pgfqpoint{0.980050in}{0.668217in}}{\pgfqpoint{0.987863in}{0.660403in}}%
\pgfpathcurveto{\pgfqpoint{0.995677in}{0.652589in}}{\pgfqpoint{1.006276in}{0.648199in}}{\pgfqpoint{1.017326in}{0.648199in}}%
\pgfpathclose%
\pgfusepath{stroke,fill}%
\end{pgfscope}%
\begin{pgfscope}%
\pgfpathrectangle{\pgfqpoint{0.787074in}{0.548769in}}{\pgfqpoint{5.062926in}{3.102590in}}%
\pgfusepath{clip}%
\pgfsetbuttcap%
\pgfsetroundjoin%
\definecolor{currentfill}{rgb}{0.121569,0.466667,0.705882}%
\pgfsetfillcolor{currentfill}%
\pgfsetlinewidth{1.003750pt}%
\definecolor{currentstroke}{rgb}{0.121569,0.466667,0.705882}%
\pgfsetstrokecolor{currentstroke}%
\pgfsetdash{}{0pt}%
\pgfpathmoveto{\pgfqpoint{1.017236in}{0.648149in}}%
\pgfpathcurveto{\pgfqpoint{1.028286in}{0.648149in}}{\pgfqpoint{1.038885in}{0.652540in}}{\pgfqpoint{1.046699in}{0.660353in}}%
\pgfpathcurveto{\pgfqpoint{1.054512in}{0.668167in}}{\pgfqpoint{1.058903in}{0.678766in}}{\pgfqpoint{1.058903in}{0.689816in}}%
\pgfpathcurveto{\pgfqpoint{1.058903in}{0.700866in}}{\pgfqpoint{1.054512in}{0.711465in}}{\pgfqpoint{1.046699in}{0.719279in}}%
\pgfpathcurveto{\pgfqpoint{1.038885in}{0.727092in}}{\pgfqpoint{1.028286in}{0.731483in}}{\pgfqpoint{1.017236in}{0.731483in}}%
\pgfpathcurveto{\pgfqpoint{1.006186in}{0.731483in}}{\pgfqpoint{0.995587in}{0.727092in}}{\pgfqpoint{0.987773in}{0.719279in}}%
\pgfpathcurveto{\pgfqpoint{0.979960in}{0.711465in}}{\pgfqpoint{0.975569in}{0.700866in}}{\pgfqpoint{0.975569in}{0.689816in}}%
\pgfpathcurveto{\pgfqpoint{0.975569in}{0.678766in}}{\pgfqpoint{0.979960in}{0.668167in}}{\pgfqpoint{0.987773in}{0.660353in}}%
\pgfpathcurveto{\pgfqpoint{0.995587in}{0.652540in}}{\pgfqpoint{1.006186in}{0.648149in}}{\pgfqpoint{1.017236in}{0.648149in}}%
\pgfpathclose%
\pgfusepath{stroke,fill}%
\end{pgfscope}%
\begin{pgfscope}%
\pgfpathrectangle{\pgfqpoint{0.787074in}{0.548769in}}{\pgfqpoint{5.062926in}{3.102590in}}%
\pgfusepath{clip}%
\pgfsetbuttcap%
\pgfsetroundjoin%
\definecolor{currentfill}{rgb}{0.121569,0.466667,0.705882}%
\pgfsetfillcolor{currentfill}%
\pgfsetlinewidth{1.003750pt}%
\definecolor{currentstroke}{rgb}{0.121569,0.466667,0.705882}%
\pgfsetstrokecolor{currentstroke}%
\pgfsetdash{}{0pt}%
\pgfpathmoveto{\pgfqpoint{4.367997in}{3.153720in}}%
\pgfpathcurveto{\pgfqpoint{4.379047in}{3.153720in}}{\pgfqpoint{4.389646in}{3.158111in}}{\pgfqpoint{4.397460in}{3.165924in}}%
\pgfpathcurveto{\pgfqpoint{4.405273in}{3.173738in}}{\pgfqpoint{4.409663in}{3.184337in}}{\pgfqpoint{4.409663in}{3.195387in}}%
\pgfpathcurveto{\pgfqpoint{4.409663in}{3.206437in}}{\pgfqpoint{4.405273in}{3.217036in}}{\pgfqpoint{4.397460in}{3.224850in}}%
\pgfpathcurveto{\pgfqpoint{4.389646in}{3.232663in}}{\pgfqpoint{4.379047in}{3.237054in}}{\pgfqpoint{4.367997in}{3.237054in}}%
\pgfpathcurveto{\pgfqpoint{4.356947in}{3.237054in}}{\pgfqpoint{4.346348in}{3.232663in}}{\pgfqpoint{4.338534in}{3.224850in}}%
\pgfpathcurveto{\pgfqpoint{4.330720in}{3.217036in}}{\pgfqpoint{4.326330in}{3.206437in}}{\pgfqpoint{4.326330in}{3.195387in}}%
\pgfpathcurveto{\pgfqpoint{4.326330in}{3.184337in}}{\pgfqpoint{4.330720in}{3.173738in}}{\pgfqpoint{4.338534in}{3.165924in}}%
\pgfpathcurveto{\pgfqpoint{4.346348in}{3.158111in}}{\pgfqpoint{4.356947in}{3.153720in}}{\pgfqpoint{4.367997in}{3.153720in}}%
\pgfpathclose%
\pgfusepath{stroke,fill}%
\end{pgfscope}%
\begin{pgfscope}%
\pgfpathrectangle{\pgfqpoint{0.787074in}{0.548769in}}{\pgfqpoint{5.062926in}{3.102590in}}%
\pgfusepath{clip}%
\pgfsetbuttcap%
\pgfsetroundjoin%
\definecolor{currentfill}{rgb}{1.000000,0.498039,0.054902}%
\pgfsetfillcolor{currentfill}%
\pgfsetlinewidth{1.003750pt}%
\definecolor{currentstroke}{rgb}{1.000000,0.498039,0.054902}%
\pgfsetstrokecolor{currentstroke}%
\pgfsetdash{}{0pt}%
\pgfpathmoveto{\pgfqpoint{3.815742in}{2.857398in}}%
\pgfpathcurveto{\pgfqpoint{3.826792in}{2.857398in}}{\pgfqpoint{3.837392in}{2.861789in}}{\pgfqpoint{3.845205in}{2.869602in}}%
\pgfpathcurveto{\pgfqpoint{3.853019in}{2.877416in}}{\pgfqpoint{3.857409in}{2.888015in}}{\pgfqpoint{3.857409in}{2.899065in}}%
\pgfpathcurveto{\pgfqpoint{3.857409in}{2.910115in}}{\pgfqpoint{3.853019in}{2.920714in}}{\pgfqpoint{3.845205in}{2.928528in}}%
\pgfpathcurveto{\pgfqpoint{3.837392in}{2.936341in}}{\pgfqpoint{3.826792in}{2.940732in}}{\pgfqpoint{3.815742in}{2.940732in}}%
\pgfpathcurveto{\pgfqpoint{3.804692in}{2.940732in}}{\pgfqpoint{3.794093in}{2.936341in}}{\pgfqpoint{3.786280in}{2.928528in}}%
\pgfpathcurveto{\pgfqpoint{3.778466in}{2.920714in}}{\pgfqpoint{3.774076in}{2.910115in}}{\pgfqpoint{3.774076in}{2.899065in}}%
\pgfpathcurveto{\pgfqpoint{3.774076in}{2.888015in}}{\pgfqpoint{3.778466in}{2.877416in}}{\pgfqpoint{3.786280in}{2.869602in}}%
\pgfpathcurveto{\pgfqpoint{3.794093in}{2.861789in}}{\pgfqpoint{3.804692in}{2.857398in}}{\pgfqpoint{3.815742in}{2.857398in}}%
\pgfpathclose%
\pgfusepath{stroke,fill}%
\end{pgfscope}%
\begin{pgfscope}%
\pgfpathrectangle{\pgfqpoint{0.787074in}{0.548769in}}{\pgfqpoint{5.062926in}{3.102590in}}%
\pgfusepath{clip}%
\pgfsetbuttcap%
\pgfsetroundjoin%
\definecolor{currentfill}{rgb}{1.000000,0.498039,0.054902}%
\pgfsetfillcolor{currentfill}%
\pgfsetlinewidth{1.003750pt}%
\definecolor{currentstroke}{rgb}{1.000000,0.498039,0.054902}%
\pgfsetstrokecolor{currentstroke}%
\pgfsetdash{}{0pt}%
\pgfpathmoveto{\pgfqpoint{3.988034in}{3.104611in}}%
\pgfpathcurveto{\pgfqpoint{3.999084in}{3.104611in}}{\pgfqpoint{4.009683in}{3.109002in}}{\pgfqpoint{4.017497in}{3.116815in}}%
\pgfpathcurveto{\pgfqpoint{4.025311in}{3.124629in}}{\pgfqpoint{4.029701in}{3.135228in}}{\pgfqpoint{4.029701in}{3.146278in}}%
\pgfpathcurveto{\pgfqpoint{4.029701in}{3.157328in}}{\pgfqpoint{4.025311in}{3.167927in}}{\pgfqpoint{4.017497in}{3.175741in}}%
\pgfpathcurveto{\pgfqpoint{4.009683in}{3.183554in}}{\pgfqpoint{3.999084in}{3.187945in}}{\pgfqpoint{3.988034in}{3.187945in}}%
\pgfpathcurveto{\pgfqpoint{3.976984in}{3.187945in}}{\pgfqpoint{3.966385in}{3.183554in}}{\pgfqpoint{3.958571in}{3.175741in}}%
\pgfpathcurveto{\pgfqpoint{3.950758in}{3.167927in}}{\pgfqpoint{3.946368in}{3.157328in}}{\pgfqpoint{3.946368in}{3.146278in}}%
\pgfpathcurveto{\pgfqpoint{3.946368in}{3.135228in}}{\pgfqpoint{3.950758in}{3.124629in}}{\pgfqpoint{3.958571in}{3.116815in}}%
\pgfpathcurveto{\pgfqpoint{3.966385in}{3.109002in}}{\pgfqpoint{3.976984in}{3.104611in}}{\pgfqpoint{3.988034in}{3.104611in}}%
\pgfpathclose%
\pgfusepath{stroke,fill}%
\end{pgfscope}%
\begin{pgfscope}%
\pgfpathrectangle{\pgfqpoint{0.787074in}{0.548769in}}{\pgfqpoint{5.062926in}{3.102590in}}%
\pgfusepath{clip}%
\pgfsetbuttcap%
\pgfsetroundjoin%
\definecolor{currentfill}{rgb}{0.121569,0.466667,0.705882}%
\pgfsetfillcolor{currentfill}%
\pgfsetlinewidth{1.003750pt}%
\definecolor{currentstroke}{rgb}{0.121569,0.466667,0.705882}%
\pgfsetstrokecolor{currentstroke}%
\pgfsetdash{}{0pt}%
\pgfpathmoveto{\pgfqpoint{1.017213in}{0.648133in}}%
\pgfpathcurveto{\pgfqpoint{1.028264in}{0.648133in}}{\pgfqpoint{1.038863in}{0.652523in}}{\pgfqpoint{1.046676in}{0.660337in}}%
\pgfpathcurveto{\pgfqpoint{1.054490in}{0.668150in}}{\pgfqpoint{1.058880in}{0.678749in}}{\pgfqpoint{1.058880in}{0.689799in}}%
\pgfpathcurveto{\pgfqpoint{1.058880in}{0.700849in}}{\pgfqpoint{1.054490in}{0.711448in}}{\pgfqpoint{1.046676in}{0.719262in}}%
\pgfpathcurveto{\pgfqpoint{1.038863in}{0.727076in}}{\pgfqpoint{1.028264in}{0.731466in}}{\pgfqpoint{1.017213in}{0.731466in}}%
\pgfpathcurveto{\pgfqpoint{1.006163in}{0.731466in}}{\pgfqpoint{0.995564in}{0.727076in}}{\pgfqpoint{0.987751in}{0.719262in}}%
\pgfpathcurveto{\pgfqpoint{0.979937in}{0.711448in}}{\pgfqpoint{0.975547in}{0.700849in}}{\pgfqpoint{0.975547in}{0.689799in}}%
\pgfpathcurveto{\pgfqpoint{0.975547in}{0.678749in}}{\pgfqpoint{0.979937in}{0.668150in}}{\pgfqpoint{0.987751in}{0.660337in}}%
\pgfpathcurveto{\pgfqpoint{0.995564in}{0.652523in}}{\pgfqpoint{1.006163in}{0.648133in}}{\pgfqpoint{1.017213in}{0.648133in}}%
\pgfpathclose%
\pgfusepath{stroke,fill}%
\end{pgfscope}%
\begin{pgfscope}%
\pgfpathrectangle{\pgfqpoint{0.787074in}{0.548769in}}{\pgfqpoint{5.062926in}{3.102590in}}%
\pgfusepath{clip}%
\pgfsetbuttcap%
\pgfsetroundjoin%
\definecolor{currentfill}{rgb}{0.121569,0.466667,0.705882}%
\pgfsetfillcolor{currentfill}%
\pgfsetlinewidth{1.003750pt}%
\definecolor{currentstroke}{rgb}{0.121569,0.466667,0.705882}%
\pgfsetstrokecolor{currentstroke}%
\pgfsetdash{}{0pt}%
\pgfpathmoveto{\pgfqpoint{1.068040in}{0.679648in}}%
\pgfpathcurveto{\pgfqpoint{1.079090in}{0.679648in}}{\pgfqpoint{1.089689in}{0.684039in}}{\pgfqpoint{1.097503in}{0.691852in}}%
\pgfpathcurveto{\pgfqpoint{1.105317in}{0.699666in}}{\pgfqpoint{1.109707in}{0.710265in}}{\pgfqpoint{1.109707in}{0.721315in}}%
\pgfpathcurveto{\pgfqpoint{1.109707in}{0.732365in}}{\pgfqpoint{1.105317in}{0.742964in}}{\pgfqpoint{1.097503in}{0.750778in}}%
\pgfpathcurveto{\pgfqpoint{1.089689in}{0.758591in}}{\pgfqpoint{1.079090in}{0.762982in}}{\pgfqpoint{1.068040in}{0.762982in}}%
\pgfpathcurveto{\pgfqpoint{1.056990in}{0.762982in}}{\pgfqpoint{1.046391in}{0.758591in}}{\pgfqpoint{1.038577in}{0.750778in}}%
\pgfpathcurveto{\pgfqpoint{1.030764in}{0.742964in}}{\pgfqpoint{1.026374in}{0.732365in}}{\pgfqpoint{1.026374in}{0.721315in}}%
\pgfpathcurveto{\pgfqpoint{1.026374in}{0.710265in}}{\pgfqpoint{1.030764in}{0.699666in}}{\pgfqpoint{1.038577in}{0.691852in}}%
\pgfpathcurveto{\pgfqpoint{1.046391in}{0.684039in}}{\pgfqpoint{1.056990in}{0.679648in}}{\pgfqpoint{1.068040in}{0.679648in}}%
\pgfpathclose%
\pgfusepath{stroke,fill}%
\end{pgfscope}%
\begin{pgfscope}%
\pgfpathrectangle{\pgfqpoint{0.787074in}{0.548769in}}{\pgfqpoint{5.062926in}{3.102590in}}%
\pgfusepath{clip}%
\pgfsetbuttcap%
\pgfsetroundjoin%
\definecolor{currentfill}{rgb}{1.000000,0.498039,0.054902}%
\pgfsetfillcolor{currentfill}%
\pgfsetlinewidth{1.003750pt}%
\definecolor{currentstroke}{rgb}{1.000000,0.498039,0.054902}%
\pgfsetstrokecolor{currentstroke}%
\pgfsetdash{}{0pt}%
\pgfpathmoveto{\pgfqpoint{4.331917in}{2.890450in}}%
\pgfpathcurveto{\pgfqpoint{4.342967in}{2.890450in}}{\pgfqpoint{4.353566in}{2.894841in}}{\pgfqpoint{4.361380in}{2.902654in}}%
\pgfpathcurveto{\pgfqpoint{4.369193in}{2.910468in}}{\pgfqpoint{4.373584in}{2.921067in}}{\pgfqpoint{4.373584in}{2.932117in}}%
\pgfpathcurveto{\pgfqpoint{4.373584in}{2.943167in}}{\pgfqpoint{4.369193in}{2.953766in}}{\pgfqpoint{4.361380in}{2.961580in}}%
\pgfpathcurveto{\pgfqpoint{4.353566in}{2.969393in}}{\pgfqpoint{4.342967in}{2.973784in}}{\pgfqpoint{4.331917in}{2.973784in}}%
\pgfpathcurveto{\pgfqpoint{4.320867in}{2.973784in}}{\pgfqpoint{4.310268in}{2.969393in}}{\pgfqpoint{4.302454in}{2.961580in}}%
\pgfpathcurveto{\pgfqpoint{4.294640in}{2.953766in}}{\pgfqpoint{4.290250in}{2.943167in}}{\pgfqpoint{4.290250in}{2.932117in}}%
\pgfpathcurveto{\pgfqpoint{4.290250in}{2.921067in}}{\pgfqpoint{4.294640in}{2.910468in}}{\pgfqpoint{4.302454in}{2.902654in}}%
\pgfpathcurveto{\pgfqpoint{4.310268in}{2.894841in}}{\pgfqpoint{4.320867in}{2.890450in}}{\pgfqpoint{4.331917in}{2.890450in}}%
\pgfpathclose%
\pgfusepath{stroke,fill}%
\end{pgfscope}%
\begin{pgfscope}%
\pgfpathrectangle{\pgfqpoint{0.787074in}{0.548769in}}{\pgfqpoint{5.062926in}{3.102590in}}%
\pgfusepath{clip}%
\pgfsetbuttcap%
\pgfsetroundjoin%
\definecolor{currentfill}{rgb}{0.121569,0.466667,0.705882}%
\pgfsetfillcolor{currentfill}%
\pgfsetlinewidth{1.003750pt}%
\definecolor{currentstroke}{rgb}{0.121569,0.466667,0.705882}%
\pgfsetstrokecolor{currentstroke}%
\pgfsetdash{}{0pt}%
\pgfpathmoveto{\pgfqpoint{1.312060in}{0.786731in}}%
\pgfpathcurveto{\pgfqpoint{1.323110in}{0.786731in}}{\pgfqpoint{1.333709in}{0.791121in}}{\pgfqpoint{1.341523in}{0.798934in}}%
\pgfpathcurveto{\pgfqpoint{1.349337in}{0.806748in}}{\pgfqpoint{1.353727in}{0.817347in}}{\pgfqpoint{1.353727in}{0.828397in}}%
\pgfpathcurveto{\pgfqpoint{1.353727in}{0.839447in}}{\pgfqpoint{1.349337in}{0.850046in}}{\pgfqpoint{1.341523in}{0.857860in}}%
\pgfpathcurveto{\pgfqpoint{1.333709in}{0.865674in}}{\pgfqpoint{1.323110in}{0.870064in}}{\pgfqpoint{1.312060in}{0.870064in}}%
\pgfpathcurveto{\pgfqpoint{1.301010in}{0.870064in}}{\pgfqpoint{1.290411in}{0.865674in}}{\pgfqpoint{1.282597in}{0.857860in}}%
\pgfpathcurveto{\pgfqpoint{1.274784in}{0.850046in}}{\pgfqpoint{1.270394in}{0.839447in}}{\pgfqpoint{1.270394in}{0.828397in}}%
\pgfpathcurveto{\pgfqpoint{1.270394in}{0.817347in}}{\pgfqpoint{1.274784in}{0.806748in}}{\pgfqpoint{1.282597in}{0.798934in}}%
\pgfpathcurveto{\pgfqpoint{1.290411in}{0.791121in}}{\pgfqpoint{1.301010in}{0.786731in}}{\pgfqpoint{1.312060in}{0.786731in}}%
\pgfpathclose%
\pgfusepath{stroke,fill}%
\end{pgfscope}%
\begin{pgfscope}%
\pgfpathrectangle{\pgfqpoint{0.787074in}{0.548769in}}{\pgfqpoint{5.062926in}{3.102590in}}%
\pgfusepath{clip}%
\pgfsetbuttcap%
\pgfsetroundjoin%
\definecolor{currentfill}{rgb}{0.839216,0.152941,0.156863}%
\pgfsetfillcolor{currentfill}%
\pgfsetlinewidth{1.003750pt}%
\definecolor{currentstroke}{rgb}{0.839216,0.152941,0.156863}%
\pgfsetstrokecolor{currentstroke}%
\pgfsetdash{}{0pt}%
\pgfpathmoveto{\pgfqpoint{4.562278in}{2.746144in}}%
\pgfpathcurveto{\pgfqpoint{4.573329in}{2.746144in}}{\pgfqpoint{4.583928in}{2.750534in}}{\pgfqpoint{4.591741in}{2.758348in}}%
\pgfpathcurveto{\pgfqpoint{4.599555in}{2.766161in}}{\pgfqpoint{4.603945in}{2.776760in}}{\pgfqpoint{4.603945in}{2.787810in}}%
\pgfpathcurveto{\pgfqpoint{4.603945in}{2.798860in}}{\pgfqpoint{4.599555in}{2.809460in}}{\pgfqpoint{4.591741in}{2.817273in}}%
\pgfpathcurveto{\pgfqpoint{4.583928in}{2.825087in}}{\pgfqpoint{4.573329in}{2.829477in}}{\pgfqpoint{4.562278in}{2.829477in}}%
\pgfpathcurveto{\pgfqpoint{4.551228in}{2.829477in}}{\pgfqpoint{4.540629in}{2.825087in}}{\pgfqpoint{4.532816in}{2.817273in}}%
\pgfpathcurveto{\pgfqpoint{4.525002in}{2.809460in}}{\pgfqpoint{4.520612in}{2.798860in}}{\pgfqpoint{4.520612in}{2.787810in}}%
\pgfpathcurveto{\pgfqpoint{4.520612in}{2.776760in}}{\pgfqpoint{4.525002in}{2.766161in}}{\pgfqpoint{4.532816in}{2.758348in}}%
\pgfpathcurveto{\pgfqpoint{4.540629in}{2.750534in}}{\pgfqpoint{4.551228in}{2.746144in}}{\pgfqpoint{4.562278in}{2.746144in}}%
\pgfpathclose%
\pgfusepath{stroke,fill}%
\end{pgfscope}%
\begin{pgfscope}%
\pgfpathrectangle{\pgfqpoint{0.787074in}{0.548769in}}{\pgfqpoint{5.062926in}{3.102590in}}%
\pgfusepath{clip}%
\pgfsetbuttcap%
\pgfsetroundjoin%
\definecolor{currentfill}{rgb}{0.121569,0.466667,0.705882}%
\pgfsetfillcolor{currentfill}%
\pgfsetlinewidth{1.003750pt}%
\definecolor{currentstroke}{rgb}{0.121569,0.466667,0.705882}%
\pgfsetstrokecolor{currentstroke}%
\pgfsetdash{}{0pt}%
\pgfpathmoveto{\pgfqpoint{1.020285in}{0.649982in}}%
\pgfpathcurveto{\pgfqpoint{1.031335in}{0.649982in}}{\pgfqpoint{1.041934in}{0.654372in}}{\pgfqpoint{1.049748in}{0.662186in}}%
\pgfpathcurveto{\pgfqpoint{1.057561in}{0.669999in}}{\pgfqpoint{1.061952in}{0.680598in}}{\pgfqpoint{1.061952in}{0.691648in}}%
\pgfpathcurveto{\pgfqpoint{1.061952in}{0.702699in}}{\pgfqpoint{1.057561in}{0.713298in}}{\pgfqpoint{1.049748in}{0.721111in}}%
\pgfpathcurveto{\pgfqpoint{1.041934in}{0.728925in}}{\pgfqpoint{1.031335in}{0.733315in}}{\pgfqpoint{1.020285in}{0.733315in}}%
\pgfpathcurveto{\pgfqpoint{1.009235in}{0.733315in}}{\pgfqpoint{0.998636in}{0.728925in}}{\pgfqpoint{0.990822in}{0.721111in}}%
\pgfpathcurveto{\pgfqpoint{0.983009in}{0.713298in}}{\pgfqpoint{0.978618in}{0.702699in}}{\pgfqpoint{0.978618in}{0.691648in}}%
\pgfpathcurveto{\pgfqpoint{0.978618in}{0.680598in}}{\pgfqpoint{0.983009in}{0.669999in}}{\pgfqpoint{0.990822in}{0.662186in}}%
\pgfpathcurveto{\pgfqpoint{0.998636in}{0.654372in}}{\pgfqpoint{1.009235in}{0.649982in}}{\pgfqpoint{1.020285in}{0.649982in}}%
\pgfpathclose%
\pgfusepath{stroke,fill}%
\end{pgfscope}%
\begin{pgfscope}%
\pgfpathrectangle{\pgfqpoint{0.787074in}{0.548769in}}{\pgfqpoint{5.062926in}{3.102590in}}%
\pgfusepath{clip}%
\pgfsetbuttcap%
\pgfsetroundjoin%
\definecolor{currentfill}{rgb}{1.000000,0.498039,0.054902}%
\pgfsetfillcolor{currentfill}%
\pgfsetlinewidth{1.003750pt}%
\definecolor{currentstroke}{rgb}{1.000000,0.498039,0.054902}%
\pgfsetstrokecolor{currentstroke}%
\pgfsetdash{}{0pt}%
\pgfpathmoveto{\pgfqpoint{3.634771in}{2.345681in}}%
\pgfpathcurveto{\pgfqpoint{3.645821in}{2.345681in}}{\pgfqpoint{3.656420in}{2.350071in}}{\pgfqpoint{3.664234in}{2.357885in}}%
\pgfpathcurveto{\pgfqpoint{3.672047in}{2.365698in}}{\pgfqpoint{3.676438in}{2.376297in}}{\pgfqpoint{3.676438in}{2.387348in}}%
\pgfpathcurveto{\pgfqpoint{3.676438in}{2.398398in}}{\pgfqpoint{3.672047in}{2.408997in}}{\pgfqpoint{3.664234in}{2.416810in}}%
\pgfpathcurveto{\pgfqpoint{3.656420in}{2.424624in}}{\pgfqpoint{3.645821in}{2.429014in}}{\pgfqpoint{3.634771in}{2.429014in}}%
\pgfpathcurveto{\pgfqpoint{3.623721in}{2.429014in}}{\pgfqpoint{3.613122in}{2.424624in}}{\pgfqpoint{3.605308in}{2.416810in}}%
\pgfpathcurveto{\pgfqpoint{3.597495in}{2.408997in}}{\pgfqpoint{3.593104in}{2.398398in}}{\pgfqpoint{3.593104in}{2.387348in}}%
\pgfpathcurveto{\pgfqpoint{3.593104in}{2.376297in}}{\pgfqpoint{3.597495in}{2.365698in}}{\pgfqpoint{3.605308in}{2.357885in}}%
\pgfpathcurveto{\pgfqpoint{3.613122in}{2.350071in}}{\pgfqpoint{3.623721in}{2.345681in}}{\pgfqpoint{3.634771in}{2.345681in}}%
\pgfpathclose%
\pgfusepath{stroke,fill}%
\end{pgfscope}%
\begin{pgfscope}%
\pgfpathrectangle{\pgfqpoint{0.787074in}{0.548769in}}{\pgfqpoint{5.062926in}{3.102590in}}%
\pgfusepath{clip}%
\pgfsetbuttcap%
\pgfsetroundjoin%
\definecolor{currentfill}{rgb}{0.121569,0.466667,0.705882}%
\pgfsetfillcolor{currentfill}%
\pgfsetlinewidth{1.003750pt}%
\definecolor{currentstroke}{rgb}{0.121569,0.466667,0.705882}%
\pgfsetstrokecolor{currentstroke}%
\pgfsetdash{}{0pt}%
\pgfpathmoveto{\pgfqpoint{1.022231in}{0.651543in}}%
\pgfpathcurveto{\pgfqpoint{1.033282in}{0.651543in}}{\pgfqpoint{1.043881in}{0.655933in}}{\pgfqpoint{1.051694in}{0.663747in}}%
\pgfpathcurveto{\pgfqpoint{1.059508in}{0.671560in}}{\pgfqpoint{1.063898in}{0.682159in}}{\pgfqpoint{1.063898in}{0.693209in}}%
\pgfpathcurveto{\pgfqpoint{1.063898in}{0.704259in}}{\pgfqpoint{1.059508in}{0.714859in}}{\pgfqpoint{1.051694in}{0.722672in}}%
\pgfpathcurveto{\pgfqpoint{1.043881in}{0.730486in}}{\pgfqpoint{1.033282in}{0.734876in}}{\pgfqpoint{1.022231in}{0.734876in}}%
\pgfpathcurveto{\pgfqpoint{1.011181in}{0.734876in}}{\pgfqpoint{1.000582in}{0.730486in}}{\pgfqpoint{0.992769in}{0.722672in}}%
\pgfpathcurveto{\pgfqpoint{0.984955in}{0.714859in}}{\pgfqpoint{0.980565in}{0.704259in}}{\pgfqpoint{0.980565in}{0.693209in}}%
\pgfpathcurveto{\pgfqpoint{0.980565in}{0.682159in}}{\pgfqpoint{0.984955in}{0.671560in}}{\pgfqpoint{0.992769in}{0.663747in}}%
\pgfpathcurveto{\pgfqpoint{1.000582in}{0.655933in}}{\pgfqpoint{1.011181in}{0.651543in}}{\pgfqpoint{1.022231in}{0.651543in}}%
\pgfpathclose%
\pgfusepath{stroke,fill}%
\end{pgfscope}%
\begin{pgfscope}%
\pgfpathrectangle{\pgfqpoint{0.787074in}{0.548769in}}{\pgfqpoint{5.062926in}{3.102590in}}%
\pgfusepath{clip}%
\pgfsetbuttcap%
\pgfsetroundjoin%
\definecolor{currentfill}{rgb}{1.000000,0.498039,0.054902}%
\pgfsetfillcolor{currentfill}%
\pgfsetlinewidth{1.003750pt}%
\definecolor{currentstroke}{rgb}{1.000000,0.498039,0.054902}%
\pgfsetstrokecolor{currentstroke}%
\pgfsetdash{}{0pt}%
\pgfpathmoveto{\pgfqpoint{4.267591in}{2.708956in}}%
\pgfpathcurveto{\pgfqpoint{4.278641in}{2.708956in}}{\pgfqpoint{4.289240in}{2.713346in}}{\pgfqpoint{4.297054in}{2.721160in}}%
\pgfpathcurveto{\pgfqpoint{4.304868in}{2.728974in}}{\pgfqpoint{4.309258in}{2.739573in}}{\pgfqpoint{4.309258in}{2.750623in}}%
\pgfpathcurveto{\pgfqpoint{4.309258in}{2.761673in}}{\pgfqpoint{4.304868in}{2.772272in}}{\pgfqpoint{4.297054in}{2.780086in}}%
\pgfpathcurveto{\pgfqpoint{4.289240in}{2.787899in}}{\pgfqpoint{4.278641in}{2.792289in}}{\pgfqpoint{4.267591in}{2.792289in}}%
\pgfpathcurveto{\pgfqpoint{4.256541in}{2.792289in}}{\pgfqpoint{4.245942in}{2.787899in}}{\pgfqpoint{4.238128in}{2.780086in}}%
\pgfpathcurveto{\pgfqpoint{4.230315in}{2.772272in}}{\pgfqpoint{4.225925in}{2.761673in}}{\pgfqpoint{4.225925in}{2.750623in}}%
\pgfpathcurveto{\pgfqpoint{4.225925in}{2.739573in}}{\pgfqpoint{4.230315in}{2.728974in}}{\pgfqpoint{4.238128in}{2.721160in}}%
\pgfpathcurveto{\pgfqpoint{4.245942in}{2.713346in}}{\pgfqpoint{4.256541in}{2.708956in}}{\pgfqpoint{4.267591in}{2.708956in}}%
\pgfpathclose%
\pgfusepath{stroke,fill}%
\end{pgfscope}%
\begin{pgfscope}%
\pgfpathrectangle{\pgfqpoint{0.787074in}{0.548769in}}{\pgfqpoint{5.062926in}{3.102590in}}%
\pgfusepath{clip}%
\pgfsetbuttcap%
\pgfsetroundjoin%
\definecolor{currentfill}{rgb}{0.121569,0.466667,0.705882}%
\pgfsetfillcolor{currentfill}%
\pgfsetlinewidth{1.003750pt}%
\definecolor{currentstroke}{rgb}{0.121569,0.466667,0.705882}%
\pgfsetstrokecolor{currentstroke}%
\pgfsetdash{}{0pt}%
\pgfpathmoveto{\pgfqpoint{1.017249in}{0.648158in}}%
\pgfpathcurveto{\pgfqpoint{1.028299in}{0.648158in}}{\pgfqpoint{1.038898in}{0.652548in}}{\pgfqpoint{1.046711in}{0.660362in}}%
\pgfpathcurveto{\pgfqpoint{1.054525in}{0.668176in}}{\pgfqpoint{1.058915in}{0.678775in}}{\pgfqpoint{1.058915in}{0.689825in}}%
\pgfpathcurveto{\pgfqpoint{1.058915in}{0.700875in}}{\pgfqpoint{1.054525in}{0.711474in}}{\pgfqpoint{1.046711in}{0.719287in}}%
\pgfpathcurveto{\pgfqpoint{1.038898in}{0.727101in}}{\pgfqpoint{1.028299in}{0.731491in}}{\pgfqpoint{1.017249in}{0.731491in}}%
\pgfpathcurveto{\pgfqpoint{1.006199in}{0.731491in}}{\pgfqpoint{0.995600in}{0.727101in}}{\pgfqpoint{0.987786in}{0.719287in}}%
\pgfpathcurveto{\pgfqpoint{0.979972in}{0.711474in}}{\pgfqpoint{0.975582in}{0.700875in}}{\pgfqpoint{0.975582in}{0.689825in}}%
\pgfpathcurveto{\pgfqpoint{0.975582in}{0.678775in}}{\pgfqpoint{0.979972in}{0.668176in}}{\pgfqpoint{0.987786in}{0.660362in}}%
\pgfpathcurveto{\pgfqpoint{0.995600in}{0.652548in}}{\pgfqpoint{1.006199in}{0.648158in}}{\pgfqpoint{1.017249in}{0.648158in}}%
\pgfpathclose%
\pgfusepath{stroke,fill}%
\end{pgfscope}%
\begin{pgfscope}%
\pgfpathrectangle{\pgfqpoint{0.787074in}{0.548769in}}{\pgfqpoint{5.062926in}{3.102590in}}%
\pgfusepath{clip}%
\pgfsetbuttcap%
\pgfsetroundjoin%
\definecolor{currentfill}{rgb}{1.000000,0.498039,0.054902}%
\pgfsetfillcolor{currentfill}%
\pgfsetlinewidth{1.003750pt}%
\definecolor{currentstroke}{rgb}{1.000000,0.498039,0.054902}%
\pgfsetstrokecolor{currentstroke}%
\pgfsetdash{}{0pt}%
\pgfpathmoveto{\pgfqpoint{4.080727in}{2.868283in}}%
\pgfpathcurveto{\pgfqpoint{4.091777in}{2.868283in}}{\pgfqpoint{4.102376in}{2.872673in}}{\pgfqpoint{4.110190in}{2.880486in}}%
\pgfpathcurveto{\pgfqpoint{4.118004in}{2.888300in}}{\pgfqpoint{4.122394in}{2.898899in}}{\pgfqpoint{4.122394in}{2.909949in}}%
\pgfpathcurveto{\pgfqpoint{4.122394in}{2.920999in}}{\pgfqpoint{4.118004in}{2.931598in}}{\pgfqpoint{4.110190in}{2.939412in}}%
\pgfpathcurveto{\pgfqpoint{4.102376in}{2.947226in}}{\pgfqpoint{4.091777in}{2.951616in}}{\pgfqpoint{4.080727in}{2.951616in}}%
\pgfpathcurveto{\pgfqpoint{4.069677in}{2.951616in}}{\pgfqpoint{4.059078in}{2.947226in}}{\pgfqpoint{4.051265in}{2.939412in}}%
\pgfpathcurveto{\pgfqpoint{4.043451in}{2.931598in}}{\pgfqpoint{4.039061in}{2.920999in}}{\pgfqpoint{4.039061in}{2.909949in}}%
\pgfpathcurveto{\pgfqpoint{4.039061in}{2.898899in}}{\pgfqpoint{4.043451in}{2.888300in}}{\pgfqpoint{4.051265in}{2.880486in}}%
\pgfpathcurveto{\pgfqpoint{4.059078in}{2.872673in}}{\pgfqpoint{4.069677in}{2.868283in}}{\pgfqpoint{4.080727in}{2.868283in}}%
\pgfpathclose%
\pgfusepath{stroke,fill}%
\end{pgfscope}%
\begin{pgfscope}%
\pgfpathrectangle{\pgfqpoint{0.787074in}{0.548769in}}{\pgfqpoint{5.062926in}{3.102590in}}%
\pgfusepath{clip}%
\pgfsetbuttcap%
\pgfsetroundjoin%
\definecolor{currentfill}{rgb}{0.121569,0.466667,0.705882}%
\pgfsetfillcolor{currentfill}%
\pgfsetlinewidth{1.003750pt}%
\definecolor{currentstroke}{rgb}{0.121569,0.466667,0.705882}%
\pgfsetstrokecolor{currentstroke}%
\pgfsetdash{}{0pt}%
\pgfpathmoveto{\pgfqpoint{1.017211in}{0.648132in}}%
\pgfpathcurveto{\pgfqpoint{1.028261in}{0.648132in}}{\pgfqpoint{1.038860in}{0.652522in}}{\pgfqpoint{1.046674in}{0.660336in}}%
\pgfpathcurveto{\pgfqpoint{1.054487in}{0.668149in}}{\pgfqpoint{1.058877in}{0.678748in}}{\pgfqpoint{1.058877in}{0.689799in}}%
\pgfpathcurveto{\pgfqpoint{1.058877in}{0.700849in}}{\pgfqpoint{1.054487in}{0.711448in}}{\pgfqpoint{1.046674in}{0.719261in}}%
\pgfpathcurveto{\pgfqpoint{1.038860in}{0.727075in}}{\pgfqpoint{1.028261in}{0.731465in}}{\pgfqpoint{1.017211in}{0.731465in}}%
\pgfpathcurveto{\pgfqpoint{1.006161in}{0.731465in}}{\pgfqpoint{0.995562in}{0.727075in}}{\pgfqpoint{0.987748in}{0.719261in}}%
\pgfpathcurveto{\pgfqpoint{0.979934in}{0.711448in}}{\pgfqpoint{0.975544in}{0.700849in}}{\pgfqpoint{0.975544in}{0.689799in}}%
\pgfpathcurveto{\pgfqpoint{0.975544in}{0.678748in}}{\pgfqpoint{0.979934in}{0.668149in}}{\pgfqpoint{0.987748in}{0.660336in}}%
\pgfpathcurveto{\pgfqpoint{0.995562in}{0.652522in}}{\pgfqpoint{1.006161in}{0.648132in}}{\pgfqpoint{1.017211in}{0.648132in}}%
\pgfpathclose%
\pgfusepath{stroke,fill}%
\end{pgfscope}%
\begin{pgfscope}%
\pgfpathrectangle{\pgfqpoint{0.787074in}{0.548769in}}{\pgfqpoint{5.062926in}{3.102590in}}%
\pgfusepath{clip}%
\pgfsetbuttcap%
\pgfsetroundjoin%
\definecolor{currentfill}{rgb}{1.000000,0.498039,0.054902}%
\pgfsetfillcolor{currentfill}%
\pgfsetlinewidth{1.003750pt}%
\definecolor{currentstroke}{rgb}{1.000000,0.498039,0.054902}%
\pgfsetstrokecolor{currentstroke}%
\pgfsetdash{}{0pt}%
\pgfpathmoveto{\pgfqpoint{4.016964in}{2.305856in}}%
\pgfpathcurveto{\pgfqpoint{4.028014in}{2.305856in}}{\pgfqpoint{4.038613in}{2.310246in}}{\pgfqpoint{4.046427in}{2.318060in}}%
\pgfpathcurveto{\pgfqpoint{4.054240in}{2.325873in}}{\pgfqpoint{4.058630in}{2.336472in}}{\pgfqpoint{4.058630in}{2.347523in}}%
\pgfpathcurveto{\pgfqpoint{4.058630in}{2.358573in}}{\pgfqpoint{4.054240in}{2.369172in}}{\pgfqpoint{4.046427in}{2.376985in}}%
\pgfpathcurveto{\pgfqpoint{4.038613in}{2.384799in}}{\pgfqpoint{4.028014in}{2.389189in}}{\pgfqpoint{4.016964in}{2.389189in}}%
\pgfpathcurveto{\pgfqpoint{4.005914in}{2.389189in}}{\pgfqpoint{3.995315in}{2.384799in}}{\pgfqpoint{3.987501in}{2.376985in}}%
\pgfpathcurveto{\pgfqpoint{3.979687in}{2.369172in}}{\pgfqpoint{3.975297in}{2.358573in}}{\pgfqpoint{3.975297in}{2.347523in}}%
\pgfpathcurveto{\pgfqpoint{3.975297in}{2.336472in}}{\pgfqpoint{3.979687in}{2.325873in}}{\pgfqpoint{3.987501in}{2.318060in}}%
\pgfpathcurveto{\pgfqpoint{3.995315in}{2.310246in}}{\pgfqpoint{4.005914in}{2.305856in}}{\pgfqpoint{4.016964in}{2.305856in}}%
\pgfpathclose%
\pgfusepath{stroke,fill}%
\end{pgfscope}%
\begin{pgfscope}%
\pgfpathrectangle{\pgfqpoint{0.787074in}{0.548769in}}{\pgfqpoint{5.062926in}{3.102590in}}%
\pgfusepath{clip}%
\pgfsetbuttcap%
\pgfsetroundjoin%
\definecolor{currentfill}{rgb}{1.000000,0.498039,0.054902}%
\pgfsetfillcolor{currentfill}%
\pgfsetlinewidth{1.003750pt}%
\definecolor{currentstroke}{rgb}{1.000000,0.498039,0.054902}%
\pgfsetstrokecolor{currentstroke}%
\pgfsetdash{}{0pt}%
\pgfpathmoveto{\pgfqpoint{3.786527in}{2.492823in}}%
\pgfpathcurveto{\pgfqpoint{3.797577in}{2.492823in}}{\pgfqpoint{3.808176in}{2.497213in}}{\pgfqpoint{3.815990in}{2.505027in}}%
\pgfpathcurveto{\pgfqpoint{3.823803in}{2.512840in}}{\pgfqpoint{3.828194in}{2.523439in}}{\pgfqpoint{3.828194in}{2.534490in}}%
\pgfpathcurveto{\pgfqpoint{3.828194in}{2.545540in}}{\pgfqpoint{3.823803in}{2.556139in}}{\pgfqpoint{3.815990in}{2.563952in}}%
\pgfpathcurveto{\pgfqpoint{3.808176in}{2.571766in}}{\pgfqpoint{3.797577in}{2.576156in}}{\pgfqpoint{3.786527in}{2.576156in}}%
\pgfpathcurveto{\pgfqpoint{3.775477in}{2.576156in}}{\pgfqpoint{3.764878in}{2.571766in}}{\pgfqpoint{3.757064in}{2.563952in}}%
\pgfpathcurveto{\pgfqpoint{3.749250in}{2.556139in}}{\pgfqpoint{3.744860in}{2.545540in}}{\pgfqpoint{3.744860in}{2.534490in}}%
\pgfpathcurveto{\pgfqpoint{3.744860in}{2.523439in}}{\pgfqpoint{3.749250in}{2.512840in}}{\pgfqpoint{3.757064in}{2.505027in}}%
\pgfpathcurveto{\pgfqpoint{3.764878in}{2.497213in}}{\pgfqpoint{3.775477in}{2.492823in}}{\pgfqpoint{3.786527in}{2.492823in}}%
\pgfpathclose%
\pgfusepath{stroke,fill}%
\end{pgfscope}%
\begin{pgfscope}%
\pgfpathrectangle{\pgfqpoint{0.787074in}{0.548769in}}{\pgfqpoint{5.062926in}{3.102590in}}%
\pgfusepath{clip}%
\pgfsetbuttcap%
\pgfsetroundjoin%
\definecolor{currentfill}{rgb}{0.121569,0.466667,0.705882}%
\pgfsetfillcolor{currentfill}%
\pgfsetlinewidth{1.003750pt}%
\definecolor{currentstroke}{rgb}{0.121569,0.466667,0.705882}%
\pgfsetstrokecolor{currentstroke}%
\pgfsetdash{}{0pt}%
\pgfpathmoveto{\pgfqpoint{1.017232in}{0.648147in}}%
\pgfpathcurveto{\pgfqpoint{1.028282in}{0.648147in}}{\pgfqpoint{1.038881in}{0.652537in}}{\pgfqpoint{1.046695in}{0.660351in}}%
\pgfpathcurveto{\pgfqpoint{1.054509in}{0.668165in}}{\pgfqpoint{1.058899in}{0.678764in}}{\pgfqpoint{1.058899in}{0.689814in}}%
\pgfpathcurveto{\pgfqpoint{1.058899in}{0.700864in}}{\pgfqpoint{1.054509in}{0.711463in}}{\pgfqpoint{1.046695in}{0.719276in}}%
\pgfpathcurveto{\pgfqpoint{1.038881in}{0.727090in}}{\pgfqpoint{1.028282in}{0.731480in}}{\pgfqpoint{1.017232in}{0.731480in}}%
\pgfpathcurveto{\pgfqpoint{1.006182in}{0.731480in}}{\pgfqpoint{0.995583in}{0.727090in}}{\pgfqpoint{0.987770in}{0.719276in}}%
\pgfpathcurveto{\pgfqpoint{0.979956in}{0.711463in}}{\pgfqpoint{0.975566in}{0.700864in}}{\pgfqpoint{0.975566in}{0.689814in}}%
\pgfpathcurveto{\pgfqpoint{0.975566in}{0.678764in}}{\pgfqpoint{0.979956in}{0.668165in}}{\pgfqpoint{0.987770in}{0.660351in}}%
\pgfpathcurveto{\pgfqpoint{0.995583in}{0.652537in}}{\pgfqpoint{1.006182in}{0.648147in}}{\pgfqpoint{1.017232in}{0.648147in}}%
\pgfpathclose%
\pgfusepath{stroke,fill}%
\end{pgfscope}%
\begin{pgfscope}%
\pgfpathrectangle{\pgfqpoint{0.787074in}{0.548769in}}{\pgfqpoint{5.062926in}{3.102590in}}%
\pgfusepath{clip}%
\pgfsetbuttcap%
\pgfsetroundjoin%
\definecolor{currentfill}{rgb}{1.000000,0.498039,0.054902}%
\pgfsetfillcolor{currentfill}%
\pgfsetlinewidth{1.003750pt}%
\definecolor{currentstroke}{rgb}{1.000000,0.498039,0.054902}%
\pgfsetstrokecolor{currentstroke}%
\pgfsetdash{}{0pt}%
\pgfpathmoveto{\pgfqpoint{5.107186in}{2.754884in}}%
\pgfpathcurveto{\pgfqpoint{5.118236in}{2.754884in}}{\pgfqpoint{5.128835in}{2.759274in}}{\pgfqpoint{5.136649in}{2.767088in}}%
\pgfpathcurveto{\pgfqpoint{5.144462in}{2.774902in}}{\pgfqpoint{5.148853in}{2.785501in}}{\pgfqpoint{5.148853in}{2.796551in}}%
\pgfpathcurveto{\pgfqpoint{5.148853in}{2.807601in}}{\pgfqpoint{5.144462in}{2.818200in}}{\pgfqpoint{5.136649in}{2.826014in}}%
\pgfpathcurveto{\pgfqpoint{5.128835in}{2.833827in}}{\pgfqpoint{5.118236in}{2.838218in}}{\pgfqpoint{5.107186in}{2.838218in}}%
\pgfpathcurveto{\pgfqpoint{5.096136in}{2.838218in}}{\pgfqpoint{5.085537in}{2.833827in}}{\pgfqpoint{5.077723in}{2.826014in}}%
\pgfpathcurveto{\pgfqpoint{5.069909in}{2.818200in}}{\pgfqpoint{5.065519in}{2.807601in}}{\pgfqpoint{5.065519in}{2.796551in}}%
\pgfpathcurveto{\pgfqpoint{5.065519in}{2.785501in}}{\pgfqpoint{5.069909in}{2.774902in}}{\pgfqpoint{5.077723in}{2.767088in}}%
\pgfpathcurveto{\pgfqpoint{5.085537in}{2.759274in}}{\pgfqpoint{5.096136in}{2.754884in}}{\pgfqpoint{5.107186in}{2.754884in}}%
\pgfpathclose%
\pgfusepath{stroke,fill}%
\end{pgfscope}%
\begin{pgfscope}%
\pgfpathrectangle{\pgfqpoint{0.787074in}{0.548769in}}{\pgfqpoint{5.062926in}{3.102590in}}%
\pgfusepath{clip}%
\pgfsetbuttcap%
\pgfsetroundjoin%
\definecolor{currentfill}{rgb}{1.000000,0.498039,0.054902}%
\pgfsetfillcolor{currentfill}%
\pgfsetlinewidth{1.003750pt}%
\definecolor{currentstroke}{rgb}{1.000000,0.498039,0.054902}%
\pgfsetstrokecolor{currentstroke}%
\pgfsetdash{}{0pt}%
\pgfpathmoveto{\pgfqpoint{5.072621in}{2.596250in}}%
\pgfpathcurveto{\pgfqpoint{5.083672in}{2.596250in}}{\pgfqpoint{5.094271in}{2.600640in}}{\pgfqpoint{5.102084in}{2.608454in}}%
\pgfpathcurveto{\pgfqpoint{5.109898in}{2.616267in}}{\pgfqpoint{5.114288in}{2.626867in}}{\pgfqpoint{5.114288in}{2.637917in}}%
\pgfpathcurveto{\pgfqpoint{5.114288in}{2.648967in}}{\pgfqpoint{5.109898in}{2.659566in}}{\pgfqpoint{5.102084in}{2.667379in}}%
\pgfpathcurveto{\pgfqpoint{5.094271in}{2.675193in}}{\pgfqpoint{5.083672in}{2.679583in}}{\pgfqpoint{5.072621in}{2.679583in}}%
\pgfpathcurveto{\pgfqpoint{5.061571in}{2.679583in}}{\pgfqpoint{5.050972in}{2.675193in}}{\pgfqpoint{5.043159in}{2.667379in}}%
\pgfpathcurveto{\pgfqpoint{5.035345in}{2.659566in}}{\pgfqpoint{5.030955in}{2.648967in}}{\pgfqpoint{5.030955in}{2.637917in}}%
\pgfpathcurveto{\pgfqpoint{5.030955in}{2.626867in}}{\pgfqpoint{5.035345in}{2.616267in}}{\pgfqpoint{5.043159in}{2.608454in}}%
\pgfpathcurveto{\pgfqpoint{5.050972in}{2.600640in}}{\pgfqpoint{5.061571in}{2.596250in}}{\pgfqpoint{5.072621in}{2.596250in}}%
\pgfpathclose%
\pgfusepath{stroke,fill}%
\end{pgfscope}%
\begin{pgfscope}%
\pgfpathrectangle{\pgfqpoint{0.787074in}{0.548769in}}{\pgfqpoint{5.062926in}{3.102590in}}%
\pgfusepath{clip}%
\pgfsetbuttcap%
\pgfsetroundjoin%
\definecolor{currentfill}{rgb}{0.121569,0.466667,0.705882}%
\pgfsetfillcolor{currentfill}%
\pgfsetlinewidth{1.003750pt}%
\definecolor{currentstroke}{rgb}{0.121569,0.466667,0.705882}%
\pgfsetstrokecolor{currentstroke}%
\pgfsetdash{}{0pt}%
\pgfpathmoveto{\pgfqpoint{1.038483in}{0.659080in}}%
\pgfpathcurveto{\pgfqpoint{1.049533in}{0.659080in}}{\pgfqpoint{1.060132in}{0.663470in}}{\pgfqpoint{1.067945in}{0.671284in}}%
\pgfpathcurveto{\pgfqpoint{1.075759in}{0.679098in}}{\pgfqpoint{1.080149in}{0.689697in}}{\pgfqpoint{1.080149in}{0.700747in}}%
\pgfpathcurveto{\pgfqpoint{1.080149in}{0.711797in}}{\pgfqpoint{1.075759in}{0.722396in}}{\pgfqpoint{1.067945in}{0.730210in}}%
\pgfpathcurveto{\pgfqpoint{1.060132in}{0.738023in}}{\pgfqpoint{1.049533in}{0.742413in}}{\pgfqpoint{1.038483in}{0.742413in}}%
\pgfpathcurveto{\pgfqpoint{1.027432in}{0.742413in}}{\pgfqpoint{1.016833in}{0.738023in}}{\pgfqpoint{1.009020in}{0.730210in}}%
\pgfpathcurveto{\pgfqpoint{1.001206in}{0.722396in}}{\pgfqpoint{0.996816in}{0.711797in}}{\pgfqpoint{0.996816in}{0.700747in}}%
\pgfpathcurveto{\pgfqpoint{0.996816in}{0.689697in}}{\pgfqpoint{1.001206in}{0.679098in}}{\pgfqpoint{1.009020in}{0.671284in}}%
\pgfpathcurveto{\pgfqpoint{1.016833in}{0.663470in}}{\pgfqpoint{1.027432in}{0.659080in}}{\pgfqpoint{1.038483in}{0.659080in}}%
\pgfpathclose%
\pgfusepath{stroke,fill}%
\end{pgfscope}%
\begin{pgfscope}%
\pgfpathrectangle{\pgfqpoint{0.787074in}{0.548769in}}{\pgfqpoint{5.062926in}{3.102590in}}%
\pgfusepath{clip}%
\pgfsetbuttcap%
\pgfsetroundjoin%
\definecolor{currentfill}{rgb}{1.000000,0.498039,0.054902}%
\pgfsetfillcolor{currentfill}%
\pgfsetlinewidth{1.003750pt}%
\definecolor{currentstroke}{rgb}{1.000000,0.498039,0.054902}%
\pgfsetstrokecolor{currentstroke}%
\pgfsetdash{}{0pt}%
\pgfpathmoveto{\pgfqpoint{4.555382in}{2.981796in}}%
\pgfpathcurveto{\pgfqpoint{4.566432in}{2.981796in}}{\pgfqpoint{4.577031in}{2.986186in}}{\pgfqpoint{4.584845in}{2.994000in}}%
\pgfpathcurveto{\pgfqpoint{4.592658in}{3.001813in}}{\pgfqpoint{4.597049in}{3.012412in}}{\pgfqpoint{4.597049in}{3.023462in}}%
\pgfpathcurveto{\pgfqpoint{4.597049in}{3.034512in}}{\pgfqpoint{4.592658in}{3.045112in}}{\pgfqpoint{4.584845in}{3.052925in}}%
\pgfpathcurveto{\pgfqpoint{4.577031in}{3.060739in}}{\pgfqpoint{4.566432in}{3.065129in}}{\pgfqpoint{4.555382in}{3.065129in}}%
\pgfpathcurveto{\pgfqpoint{4.544332in}{3.065129in}}{\pgfqpoint{4.533733in}{3.060739in}}{\pgfqpoint{4.525919in}{3.052925in}}%
\pgfpathcurveto{\pgfqpoint{4.518105in}{3.045112in}}{\pgfqpoint{4.513715in}{3.034512in}}{\pgfqpoint{4.513715in}{3.023462in}}%
\pgfpathcurveto{\pgfqpoint{4.513715in}{3.012412in}}{\pgfqpoint{4.518105in}{3.001813in}}{\pgfqpoint{4.525919in}{2.994000in}}%
\pgfpathcurveto{\pgfqpoint{4.533733in}{2.986186in}}{\pgfqpoint{4.544332in}{2.981796in}}{\pgfqpoint{4.555382in}{2.981796in}}%
\pgfpathclose%
\pgfusepath{stroke,fill}%
\end{pgfscope}%
\begin{pgfscope}%
\pgfpathrectangle{\pgfqpoint{0.787074in}{0.548769in}}{\pgfqpoint{5.062926in}{3.102590in}}%
\pgfusepath{clip}%
\pgfsetbuttcap%
\pgfsetroundjoin%
\definecolor{currentfill}{rgb}{1.000000,0.498039,0.054902}%
\pgfsetfillcolor{currentfill}%
\pgfsetlinewidth{1.003750pt}%
\definecolor{currentstroke}{rgb}{1.000000,0.498039,0.054902}%
\pgfsetstrokecolor{currentstroke}%
\pgfsetdash{}{0pt}%
\pgfpathmoveto{\pgfqpoint{4.617364in}{2.923368in}}%
\pgfpathcurveto{\pgfqpoint{4.628414in}{2.923368in}}{\pgfqpoint{4.639013in}{2.927758in}}{\pgfqpoint{4.646827in}{2.935572in}}%
\pgfpathcurveto{\pgfqpoint{4.654640in}{2.943385in}}{\pgfqpoint{4.659030in}{2.953984in}}{\pgfqpoint{4.659030in}{2.965034in}}%
\pgfpathcurveto{\pgfqpoint{4.659030in}{2.976084in}}{\pgfqpoint{4.654640in}{2.986683in}}{\pgfqpoint{4.646827in}{2.994497in}}%
\pgfpathcurveto{\pgfqpoint{4.639013in}{3.002311in}}{\pgfqpoint{4.628414in}{3.006701in}}{\pgfqpoint{4.617364in}{3.006701in}}%
\pgfpathcurveto{\pgfqpoint{4.606314in}{3.006701in}}{\pgfqpoint{4.595715in}{3.002311in}}{\pgfqpoint{4.587901in}{2.994497in}}%
\pgfpathcurveto{\pgfqpoint{4.580087in}{2.986683in}}{\pgfqpoint{4.575697in}{2.976084in}}{\pgfqpoint{4.575697in}{2.965034in}}%
\pgfpathcurveto{\pgfqpoint{4.575697in}{2.953984in}}{\pgfqpoint{4.580087in}{2.943385in}}{\pgfqpoint{4.587901in}{2.935572in}}%
\pgfpathcurveto{\pgfqpoint{4.595715in}{2.927758in}}{\pgfqpoint{4.606314in}{2.923368in}}{\pgfqpoint{4.617364in}{2.923368in}}%
\pgfpathclose%
\pgfusepath{stroke,fill}%
\end{pgfscope}%
\begin{pgfscope}%
\pgfpathrectangle{\pgfqpoint{0.787074in}{0.548769in}}{\pgfqpoint{5.062926in}{3.102590in}}%
\pgfusepath{clip}%
\pgfsetbuttcap%
\pgfsetroundjoin%
\definecolor{currentfill}{rgb}{1.000000,0.498039,0.054902}%
\pgfsetfillcolor{currentfill}%
\pgfsetlinewidth{1.003750pt}%
\definecolor{currentstroke}{rgb}{1.000000,0.498039,0.054902}%
\pgfsetstrokecolor{currentstroke}%
\pgfsetdash{}{0pt}%
\pgfpathmoveto{\pgfqpoint{4.045560in}{2.279400in}}%
\pgfpathcurveto{\pgfqpoint{4.056610in}{2.279400in}}{\pgfqpoint{4.067209in}{2.283790in}}{\pgfqpoint{4.075023in}{2.291603in}}%
\pgfpathcurveto{\pgfqpoint{4.082836in}{2.299417in}}{\pgfqpoint{4.087227in}{2.310016in}}{\pgfqpoint{4.087227in}{2.321066in}}%
\pgfpathcurveto{\pgfqpoint{4.087227in}{2.332116in}}{\pgfqpoint{4.082836in}{2.342715in}}{\pgfqpoint{4.075023in}{2.350529in}}%
\pgfpathcurveto{\pgfqpoint{4.067209in}{2.358343in}}{\pgfqpoint{4.056610in}{2.362733in}}{\pgfqpoint{4.045560in}{2.362733in}}%
\pgfpathcurveto{\pgfqpoint{4.034510in}{2.362733in}}{\pgfqpoint{4.023911in}{2.358343in}}{\pgfqpoint{4.016097in}{2.350529in}}%
\pgfpathcurveto{\pgfqpoint{4.008284in}{2.342715in}}{\pgfqpoint{4.003893in}{2.332116in}}{\pgfqpoint{4.003893in}{2.321066in}}%
\pgfpathcurveto{\pgfqpoint{4.003893in}{2.310016in}}{\pgfqpoint{4.008284in}{2.299417in}}{\pgfqpoint{4.016097in}{2.291603in}}%
\pgfpathcurveto{\pgfqpoint{4.023911in}{2.283790in}}{\pgfqpoint{4.034510in}{2.279400in}}{\pgfqpoint{4.045560in}{2.279400in}}%
\pgfpathclose%
\pgfusepath{stroke,fill}%
\end{pgfscope}%
\begin{pgfscope}%
\pgfpathrectangle{\pgfqpoint{0.787074in}{0.548769in}}{\pgfqpoint{5.062926in}{3.102590in}}%
\pgfusepath{clip}%
\pgfsetbuttcap%
\pgfsetroundjoin%
\definecolor{currentfill}{rgb}{1.000000,0.498039,0.054902}%
\pgfsetfillcolor{currentfill}%
\pgfsetlinewidth{1.003750pt}%
\definecolor{currentstroke}{rgb}{1.000000,0.498039,0.054902}%
\pgfsetstrokecolor{currentstroke}%
\pgfsetdash{}{0pt}%
\pgfpathmoveto{\pgfqpoint{4.255025in}{2.414739in}}%
\pgfpathcurveto{\pgfqpoint{4.266075in}{2.414739in}}{\pgfqpoint{4.276674in}{2.419129in}}{\pgfqpoint{4.284487in}{2.426942in}}%
\pgfpathcurveto{\pgfqpoint{4.292301in}{2.434756in}}{\pgfqpoint{4.296691in}{2.445355in}}{\pgfqpoint{4.296691in}{2.456405in}}%
\pgfpathcurveto{\pgfqpoint{4.296691in}{2.467455in}}{\pgfqpoint{4.292301in}{2.478054in}}{\pgfqpoint{4.284487in}{2.485868in}}%
\pgfpathcurveto{\pgfqpoint{4.276674in}{2.493682in}}{\pgfqpoint{4.266075in}{2.498072in}}{\pgfqpoint{4.255025in}{2.498072in}}%
\pgfpathcurveto{\pgfqpoint{4.243974in}{2.498072in}}{\pgfqpoint{4.233375in}{2.493682in}}{\pgfqpoint{4.225562in}{2.485868in}}%
\pgfpathcurveto{\pgfqpoint{4.217748in}{2.478054in}}{\pgfqpoint{4.213358in}{2.467455in}}{\pgfqpoint{4.213358in}{2.456405in}}%
\pgfpathcurveto{\pgfqpoint{4.213358in}{2.445355in}}{\pgfqpoint{4.217748in}{2.434756in}}{\pgfqpoint{4.225562in}{2.426942in}}%
\pgfpathcurveto{\pgfqpoint{4.233375in}{2.419129in}}{\pgfqpoint{4.243974in}{2.414739in}}{\pgfqpoint{4.255025in}{2.414739in}}%
\pgfpathclose%
\pgfusepath{stroke,fill}%
\end{pgfscope}%
\begin{pgfscope}%
\pgfpathrectangle{\pgfqpoint{0.787074in}{0.548769in}}{\pgfqpoint{5.062926in}{3.102590in}}%
\pgfusepath{clip}%
\pgfsetbuttcap%
\pgfsetroundjoin%
\definecolor{currentfill}{rgb}{1.000000,0.498039,0.054902}%
\pgfsetfillcolor{currentfill}%
\pgfsetlinewidth{1.003750pt}%
\definecolor{currentstroke}{rgb}{1.000000,0.498039,0.054902}%
\pgfsetstrokecolor{currentstroke}%
\pgfsetdash{}{0pt}%
\pgfpathmoveto{\pgfqpoint{4.136809in}{3.155405in}}%
\pgfpathcurveto{\pgfqpoint{4.147859in}{3.155405in}}{\pgfqpoint{4.158459in}{3.159795in}}{\pgfqpoint{4.166272in}{3.167609in}}%
\pgfpathcurveto{\pgfqpoint{4.174086in}{3.175422in}}{\pgfqpoint{4.178476in}{3.186021in}}{\pgfqpoint{4.178476in}{3.197072in}}%
\pgfpathcurveto{\pgfqpoint{4.178476in}{3.208122in}}{\pgfqpoint{4.174086in}{3.218721in}}{\pgfqpoint{4.166272in}{3.226534in}}%
\pgfpathcurveto{\pgfqpoint{4.158459in}{3.234348in}}{\pgfqpoint{4.147859in}{3.238738in}}{\pgfqpoint{4.136809in}{3.238738in}}%
\pgfpathcurveto{\pgfqpoint{4.125759in}{3.238738in}}{\pgfqpoint{4.115160in}{3.234348in}}{\pgfqpoint{4.107347in}{3.226534in}}%
\pgfpathcurveto{\pgfqpoint{4.099533in}{3.218721in}}{\pgfqpoint{4.095143in}{3.208122in}}{\pgfqpoint{4.095143in}{3.197072in}}%
\pgfpathcurveto{\pgfqpoint{4.095143in}{3.186021in}}{\pgfqpoint{4.099533in}{3.175422in}}{\pgfqpoint{4.107347in}{3.167609in}}%
\pgfpathcurveto{\pgfqpoint{4.115160in}{3.159795in}}{\pgfqpoint{4.125759in}{3.155405in}}{\pgfqpoint{4.136809in}{3.155405in}}%
\pgfpathclose%
\pgfusepath{stroke,fill}%
\end{pgfscope}%
\begin{pgfscope}%
\pgfpathrectangle{\pgfqpoint{0.787074in}{0.548769in}}{\pgfqpoint{5.062926in}{3.102590in}}%
\pgfusepath{clip}%
\pgfsetbuttcap%
\pgfsetroundjoin%
\definecolor{currentfill}{rgb}{1.000000,0.498039,0.054902}%
\pgfsetfillcolor{currentfill}%
\pgfsetlinewidth{1.003750pt}%
\definecolor{currentstroke}{rgb}{1.000000,0.498039,0.054902}%
\pgfsetstrokecolor{currentstroke}%
\pgfsetdash{}{0pt}%
\pgfpathmoveto{\pgfqpoint{4.513145in}{2.889472in}}%
\pgfpathcurveto{\pgfqpoint{4.524195in}{2.889472in}}{\pgfqpoint{4.534794in}{2.893862in}}{\pgfqpoint{4.542607in}{2.901676in}}%
\pgfpathcurveto{\pgfqpoint{4.550421in}{2.909489in}}{\pgfqpoint{4.554811in}{2.920088in}}{\pgfqpoint{4.554811in}{2.931138in}}%
\pgfpathcurveto{\pgfqpoint{4.554811in}{2.942189in}}{\pgfqpoint{4.550421in}{2.952788in}}{\pgfqpoint{4.542607in}{2.960601in}}%
\pgfpathcurveto{\pgfqpoint{4.534794in}{2.968415in}}{\pgfqpoint{4.524195in}{2.972805in}}{\pgfqpoint{4.513145in}{2.972805in}}%
\pgfpathcurveto{\pgfqpoint{4.502095in}{2.972805in}}{\pgfqpoint{4.491496in}{2.968415in}}{\pgfqpoint{4.483682in}{2.960601in}}%
\pgfpathcurveto{\pgfqpoint{4.475868in}{2.952788in}}{\pgfqpoint{4.471478in}{2.942189in}}{\pgfqpoint{4.471478in}{2.931138in}}%
\pgfpathcurveto{\pgfqpoint{4.471478in}{2.920088in}}{\pgfqpoint{4.475868in}{2.909489in}}{\pgfqpoint{4.483682in}{2.901676in}}%
\pgfpathcurveto{\pgfqpoint{4.491496in}{2.893862in}}{\pgfqpoint{4.502095in}{2.889472in}}{\pgfqpoint{4.513145in}{2.889472in}}%
\pgfpathclose%
\pgfusepath{stroke,fill}%
\end{pgfscope}%
\begin{pgfscope}%
\pgfpathrectangle{\pgfqpoint{0.787074in}{0.548769in}}{\pgfqpoint{5.062926in}{3.102590in}}%
\pgfusepath{clip}%
\pgfsetbuttcap%
\pgfsetroundjoin%
\definecolor{currentfill}{rgb}{0.121569,0.466667,0.705882}%
\pgfsetfillcolor{currentfill}%
\pgfsetlinewidth{1.003750pt}%
\definecolor{currentstroke}{rgb}{0.121569,0.466667,0.705882}%
\pgfsetstrokecolor{currentstroke}%
\pgfsetdash{}{0pt}%
\pgfpathmoveto{\pgfqpoint{1.017243in}{0.648143in}}%
\pgfpathcurveto{\pgfqpoint{1.028293in}{0.648143in}}{\pgfqpoint{1.038892in}{0.652534in}}{\pgfqpoint{1.046706in}{0.660347in}}%
\pgfpathcurveto{\pgfqpoint{1.054520in}{0.668161in}}{\pgfqpoint{1.058910in}{0.678760in}}{\pgfqpoint{1.058910in}{0.689810in}}%
\pgfpathcurveto{\pgfqpoint{1.058910in}{0.700860in}}{\pgfqpoint{1.054520in}{0.711459in}}{\pgfqpoint{1.046706in}{0.719273in}}%
\pgfpathcurveto{\pgfqpoint{1.038892in}{0.727086in}}{\pgfqpoint{1.028293in}{0.731477in}}{\pgfqpoint{1.017243in}{0.731477in}}%
\pgfpathcurveto{\pgfqpoint{1.006193in}{0.731477in}}{\pgfqpoint{0.995594in}{0.727086in}}{\pgfqpoint{0.987780in}{0.719273in}}%
\pgfpathcurveto{\pgfqpoint{0.979967in}{0.711459in}}{\pgfqpoint{0.975577in}{0.700860in}}{\pgfqpoint{0.975577in}{0.689810in}}%
\pgfpathcurveto{\pgfqpoint{0.975577in}{0.678760in}}{\pgfqpoint{0.979967in}{0.668161in}}{\pgfqpoint{0.987780in}{0.660347in}}%
\pgfpathcurveto{\pgfqpoint{0.995594in}{0.652534in}}{\pgfqpoint{1.006193in}{0.648143in}}{\pgfqpoint{1.017243in}{0.648143in}}%
\pgfpathclose%
\pgfusepath{stroke,fill}%
\end{pgfscope}%
\begin{pgfscope}%
\pgfpathrectangle{\pgfqpoint{0.787074in}{0.548769in}}{\pgfqpoint{5.062926in}{3.102590in}}%
\pgfusepath{clip}%
\pgfsetbuttcap%
\pgfsetroundjoin%
\definecolor{currentfill}{rgb}{0.121569,0.466667,0.705882}%
\pgfsetfillcolor{currentfill}%
\pgfsetlinewidth{1.003750pt}%
\definecolor{currentstroke}{rgb}{0.121569,0.466667,0.705882}%
\pgfsetstrokecolor{currentstroke}%
\pgfsetdash{}{0pt}%
\pgfpathmoveto{\pgfqpoint{1.017251in}{0.648148in}}%
\pgfpathcurveto{\pgfqpoint{1.028301in}{0.648148in}}{\pgfqpoint{1.038900in}{0.652538in}}{\pgfqpoint{1.046714in}{0.660352in}}%
\pgfpathcurveto{\pgfqpoint{1.054528in}{0.668165in}}{\pgfqpoint{1.058918in}{0.678764in}}{\pgfqpoint{1.058918in}{0.689814in}}%
\pgfpathcurveto{\pgfqpoint{1.058918in}{0.700865in}}{\pgfqpoint{1.054528in}{0.711464in}}{\pgfqpoint{1.046714in}{0.719277in}}%
\pgfpathcurveto{\pgfqpoint{1.038900in}{0.727091in}}{\pgfqpoint{1.028301in}{0.731481in}}{\pgfqpoint{1.017251in}{0.731481in}}%
\pgfpathcurveto{\pgfqpoint{1.006201in}{0.731481in}}{\pgfqpoint{0.995602in}{0.727091in}}{\pgfqpoint{0.987788in}{0.719277in}}%
\pgfpathcurveto{\pgfqpoint{0.979975in}{0.711464in}}{\pgfqpoint{0.975585in}{0.700865in}}{\pgfqpoint{0.975585in}{0.689814in}}%
\pgfpathcurveto{\pgfqpoint{0.975585in}{0.678764in}}{\pgfqpoint{0.979975in}{0.668165in}}{\pgfqpoint{0.987788in}{0.660352in}}%
\pgfpathcurveto{\pgfqpoint{0.995602in}{0.652538in}}{\pgfqpoint{1.006201in}{0.648148in}}{\pgfqpoint{1.017251in}{0.648148in}}%
\pgfpathclose%
\pgfusepath{stroke,fill}%
\end{pgfscope}%
\begin{pgfscope}%
\pgfpathrectangle{\pgfqpoint{0.787074in}{0.548769in}}{\pgfqpoint{5.062926in}{3.102590in}}%
\pgfusepath{clip}%
\pgfsetbuttcap%
\pgfsetroundjoin%
\definecolor{currentfill}{rgb}{0.121569,0.466667,0.705882}%
\pgfsetfillcolor{currentfill}%
\pgfsetlinewidth{1.003750pt}%
\definecolor{currentstroke}{rgb}{0.121569,0.466667,0.705882}%
\pgfsetstrokecolor{currentstroke}%
\pgfsetdash{}{0pt}%
\pgfpathmoveto{\pgfqpoint{1.017238in}{0.648151in}}%
\pgfpathcurveto{\pgfqpoint{1.028288in}{0.648151in}}{\pgfqpoint{1.038887in}{0.652542in}}{\pgfqpoint{1.046701in}{0.660355in}}%
\pgfpathcurveto{\pgfqpoint{1.054514in}{0.668169in}}{\pgfqpoint{1.058905in}{0.678768in}}{\pgfqpoint{1.058905in}{0.689818in}}%
\pgfpathcurveto{\pgfqpoint{1.058905in}{0.700868in}}{\pgfqpoint{1.054514in}{0.711467in}}{\pgfqpoint{1.046701in}{0.719281in}}%
\pgfpathcurveto{\pgfqpoint{1.038887in}{0.727094in}}{\pgfqpoint{1.028288in}{0.731485in}}{\pgfqpoint{1.017238in}{0.731485in}}%
\pgfpathcurveto{\pgfqpoint{1.006188in}{0.731485in}}{\pgfqpoint{0.995589in}{0.727094in}}{\pgfqpoint{0.987775in}{0.719281in}}%
\pgfpathcurveto{\pgfqpoint{0.979962in}{0.711467in}}{\pgfqpoint{0.975571in}{0.700868in}}{\pgfqpoint{0.975571in}{0.689818in}}%
\pgfpathcurveto{\pgfqpoint{0.975571in}{0.678768in}}{\pgfqpoint{0.979962in}{0.668169in}}{\pgfqpoint{0.987775in}{0.660355in}}%
\pgfpathcurveto{\pgfqpoint{0.995589in}{0.652542in}}{\pgfqpoint{1.006188in}{0.648151in}}{\pgfqpoint{1.017238in}{0.648151in}}%
\pgfpathclose%
\pgfusepath{stroke,fill}%
\end{pgfscope}%
\begin{pgfscope}%
\pgfpathrectangle{\pgfqpoint{0.787074in}{0.548769in}}{\pgfqpoint{5.062926in}{3.102590in}}%
\pgfusepath{clip}%
\pgfsetbuttcap%
\pgfsetroundjoin%
\definecolor{currentfill}{rgb}{1.000000,0.498039,0.054902}%
\pgfsetfillcolor{currentfill}%
\pgfsetlinewidth{1.003750pt}%
\definecolor{currentstroke}{rgb}{1.000000,0.498039,0.054902}%
\pgfsetstrokecolor{currentstroke}%
\pgfsetdash{}{0pt}%
\pgfpathmoveto{\pgfqpoint{4.447940in}{2.770182in}}%
\pgfpathcurveto{\pgfqpoint{4.458990in}{2.770182in}}{\pgfqpoint{4.469589in}{2.774572in}}{\pgfqpoint{4.477403in}{2.782386in}}%
\pgfpathcurveto{\pgfqpoint{4.485217in}{2.790199in}}{\pgfqpoint{4.489607in}{2.800798in}}{\pgfqpoint{4.489607in}{2.811848in}}%
\pgfpathcurveto{\pgfqpoint{4.489607in}{2.822898in}}{\pgfqpoint{4.485217in}{2.833497in}}{\pgfqpoint{4.477403in}{2.841311in}}%
\pgfpathcurveto{\pgfqpoint{4.469589in}{2.849125in}}{\pgfqpoint{4.458990in}{2.853515in}}{\pgfqpoint{4.447940in}{2.853515in}}%
\pgfpathcurveto{\pgfqpoint{4.436890in}{2.853515in}}{\pgfqpoint{4.426291in}{2.849125in}}{\pgfqpoint{4.418477in}{2.841311in}}%
\pgfpathcurveto{\pgfqpoint{4.410664in}{2.833497in}}{\pgfqpoint{4.406273in}{2.822898in}}{\pgfqpoint{4.406273in}{2.811848in}}%
\pgfpathcurveto{\pgfqpoint{4.406273in}{2.800798in}}{\pgfqpoint{4.410664in}{2.790199in}}{\pgfqpoint{4.418477in}{2.782386in}}%
\pgfpathcurveto{\pgfqpoint{4.426291in}{2.774572in}}{\pgfqpoint{4.436890in}{2.770182in}}{\pgfqpoint{4.447940in}{2.770182in}}%
\pgfpathclose%
\pgfusepath{stroke,fill}%
\end{pgfscope}%
\begin{pgfscope}%
\pgfpathrectangle{\pgfqpoint{0.787074in}{0.548769in}}{\pgfqpoint{5.062926in}{3.102590in}}%
\pgfusepath{clip}%
\pgfsetbuttcap%
\pgfsetroundjoin%
\definecolor{currentfill}{rgb}{1.000000,0.498039,0.054902}%
\pgfsetfillcolor{currentfill}%
\pgfsetlinewidth{1.003750pt}%
\definecolor{currentstroke}{rgb}{1.000000,0.498039,0.054902}%
\pgfsetstrokecolor{currentstroke}%
\pgfsetdash{}{0pt}%
\pgfpathmoveto{\pgfqpoint{4.590193in}{3.154057in}}%
\pgfpathcurveto{\pgfqpoint{4.601243in}{3.154057in}}{\pgfqpoint{4.611842in}{3.158447in}}{\pgfqpoint{4.619656in}{3.166261in}}%
\pgfpathcurveto{\pgfqpoint{4.627469in}{3.174075in}}{\pgfqpoint{4.631860in}{3.184674in}}{\pgfqpoint{4.631860in}{3.195724in}}%
\pgfpathcurveto{\pgfqpoint{4.631860in}{3.206774in}}{\pgfqpoint{4.627469in}{3.217373in}}{\pgfqpoint{4.619656in}{3.225187in}}%
\pgfpathcurveto{\pgfqpoint{4.611842in}{3.233000in}}{\pgfqpoint{4.601243in}{3.237391in}}{\pgfqpoint{4.590193in}{3.237391in}}%
\pgfpathcurveto{\pgfqpoint{4.579143in}{3.237391in}}{\pgfqpoint{4.568544in}{3.233000in}}{\pgfqpoint{4.560730in}{3.225187in}}%
\pgfpathcurveto{\pgfqpoint{4.552917in}{3.217373in}}{\pgfqpoint{4.548526in}{3.206774in}}{\pgfqpoint{4.548526in}{3.195724in}}%
\pgfpathcurveto{\pgfqpoint{4.548526in}{3.184674in}}{\pgfqpoint{4.552917in}{3.174075in}}{\pgfqpoint{4.560730in}{3.166261in}}%
\pgfpathcurveto{\pgfqpoint{4.568544in}{3.158447in}}{\pgfqpoint{4.579143in}{3.154057in}}{\pgfqpoint{4.590193in}{3.154057in}}%
\pgfpathclose%
\pgfusepath{stroke,fill}%
\end{pgfscope}%
\begin{pgfscope}%
\pgfpathrectangle{\pgfqpoint{0.787074in}{0.548769in}}{\pgfqpoint{5.062926in}{3.102590in}}%
\pgfusepath{clip}%
\pgfsetbuttcap%
\pgfsetroundjoin%
\definecolor{currentfill}{rgb}{1.000000,0.498039,0.054902}%
\pgfsetfillcolor{currentfill}%
\pgfsetlinewidth{1.003750pt}%
\definecolor{currentstroke}{rgb}{1.000000,0.498039,0.054902}%
\pgfsetstrokecolor{currentstroke}%
\pgfsetdash{}{0pt}%
\pgfpathmoveto{\pgfqpoint{4.096064in}{2.232087in}}%
\pgfpathcurveto{\pgfqpoint{4.107114in}{2.232087in}}{\pgfqpoint{4.117713in}{2.236477in}}{\pgfqpoint{4.125527in}{2.244291in}}%
\pgfpathcurveto{\pgfqpoint{4.133340in}{2.252104in}}{\pgfqpoint{4.137731in}{2.262703in}}{\pgfqpoint{4.137731in}{2.273753in}}%
\pgfpathcurveto{\pgfqpoint{4.137731in}{2.284804in}}{\pgfqpoint{4.133340in}{2.295403in}}{\pgfqpoint{4.125527in}{2.303216in}}%
\pgfpathcurveto{\pgfqpoint{4.117713in}{2.311030in}}{\pgfqpoint{4.107114in}{2.315420in}}{\pgfqpoint{4.096064in}{2.315420in}}%
\pgfpathcurveto{\pgfqpoint{4.085014in}{2.315420in}}{\pgfqpoint{4.074415in}{2.311030in}}{\pgfqpoint{4.066601in}{2.303216in}}%
\pgfpathcurveto{\pgfqpoint{4.058788in}{2.295403in}}{\pgfqpoint{4.054397in}{2.284804in}}{\pgfqpoint{4.054397in}{2.273753in}}%
\pgfpathcurveto{\pgfqpoint{4.054397in}{2.262703in}}{\pgfqpoint{4.058788in}{2.252104in}}{\pgfqpoint{4.066601in}{2.244291in}}%
\pgfpathcurveto{\pgfqpoint{4.074415in}{2.236477in}}{\pgfqpoint{4.085014in}{2.232087in}}{\pgfqpoint{4.096064in}{2.232087in}}%
\pgfpathclose%
\pgfusepath{stroke,fill}%
\end{pgfscope}%
\begin{pgfscope}%
\pgfpathrectangle{\pgfqpoint{0.787074in}{0.548769in}}{\pgfqpoint{5.062926in}{3.102590in}}%
\pgfusepath{clip}%
\pgfsetbuttcap%
\pgfsetroundjoin%
\definecolor{currentfill}{rgb}{0.121569,0.466667,0.705882}%
\pgfsetfillcolor{currentfill}%
\pgfsetlinewidth{1.003750pt}%
\definecolor{currentstroke}{rgb}{0.121569,0.466667,0.705882}%
\pgfsetstrokecolor{currentstroke}%
\pgfsetdash{}{0pt}%
\pgfpathmoveto{\pgfqpoint{1.017211in}{0.648132in}}%
\pgfpathcurveto{\pgfqpoint{1.028261in}{0.648132in}}{\pgfqpoint{1.038860in}{0.652523in}}{\pgfqpoint{1.046674in}{0.660336in}}%
\pgfpathcurveto{\pgfqpoint{1.054487in}{0.668150in}}{\pgfqpoint{1.058878in}{0.678749in}}{\pgfqpoint{1.058878in}{0.689799in}}%
\pgfpathcurveto{\pgfqpoint{1.058878in}{0.700849in}}{\pgfqpoint{1.054487in}{0.711448in}}{\pgfqpoint{1.046674in}{0.719262in}}%
\pgfpathcurveto{\pgfqpoint{1.038860in}{0.727075in}}{\pgfqpoint{1.028261in}{0.731466in}}{\pgfqpoint{1.017211in}{0.731466in}}%
\pgfpathcurveto{\pgfqpoint{1.006161in}{0.731466in}}{\pgfqpoint{0.995562in}{0.727075in}}{\pgfqpoint{0.987748in}{0.719262in}}%
\pgfpathcurveto{\pgfqpoint{0.979934in}{0.711448in}}{\pgfqpoint{0.975544in}{0.700849in}}{\pgfqpoint{0.975544in}{0.689799in}}%
\pgfpathcurveto{\pgfqpoint{0.975544in}{0.678749in}}{\pgfqpoint{0.979934in}{0.668150in}}{\pgfqpoint{0.987748in}{0.660336in}}%
\pgfpathcurveto{\pgfqpoint{0.995562in}{0.652523in}}{\pgfqpoint{1.006161in}{0.648132in}}{\pgfqpoint{1.017211in}{0.648132in}}%
\pgfpathclose%
\pgfusepath{stroke,fill}%
\end{pgfscope}%
\begin{pgfscope}%
\pgfpathrectangle{\pgfqpoint{0.787074in}{0.548769in}}{\pgfqpoint{5.062926in}{3.102590in}}%
\pgfusepath{clip}%
\pgfsetbuttcap%
\pgfsetroundjoin%
\definecolor{currentfill}{rgb}{0.121569,0.466667,0.705882}%
\pgfsetfillcolor{currentfill}%
\pgfsetlinewidth{1.003750pt}%
\definecolor{currentstroke}{rgb}{0.121569,0.466667,0.705882}%
\pgfsetstrokecolor{currentstroke}%
\pgfsetdash{}{0pt}%
\pgfpathmoveto{\pgfqpoint{1.306508in}{0.784785in}}%
\pgfpathcurveto{\pgfqpoint{1.317558in}{0.784785in}}{\pgfqpoint{1.328157in}{0.789175in}}{\pgfqpoint{1.335970in}{0.796989in}}%
\pgfpathcurveto{\pgfqpoint{1.343784in}{0.804802in}}{\pgfqpoint{1.348174in}{0.815401in}}{\pgfqpoint{1.348174in}{0.826451in}}%
\pgfpathcurveto{\pgfqpoint{1.348174in}{0.837502in}}{\pgfqpoint{1.343784in}{0.848101in}}{\pgfqpoint{1.335970in}{0.855914in}}%
\pgfpathcurveto{\pgfqpoint{1.328157in}{0.863728in}}{\pgfqpoint{1.317558in}{0.868118in}}{\pgfqpoint{1.306508in}{0.868118in}}%
\pgfpathcurveto{\pgfqpoint{1.295458in}{0.868118in}}{\pgfqpoint{1.284859in}{0.863728in}}{\pgfqpoint{1.277045in}{0.855914in}}%
\pgfpathcurveto{\pgfqpoint{1.269231in}{0.848101in}}{\pgfqpoint{1.264841in}{0.837502in}}{\pgfqpoint{1.264841in}{0.826451in}}%
\pgfpathcurveto{\pgfqpoint{1.264841in}{0.815401in}}{\pgfqpoint{1.269231in}{0.804802in}}{\pgfqpoint{1.277045in}{0.796989in}}%
\pgfpathcurveto{\pgfqpoint{1.284859in}{0.789175in}}{\pgfqpoint{1.295458in}{0.784785in}}{\pgfqpoint{1.306508in}{0.784785in}}%
\pgfpathclose%
\pgfusepath{stroke,fill}%
\end{pgfscope}%
\begin{pgfscope}%
\pgfpathrectangle{\pgfqpoint{0.787074in}{0.548769in}}{\pgfqpoint{5.062926in}{3.102590in}}%
\pgfusepath{clip}%
\pgfsetbuttcap%
\pgfsetroundjoin%
\definecolor{currentfill}{rgb}{0.121569,0.466667,0.705882}%
\pgfsetfillcolor{currentfill}%
\pgfsetlinewidth{1.003750pt}%
\definecolor{currentstroke}{rgb}{0.121569,0.466667,0.705882}%
\pgfsetstrokecolor{currentstroke}%
\pgfsetdash{}{0pt}%
\pgfpathmoveto{\pgfqpoint{4.985025in}{2.639480in}}%
\pgfpathcurveto{\pgfqpoint{4.996075in}{2.639480in}}{\pgfqpoint{5.006674in}{2.643870in}}{\pgfqpoint{5.014488in}{2.651684in}}%
\pgfpathcurveto{\pgfqpoint{5.022302in}{2.659498in}}{\pgfqpoint{5.026692in}{2.670097in}}{\pgfqpoint{5.026692in}{2.681147in}}%
\pgfpathcurveto{\pgfqpoint{5.026692in}{2.692197in}}{\pgfqpoint{5.022302in}{2.702796in}}{\pgfqpoint{5.014488in}{2.710610in}}%
\pgfpathcurveto{\pgfqpoint{5.006674in}{2.718423in}}{\pgfqpoint{4.996075in}{2.722813in}}{\pgfqpoint{4.985025in}{2.722813in}}%
\pgfpathcurveto{\pgfqpoint{4.973975in}{2.722813in}}{\pgfqpoint{4.963376in}{2.718423in}}{\pgfqpoint{4.955563in}{2.710610in}}%
\pgfpathcurveto{\pgfqpoint{4.947749in}{2.702796in}}{\pgfqpoint{4.943359in}{2.692197in}}{\pgfqpoint{4.943359in}{2.681147in}}%
\pgfpathcurveto{\pgfqpoint{4.943359in}{2.670097in}}{\pgfqpoint{4.947749in}{2.659498in}}{\pgfqpoint{4.955563in}{2.651684in}}%
\pgfpathcurveto{\pgfqpoint{4.963376in}{2.643870in}}{\pgfqpoint{4.973975in}{2.639480in}}{\pgfqpoint{4.985025in}{2.639480in}}%
\pgfpathclose%
\pgfusepath{stroke,fill}%
\end{pgfscope}%
\begin{pgfscope}%
\pgfpathrectangle{\pgfqpoint{0.787074in}{0.548769in}}{\pgfqpoint{5.062926in}{3.102590in}}%
\pgfusepath{clip}%
\pgfsetbuttcap%
\pgfsetroundjoin%
\definecolor{currentfill}{rgb}{0.121569,0.466667,0.705882}%
\pgfsetfillcolor{currentfill}%
\pgfsetlinewidth{1.003750pt}%
\definecolor{currentstroke}{rgb}{0.121569,0.466667,0.705882}%
\pgfsetstrokecolor{currentstroke}%
\pgfsetdash{}{0pt}%
\pgfpathmoveto{\pgfqpoint{1.017233in}{0.648148in}}%
\pgfpathcurveto{\pgfqpoint{1.028283in}{0.648148in}}{\pgfqpoint{1.038882in}{0.652539in}}{\pgfqpoint{1.046696in}{0.660352in}}%
\pgfpathcurveto{\pgfqpoint{1.054509in}{0.668166in}}{\pgfqpoint{1.058900in}{0.678765in}}{\pgfqpoint{1.058900in}{0.689815in}}%
\pgfpathcurveto{\pgfqpoint{1.058900in}{0.700865in}}{\pgfqpoint{1.054509in}{0.711464in}}{\pgfqpoint{1.046696in}{0.719278in}}%
\pgfpathcurveto{\pgfqpoint{1.038882in}{0.727091in}}{\pgfqpoint{1.028283in}{0.731482in}}{\pgfqpoint{1.017233in}{0.731482in}}%
\pgfpathcurveto{\pgfqpoint{1.006183in}{0.731482in}}{\pgfqpoint{0.995584in}{0.727091in}}{\pgfqpoint{0.987770in}{0.719278in}}%
\pgfpathcurveto{\pgfqpoint{0.979957in}{0.711464in}}{\pgfqpoint{0.975566in}{0.700865in}}{\pgfqpoint{0.975566in}{0.689815in}}%
\pgfpathcurveto{\pgfqpoint{0.975566in}{0.678765in}}{\pgfqpoint{0.979957in}{0.668166in}}{\pgfqpoint{0.987770in}{0.660352in}}%
\pgfpathcurveto{\pgfqpoint{0.995584in}{0.652539in}}{\pgfqpoint{1.006183in}{0.648148in}}{\pgfqpoint{1.017233in}{0.648148in}}%
\pgfpathclose%
\pgfusepath{stroke,fill}%
\end{pgfscope}%
\begin{pgfscope}%
\pgfpathrectangle{\pgfqpoint{0.787074in}{0.548769in}}{\pgfqpoint{5.062926in}{3.102590in}}%
\pgfusepath{clip}%
\pgfsetbuttcap%
\pgfsetroundjoin%
\definecolor{currentfill}{rgb}{0.121569,0.466667,0.705882}%
\pgfsetfillcolor{currentfill}%
\pgfsetlinewidth{1.003750pt}%
\definecolor{currentstroke}{rgb}{0.121569,0.466667,0.705882}%
\pgfsetstrokecolor{currentstroke}%
\pgfsetdash{}{0pt}%
\pgfpathmoveto{\pgfqpoint{1.017236in}{0.648149in}}%
\pgfpathcurveto{\pgfqpoint{1.028286in}{0.648149in}}{\pgfqpoint{1.038885in}{0.652540in}}{\pgfqpoint{1.046699in}{0.660353in}}%
\pgfpathcurveto{\pgfqpoint{1.054512in}{0.668167in}}{\pgfqpoint{1.058903in}{0.678766in}}{\pgfqpoint{1.058903in}{0.689816in}}%
\pgfpathcurveto{\pgfqpoint{1.058903in}{0.700866in}}{\pgfqpoint{1.054512in}{0.711465in}}{\pgfqpoint{1.046699in}{0.719279in}}%
\pgfpathcurveto{\pgfqpoint{1.038885in}{0.727093in}}{\pgfqpoint{1.028286in}{0.731483in}}{\pgfqpoint{1.017236in}{0.731483in}}%
\pgfpathcurveto{\pgfqpoint{1.006186in}{0.731483in}}{\pgfqpoint{0.995587in}{0.727093in}}{\pgfqpoint{0.987773in}{0.719279in}}%
\pgfpathcurveto{\pgfqpoint{0.979960in}{0.711465in}}{\pgfqpoint{0.975569in}{0.700866in}}{\pgfqpoint{0.975569in}{0.689816in}}%
\pgfpathcurveto{\pgfqpoint{0.975569in}{0.678766in}}{\pgfqpoint{0.979960in}{0.668167in}}{\pgfqpoint{0.987773in}{0.660353in}}%
\pgfpathcurveto{\pgfqpoint{0.995587in}{0.652540in}}{\pgfqpoint{1.006186in}{0.648149in}}{\pgfqpoint{1.017236in}{0.648149in}}%
\pgfpathclose%
\pgfusepath{stroke,fill}%
\end{pgfscope}%
\begin{pgfscope}%
\pgfpathrectangle{\pgfqpoint{0.787074in}{0.548769in}}{\pgfqpoint{5.062926in}{3.102590in}}%
\pgfusepath{clip}%
\pgfsetbuttcap%
\pgfsetroundjoin%
\definecolor{currentfill}{rgb}{0.121569,0.466667,0.705882}%
\pgfsetfillcolor{currentfill}%
\pgfsetlinewidth{1.003750pt}%
\definecolor{currentstroke}{rgb}{0.121569,0.466667,0.705882}%
\pgfsetstrokecolor{currentstroke}%
\pgfsetdash{}{0pt}%
\pgfpathmoveto{\pgfqpoint{4.109728in}{2.298244in}}%
\pgfpathcurveto{\pgfqpoint{4.120778in}{2.298244in}}{\pgfqpoint{4.131377in}{2.302634in}}{\pgfqpoint{4.139191in}{2.310448in}}%
\pgfpathcurveto{\pgfqpoint{4.147005in}{2.318261in}}{\pgfqpoint{4.151395in}{2.328860in}}{\pgfqpoint{4.151395in}{2.339911in}}%
\pgfpathcurveto{\pgfqpoint{4.151395in}{2.350961in}}{\pgfqpoint{4.147005in}{2.361560in}}{\pgfqpoint{4.139191in}{2.369373in}}%
\pgfpathcurveto{\pgfqpoint{4.131377in}{2.377187in}}{\pgfqpoint{4.120778in}{2.381577in}}{\pgfqpoint{4.109728in}{2.381577in}}%
\pgfpathcurveto{\pgfqpoint{4.098678in}{2.381577in}}{\pgfqpoint{4.088079in}{2.377187in}}{\pgfqpoint{4.080265in}{2.369373in}}%
\pgfpathcurveto{\pgfqpoint{4.072452in}{2.361560in}}{\pgfqpoint{4.068062in}{2.350961in}}{\pgfqpoint{4.068062in}{2.339911in}}%
\pgfpathcurveto{\pgfqpoint{4.068062in}{2.328860in}}{\pgfqpoint{4.072452in}{2.318261in}}{\pgfqpoint{4.080265in}{2.310448in}}%
\pgfpathcurveto{\pgfqpoint{4.088079in}{2.302634in}}{\pgfqpoint{4.098678in}{2.298244in}}{\pgfqpoint{4.109728in}{2.298244in}}%
\pgfpathclose%
\pgfusepath{stroke,fill}%
\end{pgfscope}%
\begin{pgfscope}%
\pgfpathrectangle{\pgfqpoint{0.787074in}{0.548769in}}{\pgfqpoint{5.062926in}{3.102590in}}%
\pgfusepath{clip}%
\pgfsetbuttcap%
\pgfsetroundjoin%
\definecolor{currentfill}{rgb}{1.000000,0.498039,0.054902}%
\pgfsetfillcolor{currentfill}%
\pgfsetlinewidth{1.003750pt}%
\definecolor{currentstroke}{rgb}{1.000000,0.498039,0.054902}%
\pgfsetstrokecolor{currentstroke}%
\pgfsetdash{}{0pt}%
\pgfpathmoveto{\pgfqpoint{4.924578in}{2.885396in}}%
\pgfpathcurveto{\pgfqpoint{4.935628in}{2.885396in}}{\pgfqpoint{4.946227in}{2.889786in}}{\pgfqpoint{4.954041in}{2.897600in}}%
\pgfpathcurveto{\pgfqpoint{4.961854in}{2.905413in}}{\pgfqpoint{4.966245in}{2.916012in}}{\pgfqpoint{4.966245in}{2.927063in}}%
\pgfpathcurveto{\pgfqpoint{4.966245in}{2.938113in}}{\pgfqpoint{4.961854in}{2.948712in}}{\pgfqpoint{4.954041in}{2.956525in}}%
\pgfpathcurveto{\pgfqpoint{4.946227in}{2.964339in}}{\pgfqpoint{4.935628in}{2.968729in}}{\pgfqpoint{4.924578in}{2.968729in}}%
\pgfpathcurveto{\pgfqpoint{4.913528in}{2.968729in}}{\pgfqpoint{4.902929in}{2.964339in}}{\pgfqpoint{4.895115in}{2.956525in}}%
\pgfpathcurveto{\pgfqpoint{4.887302in}{2.948712in}}{\pgfqpoint{4.882911in}{2.938113in}}{\pgfqpoint{4.882911in}{2.927063in}}%
\pgfpathcurveto{\pgfqpoint{4.882911in}{2.916012in}}{\pgfqpoint{4.887302in}{2.905413in}}{\pgfqpoint{4.895115in}{2.897600in}}%
\pgfpathcurveto{\pgfqpoint{4.902929in}{2.889786in}}{\pgfqpoint{4.913528in}{2.885396in}}{\pgfqpoint{4.924578in}{2.885396in}}%
\pgfpathclose%
\pgfusepath{stroke,fill}%
\end{pgfscope}%
\begin{pgfscope}%
\pgfpathrectangle{\pgfqpoint{0.787074in}{0.548769in}}{\pgfqpoint{5.062926in}{3.102590in}}%
\pgfusepath{clip}%
\pgfsetbuttcap%
\pgfsetroundjoin%
\definecolor{currentfill}{rgb}{1.000000,0.498039,0.054902}%
\pgfsetfillcolor{currentfill}%
\pgfsetlinewidth{1.003750pt}%
\definecolor{currentstroke}{rgb}{1.000000,0.498039,0.054902}%
\pgfsetstrokecolor{currentstroke}%
\pgfsetdash{}{0pt}%
\pgfpathmoveto{\pgfqpoint{3.491970in}{2.303844in}}%
\pgfpathcurveto{\pgfqpoint{3.503020in}{2.303844in}}{\pgfqpoint{3.513619in}{2.308235in}}{\pgfqpoint{3.521433in}{2.316048in}}%
\pgfpathcurveto{\pgfqpoint{3.529246in}{2.323862in}}{\pgfqpoint{3.533637in}{2.334461in}}{\pgfqpoint{3.533637in}{2.345511in}}%
\pgfpathcurveto{\pgfqpoint{3.533637in}{2.356561in}}{\pgfqpoint{3.529246in}{2.367160in}}{\pgfqpoint{3.521433in}{2.374974in}}%
\pgfpathcurveto{\pgfqpoint{3.513619in}{2.382787in}}{\pgfqpoint{3.503020in}{2.387178in}}{\pgfqpoint{3.491970in}{2.387178in}}%
\pgfpathcurveto{\pgfqpoint{3.480920in}{2.387178in}}{\pgfqpoint{3.470321in}{2.382787in}}{\pgfqpoint{3.462507in}{2.374974in}}%
\pgfpathcurveto{\pgfqpoint{3.454694in}{2.367160in}}{\pgfqpoint{3.450303in}{2.356561in}}{\pgfqpoint{3.450303in}{2.345511in}}%
\pgfpathcurveto{\pgfqpoint{3.450303in}{2.334461in}}{\pgfqpoint{3.454694in}{2.323862in}}{\pgfqpoint{3.462507in}{2.316048in}}%
\pgfpathcurveto{\pgfqpoint{3.470321in}{2.308235in}}{\pgfqpoint{3.480920in}{2.303844in}}{\pgfqpoint{3.491970in}{2.303844in}}%
\pgfpathclose%
\pgfusepath{stroke,fill}%
\end{pgfscope}%
\begin{pgfscope}%
\pgfpathrectangle{\pgfqpoint{0.787074in}{0.548769in}}{\pgfqpoint{5.062926in}{3.102590in}}%
\pgfusepath{clip}%
\pgfsetbuttcap%
\pgfsetroundjoin%
\definecolor{currentfill}{rgb}{1.000000,0.498039,0.054902}%
\pgfsetfillcolor{currentfill}%
\pgfsetlinewidth{1.003750pt}%
\definecolor{currentstroke}{rgb}{1.000000,0.498039,0.054902}%
\pgfsetstrokecolor{currentstroke}%
\pgfsetdash{}{0pt}%
\pgfpathmoveto{\pgfqpoint{4.498161in}{2.924913in}}%
\pgfpathcurveto{\pgfqpoint{4.509211in}{2.924913in}}{\pgfqpoint{4.519811in}{2.929304in}}{\pgfqpoint{4.527624in}{2.937117in}}%
\pgfpathcurveto{\pgfqpoint{4.535438in}{2.944931in}}{\pgfqpoint{4.539828in}{2.955530in}}{\pgfqpoint{4.539828in}{2.966580in}}%
\pgfpathcurveto{\pgfqpoint{4.539828in}{2.977630in}}{\pgfqpoint{4.535438in}{2.988229in}}{\pgfqpoint{4.527624in}{2.996043in}}%
\pgfpathcurveto{\pgfqpoint{4.519811in}{3.003856in}}{\pgfqpoint{4.509211in}{3.008247in}}{\pgfqpoint{4.498161in}{3.008247in}}%
\pgfpathcurveto{\pgfqpoint{4.487111in}{3.008247in}}{\pgfqpoint{4.476512in}{3.003856in}}{\pgfqpoint{4.468699in}{2.996043in}}%
\pgfpathcurveto{\pgfqpoint{4.460885in}{2.988229in}}{\pgfqpoint{4.456495in}{2.977630in}}{\pgfqpoint{4.456495in}{2.966580in}}%
\pgfpathcurveto{\pgfqpoint{4.456495in}{2.955530in}}{\pgfqpoint{4.460885in}{2.944931in}}{\pgfqpoint{4.468699in}{2.937117in}}%
\pgfpathcurveto{\pgfqpoint{4.476512in}{2.929304in}}{\pgfqpoint{4.487111in}{2.924913in}}{\pgfqpoint{4.498161in}{2.924913in}}%
\pgfpathclose%
\pgfusepath{stroke,fill}%
\end{pgfscope}%
\begin{pgfscope}%
\pgfpathrectangle{\pgfqpoint{0.787074in}{0.548769in}}{\pgfqpoint{5.062926in}{3.102590in}}%
\pgfusepath{clip}%
\pgfsetbuttcap%
\pgfsetroundjoin%
\definecolor{currentfill}{rgb}{1.000000,0.498039,0.054902}%
\pgfsetfillcolor{currentfill}%
\pgfsetlinewidth{1.003750pt}%
\definecolor{currentstroke}{rgb}{1.000000,0.498039,0.054902}%
\pgfsetstrokecolor{currentstroke}%
\pgfsetdash{}{0pt}%
\pgfpathmoveto{\pgfqpoint{3.793973in}{2.712046in}}%
\pgfpathcurveto{\pgfqpoint{3.805023in}{2.712046in}}{\pgfqpoint{3.815622in}{2.716436in}}{\pgfqpoint{3.823435in}{2.724249in}}%
\pgfpathcurveto{\pgfqpoint{3.831249in}{2.732063in}}{\pgfqpoint{3.835639in}{2.742662in}}{\pgfqpoint{3.835639in}{2.753712in}}%
\pgfpathcurveto{\pgfqpoint{3.835639in}{2.764762in}}{\pgfqpoint{3.831249in}{2.775361in}}{\pgfqpoint{3.823435in}{2.783175in}}%
\pgfpathcurveto{\pgfqpoint{3.815622in}{2.790989in}}{\pgfqpoint{3.805023in}{2.795379in}}{\pgfqpoint{3.793973in}{2.795379in}}%
\pgfpathcurveto{\pgfqpoint{3.782922in}{2.795379in}}{\pgfqpoint{3.772323in}{2.790989in}}{\pgfqpoint{3.764510in}{2.783175in}}%
\pgfpathcurveto{\pgfqpoint{3.756696in}{2.775361in}}{\pgfqpoint{3.752306in}{2.764762in}}{\pgfqpoint{3.752306in}{2.753712in}}%
\pgfpathcurveto{\pgfqpoint{3.752306in}{2.742662in}}{\pgfqpoint{3.756696in}{2.732063in}}{\pgfqpoint{3.764510in}{2.724249in}}%
\pgfpathcurveto{\pgfqpoint{3.772323in}{2.716436in}}{\pgfqpoint{3.782922in}{2.712046in}}{\pgfqpoint{3.793973in}{2.712046in}}%
\pgfpathclose%
\pgfusepath{stroke,fill}%
\end{pgfscope}%
\begin{pgfscope}%
\pgfpathrectangle{\pgfqpoint{0.787074in}{0.548769in}}{\pgfqpoint{5.062926in}{3.102590in}}%
\pgfusepath{clip}%
\pgfsetbuttcap%
\pgfsetroundjoin%
\definecolor{currentfill}{rgb}{0.121569,0.466667,0.705882}%
\pgfsetfillcolor{currentfill}%
\pgfsetlinewidth{1.003750pt}%
\definecolor{currentstroke}{rgb}{0.121569,0.466667,0.705882}%
\pgfsetstrokecolor{currentstroke}%
\pgfsetdash{}{0pt}%
\pgfpathmoveto{\pgfqpoint{5.014441in}{3.116121in}}%
\pgfpathcurveto{\pgfqpoint{5.025492in}{3.116121in}}{\pgfqpoint{5.036091in}{3.120511in}}{\pgfqpoint{5.043904in}{3.128325in}}%
\pgfpathcurveto{\pgfqpoint{5.051718in}{3.136138in}}{\pgfqpoint{5.056108in}{3.146737in}}{\pgfqpoint{5.056108in}{3.157787in}}%
\pgfpathcurveto{\pgfqpoint{5.056108in}{3.168837in}}{\pgfqpoint{5.051718in}{3.179436in}}{\pgfqpoint{5.043904in}{3.187250in}}%
\pgfpathcurveto{\pgfqpoint{5.036091in}{3.195064in}}{\pgfqpoint{5.025492in}{3.199454in}}{\pgfqpoint{5.014441in}{3.199454in}}%
\pgfpathcurveto{\pgfqpoint{5.003391in}{3.199454in}}{\pgfqpoint{4.992792in}{3.195064in}}{\pgfqpoint{4.984979in}{3.187250in}}%
\pgfpathcurveto{\pgfqpoint{4.977165in}{3.179436in}}{\pgfqpoint{4.972775in}{3.168837in}}{\pgfqpoint{4.972775in}{3.157787in}}%
\pgfpathcurveto{\pgfqpoint{4.972775in}{3.146737in}}{\pgfqpoint{4.977165in}{3.136138in}}{\pgfqpoint{4.984979in}{3.128325in}}%
\pgfpathcurveto{\pgfqpoint{4.992792in}{3.120511in}}{\pgfqpoint{5.003391in}{3.116121in}}{\pgfqpoint{5.014441in}{3.116121in}}%
\pgfpathclose%
\pgfusepath{stroke,fill}%
\end{pgfscope}%
\begin{pgfscope}%
\pgfpathrectangle{\pgfqpoint{0.787074in}{0.548769in}}{\pgfqpoint{5.062926in}{3.102590in}}%
\pgfusepath{clip}%
\pgfsetbuttcap%
\pgfsetroundjoin%
\definecolor{currentfill}{rgb}{0.839216,0.152941,0.156863}%
\pgfsetfillcolor{currentfill}%
\pgfsetlinewidth{1.003750pt}%
\definecolor{currentstroke}{rgb}{0.839216,0.152941,0.156863}%
\pgfsetstrokecolor{currentstroke}%
\pgfsetdash{}{0pt}%
\pgfpathmoveto{\pgfqpoint{3.488057in}{3.468665in}}%
\pgfpathcurveto{\pgfqpoint{3.499107in}{3.468665in}}{\pgfqpoint{3.509706in}{3.473055in}}{\pgfqpoint{3.517520in}{3.480869in}}%
\pgfpathcurveto{\pgfqpoint{3.525333in}{3.488683in}}{\pgfqpoint{3.529724in}{3.499282in}}{\pgfqpoint{3.529724in}{3.510332in}}%
\pgfpathcurveto{\pgfqpoint{3.529724in}{3.521382in}}{\pgfqpoint{3.525333in}{3.531981in}}{\pgfqpoint{3.517520in}{3.539795in}}%
\pgfpathcurveto{\pgfqpoint{3.509706in}{3.547608in}}{\pgfqpoint{3.499107in}{3.551998in}}{\pgfqpoint{3.488057in}{3.551998in}}%
\pgfpathcurveto{\pgfqpoint{3.477007in}{3.551998in}}{\pgfqpoint{3.466408in}{3.547608in}}{\pgfqpoint{3.458594in}{3.539795in}}%
\pgfpathcurveto{\pgfqpoint{3.450781in}{3.531981in}}{\pgfqpoint{3.446390in}{3.521382in}}{\pgfqpoint{3.446390in}{3.510332in}}%
\pgfpathcurveto{\pgfqpoint{3.446390in}{3.499282in}}{\pgfqpoint{3.450781in}{3.488683in}}{\pgfqpoint{3.458594in}{3.480869in}}%
\pgfpathcurveto{\pgfqpoint{3.466408in}{3.473055in}}{\pgfqpoint{3.477007in}{3.468665in}}{\pgfqpoint{3.488057in}{3.468665in}}%
\pgfpathclose%
\pgfusepath{stroke,fill}%
\end{pgfscope}%
\begin{pgfscope}%
\pgfpathrectangle{\pgfqpoint{0.787074in}{0.548769in}}{\pgfqpoint{5.062926in}{3.102590in}}%
\pgfusepath{clip}%
\pgfsetbuttcap%
\pgfsetroundjoin%
\definecolor{currentfill}{rgb}{1.000000,0.498039,0.054902}%
\pgfsetfillcolor{currentfill}%
\pgfsetlinewidth{1.003750pt}%
\definecolor{currentstroke}{rgb}{1.000000,0.498039,0.054902}%
\pgfsetstrokecolor{currentstroke}%
\pgfsetdash{}{0pt}%
\pgfpathmoveto{\pgfqpoint{4.707491in}{3.121989in}}%
\pgfpathcurveto{\pgfqpoint{4.718541in}{3.121989in}}{\pgfqpoint{4.729140in}{3.126380in}}{\pgfqpoint{4.736953in}{3.134193in}}%
\pgfpathcurveto{\pgfqpoint{4.744767in}{3.142007in}}{\pgfqpoint{4.749157in}{3.152606in}}{\pgfqpoint{4.749157in}{3.163656in}}%
\pgfpathcurveto{\pgfqpoint{4.749157in}{3.174706in}}{\pgfqpoint{4.744767in}{3.185305in}}{\pgfqpoint{4.736953in}{3.193119in}}%
\pgfpathcurveto{\pgfqpoint{4.729140in}{3.200932in}}{\pgfqpoint{4.718541in}{3.205323in}}{\pgfqpoint{4.707491in}{3.205323in}}%
\pgfpathcurveto{\pgfqpoint{4.696441in}{3.205323in}}{\pgfqpoint{4.685842in}{3.200932in}}{\pgfqpoint{4.678028in}{3.193119in}}%
\pgfpathcurveto{\pgfqpoint{4.670214in}{3.185305in}}{\pgfqpoint{4.665824in}{3.174706in}}{\pgfqpoint{4.665824in}{3.163656in}}%
\pgfpathcurveto{\pgfqpoint{4.665824in}{3.152606in}}{\pgfqpoint{4.670214in}{3.142007in}}{\pgfqpoint{4.678028in}{3.134193in}}%
\pgfpathcurveto{\pgfqpoint{4.685842in}{3.126380in}}{\pgfqpoint{4.696441in}{3.121989in}}{\pgfqpoint{4.707491in}{3.121989in}}%
\pgfpathclose%
\pgfusepath{stroke,fill}%
\end{pgfscope}%
\begin{pgfscope}%
\pgfpathrectangle{\pgfqpoint{0.787074in}{0.548769in}}{\pgfqpoint{5.062926in}{3.102590in}}%
\pgfusepath{clip}%
\pgfsetbuttcap%
\pgfsetroundjoin%
\definecolor{currentfill}{rgb}{1.000000,0.498039,0.054902}%
\pgfsetfillcolor{currentfill}%
\pgfsetlinewidth{1.003750pt}%
\definecolor{currentstroke}{rgb}{1.000000,0.498039,0.054902}%
\pgfsetstrokecolor{currentstroke}%
\pgfsetdash{}{0pt}%
\pgfpathmoveto{\pgfqpoint{3.979505in}{2.796310in}}%
\pgfpathcurveto{\pgfqpoint{3.990555in}{2.796310in}}{\pgfqpoint{4.001154in}{2.800700in}}{\pgfqpoint{4.008968in}{2.808514in}}%
\pgfpathcurveto{\pgfqpoint{4.016782in}{2.816328in}}{\pgfqpoint{4.021172in}{2.826927in}}{\pgfqpoint{4.021172in}{2.837977in}}%
\pgfpathcurveto{\pgfqpoint{4.021172in}{2.849027in}}{\pgfqpoint{4.016782in}{2.859626in}}{\pgfqpoint{4.008968in}{2.867440in}}%
\pgfpathcurveto{\pgfqpoint{4.001154in}{2.875253in}}{\pgfqpoint{3.990555in}{2.879644in}}{\pgfqpoint{3.979505in}{2.879644in}}%
\pgfpathcurveto{\pgfqpoint{3.968455in}{2.879644in}}{\pgfqpoint{3.957856in}{2.875253in}}{\pgfqpoint{3.950042in}{2.867440in}}%
\pgfpathcurveto{\pgfqpoint{3.942229in}{2.859626in}}{\pgfqpoint{3.937839in}{2.849027in}}{\pgfqpoint{3.937839in}{2.837977in}}%
\pgfpathcurveto{\pgfqpoint{3.937839in}{2.826927in}}{\pgfqpoint{3.942229in}{2.816328in}}{\pgfqpoint{3.950042in}{2.808514in}}%
\pgfpathcurveto{\pgfqpoint{3.957856in}{2.800700in}}{\pgfqpoint{3.968455in}{2.796310in}}{\pgfqpoint{3.979505in}{2.796310in}}%
\pgfpathclose%
\pgfusepath{stroke,fill}%
\end{pgfscope}%
\begin{pgfscope}%
\pgfpathrectangle{\pgfqpoint{0.787074in}{0.548769in}}{\pgfqpoint{5.062926in}{3.102590in}}%
\pgfusepath{clip}%
\pgfsetbuttcap%
\pgfsetroundjoin%
\definecolor{currentfill}{rgb}{0.121569,0.466667,0.705882}%
\pgfsetfillcolor{currentfill}%
\pgfsetlinewidth{1.003750pt}%
\definecolor{currentstroke}{rgb}{0.121569,0.466667,0.705882}%
\pgfsetstrokecolor{currentstroke}%
\pgfsetdash{}{0pt}%
\pgfpathmoveto{\pgfqpoint{2.938512in}{2.391680in}}%
\pgfpathcurveto{\pgfqpoint{2.949562in}{2.391680in}}{\pgfqpoint{2.960162in}{2.396070in}}{\pgfqpoint{2.967975in}{2.403884in}}%
\pgfpathcurveto{\pgfqpoint{2.975789in}{2.411697in}}{\pgfqpoint{2.980179in}{2.422296in}}{\pgfqpoint{2.980179in}{2.433347in}}%
\pgfpathcurveto{\pgfqpoint{2.980179in}{2.444397in}}{\pgfqpoint{2.975789in}{2.454996in}}{\pgfqpoint{2.967975in}{2.462809in}}%
\pgfpathcurveto{\pgfqpoint{2.960162in}{2.470623in}}{\pgfqpoint{2.949562in}{2.475013in}}{\pgfqpoint{2.938512in}{2.475013in}}%
\pgfpathcurveto{\pgfqpoint{2.927462in}{2.475013in}}{\pgfqpoint{2.916863in}{2.470623in}}{\pgfqpoint{2.909050in}{2.462809in}}%
\pgfpathcurveto{\pgfqpoint{2.901236in}{2.454996in}}{\pgfqpoint{2.896846in}{2.444397in}}{\pgfqpoint{2.896846in}{2.433347in}}%
\pgfpathcurveto{\pgfqpoint{2.896846in}{2.422296in}}{\pgfqpoint{2.901236in}{2.411697in}}{\pgfqpoint{2.909050in}{2.403884in}}%
\pgfpathcurveto{\pgfqpoint{2.916863in}{2.396070in}}{\pgfqpoint{2.927462in}{2.391680in}}{\pgfqpoint{2.938512in}{2.391680in}}%
\pgfpathclose%
\pgfusepath{stroke,fill}%
\end{pgfscope}%
\begin{pgfscope}%
\pgfpathrectangle{\pgfqpoint{0.787074in}{0.548769in}}{\pgfqpoint{5.062926in}{3.102590in}}%
\pgfusepath{clip}%
\pgfsetbuttcap%
\pgfsetroundjoin%
\definecolor{currentfill}{rgb}{1.000000,0.498039,0.054902}%
\pgfsetfillcolor{currentfill}%
\pgfsetlinewidth{1.003750pt}%
\definecolor{currentstroke}{rgb}{1.000000,0.498039,0.054902}%
\pgfsetstrokecolor{currentstroke}%
\pgfsetdash{}{0pt}%
\pgfpathmoveto{\pgfqpoint{3.974883in}{2.448155in}}%
\pgfpathcurveto{\pgfqpoint{3.985934in}{2.448155in}}{\pgfqpoint{3.996533in}{2.452545in}}{\pgfqpoint{4.004346in}{2.460359in}}%
\pgfpathcurveto{\pgfqpoint{4.012160in}{2.468173in}}{\pgfqpoint{4.016550in}{2.478772in}}{\pgfqpoint{4.016550in}{2.489822in}}%
\pgfpathcurveto{\pgfqpoint{4.016550in}{2.500872in}}{\pgfqpoint{4.012160in}{2.511471in}}{\pgfqpoint{4.004346in}{2.519285in}}%
\pgfpathcurveto{\pgfqpoint{3.996533in}{2.527098in}}{\pgfqpoint{3.985934in}{2.531489in}}{\pgfqpoint{3.974883in}{2.531489in}}%
\pgfpathcurveto{\pgfqpoint{3.963833in}{2.531489in}}{\pgfqpoint{3.953234in}{2.527098in}}{\pgfqpoint{3.945421in}{2.519285in}}%
\pgfpathcurveto{\pgfqpoint{3.937607in}{2.511471in}}{\pgfqpoint{3.933217in}{2.500872in}}{\pgfqpoint{3.933217in}{2.489822in}}%
\pgfpathcurveto{\pgfqpoint{3.933217in}{2.478772in}}{\pgfqpoint{3.937607in}{2.468173in}}{\pgfqpoint{3.945421in}{2.460359in}}%
\pgfpathcurveto{\pgfqpoint{3.953234in}{2.452545in}}{\pgfqpoint{3.963833in}{2.448155in}}{\pgfqpoint{3.974883in}{2.448155in}}%
\pgfpathclose%
\pgfusepath{stroke,fill}%
\end{pgfscope}%
\begin{pgfscope}%
\pgfpathrectangle{\pgfqpoint{0.787074in}{0.548769in}}{\pgfqpoint{5.062926in}{3.102590in}}%
\pgfusepath{clip}%
\pgfsetbuttcap%
\pgfsetroundjoin%
\definecolor{currentfill}{rgb}{1.000000,0.498039,0.054902}%
\pgfsetfillcolor{currentfill}%
\pgfsetlinewidth{1.003750pt}%
\definecolor{currentstroke}{rgb}{1.000000,0.498039,0.054902}%
\pgfsetstrokecolor{currentstroke}%
\pgfsetdash{}{0pt}%
\pgfpathmoveto{\pgfqpoint{5.585662in}{3.336404in}}%
\pgfpathcurveto{\pgfqpoint{5.596712in}{3.336404in}}{\pgfqpoint{5.607311in}{3.340794in}}{\pgfqpoint{5.615125in}{3.348607in}}%
\pgfpathcurveto{\pgfqpoint{5.622938in}{3.356421in}}{\pgfqpoint{5.627329in}{3.367020in}}{\pgfqpoint{5.627329in}{3.378070in}}%
\pgfpathcurveto{\pgfqpoint{5.627329in}{3.389120in}}{\pgfqpoint{5.622938in}{3.399719in}}{\pgfqpoint{5.615125in}{3.407533in}}%
\pgfpathcurveto{\pgfqpoint{5.607311in}{3.415347in}}{\pgfqpoint{5.596712in}{3.419737in}}{\pgfqpoint{5.585662in}{3.419737in}}%
\pgfpathcurveto{\pgfqpoint{5.574612in}{3.419737in}}{\pgfqpoint{5.564013in}{3.415347in}}{\pgfqpoint{5.556199in}{3.407533in}}%
\pgfpathcurveto{\pgfqpoint{5.548386in}{3.399719in}}{\pgfqpoint{5.543995in}{3.389120in}}{\pgfqpoint{5.543995in}{3.378070in}}%
\pgfpathcurveto{\pgfqpoint{5.543995in}{3.367020in}}{\pgfqpoint{5.548386in}{3.356421in}}{\pgfqpoint{5.556199in}{3.348607in}}%
\pgfpathcurveto{\pgfqpoint{5.564013in}{3.340794in}}{\pgfqpoint{5.574612in}{3.336404in}}{\pgfqpoint{5.585662in}{3.336404in}}%
\pgfpathclose%
\pgfusepath{stroke,fill}%
\end{pgfscope}%
\begin{pgfscope}%
\pgfpathrectangle{\pgfqpoint{0.787074in}{0.548769in}}{\pgfqpoint{5.062926in}{3.102590in}}%
\pgfusepath{clip}%
\pgfsetbuttcap%
\pgfsetroundjoin%
\definecolor{currentfill}{rgb}{1.000000,0.498039,0.054902}%
\pgfsetfillcolor{currentfill}%
\pgfsetlinewidth{1.003750pt}%
\definecolor{currentstroke}{rgb}{1.000000,0.498039,0.054902}%
\pgfsetstrokecolor{currentstroke}%
\pgfsetdash{}{0pt}%
\pgfpathmoveto{\pgfqpoint{4.343637in}{2.728731in}}%
\pgfpathcurveto{\pgfqpoint{4.354687in}{2.728731in}}{\pgfqpoint{4.365286in}{2.733121in}}{\pgfqpoint{4.373100in}{2.740935in}}%
\pgfpathcurveto{\pgfqpoint{4.380913in}{2.748749in}}{\pgfqpoint{4.385303in}{2.759348in}}{\pgfqpoint{4.385303in}{2.770398in}}%
\pgfpathcurveto{\pgfqpoint{4.385303in}{2.781448in}}{\pgfqpoint{4.380913in}{2.792047in}}{\pgfqpoint{4.373100in}{2.799861in}}%
\pgfpathcurveto{\pgfqpoint{4.365286in}{2.807674in}}{\pgfqpoint{4.354687in}{2.812064in}}{\pgfqpoint{4.343637in}{2.812064in}}%
\pgfpathcurveto{\pgfqpoint{4.332587in}{2.812064in}}{\pgfqpoint{4.321988in}{2.807674in}}{\pgfqpoint{4.314174in}{2.799861in}}%
\pgfpathcurveto{\pgfqpoint{4.306360in}{2.792047in}}{\pgfqpoint{4.301970in}{2.781448in}}{\pgfqpoint{4.301970in}{2.770398in}}%
\pgfpathcurveto{\pgfqpoint{4.301970in}{2.759348in}}{\pgfqpoint{4.306360in}{2.748749in}}{\pgfqpoint{4.314174in}{2.740935in}}%
\pgfpathcurveto{\pgfqpoint{4.321988in}{2.733121in}}{\pgfqpoint{4.332587in}{2.728731in}}{\pgfqpoint{4.343637in}{2.728731in}}%
\pgfpathclose%
\pgfusepath{stroke,fill}%
\end{pgfscope}%
\begin{pgfscope}%
\pgfpathrectangle{\pgfqpoint{0.787074in}{0.548769in}}{\pgfqpoint{5.062926in}{3.102590in}}%
\pgfusepath{clip}%
\pgfsetbuttcap%
\pgfsetroundjoin%
\definecolor{currentfill}{rgb}{1.000000,0.498039,0.054902}%
\pgfsetfillcolor{currentfill}%
\pgfsetlinewidth{1.003750pt}%
\definecolor{currentstroke}{rgb}{1.000000,0.498039,0.054902}%
\pgfsetstrokecolor{currentstroke}%
\pgfsetdash{}{0pt}%
\pgfpathmoveto{\pgfqpoint{3.061244in}{2.523921in}}%
\pgfpathcurveto{\pgfqpoint{3.072294in}{2.523921in}}{\pgfqpoint{3.082893in}{2.528311in}}{\pgfqpoint{3.090707in}{2.536125in}}%
\pgfpathcurveto{\pgfqpoint{3.098520in}{2.543939in}}{\pgfqpoint{3.102910in}{2.554538in}}{\pgfqpoint{3.102910in}{2.565588in}}%
\pgfpathcurveto{\pgfqpoint{3.102910in}{2.576638in}}{\pgfqpoint{3.098520in}{2.587237in}}{\pgfqpoint{3.090707in}{2.595051in}}%
\pgfpathcurveto{\pgfqpoint{3.082893in}{2.602864in}}{\pgfqpoint{3.072294in}{2.607254in}}{\pgfqpoint{3.061244in}{2.607254in}}%
\pgfpathcurveto{\pgfqpoint{3.050194in}{2.607254in}}{\pgfqpoint{3.039595in}{2.602864in}}{\pgfqpoint{3.031781in}{2.595051in}}%
\pgfpathcurveto{\pgfqpoint{3.023967in}{2.587237in}}{\pgfqpoint{3.019577in}{2.576638in}}{\pgfqpoint{3.019577in}{2.565588in}}%
\pgfpathcurveto{\pgfqpoint{3.019577in}{2.554538in}}{\pgfqpoint{3.023967in}{2.543939in}}{\pgfqpoint{3.031781in}{2.536125in}}%
\pgfpathcurveto{\pgfqpoint{3.039595in}{2.528311in}}{\pgfqpoint{3.050194in}{2.523921in}}{\pgfqpoint{3.061244in}{2.523921in}}%
\pgfpathclose%
\pgfusepath{stroke,fill}%
\end{pgfscope}%
\begin{pgfscope}%
\pgfpathrectangle{\pgfqpoint{0.787074in}{0.548769in}}{\pgfqpoint{5.062926in}{3.102590in}}%
\pgfusepath{clip}%
\pgfsetbuttcap%
\pgfsetroundjoin%
\definecolor{currentfill}{rgb}{0.121569,0.466667,0.705882}%
\pgfsetfillcolor{currentfill}%
\pgfsetlinewidth{1.003750pt}%
\definecolor{currentstroke}{rgb}{0.121569,0.466667,0.705882}%
\pgfsetstrokecolor{currentstroke}%
\pgfsetdash{}{0pt}%
\pgfpathmoveto{\pgfqpoint{4.319284in}{2.939541in}}%
\pgfpathcurveto{\pgfqpoint{4.330334in}{2.939541in}}{\pgfqpoint{4.340933in}{2.943931in}}{\pgfqpoint{4.348746in}{2.951745in}}%
\pgfpathcurveto{\pgfqpoint{4.356560in}{2.959559in}}{\pgfqpoint{4.360950in}{2.970158in}}{\pgfqpoint{4.360950in}{2.981208in}}%
\pgfpathcurveto{\pgfqpoint{4.360950in}{2.992258in}}{\pgfqpoint{4.356560in}{3.002857in}}{\pgfqpoint{4.348746in}{3.010671in}}%
\pgfpathcurveto{\pgfqpoint{4.340933in}{3.018484in}}{\pgfqpoint{4.330334in}{3.022874in}}{\pgfqpoint{4.319284in}{3.022874in}}%
\pgfpathcurveto{\pgfqpoint{4.308233in}{3.022874in}}{\pgfqpoint{4.297634in}{3.018484in}}{\pgfqpoint{4.289821in}{3.010671in}}%
\pgfpathcurveto{\pgfqpoint{4.282007in}{3.002857in}}{\pgfqpoint{4.277617in}{2.992258in}}{\pgfqpoint{4.277617in}{2.981208in}}%
\pgfpathcurveto{\pgfqpoint{4.277617in}{2.970158in}}{\pgfqpoint{4.282007in}{2.959559in}}{\pgfqpoint{4.289821in}{2.951745in}}%
\pgfpathcurveto{\pgfqpoint{4.297634in}{2.943931in}}{\pgfqpoint{4.308233in}{2.939541in}}{\pgfqpoint{4.319284in}{2.939541in}}%
\pgfpathclose%
\pgfusepath{stroke,fill}%
\end{pgfscope}%
\begin{pgfscope}%
\pgfpathrectangle{\pgfqpoint{0.787074in}{0.548769in}}{\pgfqpoint{5.062926in}{3.102590in}}%
\pgfusepath{clip}%
\pgfsetbuttcap%
\pgfsetroundjoin%
\definecolor{currentfill}{rgb}{1.000000,0.498039,0.054902}%
\pgfsetfillcolor{currentfill}%
\pgfsetlinewidth{1.003750pt}%
\definecolor{currentstroke}{rgb}{1.000000,0.498039,0.054902}%
\pgfsetstrokecolor{currentstroke}%
\pgfsetdash{}{0pt}%
\pgfpathmoveto{\pgfqpoint{4.937691in}{2.934357in}}%
\pgfpathcurveto{\pgfqpoint{4.948742in}{2.934357in}}{\pgfqpoint{4.959341in}{2.938747in}}{\pgfqpoint{4.967154in}{2.946561in}}%
\pgfpathcurveto{\pgfqpoint{4.974968in}{2.954375in}}{\pgfqpoint{4.979358in}{2.964974in}}{\pgfqpoint{4.979358in}{2.976024in}}%
\pgfpathcurveto{\pgfqpoint{4.979358in}{2.987074in}}{\pgfqpoint{4.974968in}{2.997673in}}{\pgfqpoint{4.967154in}{3.005487in}}%
\pgfpathcurveto{\pgfqpoint{4.959341in}{3.013300in}}{\pgfqpoint{4.948742in}{3.017690in}}{\pgfqpoint{4.937691in}{3.017690in}}%
\pgfpathcurveto{\pgfqpoint{4.926641in}{3.017690in}}{\pgfqpoint{4.916042in}{3.013300in}}{\pgfqpoint{4.908229in}{3.005487in}}%
\pgfpathcurveto{\pgfqpoint{4.900415in}{2.997673in}}{\pgfqpoint{4.896025in}{2.987074in}}{\pgfqpoint{4.896025in}{2.976024in}}%
\pgfpathcurveto{\pgfqpoint{4.896025in}{2.964974in}}{\pgfqpoint{4.900415in}{2.954375in}}{\pgfqpoint{4.908229in}{2.946561in}}%
\pgfpathcurveto{\pgfqpoint{4.916042in}{2.938747in}}{\pgfqpoint{4.926641in}{2.934357in}}{\pgfqpoint{4.937691in}{2.934357in}}%
\pgfpathclose%
\pgfusepath{stroke,fill}%
\end{pgfscope}%
\begin{pgfscope}%
\pgfpathrectangle{\pgfqpoint{0.787074in}{0.548769in}}{\pgfqpoint{5.062926in}{3.102590in}}%
\pgfusepath{clip}%
\pgfsetbuttcap%
\pgfsetroundjoin%
\definecolor{currentfill}{rgb}{1.000000,0.498039,0.054902}%
\pgfsetfillcolor{currentfill}%
\pgfsetlinewidth{1.003750pt}%
\definecolor{currentstroke}{rgb}{1.000000,0.498039,0.054902}%
\pgfsetstrokecolor{currentstroke}%
\pgfsetdash{}{0pt}%
\pgfpathmoveto{\pgfqpoint{4.726446in}{3.327747in}}%
\pgfpathcurveto{\pgfqpoint{4.737497in}{3.327747in}}{\pgfqpoint{4.748096in}{3.332137in}}{\pgfqpoint{4.755909in}{3.339951in}}%
\pgfpathcurveto{\pgfqpoint{4.763723in}{3.347764in}}{\pgfqpoint{4.768113in}{3.358363in}}{\pgfqpoint{4.768113in}{3.369413in}}%
\pgfpathcurveto{\pgfqpoint{4.768113in}{3.380463in}}{\pgfqpoint{4.763723in}{3.391063in}}{\pgfqpoint{4.755909in}{3.398876in}}%
\pgfpathcurveto{\pgfqpoint{4.748096in}{3.406690in}}{\pgfqpoint{4.737497in}{3.411080in}}{\pgfqpoint{4.726446in}{3.411080in}}%
\pgfpathcurveto{\pgfqpoint{4.715396in}{3.411080in}}{\pgfqpoint{4.704797in}{3.406690in}}{\pgfqpoint{4.696984in}{3.398876in}}%
\pgfpathcurveto{\pgfqpoint{4.689170in}{3.391063in}}{\pgfqpoint{4.684780in}{3.380463in}}{\pgfqpoint{4.684780in}{3.369413in}}%
\pgfpathcurveto{\pgfqpoint{4.684780in}{3.358363in}}{\pgfqpoint{4.689170in}{3.347764in}}{\pgfqpoint{4.696984in}{3.339951in}}%
\pgfpathcurveto{\pgfqpoint{4.704797in}{3.332137in}}{\pgfqpoint{4.715396in}{3.327747in}}{\pgfqpoint{4.726446in}{3.327747in}}%
\pgfpathclose%
\pgfusepath{stroke,fill}%
\end{pgfscope}%
\begin{pgfscope}%
\pgfpathrectangle{\pgfqpoint{0.787074in}{0.548769in}}{\pgfqpoint{5.062926in}{3.102590in}}%
\pgfusepath{clip}%
\pgfsetbuttcap%
\pgfsetroundjoin%
\definecolor{currentfill}{rgb}{1.000000,0.498039,0.054902}%
\pgfsetfillcolor{currentfill}%
\pgfsetlinewidth{1.003750pt}%
\definecolor{currentstroke}{rgb}{1.000000,0.498039,0.054902}%
\pgfsetstrokecolor{currentstroke}%
\pgfsetdash{}{0pt}%
\pgfpathmoveto{\pgfqpoint{4.252711in}{2.872580in}}%
\pgfpathcurveto{\pgfqpoint{4.263761in}{2.872580in}}{\pgfqpoint{4.274360in}{2.876970in}}{\pgfqpoint{4.282174in}{2.884784in}}%
\pgfpathcurveto{\pgfqpoint{4.289988in}{2.892597in}}{\pgfqpoint{4.294378in}{2.903196in}}{\pgfqpoint{4.294378in}{2.914246in}}%
\pgfpathcurveto{\pgfqpoint{4.294378in}{2.925297in}}{\pgfqpoint{4.289988in}{2.935896in}}{\pgfqpoint{4.282174in}{2.943709in}}%
\pgfpathcurveto{\pgfqpoint{4.274360in}{2.951523in}}{\pgfqpoint{4.263761in}{2.955913in}}{\pgfqpoint{4.252711in}{2.955913in}}%
\pgfpathcurveto{\pgfqpoint{4.241661in}{2.955913in}}{\pgfqpoint{4.231062in}{2.951523in}}{\pgfqpoint{4.223248in}{2.943709in}}%
\pgfpathcurveto{\pgfqpoint{4.215435in}{2.935896in}}{\pgfqpoint{4.211045in}{2.925297in}}{\pgfqpoint{4.211045in}{2.914246in}}%
\pgfpathcurveto{\pgfqpoint{4.211045in}{2.903196in}}{\pgfqpoint{4.215435in}{2.892597in}}{\pgfqpoint{4.223248in}{2.884784in}}%
\pgfpathcurveto{\pgfqpoint{4.231062in}{2.876970in}}{\pgfqpoint{4.241661in}{2.872580in}}{\pgfqpoint{4.252711in}{2.872580in}}%
\pgfpathclose%
\pgfusepath{stroke,fill}%
\end{pgfscope}%
\begin{pgfscope}%
\pgfpathrectangle{\pgfqpoint{0.787074in}{0.548769in}}{\pgfqpoint{5.062926in}{3.102590in}}%
\pgfusepath{clip}%
\pgfsetbuttcap%
\pgfsetroundjoin%
\definecolor{currentfill}{rgb}{0.839216,0.152941,0.156863}%
\pgfsetfillcolor{currentfill}%
\pgfsetlinewidth{1.003750pt}%
\definecolor{currentstroke}{rgb}{0.839216,0.152941,0.156863}%
\pgfsetstrokecolor{currentstroke}%
\pgfsetdash{}{0pt}%
\pgfpathmoveto{\pgfqpoint{4.603746in}{2.752142in}}%
\pgfpathcurveto{\pgfqpoint{4.614796in}{2.752142in}}{\pgfqpoint{4.625395in}{2.756532in}}{\pgfqpoint{4.633209in}{2.764346in}}%
\pgfpathcurveto{\pgfqpoint{4.641022in}{2.772160in}}{\pgfqpoint{4.645413in}{2.782759in}}{\pgfqpoint{4.645413in}{2.793809in}}%
\pgfpathcurveto{\pgfqpoint{4.645413in}{2.804859in}}{\pgfqpoint{4.641022in}{2.815458in}}{\pgfqpoint{4.633209in}{2.823271in}}%
\pgfpathcurveto{\pgfqpoint{4.625395in}{2.831085in}}{\pgfqpoint{4.614796in}{2.835475in}}{\pgfqpoint{4.603746in}{2.835475in}}%
\pgfpathcurveto{\pgfqpoint{4.592696in}{2.835475in}}{\pgfqpoint{4.582097in}{2.831085in}}{\pgfqpoint{4.574283in}{2.823271in}}%
\pgfpathcurveto{\pgfqpoint{4.566470in}{2.815458in}}{\pgfqpoint{4.562079in}{2.804859in}}{\pgfqpoint{4.562079in}{2.793809in}}%
\pgfpathcurveto{\pgfqpoint{4.562079in}{2.782759in}}{\pgfqpoint{4.566470in}{2.772160in}}{\pgfqpoint{4.574283in}{2.764346in}}%
\pgfpathcurveto{\pgfqpoint{4.582097in}{2.756532in}}{\pgfqpoint{4.592696in}{2.752142in}}{\pgfqpoint{4.603746in}{2.752142in}}%
\pgfpathclose%
\pgfusepath{stroke,fill}%
\end{pgfscope}%
\begin{pgfscope}%
\pgfpathrectangle{\pgfqpoint{0.787074in}{0.548769in}}{\pgfqpoint{5.062926in}{3.102590in}}%
\pgfusepath{clip}%
\pgfsetbuttcap%
\pgfsetroundjoin%
\definecolor{currentfill}{rgb}{1.000000,0.498039,0.054902}%
\pgfsetfillcolor{currentfill}%
\pgfsetlinewidth{1.003750pt}%
\definecolor{currentstroke}{rgb}{1.000000,0.498039,0.054902}%
\pgfsetstrokecolor{currentstroke}%
\pgfsetdash{}{0pt}%
\pgfpathmoveto{\pgfqpoint{3.054040in}{2.067284in}}%
\pgfpathcurveto{\pgfqpoint{3.065091in}{2.067284in}}{\pgfqpoint{3.075690in}{2.071675in}}{\pgfqpoint{3.083503in}{2.079488in}}%
\pgfpathcurveto{\pgfqpoint{3.091317in}{2.087302in}}{\pgfqpoint{3.095707in}{2.097901in}}{\pgfqpoint{3.095707in}{2.108951in}}%
\pgfpathcurveto{\pgfqpoint{3.095707in}{2.120001in}}{\pgfqpoint{3.091317in}{2.130600in}}{\pgfqpoint{3.083503in}{2.138414in}}%
\pgfpathcurveto{\pgfqpoint{3.075690in}{2.146227in}}{\pgfqpoint{3.065091in}{2.150618in}}{\pgfqpoint{3.054040in}{2.150618in}}%
\pgfpathcurveto{\pgfqpoint{3.042990in}{2.150618in}}{\pgfqpoint{3.032391in}{2.146227in}}{\pgfqpoint{3.024578in}{2.138414in}}%
\pgfpathcurveto{\pgfqpoint{3.016764in}{2.130600in}}{\pgfqpoint{3.012374in}{2.120001in}}{\pgfqpoint{3.012374in}{2.108951in}}%
\pgfpathcurveto{\pgfqpoint{3.012374in}{2.097901in}}{\pgfqpoint{3.016764in}{2.087302in}}{\pgfqpoint{3.024578in}{2.079488in}}%
\pgfpathcurveto{\pgfqpoint{3.032391in}{2.071675in}}{\pgfqpoint{3.042990in}{2.067284in}}{\pgfqpoint{3.054040in}{2.067284in}}%
\pgfpathclose%
\pgfusepath{stroke,fill}%
\end{pgfscope}%
\begin{pgfscope}%
\pgfpathrectangle{\pgfqpoint{0.787074in}{0.548769in}}{\pgfqpoint{5.062926in}{3.102590in}}%
\pgfusepath{clip}%
\pgfsetbuttcap%
\pgfsetroundjoin%
\definecolor{currentfill}{rgb}{1.000000,0.498039,0.054902}%
\pgfsetfillcolor{currentfill}%
\pgfsetlinewidth{1.003750pt}%
\definecolor{currentstroke}{rgb}{1.000000,0.498039,0.054902}%
\pgfsetstrokecolor{currentstroke}%
\pgfsetdash{}{0pt}%
\pgfpathmoveto{\pgfqpoint{5.005664in}{3.097151in}}%
\pgfpathcurveto{\pgfqpoint{5.016714in}{3.097151in}}{\pgfqpoint{5.027313in}{3.101541in}}{\pgfqpoint{5.035127in}{3.109355in}}%
\pgfpathcurveto{\pgfqpoint{5.042940in}{3.117169in}}{\pgfqpoint{5.047331in}{3.127768in}}{\pgfqpoint{5.047331in}{3.138818in}}%
\pgfpathcurveto{\pgfqpoint{5.047331in}{3.149868in}}{\pgfqpoint{5.042940in}{3.160467in}}{\pgfqpoint{5.035127in}{3.168281in}}%
\pgfpathcurveto{\pgfqpoint{5.027313in}{3.176094in}}{\pgfqpoint{5.016714in}{3.180484in}}{\pgfqpoint{5.005664in}{3.180484in}}%
\pgfpathcurveto{\pgfqpoint{4.994614in}{3.180484in}}{\pgfqpoint{4.984015in}{3.176094in}}{\pgfqpoint{4.976201in}{3.168281in}}%
\pgfpathcurveto{\pgfqpoint{4.968388in}{3.160467in}}{\pgfqpoint{4.963997in}{3.149868in}}{\pgfqpoint{4.963997in}{3.138818in}}%
\pgfpathcurveto{\pgfqpoint{4.963997in}{3.127768in}}{\pgfqpoint{4.968388in}{3.117169in}}{\pgfqpoint{4.976201in}{3.109355in}}%
\pgfpathcurveto{\pgfqpoint{4.984015in}{3.101541in}}{\pgfqpoint{4.994614in}{3.097151in}}{\pgfqpoint{5.005664in}{3.097151in}}%
\pgfpathclose%
\pgfusepath{stroke,fill}%
\end{pgfscope}%
\begin{pgfscope}%
\pgfpathrectangle{\pgfqpoint{0.787074in}{0.548769in}}{\pgfqpoint{5.062926in}{3.102590in}}%
\pgfusepath{clip}%
\pgfsetbuttcap%
\pgfsetroundjoin%
\definecolor{currentfill}{rgb}{1.000000,0.498039,0.054902}%
\pgfsetfillcolor{currentfill}%
\pgfsetlinewidth{1.003750pt}%
\definecolor{currentstroke}{rgb}{1.000000,0.498039,0.054902}%
\pgfsetstrokecolor{currentstroke}%
\pgfsetdash{}{0pt}%
\pgfpathmoveto{\pgfqpoint{4.973986in}{2.976134in}}%
\pgfpathcurveto{\pgfqpoint{4.985036in}{2.976134in}}{\pgfqpoint{4.995635in}{2.980524in}}{\pgfqpoint{5.003449in}{2.988338in}}%
\pgfpathcurveto{\pgfqpoint{5.011262in}{2.996152in}}{\pgfqpoint{5.015653in}{3.006751in}}{\pgfqpoint{5.015653in}{3.017801in}}%
\pgfpathcurveto{\pgfqpoint{5.015653in}{3.028851in}}{\pgfqpoint{5.011262in}{3.039450in}}{\pgfqpoint{5.003449in}{3.047264in}}%
\pgfpathcurveto{\pgfqpoint{4.995635in}{3.055077in}}{\pgfqpoint{4.985036in}{3.059468in}}{\pgfqpoint{4.973986in}{3.059468in}}%
\pgfpathcurveto{\pgfqpoint{4.962936in}{3.059468in}}{\pgfqpoint{4.952337in}{3.055077in}}{\pgfqpoint{4.944523in}{3.047264in}}%
\pgfpathcurveto{\pgfqpoint{4.936710in}{3.039450in}}{\pgfqpoint{4.932319in}{3.028851in}}{\pgfqpoint{4.932319in}{3.017801in}}%
\pgfpathcurveto{\pgfqpoint{4.932319in}{3.006751in}}{\pgfqpoint{4.936710in}{2.996152in}}{\pgfqpoint{4.944523in}{2.988338in}}%
\pgfpathcurveto{\pgfqpoint{4.952337in}{2.980524in}}{\pgfqpoint{4.962936in}{2.976134in}}{\pgfqpoint{4.973986in}{2.976134in}}%
\pgfpathclose%
\pgfusepath{stroke,fill}%
\end{pgfscope}%
\begin{pgfscope}%
\pgfpathrectangle{\pgfqpoint{0.787074in}{0.548769in}}{\pgfqpoint{5.062926in}{3.102590in}}%
\pgfusepath{clip}%
\pgfsetbuttcap%
\pgfsetroundjoin%
\definecolor{currentfill}{rgb}{1.000000,0.498039,0.054902}%
\pgfsetfillcolor{currentfill}%
\pgfsetlinewidth{1.003750pt}%
\definecolor{currentstroke}{rgb}{1.000000,0.498039,0.054902}%
\pgfsetstrokecolor{currentstroke}%
\pgfsetdash{}{0pt}%
\pgfpathmoveto{\pgfqpoint{3.978664in}{2.882792in}}%
\pgfpathcurveto{\pgfqpoint{3.989714in}{2.882792in}}{\pgfqpoint{4.000313in}{2.887182in}}{\pgfqpoint{4.008126in}{2.894996in}}%
\pgfpathcurveto{\pgfqpoint{4.015940in}{2.902810in}}{\pgfqpoint{4.020330in}{2.913409in}}{\pgfqpoint{4.020330in}{2.924459in}}%
\pgfpathcurveto{\pgfqpoint{4.020330in}{2.935509in}}{\pgfqpoint{4.015940in}{2.946108in}}{\pgfqpoint{4.008126in}{2.953922in}}%
\pgfpathcurveto{\pgfqpoint{4.000313in}{2.961735in}}{\pgfqpoint{3.989714in}{2.966125in}}{\pgfqpoint{3.978664in}{2.966125in}}%
\pgfpathcurveto{\pgfqpoint{3.967613in}{2.966125in}}{\pgfqpoint{3.957014in}{2.961735in}}{\pgfqpoint{3.949201in}{2.953922in}}%
\pgfpathcurveto{\pgfqpoint{3.941387in}{2.946108in}}{\pgfqpoint{3.936997in}{2.935509in}}{\pgfqpoint{3.936997in}{2.924459in}}%
\pgfpathcurveto{\pgfqpoint{3.936997in}{2.913409in}}{\pgfqpoint{3.941387in}{2.902810in}}{\pgfqpoint{3.949201in}{2.894996in}}%
\pgfpathcurveto{\pgfqpoint{3.957014in}{2.887182in}}{\pgfqpoint{3.967613in}{2.882792in}}{\pgfqpoint{3.978664in}{2.882792in}}%
\pgfpathclose%
\pgfusepath{stroke,fill}%
\end{pgfscope}%
\begin{pgfscope}%
\pgfpathrectangle{\pgfqpoint{0.787074in}{0.548769in}}{\pgfqpoint{5.062926in}{3.102590in}}%
\pgfusepath{clip}%
\pgfsetbuttcap%
\pgfsetroundjoin%
\definecolor{currentfill}{rgb}{0.839216,0.152941,0.156863}%
\pgfsetfillcolor{currentfill}%
\pgfsetlinewidth{1.003750pt}%
\definecolor{currentstroke}{rgb}{0.839216,0.152941,0.156863}%
\pgfsetstrokecolor{currentstroke}%
\pgfsetdash{}{0pt}%
\pgfpathmoveto{\pgfqpoint{5.619867in}{3.334884in}}%
\pgfpathcurveto{\pgfqpoint{5.630917in}{3.334884in}}{\pgfqpoint{5.641516in}{3.339274in}}{\pgfqpoint{5.649330in}{3.347088in}}%
\pgfpathcurveto{\pgfqpoint{5.657143in}{3.354902in}}{\pgfqpoint{5.661534in}{3.365501in}}{\pgfqpoint{5.661534in}{3.376551in}}%
\pgfpathcurveto{\pgfqpoint{5.661534in}{3.387601in}}{\pgfqpoint{5.657143in}{3.398200in}}{\pgfqpoint{5.649330in}{3.406014in}}%
\pgfpathcurveto{\pgfqpoint{5.641516in}{3.413827in}}{\pgfqpoint{5.630917in}{3.418218in}}{\pgfqpoint{5.619867in}{3.418218in}}%
\pgfpathcurveto{\pgfqpoint{5.608817in}{3.418218in}}{\pgfqpoint{5.598218in}{3.413827in}}{\pgfqpoint{5.590404in}{3.406014in}}%
\pgfpathcurveto{\pgfqpoint{5.582591in}{3.398200in}}{\pgfqpoint{5.578200in}{3.387601in}}{\pgfqpoint{5.578200in}{3.376551in}}%
\pgfpathcurveto{\pgfqpoint{5.578200in}{3.365501in}}{\pgfqpoint{5.582591in}{3.354902in}}{\pgfqpoint{5.590404in}{3.347088in}}%
\pgfpathcurveto{\pgfqpoint{5.598218in}{3.339274in}}{\pgfqpoint{5.608817in}{3.334884in}}{\pgfqpoint{5.619867in}{3.334884in}}%
\pgfpathclose%
\pgfusepath{stroke,fill}%
\end{pgfscope}%
\begin{pgfscope}%
\pgfpathrectangle{\pgfqpoint{0.787074in}{0.548769in}}{\pgfqpoint{5.062926in}{3.102590in}}%
\pgfusepath{clip}%
\pgfsetbuttcap%
\pgfsetroundjoin%
\definecolor{currentfill}{rgb}{1.000000,0.498039,0.054902}%
\pgfsetfillcolor{currentfill}%
\pgfsetlinewidth{1.003750pt}%
\definecolor{currentstroke}{rgb}{1.000000,0.498039,0.054902}%
\pgfsetstrokecolor{currentstroke}%
\pgfsetdash{}{0pt}%
\pgfpathmoveto{\pgfqpoint{4.731962in}{3.000517in}}%
\pgfpathcurveto{\pgfqpoint{4.743012in}{3.000517in}}{\pgfqpoint{4.753611in}{3.004907in}}{\pgfqpoint{4.761425in}{3.012721in}}%
\pgfpathcurveto{\pgfqpoint{4.769238in}{3.020535in}}{\pgfqpoint{4.773628in}{3.031134in}}{\pgfqpoint{4.773628in}{3.042184in}}%
\pgfpathcurveto{\pgfqpoint{4.773628in}{3.053234in}}{\pgfqpoint{4.769238in}{3.063833in}}{\pgfqpoint{4.761425in}{3.071646in}}%
\pgfpathcurveto{\pgfqpoint{4.753611in}{3.079460in}}{\pgfqpoint{4.743012in}{3.083850in}}{\pgfqpoint{4.731962in}{3.083850in}}%
\pgfpathcurveto{\pgfqpoint{4.720912in}{3.083850in}}{\pgfqpoint{4.710313in}{3.079460in}}{\pgfqpoint{4.702499in}{3.071646in}}%
\pgfpathcurveto{\pgfqpoint{4.694685in}{3.063833in}}{\pgfqpoint{4.690295in}{3.053234in}}{\pgfqpoint{4.690295in}{3.042184in}}%
\pgfpathcurveto{\pgfqpoint{4.690295in}{3.031134in}}{\pgfqpoint{4.694685in}{3.020535in}}{\pgfqpoint{4.702499in}{3.012721in}}%
\pgfpathcurveto{\pgfqpoint{4.710313in}{3.004907in}}{\pgfqpoint{4.720912in}{3.000517in}}{\pgfqpoint{4.731962in}{3.000517in}}%
\pgfpathclose%
\pgfusepath{stroke,fill}%
\end{pgfscope}%
\begin{pgfscope}%
\pgfpathrectangle{\pgfqpoint{0.787074in}{0.548769in}}{\pgfqpoint{5.062926in}{3.102590in}}%
\pgfusepath{clip}%
\pgfsetbuttcap%
\pgfsetroundjoin%
\definecolor{currentfill}{rgb}{0.121569,0.466667,0.705882}%
\pgfsetfillcolor{currentfill}%
\pgfsetlinewidth{1.003750pt}%
\definecolor{currentstroke}{rgb}{0.121569,0.466667,0.705882}%
\pgfsetstrokecolor{currentstroke}%
\pgfsetdash{}{0pt}%
\pgfpathmoveto{\pgfqpoint{3.625366in}{2.441957in}}%
\pgfpathcurveto{\pgfqpoint{3.636416in}{2.441957in}}{\pgfqpoint{3.647015in}{2.446347in}}{\pgfqpoint{3.654829in}{2.454161in}}%
\pgfpathcurveto{\pgfqpoint{3.662642in}{2.461974in}}{\pgfqpoint{3.667033in}{2.472573in}}{\pgfqpoint{3.667033in}{2.483623in}}%
\pgfpathcurveto{\pgfqpoint{3.667033in}{2.494673in}}{\pgfqpoint{3.662642in}{2.505273in}}{\pgfqpoint{3.654829in}{2.513086in}}%
\pgfpathcurveto{\pgfqpoint{3.647015in}{2.520900in}}{\pgfqpoint{3.636416in}{2.525290in}}{\pgfqpoint{3.625366in}{2.525290in}}%
\pgfpathcurveto{\pgfqpoint{3.614316in}{2.525290in}}{\pgfqpoint{3.603717in}{2.520900in}}{\pgfqpoint{3.595903in}{2.513086in}}%
\pgfpathcurveto{\pgfqpoint{3.588090in}{2.505273in}}{\pgfqpoint{3.583699in}{2.494673in}}{\pgfqpoint{3.583699in}{2.483623in}}%
\pgfpathcurveto{\pgfqpoint{3.583699in}{2.472573in}}{\pgfqpoint{3.588090in}{2.461974in}}{\pgfqpoint{3.595903in}{2.454161in}}%
\pgfpathcurveto{\pgfqpoint{3.603717in}{2.446347in}}{\pgfqpoint{3.614316in}{2.441957in}}{\pgfqpoint{3.625366in}{2.441957in}}%
\pgfpathclose%
\pgfusepath{stroke,fill}%
\end{pgfscope}%
\begin{pgfscope}%
\pgfpathrectangle{\pgfqpoint{0.787074in}{0.548769in}}{\pgfqpoint{5.062926in}{3.102590in}}%
\pgfusepath{clip}%
\pgfsetbuttcap%
\pgfsetroundjoin%
\definecolor{currentfill}{rgb}{0.121569,0.466667,0.705882}%
\pgfsetfillcolor{currentfill}%
\pgfsetlinewidth{1.003750pt}%
\definecolor{currentstroke}{rgb}{0.121569,0.466667,0.705882}%
\pgfsetstrokecolor{currentstroke}%
\pgfsetdash{}{0pt}%
\pgfpathmoveto{\pgfqpoint{1.426797in}{0.919027in}}%
\pgfpathcurveto{\pgfqpoint{1.437848in}{0.919027in}}{\pgfqpoint{1.448447in}{0.923417in}}{\pgfqpoint{1.456260in}{0.931231in}}%
\pgfpathcurveto{\pgfqpoint{1.464074in}{0.939044in}}{\pgfqpoint{1.468464in}{0.949643in}}{\pgfqpoint{1.468464in}{0.960694in}}%
\pgfpathcurveto{\pgfqpoint{1.468464in}{0.971744in}}{\pgfqpoint{1.464074in}{0.982343in}}{\pgfqpoint{1.456260in}{0.990156in}}%
\pgfpathcurveto{\pgfqpoint{1.448447in}{0.997970in}}{\pgfqpoint{1.437848in}{1.002360in}}{\pgfqpoint{1.426797in}{1.002360in}}%
\pgfpathcurveto{\pgfqpoint{1.415747in}{1.002360in}}{\pgfqpoint{1.405148in}{0.997970in}}{\pgfqpoint{1.397335in}{0.990156in}}%
\pgfpathcurveto{\pgfqpoint{1.389521in}{0.982343in}}{\pgfqpoint{1.385131in}{0.971744in}}{\pgfqpoint{1.385131in}{0.960694in}}%
\pgfpathcurveto{\pgfqpoint{1.385131in}{0.949643in}}{\pgfqpoint{1.389521in}{0.939044in}}{\pgfqpoint{1.397335in}{0.931231in}}%
\pgfpathcurveto{\pgfqpoint{1.405148in}{0.923417in}}{\pgfqpoint{1.415747in}{0.919027in}}{\pgfqpoint{1.426797in}{0.919027in}}%
\pgfpathclose%
\pgfusepath{stroke,fill}%
\end{pgfscope}%
\begin{pgfscope}%
\pgfpathrectangle{\pgfqpoint{0.787074in}{0.548769in}}{\pgfqpoint{5.062926in}{3.102590in}}%
\pgfusepath{clip}%
\pgfsetbuttcap%
\pgfsetroundjoin%
\definecolor{currentfill}{rgb}{0.839216,0.152941,0.156863}%
\pgfsetfillcolor{currentfill}%
\pgfsetlinewidth{1.003750pt}%
\definecolor{currentstroke}{rgb}{0.839216,0.152941,0.156863}%
\pgfsetstrokecolor{currentstroke}%
\pgfsetdash{}{0pt}%
\pgfpathmoveto{\pgfqpoint{5.211514in}{3.248678in}}%
\pgfpathcurveto{\pgfqpoint{5.222564in}{3.248678in}}{\pgfqpoint{5.233163in}{3.253068in}}{\pgfqpoint{5.240977in}{3.260882in}}%
\pgfpathcurveto{\pgfqpoint{5.248790in}{3.268695in}}{\pgfqpoint{5.253181in}{3.279294in}}{\pgfqpoint{5.253181in}{3.290344in}}%
\pgfpathcurveto{\pgfqpoint{5.253181in}{3.301395in}}{\pgfqpoint{5.248790in}{3.311994in}}{\pgfqpoint{5.240977in}{3.319807in}}%
\pgfpathcurveto{\pgfqpoint{5.233163in}{3.327621in}}{\pgfqpoint{5.222564in}{3.332011in}}{\pgfqpoint{5.211514in}{3.332011in}}%
\pgfpathcurveto{\pgfqpoint{5.200464in}{3.332011in}}{\pgfqpoint{5.189865in}{3.327621in}}{\pgfqpoint{5.182051in}{3.319807in}}%
\pgfpathcurveto{\pgfqpoint{5.174238in}{3.311994in}}{\pgfqpoint{5.169847in}{3.301395in}}{\pgfqpoint{5.169847in}{3.290344in}}%
\pgfpathcurveto{\pgfqpoint{5.169847in}{3.279294in}}{\pgfqpoint{5.174238in}{3.268695in}}{\pgfqpoint{5.182051in}{3.260882in}}%
\pgfpathcurveto{\pgfqpoint{5.189865in}{3.253068in}}{\pgfqpoint{5.200464in}{3.248678in}}{\pgfqpoint{5.211514in}{3.248678in}}%
\pgfpathclose%
\pgfusepath{stroke,fill}%
\end{pgfscope}%
\begin{pgfscope}%
\pgfpathrectangle{\pgfqpoint{0.787074in}{0.548769in}}{\pgfqpoint{5.062926in}{3.102590in}}%
\pgfusepath{clip}%
\pgfsetbuttcap%
\pgfsetroundjoin%
\definecolor{currentfill}{rgb}{1.000000,0.498039,0.054902}%
\pgfsetfillcolor{currentfill}%
\pgfsetlinewidth{1.003750pt}%
\definecolor{currentstroke}{rgb}{1.000000,0.498039,0.054902}%
\pgfsetstrokecolor{currentstroke}%
\pgfsetdash{}{0pt}%
\pgfpathmoveto{\pgfqpoint{4.195098in}{2.631493in}}%
\pgfpathcurveto{\pgfqpoint{4.206148in}{2.631493in}}{\pgfqpoint{4.216747in}{2.635883in}}{\pgfqpoint{4.224560in}{2.643697in}}%
\pgfpathcurveto{\pgfqpoint{4.232374in}{2.651510in}}{\pgfqpoint{4.236764in}{2.662109in}}{\pgfqpoint{4.236764in}{2.673159in}}%
\pgfpathcurveto{\pgfqpoint{4.236764in}{2.684210in}}{\pgfqpoint{4.232374in}{2.694809in}}{\pgfqpoint{4.224560in}{2.702622in}}%
\pgfpathcurveto{\pgfqpoint{4.216747in}{2.710436in}}{\pgfqpoint{4.206148in}{2.714826in}}{\pgfqpoint{4.195098in}{2.714826in}}%
\pgfpathcurveto{\pgfqpoint{4.184047in}{2.714826in}}{\pgfqpoint{4.173448in}{2.710436in}}{\pgfqpoint{4.165635in}{2.702622in}}%
\pgfpathcurveto{\pgfqpoint{4.157821in}{2.694809in}}{\pgfqpoint{4.153431in}{2.684210in}}{\pgfqpoint{4.153431in}{2.673159in}}%
\pgfpathcurveto{\pgfqpoint{4.153431in}{2.662109in}}{\pgfqpoint{4.157821in}{2.651510in}}{\pgfqpoint{4.165635in}{2.643697in}}%
\pgfpathcurveto{\pgfqpoint{4.173448in}{2.635883in}}{\pgfqpoint{4.184047in}{2.631493in}}{\pgfqpoint{4.195098in}{2.631493in}}%
\pgfpathclose%
\pgfusepath{stroke,fill}%
\end{pgfscope}%
\begin{pgfscope}%
\pgfpathrectangle{\pgfqpoint{0.787074in}{0.548769in}}{\pgfqpoint{5.062926in}{3.102590in}}%
\pgfusepath{clip}%
\pgfsetbuttcap%
\pgfsetroundjoin%
\definecolor{currentfill}{rgb}{1.000000,0.498039,0.054902}%
\pgfsetfillcolor{currentfill}%
\pgfsetlinewidth{1.003750pt}%
\definecolor{currentstroke}{rgb}{1.000000,0.498039,0.054902}%
\pgfsetstrokecolor{currentstroke}%
\pgfsetdash{}{0pt}%
\pgfpathmoveto{\pgfqpoint{3.733540in}{2.514237in}}%
\pgfpathcurveto{\pgfqpoint{3.744590in}{2.514237in}}{\pgfqpoint{3.755189in}{2.518627in}}{\pgfqpoint{3.763002in}{2.526441in}}%
\pgfpathcurveto{\pgfqpoint{3.770816in}{2.534255in}}{\pgfqpoint{3.775206in}{2.544854in}}{\pgfqpoint{3.775206in}{2.555904in}}%
\pgfpathcurveto{\pgfqpoint{3.775206in}{2.566954in}}{\pgfqpoint{3.770816in}{2.577553in}}{\pgfqpoint{3.763002in}{2.585367in}}%
\pgfpathcurveto{\pgfqpoint{3.755189in}{2.593180in}}{\pgfqpoint{3.744590in}{2.597570in}}{\pgfqpoint{3.733540in}{2.597570in}}%
\pgfpathcurveto{\pgfqpoint{3.722489in}{2.597570in}}{\pgfqpoint{3.711890in}{2.593180in}}{\pgfqpoint{3.704077in}{2.585367in}}%
\pgfpathcurveto{\pgfqpoint{3.696263in}{2.577553in}}{\pgfqpoint{3.691873in}{2.566954in}}{\pgfqpoint{3.691873in}{2.555904in}}%
\pgfpathcurveto{\pgfqpoint{3.691873in}{2.544854in}}{\pgfqpoint{3.696263in}{2.534255in}}{\pgfqpoint{3.704077in}{2.526441in}}%
\pgfpathcurveto{\pgfqpoint{3.711890in}{2.518627in}}{\pgfqpoint{3.722489in}{2.514237in}}{\pgfqpoint{3.733540in}{2.514237in}}%
\pgfpathclose%
\pgfusepath{stroke,fill}%
\end{pgfscope}%
\begin{pgfscope}%
\pgfpathrectangle{\pgfqpoint{0.787074in}{0.548769in}}{\pgfqpoint{5.062926in}{3.102590in}}%
\pgfusepath{clip}%
\pgfsetbuttcap%
\pgfsetroundjoin%
\definecolor{currentfill}{rgb}{1.000000,0.498039,0.054902}%
\pgfsetfillcolor{currentfill}%
\pgfsetlinewidth{1.003750pt}%
\definecolor{currentstroke}{rgb}{1.000000,0.498039,0.054902}%
\pgfsetstrokecolor{currentstroke}%
\pgfsetdash{}{0pt}%
\pgfpathmoveto{\pgfqpoint{4.406152in}{2.630257in}}%
\pgfpathcurveto{\pgfqpoint{4.417203in}{2.630257in}}{\pgfqpoint{4.427802in}{2.634648in}}{\pgfqpoint{4.435615in}{2.642461in}}%
\pgfpathcurveto{\pgfqpoint{4.443429in}{2.650275in}}{\pgfqpoint{4.447819in}{2.660874in}}{\pgfqpoint{4.447819in}{2.671924in}}%
\pgfpathcurveto{\pgfqpoint{4.447819in}{2.682974in}}{\pgfqpoint{4.443429in}{2.693573in}}{\pgfqpoint{4.435615in}{2.701387in}}%
\pgfpathcurveto{\pgfqpoint{4.427802in}{2.709200in}}{\pgfqpoint{4.417203in}{2.713591in}}{\pgfqpoint{4.406152in}{2.713591in}}%
\pgfpathcurveto{\pgfqpoint{4.395102in}{2.713591in}}{\pgfqpoint{4.384503in}{2.709200in}}{\pgfqpoint{4.376690in}{2.701387in}}%
\pgfpathcurveto{\pgfqpoint{4.368876in}{2.693573in}}{\pgfqpoint{4.364486in}{2.682974in}}{\pgfqpoint{4.364486in}{2.671924in}}%
\pgfpathcurveto{\pgfqpoint{4.364486in}{2.660874in}}{\pgfqpoint{4.368876in}{2.650275in}}{\pgfqpoint{4.376690in}{2.642461in}}%
\pgfpathcurveto{\pgfqpoint{4.384503in}{2.634648in}}{\pgfqpoint{4.395102in}{2.630257in}}{\pgfqpoint{4.406152in}{2.630257in}}%
\pgfpathclose%
\pgfusepath{stroke,fill}%
\end{pgfscope}%
\begin{pgfscope}%
\pgfpathrectangle{\pgfqpoint{0.787074in}{0.548769in}}{\pgfqpoint{5.062926in}{3.102590in}}%
\pgfusepath{clip}%
\pgfsetbuttcap%
\pgfsetroundjoin%
\definecolor{currentfill}{rgb}{1.000000,0.498039,0.054902}%
\pgfsetfillcolor{currentfill}%
\pgfsetlinewidth{1.003750pt}%
\definecolor{currentstroke}{rgb}{1.000000,0.498039,0.054902}%
\pgfsetstrokecolor{currentstroke}%
\pgfsetdash{}{0pt}%
\pgfpathmoveto{\pgfqpoint{4.801809in}{3.109003in}}%
\pgfpathcurveto{\pgfqpoint{4.812859in}{3.109003in}}{\pgfqpoint{4.823458in}{3.113393in}}{\pgfqpoint{4.831272in}{3.121207in}}%
\pgfpathcurveto{\pgfqpoint{4.839085in}{3.129020in}}{\pgfqpoint{4.843475in}{3.139619in}}{\pgfqpoint{4.843475in}{3.150669in}}%
\pgfpathcurveto{\pgfqpoint{4.843475in}{3.161719in}}{\pgfqpoint{4.839085in}{3.172319in}}{\pgfqpoint{4.831272in}{3.180132in}}%
\pgfpathcurveto{\pgfqpoint{4.823458in}{3.187946in}}{\pgfqpoint{4.812859in}{3.192336in}}{\pgfqpoint{4.801809in}{3.192336in}}%
\pgfpathcurveto{\pgfqpoint{4.790759in}{3.192336in}}{\pgfqpoint{4.780160in}{3.187946in}}{\pgfqpoint{4.772346in}{3.180132in}}%
\pgfpathcurveto{\pgfqpoint{4.764532in}{3.172319in}}{\pgfqpoint{4.760142in}{3.161719in}}{\pgfqpoint{4.760142in}{3.150669in}}%
\pgfpathcurveto{\pgfqpoint{4.760142in}{3.139619in}}{\pgfqpoint{4.764532in}{3.129020in}}{\pgfqpoint{4.772346in}{3.121207in}}%
\pgfpathcurveto{\pgfqpoint{4.780160in}{3.113393in}}{\pgfqpoint{4.790759in}{3.109003in}}{\pgfqpoint{4.801809in}{3.109003in}}%
\pgfpathclose%
\pgfusepath{stroke,fill}%
\end{pgfscope}%
\begin{pgfscope}%
\pgfpathrectangle{\pgfqpoint{0.787074in}{0.548769in}}{\pgfqpoint{5.062926in}{3.102590in}}%
\pgfusepath{clip}%
\pgfsetbuttcap%
\pgfsetroundjoin%
\definecolor{currentfill}{rgb}{1.000000,0.498039,0.054902}%
\pgfsetfillcolor{currentfill}%
\pgfsetlinewidth{1.003750pt}%
\definecolor{currentstroke}{rgb}{1.000000,0.498039,0.054902}%
\pgfsetstrokecolor{currentstroke}%
\pgfsetdash{}{0pt}%
\pgfpathmoveto{\pgfqpoint{3.399023in}{2.413769in}}%
\pgfpathcurveto{\pgfqpoint{3.410073in}{2.413769in}}{\pgfqpoint{3.420672in}{2.418159in}}{\pgfqpoint{3.428486in}{2.425973in}}%
\pgfpathcurveto{\pgfqpoint{3.436299in}{2.433787in}}{\pgfqpoint{3.440690in}{2.444386in}}{\pgfqpoint{3.440690in}{2.455436in}}%
\pgfpathcurveto{\pgfqpoint{3.440690in}{2.466486in}}{\pgfqpoint{3.436299in}{2.477085in}}{\pgfqpoint{3.428486in}{2.484899in}}%
\pgfpathcurveto{\pgfqpoint{3.420672in}{2.492712in}}{\pgfqpoint{3.410073in}{2.497102in}}{\pgfqpoint{3.399023in}{2.497102in}}%
\pgfpathcurveto{\pgfqpoint{3.387973in}{2.497102in}}{\pgfqpoint{3.377374in}{2.492712in}}{\pgfqpoint{3.369560in}{2.484899in}}%
\pgfpathcurveto{\pgfqpoint{3.361747in}{2.477085in}}{\pgfqpoint{3.357356in}{2.466486in}}{\pgfqpoint{3.357356in}{2.455436in}}%
\pgfpathcurveto{\pgfqpoint{3.357356in}{2.444386in}}{\pgfqpoint{3.361747in}{2.433787in}}{\pgfqpoint{3.369560in}{2.425973in}}%
\pgfpathcurveto{\pgfqpoint{3.377374in}{2.418159in}}{\pgfqpoint{3.387973in}{2.413769in}}{\pgfqpoint{3.399023in}{2.413769in}}%
\pgfpathclose%
\pgfusepath{stroke,fill}%
\end{pgfscope}%
\begin{pgfscope}%
\pgfpathrectangle{\pgfqpoint{0.787074in}{0.548769in}}{\pgfqpoint{5.062926in}{3.102590in}}%
\pgfusepath{clip}%
\pgfsetbuttcap%
\pgfsetroundjoin%
\definecolor{currentfill}{rgb}{0.121569,0.466667,0.705882}%
\pgfsetfillcolor{currentfill}%
\pgfsetlinewidth{1.003750pt}%
\definecolor{currentstroke}{rgb}{0.121569,0.466667,0.705882}%
\pgfsetstrokecolor{currentstroke}%
\pgfsetdash{}{0pt}%
\pgfpathmoveto{\pgfqpoint{4.188427in}{2.714693in}}%
\pgfpathcurveto{\pgfqpoint{4.199478in}{2.714693in}}{\pgfqpoint{4.210077in}{2.719083in}}{\pgfqpoint{4.217890in}{2.726897in}}%
\pgfpathcurveto{\pgfqpoint{4.225704in}{2.734711in}}{\pgfqpoint{4.230094in}{2.745310in}}{\pgfqpoint{4.230094in}{2.756360in}}%
\pgfpathcurveto{\pgfqpoint{4.230094in}{2.767410in}}{\pgfqpoint{4.225704in}{2.778009in}}{\pgfqpoint{4.217890in}{2.785822in}}%
\pgfpathcurveto{\pgfqpoint{4.210077in}{2.793636in}}{\pgfqpoint{4.199478in}{2.798026in}}{\pgfqpoint{4.188427in}{2.798026in}}%
\pgfpathcurveto{\pgfqpoint{4.177377in}{2.798026in}}{\pgfqpoint{4.166778in}{2.793636in}}{\pgfqpoint{4.158965in}{2.785822in}}%
\pgfpathcurveto{\pgfqpoint{4.151151in}{2.778009in}}{\pgfqpoint{4.146761in}{2.767410in}}{\pgfqpoint{4.146761in}{2.756360in}}%
\pgfpathcurveto{\pgfqpoint{4.146761in}{2.745310in}}{\pgfqpoint{4.151151in}{2.734711in}}{\pgfqpoint{4.158965in}{2.726897in}}%
\pgfpathcurveto{\pgfqpoint{4.166778in}{2.719083in}}{\pgfqpoint{4.177377in}{2.714693in}}{\pgfqpoint{4.188427in}{2.714693in}}%
\pgfpathclose%
\pgfusepath{stroke,fill}%
\end{pgfscope}%
\begin{pgfscope}%
\pgfpathrectangle{\pgfqpoint{0.787074in}{0.548769in}}{\pgfqpoint{5.062926in}{3.102590in}}%
\pgfusepath{clip}%
\pgfsetbuttcap%
\pgfsetroundjoin%
\definecolor{currentfill}{rgb}{1.000000,0.498039,0.054902}%
\pgfsetfillcolor{currentfill}%
\pgfsetlinewidth{1.003750pt}%
\definecolor{currentstroke}{rgb}{1.000000,0.498039,0.054902}%
\pgfsetstrokecolor{currentstroke}%
\pgfsetdash{}{0pt}%
\pgfpathmoveto{\pgfqpoint{3.875985in}{2.455826in}}%
\pgfpathcurveto{\pgfqpoint{3.887035in}{2.455826in}}{\pgfqpoint{3.897634in}{2.460216in}}{\pgfqpoint{3.905448in}{2.468030in}}%
\pgfpathcurveto{\pgfqpoint{3.913261in}{2.475843in}}{\pgfqpoint{3.917651in}{2.486442in}}{\pgfqpoint{3.917651in}{2.497492in}}%
\pgfpathcurveto{\pgfqpoint{3.917651in}{2.508542in}}{\pgfqpoint{3.913261in}{2.519141in}}{\pgfqpoint{3.905448in}{2.526955in}}%
\pgfpathcurveto{\pgfqpoint{3.897634in}{2.534769in}}{\pgfqpoint{3.887035in}{2.539159in}}{\pgfqpoint{3.875985in}{2.539159in}}%
\pgfpathcurveto{\pgfqpoint{3.864935in}{2.539159in}}{\pgfqpoint{3.854336in}{2.534769in}}{\pgfqpoint{3.846522in}{2.526955in}}%
\pgfpathcurveto{\pgfqpoint{3.838708in}{2.519141in}}{\pgfqpoint{3.834318in}{2.508542in}}{\pgfqpoint{3.834318in}{2.497492in}}%
\pgfpathcurveto{\pgfqpoint{3.834318in}{2.486442in}}{\pgfqpoint{3.838708in}{2.475843in}}{\pgfqpoint{3.846522in}{2.468030in}}%
\pgfpathcurveto{\pgfqpoint{3.854336in}{2.460216in}}{\pgfqpoint{3.864935in}{2.455826in}}{\pgfqpoint{3.875985in}{2.455826in}}%
\pgfpathclose%
\pgfusepath{stroke,fill}%
\end{pgfscope}%
\begin{pgfscope}%
\pgfpathrectangle{\pgfqpoint{0.787074in}{0.548769in}}{\pgfqpoint{5.062926in}{3.102590in}}%
\pgfusepath{clip}%
\pgfsetbuttcap%
\pgfsetroundjoin%
\definecolor{currentfill}{rgb}{1.000000,0.498039,0.054902}%
\pgfsetfillcolor{currentfill}%
\pgfsetlinewidth{1.003750pt}%
\definecolor{currentstroke}{rgb}{1.000000,0.498039,0.054902}%
\pgfsetstrokecolor{currentstroke}%
\pgfsetdash{}{0pt}%
\pgfpathmoveto{\pgfqpoint{3.578801in}{2.530243in}}%
\pgfpathcurveto{\pgfqpoint{3.589851in}{2.530243in}}{\pgfqpoint{3.600450in}{2.534634in}}{\pgfqpoint{3.608264in}{2.542447in}}%
\pgfpathcurveto{\pgfqpoint{3.616077in}{2.550261in}}{\pgfqpoint{3.620468in}{2.560860in}}{\pgfqpoint{3.620468in}{2.571910in}}%
\pgfpathcurveto{\pgfqpoint{3.620468in}{2.582960in}}{\pgfqpoint{3.616077in}{2.593559in}}{\pgfqpoint{3.608264in}{2.601373in}}%
\pgfpathcurveto{\pgfqpoint{3.600450in}{2.609186in}}{\pgfqpoint{3.589851in}{2.613577in}}{\pgfqpoint{3.578801in}{2.613577in}}%
\pgfpathcurveto{\pgfqpoint{3.567751in}{2.613577in}}{\pgfqpoint{3.557152in}{2.609186in}}{\pgfqpoint{3.549338in}{2.601373in}}%
\pgfpathcurveto{\pgfqpoint{3.541524in}{2.593559in}}{\pgfqpoint{3.537134in}{2.582960in}}{\pgfqpoint{3.537134in}{2.571910in}}%
\pgfpathcurveto{\pgfqpoint{3.537134in}{2.560860in}}{\pgfqpoint{3.541524in}{2.550261in}}{\pgfqpoint{3.549338in}{2.542447in}}%
\pgfpathcurveto{\pgfqpoint{3.557152in}{2.534634in}}{\pgfqpoint{3.567751in}{2.530243in}}{\pgfqpoint{3.578801in}{2.530243in}}%
\pgfpathclose%
\pgfusepath{stroke,fill}%
\end{pgfscope}%
\begin{pgfscope}%
\pgfpathrectangle{\pgfqpoint{0.787074in}{0.548769in}}{\pgfqpoint{5.062926in}{3.102590in}}%
\pgfusepath{clip}%
\pgfsetbuttcap%
\pgfsetroundjoin%
\definecolor{currentfill}{rgb}{1.000000,0.498039,0.054902}%
\pgfsetfillcolor{currentfill}%
\pgfsetlinewidth{1.003750pt}%
\definecolor{currentstroke}{rgb}{1.000000,0.498039,0.054902}%
\pgfsetstrokecolor{currentstroke}%
\pgfsetdash{}{0pt}%
\pgfpathmoveto{\pgfqpoint{5.040978in}{2.652528in}}%
\pgfpathcurveto{\pgfqpoint{5.052028in}{2.652528in}}{\pgfqpoint{5.062627in}{2.656919in}}{\pgfqpoint{5.070441in}{2.664732in}}%
\pgfpathcurveto{\pgfqpoint{5.078255in}{2.672546in}}{\pgfqpoint{5.082645in}{2.683145in}}{\pgfqpoint{5.082645in}{2.694195in}}%
\pgfpathcurveto{\pgfqpoint{5.082645in}{2.705245in}}{\pgfqpoint{5.078255in}{2.715844in}}{\pgfqpoint{5.070441in}{2.723658in}}%
\pgfpathcurveto{\pgfqpoint{5.062627in}{2.731471in}}{\pgfqpoint{5.052028in}{2.735862in}}{\pgfqpoint{5.040978in}{2.735862in}}%
\pgfpathcurveto{\pgfqpoint{5.029928in}{2.735862in}}{\pgfqpoint{5.019329in}{2.731471in}}{\pgfqpoint{5.011516in}{2.723658in}}%
\pgfpathcurveto{\pgfqpoint{5.003702in}{2.715844in}}{\pgfqpoint{4.999312in}{2.705245in}}{\pgfqpoint{4.999312in}{2.694195in}}%
\pgfpathcurveto{\pgfqpoint{4.999312in}{2.683145in}}{\pgfqpoint{5.003702in}{2.672546in}}{\pgfqpoint{5.011516in}{2.664732in}}%
\pgfpathcurveto{\pgfqpoint{5.019329in}{2.656919in}}{\pgfqpoint{5.029928in}{2.652528in}}{\pgfqpoint{5.040978in}{2.652528in}}%
\pgfpathclose%
\pgfusepath{stroke,fill}%
\end{pgfscope}%
\begin{pgfscope}%
\pgfpathrectangle{\pgfqpoint{0.787074in}{0.548769in}}{\pgfqpoint{5.062926in}{3.102590in}}%
\pgfusepath{clip}%
\pgfsetbuttcap%
\pgfsetroundjoin%
\definecolor{currentfill}{rgb}{1.000000,0.498039,0.054902}%
\pgfsetfillcolor{currentfill}%
\pgfsetlinewidth{1.003750pt}%
\definecolor{currentstroke}{rgb}{1.000000,0.498039,0.054902}%
\pgfsetstrokecolor{currentstroke}%
\pgfsetdash{}{0pt}%
\pgfpathmoveto{\pgfqpoint{3.978515in}{2.518244in}}%
\pgfpathcurveto{\pgfqpoint{3.989565in}{2.518244in}}{\pgfqpoint{4.000164in}{2.522634in}}{\pgfqpoint{4.007977in}{2.530448in}}%
\pgfpathcurveto{\pgfqpoint{4.015791in}{2.538261in}}{\pgfqpoint{4.020181in}{2.548860in}}{\pgfqpoint{4.020181in}{2.559910in}}%
\pgfpathcurveto{\pgfqpoint{4.020181in}{2.570961in}}{\pgfqpoint{4.015791in}{2.581560in}}{\pgfqpoint{4.007977in}{2.589373in}}%
\pgfpathcurveto{\pgfqpoint{4.000164in}{2.597187in}}{\pgfqpoint{3.989565in}{2.601577in}}{\pgfqpoint{3.978515in}{2.601577in}}%
\pgfpathcurveto{\pgfqpoint{3.967464in}{2.601577in}}{\pgfqpoint{3.956865in}{2.597187in}}{\pgfqpoint{3.949052in}{2.589373in}}%
\pgfpathcurveto{\pgfqpoint{3.941238in}{2.581560in}}{\pgfqpoint{3.936848in}{2.570961in}}{\pgfqpoint{3.936848in}{2.559910in}}%
\pgfpathcurveto{\pgfqpoint{3.936848in}{2.548860in}}{\pgfqpoint{3.941238in}{2.538261in}}{\pgfqpoint{3.949052in}{2.530448in}}%
\pgfpathcurveto{\pgfqpoint{3.956865in}{2.522634in}}{\pgfqpoint{3.967464in}{2.518244in}}{\pgfqpoint{3.978515in}{2.518244in}}%
\pgfpathclose%
\pgfusepath{stroke,fill}%
\end{pgfscope}%
\begin{pgfscope}%
\pgfpathrectangle{\pgfqpoint{0.787074in}{0.548769in}}{\pgfqpoint{5.062926in}{3.102590in}}%
\pgfusepath{clip}%
\pgfsetbuttcap%
\pgfsetroundjoin%
\definecolor{currentfill}{rgb}{0.839216,0.152941,0.156863}%
\pgfsetfillcolor{currentfill}%
\pgfsetlinewidth{1.003750pt}%
\definecolor{currentstroke}{rgb}{0.839216,0.152941,0.156863}%
\pgfsetstrokecolor{currentstroke}%
\pgfsetdash{}{0pt}%
\pgfpathmoveto{\pgfqpoint{3.831763in}{2.503837in}}%
\pgfpathcurveto{\pgfqpoint{3.842813in}{2.503837in}}{\pgfqpoint{3.853412in}{2.508227in}}{\pgfqpoint{3.861226in}{2.516041in}}%
\pgfpathcurveto{\pgfqpoint{3.869040in}{2.523854in}}{\pgfqpoint{3.873430in}{2.534453in}}{\pgfqpoint{3.873430in}{2.545503in}}%
\pgfpathcurveto{\pgfqpoint{3.873430in}{2.556554in}}{\pgfqpoint{3.869040in}{2.567153in}}{\pgfqpoint{3.861226in}{2.574966in}}%
\pgfpathcurveto{\pgfqpoint{3.853412in}{2.582780in}}{\pgfqpoint{3.842813in}{2.587170in}}{\pgfqpoint{3.831763in}{2.587170in}}%
\pgfpathcurveto{\pgfqpoint{3.820713in}{2.587170in}}{\pgfqpoint{3.810114in}{2.582780in}}{\pgfqpoint{3.802301in}{2.574966in}}%
\pgfpathcurveto{\pgfqpoint{3.794487in}{2.567153in}}{\pgfqpoint{3.790097in}{2.556554in}}{\pgfqpoint{3.790097in}{2.545503in}}%
\pgfpathcurveto{\pgfqpoint{3.790097in}{2.534453in}}{\pgfqpoint{3.794487in}{2.523854in}}{\pgfqpoint{3.802301in}{2.516041in}}%
\pgfpathcurveto{\pgfqpoint{3.810114in}{2.508227in}}{\pgfqpoint{3.820713in}{2.503837in}}{\pgfqpoint{3.831763in}{2.503837in}}%
\pgfpathclose%
\pgfusepath{stroke,fill}%
\end{pgfscope}%
\begin{pgfscope}%
\pgfpathrectangle{\pgfqpoint{0.787074in}{0.548769in}}{\pgfqpoint{5.062926in}{3.102590in}}%
\pgfusepath{clip}%
\pgfsetbuttcap%
\pgfsetroundjoin%
\definecolor{currentfill}{rgb}{1.000000,0.498039,0.054902}%
\pgfsetfillcolor{currentfill}%
\pgfsetlinewidth{1.003750pt}%
\definecolor{currentstroke}{rgb}{1.000000,0.498039,0.054902}%
\pgfsetstrokecolor{currentstroke}%
\pgfsetdash{}{0pt}%
\pgfpathmoveto{\pgfqpoint{4.089735in}{2.836137in}}%
\pgfpathcurveto{\pgfqpoint{4.100786in}{2.836137in}}{\pgfqpoint{4.111385in}{2.840528in}}{\pgfqpoint{4.119198in}{2.848341in}}%
\pgfpathcurveto{\pgfqpoint{4.127012in}{2.856155in}}{\pgfqpoint{4.131402in}{2.866754in}}{\pgfqpoint{4.131402in}{2.877804in}}%
\pgfpathcurveto{\pgfqpoint{4.131402in}{2.888854in}}{\pgfqpoint{4.127012in}{2.899453in}}{\pgfqpoint{4.119198in}{2.907267in}}%
\pgfpathcurveto{\pgfqpoint{4.111385in}{2.915080in}}{\pgfqpoint{4.100786in}{2.919471in}}{\pgfqpoint{4.089735in}{2.919471in}}%
\pgfpathcurveto{\pgfqpoint{4.078685in}{2.919471in}}{\pgfqpoint{4.068086in}{2.915080in}}{\pgfqpoint{4.060273in}{2.907267in}}%
\pgfpathcurveto{\pgfqpoint{4.052459in}{2.899453in}}{\pgfqpoint{4.048069in}{2.888854in}}{\pgfqpoint{4.048069in}{2.877804in}}%
\pgfpathcurveto{\pgfqpoint{4.048069in}{2.866754in}}{\pgfqpoint{4.052459in}{2.856155in}}{\pgfqpoint{4.060273in}{2.848341in}}%
\pgfpathcurveto{\pgfqpoint{4.068086in}{2.840528in}}{\pgfqpoint{4.078685in}{2.836137in}}{\pgfqpoint{4.089735in}{2.836137in}}%
\pgfpathclose%
\pgfusepath{stroke,fill}%
\end{pgfscope}%
\begin{pgfscope}%
\pgfpathrectangle{\pgfqpoint{0.787074in}{0.548769in}}{\pgfqpoint{5.062926in}{3.102590in}}%
\pgfusepath{clip}%
\pgfsetbuttcap%
\pgfsetroundjoin%
\definecolor{currentfill}{rgb}{0.839216,0.152941,0.156863}%
\pgfsetfillcolor{currentfill}%
\pgfsetlinewidth{1.003750pt}%
\definecolor{currentstroke}{rgb}{0.839216,0.152941,0.156863}%
\pgfsetstrokecolor{currentstroke}%
\pgfsetdash{}{0pt}%
\pgfpathmoveto{\pgfqpoint{4.344052in}{2.834106in}}%
\pgfpathcurveto{\pgfqpoint{4.355102in}{2.834106in}}{\pgfqpoint{4.365701in}{2.838496in}}{\pgfqpoint{4.373515in}{2.846310in}}%
\pgfpathcurveto{\pgfqpoint{4.381328in}{2.854123in}}{\pgfqpoint{4.385719in}{2.864722in}}{\pgfqpoint{4.385719in}{2.875772in}}%
\pgfpathcurveto{\pgfqpoint{4.385719in}{2.886823in}}{\pgfqpoint{4.381328in}{2.897422in}}{\pgfqpoint{4.373515in}{2.905235in}}%
\pgfpathcurveto{\pgfqpoint{4.365701in}{2.913049in}}{\pgfqpoint{4.355102in}{2.917439in}}{\pgfqpoint{4.344052in}{2.917439in}}%
\pgfpathcurveto{\pgfqpoint{4.333002in}{2.917439in}}{\pgfqpoint{4.322403in}{2.913049in}}{\pgfqpoint{4.314589in}{2.905235in}}%
\pgfpathcurveto{\pgfqpoint{4.306776in}{2.897422in}}{\pgfqpoint{4.302385in}{2.886823in}}{\pgfqpoint{4.302385in}{2.875772in}}%
\pgfpathcurveto{\pgfqpoint{4.302385in}{2.864722in}}{\pgfqpoint{4.306776in}{2.854123in}}{\pgfqpoint{4.314589in}{2.846310in}}%
\pgfpathcurveto{\pgfqpoint{4.322403in}{2.838496in}}{\pgfqpoint{4.333002in}{2.834106in}}{\pgfqpoint{4.344052in}{2.834106in}}%
\pgfpathclose%
\pgfusepath{stroke,fill}%
\end{pgfscope}%
\begin{pgfscope}%
\pgfpathrectangle{\pgfqpoint{0.787074in}{0.548769in}}{\pgfqpoint{5.062926in}{3.102590in}}%
\pgfusepath{clip}%
\pgfsetbuttcap%
\pgfsetroundjoin%
\definecolor{currentfill}{rgb}{0.121569,0.466667,0.705882}%
\pgfsetfillcolor{currentfill}%
\pgfsetlinewidth{1.003750pt}%
\definecolor{currentstroke}{rgb}{0.121569,0.466667,0.705882}%
\pgfsetstrokecolor{currentstroke}%
\pgfsetdash{}{0pt}%
\pgfpathmoveto{\pgfqpoint{4.446848in}{2.400262in}}%
\pgfpathcurveto{\pgfqpoint{4.457898in}{2.400262in}}{\pgfqpoint{4.468497in}{2.404652in}}{\pgfqpoint{4.476310in}{2.412465in}}%
\pgfpathcurveto{\pgfqpoint{4.484124in}{2.420279in}}{\pgfqpoint{4.488514in}{2.430878in}}{\pgfqpoint{4.488514in}{2.441928in}}%
\pgfpathcurveto{\pgfqpoint{4.488514in}{2.452978in}}{\pgfqpoint{4.484124in}{2.463577in}}{\pgfqpoint{4.476310in}{2.471391in}}%
\pgfpathcurveto{\pgfqpoint{4.468497in}{2.479205in}}{\pgfqpoint{4.457898in}{2.483595in}}{\pgfqpoint{4.446848in}{2.483595in}}%
\pgfpathcurveto{\pgfqpoint{4.435797in}{2.483595in}}{\pgfqpoint{4.425198in}{2.479205in}}{\pgfqpoint{4.417385in}{2.471391in}}%
\pgfpathcurveto{\pgfqpoint{4.409571in}{2.463577in}}{\pgfqpoint{4.405181in}{2.452978in}}{\pgfqpoint{4.405181in}{2.441928in}}%
\pgfpathcurveto{\pgfqpoint{4.405181in}{2.430878in}}{\pgfqpoint{4.409571in}{2.420279in}}{\pgfqpoint{4.417385in}{2.412465in}}%
\pgfpathcurveto{\pgfqpoint{4.425198in}{2.404652in}}{\pgfqpoint{4.435797in}{2.400262in}}{\pgfqpoint{4.446848in}{2.400262in}}%
\pgfpathclose%
\pgfusepath{stroke,fill}%
\end{pgfscope}%
\begin{pgfscope}%
\pgfpathrectangle{\pgfqpoint{0.787074in}{0.548769in}}{\pgfqpoint{5.062926in}{3.102590in}}%
\pgfusepath{clip}%
\pgfsetbuttcap%
\pgfsetroundjoin%
\definecolor{currentfill}{rgb}{1.000000,0.498039,0.054902}%
\pgfsetfillcolor{currentfill}%
\pgfsetlinewidth{1.003750pt}%
\definecolor{currentstroke}{rgb}{1.000000,0.498039,0.054902}%
\pgfsetstrokecolor{currentstroke}%
\pgfsetdash{}{0pt}%
\pgfpathmoveto{\pgfqpoint{3.923064in}{2.515312in}}%
\pgfpathcurveto{\pgfqpoint{3.934115in}{2.515312in}}{\pgfqpoint{3.944714in}{2.519702in}}{\pgfqpoint{3.952527in}{2.527516in}}%
\pgfpathcurveto{\pgfqpoint{3.960341in}{2.535330in}}{\pgfqpoint{3.964731in}{2.545929in}}{\pgfqpoint{3.964731in}{2.556979in}}%
\pgfpathcurveto{\pgfqpoint{3.964731in}{2.568029in}}{\pgfqpoint{3.960341in}{2.578628in}}{\pgfqpoint{3.952527in}{2.586442in}}%
\pgfpathcurveto{\pgfqpoint{3.944714in}{2.594255in}}{\pgfqpoint{3.934115in}{2.598645in}}{\pgfqpoint{3.923064in}{2.598645in}}%
\pgfpathcurveto{\pgfqpoint{3.912014in}{2.598645in}}{\pgfqpoint{3.901415in}{2.594255in}}{\pgfqpoint{3.893602in}{2.586442in}}%
\pgfpathcurveto{\pgfqpoint{3.885788in}{2.578628in}}{\pgfqpoint{3.881398in}{2.568029in}}{\pgfqpoint{3.881398in}{2.556979in}}%
\pgfpathcurveto{\pgfqpoint{3.881398in}{2.545929in}}{\pgfqpoint{3.885788in}{2.535330in}}{\pgfqpoint{3.893602in}{2.527516in}}%
\pgfpathcurveto{\pgfqpoint{3.901415in}{2.519702in}}{\pgfqpoint{3.912014in}{2.515312in}}{\pgfqpoint{3.923064in}{2.515312in}}%
\pgfpathclose%
\pgfusepath{stroke,fill}%
\end{pgfscope}%
\begin{pgfscope}%
\pgfpathrectangle{\pgfqpoint{0.787074in}{0.548769in}}{\pgfqpoint{5.062926in}{3.102590in}}%
\pgfusepath{clip}%
\pgfsetbuttcap%
\pgfsetroundjoin%
\definecolor{currentfill}{rgb}{1.000000,0.498039,0.054902}%
\pgfsetfillcolor{currentfill}%
\pgfsetlinewidth{1.003750pt}%
\definecolor{currentstroke}{rgb}{1.000000,0.498039,0.054902}%
\pgfsetstrokecolor{currentstroke}%
\pgfsetdash{}{0pt}%
\pgfpathmoveto{\pgfqpoint{4.623883in}{2.940138in}}%
\pgfpathcurveto{\pgfqpoint{4.634933in}{2.940138in}}{\pgfqpoint{4.645532in}{2.944529in}}{\pgfqpoint{4.653346in}{2.952342in}}%
\pgfpathcurveto{\pgfqpoint{4.661159in}{2.960156in}}{\pgfqpoint{4.665550in}{2.970755in}}{\pgfqpoint{4.665550in}{2.981805in}}%
\pgfpathcurveto{\pgfqpoint{4.665550in}{2.992855in}}{\pgfqpoint{4.661159in}{3.003454in}}{\pgfqpoint{4.653346in}{3.011268in}}%
\pgfpathcurveto{\pgfqpoint{4.645532in}{3.019082in}}{\pgfqpoint{4.634933in}{3.023472in}}{\pgfqpoint{4.623883in}{3.023472in}}%
\pgfpathcurveto{\pgfqpoint{4.612833in}{3.023472in}}{\pgfqpoint{4.602234in}{3.019082in}}{\pgfqpoint{4.594420in}{3.011268in}}%
\pgfpathcurveto{\pgfqpoint{4.586607in}{3.003454in}}{\pgfqpoint{4.582216in}{2.992855in}}{\pgfqpoint{4.582216in}{2.981805in}}%
\pgfpathcurveto{\pgfqpoint{4.582216in}{2.970755in}}{\pgfqpoint{4.586607in}{2.960156in}}{\pgfqpoint{4.594420in}{2.952342in}}%
\pgfpathcurveto{\pgfqpoint{4.602234in}{2.944529in}}{\pgfqpoint{4.612833in}{2.940138in}}{\pgfqpoint{4.623883in}{2.940138in}}%
\pgfpathclose%
\pgfusepath{stroke,fill}%
\end{pgfscope}%
\begin{pgfscope}%
\pgfpathrectangle{\pgfqpoint{0.787074in}{0.548769in}}{\pgfqpoint{5.062926in}{3.102590in}}%
\pgfusepath{clip}%
\pgfsetbuttcap%
\pgfsetroundjoin%
\definecolor{currentfill}{rgb}{1.000000,0.498039,0.054902}%
\pgfsetfillcolor{currentfill}%
\pgfsetlinewidth{1.003750pt}%
\definecolor{currentstroke}{rgb}{1.000000,0.498039,0.054902}%
\pgfsetstrokecolor{currentstroke}%
\pgfsetdash{}{0pt}%
\pgfpathmoveto{\pgfqpoint{4.264911in}{3.117335in}}%
\pgfpathcurveto{\pgfqpoint{4.275961in}{3.117335in}}{\pgfqpoint{4.286560in}{3.121725in}}{\pgfqpoint{4.294374in}{3.129539in}}%
\pgfpathcurveto{\pgfqpoint{4.302188in}{3.137352in}}{\pgfqpoint{4.306578in}{3.147951in}}{\pgfqpoint{4.306578in}{3.159001in}}%
\pgfpathcurveto{\pgfqpoint{4.306578in}{3.170052in}}{\pgfqpoint{4.302188in}{3.180651in}}{\pgfqpoint{4.294374in}{3.188464in}}%
\pgfpathcurveto{\pgfqpoint{4.286560in}{3.196278in}}{\pgfqpoint{4.275961in}{3.200668in}}{\pgfqpoint{4.264911in}{3.200668in}}%
\pgfpathcurveto{\pgfqpoint{4.253861in}{3.200668in}}{\pgfqpoint{4.243262in}{3.196278in}}{\pgfqpoint{4.235448in}{3.188464in}}%
\pgfpathcurveto{\pgfqpoint{4.227635in}{3.180651in}}{\pgfqpoint{4.223245in}{3.170052in}}{\pgfqpoint{4.223245in}{3.159001in}}%
\pgfpathcurveto{\pgfqpoint{4.223245in}{3.147951in}}{\pgfqpoint{4.227635in}{3.137352in}}{\pgfqpoint{4.235448in}{3.129539in}}%
\pgfpathcurveto{\pgfqpoint{4.243262in}{3.121725in}}{\pgfqpoint{4.253861in}{3.117335in}}{\pgfqpoint{4.264911in}{3.117335in}}%
\pgfpathclose%
\pgfusepath{stroke,fill}%
\end{pgfscope}%
\begin{pgfscope}%
\pgfpathrectangle{\pgfqpoint{0.787074in}{0.548769in}}{\pgfqpoint{5.062926in}{3.102590in}}%
\pgfusepath{clip}%
\pgfsetbuttcap%
\pgfsetroundjoin%
\definecolor{currentfill}{rgb}{1.000000,0.498039,0.054902}%
\pgfsetfillcolor{currentfill}%
\pgfsetlinewidth{1.003750pt}%
\definecolor{currentstroke}{rgb}{1.000000,0.498039,0.054902}%
\pgfsetstrokecolor{currentstroke}%
\pgfsetdash{}{0pt}%
\pgfpathmoveto{\pgfqpoint{2.816109in}{3.017356in}}%
\pgfpathcurveto{\pgfqpoint{2.827159in}{3.017356in}}{\pgfqpoint{2.837758in}{3.021746in}}{\pgfqpoint{2.845571in}{3.029559in}}%
\pgfpathcurveto{\pgfqpoint{2.853385in}{3.037373in}}{\pgfqpoint{2.857775in}{3.047972in}}{\pgfqpoint{2.857775in}{3.059022in}}%
\pgfpathcurveto{\pgfqpoint{2.857775in}{3.070072in}}{\pgfqpoint{2.853385in}{3.080671in}}{\pgfqpoint{2.845571in}{3.088485in}}%
\pgfpathcurveto{\pgfqpoint{2.837758in}{3.096299in}}{\pgfqpoint{2.827159in}{3.100689in}}{\pgfqpoint{2.816109in}{3.100689in}}%
\pgfpathcurveto{\pgfqpoint{2.805059in}{3.100689in}}{\pgfqpoint{2.794460in}{3.096299in}}{\pgfqpoint{2.786646in}{3.088485in}}%
\pgfpathcurveto{\pgfqpoint{2.778832in}{3.080671in}}{\pgfqpoint{2.774442in}{3.070072in}}{\pgfqpoint{2.774442in}{3.059022in}}%
\pgfpathcurveto{\pgfqpoint{2.774442in}{3.047972in}}{\pgfqpoint{2.778832in}{3.037373in}}{\pgfqpoint{2.786646in}{3.029559in}}%
\pgfpathcurveto{\pgfqpoint{2.794460in}{3.021746in}}{\pgfqpoint{2.805059in}{3.017356in}}{\pgfqpoint{2.816109in}{3.017356in}}%
\pgfpathclose%
\pgfusepath{stroke,fill}%
\end{pgfscope}%
\begin{pgfscope}%
\pgfpathrectangle{\pgfqpoint{0.787074in}{0.548769in}}{\pgfqpoint{5.062926in}{3.102590in}}%
\pgfusepath{clip}%
\pgfsetbuttcap%
\pgfsetroundjoin%
\definecolor{currentfill}{rgb}{1.000000,0.498039,0.054902}%
\pgfsetfillcolor{currentfill}%
\pgfsetlinewidth{1.003750pt}%
\definecolor{currentstroke}{rgb}{1.000000,0.498039,0.054902}%
\pgfsetstrokecolor{currentstroke}%
\pgfsetdash{}{0pt}%
\pgfpathmoveto{\pgfqpoint{3.141442in}{1.775158in}}%
\pgfpathcurveto{\pgfqpoint{3.152492in}{1.775158in}}{\pgfqpoint{3.163091in}{1.779548in}}{\pgfqpoint{3.170905in}{1.787362in}}%
\pgfpathcurveto{\pgfqpoint{3.178718in}{1.795175in}}{\pgfqpoint{3.183109in}{1.805774in}}{\pgfqpoint{3.183109in}{1.816825in}}%
\pgfpathcurveto{\pgfqpoint{3.183109in}{1.827875in}}{\pgfqpoint{3.178718in}{1.838474in}}{\pgfqpoint{3.170905in}{1.846287in}}%
\pgfpathcurveto{\pgfqpoint{3.163091in}{1.854101in}}{\pgfqpoint{3.152492in}{1.858491in}}{\pgfqpoint{3.141442in}{1.858491in}}%
\pgfpathcurveto{\pgfqpoint{3.130392in}{1.858491in}}{\pgfqpoint{3.119793in}{1.854101in}}{\pgfqpoint{3.111979in}{1.846287in}}%
\pgfpathcurveto{\pgfqpoint{3.104166in}{1.838474in}}{\pgfqpoint{3.099775in}{1.827875in}}{\pgfqpoint{3.099775in}{1.816825in}}%
\pgfpathcurveto{\pgfqpoint{3.099775in}{1.805774in}}{\pgfqpoint{3.104166in}{1.795175in}}{\pgfqpoint{3.111979in}{1.787362in}}%
\pgfpathcurveto{\pgfqpoint{3.119793in}{1.779548in}}{\pgfqpoint{3.130392in}{1.775158in}}{\pgfqpoint{3.141442in}{1.775158in}}%
\pgfpathclose%
\pgfusepath{stroke,fill}%
\end{pgfscope}%
\begin{pgfscope}%
\pgfpathrectangle{\pgfqpoint{0.787074in}{0.548769in}}{\pgfqpoint{5.062926in}{3.102590in}}%
\pgfusepath{clip}%
\pgfsetbuttcap%
\pgfsetroundjoin%
\definecolor{currentfill}{rgb}{1.000000,0.498039,0.054902}%
\pgfsetfillcolor{currentfill}%
\pgfsetlinewidth{1.003750pt}%
\definecolor{currentstroke}{rgb}{1.000000,0.498039,0.054902}%
\pgfsetstrokecolor{currentstroke}%
\pgfsetdash{}{0pt}%
\pgfpathmoveto{\pgfqpoint{4.933770in}{3.272382in}}%
\pgfpathcurveto{\pgfqpoint{4.944821in}{3.272382in}}{\pgfqpoint{4.955420in}{3.276772in}}{\pgfqpoint{4.963233in}{3.284586in}}%
\pgfpathcurveto{\pgfqpoint{4.971047in}{3.292399in}}{\pgfqpoint{4.975437in}{3.302998in}}{\pgfqpoint{4.975437in}{3.314049in}}%
\pgfpathcurveto{\pgfqpoint{4.975437in}{3.325099in}}{\pgfqpoint{4.971047in}{3.335698in}}{\pgfqpoint{4.963233in}{3.343511in}}%
\pgfpathcurveto{\pgfqpoint{4.955420in}{3.351325in}}{\pgfqpoint{4.944821in}{3.355715in}}{\pgfqpoint{4.933770in}{3.355715in}}%
\pgfpathcurveto{\pgfqpoint{4.922720in}{3.355715in}}{\pgfqpoint{4.912121in}{3.351325in}}{\pgfqpoint{4.904308in}{3.343511in}}%
\pgfpathcurveto{\pgfqpoint{4.896494in}{3.335698in}}{\pgfqpoint{4.892104in}{3.325099in}}{\pgfqpoint{4.892104in}{3.314049in}}%
\pgfpathcurveto{\pgfqpoint{4.892104in}{3.302998in}}{\pgfqpoint{4.896494in}{3.292399in}}{\pgfqpoint{4.904308in}{3.284586in}}%
\pgfpathcurveto{\pgfqpoint{4.912121in}{3.276772in}}{\pgfqpoint{4.922720in}{3.272382in}}{\pgfqpoint{4.933770in}{3.272382in}}%
\pgfpathclose%
\pgfusepath{stroke,fill}%
\end{pgfscope}%
\begin{pgfscope}%
\pgfpathrectangle{\pgfqpoint{0.787074in}{0.548769in}}{\pgfqpoint{5.062926in}{3.102590in}}%
\pgfusepath{clip}%
\pgfsetbuttcap%
\pgfsetroundjoin%
\definecolor{currentfill}{rgb}{0.121569,0.466667,0.705882}%
\pgfsetfillcolor{currentfill}%
\pgfsetlinewidth{1.003750pt}%
\definecolor{currentstroke}{rgb}{0.121569,0.466667,0.705882}%
\pgfsetstrokecolor{currentstroke}%
\pgfsetdash{}{0pt}%
\pgfpathmoveto{\pgfqpoint{1.077460in}{0.676982in}}%
\pgfpathcurveto{\pgfqpoint{1.088510in}{0.676982in}}{\pgfqpoint{1.099109in}{0.681372in}}{\pgfqpoint{1.106922in}{0.689186in}}%
\pgfpathcurveto{\pgfqpoint{1.114736in}{0.696999in}}{\pgfqpoint{1.119126in}{0.707598in}}{\pgfqpoint{1.119126in}{0.718649in}}%
\pgfpathcurveto{\pgfqpoint{1.119126in}{0.729699in}}{\pgfqpoint{1.114736in}{0.740298in}}{\pgfqpoint{1.106922in}{0.748111in}}%
\pgfpathcurveto{\pgfqpoint{1.099109in}{0.755925in}}{\pgfqpoint{1.088510in}{0.760315in}}{\pgfqpoint{1.077460in}{0.760315in}}%
\pgfpathcurveto{\pgfqpoint{1.066409in}{0.760315in}}{\pgfqpoint{1.055810in}{0.755925in}}{\pgfqpoint{1.047997in}{0.748111in}}%
\pgfpathcurveto{\pgfqpoint{1.040183in}{0.740298in}}{\pgfqpoint{1.035793in}{0.729699in}}{\pgfqpoint{1.035793in}{0.718649in}}%
\pgfpathcurveto{\pgfqpoint{1.035793in}{0.707598in}}{\pgfqpoint{1.040183in}{0.696999in}}{\pgfqpoint{1.047997in}{0.689186in}}%
\pgfpathcurveto{\pgfqpoint{1.055810in}{0.681372in}}{\pgfqpoint{1.066409in}{0.676982in}}{\pgfqpoint{1.077460in}{0.676982in}}%
\pgfpathclose%
\pgfusepath{stroke,fill}%
\end{pgfscope}%
\begin{pgfscope}%
\pgfpathrectangle{\pgfqpoint{0.787074in}{0.548769in}}{\pgfqpoint{5.062926in}{3.102590in}}%
\pgfusepath{clip}%
\pgfsetbuttcap%
\pgfsetroundjoin%
\definecolor{currentfill}{rgb}{0.839216,0.152941,0.156863}%
\pgfsetfillcolor{currentfill}%
\pgfsetlinewidth{1.003750pt}%
\definecolor{currentstroke}{rgb}{0.839216,0.152941,0.156863}%
\pgfsetstrokecolor{currentstroke}%
\pgfsetdash{}{0pt}%
\pgfpathmoveto{\pgfqpoint{3.131147in}{3.256315in}}%
\pgfpathcurveto{\pgfqpoint{3.142198in}{3.256315in}}{\pgfqpoint{3.152797in}{3.260705in}}{\pgfqpoint{3.160610in}{3.268518in}}%
\pgfpathcurveto{\pgfqpoint{3.168424in}{3.276332in}}{\pgfqpoint{3.172814in}{3.286931in}}{\pgfqpoint{3.172814in}{3.297981in}}%
\pgfpathcurveto{\pgfqpoint{3.172814in}{3.309031in}}{\pgfqpoint{3.168424in}{3.319630in}}{\pgfqpoint{3.160610in}{3.327444in}}%
\pgfpathcurveto{\pgfqpoint{3.152797in}{3.335258in}}{\pgfqpoint{3.142198in}{3.339648in}}{\pgfqpoint{3.131147in}{3.339648in}}%
\pgfpathcurveto{\pgfqpoint{3.120097in}{3.339648in}}{\pgfqpoint{3.109498in}{3.335258in}}{\pgfqpoint{3.101685in}{3.327444in}}%
\pgfpathcurveto{\pgfqpoint{3.093871in}{3.319630in}}{\pgfqpoint{3.089481in}{3.309031in}}{\pgfqpoint{3.089481in}{3.297981in}}%
\pgfpathcurveto{\pgfqpoint{3.089481in}{3.286931in}}{\pgfqpoint{3.093871in}{3.276332in}}{\pgfqpoint{3.101685in}{3.268518in}}%
\pgfpathcurveto{\pgfqpoint{3.109498in}{3.260705in}}{\pgfqpoint{3.120097in}{3.256315in}}{\pgfqpoint{3.131147in}{3.256315in}}%
\pgfpathclose%
\pgfusepath{stroke,fill}%
\end{pgfscope}%
\begin{pgfscope}%
\pgfpathrectangle{\pgfqpoint{0.787074in}{0.548769in}}{\pgfqpoint{5.062926in}{3.102590in}}%
\pgfusepath{clip}%
\pgfsetbuttcap%
\pgfsetroundjoin%
\definecolor{currentfill}{rgb}{1.000000,0.498039,0.054902}%
\pgfsetfillcolor{currentfill}%
\pgfsetlinewidth{1.003750pt}%
\definecolor{currentstroke}{rgb}{1.000000,0.498039,0.054902}%
\pgfsetstrokecolor{currentstroke}%
\pgfsetdash{}{0pt}%
\pgfpathmoveto{\pgfqpoint{5.011934in}{2.697852in}}%
\pgfpathcurveto{\pgfqpoint{5.022984in}{2.697852in}}{\pgfqpoint{5.033583in}{2.702242in}}{\pgfqpoint{5.041397in}{2.710056in}}%
\pgfpathcurveto{\pgfqpoint{5.049210in}{2.717869in}}{\pgfqpoint{5.053601in}{2.728468in}}{\pgfqpoint{5.053601in}{2.739518in}}%
\pgfpathcurveto{\pgfqpoint{5.053601in}{2.750568in}}{\pgfqpoint{5.049210in}{2.761167in}}{\pgfqpoint{5.041397in}{2.768981in}}%
\pgfpathcurveto{\pgfqpoint{5.033583in}{2.776795in}}{\pgfqpoint{5.022984in}{2.781185in}}{\pgfqpoint{5.011934in}{2.781185in}}%
\pgfpathcurveto{\pgfqpoint{5.000884in}{2.781185in}}{\pgfqpoint{4.990285in}{2.776795in}}{\pgfqpoint{4.982471in}{2.768981in}}%
\pgfpathcurveto{\pgfqpoint{4.974657in}{2.761167in}}{\pgfqpoint{4.970267in}{2.750568in}}{\pgfqpoint{4.970267in}{2.739518in}}%
\pgfpathcurveto{\pgfqpoint{4.970267in}{2.728468in}}{\pgfqpoint{4.974657in}{2.717869in}}{\pgfqpoint{4.982471in}{2.710056in}}%
\pgfpathcurveto{\pgfqpoint{4.990285in}{2.702242in}}{\pgfqpoint{5.000884in}{2.697852in}}{\pgfqpoint{5.011934in}{2.697852in}}%
\pgfpathclose%
\pgfusepath{stroke,fill}%
\end{pgfscope}%
\begin{pgfscope}%
\pgfpathrectangle{\pgfqpoint{0.787074in}{0.548769in}}{\pgfqpoint{5.062926in}{3.102590in}}%
\pgfusepath{clip}%
\pgfsetbuttcap%
\pgfsetroundjoin%
\definecolor{currentfill}{rgb}{0.121569,0.466667,0.705882}%
\pgfsetfillcolor{currentfill}%
\pgfsetlinewidth{1.003750pt}%
\definecolor{currentstroke}{rgb}{0.121569,0.466667,0.705882}%
\pgfsetstrokecolor{currentstroke}%
\pgfsetdash{}{0pt}%
\pgfpathmoveto{\pgfqpoint{1.873766in}{1.421237in}}%
\pgfpathcurveto{\pgfqpoint{1.884816in}{1.421237in}}{\pgfqpoint{1.895415in}{1.425628in}}{\pgfqpoint{1.903229in}{1.433441in}}%
\pgfpathcurveto{\pgfqpoint{1.911042in}{1.441255in}}{\pgfqpoint{1.915433in}{1.451854in}}{\pgfqpoint{1.915433in}{1.462904in}}%
\pgfpathcurveto{\pgfqpoint{1.915433in}{1.473954in}}{\pgfqpoint{1.911042in}{1.484553in}}{\pgfqpoint{1.903229in}{1.492367in}}%
\pgfpathcurveto{\pgfqpoint{1.895415in}{1.500180in}}{\pgfqpoint{1.884816in}{1.504571in}}{\pgfqpoint{1.873766in}{1.504571in}}%
\pgfpathcurveto{\pgfqpoint{1.862716in}{1.504571in}}{\pgfqpoint{1.852117in}{1.500180in}}{\pgfqpoint{1.844303in}{1.492367in}}%
\pgfpathcurveto{\pgfqpoint{1.836489in}{1.484553in}}{\pgfqpoint{1.832099in}{1.473954in}}{\pgfqpoint{1.832099in}{1.462904in}}%
\pgfpathcurveto{\pgfqpoint{1.832099in}{1.451854in}}{\pgfqpoint{1.836489in}{1.441255in}}{\pgfqpoint{1.844303in}{1.433441in}}%
\pgfpathcurveto{\pgfqpoint{1.852117in}{1.425628in}}{\pgfqpoint{1.862716in}{1.421237in}}{\pgfqpoint{1.873766in}{1.421237in}}%
\pgfpathclose%
\pgfusepath{stroke,fill}%
\end{pgfscope}%
\begin{pgfscope}%
\pgfpathrectangle{\pgfqpoint{0.787074in}{0.548769in}}{\pgfqpoint{5.062926in}{3.102590in}}%
\pgfusepath{clip}%
\pgfsetbuttcap%
\pgfsetroundjoin%
\definecolor{currentfill}{rgb}{0.121569,0.466667,0.705882}%
\pgfsetfillcolor{currentfill}%
\pgfsetlinewidth{1.003750pt}%
\definecolor{currentstroke}{rgb}{0.121569,0.466667,0.705882}%
\pgfsetstrokecolor{currentstroke}%
\pgfsetdash{}{0pt}%
\pgfpathmoveto{\pgfqpoint{2.349462in}{1.410498in}}%
\pgfpathcurveto{\pgfqpoint{2.360512in}{1.410498in}}{\pgfqpoint{2.371111in}{1.414888in}}{\pgfqpoint{2.378925in}{1.422702in}}%
\pgfpathcurveto{\pgfqpoint{2.386739in}{1.430515in}}{\pgfqpoint{2.391129in}{1.441114in}}{\pgfqpoint{2.391129in}{1.452164in}}%
\pgfpathcurveto{\pgfqpoint{2.391129in}{1.463214in}}{\pgfqpoint{2.386739in}{1.473813in}}{\pgfqpoint{2.378925in}{1.481627in}}%
\pgfpathcurveto{\pgfqpoint{2.371111in}{1.489441in}}{\pgfqpoint{2.360512in}{1.493831in}}{\pgfqpoint{2.349462in}{1.493831in}}%
\pgfpathcurveto{\pgfqpoint{2.338412in}{1.493831in}}{\pgfqpoint{2.327813in}{1.489441in}}{\pgfqpoint{2.319999in}{1.481627in}}%
\pgfpathcurveto{\pgfqpoint{2.312186in}{1.473813in}}{\pgfqpoint{2.307795in}{1.463214in}}{\pgfqpoint{2.307795in}{1.452164in}}%
\pgfpathcurveto{\pgfqpoint{2.307795in}{1.441114in}}{\pgfqpoint{2.312186in}{1.430515in}}{\pgfqpoint{2.319999in}{1.422702in}}%
\pgfpathcurveto{\pgfqpoint{2.327813in}{1.414888in}}{\pgfqpoint{2.338412in}{1.410498in}}{\pgfqpoint{2.349462in}{1.410498in}}%
\pgfpathclose%
\pgfusepath{stroke,fill}%
\end{pgfscope}%
\begin{pgfscope}%
\pgfpathrectangle{\pgfqpoint{0.787074in}{0.548769in}}{\pgfqpoint{5.062926in}{3.102590in}}%
\pgfusepath{clip}%
\pgfsetbuttcap%
\pgfsetroundjoin%
\definecolor{currentfill}{rgb}{0.121569,0.466667,0.705882}%
\pgfsetfillcolor{currentfill}%
\pgfsetlinewidth{1.003750pt}%
\definecolor{currentstroke}{rgb}{0.121569,0.466667,0.705882}%
\pgfsetstrokecolor{currentstroke}%
\pgfsetdash{}{0pt}%
\pgfpathmoveto{\pgfqpoint{1.919931in}{1.127209in}}%
\pgfpathcurveto{\pgfqpoint{1.930981in}{1.127209in}}{\pgfqpoint{1.941580in}{1.131599in}}{\pgfqpoint{1.949394in}{1.139413in}}%
\pgfpathcurveto{\pgfqpoint{1.957207in}{1.147226in}}{\pgfqpoint{1.961598in}{1.157825in}}{\pgfqpoint{1.961598in}{1.168876in}}%
\pgfpathcurveto{\pgfqpoint{1.961598in}{1.179926in}}{\pgfqpoint{1.957207in}{1.190525in}}{\pgfqpoint{1.949394in}{1.198338in}}%
\pgfpathcurveto{\pgfqpoint{1.941580in}{1.206152in}}{\pgfqpoint{1.930981in}{1.210542in}}{\pgfqpoint{1.919931in}{1.210542in}}%
\pgfpathcurveto{\pgfqpoint{1.908881in}{1.210542in}}{\pgfqpoint{1.898282in}{1.206152in}}{\pgfqpoint{1.890468in}{1.198338in}}%
\pgfpathcurveto{\pgfqpoint{1.882654in}{1.190525in}}{\pgfqpoint{1.878264in}{1.179926in}}{\pgfqpoint{1.878264in}{1.168876in}}%
\pgfpathcurveto{\pgfqpoint{1.878264in}{1.157825in}}{\pgfqpoint{1.882654in}{1.147226in}}{\pgfqpoint{1.890468in}{1.139413in}}%
\pgfpathcurveto{\pgfqpoint{1.898282in}{1.131599in}}{\pgfqpoint{1.908881in}{1.127209in}}{\pgfqpoint{1.919931in}{1.127209in}}%
\pgfpathclose%
\pgfusepath{stroke,fill}%
\end{pgfscope}%
\begin{pgfscope}%
\pgfpathrectangle{\pgfqpoint{0.787074in}{0.548769in}}{\pgfqpoint{5.062926in}{3.102590in}}%
\pgfusepath{clip}%
\pgfsetbuttcap%
\pgfsetroundjoin%
\definecolor{currentfill}{rgb}{0.121569,0.466667,0.705882}%
\pgfsetfillcolor{currentfill}%
\pgfsetlinewidth{1.003750pt}%
\definecolor{currentstroke}{rgb}{0.121569,0.466667,0.705882}%
\pgfsetstrokecolor{currentstroke}%
\pgfsetdash{}{0pt}%
\pgfpathmoveto{\pgfqpoint{1.511512in}{0.894035in}}%
\pgfpathcurveto{\pgfqpoint{1.522563in}{0.894035in}}{\pgfqpoint{1.533162in}{0.898425in}}{\pgfqpoint{1.540975in}{0.906239in}}%
\pgfpathcurveto{\pgfqpoint{1.548789in}{0.914052in}}{\pgfqpoint{1.553179in}{0.924651in}}{\pgfqpoint{1.553179in}{0.935701in}}%
\pgfpathcurveto{\pgfqpoint{1.553179in}{0.946752in}}{\pgfqpoint{1.548789in}{0.957351in}}{\pgfqpoint{1.540975in}{0.965164in}}%
\pgfpathcurveto{\pgfqpoint{1.533162in}{0.972978in}}{\pgfqpoint{1.522563in}{0.977368in}}{\pgfqpoint{1.511512in}{0.977368in}}%
\pgfpathcurveto{\pgfqpoint{1.500462in}{0.977368in}}{\pgfqpoint{1.489863in}{0.972978in}}{\pgfqpoint{1.482050in}{0.965164in}}%
\pgfpathcurveto{\pgfqpoint{1.474236in}{0.957351in}}{\pgfqpoint{1.469846in}{0.946752in}}{\pgfqpoint{1.469846in}{0.935701in}}%
\pgfpathcurveto{\pgfqpoint{1.469846in}{0.924651in}}{\pgfqpoint{1.474236in}{0.914052in}}{\pgfqpoint{1.482050in}{0.906239in}}%
\pgfpathcurveto{\pgfqpoint{1.489863in}{0.898425in}}{\pgfqpoint{1.500462in}{0.894035in}}{\pgfqpoint{1.511512in}{0.894035in}}%
\pgfpathclose%
\pgfusepath{stroke,fill}%
\end{pgfscope}%
\begin{pgfscope}%
\pgfpathrectangle{\pgfqpoint{0.787074in}{0.548769in}}{\pgfqpoint{5.062926in}{3.102590in}}%
\pgfusepath{clip}%
\pgfsetbuttcap%
\pgfsetroundjoin%
\definecolor{currentfill}{rgb}{0.839216,0.152941,0.156863}%
\pgfsetfillcolor{currentfill}%
\pgfsetlinewidth{1.003750pt}%
\definecolor{currentstroke}{rgb}{0.839216,0.152941,0.156863}%
\pgfsetstrokecolor{currentstroke}%
\pgfsetdash{}{0pt}%
\pgfpathmoveto{\pgfqpoint{5.014308in}{3.187559in}}%
\pgfpathcurveto{\pgfqpoint{5.025358in}{3.187559in}}{\pgfqpoint{5.035957in}{3.191949in}}{\pgfqpoint{5.043771in}{3.199763in}}%
\pgfpathcurveto{\pgfqpoint{5.051584in}{3.207576in}}{\pgfqpoint{5.055975in}{3.218175in}}{\pgfqpoint{5.055975in}{3.229226in}}%
\pgfpathcurveto{\pgfqpoint{5.055975in}{3.240276in}}{\pgfqpoint{5.051584in}{3.250875in}}{\pgfqpoint{5.043771in}{3.258688in}}%
\pgfpathcurveto{\pgfqpoint{5.035957in}{3.266502in}}{\pgfqpoint{5.025358in}{3.270892in}}{\pgfqpoint{5.014308in}{3.270892in}}%
\pgfpathcurveto{\pgfqpoint{5.003258in}{3.270892in}}{\pgfqpoint{4.992659in}{3.266502in}}{\pgfqpoint{4.984845in}{3.258688in}}%
\pgfpathcurveto{\pgfqpoint{4.977031in}{3.250875in}}{\pgfqpoint{4.972641in}{3.240276in}}{\pgfqpoint{4.972641in}{3.229226in}}%
\pgfpathcurveto{\pgfqpoint{4.972641in}{3.218175in}}{\pgfqpoint{4.977031in}{3.207576in}}{\pgfqpoint{4.984845in}{3.199763in}}%
\pgfpathcurveto{\pgfqpoint{4.992659in}{3.191949in}}{\pgfqpoint{5.003258in}{3.187559in}}{\pgfqpoint{5.014308in}{3.187559in}}%
\pgfpathclose%
\pgfusepath{stroke,fill}%
\end{pgfscope}%
\begin{pgfscope}%
\pgfsetbuttcap%
\pgfsetroundjoin%
\definecolor{currentfill}{rgb}{0.000000,0.000000,0.000000}%
\pgfsetfillcolor{currentfill}%
\pgfsetlinewidth{0.803000pt}%
\definecolor{currentstroke}{rgb}{0.000000,0.000000,0.000000}%
\pgfsetstrokecolor{currentstroke}%
\pgfsetdash{}{0pt}%
\pgfsys@defobject{currentmarker}{\pgfqpoint{0.000000in}{-0.048611in}}{\pgfqpoint{0.000000in}{0.000000in}}{%
\pgfpathmoveto{\pgfqpoint{0.000000in}{0.000000in}}%
\pgfpathlineto{\pgfqpoint{0.000000in}{-0.048611in}}%
\pgfusepath{stroke,fill}%
}%
\begin{pgfscope}%
\pgfsys@transformshift{1.017207in}{0.548769in}%
\pgfsys@useobject{currentmarker}{}%
\end{pgfscope}%
\end{pgfscope}%
\begin{pgfscope}%
\definecolor{textcolor}{rgb}{0.000000,0.000000,0.000000}%
\pgfsetstrokecolor{textcolor}%
\pgfsetfillcolor{textcolor}%
\pgftext[x=1.017207in,y=0.451547in,,top]{\color{textcolor}\sffamily\fontsize{10.000000}{12.000000}\selectfont \(\displaystyle {0.0}\)}%
\end{pgfscope}%
\begin{pgfscope}%
\pgfsetbuttcap%
\pgfsetroundjoin%
\definecolor{currentfill}{rgb}{0.000000,0.000000,0.000000}%
\pgfsetfillcolor{currentfill}%
\pgfsetlinewidth{0.803000pt}%
\definecolor{currentstroke}{rgb}{0.000000,0.000000,0.000000}%
\pgfsetstrokecolor{currentstroke}%
\pgfsetdash{}{0pt}%
\pgfsys@defobject{currentmarker}{\pgfqpoint{0.000000in}{-0.048611in}}{\pgfqpoint{0.000000in}{0.000000in}}{%
\pgfpathmoveto{\pgfqpoint{0.000000in}{0.000000in}}%
\pgfpathlineto{\pgfqpoint{0.000000in}{-0.048611in}}%
\pgfusepath{stroke,fill}%
}%
\begin{pgfscope}%
\pgfsys@transformshift{1.479312in}{0.548769in}%
\pgfsys@useobject{currentmarker}{}%
\end{pgfscope}%
\end{pgfscope}%
\begin{pgfscope}%
\definecolor{textcolor}{rgb}{0.000000,0.000000,0.000000}%
\pgfsetstrokecolor{textcolor}%
\pgfsetfillcolor{textcolor}%
\pgftext[x=1.479312in,y=0.451547in,,top]{\color{textcolor}\sffamily\fontsize{10.000000}{12.000000}\selectfont \(\displaystyle {0.1}\)}%
\end{pgfscope}%
\begin{pgfscope}%
\pgfsetbuttcap%
\pgfsetroundjoin%
\definecolor{currentfill}{rgb}{0.000000,0.000000,0.000000}%
\pgfsetfillcolor{currentfill}%
\pgfsetlinewidth{0.803000pt}%
\definecolor{currentstroke}{rgb}{0.000000,0.000000,0.000000}%
\pgfsetstrokecolor{currentstroke}%
\pgfsetdash{}{0pt}%
\pgfsys@defobject{currentmarker}{\pgfqpoint{0.000000in}{-0.048611in}}{\pgfqpoint{0.000000in}{0.000000in}}{%
\pgfpathmoveto{\pgfqpoint{0.000000in}{0.000000in}}%
\pgfpathlineto{\pgfqpoint{0.000000in}{-0.048611in}}%
\pgfusepath{stroke,fill}%
}%
\begin{pgfscope}%
\pgfsys@transformshift{1.941418in}{0.548769in}%
\pgfsys@useobject{currentmarker}{}%
\end{pgfscope}%
\end{pgfscope}%
\begin{pgfscope}%
\definecolor{textcolor}{rgb}{0.000000,0.000000,0.000000}%
\pgfsetstrokecolor{textcolor}%
\pgfsetfillcolor{textcolor}%
\pgftext[x=1.941418in,y=0.451547in,,top]{\color{textcolor}\sffamily\fontsize{10.000000}{12.000000}\selectfont \(\displaystyle {0.2}\)}%
\end{pgfscope}%
\begin{pgfscope}%
\pgfsetbuttcap%
\pgfsetroundjoin%
\definecolor{currentfill}{rgb}{0.000000,0.000000,0.000000}%
\pgfsetfillcolor{currentfill}%
\pgfsetlinewidth{0.803000pt}%
\definecolor{currentstroke}{rgb}{0.000000,0.000000,0.000000}%
\pgfsetstrokecolor{currentstroke}%
\pgfsetdash{}{0pt}%
\pgfsys@defobject{currentmarker}{\pgfqpoint{0.000000in}{-0.048611in}}{\pgfqpoint{0.000000in}{0.000000in}}{%
\pgfpathmoveto{\pgfqpoint{0.000000in}{0.000000in}}%
\pgfpathlineto{\pgfqpoint{0.000000in}{-0.048611in}}%
\pgfusepath{stroke,fill}%
}%
\begin{pgfscope}%
\pgfsys@transformshift{2.403523in}{0.548769in}%
\pgfsys@useobject{currentmarker}{}%
\end{pgfscope}%
\end{pgfscope}%
\begin{pgfscope}%
\definecolor{textcolor}{rgb}{0.000000,0.000000,0.000000}%
\pgfsetstrokecolor{textcolor}%
\pgfsetfillcolor{textcolor}%
\pgftext[x=2.403523in,y=0.451547in,,top]{\color{textcolor}\sffamily\fontsize{10.000000}{12.000000}\selectfont \(\displaystyle {0.3}\)}%
\end{pgfscope}%
\begin{pgfscope}%
\pgfsetbuttcap%
\pgfsetroundjoin%
\definecolor{currentfill}{rgb}{0.000000,0.000000,0.000000}%
\pgfsetfillcolor{currentfill}%
\pgfsetlinewidth{0.803000pt}%
\definecolor{currentstroke}{rgb}{0.000000,0.000000,0.000000}%
\pgfsetstrokecolor{currentstroke}%
\pgfsetdash{}{0pt}%
\pgfsys@defobject{currentmarker}{\pgfqpoint{0.000000in}{-0.048611in}}{\pgfqpoint{0.000000in}{0.000000in}}{%
\pgfpathmoveto{\pgfqpoint{0.000000in}{0.000000in}}%
\pgfpathlineto{\pgfqpoint{0.000000in}{-0.048611in}}%
\pgfusepath{stroke,fill}%
}%
\begin{pgfscope}%
\pgfsys@transformshift{2.865629in}{0.548769in}%
\pgfsys@useobject{currentmarker}{}%
\end{pgfscope}%
\end{pgfscope}%
\begin{pgfscope}%
\definecolor{textcolor}{rgb}{0.000000,0.000000,0.000000}%
\pgfsetstrokecolor{textcolor}%
\pgfsetfillcolor{textcolor}%
\pgftext[x=2.865629in,y=0.451547in,,top]{\color{textcolor}\sffamily\fontsize{10.000000}{12.000000}\selectfont \(\displaystyle {0.4}\)}%
\end{pgfscope}%
\begin{pgfscope}%
\pgfsetbuttcap%
\pgfsetroundjoin%
\definecolor{currentfill}{rgb}{0.000000,0.000000,0.000000}%
\pgfsetfillcolor{currentfill}%
\pgfsetlinewidth{0.803000pt}%
\definecolor{currentstroke}{rgb}{0.000000,0.000000,0.000000}%
\pgfsetstrokecolor{currentstroke}%
\pgfsetdash{}{0pt}%
\pgfsys@defobject{currentmarker}{\pgfqpoint{0.000000in}{-0.048611in}}{\pgfqpoint{0.000000in}{0.000000in}}{%
\pgfpathmoveto{\pgfqpoint{0.000000in}{0.000000in}}%
\pgfpathlineto{\pgfqpoint{0.000000in}{-0.048611in}}%
\pgfusepath{stroke,fill}%
}%
\begin{pgfscope}%
\pgfsys@transformshift{3.327734in}{0.548769in}%
\pgfsys@useobject{currentmarker}{}%
\end{pgfscope}%
\end{pgfscope}%
\begin{pgfscope}%
\definecolor{textcolor}{rgb}{0.000000,0.000000,0.000000}%
\pgfsetstrokecolor{textcolor}%
\pgfsetfillcolor{textcolor}%
\pgftext[x=3.327734in,y=0.451547in,,top]{\color{textcolor}\sffamily\fontsize{10.000000}{12.000000}\selectfont \(\displaystyle {0.5}\)}%
\end{pgfscope}%
\begin{pgfscope}%
\pgfsetbuttcap%
\pgfsetroundjoin%
\definecolor{currentfill}{rgb}{0.000000,0.000000,0.000000}%
\pgfsetfillcolor{currentfill}%
\pgfsetlinewidth{0.803000pt}%
\definecolor{currentstroke}{rgb}{0.000000,0.000000,0.000000}%
\pgfsetstrokecolor{currentstroke}%
\pgfsetdash{}{0pt}%
\pgfsys@defobject{currentmarker}{\pgfqpoint{0.000000in}{-0.048611in}}{\pgfqpoint{0.000000in}{0.000000in}}{%
\pgfpathmoveto{\pgfqpoint{0.000000in}{0.000000in}}%
\pgfpathlineto{\pgfqpoint{0.000000in}{-0.048611in}}%
\pgfusepath{stroke,fill}%
}%
\begin{pgfscope}%
\pgfsys@transformshift{3.789839in}{0.548769in}%
\pgfsys@useobject{currentmarker}{}%
\end{pgfscope}%
\end{pgfscope}%
\begin{pgfscope}%
\definecolor{textcolor}{rgb}{0.000000,0.000000,0.000000}%
\pgfsetstrokecolor{textcolor}%
\pgfsetfillcolor{textcolor}%
\pgftext[x=3.789839in,y=0.451547in,,top]{\color{textcolor}\sffamily\fontsize{10.000000}{12.000000}\selectfont \(\displaystyle {0.6}\)}%
\end{pgfscope}%
\begin{pgfscope}%
\pgfsetbuttcap%
\pgfsetroundjoin%
\definecolor{currentfill}{rgb}{0.000000,0.000000,0.000000}%
\pgfsetfillcolor{currentfill}%
\pgfsetlinewidth{0.803000pt}%
\definecolor{currentstroke}{rgb}{0.000000,0.000000,0.000000}%
\pgfsetstrokecolor{currentstroke}%
\pgfsetdash{}{0pt}%
\pgfsys@defobject{currentmarker}{\pgfqpoint{0.000000in}{-0.048611in}}{\pgfqpoint{0.000000in}{0.000000in}}{%
\pgfpathmoveto{\pgfqpoint{0.000000in}{0.000000in}}%
\pgfpathlineto{\pgfqpoint{0.000000in}{-0.048611in}}%
\pgfusepath{stroke,fill}%
}%
\begin{pgfscope}%
\pgfsys@transformshift{4.251945in}{0.548769in}%
\pgfsys@useobject{currentmarker}{}%
\end{pgfscope}%
\end{pgfscope}%
\begin{pgfscope}%
\definecolor{textcolor}{rgb}{0.000000,0.000000,0.000000}%
\pgfsetstrokecolor{textcolor}%
\pgfsetfillcolor{textcolor}%
\pgftext[x=4.251945in,y=0.451547in,,top]{\color{textcolor}\sffamily\fontsize{10.000000}{12.000000}\selectfont \(\displaystyle {0.7}\)}%
\end{pgfscope}%
\begin{pgfscope}%
\pgfsetbuttcap%
\pgfsetroundjoin%
\definecolor{currentfill}{rgb}{0.000000,0.000000,0.000000}%
\pgfsetfillcolor{currentfill}%
\pgfsetlinewidth{0.803000pt}%
\definecolor{currentstroke}{rgb}{0.000000,0.000000,0.000000}%
\pgfsetstrokecolor{currentstroke}%
\pgfsetdash{}{0pt}%
\pgfsys@defobject{currentmarker}{\pgfqpoint{0.000000in}{-0.048611in}}{\pgfqpoint{0.000000in}{0.000000in}}{%
\pgfpathmoveto{\pgfqpoint{0.000000in}{0.000000in}}%
\pgfpathlineto{\pgfqpoint{0.000000in}{-0.048611in}}%
\pgfusepath{stroke,fill}%
}%
\begin{pgfscope}%
\pgfsys@transformshift{4.714050in}{0.548769in}%
\pgfsys@useobject{currentmarker}{}%
\end{pgfscope}%
\end{pgfscope}%
\begin{pgfscope}%
\definecolor{textcolor}{rgb}{0.000000,0.000000,0.000000}%
\pgfsetstrokecolor{textcolor}%
\pgfsetfillcolor{textcolor}%
\pgftext[x=4.714050in,y=0.451547in,,top]{\color{textcolor}\sffamily\fontsize{10.000000}{12.000000}\selectfont \(\displaystyle {0.8}\)}%
\end{pgfscope}%
\begin{pgfscope}%
\pgfsetbuttcap%
\pgfsetroundjoin%
\definecolor{currentfill}{rgb}{0.000000,0.000000,0.000000}%
\pgfsetfillcolor{currentfill}%
\pgfsetlinewidth{0.803000pt}%
\definecolor{currentstroke}{rgb}{0.000000,0.000000,0.000000}%
\pgfsetstrokecolor{currentstroke}%
\pgfsetdash{}{0pt}%
\pgfsys@defobject{currentmarker}{\pgfqpoint{0.000000in}{-0.048611in}}{\pgfqpoint{0.000000in}{0.000000in}}{%
\pgfpathmoveto{\pgfqpoint{0.000000in}{0.000000in}}%
\pgfpathlineto{\pgfqpoint{0.000000in}{-0.048611in}}%
\pgfusepath{stroke,fill}%
}%
\begin{pgfscope}%
\pgfsys@transformshift{5.176156in}{0.548769in}%
\pgfsys@useobject{currentmarker}{}%
\end{pgfscope}%
\end{pgfscope}%
\begin{pgfscope}%
\definecolor{textcolor}{rgb}{0.000000,0.000000,0.000000}%
\pgfsetstrokecolor{textcolor}%
\pgfsetfillcolor{textcolor}%
\pgftext[x=5.176156in,y=0.451547in,,top]{\color{textcolor}\sffamily\fontsize{10.000000}{12.000000}\selectfont \(\displaystyle {0.9}\)}%
\end{pgfscope}%
\begin{pgfscope}%
\pgfsetbuttcap%
\pgfsetroundjoin%
\definecolor{currentfill}{rgb}{0.000000,0.000000,0.000000}%
\pgfsetfillcolor{currentfill}%
\pgfsetlinewidth{0.803000pt}%
\definecolor{currentstroke}{rgb}{0.000000,0.000000,0.000000}%
\pgfsetstrokecolor{currentstroke}%
\pgfsetdash{}{0pt}%
\pgfsys@defobject{currentmarker}{\pgfqpoint{0.000000in}{-0.048611in}}{\pgfqpoint{0.000000in}{0.000000in}}{%
\pgfpathmoveto{\pgfqpoint{0.000000in}{0.000000in}}%
\pgfpathlineto{\pgfqpoint{0.000000in}{-0.048611in}}%
\pgfusepath{stroke,fill}%
}%
\begin{pgfscope}%
\pgfsys@transformshift{5.638261in}{0.548769in}%
\pgfsys@useobject{currentmarker}{}%
\end{pgfscope}%
\end{pgfscope}%
\begin{pgfscope}%
\definecolor{textcolor}{rgb}{0.000000,0.000000,0.000000}%
\pgfsetstrokecolor{textcolor}%
\pgfsetfillcolor{textcolor}%
\pgftext[x=5.638261in,y=0.451547in,,top]{\color{textcolor}\sffamily\fontsize{10.000000}{12.000000}\selectfont \(\displaystyle {1.0}\)}%
\end{pgfscope}%
\begin{pgfscope}%
\definecolor{textcolor}{rgb}{0.000000,0.000000,0.000000}%
\pgfsetstrokecolor{textcolor}%
\pgfsetfillcolor{textcolor}%
\pgftext[x=3.318537in,y=0.272658in,,top]{\color{textcolor}\sffamily\fontsize{10.000000}{12.000000}\selectfont Edge Count}%
\end{pgfscope}%
\begin{pgfscope}%
\definecolor{textcolor}{rgb}{0.000000,0.000000,0.000000}%
\pgfsetstrokecolor{textcolor}%
\pgfsetfillcolor{textcolor}%
\pgftext[x=5.850000in,y=0.286547in,right,top]{\color{textcolor}\sffamily\fontsize{10.000000}{12.000000}\selectfont \(\displaystyle \times{10^{8}}{}\)}%
\end{pgfscope}%
\begin{pgfscope}%
\pgfsetbuttcap%
\pgfsetroundjoin%
\definecolor{currentfill}{rgb}{0.000000,0.000000,0.000000}%
\pgfsetfillcolor{currentfill}%
\pgfsetlinewidth{0.803000pt}%
\definecolor{currentstroke}{rgb}{0.000000,0.000000,0.000000}%
\pgfsetstrokecolor{currentstroke}%
\pgfsetdash{}{0pt}%
\pgfsys@defobject{currentmarker}{\pgfqpoint{-0.048611in}{0.000000in}}{\pgfqpoint{0.000000in}{0.000000in}}{%
\pgfpathmoveto{\pgfqpoint{0.000000in}{0.000000in}}%
\pgfpathlineto{\pgfqpoint{-0.048611in}{0.000000in}}%
\pgfusepath{stroke,fill}%
}%
\begin{pgfscope}%
\pgfsys@transformshift{0.787074in}{0.689795in}%
\pgfsys@useobject{currentmarker}{}%
\end{pgfscope}%
\end{pgfscope}%
\begin{pgfscope}%
\definecolor{textcolor}{rgb}{0.000000,0.000000,0.000000}%
\pgfsetstrokecolor{textcolor}%
\pgfsetfillcolor{textcolor}%
\pgftext[x=0.620407in, y=0.641601in, left, base]{\color{textcolor}\sffamily\fontsize{10.000000}{12.000000}\selectfont \(\displaystyle {0}\)}%
\end{pgfscope}%
\begin{pgfscope}%
\pgfsetbuttcap%
\pgfsetroundjoin%
\definecolor{currentfill}{rgb}{0.000000,0.000000,0.000000}%
\pgfsetfillcolor{currentfill}%
\pgfsetlinewidth{0.803000pt}%
\definecolor{currentstroke}{rgb}{0.000000,0.000000,0.000000}%
\pgfsetstrokecolor{currentstroke}%
\pgfsetdash{}{0pt}%
\pgfsys@defobject{currentmarker}{\pgfqpoint{-0.048611in}{0.000000in}}{\pgfqpoint{0.000000in}{0.000000in}}{%
\pgfpathmoveto{\pgfqpoint{0.000000in}{0.000000in}}%
\pgfpathlineto{\pgfqpoint{-0.048611in}{0.000000in}}%
\pgfusepath{stroke,fill}%
}%
\begin{pgfscope}%
\pgfsys@transformshift{0.787074in}{1.060560in}%
\pgfsys@useobject{currentmarker}{}%
\end{pgfscope}%
\end{pgfscope}%
\begin{pgfscope}%
\definecolor{textcolor}{rgb}{0.000000,0.000000,0.000000}%
\pgfsetstrokecolor{textcolor}%
\pgfsetfillcolor{textcolor}%
\pgftext[x=0.412073in, y=1.012365in, left, base]{\color{textcolor}\sffamily\fontsize{10.000000}{12.000000}\selectfont \(\displaystyle {2500}\)}%
\end{pgfscope}%
\begin{pgfscope}%
\pgfsetbuttcap%
\pgfsetroundjoin%
\definecolor{currentfill}{rgb}{0.000000,0.000000,0.000000}%
\pgfsetfillcolor{currentfill}%
\pgfsetlinewidth{0.803000pt}%
\definecolor{currentstroke}{rgb}{0.000000,0.000000,0.000000}%
\pgfsetstrokecolor{currentstroke}%
\pgfsetdash{}{0pt}%
\pgfsys@defobject{currentmarker}{\pgfqpoint{-0.048611in}{0.000000in}}{\pgfqpoint{0.000000in}{0.000000in}}{%
\pgfpathmoveto{\pgfqpoint{0.000000in}{0.000000in}}%
\pgfpathlineto{\pgfqpoint{-0.048611in}{0.000000in}}%
\pgfusepath{stroke,fill}%
}%
\begin{pgfscope}%
\pgfsys@transformshift{0.787074in}{1.431324in}%
\pgfsys@useobject{currentmarker}{}%
\end{pgfscope}%
\end{pgfscope}%
\begin{pgfscope}%
\definecolor{textcolor}{rgb}{0.000000,0.000000,0.000000}%
\pgfsetstrokecolor{textcolor}%
\pgfsetfillcolor{textcolor}%
\pgftext[x=0.412073in, y=1.383129in, left, base]{\color{textcolor}\sffamily\fontsize{10.000000}{12.000000}\selectfont \(\displaystyle {5000}\)}%
\end{pgfscope}%
\begin{pgfscope}%
\pgfsetbuttcap%
\pgfsetroundjoin%
\definecolor{currentfill}{rgb}{0.000000,0.000000,0.000000}%
\pgfsetfillcolor{currentfill}%
\pgfsetlinewidth{0.803000pt}%
\definecolor{currentstroke}{rgb}{0.000000,0.000000,0.000000}%
\pgfsetstrokecolor{currentstroke}%
\pgfsetdash{}{0pt}%
\pgfsys@defobject{currentmarker}{\pgfqpoint{-0.048611in}{0.000000in}}{\pgfqpoint{0.000000in}{0.000000in}}{%
\pgfpathmoveto{\pgfqpoint{0.000000in}{0.000000in}}%
\pgfpathlineto{\pgfqpoint{-0.048611in}{0.000000in}}%
\pgfusepath{stroke,fill}%
}%
\begin{pgfscope}%
\pgfsys@transformshift{0.787074in}{1.802088in}%
\pgfsys@useobject{currentmarker}{}%
\end{pgfscope}%
\end{pgfscope}%
\begin{pgfscope}%
\definecolor{textcolor}{rgb}{0.000000,0.000000,0.000000}%
\pgfsetstrokecolor{textcolor}%
\pgfsetfillcolor{textcolor}%
\pgftext[x=0.412073in, y=1.753894in, left, base]{\color{textcolor}\sffamily\fontsize{10.000000}{12.000000}\selectfont \(\displaystyle {7500}\)}%
\end{pgfscope}%
\begin{pgfscope}%
\pgfsetbuttcap%
\pgfsetroundjoin%
\definecolor{currentfill}{rgb}{0.000000,0.000000,0.000000}%
\pgfsetfillcolor{currentfill}%
\pgfsetlinewidth{0.803000pt}%
\definecolor{currentstroke}{rgb}{0.000000,0.000000,0.000000}%
\pgfsetstrokecolor{currentstroke}%
\pgfsetdash{}{0pt}%
\pgfsys@defobject{currentmarker}{\pgfqpoint{-0.048611in}{0.000000in}}{\pgfqpoint{0.000000in}{0.000000in}}{%
\pgfpathmoveto{\pgfqpoint{0.000000in}{0.000000in}}%
\pgfpathlineto{\pgfqpoint{-0.048611in}{0.000000in}}%
\pgfusepath{stroke,fill}%
}%
\begin{pgfscope}%
\pgfsys@transformshift{0.787074in}{2.172852in}%
\pgfsys@useobject{currentmarker}{}%
\end{pgfscope}%
\end{pgfscope}%
\begin{pgfscope}%
\definecolor{textcolor}{rgb}{0.000000,0.000000,0.000000}%
\pgfsetstrokecolor{textcolor}%
\pgfsetfillcolor{textcolor}%
\pgftext[x=0.342628in, y=2.124658in, left, base]{\color{textcolor}\sffamily\fontsize{10.000000}{12.000000}\selectfont \(\displaystyle {10000}\)}%
\end{pgfscope}%
\begin{pgfscope}%
\pgfsetbuttcap%
\pgfsetroundjoin%
\definecolor{currentfill}{rgb}{0.000000,0.000000,0.000000}%
\pgfsetfillcolor{currentfill}%
\pgfsetlinewidth{0.803000pt}%
\definecolor{currentstroke}{rgb}{0.000000,0.000000,0.000000}%
\pgfsetstrokecolor{currentstroke}%
\pgfsetdash{}{0pt}%
\pgfsys@defobject{currentmarker}{\pgfqpoint{-0.048611in}{0.000000in}}{\pgfqpoint{0.000000in}{0.000000in}}{%
\pgfpathmoveto{\pgfqpoint{0.000000in}{0.000000in}}%
\pgfpathlineto{\pgfqpoint{-0.048611in}{0.000000in}}%
\pgfusepath{stroke,fill}%
}%
\begin{pgfscope}%
\pgfsys@transformshift{0.787074in}{2.543617in}%
\pgfsys@useobject{currentmarker}{}%
\end{pgfscope}%
\end{pgfscope}%
\begin{pgfscope}%
\definecolor{textcolor}{rgb}{0.000000,0.000000,0.000000}%
\pgfsetstrokecolor{textcolor}%
\pgfsetfillcolor{textcolor}%
\pgftext[x=0.342628in, y=2.495422in, left, base]{\color{textcolor}\sffamily\fontsize{10.000000}{12.000000}\selectfont \(\displaystyle {12500}\)}%
\end{pgfscope}%
\begin{pgfscope}%
\pgfsetbuttcap%
\pgfsetroundjoin%
\definecolor{currentfill}{rgb}{0.000000,0.000000,0.000000}%
\pgfsetfillcolor{currentfill}%
\pgfsetlinewidth{0.803000pt}%
\definecolor{currentstroke}{rgb}{0.000000,0.000000,0.000000}%
\pgfsetstrokecolor{currentstroke}%
\pgfsetdash{}{0pt}%
\pgfsys@defobject{currentmarker}{\pgfqpoint{-0.048611in}{0.000000in}}{\pgfqpoint{0.000000in}{0.000000in}}{%
\pgfpathmoveto{\pgfqpoint{0.000000in}{0.000000in}}%
\pgfpathlineto{\pgfqpoint{-0.048611in}{0.000000in}}%
\pgfusepath{stroke,fill}%
}%
\begin{pgfscope}%
\pgfsys@transformshift{0.787074in}{2.914381in}%
\pgfsys@useobject{currentmarker}{}%
\end{pgfscope}%
\end{pgfscope}%
\begin{pgfscope}%
\definecolor{textcolor}{rgb}{0.000000,0.000000,0.000000}%
\pgfsetstrokecolor{textcolor}%
\pgfsetfillcolor{textcolor}%
\pgftext[x=0.342628in, y=2.866187in, left, base]{\color{textcolor}\sffamily\fontsize{10.000000}{12.000000}\selectfont \(\displaystyle {15000}\)}%
\end{pgfscope}%
\begin{pgfscope}%
\pgfsetbuttcap%
\pgfsetroundjoin%
\definecolor{currentfill}{rgb}{0.000000,0.000000,0.000000}%
\pgfsetfillcolor{currentfill}%
\pgfsetlinewidth{0.803000pt}%
\definecolor{currentstroke}{rgb}{0.000000,0.000000,0.000000}%
\pgfsetstrokecolor{currentstroke}%
\pgfsetdash{}{0pt}%
\pgfsys@defobject{currentmarker}{\pgfqpoint{-0.048611in}{0.000000in}}{\pgfqpoint{0.000000in}{0.000000in}}{%
\pgfpathmoveto{\pgfqpoint{0.000000in}{0.000000in}}%
\pgfpathlineto{\pgfqpoint{-0.048611in}{0.000000in}}%
\pgfusepath{stroke,fill}%
}%
\begin{pgfscope}%
\pgfsys@transformshift{0.787074in}{3.285145in}%
\pgfsys@useobject{currentmarker}{}%
\end{pgfscope}%
\end{pgfscope}%
\begin{pgfscope}%
\definecolor{textcolor}{rgb}{0.000000,0.000000,0.000000}%
\pgfsetstrokecolor{textcolor}%
\pgfsetfillcolor{textcolor}%
\pgftext[x=0.342628in, y=3.236951in, left, base]{\color{textcolor}\sffamily\fontsize{10.000000}{12.000000}\selectfont \(\displaystyle {17500}\)}%
\end{pgfscope}%
\begin{pgfscope}%
\definecolor{textcolor}{rgb}{0.000000,0.000000,0.000000}%
\pgfsetstrokecolor{textcolor}%
\pgfsetfillcolor{textcolor}%
\pgftext[x=0.287073in,y=2.100064in,,bottom,rotate=90.000000]{\color{textcolor}\sffamily\fontsize{10.000000}{12.000000}\selectfont Maximum Memory Usage (MB)}%
\end{pgfscope}%
\begin{pgfscope}%
\pgfsetrectcap%
\pgfsetmiterjoin%
\pgfsetlinewidth{0.803000pt}%
\definecolor{currentstroke}{rgb}{0.000000,0.000000,0.000000}%
\pgfsetstrokecolor{currentstroke}%
\pgfsetdash{}{0pt}%
\pgfpathmoveto{\pgfqpoint{0.787074in}{0.548769in}}%
\pgfpathlineto{\pgfqpoint{0.787074in}{3.651359in}}%
\pgfusepath{stroke}%
\end{pgfscope}%
\begin{pgfscope}%
\pgfsetrectcap%
\pgfsetmiterjoin%
\pgfsetlinewidth{0.803000pt}%
\definecolor{currentstroke}{rgb}{0.000000,0.000000,0.000000}%
\pgfsetstrokecolor{currentstroke}%
\pgfsetdash{}{0pt}%
\pgfpathmoveto{\pgfqpoint{5.850000in}{0.548769in}}%
\pgfpathlineto{\pgfqpoint{5.850000in}{3.651359in}}%
\pgfusepath{stroke}%
\end{pgfscope}%
\begin{pgfscope}%
\pgfsetrectcap%
\pgfsetmiterjoin%
\pgfsetlinewidth{0.803000pt}%
\definecolor{currentstroke}{rgb}{0.000000,0.000000,0.000000}%
\pgfsetstrokecolor{currentstroke}%
\pgfsetdash{}{0pt}%
\pgfpathmoveto{\pgfqpoint{0.787074in}{0.548769in}}%
\pgfpathlineto{\pgfqpoint{5.850000in}{0.548769in}}%
\pgfusepath{stroke}%
\end{pgfscope}%
\begin{pgfscope}%
\pgfsetrectcap%
\pgfsetmiterjoin%
\pgfsetlinewidth{0.803000pt}%
\definecolor{currentstroke}{rgb}{0.000000,0.000000,0.000000}%
\pgfsetstrokecolor{currentstroke}%
\pgfsetdash{}{0pt}%
\pgfpathmoveto{\pgfqpoint{0.787074in}{3.651359in}}%
\pgfpathlineto{\pgfqpoint{5.850000in}{3.651359in}}%
\pgfusepath{stroke}%
\end{pgfscope}%
\begin{pgfscope}%
\definecolor{textcolor}{rgb}{0.000000,0.000000,0.000000}%
\pgfsetstrokecolor{textcolor}%
\pgfsetfillcolor{textcolor}%
\pgftext[x=3.318537in,y=3.734692in,,base]{\color{textcolor}\sffamily\fontsize{12.000000}{14.400000}\selectfont Backwards}%
\end{pgfscope}%
\begin{pgfscope}%
\pgfsetbuttcap%
\pgfsetmiterjoin%
\definecolor{currentfill}{rgb}{1.000000,1.000000,1.000000}%
\pgfsetfillcolor{currentfill}%
\pgfsetfillopacity{0.800000}%
\pgfsetlinewidth{1.003750pt}%
\definecolor{currentstroke}{rgb}{0.800000,0.800000,0.800000}%
\pgfsetstrokecolor{currentstroke}%
\pgfsetstrokeopacity{0.800000}%
\pgfsetdash{}{0pt}%
\pgfpathmoveto{\pgfqpoint{0.884296in}{2.957886in}}%
\pgfpathlineto{\pgfqpoint{2.336657in}{2.957886in}}%
\pgfpathquadraticcurveto{\pgfqpoint{2.364435in}{2.957886in}}{\pgfqpoint{2.364435in}{2.985664in}}%
\pgfpathlineto{\pgfqpoint{2.364435in}{3.554136in}}%
\pgfpathquadraticcurveto{\pgfqpoint{2.364435in}{3.581914in}}{\pgfqpoint{2.336657in}{3.581914in}}%
\pgfpathlineto{\pgfqpoint{0.884296in}{3.581914in}}%
\pgfpathquadraticcurveto{\pgfqpoint{0.856518in}{3.581914in}}{\pgfqpoint{0.856518in}{3.554136in}}%
\pgfpathlineto{\pgfqpoint{0.856518in}{2.985664in}}%
\pgfpathquadraticcurveto{\pgfqpoint{0.856518in}{2.957886in}}{\pgfqpoint{0.884296in}{2.957886in}}%
\pgfpathclose%
\pgfusepath{stroke,fill}%
\end{pgfscope}%
\begin{pgfscope}%
\pgfsetbuttcap%
\pgfsetroundjoin%
\definecolor{currentfill}{rgb}{0.121569,0.466667,0.705882}%
\pgfsetfillcolor{currentfill}%
\pgfsetlinewidth{1.003750pt}%
\definecolor{currentstroke}{rgb}{0.121569,0.466667,0.705882}%
\pgfsetstrokecolor{currentstroke}%
\pgfsetdash{}{0pt}%
\pgfsys@defobject{currentmarker}{\pgfqpoint{-0.034722in}{-0.034722in}}{\pgfqpoint{0.034722in}{0.034722in}}{%
\pgfpathmoveto{\pgfqpoint{0.000000in}{-0.034722in}}%
\pgfpathcurveto{\pgfqpoint{0.009208in}{-0.034722in}}{\pgfqpoint{0.018041in}{-0.031064in}}{\pgfqpoint{0.024552in}{-0.024552in}}%
\pgfpathcurveto{\pgfqpoint{0.031064in}{-0.018041in}}{\pgfqpoint{0.034722in}{-0.009208in}}{\pgfqpoint{0.034722in}{0.000000in}}%
\pgfpathcurveto{\pgfqpoint{0.034722in}{0.009208in}}{\pgfqpoint{0.031064in}{0.018041in}}{\pgfqpoint{0.024552in}{0.024552in}}%
\pgfpathcurveto{\pgfqpoint{0.018041in}{0.031064in}}{\pgfqpoint{0.009208in}{0.034722in}}{\pgfqpoint{0.000000in}{0.034722in}}%
\pgfpathcurveto{\pgfqpoint{-0.009208in}{0.034722in}}{\pgfqpoint{-0.018041in}{0.031064in}}{\pgfqpoint{-0.024552in}{0.024552in}}%
\pgfpathcurveto{\pgfqpoint{-0.031064in}{0.018041in}}{\pgfqpoint{-0.034722in}{0.009208in}}{\pgfqpoint{-0.034722in}{0.000000in}}%
\pgfpathcurveto{\pgfqpoint{-0.034722in}{-0.009208in}}{\pgfqpoint{-0.031064in}{-0.018041in}}{\pgfqpoint{-0.024552in}{-0.024552in}}%
\pgfpathcurveto{\pgfqpoint{-0.018041in}{-0.031064in}}{\pgfqpoint{-0.009208in}{-0.034722in}}{\pgfqpoint{0.000000in}{-0.034722in}}%
\pgfpathclose%
\pgfusepath{stroke,fill}%
}%
\begin{pgfscope}%
\pgfsys@transformshift{1.050963in}{3.477748in}%
\pgfsys@useobject{currentmarker}{}%
\end{pgfscope}%
\end{pgfscope}%
\begin{pgfscope}%
\definecolor{textcolor}{rgb}{0.000000,0.000000,0.000000}%
\pgfsetstrokecolor{textcolor}%
\pgfsetfillcolor{textcolor}%
\pgftext[x=1.300963in,y=3.429136in,left,base]{\color{textcolor}\sffamily\fontsize{10.000000}{12.000000}\selectfont No Timeout}%
\end{pgfscope}%
\begin{pgfscope}%
\pgfsetbuttcap%
\pgfsetroundjoin%
\definecolor{currentfill}{rgb}{1.000000,0.498039,0.054902}%
\pgfsetfillcolor{currentfill}%
\pgfsetlinewidth{1.003750pt}%
\definecolor{currentstroke}{rgb}{1.000000,0.498039,0.054902}%
\pgfsetstrokecolor{currentstroke}%
\pgfsetdash{}{0pt}%
\pgfsys@defobject{currentmarker}{\pgfqpoint{-0.034722in}{-0.034722in}}{\pgfqpoint{0.034722in}{0.034722in}}{%
\pgfpathmoveto{\pgfqpoint{0.000000in}{-0.034722in}}%
\pgfpathcurveto{\pgfqpoint{0.009208in}{-0.034722in}}{\pgfqpoint{0.018041in}{-0.031064in}}{\pgfqpoint{0.024552in}{-0.024552in}}%
\pgfpathcurveto{\pgfqpoint{0.031064in}{-0.018041in}}{\pgfqpoint{0.034722in}{-0.009208in}}{\pgfqpoint{0.034722in}{0.000000in}}%
\pgfpathcurveto{\pgfqpoint{0.034722in}{0.009208in}}{\pgfqpoint{0.031064in}{0.018041in}}{\pgfqpoint{0.024552in}{0.024552in}}%
\pgfpathcurveto{\pgfqpoint{0.018041in}{0.031064in}}{\pgfqpoint{0.009208in}{0.034722in}}{\pgfqpoint{0.000000in}{0.034722in}}%
\pgfpathcurveto{\pgfqpoint{-0.009208in}{0.034722in}}{\pgfqpoint{-0.018041in}{0.031064in}}{\pgfqpoint{-0.024552in}{0.024552in}}%
\pgfpathcurveto{\pgfqpoint{-0.031064in}{0.018041in}}{\pgfqpoint{-0.034722in}{0.009208in}}{\pgfqpoint{-0.034722in}{0.000000in}}%
\pgfpathcurveto{\pgfqpoint{-0.034722in}{-0.009208in}}{\pgfqpoint{-0.031064in}{-0.018041in}}{\pgfqpoint{-0.024552in}{-0.024552in}}%
\pgfpathcurveto{\pgfqpoint{-0.018041in}{-0.031064in}}{\pgfqpoint{-0.009208in}{-0.034722in}}{\pgfqpoint{0.000000in}{-0.034722in}}%
\pgfpathclose%
\pgfusepath{stroke,fill}%
}%
\begin{pgfscope}%
\pgfsys@transformshift{1.050963in}{3.284136in}%
\pgfsys@useobject{currentmarker}{}%
\end{pgfscope}%
\end{pgfscope}%
\begin{pgfscope}%
\definecolor{textcolor}{rgb}{0.000000,0.000000,0.000000}%
\pgfsetstrokecolor{textcolor}%
\pgfsetfillcolor{textcolor}%
\pgftext[x=1.300963in,y=3.235525in,left,base]{\color{textcolor}\sffamily\fontsize{10.000000}{12.000000}\selectfont Time Timeout}%
\end{pgfscope}%
\begin{pgfscope}%
\pgfsetbuttcap%
\pgfsetroundjoin%
\definecolor{currentfill}{rgb}{0.839216,0.152941,0.156863}%
\pgfsetfillcolor{currentfill}%
\pgfsetlinewidth{1.003750pt}%
\definecolor{currentstroke}{rgb}{0.839216,0.152941,0.156863}%
\pgfsetstrokecolor{currentstroke}%
\pgfsetdash{}{0pt}%
\pgfsys@defobject{currentmarker}{\pgfqpoint{-0.034722in}{-0.034722in}}{\pgfqpoint{0.034722in}{0.034722in}}{%
\pgfpathmoveto{\pgfqpoint{0.000000in}{-0.034722in}}%
\pgfpathcurveto{\pgfqpoint{0.009208in}{-0.034722in}}{\pgfqpoint{0.018041in}{-0.031064in}}{\pgfqpoint{0.024552in}{-0.024552in}}%
\pgfpathcurveto{\pgfqpoint{0.031064in}{-0.018041in}}{\pgfqpoint{0.034722in}{-0.009208in}}{\pgfqpoint{0.034722in}{0.000000in}}%
\pgfpathcurveto{\pgfqpoint{0.034722in}{0.009208in}}{\pgfqpoint{0.031064in}{0.018041in}}{\pgfqpoint{0.024552in}{0.024552in}}%
\pgfpathcurveto{\pgfqpoint{0.018041in}{0.031064in}}{\pgfqpoint{0.009208in}{0.034722in}}{\pgfqpoint{0.000000in}{0.034722in}}%
\pgfpathcurveto{\pgfqpoint{-0.009208in}{0.034722in}}{\pgfqpoint{-0.018041in}{0.031064in}}{\pgfqpoint{-0.024552in}{0.024552in}}%
\pgfpathcurveto{\pgfqpoint{-0.031064in}{0.018041in}}{\pgfqpoint{-0.034722in}{0.009208in}}{\pgfqpoint{-0.034722in}{0.000000in}}%
\pgfpathcurveto{\pgfqpoint{-0.034722in}{-0.009208in}}{\pgfqpoint{-0.031064in}{-0.018041in}}{\pgfqpoint{-0.024552in}{-0.024552in}}%
\pgfpathcurveto{\pgfqpoint{-0.018041in}{-0.031064in}}{\pgfqpoint{-0.009208in}{-0.034722in}}{\pgfqpoint{0.000000in}{-0.034722in}}%
\pgfpathclose%
\pgfusepath{stroke,fill}%
}%
\begin{pgfscope}%
\pgfsys@transformshift{1.050963in}{3.090525in}%
\pgfsys@useobject{currentmarker}{}%
\end{pgfscope}%
\end{pgfscope}%
\begin{pgfscope}%
\definecolor{textcolor}{rgb}{0.000000,0.000000,0.000000}%
\pgfsetstrokecolor{textcolor}%
\pgfsetfillcolor{textcolor}%
\pgftext[x=1.300963in,y=3.041914in,left,base]{\color{textcolor}\sffamily\fontsize{10.000000}{12.000000}\selectfont Memory Timeout}%
\end{pgfscope}%
\end{pgfpicture}%
\makeatother%
\endgroup%

                }
            \end{subfigure}
        \end{subfigure}
        \caption{Maximum Memory Consumption in Comparison to the Edge Count}
        \label{f:maxmemedges}
    \end{figure}

    Also beneficial for the real-world usage of \textsc{FlowDroid} would be to estimate the memory consumption to utilize the available resources efficiently.
    In \autoref{f:maxmemtoss}, we contrast the memory consumption with the number of sources, sinks and the ratio of both.
    \autoref{f:maxmemtocodesize} shows the memory consumption in contrast to the statement, method and class count.
    The arrangement and legend are the same as in the time evaluation.
    Unlike in the time evaluation, there is only one cluster of dots: those terminating nearly instantaneous. 
    Otherwise, the dots seem to be randomly distributed. 
    All graphs indicate no correlation.

    \begin{figure}[tbp]
        \centering
        \begin{subfigure}[b]{\textwidth}
            \centering
            \begin{subfigure}[]{0.45\textwidth}
                \centering
                \resizebox{\columnwidth}{!}{
                    %% Creator: Matplotlib, PGF backend
%%
%% To include the figure in your LaTeX document, write
%%   \input{<filename>.pgf}
%%
%% Make sure the required packages are loaded in your preamble
%%   \usepackage{pgf}
%%
%% and, on pdftex
%%   \usepackage[utf8]{inputenc}\DeclareUnicodeCharacter{2212}{-}
%%
%% or, on luatex and xetex
%%   \usepackage{unicode-math}
%%
%% Figures using additional raster images can only be included by \input if
%% they are in the same directory as the main LaTeX file. For loading figures
%% from other directories you can use the `import` package
%%   \usepackage{import}
%%
%% and then include the figures with
%%   \import{<path to file>}{<filename>.pgf}
%%
%% Matplotlib used the following preamble
%%   \usepackage{amsmath}
%%   \usepackage{fontspec}
%%
\begingroup%
\makeatletter%
\begin{pgfpicture}%
\pgfpathrectangle{\pgfpointorigin}{\pgfqpoint{6.000000in}{4.000000in}}%
\pgfusepath{use as bounding box, clip}%
\begin{pgfscope}%
\pgfsetbuttcap%
\pgfsetmiterjoin%
\definecolor{currentfill}{rgb}{1.000000,1.000000,1.000000}%
\pgfsetfillcolor{currentfill}%
\pgfsetlinewidth{0.000000pt}%
\definecolor{currentstroke}{rgb}{1.000000,1.000000,1.000000}%
\pgfsetstrokecolor{currentstroke}%
\pgfsetdash{}{0pt}%
\pgfpathmoveto{\pgfqpoint{0.000000in}{0.000000in}}%
\pgfpathlineto{\pgfqpoint{6.000000in}{0.000000in}}%
\pgfpathlineto{\pgfqpoint{6.000000in}{4.000000in}}%
\pgfpathlineto{\pgfqpoint{0.000000in}{4.000000in}}%
\pgfpathclose%
\pgfusepath{fill}%
\end{pgfscope}%
\begin{pgfscope}%
\pgfsetbuttcap%
\pgfsetmiterjoin%
\definecolor{currentfill}{rgb}{1.000000,1.000000,1.000000}%
\pgfsetfillcolor{currentfill}%
\pgfsetlinewidth{0.000000pt}%
\definecolor{currentstroke}{rgb}{0.000000,0.000000,0.000000}%
\pgfsetstrokecolor{currentstroke}%
\pgfsetstrokeopacity{0.000000}%
\pgfsetdash{}{0pt}%
\pgfpathmoveto{\pgfqpoint{0.787074in}{0.548769in}}%
\pgfpathlineto{\pgfqpoint{5.850000in}{0.548769in}}%
\pgfpathlineto{\pgfqpoint{5.850000in}{3.651359in}}%
\pgfpathlineto{\pgfqpoint{0.787074in}{3.651359in}}%
\pgfpathclose%
\pgfusepath{fill}%
\end{pgfscope}%
\begin{pgfscope}%
\pgfpathrectangle{\pgfqpoint{0.787074in}{0.548769in}}{\pgfqpoint{5.062926in}{3.102590in}}%
\pgfusepath{clip}%
\pgfsetbuttcap%
\pgfsetroundjoin%
\definecolor{currentfill}{rgb}{0.121569,0.466667,0.705882}%
\pgfsetfillcolor{currentfill}%
\pgfsetlinewidth{1.003750pt}%
\definecolor{currentstroke}{rgb}{0.121569,0.466667,0.705882}%
\pgfsetstrokecolor{currentstroke}%
\pgfsetdash{}{0pt}%
\pgfpathmoveto{\pgfqpoint{1.411721in}{0.648193in}}%
\pgfpathcurveto{\pgfqpoint{1.422771in}{0.648193in}}{\pgfqpoint{1.433370in}{0.652583in}}{\pgfqpoint{1.441183in}{0.660397in}}%
\pgfpathcurveto{\pgfqpoint{1.448997in}{0.668210in}}{\pgfqpoint{1.453387in}{0.678809in}}{\pgfqpoint{1.453387in}{0.689859in}}%
\pgfpathcurveto{\pgfqpoint{1.453387in}{0.700910in}}{\pgfqpoint{1.448997in}{0.711509in}}{\pgfqpoint{1.441183in}{0.719322in}}%
\pgfpathcurveto{\pgfqpoint{1.433370in}{0.727136in}}{\pgfqpoint{1.422771in}{0.731526in}}{\pgfqpoint{1.411721in}{0.731526in}}%
\pgfpathcurveto{\pgfqpoint{1.400670in}{0.731526in}}{\pgfqpoint{1.390071in}{0.727136in}}{\pgfqpoint{1.382258in}{0.719322in}}%
\pgfpathcurveto{\pgfqpoint{1.374444in}{0.711509in}}{\pgfqpoint{1.370054in}{0.700910in}}{\pgfqpoint{1.370054in}{0.689859in}}%
\pgfpathcurveto{\pgfqpoint{1.370054in}{0.678809in}}{\pgfqpoint{1.374444in}{0.668210in}}{\pgfqpoint{1.382258in}{0.660397in}}%
\pgfpathcurveto{\pgfqpoint{1.390071in}{0.652583in}}{\pgfqpoint{1.400670in}{0.648193in}}{\pgfqpoint{1.411721in}{0.648193in}}%
\pgfpathclose%
\pgfusepath{stroke,fill}%
\end{pgfscope}%
\begin{pgfscope}%
\pgfpathrectangle{\pgfqpoint{0.787074in}{0.548769in}}{\pgfqpoint{5.062926in}{3.102590in}}%
\pgfusepath{clip}%
\pgfsetbuttcap%
\pgfsetroundjoin%
\definecolor{currentfill}{rgb}{1.000000,0.498039,0.054902}%
\pgfsetfillcolor{currentfill}%
\pgfsetlinewidth{1.003750pt}%
\definecolor{currentstroke}{rgb}{1.000000,0.498039,0.054902}%
\pgfsetstrokecolor{currentstroke}%
\pgfsetdash{}{0pt}%
\pgfpathmoveto{\pgfqpoint{2.858271in}{1.859307in}}%
\pgfpathcurveto{\pgfqpoint{2.869321in}{1.859307in}}{\pgfqpoint{2.879920in}{1.863697in}}{\pgfqpoint{2.887734in}{1.871511in}}%
\pgfpathcurveto{\pgfqpoint{2.895547in}{1.879324in}}{\pgfqpoint{2.899938in}{1.889923in}}{\pgfqpoint{2.899938in}{1.900973in}}%
\pgfpathcurveto{\pgfqpoint{2.899938in}{1.912024in}}{\pgfqpoint{2.895547in}{1.922623in}}{\pgfqpoint{2.887734in}{1.930436in}}%
\pgfpathcurveto{\pgfqpoint{2.879920in}{1.938250in}}{\pgfqpoint{2.869321in}{1.942640in}}{\pgfqpoint{2.858271in}{1.942640in}}%
\pgfpathcurveto{\pgfqpoint{2.847221in}{1.942640in}}{\pgfqpoint{2.836622in}{1.938250in}}{\pgfqpoint{2.828808in}{1.930436in}}%
\pgfpathcurveto{\pgfqpoint{2.820995in}{1.922623in}}{\pgfqpoint{2.816604in}{1.912024in}}{\pgfqpoint{2.816604in}{1.900973in}}%
\pgfpathcurveto{\pgfqpoint{2.816604in}{1.889923in}}{\pgfqpoint{2.820995in}{1.879324in}}{\pgfqpoint{2.828808in}{1.871511in}}%
\pgfpathcurveto{\pgfqpoint{2.836622in}{1.863697in}}{\pgfqpoint{2.847221in}{1.859307in}}{\pgfqpoint{2.858271in}{1.859307in}}%
\pgfpathclose%
\pgfusepath{stroke,fill}%
\end{pgfscope}%
\begin{pgfscope}%
\pgfpathrectangle{\pgfqpoint{0.787074in}{0.548769in}}{\pgfqpoint{5.062926in}{3.102590in}}%
\pgfusepath{clip}%
\pgfsetbuttcap%
\pgfsetroundjoin%
\definecolor{currentfill}{rgb}{1.000000,0.498039,0.054902}%
\pgfsetfillcolor{currentfill}%
\pgfsetlinewidth{1.003750pt}%
\definecolor{currentstroke}{rgb}{1.000000,0.498039,0.054902}%
\pgfsetstrokecolor{currentstroke}%
\pgfsetdash{}{0pt}%
\pgfpathmoveto{\pgfqpoint{1.608977in}{2.662812in}}%
\pgfpathcurveto{\pgfqpoint{1.620028in}{2.662812in}}{\pgfqpoint{1.630627in}{2.667202in}}{\pgfqpoint{1.638440in}{2.675016in}}%
\pgfpathcurveto{\pgfqpoint{1.646254in}{2.682830in}}{\pgfqpoint{1.650644in}{2.693429in}}{\pgfqpoint{1.650644in}{2.704479in}}%
\pgfpathcurveto{\pgfqpoint{1.650644in}{2.715529in}}{\pgfqpoint{1.646254in}{2.726128in}}{\pgfqpoint{1.638440in}{2.733942in}}%
\pgfpathcurveto{\pgfqpoint{1.630627in}{2.741755in}}{\pgfqpoint{1.620028in}{2.746145in}}{\pgfqpoint{1.608977in}{2.746145in}}%
\pgfpathcurveto{\pgfqpoint{1.597927in}{2.746145in}}{\pgfqpoint{1.587328in}{2.741755in}}{\pgfqpoint{1.579515in}{2.733942in}}%
\pgfpathcurveto{\pgfqpoint{1.571701in}{2.726128in}}{\pgfqpoint{1.567311in}{2.715529in}}{\pgfqpoint{1.567311in}{2.704479in}}%
\pgfpathcurveto{\pgfqpoint{1.567311in}{2.693429in}}{\pgfqpoint{1.571701in}{2.682830in}}{\pgfqpoint{1.579515in}{2.675016in}}%
\pgfpathcurveto{\pgfqpoint{1.587328in}{2.667202in}}{\pgfqpoint{1.597927in}{2.662812in}}{\pgfqpoint{1.608977in}{2.662812in}}%
\pgfpathclose%
\pgfusepath{stroke,fill}%
\end{pgfscope}%
\begin{pgfscope}%
\pgfpathrectangle{\pgfqpoint{0.787074in}{0.548769in}}{\pgfqpoint{5.062926in}{3.102590in}}%
\pgfusepath{clip}%
\pgfsetbuttcap%
\pgfsetroundjoin%
\definecolor{currentfill}{rgb}{1.000000,0.498039,0.054902}%
\pgfsetfillcolor{currentfill}%
\pgfsetlinewidth{1.003750pt}%
\definecolor{currentstroke}{rgb}{1.000000,0.498039,0.054902}%
\pgfsetstrokecolor{currentstroke}%
\pgfsetdash{}{0pt}%
\pgfpathmoveto{\pgfqpoint{2.726766in}{2.336454in}}%
\pgfpathcurveto{\pgfqpoint{2.737816in}{2.336454in}}{\pgfqpoint{2.748415in}{2.340845in}}{\pgfqpoint{2.756229in}{2.348658in}}%
\pgfpathcurveto{\pgfqpoint{2.764043in}{2.356472in}}{\pgfqpoint{2.768433in}{2.367071in}}{\pgfqpoint{2.768433in}{2.378121in}}%
\pgfpathcurveto{\pgfqpoint{2.768433in}{2.389171in}}{\pgfqpoint{2.764043in}{2.399770in}}{\pgfqpoint{2.756229in}{2.407584in}}%
\pgfpathcurveto{\pgfqpoint{2.748415in}{2.415397in}}{\pgfqpoint{2.737816in}{2.419788in}}{\pgfqpoint{2.726766in}{2.419788in}}%
\pgfpathcurveto{\pgfqpoint{2.715716in}{2.419788in}}{\pgfqpoint{2.705117in}{2.415397in}}{\pgfqpoint{2.697304in}{2.407584in}}%
\pgfpathcurveto{\pgfqpoint{2.689490in}{2.399770in}}{\pgfqpoint{2.685100in}{2.389171in}}{\pgfqpoint{2.685100in}{2.378121in}}%
\pgfpathcurveto{\pgfqpoint{2.685100in}{2.367071in}}{\pgfqpoint{2.689490in}{2.356472in}}{\pgfqpoint{2.697304in}{2.348658in}}%
\pgfpathcurveto{\pgfqpoint{2.705117in}{2.340845in}}{\pgfqpoint{2.715716in}{2.336454in}}{\pgfqpoint{2.726766in}{2.336454in}}%
\pgfpathclose%
\pgfusepath{stroke,fill}%
\end{pgfscope}%
\begin{pgfscope}%
\pgfpathrectangle{\pgfqpoint{0.787074in}{0.548769in}}{\pgfqpoint{5.062926in}{3.102590in}}%
\pgfusepath{clip}%
\pgfsetbuttcap%
\pgfsetroundjoin%
\definecolor{currentfill}{rgb}{1.000000,0.498039,0.054902}%
\pgfsetfillcolor{currentfill}%
\pgfsetlinewidth{1.003750pt}%
\definecolor{currentstroke}{rgb}{1.000000,0.498039,0.054902}%
\pgfsetstrokecolor{currentstroke}%
\pgfsetdash{}{0pt}%
\pgfpathmoveto{\pgfqpoint{2.266500in}{2.322912in}}%
\pgfpathcurveto{\pgfqpoint{2.277550in}{2.322912in}}{\pgfqpoint{2.288149in}{2.327302in}}{\pgfqpoint{2.295963in}{2.335116in}}%
\pgfpathcurveto{\pgfqpoint{2.303777in}{2.342929in}}{\pgfqpoint{2.308167in}{2.353528in}}{\pgfqpoint{2.308167in}{2.364578in}}%
\pgfpathcurveto{\pgfqpoint{2.308167in}{2.375629in}}{\pgfqpoint{2.303777in}{2.386228in}}{\pgfqpoint{2.295963in}{2.394041in}}%
\pgfpathcurveto{\pgfqpoint{2.288149in}{2.401855in}}{\pgfqpoint{2.277550in}{2.406245in}}{\pgfqpoint{2.266500in}{2.406245in}}%
\pgfpathcurveto{\pgfqpoint{2.255450in}{2.406245in}}{\pgfqpoint{2.244851in}{2.401855in}}{\pgfqpoint{2.237038in}{2.394041in}}%
\pgfpathcurveto{\pgfqpoint{2.229224in}{2.386228in}}{\pgfqpoint{2.224834in}{2.375629in}}{\pgfqpoint{2.224834in}{2.364578in}}%
\pgfpathcurveto{\pgfqpoint{2.224834in}{2.353528in}}{\pgfqpoint{2.229224in}{2.342929in}}{\pgfqpoint{2.237038in}{2.335116in}}%
\pgfpathcurveto{\pgfqpoint{2.244851in}{2.327302in}}{\pgfqpoint{2.255450in}{2.322912in}}{\pgfqpoint{2.266500in}{2.322912in}}%
\pgfpathclose%
\pgfusepath{stroke,fill}%
\end{pgfscope}%
\begin{pgfscope}%
\pgfpathrectangle{\pgfqpoint{0.787074in}{0.548769in}}{\pgfqpoint{5.062926in}{3.102590in}}%
\pgfusepath{clip}%
\pgfsetbuttcap%
\pgfsetroundjoin%
\definecolor{currentfill}{rgb}{1.000000,0.498039,0.054902}%
\pgfsetfillcolor{currentfill}%
\pgfsetlinewidth{1.003750pt}%
\definecolor{currentstroke}{rgb}{1.000000,0.498039,0.054902}%
\pgfsetstrokecolor{currentstroke}%
\pgfsetdash{}{0pt}%
\pgfpathmoveto{\pgfqpoint{2.529509in}{1.553412in}}%
\pgfpathcurveto{\pgfqpoint{2.540560in}{1.553412in}}{\pgfqpoint{2.551159in}{1.557802in}}{\pgfqpoint{2.558972in}{1.565616in}}%
\pgfpathcurveto{\pgfqpoint{2.566786in}{1.573429in}}{\pgfqpoint{2.571176in}{1.584029in}}{\pgfqpoint{2.571176in}{1.595079in}}%
\pgfpathcurveto{\pgfqpoint{2.571176in}{1.606129in}}{\pgfqpoint{2.566786in}{1.616728in}}{\pgfqpoint{2.558972in}{1.624541in}}%
\pgfpathcurveto{\pgfqpoint{2.551159in}{1.632355in}}{\pgfqpoint{2.540560in}{1.636745in}}{\pgfqpoint{2.529509in}{1.636745in}}%
\pgfpathcurveto{\pgfqpoint{2.518459in}{1.636745in}}{\pgfqpoint{2.507860in}{1.632355in}}{\pgfqpoint{2.500047in}{1.624541in}}%
\pgfpathcurveto{\pgfqpoint{2.492233in}{1.616728in}}{\pgfqpoint{2.487843in}{1.606129in}}{\pgfqpoint{2.487843in}{1.595079in}}%
\pgfpathcurveto{\pgfqpoint{2.487843in}{1.584029in}}{\pgfqpoint{2.492233in}{1.573429in}}{\pgfqpoint{2.500047in}{1.565616in}}%
\pgfpathcurveto{\pgfqpoint{2.507860in}{1.557802in}}{\pgfqpoint{2.518459in}{1.553412in}}{\pgfqpoint{2.529509in}{1.553412in}}%
\pgfpathclose%
\pgfusepath{stroke,fill}%
\end{pgfscope}%
\begin{pgfscope}%
\pgfpathrectangle{\pgfqpoint{0.787074in}{0.548769in}}{\pgfqpoint{5.062926in}{3.102590in}}%
\pgfusepath{clip}%
\pgfsetbuttcap%
\pgfsetroundjoin%
\definecolor{currentfill}{rgb}{1.000000,0.498039,0.054902}%
\pgfsetfillcolor{currentfill}%
\pgfsetlinewidth{1.003750pt}%
\definecolor{currentstroke}{rgb}{1.000000,0.498039,0.054902}%
\pgfsetstrokecolor{currentstroke}%
\pgfsetdash{}{0pt}%
\pgfpathmoveto{\pgfqpoint{2.069243in}{2.076652in}}%
\pgfpathcurveto{\pgfqpoint{2.080294in}{2.076652in}}{\pgfqpoint{2.090893in}{2.081042in}}{\pgfqpoint{2.098706in}{2.088856in}}%
\pgfpathcurveto{\pgfqpoint{2.106520in}{2.096669in}}{\pgfqpoint{2.110910in}{2.107268in}}{\pgfqpoint{2.110910in}{2.118319in}}%
\pgfpathcurveto{\pgfqpoint{2.110910in}{2.129369in}}{\pgfqpoint{2.106520in}{2.139968in}}{\pgfqpoint{2.098706in}{2.147781in}}%
\pgfpathcurveto{\pgfqpoint{2.090893in}{2.155595in}}{\pgfqpoint{2.080294in}{2.159985in}}{\pgfqpoint{2.069243in}{2.159985in}}%
\pgfpathcurveto{\pgfqpoint{2.058193in}{2.159985in}}{\pgfqpoint{2.047594in}{2.155595in}}{\pgfqpoint{2.039781in}{2.147781in}}%
\pgfpathcurveto{\pgfqpoint{2.031967in}{2.139968in}}{\pgfqpoint{2.027577in}{2.129369in}}{\pgfqpoint{2.027577in}{2.118319in}}%
\pgfpathcurveto{\pgfqpoint{2.027577in}{2.107268in}}{\pgfqpoint{2.031967in}{2.096669in}}{\pgfqpoint{2.039781in}{2.088856in}}%
\pgfpathcurveto{\pgfqpoint{2.047594in}{2.081042in}}{\pgfqpoint{2.058193in}{2.076652in}}{\pgfqpoint{2.069243in}{2.076652in}}%
\pgfpathclose%
\pgfusepath{stroke,fill}%
\end{pgfscope}%
\begin{pgfscope}%
\pgfpathrectangle{\pgfqpoint{0.787074in}{0.548769in}}{\pgfqpoint{5.062926in}{3.102590in}}%
\pgfusepath{clip}%
\pgfsetbuttcap%
\pgfsetroundjoin%
\definecolor{currentfill}{rgb}{1.000000,0.498039,0.054902}%
\pgfsetfillcolor{currentfill}%
\pgfsetlinewidth{1.003750pt}%
\definecolor{currentstroke}{rgb}{1.000000,0.498039,0.054902}%
\pgfsetstrokecolor{currentstroke}%
\pgfsetdash{}{0pt}%
\pgfpathmoveto{\pgfqpoint{1.411721in}{2.475226in}}%
\pgfpathcurveto{\pgfqpoint{1.422771in}{2.475226in}}{\pgfqpoint{1.433370in}{2.479616in}}{\pgfqpoint{1.441183in}{2.487430in}}%
\pgfpathcurveto{\pgfqpoint{1.448997in}{2.495243in}}{\pgfqpoint{1.453387in}{2.505842in}}{\pgfqpoint{1.453387in}{2.516893in}}%
\pgfpathcurveto{\pgfqpoint{1.453387in}{2.527943in}}{\pgfqpoint{1.448997in}{2.538542in}}{\pgfqpoint{1.441183in}{2.546355in}}%
\pgfpathcurveto{\pgfqpoint{1.433370in}{2.554169in}}{\pgfqpoint{1.422771in}{2.558559in}}{\pgfqpoint{1.411721in}{2.558559in}}%
\pgfpathcurveto{\pgfqpoint{1.400670in}{2.558559in}}{\pgfqpoint{1.390071in}{2.554169in}}{\pgfqpoint{1.382258in}{2.546355in}}%
\pgfpathcurveto{\pgfqpoint{1.374444in}{2.538542in}}{\pgfqpoint{1.370054in}{2.527943in}}{\pgfqpoint{1.370054in}{2.516893in}}%
\pgfpathcurveto{\pgfqpoint{1.370054in}{2.505842in}}{\pgfqpoint{1.374444in}{2.495243in}}{\pgfqpoint{1.382258in}{2.487430in}}%
\pgfpathcurveto{\pgfqpoint{1.390071in}{2.479616in}}{\pgfqpoint{1.400670in}{2.475226in}}{\pgfqpoint{1.411721in}{2.475226in}}%
\pgfpathclose%
\pgfusepath{stroke,fill}%
\end{pgfscope}%
\begin{pgfscope}%
\pgfpathrectangle{\pgfqpoint{0.787074in}{0.548769in}}{\pgfqpoint{5.062926in}{3.102590in}}%
\pgfusepath{clip}%
\pgfsetbuttcap%
\pgfsetroundjoin%
\definecolor{currentfill}{rgb}{1.000000,0.498039,0.054902}%
\pgfsetfillcolor{currentfill}%
\pgfsetlinewidth{1.003750pt}%
\definecolor{currentstroke}{rgb}{1.000000,0.498039,0.054902}%
\pgfsetstrokecolor{currentstroke}%
\pgfsetdash{}{0pt}%
\pgfpathmoveto{\pgfqpoint{1.543225in}{2.253097in}}%
\pgfpathcurveto{\pgfqpoint{1.554275in}{2.253097in}}{\pgfqpoint{1.564874in}{2.257487in}}{\pgfqpoint{1.572688in}{2.265300in}}%
\pgfpathcurveto{\pgfqpoint{1.580502in}{2.273114in}}{\pgfqpoint{1.584892in}{2.283713in}}{\pgfqpoint{1.584892in}{2.294763in}}%
\pgfpathcurveto{\pgfqpoint{1.584892in}{2.305813in}}{\pgfqpoint{1.580502in}{2.316412in}}{\pgfqpoint{1.572688in}{2.324226in}}%
\pgfpathcurveto{\pgfqpoint{1.564874in}{2.332040in}}{\pgfqpoint{1.554275in}{2.336430in}}{\pgfqpoint{1.543225in}{2.336430in}}%
\pgfpathcurveto{\pgfqpoint{1.532175in}{2.336430in}}{\pgfqpoint{1.521576in}{2.332040in}}{\pgfqpoint{1.513762in}{2.324226in}}%
\pgfpathcurveto{\pgfqpoint{1.505949in}{2.316412in}}{\pgfqpoint{1.501558in}{2.305813in}}{\pgfqpoint{1.501558in}{2.294763in}}%
\pgfpathcurveto{\pgfqpoint{1.501558in}{2.283713in}}{\pgfqpoint{1.505949in}{2.273114in}}{\pgfqpoint{1.513762in}{2.265300in}}%
\pgfpathcurveto{\pgfqpoint{1.521576in}{2.257487in}}{\pgfqpoint{1.532175in}{2.253097in}}{\pgfqpoint{1.543225in}{2.253097in}}%
\pgfpathclose%
\pgfusepath{stroke,fill}%
\end{pgfscope}%
\begin{pgfscope}%
\pgfpathrectangle{\pgfqpoint{0.787074in}{0.548769in}}{\pgfqpoint{5.062926in}{3.102590in}}%
\pgfusepath{clip}%
\pgfsetbuttcap%
\pgfsetroundjoin%
\definecolor{currentfill}{rgb}{0.121569,0.466667,0.705882}%
\pgfsetfillcolor{currentfill}%
\pgfsetlinewidth{1.003750pt}%
\definecolor{currentstroke}{rgb}{0.121569,0.466667,0.705882}%
\pgfsetstrokecolor{currentstroke}%
\pgfsetdash{}{0pt}%
\pgfpathmoveto{\pgfqpoint{1.280216in}{0.659516in}}%
\pgfpathcurveto{\pgfqpoint{1.291266in}{0.659516in}}{\pgfqpoint{1.301865in}{0.663906in}}{\pgfqpoint{1.309679in}{0.671719in}}%
\pgfpathcurveto{\pgfqpoint{1.317492in}{0.679533in}}{\pgfqpoint{1.321883in}{0.690132in}}{\pgfqpoint{1.321883in}{0.701182in}}%
\pgfpathcurveto{\pgfqpoint{1.321883in}{0.712232in}}{\pgfqpoint{1.317492in}{0.722831in}}{\pgfqpoint{1.309679in}{0.730645in}}%
\pgfpathcurveto{\pgfqpoint{1.301865in}{0.738459in}}{\pgfqpoint{1.291266in}{0.742849in}}{\pgfqpoint{1.280216in}{0.742849in}}%
\pgfpathcurveto{\pgfqpoint{1.269166in}{0.742849in}}{\pgfqpoint{1.258567in}{0.738459in}}{\pgfqpoint{1.250753in}{0.730645in}}%
\pgfpathcurveto{\pgfqpoint{1.242940in}{0.722831in}}{\pgfqpoint{1.238549in}{0.712232in}}{\pgfqpoint{1.238549in}{0.701182in}}%
\pgfpathcurveto{\pgfqpoint{1.238549in}{0.690132in}}{\pgfqpoint{1.242940in}{0.679533in}}{\pgfqpoint{1.250753in}{0.671719in}}%
\pgfpathcurveto{\pgfqpoint{1.258567in}{0.663906in}}{\pgfqpoint{1.269166in}{0.659516in}}{\pgfqpoint{1.280216in}{0.659516in}}%
\pgfpathclose%
\pgfusepath{stroke,fill}%
\end{pgfscope}%
\begin{pgfscope}%
\pgfpathrectangle{\pgfqpoint{0.787074in}{0.548769in}}{\pgfqpoint{5.062926in}{3.102590in}}%
\pgfusepath{clip}%
\pgfsetbuttcap%
\pgfsetroundjoin%
\definecolor{currentfill}{rgb}{1.000000,0.498039,0.054902}%
\pgfsetfillcolor{currentfill}%
\pgfsetlinewidth{1.003750pt}%
\definecolor{currentstroke}{rgb}{1.000000,0.498039,0.054902}%
\pgfsetstrokecolor{currentstroke}%
\pgfsetdash{}{0pt}%
\pgfpathmoveto{\pgfqpoint{1.543225in}{2.129288in}}%
\pgfpathcurveto{\pgfqpoint{1.554275in}{2.129288in}}{\pgfqpoint{1.564874in}{2.133678in}}{\pgfqpoint{1.572688in}{2.141492in}}%
\pgfpathcurveto{\pgfqpoint{1.580502in}{2.149306in}}{\pgfqpoint{1.584892in}{2.159905in}}{\pgfqpoint{1.584892in}{2.170955in}}%
\pgfpathcurveto{\pgfqpoint{1.584892in}{2.182005in}}{\pgfqpoint{1.580502in}{2.192604in}}{\pgfqpoint{1.572688in}{2.200418in}}%
\pgfpathcurveto{\pgfqpoint{1.564874in}{2.208231in}}{\pgfqpoint{1.554275in}{2.212621in}}{\pgfqpoint{1.543225in}{2.212621in}}%
\pgfpathcurveto{\pgfqpoint{1.532175in}{2.212621in}}{\pgfqpoint{1.521576in}{2.208231in}}{\pgfqpoint{1.513762in}{2.200418in}}%
\pgfpathcurveto{\pgfqpoint{1.505949in}{2.192604in}}{\pgfqpoint{1.501558in}{2.182005in}}{\pgfqpoint{1.501558in}{2.170955in}}%
\pgfpathcurveto{\pgfqpoint{1.501558in}{2.159905in}}{\pgfqpoint{1.505949in}{2.149306in}}{\pgfqpoint{1.513762in}{2.141492in}}%
\pgfpathcurveto{\pgfqpoint{1.521576in}{2.133678in}}{\pgfqpoint{1.532175in}{2.129288in}}{\pgfqpoint{1.543225in}{2.129288in}}%
\pgfpathclose%
\pgfusepath{stroke,fill}%
\end{pgfscope}%
\begin{pgfscope}%
\pgfpathrectangle{\pgfqpoint{0.787074in}{0.548769in}}{\pgfqpoint{5.062926in}{3.102590in}}%
\pgfusepath{clip}%
\pgfsetbuttcap%
\pgfsetroundjoin%
\definecolor{currentfill}{rgb}{1.000000,0.498039,0.054902}%
\pgfsetfillcolor{currentfill}%
\pgfsetlinewidth{1.003750pt}%
\definecolor{currentstroke}{rgb}{1.000000,0.498039,0.054902}%
\pgfsetstrokecolor{currentstroke}%
\pgfsetdash{}{0pt}%
\pgfpathmoveto{\pgfqpoint{3.055528in}{2.778374in}}%
\pgfpathcurveto{\pgfqpoint{3.066578in}{2.778374in}}{\pgfqpoint{3.077177in}{2.782764in}}{\pgfqpoint{3.084991in}{2.790577in}}%
\pgfpathcurveto{\pgfqpoint{3.092804in}{2.798391in}}{\pgfqpoint{3.097194in}{2.808990in}}{\pgfqpoint{3.097194in}{2.820040in}}%
\pgfpathcurveto{\pgfqpoint{3.097194in}{2.831090in}}{\pgfqpoint{3.092804in}{2.841689in}}{\pgfqpoint{3.084991in}{2.849503in}}%
\pgfpathcurveto{\pgfqpoint{3.077177in}{2.857317in}}{\pgfqpoint{3.066578in}{2.861707in}}{\pgfqpoint{3.055528in}{2.861707in}}%
\pgfpathcurveto{\pgfqpoint{3.044478in}{2.861707in}}{\pgfqpoint{3.033879in}{2.857317in}}{\pgfqpoint{3.026065in}{2.849503in}}%
\pgfpathcurveto{\pgfqpoint{3.018251in}{2.841689in}}{\pgfqpoint{3.013861in}{2.831090in}}{\pgfqpoint{3.013861in}{2.820040in}}%
\pgfpathcurveto{\pgfqpoint{3.013861in}{2.808990in}}{\pgfqpoint{3.018251in}{2.798391in}}{\pgfqpoint{3.026065in}{2.790577in}}%
\pgfpathcurveto{\pgfqpoint{3.033879in}{2.782764in}}{\pgfqpoint{3.044478in}{2.778374in}}{\pgfqpoint{3.055528in}{2.778374in}}%
\pgfpathclose%
\pgfusepath{stroke,fill}%
\end{pgfscope}%
\begin{pgfscope}%
\pgfpathrectangle{\pgfqpoint{0.787074in}{0.548769in}}{\pgfqpoint{5.062926in}{3.102590in}}%
\pgfusepath{clip}%
\pgfsetbuttcap%
\pgfsetroundjoin%
\definecolor{currentfill}{rgb}{1.000000,0.498039,0.054902}%
\pgfsetfillcolor{currentfill}%
\pgfsetlinewidth{1.003750pt}%
\definecolor{currentstroke}{rgb}{1.000000,0.498039,0.054902}%
\pgfsetstrokecolor{currentstroke}%
\pgfsetdash{}{0pt}%
\pgfpathmoveto{\pgfqpoint{1.543225in}{2.577708in}}%
\pgfpathcurveto{\pgfqpoint{1.554275in}{2.577708in}}{\pgfqpoint{1.564874in}{2.582098in}}{\pgfqpoint{1.572688in}{2.589912in}}%
\pgfpathcurveto{\pgfqpoint{1.580502in}{2.597725in}}{\pgfqpoint{1.584892in}{2.608324in}}{\pgfqpoint{1.584892in}{2.619374in}}%
\pgfpathcurveto{\pgfqpoint{1.584892in}{2.630425in}}{\pgfqpoint{1.580502in}{2.641024in}}{\pgfqpoint{1.572688in}{2.648837in}}%
\pgfpathcurveto{\pgfqpoint{1.564874in}{2.656651in}}{\pgfqpoint{1.554275in}{2.661041in}}{\pgfqpoint{1.543225in}{2.661041in}}%
\pgfpathcurveto{\pgfqpoint{1.532175in}{2.661041in}}{\pgfqpoint{1.521576in}{2.656651in}}{\pgfqpoint{1.513762in}{2.648837in}}%
\pgfpathcurveto{\pgfqpoint{1.505949in}{2.641024in}}{\pgfqpoint{1.501558in}{2.630425in}}{\pgfqpoint{1.501558in}{2.619374in}}%
\pgfpathcurveto{\pgfqpoint{1.501558in}{2.608324in}}{\pgfqpoint{1.505949in}{2.597725in}}{\pgfqpoint{1.513762in}{2.589912in}}%
\pgfpathcurveto{\pgfqpoint{1.521576in}{2.582098in}}{\pgfqpoint{1.532175in}{2.577708in}}{\pgfqpoint{1.543225in}{2.577708in}}%
\pgfpathclose%
\pgfusepath{stroke,fill}%
\end{pgfscope}%
\begin{pgfscope}%
\pgfpathrectangle{\pgfqpoint{0.787074in}{0.548769in}}{\pgfqpoint{5.062926in}{3.102590in}}%
\pgfusepath{clip}%
\pgfsetbuttcap%
\pgfsetroundjoin%
\definecolor{currentfill}{rgb}{1.000000,0.498039,0.054902}%
\pgfsetfillcolor{currentfill}%
\pgfsetlinewidth{1.003750pt}%
\definecolor{currentstroke}{rgb}{1.000000,0.498039,0.054902}%
\pgfsetstrokecolor{currentstroke}%
\pgfsetdash{}{0pt}%
\pgfpathmoveto{\pgfqpoint{1.937739in}{2.595506in}}%
\pgfpathcurveto{\pgfqpoint{1.948789in}{2.595506in}}{\pgfqpoint{1.959388in}{2.599897in}}{\pgfqpoint{1.967202in}{2.607710in}}%
\pgfpathcurveto{\pgfqpoint{1.975015in}{2.615524in}}{\pgfqpoint{1.979406in}{2.626123in}}{\pgfqpoint{1.979406in}{2.637173in}}%
\pgfpathcurveto{\pgfqpoint{1.979406in}{2.648223in}}{\pgfqpoint{1.975015in}{2.658822in}}{\pgfqpoint{1.967202in}{2.666636in}}%
\pgfpathcurveto{\pgfqpoint{1.959388in}{2.674449in}}{\pgfqpoint{1.948789in}{2.678840in}}{\pgfqpoint{1.937739in}{2.678840in}}%
\pgfpathcurveto{\pgfqpoint{1.926689in}{2.678840in}}{\pgfqpoint{1.916090in}{2.674449in}}{\pgfqpoint{1.908276in}{2.666636in}}%
\pgfpathcurveto{\pgfqpoint{1.900462in}{2.658822in}}{\pgfqpoint{1.896072in}{2.648223in}}{\pgfqpoint{1.896072in}{2.637173in}}%
\pgfpathcurveto{\pgfqpoint{1.896072in}{2.626123in}}{\pgfqpoint{1.900462in}{2.615524in}}{\pgfqpoint{1.908276in}{2.607710in}}%
\pgfpathcurveto{\pgfqpoint{1.916090in}{2.599897in}}{\pgfqpoint{1.926689in}{2.595506in}}{\pgfqpoint{1.937739in}{2.595506in}}%
\pgfpathclose%
\pgfusepath{stroke,fill}%
\end{pgfscope}%
\begin{pgfscope}%
\pgfpathrectangle{\pgfqpoint{0.787074in}{0.548769in}}{\pgfqpoint{5.062926in}{3.102590in}}%
\pgfusepath{clip}%
\pgfsetbuttcap%
\pgfsetroundjoin%
\definecolor{currentfill}{rgb}{1.000000,0.498039,0.054902}%
\pgfsetfillcolor{currentfill}%
\pgfsetlinewidth{1.003750pt}%
\definecolor{currentstroke}{rgb}{1.000000,0.498039,0.054902}%
\pgfsetstrokecolor{currentstroke}%
\pgfsetdash{}{0pt}%
\pgfpathmoveto{\pgfqpoint{1.871987in}{1.910607in}}%
\pgfpathcurveto{\pgfqpoint{1.883037in}{1.910607in}}{\pgfqpoint{1.893636in}{1.914997in}}{\pgfqpoint{1.901449in}{1.922810in}}%
\pgfpathcurveto{\pgfqpoint{1.909263in}{1.930624in}}{\pgfqpoint{1.913653in}{1.941223in}}{\pgfqpoint{1.913653in}{1.952273in}}%
\pgfpathcurveto{\pgfqpoint{1.913653in}{1.963323in}}{\pgfqpoint{1.909263in}{1.973922in}}{\pgfqpoint{1.901449in}{1.981736in}}%
\pgfpathcurveto{\pgfqpoint{1.893636in}{1.989550in}}{\pgfqpoint{1.883037in}{1.993940in}}{\pgfqpoint{1.871987in}{1.993940in}}%
\pgfpathcurveto{\pgfqpoint{1.860936in}{1.993940in}}{\pgfqpoint{1.850337in}{1.989550in}}{\pgfqpoint{1.842524in}{1.981736in}}%
\pgfpathcurveto{\pgfqpoint{1.834710in}{1.973922in}}{\pgfqpoint{1.830320in}{1.963323in}}{\pgfqpoint{1.830320in}{1.952273in}}%
\pgfpathcurveto{\pgfqpoint{1.830320in}{1.941223in}}{\pgfqpoint{1.834710in}{1.930624in}}{\pgfqpoint{1.842524in}{1.922810in}}%
\pgfpathcurveto{\pgfqpoint{1.850337in}{1.914997in}}{\pgfqpoint{1.860936in}{1.910607in}}{\pgfqpoint{1.871987in}{1.910607in}}%
\pgfpathclose%
\pgfusepath{stroke,fill}%
\end{pgfscope}%
\begin{pgfscope}%
\pgfpathrectangle{\pgfqpoint{0.787074in}{0.548769in}}{\pgfqpoint{5.062926in}{3.102590in}}%
\pgfusepath{clip}%
\pgfsetbuttcap%
\pgfsetroundjoin%
\definecolor{currentfill}{rgb}{1.000000,0.498039,0.054902}%
\pgfsetfillcolor{currentfill}%
\pgfsetlinewidth{1.003750pt}%
\definecolor{currentstroke}{rgb}{1.000000,0.498039,0.054902}%
\pgfsetstrokecolor{currentstroke}%
\pgfsetdash{}{0pt}%
\pgfpathmoveto{\pgfqpoint{1.674730in}{1.406572in}}%
\pgfpathcurveto{\pgfqpoint{1.685780in}{1.406572in}}{\pgfqpoint{1.696379in}{1.410962in}}{\pgfqpoint{1.704193in}{1.418775in}}%
\pgfpathcurveto{\pgfqpoint{1.712006in}{1.426589in}}{\pgfqpoint{1.716396in}{1.437188in}}{\pgfqpoint{1.716396in}{1.448238in}}%
\pgfpathcurveto{\pgfqpoint{1.716396in}{1.459288in}}{\pgfqpoint{1.712006in}{1.469887in}}{\pgfqpoint{1.704193in}{1.477701in}}%
\pgfpathcurveto{\pgfqpoint{1.696379in}{1.485515in}}{\pgfqpoint{1.685780in}{1.489905in}}{\pgfqpoint{1.674730in}{1.489905in}}%
\pgfpathcurveto{\pgfqpoint{1.663680in}{1.489905in}}{\pgfqpoint{1.653081in}{1.485515in}}{\pgfqpoint{1.645267in}{1.477701in}}%
\pgfpathcurveto{\pgfqpoint{1.637453in}{1.469887in}}{\pgfqpoint{1.633063in}{1.459288in}}{\pgfqpoint{1.633063in}{1.448238in}}%
\pgfpathcurveto{\pgfqpoint{1.633063in}{1.437188in}}{\pgfqpoint{1.637453in}{1.426589in}}{\pgfqpoint{1.645267in}{1.418775in}}%
\pgfpathcurveto{\pgfqpoint{1.653081in}{1.410962in}}{\pgfqpoint{1.663680in}{1.406572in}}{\pgfqpoint{1.674730in}{1.406572in}}%
\pgfpathclose%
\pgfusepath{stroke,fill}%
\end{pgfscope}%
\begin{pgfscope}%
\pgfpathrectangle{\pgfqpoint{0.787074in}{0.548769in}}{\pgfqpoint{5.062926in}{3.102590in}}%
\pgfusepath{clip}%
\pgfsetbuttcap%
\pgfsetroundjoin%
\definecolor{currentfill}{rgb}{1.000000,0.498039,0.054902}%
\pgfsetfillcolor{currentfill}%
\pgfsetlinewidth{1.003750pt}%
\definecolor{currentstroke}{rgb}{1.000000,0.498039,0.054902}%
\pgfsetstrokecolor{currentstroke}%
\pgfsetdash{}{0pt}%
\pgfpathmoveto{\pgfqpoint{2.200748in}{2.078914in}}%
\pgfpathcurveto{\pgfqpoint{2.211798in}{2.078914in}}{\pgfqpoint{2.222397in}{2.083304in}}{\pgfqpoint{2.230211in}{2.091118in}}%
\pgfpathcurveto{\pgfqpoint{2.238024in}{2.098931in}}{\pgfqpoint{2.242415in}{2.109530in}}{\pgfqpoint{2.242415in}{2.120580in}}%
\pgfpathcurveto{\pgfqpoint{2.242415in}{2.131630in}}{\pgfqpoint{2.238024in}{2.142229in}}{\pgfqpoint{2.230211in}{2.150043in}}%
\pgfpathcurveto{\pgfqpoint{2.222397in}{2.157857in}}{\pgfqpoint{2.211798in}{2.162247in}}{\pgfqpoint{2.200748in}{2.162247in}}%
\pgfpathcurveto{\pgfqpoint{2.189698in}{2.162247in}}{\pgfqpoint{2.179099in}{2.157857in}}{\pgfqpoint{2.171285in}{2.150043in}}%
\pgfpathcurveto{\pgfqpoint{2.163472in}{2.142229in}}{\pgfqpoint{2.159081in}{2.131630in}}{\pgfqpoint{2.159081in}{2.120580in}}%
\pgfpathcurveto{\pgfqpoint{2.159081in}{2.109530in}}{\pgfqpoint{2.163472in}{2.098931in}}{\pgfqpoint{2.171285in}{2.091118in}}%
\pgfpathcurveto{\pgfqpoint{2.179099in}{2.083304in}}{\pgfqpoint{2.189698in}{2.078914in}}{\pgfqpoint{2.200748in}{2.078914in}}%
\pgfpathclose%
\pgfusepath{stroke,fill}%
\end{pgfscope}%
\begin{pgfscope}%
\pgfpathrectangle{\pgfqpoint{0.787074in}{0.548769in}}{\pgfqpoint{5.062926in}{3.102590in}}%
\pgfusepath{clip}%
\pgfsetbuttcap%
\pgfsetroundjoin%
\definecolor{currentfill}{rgb}{1.000000,0.498039,0.054902}%
\pgfsetfillcolor{currentfill}%
\pgfsetlinewidth{1.003750pt}%
\definecolor{currentstroke}{rgb}{1.000000,0.498039,0.054902}%
\pgfsetstrokecolor{currentstroke}%
\pgfsetdash{}{0pt}%
\pgfpathmoveto{\pgfqpoint{1.871987in}{1.659203in}}%
\pgfpathcurveto{\pgfqpoint{1.883037in}{1.659203in}}{\pgfqpoint{1.893636in}{1.663593in}}{\pgfqpoint{1.901449in}{1.671407in}}%
\pgfpathcurveto{\pgfqpoint{1.909263in}{1.679220in}}{\pgfqpoint{1.913653in}{1.689820in}}{\pgfqpoint{1.913653in}{1.700870in}}%
\pgfpathcurveto{\pgfqpoint{1.913653in}{1.711920in}}{\pgfqpoint{1.909263in}{1.722519in}}{\pgfqpoint{1.901449in}{1.730332in}}%
\pgfpathcurveto{\pgfqpoint{1.893636in}{1.738146in}}{\pgfqpoint{1.883037in}{1.742536in}}{\pgfqpoint{1.871987in}{1.742536in}}%
\pgfpathcurveto{\pgfqpoint{1.860936in}{1.742536in}}{\pgfqpoint{1.850337in}{1.738146in}}{\pgfqpoint{1.842524in}{1.730332in}}%
\pgfpathcurveto{\pgfqpoint{1.834710in}{1.722519in}}{\pgfqpoint{1.830320in}{1.711920in}}{\pgfqpoint{1.830320in}{1.700870in}}%
\pgfpathcurveto{\pgfqpoint{1.830320in}{1.689820in}}{\pgfqpoint{1.834710in}{1.679220in}}{\pgfqpoint{1.842524in}{1.671407in}}%
\pgfpathcurveto{\pgfqpoint{1.850337in}{1.663593in}}{\pgfqpoint{1.860936in}{1.659203in}}{\pgfqpoint{1.871987in}{1.659203in}}%
\pgfpathclose%
\pgfusepath{stroke,fill}%
\end{pgfscope}%
\begin{pgfscope}%
\pgfpathrectangle{\pgfqpoint{0.787074in}{0.548769in}}{\pgfqpoint{5.062926in}{3.102590in}}%
\pgfusepath{clip}%
\pgfsetbuttcap%
\pgfsetroundjoin%
\definecolor{currentfill}{rgb}{0.121569,0.466667,0.705882}%
\pgfsetfillcolor{currentfill}%
\pgfsetlinewidth{1.003750pt}%
\definecolor{currentstroke}{rgb}{0.121569,0.466667,0.705882}%
\pgfsetstrokecolor{currentstroke}%
\pgfsetdash{}{0pt}%
\pgfpathmoveto{\pgfqpoint{5.619867in}{1.472684in}}%
\pgfpathcurveto{\pgfqpoint{5.630917in}{1.472684in}}{\pgfqpoint{5.641516in}{1.477075in}}{\pgfqpoint{5.649330in}{1.484888in}}%
\pgfpathcurveto{\pgfqpoint{5.657143in}{1.492702in}}{\pgfqpoint{5.661534in}{1.503301in}}{\pgfqpoint{5.661534in}{1.514351in}}%
\pgfpathcurveto{\pgfqpoint{5.661534in}{1.525401in}}{\pgfqpoint{5.657143in}{1.536000in}}{\pgfqpoint{5.649330in}{1.543814in}}%
\pgfpathcurveto{\pgfqpoint{5.641516in}{1.551627in}}{\pgfqpoint{5.630917in}{1.556018in}}{\pgfqpoint{5.619867in}{1.556018in}}%
\pgfpathcurveto{\pgfqpoint{5.608817in}{1.556018in}}{\pgfqpoint{5.598218in}{1.551627in}}{\pgfqpoint{5.590404in}{1.543814in}}%
\pgfpathcurveto{\pgfqpoint{5.582591in}{1.536000in}}{\pgfqpoint{5.578200in}{1.525401in}}{\pgfqpoint{5.578200in}{1.514351in}}%
\pgfpathcurveto{\pgfqpoint{5.578200in}{1.503301in}}{\pgfqpoint{5.582591in}{1.492702in}}{\pgfqpoint{5.590404in}{1.484888in}}%
\pgfpathcurveto{\pgfqpoint{5.598218in}{1.477075in}}{\pgfqpoint{5.608817in}{1.472684in}}{\pgfqpoint{5.619867in}{1.472684in}}%
\pgfpathclose%
\pgfusepath{stroke,fill}%
\end{pgfscope}%
\begin{pgfscope}%
\pgfpathrectangle{\pgfqpoint{0.787074in}{0.548769in}}{\pgfqpoint{5.062926in}{3.102590in}}%
\pgfusepath{clip}%
\pgfsetbuttcap%
\pgfsetroundjoin%
\definecolor{currentfill}{rgb}{1.000000,0.498039,0.054902}%
\pgfsetfillcolor{currentfill}%
\pgfsetlinewidth{1.003750pt}%
\definecolor{currentstroke}{rgb}{1.000000,0.498039,0.054902}%
\pgfsetstrokecolor{currentstroke}%
\pgfsetdash{}{0pt}%
\pgfpathmoveto{\pgfqpoint{1.871987in}{2.453351in}}%
\pgfpathcurveto{\pgfqpoint{1.883037in}{2.453351in}}{\pgfqpoint{1.893636in}{2.457742in}}{\pgfqpoint{1.901449in}{2.465555in}}%
\pgfpathcurveto{\pgfqpoint{1.909263in}{2.473369in}}{\pgfqpoint{1.913653in}{2.483968in}}{\pgfqpoint{1.913653in}{2.495018in}}%
\pgfpathcurveto{\pgfqpoint{1.913653in}{2.506068in}}{\pgfqpoint{1.909263in}{2.516667in}}{\pgfqpoint{1.901449in}{2.524481in}}%
\pgfpathcurveto{\pgfqpoint{1.893636in}{2.532294in}}{\pgfqpoint{1.883037in}{2.536685in}}{\pgfqpoint{1.871987in}{2.536685in}}%
\pgfpathcurveto{\pgfqpoint{1.860936in}{2.536685in}}{\pgfqpoint{1.850337in}{2.532294in}}{\pgfqpoint{1.842524in}{2.524481in}}%
\pgfpathcurveto{\pgfqpoint{1.834710in}{2.516667in}}{\pgfqpoint{1.830320in}{2.506068in}}{\pgfqpoint{1.830320in}{2.495018in}}%
\pgfpathcurveto{\pgfqpoint{1.830320in}{2.483968in}}{\pgfqpoint{1.834710in}{2.473369in}}{\pgfqpoint{1.842524in}{2.465555in}}%
\pgfpathcurveto{\pgfqpoint{1.850337in}{2.457742in}}{\pgfqpoint{1.860936in}{2.453351in}}{\pgfqpoint{1.871987in}{2.453351in}}%
\pgfpathclose%
\pgfusepath{stroke,fill}%
\end{pgfscope}%
\begin{pgfscope}%
\pgfpathrectangle{\pgfqpoint{0.787074in}{0.548769in}}{\pgfqpoint{5.062926in}{3.102590in}}%
\pgfusepath{clip}%
\pgfsetbuttcap%
\pgfsetroundjoin%
\definecolor{currentfill}{rgb}{0.121569,0.466667,0.705882}%
\pgfsetfillcolor{currentfill}%
\pgfsetlinewidth{1.003750pt}%
\definecolor{currentstroke}{rgb}{0.121569,0.466667,0.705882}%
\pgfsetstrokecolor{currentstroke}%
\pgfsetdash{}{0pt}%
\pgfpathmoveto{\pgfqpoint{1.082959in}{0.648305in}}%
\pgfpathcurveto{\pgfqpoint{1.094009in}{0.648305in}}{\pgfqpoint{1.104608in}{0.652695in}}{\pgfqpoint{1.112422in}{0.660509in}}%
\pgfpathcurveto{\pgfqpoint{1.120236in}{0.668322in}}{\pgfqpoint{1.124626in}{0.678921in}}{\pgfqpoint{1.124626in}{0.689972in}}%
\pgfpathcurveto{\pgfqpoint{1.124626in}{0.701022in}}{\pgfqpoint{1.120236in}{0.711621in}}{\pgfqpoint{1.112422in}{0.719434in}}%
\pgfpathcurveto{\pgfqpoint{1.104608in}{0.727248in}}{\pgfqpoint{1.094009in}{0.731638in}}{\pgfqpoint{1.082959in}{0.731638in}}%
\pgfpathcurveto{\pgfqpoint{1.071909in}{0.731638in}}{\pgfqpoint{1.061310in}{0.727248in}}{\pgfqpoint{1.053496in}{0.719434in}}%
\pgfpathcurveto{\pgfqpoint{1.045683in}{0.711621in}}{\pgfqpoint{1.041292in}{0.701022in}}{\pgfqpoint{1.041292in}{0.689972in}}%
\pgfpathcurveto{\pgfqpoint{1.041292in}{0.678921in}}{\pgfqpoint{1.045683in}{0.668322in}}{\pgfqpoint{1.053496in}{0.660509in}}%
\pgfpathcurveto{\pgfqpoint{1.061310in}{0.652695in}}{\pgfqpoint{1.071909in}{0.648305in}}{\pgfqpoint{1.082959in}{0.648305in}}%
\pgfpathclose%
\pgfusepath{stroke,fill}%
\end{pgfscope}%
\begin{pgfscope}%
\pgfpathrectangle{\pgfqpoint{0.787074in}{0.548769in}}{\pgfqpoint{5.062926in}{3.102590in}}%
\pgfusepath{clip}%
\pgfsetbuttcap%
\pgfsetroundjoin%
\definecolor{currentfill}{rgb}{0.121569,0.466667,0.705882}%
\pgfsetfillcolor{currentfill}%
\pgfsetlinewidth{1.003750pt}%
\definecolor{currentstroke}{rgb}{0.121569,0.466667,0.705882}%
\pgfsetstrokecolor{currentstroke}%
\pgfsetdash{}{0pt}%
\pgfpathmoveto{\pgfqpoint{1.740482in}{1.422655in}}%
\pgfpathcurveto{\pgfqpoint{1.751532in}{1.422655in}}{\pgfqpoint{1.762131in}{1.427045in}}{\pgfqpoint{1.769945in}{1.434858in}}%
\pgfpathcurveto{\pgfqpoint{1.777758in}{1.442672in}}{\pgfqpoint{1.782149in}{1.453271in}}{\pgfqpoint{1.782149in}{1.464321in}}%
\pgfpathcurveto{\pgfqpoint{1.782149in}{1.475371in}}{\pgfqpoint{1.777758in}{1.485970in}}{\pgfqpoint{1.769945in}{1.493784in}}%
\pgfpathcurveto{\pgfqpoint{1.762131in}{1.501598in}}{\pgfqpoint{1.751532in}{1.505988in}}{\pgfqpoint{1.740482in}{1.505988in}}%
\pgfpathcurveto{\pgfqpoint{1.729432in}{1.505988in}}{\pgfqpoint{1.718833in}{1.501598in}}{\pgfqpoint{1.711019in}{1.493784in}}%
\pgfpathcurveto{\pgfqpoint{1.703206in}{1.485970in}}{\pgfqpoint{1.698815in}{1.475371in}}{\pgfqpoint{1.698815in}{1.464321in}}%
\pgfpathcurveto{\pgfqpoint{1.698815in}{1.453271in}}{\pgfqpoint{1.703206in}{1.442672in}}{\pgfqpoint{1.711019in}{1.434858in}}%
\pgfpathcurveto{\pgfqpoint{1.718833in}{1.427045in}}{\pgfqpoint{1.729432in}{1.422655in}}{\pgfqpoint{1.740482in}{1.422655in}}%
\pgfpathclose%
\pgfusepath{stroke,fill}%
\end{pgfscope}%
\begin{pgfscope}%
\pgfpathrectangle{\pgfqpoint{0.787074in}{0.548769in}}{\pgfqpoint{5.062926in}{3.102590in}}%
\pgfusepath{clip}%
\pgfsetbuttcap%
\pgfsetroundjoin%
\definecolor{currentfill}{rgb}{1.000000,0.498039,0.054902}%
\pgfsetfillcolor{currentfill}%
\pgfsetlinewidth{1.003750pt}%
\definecolor{currentstroke}{rgb}{1.000000,0.498039,0.054902}%
\pgfsetstrokecolor{currentstroke}%
\pgfsetdash{}{0pt}%
\pgfpathmoveto{\pgfqpoint{1.543225in}{2.833777in}}%
\pgfpathcurveto{\pgfqpoint{1.554275in}{2.833777in}}{\pgfqpoint{1.564874in}{2.838167in}}{\pgfqpoint{1.572688in}{2.845981in}}%
\pgfpathcurveto{\pgfqpoint{1.580502in}{2.853795in}}{\pgfqpoint{1.584892in}{2.864394in}}{\pgfqpoint{1.584892in}{2.875444in}}%
\pgfpathcurveto{\pgfqpoint{1.584892in}{2.886494in}}{\pgfqpoint{1.580502in}{2.897093in}}{\pgfqpoint{1.572688in}{2.904907in}}%
\pgfpathcurveto{\pgfqpoint{1.564874in}{2.912720in}}{\pgfqpoint{1.554275in}{2.917111in}}{\pgfqpoint{1.543225in}{2.917111in}}%
\pgfpathcurveto{\pgfqpoint{1.532175in}{2.917111in}}{\pgfqpoint{1.521576in}{2.912720in}}{\pgfqpoint{1.513762in}{2.904907in}}%
\pgfpathcurveto{\pgfqpoint{1.505949in}{2.897093in}}{\pgfqpoint{1.501558in}{2.886494in}}{\pgfqpoint{1.501558in}{2.875444in}}%
\pgfpathcurveto{\pgfqpoint{1.501558in}{2.864394in}}{\pgfqpoint{1.505949in}{2.853795in}}{\pgfqpoint{1.513762in}{2.845981in}}%
\pgfpathcurveto{\pgfqpoint{1.521576in}{2.838167in}}{\pgfqpoint{1.532175in}{2.833777in}}{\pgfqpoint{1.543225in}{2.833777in}}%
\pgfpathclose%
\pgfusepath{stroke,fill}%
\end{pgfscope}%
\begin{pgfscope}%
\pgfpathrectangle{\pgfqpoint{0.787074in}{0.548769in}}{\pgfqpoint{5.062926in}{3.102590in}}%
\pgfusepath{clip}%
\pgfsetbuttcap%
\pgfsetroundjoin%
\definecolor{currentfill}{rgb}{1.000000,0.498039,0.054902}%
\pgfsetfillcolor{currentfill}%
\pgfsetlinewidth{1.003750pt}%
\definecolor{currentstroke}{rgb}{1.000000,0.498039,0.054902}%
\pgfsetstrokecolor{currentstroke}%
\pgfsetdash{}{0pt}%
\pgfpathmoveto{\pgfqpoint{3.910308in}{1.540711in}}%
\pgfpathcurveto{\pgfqpoint{3.921358in}{1.540711in}}{\pgfqpoint{3.931957in}{1.545101in}}{\pgfqpoint{3.939770in}{1.552915in}}%
\pgfpathcurveto{\pgfqpoint{3.947584in}{1.560729in}}{\pgfqpoint{3.951974in}{1.571328in}}{\pgfqpoint{3.951974in}{1.582378in}}%
\pgfpathcurveto{\pgfqpoint{3.951974in}{1.593428in}}{\pgfqpoint{3.947584in}{1.604027in}}{\pgfqpoint{3.939770in}{1.611841in}}%
\pgfpathcurveto{\pgfqpoint{3.931957in}{1.619654in}}{\pgfqpoint{3.921358in}{1.624044in}}{\pgfqpoint{3.910308in}{1.624044in}}%
\pgfpathcurveto{\pgfqpoint{3.899257in}{1.624044in}}{\pgfqpoint{3.888658in}{1.619654in}}{\pgfqpoint{3.880845in}{1.611841in}}%
\pgfpathcurveto{\pgfqpoint{3.873031in}{1.604027in}}{\pgfqpoint{3.868641in}{1.593428in}}{\pgfqpoint{3.868641in}{1.582378in}}%
\pgfpathcurveto{\pgfqpoint{3.868641in}{1.571328in}}{\pgfqpoint{3.873031in}{1.560729in}}{\pgfqpoint{3.880845in}{1.552915in}}%
\pgfpathcurveto{\pgfqpoint{3.888658in}{1.545101in}}{\pgfqpoint{3.899257in}{1.540711in}}{\pgfqpoint{3.910308in}{1.540711in}}%
\pgfpathclose%
\pgfusepath{stroke,fill}%
\end{pgfscope}%
\begin{pgfscope}%
\pgfpathrectangle{\pgfqpoint{0.787074in}{0.548769in}}{\pgfqpoint{5.062926in}{3.102590in}}%
\pgfusepath{clip}%
\pgfsetbuttcap%
\pgfsetroundjoin%
\definecolor{currentfill}{rgb}{1.000000,0.498039,0.054902}%
\pgfsetfillcolor{currentfill}%
\pgfsetlinewidth{1.003750pt}%
\definecolor{currentstroke}{rgb}{1.000000,0.498039,0.054902}%
\pgfsetstrokecolor{currentstroke}%
\pgfsetdash{}{0pt}%
\pgfpathmoveto{\pgfqpoint{2.134996in}{1.951746in}}%
\pgfpathcurveto{\pgfqpoint{2.146046in}{1.951746in}}{\pgfqpoint{2.156645in}{1.956136in}}{\pgfqpoint{2.164459in}{1.963950in}}%
\pgfpathcurveto{\pgfqpoint{2.172272in}{1.971763in}}{\pgfqpoint{2.176662in}{1.982362in}}{\pgfqpoint{2.176662in}{1.993412in}}%
\pgfpathcurveto{\pgfqpoint{2.176662in}{2.004463in}}{\pgfqpoint{2.172272in}{2.015062in}}{\pgfqpoint{2.164459in}{2.022875in}}%
\pgfpathcurveto{\pgfqpoint{2.156645in}{2.030689in}}{\pgfqpoint{2.146046in}{2.035079in}}{\pgfqpoint{2.134996in}{2.035079in}}%
\pgfpathcurveto{\pgfqpoint{2.123946in}{2.035079in}}{\pgfqpoint{2.113347in}{2.030689in}}{\pgfqpoint{2.105533in}{2.022875in}}%
\pgfpathcurveto{\pgfqpoint{2.097719in}{2.015062in}}{\pgfqpoint{2.093329in}{2.004463in}}{\pgfqpoint{2.093329in}{1.993412in}}%
\pgfpathcurveto{\pgfqpoint{2.093329in}{1.982362in}}{\pgfqpoint{2.097719in}{1.971763in}}{\pgfqpoint{2.105533in}{1.963950in}}%
\pgfpathcurveto{\pgfqpoint{2.113347in}{1.956136in}}{\pgfqpoint{2.123946in}{1.951746in}}{\pgfqpoint{2.134996in}{1.951746in}}%
\pgfpathclose%
\pgfusepath{stroke,fill}%
\end{pgfscope}%
\begin{pgfscope}%
\pgfpathrectangle{\pgfqpoint{0.787074in}{0.548769in}}{\pgfqpoint{5.062926in}{3.102590in}}%
\pgfusepath{clip}%
\pgfsetbuttcap%
\pgfsetroundjoin%
\definecolor{currentfill}{rgb}{1.000000,0.498039,0.054902}%
\pgfsetfillcolor{currentfill}%
\pgfsetlinewidth{1.003750pt}%
\definecolor{currentstroke}{rgb}{1.000000,0.498039,0.054902}%
\pgfsetstrokecolor{currentstroke}%
\pgfsetdash{}{0pt}%
\pgfpathmoveto{\pgfqpoint{1.411721in}{1.644565in}}%
\pgfpathcurveto{\pgfqpoint{1.422771in}{1.644565in}}{\pgfqpoint{1.433370in}{1.648955in}}{\pgfqpoint{1.441183in}{1.656769in}}%
\pgfpathcurveto{\pgfqpoint{1.448997in}{1.664582in}}{\pgfqpoint{1.453387in}{1.675181in}}{\pgfqpoint{1.453387in}{1.686232in}}%
\pgfpathcurveto{\pgfqpoint{1.453387in}{1.697282in}}{\pgfqpoint{1.448997in}{1.707881in}}{\pgfqpoint{1.441183in}{1.715694in}}%
\pgfpathcurveto{\pgfqpoint{1.433370in}{1.723508in}}{\pgfqpoint{1.422771in}{1.727898in}}{\pgfqpoint{1.411721in}{1.727898in}}%
\pgfpathcurveto{\pgfqpoint{1.400670in}{1.727898in}}{\pgfqpoint{1.390071in}{1.723508in}}{\pgfqpoint{1.382258in}{1.715694in}}%
\pgfpathcurveto{\pgfqpoint{1.374444in}{1.707881in}}{\pgfqpoint{1.370054in}{1.697282in}}{\pgfqpoint{1.370054in}{1.686232in}}%
\pgfpathcurveto{\pgfqpoint{1.370054in}{1.675181in}}{\pgfqpoint{1.374444in}{1.664582in}}{\pgfqpoint{1.382258in}{1.656769in}}%
\pgfpathcurveto{\pgfqpoint{1.390071in}{1.648955in}}{\pgfqpoint{1.400670in}{1.644565in}}{\pgfqpoint{1.411721in}{1.644565in}}%
\pgfpathclose%
\pgfusepath{stroke,fill}%
\end{pgfscope}%
\begin{pgfscope}%
\pgfpathrectangle{\pgfqpoint{0.787074in}{0.548769in}}{\pgfqpoint{5.062926in}{3.102590in}}%
\pgfusepath{clip}%
\pgfsetbuttcap%
\pgfsetroundjoin%
\definecolor{currentfill}{rgb}{0.121569,0.466667,0.705882}%
\pgfsetfillcolor{currentfill}%
\pgfsetlinewidth{1.003750pt}%
\definecolor{currentstroke}{rgb}{0.121569,0.466667,0.705882}%
\pgfsetstrokecolor{currentstroke}%
\pgfsetdash{}{0pt}%
\pgfpathmoveto{\pgfqpoint{1.280216in}{0.664836in}}%
\pgfpathcurveto{\pgfqpoint{1.291266in}{0.664836in}}{\pgfqpoint{1.301865in}{0.669227in}}{\pgfqpoint{1.309679in}{0.677040in}}%
\pgfpathcurveto{\pgfqpoint{1.317492in}{0.684854in}}{\pgfqpoint{1.321883in}{0.695453in}}{\pgfqpoint{1.321883in}{0.706503in}}%
\pgfpathcurveto{\pgfqpoint{1.321883in}{0.717553in}}{\pgfqpoint{1.317492in}{0.728152in}}{\pgfqpoint{1.309679in}{0.735966in}}%
\pgfpathcurveto{\pgfqpoint{1.301865in}{0.743779in}}{\pgfqpoint{1.291266in}{0.748170in}}{\pgfqpoint{1.280216in}{0.748170in}}%
\pgfpathcurveto{\pgfqpoint{1.269166in}{0.748170in}}{\pgfqpoint{1.258567in}{0.743779in}}{\pgfqpoint{1.250753in}{0.735966in}}%
\pgfpathcurveto{\pgfqpoint{1.242940in}{0.728152in}}{\pgfqpoint{1.238549in}{0.717553in}}{\pgfqpoint{1.238549in}{0.706503in}}%
\pgfpathcurveto{\pgfqpoint{1.238549in}{0.695453in}}{\pgfqpoint{1.242940in}{0.684854in}}{\pgfqpoint{1.250753in}{0.677040in}}%
\pgfpathcurveto{\pgfqpoint{1.258567in}{0.669227in}}{\pgfqpoint{1.269166in}{0.664836in}}{\pgfqpoint{1.280216in}{0.664836in}}%
\pgfpathclose%
\pgfusepath{stroke,fill}%
\end{pgfscope}%
\begin{pgfscope}%
\pgfpathrectangle{\pgfqpoint{0.787074in}{0.548769in}}{\pgfqpoint{5.062926in}{3.102590in}}%
\pgfusepath{clip}%
\pgfsetbuttcap%
\pgfsetroundjoin%
\definecolor{currentfill}{rgb}{1.000000,0.498039,0.054902}%
\pgfsetfillcolor{currentfill}%
\pgfsetlinewidth{1.003750pt}%
\definecolor{currentstroke}{rgb}{1.000000,0.498039,0.054902}%
\pgfsetstrokecolor{currentstroke}%
\pgfsetdash{}{0pt}%
\pgfpathmoveto{\pgfqpoint{1.740482in}{1.428628in}}%
\pgfpathcurveto{\pgfqpoint{1.751532in}{1.428628in}}{\pgfqpoint{1.762131in}{1.433018in}}{\pgfqpoint{1.769945in}{1.440832in}}%
\pgfpathcurveto{\pgfqpoint{1.777758in}{1.448646in}}{\pgfqpoint{1.782149in}{1.459245in}}{\pgfqpoint{1.782149in}{1.470295in}}%
\pgfpathcurveto{\pgfqpoint{1.782149in}{1.481345in}}{\pgfqpoint{1.777758in}{1.491944in}}{\pgfqpoint{1.769945in}{1.499758in}}%
\pgfpathcurveto{\pgfqpoint{1.762131in}{1.507571in}}{\pgfqpoint{1.751532in}{1.511961in}}{\pgfqpoint{1.740482in}{1.511961in}}%
\pgfpathcurveto{\pgfqpoint{1.729432in}{1.511961in}}{\pgfqpoint{1.718833in}{1.507571in}}{\pgfqpoint{1.711019in}{1.499758in}}%
\pgfpathcurveto{\pgfqpoint{1.703206in}{1.491944in}}{\pgfqpoint{1.698815in}{1.481345in}}{\pgfqpoint{1.698815in}{1.470295in}}%
\pgfpathcurveto{\pgfqpoint{1.698815in}{1.459245in}}{\pgfqpoint{1.703206in}{1.448646in}}{\pgfqpoint{1.711019in}{1.440832in}}%
\pgfpathcurveto{\pgfqpoint{1.718833in}{1.433018in}}{\pgfqpoint{1.729432in}{1.428628in}}{\pgfqpoint{1.740482in}{1.428628in}}%
\pgfpathclose%
\pgfusepath{stroke,fill}%
\end{pgfscope}%
\begin{pgfscope}%
\pgfpathrectangle{\pgfqpoint{0.787074in}{0.548769in}}{\pgfqpoint{5.062926in}{3.102590in}}%
\pgfusepath{clip}%
\pgfsetbuttcap%
\pgfsetroundjoin%
\definecolor{currentfill}{rgb}{0.121569,0.466667,0.705882}%
\pgfsetfillcolor{currentfill}%
\pgfsetlinewidth{1.003750pt}%
\definecolor{currentstroke}{rgb}{0.121569,0.466667,0.705882}%
\pgfsetstrokecolor{currentstroke}%
\pgfsetdash{}{0pt}%
\pgfpathmoveto{\pgfqpoint{1.280216in}{0.648339in}}%
\pgfpathcurveto{\pgfqpoint{1.291266in}{0.648339in}}{\pgfqpoint{1.301865in}{0.652729in}}{\pgfqpoint{1.309679in}{0.660543in}}%
\pgfpathcurveto{\pgfqpoint{1.317492in}{0.668356in}}{\pgfqpoint{1.321883in}{0.678955in}}{\pgfqpoint{1.321883in}{0.690005in}}%
\pgfpathcurveto{\pgfqpoint{1.321883in}{0.701056in}}{\pgfqpoint{1.317492in}{0.711655in}}{\pgfqpoint{1.309679in}{0.719468in}}%
\pgfpathcurveto{\pgfqpoint{1.301865in}{0.727282in}}{\pgfqpoint{1.291266in}{0.731672in}}{\pgfqpoint{1.280216in}{0.731672in}}%
\pgfpathcurveto{\pgfqpoint{1.269166in}{0.731672in}}{\pgfqpoint{1.258567in}{0.727282in}}{\pgfqpoint{1.250753in}{0.719468in}}%
\pgfpathcurveto{\pgfqpoint{1.242940in}{0.711655in}}{\pgfqpoint{1.238549in}{0.701056in}}{\pgfqpoint{1.238549in}{0.690005in}}%
\pgfpathcurveto{\pgfqpoint{1.238549in}{0.678955in}}{\pgfqpoint{1.242940in}{0.668356in}}{\pgfqpoint{1.250753in}{0.660543in}}%
\pgfpathcurveto{\pgfqpoint{1.258567in}{0.652729in}}{\pgfqpoint{1.269166in}{0.648339in}}{\pgfqpoint{1.280216in}{0.648339in}}%
\pgfpathclose%
\pgfusepath{stroke,fill}%
\end{pgfscope}%
\begin{pgfscope}%
\pgfpathrectangle{\pgfqpoint{0.787074in}{0.548769in}}{\pgfqpoint{5.062926in}{3.102590in}}%
\pgfusepath{clip}%
\pgfsetbuttcap%
\pgfsetroundjoin%
\definecolor{currentfill}{rgb}{1.000000,0.498039,0.054902}%
\pgfsetfillcolor{currentfill}%
\pgfsetlinewidth{1.003750pt}%
\definecolor{currentstroke}{rgb}{1.000000,0.498039,0.054902}%
\pgfsetstrokecolor{currentstroke}%
\pgfsetdash{}{0pt}%
\pgfpathmoveto{\pgfqpoint{2.200748in}{1.507887in}}%
\pgfpathcurveto{\pgfqpoint{2.211798in}{1.507887in}}{\pgfqpoint{2.222397in}{1.512277in}}{\pgfqpoint{2.230211in}{1.520091in}}%
\pgfpathcurveto{\pgfqpoint{2.238024in}{1.527904in}}{\pgfqpoint{2.242415in}{1.538503in}}{\pgfqpoint{2.242415in}{1.549554in}}%
\pgfpathcurveto{\pgfqpoint{2.242415in}{1.560604in}}{\pgfqpoint{2.238024in}{1.571203in}}{\pgfqpoint{2.230211in}{1.579016in}}%
\pgfpathcurveto{\pgfqpoint{2.222397in}{1.586830in}}{\pgfqpoint{2.211798in}{1.591220in}}{\pgfqpoint{2.200748in}{1.591220in}}%
\pgfpathcurveto{\pgfqpoint{2.189698in}{1.591220in}}{\pgfqpoint{2.179099in}{1.586830in}}{\pgfqpoint{2.171285in}{1.579016in}}%
\pgfpathcurveto{\pgfqpoint{2.163472in}{1.571203in}}{\pgfqpoint{2.159081in}{1.560604in}}{\pgfqpoint{2.159081in}{1.549554in}}%
\pgfpathcurveto{\pgfqpoint{2.159081in}{1.538503in}}{\pgfqpoint{2.163472in}{1.527904in}}{\pgfqpoint{2.171285in}{1.520091in}}%
\pgfpathcurveto{\pgfqpoint{2.179099in}{1.512277in}}{\pgfqpoint{2.189698in}{1.507887in}}{\pgfqpoint{2.200748in}{1.507887in}}%
\pgfpathclose%
\pgfusepath{stroke,fill}%
\end{pgfscope}%
\begin{pgfscope}%
\pgfpathrectangle{\pgfqpoint{0.787074in}{0.548769in}}{\pgfqpoint{5.062926in}{3.102590in}}%
\pgfusepath{clip}%
\pgfsetbuttcap%
\pgfsetroundjoin%
\definecolor{currentfill}{rgb}{1.000000,0.498039,0.054902}%
\pgfsetfillcolor{currentfill}%
\pgfsetlinewidth{1.003750pt}%
\definecolor{currentstroke}{rgb}{1.000000,0.498039,0.054902}%
\pgfsetstrokecolor{currentstroke}%
\pgfsetdash{}{0pt}%
\pgfpathmoveto{\pgfqpoint{2.069243in}{2.621734in}}%
\pgfpathcurveto{\pgfqpoint{2.080294in}{2.621734in}}{\pgfqpoint{2.090893in}{2.626124in}}{\pgfqpoint{2.098706in}{2.633938in}}%
\pgfpathcurveto{\pgfqpoint{2.106520in}{2.641751in}}{\pgfqpoint{2.110910in}{2.652350in}}{\pgfqpoint{2.110910in}{2.663401in}}%
\pgfpathcurveto{\pgfqpoint{2.110910in}{2.674451in}}{\pgfqpoint{2.106520in}{2.685050in}}{\pgfqpoint{2.098706in}{2.692863in}}%
\pgfpathcurveto{\pgfqpoint{2.090893in}{2.700677in}}{\pgfqpoint{2.080294in}{2.705067in}}{\pgfqpoint{2.069243in}{2.705067in}}%
\pgfpathcurveto{\pgfqpoint{2.058193in}{2.705067in}}{\pgfqpoint{2.047594in}{2.700677in}}{\pgfqpoint{2.039781in}{2.692863in}}%
\pgfpathcurveto{\pgfqpoint{2.031967in}{2.685050in}}{\pgfqpoint{2.027577in}{2.674451in}}{\pgfqpoint{2.027577in}{2.663401in}}%
\pgfpathcurveto{\pgfqpoint{2.027577in}{2.652350in}}{\pgfqpoint{2.031967in}{2.641751in}}{\pgfqpoint{2.039781in}{2.633938in}}%
\pgfpathcurveto{\pgfqpoint{2.047594in}{2.626124in}}{\pgfqpoint{2.058193in}{2.621734in}}{\pgfqpoint{2.069243in}{2.621734in}}%
\pgfpathclose%
\pgfusepath{stroke,fill}%
\end{pgfscope}%
\begin{pgfscope}%
\pgfpathrectangle{\pgfqpoint{0.787074in}{0.548769in}}{\pgfqpoint{5.062926in}{3.102590in}}%
\pgfusepath{clip}%
\pgfsetbuttcap%
\pgfsetroundjoin%
\definecolor{currentfill}{rgb}{1.000000,0.498039,0.054902}%
\pgfsetfillcolor{currentfill}%
\pgfsetlinewidth{1.003750pt}%
\definecolor{currentstroke}{rgb}{1.000000,0.498039,0.054902}%
\pgfsetstrokecolor{currentstroke}%
\pgfsetdash{}{0pt}%
\pgfpathmoveto{\pgfqpoint{4.765087in}{1.832304in}}%
\pgfpathcurveto{\pgfqpoint{4.776137in}{1.832304in}}{\pgfqpoint{4.786736in}{1.836694in}}{\pgfqpoint{4.794550in}{1.844508in}}%
\pgfpathcurveto{\pgfqpoint{4.802364in}{1.852322in}}{\pgfqpoint{4.806754in}{1.862921in}}{\pgfqpoint{4.806754in}{1.873971in}}%
\pgfpathcurveto{\pgfqpoint{4.806754in}{1.885021in}}{\pgfqpoint{4.802364in}{1.895620in}}{\pgfqpoint{4.794550in}{1.903433in}}%
\pgfpathcurveto{\pgfqpoint{4.786736in}{1.911247in}}{\pgfqpoint{4.776137in}{1.915637in}}{\pgfqpoint{4.765087in}{1.915637in}}%
\pgfpathcurveto{\pgfqpoint{4.754037in}{1.915637in}}{\pgfqpoint{4.743438in}{1.911247in}}{\pgfqpoint{4.735624in}{1.903433in}}%
\pgfpathcurveto{\pgfqpoint{4.727811in}{1.895620in}}{\pgfqpoint{4.723421in}{1.885021in}}{\pgfqpoint{4.723421in}{1.873971in}}%
\pgfpathcurveto{\pgfqpoint{4.723421in}{1.862921in}}{\pgfqpoint{4.727811in}{1.852322in}}{\pgfqpoint{4.735624in}{1.844508in}}%
\pgfpathcurveto{\pgfqpoint{4.743438in}{1.836694in}}{\pgfqpoint{4.754037in}{1.832304in}}{\pgfqpoint{4.765087in}{1.832304in}}%
\pgfpathclose%
\pgfusepath{stroke,fill}%
\end{pgfscope}%
\begin{pgfscope}%
\pgfpathrectangle{\pgfqpoint{0.787074in}{0.548769in}}{\pgfqpoint{5.062926in}{3.102590in}}%
\pgfusepath{clip}%
\pgfsetbuttcap%
\pgfsetroundjoin%
\definecolor{currentfill}{rgb}{1.000000,0.498039,0.054902}%
\pgfsetfillcolor{currentfill}%
\pgfsetlinewidth{1.003750pt}%
\definecolor{currentstroke}{rgb}{1.000000,0.498039,0.054902}%
\pgfsetstrokecolor{currentstroke}%
\pgfsetdash{}{0pt}%
\pgfpathmoveto{\pgfqpoint{1.608977in}{3.055568in}}%
\pgfpathcurveto{\pgfqpoint{1.620028in}{3.055568in}}{\pgfqpoint{1.630627in}{3.059958in}}{\pgfqpoint{1.638440in}{3.067772in}}%
\pgfpathcurveto{\pgfqpoint{1.646254in}{3.075585in}}{\pgfqpoint{1.650644in}{3.086184in}}{\pgfqpoint{1.650644in}{3.097235in}}%
\pgfpathcurveto{\pgfqpoint{1.650644in}{3.108285in}}{\pgfqpoint{1.646254in}{3.118884in}}{\pgfqpoint{1.638440in}{3.126697in}}%
\pgfpathcurveto{\pgfqpoint{1.630627in}{3.134511in}}{\pgfqpoint{1.620028in}{3.138901in}}{\pgfqpoint{1.608977in}{3.138901in}}%
\pgfpathcurveto{\pgfqpoint{1.597927in}{3.138901in}}{\pgfqpoint{1.587328in}{3.134511in}}{\pgfqpoint{1.579515in}{3.126697in}}%
\pgfpathcurveto{\pgfqpoint{1.571701in}{3.118884in}}{\pgfqpoint{1.567311in}{3.108285in}}{\pgfqpoint{1.567311in}{3.097235in}}%
\pgfpathcurveto{\pgfqpoint{1.567311in}{3.086184in}}{\pgfqpoint{1.571701in}{3.075585in}}{\pgfqpoint{1.579515in}{3.067772in}}%
\pgfpathcurveto{\pgfqpoint{1.587328in}{3.059958in}}{\pgfqpoint{1.597927in}{3.055568in}}{\pgfqpoint{1.608977in}{3.055568in}}%
\pgfpathclose%
\pgfusepath{stroke,fill}%
\end{pgfscope}%
\begin{pgfscope}%
\pgfpathrectangle{\pgfqpoint{0.787074in}{0.548769in}}{\pgfqpoint{5.062926in}{3.102590in}}%
\pgfusepath{clip}%
\pgfsetbuttcap%
\pgfsetroundjoin%
\definecolor{currentfill}{rgb}{1.000000,0.498039,0.054902}%
\pgfsetfillcolor{currentfill}%
\pgfsetlinewidth{1.003750pt}%
\definecolor{currentstroke}{rgb}{1.000000,0.498039,0.054902}%
\pgfsetstrokecolor{currentstroke}%
\pgfsetdash{}{0pt}%
\pgfpathmoveto{\pgfqpoint{1.477473in}{3.291673in}}%
\pgfpathcurveto{\pgfqpoint{1.488523in}{3.291673in}}{\pgfqpoint{1.499122in}{3.296063in}}{\pgfqpoint{1.506936in}{3.303876in}}%
\pgfpathcurveto{\pgfqpoint{1.514749in}{3.311690in}}{\pgfqpoint{1.519140in}{3.322289in}}{\pgfqpoint{1.519140in}{3.333339in}}%
\pgfpathcurveto{\pgfqpoint{1.519140in}{3.344389in}}{\pgfqpoint{1.514749in}{3.354988in}}{\pgfqpoint{1.506936in}{3.362802in}}%
\pgfpathcurveto{\pgfqpoint{1.499122in}{3.370616in}}{\pgfqpoint{1.488523in}{3.375006in}}{\pgfqpoint{1.477473in}{3.375006in}}%
\pgfpathcurveto{\pgfqpoint{1.466423in}{3.375006in}}{\pgfqpoint{1.455824in}{3.370616in}}{\pgfqpoint{1.448010in}{3.362802in}}%
\pgfpathcurveto{\pgfqpoint{1.440196in}{3.354988in}}{\pgfqpoint{1.435806in}{3.344389in}}{\pgfqpoint{1.435806in}{3.333339in}}%
\pgfpathcurveto{\pgfqpoint{1.435806in}{3.322289in}}{\pgfqpoint{1.440196in}{3.311690in}}{\pgfqpoint{1.448010in}{3.303876in}}%
\pgfpathcurveto{\pgfqpoint{1.455824in}{3.296063in}}{\pgfqpoint{1.466423in}{3.291673in}}{\pgfqpoint{1.477473in}{3.291673in}}%
\pgfpathclose%
\pgfusepath{stroke,fill}%
\end{pgfscope}%
\begin{pgfscope}%
\pgfpathrectangle{\pgfqpoint{0.787074in}{0.548769in}}{\pgfqpoint{5.062926in}{3.102590in}}%
\pgfusepath{clip}%
\pgfsetbuttcap%
\pgfsetroundjoin%
\definecolor{currentfill}{rgb}{1.000000,0.498039,0.054902}%
\pgfsetfillcolor{currentfill}%
\pgfsetlinewidth{1.003750pt}%
\definecolor{currentstroke}{rgb}{1.000000,0.498039,0.054902}%
\pgfsetstrokecolor{currentstroke}%
\pgfsetdash{}{0pt}%
\pgfpathmoveto{\pgfqpoint{1.411721in}{1.890154in}}%
\pgfpathcurveto{\pgfqpoint{1.422771in}{1.890154in}}{\pgfqpoint{1.433370in}{1.894544in}}{\pgfqpoint{1.441183in}{1.902358in}}%
\pgfpathcurveto{\pgfqpoint{1.448997in}{1.910171in}}{\pgfqpoint{1.453387in}{1.920770in}}{\pgfqpoint{1.453387in}{1.931820in}}%
\pgfpathcurveto{\pgfqpoint{1.453387in}{1.942870in}}{\pgfqpoint{1.448997in}{1.953469in}}{\pgfqpoint{1.441183in}{1.961283in}}%
\pgfpathcurveto{\pgfqpoint{1.433370in}{1.969097in}}{\pgfqpoint{1.422771in}{1.973487in}}{\pgfqpoint{1.411721in}{1.973487in}}%
\pgfpathcurveto{\pgfqpoint{1.400670in}{1.973487in}}{\pgfqpoint{1.390071in}{1.969097in}}{\pgfqpoint{1.382258in}{1.961283in}}%
\pgfpathcurveto{\pgfqpoint{1.374444in}{1.953469in}}{\pgfqpoint{1.370054in}{1.942870in}}{\pgfqpoint{1.370054in}{1.931820in}}%
\pgfpathcurveto{\pgfqpoint{1.370054in}{1.920770in}}{\pgfqpoint{1.374444in}{1.910171in}}{\pgfqpoint{1.382258in}{1.902358in}}%
\pgfpathcurveto{\pgfqpoint{1.390071in}{1.894544in}}{\pgfqpoint{1.400670in}{1.890154in}}{\pgfqpoint{1.411721in}{1.890154in}}%
\pgfpathclose%
\pgfusepath{stroke,fill}%
\end{pgfscope}%
\begin{pgfscope}%
\pgfpathrectangle{\pgfqpoint{0.787074in}{0.548769in}}{\pgfqpoint{5.062926in}{3.102590in}}%
\pgfusepath{clip}%
\pgfsetbuttcap%
\pgfsetroundjoin%
\definecolor{currentfill}{rgb}{0.121569,0.466667,0.705882}%
\pgfsetfillcolor{currentfill}%
\pgfsetlinewidth{1.003750pt}%
\definecolor{currentstroke}{rgb}{0.121569,0.466667,0.705882}%
\pgfsetstrokecolor{currentstroke}%
\pgfsetdash{}{0pt}%
\pgfpathmoveto{\pgfqpoint{1.411721in}{1.784271in}}%
\pgfpathcurveto{\pgfqpoint{1.422771in}{1.784271in}}{\pgfqpoint{1.433370in}{1.788661in}}{\pgfqpoint{1.441183in}{1.796475in}}%
\pgfpathcurveto{\pgfqpoint{1.448997in}{1.804289in}}{\pgfqpoint{1.453387in}{1.814888in}}{\pgfqpoint{1.453387in}{1.825938in}}%
\pgfpathcurveto{\pgfqpoint{1.453387in}{1.836988in}}{\pgfqpoint{1.448997in}{1.847587in}}{\pgfqpoint{1.441183in}{1.855401in}}%
\pgfpathcurveto{\pgfqpoint{1.433370in}{1.863214in}}{\pgfqpoint{1.422771in}{1.867604in}}{\pgfqpoint{1.411721in}{1.867604in}}%
\pgfpathcurveto{\pgfqpoint{1.400670in}{1.867604in}}{\pgfqpoint{1.390071in}{1.863214in}}{\pgfqpoint{1.382258in}{1.855401in}}%
\pgfpathcurveto{\pgfqpoint{1.374444in}{1.847587in}}{\pgfqpoint{1.370054in}{1.836988in}}{\pgfqpoint{1.370054in}{1.825938in}}%
\pgfpathcurveto{\pgfqpoint{1.370054in}{1.814888in}}{\pgfqpoint{1.374444in}{1.804289in}}{\pgfqpoint{1.382258in}{1.796475in}}%
\pgfpathcurveto{\pgfqpoint{1.390071in}{1.788661in}}{\pgfqpoint{1.400670in}{1.784271in}}{\pgfqpoint{1.411721in}{1.784271in}}%
\pgfpathclose%
\pgfusepath{stroke,fill}%
\end{pgfscope}%
\begin{pgfscope}%
\pgfpathrectangle{\pgfqpoint{0.787074in}{0.548769in}}{\pgfqpoint{5.062926in}{3.102590in}}%
\pgfusepath{clip}%
\pgfsetbuttcap%
\pgfsetroundjoin%
\definecolor{currentfill}{rgb}{1.000000,0.498039,0.054902}%
\pgfsetfillcolor{currentfill}%
\pgfsetlinewidth{1.003750pt}%
\definecolor{currentstroke}{rgb}{1.000000,0.498039,0.054902}%
\pgfsetstrokecolor{currentstroke}%
\pgfsetdash{}{0pt}%
\pgfpathmoveto{\pgfqpoint{1.543225in}{1.980354in}}%
\pgfpathcurveto{\pgfqpoint{1.554275in}{1.980354in}}{\pgfqpoint{1.564874in}{1.984745in}}{\pgfqpoint{1.572688in}{1.992558in}}%
\pgfpathcurveto{\pgfqpoint{1.580502in}{2.000372in}}{\pgfqpoint{1.584892in}{2.010971in}}{\pgfqpoint{1.584892in}{2.022021in}}%
\pgfpathcurveto{\pgfqpoint{1.584892in}{2.033071in}}{\pgfqpoint{1.580502in}{2.043670in}}{\pgfqpoint{1.572688in}{2.051484in}}%
\pgfpathcurveto{\pgfqpoint{1.564874in}{2.059297in}}{\pgfqpoint{1.554275in}{2.063688in}}{\pgfqpoint{1.543225in}{2.063688in}}%
\pgfpathcurveto{\pgfqpoint{1.532175in}{2.063688in}}{\pgfqpoint{1.521576in}{2.059297in}}{\pgfqpoint{1.513762in}{2.051484in}}%
\pgfpathcurveto{\pgfqpoint{1.505949in}{2.043670in}}{\pgfqpoint{1.501558in}{2.033071in}}{\pgfqpoint{1.501558in}{2.022021in}}%
\pgfpathcurveto{\pgfqpoint{1.501558in}{2.010971in}}{\pgfqpoint{1.505949in}{2.000372in}}{\pgfqpoint{1.513762in}{1.992558in}}%
\pgfpathcurveto{\pgfqpoint{1.521576in}{1.984745in}}{\pgfqpoint{1.532175in}{1.980354in}}{\pgfqpoint{1.543225in}{1.980354in}}%
\pgfpathclose%
\pgfusepath{stroke,fill}%
\end{pgfscope}%
\begin{pgfscope}%
\pgfpathrectangle{\pgfqpoint{0.787074in}{0.548769in}}{\pgfqpoint{5.062926in}{3.102590in}}%
\pgfusepath{clip}%
\pgfsetbuttcap%
\pgfsetroundjoin%
\definecolor{currentfill}{rgb}{1.000000,0.498039,0.054902}%
\pgfsetfillcolor{currentfill}%
\pgfsetlinewidth{1.003750pt}%
\definecolor{currentstroke}{rgb}{1.000000,0.498039,0.054902}%
\pgfsetstrokecolor{currentstroke}%
\pgfsetdash{}{0pt}%
\pgfpathmoveto{\pgfqpoint{2.463757in}{1.870879in}}%
\pgfpathcurveto{\pgfqpoint{2.474807in}{1.870879in}}{\pgfqpoint{2.485406in}{1.875269in}}{\pgfqpoint{2.493220in}{1.883083in}}%
\pgfpathcurveto{\pgfqpoint{2.501034in}{1.890896in}}{\pgfqpoint{2.505424in}{1.901496in}}{\pgfqpoint{2.505424in}{1.912546in}}%
\pgfpathcurveto{\pgfqpoint{2.505424in}{1.923596in}}{\pgfqpoint{2.501034in}{1.934195in}}{\pgfqpoint{2.493220in}{1.942008in}}%
\pgfpathcurveto{\pgfqpoint{2.485406in}{1.949822in}}{\pgfqpoint{2.474807in}{1.954212in}}{\pgfqpoint{2.463757in}{1.954212in}}%
\pgfpathcurveto{\pgfqpoint{2.452707in}{1.954212in}}{\pgfqpoint{2.442108in}{1.949822in}}{\pgfqpoint{2.434294in}{1.942008in}}%
\pgfpathcurveto{\pgfqpoint{2.426481in}{1.934195in}}{\pgfqpoint{2.422091in}{1.923596in}}{\pgfqpoint{2.422091in}{1.912546in}}%
\pgfpathcurveto{\pgfqpoint{2.422091in}{1.901496in}}{\pgfqpoint{2.426481in}{1.890896in}}{\pgfqpoint{2.434294in}{1.883083in}}%
\pgfpathcurveto{\pgfqpoint{2.442108in}{1.875269in}}{\pgfqpoint{2.452707in}{1.870879in}}{\pgfqpoint{2.463757in}{1.870879in}}%
\pgfpathclose%
\pgfusepath{stroke,fill}%
\end{pgfscope}%
\begin{pgfscope}%
\pgfpathrectangle{\pgfqpoint{0.787074in}{0.548769in}}{\pgfqpoint{5.062926in}{3.102590in}}%
\pgfusepath{clip}%
\pgfsetbuttcap%
\pgfsetroundjoin%
\definecolor{currentfill}{rgb}{1.000000,0.498039,0.054902}%
\pgfsetfillcolor{currentfill}%
\pgfsetlinewidth{1.003750pt}%
\definecolor{currentstroke}{rgb}{1.000000,0.498039,0.054902}%
\pgfsetstrokecolor{currentstroke}%
\pgfsetdash{}{0pt}%
\pgfpathmoveto{\pgfqpoint{2.398005in}{1.599536in}}%
\pgfpathcurveto{\pgfqpoint{2.409055in}{1.599536in}}{\pgfqpoint{2.419654in}{1.603926in}}{\pgfqpoint{2.427468in}{1.611740in}}%
\pgfpathcurveto{\pgfqpoint{2.435281in}{1.619554in}}{\pgfqpoint{2.439672in}{1.630153in}}{\pgfqpoint{2.439672in}{1.641203in}}%
\pgfpathcurveto{\pgfqpoint{2.439672in}{1.652253in}}{\pgfqpoint{2.435281in}{1.662852in}}{\pgfqpoint{2.427468in}{1.670666in}}%
\pgfpathcurveto{\pgfqpoint{2.419654in}{1.678479in}}{\pgfqpoint{2.409055in}{1.682870in}}{\pgfqpoint{2.398005in}{1.682870in}}%
\pgfpathcurveto{\pgfqpoint{2.386955in}{1.682870in}}{\pgfqpoint{2.376356in}{1.678479in}}{\pgfqpoint{2.368542in}{1.670666in}}%
\pgfpathcurveto{\pgfqpoint{2.360728in}{1.662852in}}{\pgfqpoint{2.356338in}{1.652253in}}{\pgfqpoint{2.356338in}{1.641203in}}%
\pgfpathcurveto{\pgfqpoint{2.356338in}{1.630153in}}{\pgfqpoint{2.360728in}{1.619554in}}{\pgfqpoint{2.368542in}{1.611740in}}%
\pgfpathcurveto{\pgfqpoint{2.376356in}{1.603926in}}{\pgfqpoint{2.386955in}{1.599536in}}{\pgfqpoint{2.398005in}{1.599536in}}%
\pgfpathclose%
\pgfusepath{stroke,fill}%
\end{pgfscope}%
\begin{pgfscope}%
\pgfpathrectangle{\pgfqpoint{0.787074in}{0.548769in}}{\pgfqpoint{5.062926in}{3.102590in}}%
\pgfusepath{clip}%
\pgfsetbuttcap%
\pgfsetroundjoin%
\definecolor{currentfill}{rgb}{1.000000,0.498039,0.054902}%
\pgfsetfillcolor{currentfill}%
\pgfsetlinewidth{1.003750pt}%
\definecolor{currentstroke}{rgb}{1.000000,0.498039,0.054902}%
\pgfsetstrokecolor{currentstroke}%
\pgfsetdash{}{0pt}%
\pgfpathmoveto{\pgfqpoint{2.266500in}{1.739034in}}%
\pgfpathcurveto{\pgfqpoint{2.277550in}{1.739034in}}{\pgfqpoint{2.288149in}{1.743425in}}{\pgfqpoint{2.295963in}{1.751238in}}%
\pgfpathcurveto{\pgfqpoint{2.303777in}{1.759052in}}{\pgfqpoint{2.308167in}{1.769651in}}{\pgfqpoint{2.308167in}{1.780701in}}%
\pgfpathcurveto{\pgfqpoint{2.308167in}{1.791751in}}{\pgfqpoint{2.303777in}{1.802350in}}{\pgfqpoint{2.295963in}{1.810164in}}%
\pgfpathcurveto{\pgfqpoint{2.288149in}{1.817977in}}{\pgfqpoint{2.277550in}{1.822368in}}{\pgfqpoint{2.266500in}{1.822368in}}%
\pgfpathcurveto{\pgfqpoint{2.255450in}{1.822368in}}{\pgfqpoint{2.244851in}{1.817977in}}{\pgfqpoint{2.237038in}{1.810164in}}%
\pgfpathcurveto{\pgfqpoint{2.229224in}{1.802350in}}{\pgfqpoint{2.224834in}{1.791751in}}{\pgfqpoint{2.224834in}{1.780701in}}%
\pgfpathcurveto{\pgfqpoint{2.224834in}{1.769651in}}{\pgfqpoint{2.229224in}{1.759052in}}{\pgfqpoint{2.237038in}{1.751238in}}%
\pgfpathcurveto{\pgfqpoint{2.244851in}{1.743425in}}{\pgfqpoint{2.255450in}{1.739034in}}{\pgfqpoint{2.266500in}{1.739034in}}%
\pgfpathclose%
\pgfusepath{stroke,fill}%
\end{pgfscope}%
\begin{pgfscope}%
\pgfpathrectangle{\pgfqpoint{0.787074in}{0.548769in}}{\pgfqpoint{5.062926in}{3.102590in}}%
\pgfusepath{clip}%
\pgfsetbuttcap%
\pgfsetroundjoin%
\definecolor{currentfill}{rgb}{1.000000,0.498039,0.054902}%
\pgfsetfillcolor{currentfill}%
\pgfsetlinewidth{1.003750pt}%
\definecolor{currentstroke}{rgb}{1.000000,0.498039,0.054902}%
\pgfsetstrokecolor{currentstroke}%
\pgfsetdash{}{0pt}%
\pgfpathmoveto{\pgfqpoint{1.740482in}{2.553699in}}%
\pgfpathcurveto{\pgfqpoint{1.751532in}{2.553699in}}{\pgfqpoint{1.762131in}{2.558090in}}{\pgfqpoint{1.769945in}{2.565903in}}%
\pgfpathcurveto{\pgfqpoint{1.777758in}{2.573717in}}{\pgfqpoint{1.782149in}{2.584316in}}{\pgfqpoint{1.782149in}{2.595366in}}%
\pgfpathcurveto{\pgfqpoint{1.782149in}{2.606416in}}{\pgfqpoint{1.777758in}{2.617015in}}{\pgfqpoint{1.769945in}{2.624829in}}%
\pgfpathcurveto{\pgfqpoint{1.762131in}{2.632642in}}{\pgfqpoint{1.751532in}{2.637033in}}{\pgfqpoint{1.740482in}{2.637033in}}%
\pgfpathcurveto{\pgfqpoint{1.729432in}{2.637033in}}{\pgfqpoint{1.718833in}{2.632642in}}{\pgfqpoint{1.711019in}{2.624829in}}%
\pgfpathcurveto{\pgfqpoint{1.703206in}{2.617015in}}{\pgfqpoint{1.698815in}{2.606416in}}{\pgfqpoint{1.698815in}{2.595366in}}%
\pgfpathcurveto{\pgfqpoint{1.698815in}{2.584316in}}{\pgfqpoint{1.703206in}{2.573717in}}{\pgfqpoint{1.711019in}{2.565903in}}%
\pgfpathcurveto{\pgfqpoint{1.718833in}{2.558090in}}{\pgfqpoint{1.729432in}{2.553699in}}{\pgfqpoint{1.740482in}{2.553699in}}%
\pgfpathclose%
\pgfusepath{stroke,fill}%
\end{pgfscope}%
\begin{pgfscope}%
\pgfpathrectangle{\pgfqpoint{0.787074in}{0.548769in}}{\pgfqpoint{5.062926in}{3.102590in}}%
\pgfusepath{clip}%
\pgfsetbuttcap%
\pgfsetroundjoin%
\definecolor{currentfill}{rgb}{0.121569,0.466667,0.705882}%
\pgfsetfillcolor{currentfill}%
\pgfsetlinewidth{1.003750pt}%
\definecolor{currentstroke}{rgb}{0.121569,0.466667,0.705882}%
\pgfsetstrokecolor{currentstroke}%
\pgfsetdash{}{0pt}%
\pgfpathmoveto{\pgfqpoint{1.543225in}{2.326040in}}%
\pgfpathcurveto{\pgfqpoint{1.554275in}{2.326040in}}{\pgfqpoint{1.564874in}{2.330430in}}{\pgfqpoint{1.572688in}{2.338243in}}%
\pgfpathcurveto{\pgfqpoint{1.580502in}{2.346057in}}{\pgfqpoint{1.584892in}{2.356656in}}{\pgfqpoint{1.584892in}{2.367706in}}%
\pgfpathcurveto{\pgfqpoint{1.584892in}{2.378756in}}{\pgfqpoint{1.580502in}{2.389355in}}{\pgfqpoint{1.572688in}{2.397169in}}%
\pgfpathcurveto{\pgfqpoint{1.564874in}{2.404983in}}{\pgfqpoint{1.554275in}{2.409373in}}{\pgfqpoint{1.543225in}{2.409373in}}%
\pgfpathcurveto{\pgfqpoint{1.532175in}{2.409373in}}{\pgfqpoint{1.521576in}{2.404983in}}{\pgfqpoint{1.513762in}{2.397169in}}%
\pgfpathcurveto{\pgfqpoint{1.505949in}{2.389355in}}{\pgfqpoint{1.501558in}{2.378756in}}{\pgfqpoint{1.501558in}{2.367706in}}%
\pgfpathcurveto{\pgfqpoint{1.501558in}{2.356656in}}{\pgfqpoint{1.505949in}{2.346057in}}{\pgfqpoint{1.513762in}{2.338243in}}%
\pgfpathcurveto{\pgfqpoint{1.521576in}{2.330430in}}{\pgfqpoint{1.532175in}{2.326040in}}{\pgfqpoint{1.543225in}{2.326040in}}%
\pgfpathclose%
\pgfusepath{stroke,fill}%
\end{pgfscope}%
\begin{pgfscope}%
\pgfpathrectangle{\pgfqpoint{0.787074in}{0.548769in}}{\pgfqpoint{5.062926in}{3.102590in}}%
\pgfusepath{clip}%
\pgfsetbuttcap%
\pgfsetroundjoin%
\definecolor{currentfill}{rgb}{1.000000,0.498039,0.054902}%
\pgfsetfillcolor{currentfill}%
\pgfsetlinewidth{1.003750pt}%
\definecolor{currentstroke}{rgb}{1.000000,0.498039,0.054902}%
\pgfsetstrokecolor{currentstroke}%
\pgfsetdash{}{0pt}%
\pgfpathmoveto{\pgfqpoint{1.608977in}{2.769003in}}%
\pgfpathcurveto{\pgfqpoint{1.620028in}{2.769003in}}{\pgfqpoint{1.630627in}{2.773394in}}{\pgfqpoint{1.638440in}{2.781207in}}%
\pgfpathcurveto{\pgfqpoint{1.646254in}{2.789021in}}{\pgfqpoint{1.650644in}{2.799620in}}{\pgfqpoint{1.650644in}{2.810670in}}%
\pgfpathcurveto{\pgfqpoint{1.650644in}{2.821720in}}{\pgfqpoint{1.646254in}{2.832319in}}{\pgfqpoint{1.638440in}{2.840133in}}%
\pgfpathcurveto{\pgfqpoint{1.630627in}{2.847946in}}{\pgfqpoint{1.620028in}{2.852337in}}{\pgfqpoint{1.608977in}{2.852337in}}%
\pgfpathcurveto{\pgfqpoint{1.597927in}{2.852337in}}{\pgfqpoint{1.587328in}{2.847946in}}{\pgfqpoint{1.579515in}{2.840133in}}%
\pgfpathcurveto{\pgfqpoint{1.571701in}{2.832319in}}{\pgfqpoint{1.567311in}{2.821720in}}{\pgfqpoint{1.567311in}{2.810670in}}%
\pgfpathcurveto{\pgfqpoint{1.567311in}{2.799620in}}{\pgfqpoint{1.571701in}{2.789021in}}{\pgfqpoint{1.579515in}{2.781207in}}%
\pgfpathcurveto{\pgfqpoint{1.587328in}{2.773394in}}{\pgfqpoint{1.597927in}{2.769003in}}{\pgfqpoint{1.608977in}{2.769003in}}%
\pgfpathclose%
\pgfusepath{stroke,fill}%
\end{pgfscope}%
\begin{pgfscope}%
\pgfpathrectangle{\pgfqpoint{0.787074in}{0.548769in}}{\pgfqpoint{5.062926in}{3.102590in}}%
\pgfusepath{clip}%
\pgfsetbuttcap%
\pgfsetroundjoin%
\definecolor{currentfill}{rgb}{0.121569,0.466667,0.705882}%
\pgfsetfillcolor{currentfill}%
\pgfsetlinewidth{1.003750pt}%
\definecolor{currentstroke}{rgb}{0.121569,0.466667,0.705882}%
\pgfsetstrokecolor{currentstroke}%
\pgfsetdash{}{0pt}%
\pgfpathmoveto{\pgfqpoint{1.280216in}{0.659320in}}%
\pgfpathcurveto{\pgfqpoint{1.291266in}{0.659320in}}{\pgfqpoint{1.301865in}{0.663711in}}{\pgfqpoint{1.309679in}{0.671524in}}%
\pgfpathcurveto{\pgfqpoint{1.317492in}{0.679338in}}{\pgfqpoint{1.321883in}{0.689937in}}{\pgfqpoint{1.321883in}{0.700987in}}%
\pgfpathcurveto{\pgfqpoint{1.321883in}{0.712037in}}{\pgfqpoint{1.317492in}{0.722636in}}{\pgfqpoint{1.309679in}{0.730450in}}%
\pgfpathcurveto{\pgfqpoint{1.301865in}{0.738263in}}{\pgfqpoint{1.291266in}{0.742654in}}{\pgfqpoint{1.280216in}{0.742654in}}%
\pgfpathcurveto{\pgfqpoint{1.269166in}{0.742654in}}{\pgfqpoint{1.258567in}{0.738263in}}{\pgfqpoint{1.250753in}{0.730450in}}%
\pgfpathcurveto{\pgfqpoint{1.242940in}{0.722636in}}{\pgfqpoint{1.238549in}{0.712037in}}{\pgfqpoint{1.238549in}{0.700987in}}%
\pgfpathcurveto{\pgfqpoint{1.238549in}{0.689937in}}{\pgfqpoint{1.242940in}{0.679338in}}{\pgfqpoint{1.250753in}{0.671524in}}%
\pgfpathcurveto{\pgfqpoint{1.258567in}{0.663711in}}{\pgfqpoint{1.269166in}{0.659320in}}{\pgfqpoint{1.280216in}{0.659320in}}%
\pgfpathclose%
\pgfusepath{stroke,fill}%
\end{pgfscope}%
\begin{pgfscope}%
\pgfpathrectangle{\pgfqpoint{0.787074in}{0.548769in}}{\pgfqpoint{5.062926in}{3.102590in}}%
\pgfusepath{clip}%
\pgfsetbuttcap%
\pgfsetroundjoin%
\definecolor{currentfill}{rgb}{1.000000,0.498039,0.054902}%
\pgfsetfillcolor{currentfill}%
\pgfsetlinewidth{1.003750pt}%
\definecolor{currentstroke}{rgb}{1.000000,0.498039,0.054902}%
\pgfsetstrokecolor{currentstroke}%
\pgfsetdash{}{0pt}%
\pgfpathmoveto{\pgfqpoint{2.003491in}{1.946771in}}%
\pgfpathcurveto{\pgfqpoint{2.014541in}{1.946771in}}{\pgfqpoint{2.025140in}{1.951162in}}{\pgfqpoint{2.032954in}{1.958975in}}%
\pgfpathcurveto{\pgfqpoint{2.040768in}{1.966789in}}{\pgfqpoint{2.045158in}{1.977388in}}{\pgfqpoint{2.045158in}{1.988438in}}%
\pgfpathcurveto{\pgfqpoint{2.045158in}{1.999488in}}{\pgfqpoint{2.040768in}{2.010087in}}{\pgfqpoint{2.032954in}{2.017901in}}%
\pgfpathcurveto{\pgfqpoint{2.025140in}{2.025714in}}{\pgfqpoint{2.014541in}{2.030105in}}{\pgfqpoint{2.003491in}{2.030105in}}%
\pgfpathcurveto{\pgfqpoint{1.992441in}{2.030105in}}{\pgfqpoint{1.981842in}{2.025714in}}{\pgfqpoint{1.974028in}{2.017901in}}%
\pgfpathcurveto{\pgfqpoint{1.966215in}{2.010087in}}{\pgfqpoint{1.961824in}{1.999488in}}{\pgfqpoint{1.961824in}{1.988438in}}%
\pgfpathcurveto{\pgfqpoint{1.961824in}{1.977388in}}{\pgfqpoint{1.966215in}{1.966789in}}{\pgfqpoint{1.974028in}{1.958975in}}%
\pgfpathcurveto{\pgfqpoint{1.981842in}{1.951162in}}{\pgfqpoint{1.992441in}{1.946771in}}{\pgfqpoint{2.003491in}{1.946771in}}%
\pgfpathclose%
\pgfusepath{stroke,fill}%
\end{pgfscope}%
\begin{pgfscope}%
\pgfpathrectangle{\pgfqpoint{0.787074in}{0.548769in}}{\pgfqpoint{5.062926in}{3.102590in}}%
\pgfusepath{clip}%
\pgfsetbuttcap%
\pgfsetroundjoin%
\definecolor{currentfill}{rgb}{0.121569,0.466667,0.705882}%
\pgfsetfillcolor{currentfill}%
\pgfsetlinewidth{1.003750pt}%
\definecolor{currentstroke}{rgb}{0.121569,0.466667,0.705882}%
\pgfsetstrokecolor{currentstroke}%
\pgfsetdash{}{0pt}%
\pgfpathmoveto{\pgfqpoint{1.345968in}{0.648235in}}%
\pgfpathcurveto{\pgfqpoint{1.357018in}{0.648235in}}{\pgfqpoint{1.367617in}{0.652625in}}{\pgfqpoint{1.375431in}{0.660439in}}%
\pgfpathcurveto{\pgfqpoint{1.383245in}{0.668252in}}{\pgfqpoint{1.387635in}{0.678851in}}{\pgfqpoint{1.387635in}{0.689901in}}%
\pgfpathcurveto{\pgfqpoint{1.387635in}{0.700952in}}{\pgfqpoint{1.383245in}{0.711551in}}{\pgfqpoint{1.375431in}{0.719364in}}%
\pgfpathcurveto{\pgfqpoint{1.367617in}{0.727178in}}{\pgfqpoint{1.357018in}{0.731568in}}{\pgfqpoint{1.345968in}{0.731568in}}%
\pgfpathcurveto{\pgfqpoint{1.334918in}{0.731568in}}{\pgfqpoint{1.324319in}{0.727178in}}{\pgfqpoint{1.316506in}{0.719364in}}%
\pgfpathcurveto{\pgfqpoint{1.308692in}{0.711551in}}{\pgfqpoint{1.304302in}{0.700952in}}{\pgfqpoint{1.304302in}{0.689901in}}%
\pgfpathcurveto{\pgfqpoint{1.304302in}{0.678851in}}{\pgfqpoint{1.308692in}{0.668252in}}{\pgfqpoint{1.316506in}{0.660439in}}%
\pgfpathcurveto{\pgfqpoint{1.324319in}{0.652625in}}{\pgfqpoint{1.334918in}{0.648235in}}{\pgfqpoint{1.345968in}{0.648235in}}%
\pgfpathclose%
\pgfusepath{stroke,fill}%
\end{pgfscope}%
\begin{pgfscope}%
\pgfpathrectangle{\pgfqpoint{0.787074in}{0.548769in}}{\pgfqpoint{5.062926in}{3.102590in}}%
\pgfusepath{clip}%
\pgfsetbuttcap%
\pgfsetroundjoin%
\definecolor{currentfill}{rgb}{1.000000,0.498039,0.054902}%
\pgfsetfillcolor{currentfill}%
\pgfsetlinewidth{1.003750pt}%
\definecolor{currentstroke}{rgb}{1.000000,0.498039,0.054902}%
\pgfsetstrokecolor{currentstroke}%
\pgfsetdash{}{0pt}%
\pgfpathmoveto{\pgfqpoint{1.740482in}{3.081873in}}%
\pgfpathcurveto{\pgfqpoint{1.751532in}{3.081873in}}{\pgfqpoint{1.762131in}{3.086263in}}{\pgfqpoint{1.769945in}{3.094077in}}%
\pgfpathcurveto{\pgfqpoint{1.777758in}{3.101890in}}{\pgfqpoint{1.782149in}{3.112489in}}{\pgfqpoint{1.782149in}{3.123539in}}%
\pgfpathcurveto{\pgfqpoint{1.782149in}{3.134589in}}{\pgfqpoint{1.777758in}{3.145188in}}{\pgfqpoint{1.769945in}{3.153002in}}%
\pgfpathcurveto{\pgfqpoint{1.762131in}{3.160816in}}{\pgfqpoint{1.751532in}{3.165206in}}{\pgfqpoint{1.740482in}{3.165206in}}%
\pgfpathcurveto{\pgfqpoint{1.729432in}{3.165206in}}{\pgfqpoint{1.718833in}{3.160816in}}{\pgfqpoint{1.711019in}{3.153002in}}%
\pgfpathcurveto{\pgfqpoint{1.703206in}{3.145188in}}{\pgfqpoint{1.698815in}{3.134589in}}{\pgfqpoint{1.698815in}{3.123539in}}%
\pgfpathcurveto{\pgfqpoint{1.698815in}{3.112489in}}{\pgfqpoint{1.703206in}{3.101890in}}{\pgfqpoint{1.711019in}{3.094077in}}%
\pgfpathcurveto{\pgfqpoint{1.718833in}{3.086263in}}{\pgfqpoint{1.729432in}{3.081873in}}{\pgfqpoint{1.740482in}{3.081873in}}%
\pgfpathclose%
\pgfusepath{stroke,fill}%
\end{pgfscope}%
\begin{pgfscope}%
\pgfpathrectangle{\pgfqpoint{0.787074in}{0.548769in}}{\pgfqpoint{5.062926in}{3.102590in}}%
\pgfusepath{clip}%
\pgfsetbuttcap%
\pgfsetroundjoin%
\definecolor{currentfill}{rgb}{1.000000,0.498039,0.054902}%
\pgfsetfillcolor{currentfill}%
\pgfsetlinewidth{1.003750pt}%
\definecolor{currentstroke}{rgb}{1.000000,0.498039,0.054902}%
\pgfsetstrokecolor{currentstroke}%
\pgfsetdash{}{0pt}%
\pgfpathmoveto{\pgfqpoint{1.740482in}{2.902728in}}%
\pgfpathcurveto{\pgfqpoint{1.751532in}{2.902728in}}{\pgfqpoint{1.762131in}{2.907118in}}{\pgfqpoint{1.769945in}{2.914932in}}%
\pgfpathcurveto{\pgfqpoint{1.777758in}{2.922746in}}{\pgfqpoint{1.782149in}{2.933345in}}{\pgfqpoint{1.782149in}{2.944395in}}%
\pgfpathcurveto{\pgfqpoint{1.782149in}{2.955445in}}{\pgfqpoint{1.777758in}{2.966044in}}{\pgfqpoint{1.769945in}{2.973858in}}%
\pgfpathcurveto{\pgfqpoint{1.762131in}{2.981671in}}{\pgfqpoint{1.751532in}{2.986061in}}{\pgfqpoint{1.740482in}{2.986061in}}%
\pgfpathcurveto{\pgfqpoint{1.729432in}{2.986061in}}{\pgfqpoint{1.718833in}{2.981671in}}{\pgfqpoint{1.711019in}{2.973858in}}%
\pgfpathcurveto{\pgfqpoint{1.703206in}{2.966044in}}{\pgfqpoint{1.698815in}{2.955445in}}{\pgfqpoint{1.698815in}{2.944395in}}%
\pgfpathcurveto{\pgfqpoint{1.698815in}{2.933345in}}{\pgfqpoint{1.703206in}{2.922746in}}{\pgfqpoint{1.711019in}{2.914932in}}%
\pgfpathcurveto{\pgfqpoint{1.718833in}{2.907118in}}{\pgfqpoint{1.729432in}{2.902728in}}{\pgfqpoint{1.740482in}{2.902728in}}%
\pgfpathclose%
\pgfusepath{stroke,fill}%
\end{pgfscope}%
\begin{pgfscope}%
\pgfpathrectangle{\pgfqpoint{0.787074in}{0.548769in}}{\pgfqpoint{5.062926in}{3.102590in}}%
\pgfusepath{clip}%
\pgfsetbuttcap%
\pgfsetroundjoin%
\definecolor{currentfill}{rgb}{1.000000,0.498039,0.054902}%
\pgfsetfillcolor{currentfill}%
\pgfsetlinewidth{1.003750pt}%
\definecolor{currentstroke}{rgb}{1.000000,0.498039,0.054902}%
\pgfsetstrokecolor{currentstroke}%
\pgfsetdash{}{0pt}%
\pgfpathmoveto{\pgfqpoint{1.740482in}{2.766568in}}%
\pgfpathcurveto{\pgfqpoint{1.751532in}{2.766568in}}{\pgfqpoint{1.762131in}{2.770959in}}{\pgfqpoint{1.769945in}{2.778772in}}%
\pgfpathcurveto{\pgfqpoint{1.777758in}{2.786586in}}{\pgfqpoint{1.782149in}{2.797185in}}{\pgfqpoint{1.782149in}{2.808235in}}%
\pgfpathcurveto{\pgfqpoint{1.782149in}{2.819285in}}{\pgfqpoint{1.777758in}{2.829884in}}{\pgfqpoint{1.769945in}{2.837698in}}%
\pgfpathcurveto{\pgfqpoint{1.762131in}{2.845511in}}{\pgfqpoint{1.751532in}{2.849902in}}{\pgfqpoint{1.740482in}{2.849902in}}%
\pgfpathcurveto{\pgfqpoint{1.729432in}{2.849902in}}{\pgfqpoint{1.718833in}{2.845511in}}{\pgfqpoint{1.711019in}{2.837698in}}%
\pgfpathcurveto{\pgfqpoint{1.703206in}{2.829884in}}{\pgfqpoint{1.698815in}{2.819285in}}{\pgfqpoint{1.698815in}{2.808235in}}%
\pgfpathcurveto{\pgfqpoint{1.698815in}{2.797185in}}{\pgfqpoint{1.703206in}{2.786586in}}{\pgfqpoint{1.711019in}{2.778772in}}%
\pgfpathcurveto{\pgfqpoint{1.718833in}{2.770959in}}{\pgfqpoint{1.729432in}{2.766568in}}{\pgfqpoint{1.740482in}{2.766568in}}%
\pgfpathclose%
\pgfusepath{stroke,fill}%
\end{pgfscope}%
\begin{pgfscope}%
\pgfpathrectangle{\pgfqpoint{0.787074in}{0.548769in}}{\pgfqpoint{5.062926in}{3.102590in}}%
\pgfusepath{clip}%
\pgfsetbuttcap%
\pgfsetroundjoin%
\definecolor{currentfill}{rgb}{1.000000,0.498039,0.054902}%
\pgfsetfillcolor{currentfill}%
\pgfsetlinewidth{1.003750pt}%
\definecolor{currentstroke}{rgb}{1.000000,0.498039,0.054902}%
\pgfsetstrokecolor{currentstroke}%
\pgfsetdash{}{0pt}%
\pgfpathmoveto{\pgfqpoint{1.674730in}{1.775111in}}%
\pgfpathcurveto{\pgfqpoint{1.685780in}{1.775111in}}{\pgfqpoint{1.696379in}{1.779501in}}{\pgfqpoint{1.704193in}{1.787315in}}%
\pgfpathcurveto{\pgfqpoint{1.712006in}{1.795129in}}{\pgfqpoint{1.716396in}{1.805728in}}{\pgfqpoint{1.716396in}{1.816778in}}%
\pgfpathcurveto{\pgfqpoint{1.716396in}{1.827828in}}{\pgfqpoint{1.712006in}{1.838427in}}{\pgfqpoint{1.704193in}{1.846241in}}%
\pgfpathcurveto{\pgfqpoint{1.696379in}{1.854054in}}{\pgfqpoint{1.685780in}{1.858444in}}{\pgfqpoint{1.674730in}{1.858444in}}%
\pgfpathcurveto{\pgfqpoint{1.663680in}{1.858444in}}{\pgfqpoint{1.653081in}{1.854054in}}{\pgfqpoint{1.645267in}{1.846241in}}%
\pgfpathcurveto{\pgfqpoint{1.637453in}{1.838427in}}{\pgfqpoint{1.633063in}{1.827828in}}{\pgfqpoint{1.633063in}{1.816778in}}%
\pgfpathcurveto{\pgfqpoint{1.633063in}{1.805728in}}{\pgfqpoint{1.637453in}{1.795129in}}{\pgfqpoint{1.645267in}{1.787315in}}%
\pgfpathcurveto{\pgfqpoint{1.653081in}{1.779501in}}{\pgfqpoint{1.663680in}{1.775111in}}{\pgfqpoint{1.674730in}{1.775111in}}%
\pgfpathclose%
\pgfusepath{stroke,fill}%
\end{pgfscope}%
\begin{pgfscope}%
\pgfpathrectangle{\pgfqpoint{0.787074in}{0.548769in}}{\pgfqpoint{5.062926in}{3.102590in}}%
\pgfusepath{clip}%
\pgfsetbuttcap%
\pgfsetroundjoin%
\definecolor{currentfill}{rgb}{1.000000,0.498039,0.054902}%
\pgfsetfillcolor{currentfill}%
\pgfsetlinewidth{1.003750pt}%
\definecolor{currentstroke}{rgb}{1.000000,0.498039,0.054902}%
\pgfsetstrokecolor{currentstroke}%
\pgfsetdash{}{0pt}%
\pgfpathmoveto{\pgfqpoint{1.937739in}{1.545971in}}%
\pgfpathcurveto{\pgfqpoint{1.948789in}{1.545971in}}{\pgfqpoint{1.959388in}{1.550361in}}{\pgfqpoint{1.967202in}{1.558175in}}%
\pgfpathcurveto{\pgfqpoint{1.975015in}{1.565989in}}{\pgfqpoint{1.979406in}{1.576588in}}{\pgfqpoint{1.979406in}{1.587638in}}%
\pgfpathcurveto{\pgfqpoint{1.979406in}{1.598688in}}{\pgfqpoint{1.975015in}{1.609287in}}{\pgfqpoint{1.967202in}{1.617101in}}%
\pgfpathcurveto{\pgfqpoint{1.959388in}{1.624914in}}{\pgfqpoint{1.948789in}{1.629305in}}{\pgfqpoint{1.937739in}{1.629305in}}%
\pgfpathcurveto{\pgfqpoint{1.926689in}{1.629305in}}{\pgfqpoint{1.916090in}{1.624914in}}{\pgfqpoint{1.908276in}{1.617101in}}%
\pgfpathcurveto{\pgfqpoint{1.900462in}{1.609287in}}{\pgfqpoint{1.896072in}{1.598688in}}{\pgfqpoint{1.896072in}{1.587638in}}%
\pgfpathcurveto{\pgfqpoint{1.896072in}{1.576588in}}{\pgfqpoint{1.900462in}{1.565989in}}{\pgfqpoint{1.908276in}{1.558175in}}%
\pgfpathcurveto{\pgfqpoint{1.916090in}{1.550361in}}{\pgfqpoint{1.926689in}{1.545971in}}{\pgfqpoint{1.937739in}{1.545971in}}%
\pgfpathclose%
\pgfusepath{stroke,fill}%
\end{pgfscope}%
\begin{pgfscope}%
\pgfpathrectangle{\pgfqpoint{0.787074in}{0.548769in}}{\pgfqpoint{5.062926in}{3.102590in}}%
\pgfusepath{clip}%
\pgfsetbuttcap%
\pgfsetroundjoin%
\definecolor{currentfill}{rgb}{1.000000,0.498039,0.054902}%
\pgfsetfillcolor{currentfill}%
\pgfsetlinewidth{1.003750pt}%
\definecolor{currentstroke}{rgb}{1.000000,0.498039,0.054902}%
\pgfsetstrokecolor{currentstroke}%
\pgfsetdash{}{0pt}%
\pgfpathmoveto{\pgfqpoint{1.608977in}{2.232840in}}%
\pgfpathcurveto{\pgfqpoint{1.620028in}{2.232840in}}{\pgfqpoint{1.630627in}{2.237230in}}{\pgfqpoint{1.638440in}{2.245044in}}%
\pgfpathcurveto{\pgfqpoint{1.646254in}{2.252857in}}{\pgfqpoint{1.650644in}{2.263456in}}{\pgfqpoint{1.650644in}{2.274506in}}%
\pgfpathcurveto{\pgfqpoint{1.650644in}{2.285556in}}{\pgfqpoint{1.646254in}{2.296156in}}{\pgfqpoint{1.638440in}{2.303969in}}%
\pgfpathcurveto{\pgfqpoint{1.630627in}{2.311783in}}{\pgfqpoint{1.620028in}{2.316173in}}{\pgfqpoint{1.608977in}{2.316173in}}%
\pgfpathcurveto{\pgfqpoint{1.597927in}{2.316173in}}{\pgfqpoint{1.587328in}{2.311783in}}{\pgfqpoint{1.579515in}{2.303969in}}%
\pgfpathcurveto{\pgfqpoint{1.571701in}{2.296156in}}{\pgfqpoint{1.567311in}{2.285556in}}{\pgfqpoint{1.567311in}{2.274506in}}%
\pgfpathcurveto{\pgfqpoint{1.567311in}{2.263456in}}{\pgfqpoint{1.571701in}{2.252857in}}{\pgfqpoint{1.579515in}{2.245044in}}%
\pgfpathcurveto{\pgfqpoint{1.587328in}{2.237230in}}{\pgfqpoint{1.597927in}{2.232840in}}{\pgfqpoint{1.608977in}{2.232840in}}%
\pgfpathclose%
\pgfusepath{stroke,fill}%
\end{pgfscope}%
\begin{pgfscope}%
\pgfpathrectangle{\pgfqpoint{0.787074in}{0.548769in}}{\pgfqpoint{5.062926in}{3.102590in}}%
\pgfusepath{clip}%
\pgfsetbuttcap%
\pgfsetroundjoin%
\definecolor{currentfill}{rgb}{0.121569,0.466667,0.705882}%
\pgfsetfillcolor{currentfill}%
\pgfsetlinewidth{1.003750pt}%
\definecolor{currentstroke}{rgb}{0.121569,0.466667,0.705882}%
\pgfsetstrokecolor{currentstroke}%
\pgfsetdash{}{0pt}%
\pgfpathmoveto{\pgfqpoint{1.477473in}{0.676149in}}%
\pgfpathcurveto{\pgfqpoint{1.488523in}{0.676149in}}{\pgfqpoint{1.499122in}{0.680539in}}{\pgfqpoint{1.506936in}{0.688353in}}%
\pgfpathcurveto{\pgfqpoint{1.514749in}{0.696167in}}{\pgfqpoint{1.519140in}{0.706766in}}{\pgfqpoint{1.519140in}{0.717816in}}%
\pgfpathcurveto{\pgfqpoint{1.519140in}{0.728866in}}{\pgfqpoint{1.514749in}{0.739465in}}{\pgfqpoint{1.506936in}{0.747279in}}%
\pgfpathcurveto{\pgfqpoint{1.499122in}{0.755092in}}{\pgfqpoint{1.488523in}{0.759482in}}{\pgfqpoint{1.477473in}{0.759482in}}%
\pgfpathcurveto{\pgfqpoint{1.466423in}{0.759482in}}{\pgfqpoint{1.455824in}{0.755092in}}{\pgfqpoint{1.448010in}{0.747279in}}%
\pgfpathcurveto{\pgfqpoint{1.440196in}{0.739465in}}{\pgfqpoint{1.435806in}{0.728866in}}{\pgfqpoint{1.435806in}{0.717816in}}%
\pgfpathcurveto{\pgfqpoint{1.435806in}{0.706766in}}{\pgfqpoint{1.440196in}{0.696167in}}{\pgfqpoint{1.448010in}{0.688353in}}%
\pgfpathcurveto{\pgfqpoint{1.455824in}{0.680539in}}{\pgfqpoint{1.466423in}{0.676149in}}{\pgfqpoint{1.477473in}{0.676149in}}%
\pgfpathclose%
\pgfusepath{stroke,fill}%
\end{pgfscope}%
\begin{pgfscope}%
\pgfpathrectangle{\pgfqpoint{0.787074in}{0.548769in}}{\pgfqpoint{5.062926in}{3.102590in}}%
\pgfusepath{clip}%
\pgfsetbuttcap%
\pgfsetroundjoin%
\definecolor{currentfill}{rgb}{1.000000,0.498039,0.054902}%
\pgfsetfillcolor{currentfill}%
\pgfsetlinewidth{1.003750pt}%
\definecolor{currentstroke}{rgb}{1.000000,0.498039,0.054902}%
\pgfsetstrokecolor{currentstroke}%
\pgfsetdash{}{0pt}%
\pgfpathmoveto{\pgfqpoint{1.477473in}{1.907282in}}%
\pgfpathcurveto{\pgfqpoint{1.488523in}{1.907282in}}{\pgfqpoint{1.499122in}{1.911673in}}{\pgfqpoint{1.506936in}{1.919486in}}%
\pgfpathcurveto{\pgfqpoint{1.514749in}{1.927300in}}{\pgfqpoint{1.519140in}{1.937899in}}{\pgfqpoint{1.519140in}{1.948949in}}%
\pgfpathcurveto{\pgfqpoint{1.519140in}{1.959999in}}{\pgfqpoint{1.514749in}{1.970598in}}{\pgfqpoint{1.506936in}{1.978412in}}%
\pgfpathcurveto{\pgfqpoint{1.499122in}{1.986225in}}{\pgfqpoint{1.488523in}{1.990616in}}{\pgfqpoint{1.477473in}{1.990616in}}%
\pgfpathcurveto{\pgfqpoint{1.466423in}{1.990616in}}{\pgfqpoint{1.455824in}{1.986225in}}{\pgfqpoint{1.448010in}{1.978412in}}%
\pgfpathcurveto{\pgfqpoint{1.440196in}{1.970598in}}{\pgfqpoint{1.435806in}{1.959999in}}{\pgfqpoint{1.435806in}{1.948949in}}%
\pgfpathcurveto{\pgfqpoint{1.435806in}{1.937899in}}{\pgfqpoint{1.440196in}{1.927300in}}{\pgfqpoint{1.448010in}{1.919486in}}%
\pgfpathcurveto{\pgfqpoint{1.455824in}{1.911673in}}{\pgfqpoint{1.466423in}{1.907282in}}{\pgfqpoint{1.477473in}{1.907282in}}%
\pgfpathclose%
\pgfusepath{stroke,fill}%
\end{pgfscope}%
\begin{pgfscope}%
\pgfpathrectangle{\pgfqpoint{0.787074in}{0.548769in}}{\pgfqpoint{5.062926in}{3.102590in}}%
\pgfusepath{clip}%
\pgfsetbuttcap%
\pgfsetroundjoin%
\definecolor{currentfill}{rgb}{1.000000,0.498039,0.054902}%
\pgfsetfillcolor{currentfill}%
\pgfsetlinewidth{1.003750pt}%
\definecolor{currentstroke}{rgb}{1.000000,0.498039,0.054902}%
\pgfsetstrokecolor{currentstroke}%
\pgfsetdash{}{0pt}%
\pgfpathmoveto{\pgfqpoint{1.411721in}{2.839399in}}%
\pgfpathcurveto{\pgfqpoint{1.422771in}{2.839399in}}{\pgfqpoint{1.433370in}{2.843789in}}{\pgfqpoint{1.441183in}{2.851603in}}%
\pgfpathcurveto{\pgfqpoint{1.448997in}{2.859416in}}{\pgfqpoint{1.453387in}{2.870015in}}{\pgfqpoint{1.453387in}{2.881065in}}%
\pgfpathcurveto{\pgfqpoint{1.453387in}{2.892115in}}{\pgfqpoint{1.448997in}{2.902715in}}{\pgfqpoint{1.441183in}{2.910528in}}%
\pgfpathcurveto{\pgfqpoint{1.433370in}{2.918342in}}{\pgfqpoint{1.422771in}{2.922732in}}{\pgfqpoint{1.411721in}{2.922732in}}%
\pgfpathcurveto{\pgfqpoint{1.400670in}{2.922732in}}{\pgfqpoint{1.390071in}{2.918342in}}{\pgfqpoint{1.382258in}{2.910528in}}%
\pgfpathcurveto{\pgfqpoint{1.374444in}{2.902715in}}{\pgfqpoint{1.370054in}{2.892115in}}{\pgfqpoint{1.370054in}{2.881065in}}%
\pgfpathcurveto{\pgfqpoint{1.370054in}{2.870015in}}{\pgfqpoint{1.374444in}{2.859416in}}{\pgfqpoint{1.382258in}{2.851603in}}%
\pgfpathcurveto{\pgfqpoint{1.390071in}{2.843789in}}{\pgfqpoint{1.400670in}{2.839399in}}{\pgfqpoint{1.411721in}{2.839399in}}%
\pgfpathclose%
\pgfusepath{stroke,fill}%
\end{pgfscope}%
\begin{pgfscope}%
\pgfpathrectangle{\pgfqpoint{0.787074in}{0.548769in}}{\pgfqpoint{5.062926in}{3.102590in}}%
\pgfusepath{clip}%
\pgfsetbuttcap%
\pgfsetroundjoin%
\definecolor{currentfill}{rgb}{0.121569,0.466667,0.705882}%
\pgfsetfillcolor{currentfill}%
\pgfsetlinewidth{1.003750pt}%
\definecolor{currentstroke}{rgb}{0.121569,0.466667,0.705882}%
\pgfsetstrokecolor{currentstroke}%
\pgfsetdash{}{0pt}%
\pgfpathmoveto{\pgfqpoint{1.345968in}{0.648138in}}%
\pgfpathcurveto{\pgfqpoint{1.357018in}{0.648138in}}{\pgfqpoint{1.367617in}{0.652528in}}{\pgfqpoint{1.375431in}{0.660342in}}%
\pgfpathcurveto{\pgfqpoint{1.383245in}{0.668156in}}{\pgfqpoint{1.387635in}{0.678755in}}{\pgfqpoint{1.387635in}{0.689805in}}%
\pgfpathcurveto{\pgfqpoint{1.387635in}{0.700855in}}{\pgfqpoint{1.383245in}{0.711454in}}{\pgfqpoint{1.375431in}{0.719268in}}%
\pgfpathcurveto{\pgfqpoint{1.367617in}{0.727081in}}{\pgfqpoint{1.357018in}{0.731472in}}{\pgfqpoint{1.345968in}{0.731472in}}%
\pgfpathcurveto{\pgfqpoint{1.334918in}{0.731472in}}{\pgfqpoint{1.324319in}{0.727081in}}{\pgfqpoint{1.316506in}{0.719268in}}%
\pgfpathcurveto{\pgfqpoint{1.308692in}{0.711454in}}{\pgfqpoint{1.304302in}{0.700855in}}{\pgfqpoint{1.304302in}{0.689805in}}%
\pgfpathcurveto{\pgfqpoint{1.304302in}{0.678755in}}{\pgfqpoint{1.308692in}{0.668156in}}{\pgfqpoint{1.316506in}{0.660342in}}%
\pgfpathcurveto{\pgfqpoint{1.324319in}{0.652528in}}{\pgfqpoint{1.334918in}{0.648138in}}{\pgfqpoint{1.345968in}{0.648138in}}%
\pgfpathclose%
\pgfusepath{stroke,fill}%
\end{pgfscope}%
\begin{pgfscope}%
\pgfpathrectangle{\pgfqpoint{0.787074in}{0.548769in}}{\pgfqpoint{5.062926in}{3.102590in}}%
\pgfusepath{clip}%
\pgfsetbuttcap%
\pgfsetroundjoin%
\definecolor{currentfill}{rgb}{0.121569,0.466667,0.705882}%
\pgfsetfillcolor{currentfill}%
\pgfsetlinewidth{1.003750pt}%
\definecolor{currentstroke}{rgb}{0.121569,0.466667,0.705882}%
\pgfsetstrokecolor{currentstroke}%
\pgfsetdash{}{0pt}%
\pgfpathmoveto{\pgfqpoint{1.477473in}{0.649828in}}%
\pgfpathcurveto{\pgfqpoint{1.488523in}{0.649828in}}{\pgfqpoint{1.499122in}{0.654218in}}{\pgfqpoint{1.506936in}{0.662032in}}%
\pgfpathcurveto{\pgfqpoint{1.514749in}{0.669845in}}{\pgfqpoint{1.519140in}{0.680444in}}{\pgfqpoint{1.519140in}{0.691495in}}%
\pgfpathcurveto{\pgfqpoint{1.519140in}{0.702545in}}{\pgfqpoint{1.514749in}{0.713144in}}{\pgfqpoint{1.506936in}{0.720957in}}%
\pgfpathcurveto{\pgfqpoint{1.499122in}{0.728771in}}{\pgfqpoint{1.488523in}{0.733161in}}{\pgfqpoint{1.477473in}{0.733161in}}%
\pgfpathcurveto{\pgfqpoint{1.466423in}{0.733161in}}{\pgfqpoint{1.455824in}{0.728771in}}{\pgfqpoint{1.448010in}{0.720957in}}%
\pgfpathcurveto{\pgfqpoint{1.440196in}{0.713144in}}{\pgfqpoint{1.435806in}{0.702545in}}{\pgfqpoint{1.435806in}{0.691495in}}%
\pgfpathcurveto{\pgfqpoint{1.435806in}{0.680444in}}{\pgfqpoint{1.440196in}{0.669845in}}{\pgfqpoint{1.448010in}{0.662032in}}%
\pgfpathcurveto{\pgfqpoint{1.455824in}{0.654218in}}{\pgfqpoint{1.466423in}{0.649828in}}{\pgfqpoint{1.477473in}{0.649828in}}%
\pgfpathclose%
\pgfusepath{stroke,fill}%
\end{pgfscope}%
\begin{pgfscope}%
\pgfpathrectangle{\pgfqpoint{0.787074in}{0.548769in}}{\pgfqpoint{5.062926in}{3.102590in}}%
\pgfusepath{clip}%
\pgfsetbuttcap%
\pgfsetroundjoin%
\definecolor{currentfill}{rgb}{1.000000,0.498039,0.054902}%
\pgfsetfillcolor{currentfill}%
\pgfsetlinewidth{1.003750pt}%
\definecolor{currentstroke}{rgb}{1.000000,0.498039,0.054902}%
\pgfsetstrokecolor{currentstroke}%
\pgfsetdash{}{0pt}%
\pgfpathmoveto{\pgfqpoint{1.937739in}{2.575639in}}%
\pgfpathcurveto{\pgfqpoint{1.948789in}{2.575639in}}{\pgfqpoint{1.959388in}{2.580029in}}{\pgfqpoint{1.967202in}{2.587842in}}%
\pgfpathcurveto{\pgfqpoint{1.975015in}{2.595656in}}{\pgfqpoint{1.979406in}{2.606255in}}{\pgfqpoint{1.979406in}{2.617305in}}%
\pgfpathcurveto{\pgfqpoint{1.979406in}{2.628355in}}{\pgfqpoint{1.975015in}{2.638954in}}{\pgfqpoint{1.967202in}{2.646768in}}%
\pgfpathcurveto{\pgfqpoint{1.959388in}{2.654582in}}{\pgfqpoint{1.948789in}{2.658972in}}{\pgfqpoint{1.937739in}{2.658972in}}%
\pgfpathcurveto{\pgfqpoint{1.926689in}{2.658972in}}{\pgfqpoint{1.916090in}{2.654582in}}{\pgfqpoint{1.908276in}{2.646768in}}%
\pgfpathcurveto{\pgfqpoint{1.900462in}{2.638954in}}{\pgfqpoint{1.896072in}{2.628355in}}{\pgfqpoint{1.896072in}{2.617305in}}%
\pgfpathcurveto{\pgfqpoint{1.896072in}{2.606255in}}{\pgfqpoint{1.900462in}{2.595656in}}{\pgfqpoint{1.908276in}{2.587842in}}%
\pgfpathcurveto{\pgfqpoint{1.916090in}{2.580029in}}{\pgfqpoint{1.926689in}{2.575639in}}{\pgfqpoint{1.937739in}{2.575639in}}%
\pgfpathclose%
\pgfusepath{stroke,fill}%
\end{pgfscope}%
\begin{pgfscope}%
\pgfpathrectangle{\pgfqpoint{0.787074in}{0.548769in}}{\pgfqpoint{5.062926in}{3.102590in}}%
\pgfusepath{clip}%
\pgfsetbuttcap%
\pgfsetroundjoin%
\definecolor{currentfill}{rgb}{1.000000,0.498039,0.054902}%
\pgfsetfillcolor{currentfill}%
\pgfsetlinewidth{1.003750pt}%
\definecolor{currentstroke}{rgb}{1.000000,0.498039,0.054902}%
\pgfsetstrokecolor{currentstroke}%
\pgfsetdash{}{0pt}%
\pgfpathmoveto{\pgfqpoint{2.003491in}{2.218400in}}%
\pgfpathcurveto{\pgfqpoint{2.014541in}{2.218400in}}{\pgfqpoint{2.025140in}{2.222790in}}{\pgfqpoint{2.032954in}{2.230604in}}%
\pgfpathcurveto{\pgfqpoint{2.040768in}{2.238417in}}{\pgfqpoint{2.045158in}{2.249016in}}{\pgfqpoint{2.045158in}{2.260066in}}%
\pgfpathcurveto{\pgfqpoint{2.045158in}{2.271116in}}{\pgfqpoint{2.040768in}{2.281715in}}{\pgfqpoint{2.032954in}{2.289529in}}%
\pgfpathcurveto{\pgfqpoint{2.025140in}{2.297343in}}{\pgfqpoint{2.014541in}{2.301733in}}{\pgfqpoint{2.003491in}{2.301733in}}%
\pgfpathcurveto{\pgfqpoint{1.992441in}{2.301733in}}{\pgfqpoint{1.981842in}{2.297343in}}{\pgfqpoint{1.974028in}{2.289529in}}%
\pgfpathcurveto{\pgfqpoint{1.966215in}{2.281715in}}{\pgfqpoint{1.961824in}{2.271116in}}{\pgfqpoint{1.961824in}{2.260066in}}%
\pgfpathcurveto{\pgfqpoint{1.961824in}{2.249016in}}{\pgfqpoint{1.966215in}{2.238417in}}{\pgfqpoint{1.974028in}{2.230604in}}%
\pgfpathcurveto{\pgfqpoint{1.981842in}{2.222790in}}{\pgfqpoint{1.992441in}{2.218400in}}{\pgfqpoint{2.003491in}{2.218400in}}%
\pgfpathclose%
\pgfusepath{stroke,fill}%
\end{pgfscope}%
\begin{pgfscope}%
\pgfpathrectangle{\pgfqpoint{0.787074in}{0.548769in}}{\pgfqpoint{5.062926in}{3.102590in}}%
\pgfusepath{clip}%
\pgfsetbuttcap%
\pgfsetroundjoin%
\definecolor{currentfill}{rgb}{1.000000,0.498039,0.054902}%
\pgfsetfillcolor{currentfill}%
\pgfsetlinewidth{1.003750pt}%
\definecolor{currentstroke}{rgb}{1.000000,0.498039,0.054902}%
\pgfsetstrokecolor{currentstroke}%
\pgfsetdash{}{0pt}%
\pgfpathmoveto{\pgfqpoint{2.003491in}{2.443019in}}%
\pgfpathcurveto{\pgfqpoint{2.014541in}{2.443019in}}{\pgfqpoint{2.025140in}{2.447409in}}{\pgfqpoint{2.032954in}{2.455223in}}%
\pgfpathcurveto{\pgfqpoint{2.040768in}{2.463037in}}{\pgfqpoint{2.045158in}{2.473636in}}{\pgfqpoint{2.045158in}{2.484686in}}%
\pgfpathcurveto{\pgfqpoint{2.045158in}{2.495736in}}{\pgfqpoint{2.040768in}{2.506335in}}{\pgfqpoint{2.032954in}{2.514149in}}%
\pgfpathcurveto{\pgfqpoint{2.025140in}{2.521962in}}{\pgfqpoint{2.014541in}{2.526352in}}{\pgfqpoint{2.003491in}{2.526352in}}%
\pgfpathcurveto{\pgfqpoint{1.992441in}{2.526352in}}{\pgfqpoint{1.981842in}{2.521962in}}{\pgfqpoint{1.974028in}{2.514149in}}%
\pgfpathcurveto{\pgfqpoint{1.966215in}{2.506335in}}{\pgfqpoint{1.961824in}{2.495736in}}{\pgfqpoint{1.961824in}{2.484686in}}%
\pgfpathcurveto{\pgfqpoint{1.961824in}{2.473636in}}{\pgfqpoint{1.966215in}{2.463037in}}{\pgfqpoint{1.974028in}{2.455223in}}%
\pgfpathcurveto{\pgfqpoint{1.981842in}{2.447409in}}{\pgfqpoint{1.992441in}{2.443019in}}{\pgfqpoint{2.003491in}{2.443019in}}%
\pgfpathclose%
\pgfusepath{stroke,fill}%
\end{pgfscope}%
\begin{pgfscope}%
\pgfpathrectangle{\pgfqpoint{0.787074in}{0.548769in}}{\pgfqpoint{5.062926in}{3.102590in}}%
\pgfusepath{clip}%
\pgfsetbuttcap%
\pgfsetroundjoin%
\definecolor{currentfill}{rgb}{1.000000,0.498039,0.054902}%
\pgfsetfillcolor{currentfill}%
\pgfsetlinewidth{1.003750pt}%
\definecolor{currentstroke}{rgb}{1.000000,0.498039,0.054902}%
\pgfsetstrokecolor{currentstroke}%
\pgfsetdash{}{0pt}%
\pgfpathmoveto{\pgfqpoint{2.003491in}{1.631630in}}%
\pgfpathcurveto{\pgfqpoint{2.014541in}{1.631630in}}{\pgfqpoint{2.025140in}{1.636020in}}{\pgfqpoint{2.032954in}{1.643834in}}%
\pgfpathcurveto{\pgfqpoint{2.040768in}{1.651647in}}{\pgfqpoint{2.045158in}{1.662246in}}{\pgfqpoint{2.045158in}{1.673296in}}%
\pgfpathcurveto{\pgfqpoint{2.045158in}{1.684346in}}{\pgfqpoint{2.040768in}{1.694945in}}{\pgfqpoint{2.032954in}{1.702759in}}%
\pgfpathcurveto{\pgfqpoint{2.025140in}{1.710573in}}{\pgfqpoint{2.014541in}{1.714963in}}{\pgfqpoint{2.003491in}{1.714963in}}%
\pgfpathcurveto{\pgfqpoint{1.992441in}{1.714963in}}{\pgfqpoint{1.981842in}{1.710573in}}{\pgfqpoint{1.974028in}{1.702759in}}%
\pgfpathcurveto{\pgfqpoint{1.966215in}{1.694945in}}{\pgfqpoint{1.961824in}{1.684346in}}{\pgfqpoint{1.961824in}{1.673296in}}%
\pgfpathcurveto{\pgfqpoint{1.961824in}{1.662246in}}{\pgfqpoint{1.966215in}{1.651647in}}{\pgfqpoint{1.974028in}{1.643834in}}%
\pgfpathcurveto{\pgfqpoint{1.981842in}{1.636020in}}{\pgfqpoint{1.992441in}{1.631630in}}{\pgfqpoint{2.003491in}{1.631630in}}%
\pgfpathclose%
\pgfusepath{stroke,fill}%
\end{pgfscope}%
\begin{pgfscope}%
\pgfpathrectangle{\pgfqpoint{0.787074in}{0.548769in}}{\pgfqpoint{5.062926in}{3.102590in}}%
\pgfusepath{clip}%
\pgfsetbuttcap%
\pgfsetroundjoin%
\definecolor{currentfill}{rgb}{1.000000,0.498039,0.054902}%
\pgfsetfillcolor{currentfill}%
\pgfsetlinewidth{1.003750pt}%
\definecolor{currentstroke}{rgb}{1.000000,0.498039,0.054902}%
\pgfsetstrokecolor{currentstroke}%
\pgfsetdash{}{0pt}%
\pgfpathmoveto{\pgfqpoint{1.411721in}{1.681005in}}%
\pgfpathcurveto{\pgfqpoint{1.422771in}{1.681005in}}{\pgfqpoint{1.433370in}{1.685396in}}{\pgfqpoint{1.441183in}{1.693209in}}%
\pgfpathcurveto{\pgfqpoint{1.448997in}{1.701023in}}{\pgfqpoint{1.453387in}{1.711622in}}{\pgfqpoint{1.453387in}{1.722672in}}%
\pgfpathcurveto{\pgfqpoint{1.453387in}{1.733722in}}{\pgfqpoint{1.448997in}{1.744321in}}{\pgfqpoint{1.441183in}{1.752135in}}%
\pgfpathcurveto{\pgfqpoint{1.433370in}{1.759948in}}{\pgfqpoint{1.422771in}{1.764339in}}{\pgfqpoint{1.411721in}{1.764339in}}%
\pgfpathcurveto{\pgfqpoint{1.400670in}{1.764339in}}{\pgfqpoint{1.390071in}{1.759948in}}{\pgfqpoint{1.382258in}{1.752135in}}%
\pgfpathcurveto{\pgfqpoint{1.374444in}{1.744321in}}{\pgfqpoint{1.370054in}{1.733722in}}{\pgfqpoint{1.370054in}{1.722672in}}%
\pgfpathcurveto{\pgfqpoint{1.370054in}{1.711622in}}{\pgfqpoint{1.374444in}{1.701023in}}{\pgfqpoint{1.382258in}{1.693209in}}%
\pgfpathcurveto{\pgfqpoint{1.390071in}{1.685396in}}{\pgfqpoint{1.400670in}{1.681005in}}{\pgfqpoint{1.411721in}{1.681005in}}%
\pgfpathclose%
\pgfusepath{stroke,fill}%
\end{pgfscope}%
\begin{pgfscope}%
\pgfpathrectangle{\pgfqpoint{0.787074in}{0.548769in}}{\pgfqpoint{5.062926in}{3.102590in}}%
\pgfusepath{clip}%
\pgfsetbuttcap%
\pgfsetroundjoin%
\definecolor{currentfill}{rgb}{0.121569,0.466667,0.705882}%
\pgfsetfillcolor{currentfill}%
\pgfsetlinewidth{1.003750pt}%
\definecolor{currentstroke}{rgb}{0.121569,0.466667,0.705882}%
\pgfsetstrokecolor{currentstroke}%
\pgfsetdash{}{0pt}%
\pgfpathmoveto{\pgfqpoint{1.082959in}{0.648134in}}%
\pgfpathcurveto{\pgfqpoint{1.094009in}{0.648134in}}{\pgfqpoint{1.104608in}{0.652524in}}{\pgfqpoint{1.112422in}{0.660338in}}%
\pgfpathcurveto{\pgfqpoint{1.120236in}{0.668151in}}{\pgfqpoint{1.124626in}{0.678750in}}{\pgfqpoint{1.124626in}{0.689800in}}%
\pgfpathcurveto{\pgfqpoint{1.124626in}{0.700850in}}{\pgfqpoint{1.120236in}{0.711450in}}{\pgfqpoint{1.112422in}{0.719263in}}%
\pgfpathcurveto{\pgfqpoint{1.104608in}{0.727077in}}{\pgfqpoint{1.094009in}{0.731467in}}{\pgfqpoint{1.082959in}{0.731467in}}%
\pgfpathcurveto{\pgfqpoint{1.071909in}{0.731467in}}{\pgfqpoint{1.061310in}{0.727077in}}{\pgfqpoint{1.053496in}{0.719263in}}%
\pgfpathcurveto{\pgfqpoint{1.045683in}{0.711450in}}{\pgfqpoint{1.041292in}{0.700850in}}{\pgfqpoint{1.041292in}{0.689800in}}%
\pgfpathcurveto{\pgfqpoint{1.041292in}{0.678750in}}{\pgfqpoint{1.045683in}{0.668151in}}{\pgfqpoint{1.053496in}{0.660338in}}%
\pgfpathcurveto{\pgfqpoint{1.061310in}{0.652524in}}{\pgfqpoint{1.071909in}{0.648134in}}{\pgfqpoint{1.082959in}{0.648134in}}%
\pgfpathclose%
\pgfusepath{stroke,fill}%
\end{pgfscope}%
\begin{pgfscope}%
\pgfpathrectangle{\pgfqpoint{0.787074in}{0.548769in}}{\pgfqpoint{5.062926in}{3.102590in}}%
\pgfusepath{clip}%
\pgfsetbuttcap%
\pgfsetroundjoin%
\definecolor{currentfill}{rgb}{1.000000,0.498039,0.054902}%
\pgfsetfillcolor{currentfill}%
\pgfsetlinewidth{1.003750pt}%
\definecolor{currentstroke}{rgb}{1.000000,0.498039,0.054902}%
\pgfsetstrokecolor{currentstroke}%
\pgfsetdash{}{0pt}%
\pgfpathmoveto{\pgfqpoint{1.806234in}{2.734356in}}%
\pgfpathcurveto{\pgfqpoint{1.817284in}{2.734356in}}{\pgfqpoint{1.827883in}{2.738746in}}{\pgfqpoint{1.835697in}{2.746560in}}%
\pgfpathcurveto{\pgfqpoint{1.843511in}{2.754373in}}{\pgfqpoint{1.847901in}{2.764972in}}{\pgfqpoint{1.847901in}{2.776022in}}%
\pgfpathcurveto{\pgfqpoint{1.847901in}{2.787073in}}{\pgfqpoint{1.843511in}{2.797672in}}{\pgfqpoint{1.835697in}{2.805485in}}%
\pgfpathcurveto{\pgfqpoint{1.827883in}{2.813299in}}{\pgfqpoint{1.817284in}{2.817689in}}{\pgfqpoint{1.806234in}{2.817689in}}%
\pgfpathcurveto{\pgfqpoint{1.795184in}{2.817689in}}{\pgfqpoint{1.784585in}{2.813299in}}{\pgfqpoint{1.776772in}{2.805485in}}%
\pgfpathcurveto{\pgfqpoint{1.768958in}{2.797672in}}{\pgfqpoint{1.764568in}{2.787073in}}{\pgfqpoint{1.764568in}{2.776022in}}%
\pgfpathcurveto{\pgfqpoint{1.764568in}{2.764972in}}{\pgfqpoint{1.768958in}{2.754373in}}{\pgfqpoint{1.776772in}{2.746560in}}%
\pgfpathcurveto{\pgfqpoint{1.784585in}{2.738746in}}{\pgfqpoint{1.795184in}{2.734356in}}{\pgfqpoint{1.806234in}{2.734356in}}%
\pgfpathclose%
\pgfusepath{stroke,fill}%
\end{pgfscope}%
\begin{pgfscope}%
\pgfpathrectangle{\pgfqpoint{0.787074in}{0.548769in}}{\pgfqpoint{5.062926in}{3.102590in}}%
\pgfusepath{clip}%
\pgfsetbuttcap%
\pgfsetroundjoin%
\definecolor{currentfill}{rgb}{1.000000,0.498039,0.054902}%
\pgfsetfillcolor{currentfill}%
\pgfsetlinewidth{1.003750pt}%
\definecolor{currentstroke}{rgb}{1.000000,0.498039,0.054902}%
\pgfsetstrokecolor{currentstroke}%
\pgfsetdash{}{0pt}%
\pgfpathmoveto{\pgfqpoint{1.543225in}{2.144210in}}%
\pgfpathcurveto{\pgfqpoint{1.554275in}{2.144210in}}{\pgfqpoint{1.564874in}{2.148600in}}{\pgfqpoint{1.572688in}{2.156414in}}%
\pgfpathcurveto{\pgfqpoint{1.580502in}{2.164228in}}{\pgfqpoint{1.584892in}{2.174827in}}{\pgfqpoint{1.584892in}{2.185877in}}%
\pgfpathcurveto{\pgfqpoint{1.584892in}{2.196927in}}{\pgfqpoint{1.580502in}{2.207526in}}{\pgfqpoint{1.572688in}{2.215340in}}%
\pgfpathcurveto{\pgfqpoint{1.564874in}{2.223153in}}{\pgfqpoint{1.554275in}{2.227544in}}{\pgfqpoint{1.543225in}{2.227544in}}%
\pgfpathcurveto{\pgfqpoint{1.532175in}{2.227544in}}{\pgfqpoint{1.521576in}{2.223153in}}{\pgfqpoint{1.513762in}{2.215340in}}%
\pgfpathcurveto{\pgfqpoint{1.505949in}{2.207526in}}{\pgfqpoint{1.501558in}{2.196927in}}{\pgfqpoint{1.501558in}{2.185877in}}%
\pgfpathcurveto{\pgfqpoint{1.501558in}{2.174827in}}{\pgfqpoint{1.505949in}{2.164228in}}{\pgfqpoint{1.513762in}{2.156414in}}%
\pgfpathcurveto{\pgfqpoint{1.521576in}{2.148600in}}{\pgfqpoint{1.532175in}{2.144210in}}{\pgfqpoint{1.543225in}{2.144210in}}%
\pgfpathclose%
\pgfusepath{stroke,fill}%
\end{pgfscope}%
\begin{pgfscope}%
\pgfpathrectangle{\pgfqpoint{0.787074in}{0.548769in}}{\pgfqpoint{5.062926in}{3.102590in}}%
\pgfusepath{clip}%
\pgfsetbuttcap%
\pgfsetroundjoin%
\definecolor{currentfill}{rgb}{1.000000,0.498039,0.054902}%
\pgfsetfillcolor{currentfill}%
\pgfsetlinewidth{1.003750pt}%
\definecolor{currentstroke}{rgb}{1.000000,0.498039,0.054902}%
\pgfsetstrokecolor{currentstroke}%
\pgfsetdash{}{0pt}%
\pgfpathmoveto{\pgfqpoint{1.543225in}{2.799180in}}%
\pgfpathcurveto{\pgfqpoint{1.554275in}{2.799180in}}{\pgfqpoint{1.564874in}{2.803570in}}{\pgfqpoint{1.572688in}{2.811384in}}%
\pgfpathcurveto{\pgfqpoint{1.580502in}{2.819197in}}{\pgfqpoint{1.584892in}{2.829796in}}{\pgfqpoint{1.584892in}{2.840846in}}%
\pgfpathcurveto{\pgfqpoint{1.584892in}{2.851897in}}{\pgfqpoint{1.580502in}{2.862496in}}{\pgfqpoint{1.572688in}{2.870309in}}%
\pgfpathcurveto{\pgfqpoint{1.564874in}{2.878123in}}{\pgfqpoint{1.554275in}{2.882513in}}{\pgfqpoint{1.543225in}{2.882513in}}%
\pgfpathcurveto{\pgfqpoint{1.532175in}{2.882513in}}{\pgfqpoint{1.521576in}{2.878123in}}{\pgfqpoint{1.513762in}{2.870309in}}%
\pgfpathcurveto{\pgfqpoint{1.505949in}{2.862496in}}{\pgfqpoint{1.501558in}{2.851897in}}{\pgfqpoint{1.501558in}{2.840846in}}%
\pgfpathcurveto{\pgfqpoint{1.501558in}{2.829796in}}{\pgfqpoint{1.505949in}{2.819197in}}{\pgfqpoint{1.513762in}{2.811384in}}%
\pgfpathcurveto{\pgfqpoint{1.521576in}{2.803570in}}{\pgfqpoint{1.532175in}{2.799180in}}{\pgfqpoint{1.543225in}{2.799180in}}%
\pgfpathclose%
\pgfusepath{stroke,fill}%
\end{pgfscope}%
\begin{pgfscope}%
\pgfpathrectangle{\pgfqpoint{0.787074in}{0.548769in}}{\pgfqpoint{5.062926in}{3.102590in}}%
\pgfusepath{clip}%
\pgfsetbuttcap%
\pgfsetroundjoin%
\definecolor{currentfill}{rgb}{1.000000,0.498039,0.054902}%
\pgfsetfillcolor{currentfill}%
\pgfsetlinewidth{1.003750pt}%
\definecolor{currentstroke}{rgb}{1.000000,0.498039,0.054902}%
\pgfsetstrokecolor{currentstroke}%
\pgfsetdash{}{0pt}%
\pgfpathmoveto{\pgfqpoint{1.411721in}{2.526596in}}%
\pgfpathcurveto{\pgfqpoint{1.422771in}{2.526596in}}{\pgfqpoint{1.433370in}{2.530987in}}{\pgfqpoint{1.441183in}{2.538800in}}%
\pgfpathcurveto{\pgfqpoint{1.448997in}{2.546614in}}{\pgfqpoint{1.453387in}{2.557213in}}{\pgfqpoint{1.453387in}{2.568263in}}%
\pgfpathcurveto{\pgfqpoint{1.453387in}{2.579313in}}{\pgfqpoint{1.448997in}{2.589912in}}{\pgfqpoint{1.441183in}{2.597726in}}%
\pgfpathcurveto{\pgfqpoint{1.433370in}{2.605540in}}{\pgfqpoint{1.422771in}{2.609930in}}{\pgfqpoint{1.411721in}{2.609930in}}%
\pgfpathcurveto{\pgfqpoint{1.400670in}{2.609930in}}{\pgfqpoint{1.390071in}{2.605540in}}{\pgfqpoint{1.382258in}{2.597726in}}%
\pgfpathcurveto{\pgfqpoint{1.374444in}{2.589912in}}{\pgfqpoint{1.370054in}{2.579313in}}{\pgfqpoint{1.370054in}{2.568263in}}%
\pgfpathcurveto{\pgfqpoint{1.370054in}{2.557213in}}{\pgfqpoint{1.374444in}{2.546614in}}{\pgfqpoint{1.382258in}{2.538800in}}%
\pgfpathcurveto{\pgfqpoint{1.390071in}{2.530987in}}{\pgfqpoint{1.400670in}{2.526596in}}{\pgfqpoint{1.411721in}{2.526596in}}%
\pgfpathclose%
\pgfusepath{stroke,fill}%
\end{pgfscope}%
\begin{pgfscope}%
\pgfpathrectangle{\pgfqpoint{0.787074in}{0.548769in}}{\pgfqpoint{5.062926in}{3.102590in}}%
\pgfusepath{clip}%
\pgfsetbuttcap%
\pgfsetroundjoin%
\definecolor{currentfill}{rgb}{1.000000,0.498039,0.054902}%
\pgfsetfillcolor{currentfill}%
\pgfsetlinewidth{1.003750pt}%
\definecolor{currentstroke}{rgb}{1.000000,0.498039,0.054902}%
\pgfsetstrokecolor{currentstroke}%
\pgfsetdash{}{0pt}%
\pgfpathmoveto{\pgfqpoint{1.806234in}{2.883497in}}%
\pgfpathcurveto{\pgfqpoint{1.817284in}{2.883497in}}{\pgfqpoint{1.827883in}{2.887888in}}{\pgfqpoint{1.835697in}{2.895701in}}%
\pgfpathcurveto{\pgfqpoint{1.843511in}{2.903515in}}{\pgfqpoint{1.847901in}{2.914114in}}{\pgfqpoint{1.847901in}{2.925164in}}%
\pgfpathcurveto{\pgfqpoint{1.847901in}{2.936214in}}{\pgfqpoint{1.843511in}{2.946813in}}{\pgfqpoint{1.835697in}{2.954627in}}%
\pgfpathcurveto{\pgfqpoint{1.827883in}{2.962440in}}{\pgfqpoint{1.817284in}{2.966831in}}{\pgfqpoint{1.806234in}{2.966831in}}%
\pgfpathcurveto{\pgfqpoint{1.795184in}{2.966831in}}{\pgfqpoint{1.784585in}{2.962440in}}{\pgfqpoint{1.776772in}{2.954627in}}%
\pgfpathcurveto{\pgfqpoint{1.768958in}{2.946813in}}{\pgfqpoint{1.764568in}{2.936214in}}{\pgfqpoint{1.764568in}{2.925164in}}%
\pgfpathcurveto{\pgfqpoint{1.764568in}{2.914114in}}{\pgfqpoint{1.768958in}{2.903515in}}{\pgfqpoint{1.776772in}{2.895701in}}%
\pgfpathcurveto{\pgfqpoint{1.784585in}{2.887888in}}{\pgfqpoint{1.795184in}{2.883497in}}{\pgfqpoint{1.806234in}{2.883497in}}%
\pgfpathclose%
\pgfusepath{stroke,fill}%
\end{pgfscope}%
\begin{pgfscope}%
\pgfpathrectangle{\pgfqpoint{0.787074in}{0.548769in}}{\pgfqpoint{5.062926in}{3.102590in}}%
\pgfusepath{clip}%
\pgfsetbuttcap%
\pgfsetroundjoin%
\definecolor{currentfill}{rgb}{1.000000,0.498039,0.054902}%
\pgfsetfillcolor{currentfill}%
\pgfsetlinewidth{1.003750pt}%
\definecolor{currentstroke}{rgb}{1.000000,0.498039,0.054902}%
\pgfsetstrokecolor{currentstroke}%
\pgfsetdash{}{0pt}%
\pgfpathmoveto{\pgfqpoint{2.069243in}{2.042862in}}%
\pgfpathcurveto{\pgfqpoint{2.080294in}{2.042862in}}{\pgfqpoint{2.090893in}{2.047252in}}{\pgfqpoint{2.098706in}{2.055065in}}%
\pgfpathcurveto{\pgfqpoint{2.106520in}{2.062879in}}{\pgfqpoint{2.110910in}{2.073478in}}{\pgfqpoint{2.110910in}{2.084528in}}%
\pgfpathcurveto{\pgfqpoint{2.110910in}{2.095578in}}{\pgfqpoint{2.106520in}{2.106177in}}{\pgfqpoint{2.098706in}{2.113991in}}%
\pgfpathcurveto{\pgfqpoint{2.090893in}{2.121805in}}{\pgfqpoint{2.080294in}{2.126195in}}{\pgfqpoint{2.069243in}{2.126195in}}%
\pgfpathcurveto{\pgfqpoint{2.058193in}{2.126195in}}{\pgfqpoint{2.047594in}{2.121805in}}{\pgfqpoint{2.039781in}{2.113991in}}%
\pgfpathcurveto{\pgfqpoint{2.031967in}{2.106177in}}{\pgfqpoint{2.027577in}{2.095578in}}{\pgfqpoint{2.027577in}{2.084528in}}%
\pgfpathcurveto{\pgfqpoint{2.027577in}{2.073478in}}{\pgfqpoint{2.031967in}{2.062879in}}{\pgfqpoint{2.039781in}{2.055065in}}%
\pgfpathcurveto{\pgfqpoint{2.047594in}{2.047252in}}{\pgfqpoint{2.058193in}{2.042862in}}{\pgfqpoint{2.069243in}{2.042862in}}%
\pgfpathclose%
\pgfusepath{stroke,fill}%
\end{pgfscope}%
\begin{pgfscope}%
\pgfpathrectangle{\pgfqpoint{0.787074in}{0.548769in}}{\pgfqpoint{5.062926in}{3.102590in}}%
\pgfusepath{clip}%
\pgfsetbuttcap%
\pgfsetroundjoin%
\definecolor{currentfill}{rgb}{1.000000,0.498039,0.054902}%
\pgfsetfillcolor{currentfill}%
\pgfsetlinewidth{1.003750pt}%
\definecolor{currentstroke}{rgb}{1.000000,0.498039,0.054902}%
\pgfsetstrokecolor{currentstroke}%
\pgfsetdash{}{0pt}%
\pgfpathmoveto{\pgfqpoint{1.806234in}{2.760534in}}%
\pgfpathcurveto{\pgfqpoint{1.817284in}{2.760534in}}{\pgfqpoint{1.827883in}{2.764924in}}{\pgfqpoint{1.835697in}{2.772738in}}%
\pgfpathcurveto{\pgfqpoint{1.843511in}{2.780552in}}{\pgfqpoint{1.847901in}{2.791151in}}{\pgfqpoint{1.847901in}{2.802201in}}%
\pgfpathcurveto{\pgfqpoint{1.847901in}{2.813251in}}{\pgfqpoint{1.843511in}{2.823850in}}{\pgfqpoint{1.835697in}{2.831664in}}%
\pgfpathcurveto{\pgfqpoint{1.827883in}{2.839477in}}{\pgfqpoint{1.817284in}{2.843868in}}{\pgfqpoint{1.806234in}{2.843868in}}%
\pgfpathcurveto{\pgfqpoint{1.795184in}{2.843868in}}{\pgfqpoint{1.784585in}{2.839477in}}{\pgfqpoint{1.776772in}{2.831664in}}%
\pgfpathcurveto{\pgfqpoint{1.768958in}{2.823850in}}{\pgfqpoint{1.764568in}{2.813251in}}{\pgfqpoint{1.764568in}{2.802201in}}%
\pgfpathcurveto{\pgfqpoint{1.764568in}{2.791151in}}{\pgfqpoint{1.768958in}{2.780552in}}{\pgfqpoint{1.776772in}{2.772738in}}%
\pgfpathcurveto{\pgfqpoint{1.784585in}{2.764924in}}{\pgfqpoint{1.795184in}{2.760534in}}{\pgfqpoint{1.806234in}{2.760534in}}%
\pgfpathclose%
\pgfusepath{stroke,fill}%
\end{pgfscope}%
\begin{pgfscope}%
\pgfpathrectangle{\pgfqpoint{0.787074in}{0.548769in}}{\pgfqpoint{5.062926in}{3.102590in}}%
\pgfusepath{clip}%
\pgfsetbuttcap%
\pgfsetroundjoin%
\definecolor{currentfill}{rgb}{0.121569,0.466667,0.705882}%
\pgfsetfillcolor{currentfill}%
\pgfsetlinewidth{1.003750pt}%
\definecolor{currentstroke}{rgb}{0.121569,0.466667,0.705882}%
\pgfsetstrokecolor{currentstroke}%
\pgfsetdash{}{0pt}%
\pgfpathmoveto{\pgfqpoint{1.017207in}{0.648129in}}%
\pgfpathcurveto{\pgfqpoint{1.028257in}{0.648129in}}{\pgfqpoint{1.038856in}{0.652519in}}{\pgfqpoint{1.046670in}{0.660333in}}%
\pgfpathcurveto{\pgfqpoint{1.054483in}{0.668146in}}{\pgfqpoint{1.058874in}{0.678745in}}{\pgfqpoint{1.058874in}{0.689796in}}%
\pgfpathcurveto{\pgfqpoint{1.058874in}{0.700846in}}{\pgfqpoint{1.054483in}{0.711445in}}{\pgfqpoint{1.046670in}{0.719258in}}%
\pgfpathcurveto{\pgfqpoint{1.038856in}{0.727072in}}{\pgfqpoint{1.028257in}{0.731462in}}{\pgfqpoint{1.017207in}{0.731462in}}%
\pgfpathcurveto{\pgfqpoint{1.006157in}{0.731462in}}{\pgfqpoint{0.995558in}{0.727072in}}{\pgfqpoint{0.987744in}{0.719258in}}%
\pgfpathcurveto{\pgfqpoint{0.979930in}{0.711445in}}{\pgfqpoint{0.975540in}{0.700846in}}{\pgfqpoint{0.975540in}{0.689796in}}%
\pgfpathcurveto{\pgfqpoint{0.975540in}{0.678745in}}{\pgfqpoint{0.979930in}{0.668146in}}{\pgfqpoint{0.987744in}{0.660333in}}%
\pgfpathcurveto{\pgfqpoint{0.995558in}{0.652519in}}{\pgfqpoint{1.006157in}{0.648129in}}{\pgfqpoint{1.017207in}{0.648129in}}%
\pgfpathclose%
\pgfusepath{stroke,fill}%
\end{pgfscope}%
\begin{pgfscope}%
\pgfpathrectangle{\pgfqpoint{0.787074in}{0.548769in}}{\pgfqpoint{5.062926in}{3.102590in}}%
\pgfusepath{clip}%
\pgfsetbuttcap%
\pgfsetroundjoin%
\definecolor{currentfill}{rgb}{0.121569,0.466667,0.705882}%
\pgfsetfillcolor{currentfill}%
\pgfsetlinewidth{1.003750pt}%
\definecolor{currentstroke}{rgb}{0.121569,0.466667,0.705882}%
\pgfsetstrokecolor{currentstroke}%
\pgfsetdash{}{0pt}%
\pgfpathmoveto{\pgfqpoint{1.280216in}{0.658321in}}%
\pgfpathcurveto{\pgfqpoint{1.291266in}{0.658321in}}{\pgfqpoint{1.301865in}{0.662711in}}{\pgfqpoint{1.309679in}{0.670525in}}%
\pgfpathcurveto{\pgfqpoint{1.317492in}{0.678338in}}{\pgfqpoint{1.321883in}{0.688937in}}{\pgfqpoint{1.321883in}{0.699987in}}%
\pgfpathcurveto{\pgfqpoint{1.321883in}{0.711038in}}{\pgfqpoint{1.317492in}{0.721637in}}{\pgfqpoint{1.309679in}{0.729450in}}%
\pgfpathcurveto{\pgfqpoint{1.301865in}{0.737264in}}{\pgfqpoint{1.291266in}{0.741654in}}{\pgfqpoint{1.280216in}{0.741654in}}%
\pgfpathcurveto{\pgfqpoint{1.269166in}{0.741654in}}{\pgfqpoint{1.258567in}{0.737264in}}{\pgfqpoint{1.250753in}{0.729450in}}%
\pgfpathcurveto{\pgfqpoint{1.242940in}{0.721637in}}{\pgfqpoint{1.238549in}{0.711038in}}{\pgfqpoint{1.238549in}{0.699987in}}%
\pgfpathcurveto{\pgfqpoint{1.238549in}{0.688937in}}{\pgfqpoint{1.242940in}{0.678338in}}{\pgfqpoint{1.250753in}{0.670525in}}%
\pgfpathcurveto{\pgfqpoint{1.258567in}{0.662711in}}{\pgfqpoint{1.269166in}{0.658321in}}{\pgfqpoint{1.280216in}{0.658321in}}%
\pgfpathclose%
\pgfusepath{stroke,fill}%
\end{pgfscope}%
\begin{pgfscope}%
\pgfpathrectangle{\pgfqpoint{0.787074in}{0.548769in}}{\pgfqpoint{5.062926in}{3.102590in}}%
\pgfusepath{clip}%
\pgfsetbuttcap%
\pgfsetroundjoin%
\definecolor{currentfill}{rgb}{1.000000,0.498039,0.054902}%
\pgfsetfillcolor{currentfill}%
\pgfsetlinewidth{1.003750pt}%
\definecolor{currentstroke}{rgb}{1.000000,0.498039,0.054902}%
\pgfsetstrokecolor{currentstroke}%
\pgfsetdash{}{0pt}%
\pgfpathmoveto{\pgfqpoint{1.608977in}{1.208140in}}%
\pgfpathcurveto{\pgfqpoint{1.620028in}{1.208140in}}{\pgfqpoint{1.630627in}{1.212530in}}{\pgfqpoint{1.638440in}{1.220344in}}%
\pgfpathcurveto{\pgfqpoint{1.646254in}{1.228157in}}{\pgfqpoint{1.650644in}{1.238756in}}{\pgfqpoint{1.650644in}{1.249807in}}%
\pgfpathcurveto{\pgfqpoint{1.650644in}{1.260857in}}{\pgfqpoint{1.646254in}{1.271456in}}{\pgfqpoint{1.638440in}{1.279269in}}%
\pgfpathcurveto{\pgfqpoint{1.630627in}{1.287083in}}{\pgfqpoint{1.620028in}{1.291473in}}{\pgfqpoint{1.608977in}{1.291473in}}%
\pgfpathcurveto{\pgfqpoint{1.597927in}{1.291473in}}{\pgfqpoint{1.587328in}{1.287083in}}{\pgfqpoint{1.579515in}{1.279269in}}%
\pgfpathcurveto{\pgfqpoint{1.571701in}{1.271456in}}{\pgfqpoint{1.567311in}{1.260857in}}{\pgfqpoint{1.567311in}{1.249807in}}%
\pgfpathcurveto{\pgfqpoint{1.567311in}{1.238756in}}{\pgfqpoint{1.571701in}{1.228157in}}{\pgfqpoint{1.579515in}{1.220344in}}%
\pgfpathcurveto{\pgfqpoint{1.587328in}{1.212530in}}{\pgfqpoint{1.597927in}{1.208140in}}{\pgfqpoint{1.608977in}{1.208140in}}%
\pgfpathclose%
\pgfusepath{stroke,fill}%
\end{pgfscope}%
\begin{pgfscope}%
\pgfpathrectangle{\pgfqpoint{0.787074in}{0.548769in}}{\pgfqpoint{5.062926in}{3.102590in}}%
\pgfusepath{clip}%
\pgfsetbuttcap%
\pgfsetroundjoin%
\definecolor{currentfill}{rgb}{1.000000,0.498039,0.054902}%
\pgfsetfillcolor{currentfill}%
\pgfsetlinewidth{1.003750pt}%
\definecolor{currentstroke}{rgb}{1.000000,0.498039,0.054902}%
\pgfsetstrokecolor{currentstroke}%
\pgfsetdash{}{0pt}%
\pgfpathmoveto{\pgfqpoint{1.411721in}{2.220680in}}%
\pgfpathcurveto{\pgfqpoint{1.422771in}{2.220680in}}{\pgfqpoint{1.433370in}{2.225070in}}{\pgfqpoint{1.441183in}{2.232884in}}%
\pgfpathcurveto{\pgfqpoint{1.448997in}{2.240698in}}{\pgfqpoint{1.453387in}{2.251297in}}{\pgfqpoint{1.453387in}{2.262347in}}%
\pgfpathcurveto{\pgfqpoint{1.453387in}{2.273397in}}{\pgfqpoint{1.448997in}{2.283996in}}{\pgfqpoint{1.441183in}{2.291810in}}%
\pgfpathcurveto{\pgfqpoint{1.433370in}{2.299623in}}{\pgfqpoint{1.422771in}{2.304013in}}{\pgfqpoint{1.411721in}{2.304013in}}%
\pgfpathcurveto{\pgfqpoint{1.400670in}{2.304013in}}{\pgfqpoint{1.390071in}{2.299623in}}{\pgfqpoint{1.382258in}{2.291810in}}%
\pgfpathcurveto{\pgfqpoint{1.374444in}{2.283996in}}{\pgfqpoint{1.370054in}{2.273397in}}{\pgfqpoint{1.370054in}{2.262347in}}%
\pgfpathcurveto{\pgfqpoint{1.370054in}{2.251297in}}{\pgfqpoint{1.374444in}{2.240698in}}{\pgfqpoint{1.382258in}{2.232884in}}%
\pgfpathcurveto{\pgfqpoint{1.390071in}{2.225070in}}{\pgfqpoint{1.400670in}{2.220680in}}{\pgfqpoint{1.411721in}{2.220680in}}%
\pgfpathclose%
\pgfusepath{stroke,fill}%
\end{pgfscope}%
\begin{pgfscope}%
\pgfpathrectangle{\pgfqpoint{0.787074in}{0.548769in}}{\pgfqpoint{5.062926in}{3.102590in}}%
\pgfusepath{clip}%
\pgfsetbuttcap%
\pgfsetroundjoin%
\definecolor{currentfill}{rgb}{0.121569,0.466667,0.705882}%
\pgfsetfillcolor{currentfill}%
\pgfsetlinewidth{1.003750pt}%
\definecolor{currentstroke}{rgb}{0.121569,0.466667,0.705882}%
\pgfsetstrokecolor{currentstroke}%
\pgfsetdash{}{0pt}%
\pgfpathmoveto{\pgfqpoint{1.017207in}{0.648130in}}%
\pgfpathcurveto{\pgfqpoint{1.028257in}{0.648130in}}{\pgfqpoint{1.038856in}{0.652520in}}{\pgfqpoint{1.046670in}{0.660334in}}%
\pgfpathcurveto{\pgfqpoint{1.054483in}{0.668148in}}{\pgfqpoint{1.058874in}{0.678747in}}{\pgfqpoint{1.058874in}{0.689797in}}%
\pgfpathcurveto{\pgfqpoint{1.058874in}{0.700847in}}{\pgfqpoint{1.054483in}{0.711446in}}{\pgfqpoint{1.046670in}{0.719260in}}%
\pgfpathcurveto{\pgfqpoint{1.038856in}{0.727073in}}{\pgfqpoint{1.028257in}{0.731464in}}{\pgfqpoint{1.017207in}{0.731464in}}%
\pgfpathcurveto{\pgfqpoint{1.006157in}{0.731464in}}{\pgfqpoint{0.995558in}{0.727073in}}{\pgfqpoint{0.987744in}{0.719260in}}%
\pgfpathcurveto{\pgfqpoint{0.979930in}{0.711446in}}{\pgfqpoint{0.975540in}{0.700847in}}{\pgfqpoint{0.975540in}{0.689797in}}%
\pgfpathcurveto{\pgfqpoint{0.975540in}{0.678747in}}{\pgfqpoint{0.979930in}{0.668148in}}{\pgfqpoint{0.987744in}{0.660334in}}%
\pgfpathcurveto{\pgfqpoint{0.995558in}{0.652520in}}{\pgfqpoint{1.006157in}{0.648130in}}{\pgfqpoint{1.017207in}{0.648130in}}%
\pgfpathclose%
\pgfusepath{stroke,fill}%
\end{pgfscope}%
\begin{pgfscope}%
\pgfpathrectangle{\pgfqpoint{0.787074in}{0.548769in}}{\pgfqpoint{5.062926in}{3.102590in}}%
\pgfusepath{clip}%
\pgfsetbuttcap%
\pgfsetroundjoin%
\definecolor{currentfill}{rgb}{0.121569,0.466667,0.705882}%
\pgfsetfillcolor{currentfill}%
\pgfsetlinewidth{1.003750pt}%
\definecolor{currentstroke}{rgb}{0.121569,0.466667,0.705882}%
\pgfsetstrokecolor{currentstroke}%
\pgfsetdash{}{0pt}%
\pgfpathmoveto{\pgfqpoint{1.082959in}{0.648134in}}%
\pgfpathcurveto{\pgfqpoint{1.094009in}{0.648134in}}{\pgfqpoint{1.104608in}{0.652524in}}{\pgfqpoint{1.112422in}{0.660338in}}%
\pgfpathcurveto{\pgfqpoint{1.120236in}{0.668151in}}{\pgfqpoint{1.124626in}{0.678750in}}{\pgfqpoint{1.124626in}{0.689800in}}%
\pgfpathcurveto{\pgfqpoint{1.124626in}{0.700850in}}{\pgfqpoint{1.120236in}{0.711450in}}{\pgfqpoint{1.112422in}{0.719263in}}%
\pgfpathcurveto{\pgfqpoint{1.104608in}{0.727077in}}{\pgfqpoint{1.094009in}{0.731467in}}{\pgfqpoint{1.082959in}{0.731467in}}%
\pgfpathcurveto{\pgfqpoint{1.071909in}{0.731467in}}{\pgfqpoint{1.061310in}{0.727077in}}{\pgfqpoint{1.053496in}{0.719263in}}%
\pgfpathcurveto{\pgfqpoint{1.045683in}{0.711450in}}{\pgfqpoint{1.041292in}{0.700850in}}{\pgfqpoint{1.041292in}{0.689800in}}%
\pgfpathcurveto{\pgfqpoint{1.041292in}{0.678750in}}{\pgfqpoint{1.045683in}{0.668151in}}{\pgfqpoint{1.053496in}{0.660338in}}%
\pgfpathcurveto{\pgfqpoint{1.061310in}{0.652524in}}{\pgfqpoint{1.071909in}{0.648134in}}{\pgfqpoint{1.082959in}{0.648134in}}%
\pgfpathclose%
\pgfusepath{stroke,fill}%
\end{pgfscope}%
\begin{pgfscope}%
\pgfpathrectangle{\pgfqpoint{0.787074in}{0.548769in}}{\pgfqpoint{5.062926in}{3.102590in}}%
\pgfusepath{clip}%
\pgfsetbuttcap%
\pgfsetroundjoin%
\definecolor{currentfill}{rgb}{1.000000,0.498039,0.054902}%
\pgfsetfillcolor{currentfill}%
\pgfsetlinewidth{1.003750pt}%
\definecolor{currentstroke}{rgb}{1.000000,0.498039,0.054902}%
\pgfsetstrokecolor{currentstroke}%
\pgfsetdash{}{0pt}%
\pgfpathmoveto{\pgfqpoint{1.937739in}{3.468665in}}%
\pgfpathcurveto{\pgfqpoint{1.948789in}{3.468665in}}{\pgfqpoint{1.959388in}{3.473055in}}{\pgfqpoint{1.967202in}{3.480869in}}%
\pgfpathcurveto{\pgfqpoint{1.975015in}{3.488683in}}{\pgfqpoint{1.979406in}{3.499282in}}{\pgfqpoint{1.979406in}{3.510332in}}%
\pgfpathcurveto{\pgfqpoint{1.979406in}{3.521382in}}{\pgfqpoint{1.975015in}{3.531981in}}{\pgfqpoint{1.967202in}{3.539795in}}%
\pgfpathcurveto{\pgfqpoint{1.959388in}{3.547608in}}{\pgfqpoint{1.948789in}{3.551998in}}{\pgfqpoint{1.937739in}{3.551998in}}%
\pgfpathcurveto{\pgfqpoint{1.926689in}{3.551998in}}{\pgfqpoint{1.916090in}{3.547608in}}{\pgfqpoint{1.908276in}{3.539795in}}%
\pgfpathcurveto{\pgfqpoint{1.900462in}{3.531981in}}{\pgfqpoint{1.896072in}{3.521382in}}{\pgfqpoint{1.896072in}{3.510332in}}%
\pgfpathcurveto{\pgfqpoint{1.896072in}{3.499282in}}{\pgfqpoint{1.900462in}{3.488683in}}{\pgfqpoint{1.908276in}{3.480869in}}%
\pgfpathcurveto{\pgfqpoint{1.916090in}{3.473055in}}{\pgfqpoint{1.926689in}{3.468665in}}{\pgfqpoint{1.937739in}{3.468665in}}%
\pgfpathclose%
\pgfusepath{stroke,fill}%
\end{pgfscope}%
\begin{pgfscope}%
\pgfpathrectangle{\pgfqpoint{0.787074in}{0.548769in}}{\pgfqpoint{5.062926in}{3.102590in}}%
\pgfusepath{clip}%
\pgfsetbuttcap%
\pgfsetroundjoin%
\definecolor{currentfill}{rgb}{1.000000,0.498039,0.054902}%
\pgfsetfillcolor{currentfill}%
\pgfsetlinewidth{1.003750pt}%
\definecolor{currentstroke}{rgb}{1.000000,0.498039,0.054902}%
\pgfsetstrokecolor{currentstroke}%
\pgfsetdash{}{0pt}%
\pgfpathmoveto{\pgfqpoint{1.740482in}{2.370534in}}%
\pgfpathcurveto{\pgfqpoint{1.751532in}{2.370534in}}{\pgfqpoint{1.762131in}{2.374924in}}{\pgfqpoint{1.769945in}{2.382738in}}%
\pgfpathcurveto{\pgfqpoint{1.777758in}{2.390551in}}{\pgfqpoint{1.782149in}{2.401150in}}{\pgfqpoint{1.782149in}{2.412201in}}%
\pgfpathcurveto{\pgfqpoint{1.782149in}{2.423251in}}{\pgfqpoint{1.777758in}{2.433850in}}{\pgfqpoint{1.769945in}{2.441663in}}%
\pgfpathcurveto{\pgfqpoint{1.762131in}{2.449477in}}{\pgfqpoint{1.751532in}{2.453867in}}{\pgfqpoint{1.740482in}{2.453867in}}%
\pgfpathcurveto{\pgfqpoint{1.729432in}{2.453867in}}{\pgfqpoint{1.718833in}{2.449477in}}{\pgfqpoint{1.711019in}{2.441663in}}%
\pgfpathcurveto{\pgfqpoint{1.703206in}{2.433850in}}{\pgfqpoint{1.698815in}{2.423251in}}{\pgfqpoint{1.698815in}{2.412201in}}%
\pgfpathcurveto{\pgfqpoint{1.698815in}{2.401150in}}{\pgfqpoint{1.703206in}{2.390551in}}{\pgfqpoint{1.711019in}{2.382738in}}%
\pgfpathcurveto{\pgfqpoint{1.718833in}{2.374924in}}{\pgfqpoint{1.729432in}{2.370534in}}{\pgfqpoint{1.740482in}{2.370534in}}%
\pgfpathclose%
\pgfusepath{stroke,fill}%
\end{pgfscope}%
\begin{pgfscope}%
\pgfpathrectangle{\pgfqpoint{0.787074in}{0.548769in}}{\pgfqpoint{5.062926in}{3.102590in}}%
\pgfusepath{clip}%
\pgfsetbuttcap%
\pgfsetroundjoin%
\definecolor{currentfill}{rgb}{1.000000,0.498039,0.054902}%
\pgfsetfillcolor{currentfill}%
\pgfsetlinewidth{1.003750pt}%
\definecolor{currentstroke}{rgb}{1.000000,0.498039,0.054902}%
\pgfsetstrokecolor{currentstroke}%
\pgfsetdash{}{0pt}%
\pgfpathmoveto{\pgfqpoint{1.806234in}{2.417203in}}%
\pgfpathcurveto{\pgfqpoint{1.817284in}{2.417203in}}{\pgfqpoint{1.827883in}{2.421593in}}{\pgfqpoint{1.835697in}{2.429407in}}%
\pgfpathcurveto{\pgfqpoint{1.843511in}{2.437220in}}{\pgfqpoint{1.847901in}{2.447819in}}{\pgfqpoint{1.847901in}{2.458870in}}%
\pgfpathcurveto{\pgfqpoint{1.847901in}{2.469920in}}{\pgfqpoint{1.843511in}{2.480519in}}{\pgfqpoint{1.835697in}{2.488332in}}%
\pgfpathcurveto{\pgfqpoint{1.827883in}{2.496146in}}{\pgfqpoint{1.817284in}{2.500536in}}{\pgfqpoint{1.806234in}{2.500536in}}%
\pgfpathcurveto{\pgfqpoint{1.795184in}{2.500536in}}{\pgfqpoint{1.784585in}{2.496146in}}{\pgfqpoint{1.776772in}{2.488332in}}%
\pgfpathcurveto{\pgfqpoint{1.768958in}{2.480519in}}{\pgfqpoint{1.764568in}{2.469920in}}{\pgfqpoint{1.764568in}{2.458870in}}%
\pgfpathcurveto{\pgfqpoint{1.764568in}{2.447819in}}{\pgfqpoint{1.768958in}{2.437220in}}{\pgfqpoint{1.776772in}{2.429407in}}%
\pgfpathcurveto{\pgfqpoint{1.784585in}{2.421593in}}{\pgfqpoint{1.795184in}{2.417203in}}{\pgfqpoint{1.806234in}{2.417203in}}%
\pgfpathclose%
\pgfusepath{stroke,fill}%
\end{pgfscope}%
\begin{pgfscope}%
\pgfpathrectangle{\pgfqpoint{0.787074in}{0.548769in}}{\pgfqpoint{5.062926in}{3.102590in}}%
\pgfusepath{clip}%
\pgfsetbuttcap%
\pgfsetroundjoin%
\definecolor{currentfill}{rgb}{1.000000,0.498039,0.054902}%
\pgfsetfillcolor{currentfill}%
\pgfsetlinewidth{1.003750pt}%
\definecolor{currentstroke}{rgb}{1.000000,0.498039,0.054902}%
\pgfsetstrokecolor{currentstroke}%
\pgfsetdash{}{0pt}%
\pgfpathmoveto{\pgfqpoint{1.608977in}{2.967508in}}%
\pgfpathcurveto{\pgfqpoint{1.620028in}{2.967508in}}{\pgfqpoint{1.630627in}{2.971899in}}{\pgfqpoint{1.638440in}{2.979712in}}%
\pgfpathcurveto{\pgfqpoint{1.646254in}{2.987526in}}{\pgfqpoint{1.650644in}{2.998125in}}{\pgfqpoint{1.650644in}{3.009175in}}%
\pgfpathcurveto{\pgfqpoint{1.650644in}{3.020225in}}{\pgfqpoint{1.646254in}{3.030824in}}{\pgfqpoint{1.638440in}{3.038638in}}%
\pgfpathcurveto{\pgfqpoint{1.630627in}{3.046451in}}{\pgfqpoint{1.620028in}{3.050842in}}{\pgfqpoint{1.608977in}{3.050842in}}%
\pgfpathcurveto{\pgfqpoint{1.597927in}{3.050842in}}{\pgfqpoint{1.587328in}{3.046451in}}{\pgfqpoint{1.579515in}{3.038638in}}%
\pgfpathcurveto{\pgfqpoint{1.571701in}{3.030824in}}{\pgfqpoint{1.567311in}{3.020225in}}{\pgfqpoint{1.567311in}{3.009175in}}%
\pgfpathcurveto{\pgfqpoint{1.567311in}{2.998125in}}{\pgfqpoint{1.571701in}{2.987526in}}{\pgfqpoint{1.579515in}{2.979712in}}%
\pgfpathcurveto{\pgfqpoint{1.587328in}{2.971899in}}{\pgfqpoint{1.597927in}{2.967508in}}{\pgfqpoint{1.608977in}{2.967508in}}%
\pgfpathclose%
\pgfusepath{stroke,fill}%
\end{pgfscope}%
\begin{pgfscope}%
\pgfpathrectangle{\pgfqpoint{0.787074in}{0.548769in}}{\pgfqpoint{5.062926in}{3.102590in}}%
\pgfusepath{clip}%
\pgfsetbuttcap%
\pgfsetroundjoin%
\definecolor{currentfill}{rgb}{1.000000,0.498039,0.054902}%
\pgfsetfillcolor{currentfill}%
\pgfsetlinewidth{1.003750pt}%
\definecolor{currentstroke}{rgb}{1.000000,0.498039,0.054902}%
\pgfsetstrokecolor{currentstroke}%
\pgfsetdash{}{0pt}%
\pgfpathmoveto{\pgfqpoint{1.806234in}{2.421885in}}%
\pgfpathcurveto{\pgfqpoint{1.817284in}{2.421885in}}{\pgfqpoint{1.827883in}{2.426275in}}{\pgfqpoint{1.835697in}{2.434089in}}%
\pgfpathcurveto{\pgfqpoint{1.843511in}{2.441902in}}{\pgfqpoint{1.847901in}{2.452501in}}{\pgfqpoint{1.847901in}{2.463551in}}%
\pgfpathcurveto{\pgfqpoint{1.847901in}{2.474601in}}{\pgfqpoint{1.843511in}{2.485201in}}{\pgfqpoint{1.835697in}{2.493014in}}%
\pgfpathcurveto{\pgfqpoint{1.827883in}{2.500828in}}{\pgfqpoint{1.817284in}{2.505218in}}{\pgfqpoint{1.806234in}{2.505218in}}%
\pgfpathcurveto{\pgfqpoint{1.795184in}{2.505218in}}{\pgfqpoint{1.784585in}{2.500828in}}{\pgfqpoint{1.776772in}{2.493014in}}%
\pgfpathcurveto{\pgfqpoint{1.768958in}{2.485201in}}{\pgfqpoint{1.764568in}{2.474601in}}{\pgfqpoint{1.764568in}{2.463551in}}%
\pgfpathcurveto{\pgfqpoint{1.764568in}{2.452501in}}{\pgfqpoint{1.768958in}{2.441902in}}{\pgfqpoint{1.776772in}{2.434089in}}%
\pgfpathcurveto{\pgfqpoint{1.784585in}{2.426275in}}{\pgfqpoint{1.795184in}{2.421885in}}{\pgfqpoint{1.806234in}{2.421885in}}%
\pgfpathclose%
\pgfusepath{stroke,fill}%
\end{pgfscope}%
\begin{pgfscope}%
\pgfpathrectangle{\pgfqpoint{0.787074in}{0.548769in}}{\pgfqpoint{5.062926in}{3.102590in}}%
\pgfusepath{clip}%
\pgfsetbuttcap%
\pgfsetroundjoin%
\definecolor{currentfill}{rgb}{1.000000,0.498039,0.054902}%
\pgfsetfillcolor{currentfill}%
\pgfsetlinewidth{1.003750pt}%
\definecolor{currentstroke}{rgb}{1.000000,0.498039,0.054902}%
\pgfsetstrokecolor{currentstroke}%
\pgfsetdash{}{0pt}%
\pgfpathmoveto{\pgfqpoint{3.713051in}{2.463960in}}%
\pgfpathcurveto{\pgfqpoint{3.724101in}{2.463960in}}{\pgfqpoint{3.734700in}{2.468350in}}{\pgfqpoint{3.742513in}{2.476164in}}%
\pgfpathcurveto{\pgfqpoint{3.750327in}{2.483977in}}{\pgfqpoint{3.754717in}{2.494576in}}{\pgfqpoint{3.754717in}{2.505626in}}%
\pgfpathcurveto{\pgfqpoint{3.754717in}{2.516677in}}{\pgfqpoint{3.750327in}{2.527276in}}{\pgfqpoint{3.742513in}{2.535089in}}%
\pgfpathcurveto{\pgfqpoint{3.734700in}{2.542903in}}{\pgfqpoint{3.724101in}{2.547293in}}{\pgfqpoint{3.713051in}{2.547293in}}%
\pgfpathcurveto{\pgfqpoint{3.702001in}{2.547293in}}{\pgfqpoint{3.691401in}{2.542903in}}{\pgfqpoint{3.683588in}{2.535089in}}%
\pgfpathcurveto{\pgfqpoint{3.675774in}{2.527276in}}{\pgfqpoint{3.671384in}{2.516677in}}{\pgfqpoint{3.671384in}{2.505626in}}%
\pgfpathcurveto{\pgfqpoint{3.671384in}{2.494576in}}{\pgfqpoint{3.675774in}{2.483977in}}{\pgfqpoint{3.683588in}{2.476164in}}%
\pgfpathcurveto{\pgfqpoint{3.691401in}{2.468350in}}{\pgfqpoint{3.702001in}{2.463960in}}{\pgfqpoint{3.713051in}{2.463960in}}%
\pgfpathclose%
\pgfusepath{stroke,fill}%
\end{pgfscope}%
\begin{pgfscope}%
\pgfpathrectangle{\pgfqpoint{0.787074in}{0.548769in}}{\pgfqpoint{5.062926in}{3.102590in}}%
\pgfusepath{clip}%
\pgfsetbuttcap%
\pgfsetroundjoin%
\definecolor{currentfill}{rgb}{1.000000,0.498039,0.054902}%
\pgfsetfillcolor{currentfill}%
\pgfsetlinewidth{1.003750pt}%
\definecolor{currentstroke}{rgb}{1.000000,0.498039,0.054902}%
\pgfsetstrokecolor{currentstroke}%
\pgfsetdash{}{0pt}%
\pgfpathmoveto{\pgfqpoint{2.726766in}{2.012662in}}%
\pgfpathcurveto{\pgfqpoint{2.737816in}{2.012662in}}{\pgfqpoint{2.748415in}{2.017052in}}{\pgfqpoint{2.756229in}{2.024866in}}%
\pgfpathcurveto{\pgfqpoint{2.764043in}{2.032679in}}{\pgfqpoint{2.768433in}{2.043278in}}{\pgfqpoint{2.768433in}{2.054328in}}%
\pgfpathcurveto{\pgfqpoint{2.768433in}{2.065378in}}{\pgfqpoint{2.764043in}{2.075977in}}{\pgfqpoint{2.756229in}{2.083791in}}%
\pgfpathcurveto{\pgfqpoint{2.748415in}{2.091605in}}{\pgfqpoint{2.737816in}{2.095995in}}{\pgfqpoint{2.726766in}{2.095995in}}%
\pgfpathcurveto{\pgfqpoint{2.715716in}{2.095995in}}{\pgfqpoint{2.705117in}{2.091605in}}{\pgfqpoint{2.697304in}{2.083791in}}%
\pgfpathcurveto{\pgfqpoint{2.689490in}{2.075977in}}{\pgfqpoint{2.685100in}{2.065378in}}{\pgfqpoint{2.685100in}{2.054328in}}%
\pgfpathcurveto{\pgfqpoint{2.685100in}{2.043278in}}{\pgfqpoint{2.689490in}{2.032679in}}{\pgfqpoint{2.697304in}{2.024866in}}%
\pgfpathcurveto{\pgfqpoint{2.705117in}{2.017052in}}{\pgfqpoint{2.715716in}{2.012662in}}{\pgfqpoint{2.726766in}{2.012662in}}%
\pgfpathclose%
\pgfusepath{stroke,fill}%
\end{pgfscope}%
\begin{pgfscope}%
\pgfpathrectangle{\pgfqpoint{0.787074in}{0.548769in}}{\pgfqpoint{5.062926in}{3.102590in}}%
\pgfusepath{clip}%
\pgfsetbuttcap%
\pgfsetroundjoin%
\definecolor{currentfill}{rgb}{1.000000,0.498039,0.054902}%
\pgfsetfillcolor{currentfill}%
\pgfsetlinewidth{1.003750pt}%
\definecolor{currentstroke}{rgb}{1.000000,0.498039,0.054902}%
\pgfsetstrokecolor{currentstroke}%
\pgfsetdash{}{0pt}%
\pgfpathmoveto{\pgfqpoint{2.003491in}{1.794353in}}%
\pgfpathcurveto{\pgfqpoint{2.014541in}{1.794353in}}{\pgfqpoint{2.025140in}{1.798743in}}{\pgfqpoint{2.032954in}{1.806557in}}%
\pgfpathcurveto{\pgfqpoint{2.040768in}{1.814370in}}{\pgfqpoint{2.045158in}{1.824969in}}{\pgfqpoint{2.045158in}{1.836019in}}%
\pgfpathcurveto{\pgfqpoint{2.045158in}{1.847070in}}{\pgfqpoint{2.040768in}{1.857669in}}{\pgfqpoint{2.032954in}{1.865482in}}%
\pgfpathcurveto{\pgfqpoint{2.025140in}{1.873296in}}{\pgfqpoint{2.014541in}{1.877686in}}{\pgfqpoint{2.003491in}{1.877686in}}%
\pgfpathcurveto{\pgfqpoint{1.992441in}{1.877686in}}{\pgfqpoint{1.981842in}{1.873296in}}{\pgfqpoint{1.974028in}{1.865482in}}%
\pgfpathcurveto{\pgfqpoint{1.966215in}{1.857669in}}{\pgfqpoint{1.961824in}{1.847070in}}{\pgfqpoint{1.961824in}{1.836019in}}%
\pgfpathcurveto{\pgfqpoint{1.961824in}{1.824969in}}{\pgfqpoint{1.966215in}{1.814370in}}{\pgfqpoint{1.974028in}{1.806557in}}%
\pgfpathcurveto{\pgfqpoint{1.981842in}{1.798743in}}{\pgfqpoint{1.992441in}{1.794353in}}{\pgfqpoint{2.003491in}{1.794353in}}%
\pgfpathclose%
\pgfusepath{stroke,fill}%
\end{pgfscope}%
\begin{pgfscope}%
\pgfpathrectangle{\pgfqpoint{0.787074in}{0.548769in}}{\pgfqpoint{5.062926in}{3.102590in}}%
\pgfusepath{clip}%
\pgfsetbuttcap%
\pgfsetroundjoin%
\definecolor{currentfill}{rgb}{1.000000,0.498039,0.054902}%
\pgfsetfillcolor{currentfill}%
\pgfsetlinewidth{1.003750pt}%
\definecolor{currentstroke}{rgb}{1.000000,0.498039,0.054902}%
\pgfsetstrokecolor{currentstroke}%
\pgfsetdash{}{0pt}%
\pgfpathmoveto{\pgfqpoint{1.740482in}{2.149173in}}%
\pgfpathcurveto{\pgfqpoint{1.751532in}{2.149173in}}{\pgfqpoint{1.762131in}{2.153563in}}{\pgfqpoint{1.769945in}{2.161376in}}%
\pgfpathcurveto{\pgfqpoint{1.777758in}{2.169190in}}{\pgfqpoint{1.782149in}{2.179789in}}{\pgfqpoint{1.782149in}{2.190839in}}%
\pgfpathcurveto{\pgfqpoint{1.782149in}{2.201889in}}{\pgfqpoint{1.777758in}{2.212488in}}{\pgfqpoint{1.769945in}{2.220302in}}%
\pgfpathcurveto{\pgfqpoint{1.762131in}{2.228116in}}{\pgfqpoint{1.751532in}{2.232506in}}{\pgfqpoint{1.740482in}{2.232506in}}%
\pgfpathcurveto{\pgfqpoint{1.729432in}{2.232506in}}{\pgfqpoint{1.718833in}{2.228116in}}{\pgfqpoint{1.711019in}{2.220302in}}%
\pgfpathcurveto{\pgfqpoint{1.703206in}{2.212488in}}{\pgfqpoint{1.698815in}{2.201889in}}{\pgfqpoint{1.698815in}{2.190839in}}%
\pgfpathcurveto{\pgfqpoint{1.698815in}{2.179789in}}{\pgfqpoint{1.703206in}{2.169190in}}{\pgfqpoint{1.711019in}{2.161376in}}%
\pgfpathcurveto{\pgfqpoint{1.718833in}{2.153563in}}{\pgfqpoint{1.729432in}{2.149173in}}{\pgfqpoint{1.740482in}{2.149173in}}%
\pgfpathclose%
\pgfusepath{stroke,fill}%
\end{pgfscope}%
\begin{pgfscope}%
\pgfpathrectangle{\pgfqpoint{0.787074in}{0.548769in}}{\pgfqpoint{5.062926in}{3.102590in}}%
\pgfusepath{clip}%
\pgfsetbuttcap%
\pgfsetroundjoin%
\definecolor{currentfill}{rgb}{1.000000,0.498039,0.054902}%
\pgfsetfillcolor{currentfill}%
\pgfsetlinewidth{1.003750pt}%
\definecolor{currentstroke}{rgb}{1.000000,0.498039,0.054902}%
\pgfsetstrokecolor{currentstroke}%
\pgfsetdash{}{0pt}%
\pgfpathmoveto{\pgfqpoint{2.134996in}{2.551840in}}%
\pgfpathcurveto{\pgfqpoint{2.146046in}{2.551840in}}{\pgfqpoint{2.156645in}{2.556230in}}{\pgfqpoint{2.164459in}{2.564044in}}%
\pgfpathcurveto{\pgfqpoint{2.172272in}{2.571857in}}{\pgfqpoint{2.176662in}{2.582456in}}{\pgfqpoint{2.176662in}{2.593506in}}%
\pgfpathcurveto{\pgfqpoint{2.176662in}{2.604557in}}{\pgfqpoint{2.172272in}{2.615156in}}{\pgfqpoint{2.164459in}{2.622969in}}%
\pgfpathcurveto{\pgfqpoint{2.156645in}{2.630783in}}{\pgfqpoint{2.146046in}{2.635173in}}{\pgfqpoint{2.134996in}{2.635173in}}%
\pgfpathcurveto{\pgfqpoint{2.123946in}{2.635173in}}{\pgfqpoint{2.113347in}{2.630783in}}{\pgfqpoint{2.105533in}{2.622969in}}%
\pgfpathcurveto{\pgfqpoint{2.097719in}{2.615156in}}{\pgfqpoint{2.093329in}{2.604557in}}{\pgfqpoint{2.093329in}{2.593506in}}%
\pgfpathcurveto{\pgfqpoint{2.093329in}{2.582456in}}{\pgfqpoint{2.097719in}{2.571857in}}{\pgfqpoint{2.105533in}{2.564044in}}%
\pgfpathcurveto{\pgfqpoint{2.113347in}{2.556230in}}{\pgfqpoint{2.123946in}{2.551840in}}{\pgfqpoint{2.134996in}{2.551840in}}%
\pgfpathclose%
\pgfusepath{stroke,fill}%
\end{pgfscope}%
\begin{pgfscope}%
\pgfpathrectangle{\pgfqpoint{0.787074in}{0.548769in}}{\pgfqpoint{5.062926in}{3.102590in}}%
\pgfusepath{clip}%
\pgfsetbuttcap%
\pgfsetroundjoin%
\definecolor{currentfill}{rgb}{1.000000,0.498039,0.054902}%
\pgfsetfillcolor{currentfill}%
\pgfsetlinewidth{1.003750pt}%
\definecolor{currentstroke}{rgb}{1.000000,0.498039,0.054902}%
\pgfsetstrokecolor{currentstroke}%
\pgfsetdash{}{0pt}%
\pgfpathmoveto{\pgfqpoint{1.806234in}{2.187660in}}%
\pgfpathcurveto{\pgfqpoint{1.817284in}{2.187660in}}{\pgfqpoint{1.827883in}{2.192050in}}{\pgfqpoint{1.835697in}{2.199864in}}%
\pgfpathcurveto{\pgfqpoint{1.843511in}{2.207677in}}{\pgfqpoint{1.847901in}{2.218276in}}{\pgfqpoint{1.847901in}{2.229326in}}%
\pgfpathcurveto{\pgfqpoint{1.847901in}{2.240377in}}{\pgfqpoint{1.843511in}{2.250976in}}{\pgfqpoint{1.835697in}{2.258789in}}%
\pgfpathcurveto{\pgfqpoint{1.827883in}{2.266603in}}{\pgfqpoint{1.817284in}{2.270993in}}{\pgfqpoint{1.806234in}{2.270993in}}%
\pgfpathcurveto{\pgfqpoint{1.795184in}{2.270993in}}{\pgfqpoint{1.784585in}{2.266603in}}{\pgfqpoint{1.776772in}{2.258789in}}%
\pgfpathcurveto{\pgfqpoint{1.768958in}{2.250976in}}{\pgfqpoint{1.764568in}{2.240377in}}{\pgfqpoint{1.764568in}{2.229326in}}%
\pgfpathcurveto{\pgfqpoint{1.764568in}{2.218276in}}{\pgfqpoint{1.768958in}{2.207677in}}{\pgfqpoint{1.776772in}{2.199864in}}%
\pgfpathcurveto{\pgfqpoint{1.784585in}{2.192050in}}{\pgfqpoint{1.795184in}{2.187660in}}{\pgfqpoint{1.806234in}{2.187660in}}%
\pgfpathclose%
\pgfusepath{stroke,fill}%
\end{pgfscope}%
\begin{pgfscope}%
\pgfpathrectangle{\pgfqpoint{0.787074in}{0.548769in}}{\pgfqpoint{5.062926in}{3.102590in}}%
\pgfusepath{clip}%
\pgfsetbuttcap%
\pgfsetroundjoin%
\definecolor{currentfill}{rgb}{1.000000,0.498039,0.054902}%
\pgfsetfillcolor{currentfill}%
\pgfsetlinewidth{1.003750pt}%
\definecolor{currentstroke}{rgb}{1.000000,0.498039,0.054902}%
\pgfsetstrokecolor{currentstroke}%
\pgfsetdash{}{0pt}%
\pgfpathmoveto{\pgfqpoint{1.871987in}{2.846483in}}%
\pgfpathcurveto{\pgfqpoint{1.883037in}{2.846483in}}{\pgfqpoint{1.893636in}{2.850874in}}{\pgfqpoint{1.901449in}{2.858687in}}%
\pgfpathcurveto{\pgfqpoint{1.909263in}{2.866501in}}{\pgfqpoint{1.913653in}{2.877100in}}{\pgfqpoint{1.913653in}{2.888150in}}%
\pgfpathcurveto{\pgfqpoint{1.913653in}{2.899200in}}{\pgfqpoint{1.909263in}{2.909799in}}{\pgfqpoint{1.901449in}{2.917613in}}%
\pgfpathcurveto{\pgfqpoint{1.893636in}{2.925426in}}{\pgfqpoint{1.883037in}{2.929817in}}{\pgfqpoint{1.871987in}{2.929817in}}%
\pgfpathcurveto{\pgfqpoint{1.860936in}{2.929817in}}{\pgfqpoint{1.850337in}{2.925426in}}{\pgfqpoint{1.842524in}{2.917613in}}%
\pgfpathcurveto{\pgfqpoint{1.834710in}{2.909799in}}{\pgfqpoint{1.830320in}{2.899200in}}{\pgfqpoint{1.830320in}{2.888150in}}%
\pgfpathcurveto{\pgfqpoint{1.830320in}{2.877100in}}{\pgfqpoint{1.834710in}{2.866501in}}{\pgfqpoint{1.842524in}{2.858687in}}%
\pgfpathcurveto{\pgfqpoint{1.850337in}{2.850874in}}{\pgfqpoint{1.860936in}{2.846483in}}{\pgfqpoint{1.871987in}{2.846483in}}%
\pgfpathclose%
\pgfusepath{stroke,fill}%
\end{pgfscope}%
\begin{pgfscope}%
\pgfpathrectangle{\pgfqpoint{0.787074in}{0.548769in}}{\pgfqpoint{5.062926in}{3.102590in}}%
\pgfusepath{clip}%
\pgfsetbuttcap%
\pgfsetroundjoin%
\definecolor{currentfill}{rgb}{1.000000,0.498039,0.054902}%
\pgfsetfillcolor{currentfill}%
\pgfsetlinewidth{1.003750pt}%
\definecolor{currentstroke}{rgb}{1.000000,0.498039,0.054902}%
\pgfsetstrokecolor{currentstroke}%
\pgfsetdash{}{0pt}%
\pgfpathmoveto{\pgfqpoint{2.398005in}{2.316063in}}%
\pgfpathcurveto{\pgfqpoint{2.409055in}{2.316063in}}{\pgfqpoint{2.419654in}{2.320453in}}{\pgfqpoint{2.427468in}{2.328267in}}%
\pgfpathcurveto{\pgfqpoint{2.435281in}{2.336080in}}{\pgfqpoint{2.439672in}{2.346679in}}{\pgfqpoint{2.439672in}{2.357729in}}%
\pgfpathcurveto{\pgfqpoint{2.439672in}{2.368779in}}{\pgfqpoint{2.435281in}{2.379379in}}{\pgfqpoint{2.427468in}{2.387192in}}%
\pgfpathcurveto{\pgfqpoint{2.419654in}{2.395006in}}{\pgfqpoint{2.409055in}{2.399396in}}{\pgfqpoint{2.398005in}{2.399396in}}%
\pgfpathcurveto{\pgfqpoint{2.386955in}{2.399396in}}{\pgfqpoint{2.376356in}{2.395006in}}{\pgfqpoint{2.368542in}{2.387192in}}%
\pgfpathcurveto{\pgfqpoint{2.360728in}{2.379379in}}{\pgfqpoint{2.356338in}{2.368779in}}{\pgfqpoint{2.356338in}{2.357729in}}%
\pgfpathcurveto{\pgfqpoint{2.356338in}{2.346679in}}{\pgfqpoint{2.360728in}{2.336080in}}{\pgfqpoint{2.368542in}{2.328267in}}%
\pgfpathcurveto{\pgfqpoint{2.376356in}{2.320453in}}{\pgfqpoint{2.386955in}{2.316063in}}{\pgfqpoint{2.398005in}{2.316063in}}%
\pgfpathclose%
\pgfusepath{stroke,fill}%
\end{pgfscope}%
\begin{pgfscope}%
\pgfpathrectangle{\pgfqpoint{0.787074in}{0.548769in}}{\pgfqpoint{5.062926in}{3.102590in}}%
\pgfusepath{clip}%
\pgfsetbuttcap%
\pgfsetroundjoin%
\definecolor{currentfill}{rgb}{1.000000,0.498039,0.054902}%
\pgfsetfillcolor{currentfill}%
\pgfsetlinewidth{1.003750pt}%
\definecolor{currentstroke}{rgb}{1.000000,0.498039,0.054902}%
\pgfsetstrokecolor{currentstroke}%
\pgfsetdash{}{0pt}%
\pgfpathmoveto{\pgfqpoint{2.398005in}{1.585094in}}%
\pgfpathcurveto{\pgfqpoint{2.409055in}{1.585094in}}{\pgfqpoint{2.419654in}{1.589485in}}{\pgfqpoint{2.427468in}{1.597298in}}%
\pgfpathcurveto{\pgfqpoint{2.435281in}{1.605112in}}{\pgfqpoint{2.439672in}{1.615711in}}{\pgfqpoint{2.439672in}{1.626761in}}%
\pgfpathcurveto{\pgfqpoint{2.439672in}{1.637811in}}{\pgfqpoint{2.435281in}{1.648410in}}{\pgfqpoint{2.427468in}{1.656224in}}%
\pgfpathcurveto{\pgfqpoint{2.419654in}{1.664037in}}{\pgfqpoint{2.409055in}{1.668428in}}{\pgfqpoint{2.398005in}{1.668428in}}%
\pgfpathcurveto{\pgfqpoint{2.386955in}{1.668428in}}{\pgfqpoint{2.376356in}{1.664037in}}{\pgfqpoint{2.368542in}{1.656224in}}%
\pgfpathcurveto{\pgfqpoint{2.360728in}{1.648410in}}{\pgfqpoint{2.356338in}{1.637811in}}{\pgfqpoint{2.356338in}{1.626761in}}%
\pgfpathcurveto{\pgfqpoint{2.356338in}{1.615711in}}{\pgfqpoint{2.360728in}{1.605112in}}{\pgfqpoint{2.368542in}{1.597298in}}%
\pgfpathcurveto{\pgfqpoint{2.376356in}{1.589485in}}{\pgfqpoint{2.386955in}{1.585094in}}{\pgfqpoint{2.398005in}{1.585094in}}%
\pgfpathclose%
\pgfusepath{stroke,fill}%
\end{pgfscope}%
\begin{pgfscope}%
\pgfpathrectangle{\pgfqpoint{0.787074in}{0.548769in}}{\pgfqpoint{5.062926in}{3.102590in}}%
\pgfusepath{clip}%
\pgfsetbuttcap%
\pgfsetroundjoin%
\definecolor{currentfill}{rgb}{1.000000,0.498039,0.054902}%
\pgfsetfillcolor{currentfill}%
\pgfsetlinewidth{1.003750pt}%
\definecolor{currentstroke}{rgb}{1.000000,0.498039,0.054902}%
\pgfsetstrokecolor{currentstroke}%
\pgfsetdash{}{0pt}%
\pgfpathmoveto{\pgfqpoint{3.121280in}{1.803770in}}%
\pgfpathcurveto{\pgfqpoint{3.132330in}{1.803770in}}{\pgfqpoint{3.142929in}{1.808160in}}{\pgfqpoint{3.150743in}{1.815974in}}%
\pgfpathcurveto{\pgfqpoint{3.158556in}{1.823787in}}{\pgfqpoint{3.162947in}{1.834386in}}{\pgfqpoint{3.162947in}{1.845437in}}%
\pgfpathcurveto{\pgfqpoint{3.162947in}{1.856487in}}{\pgfqpoint{3.158556in}{1.867086in}}{\pgfqpoint{3.150743in}{1.874899in}}%
\pgfpathcurveto{\pgfqpoint{3.142929in}{1.882713in}}{\pgfqpoint{3.132330in}{1.887103in}}{\pgfqpoint{3.121280in}{1.887103in}}%
\pgfpathcurveto{\pgfqpoint{3.110230in}{1.887103in}}{\pgfqpoint{3.099631in}{1.882713in}}{\pgfqpoint{3.091817in}{1.874899in}}%
\pgfpathcurveto{\pgfqpoint{3.084004in}{1.867086in}}{\pgfqpoint{3.079613in}{1.856487in}}{\pgfqpoint{3.079613in}{1.845437in}}%
\pgfpathcurveto{\pgfqpoint{3.079613in}{1.834386in}}{\pgfqpoint{3.084004in}{1.823787in}}{\pgfqpoint{3.091817in}{1.815974in}}%
\pgfpathcurveto{\pgfqpoint{3.099631in}{1.808160in}}{\pgfqpoint{3.110230in}{1.803770in}}{\pgfqpoint{3.121280in}{1.803770in}}%
\pgfpathclose%
\pgfusepath{stroke,fill}%
\end{pgfscope}%
\begin{pgfscope}%
\pgfpathrectangle{\pgfqpoint{0.787074in}{0.548769in}}{\pgfqpoint{5.062926in}{3.102590in}}%
\pgfusepath{clip}%
\pgfsetbuttcap%
\pgfsetroundjoin%
\definecolor{currentfill}{rgb}{1.000000,0.498039,0.054902}%
\pgfsetfillcolor{currentfill}%
\pgfsetlinewidth{1.003750pt}%
\definecolor{currentstroke}{rgb}{1.000000,0.498039,0.054902}%
\pgfsetstrokecolor{currentstroke}%
\pgfsetdash{}{0pt}%
\pgfpathmoveto{\pgfqpoint{1.806234in}{2.162419in}}%
\pgfpathcurveto{\pgfqpoint{1.817284in}{2.162419in}}{\pgfqpoint{1.827883in}{2.166809in}}{\pgfqpoint{1.835697in}{2.174623in}}%
\pgfpathcurveto{\pgfqpoint{1.843511in}{2.182437in}}{\pgfqpoint{1.847901in}{2.193036in}}{\pgfqpoint{1.847901in}{2.204086in}}%
\pgfpathcurveto{\pgfqpoint{1.847901in}{2.215136in}}{\pgfqpoint{1.843511in}{2.225735in}}{\pgfqpoint{1.835697in}{2.233549in}}%
\pgfpathcurveto{\pgfqpoint{1.827883in}{2.241362in}}{\pgfqpoint{1.817284in}{2.245753in}}{\pgfqpoint{1.806234in}{2.245753in}}%
\pgfpathcurveto{\pgfqpoint{1.795184in}{2.245753in}}{\pgfqpoint{1.784585in}{2.241362in}}{\pgfqpoint{1.776772in}{2.233549in}}%
\pgfpathcurveto{\pgfqpoint{1.768958in}{2.225735in}}{\pgfqpoint{1.764568in}{2.215136in}}{\pgfqpoint{1.764568in}{2.204086in}}%
\pgfpathcurveto{\pgfqpoint{1.764568in}{2.193036in}}{\pgfqpoint{1.768958in}{2.182437in}}{\pgfqpoint{1.776772in}{2.174623in}}%
\pgfpathcurveto{\pgfqpoint{1.784585in}{2.166809in}}{\pgfqpoint{1.795184in}{2.162419in}}{\pgfqpoint{1.806234in}{2.162419in}}%
\pgfpathclose%
\pgfusepath{stroke,fill}%
\end{pgfscope}%
\begin{pgfscope}%
\pgfpathrectangle{\pgfqpoint{0.787074in}{0.548769in}}{\pgfqpoint{5.062926in}{3.102590in}}%
\pgfusepath{clip}%
\pgfsetbuttcap%
\pgfsetroundjoin%
\definecolor{currentfill}{rgb}{1.000000,0.498039,0.054902}%
\pgfsetfillcolor{currentfill}%
\pgfsetlinewidth{1.003750pt}%
\definecolor{currentstroke}{rgb}{1.000000,0.498039,0.054902}%
\pgfsetstrokecolor{currentstroke}%
\pgfsetdash{}{0pt}%
\pgfpathmoveto{\pgfqpoint{2.332253in}{3.057735in}}%
\pgfpathcurveto{\pgfqpoint{2.343303in}{3.057735in}}{\pgfqpoint{2.353902in}{3.062125in}}{\pgfqpoint{2.361715in}{3.069939in}}%
\pgfpathcurveto{\pgfqpoint{2.369529in}{3.077752in}}{\pgfqpoint{2.373919in}{3.088351in}}{\pgfqpoint{2.373919in}{3.099401in}}%
\pgfpathcurveto{\pgfqpoint{2.373919in}{3.110451in}}{\pgfqpoint{2.369529in}{3.121050in}}{\pgfqpoint{2.361715in}{3.128864in}}%
\pgfpathcurveto{\pgfqpoint{2.353902in}{3.136678in}}{\pgfqpoint{2.343303in}{3.141068in}}{\pgfqpoint{2.332253in}{3.141068in}}%
\pgfpathcurveto{\pgfqpoint{2.321202in}{3.141068in}}{\pgfqpoint{2.310603in}{3.136678in}}{\pgfqpoint{2.302790in}{3.128864in}}%
\pgfpathcurveto{\pgfqpoint{2.294976in}{3.121050in}}{\pgfqpoint{2.290586in}{3.110451in}}{\pgfqpoint{2.290586in}{3.099401in}}%
\pgfpathcurveto{\pgfqpoint{2.290586in}{3.088351in}}{\pgfqpoint{2.294976in}{3.077752in}}{\pgfqpoint{2.302790in}{3.069939in}}%
\pgfpathcurveto{\pgfqpoint{2.310603in}{3.062125in}}{\pgfqpoint{2.321202in}{3.057735in}}{\pgfqpoint{2.332253in}{3.057735in}}%
\pgfpathclose%
\pgfusepath{stroke,fill}%
\end{pgfscope}%
\begin{pgfscope}%
\pgfpathrectangle{\pgfqpoint{0.787074in}{0.548769in}}{\pgfqpoint{5.062926in}{3.102590in}}%
\pgfusepath{clip}%
\pgfsetbuttcap%
\pgfsetroundjoin%
\definecolor{currentfill}{rgb}{1.000000,0.498039,0.054902}%
\pgfsetfillcolor{currentfill}%
\pgfsetlinewidth{1.003750pt}%
\definecolor{currentstroke}{rgb}{1.000000,0.498039,0.054902}%
\pgfsetstrokecolor{currentstroke}%
\pgfsetdash{}{0pt}%
\pgfpathmoveto{\pgfqpoint{2.661014in}{2.195132in}}%
\pgfpathcurveto{\pgfqpoint{2.672064in}{2.195132in}}{\pgfqpoint{2.682663in}{2.199523in}}{\pgfqpoint{2.690477in}{2.207336in}}%
\pgfpathcurveto{\pgfqpoint{2.698290in}{2.215150in}}{\pgfqpoint{2.702681in}{2.225749in}}{\pgfqpoint{2.702681in}{2.236799in}}%
\pgfpathcurveto{\pgfqpoint{2.702681in}{2.247849in}}{\pgfqpoint{2.698290in}{2.258448in}}{\pgfqpoint{2.690477in}{2.266262in}}%
\pgfpathcurveto{\pgfqpoint{2.682663in}{2.274076in}}{\pgfqpoint{2.672064in}{2.278466in}}{\pgfqpoint{2.661014in}{2.278466in}}%
\pgfpathcurveto{\pgfqpoint{2.649964in}{2.278466in}}{\pgfqpoint{2.639365in}{2.274076in}}{\pgfqpoint{2.631551in}{2.266262in}}%
\pgfpathcurveto{\pgfqpoint{2.623738in}{2.258448in}}{\pgfqpoint{2.619347in}{2.247849in}}{\pgfqpoint{2.619347in}{2.236799in}}%
\pgfpathcurveto{\pgfqpoint{2.619347in}{2.225749in}}{\pgfqpoint{2.623738in}{2.215150in}}{\pgfqpoint{2.631551in}{2.207336in}}%
\pgfpathcurveto{\pgfqpoint{2.639365in}{2.199523in}}{\pgfqpoint{2.649964in}{2.195132in}}{\pgfqpoint{2.661014in}{2.195132in}}%
\pgfpathclose%
\pgfusepath{stroke,fill}%
\end{pgfscope}%
\begin{pgfscope}%
\pgfpathrectangle{\pgfqpoint{0.787074in}{0.548769in}}{\pgfqpoint{5.062926in}{3.102590in}}%
\pgfusepath{clip}%
\pgfsetbuttcap%
\pgfsetroundjoin%
\definecolor{currentfill}{rgb}{1.000000,0.498039,0.054902}%
\pgfsetfillcolor{currentfill}%
\pgfsetlinewidth{1.003750pt}%
\definecolor{currentstroke}{rgb}{1.000000,0.498039,0.054902}%
\pgfsetstrokecolor{currentstroke}%
\pgfsetdash{}{0pt}%
\pgfpathmoveto{\pgfqpoint{1.608977in}{2.538741in}}%
\pgfpathcurveto{\pgfqpoint{1.620028in}{2.538741in}}{\pgfqpoint{1.630627in}{2.543132in}}{\pgfqpoint{1.638440in}{2.550945in}}%
\pgfpathcurveto{\pgfqpoint{1.646254in}{2.558759in}}{\pgfqpoint{1.650644in}{2.569358in}}{\pgfqpoint{1.650644in}{2.580408in}}%
\pgfpathcurveto{\pgfqpoint{1.650644in}{2.591458in}}{\pgfqpoint{1.646254in}{2.602057in}}{\pgfqpoint{1.638440in}{2.609871in}}%
\pgfpathcurveto{\pgfqpoint{1.630627in}{2.617685in}}{\pgfqpoint{1.620028in}{2.622075in}}{\pgfqpoint{1.608977in}{2.622075in}}%
\pgfpathcurveto{\pgfqpoint{1.597927in}{2.622075in}}{\pgfqpoint{1.587328in}{2.617685in}}{\pgfqpoint{1.579515in}{2.609871in}}%
\pgfpathcurveto{\pgfqpoint{1.571701in}{2.602057in}}{\pgfqpoint{1.567311in}{2.591458in}}{\pgfqpoint{1.567311in}{2.580408in}}%
\pgfpathcurveto{\pgfqpoint{1.567311in}{2.569358in}}{\pgfqpoint{1.571701in}{2.558759in}}{\pgfqpoint{1.579515in}{2.550945in}}%
\pgfpathcurveto{\pgfqpoint{1.587328in}{2.543132in}}{\pgfqpoint{1.597927in}{2.538741in}}{\pgfqpoint{1.608977in}{2.538741in}}%
\pgfpathclose%
\pgfusepath{stroke,fill}%
\end{pgfscope}%
\begin{pgfscope}%
\pgfpathrectangle{\pgfqpoint{0.787074in}{0.548769in}}{\pgfqpoint{5.062926in}{3.102590in}}%
\pgfusepath{clip}%
\pgfsetbuttcap%
\pgfsetroundjoin%
\definecolor{currentfill}{rgb}{1.000000,0.498039,0.054902}%
\pgfsetfillcolor{currentfill}%
\pgfsetlinewidth{1.003750pt}%
\definecolor{currentstroke}{rgb}{1.000000,0.498039,0.054902}%
\pgfsetstrokecolor{currentstroke}%
\pgfsetdash{}{0pt}%
\pgfpathmoveto{\pgfqpoint{2.726766in}{2.246302in}}%
\pgfpathcurveto{\pgfqpoint{2.737816in}{2.246302in}}{\pgfqpoint{2.748415in}{2.250692in}}{\pgfqpoint{2.756229in}{2.258505in}}%
\pgfpathcurveto{\pgfqpoint{2.764043in}{2.266319in}}{\pgfqpoint{2.768433in}{2.276918in}}{\pgfqpoint{2.768433in}{2.287968in}}%
\pgfpathcurveto{\pgfqpoint{2.768433in}{2.299018in}}{\pgfqpoint{2.764043in}{2.309617in}}{\pgfqpoint{2.756229in}{2.317431in}}%
\pgfpathcurveto{\pgfqpoint{2.748415in}{2.325245in}}{\pgfqpoint{2.737816in}{2.329635in}}{\pgfqpoint{2.726766in}{2.329635in}}%
\pgfpathcurveto{\pgfqpoint{2.715716in}{2.329635in}}{\pgfqpoint{2.705117in}{2.325245in}}{\pgfqpoint{2.697304in}{2.317431in}}%
\pgfpathcurveto{\pgfqpoint{2.689490in}{2.309617in}}{\pgfqpoint{2.685100in}{2.299018in}}{\pgfqpoint{2.685100in}{2.287968in}}%
\pgfpathcurveto{\pgfqpoint{2.685100in}{2.276918in}}{\pgfqpoint{2.689490in}{2.266319in}}{\pgfqpoint{2.697304in}{2.258505in}}%
\pgfpathcurveto{\pgfqpoint{2.705117in}{2.250692in}}{\pgfqpoint{2.715716in}{2.246302in}}{\pgfqpoint{2.726766in}{2.246302in}}%
\pgfpathclose%
\pgfusepath{stroke,fill}%
\end{pgfscope}%
\begin{pgfscope}%
\pgfpathrectangle{\pgfqpoint{0.787074in}{0.548769in}}{\pgfqpoint{5.062926in}{3.102590in}}%
\pgfusepath{clip}%
\pgfsetbuttcap%
\pgfsetroundjoin%
\definecolor{currentfill}{rgb}{1.000000,0.498039,0.054902}%
\pgfsetfillcolor{currentfill}%
\pgfsetlinewidth{1.003750pt}%
\definecolor{currentstroke}{rgb}{1.000000,0.498039,0.054902}%
\pgfsetstrokecolor{currentstroke}%
\pgfsetdash{}{0pt}%
\pgfpathmoveto{\pgfqpoint{2.858271in}{2.224849in}}%
\pgfpathcurveto{\pgfqpoint{2.869321in}{2.224849in}}{\pgfqpoint{2.879920in}{2.229239in}}{\pgfqpoint{2.887734in}{2.237053in}}%
\pgfpathcurveto{\pgfqpoint{2.895547in}{2.244866in}}{\pgfqpoint{2.899938in}{2.255465in}}{\pgfqpoint{2.899938in}{2.266515in}}%
\pgfpathcurveto{\pgfqpoint{2.899938in}{2.277565in}}{\pgfqpoint{2.895547in}{2.288165in}}{\pgfqpoint{2.887734in}{2.295978in}}%
\pgfpathcurveto{\pgfqpoint{2.879920in}{2.303792in}}{\pgfqpoint{2.869321in}{2.308182in}}{\pgfqpoint{2.858271in}{2.308182in}}%
\pgfpathcurveto{\pgfqpoint{2.847221in}{2.308182in}}{\pgfqpoint{2.836622in}{2.303792in}}{\pgfqpoint{2.828808in}{2.295978in}}%
\pgfpathcurveto{\pgfqpoint{2.820995in}{2.288165in}}{\pgfqpoint{2.816604in}{2.277565in}}{\pgfqpoint{2.816604in}{2.266515in}}%
\pgfpathcurveto{\pgfqpoint{2.816604in}{2.255465in}}{\pgfqpoint{2.820995in}{2.244866in}}{\pgfqpoint{2.828808in}{2.237053in}}%
\pgfpathcurveto{\pgfqpoint{2.836622in}{2.229239in}}{\pgfqpoint{2.847221in}{2.224849in}}{\pgfqpoint{2.858271in}{2.224849in}}%
\pgfpathclose%
\pgfusepath{stroke,fill}%
\end{pgfscope}%
\begin{pgfscope}%
\pgfpathrectangle{\pgfqpoint{0.787074in}{0.548769in}}{\pgfqpoint{5.062926in}{3.102590in}}%
\pgfusepath{clip}%
\pgfsetbuttcap%
\pgfsetroundjoin%
\definecolor{currentfill}{rgb}{1.000000,0.498039,0.054902}%
\pgfsetfillcolor{currentfill}%
\pgfsetlinewidth{1.003750pt}%
\definecolor{currentstroke}{rgb}{1.000000,0.498039,0.054902}%
\pgfsetstrokecolor{currentstroke}%
\pgfsetdash{}{0pt}%
\pgfpathmoveto{\pgfqpoint{2.332253in}{2.370615in}}%
\pgfpathcurveto{\pgfqpoint{2.343303in}{2.370615in}}{\pgfqpoint{2.353902in}{2.375005in}}{\pgfqpoint{2.361715in}{2.382818in}}%
\pgfpathcurveto{\pgfqpoint{2.369529in}{2.390632in}}{\pgfqpoint{2.373919in}{2.401231in}}{\pgfqpoint{2.373919in}{2.412281in}}%
\pgfpathcurveto{\pgfqpoint{2.373919in}{2.423331in}}{\pgfqpoint{2.369529in}{2.433930in}}{\pgfqpoint{2.361715in}{2.441744in}}%
\pgfpathcurveto{\pgfqpoint{2.353902in}{2.449558in}}{\pgfqpoint{2.343303in}{2.453948in}}{\pgfqpoint{2.332253in}{2.453948in}}%
\pgfpathcurveto{\pgfqpoint{2.321202in}{2.453948in}}{\pgfqpoint{2.310603in}{2.449558in}}{\pgfqpoint{2.302790in}{2.441744in}}%
\pgfpathcurveto{\pgfqpoint{2.294976in}{2.433930in}}{\pgfqpoint{2.290586in}{2.423331in}}{\pgfqpoint{2.290586in}{2.412281in}}%
\pgfpathcurveto{\pgfqpoint{2.290586in}{2.401231in}}{\pgfqpoint{2.294976in}{2.390632in}}{\pgfqpoint{2.302790in}{2.382818in}}%
\pgfpathcurveto{\pgfqpoint{2.310603in}{2.375005in}}{\pgfqpoint{2.321202in}{2.370615in}}{\pgfqpoint{2.332253in}{2.370615in}}%
\pgfpathclose%
\pgfusepath{stroke,fill}%
\end{pgfscope}%
\begin{pgfscope}%
\pgfpathrectangle{\pgfqpoint{0.787074in}{0.548769in}}{\pgfqpoint{5.062926in}{3.102590in}}%
\pgfusepath{clip}%
\pgfsetbuttcap%
\pgfsetroundjoin%
\definecolor{currentfill}{rgb}{1.000000,0.498039,0.054902}%
\pgfsetfillcolor{currentfill}%
\pgfsetlinewidth{1.003750pt}%
\definecolor{currentstroke}{rgb}{1.000000,0.498039,0.054902}%
\pgfsetstrokecolor{currentstroke}%
\pgfsetdash{}{0pt}%
\pgfpathmoveto{\pgfqpoint{2.003491in}{1.974526in}}%
\pgfpathcurveto{\pgfqpoint{2.014541in}{1.974526in}}{\pgfqpoint{2.025140in}{1.978916in}}{\pgfqpoint{2.032954in}{1.986730in}}%
\pgfpathcurveto{\pgfqpoint{2.040768in}{1.994543in}}{\pgfqpoint{2.045158in}{2.005142in}}{\pgfqpoint{2.045158in}{2.016192in}}%
\pgfpathcurveto{\pgfqpoint{2.045158in}{2.027242in}}{\pgfqpoint{2.040768in}{2.037842in}}{\pgfqpoint{2.032954in}{2.045655in}}%
\pgfpathcurveto{\pgfqpoint{2.025140in}{2.053469in}}{\pgfqpoint{2.014541in}{2.057859in}}{\pgfqpoint{2.003491in}{2.057859in}}%
\pgfpathcurveto{\pgfqpoint{1.992441in}{2.057859in}}{\pgfqpoint{1.981842in}{2.053469in}}{\pgfqpoint{1.974028in}{2.045655in}}%
\pgfpathcurveto{\pgfqpoint{1.966215in}{2.037842in}}{\pgfqpoint{1.961824in}{2.027242in}}{\pgfqpoint{1.961824in}{2.016192in}}%
\pgfpathcurveto{\pgfqpoint{1.961824in}{2.005142in}}{\pgfqpoint{1.966215in}{1.994543in}}{\pgfqpoint{1.974028in}{1.986730in}}%
\pgfpathcurveto{\pgfqpoint{1.981842in}{1.978916in}}{\pgfqpoint{1.992441in}{1.974526in}}{\pgfqpoint{2.003491in}{1.974526in}}%
\pgfpathclose%
\pgfusepath{stroke,fill}%
\end{pgfscope}%
\begin{pgfscope}%
\pgfpathrectangle{\pgfqpoint{0.787074in}{0.548769in}}{\pgfqpoint{5.062926in}{3.102590in}}%
\pgfusepath{clip}%
\pgfsetbuttcap%
\pgfsetroundjoin%
\definecolor{currentfill}{rgb}{1.000000,0.498039,0.054902}%
\pgfsetfillcolor{currentfill}%
\pgfsetlinewidth{1.003750pt}%
\definecolor{currentstroke}{rgb}{1.000000,0.498039,0.054902}%
\pgfsetstrokecolor{currentstroke}%
\pgfsetdash{}{0pt}%
\pgfpathmoveto{\pgfqpoint{1.543225in}{2.689957in}}%
\pgfpathcurveto{\pgfqpoint{1.554275in}{2.689957in}}{\pgfqpoint{1.564874in}{2.694347in}}{\pgfqpoint{1.572688in}{2.702161in}}%
\pgfpathcurveto{\pgfqpoint{1.580502in}{2.709974in}}{\pgfqpoint{1.584892in}{2.720573in}}{\pgfqpoint{1.584892in}{2.731623in}}%
\pgfpathcurveto{\pgfqpoint{1.584892in}{2.742674in}}{\pgfqpoint{1.580502in}{2.753273in}}{\pgfqpoint{1.572688in}{2.761086in}}%
\pgfpathcurveto{\pgfqpoint{1.564874in}{2.768900in}}{\pgfqpoint{1.554275in}{2.773290in}}{\pgfqpoint{1.543225in}{2.773290in}}%
\pgfpathcurveto{\pgfqpoint{1.532175in}{2.773290in}}{\pgfqpoint{1.521576in}{2.768900in}}{\pgfqpoint{1.513762in}{2.761086in}}%
\pgfpathcurveto{\pgfqpoint{1.505949in}{2.753273in}}{\pgfqpoint{1.501558in}{2.742674in}}{\pgfqpoint{1.501558in}{2.731623in}}%
\pgfpathcurveto{\pgfqpoint{1.501558in}{2.720573in}}{\pgfqpoint{1.505949in}{2.709974in}}{\pgfqpoint{1.513762in}{2.702161in}}%
\pgfpathcurveto{\pgfqpoint{1.521576in}{2.694347in}}{\pgfqpoint{1.532175in}{2.689957in}}{\pgfqpoint{1.543225in}{2.689957in}}%
\pgfpathclose%
\pgfusepath{stroke,fill}%
\end{pgfscope}%
\begin{pgfscope}%
\pgfpathrectangle{\pgfqpoint{0.787074in}{0.548769in}}{\pgfqpoint{5.062926in}{3.102590in}}%
\pgfusepath{clip}%
\pgfsetbuttcap%
\pgfsetroundjoin%
\definecolor{currentfill}{rgb}{1.000000,0.498039,0.054902}%
\pgfsetfillcolor{currentfill}%
\pgfsetlinewidth{1.003750pt}%
\definecolor{currentstroke}{rgb}{1.000000,0.498039,0.054902}%
\pgfsetstrokecolor{currentstroke}%
\pgfsetdash{}{0pt}%
\pgfpathmoveto{\pgfqpoint{3.647298in}{2.569739in}}%
\pgfpathcurveto{\pgfqpoint{3.658348in}{2.569739in}}{\pgfqpoint{3.668948in}{2.574129in}}{\pgfqpoint{3.676761in}{2.581943in}}%
\pgfpathcurveto{\pgfqpoint{3.684575in}{2.589756in}}{\pgfqpoint{3.688965in}{2.600355in}}{\pgfqpoint{3.688965in}{2.611406in}}%
\pgfpathcurveto{\pgfqpoint{3.688965in}{2.622456in}}{\pgfqpoint{3.684575in}{2.633055in}}{\pgfqpoint{3.676761in}{2.640868in}}%
\pgfpathcurveto{\pgfqpoint{3.668948in}{2.648682in}}{\pgfqpoint{3.658348in}{2.653072in}}{\pgfqpoint{3.647298in}{2.653072in}}%
\pgfpathcurveto{\pgfqpoint{3.636248in}{2.653072in}}{\pgfqpoint{3.625649in}{2.648682in}}{\pgfqpoint{3.617836in}{2.640868in}}%
\pgfpathcurveto{\pgfqpoint{3.610022in}{2.633055in}}{\pgfqpoint{3.605632in}{2.622456in}}{\pgfqpoint{3.605632in}{2.611406in}}%
\pgfpathcurveto{\pgfqpoint{3.605632in}{2.600355in}}{\pgfqpoint{3.610022in}{2.589756in}}{\pgfqpoint{3.617836in}{2.581943in}}%
\pgfpathcurveto{\pgfqpoint{3.625649in}{2.574129in}}{\pgfqpoint{3.636248in}{2.569739in}}{\pgfqpoint{3.647298in}{2.569739in}}%
\pgfpathclose%
\pgfusepath{stroke,fill}%
\end{pgfscope}%
\begin{pgfscope}%
\pgfpathrectangle{\pgfqpoint{0.787074in}{0.548769in}}{\pgfqpoint{5.062926in}{3.102590in}}%
\pgfusepath{clip}%
\pgfsetbuttcap%
\pgfsetroundjoin%
\definecolor{currentfill}{rgb}{1.000000,0.498039,0.054902}%
\pgfsetfillcolor{currentfill}%
\pgfsetlinewidth{1.003750pt}%
\definecolor{currentstroke}{rgb}{1.000000,0.498039,0.054902}%
\pgfsetstrokecolor{currentstroke}%
\pgfsetdash{}{0pt}%
\pgfpathmoveto{\pgfqpoint{3.252785in}{2.197035in}}%
\pgfpathcurveto{\pgfqpoint{3.263835in}{2.197035in}}{\pgfqpoint{3.274434in}{2.201425in}}{\pgfqpoint{3.282247in}{2.209239in}}%
\pgfpathcurveto{\pgfqpoint{3.290061in}{2.217052in}}{\pgfqpoint{3.294451in}{2.227651in}}{\pgfqpoint{3.294451in}{2.238701in}}%
\pgfpathcurveto{\pgfqpoint{3.294451in}{2.249751in}}{\pgfqpoint{3.290061in}{2.260350in}}{\pgfqpoint{3.282247in}{2.268164in}}%
\pgfpathcurveto{\pgfqpoint{3.274434in}{2.275978in}}{\pgfqpoint{3.263835in}{2.280368in}}{\pgfqpoint{3.252785in}{2.280368in}}%
\pgfpathcurveto{\pgfqpoint{3.241735in}{2.280368in}}{\pgfqpoint{3.231135in}{2.275978in}}{\pgfqpoint{3.223322in}{2.268164in}}%
\pgfpathcurveto{\pgfqpoint{3.215508in}{2.260350in}}{\pgfqpoint{3.211118in}{2.249751in}}{\pgfqpoint{3.211118in}{2.238701in}}%
\pgfpathcurveto{\pgfqpoint{3.211118in}{2.227651in}}{\pgfqpoint{3.215508in}{2.217052in}}{\pgfqpoint{3.223322in}{2.209239in}}%
\pgfpathcurveto{\pgfqpoint{3.231135in}{2.201425in}}{\pgfqpoint{3.241735in}{2.197035in}}{\pgfqpoint{3.252785in}{2.197035in}}%
\pgfpathclose%
\pgfusepath{stroke,fill}%
\end{pgfscope}%
\begin{pgfscope}%
\pgfpathrectangle{\pgfqpoint{0.787074in}{0.548769in}}{\pgfqpoint{5.062926in}{3.102590in}}%
\pgfusepath{clip}%
\pgfsetbuttcap%
\pgfsetroundjoin%
\definecolor{currentfill}{rgb}{0.121569,0.466667,0.705882}%
\pgfsetfillcolor{currentfill}%
\pgfsetlinewidth{1.003750pt}%
\definecolor{currentstroke}{rgb}{0.121569,0.466667,0.705882}%
\pgfsetstrokecolor{currentstroke}%
\pgfsetdash{}{0pt}%
\pgfpathmoveto{\pgfqpoint{1.082959in}{0.648134in}}%
\pgfpathcurveto{\pgfqpoint{1.094009in}{0.648134in}}{\pgfqpoint{1.104608in}{0.652524in}}{\pgfqpoint{1.112422in}{0.660338in}}%
\pgfpathcurveto{\pgfqpoint{1.120236in}{0.668151in}}{\pgfqpoint{1.124626in}{0.678750in}}{\pgfqpoint{1.124626in}{0.689801in}}%
\pgfpathcurveto{\pgfqpoint{1.124626in}{0.700851in}}{\pgfqpoint{1.120236in}{0.711450in}}{\pgfqpoint{1.112422in}{0.719263in}}%
\pgfpathcurveto{\pgfqpoint{1.104608in}{0.727077in}}{\pgfqpoint{1.094009in}{0.731467in}}{\pgfqpoint{1.082959in}{0.731467in}}%
\pgfpathcurveto{\pgfqpoint{1.071909in}{0.731467in}}{\pgfqpoint{1.061310in}{0.727077in}}{\pgfqpoint{1.053496in}{0.719263in}}%
\pgfpathcurveto{\pgfqpoint{1.045683in}{0.711450in}}{\pgfqpoint{1.041292in}{0.700851in}}{\pgfqpoint{1.041292in}{0.689801in}}%
\pgfpathcurveto{\pgfqpoint{1.041292in}{0.678750in}}{\pgfqpoint{1.045683in}{0.668151in}}{\pgfqpoint{1.053496in}{0.660338in}}%
\pgfpathcurveto{\pgfqpoint{1.061310in}{0.652524in}}{\pgfqpoint{1.071909in}{0.648134in}}{\pgfqpoint{1.082959in}{0.648134in}}%
\pgfpathclose%
\pgfusepath{stroke,fill}%
\end{pgfscope}%
\begin{pgfscope}%
\pgfpathrectangle{\pgfqpoint{0.787074in}{0.548769in}}{\pgfqpoint{5.062926in}{3.102590in}}%
\pgfusepath{clip}%
\pgfsetbuttcap%
\pgfsetroundjoin%
\definecolor{currentfill}{rgb}{1.000000,0.498039,0.054902}%
\pgfsetfillcolor{currentfill}%
\pgfsetlinewidth{1.003750pt}%
\definecolor{currentstroke}{rgb}{1.000000,0.498039,0.054902}%
\pgfsetstrokecolor{currentstroke}%
\pgfsetdash{}{0pt}%
\pgfpathmoveto{\pgfqpoint{2.661014in}{1.410798in}}%
\pgfpathcurveto{\pgfqpoint{2.672064in}{1.410798in}}{\pgfqpoint{2.682663in}{1.415189in}}{\pgfqpoint{2.690477in}{1.423002in}}%
\pgfpathcurveto{\pgfqpoint{2.698290in}{1.430816in}}{\pgfqpoint{2.702681in}{1.441415in}}{\pgfqpoint{2.702681in}{1.452465in}}%
\pgfpathcurveto{\pgfqpoint{2.702681in}{1.463515in}}{\pgfqpoint{2.698290in}{1.474114in}}{\pgfqpoint{2.690477in}{1.481928in}}%
\pgfpathcurveto{\pgfqpoint{2.682663in}{1.489741in}}{\pgfqpoint{2.672064in}{1.494132in}}{\pgfqpoint{2.661014in}{1.494132in}}%
\pgfpathcurveto{\pgfqpoint{2.649964in}{1.494132in}}{\pgfqpoint{2.639365in}{1.489741in}}{\pgfqpoint{2.631551in}{1.481928in}}%
\pgfpathcurveto{\pgfqpoint{2.623738in}{1.474114in}}{\pgfqpoint{2.619347in}{1.463515in}}{\pgfqpoint{2.619347in}{1.452465in}}%
\pgfpathcurveto{\pgfqpoint{2.619347in}{1.441415in}}{\pgfqpoint{2.623738in}{1.430816in}}{\pgfqpoint{2.631551in}{1.423002in}}%
\pgfpathcurveto{\pgfqpoint{2.639365in}{1.415189in}}{\pgfqpoint{2.649964in}{1.410798in}}{\pgfqpoint{2.661014in}{1.410798in}}%
\pgfpathclose%
\pgfusepath{stroke,fill}%
\end{pgfscope}%
\begin{pgfscope}%
\pgfpathrectangle{\pgfqpoint{0.787074in}{0.548769in}}{\pgfqpoint{5.062926in}{3.102590in}}%
\pgfusepath{clip}%
\pgfsetbuttcap%
\pgfsetroundjoin%
\definecolor{currentfill}{rgb}{1.000000,0.498039,0.054902}%
\pgfsetfillcolor{currentfill}%
\pgfsetlinewidth{1.003750pt}%
\definecolor{currentstroke}{rgb}{1.000000,0.498039,0.054902}%
\pgfsetstrokecolor{currentstroke}%
\pgfsetdash{}{0pt}%
\pgfpathmoveto{\pgfqpoint{1.674730in}{2.681418in}}%
\pgfpathcurveto{\pgfqpoint{1.685780in}{2.681418in}}{\pgfqpoint{1.696379in}{2.685808in}}{\pgfqpoint{1.704193in}{2.693622in}}%
\pgfpathcurveto{\pgfqpoint{1.712006in}{2.701435in}}{\pgfqpoint{1.716396in}{2.712034in}}{\pgfqpoint{1.716396in}{2.723084in}}%
\pgfpathcurveto{\pgfqpoint{1.716396in}{2.734135in}}{\pgfqpoint{1.712006in}{2.744734in}}{\pgfqpoint{1.704193in}{2.752547in}}%
\pgfpathcurveto{\pgfqpoint{1.696379in}{2.760361in}}{\pgfqpoint{1.685780in}{2.764751in}}{\pgfqpoint{1.674730in}{2.764751in}}%
\pgfpathcurveto{\pgfqpoint{1.663680in}{2.764751in}}{\pgfqpoint{1.653081in}{2.760361in}}{\pgfqpoint{1.645267in}{2.752547in}}%
\pgfpathcurveto{\pgfqpoint{1.637453in}{2.744734in}}{\pgfqpoint{1.633063in}{2.734135in}}{\pgfqpoint{1.633063in}{2.723084in}}%
\pgfpathcurveto{\pgfqpoint{1.633063in}{2.712034in}}{\pgfqpoint{1.637453in}{2.701435in}}{\pgfqpoint{1.645267in}{2.693622in}}%
\pgfpathcurveto{\pgfqpoint{1.653081in}{2.685808in}}{\pgfqpoint{1.663680in}{2.681418in}}{\pgfqpoint{1.674730in}{2.681418in}}%
\pgfpathclose%
\pgfusepath{stroke,fill}%
\end{pgfscope}%
\begin{pgfscope}%
\pgfpathrectangle{\pgfqpoint{0.787074in}{0.548769in}}{\pgfqpoint{5.062926in}{3.102590in}}%
\pgfusepath{clip}%
\pgfsetbuttcap%
\pgfsetroundjoin%
\definecolor{currentfill}{rgb}{1.000000,0.498039,0.054902}%
\pgfsetfillcolor{currentfill}%
\pgfsetlinewidth{1.003750pt}%
\definecolor{currentstroke}{rgb}{1.000000,0.498039,0.054902}%
\pgfsetstrokecolor{currentstroke}%
\pgfsetdash{}{0pt}%
\pgfpathmoveto{\pgfqpoint{1.740482in}{2.256332in}}%
\pgfpathcurveto{\pgfqpoint{1.751532in}{2.256332in}}{\pgfqpoint{1.762131in}{2.260722in}}{\pgfqpoint{1.769945in}{2.268536in}}%
\pgfpathcurveto{\pgfqpoint{1.777758in}{2.276349in}}{\pgfqpoint{1.782149in}{2.286948in}}{\pgfqpoint{1.782149in}{2.297998in}}%
\pgfpathcurveto{\pgfqpoint{1.782149in}{2.309048in}}{\pgfqpoint{1.777758in}{2.319648in}}{\pgfqpoint{1.769945in}{2.327461in}}%
\pgfpathcurveto{\pgfqpoint{1.762131in}{2.335275in}}{\pgfqpoint{1.751532in}{2.339665in}}{\pgfqpoint{1.740482in}{2.339665in}}%
\pgfpathcurveto{\pgfqpoint{1.729432in}{2.339665in}}{\pgfqpoint{1.718833in}{2.335275in}}{\pgfqpoint{1.711019in}{2.327461in}}%
\pgfpathcurveto{\pgfqpoint{1.703206in}{2.319648in}}{\pgfqpoint{1.698815in}{2.309048in}}{\pgfqpoint{1.698815in}{2.297998in}}%
\pgfpathcurveto{\pgfqpoint{1.698815in}{2.286948in}}{\pgfqpoint{1.703206in}{2.276349in}}{\pgfqpoint{1.711019in}{2.268536in}}%
\pgfpathcurveto{\pgfqpoint{1.718833in}{2.260722in}}{\pgfqpoint{1.729432in}{2.256332in}}{\pgfqpoint{1.740482in}{2.256332in}}%
\pgfpathclose%
\pgfusepath{stroke,fill}%
\end{pgfscope}%
\begin{pgfscope}%
\pgfpathrectangle{\pgfqpoint{0.787074in}{0.548769in}}{\pgfqpoint{5.062926in}{3.102590in}}%
\pgfusepath{clip}%
\pgfsetbuttcap%
\pgfsetroundjoin%
\definecolor{currentfill}{rgb}{0.121569,0.466667,0.705882}%
\pgfsetfillcolor{currentfill}%
\pgfsetlinewidth{1.003750pt}%
\definecolor{currentstroke}{rgb}{0.121569,0.466667,0.705882}%
\pgfsetstrokecolor{currentstroke}%
\pgfsetdash{}{0pt}%
\pgfpathmoveto{\pgfqpoint{2.003491in}{0.775326in}}%
\pgfpathcurveto{\pgfqpoint{2.014541in}{0.775326in}}{\pgfqpoint{2.025140in}{0.779716in}}{\pgfqpoint{2.032954in}{0.787530in}}%
\pgfpathcurveto{\pgfqpoint{2.040768in}{0.795343in}}{\pgfqpoint{2.045158in}{0.805942in}}{\pgfqpoint{2.045158in}{0.816993in}}%
\pgfpathcurveto{\pgfqpoint{2.045158in}{0.828043in}}{\pgfqpoint{2.040768in}{0.838642in}}{\pgfqpoint{2.032954in}{0.846455in}}%
\pgfpathcurveto{\pgfqpoint{2.025140in}{0.854269in}}{\pgfqpoint{2.014541in}{0.858659in}}{\pgfqpoint{2.003491in}{0.858659in}}%
\pgfpathcurveto{\pgfqpoint{1.992441in}{0.858659in}}{\pgfqpoint{1.981842in}{0.854269in}}{\pgfqpoint{1.974028in}{0.846455in}}%
\pgfpathcurveto{\pgfqpoint{1.966215in}{0.838642in}}{\pgfqpoint{1.961824in}{0.828043in}}{\pgfqpoint{1.961824in}{0.816993in}}%
\pgfpathcurveto{\pgfqpoint{1.961824in}{0.805942in}}{\pgfqpoint{1.966215in}{0.795343in}}{\pgfqpoint{1.974028in}{0.787530in}}%
\pgfpathcurveto{\pgfqpoint{1.981842in}{0.779716in}}{\pgfqpoint{1.992441in}{0.775326in}}{\pgfqpoint{2.003491in}{0.775326in}}%
\pgfpathclose%
\pgfusepath{stroke,fill}%
\end{pgfscope}%
\begin{pgfscope}%
\pgfpathrectangle{\pgfqpoint{0.787074in}{0.548769in}}{\pgfqpoint{5.062926in}{3.102590in}}%
\pgfusepath{clip}%
\pgfsetbuttcap%
\pgfsetroundjoin%
\definecolor{currentfill}{rgb}{1.000000,0.498039,0.054902}%
\pgfsetfillcolor{currentfill}%
\pgfsetlinewidth{1.003750pt}%
\definecolor{currentstroke}{rgb}{1.000000,0.498039,0.054902}%
\pgfsetstrokecolor{currentstroke}%
\pgfsetdash{}{0pt}%
\pgfpathmoveto{\pgfqpoint{1.543225in}{1.583415in}}%
\pgfpathcurveto{\pgfqpoint{1.554275in}{1.583415in}}{\pgfqpoint{1.564874in}{1.587805in}}{\pgfqpoint{1.572688in}{1.595619in}}%
\pgfpathcurveto{\pgfqpoint{1.580502in}{1.603432in}}{\pgfqpoint{1.584892in}{1.614031in}}{\pgfqpoint{1.584892in}{1.625081in}}%
\pgfpathcurveto{\pgfqpoint{1.584892in}{1.636131in}}{\pgfqpoint{1.580502in}{1.646730in}}{\pgfqpoint{1.572688in}{1.654544in}}%
\pgfpathcurveto{\pgfqpoint{1.564874in}{1.662358in}}{\pgfqpoint{1.554275in}{1.666748in}}{\pgfqpoint{1.543225in}{1.666748in}}%
\pgfpathcurveto{\pgfqpoint{1.532175in}{1.666748in}}{\pgfqpoint{1.521576in}{1.662358in}}{\pgfqpoint{1.513762in}{1.654544in}}%
\pgfpathcurveto{\pgfqpoint{1.505949in}{1.646730in}}{\pgfqpoint{1.501558in}{1.636131in}}{\pgfqpoint{1.501558in}{1.625081in}}%
\pgfpathcurveto{\pgfqpoint{1.501558in}{1.614031in}}{\pgfqpoint{1.505949in}{1.603432in}}{\pgfqpoint{1.513762in}{1.595619in}}%
\pgfpathcurveto{\pgfqpoint{1.521576in}{1.587805in}}{\pgfqpoint{1.532175in}{1.583415in}}{\pgfqpoint{1.543225in}{1.583415in}}%
\pgfpathclose%
\pgfusepath{stroke,fill}%
\end{pgfscope}%
\begin{pgfscope}%
\pgfpathrectangle{\pgfqpoint{0.787074in}{0.548769in}}{\pgfqpoint{5.062926in}{3.102590in}}%
\pgfusepath{clip}%
\pgfsetbuttcap%
\pgfsetroundjoin%
\definecolor{currentfill}{rgb}{1.000000,0.498039,0.054902}%
\pgfsetfillcolor{currentfill}%
\pgfsetlinewidth{1.003750pt}%
\definecolor{currentstroke}{rgb}{1.000000,0.498039,0.054902}%
\pgfsetstrokecolor{currentstroke}%
\pgfsetdash{}{0pt}%
\pgfpathmoveto{\pgfqpoint{2.200748in}{2.261976in}}%
\pgfpathcurveto{\pgfqpoint{2.211798in}{2.261976in}}{\pgfqpoint{2.222397in}{2.266366in}}{\pgfqpoint{2.230211in}{2.274180in}}%
\pgfpathcurveto{\pgfqpoint{2.238024in}{2.281994in}}{\pgfqpoint{2.242415in}{2.292593in}}{\pgfqpoint{2.242415in}{2.303643in}}%
\pgfpathcurveto{\pgfqpoint{2.242415in}{2.314693in}}{\pgfqpoint{2.238024in}{2.325292in}}{\pgfqpoint{2.230211in}{2.333105in}}%
\pgfpathcurveto{\pgfqpoint{2.222397in}{2.340919in}}{\pgfqpoint{2.211798in}{2.345309in}}{\pgfqpoint{2.200748in}{2.345309in}}%
\pgfpathcurveto{\pgfqpoint{2.189698in}{2.345309in}}{\pgfqpoint{2.179099in}{2.340919in}}{\pgfqpoint{2.171285in}{2.333105in}}%
\pgfpathcurveto{\pgfqpoint{2.163472in}{2.325292in}}{\pgfqpoint{2.159081in}{2.314693in}}{\pgfqpoint{2.159081in}{2.303643in}}%
\pgfpathcurveto{\pgfqpoint{2.159081in}{2.292593in}}{\pgfqpoint{2.163472in}{2.281994in}}{\pgfqpoint{2.171285in}{2.274180in}}%
\pgfpathcurveto{\pgfqpoint{2.179099in}{2.266366in}}{\pgfqpoint{2.189698in}{2.261976in}}{\pgfqpoint{2.200748in}{2.261976in}}%
\pgfpathclose%
\pgfusepath{stroke,fill}%
\end{pgfscope}%
\begin{pgfscope}%
\pgfpathrectangle{\pgfqpoint{0.787074in}{0.548769in}}{\pgfqpoint{5.062926in}{3.102590in}}%
\pgfusepath{clip}%
\pgfsetbuttcap%
\pgfsetroundjoin%
\definecolor{currentfill}{rgb}{1.000000,0.498039,0.054902}%
\pgfsetfillcolor{currentfill}%
\pgfsetlinewidth{1.003750pt}%
\definecolor{currentstroke}{rgb}{1.000000,0.498039,0.054902}%
\pgfsetstrokecolor{currentstroke}%
\pgfsetdash{}{0pt}%
\pgfpathmoveto{\pgfqpoint{2.332253in}{3.078940in}}%
\pgfpathcurveto{\pgfqpoint{2.343303in}{3.078940in}}{\pgfqpoint{2.353902in}{3.083331in}}{\pgfqpoint{2.361715in}{3.091144in}}%
\pgfpathcurveto{\pgfqpoint{2.369529in}{3.098958in}}{\pgfqpoint{2.373919in}{3.109557in}}{\pgfqpoint{2.373919in}{3.120607in}}%
\pgfpathcurveto{\pgfqpoint{2.373919in}{3.131657in}}{\pgfqpoint{2.369529in}{3.142256in}}{\pgfqpoint{2.361715in}{3.150070in}}%
\pgfpathcurveto{\pgfqpoint{2.353902in}{3.157883in}}{\pgfqpoint{2.343303in}{3.162274in}}{\pgfqpoint{2.332253in}{3.162274in}}%
\pgfpathcurveto{\pgfqpoint{2.321202in}{3.162274in}}{\pgfqpoint{2.310603in}{3.157883in}}{\pgfqpoint{2.302790in}{3.150070in}}%
\pgfpathcurveto{\pgfqpoint{2.294976in}{3.142256in}}{\pgfqpoint{2.290586in}{3.131657in}}{\pgfqpoint{2.290586in}{3.120607in}}%
\pgfpathcurveto{\pgfqpoint{2.290586in}{3.109557in}}{\pgfqpoint{2.294976in}{3.098958in}}{\pgfqpoint{2.302790in}{3.091144in}}%
\pgfpathcurveto{\pgfqpoint{2.310603in}{3.083331in}}{\pgfqpoint{2.321202in}{3.078940in}}{\pgfqpoint{2.332253in}{3.078940in}}%
\pgfpathclose%
\pgfusepath{stroke,fill}%
\end{pgfscope}%
\begin{pgfscope}%
\pgfpathrectangle{\pgfqpoint{0.787074in}{0.548769in}}{\pgfqpoint{5.062926in}{3.102590in}}%
\pgfusepath{clip}%
\pgfsetbuttcap%
\pgfsetroundjoin%
\definecolor{currentfill}{rgb}{1.000000,0.498039,0.054902}%
\pgfsetfillcolor{currentfill}%
\pgfsetlinewidth{1.003750pt}%
\definecolor{currentstroke}{rgb}{1.000000,0.498039,0.054902}%
\pgfsetstrokecolor{currentstroke}%
\pgfsetdash{}{0pt}%
\pgfpathmoveto{\pgfqpoint{2.200748in}{1.210818in}}%
\pgfpathcurveto{\pgfqpoint{2.211798in}{1.210818in}}{\pgfqpoint{2.222397in}{1.215208in}}{\pgfqpoint{2.230211in}{1.223021in}}%
\pgfpathcurveto{\pgfqpoint{2.238024in}{1.230835in}}{\pgfqpoint{2.242415in}{1.241434in}}{\pgfqpoint{2.242415in}{1.252484in}}%
\pgfpathcurveto{\pgfqpoint{2.242415in}{1.263534in}}{\pgfqpoint{2.238024in}{1.274133in}}{\pgfqpoint{2.230211in}{1.281947in}}%
\pgfpathcurveto{\pgfqpoint{2.222397in}{1.289761in}}{\pgfqpoint{2.211798in}{1.294151in}}{\pgfqpoint{2.200748in}{1.294151in}}%
\pgfpathcurveto{\pgfqpoint{2.189698in}{1.294151in}}{\pgfqpoint{2.179099in}{1.289761in}}{\pgfqpoint{2.171285in}{1.281947in}}%
\pgfpathcurveto{\pgfqpoint{2.163472in}{1.274133in}}{\pgfqpoint{2.159081in}{1.263534in}}{\pgfqpoint{2.159081in}{1.252484in}}%
\pgfpathcurveto{\pgfqpoint{2.159081in}{1.241434in}}{\pgfqpoint{2.163472in}{1.230835in}}{\pgfqpoint{2.171285in}{1.223021in}}%
\pgfpathcurveto{\pgfqpoint{2.179099in}{1.215208in}}{\pgfqpoint{2.189698in}{1.210818in}}{\pgfqpoint{2.200748in}{1.210818in}}%
\pgfpathclose%
\pgfusepath{stroke,fill}%
\end{pgfscope}%
\begin{pgfscope}%
\pgfpathrectangle{\pgfqpoint{0.787074in}{0.548769in}}{\pgfqpoint{5.062926in}{3.102590in}}%
\pgfusepath{clip}%
\pgfsetbuttcap%
\pgfsetroundjoin%
\definecolor{currentfill}{rgb}{1.000000,0.498039,0.054902}%
\pgfsetfillcolor{currentfill}%
\pgfsetlinewidth{1.003750pt}%
\definecolor{currentstroke}{rgb}{1.000000,0.498039,0.054902}%
\pgfsetstrokecolor{currentstroke}%
\pgfsetdash{}{0pt}%
\pgfpathmoveto{\pgfqpoint{2.134996in}{1.923637in}}%
\pgfpathcurveto{\pgfqpoint{2.146046in}{1.923637in}}{\pgfqpoint{2.156645in}{1.928027in}}{\pgfqpoint{2.164459in}{1.935841in}}%
\pgfpathcurveto{\pgfqpoint{2.172272in}{1.943654in}}{\pgfqpoint{2.176662in}{1.954253in}}{\pgfqpoint{2.176662in}{1.965303in}}%
\pgfpathcurveto{\pgfqpoint{2.176662in}{1.976354in}}{\pgfqpoint{2.172272in}{1.986953in}}{\pgfqpoint{2.164459in}{1.994766in}}%
\pgfpathcurveto{\pgfqpoint{2.156645in}{2.002580in}}{\pgfqpoint{2.146046in}{2.006970in}}{\pgfqpoint{2.134996in}{2.006970in}}%
\pgfpathcurveto{\pgfqpoint{2.123946in}{2.006970in}}{\pgfqpoint{2.113347in}{2.002580in}}{\pgfqpoint{2.105533in}{1.994766in}}%
\pgfpathcurveto{\pgfqpoint{2.097719in}{1.986953in}}{\pgfqpoint{2.093329in}{1.976354in}}{\pgfqpoint{2.093329in}{1.965303in}}%
\pgfpathcurveto{\pgfqpoint{2.093329in}{1.954253in}}{\pgfqpoint{2.097719in}{1.943654in}}{\pgfqpoint{2.105533in}{1.935841in}}%
\pgfpathcurveto{\pgfqpoint{2.113347in}{1.928027in}}{\pgfqpoint{2.123946in}{1.923637in}}{\pgfqpoint{2.134996in}{1.923637in}}%
\pgfpathclose%
\pgfusepath{stroke,fill}%
\end{pgfscope}%
\begin{pgfscope}%
\pgfpathrectangle{\pgfqpoint{0.787074in}{0.548769in}}{\pgfqpoint{5.062926in}{3.102590in}}%
\pgfusepath{clip}%
\pgfsetbuttcap%
\pgfsetroundjoin%
\definecolor{currentfill}{rgb}{1.000000,0.498039,0.054902}%
\pgfsetfillcolor{currentfill}%
\pgfsetlinewidth{1.003750pt}%
\definecolor{currentstroke}{rgb}{1.000000,0.498039,0.054902}%
\pgfsetstrokecolor{currentstroke}%
\pgfsetdash{}{0pt}%
\pgfpathmoveto{\pgfqpoint{2.003491in}{2.880292in}}%
\pgfpathcurveto{\pgfqpoint{2.014541in}{2.880292in}}{\pgfqpoint{2.025140in}{2.884682in}}{\pgfqpoint{2.032954in}{2.892495in}}%
\pgfpathcurveto{\pgfqpoint{2.040768in}{2.900309in}}{\pgfqpoint{2.045158in}{2.910908in}}{\pgfqpoint{2.045158in}{2.921958in}}%
\pgfpathcurveto{\pgfqpoint{2.045158in}{2.933008in}}{\pgfqpoint{2.040768in}{2.943607in}}{\pgfqpoint{2.032954in}{2.951421in}}%
\pgfpathcurveto{\pgfqpoint{2.025140in}{2.959235in}}{\pgfqpoint{2.014541in}{2.963625in}}{\pgfqpoint{2.003491in}{2.963625in}}%
\pgfpathcurveto{\pgfqpoint{1.992441in}{2.963625in}}{\pgfqpoint{1.981842in}{2.959235in}}{\pgfqpoint{1.974028in}{2.951421in}}%
\pgfpathcurveto{\pgfqpoint{1.966215in}{2.943607in}}{\pgfqpoint{1.961824in}{2.933008in}}{\pgfqpoint{1.961824in}{2.921958in}}%
\pgfpathcurveto{\pgfqpoint{1.961824in}{2.910908in}}{\pgfqpoint{1.966215in}{2.900309in}}{\pgfqpoint{1.974028in}{2.892495in}}%
\pgfpathcurveto{\pgfqpoint{1.981842in}{2.884682in}}{\pgfqpoint{1.992441in}{2.880292in}}{\pgfqpoint{2.003491in}{2.880292in}}%
\pgfpathclose%
\pgfusepath{stroke,fill}%
\end{pgfscope}%
\begin{pgfscope}%
\pgfpathrectangle{\pgfqpoint{0.787074in}{0.548769in}}{\pgfqpoint{5.062926in}{3.102590in}}%
\pgfusepath{clip}%
\pgfsetbuttcap%
\pgfsetroundjoin%
\definecolor{currentfill}{rgb}{1.000000,0.498039,0.054902}%
\pgfsetfillcolor{currentfill}%
\pgfsetlinewidth{1.003750pt}%
\definecolor{currentstroke}{rgb}{1.000000,0.498039,0.054902}%
\pgfsetstrokecolor{currentstroke}%
\pgfsetdash{}{0pt}%
\pgfpathmoveto{\pgfqpoint{2.989775in}{2.018767in}}%
\pgfpathcurveto{\pgfqpoint{3.000826in}{2.018767in}}{\pgfqpoint{3.011425in}{2.023158in}}{\pgfqpoint{3.019238in}{2.030971in}}%
\pgfpathcurveto{\pgfqpoint{3.027052in}{2.038785in}}{\pgfqpoint{3.031442in}{2.049384in}}{\pgfqpoint{3.031442in}{2.060434in}}%
\pgfpathcurveto{\pgfqpoint{3.031442in}{2.071484in}}{\pgfqpoint{3.027052in}{2.082083in}}{\pgfqpoint{3.019238in}{2.089897in}}%
\pgfpathcurveto{\pgfqpoint{3.011425in}{2.097710in}}{\pgfqpoint{3.000826in}{2.102101in}}{\pgfqpoint{2.989775in}{2.102101in}}%
\pgfpathcurveto{\pgfqpoint{2.978725in}{2.102101in}}{\pgfqpoint{2.968126in}{2.097710in}}{\pgfqpoint{2.960313in}{2.089897in}}%
\pgfpathcurveto{\pgfqpoint{2.952499in}{2.082083in}}{\pgfqpoint{2.948109in}{2.071484in}}{\pgfqpoint{2.948109in}{2.060434in}}%
\pgfpathcurveto{\pgfqpoint{2.948109in}{2.049384in}}{\pgfqpoint{2.952499in}{2.038785in}}{\pgfqpoint{2.960313in}{2.030971in}}%
\pgfpathcurveto{\pgfqpoint{2.968126in}{2.023158in}}{\pgfqpoint{2.978725in}{2.018767in}}{\pgfqpoint{2.989775in}{2.018767in}}%
\pgfpathclose%
\pgfusepath{stroke,fill}%
\end{pgfscope}%
\begin{pgfscope}%
\pgfpathrectangle{\pgfqpoint{0.787074in}{0.548769in}}{\pgfqpoint{5.062926in}{3.102590in}}%
\pgfusepath{clip}%
\pgfsetbuttcap%
\pgfsetroundjoin%
\definecolor{currentfill}{rgb}{1.000000,0.498039,0.054902}%
\pgfsetfillcolor{currentfill}%
\pgfsetlinewidth{1.003750pt}%
\definecolor{currentstroke}{rgb}{1.000000,0.498039,0.054902}%
\pgfsetstrokecolor{currentstroke}%
\pgfsetdash{}{0pt}%
\pgfpathmoveto{\pgfqpoint{4.304821in}{0.875725in}}%
\pgfpathcurveto{\pgfqpoint{4.315871in}{0.875725in}}{\pgfqpoint{4.326470in}{0.880115in}}{\pgfqpoint{4.334284in}{0.887929in}}%
\pgfpathcurveto{\pgfqpoint{4.342098in}{0.895743in}}{\pgfqpoint{4.346488in}{0.906342in}}{\pgfqpoint{4.346488in}{0.917392in}}%
\pgfpathcurveto{\pgfqpoint{4.346488in}{0.928442in}}{\pgfqpoint{4.342098in}{0.939041in}}{\pgfqpoint{4.334284in}{0.946855in}}%
\pgfpathcurveto{\pgfqpoint{4.326470in}{0.954668in}}{\pgfqpoint{4.315871in}{0.959058in}}{\pgfqpoint{4.304821in}{0.959058in}}%
\pgfpathcurveto{\pgfqpoint{4.293771in}{0.959058in}}{\pgfqpoint{4.283172in}{0.954668in}}{\pgfqpoint{4.275358in}{0.946855in}}%
\pgfpathcurveto{\pgfqpoint{4.267545in}{0.939041in}}{\pgfqpoint{4.263155in}{0.928442in}}{\pgfqpoint{4.263155in}{0.917392in}}%
\pgfpathcurveto{\pgfqpoint{4.263155in}{0.906342in}}{\pgfqpoint{4.267545in}{0.895743in}}{\pgfqpoint{4.275358in}{0.887929in}}%
\pgfpathcurveto{\pgfqpoint{4.283172in}{0.880115in}}{\pgfqpoint{4.293771in}{0.875725in}}{\pgfqpoint{4.304821in}{0.875725in}}%
\pgfpathclose%
\pgfusepath{stroke,fill}%
\end{pgfscope}%
\begin{pgfscope}%
\pgfpathrectangle{\pgfqpoint{0.787074in}{0.548769in}}{\pgfqpoint{5.062926in}{3.102590in}}%
\pgfusepath{clip}%
\pgfsetbuttcap%
\pgfsetroundjoin%
\definecolor{currentfill}{rgb}{1.000000,0.498039,0.054902}%
\pgfsetfillcolor{currentfill}%
\pgfsetlinewidth{1.003750pt}%
\definecolor{currentstroke}{rgb}{1.000000,0.498039,0.054902}%
\pgfsetstrokecolor{currentstroke}%
\pgfsetdash{}{0pt}%
\pgfpathmoveto{\pgfqpoint{1.871987in}{1.792377in}}%
\pgfpathcurveto{\pgfqpoint{1.883037in}{1.792377in}}{\pgfqpoint{1.893636in}{1.796767in}}{\pgfqpoint{1.901449in}{1.804581in}}%
\pgfpathcurveto{\pgfqpoint{1.909263in}{1.812394in}}{\pgfqpoint{1.913653in}{1.822993in}}{\pgfqpoint{1.913653in}{1.834043in}}%
\pgfpathcurveto{\pgfqpoint{1.913653in}{1.845093in}}{\pgfqpoint{1.909263in}{1.855693in}}{\pgfqpoint{1.901449in}{1.863506in}}%
\pgfpathcurveto{\pgfqpoint{1.893636in}{1.871320in}}{\pgfqpoint{1.883037in}{1.875710in}}{\pgfqpoint{1.871987in}{1.875710in}}%
\pgfpathcurveto{\pgfqpoint{1.860936in}{1.875710in}}{\pgfqpoint{1.850337in}{1.871320in}}{\pgfqpoint{1.842524in}{1.863506in}}%
\pgfpathcurveto{\pgfqpoint{1.834710in}{1.855693in}}{\pgfqpoint{1.830320in}{1.845093in}}{\pgfqpoint{1.830320in}{1.834043in}}%
\pgfpathcurveto{\pgfqpoint{1.830320in}{1.822993in}}{\pgfqpoint{1.834710in}{1.812394in}}{\pgfqpoint{1.842524in}{1.804581in}}%
\pgfpathcurveto{\pgfqpoint{1.850337in}{1.796767in}}{\pgfqpoint{1.860936in}{1.792377in}}{\pgfqpoint{1.871987in}{1.792377in}}%
\pgfpathclose%
\pgfusepath{stroke,fill}%
\end{pgfscope}%
\begin{pgfscope}%
\pgfpathrectangle{\pgfqpoint{0.787074in}{0.548769in}}{\pgfqpoint{5.062926in}{3.102590in}}%
\pgfusepath{clip}%
\pgfsetbuttcap%
\pgfsetroundjoin%
\definecolor{currentfill}{rgb}{1.000000,0.498039,0.054902}%
\pgfsetfillcolor{currentfill}%
\pgfsetlinewidth{1.003750pt}%
\definecolor{currentstroke}{rgb}{1.000000,0.498039,0.054902}%
\pgfsetstrokecolor{currentstroke}%
\pgfsetdash{}{0pt}%
\pgfpathmoveto{\pgfqpoint{1.937739in}{1.860728in}}%
\pgfpathcurveto{\pgfqpoint{1.948789in}{1.860728in}}{\pgfqpoint{1.959388in}{1.865118in}}{\pgfqpoint{1.967202in}{1.872931in}}%
\pgfpathcurveto{\pgfqpoint{1.975015in}{1.880745in}}{\pgfqpoint{1.979406in}{1.891344in}}{\pgfqpoint{1.979406in}{1.902394in}}%
\pgfpathcurveto{\pgfqpoint{1.979406in}{1.913444in}}{\pgfqpoint{1.975015in}{1.924043in}}{\pgfqpoint{1.967202in}{1.931857in}}%
\pgfpathcurveto{\pgfqpoint{1.959388in}{1.939671in}}{\pgfqpoint{1.948789in}{1.944061in}}{\pgfqpoint{1.937739in}{1.944061in}}%
\pgfpathcurveto{\pgfqpoint{1.926689in}{1.944061in}}{\pgfqpoint{1.916090in}{1.939671in}}{\pgfqpoint{1.908276in}{1.931857in}}%
\pgfpathcurveto{\pgfqpoint{1.900462in}{1.924043in}}{\pgfqpoint{1.896072in}{1.913444in}}{\pgfqpoint{1.896072in}{1.902394in}}%
\pgfpathcurveto{\pgfqpoint{1.896072in}{1.891344in}}{\pgfqpoint{1.900462in}{1.880745in}}{\pgfqpoint{1.908276in}{1.872931in}}%
\pgfpathcurveto{\pgfqpoint{1.916090in}{1.865118in}}{\pgfqpoint{1.926689in}{1.860728in}}{\pgfqpoint{1.937739in}{1.860728in}}%
\pgfpathclose%
\pgfusepath{stroke,fill}%
\end{pgfscope}%
\begin{pgfscope}%
\pgfpathrectangle{\pgfqpoint{0.787074in}{0.548769in}}{\pgfqpoint{5.062926in}{3.102590in}}%
\pgfusepath{clip}%
\pgfsetbuttcap%
\pgfsetroundjoin%
\definecolor{currentfill}{rgb}{1.000000,0.498039,0.054902}%
\pgfsetfillcolor{currentfill}%
\pgfsetlinewidth{1.003750pt}%
\definecolor{currentstroke}{rgb}{1.000000,0.498039,0.054902}%
\pgfsetstrokecolor{currentstroke}%
\pgfsetdash{}{0pt}%
\pgfpathmoveto{\pgfqpoint{1.608977in}{1.531547in}}%
\pgfpathcurveto{\pgfqpoint{1.620028in}{1.531547in}}{\pgfqpoint{1.630627in}{1.535938in}}{\pgfqpoint{1.638440in}{1.543751in}}%
\pgfpathcurveto{\pgfqpoint{1.646254in}{1.551565in}}{\pgfqpoint{1.650644in}{1.562164in}}{\pgfqpoint{1.650644in}{1.573214in}}%
\pgfpathcurveto{\pgfqpoint{1.650644in}{1.584264in}}{\pgfqpoint{1.646254in}{1.594863in}}{\pgfqpoint{1.638440in}{1.602677in}}%
\pgfpathcurveto{\pgfqpoint{1.630627in}{1.610490in}}{\pgfqpoint{1.620028in}{1.614881in}}{\pgfqpoint{1.608977in}{1.614881in}}%
\pgfpathcurveto{\pgfqpoint{1.597927in}{1.614881in}}{\pgfqpoint{1.587328in}{1.610490in}}{\pgfqpoint{1.579515in}{1.602677in}}%
\pgfpathcurveto{\pgfqpoint{1.571701in}{1.594863in}}{\pgfqpoint{1.567311in}{1.584264in}}{\pgfqpoint{1.567311in}{1.573214in}}%
\pgfpathcurveto{\pgfqpoint{1.567311in}{1.562164in}}{\pgfqpoint{1.571701in}{1.551565in}}{\pgfqpoint{1.579515in}{1.543751in}}%
\pgfpathcurveto{\pgfqpoint{1.587328in}{1.535938in}}{\pgfqpoint{1.597927in}{1.531547in}}{\pgfqpoint{1.608977in}{1.531547in}}%
\pgfpathclose%
\pgfusepath{stroke,fill}%
\end{pgfscope}%
\begin{pgfscope}%
\pgfpathrectangle{\pgfqpoint{0.787074in}{0.548769in}}{\pgfqpoint{5.062926in}{3.102590in}}%
\pgfusepath{clip}%
\pgfsetbuttcap%
\pgfsetroundjoin%
\definecolor{currentfill}{rgb}{1.000000,0.498039,0.054902}%
\pgfsetfillcolor{currentfill}%
\pgfsetlinewidth{1.003750pt}%
\definecolor{currentstroke}{rgb}{1.000000,0.498039,0.054902}%
\pgfsetstrokecolor{currentstroke}%
\pgfsetdash{}{0pt}%
\pgfpathmoveto{\pgfqpoint{2.398005in}{2.472823in}}%
\pgfpathcurveto{\pgfqpoint{2.409055in}{2.472823in}}{\pgfqpoint{2.419654in}{2.477213in}}{\pgfqpoint{2.427468in}{2.485027in}}%
\pgfpathcurveto{\pgfqpoint{2.435281in}{2.492840in}}{\pgfqpoint{2.439672in}{2.503439in}}{\pgfqpoint{2.439672in}{2.514489in}}%
\pgfpathcurveto{\pgfqpoint{2.439672in}{2.525539in}}{\pgfqpoint{2.435281in}{2.536138in}}{\pgfqpoint{2.427468in}{2.543952in}}%
\pgfpathcurveto{\pgfqpoint{2.419654in}{2.551766in}}{\pgfqpoint{2.409055in}{2.556156in}}{\pgfqpoint{2.398005in}{2.556156in}}%
\pgfpathcurveto{\pgfqpoint{2.386955in}{2.556156in}}{\pgfqpoint{2.376356in}{2.551766in}}{\pgfqpoint{2.368542in}{2.543952in}}%
\pgfpathcurveto{\pgfqpoint{2.360728in}{2.536138in}}{\pgfqpoint{2.356338in}{2.525539in}}{\pgfqpoint{2.356338in}{2.514489in}}%
\pgfpathcurveto{\pgfqpoint{2.356338in}{2.503439in}}{\pgfqpoint{2.360728in}{2.492840in}}{\pgfqpoint{2.368542in}{2.485027in}}%
\pgfpathcurveto{\pgfqpoint{2.376356in}{2.477213in}}{\pgfqpoint{2.386955in}{2.472823in}}{\pgfqpoint{2.398005in}{2.472823in}}%
\pgfpathclose%
\pgfusepath{stroke,fill}%
\end{pgfscope}%
\begin{pgfscope}%
\pgfpathrectangle{\pgfqpoint{0.787074in}{0.548769in}}{\pgfqpoint{5.062926in}{3.102590in}}%
\pgfusepath{clip}%
\pgfsetbuttcap%
\pgfsetroundjoin%
\definecolor{currentfill}{rgb}{1.000000,0.498039,0.054902}%
\pgfsetfillcolor{currentfill}%
\pgfsetlinewidth{1.003750pt}%
\definecolor{currentstroke}{rgb}{1.000000,0.498039,0.054902}%
\pgfsetstrokecolor{currentstroke}%
\pgfsetdash{}{0pt}%
\pgfpathmoveto{\pgfqpoint{2.134996in}{2.690255in}}%
\pgfpathcurveto{\pgfqpoint{2.146046in}{2.690255in}}{\pgfqpoint{2.156645in}{2.694645in}}{\pgfqpoint{2.164459in}{2.702459in}}%
\pgfpathcurveto{\pgfqpoint{2.172272in}{2.710273in}}{\pgfqpoint{2.176662in}{2.720872in}}{\pgfqpoint{2.176662in}{2.731922in}}%
\pgfpathcurveto{\pgfqpoint{2.176662in}{2.742972in}}{\pgfqpoint{2.172272in}{2.753571in}}{\pgfqpoint{2.164459in}{2.761384in}}%
\pgfpathcurveto{\pgfqpoint{2.156645in}{2.769198in}}{\pgfqpoint{2.146046in}{2.773588in}}{\pgfqpoint{2.134996in}{2.773588in}}%
\pgfpathcurveto{\pgfqpoint{2.123946in}{2.773588in}}{\pgfqpoint{2.113347in}{2.769198in}}{\pgfqpoint{2.105533in}{2.761384in}}%
\pgfpathcurveto{\pgfqpoint{2.097719in}{2.753571in}}{\pgfqpoint{2.093329in}{2.742972in}}{\pgfqpoint{2.093329in}{2.731922in}}%
\pgfpathcurveto{\pgfqpoint{2.093329in}{2.720872in}}{\pgfqpoint{2.097719in}{2.710273in}}{\pgfqpoint{2.105533in}{2.702459in}}%
\pgfpathcurveto{\pgfqpoint{2.113347in}{2.694645in}}{\pgfqpoint{2.123946in}{2.690255in}}{\pgfqpoint{2.134996in}{2.690255in}}%
\pgfpathclose%
\pgfusepath{stroke,fill}%
\end{pgfscope}%
\begin{pgfscope}%
\pgfpathrectangle{\pgfqpoint{0.787074in}{0.548769in}}{\pgfqpoint{5.062926in}{3.102590in}}%
\pgfusepath{clip}%
\pgfsetbuttcap%
\pgfsetroundjoin%
\definecolor{currentfill}{rgb}{1.000000,0.498039,0.054902}%
\pgfsetfillcolor{currentfill}%
\pgfsetlinewidth{1.003750pt}%
\definecolor{currentstroke}{rgb}{1.000000,0.498039,0.054902}%
\pgfsetstrokecolor{currentstroke}%
\pgfsetdash{}{0pt}%
\pgfpathmoveto{\pgfqpoint{1.674730in}{3.301211in}}%
\pgfpathcurveto{\pgfqpoint{1.685780in}{3.301211in}}{\pgfqpoint{1.696379in}{3.305601in}}{\pgfqpoint{1.704193in}{3.313415in}}%
\pgfpathcurveto{\pgfqpoint{1.712006in}{3.321228in}}{\pgfqpoint{1.716396in}{3.331828in}}{\pgfqpoint{1.716396in}{3.342878in}}%
\pgfpathcurveto{\pgfqpoint{1.716396in}{3.353928in}}{\pgfqpoint{1.712006in}{3.364527in}}{\pgfqpoint{1.704193in}{3.372340in}}%
\pgfpathcurveto{\pgfqpoint{1.696379in}{3.380154in}}{\pgfqpoint{1.685780in}{3.384544in}}{\pgfqpoint{1.674730in}{3.384544in}}%
\pgfpathcurveto{\pgfqpoint{1.663680in}{3.384544in}}{\pgfqpoint{1.653081in}{3.380154in}}{\pgfqpoint{1.645267in}{3.372340in}}%
\pgfpathcurveto{\pgfqpoint{1.637453in}{3.364527in}}{\pgfqpoint{1.633063in}{3.353928in}}{\pgfqpoint{1.633063in}{3.342878in}}%
\pgfpathcurveto{\pgfqpoint{1.633063in}{3.331828in}}{\pgfqpoint{1.637453in}{3.321228in}}{\pgfqpoint{1.645267in}{3.313415in}}%
\pgfpathcurveto{\pgfqpoint{1.653081in}{3.305601in}}{\pgfqpoint{1.663680in}{3.301211in}}{\pgfqpoint{1.674730in}{3.301211in}}%
\pgfpathclose%
\pgfusepath{stroke,fill}%
\end{pgfscope}%
\begin{pgfscope}%
\pgfpathrectangle{\pgfqpoint{0.787074in}{0.548769in}}{\pgfqpoint{5.062926in}{3.102590in}}%
\pgfusepath{clip}%
\pgfsetbuttcap%
\pgfsetroundjoin%
\definecolor{currentfill}{rgb}{1.000000,0.498039,0.054902}%
\pgfsetfillcolor{currentfill}%
\pgfsetlinewidth{1.003750pt}%
\definecolor{currentstroke}{rgb}{1.000000,0.498039,0.054902}%
\pgfsetstrokecolor{currentstroke}%
\pgfsetdash{}{0pt}%
\pgfpathmoveto{\pgfqpoint{2.792519in}{3.101406in}}%
\pgfpathcurveto{\pgfqpoint{2.803569in}{3.101406in}}{\pgfqpoint{2.814168in}{3.105796in}}{\pgfqpoint{2.821981in}{3.113610in}}%
\pgfpathcurveto{\pgfqpoint{2.829795in}{3.121423in}}{\pgfqpoint{2.834185in}{3.132022in}}{\pgfqpoint{2.834185in}{3.143073in}}%
\pgfpathcurveto{\pgfqpoint{2.834185in}{3.154123in}}{\pgfqpoint{2.829795in}{3.164722in}}{\pgfqpoint{2.821981in}{3.172535in}}%
\pgfpathcurveto{\pgfqpoint{2.814168in}{3.180349in}}{\pgfqpoint{2.803569in}{3.184739in}}{\pgfqpoint{2.792519in}{3.184739in}}%
\pgfpathcurveto{\pgfqpoint{2.781468in}{3.184739in}}{\pgfqpoint{2.770869in}{3.180349in}}{\pgfqpoint{2.763056in}{3.172535in}}%
\pgfpathcurveto{\pgfqpoint{2.755242in}{3.164722in}}{\pgfqpoint{2.750852in}{3.154123in}}{\pgfqpoint{2.750852in}{3.143073in}}%
\pgfpathcurveto{\pgfqpoint{2.750852in}{3.132022in}}{\pgfqpoint{2.755242in}{3.121423in}}{\pgfqpoint{2.763056in}{3.113610in}}%
\pgfpathcurveto{\pgfqpoint{2.770869in}{3.105796in}}{\pgfqpoint{2.781468in}{3.101406in}}{\pgfqpoint{2.792519in}{3.101406in}}%
\pgfpathclose%
\pgfusepath{stroke,fill}%
\end{pgfscope}%
\begin{pgfscope}%
\pgfpathrectangle{\pgfqpoint{0.787074in}{0.548769in}}{\pgfqpoint{5.062926in}{3.102590in}}%
\pgfusepath{clip}%
\pgfsetbuttcap%
\pgfsetroundjoin%
\definecolor{currentfill}{rgb}{1.000000,0.498039,0.054902}%
\pgfsetfillcolor{currentfill}%
\pgfsetlinewidth{1.003750pt}%
\definecolor{currentstroke}{rgb}{1.000000,0.498039,0.054902}%
\pgfsetstrokecolor{currentstroke}%
\pgfsetdash{}{0pt}%
\pgfpathmoveto{\pgfqpoint{2.069243in}{1.878073in}}%
\pgfpathcurveto{\pgfqpoint{2.080294in}{1.878073in}}{\pgfqpoint{2.090893in}{1.882464in}}{\pgfqpoint{2.098706in}{1.890277in}}%
\pgfpathcurveto{\pgfqpoint{2.106520in}{1.898091in}}{\pgfqpoint{2.110910in}{1.908690in}}{\pgfqpoint{2.110910in}{1.919740in}}%
\pgfpathcurveto{\pgfqpoint{2.110910in}{1.930790in}}{\pgfqpoint{2.106520in}{1.941389in}}{\pgfqpoint{2.098706in}{1.949203in}}%
\pgfpathcurveto{\pgfqpoint{2.090893in}{1.957016in}}{\pgfqpoint{2.080294in}{1.961407in}}{\pgfqpoint{2.069243in}{1.961407in}}%
\pgfpathcurveto{\pgfqpoint{2.058193in}{1.961407in}}{\pgfqpoint{2.047594in}{1.957016in}}{\pgfqpoint{2.039781in}{1.949203in}}%
\pgfpathcurveto{\pgfqpoint{2.031967in}{1.941389in}}{\pgfqpoint{2.027577in}{1.930790in}}{\pgfqpoint{2.027577in}{1.919740in}}%
\pgfpathcurveto{\pgfqpoint{2.027577in}{1.908690in}}{\pgfqpoint{2.031967in}{1.898091in}}{\pgfqpoint{2.039781in}{1.890277in}}%
\pgfpathcurveto{\pgfqpoint{2.047594in}{1.882464in}}{\pgfqpoint{2.058193in}{1.878073in}}{\pgfqpoint{2.069243in}{1.878073in}}%
\pgfpathclose%
\pgfusepath{stroke,fill}%
\end{pgfscope}%
\begin{pgfscope}%
\pgfpathrectangle{\pgfqpoint{0.787074in}{0.548769in}}{\pgfqpoint{5.062926in}{3.102590in}}%
\pgfusepath{clip}%
\pgfsetbuttcap%
\pgfsetroundjoin%
\definecolor{currentfill}{rgb}{1.000000,0.498039,0.054902}%
\pgfsetfillcolor{currentfill}%
\pgfsetlinewidth{1.003750pt}%
\definecolor{currentstroke}{rgb}{1.000000,0.498039,0.054902}%
\pgfsetstrokecolor{currentstroke}%
\pgfsetdash{}{0pt}%
\pgfpathmoveto{\pgfqpoint{2.332253in}{2.761817in}}%
\pgfpathcurveto{\pgfqpoint{2.343303in}{2.761817in}}{\pgfqpoint{2.353902in}{2.766208in}}{\pgfqpoint{2.361715in}{2.774021in}}%
\pgfpathcurveto{\pgfqpoint{2.369529in}{2.781835in}}{\pgfqpoint{2.373919in}{2.792434in}}{\pgfqpoint{2.373919in}{2.803484in}}%
\pgfpathcurveto{\pgfqpoint{2.373919in}{2.814534in}}{\pgfqpoint{2.369529in}{2.825133in}}{\pgfqpoint{2.361715in}{2.832947in}}%
\pgfpathcurveto{\pgfqpoint{2.353902in}{2.840760in}}{\pgfqpoint{2.343303in}{2.845151in}}{\pgfqpoint{2.332253in}{2.845151in}}%
\pgfpathcurveto{\pgfqpoint{2.321202in}{2.845151in}}{\pgfqpoint{2.310603in}{2.840760in}}{\pgfqpoint{2.302790in}{2.832947in}}%
\pgfpathcurveto{\pgfqpoint{2.294976in}{2.825133in}}{\pgfqpoint{2.290586in}{2.814534in}}{\pgfqpoint{2.290586in}{2.803484in}}%
\pgfpathcurveto{\pgfqpoint{2.290586in}{2.792434in}}{\pgfqpoint{2.294976in}{2.781835in}}{\pgfqpoint{2.302790in}{2.774021in}}%
\pgfpathcurveto{\pgfqpoint{2.310603in}{2.766208in}}{\pgfqpoint{2.321202in}{2.761817in}}{\pgfqpoint{2.332253in}{2.761817in}}%
\pgfpathclose%
\pgfusepath{stroke,fill}%
\end{pgfscope}%
\begin{pgfscope}%
\pgfpathrectangle{\pgfqpoint{0.787074in}{0.548769in}}{\pgfqpoint{5.062926in}{3.102590in}}%
\pgfusepath{clip}%
\pgfsetbuttcap%
\pgfsetroundjoin%
\definecolor{currentfill}{rgb}{1.000000,0.498039,0.054902}%
\pgfsetfillcolor{currentfill}%
\pgfsetlinewidth{1.003750pt}%
\definecolor{currentstroke}{rgb}{1.000000,0.498039,0.054902}%
\pgfsetstrokecolor{currentstroke}%
\pgfsetdash{}{0pt}%
\pgfpathmoveto{\pgfqpoint{3.844555in}{2.075959in}}%
\pgfpathcurveto{\pgfqpoint{3.855605in}{2.075959in}}{\pgfqpoint{3.866204in}{2.080350in}}{\pgfqpoint{3.874018in}{2.088163in}}%
\pgfpathcurveto{\pgfqpoint{3.881832in}{2.095977in}}{\pgfqpoint{3.886222in}{2.106576in}}{\pgfqpoint{3.886222in}{2.117626in}}%
\pgfpathcurveto{\pgfqpoint{3.886222in}{2.128676in}}{\pgfqpoint{3.881832in}{2.139275in}}{\pgfqpoint{3.874018in}{2.147089in}}%
\pgfpathcurveto{\pgfqpoint{3.866204in}{2.154902in}}{\pgfqpoint{3.855605in}{2.159293in}}{\pgfqpoint{3.844555in}{2.159293in}}%
\pgfpathcurveto{\pgfqpoint{3.833505in}{2.159293in}}{\pgfqpoint{3.822906in}{2.154902in}}{\pgfqpoint{3.815092in}{2.147089in}}%
\pgfpathcurveto{\pgfqpoint{3.807279in}{2.139275in}}{\pgfqpoint{3.802889in}{2.128676in}}{\pgfqpoint{3.802889in}{2.117626in}}%
\pgfpathcurveto{\pgfqpoint{3.802889in}{2.106576in}}{\pgfqpoint{3.807279in}{2.095977in}}{\pgfqpoint{3.815092in}{2.088163in}}%
\pgfpathcurveto{\pgfqpoint{3.822906in}{2.080350in}}{\pgfqpoint{3.833505in}{2.075959in}}{\pgfqpoint{3.844555in}{2.075959in}}%
\pgfpathclose%
\pgfusepath{stroke,fill}%
\end{pgfscope}%
\begin{pgfscope}%
\pgfpathrectangle{\pgfqpoint{0.787074in}{0.548769in}}{\pgfqpoint{5.062926in}{3.102590in}}%
\pgfusepath{clip}%
\pgfsetbuttcap%
\pgfsetroundjoin%
\definecolor{currentfill}{rgb}{1.000000,0.498039,0.054902}%
\pgfsetfillcolor{currentfill}%
\pgfsetlinewidth{1.003750pt}%
\definecolor{currentstroke}{rgb}{1.000000,0.498039,0.054902}%
\pgfsetstrokecolor{currentstroke}%
\pgfsetdash{}{0pt}%
\pgfpathmoveto{\pgfqpoint{2.463757in}{2.369584in}}%
\pgfpathcurveto{\pgfqpoint{2.474807in}{2.369584in}}{\pgfqpoint{2.485406in}{2.373975in}}{\pgfqpoint{2.493220in}{2.381788in}}%
\pgfpathcurveto{\pgfqpoint{2.501034in}{2.389602in}}{\pgfqpoint{2.505424in}{2.400201in}}{\pgfqpoint{2.505424in}{2.411251in}}%
\pgfpathcurveto{\pgfqpoint{2.505424in}{2.422301in}}{\pgfqpoint{2.501034in}{2.432900in}}{\pgfqpoint{2.493220in}{2.440714in}}%
\pgfpathcurveto{\pgfqpoint{2.485406in}{2.448527in}}{\pgfqpoint{2.474807in}{2.452918in}}{\pgfqpoint{2.463757in}{2.452918in}}%
\pgfpathcurveto{\pgfqpoint{2.452707in}{2.452918in}}{\pgfqpoint{2.442108in}{2.448527in}}{\pgfqpoint{2.434294in}{2.440714in}}%
\pgfpathcurveto{\pgfqpoint{2.426481in}{2.432900in}}{\pgfqpoint{2.422091in}{2.422301in}}{\pgfqpoint{2.422091in}{2.411251in}}%
\pgfpathcurveto{\pgfqpoint{2.422091in}{2.400201in}}{\pgfqpoint{2.426481in}{2.389602in}}{\pgfqpoint{2.434294in}{2.381788in}}%
\pgfpathcurveto{\pgfqpoint{2.442108in}{2.373975in}}{\pgfqpoint{2.452707in}{2.369584in}}{\pgfqpoint{2.463757in}{2.369584in}}%
\pgfpathclose%
\pgfusepath{stroke,fill}%
\end{pgfscope}%
\begin{pgfscope}%
\pgfpathrectangle{\pgfqpoint{0.787074in}{0.548769in}}{\pgfqpoint{5.062926in}{3.102590in}}%
\pgfusepath{clip}%
\pgfsetbuttcap%
\pgfsetroundjoin%
\definecolor{currentfill}{rgb}{0.121569,0.466667,0.705882}%
\pgfsetfillcolor{currentfill}%
\pgfsetlinewidth{1.003750pt}%
\definecolor{currentstroke}{rgb}{0.121569,0.466667,0.705882}%
\pgfsetstrokecolor{currentstroke}%
\pgfsetdash{}{0pt}%
\pgfpathmoveto{\pgfqpoint{1.082959in}{0.648131in}}%
\pgfpathcurveto{\pgfqpoint{1.094009in}{0.648131in}}{\pgfqpoint{1.104608in}{0.652521in}}{\pgfqpoint{1.112422in}{0.660335in}}%
\pgfpathcurveto{\pgfqpoint{1.120236in}{0.668148in}}{\pgfqpoint{1.124626in}{0.678747in}}{\pgfqpoint{1.124626in}{0.689798in}}%
\pgfpathcurveto{\pgfqpoint{1.124626in}{0.700848in}}{\pgfqpoint{1.120236in}{0.711447in}}{\pgfqpoint{1.112422in}{0.719260in}}%
\pgfpathcurveto{\pgfqpoint{1.104608in}{0.727074in}}{\pgfqpoint{1.094009in}{0.731464in}}{\pgfqpoint{1.082959in}{0.731464in}}%
\pgfpathcurveto{\pgfqpoint{1.071909in}{0.731464in}}{\pgfqpoint{1.061310in}{0.727074in}}{\pgfqpoint{1.053496in}{0.719260in}}%
\pgfpathcurveto{\pgfqpoint{1.045683in}{0.711447in}}{\pgfqpoint{1.041292in}{0.700848in}}{\pgfqpoint{1.041292in}{0.689798in}}%
\pgfpathcurveto{\pgfqpoint{1.041292in}{0.678747in}}{\pgfqpoint{1.045683in}{0.668148in}}{\pgfqpoint{1.053496in}{0.660335in}}%
\pgfpathcurveto{\pgfqpoint{1.061310in}{0.652521in}}{\pgfqpoint{1.071909in}{0.648131in}}{\pgfqpoint{1.082959in}{0.648131in}}%
\pgfpathclose%
\pgfusepath{stroke,fill}%
\end{pgfscope}%
\begin{pgfscope}%
\pgfpathrectangle{\pgfqpoint{0.787074in}{0.548769in}}{\pgfqpoint{5.062926in}{3.102590in}}%
\pgfusepath{clip}%
\pgfsetbuttcap%
\pgfsetroundjoin%
\definecolor{currentfill}{rgb}{1.000000,0.498039,0.054902}%
\pgfsetfillcolor{currentfill}%
\pgfsetlinewidth{1.003750pt}%
\definecolor{currentstroke}{rgb}{1.000000,0.498039,0.054902}%
\pgfsetstrokecolor{currentstroke}%
\pgfsetdash{}{0pt}%
\pgfpathmoveto{\pgfqpoint{4.107564in}{1.884932in}}%
\pgfpathcurveto{\pgfqpoint{4.118615in}{1.884932in}}{\pgfqpoint{4.129214in}{1.889322in}}{\pgfqpoint{4.137027in}{1.897136in}}%
\pgfpathcurveto{\pgfqpoint{4.144841in}{1.904950in}}{\pgfqpoint{4.149231in}{1.915549in}}{\pgfqpoint{4.149231in}{1.926599in}}%
\pgfpathcurveto{\pgfqpoint{4.149231in}{1.937649in}}{\pgfqpoint{4.144841in}{1.948248in}}{\pgfqpoint{4.137027in}{1.956061in}}%
\pgfpathcurveto{\pgfqpoint{4.129214in}{1.963875in}}{\pgfqpoint{4.118615in}{1.968265in}}{\pgfqpoint{4.107564in}{1.968265in}}%
\pgfpathcurveto{\pgfqpoint{4.096514in}{1.968265in}}{\pgfqpoint{4.085915in}{1.963875in}}{\pgfqpoint{4.078102in}{1.956061in}}%
\pgfpathcurveto{\pgfqpoint{4.070288in}{1.948248in}}{\pgfqpoint{4.065898in}{1.937649in}}{\pgfqpoint{4.065898in}{1.926599in}}%
\pgfpathcurveto{\pgfqpoint{4.065898in}{1.915549in}}{\pgfqpoint{4.070288in}{1.904950in}}{\pgfqpoint{4.078102in}{1.897136in}}%
\pgfpathcurveto{\pgfqpoint{4.085915in}{1.889322in}}{\pgfqpoint{4.096514in}{1.884932in}}{\pgfqpoint{4.107564in}{1.884932in}}%
\pgfpathclose%
\pgfusepath{stroke,fill}%
\end{pgfscope}%
\begin{pgfscope}%
\pgfpathrectangle{\pgfqpoint{0.787074in}{0.548769in}}{\pgfqpoint{5.062926in}{3.102590in}}%
\pgfusepath{clip}%
\pgfsetbuttcap%
\pgfsetroundjoin%
\definecolor{currentfill}{rgb}{1.000000,0.498039,0.054902}%
\pgfsetfillcolor{currentfill}%
\pgfsetlinewidth{1.003750pt}%
\definecolor{currentstroke}{rgb}{1.000000,0.498039,0.054902}%
\pgfsetstrokecolor{currentstroke}%
\pgfsetdash{}{0pt}%
\pgfpathmoveto{\pgfqpoint{1.543225in}{1.779524in}}%
\pgfpathcurveto{\pgfqpoint{1.554275in}{1.779524in}}{\pgfqpoint{1.564874in}{1.783914in}}{\pgfqpoint{1.572688in}{1.791728in}}%
\pgfpathcurveto{\pgfqpoint{1.580502in}{1.799541in}}{\pgfqpoint{1.584892in}{1.810140in}}{\pgfqpoint{1.584892in}{1.821190in}}%
\pgfpathcurveto{\pgfqpoint{1.584892in}{1.832241in}}{\pgfqpoint{1.580502in}{1.842840in}}{\pgfqpoint{1.572688in}{1.850653in}}%
\pgfpathcurveto{\pgfqpoint{1.564874in}{1.858467in}}{\pgfqpoint{1.554275in}{1.862857in}}{\pgfqpoint{1.543225in}{1.862857in}}%
\pgfpathcurveto{\pgfqpoint{1.532175in}{1.862857in}}{\pgfqpoint{1.521576in}{1.858467in}}{\pgfqpoint{1.513762in}{1.850653in}}%
\pgfpathcurveto{\pgfqpoint{1.505949in}{1.842840in}}{\pgfqpoint{1.501558in}{1.832241in}}{\pgfqpoint{1.501558in}{1.821190in}}%
\pgfpathcurveto{\pgfqpoint{1.501558in}{1.810140in}}{\pgfqpoint{1.505949in}{1.799541in}}{\pgfqpoint{1.513762in}{1.791728in}}%
\pgfpathcurveto{\pgfqpoint{1.521576in}{1.783914in}}{\pgfqpoint{1.532175in}{1.779524in}}{\pgfqpoint{1.543225in}{1.779524in}}%
\pgfpathclose%
\pgfusepath{stroke,fill}%
\end{pgfscope}%
\begin{pgfscope}%
\pgfpathrectangle{\pgfqpoint{0.787074in}{0.548769in}}{\pgfqpoint{5.062926in}{3.102590in}}%
\pgfusepath{clip}%
\pgfsetbuttcap%
\pgfsetroundjoin%
\definecolor{currentfill}{rgb}{1.000000,0.498039,0.054902}%
\pgfsetfillcolor{currentfill}%
\pgfsetlinewidth{1.003750pt}%
\definecolor{currentstroke}{rgb}{1.000000,0.498039,0.054902}%
\pgfsetstrokecolor{currentstroke}%
\pgfsetdash{}{0pt}%
\pgfpathmoveto{\pgfqpoint{1.477473in}{2.435775in}}%
\pgfpathcurveto{\pgfqpoint{1.488523in}{2.435775in}}{\pgfqpoint{1.499122in}{2.440165in}}{\pgfqpoint{1.506936in}{2.447979in}}%
\pgfpathcurveto{\pgfqpoint{1.514749in}{2.455792in}}{\pgfqpoint{1.519140in}{2.466391in}}{\pgfqpoint{1.519140in}{2.477441in}}%
\pgfpathcurveto{\pgfqpoint{1.519140in}{2.488492in}}{\pgfqpoint{1.514749in}{2.499091in}}{\pgfqpoint{1.506936in}{2.506904in}}%
\pgfpathcurveto{\pgfqpoint{1.499122in}{2.514718in}}{\pgfqpoint{1.488523in}{2.519108in}}{\pgfqpoint{1.477473in}{2.519108in}}%
\pgfpathcurveto{\pgfqpoint{1.466423in}{2.519108in}}{\pgfqpoint{1.455824in}{2.514718in}}{\pgfqpoint{1.448010in}{2.506904in}}%
\pgfpathcurveto{\pgfqpoint{1.440196in}{2.499091in}}{\pgfqpoint{1.435806in}{2.488492in}}{\pgfqpoint{1.435806in}{2.477441in}}%
\pgfpathcurveto{\pgfqpoint{1.435806in}{2.466391in}}{\pgfqpoint{1.440196in}{2.455792in}}{\pgfqpoint{1.448010in}{2.447979in}}%
\pgfpathcurveto{\pgfqpoint{1.455824in}{2.440165in}}{\pgfqpoint{1.466423in}{2.435775in}}{\pgfqpoint{1.477473in}{2.435775in}}%
\pgfpathclose%
\pgfusepath{stroke,fill}%
\end{pgfscope}%
\begin{pgfscope}%
\pgfpathrectangle{\pgfqpoint{0.787074in}{0.548769in}}{\pgfqpoint{5.062926in}{3.102590in}}%
\pgfusepath{clip}%
\pgfsetbuttcap%
\pgfsetroundjoin%
\definecolor{currentfill}{rgb}{1.000000,0.498039,0.054902}%
\pgfsetfillcolor{currentfill}%
\pgfsetlinewidth{1.003750pt}%
\definecolor{currentstroke}{rgb}{1.000000,0.498039,0.054902}%
\pgfsetstrokecolor{currentstroke}%
\pgfsetdash{}{0pt}%
\pgfpathmoveto{\pgfqpoint{2.398005in}{1.639934in}}%
\pgfpathcurveto{\pgfqpoint{2.409055in}{1.639934in}}{\pgfqpoint{2.419654in}{1.644324in}}{\pgfqpoint{2.427468in}{1.652138in}}%
\pgfpathcurveto{\pgfqpoint{2.435281in}{1.659951in}}{\pgfqpoint{2.439672in}{1.670550in}}{\pgfqpoint{2.439672in}{1.681600in}}%
\pgfpathcurveto{\pgfqpoint{2.439672in}{1.692650in}}{\pgfqpoint{2.435281in}{1.703249in}}{\pgfqpoint{2.427468in}{1.711063in}}%
\pgfpathcurveto{\pgfqpoint{2.419654in}{1.718877in}}{\pgfqpoint{2.409055in}{1.723267in}}{\pgfqpoint{2.398005in}{1.723267in}}%
\pgfpathcurveto{\pgfqpoint{2.386955in}{1.723267in}}{\pgfqpoint{2.376356in}{1.718877in}}{\pgfqpoint{2.368542in}{1.711063in}}%
\pgfpathcurveto{\pgfqpoint{2.360728in}{1.703249in}}{\pgfqpoint{2.356338in}{1.692650in}}{\pgfqpoint{2.356338in}{1.681600in}}%
\pgfpathcurveto{\pgfqpoint{2.356338in}{1.670550in}}{\pgfqpoint{2.360728in}{1.659951in}}{\pgfqpoint{2.368542in}{1.652138in}}%
\pgfpathcurveto{\pgfqpoint{2.376356in}{1.644324in}}{\pgfqpoint{2.386955in}{1.639934in}}{\pgfqpoint{2.398005in}{1.639934in}}%
\pgfpathclose%
\pgfusepath{stroke,fill}%
\end{pgfscope}%
\begin{pgfscope}%
\pgfpathrectangle{\pgfqpoint{0.787074in}{0.548769in}}{\pgfqpoint{5.062926in}{3.102590in}}%
\pgfusepath{clip}%
\pgfsetbuttcap%
\pgfsetroundjoin%
\definecolor{currentfill}{rgb}{1.000000,0.498039,0.054902}%
\pgfsetfillcolor{currentfill}%
\pgfsetlinewidth{1.003750pt}%
\definecolor{currentstroke}{rgb}{1.000000,0.498039,0.054902}%
\pgfsetstrokecolor{currentstroke}%
\pgfsetdash{}{0pt}%
\pgfpathmoveto{\pgfqpoint{1.740482in}{1.770785in}}%
\pgfpathcurveto{\pgfqpoint{1.751532in}{1.770785in}}{\pgfqpoint{1.762131in}{1.775176in}}{\pgfqpoint{1.769945in}{1.782989in}}%
\pgfpathcurveto{\pgfqpoint{1.777758in}{1.790803in}}{\pgfqpoint{1.782149in}{1.801402in}}{\pgfqpoint{1.782149in}{1.812452in}}%
\pgfpathcurveto{\pgfqpoint{1.782149in}{1.823502in}}{\pgfqpoint{1.777758in}{1.834101in}}{\pgfqpoint{1.769945in}{1.841915in}}%
\pgfpathcurveto{\pgfqpoint{1.762131in}{1.849728in}}{\pgfqpoint{1.751532in}{1.854119in}}{\pgfqpoint{1.740482in}{1.854119in}}%
\pgfpathcurveto{\pgfqpoint{1.729432in}{1.854119in}}{\pgfqpoint{1.718833in}{1.849728in}}{\pgfqpoint{1.711019in}{1.841915in}}%
\pgfpathcurveto{\pgfqpoint{1.703206in}{1.834101in}}{\pgfqpoint{1.698815in}{1.823502in}}{\pgfqpoint{1.698815in}{1.812452in}}%
\pgfpathcurveto{\pgfqpoint{1.698815in}{1.801402in}}{\pgfqpoint{1.703206in}{1.790803in}}{\pgfqpoint{1.711019in}{1.782989in}}%
\pgfpathcurveto{\pgfqpoint{1.718833in}{1.775176in}}{\pgfqpoint{1.729432in}{1.770785in}}{\pgfqpoint{1.740482in}{1.770785in}}%
\pgfpathclose%
\pgfusepath{stroke,fill}%
\end{pgfscope}%
\begin{pgfscope}%
\pgfpathrectangle{\pgfqpoint{0.787074in}{0.548769in}}{\pgfqpoint{5.062926in}{3.102590in}}%
\pgfusepath{clip}%
\pgfsetbuttcap%
\pgfsetroundjoin%
\definecolor{currentfill}{rgb}{1.000000,0.498039,0.054902}%
\pgfsetfillcolor{currentfill}%
\pgfsetlinewidth{1.003750pt}%
\definecolor{currentstroke}{rgb}{1.000000,0.498039,0.054902}%
\pgfsetstrokecolor{currentstroke}%
\pgfsetdash{}{0pt}%
\pgfpathmoveto{\pgfqpoint{2.069243in}{2.028373in}}%
\pgfpathcurveto{\pgfqpoint{2.080294in}{2.028373in}}{\pgfqpoint{2.090893in}{2.032763in}}{\pgfqpoint{2.098706in}{2.040577in}}%
\pgfpathcurveto{\pgfqpoint{2.106520in}{2.048390in}}{\pgfqpoint{2.110910in}{2.058989in}}{\pgfqpoint{2.110910in}{2.070039in}}%
\pgfpathcurveto{\pgfqpoint{2.110910in}{2.081090in}}{\pgfqpoint{2.106520in}{2.091689in}}{\pgfqpoint{2.098706in}{2.099502in}}%
\pgfpathcurveto{\pgfqpoint{2.090893in}{2.107316in}}{\pgfqpoint{2.080294in}{2.111706in}}{\pgfqpoint{2.069243in}{2.111706in}}%
\pgfpathcurveto{\pgfqpoint{2.058193in}{2.111706in}}{\pgfqpoint{2.047594in}{2.107316in}}{\pgfqpoint{2.039781in}{2.099502in}}%
\pgfpathcurveto{\pgfqpoint{2.031967in}{2.091689in}}{\pgfqpoint{2.027577in}{2.081090in}}{\pgfqpoint{2.027577in}{2.070039in}}%
\pgfpathcurveto{\pgfqpoint{2.027577in}{2.058989in}}{\pgfqpoint{2.031967in}{2.048390in}}{\pgfqpoint{2.039781in}{2.040577in}}%
\pgfpathcurveto{\pgfqpoint{2.047594in}{2.032763in}}{\pgfqpoint{2.058193in}{2.028373in}}{\pgfqpoint{2.069243in}{2.028373in}}%
\pgfpathclose%
\pgfusepath{stroke,fill}%
\end{pgfscope}%
\begin{pgfscope}%
\pgfpathrectangle{\pgfqpoint{0.787074in}{0.548769in}}{\pgfqpoint{5.062926in}{3.102590in}}%
\pgfusepath{clip}%
\pgfsetbuttcap%
\pgfsetroundjoin%
\definecolor{currentfill}{rgb}{1.000000,0.498039,0.054902}%
\pgfsetfillcolor{currentfill}%
\pgfsetlinewidth{1.003750pt}%
\definecolor{currentstroke}{rgb}{1.000000,0.498039,0.054902}%
\pgfsetstrokecolor{currentstroke}%
\pgfsetdash{}{0pt}%
\pgfpathmoveto{\pgfqpoint{3.252785in}{2.904196in}}%
\pgfpathcurveto{\pgfqpoint{3.263835in}{2.904196in}}{\pgfqpoint{3.274434in}{2.908586in}}{\pgfqpoint{3.282247in}{2.916399in}}%
\pgfpathcurveto{\pgfqpoint{3.290061in}{2.924213in}}{\pgfqpoint{3.294451in}{2.934812in}}{\pgfqpoint{3.294451in}{2.945862in}}%
\pgfpathcurveto{\pgfqpoint{3.294451in}{2.956912in}}{\pgfqpoint{3.290061in}{2.967511in}}{\pgfqpoint{3.282247in}{2.975325in}}%
\pgfpathcurveto{\pgfqpoint{3.274434in}{2.983139in}}{\pgfqpoint{3.263835in}{2.987529in}}{\pgfqpoint{3.252785in}{2.987529in}}%
\pgfpathcurveto{\pgfqpoint{3.241735in}{2.987529in}}{\pgfqpoint{3.231135in}{2.983139in}}{\pgfqpoint{3.223322in}{2.975325in}}%
\pgfpathcurveto{\pgfqpoint{3.215508in}{2.967511in}}{\pgfqpoint{3.211118in}{2.956912in}}{\pgfqpoint{3.211118in}{2.945862in}}%
\pgfpathcurveto{\pgfqpoint{3.211118in}{2.934812in}}{\pgfqpoint{3.215508in}{2.924213in}}{\pgfqpoint{3.223322in}{2.916399in}}%
\pgfpathcurveto{\pgfqpoint{3.231135in}{2.908586in}}{\pgfqpoint{3.241735in}{2.904196in}}{\pgfqpoint{3.252785in}{2.904196in}}%
\pgfpathclose%
\pgfusepath{stroke,fill}%
\end{pgfscope}%
\begin{pgfscope}%
\pgfpathrectangle{\pgfqpoint{0.787074in}{0.548769in}}{\pgfqpoint{5.062926in}{3.102590in}}%
\pgfusepath{clip}%
\pgfsetbuttcap%
\pgfsetroundjoin%
\definecolor{currentfill}{rgb}{1.000000,0.498039,0.054902}%
\pgfsetfillcolor{currentfill}%
\pgfsetlinewidth{1.003750pt}%
\definecolor{currentstroke}{rgb}{1.000000,0.498039,0.054902}%
\pgfsetstrokecolor{currentstroke}%
\pgfsetdash{}{0pt}%
\pgfpathmoveto{\pgfqpoint{1.806234in}{3.239789in}}%
\pgfpathcurveto{\pgfqpoint{1.817284in}{3.239789in}}{\pgfqpoint{1.827883in}{3.244179in}}{\pgfqpoint{1.835697in}{3.251992in}}%
\pgfpathcurveto{\pgfqpoint{1.843511in}{3.259806in}}{\pgfqpoint{1.847901in}{3.270405in}}{\pgfqpoint{1.847901in}{3.281455in}}%
\pgfpathcurveto{\pgfqpoint{1.847901in}{3.292505in}}{\pgfqpoint{1.843511in}{3.303104in}}{\pgfqpoint{1.835697in}{3.310918in}}%
\pgfpathcurveto{\pgfqpoint{1.827883in}{3.318732in}}{\pgfqpoint{1.817284in}{3.323122in}}{\pgfqpoint{1.806234in}{3.323122in}}%
\pgfpathcurveto{\pgfqpoint{1.795184in}{3.323122in}}{\pgfqpoint{1.784585in}{3.318732in}}{\pgfqpoint{1.776772in}{3.310918in}}%
\pgfpathcurveto{\pgfqpoint{1.768958in}{3.303104in}}{\pgfqpoint{1.764568in}{3.292505in}}{\pgfqpoint{1.764568in}{3.281455in}}%
\pgfpathcurveto{\pgfqpoint{1.764568in}{3.270405in}}{\pgfqpoint{1.768958in}{3.259806in}}{\pgfqpoint{1.776772in}{3.251992in}}%
\pgfpathcurveto{\pgfqpoint{1.784585in}{3.244179in}}{\pgfqpoint{1.795184in}{3.239789in}}{\pgfqpoint{1.806234in}{3.239789in}}%
\pgfpathclose%
\pgfusepath{stroke,fill}%
\end{pgfscope}%
\begin{pgfscope}%
\pgfsetbuttcap%
\pgfsetroundjoin%
\definecolor{currentfill}{rgb}{0.000000,0.000000,0.000000}%
\pgfsetfillcolor{currentfill}%
\pgfsetlinewidth{0.803000pt}%
\definecolor{currentstroke}{rgb}{0.000000,0.000000,0.000000}%
\pgfsetstrokecolor{currentstroke}%
\pgfsetdash{}{0pt}%
\pgfsys@defobject{currentmarker}{\pgfqpoint{0.000000in}{-0.048611in}}{\pgfqpoint{0.000000in}{0.000000in}}{%
\pgfpathmoveto{\pgfqpoint{0.000000in}{0.000000in}}%
\pgfpathlineto{\pgfqpoint{0.000000in}{-0.048611in}}%
\pgfusepath{stroke,fill}%
}%
\begin{pgfscope}%
\pgfsys@transformshift{0.951455in}{0.548769in}%
\pgfsys@useobject{currentmarker}{}%
\end{pgfscope}%
\end{pgfscope}%
\begin{pgfscope}%
\definecolor{textcolor}{rgb}{0.000000,0.000000,0.000000}%
\pgfsetstrokecolor{textcolor}%
\pgfsetfillcolor{textcolor}%
\pgftext[x=0.951455in,y=0.451547in,,top]{\color{textcolor}\sffamily\fontsize{10.000000}{12.000000}\selectfont \(\displaystyle {0}\)}%
\end{pgfscope}%
\begin{pgfscope}%
\pgfsetbuttcap%
\pgfsetroundjoin%
\definecolor{currentfill}{rgb}{0.000000,0.000000,0.000000}%
\pgfsetfillcolor{currentfill}%
\pgfsetlinewidth{0.803000pt}%
\definecolor{currentstroke}{rgb}{0.000000,0.000000,0.000000}%
\pgfsetstrokecolor{currentstroke}%
\pgfsetdash{}{0pt}%
\pgfsys@defobject{currentmarker}{\pgfqpoint{0.000000in}{-0.048611in}}{\pgfqpoint{0.000000in}{0.000000in}}{%
\pgfpathmoveto{\pgfqpoint{0.000000in}{0.000000in}}%
\pgfpathlineto{\pgfqpoint{0.000000in}{-0.048611in}}%
\pgfusepath{stroke,fill}%
}%
\begin{pgfscope}%
\pgfsys@transformshift{1.608977in}{0.548769in}%
\pgfsys@useobject{currentmarker}{}%
\end{pgfscope}%
\end{pgfscope}%
\begin{pgfscope}%
\definecolor{textcolor}{rgb}{0.000000,0.000000,0.000000}%
\pgfsetstrokecolor{textcolor}%
\pgfsetfillcolor{textcolor}%
\pgftext[x=1.608977in,y=0.451547in,,top]{\color{textcolor}\sffamily\fontsize{10.000000}{12.000000}\selectfont \(\displaystyle {10}\)}%
\end{pgfscope}%
\begin{pgfscope}%
\pgfsetbuttcap%
\pgfsetroundjoin%
\definecolor{currentfill}{rgb}{0.000000,0.000000,0.000000}%
\pgfsetfillcolor{currentfill}%
\pgfsetlinewidth{0.803000pt}%
\definecolor{currentstroke}{rgb}{0.000000,0.000000,0.000000}%
\pgfsetstrokecolor{currentstroke}%
\pgfsetdash{}{0pt}%
\pgfsys@defobject{currentmarker}{\pgfqpoint{0.000000in}{-0.048611in}}{\pgfqpoint{0.000000in}{0.000000in}}{%
\pgfpathmoveto{\pgfqpoint{0.000000in}{0.000000in}}%
\pgfpathlineto{\pgfqpoint{0.000000in}{-0.048611in}}%
\pgfusepath{stroke,fill}%
}%
\begin{pgfscope}%
\pgfsys@transformshift{2.266500in}{0.548769in}%
\pgfsys@useobject{currentmarker}{}%
\end{pgfscope}%
\end{pgfscope}%
\begin{pgfscope}%
\definecolor{textcolor}{rgb}{0.000000,0.000000,0.000000}%
\pgfsetstrokecolor{textcolor}%
\pgfsetfillcolor{textcolor}%
\pgftext[x=2.266500in,y=0.451547in,,top]{\color{textcolor}\sffamily\fontsize{10.000000}{12.000000}\selectfont \(\displaystyle {20}\)}%
\end{pgfscope}%
\begin{pgfscope}%
\pgfsetbuttcap%
\pgfsetroundjoin%
\definecolor{currentfill}{rgb}{0.000000,0.000000,0.000000}%
\pgfsetfillcolor{currentfill}%
\pgfsetlinewidth{0.803000pt}%
\definecolor{currentstroke}{rgb}{0.000000,0.000000,0.000000}%
\pgfsetstrokecolor{currentstroke}%
\pgfsetdash{}{0pt}%
\pgfsys@defobject{currentmarker}{\pgfqpoint{0.000000in}{-0.048611in}}{\pgfqpoint{0.000000in}{0.000000in}}{%
\pgfpathmoveto{\pgfqpoint{0.000000in}{0.000000in}}%
\pgfpathlineto{\pgfqpoint{0.000000in}{-0.048611in}}%
\pgfusepath{stroke,fill}%
}%
\begin{pgfscope}%
\pgfsys@transformshift{2.924023in}{0.548769in}%
\pgfsys@useobject{currentmarker}{}%
\end{pgfscope}%
\end{pgfscope}%
\begin{pgfscope}%
\definecolor{textcolor}{rgb}{0.000000,0.000000,0.000000}%
\pgfsetstrokecolor{textcolor}%
\pgfsetfillcolor{textcolor}%
\pgftext[x=2.924023in,y=0.451547in,,top]{\color{textcolor}\sffamily\fontsize{10.000000}{12.000000}\selectfont \(\displaystyle {30}\)}%
\end{pgfscope}%
\begin{pgfscope}%
\pgfsetbuttcap%
\pgfsetroundjoin%
\definecolor{currentfill}{rgb}{0.000000,0.000000,0.000000}%
\pgfsetfillcolor{currentfill}%
\pgfsetlinewidth{0.803000pt}%
\definecolor{currentstroke}{rgb}{0.000000,0.000000,0.000000}%
\pgfsetstrokecolor{currentstroke}%
\pgfsetdash{}{0pt}%
\pgfsys@defobject{currentmarker}{\pgfqpoint{0.000000in}{-0.048611in}}{\pgfqpoint{0.000000in}{0.000000in}}{%
\pgfpathmoveto{\pgfqpoint{0.000000in}{0.000000in}}%
\pgfpathlineto{\pgfqpoint{0.000000in}{-0.048611in}}%
\pgfusepath{stroke,fill}%
}%
\begin{pgfscope}%
\pgfsys@transformshift{3.581546in}{0.548769in}%
\pgfsys@useobject{currentmarker}{}%
\end{pgfscope}%
\end{pgfscope}%
\begin{pgfscope}%
\definecolor{textcolor}{rgb}{0.000000,0.000000,0.000000}%
\pgfsetstrokecolor{textcolor}%
\pgfsetfillcolor{textcolor}%
\pgftext[x=3.581546in,y=0.451547in,,top]{\color{textcolor}\sffamily\fontsize{10.000000}{12.000000}\selectfont \(\displaystyle {40}\)}%
\end{pgfscope}%
\begin{pgfscope}%
\pgfsetbuttcap%
\pgfsetroundjoin%
\definecolor{currentfill}{rgb}{0.000000,0.000000,0.000000}%
\pgfsetfillcolor{currentfill}%
\pgfsetlinewidth{0.803000pt}%
\definecolor{currentstroke}{rgb}{0.000000,0.000000,0.000000}%
\pgfsetstrokecolor{currentstroke}%
\pgfsetdash{}{0pt}%
\pgfsys@defobject{currentmarker}{\pgfqpoint{0.000000in}{-0.048611in}}{\pgfqpoint{0.000000in}{0.000000in}}{%
\pgfpathmoveto{\pgfqpoint{0.000000in}{0.000000in}}%
\pgfpathlineto{\pgfqpoint{0.000000in}{-0.048611in}}%
\pgfusepath{stroke,fill}%
}%
\begin{pgfscope}%
\pgfsys@transformshift{4.239069in}{0.548769in}%
\pgfsys@useobject{currentmarker}{}%
\end{pgfscope}%
\end{pgfscope}%
\begin{pgfscope}%
\definecolor{textcolor}{rgb}{0.000000,0.000000,0.000000}%
\pgfsetstrokecolor{textcolor}%
\pgfsetfillcolor{textcolor}%
\pgftext[x=4.239069in,y=0.451547in,,top]{\color{textcolor}\sffamily\fontsize{10.000000}{12.000000}\selectfont \(\displaystyle {50}\)}%
\end{pgfscope}%
\begin{pgfscope}%
\pgfsetbuttcap%
\pgfsetroundjoin%
\definecolor{currentfill}{rgb}{0.000000,0.000000,0.000000}%
\pgfsetfillcolor{currentfill}%
\pgfsetlinewidth{0.803000pt}%
\definecolor{currentstroke}{rgb}{0.000000,0.000000,0.000000}%
\pgfsetstrokecolor{currentstroke}%
\pgfsetdash{}{0pt}%
\pgfsys@defobject{currentmarker}{\pgfqpoint{0.000000in}{-0.048611in}}{\pgfqpoint{0.000000in}{0.000000in}}{%
\pgfpathmoveto{\pgfqpoint{0.000000in}{0.000000in}}%
\pgfpathlineto{\pgfqpoint{0.000000in}{-0.048611in}}%
\pgfusepath{stroke,fill}%
}%
\begin{pgfscope}%
\pgfsys@transformshift{4.896592in}{0.548769in}%
\pgfsys@useobject{currentmarker}{}%
\end{pgfscope}%
\end{pgfscope}%
\begin{pgfscope}%
\definecolor{textcolor}{rgb}{0.000000,0.000000,0.000000}%
\pgfsetstrokecolor{textcolor}%
\pgfsetfillcolor{textcolor}%
\pgftext[x=4.896592in,y=0.451547in,,top]{\color{textcolor}\sffamily\fontsize{10.000000}{12.000000}\selectfont \(\displaystyle {60}\)}%
\end{pgfscope}%
\begin{pgfscope}%
\pgfsetbuttcap%
\pgfsetroundjoin%
\definecolor{currentfill}{rgb}{0.000000,0.000000,0.000000}%
\pgfsetfillcolor{currentfill}%
\pgfsetlinewidth{0.803000pt}%
\definecolor{currentstroke}{rgb}{0.000000,0.000000,0.000000}%
\pgfsetstrokecolor{currentstroke}%
\pgfsetdash{}{0pt}%
\pgfsys@defobject{currentmarker}{\pgfqpoint{0.000000in}{-0.048611in}}{\pgfqpoint{0.000000in}{0.000000in}}{%
\pgfpathmoveto{\pgfqpoint{0.000000in}{0.000000in}}%
\pgfpathlineto{\pgfqpoint{0.000000in}{-0.048611in}}%
\pgfusepath{stroke,fill}%
}%
\begin{pgfscope}%
\pgfsys@transformshift{5.554115in}{0.548769in}%
\pgfsys@useobject{currentmarker}{}%
\end{pgfscope}%
\end{pgfscope}%
\begin{pgfscope}%
\definecolor{textcolor}{rgb}{0.000000,0.000000,0.000000}%
\pgfsetstrokecolor{textcolor}%
\pgfsetfillcolor{textcolor}%
\pgftext[x=5.554115in,y=0.451547in,,top]{\color{textcolor}\sffamily\fontsize{10.000000}{12.000000}\selectfont \(\displaystyle {70}\)}%
\end{pgfscope}%
\begin{pgfscope}%
\definecolor{textcolor}{rgb}{0.000000,0.000000,0.000000}%
\pgfsetstrokecolor{textcolor}%
\pgfsetfillcolor{textcolor}%
\pgftext[x=3.318537in,y=0.272658in,,top]{\color{textcolor}\sffamily\fontsize{10.000000}{12.000000}\selectfont Number of Sources}%
\end{pgfscope}%
\begin{pgfscope}%
\pgfsetbuttcap%
\pgfsetroundjoin%
\definecolor{currentfill}{rgb}{0.000000,0.000000,0.000000}%
\pgfsetfillcolor{currentfill}%
\pgfsetlinewidth{0.803000pt}%
\definecolor{currentstroke}{rgb}{0.000000,0.000000,0.000000}%
\pgfsetstrokecolor{currentstroke}%
\pgfsetdash{}{0pt}%
\pgfsys@defobject{currentmarker}{\pgfqpoint{-0.048611in}{0.000000in}}{\pgfqpoint{0.000000in}{0.000000in}}{%
\pgfpathmoveto{\pgfqpoint{0.000000in}{0.000000in}}%
\pgfpathlineto{\pgfqpoint{-0.048611in}{0.000000in}}%
\pgfusepath{stroke,fill}%
}%
\begin{pgfscope}%
\pgfsys@transformshift{0.787074in}{0.689795in}%
\pgfsys@useobject{currentmarker}{}%
\end{pgfscope}%
\end{pgfscope}%
\begin{pgfscope}%
\definecolor{textcolor}{rgb}{0.000000,0.000000,0.000000}%
\pgfsetstrokecolor{textcolor}%
\pgfsetfillcolor{textcolor}%
\pgftext[x=0.620407in, y=0.641601in, left, base]{\color{textcolor}\sffamily\fontsize{10.000000}{12.000000}\selectfont \(\displaystyle {0}\)}%
\end{pgfscope}%
\begin{pgfscope}%
\pgfsetbuttcap%
\pgfsetroundjoin%
\definecolor{currentfill}{rgb}{0.000000,0.000000,0.000000}%
\pgfsetfillcolor{currentfill}%
\pgfsetlinewidth{0.803000pt}%
\definecolor{currentstroke}{rgb}{0.000000,0.000000,0.000000}%
\pgfsetstrokecolor{currentstroke}%
\pgfsetdash{}{0pt}%
\pgfsys@defobject{currentmarker}{\pgfqpoint{-0.048611in}{0.000000in}}{\pgfqpoint{0.000000in}{0.000000in}}{%
\pgfpathmoveto{\pgfqpoint{0.000000in}{0.000000in}}%
\pgfpathlineto{\pgfqpoint{-0.048611in}{0.000000in}}%
\pgfusepath{stroke,fill}%
}%
\begin{pgfscope}%
\pgfsys@transformshift{0.787074in}{1.373761in}%
\pgfsys@useobject{currentmarker}{}%
\end{pgfscope}%
\end{pgfscope}%
\begin{pgfscope}%
\definecolor{textcolor}{rgb}{0.000000,0.000000,0.000000}%
\pgfsetstrokecolor{textcolor}%
\pgfsetfillcolor{textcolor}%
\pgftext[x=0.412073in, y=1.325566in, left, base]{\color{textcolor}\sffamily\fontsize{10.000000}{12.000000}\selectfont \(\displaystyle {5000}\)}%
\end{pgfscope}%
\begin{pgfscope}%
\pgfsetbuttcap%
\pgfsetroundjoin%
\definecolor{currentfill}{rgb}{0.000000,0.000000,0.000000}%
\pgfsetfillcolor{currentfill}%
\pgfsetlinewidth{0.803000pt}%
\definecolor{currentstroke}{rgb}{0.000000,0.000000,0.000000}%
\pgfsetstrokecolor{currentstroke}%
\pgfsetdash{}{0pt}%
\pgfsys@defobject{currentmarker}{\pgfqpoint{-0.048611in}{0.000000in}}{\pgfqpoint{0.000000in}{0.000000in}}{%
\pgfpathmoveto{\pgfqpoint{0.000000in}{0.000000in}}%
\pgfpathlineto{\pgfqpoint{-0.048611in}{0.000000in}}%
\pgfusepath{stroke,fill}%
}%
\begin{pgfscope}%
\pgfsys@transformshift{0.787074in}{2.057726in}%
\pgfsys@useobject{currentmarker}{}%
\end{pgfscope}%
\end{pgfscope}%
\begin{pgfscope}%
\definecolor{textcolor}{rgb}{0.000000,0.000000,0.000000}%
\pgfsetstrokecolor{textcolor}%
\pgfsetfillcolor{textcolor}%
\pgftext[x=0.342628in, y=2.009532in, left, base]{\color{textcolor}\sffamily\fontsize{10.000000}{12.000000}\selectfont \(\displaystyle {10000}\)}%
\end{pgfscope}%
\begin{pgfscope}%
\pgfsetbuttcap%
\pgfsetroundjoin%
\definecolor{currentfill}{rgb}{0.000000,0.000000,0.000000}%
\pgfsetfillcolor{currentfill}%
\pgfsetlinewidth{0.803000pt}%
\definecolor{currentstroke}{rgb}{0.000000,0.000000,0.000000}%
\pgfsetstrokecolor{currentstroke}%
\pgfsetdash{}{0pt}%
\pgfsys@defobject{currentmarker}{\pgfqpoint{-0.048611in}{0.000000in}}{\pgfqpoint{0.000000in}{0.000000in}}{%
\pgfpathmoveto{\pgfqpoint{0.000000in}{0.000000in}}%
\pgfpathlineto{\pgfqpoint{-0.048611in}{0.000000in}}%
\pgfusepath{stroke,fill}%
}%
\begin{pgfscope}%
\pgfsys@transformshift{0.787074in}{2.741692in}%
\pgfsys@useobject{currentmarker}{}%
\end{pgfscope}%
\end{pgfscope}%
\begin{pgfscope}%
\definecolor{textcolor}{rgb}{0.000000,0.000000,0.000000}%
\pgfsetstrokecolor{textcolor}%
\pgfsetfillcolor{textcolor}%
\pgftext[x=0.342628in, y=2.693498in, left, base]{\color{textcolor}\sffamily\fontsize{10.000000}{12.000000}\selectfont \(\displaystyle {15000}\)}%
\end{pgfscope}%
\begin{pgfscope}%
\pgfsetbuttcap%
\pgfsetroundjoin%
\definecolor{currentfill}{rgb}{0.000000,0.000000,0.000000}%
\pgfsetfillcolor{currentfill}%
\pgfsetlinewidth{0.803000pt}%
\definecolor{currentstroke}{rgb}{0.000000,0.000000,0.000000}%
\pgfsetstrokecolor{currentstroke}%
\pgfsetdash{}{0pt}%
\pgfsys@defobject{currentmarker}{\pgfqpoint{-0.048611in}{0.000000in}}{\pgfqpoint{0.000000in}{0.000000in}}{%
\pgfpathmoveto{\pgfqpoint{0.000000in}{0.000000in}}%
\pgfpathlineto{\pgfqpoint{-0.048611in}{0.000000in}}%
\pgfusepath{stroke,fill}%
}%
\begin{pgfscope}%
\pgfsys@transformshift{0.787074in}{3.425658in}%
\pgfsys@useobject{currentmarker}{}%
\end{pgfscope}%
\end{pgfscope}%
\begin{pgfscope}%
\definecolor{textcolor}{rgb}{0.000000,0.000000,0.000000}%
\pgfsetstrokecolor{textcolor}%
\pgfsetfillcolor{textcolor}%
\pgftext[x=0.342628in, y=3.377463in, left, base]{\color{textcolor}\sffamily\fontsize{10.000000}{12.000000}\selectfont \(\displaystyle {20000}\)}%
\end{pgfscope}%
\begin{pgfscope}%
\definecolor{textcolor}{rgb}{0.000000,0.000000,0.000000}%
\pgfsetstrokecolor{textcolor}%
\pgfsetfillcolor{textcolor}%
\pgftext[x=0.287073in,y=2.100064in,,bottom,rotate=90.000000]{\color{textcolor}\sffamily\fontsize{10.000000}{12.000000}\selectfont Maximum Memory Consumption (MB)}%
\end{pgfscope}%
\begin{pgfscope}%
\pgfsetrectcap%
\pgfsetmiterjoin%
\pgfsetlinewidth{0.803000pt}%
\definecolor{currentstroke}{rgb}{0.000000,0.000000,0.000000}%
\pgfsetstrokecolor{currentstroke}%
\pgfsetdash{}{0pt}%
\pgfpathmoveto{\pgfqpoint{0.787074in}{0.548769in}}%
\pgfpathlineto{\pgfqpoint{0.787074in}{3.651359in}}%
\pgfusepath{stroke}%
\end{pgfscope}%
\begin{pgfscope}%
\pgfsetrectcap%
\pgfsetmiterjoin%
\pgfsetlinewidth{0.803000pt}%
\definecolor{currentstroke}{rgb}{0.000000,0.000000,0.000000}%
\pgfsetstrokecolor{currentstroke}%
\pgfsetdash{}{0pt}%
\pgfpathmoveto{\pgfqpoint{5.850000in}{0.548769in}}%
\pgfpathlineto{\pgfqpoint{5.850000in}{3.651359in}}%
\pgfusepath{stroke}%
\end{pgfscope}%
\begin{pgfscope}%
\pgfsetrectcap%
\pgfsetmiterjoin%
\pgfsetlinewidth{0.803000pt}%
\definecolor{currentstroke}{rgb}{0.000000,0.000000,0.000000}%
\pgfsetstrokecolor{currentstroke}%
\pgfsetdash{}{0pt}%
\pgfpathmoveto{\pgfqpoint{0.787074in}{0.548769in}}%
\pgfpathlineto{\pgfqpoint{5.850000in}{0.548769in}}%
\pgfusepath{stroke}%
\end{pgfscope}%
\begin{pgfscope}%
\pgfsetrectcap%
\pgfsetmiterjoin%
\pgfsetlinewidth{0.803000pt}%
\definecolor{currentstroke}{rgb}{0.000000,0.000000,0.000000}%
\pgfsetstrokecolor{currentstroke}%
\pgfsetdash{}{0pt}%
\pgfpathmoveto{\pgfqpoint{0.787074in}{3.651359in}}%
\pgfpathlineto{\pgfqpoint{5.850000in}{3.651359in}}%
\pgfusepath{stroke}%
\end{pgfscope}%
\begin{pgfscope}%
\definecolor{textcolor}{rgb}{0.000000,0.000000,0.000000}%
\pgfsetstrokecolor{textcolor}%
\pgfsetfillcolor{textcolor}%
\pgftext[x=3.318537in,y=3.734692in,,base]{\color{textcolor}\sffamily\fontsize{12.000000}{14.400000}\selectfont Forward}%
\end{pgfscope}%
\begin{pgfscope}%
\pgfsetbuttcap%
\pgfsetmiterjoin%
\definecolor{currentfill}{rgb}{1.000000,1.000000,1.000000}%
\pgfsetfillcolor{currentfill}%
\pgfsetfillopacity{0.800000}%
\pgfsetlinewidth{1.003750pt}%
\definecolor{currentstroke}{rgb}{0.800000,0.800000,0.800000}%
\pgfsetstrokecolor{currentstroke}%
\pgfsetstrokeopacity{0.800000}%
\pgfsetdash{}{0pt}%
\pgfpathmoveto{\pgfqpoint{4.300417in}{2.957886in}}%
\pgfpathlineto{\pgfqpoint{5.752778in}{2.957886in}}%
\pgfpathquadraticcurveto{\pgfqpoint{5.780556in}{2.957886in}}{\pgfqpoint{5.780556in}{2.985664in}}%
\pgfpathlineto{\pgfqpoint{5.780556in}{3.554136in}}%
\pgfpathquadraticcurveto{\pgfqpoint{5.780556in}{3.581914in}}{\pgfqpoint{5.752778in}{3.581914in}}%
\pgfpathlineto{\pgfqpoint{4.300417in}{3.581914in}}%
\pgfpathquadraticcurveto{\pgfqpoint{4.272639in}{3.581914in}}{\pgfqpoint{4.272639in}{3.554136in}}%
\pgfpathlineto{\pgfqpoint{4.272639in}{2.985664in}}%
\pgfpathquadraticcurveto{\pgfqpoint{4.272639in}{2.957886in}}{\pgfqpoint{4.300417in}{2.957886in}}%
\pgfpathclose%
\pgfusepath{stroke,fill}%
\end{pgfscope}%
\begin{pgfscope}%
\pgfsetbuttcap%
\pgfsetroundjoin%
\definecolor{currentfill}{rgb}{0.121569,0.466667,0.705882}%
\pgfsetfillcolor{currentfill}%
\pgfsetlinewidth{1.003750pt}%
\definecolor{currentstroke}{rgb}{0.121569,0.466667,0.705882}%
\pgfsetstrokecolor{currentstroke}%
\pgfsetdash{}{0pt}%
\pgfsys@defobject{currentmarker}{\pgfqpoint{-0.034722in}{-0.034722in}}{\pgfqpoint{0.034722in}{0.034722in}}{%
\pgfpathmoveto{\pgfqpoint{0.000000in}{-0.034722in}}%
\pgfpathcurveto{\pgfqpoint{0.009208in}{-0.034722in}}{\pgfqpoint{0.018041in}{-0.031064in}}{\pgfqpoint{0.024552in}{-0.024552in}}%
\pgfpathcurveto{\pgfqpoint{0.031064in}{-0.018041in}}{\pgfqpoint{0.034722in}{-0.009208in}}{\pgfqpoint{0.034722in}{0.000000in}}%
\pgfpathcurveto{\pgfqpoint{0.034722in}{0.009208in}}{\pgfqpoint{0.031064in}{0.018041in}}{\pgfqpoint{0.024552in}{0.024552in}}%
\pgfpathcurveto{\pgfqpoint{0.018041in}{0.031064in}}{\pgfqpoint{0.009208in}{0.034722in}}{\pgfqpoint{0.000000in}{0.034722in}}%
\pgfpathcurveto{\pgfqpoint{-0.009208in}{0.034722in}}{\pgfqpoint{-0.018041in}{0.031064in}}{\pgfqpoint{-0.024552in}{0.024552in}}%
\pgfpathcurveto{\pgfqpoint{-0.031064in}{0.018041in}}{\pgfqpoint{-0.034722in}{0.009208in}}{\pgfqpoint{-0.034722in}{0.000000in}}%
\pgfpathcurveto{\pgfqpoint{-0.034722in}{-0.009208in}}{\pgfqpoint{-0.031064in}{-0.018041in}}{\pgfqpoint{-0.024552in}{-0.024552in}}%
\pgfpathcurveto{\pgfqpoint{-0.018041in}{-0.031064in}}{\pgfqpoint{-0.009208in}{-0.034722in}}{\pgfqpoint{0.000000in}{-0.034722in}}%
\pgfpathclose%
\pgfusepath{stroke,fill}%
}%
\begin{pgfscope}%
\pgfsys@transformshift{4.467083in}{3.477748in}%
\pgfsys@useobject{currentmarker}{}%
\end{pgfscope}%
\end{pgfscope}%
\begin{pgfscope}%
\definecolor{textcolor}{rgb}{0.000000,0.000000,0.000000}%
\pgfsetstrokecolor{textcolor}%
\pgfsetfillcolor{textcolor}%
\pgftext[x=4.717083in,y=3.429136in,left,base]{\color{textcolor}\sffamily\fontsize{10.000000}{12.000000}\selectfont No Timeout}%
\end{pgfscope}%
\begin{pgfscope}%
\pgfsetbuttcap%
\pgfsetroundjoin%
\definecolor{currentfill}{rgb}{1.000000,0.498039,0.054902}%
\pgfsetfillcolor{currentfill}%
\pgfsetlinewidth{1.003750pt}%
\definecolor{currentstroke}{rgb}{1.000000,0.498039,0.054902}%
\pgfsetstrokecolor{currentstroke}%
\pgfsetdash{}{0pt}%
\pgfsys@defobject{currentmarker}{\pgfqpoint{-0.034722in}{-0.034722in}}{\pgfqpoint{0.034722in}{0.034722in}}{%
\pgfpathmoveto{\pgfqpoint{0.000000in}{-0.034722in}}%
\pgfpathcurveto{\pgfqpoint{0.009208in}{-0.034722in}}{\pgfqpoint{0.018041in}{-0.031064in}}{\pgfqpoint{0.024552in}{-0.024552in}}%
\pgfpathcurveto{\pgfqpoint{0.031064in}{-0.018041in}}{\pgfqpoint{0.034722in}{-0.009208in}}{\pgfqpoint{0.034722in}{0.000000in}}%
\pgfpathcurveto{\pgfqpoint{0.034722in}{0.009208in}}{\pgfqpoint{0.031064in}{0.018041in}}{\pgfqpoint{0.024552in}{0.024552in}}%
\pgfpathcurveto{\pgfqpoint{0.018041in}{0.031064in}}{\pgfqpoint{0.009208in}{0.034722in}}{\pgfqpoint{0.000000in}{0.034722in}}%
\pgfpathcurveto{\pgfqpoint{-0.009208in}{0.034722in}}{\pgfqpoint{-0.018041in}{0.031064in}}{\pgfqpoint{-0.024552in}{0.024552in}}%
\pgfpathcurveto{\pgfqpoint{-0.031064in}{0.018041in}}{\pgfqpoint{-0.034722in}{0.009208in}}{\pgfqpoint{-0.034722in}{0.000000in}}%
\pgfpathcurveto{\pgfqpoint{-0.034722in}{-0.009208in}}{\pgfqpoint{-0.031064in}{-0.018041in}}{\pgfqpoint{-0.024552in}{-0.024552in}}%
\pgfpathcurveto{\pgfqpoint{-0.018041in}{-0.031064in}}{\pgfqpoint{-0.009208in}{-0.034722in}}{\pgfqpoint{0.000000in}{-0.034722in}}%
\pgfpathclose%
\pgfusepath{stroke,fill}%
}%
\begin{pgfscope}%
\pgfsys@transformshift{4.467083in}{3.284136in}%
\pgfsys@useobject{currentmarker}{}%
\end{pgfscope}%
\end{pgfscope}%
\begin{pgfscope}%
\definecolor{textcolor}{rgb}{0.000000,0.000000,0.000000}%
\pgfsetstrokecolor{textcolor}%
\pgfsetfillcolor{textcolor}%
\pgftext[x=4.717083in,y=3.235525in,left,base]{\color{textcolor}\sffamily\fontsize{10.000000}{12.000000}\selectfont Time Timeout}%
\end{pgfscope}%
\begin{pgfscope}%
\pgfsetbuttcap%
\pgfsetroundjoin%
\definecolor{currentfill}{rgb}{0.839216,0.152941,0.156863}%
\pgfsetfillcolor{currentfill}%
\pgfsetlinewidth{1.003750pt}%
\definecolor{currentstroke}{rgb}{0.839216,0.152941,0.156863}%
\pgfsetstrokecolor{currentstroke}%
\pgfsetdash{}{0pt}%
\pgfsys@defobject{currentmarker}{\pgfqpoint{-0.034722in}{-0.034722in}}{\pgfqpoint{0.034722in}{0.034722in}}{%
\pgfpathmoveto{\pgfqpoint{0.000000in}{-0.034722in}}%
\pgfpathcurveto{\pgfqpoint{0.009208in}{-0.034722in}}{\pgfqpoint{0.018041in}{-0.031064in}}{\pgfqpoint{0.024552in}{-0.024552in}}%
\pgfpathcurveto{\pgfqpoint{0.031064in}{-0.018041in}}{\pgfqpoint{0.034722in}{-0.009208in}}{\pgfqpoint{0.034722in}{0.000000in}}%
\pgfpathcurveto{\pgfqpoint{0.034722in}{0.009208in}}{\pgfqpoint{0.031064in}{0.018041in}}{\pgfqpoint{0.024552in}{0.024552in}}%
\pgfpathcurveto{\pgfqpoint{0.018041in}{0.031064in}}{\pgfqpoint{0.009208in}{0.034722in}}{\pgfqpoint{0.000000in}{0.034722in}}%
\pgfpathcurveto{\pgfqpoint{-0.009208in}{0.034722in}}{\pgfqpoint{-0.018041in}{0.031064in}}{\pgfqpoint{-0.024552in}{0.024552in}}%
\pgfpathcurveto{\pgfqpoint{-0.031064in}{0.018041in}}{\pgfqpoint{-0.034722in}{0.009208in}}{\pgfqpoint{-0.034722in}{0.000000in}}%
\pgfpathcurveto{\pgfqpoint{-0.034722in}{-0.009208in}}{\pgfqpoint{-0.031064in}{-0.018041in}}{\pgfqpoint{-0.024552in}{-0.024552in}}%
\pgfpathcurveto{\pgfqpoint{-0.018041in}{-0.031064in}}{\pgfqpoint{-0.009208in}{-0.034722in}}{\pgfqpoint{0.000000in}{-0.034722in}}%
\pgfpathclose%
\pgfusepath{stroke,fill}%
}%
\begin{pgfscope}%
\pgfsys@transformshift{4.467083in}{3.090525in}%
\pgfsys@useobject{currentmarker}{}%
\end{pgfscope}%
\end{pgfscope}%
\begin{pgfscope}%
\definecolor{textcolor}{rgb}{0.000000,0.000000,0.000000}%
\pgfsetstrokecolor{textcolor}%
\pgfsetfillcolor{textcolor}%
\pgftext[x=4.717083in,y=3.041914in,left,base]{\color{textcolor}\sffamily\fontsize{10.000000}{12.000000}\selectfont Memory Timeout}%
\end{pgfscope}%
\end{pgfpicture}%
\makeatother%
\endgroup%

                }
            \end{subfigure}
            \qquad
            \begin{subfigure}[]{0.45\textwidth}
                \centering
                \resizebox{\columnwidth}{!}{
                    %% Creator: Matplotlib, PGF backend
%%
%% To include the figure in your LaTeX document, write
%%   \input{<filename>.pgf}
%%
%% Make sure the required packages are loaded in your preamble
%%   \usepackage{pgf}
%%
%% and, on pdftex
%%   \usepackage[utf8]{inputenc}\DeclareUnicodeCharacter{2212}{-}
%%
%% or, on luatex and xetex
%%   \usepackage{unicode-math}
%%
%% Figures using additional raster images can only be included by \input if
%% they are in the same directory as the main LaTeX file. For loading figures
%% from other directories you can use the `import` package
%%   \usepackage{import}
%%
%% and then include the figures with
%%   \import{<path to file>}{<filename>.pgf}
%%
%% Matplotlib used the following preamble
%%   \usepackage{amsmath}
%%   \usepackage{fontspec}
%%
\begingroup%
\makeatletter%
\begin{pgfpicture}%
\pgfpathrectangle{\pgfpointorigin}{\pgfqpoint{6.000000in}{4.000000in}}%
\pgfusepath{use as bounding box, clip}%
\begin{pgfscope}%
\pgfsetbuttcap%
\pgfsetmiterjoin%
\definecolor{currentfill}{rgb}{1.000000,1.000000,1.000000}%
\pgfsetfillcolor{currentfill}%
\pgfsetlinewidth{0.000000pt}%
\definecolor{currentstroke}{rgb}{1.000000,1.000000,1.000000}%
\pgfsetstrokecolor{currentstroke}%
\pgfsetdash{}{0pt}%
\pgfpathmoveto{\pgfqpoint{0.000000in}{0.000000in}}%
\pgfpathlineto{\pgfqpoint{6.000000in}{0.000000in}}%
\pgfpathlineto{\pgfqpoint{6.000000in}{4.000000in}}%
\pgfpathlineto{\pgfqpoint{0.000000in}{4.000000in}}%
\pgfpathclose%
\pgfusepath{fill}%
\end{pgfscope}%
\begin{pgfscope}%
\pgfsetbuttcap%
\pgfsetmiterjoin%
\definecolor{currentfill}{rgb}{1.000000,1.000000,1.000000}%
\pgfsetfillcolor{currentfill}%
\pgfsetlinewidth{0.000000pt}%
\definecolor{currentstroke}{rgb}{0.000000,0.000000,0.000000}%
\pgfsetstrokecolor{currentstroke}%
\pgfsetstrokeopacity{0.000000}%
\pgfsetdash{}{0pt}%
\pgfpathmoveto{\pgfqpoint{0.787074in}{0.548769in}}%
\pgfpathlineto{\pgfqpoint{5.850000in}{0.548769in}}%
\pgfpathlineto{\pgfqpoint{5.850000in}{3.651359in}}%
\pgfpathlineto{\pgfqpoint{0.787074in}{3.651359in}}%
\pgfpathclose%
\pgfusepath{fill}%
\end{pgfscope}%
\begin{pgfscope}%
\pgfpathrectangle{\pgfqpoint{0.787074in}{0.548769in}}{\pgfqpoint{5.062926in}{3.102590in}}%
\pgfusepath{clip}%
\pgfsetbuttcap%
\pgfsetroundjoin%
\definecolor{currentfill}{rgb}{0.121569,0.466667,0.705882}%
\pgfsetfillcolor{currentfill}%
\pgfsetlinewidth{1.003750pt}%
\definecolor{currentstroke}{rgb}{0.121569,0.466667,0.705882}%
\pgfsetstrokecolor{currentstroke}%
\pgfsetdash{}{0pt}%
\pgfpathmoveto{\pgfqpoint{1.411721in}{0.676730in}}%
\pgfpathcurveto{\pgfqpoint{1.422771in}{0.676730in}}{\pgfqpoint{1.433370in}{0.681121in}}{\pgfqpoint{1.441183in}{0.688934in}}%
\pgfpathcurveto{\pgfqpoint{1.448997in}{0.696748in}}{\pgfqpoint{1.453387in}{0.707347in}}{\pgfqpoint{1.453387in}{0.718397in}}%
\pgfpathcurveto{\pgfqpoint{1.453387in}{0.729447in}}{\pgfqpoint{1.448997in}{0.740046in}}{\pgfqpoint{1.441183in}{0.747860in}}%
\pgfpathcurveto{\pgfqpoint{1.433370in}{0.755673in}}{\pgfqpoint{1.422771in}{0.760064in}}{\pgfqpoint{1.411721in}{0.760064in}}%
\pgfpathcurveto{\pgfqpoint{1.400670in}{0.760064in}}{\pgfqpoint{1.390071in}{0.755673in}}{\pgfqpoint{1.382258in}{0.747860in}}%
\pgfpathcurveto{\pgfqpoint{1.374444in}{0.740046in}}{\pgfqpoint{1.370054in}{0.729447in}}{\pgfqpoint{1.370054in}{0.718397in}}%
\pgfpathcurveto{\pgfqpoint{1.370054in}{0.707347in}}{\pgfqpoint{1.374444in}{0.696748in}}{\pgfqpoint{1.382258in}{0.688934in}}%
\pgfpathcurveto{\pgfqpoint{1.390071in}{0.681121in}}{\pgfqpoint{1.400670in}{0.676730in}}{\pgfqpoint{1.411721in}{0.676730in}}%
\pgfpathclose%
\pgfusepath{stroke,fill}%
\end{pgfscope}%
\begin{pgfscope}%
\pgfpathrectangle{\pgfqpoint{0.787074in}{0.548769in}}{\pgfqpoint{5.062926in}{3.102590in}}%
\pgfusepath{clip}%
\pgfsetbuttcap%
\pgfsetroundjoin%
\definecolor{currentfill}{rgb}{1.000000,0.498039,0.054902}%
\pgfsetfillcolor{currentfill}%
\pgfsetlinewidth{1.003750pt}%
\definecolor{currentstroke}{rgb}{1.000000,0.498039,0.054902}%
\pgfsetstrokecolor{currentstroke}%
\pgfsetdash{}{0pt}%
\pgfpathmoveto{\pgfqpoint{2.858271in}{1.501451in}}%
\pgfpathcurveto{\pgfqpoint{2.869321in}{1.501451in}}{\pgfqpoint{2.879920in}{1.505841in}}{\pgfqpoint{2.887734in}{1.513655in}}%
\pgfpathcurveto{\pgfqpoint{2.895547in}{1.521468in}}{\pgfqpoint{2.899938in}{1.532067in}}{\pgfqpoint{2.899938in}{1.543118in}}%
\pgfpathcurveto{\pgfqpoint{2.899938in}{1.554168in}}{\pgfqpoint{2.895547in}{1.564767in}}{\pgfqpoint{2.887734in}{1.572580in}}%
\pgfpathcurveto{\pgfqpoint{2.879920in}{1.580394in}}{\pgfqpoint{2.869321in}{1.584784in}}{\pgfqpoint{2.858271in}{1.584784in}}%
\pgfpathcurveto{\pgfqpoint{2.847221in}{1.584784in}}{\pgfqpoint{2.836622in}{1.580394in}}{\pgfqpoint{2.828808in}{1.572580in}}%
\pgfpathcurveto{\pgfqpoint{2.820995in}{1.564767in}}{\pgfqpoint{2.816604in}{1.554168in}}{\pgfqpoint{2.816604in}{1.543118in}}%
\pgfpathcurveto{\pgfqpoint{2.816604in}{1.532067in}}{\pgfqpoint{2.820995in}{1.521468in}}{\pgfqpoint{2.828808in}{1.513655in}}%
\pgfpathcurveto{\pgfqpoint{2.836622in}{1.505841in}}{\pgfqpoint{2.847221in}{1.501451in}}{\pgfqpoint{2.858271in}{1.501451in}}%
\pgfpathclose%
\pgfusepath{stroke,fill}%
\end{pgfscope}%
\begin{pgfscope}%
\pgfpathrectangle{\pgfqpoint{0.787074in}{0.548769in}}{\pgfqpoint{5.062926in}{3.102590in}}%
\pgfusepath{clip}%
\pgfsetbuttcap%
\pgfsetroundjoin%
\definecolor{currentfill}{rgb}{1.000000,0.498039,0.054902}%
\pgfsetfillcolor{currentfill}%
\pgfsetlinewidth{1.003750pt}%
\definecolor{currentstroke}{rgb}{1.000000,0.498039,0.054902}%
\pgfsetstrokecolor{currentstroke}%
\pgfsetdash{}{0pt}%
\pgfpathmoveto{\pgfqpoint{1.608977in}{2.589003in}}%
\pgfpathcurveto{\pgfqpoint{1.620028in}{2.589003in}}{\pgfqpoint{1.630627in}{2.593393in}}{\pgfqpoint{1.638440in}{2.601207in}}%
\pgfpathcurveto{\pgfqpoint{1.646254in}{2.609020in}}{\pgfqpoint{1.650644in}{2.619619in}}{\pgfqpoint{1.650644in}{2.630670in}}%
\pgfpathcurveto{\pgfqpoint{1.650644in}{2.641720in}}{\pgfqpoint{1.646254in}{2.652319in}}{\pgfqpoint{1.638440in}{2.660132in}}%
\pgfpathcurveto{\pgfqpoint{1.630627in}{2.667946in}}{\pgfqpoint{1.620028in}{2.672336in}}{\pgfqpoint{1.608977in}{2.672336in}}%
\pgfpathcurveto{\pgfqpoint{1.597927in}{2.672336in}}{\pgfqpoint{1.587328in}{2.667946in}}{\pgfqpoint{1.579515in}{2.660132in}}%
\pgfpathcurveto{\pgfqpoint{1.571701in}{2.652319in}}{\pgfqpoint{1.567311in}{2.641720in}}{\pgfqpoint{1.567311in}{2.630670in}}%
\pgfpathcurveto{\pgfqpoint{1.567311in}{2.619619in}}{\pgfqpoint{1.571701in}{2.609020in}}{\pgfqpoint{1.579515in}{2.601207in}}%
\pgfpathcurveto{\pgfqpoint{1.587328in}{2.593393in}}{\pgfqpoint{1.597927in}{2.589003in}}{\pgfqpoint{1.608977in}{2.589003in}}%
\pgfpathclose%
\pgfusepath{stroke,fill}%
\end{pgfscope}%
\begin{pgfscope}%
\pgfpathrectangle{\pgfqpoint{0.787074in}{0.548769in}}{\pgfqpoint{5.062926in}{3.102590in}}%
\pgfusepath{clip}%
\pgfsetbuttcap%
\pgfsetroundjoin%
\definecolor{currentfill}{rgb}{1.000000,0.498039,0.054902}%
\pgfsetfillcolor{currentfill}%
\pgfsetlinewidth{1.003750pt}%
\definecolor{currentstroke}{rgb}{1.000000,0.498039,0.054902}%
\pgfsetstrokecolor{currentstroke}%
\pgfsetdash{}{0pt}%
\pgfpathmoveto{\pgfqpoint{2.726766in}{1.800918in}}%
\pgfpathcurveto{\pgfqpoint{2.737816in}{1.800918in}}{\pgfqpoint{2.748415in}{1.805309in}}{\pgfqpoint{2.756229in}{1.813122in}}%
\pgfpathcurveto{\pgfqpoint{2.764043in}{1.820936in}}{\pgfqpoint{2.768433in}{1.831535in}}{\pgfqpoint{2.768433in}{1.842585in}}%
\pgfpathcurveto{\pgfqpoint{2.768433in}{1.853635in}}{\pgfqpoint{2.764043in}{1.864234in}}{\pgfqpoint{2.756229in}{1.872048in}}%
\pgfpathcurveto{\pgfqpoint{2.748415in}{1.879862in}}{\pgfqpoint{2.737816in}{1.884252in}}{\pgfqpoint{2.726766in}{1.884252in}}%
\pgfpathcurveto{\pgfqpoint{2.715716in}{1.884252in}}{\pgfqpoint{2.705117in}{1.879862in}}{\pgfqpoint{2.697304in}{1.872048in}}%
\pgfpathcurveto{\pgfqpoint{2.689490in}{1.864234in}}{\pgfqpoint{2.685100in}{1.853635in}}{\pgfqpoint{2.685100in}{1.842585in}}%
\pgfpathcurveto{\pgfqpoint{2.685100in}{1.831535in}}{\pgfqpoint{2.689490in}{1.820936in}}{\pgfqpoint{2.697304in}{1.813122in}}%
\pgfpathcurveto{\pgfqpoint{2.705117in}{1.805309in}}{\pgfqpoint{2.715716in}{1.800918in}}{\pgfqpoint{2.726766in}{1.800918in}}%
\pgfpathclose%
\pgfusepath{stroke,fill}%
\end{pgfscope}%
\begin{pgfscope}%
\pgfpathrectangle{\pgfqpoint{0.787074in}{0.548769in}}{\pgfqpoint{5.062926in}{3.102590in}}%
\pgfusepath{clip}%
\pgfsetbuttcap%
\pgfsetroundjoin%
\definecolor{currentfill}{rgb}{0.121569,0.466667,0.705882}%
\pgfsetfillcolor{currentfill}%
\pgfsetlinewidth{1.003750pt}%
\definecolor{currentstroke}{rgb}{0.121569,0.466667,0.705882}%
\pgfsetstrokecolor{currentstroke}%
\pgfsetdash{}{0pt}%
\pgfpathmoveto{\pgfqpoint{2.266500in}{0.650081in}}%
\pgfpathcurveto{\pgfqpoint{2.277550in}{0.650081in}}{\pgfqpoint{2.288149in}{0.654472in}}{\pgfqpoint{2.295963in}{0.662285in}}%
\pgfpathcurveto{\pgfqpoint{2.303777in}{0.670099in}}{\pgfqpoint{2.308167in}{0.680698in}}{\pgfqpoint{2.308167in}{0.691748in}}%
\pgfpathcurveto{\pgfqpoint{2.308167in}{0.702798in}}{\pgfqpoint{2.303777in}{0.713397in}}{\pgfqpoint{2.295963in}{0.721211in}}%
\pgfpathcurveto{\pgfqpoint{2.288149in}{0.729024in}}{\pgfqpoint{2.277550in}{0.733415in}}{\pgfqpoint{2.266500in}{0.733415in}}%
\pgfpathcurveto{\pgfqpoint{2.255450in}{0.733415in}}{\pgfqpoint{2.244851in}{0.729024in}}{\pgfqpoint{2.237038in}{0.721211in}}%
\pgfpathcurveto{\pgfqpoint{2.229224in}{0.713397in}}{\pgfqpoint{2.224834in}{0.702798in}}{\pgfqpoint{2.224834in}{0.691748in}}%
\pgfpathcurveto{\pgfqpoint{2.224834in}{0.680698in}}{\pgfqpoint{2.229224in}{0.670099in}}{\pgfqpoint{2.237038in}{0.662285in}}%
\pgfpathcurveto{\pgfqpoint{2.244851in}{0.654472in}}{\pgfqpoint{2.255450in}{0.650081in}}{\pgfqpoint{2.266500in}{0.650081in}}%
\pgfpathclose%
\pgfusepath{stroke,fill}%
\end{pgfscope}%
\begin{pgfscope}%
\pgfpathrectangle{\pgfqpoint{0.787074in}{0.548769in}}{\pgfqpoint{5.062926in}{3.102590in}}%
\pgfusepath{clip}%
\pgfsetbuttcap%
\pgfsetroundjoin%
\definecolor{currentfill}{rgb}{1.000000,0.498039,0.054902}%
\pgfsetfillcolor{currentfill}%
\pgfsetlinewidth{1.003750pt}%
\definecolor{currentstroke}{rgb}{1.000000,0.498039,0.054902}%
\pgfsetstrokecolor{currentstroke}%
\pgfsetdash{}{0pt}%
\pgfpathmoveto{\pgfqpoint{2.529509in}{2.499787in}}%
\pgfpathcurveto{\pgfqpoint{2.540560in}{2.499787in}}{\pgfqpoint{2.551159in}{2.504177in}}{\pgfqpoint{2.558972in}{2.511991in}}%
\pgfpathcurveto{\pgfqpoint{2.566786in}{2.519804in}}{\pgfqpoint{2.571176in}{2.530403in}}{\pgfqpoint{2.571176in}{2.541454in}}%
\pgfpathcurveto{\pgfqpoint{2.571176in}{2.552504in}}{\pgfqpoint{2.566786in}{2.563103in}}{\pgfqpoint{2.558972in}{2.570916in}}%
\pgfpathcurveto{\pgfqpoint{2.551159in}{2.578730in}}{\pgfqpoint{2.540560in}{2.583120in}}{\pgfqpoint{2.529509in}{2.583120in}}%
\pgfpathcurveto{\pgfqpoint{2.518459in}{2.583120in}}{\pgfqpoint{2.507860in}{2.578730in}}{\pgfqpoint{2.500047in}{2.570916in}}%
\pgfpathcurveto{\pgfqpoint{2.492233in}{2.563103in}}{\pgfqpoint{2.487843in}{2.552504in}}{\pgfqpoint{2.487843in}{2.541454in}}%
\pgfpathcurveto{\pgfqpoint{2.487843in}{2.530403in}}{\pgfqpoint{2.492233in}{2.519804in}}{\pgfqpoint{2.500047in}{2.511991in}}%
\pgfpathcurveto{\pgfqpoint{2.507860in}{2.504177in}}{\pgfqpoint{2.518459in}{2.499787in}}{\pgfqpoint{2.529509in}{2.499787in}}%
\pgfpathclose%
\pgfusepath{stroke,fill}%
\end{pgfscope}%
\begin{pgfscope}%
\pgfpathrectangle{\pgfqpoint{0.787074in}{0.548769in}}{\pgfqpoint{5.062926in}{3.102590in}}%
\pgfusepath{clip}%
\pgfsetbuttcap%
\pgfsetroundjoin%
\definecolor{currentfill}{rgb}{1.000000,0.498039,0.054902}%
\pgfsetfillcolor{currentfill}%
\pgfsetlinewidth{1.003750pt}%
\definecolor{currentstroke}{rgb}{1.000000,0.498039,0.054902}%
\pgfsetstrokecolor{currentstroke}%
\pgfsetdash{}{0pt}%
\pgfpathmoveto{\pgfqpoint{2.069243in}{2.338512in}}%
\pgfpathcurveto{\pgfqpoint{2.080294in}{2.338512in}}{\pgfqpoint{2.090893in}{2.342903in}}{\pgfqpoint{2.098706in}{2.350716in}}%
\pgfpathcurveto{\pgfqpoint{2.106520in}{2.358530in}}{\pgfqpoint{2.110910in}{2.369129in}}{\pgfqpoint{2.110910in}{2.380179in}}%
\pgfpathcurveto{\pgfqpoint{2.110910in}{2.391229in}}{\pgfqpoint{2.106520in}{2.401828in}}{\pgfqpoint{2.098706in}{2.409642in}}%
\pgfpathcurveto{\pgfqpoint{2.090893in}{2.417455in}}{\pgfqpoint{2.080294in}{2.421846in}}{\pgfqpoint{2.069243in}{2.421846in}}%
\pgfpathcurveto{\pgfqpoint{2.058193in}{2.421846in}}{\pgfqpoint{2.047594in}{2.417455in}}{\pgfqpoint{2.039781in}{2.409642in}}%
\pgfpathcurveto{\pgfqpoint{2.031967in}{2.401828in}}{\pgfqpoint{2.027577in}{2.391229in}}{\pgfqpoint{2.027577in}{2.380179in}}%
\pgfpathcurveto{\pgfqpoint{2.027577in}{2.369129in}}{\pgfqpoint{2.031967in}{2.358530in}}{\pgfqpoint{2.039781in}{2.350716in}}%
\pgfpathcurveto{\pgfqpoint{2.047594in}{2.342903in}}{\pgfqpoint{2.058193in}{2.338512in}}{\pgfqpoint{2.069243in}{2.338512in}}%
\pgfpathclose%
\pgfusepath{stroke,fill}%
\end{pgfscope}%
\begin{pgfscope}%
\pgfpathrectangle{\pgfqpoint{0.787074in}{0.548769in}}{\pgfqpoint{5.062926in}{3.102590in}}%
\pgfusepath{clip}%
\pgfsetbuttcap%
\pgfsetroundjoin%
\definecolor{currentfill}{rgb}{0.121569,0.466667,0.705882}%
\pgfsetfillcolor{currentfill}%
\pgfsetlinewidth{1.003750pt}%
\definecolor{currentstroke}{rgb}{0.121569,0.466667,0.705882}%
\pgfsetstrokecolor{currentstroke}%
\pgfsetdash{}{0pt}%
\pgfpathmoveto{\pgfqpoint{1.411721in}{0.648162in}}%
\pgfpathcurveto{\pgfqpoint{1.422771in}{0.648162in}}{\pgfqpoint{1.433370in}{0.652552in}}{\pgfqpoint{1.441183in}{0.660365in}}%
\pgfpathcurveto{\pgfqpoint{1.448997in}{0.668179in}}{\pgfqpoint{1.453387in}{0.678778in}}{\pgfqpoint{1.453387in}{0.689828in}}%
\pgfpathcurveto{\pgfqpoint{1.453387in}{0.700878in}}{\pgfqpoint{1.448997in}{0.711477in}}{\pgfqpoint{1.441183in}{0.719291in}}%
\pgfpathcurveto{\pgfqpoint{1.433370in}{0.727105in}}{\pgfqpoint{1.422771in}{0.731495in}}{\pgfqpoint{1.411721in}{0.731495in}}%
\pgfpathcurveto{\pgfqpoint{1.400670in}{0.731495in}}{\pgfqpoint{1.390071in}{0.727105in}}{\pgfqpoint{1.382258in}{0.719291in}}%
\pgfpathcurveto{\pgfqpoint{1.374444in}{0.711477in}}{\pgfqpoint{1.370054in}{0.700878in}}{\pgfqpoint{1.370054in}{0.689828in}}%
\pgfpathcurveto{\pgfqpoint{1.370054in}{0.678778in}}{\pgfqpoint{1.374444in}{0.668179in}}{\pgfqpoint{1.382258in}{0.660365in}}%
\pgfpathcurveto{\pgfqpoint{1.390071in}{0.652552in}}{\pgfqpoint{1.400670in}{0.648162in}}{\pgfqpoint{1.411721in}{0.648162in}}%
\pgfpathclose%
\pgfusepath{stroke,fill}%
\end{pgfscope}%
\begin{pgfscope}%
\pgfpathrectangle{\pgfqpoint{0.787074in}{0.548769in}}{\pgfqpoint{5.062926in}{3.102590in}}%
\pgfusepath{clip}%
\pgfsetbuttcap%
\pgfsetroundjoin%
\definecolor{currentfill}{rgb}{0.121569,0.466667,0.705882}%
\pgfsetfillcolor{currentfill}%
\pgfsetlinewidth{1.003750pt}%
\definecolor{currentstroke}{rgb}{0.121569,0.466667,0.705882}%
\pgfsetstrokecolor{currentstroke}%
\pgfsetdash{}{0pt}%
\pgfpathmoveto{\pgfqpoint{1.543225in}{0.840903in}}%
\pgfpathcurveto{\pgfqpoint{1.554275in}{0.840903in}}{\pgfqpoint{1.564874in}{0.845293in}}{\pgfqpoint{1.572688in}{0.853107in}}%
\pgfpathcurveto{\pgfqpoint{1.580502in}{0.860920in}}{\pgfqpoint{1.584892in}{0.871520in}}{\pgfqpoint{1.584892in}{0.882570in}}%
\pgfpathcurveto{\pgfqpoint{1.584892in}{0.893620in}}{\pgfqpoint{1.580502in}{0.904219in}}{\pgfqpoint{1.572688in}{0.912032in}}%
\pgfpathcurveto{\pgfqpoint{1.564874in}{0.919846in}}{\pgfqpoint{1.554275in}{0.924236in}}{\pgfqpoint{1.543225in}{0.924236in}}%
\pgfpathcurveto{\pgfqpoint{1.532175in}{0.924236in}}{\pgfqpoint{1.521576in}{0.919846in}}{\pgfqpoint{1.513762in}{0.912032in}}%
\pgfpathcurveto{\pgfqpoint{1.505949in}{0.904219in}}{\pgfqpoint{1.501558in}{0.893620in}}{\pgfqpoint{1.501558in}{0.882570in}}%
\pgfpathcurveto{\pgfqpoint{1.501558in}{0.871520in}}{\pgfqpoint{1.505949in}{0.860920in}}{\pgfqpoint{1.513762in}{0.853107in}}%
\pgfpathcurveto{\pgfqpoint{1.521576in}{0.845293in}}{\pgfqpoint{1.532175in}{0.840903in}}{\pgfqpoint{1.543225in}{0.840903in}}%
\pgfpathclose%
\pgfusepath{stroke,fill}%
\end{pgfscope}%
\begin{pgfscope}%
\pgfpathrectangle{\pgfqpoint{0.787074in}{0.548769in}}{\pgfqpoint{5.062926in}{3.102590in}}%
\pgfusepath{clip}%
\pgfsetbuttcap%
\pgfsetroundjoin%
\definecolor{currentfill}{rgb}{0.121569,0.466667,0.705882}%
\pgfsetfillcolor{currentfill}%
\pgfsetlinewidth{1.003750pt}%
\definecolor{currentstroke}{rgb}{0.121569,0.466667,0.705882}%
\pgfsetstrokecolor{currentstroke}%
\pgfsetdash{}{0pt}%
\pgfpathmoveto{\pgfqpoint{1.280216in}{0.787071in}}%
\pgfpathcurveto{\pgfqpoint{1.291266in}{0.787071in}}{\pgfqpoint{1.301865in}{0.791461in}}{\pgfqpoint{1.309679in}{0.799275in}}%
\pgfpathcurveto{\pgfqpoint{1.317492in}{0.807089in}}{\pgfqpoint{1.321883in}{0.817688in}}{\pgfqpoint{1.321883in}{0.828738in}}%
\pgfpathcurveto{\pgfqpoint{1.321883in}{0.839788in}}{\pgfqpoint{1.317492in}{0.850387in}}{\pgfqpoint{1.309679in}{0.858201in}}%
\pgfpathcurveto{\pgfqpoint{1.301865in}{0.866014in}}{\pgfqpoint{1.291266in}{0.870404in}}{\pgfqpoint{1.280216in}{0.870404in}}%
\pgfpathcurveto{\pgfqpoint{1.269166in}{0.870404in}}{\pgfqpoint{1.258567in}{0.866014in}}{\pgfqpoint{1.250753in}{0.858201in}}%
\pgfpathcurveto{\pgfqpoint{1.242940in}{0.850387in}}{\pgfqpoint{1.238549in}{0.839788in}}{\pgfqpoint{1.238549in}{0.828738in}}%
\pgfpathcurveto{\pgfqpoint{1.238549in}{0.817688in}}{\pgfqpoint{1.242940in}{0.807089in}}{\pgfqpoint{1.250753in}{0.799275in}}%
\pgfpathcurveto{\pgfqpoint{1.258567in}{0.791461in}}{\pgfqpoint{1.269166in}{0.787071in}}{\pgfqpoint{1.280216in}{0.787071in}}%
\pgfpathclose%
\pgfusepath{stroke,fill}%
\end{pgfscope}%
\begin{pgfscope}%
\pgfpathrectangle{\pgfqpoint{0.787074in}{0.548769in}}{\pgfqpoint{5.062926in}{3.102590in}}%
\pgfusepath{clip}%
\pgfsetbuttcap%
\pgfsetroundjoin%
\definecolor{currentfill}{rgb}{0.121569,0.466667,0.705882}%
\pgfsetfillcolor{currentfill}%
\pgfsetlinewidth{1.003750pt}%
\definecolor{currentstroke}{rgb}{0.121569,0.466667,0.705882}%
\pgfsetstrokecolor{currentstroke}%
\pgfsetdash{}{0pt}%
\pgfpathmoveto{\pgfqpoint{1.543225in}{0.660658in}}%
\pgfpathcurveto{\pgfqpoint{1.554275in}{0.660658in}}{\pgfqpoint{1.564874in}{0.665048in}}{\pgfqpoint{1.572688in}{0.672862in}}%
\pgfpathcurveto{\pgfqpoint{1.580502in}{0.680676in}}{\pgfqpoint{1.584892in}{0.691275in}}{\pgfqpoint{1.584892in}{0.702325in}}%
\pgfpathcurveto{\pgfqpoint{1.584892in}{0.713375in}}{\pgfqpoint{1.580502in}{0.723974in}}{\pgfqpoint{1.572688in}{0.731788in}}%
\pgfpathcurveto{\pgfqpoint{1.564874in}{0.739601in}}{\pgfqpoint{1.554275in}{0.743991in}}{\pgfqpoint{1.543225in}{0.743991in}}%
\pgfpathcurveto{\pgfqpoint{1.532175in}{0.743991in}}{\pgfqpoint{1.521576in}{0.739601in}}{\pgfqpoint{1.513762in}{0.731788in}}%
\pgfpathcurveto{\pgfqpoint{1.505949in}{0.723974in}}{\pgfqpoint{1.501558in}{0.713375in}}{\pgfqpoint{1.501558in}{0.702325in}}%
\pgfpathcurveto{\pgfqpoint{1.501558in}{0.691275in}}{\pgfqpoint{1.505949in}{0.680676in}}{\pgfqpoint{1.513762in}{0.672862in}}%
\pgfpathcurveto{\pgfqpoint{1.521576in}{0.665048in}}{\pgfqpoint{1.532175in}{0.660658in}}{\pgfqpoint{1.543225in}{0.660658in}}%
\pgfpathclose%
\pgfusepath{stroke,fill}%
\end{pgfscope}%
\begin{pgfscope}%
\pgfpathrectangle{\pgfqpoint{0.787074in}{0.548769in}}{\pgfqpoint{5.062926in}{3.102590in}}%
\pgfusepath{clip}%
\pgfsetbuttcap%
\pgfsetroundjoin%
\definecolor{currentfill}{rgb}{0.121569,0.466667,0.705882}%
\pgfsetfillcolor{currentfill}%
\pgfsetlinewidth{1.003750pt}%
\definecolor{currentstroke}{rgb}{0.121569,0.466667,0.705882}%
\pgfsetstrokecolor{currentstroke}%
\pgfsetdash{}{0pt}%
\pgfpathmoveto{\pgfqpoint{3.055528in}{0.648155in}}%
\pgfpathcurveto{\pgfqpoint{3.066578in}{0.648155in}}{\pgfqpoint{3.077177in}{0.652546in}}{\pgfqpoint{3.084991in}{0.660359in}}%
\pgfpathcurveto{\pgfqpoint{3.092804in}{0.668173in}}{\pgfqpoint{3.097194in}{0.678772in}}{\pgfqpoint{3.097194in}{0.689822in}}%
\pgfpathcurveto{\pgfqpoint{3.097194in}{0.700872in}}{\pgfqpoint{3.092804in}{0.711471in}}{\pgfqpoint{3.084991in}{0.719285in}}%
\pgfpathcurveto{\pgfqpoint{3.077177in}{0.727098in}}{\pgfqpoint{3.066578in}{0.731489in}}{\pgfqpoint{3.055528in}{0.731489in}}%
\pgfpathcurveto{\pgfqpoint{3.044478in}{0.731489in}}{\pgfqpoint{3.033879in}{0.727098in}}{\pgfqpoint{3.026065in}{0.719285in}}%
\pgfpathcurveto{\pgfqpoint{3.018251in}{0.711471in}}{\pgfqpoint{3.013861in}{0.700872in}}{\pgfqpoint{3.013861in}{0.689822in}}%
\pgfpathcurveto{\pgfqpoint{3.013861in}{0.678772in}}{\pgfqpoint{3.018251in}{0.668173in}}{\pgfqpoint{3.026065in}{0.660359in}}%
\pgfpathcurveto{\pgfqpoint{3.033879in}{0.652546in}}{\pgfqpoint{3.044478in}{0.648155in}}{\pgfqpoint{3.055528in}{0.648155in}}%
\pgfpathclose%
\pgfusepath{stroke,fill}%
\end{pgfscope}%
\begin{pgfscope}%
\pgfpathrectangle{\pgfqpoint{0.787074in}{0.548769in}}{\pgfqpoint{5.062926in}{3.102590in}}%
\pgfusepath{clip}%
\pgfsetbuttcap%
\pgfsetroundjoin%
\definecolor{currentfill}{rgb}{0.121569,0.466667,0.705882}%
\pgfsetfillcolor{currentfill}%
\pgfsetlinewidth{1.003750pt}%
\definecolor{currentstroke}{rgb}{0.121569,0.466667,0.705882}%
\pgfsetstrokecolor{currentstroke}%
\pgfsetdash{}{0pt}%
\pgfpathmoveto{\pgfqpoint{1.543225in}{0.648148in}}%
\pgfpathcurveto{\pgfqpoint{1.554275in}{0.648148in}}{\pgfqpoint{1.564874in}{0.652539in}}{\pgfqpoint{1.572688in}{0.660352in}}%
\pgfpathcurveto{\pgfqpoint{1.580502in}{0.668166in}}{\pgfqpoint{1.584892in}{0.678765in}}{\pgfqpoint{1.584892in}{0.689815in}}%
\pgfpathcurveto{\pgfqpoint{1.584892in}{0.700865in}}{\pgfqpoint{1.580502in}{0.711464in}}{\pgfqpoint{1.572688in}{0.719278in}}%
\pgfpathcurveto{\pgfqpoint{1.564874in}{0.727091in}}{\pgfqpoint{1.554275in}{0.731482in}}{\pgfqpoint{1.543225in}{0.731482in}}%
\pgfpathcurveto{\pgfqpoint{1.532175in}{0.731482in}}{\pgfqpoint{1.521576in}{0.727091in}}{\pgfqpoint{1.513762in}{0.719278in}}%
\pgfpathcurveto{\pgfqpoint{1.505949in}{0.711464in}}{\pgfqpoint{1.501558in}{0.700865in}}{\pgfqpoint{1.501558in}{0.689815in}}%
\pgfpathcurveto{\pgfqpoint{1.501558in}{0.678765in}}{\pgfqpoint{1.505949in}{0.668166in}}{\pgfqpoint{1.513762in}{0.660352in}}%
\pgfpathcurveto{\pgfqpoint{1.521576in}{0.652539in}}{\pgfqpoint{1.532175in}{0.648148in}}{\pgfqpoint{1.543225in}{0.648148in}}%
\pgfpathclose%
\pgfusepath{stroke,fill}%
\end{pgfscope}%
\begin{pgfscope}%
\pgfpathrectangle{\pgfqpoint{0.787074in}{0.548769in}}{\pgfqpoint{5.062926in}{3.102590in}}%
\pgfusepath{clip}%
\pgfsetbuttcap%
\pgfsetroundjoin%
\definecolor{currentfill}{rgb}{1.000000,0.498039,0.054902}%
\pgfsetfillcolor{currentfill}%
\pgfsetlinewidth{1.003750pt}%
\definecolor{currentstroke}{rgb}{1.000000,0.498039,0.054902}%
\pgfsetstrokecolor{currentstroke}%
\pgfsetdash{}{0pt}%
\pgfpathmoveto{\pgfqpoint{1.937739in}{1.766469in}}%
\pgfpathcurveto{\pgfqpoint{1.948789in}{1.766469in}}{\pgfqpoint{1.959388in}{1.770859in}}{\pgfqpoint{1.967202in}{1.778672in}}%
\pgfpathcurveto{\pgfqpoint{1.975015in}{1.786486in}}{\pgfqpoint{1.979406in}{1.797085in}}{\pgfqpoint{1.979406in}{1.808135in}}%
\pgfpathcurveto{\pgfqpoint{1.979406in}{1.819185in}}{\pgfqpoint{1.975015in}{1.829784in}}{\pgfqpoint{1.967202in}{1.837598in}}%
\pgfpathcurveto{\pgfqpoint{1.959388in}{1.845412in}}{\pgfqpoint{1.948789in}{1.849802in}}{\pgfqpoint{1.937739in}{1.849802in}}%
\pgfpathcurveto{\pgfqpoint{1.926689in}{1.849802in}}{\pgfqpoint{1.916090in}{1.845412in}}{\pgfqpoint{1.908276in}{1.837598in}}%
\pgfpathcurveto{\pgfqpoint{1.900462in}{1.829784in}}{\pgfqpoint{1.896072in}{1.819185in}}{\pgfqpoint{1.896072in}{1.808135in}}%
\pgfpathcurveto{\pgfqpoint{1.896072in}{1.797085in}}{\pgfqpoint{1.900462in}{1.786486in}}{\pgfqpoint{1.908276in}{1.778672in}}%
\pgfpathcurveto{\pgfqpoint{1.916090in}{1.770859in}}{\pgfqpoint{1.926689in}{1.766469in}}{\pgfqpoint{1.937739in}{1.766469in}}%
\pgfpathclose%
\pgfusepath{stroke,fill}%
\end{pgfscope}%
\begin{pgfscope}%
\pgfpathrectangle{\pgfqpoint{0.787074in}{0.548769in}}{\pgfqpoint{5.062926in}{3.102590in}}%
\pgfusepath{clip}%
\pgfsetbuttcap%
\pgfsetroundjoin%
\definecolor{currentfill}{rgb}{1.000000,0.498039,0.054902}%
\pgfsetfillcolor{currentfill}%
\pgfsetlinewidth{1.003750pt}%
\definecolor{currentstroke}{rgb}{1.000000,0.498039,0.054902}%
\pgfsetstrokecolor{currentstroke}%
\pgfsetdash{}{0pt}%
\pgfpathmoveto{\pgfqpoint{1.871987in}{2.575149in}}%
\pgfpathcurveto{\pgfqpoint{1.883037in}{2.575149in}}{\pgfqpoint{1.893636in}{2.579539in}}{\pgfqpoint{1.901449in}{2.587353in}}%
\pgfpathcurveto{\pgfqpoint{1.909263in}{2.595166in}}{\pgfqpoint{1.913653in}{2.605765in}}{\pgfqpoint{1.913653in}{2.616816in}}%
\pgfpathcurveto{\pgfqpoint{1.913653in}{2.627866in}}{\pgfqpoint{1.909263in}{2.638465in}}{\pgfqpoint{1.901449in}{2.646278in}}%
\pgfpathcurveto{\pgfqpoint{1.893636in}{2.654092in}}{\pgfqpoint{1.883037in}{2.658482in}}{\pgfqpoint{1.871987in}{2.658482in}}%
\pgfpathcurveto{\pgfqpoint{1.860936in}{2.658482in}}{\pgfqpoint{1.850337in}{2.654092in}}{\pgfqpoint{1.842524in}{2.646278in}}%
\pgfpathcurveto{\pgfqpoint{1.834710in}{2.638465in}}{\pgfqpoint{1.830320in}{2.627866in}}{\pgfqpoint{1.830320in}{2.616816in}}%
\pgfpathcurveto{\pgfqpoint{1.830320in}{2.605765in}}{\pgfqpoint{1.834710in}{2.595166in}}{\pgfqpoint{1.842524in}{2.587353in}}%
\pgfpathcurveto{\pgfqpoint{1.850337in}{2.579539in}}{\pgfqpoint{1.860936in}{2.575149in}}{\pgfqpoint{1.871987in}{2.575149in}}%
\pgfpathclose%
\pgfusepath{stroke,fill}%
\end{pgfscope}%
\begin{pgfscope}%
\pgfpathrectangle{\pgfqpoint{0.787074in}{0.548769in}}{\pgfqpoint{5.062926in}{3.102590in}}%
\pgfusepath{clip}%
\pgfsetbuttcap%
\pgfsetroundjoin%
\definecolor{currentfill}{rgb}{0.121569,0.466667,0.705882}%
\pgfsetfillcolor{currentfill}%
\pgfsetlinewidth{1.003750pt}%
\definecolor{currentstroke}{rgb}{0.121569,0.466667,0.705882}%
\pgfsetstrokecolor{currentstroke}%
\pgfsetdash{}{0pt}%
\pgfpathmoveto{\pgfqpoint{1.674730in}{0.825330in}}%
\pgfpathcurveto{\pgfqpoint{1.685780in}{0.825330in}}{\pgfqpoint{1.696379in}{0.829720in}}{\pgfqpoint{1.704193in}{0.837534in}}%
\pgfpathcurveto{\pgfqpoint{1.712006in}{0.845347in}}{\pgfqpoint{1.716396in}{0.855946in}}{\pgfqpoint{1.716396in}{0.866996in}}%
\pgfpathcurveto{\pgfqpoint{1.716396in}{0.878046in}}{\pgfqpoint{1.712006in}{0.888646in}}{\pgfqpoint{1.704193in}{0.896459in}}%
\pgfpathcurveto{\pgfqpoint{1.696379in}{0.904273in}}{\pgfqpoint{1.685780in}{0.908663in}}{\pgfqpoint{1.674730in}{0.908663in}}%
\pgfpathcurveto{\pgfqpoint{1.663680in}{0.908663in}}{\pgfqpoint{1.653081in}{0.904273in}}{\pgfqpoint{1.645267in}{0.896459in}}%
\pgfpathcurveto{\pgfqpoint{1.637453in}{0.888646in}}{\pgfqpoint{1.633063in}{0.878046in}}{\pgfqpoint{1.633063in}{0.866996in}}%
\pgfpathcurveto{\pgfqpoint{1.633063in}{0.855946in}}{\pgfqpoint{1.637453in}{0.845347in}}{\pgfqpoint{1.645267in}{0.837534in}}%
\pgfpathcurveto{\pgfqpoint{1.653081in}{0.829720in}}{\pgfqpoint{1.663680in}{0.825330in}}{\pgfqpoint{1.674730in}{0.825330in}}%
\pgfpathclose%
\pgfusepath{stroke,fill}%
\end{pgfscope}%
\begin{pgfscope}%
\pgfpathrectangle{\pgfqpoint{0.787074in}{0.548769in}}{\pgfqpoint{5.062926in}{3.102590in}}%
\pgfusepath{clip}%
\pgfsetbuttcap%
\pgfsetroundjoin%
\definecolor{currentfill}{rgb}{1.000000,0.498039,0.054902}%
\pgfsetfillcolor{currentfill}%
\pgfsetlinewidth{1.003750pt}%
\definecolor{currentstroke}{rgb}{1.000000,0.498039,0.054902}%
\pgfsetstrokecolor{currentstroke}%
\pgfsetdash{}{0pt}%
\pgfpathmoveto{\pgfqpoint{2.200748in}{1.508166in}}%
\pgfpathcurveto{\pgfqpoint{2.211798in}{1.508166in}}{\pgfqpoint{2.222397in}{1.512556in}}{\pgfqpoint{2.230211in}{1.520370in}}%
\pgfpathcurveto{\pgfqpoint{2.238024in}{1.528184in}}{\pgfqpoint{2.242415in}{1.538783in}}{\pgfqpoint{2.242415in}{1.549833in}}%
\pgfpathcurveto{\pgfqpoint{2.242415in}{1.560883in}}{\pgfqpoint{2.238024in}{1.571482in}}{\pgfqpoint{2.230211in}{1.579296in}}%
\pgfpathcurveto{\pgfqpoint{2.222397in}{1.587109in}}{\pgfqpoint{2.211798in}{1.591499in}}{\pgfqpoint{2.200748in}{1.591499in}}%
\pgfpathcurveto{\pgfqpoint{2.189698in}{1.591499in}}{\pgfqpoint{2.179099in}{1.587109in}}{\pgfqpoint{2.171285in}{1.579296in}}%
\pgfpathcurveto{\pgfqpoint{2.163472in}{1.571482in}}{\pgfqpoint{2.159081in}{1.560883in}}{\pgfqpoint{2.159081in}{1.549833in}}%
\pgfpathcurveto{\pgfqpoint{2.159081in}{1.538783in}}{\pgfqpoint{2.163472in}{1.528184in}}{\pgfqpoint{2.171285in}{1.520370in}}%
\pgfpathcurveto{\pgfqpoint{2.179099in}{1.512556in}}{\pgfqpoint{2.189698in}{1.508166in}}{\pgfqpoint{2.200748in}{1.508166in}}%
\pgfpathclose%
\pgfusepath{stroke,fill}%
\end{pgfscope}%
\begin{pgfscope}%
\pgfpathrectangle{\pgfqpoint{0.787074in}{0.548769in}}{\pgfqpoint{5.062926in}{3.102590in}}%
\pgfusepath{clip}%
\pgfsetbuttcap%
\pgfsetroundjoin%
\definecolor{currentfill}{rgb}{1.000000,0.498039,0.054902}%
\pgfsetfillcolor{currentfill}%
\pgfsetlinewidth{1.003750pt}%
\definecolor{currentstroke}{rgb}{1.000000,0.498039,0.054902}%
\pgfsetstrokecolor{currentstroke}%
\pgfsetdash{}{0pt}%
\pgfpathmoveto{\pgfqpoint{1.871987in}{2.241293in}}%
\pgfpathcurveto{\pgfqpoint{1.883037in}{2.241293in}}{\pgfqpoint{1.893636in}{2.245684in}}{\pgfqpoint{1.901449in}{2.253497in}}%
\pgfpathcurveto{\pgfqpoint{1.909263in}{2.261311in}}{\pgfqpoint{1.913653in}{2.271910in}}{\pgfqpoint{1.913653in}{2.282960in}}%
\pgfpathcurveto{\pgfqpoint{1.913653in}{2.294010in}}{\pgfqpoint{1.909263in}{2.304609in}}{\pgfqpoint{1.901449in}{2.312423in}}%
\pgfpathcurveto{\pgfqpoint{1.893636in}{2.320236in}}{\pgfqpoint{1.883037in}{2.324627in}}{\pgfqpoint{1.871987in}{2.324627in}}%
\pgfpathcurveto{\pgfqpoint{1.860936in}{2.324627in}}{\pgfqpoint{1.850337in}{2.320236in}}{\pgfqpoint{1.842524in}{2.312423in}}%
\pgfpathcurveto{\pgfqpoint{1.834710in}{2.304609in}}{\pgfqpoint{1.830320in}{2.294010in}}{\pgfqpoint{1.830320in}{2.282960in}}%
\pgfpathcurveto{\pgfqpoint{1.830320in}{2.271910in}}{\pgfqpoint{1.834710in}{2.261311in}}{\pgfqpoint{1.842524in}{2.253497in}}%
\pgfpathcurveto{\pgfqpoint{1.850337in}{2.245684in}}{\pgfqpoint{1.860936in}{2.241293in}}{\pgfqpoint{1.871987in}{2.241293in}}%
\pgfpathclose%
\pgfusepath{stroke,fill}%
\end{pgfscope}%
\begin{pgfscope}%
\pgfpathrectangle{\pgfqpoint{0.787074in}{0.548769in}}{\pgfqpoint{5.062926in}{3.102590in}}%
\pgfusepath{clip}%
\pgfsetbuttcap%
\pgfsetroundjoin%
\definecolor{currentfill}{rgb}{0.121569,0.466667,0.705882}%
\pgfsetfillcolor{currentfill}%
\pgfsetlinewidth{1.003750pt}%
\definecolor{currentstroke}{rgb}{0.121569,0.466667,0.705882}%
\pgfsetstrokecolor{currentstroke}%
\pgfsetdash{}{0pt}%
\pgfpathmoveto{\pgfqpoint{5.619867in}{0.648150in}}%
\pgfpathcurveto{\pgfqpoint{5.630917in}{0.648150in}}{\pgfqpoint{5.641516in}{0.652540in}}{\pgfqpoint{5.649330in}{0.660353in}}%
\pgfpathcurveto{\pgfqpoint{5.657143in}{0.668167in}}{\pgfqpoint{5.661534in}{0.678766in}}{\pgfqpoint{5.661534in}{0.689816in}}%
\pgfpathcurveto{\pgfqpoint{5.661534in}{0.700866in}}{\pgfqpoint{5.657143in}{0.711465in}}{\pgfqpoint{5.649330in}{0.719279in}}%
\pgfpathcurveto{\pgfqpoint{5.641516in}{0.727093in}}{\pgfqpoint{5.630917in}{0.731483in}}{\pgfqpoint{5.619867in}{0.731483in}}%
\pgfpathcurveto{\pgfqpoint{5.608817in}{0.731483in}}{\pgfqpoint{5.598218in}{0.727093in}}{\pgfqpoint{5.590404in}{0.719279in}}%
\pgfpathcurveto{\pgfqpoint{5.582591in}{0.711465in}}{\pgfqpoint{5.578200in}{0.700866in}}{\pgfqpoint{5.578200in}{0.689816in}}%
\pgfpathcurveto{\pgfqpoint{5.578200in}{0.678766in}}{\pgfqpoint{5.582591in}{0.668167in}}{\pgfqpoint{5.590404in}{0.660353in}}%
\pgfpathcurveto{\pgfqpoint{5.598218in}{0.652540in}}{\pgfqpoint{5.608817in}{0.648150in}}{\pgfqpoint{5.619867in}{0.648150in}}%
\pgfpathclose%
\pgfusepath{stroke,fill}%
\end{pgfscope}%
\begin{pgfscope}%
\pgfpathrectangle{\pgfqpoint{0.787074in}{0.548769in}}{\pgfqpoint{5.062926in}{3.102590in}}%
\pgfusepath{clip}%
\pgfsetbuttcap%
\pgfsetroundjoin%
\definecolor{currentfill}{rgb}{1.000000,0.498039,0.054902}%
\pgfsetfillcolor{currentfill}%
\pgfsetlinewidth{1.003750pt}%
\definecolor{currentstroke}{rgb}{1.000000,0.498039,0.054902}%
\pgfsetstrokecolor{currentstroke}%
\pgfsetdash{}{0pt}%
\pgfpathmoveto{\pgfqpoint{1.871987in}{2.050079in}}%
\pgfpathcurveto{\pgfqpoint{1.883037in}{2.050079in}}{\pgfqpoint{1.893636in}{2.054470in}}{\pgfqpoint{1.901449in}{2.062283in}}%
\pgfpathcurveto{\pgfqpoint{1.909263in}{2.070097in}}{\pgfqpoint{1.913653in}{2.080696in}}{\pgfqpoint{1.913653in}{2.091746in}}%
\pgfpathcurveto{\pgfqpoint{1.913653in}{2.102796in}}{\pgfqpoint{1.909263in}{2.113395in}}{\pgfqpoint{1.901449in}{2.121209in}}%
\pgfpathcurveto{\pgfqpoint{1.893636in}{2.129022in}}{\pgfqpoint{1.883037in}{2.133413in}}{\pgfqpoint{1.871987in}{2.133413in}}%
\pgfpathcurveto{\pgfqpoint{1.860936in}{2.133413in}}{\pgfqpoint{1.850337in}{2.129022in}}{\pgfqpoint{1.842524in}{2.121209in}}%
\pgfpathcurveto{\pgfqpoint{1.834710in}{2.113395in}}{\pgfqpoint{1.830320in}{2.102796in}}{\pgfqpoint{1.830320in}{2.091746in}}%
\pgfpathcurveto{\pgfqpoint{1.830320in}{2.080696in}}{\pgfqpoint{1.834710in}{2.070097in}}{\pgfqpoint{1.842524in}{2.062283in}}%
\pgfpathcurveto{\pgfqpoint{1.850337in}{2.054470in}}{\pgfqpoint{1.860936in}{2.050079in}}{\pgfqpoint{1.871987in}{2.050079in}}%
\pgfpathclose%
\pgfusepath{stroke,fill}%
\end{pgfscope}%
\begin{pgfscope}%
\pgfpathrectangle{\pgfqpoint{0.787074in}{0.548769in}}{\pgfqpoint{5.062926in}{3.102590in}}%
\pgfusepath{clip}%
\pgfsetbuttcap%
\pgfsetroundjoin%
\definecolor{currentfill}{rgb}{0.121569,0.466667,0.705882}%
\pgfsetfillcolor{currentfill}%
\pgfsetlinewidth{1.003750pt}%
\definecolor{currentstroke}{rgb}{0.121569,0.466667,0.705882}%
\pgfsetstrokecolor{currentstroke}%
\pgfsetdash{}{0pt}%
\pgfpathmoveto{\pgfqpoint{1.082959in}{0.648148in}}%
\pgfpathcurveto{\pgfqpoint{1.094009in}{0.648148in}}{\pgfqpoint{1.104608in}{0.652539in}}{\pgfqpoint{1.112422in}{0.660352in}}%
\pgfpathcurveto{\pgfqpoint{1.120236in}{0.668166in}}{\pgfqpoint{1.124626in}{0.678765in}}{\pgfqpoint{1.124626in}{0.689815in}}%
\pgfpathcurveto{\pgfqpoint{1.124626in}{0.700865in}}{\pgfqpoint{1.120236in}{0.711464in}}{\pgfqpoint{1.112422in}{0.719278in}}%
\pgfpathcurveto{\pgfqpoint{1.104608in}{0.727091in}}{\pgfqpoint{1.094009in}{0.731482in}}{\pgfqpoint{1.082959in}{0.731482in}}%
\pgfpathcurveto{\pgfqpoint{1.071909in}{0.731482in}}{\pgfqpoint{1.061310in}{0.727091in}}{\pgfqpoint{1.053496in}{0.719278in}}%
\pgfpathcurveto{\pgfqpoint{1.045683in}{0.711464in}}{\pgfqpoint{1.041292in}{0.700865in}}{\pgfqpoint{1.041292in}{0.689815in}}%
\pgfpathcurveto{\pgfqpoint{1.041292in}{0.678765in}}{\pgfqpoint{1.045683in}{0.668166in}}{\pgfqpoint{1.053496in}{0.660352in}}%
\pgfpathcurveto{\pgfqpoint{1.061310in}{0.652539in}}{\pgfqpoint{1.071909in}{0.648148in}}{\pgfqpoint{1.082959in}{0.648148in}}%
\pgfpathclose%
\pgfusepath{stroke,fill}%
\end{pgfscope}%
\begin{pgfscope}%
\pgfpathrectangle{\pgfqpoint{0.787074in}{0.548769in}}{\pgfqpoint{5.062926in}{3.102590in}}%
\pgfusepath{clip}%
\pgfsetbuttcap%
\pgfsetroundjoin%
\definecolor{currentfill}{rgb}{0.121569,0.466667,0.705882}%
\pgfsetfillcolor{currentfill}%
\pgfsetlinewidth{1.003750pt}%
\definecolor{currentstroke}{rgb}{0.121569,0.466667,0.705882}%
\pgfsetstrokecolor{currentstroke}%
\pgfsetdash{}{0pt}%
\pgfpathmoveto{\pgfqpoint{1.740482in}{0.648168in}}%
\pgfpathcurveto{\pgfqpoint{1.751532in}{0.648168in}}{\pgfqpoint{1.762131in}{0.652558in}}{\pgfqpoint{1.769945in}{0.660372in}}%
\pgfpathcurveto{\pgfqpoint{1.777758in}{0.668185in}}{\pgfqpoint{1.782149in}{0.678784in}}{\pgfqpoint{1.782149in}{0.689834in}}%
\pgfpathcurveto{\pgfqpoint{1.782149in}{0.700885in}}{\pgfqpoint{1.777758in}{0.711484in}}{\pgfqpoint{1.769945in}{0.719297in}}%
\pgfpathcurveto{\pgfqpoint{1.762131in}{0.727111in}}{\pgfqpoint{1.751532in}{0.731501in}}{\pgfqpoint{1.740482in}{0.731501in}}%
\pgfpathcurveto{\pgfqpoint{1.729432in}{0.731501in}}{\pgfqpoint{1.718833in}{0.727111in}}{\pgfqpoint{1.711019in}{0.719297in}}%
\pgfpathcurveto{\pgfqpoint{1.703206in}{0.711484in}}{\pgfqpoint{1.698815in}{0.700885in}}{\pgfqpoint{1.698815in}{0.689834in}}%
\pgfpathcurveto{\pgfqpoint{1.698815in}{0.678784in}}{\pgfqpoint{1.703206in}{0.668185in}}{\pgfqpoint{1.711019in}{0.660372in}}%
\pgfpathcurveto{\pgfqpoint{1.718833in}{0.652558in}}{\pgfqpoint{1.729432in}{0.648168in}}{\pgfqpoint{1.740482in}{0.648168in}}%
\pgfpathclose%
\pgfusepath{stroke,fill}%
\end{pgfscope}%
\begin{pgfscope}%
\pgfpathrectangle{\pgfqpoint{0.787074in}{0.548769in}}{\pgfqpoint{5.062926in}{3.102590in}}%
\pgfusepath{clip}%
\pgfsetbuttcap%
\pgfsetroundjoin%
\definecolor{currentfill}{rgb}{0.121569,0.466667,0.705882}%
\pgfsetfillcolor{currentfill}%
\pgfsetlinewidth{1.003750pt}%
\definecolor{currentstroke}{rgb}{0.121569,0.466667,0.705882}%
\pgfsetstrokecolor{currentstroke}%
\pgfsetdash{}{0pt}%
\pgfpathmoveto{\pgfqpoint{1.543225in}{0.861755in}}%
\pgfpathcurveto{\pgfqpoint{1.554275in}{0.861755in}}{\pgfqpoint{1.564874in}{0.866145in}}{\pgfqpoint{1.572688in}{0.873958in}}%
\pgfpathcurveto{\pgfqpoint{1.580502in}{0.881772in}}{\pgfqpoint{1.584892in}{0.892371in}}{\pgfqpoint{1.584892in}{0.903421in}}%
\pgfpathcurveto{\pgfqpoint{1.584892in}{0.914471in}}{\pgfqpoint{1.580502in}{0.925070in}}{\pgfqpoint{1.572688in}{0.932884in}}%
\pgfpathcurveto{\pgfqpoint{1.564874in}{0.940698in}}{\pgfqpoint{1.554275in}{0.945088in}}{\pgfqpoint{1.543225in}{0.945088in}}%
\pgfpathcurveto{\pgfqpoint{1.532175in}{0.945088in}}{\pgfqpoint{1.521576in}{0.940698in}}{\pgfqpoint{1.513762in}{0.932884in}}%
\pgfpathcurveto{\pgfqpoint{1.505949in}{0.925070in}}{\pgfqpoint{1.501558in}{0.914471in}}{\pgfqpoint{1.501558in}{0.903421in}}%
\pgfpathcurveto{\pgfqpoint{1.501558in}{0.892371in}}{\pgfqpoint{1.505949in}{0.881772in}}{\pgfqpoint{1.513762in}{0.873958in}}%
\pgfpathcurveto{\pgfqpoint{1.521576in}{0.866145in}}{\pgfqpoint{1.532175in}{0.861755in}}{\pgfqpoint{1.543225in}{0.861755in}}%
\pgfpathclose%
\pgfusepath{stroke,fill}%
\end{pgfscope}%
\begin{pgfscope}%
\pgfpathrectangle{\pgfqpoint{0.787074in}{0.548769in}}{\pgfqpoint{5.062926in}{3.102590in}}%
\pgfusepath{clip}%
\pgfsetbuttcap%
\pgfsetroundjoin%
\definecolor{currentfill}{rgb}{0.121569,0.466667,0.705882}%
\pgfsetfillcolor{currentfill}%
\pgfsetlinewidth{1.003750pt}%
\definecolor{currentstroke}{rgb}{0.121569,0.466667,0.705882}%
\pgfsetstrokecolor{currentstroke}%
\pgfsetdash{}{0pt}%
\pgfpathmoveto{\pgfqpoint{3.910308in}{1.953744in}}%
\pgfpathcurveto{\pgfqpoint{3.921358in}{1.953744in}}{\pgfqpoint{3.931957in}{1.958135in}}{\pgfqpoint{3.939770in}{1.965948in}}%
\pgfpathcurveto{\pgfqpoint{3.947584in}{1.973762in}}{\pgfqpoint{3.951974in}{1.984361in}}{\pgfqpoint{3.951974in}{1.995411in}}%
\pgfpathcurveto{\pgfqpoint{3.951974in}{2.006461in}}{\pgfqpoint{3.947584in}{2.017060in}}{\pgfqpoint{3.939770in}{2.024874in}}%
\pgfpathcurveto{\pgfqpoint{3.931957in}{2.032687in}}{\pgfqpoint{3.921358in}{2.037078in}}{\pgfqpoint{3.910308in}{2.037078in}}%
\pgfpathcurveto{\pgfqpoint{3.899257in}{2.037078in}}{\pgfqpoint{3.888658in}{2.032687in}}{\pgfqpoint{3.880845in}{2.024874in}}%
\pgfpathcurveto{\pgfqpoint{3.873031in}{2.017060in}}{\pgfqpoint{3.868641in}{2.006461in}}{\pgfqpoint{3.868641in}{1.995411in}}%
\pgfpathcurveto{\pgfqpoint{3.868641in}{1.984361in}}{\pgfqpoint{3.873031in}{1.973762in}}{\pgfqpoint{3.880845in}{1.965948in}}%
\pgfpathcurveto{\pgfqpoint{3.888658in}{1.958135in}}{\pgfqpoint{3.899257in}{1.953744in}}{\pgfqpoint{3.910308in}{1.953744in}}%
\pgfpathclose%
\pgfusepath{stroke,fill}%
\end{pgfscope}%
\begin{pgfscope}%
\pgfpathrectangle{\pgfqpoint{0.787074in}{0.548769in}}{\pgfqpoint{5.062926in}{3.102590in}}%
\pgfusepath{clip}%
\pgfsetbuttcap%
\pgfsetroundjoin%
\definecolor{currentfill}{rgb}{1.000000,0.498039,0.054902}%
\pgfsetfillcolor{currentfill}%
\pgfsetlinewidth{1.003750pt}%
\definecolor{currentstroke}{rgb}{1.000000,0.498039,0.054902}%
\pgfsetstrokecolor{currentstroke}%
\pgfsetdash{}{0pt}%
\pgfpathmoveto{\pgfqpoint{2.134996in}{1.401411in}}%
\pgfpathcurveto{\pgfqpoint{2.146046in}{1.401411in}}{\pgfqpoint{2.156645in}{1.405801in}}{\pgfqpoint{2.164459in}{1.413615in}}%
\pgfpathcurveto{\pgfqpoint{2.172272in}{1.421429in}}{\pgfqpoint{2.176662in}{1.432028in}}{\pgfqpoint{2.176662in}{1.443078in}}%
\pgfpathcurveto{\pgfqpoint{2.176662in}{1.454128in}}{\pgfqpoint{2.172272in}{1.464727in}}{\pgfqpoint{2.164459in}{1.472540in}}%
\pgfpathcurveto{\pgfqpoint{2.156645in}{1.480354in}}{\pgfqpoint{2.146046in}{1.484744in}}{\pgfqpoint{2.134996in}{1.484744in}}%
\pgfpathcurveto{\pgfqpoint{2.123946in}{1.484744in}}{\pgfqpoint{2.113347in}{1.480354in}}{\pgfqpoint{2.105533in}{1.472540in}}%
\pgfpathcurveto{\pgfqpoint{2.097719in}{1.464727in}}{\pgfqpoint{2.093329in}{1.454128in}}{\pgfqpoint{2.093329in}{1.443078in}}%
\pgfpathcurveto{\pgfqpoint{2.093329in}{1.432028in}}{\pgfqpoint{2.097719in}{1.421429in}}{\pgfqpoint{2.105533in}{1.413615in}}%
\pgfpathcurveto{\pgfqpoint{2.113347in}{1.405801in}}{\pgfqpoint{2.123946in}{1.401411in}}{\pgfqpoint{2.134996in}{1.401411in}}%
\pgfpathclose%
\pgfusepath{stroke,fill}%
\end{pgfscope}%
\begin{pgfscope}%
\pgfpathrectangle{\pgfqpoint{0.787074in}{0.548769in}}{\pgfqpoint{5.062926in}{3.102590in}}%
\pgfusepath{clip}%
\pgfsetbuttcap%
\pgfsetroundjoin%
\definecolor{currentfill}{rgb}{0.121569,0.466667,0.705882}%
\pgfsetfillcolor{currentfill}%
\pgfsetlinewidth{1.003750pt}%
\definecolor{currentstroke}{rgb}{0.121569,0.466667,0.705882}%
\pgfsetstrokecolor{currentstroke}%
\pgfsetdash{}{0pt}%
\pgfpathmoveto{\pgfqpoint{1.411721in}{0.648141in}}%
\pgfpathcurveto{\pgfqpoint{1.422771in}{0.648141in}}{\pgfqpoint{1.433370in}{0.652531in}}{\pgfqpoint{1.441183in}{0.660344in}}%
\pgfpathcurveto{\pgfqpoint{1.448997in}{0.668158in}}{\pgfqpoint{1.453387in}{0.678757in}}{\pgfqpoint{1.453387in}{0.689807in}}%
\pgfpathcurveto{\pgfqpoint{1.453387in}{0.700857in}}{\pgfqpoint{1.448997in}{0.711456in}}{\pgfqpoint{1.441183in}{0.719270in}}%
\pgfpathcurveto{\pgfqpoint{1.433370in}{0.727084in}}{\pgfqpoint{1.422771in}{0.731474in}}{\pgfqpoint{1.411721in}{0.731474in}}%
\pgfpathcurveto{\pgfqpoint{1.400670in}{0.731474in}}{\pgfqpoint{1.390071in}{0.727084in}}{\pgfqpoint{1.382258in}{0.719270in}}%
\pgfpathcurveto{\pgfqpoint{1.374444in}{0.711456in}}{\pgfqpoint{1.370054in}{0.700857in}}{\pgfqpoint{1.370054in}{0.689807in}}%
\pgfpathcurveto{\pgfqpoint{1.370054in}{0.678757in}}{\pgfqpoint{1.374444in}{0.668158in}}{\pgfqpoint{1.382258in}{0.660344in}}%
\pgfpathcurveto{\pgfqpoint{1.390071in}{0.652531in}}{\pgfqpoint{1.400670in}{0.648141in}}{\pgfqpoint{1.411721in}{0.648141in}}%
\pgfpathclose%
\pgfusepath{stroke,fill}%
\end{pgfscope}%
\begin{pgfscope}%
\pgfpathrectangle{\pgfqpoint{0.787074in}{0.548769in}}{\pgfqpoint{5.062926in}{3.102590in}}%
\pgfusepath{clip}%
\pgfsetbuttcap%
\pgfsetroundjoin%
\definecolor{currentfill}{rgb}{0.121569,0.466667,0.705882}%
\pgfsetfillcolor{currentfill}%
\pgfsetlinewidth{1.003750pt}%
\definecolor{currentstroke}{rgb}{0.121569,0.466667,0.705882}%
\pgfsetstrokecolor{currentstroke}%
\pgfsetdash{}{0pt}%
\pgfpathmoveto{\pgfqpoint{1.280216in}{0.787322in}}%
\pgfpathcurveto{\pgfqpoint{1.291266in}{0.787322in}}{\pgfqpoint{1.301865in}{0.791712in}}{\pgfqpoint{1.309679in}{0.799526in}}%
\pgfpathcurveto{\pgfqpoint{1.317492in}{0.807339in}}{\pgfqpoint{1.321883in}{0.817938in}}{\pgfqpoint{1.321883in}{0.828988in}}%
\pgfpathcurveto{\pgfqpoint{1.321883in}{0.840039in}}{\pgfqpoint{1.317492in}{0.850638in}}{\pgfqpoint{1.309679in}{0.858451in}}%
\pgfpathcurveto{\pgfqpoint{1.301865in}{0.866265in}}{\pgfqpoint{1.291266in}{0.870655in}}{\pgfqpoint{1.280216in}{0.870655in}}%
\pgfpathcurveto{\pgfqpoint{1.269166in}{0.870655in}}{\pgfqpoint{1.258567in}{0.866265in}}{\pgfqpoint{1.250753in}{0.858451in}}%
\pgfpathcurveto{\pgfqpoint{1.242940in}{0.850638in}}{\pgfqpoint{1.238549in}{0.840039in}}{\pgfqpoint{1.238549in}{0.828988in}}%
\pgfpathcurveto{\pgfqpoint{1.238549in}{0.817938in}}{\pgfqpoint{1.242940in}{0.807339in}}{\pgfqpoint{1.250753in}{0.799526in}}%
\pgfpathcurveto{\pgfqpoint{1.258567in}{0.791712in}}{\pgfqpoint{1.269166in}{0.787322in}}{\pgfqpoint{1.280216in}{0.787322in}}%
\pgfpathclose%
\pgfusepath{stroke,fill}%
\end{pgfscope}%
\begin{pgfscope}%
\pgfpathrectangle{\pgfqpoint{0.787074in}{0.548769in}}{\pgfqpoint{5.062926in}{3.102590in}}%
\pgfusepath{clip}%
\pgfsetbuttcap%
\pgfsetroundjoin%
\definecolor{currentfill}{rgb}{0.121569,0.466667,0.705882}%
\pgfsetfillcolor{currentfill}%
\pgfsetlinewidth{1.003750pt}%
\definecolor{currentstroke}{rgb}{0.121569,0.466667,0.705882}%
\pgfsetstrokecolor{currentstroke}%
\pgfsetdash{}{0pt}%
\pgfpathmoveto{\pgfqpoint{1.740482in}{1.729931in}}%
\pgfpathcurveto{\pgfqpoint{1.751532in}{1.729931in}}{\pgfqpoint{1.762131in}{1.734321in}}{\pgfqpoint{1.769945in}{1.742135in}}%
\pgfpathcurveto{\pgfqpoint{1.777758in}{1.749948in}}{\pgfqpoint{1.782149in}{1.760547in}}{\pgfqpoint{1.782149in}{1.771597in}}%
\pgfpathcurveto{\pgfqpoint{1.782149in}{1.782647in}}{\pgfqpoint{1.777758in}{1.793246in}}{\pgfqpoint{1.769945in}{1.801060in}}%
\pgfpathcurveto{\pgfqpoint{1.762131in}{1.808874in}}{\pgfqpoint{1.751532in}{1.813264in}}{\pgfqpoint{1.740482in}{1.813264in}}%
\pgfpathcurveto{\pgfqpoint{1.729432in}{1.813264in}}{\pgfqpoint{1.718833in}{1.808874in}}{\pgfqpoint{1.711019in}{1.801060in}}%
\pgfpathcurveto{\pgfqpoint{1.703206in}{1.793246in}}{\pgfqpoint{1.698815in}{1.782647in}}{\pgfqpoint{1.698815in}{1.771597in}}%
\pgfpathcurveto{\pgfqpoint{1.698815in}{1.760547in}}{\pgfqpoint{1.703206in}{1.749948in}}{\pgfqpoint{1.711019in}{1.742135in}}%
\pgfpathcurveto{\pgfqpoint{1.718833in}{1.734321in}}{\pgfqpoint{1.729432in}{1.729931in}}{\pgfqpoint{1.740482in}{1.729931in}}%
\pgfpathclose%
\pgfusepath{stroke,fill}%
\end{pgfscope}%
\begin{pgfscope}%
\pgfpathrectangle{\pgfqpoint{0.787074in}{0.548769in}}{\pgfqpoint{5.062926in}{3.102590in}}%
\pgfusepath{clip}%
\pgfsetbuttcap%
\pgfsetroundjoin%
\definecolor{currentfill}{rgb}{0.121569,0.466667,0.705882}%
\pgfsetfillcolor{currentfill}%
\pgfsetlinewidth{1.003750pt}%
\definecolor{currentstroke}{rgb}{0.121569,0.466667,0.705882}%
\pgfsetstrokecolor{currentstroke}%
\pgfsetdash{}{0pt}%
\pgfpathmoveto{\pgfqpoint{1.280216in}{0.648129in}}%
\pgfpathcurveto{\pgfqpoint{1.291266in}{0.648129in}}{\pgfqpoint{1.301865in}{0.652519in}}{\pgfqpoint{1.309679in}{0.660333in}}%
\pgfpathcurveto{\pgfqpoint{1.317492in}{0.668146in}}{\pgfqpoint{1.321883in}{0.678745in}}{\pgfqpoint{1.321883in}{0.689796in}}%
\pgfpathcurveto{\pgfqpoint{1.321883in}{0.700846in}}{\pgfqpoint{1.317492in}{0.711445in}}{\pgfqpoint{1.309679in}{0.719258in}}%
\pgfpathcurveto{\pgfqpoint{1.301865in}{0.727072in}}{\pgfqpoint{1.291266in}{0.731462in}}{\pgfqpoint{1.280216in}{0.731462in}}%
\pgfpathcurveto{\pgfqpoint{1.269166in}{0.731462in}}{\pgfqpoint{1.258567in}{0.727072in}}{\pgfqpoint{1.250753in}{0.719258in}}%
\pgfpathcurveto{\pgfqpoint{1.242940in}{0.711445in}}{\pgfqpoint{1.238549in}{0.700846in}}{\pgfqpoint{1.238549in}{0.689796in}}%
\pgfpathcurveto{\pgfqpoint{1.238549in}{0.678745in}}{\pgfqpoint{1.242940in}{0.668146in}}{\pgfqpoint{1.250753in}{0.660333in}}%
\pgfpathcurveto{\pgfqpoint{1.258567in}{0.652519in}}{\pgfqpoint{1.269166in}{0.648129in}}{\pgfqpoint{1.280216in}{0.648129in}}%
\pgfpathclose%
\pgfusepath{stroke,fill}%
\end{pgfscope}%
\begin{pgfscope}%
\pgfpathrectangle{\pgfqpoint{0.787074in}{0.548769in}}{\pgfqpoint{5.062926in}{3.102590in}}%
\pgfusepath{clip}%
\pgfsetbuttcap%
\pgfsetroundjoin%
\definecolor{currentfill}{rgb}{1.000000,0.498039,0.054902}%
\pgfsetfillcolor{currentfill}%
\pgfsetlinewidth{1.003750pt}%
\definecolor{currentstroke}{rgb}{1.000000,0.498039,0.054902}%
\pgfsetstrokecolor{currentstroke}%
\pgfsetdash{}{0pt}%
\pgfpathmoveto{\pgfqpoint{2.200748in}{1.417750in}}%
\pgfpathcurveto{\pgfqpoint{2.211798in}{1.417750in}}{\pgfqpoint{2.222397in}{1.422140in}}{\pgfqpoint{2.230211in}{1.429954in}}%
\pgfpathcurveto{\pgfqpoint{2.238024in}{1.437768in}}{\pgfqpoint{2.242415in}{1.448367in}}{\pgfqpoint{2.242415in}{1.459417in}}%
\pgfpathcurveto{\pgfqpoint{2.242415in}{1.470467in}}{\pgfqpoint{2.238024in}{1.481066in}}{\pgfqpoint{2.230211in}{1.488880in}}%
\pgfpathcurveto{\pgfqpoint{2.222397in}{1.496693in}}{\pgfqpoint{2.211798in}{1.501083in}}{\pgfqpoint{2.200748in}{1.501083in}}%
\pgfpathcurveto{\pgfqpoint{2.189698in}{1.501083in}}{\pgfqpoint{2.179099in}{1.496693in}}{\pgfqpoint{2.171285in}{1.488880in}}%
\pgfpathcurveto{\pgfqpoint{2.163472in}{1.481066in}}{\pgfqpoint{2.159081in}{1.470467in}}{\pgfqpoint{2.159081in}{1.459417in}}%
\pgfpathcurveto{\pgfqpoint{2.159081in}{1.448367in}}{\pgfqpoint{2.163472in}{1.437768in}}{\pgfqpoint{2.171285in}{1.429954in}}%
\pgfpathcurveto{\pgfqpoint{2.179099in}{1.422140in}}{\pgfqpoint{2.189698in}{1.417750in}}{\pgfqpoint{2.200748in}{1.417750in}}%
\pgfpathclose%
\pgfusepath{stroke,fill}%
\end{pgfscope}%
\begin{pgfscope}%
\pgfpathrectangle{\pgfqpoint{0.787074in}{0.548769in}}{\pgfqpoint{5.062926in}{3.102590in}}%
\pgfusepath{clip}%
\pgfsetbuttcap%
\pgfsetroundjoin%
\definecolor{currentfill}{rgb}{1.000000,0.498039,0.054902}%
\pgfsetfillcolor{currentfill}%
\pgfsetlinewidth{1.003750pt}%
\definecolor{currentstroke}{rgb}{1.000000,0.498039,0.054902}%
\pgfsetstrokecolor{currentstroke}%
\pgfsetdash{}{0pt}%
\pgfpathmoveto{\pgfqpoint{2.069243in}{2.301800in}}%
\pgfpathcurveto{\pgfqpoint{2.080294in}{2.301800in}}{\pgfqpoint{2.090893in}{2.306191in}}{\pgfqpoint{2.098706in}{2.314004in}}%
\pgfpathcurveto{\pgfqpoint{2.106520in}{2.321818in}}{\pgfqpoint{2.110910in}{2.332417in}}{\pgfqpoint{2.110910in}{2.343467in}}%
\pgfpathcurveto{\pgfqpoint{2.110910in}{2.354517in}}{\pgfqpoint{2.106520in}{2.365116in}}{\pgfqpoint{2.098706in}{2.372930in}}%
\pgfpathcurveto{\pgfqpoint{2.090893in}{2.380743in}}{\pgfqpoint{2.080294in}{2.385134in}}{\pgfqpoint{2.069243in}{2.385134in}}%
\pgfpathcurveto{\pgfqpoint{2.058193in}{2.385134in}}{\pgfqpoint{2.047594in}{2.380743in}}{\pgfqpoint{2.039781in}{2.372930in}}%
\pgfpathcurveto{\pgfqpoint{2.031967in}{2.365116in}}{\pgfqpoint{2.027577in}{2.354517in}}{\pgfqpoint{2.027577in}{2.343467in}}%
\pgfpathcurveto{\pgfqpoint{2.027577in}{2.332417in}}{\pgfqpoint{2.031967in}{2.321818in}}{\pgfqpoint{2.039781in}{2.314004in}}%
\pgfpathcurveto{\pgfqpoint{2.047594in}{2.306191in}}{\pgfqpoint{2.058193in}{2.301800in}}{\pgfqpoint{2.069243in}{2.301800in}}%
\pgfpathclose%
\pgfusepath{stroke,fill}%
\end{pgfscope}%
\begin{pgfscope}%
\pgfpathrectangle{\pgfqpoint{0.787074in}{0.548769in}}{\pgfqpoint{5.062926in}{3.102590in}}%
\pgfusepath{clip}%
\pgfsetbuttcap%
\pgfsetroundjoin%
\definecolor{currentfill}{rgb}{0.121569,0.466667,0.705882}%
\pgfsetfillcolor{currentfill}%
\pgfsetlinewidth{1.003750pt}%
\definecolor{currentstroke}{rgb}{0.121569,0.466667,0.705882}%
\pgfsetstrokecolor{currentstroke}%
\pgfsetdash{}{0pt}%
\pgfpathmoveto{\pgfqpoint{4.765087in}{0.652202in}}%
\pgfpathcurveto{\pgfqpoint{4.776137in}{0.652202in}}{\pgfqpoint{4.786736in}{0.656593in}}{\pgfqpoint{4.794550in}{0.664406in}}%
\pgfpathcurveto{\pgfqpoint{4.802364in}{0.672220in}}{\pgfqpoint{4.806754in}{0.682819in}}{\pgfqpoint{4.806754in}{0.693869in}}%
\pgfpathcurveto{\pgfqpoint{4.806754in}{0.704919in}}{\pgfqpoint{4.802364in}{0.715518in}}{\pgfqpoint{4.794550in}{0.723332in}}%
\pgfpathcurveto{\pgfqpoint{4.786736in}{0.731145in}}{\pgfqpoint{4.776137in}{0.735536in}}{\pgfqpoint{4.765087in}{0.735536in}}%
\pgfpathcurveto{\pgfqpoint{4.754037in}{0.735536in}}{\pgfqpoint{4.743438in}{0.731145in}}{\pgfqpoint{4.735624in}{0.723332in}}%
\pgfpathcurveto{\pgfqpoint{4.727811in}{0.715518in}}{\pgfqpoint{4.723421in}{0.704919in}}{\pgfqpoint{4.723421in}{0.693869in}}%
\pgfpathcurveto{\pgfqpoint{4.723421in}{0.682819in}}{\pgfqpoint{4.727811in}{0.672220in}}{\pgfqpoint{4.735624in}{0.664406in}}%
\pgfpathcurveto{\pgfqpoint{4.743438in}{0.656593in}}{\pgfqpoint{4.754037in}{0.652202in}}{\pgfqpoint{4.765087in}{0.652202in}}%
\pgfpathclose%
\pgfusepath{stroke,fill}%
\end{pgfscope}%
\begin{pgfscope}%
\pgfpathrectangle{\pgfqpoint{0.787074in}{0.548769in}}{\pgfqpoint{5.062926in}{3.102590in}}%
\pgfusepath{clip}%
\pgfsetbuttcap%
\pgfsetroundjoin%
\definecolor{currentfill}{rgb}{0.121569,0.466667,0.705882}%
\pgfsetfillcolor{currentfill}%
\pgfsetlinewidth{1.003750pt}%
\definecolor{currentstroke}{rgb}{0.121569,0.466667,0.705882}%
\pgfsetstrokecolor{currentstroke}%
\pgfsetdash{}{0pt}%
\pgfpathmoveto{\pgfqpoint{1.608977in}{0.648180in}}%
\pgfpathcurveto{\pgfqpoint{1.620028in}{0.648180in}}{\pgfqpoint{1.630627in}{0.652571in}}{\pgfqpoint{1.638440in}{0.660384in}}%
\pgfpathcurveto{\pgfqpoint{1.646254in}{0.668198in}}{\pgfqpoint{1.650644in}{0.678797in}}{\pgfqpoint{1.650644in}{0.689847in}}%
\pgfpathcurveto{\pgfqpoint{1.650644in}{0.700897in}}{\pgfqpoint{1.646254in}{0.711496in}}{\pgfqpoint{1.638440in}{0.719310in}}%
\pgfpathcurveto{\pgfqpoint{1.630627in}{0.727123in}}{\pgfqpoint{1.620028in}{0.731514in}}{\pgfqpoint{1.608977in}{0.731514in}}%
\pgfpathcurveto{\pgfqpoint{1.597927in}{0.731514in}}{\pgfqpoint{1.587328in}{0.727123in}}{\pgfqpoint{1.579515in}{0.719310in}}%
\pgfpathcurveto{\pgfqpoint{1.571701in}{0.711496in}}{\pgfqpoint{1.567311in}{0.700897in}}{\pgfqpoint{1.567311in}{0.689847in}}%
\pgfpathcurveto{\pgfqpoint{1.567311in}{0.678797in}}{\pgfqpoint{1.571701in}{0.668198in}}{\pgfqpoint{1.579515in}{0.660384in}}%
\pgfpathcurveto{\pgfqpoint{1.587328in}{0.652571in}}{\pgfqpoint{1.597927in}{0.648180in}}{\pgfqpoint{1.608977in}{0.648180in}}%
\pgfpathclose%
\pgfusepath{stroke,fill}%
\end{pgfscope}%
\begin{pgfscope}%
\pgfpathrectangle{\pgfqpoint{0.787074in}{0.548769in}}{\pgfqpoint{5.062926in}{3.102590in}}%
\pgfusepath{clip}%
\pgfsetbuttcap%
\pgfsetroundjoin%
\definecolor{currentfill}{rgb}{0.121569,0.466667,0.705882}%
\pgfsetfillcolor{currentfill}%
\pgfsetlinewidth{1.003750pt}%
\definecolor{currentstroke}{rgb}{0.121569,0.466667,0.705882}%
\pgfsetstrokecolor{currentstroke}%
\pgfsetdash{}{0pt}%
\pgfpathmoveto{\pgfqpoint{1.477473in}{0.648153in}}%
\pgfpathcurveto{\pgfqpoint{1.488523in}{0.648153in}}{\pgfqpoint{1.499122in}{0.652543in}}{\pgfqpoint{1.506936in}{0.660357in}}%
\pgfpathcurveto{\pgfqpoint{1.514749in}{0.668170in}}{\pgfqpoint{1.519140in}{0.678769in}}{\pgfqpoint{1.519140in}{0.689819in}}%
\pgfpathcurveto{\pgfqpoint{1.519140in}{0.700870in}}{\pgfqpoint{1.514749in}{0.711469in}}{\pgfqpoint{1.506936in}{0.719282in}}%
\pgfpathcurveto{\pgfqpoint{1.499122in}{0.727096in}}{\pgfqpoint{1.488523in}{0.731486in}}{\pgfqpoint{1.477473in}{0.731486in}}%
\pgfpathcurveto{\pgfqpoint{1.466423in}{0.731486in}}{\pgfqpoint{1.455824in}{0.727096in}}{\pgfqpoint{1.448010in}{0.719282in}}%
\pgfpathcurveto{\pgfqpoint{1.440196in}{0.711469in}}{\pgfqpoint{1.435806in}{0.700870in}}{\pgfqpoint{1.435806in}{0.689819in}}%
\pgfpathcurveto{\pgfqpoint{1.435806in}{0.678769in}}{\pgfqpoint{1.440196in}{0.668170in}}{\pgfqpoint{1.448010in}{0.660357in}}%
\pgfpathcurveto{\pgfqpoint{1.455824in}{0.652543in}}{\pgfqpoint{1.466423in}{0.648153in}}{\pgfqpoint{1.477473in}{0.648153in}}%
\pgfpathclose%
\pgfusepath{stroke,fill}%
\end{pgfscope}%
\begin{pgfscope}%
\pgfpathrectangle{\pgfqpoint{0.787074in}{0.548769in}}{\pgfqpoint{5.062926in}{3.102590in}}%
\pgfusepath{clip}%
\pgfsetbuttcap%
\pgfsetroundjoin%
\definecolor{currentfill}{rgb}{0.121569,0.466667,0.705882}%
\pgfsetfillcolor{currentfill}%
\pgfsetlinewidth{1.003750pt}%
\definecolor{currentstroke}{rgb}{0.121569,0.466667,0.705882}%
\pgfsetstrokecolor{currentstroke}%
\pgfsetdash{}{0pt}%
\pgfpathmoveto{\pgfqpoint{1.411721in}{0.648456in}}%
\pgfpathcurveto{\pgfqpoint{1.422771in}{0.648456in}}{\pgfqpoint{1.433370in}{0.652846in}}{\pgfqpoint{1.441183in}{0.660660in}}%
\pgfpathcurveto{\pgfqpoint{1.448997in}{0.668474in}}{\pgfqpoint{1.453387in}{0.679073in}}{\pgfqpoint{1.453387in}{0.690123in}}%
\pgfpathcurveto{\pgfqpoint{1.453387in}{0.701173in}}{\pgfqpoint{1.448997in}{0.711772in}}{\pgfqpoint{1.441183in}{0.719586in}}%
\pgfpathcurveto{\pgfqpoint{1.433370in}{0.727399in}}{\pgfqpoint{1.422771in}{0.731789in}}{\pgfqpoint{1.411721in}{0.731789in}}%
\pgfpathcurveto{\pgfqpoint{1.400670in}{0.731789in}}{\pgfqpoint{1.390071in}{0.727399in}}{\pgfqpoint{1.382258in}{0.719586in}}%
\pgfpathcurveto{\pgfqpoint{1.374444in}{0.711772in}}{\pgfqpoint{1.370054in}{0.701173in}}{\pgfqpoint{1.370054in}{0.690123in}}%
\pgfpathcurveto{\pgfqpoint{1.370054in}{0.679073in}}{\pgfqpoint{1.374444in}{0.668474in}}{\pgfqpoint{1.382258in}{0.660660in}}%
\pgfpathcurveto{\pgfqpoint{1.390071in}{0.652846in}}{\pgfqpoint{1.400670in}{0.648456in}}{\pgfqpoint{1.411721in}{0.648456in}}%
\pgfpathclose%
\pgfusepath{stroke,fill}%
\end{pgfscope}%
\begin{pgfscope}%
\pgfpathrectangle{\pgfqpoint{0.787074in}{0.548769in}}{\pgfqpoint{5.062926in}{3.102590in}}%
\pgfusepath{clip}%
\pgfsetbuttcap%
\pgfsetroundjoin%
\definecolor{currentfill}{rgb}{0.121569,0.466667,0.705882}%
\pgfsetfillcolor{currentfill}%
\pgfsetlinewidth{1.003750pt}%
\definecolor{currentstroke}{rgb}{0.121569,0.466667,0.705882}%
\pgfsetstrokecolor{currentstroke}%
\pgfsetdash{}{0pt}%
\pgfpathmoveto{\pgfqpoint{1.411721in}{0.648200in}}%
\pgfpathcurveto{\pgfqpoint{1.422771in}{0.648200in}}{\pgfqpoint{1.433370in}{0.652590in}}{\pgfqpoint{1.441183in}{0.660404in}}%
\pgfpathcurveto{\pgfqpoint{1.448997in}{0.668217in}}{\pgfqpoint{1.453387in}{0.678816in}}{\pgfqpoint{1.453387in}{0.689866in}}%
\pgfpathcurveto{\pgfqpoint{1.453387in}{0.700917in}}{\pgfqpoint{1.448997in}{0.711516in}}{\pgfqpoint{1.441183in}{0.719329in}}%
\pgfpathcurveto{\pgfqpoint{1.433370in}{0.727143in}}{\pgfqpoint{1.422771in}{0.731533in}}{\pgfqpoint{1.411721in}{0.731533in}}%
\pgfpathcurveto{\pgfqpoint{1.400670in}{0.731533in}}{\pgfqpoint{1.390071in}{0.727143in}}{\pgfqpoint{1.382258in}{0.719329in}}%
\pgfpathcurveto{\pgfqpoint{1.374444in}{0.711516in}}{\pgfqpoint{1.370054in}{0.700917in}}{\pgfqpoint{1.370054in}{0.689866in}}%
\pgfpathcurveto{\pgfqpoint{1.370054in}{0.678816in}}{\pgfqpoint{1.374444in}{0.668217in}}{\pgfqpoint{1.382258in}{0.660404in}}%
\pgfpathcurveto{\pgfqpoint{1.390071in}{0.652590in}}{\pgfqpoint{1.400670in}{0.648200in}}{\pgfqpoint{1.411721in}{0.648200in}}%
\pgfpathclose%
\pgfusepath{stroke,fill}%
\end{pgfscope}%
\begin{pgfscope}%
\pgfpathrectangle{\pgfqpoint{0.787074in}{0.548769in}}{\pgfqpoint{5.062926in}{3.102590in}}%
\pgfusepath{clip}%
\pgfsetbuttcap%
\pgfsetroundjoin%
\definecolor{currentfill}{rgb}{0.121569,0.466667,0.705882}%
\pgfsetfillcolor{currentfill}%
\pgfsetlinewidth{1.003750pt}%
\definecolor{currentstroke}{rgb}{0.121569,0.466667,0.705882}%
\pgfsetstrokecolor{currentstroke}%
\pgfsetdash{}{0pt}%
\pgfpathmoveto{\pgfqpoint{1.543225in}{0.648150in}}%
\pgfpathcurveto{\pgfqpoint{1.554275in}{0.648150in}}{\pgfqpoint{1.564874in}{0.652540in}}{\pgfqpoint{1.572688in}{0.660354in}}%
\pgfpathcurveto{\pgfqpoint{1.580502in}{0.668167in}}{\pgfqpoint{1.584892in}{0.678766in}}{\pgfqpoint{1.584892in}{0.689816in}}%
\pgfpathcurveto{\pgfqpoint{1.584892in}{0.700866in}}{\pgfqpoint{1.580502in}{0.711465in}}{\pgfqpoint{1.572688in}{0.719279in}}%
\pgfpathcurveto{\pgfqpoint{1.564874in}{0.727093in}}{\pgfqpoint{1.554275in}{0.731483in}}{\pgfqpoint{1.543225in}{0.731483in}}%
\pgfpathcurveto{\pgfqpoint{1.532175in}{0.731483in}}{\pgfqpoint{1.521576in}{0.727093in}}{\pgfqpoint{1.513762in}{0.719279in}}%
\pgfpathcurveto{\pgfqpoint{1.505949in}{0.711465in}}{\pgfqpoint{1.501558in}{0.700866in}}{\pgfqpoint{1.501558in}{0.689816in}}%
\pgfpathcurveto{\pgfqpoint{1.501558in}{0.678766in}}{\pgfqpoint{1.505949in}{0.668167in}}{\pgfqpoint{1.513762in}{0.660354in}}%
\pgfpathcurveto{\pgfqpoint{1.521576in}{0.652540in}}{\pgfqpoint{1.532175in}{0.648150in}}{\pgfqpoint{1.543225in}{0.648150in}}%
\pgfpathclose%
\pgfusepath{stroke,fill}%
\end{pgfscope}%
\begin{pgfscope}%
\pgfpathrectangle{\pgfqpoint{0.787074in}{0.548769in}}{\pgfqpoint{5.062926in}{3.102590in}}%
\pgfusepath{clip}%
\pgfsetbuttcap%
\pgfsetroundjoin%
\definecolor{currentfill}{rgb}{1.000000,0.498039,0.054902}%
\pgfsetfillcolor{currentfill}%
\pgfsetlinewidth{1.003750pt}%
\definecolor{currentstroke}{rgb}{1.000000,0.498039,0.054902}%
\pgfsetstrokecolor{currentstroke}%
\pgfsetdash{}{0pt}%
\pgfpathmoveto{\pgfqpoint{2.463757in}{3.030917in}}%
\pgfpathcurveto{\pgfqpoint{2.474807in}{3.030917in}}{\pgfqpoint{2.485406in}{3.035308in}}{\pgfqpoint{2.493220in}{3.043121in}}%
\pgfpathcurveto{\pgfqpoint{2.501034in}{3.050935in}}{\pgfqpoint{2.505424in}{3.061534in}}{\pgfqpoint{2.505424in}{3.072584in}}%
\pgfpathcurveto{\pgfqpoint{2.505424in}{3.083634in}}{\pgfqpoint{2.501034in}{3.094233in}}{\pgfqpoint{2.493220in}{3.102047in}}%
\pgfpathcurveto{\pgfqpoint{2.485406in}{3.109861in}}{\pgfqpoint{2.474807in}{3.114251in}}{\pgfqpoint{2.463757in}{3.114251in}}%
\pgfpathcurveto{\pgfqpoint{2.452707in}{3.114251in}}{\pgfqpoint{2.442108in}{3.109861in}}{\pgfqpoint{2.434294in}{3.102047in}}%
\pgfpathcurveto{\pgfqpoint{2.426481in}{3.094233in}}{\pgfqpoint{2.422091in}{3.083634in}}{\pgfqpoint{2.422091in}{3.072584in}}%
\pgfpathcurveto{\pgfqpoint{2.422091in}{3.061534in}}{\pgfqpoint{2.426481in}{3.050935in}}{\pgfqpoint{2.434294in}{3.043121in}}%
\pgfpathcurveto{\pgfqpoint{2.442108in}{3.035308in}}{\pgfqpoint{2.452707in}{3.030917in}}{\pgfqpoint{2.463757in}{3.030917in}}%
\pgfpathclose%
\pgfusepath{stroke,fill}%
\end{pgfscope}%
\begin{pgfscope}%
\pgfpathrectangle{\pgfqpoint{0.787074in}{0.548769in}}{\pgfqpoint{5.062926in}{3.102590in}}%
\pgfusepath{clip}%
\pgfsetbuttcap%
\pgfsetroundjoin%
\definecolor{currentfill}{rgb}{1.000000,0.498039,0.054902}%
\pgfsetfillcolor{currentfill}%
\pgfsetlinewidth{1.003750pt}%
\definecolor{currentstroke}{rgb}{1.000000,0.498039,0.054902}%
\pgfsetstrokecolor{currentstroke}%
\pgfsetdash{}{0pt}%
\pgfpathmoveto{\pgfqpoint{2.398005in}{2.938746in}}%
\pgfpathcurveto{\pgfqpoint{2.409055in}{2.938746in}}{\pgfqpoint{2.419654in}{2.943136in}}{\pgfqpoint{2.427468in}{2.950950in}}%
\pgfpathcurveto{\pgfqpoint{2.435281in}{2.958764in}}{\pgfqpoint{2.439672in}{2.969363in}}{\pgfqpoint{2.439672in}{2.980413in}}%
\pgfpathcurveto{\pgfqpoint{2.439672in}{2.991463in}}{\pgfqpoint{2.435281in}{3.002062in}}{\pgfqpoint{2.427468in}{3.009876in}}%
\pgfpathcurveto{\pgfqpoint{2.419654in}{3.017689in}}{\pgfqpoint{2.409055in}{3.022079in}}{\pgfqpoint{2.398005in}{3.022079in}}%
\pgfpathcurveto{\pgfqpoint{2.386955in}{3.022079in}}{\pgfqpoint{2.376356in}{3.017689in}}{\pgfqpoint{2.368542in}{3.009876in}}%
\pgfpathcurveto{\pgfqpoint{2.360728in}{3.002062in}}{\pgfqpoint{2.356338in}{2.991463in}}{\pgfqpoint{2.356338in}{2.980413in}}%
\pgfpathcurveto{\pgfqpoint{2.356338in}{2.969363in}}{\pgfqpoint{2.360728in}{2.958764in}}{\pgfqpoint{2.368542in}{2.950950in}}%
\pgfpathcurveto{\pgfqpoint{2.376356in}{2.943136in}}{\pgfqpoint{2.386955in}{2.938746in}}{\pgfqpoint{2.398005in}{2.938746in}}%
\pgfpathclose%
\pgfusepath{stroke,fill}%
\end{pgfscope}%
\begin{pgfscope}%
\pgfpathrectangle{\pgfqpoint{0.787074in}{0.548769in}}{\pgfqpoint{5.062926in}{3.102590in}}%
\pgfusepath{clip}%
\pgfsetbuttcap%
\pgfsetroundjoin%
\definecolor{currentfill}{rgb}{1.000000,0.498039,0.054902}%
\pgfsetfillcolor{currentfill}%
\pgfsetlinewidth{1.003750pt}%
\definecolor{currentstroke}{rgb}{1.000000,0.498039,0.054902}%
\pgfsetstrokecolor{currentstroke}%
\pgfsetdash{}{0pt}%
\pgfpathmoveto{\pgfqpoint{2.266500in}{2.706441in}}%
\pgfpathcurveto{\pgfqpoint{2.277550in}{2.706441in}}{\pgfqpoint{2.288149in}{2.710831in}}{\pgfqpoint{2.295963in}{2.718645in}}%
\pgfpathcurveto{\pgfqpoint{2.303777in}{2.726459in}}{\pgfqpoint{2.308167in}{2.737058in}}{\pgfqpoint{2.308167in}{2.748108in}}%
\pgfpathcurveto{\pgfqpoint{2.308167in}{2.759158in}}{\pgfqpoint{2.303777in}{2.769757in}}{\pgfqpoint{2.295963in}{2.777571in}}%
\pgfpathcurveto{\pgfqpoint{2.288149in}{2.785384in}}{\pgfqpoint{2.277550in}{2.789774in}}{\pgfqpoint{2.266500in}{2.789774in}}%
\pgfpathcurveto{\pgfqpoint{2.255450in}{2.789774in}}{\pgfqpoint{2.244851in}{2.785384in}}{\pgfqpoint{2.237038in}{2.777571in}}%
\pgfpathcurveto{\pgfqpoint{2.229224in}{2.769757in}}{\pgfqpoint{2.224834in}{2.759158in}}{\pgfqpoint{2.224834in}{2.748108in}}%
\pgfpathcurveto{\pgfqpoint{2.224834in}{2.737058in}}{\pgfqpoint{2.229224in}{2.726459in}}{\pgfqpoint{2.237038in}{2.718645in}}%
\pgfpathcurveto{\pgfqpoint{2.244851in}{2.710831in}}{\pgfqpoint{2.255450in}{2.706441in}}{\pgfqpoint{2.266500in}{2.706441in}}%
\pgfpathclose%
\pgfusepath{stroke,fill}%
\end{pgfscope}%
\begin{pgfscope}%
\pgfpathrectangle{\pgfqpoint{0.787074in}{0.548769in}}{\pgfqpoint{5.062926in}{3.102590in}}%
\pgfusepath{clip}%
\pgfsetbuttcap%
\pgfsetroundjoin%
\definecolor{currentfill}{rgb}{0.121569,0.466667,0.705882}%
\pgfsetfillcolor{currentfill}%
\pgfsetlinewidth{1.003750pt}%
\definecolor{currentstroke}{rgb}{0.121569,0.466667,0.705882}%
\pgfsetstrokecolor{currentstroke}%
\pgfsetdash{}{0pt}%
\pgfpathmoveto{\pgfqpoint{1.740482in}{0.648133in}}%
\pgfpathcurveto{\pgfqpoint{1.751532in}{0.648133in}}{\pgfqpoint{1.762131in}{0.652523in}}{\pgfqpoint{1.769945in}{0.660337in}}%
\pgfpathcurveto{\pgfqpoint{1.777758in}{0.668150in}}{\pgfqpoint{1.782149in}{0.678749in}}{\pgfqpoint{1.782149in}{0.689799in}}%
\pgfpathcurveto{\pgfqpoint{1.782149in}{0.700849in}}{\pgfqpoint{1.777758in}{0.711448in}}{\pgfqpoint{1.769945in}{0.719262in}}%
\pgfpathcurveto{\pgfqpoint{1.762131in}{0.727076in}}{\pgfqpoint{1.751532in}{0.731466in}}{\pgfqpoint{1.740482in}{0.731466in}}%
\pgfpathcurveto{\pgfqpoint{1.729432in}{0.731466in}}{\pgfqpoint{1.718833in}{0.727076in}}{\pgfqpoint{1.711019in}{0.719262in}}%
\pgfpathcurveto{\pgfqpoint{1.703206in}{0.711448in}}{\pgfqpoint{1.698815in}{0.700849in}}{\pgfqpoint{1.698815in}{0.689799in}}%
\pgfpathcurveto{\pgfqpoint{1.698815in}{0.678749in}}{\pgfqpoint{1.703206in}{0.668150in}}{\pgfqpoint{1.711019in}{0.660337in}}%
\pgfpathcurveto{\pgfqpoint{1.718833in}{0.652523in}}{\pgfqpoint{1.729432in}{0.648133in}}{\pgfqpoint{1.740482in}{0.648133in}}%
\pgfpathclose%
\pgfusepath{stroke,fill}%
\end{pgfscope}%
\begin{pgfscope}%
\pgfpathrectangle{\pgfqpoint{0.787074in}{0.548769in}}{\pgfqpoint{5.062926in}{3.102590in}}%
\pgfusepath{clip}%
\pgfsetbuttcap%
\pgfsetroundjoin%
\definecolor{currentfill}{rgb}{0.121569,0.466667,0.705882}%
\pgfsetfillcolor{currentfill}%
\pgfsetlinewidth{1.003750pt}%
\definecolor{currentstroke}{rgb}{0.121569,0.466667,0.705882}%
\pgfsetstrokecolor{currentstroke}%
\pgfsetdash{}{0pt}%
\pgfpathmoveto{\pgfqpoint{1.543225in}{0.678236in}}%
\pgfpathcurveto{\pgfqpoint{1.554275in}{0.678236in}}{\pgfqpoint{1.564874in}{0.682626in}}{\pgfqpoint{1.572688in}{0.690439in}}%
\pgfpathcurveto{\pgfqpoint{1.580502in}{0.698253in}}{\pgfqpoint{1.584892in}{0.708852in}}{\pgfqpoint{1.584892in}{0.719902in}}%
\pgfpathcurveto{\pgfqpoint{1.584892in}{0.730952in}}{\pgfqpoint{1.580502in}{0.741551in}}{\pgfqpoint{1.572688in}{0.749365in}}%
\pgfpathcurveto{\pgfqpoint{1.564874in}{0.757179in}}{\pgfqpoint{1.554275in}{0.761569in}}{\pgfqpoint{1.543225in}{0.761569in}}%
\pgfpathcurveto{\pgfqpoint{1.532175in}{0.761569in}}{\pgfqpoint{1.521576in}{0.757179in}}{\pgfqpoint{1.513762in}{0.749365in}}%
\pgfpathcurveto{\pgfqpoint{1.505949in}{0.741551in}}{\pgfqpoint{1.501558in}{0.730952in}}{\pgfqpoint{1.501558in}{0.719902in}}%
\pgfpathcurveto{\pgfqpoint{1.501558in}{0.708852in}}{\pgfqpoint{1.505949in}{0.698253in}}{\pgfqpoint{1.513762in}{0.690439in}}%
\pgfpathcurveto{\pgfqpoint{1.521576in}{0.682626in}}{\pgfqpoint{1.532175in}{0.678236in}}{\pgfqpoint{1.543225in}{0.678236in}}%
\pgfpathclose%
\pgfusepath{stroke,fill}%
\end{pgfscope}%
\begin{pgfscope}%
\pgfpathrectangle{\pgfqpoint{0.787074in}{0.548769in}}{\pgfqpoint{5.062926in}{3.102590in}}%
\pgfusepath{clip}%
\pgfsetbuttcap%
\pgfsetroundjoin%
\definecolor{currentfill}{rgb}{1.000000,0.498039,0.054902}%
\pgfsetfillcolor{currentfill}%
\pgfsetlinewidth{1.003750pt}%
\definecolor{currentstroke}{rgb}{1.000000,0.498039,0.054902}%
\pgfsetstrokecolor{currentstroke}%
\pgfsetdash{}{0pt}%
\pgfpathmoveto{\pgfqpoint{1.608977in}{3.011191in}}%
\pgfpathcurveto{\pgfqpoint{1.620028in}{3.011191in}}{\pgfqpoint{1.630627in}{3.015581in}}{\pgfqpoint{1.638440in}{3.023394in}}%
\pgfpathcurveto{\pgfqpoint{1.646254in}{3.031208in}}{\pgfqpoint{1.650644in}{3.041807in}}{\pgfqpoint{1.650644in}{3.052857in}}%
\pgfpathcurveto{\pgfqpoint{1.650644in}{3.063907in}}{\pgfqpoint{1.646254in}{3.074506in}}{\pgfqpoint{1.638440in}{3.082320in}}%
\pgfpathcurveto{\pgfqpoint{1.630627in}{3.090134in}}{\pgfqpoint{1.620028in}{3.094524in}}{\pgfqpoint{1.608977in}{3.094524in}}%
\pgfpathcurveto{\pgfqpoint{1.597927in}{3.094524in}}{\pgfqpoint{1.587328in}{3.090134in}}{\pgfqpoint{1.579515in}{3.082320in}}%
\pgfpathcurveto{\pgfqpoint{1.571701in}{3.074506in}}{\pgfqpoint{1.567311in}{3.063907in}}{\pgfqpoint{1.567311in}{3.052857in}}%
\pgfpathcurveto{\pgfqpoint{1.567311in}{3.041807in}}{\pgfqpoint{1.571701in}{3.031208in}}{\pgfqpoint{1.579515in}{3.023394in}}%
\pgfpathcurveto{\pgfqpoint{1.587328in}{3.015581in}}{\pgfqpoint{1.597927in}{3.011191in}}{\pgfqpoint{1.608977in}{3.011191in}}%
\pgfpathclose%
\pgfusepath{stroke,fill}%
\end{pgfscope}%
\begin{pgfscope}%
\pgfpathrectangle{\pgfqpoint{0.787074in}{0.548769in}}{\pgfqpoint{5.062926in}{3.102590in}}%
\pgfusepath{clip}%
\pgfsetbuttcap%
\pgfsetroundjoin%
\definecolor{currentfill}{rgb}{0.121569,0.466667,0.705882}%
\pgfsetfillcolor{currentfill}%
\pgfsetlinewidth{1.003750pt}%
\definecolor{currentstroke}{rgb}{0.121569,0.466667,0.705882}%
\pgfsetstrokecolor{currentstroke}%
\pgfsetdash{}{0pt}%
\pgfpathmoveto{\pgfqpoint{1.280216in}{0.787280in}}%
\pgfpathcurveto{\pgfqpoint{1.291266in}{0.787280in}}{\pgfqpoint{1.301865in}{0.791670in}}{\pgfqpoint{1.309679in}{0.799484in}}%
\pgfpathcurveto{\pgfqpoint{1.317492in}{0.807297in}}{\pgfqpoint{1.321883in}{0.817896in}}{\pgfqpoint{1.321883in}{0.828946in}}%
\pgfpathcurveto{\pgfqpoint{1.321883in}{0.839997in}}{\pgfqpoint{1.317492in}{0.850596in}}{\pgfqpoint{1.309679in}{0.858409in}}%
\pgfpathcurveto{\pgfqpoint{1.301865in}{0.866223in}}{\pgfqpoint{1.291266in}{0.870613in}}{\pgfqpoint{1.280216in}{0.870613in}}%
\pgfpathcurveto{\pgfqpoint{1.269166in}{0.870613in}}{\pgfqpoint{1.258567in}{0.866223in}}{\pgfqpoint{1.250753in}{0.858409in}}%
\pgfpathcurveto{\pgfqpoint{1.242940in}{0.850596in}}{\pgfqpoint{1.238549in}{0.839997in}}{\pgfqpoint{1.238549in}{0.828946in}}%
\pgfpathcurveto{\pgfqpoint{1.238549in}{0.817896in}}{\pgfqpoint{1.242940in}{0.807297in}}{\pgfqpoint{1.250753in}{0.799484in}}%
\pgfpathcurveto{\pgfqpoint{1.258567in}{0.791670in}}{\pgfqpoint{1.269166in}{0.787280in}}{\pgfqpoint{1.280216in}{0.787280in}}%
\pgfpathclose%
\pgfusepath{stroke,fill}%
\end{pgfscope}%
\begin{pgfscope}%
\pgfpathrectangle{\pgfqpoint{0.787074in}{0.548769in}}{\pgfqpoint{5.062926in}{3.102590in}}%
\pgfusepath{clip}%
\pgfsetbuttcap%
\pgfsetroundjoin%
\definecolor{currentfill}{rgb}{1.000000,0.498039,0.054902}%
\pgfsetfillcolor{currentfill}%
\pgfsetlinewidth{1.003750pt}%
\definecolor{currentstroke}{rgb}{1.000000,0.498039,0.054902}%
\pgfsetstrokecolor{currentstroke}%
\pgfsetdash{}{0pt}%
\pgfpathmoveto{\pgfqpoint{2.003491in}{3.468665in}}%
\pgfpathcurveto{\pgfqpoint{2.014541in}{3.468665in}}{\pgfqpoint{2.025140in}{3.473055in}}{\pgfqpoint{2.032954in}{3.480869in}}%
\pgfpathcurveto{\pgfqpoint{2.040768in}{3.488683in}}{\pgfqpoint{2.045158in}{3.499282in}}{\pgfqpoint{2.045158in}{3.510332in}}%
\pgfpathcurveto{\pgfqpoint{2.045158in}{3.521382in}}{\pgfqpoint{2.040768in}{3.531981in}}{\pgfqpoint{2.032954in}{3.539795in}}%
\pgfpathcurveto{\pgfqpoint{2.025140in}{3.547608in}}{\pgfqpoint{2.014541in}{3.551998in}}{\pgfqpoint{2.003491in}{3.551998in}}%
\pgfpathcurveto{\pgfqpoint{1.992441in}{3.551998in}}{\pgfqpoint{1.981842in}{3.547608in}}{\pgfqpoint{1.974028in}{3.539795in}}%
\pgfpathcurveto{\pgfqpoint{1.966215in}{3.531981in}}{\pgfqpoint{1.961824in}{3.521382in}}{\pgfqpoint{1.961824in}{3.510332in}}%
\pgfpathcurveto{\pgfqpoint{1.961824in}{3.499282in}}{\pgfqpoint{1.966215in}{3.488683in}}{\pgfqpoint{1.974028in}{3.480869in}}%
\pgfpathcurveto{\pgfqpoint{1.981842in}{3.473055in}}{\pgfqpoint{1.992441in}{3.468665in}}{\pgfqpoint{2.003491in}{3.468665in}}%
\pgfpathclose%
\pgfusepath{stroke,fill}%
\end{pgfscope}%
\begin{pgfscope}%
\pgfpathrectangle{\pgfqpoint{0.787074in}{0.548769in}}{\pgfqpoint{5.062926in}{3.102590in}}%
\pgfusepath{clip}%
\pgfsetbuttcap%
\pgfsetroundjoin%
\definecolor{currentfill}{rgb}{0.121569,0.466667,0.705882}%
\pgfsetfillcolor{currentfill}%
\pgfsetlinewidth{1.003750pt}%
\definecolor{currentstroke}{rgb}{0.121569,0.466667,0.705882}%
\pgfsetstrokecolor{currentstroke}%
\pgfsetdash{}{0pt}%
\pgfpathmoveto{\pgfqpoint{1.345968in}{0.649989in}}%
\pgfpathcurveto{\pgfqpoint{1.357018in}{0.649989in}}{\pgfqpoint{1.367617in}{0.654379in}}{\pgfqpoint{1.375431in}{0.662193in}}%
\pgfpathcurveto{\pgfqpoint{1.383245in}{0.670006in}}{\pgfqpoint{1.387635in}{0.680605in}}{\pgfqpoint{1.387635in}{0.691655in}}%
\pgfpathcurveto{\pgfqpoint{1.387635in}{0.702706in}}{\pgfqpoint{1.383245in}{0.713305in}}{\pgfqpoint{1.375431in}{0.721118in}}%
\pgfpathcurveto{\pgfqpoint{1.367617in}{0.728932in}}{\pgfqpoint{1.357018in}{0.733322in}}{\pgfqpoint{1.345968in}{0.733322in}}%
\pgfpathcurveto{\pgfqpoint{1.334918in}{0.733322in}}{\pgfqpoint{1.324319in}{0.728932in}}{\pgfqpoint{1.316506in}{0.721118in}}%
\pgfpathcurveto{\pgfqpoint{1.308692in}{0.713305in}}{\pgfqpoint{1.304302in}{0.702706in}}{\pgfqpoint{1.304302in}{0.691655in}}%
\pgfpathcurveto{\pgfqpoint{1.304302in}{0.680605in}}{\pgfqpoint{1.308692in}{0.670006in}}{\pgfqpoint{1.316506in}{0.662193in}}%
\pgfpathcurveto{\pgfqpoint{1.324319in}{0.654379in}}{\pgfqpoint{1.334918in}{0.649989in}}{\pgfqpoint{1.345968in}{0.649989in}}%
\pgfpathclose%
\pgfusepath{stroke,fill}%
\end{pgfscope}%
\begin{pgfscope}%
\pgfpathrectangle{\pgfqpoint{0.787074in}{0.548769in}}{\pgfqpoint{5.062926in}{3.102590in}}%
\pgfusepath{clip}%
\pgfsetbuttcap%
\pgfsetroundjoin%
\definecolor{currentfill}{rgb}{1.000000,0.498039,0.054902}%
\pgfsetfillcolor{currentfill}%
\pgfsetlinewidth{1.003750pt}%
\definecolor{currentstroke}{rgb}{1.000000,0.498039,0.054902}%
\pgfsetstrokecolor{currentstroke}%
\pgfsetdash{}{0pt}%
\pgfpathmoveto{\pgfqpoint{1.740482in}{2.212517in}}%
\pgfpathcurveto{\pgfqpoint{1.751532in}{2.212517in}}{\pgfqpoint{1.762131in}{2.216907in}}{\pgfqpoint{1.769945in}{2.224721in}}%
\pgfpathcurveto{\pgfqpoint{1.777758in}{2.232535in}}{\pgfqpoint{1.782149in}{2.243134in}}{\pgfqpoint{1.782149in}{2.254184in}}%
\pgfpathcurveto{\pgfqpoint{1.782149in}{2.265234in}}{\pgfqpoint{1.777758in}{2.275833in}}{\pgfqpoint{1.769945in}{2.283647in}}%
\pgfpathcurveto{\pgfqpoint{1.762131in}{2.291460in}}{\pgfqpoint{1.751532in}{2.295850in}}{\pgfqpoint{1.740482in}{2.295850in}}%
\pgfpathcurveto{\pgfqpoint{1.729432in}{2.295850in}}{\pgfqpoint{1.718833in}{2.291460in}}{\pgfqpoint{1.711019in}{2.283647in}}%
\pgfpathcurveto{\pgfqpoint{1.703206in}{2.275833in}}{\pgfqpoint{1.698815in}{2.265234in}}{\pgfqpoint{1.698815in}{2.254184in}}%
\pgfpathcurveto{\pgfqpoint{1.698815in}{2.243134in}}{\pgfqpoint{1.703206in}{2.232535in}}{\pgfqpoint{1.711019in}{2.224721in}}%
\pgfpathcurveto{\pgfqpoint{1.718833in}{2.216907in}}{\pgfqpoint{1.729432in}{2.212517in}}{\pgfqpoint{1.740482in}{2.212517in}}%
\pgfpathclose%
\pgfusepath{stroke,fill}%
\end{pgfscope}%
\begin{pgfscope}%
\pgfpathrectangle{\pgfqpoint{0.787074in}{0.548769in}}{\pgfqpoint{5.062926in}{3.102590in}}%
\pgfusepath{clip}%
\pgfsetbuttcap%
\pgfsetroundjoin%
\definecolor{currentfill}{rgb}{0.121569,0.466667,0.705882}%
\pgfsetfillcolor{currentfill}%
\pgfsetlinewidth{1.003750pt}%
\definecolor{currentstroke}{rgb}{0.121569,0.466667,0.705882}%
\pgfsetstrokecolor{currentstroke}%
\pgfsetdash{}{0pt}%
\pgfpathmoveto{\pgfqpoint{1.740482in}{0.651678in}}%
\pgfpathcurveto{\pgfqpoint{1.751532in}{0.651678in}}{\pgfqpoint{1.762131in}{0.656068in}}{\pgfqpoint{1.769945in}{0.663882in}}%
\pgfpathcurveto{\pgfqpoint{1.777758in}{0.671695in}}{\pgfqpoint{1.782149in}{0.682295in}}{\pgfqpoint{1.782149in}{0.693345in}}%
\pgfpathcurveto{\pgfqpoint{1.782149in}{0.704395in}}{\pgfqpoint{1.777758in}{0.714994in}}{\pgfqpoint{1.769945in}{0.722807in}}%
\pgfpathcurveto{\pgfqpoint{1.762131in}{0.730621in}}{\pgfqpoint{1.751532in}{0.735011in}}{\pgfqpoint{1.740482in}{0.735011in}}%
\pgfpathcurveto{\pgfqpoint{1.729432in}{0.735011in}}{\pgfqpoint{1.718833in}{0.730621in}}{\pgfqpoint{1.711019in}{0.722807in}}%
\pgfpathcurveto{\pgfqpoint{1.703206in}{0.714994in}}{\pgfqpoint{1.698815in}{0.704395in}}{\pgfqpoint{1.698815in}{0.693345in}}%
\pgfpathcurveto{\pgfqpoint{1.698815in}{0.682295in}}{\pgfqpoint{1.703206in}{0.671695in}}{\pgfqpoint{1.711019in}{0.663882in}}%
\pgfpathcurveto{\pgfqpoint{1.718833in}{0.656068in}}{\pgfqpoint{1.729432in}{0.651678in}}{\pgfqpoint{1.740482in}{0.651678in}}%
\pgfpathclose%
\pgfusepath{stroke,fill}%
\end{pgfscope}%
\begin{pgfscope}%
\pgfpathrectangle{\pgfqpoint{0.787074in}{0.548769in}}{\pgfqpoint{5.062926in}{3.102590in}}%
\pgfusepath{clip}%
\pgfsetbuttcap%
\pgfsetroundjoin%
\definecolor{currentfill}{rgb}{1.000000,0.498039,0.054902}%
\pgfsetfillcolor{currentfill}%
\pgfsetlinewidth{1.003750pt}%
\definecolor{currentstroke}{rgb}{1.000000,0.498039,0.054902}%
\pgfsetstrokecolor{currentstroke}%
\pgfsetdash{}{0pt}%
\pgfpathmoveto{\pgfqpoint{1.740482in}{2.584775in}}%
\pgfpathcurveto{\pgfqpoint{1.751532in}{2.584775in}}{\pgfqpoint{1.762131in}{2.589166in}}{\pgfqpoint{1.769945in}{2.596979in}}%
\pgfpathcurveto{\pgfqpoint{1.777758in}{2.604793in}}{\pgfqpoint{1.782149in}{2.615392in}}{\pgfqpoint{1.782149in}{2.626442in}}%
\pgfpathcurveto{\pgfqpoint{1.782149in}{2.637492in}}{\pgfqpoint{1.777758in}{2.648091in}}{\pgfqpoint{1.769945in}{2.655905in}}%
\pgfpathcurveto{\pgfqpoint{1.762131in}{2.663718in}}{\pgfqpoint{1.751532in}{2.668109in}}{\pgfqpoint{1.740482in}{2.668109in}}%
\pgfpathcurveto{\pgfqpoint{1.729432in}{2.668109in}}{\pgfqpoint{1.718833in}{2.663718in}}{\pgfqpoint{1.711019in}{2.655905in}}%
\pgfpathcurveto{\pgfqpoint{1.703206in}{2.648091in}}{\pgfqpoint{1.698815in}{2.637492in}}{\pgfqpoint{1.698815in}{2.626442in}}%
\pgfpathcurveto{\pgfqpoint{1.698815in}{2.615392in}}{\pgfqpoint{1.703206in}{2.604793in}}{\pgfqpoint{1.711019in}{2.596979in}}%
\pgfpathcurveto{\pgfqpoint{1.718833in}{2.589166in}}{\pgfqpoint{1.729432in}{2.584775in}}{\pgfqpoint{1.740482in}{2.584775in}}%
\pgfpathclose%
\pgfusepath{stroke,fill}%
\end{pgfscope}%
\begin{pgfscope}%
\pgfpathrectangle{\pgfqpoint{0.787074in}{0.548769in}}{\pgfqpoint{5.062926in}{3.102590in}}%
\pgfusepath{clip}%
\pgfsetbuttcap%
\pgfsetroundjoin%
\definecolor{currentfill}{rgb}{0.121569,0.466667,0.705882}%
\pgfsetfillcolor{currentfill}%
\pgfsetlinewidth{1.003750pt}%
\definecolor{currentstroke}{rgb}{0.121569,0.466667,0.705882}%
\pgfsetstrokecolor{currentstroke}%
\pgfsetdash{}{0pt}%
\pgfpathmoveto{\pgfqpoint{1.674730in}{0.648158in}}%
\pgfpathcurveto{\pgfqpoint{1.685780in}{0.648158in}}{\pgfqpoint{1.696379in}{0.652548in}}{\pgfqpoint{1.704193in}{0.660362in}}%
\pgfpathcurveto{\pgfqpoint{1.712006in}{0.668176in}}{\pgfqpoint{1.716396in}{0.678775in}}{\pgfqpoint{1.716396in}{0.689825in}}%
\pgfpathcurveto{\pgfqpoint{1.716396in}{0.700875in}}{\pgfqpoint{1.712006in}{0.711474in}}{\pgfqpoint{1.704193in}{0.719288in}}%
\pgfpathcurveto{\pgfqpoint{1.696379in}{0.727101in}}{\pgfqpoint{1.685780in}{0.731492in}}{\pgfqpoint{1.674730in}{0.731492in}}%
\pgfpathcurveto{\pgfqpoint{1.663680in}{0.731492in}}{\pgfqpoint{1.653081in}{0.727101in}}{\pgfqpoint{1.645267in}{0.719288in}}%
\pgfpathcurveto{\pgfqpoint{1.637453in}{0.711474in}}{\pgfqpoint{1.633063in}{0.700875in}}{\pgfqpoint{1.633063in}{0.689825in}}%
\pgfpathcurveto{\pgfqpoint{1.633063in}{0.678775in}}{\pgfqpoint{1.637453in}{0.668176in}}{\pgfqpoint{1.645267in}{0.660362in}}%
\pgfpathcurveto{\pgfqpoint{1.653081in}{0.652548in}}{\pgfqpoint{1.663680in}{0.648158in}}{\pgfqpoint{1.674730in}{0.648158in}}%
\pgfpathclose%
\pgfusepath{stroke,fill}%
\end{pgfscope}%
\begin{pgfscope}%
\pgfpathrectangle{\pgfqpoint{0.787074in}{0.548769in}}{\pgfqpoint{5.062926in}{3.102590in}}%
\pgfusepath{clip}%
\pgfsetbuttcap%
\pgfsetroundjoin%
\definecolor{currentfill}{rgb}{1.000000,0.498039,0.054902}%
\pgfsetfillcolor{currentfill}%
\pgfsetlinewidth{1.003750pt}%
\definecolor{currentstroke}{rgb}{1.000000,0.498039,0.054902}%
\pgfsetstrokecolor{currentstroke}%
\pgfsetdash{}{0pt}%
\pgfpathmoveto{\pgfqpoint{1.937739in}{2.981281in}}%
\pgfpathcurveto{\pgfqpoint{1.948789in}{2.981281in}}{\pgfqpoint{1.959388in}{2.985671in}}{\pgfqpoint{1.967202in}{2.993484in}}%
\pgfpathcurveto{\pgfqpoint{1.975015in}{3.001298in}}{\pgfqpoint{1.979406in}{3.011897in}}{\pgfqpoint{1.979406in}{3.022947in}}%
\pgfpathcurveto{\pgfqpoint{1.979406in}{3.033997in}}{\pgfqpoint{1.975015in}{3.044596in}}{\pgfqpoint{1.967202in}{3.052410in}}%
\pgfpathcurveto{\pgfqpoint{1.959388in}{3.060224in}}{\pgfqpoint{1.948789in}{3.064614in}}{\pgfqpoint{1.937739in}{3.064614in}}%
\pgfpathcurveto{\pgfqpoint{1.926689in}{3.064614in}}{\pgfqpoint{1.916090in}{3.060224in}}{\pgfqpoint{1.908276in}{3.052410in}}%
\pgfpathcurveto{\pgfqpoint{1.900462in}{3.044596in}}{\pgfqpoint{1.896072in}{3.033997in}}{\pgfqpoint{1.896072in}{3.022947in}}%
\pgfpathcurveto{\pgfqpoint{1.896072in}{3.011897in}}{\pgfqpoint{1.900462in}{3.001298in}}{\pgfqpoint{1.908276in}{2.993484in}}%
\pgfpathcurveto{\pgfqpoint{1.916090in}{2.985671in}}{\pgfqpoint{1.926689in}{2.981281in}}{\pgfqpoint{1.937739in}{2.981281in}}%
\pgfpathclose%
\pgfusepath{stroke,fill}%
\end{pgfscope}%
\begin{pgfscope}%
\pgfpathrectangle{\pgfqpoint{0.787074in}{0.548769in}}{\pgfqpoint{5.062926in}{3.102590in}}%
\pgfusepath{clip}%
\pgfsetbuttcap%
\pgfsetroundjoin%
\definecolor{currentfill}{rgb}{0.121569,0.466667,0.705882}%
\pgfsetfillcolor{currentfill}%
\pgfsetlinewidth{1.003750pt}%
\definecolor{currentstroke}{rgb}{0.121569,0.466667,0.705882}%
\pgfsetstrokecolor{currentstroke}%
\pgfsetdash{}{0pt}%
\pgfpathmoveto{\pgfqpoint{1.608977in}{0.648132in}}%
\pgfpathcurveto{\pgfqpoint{1.620028in}{0.648132in}}{\pgfqpoint{1.630627in}{0.652522in}}{\pgfqpoint{1.638440in}{0.660336in}}%
\pgfpathcurveto{\pgfqpoint{1.646254in}{0.668149in}}{\pgfqpoint{1.650644in}{0.678749in}}{\pgfqpoint{1.650644in}{0.689799in}}%
\pgfpathcurveto{\pgfqpoint{1.650644in}{0.700849in}}{\pgfqpoint{1.646254in}{0.711448in}}{\pgfqpoint{1.638440in}{0.719261in}}%
\pgfpathcurveto{\pgfqpoint{1.630627in}{0.727075in}}{\pgfqpoint{1.620028in}{0.731465in}}{\pgfqpoint{1.608977in}{0.731465in}}%
\pgfpathcurveto{\pgfqpoint{1.597927in}{0.731465in}}{\pgfqpoint{1.587328in}{0.727075in}}{\pgfqpoint{1.579515in}{0.719261in}}%
\pgfpathcurveto{\pgfqpoint{1.571701in}{0.711448in}}{\pgfqpoint{1.567311in}{0.700849in}}{\pgfqpoint{1.567311in}{0.689799in}}%
\pgfpathcurveto{\pgfqpoint{1.567311in}{0.678749in}}{\pgfqpoint{1.571701in}{0.668149in}}{\pgfqpoint{1.579515in}{0.660336in}}%
\pgfpathcurveto{\pgfqpoint{1.587328in}{0.652522in}}{\pgfqpoint{1.597927in}{0.648132in}}{\pgfqpoint{1.608977in}{0.648132in}}%
\pgfpathclose%
\pgfusepath{stroke,fill}%
\end{pgfscope}%
\begin{pgfscope}%
\pgfpathrectangle{\pgfqpoint{0.787074in}{0.548769in}}{\pgfqpoint{5.062926in}{3.102590in}}%
\pgfusepath{clip}%
\pgfsetbuttcap%
\pgfsetroundjoin%
\definecolor{currentfill}{rgb}{1.000000,0.498039,0.054902}%
\pgfsetfillcolor{currentfill}%
\pgfsetlinewidth{1.003750pt}%
\definecolor{currentstroke}{rgb}{1.000000,0.498039,0.054902}%
\pgfsetstrokecolor{currentstroke}%
\pgfsetdash{}{0pt}%
\pgfpathmoveto{\pgfqpoint{1.477473in}{2.254043in}}%
\pgfpathcurveto{\pgfqpoint{1.488523in}{2.254043in}}{\pgfqpoint{1.499122in}{2.258433in}}{\pgfqpoint{1.506936in}{2.266247in}}%
\pgfpathcurveto{\pgfqpoint{1.514749in}{2.274060in}}{\pgfqpoint{1.519140in}{2.284659in}}{\pgfqpoint{1.519140in}{2.295709in}}%
\pgfpathcurveto{\pgfqpoint{1.519140in}{2.306759in}}{\pgfqpoint{1.514749in}{2.317359in}}{\pgfqpoint{1.506936in}{2.325172in}}%
\pgfpathcurveto{\pgfqpoint{1.499122in}{2.332986in}}{\pgfqpoint{1.488523in}{2.337376in}}{\pgfqpoint{1.477473in}{2.337376in}}%
\pgfpathcurveto{\pgfqpoint{1.466423in}{2.337376in}}{\pgfqpoint{1.455824in}{2.332986in}}{\pgfqpoint{1.448010in}{2.325172in}}%
\pgfpathcurveto{\pgfqpoint{1.440196in}{2.317359in}}{\pgfqpoint{1.435806in}{2.306759in}}{\pgfqpoint{1.435806in}{2.295709in}}%
\pgfpathcurveto{\pgfqpoint{1.435806in}{2.284659in}}{\pgfqpoint{1.440196in}{2.274060in}}{\pgfqpoint{1.448010in}{2.266247in}}%
\pgfpathcurveto{\pgfqpoint{1.455824in}{2.258433in}}{\pgfqpoint{1.466423in}{2.254043in}}{\pgfqpoint{1.477473in}{2.254043in}}%
\pgfpathclose%
\pgfusepath{stroke,fill}%
\end{pgfscope}%
\begin{pgfscope}%
\pgfpathrectangle{\pgfqpoint{0.787074in}{0.548769in}}{\pgfqpoint{5.062926in}{3.102590in}}%
\pgfusepath{clip}%
\pgfsetbuttcap%
\pgfsetroundjoin%
\definecolor{currentfill}{rgb}{1.000000,0.498039,0.054902}%
\pgfsetfillcolor{currentfill}%
\pgfsetlinewidth{1.003750pt}%
\definecolor{currentstroke}{rgb}{1.000000,0.498039,0.054902}%
\pgfsetstrokecolor{currentstroke}%
\pgfsetdash{}{0pt}%
\pgfpathmoveto{\pgfqpoint{1.477473in}{2.386212in}}%
\pgfpathcurveto{\pgfqpoint{1.488523in}{2.386212in}}{\pgfqpoint{1.499122in}{2.390603in}}{\pgfqpoint{1.506936in}{2.398416in}}%
\pgfpathcurveto{\pgfqpoint{1.514749in}{2.406230in}}{\pgfqpoint{1.519140in}{2.416829in}}{\pgfqpoint{1.519140in}{2.427879in}}%
\pgfpathcurveto{\pgfqpoint{1.519140in}{2.438929in}}{\pgfqpoint{1.514749in}{2.449528in}}{\pgfqpoint{1.506936in}{2.457342in}}%
\pgfpathcurveto{\pgfqpoint{1.499122in}{2.465155in}}{\pgfqpoint{1.488523in}{2.469546in}}{\pgfqpoint{1.477473in}{2.469546in}}%
\pgfpathcurveto{\pgfqpoint{1.466423in}{2.469546in}}{\pgfqpoint{1.455824in}{2.465155in}}{\pgfqpoint{1.448010in}{2.457342in}}%
\pgfpathcurveto{\pgfqpoint{1.440196in}{2.449528in}}{\pgfqpoint{1.435806in}{2.438929in}}{\pgfqpoint{1.435806in}{2.427879in}}%
\pgfpathcurveto{\pgfqpoint{1.435806in}{2.416829in}}{\pgfqpoint{1.440196in}{2.406230in}}{\pgfqpoint{1.448010in}{2.398416in}}%
\pgfpathcurveto{\pgfqpoint{1.455824in}{2.390603in}}{\pgfqpoint{1.466423in}{2.386212in}}{\pgfqpoint{1.477473in}{2.386212in}}%
\pgfpathclose%
\pgfusepath{stroke,fill}%
\end{pgfscope}%
\begin{pgfscope}%
\pgfpathrectangle{\pgfqpoint{0.787074in}{0.548769in}}{\pgfqpoint{5.062926in}{3.102590in}}%
\pgfusepath{clip}%
\pgfsetbuttcap%
\pgfsetroundjoin%
\definecolor{currentfill}{rgb}{0.121569,0.466667,0.705882}%
\pgfsetfillcolor{currentfill}%
\pgfsetlinewidth{1.003750pt}%
\definecolor{currentstroke}{rgb}{0.121569,0.466667,0.705882}%
\pgfsetstrokecolor{currentstroke}%
\pgfsetdash{}{0pt}%
\pgfpathmoveto{\pgfqpoint{1.411721in}{0.648147in}}%
\pgfpathcurveto{\pgfqpoint{1.422771in}{0.648147in}}{\pgfqpoint{1.433370in}{0.652537in}}{\pgfqpoint{1.441183in}{0.660351in}}%
\pgfpathcurveto{\pgfqpoint{1.448997in}{0.668165in}}{\pgfqpoint{1.453387in}{0.678764in}}{\pgfqpoint{1.453387in}{0.689814in}}%
\pgfpathcurveto{\pgfqpoint{1.453387in}{0.700864in}}{\pgfqpoint{1.448997in}{0.711463in}}{\pgfqpoint{1.441183in}{0.719277in}}%
\pgfpathcurveto{\pgfqpoint{1.433370in}{0.727090in}}{\pgfqpoint{1.422771in}{0.731480in}}{\pgfqpoint{1.411721in}{0.731480in}}%
\pgfpathcurveto{\pgfqpoint{1.400670in}{0.731480in}}{\pgfqpoint{1.390071in}{0.727090in}}{\pgfqpoint{1.382258in}{0.719277in}}%
\pgfpathcurveto{\pgfqpoint{1.374444in}{0.711463in}}{\pgfqpoint{1.370054in}{0.700864in}}{\pgfqpoint{1.370054in}{0.689814in}}%
\pgfpathcurveto{\pgfqpoint{1.370054in}{0.678764in}}{\pgfqpoint{1.374444in}{0.668165in}}{\pgfqpoint{1.382258in}{0.660351in}}%
\pgfpathcurveto{\pgfqpoint{1.390071in}{0.652537in}}{\pgfqpoint{1.400670in}{0.648147in}}{\pgfqpoint{1.411721in}{0.648147in}}%
\pgfpathclose%
\pgfusepath{stroke,fill}%
\end{pgfscope}%
\begin{pgfscope}%
\pgfpathrectangle{\pgfqpoint{0.787074in}{0.548769in}}{\pgfqpoint{5.062926in}{3.102590in}}%
\pgfusepath{clip}%
\pgfsetbuttcap%
\pgfsetroundjoin%
\definecolor{currentfill}{rgb}{1.000000,0.498039,0.054902}%
\pgfsetfillcolor{currentfill}%
\pgfsetlinewidth{1.003750pt}%
\definecolor{currentstroke}{rgb}{1.000000,0.498039,0.054902}%
\pgfsetstrokecolor{currentstroke}%
\pgfsetdash{}{0pt}%
\pgfpathmoveto{\pgfqpoint{1.345968in}{2.590063in}}%
\pgfpathcurveto{\pgfqpoint{1.357018in}{2.590063in}}{\pgfqpoint{1.367617in}{2.594453in}}{\pgfqpoint{1.375431in}{2.602267in}}%
\pgfpathcurveto{\pgfqpoint{1.383245in}{2.610081in}}{\pgfqpoint{1.387635in}{2.620680in}}{\pgfqpoint{1.387635in}{2.631730in}}%
\pgfpathcurveto{\pgfqpoint{1.387635in}{2.642780in}}{\pgfqpoint{1.383245in}{2.653379in}}{\pgfqpoint{1.375431in}{2.661193in}}%
\pgfpathcurveto{\pgfqpoint{1.367617in}{2.669006in}}{\pgfqpoint{1.357018in}{2.673397in}}{\pgfqpoint{1.345968in}{2.673397in}}%
\pgfpathcurveto{\pgfqpoint{1.334918in}{2.673397in}}{\pgfqpoint{1.324319in}{2.669006in}}{\pgfqpoint{1.316506in}{2.661193in}}%
\pgfpathcurveto{\pgfqpoint{1.308692in}{2.653379in}}{\pgfqpoint{1.304302in}{2.642780in}}{\pgfqpoint{1.304302in}{2.631730in}}%
\pgfpathcurveto{\pgfqpoint{1.304302in}{2.620680in}}{\pgfqpoint{1.308692in}{2.610081in}}{\pgfqpoint{1.316506in}{2.602267in}}%
\pgfpathcurveto{\pgfqpoint{1.324319in}{2.594453in}}{\pgfqpoint{1.334918in}{2.590063in}}{\pgfqpoint{1.345968in}{2.590063in}}%
\pgfpathclose%
\pgfusepath{stroke,fill}%
\end{pgfscope}%
\begin{pgfscope}%
\pgfpathrectangle{\pgfqpoint{0.787074in}{0.548769in}}{\pgfqpoint{5.062926in}{3.102590in}}%
\pgfusepath{clip}%
\pgfsetbuttcap%
\pgfsetroundjoin%
\definecolor{currentfill}{rgb}{1.000000,0.498039,0.054902}%
\pgfsetfillcolor{currentfill}%
\pgfsetlinewidth{1.003750pt}%
\definecolor{currentstroke}{rgb}{1.000000,0.498039,0.054902}%
\pgfsetstrokecolor{currentstroke}%
\pgfsetdash{}{0pt}%
\pgfpathmoveto{\pgfqpoint{1.477473in}{2.389366in}}%
\pgfpathcurveto{\pgfqpoint{1.488523in}{2.389366in}}{\pgfqpoint{1.499122in}{2.393756in}}{\pgfqpoint{1.506936in}{2.401570in}}%
\pgfpathcurveto{\pgfqpoint{1.514749in}{2.409383in}}{\pgfqpoint{1.519140in}{2.419982in}}{\pgfqpoint{1.519140in}{2.431032in}}%
\pgfpathcurveto{\pgfqpoint{1.519140in}{2.442082in}}{\pgfqpoint{1.514749in}{2.452682in}}{\pgfqpoint{1.506936in}{2.460495in}}%
\pgfpathcurveto{\pgfqpoint{1.499122in}{2.468309in}}{\pgfqpoint{1.488523in}{2.472699in}}{\pgfqpoint{1.477473in}{2.472699in}}%
\pgfpathcurveto{\pgfqpoint{1.466423in}{2.472699in}}{\pgfqpoint{1.455824in}{2.468309in}}{\pgfqpoint{1.448010in}{2.460495in}}%
\pgfpathcurveto{\pgfqpoint{1.440196in}{2.452682in}}{\pgfqpoint{1.435806in}{2.442082in}}{\pgfqpoint{1.435806in}{2.431032in}}%
\pgfpathcurveto{\pgfqpoint{1.435806in}{2.419982in}}{\pgfqpoint{1.440196in}{2.409383in}}{\pgfqpoint{1.448010in}{2.401570in}}%
\pgfpathcurveto{\pgfqpoint{1.455824in}{2.393756in}}{\pgfqpoint{1.466423in}{2.389366in}}{\pgfqpoint{1.477473in}{2.389366in}}%
\pgfpathclose%
\pgfusepath{stroke,fill}%
\end{pgfscope}%
\begin{pgfscope}%
\pgfpathrectangle{\pgfqpoint{0.787074in}{0.548769in}}{\pgfqpoint{5.062926in}{3.102590in}}%
\pgfusepath{clip}%
\pgfsetbuttcap%
\pgfsetroundjoin%
\definecolor{currentfill}{rgb}{0.121569,0.466667,0.705882}%
\pgfsetfillcolor{currentfill}%
\pgfsetlinewidth{1.003750pt}%
\definecolor{currentstroke}{rgb}{0.121569,0.466667,0.705882}%
\pgfsetstrokecolor{currentstroke}%
\pgfsetdash{}{0pt}%
\pgfpathmoveto{\pgfqpoint{1.937739in}{0.658967in}}%
\pgfpathcurveto{\pgfqpoint{1.948789in}{0.658967in}}{\pgfqpoint{1.959388in}{0.663357in}}{\pgfqpoint{1.967202in}{0.671170in}}%
\pgfpathcurveto{\pgfqpoint{1.975015in}{0.678984in}}{\pgfqpoint{1.979406in}{0.689583in}}{\pgfqpoint{1.979406in}{0.700633in}}%
\pgfpathcurveto{\pgfqpoint{1.979406in}{0.711683in}}{\pgfqpoint{1.975015in}{0.722282in}}{\pgfqpoint{1.967202in}{0.730096in}}%
\pgfpathcurveto{\pgfqpoint{1.959388in}{0.737910in}}{\pgfqpoint{1.948789in}{0.742300in}}{\pgfqpoint{1.937739in}{0.742300in}}%
\pgfpathcurveto{\pgfqpoint{1.926689in}{0.742300in}}{\pgfqpoint{1.916090in}{0.737910in}}{\pgfqpoint{1.908276in}{0.730096in}}%
\pgfpathcurveto{\pgfqpoint{1.900462in}{0.722282in}}{\pgfqpoint{1.896072in}{0.711683in}}{\pgfqpoint{1.896072in}{0.700633in}}%
\pgfpathcurveto{\pgfqpoint{1.896072in}{0.689583in}}{\pgfqpoint{1.900462in}{0.678984in}}{\pgfqpoint{1.908276in}{0.671170in}}%
\pgfpathcurveto{\pgfqpoint{1.916090in}{0.663357in}}{\pgfqpoint{1.926689in}{0.658967in}}{\pgfqpoint{1.937739in}{0.658967in}}%
\pgfpathclose%
\pgfusepath{stroke,fill}%
\end{pgfscope}%
\begin{pgfscope}%
\pgfpathrectangle{\pgfqpoint{0.787074in}{0.548769in}}{\pgfqpoint{5.062926in}{3.102590in}}%
\pgfusepath{clip}%
\pgfsetbuttcap%
\pgfsetroundjoin%
\definecolor{currentfill}{rgb}{1.000000,0.498039,0.054902}%
\pgfsetfillcolor{currentfill}%
\pgfsetlinewidth{1.003750pt}%
\definecolor{currentstroke}{rgb}{1.000000,0.498039,0.054902}%
\pgfsetstrokecolor{currentstroke}%
\pgfsetdash{}{0pt}%
\pgfpathmoveto{\pgfqpoint{2.003491in}{2.623396in}}%
\pgfpathcurveto{\pgfqpoint{2.014541in}{2.623396in}}{\pgfqpoint{2.025140in}{2.627787in}}{\pgfqpoint{2.032954in}{2.635600in}}%
\pgfpathcurveto{\pgfqpoint{2.040768in}{2.643414in}}{\pgfqpoint{2.045158in}{2.654013in}}{\pgfqpoint{2.045158in}{2.665063in}}%
\pgfpathcurveto{\pgfqpoint{2.045158in}{2.676113in}}{\pgfqpoint{2.040768in}{2.686712in}}{\pgfqpoint{2.032954in}{2.694526in}}%
\pgfpathcurveto{\pgfqpoint{2.025140in}{2.702339in}}{\pgfqpoint{2.014541in}{2.706730in}}{\pgfqpoint{2.003491in}{2.706730in}}%
\pgfpathcurveto{\pgfqpoint{1.992441in}{2.706730in}}{\pgfqpoint{1.981842in}{2.702339in}}{\pgfqpoint{1.974028in}{2.694526in}}%
\pgfpathcurveto{\pgfqpoint{1.966215in}{2.686712in}}{\pgfqpoint{1.961824in}{2.676113in}}{\pgfqpoint{1.961824in}{2.665063in}}%
\pgfpathcurveto{\pgfqpoint{1.961824in}{2.654013in}}{\pgfqpoint{1.966215in}{2.643414in}}{\pgfqpoint{1.974028in}{2.635600in}}%
\pgfpathcurveto{\pgfqpoint{1.981842in}{2.627787in}}{\pgfqpoint{1.992441in}{2.623396in}}{\pgfqpoint{2.003491in}{2.623396in}}%
\pgfpathclose%
\pgfusepath{stroke,fill}%
\end{pgfscope}%
\begin{pgfscope}%
\pgfpathrectangle{\pgfqpoint{0.787074in}{0.548769in}}{\pgfqpoint{5.062926in}{3.102590in}}%
\pgfusepath{clip}%
\pgfsetbuttcap%
\pgfsetroundjoin%
\definecolor{currentfill}{rgb}{1.000000,0.498039,0.054902}%
\pgfsetfillcolor{currentfill}%
\pgfsetlinewidth{1.003750pt}%
\definecolor{currentstroke}{rgb}{1.000000,0.498039,0.054902}%
\pgfsetstrokecolor{currentstroke}%
\pgfsetdash{}{0pt}%
\pgfpathmoveto{\pgfqpoint{2.003491in}{2.816554in}}%
\pgfpathcurveto{\pgfqpoint{2.014541in}{2.816554in}}{\pgfqpoint{2.025140in}{2.820944in}}{\pgfqpoint{2.032954in}{2.828758in}}%
\pgfpathcurveto{\pgfqpoint{2.040768in}{2.836572in}}{\pgfqpoint{2.045158in}{2.847171in}}{\pgfqpoint{2.045158in}{2.858221in}}%
\pgfpathcurveto{\pgfqpoint{2.045158in}{2.869271in}}{\pgfqpoint{2.040768in}{2.879870in}}{\pgfqpoint{2.032954in}{2.887684in}}%
\pgfpathcurveto{\pgfqpoint{2.025140in}{2.895497in}}{\pgfqpoint{2.014541in}{2.899887in}}{\pgfqpoint{2.003491in}{2.899887in}}%
\pgfpathcurveto{\pgfqpoint{1.992441in}{2.899887in}}{\pgfqpoint{1.981842in}{2.895497in}}{\pgfqpoint{1.974028in}{2.887684in}}%
\pgfpathcurveto{\pgfqpoint{1.966215in}{2.879870in}}{\pgfqpoint{1.961824in}{2.869271in}}{\pgfqpoint{1.961824in}{2.858221in}}%
\pgfpathcurveto{\pgfqpoint{1.961824in}{2.847171in}}{\pgfqpoint{1.966215in}{2.836572in}}{\pgfqpoint{1.974028in}{2.828758in}}%
\pgfpathcurveto{\pgfqpoint{1.981842in}{2.820944in}}{\pgfqpoint{1.992441in}{2.816554in}}{\pgfqpoint{2.003491in}{2.816554in}}%
\pgfpathclose%
\pgfusepath{stroke,fill}%
\end{pgfscope}%
\begin{pgfscope}%
\pgfpathrectangle{\pgfqpoint{0.787074in}{0.548769in}}{\pgfqpoint{5.062926in}{3.102590in}}%
\pgfusepath{clip}%
\pgfsetbuttcap%
\pgfsetroundjoin%
\definecolor{currentfill}{rgb}{1.000000,0.498039,0.054902}%
\pgfsetfillcolor{currentfill}%
\pgfsetlinewidth{1.003750pt}%
\definecolor{currentstroke}{rgb}{1.000000,0.498039,0.054902}%
\pgfsetstrokecolor{currentstroke}%
\pgfsetdash{}{0pt}%
\pgfpathmoveto{\pgfqpoint{2.003491in}{2.998783in}}%
\pgfpathcurveto{\pgfqpoint{2.014541in}{2.998783in}}{\pgfqpoint{2.025140in}{3.003173in}}{\pgfqpoint{2.032954in}{3.010987in}}%
\pgfpathcurveto{\pgfqpoint{2.040768in}{3.018800in}}{\pgfqpoint{2.045158in}{3.029399in}}{\pgfqpoint{2.045158in}{3.040450in}}%
\pgfpathcurveto{\pgfqpoint{2.045158in}{3.051500in}}{\pgfqpoint{2.040768in}{3.062099in}}{\pgfqpoint{2.032954in}{3.069912in}}%
\pgfpathcurveto{\pgfqpoint{2.025140in}{3.077726in}}{\pgfqpoint{2.014541in}{3.082116in}}{\pgfqpoint{2.003491in}{3.082116in}}%
\pgfpathcurveto{\pgfqpoint{1.992441in}{3.082116in}}{\pgfqpoint{1.981842in}{3.077726in}}{\pgfqpoint{1.974028in}{3.069912in}}%
\pgfpathcurveto{\pgfqpoint{1.966215in}{3.062099in}}{\pgfqpoint{1.961824in}{3.051500in}}{\pgfqpoint{1.961824in}{3.040450in}}%
\pgfpathcurveto{\pgfqpoint{1.961824in}{3.029399in}}{\pgfqpoint{1.966215in}{3.018800in}}{\pgfqpoint{1.974028in}{3.010987in}}%
\pgfpathcurveto{\pgfqpoint{1.981842in}{3.003173in}}{\pgfqpoint{1.992441in}{2.998783in}}{\pgfqpoint{2.003491in}{2.998783in}}%
\pgfpathclose%
\pgfusepath{stroke,fill}%
\end{pgfscope}%
\begin{pgfscope}%
\pgfpathrectangle{\pgfqpoint{0.787074in}{0.548769in}}{\pgfqpoint{5.062926in}{3.102590in}}%
\pgfusepath{clip}%
\pgfsetbuttcap%
\pgfsetroundjoin%
\definecolor{currentfill}{rgb}{1.000000,0.498039,0.054902}%
\pgfsetfillcolor{currentfill}%
\pgfsetlinewidth{1.003750pt}%
\definecolor{currentstroke}{rgb}{1.000000,0.498039,0.054902}%
\pgfsetstrokecolor{currentstroke}%
\pgfsetdash{}{0pt}%
\pgfpathmoveto{\pgfqpoint{1.411721in}{2.057042in}}%
\pgfpathcurveto{\pgfqpoint{1.422771in}{2.057042in}}{\pgfqpoint{1.433370in}{2.061432in}}{\pgfqpoint{1.441183in}{2.069246in}}%
\pgfpathcurveto{\pgfqpoint{1.448997in}{2.077059in}}{\pgfqpoint{1.453387in}{2.087659in}}{\pgfqpoint{1.453387in}{2.098709in}}%
\pgfpathcurveto{\pgfqpoint{1.453387in}{2.109759in}}{\pgfqpoint{1.448997in}{2.120358in}}{\pgfqpoint{1.441183in}{2.128171in}}%
\pgfpathcurveto{\pgfqpoint{1.433370in}{2.135985in}}{\pgfqpoint{1.422771in}{2.140375in}}{\pgfqpoint{1.411721in}{2.140375in}}%
\pgfpathcurveto{\pgfqpoint{1.400670in}{2.140375in}}{\pgfqpoint{1.390071in}{2.135985in}}{\pgfqpoint{1.382258in}{2.128171in}}%
\pgfpathcurveto{\pgfqpoint{1.374444in}{2.120358in}}{\pgfqpoint{1.370054in}{2.109759in}}{\pgfqpoint{1.370054in}{2.098709in}}%
\pgfpathcurveto{\pgfqpoint{1.370054in}{2.087659in}}{\pgfqpoint{1.374444in}{2.077059in}}{\pgfqpoint{1.382258in}{2.069246in}}%
\pgfpathcurveto{\pgfqpoint{1.390071in}{2.061432in}}{\pgfqpoint{1.400670in}{2.057042in}}{\pgfqpoint{1.411721in}{2.057042in}}%
\pgfpathclose%
\pgfusepath{stroke,fill}%
\end{pgfscope}%
\begin{pgfscope}%
\pgfpathrectangle{\pgfqpoint{0.787074in}{0.548769in}}{\pgfqpoint{5.062926in}{3.102590in}}%
\pgfusepath{clip}%
\pgfsetbuttcap%
\pgfsetroundjoin%
\definecolor{currentfill}{rgb}{1.000000,0.498039,0.054902}%
\pgfsetfillcolor{currentfill}%
\pgfsetlinewidth{1.003750pt}%
\definecolor{currentstroke}{rgb}{1.000000,0.498039,0.054902}%
\pgfsetstrokecolor{currentstroke}%
\pgfsetdash{}{0pt}%
\pgfpathmoveto{\pgfqpoint{1.082959in}{3.349495in}}%
\pgfpathcurveto{\pgfqpoint{1.094009in}{3.349495in}}{\pgfqpoint{1.104608in}{3.353886in}}{\pgfqpoint{1.112422in}{3.361699in}}%
\pgfpathcurveto{\pgfqpoint{1.120236in}{3.369513in}}{\pgfqpoint{1.124626in}{3.380112in}}{\pgfqpoint{1.124626in}{3.391162in}}%
\pgfpathcurveto{\pgfqpoint{1.124626in}{3.402212in}}{\pgfqpoint{1.120236in}{3.412811in}}{\pgfqpoint{1.112422in}{3.420625in}}%
\pgfpathcurveto{\pgfqpoint{1.104608in}{3.428438in}}{\pgfqpoint{1.094009in}{3.432829in}}{\pgfqpoint{1.082959in}{3.432829in}}%
\pgfpathcurveto{\pgfqpoint{1.071909in}{3.432829in}}{\pgfqpoint{1.061310in}{3.428438in}}{\pgfqpoint{1.053496in}{3.420625in}}%
\pgfpathcurveto{\pgfqpoint{1.045683in}{3.412811in}}{\pgfqpoint{1.041292in}{3.402212in}}{\pgfqpoint{1.041292in}{3.391162in}}%
\pgfpathcurveto{\pgfqpoint{1.041292in}{3.380112in}}{\pgfqpoint{1.045683in}{3.369513in}}{\pgfqpoint{1.053496in}{3.361699in}}%
\pgfpathcurveto{\pgfqpoint{1.061310in}{3.353886in}}{\pgfqpoint{1.071909in}{3.349495in}}{\pgfqpoint{1.082959in}{3.349495in}}%
\pgfpathclose%
\pgfusepath{stroke,fill}%
\end{pgfscope}%
\begin{pgfscope}%
\pgfpathrectangle{\pgfqpoint{0.787074in}{0.548769in}}{\pgfqpoint{5.062926in}{3.102590in}}%
\pgfusepath{clip}%
\pgfsetbuttcap%
\pgfsetroundjoin%
\definecolor{currentfill}{rgb}{1.000000,0.498039,0.054902}%
\pgfsetfillcolor{currentfill}%
\pgfsetlinewidth{1.003750pt}%
\definecolor{currentstroke}{rgb}{1.000000,0.498039,0.054902}%
\pgfsetstrokecolor{currentstroke}%
\pgfsetdash{}{0pt}%
\pgfpathmoveto{\pgfqpoint{1.806234in}{2.864404in}}%
\pgfpathcurveto{\pgfqpoint{1.817284in}{2.864404in}}{\pgfqpoint{1.827883in}{2.868794in}}{\pgfqpoint{1.835697in}{2.876608in}}%
\pgfpathcurveto{\pgfqpoint{1.843511in}{2.884422in}}{\pgfqpoint{1.847901in}{2.895021in}}{\pgfqpoint{1.847901in}{2.906071in}}%
\pgfpathcurveto{\pgfqpoint{1.847901in}{2.917121in}}{\pgfqpoint{1.843511in}{2.927720in}}{\pgfqpoint{1.835697in}{2.935534in}}%
\pgfpathcurveto{\pgfqpoint{1.827883in}{2.943347in}}{\pgfqpoint{1.817284in}{2.947737in}}{\pgfqpoint{1.806234in}{2.947737in}}%
\pgfpathcurveto{\pgfqpoint{1.795184in}{2.947737in}}{\pgfqpoint{1.784585in}{2.943347in}}{\pgfqpoint{1.776772in}{2.935534in}}%
\pgfpathcurveto{\pgfqpoint{1.768958in}{2.927720in}}{\pgfqpoint{1.764568in}{2.917121in}}{\pgfqpoint{1.764568in}{2.906071in}}%
\pgfpathcurveto{\pgfqpoint{1.764568in}{2.895021in}}{\pgfqpoint{1.768958in}{2.884422in}}{\pgfqpoint{1.776772in}{2.876608in}}%
\pgfpathcurveto{\pgfqpoint{1.784585in}{2.868794in}}{\pgfqpoint{1.795184in}{2.864404in}}{\pgfqpoint{1.806234in}{2.864404in}}%
\pgfpathclose%
\pgfusepath{stroke,fill}%
\end{pgfscope}%
\begin{pgfscope}%
\pgfpathrectangle{\pgfqpoint{0.787074in}{0.548769in}}{\pgfqpoint{5.062926in}{3.102590in}}%
\pgfusepath{clip}%
\pgfsetbuttcap%
\pgfsetroundjoin%
\definecolor{currentfill}{rgb}{0.121569,0.466667,0.705882}%
\pgfsetfillcolor{currentfill}%
\pgfsetlinewidth{1.003750pt}%
\definecolor{currentstroke}{rgb}{0.121569,0.466667,0.705882}%
\pgfsetstrokecolor{currentstroke}%
\pgfsetdash{}{0pt}%
\pgfpathmoveto{\pgfqpoint{1.543225in}{0.648143in}}%
\pgfpathcurveto{\pgfqpoint{1.554275in}{0.648143in}}{\pgfqpoint{1.564874in}{0.652534in}}{\pgfqpoint{1.572688in}{0.660347in}}%
\pgfpathcurveto{\pgfqpoint{1.580502in}{0.668161in}}{\pgfqpoint{1.584892in}{0.678760in}}{\pgfqpoint{1.584892in}{0.689810in}}%
\pgfpathcurveto{\pgfqpoint{1.584892in}{0.700860in}}{\pgfqpoint{1.580502in}{0.711459in}}{\pgfqpoint{1.572688in}{0.719273in}}%
\pgfpathcurveto{\pgfqpoint{1.564874in}{0.727086in}}{\pgfqpoint{1.554275in}{0.731477in}}{\pgfqpoint{1.543225in}{0.731477in}}%
\pgfpathcurveto{\pgfqpoint{1.532175in}{0.731477in}}{\pgfqpoint{1.521576in}{0.727086in}}{\pgfqpoint{1.513762in}{0.719273in}}%
\pgfpathcurveto{\pgfqpoint{1.505949in}{0.711459in}}{\pgfqpoint{1.501558in}{0.700860in}}{\pgfqpoint{1.501558in}{0.689810in}}%
\pgfpathcurveto{\pgfqpoint{1.501558in}{0.678760in}}{\pgfqpoint{1.505949in}{0.668161in}}{\pgfqpoint{1.513762in}{0.660347in}}%
\pgfpathcurveto{\pgfqpoint{1.521576in}{0.652534in}}{\pgfqpoint{1.532175in}{0.648143in}}{\pgfqpoint{1.543225in}{0.648143in}}%
\pgfpathclose%
\pgfusepath{stroke,fill}%
\end{pgfscope}%
\begin{pgfscope}%
\pgfpathrectangle{\pgfqpoint{0.787074in}{0.548769in}}{\pgfqpoint{5.062926in}{3.102590in}}%
\pgfusepath{clip}%
\pgfsetbuttcap%
\pgfsetroundjoin%
\definecolor{currentfill}{rgb}{0.121569,0.466667,0.705882}%
\pgfsetfillcolor{currentfill}%
\pgfsetlinewidth{1.003750pt}%
\definecolor{currentstroke}{rgb}{0.121569,0.466667,0.705882}%
\pgfsetstrokecolor{currentstroke}%
\pgfsetdash{}{0pt}%
\pgfpathmoveto{\pgfqpoint{1.543225in}{0.648148in}}%
\pgfpathcurveto{\pgfqpoint{1.554275in}{0.648148in}}{\pgfqpoint{1.564874in}{0.652538in}}{\pgfqpoint{1.572688in}{0.660352in}}%
\pgfpathcurveto{\pgfqpoint{1.580502in}{0.668165in}}{\pgfqpoint{1.584892in}{0.678764in}}{\pgfqpoint{1.584892in}{0.689815in}}%
\pgfpathcurveto{\pgfqpoint{1.584892in}{0.700865in}}{\pgfqpoint{1.580502in}{0.711464in}}{\pgfqpoint{1.572688in}{0.719277in}}%
\pgfpathcurveto{\pgfqpoint{1.564874in}{0.727091in}}{\pgfqpoint{1.554275in}{0.731481in}}{\pgfqpoint{1.543225in}{0.731481in}}%
\pgfpathcurveto{\pgfqpoint{1.532175in}{0.731481in}}{\pgfqpoint{1.521576in}{0.727091in}}{\pgfqpoint{1.513762in}{0.719277in}}%
\pgfpathcurveto{\pgfqpoint{1.505949in}{0.711464in}}{\pgfqpoint{1.501558in}{0.700865in}}{\pgfqpoint{1.501558in}{0.689815in}}%
\pgfpathcurveto{\pgfqpoint{1.501558in}{0.678764in}}{\pgfqpoint{1.505949in}{0.668165in}}{\pgfqpoint{1.513762in}{0.660352in}}%
\pgfpathcurveto{\pgfqpoint{1.521576in}{0.652538in}}{\pgfqpoint{1.532175in}{0.648148in}}{\pgfqpoint{1.543225in}{0.648148in}}%
\pgfpathclose%
\pgfusepath{stroke,fill}%
\end{pgfscope}%
\begin{pgfscope}%
\pgfpathrectangle{\pgfqpoint{0.787074in}{0.548769in}}{\pgfqpoint{5.062926in}{3.102590in}}%
\pgfusepath{clip}%
\pgfsetbuttcap%
\pgfsetroundjoin%
\definecolor{currentfill}{rgb}{0.121569,0.466667,0.705882}%
\pgfsetfillcolor{currentfill}%
\pgfsetlinewidth{1.003750pt}%
\definecolor{currentstroke}{rgb}{0.121569,0.466667,0.705882}%
\pgfsetstrokecolor{currentstroke}%
\pgfsetdash{}{0pt}%
\pgfpathmoveto{\pgfqpoint{1.411721in}{0.648152in}}%
\pgfpathcurveto{\pgfqpoint{1.422771in}{0.648152in}}{\pgfqpoint{1.433370in}{0.652542in}}{\pgfqpoint{1.441183in}{0.660355in}}%
\pgfpathcurveto{\pgfqpoint{1.448997in}{0.668169in}}{\pgfqpoint{1.453387in}{0.678768in}}{\pgfqpoint{1.453387in}{0.689818in}}%
\pgfpathcurveto{\pgfqpoint{1.453387in}{0.700868in}}{\pgfqpoint{1.448997in}{0.711467in}}{\pgfqpoint{1.441183in}{0.719281in}}%
\pgfpathcurveto{\pgfqpoint{1.433370in}{0.727095in}}{\pgfqpoint{1.422771in}{0.731485in}}{\pgfqpoint{1.411721in}{0.731485in}}%
\pgfpathcurveto{\pgfqpoint{1.400670in}{0.731485in}}{\pgfqpoint{1.390071in}{0.727095in}}{\pgfqpoint{1.382258in}{0.719281in}}%
\pgfpathcurveto{\pgfqpoint{1.374444in}{0.711467in}}{\pgfqpoint{1.370054in}{0.700868in}}{\pgfqpoint{1.370054in}{0.689818in}}%
\pgfpathcurveto{\pgfqpoint{1.370054in}{0.678768in}}{\pgfqpoint{1.374444in}{0.668169in}}{\pgfqpoint{1.382258in}{0.660355in}}%
\pgfpathcurveto{\pgfqpoint{1.390071in}{0.652542in}}{\pgfqpoint{1.400670in}{0.648152in}}{\pgfqpoint{1.411721in}{0.648152in}}%
\pgfpathclose%
\pgfusepath{stroke,fill}%
\end{pgfscope}%
\begin{pgfscope}%
\pgfpathrectangle{\pgfqpoint{0.787074in}{0.548769in}}{\pgfqpoint{5.062926in}{3.102590in}}%
\pgfusepath{clip}%
\pgfsetbuttcap%
\pgfsetroundjoin%
\definecolor{currentfill}{rgb}{1.000000,0.498039,0.054902}%
\pgfsetfillcolor{currentfill}%
\pgfsetlinewidth{1.003750pt}%
\definecolor{currentstroke}{rgb}{1.000000,0.498039,0.054902}%
\pgfsetstrokecolor{currentstroke}%
\pgfsetdash{}{0pt}%
\pgfpathmoveto{\pgfqpoint{1.806234in}{2.884914in}}%
\pgfpathcurveto{\pgfqpoint{1.817284in}{2.884914in}}{\pgfqpoint{1.827883in}{2.889305in}}{\pgfqpoint{1.835697in}{2.897118in}}%
\pgfpathcurveto{\pgfqpoint{1.843511in}{2.904932in}}{\pgfqpoint{1.847901in}{2.915531in}}{\pgfqpoint{1.847901in}{2.926581in}}%
\pgfpathcurveto{\pgfqpoint{1.847901in}{2.937631in}}{\pgfqpoint{1.843511in}{2.948230in}}{\pgfqpoint{1.835697in}{2.956044in}}%
\pgfpathcurveto{\pgfqpoint{1.827883in}{2.963857in}}{\pgfqpoint{1.817284in}{2.968248in}}{\pgfqpoint{1.806234in}{2.968248in}}%
\pgfpathcurveto{\pgfqpoint{1.795184in}{2.968248in}}{\pgfqpoint{1.784585in}{2.963857in}}{\pgfqpoint{1.776772in}{2.956044in}}%
\pgfpathcurveto{\pgfqpoint{1.768958in}{2.948230in}}{\pgfqpoint{1.764568in}{2.937631in}}{\pgfqpoint{1.764568in}{2.926581in}}%
\pgfpathcurveto{\pgfqpoint{1.764568in}{2.915531in}}{\pgfqpoint{1.768958in}{2.904932in}}{\pgfqpoint{1.776772in}{2.897118in}}%
\pgfpathcurveto{\pgfqpoint{1.784585in}{2.889305in}}{\pgfqpoint{1.795184in}{2.884914in}}{\pgfqpoint{1.806234in}{2.884914in}}%
\pgfpathclose%
\pgfusepath{stroke,fill}%
\end{pgfscope}%
\begin{pgfscope}%
\pgfpathrectangle{\pgfqpoint{0.787074in}{0.548769in}}{\pgfqpoint{5.062926in}{3.102590in}}%
\pgfusepath{clip}%
\pgfsetbuttcap%
\pgfsetroundjoin%
\definecolor{currentfill}{rgb}{1.000000,0.498039,0.054902}%
\pgfsetfillcolor{currentfill}%
\pgfsetlinewidth{1.003750pt}%
\definecolor{currentstroke}{rgb}{1.000000,0.498039,0.054902}%
\pgfsetstrokecolor{currentstroke}%
\pgfsetdash{}{0pt}%
\pgfpathmoveto{\pgfqpoint{2.069243in}{2.875764in}}%
\pgfpathcurveto{\pgfqpoint{2.080294in}{2.875764in}}{\pgfqpoint{2.090893in}{2.880154in}}{\pgfqpoint{2.098706in}{2.887968in}}%
\pgfpathcurveto{\pgfqpoint{2.106520in}{2.895781in}}{\pgfqpoint{2.110910in}{2.906380in}}{\pgfqpoint{2.110910in}{2.917430in}}%
\pgfpathcurveto{\pgfqpoint{2.110910in}{2.928481in}}{\pgfqpoint{2.106520in}{2.939080in}}{\pgfqpoint{2.098706in}{2.946893in}}%
\pgfpathcurveto{\pgfqpoint{2.090893in}{2.954707in}}{\pgfqpoint{2.080294in}{2.959097in}}{\pgfqpoint{2.069243in}{2.959097in}}%
\pgfpathcurveto{\pgfqpoint{2.058193in}{2.959097in}}{\pgfqpoint{2.047594in}{2.954707in}}{\pgfqpoint{2.039781in}{2.946893in}}%
\pgfpathcurveto{\pgfqpoint{2.031967in}{2.939080in}}{\pgfqpoint{2.027577in}{2.928481in}}{\pgfqpoint{2.027577in}{2.917430in}}%
\pgfpathcurveto{\pgfqpoint{2.027577in}{2.906380in}}{\pgfqpoint{2.031967in}{2.895781in}}{\pgfqpoint{2.039781in}{2.887968in}}%
\pgfpathcurveto{\pgfqpoint{2.047594in}{2.880154in}}{\pgfqpoint{2.058193in}{2.875764in}}{\pgfqpoint{2.069243in}{2.875764in}}%
\pgfpathclose%
\pgfusepath{stroke,fill}%
\end{pgfscope}%
\begin{pgfscope}%
\pgfpathrectangle{\pgfqpoint{0.787074in}{0.548769in}}{\pgfqpoint{5.062926in}{3.102590in}}%
\pgfusepath{clip}%
\pgfsetbuttcap%
\pgfsetroundjoin%
\definecolor{currentfill}{rgb}{1.000000,0.498039,0.054902}%
\pgfsetfillcolor{currentfill}%
\pgfsetlinewidth{1.003750pt}%
\definecolor{currentstroke}{rgb}{1.000000,0.498039,0.054902}%
\pgfsetstrokecolor{currentstroke}%
\pgfsetdash{}{0pt}%
\pgfpathmoveto{\pgfqpoint{1.806234in}{2.086876in}}%
\pgfpathcurveto{\pgfqpoint{1.817284in}{2.086876in}}{\pgfqpoint{1.827883in}{2.091266in}}{\pgfqpoint{1.835697in}{2.099080in}}%
\pgfpathcurveto{\pgfqpoint{1.843511in}{2.106893in}}{\pgfqpoint{1.847901in}{2.117492in}}{\pgfqpoint{1.847901in}{2.128542in}}%
\pgfpathcurveto{\pgfqpoint{1.847901in}{2.139593in}}{\pgfqpoint{1.843511in}{2.150192in}}{\pgfqpoint{1.835697in}{2.158005in}}%
\pgfpathcurveto{\pgfqpoint{1.827883in}{2.165819in}}{\pgfqpoint{1.817284in}{2.170209in}}{\pgfqpoint{1.806234in}{2.170209in}}%
\pgfpathcurveto{\pgfqpoint{1.795184in}{2.170209in}}{\pgfqpoint{1.784585in}{2.165819in}}{\pgfqpoint{1.776772in}{2.158005in}}%
\pgfpathcurveto{\pgfqpoint{1.768958in}{2.150192in}}{\pgfqpoint{1.764568in}{2.139593in}}{\pgfqpoint{1.764568in}{2.128542in}}%
\pgfpathcurveto{\pgfqpoint{1.764568in}{2.117492in}}{\pgfqpoint{1.768958in}{2.106893in}}{\pgfqpoint{1.776772in}{2.099080in}}%
\pgfpathcurveto{\pgfqpoint{1.784585in}{2.091266in}}{\pgfqpoint{1.795184in}{2.086876in}}{\pgfqpoint{1.806234in}{2.086876in}}%
\pgfpathclose%
\pgfusepath{stroke,fill}%
\end{pgfscope}%
\begin{pgfscope}%
\pgfpathrectangle{\pgfqpoint{0.787074in}{0.548769in}}{\pgfqpoint{5.062926in}{3.102590in}}%
\pgfusepath{clip}%
\pgfsetbuttcap%
\pgfsetroundjoin%
\definecolor{currentfill}{rgb}{0.121569,0.466667,0.705882}%
\pgfsetfillcolor{currentfill}%
\pgfsetlinewidth{1.003750pt}%
\definecolor{currentstroke}{rgb}{0.121569,0.466667,0.705882}%
\pgfsetstrokecolor{currentstroke}%
\pgfsetdash{}{0pt}%
\pgfpathmoveto{\pgfqpoint{1.017207in}{0.648132in}}%
\pgfpathcurveto{\pgfqpoint{1.028257in}{0.648132in}}{\pgfqpoint{1.038856in}{0.652523in}}{\pgfqpoint{1.046670in}{0.660336in}}%
\pgfpathcurveto{\pgfqpoint{1.054483in}{0.668150in}}{\pgfqpoint{1.058874in}{0.678749in}}{\pgfqpoint{1.058874in}{0.689799in}}%
\pgfpathcurveto{\pgfqpoint{1.058874in}{0.700849in}}{\pgfqpoint{1.054483in}{0.711448in}}{\pgfqpoint{1.046670in}{0.719262in}}%
\pgfpathcurveto{\pgfqpoint{1.038856in}{0.727075in}}{\pgfqpoint{1.028257in}{0.731466in}}{\pgfqpoint{1.017207in}{0.731466in}}%
\pgfpathcurveto{\pgfqpoint{1.006157in}{0.731466in}}{\pgfqpoint{0.995558in}{0.727075in}}{\pgfqpoint{0.987744in}{0.719262in}}%
\pgfpathcurveto{\pgfqpoint{0.979930in}{0.711448in}}{\pgfqpoint{0.975540in}{0.700849in}}{\pgfqpoint{0.975540in}{0.689799in}}%
\pgfpathcurveto{\pgfqpoint{0.975540in}{0.678749in}}{\pgfqpoint{0.979930in}{0.668150in}}{\pgfqpoint{0.987744in}{0.660336in}}%
\pgfpathcurveto{\pgfqpoint{0.995558in}{0.652523in}}{\pgfqpoint{1.006157in}{0.648132in}}{\pgfqpoint{1.017207in}{0.648132in}}%
\pgfpathclose%
\pgfusepath{stroke,fill}%
\end{pgfscope}%
\begin{pgfscope}%
\pgfpathrectangle{\pgfqpoint{0.787074in}{0.548769in}}{\pgfqpoint{5.062926in}{3.102590in}}%
\pgfusepath{clip}%
\pgfsetbuttcap%
\pgfsetroundjoin%
\definecolor{currentfill}{rgb}{0.121569,0.466667,0.705882}%
\pgfsetfillcolor{currentfill}%
\pgfsetlinewidth{1.003750pt}%
\definecolor{currentstroke}{rgb}{0.121569,0.466667,0.705882}%
\pgfsetstrokecolor{currentstroke}%
\pgfsetdash{}{0pt}%
\pgfpathmoveto{\pgfqpoint{1.280216in}{0.787529in}}%
\pgfpathcurveto{\pgfqpoint{1.291266in}{0.787529in}}{\pgfqpoint{1.301865in}{0.791919in}}{\pgfqpoint{1.309679in}{0.799733in}}%
\pgfpathcurveto{\pgfqpoint{1.317492in}{0.807547in}}{\pgfqpoint{1.321883in}{0.818146in}}{\pgfqpoint{1.321883in}{0.829196in}}%
\pgfpathcurveto{\pgfqpoint{1.321883in}{0.840246in}}{\pgfqpoint{1.317492in}{0.850845in}}{\pgfqpoint{1.309679in}{0.858658in}}%
\pgfpathcurveto{\pgfqpoint{1.301865in}{0.866472in}}{\pgfqpoint{1.291266in}{0.870862in}}{\pgfqpoint{1.280216in}{0.870862in}}%
\pgfpathcurveto{\pgfqpoint{1.269166in}{0.870862in}}{\pgfqpoint{1.258567in}{0.866472in}}{\pgfqpoint{1.250753in}{0.858658in}}%
\pgfpathcurveto{\pgfqpoint{1.242940in}{0.850845in}}{\pgfqpoint{1.238549in}{0.840246in}}{\pgfqpoint{1.238549in}{0.829196in}}%
\pgfpathcurveto{\pgfqpoint{1.238549in}{0.818146in}}{\pgfqpoint{1.242940in}{0.807547in}}{\pgfqpoint{1.250753in}{0.799733in}}%
\pgfpathcurveto{\pgfqpoint{1.258567in}{0.791919in}}{\pgfqpoint{1.269166in}{0.787529in}}{\pgfqpoint{1.280216in}{0.787529in}}%
\pgfpathclose%
\pgfusepath{stroke,fill}%
\end{pgfscope}%
\begin{pgfscope}%
\pgfpathrectangle{\pgfqpoint{0.787074in}{0.548769in}}{\pgfqpoint{5.062926in}{3.102590in}}%
\pgfusepath{clip}%
\pgfsetbuttcap%
\pgfsetroundjoin%
\definecolor{currentfill}{rgb}{0.121569,0.466667,0.705882}%
\pgfsetfillcolor{currentfill}%
\pgfsetlinewidth{1.003750pt}%
\definecolor{currentstroke}{rgb}{0.121569,0.466667,0.705882}%
\pgfsetstrokecolor{currentstroke}%
\pgfsetdash{}{0pt}%
\pgfpathmoveto{\pgfqpoint{1.608977in}{2.073153in}}%
\pgfpathcurveto{\pgfqpoint{1.620028in}{2.073153in}}{\pgfqpoint{1.630627in}{2.077543in}}{\pgfqpoint{1.638440in}{2.085357in}}%
\pgfpathcurveto{\pgfqpoint{1.646254in}{2.093170in}}{\pgfqpoint{1.650644in}{2.103769in}}{\pgfqpoint{1.650644in}{2.114820in}}%
\pgfpathcurveto{\pgfqpoint{1.650644in}{2.125870in}}{\pgfqpoint{1.646254in}{2.136469in}}{\pgfqpoint{1.638440in}{2.144282in}}%
\pgfpathcurveto{\pgfqpoint{1.630627in}{2.152096in}}{\pgfqpoint{1.620028in}{2.156486in}}{\pgfqpoint{1.608977in}{2.156486in}}%
\pgfpathcurveto{\pgfqpoint{1.597927in}{2.156486in}}{\pgfqpoint{1.587328in}{2.152096in}}{\pgfqpoint{1.579515in}{2.144282in}}%
\pgfpathcurveto{\pgfqpoint{1.571701in}{2.136469in}}{\pgfqpoint{1.567311in}{2.125870in}}{\pgfqpoint{1.567311in}{2.114820in}}%
\pgfpathcurveto{\pgfqpoint{1.567311in}{2.103769in}}{\pgfqpoint{1.571701in}{2.093170in}}{\pgfqpoint{1.579515in}{2.085357in}}%
\pgfpathcurveto{\pgfqpoint{1.587328in}{2.077543in}}{\pgfqpoint{1.597927in}{2.073153in}}{\pgfqpoint{1.608977in}{2.073153in}}%
\pgfpathclose%
\pgfusepath{stroke,fill}%
\end{pgfscope}%
\begin{pgfscope}%
\pgfpathrectangle{\pgfqpoint{0.787074in}{0.548769in}}{\pgfqpoint{5.062926in}{3.102590in}}%
\pgfusepath{clip}%
\pgfsetbuttcap%
\pgfsetroundjoin%
\definecolor{currentfill}{rgb}{0.121569,0.466667,0.705882}%
\pgfsetfillcolor{currentfill}%
\pgfsetlinewidth{1.003750pt}%
\definecolor{currentstroke}{rgb}{0.121569,0.466667,0.705882}%
\pgfsetstrokecolor{currentstroke}%
\pgfsetdash{}{0pt}%
\pgfpathmoveto{\pgfqpoint{1.411721in}{0.648148in}}%
\pgfpathcurveto{\pgfqpoint{1.422771in}{0.648148in}}{\pgfqpoint{1.433370in}{0.652539in}}{\pgfqpoint{1.441183in}{0.660352in}}%
\pgfpathcurveto{\pgfqpoint{1.448997in}{0.668166in}}{\pgfqpoint{1.453387in}{0.678765in}}{\pgfqpoint{1.453387in}{0.689815in}}%
\pgfpathcurveto{\pgfqpoint{1.453387in}{0.700865in}}{\pgfqpoint{1.448997in}{0.711464in}}{\pgfqpoint{1.441183in}{0.719278in}}%
\pgfpathcurveto{\pgfqpoint{1.433370in}{0.727091in}}{\pgfqpoint{1.422771in}{0.731482in}}{\pgfqpoint{1.411721in}{0.731482in}}%
\pgfpathcurveto{\pgfqpoint{1.400670in}{0.731482in}}{\pgfqpoint{1.390071in}{0.727091in}}{\pgfqpoint{1.382258in}{0.719278in}}%
\pgfpathcurveto{\pgfqpoint{1.374444in}{0.711464in}}{\pgfqpoint{1.370054in}{0.700865in}}{\pgfqpoint{1.370054in}{0.689815in}}%
\pgfpathcurveto{\pgfqpoint{1.370054in}{0.678765in}}{\pgfqpoint{1.374444in}{0.668166in}}{\pgfqpoint{1.382258in}{0.660352in}}%
\pgfpathcurveto{\pgfqpoint{1.390071in}{0.652539in}}{\pgfqpoint{1.400670in}{0.648148in}}{\pgfqpoint{1.411721in}{0.648148in}}%
\pgfpathclose%
\pgfusepath{stroke,fill}%
\end{pgfscope}%
\begin{pgfscope}%
\pgfpathrectangle{\pgfqpoint{0.787074in}{0.548769in}}{\pgfqpoint{5.062926in}{3.102590in}}%
\pgfusepath{clip}%
\pgfsetbuttcap%
\pgfsetroundjoin%
\definecolor{currentfill}{rgb}{0.121569,0.466667,0.705882}%
\pgfsetfillcolor{currentfill}%
\pgfsetlinewidth{1.003750pt}%
\definecolor{currentstroke}{rgb}{0.121569,0.466667,0.705882}%
\pgfsetstrokecolor{currentstroke}%
\pgfsetdash{}{0pt}%
\pgfpathmoveto{\pgfqpoint{1.017207in}{0.648150in}}%
\pgfpathcurveto{\pgfqpoint{1.028257in}{0.648150in}}{\pgfqpoint{1.038856in}{0.652540in}}{\pgfqpoint{1.046670in}{0.660353in}}%
\pgfpathcurveto{\pgfqpoint{1.054483in}{0.668167in}}{\pgfqpoint{1.058874in}{0.678766in}}{\pgfqpoint{1.058874in}{0.689816in}}%
\pgfpathcurveto{\pgfqpoint{1.058874in}{0.700866in}}{\pgfqpoint{1.054483in}{0.711465in}}{\pgfqpoint{1.046670in}{0.719279in}}%
\pgfpathcurveto{\pgfqpoint{1.038856in}{0.727093in}}{\pgfqpoint{1.028257in}{0.731483in}}{\pgfqpoint{1.017207in}{0.731483in}}%
\pgfpathcurveto{\pgfqpoint{1.006157in}{0.731483in}}{\pgfqpoint{0.995558in}{0.727093in}}{\pgfqpoint{0.987744in}{0.719279in}}%
\pgfpathcurveto{\pgfqpoint{0.979930in}{0.711465in}}{\pgfqpoint{0.975540in}{0.700866in}}{\pgfqpoint{0.975540in}{0.689816in}}%
\pgfpathcurveto{\pgfqpoint{0.975540in}{0.678766in}}{\pgfqpoint{0.979930in}{0.668167in}}{\pgfqpoint{0.987744in}{0.660353in}}%
\pgfpathcurveto{\pgfqpoint{0.995558in}{0.652540in}}{\pgfqpoint{1.006157in}{0.648150in}}{\pgfqpoint{1.017207in}{0.648150in}}%
\pgfpathclose%
\pgfusepath{stroke,fill}%
\end{pgfscope}%
\begin{pgfscope}%
\pgfpathrectangle{\pgfqpoint{0.787074in}{0.548769in}}{\pgfqpoint{5.062926in}{3.102590in}}%
\pgfusepath{clip}%
\pgfsetbuttcap%
\pgfsetroundjoin%
\definecolor{currentfill}{rgb}{0.121569,0.466667,0.705882}%
\pgfsetfillcolor{currentfill}%
\pgfsetlinewidth{1.003750pt}%
\definecolor{currentstroke}{rgb}{0.121569,0.466667,0.705882}%
\pgfsetstrokecolor{currentstroke}%
\pgfsetdash{}{0pt}%
\pgfpathmoveto{\pgfqpoint{1.082959in}{2.326735in}}%
\pgfpathcurveto{\pgfqpoint{1.094009in}{2.326735in}}{\pgfqpoint{1.104608in}{2.331125in}}{\pgfqpoint{1.112422in}{2.338939in}}%
\pgfpathcurveto{\pgfqpoint{1.120236in}{2.346753in}}{\pgfqpoint{1.124626in}{2.357352in}}{\pgfqpoint{1.124626in}{2.368402in}}%
\pgfpathcurveto{\pgfqpoint{1.124626in}{2.379452in}}{\pgfqpoint{1.120236in}{2.390051in}}{\pgfqpoint{1.112422in}{2.397865in}}%
\pgfpathcurveto{\pgfqpoint{1.104608in}{2.405678in}}{\pgfqpoint{1.094009in}{2.410068in}}{\pgfqpoint{1.082959in}{2.410068in}}%
\pgfpathcurveto{\pgfqpoint{1.071909in}{2.410068in}}{\pgfqpoint{1.061310in}{2.405678in}}{\pgfqpoint{1.053496in}{2.397865in}}%
\pgfpathcurveto{\pgfqpoint{1.045683in}{2.390051in}}{\pgfqpoint{1.041292in}{2.379452in}}{\pgfqpoint{1.041292in}{2.368402in}}%
\pgfpathcurveto{\pgfqpoint{1.041292in}{2.357352in}}{\pgfqpoint{1.045683in}{2.346753in}}{\pgfqpoint{1.053496in}{2.338939in}}%
\pgfpathcurveto{\pgfqpoint{1.061310in}{2.331125in}}{\pgfqpoint{1.071909in}{2.326735in}}{\pgfqpoint{1.082959in}{2.326735in}}%
\pgfpathclose%
\pgfusepath{stroke,fill}%
\end{pgfscope}%
\begin{pgfscope}%
\pgfpathrectangle{\pgfqpoint{0.787074in}{0.548769in}}{\pgfqpoint{5.062926in}{3.102590in}}%
\pgfusepath{clip}%
\pgfsetbuttcap%
\pgfsetroundjoin%
\definecolor{currentfill}{rgb}{1.000000,0.498039,0.054902}%
\pgfsetfillcolor{currentfill}%
\pgfsetlinewidth{1.003750pt}%
\definecolor{currentstroke}{rgb}{1.000000,0.498039,0.054902}%
\pgfsetstrokecolor{currentstroke}%
\pgfsetdash{}{0pt}%
\pgfpathmoveto{\pgfqpoint{1.937739in}{2.612427in}}%
\pgfpathcurveto{\pgfqpoint{1.948789in}{2.612427in}}{\pgfqpoint{1.959388in}{2.616817in}}{\pgfqpoint{1.967202in}{2.624631in}}%
\pgfpathcurveto{\pgfqpoint{1.975015in}{2.632444in}}{\pgfqpoint{1.979406in}{2.643043in}}{\pgfqpoint{1.979406in}{2.654094in}}%
\pgfpathcurveto{\pgfqpoint{1.979406in}{2.665144in}}{\pgfqpoint{1.975015in}{2.675743in}}{\pgfqpoint{1.967202in}{2.683556in}}%
\pgfpathcurveto{\pgfqpoint{1.959388in}{2.691370in}}{\pgfqpoint{1.948789in}{2.695760in}}{\pgfqpoint{1.937739in}{2.695760in}}%
\pgfpathcurveto{\pgfqpoint{1.926689in}{2.695760in}}{\pgfqpoint{1.916090in}{2.691370in}}{\pgfqpoint{1.908276in}{2.683556in}}%
\pgfpathcurveto{\pgfqpoint{1.900462in}{2.675743in}}{\pgfqpoint{1.896072in}{2.665144in}}{\pgfqpoint{1.896072in}{2.654094in}}%
\pgfpathcurveto{\pgfqpoint{1.896072in}{2.643043in}}{\pgfqpoint{1.900462in}{2.632444in}}{\pgfqpoint{1.908276in}{2.624631in}}%
\pgfpathcurveto{\pgfqpoint{1.916090in}{2.616817in}}{\pgfqpoint{1.926689in}{2.612427in}}{\pgfqpoint{1.937739in}{2.612427in}}%
\pgfpathclose%
\pgfusepath{stroke,fill}%
\end{pgfscope}%
\begin{pgfscope}%
\pgfpathrectangle{\pgfqpoint{0.787074in}{0.548769in}}{\pgfqpoint{5.062926in}{3.102590in}}%
\pgfusepath{clip}%
\pgfsetbuttcap%
\pgfsetroundjoin%
\definecolor{currentfill}{rgb}{1.000000,0.498039,0.054902}%
\pgfsetfillcolor{currentfill}%
\pgfsetlinewidth{1.003750pt}%
\definecolor{currentstroke}{rgb}{1.000000,0.498039,0.054902}%
\pgfsetstrokecolor{currentstroke}%
\pgfsetdash{}{0pt}%
\pgfpathmoveto{\pgfqpoint{1.740482in}{2.467225in}}%
\pgfpathcurveto{\pgfqpoint{1.751532in}{2.467225in}}{\pgfqpoint{1.762131in}{2.471616in}}{\pgfqpoint{1.769945in}{2.479429in}}%
\pgfpathcurveto{\pgfqpoint{1.777758in}{2.487243in}}{\pgfqpoint{1.782149in}{2.497842in}}{\pgfqpoint{1.782149in}{2.508892in}}%
\pgfpathcurveto{\pgfqpoint{1.782149in}{2.519942in}}{\pgfqpoint{1.777758in}{2.530541in}}{\pgfqpoint{1.769945in}{2.538355in}}%
\pgfpathcurveto{\pgfqpoint{1.762131in}{2.546168in}}{\pgfqpoint{1.751532in}{2.550559in}}{\pgfqpoint{1.740482in}{2.550559in}}%
\pgfpathcurveto{\pgfqpoint{1.729432in}{2.550559in}}{\pgfqpoint{1.718833in}{2.546168in}}{\pgfqpoint{1.711019in}{2.538355in}}%
\pgfpathcurveto{\pgfqpoint{1.703206in}{2.530541in}}{\pgfqpoint{1.698815in}{2.519942in}}{\pgfqpoint{1.698815in}{2.508892in}}%
\pgfpathcurveto{\pgfqpoint{1.698815in}{2.497842in}}{\pgfqpoint{1.703206in}{2.487243in}}{\pgfqpoint{1.711019in}{2.479429in}}%
\pgfpathcurveto{\pgfqpoint{1.718833in}{2.471616in}}{\pgfqpoint{1.729432in}{2.467225in}}{\pgfqpoint{1.740482in}{2.467225in}}%
\pgfpathclose%
\pgfusepath{stroke,fill}%
\end{pgfscope}%
\begin{pgfscope}%
\pgfpathrectangle{\pgfqpoint{0.787074in}{0.548769in}}{\pgfqpoint{5.062926in}{3.102590in}}%
\pgfusepath{clip}%
\pgfsetbuttcap%
\pgfsetroundjoin%
\definecolor{currentfill}{rgb}{1.000000,0.498039,0.054902}%
\pgfsetfillcolor{currentfill}%
\pgfsetlinewidth{1.003750pt}%
\definecolor{currentstroke}{rgb}{1.000000,0.498039,0.054902}%
\pgfsetstrokecolor{currentstroke}%
\pgfsetdash{}{0pt}%
\pgfpathmoveto{\pgfqpoint{1.806234in}{2.185530in}}%
\pgfpathcurveto{\pgfqpoint{1.817284in}{2.185530in}}{\pgfqpoint{1.827883in}{2.189921in}}{\pgfqpoint{1.835697in}{2.197734in}}%
\pgfpathcurveto{\pgfqpoint{1.843511in}{2.205548in}}{\pgfqpoint{1.847901in}{2.216147in}}{\pgfqpoint{1.847901in}{2.227197in}}%
\pgfpathcurveto{\pgfqpoint{1.847901in}{2.238247in}}{\pgfqpoint{1.843511in}{2.248846in}}{\pgfqpoint{1.835697in}{2.256660in}}%
\pgfpathcurveto{\pgfqpoint{1.827883in}{2.264473in}}{\pgfqpoint{1.817284in}{2.268864in}}{\pgfqpoint{1.806234in}{2.268864in}}%
\pgfpathcurveto{\pgfqpoint{1.795184in}{2.268864in}}{\pgfqpoint{1.784585in}{2.264473in}}{\pgfqpoint{1.776772in}{2.256660in}}%
\pgfpathcurveto{\pgfqpoint{1.768958in}{2.248846in}}{\pgfqpoint{1.764568in}{2.238247in}}{\pgfqpoint{1.764568in}{2.227197in}}%
\pgfpathcurveto{\pgfqpoint{1.764568in}{2.216147in}}{\pgfqpoint{1.768958in}{2.205548in}}{\pgfqpoint{1.776772in}{2.197734in}}%
\pgfpathcurveto{\pgfqpoint{1.784585in}{2.189921in}}{\pgfqpoint{1.795184in}{2.185530in}}{\pgfqpoint{1.806234in}{2.185530in}}%
\pgfpathclose%
\pgfusepath{stroke,fill}%
\end{pgfscope}%
\begin{pgfscope}%
\pgfpathrectangle{\pgfqpoint{0.787074in}{0.548769in}}{\pgfqpoint{5.062926in}{3.102590in}}%
\pgfusepath{clip}%
\pgfsetbuttcap%
\pgfsetroundjoin%
\definecolor{currentfill}{rgb}{0.121569,0.466667,0.705882}%
\pgfsetfillcolor{currentfill}%
\pgfsetlinewidth{1.003750pt}%
\definecolor{currentstroke}{rgb}{0.121569,0.466667,0.705882}%
\pgfsetstrokecolor{currentstroke}%
\pgfsetdash{}{0pt}%
\pgfpathmoveto{\pgfqpoint{1.608977in}{2.666690in}}%
\pgfpathcurveto{\pgfqpoint{1.620028in}{2.666690in}}{\pgfqpoint{1.630627in}{2.671080in}}{\pgfqpoint{1.638440in}{2.678894in}}%
\pgfpathcurveto{\pgfqpoint{1.646254in}{2.686707in}}{\pgfqpoint{1.650644in}{2.697306in}}{\pgfqpoint{1.650644in}{2.708356in}}%
\pgfpathcurveto{\pgfqpoint{1.650644in}{2.719407in}}{\pgfqpoint{1.646254in}{2.730006in}}{\pgfqpoint{1.638440in}{2.737819in}}%
\pgfpathcurveto{\pgfqpoint{1.630627in}{2.745633in}}{\pgfqpoint{1.620028in}{2.750023in}}{\pgfqpoint{1.608977in}{2.750023in}}%
\pgfpathcurveto{\pgfqpoint{1.597927in}{2.750023in}}{\pgfqpoint{1.587328in}{2.745633in}}{\pgfqpoint{1.579515in}{2.737819in}}%
\pgfpathcurveto{\pgfqpoint{1.571701in}{2.730006in}}{\pgfqpoint{1.567311in}{2.719407in}}{\pgfqpoint{1.567311in}{2.708356in}}%
\pgfpathcurveto{\pgfqpoint{1.567311in}{2.697306in}}{\pgfqpoint{1.571701in}{2.686707in}}{\pgfqpoint{1.579515in}{2.678894in}}%
\pgfpathcurveto{\pgfqpoint{1.587328in}{2.671080in}}{\pgfqpoint{1.597927in}{2.666690in}}{\pgfqpoint{1.608977in}{2.666690in}}%
\pgfpathclose%
\pgfusepath{stroke,fill}%
\end{pgfscope}%
\begin{pgfscope}%
\pgfpathrectangle{\pgfqpoint{0.787074in}{0.548769in}}{\pgfqpoint{5.062926in}{3.102590in}}%
\pgfusepath{clip}%
\pgfsetbuttcap%
\pgfsetroundjoin%
\definecolor{currentfill}{rgb}{1.000000,0.498039,0.054902}%
\pgfsetfillcolor{currentfill}%
\pgfsetlinewidth{1.003750pt}%
\definecolor{currentstroke}{rgb}{1.000000,0.498039,0.054902}%
\pgfsetstrokecolor{currentstroke}%
\pgfsetdash{}{0pt}%
\pgfpathmoveto{\pgfqpoint{1.806234in}{2.209697in}}%
\pgfpathcurveto{\pgfqpoint{1.817284in}{2.209697in}}{\pgfqpoint{1.827883in}{2.214087in}}{\pgfqpoint{1.835697in}{2.221901in}}%
\pgfpathcurveto{\pgfqpoint{1.843511in}{2.229714in}}{\pgfqpoint{1.847901in}{2.240313in}}{\pgfqpoint{1.847901in}{2.251363in}}%
\pgfpathcurveto{\pgfqpoint{1.847901in}{2.262414in}}{\pgfqpoint{1.843511in}{2.273013in}}{\pgfqpoint{1.835697in}{2.280826in}}%
\pgfpathcurveto{\pgfqpoint{1.827883in}{2.288640in}}{\pgfqpoint{1.817284in}{2.293030in}}{\pgfqpoint{1.806234in}{2.293030in}}%
\pgfpathcurveto{\pgfqpoint{1.795184in}{2.293030in}}{\pgfqpoint{1.784585in}{2.288640in}}{\pgfqpoint{1.776772in}{2.280826in}}%
\pgfpathcurveto{\pgfqpoint{1.768958in}{2.273013in}}{\pgfqpoint{1.764568in}{2.262414in}}{\pgfqpoint{1.764568in}{2.251363in}}%
\pgfpathcurveto{\pgfqpoint{1.764568in}{2.240313in}}{\pgfqpoint{1.768958in}{2.229714in}}{\pgfqpoint{1.776772in}{2.221901in}}%
\pgfpathcurveto{\pgfqpoint{1.784585in}{2.214087in}}{\pgfqpoint{1.795184in}{2.209697in}}{\pgfqpoint{1.806234in}{2.209697in}}%
\pgfpathclose%
\pgfusepath{stroke,fill}%
\end{pgfscope}%
\begin{pgfscope}%
\pgfpathrectangle{\pgfqpoint{0.787074in}{0.548769in}}{\pgfqpoint{5.062926in}{3.102590in}}%
\pgfusepath{clip}%
\pgfsetbuttcap%
\pgfsetroundjoin%
\definecolor{currentfill}{rgb}{1.000000,0.498039,0.054902}%
\pgfsetfillcolor{currentfill}%
\pgfsetlinewidth{1.003750pt}%
\definecolor{currentstroke}{rgb}{1.000000,0.498039,0.054902}%
\pgfsetstrokecolor{currentstroke}%
\pgfsetdash{}{0pt}%
\pgfpathmoveto{\pgfqpoint{3.713051in}{1.987916in}}%
\pgfpathcurveto{\pgfqpoint{3.724101in}{1.987916in}}{\pgfqpoint{3.734700in}{1.992306in}}{\pgfqpoint{3.742513in}{2.000120in}}%
\pgfpathcurveto{\pgfqpoint{3.750327in}{2.007934in}}{\pgfqpoint{3.754717in}{2.018533in}}{\pgfqpoint{3.754717in}{2.029583in}}%
\pgfpathcurveto{\pgfqpoint{3.754717in}{2.040633in}}{\pgfqpoint{3.750327in}{2.051232in}}{\pgfqpoint{3.742513in}{2.059045in}}%
\pgfpathcurveto{\pgfqpoint{3.734700in}{2.066859in}}{\pgfqpoint{3.724101in}{2.071249in}}{\pgfqpoint{3.713051in}{2.071249in}}%
\pgfpathcurveto{\pgfqpoint{3.702001in}{2.071249in}}{\pgfqpoint{3.691401in}{2.066859in}}{\pgfqpoint{3.683588in}{2.059045in}}%
\pgfpathcurveto{\pgfqpoint{3.675774in}{2.051232in}}{\pgfqpoint{3.671384in}{2.040633in}}{\pgfqpoint{3.671384in}{2.029583in}}%
\pgfpathcurveto{\pgfqpoint{3.671384in}{2.018533in}}{\pgfqpoint{3.675774in}{2.007934in}}{\pgfqpoint{3.683588in}{2.000120in}}%
\pgfpathcurveto{\pgfqpoint{3.691401in}{1.992306in}}{\pgfqpoint{3.702001in}{1.987916in}}{\pgfqpoint{3.713051in}{1.987916in}}%
\pgfpathclose%
\pgfusepath{stroke,fill}%
\end{pgfscope}%
\begin{pgfscope}%
\pgfpathrectangle{\pgfqpoint{0.787074in}{0.548769in}}{\pgfqpoint{5.062926in}{3.102590in}}%
\pgfusepath{clip}%
\pgfsetbuttcap%
\pgfsetroundjoin%
\definecolor{currentfill}{rgb}{1.000000,0.498039,0.054902}%
\pgfsetfillcolor{currentfill}%
\pgfsetlinewidth{1.003750pt}%
\definecolor{currentstroke}{rgb}{1.000000,0.498039,0.054902}%
\pgfsetstrokecolor{currentstroke}%
\pgfsetdash{}{0pt}%
\pgfpathmoveto{\pgfqpoint{2.726766in}{2.610441in}}%
\pgfpathcurveto{\pgfqpoint{2.737816in}{2.610441in}}{\pgfqpoint{2.748415in}{2.614831in}}{\pgfqpoint{2.756229in}{2.622645in}}%
\pgfpathcurveto{\pgfqpoint{2.764043in}{2.630459in}}{\pgfqpoint{2.768433in}{2.641058in}}{\pgfqpoint{2.768433in}{2.652108in}}%
\pgfpathcurveto{\pgfqpoint{2.768433in}{2.663158in}}{\pgfqpoint{2.764043in}{2.673757in}}{\pgfqpoint{2.756229in}{2.681571in}}%
\pgfpathcurveto{\pgfqpoint{2.748415in}{2.689384in}}{\pgfqpoint{2.737816in}{2.693774in}}{\pgfqpoint{2.726766in}{2.693774in}}%
\pgfpathcurveto{\pgfqpoint{2.715716in}{2.693774in}}{\pgfqpoint{2.705117in}{2.689384in}}{\pgfqpoint{2.697304in}{2.681571in}}%
\pgfpathcurveto{\pgfqpoint{2.689490in}{2.673757in}}{\pgfqpoint{2.685100in}{2.663158in}}{\pgfqpoint{2.685100in}{2.652108in}}%
\pgfpathcurveto{\pgfqpoint{2.685100in}{2.641058in}}{\pgfqpoint{2.689490in}{2.630459in}}{\pgfqpoint{2.697304in}{2.622645in}}%
\pgfpathcurveto{\pgfqpoint{2.705117in}{2.614831in}}{\pgfqpoint{2.715716in}{2.610441in}}{\pgfqpoint{2.726766in}{2.610441in}}%
\pgfpathclose%
\pgfusepath{stroke,fill}%
\end{pgfscope}%
\begin{pgfscope}%
\pgfpathrectangle{\pgfqpoint{0.787074in}{0.548769in}}{\pgfqpoint{5.062926in}{3.102590in}}%
\pgfusepath{clip}%
\pgfsetbuttcap%
\pgfsetroundjoin%
\definecolor{currentfill}{rgb}{1.000000,0.498039,0.054902}%
\pgfsetfillcolor{currentfill}%
\pgfsetlinewidth{1.003750pt}%
\definecolor{currentstroke}{rgb}{1.000000,0.498039,0.054902}%
\pgfsetstrokecolor{currentstroke}%
\pgfsetdash{}{0pt}%
\pgfpathmoveto{\pgfqpoint{2.003491in}{1.913221in}}%
\pgfpathcurveto{\pgfqpoint{2.014541in}{1.913221in}}{\pgfqpoint{2.025140in}{1.917612in}}{\pgfqpoint{2.032954in}{1.925425in}}%
\pgfpathcurveto{\pgfqpoint{2.040768in}{1.933239in}}{\pgfqpoint{2.045158in}{1.943838in}}{\pgfqpoint{2.045158in}{1.954888in}}%
\pgfpathcurveto{\pgfqpoint{2.045158in}{1.965938in}}{\pgfqpoint{2.040768in}{1.976537in}}{\pgfqpoint{2.032954in}{1.984351in}}%
\pgfpathcurveto{\pgfqpoint{2.025140in}{1.992164in}}{\pgfqpoint{2.014541in}{1.996555in}}{\pgfqpoint{2.003491in}{1.996555in}}%
\pgfpathcurveto{\pgfqpoint{1.992441in}{1.996555in}}{\pgfqpoint{1.981842in}{1.992164in}}{\pgfqpoint{1.974028in}{1.984351in}}%
\pgfpathcurveto{\pgfqpoint{1.966215in}{1.976537in}}{\pgfqpoint{1.961824in}{1.965938in}}{\pgfqpoint{1.961824in}{1.954888in}}%
\pgfpathcurveto{\pgfqpoint{1.961824in}{1.943838in}}{\pgfqpoint{1.966215in}{1.933239in}}{\pgfqpoint{1.974028in}{1.925425in}}%
\pgfpathcurveto{\pgfqpoint{1.981842in}{1.917612in}}{\pgfqpoint{1.992441in}{1.913221in}}{\pgfqpoint{2.003491in}{1.913221in}}%
\pgfpathclose%
\pgfusepath{stroke,fill}%
\end{pgfscope}%
\begin{pgfscope}%
\pgfpathrectangle{\pgfqpoint{0.787074in}{0.548769in}}{\pgfqpoint{5.062926in}{3.102590in}}%
\pgfusepath{clip}%
\pgfsetbuttcap%
\pgfsetroundjoin%
\definecolor{currentfill}{rgb}{1.000000,0.498039,0.054902}%
\pgfsetfillcolor{currentfill}%
\pgfsetlinewidth{1.003750pt}%
\definecolor{currentstroke}{rgb}{1.000000,0.498039,0.054902}%
\pgfsetstrokecolor{currentstroke}%
\pgfsetdash{}{0pt}%
\pgfpathmoveto{\pgfqpoint{1.740482in}{2.130198in}}%
\pgfpathcurveto{\pgfqpoint{1.751532in}{2.130198in}}{\pgfqpoint{1.762131in}{2.134588in}}{\pgfqpoint{1.769945in}{2.142402in}}%
\pgfpathcurveto{\pgfqpoint{1.777758in}{2.150216in}}{\pgfqpoint{1.782149in}{2.160815in}}{\pgfqpoint{1.782149in}{2.171865in}}%
\pgfpathcurveto{\pgfqpoint{1.782149in}{2.182915in}}{\pgfqpoint{1.777758in}{2.193514in}}{\pgfqpoint{1.769945in}{2.201327in}}%
\pgfpathcurveto{\pgfqpoint{1.762131in}{2.209141in}}{\pgfqpoint{1.751532in}{2.213531in}}{\pgfqpoint{1.740482in}{2.213531in}}%
\pgfpathcurveto{\pgfqpoint{1.729432in}{2.213531in}}{\pgfqpoint{1.718833in}{2.209141in}}{\pgfqpoint{1.711019in}{2.201327in}}%
\pgfpathcurveto{\pgfqpoint{1.703206in}{2.193514in}}{\pgfqpoint{1.698815in}{2.182915in}}{\pgfqpoint{1.698815in}{2.171865in}}%
\pgfpathcurveto{\pgfqpoint{1.698815in}{2.160815in}}{\pgfqpoint{1.703206in}{2.150216in}}{\pgfqpoint{1.711019in}{2.142402in}}%
\pgfpathcurveto{\pgfqpoint{1.718833in}{2.134588in}}{\pgfqpoint{1.729432in}{2.130198in}}{\pgfqpoint{1.740482in}{2.130198in}}%
\pgfpathclose%
\pgfusepath{stroke,fill}%
\end{pgfscope}%
\begin{pgfscope}%
\pgfpathrectangle{\pgfqpoint{0.787074in}{0.548769in}}{\pgfqpoint{5.062926in}{3.102590in}}%
\pgfusepath{clip}%
\pgfsetbuttcap%
\pgfsetroundjoin%
\definecolor{currentfill}{rgb}{1.000000,0.498039,0.054902}%
\pgfsetfillcolor{currentfill}%
\pgfsetlinewidth{1.003750pt}%
\definecolor{currentstroke}{rgb}{1.000000,0.498039,0.054902}%
\pgfsetstrokecolor{currentstroke}%
\pgfsetdash{}{0pt}%
\pgfpathmoveto{\pgfqpoint{2.134996in}{1.747432in}}%
\pgfpathcurveto{\pgfqpoint{2.146046in}{1.747432in}}{\pgfqpoint{2.156645in}{1.751822in}}{\pgfqpoint{2.164459in}{1.759636in}}%
\pgfpathcurveto{\pgfqpoint{2.172272in}{1.767449in}}{\pgfqpoint{2.176662in}{1.778048in}}{\pgfqpoint{2.176662in}{1.789098in}}%
\pgfpathcurveto{\pgfqpoint{2.176662in}{1.800148in}}{\pgfqpoint{2.172272in}{1.810748in}}{\pgfqpoint{2.164459in}{1.818561in}}%
\pgfpathcurveto{\pgfqpoint{2.156645in}{1.826375in}}{\pgfqpoint{2.146046in}{1.830765in}}{\pgfqpoint{2.134996in}{1.830765in}}%
\pgfpathcurveto{\pgfqpoint{2.123946in}{1.830765in}}{\pgfqpoint{2.113347in}{1.826375in}}{\pgfqpoint{2.105533in}{1.818561in}}%
\pgfpathcurveto{\pgfqpoint{2.097719in}{1.810748in}}{\pgfqpoint{2.093329in}{1.800148in}}{\pgfqpoint{2.093329in}{1.789098in}}%
\pgfpathcurveto{\pgfqpoint{2.093329in}{1.778048in}}{\pgfqpoint{2.097719in}{1.767449in}}{\pgfqpoint{2.105533in}{1.759636in}}%
\pgfpathcurveto{\pgfqpoint{2.113347in}{1.751822in}}{\pgfqpoint{2.123946in}{1.747432in}}{\pgfqpoint{2.134996in}{1.747432in}}%
\pgfpathclose%
\pgfusepath{stroke,fill}%
\end{pgfscope}%
\begin{pgfscope}%
\pgfpathrectangle{\pgfqpoint{0.787074in}{0.548769in}}{\pgfqpoint{5.062926in}{3.102590in}}%
\pgfusepath{clip}%
\pgfsetbuttcap%
\pgfsetroundjoin%
\definecolor{currentfill}{rgb}{1.000000,0.498039,0.054902}%
\pgfsetfillcolor{currentfill}%
\pgfsetlinewidth{1.003750pt}%
\definecolor{currentstroke}{rgb}{1.000000,0.498039,0.054902}%
\pgfsetstrokecolor{currentstroke}%
\pgfsetdash{}{0pt}%
\pgfpathmoveto{\pgfqpoint{1.806234in}{2.135660in}}%
\pgfpathcurveto{\pgfqpoint{1.817284in}{2.135660in}}{\pgfqpoint{1.827883in}{2.140050in}}{\pgfqpoint{1.835697in}{2.147864in}}%
\pgfpathcurveto{\pgfqpoint{1.843511in}{2.155678in}}{\pgfqpoint{1.847901in}{2.166277in}}{\pgfqpoint{1.847901in}{2.177327in}}%
\pgfpathcurveto{\pgfqpoint{1.847901in}{2.188377in}}{\pgfqpoint{1.843511in}{2.198976in}}{\pgfqpoint{1.835697in}{2.206789in}}%
\pgfpathcurveto{\pgfqpoint{1.827883in}{2.214603in}}{\pgfqpoint{1.817284in}{2.218993in}}{\pgfqpoint{1.806234in}{2.218993in}}%
\pgfpathcurveto{\pgfqpoint{1.795184in}{2.218993in}}{\pgfqpoint{1.784585in}{2.214603in}}{\pgfqpoint{1.776772in}{2.206789in}}%
\pgfpathcurveto{\pgfqpoint{1.768958in}{2.198976in}}{\pgfqpoint{1.764568in}{2.188377in}}{\pgfqpoint{1.764568in}{2.177327in}}%
\pgfpathcurveto{\pgfqpoint{1.764568in}{2.166277in}}{\pgfqpoint{1.768958in}{2.155678in}}{\pgfqpoint{1.776772in}{2.147864in}}%
\pgfpathcurveto{\pgfqpoint{1.784585in}{2.140050in}}{\pgfqpoint{1.795184in}{2.135660in}}{\pgfqpoint{1.806234in}{2.135660in}}%
\pgfpathclose%
\pgfusepath{stroke,fill}%
\end{pgfscope}%
\begin{pgfscope}%
\pgfpathrectangle{\pgfqpoint{0.787074in}{0.548769in}}{\pgfqpoint{5.062926in}{3.102590in}}%
\pgfusepath{clip}%
\pgfsetbuttcap%
\pgfsetroundjoin%
\definecolor{currentfill}{rgb}{1.000000,0.498039,0.054902}%
\pgfsetfillcolor{currentfill}%
\pgfsetlinewidth{1.003750pt}%
\definecolor{currentstroke}{rgb}{1.000000,0.498039,0.054902}%
\pgfsetstrokecolor{currentstroke}%
\pgfsetdash{}{0pt}%
\pgfpathmoveto{\pgfqpoint{1.871987in}{1.976501in}}%
\pgfpathcurveto{\pgfqpoint{1.883037in}{1.976501in}}{\pgfqpoint{1.893636in}{1.980891in}}{\pgfqpoint{1.901449in}{1.988704in}}%
\pgfpathcurveto{\pgfqpoint{1.909263in}{1.996518in}}{\pgfqpoint{1.913653in}{2.007117in}}{\pgfqpoint{1.913653in}{2.018167in}}%
\pgfpathcurveto{\pgfqpoint{1.913653in}{2.029217in}}{\pgfqpoint{1.909263in}{2.039816in}}{\pgfqpoint{1.901449in}{2.047630in}}%
\pgfpathcurveto{\pgfqpoint{1.893636in}{2.055444in}}{\pgfqpoint{1.883037in}{2.059834in}}{\pgfqpoint{1.871987in}{2.059834in}}%
\pgfpathcurveto{\pgfqpoint{1.860936in}{2.059834in}}{\pgfqpoint{1.850337in}{2.055444in}}{\pgfqpoint{1.842524in}{2.047630in}}%
\pgfpathcurveto{\pgfqpoint{1.834710in}{2.039816in}}{\pgfqpoint{1.830320in}{2.029217in}}{\pgfqpoint{1.830320in}{2.018167in}}%
\pgfpathcurveto{\pgfqpoint{1.830320in}{2.007117in}}{\pgfqpoint{1.834710in}{1.996518in}}{\pgfqpoint{1.842524in}{1.988704in}}%
\pgfpathcurveto{\pgfqpoint{1.850337in}{1.980891in}}{\pgfqpoint{1.860936in}{1.976501in}}{\pgfqpoint{1.871987in}{1.976501in}}%
\pgfpathclose%
\pgfusepath{stroke,fill}%
\end{pgfscope}%
\begin{pgfscope}%
\pgfpathrectangle{\pgfqpoint{0.787074in}{0.548769in}}{\pgfqpoint{5.062926in}{3.102590in}}%
\pgfusepath{clip}%
\pgfsetbuttcap%
\pgfsetroundjoin%
\definecolor{currentfill}{rgb}{1.000000,0.498039,0.054902}%
\pgfsetfillcolor{currentfill}%
\pgfsetlinewidth{1.003750pt}%
\definecolor{currentstroke}{rgb}{1.000000,0.498039,0.054902}%
\pgfsetstrokecolor{currentstroke}%
\pgfsetdash{}{0pt}%
\pgfpathmoveto{\pgfqpoint{2.398005in}{2.141633in}}%
\pgfpathcurveto{\pgfqpoint{2.409055in}{2.141633in}}{\pgfqpoint{2.419654in}{2.146024in}}{\pgfqpoint{2.427468in}{2.153837in}}%
\pgfpathcurveto{\pgfqpoint{2.435281in}{2.161651in}}{\pgfqpoint{2.439672in}{2.172250in}}{\pgfqpoint{2.439672in}{2.183300in}}%
\pgfpathcurveto{\pgfqpoint{2.439672in}{2.194350in}}{\pgfqpoint{2.435281in}{2.204949in}}{\pgfqpoint{2.427468in}{2.212763in}}%
\pgfpathcurveto{\pgfqpoint{2.419654in}{2.220577in}}{\pgfqpoint{2.409055in}{2.224967in}}{\pgfqpoint{2.398005in}{2.224967in}}%
\pgfpathcurveto{\pgfqpoint{2.386955in}{2.224967in}}{\pgfqpoint{2.376356in}{2.220577in}}{\pgfqpoint{2.368542in}{2.212763in}}%
\pgfpathcurveto{\pgfqpoint{2.360728in}{2.204949in}}{\pgfqpoint{2.356338in}{2.194350in}}{\pgfqpoint{2.356338in}{2.183300in}}%
\pgfpathcurveto{\pgfqpoint{2.356338in}{2.172250in}}{\pgfqpoint{2.360728in}{2.161651in}}{\pgfqpoint{2.368542in}{2.153837in}}%
\pgfpathcurveto{\pgfqpoint{2.376356in}{2.146024in}}{\pgfqpoint{2.386955in}{2.141633in}}{\pgfqpoint{2.398005in}{2.141633in}}%
\pgfpathclose%
\pgfusepath{stroke,fill}%
\end{pgfscope}%
\begin{pgfscope}%
\pgfpathrectangle{\pgfqpoint{0.787074in}{0.548769in}}{\pgfqpoint{5.062926in}{3.102590in}}%
\pgfusepath{clip}%
\pgfsetbuttcap%
\pgfsetroundjoin%
\definecolor{currentfill}{rgb}{1.000000,0.498039,0.054902}%
\pgfsetfillcolor{currentfill}%
\pgfsetlinewidth{1.003750pt}%
\definecolor{currentstroke}{rgb}{1.000000,0.498039,0.054902}%
\pgfsetstrokecolor{currentstroke}%
\pgfsetdash{}{0pt}%
\pgfpathmoveto{\pgfqpoint{2.398005in}{2.039016in}}%
\pgfpathcurveto{\pgfqpoint{2.409055in}{2.039016in}}{\pgfqpoint{2.419654in}{2.043407in}}{\pgfqpoint{2.427468in}{2.051220in}}%
\pgfpathcurveto{\pgfqpoint{2.435281in}{2.059034in}}{\pgfqpoint{2.439672in}{2.069633in}}{\pgfqpoint{2.439672in}{2.080683in}}%
\pgfpathcurveto{\pgfqpoint{2.439672in}{2.091733in}}{\pgfqpoint{2.435281in}{2.102332in}}{\pgfqpoint{2.427468in}{2.110146in}}%
\pgfpathcurveto{\pgfqpoint{2.419654in}{2.117960in}}{\pgfqpoint{2.409055in}{2.122350in}}{\pgfqpoint{2.398005in}{2.122350in}}%
\pgfpathcurveto{\pgfqpoint{2.386955in}{2.122350in}}{\pgfqpoint{2.376356in}{2.117960in}}{\pgfqpoint{2.368542in}{2.110146in}}%
\pgfpathcurveto{\pgfqpoint{2.360728in}{2.102332in}}{\pgfqpoint{2.356338in}{2.091733in}}{\pgfqpoint{2.356338in}{2.080683in}}%
\pgfpathcurveto{\pgfqpoint{2.356338in}{2.069633in}}{\pgfqpoint{2.360728in}{2.059034in}}{\pgfqpoint{2.368542in}{2.051220in}}%
\pgfpathcurveto{\pgfqpoint{2.376356in}{2.043407in}}{\pgfqpoint{2.386955in}{2.039016in}}{\pgfqpoint{2.398005in}{2.039016in}}%
\pgfpathclose%
\pgfusepath{stroke,fill}%
\end{pgfscope}%
\begin{pgfscope}%
\pgfpathrectangle{\pgfqpoint{0.787074in}{0.548769in}}{\pgfqpoint{5.062926in}{3.102590in}}%
\pgfusepath{clip}%
\pgfsetbuttcap%
\pgfsetroundjoin%
\definecolor{currentfill}{rgb}{0.121569,0.466667,0.705882}%
\pgfsetfillcolor{currentfill}%
\pgfsetlinewidth{1.003750pt}%
\definecolor{currentstroke}{rgb}{0.121569,0.466667,0.705882}%
\pgfsetstrokecolor{currentstroke}%
\pgfsetdash{}{0pt}%
\pgfpathmoveto{\pgfqpoint{3.121280in}{1.981410in}}%
\pgfpathcurveto{\pgfqpoint{3.132330in}{1.981410in}}{\pgfqpoint{3.142929in}{1.985801in}}{\pgfqpoint{3.150743in}{1.993614in}}%
\pgfpathcurveto{\pgfqpoint{3.158556in}{2.001428in}}{\pgfqpoint{3.162947in}{2.012027in}}{\pgfqpoint{3.162947in}{2.023077in}}%
\pgfpathcurveto{\pgfqpoint{3.162947in}{2.034127in}}{\pgfqpoint{3.158556in}{2.044726in}}{\pgfqpoint{3.150743in}{2.052540in}}%
\pgfpathcurveto{\pgfqpoint{3.142929in}{2.060353in}}{\pgfqpoint{3.132330in}{2.064744in}}{\pgfqpoint{3.121280in}{2.064744in}}%
\pgfpathcurveto{\pgfqpoint{3.110230in}{2.064744in}}{\pgfqpoint{3.099631in}{2.060353in}}{\pgfqpoint{3.091817in}{2.052540in}}%
\pgfpathcurveto{\pgfqpoint{3.084004in}{2.044726in}}{\pgfqpoint{3.079613in}{2.034127in}}{\pgfqpoint{3.079613in}{2.023077in}}%
\pgfpathcurveto{\pgfqpoint{3.079613in}{2.012027in}}{\pgfqpoint{3.084004in}{2.001428in}}{\pgfqpoint{3.091817in}{1.993614in}}%
\pgfpathcurveto{\pgfqpoint{3.099631in}{1.985801in}}{\pgfqpoint{3.110230in}{1.981410in}}{\pgfqpoint{3.121280in}{1.981410in}}%
\pgfpathclose%
\pgfusepath{stroke,fill}%
\end{pgfscope}%
\begin{pgfscope}%
\pgfpathrectangle{\pgfqpoint{0.787074in}{0.548769in}}{\pgfqpoint{5.062926in}{3.102590in}}%
\pgfusepath{clip}%
\pgfsetbuttcap%
\pgfsetroundjoin%
\definecolor{currentfill}{rgb}{1.000000,0.498039,0.054902}%
\pgfsetfillcolor{currentfill}%
\pgfsetlinewidth{1.003750pt}%
\definecolor{currentstroke}{rgb}{1.000000,0.498039,0.054902}%
\pgfsetstrokecolor{currentstroke}%
\pgfsetdash{}{0pt}%
\pgfpathmoveto{\pgfqpoint{1.806234in}{2.386079in}}%
\pgfpathcurveto{\pgfqpoint{1.817284in}{2.386079in}}{\pgfqpoint{1.827883in}{2.390469in}}{\pgfqpoint{1.835697in}{2.398283in}}%
\pgfpathcurveto{\pgfqpoint{1.843511in}{2.406097in}}{\pgfqpoint{1.847901in}{2.416696in}}{\pgfqpoint{1.847901in}{2.427746in}}%
\pgfpathcurveto{\pgfqpoint{1.847901in}{2.438796in}}{\pgfqpoint{1.843511in}{2.449395in}}{\pgfqpoint{1.835697in}{2.457209in}}%
\pgfpathcurveto{\pgfqpoint{1.827883in}{2.465022in}}{\pgfqpoint{1.817284in}{2.469412in}}{\pgfqpoint{1.806234in}{2.469412in}}%
\pgfpathcurveto{\pgfqpoint{1.795184in}{2.469412in}}{\pgfqpoint{1.784585in}{2.465022in}}{\pgfqpoint{1.776772in}{2.457209in}}%
\pgfpathcurveto{\pgfqpoint{1.768958in}{2.449395in}}{\pgfqpoint{1.764568in}{2.438796in}}{\pgfqpoint{1.764568in}{2.427746in}}%
\pgfpathcurveto{\pgfqpoint{1.764568in}{2.416696in}}{\pgfqpoint{1.768958in}{2.406097in}}{\pgfqpoint{1.776772in}{2.398283in}}%
\pgfpathcurveto{\pgfqpoint{1.784585in}{2.390469in}}{\pgfqpoint{1.795184in}{2.386079in}}{\pgfqpoint{1.806234in}{2.386079in}}%
\pgfpathclose%
\pgfusepath{stroke,fill}%
\end{pgfscope}%
\begin{pgfscope}%
\pgfpathrectangle{\pgfqpoint{0.787074in}{0.548769in}}{\pgfqpoint{5.062926in}{3.102590in}}%
\pgfusepath{clip}%
\pgfsetbuttcap%
\pgfsetroundjoin%
\definecolor{currentfill}{rgb}{1.000000,0.498039,0.054902}%
\pgfsetfillcolor{currentfill}%
\pgfsetlinewidth{1.003750pt}%
\definecolor{currentstroke}{rgb}{1.000000,0.498039,0.054902}%
\pgfsetstrokecolor{currentstroke}%
\pgfsetdash{}{0pt}%
\pgfpathmoveto{\pgfqpoint{2.332253in}{2.446539in}}%
\pgfpathcurveto{\pgfqpoint{2.343303in}{2.446539in}}{\pgfqpoint{2.353902in}{2.450929in}}{\pgfqpoint{2.361715in}{2.458743in}}%
\pgfpathcurveto{\pgfqpoint{2.369529in}{2.466556in}}{\pgfqpoint{2.373919in}{2.477155in}}{\pgfqpoint{2.373919in}{2.488206in}}%
\pgfpathcurveto{\pgfqpoint{2.373919in}{2.499256in}}{\pgfqpoint{2.369529in}{2.509855in}}{\pgfqpoint{2.361715in}{2.517668in}}%
\pgfpathcurveto{\pgfqpoint{2.353902in}{2.525482in}}{\pgfqpoint{2.343303in}{2.529872in}}{\pgfqpoint{2.332253in}{2.529872in}}%
\pgfpathcurveto{\pgfqpoint{2.321202in}{2.529872in}}{\pgfqpoint{2.310603in}{2.525482in}}{\pgfqpoint{2.302790in}{2.517668in}}%
\pgfpathcurveto{\pgfqpoint{2.294976in}{2.509855in}}{\pgfqpoint{2.290586in}{2.499256in}}{\pgfqpoint{2.290586in}{2.488206in}}%
\pgfpathcurveto{\pgfqpoint{2.290586in}{2.477155in}}{\pgfqpoint{2.294976in}{2.466556in}}{\pgfqpoint{2.302790in}{2.458743in}}%
\pgfpathcurveto{\pgfqpoint{2.310603in}{2.450929in}}{\pgfqpoint{2.321202in}{2.446539in}}{\pgfqpoint{2.332253in}{2.446539in}}%
\pgfpathclose%
\pgfusepath{stroke,fill}%
\end{pgfscope}%
\begin{pgfscope}%
\pgfpathrectangle{\pgfqpoint{0.787074in}{0.548769in}}{\pgfqpoint{5.062926in}{3.102590in}}%
\pgfusepath{clip}%
\pgfsetbuttcap%
\pgfsetroundjoin%
\definecolor{currentfill}{rgb}{1.000000,0.498039,0.054902}%
\pgfsetfillcolor{currentfill}%
\pgfsetlinewidth{1.003750pt}%
\definecolor{currentstroke}{rgb}{1.000000,0.498039,0.054902}%
\pgfsetstrokecolor{currentstroke}%
\pgfsetdash{}{0pt}%
\pgfpathmoveto{\pgfqpoint{2.661014in}{2.604778in}}%
\pgfpathcurveto{\pgfqpoint{2.672064in}{2.604778in}}{\pgfqpoint{2.682663in}{2.609168in}}{\pgfqpoint{2.690477in}{2.616982in}}%
\pgfpathcurveto{\pgfqpoint{2.698290in}{2.624795in}}{\pgfqpoint{2.702681in}{2.635394in}}{\pgfqpoint{2.702681in}{2.646444in}}%
\pgfpathcurveto{\pgfqpoint{2.702681in}{2.657495in}}{\pgfqpoint{2.698290in}{2.668094in}}{\pgfqpoint{2.690477in}{2.675907in}}%
\pgfpathcurveto{\pgfqpoint{2.682663in}{2.683721in}}{\pgfqpoint{2.672064in}{2.688111in}}{\pgfqpoint{2.661014in}{2.688111in}}%
\pgfpathcurveto{\pgfqpoint{2.649964in}{2.688111in}}{\pgfqpoint{2.639365in}{2.683721in}}{\pgfqpoint{2.631551in}{2.675907in}}%
\pgfpathcurveto{\pgfqpoint{2.623738in}{2.668094in}}{\pgfqpoint{2.619347in}{2.657495in}}{\pgfqpoint{2.619347in}{2.646444in}}%
\pgfpathcurveto{\pgfqpoint{2.619347in}{2.635394in}}{\pgfqpoint{2.623738in}{2.624795in}}{\pgfqpoint{2.631551in}{2.616982in}}%
\pgfpathcurveto{\pgfqpoint{2.639365in}{2.609168in}}{\pgfqpoint{2.649964in}{2.604778in}}{\pgfqpoint{2.661014in}{2.604778in}}%
\pgfpathclose%
\pgfusepath{stroke,fill}%
\end{pgfscope}%
\begin{pgfscope}%
\pgfpathrectangle{\pgfqpoint{0.787074in}{0.548769in}}{\pgfqpoint{5.062926in}{3.102590in}}%
\pgfusepath{clip}%
\pgfsetbuttcap%
\pgfsetroundjoin%
\definecolor{currentfill}{rgb}{1.000000,0.498039,0.054902}%
\pgfsetfillcolor{currentfill}%
\pgfsetlinewidth{1.003750pt}%
\definecolor{currentstroke}{rgb}{1.000000,0.498039,0.054902}%
\pgfsetstrokecolor{currentstroke}%
\pgfsetdash{}{0pt}%
\pgfpathmoveto{\pgfqpoint{1.608977in}{2.291595in}}%
\pgfpathcurveto{\pgfqpoint{1.620028in}{2.291595in}}{\pgfqpoint{1.630627in}{2.295985in}}{\pgfqpoint{1.638440in}{2.303799in}}%
\pgfpathcurveto{\pgfqpoint{1.646254in}{2.311612in}}{\pgfqpoint{1.650644in}{2.322211in}}{\pgfqpoint{1.650644in}{2.333261in}}%
\pgfpathcurveto{\pgfqpoint{1.650644in}{2.344312in}}{\pgfqpoint{1.646254in}{2.354911in}}{\pgfqpoint{1.638440in}{2.362724in}}%
\pgfpathcurveto{\pgfqpoint{1.630627in}{2.370538in}}{\pgfqpoint{1.620028in}{2.374928in}}{\pgfqpoint{1.608977in}{2.374928in}}%
\pgfpathcurveto{\pgfqpoint{1.597927in}{2.374928in}}{\pgfqpoint{1.587328in}{2.370538in}}{\pgfqpoint{1.579515in}{2.362724in}}%
\pgfpathcurveto{\pgfqpoint{1.571701in}{2.354911in}}{\pgfqpoint{1.567311in}{2.344312in}}{\pgfqpoint{1.567311in}{2.333261in}}%
\pgfpathcurveto{\pgfqpoint{1.567311in}{2.322211in}}{\pgfqpoint{1.571701in}{2.311612in}}{\pgfqpoint{1.579515in}{2.303799in}}%
\pgfpathcurveto{\pgfqpoint{1.587328in}{2.295985in}}{\pgfqpoint{1.597927in}{2.291595in}}{\pgfqpoint{1.608977in}{2.291595in}}%
\pgfpathclose%
\pgfusepath{stroke,fill}%
\end{pgfscope}%
\begin{pgfscope}%
\pgfpathrectangle{\pgfqpoint{0.787074in}{0.548769in}}{\pgfqpoint{5.062926in}{3.102590in}}%
\pgfusepath{clip}%
\pgfsetbuttcap%
\pgfsetroundjoin%
\definecolor{currentfill}{rgb}{0.121569,0.466667,0.705882}%
\pgfsetfillcolor{currentfill}%
\pgfsetlinewidth{1.003750pt}%
\definecolor{currentstroke}{rgb}{0.121569,0.466667,0.705882}%
\pgfsetstrokecolor{currentstroke}%
\pgfsetdash{}{0pt}%
\pgfpathmoveto{\pgfqpoint{2.726766in}{1.765305in}}%
\pgfpathcurveto{\pgfqpoint{2.737816in}{1.765305in}}{\pgfqpoint{2.748415in}{1.769696in}}{\pgfqpoint{2.756229in}{1.777509in}}%
\pgfpathcurveto{\pgfqpoint{2.764043in}{1.785323in}}{\pgfqpoint{2.768433in}{1.795922in}}{\pgfqpoint{2.768433in}{1.806972in}}%
\pgfpathcurveto{\pgfqpoint{2.768433in}{1.818022in}}{\pgfqpoint{2.764043in}{1.828621in}}{\pgfqpoint{2.756229in}{1.836435in}}%
\pgfpathcurveto{\pgfqpoint{2.748415in}{1.844249in}}{\pgfqpoint{2.737816in}{1.848639in}}{\pgfqpoint{2.726766in}{1.848639in}}%
\pgfpathcurveto{\pgfqpoint{2.715716in}{1.848639in}}{\pgfqpoint{2.705117in}{1.844249in}}{\pgfqpoint{2.697304in}{1.836435in}}%
\pgfpathcurveto{\pgfqpoint{2.689490in}{1.828621in}}{\pgfqpoint{2.685100in}{1.818022in}}{\pgfqpoint{2.685100in}{1.806972in}}%
\pgfpathcurveto{\pgfqpoint{2.685100in}{1.795922in}}{\pgfqpoint{2.689490in}{1.785323in}}{\pgfqpoint{2.697304in}{1.777509in}}%
\pgfpathcurveto{\pgfqpoint{2.705117in}{1.769696in}}{\pgfqpoint{2.715716in}{1.765305in}}{\pgfqpoint{2.726766in}{1.765305in}}%
\pgfpathclose%
\pgfusepath{stroke,fill}%
\end{pgfscope}%
\begin{pgfscope}%
\pgfpathrectangle{\pgfqpoint{0.787074in}{0.548769in}}{\pgfqpoint{5.062926in}{3.102590in}}%
\pgfusepath{clip}%
\pgfsetbuttcap%
\pgfsetroundjoin%
\definecolor{currentfill}{rgb}{1.000000,0.498039,0.054902}%
\pgfsetfillcolor{currentfill}%
\pgfsetlinewidth{1.003750pt}%
\definecolor{currentstroke}{rgb}{1.000000,0.498039,0.054902}%
\pgfsetstrokecolor{currentstroke}%
\pgfsetdash{}{0pt}%
\pgfpathmoveto{\pgfqpoint{2.858271in}{2.168495in}}%
\pgfpathcurveto{\pgfqpoint{2.869321in}{2.168495in}}{\pgfqpoint{2.879920in}{2.172885in}}{\pgfqpoint{2.887734in}{2.180698in}}%
\pgfpathcurveto{\pgfqpoint{2.895547in}{2.188512in}}{\pgfqpoint{2.899938in}{2.199111in}}{\pgfqpoint{2.899938in}{2.210161in}}%
\pgfpathcurveto{\pgfqpoint{2.899938in}{2.221211in}}{\pgfqpoint{2.895547in}{2.231810in}}{\pgfqpoint{2.887734in}{2.239624in}}%
\pgfpathcurveto{\pgfqpoint{2.879920in}{2.247438in}}{\pgfqpoint{2.869321in}{2.251828in}}{\pgfqpoint{2.858271in}{2.251828in}}%
\pgfpathcurveto{\pgfqpoint{2.847221in}{2.251828in}}{\pgfqpoint{2.836622in}{2.247438in}}{\pgfqpoint{2.828808in}{2.239624in}}%
\pgfpathcurveto{\pgfqpoint{2.820995in}{2.231810in}}{\pgfqpoint{2.816604in}{2.221211in}}{\pgfqpoint{2.816604in}{2.210161in}}%
\pgfpathcurveto{\pgfqpoint{2.816604in}{2.199111in}}{\pgfqpoint{2.820995in}{2.188512in}}{\pgfqpoint{2.828808in}{2.180698in}}%
\pgfpathcurveto{\pgfqpoint{2.836622in}{2.172885in}}{\pgfqpoint{2.847221in}{2.168495in}}{\pgfqpoint{2.858271in}{2.168495in}}%
\pgfpathclose%
\pgfusepath{stroke,fill}%
\end{pgfscope}%
\begin{pgfscope}%
\pgfpathrectangle{\pgfqpoint{0.787074in}{0.548769in}}{\pgfqpoint{5.062926in}{3.102590in}}%
\pgfusepath{clip}%
\pgfsetbuttcap%
\pgfsetroundjoin%
\definecolor{currentfill}{rgb}{1.000000,0.498039,0.054902}%
\pgfsetfillcolor{currentfill}%
\pgfsetlinewidth{1.003750pt}%
\definecolor{currentstroke}{rgb}{1.000000,0.498039,0.054902}%
\pgfsetstrokecolor{currentstroke}%
\pgfsetdash{}{0pt}%
\pgfpathmoveto{\pgfqpoint{2.332253in}{2.330781in}}%
\pgfpathcurveto{\pgfqpoint{2.343303in}{2.330781in}}{\pgfqpoint{2.353902in}{2.335171in}}{\pgfqpoint{2.361715in}{2.342985in}}%
\pgfpathcurveto{\pgfqpoint{2.369529in}{2.350799in}}{\pgfqpoint{2.373919in}{2.361398in}}{\pgfqpoint{2.373919in}{2.372448in}}%
\pgfpathcurveto{\pgfqpoint{2.373919in}{2.383498in}}{\pgfqpoint{2.369529in}{2.394097in}}{\pgfqpoint{2.361715in}{2.401910in}}%
\pgfpathcurveto{\pgfqpoint{2.353902in}{2.409724in}}{\pgfqpoint{2.343303in}{2.414114in}}{\pgfqpoint{2.332253in}{2.414114in}}%
\pgfpathcurveto{\pgfqpoint{2.321202in}{2.414114in}}{\pgfqpoint{2.310603in}{2.409724in}}{\pgfqpoint{2.302790in}{2.401910in}}%
\pgfpathcurveto{\pgfqpoint{2.294976in}{2.394097in}}{\pgfqpoint{2.290586in}{2.383498in}}{\pgfqpoint{2.290586in}{2.372448in}}%
\pgfpathcurveto{\pgfqpoint{2.290586in}{2.361398in}}{\pgfqpoint{2.294976in}{2.350799in}}{\pgfqpoint{2.302790in}{2.342985in}}%
\pgfpathcurveto{\pgfqpoint{2.310603in}{2.335171in}}{\pgfqpoint{2.321202in}{2.330781in}}{\pgfqpoint{2.332253in}{2.330781in}}%
\pgfpathclose%
\pgfusepath{stroke,fill}%
\end{pgfscope}%
\begin{pgfscope}%
\pgfpathrectangle{\pgfqpoint{0.787074in}{0.548769in}}{\pgfqpoint{5.062926in}{3.102590in}}%
\pgfusepath{clip}%
\pgfsetbuttcap%
\pgfsetroundjoin%
\definecolor{currentfill}{rgb}{1.000000,0.498039,0.054902}%
\pgfsetfillcolor{currentfill}%
\pgfsetlinewidth{1.003750pt}%
\definecolor{currentstroke}{rgb}{1.000000,0.498039,0.054902}%
\pgfsetstrokecolor{currentstroke}%
\pgfsetdash{}{0pt}%
\pgfpathmoveto{\pgfqpoint{2.003491in}{2.341609in}}%
\pgfpathcurveto{\pgfqpoint{2.014541in}{2.341609in}}{\pgfqpoint{2.025140in}{2.345999in}}{\pgfqpoint{2.032954in}{2.353813in}}%
\pgfpathcurveto{\pgfqpoint{2.040768in}{2.361626in}}{\pgfqpoint{2.045158in}{2.372225in}}{\pgfqpoint{2.045158in}{2.383275in}}%
\pgfpathcurveto{\pgfqpoint{2.045158in}{2.394326in}}{\pgfqpoint{2.040768in}{2.404925in}}{\pgfqpoint{2.032954in}{2.412738in}}%
\pgfpathcurveto{\pgfqpoint{2.025140in}{2.420552in}}{\pgfqpoint{2.014541in}{2.424942in}}{\pgfqpoint{2.003491in}{2.424942in}}%
\pgfpathcurveto{\pgfqpoint{1.992441in}{2.424942in}}{\pgfqpoint{1.981842in}{2.420552in}}{\pgfqpoint{1.974028in}{2.412738in}}%
\pgfpathcurveto{\pgfqpoint{1.966215in}{2.404925in}}{\pgfqpoint{1.961824in}{2.394326in}}{\pgfqpoint{1.961824in}{2.383275in}}%
\pgfpathcurveto{\pgfqpoint{1.961824in}{2.372225in}}{\pgfqpoint{1.966215in}{2.361626in}}{\pgfqpoint{1.974028in}{2.353813in}}%
\pgfpathcurveto{\pgfqpoint{1.981842in}{2.345999in}}{\pgfqpoint{1.992441in}{2.341609in}}{\pgfqpoint{2.003491in}{2.341609in}}%
\pgfpathclose%
\pgfusepath{stroke,fill}%
\end{pgfscope}%
\begin{pgfscope}%
\pgfpathrectangle{\pgfqpoint{0.787074in}{0.548769in}}{\pgfqpoint{5.062926in}{3.102590in}}%
\pgfusepath{clip}%
\pgfsetbuttcap%
\pgfsetroundjoin%
\definecolor{currentfill}{rgb}{1.000000,0.498039,0.054902}%
\pgfsetfillcolor{currentfill}%
\pgfsetlinewidth{1.003750pt}%
\definecolor{currentstroke}{rgb}{1.000000,0.498039,0.054902}%
\pgfsetstrokecolor{currentstroke}%
\pgfsetdash{}{0pt}%
\pgfpathmoveto{\pgfqpoint{1.543225in}{2.708305in}}%
\pgfpathcurveto{\pgfqpoint{1.554275in}{2.708305in}}{\pgfqpoint{1.564874in}{2.712695in}}{\pgfqpoint{1.572688in}{2.720509in}}%
\pgfpathcurveto{\pgfqpoint{1.580502in}{2.728322in}}{\pgfqpoint{1.584892in}{2.738921in}}{\pgfqpoint{1.584892in}{2.749972in}}%
\pgfpathcurveto{\pgfqpoint{1.584892in}{2.761022in}}{\pgfqpoint{1.580502in}{2.771621in}}{\pgfqpoint{1.572688in}{2.779434in}}%
\pgfpathcurveto{\pgfqpoint{1.564874in}{2.787248in}}{\pgfqpoint{1.554275in}{2.791638in}}{\pgfqpoint{1.543225in}{2.791638in}}%
\pgfpathcurveto{\pgfqpoint{1.532175in}{2.791638in}}{\pgfqpoint{1.521576in}{2.787248in}}{\pgfqpoint{1.513762in}{2.779434in}}%
\pgfpathcurveto{\pgfqpoint{1.505949in}{2.771621in}}{\pgfqpoint{1.501558in}{2.761022in}}{\pgfqpoint{1.501558in}{2.749972in}}%
\pgfpathcurveto{\pgfqpoint{1.501558in}{2.738921in}}{\pgfqpoint{1.505949in}{2.728322in}}{\pgfqpoint{1.513762in}{2.720509in}}%
\pgfpathcurveto{\pgfqpoint{1.521576in}{2.712695in}}{\pgfqpoint{1.532175in}{2.708305in}}{\pgfqpoint{1.543225in}{2.708305in}}%
\pgfpathclose%
\pgfusepath{stroke,fill}%
\end{pgfscope}%
\begin{pgfscope}%
\pgfpathrectangle{\pgfqpoint{0.787074in}{0.548769in}}{\pgfqpoint{5.062926in}{3.102590in}}%
\pgfusepath{clip}%
\pgfsetbuttcap%
\pgfsetroundjoin%
\definecolor{currentfill}{rgb}{1.000000,0.498039,0.054902}%
\pgfsetfillcolor{currentfill}%
\pgfsetlinewidth{1.003750pt}%
\definecolor{currentstroke}{rgb}{1.000000,0.498039,0.054902}%
\pgfsetstrokecolor{currentstroke}%
\pgfsetdash{}{0pt}%
\pgfpathmoveto{\pgfqpoint{3.647298in}{2.886489in}}%
\pgfpathcurveto{\pgfqpoint{3.658348in}{2.886489in}}{\pgfqpoint{3.668948in}{2.890880in}}{\pgfqpoint{3.676761in}{2.898693in}}%
\pgfpathcurveto{\pgfqpoint{3.684575in}{2.906507in}}{\pgfqpoint{3.688965in}{2.917106in}}{\pgfqpoint{3.688965in}{2.928156in}}%
\pgfpathcurveto{\pgfqpoint{3.688965in}{2.939206in}}{\pgfqpoint{3.684575in}{2.949805in}}{\pgfqpoint{3.676761in}{2.957619in}}%
\pgfpathcurveto{\pgfqpoint{3.668948in}{2.965433in}}{\pgfqpoint{3.658348in}{2.969823in}}{\pgfqpoint{3.647298in}{2.969823in}}%
\pgfpathcurveto{\pgfqpoint{3.636248in}{2.969823in}}{\pgfqpoint{3.625649in}{2.965433in}}{\pgfqpoint{3.617836in}{2.957619in}}%
\pgfpathcurveto{\pgfqpoint{3.610022in}{2.949805in}}{\pgfqpoint{3.605632in}{2.939206in}}{\pgfqpoint{3.605632in}{2.928156in}}%
\pgfpathcurveto{\pgfqpoint{3.605632in}{2.917106in}}{\pgfqpoint{3.610022in}{2.906507in}}{\pgfqpoint{3.617836in}{2.898693in}}%
\pgfpathcurveto{\pgfqpoint{3.625649in}{2.890880in}}{\pgfqpoint{3.636248in}{2.886489in}}{\pgfqpoint{3.647298in}{2.886489in}}%
\pgfpathclose%
\pgfusepath{stroke,fill}%
\end{pgfscope}%
\begin{pgfscope}%
\pgfpathrectangle{\pgfqpoint{0.787074in}{0.548769in}}{\pgfqpoint{5.062926in}{3.102590in}}%
\pgfusepath{clip}%
\pgfsetbuttcap%
\pgfsetroundjoin%
\definecolor{currentfill}{rgb}{1.000000,0.498039,0.054902}%
\pgfsetfillcolor{currentfill}%
\pgfsetlinewidth{1.003750pt}%
\definecolor{currentstroke}{rgb}{1.000000,0.498039,0.054902}%
\pgfsetstrokecolor{currentstroke}%
\pgfsetdash{}{0pt}%
\pgfpathmoveto{\pgfqpoint{3.252785in}{2.003186in}}%
\pgfpathcurveto{\pgfqpoint{3.263835in}{2.003186in}}{\pgfqpoint{3.274434in}{2.007576in}}{\pgfqpoint{3.282247in}{2.015390in}}%
\pgfpathcurveto{\pgfqpoint{3.290061in}{2.023204in}}{\pgfqpoint{3.294451in}{2.033803in}}{\pgfqpoint{3.294451in}{2.044853in}}%
\pgfpathcurveto{\pgfqpoint{3.294451in}{2.055903in}}{\pgfqpoint{3.290061in}{2.066502in}}{\pgfqpoint{3.282247in}{2.074316in}}%
\pgfpathcurveto{\pgfqpoint{3.274434in}{2.082129in}}{\pgfqpoint{3.263835in}{2.086520in}}{\pgfqpoint{3.252785in}{2.086520in}}%
\pgfpathcurveto{\pgfqpoint{3.241735in}{2.086520in}}{\pgfqpoint{3.231135in}{2.082129in}}{\pgfqpoint{3.223322in}{2.074316in}}%
\pgfpathcurveto{\pgfqpoint{3.215508in}{2.066502in}}{\pgfqpoint{3.211118in}{2.055903in}}{\pgfqpoint{3.211118in}{2.044853in}}%
\pgfpathcurveto{\pgfqpoint{3.211118in}{2.033803in}}{\pgfqpoint{3.215508in}{2.023204in}}{\pgfqpoint{3.223322in}{2.015390in}}%
\pgfpathcurveto{\pgfqpoint{3.231135in}{2.007576in}}{\pgfqpoint{3.241735in}{2.003186in}}{\pgfqpoint{3.252785in}{2.003186in}}%
\pgfpathclose%
\pgfusepath{stroke,fill}%
\end{pgfscope}%
\begin{pgfscope}%
\pgfpathrectangle{\pgfqpoint{0.787074in}{0.548769in}}{\pgfqpoint{5.062926in}{3.102590in}}%
\pgfusepath{clip}%
\pgfsetbuttcap%
\pgfsetroundjoin%
\definecolor{currentfill}{rgb}{0.121569,0.466667,0.705882}%
\pgfsetfillcolor{currentfill}%
\pgfsetlinewidth{1.003750pt}%
\definecolor{currentstroke}{rgb}{0.121569,0.466667,0.705882}%
\pgfsetstrokecolor{currentstroke}%
\pgfsetdash{}{0pt}%
\pgfpathmoveto{\pgfqpoint{1.082959in}{0.942348in}}%
\pgfpathcurveto{\pgfqpoint{1.094009in}{0.942348in}}{\pgfqpoint{1.104608in}{0.946738in}}{\pgfqpoint{1.112422in}{0.954552in}}%
\pgfpathcurveto{\pgfqpoint{1.120236in}{0.962365in}}{\pgfqpoint{1.124626in}{0.972964in}}{\pgfqpoint{1.124626in}{0.984014in}}%
\pgfpathcurveto{\pgfqpoint{1.124626in}{0.995065in}}{\pgfqpoint{1.120236in}{1.005664in}}{\pgfqpoint{1.112422in}{1.013477in}}%
\pgfpathcurveto{\pgfqpoint{1.104608in}{1.021291in}}{\pgfqpoint{1.094009in}{1.025681in}}{\pgfqpoint{1.082959in}{1.025681in}}%
\pgfpathcurveto{\pgfqpoint{1.071909in}{1.025681in}}{\pgfqpoint{1.061310in}{1.021291in}}{\pgfqpoint{1.053496in}{1.013477in}}%
\pgfpathcurveto{\pgfqpoint{1.045683in}{1.005664in}}{\pgfqpoint{1.041292in}{0.995065in}}{\pgfqpoint{1.041292in}{0.984014in}}%
\pgfpathcurveto{\pgfqpoint{1.041292in}{0.972964in}}{\pgfqpoint{1.045683in}{0.962365in}}{\pgfqpoint{1.053496in}{0.954552in}}%
\pgfpathcurveto{\pgfqpoint{1.061310in}{0.946738in}}{\pgfqpoint{1.071909in}{0.942348in}}{\pgfqpoint{1.082959in}{0.942348in}}%
\pgfpathclose%
\pgfusepath{stroke,fill}%
\end{pgfscope}%
\begin{pgfscope}%
\pgfpathrectangle{\pgfqpoint{0.787074in}{0.548769in}}{\pgfqpoint{5.062926in}{3.102590in}}%
\pgfusepath{clip}%
\pgfsetbuttcap%
\pgfsetroundjoin%
\definecolor{currentfill}{rgb}{1.000000,0.498039,0.054902}%
\pgfsetfillcolor{currentfill}%
\pgfsetlinewidth{1.003750pt}%
\definecolor{currentstroke}{rgb}{1.000000,0.498039,0.054902}%
\pgfsetstrokecolor{currentstroke}%
\pgfsetdash{}{0pt}%
\pgfpathmoveto{\pgfqpoint{2.661014in}{2.620166in}}%
\pgfpathcurveto{\pgfqpoint{2.672064in}{2.620166in}}{\pgfqpoint{2.682663in}{2.624556in}}{\pgfqpoint{2.690477in}{2.632370in}}%
\pgfpathcurveto{\pgfqpoint{2.698290in}{2.640183in}}{\pgfqpoint{2.702681in}{2.650782in}}{\pgfqpoint{2.702681in}{2.661832in}}%
\pgfpathcurveto{\pgfqpoint{2.702681in}{2.672883in}}{\pgfqpoint{2.698290in}{2.683482in}}{\pgfqpoint{2.690477in}{2.691295in}}%
\pgfpathcurveto{\pgfqpoint{2.682663in}{2.699109in}}{\pgfqpoint{2.672064in}{2.703499in}}{\pgfqpoint{2.661014in}{2.703499in}}%
\pgfpathcurveto{\pgfqpoint{2.649964in}{2.703499in}}{\pgfqpoint{2.639365in}{2.699109in}}{\pgfqpoint{2.631551in}{2.691295in}}%
\pgfpathcurveto{\pgfqpoint{2.623738in}{2.683482in}}{\pgfqpoint{2.619347in}{2.672883in}}{\pgfqpoint{2.619347in}{2.661832in}}%
\pgfpathcurveto{\pgfqpoint{2.619347in}{2.650782in}}{\pgfqpoint{2.623738in}{2.640183in}}{\pgfqpoint{2.631551in}{2.632370in}}%
\pgfpathcurveto{\pgfqpoint{2.639365in}{2.624556in}}{\pgfqpoint{2.649964in}{2.620166in}}{\pgfqpoint{2.661014in}{2.620166in}}%
\pgfpathclose%
\pgfusepath{stroke,fill}%
\end{pgfscope}%
\begin{pgfscope}%
\pgfpathrectangle{\pgfqpoint{0.787074in}{0.548769in}}{\pgfqpoint{5.062926in}{3.102590in}}%
\pgfusepath{clip}%
\pgfsetbuttcap%
\pgfsetroundjoin%
\definecolor{currentfill}{rgb}{1.000000,0.498039,0.054902}%
\pgfsetfillcolor{currentfill}%
\pgfsetlinewidth{1.003750pt}%
\definecolor{currentstroke}{rgb}{1.000000,0.498039,0.054902}%
\pgfsetstrokecolor{currentstroke}%
\pgfsetdash{}{0pt}%
\pgfpathmoveto{\pgfqpoint{1.674730in}{2.991999in}}%
\pgfpathcurveto{\pgfqpoint{1.685780in}{2.991999in}}{\pgfqpoint{1.696379in}{2.996389in}}{\pgfqpoint{1.704193in}{3.004203in}}%
\pgfpathcurveto{\pgfqpoint{1.712006in}{3.012016in}}{\pgfqpoint{1.716396in}{3.022615in}}{\pgfqpoint{1.716396in}{3.033665in}}%
\pgfpathcurveto{\pgfqpoint{1.716396in}{3.044716in}}{\pgfqpoint{1.712006in}{3.055315in}}{\pgfqpoint{1.704193in}{3.063128in}}%
\pgfpathcurveto{\pgfqpoint{1.696379in}{3.070942in}}{\pgfqpoint{1.685780in}{3.075332in}}{\pgfqpoint{1.674730in}{3.075332in}}%
\pgfpathcurveto{\pgfqpoint{1.663680in}{3.075332in}}{\pgfqpoint{1.653081in}{3.070942in}}{\pgfqpoint{1.645267in}{3.063128in}}%
\pgfpathcurveto{\pgfqpoint{1.637453in}{3.055315in}}{\pgfqpoint{1.633063in}{3.044716in}}{\pgfqpoint{1.633063in}{3.033665in}}%
\pgfpathcurveto{\pgfqpoint{1.633063in}{3.022615in}}{\pgfqpoint{1.637453in}{3.012016in}}{\pgfqpoint{1.645267in}{3.004203in}}%
\pgfpathcurveto{\pgfqpoint{1.653081in}{2.996389in}}{\pgfqpoint{1.663680in}{2.991999in}}{\pgfqpoint{1.674730in}{2.991999in}}%
\pgfpathclose%
\pgfusepath{stroke,fill}%
\end{pgfscope}%
\begin{pgfscope}%
\pgfpathrectangle{\pgfqpoint{0.787074in}{0.548769in}}{\pgfqpoint{5.062926in}{3.102590in}}%
\pgfusepath{clip}%
\pgfsetbuttcap%
\pgfsetroundjoin%
\definecolor{currentfill}{rgb}{1.000000,0.498039,0.054902}%
\pgfsetfillcolor{currentfill}%
\pgfsetlinewidth{1.003750pt}%
\definecolor{currentstroke}{rgb}{1.000000,0.498039,0.054902}%
\pgfsetstrokecolor{currentstroke}%
\pgfsetdash{}{0pt}%
\pgfpathmoveto{\pgfqpoint{1.740482in}{3.136678in}}%
\pgfpathcurveto{\pgfqpoint{1.751532in}{3.136678in}}{\pgfqpoint{1.762131in}{3.141068in}}{\pgfqpoint{1.769945in}{3.148882in}}%
\pgfpathcurveto{\pgfqpoint{1.777758in}{3.156695in}}{\pgfqpoint{1.782149in}{3.167294in}}{\pgfqpoint{1.782149in}{3.178345in}}%
\pgfpathcurveto{\pgfqpoint{1.782149in}{3.189395in}}{\pgfqpoint{1.777758in}{3.199994in}}{\pgfqpoint{1.769945in}{3.207807in}}%
\pgfpathcurveto{\pgfqpoint{1.762131in}{3.215621in}}{\pgfqpoint{1.751532in}{3.220011in}}{\pgfqpoint{1.740482in}{3.220011in}}%
\pgfpathcurveto{\pgfqpoint{1.729432in}{3.220011in}}{\pgfqpoint{1.718833in}{3.215621in}}{\pgfqpoint{1.711019in}{3.207807in}}%
\pgfpathcurveto{\pgfqpoint{1.703206in}{3.199994in}}{\pgfqpoint{1.698815in}{3.189395in}}{\pgfqpoint{1.698815in}{3.178345in}}%
\pgfpathcurveto{\pgfqpoint{1.698815in}{3.167294in}}{\pgfqpoint{1.703206in}{3.156695in}}{\pgfqpoint{1.711019in}{3.148882in}}%
\pgfpathcurveto{\pgfqpoint{1.718833in}{3.141068in}}{\pgfqpoint{1.729432in}{3.136678in}}{\pgfqpoint{1.740482in}{3.136678in}}%
\pgfpathclose%
\pgfusepath{stroke,fill}%
\end{pgfscope}%
\begin{pgfscope}%
\pgfpathrectangle{\pgfqpoint{0.787074in}{0.548769in}}{\pgfqpoint{5.062926in}{3.102590in}}%
\pgfusepath{clip}%
\pgfsetbuttcap%
\pgfsetroundjoin%
\definecolor{currentfill}{rgb}{1.000000,0.498039,0.054902}%
\pgfsetfillcolor{currentfill}%
\pgfsetlinewidth{1.003750pt}%
\definecolor{currentstroke}{rgb}{1.000000,0.498039,0.054902}%
\pgfsetstrokecolor{currentstroke}%
\pgfsetdash{}{0pt}%
\pgfpathmoveto{\pgfqpoint{2.003491in}{2.684169in}}%
\pgfpathcurveto{\pgfqpoint{2.014541in}{2.684169in}}{\pgfqpoint{2.025140in}{2.688559in}}{\pgfqpoint{2.032954in}{2.696373in}}%
\pgfpathcurveto{\pgfqpoint{2.040768in}{2.704186in}}{\pgfqpoint{2.045158in}{2.714785in}}{\pgfqpoint{2.045158in}{2.725835in}}%
\pgfpathcurveto{\pgfqpoint{2.045158in}{2.736886in}}{\pgfqpoint{2.040768in}{2.747485in}}{\pgfqpoint{2.032954in}{2.755298in}}%
\pgfpathcurveto{\pgfqpoint{2.025140in}{2.763112in}}{\pgfqpoint{2.014541in}{2.767502in}}{\pgfqpoint{2.003491in}{2.767502in}}%
\pgfpathcurveto{\pgfqpoint{1.992441in}{2.767502in}}{\pgfqpoint{1.981842in}{2.763112in}}{\pgfqpoint{1.974028in}{2.755298in}}%
\pgfpathcurveto{\pgfqpoint{1.966215in}{2.747485in}}{\pgfqpoint{1.961824in}{2.736886in}}{\pgfqpoint{1.961824in}{2.725835in}}%
\pgfpathcurveto{\pgfqpoint{1.961824in}{2.714785in}}{\pgfqpoint{1.966215in}{2.704186in}}{\pgfqpoint{1.974028in}{2.696373in}}%
\pgfpathcurveto{\pgfqpoint{1.981842in}{2.688559in}}{\pgfqpoint{1.992441in}{2.684169in}}{\pgfqpoint{2.003491in}{2.684169in}}%
\pgfpathclose%
\pgfusepath{stroke,fill}%
\end{pgfscope}%
\begin{pgfscope}%
\pgfpathrectangle{\pgfqpoint{0.787074in}{0.548769in}}{\pgfqpoint{5.062926in}{3.102590in}}%
\pgfusepath{clip}%
\pgfsetbuttcap%
\pgfsetroundjoin%
\definecolor{currentfill}{rgb}{1.000000,0.498039,0.054902}%
\pgfsetfillcolor{currentfill}%
\pgfsetlinewidth{1.003750pt}%
\definecolor{currentstroke}{rgb}{1.000000,0.498039,0.054902}%
\pgfsetstrokecolor{currentstroke}%
\pgfsetdash{}{0pt}%
\pgfpathmoveto{\pgfqpoint{1.543225in}{2.925050in}}%
\pgfpathcurveto{\pgfqpoint{1.554275in}{2.925050in}}{\pgfqpoint{1.564874in}{2.929440in}}{\pgfqpoint{1.572688in}{2.937253in}}%
\pgfpathcurveto{\pgfqpoint{1.580502in}{2.945067in}}{\pgfqpoint{1.584892in}{2.955666in}}{\pgfqpoint{1.584892in}{2.966716in}}%
\pgfpathcurveto{\pgfqpoint{1.584892in}{2.977766in}}{\pgfqpoint{1.580502in}{2.988365in}}{\pgfqpoint{1.572688in}{2.996179in}}%
\pgfpathcurveto{\pgfqpoint{1.564874in}{3.003993in}}{\pgfqpoint{1.554275in}{3.008383in}}{\pgfqpoint{1.543225in}{3.008383in}}%
\pgfpathcurveto{\pgfqpoint{1.532175in}{3.008383in}}{\pgfqpoint{1.521576in}{3.003993in}}{\pgfqpoint{1.513762in}{2.996179in}}%
\pgfpathcurveto{\pgfqpoint{1.505949in}{2.988365in}}{\pgfqpoint{1.501558in}{2.977766in}}{\pgfqpoint{1.501558in}{2.966716in}}%
\pgfpathcurveto{\pgfqpoint{1.501558in}{2.955666in}}{\pgfqpoint{1.505949in}{2.945067in}}{\pgfqpoint{1.513762in}{2.937253in}}%
\pgfpathcurveto{\pgfqpoint{1.521576in}{2.929440in}}{\pgfqpoint{1.532175in}{2.925050in}}{\pgfqpoint{1.543225in}{2.925050in}}%
\pgfpathclose%
\pgfusepath{stroke,fill}%
\end{pgfscope}%
\begin{pgfscope}%
\pgfpathrectangle{\pgfqpoint{0.787074in}{0.548769in}}{\pgfqpoint{5.062926in}{3.102590in}}%
\pgfusepath{clip}%
\pgfsetbuttcap%
\pgfsetroundjoin%
\definecolor{currentfill}{rgb}{0.121569,0.466667,0.705882}%
\pgfsetfillcolor{currentfill}%
\pgfsetlinewidth{1.003750pt}%
\definecolor{currentstroke}{rgb}{0.121569,0.466667,0.705882}%
\pgfsetstrokecolor{currentstroke}%
\pgfsetdash{}{0pt}%
\pgfpathmoveto{\pgfqpoint{2.200748in}{2.772349in}}%
\pgfpathcurveto{\pgfqpoint{2.211798in}{2.772349in}}{\pgfqpoint{2.222397in}{2.776739in}}{\pgfqpoint{2.230211in}{2.784553in}}%
\pgfpathcurveto{\pgfqpoint{2.238024in}{2.792367in}}{\pgfqpoint{2.242415in}{2.802966in}}{\pgfqpoint{2.242415in}{2.814016in}}%
\pgfpathcurveto{\pgfqpoint{2.242415in}{2.825066in}}{\pgfqpoint{2.238024in}{2.835665in}}{\pgfqpoint{2.230211in}{2.843479in}}%
\pgfpathcurveto{\pgfqpoint{2.222397in}{2.851292in}}{\pgfqpoint{2.211798in}{2.855683in}}{\pgfqpoint{2.200748in}{2.855683in}}%
\pgfpathcurveto{\pgfqpoint{2.189698in}{2.855683in}}{\pgfqpoint{2.179099in}{2.851292in}}{\pgfqpoint{2.171285in}{2.843479in}}%
\pgfpathcurveto{\pgfqpoint{2.163472in}{2.835665in}}{\pgfqpoint{2.159081in}{2.825066in}}{\pgfqpoint{2.159081in}{2.814016in}}%
\pgfpathcurveto{\pgfqpoint{2.159081in}{2.802966in}}{\pgfqpoint{2.163472in}{2.792367in}}{\pgfqpoint{2.171285in}{2.784553in}}%
\pgfpathcurveto{\pgfqpoint{2.179099in}{2.776739in}}{\pgfqpoint{2.189698in}{2.772349in}}{\pgfqpoint{2.200748in}{2.772349in}}%
\pgfpathclose%
\pgfusepath{stroke,fill}%
\end{pgfscope}%
\begin{pgfscope}%
\pgfpathrectangle{\pgfqpoint{0.787074in}{0.548769in}}{\pgfqpoint{5.062926in}{3.102590in}}%
\pgfusepath{clip}%
\pgfsetbuttcap%
\pgfsetroundjoin%
\definecolor{currentfill}{rgb}{1.000000,0.498039,0.054902}%
\pgfsetfillcolor{currentfill}%
\pgfsetlinewidth{1.003750pt}%
\definecolor{currentstroke}{rgb}{1.000000,0.498039,0.054902}%
\pgfsetstrokecolor{currentstroke}%
\pgfsetdash{}{0pt}%
\pgfpathmoveto{\pgfqpoint{2.332253in}{3.273091in}}%
\pgfpathcurveto{\pgfqpoint{2.343303in}{3.273091in}}{\pgfqpoint{2.353902in}{3.277482in}}{\pgfqpoint{2.361715in}{3.285295in}}%
\pgfpathcurveto{\pgfqpoint{2.369529in}{3.293109in}}{\pgfqpoint{2.373919in}{3.303708in}}{\pgfqpoint{2.373919in}{3.314758in}}%
\pgfpathcurveto{\pgfqpoint{2.373919in}{3.325808in}}{\pgfqpoint{2.369529in}{3.336407in}}{\pgfqpoint{2.361715in}{3.344221in}}%
\pgfpathcurveto{\pgfqpoint{2.353902in}{3.352035in}}{\pgfqpoint{2.343303in}{3.356425in}}{\pgfqpoint{2.332253in}{3.356425in}}%
\pgfpathcurveto{\pgfqpoint{2.321202in}{3.356425in}}{\pgfqpoint{2.310603in}{3.352035in}}{\pgfqpoint{2.302790in}{3.344221in}}%
\pgfpathcurveto{\pgfqpoint{2.294976in}{3.336407in}}{\pgfqpoint{2.290586in}{3.325808in}}{\pgfqpoint{2.290586in}{3.314758in}}%
\pgfpathcurveto{\pgfqpoint{2.290586in}{3.303708in}}{\pgfqpoint{2.294976in}{3.293109in}}{\pgfqpoint{2.302790in}{3.285295in}}%
\pgfpathcurveto{\pgfqpoint{2.310603in}{3.277482in}}{\pgfqpoint{2.321202in}{3.273091in}}{\pgfqpoint{2.332253in}{3.273091in}}%
\pgfpathclose%
\pgfusepath{stroke,fill}%
\end{pgfscope}%
\begin{pgfscope}%
\pgfpathrectangle{\pgfqpoint{0.787074in}{0.548769in}}{\pgfqpoint{5.062926in}{3.102590in}}%
\pgfusepath{clip}%
\pgfsetbuttcap%
\pgfsetroundjoin%
\definecolor{currentfill}{rgb}{1.000000,0.498039,0.054902}%
\pgfsetfillcolor{currentfill}%
\pgfsetlinewidth{1.003750pt}%
\definecolor{currentstroke}{rgb}{1.000000,0.498039,0.054902}%
\pgfsetstrokecolor{currentstroke}%
\pgfsetdash{}{0pt}%
\pgfpathmoveto{\pgfqpoint{2.200748in}{2.433412in}}%
\pgfpathcurveto{\pgfqpoint{2.211798in}{2.433412in}}{\pgfqpoint{2.222397in}{2.437803in}}{\pgfqpoint{2.230211in}{2.445616in}}%
\pgfpathcurveto{\pgfqpoint{2.238024in}{2.453430in}}{\pgfqpoint{2.242415in}{2.464029in}}{\pgfqpoint{2.242415in}{2.475079in}}%
\pgfpathcurveto{\pgfqpoint{2.242415in}{2.486129in}}{\pgfqpoint{2.238024in}{2.496728in}}{\pgfqpoint{2.230211in}{2.504542in}}%
\pgfpathcurveto{\pgfqpoint{2.222397in}{2.512356in}}{\pgfqpoint{2.211798in}{2.516746in}}{\pgfqpoint{2.200748in}{2.516746in}}%
\pgfpathcurveto{\pgfqpoint{2.189698in}{2.516746in}}{\pgfqpoint{2.179099in}{2.512356in}}{\pgfqpoint{2.171285in}{2.504542in}}%
\pgfpathcurveto{\pgfqpoint{2.163472in}{2.496728in}}{\pgfqpoint{2.159081in}{2.486129in}}{\pgfqpoint{2.159081in}{2.475079in}}%
\pgfpathcurveto{\pgfqpoint{2.159081in}{2.464029in}}{\pgfqpoint{2.163472in}{2.453430in}}{\pgfqpoint{2.171285in}{2.445616in}}%
\pgfpathcurveto{\pgfqpoint{2.179099in}{2.437803in}}{\pgfqpoint{2.189698in}{2.433412in}}{\pgfqpoint{2.200748in}{2.433412in}}%
\pgfpathclose%
\pgfusepath{stroke,fill}%
\end{pgfscope}%
\begin{pgfscope}%
\pgfpathrectangle{\pgfqpoint{0.787074in}{0.548769in}}{\pgfqpoint{5.062926in}{3.102590in}}%
\pgfusepath{clip}%
\pgfsetbuttcap%
\pgfsetroundjoin%
\definecolor{currentfill}{rgb}{1.000000,0.498039,0.054902}%
\pgfsetfillcolor{currentfill}%
\pgfsetlinewidth{1.003750pt}%
\definecolor{currentstroke}{rgb}{1.000000,0.498039,0.054902}%
\pgfsetstrokecolor{currentstroke}%
\pgfsetdash{}{0pt}%
\pgfpathmoveto{\pgfqpoint{2.134996in}{2.784629in}}%
\pgfpathcurveto{\pgfqpoint{2.146046in}{2.784629in}}{\pgfqpoint{2.156645in}{2.789019in}}{\pgfqpoint{2.164459in}{2.796833in}}%
\pgfpathcurveto{\pgfqpoint{2.172272in}{2.804646in}}{\pgfqpoint{2.176662in}{2.815245in}}{\pgfqpoint{2.176662in}{2.826296in}}%
\pgfpathcurveto{\pgfqpoint{2.176662in}{2.837346in}}{\pgfqpoint{2.172272in}{2.847945in}}{\pgfqpoint{2.164459in}{2.855758in}}%
\pgfpathcurveto{\pgfqpoint{2.156645in}{2.863572in}}{\pgfqpoint{2.146046in}{2.867962in}}{\pgfqpoint{2.134996in}{2.867962in}}%
\pgfpathcurveto{\pgfqpoint{2.123946in}{2.867962in}}{\pgfqpoint{2.113347in}{2.863572in}}{\pgfqpoint{2.105533in}{2.855758in}}%
\pgfpathcurveto{\pgfqpoint{2.097719in}{2.847945in}}{\pgfqpoint{2.093329in}{2.837346in}}{\pgfqpoint{2.093329in}{2.826296in}}%
\pgfpathcurveto{\pgfqpoint{2.093329in}{2.815245in}}{\pgfqpoint{2.097719in}{2.804646in}}{\pgfqpoint{2.105533in}{2.796833in}}%
\pgfpathcurveto{\pgfqpoint{2.113347in}{2.789019in}}{\pgfqpoint{2.123946in}{2.784629in}}{\pgfqpoint{2.134996in}{2.784629in}}%
\pgfpathclose%
\pgfusepath{stroke,fill}%
\end{pgfscope}%
\begin{pgfscope}%
\pgfpathrectangle{\pgfqpoint{0.787074in}{0.548769in}}{\pgfqpoint{5.062926in}{3.102590in}}%
\pgfusepath{clip}%
\pgfsetbuttcap%
\pgfsetroundjoin%
\definecolor{currentfill}{rgb}{1.000000,0.498039,0.054902}%
\pgfsetfillcolor{currentfill}%
\pgfsetlinewidth{1.003750pt}%
\definecolor{currentstroke}{rgb}{1.000000,0.498039,0.054902}%
\pgfsetstrokecolor{currentstroke}%
\pgfsetdash{}{0pt}%
\pgfpathmoveto{\pgfqpoint{2.003491in}{2.406321in}}%
\pgfpathcurveto{\pgfqpoint{2.014541in}{2.406321in}}{\pgfqpoint{2.025140in}{2.410712in}}{\pgfqpoint{2.032954in}{2.418525in}}%
\pgfpathcurveto{\pgfqpoint{2.040768in}{2.426339in}}{\pgfqpoint{2.045158in}{2.436938in}}{\pgfqpoint{2.045158in}{2.447988in}}%
\pgfpathcurveto{\pgfqpoint{2.045158in}{2.459038in}}{\pgfqpoint{2.040768in}{2.469637in}}{\pgfqpoint{2.032954in}{2.477451in}}%
\pgfpathcurveto{\pgfqpoint{2.025140in}{2.485264in}}{\pgfqpoint{2.014541in}{2.489655in}}{\pgfqpoint{2.003491in}{2.489655in}}%
\pgfpathcurveto{\pgfqpoint{1.992441in}{2.489655in}}{\pgfqpoint{1.981842in}{2.485264in}}{\pgfqpoint{1.974028in}{2.477451in}}%
\pgfpathcurveto{\pgfqpoint{1.966215in}{2.469637in}}{\pgfqpoint{1.961824in}{2.459038in}}{\pgfqpoint{1.961824in}{2.447988in}}%
\pgfpathcurveto{\pgfqpoint{1.961824in}{2.436938in}}{\pgfqpoint{1.966215in}{2.426339in}}{\pgfqpoint{1.974028in}{2.418525in}}%
\pgfpathcurveto{\pgfqpoint{1.981842in}{2.410712in}}{\pgfqpoint{1.992441in}{2.406321in}}{\pgfqpoint{2.003491in}{2.406321in}}%
\pgfpathclose%
\pgfusepath{stroke,fill}%
\end{pgfscope}%
\begin{pgfscope}%
\pgfpathrectangle{\pgfqpoint{0.787074in}{0.548769in}}{\pgfqpoint{5.062926in}{3.102590in}}%
\pgfusepath{clip}%
\pgfsetbuttcap%
\pgfsetroundjoin%
\definecolor{currentfill}{rgb}{1.000000,0.498039,0.054902}%
\pgfsetfillcolor{currentfill}%
\pgfsetlinewidth{1.003750pt}%
\definecolor{currentstroke}{rgb}{1.000000,0.498039,0.054902}%
\pgfsetstrokecolor{currentstroke}%
\pgfsetdash{}{0pt}%
\pgfpathmoveto{\pgfqpoint{2.989775in}{2.705375in}}%
\pgfpathcurveto{\pgfqpoint{3.000826in}{2.705375in}}{\pgfqpoint{3.011425in}{2.709765in}}{\pgfqpoint{3.019238in}{2.717578in}}%
\pgfpathcurveto{\pgfqpoint{3.027052in}{2.725392in}}{\pgfqpoint{3.031442in}{2.735991in}}{\pgfqpoint{3.031442in}{2.747041in}}%
\pgfpathcurveto{\pgfqpoint{3.031442in}{2.758091in}}{\pgfqpoint{3.027052in}{2.768690in}}{\pgfqpoint{3.019238in}{2.776504in}}%
\pgfpathcurveto{\pgfqpoint{3.011425in}{2.784318in}}{\pgfqpoint{3.000826in}{2.788708in}}{\pgfqpoint{2.989775in}{2.788708in}}%
\pgfpathcurveto{\pgfqpoint{2.978725in}{2.788708in}}{\pgfqpoint{2.968126in}{2.784318in}}{\pgfqpoint{2.960313in}{2.776504in}}%
\pgfpathcurveto{\pgfqpoint{2.952499in}{2.768690in}}{\pgfqpoint{2.948109in}{2.758091in}}{\pgfqpoint{2.948109in}{2.747041in}}%
\pgfpathcurveto{\pgfqpoint{2.948109in}{2.735991in}}{\pgfqpoint{2.952499in}{2.725392in}}{\pgfqpoint{2.960313in}{2.717578in}}%
\pgfpathcurveto{\pgfqpoint{2.968126in}{2.709765in}}{\pgfqpoint{2.978725in}{2.705375in}}{\pgfqpoint{2.989775in}{2.705375in}}%
\pgfpathclose%
\pgfusepath{stroke,fill}%
\end{pgfscope}%
\begin{pgfscope}%
\pgfpathrectangle{\pgfqpoint{0.787074in}{0.548769in}}{\pgfqpoint{5.062926in}{3.102590in}}%
\pgfusepath{clip}%
\pgfsetbuttcap%
\pgfsetroundjoin%
\definecolor{currentfill}{rgb}{1.000000,0.498039,0.054902}%
\pgfsetfillcolor{currentfill}%
\pgfsetlinewidth{1.003750pt}%
\definecolor{currentstroke}{rgb}{1.000000,0.498039,0.054902}%
\pgfsetstrokecolor{currentstroke}%
\pgfsetdash{}{0pt}%
\pgfpathmoveto{\pgfqpoint{4.304821in}{2.307212in}}%
\pgfpathcurveto{\pgfqpoint{4.315871in}{2.307212in}}{\pgfqpoint{4.326470in}{2.311602in}}{\pgfqpoint{4.334284in}{2.319416in}}%
\pgfpathcurveto{\pgfqpoint{4.342098in}{2.327230in}}{\pgfqpoint{4.346488in}{2.337829in}}{\pgfqpoint{4.346488in}{2.348879in}}%
\pgfpathcurveto{\pgfqpoint{4.346488in}{2.359929in}}{\pgfqpoint{4.342098in}{2.370528in}}{\pgfqpoint{4.334284in}{2.378342in}}%
\pgfpathcurveto{\pgfqpoint{4.326470in}{2.386155in}}{\pgfqpoint{4.315871in}{2.390545in}}{\pgfqpoint{4.304821in}{2.390545in}}%
\pgfpathcurveto{\pgfqpoint{4.293771in}{2.390545in}}{\pgfqpoint{4.283172in}{2.386155in}}{\pgfqpoint{4.275358in}{2.378342in}}%
\pgfpathcurveto{\pgfqpoint{4.267545in}{2.370528in}}{\pgfqpoint{4.263155in}{2.359929in}}{\pgfqpoint{4.263155in}{2.348879in}}%
\pgfpathcurveto{\pgfqpoint{4.263155in}{2.337829in}}{\pgfqpoint{4.267545in}{2.327230in}}{\pgfqpoint{4.275358in}{2.319416in}}%
\pgfpathcurveto{\pgfqpoint{4.283172in}{2.311602in}}{\pgfqpoint{4.293771in}{2.307212in}}{\pgfqpoint{4.304821in}{2.307212in}}%
\pgfpathclose%
\pgfusepath{stroke,fill}%
\end{pgfscope}%
\begin{pgfscope}%
\pgfpathrectangle{\pgfqpoint{0.787074in}{0.548769in}}{\pgfqpoint{5.062926in}{3.102590in}}%
\pgfusepath{clip}%
\pgfsetbuttcap%
\pgfsetroundjoin%
\definecolor{currentfill}{rgb}{1.000000,0.498039,0.054902}%
\pgfsetfillcolor{currentfill}%
\pgfsetlinewidth{1.003750pt}%
\definecolor{currentstroke}{rgb}{1.000000,0.498039,0.054902}%
\pgfsetstrokecolor{currentstroke}%
\pgfsetdash{}{0pt}%
\pgfpathmoveto{\pgfqpoint{1.871987in}{3.001341in}}%
\pgfpathcurveto{\pgfqpoint{1.883037in}{3.001341in}}{\pgfqpoint{1.893636in}{3.005732in}}{\pgfqpoint{1.901449in}{3.013545in}}%
\pgfpathcurveto{\pgfqpoint{1.909263in}{3.021359in}}{\pgfqpoint{1.913653in}{3.031958in}}{\pgfqpoint{1.913653in}{3.043008in}}%
\pgfpathcurveto{\pgfqpoint{1.913653in}{3.054058in}}{\pgfqpoint{1.909263in}{3.064657in}}{\pgfqpoint{1.901449in}{3.072471in}}%
\pgfpathcurveto{\pgfqpoint{1.893636in}{3.080284in}}{\pgfqpoint{1.883037in}{3.084675in}}{\pgfqpoint{1.871987in}{3.084675in}}%
\pgfpathcurveto{\pgfqpoint{1.860936in}{3.084675in}}{\pgfqpoint{1.850337in}{3.080284in}}{\pgfqpoint{1.842524in}{3.072471in}}%
\pgfpathcurveto{\pgfqpoint{1.834710in}{3.064657in}}{\pgfqpoint{1.830320in}{3.054058in}}{\pgfqpoint{1.830320in}{3.043008in}}%
\pgfpathcurveto{\pgfqpoint{1.830320in}{3.031958in}}{\pgfqpoint{1.834710in}{3.021359in}}{\pgfqpoint{1.842524in}{3.013545in}}%
\pgfpathcurveto{\pgfqpoint{1.850337in}{3.005732in}}{\pgfqpoint{1.860936in}{3.001341in}}{\pgfqpoint{1.871987in}{3.001341in}}%
\pgfpathclose%
\pgfusepath{stroke,fill}%
\end{pgfscope}%
\begin{pgfscope}%
\pgfpathrectangle{\pgfqpoint{0.787074in}{0.548769in}}{\pgfqpoint{5.062926in}{3.102590in}}%
\pgfusepath{clip}%
\pgfsetbuttcap%
\pgfsetroundjoin%
\definecolor{currentfill}{rgb}{1.000000,0.498039,0.054902}%
\pgfsetfillcolor{currentfill}%
\pgfsetlinewidth{1.003750pt}%
\definecolor{currentstroke}{rgb}{1.000000,0.498039,0.054902}%
\pgfsetstrokecolor{currentstroke}%
\pgfsetdash{}{0pt}%
\pgfpathmoveto{\pgfqpoint{1.937739in}{2.401858in}}%
\pgfpathcurveto{\pgfqpoint{1.948789in}{2.401858in}}{\pgfqpoint{1.959388in}{2.406248in}}{\pgfqpoint{1.967202in}{2.414062in}}%
\pgfpathcurveto{\pgfqpoint{1.975015in}{2.421875in}}{\pgfqpoint{1.979406in}{2.432474in}}{\pgfqpoint{1.979406in}{2.443524in}}%
\pgfpathcurveto{\pgfqpoint{1.979406in}{2.454575in}}{\pgfqpoint{1.975015in}{2.465174in}}{\pgfqpoint{1.967202in}{2.472987in}}%
\pgfpathcurveto{\pgfqpoint{1.959388in}{2.480801in}}{\pgfqpoint{1.948789in}{2.485191in}}{\pgfqpoint{1.937739in}{2.485191in}}%
\pgfpathcurveto{\pgfqpoint{1.926689in}{2.485191in}}{\pgfqpoint{1.916090in}{2.480801in}}{\pgfqpoint{1.908276in}{2.472987in}}%
\pgfpathcurveto{\pgfqpoint{1.900462in}{2.465174in}}{\pgfqpoint{1.896072in}{2.454575in}}{\pgfqpoint{1.896072in}{2.443524in}}%
\pgfpathcurveto{\pgfqpoint{1.896072in}{2.432474in}}{\pgfqpoint{1.900462in}{2.421875in}}{\pgfqpoint{1.908276in}{2.414062in}}%
\pgfpathcurveto{\pgfqpoint{1.916090in}{2.406248in}}{\pgfqpoint{1.926689in}{2.401858in}}{\pgfqpoint{1.937739in}{2.401858in}}%
\pgfpathclose%
\pgfusepath{stroke,fill}%
\end{pgfscope}%
\begin{pgfscope}%
\pgfpathrectangle{\pgfqpoint{0.787074in}{0.548769in}}{\pgfqpoint{5.062926in}{3.102590in}}%
\pgfusepath{clip}%
\pgfsetbuttcap%
\pgfsetroundjoin%
\definecolor{currentfill}{rgb}{0.121569,0.466667,0.705882}%
\pgfsetfillcolor{currentfill}%
\pgfsetlinewidth{1.003750pt}%
\definecolor{currentstroke}{rgb}{0.121569,0.466667,0.705882}%
\pgfsetstrokecolor{currentstroke}%
\pgfsetdash{}{0pt}%
\pgfpathmoveto{\pgfqpoint{1.608977in}{2.416209in}}%
\pgfpathcurveto{\pgfqpoint{1.620028in}{2.416209in}}{\pgfqpoint{1.630627in}{2.420599in}}{\pgfqpoint{1.638440in}{2.428413in}}%
\pgfpathcurveto{\pgfqpoint{1.646254in}{2.436227in}}{\pgfqpoint{1.650644in}{2.446826in}}{\pgfqpoint{1.650644in}{2.457876in}}%
\pgfpathcurveto{\pgfqpoint{1.650644in}{2.468926in}}{\pgfqpoint{1.646254in}{2.479525in}}{\pgfqpoint{1.638440in}{2.487339in}}%
\pgfpathcurveto{\pgfqpoint{1.630627in}{2.495152in}}{\pgfqpoint{1.620028in}{2.499543in}}{\pgfqpoint{1.608977in}{2.499543in}}%
\pgfpathcurveto{\pgfqpoint{1.597927in}{2.499543in}}{\pgfqpoint{1.587328in}{2.495152in}}{\pgfqpoint{1.579515in}{2.487339in}}%
\pgfpathcurveto{\pgfqpoint{1.571701in}{2.479525in}}{\pgfqpoint{1.567311in}{2.468926in}}{\pgfqpoint{1.567311in}{2.457876in}}%
\pgfpathcurveto{\pgfqpoint{1.567311in}{2.446826in}}{\pgfqpoint{1.571701in}{2.436227in}}{\pgfqpoint{1.579515in}{2.428413in}}%
\pgfpathcurveto{\pgfqpoint{1.587328in}{2.420599in}}{\pgfqpoint{1.597927in}{2.416209in}}{\pgfqpoint{1.608977in}{2.416209in}}%
\pgfpathclose%
\pgfusepath{stroke,fill}%
\end{pgfscope}%
\begin{pgfscope}%
\pgfpathrectangle{\pgfqpoint{0.787074in}{0.548769in}}{\pgfqpoint{5.062926in}{3.102590in}}%
\pgfusepath{clip}%
\pgfsetbuttcap%
\pgfsetroundjoin%
\definecolor{currentfill}{rgb}{1.000000,0.498039,0.054902}%
\pgfsetfillcolor{currentfill}%
\pgfsetlinewidth{1.003750pt}%
\definecolor{currentstroke}{rgb}{1.000000,0.498039,0.054902}%
\pgfsetstrokecolor{currentstroke}%
\pgfsetdash{}{0pt}%
\pgfpathmoveto{\pgfqpoint{2.398005in}{3.029407in}}%
\pgfpathcurveto{\pgfqpoint{2.409055in}{3.029407in}}{\pgfqpoint{2.419654in}{3.033797in}}{\pgfqpoint{2.427468in}{3.041611in}}%
\pgfpathcurveto{\pgfqpoint{2.435281in}{3.049424in}}{\pgfqpoint{2.439672in}{3.060023in}}{\pgfqpoint{2.439672in}{3.071073in}}%
\pgfpathcurveto{\pgfqpoint{2.439672in}{3.082124in}}{\pgfqpoint{2.435281in}{3.092723in}}{\pgfqpoint{2.427468in}{3.100536in}}%
\pgfpathcurveto{\pgfqpoint{2.419654in}{3.108350in}}{\pgfqpoint{2.409055in}{3.112740in}}{\pgfqpoint{2.398005in}{3.112740in}}%
\pgfpathcurveto{\pgfqpoint{2.386955in}{3.112740in}}{\pgfqpoint{2.376356in}{3.108350in}}{\pgfqpoint{2.368542in}{3.100536in}}%
\pgfpathcurveto{\pgfqpoint{2.360728in}{3.092723in}}{\pgfqpoint{2.356338in}{3.082124in}}{\pgfqpoint{2.356338in}{3.071073in}}%
\pgfpathcurveto{\pgfqpoint{2.356338in}{3.060023in}}{\pgfqpoint{2.360728in}{3.049424in}}{\pgfqpoint{2.368542in}{3.041611in}}%
\pgfpathcurveto{\pgfqpoint{2.376356in}{3.033797in}}{\pgfqpoint{2.386955in}{3.029407in}}{\pgfqpoint{2.398005in}{3.029407in}}%
\pgfpathclose%
\pgfusepath{stroke,fill}%
\end{pgfscope}%
\begin{pgfscope}%
\pgfpathrectangle{\pgfqpoint{0.787074in}{0.548769in}}{\pgfqpoint{5.062926in}{3.102590in}}%
\pgfusepath{clip}%
\pgfsetbuttcap%
\pgfsetroundjoin%
\definecolor{currentfill}{rgb}{1.000000,0.498039,0.054902}%
\pgfsetfillcolor{currentfill}%
\pgfsetlinewidth{1.003750pt}%
\definecolor{currentstroke}{rgb}{1.000000,0.498039,0.054902}%
\pgfsetstrokecolor{currentstroke}%
\pgfsetdash{}{0pt}%
\pgfpathmoveto{\pgfqpoint{2.134996in}{2.943038in}}%
\pgfpathcurveto{\pgfqpoint{2.146046in}{2.943038in}}{\pgfqpoint{2.156645in}{2.947429in}}{\pgfqpoint{2.164459in}{2.955242in}}%
\pgfpathcurveto{\pgfqpoint{2.172272in}{2.963056in}}{\pgfqpoint{2.176662in}{2.973655in}}{\pgfqpoint{2.176662in}{2.984705in}}%
\pgfpathcurveto{\pgfqpoint{2.176662in}{2.995755in}}{\pgfqpoint{2.172272in}{3.006354in}}{\pgfqpoint{2.164459in}{3.014168in}}%
\pgfpathcurveto{\pgfqpoint{2.156645in}{3.021981in}}{\pgfqpoint{2.146046in}{3.026372in}}{\pgfqpoint{2.134996in}{3.026372in}}%
\pgfpathcurveto{\pgfqpoint{2.123946in}{3.026372in}}{\pgfqpoint{2.113347in}{3.021981in}}{\pgfqpoint{2.105533in}{3.014168in}}%
\pgfpathcurveto{\pgfqpoint{2.097719in}{3.006354in}}{\pgfqpoint{2.093329in}{2.995755in}}{\pgfqpoint{2.093329in}{2.984705in}}%
\pgfpathcurveto{\pgfqpoint{2.093329in}{2.973655in}}{\pgfqpoint{2.097719in}{2.963056in}}{\pgfqpoint{2.105533in}{2.955242in}}%
\pgfpathcurveto{\pgfqpoint{2.113347in}{2.947429in}}{\pgfqpoint{2.123946in}{2.943038in}}{\pgfqpoint{2.134996in}{2.943038in}}%
\pgfpathclose%
\pgfusepath{stroke,fill}%
\end{pgfscope}%
\begin{pgfscope}%
\pgfpathrectangle{\pgfqpoint{0.787074in}{0.548769in}}{\pgfqpoint{5.062926in}{3.102590in}}%
\pgfusepath{clip}%
\pgfsetbuttcap%
\pgfsetroundjoin%
\definecolor{currentfill}{rgb}{1.000000,0.498039,0.054902}%
\pgfsetfillcolor{currentfill}%
\pgfsetlinewidth{1.003750pt}%
\definecolor{currentstroke}{rgb}{1.000000,0.498039,0.054902}%
\pgfsetstrokecolor{currentstroke}%
\pgfsetdash{}{0pt}%
\pgfpathmoveto{\pgfqpoint{1.674730in}{2.587341in}}%
\pgfpathcurveto{\pgfqpoint{1.685780in}{2.587341in}}{\pgfqpoint{1.696379in}{2.591732in}}{\pgfqpoint{1.704193in}{2.599545in}}%
\pgfpathcurveto{\pgfqpoint{1.712006in}{2.607359in}}{\pgfqpoint{1.716396in}{2.617958in}}{\pgfqpoint{1.716396in}{2.629008in}}%
\pgfpathcurveto{\pgfqpoint{1.716396in}{2.640058in}}{\pgfqpoint{1.712006in}{2.650657in}}{\pgfqpoint{1.704193in}{2.658471in}}%
\pgfpathcurveto{\pgfqpoint{1.696379in}{2.666284in}}{\pgfqpoint{1.685780in}{2.670675in}}{\pgfqpoint{1.674730in}{2.670675in}}%
\pgfpathcurveto{\pgfqpoint{1.663680in}{2.670675in}}{\pgfqpoint{1.653081in}{2.666284in}}{\pgfqpoint{1.645267in}{2.658471in}}%
\pgfpathcurveto{\pgfqpoint{1.637453in}{2.650657in}}{\pgfqpoint{1.633063in}{2.640058in}}{\pgfqpoint{1.633063in}{2.629008in}}%
\pgfpathcurveto{\pgfqpoint{1.633063in}{2.617958in}}{\pgfqpoint{1.637453in}{2.607359in}}{\pgfqpoint{1.645267in}{2.599545in}}%
\pgfpathcurveto{\pgfqpoint{1.653081in}{2.591732in}}{\pgfqpoint{1.663680in}{2.587341in}}{\pgfqpoint{1.674730in}{2.587341in}}%
\pgfpathclose%
\pgfusepath{stroke,fill}%
\end{pgfscope}%
\begin{pgfscope}%
\pgfpathrectangle{\pgfqpoint{0.787074in}{0.548769in}}{\pgfqpoint{5.062926in}{3.102590in}}%
\pgfusepath{clip}%
\pgfsetbuttcap%
\pgfsetroundjoin%
\definecolor{currentfill}{rgb}{1.000000,0.498039,0.054902}%
\pgfsetfillcolor{currentfill}%
\pgfsetlinewidth{1.003750pt}%
\definecolor{currentstroke}{rgb}{1.000000,0.498039,0.054902}%
\pgfsetstrokecolor{currentstroke}%
\pgfsetdash{}{0pt}%
\pgfpathmoveto{\pgfqpoint{2.792519in}{2.817920in}}%
\pgfpathcurveto{\pgfqpoint{2.803569in}{2.817920in}}{\pgfqpoint{2.814168in}{2.822310in}}{\pgfqpoint{2.821981in}{2.830124in}}%
\pgfpathcurveto{\pgfqpoint{2.829795in}{2.837937in}}{\pgfqpoint{2.834185in}{2.848536in}}{\pgfqpoint{2.834185in}{2.859586in}}%
\pgfpathcurveto{\pgfqpoint{2.834185in}{2.870637in}}{\pgfqpoint{2.829795in}{2.881236in}}{\pgfqpoint{2.821981in}{2.889049in}}%
\pgfpathcurveto{\pgfqpoint{2.814168in}{2.896863in}}{\pgfqpoint{2.803569in}{2.901253in}}{\pgfqpoint{2.792519in}{2.901253in}}%
\pgfpathcurveto{\pgfqpoint{2.781468in}{2.901253in}}{\pgfqpoint{2.770869in}{2.896863in}}{\pgfqpoint{2.763056in}{2.889049in}}%
\pgfpathcurveto{\pgfqpoint{2.755242in}{2.881236in}}{\pgfqpoint{2.750852in}{2.870637in}}{\pgfqpoint{2.750852in}{2.859586in}}%
\pgfpathcurveto{\pgfqpoint{2.750852in}{2.848536in}}{\pgfqpoint{2.755242in}{2.837937in}}{\pgfqpoint{2.763056in}{2.830124in}}%
\pgfpathcurveto{\pgfqpoint{2.770869in}{2.822310in}}{\pgfqpoint{2.781468in}{2.817920in}}{\pgfqpoint{2.792519in}{2.817920in}}%
\pgfpathclose%
\pgfusepath{stroke,fill}%
\end{pgfscope}%
\begin{pgfscope}%
\pgfpathrectangle{\pgfqpoint{0.787074in}{0.548769in}}{\pgfqpoint{5.062926in}{3.102590in}}%
\pgfusepath{clip}%
\pgfsetbuttcap%
\pgfsetroundjoin%
\definecolor{currentfill}{rgb}{1.000000,0.498039,0.054902}%
\pgfsetfillcolor{currentfill}%
\pgfsetlinewidth{1.003750pt}%
\definecolor{currentstroke}{rgb}{1.000000,0.498039,0.054902}%
\pgfsetstrokecolor{currentstroke}%
\pgfsetdash{}{0pt}%
\pgfpathmoveto{\pgfqpoint{2.069243in}{2.246871in}}%
\pgfpathcurveto{\pgfqpoint{2.080294in}{2.246871in}}{\pgfqpoint{2.090893in}{2.251261in}}{\pgfqpoint{2.098706in}{2.259075in}}%
\pgfpathcurveto{\pgfqpoint{2.106520in}{2.266888in}}{\pgfqpoint{2.110910in}{2.277487in}}{\pgfqpoint{2.110910in}{2.288537in}}%
\pgfpathcurveto{\pgfqpoint{2.110910in}{2.299587in}}{\pgfqpoint{2.106520in}{2.310186in}}{\pgfqpoint{2.098706in}{2.318000in}}%
\pgfpathcurveto{\pgfqpoint{2.090893in}{2.325814in}}{\pgfqpoint{2.080294in}{2.330204in}}{\pgfqpoint{2.069243in}{2.330204in}}%
\pgfpathcurveto{\pgfqpoint{2.058193in}{2.330204in}}{\pgfqpoint{2.047594in}{2.325814in}}{\pgfqpoint{2.039781in}{2.318000in}}%
\pgfpathcurveto{\pgfqpoint{2.031967in}{2.310186in}}{\pgfqpoint{2.027577in}{2.299587in}}{\pgfqpoint{2.027577in}{2.288537in}}%
\pgfpathcurveto{\pgfqpoint{2.027577in}{2.277487in}}{\pgfqpoint{2.031967in}{2.266888in}}{\pgfqpoint{2.039781in}{2.259075in}}%
\pgfpathcurveto{\pgfqpoint{2.047594in}{2.251261in}}{\pgfqpoint{2.058193in}{2.246871in}}{\pgfqpoint{2.069243in}{2.246871in}}%
\pgfpathclose%
\pgfusepath{stroke,fill}%
\end{pgfscope}%
\begin{pgfscope}%
\pgfpathrectangle{\pgfqpoint{0.787074in}{0.548769in}}{\pgfqpoint{5.062926in}{3.102590in}}%
\pgfusepath{clip}%
\pgfsetbuttcap%
\pgfsetroundjoin%
\definecolor{currentfill}{rgb}{1.000000,0.498039,0.054902}%
\pgfsetfillcolor{currentfill}%
\pgfsetlinewidth{1.003750pt}%
\definecolor{currentstroke}{rgb}{1.000000,0.498039,0.054902}%
\pgfsetstrokecolor{currentstroke}%
\pgfsetdash{}{0pt}%
\pgfpathmoveto{\pgfqpoint{2.332253in}{1.776148in}}%
\pgfpathcurveto{\pgfqpoint{2.343303in}{1.776148in}}{\pgfqpoint{2.353902in}{1.780538in}}{\pgfqpoint{2.361715in}{1.788352in}}%
\pgfpathcurveto{\pgfqpoint{2.369529in}{1.796166in}}{\pgfqpoint{2.373919in}{1.806765in}}{\pgfqpoint{2.373919in}{1.817815in}}%
\pgfpathcurveto{\pgfqpoint{2.373919in}{1.828865in}}{\pgfqpoint{2.369529in}{1.839464in}}{\pgfqpoint{2.361715in}{1.847278in}}%
\pgfpathcurveto{\pgfqpoint{2.353902in}{1.855091in}}{\pgfqpoint{2.343303in}{1.859481in}}{\pgfqpoint{2.332253in}{1.859481in}}%
\pgfpathcurveto{\pgfqpoint{2.321202in}{1.859481in}}{\pgfqpoint{2.310603in}{1.855091in}}{\pgfqpoint{2.302790in}{1.847278in}}%
\pgfpathcurveto{\pgfqpoint{2.294976in}{1.839464in}}{\pgfqpoint{2.290586in}{1.828865in}}{\pgfqpoint{2.290586in}{1.817815in}}%
\pgfpathcurveto{\pgfqpoint{2.290586in}{1.806765in}}{\pgfqpoint{2.294976in}{1.796166in}}{\pgfqpoint{2.302790in}{1.788352in}}%
\pgfpathcurveto{\pgfqpoint{2.310603in}{1.780538in}}{\pgfqpoint{2.321202in}{1.776148in}}{\pgfqpoint{2.332253in}{1.776148in}}%
\pgfpathclose%
\pgfusepath{stroke,fill}%
\end{pgfscope}%
\begin{pgfscope}%
\pgfpathrectangle{\pgfqpoint{0.787074in}{0.548769in}}{\pgfqpoint{5.062926in}{3.102590in}}%
\pgfusepath{clip}%
\pgfsetbuttcap%
\pgfsetroundjoin%
\definecolor{currentfill}{rgb}{1.000000,0.498039,0.054902}%
\pgfsetfillcolor{currentfill}%
\pgfsetlinewidth{1.003750pt}%
\definecolor{currentstroke}{rgb}{1.000000,0.498039,0.054902}%
\pgfsetstrokecolor{currentstroke}%
\pgfsetdash{}{0pt}%
\pgfpathmoveto{\pgfqpoint{3.844555in}{2.992883in}}%
\pgfpathcurveto{\pgfqpoint{3.855605in}{2.992883in}}{\pgfqpoint{3.866204in}{2.997273in}}{\pgfqpoint{3.874018in}{3.005087in}}%
\pgfpathcurveto{\pgfqpoint{3.881832in}{3.012900in}}{\pgfqpoint{3.886222in}{3.023499in}}{\pgfqpoint{3.886222in}{3.034549in}}%
\pgfpathcurveto{\pgfqpoint{3.886222in}{3.045599in}}{\pgfqpoint{3.881832in}{3.056199in}}{\pgfqpoint{3.874018in}{3.064012in}}%
\pgfpathcurveto{\pgfqpoint{3.866204in}{3.071826in}}{\pgfqpoint{3.855605in}{3.076216in}}{\pgfqpoint{3.844555in}{3.076216in}}%
\pgfpathcurveto{\pgfqpoint{3.833505in}{3.076216in}}{\pgfqpoint{3.822906in}{3.071826in}}{\pgfqpoint{3.815092in}{3.064012in}}%
\pgfpathcurveto{\pgfqpoint{3.807279in}{3.056199in}}{\pgfqpoint{3.802889in}{3.045599in}}{\pgfqpoint{3.802889in}{3.034549in}}%
\pgfpathcurveto{\pgfqpoint{3.802889in}{3.023499in}}{\pgfqpoint{3.807279in}{3.012900in}}{\pgfqpoint{3.815092in}{3.005087in}}%
\pgfpathcurveto{\pgfqpoint{3.822906in}{2.997273in}}{\pgfqpoint{3.833505in}{2.992883in}}{\pgfqpoint{3.844555in}{2.992883in}}%
\pgfpathclose%
\pgfusepath{stroke,fill}%
\end{pgfscope}%
\begin{pgfscope}%
\pgfpathrectangle{\pgfqpoint{0.787074in}{0.548769in}}{\pgfqpoint{5.062926in}{3.102590in}}%
\pgfusepath{clip}%
\pgfsetbuttcap%
\pgfsetroundjoin%
\definecolor{currentfill}{rgb}{1.000000,0.498039,0.054902}%
\pgfsetfillcolor{currentfill}%
\pgfsetlinewidth{1.003750pt}%
\definecolor{currentstroke}{rgb}{1.000000,0.498039,0.054902}%
\pgfsetstrokecolor{currentstroke}%
\pgfsetdash{}{0pt}%
\pgfpathmoveto{\pgfqpoint{2.463757in}{1.537669in}}%
\pgfpathcurveto{\pgfqpoint{2.474807in}{1.537669in}}{\pgfqpoint{2.485406in}{1.542060in}}{\pgfqpoint{2.493220in}{1.549873in}}%
\pgfpathcurveto{\pgfqpoint{2.501034in}{1.557687in}}{\pgfqpoint{2.505424in}{1.568286in}}{\pgfqpoint{2.505424in}{1.579336in}}%
\pgfpathcurveto{\pgfqpoint{2.505424in}{1.590386in}}{\pgfqpoint{2.501034in}{1.600985in}}{\pgfqpoint{2.493220in}{1.608799in}}%
\pgfpathcurveto{\pgfqpoint{2.485406in}{1.616612in}}{\pgfqpoint{2.474807in}{1.621003in}}{\pgfqpoint{2.463757in}{1.621003in}}%
\pgfpathcurveto{\pgfqpoint{2.452707in}{1.621003in}}{\pgfqpoint{2.442108in}{1.616612in}}{\pgfqpoint{2.434294in}{1.608799in}}%
\pgfpathcurveto{\pgfqpoint{2.426481in}{1.600985in}}{\pgfqpoint{2.422091in}{1.590386in}}{\pgfqpoint{2.422091in}{1.579336in}}%
\pgfpathcurveto{\pgfqpoint{2.422091in}{1.568286in}}{\pgfqpoint{2.426481in}{1.557687in}}{\pgfqpoint{2.434294in}{1.549873in}}%
\pgfpathcurveto{\pgfqpoint{2.442108in}{1.542060in}}{\pgfqpoint{2.452707in}{1.537669in}}{\pgfqpoint{2.463757in}{1.537669in}}%
\pgfpathclose%
\pgfusepath{stroke,fill}%
\end{pgfscope}%
\begin{pgfscope}%
\pgfpathrectangle{\pgfqpoint{0.787074in}{0.548769in}}{\pgfqpoint{5.062926in}{3.102590in}}%
\pgfusepath{clip}%
\pgfsetbuttcap%
\pgfsetroundjoin%
\definecolor{currentfill}{rgb}{0.121569,0.466667,0.705882}%
\pgfsetfillcolor{currentfill}%
\pgfsetlinewidth{1.003750pt}%
\definecolor{currentstroke}{rgb}{0.121569,0.466667,0.705882}%
\pgfsetstrokecolor{currentstroke}%
\pgfsetdash{}{0pt}%
\pgfpathmoveto{\pgfqpoint{1.082959in}{0.681479in}}%
\pgfpathcurveto{\pgfqpoint{1.094009in}{0.681479in}}{\pgfqpoint{1.104608in}{0.685869in}}{\pgfqpoint{1.112422in}{0.693683in}}%
\pgfpathcurveto{\pgfqpoint{1.120236in}{0.701497in}}{\pgfqpoint{1.124626in}{0.712096in}}{\pgfqpoint{1.124626in}{0.723146in}}%
\pgfpathcurveto{\pgfqpoint{1.124626in}{0.734196in}}{\pgfqpoint{1.120236in}{0.744795in}}{\pgfqpoint{1.112422in}{0.752609in}}%
\pgfpathcurveto{\pgfqpoint{1.104608in}{0.760422in}}{\pgfqpoint{1.094009in}{0.764812in}}{\pgfqpoint{1.082959in}{0.764812in}}%
\pgfpathcurveto{\pgfqpoint{1.071909in}{0.764812in}}{\pgfqpoint{1.061310in}{0.760422in}}{\pgfqpoint{1.053496in}{0.752609in}}%
\pgfpathcurveto{\pgfqpoint{1.045683in}{0.744795in}}{\pgfqpoint{1.041292in}{0.734196in}}{\pgfqpoint{1.041292in}{0.723146in}}%
\pgfpathcurveto{\pgfqpoint{1.041292in}{0.712096in}}{\pgfqpoint{1.045683in}{0.701497in}}{\pgfqpoint{1.053496in}{0.693683in}}%
\pgfpathcurveto{\pgfqpoint{1.061310in}{0.685869in}}{\pgfqpoint{1.071909in}{0.681479in}}{\pgfqpoint{1.082959in}{0.681479in}}%
\pgfpathclose%
\pgfusepath{stroke,fill}%
\end{pgfscope}%
\begin{pgfscope}%
\pgfpathrectangle{\pgfqpoint{0.787074in}{0.548769in}}{\pgfqpoint{5.062926in}{3.102590in}}%
\pgfusepath{clip}%
\pgfsetbuttcap%
\pgfsetroundjoin%
\definecolor{currentfill}{rgb}{1.000000,0.498039,0.054902}%
\pgfsetfillcolor{currentfill}%
\pgfsetlinewidth{1.003750pt}%
\definecolor{currentstroke}{rgb}{1.000000,0.498039,0.054902}%
\pgfsetstrokecolor{currentstroke}%
\pgfsetdash{}{0pt}%
\pgfpathmoveto{\pgfqpoint{4.107564in}{2.885059in}}%
\pgfpathcurveto{\pgfqpoint{4.118615in}{2.885059in}}{\pgfqpoint{4.129214in}{2.889449in}}{\pgfqpoint{4.137027in}{2.897263in}}%
\pgfpathcurveto{\pgfqpoint{4.144841in}{2.905076in}}{\pgfqpoint{4.149231in}{2.915675in}}{\pgfqpoint{4.149231in}{2.926726in}}%
\pgfpathcurveto{\pgfqpoint{4.149231in}{2.937776in}}{\pgfqpoint{4.144841in}{2.948375in}}{\pgfqpoint{4.137027in}{2.956188in}}%
\pgfpathcurveto{\pgfqpoint{4.129214in}{2.964002in}}{\pgfqpoint{4.118615in}{2.968392in}}{\pgfqpoint{4.107564in}{2.968392in}}%
\pgfpathcurveto{\pgfqpoint{4.096514in}{2.968392in}}{\pgfqpoint{4.085915in}{2.964002in}}{\pgfqpoint{4.078102in}{2.956188in}}%
\pgfpathcurveto{\pgfqpoint{4.070288in}{2.948375in}}{\pgfqpoint{4.065898in}{2.937776in}}{\pgfqpoint{4.065898in}{2.926726in}}%
\pgfpathcurveto{\pgfqpoint{4.065898in}{2.915675in}}{\pgfqpoint{4.070288in}{2.905076in}}{\pgfqpoint{4.078102in}{2.897263in}}%
\pgfpathcurveto{\pgfqpoint{4.085915in}{2.889449in}}{\pgfqpoint{4.096514in}{2.885059in}}{\pgfqpoint{4.107564in}{2.885059in}}%
\pgfpathclose%
\pgfusepath{stroke,fill}%
\end{pgfscope}%
\begin{pgfscope}%
\pgfpathrectangle{\pgfqpoint{0.787074in}{0.548769in}}{\pgfqpoint{5.062926in}{3.102590in}}%
\pgfusepath{clip}%
\pgfsetbuttcap%
\pgfsetroundjoin%
\definecolor{currentfill}{rgb}{1.000000,0.498039,0.054902}%
\pgfsetfillcolor{currentfill}%
\pgfsetlinewidth{1.003750pt}%
\definecolor{currentstroke}{rgb}{1.000000,0.498039,0.054902}%
\pgfsetstrokecolor{currentstroke}%
\pgfsetdash{}{0pt}%
\pgfpathmoveto{\pgfqpoint{1.543225in}{2.017445in}}%
\pgfpathcurveto{\pgfqpoint{1.554275in}{2.017445in}}{\pgfqpoint{1.564874in}{2.021836in}}{\pgfqpoint{1.572688in}{2.029649in}}%
\pgfpathcurveto{\pgfqpoint{1.580502in}{2.037463in}}{\pgfqpoint{1.584892in}{2.048062in}}{\pgfqpoint{1.584892in}{2.059112in}}%
\pgfpathcurveto{\pgfqpoint{1.584892in}{2.070162in}}{\pgfqpoint{1.580502in}{2.080761in}}{\pgfqpoint{1.572688in}{2.088575in}}%
\pgfpathcurveto{\pgfqpoint{1.564874in}{2.096388in}}{\pgfqpoint{1.554275in}{2.100779in}}{\pgfqpoint{1.543225in}{2.100779in}}%
\pgfpathcurveto{\pgfqpoint{1.532175in}{2.100779in}}{\pgfqpoint{1.521576in}{2.096388in}}{\pgfqpoint{1.513762in}{2.088575in}}%
\pgfpathcurveto{\pgfqpoint{1.505949in}{2.080761in}}{\pgfqpoint{1.501558in}{2.070162in}}{\pgfqpoint{1.501558in}{2.059112in}}%
\pgfpathcurveto{\pgfqpoint{1.501558in}{2.048062in}}{\pgfqpoint{1.505949in}{2.037463in}}{\pgfqpoint{1.513762in}{2.029649in}}%
\pgfpathcurveto{\pgfqpoint{1.521576in}{2.021836in}}{\pgfqpoint{1.532175in}{2.017445in}}{\pgfqpoint{1.543225in}{2.017445in}}%
\pgfpathclose%
\pgfusepath{stroke,fill}%
\end{pgfscope}%
\begin{pgfscope}%
\pgfpathrectangle{\pgfqpoint{0.787074in}{0.548769in}}{\pgfqpoint{5.062926in}{3.102590in}}%
\pgfusepath{clip}%
\pgfsetbuttcap%
\pgfsetroundjoin%
\definecolor{currentfill}{rgb}{0.121569,0.466667,0.705882}%
\pgfsetfillcolor{currentfill}%
\pgfsetlinewidth{1.003750pt}%
\definecolor{currentstroke}{rgb}{0.121569,0.466667,0.705882}%
\pgfsetstrokecolor{currentstroke}%
\pgfsetdash{}{0pt}%
\pgfpathmoveto{\pgfqpoint{1.477473in}{2.239976in}}%
\pgfpathcurveto{\pgfqpoint{1.488523in}{2.239976in}}{\pgfqpoint{1.499122in}{2.244367in}}{\pgfqpoint{1.506936in}{2.252180in}}%
\pgfpathcurveto{\pgfqpoint{1.514749in}{2.259994in}}{\pgfqpoint{1.519140in}{2.270593in}}{\pgfqpoint{1.519140in}{2.281643in}}%
\pgfpathcurveto{\pgfqpoint{1.519140in}{2.292693in}}{\pgfqpoint{1.514749in}{2.303292in}}{\pgfqpoint{1.506936in}{2.311106in}}%
\pgfpathcurveto{\pgfqpoint{1.499122in}{2.318919in}}{\pgfqpoint{1.488523in}{2.323310in}}{\pgfqpoint{1.477473in}{2.323310in}}%
\pgfpathcurveto{\pgfqpoint{1.466423in}{2.323310in}}{\pgfqpoint{1.455824in}{2.318919in}}{\pgfqpoint{1.448010in}{2.311106in}}%
\pgfpathcurveto{\pgfqpoint{1.440196in}{2.303292in}}{\pgfqpoint{1.435806in}{2.292693in}}{\pgfqpoint{1.435806in}{2.281643in}}%
\pgfpathcurveto{\pgfqpoint{1.435806in}{2.270593in}}{\pgfqpoint{1.440196in}{2.259994in}}{\pgfqpoint{1.448010in}{2.252180in}}%
\pgfpathcurveto{\pgfqpoint{1.455824in}{2.244367in}}{\pgfqpoint{1.466423in}{2.239976in}}{\pgfqpoint{1.477473in}{2.239976in}}%
\pgfpathclose%
\pgfusepath{stroke,fill}%
\end{pgfscope}%
\begin{pgfscope}%
\pgfpathrectangle{\pgfqpoint{0.787074in}{0.548769in}}{\pgfqpoint{5.062926in}{3.102590in}}%
\pgfusepath{clip}%
\pgfsetbuttcap%
\pgfsetroundjoin%
\definecolor{currentfill}{rgb}{0.121569,0.466667,0.705882}%
\pgfsetfillcolor{currentfill}%
\pgfsetlinewidth{1.003750pt}%
\definecolor{currentstroke}{rgb}{0.121569,0.466667,0.705882}%
\pgfsetstrokecolor{currentstroke}%
\pgfsetdash{}{0pt}%
\pgfpathmoveto{\pgfqpoint{2.398005in}{2.686510in}}%
\pgfpathcurveto{\pgfqpoint{2.409055in}{2.686510in}}{\pgfqpoint{2.419654in}{2.690900in}}{\pgfqpoint{2.427468in}{2.698714in}}%
\pgfpathcurveto{\pgfqpoint{2.435281in}{2.706527in}}{\pgfqpoint{2.439672in}{2.717126in}}{\pgfqpoint{2.439672in}{2.728177in}}%
\pgfpathcurveto{\pgfqpoint{2.439672in}{2.739227in}}{\pgfqpoint{2.435281in}{2.749826in}}{\pgfqpoint{2.427468in}{2.757639in}}%
\pgfpathcurveto{\pgfqpoint{2.419654in}{2.765453in}}{\pgfqpoint{2.409055in}{2.769843in}}{\pgfqpoint{2.398005in}{2.769843in}}%
\pgfpathcurveto{\pgfqpoint{2.386955in}{2.769843in}}{\pgfqpoint{2.376356in}{2.765453in}}{\pgfqpoint{2.368542in}{2.757639in}}%
\pgfpathcurveto{\pgfqpoint{2.360728in}{2.749826in}}{\pgfqpoint{2.356338in}{2.739227in}}{\pgfqpoint{2.356338in}{2.728177in}}%
\pgfpathcurveto{\pgfqpoint{2.356338in}{2.717126in}}{\pgfqpoint{2.360728in}{2.706527in}}{\pgfqpoint{2.368542in}{2.698714in}}%
\pgfpathcurveto{\pgfqpoint{2.376356in}{2.690900in}}{\pgfqpoint{2.386955in}{2.686510in}}{\pgfqpoint{2.398005in}{2.686510in}}%
\pgfpathclose%
\pgfusepath{stroke,fill}%
\end{pgfscope}%
\begin{pgfscope}%
\pgfpathrectangle{\pgfqpoint{0.787074in}{0.548769in}}{\pgfqpoint{5.062926in}{3.102590in}}%
\pgfusepath{clip}%
\pgfsetbuttcap%
\pgfsetroundjoin%
\definecolor{currentfill}{rgb}{1.000000,0.498039,0.054902}%
\pgfsetfillcolor{currentfill}%
\pgfsetlinewidth{1.003750pt}%
\definecolor{currentstroke}{rgb}{1.000000,0.498039,0.054902}%
\pgfsetstrokecolor{currentstroke}%
\pgfsetdash{}{0pt}%
\pgfpathmoveto{\pgfqpoint{1.740482in}{2.173816in}}%
\pgfpathcurveto{\pgfqpoint{1.751532in}{2.173816in}}{\pgfqpoint{1.762131in}{2.178206in}}{\pgfqpoint{1.769945in}{2.186020in}}%
\pgfpathcurveto{\pgfqpoint{1.777758in}{2.193834in}}{\pgfqpoint{1.782149in}{2.204433in}}{\pgfqpoint{1.782149in}{2.215483in}}%
\pgfpathcurveto{\pgfqpoint{1.782149in}{2.226533in}}{\pgfqpoint{1.777758in}{2.237132in}}{\pgfqpoint{1.769945in}{2.244946in}}%
\pgfpathcurveto{\pgfqpoint{1.762131in}{2.252759in}}{\pgfqpoint{1.751532in}{2.257150in}}{\pgfqpoint{1.740482in}{2.257150in}}%
\pgfpathcurveto{\pgfqpoint{1.729432in}{2.257150in}}{\pgfqpoint{1.718833in}{2.252759in}}{\pgfqpoint{1.711019in}{2.244946in}}%
\pgfpathcurveto{\pgfqpoint{1.703206in}{2.237132in}}{\pgfqpoint{1.698815in}{2.226533in}}{\pgfqpoint{1.698815in}{2.215483in}}%
\pgfpathcurveto{\pgfqpoint{1.698815in}{2.204433in}}{\pgfqpoint{1.703206in}{2.193834in}}{\pgfqpoint{1.711019in}{2.186020in}}%
\pgfpathcurveto{\pgfqpoint{1.718833in}{2.178206in}}{\pgfqpoint{1.729432in}{2.173816in}}{\pgfqpoint{1.740482in}{2.173816in}}%
\pgfpathclose%
\pgfusepath{stroke,fill}%
\end{pgfscope}%
\begin{pgfscope}%
\pgfpathrectangle{\pgfqpoint{0.787074in}{0.548769in}}{\pgfqpoint{5.062926in}{3.102590in}}%
\pgfusepath{clip}%
\pgfsetbuttcap%
\pgfsetroundjoin%
\definecolor{currentfill}{rgb}{0.121569,0.466667,0.705882}%
\pgfsetfillcolor{currentfill}%
\pgfsetlinewidth{1.003750pt}%
\definecolor{currentstroke}{rgb}{0.121569,0.466667,0.705882}%
\pgfsetstrokecolor{currentstroke}%
\pgfsetdash{}{0pt}%
\pgfpathmoveto{\pgfqpoint{2.069243in}{1.676903in}}%
\pgfpathcurveto{\pgfqpoint{2.080294in}{1.676903in}}{\pgfqpoint{2.090893in}{1.681293in}}{\pgfqpoint{2.098706in}{1.689107in}}%
\pgfpathcurveto{\pgfqpoint{2.106520in}{1.696921in}}{\pgfqpoint{2.110910in}{1.707520in}}{\pgfqpoint{2.110910in}{1.718570in}}%
\pgfpathcurveto{\pgfqpoint{2.110910in}{1.729620in}}{\pgfqpoint{2.106520in}{1.740219in}}{\pgfqpoint{2.098706in}{1.748032in}}%
\pgfpathcurveto{\pgfqpoint{2.090893in}{1.755846in}}{\pgfqpoint{2.080294in}{1.760236in}}{\pgfqpoint{2.069243in}{1.760236in}}%
\pgfpathcurveto{\pgfqpoint{2.058193in}{1.760236in}}{\pgfqpoint{2.047594in}{1.755846in}}{\pgfqpoint{2.039781in}{1.748032in}}%
\pgfpathcurveto{\pgfqpoint{2.031967in}{1.740219in}}{\pgfqpoint{2.027577in}{1.729620in}}{\pgfqpoint{2.027577in}{1.718570in}}%
\pgfpathcurveto{\pgfqpoint{2.027577in}{1.707520in}}{\pgfqpoint{2.031967in}{1.696921in}}{\pgfqpoint{2.039781in}{1.689107in}}%
\pgfpathcurveto{\pgfqpoint{2.047594in}{1.681293in}}{\pgfqpoint{2.058193in}{1.676903in}}{\pgfqpoint{2.069243in}{1.676903in}}%
\pgfpathclose%
\pgfusepath{stroke,fill}%
\end{pgfscope}%
\begin{pgfscope}%
\pgfpathrectangle{\pgfqpoint{0.787074in}{0.548769in}}{\pgfqpoint{5.062926in}{3.102590in}}%
\pgfusepath{clip}%
\pgfsetbuttcap%
\pgfsetroundjoin%
\definecolor{currentfill}{rgb}{0.121569,0.466667,0.705882}%
\pgfsetfillcolor{currentfill}%
\pgfsetlinewidth{1.003750pt}%
\definecolor{currentstroke}{rgb}{0.121569,0.466667,0.705882}%
\pgfsetstrokecolor{currentstroke}%
\pgfsetdash{}{0pt}%
\pgfpathmoveto{\pgfqpoint{3.252785in}{1.092741in}}%
\pgfpathcurveto{\pgfqpoint{3.263835in}{1.092741in}}{\pgfqpoint{3.274434in}{1.097131in}}{\pgfqpoint{3.282247in}{1.104945in}}%
\pgfpathcurveto{\pgfqpoint{3.290061in}{1.112758in}}{\pgfqpoint{3.294451in}{1.123357in}}{\pgfqpoint{3.294451in}{1.134408in}}%
\pgfpathcurveto{\pgfqpoint{3.294451in}{1.145458in}}{\pgfqpoint{3.290061in}{1.156057in}}{\pgfqpoint{3.282247in}{1.163870in}}%
\pgfpathcurveto{\pgfqpoint{3.274434in}{1.171684in}}{\pgfqpoint{3.263835in}{1.176074in}}{\pgfqpoint{3.252785in}{1.176074in}}%
\pgfpathcurveto{\pgfqpoint{3.241735in}{1.176074in}}{\pgfqpoint{3.231135in}{1.171684in}}{\pgfqpoint{3.223322in}{1.163870in}}%
\pgfpathcurveto{\pgfqpoint{3.215508in}{1.156057in}}{\pgfqpoint{3.211118in}{1.145458in}}{\pgfqpoint{3.211118in}{1.134408in}}%
\pgfpathcurveto{\pgfqpoint{3.211118in}{1.123357in}}{\pgfqpoint{3.215508in}{1.112758in}}{\pgfqpoint{3.223322in}{1.104945in}}%
\pgfpathcurveto{\pgfqpoint{3.231135in}{1.097131in}}{\pgfqpoint{3.241735in}{1.092741in}}{\pgfqpoint{3.252785in}{1.092741in}}%
\pgfpathclose%
\pgfusepath{stroke,fill}%
\end{pgfscope}%
\begin{pgfscope}%
\pgfpathrectangle{\pgfqpoint{0.787074in}{0.548769in}}{\pgfqpoint{5.062926in}{3.102590in}}%
\pgfusepath{clip}%
\pgfsetbuttcap%
\pgfsetroundjoin%
\definecolor{currentfill}{rgb}{1.000000,0.498039,0.054902}%
\pgfsetfillcolor{currentfill}%
\pgfsetlinewidth{1.003750pt}%
\definecolor{currentstroke}{rgb}{1.000000,0.498039,0.054902}%
\pgfsetstrokecolor{currentstroke}%
\pgfsetdash{}{0pt}%
\pgfpathmoveto{\pgfqpoint{1.806234in}{2.255576in}}%
\pgfpathcurveto{\pgfqpoint{1.817284in}{2.255576in}}{\pgfqpoint{1.827883in}{2.259966in}}{\pgfqpoint{1.835697in}{2.267780in}}%
\pgfpathcurveto{\pgfqpoint{1.843511in}{2.275593in}}{\pgfqpoint{1.847901in}{2.286192in}}{\pgfqpoint{1.847901in}{2.297243in}}%
\pgfpathcurveto{\pgfqpoint{1.847901in}{2.308293in}}{\pgfqpoint{1.843511in}{2.318892in}}{\pgfqpoint{1.835697in}{2.326705in}}%
\pgfpathcurveto{\pgfqpoint{1.827883in}{2.334519in}}{\pgfqpoint{1.817284in}{2.338909in}}{\pgfqpoint{1.806234in}{2.338909in}}%
\pgfpathcurveto{\pgfqpoint{1.795184in}{2.338909in}}{\pgfqpoint{1.784585in}{2.334519in}}{\pgfqpoint{1.776772in}{2.326705in}}%
\pgfpathcurveto{\pgfqpoint{1.768958in}{2.318892in}}{\pgfqpoint{1.764568in}{2.308293in}}{\pgfqpoint{1.764568in}{2.297243in}}%
\pgfpathcurveto{\pgfqpoint{1.764568in}{2.286192in}}{\pgfqpoint{1.768958in}{2.275593in}}{\pgfqpoint{1.776772in}{2.267780in}}%
\pgfpathcurveto{\pgfqpoint{1.784585in}{2.259966in}}{\pgfqpoint{1.795184in}{2.255576in}}{\pgfqpoint{1.806234in}{2.255576in}}%
\pgfpathclose%
\pgfusepath{stroke,fill}%
\end{pgfscope}%
\begin{pgfscope}%
\pgfsetbuttcap%
\pgfsetroundjoin%
\definecolor{currentfill}{rgb}{0.000000,0.000000,0.000000}%
\pgfsetfillcolor{currentfill}%
\pgfsetlinewidth{0.803000pt}%
\definecolor{currentstroke}{rgb}{0.000000,0.000000,0.000000}%
\pgfsetstrokecolor{currentstroke}%
\pgfsetdash{}{0pt}%
\pgfsys@defobject{currentmarker}{\pgfqpoint{0.000000in}{-0.048611in}}{\pgfqpoint{0.000000in}{0.000000in}}{%
\pgfpathmoveto{\pgfqpoint{0.000000in}{0.000000in}}%
\pgfpathlineto{\pgfqpoint{0.000000in}{-0.048611in}}%
\pgfusepath{stroke,fill}%
}%
\begin{pgfscope}%
\pgfsys@transformshift{0.951455in}{0.548769in}%
\pgfsys@useobject{currentmarker}{}%
\end{pgfscope}%
\end{pgfscope}%
\begin{pgfscope}%
\definecolor{textcolor}{rgb}{0.000000,0.000000,0.000000}%
\pgfsetstrokecolor{textcolor}%
\pgfsetfillcolor{textcolor}%
\pgftext[x=0.951455in,y=0.451547in,,top]{\color{textcolor}\sffamily\fontsize{10.000000}{12.000000}\selectfont \(\displaystyle {0}\)}%
\end{pgfscope}%
\begin{pgfscope}%
\pgfsetbuttcap%
\pgfsetroundjoin%
\definecolor{currentfill}{rgb}{0.000000,0.000000,0.000000}%
\pgfsetfillcolor{currentfill}%
\pgfsetlinewidth{0.803000pt}%
\definecolor{currentstroke}{rgb}{0.000000,0.000000,0.000000}%
\pgfsetstrokecolor{currentstroke}%
\pgfsetdash{}{0pt}%
\pgfsys@defobject{currentmarker}{\pgfqpoint{0.000000in}{-0.048611in}}{\pgfqpoint{0.000000in}{0.000000in}}{%
\pgfpathmoveto{\pgfqpoint{0.000000in}{0.000000in}}%
\pgfpathlineto{\pgfqpoint{0.000000in}{-0.048611in}}%
\pgfusepath{stroke,fill}%
}%
\begin{pgfscope}%
\pgfsys@transformshift{1.608977in}{0.548769in}%
\pgfsys@useobject{currentmarker}{}%
\end{pgfscope}%
\end{pgfscope}%
\begin{pgfscope}%
\definecolor{textcolor}{rgb}{0.000000,0.000000,0.000000}%
\pgfsetstrokecolor{textcolor}%
\pgfsetfillcolor{textcolor}%
\pgftext[x=1.608977in,y=0.451547in,,top]{\color{textcolor}\sffamily\fontsize{10.000000}{12.000000}\selectfont \(\displaystyle {10}\)}%
\end{pgfscope}%
\begin{pgfscope}%
\pgfsetbuttcap%
\pgfsetroundjoin%
\definecolor{currentfill}{rgb}{0.000000,0.000000,0.000000}%
\pgfsetfillcolor{currentfill}%
\pgfsetlinewidth{0.803000pt}%
\definecolor{currentstroke}{rgb}{0.000000,0.000000,0.000000}%
\pgfsetstrokecolor{currentstroke}%
\pgfsetdash{}{0pt}%
\pgfsys@defobject{currentmarker}{\pgfqpoint{0.000000in}{-0.048611in}}{\pgfqpoint{0.000000in}{0.000000in}}{%
\pgfpathmoveto{\pgfqpoint{0.000000in}{0.000000in}}%
\pgfpathlineto{\pgfqpoint{0.000000in}{-0.048611in}}%
\pgfusepath{stroke,fill}%
}%
\begin{pgfscope}%
\pgfsys@transformshift{2.266500in}{0.548769in}%
\pgfsys@useobject{currentmarker}{}%
\end{pgfscope}%
\end{pgfscope}%
\begin{pgfscope}%
\definecolor{textcolor}{rgb}{0.000000,0.000000,0.000000}%
\pgfsetstrokecolor{textcolor}%
\pgfsetfillcolor{textcolor}%
\pgftext[x=2.266500in,y=0.451547in,,top]{\color{textcolor}\sffamily\fontsize{10.000000}{12.000000}\selectfont \(\displaystyle {20}\)}%
\end{pgfscope}%
\begin{pgfscope}%
\pgfsetbuttcap%
\pgfsetroundjoin%
\definecolor{currentfill}{rgb}{0.000000,0.000000,0.000000}%
\pgfsetfillcolor{currentfill}%
\pgfsetlinewidth{0.803000pt}%
\definecolor{currentstroke}{rgb}{0.000000,0.000000,0.000000}%
\pgfsetstrokecolor{currentstroke}%
\pgfsetdash{}{0pt}%
\pgfsys@defobject{currentmarker}{\pgfqpoint{0.000000in}{-0.048611in}}{\pgfqpoint{0.000000in}{0.000000in}}{%
\pgfpathmoveto{\pgfqpoint{0.000000in}{0.000000in}}%
\pgfpathlineto{\pgfqpoint{0.000000in}{-0.048611in}}%
\pgfusepath{stroke,fill}%
}%
\begin{pgfscope}%
\pgfsys@transformshift{2.924023in}{0.548769in}%
\pgfsys@useobject{currentmarker}{}%
\end{pgfscope}%
\end{pgfscope}%
\begin{pgfscope}%
\definecolor{textcolor}{rgb}{0.000000,0.000000,0.000000}%
\pgfsetstrokecolor{textcolor}%
\pgfsetfillcolor{textcolor}%
\pgftext[x=2.924023in,y=0.451547in,,top]{\color{textcolor}\sffamily\fontsize{10.000000}{12.000000}\selectfont \(\displaystyle {30}\)}%
\end{pgfscope}%
\begin{pgfscope}%
\pgfsetbuttcap%
\pgfsetroundjoin%
\definecolor{currentfill}{rgb}{0.000000,0.000000,0.000000}%
\pgfsetfillcolor{currentfill}%
\pgfsetlinewidth{0.803000pt}%
\definecolor{currentstroke}{rgb}{0.000000,0.000000,0.000000}%
\pgfsetstrokecolor{currentstroke}%
\pgfsetdash{}{0pt}%
\pgfsys@defobject{currentmarker}{\pgfqpoint{0.000000in}{-0.048611in}}{\pgfqpoint{0.000000in}{0.000000in}}{%
\pgfpathmoveto{\pgfqpoint{0.000000in}{0.000000in}}%
\pgfpathlineto{\pgfqpoint{0.000000in}{-0.048611in}}%
\pgfusepath{stroke,fill}%
}%
\begin{pgfscope}%
\pgfsys@transformshift{3.581546in}{0.548769in}%
\pgfsys@useobject{currentmarker}{}%
\end{pgfscope}%
\end{pgfscope}%
\begin{pgfscope}%
\definecolor{textcolor}{rgb}{0.000000,0.000000,0.000000}%
\pgfsetstrokecolor{textcolor}%
\pgfsetfillcolor{textcolor}%
\pgftext[x=3.581546in,y=0.451547in,,top]{\color{textcolor}\sffamily\fontsize{10.000000}{12.000000}\selectfont \(\displaystyle {40}\)}%
\end{pgfscope}%
\begin{pgfscope}%
\pgfsetbuttcap%
\pgfsetroundjoin%
\definecolor{currentfill}{rgb}{0.000000,0.000000,0.000000}%
\pgfsetfillcolor{currentfill}%
\pgfsetlinewidth{0.803000pt}%
\definecolor{currentstroke}{rgb}{0.000000,0.000000,0.000000}%
\pgfsetstrokecolor{currentstroke}%
\pgfsetdash{}{0pt}%
\pgfsys@defobject{currentmarker}{\pgfqpoint{0.000000in}{-0.048611in}}{\pgfqpoint{0.000000in}{0.000000in}}{%
\pgfpathmoveto{\pgfqpoint{0.000000in}{0.000000in}}%
\pgfpathlineto{\pgfqpoint{0.000000in}{-0.048611in}}%
\pgfusepath{stroke,fill}%
}%
\begin{pgfscope}%
\pgfsys@transformshift{4.239069in}{0.548769in}%
\pgfsys@useobject{currentmarker}{}%
\end{pgfscope}%
\end{pgfscope}%
\begin{pgfscope}%
\definecolor{textcolor}{rgb}{0.000000,0.000000,0.000000}%
\pgfsetstrokecolor{textcolor}%
\pgfsetfillcolor{textcolor}%
\pgftext[x=4.239069in,y=0.451547in,,top]{\color{textcolor}\sffamily\fontsize{10.000000}{12.000000}\selectfont \(\displaystyle {50}\)}%
\end{pgfscope}%
\begin{pgfscope}%
\pgfsetbuttcap%
\pgfsetroundjoin%
\definecolor{currentfill}{rgb}{0.000000,0.000000,0.000000}%
\pgfsetfillcolor{currentfill}%
\pgfsetlinewidth{0.803000pt}%
\definecolor{currentstroke}{rgb}{0.000000,0.000000,0.000000}%
\pgfsetstrokecolor{currentstroke}%
\pgfsetdash{}{0pt}%
\pgfsys@defobject{currentmarker}{\pgfqpoint{0.000000in}{-0.048611in}}{\pgfqpoint{0.000000in}{0.000000in}}{%
\pgfpathmoveto{\pgfqpoint{0.000000in}{0.000000in}}%
\pgfpathlineto{\pgfqpoint{0.000000in}{-0.048611in}}%
\pgfusepath{stroke,fill}%
}%
\begin{pgfscope}%
\pgfsys@transformshift{4.896592in}{0.548769in}%
\pgfsys@useobject{currentmarker}{}%
\end{pgfscope}%
\end{pgfscope}%
\begin{pgfscope}%
\definecolor{textcolor}{rgb}{0.000000,0.000000,0.000000}%
\pgfsetstrokecolor{textcolor}%
\pgfsetfillcolor{textcolor}%
\pgftext[x=4.896592in,y=0.451547in,,top]{\color{textcolor}\sffamily\fontsize{10.000000}{12.000000}\selectfont \(\displaystyle {60}\)}%
\end{pgfscope}%
\begin{pgfscope}%
\pgfsetbuttcap%
\pgfsetroundjoin%
\definecolor{currentfill}{rgb}{0.000000,0.000000,0.000000}%
\pgfsetfillcolor{currentfill}%
\pgfsetlinewidth{0.803000pt}%
\definecolor{currentstroke}{rgb}{0.000000,0.000000,0.000000}%
\pgfsetstrokecolor{currentstroke}%
\pgfsetdash{}{0pt}%
\pgfsys@defobject{currentmarker}{\pgfqpoint{0.000000in}{-0.048611in}}{\pgfqpoint{0.000000in}{0.000000in}}{%
\pgfpathmoveto{\pgfqpoint{0.000000in}{0.000000in}}%
\pgfpathlineto{\pgfqpoint{0.000000in}{-0.048611in}}%
\pgfusepath{stroke,fill}%
}%
\begin{pgfscope}%
\pgfsys@transformshift{5.554115in}{0.548769in}%
\pgfsys@useobject{currentmarker}{}%
\end{pgfscope}%
\end{pgfscope}%
\begin{pgfscope}%
\definecolor{textcolor}{rgb}{0.000000,0.000000,0.000000}%
\pgfsetstrokecolor{textcolor}%
\pgfsetfillcolor{textcolor}%
\pgftext[x=5.554115in,y=0.451547in,,top]{\color{textcolor}\sffamily\fontsize{10.000000}{12.000000}\selectfont \(\displaystyle {70}\)}%
\end{pgfscope}%
\begin{pgfscope}%
\definecolor{textcolor}{rgb}{0.000000,0.000000,0.000000}%
\pgfsetstrokecolor{textcolor}%
\pgfsetfillcolor{textcolor}%
\pgftext[x=3.318537in,y=0.272658in,,top]{\color{textcolor}\sffamily\fontsize{10.000000}{12.000000}\selectfont Number of Sources}%
\end{pgfscope}%
\begin{pgfscope}%
\pgfsetbuttcap%
\pgfsetroundjoin%
\definecolor{currentfill}{rgb}{0.000000,0.000000,0.000000}%
\pgfsetfillcolor{currentfill}%
\pgfsetlinewidth{0.803000pt}%
\definecolor{currentstroke}{rgb}{0.000000,0.000000,0.000000}%
\pgfsetstrokecolor{currentstroke}%
\pgfsetdash{}{0pt}%
\pgfsys@defobject{currentmarker}{\pgfqpoint{-0.048611in}{0.000000in}}{\pgfqpoint{0.000000in}{0.000000in}}{%
\pgfpathmoveto{\pgfqpoint{0.000000in}{0.000000in}}%
\pgfpathlineto{\pgfqpoint{-0.048611in}{0.000000in}}%
\pgfusepath{stroke,fill}%
}%
\begin{pgfscope}%
\pgfsys@transformshift{0.787074in}{0.689795in}%
\pgfsys@useobject{currentmarker}{}%
\end{pgfscope}%
\end{pgfscope}%
\begin{pgfscope}%
\definecolor{textcolor}{rgb}{0.000000,0.000000,0.000000}%
\pgfsetstrokecolor{textcolor}%
\pgfsetfillcolor{textcolor}%
\pgftext[x=0.620407in, y=0.641601in, left, base]{\color{textcolor}\sffamily\fontsize{10.000000}{12.000000}\selectfont \(\displaystyle {0}\)}%
\end{pgfscope}%
\begin{pgfscope}%
\pgfsetbuttcap%
\pgfsetroundjoin%
\definecolor{currentfill}{rgb}{0.000000,0.000000,0.000000}%
\pgfsetfillcolor{currentfill}%
\pgfsetlinewidth{0.803000pt}%
\definecolor{currentstroke}{rgb}{0.000000,0.000000,0.000000}%
\pgfsetstrokecolor{currentstroke}%
\pgfsetdash{}{0pt}%
\pgfsys@defobject{currentmarker}{\pgfqpoint{-0.048611in}{0.000000in}}{\pgfqpoint{0.000000in}{0.000000in}}{%
\pgfpathmoveto{\pgfqpoint{0.000000in}{0.000000in}}%
\pgfpathlineto{\pgfqpoint{-0.048611in}{0.000000in}}%
\pgfusepath{stroke,fill}%
}%
\begin{pgfscope}%
\pgfsys@transformshift{0.787074in}{1.062854in}%
\pgfsys@useobject{currentmarker}{}%
\end{pgfscope}%
\end{pgfscope}%
\begin{pgfscope}%
\definecolor{textcolor}{rgb}{0.000000,0.000000,0.000000}%
\pgfsetstrokecolor{textcolor}%
\pgfsetfillcolor{textcolor}%
\pgftext[x=0.412073in, y=1.014659in, left, base]{\color{textcolor}\sffamily\fontsize{10.000000}{12.000000}\selectfont \(\displaystyle {2500}\)}%
\end{pgfscope}%
\begin{pgfscope}%
\pgfsetbuttcap%
\pgfsetroundjoin%
\definecolor{currentfill}{rgb}{0.000000,0.000000,0.000000}%
\pgfsetfillcolor{currentfill}%
\pgfsetlinewidth{0.803000pt}%
\definecolor{currentstroke}{rgb}{0.000000,0.000000,0.000000}%
\pgfsetstrokecolor{currentstroke}%
\pgfsetdash{}{0pt}%
\pgfsys@defobject{currentmarker}{\pgfqpoint{-0.048611in}{0.000000in}}{\pgfqpoint{0.000000in}{0.000000in}}{%
\pgfpathmoveto{\pgfqpoint{0.000000in}{0.000000in}}%
\pgfpathlineto{\pgfqpoint{-0.048611in}{0.000000in}}%
\pgfusepath{stroke,fill}%
}%
\begin{pgfscope}%
\pgfsys@transformshift{0.787074in}{1.435912in}%
\pgfsys@useobject{currentmarker}{}%
\end{pgfscope}%
\end{pgfscope}%
\begin{pgfscope}%
\definecolor{textcolor}{rgb}{0.000000,0.000000,0.000000}%
\pgfsetstrokecolor{textcolor}%
\pgfsetfillcolor{textcolor}%
\pgftext[x=0.412073in, y=1.387718in, left, base]{\color{textcolor}\sffamily\fontsize{10.000000}{12.000000}\selectfont \(\displaystyle {5000}\)}%
\end{pgfscope}%
\begin{pgfscope}%
\pgfsetbuttcap%
\pgfsetroundjoin%
\definecolor{currentfill}{rgb}{0.000000,0.000000,0.000000}%
\pgfsetfillcolor{currentfill}%
\pgfsetlinewidth{0.803000pt}%
\definecolor{currentstroke}{rgb}{0.000000,0.000000,0.000000}%
\pgfsetstrokecolor{currentstroke}%
\pgfsetdash{}{0pt}%
\pgfsys@defobject{currentmarker}{\pgfqpoint{-0.048611in}{0.000000in}}{\pgfqpoint{0.000000in}{0.000000in}}{%
\pgfpathmoveto{\pgfqpoint{0.000000in}{0.000000in}}%
\pgfpathlineto{\pgfqpoint{-0.048611in}{0.000000in}}%
\pgfusepath{stroke,fill}%
}%
\begin{pgfscope}%
\pgfsys@transformshift{0.787074in}{1.808971in}%
\pgfsys@useobject{currentmarker}{}%
\end{pgfscope}%
\end{pgfscope}%
\begin{pgfscope}%
\definecolor{textcolor}{rgb}{0.000000,0.000000,0.000000}%
\pgfsetstrokecolor{textcolor}%
\pgfsetfillcolor{textcolor}%
\pgftext[x=0.412073in, y=1.760776in, left, base]{\color{textcolor}\sffamily\fontsize{10.000000}{12.000000}\selectfont \(\displaystyle {7500}\)}%
\end{pgfscope}%
\begin{pgfscope}%
\pgfsetbuttcap%
\pgfsetroundjoin%
\definecolor{currentfill}{rgb}{0.000000,0.000000,0.000000}%
\pgfsetfillcolor{currentfill}%
\pgfsetlinewidth{0.803000pt}%
\definecolor{currentstroke}{rgb}{0.000000,0.000000,0.000000}%
\pgfsetstrokecolor{currentstroke}%
\pgfsetdash{}{0pt}%
\pgfsys@defobject{currentmarker}{\pgfqpoint{-0.048611in}{0.000000in}}{\pgfqpoint{0.000000in}{0.000000in}}{%
\pgfpathmoveto{\pgfqpoint{0.000000in}{0.000000in}}%
\pgfpathlineto{\pgfqpoint{-0.048611in}{0.000000in}}%
\pgfusepath{stroke,fill}%
}%
\begin{pgfscope}%
\pgfsys@transformshift{0.787074in}{2.182029in}%
\pgfsys@useobject{currentmarker}{}%
\end{pgfscope}%
\end{pgfscope}%
\begin{pgfscope}%
\definecolor{textcolor}{rgb}{0.000000,0.000000,0.000000}%
\pgfsetstrokecolor{textcolor}%
\pgfsetfillcolor{textcolor}%
\pgftext[x=0.342628in, y=2.133835in, left, base]{\color{textcolor}\sffamily\fontsize{10.000000}{12.000000}\selectfont \(\displaystyle {10000}\)}%
\end{pgfscope}%
\begin{pgfscope}%
\pgfsetbuttcap%
\pgfsetroundjoin%
\definecolor{currentfill}{rgb}{0.000000,0.000000,0.000000}%
\pgfsetfillcolor{currentfill}%
\pgfsetlinewidth{0.803000pt}%
\definecolor{currentstroke}{rgb}{0.000000,0.000000,0.000000}%
\pgfsetstrokecolor{currentstroke}%
\pgfsetdash{}{0pt}%
\pgfsys@defobject{currentmarker}{\pgfqpoint{-0.048611in}{0.000000in}}{\pgfqpoint{0.000000in}{0.000000in}}{%
\pgfpathmoveto{\pgfqpoint{0.000000in}{0.000000in}}%
\pgfpathlineto{\pgfqpoint{-0.048611in}{0.000000in}}%
\pgfusepath{stroke,fill}%
}%
\begin{pgfscope}%
\pgfsys@transformshift{0.787074in}{2.555088in}%
\pgfsys@useobject{currentmarker}{}%
\end{pgfscope}%
\end{pgfscope}%
\begin{pgfscope}%
\definecolor{textcolor}{rgb}{0.000000,0.000000,0.000000}%
\pgfsetstrokecolor{textcolor}%
\pgfsetfillcolor{textcolor}%
\pgftext[x=0.342628in, y=2.506893in, left, base]{\color{textcolor}\sffamily\fontsize{10.000000}{12.000000}\selectfont \(\displaystyle {12500}\)}%
\end{pgfscope}%
\begin{pgfscope}%
\pgfsetbuttcap%
\pgfsetroundjoin%
\definecolor{currentfill}{rgb}{0.000000,0.000000,0.000000}%
\pgfsetfillcolor{currentfill}%
\pgfsetlinewidth{0.803000pt}%
\definecolor{currentstroke}{rgb}{0.000000,0.000000,0.000000}%
\pgfsetstrokecolor{currentstroke}%
\pgfsetdash{}{0pt}%
\pgfsys@defobject{currentmarker}{\pgfqpoint{-0.048611in}{0.000000in}}{\pgfqpoint{0.000000in}{0.000000in}}{%
\pgfpathmoveto{\pgfqpoint{0.000000in}{0.000000in}}%
\pgfpathlineto{\pgfqpoint{-0.048611in}{0.000000in}}%
\pgfusepath{stroke,fill}%
}%
\begin{pgfscope}%
\pgfsys@transformshift{0.787074in}{2.928146in}%
\pgfsys@useobject{currentmarker}{}%
\end{pgfscope}%
\end{pgfscope}%
\begin{pgfscope}%
\definecolor{textcolor}{rgb}{0.000000,0.000000,0.000000}%
\pgfsetstrokecolor{textcolor}%
\pgfsetfillcolor{textcolor}%
\pgftext[x=0.342628in, y=2.879952in, left, base]{\color{textcolor}\sffamily\fontsize{10.000000}{12.000000}\selectfont \(\displaystyle {15000}\)}%
\end{pgfscope}%
\begin{pgfscope}%
\pgfsetbuttcap%
\pgfsetroundjoin%
\definecolor{currentfill}{rgb}{0.000000,0.000000,0.000000}%
\pgfsetfillcolor{currentfill}%
\pgfsetlinewidth{0.803000pt}%
\definecolor{currentstroke}{rgb}{0.000000,0.000000,0.000000}%
\pgfsetstrokecolor{currentstroke}%
\pgfsetdash{}{0pt}%
\pgfsys@defobject{currentmarker}{\pgfqpoint{-0.048611in}{0.000000in}}{\pgfqpoint{0.000000in}{0.000000in}}{%
\pgfpathmoveto{\pgfqpoint{0.000000in}{0.000000in}}%
\pgfpathlineto{\pgfqpoint{-0.048611in}{0.000000in}}%
\pgfusepath{stroke,fill}%
}%
\begin{pgfscope}%
\pgfsys@transformshift{0.787074in}{3.301204in}%
\pgfsys@useobject{currentmarker}{}%
\end{pgfscope}%
\end{pgfscope}%
\begin{pgfscope}%
\definecolor{textcolor}{rgb}{0.000000,0.000000,0.000000}%
\pgfsetstrokecolor{textcolor}%
\pgfsetfillcolor{textcolor}%
\pgftext[x=0.342628in, y=3.253010in, left, base]{\color{textcolor}\sffamily\fontsize{10.000000}{12.000000}\selectfont \(\displaystyle {17500}\)}%
\end{pgfscope}%
\begin{pgfscope}%
\definecolor{textcolor}{rgb}{0.000000,0.000000,0.000000}%
\pgfsetstrokecolor{textcolor}%
\pgfsetfillcolor{textcolor}%
\pgftext[x=0.287073in,y=2.100064in,,bottom,rotate=90.000000]{\color{textcolor}\sffamily\fontsize{10.000000}{12.000000}\selectfont Maximum Memory Usage (MB)}%
\end{pgfscope}%
\begin{pgfscope}%
\pgfsetrectcap%
\pgfsetmiterjoin%
\pgfsetlinewidth{0.803000pt}%
\definecolor{currentstroke}{rgb}{0.000000,0.000000,0.000000}%
\pgfsetstrokecolor{currentstroke}%
\pgfsetdash{}{0pt}%
\pgfpathmoveto{\pgfqpoint{0.787074in}{0.548769in}}%
\pgfpathlineto{\pgfqpoint{0.787074in}{3.651359in}}%
\pgfusepath{stroke}%
\end{pgfscope}%
\begin{pgfscope}%
\pgfsetrectcap%
\pgfsetmiterjoin%
\pgfsetlinewidth{0.803000pt}%
\definecolor{currentstroke}{rgb}{0.000000,0.000000,0.000000}%
\pgfsetstrokecolor{currentstroke}%
\pgfsetdash{}{0pt}%
\pgfpathmoveto{\pgfqpoint{5.850000in}{0.548769in}}%
\pgfpathlineto{\pgfqpoint{5.850000in}{3.651359in}}%
\pgfusepath{stroke}%
\end{pgfscope}%
\begin{pgfscope}%
\pgfsetrectcap%
\pgfsetmiterjoin%
\pgfsetlinewidth{0.803000pt}%
\definecolor{currentstroke}{rgb}{0.000000,0.000000,0.000000}%
\pgfsetstrokecolor{currentstroke}%
\pgfsetdash{}{0pt}%
\pgfpathmoveto{\pgfqpoint{0.787074in}{0.548769in}}%
\pgfpathlineto{\pgfqpoint{5.850000in}{0.548769in}}%
\pgfusepath{stroke}%
\end{pgfscope}%
\begin{pgfscope}%
\pgfsetrectcap%
\pgfsetmiterjoin%
\pgfsetlinewidth{0.803000pt}%
\definecolor{currentstroke}{rgb}{0.000000,0.000000,0.000000}%
\pgfsetstrokecolor{currentstroke}%
\pgfsetdash{}{0pt}%
\pgfpathmoveto{\pgfqpoint{0.787074in}{3.651359in}}%
\pgfpathlineto{\pgfqpoint{5.850000in}{3.651359in}}%
\pgfusepath{stroke}%
\end{pgfscope}%
\begin{pgfscope}%
\definecolor{textcolor}{rgb}{0.000000,0.000000,0.000000}%
\pgfsetstrokecolor{textcolor}%
\pgfsetfillcolor{textcolor}%
\pgftext[x=3.318537in,y=3.734692in,,base]{\color{textcolor}\sffamily\fontsize{12.000000}{14.400000}\selectfont Backward}%
\end{pgfscope}%
\begin{pgfscope}%
\pgfsetbuttcap%
\pgfsetmiterjoin%
\definecolor{currentfill}{rgb}{1.000000,1.000000,1.000000}%
\pgfsetfillcolor{currentfill}%
\pgfsetfillopacity{0.800000}%
\pgfsetlinewidth{1.003750pt}%
\definecolor{currentstroke}{rgb}{0.800000,0.800000,0.800000}%
\pgfsetstrokecolor{currentstroke}%
\pgfsetstrokeopacity{0.800000}%
\pgfsetdash{}{0pt}%
\pgfpathmoveto{\pgfqpoint{4.300417in}{2.957886in}}%
\pgfpathlineto{\pgfqpoint{5.752778in}{2.957886in}}%
\pgfpathquadraticcurveto{\pgfqpoint{5.780556in}{2.957886in}}{\pgfqpoint{5.780556in}{2.985664in}}%
\pgfpathlineto{\pgfqpoint{5.780556in}{3.554136in}}%
\pgfpathquadraticcurveto{\pgfqpoint{5.780556in}{3.581914in}}{\pgfqpoint{5.752778in}{3.581914in}}%
\pgfpathlineto{\pgfqpoint{4.300417in}{3.581914in}}%
\pgfpathquadraticcurveto{\pgfqpoint{4.272639in}{3.581914in}}{\pgfqpoint{4.272639in}{3.554136in}}%
\pgfpathlineto{\pgfqpoint{4.272639in}{2.985664in}}%
\pgfpathquadraticcurveto{\pgfqpoint{4.272639in}{2.957886in}}{\pgfqpoint{4.300417in}{2.957886in}}%
\pgfpathclose%
\pgfusepath{stroke,fill}%
\end{pgfscope}%
\begin{pgfscope}%
\pgfsetbuttcap%
\pgfsetroundjoin%
\definecolor{currentfill}{rgb}{0.121569,0.466667,0.705882}%
\pgfsetfillcolor{currentfill}%
\pgfsetlinewidth{1.003750pt}%
\definecolor{currentstroke}{rgb}{0.121569,0.466667,0.705882}%
\pgfsetstrokecolor{currentstroke}%
\pgfsetdash{}{0pt}%
\pgfsys@defobject{currentmarker}{\pgfqpoint{-0.034722in}{-0.034722in}}{\pgfqpoint{0.034722in}{0.034722in}}{%
\pgfpathmoveto{\pgfqpoint{0.000000in}{-0.034722in}}%
\pgfpathcurveto{\pgfqpoint{0.009208in}{-0.034722in}}{\pgfqpoint{0.018041in}{-0.031064in}}{\pgfqpoint{0.024552in}{-0.024552in}}%
\pgfpathcurveto{\pgfqpoint{0.031064in}{-0.018041in}}{\pgfqpoint{0.034722in}{-0.009208in}}{\pgfqpoint{0.034722in}{0.000000in}}%
\pgfpathcurveto{\pgfqpoint{0.034722in}{0.009208in}}{\pgfqpoint{0.031064in}{0.018041in}}{\pgfqpoint{0.024552in}{0.024552in}}%
\pgfpathcurveto{\pgfqpoint{0.018041in}{0.031064in}}{\pgfqpoint{0.009208in}{0.034722in}}{\pgfqpoint{0.000000in}{0.034722in}}%
\pgfpathcurveto{\pgfqpoint{-0.009208in}{0.034722in}}{\pgfqpoint{-0.018041in}{0.031064in}}{\pgfqpoint{-0.024552in}{0.024552in}}%
\pgfpathcurveto{\pgfqpoint{-0.031064in}{0.018041in}}{\pgfqpoint{-0.034722in}{0.009208in}}{\pgfqpoint{-0.034722in}{0.000000in}}%
\pgfpathcurveto{\pgfqpoint{-0.034722in}{-0.009208in}}{\pgfqpoint{-0.031064in}{-0.018041in}}{\pgfqpoint{-0.024552in}{-0.024552in}}%
\pgfpathcurveto{\pgfqpoint{-0.018041in}{-0.031064in}}{\pgfqpoint{-0.009208in}{-0.034722in}}{\pgfqpoint{0.000000in}{-0.034722in}}%
\pgfpathclose%
\pgfusepath{stroke,fill}%
}%
\begin{pgfscope}%
\pgfsys@transformshift{4.467083in}{3.477748in}%
\pgfsys@useobject{currentmarker}{}%
\end{pgfscope}%
\end{pgfscope}%
\begin{pgfscope}%
\definecolor{textcolor}{rgb}{0.000000,0.000000,0.000000}%
\pgfsetstrokecolor{textcolor}%
\pgfsetfillcolor{textcolor}%
\pgftext[x=4.717083in,y=3.429136in,left,base]{\color{textcolor}\sffamily\fontsize{10.000000}{12.000000}\selectfont No Timeout}%
\end{pgfscope}%
\begin{pgfscope}%
\pgfsetbuttcap%
\pgfsetroundjoin%
\definecolor{currentfill}{rgb}{1.000000,0.498039,0.054902}%
\pgfsetfillcolor{currentfill}%
\pgfsetlinewidth{1.003750pt}%
\definecolor{currentstroke}{rgb}{1.000000,0.498039,0.054902}%
\pgfsetstrokecolor{currentstroke}%
\pgfsetdash{}{0pt}%
\pgfsys@defobject{currentmarker}{\pgfqpoint{-0.034722in}{-0.034722in}}{\pgfqpoint{0.034722in}{0.034722in}}{%
\pgfpathmoveto{\pgfqpoint{0.000000in}{-0.034722in}}%
\pgfpathcurveto{\pgfqpoint{0.009208in}{-0.034722in}}{\pgfqpoint{0.018041in}{-0.031064in}}{\pgfqpoint{0.024552in}{-0.024552in}}%
\pgfpathcurveto{\pgfqpoint{0.031064in}{-0.018041in}}{\pgfqpoint{0.034722in}{-0.009208in}}{\pgfqpoint{0.034722in}{0.000000in}}%
\pgfpathcurveto{\pgfqpoint{0.034722in}{0.009208in}}{\pgfqpoint{0.031064in}{0.018041in}}{\pgfqpoint{0.024552in}{0.024552in}}%
\pgfpathcurveto{\pgfqpoint{0.018041in}{0.031064in}}{\pgfqpoint{0.009208in}{0.034722in}}{\pgfqpoint{0.000000in}{0.034722in}}%
\pgfpathcurveto{\pgfqpoint{-0.009208in}{0.034722in}}{\pgfqpoint{-0.018041in}{0.031064in}}{\pgfqpoint{-0.024552in}{0.024552in}}%
\pgfpathcurveto{\pgfqpoint{-0.031064in}{0.018041in}}{\pgfqpoint{-0.034722in}{0.009208in}}{\pgfqpoint{-0.034722in}{0.000000in}}%
\pgfpathcurveto{\pgfqpoint{-0.034722in}{-0.009208in}}{\pgfqpoint{-0.031064in}{-0.018041in}}{\pgfqpoint{-0.024552in}{-0.024552in}}%
\pgfpathcurveto{\pgfqpoint{-0.018041in}{-0.031064in}}{\pgfqpoint{-0.009208in}{-0.034722in}}{\pgfqpoint{0.000000in}{-0.034722in}}%
\pgfpathclose%
\pgfusepath{stroke,fill}%
}%
\begin{pgfscope}%
\pgfsys@transformshift{4.467083in}{3.284136in}%
\pgfsys@useobject{currentmarker}{}%
\end{pgfscope}%
\end{pgfscope}%
\begin{pgfscope}%
\definecolor{textcolor}{rgb}{0.000000,0.000000,0.000000}%
\pgfsetstrokecolor{textcolor}%
\pgfsetfillcolor{textcolor}%
\pgftext[x=4.717083in,y=3.235525in,left,base]{\color{textcolor}\sffamily\fontsize{10.000000}{12.000000}\selectfont Time Timeout}%
\end{pgfscope}%
\begin{pgfscope}%
\pgfsetbuttcap%
\pgfsetroundjoin%
\definecolor{currentfill}{rgb}{0.839216,0.152941,0.156863}%
\pgfsetfillcolor{currentfill}%
\pgfsetlinewidth{1.003750pt}%
\definecolor{currentstroke}{rgb}{0.839216,0.152941,0.156863}%
\pgfsetstrokecolor{currentstroke}%
\pgfsetdash{}{0pt}%
\pgfsys@defobject{currentmarker}{\pgfqpoint{-0.034722in}{-0.034722in}}{\pgfqpoint{0.034722in}{0.034722in}}{%
\pgfpathmoveto{\pgfqpoint{0.000000in}{-0.034722in}}%
\pgfpathcurveto{\pgfqpoint{0.009208in}{-0.034722in}}{\pgfqpoint{0.018041in}{-0.031064in}}{\pgfqpoint{0.024552in}{-0.024552in}}%
\pgfpathcurveto{\pgfqpoint{0.031064in}{-0.018041in}}{\pgfqpoint{0.034722in}{-0.009208in}}{\pgfqpoint{0.034722in}{0.000000in}}%
\pgfpathcurveto{\pgfqpoint{0.034722in}{0.009208in}}{\pgfqpoint{0.031064in}{0.018041in}}{\pgfqpoint{0.024552in}{0.024552in}}%
\pgfpathcurveto{\pgfqpoint{0.018041in}{0.031064in}}{\pgfqpoint{0.009208in}{0.034722in}}{\pgfqpoint{0.000000in}{0.034722in}}%
\pgfpathcurveto{\pgfqpoint{-0.009208in}{0.034722in}}{\pgfqpoint{-0.018041in}{0.031064in}}{\pgfqpoint{-0.024552in}{0.024552in}}%
\pgfpathcurveto{\pgfqpoint{-0.031064in}{0.018041in}}{\pgfqpoint{-0.034722in}{0.009208in}}{\pgfqpoint{-0.034722in}{0.000000in}}%
\pgfpathcurveto{\pgfqpoint{-0.034722in}{-0.009208in}}{\pgfqpoint{-0.031064in}{-0.018041in}}{\pgfqpoint{-0.024552in}{-0.024552in}}%
\pgfpathcurveto{\pgfqpoint{-0.018041in}{-0.031064in}}{\pgfqpoint{-0.009208in}{-0.034722in}}{\pgfqpoint{0.000000in}{-0.034722in}}%
\pgfpathclose%
\pgfusepath{stroke,fill}%
}%
\begin{pgfscope}%
\pgfsys@transformshift{4.467083in}{3.090525in}%
\pgfsys@useobject{currentmarker}{}%
\end{pgfscope}%
\end{pgfscope}%
\begin{pgfscope}%
\definecolor{textcolor}{rgb}{0.000000,0.000000,0.000000}%
\pgfsetstrokecolor{textcolor}%
\pgfsetfillcolor{textcolor}%
\pgftext[x=4.717083in,y=3.041914in,left,base]{\color{textcolor}\sffamily\fontsize{10.000000}{12.000000}\selectfont Memory Timeout}%
\end{pgfscope}%
\end{pgfpicture}%
\makeatother%
\endgroup%

                }
            \end{subfigure}
            \caption{Sources}
        \end{subfigure}
        \bigbreak
        \begin{subfigure}[b]{\textwidth}
            \centering
            \begin{subfigure}[]{0.45\textwidth}
                \centering
                \resizebox{\columnwidth}{!}{
                    %% Creator: Matplotlib, PGF backend
%%
%% To include the figure in your LaTeX document, write
%%   \input{<filename>.pgf}
%%
%% Make sure the required packages are loaded in your preamble
%%   \usepackage{pgf}
%%
%% and, on pdftex
%%   \usepackage[utf8]{inputenc}\DeclareUnicodeCharacter{2212}{-}
%%
%% or, on luatex and xetex
%%   \usepackage{unicode-math}
%%
%% Figures using additional raster images can only be included by \input if
%% they are in the same directory as the main LaTeX file. For loading figures
%% from other directories you can use the `import` package
%%   \usepackage{import}
%%
%% and then include the figures with
%%   \import{<path to file>}{<filename>.pgf}
%%
%% Matplotlib used the following preamble
%%   \usepackage{amsmath}
%%   \usepackage{fontspec}
%%
\begingroup%
\makeatletter%
\begin{pgfpicture}%
\pgfpathrectangle{\pgfpointorigin}{\pgfqpoint{6.000000in}{4.000000in}}%
\pgfusepath{use as bounding box, clip}%
\begin{pgfscope}%
\pgfsetbuttcap%
\pgfsetmiterjoin%
\definecolor{currentfill}{rgb}{1.000000,1.000000,1.000000}%
\pgfsetfillcolor{currentfill}%
\pgfsetlinewidth{0.000000pt}%
\definecolor{currentstroke}{rgb}{1.000000,1.000000,1.000000}%
\pgfsetstrokecolor{currentstroke}%
\pgfsetdash{}{0pt}%
\pgfpathmoveto{\pgfqpoint{0.000000in}{0.000000in}}%
\pgfpathlineto{\pgfqpoint{6.000000in}{0.000000in}}%
\pgfpathlineto{\pgfqpoint{6.000000in}{4.000000in}}%
\pgfpathlineto{\pgfqpoint{0.000000in}{4.000000in}}%
\pgfpathclose%
\pgfusepath{fill}%
\end{pgfscope}%
\begin{pgfscope}%
\pgfsetbuttcap%
\pgfsetmiterjoin%
\definecolor{currentfill}{rgb}{1.000000,1.000000,1.000000}%
\pgfsetfillcolor{currentfill}%
\pgfsetlinewidth{0.000000pt}%
\definecolor{currentstroke}{rgb}{0.000000,0.000000,0.000000}%
\pgfsetstrokecolor{currentstroke}%
\pgfsetstrokeopacity{0.000000}%
\pgfsetdash{}{0pt}%
\pgfpathmoveto{\pgfqpoint{0.787074in}{0.548769in}}%
\pgfpathlineto{\pgfqpoint{5.850000in}{0.548769in}}%
\pgfpathlineto{\pgfqpoint{5.850000in}{3.651359in}}%
\pgfpathlineto{\pgfqpoint{0.787074in}{3.651359in}}%
\pgfpathclose%
\pgfusepath{fill}%
\end{pgfscope}%
\begin{pgfscope}%
\pgfpathrectangle{\pgfqpoint{0.787074in}{0.548769in}}{\pgfqpoint{5.062926in}{3.102590in}}%
\pgfusepath{clip}%
\pgfsetbuttcap%
\pgfsetroundjoin%
\definecolor{currentfill}{rgb}{0.121569,0.466667,0.705882}%
\pgfsetfillcolor{currentfill}%
\pgfsetlinewidth{1.003750pt}%
\definecolor{currentstroke}{rgb}{0.121569,0.466667,0.705882}%
\pgfsetstrokecolor{currentstroke}%
\pgfsetdash{}{0pt}%
\pgfpathmoveto{\pgfqpoint{1.080257in}{0.648193in}}%
\pgfpathcurveto{\pgfqpoint{1.091307in}{0.648193in}}{\pgfqpoint{1.101906in}{0.652583in}}{\pgfqpoint{1.109720in}{0.660397in}}%
\pgfpathcurveto{\pgfqpoint{1.117533in}{0.668210in}}{\pgfqpoint{1.121924in}{0.678809in}}{\pgfqpoint{1.121924in}{0.689859in}}%
\pgfpathcurveto{\pgfqpoint{1.121924in}{0.700910in}}{\pgfqpoint{1.117533in}{0.711509in}}{\pgfqpoint{1.109720in}{0.719322in}}%
\pgfpathcurveto{\pgfqpoint{1.101906in}{0.727136in}}{\pgfqpoint{1.091307in}{0.731526in}}{\pgfqpoint{1.080257in}{0.731526in}}%
\pgfpathcurveto{\pgfqpoint{1.069207in}{0.731526in}}{\pgfqpoint{1.058608in}{0.727136in}}{\pgfqpoint{1.050794in}{0.719322in}}%
\pgfpathcurveto{\pgfqpoint{1.042981in}{0.711509in}}{\pgfqpoint{1.038590in}{0.700910in}}{\pgfqpoint{1.038590in}{0.689859in}}%
\pgfpathcurveto{\pgfqpoint{1.038590in}{0.678809in}}{\pgfqpoint{1.042981in}{0.668210in}}{\pgfqpoint{1.050794in}{0.660397in}}%
\pgfpathcurveto{\pgfqpoint{1.058608in}{0.652583in}}{\pgfqpoint{1.069207in}{0.648193in}}{\pgfqpoint{1.080257in}{0.648193in}}%
\pgfpathclose%
\pgfusepath{stroke,fill}%
\end{pgfscope}%
\begin{pgfscope}%
\pgfpathrectangle{\pgfqpoint{0.787074in}{0.548769in}}{\pgfqpoint{5.062926in}{3.102590in}}%
\pgfusepath{clip}%
\pgfsetbuttcap%
\pgfsetroundjoin%
\definecolor{currentfill}{rgb}{1.000000,0.498039,0.054902}%
\pgfsetfillcolor{currentfill}%
\pgfsetlinewidth{1.003750pt}%
\definecolor{currentstroke}{rgb}{1.000000,0.498039,0.054902}%
\pgfsetstrokecolor{currentstroke}%
\pgfsetdash{}{0pt}%
\pgfpathmoveto{\pgfqpoint{2.152109in}{1.859307in}}%
\pgfpathcurveto{\pgfqpoint{2.163159in}{1.859307in}}{\pgfqpoint{2.173759in}{1.863697in}}{\pgfqpoint{2.181572in}{1.871511in}}%
\pgfpathcurveto{\pgfqpoint{2.189386in}{1.879324in}}{\pgfqpoint{2.193776in}{1.889923in}}{\pgfqpoint{2.193776in}{1.900973in}}%
\pgfpathcurveto{\pgfqpoint{2.193776in}{1.912024in}}{\pgfqpoint{2.189386in}{1.922623in}}{\pgfqpoint{2.181572in}{1.930436in}}%
\pgfpathcurveto{\pgfqpoint{2.173759in}{1.938250in}}{\pgfqpoint{2.163159in}{1.942640in}}{\pgfqpoint{2.152109in}{1.942640in}}%
\pgfpathcurveto{\pgfqpoint{2.141059in}{1.942640in}}{\pgfqpoint{2.130460in}{1.938250in}}{\pgfqpoint{2.122647in}{1.930436in}}%
\pgfpathcurveto{\pgfqpoint{2.114833in}{1.922623in}}{\pgfqpoint{2.110443in}{1.912024in}}{\pgfqpoint{2.110443in}{1.900973in}}%
\pgfpathcurveto{\pgfqpoint{2.110443in}{1.889923in}}{\pgfqpoint{2.114833in}{1.879324in}}{\pgfqpoint{2.122647in}{1.871511in}}%
\pgfpathcurveto{\pgfqpoint{2.130460in}{1.863697in}}{\pgfqpoint{2.141059in}{1.859307in}}{\pgfqpoint{2.152109in}{1.859307in}}%
\pgfpathclose%
\pgfusepath{stroke,fill}%
\end{pgfscope}%
\begin{pgfscope}%
\pgfpathrectangle{\pgfqpoint{0.787074in}{0.548769in}}{\pgfqpoint{5.062926in}{3.102590in}}%
\pgfusepath{clip}%
\pgfsetbuttcap%
\pgfsetroundjoin%
\definecolor{currentfill}{rgb}{1.000000,0.498039,0.054902}%
\pgfsetfillcolor{currentfill}%
\pgfsetlinewidth{1.003750pt}%
\definecolor{currentstroke}{rgb}{1.000000,0.498039,0.054902}%
\pgfsetstrokecolor{currentstroke}%
\pgfsetdash{}{0pt}%
\pgfpathmoveto{\pgfqpoint{1.017207in}{2.662812in}}%
\pgfpathcurveto{\pgfqpoint{1.028257in}{2.662812in}}{\pgfqpoint{1.038856in}{2.667202in}}{\pgfqpoint{1.046670in}{2.675016in}}%
\pgfpathcurveto{\pgfqpoint{1.054483in}{2.682830in}}{\pgfqpoint{1.058874in}{2.693429in}}{\pgfqpoint{1.058874in}{2.704479in}}%
\pgfpathcurveto{\pgfqpoint{1.058874in}{2.715529in}}{\pgfqpoint{1.054483in}{2.726128in}}{\pgfqpoint{1.046670in}{2.733942in}}%
\pgfpathcurveto{\pgfqpoint{1.038856in}{2.741755in}}{\pgfqpoint{1.028257in}{2.746145in}}{\pgfqpoint{1.017207in}{2.746145in}}%
\pgfpathcurveto{\pgfqpoint{1.006157in}{2.746145in}}{\pgfqpoint{0.995558in}{2.741755in}}{\pgfqpoint{0.987744in}{2.733942in}}%
\pgfpathcurveto{\pgfqpoint{0.979930in}{2.726128in}}{\pgfqpoint{0.975540in}{2.715529in}}{\pgfqpoint{0.975540in}{2.704479in}}%
\pgfpathcurveto{\pgfqpoint{0.975540in}{2.693429in}}{\pgfqpoint{0.979930in}{2.682830in}}{\pgfqpoint{0.987744in}{2.675016in}}%
\pgfpathcurveto{\pgfqpoint{0.995558in}{2.667202in}}{\pgfqpoint{1.006157in}{2.662812in}}{\pgfqpoint{1.017207in}{2.662812in}}%
\pgfpathclose%
\pgfusepath{stroke,fill}%
\end{pgfscope}%
\begin{pgfscope}%
\pgfpathrectangle{\pgfqpoint{0.787074in}{0.548769in}}{\pgfqpoint{5.062926in}{3.102590in}}%
\pgfusepath{clip}%
\pgfsetbuttcap%
\pgfsetroundjoin%
\definecolor{currentfill}{rgb}{1.000000,0.498039,0.054902}%
\pgfsetfillcolor{currentfill}%
\pgfsetlinewidth{1.003750pt}%
\definecolor{currentstroke}{rgb}{1.000000,0.498039,0.054902}%
\pgfsetstrokecolor{currentstroke}%
\pgfsetdash{}{0pt}%
\pgfpathmoveto{\pgfqpoint{2.530410in}{2.336454in}}%
\pgfpathcurveto{\pgfqpoint{2.541460in}{2.336454in}}{\pgfqpoint{2.552059in}{2.340845in}}{\pgfqpoint{2.559873in}{2.348658in}}%
\pgfpathcurveto{\pgfqpoint{2.567687in}{2.356472in}}{\pgfqpoint{2.572077in}{2.367071in}}{\pgfqpoint{2.572077in}{2.378121in}}%
\pgfpathcurveto{\pgfqpoint{2.572077in}{2.389171in}}{\pgfqpoint{2.567687in}{2.399770in}}{\pgfqpoint{2.559873in}{2.407584in}}%
\pgfpathcurveto{\pgfqpoint{2.552059in}{2.415397in}}{\pgfqpoint{2.541460in}{2.419788in}}{\pgfqpoint{2.530410in}{2.419788in}}%
\pgfpathcurveto{\pgfqpoint{2.519360in}{2.419788in}}{\pgfqpoint{2.508761in}{2.415397in}}{\pgfqpoint{2.500947in}{2.407584in}}%
\pgfpathcurveto{\pgfqpoint{2.493134in}{2.399770in}}{\pgfqpoint{2.488744in}{2.389171in}}{\pgfqpoint{2.488744in}{2.378121in}}%
\pgfpathcurveto{\pgfqpoint{2.488744in}{2.367071in}}{\pgfqpoint{2.493134in}{2.356472in}}{\pgfqpoint{2.500947in}{2.348658in}}%
\pgfpathcurveto{\pgfqpoint{2.508761in}{2.340845in}}{\pgfqpoint{2.519360in}{2.336454in}}{\pgfqpoint{2.530410in}{2.336454in}}%
\pgfpathclose%
\pgfusepath{stroke,fill}%
\end{pgfscope}%
\begin{pgfscope}%
\pgfpathrectangle{\pgfqpoint{0.787074in}{0.548769in}}{\pgfqpoint{5.062926in}{3.102590in}}%
\pgfusepath{clip}%
\pgfsetbuttcap%
\pgfsetroundjoin%
\definecolor{currentfill}{rgb}{0.121569,0.466667,0.705882}%
\pgfsetfillcolor{currentfill}%
\pgfsetlinewidth{1.003750pt}%
\definecolor{currentstroke}{rgb}{0.121569,0.466667,0.705882}%
\pgfsetstrokecolor{currentstroke}%
\pgfsetdash{}{0pt}%
\pgfpathmoveto{\pgfqpoint{1.458558in}{2.322912in}}%
\pgfpathcurveto{\pgfqpoint{1.469608in}{2.322912in}}{\pgfqpoint{1.480207in}{2.327302in}}{\pgfqpoint{1.488021in}{2.335116in}}%
\pgfpathcurveto{\pgfqpoint{1.495834in}{2.342929in}}{\pgfqpoint{1.500224in}{2.353528in}}{\pgfqpoint{1.500224in}{2.364578in}}%
\pgfpathcurveto{\pgfqpoint{1.500224in}{2.375629in}}{\pgfqpoint{1.495834in}{2.386228in}}{\pgfqpoint{1.488021in}{2.394041in}}%
\pgfpathcurveto{\pgfqpoint{1.480207in}{2.401855in}}{\pgfqpoint{1.469608in}{2.406245in}}{\pgfqpoint{1.458558in}{2.406245in}}%
\pgfpathcurveto{\pgfqpoint{1.447508in}{2.406245in}}{\pgfqpoint{1.436909in}{2.401855in}}{\pgfqpoint{1.429095in}{2.394041in}}%
\pgfpathcurveto{\pgfqpoint{1.421281in}{2.386228in}}{\pgfqpoint{1.416891in}{2.375629in}}{\pgfqpoint{1.416891in}{2.364578in}}%
\pgfpathcurveto{\pgfqpoint{1.416891in}{2.353528in}}{\pgfqpoint{1.421281in}{2.342929in}}{\pgfqpoint{1.429095in}{2.335116in}}%
\pgfpathcurveto{\pgfqpoint{1.436909in}{2.327302in}}{\pgfqpoint{1.447508in}{2.322912in}}{\pgfqpoint{1.458558in}{2.322912in}}%
\pgfpathclose%
\pgfusepath{stroke,fill}%
\end{pgfscope}%
\begin{pgfscope}%
\pgfpathrectangle{\pgfqpoint{0.787074in}{0.548769in}}{\pgfqpoint{5.062926in}{3.102590in}}%
\pgfusepath{clip}%
\pgfsetbuttcap%
\pgfsetroundjoin%
\definecolor{currentfill}{rgb}{1.000000,0.498039,0.054902}%
\pgfsetfillcolor{currentfill}%
\pgfsetlinewidth{1.003750pt}%
\definecolor{currentstroke}{rgb}{1.000000,0.498039,0.054902}%
\pgfsetstrokecolor{currentstroke}%
\pgfsetdash{}{0pt}%
\pgfpathmoveto{\pgfqpoint{4.169714in}{1.553412in}}%
\pgfpathcurveto{\pgfqpoint{4.180764in}{1.553412in}}{\pgfqpoint{4.191363in}{1.557802in}}{\pgfqpoint{4.199177in}{1.565616in}}%
\pgfpathcurveto{\pgfqpoint{4.206990in}{1.573429in}}{\pgfqpoint{4.211380in}{1.584029in}}{\pgfqpoint{4.211380in}{1.595079in}}%
\pgfpathcurveto{\pgfqpoint{4.211380in}{1.606129in}}{\pgfqpoint{4.206990in}{1.616728in}}{\pgfqpoint{4.199177in}{1.624541in}}%
\pgfpathcurveto{\pgfqpoint{4.191363in}{1.632355in}}{\pgfqpoint{4.180764in}{1.636745in}}{\pgfqpoint{4.169714in}{1.636745in}}%
\pgfpathcurveto{\pgfqpoint{4.158664in}{1.636745in}}{\pgfqpoint{4.148065in}{1.632355in}}{\pgfqpoint{4.140251in}{1.624541in}}%
\pgfpathcurveto{\pgfqpoint{4.132437in}{1.616728in}}{\pgfqpoint{4.128047in}{1.606129in}}{\pgfqpoint{4.128047in}{1.595079in}}%
\pgfpathcurveto{\pgfqpoint{4.128047in}{1.584029in}}{\pgfqpoint{4.132437in}{1.573429in}}{\pgfqpoint{4.140251in}{1.565616in}}%
\pgfpathcurveto{\pgfqpoint{4.148065in}{1.557802in}}{\pgfqpoint{4.158664in}{1.553412in}}{\pgfqpoint{4.169714in}{1.553412in}}%
\pgfpathclose%
\pgfusepath{stroke,fill}%
\end{pgfscope}%
\begin{pgfscope}%
\pgfpathrectangle{\pgfqpoint{0.787074in}{0.548769in}}{\pgfqpoint{5.062926in}{3.102590in}}%
\pgfusepath{clip}%
\pgfsetbuttcap%
\pgfsetroundjoin%
\definecolor{currentfill}{rgb}{1.000000,0.498039,0.054902}%
\pgfsetfillcolor{currentfill}%
\pgfsetlinewidth{1.003750pt}%
\definecolor{currentstroke}{rgb}{1.000000,0.498039,0.054902}%
\pgfsetstrokecolor{currentstroke}%
\pgfsetdash{}{0pt}%
\pgfpathmoveto{\pgfqpoint{1.962959in}{2.076652in}}%
\pgfpathcurveto{\pgfqpoint{1.974009in}{2.076652in}}{\pgfqpoint{1.984608in}{2.081042in}}{\pgfqpoint{1.992422in}{2.088856in}}%
\pgfpathcurveto{\pgfqpoint{2.000235in}{2.096669in}}{\pgfqpoint{2.004626in}{2.107268in}}{\pgfqpoint{2.004626in}{2.118319in}}%
\pgfpathcurveto{\pgfqpoint{2.004626in}{2.129369in}}{\pgfqpoint{2.000235in}{2.139968in}}{\pgfqpoint{1.992422in}{2.147781in}}%
\pgfpathcurveto{\pgfqpoint{1.984608in}{2.155595in}}{\pgfqpoint{1.974009in}{2.159985in}}{\pgfqpoint{1.962959in}{2.159985in}}%
\pgfpathcurveto{\pgfqpoint{1.951909in}{2.159985in}}{\pgfqpoint{1.941310in}{2.155595in}}{\pgfqpoint{1.933496in}{2.147781in}}%
\pgfpathcurveto{\pgfqpoint{1.925683in}{2.139968in}}{\pgfqpoint{1.921292in}{2.129369in}}{\pgfqpoint{1.921292in}{2.118319in}}%
\pgfpathcurveto{\pgfqpoint{1.921292in}{2.107268in}}{\pgfqpoint{1.925683in}{2.096669in}}{\pgfqpoint{1.933496in}{2.088856in}}%
\pgfpathcurveto{\pgfqpoint{1.941310in}{2.081042in}}{\pgfqpoint{1.951909in}{2.076652in}}{\pgfqpoint{1.962959in}{2.076652in}}%
\pgfpathclose%
\pgfusepath{stroke,fill}%
\end{pgfscope}%
\begin{pgfscope}%
\pgfpathrectangle{\pgfqpoint{0.787074in}{0.548769in}}{\pgfqpoint{5.062926in}{3.102590in}}%
\pgfusepath{clip}%
\pgfsetbuttcap%
\pgfsetroundjoin%
\definecolor{currentfill}{rgb}{1.000000,0.498039,0.054902}%
\pgfsetfillcolor{currentfill}%
\pgfsetlinewidth{1.003750pt}%
\definecolor{currentstroke}{rgb}{1.000000,0.498039,0.054902}%
\pgfsetstrokecolor{currentstroke}%
\pgfsetdash{}{0pt}%
\pgfpathmoveto{\pgfqpoint{1.710758in}{2.475226in}}%
\pgfpathcurveto{\pgfqpoint{1.721809in}{2.475226in}}{\pgfqpoint{1.732408in}{2.479616in}}{\pgfqpoint{1.740221in}{2.487430in}}%
\pgfpathcurveto{\pgfqpoint{1.748035in}{2.495243in}}{\pgfqpoint{1.752425in}{2.505842in}}{\pgfqpoint{1.752425in}{2.516893in}}%
\pgfpathcurveto{\pgfqpoint{1.752425in}{2.527943in}}{\pgfqpoint{1.748035in}{2.538542in}}{\pgfqpoint{1.740221in}{2.546355in}}%
\pgfpathcurveto{\pgfqpoint{1.732408in}{2.554169in}}{\pgfqpoint{1.721809in}{2.558559in}}{\pgfqpoint{1.710758in}{2.558559in}}%
\pgfpathcurveto{\pgfqpoint{1.699708in}{2.558559in}}{\pgfqpoint{1.689109in}{2.554169in}}{\pgfqpoint{1.681296in}{2.546355in}}%
\pgfpathcurveto{\pgfqpoint{1.673482in}{2.538542in}}{\pgfqpoint{1.669092in}{2.527943in}}{\pgfqpoint{1.669092in}{2.516893in}}%
\pgfpathcurveto{\pgfqpoint{1.669092in}{2.505842in}}{\pgfqpoint{1.673482in}{2.495243in}}{\pgfqpoint{1.681296in}{2.487430in}}%
\pgfpathcurveto{\pgfqpoint{1.689109in}{2.479616in}}{\pgfqpoint{1.699708in}{2.475226in}}{\pgfqpoint{1.710758in}{2.475226in}}%
\pgfpathclose%
\pgfusepath{stroke,fill}%
\end{pgfscope}%
\begin{pgfscope}%
\pgfpathrectangle{\pgfqpoint{0.787074in}{0.548769in}}{\pgfqpoint{5.062926in}{3.102590in}}%
\pgfusepath{clip}%
\pgfsetbuttcap%
\pgfsetroundjoin%
\definecolor{currentfill}{rgb}{1.000000,0.498039,0.054902}%
\pgfsetfillcolor{currentfill}%
\pgfsetlinewidth{1.003750pt}%
\definecolor{currentstroke}{rgb}{1.000000,0.498039,0.054902}%
\pgfsetstrokecolor{currentstroke}%
\pgfsetdash{}{0pt}%
\pgfpathmoveto{\pgfqpoint{1.395508in}{2.253097in}}%
\pgfpathcurveto{\pgfqpoint{1.406558in}{2.253097in}}{\pgfqpoint{1.417157in}{2.257487in}}{\pgfqpoint{1.424970in}{2.265300in}}%
\pgfpathcurveto{\pgfqpoint{1.432784in}{2.273114in}}{\pgfqpoint{1.437174in}{2.283713in}}{\pgfqpoint{1.437174in}{2.294763in}}%
\pgfpathcurveto{\pgfqpoint{1.437174in}{2.305813in}}{\pgfqpoint{1.432784in}{2.316412in}}{\pgfqpoint{1.424970in}{2.324226in}}%
\pgfpathcurveto{\pgfqpoint{1.417157in}{2.332040in}}{\pgfqpoint{1.406558in}{2.336430in}}{\pgfqpoint{1.395508in}{2.336430in}}%
\pgfpathcurveto{\pgfqpoint{1.384458in}{2.336430in}}{\pgfqpoint{1.373859in}{2.332040in}}{\pgfqpoint{1.366045in}{2.324226in}}%
\pgfpathcurveto{\pgfqpoint{1.358231in}{2.316412in}}{\pgfqpoint{1.353841in}{2.305813in}}{\pgfqpoint{1.353841in}{2.294763in}}%
\pgfpathcurveto{\pgfqpoint{1.353841in}{2.283713in}}{\pgfqpoint{1.358231in}{2.273114in}}{\pgfqpoint{1.366045in}{2.265300in}}%
\pgfpathcurveto{\pgfqpoint{1.373859in}{2.257487in}}{\pgfqpoint{1.384458in}{2.253097in}}{\pgfqpoint{1.395508in}{2.253097in}}%
\pgfpathclose%
\pgfusepath{stroke,fill}%
\end{pgfscope}%
\begin{pgfscope}%
\pgfpathrectangle{\pgfqpoint{0.787074in}{0.548769in}}{\pgfqpoint{5.062926in}{3.102590in}}%
\pgfusepath{clip}%
\pgfsetbuttcap%
\pgfsetroundjoin%
\definecolor{currentfill}{rgb}{0.121569,0.466667,0.705882}%
\pgfsetfillcolor{currentfill}%
\pgfsetlinewidth{1.003750pt}%
\definecolor{currentstroke}{rgb}{0.121569,0.466667,0.705882}%
\pgfsetstrokecolor{currentstroke}%
\pgfsetdash{}{0pt}%
\pgfpathmoveto{\pgfqpoint{1.395508in}{0.659516in}}%
\pgfpathcurveto{\pgfqpoint{1.406558in}{0.659516in}}{\pgfqpoint{1.417157in}{0.663906in}}{\pgfqpoint{1.424970in}{0.671719in}}%
\pgfpathcurveto{\pgfqpoint{1.432784in}{0.679533in}}{\pgfqpoint{1.437174in}{0.690132in}}{\pgfqpoint{1.437174in}{0.701182in}}%
\pgfpathcurveto{\pgfqpoint{1.437174in}{0.712232in}}{\pgfqpoint{1.432784in}{0.722831in}}{\pgfqpoint{1.424970in}{0.730645in}}%
\pgfpathcurveto{\pgfqpoint{1.417157in}{0.738459in}}{\pgfqpoint{1.406558in}{0.742849in}}{\pgfqpoint{1.395508in}{0.742849in}}%
\pgfpathcurveto{\pgfqpoint{1.384458in}{0.742849in}}{\pgfqpoint{1.373859in}{0.738459in}}{\pgfqpoint{1.366045in}{0.730645in}}%
\pgfpathcurveto{\pgfqpoint{1.358231in}{0.722831in}}{\pgfqpoint{1.353841in}{0.712232in}}{\pgfqpoint{1.353841in}{0.701182in}}%
\pgfpathcurveto{\pgfqpoint{1.353841in}{0.690132in}}{\pgfqpoint{1.358231in}{0.679533in}}{\pgfqpoint{1.366045in}{0.671719in}}%
\pgfpathcurveto{\pgfqpoint{1.373859in}{0.663906in}}{\pgfqpoint{1.384458in}{0.659516in}}{\pgfqpoint{1.395508in}{0.659516in}}%
\pgfpathclose%
\pgfusepath{stroke,fill}%
\end{pgfscope}%
\begin{pgfscope}%
\pgfpathrectangle{\pgfqpoint{0.787074in}{0.548769in}}{\pgfqpoint{5.062926in}{3.102590in}}%
\pgfusepath{clip}%
\pgfsetbuttcap%
\pgfsetroundjoin%
\definecolor{currentfill}{rgb}{1.000000,0.498039,0.054902}%
\pgfsetfillcolor{currentfill}%
\pgfsetlinewidth{1.003750pt}%
\definecolor{currentstroke}{rgb}{1.000000,0.498039,0.054902}%
\pgfsetstrokecolor{currentstroke}%
\pgfsetdash{}{0pt}%
\pgfpathmoveto{\pgfqpoint{1.080257in}{2.129288in}}%
\pgfpathcurveto{\pgfqpoint{1.091307in}{2.129288in}}{\pgfqpoint{1.101906in}{2.133678in}}{\pgfqpoint{1.109720in}{2.141492in}}%
\pgfpathcurveto{\pgfqpoint{1.117533in}{2.149306in}}{\pgfqpoint{1.121924in}{2.159905in}}{\pgfqpoint{1.121924in}{2.170955in}}%
\pgfpathcurveto{\pgfqpoint{1.121924in}{2.182005in}}{\pgfqpoint{1.117533in}{2.192604in}}{\pgfqpoint{1.109720in}{2.200418in}}%
\pgfpathcurveto{\pgfqpoint{1.101906in}{2.208231in}}{\pgfqpoint{1.091307in}{2.212621in}}{\pgfqpoint{1.080257in}{2.212621in}}%
\pgfpathcurveto{\pgfqpoint{1.069207in}{2.212621in}}{\pgfqpoint{1.058608in}{2.208231in}}{\pgfqpoint{1.050794in}{2.200418in}}%
\pgfpathcurveto{\pgfqpoint{1.042981in}{2.192604in}}{\pgfqpoint{1.038590in}{2.182005in}}{\pgfqpoint{1.038590in}{2.170955in}}%
\pgfpathcurveto{\pgfqpoint{1.038590in}{2.159905in}}{\pgfqpoint{1.042981in}{2.149306in}}{\pgfqpoint{1.050794in}{2.141492in}}%
\pgfpathcurveto{\pgfqpoint{1.058608in}{2.133678in}}{\pgfqpoint{1.069207in}{2.129288in}}{\pgfqpoint{1.080257in}{2.129288in}}%
\pgfpathclose%
\pgfusepath{stroke,fill}%
\end{pgfscope}%
\begin{pgfscope}%
\pgfpathrectangle{\pgfqpoint{0.787074in}{0.548769in}}{\pgfqpoint{5.062926in}{3.102590in}}%
\pgfusepath{clip}%
\pgfsetbuttcap%
\pgfsetroundjoin%
\definecolor{currentfill}{rgb}{1.000000,0.498039,0.054902}%
\pgfsetfillcolor{currentfill}%
\pgfsetlinewidth{1.003750pt}%
\definecolor{currentstroke}{rgb}{1.000000,0.498039,0.054902}%
\pgfsetstrokecolor{currentstroke}%
\pgfsetdash{}{0pt}%
\pgfpathmoveto{\pgfqpoint{1.521608in}{2.778374in}}%
\pgfpathcurveto{\pgfqpoint{1.532658in}{2.778374in}}{\pgfqpoint{1.543257in}{2.782764in}}{\pgfqpoint{1.551071in}{2.790577in}}%
\pgfpathcurveto{\pgfqpoint{1.558884in}{2.798391in}}{\pgfqpoint{1.563275in}{2.808990in}}{\pgfqpoint{1.563275in}{2.820040in}}%
\pgfpathcurveto{\pgfqpoint{1.563275in}{2.831090in}}{\pgfqpoint{1.558884in}{2.841689in}}{\pgfqpoint{1.551071in}{2.849503in}}%
\pgfpathcurveto{\pgfqpoint{1.543257in}{2.857317in}}{\pgfqpoint{1.532658in}{2.861707in}}{\pgfqpoint{1.521608in}{2.861707in}}%
\pgfpathcurveto{\pgfqpoint{1.510558in}{2.861707in}}{\pgfqpoint{1.499959in}{2.857317in}}{\pgfqpoint{1.492145in}{2.849503in}}%
\pgfpathcurveto{\pgfqpoint{1.484332in}{2.841689in}}{\pgfqpoint{1.479941in}{2.831090in}}{\pgfqpoint{1.479941in}{2.820040in}}%
\pgfpathcurveto{\pgfqpoint{1.479941in}{2.808990in}}{\pgfqpoint{1.484332in}{2.798391in}}{\pgfqpoint{1.492145in}{2.790577in}}%
\pgfpathcurveto{\pgfqpoint{1.499959in}{2.782764in}}{\pgfqpoint{1.510558in}{2.778374in}}{\pgfqpoint{1.521608in}{2.778374in}}%
\pgfpathclose%
\pgfusepath{stroke,fill}%
\end{pgfscope}%
\begin{pgfscope}%
\pgfpathrectangle{\pgfqpoint{0.787074in}{0.548769in}}{\pgfqpoint{5.062926in}{3.102590in}}%
\pgfusepath{clip}%
\pgfsetbuttcap%
\pgfsetroundjoin%
\definecolor{currentfill}{rgb}{0.121569,0.466667,0.705882}%
\pgfsetfillcolor{currentfill}%
\pgfsetlinewidth{1.003750pt}%
\definecolor{currentstroke}{rgb}{0.121569,0.466667,0.705882}%
\pgfsetstrokecolor{currentstroke}%
\pgfsetdash{}{0pt}%
\pgfpathmoveto{\pgfqpoint{1.332458in}{2.577708in}}%
\pgfpathcurveto{\pgfqpoint{1.343508in}{2.577708in}}{\pgfqpoint{1.354107in}{2.582098in}}{\pgfqpoint{1.361920in}{2.589912in}}%
\pgfpathcurveto{\pgfqpoint{1.369734in}{2.597725in}}{\pgfqpoint{1.374124in}{2.608324in}}{\pgfqpoint{1.374124in}{2.619374in}}%
\pgfpathcurveto{\pgfqpoint{1.374124in}{2.630425in}}{\pgfqpoint{1.369734in}{2.641024in}}{\pgfqpoint{1.361920in}{2.648837in}}%
\pgfpathcurveto{\pgfqpoint{1.354107in}{2.656651in}}{\pgfqpoint{1.343508in}{2.661041in}}{\pgfqpoint{1.332458in}{2.661041in}}%
\pgfpathcurveto{\pgfqpoint{1.321407in}{2.661041in}}{\pgfqpoint{1.310808in}{2.656651in}}{\pgfqpoint{1.302995in}{2.648837in}}%
\pgfpathcurveto{\pgfqpoint{1.295181in}{2.641024in}}{\pgfqpoint{1.290791in}{2.630425in}}{\pgfqpoint{1.290791in}{2.619374in}}%
\pgfpathcurveto{\pgfqpoint{1.290791in}{2.608324in}}{\pgfqpoint{1.295181in}{2.597725in}}{\pgfqpoint{1.302995in}{2.589912in}}%
\pgfpathcurveto{\pgfqpoint{1.310808in}{2.582098in}}{\pgfqpoint{1.321407in}{2.577708in}}{\pgfqpoint{1.332458in}{2.577708in}}%
\pgfpathclose%
\pgfusepath{stroke,fill}%
\end{pgfscope}%
\begin{pgfscope}%
\pgfpathrectangle{\pgfqpoint{0.787074in}{0.548769in}}{\pgfqpoint{5.062926in}{3.102590in}}%
\pgfusepath{clip}%
\pgfsetbuttcap%
\pgfsetroundjoin%
\definecolor{currentfill}{rgb}{1.000000,0.498039,0.054902}%
\pgfsetfillcolor{currentfill}%
\pgfsetlinewidth{1.003750pt}%
\definecolor{currentstroke}{rgb}{1.000000,0.498039,0.054902}%
\pgfsetstrokecolor{currentstroke}%
\pgfsetdash{}{0pt}%
\pgfpathmoveto{\pgfqpoint{1.521608in}{2.595506in}}%
\pgfpathcurveto{\pgfqpoint{1.532658in}{2.595506in}}{\pgfqpoint{1.543257in}{2.599897in}}{\pgfqpoint{1.551071in}{2.607710in}}%
\pgfpathcurveto{\pgfqpoint{1.558884in}{2.615524in}}{\pgfqpoint{1.563275in}{2.626123in}}{\pgfqpoint{1.563275in}{2.637173in}}%
\pgfpathcurveto{\pgfqpoint{1.563275in}{2.648223in}}{\pgfqpoint{1.558884in}{2.658822in}}{\pgfqpoint{1.551071in}{2.666636in}}%
\pgfpathcurveto{\pgfqpoint{1.543257in}{2.674449in}}{\pgfqpoint{1.532658in}{2.678840in}}{\pgfqpoint{1.521608in}{2.678840in}}%
\pgfpathcurveto{\pgfqpoint{1.510558in}{2.678840in}}{\pgfqpoint{1.499959in}{2.674449in}}{\pgfqpoint{1.492145in}{2.666636in}}%
\pgfpathcurveto{\pgfqpoint{1.484332in}{2.658822in}}{\pgfqpoint{1.479941in}{2.648223in}}{\pgfqpoint{1.479941in}{2.637173in}}%
\pgfpathcurveto{\pgfqpoint{1.479941in}{2.626123in}}{\pgfqpoint{1.484332in}{2.615524in}}{\pgfqpoint{1.492145in}{2.607710in}}%
\pgfpathcurveto{\pgfqpoint{1.499959in}{2.599897in}}{\pgfqpoint{1.510558in}{2.595506in}}{\pgfqpoint{1.521608in}{2.595506in}}%
\pgfpathclose%
\pgfusepath{stroke,fill}%
\end{pgfscope}%
\begin{pgfscope}%
\pgfpathrectangle{\pgfqpoint{0.787074in}{0.548769in}}{\pgfqpoint{5.062926in}{3.102590in}}%
\pgfusepath{clip}%
\pgfsetbuttcap%
\pgfsetroundjoin%
\definecolor{currentfill}{rgb}{1.000000,0.498039,0.054902}%
\pgfsetfillcolor{currentfill}%
\pgfsetlinewidth{1.003750pt}%
\definecolor{currentstroke}{rgb}{1.000000,0.498039,0.054902}%
\pgfsetstrokecolor{currentstroke}%
\pgfsetdash{}{0pt}%
\pgfpathmoveto{\pgfqpoint{2.467360in}{1.910607in}}%
\pgfpathcurveto{\pgfqpoint{2.478410in}{1.910607in}}{\pgfqpoint{2.489009in}{1.914997in}}{\pgfqpoint{2.496823in}{1.922810in}}%
\pgfpathcurveto{\pgfqpoint{2.504636in}{1.930624in}}{\pgfqpoint{2.509027in}{1.941223in}}{\pgfqpoint{2.509027in}{1.952273in}}%
\pgfpathcurveto{\pgfqpoint{2.509027in}{1.963323in}}{\pgfqpoint{2.504636in}{1.973922in}}{\pgfqpoint{2.496823in}{1.981736in}}%
\pgfpathcurveto{\pgfqpoint{2.489009in}{1.989550in}}{\pgfqpoint{2.478410in}{1.993940in}}{\pgfqpoint{2.467360in}{1.993940in}}%
\pgfpathcurveto{\pgfqpoint{2.456310in}{1.993940in}}{\pgfqpoint{2.445711in}{1.989550in}}{\pgfqpoint{2.437897in}{1.981736in}}%
\pgfpathcurveto{\pgfqpoint{2.430084in}{1.973922in}}{\pgfqpoint{2.425693in}{1.963323in}}{\pgfqpoint{2.425693in}{1.952273in}}%
\pgfpathcurveto{\pgfqpoint{2.425693in}{1.941223in}}{\pgfqpoint{2.430084in}{1.930624in}}{\pgfqpoint{2.437897in}{1.922810in}}%
\pgfpathcurveto{\pgfqpoint{2.445711in}{1.914997in}}{\pgfqpoint{2.456310in}{1.910607in}}{\pgfqpoint{2.467360in}{1.910607in}}%
\pgfpathclose%
\pgfusepath{stroke,fill}%
\end{pgfscope}%
\begin{pgfscope}%
\pgfpathrectangle{\pgfqpoint{0.787074in}{0.548769in}}{\pgfqpoint{5.062926in}{3.102590in}}%
\pgfusepath{clip}%
\pgfsetbuttcap%
\pgfsetroundjoin%
\definecolor{currentfill}{rgb}{1.000000,0.498039,0.054902}%
\pgfsetfillcolor{currentfill}%
\pgfsetlinewidth{1.003750pt}%
\definecolor{currentstroke}{rgb}{1.000000,0.498039,0.054902}%
\pgfsetstrokecolor{currentstroke}%
\pgfsetdash{}{0pt}%
\pgfpathmoveto{\pgfqpoint{1.584658in}{1.406572in}}%
\pgfpathcurveto{\pgfqpoint{1.595708in}{1.406572in}}{\pgfqpoint{1.606307in}{1.410962in}}{\pgfqpoint{1.614121in}{1.418775in}}%
\pgfpathcurveto{\pgfqpoint{1.621935in}{1.426589in}}{\pgfqpoint{1.626325in}{1.437188in}}{\pgfqpoint{1.626325in}{1.448238in}}%
\pgfpathcurveto{\pgfqpoint{1.626325in}{1.459288in}}{\pgfqpoint{1.621935in}{1.469887in}}{\pgfqpoint{1.614121in}{1.477701in}}%
\pgfpathcurveto{\pgfqpoint{1.606307in}{1.485515in}}{\pgfqpoint{1.595708in}{1.489905in}}{\pgfqpoint{1.584658in}{1.489905in}}%
\pgfpathcurveto{\pgfqpoint{1.573608in}{1.489905in}}{\pgfqpoint{1.563009in}{1.485515in}}{\pgfqpoint{1.555195in}{1.477701in}}%
\pgfpathcurveto{\pgfqpoint{1.547382in}{1.469887in}}{\pgfqpoint{1.542991in}{1.459288in}}{\pgfqpoint{1.542991in}{1.448238in}}%
\pgfpathcurveto{\pgfqpoint{1.542991in}{1.437188in}}{\pgfqpoint{1.547382in}{1.426589in}}{\pgfqpoint{1.555195in}{1.418775in}}%
\pgfpathcurveto{\pgfqpoint{1.563009in}{1.410962in}}{\pgfqpoint{1.573608in}{1.406572in}}{\pgfqpoint{1.584658in}{1.406572in}}%
\pgfpathclose%
\pgfusepath{stroke,fill}%
\end{pgfscope}%
\begin{pgfscope}%
\pgfpathrectangle{\pgfqpoint{0.787074in}{0.548769in}}{\pgfqpoint{5.062926in}{3.102590in}}%
\pgfusepath{clip}%
\pgfsetbuttcap%
\pgfsetroundjoin%
\definecolor{currentfill}{rgb}{1.000000,0.498039,0.054902}%
\pgfsetfillcolor{currentfill}%
\pgfsetlinewidth{1.003750pt}%
\definecolor{currentstroke}{rgb}{1.000000,0.498039,0.054902}%
\pgfsetstrokecolor{currentstroke}%
\pgfsetdash{}{0pt}%
\pgfpathmoveto{\pgfqpoint{1.584658in}{2.078914in}}%
\pgfpathcurveto{\pgfqpoint{1.595708in}{2.078914in}}{\pgfqpoint{1.606307in}{2.083304in}}{\pgfqpoint{1.614121in}{2.091118in}}%
\pgfpathcurveto{\pgfqpoint{1.621935in}{2.098931in}}{\pgfqpoint{1.626325in}{2.109530in}}{\pgfqpoint{1.626325in}{2.120580in}}%
\pgfpathcurveto{\pgfqpoint{1.626325in}{2.131630in}}{\pgfqpoint{1.621935in}{2.142229in}}{\pgfqpoint{1.614121in}{2.150043in}}%
\pgfpathcurveto{\pgfqpoint{1.606307in}{2.157857in}}{\pgfqpoint{1.595708in}{2.162247in}}{\pgfqpoint{1.584658in}{2.162247in}}%
\pgfpathcurveto{\pgfqpoint{1.573608in}{2.162247in}}{\pgfqpoint{1.563009in}{2.157857in}}{\pgfqpoint{1.555195in}{2.150043in}}%
\pgfpathcurveto{\pgfqpoint{1.547382in}{2.142229in}}{\pgfqpoint{1.542991in}{2.131630in}}{\pgfqpoint{1.542991in}{2.120580in}}%
\pgfpathcurveto{\pgfqpoint{1.542991in}{2.109530in}}{\pgfqpoint{1.547382in}{2.098931in}}{\pgfqpoint{1.555195in}{2.091118in}}%
\pgfpathcurveto{\pgfqpoint{1.563009in}{2.083304in}}{\pgfqpoint{1.573608in}{2.078914in}}{\pgfqpoint{1.584658in}{2.078914in}}%
\pgfpathclose%
\pgfusepath{stroke,fill}%
\end{pgfscope}%
\begin{pgfscope}%
\pgfpathrectangle{\pgfqpoint{0.787074in}{0.548769in}}{\pgfqpoint{5.062926in}{3.102590in}}%
\pgfusepath{clip}%
\pgfsetbuttcap%
\pgfsetroundjoin%
\definecolor{currentfill}{rgb}{1.000000,0.498039,0.054902}%
\pgfsetfillcolor{currentfill}%
\pgfsetlinewidth{1.003750pt}%
\definecolor{currentstroke}{rgb}{1.000000,0.498039,0.054902}%
\pgfsetstrokecolor{currentstroke}%
\pgfsetdash{}{0pt}%
\pgfpathmoveto{\pgfqpoint{1.647708in}{1.659203in}}%
\pgfpathcurveto{\pgfqpoint{1.658758in}{1.659203in}}{\pgfqpoint{1.669357in}{1.663593in}}{\pgfqpoint{1.677171in}{1.671407in}}%
\pgfpathcurveto{\pgfqpoint{1.684985in}{1.679220in}}{\pgfqpoint{1.689375in}{1.689820in}}{\pgfqpoint{1.689375in}{1.700870in}}%
\pgfpathcurveto{\pgfqpoint{1.689375in}{1.711920in}}{\pgfqpoint{1.684985in}{1.722519in}}{\pgfqpoint{1.677171in}{1.730332in}}%
\pgfpathcurveto{\pgfqpoint{1.669357in}{1.738146in}}{\pgfqpoint{1.658758in}{1.742536in}}{\pgfqpoint{1.647708in}{1.742536in}}%
\pgfpathcurveto{\pgfqpoint{1.636658in}{1.742536in}}{\pgfqpoint{1.626059in}{1.738146in}}{\pgfqpoint{1.618245in}{1.730332in}}%
\pgfpathcurveto{\pgfqpoint{1.610432in}{1.722519in}}{\pgfqpoint{1.606042in}{1.711920in}}{\pgfqpoint{1.606042in}{1.700870in}}%
\pgfpathcurveto{\pgfqpoint{1.606042in}{1.689820in}}{\pgfqpoint{1.610432in}{1.679220in}}{\pgfqpoint{1.618245in}{1.671407in}}%
\pgfpathcurveto{\pgfqpoint{1.626059in}{1.663593in}}{\pgfqpoint{1.636658in}{1.659203in}}{\pgfqpoint{1.647708in}{1.659203in}}%
\pgfpathclose%
\pgfusepath{stroke,fill}%
\end{pgfscope}%
\begin{pgfscope}%
\pgfpathrectangle{\pgfqpoint{0.787074in}{0.548769in}}{\pgfqpoint{5.062926in}{3.102590in}}%
\pgfusepath{clip}%
\pgfsetbuttcap%
\pgfsetroundjoin%
\definecolor{currentfill}{rgb}{0.121569,0.466667,0.705882}%
\pgfsetfillcolor{currentfill}%
\pgfsetlinewidth{1.003750pt}%
\definecolor{currentstroke}{rgb}{0.121569,0.466667,0.705882}%
\pgfsetstrokecolor{currentstroke}%
\pgfsetdash{}{0pt}%
\pgfpathmoveto{\pgfqpoint{1.332458in}{1.472684in}}%
\pgfpathcurveto{\pgfqpoint{1.343508in}{1.472684in}}{\pgfqpoint{1.354107in}{1.477075in}}{\pgfqpoint{1.361920in}{1.484888in}}%
\pgfpathcurveto{\pgfqpoint{1.369734in}{1.492702in}}{\pgfqpoint{1.374124in}{1.503301in}}{\pgfqpoint{1.374124in}{1.514351in}}%
\pgfpathcurveto{\pgfqpoint{1.374124in}{1.525401in}}{\pgfqpoint{1.369734in}{1.536000in}}{\pgfqpoint{1.361920in}{1.543814in}}%
\pgfpathcurveto{\pgfqpoint{1.354107in}{1.551627in}}{\pgfqpoint{1.343508in}{1.556018in}}{\pgfqpoint{1.332458in}{1.556018in}}%
\pgfpathcurveto{\pgfqpoint{1.321407in}{1.556018in}}{\pgfqpoint{1.310808in}{1.551627in}}{\pgfqpoint{1.302995in}{1.543814in}}%
\pgfpathcurveto{\pgfqpoint{1.295181in}{1.536000in}}{\pgfqpoint{1.290791in}{1.525401in}}{\pgfqpoint{1.290791in}{1.514351in}}%
\pgfpathcurveto{\pgfqpoint{1.290791in}{1.503301in}}{\pgfqpoint{1.295181in}{1.492702in}}{\pgfqpoint{1.302995in}{1.484888in}}%
\pgfpathcurveto{\pgfqpoint{1.310808in}{1.477075in}}{\pgfqpoint{1.321407in}{1.472684in}}{\pgfqpoint{1.332458in}{1.472684in}}%
\pgfpathclose%
\pgfusepath{stroke,fill}%
\end{pgfscope}%
\begin{pgfscope}%
\pgfpathrectangle{\pgfqpoint{0.787074in}{0.548769in}}{\pgfqpoint{5.062926in}{3.102590in}}%
\pgfusepath{clip}%
\pgfsetbuttcap%
\pgfsetroundjoin%
\definecolor{currentfill}{rgb}{1.000000,0.498039,0.054902}%
\pgfsetfillcolor{currentfill}%
\pgfsetlinewidth{1.003750pt}%
\definecolor{currentstroke}{rgb}{1.000000,0.498039,0.054902}%
\pgfsetstrokecolor{currentstroke}%
\pgfsetdash{}{0pt}%
\pgfpathmoveto{\pgfqpoint{2.467360in}{2.453351in}}%
\pgfpathcurveto{\pgfqpoint{2.478410in}{2.453351in}}{\pgfqpoint{2.489009in}{2.457742in}}{\pgfqpoint{2.496823in}{2.465555in}}%
\pgfpathcurveto{\pgfqpoint{2.504636in}{2.473369in}}{\pgfqpoint{2.509027in}{2.483968in}}{\pgfqpoint{2.509027in}{2.495018in}}%
\pgfpathcurveto{\pgfqpoint{2.509027in}{2.506068in}}{\pgfqpoint{2.504636in}{2.516667in}}{\pgfqpoint{2.496823in}{2.524481in}}%
\pgfpathcurveto{\pgfqpoint{2.489009in}{2.532294in}}{\pgfqpoint{2.478410in}{2.536685in}}{\pgfqpoint{2.467360in}{2.536685in}}%
\pgfpathcurveto{\pgfqpoint{2.456310in}{2.536685in}}{\pgfqpoint{2.445711in}{2.532294in}}{\pgfqpoint{2.437897in}{2.524481in}}%
\pgfpathcurveto{\pgfqpoint{2.430084in}{2.516667in}}{\pgfqpoint{2.425693in}{2.506068in}}{\pgfqpoint{2.425693in}{2.495018in}}%
\pgfpathcurveto{\pgfqpoint{2.425693in}{2.483968in}}{\pgfqpoint{2.430084in}{2.473369in}}{\pgfqpoint{2.437897in}{2.465555in}}%
\pgfpathcurveto{\pgfqpoint{2.445711in}{2.457742in}}{\pgfqpoint{2.456310in}{2.453351in}}{\pgfqpoint{2.467360in}{2.453351in}}%
\pgfpathclose%
\pgfusepath{stroke,fill}%
\end{pgfscope}%
\begin{pgfscope}%
\pgfpathrectangle{\pgfqpoint{0.787074in}{0.548769in}}{\pgfqpoint{5.062926in}{3.102590in}}%
\pgfusepath{clip}%
\pgfsetbuttcap%
\pgfsetroundjoin%
\definecolor{currentfill}{rgb}{0.121569,0.466667,0.705882}%
\pgfsetfillcolor{currentfill}%
\pgfsetlinewidth{1.003750pt}%
\definecolor{currentstroke}{rgb}{0.121569,0.466667,0.705882}%
\pgfsetstrokecolor{currentstroke}%
\pgfsetdash{}{0pt}%
\pgfpathmoveto{\pgfqpoint{1.332458in}{0.648305in}}%
\pgfpathcurveto{\pgfqpoint{1.343508in}{0.648305in}}{\pgfqpoint{1.354107in}{0.652695in}}{\pgfqpoint{1.361920in}{0.660509in}}%
\pgfpathcurveto{\pgfqpoint{1.369734in}{0.668322in}}{\pgfqpoint{1.374124in}{0.678921in}}{\pgfqpoint{1.374124in}{0.689972in}}%
\pgfpathcurveto{\pgfqpoint{1.374124in}{0.701022in}}{\pgfqpoint{1.369734in}{0.711621in}}{\pgfqpoint{1.361920in}{0.719434in}}%
\pgfpathcurveto{\pgfqpoint{1.354107in}{0.727248in}}{\pgfqpoint{1.343508in}{0.731638in}}{\pgfqpoint{1.332458in}{0.731638in}}%
\pgfpathcurveto{\pgfqpoint{1.321407in}{0.731638in}}{\pgfqpoint{1.310808in}{0.727248in}}{\pgfqpoint{1.302995in}{0.719434in}}%
\pgfpathcurveto{\pgfqpoint{1.295181in}{0.711621in}}{\pgfqpoint{1.290791in}{0.701022in}}{\pgfqpoint{1.290791in}{0.689972in}}%
\pgfpathcurveto{\pgfqpoint{1.290791in}{0.678921in}}{\pgfqpoint{1.295181in}{0.668322in}}{\pgfqpoint{1.302995in}{0.660509in}}%
\pgfpathcurveto{\pgfqpoint{1.310808in}{0.652695in}}{\pgfqpoint{1.321407in}{0.648305in}}{\pgfqpoint{1.332458in}{0.648305in}}%
\pgfpathclose%
\pgfusepath{stroke,fill}%
\end{pgfscope}%
\begin{pgfscope}%
\pgfpathrectangle{\pgfqpoint{0.787074in}{0.548769in}}{\pgfqpoint{5.062926in}{3.102590in}}%
\pgfusepath{clip}%
\pgfsetbuttcap%
\pgfsetroundjoin%
\definecolor{currentfill}{rgb}{0.121569,0.466667,0.705882}%
\pgfsetfillcolor{currentfill}%
\pgfsetlinewidth{1.003750pt}%
\definecolor{currentstroke}{rgb}{0.121569,0.466667,0.705882}%
\pgfsetstrokecolor{currentstroke}%
\pgfsetdash{}{0pt}%
\pgfpathmoveto{\pgfqpoint{1.017207in}{1.422655in}}%
\pgfpathcurveto{\pgfqpoint{1.028257in}{1.422655in}}{\pgfqpoint{1.038856in}{1.427045in}}{\pgfqpoint{1.046670in}{1.434858in}}%
\pgfpathcurveto{\pgfqpoint{1.054483in}{1.442672in}}{\pgfqpoint{1.058874in}{1.453271in}}{\pgfqpoint{1.058874in}{1.464321in}}%
\pgfpathcurveto{\pgfqpoint{1.058874in}{1.475371in}}{\pgfqpoint{1.054483in}{1.485970in}}{\pgfqpoint{1.046670in}{1.493784in}}%
\pgfpathcurveto{\pgfqpoint{1.038856in}{1.501598in}}{\pgfqpoint{1.028257in}{1.505988in}}{\pgfqpoint{1.017207in}{1.505988in}}%
\pgfpathcurveto{\pgfqpoint{1.006157in}{1.505988in}}{\pgfqpoint{0.995558in}{1.501598in}}{\pgfqpoint{0.987744in}{1.493784in}}%
\pgfpathcurveto{\pgfqpoint{0.979930in}{1.485970in}}{\pgfqpoint{0.975540in}{1.475371in}}{\pgfqpoint{0.975540in}{1.464321in}}%
\pgfpathcurveto{\pgfqpoint{0.975540in}{1.453271in}}{\pgfqpoint{0.979930in}{1.442672in}}{\pgfqpoint{0.987744in}{1.434858in}}%
\pgfpathcurveto{\pgfqpoint{0.995558in}{1.427045in}}{\pgfqpoint{1.006157in}{1.422655in}}{\pgfqpoint{1.017207in}{1.422655in}}%
\pgfpathclose%
\pgfusepath{stroke,fill}%
\end{pgfscope}%
\begin{pgfscope}%
\pgfpathrectangle{\pgfqpoint{0.787074in}{0.548769in}}{\pgfqpoint{5.062926in}{3.102590in}}%
\pgfusepath{clip}%
\pgfsetbuttcap%
\pgfsetroundjoin%
\definecolor{currentfill}{rgb}{1.000000,0.498039,0.054902}%
\pgfsetfillcolor{currentfill}%
\pgfsetlinewidth{1.003750pt}%
\definecolor{currentstroke}{rgb}{1.000000,0.498039,0.054902}%
\pgfsetstrokecolor{currentstroke}%
\pgfsetdash{}{0pt}%
\pgfpathmoveto{\pgfqpoint{1.647708in}{2.833777in}}%
\pgfpathcurveto{\pgfqpoint{1.658758in}{2.833777in}}{\pgfqpoint{1.669357in}{2.838167in}}{\pgfqpoint{1.677171in}{2.845981in}}%
\pgfpathcurveto{\pgfqpoint{1.684985in}{2.853795in}}{\pgfqpoint{1.689375in}{2.864394in}}{\pgfqpoint{1.689375in}{2.875444in}}%
\pgfpathcurveto{\pgfqpoint{1.689375in}{2.886494in}}{\pgfqpoint{1.684985in}{2.897093in}}{\pgfqpoint{1.677171in}{2.904907in}}%
\pgfpathcurveto{\pgfqpoint{1.669357in}{2.912720in}}{\pgfqpoint{1.658758in}{2.917111in}}{\pgfqpoint{1.647708in}{2.917111in}}%
\pgfpathcurveto{\pgfqpoint{1.636658in}{2.917111in}}{\pgfqpoint{1.626059in}{2.912720in}}{\pgfqpoint{1.618245in}{2.904907in}}%
\pgfpathcurveto{\pgfqpoint{1.610432in}{2.897093in}}{\pgfqpoint{1.606042in}{2.886494in}}{\pgfqpoint{1.606042in}{2.875444in}}%
\pgfpathcurveto{\pgfqpoint{1.606042in}{2.864394in}}{\pgfqpoint{1.610432in}{2.853795in}}{\pgfqpoint{1.618245in}{2.845981in}}%
\pgfpathcurveto{\pgfqpoint{1.626059in}{2.838167in}}{\pgfqpoint{1.636658in}{2.833777in}}{\pgfqpoint{1.647708in}{2.833777in}}%
\pgfpathclose%
\pgfusepath{stroke,fill}%
\end{pgfscope}%
\begin{pgfscope}%
\pgfpathrectangle{\pgfqpoint{0.787074in}{0.548769in}}{\pgfqpoint{5.062926in}{3.102590in}}%
\pgfusepath{clip}%
\pgfsetbuttcap%
\pgfsetroundjoin%
\definecolor{currentfill}{rgb}{0.121569,0.466667,0.705882}%
\pgfsetfillcolor{currentfill}%
\pgfsetlinewidth{1.003750pt}%
\definecolor{currentstroke}{rgb}{0.121569,0.466667,0.705882}%
\pgfsetstrokecolor{currentstroke}%
\pgfsetdash{}{0pt}%
\pgfpathmoveto{\pgfqpoint{4.043614in}{1.540711in}}%
\pgfpathcurveto{\pgfqpoint{4.054664in}{1.540711in}}{\pgfqpoint{4.065263in}{1.545101in}}{\pgfqpoint{4.073076in}{1.552915in}}%
\pgfpathcurveto{\pgfqpoint{4.080890in}{1.560729in}}{\pgfqpoint{4.085280in}{1.571328in}}{\pgfqpoint{4.085280in}{1.582378in}}%
\pgfpathcurveto{\pgfqpoint{4.085280in}{1.593428in}}{\pgfqpoint{4.080890in}{1.604027in}}{\pgfqpoint{4.073076in}{1.611841in}}%
\pgfpathcurveto{\pgfqpoint{4.065263in}{1.619654in}}{\pgfqpoint{4.054664in}{1.624044in}}{\pgfqpoint{4.043614in}{1.624044in}}%
\pgfpathcurveto{\pgfqpoint{4.032563in}{1.624044in}}{\pgfqpoint{4.021964in}{1.619654in}}{\pgfqpoint{4.014151in}{1.611841in}}%
\pgfpathcurveto{\pgfqpoint{4.006337in}{1.604027in}}{\pgfqpoint{4.001947in}{1.593428in}}{\pgfqpoint{4.001947in}{1.582378in}}%
\pgfpathcurveto{\pgfqpoint{4.001947in}{1.571328in}}{\pgfqpoint{4.006337in}{1.560729in}}{\pgfqpoint{4.014151in}{1.552915in}}%
\pgfpathcurveto{\pgfqpoint{4.021964in}{1.545101in}}{\pgfqpoint{4.032563in}{1.540711in}}{\pgfqpoint{4.043614in}{1.540711in}}%
\pgfpathclose%
\pgfusepath{stroke,fill}%
\end{pgfscope}%
\begin{pgfscope}%
\pgfpathrectangle{\pgfqpoint{0.787074in}{0.548769in}}{\pgfqpoint{5.062926in}{3.102590in}}%
\pgfusepath{clip}%
\pgfsetbuttcap%
\pgfsetroundjoin%
\definecolor{currentfill}{rgb}{0.121569,0.466667,0.705882}%
\pgfsetfillcolor{currentfill}%
\pgfsetlinewidth{1.003750pt}%
\definecolor{currentstroke}{rgb}{0.121569,0.466667,0.705882}%
\pgfsetstrokecolor{currentstroke}%
\pgfsetdash{}{0pt}%
\pgfpathmoveto{\pgfqpoint{1.269407in}{1.951746in}}%
\pgfpathcurveto{\pgfqpoint{1.280458in}{1.951746in}}{\pgfqpoint{1.291057in}{1.956136in}}{\pgfqpoint{1.298870in}{1.963950in}}%
\pgfpathcurveto{\pgfqpoint{1.306684in}{1.971763in}}{\pgfqpoint{1.311074in}{1.982362in}}{\pgfqpoint{1.311074in}{1.993412in}}%
\pgfpathcurveto{\pgfqpoint{1.311074in}{2.004463in}}{\pgfqpoint{1.306684in}{2.015062in}}{\pgfqpoint{1.298870in}{2.022875in}}%
\pgfpathcurveto{\pgfqpoint{1.291057in}{2.030689in}}{\pgfqpoint{1.280458in}{2.035079in}}{\pgfqpoint{1.269407in}{2.035079in}}%
\pgfpathcurveto{\pgfqpoint{1.258357in}{2.035079in}}{\pgfqpoint{1.247758in}{2.030689in}}{\pgfqpoint{1.239945in}{2.022875in}}%
\pgfpathcurveto{\pgfqpoint{1.232131in}{2.015062in}}{\pgfqpoint{1.227741in}{2.004463in}}{\pgfqpoint{1.227741in}{1.993412in}}%
\pgfpathcurveto{\pgfqpoint{1.227741in}{1.982362in}}{\pgfqpoint{1.232131in}{1.971763in}}{\pgfqpoint{1.239945in}{1.963950in}}%
\pgfpathcurveto{\pgfqpoint{1.247758in}{1.956136in}}{\pgfqpoint{1.258357in}{1.951746in}}{\pgfqpoint{1.269407in}{1.951746in}}%
\pgfpathclose%
\pgfusepath{stroke,fill}%
\end{pgfscope}%
\begin{pgfscope}%
\pgfpathrectangle{\pgfqpoint{0.787074in}{0.548769in}}{\pgfqpoint{5.062926in}{3.102590in}}%
\pgfusepath{clip}%
\pgfsetbuttcap%
\pgfsetroundjoin%
\definecolor{currentfill}{rgb}{1.000000,0.498039,0.054902}%
\pgfsetfillcolor{currentfill}%
\pgfsetlinewidth{1.003750pt}%
\definecolor{currentstroke}{rgb}{1.000000,0.498039,0.054902}%
\pgfsetstrokecolor{currentstroke}%
\pgfsetdash{}{0pt}%
\pgfpathmoveto{\pgfqpoint{1.080257in}{1.644565in}}%
\pgfpathcurveto{\pgfqpoint{1.091307in}{1.644565in}}{\pgfqpoint{1.101906in}{1.648955in}}{\pgfqpoint{1.109720in}{1.656769in}}%
\pgfpathcurveto{\pgfqpoint{1.117533in}{1.664582in}}{\pgfqpoint{1.121924in}{1.675181in}}{\pgfqpoint{1.121924in}{1.686232in}}%
\pgfpathcurveto{\pgfqpoint{1.121924in}{1.697282in}}{\pgfqpoint{1.117533in}{1.707881in}}{\pgfqpoint{1.109720in}{1.715694in}}%
\pgfpathcurveto{\pgfqpoint{1.101906in}{1.723508in}}{\pgfqpoint{1.091307in}{1.727898in}}{\pgfqpoint{1.080257in}{1.727898in}}%
\pgfpathcurveto{\pgfqpoint{1.069207in}{1.727898in}}{\pgfqpoint{1.058608in}{1.723508in}}{\pgfqpoint{1.050794in}{1.715694in}}%
\pgfpathcurveto{\pgfqpoint{1.042981in}{1.707881in}}{\pgfqpoint{1.038590in}{1.697282in}}{\pgfqpoint{1.038590in}{1.686232in}}%
\pgfpathcurveto{\pgfqpoint{1.038590in}{1.675181in}}{\pgfqpoint{1.042981in}{1.664582in}}{\pgfqpoint{1.050794in}{1.656769in}}%
\pgfpathcurveto{\pgfqpoint{1.058608in}{1.648955in}}{\pgfqpoint{1.069207in}{1.644565in}}{\pgfqpoint{1.080257in}{1.644565in}}%
\pgfpathclose%
\pgfusepath{stroke,fill}%
\end{pgfscope}%
\begin{pgfscope}%
\pgfpathrectangle{\pgfqpoint{0.787074in}{0.548769in}}{\pgfqpoint{5.062926in}{3.102590in}}%
\pgfusepath{clip}%
\pgfsetbuttcap%
\pgfsetroundjoin%
\definecolor{currentfill}{rgb}{0.121569,0.466667,0.705882}%
\pgfsetfillcolor{currentfill}%
\pgfsetlinewidth{1.003750pt}%
\definecolor{currentstroke}{rgb}{0.121569,0.466667,0.705882}%
\pgfsetstrokecolor{currentstroke}%
\pgfsetdash{}{0pt}%
\pgfpathmoveto{\pgfqpoint{1.395508in}{0.664836in}}%
\pgfpathcurveto{\pgfqpoint{1.406558in}{0.664836in}}{\pgfqpoint{1.417157in}{0.669227in}}{\pgfqpoint{1.424970in}{0.677040in}}%
\pgfpathcurveto{\pgfqpoint{1.432784in}{0.684854in}}{\pgfqpoint{1.437174in}{0.695453in}}{\pgfqpoint{1.437174in}{0.706503in}}%
\pgfpathcurveto{\pgfqpoint{1.437174in}{0.717553in}}{\pgfqpoint{1.432784in}{0.728152in}}{\pgfqpoint{1.424970in}{0.735966in}}%
\pgfpathcurveto{\pgfqpoint{1.417157in}{0.743779in}}{\pgfqpoint{1.406558in}{0.748170in}}{\pgfqpoint{1.395508in}{0.748170in}}%
\pgfpathcurveto{\pgfqpoint{1.384458in}{0.748170in}}{\pgfqpoint{1.373859in}{0.743779in}}{\pgfqpoint{1.366045in}{0.735966in}}%
\pgfpathcurveto{\pgfqpoint{1.358231in}{0.728152in}}{\pgfqpoint{1.353841in}{0.717553in}}{\pgfqpoint{1.353841in}{0.706503in}}%
\pgfpathcurveto{\pgfqpoint{1.353841in}{0.695453in}}{\pgfqpoint{1.358231in}{0.684854in}}{\pgfqpoint{1.366045in}{0.677040in}}%
\pgfpathcurveto{\pgfqpoint{1.373859in}{0.669227in}}{\pgfqpoint{1.384458in}{0.664836in}}{\pgfqpoint{1.395508in}{0.664836in}}%
\pgfpathclose%
\pgfusepath{stroke,fill}%
\end{pgfscope}%
\begin{pgfscope}%
\pgfpathrectangle{\pgfqpoint{0.787074in}{0.548769in}}{\pgfqpoint{5.062926in}{3.102590in}}%
\pgfusepath{clip}%
\pgfsetbuttcap%
\pgfsetroundjoin%
\definecolor{currentfill}{rgb}{0.121569,0.466667,0.705882}%
\pgfsetfillcolor{currentfill}%
\pgfsetlinewidth{1.003750pt}%
\definecolor{currentstroke}{rgb}{0.121569,0.466667,0.705882}%
\pgfsetstrokecolor{currentstroke}%
\pgfsetdash{}{0pt}%
\pgfpathmoveto{\pgfqpoint{1.962959in}{1.428628in}}%
\pgfpathcurveto{\pgfqpoint{1.974009in}{1.428628in}}{\pgfqpoint{1.984608in}{1.433018in}}{\pgfqpoint{1.992422in}{1.440832in}}%
\pgfpathcurveto{\pgfqpoint{2.000235in}{1.448646in}}{\pgfqpoint{2.004626in}{1.459245in}}{\pgfqpoint{2.004626in}{1.470295in}}%
\pgfpathcurveto{\pgfqpoint{2.004626in}{1.481345in}}{\pgfqpoint{2.000235in}{1.491944in}}{\pgfqpoint{1.992422in}{1.499758in}}%
\pgfpathcurveto{\pgfqpoint{1.984608in}{1.507571in}}{\pgfqpoint{1.974009in}{1.511961in}}{\pgfqpoint{1.962959in}{1.511961in}}%
\pgfpathcurveto{\pgfqpoint{1.951909in}{1.511961in}}{\pgfqpoint{1.941310in}{1.507571in}}{\pgfqpoint{1.933496in}{1.499758in}}%
\pgfpathcurveto{\pgfqpoint{1.925683in}{1.491944in}}{\pgfqpoint{1.921292in}{1.481345in}}{\pgfqpoint{1.921292in}{1.470295in}}%
\pgfpathcurveto{\pgfqpoint{1.921292in}{1.459245in}}{\pgfqpoint{1.925683in}{1.448646in}}{\pgfqpoint{1.933496in}{1.440832in}}%
\pgfpathcurveto{\pgfqpoint{1.941310in}{1.433018in}}{\pgfqpoint{1.951909in}{1.428628in}}{\pgfqpoint{1.962959in}{1.428628in}}%
\pgfpathclose%
\pgfusepath{stroke,fill}%
\end{pgfscope}%
\begin{pgfscope}%
\pgfpathrectangle{\pgfqpoint{0.787074in}{0.548769in}}{\pgfqpoint{5.062926in}{3.102590in}}%
\pgfusepath{clip}%
\pgfsetbuttcap%
\pgfsetroundjoin%
\definecolor{currentfill}{rgb}{0.121569,0.466667,0.705882}%
\pgfsetfillcolor{currentfill}%
\pgfsetlinewidth{1.003750pt}%
\definecolor{currentstroke}{rgb}{0.121569,0.466667,0.705882}%
\pgfsetstrokecolor{currentstroke}%
\pgfsetdash{}{0pt}%
\pgfpathmoveto{\pgfqpoint{1.017207in}{0.648339in}}%
\pgfpathcurveto{\pgfqpoint{1.028257in}{0.648339in}}{\pgfqpoint{1.038856in}{0.652729in}}{\pgfqpoint{1.046670in}{0.660543in}}%
\pgfpathcurveto{\pgfqpoint{1.054483in}{0.668356in}}{\pgfqpoint{1.058874in}{0.678955in}}{\pgfqpoint{1.058874in}{0.690005in}}%
\pgfpathcurveto{\pgfqpoint{1.058874in}{0.701056in}}{\pgfqpoint{1.054483in}{0.711655in}}{\pgfqpoint{1.046670in}{0.719468in}}%
\pgfpathcurveto{\pgfqpoint{1.038856in}{0.727282in}}{\pgfqpoint{1.028257in}{0.731672in}}{\pgfqpoint{1.017207in}{0.731672in}}%
\pgfpathcurveto{\pgfqpoint{1.006157in}{0.731672in}}{\pgfqpoint{0.995558in}{0.727282in}}{\pgfqpoint{0.987744in}{0.719468in}}%
\pgfpathcurveto{\pgfqpoint{0.979930in}{0.711655in}}{\pgfqpoint{0.975540in}{0.701056in}}{\pgfqpoint{0.975540in}{0.690005in}}%
\pgfpathcurveto{\pgfqpoint{0.975540in}{0.678955in}}{\pgfqpoint{0.979930in}{0.668356in}}{\pgfqpoint{0.987744in}{0.660543in}}%
\pgfpathcurveto{\pgfqpoint{0.995558in}{0.652729in}}{\pgfqpoint{1.006157in}{0.648339in}}{\pgfqpoint{1.017207in}{0.648339in}}%
\pgfpathclose%
\pgfusepath{stroke,fill}%
\end{pgfscope}%
\begin{pgfscope}%
\pgfpathrectangle{\pgfqpoint{0.787074in}{0.548769in}}{\pgfqpoint{5.062926in}{3.102590in}}%
\pgfusepath{clip}%
\pgfsetbuttcap%
\pgfsetroundjoin%
\definecolor{currentfill}{rgb}{1.000000,0.498039,0.054902}%
\pgfsetfillcolor{currentfill}%
\pgfsetlinewidth{1.003750pt}%
\definecolor{currentstroke}{rgb}{1.000000,0.498039,0.054902}%
\pgfsetstrokecolor{currentstroke}%
\pgfsetdash{}{0pt}%
\pgfpathmoveto{\pgfqpoint{1.962959in}{1.507887in}}%
\pgfpathcurveto{\pgfqpoint{1.974009in}{1.507887in}}{\pgfqpoint{1.984608in}{1.512277in}}{\pgfqpoint{1.992422in}{1.520091in}}%
\pgfpathcurveto{\pgfqpoint{2.000235in}{1.527904in}}{\pgfqpoint{2.004626in}{1.538503in}}{\pgfqpoint{2.004626in}{1.549554in}}%
\pgfpathcurveto{\pgfqpoint{2.004626in}{1.560604in}}{\pgfqpoint{2.000235in}{1.571203in}}{\pgfqpoint{1.992422in}{1.579016in}}%
\pgfpathcurveto{\pgfqpoint{1.984608in}{1.586830in}}{\pgfqpoint{1.974009in}{1.591220in}}{\pgfqpoint{1.962959in}{1.591220in}}%
\pgfpathcurveto{\pgfqpoint{1.951909in}{1.591220in}}{\pgfqpoint{1.941310in}{1.586830in}}{\pgfqpoint{1.933496in}{1.579016in}}%
\pgfpathcurveto{\pgfqpoint{1.925683in}{1.571203in}}{\pgfqpoint{1.921292in}{1.560604in}}{\pgfqpoint{1.921292in}{1.549554in}}%
\pgfpathcurveto{\pgfqpoint{1.921292in}{1.538503in}}{\pgfqpoint{1.925683in}{1.527904in}}{\pgfqpoint{1.933496in}{1.520091in}}%
\pgfpathcurveto{\pgfqpoint{1.941310in}{1.512277in}}{\pgfqpoint{1.951909in}{1.507887in}}{\pgfqpoint{1.962959in}{1.507887in}}%
\pgfpathclose%
\pgfusepath{stroke,fill}%
\end{pgfscope}%
\begin{pgfscope}%
\pgfpathrectangle{\pgfqpoint{0.787074in}{0.548769in}}{\pgfqpoint{5.062926in}{3.102590in}}%
\pgfusepath{clip}%
\pgfsetbuttcap%
\pgfsetroundjoin%
\definecolor{currentfill}{rgb}{1.000000,0.498039,0.054902}%
\pgfsetfillcolor{currentfill}%
\pgfsetlinewidth{1.003750pt}%
\definecolor{currentstroke}{rgb}{1.000000,0.498039,0.054902}%
\pgfsetstrokecolor{currentstroke}%
\pgfsetdash{}{0pt}%
\pgfpathmoveto{\pgfqpoint{1.836859in}{2.621734in}}%
\pgfpathcurveto{\pgfqpoint{1.847909in}{2.621734in}}{\pgfqpoint{1.858508in}{2.626124in}}{\pgfqpoint{1.866321in}{2.633938in}}%
\pgfpathcurveto{\pgfqpoint{1.874135in}{2.641751in}}{\pgfqpoint{1.878525in}{2.652350in}}{\pgfqpoint{1.878525in}{2.663401in}}%
\pgfpathcurveto{\pgfqpoint{1.878525in}{2.674451in}}{\pgfqpoint{1.874135in}{2.685050in}}{\pgfqpoint{1.866321in}{2.692863in}}%
\pgfpathcurveto{\pgfqpoint{1.858508in}{2.700677in}}{\pgfqpoint{1.847909in}{2.705067in}}{\pgfqpoint{1.836859in}{2.705067in}}%
\pgfpathcurveto{\pgfqpoint{1.825809in}{2.705067in}}{\pgfqpoint{1.815209in}{2.700677in}}{\pgfqpoint{1.807396in}{2.692863in}}%
\pgfpathcurveto{\pgfqpoint{1.799582in}{2.685050in}}{\pgfqpoint{1.795192in}{2.674451in}}{\pgfqpoint{1.795192in}{2.663401in}}%
\pgfpathcurveto{\pgfqpoint{1.795192in}{2.652350in}}{\pgfqpoint{1.799582in}{2.641751in}}{\pgfqpoint{1.807396in}{2.633938in}}%
\pgfpathcurveto{\pgfqpoint{1.815209in}{2.626124in}}{\pgfqpoint{1.825809in}{2.621734in}}{\pgfqpoint{1.836859in}{2.621734in}}%
\pgfpathclose%
\pgfusepath{stroke,fill}%
\end{pgfscope}%
\begin{pgfscope}%
\pgfpathrectangle{\pgfqpoint{0.787074in}{0.548769in}}{\pgfqpoint{5.062926in}{3.102590in}}%
\pgfusepath{clip}%
\pgfsetbuttcap%
\pgfsetroundjoin%
\definecolor{currentfill}{rgb}{1.000000,0.498039,0.054902}%
\pgfsetfillcolor{currentfill}%
\pgfsetlinewidth{1.003750pt}%
\definecolor{currentstroke}{rgb}{1.000000,0.498039,0.054902}%
\pgfsetstrokecolor{currentstroke}%
\pgfsetdash{}{0pt}%
\pgfpathmoveto{\pgfqpoint{2.089059in}{1.832304in}}%
\pgfpathcurveto{\pgfqpoint{2.100109in}{1.832304in}}{\pgfqpoint{2.110708in}{1.836694in}}{\pgfqpoint{2.118522in}{1.844508in}}%
\pgfpathcurveto{\pgfqpoint{2.126336in}{1.852322in}}{\pgfqpoint{2.130726in}{1.862921in}}{\pgfqpoint{2.130726in}{1.873971in}}%
\pgfpathcurveto{\pgfqpoint{2.130726in}{1.885021in}}{\pgfqpoint{2.126336in}{1.895620in}}{\pgfqpoint{2.118522in}{1.903433in}}%
\pgfpathcurveto{\pgfqpoint{2.110708in}{1.911247in}}{\pgfqpoint{2.100109in}{1.915637in}}{\pgfqpoint{2.089059in}{1.915637in}}%
\pgfpathcurveto{\pgfqpoint{2.078009in}{1.915637in}}{\pgfqpoint{2.067410in}{1.911247in}}{\pgfqpoint{2.059596in}{1.903433in}}%
\pgfpathcurveto{\pgfqpoint{2.051783in}{1.895620in}}{\pgfqpoint{2.047393in}{1.885021in}}{\pgfqpoint{2.047393in}{1.873971in}}%
\pgfpathcurveto{\pgfqpoint{2.047393in}{1.862921in}}{\pgfqpoint{2.051783in}{1.852322in}}{\pgfqpoint{2.059596in}{1.844508in}}%
\pgfpathcurveto{\pgfqpoint{2.067410in}{1.836694in}}{\pgfqpoint{2.078009in}{1.832304in}}{\pgfqpoint{2.089059in}{1.832304in}}%
\pgfpathclose%
\pgfusepath{stroke,fill}%
\end{pgfscope}%
\begin{pgfscope}%
\pgfpathrectangle{\pgfqpoint{0.787074in}{0.548769in}}{\pgfqpoint{5.062926in}{3.102590in}}%
\pgfusepath{clip}%
\pgfsetbuttcap%
\pgfsetroundjoin%
\definecolor{currentfill}{rgb}{1.000000,0.498039,0.054902}%
\pgfsetfillcolor{currentfill}%
\pgfsetlinewidth{1.003750pt}%
\definecolor{currentstroke}{rgb}{1.000000,0.498039,0.054902}%
\pgfsetstrokecolor{currentstroke}%
\pgfsetdash{}{0pt}%
\pgfpathmoveto{\pgfqpoint{1.206357in}{3.055568in}}%
\pgfpathcurveto{\pgfqpoint{1.217407in}{3.055568in}}{\pgfqpoint{1.228006in}{3.059958in}}{\pgfqpoint{1.235820in}{3.067772in}}%
\pgfpathcurveto{\pgfqpoint{1.243634in}{3.075585in}}{\pgfqpoint{1.248024in}{3.086184in}}{\pgfqpoint{1.248024in}{3.097235in}}%
\pgfpathcurveto{\pgfqpoint{1.248024in}{3.108285in}}{\pgfqpoint{1.243634in}{3.118884in}}{\pgfqpoint{1.235820in}{3.126697in}}%
\pgfpathcurveto{\pgfqpoint{1.228006in}{3.134511in}}{\pgfqpoint{1.217407in}{3.138901in}}{\pgfqpoint{1.206357in}{3.138901in}}%
\pgfpathcurveto{\pgfqpoint{1.195307in}{3.138901in}}{\pgfqpoint{1.184708in}{3.134511in}}{\pgfqpoint{1.176894in}{3.126697in}}%
\pgfpathcurveto{\pgfqpoint{1.169081in}{3.118884in}}{\pgfqpoint{1.164691in}{3.108285in}}{\pgfqpoint{1.164691in}{3.097235in}}%
\pgfpathcurveto{\pgfqpoint{1.164691in}{3.086184in}}{\pgfqpoint{1.169081in}{3.075585in}}{\pgfqpoint{1.176894in}{3.067772in}}%
\pgfpathcurveto{\pgfqpoint{1.184708in}{3.059958in}}{\pgfqpoint{1.195307in}{3.055568in}}{\pgfqpoint{1.206357in}{3.055568in}}%
\pgfpathclose%
\pgfusepath{stroke,fill}%
\end{pgfscope}%
\begin{pgfscope}%
\pgfpathrectangle{\pgfqpoint{0.787074in}{0.548769in}}{\pgfqpoint{5.062926in}{3.102590in}}%
\pgfusepath{clip}%
\pgfsetbuttcap%
\pgfsetroundjoin%
\definecolor{currentfill}{rgb}{1.000000,0.498039,0.054902}%
\pgfsetfillcolor{currentfill}%
\pgfsetlinewidth{1.003750pt}%
\definecolor{currentstroke}{rgb}{1.000000,0.498039,0.054902}%
\pgfsetstrokecolor{currentstroke}%
\pgfsetdash{}{0pt}%
\pgfpathmoveto{\pgfqpoint{1.206357in}{3.291673in}}%
\pgfpathcurveto{\pgfqpoint{1.217407in}{3.291673in}}{\pgfqpoint{1.228006in}{3.296063in}}{\pgfqpoint{1.235820in}{3.303876in}}%
\pgfpathcurveto{\pgfqpoint{1.243634in}{3.311690in}}{\pgfqpoint{1.248024in}{3.322289in}}{\pgfqpoint{1.248024in}{3.333339in}}%
\pgfpathcurveto{\pgfqpoint{1.248024in}{3.344389in}}{\pgfqpoint{1.243634in}{3.354988in}}{\pgfqpoint{1.235820in}{3.362802in}}%
\pgfpathcurveto{\pgfqpoint{1.228006in}{3.370616in}}{\pgfqpoint{1.217407in}{3.375006in}}{\pgfqpoint{1.206357in}{3.375006in}}%
\pgfpathcurveto{\pgfqpoint{1.195307in}{3.375006in}}{\pgfqpoint{1.184708in}{3.370616in}}{\pgfqpoint{1.176894in}{3.362802in}}%
\pgfpathcurveto{\pgfqpoint{1.169081in}{3.354988in}}{\pgfqpoint{1.164691in}{3.344389in}}{\pgfqpoint{1.164691in}{3.333339in}}%
\pgfpathcurveto{\pgfqpoint{1.164691in}{3.322289in}}{\pgfqpoint{1.169081in}{3.311690in}}{\pgfqpoint{1.176894in}{3.303876in}}%
\pgfpathcurveto{\pgfqpoint{1.184708in}{3.296063in}}{\pgfqpoint{1.195307in}{3.291673in}}{\pgfqpoint{1.206357in}{3.291673in}}%
\pgfpathclose%
\pgfusepath{stroke,fill}%
\end{pgfscope}%
\begin{pgfscope}%
\pgfpathrectangle{\pgfqpoint{0.787074in}{0.548769in}}{\pgfqpoint{5.062926in}{3.102590in}}%
\pgfusepath{clip}%
\pgfsetbuttcap%
\pgfsetroundjoin%
\definecolor{currentfill}{rgb}{1.000000,0.498039,0.054902}%
\pgfsetfillcolor{currentfill}%
\pgfsetlinewidth{1.003750pt}%
\definecolor{currentstroke}{rgb}{1.000000,0.498039,0.054902}%
\pgfsetstrokecolor{currentstroke}%
\pgfsetdash{}{0pt}%
\pgfpathmoveto{\pgfqpoint{1.647708in}{1.890154in}}%
\pgfpathcurveto{\pgfqpoint{1.658758in}{1.890154in}}{\pgfqpoint{1.669357in}{1.894544in}}{\pgfqpoint{1.677171in}{1.902358in}}%
\pgfpathcurveto{\pgfqpoint{1.684985in}{1.910171in}}{\pgfqpoint{1.689375in}{1.920770in}}{\pgfqpoint{1.689375in}{1.931820in}}%
\pgfpathcurveto{\pgfqpoint{1.689375in}{1.942870in}}{\pgfqpoint{1.684985in}{1.953469in}}{\pgfqpoint{1.677171in}{1.961283in}}%
\pgfpathcurveto{\pgfqpoint{1.669357in}{1.969097in}}{\pgfqpoint{1.658758in}{1.973487in}}{\pgfqpoint{1.647708in}{1.973487in}}%
\pgfpathcurveto{\pgfqpoint{1.636658in}{1.973487in}}{\pgfqpoint{1.626059in}{1.969097in}}{\pgfqpoint{1.618245in}{1.961283in}}%
\pgfpathcurveto{\pgfqpoint{1.610432in}{1.953469in}}{\pgfqpoint{1.606042in}{1.942870in}}{\pgfqpoint{1.606042in}{1.931820in}}%
\pgfpathcurveto{\pgfqpoint{1.606042in}{1.920770in}}{\pgfqpoint{1.610432in}{1.910171in}}{\pgfqpoint{1.618245in}{1.902358in}}%
\pgfpathcurveto{\pgfqpoint{1.626059in}{1.894544in}}{\pgfqpoint{1.636658in}{1.890154in}}{\pgfqpoint{1.647708in}{1.890154in}}%
\pgfpathclose%
\pgfusepath{stroke,fill}%
\end{pgfscope}%
\begin{pgfscope}%
\pgfpathrectangle{\pgfqpoint{0.787074in}{0.548769in}}{\pgfqpoint{5.062926in}{3.102590in}}%
\pgfusepath{clip}%
\pgfsetbuttcap%
\pgfsetroundjoin%
\definecolor{currentfill}{rgb}{0.121569,0.466667,0.705882}%
\pgfsetfillcolor{currentfill}%
\pgfsetlinewidth{1.003750pt}%
\definecolor{currentstroke}{rgb}{0.121569,0.466667,0.705882}%
\pgfsetstrokecolor{currentstroke}%
\pgfsetdash{}{0pt}%
\pgfpathmoveto{\pgfqpoint{1.080257in}{1.784271in}}%
\pgfpathcurveto{\pgfqpoint{1.091307in}{1.784271in}}{\pgfqpoint{1.101906in}{1.788661in}}{\pgfqpoint{1.109720in}{1.796475in}}%
\pgfpathcurveto{\pgfqpoint{1.117533in}{1.804289in}}{\pgfqpoint{1.121924in}{1.814888in}}{\pgfqpoint{1.121924in}{1.825938in}}%
\pgfpathcurveto{\pgfqpoint{1.121924in}{1.836988in}}{\pgfqpoint{1.117533in}{1.847587in}}{\pgfqpoint{1.109720in}{1.855401in}}%
\pgfpathcurveto{\pgfqpoint{1.101906in}{1.863214in}}{\pgfqpoint{1.091307in}{1.867604in}}{\pgfqpoint{1.080257in}{1.867604in}}%
\pgfpathcurveto{\pgfqpoint{1.069207in}{1.867604in}}{\pgfqpoint{1.058608in}{1.863214in}}{\pgfqpoint{1.050794in}{1.855401in}}%
\pgfpathcurveto{\pgfqpoint{1.042981in}{1.847587in}}{\pgfqpoint{1.038590in}{1.836988in}}{\pgfqpoint{1.038590in}{1.825938in}}%
\pgfpathcurveto{\pgfqpoint{1.038590in}{1.814888in}}{\pgfqpoint{1.042981in}{1.804289in}}{\pgfqpoint{1.050794in}{1.796475in}}%
\pgfpathcurveto{\pgfqpoint{1.058608in}{1.788661in}}{\pgfqpoint{1.069207in}{1.784271in}}{\pgfqpoint{1.080257in}{1.784271in}}%
\pgfpathclose%
\pgfusepath{stroke,fill}%
\end{pgfscope}%
\begin{pgfscope}%
\pgfpathrectangle{\pgfqpoint{0.787074in}{0.548769in}}{\pgfqpoint{5.062926in}{3.102590in}}%
\pgfusepath{clip}%
\pgfsetbuttcap%
\pgfsetroundjoin%
\definecolor{currentfill}{rgb}{1.000000,0.498039,0.054902}%
\pgfsetfillcolor{currentfill}%
\pgfsetlinewidth{1.003750pt}%
\definecolor{currentstroke}{rgb}{1.000000,0.498039,0.054902}%
\pgfsetstrokecolor{currentstroke}%
\pgfsetdash{}{0pt}%
\pgfpathmoveto{\pgfqpoint{1.332458in}{1.980354in}}%
\pgfpathcurveto{\pgfqpoint{1.343508in}{1.980354in}}{\pgfqpoint{1.354107in}{1.984745in}}{\pgfqpoint{1.361920in}{1.992558in}}%
\pgfpathcurveto{\pgfqpoint{1.369734in}{2.000372in}}{\pgfqpoint{1.374124in}{2.010971in}}{\pgfqpoint{1.374124in}{2.022021in}}%
\pgfpathcurveto{\pgfqpoint{1.374124in}{2.033071in}}{\pgfqpoint{1.369734in}{2.043670in}}{\pgfqpoint{1.361920in}{2.051484in}}%
\pgfpathcurveto{\pgfqpoint{1.354107in}{2.059297in}}{\pgfqpoint{1.343508in}{2.063688in}}{\pgfqpoint{1.332458in}{2.063688in}}%
\pgfpathcurveto{\pgfqpoint{1.321407in}{2.063688in}}{\pgfqpoint{1.310808in}{2.059297in}}{\pgfqpoint{1.302995in}{2.051484in}}%
\pgfpathcurveto{\pgfqpoint{1.295181in}{2.043670in}}{\pgfqpoint{1.290791in}{2.033071in}}{\pgfqpoint{1.290791in}{2.022021in}}%
\pgfpathcurveto{\pgfqpoint{1.290791in}{2.010971in}}{\pgfqpoint{1.295181in}{2.000372in}}{\pgfqpoint{1.302995in}{1.992558in}}%
\pgfpathcurveto{\pgfqpoint{1.310808in}{1.984745in}}{\pgfqpoint{1.321407in}{1.980354in}}{\pgfqpoint{1.332458in}{1.980354in}}%
\pgfpathclose%
\pgfusepath{stroke,fill}%
\end{pgfscope}%
\begin{pgfscope}%
\pgfpathrectangle{\pgfqpoint{0.787074in}{0.548769in}}{\pgfqpoint{5.062926in}{3.102590in}}%
\pgfusepath{clip}%
\pgfsetbuttcap%
\pgfsetroundjoin%
\definecolor{currentfill}{rgb}{0.121569,0.466667,0.705882}%
\pgfsetfillcolor{currentfill}%
\pgfsetlinewidth{1.003750pt}%
\definecolor{currentstroke}{rgb}{0.121569,0.466667,0.705882}%
\pgfsetstrokecolor{currentstroke}%
\pgfsetdash{}{0pt}%
\pgfpathmoveto{\pgfqpoint{2.530410in}{1.870879in}}%
\pgfpathcurveto{\pgfqpoint{2.541460in}{1.870879in}}{\pgfqpoint{2.552059in}{1.875269in}}{\pgfqpoint{2.559873in}{1.883083in}}%
\pgfpathcurveto{\pgfqpoint{2.567687in}{1.890896in}}{\pgfqpoint{2.572077in}{1.901496in}}{\pgfqpoint{2.572077in}{1.912546in}}%
\pgfpathcurveto{\pgfqpoint{2.572077in}{1.923596in}}{\pgfqpoint{2.567687in}{1.934195in}}{\pgfqpoint{2.559873in}{1.942008in}}%
\pgfpathcurveto{\pgfqpoint{2.552059in}{1.949822in}}{\pgfqpoint{2.541460in}{1.954212in}}{\pgfqpoint{2.530410in}{1.954212in}}%
\pgfpathcurveto{\pgfqpoint{2.519360in}{1.954212in}}{\pgfqpoint{2.508761in}{1.949822in}}{\pgfqpoint{2.500947in}{1.942008in}}%
\pgfpathcurveto{\pgfqpoint{2.493134in}{1.934195in}}{\pgfqpoint{2.488744in}{1.923596in}}{\pgfqpoint{2.488744in}{1.912546in}}%
\pgfpathcurveto{\pgfqpoint{2.488744in}{1.901496in}}{\pgfqpoint{2.493134in}{1.890896in}}{\pgfqpoint{2.500947in}{1.883083in}}%
\pgfpathcurveto{\pgfqpoint{2.508761in}{1.875269in}}{\pgfqpoint{2.519360in}{1.870879in}}{\pgfqpoint{2.530410in}{1.870879in}}%
\pgfpathclose%
\pgfusepath{stroke,fill}%
\end{pgfscope}%
\begin{pgfscope}%
\pgfpathrectangle{\pgfqpoint{0.787074in}{0.548769in}}{\pgfqpoint{5.062926in}{3.102590in}}%
\pgfusepath{clip}%
\pgfsetbuttcap%
\pgfsetroundjoin%
\definecolor{currentfill}{rgb}{1.000000,0.498039,0.054902}%
\pgfsetfillcolor{currentfill}%
\pgfsetlinewidth{1.003750pt}%
\definecolor{currentstroke}{rgb}{1.000000,0.498039,0.054902}%
\pgfsetstrokecolor{currentstroke}%
\pgfsetdash{}{0pt}%
\pgfpathmoveto{\pgfqpoint{2.404310in}{1.599536in}}%
\pgfpathcurveto{\pgfqpoint{2.415360in}{1.599536in}}{\pgfqpoint{2.425959in}{1.603926in}}{\pgfqpoint{2.433773in}{1.611740in}}%
\pgfpathcurveto{\pgfqpoint{2.441586in}{1.619554in}}{\pgfqpoint{2.445977in}{1.630153in}}{\pgfqpoint{2.445977in}{1.641203in}}%
\pgfpathcurveto{\pgfqpoint{2.445977in}{1.652253in}}{\pgfqpoint{2.441586in}{1.662852in}}{\pgfqpoint{2.433773in}{1.670666in}}%
\pgfpathcurveto{\pgfqpoint{2.425959in}{1.678479in}}{\pgfqpoint{2.415360in}{1.682870in}}{\pgfqpoint{2.404310in}{1.682870in}}%
\pgfpathcurveto{\pgfqpoint{2.393260in}{1.682870in}}{\pgfqpoint{2.382661in}{1.678479in}}{\pgfqpoint{2.374847in}{1.670666in}}%
\pgfpathcurveto{\pgfqpoint{2.367033in}{1.662852in}}{\pgfqpoint{2.362643in}{1.652253in}}{\pgfqpoint{2.362643in}{1.641203in}}%
\pgfpathcurveto{\pgfqpoint{2.362643in}{1.630153in}}{\pgfqpoint{2.367033in}{1.619554in}}{\pgfqpoint{2.374847in}{1.611740in}}%
\pgfpathcurveto{\pgfqpoint{2.382661in}{1.603926in}}{\pgfqpoint{2.393260in}{1.599536in}}{\pgfqpoint{2.404310in}{1.599536in}}%
\pgfpathclose%
\pgfusepath{stroke,fill}%
\end{pgfscope}%
\begin{pgfscope}%
\pgfpathrectangle{\pgfqpoint{0.787074in}{0.548769in}}{\pgfqpoint{5.062926in}{3.102590in}}%
\pgfusepath{clip}%
\pgfsetbuttcap%
\pgfsetroundjoin%
\definecolor{currentfill}{rgb}{0.121569,0.466667,0.705882}%
\pgfsetfillcolor{currentfill}%
\pgfsetlinewidth{1.003750pt}%
\definecolor{currentstroke}{rgb}{0.121569,0.466667,0.705882}%
\pgfsetstrokecolor{currentstroke}%
\pgfsetdash{}{0pt}%
\pgfpathmoveto{\pgfqpoint{2.656510in}{1.739034in}}%
\pgfpathcurveto{\pgfqpoint{2.667561in}{1.739034in}}{\pgfqpoint{2.678160in}{1.743425in}}{\pgfqpoint{2.685973in}{1.751238in}}%
\pgfpathcurveto{\pgfqpoint{2.693787in}{1.759052in}}{\pgfqpoint{2.698177in}{1.769651in}}{\pgfqpoint{2.698177in}{1.780701in}}%
\pgfpathcurveto{\pgfqpoint{2.698177in}{1.791751in}}{\pgfqpoint{2.693787in}{1.802350in}}{\pgfqpoint{2.685973in}{1.810164in}}%
\pgfpathcurveto{\pgfqpoint{2.678160in}{1.817977in}}{\pgfqpoint{2.667561in}{1.822368in}}{\pgfqpoint{2.656510in}{1.822368in}}%
\pgfpathcurveto{\pgfqpoint{2.645460in}{1.822368in}}{\pgfqpoint{2.634861in}{1.817977in}}{\pgfqpoint{2.627048in}{1.810164in}}%
\pgfpathcurveto{\pgfqpoint{2.619234in}{1.802350in}}{\pgfqpoint{2.614844in}{1.791751in}}{\pgfqpoint{2.614844in}{1.780701in}}%
\pgfpathcurveto{\pgfqpoint{2.614844in}{1.769651in}}{\pgfqpoint{2.619234in}{1.759052in}}{\pgfqpoint{2.627048in}{1.751238in}}%
\pgfpathcurveto{\pgfqpoint{2.634861in}{1.743425in}}{\pgfqpoint{2.645460in}{1.739034in}}{\pgfqpoint{2.656510in}{1.739034in}}%
\pgfpathclose%
\pgfusepath{stroke,fill}%
\end{pgfscope}%
\begin{pgfscope}%
\pgfpathrectangle{\pgfqpoint{0.787074in}{0.548769in}}{\pgfqpoint{5.062926in}{3.102590in}}%
\pgfusepath{clip}%
\pgfsetbuttcap%
\pgfsetroundjoin%
\definecolor{currentfill}{rgb}{1.000000,0.498039,0.054902}%
\pgfsetfillcolor{currentfill}%
\pgfsetlinewidth{1.003750pt}%
\definecolor{currentstroke}{rgb}{1.000000,0.498039,0.054902}%
\pgfsetstrokecolor{currentstroke}%
\pgfsetdash{}{0pt}%
\pgfpathmoveto{\pgfqpoint{1.080257in}{2.553699in}}%
\pgfpathcurveto{\pgfqpoint{1.091307in}{2.553699in}}{\pgfqpoint{1.101906in}{2.558090in}}{\pgfqpoint{1.109720in}{2.565903in}}%
\pgfpathcurveto{\pgfqpoint{1.117533in}{2.573717in}}{\pgfqpoint{1.121924in}{2.584316in}}{\pgfqpoint{1.121924in}{2.595366in}}%
\pgfpathcurveto{\pgfqpoint{1.121924in}{2.606416in}}{\pgfqpoint{1.117533in}{2.617015in}}{\pgfqpoint{1.109720in}{2.624829in}}%
\pgfpathcurveto{\pgfqpoint{1.101906in}{2.632642in}}{\pgfqpoint{1.091307in}{2.637033in}}{\pgfqpoint{1.080257in}{2.637033in}}%
\pgfpathcurveto{\pgfqpoint{1.069207in}{2.637033in}}{\pgfqpoint{1.058608in}{2.632642in}}{\pgfqpoint{1.050794in}{2.624829in}}%
\pgfpathcurveto{\pgfqpoint{1.042981in}{2.617015in}}{\pgfqpoint{1.038590in}{2.606416in}}{\pgfqpoint{1.038590in}{2.595366in}}%
\pgfpathcurveto{\pgfqpoint{1.038590in}{2.584316in}}{\pgfqpoint{1.042981in}{2.573717in}}{\pgfqpoint{1.050794in}{2.565903in}}%
\pgfpathcurveto{\pgfqpoint{1.058608in}{2.558090in}}{\pgfqpoint{1.069207in}{2.553699in}}{\pgfqpoint{1.080257in}{2.553699in}}%
\pgfpathclose%
\pgfusepath{stroke,fill}%
\end{pgfscope}%
\begin{pgfscope}%
\pgfpathrectangle{\pgfqpoint{0.787074in}{0.548769in}}{\pgfqpoint{5.062926in}{3.102590in}}%
\pgfusepath{clip}%
\pgfsetbuttcap%
\pgfsetroundjoin%
\definecolor{currentfill}{rgb}{0.121569,0.466667,0.705882}%
\pgfsetfillcolor{currentfill}%
\pgfsetlinewidth{1.003750pt}%
\definecolor{currentstroke}{rgb}{0.121569,0.466667,0.705882}%
\pgfsetstrokecolor{currentstroke}%
\pgfsetdash{}{0pt}%
\pgfpathmoveto{\pgfqpoint{1.143307in}{2.326040in}}%
\pgfpathcurveto{\pgfqpoint{1.154357in}{2.326040in}}{\pgfqpoint{1.164956in}{2.330430in}}{\pgfqpoint{1.172770in}{2.338243in}}%
\pgfpathcurveto{\pgfqpoint{1.180584in}{2.346057in}}{\pgfqpoint{1.184974in}{2.356656in}}{\pgfqpoint{1.184974in}{2.367706in}}%
\pgfpathcurveto{\pgfqpoint{1.184974in}{2.378756in}}{\pgfqpoint{1.180584in}{2.389355in}}{\pgfqpoint{1.172770in}{2.397169in}}%
\pgfpathcurveto{\pgfqpoint{1.164956in}{2.404983in}}{\pgfqpoint{1.154357in}{2.409373in}}{\pgfqpoint{1.143307in}{2.409373in}}%
\pgfpathcurveto{\pgfqpoint{1.132257in}{2.409373in}}{\pgfqpoint{1.121658in}{2.404983in}}{\pgfqpoint{1.113844in}{2.397169in}}%
\pgfpathcurveto{\pgfqpoint{1.106031in}{2.389355in}}{\pgfqpoint{1.101640in}{2.378756in}}{\pgfqpoint{1.101640in}{2.367706in}}%
\pgfpathcurveto{\pgfqpoint{1.101640in}{2.356656in}}{\pgfqpoint{1.106031in}{2.346057in}}{\pgfqpoint{1.113844in}{2.338243in}}%
\pgfpathcurveto{\pgfqpoint{1.121658in}{2.330430in}}{\pgfqpoint{1.132257in}{2.326040in}}{\pgfqpoint{1.143307in}{2.326040in}}%
\pgfpathclose%
\pgfusepath{stroke,fill}%
\end{pgfscope}%
\begin{pgfscope}%
\pgfpathrectangle{\pgfqpoint{0.787074in}{0.548769in}}{\pgfqpoint{5.062926in}{3.102590in}}%
\pgfusepath{clip}%
\pgfsetbuttcap%
\pgfsetroundjoin%
\definecolor{currentfill}{rgb}{1.000000,0.498039,0.054902}%
\pgfsetfillcolor{currentfill}%
\pgfsetlinewidth{1.003750pt}%
\definecolor{currentstroke}{rgb}{1.000000,0.498039,0.054902}%
\pgfsetstrokecolor{currentstroke}%
\pgfsetdash{}{0pt}%
\pgfpathmoveto{\pgfqpoint{2.467360in}{2.769003in}}%
\pgfpathcurveto{\pgfqpoint{2.478410in}{2.769003in}}{\pgfqpoint{2.489009in}{2.773394in}}{\pgfqpoint{2.496823in}{2.781207in}}%
\pgfpathcurveto{\pgfqpoint{2.504636in}{2.789021in}}{\pgfqpoint{2.509027in}{2.799620in}}{\pgfqpoint{2.509027in}{2.810670in}}%
\pgfpathcurveto{\pgfqpoint{2.509027in}{2.821720in}}{\pgfqpoint{2.504636in}{2.832319in}}{\pgfqpoint{2.496823in}{2.840133in}}%
\pgfpathcurveto{\pgfqpoint{2.489009in}{2.847946in}}{\pgfqpoint{2.478410in}{2.852337in}}{\pgfqpoint{2.467360in}{2.852337in}}%
\pgfpathcurveto{\pgfqpoint{2.456310in}{2.852337in}}{\pgfqpoint{2.445711in}{2.847946in}}{\pgfqpoint{2.437897in}{2.840133in}}%
\pgfpathcurveto{\pgfqpoint{2.430084in}{2.832319in}}{\pgfqpoint{2.425693in}{2.821720in}}{\pgfqpoint{2.425693in}{2.810670in}}%
\pgfpathcurveto{\pgfqpoint{2.425693in}{2.799620in}}{\pgfqpoint{2.430084in}{2.789021in}}{\pgfqpoint{2.437897in}{2.781207in}}%
\pgfpathcurveto{\pgfqpoint{2.445711in}{2.773394in}}{\pgfqpoint{2.456310in}{2.769003in}}{\pgfqpoint{2.467360in}{2.769003in}}%
\pgfpathclose%
\pgfusepath{stroke,fill}%
\end{pgfscope}%
\begin{pgfscope}%
\pgfpathrectangle{\pgfqpoint{0.787074in}{0.548769in}}{\pgfqpoint{5.062926in}{3.102590in}}%
\pgfusepath{clip}%
\pgfsetbuttcap%
\pgfsetroundjoin%
\definecolor{currentfill}{rgb}{0.121569,0.466667,0.705882}%
\pgfsetfillcolor{currentfill}%
\pgfsetlinewidth{1.003750pt}%
\definecolor{currentstroke}{rgb}{0.121569,0.466667,0.705882}%
\pgfsetstrokecolor{currentstroke}%
\pgfsetdash{}{0pt}%
\pgfpathmoveto{\pgfqpoint{1.395508in}{0.659320in}}%
\pgfpathcurveto{\pgfqpoint{1.406558in}{0.659320in}}{\pgfqpoint{1.417157in}{0.663711in}}{\pgfqpoint{1.424970in}{0.671524in}}%
\pgfpathcurveto{\pgfqpoint{1.432784in}{0.679338in}}{\pgfqpoint{1.437174in}{0.689937in}}{\pgfqpoint{1.437174in}{0.700987in}}%
\pgfpathcurveto{\pgfqpoint{1.437174in}{0.712037in}}{\pgfqpoint{1.432784in}{0.722636in}}{\pgfqpoint{1.424970in}{0.730450in}}%
\pgfpathcurveto{\pgfqpoint{1.417157in}{0.738263in}}{\pgfqpoint{1.406558in}{0.742654in}}{\pgfqpoint{1.395508in}{0.742654in}}%
\pgfpathcurveto{\pgfqpoint{1.384458in}{0.742654in}}{\pgfqpoint{1.373859in}{0.738263in}}{\pgfqpoint{1.366045in}{0.730450in}}%
\pgfpathcurveto{\pgfqpoint{1.358231in}{0.722636in}}{\pgfqpoint{1.353841in}{0.712037in}}{\pgfqpoint{1.353841in}{0.700987in}}%
\pgfpathcurveto{\pgfqpoint{1.353841in}{0.689937in}}{\pgfqpoint{1.358231in}{0.679338in}}{\pgfqpoint{1.366045in}{0.671524in}}%
\pgfpathcurveto{\pgfqpoint{1.373859in}{0.663711in}}{\pgfqpoint{1.384458in}{0.659320in}}{\pgfqpoint{1.395508in}{0.659320in}}%
\pgfpathclose%
\pgfusepath{stroke,fill}%
\end{pgfscope}%
\begin{pgfscope}%
\pgfpathrectangle{\pgfqpoint{0.787074in}{0.548769in}}{\pgfqpoint{5.062926in}{3.102590in}}%
\pgfusepath{clip}%
\pgfsetbuttcap%
\pgfsetroundjoin%
\definecolor{currentfill}{rgb}{1.000000,0.498039,0.054902}%
\pgfsetfillcolor{currentfill}%
\pgfsetlinewidth{1.003750pt}%
\definecolor{currentstroke}{rgb}{1.000000,0.498039,0.054902}%
\pgfsetstrokecolor{currentstroke}%
\pgfsetdash{}{0pt}%
\pgfpathmoveto{\pgfqpoint{1.836859in}{1.946771in}}%
\pgfpathcurveto{\pgfqpoint{1.847909in}{1.946771in}}{\pgfqpoint{1.858508in}{1.951162in}}{\pgfqpoint{1.866321in}{1.958975in}}%
\pgfpathcurveto{\pgfqpoint{1.874135in}{1.966789in}}{\pgfqpoint{1.878525in}{1.977388in}}{\pgfqpoint{1.878525in}{1.988438in}}%
\pgfpathcurveto{\pgfqpoint{1.878525in}{1.999488in}}{\pgfqpoint{1.874135in}{2.010087in}}{\pgfqpoint{1.866321in}{2.017901in}}%
\pgfpathcurveto{\pgfqpoint{1.858508in}{2.025714in}}{\pgfqpoint{1.847909in}{2.030105in}}{\pgfqpoint{1.836859in}{2.030105in}}%
\pgfpathcurveto{\pgfqpoint{1.825809in}{2.030105in}}{\pgfqpoint{1.815209in}{2.025714in}}{\pgfqpoint{1.807396in}{2.017901in}}%
\pgfpathcurveto{\pgfqpoint{1.799582in}{2.010087in}}{\pgfqpoint{1.795192in}{1.999488in}}{\pgfqpoint{1.795192in}{1.988438in}}%
\pgfpathcurveto{\pgfqpoint{1.795192in}{1.977388in}}{\pgfqpoint{1.799582in}{1.966789in}}{\pgfqpoint{1.807396in}{1.958975in}}%
\pgfpathcurveto{\pgfqpoint{1.815209in}{1.951162in}}{\pgfqpoint{1.825809in}{1.946771in}}{\pgfqpoint{1.836859in}{1.946771in}}%
\pgfpathclose%
\pgfusepath{stroke,fill}%
\end{pgfscope}%
\begin{pgfscope}%
\pgfpathrectangle{\pgfqpoint{0.787074in}{0.548769in}}{\pgfqpoint{5.062926in}{3.102590in}}%
\pgfusepath{clip}%
\pgfsetbuttcap%
\pgfsetroundjoin%
\definecolor{currentfill}{rgb}{0.121569,0.466667,0.705882}%
\pgfsetfillcolor{currentfill}%
\pgfsetlinewidth{1.003750pt}%
\definecolor{currentstroke}{rgb}{0.121569,0.466667,0.705882}%
\pgfsetstrokecolor{currentstroke}%
\pgfsetdash{}{0pt}%
\pgfpathmoveto{\pgfqpoint{1.269407in}{0.648235in}}%
\pgfpathcurveto{\pgfqpoint{1.280458in}{0.648235in}}{\pgfqpoint{1.291057in}{0.652625in}}{\pgfqpoint{1.298870in}{0.660439in}}%
\pgfpathcurveto{\pgfqpoint{1.306684in}{0.668252in}}{\pgfqpoint{1.311074in}{0.678851in}}{\pgfqpoint{1.311074in}{0.689901in}}%
\pgfpathcurveto{\pgfqpoint{1.311074in}{0.700952in}}{\pgfqpoint{1.306684in}{0.711551in}}{\pgfqpoint{1.298870in}{0.719364in}}%
\pgfpathcurveto{\pgfqpoint{1.291057in}{0.727178in}}{\pgfqpoint{1.280458in}{0.731568in}}{\pgfqpoint{1.269407in}{0.731568in}}%
\pgfpathcurveto{\pgfqpoint{1.258357in}{0.731568in}}{\pgfqpoint{1.247758in}{0.727178in}}{\pgfqpoint{1.239945in}{0.719364in}}%
\pgfpathcurveto{\pgfqpoint{1.232131in}{0.711551in}}{\pgfqpoint{1.227741in}{0.700952in}}{\pgfqpoint{1.227741in}{0.689901in}}%
\pgfpathcurveto{\pgfqpoint{1.227741in}{0.678851in}}{\pgfqpoint{1.232131in}{0.668252in}}{\pgfqpoint{1.239945in}{0.660439in}}%
\pgfpathcurveto{\pgfqpoint{1.247758in}{0.652625in}}{\pgfqpoint{1.258357in}{0.648235in}}{\pgfqpoint{1.269407in}{0.648235in}}%
\pgfpathclose%
\pgfusepath{stroke,fill}%
\end{pgfscope}%
\begin{pgfscope}%
\pgfpathrectangle{\pgfqpoint{0.787074in}{0.548769in}}{\pgfqpoint{5.062926in}{3.102590in}}%
\pgfusepath{clip}%
\pgfsetbuttcap%
\pgfsetroundjoin%
\definecolor{currentfill}{rgb}{0.121569,0.466667,0.705882}%
\pgfsetfillcolor{currentfill}%
\pgfsetlinewidth{1.003750pt}%
\definecolor{currentstroke}{rgb}{0.121569,0.466667,0.705882}%
\pgfsetstrokecolor{currentstroke}%
\pgfsetdash{}{0pt}%
\pgfpathmoveto{\pgfqpoint{2.971761in}{3.081873in}}%
\pgfpathcurveto{\pgfqpoint{2.982811in}{3.081873in}}{\pgfqpoint{2.993410in}{3.086263in}}{\pgfqpoint{3.001224in}{3.094077in}}%
\pgfpathcurveto{\pgfqpoint{3.009038in}{3.101890in}}{\pgfqpoint{3.013428in}{3.112489in}}{\pgfqpoint{3.013428in}{3.123539in}}%
\pgfpathcurveto{\pgfqpoint{3.013428in}{3.134589in}}{\pgfqpoint{3.009038in}{3.145188in}}{\pgfqpoint{3.001224in}{3.153002in}}%
\pgfpathcurveto{\pgfqpoint{2.993410in}{3.160816in}}{\pgfqpoint{2.982811in}{3.165206in}}{\pgfqpoint{2.971761in}{3.165206in}}%
\pgfpathcurveto{\pgfqpoint{2.960711in}{3.165206in}}{\pgfqpoint{2.950112in}{3.160816in}}{\pgfqpoint{2.942298in}{3.153002in}}%
\pgfpathcurveto{\pgfqpoint{2.934485in}{3.145188in}}{\pgfqpoint{2.930094in}{3.134589in}}{\pgfqpoint{2.930094in}{3.123539in}}%
\pgfpathcurveto{\pgfqpoint{2.930094in}{3.112489in}}{\pgfqpoint{2.934485in}{3.101890in}}{\pgfqpoint{2.942298in}{3.094077in}}%
\pgfpathcurveto{\pgfqpoint{2.950112in}{3.086263in}}{\pgfqpoint{2.960711in}{3.081873in}}{\pgfqpoint{2.971761in}{3.081873in}}%
\pgfpathclose%
\pgfusepath{stroke,fill}%
\end{pgfscope}%
\begin{pgfscope}%
\pgfpathrectangle{\pgfqpoint{0.787074in}{0.548769in}}{\pgfqpoint{5.062926in}{3.102590in}}%
\pgfusepath{clip}%
\pgfsetbuttcap%
\pgfsetroundjoin%
\definecolor{currentfill}{rgb}{1.000000,0.498039,0.054902}%
\pgfsetfillcolor{currentfill}%
\pgfsetlinewidth{1.003750pt}%
\definecolor{currentstroke}{rgb}{1.000000,0.498039,0.054902}%
\pgfsetstrokecolor{currentstroke}%
\pgfsetdash{}{0pt}%
\pgfpathmoveto{\pgfqpoint{1.080257in}{2.902728in}}%
\pgfpathcurveto{\pgfqpoint{1.091307in}{2.902728in}}{\pgfqpoint{1.101906in}{2.907118in}}{\pgfqpoint{1.109720in}{2.914932in}}%
\pgfpathcurveto{\pgfqpoint{1.117533in}{2.922746in}}{\pgfqpoint{1.121924in}{2.933345in}}{\pgfqpoint{1.121924in}{2.944395in}}%
\pgfpathcurveto{\pgfqpoint{1.121924in}{2.955445in}}{\pgfqpoint{1.117533in}{2.966044in}}{\pgfqpoint{1.109720in}{2.973858in}}%
\pgfpathcurveto{\pgfqpoint{1.101906in}{2.981671in}}{\pgfqpoint{1.091307in}{2.986061in}}{\pgfqpoint{1.080257in}{2.986061in}}%
\pgfpathcurveto{\pgfqpoint{1.069207in}{2.986061in}}{\pgfqpoint{1.058608in}{2.981671in}}{\pgfqpoint{1.050794in}{2.973858in}}%
\pgfpathcurveto{\pgfqpoint{1.042981in}{2.966044in}}{\pgfqpoint{1.038590in}{2.955445in}}{\pgfqpoint{1.038590in}{2.944395in}}%
\pgfpathcurveto{\pgfqpoint{1.038590in}{2.933345in}}{\pgfqpoint{1.042981in}{2.922746in}}{\pgfqpoint{1.050794in}{2.914932in}}%
\pgfpathcurveto{\pgfqpoint{1.058608in}{2.907118in}}{\pgfqpoint{1.069207in}{2.902728in}}{\pgfqpoint{1.080257in}{2.902728in}}%
\pgfpathclose%
\pgfusepath{stroke,fill}%
\end{pgfscope}%
\begin{pgfscope}%
\pgfpathrectangle{\pgfqpoint{0.787074in}{0.548769in}}{\pgfqpoint{5.062926in}{3.102590in}}%
\pgfusepath{clip}%
\pgfsetbuttcap%
\pgfsetroundjoin%
\definecolor{currentfill}{rgb}{1.000000,0.498039,0.054902}%
\pgfsetfillcolor{currentfill}%
\pgfsetlinewidth{1.003750pt}%
\definecolor{currentstroke}{rgb}{1.000000,0.498039,0.054902}%
\pgfsetstrokecolor{currentstroke}%
\pgfsetdash{}{0pt}%
\pgfpathmoveto{\pgfqpoint{1.269407in}{2.766568in}}%
\pgfpathcurveto{\pgfqpoint{1.280458in}{2.766568in}}{\pgfqpoint{1.291057in}{2.770959in}}{\pgfqpoint{1.298870in}{2.778772in}}%
\pgfpathcurveto{\pgfqpoint{1.306684in}{2.786586in}}{\pgfqpoint{1.311074in}{2.797185in}}{\pgfqpoint{1.311074in}{2.808235in}}%
\pgfpathcurveto{\pgfqpoint{1.311074in}{2.819285in}}{\pgfqpoint{1.306684in}{2.829884in}}{\pgfqpoint{1.298870in}{2.837698in}}%
\pgfpathcurveto{\pgfqpoint{1.291057in}{2.845511in}}{\pgfqpoint{1.280458in}{2.849902in}}{\pgfqpoint{1.269407in}{2.849902in}}%
\pgfpathcurveto{\pgfqpoint{1.258357in}{2.849902in}}{\pgfqpoint{1.247758in}{2.845511in}}{\pgfqpoint{1.239945in}{2.837698in}}%
\pgfpathcurveto{\pgfqpoint{1.232131in}{2.829884in}}{\pgfqpoint{1.227741in}{2.819285in}}{\pgfqpoint{1.227741in}{2.808235in}}%
\pgfpathcurveto{\pgfqpoint{1.227741in}{2.797185in}}{\pgfqpoint{1.232131in}{2.786586in}}{\pgfqpoint{1.239945in}{2.778772in}}%
\pgfpathcurveto{\pgfqpoint{1.247758in}{2.770959in}}{\pgfqpoint{1.258357in}{2.766568in}}{\pgfqpoint{1.269407in}{2.766568in}}%
\pgfpathclose%
\pgfusepath{stroke,fill}%
\end{pgfscope}%
\begin{pgfscope}%
\pgfpathrectangle{\pgfqpoint{0.787074in}{0.548769in}}{\pgfqpoint{5.062926in}{3.102590in}}%
\pgfusepath{clip}%
\pgfsetbuttcap%
\pgfsetroundjoin%
\definecolor{currentfill}{rgb}{1.000000,0.498039,0.054902}%
\pgfsetfillcolor{currentfill}%
\pgfsetlinewidth{1.003750pt}%
\definecolor{currentstroke}{rgb}{1.000000,0.498039,0.054902}%
\pgfsetstrokecolor{currentstroke}%
\pgfsetdash{}{0pt}%
\pgfpathmoveto{\pgfqpoint{1.080257in}{1.775111in}}%
\pgfpathcurveto{\pgfqpoint{1.091307in}{1.775111in}}{\pgfqpoint{1.101906in}{1.779501in}}{\pgfqpoint{1.109720in}{1.787315in}}%
\pgfpathcurveto{\pgfqpoint{1.117533in}{1.795129in}}{\pgfqpoint{1.121924in}{1.805728in}}{\pgfqpoint{1.121924in}{1.816778in}}%
\pgfpathcurveto{\pgfqpoint{1.121924in}{1.827828in}}{\pgfqpoint{1.117533in}{1.838427in}}{\pgfqpoint{1.109720in}{1.846241in}}%
\pgfpathcurveto{\pgfqpoint{1.101906in}{1.854054in}}{\pgfqpoint{1.091307in}{1.858444in}}{\pgfqpoint{1.080257in}{1.858444in}}%
\pgfpathcurveto{\pgfqpoint{1.069207in}{1.858444in}}{\pgfqpoint{1.058608in}{1.854054in}}{\pgfqpoint{1.050794in}{1.846241in}}%
\pgfpathcurveto{\pgfqpoint{1.042981in}{1.838427in}}{\pgfqpoint{1.038590in}{1.827828in}}{\pgfqpoint{1.038590in}{1.816778in}}%
\pgfpathcurveto{\pgfqpoint{1.038590in}{1.805728in}}{\pgfqpoint{1.042981in}{1.795129in}}{\pgfqpoint{1.050794in}{1.787315in}}%
\pgfpathcurveto{\pgfqpoint{1.058608in}{1.779501in}}{\pgfqpoint{1.069207in}{1.775111in}}{\pgfqpoint{1.080257in}{1.775111in}}%
\pgfpathclose%
\pgfusepath{stroke,fill}%
\end{pgfscope}%
\begin{pgfscope}%
\pgfpathrectangle{\pgfqpoint{0.787074in}{0.548769in}}{\pgfqpoint{5.062926in}{3.102590in}}%
\pgfusepath{clip}%
\pgfsetbuttcap%
\pgfsetroundjoin%
\definecolor{currentfill}{rgb}{1.000000,0.498039,0.054902}%
\pgfsetfillcolor{currentfill}%
\pgfsetlinewidth{1.003750pt}%
\definecolor{currentstroke}{rgb}{1.000000,0.498039,0.054902}%
\pgfsetstrokecolor{currentstroke}%
\pgfsetdash{}{0pt}%
\pgfpathmoveto{\pgfqpoint{1.332458in}{1.545971in}}%
\pgfpathcurveto{\pgfqpoint{1.343508in}{1.545971in}}{\pgfqpoint{1.354107in}{1.550361in}}{\pgfqpoint{1.361920in}{1.558175in}}%
\pgfpathcurveto{\pgfqpoint{1.369734in}{1.565989in}}{\pgfqpoint{1.374124in}{1.576588in}}{\pgfqpoint{1.374124in}{1.587638in}}%
\pgfpathcurveto{\pgfqpoint{1.374124in}{1.598688in}}{\pgfqpoint{1.369734in}{1.609287in}}{\pgfqpoint{1.361920in}{1.617101in}}%
\pgfpathcurveto{\pgfqpoint{1.354107in}{1.624914in}}{\pgfqpoint{1.343508in}{1.629305in}}{\pgfqpoint{1.332458in}{1.629305in}}%
\pgfpathcurveto{\pgfqpoint{1.321407in}{1.629305in}}{\pgfqpoint{1.310808in}{1.624914in}}{\pgfqpoint{1.302995in}{1.617101in}}%
\pgfpathcurveto{\pgfqpoint{1.295181in}{1.609287in}}{\pgfqpoint{1.290791in}{1.598688in}}{\pgfqpoint{1.290791in}{1.587638in}}%
\pgfpathcurveto{\pgfqpoint{1.290791in}{1.576588in}}{\pgfqpoint{1.295181in}{1.565989in}}{\pgfqpoint{1.302995in}{1.558175in}}%
\pgfpathcurveto{\pgfqpoint{1.310808in}{1.550361in}}{\pgfqpoint{1.321407in}{1.545971in}}{\pgfqpoint{1.332458in}{1.545971in}}%
\pgfpathclose%
\pgfusepath{stroke,fill}%
\end{pgfscope}%
\begin{pgfscope}%
\pgfpathrectangle{\pgfqpoint{0.787074in}{0.548769in}}{\pgfqpoint{5.062926in}{3.102590in}}%
\pgfusepath{clip}%
\pgfsetbuttcap%
\pgfsetroundjoin%
\definecolor{currentfill}{rgb}{0.121569,0.466667,0.705882}%
\pgfsetfillcolor{currentfill}%
\pgfsetlinewidth{1.003750pt}%
\definecolor{currentstroke}{rgb}{0.121569,0.466667,0.705882}%
\pgfsetstrokecolor{currentstroke}%
\pgfsetdash{}{0pt}%
\pgfpathmoveto{\pgfqpoint{1.080257in}{2.232840in}}%
\pgfpathcurveto{\pgfqpoint{1.091307in}{2.232840in}}{\pgfqpoint{1.101906in}{2.237230in}}{\pgfqpoint{1.109720in}{2.245044in}}%
\pgfpathcurveto{\pgfqpoint{1.117533in}{2.252857in}}{\pgfqpoint{1.121924in}{2.263456in}}{\pgfqpoint{1.121924in}{2.274506in}}%
\pgfpathcurveto{\pgfqpoint{1.121924in}{2.285556in}}{\pgfqpoint{1.117533in}{2.296156in}}{\pgfqpoint{1.109720in}{2.303969in}}%
\pgfpathcurveto{\pgfqpoint{1.101906in}{2.311783in}}{\pgfqpoint{1.091307in}{2.316173in}}{\pgfqpoint{1.080257in}{2.316173in}}%
\pgfpathcurveto{\pgfqpoint{1.069207in}{2.316173in}}{\pgfqpoint{1.058608in}{2.311783in}}{\pgfqpoint{1.050794in}{2.303969in}}%
\pgfpathcurveto{\pgfqpoint{1.042981in}{2.296156in}}{\pgfqpoint{1.038590in}{2.285556in}}{\pgfqpoint{1.038590in}{2.274506in}}%
\pgfpathcurveto{\pgfqpoint{1.038590in}{2.263456in}}{\pgfqpoint{1.042981in}{2.252857in}}{\pgfqpoint{1.050794in}{2.245044in}}%
\pgfpathcurveto{\pgfqpoint{1.058608in}{2.237230in}}{\pgfqpoint{1.069207in}{2.232840in}}{\pgfqpoint{1.080257in}{2.232840in}}%
\pgfpathclose%
\pgfusepath{stroke,fill}%
\end{pgfscope}%
\begin{pgfscope}%
\pgfpathrectangle{\pgfqpoint{0.787074in}{0.548769in}}{\pgfqpoint{5.062926in}{3.102590in}}%
\pgfusepath{clip}%
\pgfsetbuttcap%
\pgfsetroundjoin%
\definecolor{currentfill}{rgb}{0.121569,0.466667,0.705882}%
\pgfsetfillcolor{currentfill}%
\pgfsetlinewidth{1.003750pt}%
\definecolor{currentstroke}{rgb}{0.121569,0.466667,0.705882}%
\pgfsetstrokecolor{currentstroke}%
\pgfsetdash{}{0pt}%
\pgfpathmoveto{\pgfqpoint{1.521608in}{0.676149in}}%
\pgfpathcurveto{\pgfqpoint{1.532658in}{0.676149in}}{\pgfqpoint{1.543257in}{0.680539in}}{\pgfqpoint{1.551071in}{0.688353in}}%
\pgfpathcurveto{\pgfqpoint{1.558884in}{0.696167in}}{\pgfqpoint{1.563275in}{0.706766in}}{\pgfqpoint{1.563275in}{0.717816in}}%
\pgfpathcurveto{\pgfqpoint{1.563275in}{0.728866in}}{\pgfqpoint{1.558884in}{0.739465in}}{\pgfqpoint{1.551071in}{0.747279in}}%
\pgfpathcurveto{\pgfqpoint{1.543257in}{0.755092in}}{\pgfqpoint{1.532658in}{0.759482in}}{\pgfqpoint{1.521608in}{0.759482in}}%
\pgfpathcurveto{\pgfqpoint{1.510558in}{0.759482in}}{\pgfqpoint{1.499959in}{0.755092in}}{\pgfqpoint{1.492145in}{0.747279in}}%
\pgfpathcurveto{\pgfqpoint{1.484332in}{0.739465in}}{\pgfqpoint{1.479941in}{0.728866in}}{\pgfqpoint{1.479941in}{0.717816in}}%
\pgfpathcurveto{\pgfqpoint{1.479941in}{0.706766in}}{\pgfqpoint{1.484332in}{0.696167in}}{\pgfqpoint{1.492145in}{0.688353in}}%
\pgfpathcurveto{\pgfqpoint{1.499959in}{0.680539in}}{\pgfqpoint{1.510558in}{0.676149in}}{\pgfqpoint{1.521608in}{0.676149in}}%
\pgfpathclose%
\pgfusepath{stroke,fill}%
\end{pgfscope}%
\begin{pgfscope}%
\pgfpathrectangle{\pgfqpoint{0.787074in}{0.548769in}}{\pgfqpoint{5.062926in}{3.102590in}}%
\pgfusepath{clip}%
\pgfsetbuttcap%
\pgfsetroundjoin%
\definecolor{currentfill}{rgb}{1.000000,0.498039,0.054902}%
\pgfsetfillcolor{currentfill}%
\pgfsetlinewidth{1.003750pt}%
\definecolor{currentstroke}{rgb}{1.000000,0.498039,0.054902}%
\pgfsetstrokecolor{currentstroke}%
\pgfsetdash{}{0pt}%
\pgfpathmoveto{\pgfqpoint{1.143307in}{1.907282in}}%
\pgfpathcurveto{\pgfqpoint{1.154357in}{1.907282in}}{\pgfqpoint{1.164956in}{1.911673in}}{\pgfqpoint{1.172770in}{1.919486in}}%
\pgfpathcurveto{\pgfqpoint{1.180584in}{1.927300in}}{\pgfqpoint{1.184974in}{1.937899in}}{\pgfqpoint{1.184974in}{1.948949in}}%
\pgfpathcurveto{\pgfqpoint{1.184974in}{1.959999in}}{\pgfqpoint{1.180584in}{1.970598in}}{\pgfqpoint{1.172770in}{1.978412in}}%
\pgfpathcurveto{\pgfqpoint{1.164956in}{1.986225in}}{\pgfqpoint{1.154357in}{1.990616in}}{\pgfqpoint{1.143307in}{1.990616in}}%
\pgfpathcurveto{\pgfqpoint{1.132257in}{1.990616in}}{\pgfqpoint{1.121658in}{1.986225in}}{\pgfqpoint{1.113844in}{1.978412in}}%
\pgfpathcurveto{\pgfqpoint{1.106031in}{1.970598in}}{\pgfqpoint{1.101640in}{1.959999in}}{\pgfqpoint{1.101640in}{1.948949in}}%
\pgfpathcurveto{\pgfqpoint{1.101640in}{1.937899in}}{\pgfqpoint{1.106031in}{1.927300in}}{\pgfqpoint{1.113844in}{1.919486in}}%
\pgfpathcurveto{\pgfqpoint{1.121658in}{1.911673in}}{\pgfqpoint{1.132257in}{1.907282in}}{\pgfqpoint{1.143307in}{1.907282in}}%
\pgfpathclose%
\pgfusepath{stroke,fill}%
\end{pgfscope}%
\begin{pgfscope}%
\pgfpathrectangle{\pgfqpoint{0.787074in}{0.548769in}}{\pgfqpoint{5.062926in}{3.102590in}}%
\pgfusepath{clip}%
\pgfsetbuttcap%
\pgfsetroundjoin%
\definecolor{currentfill}{rgb}{1.000000,0.498039,0.054902}%
\pgfsetfillcolor{currentfill}%
\pgfsetlinewidth{1.003750pt}%
\definecolor{currentstroke}{rgb}{1.000000,0.498039,0.054902}%
\pgfsetstrokecolor{currentstroke}%
\pgfsetdash{}{0pt}%
\pgfpathmoveto{\pgfqpoint{1.332458in}{2.839399in}}%
\pgfpathcurveto{\pgfqpoint{1.343508in}{2.839399in}}{\pgfqpoint{1.354107in}{2.843789in}}{\pgfqpoint{1.361920in}{2.851603in}}%
\pgfpathcurveto{\pgfqpoint{1.369734in}{2.859416in}}{\pgfqpoint{1.374124in}{2.870015in}}{\pgfqpoint{1.374124in}{2.881065in}}%
\pgfpathcurveto{\pgfqpoint{1.374124in}{2.892115in}}{\pgfqpoint{1.369734in}{2.902715in}}{\pgfqpoint{1.361920in}{2.910528in}}%
\pgfpathcurveto{\pgfqpoint{1.354107in}{2.918342in}}{\pgfqpoint{1.343508in}{2.922732in}}{\pgfqpoint{1.332458in}{2.922732in}}%
\pgfpathcurveto{\pgfqpoint{1.321407in}{2.922732in}}{\pgfqpoint{1.310808in}{2.918342in}}{\pgfqpoint{1.302995in}{2.910528in}}%
\pgfpathcurveto{\pgfqpoint{1.295181in}{2.902715in}}{\pgfqpoint{1.290791in}{2.892115in}}{\pgfqpoint{1.290791in}{2.881065in}}%
\pgfpathcurveto{\pgfqpoint{1.290791in}{2.870015in}}{\pgfqpoint{1.295181in}{2.859416in}}{\pgfqpoint{1.302995in}{2.851603in}}%
\pgfpathcurveto{\pgfqpoint{1.310808in}{2.843789in}}{\pgfqpoint{1.321407in}{2.839399in}}{\pgfqpoint{1.332458in}{2.839399in}}%
\pgfpathclose%
\pgfusepath{stroke,fill}%
\end{pgfscope}%
\begin{pgfscope}%
\pgfpathrectangle{\pgfqpoint{0.787074in}{0.548769in}}{\pgfqpoint{5.062926in}{3.102590in}}%
\pgfusepath{clip}%
\pgfsetbuttcap%
\pgfsetroundjoin%
\definecolor{currentfill}{rgb}{0.121569,0.466667,0.705882}%
\pgfsetfillcolor{currentfill}%
\pgfsetlinewidth{1.003750pt}%
\definecolor{currentstroke}{rgb}{0.121569,0.466667,0.705882}%
\pgfsetstrokecolor{currentstroke}%
\pgfsetdash{}{0pt}%
\pgfpathmoveto{\pgfqpoint{2.152109in}{0.648138in}}%
\pgfpathcurveto{\pgfqpoint{2.163159in}{0.648138in}}{\pgfqpoint{2.173759in}{0.652528in}}{\pgfqpoint{2.181572in}{0.660342in}}%
\pgfpathcurveto{\pgfqpoint{2.189386in}{0.668156in}}{\pgfqpoint{2.193776in}{0.678755in}}{\pgfqpoint{2.193776in}{0.689805in}}%
\pgfpathcurveto{\pgfqpoint{2.193776in}{0.700855in}}{\pgfqpoint{2.189386in}{0.711454in}}{\pgfqpoint{2.181572in}{0.719268in}}%
\pgfpathcurveto{\pgfqpoint{2.173759in}{0.727081in}}{\pgfqpoint{2.163159in}{0.731472in}}{\pgfqpoint{2.152109in}{0.731472in}}%
\pgfpathcurveto{\pgfqpoint{2.141059in}{0.731472in}}{\pgfqpoint{2.130460in}{0.727081in}}{\pgfqpoint{2.122647in}{0.719268in}}%
\pgfpathcurveto{\pgfqpoint{2.114833in}{0.711454in}}{\pgfqpoint{2.110443in}{0.700855in}}{\pgfqpoint{2.110443in}{0.689805in}}%
\pgfpathcurveto{\pgfqpoint{2.110443in}{0.678755in}}{\pgfqpoint{2.114833in}{0.668156in}}{\pgfqpoint{2.122647in}{0.660342in}}%
\pgfpathcurveto{\pgfqpoint{2.130460in}{0.652528in}}{\pgfqpoint{2.141059in}{0.648138in}}{\pgfqpoint{2.152109in}{0.648138in}}%
\pgfpathclose%
\pgfusepath{stroke,fill}%
\end{pgfscope}%
\begin{pgfscope}%
\pgfpathrectangle{\pgfqpoint{0.787074in}{0.548769in}}{\pgfqpoint{5.062926in}{3.102590in}}%
\pgfusepath{clip}%
\pgfsetbuttcap%
\pgfsetroundjoin%
\definecolor{currentfill}{rgb}{0.121569,0.466667,0.705882}%
\pgfsetfillcolor{currentfill}%
\pgfsetlinewidth{1.003750pt}%
\definecolor{currentstroke}{rgb}{0.121569,0.466667,0.705882}%
\pgfsetstrokecolor{currentstroke}%
\pgfsetdash{}{0pt}%
\pgfpathmoveto{\pgfqpoint{2.026009in}{0.649828in}}%
\pgfpathcurveto{\pgfqpoint{2.037059in}{0.649828in}}{\pgfqpoint{2.047658in}{0.654218in}}{\pgfqpoint{2.055472in}{0.662032in}}%
\pgfpathcurveto{\pgfqpoint{2.063285in}{0.669845in}}{\pgfqpoint{2.067676in}{0.680444in}}{\pgfqpoint{2.067676in}{0.691495in}}%
\pgfpathcurveto{\pgfqpoint{2.067676in}{0.702545in}}{\pgfqpoint{2.063285in}{0.713144in}}{\pgfqpoint{2.055472in}{0.720957in}}%
\pgfpathcurveto{\pgfqpoint{2.047658in}{0.728771in}}{\pgfqpoint{2.037059in}{0.733161in}}{\pgfqpoint{2.026009in}{0.733161in}}%
\pgfpathcurveto{\pgfqpoint{2.014959in}{0.733161in}}{\pgfqpoint{2.004360in}{0.728771in}}{\pgfqpoint{1.996546in}{0.720957in}}%
\pgfpathcurveto{\pgfqpoint{1.988733in}{0.713144in}}{\pgfqpoint{1.984342in}{0.702545in}}{\pgfqpoint{1.984342in}{0.691495in}}%
\pgfpathcurveto{\pgfqpoint{1.984342in}{0.680444in}}{\pgfqpoint{1.988733in}{0.669845in}}{\pgfqpoint{1.996546in}{0.662032in}}%
\pgfpathcurveto{\pgfqpoint{2.004360in}{0.654218in}}{\pgfqpoint{2.014959in}{0.649828in}}{\pgfqpoint{2.026009in}{0.649828in}}%
\pgfpathclose%
\pgfusepath{stroke,fill}%
\end{pgfscope}%
\begin{pgfscope}%
\pgfpathrectangle{\pgfqpoint{0.787074in}{0.548769in}}{\pgfqpoint{5.062926in}{3.102590in}}%
\pgfusepath{clip}%
\pgfsetbuttcap%
\pgfsetroundjoin%
\definecolor{currentfill}{rgb}{0.121569,0.466667,0.705882}%
\pgfsetfillcolor{currentfill}%
\pgfsetlinewidth{1.003750pt}%
\definecolor{currentstroke}{rgb}{0.121569,0.466667,0.705882}%
\pgfsetstrokecolor{currentstroke}%
\pgfsetdash{}{0pt}%
\pgfpathmoveto{\pgfqpoint{1.962959in}{2.575639in}}%
\pgfpathcurveto{\pgfqpoint{1.974009in}{2.575639in}}{\pgfqpoint{1.984608in}{2.580029in}}{\pgfqpoint{1.992422in}{2.587842in}}%
\pgfpathcurveto{\pgfqpoint{2.000235in}{2.595656in}}{\pgfqpoint{2.004626in}{2.606255in}}{\pgfqpoint{2.004626in}{2.617305in}}%
\pgfpathcurveto{\pgfqpoint{2.004626in}{2.628355in}}{\pgfqpoint{2.000235in}{2.638954in}}{\pgfqpoint{1.992422in}{2.646768in}}%
\pgfpathcurveto{\pgfqpoint{1.984608in}{2.654582in}}{\pgfqpoint{1.974009in}{2.658972in}}{\pgfqpoint{1.962959in}{2.658972in}}%
\pgfpathcurveto{\pgfqpoint{1.951909in}{2.658972in}}{\pgfqpoint{1.941310in}{2.654582in}}{\pgfqpoint{1.933496in}{2.646768in}}%
\pgfpathcurveto{\pgfqpoint{1.925683in}{2.638954in}}{\pgfqpoint{1.921292in}{2.628355in}}{\pgfqpoint{1.921292in}{2.617305in}}%
\pgfpathcurveto{\pgfqpoint{1.921292in}{2.606255in}}{\pgfqpoint{1.925683in}{2.595656in}}{\pgfqpoint{1.933496in}{2.587842in}}%
\pgfpathcurveto{\pgfqpoint{1.941310in}{2.580029in}}{\pgfqpoint{1.951909in}{2.575639in}}{\pgfqpoint{1.962959in}{2.575639in}}%
\pgfpathclose%
\pgfusepath{stroke,fill}%
\end{pgfscope}%
\begin{pgfscope}%
\pgfpathrectangle{\pgfqpoint{0.787074in}{0.548769in}}{\pgfqpoint{5.062926in}{3.102590in}}%
\pgfusepath{clip}%
\pgfsetbuttcap%
\pgfsetroundjoin%
\definecolor{currentfill}{rgb}{1.000000,0.498039,0.054902}%
\pgfsetfillcolor{currentfill}%
\pgfsetlinewidth{1.003750pt}%
\definecolor{currentstroke}{rgb}{1.000000,0.498039,0.054902}%
\pgfsetstrokecolor{currentstroke}%
\pgfsetdash{}{0pt}%
\pgfpathmoveto{\pgfqpoint{2.404310in}{2.218400in}}%
\pgfpathcurveto{\pgfqpoint{2.415360in}{2.218400in}}{\pgfqpoint{2.425959in}{2.222790in}}{\pgfqpoint{2.433773in}{2.230604in}}%
\pgfpathcurveto{\pgfqpoint{2.441586in}{2.238417in}}{\pgfqpoint{2.445977in}{2.249016in}}{\pgfqpoint{2.445977in}{2.260066in}}%
\pgfpathcurveto{\pgfqpoint{2.445977in}{2.271116in}}{\pgfqpoint{2.441586in}{2.281715in}}{\pgfqpoint{2.433773in}{2.289529in}}%
\pgfpathcurveto{\pgfqpoint{2.425959in}{2.297343in}}{\pgfqpoint{2.415360in}{2.301733in}}{\pgfqpoint{2.404310in}{2.301733in}}%
\pgfpathcurveto{\pgfqpoint{2.393260in}{2.301733in}}{\pgfqpoint{2.382661in}{2.297343in}}{\pgfqpoint{2.374847in}{2.289529in}}%
\pgfpathcurveto{\pgfqpoint{2.367033in}{2.281715in}}{\pgfqpoint{2.362643in}{2.271116in}}{\pgfqpoint{2.362643in}{2.260066in}}%
\pgfpathcurveto{\pgfqpoint{2.362643in}{2.249016in}}{\pgfqpoint{2.367033in}{2.238417in}}{\pgfqpoint{2.374847in}{2.230604in}}%
\pgfpathcurveto{\pgfqpoint{2.382661in}{2.222790in}}{\pgfqpoint{2.393260in}{2.218400in}}{\pgfqpoint{2.404310in}{2.218400in}}%
\pgfpathclose%
\pgfusepath{stroke,fill}%
\end{pgfscope}%
\begin{pgfscope}%
\pgfpathrectangle{\pgfqpoint{0.787074in}{0.548769in}}{\pgfqpoint{5.062926in}{3.102590in}}%
\pgfusepath{clip}%
\pgfsetbuttcap%
\pgfsetroundjoin%
\definecolor{currentfill}{rgb}{1.000000,0.498039,0.054902}%
\pgfsetfillcolor{currentfill}%
\pgfsetlinewidth{1.003750pt}%
\definecolor{currentstroke}{rgb}{1.000000,0.498039,0.054902}%
\pgfsetstrokecolor{currentstroke}%
\pgfsetdash{}{0pt}%
\pgfpathmoveto{\pgfqpoint{1.899909in}{2.443019in}}%
\pgfpathcurveto{\pgfqpoint{1.910959in}{2.443019in}}{\pgfqpoint{1.921558in}{2.447409in}}{\pgfqpoint{1.929372in}{2.455223in}}%
\pgfpathcurveto{\pgfqpoint{1.937185in}{2.463037in}}{\pgfqpoint{1.941575in}{2.473636in}}{\pgfqpoint{1.941575in}{2.484686in}}%
\pgfpathcurveto{\pgfqpoint{1.941575in}{2.495736in}}{\pgfqpoint{1.937185in}{2.506335in}}{\pgfqpoint{1.929372in}{2.514149in}}%
\pgfpathcurveto{\pgfqpoint{1.921558in}{2.521962in}}{\pgfqpoint{1.910959in}{2.526352in}}{\pgfqpoint{1.899909in}{2.526352in}}%
\pgfpathcurveto{\pgfqpoint{1.888859in}{2.526352in}}{\pgfqpoint{1.878260in}{2.521962in}}{\pgfqpoint{1.870446in}{2.514149in}}%
\pgfpathcurveto{\pgfqpoint{1.862632in}{2.506335in}}{\pgfqpoint{1.858242in}{2.495736in}}{\pgfqpoint{1.858242in}{2.484686in}}%
\pgfpathcurveto{\pgfqpoint{1.858242in}{2.473636in}}{\pgfqpoint{1.862632in}{2.463037in}}{\pgfqpoint{1.870446in}{2.455223in}}%
\pgfpathcurveto{\pgfqpoint{1.878260in}{2.447409in}}{\pgfqpoint{1.888859in}{2.443019in}}{\pgfqpoint{1.899909in}{2.443019in}}%
\pgfpathclose%
\pgfusepath{stroke,fill}%
\end{pgfscope}%
\begin{pgfscope}%
\pgfpathrectangle{\pgfqpoint{0.787074in}{0.548769in}}{\pgfqpoint{5.062926in}{3.102590in}}%
\pgfusepath{clip}%
\pgfsetbuttcap%
\pgfsetroundjoin%
\definecolor{currentfill}{rgb}{0.121569,0.466667,0.705882}%
\pgfsetfillcolor{currentfill}%
\pgfsetlinewidth{1.003750pt}%
\definecolor{currentstroke}{rgb}{0.121569,0.466667,0.705882}%
\pgfsetstrokecolor{currentstroke}%
\pgfsetdash{}{0pt}%
\pgfpathmoveto{\pgfqpoint{1.773809in}{1.631630in}}%
\pgfpathcurveto{\pgfqpoint{1.784859in}{1.631630in}}{\pgfqpoint{1.795458in}{1.636020in}}{\pgfqpoint{1.803271in}{1.643834in}}%
\pgfpathcurveto{\pgfqpoint{1.811085in}{1.651647in}}{\pgfqpoint{1.815475in}{1.662246in}}{\pgfqpoint{1.815475in}{1.673296in}}%
\pgfpathcurveto{\pgfqpoint{1.815475in}{1.684346in}}{\pgfqpoint{1.811085in}{1.694945in}}{\pgfqpoint{1.803271in}{1.702759in}}%
\pgfpathcurveto{\pgfqpoint{1.795458in}{1.710573in}}{\pgfqpoint{1.784859in}{1.714963in}}{\pgfqpoint{1.773809in}{1.714963in}}%
\pgfpathcurveto{\pgfqpoint{1.762758in}{1.714963in}}{\pgfqpoint{1.752159in}{1.710573in}}{\pgfqpoint{1.744346in}{1.702759in}}%
\pgfpathcurveto{\pgfqpoint{1.736532in}{1.694945in}}{\pgfqpoint{1.732142in}{1.684346in}}{\pgfqpoint{1.732142in}{1.673296in}}%
\pgfpathcurveto{\pgfqpoint{1.732142in}{1.662246in}}{\pgfqpoint{1.736532in}{1.651647in}}{\pgfqpoint{1.744346in}{1.643834in}}%
\pgfpathcurveto{\pgfqpoint{1.752159in}{1.636020in}}{\pgfqpoint{1.762758in}{1.631630in}}{\pgfqpoint{1.773809in}{1.631630in}}%
\pgfpathclose%
\pgfusepath{stroke,fill}%
\end{pgfscope}%
\begin{pgfscope}%
\pgfpathrectangle{\pgfqpoint{0.787074in}{0.548769in}}{\pgfqpoint{5.062926in}{3.102590in}}%
\pgfusepath{clip}%
\pgfsetbuttcap%
\pgfsetroundjoin%
\definecolor{currentfill}{rgb}{1.000000,0.498039,0.054902}%
\pgfsetfillcolor{currentfill}%
\pgfsetlinewidth{1.003750pt}%
\definecolor{currentstroke}{rgb}{1.000000,0.498039,0.054902}%
\pgfsetstrokecolor{currentstroke}%
\pgfsetdash{}{0pt}%
\pgfpathmoveto{\pgfqpoint{1.710758in}{1.681005in}}%
\pgfpathcurveto{\pgfqpoint{1.721809in}{1.681005in}}{\pgfqpoint{1.732408in}{1.685396in}}{\pgfqpoint{1.740221in}{1.693209in}}%
\pgfpathcurveto{\pgfqpoint{1.748035in}{1.701023in}}{\pgfqpoint{1.752425in}{1.711622in}}{\pgfqpoint{1.752425in}{1.722672in}}%
\pgfpathcurveto{\pgfqpoint{1.752425in}{1.733722in}}{\pgfqpoint{1.748035in}{1.744321in}}{\pgfqpoint{1.740221in}{1.752135in}}%
\pgfpathcurveto{\pgfqpoint{1.732408in}{1.759948in}}{\pgfqpoint{1.721809in}{1.764339in}}{\pgfqpoint{1.710758in}{1.764339in}}%
\pgfpathcurveto{\pgfqpoint{1.699708in}{1.764339in}}{\pgfqpoint{1.689109in}{1.759948in}}{\pgfqpoint{1.681296in}{1.752135in}}%
\pgfpathcurveto{\pgfqpoint{1.673482in}{1.744321in}}{\pgfqpoint{1.669092in}{1.733722in}}{\pgfqpoint{1.669092in}{1.722672in}}%
\pgfpathcurveto{\pgfqpoint{1.669092in}{1.711622in}}{\pgfqpoint{1.673482in}{1.701023in}}{\pgfqpoint{1.681296in}{1.693209in}}%
\pgfpathcurveto{\pgfqpoint{1.689109in}{1.685396in}}{\pgfqpoint{1.699708in}{1.681005in}}{\pgfqpoint{1.710758in}{1.681005in}}%
\pgfpathclose%
\pgfusepath{stroke,fill}%
\end{pgfscope}%
\begin{pgfscope}%
\pgfpathrectangle{\pgfqpoint{0.787074in}{0.548769in}}{\pgfqpoint{5.062926in}{3.102590in}}%
\pgfusepath{clip}%
\pgfsetbuttcap%
\pgfsetroundjoin%
\definecolor{currentfill}{rgb}{0.121569,0.466667,0.705882}%
\pgfsetfillcolor{currentfill}%
\pgfsetlinewidth{1.003750pt}%
\definecolor{currentstroke}{rgb}{0.121569,0.466667,0.705882}%
\pgfsetstrokecolor{currentstroke}%
\pgfsetdash{}{0pt}%
\pgfpathmoveto{\pgfqpoint{1.647708in}{0.648134in}}%
\pgfpathcurveto{\pgfqpoint{1.658758in}{0.648134in}}{\pgfqpoint{1.669357in}{0.652524in}}{\pgfqpoint{1.677171in}{0.660338in}}%
\pgfpathcurveto{\pgfqpoint{1.684985in}{0.668151in}}{\pgfqpoint{1.689375in}{0.678750in}}{\pgfqpoint{1.689375in}{0.689800in}}%
\pgfpathcurveto{\pgfqpoint{1.689375in}{0.700850in}}{\pgfqpoint{1.684985in}{0.711450in}}{\pgfqpoint{1.677171in}{0.719263in}}%
\pgfpathcurveto{\pgfqpoint{1.669357in}{0.727077in}}{\pgfqpoint{1.658758in}{0.731467in}}{\pgfqpoint{1.647708in}{0.731467in}}%
\pgfpathcurveto{\pgfqpoint{1.636658in}{0.731467in}}{\pgfqpoint{1.626059in}{0.727077in}}{\pgfqpoint{1.618245in}{0.719263in}}%
\pgfpathcurveto{\pgfqpoint{1.610432in}{0.711450in}}{\pgfqpoint{1.606042in}{0.700850in}}{\pgfqpoint{1.606042in}{0.689800in}}%
\pgfpathcurveto{\pgfqpoint{1.606042in}{0.678750in}}{\pgfqpoint{1.610432in}{0.668151in}}{\pgfqpoint{1.618245in}{0.660338in}}%
\pgfpathcurveto{\pgfqpoint{1.626059in}{0.652524in}}{\pgfqpoint{1.636658in}{0.648134in}}{\pgfqpoint{1.647708in}{0.648134in}}%
\pgfpathclose%
\pgfusepath{stroke,fill}%
\end{pgfscope}%
\begin{pgfscope}%
\pgfpathrectangle{\pgfqpoint{0.787074in}{0.548769in}}{\pgfqpoint{5.062926in}{3.102590in}}%
\pgfusepath{clip}%
\pgfsetbuttcap%
\pgfsetroundjoin%
\definecolor{currentfill}{rgb}{1.000000,0.498039,0.054902}%
\pgfsetfillcolor{currentfill}%
\pgfsetlinewidth{1.003750pt}%
\definecolor{currentstroke}{rgb}{1.000000,0.498039,0.054902}%
\pgfsetstrokecolor{currentstroke}%
\pgfsetdash{}{0pt}%
\pgfpathmoveto{\pgfqpoint{1.710758in}{2.734356in}}%
\pgfpathcurveto{\pgfqpoint{1.721809in}{2.734356in}}{\pgfqpoint{1.732408in}{2.738746in}}{\pgfqpoint{1.740221in}{2.746560in}}%
\pgfpathcurveto{\pgfqpoint{1.748035in}{2.754373in}}{\pgfqpoint{1.752425in}{2.764972in}}{\pgfqpoint{1.752425in}{2.776022in}}%
\pgfpathcurveto{\pgfqpoint{1.752425in}{2.787073in}}{\pgfqpoint{1.748035in}{2.797672in}}{\pgfqpoint{1.740221in}{2.805485in}}%
\pgfpathcurveto{\pgfqpoint{1.732408in}{2.813299in}}{\pgfqpoint{1.721809in}{2.817689in}}{\pgfqpoint{1.710758in}{2.817689in}}%
\pgfpathcurveto{\pgfqpoint{1.699708in}{2.817689in}}{\pgfqpoint{1.689109in}{2.813299in}}{\pgfqpoint{1.681296in}{2.805485in}}%
\pgfpathcurveto{\pgfqpoint{1.673482in}{2.797672in}}{\pgfqpoint{1.669092in}{2.787073in}}{\pgfqpoint{1.669092in}{2.776022in}}%
\pgfpathcurveto{\pgfqpoint{1.669092in}{2.764972in}}{\pgfqpoint{1.673482in}{2.754373in}}{\pgfqpoint{1.681296in}{2.746560in}}%
\pgfpathcurveto{\pgfqpoint{1.689109in}{2.738746in}}{\pgfqpoint{1.699708in}{2.734356in}}{\pgfqpoint{1.710758in}{2.734356in}}%
\pgfpathclose%
\pgfusepath{stroke,fill}%
\end{pgfscope}%
\begin{pgfscope}%
\pgfpathrectangle{\pgfqpoint{0.787074in}{0.548769in}}{\pgfqpoint{5.062926in}{3.102590in}}%
\pgfusepath{clip}%
\pgfsetbuttcap%
\pgfsetroundjoin%
\definecolor{currentfill}{rgb}{1.000000,0.498039,0.054902}%
\pgfsetfillcolor{currentfill}%
\pgfsetlinewidth{1.003750pt}%
\definecolor{currentstroke}{rgb}{1.000000,0.498039,0.054902}%
\pgfsetstrokecolor{currentstroke}%
\pgfsetdash{}{0pt}%
\pgfpathmoveto{\pgfqpoint{1.332458in}{2.144210in}}%
\pgfpathcurveto{\pgfqpoint{1.343508in}{2.144210in}}{\pgfqpoint{1.354107in}{2.148600in}}{\pgfqpoint{1.361920in}{2.156414in}}%
\pgfpathcurveto{\pgfqpoint{1.369734in}{2.164228in}}{\pgfqpoint{1.374124in}{2.174827in}}{\pgfqpoint{1.374124in}{2.185877in}}%
\pgfpathcurveto{\pgfqpoint{1.374124in}{2.196927in}}{\pgfqpoint{1.369734in}{2.207526in}}{\pgfqpoint{1.361920in}{2.215340in}}%
\pgfpathcurveto{\pgfqpoint{1.354107in}{2.223153in}}{\pgfqpoint{1.343508in}{2.227544in}}{\pgfqpoint{1.332458in}{2.227544in}}%
\pgfpathcurveto{\pgfqpoint{1.321407in}{2.227544in}}{\pgfqpoint{1.310808in}{2.223153in}}{\pgfqpoint{1.302995in}{2.215340in}}%
\pgfpathcurveto{\pgfqpoint{1.295181in}{2.207526in}}{\pgfqpoint{1.290791in}{2.196927in}}{\pgfqpoint{1.290791in}{2.185877in}}%
\pgfpathcurveto{\pgfqpoint{1.290791in}{2.174827in}}{\pgfqpoint{1.295181in}{2.164228in}}{\pgfqpoint{1.302995in}{2.156414in}}%
\pgfpathcurveto{\pgfqpoint{1.310808in}{2.148600in}}{\pgfqpoint{1.321407in}{2.144210in}}{\pgfqpoint{1.332458in}{2.144210in}}%
\pgfpathclose%
\pgfusepath{stroke,fill}%
\end{pgfscope}%
\begin{pgfscope}%
\pgfpathrectangle{\pgfqpoint{0.787074in}{0.548769in}}{\pgfqpoint{5.062926in}{3.102590in}}%
\pgfusepath{clip}%
\pgfsetbuttcap%
\pgfsetroundjoin%
\definecolor{currentfill}{rgb}{1.000000,0.498039,0.054902}%
\pgfsetfillcolor{currentfill}%
\pgfsetlinewidth{1.003750pt}%
\definecolor{currentstroke}{rgb}{1.000000,0.498039,0.054902}%
\pgfsetstrokecolor{currentstroke}%
\pgfsetdash{}{0pt}%
\pgfpathmoveto{\pgfqpoint{1.080257in}{2.799180in}}%
\pgfpathcurveto{\pgfqpoint{1.091307in}{2.799180in}}{\pgfqpoint{1.101906in}{2.803570in}}{\pgfqpoint{1.109720in}{2.811384in}}%
\pgfpathcurveto{\pgfqpoint{1.117533in}{2.819197in}}{\pgfqpoint{1.121924in}{2.829796in}}{\pgfqpoint{1.121924in}{2.840846in}}%
\pgfpathcurveto{\pgfqpoint{1.121924in}{2.851897in}}{\pgfqpoint{1.117533in}{2.862496in}}{\pgfqpoint{1.109720in}{2.870309in}}%
\pgfpathcurveto{\pgfqpoint{1.101906in}{2.878123in}}{\pgfqpoint{1.091307in}{2.882513in}}{\pgfqpoint{1.080257in}{2.882513in}}%
\pgfpathcurveto{\pgfqpoint{1.069207in}{2.882513in}}{\pgfqpoint{1.058608in}{2.878123in}}{\pgfqpoint{1.050794in}{2.870309in}}%
\pgfpathcurveto{\pgfqpoint{1.042981in}{2.862496in}}{\pgfqpoint{1.038590in}{2.851897in}}{\pgfqpoint{1.038590in}{2.840846in}}%
\pgfpathcurveto{\pgfqpoint{1.038590in}{2.829796in}}{\pgfqpoint{1.042981in}{2.819197in}}{\pgfqpoint{1.050794in}{2.811384in}}%
\pgfpathcurveto{\pgfqpoint{1.058608in}{2.803570in}}{\pgfqpoint{1.069207in}{2.799180in}}{\pgfqpoint{1.080257in}{2.799180in}}%
\pgfpathclose%
\pgfusepath{stroke,fill}%
\end{pgfscope}%
\begin{pgfscope}%
\pgfpathrectangle{\pgfqpoint{0.787074in}{0.548769in}}{\pgfqpoint{5.062926in}{3.102590in}}%
\pgfusepath{clip}%
\pgfsetbuttcap%
\pgfsetroundjoin%
\definecolor{currentfill}{rgb}{1.000000,0.498039,0.054902}%
\pgfsetfillcolor{currentfill}%
\pgfsetlinewidth{1.003750pt}%
\definecolor{currentstroke}{rgb}{1.000000,0.498039,0.054902}%
\pgfsetstrokecolor{currentstroke}%
\pgfsetdash{}{0pt}%
\pgfpathmoveto{\pgfqpoint{1.458558in}{2.526596in}}%
\pgfpathcurveto{\pgfqpoint{1.469608in}{2.526596in}}{\pgfqpoint{1.480207in}{2.530987in}}{\pgfqpoint{1.488021in}{2.538800in}}%
\pgfpathcurveto{\pgfqpoint{1.495834in}{2.546614in}}{\pgfqpoint{1.500224in}{2.557213in}}{\pgfqpoint{1.500224in}{2.568263in}}%
\pgfpathcurveto{\pgfqpoint{1.500224in}{2.579313in}}{\pgfqpoint{1.495834in}{2.589912in}}{\pgfqpoint{1.488021in}{2.597726in}}%
\pgfpathcurveto{\pgfqpoint{1.480207in}{2.605540in}}{\pgfqpoint{1.469608in}{2.609930in}}{\pgfqpoint{1.458558in}{2.609930in}}%
\pgfpathcurveto{\pgfqpoint{1.447508in}{2.609930in}}{\pgfqpoint{1.436909in}{2.605540in}}{\pgfqpoint{1.429095in}{2.597726in}}%
\pgfpathcurveto{\pgfqpoint{1.421281in}{2.589912in}}{\pgfqpoint{1.416891in}{2.579313in}}{\pgfqpoint{1.416891in}{2.568263in}}%
\pgfpathcurveto{\pgfqpoint{1.416891in}{2.557213in}}{\pgfqpoint{1.421281in}{2.546614in}}{\pgfqpoint{1.429095in}{2.538800in}}%
\pgfpathcurveto{\pgfqpoint{1.436909in}{2.530987in}}{\pgfqpoint{1.447508in}{2.526596in}}{\pgfqpoint{1.458558in}{2.526596in}}%
\pgfpathclose%
\pgfusepath{stroke,fill}%
\end{pgfscope}%
\begin{pgfscope}%
\pgfpathrectangle{\pgfqpoint{0.787074in}{0.548769in}}{\pgfqpoint{5.062926in}{3.102590in}}%
\pgfusepath{clip}%
\pgfsetbuttcap%
\pgfsetroundjoin%
\definecolor{currentfill}{rgb}{1.000000,0.498039,0.054902}%
\pgfsetfillcolor{currentfill}%
\pgfsetlinewidth{1.003750pt}%
\definecolor{currentstroke}{rgb}{1.000000,0.498039,0.054902}%
\pgfsetstrokecolor{currentstroke}%
\pgfsetdash{}{0pt}%
\pgfpathmoveto{\pgfqpoint{1.269407in}{2.883497in}}%
\pgfpathcurveto{\pgfqpoint{1.280458in}{2.883497in}}{\pgfqpoint{1.291057in}{2.887888in}}{\pgfqpoint{1.298870in}{2.895701in}}%
\pgfpathcurveto{\pgfqpoint{1.306684in}{2.903515in}}{\pgfqpoint{1.311074in}{2.914114in}}{\pgfqpoint{1.311074in}{2.925164in}}%
\pgfpathcurveto{\pgfqpoint{1.311074in}{2.936214in}}{\pgfqpoint{1.306684in}{2.946813in}}{\pgfqpoint{1.298870in}{2.954627in}}%
\pgfpathcurveto{\pgfqpoint{1.291057in}{2.962440in}}{\pgfqpoint{1.280458in}{2.966831in}}{\pgfqpoint{1.269407in}{2.966831in}}%
\pgfpathcurveto{\pgfqpoint{1.258357in}{2.966831in}}{\pgfqpoint{1.247758in}{2.962440in}}{\pgfqpoint{1.239945in}{2.954627in}}%
\pgfpathcurveto{\pgfqpoint{1.232131in}{2.946813in}}{\pgfqpoint{1.227741in}{2.936214in}}{\pgfqpoint{1.227741in}{2.925164in}}%
\pgfpathcurveto{\pgfqpoint{1.227741in}{2.914114in}}{\pgfqpoint{1.232131in}{2.903515in}}{\pgfqpoint{1.239945in}{2.895701in}}%
\pgfpathcurveto{\pgfqpoint{1.247758in}{2.887888in}}{\pgfqpoint{1.258357in}{2.883497in}}{\pgfqpoint{1.269407in}{2.883497in}}%
\pgfpathclose%
\pgfusepath{stroke,fill}%
\end{pgfscope}%
\begin{pgfscope}%
\pgfpathrectangle{\pgfqpoint{0.787074in}{0.548769in}}{\pgfqpoint{5.062926in}{3.102590in}}%
\pgfusepath{clip}%
\pgfsetbuttcap%
\pgfsetroundjoin%
\definecolor{currentfill}{rgb}{1.000000,0.498039,0.054902}%
\pgfsetfillcolor{currentfill}%
\pgfsetlinewidth{1.003750pt}%
\definecolor{currentstroke}{rgb}{1.000000,0.498039,0.054902}%
\pgfsetstrokecolor{currentstroke}%
\pgfsetdash{}{0pt}%
\pgfpathmoveto{\pgfqpoint{1.962959in}{2.042862in}}%
\pgfpathcurveto{\pgfqpoint{1.974009in}{2.042862in}}{\pgfqpoint{1.984608in}{2.047252in}}{\pgfqpoint{1.992422in}{2.055065in}}%
\pgfpathcurveto{\pgfqpoint{2.000235in}{2.062879in}}{\pgfqpoint{2.004626in}{2.073478in}}{\pgfqpoint{2.004626in}{2.084528in}}%
\pgfpathcurveto{\pgfqpoint{2.004626in}{2.095578in}}{\pgfqpoint{2.000235in}{2.106177in}}{\pgfqpoint{1.992422in}{2.113991in}}%
\pgfpathcurveto{\pgfqpoint{1.984608in}{2.121805in}}{\pgfqpoint{1.974009in}{2.126195in}}{\pgfqpoint{1.962959in}{2.126195in}}%
\pgfpathcurveto{\pgfqpoint{1.951909in}{2.126195in}}{\pgfqpoint{1.941310in}{2.121805in}}{\pgfqpoint{1.933496in}{2.113991in}}%
\pgfpathcurveto{\pgfqpoint{1.925683in}{2.106177in}}{\pgfqpoint{1.921292in}{2.095578in}}{\pgfqpoint{1.921292in}{2.084528in}}%
\pgfpathcurveto{\pgfqpoint{1.921292in}{2.073478in}}{\pgfqpoint{1.925683in}{2.062879in}}{\pgfqpoint{1.933496in}{2.055065in}}%
\pgfpathcurveto{\pgfqpoint{1.941310in}{2.047252in}}{\pgfqpoint{1.951909in}{2.042862in}}{\pgfqpoint{1.962959in}{2.042862in}}%
\pgfpathclose%
\pgfusepath{stroke,fill}%
\end{pgfscope}%
\begin{pgfscope}%
\pgfpathrectangle{\pgfqpoint{0.787074in}{0.548769in}}{\pgfqpoint{5.062926in}{3.102590in}}%
\pgfusepath{clip}%
\pgfsetbuttcap%
\pgfsetroundjoin%
\definecolor{currentfill}{rgb}{1.000000,0.498039,0.054902}%
\pgfsetfillcolor{currentfill}%
\pgfsetlinewidth{1.003750pt}%
\definecolor{currentstroke}{rgb}{1.000000,0.498039,0.054902}%
\pgfsetstrokecolor{currentstroke}%
\pgfsetdash{}{0pt}%
\pgfpathmoveto{\pgfqpoint{1.395508in}{2.760534in}}%
\pgfpathcurveto{\pgfqpoint{1.406558in}{2.760534in}}{\pgfqpoint{1.417157in}{2.764924in}}{\pgfqpoint{1.424970in}{2.772738in}}%
\pgfpathcurveto{\pgfqpoint{1.432784in}{2.780552in}}{\pgfqpoint{1.437174in}{2.791151in}}{\pgfqpoint{1.437174in}{2.802201in}}%
\pgfpathcurveto{\pgfqpoint{1.437174in}{2.813251in}}{\pgfqpoint{1.432784in}{2.823850in}}{\pgfqpoint{1.424970in}{2.831664in}}%
\pgfpathcurveto{\pgfqpoint{1.417157in}{2.839477in}}{\pgfqpoint{1.406558in}{2.843868in}}{\pgfqpoint{1.395508in}{2.843868in}}%
\pgfpathcurveto{\pgfqpoint{1.384458in}{2.843868in}}{\pgfqpoint{1.373859in}{2.839477in}}{\pgfqpoint{1.366045in}{2.831664in}}%
\pgfpathcurveto{\pgfqpoint{1.358231in}{2.823850in}}{\pgfqpoint{1.353841in}{2.813251in}}{\pgfqpoint{1.353841in}{2.802201in}}%
\pgfpathcurveto{\pgfqpoint{1.353841in}{2.791151in}}{\pgfqpoint{1.358231in}{2.780552in}}{\pgfqpoint{1.366045in}{2.772738in}}%
\pgfpathcurveto{\pgfqpoint{1.373859in}{2.764924in}}{\pgfqpoint{1.384458in}{2.760534in}}{\pgfqpoint{1.395508in}{2.760534in}}%
\pgfpathclose%
\pgfusepath{stroke,fill}%
\end{pgfscope}%
\begin{pgfscope}%
\pgfpathrectangle{\pgfqpoint{0.787074in}{0.548769in}}{\pgfqpoint{5.062926in}{3.102590in}}%
\pgfusepath{clip}%
\pgfsetbuttcap%
\pgfsetroundjoin%
\definecolor{currentfill}{rgb}{0.121569,0.466667,0.705882}%
\pgfsetfillcolor{currentfill}%
\pgfsetlinewidth{1.003750pt}%
\definecolor{currentstroke}{rgb}{0.121569,0.466667,0.705882}%
\pgfsetstrokecolor{currentstroke}%
\pgfsetdash{}{0pt}%
\pgfpathmoveto{\pgfqpoint{1.080257in}{0.648129in}}%
\pgfpathcurveto{\pgfqpoint{1.091307in}{0.648129in}}{\pgfqpoint{1.101906in}{0.652519in}}{\pgfqpoint{1.109720in}{0.660333in}}%
\pgfpathcurveto{\pgfqpoint{1.117533in}{0.668146in}}{\pgfqpoint{1.121924in}{0.678745in}}{\pgfqpoint{1.121924in}{0.689796in}}%
\pgfpathcurveto{\pgfqpoint{1.121924in}{0.700846in}}{\pgfqpoint{1.117533in}{0.711445in}}{\pgfqpoint{1.109720in}{0.719258in}}%
\pgfpathcurveto{\pgfqpoint{1.101906in}{0.727072in}}{\pgfqpoint{1.091307in}{0.731462in}}{\pgfqpoint{1.080257in}{0.731462in}}%
\pgfpathcurveto{\pgfqpoint{1.069207in}{0.731462in}}{\pgfqpoint{1.058608in}{0.727072in}}{\pgfqpoint{1.050794in}{0.719258in}}%
\pgfpathcurveto{\pgfqpoint{1.042981in}{0.711445in}}{\pgfqpoint{1.038590in}{0.700846in}}{\pgfqpoint{1.038590in}{0.689796in}}%
\pgfpathcurveto{\pgfqpoint{1.038590in}{0.678745in}}{\pgfqpoint{1.042981in}{0.668146in}}{\pgfqpoint{1.050794in}{0.660333in}}%
\pgfpathcurveto{\pgfqpoint{1.058608in}{0.652519in}}{\pgfqpoint{1.069207in}{0.648129in}}{\pgfqpoint{1.080257in}{0.648129in}}%
\pgfpathclose%
\pgfusepath{stroke,fill}%
\end{pgfscope}%
\begin{pgfscope}%
\pgfpathrectangle{\pgfqpoint{0.787074in}{0.548769in}}{\pgfqpoint{5.062926in}{3.102590in}}%
\pgfusepath{clip}%
\pgfsetbuttcap%
\pgfsetroundjoin%
\definecolor{currentfill}{rgb}{0.121569,0.466667,0.705882}%
\pgfsetfillcolor{currentfill}%
\pgfsetlinewidth{1.003750pt}%
\definecolor{currentstroke}{rgb}{0.121569,0.466667,0.705882}%
\pgfsetstrokecolor{currentstroke}%
\pgfsetdash{}{0pt}%
\pgfpathmoveto{\pgfqpoint{1.395508in}{0.658321in}}%
\pgfpathcurveto{\pgfqpoint{1.406558in}{0.658321in}}{\pgfqpoint{1.417157in}{0.662711in}}{\pgfqpoint{1.424970in}{0.670525in}}%
\pgfpathcurveto{\pgfqpoint{1.432784in}{0.678338in}}{\pgfqpoint{1.437174in}{0.688937in}}{\pgfqpoint{1.437174in}{0.699987in}}%
\pgfpathcurveto{\pgfqpoint{1.437174in}{0.711038in}}{\pgfqpoint{1.432784in}{0.721637in}}{\pgfqpoint{1.424970in}{0.729450in}}%
\pgfpathcurveto{\pgfqpoint{1.417157in}{0.737264in}}{\pgfqpoint{1.406558in}{0.741654in}}{\pgfqpoint{1.395508in}{0.741654in}}%
\pgfpathcurveto{\pgfqpoint{1.384458in}{0.741654in}}{\pgfqpoint{1.373859in}{0.737264in}}{\pgfqpoint{1.366045in}{0.729450in}}%
\pgfpathcurveto{\pgfqpoint{1.358231in}{0.721637in}}{\pgfqpoint{1.353841in}{0.711038in}}{\pgfqpoint{1.353841in}{0.699987in}}%
\pgfpathcurveto{\pgfqpoint{1.353841in}{0.688937in}}{\pgfqpoint{1.358231in}{0.678338in}}{\pgfqpoint{1.366045in}{0.670525in}}%
\pgfpathcurveto{\pgfqpoint{1.373859in}{0.662711in}}{\pgfqpoint{1.384458in}{0.658321in}}{\pgfqpoint{1.395508in}{0.658321in}}%
\pgfpathclose%
\pgfusepath{stroke,fill}%
\end{pgfscope}%
\begin{pgfscope}%
\pgfpathrectangle{\pgfqpoint{0.787074in}{0.548769in}}{\pgfqpoint{5.062926in}{3.102590in}}%
\pgfusepath{clip}%
\pgfsetbuttcap%
\pgfsetroundjoin%
\definecolor{currentfill}{rgb}{0.121569,0.466667,0.705882}%
\pgfsetfillcolor{currentfill}%
\pgfsetlinewidth{1.003750pt}%
\definecolor{currentstroke}{rgb}{0.121569,0.466667,0.705882}%
\pgfsetstrokecolor{currentstroke}%
\pgfsetdash{}{0pt}%
\pgfpathmoveto{\pgfqpoint{1.269407in}{1.208140in}}%
\pgfpathcurveto{\pgfqpoint{1.280458in}{1.208140in}}{\pgfqpoint{1.291057in}{1.212530in}}{\pgfqpoint{1.298870in}{1.220344in}}%
\pgfpathcurveto{\pgfqpoint{1.306684in}{1.228157in}}{\pgfqpoint{1.311074in}{1.238756in}}{\pgfqpoint{1.311074in}{1.249807in}}%
\pgfpathcurveto{\pgfqpoint{1.311074in}{1.260857in}}{\pgfqpoint{1.306684in}{1.271456in}}{\pgfqpoint{1.298870in}{1.279269in}}%
\pgfpathcurveto{\pgfqpoint{1.291057in}{1.287083in}}{\pgfqpoint{1.280458in}{1.291473in}}{\pgfqpoint{1.269407in}{1.291473in}}%
\pgfpathcurveto{\pgfqpoint{1.258357in}{1.291473in}}{\pgfqpoint{1.247758in}{1.287083in}}{\pgfqpoint{1.239945in}{1.279269in}}%
\pgfpathcurveto{\pgfqpoint{1.232131in}{1.271456in}}{\pgfqpoint{1.227741in}{1.260857in}}{\pgfqpoint{1.227741in}{1.249807in}}%
\pgfpathcurveto{\pgfqpoint{1.227741in}{1.238756in}}{\pgfqpoint{1.232131in}{1.228157in}}{\pgfqpoint{1.239945in}{1.220344in}}%
\pgfpathcurveto{\pgfqpoint{1.247758in}{1.212530in}}{\pgfqpoint{1.258357in}{1.208140in}}{\pgfqpoint{1.269407in}{1.208140in}}%
\pgfpathclose%
\pgfusepath{stroke,fill}%
\end{pgfscope}%
\begin{pgfscope}%
\pgfpathrectangle{\pgfqpoint{0.787074in}{0.548769in}}{\pgfqpoint{5.062926in}{3.102590in}}%
\pgfusepath{clip}%
\pgfsetbuttcap%
\pgfsetroundjoin%
\definecolor{currentfill}{rgb}{1.000000,0.498039,0.054902}%
\pgfsetfillcolor{currentfill}%
\pgfsetlinewidth{1.003750pt}%
\definecolor{currentstroke}{rgb}{1.000000,0.498039,0.054902}%
\pgfsetstrokecolor{currentstroke}%
\pgfsetdash{}{0pt}%
\pgfpathmoveto{\pgfqpoint{1.332458in}{2.220680in}}%
\pgfpathcurveto{\pgfqpoint{1.343508in}{2.220680in}}{\pgfqpoint{1.354107in}{2.225070in}}{\pgfqpoint{1.361920in}{2.232884in}}%
\pgfpathcurveto{\pgfqpoint{1.369734in}{2.240698in}}{\pgfqpoint{1.374124in}{2.251297in}}{\pgfqpoint{1.374124in}{2.262347in}}%
\pgfpathcurveto{\pgfqpoint{1.374124in}{2.273397in}}{\pgfqpoint{1.369734in}{2.283996in}}{\pgfqpoint{1.361920in}{2.291810in}}%
\pgfpathcurveto{\pgfqpoint{1.354107in}{2.299623in}}{\pgfqpoint{1.343508in}{2.304013in}}{\pgfqpoint{1.332458in}{2.304013in}}%
\pgfpathcurveto{\pgfqpoint{1.321407in}{2.304013in}}{\pgfqpoint{1.310808in}{2.299623in}}{\pgfqpoint{1.302995in}{2.291810in}}%
\pgfpathcurveto{\pgfqpoint{1.295181in}{2.283996in}}{\pgfqpoint{1.290791in}{2.273397in}}{\pgfqpoint{1.290791in}{2.262347in}}%
\pgfpathcurveto{\pgfqpoint{1.290791in}{2.251297in}}{\pgfqpoint{1.295181in}{2.240698in}}{\pgfqpoint{1.302995in}{2.232884in}}%
\pgfpathcurveto{\pgfqpoint{1.310808in}{2.225070in}}{\pgfqpoint{1.321407in}{2.220680in}}{\pgfqpoint{1.332458in}{2.220680in}}%
\pgfpathclose%
\pgfusepath{stroke,fill}%
\end{pgfscope}%
\begin{pgfscope}%
\pgfpathrectangle{\pgfqpoint{0.787074in}{0.548769in}}{\pgfqpoint{5.062926in}{3.102590in}}%
\pgfusepath{clip}%
\pgfsetbuttcap%
\pgfsetroundjoin%
\definecolor{currentfill}{rgb}{0.121569,0.466667,0.705882}%
\pgfsetfillcolor{currentfill}%
\pgfsetlinewidth{1.003750pt}%
\definecolor{currentstroke}{rgb}{0.121569,0.466667,0.705882}%
\pgfsetstrokecolor{currentstroke}%
\pgfsetdash{}{0pt}%
\pgfpathmoveto{\pgfqpoint{1.332458in}{0.648130in}}%
\pgfpathcurveto{\pgfqpoint{1.343508in}{0.648130in}}{\pgfqpoint{1.354107in}{0.652520in}}{\pgfqpoint{1.361920in}{0.660334in}}%
\pgfpathcurveto{\pgfqpoint{1.369734in}{0.668148in}}{\pgfqpoint{1.374124in}{0.678747in}}{\pgfqpoint{1.374124in}{0.689797in}}%
\pgfpathcurveto{\pgfqpoint{1.374124in}{0.700847in}}{\pgfqpoint{1.369734in}{0.711446in}}{\pgfqpoint{1.361920in}{0.719260in}}%
\pgfpathcurveto{\pgfqpoint{1.354107in}{0.727073in}}{\pgfqpoint{1.343508in}{0.731464in}}{\pgfqpoint{1.332458in}{0.731464in}}%
\pgfpathcurveto{\pgfqpoint{1.321407in}{0.731464in}}{\pgfqpoint{1.310808in}{0.727073in}}{\pgfqpoint{1.302995in}{0.719260in}}%
\pgfpathcurveto{\pgfqpoint{1.295181in}{0.711446in}}{\pgfqpoint{1.290791in}{0.700847in}}{\pgfqpoint{1.290791in}{0.689797in}}%
\pgfpathcurveto{\pgfqpoint{1.290791in}{0.678747in}}{\pgfqpoint{1.295181in}{0.668148in}}{\pgfqpoint{1.302995in}{0.660334in}}%
\pgfpathcurveto{\pgfqpoint{1.310808in}{0.652520in}}{\pgfqpoint{1.321407in}{0.648130in}}{\pgfqpoint{1.332458in}{0.648130in}}%
\pgfpathclose%
\pgfusepath{stroke,fill}%
\end{pgfscope}%
\begin{pgfscope}%
\pgfpathrectangle{\pgfqpoint{0.787074in}{0.548769in}}{\pgfqpoint{5.062926in}{3.102590in}}%
\pgfusepath{clip}%
\pgfsetbuttcap%
\pgfsetroundjoin%
\definecolor{currentfill}{rgb}{0.121569,0.466667,0.705882}%
\pgfsetfillcolor{currentfill}%
\pgfsetlinewidth{1.003750pt}%
\definecolor{currentstroke}{rgb}{0.121569,0.466667,0.705882}%
\pgfsetstrokecolor{currentstroke}%
\pgfsetdash{}{0pt}%
\pgfpathmoveto{\pgfqpoint{1.080257in}{0.648134in}}%
\pgfpathcurveto{\pgfqpoint{1.091307in}{0.648134in}}{\pgfqpoint{1.101906in}{0.652524in}}{\pgfqpoint{1.109720in}{0.660338in}}%
\pgfpathcurveto{\pgfqpoint{1.117533in}{0.668151in}}{\pgfqpoint{1.121924in}{0.678750in}}{\pgfqpoint{1.121924in}{0.689800in}}%
\pgfpathcurveto{\pgfqpoint{1.121924in}{0.700850in}}{\pgfqpoint{1.117533in}{0.711450in}}{\pgfqpoint{1.109720in}{0.719263in}}%
\pgfpathcurveto{\pgfqpoint{1.101906in}{0.727077in}}{\pgfqpoint{1.091307in}{0.731467in}}{\pgfqpoint{1.080257in}{0.731467in}}%
\pgfpathcurveto{\pgfqpoint{1.069207in}{0.731467in}}{\pgfqpoint{1.058608in}{0.727077in}}{\pgfqpoint{1.050794in}{0.719263in}}%
\pgfpathcurveto{\pgfqpoint{1.042981in}{0.711450in}}{\pgfqpoint{1.038590in}{0.700850in}}{\pgfqpoint{1.038590in}{0.689800in}}%
\pgfpathcurveto{\pgfqpoint{1.038590in}{0.678750in}}{\pgfqpoint{1.042981in}{0.668151in}}{\pgfqpoint{1.050794in}{0.660338in}}%
\pgfpathcurveto{\pgfqpoint{1.058608in}{0.652524in}}{\pgfqpoint{1.069207in}{0.648134in}}{\pgfqpoint{1.080257in}{0.648134in}}%
\pgfpathclose%
\pgfusepath{stroke,fill}%
\end{pgfscope}%
\begin{pgfscope}%
\pgfpathrectangle{\pgfqpoint{0.787074in}{0.548769in}}{\pgfqpoint{5.062926in}{3.102590in}}%
\pgfusepath{clip}%
\pgfsetbuttcap%
\pgfsetroundjoin%
\definecolor{currentfill}{rgb}{1.000000,0.498039,0.054902}%
\pgfsetfillcolor{currentfill}%
\pgfsetlinewidth{1.003750pt}%
\definecolor{currentstroke}{rgb}{1.000000,0.498039,0.054902}%
\pgfsetstrokecolor{currentstroke}%
\pgfsetdash{}{0pt}%
\pgfpathmoveto{\pgfqpoint{1.017207in}{3.468665in}}%
\pgfpathcurveto{\pgfqpoint{1.028257in}{3.468665in}}{\pgfqpoint{1.038856in}{3.473055in}}{\pgfqpoint{1.046670in}{3.480869in}}%
\pgfpathcurveto{\pgfqpoint{1.054483in}{3.488683in}}{\pgfqpoint{1.058874in}{3.499282in}}{\pgfqpoint{1.058874in}{3.510332in}}%
\pgfpathcurveto{\pgfqpoint{1.058874in}{3.521382in}}{\pgfqpoint{1.054483in}{3.531981in}}{\pgfqpoint{1.046670in}{3.539795in}}%
\pgfpathcurveto{\pgfqpoint{1.038856in}{3.547608in}}{\pgfqpoint{1.028257in}{3.551998in}}{\pgfqpoint{1.017207in}{3.551998in}}%
\pgfpathcurveto{\pgfqpoint{1.006157in}{3.551998in}}{\pgfqpoint{0.995558in}{3.547608in}}{\pgfqpoint{0.987744in}{3.539795in}}%
\pgfpathcurveto{\pgfqpoint{0.979930in}{3.531981in}}{\pgfqpoint{0.975540in}{3.521382in}}{\pgfqpoint{0.975540in}{3.510332in}}%
\pgfpathcurveto{\pgfqpoint{0.975540in}{3.499282in}}{\pgfqpoint{0.979930in}{3.488683in}}{\pgfqpoint{0.987744in}{3.480869in}}%
\pgfpathcurveto{\pgfqpoint{0.995558in}{3.473055in}}{\pgfqpoint{1.006157in}{3.468665in}}{\pgfqpoint{1.017207in}{3.468665in}}%
\pgfpathclose%
\pgfusepath{stroke,fill}%
\end{pgfscope}%
\begin{pgfscope}%
\pgfpathrectangle{\pgfqpoint{0.787074in}{0.548769in}}{\pgfqpoint{5.062926in}{3.102590in}}%
\pgfusepath{clip}%
\pgfsetbuttcap%
\pgfsetroundjoin%
\definecolor{currentfill}{rgb}{0.121569,0.466667,0.705882}%
\pgfsetfillcolor{currentfill}%
\pgfsetlinewidth{1.003750pt}%
\definecolor{currentstroke}{rgb}{0.121569,0.466667,0.705882}%
\pgfsetstrokecolor{currentstroke}%
\pgfsetdash{}{0pt}%
\pgfpathmoveto{\pgfqpoint{1.269407in}{2.370534in}}%
\pgfpathcurveto{\pgfqpoint{1.280458in}{2.370534in}}{\pgfqpoint{1.291057in}{2.374924in}}{\pgfqpoint{1.298870in}{2.382738in}}%
\pgfpathcurveto{\pgfqpoint{1.306684in}{2.390551in}}{\pgfqpoint{1.311074in}{2.401150in}}{\pgfqpoint{1.311074in}{2.412201in}}%
\pgfpathcurveto{\pgfqpoint{1.311074in}{2.423251in}}{\pgfqpoint{1.306684in}{2.433850in}}{\pgfqpoint{1.298870in}{2.441663in}}%
\pgfpathcurveto{\pgfqpoint{1.291057in}{2.449477in}}{\pgfqpoint{1.280458in}{2.453867in}}{\pgfqpoint{1.269407in}{2.453867in}}%
\pgfpathcurveto{\pgfqpoint{1.258357in}{2.453867in}}{\pgfqpoint{1.247758in}{2.449477in}}{\pgfqpoint{1.239945in}{2.441663in}}%
\pgfpathcurveto{\pgfqpoint{1.232131in}{2.433850in}}{\pgfqpoint{1.227741in}{2.423251in}}{\pgfqpoint{1.227741in}{2.412201in}}%
\pgfpathcurveto{\pgfqpoint{1.227741in}{2.401150in}}{\pgfqpoint{1.232131in}{2.390551in}}{\pgfqpoint{1.239945in}{2.382738in}}%
\pgfpathcurveto{\pgfqpoint{1.247758in}{2.374924in}}{\pgfqpoint{1.258357in}{2.370534in}}{\pgfqpoint{1.269407in}{2.370534in}}%
\pgfpathclose%
\pgfusepath{stroke,fill}%
\end{pgfscope}%
\begin{pgfscope}%
\pgfpathrectangle{\pgfqpoint{0.787074in}{0.548769in}}{\pgfqpoint{5.062926in}{3.102590in}}%
\pgfusepath{clip}%
\pgfsetbuttcap%
\pgfsetroundjoin%
\definecolor{currentfill}{rgb}{1.000000,0.498039,0.054902}%
\pgfsetfillcolor{currentfill}%
\pgfsetlinewidth{1.003750pt}%
\definecolor{currentstroke}{rgb}{1.000000,0.498039,0.054902}%
\pgfsetstrokecolor{currentstroke}%
\pgfsetdash{}{0pt}%
\pgfpathmoveto{\pgfqpoint{3.034811in}{2.417203in}}%
\pgfpathcurveto{\pgfqpoint{3.045861in}{2.417203in}}{\pgfqpoint{3.056460in}{2.421593in}}{\pgfqpoint{3.064274in}{2.429407in}}%
\pgfpathcurveto{\pgfqpoint{3.072088in}{2.437220in}}{\pgfqpoint{3.076478in}{2.447819in}}{\pgfqpoint{3.076478in}{2.458870in}}%
\pgfpathcurveto{\pgfqpoint{3.076478in}{2.469920in}}{\pgfqpoint{3.072088in}{2.480519in}}{\pgfqpoint{3.064274in}{2.488332in}}%
\pgfpathcurveto{\pgfqpoint{3.056460in}{2.496146in}}{\pgfqpoint{3.045861in}{2.500536in}}{\pgfqpoint{3.034811in}{2.500536in}}%
\pgfpathcurveto{\pgfqpoint{3.023761in}{2.500536in}}{\pgfqpoint{3.013162in}{2.496146in}}{\pgfqpoint{3.005349in}{2.488332in}}%
\pgfpathcurveto{\pgfqpoint{2.997535in}{2.480519in}}{\pgfqpoint{2.993145in}{2.469920in}}{\pgfqpoint{2.993145in}{2.458870in}}%
\pgfpathcurveto{\pgfqpoint{2.993145in}{2.447819in}}{\pgfqpoint{2.997535in}{2.437220in}}{\pgfqpoint{3.005349in}{2.429407in}}%
\pgfpathcurveto{\pgfqpoint{3.013162in}{2.421593in}}{\pgfqpoint{3.023761in}{2.417203in}}{\pgfqpoint{3.034811in}{2.417203in}}%
\pgfpathclose%
\pgfusepath{stroke,fill}%
\end{pgfscope}%
\begin{pgfscope}%
\pgfpathrectangle{\pgfqpoint{0.787074in}{0.548769in}}{\pgfqpoint{5.062926in}{3.102590in}}%
\pgfusepath{clip}%
\pgfsetbuttcap%
\pgfsetroundjoin%
\definecolor{currentfill}{rgb}{1.000000,0.498039,0.054902}%
\pgfsetfillcolor{currentfill}%
\pgfsetlinewidth{1.003750pt}%
\definecolor{currentstroke}{rgb}{1.000000,0.498039,0.054902}%
\pgfsetstrokecolor{currentstroke}%
\pgfsetdash{}{0pt}%
\pgfpathmoveto{\pgfqpoint{1.836859in}{2.967508in}}%
\pgfpathcurveto{\pgfqpoint{1.847909in}{2.967508in}}{\pgfqpoint{1.858508in}{2.971899in}}{\pgfqpoint{1.866321in}{2.979712in}}%
\pgfpathcurveto{\pgfqpoint{1.874135in}{2.987526in}}{\pgfqpoint{1.878525in}{2.998125in}}{\pgfqpoint{1.878525in}{3.009175in}}%
\pgfpathcurveto{\pgfqpoint{1.878525in}{3.020225in}}{\pgfqpoint{1.874135in}{3.030824in}}{\pgfqpoint{1.866321in}{3.038638in}}%
\pgfpathcurveto{\pgfqpoint{1.858508in}{3.046451in}}{\pgfqpoint{1.847909in}{3.050842in}}{\pgfqpoint{1.836859in}{3.050842in}}%
\pgfpathcurveto{\pgfqpoint{1.825809in}{3.050842in}}{\pgfqpoint{1.815209in}{3.046451in}}{\pgfqpoint{1.807396in}{3.038638in}}%
\pgfpathcurveto{\pgfqpoint{1.799582in}{3.030824in}}{\pgfqpoint{1.795192in}{3.020225in}}{\pgfqpoint{1.795192in}{3.009175in}}%
\pgfpathcurveto{\pgfqpoint{1.795192in}{2.998125in}}{\pgfqpoint{1.799582in}{2.987526in}}{\pgfqpoint{1.807396in}{2.979712in}}%
\pgfpathcurveto{\pgfqpoint{1.815209in}{2.971899in}}{\pgfqpoint{1.825809in}{2.967508in}}{\pgfqpoint{1.836859in}{2.967508in}}%
\pgfpathclose%
\pgfusepath{stroke,fill}%
\end{pgfscope}%
\begin{pgfscope}%
\pgfpathrectangle{\pgfqpoint{0.787074in}{0.548769in}}{\pgfqpoint{5.062926in}{3.102590in}}%
\pgfusepath{clip}%
\pgfsetbuttcap%
\pgfsetroundjoin%
\definecolor{currentfill}{rgb}{1.000000,0.498039,0.054902}%
\pgfsetfillcolor{currentfill}%
\pgfsetlinewidth{1.003750pt}%
\definecolor{currentstroke}{rgb}{1.000000,0.498039,0.054902}%
\pgfsetstrokecolor{currentstroke}%
\pgfsetdash{}{0pt}%
\pgfpathmoveto{\pgfqpoint{1.332458in}{2.421885in}}%
\pgfpathcurveto{\pgfqpoint{1.343508in}{2.421885in}}{\pgfqpoint{1.354107in}{2.426275in}}{\pgfqpoint{1.361920in}{2.434089in}}%
\pgfpathcurveto{\pgfqpoint{1.369734in}{2.441902in}}{\pgfqpoint{1.374124in}{2.452501in}}{\pgfqpoint{1.374124in}{2.463551in}}%
\pgfpathcurveto{\pgfqpoint{1.374124in}{2.474601in}}{\pgfqpoint{1.369734in}{2.485201in}}{\pgfqpoint{1.361920in}{2.493014in}}%
\pgfpathcurveto{\pgfqpoint{1.354107in}{2.500828in}}{\pgfqpoint{1.343508in}{2.505218in}}{\pgfqpoint{1.332458in}{2.505218in}}%
\pgfpathcurveto{\pgfqpoint{1.321407in}{2.505218in}}{\pgfqpoint{1.310808in}{2.500828in}}{\pgfqpoint{1.302995in}{2.493014in}}%
\pgfpathcurveto{\pgfqpoint{1.295181in}{2.485201in}}{\pgfqpoint{1.290791in}{2.474601in}}{\pgfqpoint{1.290791in}{2.463551in}}%
\pgfpathcurveto{\pgfqpoint{1.290791in}{2.452501in}}{\pgfqpoint{1.295181in}{2.441902in}}{\pgfqpoint{1.302995in}{2.434089in}}%
\pgfpathcurveto{\pgfqpoint{1.310808in}{2.426275in}}{\pgfqpoint{1.321407in}{2.421885in}}{\pgfqpoint{1.332458in}{2.421885in}}%
\pgfpathclose%
\pgfusepath{stroke,fill}%
\end{pgfscope}%
\begin{pgfscope}%
\pgfpathrectangle{\pgfqpoint{0.787074in}{0.548769in}}{\pgfqpoint{5.062926in}{3.102590in}}%
\pgfusepath{clip}%
\pgfsetbuttcap%
\pgfsetroundjoin%
\definecolor{currentfill}{rgb}{0.121569,0.466667,0.705882}%
\pgfsetfillcolor{currentfill}%
\pgfsetlinewidth{1.003750pt}%
\definecolor{currentstroke}{rgb}{0.121569,0.466667,0.705882}%
\pgfsetstrokecolor{currentstroke}%
\pgfsetdash{}{0pt}%
\pgfpathmoveto{\pgfqpoint{3.917513in}{2.463960in}}%
\pgfpathcurveto{\pgfqpoint{3.928563in}{2.463960in}}{\pgfqpoint{3.939162in}{2.468350in}}{\pgfqpoint{3.946976in}{2.476164in}}%
\pgfpathcurveto{\pgfqpoint{3.954790in}{2.483977in}}{\pgfqpoint{3.959180in}{2.494576in}}{\pgfqpoint{3.959180in}{2.505626in}}%
\pgfpathcurveto{\pgfqpoint{3.959180in}{2.516677in}}{\pgfqpoint{3.954790in}{2.527276in}}{\pgfqpoint{3.946976in}{2.535089in}}%
\pgfpathcurveto{\pgfqpoint{3.939162in}{2.542903in}}{\pgfqpoint{3.928563in}{2.547293in}}{\pgfqpoint{3.917513in}{2.547293in}}%
\pgfpathcurveto{\pgfqpoint{3.906463in}{2.547293in}}{\pgfqpoint{3.895864in}{2.542903in}}{\pgfqpoint{3.888050in}{2.535089in}}%
\pgfpathcurveto{\pgfqpoint{3.880237in}{2.527276in}}{\pgfqpoint{3.875847in}{2.516677in}}{\pgfqpoint{3.875847in}{2.505626in}}%
\pgfpathcurveto{\pgfqpoint{3.875847in}{2.494576in}}{\pgfqpoint{3.880237in}{2.483977in}}{\pgfqpoint{3.888050in}{2.476164in}}%
\pgfpathcurveto{\pgfqpoint{3.895864in}{2.468350in}}{\pgfqpoint{3.906463in}{2.463960in}}{\pgfqpoint{3.917513in}{2.463960in}}%
\pgfpathclose%
\pgfusepath{stroke,fill}%
\end{pgfscope}%
\begin{pgfscope}%
\pgfpathrectangle{\pgfqpoint{0.787074in}{0.548769in}}{\pgfqpoint{5.062926in}{3.102590in}}%
\pgfusepath{clip}%
\pgfsetbuttcap%
\pgfsetroundjoin%
\definecolor{currentfill}{rgb}{1.000000,0.498039,0.054902}%
\pgfsetfillcolor{currentfill}%
\pgfsetlinewidth{1.003750pt}%
\definecolor{currentstroke}{rgb}{1.000000,0.498039,0.054902}%
\pgfsetstrokecolor{currentstroke}%
\pgfsetdash{}{0pt}%
\pgfpathmoveto{\pgfqpoint{3.854463in}{2.012662in}}%
\pgfpathcurveto{\pgfqpoint{3.865513in}{2.012662in}}{\pgfqpoint{3.876112in}{2.017052in}}{\pgfqpoint{3.883926in}{2.024866in}}%
\pgfpathcurveto{\pgfqpoint{3.891740in}{2.032679in}}{\pgfqpoint{3.896130in}{2.043278in}}{\pgfqpoint{3.896130in}{2.054328in}}%
\pgfpathcurveto{\pgfqpoint{3.896130in}{2.065378in}}{\pgfqpoint{3.891740in}{2.075977in}}{\pgfqpoint{3.883926in}{2.083791in}}%
\pgfpathcurveto{\pgfqpoint{3.876112in}{2.091605in}}{\pgfqpoint{3.865513in}{2.095995in}}{\pgfqpoint{3.854463in}{2.095995in}}%
\pgfpathcurveto{\pgfqpoint{3.843413in}{2.095995in}}{\pgfqpoint{3.832814in}{2.091605in}}{\pgfqpoint{3.825000in}{2.083791in}}%
\pgfpathcurveto{\pgfqpoint{3.817187in}{2.075977in}}{\pgfqpoint{3.812796in}{2.065378in}}{\pgfqpoint{3.812796in}{2.054328in}}%
\pgfpathcurveto{\pgfqpoint{3.812796in}{2.043278in}}{\pgfqpoint{3.817187in}{2.032679in}}{\pgfqpoint{3.825000in}{2.024866in}}%
\pgfpathcurveto{\pgfqpoint{3.832814in}{2.017052in}}{\pgfqpoint{3.843413in}{2.012662in}}{\pgfqpoint{3.854463in}{2.012662in}}%
\pgfpathclose%
\pgfusepath{stroke,fill}%
\end{pgfscope}%
\begin{pgfscope}%
\pgfpathrectangle{\pgfqpoint{0.787074in}{0.548769in}}{\pgfqpoint{5.062926in}{3.102590in}}%
\pgfusepath{clip}%
\pgfsetbuttcap%
\pgfsetroundjoin%
\definecolor{currentfill}{rgb}{1.000000,0.498039,0.054902}%
\pgfsetfillcolor{currentfill}%
\pgfsetlinewidth{1.003750pt}%
\definecolor{currentstroke}{rgb}{1.000000,0.498039,0.054902}%
\pgfsetstrokecolor{currentstroke}%
\pgfsetdash{}{0pt}%
\pgfpathmoveto{\pgfqpoint{1.521608in}{1.794353in}}%
\pgfpathcurveto{\pgfqpoint{1.532658in}{1.794353in}}{\pgfqpoint{1.543257in}{1.798743in}}{\pgfqpoint{1.551071in}{1.806557in}}%
\pgfpathcurveto{\pgfqpoint{1.558884in}{1.814370in}}{\pgfqpoint{1.563275in}{1.824969in}}{\pgfqpoint{1.563275in}{1.836019in}}%
\pgfpathcurveto{\pgfqpoint{1.563275in}{1.847070in}}{\pgfqpoint{1.558884in}{1.857669in}}{\pgfqpoint{1.551071in}{1.865482in}}%
\pgfpathcurveto{\pgfqpoint{1.543257in}{1.873296in}}{\pgfqpoint{1.532658in}{1.877686in}}{\pgfqpoint{1.521608in}{1.877686in}}%
\pgfpathcurveto{\pgfqpoint{1.510558in}{1.877686in}}{\pgfqpoint{1.499959in}{1.873296in}}{\pgfqpoint{1.492145in}{1.865482in}}%
\pgfpathcurveto{\pgfqpoint{1.484332in}{1.857669in}}{\pgfqpoint{1.479941in}{1.847070in}}{\pgfqpoint{1.479941in}{1.836019in}}%
\pgfpathcurveto{\pgfqpoint{1.479941in}{1.824969in}}{\pgfqpoint{1.484332in}{1.814370in}}{\pgfqpoint{1.492145in}{1.806557in}}%
\pgfpathcurveto{\pgfqpoint{1.499959in}{1.798743in}}{\pgfqpoint{1.510558in}{1.794353in}}{\pgfqpoint{1.521608in}{1.794353in}}%
\pgfpathclose%
\pgfusepath{stroke,fill}%
\end{pgfscope}%
\begin{pgfscope}%
\pgfpathrectangle{\pgfqpoint{0.787074in}{0.548769in}}{\pgfqpoint{5.062926in}{3.102590in}}%
\pgfusepath{clip}%
\pgfsetbuttcap%
\pgfsetroundjoin%
\definecolor{currentfill}{rgb}{0.121569,0.466667,0.705882}%
\pgfsetfillcolor{currentfill}%
\pgfsetlinewidth{1.003750pt}%
\definecolor{currentstroke}{rgb}{0.121569,0.466667,0.705882}%
\pgfsetstrokecolor{currentstroke}%
\pgfsetdash{}{0pt}%
\pgfpathmoveto{\pgfqpoint{3.097861in}{2.149173in}}%
\pgfpathcurveto{\pgfqpoint{3.108912in}{2.149173in}}{\pgfqpoint{3.119511in}{2.153563in}}{\pgfqpoint{3.127324in}{2.161376in}}%
\pgfpathcurveto{\pgfqpoint{3.135138in}{2.169190in}}{\pgfqpoint{3.139528in}{2.179789in}}{\pgfqpoint{3.139528in}{2.190839in}}%
\pgfpathcurveto{\pgfqpoint{3.139528in}{2.201889in}}{\pgfqpoint{3.135138in}{2.212488in}}{\pgfqpoint{3.127324in}{2.220302in}}%
\pgfpathcurveto{\pgfqpoint{3.119511in}{2.228116in}}{\pgfqpoint{3.108912in}{2.232506in}}{\pgfqpoint{3.097861in}{2.232506in}}%
\pgfpathcurveto{\pgfqpoint{3.086811in}{2.232506in}}{\pgfqpoint{3.076212in}{2.228116in}}{\pgfqpoint{3.068399in}{2.220302in}}%
\pgfpathcurveto{\pgfqpoint{3.060585in}{2.212488in}}{\pgfqpoint{3.056195in}{2.201889in}}{\pgfqpoint{3.056195in}{2.190839in}}%
\pgfpathcurveto{\pgfqpoint{3.056195in}{2.179789in}}{\pgfqpoint{3.060585in}{2.169190in}}{\pgfqpoint{3.068399in}{2.161376in}}%
\pgfpathcurveto{\pgfqpoint{3.076212in}{2.153563in}}{\pgfqpoint{3.086811in}{2.149173in}}{\pgfqpoint{3.097861in}{2.149173in}}%
\pgfpathclose%
\pgfusepath{stroke,fill}%
\end{pgfscope}%
\begin{pgfscope}%
\pgfpathrectangle{\pgfqpoint{0.787074in}{0.548769in}}{\pgfqpoint{5.062926in}{3.102590in}}%
\pgfusepath{clip}%
\pgfsetbuttcap%
\pgfsetroundjoin%
\definecolor{currentfill}{rgb}{1.000000,0.498039,0.054902}%
\pgfsetfillcolor{currentfill}%
\pgfsetlinewidth{1.003750pt}%
\definecolor{currentstroke}{rgb}{1.000000,0.498039,0.054902}%
\pgfsetstrokecolor{currentstroke}%
\pgfsetdash{}{0pt}%
\pgfpathmoveto{\pgfqpoint{2.719561in}{2.551840in}}%
\pgfpathcurveto{\pgfqpoint{2.730611in}{2.551840in}}{\pgfqpoint{2.741210in}{2.556230in}}{\pgfqpoint{2.749023in}{2.564044in}}%
\pgfpathcurveto{\pgfqpoint{2.756837in}{2.571857in}}{\pgfqpoint{2.761227in}{2.582456in}}{\pgfqpoint{2.761227in}{2.593506in}}%
\pgfpathcurveto{\pgfqpoint{2.761227in}{2.604557in}}{\pgfqpoint{2.756837in}{2.615156in}}{\pgfqpoint{2.749023in}{2.622969in}}%
\pgfpathcurveto{\pgfqpoint{2.741210in}{2.630783in}}{\pgfqpoint{2.730611in}{2.635173in}}{\pgfqpoint{2.719561in}{2.635173in}}%
\pgfpathcurveto{\pgfqpoint{2.708510in}{2.635173in}}{\pgfqpoint{2.697911in}{2.630783in}}{\pgfqpoint{2.690098in}{2.622969in}}%
\pgfpathcurveto{\pgfqpoint{2.682284in}{2.615156in}}{\pgfqpoint{2.677894in}{2.604557in}}{\pgfqpoint{2.677894in}{2.593506in}}%
\pgfpathcurveto{\pgfqpoint{2.677894in}{2.582456in}}{\pgfqpoint{2.682284in}{2.571857in}}{\pgfqpoint{2.690098in}{2.564044in}}%
\pgfpathcurveto{\pgfqpoint{2.697911in}{2.556230in}}{\pgfqpoint{2.708510in}{2.551840in}}{\pgfqpoint{2.719561in}{2.551840in}}%
\pgfpathclose%
\pgfusepath{stroke,fill}%
\end{pgfscope}%
\begin{pgfscope}%
\pgfpathrectangle{\pgfqpoint{0.787074in}{0.548769in}}{\pgfqpoint{5.062926in}{3.102590in}}%
\pgfusepath{clip}%
\pgfsetbuttcap%
\pgfsetroundjoin%
\definecolor{currentfill}{rgb}{1.000000,0.498039,0.054902}%
\pgfsetfillcolor{currentfill}%
\pgfsetlinewidth{1.003750pt}%
\definecolor{currentstroke}{rgb}{1.000000,0.498039,0.054902}%
\pgfsetstrokecolor{currentstroke}%
\pgfsetdash{}{0pt}%
\pgfpathmoveto{\pgfqpoint{1.143307in}{2.187660in}}%
\pgfpathcurveto{\pgfqpoint{1.154357in}{2.187660in}}{\pgfqpoint{1.164956in}{2.192050in}}{\pgfqpoint{1.172770in}{2.199864in}}%
\pgfpathcurveto{\pgfqpoint{1.180584in}{2.207677in}}{\pgfqpoint{1.184974in}{2.218276in}}{\pgfqpoint{1.184974in}{2.229326in}}%
\pgfpathcurveto{\pgfqpoint{1.184974in}{2.240377in}}{\pgfqpoint{1.180584in}{2.250976in}}{\pgfqpoint{1.172770in}{2.258789in}}%
\pgfpathcurveto{\pgfqpoint{1.164956in}{2.266603in}}{\pgfqpoint{1.154357in}{2.270993in}}{\pgfqpoint{1.143307in}{2.270993in}}%
\pgfpathcurveto{\pgfqpoint{1.132257in}{2.270993in}}{\pgfqpoint{1.121658in}{2.266603in}}{\pgfqpoint{1.113844in}{2.258789in}}%
\pgfpathcurveto{\pgfqpoint{1.106031in}{2.250976in}}{\pgfqpoint{1.101640in}{2.240377in}}{\pgfqpoint{1.101640in}{2.229326in}}%
\pgfpathcurveto{\pgfqpoint{1.101640in}{2.218276in}}{\pgfqpoint{1.106031in}{2.207677in}}{\pgfqpoint{1.113844in}{2.199864in}}%
\pgfpathcurveto{\pgfqpoint{1.121658in}{2.192050in}}{\pgfqpoint{1.132257in}{2.187660in}}{\pgfqpoint{1.143307in}{2.187660in}}%
\pgfpathclose%
\pgfusepath{stroke,fill}%
\end{pgfscope}%
\begin{pgfscope}%
\pgfpathrectangle{\pgfqpoint{0.787074in}{0.548769in}}{\pgfqpoint{5.062926in}{3.102590in}}%
\pgfusepath{clip}%
\pgfsetbuttcap%
\pgfsetroundjoin%
\definecolor{currentfill}{rgb}{1.000000,0.498039,0.054902}%
\pgfsetfillcolor{currentfill}%
\pgfsetlinewidth{1.003750pt}%
\definecolor{currentstroke}{rgb}{1.000000,0.498039,0.054902}%
\pgfsetstrokecolor{currentstroke}%
\pgfsetdash{}{0pt}%
\pgfpathmoveto{\pgfqpoint{2.530410in}{2.846483in}}%
\pgfpathcurveto{\pgfqpoint{2.541460in}{2.846483in}}{\pgfqpoint{2.552059in}{2.850874in}}{\pgfqpoint{2.559873in}{2.858687in}}%
\pgfpathcurveto{\pgfqpoint{2.567687in}{2.866501in}}{\pgfqpoint{2.572077in}{2.877100in}}{\pgfqpoint{2.572077in}{2.888150in}}%
\pgfpathcurveto{\pgfqpoint{2.572077in}{2.899200in}}{\pgfqpoint{2.567687in}{2.909799in}}{\pgfqpoint{2.559873in}{2.917613in}}%
\pgfpathcurveto{\pgfqpoint{2.552059in}{2.925426in}}{\pgfqpoint{2.541460in}{2.929817in}}{\pgfqpoint{2.530410in}{2.929817in}}%
\pgfpathcurveto{\pgfqpoint{2.519360in}{2.929817in}}{\pgfqpoint{2.508761in}{2.925426in}}{\pgfqpoint{2.500947in}{2.917613in}}%
\pgfpathcurveto{\pgfqpoint{2.493134in}{2.909799in}}{\pgfqpoint{2.488744in}{2.899200in}}{\pgfqpoint{2.488744in}{2.888150in}}%
\pgfpathcurveto{\pgfqpoint{2.488744in}{2.877100in}}{\pgfqpoint{2.493134in}{2.866501in}}{\pgfqpoint{2.500947in}{2.858687in}}%
\pgfpathcurveto{\pgfqpoint{2.508761in}{2.850874in}}{\pgfqpoint{2.519360in}{2.846483in}}{\pgfqpoint{2.530410in}{2.846483in}}%
\pgfpathclose%
\pgfusepath{stroke,fill}%
\end{pgfscope}%
\begin{pgfscope}%
\pgfpathrectangle{\pgfqpoint{0.787074in}{0.548769in}}{\pgfqpoint{5.062926in}{3.102590in}}%
\pgfusepath{clip}%
\pgfsetbuttcap%
\pgfsetroundjoin%
\definecolor{currentfill}{rgb}{1.000000,0.498039,0.054902}%
\pgfsetfillcolor{currentfill}%
\pgfsetlinewidth{1.003750pt}%
\definecolor{currentstroke}{rgb}{1.000000,0.498039,0.054902}%
\pgfsetstrokecolor{currentstroke}%
\pgfsetdash{}{0pt}%
\pgfpathmoveto{\pgfqpoint{2.089059in}{2.316063in}}%
\pgfpathcurveto{\pgfqpoint{2.100109in}{2.316063in}}{\pgfqpoint{2.110708in}{2.320453in}}{\pgfqpoint{2.118522in}{2.328267in}}%
\pgfpathcurveto{\pgfqpoint{2.126336in}{2.336080in}}{\pgfqpoint{2.130726in}{2.346679in}}{\pgfqpoint{2.130726in}{2.357729in}}%
\pgfpathcurveto{\pgfqpoint{2.130726in}{2.368779in}}{\pgfqpoint{2.126336in}{2.379379in}}{\pgfqpoint{2.118522in}{2.387192in}}%
\pgfpathcurveto{\pgfqpoint{2.110708in}{2.395006in}}{\pgfqpoint{2.100109in}{2.399396in}}{\pgfqpoint{2.089059in}{2.399396in}}%
\pgfpathcurveto{\pgfqpoint{2.078009in}{2.399396in}}{\pgfqpoint{2.067410in}{2.395006in}}{\pgfqpoint{2.059596in}{2.387192in}}%
\pgfpathcurveto{\pgfqpoint{2.051783in}{2.379379in}}{\pgfqpoint{2.047393in}{2.368779in}}{\pgfqpoint{2.047393in}{2.357729in}}%
\pgfpathcurveto{\pgfqpoint{2.047393in}{2.346679in}}{\pgfqpoint{2.051783in}{2.336080in}}{\pgfqpoint{2.059596in}{2.328267in}}%
\pgfpathcurveto{\pgfqpoint{2.067410in}{2.320453in}}{\pgfqpoint{2.078009in}{2.316063in}}{\pgfqpoint{2.089059in}{2.316063in}}%
\pgfpathclose%
\pgfusepath{stroke,fill}%
\end{pgfscope}%
\begin{pgfscope}%
\pgfpathrectangle{\pgfqpoint{0.787074in}{0.548769in}}{\pgfqpoint{5.062926in}{3.102590in}}%
\pgfusepath{clip}%
\pgfsetbuttcap%
\pgfsetroundjoin%
\definecolor{currentfill}{rgb}{1.000000,0.498039,0.054902}%
\pgfsetfillcolor{currentfill}%
\pgfsetlinewidth{1.003750pt}%
\definecolor{currentstroke}{rgb}{1.000000,0.498039,0.054902}%
\pgfsetstrokecolor{currentstroke}%
\pgfsetdash{}{0pt}%
\pgfpathmoveto{\pgfqpoint{2.026009in}{1.585094in}}%
\pgfpathcurveto{\pgfqpoint{2.037059in}{1.585094in}}{\pgfqpoint{2.047658in}{1.589485in}}{\pgfqpoint{2.055472in}{1.597298in}}%
\pgfpathcurveto{\pgfqpoint{2.063285in}{1.605112in}}{\pgfqpoint{2.067676in}{1.615711in}}{\pgfqpoint{2.067676in}{1.626761in}}%
\pgfpathcurveto{\pgfqpoint{2.067676in}{1.637811in}}{\pgfqpoint{2.063285in}{1.648410in}}{\pgfqpoint{2.055472in}{1.656224in}}%
\pgfpathcurveto{\pgfqpoint{2.047658in}{1.664037in}}{\pgfqpoint{2.037059in}{1.668428in}}{\pgfqpoint{2.026009in}{1.668428in}}%
\pgfpathcurveto{\pgfqpoint{2.014959in}{1.668428in}}{\pgfqpoint{2.004360in}{1.664037in}}{\pgfqpoint{1.996546in}{1.656224in}}%
\pgfpathcurveto{\pgfqpoint{1.988733in}{1.648410in}}{\pgfqpoint{1.984342in}{1.637811in}}{\pgfqpoint{1.984342in}{1.626761in}}%
\pgfpathcurveto{\pgfqpoint{1.984342in}{1.615711in}}{\pgfqpoint{1.988733in}{1.605112in}}{\pgfqpoint{1.996546in}{1.597298in}}%
\pgfpathcurveto{\pgfqpoint{2.004360in}{1.589485in}}{\pgfqpoint{2.014959in}{1.585094in}}{\pgfqpoint{2.026009in}{1.585094in}}%
\pgfpathclose%
\pgfusepath{stroke,fill}%
\end{pgfscope}%
\begin{pgfscope}%
\pgfpathrectangle{\pgfqpoint{0.787074in}{0.548769in}}{\pgfqpoint{5.062926in}{3.102590in}}%
\pgfusepath{clip}%
\pgfsetbuttcap%
\pgfsetroundjoin%
\definecolor{currentfill}{rgb}{0.121569,0.466667,0.705882}%
\pgfsetfillcolor{currentfill}%
\pgfsetlinewidth{1.003750pt}%
\definecolor{currentstroke}{rgb}{0.121569,0.466667,0.705882}%
\pgfsetstrokecolor{currentstroke}%
\pgfsetdash{}{0pt}%
\pgfpathmoveto{\pgfqpoint{2.026009in}{1.803770in}}%
\pgfpathcurveto{\pgfqpoint{2.037059in}{1.803770in}}{\pgfqpoint{2.047658in}{1.808160in}}{\pgfqpoint{2.055472in}{1.815974in}}%
\pgfpathcurveto{\pgfqpoint{2.063285in}{1.823787in}}{\pgfqpoint{2.067676in}{1.834386in}}{\pgfqpoint{2.067676in}{1.845437in}}%
\pgfpathcurveto{\pgfqpoint{2.067676in}{1.856487in}}{\pgfqpoint{2.063285in}{1.867086in}}{\pgfqpoint{2.055472in}{1.874899in}}%
\pgfpathcurveto{\pgfqpoint{2.047658in}{1.882713in}}{\pgfqpoint{2.037059in}{1.887103in}}{\pgfqpoint{2.026009in}{1.887103in}}%
\pgfpathcurveto{\pgfqpoint{2.014959in}{1.887103in}}{\pgfqpoint{2.004360in}{1.882713in}}{\pgfqpoint{1.996546in}{1.874899in}}%
\pgfpathcurveto{\pgfqpoint{1.988733in}{1.867086in}}{\pgfqpoint{1.984342in}{1.856487in}}{\pgfqpoint{1.984342in}{1.845437in}}%
\pgfpathcurveto{\pgfqpoint{1.984342in}{1.834386in}}{\pgfqpoint{1.988733in}{1.823787in}}{\pgfqpoint{1.996546in}{1.815974in}}%
\pgfpathcurveto{\pgfqpoint{2.004360in}{1.808160in}}{\pgfqpoint{2.014959in}{1.803770in}}{\pgfqpoint{2.026009in}{1.803770in}}%
\pgfpathclose%
\pgfusepath{stroke,fill}%
\end{pgfscope}%
\begin{pgfscope}%
\pgfpathrectangle{\pgfqpoint{0.787074in}{0.548769in}}{\pgfqpoint{5.062926in}{3.102590in}}%
\pgfusepath{clip}%
\pgfsetbuttcap%
\pgfsetroundjoin%
\definecolor{currentfill}{rgb}{1.000000,0.498039,0.054902}%
\pgfsetfillcolor{currentfill}%
\pgfsetlinewidth{1.003750pt}%
\definecolor{currentstroke}{rgb}{1.000000,0.498039,0.054902}%
\pgfsetstrokecolor{currentstroke}%
\pgfsetdash{}{0pt}%
\pgfpathmoveto{\pgfqpoint{1.647708in}{2.162419in}}%
\pgfpathcurveto{\pgfqpoint{1.658758in}{2.162419in}}{\pgfqpoint{1.669357in}{2.166809in}}{\pgfqpoint{1.677171in}{2.174623in}}%
\pgfpathcurveto{\pgfqpoint{1.684985in}{2.182437in}}{\pgfqpoint{1.689375in}{2.193036in}}{\pgfqpoint{1.689375in}{2.204086in}}%
\pgfpathcurveto{\pgfqpoint{1.689375in}{2.215136in}}{\pgfqpoint{1.684985in}{2.225735in}}{\pgfqpoint{1.677171in}{2.233549in}}%
\pgfpathcurveto{\pgfqpoint{1.669357in}{2.241362in}}{\pgfqpoint{1.658758in}{2.245753in}}{\pgfqpoint{1.647708in}{2.245753in}}%
\pgfpathcurveto{\pgfqpoint{1.636658in}{2.245753in}}{\pgfqpoint{1.626059in}{2.241362in}}{\pgfqpoint{1.618245in}{2.233549in}}%
\pgfpathcurveto{\pgfqpoint{1.610432in}{2.225735in}}{\pgfqpoint{1.606042in}{2.215136in}}{\pgfqpoint{1.606042in}{2.204086in}}%
\pgfpathcurveto{\pgfqpoint{1.606042in}{2.193036in}}{\pgfqpoint{1.610432in}{2.182437in}}{\pgfqpoint{1.618245in}{2.174623in}}%
\pgfpathcurveto{\pgfqpoint{1.626059in}{2.166809in}}{\pgfqpoint{1.636658in}{2.162419in}}{\pgfqpoint{1.647708in}{2.162419in}}%
\pgfpathclose%
\pgfusepath{stroke,fill}%
\end{pgfscope}%
\begin{pgfscope}%
\pgfpathrectangle{\pgfqpoint{0.787074in}{0.548769in}}{\pgfqpoint{5.062926in}{3.102590in}}%
\pgfusepath{clip}%
\pgfsetbuttcap%
\pgfsetroundjoin%
\definecolor{currentfill}{rgb}{1.000000,0.498039,0.054902}%
\pgfsetfillcolor{currentfill}%
\pgfsetlinewidth{1.003750pt}%
\definecolor{currentstroke}{rgb}{1.000000,0.498039,0.054902}%
\pgfsetstrokecolor{currentstroke}%
\pgfsetdash{}{0pt}%
\pgfpathmoveto{\pgfqpoint{5.619867in}{3.057735in}}%
\pgfpathcurveto{\pgfqpoint{5.630917in}{3.057735in}}{\pgfqpoint{5.641516in}{3.062125in}}{\pgfqpoint{5.649330in}{3.069939in}}%
\pgfpathcurveto{\pgfqpoint{5.657143in}{3.077752in}}{\pgfqpoint{5.661534in}{3.088351in}}{\pgfqpoint{5.661534in}{3.099401in}}%
\pgfpathcurveto{\pgfqpoint{5.661534in}{3.110451in}}{\pgfqpoint{5.657143in}{3.121050in}}{\pgfqpoint{5.649330in}{3.128864in}}%
\pgfpathcurveto{\pgfqpoint{5.641516in}{3.136678in}}{\pgfqpoint{5.630917in}{3.141068in}}{\pgfqpoint{5.619867in}{3.141068in}}%
\pgfpathcurveto{\pgfqpoint{5.608817in}{3.141068in}}{\pgfqpoint{5.598218in}{3.136678in}}{\pgfqpoint{5.590404in}{3.128864in}}%
\pgfpathcurveto{\pgfqpoint{5.582591in}{3.121050in}}{\pgfqpoint{5.578200in}{3.110451in}}{\pgfqpoint{5.578200in}{3.099401in}}%
\pgfpathcurveto{\pgfqpoint{5.578200in}{3.088351in}}{\pgfqpoint{5.582591in}{3.077752in}}{\pgfqpoint{5.590404in}{3.069939in}}%
\pgfpathcurveto{\pgfqpoint{5.598218in}{3.062125in}}{\pgfqpoint{5.608817in}{3.057735in}}{\pgfqpoint{5.619867in}{3.057735in}}%
\pgfpathclose%
\pgfusepath{stroke,fill}%
\end{pgfscope}%
\begin{pgfscope}%
\pgfpathrectangle{\pgfqpoint{0.787074in}{0.548769in}}{\pgfqpoint{5.062926in}{3.102590in}}%
\pgfusepath{clip}%
\pgfsetbuttcap%
\pgfsetroundjoin%
\definecolor{currentfill}{rgb}{0.121569,0.466667,0.705882}%
\pgfsetfillcolor{currentfill}%
\pgfsetlinewidth{1.003750pt}%
\definecolor{currentstroke}{rgb}{0.121569,0.466667,0.705882}%
\pgfsetstrokecolor{currentstroke}%
\pgfsetdash{}{0pt}%
\pgfpathmoveto{\pgfqpoint{2.656510in}{2.195132in}}%
\pgfpathcurveto{\pgfqpoint{2.667561in}{2.195132in}}{\pgfqpoint{2.678160in}{2.199523in}}{\pgfqpoint{2.685973in}{2.207336in}}%
\pgfpathcurveto{\pgfqpoint{2.693787in}{2.215150in}}{\pgfqpoint{2.698177in}{2.225749in}}{\pgfqpoint{2.698177in}{2.236799in}}%
\pgfpathcurveto{\pgfqpoint{2.698177in}{2.247849in}}{\pgfqpoint{2.693787in}{2.258448in}}{\pgfqpoint{2.685973in}{2.266262in}}%
\pgfpathcurveto{\pgfqpoint{2.678160in}{2.274076in}}{\pgfqpoint{2.667561in}{2.278466in}}{\pgfqpoint{2.656510in}{2.278466in}}%
\pgfpathcurveto{\pgfqpoint{2.645460in}{2.278466in}}{\pgfqpoint{2.634861in}{2.274076in}}{\pgfqpoint{2.627048in}{2.266262in}}%
\pgfpathcurveto{\pgfqpoint{2.619234in}{2.258448in}}{\pgfqpoint{2.614844in}{2.247849in}}{\pgfqpoint{2.614844in}{2.236799in}}%
\pgfpathcurveto{\pgfqpoint{2.614844in}{2.225749in}}{\pgfqpoint{2.619234in}{2.215150in}}{\pgfqpoint{2.627048in}{2.207336in}}%
\pgfpathcurveto{\pgfqpoint{2.634861in}{2.199523in}}{\pgfqpoint{2.645460in}{2.195132in}}{\pgfqpoint{2.656510in}{2.195132in}}%
\pgfpathclose%
\pgfusepath{stroke,fill}%
\end{pgfscope}%
\begin{pgfscope}%
\pgfpathrectangle{\pgfqpoint{0.787074in}{0.548769in}}{\pgfqpoint{5.062926in}{3.102590in}}%
\pgfusepath{clip}%
\pgfsetbuttcap%
\pgfsetroundjoin%
\definecolor{currentfill}{rgb}{1.000000,0.498039,0.054902}%
\pgfsetfillcolor{currentfill}%
\pgfsetlinewidth{1.003750pt}%
\definecolor{currentstroke}{rgb}{1.000000,0.498039,0.054902}%
\pgfsetstrokecolor{currentstroke}%
\pgfsetdash{}{0pt}%
\pgfpathmoveto{\pgfqpoint{1.899909in}{2.538741in}}%
\pgfpathcurveto{\pgfqpoint{1.910959in}{2.538741in}}{\pgfqpoint{1.921558in}{2.543132in}}{\pgfqpoint{1.929372in}{2.550945in}}%
\pgfpathcurveto{\pgfqpoint{1.937185in}{2.558759in}}{\pgfqpoint{1.941575in}{2.569358in}}{\pgfqpoint{1.941575in}{2.580408in}}%
\pgfpathcurveto{\pgfqpoint{1.941575in}{2.591458in}}{\pgfqpoint{1.937185in}{2.602057in}}{\pgfqpoint{1.929372in}{2.609871in}}%
\pgfpathcurveto{\pgfqpoint{1.921558in}{2.617685in}}{\pgfqpoint{1.910959in}{2.622075in}}{\pgfqpoint{1.899909in}{2.622075in}}%
\pgfpathcurveto{\pgfqpoint{1.888859in}{2.622075in}}{\pgfqpoint{1.878260in}{2.617685in}}{\pgfqpoint{1.870446in}{2.609871in}}%
\pgfpathcurveto{\pgfqpoint{1.862632in}{2.602057in}}{\pgfqpoint{1.858242in}{2.591458in}}{\pgfqpoint{1.858242in}{2.580408in}}%
\pgfpathcurveto{\pgfqpoint{1.858242in}{2.569358in}}{\pgfqpoint{1.862632in}{2.558759in}}{\pgfqpoint{1.870446in}{2.550945in}}%
\pgfpathcurveto{\pgfqpoint{1.878260in}{2.543132in}}{\pgfqpoint{1.888859in}{2.538741in}}{\pgfqpoint{1.899909in}{2.538741in}}%
\pgfpathclose%
\pgfusepath{stroke,fill}%
\end{pgfscope}%
\begin{pgfscope}%
\pgfpathrectangle{\pgfqpoint{0.787074in}{0.548769in}}{\pgfqpoint{5.062926in}{3.102590in}}%
\pgfusepath{clip}%
\pgfsetbuttcap%
\pgfsetroundjoin%
\definecolor{currentfill}{rgb}{1.000000,0.498039,0.054902}%
\pgfsetfillcolor{currentfill}%
\pgfsetlinewidth{1.003750pt}%
\definecolor{currentstroke}{rgb}{1.000000,0.498039,0.054902}%
\pgfsetstrokecolor{currentstroke}%
\pgfsetdash{}{0pt}%
\pgfpathmoveto{\pgfqpoint{2.467360in}{2.246302in}}%
\pgfpathcurveto{\pgfqpoint{2.478410in}{2.246302in}}{\pgfqpoint{2.489009in}{2.250692in}}{\pgfqpoint{2.496823in}{2.258505in}}%
\pgfpathcurveto{\pgfqpoint{2.504636in}{2.266319in}}{\pgfqpoint{2.509027in}{2.276918in}}{\pgfqpoint{2.509027in}{2.287968in}}%
\pgfpathcurveto{\pgfqpoint{2.509027in}{2.299018in}}{\pgfqpoint{2.504636in}{2.309617in}}{\pgfqpoint{2.496823in}{2.317431in}}%
\pgfpathcurveto{\pgfqpoint{2.489009in}{2.325245in}}{\pgfqpoint{2.478410in}{2.329635in}}{\pgfqpoint{2.467360in}{2.329635in}}%
\pgfpathcurveto{\pgfqpoint{2.456310in}{2.329635in}}{\pgfqpoint{2.445711in}{2.325245in}}{\pgfqpoint{2.437897in}{2.317431in}}%
\pgfpathcurveto{\pgfqpoint{2.430084in}{2.309617in}}{\pgfqpoint{2.425693in}{2.299018in}}{\pgfqpoint{2.425693in}{2.287968in}}%
\pgfpathcurveto{\pgfqpoint{2.425693in}{2.276918in}}{\pgfqpoint{2.430084in}{2.266319in}}{\pgfqpoint{2.437897in}{2.258505in}}%
\pgfpathcurveto{\pgfqpoint{2.445711in}{2.250692in}}{\pgfqpoint{2.456310in}{2.246302in}}{\pgfqpoint{2.467360in}{2.246302in}}%
\pgfpathclose%
\pgfusepath{stroke,fill}%
\end{pgfscope}%
\begin{pgfscope}%
\pgfpathrectangle{\pgfqpoint{0.787074in}{0.548769in}}{\pgfqpoint{5.062926in}{3.102590in}}%
\pgfusepath{clip}%
\pgfsetbuttcap%
\pgfsetroundjoin%
\definecolor{currentfill}{rgb}{0.121569,0.466667,0.705882}%
\pgfsetfillcolor{currentfill}%
\pgfsetlinewidth{1.003750pt}%
\definecolor{currentstroke}{rgb}{0.121569,0.466667,0.705882}%
\pgfsetstrokecolor{currentstroke}%
\pgfsetdash{}{0pt}%
\pgfpathmoveto{\pgfqpoint{2.278210in}{2.224849in}}%
\pgfpathcurveto{\pgfqpoint{2.289260in}{2.224849in}}{\pgfqpoint{2.299859in}{2.229239in}}{\pgfqpoint{2.307672in}{2.237053in}}%
\pgfpathcurveto{\pgfqpoint{2.315486in}{2.244866in}}{\pgfqpoint{2.319876in}{2.255465in}}{\pgfqpoint{2.319876in}{2.266515in}}%
\pgfpathcurveto{\pgfqpoint{2.319876in}{2.277565in}}{\pgfqpoint{2.315486in}{2.288165in}}{\pgfqpoint{2.307672in}{2.295978in}}%
\pgfpathcurveto{\pgfqpoint{2.299859in}{2.303792in}}{\pgfqpoint{2.289260in}{2.308182in}}{\pgfqpoint{2.278210in}{2.308182in}}%
\pgfpathcurveto{\pgfqpoint{2.267159in}{2.308182in}}{\pgfqpoint{2.256560in}{2.303792in}}{\pgfqpoint{2.248747in}{2.295978in}}%
\pgfpathcurveto{\pgfqpoint{2.240933in}{2.288165in}}{\pgfqpoint{2.236543in}{2.277565in}}{\pgfqpoint{2.236543in}{2.266515in}}%
\pgfpathcurveto{\pgfqpoint{2.236543in}{2.255465in}}{\pgfqpoint{2.240933in}{2.244866in}}{\pgfqpoint{2.248747in}{2.237053in}}%
\pgfpathcurveto{\pgfqpoint{2.256560in}{2.229239in}}{\pgfqpoint{2.267159in}{2.224849in}}{\pgfqpoint{2.278210in}{2.224849in}}%
\pgfpathclose%
\pgfusepath{stroke,fill}%
\end{pgfscope}%
\begin{pgfscope}%
\pgfpathrectangle{\pgfqpoint{0.787074in}{0.548769in}}{\pgfqpoint{5.062926in}{3.102590in}}%
\pgfusepath{clip}%
\pgfsetbuttcap%
\pgfsetroundjoin%
\definecolor{currentfill}{rgb}{0.121569,0.466667,0.705882}%
\pgfsetfillcolor{currentfill}%
\pgfsetlinewidth{1.003750pt}%
\definecolor{currentstroke}{rgb}{0.121569,0.466667,0.705882}%
\pgfsetstrokecolor{currentstroke}%
\pgfsetdash{}{0pt}%
\pgfpathmoveto{\pgfqpoint{2.341260in}{2.370615in}}%
\pgfpathcurveto{\pgfqpoint{2.352310in}{2.370615in}}{\pgfqpoint{2.362909in}{2.375005in}}{\pgfqpoint{2.370723in}{2.382818in}}%
\pgfpathcurveto{\pgfqpoint{2.378536in}{2.390632in}}{\pgfqpoint{2.382926in}{2.401231in}}{\pgfqpoint{2.382926in}{2.412281in}}%
\pgfpathcurveto{\pgfqpoint{2.382926in}{2.423331in}}{\pgfqpoint{2.378536in}{2.433930in}}{\pgfqpoint{2.370723in}{2.441744in}}%
\pgfpathcurveto{\pgfqpoint{2.362909in}{2.449558in}}{\pgfqpoint{2.352310in}{2.453948in}}{\pgfqpoint{2.341260in}{2.453948in}}%
\pgfpathcurveto{\pgfqpoint{2.330210in}{2.453948in}}{\pgfqpoint{2.319611in}{2.449558in}}{\pgfqpoint{2.311797in}{2.441744in}}%
\pgfpathcurveto{\pgfqpoint{2.303983in}{2.433930in}}{\pgfqpoint{2.299593in}{2.423331in}}{\pgfqpoint{2.299593in}{2.412281in}}%
\pgfpathcurveto{\pgfqpoint{2.299593in}{2.401231in}}{\pgfqpoint{2.303983in}{2.390632in}}{\pgfqpoint{2.311797in}{2.382818in}}%
\pgfpathcurveto{\pgfqpoint{2.319611in}{2.375005in}}{\pgfqpoint{2.330210in}{2.370615in}}{\pgfqpoint{2.341260in}{2.370615in}}%
\pgfpathclose%
\pgfusepath{stroke,fill}%
\end{pgfscope}%
\begin{pgfscope}%
\pgfpathrectangle{\pgfqpoint{0.787074in}{0.548769in}}{\pgfqpoint{5.062926in}{3.102590in}}%
\pgfusepath{clip}%
\pgfsetbuttcap%
\pgfsetroundjoin%
\definecolor{currentfill}{rgb}{0.121569,0.466667,0.705882}%
\pgfsetfillcolor{currentfill}%
\pgfsetlinewidth{1.003750pt}%
\definecolor{currentstroke}{rgb}{0.121569,0.466667,0.705882}%
\pgfsetstrokecolor{currentstroke}%
\pgfsetdash{}{0pt}%
\pgfpathmoveto{\pgfqpoint{2.026009in}{1.974526in}}%
\pgfpathcurveto{\pgfqpoint{2.037059in}{1.974526in}}{\pgfqpoint{2.047658in}{1.978916in}}{\pgfqpoint{2.055472in}{1.986730in}}%
\pgfpathcurveto{\pgfqpoint{2.063285in}{1.994543in}}{\pgfqpoint{2.067676in}{2.005142in}}{\pgfqpoint{2.067676in}{2.016192in}}%
\pgfpathcurveto{\pgfqpoint{2.067676in}{2.027242in}}{\pgfqpoint{2.063285in}{2.037842in}}{\pgfqpoint{2.055472in}{2.045655in}}%
\pgfpathcurveto{\pgfqpoint{2.047658in}{2.053469in}}{\pgfqpoint{2.037059in}{2.057859in}}{\pgfqpoint{2.026009in}{2.057859in}}%
\pgfpathcurveto{\pgfqpoint{2.014959in}{2.057859in}}{\pgfqpoint{2.004360in}{2.053469in}}{\pgfqpoint{1.996546in}{2.045655in}}%
\pgfpathcurveto{\pgfqpoint{1.988733in}{2.037842in}}{\pgfqpoint{1.984342in}{2.027242in}}{\pgfqpoint{1.984342in}{2.016192in}}%
\pgfpathcurveto{\pgfqpoint{1.984342in}{2.005142in}}{\pgfqpoint{1.988733in}{1.994543in}}{\pgfqpoint{1.996546in}{1.986730in}}%
\pgfpathcurveto{\pgfqpoint{2.004360in}{1.978916in}}{\pgfqpoint{2.014959in}{1.974526in}}{\pgfqpoint{2.026009in}{1.974526in}}%
\pgfpathclose%
\pgfusepath{stroke,fill}%
\end{pgfscope}%
\begin{pgfscope}%
\pgfpathrectangle{\pgfqpoint{0.787074in}{0.548769in}}{\pgfqpoint{5.062926in}{3.102590in}}%
\pgfusepath{clip}%
\pgfsetbuttcap%
\pgfsetroundjoin%
\definecolor{currentfill}{rgb}{1.000000,0.498039,0.054902}%
\pgfsetfillcolor{currentfill}%
\pgfsetlinewidth{1.003750pt}%
\definecolor{currentstroke}{rgb}{1.000000,0.498039,0.054902}%
\pgfsetstrokecolor{currentstroke}%
\pgfsetdash{}{0pt}%
\pgfpathmoveto{\pgfqpoint{2.341260in}{2.689957in}}%
\pgfpathcurveto{\pgfqpoint{2.352310in}{2.689957in}}{\pgfqpoint{2.362909in}{2.694347in}}{\pgfqpoint{2.370723in}{2.702161in}}%
\pgfpathcurveto{\pgfqpoint{2.378536in}{2.709974in}}{\pgfqpoint{2.382926in}{2.720573in}}{\pgfqpoint{2.382926in}{2.731623in}}%
\pgfpathcurveto{\pgfqpoint{2.382926in}{2.742674in}}{\pgfqpoint{2.378536in}{2.753273in}}{\pgfqpoint{2.370723in}{2.761086in}}%
\pgfpathcurveto{\pgfqpoint{2.362909in}{2.768900in}}{\pgfqpoint{2.352310in}{2.773290in}}{\pgfqpoint{2.341260in}{2.773290in}}%
\pgfpathcurveto{\pgfqpoint{2.330210in}{2.773290in}}{\pgfqpoint{2.319611in}{2.768900in}}{\pgfqpoint{2.311797in}{2.761086in}}%
\pgfpathcurveto{\pgfqpoint{2.303983in}{2.753273in}}{\pgfqpoint{2.299593in}{2.742674in}}{\pgfqpoint{2.299593in}{2.731623in}}%
\pgfpathcurveto{\pgfqpoint{2.299593in}{2.720573in}}{\pgfqpoint{2.303983in}{2.709974in}}{\pgfqpoint{2.311797in}{2.702161in}}%
\pgfpathcurveto{\pgfqpoint{2.319611in}{2.694347in}}{\pgfqpoint{2.330210in}{2.689957in}}{\pgfqpoint{2.341260in}{2.689957in}}%
\pgfpathclose%
\pgfusepath{stroke,fill}%
\end{pgfscope}%
\begin{pgfscope}%
\pgfpathrectangle{\pgfqpoint{0.787074in}{0.548769in}}{\pgfqpoint{5.062926in}{3.102590in}}%
\pgfusepath{clip}%
\pgfsetbuttcap%
\pgfsetroundjoin%
\definecolor{currentfill}{rgb}{0.121569,0.466667,0.705882}%
\pgfsetfillcolor{currentfill}%
\pgfsetlinewidth{1.003750pt}%
\definecolor{currentstroke}{rgb}{0.121569,0.466667,0.705882}%
\pgfsetstrokecolor{currentstroke}%
\pgfsetdash{}{0pt}%
\pgfpathmoveto{\pgfqpoint{4.863265in}{2.569739in}}%
\pgfpathcurveto{\pgfqpoint{4.874315in}{2.569739in}}{\pgfqpoint{4.884914in}{2.574129in}}{\pgfqpoint{4.892728in}{2.581943in}}%
\pgfpathcurveto{\pgfqpoint{4.900542in}{2.589756in}}{\pgfqpoint{4.904932in}{2.600355in}}{\pgfqpoint{4.904932in}{2.611406in}}%
\pgfpathcurveto{\pgfqpoint{4.904932in}{2.622456in}}{\pgfqpoint{4.900542in}{2.633055in}}{\pgfqpoint{4.892728in}{2.640868in}}%
\pgfpathcurveto{\pgfqpoint{4.884914in}{2.648682in}}{\pgfqpoint{4.874315in}{2.653072in}}{\pgfqpoint{4.863265in}{2.653072in}}%
\pgfpathcurveto{\pgfqpoint{4.852215in}{2.653072in}}{\pgfqpoint{4.841616in}{2.648682in}}{\pgfqpoint{4.833803in}{2.640868in}}%
\pgfpathcurveto{\pgfqpoint{4.825989in}{2.633055in}}{\pgfqpoint{4.821599in}{2.622456in}}{\pgfqpoint{4.821599in}{2.611406in}}%
\pgfpathcurveto{\pgfqpoint{4.821599in}{2.600355in}}{\pgfqpoint{4.825989in}{2.589756in}}{\pgfqpoint{4.833803in}{2.581943in}}%
\pgfpathcurveto{\pgfqpoint{4.841616in}{2.574129in}}{\pgfqpoint{4.852215in}{2.569739in}}{\pgfqpoint{4.863265in}{2.569739in}}%
\pgfpathclose%
\pgfusepath{stroke,fill}%
\end{pgfscope}%
\begin{pgfscope}%
\pgfpathrectangle{\pgfqpoint{0.787074in}{0.548769in}}{\pgfqpoint{5.062926in}{3.102590in}}%
\pgfusepath{clip}%
\pgfsetbuttcap%
\pgfsetroundjoin%
\definecolor{currentfill}{rgb}{0.121569,0.466667,0.705882}%
\pgfsetfillcolor{currentfill}%
\pgfsetlinewidth{1.003750pt}%
\definecolor{currentstroke}{rgb}{0.121569,0.466667,0.705882}%
\pgfsetstrokecolor{currentstroke}%
\pgfsetdash{}{0pt}%
\pgfpathmoveto{\pgfqpoint{2.719561in}{2.197035in}}%
\pgfpathcurveto{\pgfqpoint{2.730611in}{2.197035in}}{\pgfqpoint{2.741210in}{2.201425in}}{\pgfqpoint{2.749023in}{2.209239in}}%
\pgfpathcurveto{\pgfqpoint{2.756837in}{2.217052in}}{\pgfqpoint{2.761227in}{2.227651in}}{\pgfqpoint{2.761227in}{2.238701in}}%
\pgfpathcurveto{\pgfqpoint{2.761227in}{2.249751in}}{\pgfqpoint{2.756837in}{2.260350in}}{\pgfqpoint{2.749023in}{2.268164in}}%
\pgfpathcurveto{\pgfqpoint{2.741210in}{2.275978in}}{\pgfqpoint{2.730611in}{2.280368in}}{\pgfqpoint{2.719561in}{2.280368in}}%
\pgfpathcurveto{\pgfqpoint{2.708510in}{2.280368in}}{\pgfqpoint{2.697911in}{2.275978in}}{\pgfqpoint{2.690098in}{2.268164in}}%
\pgfpathcurveto{\pgfqpoint{2.682284in}{2.260350in}}{\pgfqpoint{2.677894in}{2.249751in}}{\pgfqpoint{2.677894in}{2.238701in}}%
\pgfpathcurveto{\pgfqpoint{2.677894in}{2.227651in}}{\pgfqpoint{2.682284in}{2.217052in}}{\pgfqpoint{2.690098in}{2.209239in}}%
\pgfpathcurveto{\pgfqpoint{2.697911in}{2.201425in}}{\pgfqpoint{2.708510in}{2.197035in}}{\pgfqpoint{2.719561in}{2.197035in}}%
\pgfpathclose%
\pgfusepath{stroke,fill}%
\end{pgfscope}%
\begin{pgfscope}%
\pgfpathrectangle{\pgfqpoint{0.787074in}{0.548769in}}{\pgfqpoint{5.062926in}{3.102590in}}%
\pgfusepath{clip}%
\pgfsetbuttcap%
\pgfsetroundjoin%
\definecolor{currentfill}{rgb}{0.121569,0.466667,0.705882}%
\pgfsetfillcolor{currentfill}%
\pgfsetlinewidth{1.003750pt}%
\definecolor{currentstroke}{rgb}{0.121569,0.466667,0.705882}%
\pgfsetstrokecolor{currentstroke}%
\pgfsetdash{}{0pt}%
\pgfpathmoveto{\pgfqpoint{1.269407in}{0.648134in}}%
\pgfpathcurveto{\pgfqpoint{1.280458in}{0.648134in}}{\pgfqpoint{1.291057in}{0.652524in}}{\pgfqpoint{1.298870in}{0.660338in}}%
\pgfpathcurveto{\pgfqpoint{1.306684in}{0.668151in}}{\pgfqpoint{1.311074in}{0.678750in}}{\pgfqpoint{1.311074in}{0.689801in}}%
\pgfpathcurveto{\pgfqpoint{1.311074in}{0.700851in}}{\pgfqpoint{1.306684in}{0.711450in}}{\pgfqpoint{1.298870in}{0.719263in}}%
\pgfpathcurveto{\pgfqpoint{1.291057in}{0.727077in}}{\pgfqpoint{1.280458in}{0.731467in}}{\pgfqpoint{1.269407in}{0.731467in}}%
\pgfpathcurveto{\pgfqpoint{1.258357in}{0.731467in}}{\pgfqpoint{1.247758in}{0.727077in}}{\pgfqpoint{1.239945in}{0.719263in}}%
\pgfpathcurveto{\pgfqpoint{1.232131in}{0.711450in}}{\pgfqpoint{1.227741in}{0.700851in}}{\pgfqpoint{1.227741in}{0.689801in}}%
\pgfpathcurveto{\pgfqpoint{1.227741in}{0.678750in}}{\pgfqpoint{1.232131in}{0.668151in}}{\pgfqpoint{1.239945in}{0.660338in}}%
\pgfpathcurveto{\pgfqpoint{1.247758in}{0.652524in}}{\pgfqpoint{1.258357in}{0.648134in}}{\pgfqpoint{1.269407in}{0.648134in}}%
\pgfpathclose%
\pgfusepath{stroke,fill}%
\end{pgfscope}%
\begin{pgfscope}%
\pgfpathrectangle{\pgfqpoint{0.787074in}{0.548769in}}{\pgfqpoint{5.062926in}{3.102590in}}%
\pgfusepath{clip}%
\pgfsetbuttcap%
\pgfsetroundjoin%
\definecolor{currentfill}{rgb}{1.000000,0.498039,0.054902}%
\pgfsetfillcolor{currentfill}%
\pgfsetlinewidth{1.003750pt}%
\definecolor{currentstroke}{rgb}{1.000000,0.498039,0.054902}%
\pgfsetstrokecolor{currentstroke}%
\pgfsetdash{}{0pt}%
\pgfpathmoveto{\pgfqpoint{2.782611in}{1.410798in}}%
\pgfpathcurveto{\pgfqpoint{2.793661in}{1.410798in}}{\pgfqpoint{2.804260in}{1.415189in}}{\pgfqpoint{2.812074in}{1.423002in}}%
\pgfpathcurveto{\pgfqpoint{2.819887in}{1.430816in}}{\pgfqpoint{2.824277in}{1.441415in}}{\pgfqpoint{2.824277in}{1.452465in}}%
\pgfpathcurveto{\pgfqpoint{2.824277in}{1.463515in}}{\pgfqpoint{2.819887in}{1.474114in}}{\pgfqpoint{2.812074in}{1.481928in}}%
\pgfpathcurveto{\pgfqpoint{2.804260in}{1.489741in}}{\pgfqpoint{2.793661in}{1.494132in}}{\pgfqpoint{2.782611in}{1.494132in}}%
\pgfpathcurveto{\pgfqpoint{2.771561in}{1.494132in}}{\pgfqpoint{2.760962in}{1.489741in}}{\pgfqpoint{2.753148in}{1.481928in}}%
\pgfpathcurveto{\pgfqpoint{2.745334in}{1.474114in}}{\pgfqpoint{2.740944in}{1.463515in}}{\pgfqpoint{2.740944in}{1.452465in}}%
\pgfpathcurveto{\pgfqpoint{2.740944in}{1.441415in}}{\pgfqpoint{2.745334in}{1.430816in}}{\pgfqpoint{2.753148in}{1.423002in}}%
\pgfpathcurveto{\pgfqpoint{2.760962in}{1.415189in}}{\pgfqpoint{2.771561in}{1.410798in}}{\pgfqpoint{2.782611in}{1.410798in}}%
\pgfpathclose%
\pgfusepath{stroke,fill}%
\end{pgfscope}%
\begin{pgfscope}%
\pgfpathrectangle{\pgfqpoint{0.787074in}{0.548769in}}{\pgfqpoint{5.062926in}{3.102590in}}%
\pgfusepath{clip}%
\pgfsetbuttcap%
\pgfsetroundjoin%
\definecolor{currentfill}{rgb}{1.000000,0.498039,0.054902}%
\pgfsetfillcolor{currentfill}%
\pgfsetlinewidth{1.003750pt}%
\definecolor{currentstroke}{rgb}{1.000000,0.498039,0.054902}%
\pgfsetstrokecolor{currentstroke}%
\pgfsetdash{}{0pt}%
\pgfpathmoveto{\pgfqpoint{1.836859in}{2.681418in}}%
\pgfpathcurveto{\pgfqpoint{1.847909in}{2.681418in}}{\pgfqpoint{1.858508in}{2.685808in}}{\pgfqpoint{1.866321in}{2.693622in}}%
\pgfpathcurveto{\pgfqpoint{1.874135in}{2.701435in}}{\pgfqpoint{1.878525in}{2.712034in}}{\pgfqpoint{1.878525in}{2.723084in}}%
\pgfpathcurveto{\pgfqpoint{1.878525in}{2.734135in}}{\pgfqpoint{1.874135in}{2.744734in}}{\pgfqpoint{1.866321in}{2.752547in}}%
\pgfpathcurveto{\pgfqpoint{1.858508in}{2.760361in}}{\pgfqpoint{1.847909in}{2.764751in}}{\pgfqpoint{1.836859in}{2.764751in}}%
\pgfpathcurveto{\pgfqpoint{1.825809in}{2.764751in}}{\pgfqpoint{1.815209in}{2.760361in}}{\pgfqpoint{1.807396in}{2.752547in}}%
\pgfpathcurveto{\pgfqpoint{1.799582in}{2.744734in}}{\pgfqpoint{1.795192in}{2.734135in}}{\pgfqpoint{1.795192in}{2.723084in}}%
\pgfpathcurveto{\pgfqpoint{1.795192in}{2.712034in}}{\pgfqpoint{1.799582in}{2.701435in}}{\pgfqpoint{1.807396in}{2.693622in}}%
\pgfpathcurveto{\pgfqpoint{1.815209in}{2.685808in}}{\pgfqpoint{1.825809in}{2.681418in}}{\pgfqpoint{1.836859in}{2.681418in}}%
\pgfpathclose%
\pgfusepath{stroke,fill}%
\end{pgfscope}%
\begin{pgfscope}%
\pgfpathrectangle{\pgfqpoint{0.787074in}{0.548769in}}{\pgfqpoint{5.062926in}{3.102590in}}%
\pgfusepath{clip}%
\pgfsetbuttcap%
\pgfsetroundjoin%
\definecolor{currentfill}{rgb}{1.000000,0.498039,0.054902}%
\pgfsetfillcolor{currentfill}%
\pgfsetlinewidth{1.003750pt}%
\definecolor{currentstroke}{rgb}{1.000000,0.498039,0.054902}%
\pgfsetstrokecolor{currentstroke}%
\pgfsetdash{}{0pt}%
\pgfpathmoveto{\pgfqpoint{1.962959in}{2.256332in}}%
\pgfpathcurveto{\pgfqpoint{1.974009in}{2.256332in}}{\pgfqpoint{1.984608in}{2.260722in}}{\pgfqpoint{1.992422in}{2.268536in}}%
\pgfpathcurveto{\pgfqpoint{2.000235in}{2.276349in}}{\pgfqpoint{2.004626in}{2.286948in}}{\pgfqpoint{2.004626in}{2.297998in}}%
\pgfpathcurveto{\pgfqpoint{2.004626in}{2.309048in}}{\pgfqpoint{2.000235in}{2.319648in}}{\pgfqpoint{1.992422in}{2.327461in}}%
\pgfpathcurveto{\pgfqpoint{1.984608in}{2.335275in}}{\pgfqpoint{1.974009in}{2.339665in}}{\pgfqpoint{1.962959in}{2.339665in}}%
\pgfpathcurveto{\pgfqpoint{1.951909in}{2.339665in}}{\pgfqpoint{1.941310in}{2.335275in}}{\pgfqpoint{1.933496in}{2.327461in}}%
\pgfpathcurveto{\pgfqpoint{1.925683in}{2.319648in}}{\pgfqpoint{1.921292in}{2.309048in}}{\pgfqpoint{1.921292in}{2.297998in}}%
\pgfpathcurveto{\pgfqpoint{1.921292in}{2.286948in}}{\pgfqpoint{1.925683in}{2.276349in}}{\pgfqpoint{1.933496in}{2.268536in}}%
\pgfpathcurveto{\pgfqpoint{1.941310in}{2.260722in}}{\pgfqpoint{1.951909in}{2.256332in}}{\pgfqpoint{1.962959in}{2.256332in}}%
\pgfpathclose%
\pgfusepath{stroke,fill}%
\end{pgfscope}%
\begin{pgfscope}%
\pgfpathrectangle{\pgfqpoint{0.787074in}{0.548769in}}{\pgfqpoint{5.062926in}{3.102590in}}%
\pgfusepath{clip}%
\pgfsetbuttcap%
\pgfsetroundjoin%
\definecolor{currentfill}{rgb}{0.121569,0.466667,0.705882}%
\pgfsetfillcolor{currentfill}%
\pgfsetlinewidth{1.003750pt}%
\definecolor{currentstroke}{rgb}{0.121569,0.466667,0.705882}%
\pgfsetstrokecolor{currentstroke}%
\pgfsetdash{}{0pt}%
\pgfpathmoveto{\pgfqpoint{1.458558in}{0.775326in}}%
\pgfpathcurveto{\pgfqpoint{1.469608in}{0.775326in}}{\pgfqpoint{1.480207in}{0.779716in}}{\pgfqpoint{1.488021in}{0.787530in}}%
\pgfpathcurveto{\pgfqpoint{1.495834in}{0.795343in}}{\pgfqpoint{1.500224in}{0.805942in}}{\pgfqpoint{1.500224in}{0.816993in}}%
\pgfpathcurveto{\pgfqpoint{1.500224in}{0.828043in}}{\pgfqpoint{1.495834in}{0.838642in}}{\pgfqpoint{1.488021in}{0.846455in}}%
\pgfpathcurveto{\pgfqpoint{1.480207in}{0.854269in}}{\pgfqpoint{1.469608in}{0.858659in}}{\pgfqpoint{1.458558in}{0.858659in}}%
\pgfpathcurveto{\pgfqpoint{1.447508in}{0.858659in}}{\pgfqpoint{1.436909in}{0.854269in}}{\pgfqpoint{1.429095in}{0.846455in}}%
\pgfpathcurveto{\pgfqpoint{1.421281in}{0.838642in}}{\pgfqpoint{1.416891in}{0.828043in}}{\pgfqpoint{1.416891in}{0.816993in}}%
\pgfpathcurveto{\pgfqpoint{1.416891in}{0.805942in}}{\pgfqpoint{1.421281in}{0.795343in}}{\pgfqpoint{1.429095in}{0.787530in}}%
\pgfpathcurveto{\pgfqpoint{1.436909in}{0.779716in}}{\pgfqpoint{1.447508in}{0.775326in}}{\pgfqpoint{1.458558in}{0.775326in}}%
\pgfpathclose%
\pgfusepath{stroke,fill}%
\end{pgfscope}%
\begin{pgfscope}%
\pgfpathrectangle{\pgfqpoint{0.787074in}{0.548769in}}{\pgfqpoint{5.062926in}{3.102590in}}%
\pgfusepath{clip}%
\pgfsetbuttcap%
\pgfsetroundjoin%
\definecolor{currentfill}{rgb}{1.000000,0.498039,0.054902}%
\pgfsetfillcolor{currentfill}%
\pgfsetlinewidth{1.003750pt}%
\definecolor{currentstroke}{rgb}{1.000000,0.498039,0.054902}%
\pgfsetstrokecolor{currentstroke}%
\pgfsetdash{}{0pt}%
\pgfpathmoveto{\pgfqpoint{2.719561in}{1.583415in}}%
\pgfpathcurveto{\pgfqpoint{2.730611in}{1.583415in}}{\pgfqpoint{2.741210in}{1.587805in}}{\pgfqpoint{2.749023in}{1.595619in}}%
\pgfpathcurveto{\pgfqpoint{2.756837in}{1.603432in}}{\pgfqpoint{2.761227in}{1.614031in}}{\pgfqpoint{2.761227in}{1.625081in}}%
\pgfpathcurveto{\pgfqpoint{2.761227in}{1.636131in}}{\pgfqpoint{2.756837in}{1.646730in}}{\pgfqpoint{2.749023in}{1.654544in}}%
\pgfpathcurveto{\pgfqpoint{2.741210in}{1.662358in}}{\pgfqpoint{2.730611in}{1.666748in}}{\pgfqpoint{2.719561in}{1.666748in}}%
\pgfpathcurveto{\pgfqpoint{2.708510in}{1.666748in}}{\pgfqpoint{2.697911in}{1.662358in}}{\pgfqpoint{2.690098in}{1.654544in}}%
\pgfpathcurveto{\pgfqpoint{2.682284in}{1.646730in}}{\pgfqpoint{2.677894in}{1.636131in}}{\pgfqpoint{2.677894in}{1.625081in}}%
\pgfpathcurveto{\pgfqpoint{2.677894in}{1.614031in}}{\pgfqpoint{2.682284in}{1.603432in}}{\pgfqpoint{2.690098in}{1.595619in}}%
\pgfpathcurveto{\pgfqpoint{2.697911in}{1.587805in}}{\pgfqpoint{2.708510in}{1.583415in}}{\pgfqpoint{2.719561in}{1.583415in}}%
\pgfpathclose%
\pgfusepath{stroke,fill}%
\end{pgfscope}%
\begin{pgfscope}%
\pgfpathrectangle{\pgfqpoint{0.787074in}{0.548769in}}{\pgfqpoint{5.062926in}{3.102590in}}%
\pgfusepath{clip}%
\pgfsetbuttcap%
\pgfsetroundjoin%
\definecolor{currentfill}{rgb}{0.121569,0.466667,0.705882}%
\pgfsetfillcolor{currentfill}%
\pgfsetlinewidth{1.003750pt}%
\definecolor{currentstroke}{rgb}{0.121569,0.466667,0.705882}%
\pgfsetstrokecolor{currentstroke}%
\pgfsetdash{}{0pt}%
\pgfpathmoveto{\pgfqpoint{2.530410in}{2.261976in}}%
\pgfpathcurveto{\pgfqpoint{2.541460in}{2.261976in}}{\pgfqpoint{2.552059in}{2.266366in}}{\pgfqpoint{2.559873in}{2.274180in}}%
\pgfpathcurveto{\pgfqpoint{2.567687in}{2.281994in}}{\pgfqpoint{2.572077in}{2.292593in}}{\pgfqpoint{2.572077in}{2.303643in}}%
\pgfpathcurveto{\pgfqpoint{2.572077in}{2.314693in}}{\pgfqpoint{2.567687in}{2.325292in}}{\pgfqpoint{2.559873in}{2.333105in}}%
\pgfpathcurveto{\pgfqpoint{2.552059in}{2.340919in}}{\pgfqpoint{2.541460in}{2.345309in}}{\pgfqpoint{2.530410in}{2.345309in}}%
\pgfpathcurveto{\pgfqpoint{2.519360in}{2.345309in}}{\pgfqpoint{2.508761in}{2.340919in}}{\pgfqpoint{2.500947in}{2.333105in}}%
\pgfpathcurveto{\pgfqpoint{2.493134in}{2.325292in}}{\pgfqpoint{2.488744in}{2.314693in}}{\pgfqpoint{2.488744in}{2.303643in}}%
\pgfpathcurveto{\pgfqpoint{2.488744in}{2.292593in}}{\pgfqpoint{2.493134in}{2.281994in}}{\pgfqpoint{2.500947in}{2.274180in}}%
\pgfpathcurveto{\pgfqpoint{2.508761in}{2.266366in}}{\pgfqpoint{2.519360in}{2.261976in}}{\pgfqpoint{2.530410in}{2.261976in}}%
\pgfpathclose%
\pgfusepath{stroke,fill}%
\end{pgfscope}%
\begin{pgfscope}%
\pgfpathrectangle{\pgfqpoint{0.787074in}{0.548769in}}{\pgfqpoint{5.062926in}{3.102590in}}%
\pgfusepath{clip}%
\pgfsetbuttcap%
\pgfsetroundjoin%
\definecolor{currentfill}{rgb}{0.121569,0.466667,0.705882}%
\pgfsetfillcolor{currentfill}%
\pgfsetlinewidth{1.003750pt}%
\definecolor{currentstroke}{rgb}{0.121569,0.466667,0.705882}%
\pgfsetstrokecolor{currentstroke}%
\pgfsetdash{}{0pt}%
\pgfpathmoveto{\pgfqpoint{3.160912in}{3.078940in}}%
\pgfpathcurveto{\pgfqpoint{3.171962in}{3.078940in}}{\pgfqpoint{3.182561in}{3.083331in}}{\pgfqpoint{3.190374in}{3.091144in}}%
\pgfpathcurveto{\pgfqpoint{3.198188in}{3.098958in}}{\pgfqpoint{3.202578in}{3.109557in}}{\pgfqpoint{3.202578in}{3.120607in}}%
\pgfpathcurveto{\pgfqpoint{3.202578in}{3.131657in}}{\pgfqpoint{3.198188in}{3.142256in}}{\pgfqpoint{3.190374in}{3.150070in}}%
\pgfpathcurveto{\pgfqpoint{3.182561in}{3.157883in}}{\pgfqpoint{3.171962in}{3.162274in}}{\pgfqpoint{3.160912in}{3.162274in}}%
\pgfpathcurveto{\pgfqpoint{3.149861in}{3.162274in}}{\pgfqpoint{3.139262in}{3.157883in}}{\pgfqpoint{3.131449in}{3.150070in}}%
\pgfpathcurveto{\pgfqpoint{3.123635in}{3.142256in}}{\pgfqpoint{3.119245in}{3.131657in}}{\pgfqpoint{3.119245in}{3.120607in}}%
\pgfpathcurveto{\pgfqpoint{3.119245in}{3.109557in}}{\pgfqpoint{3.123635in}{3.098958in}}{\pgfqpoint{3.131449in}{3.091144in}}%
\pgfpathcurveto{\pgfqpoint{3.139262in}{3.083331in}}{\pgfqpoint{3.149861in}{3.078940in}}{\pgfqpoint{3.160912in}{3.078940in}}%
\pgfpathclose%
\pgfusepath{stroke,fill}%
\end{pgfscope}%
\begin{pgfscope}%
\pgfpathrectangle{\pgfqpoint{0.787074in}{0.548769in}}{\pgfqpoint{5.062926in}{3.102590in}}%
\pgfusepath{clip}%
\pgfsetbuttcap%
\pgfsetroundjoin%
\definecolor{currentfill}{rgb}{1.000000,0.498039,0.054902}%
\pgfsetfillcolor{currentfill}%
\pgfsetlinewidth{1.003750pt}%
\definecolor{currentstroke}{rgb}{1.000000,0.498039,0.054902}%
\pgfsetstrokecolor{currentstroke}%
\pgfsetdash{}{0pt}%
\pgfpathmoveto{\pgfqpoint{1.836859in}{1.210818in}}%
\pgfpathcurveto{\pgfqpoint{1.847909in}{1.210818in}}{\pgfqpoint{1.858508in}{1.215208in}}{\pgfqpoint{1.866321in}{1.223021in}}%
\pgfpathcurveto{\pgfqpoint{1.874135in}{1.230835in}}{\pgfqpoint{1.878525in}{1.241434in}}{\pgfqpoint{1.878525in}{1.252484in}}%
\pgfpathcurveto{\pgfqpoint{1.878525in}{1.263534in}}{\pgfqpoint{1.874135in}{1.274133in}}{\pgfqpoint{1.866321in}{1.281947in}}%
\pgfpathcurveto{\pgfqpoint{1.858508in}{1.289761in}}{\pgfqpoint{1.847909in}{1.294151in}}{\pgfqpoint{1.836859in}{1.294151in}}%
\pgfpathcurveto{\pgfqpoint{1.825809in}{1.294151in}}{\pgfqpoint{1.815209in}{1.289761in}}{\pgfqpoint{1.807396in}{1.281947in}}%
\pgfpathcurveto{\pgfqpoint{1.799582in}{1.274133in}}{\pgfqpoint{1.795192in}{1.263534in}}{\pgfqpoint{1.795192in}{1.252484in}}%
\pgfpathcurveto{\pgfqpoint{1.795192in}{1.241434in}}{\pgfqpoint{1.799582in}{1.230835in}}{\pgfqpoint{1.807396in}{1.223021in}}%
\pgfpathcurveto{\pgfqpoint{1.815209in}{1.215208in}}{\pgfqpoint{1.825809in}{1.210818in}}{\pgfqpoint{1.836859in}{1.210818in}}%
\pgfpathclose%
\pgfusepath{stroke,fill}%
\end{pgfscope}%
\begin{pgfscope}%
\pgfpathrectangle{\pgfqpoint{0.787074in}{0.548769in}}{\pgfqpoint{5.062926in}{3.102590in}}%
\pgfusepath{clip}%
\pgfsetbuttcap%
\pgfsetroundjoin%
\definecolor{currentfill}{rgb}{1.000000,0.498039,0.054902}%
\pgfsetfillcolor{currentfill}%
\pgfsetlinewidth{1.003750pt}%
\definecolor{currentstroke}{rgb}{1.000000,0.498039,0.054902}%
\pgfsetstrokecolor{currentstroke}%
\pgfsetdash{}{0pt}%
\pgfpathmoveto{\pgfqpoint{2.278210in}{1.923637in}}%
\pgfpathcurveto{\pgfqpoint{2.289260in}{1.923637in}}{\pgfqpoint{2.299859in}{1.928027in}}{\pgfqpoint{2.307672in}{1.935841in}}%
\pgfpathcurveto{\pgfqpoint{2.315486in}{1.943654in}}{\pgfqpoint{2.319876in}{1.954253in}}{\pgfqpoint{2.319876in}{1.965303in}}%
\pgfpathcurveto{\pgfqpoint{2.319876in}{1.976354in}}{\pgfqpoint{2.315486in}{1.986953in}}{\pgfqpoint{2.307672in}{1.994766in}}%
\pgfpathcurveto{\pgfqpoint{2.299859in}{2.002580in}}{\pgfqpoint{2.289260in}{2.006970in}}{\pgfqpoint{2.278210in}{2.006970in}}%
\pgfpathcurveto{\pgfqpoint{2.267159in}{2.006970in}}{\pgfqpoint{2.256560in}{2.002580in}}{\pgfqpoint{2.248747in}{1.994766in}}%
\pgfpathcurveto{\pgfqpoint{2.240933in}{1.986953in}}{\pgfqpoint{2.236543in}{1.976354in}}{\pgfqpoint{2.236543in}{1.965303in}}%
\pgfpathcurveto{\pgfqpoint{2.236543in}{1.954253in}}{\pgfqpoint{2.240933in}{1.943654in}}{\pgfqpoint{2.248747in}{1.935841in}}%
\pgfpathcurveto{\pgfqpoint{2.256560in}{1.928027in}}{\pgfqpoint{2.267159in}{1.923637in}}{\pgfqpoint{2.278210in}{1.923637in}}%
\pgfpathclose%
\pgfusepath{stroke,fill}%
\end{pgfscope}%
\begin{pgfscope}%
\pgfpathrectangle{\pgfqpoint{0.787074in}{0.548769in}}{\pgfqpoint{5.062926in}{3.102590in}}%
\pgfusepath{clip}%
\pgfsetbuttcap%
\pgfsetroundjoin%
\definecolor{currentfill}{rgb}{0.121569,0.466667,0.705882}%
\pgfsetfillcolor{currentfill}%
\pgfsetlinewidth{1.003750pt}%
\definecolor{currentstroke}{rgb}{0.121569,0.466667,0.705882}%
\pgfsetstrokecolor{currentstroke}%
\pgfsetdash{}{0pt}%
\pgfpathmoveto{\pgfqpoint{1.458558in}{2.880292in}}%
\pgfpathcurveto{\pgfqpoint{1.469608in}{2.880292in}}{\pgfqpoint{1.480207in}{2.884682in}}{\pgfqpoint{1.488021in}{2.892495in}}%
\pgfpathcurveto{\pgfqpoint{1.495834in}{2.900309in}}{\pgfqpoint{1.500224in}{2.910908in}}{\pgfqpoint{1.500224in}{2.921958in}}%
\pgfpathcurveto{\pgfqpoint{1.500224in}{2.933008in}}{\pgfqpoint{1.495834in}{2.943607in}}{\pgfqpoint{1.488021in}{2.951421in}}%
\pgfpathcurveto{\pgfqpoint{1.480207in}{2.959235in}}{\pgfqpoint{1.469608in}{2.963625in}}{\pgfqpoint{1.458558in}{2.963625in}}%
\pgfpathcurveto{\pgfqpoint{1.447508in}{2.963625in}}{\pgfqpoint{1.436909in}{2.959235in}}{\pgfqpoint{1.429095in}{2.951421in}}%
\pgfpathcurveto{\pgfqpoint{1.421281in}{2.943607in}}{\pgfqpoint{1.416891in}{2.933008in}}{\pgfqpoint{1.416891in}{2.921958in}}%
\pgfpathcurveto{\pgfqpoint{1.416891in}{2.910908in}}{\pgfqpoint{1.421281in}{2.900309in}}{\pgfqpoint{1.429095in}{2.892495in}}%
\pgfpathcurveto{\pgfqpoint{1.436909in}{2.884682in}}{\pgfqpoint{1.447508in}{2.880292in}}{\pgfqpoint{1.458558in}{2.880292in}}%
\pgfpathclose%
\pgfusepath{stroke,fill}%
\end{pgfscope}%
\begin{pgfscope}%
\pgfpathrectangle{\pgfqpoint{0.787074in}{0.548769in}}{\pgfqpoint{5.062926in}{3.102590in}}%
\pgfusepath{clip}%
\pgfsetbuttcap%
\pgfsetroundjoin%
\definecolor{currentfill}{rgb}{0.121569,0.466667,0.705882}%
\pgfsetfillcolor{currentfill}%
\pgfsetlinewidth{1.003750pt}%
\definecolor{currentstroke}{rgb}{0.121569,0.466667,0.705882}%
\pgfsetstrokecolor{currentstroke}%
\pgfsetdash{}{0pt}%
\pgfpathmoveto{\pgfqpoint{2.404310in}{2.018767in}}%
\pgfpathcurveto{\pgfqpoint{2.415360in}{2.018767in}}{\pgfqpoint{2.425959in}{2.023158in}}{\pgfqpoint{2.433773in}{2.030971in}}%
\pgfpathcurveto{\pgfqpoint{2.441586in}{2.038785in}}{\pgfqpoint{2.445977in}{2.049384in}}{\pgfqpoint{2.445977in}{2.060434in}}%
\pgfpathcurveto{\pgfqpoint{2.445977in}{2.071484in}}{\pgfqpoint{2.441586in}{2.082083in}}{\pgfqpoint{2.433773in}{2.089897in}}%
\pgfpathcurveto{\pgfqpoint{2.425959in}{2.097710in}}{\pgfqpoint{2.415360in}{2.102101in}}{\pgfqpoint{2.404310in}{2.102101in}}%
\pgfpathcurveto{\pgfqpoint{2.393260in}{2.102101in}}{\pgfqpoint{2.382661in}{2.097710in}}{\pgfqpoint{2.374847in}{2.089897in}}%
\pgfpathcurveto{\pgfqpoint{2.367033in}{2.082083in}}{\pgfqpoint{2.362643in}{2.071484in}}{\pgfqpoint{2.362643in}{2.060434in}}%
\pgfpathcurveto{\pgfqpoint{2.362643in}{2.049384in}}{\pgfqpoint{2.367033in}{2.038785in}}{\pgfqpoint{2.374847in}{2.030971in}}%
\pgfpathcurveto{\pgfqpoint{2.382661in}{2.023158in}}{\pgfqpoint{2.393260in}{2.018767in}}{\pgfqpoint{2.404310in}{2.018767in}}%
\pgfpathclose%
\pgfusepath{stroke,fill}%
\end{pgfscope}%
\begin{pgfscope}%
\pgfpathrectangle{\pgfqpoint{0.787074in}{0.548769in}}{\pgfqpoint{5.062926in}{3.102590in}}%
\pgfusepath{clip}%
\pgfsetbuttcap%
\pgfsetroundjoin%
\definecolor{currentfill}{rgb}{1.000000,0.498039,0.054902}%
\pgfsetfillcolor{currentfill}%
\pgfsetlinewidth{1.003750pt}%
\definecolor{currentstroke}{rgb}{1.000000,0.498039,0.054902}%
\pgfsetstrokecolor{currentstroke}%
\pgfsetdash{}{0pt}%
\pgfpathmoveto{\pgfqpoint{2.719561in}{0.875725in}}%
\pgfpathcurveto{\pgfqpoint{2.730611in}{0.875725in}}{\pgfqpoint{2.741210in}{0.880115in}}{\pgfqpoint{2.749023in}{0.887929in}}%
\pgfpathcurveto{\pgfqpoint{2.756837in}{0.895743in}}{\pgfqpoint{2.761227in}{0.906342in}}{\pgfqpoint{2.761227in}{0.917392in}}%
\pgfpathcurveto{\pgfqpoint{2.761227in}{0.928442in}}{\pgfqpoint{2.756837in}{0.939041in}}{\pgfqpoint{2.749023in}{0.946855in}}%
\pgfpathcurveto{\pgfqpoint{2.741210in}{0.954668in}}{\pgfqpoint{2.730611in}{0.959058in}}{\pgfqpoint{2.719561in}{0.959058in}}%
\pgfpathcurveto{\pgfqpoint{2.708510in}{0.959058in}}{\pgfqpoint{2.697911in}{0.954668in}}{\pgfqpoint{2.690098in}{0.946855in}}%
\pgfpathcurveto{\pgfqpoint{2.682284in}{0.939041in}}{\pgfqpoint{2.677894in}{0.928442in}}{\pgfqpoint{2.677894in}{0.917392in}}%
\pgfpathcurveto{\pgfqpoint{2.677894in}{0.906342in}}{\pgfqpoint{2.682284in}{0.895743in}}{\pgfqpoint{2.690098in}{0.887929in}}%
\pgfpathcurveto{\pgfqpoint{2.697911in}{0.880115in}}{\pgfqpoint{2.708510in}{0.875725in}}{\pgfqpoint{2.719561in}{0.875725in}}%
\pgfpathclose%
\pgfusepath{stroke,fill}%
\end{pgfscope}%
\begin{pgfscope}%
\pgfpathrectangle{\pgfqpoint{0.787074in}{0.548769in}}{\pgfqpoint{5.062926in}{3.102590in}}%
\pgfusepath{clip}%
\pgfsetbuttcap%
\pgfsetroundjoin%
\definecolor{currentfill}{rgb}{1.000000,0.498039,0.054902}%
\pgfsetfillcolor{currentfill}%
\pgfsetlinewidth{1.003750pt}%
\definecolor{currentstroke}{rgb}{1.000000,0.498039,0.054902}%
\pgfsetstrokecolor{currentstroke}%
\pgfsetdash{}{0pt}%
\pgfpathmoveto{\pgfqpoint{2.278210in}{1.792377in}}%
\pgfpathcurveto{\pgfqpoint{2.289260in}{1.792377in}}{\pgfqpoint{2.299859in}{1.796767in}}{\pgfqpoint{2.307672in}{1.804581in}}%
\pgfpathcurveto{\pgfqpoint{2.315486in}{1.812394in}}{\pgfqpoint{2.319876in}{1.822993in}}{\pgfqpoint{2.319876in}{1.834043in}}%
\pgfpathcurveto{\pgfqpoint{2.319876in}{1.845093in}}{\pgfqpoint{2.315486in}{1.855693in}}{\pgfqpoint{2.307672in}{1.863506in}}%
\pgfpathcurveto{\pgfqpoint{2.299859in}{1.871320in}}{\pgfqpoint{2.289260in}{1.875710in}}{\pgfqpoint{2.278210in}{1.875710in}}%
\pgfpathcurveto{\pgfqpoint{2.267159in}{1.875710in}}{\pgfqpoint{2.256560in}{1.871320in}}{\pgfqpoint{2.248747in}{1.863506in}}%
\pgfpathcurveto{\pgfqpoint{2.240933in}{1.855693in}}{\pgfqpoint{2.236543in}{1.845093in}}{\pgfqpoint{2.236543in}{1.834043in}}%
\pgfpathcurveto{\pgfqpoint{2.236543in}{1.822993in}}{\pgfqpoint{2.240933in}{1.812394in}}{\pgfqpoint{2.248747in}{1.804581in}}%
\pgfpathcurveto{\pgfqpoint{2.256560in}{1.796767in}}{\pgfqpoint{2.267159in}{1.792377in}}{\pgfqpoint{2.278210in}{1.792377in}}%
\pgfpathclose%
\pgfusepath{stroke,fill}%
\end{pgfscope}%
\begin{pgfscope}%
\pgfpathrectangle{\pgfqpoint{0.787074in}{0.548769in}}{\pgfqpoint{5.062926in}{3.102590in}}%
\pgfusepath{clip}%
\pgfsetbuttcap%
\pgfsetroundjoin%
\definecolor{currentfill}{rgb}{1.000000,0.498039,0.054902}%
\pgfsetfillcolor{currentfill}%
\pgfsetlinewidth{1.003750pt}%
\definecolor{currentstroke}{rgb}{1.000000,0.498039,0.054902}%
\pgfsetstrokecolor{currentstroke}%
\pgfsetdash{}{0pt}%
\pgfpathmoveto{\pgfqpoint{1.647708in}{1.860728in}}%
\pgfpathcurveto{\pgfqpoint{1.658758in}{1.860728in}}{\pgfqpoint{1.669357in}{1.865118in}}{\pgfqpoint{1.677171in}{1.872931in}}%
\pgfpathcurveto{\pgfqpoint{1.684985in}{1.880745in}}{\pgfqpoint{1.689375in}{1.891344in}}{\pgfqpoint{1.689375in}{1.902394in}}%
\pgfpathcurveto{\pgfqpoint{1.689375in}{1.913444in}}{\pgfqpoint{1.684985in}{1.924043in}}{\pgfqpoint{1.677171in}{1.931857in}}%
\pgfpathcurveto{\pgfqpoint{1.669357in}{1.939671in}}{\pgfqpoint{1.658758in}{1.944061in}}{\pgfqpoint{1.647708in}{1.944061in}}%
\pgfpathcurveto{\pgfqpoint{1.636658in}{1.944061in}}{\pgfqpoint{1.626059in}{1.939671in}}{\pgfqpoint{1.618245in}{1.931857in}}%
\pgfpathcurveto{\pgfqpoint{1.610432in}{1.924043in}}{\pgfqpoint{1.606042in}{1.913444in}}{\pgfqpoint{1.606042in}{1.902394in}}%
\pgfpathcurveto{\pgfqpoint{1.606042in}{1.891344in}}{\pgfqpoint{1.610432in}{1.880745in}}{\pgfqpoint{1.618245in}{1.872931in}}%
\pgfpathcurveto{\pgfqpoint{1.626059in}{1.865118in}}{\pgfqpoint{1.636658in}{1.860728in}}{\pgfqpoint{1.647708in}{1.860728in}}%
\pgfpathclose%
\pgfusepath{stroke,fill}%
\end{pgfscope}%
\begin{pgfscope}%
\pgfpathrectangle{\pgfqpoint{0.787074in}{0.548769in}}{\pgfqpoint{5.062926in}{3.102590in}}%
\pgfusepath{clip}%
\pgfsetbuttcap%
\pgfsetroundjoin%
\definecolor{currentfill}{rgb}{0.121569,0.466667,0.705882}%
\pgfsetfillcolor{currentfill}%
\pgfsetlinewidth{1.003750pt}%
\definecolor{currentstroke}{rgb}{0.121569,0.466667,0.705882}%
\pgfsetstrokecolor{currentstroke}%
\pgfsetdash{}{0pt}%
\pgfpathmoveto{\pgfqpoint{3.034811in}{1.531547in}}%
\pgfpathcurveto{\pgfqpoint{3.045861in}{1.531547in}}{\pgfqpoint{3.056460in}{1.535938in}}{\pgfqpoint{3.064274in}{1.543751in}}%
\pgfpathcurveto{\pgfqpoint{3.072088in}{1.551565in}}{\pgfqpoint{3.076478in}{1.562164in}}{\pgfqpoint{3.076478in}{1.573214in}}%
\pgfpathcurveto{\pgfqpoint{3.076478in}{1.584264in}}{\pgfqpoint{3.072088in}{1.594863in}}{\pgfqpoint{3.064274in}{1.602677in}}%
\pgfpathcurveto{\pgfqpoint{3.056460in}{1.610490in}}{\pgfqpoint{3.045861in}{1.614881in}}{\pgfqpoint{3.034811in}{1.614881in}}%
\pgfpathcurveto{\pgfqpoint{3.023761in}{1.614881in}}{\pgfqpoint{3.013162in}{1.610490in}}{\pgfqpoint{3.005349in}{1.602677in}}%
\pgfpathcurveto{\pgfqpoint{2.997535in}{1.594863in}}{\pgfqpoint{2.993145in}{1.584264in}}{\pgfqpoint{2.993145in}{1.573214in}}%
\pgfpathcurveto{\pgfqpoint{2.993145in}{1.562164in}}{\pgfqpoint{2.997535in}{1.551565in}}{\pgfqpoint{3.005349in}{1.543751in}}%
\pgfpathcurveto{\pgfqpoint{3.013162in}{1.535938in}}{\pgfqpoint{3.023761in}{1.531547in}}{\pgfqpoint{3.034811in}{1.531547in}}%
\pgfpathclose%
\pgfusepath{stroke,fill}%
\end{pgfscope}%
\begin{pgfscope}%
\pgfpathrectangle{\pgfqpoint{0.787074in}{0.548769in}}{\pgfqpoint{5.062926in}{3.102590in}}%
\pgfusepath{clip}%
\pgfsetbuttcap%
\pgfsetroundjoin%
\definecolor{currentfill}{rgb}{1.000000,0.498039,0.054902}%
\pgfsetfillcolor{currentfill}%
\pgfsetlinewidth{1.003750pt}%
\definecolor{currentstroke}{rgb}{1.000000,0.498039,0.054902}%
\pgfsetstrokecolor{currentstroke}%
\pgfsetdash{}{0pt}%
\pgfpathmoveto{\pgfqpoint{1.521608in}{2.472823in}}%
\pgfpathcurveto{\pgfqpoint{1.532658in}{2.472823in}}{\pgfqpoint{1.543257in}{2.477213in}}{\pgfqpoint{1.551071in}{2.485027in}}%
\pgfpathcurveto{\pgfqpoint{1.558884in}{2.492840in}}{\pgfqpoint{1.563275in}{2.503439in}}{\pgfqpoint{1.563275in}{2.514489in}}%
\pgfpathcurveto{\pgfqpoint{1.563275in}{2.525539in}}{\pgfqpoint{1.558884in}{2.536138in}}{\pgfqpoint{1.551071in}{2.543952in}}%
\pgfpathcurveto{\pgfqpoint{1.543257in}{2.551766in}}{\pgfqpoint{1.532658in}{2.556156in}}{\pgfqpoint{1.521608in}{2.556156in}}%
\pgfpathcurveto{\pgfqpoint{1.510558in}{2.556156in}}{\pgfqpoint{1.499959in}{2.551766in}}{\pgfqpoint{1.492145in}{2.543952in}}%
\pgfpathcurveto{\pgfqpoint{1.484332in}{2.536138in}}{\pgfqpoint{1.479941in}{2.525539in}}{\pgfqpoint{1.479941in}{2.514489in}}%
\pgfpathcurveto{\pgfqpoint{1.479941in}{2.503439in}}{\pgfqpoint{1.484332in}{2.492840in}}{\pgfqpoint{1.492145in}{2.485027in}}%
\pgfpathcurveto{\pgfqpoint{1.499959in}{2.477213in}}{\pgfqpoint{1.510558in}{2.472823in}}{\pgfqpoint{1.521608in}{2.472823in}}%
\pgfpathclose%
\pgfusepath{stroke,fill}%
\end{pgfscope}%
\begin{pgfscope}%
\pgfpathrectangle{\pgfqpoint{0.787074in}{0.548769in}}{\pgfqpoint{5.062926in}{3.102590in}}%
\pgfusepath{clip}%
\pgfsetbuttcap%
\pgfsetroundjoin%
\definecolor{currentfill}{rgb}{0.121569,0.466667,0.705882}%
\pgfsetfillcolor{currentfill}%
\pgfsetlinewidth{1.003750pt}%
\definecolor{currentstroke}{rgb}{0.121569,0.466667,0.705882}%
\pgfsetstrokecolor{currentstroke}%
\pgfsetdash{}{0pt}%
\pgfpathmoveto{\pgfqpoint{2.530410in}{2.690255in}}%
\pgfpathcurveto{\pgfqpoint{2.541460in}{2.690255in}}{\pgfqpoint{2.552059in}{2.694645in}}{\pgfqpoint{2.559873in}{2.702459in}}%
\pgfpathcurveto{\pgfqpoint{2.567687in}{2.710273in}}{\pgfqpoint{2.572077in}{2.720872in}}{\pgfqpoint{2.572077in}{2.731922in}}%
\pgfpathcurveto{\pgfqpoint{2.572077in}{2.742972in}}{\pgfqpoint{2.567687in}{2.753571in}}{\pgfqpoint{2.559873in}{2.761384in}}%
\pgfpathcurveto{\pgfqpoint{2.552059in}{2.769198in}}{\pgfqpoint{2.541460in}{2.773588in}}{\pgfqpoint{2.530410in}{2.773588in}}%
\pgfpathcurveto{\pgfqpoint{2.519360in}{2.773588in}}{\pgfqpoint{2.508761in}{2.769198in}}{\pgfqpoint{2.500947in}{2.761384in}}%
\pgfpathcurveto{\pgfqpoint{2.493134in}{2.753571in}}{\pgfqpoint{2.488744in}{2.742972in}}{\pgfqpoint{2.488744in}{2.731922in}}%
\pgfpathcurveto{\pgfqpoint{2.488744in}{2.720872in}}{\pgfqpoint{2.493134in}{2.710273in}}{\pgfqpoint{2.500947in}{2.702459in}}%
\pgfpathcurveto{\pgfqpoint{2.508761in}{2.694645in}}{\pgfqpoint{2.519360in}{2.690255in}}{\pgfqpoint{2.530410in}{2.690255in}}%
\pgfpathclose%
\pgfusepath{stroke,fill}%
\end{pgfscope}%
\begin{pgfscope}%
\pgfpathrectangle{\pgfqpoint{0.787074in}{0.548769in}}{\pgfqpoint{5.062926in}{3.102590in}}%
\pgfusepath{clip}%
\pgfsetbuttcap%
\pgfsetroundjoin%
\definecolor{currentfill}{rgb}{1.000000,0.498039,0.054902}%
\pgfsetfillcolor{currentfill}%
\pgfsetlinewidth{1.003750pt}%
\definecolor{currentstroke}{rgb}{1.000000,0.498039,0.054902}%
\pgfsetstrokecolor{currentstroke}%
\pgfsetdash{}{0pt}%
\pgfpathmoveto{\pgfqpoint{1.143307in}{3.301211in}}%
\pgfpathcurveto{\pgfqpoint{1.154357in}{3.301211in}}{\pgfqpoint{1.164956in}{3.305601in}}{\pgfqpoint{1.172770in}{3.313415in}}%
\pgfpathcurveto{\pgfqpoint{1.180584in}{3.321228in}}{\pgfqpoint{1.184974in}{3.331828in}}{\pgfqpoint{1.184974in}{3.342878in}}%
\pgfpathcurveto{\pgfqpoint{1.184974in}{3.353928in}}{\pgfqpoint{1.180584in}{3.364527in}}{\pgfqpoint{1.172770in}{3.372340in}}%
\pgfpathcurveto{\pgfqpoint{1.164956in}{3.380154in}}{\pgfqpoint{1.154357in}{3.384544in}}{\pgfqpoint{1.143307in}{3.384544in}}%
\pgfpathcurveto{\pgfqpoint{1.132257in}{3.384544in}}{\pgfqpoint{1.121658in}{3.380154in}}{\pgfqpoint{1.113844in}{3.372340in}}%
\pgfpathcurveto{\pgfqpoint{1.106031in}{3.364527in}}{\pgfqpoint{1.101640in}{3.353928in}}{\pgfqpoint{1.101640in}{3.342878in}}%
\pgfpathcurveto{\pgfqpoint{1.101640in}{3.331828in}}{\pgfqpoint{1.106031in}{3.321228in}}{\pgfqpoint{1.113844in}{3.313415in}}%
\pgfpathcurveto{\pgfqpoint{1.121658in}{3.305601in}}{\pgfqpoint{1.132257in}{3.301211in}}{\pgfqpoint{1.143307in}{3.301211in}}%
\pgfpathclose%
\pgfusepath{stroke,fill}%
\end{pgfscope}%
\begin{pgfscope}%
\pgfpathrectangle{\pgfqpoint{0.787074in}{0.548769in}}{\pgfqpoint{5.062926in}{3.102590in}}%
\pgfusepath{clip}%
\pgfsetbuttcap%
\pgfsetroundjoin%
\definecolor{currentfill}{rgb}{0.121569,0.466667,0.705882}%
\pgfsetfillcolor{currentfill}%
\pgfsetlinewidth{1.003750pt}%
\definecolor{currentstroke}{rgb}{0.121569,0.466667,0.705882}%
\pgfsetstrokecolor{currentstroke}%
\pgfsetdash{}{0pt}%
\pgfpathmoveto{\pgfqpoint{2.719561in}{3.101406in}}%
\pgfpathcurveto{\pgfqpoint{2.730611in}{3.101406in}}{\pgfqpoint{2.741210in}{3.105796in}}{\pgfqpoint{2.749023in}{3.113610in}}%
\pgfpathcurveto{\pgfqpoint{2.756837in}{3.121423in}}{\pgfqpoint{2.761227in}{3.132022in}}{\pgfqpoint{2.761227in}{3.143073in}}%
\pgfpathcurveto{\pgfqpoint{2.761227in}{3.154123in}}{\pgfqpoint{2.756837in}{3.164722in}}{\pgfqpoint{2.749023in}{3.172535in}}%
\pgfpathcurveto{\pgfqpoint{2.741210in}{3.180349in}}{\pgfqpoint{2.730611in}{3.184739in}}{\pgfqpoint{2.719561in}{3.184739in}}%
\pgfpathcurveto{\pgfqpoint{2.708510in}{3.184739in}}{\pgfqpoint{2.697911in}{3.180349in}}{\pgfqpoint{2.690098in}{3.172535in}}%
\pgfpathcurveto{\pgfqpoint{2.682284in}{3.164722in}}{\pgfqpoint{2.677894in}{3.154123in}}{\pgfqpoint{2.677894in}{3.143073in}}%
\pgfpathcurveto{\pgfqpoint{2.677894in}{3.132022in}}{\pgfqpoint{2.682284in}{3.121423in}}{\pgfqpoint{2.690098in}{3.113610in}}%
\pgfpathcurveto{\pgfqpoint{2.697911in}{3.105796in}}{\pgfqpoint{2.708510in}{3.101406in}}{\pgfqpoint{2.719561in}{3.101406in}}%
\pgfpathclose%
\pgfusepath{stroke,fill}%
\end{pgfscope}%
\begin{pgfscope}%
\pgfpathrectangle{\pgfqpoint{0.787074in}{0.548769in}}{\pgfqpoint{5.062926in}{3.102590in}}%
\pgfusepath{clip}%
\pgfsetbuttcap%
\pgfsetroundjoin%
\definecolor{currentfill}{rgb}{1.000000,0.498039,0.054902}%
\pgfsetfillcolor{currentfill}%
\pgfsetlinewidth{1.003750pt}%
\definecolor{currentstroke}{rgb}{1.000000,0.498039,0.054902}%
\pgfsetstrokecolor{currentstroke}%
\pgfsetdash{}{0pt}%
\pgfpathmoveto{\pgfqpoint{2.026009in}{1.878073in}}%
\pgfpathcurveto{\pgfqpoint{2.037059in}{1.878073in}}{\pgfqpoint{2.047658in}{1.882464in}}{\pgfqpoint{2.055472in}{1.890277in}}%
\pgfpathcurveto{\pgfqpoint{2.063285in}{1.898091in}}{\pgfqpoint{2.067676in}{1.908690in}}{\pgfqpoint{2.067676in}{1.919740in}}%
\pgfpathcurveto{\pgfqpoint{2.067676in}{1.930790in}}{\pgfqpoint{2.063285in}{1.941389in}}{\pgfqpoint{2.055472in}{1.949203in}}%
\pgfpathcurveto{\pgfqpoint{2.047658in}{1.957016in}}{\pgfqpoint{2.037059in}{1.961407in}}{\pgfqpoint{2.026009in}{1.961407in}}%
\pgfpathcurveto{\pgfqpoint{2.014959in}{1.961407in}}{\pgfqpoint{2.004360in}{1.957016in}}{\pgfqpoint{1.996546in}{1.949203in}}%
\pgfpathcurveto{\pgfqpoint{1.988733in}{1.941389in}}{\pgfqpoint{1.984342in}{1.930790in}}{\pgfqpoint{1.984342in}{1.919740in}}%
\pgfpathcurveto{\pgfqpoint{1.984342in}{1.908690in}}{\pgfqpoint{1.988733in}{1.898091in}}{\pgfqpoint{1.996546in}{1.890277in}}%
\pgfpathcurveto{\pgfqpoint{2.004360in}{1.882464in}}{\pgfqpoint{2.014959in}{1.878073in}}{\pgfqpoint{2.026009in}{1.878073in}}%
\pgfpathclose%
\pgfusepath{stroke,fill}%
\end{pgfscope}%
\begin{pgfscope}%
\pgfpathrectangle{\pgfqpoint{0.787074in}{0.548769in}}{\pgfqpoint{5.062926in}{3.102590in}}%
\pgfusepath{clip}%
\pgfsetbuttcap%
\pgfsetroundjoin%
\definecolor{currentfill}{rgb}{1.000000,0.498039,0.054902}%
\pgfsetfillcolor{currentfill}%
\pgfsetlinewidth{1.003750pt}%
\definecolor{currentstroke}{rgb}{1.000000,0.498039,0.054902}%
\pgfsetstrokecolor{currentstroke}%
\pgfsetdash{}{0pt}%
\pgfpathmoveto{\pgfqpoint{1.395508in}{2.761817in}}%
\pgfpathcurveto{\pgfqpoint{1.406558in}{2.761817in}}{\pgfqpoint{1.417157in}{2.766208in}}{\pgfqpoint{1.424970in}{2.774021in}}%
\pgfpathcurveto{\pgfqpoint{1.432784in}{2.781835in}}{\pgfqpoint{1.437174in}{2.792434in}}{\pgfqpoint{1.437174in}{2.803484in}}%
\pgfpathcurveto{\pgfqpoint{1.437174in}{2.814534in}}{\pgfqpoint{1.432784in}{2.825133in}}{\pgfqpoint{1.424970in}{2.832947in}}%
\pgfpathcurveto{\pgfqpoint{1.417157in}{2.840760in}}{\pgfqpoint{1.406558in}{2.845151in}}{\pgfqpoint{1.395508in}{2.845151in}}%
\pgfpathcurveto{\pgfqpoint{1.384458in}{2.845151in}}{\pgfqpoint{1.373859in}{2.840760in}}{\pgfqpoint{1.366045in}{2.832947in}}%
\pgfpathcurveto{\pgfqpoint{1.358231in}{2.825133in}}{\pgfqpoint{1.353841in}{2.814534in}}{\pgfqpoint{1.353841in}{2.803484in}}%
\pgfpathcurveto{\pgfqpoint{1.353841in}{2.792434in}}{\pgfqpoint{1.358231in}{2.781835in}}{\pgfqpoint{1.366045in}{2.774021in}}%
\pgfpathcurveto{\pgfqpoint{1.373859in}{2.766208in}}{\pgfqpoint{1.384458in}{2.761817in}}{\pgfqpoint{1.395508in}{2.761817in}}%
\pgfpathclose%
\pgfusepath{stroke,fill}%
\end{pgfscope}%
\begin{pgfscope}%
\pgfpathrectangle{\pgfqpoint{0.787074in}{0.548769in}}{\pgfqpoint{5.062926in}{3.102590in}}%
\pgfusepath{clip}%
\pgfsetbuttcap%
\pgfsetroundjoin%
\definecolor{currentfill}{rgb}{1.000000,0.498039,0.054902}%
\pgfsetfillcolor{currentfill}%
\pgfsetlinewidth{1.003750pt}%
\definecolor{currentstroke}{rgb}{1.000000,0.498039,0.054902}%
\pgfsetstrokecolor{currentstroke}%
\pgfsetdash{}{0pt}%
\pgfpathmoveto{\pgfqpoint{2.845661in}{2.075959in}}%
\pgfpathcurveto{\pgfqpoint{2.856711in}{2.075959in}}{\pgfqpoint{2.867310in}{2.080350in}}{\pgfqpoint{2.875124in}{2.088163in}}%
\pgfpathcurveto{\pgfqpoint{2.882937in}{2.095977in}}{\pgfqpoint{2.887328in}{2.106576in}}{\pgfqpoint{2.887328in}{2.117626in}}%
\pgfpathcurveto{\pgfqpoint{2.887328in}{2.128676in}}{\pgfqpoint{2.882937in}{2.139275in}}{\pgfqpoint{2.875124in}{2.147089in}}%
\pgfpathcurveto{\pgfqpoint{2.867310in}{2.154902in}}{\pgfqpoint{2.856711in}{2.159293in}}{\pgfqpoint{2.845661in}{2.159293in}}%
\pgfpathcurveto{\pgfqpoint{2.834611in}{2.159293in}}{\pgfqpoint{2.824012in}{2.154902in}}{\pgfqpoint{2.816198in}{2.147089in}}%
\pgfpathcurveto{\pgfqpoint{2.808384in}{2.139275in}}{\pgfqpoint{2.803994in}{2.128676in}}{\pgfqpoint{2.803994in}{2.117626in}}%
\pgfpathcurveto{\pgfqpoint{2.803994in}{2.106576in}}{\pgfqpoint{2.808384in}{2.095977in}}{\pgfqpoint{2.816198in}{2.088163in}}%
\pgfpathcurveto{\pgfqpoint{2.824012in}{2.080350in}}{\pgfqpoint{2.834611in}{2.075959in}}{\pgfqpoint{2.845661in}{2.075959in}}%
\pgfpathclose%
\pgfusepath{stroke,fill}%
\end{pgfscope}%
\begin{pgfscope}%
\pgfpathrectangle{\pgfqpoint{0.787074in}{0.548769in}}{\pgfqpoint{5.062926in}{3.102590in}}%
\pgfusepath{clip}%
\pgfsetbuttcap%
\pgfsetroundjoin%
\definecolor{currentfill}{rgb}{1.000000,0.498039,0.054902}%
\pgfsetfillcolor{currentfill}%
\pgfsetlinewidth{1.003750pt}%
\definecolor{currentstroke}{rgb}{1.000000,0.498039,0.054902}%
\pgfsetstrokecolor{currentstroke}%
\pgfsetdash{}{0pt}%
\pgfpathmoveto{\pgfqpoint{2.908711in}{2.369584in}}%
\pgfpathcurveto{\pgfqpoint{2.919761in}{2.369584in}}{\pgfqpoint{2.930360in}{2.373975in}}{\pgfqpoint{2.938174in}{2.381788in}}%
\pgfpathcurveto{\pgfqpoint{2.945987in}{2.389602in}}{\pgfqpoint{2.950378in}{2.400201in}}{\pgfqpoint{2.950378in}{2.411251in}}%
\pgfpathcurveto{\pgfqpoint{2.950378in}{2.422301in}}{\pgfqpoint{2.945987in}{2.432900in}}{\pgfqpoint{2.938174in}{2.440714in}}%
\pgfpathcurveto{\pgfqpoint{2.930360in}{2.448527in}}{\pgfqpoint{2.919761in}{2.452918in}}{\pgfqpoint{2.908711in}{2.452918in}}%
\pgfpathcurveto{\pgfqpoint{2.897661in}{2.452918in}}{\pgfqpoint{2.887062in}{2.448527in}}{\pgfqpoint{2.879248in}{2.440714in}}%
\pgfpathcurveto{\pgfqpoint{2.871435in}{2.432900in}}{\pgfqpoint{2.867044in}{2.422301in}}{\pgfqpoint{2.867044in}{2.411251in}}%
\pgfpathcurveto{\pgfqpoint{2.867044in}{2.400201in}}{\pgfqpoint{2.871435in}{2.389602in}}{\pgfqpoint{2.879248in}{2.381788in}}%
\pgfpathcurveto{\pgfqpoint{2.887062in}{2.373975in}}{\pgfqpoint{2.897661in}{2.369584in}}{\pgfqpoint{2.908711in}{2.369584in}}%
\pgfpathclose%
\pgfusepath{stroke,fill}%
\end{pgfscope}%
\begin{pgfscope}%
\pgfpathrectangle{\pgfqpoint{0.787074in}{0.548769in}}{\pgfqpoint{5.062926in}{3.102590in}}%
\pgfusepath{clip}%
\pgfsetbuttcap%
\pgfsetroundjoin%
\definecolor{currentfill}{rgb}{0.121569,0.466667,0.705882}%
\pgfsetfillcolor{currentfill}%
\pgfsetlinewidth{1.003750pt}%
\definecolor{currentstroke}{rgb}{0.121569,0.466667,0.705882}%
\pgfsetstrokecolor{currentstroke}%
\pgfsetdash{}{0pt}%
\pgfpathmoveto{\pgfqpoint{1.773809in}{0.648131in}}%
\pgfpathcurveto{\pgfqpoint{1.784859in}{0.648131in}}{\pgfqpoint{1.795458in}{0.652521in}}{\pgfqpoint{1.803271in}{0.660335in}}%
\pgfpathcurveto{\pgfqpoint{1.811085in}{0.668148in}}{\pgfqpoint{1.815475in}{0.678747in}}{\pgfqpoint{1.815475in}{0.689798in}}%
\pgfpathcurveto{\pgfqpoint{1.815475in}{0.700848in}}{\pgfqpoint{1.811085in}{0.711447in}}{\pgfqpoint{1.803271in}{0.719260in}}%
\pgfpathcurveto{\pgfqpoint{1.795458in}{0.727074in}}{\pgfqpoint{1.784859in}{0.731464in}}{\pgfqpoint{1.773809in}{0.731464in}}%
\pgfpathcurveto{\pgfqpoint{1.762758in}{0.731464in}}{\pgfqpoint{1.752159in}{0.727074in}}{\pgfqpoint{1.744346in}{0.719260in}}%
\pgfpathcurveto{\pgfqpoint{1.736532in}{0.711447in}}{\pgfqpoint{1.732142in}{0.700848in}}{\pgfqpoint{1.732142in}{0.689798in}}%
\pgfpathcurveto{\pgfqpoint{1.732142in}{0.678747in}}{\pgfqpoint{1.736532in}{0.668148in}}{\pgfqpoint{1.744346in}{0.660335in}}%
\pgfpathcurveto{\pgfqpoint{1.752159in}{0.652521in}}{\pgfqpoint{1.762758in}{0.648131in}}{\pgfqpoint{1.773809in}{0.648131in}}%
\pgfpathclose%
\pgfusepath{stroke,fill}%
\end{pgfscope}%
\begin{pgfscope}%
\pgfpathrectangle{\pgfqpoint{0.787074in}{0.548769in}}{\pgfqpoint{5.062926in}{3.102590in}}%
\pgfusepath{clip}%
\pgfsetbuttcap%
\pgfsetroundjoin%
\definecolor{currentfill}{rgb}{1.000000,0.498039,0.054902}%
\pgfsetfillcolor{currentfill}%
\pgfsetlinewidth{1.003750pt}%
\definecolor{currentstroke}{rgb}{1.000000,0.498039,0.054902}%
\pgfsetstrokecolor{currentstroke}%
\pgfsetdash{}{0pt}%
\pgfpathmoveto{\pgfqpoint{1.836859in}{1.884932in}}%
\pgfpathcurveto{\pgfqpoint{1.847909in}{1.884932in}}{\pgfqpoint{1.858508in}{1.889322in}}{\pgfqpoint{1.866321in}{1.897136in}}%
\pgfpathcurveto{\pgfqpoint{1.874135in}{1.904950in}}{\pgfqpoint{1.878525in}{1.915549in}}{\pgfqpoint{1.878525in}{1.926599in}}%
\pgfpathcurveto{\pgfqpoint{1.878525in}{1.937649in}}{\pgfqpoint{1.874135in}{1.948248in}}{\pgfqpoint{1.866321in}{1.956061in}}%
\pgfpathcurveto{\pgfqpoint{1.858508in}{1.963875in}}{\pgfqpoint{1.847909in}{1.968265in}}{\pgfqpoint{1.836859in}{1.968265in}}%
\pgfpathcurveto{\pgfqpoint{1.825809in}{1.968265in}}{\pgfqpoint{1.815209in}{1.963875in}}{\pgfqpoint{1.807396in}{1.956061in}}%
\pgfpathcurveto{\pgfqpoint{1.799582in}{1.948248in}}{\pgfqpoint{1.795192in}{1.937649in}}{\pgfqpoint{1.795192in}{1.926599in}}%
\pgfpathcurveto{\pgfqpoint{1.795192in}{1.915549in}}{\pgfqpoint{1.799582in}{1.904950in}}{\pgfqpoint{1.807396in}{1.897136in}}%
\pgfpathcurveto{\pgfqpoint{1.815209in}{1.889322in}}{\pgfqpoint{1.825809in}{1.884932in}}{\pgfqpoint{1.836859in}{1.884932in}}%
\pgfpathclose%
\pgfusepath{stroke,fill}%
\end{pgfscope}%
\begin{pgfscope}%
\pgfpathrectangle{\pgfqpoint{0.787074in}{0.548769in}}{\pgfqpoint{5.062926in}{3.102590in}}%
\pgfusepath{clip}%
\pgfsetbuttcap%
\pgfsetroundjoin%
\definecolor{currentfill}{rgb}{1.000000,0.498039,0.054902}%
\pgfsetfillcolor{currentfill}%
\pgfsetlinewidth{1.003750pt}%
\definecolor{currentstroke}{rgb}{1.000000,0.498039,0.054902}%
\pgfsetstrokecolor{currentstroke}%
\pgfsetdash{}{0pt}%
\pgfpathmoveto{\pgfqpoint{1.017207in}{1.779524in}}%
\pgfpathcurveto{\pgfqpoint{1.028257in}{1.779524in}}{\pgfqpoint{1.038856in}{1.783914in}}{\pgfqpoint{1.046670in}{1.791728in}}%
\pgfpathcurveto{\pgfqpoint{1.054483in}{1.799541in}}{\pgfqpoint{1.058874in}{1.810140in}}{\pgfqpoint{1.058874in}{1.821190in}}%
\pgfpathcurveto{\pgfqpoint{1.058874in}{1.832241in}}{\pgfqpoint{1.054483in}{1.842840in}}{\pgfqpoint{1.046670in}{1.850653in}}%
\pgfpathcurveto{\pgfqpoint{1.038856in}{1.858467in}}{\pgfqpoint{1.028257in}{1.862857in}}{\pgfqpoint{1.017207in}{1.862857in}}%
\pgfpathcurveto{\pgfqpoint{1.006157in}{1.862857in}}{\pgfqpoint{0.995558in}{1.858467in}}{\pgfqpoint{0.987744in}{1.850653in}}%
\pgfpathcurveto{\pgfqpoint{0.979930in}{1.842840in}}{\pgfqpoint{0.975540in}{1.832241in}}{\pgfqpoint{0.975540in}{1.821190in}}%
\pgfpathcurveto{\pgfqpoint{0.975540in}{1.810140in}}{\pgfqpoint{0.979930in}{1.799541in}}{\pgfqpoint{0.987744in}{1.791728in}}%
\pgfpathcurveto{\pgfqpoint{0.995558in}{1.783914in}}{\pgfqpoint{1.006157in}{1.779524in}}{\pgfqpoint{1.017207in}{1.779524in}}%
\pgfpathclose%
\pgfusepath{stroke,fill}%
\end{pgfscope}%
\begin{pgfscope}%
\pgfpathrectangle{\pgfqpoint{0.787074in}{0.548769in}}{\pgfqpoint{5.062926in}{3.102590in}}%
\pgfusepath{clip}%
\pgfsetbuttcap%
\pgfsetroundjoin%
\definecolor{currentfill}{rgb}{1.000000,0.498039,0.054902}%
\pgfsetfillcolor{currentfill}%
\pgfsetlinewidth{1.003750pt}%
\definecolor{currentstroke}{rgb}{1.000000,0.498039,0.054902}%
\pgfsetstrokecolor{currentstroke}%
\pgfsetdash{}{0pt}%
\pgfpathmoveto{\pgfqpoint{1.269407in}{2.435775in}}%
\pgfpathcurveto{\pgfqpoint{1.280458in}{2.435775in}}{\pgfqpoint{1.291057in}{2.440165in}}{\pgfqpoint{1.298870in}{2.447979in}}%
\pgfpathcurveto{\pgfqpoint{1.306684in}{2.455792in}}{\pgfqpoint{1.311074in}{2.466391in}}{\pgfqpoint{1.311074in}{2.477441in}}%
\pgfpathcurveto{\pgfqpoint{1.311074in}{2.488492in}}{\pgfqpoint{1.306684in}{2.499091in}}{\pgfqpoint{1.298870in}{2.506904in}}%
\pgfpathcurveto{\pgfqpoint{1.291057in}{2.514718in}}{\pgfqpoint{1.280458in}{2.519108in}}{\pgfqpoint{1.269407in}{2.519108in}}%
\pgfpathcurveto{\pgfqpoint{1.258357in}{2.519108in}}{\pgfqpoint{1.247758in}{2.514718in}}{\pgfqpoint{1.239945in}{2.506904in}}%
\pgfpathcurveto{\pgfqpoint{1.232131in}{2.499091in}}{\pgfqpoint{1.227741in}{2.488492in}}{\pgfqpoint{1.227741in}{2.477441in}}%
\pgfpathcurveto{\pgfqpoint{1.227741in}{2.466391in}}{\pgfqpoint{1.232131in}{2.455792in}}{\pgfqpoint{1.239945in}{2.447979in}}%
\pgfpathcurveto{\pgfqpoint{1.247758in}{2.440165in}}{\pgfqpoint{1.258357in}{2.435775in}}{\pgfqpoint{1.269407in}{2.435775in}}%
\pgfpathclose%
\pgfusepath{stroke,fill}%
\end{pgfscope}%
\begin{pgfscope}%
\pgfpathrectangle{\pgfqpoint{0.787074in}{0.548769in}}{\pgfqpoint{5.062926in}{3.102590in}}%
\pgfusepath{clip}%
\pgfsetbuttcap%
\pgfsetroundjoin%
\definecolor{currentfill}{rgb}{1.000000,0.498039,0.054902}%
\pgfsetfillcolor{currentfill}%
\pgfsetlinewidth{1.003750pt}%
\definecolor{currentstroke}{rgb}{1.000000,0.498039,0.054902}%
\pgfsetstrokecolor{currentstroke}%
\pgfsetdash{}{0pt}%
\pgfpathmoveto{\pgfqpoint{2.215159in}{1.639934in}}%
\pgfpathcurveto{\pgfqpoint{2.226210in}{1.639934in}}{\pgfqpoint{2.236809in}{1.644324in}}{\pgfqpoint{2.244622in}{1.652138in}}%
\pgfpathcurveto{\pgfqpoint{2.252436in}{1.659951in}}{\pgfqpoint{2.256826in}{1.670550in}}{\pgfqpoint{2.256826in}{1.681600in}}%
\pgfpathcurveto{\pgfqpoint{2.256826in}{1.692650in}}{\pgfqpoint{2.252436in}{1.703249in}}{\pgfqpoint{2.244622in}{1.711063in}}%
\pgfpathcurveto{\pgfqpoint{2.236809in}{1.718877in}}{\pgfqpoint{2.226210in}{1.723267in}}{\pgfqpoint{2.215159in}{1.723267in}}%
\pgfpathcurveto{\pgfqpoint{2.204109in}{1.723267in}}{\pgfqpoint{2.193510in}{1.718877in}}{\pgfqpoint{2.185697in}{1.711063in}}%
\pgfpathcurveto{\pgfqpoint{2.177883in}{1.703249in}}{\pgfqpoint{2.173493in}{1.692650in}}{\pgfqpoint{2.173493in}{1.681600in}}%
\pgfpathcurveto{\pgfqpoint{2.173493in}{1.670550in}}{\pgfqpoint{2.177883in}{1.659951in}}{\pgfqpoint{2.185697in}{1.652138in}}%
\pgfpathcurveto{\pgfqpoint{2.193510in}{1.644324in}}{\pgfqpoint{2.204109in}{1.639934in}}{\pgfqpoint{2.215159in}{1.639934in}}%
\pgfpathclose%
\pgfusepath{stroke,fill}%
\end{pgfscope}%
\begin{pgfscope}%
\pgfpathrectangle{\pgfqpoint{0.787074in}{0.548769in}}{\pgfqpoint{5.062926in}{3.102590in}}%
\pgfusepath{clip}%
\pgfsetbuttcap%
\pgfsetroundjoin%
\definecolor{currentfill}{rgb}{0.121569,0.466667,0.705882}%
\pgfsetfillcolor{currentfill}%
\pgfsetlinewidth{1.003750pt}%
\definecolor{currentstroke}{rgb}{0.121569,0.466667,0.705882}%
\pgfsetstrokecolor{currentstroke}%
\pgfsetdash{}{0pt}%
\pgfpathmoveto{\pgfqpoint{2.152109in}{1.770785in}}%
\pgfpathcurveto{\pgfqpoint{2.163159in}{1.770785in}}{\pgfqpoint{2.173759in}{1.775176in}}{\pgfqpoint{2.181572in}{1.782989in}}%
\pgfpathcurveto{\pgfqpoint{2.189386in}{1.790803in}}{\pgfqpoint{2.193776in}{1.801402in}}{\pgfqpoint{2.193776in}{1.812452in}}%
\pgfpathcurveto{\pgfqpoint{2.193776in}{1.823502in}}{\pgfqpoint{2.189386in}{1.834101in}}{\pgfqpoint{2.181572in}{1.841915in}}%
\pgfpathcurveto{\pgfqpoint{2.173759in}{1.849728in}}{\pgfqpoint{2.163159in}{1.854119in}}{\pgfqpoint{2.152109in}{1.854119in}}%
\pgfpathcurveto{\pgfqpoint{2.141059in}{1.854119in}}{\pgfqpoint{2.130460in}{1.849728in}}{\pgfqpoint{2.122647in}{1.841915in}}%
\pgfpathcurveto{\pgfqpoint{2.114833in}{1.834101in}}{\pgfqpoint{2.110443in}{1.823502in}}{\pgfqpoint{2.110443in}{1.812452in}}%
\pgfpathcurveto{\pgfqpoint{2.110443in}{1.801402in}}{\pgfqpoint{2.114833in}{1.790803in}}{\pgfqpoint{2.122647in}{1.782989in}}%
\pgfpathcurveto{\pgfqpoint{2.130460in}{1.775176in}}{\pgfqpoint{2.141059in}{1.770785in}}{\pgfqpoint{2.152109in}{1.770785in}}%
\pgfpathclose%
\pgfusepath{stroke,fill}%
\end{pgfscope}%
\begin{pgfscope}%
\pgfpathrectangle{\pgfqpoint{0.787074in}{0.548769in}}{\pgfqpoint{5.062926in}{3.102590in}}%
\pgfusepath{clip}%
\pgfsetbuttcap%
\pgfsetroundjoin%
\definecolor{currentfill}{rgb}{1.000000,0.498039,0.054902}%
\pgfsetfillcolor{currentfill}%
\pgfsetlinewidth{1.003750pt}%
\definecolor{currentstroke}{rgb}{1.000000,0.498039,0.054902}%
\pgfsetstrokecolor{currentstroke}%
\pgfsetdash{}{0pt}%
\pgfpathmoveto{\pgfqpoint{2.152109in}{2.028373in}}%
\pgfpathcurveto{\pgfqpoint{2.163159in}{2.028373in}}{\pgfqpoint{2.173759in}{2.032763in}}{\pgfqpoint{2.181572in}{2.040577in}}%
\pgfpathcurveto{\pgfqpoint{2.189386in}{2.048390in}}{\pgfqpoint{2.193776in}{2.058989in}}{\pgfqpoint{2.193776in}{2.070039in}}%
\pgfpathcurveto{\pgfqpoint{2.193776in}{2.081090in}}{\pgfqpoint{2.189386in}{2.091689in}}{\pgfqpoint{2.181572in}{2.099502in}}%
\pgfpathcurveto{\pgfqpoint{2.173759in}{2.107316in}}{\pgfqpoint{2.163159in}{2.111706in}}{\pgfqpoint{2.152109in}{2.111706in}}%
\pgfpathcurveto{\pgfqpoint{2.141059in}{2.111706in}}{\pgfqpoint{2.130460in}{2.107316in}}{\pgfqpoint{2.122647in}{2.099502in}}%
\pgfpathcurveto{\pgfqpoint{2.114833in}{2.091689in}}{\pgfqpoint{2.110443in}{2.081090in}}{\pgfqpoint{2.110443in}{2.070039in}}%
\pgfpathcurveto{\pgfqpoint{2.110443in}{2.058989in}}{\pgfqpoint{2.114833in}{2.048390in}}{\pgfqpoint{2.122647in}{2.040577in}}%
\pgfpathcurveto{\pgfqpoint{2.130460in}{2.032763in}}{\pgfqpoint{2.141059in}{2.028373in}}{\pgfqpoint{2.152109in}{2.028373in}}%
\pgfpathclose%
\pgfusepath{stroke,fill}%
\end{pgfscope}%
\begin{pgfscope}%
\pgfpathrectangle{\pgfqpoint{0.787074in}{0.548769in}}{\pgfqpoint{5.062926in}{3.102590in}}%
\pgfusepath{clip}%
\pgfsetbuttcap%
\pgfsetroundjoin%
\definecolor{currentfill}{rgb}{1.000000,0.498039,0.054902}%
\pgfsetfillcolor{currentfill}%
\pgfsetlinewidth{1.003750pt}%
\definecolor{currentstroke}{rgb}{1.000000,0.498039,0.054902}%
\pgfsetstrokecolor{currentstroke}%
\pgfsetdash{}{0pt}%
\pgfpathmoveto{\pgfqpoint{3.097861in}{2.904196in}}%
\pgfpathcurveto{\pgfqpoint{3.108912in}{2.904196in}}{\pgfqpoint{3.119511in}{2.908586in}}{\pgfqpoint{3.127324in}{2.916399in}}%
\pgfpathcurveto{\pgfqpoint{3.135138in}{2.924213in}}{\pgfqpoint{3.139528in}{2.934812in}}{\pgfqpoint{3.139528in}{2.945862in}}%
\pgfpathcurveto{\pgfqpoint{3.139528in}{2.956912in}}{\pgfqpoint{3.135138in}{2.967511in}}{\pgfqpoint{3.127324in}{2.975325in}}%
\pgfpathcurveto{\pgfqpoint{3.119511in}{2.983139in}}{\pgfqpoint{3.108912in}{2.987529in}}{\pgfqpoint{3.097861in}{2.987529in}}%
\pgfpathcurveto{\pgfqpoint{3.086811in}{2.987529in}}{\pgfqpoint{3.076212in}{2.983139in}}{\pgfqpoint{3.068399in}{2.975325in}}%
\pgfpathcurveto{\pgfqpoint{3.060585in}{2.967511in}}{\pgfqpoint{3.056195in}{2.956912in}}{\pgfqpoint{3.056195in}{2.945862in}}%
\pgfpathcurveto{\pgfqpoint{3.056195in}{2.934812in}}{\pgfqpoint{3.060585in}{2.924213in}}{\pgfqpoint{3.068399in}{2.916399in}}%
\pgfpathcurveto{\pgfqpoint{3.076212in}{2.908586in}}{\pgfqpoint{3.086811in}{2.904196in}}{\pgfqpoint{3.097861in}{2.904196in}}%
\pgfpathclose%
\pgfusepath{stroke,fill}%
\end{pgfscope}%
\begin{pgfscope}%
\pgfpathrectangle{\pgfqpoint{0.787074in}{0.548769in}}{\pgfqpoint{5.062926in}{3.102590in}}%
\pgfusepath{clip}%
\pgfsetbuttcap%
\pgfsetroundjoin%
\definecolor{currentfill}{rgb}{1.000000,0.498039,0.054902}%
\pgfsetfillcolor{currentfill}%
\pgfsetlinewidth{1.003750pt}%
\definecolor{currentstroke}{rgb}{1.000000,0.498039,0.054902}%
\pgfsetstrokecolor{currentstroke}%
\pgfsetdash{}{0pt}%
\pgfpathmoveto{\pgfqpoint{1.395508in}{3.239789in}}%
\pgfpathcurveto{\pgfqpoint{1.406558in}{3.239789in}}{\pgfqpoint{1.417157in}{3.244179in}}{\pgfqpoint{1.424970in}{3.251992in}}%
\pgfpathcurveto{\pgfqpoint{1.432784in}{3.259806in}}{\pgfqpoint{1.437174in}{3.270405in}}{\pgfqpoint{1.437174in}{3.281455in}}%
\pgfpathcurveto{\pgfqpoint{1.437174in}{3.292505in}}{\pgfqpoint{1.432784in}{3.303104in}}{\pgfqpoint{1.424970in}{3.310918in}}%
\pgfpathcurveto{\pgfqpoint{1.417157in}{3.318732in}}{\pgfqpoint{1.406558in}{3.323122in}}{\pgfqpoint{1.395508in}{3.323122in}}%
\pgfpathcurveto{\pgfqpoint{1.384458in}{3.323122in}}{\pgfqpoint{1.373859in}{3.318732in}}{\pgfqpoint{1.366045in}{3.310918in}}%
\pgfpathcurveto{\pgfqpoint{1.358231in}{3.303104in}}{\pgfqpoint{1.353841in}{3.292505in}}{\pgfqpoint{1.353841in}{3.281455in}}%
\pgfpathcurveto{\pgfqpoint{1.353841in}{3.270405in}}{\pgfqpoint{1.358231in}{3.259806in}}{\pgfqpoint{1.366045in}{3.251992in}}%
\pgfpathcurveto{\pgfqpoint{1.373859in}{3.244179in}}{\pgfqpoint{1.384458in}{3.239789in}}{\pgfqpoint{1.395508in}{3.239789in}}%
\pgfpathclose%
\pgfusepath{stroke,fill}%
\end{pgfscope}%
\begin{pgfscope}%
\pgfsetbuttcap%
\pgfsetroundjoin%
\definecolor{currentfill}{rgb}{0.000000,0.000000,0.000000}%
\pgfsetfillcolor{currentfill}%
\pgfsetlinewidth{0.803000pt}%
\definecolor{currentstroke}{rgb}{0.000000,0.000000,0.000000}%
\pgfsetstrokecolor{currentstroke}%
\pgfsetdash{}{0pt}%
\pgfsys@defobject{currentmarker}{\pgfqpoint{0.000000in}{-0.048611in}}{\pgfqpoint{0.000000in}{0.000000in}}{%
\pgfpathmoveto{\pgfqpoint{0.000000in}{0.000000in}}%
\pgfpathlineto{\pgfqpoint{0.000000in}{-0.048611in}}%
\pgfusepath{stroke,fill}%
}%
\begin{pgfscope}%
\pgfsys@transformshift{0.954157in}{0.548769in}%
\pgfsys@useobject{currentmarker}{}%
\end{pgfscope}%
\end{pgfscope}%
\begin{pgfscope}%
\definecolor{textcolor}{rgb}{0.000000,0.000000,0.000000}%
\pgfsetstrokecolor{textcolor}%
\pgfsetfillcolor{textcolor}%
\pgftext[x=0.954157in,y=0.451547in,,top]{\color{textcolor}\sffamily\fontsize{10.000000}{12.000000}\selectfont \(\displaystyle {0}\)}%
\end{pgfscope}%
\begin{pgfscope}%
\pgfsetbuttcap%
\pgfsetroundjoin%
\definecolor{currentfill}{rgb}{0.000000,0.000000,0.000000}%
\pgfsetfillcolor{currentfill}%
\pgfsetlinewidth{0.803000pt}%
\definecolor{currentstroke}{rgb}{0.000000,0.000000,0.000000}%
\pgfsetstrokecolor{currentstroke}%
\pgfsetdash{}{0pt}%
\pgfsys@defobject{currentmarker}{\pgfqpoint{0.000000in}{-0.048611in}}{\pgfqpoint{0.000000in}{0.000000in}}{%
\pgfpathmoveto{\pgfqpoint{0.000000in}{0.000000in}}%
\pgfpathlineto{\pgfqpoint{0.000000in}{-0.048611in}}%
\pgfusepath{stroke,fill}%
}%
\begin{pgfscope}%
\pgfsys@transformshift{1.584658in}{0.548769in}%
\pgfsys@useobject{currentmarker}{}%
\end{pgfscope}%
\end{pgfscope}%
\begin{pgfscope}%
\definecolor{textcolor}{rgb}{0.000000,0.000000,0.000000}%
\pgfsetstrokecolor{textcolor}%
\pgfsetfillcolor{textcolor}%
\pgftext[x=1.584658in,y=0.451547in,,top]{\color{textcolor}\sffamily\fontsize{10.000000}{12.000000}\selectfont \(\displaystyle {10}\)}%
\end{pgfscope}%
\begin{pgfscope}%
\pgfsetbuttcap%
\pgfsetroundjoin%
\definecolor{currentfill}{rgb}{0.000000,0.000000,0.000000}%
\pgfsetfillcolor{currentfill}%
\pgfsetlinewidth{0.803000pt}%
\definecolor{currentstroke}{rgb}{0.000000,0.000000,0.000000}%
\pgfsetstrokecolor{currentstroke}%
\pgfsetdash{}{0pt}%
\pgfsys@defobject{currentmarker}{\pgfqpoint{0.000000in}{-0.048611in}}{\pgfqpoint{0.000000in}{0.000000in}}{%
\pgfpathmoveto{\pgfqpoint{0.000000in}{0.000000in}}%
\pgfpathlineto{\pgfqpoint{0.000000in}{-0.048611in}}%
\pgfusepath{stroke,fill}%
}%
\begin{pgfscope}%
\pgfsys@transformshift{2.215159in}{0.548769in}%
\pgfsys@useobject{currentmarker}{}%
\end{pgfscope}%
\end{pgfscope}%
\begin{pgfscope}%
\definecolor{textcolor}{rgb}{0.000000,0.000000,0.000000}%
\pgfsetstrokecolor{textcolor}%
\pgfsetfillcolor{textcolor}%
\pgftext[x=2.215159in,y=0.451547in,,top]{\color{textcolor}\sffamily\fontsize{10.000000}{12.000000}\selectfont \(\displaystyle {20}\)}%
\end{pgfscope}%
\begin{pgfscope}%
\pgfsetbuttcap%
\pgfsetroundjoin%
\definecolor{currentfill}{rgb}{0.000000,0.000000,0.000000}%
\pgfsetfillcolor{currentfill}%
\pgfsetlinewidth{0.803000pt}%
\definecolor{currentstroke}{rgb}{0.000000,0.000000,0.000000}%
\pgfsetstrokecolor{currentstroke}%
\pgfsetdash{}{0pt}%
\pgfsys@defobject{currentmarker}{\pgfqpoint{0.000000in}{-0.048611in}}{\pgfqpoint{0.000000in}{0.000000in}}{%
\pgfpathmoveto{\pgfqpoint{0.000000in}{0.000000in}}%
\pgfpathlineto{\pgfqpoint{0.000000in}{-0.048611in}}%
\pgfusepath{stroke,fill}%
}%
\begin{pgfscope}%
\pgfsys@transformshift{2.845661in}{0.548769in}%
\pgfsys@useobject{currentmarker}{}%
\end{pgfscope}%
\end{pgfscope}%
\begin{pgfscope}%
\definecolor{textcolor}{rgb}{0.000000,0.000000,0.000000}%
\pgfsetstrokecolor{textcolor}%
\pgfsetfillcolor{textcolor}%
\pgftext[x=2.845661in,y=0.451547in,,top]{\color{textcolor}\sffamily\fontsize{10.000000}{12.000000}\selectfont \(\displaystyle {30}\)}%
\end{pgfscope}%
\begin{pgfscope}%
\pgfsetbuttcap%
\pgfsetroundjoin%
\definecolor{currentfill}{rgb}{0.000000,0.000000,0.000000}%
\pgfsetfillcolor{currentfill}%
\pgfsetlinewidth{0.803000pt}%
\definecolor{currentstroke}{rgb}{0.000000,0.000000,0.000000}%
\pgfsetstrokecolor{currentstroke}%
\pgfsetdash{}{0pt}%
\pgfsys@defobject{currentmarker}{\pgfqpoint{0.000000in}{-0.048611in}}{\pgfqpoint{0.000000in}{0.000000in}}{%
\pgfpathmoveto{\pgfqpoint{0.000000in}{0.000000in}}%
\pgfpathlineto{\pgfqpoint{0.000000in}{-0.048611in}}%
\pgfusepath{stroke,fill}%
}%
\begin{pgfscope}%
\pgfsys@transformshift{3.476162in}{0.548769in}%
\pgfsys@useobject{currentmarker}{}%
\end{pgfscope}%
\end{pgfscope}%
\begin{pgfscope}%
\definecolor{textcolor}{rgb}{0.000000,0.000000,0.000000}%
\pgfsetstrokecolor{textcolor}%
\pgfsetfillcolor{textcolor}%
\pgftext[x=3.476162in,y=0.451547in,,top]{\color{textcolor}\sffamily\fontsize{10.000000}{12.000000}\selectfont \(\displaystyle {40}\)}%
\end{pgfscope}%
\begin{pgfscope}%
\pgfsetbuttcap%
\pgfsetroundjoin%
\definecolor{currentfill}{rgb}{0.000000,0.000000,0.000000}%
\pgfsetfillcolor{currentfill}%
\pgfsetlinewidth{0.803000pt}%
\definecolor{currentstroke}{rgb}{0.000000,0.000000,0.000000}%
\pgfsetstrokecolor{currentstroke}%
\pgfsetdash{}{0pt}%
\pgfsys@defobject{currentmarker}{\pgfqpoint{0.000000in}{-0.048611in}}{\pgfqpoint{0.000000in}{0.000000in}}{%
\pgfpathmoveto{\pgfqpoint{0.000000in}{0.000000in}}%
\pgfpathlineto{\pgfqpoint{0.000000in}{-0.048611in}}%
\pgfusepath{stroke,fill}%
}%
\begin{pgfscope}%
\pgfsys@transformshift{4.106664in}{0.548769in}%
\pgfsys@useobject{currentmarker}{}%
\end{pgfscope}%
\end{pgfscope}%
\begin{pgfscope}%
\definecolor{textcolor}{rgb}{0.000000,0.000000,0.000000}%
\pgfsetstrokecolor{textcolor}%
\pgfsetfillcolor{textcolor}%
\pgftext[x=4.106664in,y=0.451547in,,top]{\color{textcolor}\sffamily\fontsize{10.000000}{12.000000}\selectfont \(\displaystyle {50}\)}%
\end{pgfscope}%
\begin{pgfscope}%
\pgfsetbuttcap%
\pgfsetroundjoin%
\definecolor{currentfill}{rgb}{0.000000,0.000000,0.000000}%
\pgfsetfillcolor{currentfill}%
\pgfsetlinewidth{0.803000pt}%
\definecolor{currentstroke}{rgb}{0.000000,0.000000,0.000000}%
\pgfsetstrokecolor{currentstroke}%
\pgfsetdash{}{0pt}%
\pgfsys@defobject{currentmarker}{\pgfqpoint{0.000000in}{-0.048611in}}{\pgfqpoint{0.000000in}{0.000000in}}{%
\pgfpathmoveto{\pgfqpoint{0.000000in}{0.000000in}}%
\pgfpathlineto{\pgfqpoint{0.000000in}{-0.048611in}}%
\pgfusepath{stroke,fill}%
}%
\begin{pgfscope}%
\pgfsys@transformshift{4.737165in}{0.548769in}%
\pgfsys@useobject{currentmarker}{}%
\end{pgfscope}%
\end{pgfscope}%
\begin{pgfscope}%
\definecolor{textcolor}{rgb}{0.000000,0.000000,0.000000}%
\pgfsetstrokecolor{textcolor}%
\pgfsetfillcolor{textcolor}%
\pgftext[x=4.737165in,y=0.451547in,,top]{\color{textcolor}\sffamily\fontsize{10.000000}{12.000000}\selectfont \(\displaystyle {60}\)}%
\end{pgfscope}%
\begin{pgfscope}%
\pgfsetbuttcap%
\pgfsetroundjoin%
\definecolor{currentfill}{rgb}{0.000000,0.000000,0.000000}%
\pgfsetfillcolor{currentfill}%
\pgfsetlinewidth{0.803000pt}%
\definecolor{currentstroke}{rgb}{0.000000,0.000000,0.000000}%
\pgfsetstrokecolor{currentstroke}%
\pgfsetdash{}{0pt}%
\pgfsys@defobject{currentmarker}{\pgfqpoint{0.000000in}{-0.048611in}}{\pgfqpoint{0.000000in}{0.000000in}}{%
\pgfpathmoveto{\pgfqpoint{0.000000in}{0.000000in}}%
\pgfpathlineto{\pgfqpoint{0.000000in}{-0.048611in}}%
\pgfusepath{stroke,fill}%
}%
\begin{pgfscope}%
\pgfsys@transformshift{5.367666in}{0.548769in}%
\pgfsys@useobject{currentmarker}{}%
\end{pgfscope}%
\end{pgfscope}%
\begin{pgfscope}%
\definecolor{textcolor}{rgb}{0.000000,0.000000,0.000000}%
\pgfsetstrokecolor{textcolor}%
\pgfsetfillcolor{textcolor}%
\pgftext[x=5.367666in,y=0.451547in,,top]{\color{textcolor}\sffamily\fontsize{10.000000}{12.000000}\selectfont \(\displaystyle {70}\)}%
\end{pgfscope}%
\begin{pgfscope}%
\definecolor{textcolor}{rgb}{0.000000,0.000000,0.000000}%
\pgfsetstrokecolor{textcolor}%
\pgfsetfillcolor{textcolor}%
\pgftext[x=3.318537in,y=0.272658in,,top]{\color{textcolor}\sffamily\fontsize{10.000000}{12.000000}\selectfont Number of Sinks}%
\end{pgfscope}%
\begin{pgfscope}%
\pgfsetbuttcap%
\pgfsetroundjoin%
\definecolor{currentfill}{rgb}{0.000000,0.000000,0.000000}%
\pgfsetfillcolor{currentfill}%
\pgfsetlinewidth{0.803000pt}%
\definecolor{currentstroke}{rgb}{0.000000,0.000000,0.000000}%
\pgfsetstrokecolor{currentstroke}%
\pgfsetdash{}{0pt}%
\pgfsys@defobject{currentmarker}{\pgfqpoint{-0.048611in}{0.000000in}}{\pgfqpoint{0.000000in}{0.000000in}}{%
\pgfpathmoveto{\pgfqpoint{0.000000in}{0.000000in}}%
\pgfpathlineto{\pgfqpoint{-0.048611in}{0.000000in}}%
\pgfusepath{stroke,fill}%
}%
\begin{pgfscope}%
\pgfsys@transformshift{0.787074in}{0.689795in}%
\pgfsys@useobject{currentmarker}{}%
\end{pgfscope}%
\end{pgfscope}%
\begin{pgfscope}%
\definecolor{textcolor}{rgb}{0.000000,0.000000,0.000000}%
\pgfsetstrokecolor{textcolor}%
\pgfsetfillcolor{textcolor}%
\pgftext[x=0.620407in, y=0.641601in, left, base]{\color{textcolor}\sffamily\fontsize{10.000000}{12.000000}\selectfont \(\displaystyle {0}\)}%
\end{pgfscope}%
\begin{pgfscope}%
\pgfsetbuttcap%
\pgfsetroundjoin%
\definecolor{currentfill}{rgb}{0.000000,0.000000,0.000000}%
\pgfsetfillcolor{currentfill}%
\pgfsetlinewidth{0.803000pt}%
\definecolor{currentstroke}{rgb}{0.000000,0.000000,0.000000}%
\pgfsetstrokecolor{currentstroke}%
\pgfsetdash{}{0pt}%
\pgfsys@defobject{currentmarker}{\pgfqpoint{-0.048611in}{0.000000in}}{\pgfqpoint{0.000000in}{0.000000in}}{%
\pgfpathmoveto{\pgfqpoint{0.000000in}{0.000000in}}%
\pgfpathlineto{\pgfqpoint{-0.048611in}{0.000000in}}%
\pgfusepath{stroke,fill}%
}%
\begin{pgfscope}%
\pgfsys@transformshift{0.787074in}{1.373761in}%
\pgfsys@useobject{currentmarker}{}%
\end{pgfscope}%
\end{pgfscope}%
\begin{pgfscope}%
\definecolor{textcolor}{rgb}{0.000000,0.000000,0.000000}%
\pgfsetstrokecolor{textcolor}%
\pgfsetfillcolor{textcolor}%
\pgftext[x=0.412073in, y=1.325566in, left, base]{\color{textcolor}\sffamily\fontsize{10.000000}{12.000000}\selectfont \(\displaystyle {5000}\)}%
\end{pgfscope}%
\begin{pgfscope}%
\pgfsetbuttcap%
\pgfsetroundjoin%
\definecolor{currentfill}{rgb}{0.000000,0.000000,0.000000}%
\pgfsetfillcolor{currentfill}%
\pgfsetlinewidth{0.803000pt}%
\definecolor{currentstroke}{rgb}{0.000000,0.000000,0.000000}%
\pgfsetstrokecolor{currentstroke}%
\pgfsetdash{}{0pt}%
\pgfsys@defobject{currentmarker}{\pgfqpoint{-0.048611in}{0.000000in}}{\pgfqpoint{0.000000in}{0.000000in}}{%
\pgfpathmoveto{\pgfqpoint{0.000000in}{0.000000in}}%
\pgfpathlineto{\pgfqpoint{-0.048611in}{0.000000in}}%
\pgfusepath{stroke,fill}%
}%
\begin{pgfscope}%
\pgfsys@transformshift{0.787074in}{2.057726in}%
\pgfsys@useobject{currentmarker}{}%
\end{pgfscope}%
\end{pgfscope}%
\begin{pgfscope}%
\definecolor{textcolor}{rgb}{0.000000,0.000000,0.000000}%
\pgfsetstrokecolor{textcolor}%
\pgfsetfillcolor{textcolor}%
\pgftext[x=0.342628in, y=2.009532in, left, base]{\color{textcolor}\sffamily\fontsize{10.000000}{12.000000}\selectfont \(\displaystyle {10000}\)}%
\end{pgfscope}%
\begin{pgfscope}%
\pgfsetbuttcap%
\pgfsetroundjoin%
\definecolor{currentfill}{rgb}{0.000000,0.000000,0.000000}%
\pgfsetfillcolor{currentfill}%
\pgfsetlinewidth{0.803000pt}%
\definecolor{currentstroke}{rgb}{0.000000,0.000000,0.000000}%
\pgfsetstrokecolor{currentstroke}%
\pgfsetdash{}{0pt}%
\pgfsys@defobject{currentmarker}{\pgfqpoint{-0.048611in}{0.000000in}}{\pgfqpoint{0.000000in}{0.000000in}}{%
\pgfpathmoveto{\pgfqpoint{0.000000in}{0.000000in}}%
\pgfpathlineto{\pgfqpoint{-0.048611in}{0.000000in}}%
\pgfusepath{stroke,fill}%
}%
\begin{pgfscope}%
\pgfsys@transformshift{0.787074in}{2.741692in}%
\pgfsys@useobject{currentmarker}{}%
\end{pgfscope}%
\end{pgfscope}%
\begin{pgfscope}%
\definecolor{textcolor}{rgb}{0.000000,0.000000,0.000000}%
\pgfsetstrokecolor{textcolor}%
\pgfsetfillcolor{textcolor}%
\pgftext[x=0.342628in, y=2.693498in, left, base]{\color{textcolor}\sffamily\fontsize{10.000000}{12.000000}\selectfont \(\displaystyle {15000}\)}%
\end{pgfscope}%
\begin{pgfscope}%
\pgfsetbuttcap%
\pgfsetroundjoin%
\definecolor{currentfill}{rgb}{0.000000,0.000000,0.000000}%
\pgfsetfillcolor{currentfill}%
\pgfsetlinewidth{0.803000pt}%
\definecolor{currentstroke}{rgb}{0.000000,0.000000,0.000000}%
\pgfsetstrokecolor{currentstroke}%
\pgfsetdash{}{0pt}%
\pgfsys@defobject{currentmarker}{\pgfqpoint{-0.048611in}{0.000000in}}{\pgfqpoint{0.000000in}{0.000000in}}{%
\pgfpathmoveto{\pgfqpoint{0.000000in}{0.000000in}}%
\pgfpathlineto{\pgfqpoint{-0.048611in}{0.000000in}}%
\pgfusepath{stroke,fill}%
}%
\begin{pgfscope}%
\pgfsys@transformshift{0.787074in}{3.425658in}%
\pgfsys@useobject{currentmarker}{}%
\end{pgfscope}%
\end{pgfscope}%
\begin{pgfscope}%
\definecolor{textcolor}{rgb}{0.000000,0.000000,0.000000}%
\pgfsetstrokecolor{textcolor}%
\pgfsetfillcolor{textcolor}%
\pgftext[x=0.342628in, y=3.377463in, left, base]{\color{textcolor}\sffamily\fontsize{10.000000}{12.000000}\selectfont \(\displaystyle {20000}\)}%
\end{pgfscope}%
\begin{pgfscope}%
\definecolor{textcolor}{rgb}{0.000000,0.000000,0.000000}%
\pgfsetstrokecolor{textcolor}%
\pgfsetfillcolor{textcolor}%
\pgftext[x=0.287073in,y=2.100064in,,bottom,rotate=90.000000]{\color{textcolor}\sffamily\fontsize{10.000000}{12.000000}\selectfont Maximum Memory Usage (MB)}%
\end{pgfscope}%
\begin{pgfscope}%
\pgfsetrectcap%
\pgfsetmiterjoin%
\pgfsetlinewidth{0.803000pt}%
\definecolor{currentstroke}{rgb}{0.000000,0.000000,0.000000}%
\pgfsetstrokecolor{currentstroke}%
\pgfsetdash{}{0pt}%
\pgfpathmoveto{\pgfqpoint{0.787074in}{0.548769in}}%
\pgfpathlineto{\pgfqpoint{0.787074in}{3.651359in}}%
\pgfusepath{stroke}%
\end{pgfscope}%
\begin{pgfscope}%
\pgfsetrectcap%
\pgfsetmiterjoin%
\pgfsetlinewidth{0.803000pt}%
\definecolor{currentstroke}{rgb}{0.000000,0.000000,0.000000}%
\pgfsetstrokecolor{currentstroke}%
\pgfsetdash{}{0pt}%
\pgfpathmoveto{\pgfqpoint{5.850000in}{0.548769in}}%
\pgfpathlineto{\pgfqpoint{5.850000in}{3.651359in}}%
\pgfusepath{stroke}%
\end{pgfscope}%
\begin{pgfscope}%
\pgfsetrectcap%
\pgfsetmiterjoin%
\pgfsetlinewidth{0.803000pt}%
\definecolor{currentstroke}{rgb}{0.000000,0.000000,0.000000}%
\pgfsetstrokecolor{currentstroke}%
\pgfsetdash{}{0pt}%
\pgfpathmoveto{\pgfqpoint{0.787074in}{0.548769in}}%
\pgfpathlineto{\pgfqpoint{5.850000in}{0.548769in}}%
\pgfusepath{stroke}%
\end{pgfscope}%
\begin{pgfscope}%
\pgfsetrectcap%
\pgfsetmiterjoin%
\pgfsetlinewidth{0.803000pt}%
\definecolor{currentstroke}{rgb}{0.000000,0.000000,0.000000}%
\pgfsetstrokecolor{currentstroke}%
\pgfsetdash{}{0pt}%
\pgfpathmoveto{\pgfqpoint{0.787074in}{3.651359in}}%
\pgfpathlineto{\pgfqpoint{5.850000in}{3.651359in}}%
\pgfusepath{stroke}%
\end{pgfscope}%
\begin{pgfscope}%
\definecolor{textcolor}{rgb}{0.000000,0.000000,0.000000}%
\pgfsetstrokecolor{textcolor}%
\pgfsetfillcolor{textcolor}%
\pgftext[x=3.318537in,y=3.734692in,,base]{\color{textcolor}\sffamily\fontsize{12.000000}{14.400000}\selectfont Forward}%
\end{pgfscope}%
\begin{pgfscope}%
\pgfsetbuttcap%
\pgfsetmiterjoin%
\definecolor{currentfill}{rgb}{1.000000,1.000000,1.000000}%
\pgfsetfillcolor{currentfill}%
\pgfsetfillopacity{0.800000}%
\pgfsetlinewidth{1.003750pt}%
\definecolor{currentstroke}{rgb}{0.800000,0.800000,0.800000}%
\pgfsetstrokecolor{currentstroke}%
\pgfsetstrokeopacity{0.800000}%
\pgfsetdash{}{0pt}%
\pgfpathmoveto{\pgfqpoint{4.300417in}{0.618213in}}%
\pgfpathlineto{\pgfqpoint{5.752778in}{0.618213in}}%
\pgfpathquadraticcurveto{\pgfqpoint{5.780556in}{0.618213in}}{\pgfqpoint{5.780556in}{0.645991in}}%
\pgfpathlineto{\pgfqpoint{5.780556in}{1.214463in}}%
\pgfpathquadraticcurveto{\pgfqpoint{5.780556in}{1.242241in}}{\pgfqpoint{5.752778in}{1.242241in}}%
\pgfpathlineto{\pgfqpoint{4.300417in}{1.242241in}}%
\pgfpathquadraticcurveto{\pgfqpoint{4.272639in}{1.242241in}}{\pgfqpoint{4.272639in}{1.214463in}}%
\pgfpathlineto{\pgfqpoint{4.272639in}{0.645991in}}%
\pgfpathquadraticcurveto{\pgfqpoint{4.272639in}{0.618213in}}{\pgfqpoint{4.300417in}{0.618213in}}%
\pgfpathclose%
\pgfusepath{stroke,fill}%
\end{pgfscope}%
\begin{pgfscope}%
\pgfsetbuttcap%
\pgfsetroundjoin%
\definecolor{currentfill}{rgb}{0.121569,0.466667,0.705882}%
\pgfsetfillcolor{currentfill}%
\pgfsetlinewidth{1.003750pt}%
\definecolor{currentstroke}{rgb}{0.121569,0.466667,0.705882}%
\pgfsetstrokecolor{currentstroke}%
\pgfsetdash{}{0pt}%
\pgfsys@defobject{currentmarker}{\pgfqpoint{-0.034722in}{-0.034722in}}{\pgfqpoint{0.034722in}{0.034722in}}{%
\pgfpathmoveto{\pgfqpoint{0.000000in}{-0.034722in}}%
\pgfpathcurveto{\pgfqpoint{0.009208in}{-0.034722in}}{\pgfqpoint{0.018041in}{-0.031064in}}{\pgfqpoint{0.024552in}{-0.024552in}}%
\pgfpathcurveto{\pgfqpoint{0.031064in}{-0.018041in}}{\pgfqpoint{0.034722in}{-0.009208in}}{\pgfqpoint{0.034722in}{0.000000in}}%
\pgfpathcurveto{\pgfqpoint{0.034722in}{0.009208in}}{\pgfqpoint{0.031064in}{0.018041in}}{\pgfqpoint{0.024552in}{0.024552in}}%
\pgfpathcurveto{\pgfqpoint{0.018041in}{0.031064in}}{\pgfqpoint{0.009208in}{0.034722in}}{\pgfqpoint{0.000000in}{0.034722in}}%
\pgfpathcurveto{\pgfqpoint{-0.009208in}{0.034722in}}{\pgfqpoint{-0.018041in}{0.031064in}}{\pgfqpoint{-0.024552in}{0.024552in}}%
\pgfpathcurveto{\pgfqpoint{-0.031064in}{0.018041in}}{\pgfqpoint{-0.034722in}{0.009208in}}{\pgfqpoint{-0.034722in}{0.000000in}}%
\pgfpathcurveto{\pgfqpoint{-0.034722in}{-0.009208in}}{\pgfqpoint{-0.031064in}{-0.018041in}}{\pgfqpoint{-0.024552in}{-0.024552in}}%
\pgfpathcurveto{\pgfqpoint{-0.018041in}{-0.031064in}}{\pgfqpoint{-0.009208in}{-0.034722in}}{\pgfqpoint{0.000000in}{-0.034722in}}%
\pgfpathclose%
\pgfusepath{stroke,fill}%
}%
\begin{pgfscope}%
\pgfsys@transformshift{4.467083in}{1.138074in}%
\pgfsys@useobject{currentmarker}{}%
\end{pgfscope}%
\end{pgfscope}%
\begin{pgfscope}%
\definecolor{textcolor}{rgb}{0.000000,0.000000,0.000000}%
\pgfsetstrokecolor{textcolor}%
\pgfsetfillcolor{textcolor}%
\pgftext[x=4.717083in,y=1.089463in,left,base]{\color{textcolor}\sffamily\fontsize{10.000000}{12.000000}\selectfont No Timeout}%
\end{pgfscope}%
\begin{pgfscope}%
\pgfsetbuttcap%
\pgfsetroundjoin%
\definecolor{currentfill}{rgb}{1.000000,0.498039,0.054902}%
\pgfsetfillcolor{currentfill}%
\pgfsetlinewidth{1.003750pt}%
\definecolor{currentstroke}{rgb}{1.000000,0.498039,0.054902}%
\pgfsetstrokecolor{currentstroke}%
\pgfsetdash{}{0pt}%
\pgfsys@defobject{currentmarker}{\pgfqpoint{-0.034722in}{-0.034722in}}{\pgfqpoint{0.034722in}{0.034722in}}{%
\pgfpathmoveto{\pgfqpoint{0.000000in}{-0.034722in}}%
\pgfpathcurveto{\pgfqpoint{0.009208in}{-0.034722in}}{\pgfqpoint{0.018041in}{-0.031064in}}{\pgfqpoint{0.024552in}{-0.024552in}}%
\pgfpathcurveto{\pgfqpoint{0.031064in}{-0.018041in}}{\pgfqpoint{0.034722in}{-0.009208in}}{\pgfqpoint{0.034722in}{0.000000in}}%
\pgfpathcurveto{\pgfqpoint{0.034722in}{0.009208in}}{\pgfqpoint{0.031064in}{0.018041in}}{\pgfqpoint{0.024552in}{0.024552in}}%
\pgfpathcurveto{\pgfqpoint{0.018041in}{0.031064in}}{\pgfqpoint{0.009208in}{0.034722in}}{\pgfqpoint{0.000000in}{0.034722in}}%
\pgfpathcurveto{\pgfqpoint{-0.009208in}{0.034722in}}{\pgfqpoint{-0.018041in}{0.031064in}}{\pgfqpoint{-0.024552in}{0.024552in}}%
\pgfpathcurveto{\pgfqpoint{-0.031064in}{0.018041in}}{\pgfqpoint{-0.034722in}{0.009208in}}{\pgfqpoint{-0.034722in}{0.000000in}}%
\pgfpathcurveto{\pgfqpoint{-0.034722in}{-0.009208in}}{\pgfqpoint{-0.031064in}{-0.018041in}}{\pgfqpoint{-0.024552in}{-0.024552in}}%
\pgfpathcurveto{\pgfqpoint{-0.018041in}{-0.031064in}}{\pgfqpoint{-0.009208in}{-0.034722in}}{\pgfqpoint{0.000000in}{-0.034722in}}%
\pgfpathclose%
\pgfusepath{stroke,fill}%
}%
\begin{pgfscope}%
\pgfsys@transformshift{4.467083in}{0.944463in}%
\pgfsys@useobject{currentmarker}{}%
\end{pgfscope}%
\end{pgfscope}%
\begin{pgfscope}%
\definecolor{textcolor}{rgb}{0.000000,0.000000,0.000000}%
\pgfsetstrokecolor{textcolor}%
\pgfsetfillcolor{textcolor}%
\pgftext[x=4.717083in,y=0.895852in,left,base]{\color{textcolor}\sffamily\fontsize{10.000000}{12.000000}\selectfont Time Timeout}%
\end{pgfscope}%
\begin{pgfscope}%
\pgfsetbuttcap%
\pgfsetroundjoin%
\definecolor{currentfill}{rgb}{0.839216,0.152941,0.156863}%
\pgfsetfillcolor{currentfill}%
\pgfsetlinewidth{1.003750pt}%
\definecolor{currentstroke}{rgb}{0.839216,0.152941,0.156863}%
\pgfsetstrokecolor{currentstroke}%
\pgfsetdash{}{0pt}%
\pgfsys@defobject{currentmarker}{\pgfqpoint{-0.034722in}{-0.034722in}}{\pgfqpoint{0.034722in}{0.034722in}}{%
\pgfpathmoveto{\pgfqpoint{0.000000in}{-0.034722in}}%
\pgfpathcurveto{\pgfqpoint{0.009208in}{-0.034722in}}{\pgfqpoint{0.018041in}{-0.031064in}}{\pgfqpoint{0.024552in}{-0.024552in}}%
\pgfpathcurveto{\pgfqpoint{0.031064in}{-0.018041in}}{\pgfqpoint{0.034722in}{-0.009208in}}{\pgfqpoint{0.034722in}{0.000000in}}%
\pgfpathcurveto{\pgfqpoint{0.034722in}{0.009208in}}{\pgfqpoint{0.031064in}{0.018041in}}{\pgfqpoint{0.024552in}{0.024552in}}%
\pgfpathcurveto{\pgfqpoint{0.018041in}{0.031064in}}{\pgfqpoint{0.009208in}{0.034722in}}{\pgfqpoint{0.000000in}{0.034722in}}%
\pgfpathcurveto{\pgfqpoint{-0.009208in}{0.034722in}}{\pgfqpoint{-0.018041in}{0.031064in}}{\pgfqpoint{-0.024552in}{0.024552in}}%
\pgfpathcurveto{\pgfqpoint{-0.031064in}{0.018041in}}{\pgfqpoint{-0.034722in}{0.009208in}}{\pgfqpoint{-0.034722in}{0.000000in}}%
\pgfpathcurveto{\pgfqpoint{-0.034722in}{-0.009208in}}{\pgfqpoint{-0.031064in}{-0.018041in}}{\pgfqpoint{-0.024552in}{-0.024552in}}%
\pgfpathcurveto{\pgfqpoint{-0.018041in}{-0.031064in}}{\pgfqpoint{-0.009208in}{-0.034722in}}{\pgfqpoint{0.000000in}{-0.034722in}}%
\pgfpathclose%
\pgfusepath{stroke,fill}%
}%
\begin{pgfscope}%
\pgfsys@transformshift{4.467083in}{0.750852in}%
\pgfsys@useobject{currentmarker}{}%
\end{pgfscope}%
\end{pgfscope}%
\begin{pgfscope}%
\definecolor{textcolor}{rgb}{0.000000,0.000000,0.000000}%
\pgfsetstrokecolor{textcolor}%
\pgfsetfillcolor{textcolor}%
\pgftext[x=4.717083in,y=0.702241in,left,base]{\color{textcolor}\sffamily\fontsize{10.000000}{12.000000}\selectfont Memory Timeout}%
\end{pgfscope}%
\end{pgfpicture}%
\makeatother%
\endgroup%

                }
            \end{subfigure}
            \qquad
            \begin{subfigure}[]{0.45\textwidth}
                \centering
                \resizebox{\columnwidth}{!}{
                    %% Creator: Matplotlib, PGF backend
%%
%% To include the figure in your LaTeX document, write
%%   \input{<filename>.pgf}
%%
%% Make sure the required packages are loaded in your preamble
%%   \usepackage{pgf}
%%
%% and, on pdftex
%%   \usepackage[utf8]{inputenc}\DeclareUnicodeCharacter{2212}{-}
%%
%% or, on luatex and xetex
%%   \usepackage{unicode-math}
%%
%% Figures using additional raster images can only be included by \input if
%% they are in the same directory as the main LaTeX file. For loading figures
%% from other directories you can use the `import` package
%%   \usepackage{import}
%%
%% and then include the figures with
%%   \import{<path to file>}{<filename>.pgf}
%%
%% Matplotlib used the following preamble
%%   \usepackage{amsmath}
%%   \usepackage{fontspec}
%%
\begingroup%
\makeatletter%
\begin{pgfpicture}%
\pgfpathrectangle{\pgfpointorigin}{\pgfqpoint{6.000000in}{4.000000in}}%
\pgfusepath{use as bounding box, clip}%
\begin{pgfscope}%
\pgfsetbuttcap%
\pgfsetmiterjoin%
\definecolor{currentfill}{rgb}{1.000000,1.000000,1.000000}%
\pgfsetfillcolor{currentfill}%
\pgfsetlinewidth{0.000000pt}%
\definecolor{currentstroke}{rgb}{1.000000,1.000000,1.000000}%
\pgfsetstrokecolor{currentstroke}%
\pgfsetdash{}{0pt}%
\pgfpathmoveto{\pgfqpoint{0.000000in}{0.000000in}}%
\pgfpathlineto{\pgfqpoint{6.000000in}{0.000000in}}%
\pgfpathlineto{\pgfqpoint{6.000000in}{4.000000in}}%
\pgfpathlineto{\pgfqpoint{0.000000in}{4.000000in}}%
\pgfpathclose%
\pgfusepath{fill}%
\end{pgfscope}%
\begin{pgfscope}%
\pgfsetbuttcap%
\pgfsetmiterjoin%
\definecolor{currentfill}{rgb}{1.000000,1.000000,1.000000}%
\pgfsetfillcolor{currentfill}%
\pgfsetlinewidth{0.000000pt}%
\definecolor{currentstroke}{rgb}{0.000000,0.000000,0.000000}%
\pgfsetstrokecolor{currentstroke}%
\pgfsetstrokeopacity{0.000000}%
\pgfsetdash{}{0pt}%
\pgfpathmoveto{\pgfqpoint{0.787074in}{0.548769in}}%
\pgfpathlineto{\pgfqpoint{5.850000in}{0.548769in}}%
\pgfpathlineto{\pgfqpoint{5.850000in}{3.651359in}}%
\pgfpathlineto{\pgfqpoint{0.787074in}{3.651359in}}%
\pgfpathclose%
\pgfusepath{fill}%
\end{pgfscope}%
\begin{pgfscope}%
\pgfpathrectangle{\pgfqpoint{0.787074in}{0.548769in}}{\pgfqpoint{5.062926in}{3.102590in}}%
\pgfusepath{clip}%
\pgfsetbuttcap%
\pgfsetroundjoin%
\definecolor{currentfill}{rgb}{0.121569,0.466667,0.705882}%
\pgfsetfillcolor{currentfill}%
\pgfsetlinewidth{1.003750pt}%
\definecolor{currentstroke}{rgb}{0.121569,0.466667,0.705882}%
\pgfsetstrokecolor{currentstroke}%
\pgfsetdash{}{0pt}%
\pgfpathmoveto{\pgfqpoint{1.080257in}{0.676616in}}%
\pgfpathcurveto{\pgfqpoint{1.091307in}{0.676616in}}{\pgfqpoint{1.101906in}{0.681006in}}{\pgfqpoint{1.109720in}{0.688820in}}%
\pgfpathcurveto{\pgfqpoint{1.117533in}{0.696633in}}{\pgfqpoint{1.121924in}{0.707232in}}{\pgfqpoint{1.121924in}{0.718283in}}%
\pgfpathcurveto{\pgfqpoint{1.121924in}{0.729333in}}{\pgfqpoint{1.117533in}{0.739932in}}{\pgfqpoint{1.109720in}{0.747745in}}%
\pgfpathcurveto{\pgfqpoint{1.101906in}{0.755559in}}{\pgfqpoint{1.091307in}{0.759949in}}{\pgfqpoint{1.080257in}{0.759949in}}%
\pgfpathcurveto{\pgfqpoint{1.069207in}{0.759949in}}{\pgfqpoint{1.058608in}{0.755559in}}{\pgfqpoint{1.050794in}{0.747745in}}%
\pgfpathcurveto{\pgfqpoint{1.042981in}{0.739932in}}{\pgfqpoint{1.038590in}{0.729333in}}{\pgfqpoint{1.038590in}{0.718283in}}%
\pgfpathcurveto{\pgfqpoint{1.038590in}{0.707232in}}{\pgfqpoint{1.042981in}{0.696633in}}{\pgfqpoint{1.050794in}{0.688820in}}%
\pgfpathcurveto{\pgfqpoint{1.058608in}{0.681006in}}{\pgfqpoint{1.069207in}{0.676616in}}{\pgfqpoint{1.080257in}{0.676616in}}%
\pgfpathclose%
\pgfusepath{stroke,fill}%
\end{pgfscope}%
\begin{pgfscope}%
\pgfpathrectangle{\pgfqpoint{0.787074in}{0.548769in}}{\pgfqpoint{5.062926in}{3.102590in}}%
\pgfusepath{clip}%
\pgfsetbuttcap%
\pgfsetroundjoin%
\definecolor{currentfill}{rgb}{0.121569,0.466667,0.705882}%
\pgfsetfillcolor{currentfill}%
\pgfsetlinewidth{1.003750pt}%
\definecolor{currentstroke}{rgb}{0.121569,0.466667,0.705882}%
\pgfsetstrokecolor{currentstroke}%
\pgfsetdash{}{0pt}%
\pgfpathmoveto{\pgfqpoint{2.152109in}{2.052247in}}%
\pgfpathcurveto{\pgfqpoint{2.163159in}{2.052247in}}{\pgfqpoint{2.173759in}{2.056638in}}{\pgfqpoint{2.181572in}{2.064451in}}%
\pgfpathcurveto{\pgfqpoint{2.189386in}{2.072265in}}{\pgfqpoint{2.193776in}{2.082864in}}{\pgfqpoint{2.193776in}{2.093914in}}%
\pgfpathcurveto{\pgfqpoint{2.193776in}{2.104964in}}{\pgfqpoint{2.189386in}{2.115563in}}{\pgfqpoint{2.181572in}{2.123377in}}%
\pgfpathcurveto{\pgfqpoint{2.173759in}{2.131190in}}{\pgfqpoint{2.163159in}{2.135581in}}{\pgfqpoint{2.152109in}{2.135581in}}%
\pgfpathcurveto{\pgfqpoint{2.141059in}{2.135581in}}{\pgfqpoint{2.130460in}{2.131190in}}{\pgfqpoint{2.122647in}{2.123377in}}%
\pgfpathcurveto{\pgfqpoint{2.114833in}{2.115563in}}{\pgfqpoint{2.110443in}{2.104964in}}{\pgfqpoint{2.110443in}{2.093914in}}%
\pgfpathcurveto{\pgfqpoint{2.110443in}{2.082864in}}{\pgfqpoint{2.114833in}{2.072265in}}{\pgfqpoint{2.122647in}{2.064451in}}%
\pgfpathcurveto{\pgfqpoint{2.130460in}{2.056638in}}{\pgfqpoint{2.141059in}{2.052247in}}{\pgfqpoint{2.152109in}{2.052247in}}%
\pgfpathclose%
\pgfusepath{stroke,fill}%
\end{pgfscope}%
\begin{pgfscope}%
\pgfpathrectangle{\pgfqpoint{0.787074in}{0.548769in}}{\pgfqpoint{5.062926in}{3.102590in}}%
\pgfusepath{clip}%
\pgfsetbuttcap%
\pgfsetroundjoin%
\definecolor{currentfill}{rgb}{1.000000,0.498039,0.054902}%
\pgfsetfillcolor{currentfill}%
\pgfsetlinewidth{1.003750pt}%
\definecolor{currentstroke}{rgb}{1.000000,0.498039,0.054902}%
\pgfsetstrokecolor{currentstroke}%
\pgfsetdash{}{0pt}%
\pgfpathmoveto{\pgfqpoint{1.017207in}{2.417739in}}%
\pgfpathcurveto{\pgfqpoint{1.028257in}{2.417739in}}{\pgfqpoint{1.038856in}{2.422129in}}{\pgfqpoint{1.046670in}{2.429943in}}%
\pgfpathcurveto{\pgfqpoint{1.054483in}{2.437757in}}{\pgfqpoint{1.058874in}{2.448356in}}{\pgfqpoint{1.058874in}{2.459406in}}%
\pgfpathcurveto{\pgfqpoint{1.058874in}{2.470456in}}{\pgfqpoint{1.054483in}{2.481055in}}{\pgfqpoint{1.046670in}{2.488869in}}%
\pgfpathcurveto{\pgfqpoint{1.038856in}{2.496682in}}{\pgfqpoint{1.028257in}{2.501072in}}{\pgfqpoint{1.017207in}{2.501072in}}%
\pgfpathcurveto{\pgfqpoint{1.006157in}{2.501072in}}{\pgfqpoint{0.995558in}{2.496682in}}{\pgfqpoint{0.987744in}{2.488869in}}%
\pgfpathcurveto{\pgfqpoint{0.979930in}{2.481055in}}{\pgfqpoint{0.975540in}{2.470456in}}{\pgfqpoint{0.975540in}{2.459406in}}%
\pgfpathcurveto{\pgfqpoint{0.975540in}{2.448356in}}{\pgfqpoint{0.979930in}{2.437757in}}{\pgfqpoint{0.987744in}{2.429943in}}%
\pgfpathcurveto{\pgfqpoint{0.995558in}{2.422129in}}{\pgfqpoint{1.006157in}{2.417739in}}{\pgfqpoint{1.017207in}{2.417739in}}%
\pgfpathclose%
\pgfusepath{stroke,fill}%
\end{pgfscope}%
\begin{pgfscope}%
\pgfpathrectangle{\pgfqpoint{0.787074in}{0.548769in}}{\pgfqpoint{5.062926in}{3.102590in}}%
\pgfusepath{clip}%
\pgfsetbuttcap%
\pgfsetroundjoin%
\definecolor{currentfill}{rgb}{1.000000,0.498039,0.054902}%
\pgfsetfillcolor{currentfill}%
\pgfsetlinewidth{1.003750pt}%
\definecolor{currentstroke}{rgb}{1.000000,0.498039,0.054902}%
\pgfsetstrokecolor{currentstroke}%
\pgfsetdash{}{0pt}%
\pgfpathmoveto{\pgfqpoint{2.530410in}{2.242926in}}%
\pgfpathcurveto{\pgfqpoint{2.541460in}{2.242926in}}{\pgfqpoint{2.552059in}{2.247317in}}{\pgfqpoint{2.559873in}{2.255130in}}%
\pgfpathcurveto{\pgfqpoint{2.567687in}{2.262944in}}{\pgfqpoint{2.572077in}{2.273543in}}{\pgfqpoint{2.572077in}{2.284593in}}%
\pgfpathcurveto{\pgfqpoint{2.572077in}{2.295643in}}{\pgfqpoint{2.567687in}{2.306242in}}{\pgfqpoint{2.559873in}{2.314056in}}%
\pgfpathcurveto{\pgfqpoint{2.552059in}{2.321869in}}{\pgfqpoint{2.541460in}{2.326260in}}{\pgfqpoint{2.530410in}{2.326260in}}%
\pgfpathcurveto{\pgfqpoint{2.519360in}{2.326260in}}{\pgfqpoint{2.508761in}{2.321869in}}{\pgfqpoint{2.500947in}{2.314056in}}%
\pgfpathcurveto{\pgfqpoint{2.493134in}{2.306242in}}{\pgfqpoint{2.488744in}{2.295643in}}{\pgfqpoint{2.488744in}{2.284593in}}%
\pgfpathcurveto{\pgfqpoint{2.488744in}{2.273543in}}{\pgfqpoint{2.493134in}{2.262944in}}{\pgfqpoint{2.500947in}{2.255130in}}%
\pgfpathcurveto{\pgfqpoint{2.508761in}{2.247317in}}{\pgfqpoint{2.519360in}{2.242926in}}{\pgfqpoint{2.530410in}{2.242926in}}%
\pgfpathclose%
\pgfusepath{stroke,fill}%
\end{pgfscope}%
\begin{pgfscope}%
\pgfpathrectangle{\pgfqpoint{0.787074in}{0.548769in}}{\pgfqpoint{5.062926in}{3.102590in}}%
\pgfusepath{clip}%
\pgfsetbuttcap%
\pgfsetroundjoin%
\definecolor{currentfill}{rgb}{0.121569,0.466667,0.705882}%
\pgfsetfillcolor{currentfill}%
\pgfsetlinewidth{1.003750pt}%
\definecolor{currentstroke}{rgb}{0.121569,0.466667,0.705882}%
\pgfsetstrokecolor{currentstroke}%
\pgfsetdash{}{0pt}%
\pgfpathmoveto{\pgfqpoint{1.458558in}{0.650065in}}%
\pgfpathcurveto{\pgfqpoint{1.469608in}{0.650065in}}{\pgfqpoint{1.480207in}{0.654455in}}{\pgfqpoint{1.488021in}{0.662269in}}%
\pgfpathcurveto{\pgfqpoint{1.495834in}{0.670082in}}{\pgfqpoint{1.500224in}{0.680681in}}{\pgfqpoint{1.500224in}{0.691731in}}%
\pgfpathcurveto{\pgfqpoint{1.500224in}{0.702781in}}{\pgfqpoint{1.495834in}{0.713380in}}{\pgfqpoint{1.488021in}{0.721194in}}%
\pgfpathcurveto{\pgfqpoint{1.480207in}{0.729008in}}{\pgfqpoint{1.469608in}{0.733398in}}{\pgfqpoint{1.458558in}{0.733398in}}%
\pgfpathcurveto{\pgfqpoint{1.447508in}{0.733398in}}{\pgfqpoint{1.436909in}{0.729008in}}{\pgfqpoint{1.429095in}{0.721194in}}%
\pgfpathcurveto{\pgfqpoint{1.421281in}{0.713380in}}{\pgfqpoint{1.416891in}{0.702781in}}{\pgfqpoint{1.416891in}{0.691731in}}%
\pgfpathcurveto{\pgfqpoint{1.416891in}{0.680681in}}{\pgfqpoint{1.421281in}{0.670082in}}{\pgfqpoint{1.429095in}{0.662269in}}%
\pgfpathcurveto{\pgfqpoint{1.436909in}{0.654455in}}{\pgfqpoint{1.447508in}{0.650065in}}{\pgfqpoint{1.458558in}{0.650065in}}%
\pgfpathclose%
\pgfusepath{stroke,fill}%
\end{pgfscope}%
\begin{pgfscope}%
\pgfpathrectangle{\pgfqpoint{0.787074in}{0.548769in}}{\pgfqpoint{5.062926in}{3.102590in}}%
\pgfusepath{clip}%
\pgfsetbuttcap%
\pgfsetroundjoin%
\definecolor{currentfill}{rgb}{0.121569,0.466667,0.705882}%
\pgfsetfillcolor{currentfill}%
\pgfsetlinewidth{1.003750pt}%
\definecolor{currentstroke}{rgb}{0.121569,0.466667,0.705882}%
\pgfsetstrokecolor{currentstroke}%
\pgfsetdash{}{0pt}%
\pgfpathmoveto{\pgfqpoint{4.169714in}{2.492055in}}%
\pgfpathcurveto{\pgfqpoint{4.180764in}{2.492055in}}{\pgfqpoint{4.191363in}{2.496445in}}{\pgfqpoint{4.199177in}{2.504258in}}%
\pgfpathcurveto{\pgfqpoint{4.206990in}{2.512072in}}{\pgfqpoint{4.211380in}{2.522671in}}{\pgfqpoint{4.211380in}{2.533721in}}%
\pgfpathcurveto{\pgfqpoint{4.211380in}{2.544771in}}{\pgfqpoint{4.206990in}{2.555370in}}{\pgfqpoint{4.199177in}{2.563184in}}%
\pgfpathcurveto{\pgfqpoint{4.191363in}{2.570998in}}{\pgfqpoint{4.180764in}{2.575388in}}{\pgfqpoint{4.169714in}{2.575388in}}%
\pgfpathcurveto{\pgfqpoint{4.158664in}{2.575388in}}{\pgfqpoint{4.148065in}{2.570998in}}{\pgfqpoint{4.140251in}{2.563184in}}%
\pgfpathcurveto{\pgfqpoint{4.132437in}{2.555370in}}{\pgfqpoint{4.128047in}{2.544771in}}{\pgfqpoint{4.128047in}{2.533721in}}%
\pgfpathcurveto{\pgfqpoint{4.128047in}{2.522671in}}{\pgfqpoint{4.132437in}{2.512072in}}{\pgfqpoint{4.140251in}{2.504258in}}%
\pgfpathcurveto{\pgfqpoint{4.148065in}{2.496445in}}{\pgfqpoint{4.158664in}{2.492055in}}{\pgfqpoint{4.169714in}{2.492055in}}%
\pgfpathclose%
\pgfusepath{stroke,fill}%
\end{pgfscope}%
\begin{pgfscope}%
\pgfpathrectangle{\pgfqpoint{0.787074in}{0.548769in}}{\pgfqpoint{5.062926in}{3.102590in}}%
\pgfusepath{clip}%
\pgfsetbuttcap%
\pgfsetroundjoin%
\definecolor{currentfill}{rgb}{1.000000,0.498039,0.054902}%
\pgfsetfillcolor{currentfill}%
\pgfsetlinewidth{1.003750pt}%
\definecolor{currentstroke}{rgb}{1.000000,0.498039,0.054902}%
\pgfsetstrokecolor{currentstroke}%
\pgfsetdash{}{0pt}%
\pgfpathmoveto{\pgfqpoint{1.962959in}{2.495853in}}%
\pgfpathcurveto{\pgfqpoint{1.974009in}{2.495853in}}{\pgfqpoint{1.984608in}{2.500244in}}{\pgfqpoint{1.992422in}{2.508057in}}%
\pgfpathcurveto{\pgfqpoint{2.000235in}{2.515871in}}{\pgfqpoint{2.004626in}{2.526470in}}{\pgfqpoint{2.004626in}{2.537520in}}%
\pgfpathcurveto{\pgfqpoint{2.004626in}{2.548570in}}{\pgfqpoint{2.000235in}{2.559169in}}{\pgfqpoint{1.992422in}{2.566983in}}%
\pgfpathcurveto{\pgfqpoint{1.984608in}{2.574797in}}{\pgfqpoint{1.974009in}{2.579187in}}{\pgfqpoint{1.962959in}{2.579187in}}%
\pgfpathcurveto{\pgfqpoint{1.951909in}{2.579187in}}{\pgfqpoint{1.941310in}{2.574797in}}{\pgfqpoint{1.933496in}{2.566983in}}%
\pgfpathcurveto{\pgfqpoint{1.925683in}{2.559169in}}{\pgfqpoint{1.921292in}{2.548570in}}{\pgfqpoint{1.921292in}{2.537520in}}%
\pgfpathcurveto{\pgfqpoint{1.921292in}{2.526470in}}{\pgfqpoint{1.925683in}{2.515871in}}{\pgfqpoint{1.933496in}{2.508057in}}%
\pgfpathcurveto{\pgfqpoint{1.941310in}{2.500244in}}{\pgfqpoint{1.951909in}{2.495853in}}{\pgfqpoint{1.962959in}{2.495853in}}%
\pgfpathclose%
\pgfusepath{stroke,fill}%
\end{pgfscope}%
\begin{pgfscope}%
\pgfpathrectangle{\pgfqpoint{0.787074in}{0.548769in}}{\pgfqpoint{5.062926in}{3.102590in}}%
\pgfusepath{clip}%
\pgfsetbuttcap%
\pgfsetroundjoin%
\definecolor{currentfill}{rgb}{0.121569,0.466667,0.705882}%
\pgfsetfillcolor{currentfill}%
\pgfsetlinewidth{1.003750pt}%
\definecolor{currentstroke}{rgb}{0.121569,0.466667,0.705882}%
\pgfsetstrokecolor{currentstroke}%
\pgfsetdash{}{0pt}%
\pgfpathmoveto{\pgfqpoint{1.710758in}{0.648161in}}%
\pgfpathcurveto{\pgfqpoint{1.721809in}{0.648161in}}{\pgfqpoint{1.732408in}{0.652552in}}{\pgfqpoint{1.740221in}{0.660365in}}%
\pgfpathcurveto{\pgfqpoint{1.748035in}{0.668179in}}{\pgfqpoint{1.752425in}{0.678778in}}{\pgfqpoint{1.752425in}{0.689828in}}%
\pgfpathcurveto{\pgfqpoint{1.752425in}{0.700878in}}{\pgfqpoint{1.748035in}{0.711477in}}{\pgfqpoint{1.740221in}{0.719291in}}%
\pgfpathcurveto{\pgfqpoint{1.732408in}{0.727104in}}{\pgfqpoint{1.721809in}{0.731495in}}{\pgfqpoint{1.710758in}{0.731495in}}%
\pgfpathcurveto{\pgfqpoint{1.699708in}{0.731495in}}{\pgfqpoint{1.689109in}{0.727104in}}{\pgfqpoint{1.681296in}{0.719291in}}%
\pgfpathcurveto{\pgfqpoint{1.673482in}{0.711477in}}{\pgfqpoint{1.669092in}{0.700878in}}{\pgfqpoint{1.669092in}{0.689828in}}%
\pgfpathcurveto{\pgfqpoint{1.669092in}{0.678778in}}{\pgfqpoint{1.673482in}{0.668179in}}{\pgfqpoint{1.681296in}{0.660365in}}%
\pgfpathcurveto{\pgfqpoint{1.689109in}{0.652552in}}{\pgfqpoint{1.699708in}{0.648161in}}{\pgfqpoint{1.710758in}{0.648161in}}%
\pgfpathclose%
\pgfusepath{stroke,fill}%
\end{pgfscope}%
\begin{pgfscope}%
\pgfpathrectangle{\pgfqpoint{0.787074in}{0.548769in}}{\pgfqpoint{5.062926in}{3.102590in}}%
\pgfusepath{clip}%
\pgfsetbuttcap%
\pgfsetroundjoin%
\definecolor{currentfill}{rgb}{0.121569,0.466667,0.705882}%
\pgfsetfillcolor{currentfill}%
\pgfsetlinewidth{1.003750pt}%
\definecolor{currentstroke}{rgb}{0.121569,0.466667,0.705882}%
\pgfsetstrokecolor{currentstroke}%
\pgfsetdash{}{0pt}%
\pgfpathmoveto{\pgfqpoint{1.395508in}{0.839743in}}%
\pgfpathcurveto{\pgfqpoint{1.406558in}{0.839743in}}{\pgfqpoint{1.417157in}{0.844134in}}{\pgfqpoint{1.424970in}{0.851947in}}%
\pgfpathcurveto{\pgfqpoint{1.432784in}{0.859761in}}{\pgfqpoint{1.437174in}{0.870360in}}{\pgfqpoint{1.437174in}{0.881410in}}%
\pgfpathcurveto{\pgfqpoint{1.437174in}{0.892460in}}{\pgfqpoint{1.432784in}{0.903059in}}{\pgfqpoint{1.424970in}{0.910873in}}%
\pgfpathcurveto{\pgfqpoint{1.417157in}{0.918687in}}{\pgfqpoint{1.406558in}{0.923077in}}{\pgfqpoint{1.395508in}{0.923077in}}%
\pgfpathcurveto{\pgfqpoint{1.384458in}{0.923077in}}{\pgfqpoint{1.373859in}{0.918687in}}{\pgfqpoint{1.366045in}{0.910873in}}%
\pgfpathcurveto{\pgfqpoint{1.358231in}{0.903059in}}{\pgfqpoint{1.353841in}{0.892460in}}{\pgfqpoint{1.353841in}{0.881410in}}%
\pgfpathcurveto{\pgfqpoint{1.353841in}{0.870360in}}{\pgfqpoint{1.358231in}{0.859761in}}{\pgfqpoint{1.366045in}{0.851947in}}%
\pgfpathcurveto{\pgfqpoint{1.373859in}{0.844134in}}{\pgfqpoint{1.384458in}{0.839743in}}{\pgfqpoint{1.395508in}{0.839743in}}%
\pgfpathclose%
\pgfusepath{stroke,fill}%
\end{pgfscope}%
\begin{pgfscope}%
\pgfpathrectangle{\pgfqpoint{0.787074in}{0.548769in}}{\pgfqpoint{5.062926in}{3.102590in}}%
\pgfusepath{clip}%
\pgfsetbuttcap%
\pgfsetroundjoin%
\definecolor{currentfill}{rgb}{0.121569,0.466667,0.705882}%
\pgfsetfillcolor{currentfill}%
\pgfsetlinewidth{1.003750pt}%
\definecolor{currentstroke}{rgb}{0.121569,0.466667,0.705882}%
\pgfsetstrokecolor{currentstroke}%
\pgfsetdash{}{0pt}%
\pgfpathmoveto{\pgfqpoint{1.395508in}{0.786120in}}%
\pgfpathcurveto{\pgfqpoint{1.406558in}{0.786120in}}{\pgfqpoint{1.417157in}{0.790510in}}{\pgfqpoint{1.424970in}{0.798324in}}%
\pgfpathcurveto{\pgfqpoint{1.432784in}{0.806137in}}{\pgfqpoint{1.437174in}{0.816736in}}{\pgfqpoint{1.437174in}{0.827787in}}%
\pgfpathcurveto{\pgfqpoint{1.437174in}{0.838837in}}{\pgfqpoint{1.432784in}{0.849436in}}{\pgfqpoint{1.424970in}{0.857249in}}%
\pgfpathcurveto{\pgfqpoint{1.417157in}{0.865063in}}{\pgfqpoint{1.406558in}{0.869453in}}{\pgfqpoint{1.395508in}{0.869453in}}%
\pgfpathcurveto{\pgfqpoint{1.384458in}{0.869453in}}{\pgfqpoint{1.373859in}{0.865063in}}{\pgfqpoint{1.366045in}{0.857249in}}%
\pgfpathcurveto{\pgfqpoint{1.358231in}{0.849436in}}{\pgfqpoint{1.353841in}{0.838837in}}{\pgfqpoint{1.353841in}{0.827787in}}%
\pgfpathcurveto{\pgfqpoint{1.353841in}{0.816736in}}{\pgfqpoint{1.358231in}{0.806137in}}{\pgfqpoint{1.366045in}{0.798324in}}%
\pgfpathcurveto{\pgfqpoint{1.373859in}{0.790510in}}{\pgfqpoint{1.384458in}{0.786120in}}{\pgfqpoint{1.395508in}{0.786120in}}%
\pgfpathclose%
\pgfusepath{stroke,fill}%
\end{pgfscope}%
\begin{pgfscope}%
\pgfpathrectangle{\pgfqpoint{0.787074in}{0.548769in}}{\pgfqpoint{5.062926in}{3.102590in}}%
\pgfusepath{clip}%
\pgfsetbuttcap%
\pgfsetroundjoin%
\definecolor{currentfill}{rgb}{0.121569,0.466667,0.705882}%
\pgfsetfillcolor{currentfill}%
\pgfsetlinewidth{1.003750pt}%
\definecolor{currentstroke}{rgb}{0.121569,0.466667,0.705882}%
\pgfsetstrokecolor{currentstroke}%
\pgfsetdash{}{0pt}%
\pgfpathmoveto{\pgfqpoint{1.080257in}{0.660513in}}%
\pgfpathcurveto{\pgfqpoint{1.091307in}{0.660513in}}{\pgfqpoint{1.101906in}{0.664903in}}{\pgfqpoint{1.109720in}{0.672717in}}%
\pgfpathcurveto{\pgfqpoint{1.117533in}{0.680531in}}{\pgfqpoint{1.121924in}{0.691130in}}{\pgfqpoint{1.121924in}{0.702180in}}%
\pgfpathcurveto{\pgfqpoint{1.121924in}{0.713230in}}{\pgfqpoint{1.117533in}{0.723829in}}{\pgfqpoint{1.109720in}{0.731643in}}%
\pgfpathcurveto{\pgfqpoint{1.101906in}{0.739456in}}{\pgfqpoint{1.091307in}{0.743847in}}{\pgfqpoint{1.080257in}{0.743847in}}%
\pgfpathcurveto{\pgfqpoint{1.069207in}{0.743847in}}{\pgfqpoint{1.058608in}{0.739456in}}{\pgfqpoint{1.050794in}{0.731643in}}%
\pgfpathcurveto{\pgfqpoint{1.042981in}{0.723829in}}{\pgfqpoint{1.038590in}{0.713230in}}{\pgfqpoint{1.038590in}{0.702180in}}%
\pgfpathcurveto{\pgfqpoint{1.038590in}{0.691130in}}{\pgfqpoint{1.042981in}{0.680531in}}{\pgfqpoint{1.050794in}{0.672717in}}%
\pgfpathcurveto{\pgfqpoint{1.058608in}{0.664903in}}{\pgfqpoint{1.069207in}{0.660513in}}{\pgfqpoint{1.080257in}{0.660513in}}%
\pgfpathclose%
\pgfusepath{stroke,fill}%
\end{pgfscope}%
\begin{pgfscope}%
\pgfpathrectangle{\pgfqpoint{0.787074in}{0.548769in}}{\pgfqpoint{5.062926in}{3.102590in}}%
\pgfusepath{clip}%
\pgfsetbuttcap%
\pgfsetroundjoin%
\definecolor{currentfill}{rgb}{0.121569,0.466667,0.705882}%
\pgfsetfillcolor{currentfill}%
\pgfsetlinewidth{1.003750pt}%
\definecolor{currentstroke}{rgb}{0.121569,0.466667,0.705882}%
\pgfsetstrokecolor{currentstroke}%
\pgfsetdash{}{0pt}%
\pgfpathmoveto{\pgfqpoint{1.521608in}{0.648155in}}%
\pgfpathcurveto{\pgfqpoint{1.532658in}{0.648155in}}{\pgfqpoint{1.543257in}{0.652545in}}{\pgfqpoint{1.551071in}{0.660359in}}%
\pgfpathcurveto{\pgfqpoint{1.558884in}{0.668173in}}{\pgfqpoint{1.563275in}{0.678772in}}{\pgfqpoint{1.563275in}{0.689822in}}%
\pgfpathcurveto{\pgfqpoint{1.563275in}{0.700872in}}{\pgfqpoint{1.558884in}{0.711471in}}{\pgfqpoint{1.551071in}{0.719285in}}%
\pgfpathcurveto{\pgfqpoint{1.543257in}{0.727098in}}{\pgfqpoint{1.532658in}{0.731489in}}{\pgfqpoint{1.521608in}{0.731489in}}%
\pgfpathcurveto{\pgfqpoint{1.510558in}{0.731489in}}{\pgfqpoint{1.499959in}{0.727098in}}{\pgfqpoint{1.492145in}{0.719285in}}%
\pgfpathcurveto{\pgfqpoint{1.484332in}{0.711471in}}{\pgfqpoint{1.479941in}{0.700872in}}{\pgfqpoint{1.479941in}{0.689822in}}%
\pgfpathcurveto{\pgfqpoint{1.479941in}{0.678772in}}{\pgfqpoint{1.484332in}{0.668173in}}{\pgfqpoint{1.492145in}{0.660359in}}%
\pgfpathcurveto{\pgfqpoint{1.499959in}{0.652545in}}{\pgfqpoint{1.510558in}{0.648155in}}{\pgfqpoint{1.521608in}{0.648155in}}%
\pgfpathclose%
\pgfusepath{stroke,fill}%
\end{pgfscope}%
\begin{pgfscope}%
\pgfpathrectangle{\pgfqpoint{0.787074in}{0.548769in}}{\pgfqpoint{5.062926in}{3.102590in}}%
\pgfusepath{clip}%
\pgfsetbuttcap%
\pgfsetroundjoin%
\definecolor{currentfill}{rgb}{0.121569,0.466667,0.705882}%
\pgfsetfillcolor{currentfill}%
\pgfsetlinewidth{1.003750pt}%
\definecolor{currentstroke}{rgb}{0.121569,0.466667,0.705882}%
\pgfsetstrokecolor{currentstroke}%
\pgfsetdash{}{0pt}%
\pgfpathmoveto{\pgfqpoint{1.332458in}{0.648148in}}%
\pgfpathcurveto{\pgfqpoint{1.343508in}{0.648148in}}{\pgfqpoint{1.354107in}{0.652538in}}{\pgfqpoint{1.361920in}{0.660352in}}%
\pgfpathcurveto{\pgfqpoint{1.369734in}{0.668166in}}{\pgfqpoint{1.374124in}{0.678765in}}{\pgfqpoint{1.374124in}{0.689815in}}%
\pgfpathcurveto{\pgfqpoint{1.374124in}{0.700865in}}{\pgfqpoint{1.369734in}{0.711464in}}{\pgfqpoint{1.361920in}{0.719278in}}%
\pgfpathcurveto{\pgfqpoint{1.354107in}{0.727091in}}{\pgfqpoint{1.343508in}{0.731482in}}{\pgfqpoint{1.332458in}{0.731482in}}%
\pgfpathcurveto{\pgfqpoint{1.321407in}{0.731482in}}{\pgfqpoint{1.310808in}{0.727091in}}{\pgfqpoint{1.302995in}{0.719278in}}%
\pgfpathcurveto{\pgfqpoint{1.295181in}{0.711464in}}{\pgfqpoint{1.290791in}{0.700865in}}{\pgfqpoint{1.290791in}{0.689815in}}%
\pgfpathcurveto{\pgfqpoint{1.290791in}{0.678765in}}{\pgfqpoint{1.295181in}{0.668166in}}{\pgfqpoint{1.302995in}{0.660352in}}%
\pgfpathcurveto{\pgfqpoint{1.310808in}{0.652538in}}{\pgfqpoint{1.321407in}{0.648148in}}{\pgfqpoint{1.332458in}{0.648148in}}%
\pgfpathclose%
\pgfusepath{stroke,fill}%
\end{pgfscope}%
\begin{pgfscope}%
\pgfpathrectangle{\pgfqpoint{0.787074in}{0.548769in}}{\pgfqpoint{5.062926in}{3.102590in}}%
\pgfusepath{clip}%
\pgfsetbuttcap%
\pgfsetroundjoin%
\definecolor{currentfill}{rgb}{1.000000,0.498039,0.054902}%
\pgfsetfillcolor{currentfill}%
\pgfsetlinewidth{1.003750pt}%
\definecolor{currentstroke}{rgb}{1.000000,0.498039,0.054902}%
\pgfsetstrokecolor{currentstroke}%
\pgfsetdash{}{0pt}%
\pgfpathmoveto{\pgfqpoint{1.521608in}{2.337155in}}%
\pgfpathcurveto{\pgfqpoint{1.532658in}{2.337155in}}{\pgfqpoint{1.543257in}{2.341545in}}{\pgfqpoint{1.551071in}{2.349358in}}%
\pgfpathcurveto{\pgfqpoint{1.558884in}{2.357172in}}{\pgfqpoint{1.563275in}{2.367771in}}{\pgfqpoint{1.563275in}{2.378821in}}%
\pgfpathcurveto{\pgfqpoint{1.563275in}{2.389871in}}{\pgfqpoint{1.558884in}{2.400470in}}{\pgfqpoint{1.551071in}{2.408284in}}%
\pgfpathcurveto{\pgfqpoint{1.543257in}{2.416098in}}{\pgfqpoint{1.532658in}{2.420488in}}{\pgfqpoint{1.521608in}{2.420488in}}%
\pgfpathcurveto{\pgfqpoint{1.510558in}{2.420488in}}{\pgfqpoint{1.499959in}{2.416098in}}{\pgfqpoint{1.492145in}{2.408284in}}%
\pgfpathcurveto{\pgfqpoint{1.484332in}{2.400470in}}{\pgfqpoint{1.479941in}{2.389871in}}{\pgfqpoint{1.479941in}{2.378821in}}%
\pgfpathcurveto{\pgfqpoint{1.479941in}{2.367771in}}{\pgfqpoint{1.484332in}{2.357172in}}{\pgfqpoint{1.492145in}{2.349358in}}%
\pgfpathcurveto{\pgfqpoint{1.499959in}{2.341545in}}{\pgfqpoint{1.510558in}{2.337155in}}{\pgfqpoint{1.521608in}{2.337155in}}%
\pgfpathclose%
\pgfusepath{stroke,fill}%
\end{pgfscope}%
\begin{pgfscope}%
\pgfpathrectangle{\pgfqpoint{0.787074in}{0.548769in}}{\pgfqpoint{5.062926in}{3.102590in}}%
\pgfusepath{clip}%
\pgfsetbuttcap%
\pgfsetroundjoin%
\definecolor{currentfill}{rgb}{1.000000,0.498039,0.054902}%
\pgfsetfillcolor{currentfill}%
\pgfsetlinewidth{1.003750pt}%
\definecolor{currentstroke}{rgb}{1.000000,0.498039,0.054902}%
\pgfsetstrokecolor{currentstroke}%
\pgfsetdash{}{0pt}%
\pgfpathmoveto{\pgfqpoint{2.467360in}{2.913319in}}%
\pgfpathcurveto{\pgfqpoint{2.478410in}{2.913319in}}{\pgfqpoint{2.489009in}{2.917709in}}{\pgfqpoint{2.496823in}{2.925523in}}%
\pgfpathcurveto{\pgfqpoint{2.504636in}{2.933336in}}{\pgfqpoint{2.509027in}{2.943935in}}{\pgfqpoint{2.509027in}{2.954985in}}%
\pgfpathcurveto{\pgfqpoint{2.509027in}{2.966036in}}{\pgfqpoint{2.504636in}{2.976635in}}{\pgfqpoint{2.496823in}{2.984448in}}%
\pgfpathcurveto{\pgfqpoint{2.489009in}{2.992262in}}{\pgfqpoint{2.478410in}{2.996652in}}{\pgfqpoint{2.467360in}{2.996652in}}%
\pgfpathcurveto{\pgfqpoint{2.456310in}{2.996652in}}{\pgfqpoint{2.445711in}{2.992262in}}{\pgfqpoint{2.437897in}{2.984448in}}%
\pgfpathcurveto{\pgfqpoint{2.430084in}{2.976635in}}{\pgfqpoint{2.425693in}{2.966036in}}{\pgfqpoint{2.425693in}{2.954985in}}%
\pgfpathcurveto{\pgfqpoint{2.425693in}{2.943935in}}{\pgfqpoint{2.430084in}{2.933336in}}{\pgfqpoint{2.437897in}{2.925523in}}%
\pgfpathcurveto{\pgfqpoint{2.445711in}{2.917709in}}{\pgfqpoint{2.456310in}{2.913319in}}{\pgfqpoint{2.467360in}{2.913319in}}%
\pgfpathclose%
\pgfusepath{stroke,fill}%
\end{pgfscope}%
\begin{pgfscope}%
\pgfpathrectangle{\pgfqpoint{0.787074in}{0.548769in}}{\pgfqpoint{5.062926in}{3.102590in}}%
\pgfusepath{clip}%
\pgfsetbuttcap%
\pgfsetroundjoin%
\definecolor{currentfill}{rgb}{0.121569,0.466667,0.705882}%
\pgfsetfillcolor{currentfill}%
\pgfsetlinewidth{1.003750pt}%
\definecolor{currentstroke}{rgb}{0.121569,0.466667,0.705882}%
\pgfsetstrokecolor{currentstroke}%
\pgfsetdash{}{0pt}%
\pgfpathmoveto{\pgfqpoint{1.584658in}{0.825123in}}%
\pgfpathcurveto{\pgfqpoint{1.595708in}{0.825123in}}{\pgfqpoint{1.606307in}{0.829514in}}{\pgfqpoint{1.614121in}{0.837327in}}%
\pgfpathcurveto{\pgfqpoint{1.621935in}{0.845141in}}{\pgfqpoint{1.626325in}{0.855740in}}{\pgfqpoint{1.626325in}{0.866790in}}%
\pgfpathcurveto{\pgfqpoint{1.626325in}{0.877840in}}{\pgfqpoint{1.621935in}{0.888439in}}{\pgfqpoint{1.614121in}{0.896253in}}%
\pgfpathcurveto{\pgfqpoint{1.606307in}{0.904067in}}{\pgfqpoint{1.595708in}{0.908457in}}{\pgfqpoint{1.584658in}{0.908457in}}%
\pgfpathcurveto{\pgfqpoint{1.573608in}{0.908457in}}{\pgfqpoint{1.563009in}{0.904067in}}{\pgfqpoint{1.555195in}{0.896253in}}%
\pgfpathcurveto{\pgfqpoint{1.547382in}{0.888439in}}{\pgfqpoint{1.542991in}{0.877840in}}{\pgfqpoint{1.542991in}{0.866790in}}%
\pgfpathcurveto{\pgfqpoint{1.542991in}{0.855740in}}{\pgfqpoint{1.547382in}{0.845141in}}{\pgfqpoint{1.555195in}{0.837327in}}%
\pgfpathcurveto{\pgfqpoint{1.563009in}{0.829514in}}{\pgfqpoint{1.573608in}{0.825123in}}{\pgfqpoint{1.584658in}{0.825123in}}%
\pgfpathclose%
\pgfusepath{stroke,fill}%
\end{pgfscope}%
\begin{pgfscope}%
\pgfpathrectangle{\pgfqpoint{0.787074in}{0.548769in}}{\pgfqpoint{5.062926in}{3.102590in}}%
\pgfusepath{clip}%
\pgfsetbuttcap%
\pgfsetroundjoin%
\definecolor{currentfill}{rgb}{1.000000,0.498039,0.054902}%
\pgfsetfillcolor{currentfill}%
\pgfsetlinewidth{1.003750pt}%
\definecolor{currentstroke}{rgb}{1.000000,0.498039,0.054902}%
\pgfsetstrokecolor{currentstroke}%
\pgfsetdash{}{0pt}%
\pgfpathmoveto{\pgfqpoint{1.584658in}{1.776870in}}%
\pgfpathcurveto{\pgfqpoint{1.595708in}{1.776870in}}{\pgfqpoint{1.606307in}{1.781260in}}{\pgfqpoint{1.614121in}{1.789074in}}%
\pgfpathcurveto{\pgfqpoint{1.621935in}{1.796887in}}{\pgfqpoint{1.626325in}{1.807486in}}{\pgfqpoint{1.626325in}{1.818537in}}%
\pgfpathcurveto{\pgfqpoint{1.626325in}{1.829587in}}{\pgfqpoint{1.621935in}{1.840186in}}{\pgfqpoint{1.614121in}{1.847999in}}%
\pgfpathcurveto{\pgfqpoint{1.606307in}{1.855813in}}{\pgfqpoint{1.595708in}{1.860203in}}{\pgfqpoint{1.584658in}{1.860203in}}%
\pgfpathcurveto{\pgfqpoint{1.573608in}{1.860203in}}{\pgfqpoint{1.563009in}{1.855813in}}{\pgfqpoint{1.555195in}{1.847999in}}%
\pgfpathcurveto{\pgfqpoint{1.547382in}{1.840186in}}{\pgfqpoint{1.542991in}{1.829587in}}{\pgfqpoint{1.542991in}{1.818537in}}%
\pgfpathcurveto{\pgfqpoint{1.542991in}{1.807486in}}{\pgfqpoint{1.547382in}{1.796887in}}{\pgfqpoint{1.555195in}{1.789074in}}%
\pgfpathcurveto{\pgfqpoint{1.563009in}{1.781260in}}{\pgfqpoint{1.573608in}{1.776870in}}{\pgfqpoint{1.584658in}{1.776870in}}%
\pgfpathclose%
\pgfusepath{stroke,fill}%
\end{pgfscope}%
\begin{pgfscope}%
\pgfpathrectangle{\pgfqpoint{0.787074in}{0.548769in}}{\pgfqpoint{5.062926in}{3.102590in}}%
\pgfusepath{clip}%
\pgfsetbuttcap%
\pgfsetroundjoin%
\definecolor{currentfill}{rgb}{1.000000,0.498039,0.054902}%
\pgfsetfillcolor{currentfill}%
\pgfsetlinewidth{1.003750pt}%
\definecolor{currentstroke}{rgb}{1.000000,0.498039,0.054902}%
\pgfsetstrokecolor{currentstroke}%
\pgfsetdash{}{0pt}%
\pgfpathmoveto{\pgfqpoint{1.647708in}{2.789062in}}%
\pgfpathcurveto{\pgfqpoint{1.658758in}{2.789062in}}{\pgfqpoint{1.669357in}{2.793453in}}{\pgfqpoint{1.677171in}{2.801266in}}%
\pgfpathcurveto{\pgfqpoint{1.684985in}{2.809080in}}{\pgfqpoint{1.689375in}{2.819679in}}{\pgfqpoint{1.689375in}{2.830729in}}%
\pgfpathcurveto{\pgfqpoint{1.689375in}{2.841779in}}{\pgfqpoint{1.684985in}{2.852378in}}{\pgfqpoint{1.677171in}{2.860192in}}%
\pgfpathcurveto{\pgfqpoint{1.669357in}{2.868005in}}{\pgfqpoint{1.658758in}{2.872396in}}{\pgfqpoint{1.647708in}{2.872396in}}%
\pgfpathcurveto{\pgfqpoint{1.636658in}{2.872396in}}{\pgfqpoint{1.626059in}{2.868005in}}{\pgfqpoint{1.618245in}{2.860192in}}%
\pgfpathcurveto{\pgfqpoint{1.610432in}{2.852378in}}{\pgfqpoint{1.606042in}{2.841779in}}{\pgfqpoint{1.606042in}{2.830729in}}%
\pgfpathcurveto{\pgfqpoint{1.606042in}{2.819679in}}{\pgfqpoint{1.610432in}{2.809080in}}{\pgfqpoint{1.618245in}{2.801266in}}%
\pgfpathcurveto{\pgfqpoint{1.626059in}{2.793453in}}{\pgfqpoint{1.636658in}{2.789062in}}{\pgfqpoint{1.647708in}{2.789062in}}%
\pgfpathclose%
\pgfusepath{stroke,fill}%
\end{pgfscope}%
\begin{pgfscope}%
\pgfpathrectangle{\pgfqpoint{0.787074in}{0.548769in}}{\pgfqpoint{5.062926in}{3.102590in}}%
\pgfusepath{clip}%
\pgfsetbuttcap%
\pgfsetroundjoin%
\definecolor{currentfill}{rgb}{0.121569,0.466667,0.705882}%
\pgfsetfillcolor{currentfill}%
\pgfsetlinewidth{1.003750pt}%
\definecolor{currentstroke}{rgb}{0.121569,0.466667,0.705882}%
\pgfsetstrokecolor{currentstroke}%
\pgfsetdash{}{0pt}%
\pgfpathmoveto{\pgfqpoint{1.332458in}{0.648149in}}%
\pgfpathcurveto{\pgfqpoint{1.343508in}{0.648149in}}{\pgfqpoint{1.354107in}{0.652540in}}{\pgfqpoint{1.361920in}{0.660353in}}%
\pgfpathcurveto{\pgfqpoint{1.369734in}{0.668167in}}{\pgfqpoint{1.374124in}{0.678766in}}{\pgfqpoint{1.374124in}{0.689816in}}%
\pgfpathcurveto{\pgfqpoint{1.374124in}{0.700866in}}{\pgfqpoint{1.369734in}{0.711465in}}{\pgfqpoint{1.361920in}{0.719279in}}%
\pgfpathcurveto{\pgfqpoint{1.354107in}{0.727093in}}{\pgfqpoint{1.343508in}{0.731483in}}{\pgfqpoint{1.332458in}{0.731483in}}%
\pgfpathcurveto{\pgfqpoint{1.321407in}{0.731483in}}{\pgfqpoint{1.310808in}{0.727093in}}{\pgfqpoint{1.302995in}{0.719279in}}%
\pgfpathcurveto{\pgfqpoint{1.295181in}{0.711465in}}{\pgfqpoint{1.290791in}{0.700866in}}{\pgfqpoint{1.290791in}{0.689816in}}%
\pgfpathcurveto{\pgfqpoint{1.290791in}{0.678766in}}{\pgfqpoint{1.295181in}{0.668167in}}{\pgfqpoint{1.302995in}{0.660353in}}%
\pgfpathcurveto{\pgfqpoint{1.310808in}{0.652540in}}{\pgfqpoint{1.321407in}{0.648149in}}{\pgfqpoint{1.332458in}{0.648149in}}%
\pgfpathclose%
\pgfusepath{stroke,fill}%
\end{pgfscope}%
\begin{pgfscope}%
\pgfpathrectangle{\pgfqpoint{0.787074in}{0.548769in}}{\pgfqpoint{5.062926in}{3.102590in}}%
\pgfusepath{clip}%
\pgfsetbuttcap%
\pgfsetroundjoin%
\definecolor{currentfill}{rgb}{1.000000,0.498039,0.054902}%
\pgfsetfillcolor{currentfill}%
\pgfsetlinewidth{1.003750pt}%
\definecolor{currentstroke}{rgb}{1.000000,0.498039,0.054902}%
\pgfsetstrokecolor{currentstroke}%
\pgfsetdash{}{0pt}%
\pgfpathmoveto{\pgfqpoint{2.467360in}{2.789836in}}%
\pgfpathcurveto{\pgfqpoint{2.478410in}{2.789836in}}{\pgfqpoint{2.489009in}{2.794226in}}{\pgfqpoint{2.496823in}{2.802040in}}%
\pgfpathcurveto{\pgfqpoint{2.504636in}{2.809854in}}{\pgfqpoint{2.509027in}{2.820453in}}{\pgfqpoint{2.509027in}{2.831503in}}%
\pgfpathcurveto{\pgfqpoint{2.509027in}{2.842553in}}{\pgfqpoint{2.504636in}{2.853152in}}{\pgfqpoint{2.496823in}{2.860966in}}%
\pgfpathcurveto{\pgfqpoint{2.489009in}{2.868779in}}{\pgfqpoint{2.478410in}{2.873169in}}{\pgfqpoint{2.467360in}{2.873169in}}%
\pgfpathcurveto{\pgfqpoint{2.456310in}{2.873169in}}{\pgfqpoint{2.445711in}{2.868779in}}{\pgfqpoint{2.437897in}{2.860966in}}%
\pgfpathcurveto{\pgfqpoint{2.430084in}{2.853152in}}{\pgfqpoint{2.425693in}{2.842553in}}{\pgfqpoint{2.425693in}{2.831503in}}%
\pgfpathcurveto{\pgfqpoint{2.425693in}{2.820453in}}{\pgfqpoint{2.430084in}{2.809854in}}{\pgfqpoint{2.437897in}{2.802040in}}%
\pgfpathcurveto{\pgfqpoint{2.445711in}{2.794226in}}{\pgfqpoint{2.456310in}{2.789836in}}{\pgfqpoint{2.467360in}{2.789836in}}%
\pgfpathclose%
\pgfusepath{stroke,fill}%
\end{pgfscope}%
\begin{pgfscope}%
\pgfpathrectangle{\pgfqpoint{0.787074in}{0.548769in}}{\pgfqpoint{5.062926in}{3.102590in}}%
\pgfusepath{clip}%
\pgfsetbuttcap%
\pgfsetroundjoin%
\definecolor{currentfill}{rgb}{0.121569,0.466667,0.705882}%
\pgfsetfillcolor{currentfill}%
\pgfsetlinewidth{1.003750pt}%
\definecolor{currentstroke}{rgb}{0.121569,0.466667,0.705882}%
\pgfsetstrokecolor{currentstroke}%
\pgfsetdash{}{0pt}%
\pgfpathmoveto{\pgfqpoint{1.332458in}{0.648148in}}%
\pgfpathcurveto{\pgfqpoint{1.343508in}{0.648148in}}{\pgfqpoint{1.354107in}{0.652539in}}{\pgfqpoint{1.361920in}{0.660352in}}%
\pgfpathcurveto{\pgfqpoint{1.369734in}{0.668166in}}{\pgfqpoint{1.374124in}{0.678765in}}{\pgfqpoint{1.374124in}{0.689815in}}%
\pgfpathcurveto{\pgfqpoint{1.374124in}{0.700865in}}{\pgfqpoint{1.369734in}{0.711464in}}{\pgfqpoint{1.361920in}{0.719278in}}%
\pgfpathcurveto{\pgfqpoint{1.354107in}{0.727091in}}{\pgfqpoint{1.343508in}{0.731482in}}{\pgfqpoint{1.332458in}{0.731482in}}%
\pgfpathcurveto{\pgfqpoint{1.321407in}{0.731482in}}{\pgfqpoint{1.310808in}{0.727091in}}{\pgfqpoint{1.302995in}{0.719278in}}%
\pgfpathcurveto{\pgfqpoint{1.295181in}{0.711464in}}{\pgfqpoint{1.290791in}{0.700865in}}{\pgfqpoint{1.290791in}{0.689815in}}%
\pgfpathcurveto{\pgfqpoint{1.290791in}{0.678765in}}{\pgfqpoint{1.295181in}{0.668166in}}{\pgfqpoint{1.302995in}{0.660352in}}%
\pgfpathcurveto{\pgfqpoint{1.310808in}{0.652539in}}{\pgfqpoint{1.321407in}{0.648148in}}{\pgfqpoint{1.332458in}{0.648148in}}%
\pgfpathclose%
\pgfusepath{stroke,fill}%
\end{pgfscope}%
\begin{pgfscope}%
\pgfpathrectangle{\pgfqpoint{0.787074in}{0.548769in}}{\pgfqpoint{5.062926in}{3.102590in}}%
\pgfusepath{clip}%
\pgfsetbuttcap%
\pgfsetroundjoin%
\definecolor{currentfill}{rgb}{0.121569,0.466667,0.705882}%
\pgfsetfillcolor{currentfill}%
\pgfsetlinewidth{1.003750pt}%
\definecolor{currentstroke}{rgb}{0.121569,0.466667,0.705882}%
\pgfsetstrokecolor{currentstroke}%
\pgfsetdash{}{0pt}%
\pgfpathmoveto{\pgfqpoint{1.017207in}{0.648168in}}%
\pgfpathcurveto{\pgfqpoint{1.028257in}{0.648168in}}{\pgfqpoint{1.038856in}{0.652558in}}{\pgfqpoint{1.046670in}{0.660371in}}%
\pgfpathcurveto{\pgfqpoint{1.054483in}{0.668185in}}{\pgfqpoint{1.058874in}{0.678784in}}{\pgfqpoint{1.058874in}{0.689834in}}%
\pgfpathcurveto{\pgfqpoint{1.058874in}{0.700884in}}{\pgfqpoint{1.054483in}{0.711483in}}{\pgfqpoint{1.046670in}{0.719297in}}%
\pgfpathcurveto{\pgfqpoint{1.038856in}{0.727111in}}{\pgfqpoint{1.028257in}{0.731501in}}{\pgfqpoint{1.017207in}{0.731501in}}%
\pgfpathcurveto{\pgfqpoint{1.006157in}{0.731501in}}{\pgfqpoint{0.995558in}{0.727111in}}{\pgfqpoint{0.987744in}{0.719297in}}%
\pgfpathcurveto{\pgfqpoint{0.979930in}{0.711483in}}{\pgfqpoint{0.975540in}{0.700884in}}{\pgfqpoint{0.975540in}{0.689834in}}%
\pgfpathcurveto{\pgfqpoint{0.975540in}{0.678784in}}{\pgfqpoint{0.979930in}{0.668185in}}{\pgfqpoint{0.987744in}{0.660371in}}%
\pgfpathcurveto{\pgfqpoint{0.995558in}{0.652558in}}{\pgfqpoint{1.006157in}{0.648168in}}{\pgfqpoint{1.017207in}{0.648168in}}%
\pgfpathclose%
\pgfusepath{stroke,fill}%
\end{pgfscope}%
\begin{pgfscope}%
\pgfpathrectangle{\pgfqpoint{0.787074in}{0.548769in}}{\pgfqpoint{5.062926in}{3.102590in}}%
\pgfusepath{clip}%
\pgfsetbuttcap%
\pgfsetroundjoin%
\definecolor{currentfill}{rgb}{0.121569,0.466667,0.705882}%
\pgfsetfillcolor{currentfill}%
\pgfsetlinewidth{1.003750pt}%
\definecolor{currentstroke}{rgb}{0.121569,0.466667,0.705882}%
\pgfsetstrokecolor{currentstroke}%
\pgfsetdash{}{0pt}%
\pgfpathmoveto{\pgfqpoint{1.647708in}{0.858168in}}%
\pgfpathcurveto{\pgfqpoint{1.658758in}{0.858168in}}{\pgfqpoint{1.669357in}{0.862558in}}{\pgfqpoint{1.677171in}{0.870372in}}%
\pgfpathcurveto{\pgfqpoint{1.684985in}{0.878185in}}{\pgfqpoint{1.689375in}{0.888784in}}{\pgfqpoint{1.689375in}{0.899835in}}%
\pgfpathcurveto{\pgfqpoint{1.689375in}{0.910885in}}{\pgfqpoint{1.684985in}{0.921484in}}{\pgfqpoint{1.677171in}{0.929297in}}%
\pgfpathcurveto{\pgfqpoint{1.669357in}{0.937111in}}{\pgfqpoint{1.658758in}{0.941501in}}{\pgfqpoint{1.647708in}{0.941501in}}%
\pgfpathcurveto{\pgfqpoint{1.636658in}{0.941501in}}{\pgfqpoint{1.626059in}{0.937111in}}{\pgfqpoint{1.618245in}{0.929297in}}%
\pgfpathcurveto{\pgfqpoint{1.610432in}{0.921484in}}{\pgfqpoint{1.606042in}{0.910885in}}{\pgfqpoint{1.606042in}{0.899835in}}%
\pgfpathcurveto{\pgfqpoint{1.606042in}{0.888784in}}{\pgfqpoint{1.610432in}{0.878185in}}{\pgfqpoint{1.618245in}{0.870372in}}%
\pgfpathcurveto{\pgfqpoint{1.626059in}{0.862558in}}{\pgfqpoint{1.636658in}{0.858168in}}{\pgfqpoint{1.647708in}{0.858168in}}%
\pgfpathclose%
\pgfusepath{stroke,fill}%
\end{pgfscope}%
\begin{pgfscope}%
\pgfpathrectangle{\pgfqpoint{0.787074in}{0.548769in}}{\pgfqpoint{5.062926in}{3.102590in}}%
\pgfusepath{clip}%
\pgfsetbuttcap%
\pgfsetroundjoin%
\definecolor{currentfill}{rgb}{1.000000,0.498039,0.054902}%
\pgfsetfillcolor{currentfill}%
\pgfsetlinewidth{1.003750pt}%
\definecolor{currentstroke}{rgb}{1.000000,0.498039,0.054902}%
\pgfsetstrokecolor{currentstroke}%
\pgfsetdash{}{0pt}%
\pgfpathmoveto{\pgfqpoint{4.043614in}{2.478569in}}%
\pgfpathcurveto{\pgfqpoint{4.054664in}{2.478569in}}{\pgfqpoint{4.065263in}{2.482959in}}{\pgfqpoint{4.073076in}{2.490773in}}%
\pgfpathcurveto{\pgfqpoint{4.080890in}{2.498586in}}{\pgfqpoint{4.085280in}{2.509185in}}{\pgfqpoint{4.085280in}{2.520235in}}%
\pgfpathcurveto{\pgfqpoint{4.085280in}{2.531286in}}{\pgfqpoint{4.080890in}{2.541885in}}{\pgfqpoint{4.073076in}{2.549698in}}%
\pgfpathcurveto{\pgfqpoint{4.065263in}{2.557512in}}{\pgfqpoint{4.054664in}{2.561902in}}{\pgfqpoint{4.043614in}{2.561902in}}%
\pgfpathcurveto{\pgfqpoint{4.032563in}{2.561902in}}{\pgfqpoint{4.021964in}{2.557512in}}{\pgfqpoint{4.014151in}{2.549698in}}%
\pgfpathcurveto{\pgfqpoint{4.006337in}{2.541885in}}{\pgfqpoint{4.001947in}{2.531286in}}{\pgfqpoint{4.001947in}{2.520235in}}%
\pgfpathcurveto{\pgfqpoint{4.001947in}{2.509185in}}{\pgfqpoint{4.006337in}{2.498586in}}{\pgfqpoint{4.014151in}{2.490773in}}%
\pgfpathcurveto{\pgfqpoint{4.021964in}{2.482959in}}{\pgfqpoint{4.032563in}{2.478569in}}{\pgfqpoint{4.043614in}{2.478569in}}%
\pgfpathclose%
\pgfusepath{stroke,fill}%
\end{pgfscope}%
\begin{pgfscope}%
\pgfpathrectangle{\pgfqpoint{0.787074in}{0.548769in}}{\pgfqpoint{5.062926in}{3.102590in}}%
\pgfusepath{clip}%
\pgfsetbuttcap%
\pgfsetroundjoin%
\definecolor{currentfill}{rgb}{1.000000,0.498039,0.054902}%
\pgfsetfillcolor{currentfill}%
\pgfsetlinewidth{1.003750pt}%
\definecolor{currentstroke}{rgb}{1.000000,0.498039,0.054902}%
\pgfsetstrokecolor{currentstroke}%
\pgfsetdash{}{0pt}%
\pgfpathmoveto{\pgfqpoint{1.269407in}{1.627903in}}%
\pgfpathcurveto{\pgfqpoint{1.280458in}{1.627903in}}{\pgfqpoint{1.291057in}{1.632294in}}{\pgfqpoint{1.298870in}{1.640107in}}%
\pgfpathcurveto{\pgfqpoint{1.306684in}{1.647921in}}{\pgfqpoint{1.311074in}{1.658520in}}{\pgfqpoint{1.311074in}{1.669570in}}%
\pgfpathcurveto{\pgfqpoint{1.311074in}{1.680620in}}{\pgfqpoint{1.306684in}{1.691219in}}{\pgfqpoint{1.298870in}{1.699033in}}%
\pgfpathcurveto{\pgfqpoint{1.291057in}{1.706846in}}{\pgfqpoint{1.280458in}{1.711237in}}{\pgfqpoint{1.269407in}{1.711237in}}%
\pgfpathcurveto{\pgfqpoint{1.258357in}{1.711237in}}{\pgfqpoint{1.247758in}{1.706846in}}{\pgfqpoint{1.239945in}{1.699033in}}%
\pgfpathcurveto{\pgfqpoint{1.232131in}{1.691219in}}{\pgfqpoint{1.227741in}{1.680620in}}{\pgfqpoint{1.227741in}{1.669570in}}%
\pgfpathcurveto{\pgfqpoint{1.227741in}{1.658520in}}{\pgfqpoint{1.232131in}{1.647921in}}{\pgfqpoint{1.239945in}{1.640107in}}%
\pgfpathcurveto{\pgfqpoint{1.247758in}{1.632294in}}{\pgfqpoint{1.258357in}{1.627903in}}{\pgfqpoint{1.269407in}{1.627903in}}%
\pgfpathclose%
\pgfusepath{stroke,fill}%
\end{pgfscope}%
\begin{pgfscope}%
\pgfpathrectangle{\pgfqpoint{0.787074in}{0.548769in}}{\pgfqpoint{5.062926in}{3.102590in}}%
\pgfusepath{clip}%
\pgfsetbuttcap%
\pgfsetroundjoin%
\definecolor{currentfill}{rgb}{0.121569,0.466667,0.705882}%
\pgfsetfillcolor{currentfill}%
\pgfsetlinewidth{1.003750pt}%
\definecolor{currentstroke}{rgb}{0.121569,0.466667,0.705882}%
\pgfsetstrokecolor{currentstroke}%
\pgfsetdash{}{0pt}%
\pgfpathmoveto{\pgfqpoint{1.080257in}{0.648140in}}%
\pgfpathcurveto{\pgfqpoint{1.091307in}{0.648140in}}{\pgfqpoint{1.101906in}{0.652531in}}{\pgfqpoint{1.109720in}{0.660344in}}%
\pgfpathcurveto{\pgfqpoint{1.117533in}{0.668158in}}{\pgfqpoint{1.121924in}{0.678757in}}{\pgfqpoint{1.121924in}{0.689807in}}%
\pgfpathcurveto{\pgfqpoint{1.121924in}{0.700857in}}{\pgfqpoint{1.117533in}{0.711456in}}{\pgfqpoint{1.109720in}{0.719270in}}%
\pgfpathcurveto{\pgfqpoint{1.101906in}{0.727084in}}{\pgfqpoint{1.091307in}{0.731474in}}{\pgfqpoint{1.080257in}{0.731474in}}%
\pgfpathcurveto{\pgfqpoint{1.069207in}{0.731474in}}{\pgfqpoint{1.058608in}{0.727084in}}{\pgfqpoint{1.050794in}{0.719270in}}%
\pgfpathcurveto{\pgfqpoint{1.042981in}{0.711456in}}{\pgfqpoint{1.038590in}{0.700857in}}{\pgfqpoint{1.038590in}{0.689807in}}%
\pgfpathcurveto{\pgfqpoint{1.038590in}{0.678757in}}{\pgfqpoint{1.042981in}{0.668158in}}{\pgfqpoint{1.050794in}{0.660344in}}%
\pgfpathcurveto{\pgfqpoint{1.058608in}{0.652531in}}{\pgfqpoint{1.069207in}{0.648140in}}{\pgfqpoint{1.080257in}{0.648140in}}%
\pgfpathclose%
\pgfusepath{stroke,fill}%
\end{pgfscope}%
\begin{pgfscope}%
\pgfpathrectangle{\pgfqpoint{0.787074in}{0.548769in}}{\pgfqpoint{5.062926in}{3.102590in}}%
\pgfusepath{clip}%
\pgfsetbuttcap%
\pgfsetroundjoin%
\definecolor{currentfill}{rgb}{0.121569,0.466667,0.705882}%
\pgfsetfillcolor{currentfill}%
\pgfsetlinewidth{1.003750pt}%
\definecolor{currentstroke}{rgb}{0.121569,0.466667,0.705882}%
\pgfsetstrokecolor{currentstroke}%
\pgfsetdash{}{0pt}%
\pgfpathmoveto{\pgfqpoint{1.395508in}{0.786141in}}%
\pgfpathcurveto{\pgfqpoint{1.406558in}{0.786141in}}{\pgfqpoint{1.417157in}{0.790531in}}{\pgfqpoint{1.424970in}{0.798345in}}%
\pgfpathcurveto{\pgfqpoint{1.432784in}{0.806159in}}{\pgfqpoint{1.437174in}{0.816758in}}{\pgfqpoint{1.437174in}{0.827808in}}%
\pgfpathcurveto{\pgfqpoint{1.437174in}{0.838858in}}{\pgfqpoint{1.432784in}{0.849457in}}{\pgfqpoint{1.424970in}{0.857271in}}%
\pgfpathcurveto{\pgfqpoint{1.417157in}{0.865084in}}{\pgfqpoint{1.406558in}{0.869474in}}{\pgfqpoint{1.395508in}{0.869474in}}%
\pgfpathcurveto{\pgfqpoint{1.384458in}{0.869474in}}{\pgfqpoint{1.373859in}{0.865084in}}{\pgfqpoint{1.366045in}{0.857271in}}%
\pgfpathcurveto{\pgfqpoint{1.358231in}{0.849457in}}{\pgfqpoint{1.353841in}{0.838858in}}{\pgfqpoint{1.353841in}{0.827808in}}%
\pgfpathcurveto{\pgfqpoint{1.353841in}{0.816758in}}{\pgfqpoint{1.358231in}{0.806159in}}{\pgfqpoint{1.366045in}{0.798345in}}%
\pgfpathcurveto{\pgfqpoint{1.373859in}{0.790531in}}{\pgfqpoint{1.384458in}{0.786141in}}{\pgfqpoint{1.395508in}{0.786141in}}%
\pgfpathclose%
\pgfusepath{stroke,fill}%
\end{pgfscope}%
\begin{pgfscope}%
\pgfpathrectangle{\pgfqpoint{0.787074in}{0.548769in}}{\pgfqpoint{5.062926in}{3.102590in}}%
\pgfusepath{clip}%
\pgfsetbuttcap%
\pgfsetroundjoin%
\definecolor{currentfill}{rgb}{1.000000,0.498039,0.054902}%
\pgfsetfillcolor{currentfill}%
\pgfsetlinewidth{1.003750pt}%
\definecolor{currentstroke}{rgb}{1.000000,0.498039,0.054902}%
\pgfsetstrokecolor{currentstroke}%
\pgfsetdash{}{0pt}%
\pgfpathmoveto{\pgfqpoint{1.962959in}{2.612758in}}%
\pgfpathcurveto{\pgfqpoint{1.974009in}{2.612758in}}{\pgfqpoint{1.984608in}{2.617148in}}{\pgfqpoint{1.992422in}{2.624962in}}%
\pgfpathcurveto{\pgfqpoint{2.000235in}{2.632775in}}{\pgfqpoint{2.004626in}{2.643374in}}{\pgfqpoint{2.004626in}{2.654424in}}%
\pgfpathcurveto{\pgfqpoint{2.004626in}{2.665474in}}{\pgfqpoint{2.000235in}{2.676074in}}{\pgfqpoint{1.992422in}{2.683887in}}%
\pgfpathcurveto{\pgfqpoint{1.984608in}{2.691701in}}{\pgfqpoint{1.974009in}{2.696091in}}{\pgfqpoint{1.962959in}{2.696091in}}%
\pgfpathcurveto{\pgfqpoint{1.951909in}{2.696091in}}{\pgfqpoint{1.941310in}{2.691701in}}{\pgfqpoint{1.933496in}{2.683887in}}%
\pgfpathcurveto{\pgfqpoint{1.925683in}{2.676074in}}{\pgfqpoint{1.921292in}{2.665474in}}{\pgfqpoint{1.921292in}{2.654424in}}%
\pgfpathcurveto{\pgfqpoint{1.921292in}{2.643374in}}{\pgfqpoint{1.925683in}{2.632775in}}{\pgfqpoint{1.933496in}{2.624962in}}%
\pgfpathcurveto{\pgfqpoint{1.941310in}{2.617148in}}{\pgfqpoint{1.951909in}{2.612758in}}{\pgfqpoint{1.962959in}{2.612758in}}%
\pgfpathclose%
\pgfusepath{stroke,fill}%
\end{pgfscope}%
\begin{pgfscope}%
\pgfpathrectangle{\pgfqpoint{0.787074in}{0.548769in}}{\pgfqpoint{5.062926in}{3.102590in}}%
\pgfusepath{clip}%
\pgfsetbuttcap%
\pgfsetroundjoin%
\definecolor{currentfill}{rgb}{0.121569,0.466667,0.705882}%
\pgfsetfillcolor{currentfill}%
\pgfsetlinewidth{1.003750pt}%
\definecolor{currentstroke}{rgb}{0.121569,0.466667,0.705882}%
\pgfsetstrokecolor{currentstroke}%
\pgfsetdash{}{0pt}%
\pgfpathmoveto{\pgfqpoint{1.017207in}{0.648129in}}%
\pgfpathcurveto{\pgfqpoint{1.028257in}{0.648129in}}{\pgfqpoint{1.038856in}{0.652519in}}{\pgfqpoint{1.046670in}{0.660333in}}%
\pgfpathcurveto{\pgfqpoint{1.054483in}{0.668146in}}{\pgfqpoint{1.058874in}{0.678745in}}{\pgfqpoint{1.058874in}{0.689796in}}%
\pgfpathcurveto{\pgfqpoint{1.058874in}{0.700846in}}{\pgfqpoint{1.054483in}{0.711445in}}{\pgfqpoint{1.046670in}{0.719258in}}%
\pgfpathcurveto{\pgfqpoint{1.038856in}{0.727072in}}{\pgfqpoint{1.028257in}{0.731462in}}{\pgfqpoint{1.017207in}{0.731462in}}%
\pgfpathcurveto{\pgfqpoint{1.006157in}{0.731462in}}{\pgfqpoint{0.995558in}{0.727072in}}{\pgfqpoint{0.987744in}{0.719258in}}%
\pgfpathcurveto{\pgfqpoint{0.979930in}{0.711445in}}{\pgfqpoint{0.975540in}{0.700846in}}{\pgfqpoint{0.975540in}{0.689796in}}%
\pgfpathcurveto{\pgfqpoint{0.975540in}{0.678745in}}{\pgfqpoint{0.979930in}{0.668146in}}{\pgfqpoint{0.987744in}{0.660333in}}%
\pgfpathcurveto{\pgfqpoint{0.995558in}{0.652519in}}{\pgfqpoint{1.006157in}{0.648129in}}{\pgfqpoint{1.017207in}{0.648129in}}%
\pgfpathclose%
\pgfusepath{stroke,fill}%
\end{pgfscope}%
\begin{pgfscope}%
\pgfpathrectangle{\pgfqpoint{0.787074in}{0.548769in}}{\pgfqpoint{5.062926in}{3.102590in}}%
\pgfusepath{clip}%
\pgfsetbuttcap%
\pgfsetroundjoin%
\definecolor{currentfill}{rgb}{1.000000,0.498039,0.054902}%
\pgfsetfillcolor{currentfill}%
\pgfsetlinewidth{1.003750pt}%
\definecolor{currentstroke}{rgb}{1.000000,0.498039,0.054902}%
\pgfsetstrokecolor{currentstroke}%
\pgfsetdash{}{0pt}%
\pgfpathmoveto{\pgfqpoint{1.962959in}{2.514801in}}%
\pgfpathcurveto{\pgfqpoint{1.974009in}{2.514801in}}{\pgfqpoint{1.984608in}{2.519191in}}{\pgfqpoint{1.992422in}{2.527004in}}%
\pgfpathcurveto{\pgfqpoint{2.000235in}{2.534818in}}{\pgfqpoint{2.004626in}{2.545417in}}{\pgfqpoint{2.004626in}{2.556467in}}%
\pgfpathcurveto{\pgfqpoint{2.004626in}{2.567517in}}{\pgfqpoint{2.000235in}{2.578116in}}{\pgfqpoint{1.992422in}{2.585930in}}%
\pgfpathcurveto{\pgfqpoint{1.984608in}{2.593744in}}{\pgfqpoint{1.974009in}{2.598134in}}{\pgfqpoint{1.962959in}{2.598134in}}%
\pgfpathcurveto{\pgfqpoint{1.951909in}{2.598134in}}{\pgfqpoint{1.941310in}{2.593744in}}{\pgfqpoint{1.933496in}{2.585930in}}%
\pgfpathcurveto{\pgfqpoint{1.925683in}{2.578116in}}{\pgfqpoint{1.921292in}{2.567517in}}{\pgfqpoint{1.921292in}{2.556467in}}%
\pgfpathcurveto{\pgfqpoint{1.921292in}{2.545417in}}{\pgfqpoint{1.925683in}{2.534818in}}{\pgfqpoint{1.933496in}{2.527004in}}%
\pgfpathcurveto{\pgfqpoint{1.941310in}{2.519191in}}{\pgfqpoint{1.951909in}{2.514801in}}{\pgfqpoint{1.962959in}{2.514801in}}%
\pgfpathclose%
\pgfusepath{stroke,fill}%
\end{pgfscope}%
\begin{pgfscope}%
\pgfpathrectangle{\pgfqpoint{0.787074in}{0.548769in}}{\pgfqpoint{5.062926in}{3.102590in}}%
\pgfusepath{clip}%
\pgfsetbuttcap%
\pgfsetroundjoin%
\definecolor{currentfill}{rgb}{1.000000,0.498039,0.054902}%
\pgfsetfillcolor{currentfill}%
\pgfsetlinewidth{1.003750pt}%
\definecolor{currentstroke}{rgb}{1.000000,0.498039,0.054902}%
\pgfsetstrokecolor{currentstroke}%
\pgfsetdash{}{0pt}%
\pgfpathmoveto{\pgfqpoint{1.836859in}{2.647988in}}%
\pgfpathcurveto{\pgfqpoint{1.847909in}{2.647988in}}{\pgfqpoint{1.858508in}{2.652378in}}{\pgfqpoint{1.866321in}{2.660191in}}%
\pgfpathcurveto{\pgfqpoint{1.874135in}{2.668005in}}{\pgfqpoint{1.878525in}{2.678604in}}{\pgfqpoint{1.878525in}{2.689654in}}%
\pgfpathcurveto{\pgfqpoint{1.878525in}{2.700704in}}{\pgfqpoint{1.874135in}{2.711303in}}{\pgfqpoint{1.866321in}{2.719117in}}%
\pgfpathcurveto{\pgfqpoint{1.858508in}{2.726931in}}{\pgfqpoint{1.847909in}{2.731321in}}{\pgfqpoint{1.836859in}{2.731321in}}%
\pgfpathcurveto{\pgfqpoint{1.825809in}{2.731321in}}{\pgfqpoint{1.815209in}{2.726931in}}{\pgfqpoint{1.807396in}{2.719117in}}%
\pgfpathcurveto{\pgfqpoint{1.799582in}{2.711303in}}{\pgfqpoint{1.795192in}{2.700704in}}{\pgfqpoint{1.795192in}{2.689654in}}%
\pgfpathcurveto{\pgfqpoint{1.795192in}{2.678604in}}{\pgfqpoint{1.799582in}{2.668005in}}{\pgfqpoint{1.807396in}{2.660191in}}%
\pgfpathcurveto{\pgfqpoint{1.815209in}{2.652378in}}{\pgfqpoint{1.825809in}{2.647988in}}{\pgfqpoint{1.836859in}{2.647988in}}%
\pgfpathclose%
\pgfusepath{stroke,fill}%
\end{pgfscope}%
\begin{pgfscope}%
\pgfpathrectangle{\pgfqpoint{0.787074in}{0.548769in}}{\pgfqpoint{5.062926in}{3.102590in}}%
\pgfusepath{clip}%
\pgfsetbuttcap%
\pgfsetroundjoin%
\definecolor{currentfill}{rgb}{0.121569,0.466667,0.705882}%
\pgfsetfillcolor{currentfill}%
\pgfsetlinewidth{1.003750pt}%
\definecolor{currentstroke}{rgb}{0.121569,0.466667,0.705882}%
\pgfsetstrokecolor{currentstroke}%
\pgfsetdash{}{0pt}%
\pgfpathmoveto{\pgfqpoint{2.089059in}{0.652182in}}%
\pgfpathcurveto{\pgfqpoint{2.100109in}{0.652182in}}{\pgfqpoint{2.110708in}{0.656572in}}{\pgfqpoint{2.118522in}{0.664386in}}%
\pgfpathcurveto{\pgfqpoint{2.126336in}{0.672199in}}{\pgfqpoint{2.130726in}{0.682798in}}{\pgfqpoint{2.130726in}{0.693848in}}%
\pgfpathcurveto{\pgfqpoint{2.130726in}{0.704899in}}{\pgfqpoint{2.126336in}{0.715498in}}{\pgfqpoint{2.118522in}{0.723311in}}%
\pgfpathcurveto{\pgfqpoint{2.110708in}{0.731125in}}{\pgfqpoint{2.100109in}{0.735515in}}{\pgfqpoint{2.089059in}{0.735515in}}%
\pgfpathcurveto{\pgfqpoint{2.078009in}{0.735515in}}{\pgfqpoint{2.067410in}{0.731125in}}{\pgfqpoint{2.059596in}{0.723311in}}%
\pgfpathcurveto{\pgfqpoint{2.051783in}{0.715498in}}{\pgfqpoint{2.047393in}{0.704899in}}{\pgfqpoint{2.047393in}{0.693848in}}%
\pgfpathcurveto{\pgfqpoint{2.047393in}{0.682798in}}{\pgfqpoint{2.051783in}{0.672199in}}{\pgfqpoint{2.059596in}{0.664386in}}%
\pgfpathcurveto{\pgfqpoint{2.067410in}{0.656572in}}{\pgfqpoint{2.078009in}{0.652182in}}{\pgfqpoint{2.089059in}{0.652182in}}%
\pgfpathclose%
\pgfusepath{stroke,fill}%
\end{pgfscope}%
\begin{pgfscope}%
\pgfpathrectangle{\pgfqpoint{0.787074in}{0.548769in}}{\pgfqpoint{5.062926in}{3.102590in}}%
\pgfusepath{clip}%
\pgfsetbuttcap%
\pgfsetroundjoin%
\definecolor{currentfill}{rgb}{0.121569,0.466667,0.705882}%
\pgfsetfillcolor{currentfill}%
\pgfsetlinewidth{1.003750pt}%
\definecolor{currentstroke}{rgb}{0.121569,0.466667,0.705882}%
\pgfsetstrokecolor{currentstroke}%
\pgfsetdash{}{0pt}%
\pgfpathmoveto{\pgfqpoint{1.206357in}{0.648180in}}%
\pgfpathcurveto{\pgfqpoint{1.217407in}{0.648180in}}{\pgfqpoint{1.228006in}{0.652570in}}{\pgfqpoint{1.235820in}{0.660384in}}%
\pgfpathcurveto{\pgfqpoint{1.243634in}{0.668197in}}{\pgfqpoint{1.248024in}{0.678797in}}{\pgfqpoint{1.248024in}{0.689847in}}%
\pgfpathcurveto{\pgfqpoint{1.248024in}{0.700897in}}{\pgfqpoint{1.243634in}{0.711496in}}{\pgfqpoint{1.235820in}{0.719309in}}%
\pgfpathcurveto{\pgfqpoint{1.228006in}{0.727123in}}{\pgfqpoint{1.217407in}{0.731513in}}{\pgfqpoint{1.206357in}{0.731513in}}%
\pgfpathcurveto{\pgfqpoint{1.195307in}{0.731513in}}{\pgfqpoint{1.184708in}{0.727123in}}{\pgfqpoint{1.176894in}{0.719309in}}%
\pgfpathcurveto{\pgfqpoint{1.169081in}{0.711496in}}{\pgfqpoint{1.164691in}{0.700897in}}{\pgfqpoint{1.164691in}{0.689847in}}%
\pgfpathcurveto{\pgfqpoint{1.164691in}{0.678797in}}{\pgfqpoint{1.169081in}{0.668197in}}{\pgfqpoint{1.176894in}{0.660384in}}%
\pgfpathcurveto{\pgfqpoint{1.184708in}{0.652570in}}{\pgfqpoint{1.195307in}{0.648180in}}{\pgfqpoint{1.206357in}{0.648180in}}%
\pgfpathclose%
\pgfusepath{stroke,fill}%
\end{pgfscope}%
\begin{pgfscope}%
\pgfpathrectangle{\pgfqpoint{0.787074in}{0.548769in}}{\pgfqpoint{5.062926in}{3.102590in}}%
\pgfusepath{clip}%
\pgfsetbuttcap%
\pgfsetroundjoin%
\definecolor{currentfill}{rgb}{0.121569,0.466667,0.705882}%
\pgfsetfillcolor{currentfill}%
\pgfsetlinewidth{1.003750pt}%
\definecolor{currentstroke}{rgb}{0.121569,0.466667,0.705882}%
\pgfsetstrokecolor{currentstroke}%
\pgfsetdash{}{0pt}%
\pgfpathmoveto{\pgfqpoint{1.206357in}{0.648153in}}%
\pgfpathcurveto{\pgfqpoint{1.217407in}{0.648153in}}{\pgfqpoint{1.228006in}{0.652543in}}{\pgfqpoint{1.235820in}{0.660356in}}%
\pgfpathcurveto{\pgfqpoint{1.243634in}{0.668170in}}{\pgfqpoint{1.248024in}{0.678769in}}{\pgfqpoint{1.248024in}{0.689819in}}%
\pgfpathcurveto{\pgfqpoint{1.248024in}{0.700869in}}{\pgfqpoint{1.243634in}{0.711468in}}{\pgfqpoint{1.235820in}{0.719282in}}%
\pgfpathcurveto{\pgfqpoint{1.228006in}{0.727096in}}{\pgfqpoint{1.217407in}{0.731486in}}{\pgfqpoint{1.206357in}{0.731486in}}%
\pgfpathcurveto{\pgfqpoint{1.195307in}{0.731486in}}{\pgfqpoint{1.184708in}{0.727096in}}{\pgfqpoint{1.176894in}{0.719282in}}%
\pgfpathcurveto{\pgfqpoint{1.169081in}{0.711468in}}{\pgfqpoint{1.164691in}{0.700869in}}{\pgfqpoint{1.164691in}{0.689819in}}%
\pgfpathcurveto{\pgfqpoint{1.164691in}{0.678769in}}{\pgfqpoint{1.169081in}{0.668170in}}{\pgfqpoint{1.176894in}{0.660356in}}%
\pgfpathcurveto{\pgfqpoint{1.184708in}{0.652543in}}{\pgfqpoint{1.195307in}{0.648153in}}{\pgfqpoint{1.206357in}{0.648153in}}%
\pgfpathclose%
\pgfusepath{stroke,fill}%
\end{pgfscope}%
\begin{pgfscope}%
\pgfpathrectangle{\pgfqpoint{0.787074in}{0.548769in}}{\pgfqpoint{5.062926in}{3.102590in}}%
\pgfusepath{clip}%
\pgfsetbuttcap%
\pgfsetroundjoin%
\definecolor{currentfill}{rgb}{0.121569,0.466667,0.705882}%
\pgfsetfillcolor{currentfill}%
\pgfsetlinewidth{1.003750pt}%
\definecolor{currentstroke}{rgb}{0.121569,0.466667,0.705882}%
\pgfsetstrokecolor{currentstroke}%
\pgfsetdash{}{0pt}%
\pgfpathmoveto{\pgfqpoint{1.647708in}{0.648454in}}%
\pgfpathcurveto{\pgfqpoint{1.658758in}{0.648454in}}{\pgfqpoint{1.669357in}{0.652844in}}{\pgfqpoint{1.677171in}{0.660658in}}%
\pgfpathcurveto{\pgfqpoint{1.684985in}{0.668471in}}{\pgfqpoint{1.689375in}{0.679070in}}{\pgfqpoint{1.689375in}{0.690121in}}%
\pgfpathcurveto{\pgfqpoint{1.689375in}{0.701171in}}{\pgfqpoint{1.684985in}{0.711770in}}{\pgfqpoint{1.677171in}{0.719583in}}%
\pgfpathcurveto{\pgfqpoint{1.669357in}{0.727397in}}{\pgfqpoint{1.658758in}{0.731787in}}{\pgfqpoint{1.647708in}{0.731787in}}%
\pgfpathcurveto{\pgfqpoint{1.636658in}{0.731787in}}{\pgfqpoint{1.626059in}{0.727397in}}{\pgfqpoint{1.618245in}{0.719583in}}%
\pgfpathcurveto{\pgfqpoint{1.610432in}{0.711770in}}{\pgfqpoint{1.606042in}{0.701171in}}{\pgfqpoint{1.606042in}{0.690121in}}%
\pgfpathcurveto{\pgfqpoint{1.606042in}{0.679070in}}{\pgfqpoint{1.610432in}{0.668471in}}{\pgfqpoint{1.618245in}{0.660658in}}%
\pgfpathcurveto{\pgfqpoint{1.626059in}{0.652844in}}{\pgfqpoint{1.636658in}{0.648454in}}{\pgfqpoint{1.647708in}{0.648454in}}%
\pgfpathclose%
\pgfusepath{stroke,fill}%
\end{pgfscope}%
\begin{pgfscope}%
\pgfpathrectangle{\pgfqpoint{0.787074in}{0.548769in}}{\pgfqpoint{5.062926in}{3.102590in}}%
\pgfusepath{clip}%
\pgfsetbuttcap%
\pgfsetroundjoin%
\definecolor{currentfill}{rgb}{0.121569,0.466667,0.705882}%
\pgfsetfillcolor{currentfill}%
\pgfsetlinewidth{1.003750pt}%
\definecolor{currentstroke}{rgb}{0.121569,0.466667,0.705882}%
\pgfsetstrokecolor{currentstroke}%
\pgfsetdash{}{0pt}%
\pgfpathmoveto{\pgfqpoint{1.080257in}{0.648199in}}%
\pgfpathcurveto{\pgfqpoint{1.091307in}{0.648199in}}{\pgfqpoint{1.101906in}{0.652589in}}{\pgfqpoint{1.109720in}{0.660403in}}%
\pgfpathcurveto{\pgfqpoint{1.117533in}{0.668217in}}{\pgfqpoint{1.121924in}{0.678816in}}{\pgfqpoint{1.121924in}{0.689866in}}%
\pgfpathcurveto{\pgfqpoint{1.121924in}{0.700916in}}{\pgfqpoint{1.117533in}{0.711515in}}{\pgfqpoint{1.109720in}{0.719329in}}%
\pgfpathcurveto{\pgfqpoint{1.101906in}{0.727142in}}{\pgfqpoint{1.091307in}{0.731533in}}{\pgfqpoint{1.080257in}{0.731533in}}%
\pgfpathcurveto{\pgfqpoint{1.069207in}{0.731533in}}{\pgfqpoint{1.058608in}{0.727142in}}{\pgfqpoint{1.050794in}{0.719329in}}%
\pgfpathcurveto{\pgfqpoint{1.042981in}{0.711515in}}{\pgfqpoint{1.038590in}{0.700916in}}{\pgfqpoint{1.038590in}{0.689866in}}%
\pgfpathcurveto{\pgfqpoint{1.038590in}{0.678816in}}{\pgfqpoint{1.042981in}{0.668217in}}{\pgfqpoint{1.050794in}{0.660403in}}%
\pgfpathcurveto{\pgfqpoint{1.058608in}{0.652589in}}{\pgfqpoint{1.069207in}{0.648199in}}{\pgfqpoint{1.080257in}{0.648199in}}%
\pgfpathclose%
\pgfusepath{stroke,fill}%
\end{pgfscope}%
\begin{pgfscope}%
\pgfpathrectangle{\pgfqpoint{0.787074in}{0.548769in}}{\pgfqpoint{5.062926in}{3.102590in}}%
\pgfusepath{clip}%
\pgfsetbuttcap%
\pgfsetroundjoin%
\definecolor{currentfill}{rgb}{0.121569,0.466667,0.705882}%
\pgfsetfillcolor{currentfill}%
\pgfsetlinewidth{1.003750pt}%
\definecolor{currentstroke}{rgb}{0.121569,0.466667,0.705882}%
\pgfsetstrokecolor{currentstroke}%
\pgfsetdash{}{0pt}%
\pgfpathmoveto{\pgfqpoint{1.332458in}{0.648149in}}%
\pgfpathcurveto{\pgfqpoint{1.343508in}{0.648149in}}{\pgfqpoint{1.354107in}{0.652540in}}{\pgfqpoint{1.361920in}{0.660353in}}%
\pgfpathcurveto{\pgfqpoint{1.369734in}{0.668167in}}{\pgfqpoint{1.374124in}{0.678766in}}{\pgfqpoint{1.374124in}{0.689816in}}%
\pgfpathcurveto{\pgfqpoint{1.374124in}{0.700866in}}{\pgfqpoint{1.369734in}{0.711465in}}{\pgfqpoint{1.361920in}{0.719279in}}%
\pgfpathcurveto{\pgfqpoint{1.354107in}{0.727092in}}{\pgfqpoint{1.343508in}{0.731483in}}{\pgfqpoint{1.332458in}{0.731483in}}%
\pgfpathcurveto{\pgfqpoint{1.321407in}{0.731483in}}{\pgfqpoint{1.310808in}{0.727092in}}{\pgfqpoint{1.302995in}{0.719279in}}%
\pgfpathcurveto{\pgfqpoint{1.295181in}{0.711465in}}{\pgfqpoint{1.290791in}{0.700866in}}{\pgfqpoint{1.290791in}{0.689816in}}%
\pgfpathcurveto{\pgfqpoint{1.290791in}{0.678766in}}{\pgfqpoint{1.295181in}{0.668167in}}{\pgfqpoint{1.302995in}{0.660353in}}%
\pgfpathcurveto{\pgfqpoint{1.310808in}{0.652540in}}{\pgfqpoint{1.321407in}{0.648149in}}{\pgfqpoint{1.332458in}{0.648149in}}%
\pgfpathclose%
\pgfusepath{stroke,fill}%
\end{pgfscope}%
\begin{pgfscope}%
\pgfpathrectangle{\pgfqpoint{0.787074in}{0.548769in}}{\pgfqpoint{5.062926in}{3.102590in}}%
\pgfusepath{clip}%
\pgfsetbuttcap%
\pgfsetroundjoin%
\definecolor{currentfill}{rgb}{0.121569,0.466667,0.705882}%
\pgfsetfillcolor{currentfill}%
\pgfsetlinewidth{1.003750pt}%
\definecolor{currentstroke}{rgb}{0.121569,0.466667,0.705882}%
\pgfsetstrokecolor{currentstroke}%
\pgfsetdash{}{0pt}%
\pgfpathmoveto{\pgfqpoint{2.530410in}{3.153720in}}%
\pgfpathcurveto{\pgfqpoint{2.541460in}{3.153720in}}{\pgfqpoint{2.552059in}{3.158111in}}{\pgfqpoint{2.559873in}{3.165924in}}%
\pgfpathcurveto{\pgfqpoint{2.567687in}{3.173738in}}{\pgfqpoint{2.572077in}{3.184337in}}{\pgfqpoint{2.572077in}{3.195387in}}%
\pgfpathcurveto{\pgfqpoint{2.572077in}{3.206437in}}{\pgfqpoint{2.567687in}{3.217036in}}{\pgfqpoint{2.559873in}{3.224850in}}%
\pgfpathcurveto{\pgfqpoint{2.552059in}{3.232663in}}{\pgfqpoint{2.541460in}{3.237054in}}{\pgfqpoint{2.530410in}{3.237054in}}%
\pgfpathcurveto{\pgfqpoint{2.519360in}{3.237054in}}{\pgfqpoint{2.508761in}{3.232663in}}{\pgfqpoint{2.500947in}{3.224850in}}%
\pgfpathcurveto{\pgfqpoint{2.493134in}{3.217036in}}{\pgfqpoint{2.488744in}{3.206437in}}{\pgfqpoint{2.488744in}{3.195387in}}%
\pgfpathcurveto{\pgfqpoint{2.488744in}{3.184337in}}{\pgfqpoint{2.493134in}{3.173738in}}{\pgfqpoint{2.500947in}{3.165924in}}%
\pgfpathcurveto{\pgfqpoint{2.508761in}{3.158111in}}{\pgfqpoint{2.519360in}{3.153720in}}{\pgfqpoint{2.530410in}{3.153720in}}%
\pgfpathclose%
\pgfusepath{stroke,fill}%
\end{pgfscope}%
\begin{pgfscope}%
\pgfpathrectangle{\pgfqpoint{0.787074in}{0.548769in}}{\pgfqpoint{5.062926in}{3.102590in}}%
\pgfusepath{clip}%
\pgfsetbuttcap%
\pgfsetroundjoin%
\definecolor{currentfill}{rgb}{1.000000,0.498039,0.054902}%
\pgfsetfillcolor{currentfill}%
\pgfsetlinewidth{1.003750pt}%
\definecolor{currentstroke}{rgb}{1.000000,0.498039,0.054902}%
\pgfsetstrokecolor{currentstroke}%
\pgfsetdash{}{0pt}%
\pgfpathmoveto{\pgfqpoint{2.404310in}{2.857398in}}%
\pgfpathcurveto{\pgfqpoint{2.415360in}{2.857398in}}{\pgfqpoint{2.425959in}{2.861789in}}{\pgfqpoint{2.433773in}{2.869602in}}%
\pgfpathcurveto{\pgfqpoint{2.441586in}{2.877416in}}{\pgfqpoint{2.445977in}{2.888015in}}{\pgfqpoint{2.445977in}{2.899065in}}%
\pgfpathcurveto{\pgfqpoint{2.445977in}{2.910115in}}{\pgfqpoint{2.441586in}{2.920714in}}{\pgfqpoint{2.433773in}{2.928528in}}%
\pgfpathcurveto{\pgfqpoint{2.425959in}{2.936341in}}{\pgfqpoint{2.415360in}{2.940732in}}{\pgfqpoint{2.404310in}{2.940732in}}%
\pgfpathcurveto{\pgfqpoint{2.393260in}{2.940732in}}{\pgfqpoint{2.382661in}{2.936341in}}{\pgfqpoint{2.374847in}{2.928528in}}%
\pgfpathcurveto{\pgfqpoint{2.367033in}{2.920714in}}{\pgfqpoint{2.362643in}{2.910115in}}{\pgfqpoint{2.362643in}{2.899065in}}%
\pgfpathcurveto{\pgfqpoint{2.362643in}{2.888015in}}{\pgfqpoint{2.367033in}{2.877416in}}{\pgfqpoint{2.374847in}{2.869602in}}%
\pgfpathcurveto{\pgfqpoint{2.382661in}{2.861789in}}{\pgfqpoint{2.393260in}{2.857398in}}{\pgfqpoint{2.404310in}{2.857398in}}%
\pgfpathclose%
\pgfusepath{stroke,fill}%
\end{pgfscope}%
\begin{pgfscope}%
\pgfpathrectangle{\pgfqpoint{0.787074in}{0.548769in}}{\pgfqpoint{5.062926in}{3.102590in}}%
\pgfusepath{clip}%
\pgfsetbuttcap%
\pgfsetroundjoin%
\definecolor{currentfill}{rgb}{1.000000,0.498039,0.054902}%
\pgfsetfillcolor{currentfill}%
\pgfsetlinewidth{1.003750pt}%
\definecolor{currentstroke}{rgb}{1.000000,0.498039,0.054902}%
\pgfsetstrokecolor{currentstroke}%
\pgfsetdash{}{0pt}%
\pgfpathmoveto{\pgfqpoint{2.656510in}{3.104611in}}%
\pgfpathcurveto{\pgfqpoint{2.667561in}{3.104611in}}{\pgfqpoint{2.678160in}{3.109002in}}{\pgfqpoint{2.685973in}{3.116815in}}%
\pgfpathcurveto{\pgfqpoint{2.693787in}{3.124629in}}{\pgfqpoint{2.698177in}{3.135228in}}{\pgfqpoint{2.698177in}{3.146278in}}%
\pgfpathcurveto{\pgfqpoint{2.698177in}{3.157328in}}{\pgfqpoint{2.693787in}{3.167927in}}{\pgfqpoint{2.685973in}{3.175741in}}%
\pgfpathcurveto{\pgfqpoint{2.678160in}{3.183554in}}{\pgfqpoint{2.667561in}{3.187945in}}{\pgfqpoint{2.656510in}{3.187945in}}%
\pgfpathcurveto{\pgfqpoint{2.645460in}{3.187945in}}{\pgfqpoint{2.634861in}{3.183554in}}{\pgfqpoint{2.627048in}{3.175741in}}%
\pgfpathcurveto{\pgfqpoint{2.619234in}{3.167927in}}{\pgfqpoint{2.614844in}{3.157328in}}{\pgfqpoint{2.614844in}{3.146278in}}%
\pgfpathcurveto{\pgfqpoint{2.614844in}{3.135228in}}{\pgfqpoint{2.619234in}{3.124629in}}{\pgfqpoint{2.627048in}{3.116815in}}%
\pgfpathcurveto{\pgfqpoint{2.634861in}{3.109002in}}{\pgfqpoint{2.645460in}{3.104611in}}{\pgfqpoint{2.656510in}{3.104611in}}%
\pgfpathclose%
\pgfusepath{stroke,fill}%
\end{pgfscope}%
\begin{pgfscope}%
\pgfpathrectangle{\pgfqpoint{0.787074in}{0.548769in}}{\pgfqpoint{5.062926in}{3.102590in}}%
\pgfusepath{clip}%
\pgfsetbuttcap%
\pgfsetroundjoin%
\definecolor{currentfill}{rgb}{0.121569,0.466667,0.705882}%
\pgfsetfillcolor{currentfill}%
\pgfsetlinewidth{1.003750pt}%
\definecolor{currentstroke}{rgb}{0.121569,0.466667,0.705882}%
\pgfsetstrokecolor{currentstroke}%
\pgfsetdash{}{0pt}%
\pgfpathmoveto{\pgfqpoint{1.080257in}{0.648133in}}%
\pgfpathcurveto{\pgfqpoint{1.091307in}{0.648133in}}{\pgfqpoint{1.101906in}{0.652523in}}{\pgfqpoint{1.109720in}{0.660337in}}%
\pgfpathcurveto{\pgfqpoint{1.117533in}{0.668150in}}{\pgfqpoint{1.121924in}{0.678749in}}{\pgfqpoint{1.121924in}{0.689799in}}%
\pgfpathcurveto{\pgfqpoint{1.121924in}{0.700849in}}{\pgfqpoint{1.117533in}{0.711448in}}{\pgfqpoint{1.109720in}{0.719262in}}%
\pgfpathcurveto{\pgfqpoint{1.101906in}{0.727076in}}{\pgfqpoint{1.091307in}{0.731466in}}{\pgfqpoint{1.080257in}{0.731466in}}%
\pgfpathcurveto{\pgfqpoint{1.069207in}{0.731466in}}{\pgfqpoint{1.058608in}{0.727076in}}{\pgfqpoint{1.050794in}{0.719262in}}%
\pgfpathcurveto{\pgfqpoint{1.042981in}{0.711448in}}{\pgfqpoint{1.038590in}{0.700849in}}{\pgfqpoint{1.038590in}{0.689799in}}%
\pgfpathcurveto{\pgfqpoint{1.038590in}{0.678749in}}{\pgfqpoint{1.042981in}{0.668150in}}{\pgfqpoint{1.050794in}{0.660337in}}%
\pgfpathcurveto{\pgfqpoint{1.058608in}{0.652523in}}{\pgfqpoint{1.069207in}{0.648133in}}{\pgfqpoint{1.080257in}{0.648133in}}%
\pgfpathclose%
\pgfusepath{stroke,fill}%
\end{pgfscope}%
\begin{pgfscope}%
\pgfpathrectangle{\pgfqpoint{0.787074in}{0.548769in}}{\pgfqpoint{5.062926in}{3.102590in}}%
\pgfusepath{clip}%
\pgfsetbuttcap%
\pgfsetroundjoin%
\definecolor{currentfill}{rgb}{0.121569,0.466667,0.705882}%
\pgfsetfillcolor{currentfill}%
\pgfsetlinewidth{1.003750pt}%
\definecolor{currentstroke}{rgb}{0.121569,0.466667,0.705882}%
\pgfsetstrokecolor{currentstroke}%
\pgfsetdash{}{0pt}%
\pgfpathmoveto{\pgfqpoint{1.143307in}{0.679648in}}%
\pgfpathcurveto{\pgfqpoint{1.154357in}{0.679648in}}{\pgfqpoint{1.164956in}{0.684039in}}{\pgfqpoint{1.172770in}{0.691852in}}%
\pgfpathcurveto{\pgfqpoint{1.180584in}{0.699666in}}{\pgfqpoint{1.184974in}{0.710265in}}{\pgfqpoint{1.184974in}{0.721315in}}%
\pgfpathcurveto{\pgfqpoint{1.184974in}{0.732365in}}{\pgfqpoint{1.180584in}{0.742964in}}{\pgfqpoint{1.172770in}{0.750778in}}%
\pgfpathcurveto{\pgfqpoint{1.164956in}{0.758591in}}{\pgfqpoint{1.154357in}{0.762982in}}{\pgfqpoint{1.143307in}{0.762982in}}%
\pgfpathcurveto{\pgfqpoint{1.132257in}{0.762982in}}{\pgfqpoint{1.121658in}{0.758591in}}{\pgfqpoint{1.113844in}{0.750778in}}%
\pgfpathcurveto{\pgfqpoint{1.106031in}{0.742964in}}{\pgfqpoint{1.101640in}{0.732365in}}{\pgfqpoint{1.101640in}{0.721315in}}%
\pgfpathcurveto{\pgfqpoint{1.101640in}{0.710265in}}{\pgfqpoint{1.106031in}{0.699666in}}{\pgfqpoint{1.113844in}{0.691852in}}%
\pgfpathcurveto{\pgfqpoint{1.121658in}{0.684039in}}{\pgfqpoint{1.132257in}{0.679648in}}{\pgfqpoint{1.143307in}{0.679648in}}%
\pgfpathclose%
\pgfusepath{stroke,fill}%
\end{pgfscope}%
\begin{pgfscope}%
\pgfpathrectangle{\pgfqpoint{0.787074in}{0.548769in}}{\pgfqpoint{5.062926in}{3.102590in}}%
\pgfusepath{clip}%
\pgfsetbuttcap%
\pgfsetroundjoin%
\definecolor{currentfill}{rgb}{1.000000,0.498039,0.054902}%
\pgfsetfillcolor{currentfill}%
\pgfsetlinewidth{1.003750pt}%
\definecolor{currentstroke}{rgb}{1.000000,0.498039,0.054902}%
\pgfsetstrokecolor{currentstroke}%
\pgfsetdash{}{0pt}%
\pgfpathmoveto{\pgfqpoint{2.467360in}{2.890450in}}%
\pgfpathcurveto{\pgfqpoint{2.478410in}{2.890450in}}{\pgfqpoint{2.489009in}{2.894841in}}{\pgfqpoint{2.496823in}{2.902654in}}%
\pgfpathcurveto{\pgfqpoint{2.504636in}{2.910468in}}{\pgfqpoint{2.509027in}{2.921067in}}{\pgfqpoint{2.509027in}{2.932117in}}%
\pgfpathcurveto{\pgfqpoint{2.509027in}{2.943167in}}{\pgfqpoint{2.504636in}{2.953766in}}{\pgfqpoint{2.496823in}{2.961580in}}%
\pgfpathcurveto{\pgfqpoint{2.489009in}{2.969393in}}{\pgfqpoint{2.478410in}{2.973784in}}{\pgfqpoint{2.467360in}{2.973784in}}%
\pgfpathcurveto{\pgfqpoint{2.456310in}{2.973784in}}{\pgfqpoint{2.445711in}{2.969393in}}{\pgfqpoint{2.437897in}{2.961580in}}%
\pgfpathcurveto{\pgfqpoint{2.430084in}{2.953766in}}{\pgfqpoint{2.425693in}{2.943167in}}{\pgfqpoint{2.425693in}{2.932117in}}%
\pgfpathcurveto{\pgfqpoint{2.425693in}{2.921067in}}{\pgfqpoint{2.430084in}{2.910468in}}{\pgfqpoint{2.437897in}{2.902654in}}%
\pgfpathcurveto{\pgfqpoint{2.445711in}{2.894841in}}{\pgfqpoint{2.456310in}{2.890450in}}{\pgfqpoint{2.467360in}{2.890450in}}%
\pgfpathclose%
\pgfusepath{stroke,fill}%
\end{pgfscope}%
\begin{pgfscope}%
\pgfpathrectangle{\pgfqpoint{0.787074in}{0.548769in}}{\pgfqpoint{5.062926in}{3.102590in}}%
\pgfusepath{clip}%
\pgfsetbuttcap%
\pgfsetroundjoin%
\definecolor{currentfill}{rgb}{0.121569,0.466667,0.705882}%
\pgfsetfillcolor{currentfill}%
\pgfsetlinewidth{1.003750pt}%
\definecolor{currentstroke}{rgb}{0.121569,0.466667,0.705882}%
\pgfsetstrokecolor{currentstroke}%
\pgfsetdash{}{0pt}%
\pgfpathmoveto{\pgfqpoint{1.395508in}{0.786731in}}%
\pgfpathcurveto{\pgfqpoint{1.406558in}{0.786731in}}{\pgfqpoint{1.417157in}{0.791121in}}{\pgfqpoint{1.424970in}{0.798934in}}%
\pgfpathcurveto{\pgfqpoint{1.432784in}{0.806748in}}{\pgfqpoint{1.437174in}{0.817347in}}{\pgfqpoint{1.437174in}{0.828397in}}%
\pgfpathcurveto{\pgfqpoint{1.437174in}{0.839447in}}{\pgfqpoint{1.432784in}{0.850046in}}{\pgfqpoint{1.424970in}{0.857860in}}%
\pgfpathcurveto{\pgfqpoint{1.417157in}{0.865674in}}{\pgfqpoint{1.406558in}{0.870064in}}{\pgfqpoint{1.395508in}{0.870064in}}%
\pgfpathcurveto{\pgfqpoint{1.384458in}{0.870064in}}{\pgfqpoint{1.373859in}{0.865674in}}{\pgfqpoint{1.366045in}{0.857860in}}%
\pgfpathcurveto{\pgfqpoint{1.358231in}{0.850046in}}{\pgfqpoint{1.353841in}{0.839447in}}{\pgfqpoint{1.353841in}{0.828397in}}%
\pgfpathcurveto{\pgfqpoint{1.353841in}{0.817347in}}{\pgfqpoint{1.358231in}{0.806748in}}{\pgfqpoint{1.366045in}{0.798934in}}%
\pgfpathcurveto{\pgfqpoint{1.373859in}{0.791121in}}{\pgfqpoint{1.384458in}{0.786731in}}{\pgfqpoint{1.395508in}{0.786731in}}%
\pgfpathclose%
\pgfusepath{stroke,fill}%
\end{pgfscope}%
\begin{pgfscope}%
\pgfpathrectangle{\pgfqpoint{0.787074in}{0.548769in}}{\pgfqpoint{5.062926in}{3.102590in}}%
\pgfusepath{clip}%
\pgfsetbuttcap%
\pgfsetroundjoin%
\definecolor{currentfill}{rgb}{0.839216,0.152941,0.156863}%
\pgfsetfillcolor{currentfill}%
\pgfsetlinewidth{1.003750pt}%
\definecolor{currentstroke}{rgb}{0.839216,0.152941,0.156863}%
\pgfsetstrokecolor{currentstroke}%
\pgfsetdash{}{0pt}%
\pgfpathmoveto{\pgfqpoint{1.836859in}{2.746144in}}%
\pgfpathcurveto{\pgfqpoint{1.847909in}{2.746144in}}{\pgfqpoint{1.858508in}{2.750534in}}{\pgfqpoint{1.866321in}{2.758348in}}%
\pgfpathcurveto{\pgfqpoint{1.874135in}{2.766161in}}{\pgfqpoint{1.878525in}{2.776760in}}{\pgfqpoint{1.878525in}{2.787810in}}%
\pgfpathcurveto{\pgfqpoint{1.878525in}{2.798860in}}{\pgfqpoint{1.874135in}{2.809460in}}{\pgfqpoint{1.866321in}{2.817273in}}%
\pgfpathcurveto{\pgfqpoint{1.858508in}{2.825087in}}{\pgfqpoint{1.847909in}{2.829477in}}{\pgfqpoint{1.836859in}{2.829477in}}%
\pgfpathcurveto{\pgfqpoint{1.825809in}{2.829477in}}{\pgfqpoint{1.815209in}{2.825087in}}{\pgfqpoint{1.807396in}{2.817273in}}%
\pgfpathcurveto{\pgfqpoint{1.799582in}{2.809460in}}{\pgfqpoint{1.795192in}{2.798860in}}{\pgfqpoint{1.795192in}{2.787810in}}%
\pgfpathcurveto{\pgfqpoint{1.795192in}{2.776760in}}{\pgfqpoint{1.799582in}{2.766161in}}{\pgfqpoint{1.807396in}{2.758348in}}%
\pgfpathcurveto{\pgfqpoint{1.815209in}{2.750534in}}{\pgfqpoint{1.825809in}{2.746144in}}{\pgfqpoint{1.836859in}{2.746144in}}%
\pgfpathclose%
\pgfusepath{stroke,fill}%
\end{pgfscope}%
\begin{pgfscope}%
\pgfpathrectangle{\pgfqpoint{0.787074in}{0.548769in}}{\pgfqpoint{5.062926in}{3.102590in}}%
\pgfusepath{clip}%
\pgfsetbuttcap%
\pgfsetroundjoin%
\definecolor{currentfill}{rgb}{0.121569,0.466667,0.705882}%
\pgfsetfillcolor{currentfill}%
\pgfsetlinewidth{1.003750pt}%
\definecolor{currentstroke}{rgb}{0.121569,0.466667,0.705882}%
\pgfsetstrokecolor{currentstroke}%
\pgfsetdash{}{0pt}%
\pgfpathmoveto{\pgfqpoint{1.269407in}{0.649982in}}%
\pgfpathcurveto{\pgfqpoint{1.280458in}{0.649982in}}{\pgfqpoint{1.291057in}{0.654372in}}{\pgfqpoint{1.298870in}{0.662186in}}%
\pgfpathcurveto{\pgfqpoint{1.306684in}{0.669999in}}{\pgfqpoint{1.311074in}{0.680598in}}{\pgfqpoint{1.311074in}{0.691648in}}%
\pgfpathcurveto{\pgfqpoint{1.311074in}{0.702699in}}{\pgfqpoint{1.306684in}{0.713298in}}{\pgfqpoint{1.298870in}{0.721111in}}%
\pgfpathcurveto{\pgfqpoint{1.291057in}{0.728925in}}{\pgfqpoint{1.280458in}{0.733315in}}{\pgfqpoint{1.269407in}{0.733315in}}%
\pgfpathcurveto{\pgfqpoint{1.258357in}{0.733315in}}{\pgfqpoint{1.247758in}{0.728925in}}{\pgfqpoint{1.239945in}{0.721111in}}%
\pgfpathcurveto{\pgfqpoint{1.232131in}{0.713298in}}{\pgfqpoint{1.227741in}{0.702699in}}{\pgfqpoint{1.227741in}{0.691648in}}%
\pgfpathcurveto{\pgfqpoint{1.227741in}{0.680598in}}{\pgfqpoint{1.232131in}{0.669999in}}{\pgfqpoint{1.239945in}{0.662186in}}%
\pgfpathcurveto{\pgfqpoint{1.247758in}{0.654372in}}{\pgfqpoint{1.258357in}{0.649982in}}{\pgfqpoint{1.269407in}{0.649982in}}%
\pgfpathclose%
\pgfusepath{stroke,fill}%
\end{pgfscope}%
\begin{pgfscope}%
\pgfpathrectangle{\pgfqpoint{0.787074in}{0.548769in}}{\pgfqpoint{5.062926in}{3.102590in}}%
\pgfusepath{clip}%
\pgfsetbuttcap%
\pgfsetroundjoin%
\definecolor{currentfill}{rgb}{1.000000,0.498039,0.054902}%
\pgfsetfillcolor{currentfill}%
\pgfsetlinewidth{1.003750pt}%
\definecolor{currentstroke}{rgb}{1.000000,0.498039,0.054902}%
\pgfsetstrokecolor{currentstroke}%
\pgfsetdash{}{0pt}%
\pgfpathmoveto{\pgfqpoint{2.971761in}{2.345681in}}%
\pgfpathcurveto{\pgfqpoint{2.982811in}{2.345681in}}{\pgfqpoint{2.993410in}{2.350071in}}{\pgfqpoint{3.001224in}{2.357885in}}%
\pgfpathcurveto{\pgfqpoint{3.009038in}{2.365698in}}{\pgfqpoint{3.013428in}{2.376297in}}{\pgfqpoint{3.013428in}{2.387348in}}%
\pgfpathcurveto{\pgfqpoint{3.013428in}{2.398398in}}{\pgfqpoint{3.009038in}{2.408997in}}{\pgfqpoint{3.001224in}{2.416810in}}%
\pgfpathcurveto{\pgfqpoint{2.993410in}{2.424624in}}{\pgfqpoint{2.982811in}{2.429014in}}{\pgfqpoint{2.971761in}{2.429014in}}%
\pgfpathcurveto{\pgfqpoint{2.960711in}{2.429014in}}{\pgfqpoint{2.950112in}{2.424624in}}{\pgfqpoint{2.942298in}{2.416810in}}%
\pgfpathcurveto{\pgfqpoint{2.934485in}{2.408997in}}{\pgfqpoint{2.930094in}{2.398398in}}{\pgfqpoint{2.930094in}{2.387348in}}%
\pgfpathcurveto{\pgfqpoint{2.930094in}{2.376297in}}{\pgfqpoint{2.934485in}{2.365698in}}{\pgfqpoint{2.942298in}{2.357885in}}%
\pgfpathcurveto{\pgfqpoint{2.950112in}{2.350071in}}{\pgfqpoint{2.960711in}{2.345681in}}{\pgfqpoint{2.971761in}{2.345681in}}%
\pgfpathclose%
\pgfusepath{stroke,fill}%
\end{pgfscope}%
\begin{pgfscope}%
\pgfpathrectangle{\pgfqpoint{0.787074in}{0.548769in}}{\pgfqpoint{5.062926in}{3.102590in}}%
\pgfusepath{clip}%
\pgfsetbuttcap%
\pgfsetroundjoin%
\definecolor{currentfill}{rgb}{0.121569,0.466667,0.705882}%
\pgfsetfillcolor{currentfill}%
\pgfsetlinewidth{1.003750pt}%
\definecolor{currentstroke}{rgb}{0.121569,0.466667,0.705882}%
\pgfsetstrokecolor{currentstroke}%
\pgfsetdash{}{0pt}%
\pgfpathmoveto{\pgfqpoint{1.080257in}{0.651543in}}%
\pgfpathcurveto{\pgfqpoint{1.091307in}{0.651543in}}{\pgfqpoint{1.101906in}{0.655933in}}{\pgfqpoint{1.109720in}{0.663747in}}%
\pgfpathcurveto{\pgfqpoint{1.117533in}{0.671560in}}{\pgfqpoint{1.121924in}{0.682159in}}{\pgfqpoint{1.121924in}{0.693209in}}%
\pgfpathcurveto{\pgfqpoint{1.121924in}{0.704259in}}{\pgfqpoint{1.117533in}{0.714859in}}{\pgfqpoint{1.109720in}{0.722672in}}%
\pgfpathcurveto{\pgfqpoint{1.101906in}{0.730486in}}{\pgfqpoint{1.091307in}{0.734876in}}{\pgfqpoint{1.080257in}{0.734876in}}%
\pgfpathcurveto{\pgfqpoint{1.069207in}{0.734876in}}{\pgfqpoint{1.058608in}{0.730486in}}{\pgfqpoint{1.050794in}{0.722672in}}%
\pgfpathcurveto{\pgfqpoint{1.042981in}{0.714859in}}{\pgfqpoint{1.038590in}{0.704259in}}{\pgfqpoint{1.038590in}{0.693209in}}%
\pgfpathcurveto{\pgfqpoint{1.038590in}{0.682159in}}{\pgfqpoint{1.042981in}{0.671560in}}{\pgfqpoint{1.050794in}{0.663747in}}%
\pgfpathcurveto{\pgfqpoint{1.058608in}{0.655933in}}{\pgfqpoint{1.069207in}{0.651543in}}{\pgfqpoint{1.080257in}{0.651543in}}%
\pgfpathclose%
\pgfusepath{stroke,fill}%
\end{pgfscope}%
\begin{pgfscope}%
\pgfpathrectangle{\pgfqpoint{0.787074in}{0.548769in}}{\pgfqpoint{5.062926in}{3.102590in}}%
\pgfusepath{clip}%
\pgfsetbuttcap%
\pgfsetroundjoin%
\definecolor{currentfill}{rgb}{1.000000,0.498039,0.054902}%
\pgfsetfillcolor{currentfill}%
\pgfsetlinewidth{1.003750pt}%
\definecolor{currentstroke}{rgb}{1.000000,0.498039,0.054902}%
\pgfsetstrokecolor{currentstroke}%
\pgfsetdash{}{0pt}%
\pgfpathmoveto{\pgfqpoint{1.269407in}{2.708956in}}%
\pgfpathcurveto{\pgfqpoint{1.280458in}{2.708956in}}{\pgfqpoint{1.291057in}{2.713346in}}{\pgfqpoint{1.298870in}{2.721160in}}%
\pgfpathcurveto{\pgfqpoint{1.306684in}{2.728974in}}{\pgfqpoint{1.311074in}{2.739573in}}{\pgfqpoint{1.311074in}{2.750623in}}%
\pgfpathcurveto{\pgfqpoint{1.311074in}{2.761673in}}{\pgfqpoint{1.306684in}{2.772272in}}{\pgfqpoint{1.298870in}{2.780086in}}%
\pgfpathcurveto{\pgfqpoint{1.291057in}{2.787899in}}{\pgfqpoint{1.280458in}{2.792289in}}{\pgfqpoint{1.269407in}{2.792289in}}%
\pgfpathcurveto{\pgfqpoint{1.258357in}{2.792289in}}{\pgfqpoint{1.247758in}{2.787899in}}{\pgfqpoint{1.239945in}{2.780086in}}%
\pgfpathcurveto{\pgfqpoint{1.232131in}{2.772272in}}{\pgfqpoint{1.227741in}{2.761673in}}{\pgfqpoint{1.227741in}{2.750623in}}%
\pgfpathcurveto{\pgfqpoint{1.227741in}{2.739573in}}{\pgfqpoint{1.232131in}{2.728974in}}{\pgfqpoint{1.239945in}{2.721160in}}%
\pgfpathcurveto{\pgfqpoint{1.247758in}{2.713346in}}{\pgfqpoint{1.258357in}{2.708956in}}{\pgfqpoint{1.269407in}{2.708956in}}%
\pgfpathclose%
\pgfusepath{stroke,fill}%
\end{pgfscope}%
\begin{pgfscope}%
\pgfpathrectangle{\pgfqpoint{0.787074in}{0.548769in}}{\pgfqpoint{5.062926in}{3.102590in}}%
\pgfusepath{clip}%
\pgfsetbuttcap%
\pgfsetroundjoin%
\definecolor{currentfill}{rgb}{0.121569,0.466667,0.705882}%
\pgfsetfillcolor{currentfill}%
\pgfsetlinewidth{1.003750pt}%
\definecolor{currentstroke}{rgb}{0.121569,0.466667,0.705882}%
\pgfsetstrokecolor{currentstroke}%
\pgfsetdash{}{0pt}%
\pgfpathmoveto{\pgfqpoint{1.080257in}{0.648158in}}%
\pgfpathcurveto{\pgfqpoint{1.091307in}{0.648158in}}{\pgfqpoint{1.101906in}{0.652548in}}{\pgfqpoint{1.109720in}{0.660362in}}%
\pgfpathcurveto{\pgfqpoint{1.117533in}{0.668176in}}{\pgfqpoint{1.121924in}{0.678775in}}{\pgfqpoint{1.121924in}{0.689825in}}%
\pgfpathcurveto{\pgfqpoint{1.121924in}{0.700875in}}{\pgfqpoint{1.117533in}{0.711474in}}{\pgfqpoint{1.109720in}{0.719287in}}%
\pgfpathcurveto{\pgfqpoint{1.101906in}{0.727101in}}{\pgfqpoint{1.091307in}{0.731491in}}{\pgfqpoint{1.080257in}{0.731491in}}%
\pgfpathcurveto{\pgfqpoint{1.069207in}{0.731491in}}{\pgfqpoint{1.058608in}{0.727101in}}{\pgfqpoint{1.050794in}{0.719287in}}%
\pgfpathcurveto{\pgfqpoint{1.042981in}{0.711474in}}{\pgfqpoint{1.038590in}{0.700875in}}{\pgfqpoint{1.038590in}{0.689825in}}%
\pgfpathcurveto{\pgfqpoint{1.038590in}{0.678775in}}{\pgfqpoint{1.042981in}{0.668176in}}{\pgfqpoint{1.050794in}{0.660362in}}%
\pgfpathcurveto{\pgfqpoint{1.058608in}{0.652548in}}{\pgfqpoint{1.069207in}{0.648158in}}{\pgfqpoint{1.080257in}{0.648158in}}%
\pgfpathclose%
\pgfusepath{stroke,fill}%
\end{pgfscope}%
\begin{pgfscope}%
\pgfpathrectangle{\pgfqpoint{0.787074in}{0.548769in}}{\pgfqpoint{5.062926in}{3.102590in}}%
\pgfusepath{clip}%
\pgfsetbuttcap%
\pgfsetroundjoin%
\definecolor{currentfill}{rgb}{1.000000,0.498039,0.054902}%
\pgfsetfillcolor{currentfill}%
\pgfsetlinewidth{1.003750pt}%
\definecolor{currentstroke}{rgb}{1.000000,0.498039,0.054902}%
\pgfsetstrokecolor{currentstroke}%
\pgfsetdash{}{0pt}%
\pgfpathmoveto{\pgfqpoint{1.332458in}{2.868283in}}%
\pgfpathcurveto{\pgfqpoint{1.343508in}{2.868283in}}{\pgfqpoint{1.354107in}{2.872673in}}{\pgfqpoint{1.361920in}{2.880486in}}%
\pgfpathcurveto{\pgfqpoint{1.369734in}{2.888300in}}{\pgfqpoint{1.374124in}{2.898899in}}{\pgfqpoint{1.374124in}{2.909949in}}%
\pgfpathcurveto{\pgfqpoint{1.374124in}{2.920999in}}{\pgfqpoint{1.369734in}{2.931598in}}{\pgfqpoint{1.361920in}{2.939412in}}%
\pgfpathcurveto{\pgfqpoint{1.354107in}{2.947226in}}{\pgfqpoint{1.343508in}{2.951616in}}{\pgfqpoint{1.332458in}{2.951616in}}%
\pgfpathcurveto{\pgfqpoint{1.321407in}{2.951616in}}{\pgfqpoint{1.310808in}{2.947226in}}{\pgfqpoint{1.302995in}{2.939412in}}%
\pgfpathcurveto{\pgfqpoint{1.295181in}{2.931598in}}{\pgfqpoint{1.290791in}{2.920999in}}{\pgfqpoint{1.290791in}{2.909949in}}%
\pgfpathcurveto{\pgfqpoint{1.290791in}{2.898899in}}{\pgfqpoint{1.295181in}{2.888300in}}{\pgfqpoint{1.302995in}{2.880486in}}%
\pgfpathcurveto{\pgfqpoint{1.310808in}{2.872673in}}{\pgfqpoint{1.321407in}{2.868283in}}{\pgfqpoint{1.332458in}{2.868283in}}%
\pgfpathclose%
\pgfusepath{stroke,fill}%
\end{pgfscope}%
\begin{pgfscope}%
\pgfpathrectangle{\pgfqpoint{0.787074in}{0.548769in}}{\pgfqpoint{5.062926in}{3.102590in}}%
\pgfusepath{clip}%
\pgfsetbuttcap%
\pgfsetroundjoin%
\definecolor{currentfill}{rgb}{0.121569,0.466667,0.705882}%
\pgfsetfillcolor{currentfill}%
\pgfsetlinewidth{1.003750pt}%
\definecolor{currentstroke}{rgb}{0.121569,0.466667,0.705882}%
\pgfsetstrokecolor{currentstroke}%
\pgfsetdash{}{0pt}%
\pgfpathmoveto{\pgfqpoint{1.080257in}{0.648132in}}%
\pgfpathcurveto{\pgfqpoint{1.091307in}{0.648132in}}{\pgfqpoint{1.101906in}{0.652522in}}{\pgfqpoint{1.109720in}{0.660336in}}%
\pgfpathcurveto{\pgfqpoint{1.117533in}{0.668149in}}{\pgfqpoint{1.121924in}{0.678748in}}{\pgfqpoint{1.121924in}{0.689799in}}%
\pgfpathcurveto{\pgfqpoint{1.121924in}{0.700849in}}{\pgfqpoint{1.117533in}{0.711448in}}{\pgfqpoint{1.109720in}{0.719261in}}%
\pgfpathcurveto{\pgfqpoint{1.101906in}{0.727075in}}{\pgfqpoint{1.091307in}{0.731465in}}{\pgfqpoint{1.080257in}{0.731465in}}%
\pgfpathcurveto{\pgfqpoint{1.069207in}{0.731465in}}{\pgfqpoint{1.058608in}{0.727075in}}{\pgfqpoint{1.050794in}{0.719261in}}%
\pgfpathcurveto{\pgfqpoint{1.042981in}{0.711448in}}{\pgfqpoint{1.038590in}{0.700849in}}{\pgfqpoint{1.038590in}{0.689799in}}%
\pgfpathcurveto{\pgfqpoint{1.038590in}{0.678748in}}{\pgfqpoint{1.042981in}{0.668149in}}{\pgfqpoint{1.050794in}{0.660336in}}%
\pgfpathcurveto{\pgfqpoint{1.058608in}{0.652522in}}{\pgfqpoint{1.069207in}{0.648132in}}{\pgfqpoint{1.080257in}{0.648132in}}%
\pgfpathclose%
\pgfusepath{stroke,fill}%
\end{pgfscope}%
\begin{pgfscope}%
\pgfpathrectangle{\pgfqpoint{0.787074in}{0.548769in}}{\pgfqpoint{5.062926in}{3.102590in}}%
\pgfusepath{clip}%
\pgfsetbuttcap%
\pgfsetroundjoin%
\definecolor{currentfill}{rgb}{1.000000,0.498039,0.054902}%
\pgfsetfillcolor{currentfill}%
\pgfsetlinewidth{1.003750pt}%
\definecolor{currentstroke}{rgb}{1.000000,0.498039,0.054902}%
\pgfsetstrokecolor{currentstroke}%
\pgfsetdash{}{0pt}%
\pgfpathmoveto{\pgfqpoint{1.521608in}{2.305856in}}%
\pgfpathcurveto{\pgfqpoint{1.532658in}{2.305856in}}{\pgfqpoint{1.543257in}{2.310246in}}{\pgfqpoint{1.551071in}{2.318060in}}%
\pgfpathcurveto{\pgfqpoint{1.558884in}{2.325873in}}{\pgfqpoint{1.563275in}{2.336472in}}{\pgfqpoint{1.563275in}{2.347523in}}%
\pgfpathcurveto{\pgfqpoint{1.563275in}{2.358573in}}{\pgfqpoint{1.558884in}{2.369172in}}{\pgfqpoint{1.551071in}{2.376985in}}%
\pgfpathcurveto{\pgfqpoint{1.543257in}{2.384799in}}{\pgfqpoint{1.532658in}{2.389189in}}{\pgfqpoint{1.521608in}{2.389189in}}%
\pgfpathcurveto{\pgfqpoint{1.510558in}{2.389189in}}{\pgfqpoint{1.499959in}{2.384799in}}{\pgfqpoint{1.492145in}{2.376985in}}%
\pgfpathcurveto{\pgfqpoint{1.484332in}{2.369172in}}{\pgfqpoint{1.479941in}{2.358573in}}{\pgfqpoint{1.479941in}{2.347523in}}%
\pgfpathcurveto{\pgfqpoint{1.479941in}{2.336472in}}{\pgfqpoint{1.484332in}{2.325873in}}{\pgfqpoint{1.492145in}{2.318060in}}%
\pgfpathcurveto{\pgfqpoint{1.499959in}{2.310246in}}{\pgfqpoint{1.510558in}{2.305856in}}{\pgfqpoint{1.521608in}{2.305856in}}%
\pgfpathclose%
\pgfusepath{stroke,fill}%
\end{pgfscope}%
\begin{pgfscope}%
\pgfpathrectangle{\pgfqpoint{0.787074in}{0.548769in}}{\pgfqpoint{5.062926in}{3.102590in}}%
\pgfusepath{clip}%
\pgfsetbuttcap%
\pgfsetroundjoin%
\definecolor{currentfill}{rgb}{1.000000,0.498039,0.054902}%
\pgfsetfillcolor{currentfill}%
\pgfsetlinewidth{1.003750pt}%
\definecolor{currentstroke}{rgb}{1.000000,0.498039,0.054902}%
\pgfsetstrokecolor{currentstroke}%
\pgfsetdash{}{0pt}%
\pgfpathmoveto{\pgfqpoint{1.143307in}{2.492823in}}%
\pgfpathcurveto{\pgfqpoint{1.154357in}{2.492823in}}{\pgfqpoint{1.164956in}{2.497213in}}{\pgfqpoint{1.172770in}{2.505027in}}%
\pgfpathcurveto{\pgfqpoint{1.180584in}{2.512840in}}{\pgfqpoint{1.184974in}{2.523439in}}{\pgfqpoint{1.184974in}{2.534490in}}%
\pgfpathcurveto{\pgfqpoint{1.184974in}{2.545540in}}{\pgfqpoint{1.180584in}{2.556139in}}{\pgfqpoint{1.172770in}{2.563952in}}%
\pgfpathcurveto{\pgfqpoint{1.164956in}{2.571766in}}{\pgfqpoint{1.154357in}{2.576156in}}{\pgfqpoint{1.143307in}{2.576156in}}%
\pgfpathcurveto{\pgfqpoint{1.132257in}{2.576156in}}{\pgfqpoint{1.121658in}{2.571766in}}{\pgfqpoint{1.113844in}{2.563952in}}%
\pgfpathcurveto{\pgfqpoint{1.106031in}{2.556139in}}{\pgfqpoint{1.101640in}{2.545540in}}{\pgfqpoint{1.101640in}{2.534490in}}%
\pgfpathcurveto{\pgfqpoint{1.101640in}{2.523439in}}{\pgfqpoint{1.106031in}{2.512840in}}{\pgfqpoint{1.113844in}{2.505027in}}%
\pgfpathcurveto{\pgfqpoint{1.121658in}{2.497213in}}{\pgfqpoint{1.132257in}{2.492823in}}{\pgfqpoint{1.143307in}{2.492823in}}%
\pgfpathclose%
\pgfusepath{stroke,fill}%
\end{pgfscope}%
\begin{pgfscope}%
\pgfpathrectangle{\pgfqpoint{0.787074in}{0.548769in}}{\pgfqpoint{5.062926in}{3.102590in}}%
\pgfusepath{clip}%
\pgfsetbuttcap%
\pgfsetroundjoin%
\definecolor{currentfill}{rgb}{0.121569,0.466667,0.705882}%
\pgfsetfillcolor{currentfill}%
\pgfsetlinewidth{1.003750pt}%
\definecolor{currentstroke}{rgb}{0.121569,0.466667,0.705882}%
\pgfsetstrokecolor{currentstroke}%
\pgfsetdash{}{0pt}%
\pgfpathmoveto{\pgfqpoint{1.332458in}{0.648147in}}%
\pgfpathcurveto{\pgfqpoint{1.343508in}{0.648147in}}{\pgfqpoint{1.354107in}{0.652537in}}{\pgfqpoint{1.361920in}{0.660351in}}%
\pgfpathcurveto{\pgfqpoint{1.369734in}{0.668165in}}{\pgfqpoint{1.374124in}{0.678764in}}{\pgfqpoint{1.374124in}{0.689814in}}%
\pgfpathcurveto{\pgfqpoint{1.374124in}{0.700864in}}{\pgfqpoint{1.369734in}{0.711463in}}{\pgfqpoint{1.361920in}{0.719276in}}%
\pgfpathcurveto{\pgfqpoint{1.354107in}{0.727090in}}{\pgfqpoint{1.343508in}{0.731480in}}{\pgfqpoint{1.332458in}{0.731480in}}%
\pgfpathcurveto{\pgfqpoint{1.321407in}{0.731480in}}{\pgfqpoint{1.310808in}{0.727090in}}{\pgfqpoint{1.302995in}{0.719276in}}%
\pgfpathcurveto{\pgfqpoint{1.295181in}{0.711463in}}{\pgfqpoint{1.290791in}{0.700864in}}{\pgfqpoint{1.290791in}{0.689814in}}%
\pgfpathcurveto{\pgfqpoint{1.290791in}{0.678764in}}{\pgfqpoint{1.295181in}{0.668165in}}{\pgfqpoint{1.302995in}{0.660351in}}%
\pgfpathcurveto{\pgfqpoint{1.310808in}{0.652537in}}{\pgfqpoint{1.321407in}{0.648147in}}{\pgfqpoint{1.332458in}{0.648147in}}%
\pgfpathclose%
\pgfusepath{stroke,fill}%
\end{pgfscope}%
\begin{pgfscope}%
\pgfpathrectangle{\pgfqpoint{0.787074in}{0.548769in}}{\pgfqpoint{5.062926in}{3.102590in}}%
\pgfusepath{clip}%
\pgfsetbuttcap%
\pgfsetroundjoin%
\definecolor{currentfill}{rgb}{1.000000,0.498039,0.054902}%
\pgfsetfillcolor{currentfill}%
\pgfsetlinewidth{1.003750pt}%
\definecolor{currentstroke}{rgb}{1.000000,0.498039,0.054902}%
\pgfsetstrokecolor{currentstroke}%
\pgfsetdash{}{0pt}%
\pgfpathmoveto{\pgfqpoint{2.152109in}{2.754884in}}%
\pgfpathcurveto{\pgfqpoint{2.163159in}{2.754884in}}{\pgfqpoint{2.173759in}{2.759274in}}{\pgfqpoint{2.181572in}{2.767088in}}%
\pgfpathcurveto{\pgfqpoint{2.189386in}{2.774902in}}{\pgfqpoint{2.193776in}{2.785501in}}{\pgfqpoint{2.193776in}{2.796551in}}%
\pgfpathcurveto{\pgfqpoint{2.193776in}{2.807601in}}{\pgfqpoint{2.189386in}{2.818200in}}{\pgfqpoint{2.181572in}{2.826014in}}%
\pgfpathcurveto{\pgfqpoint{2.173759in}{2.833827in}}{\pgfqpoint{2.163159in}{2.838218in}}{\pgfqpoint{2.152109in}{2.838218in}}%
\pgfpathcurveto{\pgfqpoint{2.141059in}{2.838218in}}{\pgfqpoint{2.130460in}{2.833827in}}{\pgfqpoint{2.122647in}{2.826014in}}%
\pgfpathcurveto{\pgfqpoint{2.114833in}{2.818200in}}{\pgfqpoint{2.110443in}{2.807601in}}{\pgfqpoint{2.110443in}{2.796551in}}%
\pgfpathcurveto{\pgfqpoint{2.110443in}{2.785501in}}{\pgfqpoint{2.114833in}{2.774902in}}{\pgfqpoint{2.122647in}{2.767088in}}%
\pgfpathcurveto{\pgfqpoint{2.130460in}{2.759274in}}{\pgfqpoint{2.141059in}{2.754884in}}{\pgfqpoint{2.152109in}{2.754884in}}%
\pgfpathclose%
\pgfusepath{stroke,fill}%
\end{pgfscope}%
\begin{pgfscope}%
\pgfpathrectangle{\pgfqpoint{0.787074in}{0.548769in}}{\pgfqpoint{5.062926in}{3.102590in}}%
\pgfusepath{clip}%
\pgfsetbuttcap%
\pgfsetroundjoin%
\definecolor{currentfill}{rgb}{1.000000,0.498039,0.054902}%
\pgfsetfillcolor{currentfill}%
\pgfsetlinewidth{1.003750pt}%
\definecolor{currentstroke}{rgb}{1.000000,0.498039,0.054902}%
\pgfsetstrokecolor{currentstroke}%
\pgfsetdash{}{0pt}%
\pgfpathmoveto{\pgfqpoint{2.026009in}{2.596250in}}%
\pgfpathcurveto{\pgfqpoint{2.037059in}{2.596250in}}{\pgfqpoint{2.047658in}{2.600640in}}{\pgfqpoint{2.055472in}{2.608454in}}%
\pgfpathcurveto{\pgfqpoint{2.063285in}{2.616267in}}{\pgfqpoint{2.067676in}{2.626867in}}{\pgfqpoint{2.067676in}{2.637917in}}%
\pgfpathcurveto{\pgfqpoint{2.067676in}{2.648967in}}{\pgfqpoint{2.063285in}{2.659566in}}{\pgfqpoint{2.055472in}{2.667379in}}%
\pgfpathcurveto{\pgfqpoint{2.047658in}{2.675193in}}{\pgfqpoint{2.037059in}{2.679583in}}{\pgfqpoint{2.026009in}{2.679583in}}%
\pgfpathcurveto{\pgfqpoint{2.014959in}{2.679583in}}{\pgfqpoint{2.004360in}{2.675193in}}{\pgfqpoint{1.996546in}{2.667379in}}%
\pgfpathcurveto{\pgfqpoint{1.988733in}{2.659566in}}{\pgfqpoint{1.984342in}{2.648967in}}{\pgfqpoint{1.984342in}{2.637917in}}%
\pgfpathcurveto{\pgfqpoint{1.984342in}{2.626867in}}{\pgfqpoint{1.988733in}{2.616267in}}{\pgfqpoint{1.996546in}{2.608454in}}%
\pgfpathcurveto{\pgfqpoint{2.004360in}{2.600640in}}{\pgfqpoint{2.014959in}{2.596250in}}{\pgfqpoint{2.026009in}{2.596250in}}%
\pgfpathclose%
\pgfusepath{stroke,fill}%
\end{pgfscope}%
\begin{pgfscope}%
\pgfpathrectangle{\pgfqpoint{0.787074in}{0.548769in}}{\pgfqpoint{5.062926in}{3.102590in}}%
\pgfusepath{clip}%
\pgfsetbuttcap%
\pgfsetroundjoin%
\definecolor{currentfill}{rgb}{0.121569,0.466667,0.705882}%
\pgfsetfillcolor{currentfill}%
\pgfsetlinewidth{1.003750pt}%
\definecolor{currentstroke}{rgb}{0.121569,0.466667,0.705882}%
\pgfsetstrokecolor{currentstroke}%
\pgfsetdash{}{0pt}%
\pgfpathmoveto{\pgfqpoint{1.962959in}{0.659080in}}%
\pgfpathcurveto{\pgfqpoint{1.974009in}{0.659080in}}{\pgfqpoint{1.984608in}{0.663470in}}{\pgfqpoint{1.992422in}{0.671284in}}%
\pgfpathcurveto{\pgfqpoint{2.000235in}{0.679098in}}{\pgfqpoint{2.004626in}{0.689697in}}{\pgfqpoint{2.004626in}{0.700747in}}%
\pgfpathcurveto{\pgfqpoint{2.004626in}{0.711797in}}{\pgfqpoint{2.000235in}{0.722396in}}{\pgfqpoint{1.992422in}{0.730210in}}%
\pgfpathcurveto{\pgfqpoint{1.984608in}{0.738023in}}{\pgfqpoint{1.974009in}{0.742413in}}{\pgfqpoint{1.962959in}{0.742413in}}%
\pgfpathcurveto{\pgfqpoint{1.951909in}{0.742413in}}{\pgfqpoint{1.941310in}{0.738023in}}{\pgfqpoint{1.933496in}{0.730210in}}%
\pgfpathcurveto{\pgfqpoint{1.925683in}{0.722396in}}{\pgfqpoint{1.921292in}{0.711797in}}{\pgfqpoint{1.921292in}{0.700747in}}%
\pgfpathcurveto{\pgfqpoint{1.921292in}{0.689697in}}{\pgfqpoint{1.925683in}{0.679098in}}{\pgfqpoint{1.933496in}{0.671284in}}%
\pgfpathcurveto{\pgfqpoint{1.941310in}{0.663470in}}{\pgfqpoint{1.951909in}{0.659080in}}{\pgfqpoint{1.962959in}{0.659080in}}%
\pgfpathclose%
\pgfusepath{stroke,fill}%
\end{pgfscope}%
\begin{pgfscope}%
\pgfpathrectangle{\pgfqpoint{0.787074in}{0.548769in}}{\pgfqpoint{5.062926in}{3.102590in}}%
\pgfusepath{clip}%
\pgfsetbuttcap%
\pgfsetroundjoin%
\definecolor{currentfill}{rgb}{1.000000,0.498039,0.054902}%
\pgfsetfillcolor{currentfill}%
\pgfsetlinewidth{1.003750pt}%
\definecolor{currentstroke}{rgb}{1.000000,0.498039,0.054902}%
\pgfsetstrokecolor{currentstroke}%
\pgfsetdash{}{0pt}%
\pgfpathmoveto{\pgfqpoint{2.404310in}{2.981796in}}%
\pgfpathcurveto{\pgfqpoint{2.415360in}{2.981796in}}{\pgfqpoint{2.425959in}{2.986186in}}{\pgfqpoint{2.433773in}{2.994000in}}%
\pgfpathcurveto{\pgfqpoint{2.441586in}{3.001813in}}{\pgfqpoint{2.445977in}{3.012412in}}{\pgfqpoint{2.445977in}{3.023462in}}%
\pgfpathcurveto{\pgfqpoint{2.445977in}{3.034512in}}{\pgfqpoint{2.441586in}{3.045112in}}{\pgfqpoint{2.433773in}{3.052925in}}%
\pgfpathcurveto{\pgfqpoint{2.425959in}{3.060739in}}{\pgfqpoint{2.415360in}{3.065129in}}{\pgfqpoint{2.404310in}{3.065129in}}%
\pgfpathcurveto{\pgfqpoint{2.393260in}{3.065129in}}{\pgfqpoint{2.382661in}{3.060739in}}{\pgfqpoint{2.374847in}{3.052925in}}%
\pgfpathcurveto{\pgfqpoint{2.367033in}{3.045112in}}{\pgfqpoint{2.362643in}{3.034512in}}{\pgfqpoint{2.362643in}{3.023462in}}%
\pgfpathcurveto{\pgfqpoint{2.362643in}{3.012412in}}{\pgfqpoint{2.367033in}{3.001813in}}{\pgfqpoint{2.374847in}{2.994000in}}%
\pgfpathcurveto{\pgfqpoint{2.382661in}{2.986186in}}{\pgfqpoint{2.393260in}{2.981796in}}{\pgfqpoint{2.404310in}{2.981796in}}%
\pgfpathclose%
\pgfusepath{stroke,fill}%
\end{pgfscope}%
\begin{pgfscope}%
\pgfpathrectangle{\pgfqpoint{0.787074in}{0.548769in}}{\pgfqpoint{5.062926in}{3.102590in}}%
\pgfusepath{clip}%
\pgfsetbuttcap%
\pgfsetroundjoin%
\definecolor{currentfill}{rgb}{1.000000,0.498039,0.054902}%
\pgfsetfillcolor{currentfill}%
\pgfsetlinewidth{1.003750pt}%
\definecolor{currentstroke}{rgb}{1.000000,0.498039,0.054902}%
\pgfsetstrokecolor{currentstroke}%
\pgfsetdash{}{0pt}%
\pgfpathmoveto{\pgfqpoint{1.899909in}{2.923368in}}%
\pgfpathcurveto{\pgfqpoint{1.910959in}{2.923368in}}{\pgfqpoint{1.921558in}{2.927758in}}{\pgfqpoint{1.929372in}{2.935572in}}%
\pgfpathcurveto{\pgfqpoint{1.937185in}{2.943385in}}{\pgfqpoint{1.941575in}{2.953984in}}{\pgfqpoint{1.941575in}{2.965034in}}%
\pgfpathcurveto{\pgfqpoint{1.941575in}{2.976084in}}{\pgfqpoint{1.937185in}{2.986683in}}{\pgfqpoint{1.929372in}{2.994497in}}%
\pgfpathcurveto{\pgfqpoint{1.921558in}{3.002311in}}{\pgfqpoint{1.910959in}{3.006701in}}{\pgfqpoint{1.899909in}{3.006701in}}%
\pgfpathcurveto{\pgfqpoint{1.888859in}{3.006701in}}{\pgfqpoint{1.878260in}{3.002311in}}{\pgfqpoint{1.870446in}{2.994497in}}%
\pgfpathcurveto{\pgfqpoint{1.862632in}{2.986683in}}{\pgfqpoint{1.858242in}{2.976084in}}{\pgfqpoint{1.858242in}{2.965034in}}%
\pgfpathcurveto{\pgfqpoint{1.858242in}{2.953984in}}{\pgfqpoint{1.862632in}{2.943385in}}{\pgfqpoint{1.870446in}{2.935572in}}%
\pgfpathcurveto{\pgfqpoint{1.878260in}{2.927758in}}{\pgfqpoint{1.888859in}{2.923368in}}{\pgfqpoint{1.899909in}{2.923368in}}%
\pgfpathclose%
\pgfusepath{stroke,fill}%
\end{pgfscope}%
\begin{pgfscope}%
\pgfpathrectangle{\pgfqpoint{0.787074in}{0.548769in}}{\pgfqpoint{5.062926in}{3.102590in}}%
\pgfusepath{clip}%
\pgfsetbuttcap%
\pgfsetroundjoin%
\definecolor{currentfill}{rgb}{1.000000,0.498039,0.054902}%
\pgfsetfillcolor{currentfill}%
\pgfsetlinewidth{1.003750pt}%
\definecolor{currentstroke}{rgb}{1.000000,0.498039,0.054902}%
\pgfsetstrokecolor{currentstroke}%
\pgfsetdash{}{0pt}%
\pgfpathmoveto{\pgfqpoint{1.773809in}{2.279400in}}%
\pgfpathcurveto{\pgfqpoint{1.784859in}{2.279400in}}{\pgfqpoint{1.795458in}{2.283790in}}{\pgfqpoint{1.803271in}{2.291603in}}%
\pgfpathcurveto{\pgfqpoint{1.811085in}{2.299417in}}{\pgfqpoint{1.815475in}{2.310016in}}{\pgfqpoint{1.815475in}{2.321066in}}%
\pgfpathcurveto{\pgfqpoint{1.815475in}{2.332116in}}{\pgfqpoint{1.811085in}{2.342715in}}{\pgfqpoint{1.803271in}{2.350529in}}%
\pgfpathcurveto{\pgfqpoint{1.795458in}{2.358343in}}{\pgfqpoint{1.784859in}{2.362733in}}{\pgfqpoint{1.773809in}{2.362733in}}%
\pgfpathcurveto{\pgfqpoint{1.762758in}{2.362733in}}{\pgfqpoint{1.752159in}{2.358343in}}{\pgfqpoint{1.744346in}{2.350529in}}%
\pgfpathcurveto{\pgfqpoint{1.736532in}{2.342715in}}{\pgfqpoint{1.732142in}{2.332116in}}{\pgfqpoint{1.732142in}{2.321066in}}%
\pgfpathcurveto{\pgfqpoint{1.732142in}{2.310016in}}{\pgfqpoint{1.736532in}{2.299417in}}{\pgfqpoint{1.744346in}{2.291603in}}%
\pgfpathcurveto{\pgfqpoint{1.752159in}{2.283790in}}{\pgfqpoint{1.762758in}{2.279400in}}{\pgfqpoint{1.773809in}{2.279400in}}%
\pgfpathclose%
\pgfusepath{stroke,fill}%
\end{pgfscope}%
\begin{pgfscope}%
\pgfpathrectangle{\pgfqpoint{0.787074in}{0.548769in}}{\pgfqpoint{5.062926in}{3.102590in}}%
\pgfusepath{clip}%
\pgfsetbuttcap%
\pgfsetroundjoin%
\definecolor{currentfill}{rgb}{1.000000,0.498039,0.054902}%
\pgfsetfillcolor{currentfill}%
\pgfsetlinewidth{1.003750pt}%
\definecolor{currentstroke}{rgb}{1.000000,0.498039,0.054902}%
\pgfsetstrokecolor{currentstroke}%
\pgfsetdash{}{0pt}%
\pgfpathmoveto{\pgfqpoint{1.710758in}{2.414739in}}%
\pgfpathcurveto{\pgfqpoint{1.721809in}{2.414739in}}{\pgfqpoint{1.732408in}{2.419129in}}{\pgfqpoint{1.740221in}{2.426942in}}%
\pgfpathcurveto{\pgfqpoint{1.748035in}{2.434756in}}{\pgfqpoint{1.752425in}{2.445355in}}{\pgfqpoint{1.752425in}{2.456405in}}%
\pgfpathcurveto{\pgfqpoint{1.752425in}{2.467455in}}{\pgfqpoint{1.748035in}{2.478054in}}{\pgfqpoint{1.740221in}{2.485868in}}%
\pgfpathcurveto{\pgfqpoint{1.732408in}{2.493682in}}{\pgfqpoint{1.721809in}{2.498072in}}{\pgfqpoint{1.710758in}{2.498072in}}%
\pgfpathcurveto{\pgfqpoint{1.699708in}{2.498072in}}{\pgfqpoint{1.689109in}{2.493682in}}{\pgfqpoint{1.681296in}{2.485868in}}%
\pgfpathcurveto{\pgfqpoint{1.673482in}{2.478054in}}{\pgfqpoint{1.669092in}{2.467455in}}{\pgfqpoint{1.669092in}{2.456405in}}%
\pgfpathcurveto{\pgfqpoint{1.669092in}{2.445355in}}{\pgfqpoint{1.673482in}{2.434756in}}{\pgfqpoint{1.681296in}{2.426942in}}%
\pgfpathcurveto{\pgfqpoint{1.689109in}{2.419129in}}{\pgfqpoint{1.699708in}{2.414739in}}{\pgfqpoint{1.710758in}{2.414739in}}%
\pgfpathclose%
\pgfusepath{stroke,fill}%
\end{pgfscope}%
\begin{pgfscope}%
\pgfpathrectangle{\pgfqpoint{0.787074in}{0.548769in}}{\pgfqpoint{5.062926in}{3.102590in}}%
\pgfusepath{clip}%
\pgfsetbuttcap%
\pgfsetroundjoin%
\definecolor{currentfill}{rgb}{1.000000,0.498039,0.054902}%
\pgfsetfillcolor{currentfill}%
\pgfsetlinewidth{1.003750pt}%
\definecolor{currentstroke}{rgb}{1.000000,0.498039,0.054902}%
\pgfsetstrokecolor{currentstroke}%
\pgfsetdash{}{0pt}%
\pgfpathmoveto{\pgfqpoint{1.647708in}{3.155405in}}%
\pgfpathcurveto{\pgfqpoint{1.658758in}{3.155405in}}{\pgfqpoint{1.669357in}{3.159795in}}{\pgfqpoint{1.677171in}{3.167609in}}%
\pgfpathcurveto{\pgfqpoint{1.684985in}{3.175422in}}{\pgfqpoint{1.689375in}{3.186021in}}{\pgfqpoint{1.689375in}{3.197072in}}%
\pgfpathcurveto{\pgfqpoint{1.689375in}{3.208122in}}{\pgfqpoint{1.684985in}{3.218721in}}{\pgfqpoint{1.677171in}{3.226534in}}%
\pgfpathcurveto{\pgfqpoint{1.669357in}{3.234348in}}{\pgfqpoint{1.658758in}{3.238738in}}{\pgfqpoint{1.647708in}{3.238738in}}%
\pgfpathcurveto{\pgfqpoint{1.636658in}{3.238738in}}{\pgfqpoint{1.626059in}{3.234348in}}{\pgfqpoint{1.618245in}{3.226534in}}%
\pgfpathcurveto{\pgfqpoint{1.610432in}{3.218721in}}{\pgfqpoint{1.606042in}{3.208122in}}{\pgfqpoint{1.606042in}{3.197072in}}%
\pgfpathcurveto{\pgfqpoint{1.606042in}{3.186021in}}{\pgfqpoint{1.610432in}{3.175422in}}{\pgfqpoint{1.618245in}{3.167609in}}%
\pgfpathcurveto{\pgfqpoint{1.626059in}{3.159795in}}{\pgfqpoint{1.636658in}{3.155405in}}{\pgfqpoint{1.647708in}{3.155405in}}%
\pgfpathclose%
\pgfusepath{stroke,fill}%
\end{pgfscope}%
\begin{pgfscope}%
\pgfpathrectangle{\pgfqpoint{0.787074in}{0.548769in}}{\pgfqpoint{5.062926in}{3.102590in}}%
\pgfusepath{clip}%
\pgfsetbuttcap%
\pgfsetroundjoin%
\definecolor{currentfill}{rgb}{1.000000,0.498039,0.054902}%
\pgfsetfillcolor{currentfill}%
\pgfsetlinewidth{1.003750pt}%
\definecolor{currentstroke}{rgb}{1.000000,0.498039,0.054902}%
\pgfsetstrokecolor{currentstroke}%
\pgfsetdash{}{0pt}%
\pgfpathmoveto{\pgfqpoint{1.710758in}{2.889472in}}%
\pgfpathcurveto{\pgfqpoint{1.721809in}{2.889472in}}{\pgfqpoint{1.732408in}{2.893862in}}{\pgfqpoint{1.740221in}{2.901676in}}%
\pgfpathcurveto{\pgfqpoint{1.748035in}{2.909489in}}{\pgfqpoint{1.752425in}{2.920088in}}{\pgfqpoint{1.752425in}{2.931138in}}%
\pgfpathcurveto{\pgfqpoint{1.752425in}{2.942189in}}{\pgfqpoint{1.748035in}{2.952788in}}{\pgfqpoint{1.740221in}{2.960601in}}%
\pgfpathcurveto{\pgfqpoint{1.732408in}{2.968415in}}{\pgfqpoint{1.721809in}{2.972805in}}{\pgfqpoint{1.710758in}{2.972805in}}%
\pgfpathcurveto{\pgfqpoint{1.699708in}{2.972805in}}{\pgfqpoint{1.689109in}{2.968415in}}{\pgfqpoint{1.681296in}{2.960601in}}%
\pgfpathcurveto{\pgfqpoint{1.673482in}{2.952788in}}{\pgfqpoint{1.669092in}{2.942189in}}{\pgfqpoint{1.669092in}{2.931138in}}%
\pgfpathcurveto{\pgfqpoint{1.669092in}{2.920088in}}{\pgfqpoint{1.673482in}{2.909489in}}{\pgfqpoint{1.681296in}{2.901676in}}%
\pgfpathcurveto{\pgfqpoint{1.689109in}{2.893862in}}{\pgfqpoint{1.699708in}{2.889472in}}{\pgfqpoint{1.710758in}{2.889472in}}%
\pgfpathclose%
\pgfusepath{stroke,fill}%
\end{pgfscope}%
\begin{pgfscope}%
\pgfpathrectangle{\pgfqpoint{0.787074in}{0.548769in}}{\pgfqpoint{5.062926in}{3.102590in}}%
\pgfusepath{clip}%
\pgfsetbuttcap%
\pgfsetroundjoin%
\definecolor{currentfill}{rgb}{0.121569,0.466667,0.705882}%
\pgfsetfillcolor{currentfill}%
\pgfsetlinewidth{1.003750pt}%
\definecolor{currentstroke}{rgb}{0.121569,0.466667,0.705882}%
\pgfsetstrokecolor{currentstroke}%
\pgfsetdash{}{0pt}%
\pgfpathmoveto{\pgfqpoint{1.332458in}{0.648143in}}%
\pgfpathcurveto{\pgfqpoint{1.343508in}{0.648143in}}{\pgfqpoint{1.354107in}{0.652534in}}{\pgfqpoint{1.361920in}{0.660347in}}%
\pgfpathcurveto{\pgfqpoint{1.369734in}{0.668161in}}{\pgfqpoint{1.374124in}{0.678760in}}{\pgfqpoint{1.374124in}{0.689810in}}%
\pgfpathcurveto{\pgfqpoint{1.374124in}{0.700860in}}{\pgfqpoint{1.369734in}{0.711459in}}{\pgfqpoint{1.361920in}{0.719273in}}%
\pgfpathcurveto{\pgfqpoint{1.354107in}{0.727086in}}{\pgfqpoint{1.343508in}{0.731477in}}{\pgfqpoint{1.332458in}{0.731477in}}%
\pgfpathcurveto{\pgfqpoint{1.321407in}{0.731477in}}{\pgfqpoint{1.310808in}{0.727086in}}{\pgfqpoint{1.302995in}{0.719273in}}%
\pgfpathcurveto{\pgfqpoint{1.295181in}{0.711459in}}{\pgfqpoint{1.290791in}{0.700860in}}{\pgfqpoint{1.290791in}{0.689810in}}%
\pgfpathcurveto{\pgfqpoint{1.290791in}{0.678760in}}{\pgfqpoint{1.295181in}{0.668161in}}{\pgfqpoint{1.302995in}{0.660347in}}%
\pgfpathcurveto{\pgfqpoint{1.310808in}{0.652534in}}{\pgfqpoint{1.321407in}{0.648143in}}{\pgfqpoint{1.332458in}{0.648143in}}%
\pgfpathclose%
\pgfusepath{stroke,fill}%
\end{pgfscope}%
\begin{pgfscope}%
\pgfpathrectangle{\pgfqpoint{0.787074in}{0.548769in}}{\pgfqpoint{5.062926in}{3.102590in}}%
\pgfusepath{clip}%
\pgfsetbuttcap%
\pgfsetroundjoin%
\definecolor{currentfill}{rgb}{0.121569,0.466667,0.705882}%
\pgfsetfillcolor{currentfill}%
\pgfsetlinewidth{1.003750pt}%
\definecolor{currentstroke}{rgb}{0.121569,0.466667,0.705882}%
\pgfsetstrokecolor{currentstroke}%
\pgfsetdash{}{0pt}%
\pgfpathmoveto{\pgfqpoint{1.080257in}{0.648148in}}%
\pgfpathcurveto{\pgfqpoint{1.091307in}{0.648148in}}{\pgfqpoint{1.101906in}{0.652538in}}{\pgfqpoint{1.109720in}{0.660352in}}%
\pgfpathcurveto{\pgfqpoint{1.117533in}{0.668165in}}{\pgfqpoint{1.121924in}{0.678764in}}{\pgfqpoint{1.121924in}{0.689814in}}%
\pgfpathcurveto{\pgfqpoint{1.121924in}{0.700865in}}{\pgfqpoint{1.117533in}{0.711464in}}{\pgfqpoint{1.109720in}{0.719277in}}%
\pgfpathcurveto{\pgfqpoint{1.101906in}{0.727091in}}{\pgfqpoint{1.091307in}{0.731481in}}{\pgfqpoint{1.080257in}{0.731481in}}%
\pgfpathcurveto{\pgfqpoint{1.069207in}{0.731481in}}{\pgfqpoint{1.058608in}{0.727091in}}{\pgfqpoint{1.050794in}{0.719277in}}%
\pgfpathcurveto{\pgfqpoint{1.042981in}{0.711464in}}{\pgfqpoint{1.038590in}{0.700865in}}{\pgfqpoint{1.038590in}{0.689814in}}%
\pgfpathcurveto{\pgfqpoint{1.038590in}{0.678764in}}{\pgfqpoint{1.042981in}{0.668165in}}{\pgfqpoint{1.050794in}{0.660352in}}%
\pgfpathcurveto{\pgfqpoint{1.058608in}{0.652538in}}{\pgfqpoint{1.069207in}{0.648148in}}{\pgfqpoint{1.080257in}{0.648148in}}%
\pgfpathclose%
\pgfusepath{stroke,fill}%
\end{pgfscope}%
\begin{pgfscope}%
\pgfpathrectangle{\pgfqpoint{0.787074in}{0.548769in}}{\pgfqpoint{5.062926in}{3.102590in}}%
\pgfusepath{clip}%
\pgfsetbuttcap%
\pgfsetroundjoin%
\definecolor{currentfill}{rgb}{0.121569,0.466667,0.705882}%
\pgfsetfillcolor{currentfill}%
\pgfsetlinewidth{1.003750pt}%
\definecolor{currentstroke}{rgb}{0.121569,0.466667,0.705882}%
\pgfsetstrokecolor{currentstroke}%
\pgfsetdash{}{0pt}%
\pgfpathmoveto{\pgfqpoint{1.458558in}{0.648151in}}%
\pgfpathcurveto{\pgfqpoint{1.469608in}{0.648151in}}{\pgfqpoint{1.480207in}{0.652542in}}{\pgfqpoint{1.488021in}{0.660355in}}%
\pgfpathcurveto{\pgfqpoint{1.495834in}{0.668169in}}{\pgfqpoint{1.500224in}{0.678768in}}{\pgfqpoint{1.500224in}{0.689818in}}%
\pgfpathcurveto{\pgfqpoint{1.500224in}{0.700868in}}{\pgfqpoint{1.495834in}{0.711467in}}{\pgfqpoint{1.488021in}{0.719281in}}%
\pgfpathcurveto{\pgfqpoint{1.480207in}{0.727094in}}{\pgfqpoint{1.469608in}{0.731485in}}{\pgfqpoint{1.458558in}{0.731485in}}%
\pgfpathcurveto{\pgfqpoint{1.447508in}{0.731485in}}{\pgfqpoint{1.436909in}{0.727094in}}{\pgfqpoint{1.429095in}{0.719281in}}%
\pgfpathcurveto{\pgfqpoint{1.421281in}{0.711467in}}{\pgfqpoint{1.416891in}{0.700868in}}{\pgfqpoint{1.416891in}{0.689818in}}%
\pgfpathcurveto{\pgfqpoint{1.416891in}{0.678768in}}{\pgfqpoint{1.421281in}{0.668169in}}{\pgfqpoint{1.429095in}{0.660355in}}%
\pgfpathcurveto{\pgfqpoint{1.436909in}{0.652542in}}{\pgfqpoint{1.447508in}{0.648151in}}{\pgfqpoint{1.458558in}{0.648151in}}%
\pgfpathclose%
\pgfusepath{stroke,fill}%
\end{pgfscope}%
\begin{pgfscope}%
\pgfpathrectangle{\pgfqpoint{0.787074in}{0.548769in}}{\pgfqpoint{5.062926in}{3.102590in}}%
\pgfusepath{clip}%
\pgfsetbuttcap%
\pgfsetroundjoin%
\definecolor{currentfill}{rgb}{1.000000,0.498039,0.054902}%
\pgfsetfillcolor{currentfill}%
\pgfsetlinewidth{1.003750pt}%
\definecolor{currentstroke}{rgb}{1.000000,0.498039,0.054902}%
\pgfsetstrokecolor{currentstroke}%
\pgfsetdash{}{0pt}%
\pgfpathmoveto{\pgfqpoint{1.269407in}{2.770182in}}%
\pgfpathcurveto{\pgfqpoint{1.280458in}{2.770182in}}{\pgfqpoint{1.291057in}{2.774572in}}{\pgfqpoint{1.298870in}{2.782386in}}%
\pgfpathcurveto{\pgfqpoint{1.306684in}{2.790199in}}{\pgfqpoint{1.311074in}{2.800798in}}{\pgfqpoint{1.311074in}{2.811848in}}%
\pgfpathcurveto{\pgfqpoint{1.311074in}{2.822898in}}{\pgfqpoint{1.306684in}{2.833497in}}{\pgfqpoint{1.298870in}{2.841311in}}%
\pgfpathcurveto{\pgfqpoint{1.291057in}{2.849125in}}{\pgfqpoint{1.280458in}{2.853515in}}{\pgfqpoint{1.269407in}{2.853515in}}%
\pgfpathcurveto{\pgfqpoint{1.258357in}{2.853515in}}{\pgfqpoint{1.247758in}{2.849125in}}{\pgfqpoint{1.239945in}{2.841311in}}%
\pgfpathcurveto{\pgfqpoint{1.232131in}{2.833497in}}{\pgfqpoint{1.227741in}{2.822898in}}{\pgfqpoint{1.227741in}{2.811848in}}%
\pgfpathcurveto{\pgfqpoint{1.227741in}{2.800798in}}{\pgfqpoint{1.232131in}{2.790199in}}{\pgfqpoint{1.239945in}{2.782386in}}%
\pgfpathcurveto{\pgfqpoint{1.247758in}{2.774572in}}{\pgfqpoint{1.258357in}{2.770182in}}{\pgfqpoint{1.269407in}{2.770182in}}%
\pgfpathclose%
\pgfusepath{stroke,fill}%
\end{pgfscope}%
\begin{pgfscope}%
\pgfpathrectangle{\pgfqpoint{0.787074in}{0.548769in}}{\pgfqpoint{5.062926in}{3.102590in}}%
\pgfusepath{clip}%
\pgfsetbuttcap%
\pgfsetroundjoin%
\definecolor{currentfill}{rgb}{1.000000,0.498039,0.054902}%
\pgfsetfillcolor{currentfill}%
\pgfsetlinewidth{1.003750pt}%
\definecolor{currentstroke}{rgb}{1.000000,0.498039,0.054902}%
\pgfsetstrokecolor{currentstroke}%
\pgfsetdash{}{0pt}%
\pgfpathmoveto{\pgfqpoint{1.962959in}{3.154057in}}%
\pgfpathcurveto{\pgfqpoint{1.974009in}{3.154057in}}{\pgfqpoint{1.984608in}{3.158447in}}{\pgfqpoint{1.992422in}{3.166261in}}%
\pgfpathcurveto{\pgfqpoint{2.000235in}{3.174075in}}{\pgfqpoint{2.004626in}{3.184674in}}{\pgfqpoint{2.004626in}{3.195724in}}%
\pgfpathcurveto{\pgfqpoint{2.004626in}{3.206774in}}{\pgfqpoint{2.000235in}{3.217373in}}{\pgfqpoint{1.992422in}{3.225187in}}%
\pgfpathcurveto{\pgfqpoint{1.984608in}{3.233000in}}{\pgfqpoint{1.974009in}{3.237391in}}{\pgfqpoint{1.962959in}{3.237391in}}%
\pgfpathcurveto{\pgfqpoint{1.951909in}{3.237391in}}{\pgfqpoint{1.941310in}{3.233000in}}{\pgfqpoint{1.933496in}{3.225187in}}%
\pgfpathcurveto{\pgfqpoint{1.925683in}{3.217373in}}{\pgfqpoint{1.921292in}{3.206774in}}{\pgfqpoint{1.921292in}{3.195724in}}%
\pgfpathcurveto{\pgfqpoint{1.921292in}{3.184674in}}{\pgfqpoint{1.925683in}{3.174075in}}{\pgfqpoint{1.933496in}{3.166261in}}%
\pgfpathcurveto{\pgfqpoint{1.941310in}{3.158447in}}{\pgfqpoint{1.951909in}{3.154057in}}{\pgfqpoint{1.962959in}{3.154057in}}%
\pgfpathclose%
\pgfusepath{stroke,fill}%
\end{pgfscope}%
\begin{pgfscope}%
\pgfpathrectangle{\pgfqpoint{0.787074in}{0.548769in}}{\pgfqpoint{5.062926in}{3.102590in}}%
\pgfusepath{clip}%
\pgfsetbuttcap%
\pgfsetroundjoin%
\definecolor{currentfill}{rgb}{1.000000,0.498039,0.054902}%
\pgfsetfillcolor{currentfill}%
\pgfsetlinewidth{1.003750pt}%
\definecolor{currentstroke}{rgb}{1.000000,0.498039,0.054902}%
\pgfsetstrokecolor{currentstroke}%
\pgfsetdash{}{0pt}%
\pgfpathmoveto{\pgfqpoint{1.395508in}{2.232087in}}%
\pgfpathcurveto{\pgfqpoint{1.406558in}{2.232087in}}{\pgfqpoint{1.417157in}{2.236477in}}{\pgfqpoint{1.424970in}{2.244291in}}%
\pgfpathcurveto{\pgfqpoint{1.432784in}{2.252104in}}{\pgfqpoint{1.437174in}{2.262703in}}{\pgfqpoint{1.437174in}{2.273753in}}%
\pgfpathcurveto{\pgfqpoint{1.437174in}{2.284804in}}{\pgfqpoint{1.432784in}{2.295403in}}{\pgfqpoint{1.424970in}{2.303216in}}%
\pgfpathcurveto{\pgfqpoint{1.417157in}{2.311030in}}{\pgfqpoint{1.406558in}{2.315420in}}{\pgfqpoint{1.395508in}{2.315420in}}%
\pgfpathcurveto{\pgfqpoint{1.384458in}{2.315420in}}{\pgfqpoint{1.373859in}{2.311030in}}{\pgfqpoint{1.366045in}{2.303216in}}%
\pgfpathcurveto{\pgfqpoint{1.358231in}{2.295403in}}{\pgfqpoint{1.353841in}{2.284804in}}{\pgfqpoint{1.353841in}{2.273753in}}%
\pgfpathcurveto{\pgfqpoint{1.353841in}{2.262703in}}{\pgfqpoint{1.358231in}{2.252104in}}{\pgfqpoint{1.366045in}{2.244291in}}%
\pgfpathcurveto{\pgfqpoint{1.373859in}{2.236477in}}{\pgfqpoint{1.384458in}{2.232087in}}{\pgfqpoint{1.395508in}{2.232087in}}%
\pgfpathclose%
\pgfusepath{stroke,fill}%
\end{pgfscope}%
\begin{pgfscope}%
\pgfpathrectangle{\pgfqpoint{0.787074in}{0.548769in}}{\pgfqpoint{5.062926in}{3.102590in}}%
\pgfusepath{clip}%
\pgfsetbuttcap%
\pgfsetroundjoin%
\definecolor{currentfill}{rgb}{0.121569,0.466667,0.705882}%
\pgfsetfillcolor{currentfill}%
\pgfsetlinewidth{1.003750pt}%
\definecolor{currentstroke}{rgb}{0.121569,0.466667,0.705882}%
\pgfsetstrokecolor{currentstroke}%
\pgfsetdash{}{0pt}%
\pgfpathmoveto{\pgfqpoint{1.080257in}{0.648132in}}%
\pgfpathcurveto{\pgfqpoint{1.091307in}{0.648132in}}{\pgfqpoint{1.101906in}{0.652523in}}{\pgfqpoint{1.109720in}{0.660336in}}%
\pgfpathcurveto{\pgfqpoint{1.117533in}{0.668150in}}{\pgfqpoint{1.121924in}{0.678749in}}{\pgfqpoint{1.121924in}{0.689799in}}%
\pgfpathcurveto{\pgfqpoint{1.121924in}{0.700849in}}{\pgfqpoint{1.117533in}{0.711448in}}{\pgfqpoint{1.109720in}{0.719262in}}%
\pgfpathcurveto{\pgfqpoint{1.101906in}{0.727075in}}{\pgfqpoint{1.091307in}{0.731466in}}{\pgfqpoint{1.080257in}{0.731466in}}%
\pgfpathcurveto{\pgfqpoint{1.069207in}{0.731466in}}{\pgfqpoint{1.058608in}{0.727075in}}{\pgfqpoint{1.050794in}{0.719262in}}%
\pgfpathcurveto{\pgfqpoint{1.042981in}{0.711448in}}{\pgfqpoint{1.038590in}{0.700849in}}{\pgfqpoint{1.038590in}{0.689799in}}%
\pgfpathcurveto{\pgfqpoint{1.038590in}{0.678749in}}{\pgfqpoint{1.042981in}{0.668150in}}{\pgfqpoint{1.050794in}{0.660336in}}%
\pgfpathcurveto{\pgfqpoint{1.058608in}{0.652523in}}{\pgfqpoint{1.069207in}{0.648132in}}{\pgfqpoint{1.080257in}{0.648132in}}%
\pgfpathclose%
\pgfusepath{stroke,fill}%
\end{pgfscope}%
\begin{pgfscope}%
\pgfpathrectangle{\pgfqpoint{0.787074in}{0.548769in}}{\pgfqpoint{5.062926in}{3.102590in}}%
\pgfusepath{clip}%
\pgfsetbuttcap%
\pgfsetroundjoin%
\definecolor{currentfill}{rgb}{0.121569,0.466667,0.705882}%
\pgfsetfillcolor{currentfill}%
\pgfsetlinewidth{1.003750pt}%
\definecolor{currentstroke}{rgb}{0.121569,0.466667,0.705882}%
\pgfsetstrokecolor{currentstroke}%
\pgfsetdash{}{0pt}%
\pgfpathmoveto{\pgfqpoint{1.395508in}{0.784785in}}%
\pgfpathcurveto{\pgfqpoint{1.406558in}{0.784785in}}{\pgfqpoint{1.417157in}{0.789175in}}{\pgfqpoint{1.424970in}{0.796989in}}%
\pgfpathcurveto{\pgfqpoint{1.432784in}{0.804802in}}{\pgfqpoint{1.437174in}{0.815401in}}{\pgfqpoint{1.437174in}{0.826451in}}%
\pgfpathcurveto{\pgfqpoint{1.437174in}{0.837502in}}{\pgfqpoint{1.432784in}{0.848101in}}{\pgfqpoint{1.424970in}{0.855914in}}%
\pgfpathcurveto{\pgfqpoint{1.417157in}{0.863728in}}{\pgfqpoint{1.406558in}{0.868118in}}{\pgfqpoint{1.395508in}{0.868118in}}%
\pgfpathcurveto{\pgfqpoint{1.384458in}{0.868118in}}{\pgfqpoint{1.373859in}{0.863728in}}{\pgfqpoint{1.366045in}{0.855914in}}%
\pgfpathcurveto{\pgfqpoint{1.358231in}{0.848101in}}{\pgfqpoint{1.353841in}{0.837502in}}{\pgfqpoint{1.353841in}{0.826451in}}%
\pgfpathcurveto{\pgfqpoint{1.353841in}{0.815401in}}{\pgfqpoint{1.358231in}{0.804802in}}{\pgfqpoint{1.366045in}{0.796989in}}%
\pgfpathcurveto{\pgfqpoint{1.373859in}{0.789175in}}{\pgfqpoint{1.384458in}{0.784785in}}{\pgfqpoint{1.395508in}{0.784785in}}%
\pgfpathclose%
\pgfusepath{stroke,fill}%
\end{pgfscope}%
\begin{pgfscope}%
\pgfpathrectangle{\pgfqpoint{0.787074in}{0.548769in}}{\pgfqpoint{5.062926in}{3.102590in}}%
\pgfusepath{clip}%
\pgfsetbuttcap%
\pgfsetroundjoin%
\definecolor{currentfill}{rgb}{0.121569,0.466667,0.705882}%
\pgfsetfillcolor{currentfill}%
\pgfsetlinewidth{1.003750pt}%
\definecolor{currentstroke}{rgb}{0.121569,0.466667,0.705882}%
\pgfsetstrokecolor{currentstroke}%
\pgfsetdash{}{0pt}%
\pgfpathmoveto{\pgfqpoint{1.269407in}{2.639480in}}%
\pgfpathcurveto{\pgfqpoint{1.280458in}{2.639480in}}{\pgfqpoint{1.291057in}{2.643870in}}{\pgfqpoint{1.298870in}{2.651684in}}%
\pgfpathcurveto{\pgfqpoint{1.306684in}{2.659498in}}{\pgfqpoint{1.311074in}{2.670097in}}{\pgfqpoint{1.311074in}{2.681147in}}%
\pgfpathcurveto{\pgfqpoint{1.311074in}{2.692197in}}{\pgfqpoint{1.306684in}{2.702796in}}{\pgfqpoint{1.298870in}{2.710610in}}%
\pgfpathcurveto{\pgfqpoint{1.291057in}{2.718423in}}{\pgfqpoint{1.280458in}{2.722813in}}{\pgfqpoint{1.269407in}{2.722813in}}%
\pgfpathcurveto{\pgfqpoint{1.258357in}{2.722813in}}{\pgfqpoint{1.247758in}{2.718423in}}{\pgfqpoint{1.239945in}{2.710610in}}%
\pgfpathcurveto{\pgfqpoint{1.232131in}{2.702796in}}{\pgfqpoint{1.227741in}{2.692197in}}{\pgfqpoint{1.227741in}{2.681147in}}%
\pgfpathcurveto{\pgfqpoint{1.227741in}{2.670097in}}{\pgfqpoint{1.232131in}{2.659498in}}{\pgfqpoint{1.239945in}{2.651684in}}%
\pgfpathcurveto{\pgfqpoint{1.247758in}{2.643870in}}{\pgfqpoint{1.258357in}{2.639480in}}{\pgfqpoint{1.269407in}{2.639480in}}%
\pgfpathclose%
\pgfusepath{stroke,fill}%
\end{pgfscope}%
\begin{pgfscope}%
\pgfpathrectangle{\pgfqpoint{0.787074in}{0.548769in}}{\pgfqpoint{5.062926in}{3.102590in}}%
\pgfusepath{clip}%
\pgfsetbuttcap%
\pgfsetroundjoin%
\definecolor{currentfill}{rgb}{0.121569,0.466667,0.705882}%
\pgfsetfillcolor{currentfill}%
\pgfsetlinewidth{1.003750pt}%
\definecolor{currentstroke}{rgb}{0.121569,0.466667,0.705882}%
\pgfsetstrokecolor{currentstroke}%
\pgfsetdash{}{0pt}%
\pgfpathmoveto{\pgfqpoint{1.332458in}{0.648148in}}%
\pgfpathcurveto{\pgfqpoint{1.343508in}{0.648148in}}{\pgfqpoint{1.354107in}{0.652539in}}{\pgfqpoint{1.361920in}{0.660352in}}%
\pgfpathcurveto{\pgfqpoint{1.369734in}{0.668166in}}{\pgfqpoint{1.374124in}{0.678765in}}{\pgfqpoint{1.374124in}{0.689815in}}%
\pgfpathcurveto{\pgfqpoint{1.374124in}{0.700865in}}{\pgfqpoint{1.369734in}{0.711464in}}{\pgfqpoint{1.361920in}{0.719278in}}%
\pgfpathcurveto{\pgfqpoint{1.354107in}{0.727091in}}{\pgfqpoint{1.343508in}{0.731482in}}{\pgfqpoint{1.332458in}{0.731482in}}%
\pgfpathcurveto{\pgfqpoint{1.321407in}{0.731482in}}{\pgfqpoint{1.310808in}{0.727091in}}{\pgfqpoint{1.302995in}{0.719278in}}%
\pgfpathcurveto{\pgfqpoint{1.295181in}{0.711464in}}{\pgfqpoint{1.290791in}{0.700865in}}{\pgfqpoint{1.290791in}{0.689815in}}%
\pgfpathcurveto{\pgfqpoint{1.290791in}{0.678765in}}{\pgfqpoint{1.295181in}{0.668166in}}{\pgfqpoint{1.302995in}{0.660352in}}%
\pgfpathcurveto{\pgfqpoint{1.310808in}{0.652539in}}{\pgfqpoint{1.321407in}{0.648148in}}{\pgfqpoint{1.332458in}{0.648148in}}%
\pgfpathclose%
\pgfusepath{stroke,fill}%
\end{pgfscope}%
\begin{pgfscope}%
\pgfpathrectangle{\pgfqpoint{0.787074in}{0.548769in}}{\pgfqpoint{5.062926in}{3.102590in}}%
\pgfusepath{clip}%
\pgfsetbuttcap%
\pgfsetroundjoin%
\definecolor{currentfill}{rgb}{0.121569,0.466667,0.705882}%
\pgfsetfillcolor{currentfill}%
\pgfsetlinewidth{1.003750pt}%
\definecolor{currentstroke}{rgb}{0.121569,0.466667,0.705882}%
\pgfsetstrokecolor{currentstroke}%
\pgfsetdash{}{0pt}%
\pgfpathmoveto{\pgfqpoint{1.332458in}{0.648149in}}%
\pgfpathcurveto{\pgfqpoint{1.343508in}{0.648149in}}{\pgfqpoint{1.354107in}{0.652540in}}{\pgfqpoint{1.361920in}{0.660353in}}%
\pgfpathcurveto{\pgfqpoint{1.369734in}{0.668167in}}{\pgfqpoint{1.374124in}{0.678766in}}{\pgfqpoint{1.374124in}{0.689816in}}%
\pgfpathcurveto{\pgfqpoint{1.374124in}{0.700866in}}{\pgfqpoint{1.369734in}{0.711465in}}{\pgfqpoint{1.361920in}{0.719279in}}%
\pgfpathcurveto{\pgfqpoint{1.354107in}{0.727093in}}{\pgfqpoint{1.343508in}{0.731483in}}{\pgfqpoint{1.332458in}{0.731483in}}%
\pgfpathcurveto{\pgfqpoint{1.321407in}{0.731483in}}{\pgfqpoint{1.310808in}{0.727093in}}{\pgfqpoint{1.302995in}{0.719279in}}%
\pgfpathcurveto{\pgfqpoint{1.295181in}{0.711465in}}{\pgfqpoint{1.290791in}{0.700866in}}{\pgfqpoint{1.290791in}{0.689816in}}%
\pgfpathcurveto{\pgfqpoint{1.290791in}{0.678766in}}{\pgfqpoint{1.295181in}{0.668167in}}{\pgfqpoint{1.302995in}{0.660353in}}%
\pgfpathcurveto{\pgfqpoint{1.310808in}{0.652540in}}{\pgfqpoint{1.321407in}{0.648149in}}{\pgfqpoint{1.332458in}{0.648149in}}%
\pgfpathclose%
\pgfusepath{stroke,fill}%
\end{pgfscope}%
\begin{pgfscope}%
\pgfpathrectangle{\pgfqpoint{0.787074in}{0.548769in}}{\pgfqpoint{5.062926in}{3.102590in}}%
\pgfusepath{clip}%
\pgfsetbuttcap%
\pgfsetroundjoin%
\definecolor{currentfill}{rgb}{0.121569,0.466667,0.705882}%
\pgfsetfillcolor{currentfill}%
\pgfsetlinewidth{1.003750pt}%
\definecolor{currentstroke}{rgb}{0.121569,0.466667,0.705882}%
\pgfsetstrokecolor{currentstroke}%
\pgfsetdash{}{0pt}%
\pgfpathmoveto{\pgfqpoint{1.080257in}{2.298244in}}%
\pgfpathcurveto{\pgfqpoint{1.091307in}{2.298244in}}{\pgfqpoint{1.101906in}{2.302634in}}{\pgfqpoint{1.109720in}{2.310448in}}%
\pgfpathcurveto{\pgfqpoint{1.117533in}{2.318261in}}{\pgfqpoint{1.121924in}{2.328860in}}{\pgfqpoint{1.121924in}{2.339911in}}%
\pgfpathcurveto{\pgfqpoint{1.121924in}{2.350961in}}{\pgfqpoint{1.117533in}{2.361560in}}{\pgfqpoint{1.109720in}{2.369373in}}%
\pgfpathcurveto{\pgfqpoint{1.101906in}{2.377187in}}{\pgfqpoint{1.091307in}{2.381577in}}{\pgfqpoint{1.080257in}{2.381577in}}%
\pgfpathcurveto{\pgfqpoint{1.069207in}{2.381577in}}{\pgfqpoint{1.058608in}{2.377187in}}{\pgfqpoint{1.050794in}{2.369373in}}%
\pgfpathcurveto{\pgfqpoint{1.042981in}{2.361560in}}{\pgfqpoint{1.038590in}{2.350961in}}{\pgfqpoint{1.038590in}{2.339911in}}%
\pgfpathcurveto{\pgfqpoint{1.038590in}{2.328860in}}{\pgfqpoint{1.042981in}{2.318261in}}{\pgfqpoint{1.050794in}{2.310448in}}%
\pgfpathcurveto{\pgfqpoint{1.058608in}{2.302634in}}{\pgfqpoint{1.069207in}{2.298244in}}{\pgfqpoint{1.080257in}{2.298244in}}%
\pgfpathclose%
\pgfusepath{stroke,fill}%
\end{pgfscope}%
\begin{pgfscope}%
\pgfpathrectangle{\pgfqpoint{0.787074in}{0.548769in}}{\pgfqpoint{5.062926in}{3.102590in}}%
\pgfusepath{clip}%
\pgfsetbuttcap%
\pgfsetroundjoin%
\definecolor{currentfill}{rgb}{1.000000,0.498039,0.054902}%
\pgfsetfillcolor{currentfill}%
\pgfsetlinewidth{1.003750pt}%
\definecolor{currentstroke}{rgb}{1.000000,0.498039,0.054902}%
\pgfsetstrokecolor{currentstroke}%
\pgfsetdash{}{0pt}%
\pgfpathmoveto{\pgfqpoint{1.017207in}{2.885396in}}%
\pgfpathcurveto{\pgfqpoint{1.028257in}{2.885396in}}{\pgfqpoint{1.038856in}{2.889786in}}{\pgfqpoint{1.046670in}{2.897600in}}%
\pgfpathcurveto{\pgfqpoint{1.054483in}{2.905413in}}{\pgfqpoint{1.058874in}{2.916012in}}{\pgfqpoint{1.058874in}{2.927063in}}%
\pgfpathcurveto{\pgfqpoint{1.058874in}{2.938113in}}{\pgfqpoint{1.054483in}{2.948712in}}{\pgfqpoint{1.046670in}{2.956525in}}%
\pgfpathcurveto{\pgfqpoint{1.038856in}{2.964339in}}{\pgfqpoint{1.028257in}{2.968729in}}{\pgfqpoint{1.017207in}{2.968729in}}%
\pgfpathcurveto{\pgfqpoint{1.006157in}{2.968729in}}{\pgfqpoint{0.995558in}{2.964339in}}{\pgfqpoint{0.987744in}{2.956525in}}%
\pgfpathcurveto{\pgfqpoint{0.979930in}{2.948712in}}{\pgfqpoint{0.975540in}{2.938113in}}{\pgfqpoint{0.975540in}{2.927063in}}%
\pgfpathcurveto{\pgfqpoint{0.975540in}{2.916012in}}{\pgfqpoint{0.979930in}{2.905413in}}{\pgfqpoint{0.987744in}{2.897600in}}%
\pgfpathcurveto{\pgfqpoint{0.995558in}{2.889786in}}{\pgfqpoint{1.006157in}{2.885396in}}{\pgfqpoint{1.017207in}{2.885396in}}%
\pgfpathclose%
\pgfusepath{stroke,fill}%
\end{pgfscope}%
\begin{pgfscope}%
\pgfpathrectangle{\pgfqpoint{0.787074in}{0.548769in}}{\pgfqpoint{5.062926in}{3.102590in}}%
\pgfusepath{clip}%
\pgfsetbuttcap%
\pgfsetroundjoin%
\definecolor{currentfill}{rgb}{1.000000,0.498039,0.054902}%
\pgfsetfillcolor{currentfill}%
\pgfsetlinewidth{1.003750pt}%
\definecolor{currentstroke}{rgb}{1.000000,0.498039,0.054902}%
\pgfsetstrokecolor{currentstroke}%
\pgfsetdash{}{0pt}%
\pgfpathmoveto{\pgfqpoint{3.034811in}{2.303844in}}%
\pgfpathcurveto{\pgfqpoint{3.045861in}{2.303844in}}{\pgfqpoint{3.056460in}{2.308235in}}{\pgfqpoint{3.064274in}{2.316048in}}%
\pgfpathcurveto{\pgfqpoint{3.072088in}{2.323862in}}{\pgfqpoint{3.076478in}{2.334461in}}{\pgfqpoint{3.076478in}{2.345511in}}%
\pgfpathcurveto{\pgfqpoint{3.076478in}{2.356561in}}{\pgfqpoint{3.072088in}{2.367160in}}{\pgfqpoint{3.064274in}{2.374974in}}%
\pgfpathcurveto{\pgfqpoint{3.056460in}{2.382787in}}{\pgfqpoint{3.045861in}{2.387178in}}{\pgfqpoint{3.034811in}{2.387178in}}%
\pgfpathcurveto{\pgfqpoint{3.023761in}{2.387178in}}{\pgfqpoint{3.013162in}{2.382787in}}{\pgfqpoint{3.005349in}{2.374974in}}%
\pgfpathcurveto{\pgfqpoint{2.997535in}{2.367160in}}{\pgfqpoint{2.993145in}{2.356561in}}{\pgfqpoint{2.993145in}{2.345511in}}%
\pgfpathcurveto{\pgfqpoint{2.993145in}{2.334461in}}{\pgfqpoint{2.997535in}{2.323862in}}{\pgfqpoint{3.005349in}{2.316048in}}%
\pgfpathcurveto{\pgfqpoint{3.013162in}{2.308235in}}{\pgfqpoint{3.023761in}{2.303844in}}{\pgfqpoint{3.034811in}{2.303844in}}%
\pgfpathclose%
\pgfusepath{stroke,fill}%
\end{pgfscope}%
\begin{pgfscope}%
\pgfpathrectangle{\pgfqpoint{0.787074in}{0.548769in}}{\pgfqpoint{5.062926in}{3.102590in}}%
\pgfusepath{clip}%
\pgfsetbuttcap%
\pgfsetroundjoin%
\definecolor{currentfill}{rgb}{1.000000,0.498039,0.054902}%
\pgfsetfillcolor{currentfill}%
\pgfsetlinewidth{1.003750pt}%
\definecolor{currentstroke}{rgb}{1.000000,0.498039,0.054902}%
\pgfsetstrokecolor{currentstroke}%
\pgfsetdash{}{0pt}%
\pgfpathmoveto{\pgfqpoint{1.836859in}{2.924913in}}%
\pgfpathcurveto{\pgfqpoint{1.847909in}{2.924913in}}{\pgfqpoint{1.858508in}{2.929304in}}{\pgfqpoint{1.866321in}{2.937117in}}%
\pgfpathcurveto{\pgfqpoint{1.874135in}{2.944931in}}{\pgfqpoint{1.878525in}{2.955530in}}{\pgfqpoint{1.878525in}{2.966580in}}%
\pgfpathcurveto{\pgfqpoint{1.878525in}{2.977630in}}{\pgfqpoint{1.874135in}{2.988229in}}{\pgfqpoint{1.866321in}{2.996043in}}%
\pgfpathcurveto{\pgfqpoint{1.858508in}{3.003856in}}{\pgfqpoint{1.847909in}{3.008247in}}{\pgfqpoint{1.836859in}{3.008247in}}%
\pgfpathcurveto{\pgfqpoint{1.825809in}{3.008247in}}{\pgfqpoint{1.815209in}{3.003856in}}{\pgfqpoint{1.807396in}{2.996043in}}%
\pgfpathcurveto{\pgfqpoint{1.799582in}{2.988229in}}{\pgfqpoint{1.795192in}{2.977630in}}{\pgfqpoint{1.795192in}{2.966580in}}%
\pgfpathcurveto{\pgfqpoint{1.795192in}{2.955530in}}{\pgfqpoint{1.799582in}{2.944931in}}{\pgfqpoint{1.807396in}{2.937117in}}%
\pgfpathcurveto{\pgfqpoint{1.815209in}{2.929304in}}{\pgfqpoint{1.825809in}{2.924913in}}{\pgfqpoint{1.836859in}{2.924913in}}%
\pgfpathclose%
\pgfusepath{stroke,fill}%
\end{pgfscope}%
\begin{pgfscope}%
\pgfpathrectangle{\pgfqpoint{0.787074in}{0.548769in}}{\pgfqpoint{5.062926in}{3.102590in}}%
\pgfusepath{clip}%
\pgfsetbuttcap%
\pgfsetroundjoin%
\definecolor{currentfill}{rgb}{1.000000,0.498039,0.054902}%
\pgfsetfillcolor{currentfill}%
\pgfsetlinewidth{1.003750pt}%
\definecolor{currentstroke}{rgb}{1.000000,0.498039,0.054902}%
\pgfsetstrokecolor{currentstroke}%
\pgfsetdash{}{0pt}%
\pgfpathmoveto{\pgfqpoint{1.332458in}{2.712046in}}%
\pgfpathcurveto{\pgfqpoint{1.343508in}{2.712046in}}{\pgfqpoint{1.354107in}{2.716436in}}{\pgfqpoint{1.361920in}{2.724249in}}%
\pgfpathcurveto{\pgfqpoint{1.369734in}{2.732063in}}{\pgfqpoint{1.374124in}{2.742662in}}{\pgfqpoint{1.374124in}{2.753712in}}%
\pgfpathcurveto{\pgfqpoint{1.374124in}{2.764762in}}{\pgfqpoint{1.369734in}{2.775361in}}{\pgfqpoint{1.361920in}{2.783175in}}%
\pgfpathcurveto{\pgfqpoint{1.354107in}{2.790989in}}{\pgfqpoint{1.343508in}{2.795379in}}{\pgfqpoint{1.332458in}{2.795379in}}%
\pgfpathcurveto{\pgfqpoint{1.321407in}{2.795379in}}{\pgfqpoint{1.310808in}{2.790989in}}{\pgfqpoint{1.302995in}{2.783175in}}%
\pgfpathcurveto{\pgfqpoint{1.295181in}{2.775361in}}{\pgfqpoint{1.290791in}{2.764762in}}{\pgfqpoint{1.290791in}{2.753712in}}%
\pgfpathcurveto{\pgfqpoint{1.290791in}{2.742662in}}{\pgfqpoint{1.295181in}{2.732063in}}{\pgfqpoint{1.302995in}{2.724249in}}%
\pgfpathcurveto{\pgfqpoint{1.310808in}{2.716436in}}{\pgfqpoint{1.321407in}{2.712046in}}{\pgfqpoint{1.332458in}{2.712046in}}%
\pgfpathclose%
\pgfusepath{stroke,fill}%
\end{pgfscope}%
\begin{pgfscope}%
\pgfpathrectangle{\pgfqpoint{0.787074in}{0.548769in}}{\pgfqpoint{5.062926in}{3.102590in}}%
\pgfusepath{clip}%
\pgfsetbuttcap%
\pgfsetroundjoin%
\definecolor{currentfill}{rgb}{0.121569,0.466667,0.705882}%
\pgfsetfillcolor{currentfill}%
\pgfsetlinewidth{1.003750pt}%
\definecolor{currentstroke}{rgb}{0.121569,0.466667,0.705882}%
\pgfsetstrokecolor{currentstroke}%
\pgfsetdash{}{0pt}%
\pgfpathmoveto{\pgfqpoint{3.917513in}{3.116121in}}%
\pgfpathcurveto{\pgfqpoint{3.928563in}{3.116121in}}{\pgfqpoint{3.939162in}{3.120511in}}{\pgfqpoint{3.946976in}{3.128325in}}%
\pgfpathcurveto{\pgfqpoint{3.954790in}{3.136138in}}{\pgfqpoint{3.959180in}{3.146737in}}{\pgfqpoint{3.959180in}{3.157787in}}%
\pgfpathcurveto{\pgfqpoint{3.959180in}{3.168837in}}{\pgfqpoint{3.954790in}{3.179436in}}{\pgfqpoint{3.946976in}{3.187250in}}%
\pgfpathcurveto{\pgfqpoint{3.939162in}{3.195064in}}{\pgfqpoint{3.928563in}{3.199454in}}{\pgfqpoint{3.917513in}{3.199454in}}%
\pgfpathcurveto{\pgfqpoint{3.906463in}{3.199454in}}{\pgfqpoint{3.895864in}{3.195064in}}{\pgfqpoint{3.888050in}{3.187250in}}%
\pgfpathcurveto{\pgfqpoint{3.880237in}{3.179436in}}{\pgfqpoint{3.875847in}{3.168837in}}{\pgfqpoint{3.875847in}{3.157787in}}%
\pgfpathcurveto{\pgfqpoint{3.875847in}{3.146737in}}{\pgfqpoint{3.880237in}{3.136138in}}{\pgfqpoint{3.888050in}{3.128325in}}%
\pgfpathcurveto{\pgfqpoint{3.895864in}{3.120511in}}{\pgfqpoint{3.906463in}{3.116121in}}{\pgfqpoint{3.917513in}{3.116121in}}%
\pgfpathclose%
\pgfusepath{stroke,fill}%
\end{pgfscope}%
\begin{pgfscope}%
\pgfpathrectangle{\pgfqpoint{0.787074in}{0.548769in}}{\pgfqpoint{5.062926in}{3.102590in}}%
\pgfusepath{clip}%
\pgfsetbuttcap%
\pgfsetroundjoin%
\definecolor{currentfill}{rgb}{0.839216,0.152941,0.156863}%
\pgfsetfillcolor{currentfill}%
\pgfsetlinewidth{1.003750pt}%
\definecolor{currentstroke}{rgb}{0.839216,0.152941,0.156863}%
\pgfsetstrokecolor{currentstroke}%
\pgfsetdash{}{0pt}%
\pgfpathmoveto{\pgfqpoint{3.854463in}{3.468665in}}%
\pgfpathcurveto{\pgfqpoint{3.865513in}{3.468665in}}{\pgfqpoint{3.876112in}{3.473055in}}{\pgfqpoint{3.883926in}{3.480869in}}%
\pgfpathcurveto{\pgfqpoint{3.891740in}{3.488683in}}{\pgfqpoint{3.896130in}{3.499282in}}{\pgfqpoint{3.896130in}{3.510332in}}%
\pgfpathcurveto{\pgfqpoint{3.896130in}{3.521382in}}{\pgfqpoint{3.891740in}{3.531981in}}{\pgfqpoint{3.883926in}{3.539795in}}%
\pgfpathcurveto{\pgfqpoint{3.876112in}{3.547608in}}{\pgfqpoint{3.865513in}{3.551998in}}{\pgfqpoint{3.854463in}{3.551998in}}%
\pgfpathcurveto{\pgfqpoint{3.843413in}{3.551998in}}{\pgfqpoint{3.832814in}{3.547608in}}{\pgfqpoint{3.825000in}{3.539795in}}%
\pgfpathcurveto{\pgfqpoint{3.817187in}{3.531981in}}{\pgfqpoint{3.812796in}{3.521382in}}{\pgfqpoint{3.812796in}{3.510332in}}%
\pgfpathcurveto{\pgfqpoint{3.812796in}{3.499282in}}{\pgfqpoint{3.817187in}{3.488683in}}{\pgfqpoint{3.825000in}{3.480869in}}%
\pgfpathcurveto{\pgfqpoint{3.832814in}{3.473055in}}{\pgfqpoint{3.843413in}{3.468665in}}{\pgfqpoint{3.854463in}{3.468665in}}%
\pgfpathclose%
\pgfusepath{stroke,fill}%
\end{pgfscope}%
\begin{pgfscope}%
\pgfpathrectangle{\pgfqpoint{0.787074in}{0.548769in}}{\pgfqpoint{5.062926in}{3.102590in}}%
\pgfusepath{clip}%
\pgfsetbuttcap%
\pgfsetroundjoin%
\definecolor{currentfill}{rgb}{1.000000,0.498039,0.054902}%
\pgfsetfillcolor{currentfill}%
\pgfsetlinewidth{1.003750pt}%
\definecolor{currentstroke}{rgb}{1.000000,0.498039,0.054902}%
\pgfsetstrokecolor{currentstroke}%
\pgfsetdash{}{0pt}%
\pgfpathmoveto{\pgfqpoint{1.521608in}{3.121989in}}%
\pgfpathcurveto{\pgfqpoint{1.532658in}{3.121989in}}{\pgfqpoint{1.543257in}{3.126380in}}{\pgfqpoint{1.551071in}{3.134193in}}%
\pgfpathcurveto{\pgfqpoint{1.558884in}{3.142007in}}{\pgfqpoint{1.563275in}{3.152606in}}{\pgfqpoint{1.563275in}{3.163656in}}%
\pgfpathcurveto{\pgfqpoint{1.563275in}{3.174706in}}{\pgfqpoint{1.558884in}{3.185305in}}{\pgfqpoint{1.551071in}{3.193119in}}%
\pgfpathcurveto{\pgfqpoint{1.543257in}{3.200932in}}{\pgfqpoint{1.532658in}{3.205323in}}{\pgfqpoint{1.521608in}{3.205323in}}%
\pgfpathcurveto{\pgfqpoint{1.510558in}{3.205323in}}{\pgfqpoint{1.499959in}{3.200932in}}{\pgfqpoint{1.492145in}{3.193119in}}%
\pgfpathcurveto{\pgfqpoint{1.484332in}{3.185305in}}{\pgfqpoint{1.479941in}{3.174706in}}{\pgfqpoint{1.479941in}{3.163656in}}%
\pgfpathcurveto{\pgfqpoint{1.479941in}{3.152606in}}{\pgfqpoint{1.484332in}{3.142007in}}{\pgfqpoint{1.492145in}{3.134193in}}%
\pgfpathcurveto{\pgfqpoint{1.499959in}{3.126380in}}{\pgfqpoint{1.510558in}{3.121989in}}{\pgfqpoint{1.521608in}{3.121989in}}%
\pgfpathclose%
\pgfusepath{stroke,fill}%
\end{pgfscope}%
\begin{pgfscope}%
\pgfpathrectangle{\pgfqpoint{0.787074in}{0.548769in}}{\pgfqpoint{5.062926in}{3.102590in}}%
\pgfusepath{clip}%
\pgfsetbuttcap%
\pgfsetroundjoin%
\definecolor{currentfill}{rgb}{1.000000,0.498039,0.054902}%
\pgfsetfillcolor{currentfill}%
\pgfsetlinewidth{1.003750pt}%
\definecolor{currentstroke}{rgb}{1.000000,0.498039,0.054902}%
\pgfsetstrokecolor{currentstroke}%
\pgfsetdash{}{0pt}%
\pgfpathmoveto{\pgfqpoint{3.097861in}{2.796310in}}%
\pgfpathcurveto{\pgfqpoint{3.108912in}{2.796310in}}{\pgfqpoint{3.119511in}{2.800700in}}{\pgfqpoint{3.127324in}{2.808514in}}%
\pgfpathcurveto{\pgfqpoint{3.135138in}{2.816328in}}{\pgfqpoint{3.139528in}{2.826927in}}{\pgfqpoint{3.139528in}{2.837977in}}%
\pgfpathcurveto{\pgfqpoint{3.139528in}{2.849027in}}{\pgfqpoint{3.135138in}{2.859626in}}{\pgfqpoint{3.127324in}{2.867440in}}%
\pgfpathcurveto{\pgfqpoint{3.119511in}{2.875253in}}{\pgfqpoint{3.108912in}{2.879644in}}{\pgfqpoint{3.097861in}{2.879644in}}%
\pgfpathcurveto{\pgfqpoint{3.086811in}{2.879644in}}{\pgfqpoint{3.076212in}{2.875253in}}{\pgfqpoint{3.068399in}{2.867440in}}%
\pgfpathcurveto{\pgfqpoint{3.060585in}{2.859626in}}{\pgfqpoint{3.056195in}{2.849027in}}{\pgfqpoint{3.056195in}{2.837977in}}%
\pgfpathcurveto{\pgfqpoint{3.056195in}{2.826927in}}{\pgfqpoint{3.060585in}{2.816328in}}{\pgfqpoint{3.068399in}{2.808514in}}%
\pgfpathcurveto{\pgfqpoint{3.076212in}{2.800700in}}{\pgfqpoint{3.086811in}{2.796310in}}{\pgfqpoint{3.097861in}{2.796310in}}%
\pgfpathclose%
\pgfusepath{stroke,fill}%
\end{pgfscope}%
\begin{pgfscope}%
\pgfpathrectangle{\pgfqpoint{0.787074in}{0.548769in}}{\pgfqpoint{5.062926in}{3.102590in}}%
\pgfusepath{clip}%
\pgfsetbuttcap%
\pgfsetroundjoin%
\definecolor{currentfill}{rgb}{0.121569,0.466667,0.705882}%
\pgfsetfillcolor{currentfill}%
\pgfsetlinewidth{1.003750pt}%
\definecolor{currentstroke}{rgb}{0.121569,0.466667,0.705882}%
\pgfsetstrokecolor{currentstroke}%
\pgfsetdash{}{0pt}%
\pgfpathmoveto{\pgfqpoint{2.719561in}{2.391680in}}%
\pgfpathcurveto{\pgfqpoint{2.730611in}{2.391680in}}{\pgfqpoint{2.741210in}{2.396070in}}{\pgfqpoint{2.749023in}{2.403884in}}%
\pgfpathcurveto{\pgfqpoint{2.756837in}{2.411697in}}{\pgfqpoint{2.761227in}{2.422296in}}{\pgfqpoint{2.761227in}{2.433347in}}%
\pgfpathcurveto{\pgfqpoint{2.761227in}{2.444397in}}{\pgfqpoint{2.756837in}{2.454996in}}{\pgfqpoint{2.749023in}{2.462809in}}%
\pgfpathcurveto{\pgfqpoint{2.741210in}{2.470623in}}{\pgfqpoint{2.730611in}{2.475013in}}{\pgfqpoint{2.719561in}{2.475013in}}%
\pgfpathcurveto{\pgfqpoint{2.708510in}{2.475013in}}{\pgfqpoint{2.697911in}{2.470623in}}{\pgfqpoint{2.690098in}{2.462809in}}%
\pgfpathcurveto{\pgfqpoint{2.682284in}{2.454996in}}{\pgfqpoint{2.677894in}{2.444397in}}{\pgfqpoint{2.677894in}{2.433347in}}%
\pgfpathcurveto{\pgfqpoint{2.677894in}{2.422296in}}{\pgfqpoint{2.682284in}{2.411697in}}{\pgfqpoint{2.690098in}{2.403884in}}%
\pgfpathcurveto{\pgfqpoint{2.697911in}{2.396070in}}{\pgfqpoint{2.708510in}{2.391680in}}{\pgfqpoint{2.719561in}{2.391680in}}%
\pgfpathclose%
\pgfusepath{stroke,fill}%
\end{pgfscope}%
\begin{pgfscope}%
\pgfpathrectangle{\pgfqpoint{0.787074in}{0.548769in}}{\pgfqpoint{5.062926in}{3.102590in}}%
\pgfusepath{clip}%
\pgfsetbuttcap%
\pgfsetroundjoin%
\definecolor{currentfill}{rgb}{1.000000,0.498039,0.054902}%
\pgfsetfillcolor{currentfill}%
\pgfsetlinewidth{1.003750pt}%
\definecolor{currentstroke}{rgb}{1.000000,0.498039,0.054902}%
\pgfsetstrokecolor{currentstroke}%
\pgfsetdash{}{0pt}%
\pgfpathmoveto{\pgfqpoint{1.143307in}{2.448155in}}%
\pgfpathcurveto{\pgfqpoint{1.154357in}{2.448155in}}{\pgfqpoint{1.164956in}{2.452545in}}{\pgfqpoint{1.172770in}{2.460359in}}%
\pgfpathcurveto{\pgfqpoint{1.180584in}{2.468173in}}{\pgfqpoint{1.184974in}{2.478772in}}{\pgfqpoint{1.184974in}{2.489822in}}%
\pgfpathcurveto{\pgfqpoint{1.184974in}{2.500872in}}{\pgfqpoint{1.180584in}{2.511471in}}{\pgfqpoint{1.172770in}{2.519285in}}%
\pgfpathcurveto{\pgfqpoint{1.164956in}{2.527098in}}{\pgfqpoint{1.154357in}{2.531489in}}{\pgfqpoint{1.143307in}{2.531489in}}%
\pgfpathcurveto{\pgfqpoint{1.132257in}{2.531489in}}{\pgfqpoint{1.121658in}{2.527098in}}{\pgfqpoint{1.113844in}{2.519285in}}%
\pgfpathcurveto{\pgfqpoint{1.106031in}{2.511471in}}{\pgfqpoint{1.101640in}{2.500872in}}{\pgfqpoint{1.101640in}{2.489822in}}%
\pgfpathcurveto{\pgfqpoint{1.101640in}{2.478772in}}{\pgfqpoint{1.106031in}{2.468173in}}{\pgfqpoint{1.113844in}{2.460359in}}%
\pgfpathcurveto{\pgfqpoint{1.121658in}{2.452545in}}{\pgfqpoint{1.132257in}{2.448155in}}{\pgfqpoint{1.143307in}{2.448155in}}%
\pgfpathclose%
\pgfusepath{stroke,fill}%
\end{pgfscope}%
\begin{pgfscope}%
\pgfpathrectangle{\pgfqpoint{0.787074in}{0.548769in}}{\pgfqpoint{5.062926in}{3.102590in}}%
\pgfusepath{clip}%
\pgfsetbuttcap%
\pgfsetroundjoin%
\definecolor{currentfill}{rgb}{1.000000,0.498039,0.054902}%
\pgfsetfillcolor{currentfill}%
\pgfsetlinewidth{1.003750pt}%
\definecolor{currentstroke}{rgb}{1.000000,0.498039,0.054902}%
\pgfsetstrokecolor{currentstroke}%
\pgfsetdash{}{0pt}%
\pgfpathmoveto{\pgfqpoint{2.530410in}{3.336404in}}%
\pgfpathcurveto{\pgfqpoint{2.541460in}{3.336404in}}{\pgfqpoint{2.552059in}{3.340794in}}{\pgfqpoint{2.559873in}{3.348607in}}%
\pgfpathcurveto{\pgfqpoint{2.567687in}{3.356421in}}{\pgfqpoint{2.572077in}{3.367020in}}{\pgfqpoint{2.572077in}{3.378070in}}%
\pgfpathcurveto{\pgfqpoint{2.572077in}{3.389120in}}{\pgfqpoint{2.567687in}{3.399719in}}{\pgfqpoint{2.559873in}{3.407533in}}%
\pgfpathcurveto{\pgfqpoint{2.552059in}{3.415347in}}{\pgfqpoint{2.541460in}{3.419737in}}{\pgfqpoint{2.530410in}{3.419737in}}%
\pgfpathcurveto{\pgfqpoint{2.519360in}{3.419737in}}{\pgfqpoint{2.508761in}{3.415347in}}{\pgfqpoint{2.500947in}{3.407533in}}%
\pgfpathcurveto{\pgfqpoint{2.493134in}{3.399719in}}{\pgfqpoint{2.488744in}{3.389120in}}{\pgfqpoint{2.488744in}{3.378070in}}%
\pgfpathcurveto{\pgfqpoint{2.488744in}{3.367020in}}{\pgfqpoint{2.493134in}{3.356421in}}{\pgfqpoint{2.500947in}{3.348607in}}%
\pgfpathcurveto{\pgfqpoint{2.508761in}{3.340794in}}{\pgfqpoint{2.519360in}{3.336404in}}{\pgfqpoint{2.530410in}{3.336404in}}%
\pgfpathclose%
\pgfusepath{stroke,fill}%
\end{pgfscope}%
\begin{pgfscope}%
\pgfpathrectangle{\pgfqpoint{0.787074in}{0.548769in}}{\pgfqpoint{5.062926in}{3.102590in}}%
\pgfusepath{clip}%
\pgfsetbuttcap%
\pgfsetroundjoin%
\definecolor{currentfill}{rgb}{1.000000,0.498039,0.054902}%
\pgfsetfillcolor{currentfill}%
\pgfsetlinewidth{1.003750pt}%
\definecolor{currentstroke}{rgb}{1.000000,0.498039,0.054902}%
\pgfsetstrokecolor{currentstroke}%
\pgfsetdash{}{0pt}%
\pgfpathmoveto{\pgfqpoint{2.089059in}{2.728731in}}%
\pgfpathcurveto{\pgfqpoint{2.100109in}{2.728731in}}{\pgfqpoint{2.110708in}{2.733121in}}{\pgfqpoint{2.118522in}{2.740935in}}%
\pgfpathcurveto{\pgfqpoint{2.126336in}{2.748749in}}{\pgfqpoint{2.130726in}{2.759348in}}{\pgfqpoint{2.130726in}{2.770398in}}%
\pgfpathcurveto{\pgfqpoint{2.130726in}{2.781448in}}{\pgfqpoint{2.126336in}{2.792047in}}{\pgfqpoint{2.118522in}{2.799861in}}%
\pgfpathcurveto{\pgfqpoint{2.110708in}{2.807674in}}{\pgfqpoint{2.100109in}{2.812064in}}{\pgfqpoint{2.089059in}{2.812064in}}%
\pgfpathcurveto{\pgfqpoint{2.078009in}{2.812064in}}{\pgfqpoint{2.067410in}{2.807674in}}{\pgfqpoint{2.059596in}{2.799861in}}%
\pgfpathcurveto{\pgfqpoint{2.051783in}{2.792047in}}{\pgfqpoint{2.047393in}{2.781448in}}{\pgfqpoint{2.047393in}{2.770398in}}%
\pgfpathcurveto{\pgfqpoint{2.047393in}{2.759348in}}{\pgfqpoint{2.051783in}{2.748749in}}{\pgfqpoint{2.059596in}{2.740935in}}%
\pgfpathcurveto{\pgfqpoint{2.067410in}{2.733121in}}{\pgfqpoint{2.078009in}{2.728731in}}{\pgfqpoint{2.089059in}{2.728731in}}%
\pgfpathclose%
\pgfusepath{stroke,fill}%
\end{pgfscope}%
\begin{pgfscope}%
\pgfpathrectangle{\pgfqpoint{0.787074in}{0.548769in}}{\pgfqpoint{5.062926in}{3.102590in}}%
\pgfusepath{clip}%
\pgfsetbuttcap%
\pgfsetroundjoin%
\definecolor{currentfill}{rgb}{1.000000,0.498039,0.054902}%
\pgfsetfillcolor{currentfill}%
\pgfsetlinewidth{1.003750pt}%
\definecolor{currentstroke}{rgb}{1.000000,0.498039,0.054902}%
\pgfsetstrokecolor{currentstroke}%
\pgfsetdash{}{0pt}%
\pgfpathmoveto{\pgfqpoint{2.026009in}{2.523921in}}%
\pgfpathcurveto{\pgfqpoint{2.037059in}{2.523921in}}{\pgfqpoint{2.047658in}{2.528311in}}{\pgfqpoint{2.055472in}{2.536125in}}%
\pgfpathcurveto{\pgfqpoint{2.063285in}{2.543939in}}{\pgfqpoint{2.067676in}{2.554538in}}{\pgfqpoint{2.067676in}{2.565588in}}%
\pgfpathcurveto{\pgfqpoint{2.067676in}{2.576638in}}{\pgfqpoint{2.063285in}{2.587237in}}{\pgfqpoint{2.055472in}{2.595051in}}%
\pgfpathcurveto{\pgfqpoint{2.047658in}{2.602864in}}{\pgfqpoint{2.037059in}{2.607254in}}{\pgfqpoint{2.026009in}{2.607254in}}%
\pgfpathcurveto{\pgfqpoint{2.014959in}{2.607254in}}{\pgfqpoint{2.004360in}{2.602864in}}{\pgfqpoint{1.996546in}{2.595051in}}%
\pgfpathcurveto{\pgfqpoint{1.988733in}{2.587237in}}{\pgfqpoint{1.984342in}{2.576638in}}{\pgfqpoint{1.984342in}{2.565588in}}%
\pgfpathcurveto{\pgfqpoint{1.984342in}{2.554538in}}{\pgfqpoint{1.988733in}{2.543939in}}{\pgfqpoint{1.996546in}{2.536125in}}%
\pgfpathcurveto{\pgfqpoint{2.004360in}{2.528311in}}{\pgfqpoint{2.014959in}{2.523921in}}{\pgfqpoint{2.026009in}{2.523921in}}%
\pgfpathclose%
\pgfusepath{stroke,fill}%
\end{pgfscope}%
\begin{pgfscope}%
\pgfpathrectangle{\pgfqpoint{0.787074in}{0.548769in}}{\pgfqpoint{5.062926in}{3.102590in}}%
\pgfusepath{clip}%
\pgfsetbuttcap%
\pgfsetroundjoin%
\definecolor{currentfill}{rgb}{0.121569,0.466667,0.705882}%
\pgfsetfillcolor{currentfill}%
\pgfsetlinewidth{1.003750pt}%
\definecolor{currentstroke}{rgb}{0.121569,0.466667,0.705882}%
\pgfsetstrokecolor{currentstroke}%
\pgfsetdash{}{0pt}%
\pgfpathmoveto{\pgfqpoint{2.026009in}{2.939541in}}%
\pgfpathcurveto{\pgfqpoint{2.037059in}{2.939541in}}{\pgfqpoint{2.047658in}{2.943931in}}{\pgfqpoint{2.055472in}{2.951745in}}%
\pgfpathcurveto{\pgfqpoint{2.063285in}{2.959559in}}{\pgfqpoint{2.067676in}{2.970158in}}{\pgfqpoint{2.067676in}{2.981208in}}%
\pgfpathcurveto{\pgfqpoint{2.067676in}{2.992258in}}{\pgfqpoint{2.063285in}{3.002857in}}{\pgfqpoint{2.055472in}{3.010671in}}%
\pgfpathcurveto{\pgfqpoint{2.047658in}{3.018484in}}{\pgfqpoint{2.037059in}{3.022874in}}{\pgfqpoint{2.026009in}{3.022874in}}%
\pgfpathcurveto{\pgfqpoint{2.014959in}{3.022874in}}{\pgfqpoint{2.004360in}{3.018484in}}{\pgfqpoint{1.996546in}{3.010671in}}%
\pgfpathcurveto{\pgfqpoint{1.988733in}{3.002857in}}{\pgfqpoint{1.984342in}{2.992258in}}{\pgfqpoint{1.984342in}{2.981208in}}%
\pgfpathcurveto{\pgfqpoint{1.984342in}{2.970158in}}{\pgfqpoint{1.988733in}{2.959559in}}{\pgfqpoint{1.996546in}{2.951745in}}%
\pgfpathcurveto{\pgfqpoint{2.004360in}{2.943931in}}{\pgfqpoint{2.014959in}{2.939541in}}{\pgfqpoint{2.026009in}{2.939541in}}%
\pgfpathclose%
\pgfusepath{stroke,fill}%
\end{pgfscope}%
\begin{pgfscope}%
\pgfpathrectangle{\pgfqpoint{0.787074in}{0.548769in}}{\pgfqpoint{5.062926in}{3.102590in}}%
\pgfusepath{clip}%
\pgfsetbuttcap%
\pgfsetroundjoin%
\definecolor{currentfill}{rgb}{1.000000,0.498039,0.054902}%
\pgfsetfillcolor{currentfill}%
\pgfsetlinewidth{1.003750pt}%
\definecolor{currentstroke}{rgb}{1.000000,0.498039,0.054902}%
\pgfsetstrokecolor{currentstroke}%
\pgfsetdash{}{0pt}%
\pgfpathmoveto{\pgfqpoint{1.647708in}{2.934357in}}%
\pgfpathcurveto{\pgfqpoint{1.658758in}{2.934357in}}{\pgfqpoint{1.669357in}{2.938747in}}{\pgfqpoint{1.677171in}{2.946561in}}%
\pgfpathcurveto{\pgfqpoint{1.684985in}{2.954375in}}{\pgfqpoint{1.689375in}{2.964974in}}{\pgfqpoint{1.689375in}{2.976024in}}%
\pgfpathcurveto{\pgfqpoint{1.689375in}{2.987074in}}{\pgfqpoint{1.684985in}{2.997673in}}{\pgfqpoint{1.677171in}{3.005487in}}%
\pgfpathcurveto{\pgfqpoint{1.669357in}{3.013300in}}{\pgfqpoint{1.658758in}{3.017690in}}{\pgfqpoint{1.647708in}{3.017690in}}%
\pgfpathcurveto{\pgfqpoint{1.636658in}{3.017690in}}{\pgfqpoint{1.626059in}{3.013300in}}{\pgfqpoint{1.618245in}{3.005487in}}%
\pgfpathcurveto{\pgfqpoint{1.610432in}{2.997673in}}{\pgfqpoint{1.606042in}{2.987074in}}{\pgfqpoint{1.606042in}{2.976024in}}%
\pgfpathcurveto{\pgfqpoint{1.606042in}{2.964974in}}{\pgfqpoint{1.610432in}{2.954375in}}{\pgfqpoint{1.618245in}{2.946561in}}%
\pgfpathcurveto{\pgfqpoint{1.626059in}{2.938747in}}{\pgfqpoint{1.636658in}{2.934357in}}{\pgfqpoint{1.647708in}{2.934357in}}%
\pgfpathclose%
\pgfusepath{stroke,fill}%
\end{pgfscope}%
\begin{pgfscope}%
\pgfpathrectangle{\pgfqpoint{0.787074in}{0.548769in}}{\pgfqpoint{5.062926in}{3.102590in}}%
\pgfusepath{clip}%
\pgfsetbuttcap%
\pgfsetroundjoin%
\definecolor{currentfill}{rgb}{1.000000,0.498039,0.054902}%
\pgfsetfillcolor{currentfill}%
\pgfsetlinewidth{1.003750pt}%
\definecolor{currentstroke}{rgb}{1.000000,0.498039,0.054902}%
\pgfsetstrokecolor{currentstroke}%
\pgfsetdash{}{0pt}%
\pgfpathmoveto{\pgfqpoint{5.619867in}{3.327747in}}%
\pgfpathcurveto{\pgfqpoint{5.630917in}{3.327747in}}{\pgfqpoint{5.641516in}{3.332137in}}{\pgfqpoint{5.649330in}{3.339951in}}%
\pgfpathcurveto{\pgfqpoint{5.657143in}{3.347764in}}{\pgfqpoint{5.661534in}{3.358363in}}{\pgfqpoint{5.661534in}{3.369413in}}%
\pgfpathcurveto{\pgfqpoint{5.661534in}{3.380463in}}{\pgfqpoint{5.657143in}{3.391063in}}{\pgfqpoint{5.649330in}{3.398876in}}%
\pgfpathcurveto{\pgfqpoint{5.641516in}{3.406690in}}{\pgfqpoint{5.630917in}{3.411080in}}{\pgfqpoint{5.619867in}{3.411080in}}%
\pgfpathcurveto{\pgfqpoint{5.608817in}{3.411080in}}{\pgfqpoint{5.598218in}{3.406690in}}{\pgfqpoint{5.590404in}{3.398876in}}%
\pgfpathcurveto{\pgfqpoint{5.582591in}{3.391063in}}{\pgfqpoint{5.578200in}{3.380463in}}{\pgfqpoint{5.578200in}{3.369413in}}%
\pgfpathcurveto{\pgfqpoint{5.578200in}{3.358363in}}{\pgfqpoint{5.582591in}{3.347764in}}{\pgfqpoint{5.590404in}{3.339951in}}%
\pgfpathcurveto{\pgfqpoint{5.598218in}{3.332137in}}{\pgfqpoint{5.608817in}{3.327747in}}{\pgfqpoint{5.619867in}{3.327747in}}%
\pgfpathclose%
\pgfusepath{stroke,fill}%
\end{pgfscope}%
\begin{pgfscope}%
\pgfpathrectangle{\pgfqpoint{0.787074in}{0.548769in}}{\pgfqpoint{5.062926in}{3.102590in}}%
\pgfusepath{clip}%
\pgfsetbuttcap%
\pgfsetroundjoin%
\definecolor{currentfill}{rgb}{1.000000,0.498039,0.054902}%
\pgfsetfillcolor{currentfill}%
\pgfsetlinewidth{1.003750pt}%
\definecolor{currentstroke}{rgb}{1.000000,0.498039,0.054902}%
\pgfsetstrokecolor{currentstroke}%
\pgfsetdash{}{0pt}%
\pgfpathmoveto{\pgfqpoint{2.656510in}{2.872580in}}%
\pgfpathcurveto{\pgfqpoint{2.667561in}{2.872580in}}{\pgfqpoint{2.678160in}{2.876970in}}{\pgfqpoint{2.685973in}{2.884784in}}%
\pgfpathcurveto{\pgfqpoint{2.693787in}{2.892597in}}{\pgfqpoint{2.698177in}{2.903196in}}{\pgfqpoint{2.698177in}{2.914246in}}%
\pgfpathcurveto{\pgfqpoint{2.698177in}{2.925297in}}{\pgfqpoint{2.693787in}{2.935896in}}{\pgfqpoint{2.685973in}{2.943709in}}%
\pgfpathcurveto{\pgfqpoint{2.678160in}{2.951523in}}{\pgfqpoint{2.667561in}{2.955913in}}{\pgfqpoint{2.656510in}{2.955913in}}%
\pgfpathcurveto{\pgfqpoint{2.645460in}{2.955913in}}{\pgfqpoint{2.634861in}{2.951523in}}{\pgfqpoint{2.627048in}{2.943709in}}%
\pgfpathcurveto{\pgfqpoint{2.619234in}{2.935896in}}{\pgfqpoint{2.614844in}{2.925297in}}{\pgfqpoint{2.614844in}{2.914246in}}%
\pgfpathcurveto{\pgfqpoint{2.614844in}{2.903196in}}{\pgfqpoint{2.619234in}{2.892597in}}{\pgfqpoint{2.627048in}{2.884784in}}%
\pgfpathcurveto{\pgfqpoint{2.634861in}{2.876970in}}{\pgfqpoint{2.645460in}{2.872580in}}{\pgfqpoint{2.656510in}{2.872580in}}%
\pgfpathclose%
\pgfusepath{stroke,fill}%
\end{pgfscope}%
\begin{pgfscope}%
\pgfpathrectangle{\pgfqpoint{0.787074in}{0.548769in}}{\pgfqpoint{5.062926in}{3.102590in}}%
\pgfusepath{clip}%
\pgfsetbuttcap%
\pgfsetroundjoin%
\definecolor{currentfill}{rgb}{0.839216,0.152941,0.156863}%
\pgfsetfillcolor{currentfill}%
\pgfsetlinewidth{1.003750pt}%
\definecolor{currentstroke}{rgb}{0.839216,0.152941,0.156863}%
\pgfsetstrokecolor{currentstroke}%
\pgfsetdash{}{0pt}%
\pgfpathmoveto{\pgfqpoint{1.899909in}{2.752142in}}%
\pgfpathcurveto{\pgfqpoint{1.910959in}{2.752142in}}{\pgfqpoint{1.921558in}{2.756532in}}{\pgfqpoint{1.929372in}{2.764346in}}%
\pgfpathcurveto{\pgfqpoint{1.937185in}{2.772160in}}{\pgfqpoint{1.941575in}{2.782759in}}{\pgfqpoint{1.941575in}{2.793809in}}%
\pgfpathcurveto{\pgfqpoint{1.941575in}{2.804859in}}{\pgfqpoint{1.937185in}{2.815458in}}{\pgfqpoint{1.929372in}{2.823271in}}%
\pgfpathcurveto{\pgfqpoint{1.921558in}{2.831085in}}{\pgfqpoint{1.910959in}{2.835475in}}{\pgfqpoint{1.899909in}{2.835475in}}%
\pgfpathcurveto{\pgfqpoint{1.888859in}{2.835475in}}{\pgfqpoint{1.878260in}{2.831085in}}{\pgfqpoint{1.870446in}{2.823271in}}%
\pgfpathcurveto{\pgfqpoint{1.862632in}{2.815458in}}{\pgfqpoint{1.858242in}{2.804859in}}{\pgfqpoint{1.858242in}{2.793809in}}%
\pgfpathcurveto{\pgfqpoint{1.858242in}{2.782759in}}{\pgfqpoint{1.862632in}{2.772160in}}{\pgfqpoint{1.870446in}{2.764346in}}%
\pgfpathcurveto{\pgfqpoint{1.878260in}{2.756532in}}{\pgfqpoint{1.888859in}{2.752142in}}{\pgfqpoint{1.899909in}{2.752142in}}%
\pgfpathclose%
\pgfusepath{stroke,fill}%
\end{pgfscope}%
\begin{pgfscope}%
\pgfpathrectangle{\pgfqpoint{0.787074in}{0.548769in}}{\pgfqpoint{5.062926in}{3.102590in}}%
\pgfusepath{clip}%
\pgfsetbuttcap%
\pgfsetroundjoin%
\definecolor{currentfill}{rgb}{1.000000,0.498039,0.054902}%
\pgfsetfillcolor{currentfill}%
\pgfsetlinewidth{1.003750pt}%
\definecolor{currentstroke}{rgb}{1.000000,0.498039,0.054902}%
\pgfsetstrokecolor{currentstroke}%
\pgfsetdash{}{0pt}%
\pgfpathmoveto{\pgfqpoint{2.467360in}{2.067284in}}%
\pgfpathcurveto{\pgfqpoint{2.478410in}{2.067284in}}{\pgfqpoint{2.489009in}{2.071675in}}{\pgfqpoint{2.496823in}{2.079488in}}%
\pgfpathcurveto{\pgfqpoint{2.504636in}{2.087302in}}{\pgfqpoint{2.509027in}{2.097901in}}{\pgfqpoint{2.509027in}{2.108951in}}%
\pgfpathcurveto{\pgfqpoint{2.509027in}{2.120001in}}{\pgfqpoint{2.504636in}{2.130600in}}{\pgfqpoint{2.496823in}{2.138414in}}%
\pgfpathcurveto{\pgfqpoint{2.489009in}{2.146227in}}{\pgfqpoint{2.478410in}{2.150618in}}{\pgfqpoint{2.467360in}{2.150618in}}%
\pgfpathcurveto{\pgfqpoint{2.456310in}{2.150618in}}{\pgfqpoint{2.445711in}{2.146227in}}{\pgfqpoint{2.437897in}{2.138414in}}%
\pgfpathcurveto{\pgfqpoint{2.430084in}{2.130600in}}{\pgfqpoint{2.425693in}{2.120001in}}{\pgfqpoint{2.425693in}{2.108951in}}%
\pgfpathcurveto{\pgfqpoint{2.425693in}{2.097901in}}{\pgfqpoint{2.430084in}{2.087302in}}{\pgfqpoint{2.437897in}{2.079488in}}%
\pgfpathcurveto{\pgfqpoint{2.445711in}{2.071675in}}{\pgfqpoint{2.456310in}{2.067284in}}{\pgfqpoint{2.467360in}{2.067284in}}%
\pgfpathclose%
\pgfusepath{stroke,fill}%
\end{pgfscope}%
\begin{pgfscope}%
\pgfpathrectangle{\pgfqpoint{0.787074in}{0.548769in}}{\pgfqpoint{5.062926in}{3.102590in}}%
\pgfusepath{clip}%
\pgfsetbuttcap%
\pgfsetroundjoin%
\definecolor{currentfill}{rgb}{1.000000,0.498039,0.054902}%
\pgfsetfillcolor{currentfill}%
\pgfsetlinewidth{1.003750pt}%
\definecolor{currentstroke}{rgb}{1.000000,0.498039,0.054902}%
\pgfsetstrokecolor{currentstroke}%
\pgfsetdash{}{0pt}%
\pgfpathmoveto{\pgfqpoint{2.278210in}{3.097151in}}%
\pgfpathcurveto{\pgfqpoint{2.289260in}{3.097151in}}{\pgfqpoint{2.299859in}{3.101541in}}{\pgfqpoint{2.307672in}{3.109355in}}%
\pgfpathcurveto{\pgfqpoint{2.315486in}{3.117169in}}{\pgfqpoint{2.319876in}{3.127768in}}{\pgfqpoint{2.319876in}{3.138818in}}%
\pgfpathcurveto{\pgfqpoint{2.319876in}{3.149868in}}{\pgfqpoint{2.315486in}{3.160467in}}{\pgfqpoint{2.307672in}{3.168281in}}%
\pgfpathcurveto{\pgfqpoint{2.299859in}{3.176094in}}{\pgfqpoint{2.289260in}{3.180484in}}{\pgfqpoint{2.278210in}{3.180484in}}%
\pgfpathcurveto{\pgfqpoint{2.267159in}{3.180484in}}{\pgfqpoint{2.256560in}{3.176094in}}{\pgfqpoint{2.248747in}{3.168281in}}%
\pgfpathcurveto{\pgfqpoint{2.240933in}{3.160467in}}{\pgfqpoint{2.236543in}{3.149868in}}{\pgfqpoint{2.236543in}{3.138818in}}%
\pgfpathcurveto{\pgfqpoint{2.236543in}{3.127768in}}{\pgfqpoint{2.240933in}{3.117169in}}{\pgfqpoint{2.248747in}{3.109355in}}%
\pgfpathcurveto{\pgfqpoint{2.256560in}{3.101541in}}{\pgfqpoint{2.267159in}{3.097151in}}{\pgfqpoint{2.278210in}{3.097151in}}%
\pgfpathclose%
\pgfusepath{stroke,fill}%
\end{pgfscope}%
\begin{pgfscope}%
\pgfpathrectangle{\pgfqpoint{0.787074in}{0.548769in}}{\pgfqpoint{5.062926in}{3.102590in}}%
\pgfusepath{clip}%
\pgfsetbuttcap%
\pgfsetroundjoin%
\definecolor{currentfill}{rgb}{1.000000,0.498039,0.054902}%
\pgfsetfillcolor{currentfill}%
\pgfsetlinewidth{1.003750pt}%
\definecolor{currentstroke}{rgb}{1.000000,0.498039,0.054902}%
\pgfsetstrokecolor{currentstroke}%
\pgfsetdash{}{0pt}%
\pgfpathmoveto{\pgfqpoint{2.341260in}{2.976134in}}%
\pgfpathcurveto{\pgfqpoint{2.352310in}{2.976134in}}{\pgfqpoint{2.362909in}{2.980524in}}{\pgfqpoint{2.370723in}{2.988338in}}%
\pgfpathcurveto{\pgfqpoint{2.378536in}{2.996152in}}{\pgfqpoint{2.382926in}{3.006751in}}{\pgfqpoint{2.382926in}{3.017801in}}%
\pgfpathcurveto{\pgfqpoint{2.382926in}{3.028851in}}{\pgfqpoint{2.378536in}{3.039450in}}{\pgfqpoint{2.370723in}{3.047264in}}%
\pgfpathcurveto{\pgfqpoint{2.362909in}{3.055077in}}{\pgfqpoint{2.352310in}{3.059468in}}{\pgfqpoint{2.341260in}{3.059468in}}%
\pgfpathcurveto{\pgfqpoint{2.330210in}{3.059468in}}{\pgfqpoint{2.319611in}{3.055077in}}{\pgfqpoint{2.311797in}{3.047264in}}%
\pgfpathcurveto{\pgfqpoint{2.303983in}{3.039450in}}{\pgfqpoint{2.299593in}{3.028851in}}{\pgfqpoint{2.299593in}{3.017801in}}%
\pgfpathcurveto{\pgfqpoint{2.299593in}{3.006751in}}{\pgfqpoint{2.303983in}{2.996152in}}{\pgfqpoint{2.311797in}{2.988338in}}%
\pgfpathcurveto{\pgfqpoint{2.319611in}{2.980524in}}{\pgfqpoint{2.330210in}{2.976134in}}{\pgfqpoint{2.341260in}{2.976134in}}%
\pgfpathclose%
\pgfusepath{stroke,fill}%
\end{pgfscope}%
\begin{pgfscope}%
\pgfpathrectangle{\pgfqpoint{0.787074in}{0.548769in}}{\pgfqpoint{5.062926in}{3.102590in}}%
\pgfusepath{clip}%
\pgfsetbuttcap%
\pgfsetroundjoin%
\definecolor{currentfill}{rgb}{1.000000,0.498039,0.054902}%
\pgfsetfillcolor{currentfill}%
\pgfsetlinewidth{1.003750pt}%
\definecolor{currentstroke}{rgb}{1.000000,0.498039,0.054902}%
\pgfsetstrokecolor{currentstroke}%
\pgfsetdash{}{0pt}%
\pgfpathmoveto{\pgfqpoint{2.026009in}{2.882792in}}%
\pgfpathcurveto{\pgfqpoint{2.037059in}{2.882792in}}{\pgfqpoint{2.047658in}{2.887182in}}{\pgfqpoint{2.055472in}{2.894996in}}%
\pgfpathcurveto{\pgfqpoint{2.063285in}{2.902810in}}{\pgfqpoint{2.067676in}{2.913409in}}{\pgfqpoint{2.067676in}{2.924459in}}%
\pgfpathcurveto{\pgfqpoint{2.067676in}{2.935509in}}{\pgfqpoint{2.063285in}{2.946108in}}{\pgfqpoint{2.055472in}{2.953922in}}%
\pgfpathcurveto{\pgfqpoint{2.047658in}{2.961735in}}{\pgfqpoint{2.037059in}{2.966125in}}{\pgfqpoint{2.026009in}{2.966125in}}%
\pgfpathcurveto{\pgfqpoint{2.014959in}{2.966125in}}{\pgfqpoint{2.004360in}{2.961735in}}{\pgfqpoint{1.996546in}{2.953922in}}%
\pgfpathcurveto{\pgfqpoint{1.988733in}{2.946108in}}{\pgfqpoint{1.984342in}{2.935509in}}{\pgfqpoint{1.984342in}{2.924459in}}%
\pgfpathcurveto{\pgfqpoint{1.984342in}{2.913409in}}{\pgfqpoint{1.988733in}{2.902810in}}{\pgfqpoint{1.996546in}{2.894996in}}%
\pgfpathcurveto{\pgfqpoint{2.004360in}{2.887182in}}{\pgfqpoint{2.014959in}{2.882792in}}{\pgfqpoint{2.026009in}{2.882792in}}%
\pgfpathclose%
\pgfusepath{stroke,fill}%
\end{pgfscope}%
\begin{pgfscope}%
\pgfpathrectangle{\pgfqpoint{0.787074in}{0.548769in}}{\pgfqpoint{5.062926in}{3.102590in}}%
\pgfusepath{clip}%
\pgfsetbuttcap%
\pgfsetroundjoin%
\definecolor{currentfill}{rgb}{0.839216,0.152941,0.156863}%
\pgfsetfillcolor{currentfill}%
\pgfsetlinewidth{1.003750pt}%
\definecolor{currentstroke}{rgb}{0.839216,0.152941,0.156863}%
\pgfsetstrokecolor{currentstroke}%
\pgfsetdash{}{0pt}%
\pgfpathmoveto{\pgfqpoint{2.341260in}{3.334884in}}%
\pgfpathcurveto{\pgfqpoint{2.352310in}{3.334884in}}{\pgfqpoint{2.362909in}{3.339274in}}{\pgfqpoint{2.370723in}{3.347088in}}%
\pgfpathcurveto{\pgfqpoint{2.378536in}{3.354902in}}{\pgfqpoint{2.382926in}{3.365501in}}{\pgfqpoint{2.382926in}{3.376551in}}%
\pgfpathcurveto{\pgfqpoint{2.382926in}{3.387601in}}{\pgfqpoint{2.378536in}{3.398200in}}{\pgfqpoint{2.370723in}{3.406014in}}%
\pgfpathcurveto{\pgfqpoint{2.362909in}{3.413827in}}{\pgfqpoint{2.352310in}{3.418218in}}{\pgfqpoint{2.341260in}{3.418218in}}%
\pgfpathcurveto{\pgfqpoint{2.330210in}{3.418218in}}{\pgfqpoint{2.319611in}{3.413827in}}{\pgfqpoint{2.311797in}{3.406014in}}%
\pgfpathcurveto{\pgfqpoint{2.303983in}{3.398200in}}{\pgfqpoint{2.299593in}{3.387601in}}{\pgfqpoint{2.299593in}{3.376551in}}%
\pgfpathcurveto{\pgfqpoint{2.299593in}{3.365501in}}{\pgfqpoint{2.303983in}{3.354902in}}{\pgfqpoint{2.311797in}{3.347088in}}%
\pgfpathcurveto{\pgfqpoint{2.319611in}{3.339274in}}{\pgfqpoint{2.330210in}{3.334884in}}{\pgfqpoint{2.341260in}{3.334884in}}%
\pgfpathclose%
\pgfusepath{stroke,fill}%
\end{pgfscope}%
\begin{pgfscope}%
\pgfpathrectangle{\pgfqpoint{0.787074in}{0.548769in}}{\pgfqpoint{5.062926in}{3.102590in}}%
\pgfusepath{clip}%
\pgfsetbuttcap%
\pgfsetroundjoin%
\definecolor{currentfill}{rgb}{1.000000,0.498039,0.054902}%
\pgfsetfillcolor{currentfill}%
\pgfsetlinewidth{1.003750pt}%
\definecolor{currentstroke}{rgb}{1.000000,0.498039,0.054902}%
\pgfsetstrokecolor{currentstroke}%
\pgfsetdash{}{0pt}%
\pgfpathmoveto{\pgfqpoint{4.863265in}{3.000517in}}%
\pgfpathcurveto{\pgfqpoint{4.874315in}{3.000517in}}{\pgfqpoint{4.884914in}{3.004907in}}{\pgfqpoint{4.892728in}{3.012721in}}%
\pgfpathcurveto{\pgfqpoint{4.900542in}{3.020535in}}{\pgfqpoint{4.904932in}{3.031134in}}{\pgfqpoint{4.904932in}{3.042184in}}%
\pgfpathcurveto{\pgfqpoint{4.904932in}{3.053234in}}{\pgfqpoint{4.900542in}{3.063833in}}{\pgfqpoint{4.892728in}{3.071646in}}%
\pgfpathcurveto{\pgfqpoint{4.884914in}{3.079460in}}{\pgfqpoint{4.874315in}{3.083850in}}{\pgfqpoint{4.863265in}{3.083850in}}%
\pgfpathcurveto{\pgfqpoint{4.852215in}{3.083850in}}{\pgfqpoint{4.841616in}{3.079460in}}{\pgfqpoint{4.833803in}{3.071646in}}%
\pgfpathcurveto{\pgfqpoint{4.825989in}{3.063833in}}{\pgfqpoint{4.821599in}{3.053234in}}{\pgfqpoint{4.821599in}{3.042184in}}%
\pgfpathcurveto{\pgfqpoint{4.821599in}{3.031134in}}{\pgfqpoint{4.825989in}{3.020535in}}{\pgfqpoint{4.833803in}{3.012721in}}%
\pgfpathcurveto{\pgfqpoint{4.841616in}{3.004907in}}{\pgfqpoint{4.852215in}{3.000517in}}{\pgfqpoint{4.863265in}{3.000517in}}%
\pgfpathclose%
\pgfusepath{stroke,fill}%
\end{pgfscope}%
\begin{pgfscope}%
\pgfpathrectangle{\pgfqpoint{0.787074in}{0.548769in}}{\pgfqpoint{5.062926in}{3.102590in}}%
\pgfusepath{clip}%
\pgfsetbuttcap%
\pgfsetroundjoin%
\definecolor{currentfill}{rgb}{0.121569,0.466667,0.705882}%
\pgfsetfillcolor{currentfill}%
\pgfsetlinewidth{1.003750pt}%
\definecolor{currentstroke}{rgb}{0.121569,0.466667,0.705882}%
\pgfsetstrokecolor{currentstroke}%
\pgfsetdash{}{0pt}%
\pgfpathmoveto{\pgfqpoint{2.719561in}{2.441957in}}%
\pgfpathcurveto{\pgfqpoint{2.730611in}{2.441957in}}{\pgfqpoint{2.741210in}{2.446347in}}{\pgfqpoint{2.749023in}{2.454161in}}%
\pgfpathcurveto{\pgfqpoint{2.756837in}{2.461974in}}{\pgfqpoint{2.761227in}{2.472573in}}{\pgfqpoint{2.761227in}{2.483623in}}%
\pgfpathcurveto{\pgfqpoint{2.761227in}{2.494673in}}{\pgfqpoint{2.756837in}{2.505273in}}{\pgfqpoint{2.749023in}{2.513086in}}%
\pgfpathcurveto{\pgfqpoint{2.741210in}{2.520900in}}{\pgfqpoint{2.730611in}{2.525290in}}{\pgfqpoint{2.719561in}{2.525290in}}%
\pgfpathcurveto{\pgfqpoint{2.708510in}{2.525290in}}{\pgfqpoint{2.697911in}{2.520900in}}{\pgfqpoint{2.690098in}{2.513086in}}%
\pgfpathcurveto{\pgfqpoint{2.682284in}{2.505273in}}{\pgfqpoint{2.677894in}{2.494673in}}{\pgfqpoint{2.677894in}{2.483623in}}%
\pgfpathcurveto{\pgfqpoint{2.677894in}{2.472573in}}{\pgfqpoint{2.682284in}{2.461974in}}{\pgfqpoint{2.690098in}{2.454161in}}%
\pgfpathcurveto{\pgfqpoint{2.697911in}{2.446347in}}{\pgfqpoint{2.708510in}{2.441957in}}{\pgfqpoint{2.719561in}{2.441957in}}%
\pgfpathclose%
\pgfusepath{stroke,fill}%
\end{pgfscope}%
\begin{pgfscope}%
\pgfpathrectangle{\pgfqpoint{0.787074in}{0.548769in}}{\pgfqpoint{5.062926in}{3.102590in}}%
\pgfusepath{clip}%
\pgfsetbuttcap%
\pgfsetroundjoin%
\definecolor{currentfill}{rgb}{0.121569,0.466667,0.705882}%
\pgfsetfillcolor{currentfill}%
\pgfsetlinewidth{1.003750pt}%
\definecolor{currentstroke}{rgb}{0.121569,0.466667,0.705882}%
\pgfsetstrokecolor{currentstroke}%
\pgfsetdash{}{0pt}%
\pgfpathmoveto{\pgfqpoint{1.269407in}{0.919027in}}%
\pgfpathcurveto{\pgfqpoint{1.280458in}{0.919027in}}{\pgfqpoint{1.291057in}{0.923417in}}{\pgfqpoint{1.298870in}{0.931231in}}%
\pgfpathcurveto{\pgfqpoint{1.306684in}{0.939044in}}{\pgfqpoint{1.311074in}{0.949643in}}{\pgfqpoint{1.311074in}{0.960694in}}%
\pgfpathcurveto{\pgfqpoint{1.311074in}{0.971744in}}{\pgfqpoint{1.306684in}{0.982343in}}{\pgfqpoint{1.298870in}{0.990156in}}%
\pgfpathcurveto{\pgfqpoint{1.291057in}{0.997970in}}{\pgfqpoint{1.280458in}{1.002360in}}{\pgfqpoint{1.269407in}{1.002360in}}%
\pgfpathcurveto{\pgfqpoint{1.258357in}{1.002360in}}{\pgfqpoint{1.247758in}{0.997970in}}{\pgfqpoint{1.239945in}{0.990156in}}%
\pgfpathcurveto{\pgfqpoint{1.232131in}{0.982343in}}{\pgfqpoint{1.227741in}{0.971744in}}{\pgfqpoint{1.227741in}{0.960694in}}%
\pgfpathcurveto{\pgfqpoint{1.227741in}{0.949643in}}{\pgfqpoint{1.232131in}{0.939044in}}{\pgfqpoint{1.239945in}{0.931231in}}%
\pgfpathcurveto{\pgfqpoint{1.247758in}{0.923417in}}{\pgfqpoint{1.258357in}{0.919027in}}{\pgfqpoint{1.269407in}{0.919027in}}%
\pgfpathclose%
\pgfusepath{stroke,fill}%
\end{pgfscope}%
\begin{pgfscope}%
\pgfpathrectangle{\pgfqpoint{0.787074in}{0.548769in}}{\pgfqpoint{5.062926in}{3.102590in}}%
\pgfusepath{clip}%
\pgfsetbuttcap%
\pgfsetroundjoin%
\definecolor{currentfill}{rgb}{0.839216,0.152941,0.156863}%
\pgfsetfillcolor{currentfill}%
\pgfsetlinewidth{1.003750pt}%
\definecolor{currentstroke}{rgb}{0.839216,0.152941,0.156863}%
\pgfsetstrokecolor{currentstroke}%
\pgfsetdash{}{0pt}%
\pgfpathmoveto{\pgfqpoint{2.782611in}{3.248678in}}%
\pgfpathcurveto{\pgfqpoint{2.793661in}{3.248678in}}{\pgfqpoint{2.804260in}{3.253068in}}{\pgfqpoint{2.812074in}{3.260882in}}%
\pgfpathcurveto{\pgfqpoint{2.819887in}{3.268695in}}{\pgfqpoint{2.824277in}{3.279294in}}{\pgfqpoint{2.824277in}{3.290344in}}%
\pgfpathcurveto{\pgfqpoint{2.824277in}{3.301395in}}{\pgfqpoint{2.819887in}{3.311994in}}{\pgfqpoint{2.812074in}{3.319807in}}%
\pgfpathcurveto{\pgfqpoint{2.804260in}{3.327621in}}{\pgfqpoint{2.793661in}{3.332011in}}{\pgfqpoint{2.782611in}{3.332011in}}%
\pgfpathcurveto{\pgfqpoint{2.771561in}{3.332011in}}{\pgfqpoint{2.760962in}{3.327621in}}{\pgfqpoint{2.753148in}{3.319807in}}%
\pgfpathcurveto{\pgfqpoint{2.745334in}{3.311994in}}{\pgfqpoint{2.740944in}{3.301395in}}{\pgfqpoint{2.740944in}{3.290344in}}%
\pgfpathcurveto{\pgfqpoint{2.740944in}{3.279294in}}{\pgfqpoint{2.745334in}{3.268695in}}{\pgfqpoint{2.753148in}{3.260882in}}%
\pgfpathcurveto{\pgfqpoint{2.760962in}{3.253068in}}{\pgfqpoint{2.771561in}{3.248678in}}{\pgfqpoint{2.782611in}{3.248678in}}%
\pgfpathclose%
\pgfusepath{stroke,fill}%
\end{pgfscope}%
\begin{pgfscope}%
\pgfpathrectangle{\pgfqpoint{0.787074in}{0.548769in}}{\pgfqpoint{5.062926in}{3.102590in}}%
\pgfusepath{clip}%
\pgfsetbuttcap%
\pgfsetroundjoin%
\definecolor{currentfill}{rgb}{1.000000,0.498039,0.054902}%
\pgfsetfillcolor{currentfill}%
\pgfsetlinewidth{1.003750pt}%
\definecolor{currentstroke}{rgb}{1.000000,0.498039,0.054902}%
\pgfsetstrokecolor{currentstroke}%
\pgfsetdash{}{0pt}%
\pgfpathmoveto{\pgfqpoint{1.836859in}{2.631493in}}%
\pgfpathcurveto{\pgfqpoint{1.847909in}{2.631493in}}{\pgfqpoint{1.858508in}{2.635883in}}{\pgfqpoint{1.866321in}{2.643697in}}%
\pgfpathcurveto{\pgfqpoint{1.874135in}{2.651510in}}{\pgfqpoint{1.878525in}{2.662109in}}{\pgfqpoint{1.878525in}{2.673159in}}%
\pgfpathcurveto{\pgfqpoint{1.878525in}{2.684210in}}{\pgfqpoint{1.874135in}{2.694809in}}{\pgfqpoint{1.866321in}{2.702622in}}%
\pgfpathcurveto{\pgfqpoint{1.858508in}{2.710436in}}{\pgfqpoint{1.847909in}{2.714826in}}{\pgfqpoint{1.836859in}{2.714826in}}%
\pgfpathcurveto{\pgfqpoint{1.825809in}{2.714826in}}{\pgfqpoint{1.815209in}{2.710436in}}{\pgfqpoint{1.807396in}{2.702622in}}%
\pgfpathcurveto{\pgfqpoint{1.799582in}{2.694809in}}{\pgfqpoint{1.795192in}{2.684210in}}{\pgfqpoint{1.795192in}{2.673159in}}%
\pgfpathcurveto{\pgfqpoint{1.795192in}{2.662109in}}{\pgfqpoint{1.799582in}{2.651510in}}{\pgfqpoint{1.807396in}{2.643697in}}%
\pgfpathcurveto{\pgfqpoint{1.815209in}{2.635883in}}{\pgfqpoint{1.825809in}{2.631493in}}{\pgfqpoint{1.836859in}{2.631493in}}%
\pgfpathclose%
\pgfusepath{stroke,fill}%
\end{pgfscope}%
\begin{pgfscope}%
\pgfpathrectangle{\pgfqpoint{0.787074in}{0.548769in}}{\pgfqpoint{5.062926in}{3.102590in}}%
\pgfusepath{clip}%
\pgfsetbuttcap%
\pgfsetroundjoin%
\definecolor{currentfill}{rgb}{1.000000,0.498039,0.054902}%
\pgfsetfillcolor{currentfill}%
\pgfsetlinewidth{1.003750pt}%
\definecolor{currentstroke}{rgb}{1.000000,0.498039,0.054902}%
\pgfsetstrokecolor{currentstroke}%
\pgfsetdash{}{0pt}%
\pgfpathmoveto{\pgfqpoint{1.962959in}{2.514237in}}%
\pgfpathcurveto{\pgfqpoint{1.974009in}{2.514237in}}{\pgfqpoint{1.984608in}{2.518627in}}{\pgfqpoint{1.992422in}{2.526441in}}%
\pgfpathcurveto{\pgfqpoint{2.000235in}{2.534255in}}{\pgfqpoint{2.004626in}{2.544854in}}{\pgfqpoint{2.004626in}{2.555904in}}%
\pgfpathcurveto{\pgfqpoint{2.004626in}{2.566954in}}{\pgfqpoint{2.000235in}{2.577553in}}{\pgfqpoint{1.992422in}{2.585367in}}%
\pgfpathcurveto{\pgfqpoint{1.984608in}{2.593180in}}{\pgfqpoint{1.974009in}{2.597570in}}{\pgfqpoint{1.962959in}{2.597570in}}%
\pgfpathcurveto{\pgfqpoint{1.951909in}{2.597570in}}{\pgfqpoint{1.941310in}{2.593180in}}{\pgfqpoint{1.933496in}{2.585367in}}%
\pgfpathcurveto{\pgfqpoint{1.925683in}{2.577553in}}{\pgfqpoint{1.921292in}{2.566954in}}{\pgfqpoint{1.921292in}{2.555904in}}%
\pgfpathcurveto{\pgfqpoint{1.921292in}{2.544854in}}{\pgfqpoint{1.925683in}{2.534255in}}{\pgfqpoint{1.933496in}{2.526441in}}%
\pgfpathcurveto{\pgfqpoint{1.941310in}{2.518627in}}{\pgfqpoint{1.951909in}{2.514237in}}{\pgfqpoint{1.962959in}{2.514237in}}%
\pgfpathclose%
\pgfusepath{stroke,fill}%
\end{pgfscope}%
\begin{pgfscope}%
\pgfpathrectangle{\pgfqpoint{0.787074in}{0.548769in}}{\pgfqpoint{5.062926in}{3.102590in}}%
\pgfusepath{clip}%
\pgfsetbuttcap%
\pgfsetroundjoin%
\definecolor{currentfill}{rgb}{1.000000,0.498039,0.054902}%
\pgfsetfillcolor{currentfill}%
\pgfsetlinewidth{1.003750pt}%
\definecolor{currentstroke}{rgb}{1.000000,0.498039,0.054902}%
\pgfsetstrokecolor{currentstroke}%
\pgfsetdash{}{0pt}%
\pgfpathmoveto{\pgfqpoint{1.458558in}{2.630257in}}%
\pgfpathcurveto{\pgfqpoint{1.469608in}{2.630257in}}{\pgfqpoint{1.480207in}{2.634648in}}{\pgfqpoint{1.488021in}{2.642461in}}%
\pgfpathcurveto{\pgfqpoint{1.495834in}{2.650275in}}{\pgfqpoint{1.500224in}{2.660874in}}{\pgfqpoint{1.500224in}{2.671924in}}%
\pgfpathcurveto{\pgfqpoint{1.500224in}{2.682974in}}{\pgfqpoint{1.495834in}{2.693573in}}{\pgfqpoint{1.488021in}{2.701387in}}%
\pgfpathcurveto{\pgfqpoint{1.480207in}{2.709200in}}{\pgfqpoint{1.469608in}{2.713591in}}{\pgfqpoint{1.458558in}{2.713591in}}%
\pgfpathcurveto{\pgfqpoint{1.447508in}{2.713591in}}{\pgfqpoint{1.436909in}{2.709200in}}{\pgfqpoint{1.429095in}{2.701387in}}%
\pgfpathcurveto{\pgfqpoint{1.421281in}{2.693573in}}{\pgfqpoint{1.416891in}{2.682974in}}{\pgfqpoint{1.416891in}{2.671924in}}%
\pgfpathcurveto{\pgfqpoint{1.416891in}{2.660874in}}{\pgfqpoint{1.421281in}{2.650275in}}{\pgfqpoint{1.429095in}{2.642461in}}%
\pgfpathcurveto{\pgfqpoint{1.436909in}{2.634648in}}{\pgfqpoint{1.447508in}{2.630257in}}{\pgfqpoint{1.458558in}{2.630257in}}%
\pgfpathclose%
\pgfusepath{stroke,fill}%
\end{pgfscope}%
\begin{pgfscope}%
\pgfpathrectangle{\pgfqpoint{0.787074in}{0.548769in}}{\pgfqpoint{5.062926in}{3.102590in}}%
\pgfusepath{clip}%
\pgfsetbuttcap%
\pgfsetroundjoin%
\definecolor{currentfill}{rgb}{1.000000,0.498039,0.054902}%
\pgfsetfillcolor{currentfill}%
\pgfsetlinewidth{1.003750pt}%
\definecolor{currentstroke}{rgb}{1.000000,0.498039,0.054902}%
\pgfsetstrokecolor{currentstroke}%
\pgfsetdash{}{0pt}%
\pgfpathmoveto{\pgfqpoint{2.719561in}{3.109003in}}%
\pgfpathcurveto{\pgfqpoint{2.730611in}{3.109003in}}{\pgfqpoint{2.741210in}{3.113393in}}{\pgfqpoint{2.749023in}{3.121207in}}%
\pgfpathcurveto{\pgfqpoint{2.756837in}{3.129020in}}{\pgfqpoint{2.761227in}{3.139619in}}{\pgfqpoint{2.761227in}{3.150669in}}%
\pgfpathcurveto{\pgfqpoint{2.761227in}{3.161719in}}{\pgfqpoint{2.756837in}{3.172319in}}{\pgfqpoint{2.749023in}{3.180132in}}%
\pgfpathcurveto{\pgfqpoint{2.741210in}{3.187946in}}{\pgfqpoint{2.730611in}{3.192336in}}{\pgfqpoint{2.719561in}{3.192336in}}%
\pgfpathcurveto{\pgfqpoint{2.708510in}{3.192336in}}{\pgfqpoint{2.697911in}{3.187946in}}{\pgfqpoint{2.690098in}{3.180132in}}%
\pgfpathcurveto{\pgfqpoint{2.682284in}{3.172319in}}{\pgfqpoint{2.677894in}{3.161719in}}{\pgfqpoint{2.677894in}{3.150669in}}%
\pgfpathcurveto{\pgfqpoint{2.677894in}{3.139619in}}{\pgfqpoint{2.682284in}{3.129020in}}{\pgfqpoint{2.690098in}{3.121207in}}%
\pgfpathcurveto{\pgfqpoint{2.697911in}{3.113393in}}{\pgfqpoint{2.708510in}{3.109003in}}{\pgfqpoint{2.719561in}{3.109003in}}%
\pgfpathclose%
\pgfusepath{stroke,fill}%
\end{pgfscope}%
\begin{pgfscope}%
\pgfpathrectangle{\pgfqpoint{0.787074in}{0.548769in}}{\pgfqpoint{5.062926in}{3.102590in}}%
\pgfusepath{clip}%
\pgfsetbuttcap%
\pgfsetroundjoin%
\definecolor{currentfill}{rgb}{1.000000,0.498039,0.054902}%
\pgfsetfillcolor{currentfill}%
\pgfsetlinewidth{1.003750pt}%
\definecolor{currentstroke}{rgb}{1.000000,0.498039,0.054902}%
\pgfsetstrokecolor{currentstroke}%
\pgfsetdash{}{0pt}%
\pgfpathmoveto{\pgfqpoint{2.530410in}{2.413769in}}%
\pgfpathcurveto{\pgfqpoint{2.541460in}{2.413769in}}{\pgfqpoint{2.552059in}{2.418159in}}{\pgfqpoint{2.559873in}{2.425973in}}%
\pgfpathcurveto{\pgfqpoint{2.567687in}{2.433787in}}{\pgfqpoint{2.572077in}{2.444386in}}{\pgfqpoint{2.572077in}{2.455436in}}%
\pgfpathcurveto{\pgfqpoint{2.572077in}{2.466486in}}{\pgfqpoint{2.567687in}{2.477085in}}{\pgfqpoint{2.559873in}{2.484899in}}%
\pgfpathcurveto{\pgfqpoint{2.552059in}{2.492712in}}{\pgfqpoint{2.541460in}{2.497102in}}{\pgfqpoint{2.530410in}{2.497102in}}%
\pgfpathcurveto{\pgfqpoint{2.519360in}{2.497102in}}{\pgfqpoint{2.508761in}{2.492712in}}{\pgfqpoint{2.500947in}{2.484899in}}%
\pgfpathcurveto{\pgfqpoint{2.493134in}{2.477085in}}{\pgfqpoint{2.488744in}{2.466486in}}{\pgfqpoint{2.488744in}{2.455436in}}%
\pgfpathcurveto{\pgfqpoint{2.488744in}{2.444386in}}{\pgfqpoint{2.493134in}{2.433787in}}{\pgfqpoint{2.500947in}{2.425973in}}%
\pgfpathcurveto{\pgfqpoint{2.508761in}{2.418159in}}{\pgfqpoint{2.519360in}{2.413769in}}{\pgfqpoint{2.530410in}{2.413769in}}%
\pgfpathclose%
\pgfusepath{stroke,fill}%
\end{pgfscope}%
\begin{pgfscope}%
\pgfpathrectangle{\pgfqpoint{0.787074in}{0.548769in}}{\pgfqpoint{5.062926in}{3.102590in}}%
\pgfusepath{clip}%
\pgfsetbuttcap%
\pgfsetroundjoin%
\definecolor{currentfill}{rgb}{0.121569,0.466667,0.705882}%
\pgfsetfillcolor{currentfill}%
\pgfsetlinewidth{1.003750pt}%
\definecolor{currentstroke}{rgb}{0.121569,0.466667,0.705882}%
\pgfsetstrokecolor{currentstroke}%
\pgfsetdash{}{0pt}%
\pgfpathmoveto{\pgfqpoint{3.160912in}{2.714693in}}%
\pgfpathcurveto{\pgfqpoint{3.171962in}{2.714693in}}{\pgfqpoint{3.182561in}{2.719083in}}{\pgfqpoint{3.190374in}{2.726897in}}%
\pgfpathcurveto{\pgfqpoint{3.198188in}{2.734711in}}{\pgfqpoint{3.202578in}{2.745310in}}{\pgfqpoint{3.202578in}{2.756360in}}%
\pgfpathcurveto{\pgfqpoint{3.202578in}{2.767410in}}{\pgfqpoint{3.198188in}{2.778009in}}{\pgfqpoint{3.190374in}{2.785822in}}%
\pgfpathcurveto{\pgfqpoint{3.182561in}{2.793636in}}{\pgfqpoint{3.171962in}{2.798026in}}{\pgfqpoint{3.160912in}{2.798026in}}%
\pgfpathcurveto{\pgfqpoint{3.149861in}{2.798026in}}{\pgfqpoint{3.139262in}{2.793636in}}{\pgfqpoint{3.131449in}{2.785822in}}%
\pgfpathcurveto{\pgfqpoint{3.123635in}{2.778009in}}{\pgfqpoint{3.119245in}{2.767410in}}{\pgfqpoint{3.119245in}{2.756360in}}%
\pgfpathcurveto{\pgfqpoint{3.119245in}{2.745310in}}{\pgfqpoint{3.123635in}{2.734711in}}{\pgfqpoint{3.131449in}{2.726897in}}%
\pgfpathcurveto{\pgfqpoint{3.139262in}{2.719083in}}{\pgfqpoint{3.149861in}{2.714693in}}{\pgfqpoint{3.160912in}{2.714693in}}%
\pgfpathclose%
\pgfusepath{stroke,fill}%
\end{pgfscope}%
\begin{pgfscope}%
\pgfpathrectangle{\pgfqpoint{0.787074in}{0.548769in}}{\pgfqpoint{5.062926in}{3.102590in}}%
\pgfusepath{clip}%
\pgfsetbuttcap%
\pgfsetroundjoin%
\definecolor{currentfill}{rgb}{1.000000,0.498039,0.054902}%
\pgfsetfillcolor{currentfill}%
\pgfsetlinewidth{1.003750pt}%
\definecolor{currentstroke}{rgb}{1.000000,0.498039,0.054902}%
\pgfsetstrokecolor{currentstroke}%
\pgfsetdash{}{0pt}%
\pgfpathmoveto{\pgfqpoint{1.836859in}{2.455826in}}%
\pgfpathcurveto{\pgfqpoint{1.847909in}{2.455826in}}{\pgfqpoint{1.858508in}{2.460216in}}{\pgfqpoint{1.866321in}{2.468030in}}%
\pgfpathcurveto{\pgfqpoint{1.874135in}{2.475843in}}{\pgfqpoint{1.878525in}{2.486442in}}{\pgfqpoint{1.878525in}{2.497492in}}%
\pgfpathcurveto{\pgfqpoint{1.878525in}{2.508542in}}{\pgfqpoint{1.874135in}{2.519141in}}{\pgfqpoint{1.866321in}{2.526955in}}%
\pgfpathcurveto{\pgfqpoint{1.858508in}{2.534769in}}{\pgfqpoint{1.847909in}{2.539159in}}{\pgfqpoint{1.836859in}{2.539159in}}%
\pgfpathcurveto{\pgfqpoint{1.825809in}{2.539159in}}{\pgfqpoint{1.815209in}{2.534769in}}{\pgfqpoint{1.807396in}{2.526955in}}%
\pgfpathcurveto{\pgfqpoint{1.799582in}{2.519141in}}{\pgfqpoint{1.795192in}{2.508542in}}{\pgfqpoint{1.795192in}{2.497492in}}%
\pgfpathcurveto{\pgfqpoint{1.795192in}{2.486442in}}{\pgfqpoint{1.799582in}{2.475843in}}{\pgfqpoint{1.807396in}{2.468030in}}%
\pgfpathcurveto{\pgfqpoint{1.815209in}{2.460216in}}{\pgfqpoint{1.825809in}{2.455826in}}{\pgfqpoint{1.836859in}{2.455826in}}%
\pgfpathclose%
\pgfusepath{stroke,fill}%
\end{pgfscope}%
\begin{pgfscope}%
\pgfpathrectangle{\pgfqpoint{0.787074in}{0.548769in}}{\pgfqpoint{5.062926in}{3.102590in}}%
\pgfusepath{clip}%
\pgfsetbuttcap%
\pgfsetroundjoin%
\definecolor{currentfill}{rgb}{1.000000,0.498039,0.054902}%
\pgfsetfillcolor{currentfill}%
\pgfsetlinewidth{1.003750pt}%
\definecolor{currentstroke}{rgb}{1.000000,0.498039,0.054902}%
\pgfsetstrokecolor{currentstroke}%
\pgfsetdash{}{0pt}%
\pgfpathmoveto{\pgfqpoint{2.278210in}{2.530243in}}%
\pgfpathcurveto{\pgfqpoint{2.289260in}{2.530243in}}{\pgfqpoint{2.299859in}{2.534634in}}{\pgfqpoint{2.307672in}{2.542447in}}%
\pgfpathcurveto{\pgfqpoint{2.315486in}{2.550261in}}{\pgfqpoint{2.319876in}{2.560860in}}{\pgfqpoint{2.319876in}{2.571910in}}%
\pgfpathcurveto{\pgfqpoint{2.319876in}{2.582960in}}{\pgfqpoint{2.315486in}{2.593559in}}{\pgfqpoint{2.307672in}{2.601373in}}%
\pgfpathcurveto{\pgfqpoint{2.299859in}{2.609186in}}{\pgfqpoint{2.289260in}{2.613577in}}{\pgfqpoint{2.278210in}{2.613577in}}%
\pgfpathcurveto{\pgfqpoint{2.267159in}{2.613577in}}{\pgfqpoint{2.256560in}{2.609186in}}{\pgfqpoint{2.248747in}{2.601373in}}%
\pgfpathcurveto{\pgfqpoint{2.240933in}{2.593559in}}{\pgfqpoint{2.236543in}{2.582960in}}{\pgfqpoint{2.236543in}{2.571910in}}%
\pgfpathcurveto{\pgfqpoint{2.236543in}{2.560860in}}{\pgfqpoint{2.240933in}{2.550261in}}{\pgfqpoint{2.248747in}{2.542447in}}%
\pgfpathcurveto{\pgfqpoint{2.256560in}{2.534634in}}{\pgfqpoint{2.267159in}{2.530243in}}{\pgfqpoint{2.278210in}{2.530243in}}%
\pgfpathclose%
\pgfusepath{stroke,fill}%
\end{pgfscope}%
\begin{pgfscope}%
\pgfpathrectangle{\pgfqpoint{0.787074in}{0.548769in}}{\pgfqpoint{5.062926in}{3.102590in}}%
\pgfusepath{clip}%
\pgfsetbuttcap%
\pgfsetroundjoin%
\definecolor{currentfill}{rgb}{1.000000,0.498039,0.054902}%
\pgfsetfillcolor{currentfill}%
\pgfsetlinewidth{1.003750pt}%
\definecolor{currentstroke}{rgb}{1.000000,0.498039,0.054902}%
\pgfsetstrokecolor{currentstroke}%
\pgfsetdash{}{0pt}%
\pgfpathmoveto{\pgfqpoint{1.458558in}{2.652528in}}%
\pgfpathcurveto{\pgfqpoint{1.469608in}{2.652528in}}{\pgfqpoint{1.480207in}{2.656919in}}{\pgfqpoint{1.488021in}{2.664732in}}%
\pgfpathcurveto{\pgfqpoint{1.495834in}{2.672546in}}{\pgfqpoint{1.500224in}{2.683145in}}{\pgfqpoint{1.500224in}{2.694195in}}%
\pgfpathcurveto{\pgfqpoint{1.500224in}{2.705245in}}{\pgfqpoint{1.495834in}{2.715844in}}{\pgfqpoint{1.488021in}{2.723658in}}%
\pgfpathcurveto{\pgfqpoint{1.480207in}{2.731471in}}{\pgfqpoint{1.469608in}{2.735862in}}{\pgfqpoint{1.458558in}{2.735862in}}%
\pgfpathcurveto{\pgfqpoint{1.447508in}{2.735862in}}{\pgfqpoint{1.436909in}{2.731471in}}{\pgfqpoint{1.429095in}{2.723658in}}%
\pgfpathcurveto{\pgfqpoint{1.421281in}{2.715844in}}{\pgfqpoint{1.416891in}{2.705245in}}{\pgfqpoint{1.416891in}{2.694195in}}%
\pgfpathcurveto{\pgfqpoint{1.416891in}{2.683145in}}{\pgfqpoint{1.421281in}{2.672546in}}{\pgfqpoint{1.429095in}{2.664732in}}%
\pgfpathcurveto{\pgfqpoint{1.436909in}{2.656919in}}{\pgfqpoint{1.447508in}{2.652528in}}{\pgfqpoint{1.458558in}{2.652528in}}%
\pgfpathclose%
\pgfusepath{stroke,fill}%
\end{pgfscope}%
\begin{pgfscope}%
\pgfpathrectangle{\pgfqpoint{0.787074in}{0.548769in}}{\pgfqpoint{5.062926in}{3.102590in}}%
\pgfusepath{clip}%
\pgfsetbuttcap%
\pgfsetroundjoin%
\definecolor{currentfill}{rgb}{1.000000,0.498039,0.054902}%
\pgfsetfillcolor{currentfill}%
\pgfsetlinewidth{1.003750pt}%
\definecolor{currentstroke}{rgb}{1.000000,0.498039,0.054902}%
\pgfsetstrokecolor{currentstroke}%
\pgfsetdash{}{0pt}%
\pgfpathmoveto{\pgfqpoint{2.404310in}{2.518244in}}%
\pgfpathcurveto{\pgfqpoint{2.415360in}{2.518244in}}{\pgfqpoint{2.425959in}{2.522634in}}{\pgfqpoint{2.433773in}{2.530448in}}%
\pgfpathcurveto{\pgfqpoint{2.441586in}{2.538261in}}{\pgfqpoint{2.445977in}{2.548860in}}{\pgfqpoint{2.445977in}{2.559910in}}%
\pgfpathcurveto{\pgfqpoint{2.445977in}{2.570961in}}{\pgfqpoint{2.441586in}{2.581560in}}{\pgfqpoint{2.433773in}{2.589373in}}%
\pgfpathcurveto{\pgfqpoint{2.425959in}{2.597187in}}{\pgfqpoint{2.415360in}{2.601577in}}{\pgfqpoint{2.404310in}{2.601577in}}%
\pgfpathcurveto{\pgfqpoint{2.393260in}{2.601577in}}{\pgfqpoint{2.382661in}{2.597187in}}{\pgfqpoint{2.374847in}{2.589373in}}%
\pgfpathcurveto{\pgfqpoint{2.367033in}{2.581560in}}{\pgfqpoint{2.362643in}{2.570961in}}{\pgfqpoint{2.362643in}{2.559910in}}%
\pgfpathcurveto{\pgfqpoint{2.362643in}{2.548860in}}{\pgfqpoint{2.367033in}{2.538261in}}{\pgfqpoint{2.374847in}{2.530448in}}%
\pgfpathcurveto{\pgfqpoint{2.382661in}{2.522634in}}{\pgfqpoint{2.393260in}{2.518244in}}{\pgfqpoint{2.404310in}{2.518244in}}%
\pgfpathclose%
\pgfusepath{stroke,fill}%
\end{pgfscope}%
\begin{pgfscope}%
\pgfpathrectangle{\pgfqpoint{0.787074in}{0.548769in}}{\pgfqpoint{5.062926in}{3.102590in}}%
\pgfusepath{clip}%
\pgfsetbuttcap%
\pgfsetroundjoin%
\definecolor{currentfill}{rgb}{0.839216,0.152941,0.156863}%
\pgfsetfillcolor{currentfill}%
\pgfsetlinewidth{1.003750pt}%
\definecolor{currentstroke}{rgb}{0.839216,0.152941,0.156863}%
\pgfsetstrokecolor{currentstroke}%
\pgfsetdash{}{0pt}%
\pgfpathmoveto{\pgfqpoint{2.719561in}{2.503837in}}%
\pgfpathcurveto{\pgfqpoint{2.730611in}{2.503837in}}{\pgfqpoint{2.741210in}{2.508227in}}{\pgfqpoint{2.749023in}{2.516041in}}%
\pgfpathcurveto{\pgfqpoint{2.756837in}{2.523854in}}{\pgfqpoint{2.761227in}{2.534453in}}{\pgfqpoint{2.761227in}{2.545503in}}%
\pgfpathcurveto{\pgfqpoint{2.761227in}{2.556554in}}{\pgfqpoint{2.756837in}{2.567153in}}{\pgfqpoint{2.749023in}{2.574966in}}%
\pgfpathcurveto{\pgfqpoint{2.741210in}{2.582780in}}{\pgfqpoint{2.730611in}{2.587170in}}{\pgfqpoint{2.719561in}{2.587170in}}%
\pgfpathcurveto{\pgfqpoint{2.708510in}{2.587170in}}{\pgfqpoint{2.697911in}{2.582780in}}{\pgfqpoint{2.690098in}{2.574966in}}%
\pgfpathcurveto{\pgfqpoint{2.682284in}{2.567153in}}{\pgfqpoint{2.677894in}{2.556554in}}{\pgfqpoint{2.677894in}{2.545503in}}%
\pgfpathcurveto{\pgfqpoint{2.677894in}{2.534453in}}{\pgfqpoint{2.682284in}{2.523854in}}{\pgfqpoint{2.690098in}{2.516041in}}%
\pgfpathcurveto{\pgfqpoint{2.697911in}{2.508227in}}{\pgfqpoint{2.708510in}{2.503837in}}{\pgfqpoint{2.719561in}{2.503837in}}%
\pgfpathclose%
\pgfusepath{stroke,fill}%
\end{pgfscope}%
\begin{pgfscope}%
\pgfpathrectangle{\pgfqpoint{0.787074in}{0.548769in}}{\pgfqpoint{5.062926in}{3.102590in}}%
\pgfusepath{clip}%
\pgfsetbuttcap%
\pgfsetroundjoin%
\definecolor{currentfill}{rgb}{1.000000,0.498039,0.054902}%
\pgfsetfillcolor{currentfill}%
\pgfsetlinewidth{1.003750pt}%
\definecolor{currentstroke}{rgb}{1.000000,0.498039,0.054902}%
\pgfsetstrokecolor{currentstroke}%
\pgfsetdash{}{0pt}%
\pgfpathmoveto{\pgfqpoint{2.278210in}{2.836137in}}%
\pgfpathcurveto{\pgfqpoint{2.289260in}{2.836137in}}{\pgfqpoint{2.299859in}{2.840528in}}{\pgfqpoint{2.307672in}{2.848341in}}%
\pgfpathcurveto{\pgfqpoint{2.315486in}{2.856155in}}{\pgfqpoint{2.319876in}{2.866754in}}{\pgfqpoint{2.319876in}{2.877804in}}%
\pgfpathcurveto{\pgfqpoint{2.319876in}{2.888854in}}{\pgfqpoint{2.315486in}{2.899453in}}{\pgfqpoint{2.307672in}{2.907267in}}%
\pgfpathcurveto{\pgfqpoint{2.299859in}{2.915080in}}{\pgfqpoint{2.289260in}{2.919471in}}{\pgfqpoint{2.278210in}{2.919471in}}%
\pgfpathcurveto{\pgfqpoint{2.267159in}{2.919471in}}{\pgfqpoint{2.256560in}{2.915080in}}{\pgfqpoint{2.248747in}{2.907267in}}%
\pgfpathcurveto{\pgfqpoint{2.240933in}{2.899453in}}{\pgfqpoint{2.236543in}{2.888854in}}{\pgfqpoint{2.236543in}{2.877804in}}%
\pgfpathcurveto{\pgfqpoint{2.236543in}{2.866754in}}{\pgfqpoint{2.240933in}{2.856155in}}{\pgfqpoint{2.248747in}{2.848341in}}%
\pgfpathcurveto{\pgfqpoint{2.256560in}{2.840528in}}{\pgfqpoint{2.267159in}{2.836137in}}{\pgfqpoint{2.278210in}{2.836137in}}%
\pgfpathclose%
\pgfusepath{stroke,fill}%
\end{pgfscope}%
\begin{pgfscope}%
\pgfpathrectangle{\pgfqpoint{0.787074in}{0.548769in}}{\pgfqpoint{5.062926in}{3.102590in}}%
\pgfusepath{clip}%
\pgfsetbuttcap%
\pgfsetroundjoin%
\definecolor{currentfill}{rgb}{0.839216,0.152941,0.156863}%
\pgfsetfillcolor{currentfill}%
\pgfsetlinewidth{1.003750pt}%
\definecolor{currentstroke}{rgb}{0.839216,0.152941,0.156863}%
\pgfsetstrokecolor{currentstroke}%
\pgfsetdash{}{0pt}%
\pgfpathmoveto{\pgfqpoint{1.647708in}{2.834106in}}%
\pgfpathcurveto{\pgfqpoint{1.658758in}{2.834106in}}{\pgfqpoint{1.669357in}{2.838496in}}{\pgfqpoint{1.677171in}{2.846310in}}%
\pgfpathcurveto{\pgfqpoint{1.684985in}{2.854123in}}{\pgfqpoint{1.689375in}{2.864722in}}{\pgfqpoint{1.689375in}{2.875772in}}%
\pgfpathcurveto{\pgfqpoint{1.689375in}{2.886823in}}{\pgfqpoint{1.684985in}{2.897422in}}{\pgfqpoint{1.677171in}{2.905235in}}%
\pgfpathcurveto{\pgfqpoint{1.669357in}{2.913049in}}{\pgfqpoint{1.658758in}{2.917439in}}{\pgfqpoint{1.647708in}{2.917439in}}%
\pgfpathcurveto{\pgfqpoint{1.636658in}{2.917439in}}{\pgfqpoint{1.626059in}{2.913049in}}{\pgfqpoint{1.618245in}{2.905235in}}%
\pgfpathcurveto{\pgfqpoint{1.610432in}{2.897422in}}{\pgfqpoint{1.606042in}{2.886823in}}{\pgfqpoint{1.606042in}{2.875772in}}%
\pgfpathcurveto{\pgfqpoint{1.606042in}{2.864722in}}{\pgfqpoint{1.610432in}{2.854123in}}{\pgfqpoint{1.618245in}{2.846310in}}%
\pgfpathcurveto{\pgfqpoint{1.626059in}{2.838496in}}{\pgfqpoint{1.636658in}{2.834106in}}{\pgfqpoint{1.647708in}{2.834106in}}%
\pgfpathclose%
\pgfusepath{stroke,fill}%
\end{pgfscope}%
\begin{pgfscope}%
\pgfpathrectangle{\pgfqpoint{0.787074in}{0.548769in}}{\pgfqpoint{5.062926in}{3.102590in}}%
\pgfusepath{clip}%
\pgfsetbuttcap%
\pgfsetroundjoin%
\definecolor{currentfill}{rgb}{0.121569,0.466667,0.705882}%
\pgfsetfillcolor{currentfill}%
\pgfsetlinewidth{1.003750pt}%
\definecolor{currentstroke}{rgb}{0.121569,0.466667,0.705882}%
\pgfsetstrokecolor{currentstroke}%
\pgfsetdash{}{0pt}%
\pgfpathmoveto{\pgfqpoint{3.034811in}{2.400262in}}%
\pgfpathcurveto{\pgfqpoint{3.045861in}{2.400262in}}{\pgfqpoint{3.056460in}{2.404652in}}{\pgfqpoint{3.064274in}{2.412465in}}%
\pgfpathcurveto{\pgfqpoint{3.072088in}{2.420279in}}{\pgfqpoint{3.076478in}{2.430878in}}{\pgfqpoint{3.076478in}{2.441928in}}%
\pgfpathcurveto{\pgfqpoint{3.076478in}{2.452978in}}{\pgfqpoint{3.072088in}{2.463577in}}{\pgfqpoint{3.064274in}{2.471391in}}%
\pgfpathcurveto{\pgfqpoint{3.056460in}{2.479205in}}{\pgfqpoint{3.045861in}{2.483595in}}{\pgfqpoint{3.034811in}{2.483595in}}%
\pgfpathcurveto{\pgfqpoint{3.023761in}{2.483595in}}{\pgfqpoint{3.013162in}{2.479205in}}{\pgfqpoint{3.005349in}{2.471391in}}%
\pgfpathcurveto{\pgfqpoint{2.997535in}{2.463577in}}{\pgfqpoint{2.993145in}{2.452978in}}{\pgfqpoint{2.993145in}{2.441928in}}%
\pgfpathcurveto{\pgfqpoint{2.993145in}{2.430878in}}{\pgfqpoint{2.997535in}{2.420279in}}{\pgfqpoint{3.005349in}{2.412465in}}%
\pgfpathcurveto{\pgfqpoint{3.013162in}{2.404652in}}{\pgfqpoint{3.023761in}{2.400262in}}{\pgfqpoint{3.034811in}{2.400262in}}%
\pgfpathclose%
\pgfusepath{stroke,fill}%
\end{pgfscope}%
\begin{pgfscope}%
\pgfpathrectangle{\pgfqpoint{0.787074in}{0.548769in}}{\pgfqpoint{5.062926in}{3.102590in}}%
\pgfusepath{clip}%
\pgfsetbuttcap%
\pgfsetroundjoin%
\definecolor{currentfill}{rgb}{1.000000,0.498039,0.054902}%
\pgfsetfillcolor{currentfill}%
\pgfsetlinewidth{1.003750pt}%
\definecolor{currentstroke}{rgb}{1.000000,0.498039,0.054902}%
\pgfsetstrokecolor{currentstroke}%
\pgfsetdash{}{0pt}%
\pgfpathmoveto{\pgfqpoint{1.521608in}{2.515312in}}%
\pgfpathcurveto{\pgfqpoint{1.532658in}{2.515312in}}{\pgfqpoint{1.543257in}{2.519702in}}{\pgfqpoint{1.551071in}{2.527516in}}%
\pgfpathcurveto{\pgfqpoint{1.558884in}{2.535330in}}{\pgfqpoint{1.563275in}{2.545929in}}{\pgfqpoint{1.563275in}{2.556979in}}%
\pgfpathcurveto{\pgfqpoint{1.563275in}{2.568029in}}{\pgfqpoint{1.558884in}{2.578628in}}{\pgfqpoint{1.551071in}{2.586442in}}%
\pgfpathcurveto{\pgfqpoint{1.543257in}{2.594255in}}{\pgfqpoint{1.532658in}{2.598645in}}{\pgfqpoint{1.521608in}{2.598645in}}%
\pgfpathcurveto{\pgfqpoint{1.510558in}{2.598645in}}{\pgfqpoint{1.499959in}{2.594255in}}{\pgfqpoint{1.492145in}{2.586442in}}%
\pgfpathcurveto{\pgfqpoint{1.484332in}{2.578628in}}{\pgfqpoint{1.479941in}{2.568029in}}{\pgfqpoint{1.479941in}{2.556979in}}%
\pgfpathcurveto{\pgfqpoint{1.479941in}{2.545929in}}{\pgfqpoint{1.484332in}{2.535330in}}{\pgfqpoint{1.492145in}{2.527516in}}%
\pgfpathcurveto{\pgfqpoint{1.499959in}{2.519702in}}{\pgfqpoint{1.510558in}{2.515312in}}{\pgfqpoint{1.521608in}{2.515312in}}%
\pgfpathclose%
\pgfusepath{stroke,fill}%
\end{pgfscope}%
\begin{pgfscope}%
\pgfpathrectangle{\pgfqpoint{0.787074in}{0.548769in}}{\pgfqpoint{5.062926in}{3.102590in}}%
\pgfusepath{clip}%
\pgfsetbuttcap%
\pgfsetroundjoin%
\definecolor{currentfill}{rgb}{1.000000,0.498039,0.054902}%
\pgfsetfillcolor{currentfill}%
\pgfsetlinewidth{1.003750pt}%
\definecolor{currentstroke}{rgb}{1.000000,0.498039,0.054902}%
\pgfsetstrokecolor{currentstroke}%
\pgfsetdash{}{0pt}%
\pgfpathmoveto{\pgfqpoint{2.530410in}{2.940138in}}%
\pgfpathcurveto{\pgfqpoint{2.541460in}{2.940138in}}{\pgfqpoint{2.552059in}{2.944529in}}{\pgfqpoint{2.559873in}{2.952342in}}%
\pgfpathcurveto{\pgfqpoint{2.567687in}{2.960156in}}{\pgfqpoint{2.572077in}{2.970755in}}{\pgfqpoint{2.572077in}{2.981805in}}%
\pgfpathcurveto{\pgfqpoint{2.572077in}{2.992855in}}{\pgfqpoint{2.567687in}{3.003454in}}{\pgfqpoint{2.559873in}{3.011268in}}%
\pgfpathcurveto{\pgfqpoint{2.552059in}{3.019082in}}{\pgfqpoint{2.541460in}{3.023472in}}{\pgfqpoint{2.530410in}{3.023472in}}%
\pgfpathcurveto{\pgfqpoint{2.519360in}{3.023472in}}{\pgfqpoint{2.508761in}{3.019082in}}{\pgfqpoint{2.500947in}{3.011268in}}%
\pgfpathcurveto{\pgfqpoint{2.493134in}{3.003454in}}{\pgfqpoint{2.488744in}{2.992855in}}{\pgfqpoint{2.488744in}{2.981805in}}%
\pgfpathcurveto{\pgfqpoint{2.488744in}{2.970755in}}{\pgfqpoint{2.493134in}{2.960156in}}{\pgfqpoint{2.500947in}{2.952342in}}%
\pgfpathcurveto{\pgfqpoint{2.508761in}{2.944529in}}{\pgfqpoint{2.519360in}{2.940138in}}{\pgfqpoint{2.530410in}{2.940138in}}%
\pgfpathclose%
\pgfusepath{stroke,fill}%
\end{pgfscope}%
\begin{pgfscope}%
\pgfpathrectangle{\pgfqpoint{0.787074in}{0.548769in}}{\pgfqpoint{5.062926in}{3.102590in}}%
\pgfusepath{clip}%
\pgfsetbuttcap%
\pgfsetroundjoin%
\definecolor{currentfill}{rgb}{1.000000,0.498039,0.054902}%
\pgfsetfillcolor{currentfill}%
\pgfsetlinewidth{1.003750pt}%
\definecolor{currentstroke}{rgb}{1.000000,0.498039,0.054902}%
\pgfsetstrokecolor{currentstroke}%
\pgfsetdash{}{0pt}%
\pgfpathmoveto{\pgfqpoint{1.143307in}{3.117335in}}%
\pgfpathcurveto{\pgfqpoint{1.154357in}{3.117335in}}{\pgfqpoint{1.164956in}{3.121725in}}{\pgfqpoint{1.172770in}{3.129539in}}%
\pgfpathcurveto{\pgfqpoint{1.180584in}{3.137352in}}{\pgfqpoint{1.184974in}{3.147951in}}{\pgfqpoint{1.184974in}{3.159001in}}%
\pgfpathcurveto{\pgfqpoint{1.184974in}{3.170052in}}{\pgfqpoint{1.180584in}{3.180651in}}{\pgfqpoint{1.172770in}{3.188464in}}%
\pgfpathcurveto{\pgfqpoint{1.164956in}{3.196278in}}{\pgfqpoint{1.154357in}{3.200668in}}{\pgfqpoint{1.143307in}{3.200668in}}%
\pgfpathcurveto{\pgfqpoint{1.132257in}{3.200668in}}{\pgfqpoint{1.121658in}{3.196278in}}{\pgfqpoint{1.113844in}{3.188464in}}%
\pgfpathcurveto{\pgfqpoint{1.106031in}{3.180651in}}{\pgfqpoint{1.101640in}{3.170052in}}{\pgfqpoint{1.101640in}{3.159001in}}%
\pgfpathcurveto{\pgfqpoint{1.101640in}{3.147951in}}{\pgfqpoint{1.106031in}{3.137352in}}{\pgfqpoint{1.113844in}{3.129539in}}%
\pgfpathcurveto{\pgfqpoint{1.121658in}{3.121725in}}{\pgfqpoint{1.132257in}{3.117335in}}{\pgfqpoint{1.143307in}{3.117335in}}%
\pgfpathclose%
\pgfusepath{stroke,fill}%
\end{pgfscope}%
\begin{pgfscope}%
\pgfpathrectangle{\pgfqpoint{0.787074in}{0.548769in}}{\pgfqpoint{5.062926in}{3.102590in}}%
\pgfusepath{clip}%
\pgfsetbuttcap%
\pgfsetroundjoin%
\definecolor{currentfill}{rgb}{1.000000,0.498039,0.054902}%
\pgfsetfillcolor{currentfill}%
\pgfsetlinewidth{1.003750pt}%
\definecolor{currentstroke}{rgb}{1.000000,0.498039,0.054902}%
\pgfsetstrokecolor{currentstroke}%
\pgfsetdash{}{0pt}%
\pgfpathmoveto{\pgfqpoint{2.719561in}{3.017356in}}%
\pgfpathcurveto{\pgfqpoint{2.730611in}{3.017356in}}{\pgfqpoint{2.741210in}{3.021746in}}{\pgfqpoint{2.749023in}{3.029559in}}%
\pgfpathcurveto{\pgfqpoint{2.756837in}{3.037373in}}{\pgfqpoint{2.761227in}{3.047972in}}{\pgfqpoint{2.761227in}{3.059022in}}%
\pgfpathcurveto{\pgfqpoint{2.761227in}{3.070072in}}{\pgfqpoint{2.756837in}{3.080671in}}{\pgfqpoint{2.749023in}{3.088485in}}%
\pgfpathcurveto{\pgfqpoint{2.741210in}{3.096299in}}{\pgfqpoint{2.730611in}{3.100689in}}{\pgfqpoint{2.719561in}{3.100689in}}%
\pgfpathcurveto{\pgfqpoint{2.708510in}{3.100689in}}{\pgfqpoint{2.697911in}{3.096299in}}{\pgfqpoint{2.690098in}{3.088485in}}%
\pgfpathcurveto{\pgfqpoint{2.682284in}{3.080671in}}{\pgfqpoint{2.677894in}{3.070072in}}{\pgfqpoint{2.677894in}{3.059022in}}%
\pgfpathcurveto{\pgfqpoint{2.677894in}{3.047972in}}{\pgfqpoint{2.682284in}{3.037373in}}{\pgfqpoint{2.690098in}{3.029559in}}%
\pgfpathcurveto{\pgfqpoint{2.697911in}{3.021746in}}{\pgfqpoint{2.708510in}{3.017356in}}{\pgfqpoint{2.719561in}{3.017356in}}%
\pgfpathclose%
\pgfusepath{stroke,fill}%
\end{pgfscope}%
\begin{pgfscope}%
\pgfpathrectangle{\pgfqpoint{0.787074in}{0.548769in}}{\pgfqpoint{5.062926in}{3.102590in}}%
\pgfusepath{clip}%
\pgfsetbuttcap%
\pgfsetroundjoin%
\definecolor{currentfill}{rgb}{1.000000,0.498039,0.054902}%
\pgfsetfillcolor{currentfill}%
\pgfsetlinewidth{1.003750pt}%
\definecolor{currentstroke}{rgb}{1.000000,0.498039,0.054902}%
\pgfsetstrokecolor{currentstroke}%
\pgfsetdash{}{0pt}%
\pgfpathmoveto{\pgfqpoint{2.026009in}{1.775158in}}%
\pgfpathcurveto{\pgfqpoint{2.037059in}{1.775158in}}{\pgfqpoint{2.047658in}{1.779548in}}{\pgfqpoint{2.055472in}{1.787362in}}%
\pgfpathcurveto{\pgfqpoint{2.063285in}{1.795175in}}{\pgfqpoint{2.067676in}{1.805774in}}{\pgfqpoint{2.067676in}{1.816825in}}%
\pgfpathcurveto{\pgfqpoint{2.067676in}{1.827875in}}{\pgfqpoint{2.063285in}{1.838474in}}{\pgfqpoint{2.055472in}{1.846287in}}%
\pgfpathcurveto{\pgfqpoint{2.047658in}{1.854101in}}{\pgfqpoint{2.037059in}{1.858491in}}{\pgfqpoint{2.026009in}{1.858491in}}%
\pgfpathcurveto{\pgfqpoint{2.014959in}{1.858491in}}{\pgfqpoint{2.004360in}{1.854101in}}{\pgfqpoint{1.996546in}{1.846287in}}%
\pgfpathcurveto{\pgfqpoint{1.988733in}{1.838474in}}{\pgfqpoint{1.984342in}{1.827875in}}{\pgfqpoint{1.984342in}{1.816825in}}%
\pgfpathcurveto{\pgfqpoint{1.984342in}{1.805774in}}{\pgfqpoint{1.988733in}{1.795175in}}{\pgfqpoint{1.996546in}{1.787362in}}%
\pgfpathcurveto{\pgfqpoint{2.004360in}{1.779548in}}{\pgfqpoint{2.014959in}{1.775158in}}{\pgfqpoint{2.026009in}{1.775158in}}%
\pgfpathclose%
\pgfusepath{stroke,fill}%
\end{pgfscope}%
\begin{pgfscope}%
\pgfpathrectangle{\pgfqpoint{0.787074in}{0.548769in}}{\pgfqpoint{5.062926in}{3.102590in}}%
\pgfusepath{clip}%
\pgfsetbuttcap%
\pgfsetroundjoin%
\definecolor{currentfill}{rgb}{1.000000,0.498039,0.054902}%
\pgfsetfillcolor{currentfill}%
\pgfsetlinewidth{1.003750pt}%
\definecolor{currentstroke}{rgb}{1.000000,0.498039,0.054902}%
\pgfsetstrokecolor{currentstroke}%
\pgfsetdash{}{0pt}%
\pgfpathmoveto{\pgfqpoint{2.908711in}{3.272382in}}%
\pgfpathcurveto{\pgfqpoint{2.919761in}{3.272382in}}{\pgfqpoint{2.930360in}{3.276772in}}{\pgfqpoint{2.938174in}{3.284586in}}%
\pgfpathcurveto{\pgfqpoint{2.945987in}{3.292399in}}{\pgfqpoint{2.950378in}{3.302998in}}{\pgfqpoint{2.950378in}{3.314049in}}%
\pgfpathcurveto{\pgfqpoint{2.950378in}{3.325099in}}{\pgfqpoint{2.945987in}{3.335698in}}{\pgfqpoint{2.938174in}{3.343511in}}%
\pgfpathcurveto{\pgfqpoint{2.930360in}{3.351325in}}{\pgfqpoint{2.919761in}{3.355715in}}{\pgfqpoint{2.908711in}{3.355715in}}%
\pgfpathcurveto{\pgfqpoint{2.897661in}{3.355715in}}{\pgfqpoint{2.887062in}{3.351325in}}{\pgfqpoint{2.879248in}{3.343511in}}%
\pgfpathcurveto{\pgfqpoint{2.871435in}{3.335698in}}{\pgfqpoint{2.867044in}{3.325099in}}{\pgfqpoint{2.867044in}{3.314049in}}%
\pgfpathcurveto{\pgfqpoint{2.867044in}{3.302998in}}{\pgfqpoint{2.871435in}{3.292399in}}{\pgfqpoint{2.879248in}{3.284586in}}%
\pgfpathcurveto{\pgfqpoint{2.887062in}{3.276772in}}{\pgfqpoint{2.897661in}{3.272382in}}{\pgfqpoint{2.908711in}{3.272382in}}%
\pgfpathclose%
\pgfusepath{stroke,fill}%
\end{pgfscope}%
\begin{pgfscope}%
\pgfpathrectangle{\pgfqpoint{0.787074in}{0.548769in}}{\pgfqpoint{5.062926in}{3.102590in}}%
\pgfusepath{clip}%
\pgfsetbuttcap%
\pgfsetroundjoin%
\definecolor{currentfill}{rgb}{0.121569,0.466667,0.705882}%
\pgfsetfillcolor{currentfill}%
\pgfsetlinewidth{1.003750pt}%
\definecolor{currentstroke}{rgb}{0.121569,0.466667,0.705882}%
\pgfsetstrokecolor{currentstroke}%
\pgfsetdash{}{0pt}%
\pgfpathmoveto{\pgfqpoint{1.773809in}{0.676982in}}%
\pgfpathcurveto{\pgfqpoint{1.784859in}{0.676982in}}{\pgfqpoint{1.795458in}{0.681372in}}{\pgfqpoint{1.803271in}{0.689186in}}%
\pgfpathcurveto{\pgfqpoint{1.811085in}{0.696999in}}{\pgfqpoint{1.815475in}{0.707598in}}{\pgfqpoint{1.815475in}{0.718649in}}%
\pgfpathcurveto{\pgfqpoint{1.815475in}{0.729699in}}{\pgfqpoint{1.811085in}{0.740298in}}{\pgfqpoint{1.803271in}{0.748111in}}%
\pgfpathcurveto{\pgfqpoint{1.795458in}{0.755925in}}{\pgfqpoint{1.784859in}{0.760315in}}{\pgfqpoint{1.773809in}{0.760315in}}%
\pgfpathcurveto{\pgfqpoint{1.762758in}{0.760315in}}{\pgfqpoint{1.752159in}{0.755925in}}{\pgfqpoint{1.744346in}{0.748111in}}%
\pgfpathcurveto{\pgfqpoint{1.736532in}{0.740298in}}{\pgfqpoint{1.732142in}{0.729699in}}{\pgfqpoint{1.732142in}{0.718649in}}%
\pgfpathcurveto{\pgfqpoint{1.732142in}{0.707598in}}{\pgfqpoint{1.736532in}{0.696999in}}{\pgfqpoint{1.744346in}{0.689186in}}%
\pgfpathcurveto{\pgfqpoint{1.752159in}{0.681372in}}{\pgfqpoint{1.762758in}{0.676982in}}{\pgfqpoint{1.773809in}{0.676982in}}%
\pgfpathclose%
\pgfusepath{stroke,fill}%
\end{pgfscope}%
\begin{pgfscope}%
\pgfpathrectangle{\pgfqpoint{0.787074in}{0.548769in}}{\pgfqpoint{5.062926in}{3.102590in}}%
\pgfusepath{clip}%
\pgfsetbuttcap%
\pgfsetroundjoin%
\definecolor{currentfill}{rgb}{0.839216,0.152941,0.156863}%
\pgfsetfillcolor{currentfill}%
\pgfsetlinewidth{1.003750pt}%
\definecolor{currentstroke}{rgb}{0.839216,0.152941,0.156863}%
\pgfsetstrokecolor{currentstroke}%
\pgfsetdash{}{0pt}%
\pgfpathmoveto{\pgfqpoint{1.836859in}{3.256315in}}%
\pgfpathcurveto{\pgfqpoint{1.847909in}{3.256315in}}{\pgfqpoint{1.858508in}{3.260705in}}{\pgfqpoint{1.866321in}{3.268518in}}%
\pgfpathcurveto{\pgfqpoint{1.874135in}{3.276332in}}{\pgfqpoint{1.878525in}{3.286931in}}{\pgfqpoint{1.878525in}{3.297981in}}%
\pgfpathcurveto{\pgfqpoint{1.878525in}{3.309031in}}{\pgfqpoint{1.874135in}{3.319630in}}{\pgfqpoint{1.866321in}{3.327444in}}%
\pgfpathcurveto{\pgfqpoint{1.858508in}{3.335258in}}{\pgfqpoint{1.847909in}{3.339648in}}{\pgfqpoint{1.836859in}{3.339648in}}%
\pgfpathcurveto{\pgfqpoint{1.825809in}{3.339648in}}{\pgfqpoint{1.815209in}{3.335258in}}{\pgfqpoint{1.807396in}{3.327444in}}%
\pgfpathcurveto{\pgfqpoint{1.799582in}{3.319630in}}{\pgfqpoint{1.795192in}{3.309031in}}{\pgfqpoint{1.795192in}{3.297981in}}%
\pgfpathcurveto{\pgfqpoint{1.795192in}{3.286931in}}{\pgfqpoint{1.799582in}{3.276332in}}{\pgfqpoint{1.807396in}{3.268518in}}%
\pgfpathcurveto{\pgfqpoint{1.815209in}{3.260705in}}{\pgfqpoint{1.825809in}{3.256315in}}{\pgfqpoint{1.836859in}{3.256315in}}%
\pgfpathclose%
\pgfusepath{stroke,fill}%
\end{pgfscope}%
\begin{pgfscope}%
\pgfpathrectangle{\pgfqpoint{0.787074in}{0.548769in}}{\pgfqpoint{5.062926in}{3.102590in}}%
\pgfusepath{clip}%
\pgfsetbuttcap%
\pgfsetroundjoin%
\definecolor{currentfill}{rgb}{1.000000,0.498039,0.054902}%
\pgfsetfillcolor{currentfill}%
\pgfsetlinewidth{1.003750pt}%
\definecolor{currentstroke}{rgb}{1.000000,0.498039,0.054902}%
\pgfsetstrokecolor{currentstroke}%
\pgfsetdash{}{0pt}%
\pgfpathmoveto{\pgfqpoint{1.017207in}{2.697852in}}%
\pgfpathcurveto{\pgfqpoint{1.028257in}{2.697852in}}{\pgfqpoint{1.038856in}{2.702242in}}{\pgfqpoint{1.046670in}{2.710056in}}%
\pgfpathcurveto{\pgfqpoint{1.054483in}{2.717869in}}{\pgfqpoint{1.058874in}{2.728468in}}{\pgfqpoint{1.058874in}{2.739518in}}%
\pgfpathcurveto{\pgfqpoint{1.058874in}{2.750568in}}{\pgfqpoint{1.054483in}{2.761167in}}{\pgfqpoint{1.046670in}{2.768981in}}%
\pgfpathcurveto{\pgfqpoint{1.038856in}{2.776795in}}{\pgfqpoint{1.028257in}{2.781185in}}{\pgfqpoint{1.017207in}{2.781185in}}%
\pgfpathcurveto{\pgfqpoint{1.006157in}{2.781185in}}{\pgfqpoint{0.995558in}{2.776795in}}{\pgfqpoint{0.987744in}{2.768981in}}%
\pgfpathcurveto{\pgfqpoint{0.979930in}{2.761167in}}{\pgfqpoint{0.975540in}{2.750568in}}{\pgfqpoint{0.975540in}{2.739518in}}%
\pgfpathcurveto{\pgfqpoint{0.975540in}{2.728468in}}{\pgfqpoint{0.979930in}{2.717869in}}{\pgfqpoint{0.987744in}{2.710056in}}%
\pgfpathcurveto{\pgfqpoint{0.995558in}{2.702242in}}{\pgfqpoint{1.006157in}{2.697852in}}{\pgfqpoint{1.017207in}{2.697852in}}%
\pgfpathclose%
\pgfusepath{stroke,fill}%
\end{pgfscope}%
\begin{pgfscope}%
\pgfpathrectangle{\pgfqpoint{0.787074in}{0.548769in}}{\pgfqpoint{5.062926in}{3.102590in}}%
\pgfusepath{clip}%
\pgfsetbuttcap%
\pgfsetroundjoin%
\definecolor{currentfill}{rgb}{0.121569,0.466667,0.705882}%
\pgfsetfillcolor{currentfill}%
\pgfsetlinewidth{1.003750pt}%
\definecolor{currentstroke}{rgb}{0.121569,0.466667,0.705882}%
\pgfsetstrokecolor{currentstroke}%
\pgfsetdash{}{0pt}%
\pgfpathmoveto{\pgfqpoint{1.269407in}{1.421237in}}%
\pgfpathcurveto{\pgfqpoint{1.280458in}{1.421237in}}{\pgfqpoint{1.291057in}{1.425628in}}{\pgfqpoint{1.298870in}{1.433441in}}%
\pgfpathcurveto{\pgfqpoint{1.306684in}{1.441255in}}{\pgfqpoint{1.311074in}{1.451854in}}{\pgfqpoint{1.311074in}{1.462904in}}%
\pgfpathcurveto{\pgfqpoint{1.311074in}{1.473954in}}{\pgfqpoint{1.306684in}{1.484553in}}{\pgfqpoint{1.298870in}{1.492367in}}%
\pgfpathcurveto{\pgfqpoint{1.291057in}{1.500180in}}{\pgfqpoint{1.280458in}{1.504571in}}{\pgfqpoint{1.269407in}{1.504571in}}%
\pgfpathcurveto{\pgfqpoint{1.258357in}{1.504571in}}{\pgfqpoint{1.247758in}{1.500180in}}{\pgfqpoint{1.239945in}{1.492367in}}%
\pgfpathcurveto{\pgfqpoint{1.232131in}{1.484553in}}{\pgfqpoint{1.227741in}{1.473954in}}{\pgfqpoint{1.227741in}{1.462904in}}%
\pgfpathcurveto{\pgfqpoint{1.227741in}{1.451854in}}{\pgfqpoint{1.232131in}{1.441255in}}{\pgfqpoint{1.239945in}{1.433441in}}%
\pgfpathcurveto{\pgfqpoint{1.247758in}{1.425628in}}{\pgfqpoint{1.258357in}{1.421237in}}{\pgfqpoint{1.269407in}{1.421237in}}%
\pgfpathclose%
\pgfusepath{stroke,fill}%
\end{pgfscope}%
\begin{pgfscope}%
\pgfpathrectangle{\pgfqpoint{0.787074in}{0.548769in}}{\pgfqpoint{5.062926in}{3.102590in}}%
\pgfusepath{clip}%
\pgfsetbuttcap%
\pgfsetroundjoin%
\definecolor{currentfill}{rgb}{0.121569,0.466667,0.705882}%
\pgfsetfillcolor{currentfill}%
\pgfsetlinewidth{1.003750pt}%
\definecolor{currentstroke}{rgb}{0.121569,0.466667,0.705882}%
\pgfsetstrokecolor{currentstroke}%
\pgfsetdash{}{0pt}%
\pgfpathmoveto{\pgfqpoint{2.215159in}{1.410498in}}%
\pgfpathcurveto{\pgfqpoint{2.226210in}{1.410498in}}{\pgfqpoint{2.236809in}{1.414888in}}{\pgfqpoint{2.244622in}{1.422702in}}%
\pgfpathcurveto{\pgfqpoint{2.252436in}{1.430515in}}{\pgfqpoint{2.256826in}{1.441114in}}{\pgfqpoint{2.256826in}{1.452164in}}%
\pgfpathcurveto{\pgfqpoint{2.256826in}{1.463214in}}{\pgfqpoint{2.252436in}{1.473813in}}{\pgfqpoint{2.244622in}{1.481627in}}%
\pgfpathcurveto{\pgfqpoint{2.236809in}{1.489441in}}{\pgfqpoint{2.226210in}{1.493831in}}{\pgfqpoint{2.215159in}{1.493831in}}%
\pgfpathcurveto{\pgfqpoint{2.204109in}{1.493831in}}{\pgfqpoint{2.193510in}{1.489441in}}{\pgfqpoint{2.185697in}{1.481627in}}%
\pgfpathcurveto{\pgfqpoint{2.177883in}{1.473813in}}{\pgfqpoint{2.173493in}{1.463214in}}{\pgfqpoint{2.173493in}{1.452164in}}%
\pgfpathcurveto{\pgfqpoint{2.173493in}{1.441114in}}{\pgfqpoint{2.177883in}{1.430515in}}{\pgfqpoint{2.185697in}{1.422702in}}%
\pgfpathcurveto{\pgfqpoint{2.193510in}{1.414888in}}{\pgfqpoint{2.204109in}{1.410498in}}{\pgfqpoint{2.215159in}{1.410498in}}%
\pgfpathclose%
\pgfusepath{stroke,fill}%
\end{pgfscope}%
\begin{pgfscope}%
\pgfpathrectangle{\pgfqpoint{0.787074in}{0.548769in}}{\pgfqpoint{5.062926in}{3.102590in}}%
\pgfusepath{clip}%
\pgfsetbuttcap%
\pgfsetroundjoin%
\definecolor{currentfill}{rgb}{0.121569,0.466667,0.705882}%
\pgfsetfillcolor{currentfill}%
\pgfsetlinewidth{1.003750pt}%
\definecolor{currentstroke}{rgb}{0.121569,0.466667,0.705882}%
\pgfsetstrokecolor{currentstroke}%
\pgfsetdash{}{0pt}%
\pgfpathmoveto{\pgfqpoint{2.152109in}{1.127209in}}%
\pgfpathcurveto{\pgfqpoint{2.163159in}{1.127209in}}{\pgfqpoint{2.173759in}{1.131599in}}{\pgfqpoint{2.181572in}{1.139413in}}%
\pgfpathcurveto{\pgfqpoint{2.189386in}{1.147226in}}{\pgfqpoint{2.193776in}{1.157825in}}{\pgfqpoint{2.193776in}{1.168876in}}%
\pgfpathcurveto{\pgfqpoint{2.193776in}{1.179926in}}{\pgfqpoint{2.189386in}{1.190525in}}{\pgfqpoint{2.181572in}{1.198338in}}%
\pgfpathcurveto{\pgfqpoint{2.173759in}{1.206152in}}{\pgfqpoint{2.163159in}{1.210542in}}{\pgfqpoint{2.152109in}{1.210542in}}%
\pgfpathcurveto{\pgfqpoint{2.141059in}{1.210542in}}{\pgfqpoint{2.130460in}{1.206152in}}{\pgfqpoint{2.122647in}{1.198338in}}%
\pgfpathcurveto{\pgfqpoint{2.114833in}{1.190525in}}{\pgfqpoint{2.110443in}{1.179926in}}{\pgfqpoint{2.110443in}{1.168876in}}%
\pgfpathcurveto{\pgfqpoint{2.110443in}{1.157825in}}{\pgfqpoint{2.114833in}{1.147226in}}{\pgfqpoint{2.122647in}{1.139413in}}%
\pgfpathcurveto{\pgfqpoint{2.130460in}{1.131599in}}{\pgfqpoint{2.141059in}{1.127209in}}{\pgfqpoint{2.152109in}{1.127209in}}%
\pgfpathclose%
\pgfusepath{stroke,fill}%
\end{pgfscope}%
\begin{pgfscope}%
\pgfpathrectangle{\pgfqpoint{0.787074in}{0.548769in}}{\pgfqpoint{5.062926in}{3.102590in}}%
\pgfusepath{clip}%
\pgfsetbuttcap%
\pgfsetroundjoin%
\definecolor{currentfill}{rgb}{0.121569,0.466667,0.705882}%
\pgfsetfillcolor{currentfill}%
\pgfsetlinewidth{1.003750pt}%
\definecolor{currentstroke}{rgb}{0.121569,0.466667,0.705882}%
\pgfsetstrokecolor{currentstroke}%
\pgfsetdash{}{0pt}%
\pgfpathmoveto{\pgfqpoint{3.097861in}{0.894035in}}%
\pgfpathcurveto{\pgfqpoint{3.108912in}{0.894035in}}{\pgfqpoint{3.119511in}{0.898425in}}{\pgfqpoint{3.127324in}{0.906239in}}%
\pgfpathcurveto{\pgfqpoint{3.135138in}{0.914052in}}{\pgfqpoint{3.139528in}{0.924651in}}{\pgfqpoint{3.139528in}{0.935701in}}%
\pgfpathcurveto{\pgfqpoint{3.139528in}{0.946752in}}{\pgfqpoint{3.135138in}{0.957351in}}{\pgfqpoint{3.127324in}{0.965164in}}%
\pgfpathcurveto{\pgfqpoint{3.119511in}{0.972978in}}{\pgfqpoint{3.108912in}{0.977368in}}{\pgfqpoint{3.097861in}{0.977368in}}%
\pgfpathcurveto{\pgfqpoint{3.086811in}{0.977368in}}{\pgfqpoint{3.076212in}{0.972978in}}{\pgfqpoint{3.068399in}{0.965164in}}%
\pgfpathcurveto{\pgfqpoint{3.060585in}{0.957351in}}{\pgfqpoint{3.056195in}{0.946752in}}{\pgfqpoint{3.056195in}{0.935701in}}%
\pgfpathcurveto{\pgfqpoint{3.056195in}{0.924651in}}{\pgfqpoint{3.060585in}{0.914052in}}{\pgfqpoint{3.068399in}{0.906239in}}%
\pgfpathcurveto{\pgfqpoint{3.076212in}{0.898425in}}{\pgfqpoint{3.086811in}{0.894035in}}{\pgfqpoint{3.097861in}{0.894035in}}%
\pgfpathclose%
\pgfusepath{stroke,fill}%
\end{pgfscope}%
\begin{pgfscope}%
\pgfpathrectangle{\pgfqpoint{0.787074in}{0.548769in}}{\pgfqpoint{5.062926in}{3.102590in}}%
\pgfusepath{clip}%
\pgfsetbuttcap%
\pgfsetroundjoin%
\definecolor{currentfill}{rgb}{0.839216,0.152941,0.156863}%
\pgfsetfillcolor{currentfill}%
\pgfsetlinewidth{1.003750pt}%
\definecolor{currentstroke}{rgb}{0.839216,0.152941,0.156863}%
\pgfsetstrokecolor{currentstroke}%
\pgfsetdash{}{0pt}%
\pgfpathmoveto{\pgfqpoint{1.395508in}{3.187559in}}%
\pgfpathcurveto{\pgfqpoint{1.406558in}{3.187559in}}{\pgfqpoint{1.417157in}{3.191949in}}{\pgfqpoint{1.424970in}{3.199763in}}%
\pgfpathcurveto{\pgfqpoint{1.432784in}{3.207576in}}{\pgfqpoint{1.437174in}{3.218175in}}{\pgfqpoint{1.437174in}{3.229226in}}%
\pgfpathcurveto{\pgfqpoint{1.437174in}{3.240276in}}{\pgfqpoint{1.432784in}{3.250875in}}{\pgfqpoint{1.424970in}{3.258688in}}%
\pgfpathcurveto{\pgfqpoint{1.417157in}{3.266502in}}{\pgfqpoint{1.406558in}{3.270892in}}{\pgfqpoint{1.395508in}{3.270892in}}%
\pgfpathcurveto{\pgfqpoint{1.384458in}{3.270892in}}{\pgfqpoint{1.373859in}{3.266502in}}{\pgfqpoint{1.366045in}{3.258688in}}%
\pgfpathcurveto{\pgfqpoint{1.358231in}{3.250875in}}{\pgfqpoint{1.353841in}{3.240276in}}{\pgfqpoint{1.353841in}{3.229226in}}%
\pgfpathcurveto{\pgfqpoint{1.353841in}{3.218175in}}{\pgfqpoint{1.358231in}{3.207576in}}{\pgfqpoint{1.366045in}{3.199763in}}%
\pgfpathcurveto{\pgfqpoint{1.373859in}{3.191949in}}{\pgfqpoint{1.384458in}{3.187559in}}{\pgfqpoint{1.395508in}{3.187559in}}%
\pgfpathclose%
\pgfusepath{stroke,fill}%
\end{pgfscope}%
\begin{pgfscope}%
\pgfsetbuttcap%
\pgfsetroundjoin%
\definecolor{currentfill}{rgb}{0.000000,0.000000,0.000000}%
\pgfsetfillcolor{currentfill}%
\pgfsetlinewidth{0.803000pt}%
\definecolor{currentstroke}{rgb}{0.000000,0.000000,0.000000}%
\pgfsetstrokecolor{currentstroke}%
\pgfsetdash{}{0pt}%
\pgfsys@defobject{currentmarker}{\pgfqpoint{0.000000in}{-0.048611in}}{\pgfqpoint{0.000000in}{0.000000in}}{%
\pgfpathmoveto{\pgfqpoint{0.000000in}{0.000000in}}%
\pgfpathlineto{\pgfqpoint{0.000000in}{-0.048611in}}%
\pgfusepath{stroke,fill}%
}%
\begin{pgfscope}%
\pgfsys@transformshift{0.954157in}{0.548769in}%
\pgfsys@useobject{currentmarker}{}%
\end{pgfscope}%
\end{pgfscope}%
\begin{pgfscope}%
\definecolor{textcolor}{rgb}{0.000000,0.000000,0.000000}%
\pgfsetstrokecolor{textcolor}%
\pgfsetfillcolor{textcolor}%
\pgftext[x=0.954157in,y=0.451547in,,top]{\color{textcolor}\sffamily\fontsize{10.000000}{12.000000}\selectfont \(\displaystyle {0}\)}%
\end{pgfscope}%
\begin{pgfscope}%
\pgfsetbuttcap%
\pgfsetroundjoin%
\definecolor{currentfill}{rgb}{0.000000,0.000000,0.000000}%
\pgfsetfillcolor{currentfill}%
\pgfsetlinewidth{0.803000pt}%
\definecolor{currentstroke}{rgb}{0.000000,0.000000,0.000000}%
\pgfsetstrokecolor{currentstroke}%
\pgfsetdash{}{0pt}%
\pgfsys@defobject{currentmarker}{\pgfqpoint{0.000000in}{-0.048611in}}{\pgfqpoint{0.000000in}{0.000000in}}{%
\pgfpathmoveto{\pgfqpoint{0.000000in}{0.000000in}}%
\pgfpathlineto{\pgfqpoint{0.000000in}{-0.048611in}}%
\pgfusepath{stroke,fill}%
}%
\begin{pgfscope}%
\pgfsys@transformshift{1.584658in}{0.548769in}%
\pgfsys@useobject{currentmarker}{}%
\end{pgfscope}%
\end{pgfscope}%
\begin{pgfscope}%
\definecolor{textcolor}{rgb}{0.000000,0.000000,0.000000}%
\pgfsetstrokecolor{textcolor}%
\pgfsetfillcolor{textcolor}%
\pgftext[x=1.584658in,y=0.451547in,,top]{\color{textcolor}\sffamily\fontsize{10.000000}{12.000000}\selectfont \(\displaystyle {10}\)}%
\end{pgfscope}%
\begin{pgfscope}%
\pgfsetbuttcap%
\pgfsetroundjoin%
\definecolor{currentfill}{rgb}{0.000000,0.000000,0.000000}%
\pgfsetfillcolor{currentfill}%
\pgfsetlinewidth{0.803000pt}%
\definecolor{currentstroke}{rgb}{0.000000,0.000000,0.000000}%
\pgfsetstrokecolor{currentstroke}%
\pgfsetdash{}{0pt}%
\pgfsys@defobject{currentmarker}{\pgfqpoint{0.000000in}{-0.048611in}}{\pgfqpoint{0.000000in}{0.000000in}}{%
\pgfpathmoveto{\pgfqpoint{0.000000in}{0.000000in}}%
\pgfpathlineto{\pgfqpoint{0.000000in}{-0.048611in}}%
\pgfusepath{stroke,fill}%
}%
\begin{pgfscope}%
\pgfsys@transformshift{2.215159in}{0.548769in}%
\pgfsys@useobject{currentmarker}{}%
\end{pgfscope}%
\end{pgfscope}%
\begin{pgfscope}%
\definecolor{textcolor}{rgb}{0.000000,0.000000,0.000000}%
\pgfsetstrokecolor{textcolor}%
\pgfsetfillcolor{textcolor}%
\pgftext[x=2.215159in,y=0.451547in,,top]{\color{textcolor}\sffamily\fontsize{10.000000}{12.000000}\selectfont \(\displaystyle {20}\)}%
\end{pgfscope}%
\begin{pgfscope}%
\pgfsetbuttcap%
\pgfsetroundjoin%
\definecolor{currentfill}{rgb}{0.000000,0.000000,0.000000}%
\pgfsetfillcolor{currentfill}%
\pgfsetlinewidth{0.803000pt}%
\definecolor{currentstroke}{rgb}{0.000000,0.000000,0.000000}%
\pgfsetstrokecolor{currentstroke}%
\pgfsetdash{}{0pt}%
\pgfsys@defobject{currentmarker}{\pgfqpoint{0.000000in}{-0.048611in}}{\pgfqpoint{0.000000in}{0.000000in}}{%
\pgfpathmoveto{\pgfqpoint{0.000000in}{0.000000in}}%
\pgfpathlineto{\pgfqpoint{0.000000in}{-0.048611in}}%
\pgfusepath{stroke,fill}%
}%
\begin{pgfscope}%
\pgfsys@transformshift{2.845661in}{0.548769in}%
\pgfsys@useobject{currentmarker}{}%
\end{pgfscope}%
\end{pgfscope}%
\begin{pgfscope}%
\definecolor{textcolor}{rgb}{0.000000,0.000000,0.000000}%
\pgfsetstrokecolor{textcolor}%
\pgfsetfillcolor{textcolor}%
\pgftext[x=2.845661in,y=0.451547in,,top]{\color{textcolor}\sffamily\fontsize{10.000000}{12.000000}\selectfont \(\displaystyle {30}\)}%
\end{pgfscope}%
\begin{pgfscope}%
\pgfsetbuttcap%
\pgfsetroundjoin%
\definecolor{currentfill}{rgb}{0.000000,0.000000,0.000000}%
\pgfsetfillcolor{currentfill}%
\pgfsetlinewidth{0.803000pt}%
\definecolor{currentstroke}{rgb}{0.000000,0.000000,0.000000}%
\pgfsetstrokecolor{currentstroke}%
\pgfsetdash{}{0pt}%
\pgfsys@defobject{currentmarker}{\pgfqpoint{0.000000in}{-0.048611in}}{\pgfqpoint{0.000000in}{0.000000in}}{%
\pgfpathmoveto{\pgfqpoint{0.000000in}{0.000000in}}%
\pgfpathlineto{\pgfqpoint{0.000000in}{-0.048611in}}%
\pgfusepath{stroke,fill}%
}%
\begin{pgfscope}%
\pgfsys@transformshift{3.476162in}{0.548769in}%
\pgfsys@useobject{currentmarker}{}%
\end{pgfscope}%
\end{pgfscope}%
\begin{pgfscope}%
\definecolor{textcolor}{rgb}{0.000000,0.000000,0.000000}%
\pgfsetstrokecolor{textcolor}%
\pgfsetfillcolor{textcolor}%
\pgftext[x=3.476162in,y=0.451547in,,top]{\color{textcolor}\sffamily\fontsize{10.000000}{12.000000}\selectfont \(\displaystyle {40}\)}%
\end{pgfscope}%
\begin{pgfscope}%
\pgfsetbuttcap%
\pgfsetroundjoin%
\definecolor{currentfill}{rgb}{0.000000,0.000000,0.000000}%
\pgfsetfillcolor{currentfill}%
\pgfsetlinewidth{0.803000pt}%
\definecolor{currentstroke}{rgb}{0.000000,0.000000,0.000000}%
\pgfsetstrokecolor{currentstroke}%
\pgfsetdash{}{0pt}%
\pgfsys@defobject{currentmarker}{\pgfqpoint{0.000000in}{-0.048611in}}{\pgfqpoint{0.000000in}{0.000000in}}{%
\pgfpathmoveto{\pgfqpoint{0.000000in}{0.000000in}}%
\pgfpathlineto{\pgfqpoint{0.000000in}{-0.048611in}}%
\pgfusepath{stroke,fill}%
}%
\begin{pgfscope}%
\pgfsys@transformshift{4.106664in}{0.548769in}%
\pgfsys@useobject{currentmarker}{}%
\end{pgfscope}%
\end{pgfscope}%
\begin{pgfscope}%
\definecolor{textcolor}{rgb}{0.000000,0.000000,0.000000}%
\pgfsetstrokecolor{textcolor}%
\pgfsetfillcolor{textcolor}%
\pgftext[x=4.106664in,y=0.451547in,,top]{\color{textcolor}\sffamily\fontsize{10.000000}{12.000000}\selectfont \(\displaystyle {50}\)}%
\end{pgfscope}%
\begin{pgfscope}%
\pgfsetbuttcap%
\pgfsetroundjoin%
\definecolor{currentfill}{rgb}{0.000000,0.000000,0.000000}%
\pgfsetfillcolor{currentfill}%
\pgfsetlinewidth{0.803000pt}%
\definecolor{currentstroke}{rgb}{0.000000,0.000000,0.000000}%
\pgfsetstrokecolor{currentstroke}%
\pgfsetdash{}{0pt}%
\pgfsys@defobject{currentmarker}{\pgfqpoint{0.000000in}{-0.048611in}}{\pgfqpoint{0.000000in}{0.000000in}}{%
\pgfpathmoveto{\pgfqpoint{0.000000in}{0.000000in}}%
\pgfpathlineto{\pgfqpoint{0.000000in}{-0.048611in}}%
\pgfusepath{stroke,fill}%
}%
\begin{pgfscope}%
\pgfsys@transformshift{4.737165in}{0.548769in}%
\pgfsys@useobject{currentmarker}{}%
\end{pgfscope}%
\end{pgfscope}%
\begin{pgfscope}%
\definecolor{textcolor}{rgb}{0.000000,0.000000,0.000000}%
\pgfsetstrokecolor{textcolor}%
\pgfsetfillcolor{textcolor}%
\pgftext[x=4.737165in,y=0.451547in,,top]{\color{textcolor}\sffamily\fontsize{10.000000}{12.000000}\selectfont \(\displaystyle {60}\)}%
\end{pgfscope}%
\begin{pgfscope}%
\pgfsetbuttcap%
\pgfsetroundjoin%
\definecolor{currentfill}{rgb}{0.000000,0.000000,0.000000}%
\pgfsetfillcolor{currentfill}%
\pgfsetlinewidth{0.803000pt}%
\definecolor{currentstroke}{rgb}{0.000000,0.000000,0.000000}%
\pgfsetstrokecolor{currentstroke}%
\pgfsetdash{}{0pt}%
\pgfsys@defobject{currentmarker}{\pgfqpoint{0.000000in}{-0.048611in}}{\pgfqpoint{0.000000in}{0.000000in}}{%
\pgfpathmoveto{\pgfqpoint{0.000000in}{0.000000in}}%
\pgfpathlineto{\pgfqpoint{0.000000in}{-0.048611in}}%
\pgfusepath{stroke,fill}%
}%
\begin{pgfscope}%
\pgfsys@transformshift{5.367666in}{0.548769in}%
\pgfsys@useobject{currentmarker}{}%
\end{pgfscope}%
\end{pgfscope}%
\begin{pgfscope}%
\definecolor{textcolor}{rgb}{0.000000,0.000000,0.000000}%
\pgfsetstrokecolor{textcolor}%
\pgfsetfillcolor{textcolor}%
\pgftext[x=5.367666in,y=0.451547in,,top]{\color{textcolor}\sffamily\fontsize{10.000000}{12.000000}\selectfont \(\displaystyle {70}\)}%
\end{pgfscope}%
\begin{pgfscope}%
\definecolor{textcolor}{rgb}{0.000000,0.000000,0.000000}%
\pgfsetstrokecolor{textcolor}%
\pgfsetfillcolor{textcolor}%
\pgftext[x=3.318537in,y=0.272658in,,top]{\color{textcolor}\sffamily\fontsize{10.000000}{12.000000}\selectfont Number of Sinks}%
\end{pgfscope}%
\begin{pgfscope}%
\pgfsetbuttcap%
\pgfsetroundjoin%
\definecolor{currentfill}{rgb}{0.000000,0.000000,0.000000}%
\pgfsetfillcolor{currentfill}%
\pgfsetlinewidth{0.803000pt}%
\definecolor{currentstroke}{rgb}{0.000000,0.000000,0.000000}%
\pgfsetstrokecolor{currentstroke}%
\pgfsetdash{}{0pt}%
\pgfsys@defobject{currentmarker}{\pgfqpoint{-0.048611in}{0.000000in}}{\pgfqpoint{0.000000in}{0.000000in}}{%
\pgfpathmoveto{\pgfqpoint{0.000000in}{0.000000in}}%
\pgfpathlineto{\pgfqpoint{-0.048611in}{0.000000in}}%
\pgfusepath{stroke,fill}%
}%
\begin{pgfscope}%
\pgfsys@transformshift{0.787074in}{0.689795in}%
\pgfsys@useobject{currentmarker}{}%
\end{pgfscope}%
\end{pgfscope}%
\begin{pgfscope}%
\definecolor{textcolor}{rgb}{0.000000,0.000000,0.000000}%
\pgfsetstrokecolor{textcolor}%
\pgfsetfillcolor{textcolor}%
\pgftext[x=0.620407in, y=0.641601in, left, base]{\color{textcolor}\sffamily\fontsize{10.000000}{12.000000}\selectfont \(\displaystyle {0}\)}%
\end{pgfscope}%
\begin{pgfscope}%
\pgfsetbuttcap%
\pgfsetroundjoin%
\definecolor{currentfill}{rgb}{0.000000,0.000000,0.000000}%
\pgfsetfillcolor{currentfill}%
\pgfsetlinewidth{0.803000pt}%
\definecolor{currentstroke}{rgb}{0.000000,0.000000,0.000000}%
\pgfsetstrokecolor{currentstroke}%
\pgfsetdash{}{0pt}%
\pgfsys@defobject{currentmarker}{\pgfqpoint{-0.048611in}{0.000000in}}{\pgfqpoint{0.000000in}{0.000000in}}{%
\pgfpathmoveto{\pgfqpoint{0.000000in}{0.000000in}}%
\pgfpathlineto{\pgfqpoint{-0.048611in}{0.000000in}}%
\pgfusepath{stroke,fill}%
}%
\begin{pgfscope}%
\pgfsys@transformshift{0.787074in}{1.060560in}%
\pgfsys@useobject{currentmarker}{}%
\end{pgfscope}%
\end{pgfscope}%
\begin{pgfscope}%
\definecolor{textcolor}{rgb}{0.000000,0.000000,0.000000}%
\pgfsetstrokecolor{textcolor}%
\pgfsetfillcolor{textcolor}%
\pgftext[x=0.412073in, y=1.012365in, left, base]{\color{textcolor}\sffamily\fontsize{10.000000}{12.000000}\selectfont \(\displaystyle {2500}\)}%
\end{pgfscope}%
\begin{pgfscope}%
\pgfsetbuttcap%
\pgfsetroundjoin%
\definecolor{currentfill}{rgb}{0.000000,0.000000,0.000000}%
\pgfsetfillcolor{currentfill}%
\pgfsetlinewidth{0.803000pt}%
\definecolor{currentstroke}{rgb}{0.000000,0.000000,0.000000}%
\pgfsetstrokecolor{currentstroke}%
\pgfsetdash{}{0pt}%
\pgfsys@defobject{currentmarker}{\pgfqpoint{-0.048611in}{0.000000in}}{\pgfqpoint{0.000000in}{0.000000in}}{%
\pgfpathmoveto{\pgfqpoint{0.000000in}{0.000000in}}%
\pgfpathlineto{\pgfqpoint{-0.048611in}{0.000000in}}%
\pgfusepath{stroke,fill}%
}%
\begin{pgfscope}%
\pgfsys@transformshift{0.787074in}{1.431324in}%
\pgfsys@useobject{currentmarker}{}%
\end{pgfscope}%
\end{pgfscope}%
\begin{pgfscope}%
\definecolor{textcolor}{rgb}{0.000000,0.000000,0.000000}%
\pgfsetstrokecolor{textcolor}%
\pgfsetfillcolor{textcolor}%
\pgftext[x=0.412073in, y=1.383129in, left, base]{\color{textcolor}\sffamily\fontsize{10.000000}{12.000000}\selectfont \(\displaystyle {5000}\)}%
\end{pgfscope}%
\begin{pgfscope}%
\pgfsetbuttcap%
\pgfsetroundjoin%
\definecolor{currentfill}{rgb}{0.000000,0.000000,0.000000}%
\pgfsetfillcolor{currentfill}%
\pgfsetlinewidth{0.803000pt}%
\definecolor{currentstroke}{rgb}{0.000000,0.000000,0.000000}%
\pgfsetstrokecolor{currentstroke}%
\pgfsetdash{}{0pt}%
\pgfsys@defobject{currentmarker}{\pgfqpoint{-0.048611in}{0.000000in}}{\pgfqpoint{0.000000in}{0.000000in}}{%
\pgfpathmoveto{\pgfqpoint{0.000000in}{0.000000in}}%
\pgfpathlineto{\pgfqpoint{-0.048611in}{0.000000in}}%
\pgfusepath{stroke,fill}%
}%
\begin{pgfscope}%
\pgfsys@transformshift{0.787074in}{1.802088in}%
\pgfsys@useobject{currentmarker}{}%
\end{pgfscope}%
\end{pgfscope}%
\begin{pgfscope}%
\definecolor{textcolor}{rgb}{0.000000,0.000000,0.000000}%
\pgfsetstrokecolor{textcolor}%
\pgfsetfillcolor{textcolor}%
\pgftext[x=0.412073in, y=1.753894in, left, base]{\color{textcolor}\sffamily\fontsize{10.000000}{12.000000}\selectfont \(\displaystyle {7500}\)}%
\end{pgfscope}%
\begin{pgfscope}%
\pgfsetbuttcap%
\pgfsetroundjoin%
\definecolor{currentfill}{rgb}{0.000000,0.000000,0.000000}%
\pgfsetfillcolor{currentfill}%
\pgfsetlinewidth{0.803000pt}%
\definecolor{currentstroke}{rgb}{0.000000,0.000000,0.000000}%
\pgfsetstrokecolor{currentstroke}%
\pgfsetdash{}{0pt}%
\pgfsys@defobject{currentmarker}{\pgfqpoint{-0.048611in}{0.000000in}}{\pgfqpoint{0.000000in}{0.000000in}}{%
\pgfpathmoveto{\pgfqpoint{0.000000in}{0.000000in}}%
\pgfpathlineto{\pgfqpoint{-0.048611in}{0.000000in}}%
\pgfusepath{stroke,fill}%
}%
\begin{pgfscope}%
\pgfsys@transformshift{0.787074in}{2.172852in}%
\pgfsys@useobject{currentmarker}{}%
\end{pgfscope}%
\end{pgfscope}%
\begin{pgfscope}%
\definecolor{textcolor}{rgb}{0.000000,0.000000,0.000000}%
\pgfsetstrokecolor{textcolor}%
\pgfsetfillcolor{textcolor}%
\pgftext[x=0.342628in, y=2.124658in, left, base]{\color{textcolor}\sffamily\fontsize{10.000000}{12.000000}\selectfont \(\displaystyle {10000}\)}%
\end{pgfscope}%
\begin{pgfscope}%
\pgfsetbuttcap%
\pgfsetroundjoin%
\definecolor{currentfill}{rgb}{0.000000,0.000000,0.000000}%
\pgfsetfillcolor{currentfill}%
\pgfsetlinewidth{0.803000pt}%
\definecolor{currentstroke}{rgb}{0.000000,0.000000,0.000000}%
\pgfsetstrokecolor{currentstroke}%
\pgfsetdash{}{0pt}%
\pgfsys@defobject{currentmarker}{\pgfqpoint{-0.048611in}{0.000000in}}{\pgfqpoint{0.000000in}{0.000000in}}{%
\pgfpathmoveto{\pgfqpoint{0.000000in}{0.000000in}}%
\pgfpathlineto{\pgfqpoint{-0.048611in}{0.000000in}}%
\pgfusepath{stroke,fill}%
}%
\begin{pgfscope}%
\pgfsys@transformshift{0.787074in}{2.543617in}%
\pgfsys@useobject{currentmarker}{}%
\end{pgfscope}%
\end{pgfscope}%
\begin{pgfscope}%
\definecolor{textcolor}{rgb}{0.000000,0.000000,0.000000}%
\pgfsetstrokecolor{textcolor}%
\pgfsetfillcolor{textcolor}%
\pgftext[x=0.342628in, y=2.495422in, left, base]{\color{textcolor}\sffamily\fontsize{10.000000}{12.000000}\selectfont \(\displaystyle {12500}\)}%
\end{pgfscope}%
\begin{pgfscope}%
\pgfsetbuttcap%
\pgfsetroundjoin%
\definecolor{currentfill}{rgb}{0.000000,0.000000,0.000000}%
\pgfsetfillcolor{currentfill}%
\pgfsetlinewidth{0.803000pt}%
\definecolor{currentstroke}{rgb}{0.000000,0.000000,0.000000}%
\pgfsetstrokecolor{currentstroke}%
\pgfsetdash{}{0pt}%
\pgfsys@defobject{currentmarker}{\pgfqpoint{-0.048611in}{0.000000in}}{\pgfqpoint{0.000000in}{0.000000in}}{%
\pgfpathmoveto{\pgfqpoint{0.000000in}{0.000000in}}%
\pgfpathlineto{\pgfqpoint{-0.048611in}{0.000000in}}%
\pgfusepath{stroke,fill}%
}%
\begin{pgfscope}%
\pgfsys@transformshift{0.787074in}{2.914381in}%
\pgfsys@useobject{currentmarker}{}%
\end{pgfscope}%
\end{pgfscope}%
\begin{pgfscope}%
\definecolor{textcolor}{rgb}{0.000000,0.000000,0.000000}%
\pgfsetstrokecolor{textcolor}%
\pgfsetfillcolor{textcolor}%
\pgftext[x=0.342628in, y=2.866187in, left, base]{\color{textcolor}\sffamily\fontsize{10.000000}{12.000000}\selectfont \(\displaystyle {15000}\)}%
\end{pgfscope}%
\begin{pgfscope}%
\pgfsetbuttcap%
\pgfsetroundjoin%
\definecolor{currentfill}{rgb}{0.000000,0.000000,0.000000}%
\pgfsetfillcolor{currentfill}%
\pgfsetlinewidth{0.803000pt}%
\definecolor{currentstroke}{rgb}{0.000000,0.000000,0.000000}%
\pgfsetstrokecolor{currentstroke}%
\pgfsetdash{}{0pt}%
\pgfsys@defobject{currentmarker}{\pgfqpoint{-0.048611in}{0.000000in}}{\pgfqpoint{0.000000in}{0.000000in}}{%
\pgfpathmoveto{\pgfqpoint{0.000000in}{0.000000in}}%
\pgfpathlineto{\pgfqpoint{-0.048611in}{0.000000in}}%
\pgfusepath{stroke,fill}%
}%
\begin{pgfscope}%
\pgfsys@transformshift{0.787074in}{3.285145in}%
\pgfsys@useobject{currentmarker}{}%
\end{pgfscope}%
\end{pgfscope}%
\begin{pgfscope}%
\definecolor{textcolor}{rgb}{0.000000,0.000000,0.000000}%
\pgfsetstrokecolor{textcolor}%
\pgfsetfillcolor{textcolor}%
\pgftext[x=0.342628in, y=3.236951in, left, base]{\color{textcolor}\sffamily\fontsize{10.000000}{12.000000}\selectfont \(\displaystyle {17500}\)}%
\end{pgfscope}%
\begin{pgfscope}%
\definecolor{textcolor}{rgb}{0.000000,0.000000,0.000000}%
\pgfsetstrokecolor{textcolor}%
\pgfsetfillcolor{textcolor}%
\pgftext[x=0.287073in,y=2.100064in,,bottom,rotate=90.000000]{\color{textcolor}\sffamily\fontsize{10.000000}{12.000000}\selectfont Maximum Memory Usage (MB)}%
\end{pgfscope}%
\begin{pgfscope}%
\pgfsetrectcap%
\pgfsetmiterjoin%
\pgfsetlinewidth{0.803000pt}%
\definecolor{currentstroke}{rgb}{0.000000,0.000000,0.000000}%
\pgfsetstrokecolor{currentstroke}%
\pgfsetdash{}{0pt}%
\pgfpathmoveto{\pgfqpoint{0.787074in}{0.548769in}}%
\pgfpathlineto{\pgfqpoint{0.787074in}{3.651359in}}%
\pgfusepath{stroke}%
\end{pgfscope}%
\begin{pgfscope}%
\pgfsetrectcap%
\pgfsetmiterjoin%
\pgfsetlinewidth{0.803000pt}%
\definecolor{currentstroke}{rgb}{0.000000,0.000000,0.000000}%
\pgfsetstrokecolor{currentstroke}%
\pgfsetdash{}{0pt}%
\pgfpathmoveto{\pgfqpoint{5.850000in}{0.548769in}}%
\pgfpathlineto{\pgfqpoint{5.850000in}{3.651359in}}%
\pgfusepath{stroke}%
\end{pgfscope}%
\begin{pgfscope}%
\pgfsetrectcap%
\pgfsetmiterjoin%
\pgfsetlinewidth{0.803000pt}%
\definecolor{currentstroke}{rgb}{0.000000,0.000000,0.000000}%
\pgfsetstrokecolor{currentstroke}%
\pgfsetdash{}{0pt}%
\pgfpathmoveto{\pgfqpoint{0.787074in}{0.548769in}}%
\pgfpathlineto{\pgfqpoint{5.850000in}{0.548769in}}%
\pgfusepath{stroke}%
\end{pgfscope}%
\begin{pgfscope}%
\pgfsetrectcap%
\pgfsetmiterjoin%
\pgfsetlinewidth{0.803000pt}%
\definecolor{currentstroke}{rgb}{0.000000,0.000000,0.000000}%
\pgfsetstrokecolor{currentstroke}%
\pgfsetdash{}{0pt}%
\pgfpathmoveto{\pgfqpoint{0.787074in}{3.651359in}}%
\pgfpathlineto{\pgfqpoint{5.850000in}{3.651359in}}%
\pgfusepath{stroke}%
\end{pgfscope}%
\begin{pgfscope}%
\definecolor{textcolor}{rgb}{0.000000,0.000000,0.000000}%
\pgfsetstrokecolor{textcolor}%
\pgfsetfillcolor{textcolor}%
\pgftext[x=3.318537in,y=3.734692in,,base]{\color{textcolor}\sffamily\fontsize{12.000000}{14.400000}\selectfont Backwards}%
\end{pgfscope}%
\begin{pgfscope}%
\pgfsetbuttcap%
\pgfsetmiterjoin%
\definecolor{currentfill}{rgb}{1.000000,1.000000,1.000000}%
\pgfsetfillcolor{currentfill}%
\pgfsetfillopacity{0.800000}%
\pgfsetlinewidth{1.003750pt}%
\definecolor{currentstroke}{rgb}{0.800000,0.800000,0.800000}%
\pgfsetstrokecolor{currentstroke}%
\pgfsetstrokeopacity{0.800000}%
\pgfsetdash{}{0pt}%
\pgfpathmoveto{\pgfqpoint{4.300417in}{0.618213in}}%
\pgfpathlineto{\pgfqpoint{5.752778in}{0.618213in}}%
\pgfpathquadraticcurveto{\pgfqpoint{5.780556in}{0.618213in}}{\pgfqpoint{5.780556in}{0.645991in}}%
\pgfpathlineto{\pgfqpoint{5.780556in}{1.214463in}}%
\pgfpathquadraticcurveto{\pgfqpoint{5.780556in}{1.242241in}}{\pgfqpoint{5.752778in}{1.242241in}}%
\pgfpathlineto{\pgfqpoint{4.300417in}{1.242241in}}%
\pgfpathquadraticcurveto{\pgfqpoint{4.272639in}{1.242241in}}{\pgfqpoint{4.272639in}{1.214463in}}%
\pgfpathlineto{\pgfqpoint{4.272639in}{0.645991in}}%
\pgfpathquadraticcurveto{\pgfqpoint{4.272639in}{0.618213in}}{\pgfqpoint{4.300417in}{0.618213in}}%
\pgfpathclose%
\pgfusepath{stroke,fill}%
\end{pgfscope}%
\begin{pgfscope}%
\pgfsetbuttcap%
\pgfsetroundjoin%
\definecolor{currentfill}{rgb}{0.121569,0.466667,0.705882}%
\pgfsetfillcolor{currentfill}%
\pgfsetlinewidth{1.003750pt}%
\definecolor{currentstroke}{rgb}{0.121569,0.466667,0.705882}%
\pgfsetstrokecolor{currentstroke}%
\pgfsetdash{}{0pt}%
\pgfsys@defobject{currentmarker}{\pgfqpoint{-0.034722in}{-0.034722in}}{\pgfqpoint{0.034722in}{0.034722in}}{%
\pgfpathmoveto{\pgfqpoint{0.000000in}{-0.034722in}}%
\pgfpathcurveto{\pgfqpoint{0.009208in}{-0.034722in}}{\pgfqpoint{0.018041in}{-0.031064in}}{\pgfqpoint{0.024552in}{-0.024552in}}%
\pgfpathcurveto{\pgfqpoint{0.031064in}{-0.018041in}}{\pgfqpoint{0.034722in}{-0.009208in}}{\pgfqpoint{0.034722in}{0.000000in}}%
\pgfpathcurveto{\pgfqpoint{0.034722in}{0.009208in}}{\pgfqpoint{0.031064in}{0.018041in}}{\pgfqpoint{0.024552in}{0.024552in}}%
\pgfpathcurveto{\pgfqpoint{0.018041in}{0.031064in}}{\pgfqpoint{0.009208in}{0.034722in}}{\pgfqpoint{0.000000in}{0.034722in}}%
\pgfpathcurveto{\pgfqpoint{-0.009208in}{0.034722in}}{\pgfqpoint{-0.018041in}{0.031064in}}{\pgfqpoint{-0.024552in}{0.024552in}}%
\pgfpathcurveto{\pgfqpoint{-0.031064in}{0.018041in}}{\pgfqpoint{-0.034722in}{0.009208in}}{\pgfqpoint{-0.034722in}{0.000000in}}%
\pgfpathcurveto{\pgfqpoint{-0.034722in}{-0.009208in}}{\pgfqpoint{-0.031064in}{-0.018041in}}{\pgfqpoint{-0.024552in}{-0.024552in}}%
\pgfpathcurveto{\pgfqpoint{-0.018041in}{-0.031064in}}{\pgfqpoint{-0.009208in}{-0.034722in}}{\pgfqpoint{0.000000in}{-0.034722in}}%
\pgfpathclose%
\pgfusepath{stroke,fill}%
}%
\begin{pgfscope}%
\pgfsys@transformshift{4.467083in}{1.138074in}%
\pgfsys@useobject{currentmarker}{}%
\end{pgfscope}%
\end{pgfscope}%
\begin{pgfscope}%
\definecolor{textcolor}{rgb}{0.000000,0.000000,0.000000}%
\pgfsetstrokecolor{textcolor}%
\pgfsetfillcolor{textcolor}%
\pgftext[x=4.717083in,y=1.089463in,left,base]{\color{textcolor}\sffamily\fontsize{10.000000}{12.000000}\selectfont No Timeout}%
\end{pgfscope}%
\begin{pgfscope}%
\pgfsetbuttcap%
\pgfsetroundjoin%
\definecolor{currentfill}{rgb}{1.000000,0.498039,0.054902}%
\pgfsetfillcolor{currentfill}%
\pgfsetlinewidth{1.003750pt}%
\definecolor{currentstroke}{rgb}{1.000000,0.498039,0.054902}%
\pgfsetstrokecolor{currentstroke}%
\pgfsetdash{}{0pt}%
\pgfsys@defobject{currentmarker}{\pgfqpoint{-0.034722in}{-0.034722in}}{\pgfqpoint{0.034722in}{0.034722in}}{%
\pgfpathmoveto{\pgfqpoint{0.000000in}{-0.034722in}}%
\pgfpathcurveto{\pgfqpoint{0.009208in}{-0.034722in}}{\pgfqpoint{0.018041in}{-0.031064in}}{\pgfqpoint{0.024552in}{-0.024552in}}%
\pgfpathcurveto{\pgfqpoint{0.031064in}{-0.018041in}}{\pgfqpoint{0.034722in}{-0.009208in}}{\pgfqpoint{0.034722in}{0.000000in}}%
\pgfpathcurveto{\pgfqpoint{0.034722in}{0.009208in}}{\pgfqpoint{0.031064in}{0.018041in}}{\pgfqpoint{0.024552in}{0.024552in}}%
\pgfpathcurveto{\pgfqpoint{0.018041in}{0.031064in}}{\pgfqpoint{0.009208in}{0.034722in}}{\pgfqpoint{0.000000in}{0.034722in}}%
\pgfpathcurveto{\pgfqpoint{-0.009208in}{0.034722in}}{\pgfqpoint{-0.018041in}{0.031064in}}{\pgfqpoint{-0.024552in}{0.024552in}}%
\pgfpathcurveto{\pgfqpoint{-0.031064in}{0.018041in}}{\pgfqpoint{-0.034722in}{0.009208in}}{\pgfqpoint{-0.034722in}{0.000000in}}%
\pgfpathcurveto{\pgfqpoint{-0.034722in}{-0.009208in}}{\pgfqpoint{-0.031064in}{-0.018041in}}{\pgfqpoint{-0.024552in}{-0.024552in}}%
\pgfpathcurveto{\pgfqpoint{-0.018041in}{-0.031064in}}{\pgfqpoint{-0.009208in}{-0.034722in}}{\pgfqpoint{0.000000in}{-0.034722in}}%
\pgfpathclose%
\pgfusepath{stroke,fill}%
}%
\begin{pgfscope}%
\pgfsys@transformshift{4.467083in}{0.944463in}%
\pgfsys@useobject{currentmarker}{}%
\end{pgfscope}%
\end{pgfscope}%
\begin{pgfscope}%
\definecolor{textcolor}{rgb}{0.000000,0.000000,0.000000}%
\pgfsetstrokecolor{textcolor}%
\pgfsetfillcolor{textcolor}%
\pgftext[x=4.717083in,y=0.895852in,left,base]{\color{textcolor}\sffamily\fontsize{10.000000}{12.000000}\selectfont Time Timeout}%
\end{pgfscope}%
\begin{pgfscope}%
\pgfsetbuttcap%
\pgfsetroundjoin%
\definecolor{currentfill}{rgb}{0.839216,0.152941,0.156863}%
\pgfsetfillcolor{currentfill}%
\pgfsetlinewidth{1.003750pt}%
\definecolor{currentstroke}{rgb}{0.839216,0.152941,0.156863}%
\pgfsetstrokecolor{currentstroke}%
\pgfsetdash{}{0pt}%
\pgfsys@defobject{currentmarker}{\pgfqpoint{-0.034722in}{-0.034722in}}{\pgfqpoint{0.034722in}{0.034722in}}{%
\pgfpathmoveto{\pgfqpoint{0.000000in}{-0.034722in}}%
\pgfpathcurveto{\pgfqpoint{0.009208in}{-0.034722in}}{\pgfqpoint{0.018041in}{-0.031064in}}{\pgfqpoint{0.024552in}{-0.024552in}}%
\pgfpathcurveto{\pgfqpoint{0.031064in}{-0.018041in}}{\pgfqpoint{0.034722in}{-0.009208in}}{\pgfqpoint{0.034722in}{0.000000in}}%
\pgfpathcurveto{\pgfqpoint{0.034722in}{0.009208in}}{\pgfqpoint{0.031064in}{0.018041in}}{\pgfqpoint{0.024552in}{0.024552in}}%
\pgfpathcurveto{\pgfqpoint{0.018041in}{0.031064in}}{\pgfqpoint{0.009208in}{0.034722in}}{\pgfqpoint{0.000000in}{0.034722in}}%
\pgfpathcurveto{\pgfqpoint{-0.009208in}{0.034722in}}{\pgfqpoint{-0.018041in}{0.031064in}}{\pgfqpoint{-0.024552in}{0.024552in}}%
\pgfpathcurveto{\pgfqpoint{-0.031064in}{0.018041in}}{\pgfqpoint{-0.034722in}{0.009208in}}{\pgfqpoint{-0.034722in}{0.000000in}}%
\pgfpathcurveto{\pgfqpoint{-0.034722in}{-0.009208in}}{\pgfqpoint{-0.031064in}{-0.018041in}}{\pgfqpoint{-0.024552in}{-0.024552in}}%
\pgfpathcurveto{\pgfqpoint{-0.018041in}{-0.031064in}}{\pgfqpoint{-0.009208in}{-0.034722in}}{\pgfqpoint{0.000000in}{-0.034722in}}%
\pgfpathclose%
\pgfusepath{stroke,fill}%
}%
\begin{pgfscope}%
\pgfsys@transformshift{4.467083in}{0.750852in}%
\pgfsys@useobject{currentmarker}{}%
\end{pgfscope}%
\end{pgfscope}%
\begin{pgfscope}%
\definecolor{textcolor}{rgb}{0.000000,0.000000,0.000000}%
\pgfsetstrokecolor{textcolor}%
\pgfsetfillcolor{textcolor}%
\pgftext[x=4.717083in,y=0.702241in,left,base]{\color{textcolor}\sffamily\fontsize{10.000000}{12.000000}\selectfont Memory Timeout}%
\end{pgfscope}%
\end{pgfpicture}%
\makeatother%
\endgroup%

                }
            \end{subfigure}
            \caption{Sinks}
        \end{subfigure}
        \bigbreak
        \begin{subfigure}[b]{\textwidth}
            \centering
            \begin{subfigure}[]{0.45\textwidth}
                \centering
                \resizebox{\columnwidth}{!}{
                    %% Creator: Matplotlib, PGF backend
%%
%% To include the figure in your LaTeX document, write
%%   \input{<filename>.pgf}
%%
%% Make sure the required packages are loaded in your preamble
%%   \usepackage{pgf}
%%
%% and, on pdftex
%%   \usepackage[utf8]{inputenc}\DeclareUnicodeCharacter{2212}{-}
%%
%% or, on luatex and xetex
%%   \usepackage{unicode-math}
%%
%% Figures using additional raster images can only be included by \input if
%% they are in the same directory as the main LaTeX file. For loading figures
%% from other directories you can use the `import` package
%%   \usepackage{import}
%%
%% and then include the figures with
%%   \import{<path to file>}{<filename>.pgf}
%%
%% Matplotlib used the following preamble
%%   \usepackage{amsmath}
%%   \usepackage{fontspec}
%%
\begingroup%
\makeatletter%
\begin{pgfpicture}%
\pgfpathrectangle{\pgfpointorigin}{\pgfqpoint{6.000000in}{4.000000in}}%
\pgfusepath{use as bounding box, clip}%
\begin{pgfscope}%
\pgfsetbuttcap%
\pgfsetmiterjoin%
\definecolor{currentfill}{rgb}{1.000000,1.000000,1.000000}%
\pgfsetfillcolor{currentfill}%
\pgfsetlinewidth{0.000000pt}%
\definecolor{currentstroke}{rgb}{1.000000,1.000000,1.000000}%
\pgfsetstrokecolor{currentstroke}%
\pgfsetdash{}{0pt}%
\pgfpathmoveto{\pgfqpoint{0.000000in}{0.000000in}}%
\pgfpathlineto{\pgfqpoint{6.000000in}{0.000000in}}%
\pgfpathlineto{\pgfqpoint{6.000000in}{4.000000in}}%
\pgfpathlineto{\pgfqpoint{0.000000in}{4.000000in}}%
\pgfpathclose%
\pgfusepath{fill}%
\end{pgfscope}%
\begin{pgfscope}%
\pgfsetbuttcap%
\pgfsetmiterjoin%
\definecolor{currentfill}{rgb}{1.000000,1.000000,1.000000}%
\pgfsetfillcolor{currentfill}%
\pgfsetlinewidth{0.000000pt}%
\definecolor{currentstroke}{rgb}{0.000000,0.000000,0.000000}%
\pgfsetstrokecolor{currentstroke}%
\pgfsetstrokeopacity{0.000000}%
\pgfsetdash{}{0pt}%
\pgfpathmoveto{\pgfqpoint{0.787074in}{0.548769in}}%
\pgfpathlineto{\pgfqpoint{5.850000in}{0.548769in}}%
\pgfpathlineto{\pgfqpoint{5.850000in}{3.651359in}}%
\pgfpathlineto{\pgfqpoint{0.787074in}{3.651359in}}%
\pgfpathclose%
\pgfusepath{fill}%
\end{pgfscope}%
\begin{pgfscope}%
\pgfpathrectangle{\pgfqpoint{0.787074in}{0.548769in}}{\pgfqpoint{5.062926in}{3.102590in}}%
\pgfusepath{clip}%
\pgfsetbuttcap%
\pgfsetroundjoin%
\definecolor{currentfill}{rgb}{0.121569,0.466667,0.705882}%
\pgfsetfillcolor{currentfill}%
\pgfsetlinewidth{1.003750pt}%
\definecolor{currentstroke}{rgb}{0.121569,0.466667,0.705882}%
\pgfsetstrokecolor{currentstroke}%
\pgfsetdash{}{0pt}%
\pgfpathmoveto{\pgfqpoint{1.173922in}{0.648193in}}%
\pgfpathcurveto{\pgfqpoint{1.184972in}{0.648193in}}{\pgfqpoint{1.195571in}{0.652583in}}{\pgfqpoint{1.203385in}{0.660397in}}%
\pgfpathcurveto{\pgfqpoint{1.211199in}{0.668210in}}{\pgfqpoint{1.215589in}{0.678809in}}{\pgfqpoint{1.215589in}{0.689859in}}%
\pgfpathcurveto{\pgfqpoint{1.215589in}{0.700910in}}{\pgfqpoint{1.211199in}{0.711509in}}{\pgfqpoint{1.203385in}{0.719322in}}%
\pgfpathcurveto{\pgfqpoint{1.195571in}{0.727136in}}{\pgfqpoint{1.184972in}{0.731526in}}{\pgfqpoint{1.173922in}{0.731526in}}%
\pgfpathcurveto{\pgfqpoint{1.162872in}{0.731526in}}{\pgfqpoint{1.152273in}{0.727136in}}{\pgfqpoint{1.144459in}{0.719322in}}%
\pgfpathcurveto{\pgfqpoint{1.136646in}{0.711509in}}{\pgfqpoint{1.132255in}{0.700910in}}{\pgfqpoint{1.132255in}{0.689859in}}%
\pgfpathcurveto{\pgfqpoint{1.132255in}{0.678809in}}{\pgfqpoint{1.136646in}{0.668210in}}{\pgfqpoint{1.144459in}{0.660397in}}%
\pgfpathcurveto{\pgfqpoint{1.152273in}{0.652583in}}{\pgfqpoint{1.162872in}{0.648193in}}{\pgfqpoint{1.173922in}{0.648193in}}%
\pgfpathclose%
\pgfusepath{stroke,fill}%
\end{pgfscope}%
\begin{pgfscope}%
\pgfpathrectangle{\pgfqpoint{0.787074in}{0.548769in}}{\pgfqpoint{5.062926in}{3.102590in}}%
\pgfusepath{clip}%
\pgfsetbuttcap%
\pgfsetroundjoin%
\definecolor{currentfill}{rgb}{1.000000,0.498039,0.054902}%
\pgfsetfillcolor{currentfill}%
\pgfsetlinewidth{1.003750pt}%
\definecolor{currentstroke}{rgb}{1.000000,0.498039,0.054902}%
\pgfsetstrokecolor{currentstroke}%
\pgfsetdash{}{0pt}%
\pgfpathmoveto{\pgfqpoint{1.438247in}{1.859307in}}%
\pgfpathcurveto{\pgfqpoint{1.449297in}{1.859307in}}{\pgfqpoint{1.459896in}{1.863697in}}{\pgfqpoint{1.467710in}{1.871511in}}%
\pgfpathcurveto{\pgfqpoint{1.475523in}{1.879324in}}{\pgfqpoint{1.479914in}{1.889923in}}{\pgfqpoint{1.479914in}{1.900973in}}%
\pgfpathcurveto{\pgfqpoint{1.479914in}{1.912024in}}{\pgfqpoint{1.475523in}{1.922623in}}{\pgfqpoint{1.467710in}{1.930436in}}%
\pgfpathcurveto{\pgfqpoint{1.459896in}{1.938250in}}{\pgfqpoint{1.449297in}{1.942640in}}{\pgfqpoint{1.438247in}{1.942640in}}%
\pgfpathcurveto{\pgfqpoint{1.427197in}{1.942640in}}{\pgfqpoint{1.416598in}{1.938250in}}{\pgfqpoint{1.408784in}{1.930436in}}%
\pgfpathcurveto{\pgfqpoint{1.400971in}{1.922623in}}{\pgfqpoint{1.396580in}{1.912024in}}{\pgfqpoint{1.396580in}{1.900973in}}%
\pgfpathcurveto{\pgfqpoint{1.396580in}{1.889923in}}{\pgfqpoint{1.400971in}{1.879324in}}{\pgfqpoint{1.408784in}{1.871511in}}%
\pgfpathcurveto{\pgfqpoint{1.416598in}{1.863697in}}{\pgfqpoint{1.427197in}{1.859307in}}{\pgfqpoint{1.438247in}{1.859307in}}%
\pgfpathclose%
\pgfusepath{stroke,fill}%
\end{pgfscope}%
\begin{pgfscope}%
\pgfpathrectangle{\pgfqpoint{0.787074in}{0.548769in}}{\pgfqpoint{5.062926in}{3.102590in}}%
\pgfusepath{clip}%
\pgfsetbuttcap%
\pgfsetroundjoin%
\definecolor{currentfill}{rgb}{1.000000,0.498039,0.054902}%
\pgfsetfillcolor{currentfill}%
\pgfsetlinewidth{1.003750pt}%
\definecolor{currentstroke}{rgb}{1.000000,0.498039,0.054902}%
\pgfsetstrokecolor{currentstroke}%
\pgfsetdash{}{0pt}%
\pgfpathmoveto{\pgfqpoint{1.041055in}{2.662812in}}%
\pgfpathcurveto{\pgfqpoint{1.052105in}{2.662812in}}{\pgfqpoint{1.062704in}{2.667202in}}{\pgfqpoint{1.070518in}{2.675016in}}%
\pgfpathcurveto{\pgfqpoint{1.078331in}{2.682830in}}{\pgfqpoint{1.082721in}{2.693429in}}{\pgfqpoint{1.082721in}{2.704479in}}%
\pgfpathcurveto{\pgfqpoint{1.082721in}{2.715529in}}{\pgfqpoint{1.078331in}{2.726128in}}{\pgfqpoint{1.070518in}{2.733942in}}%
\pgfpathcurveto{\pgfqpoint{1.062704in}{2.741755in}}{\pgfqpoint{1.052105in}{2.746145in}}{\pgfqpoint{1.041055in}{2.746145in}}%
\pgfpathcurveto{\pgfqpoint{1.030005in}{2.746145in}}{\pgfqpoint{1.019406in}{2.741755in}}{\pgfqpoint{1.011592in}{2.733942in}}%
\pgfpathcurveto{\pgfqpoint{1.003778in}{2.726128in}}{\pgfqpoint{0.999388in}{2.715529in}}{\pgfqpoint{0.999388in}{2.704479in}}%
\pgfpathcurveto{\pgfqpoint{0.999388in}{2.693429in}}{\pgfqpoint{1.003778in}{2.682830in}}{\pgfqpoint{1.011592in}{2.675016in}}%
\pgfpathcurveto{\pgfqpoint{1.019406in}{2.667202in}}{\pgfqpoint{1.030005in}{2.662812in}}{\pgfqpoint{1.041055in}{2.662812in}}%
\pgfpathclose%
\pgfusepath{stroke,fill}%
\end{pgfscope}%
\begin{pgfscope}%
\pgfpathrectangle{\pgfqpoint{0.787074in}{0.548769in}}{\pgfqpoint{5.062926in}{3.102590in}}%
\pgfusepath{clip}%
\pgfsetbuttcap%
\pgfsetroundjoin%
\definecolor{currentfill}{rgb}{1.000000,0.498039,0.054902}%
\pgfsetfillcolor{currentfill}%
\pgfsetlinewidth{1.003750pt}%
\definecolor{currentstroke}{rgb}{1.000000,0.498039,0.054902}%
\pgfsetstrokecolor{currentstroke}%
\pgfsetdash{}{0pt}%
\pgfpathmoveto{\pgfqpoint{1.631955in}{2.336454in}}%
\pgfpathcurveto{\pgfqpoint{1.643005in}{2.336454in}}{\pgfqpoint{1.653604in}{2.340845in}}{\pgfqpoint{1.661418in}{2.348658in}}%
\pgfpathcurveto{\pgfqpoint{1.669231in}{2.356472in}}{\pgfqpoint{1.673621in}{2.367071in}}{\pgfqpoint{1.673621in}{2.378121in}}%
\pgfpathcurveto{\pgfqpoint{1.673621in}{2.389171in}}{\pgfqpoint{1.669231in}{2.399770in}}{\pgfqpoint{1.661418in}{2.407584in}}%
\pgfpathcurveto{\pgfqpoint{1.653604in}{2.415397in}}{\pgfqpoint{1.643005in}{2.419788in}}{\pgfqpoint{1.631955in}{2.419788in}}%
\pgfpathcurveto{\pgfqpoint{1.620905in}{2.419788in}}{\pgfqpoint{1.610306in}{2.415397in}}{\pgfqpoint{1.602492in}{2.407584in}}%
\pgfpathcurveto{\pgfqpoint{1.594678in}{2.399770in}}{\pgfqpoint{1.590288in}{2.389171in}}{\pgfqpoint{1.590288in}{2.378121in}}%
\pgfpathcurveto{\pgfqpoint{1.590288in}{2.367071in}}{\pgfqpoint{1.594678in}{2.356472in}}{\pgfqpoint{1.602492in}{2.348658in}}%
\pgfpathcurveto{\pgfqpoint{1.610306in}{2.340845in}}{\pgfqpoint{1.620905in}{2.336454in}}{\pgfqpoint{1.631955in}{2.336454in}}%
\pgfpathclose%
\pgfusepath{stroke,fill}%
\end{pgfscope}%
\begin{pgfscope}%
\pgfpathrectangle{\pgfqpoint{0.787074in}{0.548769in}}{\pgfqpoint{5.062926in}{3.102590in}}%
\pgfusepath{clip}%
\pgfsetbuttcap%
\pgfsetroundjoin%
\definecolor{currentfill}{rgb}{1.000000,0.498039,0.054902}%
\pgfsetfillcolor{currentfill}%
\pgfsetlinewidth{1.003750pt}%
\definecolor{currentstroke}{rgb}{1.000000,0.498039,0.054902}%
\pgfsetstrokecolor{currentstroke}%
\pgfsetdash{}{0pt}%
\pgfpathmoveto{\pgfqpoint{1.255687in}{2.322912in}}%
\pgfpathcurveto{\pgfqpoint{1.266737in}{2.322912in}}{\pgfqpoint{1.277336in}{2.327302in}}{\pgfqpoint{1.285149in}{2.335116in}}%
\pgfpathcurveto{\pgfqpoint{1.292963in}{2.342929in}}{\pgfqpoint{1.297353in}{2.353528in}}{\pgfqpoint{1.297353in}{2.364578in}}%
\pgfpathcurveto{\pgfqpoint{1.297353in}{2.375629in}}{\pgfqpoint{1.292963in}{2.386228in}}{\pgfqpoint{1.285149in}{2.394041in}}%
\pgfpathcurveto{\pgfqpoint{1.277336in}{2.401855in}}{\pgfqpoint{1.266737in}{2.406245in}}{\pgfqpoint{1.255687in}{2.406245in}}%
\pgfpathcurveto{\pgfqpoint{1.244637in}{2.406245in}}{\pgfqpoint{1.234037in}{2.401855in}}{\pgfqpoint{1.226224in}{2.394041in}}%
\pgfpathcurveto{\pgfqpoint{1.218410in}{2.386228in}}{\pgfqpoint{1.214020in}{2.375629in}}{\pgfqpoint{1.214020in}{2.364578in}}%
\pgfpathcurveto{\pgfqpoint{1.214020in}{2.353528in}}{\pgfqpoint{1.218410in}{2.342929in}}{\pgfqpoint{1.226224in}{2.335116in}}%
\pgfpathcurveto{\pgfqpoint{1.234037in}{2.327302in}}{\pgfqpoint{1.244637in}{2.322912in}}{\pgfqpoint{1.255687in}{2.322912in}}%
\pgfpathclose%
\pgfusepath{stroke,fill}%
\end{pgfscope}%
\begin{pgfscope}%
\pgfpathrectangle{\pgfqpoint{0.787074in}{0.548769in}}{\pgfqpoint{5.062926in}{3.102590in}}%
\pgfusepath{clip}%
\pgfsetbuttcap%
\pgfsetroundjoin%
\definecolor{currentfill}{rgb}{1.000000,0.498039,0.054902}%
\pgfsetfillcolor{currentfill}%
\pgfsetlinewidth{1.003750pt}%
\definecolor{currentstroke}{rgb}{1.000000,0.498039,0.054902}%
\pgfsetstrokecolor{currentstroke}%
\pgfsetdash{}{0pt}%
\pgfpathmoveto{\pgfqpoint{2.489820in}{1.553412in}}%
\pgfpathcurveto{\pgfqpoint{2.500870in}{1.553412in}}{\pgfqpoint{2.511469in}{1.557802in}}{\pgfqpoint{2.519282in}{1.565616in}}%
\pgfpathcurveto{\pgfqpoint{2.527096in}{1.573429in}}{\pgfqpoint{2.531486in}{1.584029in}}{\pgfqpoint{2.531486in}{1.595079in}}%
\pgfpathcurveto{\pgfqpoint{2.531486in}{1.606129in}}{\pgfqpoint{2.527096in}{1.616728in}}{\pgfqpoint{2.519282in}{1.624541in}}%
\pgfpathcurveto{\pgfqpoint{2.511469in}{1.632355in}}{\pgfqpoint{2.500870in}{1.636745in}}{\pgfqpoint{2.489820in}{1.636745in}}%
\pgfpathcurveto{\pgfqpoint{2.478769in}{1.636745in}}{\pgfqpoint{2.468170in}{1.632355in}}{\pgfqpoint{2.460357in}{1.624541in}}%
\pgfpathcurveto{\pgfqpoint{2.452543in}{1.616728in}}{\pgfqpoint{2.448153in}{1.606129in}}{\pgfqpoint{2.448153in}{1.595079in}}%
\pgfpathcurveto{\pgfqpoint{2.448153in}{1.584029in}}{\pgfqpoint{2.452543in}{1.573429in}}{\pgfqpoint{2.460357in}{1.565616in}}%
\pgfpathcurveto{\pgfqpoint{2.468170in}{1.557802in}}{\pgfqpoint{2.478769in}{1.553412in}}{\pgfqpoint{2.489820in}{1.553412in}}%
\pgfpathclose%
\pgfusepath{stroke,fill}%
\end{pgfscope}%
\begin{pgfscope}%
\pgfpathrectangle{\pgfqpoint{0.787074in}{0.548769in}}{\pgfqpoint{5.062926in}{3.102590in}}%
\pgfusepath{clip}%
\pgfsetbuttcap%
\pgfsetroundjoin%
\definecolor{currentfill}{rgb}{1.000000,0.498039,0.054902}%
\pgfsetfillcolor{currentfill}%
\pgfsetlinewidth{1.003750pt}%
\definecolor{currentstroke}{rgb}{1.000000,0.498039,0.054902}%
\pgfsetstrokecolor{currentstroke}%
\pgfsetdash{}{0pt}%
\pgfpathmoveto{\pgfqpoint{1.642866in}{2.076652in}}%
\pgfpathcurveto{\pgfqpoint{1.653916in}{2.076652in}}{\pgfqpoint{1.664515in}{2.081042in}}{\pgfqpoint{1.672328in}{2.088856in}}%
\pgfpathcurveto{\pgfqpoint{1.680142in}{2.096669in}}{\pgfqpoint{1.684532in}{2.107268in}}{\pgfqpoint{1.684532in}{2.118319in}}%
\pgfpathcurveto{\pgfqpoint{1.684532in}{2.129369in}}{\pgfqpoint{1.680142in}{2.139968in}}{\pgfqpoint{1.672328in}{2.147781in}}%
\pgfpathcurveto{\pgfqpoint{1.664515in}{2.155595in}}{\pgfqpoint{1.653916in}{2.159985in}}{\pgfqpoint{1.642866in}{2.159985in}}%
\pgfpathcurveto{\pgfqpoint{1.631815in}{2.159985in}}{\pgfqpoint{1.621216in}{2.155595in}}{\pgfqpoint{1.613403in}{2.147781in}}%
\pgfpathcurveto{\pgfqpoint{1.605589in}{2.139968in}}{\pgfqpoint{1.601199in}{2.129369in}}{\pgfqpoint{1.601199in}{2.118319in}}%
\pgfpathcurveto{\pgfqpoint{1.601199in}{2.107268in}}{\pgfqpoint{1.605589in}{2.096669in}}{\pgfqpoint{1.613403in}{2.088856in}}%
\pgfpathcurveto{\pgfqpoint{1.621216in}{2.081042in}}{\pgfqpoint{1.631815in}{2.076652in}}{\pgfqpoint{1.642866in}{2.076652in}}%
\pgfpathclose%
\pgfusepath{stroke,fill}%
\end{pgfscope}%
\begin{pgfscope}%
\pgfpathrectangle{\pgfqpoint{0.787074in}{0.548769in}}{\pgfqpoint{5.062926in}{3.102590in}}%
\pgfusepath{clip}%
\pgfsetbuttcap%
\pgfsetroundjoin%
\definecolor{currentfill}{rgb}{1.000000,0.498039,0.054902}%
\pgfsetfillcolor{currentfill}%
\pgfsetlinewidth{1.003750pt}%
\definecolor{currentstroke}{rgb}{1.000000,0.498039,0.054902}%
\pgfsetstrokecolor{currentstroke}%
\pgfsetdash{}{0pt}%
\pgfpathmoveto{\pgfqpoint{2.195978in}{2.475226in}}%
\pgfpathcurveto{\pgfqpoint{2.207029in}{2.475226in}}{\pgfqpoint{2.217628in}{2.479616in}}{\pgfqpoint{2.225441in}{2.487430in}}%
\pgfpathcurveto{\pgfqpoint{2.233255in}{2.495243in}}{\pgfqpoint{2.237645in}{2.505842in}}{\pgfqpoint{2.237645in}{2.516893in}}%
\pgfpathcurveto{\pgfqpoint{2.237645in}{2.527943in}}{\pgfqpoint{2.233255in}{2.538542in}}{\pgfqpoint{2.225441in}{2.546355in}}%
\pgfpathcurveto{\pgfqpoint{2.217628in}{2.554169in}}{\pgfqpoint{2.207029in}{2.558559in}}{\pgfqpoint{2.195978in}{2.558559in}}%
\pgfpathcurveto{\pgfqpoint{2.184928in}{2.558559in}}{\pgfqpoint{2.174329in}{2.554169in}}{\pgfqpoint{2.166516in}{2.546355in}}%
\pgfpathcurveto{\pgfqpoint{2.158702in}{2.538542in}}{\pgfqpoint{2.154312in}{2.527943in}}{\pgfqpoint{2.154312in}{2.516893in}}%
\pgfpathcurveto{\pgfqpoint{2.154312in}{2.505842in}}{\pgfqpoint{2.158702in}{2.495243in}}{\pgfqpoint{2.166516in}{2.487430in}}%
\pgfpathcurveto{\pgfqpoint{2.174329in}{2.479616in}}{\pgfqpoint{2.184928in}{2.475226in}}{\pgfqpoint{2.195978in}{2.475226in}}%
\pgfpathclose%
\pgfusepath{stroke,fill}%
\end{pgfscope}%
\begin{pgfscope}%
\pgfpathrectangle{\pgfqpoint{0.787074in}{0.548769in}}{\pgfqpoint{5.062926in}{3.102590in}}%
\pgfusepath{clip}%
\pgfsetbuttcap%
\pgfsetroundjoin%
\definecolor{currentfill}{rgb}{1.000000,0.498039,0.054902}%
\pgfsetfillcolor{currentfill}%
\pgfsetlinewidth{1.003750pt}%
\definecolor{currentstroke}{rgb}{1.000000,0.498039,0.054902}%
\pgfsetstrokecolor{currentstroke}%
\pgfsetdash{}{0pt}%
\pgfpathmoveto{\pgfqpoint{1.525964in}{2.253097in}}%
\pgfpathcurveto{\pgfqpoint{1.537014in}{2.253097in}}{\pgfqpoint{1.547613in}{2.257487in}}{\pgfqpoint{1.555427in}{2.265300in}}%
\pgfpathcurveto{\pgfqpoint{1.563240in}{2.273114in}}{\pgfqpoint{1.567630in}{2.283713in}}{\pgfqpoint{1.567630in}{2.294763in}}%
\pgfpathcurveto{\pgfqpoint{1.567630in}{2.305813in}}{\pgfqpoint{1.563240in}{2.316412in}}{\pgfqpoint{1.555427in}{2.324226in}}%
\pgfpathcurveto{\pgfqpoint{1.547613in}{2.332040in}}{\pgfqpoint{1.537014in}{2.336430in}}{\pgfqpoint{1.525964in}{2.336430in}}%
\pgfpathcurveto{\pgfqpoint{1.514914in}{2.336430in}}{\pgfqpoint{1.504315in}{2.332040in}}{\pgfqpoint{1.496501in}{2.324226in}}%
\pgfpathcurveto{\pgfqpoint{1.488687in}{2.316412in}}{\pgfqpoint{1.484297in}{2.305813in}}{\pgfqpoint{1.484297in}{2.294763in}}%
\pgfpathcurveto{\pgfqpoint{1.484297in}{2.283713in}}{\pgfqpoint{1.488687in}{2.273114in}}{\pgfqpoint{1.496501in}{2.265300in}}%
\pgfpathcurveto{\pgfqpoint{1.504315in}{2.257487in}}{\pgfqpoint{1.514914in}{2.253097in}}{\pgfqpoint{1.525964in}{2.253097in}}%
\pgfpathclose%
\pgfusepath{stroke,fill}%
\end{pgfscope}%
\begin{pgfscope}%
\pgfpathrectangle{\pgfqpoint{0.787074in}{0.548769in}}{\pgfqpoint{5.062926in}{3.102590in}}%
\pgfusepath{clip}%
\pgfsetbuttcap%
\pgfsetroundjoin%
\definecolor{currentfill}{rgb}{0.121569,0.466667,0.705882}%
\pgfsetfillcolor{currentfill}%
\pgfsetlinewidth{1.003750pt}%
\definecolor{currentstroke}{rgb}{0.121569,0.466667,0.705882}%
\pgfsetstrokecolor{currentstroke}%
\pgfsetdash{}{0pt}%
\pgfpathmoveto{\pgfqpoint{1.971126in}{0.659516in}}%
\pgfpathcurveto{\pgfqpoint{1.982176in}{0.659516in}}{\pgfqpoint{1.992775in}{0.663906in}}{\pgfqpoint{2.000589in}{0.671719in}}%
\pgfpathcurveto{\pgfqpoint{2.008402in}{0.679533in}}{\pgfqpoint{2.012793in}{0.690132in}}{\pgfqpoint{2.012793in}{0.701182in}}%
\pgfpathcurveto{\pgfqpoint{2.012793in}{0.712232in}}{\pgfqpoint{2.008402in}{0.722831in}}{\pgfqpoint{2.000589in}{0.730645in}}%
\pgfpathcurveto{\pgfqpoint{1.992775in}{0.738459in}}{\pgfqpoint{1.982176in}{0.742849in}}{\pgfqpoint{1.971126in}{0.742849in}}%
\pgfpathcurveto{\pgfqpoint{1.960076in}{0.742849in}}{\pgfqpoint{1.949477in}{0.738459in}}{\pgfqpoint{1.941663in}{0.730645in}}%
\pgfpathcurveto{\pgfqpoint{1.933850in}{0.722831in}}{\pgfqpoint{1.929459in}{0.712232in}}{\pgfqpoint{1.929459in}{0.701182in}}%
\pgfpathcurveto{\pgfqpoint{1.929459in}{0.690132in}}{\pgfqpoint{1.933850in}{0.679533in}}{\pgfqpoint{1.941663in}{0.671719in}}%
\pgfpathcurveto{\pgfqpoint{1.949477in}{0.663906in}}{\pgfqpoint{1.960076in}{0.659516in}}{\pgfqpoint{1.971126in}{0.659516in}}%
\pgfpathclose%
\pgfusepath{stroke,fill}%
\end{pgfscope}%
\begin{pgfscope}%
\pgfpathrectangle{\pgfqpoint{0.787074in}{0.548769in}}{\pgfqpoint{5.062926in}{3.102590in}}%
\pgfusepath{clip}%
\pgfsetbuttcap%
\pgfsetroundjoin%
\definecolor{currentfill}{rgb}{1.000000,0.498039,0.054902}%
\pgfsetfillcolor{currentfill}%
\pgfsetlinewidth{1.003750pt}%
\definecolor{currentstroke}{rgb}{1.000000,0.498039,0.054902}%
\pgfsetstrokecolor{currentstroke}%
\pgfsetdash{}{0pt}%
\pgfpathmoveto{\pgfqpoint{1.128497in}{2.129288in}}%
\pgfpathcurveto{\pgfqpoint{1.139548in}{2.129288in}}{\pgfqpoint{1.150147in}{2.133678in}}{\pgfqpoint{1.157960in}{2.141492in}}%
\pgfpathcurveto{\pgfqpoint{1.165774in}{2.149306in}}{\pgfqpoint{1.170164in}{2.159905in}}{\pgfqpoint{1.170164in}{2.170955in}}%
\pgfpathcurveto{\pgfqpoint{1.170164in}{2.182005in}}{\pgfqpoint{1.165774in}{2.192604in}}{\pgfqpoint{1.157960in}{2.200418in}}%
\pgfpathcurveto{\pgfqpoint{1.150147in}{2.208231in}}{\pgfqpoint{1.139548in}{2.212621in}}{\pgfqpoint{1.128497in}{2.212621in}}%
\pgfpathcurveto{\pgfqpoint{1.117447in}{2.212621in}}{\pgfqpoint{1.106848in}{2.208231in}}{\pgfqpoint{1.099035in}{2.200418in}}%
\pgfpathcurveto{\pgfqpoint{1.091221in}{2.192604in}}{\pgfqpoint{1.086831in}{2.182005in}}{\pgfqpoint{1.086831in}{2.170955in}}%
\pgfpathcurveto{\pgfqpoint{1.086831in}{2.159905in}}{\pgfqpoint{1.091221in}{2.149306in}}{\pgfqpoint{1.099035in}{2.141492in}}%
\pgfpathcurveto{\pgfqpoint{1.106848in}{2.133678in}}{\pgfqpoint{1.117447in}{2.129288in}}{\pgfqpoint{1.128497in}{2.129288in}}%
\pgfpathclose%
\pgfusepath{stroke,fill}%
\end{pgfscope}%
\begin{pgfscope}%
\pgfpathrectangle{\pgfqpoint{0.787074in}{0.548769in}}{\pgfqpoint{5.062926in}{3.102590in}}%
\pgfusepath{clip}%
\pgfsetbuttcap%
\pgfsetroundjoin%
\definecolor{currentfill}{rgb}{1.000000,0.498039,0.054902}%
\pgfsetfillcolor{currentfill}%
\pgfsetlinewidth{1.003750pt}%
\definecolor{currentstroke}{rgb}{1.000000,0.498039,0.054902}%
\pgfsetstrokecolor{currentstroke}%
\pgfsetdash{}{0pt}%
\pgfpathmoveto{\pgfqpoint{1.170728in}{2.778374in}}%
\pgfpathcurveto{\pgfqpoint{1.181778in}{2.778374in}}{\pgfqpoint{1.192377in}{2.782764in}}{\pgfqpoint{1.200191in}{2.790577in}}%
\pgfpathcurveto{\pgfqpoint{1.208005in}{2.798391in}}{\pgfqpoint{1.212395in}{2.808990in}}{\pgfqpoint{1.212395in}{2.820040in}}%
\pgfpathcurveto{\pgfqpoint{1.212395in}{2.831090in}}{\pgfqpoint{1.208005in}{2.841689in}}{\pgfqpoint{1.200191in}{2.849503in}}%
\pgfpathcurveto{\pgfqpoint{1.192377in}{2.857317in}}{\pgfqpoint{1.181778in}{2.861707in}}{\pgfqpoint{1.170728in}{2.861707in}}%
\pgfpathcurveto{\pgfqpoint{1.159678in}{2.861707in}}{\pgfqpoint{1.149079in}{2.857317in}}{\pgfqpoint{1.141265in}{2.849503in}}%
\pgfpathcurveto{\pgfqpoint{1.133452in}{2.841689in}}{\pgfqpoint{1.129062in}{2.831090in}}{\pgfqpoint{1.129062in}{2.820040in}}%
\pgfpathcurveto{\pgfqpoint{1.129062in}{2.808990in}}{\pgfqpoint{1.133452in}{2.798391in}}{\pgfqpoint{1.141265in}{2.790577in}}%
\pgfpathcurveto{\pgfqpoint{1.149079in}{2.782764in}}{\pgfqpoint{1.159678in}{2.778374in}}{\pgfqpoint{1.170728in}{2.778374in}}%
\pgfpathclose%
\pgfusepath{stroke,fill}%
\end{pgfscope}%
\begin{pgfscope}%
\pgfpathrectangle{\pgfqpoint{0.787074in}{0.548769in}}{\pgfqpoint{5.062926in}{3.102590in}}%
\pgfusepath{clip}%
\pgfsetbuttcap%
\pgfsetroundjoin%
\definecolor{currentfill}{rgb}{1.000000,0.498039,0.054902}%
\pgfsetfillcolor{currentfill}%
\pgfsetlinewidth{1.003750pt}%
\definecolor{currentstroke}{rgb}{1.000000,0.498039,0.054902}%
\pgfsetstrokecolor{currentstroke}%
\pgfsetdash{}{0pt}%
\pgfpathmoveto{\pgfqpoint{1.446470in}{2.577708in}}%
\pgfpathcurveto{\pgfqpoint{1.457521in}{2.577708in}}{\pgfqpoint{1.468120in}{2.582098in}}{\pgfqpoint{1.475933in}{2.589912in}}%
\pgfpathcurveto{\pgfqpoint{1.483747in}{2.597725in}}{\pgfqpoint{1.488137in}{2.608324in}}{\pgfqpoint{1.488137in}{2.619374in}}%
\pgfpathcurveto{\pgfqpoint{1.488137in}{2.630425in}}{\pgfqpoint{1.483747in}{2.641024in}}{\pgfqpoint{1.475933in}{2.648837in}}%
\pgfpathcurveto{\pgfqpoint{1.468120in}{2.656651in}}{\pgfqpoint{1.457521in}{2.661041in}}{\pgfqpoint{1.446470in}{2.661041in}}%
\pgfpathcurveto{\pgfqpoint{1.435420in}{2.661041in}}{\pgfqpoint{1.424821in}{2.656651in}}{\pgfqpoint{1.417008in}{2.648837in}}%
\pgfpathcurveto{\pgfqpoint{1.409194in}{2.641024in}}{\pgfqpoint{1.404804in}{2.630425in}}{\pgfqpoint{1.404804in}{2.619374in}}%
\pgfpathcurveto{\pgfqpoint{1.404804in}{2.608324in}}{\pgfqpoint{1.409194in}{2.597725in}}{\pgfqpoint{1.417008in}{2.589912in}}%
\pgfpathcurveto{\pgfqpoint{1.424821in}{2.582098in}}{\pgfqpoint{1.435420in}{2.577708in}}{\pgfqpoint{1.446470in}{2.577708in}}%
\pgfpathclose%
\pgfusepath{stroke,fill}%
\end{pgfscope}%
\begin{pgfscope}%
\pgfpathrectangle{\pgfqpoint{0.787074in}{0.548769in}}{\pgfqpoint{5.062926in}{3.102590in}}%
\pgfusepath{clip}%
\pgfsetbuttcap%
\pgfsetroundjoin%
\definecolor{currentfill}{rgb}{1.000000,0.498039,0.054902}%
\pgfsetfillcolor{currentfill}%
\pgfsetlinewidth{1.003750pt}%
\definecolor{currentstroke}{rgb}{1.000000,0.498039,0.054902}%
\pgfsetstrokecolor{currentstroke}%
\pgfsetdash{}{0pt}%
\pgfpathmoveto{\pgfqpoint{1.398775in}{2.595506in}}%
\pgfpathcurveto{\pgfqpoint{1.409825in}{2.595506in}}{\pgfqpoint{1.420424in}{2.599897in}}{\pgfqpoint{1.428237in}{2.607710in}}%
\pgfpathcurveto{\pgfqpoint{1.436051in}{2.615524in}}{\pgfqpoint{1.440441in}{2.626123in}}{\pgfqpoint{1.440441in}{2.637173in}}%
\pgfpathcurveto{\pgfqpoint{1.440441in}{2.648223in}}{\pgfqpoint{1.436051in}{2.658822in}}{\pgfqpoint{1.428237in}{2.666636in}}%
\pgfpathcurveto{\pgfqpoint{1.420424in}{2.674449in}}{\pgfqpoint{1.409825in}{2.678840in}}{\pgfqpoint{1.398775in}{2.678840in}}%
\pgfpathcurveto{\pgfqpoint{1.387724in}{2.678840in}}{\pgfqpoint{1.377125in}{2.674449in}}{\pgfqpoint{1.369312in}{2.666636in}}%
\pgfpathcurveto{\pgfqpoint{1.361498in}{2.658822in}}{\pgfqpoint{1.357108in}{2.648223in}}{\pgfqpoint{1.357108in}{2.637173in}}%
\pgfpathcurveto{\pgfqpoint{1.357108in}{2.626123in}}{\pgfqpoint{1.361498in}{2.615524in}}{\pgfqpoint{1.369312in}{2.607710in}}%
\pgfpathcurveto{\pgfqpoint{1.377125in}{2.599897in}}{\pgfqpoint{1.387724in}{2.595506in}}{\pgfqpoint{1.398775in}{2.595506in}}%
\pgfpathclose%
\pgfusepath{stroke,fill}%
\end{pgfscope}%
\begin{pgfscope}%
\pgfpathrectangle{\pgfqpoint{0.787074in}{0.548769in}}{\pgfqpoint{5.062926in}{3.102590in}}%
\pgfusepath{clip}%
\pgfsetbuttcap%
\pgfsetroundjoin%
\definecolor{currentfill}{rgb}{1.000000,0.498039,0.054902}%
\pgfsetfillcolor{currentfill}%
\pgfsetlinewidth{1.003750pt}%
\definecolor{currentstroke}{rgb}{1.000000,0.498039,0.054902}%
\pgfsetstrokecolor{currentstroke}%
\pgfsetdash{}{0pt}%
\pgfpathmoveto{\pgfqpoint{2.195978in}{1.910607in}}%
\pgfpathcurveto{\pgfqpoint{2.207029in}{1.910607in}}{\pgfqpoint{2.217628in}{1.914997in}}{\pgfqpoint{2.225441in}{1.922810in}}%
\pgfpathcurveto{\pgfqpoint{2.233255in}{1.930624in}}{\pgfqpoint{2.237645in}{1.941223in}}{\pgfqpoint{2.237645in}{1.952273in}}%
\pgfpathcurveto{\pgfqpoint{2.237645in}{1.963323in}}{\pgfqpoint{2.233255in}{1.973922in}}{\pgfqpoint{2.225441in}{1.981736in}}%
\pgfpathcurveto{\pgfqpoint{2.217628in}{1.989550in}}{\pgfqpoint{2.207029in}{1.993940in}}{\pgfqpoint{2.195978in}{1.993940in}}%
\pgfpathcurveto{\pgfqpoint{2.184928in}{1.993940in}}{\pgfqpoint{2.174329in}{1.989550in}}{\pgfqpoint{2.166516in}{1.981736in}}%
\pgfpathcurveto{\pgfqpoint{2.158702in}{1.973922in}}{\pgfqpoint{2.154312in}{1.963323in}}{\pgfqpoint{2.154312in}{1.952273in}}%
\pgfpathcurveto{\pgfqpoint{2.154312in}{1.941223in}}{\pgfqpoint{2.158702in}{1.930624in}}{\pgfqpoint{2.166516in}{1.922810in}}%
\pgfpathcurveto{\pgfqpoint{2.174329in}{1.914997in}}{\pgfqpoint{2.184928in}{1.910607in}}{\pgfqpoint{2.195978in}{1.910607in}}%
\pgfpathclose%
\pgfusepath{stroke,fill}%
\end{pgfscope}%
\begin{pgfscope}%
\pgfpathrectangle{\pgfqpoint{0.787074in}{0.548769in}}{\pgfqpoint{5.062926in}{3.102590in}}%
\pgfusepath{clip}%
\pgfsetbuttcap%
\pgfsetroundjoin%
\definecolor{currentfill}{rgb}{1.000000,0.498039,0.054902}%
\pgfsetfillcolor{currentfill}%
\pgfsetlinewidth{1.003750pt}%
\definecolor{currentstroke}{rgb}{1.000000,0.498039,0.054902}%
\pgfsetstrokecolor{currentstroke}%
\pgfsetdash{}{0pt}%
\pgfpathmoveto{\pgfqpoint{1.619910in}{1.406572in}}%
\pgfpathcurveto{\pgfqpoint{1.630960in}{1.406572in}}{\pgfqpoint{1.641560in}{1.410962in}}{\pgfqpoint{1.649373in}{1.418775in}}%
\pgfpathcurveto{\pgfqpoint{1.657187in}{1.426589in}}{\pgfqpoint{1.661577in}{1.437188in}}{\pgfqpoint{1.661577in}{1.448238in}}%
\pgfpathcurveto{\pgfqpoint{1.661577in}{1.459288in}}{\pgfqpoint{1.657187in}{1.469887in}}{\pgfqpoint{1.649373in}{1.477701in}}%
\pgfpathcurveto{\pgfqpoint{1.641560in}{1.485515in}}{\pgfqpoint{1.630960in}{1.489905in}}{\pgfqpoint{1.619910in}{1.489905in}}%
\pgfpathcurveto{\pgfqpoint{1.608860in}{1.489905in}}{\pgfqpoint{1.598261in}{1.485515in}}{\pgfqpoint{1.590448in}{1.477701in}}%
\pgfpathcurveto{\pgfqpoint{1.582634in}{1.469887in}}{\pgfqpoint{1.578244in}{1.459288in}}{\pgfqpoint{1.578244in}{1.448238in}}%
\pgfpathcurveto{\pgfqpoint{1.578244in}{1.437188in}}{\pgfqpoint{1.582634in}{1.426589in}}{\pgfqpoint{1.590448in}{1.418775in}}%
\pgfpathcurveto{\pgfqpoint{1.598261in}{1.410962in}}{\pgfqpoint{1.608860in}{1.406572in}}{\pgfqpoint{1.619910in}{1.406572in}}%
\pgfpathclose%
\pgfusepath{stroke,fill}%
\end{pgfscope}%
\begin{pgfscope}%
\pgfpathrectangle{\pgfqpoint{0.787074in}{0.548769in}}{\pgfqpoint{5.062926in}{3.102590in}}%
\pgfusepath{clip}%
\pgfsetbuttcap%
\pgfsetroundjoin%
\definecolor{currentfill}{rgb}{1.000000,0.498039,0.054902}%
\pgfsetfillcolor{currentfill}%
\pgfsetlinewidth{1.003750pt}%
\definecolor{currentstroke}{rgb}{1.000000,0.498039,0.054902}%
\pgfsetstrokecolor{currentstroke}%
\pgfsetdash{}{0pt}%
\pgfpathmoveto{\pgfqpoint{1.346058in}{2.078914in}}%
\pgfpathcurveto{\pgfqpoint{1.357108in}{2.078914in}}{\pgfqpoint{1.367707in}{2.083304in}}{\pgfqpoint{1.375521in}{2.091118in}}%
\pgfpathcurveto{\pgfqpoint{1.383334in}{2.098931in}}{\pgfqpoint{1.387725in}{2.109530in}}{\pgfqpoint{1.387725in}{2.120580in}}%
\pgfpathcurveto{\pgfqpoint{1.387725in}{2.131630in}}{\pgfqpoint{1.383334in}{2.142229in}}{\pgfqpoint{1.375521in}{2.150043in}}%
\pgfpathcurveto{\pgfqpoint{1.367707in}{2.157857in}}{\pgfqpoint{1.357108in}{2.162247in}}{\pgfqpoint{1.346058in}{2.162247in}}%
\pgfpathcurveto{\pgfqpoint{1.335008in}{2.162247in}}{\pgfqpoint{1.324409in}{2.157857in}}{\pgfqpoint{1.316595in}{2.150043in}}%
\pgfpathcurveto{\pgfqpoint{1.308782in}{2.142229in}}{\pgfqpoint{1.304391in}{2.131630in}}{\pgfqpoint{1.304391in}{2.120580in}}%
\pgfpathcurveto{\pgfqpoint{1.304391in}{2.109530in}}{\pgfqpoint{1.308782in}{2.098931in}}{\pgfqpoint{1.316595in}{2.091118in}}%
\pgfpathcurveto{\pgfqpoint{1.324409in}{2.083304in}}{\pgfqpoint{1.335008in}{2.078914in}}{\pgfqpoint{1.346058in}{2.078914in}}%
\pgfpathclose%
\pgfusepath{stroke,fill}%
\end{pgfscope}%
\begin{pgfscope}%
\pgfpathrectangle{\pgfqpoint{0.787074in}{0.548769in}}{\pgfqpoint{5.062926in}{3.102590in}}%
\pgfusepath{clip}%
\pgfsetbuttcap%
\pgfsetroundjoin%
\definecolor{currentfill}{rgb}{1.000000,0.498039,0.054902}%
\pgfsetfillcolor{currentfill}%
\pgfsetlinewidth{1.003750pt}%
\definecolor{currentstroke}{rgb}{1.000000,0.498039,0.054902}%
\pgfsetstrokecolor{currentstroke}%
\pgfsetdash{}{0pt}%
\pgfpathmoveto{\pgfqpoint{1.531642in}{1.659203in}}%
\pgfpathcurveto{\pgfqpoint{1.542692in}{1.659203in}}{\pgfqpoint{1.553291in}{1.663593in}}{\pgfqpoint{1.561105in}{1.671407in}}%
\pgfpathcurveto{\pgfqpoint{1.568918in}{1.679220in}}{\pgfqpoint{1.573309in}{1.689820in}}{\pgfqpoint{1.573309in}{1.700870in}}%
\pgfpathcurveto{\pgfqpoint{1.573309in}{1.711920in}}{\pgfqpoint{1.568918in}{1.722519in}}{\pgfqpoint{1.561105in}{1.730332in}}%
\pgfpathcurveto{\pgfqpoint{1.553291in}{1.738146in}}{\pgfqpoint{1.542692in}{1.742536in}}{\pgfqpoint{1.531642in}{1.742536in}}%
\pgfpathcurveto{\pgfqpoint{1.520592in}{1.742536in}}{\pgfqpoint{1.509993in}{1.738146in}}{\pgfqpoint{1.502179in}{1.730332in}}%
\pgfpathcurveto{\pgfqpoint{1.494365in}{1.722519in}}{\pgfqpoint{1.489975in}{1.711920in}}{\pgfqpoint{1.489975in}{1.700870in}}%
\pgfpathcurveto{\pgfqpoint{1.489975in}{1.689820in}}{\pgfqpoint{1.494365in}{1.679220in}}{\pgfqpoint{1.502179in}{1.671407in}}%
\pgfpathcurveto{\pgfqpoint{1.509993in}{1.663593in}}{\pgfqpoint{1.520592in}{1.659203in}}{\pgfqpoint{1.531642in}{1.659203in}}%
\pgfpathclose%
\pgfusepath{stroke,fill}%
\end{pgfscope}%
\begin{pgfscope}%
\pgfpathrectangle{\pgfqpoint{0.787074in}{0.548769in}}{\pgfqpoint{5.062926in}{3.102590in}}%
\pgfusepath{clip}%
\pgfsetbuttcap%
\pgfsetroundjoin%
\definecolor{currentfill}{rgb}{0.121569,0.466667,0.705882}%
\pgfsetfillcolor{currentfill}%
\pgfsetlinewidth{1.003750pt}%
\definecolor{currentstroke}{rgb}{0.121569,0.466667,0.705882}%
\pgfsetstrokecolor{currentstroke}%
\pgfsetdash{}{0pt}%
\pgfpathmoveto{\pgfqpoint{1.029971in}{1.472684in}}%
\pgfpathcurveto{\pgfqpoint{1.041021in}{1.472684in}}{\pgfqpoint{1.051620in}{1.477075in}}{\pgfqpoint{1.059433in}{1.484888in}}%
\pgfpathcurveto{\pgfqpoint{1.067247in}{1.492702in}}{\pgfqpoint{1.071637in}{1.503301in}}{\pgfqpoint{1.071637in}{1.514351in}}%
\pgfpathcurveto{\pgfqpoint{1.071637in}{1.525401in}}{\pgfqpoint{1.067247in}{1.536000in}}{\pgfqpoint{1.059433in}{1.543814in}}%
\pgfpathcurveto{\pgfqpoint{1.051620in}{1.551627in}}{\pgfqpoint{1.041021in}{1.556018in}}{\pgfqpoint{1.029971in}{1.556018in}}%
\pgfpathcurveto{\pgfqpoint{1.018920in}{1.556018in}}{\pgfqpoint{1.008321in}{1.551627in}}{\pgfqpoint{1.000508in}{1.543814in}}%
\pgfpathcurveto{\pgfqpoint{0.992694in}{1.536000in}}{\pgfqpoint{0.988304in}{1.525401in}}{\pgfqpoint{0.988304in}{1.514351in}}%
\pgfpathcurveto{\pgfqpoint{0.988304in}{1.503301in}}{\pgfqpoint{0.992694in}{1.492702in}}{\pgfqpoint{1.000508in}{1.484888in}}%
\pgfpathcurveto{\pgfqpoint{1.008321in}{1.477075in}}{\pgfqpoint{1.018920in}{1.472684in}}{\pgfqpoint{1.029971in}{1.472684in}}%
\pgfpathclose%
\pgfusepath{stroke,fill}%
\end{pgfscope}%
\begin{pgfscope}%
\pgfpathrectangle{\pgfqpoint{0.787074in}{0.548769in}}{\pgfqpoint{5.062926in}{3.102590in}}%
\pgfusepath{clip}%
\pgfsetbuttcap%
\pgfsetroundjoin%
\definecolor{currentfill}{rgb}{1.000000,0.498039,0.054902}%
\pgfsetfillcolor{currentfill}%
\pgfsetlinewidth{1.003750pt}%
\definecolor{currentstroke}{rgb}{1.000000,0.498039,0.054902}%
\pgfsetstrokecolor{currentstroke}%
\pgfsetdash{}{0pt}%
\pgfpathmoveto{\pgfqpoint{2.195978in}{2.453351in}}%
\pgfpathcurveto{\pgfqpoint{2.207029in}{2.453351in}}{\pgfqpoint{2.217628in}{2.457742in}}{\pgfqpoint{2.225441in}{2.465555in}}%
\pgfpathcurveto{\pgfqpoint{2.233255in}{2.473369in}}{\pgfqpoint{2.237645in}{2.483968in}}{\pgfqpoint{2.237645in}{2.495018in}}%
\pgfpathcurveto{\pgfqpoint{2.237645in}{2.506068in}}{\pgfqpoint{2.233255in}{2.516667in}}{\pgfqpoint{2.225441in}{2.524481in}}%
\pgfpathcurveto{\pgfqpoint{2.217628in}{2.532294in}}{\pgfqpoint{2.207029in}{2.536685in}}{\pgfqpoint{2.195978in}{2.536685in}}%
\pgfpathcurveto{\pgfqpoint{2.184928in}{2.536685in}}{\pgfqpoint{2.174329in}{2.532294in}}{\pgfqpoint{2.166516in}{2.524481in}}%
\pgfpathcurveto{\pgfqpoint{2.158702in}{2.516667in}}{\pgfqpoint{2.154312in}{2.506068in}}{\pgfqpoint{2.154312in}{2.495018in}}%
\pgfpathcurveto{\pgfqpoint{2.154312in}{2.483968in}}{\pgfqpoint{2.158702in}{2.473369in}}{\pgfqpoint{2.166516in}{2.465555in}}%
\pgfpathcurveto{\pgfqpoint{2.174329in}{2.457742in}}{\pgfqpoint{2.184928in}{2.453351in}}{\pgfqpoint{2.195978in}{2.453351in}}%
\pgfpathclose%
\pgfusepath{stroke,fill}%
\end{pgfscope}%
\begin{pgfscope}%
\pgfpathrectangle{\pgfqpoint{0.787074in}{0.548769in}}{\pgfqpoint{5.062926in}{3.102590in}}%
\pgfusepath{clip}%
\pgfsetbuttcap%
\pgfsetroundjoin%
\definecolor{currentfill}{rgb}{0.121569,0.466667,0.705882}%
\pgfsetfillcolor{currentfill}%
\pgfsetlinewidth{1.003750pt}%
\definecolor{currentstroke}{rgb}{0.121569,0.466667,0.705882}%
\pgfsetstrokecolor{currentstroke}%
\pgfsetdash{}{0pt}%
\pgfpathmoveto{\pgfqpoint{3.115829in}{0.648305in}}%
\pgfpathcurveto{\pgfqpoint{3.126879in}{0.648305in}}{\pgfqpoint{3.137478in}{0.652695in}}{\pgfqpoint{3.145292in}{0.660509in}}%
\pgfpathcurveto{\pgfqpoint{3.153105in}{0.668322in}}{\pgfqpoint{3.157496in}{0.678921in}}{\pgfqpoint{3.157496in}{0.689972in}}%
\pgfpathcurveto{\pgfqpoint{3.157496in}{0.701022in}}{\pgfqpoint{3.153105in}{0.711621in}}{\pgfqpoint{3.145292in}{0.719434in}}%
\pgfpathcurveto{\pgfqpoint{3.137478in}{0.727248in}}{\pgfqpoint{3.126879in}{0.731638in}}{\pgfqpoint{3.115829in}{0.731638in}}%
\pgfpathcurveto{\pgfqpoint{3.104779in}{0.731638in}}{\pgfqpoint{3.094180in}{0.727248in}}{\pgfqpoint{3.086366in}{0.719434in}}%
\pgfpathcurveto{\pgfqpoint{3.078553in}{0.711621in}}{\pgfqpoint{3.074162in}{0.701022in}}{\pgfqpoint{3.074162in}{0.689972in}}%
\pgfpathcurveto{\pgfqpoint{3.074162in}{0.678921in}}{\pgfqpoint{3.078553in}{0.668322in}}{\pgfqpoint{3.086366in}{0.660509in}}%
\pgfpathcurveto{\pgfqpoint{3.094180in}{0.652695in}}{\pgfqpoint{3.104779in}{0.648305in}}{\pgfqpoint{3.115829in}{0.648305in}}%
\pgfpathclose%
\pgfusepath{stroke,fill}%
\end{pgfscope}%
\begin{pgfscope}%
\pgfpathrectangle{\pgfqpoint{0.787074in}{0.548769in}}{\pgfqpoint{5.062926in}{3.102590in}}%
\pgfusepath{clip}%
\pgfsetbuttcap%
\pgfsetroundjoin%
\definecolor{currentfill}{rgb}{0.121569,0.466667,0.705882}%
\pgfsetfillcolor{currentfill}%
\pgfsetlinewidth{1.003750pt}%
\definecolor{currentstroke}{rgb}{0.121569,0.466667,0.705882}%
\pgfsetstrokecolor{currentstroke}%
\pgfsetdash{}{0pt}%
\pgfpathmoveto{\pgfqpoint{1.029131in}{1.422655in}}%
\pgfpathcurveto{\pgfqpoint{1.040181in}{1.422655in}}{\pgfqpoint{1.050780in}{1.427045in}}{\pgfqpoint{1.058594in}{1.434858in}}%
\pgfpathcurveto{\pgfqpoint{1.066407in}{1.442672in}}{\pgfqpoint{1.070798in}{1.453271in}}{\pgfqpoint{1.070798in}{1.464321in}}%
\pgfpathcurveto{\pgfqpoint{1.070798in}{1.475371in}}{\pgfqpoint{1.066407in}{1.485970in}}{\pgfqpoint{1.058594in}{1.493784in}}%
\pgfpathcurveto{\pgfqpoint{1.050780in}{1.501598in}}{\pgfqpoint{1.040181in}{1.505988in}}{\pgfqpoint{1.029131in}{1.505988in}}%
\pgfpathcurveto{\pgfqpoint{1.018081in}{1.505988in}}{\pgfqpoint{1.007482in}{1.501598in}}{\pgfqpoint{0.999668in}{1.493784in}}%
\pgfpathcurveto{\pgfqpoint{0.991854in}{1.485970in}}{\pgfqpoint{0.987464in}{1.475371in}}{\pgfqpoint{0.987464in}{1.464321in}}%
\pgfpathcurveto{\pgfqpoint{0.987464in}{1.453271in}}{\pgfqpoint{0.991854in}{1.442672in}}{\pgfqpoint{0.999668in}{1.434858in}}%
\pgfpathcurveto{\pgfqpoint{1.007482in}{1.427045in}}{\pgfqpoint{1.018081in}{1.422655in}}{\pgfqpoint{1.029131in}{1.422655in}}%
\pgfpathclose%
\pgfusepath{stroke,fill}%
\end{pgfscope}%
\begin{pgfscope}%
\pgfpathrectangle{\pgfqpoint{0.787074in}{0.548769in}}{\pgfqpoint{5.062926in}{3.102590in}}%
\pgfusepath{clip}%
\pgfsetbuttcap%
\pgfsetroundjoin%
\definecolor{currentfill}{rgb}{1.000000,0.498039,0.054902}%
\pgfsetfillcolor{currentfill}%
\pgfsetlinewidth{1.003750pt}%
\definecolor{currentstroke}{rgb}{1.000000,0.498039,0.054902}%
\pgfsetstrokecolor{currentstroke}%
\pgfsetdash{}{0pt}%
\pgfpathmoveto{\pgfqpoint{1.843937in}{2.833777in}}%
\pgfpathcurveto{\pgfqpoint{1.854987in}{2.833777in}}{\pgfqpoint{1.865586in}{2.838167in}}{\pgfqpoint{1.873400in}{2.845981in}}%
\pgfpathcurveto{\pgfqpoint{1.881213in}{2.853795in}}{\pgfqpoint{1.885603in}{2.864394in}}{\pgfqpoint{1.885603in}{2.875444in}}%
\pgfpathcurveto{\pgfqpoint{1.885603in}{2.886494in}}{\pgfqpoint{1.881213in}{2.897093in}}{\pgfqpoint{1.873400in}{2.904907in}}%
\pgfpathcurveto{\pgfqpoint{1.865586in}{2.912720in}}{\pgfqpoint{1.854987in}{2.917111in}}{\pgfqpoint{1.843937in}{2.917111in}}%
\pgfpathcurveto{\pgfqpoint{1.832887in}{2.917111in}}{\pgfqpoint{1.822288in}{2.912720in}}{\pgfqpoint{1.814474in}{2.904907in}}%
\pgfpathcurveto{\pgfqpoint{1.806660in}{2.897093in}}{\pgfqpoint{1.802270in}{2.886494in}}{\pgfqpoint{1.802270in}{2.875444in}}%
\pgfpathcurveto{\pgfqpoint{1.802270in}{2.864394in}}{\pgfqpoint{1.806660in}{2.853795in}}{\pgfqpoint{1.814474in}{2.845981in}}%
\pgfpathcurveto{\pgfqpoint{1.822288in}{2.838167in}}{\pgfqpoint{1.832887in}{2.833777in}}{\pgfqpoint{1.843937in}{2.833777in}}%
\pgfpathclose%
\pgfusepath{stroke,fill}%
\end{pgfscope}%
\begin{pgfscope}%
\pgfpathrectangle{\pgfqpoint{0.787074in}{0.548769in}}{\pgfqpoint{5.062926in}{3.102590in}}%
\pgfusepath{clip}%
\pgfsetbuttcap%
\pgfsetroundjoin%
\definecolor{currentfill}{rgb}{1.000000,0.498039,0.054902}%
\pgfsetfillcolor{currentfill}%
\pgfsetlinewidth{1.003750pt}%
\definecolor{currentstroke}{rgb}{1.000000,0.498039,0.054902}%
\pgfsetstrokecolor{currentstroke}%
\pgfsetdash{}{0pt}%
\pgfpathmoveto{\pgfqpoint{1.748545in}{1.540711in}}%
\pgfpathcurveto{\pgfqpoint{1.759595in}{1.540711in}}{\pgfqpoint{1.770194in}{1.545101in}}{\pgfqpoint{1.778008in}{1.552915in}}%
\pgfpathcurveto{\pgfqpoint{1.785821in}{1.560729in}}{\pgfqpoint{1.790212in}{1.571328in}}{\pgfqpoint{1.790212in}{1.582378in}}%
\pgfpathcurveto{\pgfqpoint{1.790212in}{1.593428in}}{\pgfqpoint{1.785821in}{1.604027in}}{\pgfqpoint{1.778008in}{1.611841in}}%
\pgfpathcurveto{\pgfqpoint{1.770194in}{1.619654in}}{\pgfqpoint{1.759595in}{1.624044in}}{\pgfqpoint{1.748545in}{1.624044in}}%
\pgfpathcurveto{\pgfqpoint{1.737495in}{1.624044in}}{\pgfqpoint{1.726896in}{1.619654in}}{\pgfqpoint{1.719082in}{1.611841in}}%
\pgfpathcurveto{\pgfqpoint{1.711268in}{1.604027in}}{\pgfqpoint{1.706878in}{1.593428in}}{\pgfqpoint{1.706878in}{1.582378in}}%
\pgfpathcurveto{\pgfqpoint{1.706878in}{1.571328in}}{\pgfqpoint{1.711268in}{1.560729in}}{\pgfqpoint{1.719082in}{1.552915in}}%
\pgfpathcurveto{\pgfqpoint{1.726896in}{1.545101in}}{\pgfqpoint{1.737495in}{1.540711in}}{\pgfqpoint{1.748545in}{1.540711in}}%
\pgfpathclose%
\pgfusepath{stroke,fill}%
\end{pgfscope}%
\begin{pgfscope}%
\pgfpathrectangle{\pgfqpoint{0.787074in}{0.548769in}}{\pgfqpoint{5.062926in}{3.102590in}}%
\pgfusepath{clip}%
\pgfsetbuttcap%
\pgfsetroundjoin%
\definecolor{currentfill}{rgb}{1.000000,0.498039,0.054902}%
\pgfsetfillcolor{currentfill}%
\pgfsetlinewidth{1.003750pt}%
\definecolor{currentstroke}{rgb}{1.000000,0.498039,0.054902}%
\pgfsetstrokecolor{currentstroke}%
\pgfsetdash{}{0pt}%
\pgfpathmoveto{\pgfqpoint{1.168244in}{1.951746in}}%
\pgfpathcurveto{\pgfqpoint{1.179294in}{1.951746in}}{\pgfqpoint{1.189893in}{1.956136in}}{\pgfqpoint{1.197707in}{1.963950in}}%
\pgfpathcurveto{\pgfqpoint{1.205520in}{1.971763in}}{\pgfqpoint{1.209911in}{1.982362in}}{\pgfqpoint{1.209911in}{1.993412in}}%
\pgfpathcurveto{\pgfqpoint{1.209911in}{2.004463in}}{\pgfqpoint{1.205520in}{2.015062in}}{\pgfqpoint{1.197707in}{2.022875in}}%
\pgfpathcurveto{\pgfqpoint{1.189893in}{2.030689in}}{\pgfqpoint{1.179294in}{2.035079in}}{\pgfqpoint{1.168244in}{2.035079in}}%
\pgfpathcurveto{\pgfqpoint{1.157194in}{2.035079in}}{\pgfqpoint{1.146595in}{2.030689in}}{\pgfqpoint{1.138781in}{2.022875in}}%
\pgfpathcurveto{\pgfqpoint{1.130968in}{2.015062in}}{\pgfqpoint{1.126577in}{2.004463in}}{\pgfqpoint{1.126577in}{1.993412in}}%
\pgfpathcurveto{\pgfqpoint{1.126577in}{1.982362in}}{\pgfqpoint{1.130968in}{1.971763in}}{\pgfqpoint{1.138781in}{1.963950in}}%
\pgfpathcurveto{\pgfqpoint{1.146595in}{1.956136in}}{\pgfqpoint{1.157194in}{1.951746in}}{\pgfqpoint{1.168244in}{1.951746in}}%
\pgfpathclose%
\pgfusepath{stroke,fill}%
\end{pgfscope}%
\begin{pgfscope}%
\pgfpathrectangle{\pgfqpoint{0.787074in}{0.548769in}}{\pgfqpoint{5.062926in}{3.102590in}}%
\pgfusepath{clip}%
\pgfsetbuttcap%
\pgfsetroundjoin%
\definecolor{currentfill}{rgb}{1.000000,0.498039,0.054902}%
\pgfsetfillcolor{currentfill}%
\pgfsetlinewidth{1.003750pt}%
\definecolor{currentstroke}{rgb}{1.000000,0.498039,0.054902}%
\pgfsetstrokecolor{currentstroke}%
\pgfsetdash{}{0pt}%
\pgfpathmoveto{\pgfqpoint{1.173922in}{1.644565in}}%
\pgfpathcurveto{\pgfqpoint{1.184972in}{1.644565in}}{\pgfqpoint{1.195571in}{1.648955in}}{\pgfqpoint{1.203385in}{1.656769in}}%
\pgfpathcurveto{\pgfqpoint{1.211199in}{1.664582in}}{\pgfqpoint{1.215589in}{1.675181in}}{\pgfqpoint{1.215589in}{1.686232in}}%
\pgfpathcurveto{\pgfqpoint{1.215589in}{1.697282in}}{\pgfqpoint{1.211199in}{1.707881in}}{\pgfqpoint{1.203385in}{1.715694in}}%
\pgfpathcurveto{\pgfqpoint{1.195571in}{1.723508in}}{\pgfqpoint{1.184972in}{1.727898in}}{\pgfqpoint{1.173922in}{1.727898in}}%
\pgfpathcurveto{\pgfqpoint{1.162872in}{1.727898in}}{\pgfqpoint{1.152273in}{1.723508in}}{\pgfqpoint{1.144459in}{1.715694in}}%
\pgfpathcurveto{\pgfqpoint{1.136646in}{1.707881in}}{\pgfqpoint{1.132255in}{1.697282in}}{\pgfqpoint{1.132255in}{1.686232in}}%
\pgfpathcurveto{\pgfqpoint{1.132255in}{1.675181in}}{\pgfqpoint{1.136646in}{1.664582in}}{\pgfqpoint{1.144459in}{1.656769in}}%
\pgfpathcurveto{\pgfqpoint{1.152273in}{1.648955in}}{\pgfqpoint{1.162872in}{1.644565in}}{\pgfqpoint{1.173922in}{1.644565in}}%
\pgfpathclose%
\pgfusepath{stroke,fill}%
\end{pgfscope}%
\begin{pgfscope}%
\pgfpathrectangle{\pgfqpoint{0.787074in}{0.548769in}}{\pgfqpoint{5.062926in}{3.102590in}}%
\pgfusepath{clip}%
\pgfsetbuttcap%
\pgfsetroundjoin%
\definecolor{currentfill}{rgb}{0.121569,0.466667,0.705882}%
\pgfsetfillcolor{currentfill}%
\pgfsetlinewidth{1.003750pt}%
\definecolor{currentstroke}{rgb}{0.121569,0.466667,0.705882}%
\pgfsetstrokecolor{currentstroke}%
\pgfsetdash{}{0pt}%
\pgfpathmoveto{\pgfqpoint{1.971126in}{0.664836in}}%
\pgfpathcurveto{\pgfqpoint{1.982176in}{0.664836in}}{\pgfqpoint{1.992775in}{0.669227in}}{\pgfqpoint{2.000589in}{0.677040in}}%
\pgfpathcurveto{\pgfqpoint{2.008402in}{0.684854in}}{\pgfqpoint{2.012793in}{0.695453in}}{\pgfqpoint{2.012793in}{0.706503in}}%
\pgfpathcurveto{\pgfqpoint{2.012793in}{0.717553in}}{\pgfqpoint{2.008402in}{0.728152in}}{\pgfqpoint{2.000589in}{0.735966in}}%
\pgfpathcurveto{\pgfqpoint{1.992775in}{0.743779in}}{\pgfqpoint{1.982176in}{0.748170in}}{\pgfqpoint{1.971126in}{0.748170in}}%
\pgfpathcurveto{\pgfqpoint{1.960076in}{0.748170in}}{\pgfqpoint{1.949477in}{0.743779in}}{\pgfqpoint{1.941663in}{0.735966in}}%
\pgfpathcurveto{\pgfqpoint{1.933850in}{0.728152in}}{\pgfqpoint{1.929459in}{0.717553in}}{\pgfqpoint{1.929459in}{0.706503in}}%
\pgfpathcurveto{\pgfqpoint{1.929459in}{0.695453in}}{\pgfqpoint{1.933850in}{0.684854in}}{\pgfqpoint{1.941663in}{0.677040in}}%
\pgfpathcurveto{\pgfqpoint{1.949477in}{0.669227in}}{\pgfqpoint{1.960076in}{0.664836in}}{\pgfqpoint{1.971126in}{0.664836in}}%
\pgfpathclose%
\pgfusepath{stroke,fill}%
\end{pgfscope}%
\begin{pgfscope}%
\pgfpathrectangle{\pgfqpoint{0.787074in}{0.548769in}}{\pgfqpoint{5.062926in}{3.102590in}}%
\pgfusepath{clip}%
\pgfsetbuttcap%
\pgfsetroundjoin%
\definecolor{currentfill}{rgb}{1.000000,0.498039,0.054902}%
\pgfsetfillcolor{currentfill}%
\pgfsetlinewidth{1.003750pt}%
\definecolor{currentstroke}{rgb}{1.000000,0.498039,0.054902}%
\pgfsetstrokecolor{currentstroke}%
\pgfsetdash{}{0pt}%
\pgfpathmoveto{\pgfqpoint{1.923430in}{1.428628in}}%
\pgfpathcurveto{\pgfqpoint{1.934480in}{1.428628in}}{\pgfqpoint{1.945079in}{1.433018in}}{\pgfqpoint{1.952893in}{1.440832in}}%
\pgfpathcurveto{\pgfqpoint{1.960706in}{1.448646in}}{\pgfqpoint{1.965097in}{1.459245in}}{\pgfqpoint{1.965097in}{1.470295in}}%
\pgfpathcurveto{\pgfqpoint{1.965097in}{1.481345in}}{\pgfqpoint{1.960706in}{1.491944in}}{\pgfqpoint{1.952893in}{1.499758in}}%
\pgfpathcurveto{\pgfqpoint{1.945079in}{1.507571in}}{\pgfqpoint{1.934480in}{1.511961in}}{\pgfqpoint{1.923430in}{1.511961in}}%
\pgfpathcurveto{\pgfqpoint{1.912380in}{1.511961in}}{\pgfqpoint{1.901781in}{1.507571in}}{\pgfqpoint{1.893967in}{1.499758in}}%
\pgfpathcurveto{\pgfqpoint{1.886154in}{1.491944in}}{\pgfqpoint{1.881763in}{1.481345in}}{\pgfqpoint{1.881763in}{1.470295in}}%
\pgfpathcurveto{\pgfqpoint{1.881763in}{1.459245in}}{\pgfqpoint{1.886154in}{1.448646in}}{\pgfqpoint{1.893967in}{1.440832in}}%
\pgfpathcurveto{\pgfqpoint{1.901781in}{1.433018in}}{\pgfqpoint{1.912380in}{1.428628in}}{\pgfqpoint{1.923430in}{1.428628in}}%
\pgfpathclose%
\pgfusepath{stroke,fill}%
\end{pgfscope}%
\begin{pgfscope}%
\pgfpathrectangle{\pgfqpoint{0.787074in}{0.548769in}}{\pgfqpoint{5.062926in}{3.102590in}}%
\pgfusepath{clip}%
\pgfsetbuttcap%
\pgfsetroundjoin%
\definecolor{currentfill}{rgb}{0.121569,0.466667,0.705882}%
\pgfsetfillcolor{currentfill}%
\pgfsetlinewidth{1.003750pt}%
\definecolor{currentstroke}{rgb}{0.121569,0.466667,0.705882}%
\pgfsetstrokecolor{currentstroke}%
\pgfsetdash{}{0pt}%
\pgfpathmoveto{\pgfqpoint{1.112599in}{0.648339in}}%
\pgfpathcurveto{\pgfqpoint{1.123649in}{0.648339in}}{\pgfqpoint{1.134248in}{0.652729in}}{\pgfqpoint{1.142062in}{0.660543in}}%
\pgfpathcurveto{\pgfqpoint{1.149875in}{0.668356in}}{\pgfqpoint{1.154265in}{0.678955in}}{\pgfqpoint{1.154265in}{0.690005in}}%
\pgfpathcurveto{\pgfqpoint{1.154265in}{0.701056in}}{\pgfqpoint{1.149875in}{0.711655in}}{\pgfqpoint{1.142062in}{0.719468in}}%
\pgfpathcurveto{\pgfqpoint{1.134248in}{0.727282in}}{\pgfqpoint{1.123649in}{0.731672in}}{\pgfqpoint{1.112599in}{0.731672in}}%
\pgfpathcurveto{\pgfqpoint{1.101549in}{0.731672in}}{\pgfqpoint{1.090950in}{0.727282in}}{\pgfqpoint{1.083136in}{0.719468in}}%
\pgfpathcurveto{\pgfqpoint{1.075322in}{0.711655in}}{\pgfqpoint{1.070932in}{0.701056in}}{\pgfqpoint{1.070932in}{0.690005in}}%
\pgfpathcurveto{\pgfqpoint{1.070932in}{0.678955in}}{\pgfqpoint{1.075322in}{0.668356in}}{\pgfqpoint{1.083136in}{0.660543in}}%
\pgfpathcurveto{\pgfqpoint{1.090950in}{0.652729in}}{\pgfqpoint{1.101549in}{0.648339in}}{\pgfqpoint{1.112599in}{0.648339in}}%
\pgfpathclose%
\pgfusepath{stroke,fill}%
\end{pgfscope}%
\begin{pgfscope}%
\pgfpathrectangle{\pgfqpoint{0.787074in}{0.548769in}}{\pgfqpoint{5.062926in}{3.102590in}}%
\pgfusepath{clip}%
\pgfsetbuttcap%
\pgfsetroundjoin%
\definecolor{currentfill}{rgb}{1.000000,0.498039,0.054902}%
\pgfsetfillcolor{currentfill}%
\pgfsetlinewidth{1.003750pt}%
\definecolor{currentstroke}{rgb}{1.000000,0.498039,0.054902}%
\pgfsetstrokecolor{currentstroke}%
\pgfsetdash{}{0pt}%
\pgfpathmoveto{\pgfqpoint{1.571986in}{1.507887in}}%
\pgfpathcurveto{\pgfqpoint{1.583036in}{1.507887in}}{\pgfqpoint{1.593635in}{1.512277in}}{\pgfqpoint{1.601449in}{1.520091in}}%
\pgfpathcurveto{\pgfqpoint{1.609263in}{1.527904in}}{\pgfqpoint{1.613653in}{1.538503in}}{\pgfqpoint{1.613653in}{1.549554in}}%
\pgfpathcurveto{\pgfqpoint{1.613653in}{1.560604in}}{\pgfqpoint{1.609263in}{1.571203in}}{\pgfqpoint{1.601449in}{1.579016in}}%
\pgfpathcurveto{\pgfqpoint{1.593635in}{1.586830in}}{\pgfqpoint{1.583036in}{1.591220in}}{\pgfqpoint{1.571986in}{1.591220in}}%
\pgfpathcurveto{\pgfqpoint{1.560936in}{1.591220in}}{\pgfqpoint{1.550337in}{1.586830in}}{\pgfqpoint{1.542523in}{1.579016in}}%
\pgfpathcurveto{\pgfqpoint{1.534710in}{1.571203in}}{\pgfqpoint{1.530320in}{1.560604in}}{\pgfqpoint{1.530320in}{1.549554in}}%
\pgfpathcurveto{\pgfqpoint{1.530320in}{1.538503in}}{\pgfqpoint{1.534710in}{1.527904in}}{\pgfqpoint{1.542523in}{1.520091in}}%
\pgfpathcurveto{\pgfqpoint{1.550337in}{1.512277in}}{\pgfqpoint{1.560936in}{1.507887in}}{\pgfqpoint{1.571986in}{1.507887in}}%
\pgfpathclose%
\pgfusepath{stroke,fill}%
\end{pgfscope}%
\begin{pgfscope}%
\pgfpathrectangle{\pgfqpoint{0.787074in}{0.548769in}}{\pgfqpoint{5.062926in}{3.102590in}}%
\pgfusepath{clip}%
\pgfsetbuttcap%
\pgfsetroundjoin%
\definecolor{currentfill}{rgb}{1.000000,0.498039,0.054902}%
\pgfsetfillcolor{currentfill}%
\pgfsetlinewidth{1.003750pt}%
\definecolor{currentstroke}{rgb}{1.000000,0.498039,0.054902}%
\pgfsetstrokecolor{currentstroke}%
\pgfsetdash{}{0pt}%
\pgfpathmoveto{\pgfqpoint{1.558696in}{2.621734in}}%
\pgfpathcurveto{\pgfqpoint{1.569746in}{2.621734in}}{\pgfqpoint{1.580345in}{2.626124in}}{\pgfqpoint{1.588159in}{2.633938in}}%
\pgfpathcurveto{\pgfqpoint{1.595973in}{2.641751in}}{\pgfqpoint{1.600363in}{2.652350in}}{\pgfqpoint{1.600363in}{2.663401in}}%
\pgfpathcurveto{\pgfqpoint{1.600363in}{2.674451in}}{\pgfqpoint{1.595973in}{2.685050in}}{\pgfqpoint{1.588159in}{2.692863in}}%
\pgfpathcurveto{\pgfqpoint{1.580345in}{2.700677in}}{\pgfqpoint{1.569746in}{2.705067in}}{\pgfqpoint{1.558696in}{2.705067in}}%
\pgfpathcurveto{\pgfqpoint{1.547646in}{2.705067in}}{\pgfqpoint{1.537047in}{2.700677in}}{\pgfqpoint{1.529233in}{2.692863in}}%
\pgfpathcurveto{\pgfqpoint{1.521420in}{2.685050in}}{\pgfqpoint{1.517030in}{2.674451in}}{\pgfqpoint{1.517030in}{2.663401in}}%
\pgfpathcurveto{\pgfqpoint{1.517030in}{2.652350in}}{\pgfqpoint{1.521420in}{2.641751in}}{\pgfqpoint{1.529233in}{2.633938in}}%
\pgfpathcurveto{\pgfqpoint{1.537047in}{2.626124in}}{\pgfqpoint{1.547646in}{2.621734in}}{\pgfqpoint{1.558696in}{2.621734in}}%
\pgfpathclose%
\pgfusepath{stroke,fill}%
\end{pgfscope}%
\begin{pgfscope}%
\pgfpathrectangle{\pgfqpoint{0.787074in}{0.548769in}}{\pgfqpoint{5.062926in}{3.102590in}}%
\pgfusepath{clip}%
\pgfsetbuttcap%
\pgfsetroundjoin%
\definecolor{currentfill}{rgb}{1.000000,0.498039,0.054902}%
\pgfsetfillcolor{currentfill}%
\pgfsetlinewidth{1.003750pt}%
\definecolor{currentstroke}{rgb}{1.000000,0.498039,0.054902}%
\pgfsetstrokecolor{currentstroke}%
\pgfsetdash{}{0pt}%
\pgfpathmoveto{\pgfqpoint{1.191544in}{1.832304in}}%
\pgfpathcurveto{\pgfqpoint{1.202594in}{1.832304in}}{\pgfqpoint{1.213193in}{1.836694in}}{\pgfqpoint{1.221007in}{1.844508in}}%
\pgfpathcurveto{\pgfqpoint{1.228820in}{1.852322in}}{\pgfqpoint{1.233210in}{1.862921in}}{\pgfqpoint{1.233210in}{1.873971in}}%
\pgfpathcurveto{\pgfqpoint{1.233210in}{1.885021in}}{\pgfqpoint{1.228820in}{1.895620in}}{\pgfqpoint{1.221007in}{1.903433in}}%
\pgfpathcurveto{\pgfqpoint{1.213193in}{1.911247in}}{\pgfqpoint{1.202594in}{1.915637in}}{\pgfqpoint{1.191544in}{1.915637in}}%
\pgfpathcurveto{\pgfqpoint{1.180494in}{1.915637in}}{\pgfqpoint{1.169895in}{1.911247in}}{\pgfqpoint{1.162081in}{1.903433in}}%
\pgfpathcurveto{\pgfqpoint{1.154267in}{1.895620in}}{\pgfqpoint{1.149877in}{1.885021in}}{\pgfqpoint{1.149877in}{1.873971in}}%
\pgfpathcurveto{\pgfqpoint{1.149877in}{1.862921in}}{\pgfqpoint{1.154267in}{1.852322in}}{\pgfqpoint{1.162081in}{1.844508in}}%
\pgfpathcurveto{\pgfqpoint{1.169895in}{1.836694in}}{\pgfqpoint{1.180494in}{1.832304in}}{\pgfqpoint{1.191544in}{1.832304in}}%
\pgfpathclose%
\pgfusepath{stroke,fill}%
\end{pgfscope}%
\begin{pgfscope}%
\pgfpathrectangle{\pgfqpoint{0.787074in}{0.548769in}}{\pgfqpoint{5.062926in}{3.102590in}}%
\pgfusepath{clip}%
\pgfsetbuttcap%
\pgfsetroundjoin%
\definecolor{currentfill}{rgb}{1.000000,0.498039,0.054902}%
\pgfsetfillcolor{currentfill}%
\pgfsetlinewidth{1.003750pt}%
\definecolor{currentstroke}{rgb}{1.000000,0.498039,0.054902}%
\pgfsetstrokecolor{currentstroke}%
\pgfsetdash{}{0pt}%
\pgfpathmoveto{\pgfqpoint{1.255687in}{3.055568in}}%
\pgfpathcurveto{\pgfqpoint{1.266737in}{3.055568in}}{\pgfqpoint{1.277336in}{3.059958in}}{\pgfqpoint{1.285149in}{3.067772in}}%
\pgfpathcurveto{\pgfqpoint{1.292963in}{3.075585in}}{\pgfqpoint{1.297353in}{3.086184in}}{\pgfqpoint{1.297353in}{3.097235in}}%
\pgfpathcurveto{\pgfqpoint{1.297353in}{3.108285in}}{\pgfqpoint{1.292963in}{3.118884in}}{\pgfqpoint{1.285149in}{3.126697in}}%
\pgfpathcurveto{\pgfqpoint{1.277336in}{3.134511in}}{\pgfqpoint{1.266737in}{3.138901in}}{\pgfqpoint{1.255687in}{3.138901in}}%
\pgfpathcurveto{\pgfqpoint{1.244637in}{3.138901in}}{\pgfqpoint{1.234037in}{3.134511in}}{\pgfqpoint{1.226224in}{3.126697in}}%
\pgfpathcurveto{\pgfqpoint{1.218410in}{3.118884in}}{\pgfqpoint{1.214020in}{3.108285in}}{\pgfqpoint{1.214020in}{3.097235in}}%
\pgfpathcurveto{\pgfqpoint{1.214020in}{3.086184in}}{\pgfqpoint{1.218410in}{3.075585in}}{\pgfqpoint{1.226224in}{3.067772in}}%
\pgfpathcurveto{\pgfqpoint{1.234037in}{3.059958in}}{\pgfqpoint{1.244637in}{3.055568in}}{\pgfqpoint{1.255687in}{3.055568in}}%
\pgfpathclose%
\pgfusepath{stroke,fill}%
\end{pgfscope}%
\begin{pgfscope}%
\pgfpathrectangle{\pgfqpoint{0.787074in}{0.548769in}}{\pgfqpoint{5.062926in}{3.102590in}}%
\pgfusepath{clip}%
\pgfsetbuttcap%
\pgfsetroundjoin%
\definecolor{currentfill}{rgb}{1.000000,0.498039,0.054902}%
\pgfsetfillcolor{currentfill}%
\pgfsetlinewidth{1.003750pt}%
\definecolor{currentstroke}{rgb}{1.000000,0.498039,0.054902}%
\pgfsetstrokecolor{currentstroke}%
\pgfsetdash{}{0pt}%
\pgfpathmoveto{\pgfqpoint{1.327231in}{3.291673in}}%
\pgfpathcurveto{\pgfqpoint{1.338281in}{3.291673in}}{\pgfqpoint{1.348880in}{3.296063in}}{\pgfqpoint{1.356693in}{3.303876in}}%
\pgfpathcurveto{\pgfqpoint{1.364507in}{3.311690in}}{\pgfqpoint{1.368897in}{3.322289in}}{\pgfqpoint{1.368897in}{3.333339in}}%
\pgfpathcurveto{\pgfqpoint{1.368897in}{3.344389in}}{\pgfqpoint{1.364507in}{3.354988in}}{\pgfqpoint{1.356693in}{3.362802in}}%
\pgfpathcurveto{\pgfqpoint{1.348880in}{3.370616in}}{\pgfqpoint{1.338281in}{3.375006in}}{\pgfqpoint{1.327231in}{3.375006in}}%
\pgfpathcurveto{\pgfqpoint{1.316180in}{3.375006in}}{\pgfqpoint{1.305581in}{3.370616in}}{\pgfqpoint{1.297768in}{3.362802in}}%
\pgfpathcurveto{\pgfqpoint{1.289954in}{3.354988in}}{\pgfqpoint{1.285564in}{3.344389in}}{\pgfqpoint{1.285564in}{3.333339in}}%
\pgfpathcurveto{\pgfqpoint{1.285564in}{3.322289in}}{\pgfqpoint{1.289954in}{3.311690in}}{\pgfqpoint{1.297768in}{3.303876in}}%
\pgfpathcurveto{\pgfqpoint{1.305581in}{3.296063in}}{\pgfqpoint{1.316180in}{3.291673in}}{\pgfqpoint{1.327231in}{3.291673in}}%
\pgfpathclose%
\pgfusepath{stroke,fill}%
\end{pgfscope}%
\begin{pgfscope}%
\pgfpathrectangle{\pgfqpoint{0.787074in}{0.548769in}}{\pgfqpoint{5.062926in}{3.102590in}}%
\pgfusepath{clip}%
\pgfsetbuttcap%
\pgfsetroundjoin%
\definecolor{currentfill}{rgb}{1.000000,0.498039,0.054902}%
\pgfsetfillcolor{currentfill}%
\pgfsetlinewidth{1.003750pt}%
\definecolor{currentstroke}{rgb}{1.000000,0.498039,0.054902}%
\pgfsetstrokecolor{currentstroke}%
\pgfsetdash{}{0pt}%
\pgfpathmoveto{\pgfqpoint{2.093773in}{1.890154in}}%
\pgfpathcurveto{\pgfqpoint{2.104823in}{1.890154in}}{\pgfqpoint{2.115422in}{1.894544in}}{\pgfqpoint{2.123236in}{1.902358in}}%
\pgfpathcurveto{\pgfqpoint{2.131049in}{1.910171in}}{\pgfqpoint{2.135439in}{1.920770in}}{\pgfqpoint{2.135439in}{1.931820in}}%
\pgfpathcurveto{\pgfqpoint{2.135439in}{1.942870in}}{\pgfqpoint{2.131049in}{1.953469in}}{\pgfqpoint{2.123236in}{1.961283in}}%
\pgfpathcurveto{\pgfqpoint{2.115422in}{1.969097in}}{\pgfqpoint{2.104823in}{1.973487in}}{\pgfqpoint{2.093773in}{1.973487in}}%
\pgfpathcurveto{\pgfqpoint{2.082723in}{1.973487in}}{\pgfqpoint{2.072124in}{1.969097in}}{\pgfqpoint{2.064310in}{1.961283in}}%
\pgfpathcurveto{\pgfqpoint{2.056496in}{1.953469in}}{\pgfqpoint{2.052106in}{1.942870in}}{\pgfqpoint{2.052106in}{1.931820in}}%
\pgfpathcurveto{\pgfqpoint{2.052106in}{1.920770in}}{\pgfqpoint{2.056496in}{1.910171in}}{\pgfqpoint{2.064310in}{1.902358in}}%
\pgfpathcurveto{\pgfqpoint{2.072124in}{1.894544in}}{\pgfqpoint{2.082723in}{1.890154in}}{\pgfqpoint{2.093773in}{1.890154in}}%
\pgfpathclose%
\pgfusepath{stroke,fill}%
\end{pgfscope}%
\begin{pgfscope}%
\pgfpathrectangle{\pgfqpoint{0.787074in}{0.548769in}}{\pgfqpoint{5.062926in}{3.102590in}}%
\pgfusepath{clip}%
\pgfsetbuttcap%
\pgfsetroundjoin%
\definecolor{currentfill}{rgb}{0.121569,0.466667,0.705882}%
\pgfsetfillcolor{currentfill}%
\pgfsetlinewidth{1.003750pt}%
\definecolor{currentstroke}{rgb}{0.121569,0.466667,0.705882}%
\pgfsetstrokecolor{currentstroke}%
\pgfsetdash{}{0pt}%
\pgfpathmoveto{\pgfqpoint{1.173922in}{1.784271in}}%
\pgfpathcurveto{\pgfqpoint{1.184972in}{1.784271in}}{\pgfqpoint{1.195571in}{1.788661in}}{\pgfqpoint{1.203385in}{1.796475in}}%
\pgfpathcurveto{\pgfqpoint{1.211199in}{1.804289in}}{\pgfqpoint{1.215589in}{1.814888in}}{\pgfqpoint{1.215589in}{1.825938in}}%
\pgfpathcurveto{\pgfqpoint{1.215589in}{1.836988in}}{\pgfqpoint{1.211199in}{1.847587in}}{\pgfqpoint{1.203385in}{1.855401in}}%
\pgfpathcurveto{\pgfqpoint{1.195571in}{1.863214in}}{\pgfqpoint{1.184972in}{1.867604in}}{\pgfqpoint{1.173922in}{1.867604in}}%
\pgfpathcurveto{\pgfqpoint{1.162872in}{1.867604in}}{\pgfqpoint{1.152273in}{1.863214in}}{\pgfqpoint{1.144459in}{1.855401in}}%
\pgfpathcurveto{\pgfqpoint{1.136646in}{1.847587in}}{\pgfqpoint{1.132255in}{1.836988in}}{\pgfqpoint{1.132255in}{1.825938in}}%
\pgfpathcurveto{\pgfqpoint{1.132255in}{1.814888in}}{\pgfqpoint{1.136646in}{1.804289in}}{\pgfqpoint{1.144459in}{1.796475in}}%
\pgfpathcurveto{\pgfqpoint{1.152273in}{1.788661in}}{\pgfqpoint{1.162872in}{1.784271in}}{\pgfqpoint{1.173922in}{1.784271in}}%
\pgfpathclose%
\pgfusepath{stroke,fill}%
\end{pgfscope}%
\begin{pgfscope}%
\pgfpathrectangle{\pgfqpoint{0.787074in}{0.548769in}}{\pgfqpoint{5.062926in}{3.102590in}}%
\pgfusepath{clip}%
\pgfsetbuttcap%
\pgfsetroundjoin%
\definecolor{currentfill}{rgb}{1.000000,0.498039,0.054902}%
\pgfsetfillcolor{currentfill}%
\pgfsetlinewidth{1.003750pt}%
\definecolor{currentstroke}{rgb}{1.000000,0.498039,0.054902}%
\pgfsetstrokecolor{currentstroke}%
\pgfsetdash{}{0pt}%
\pgfpathmoveto{\pgfqpoint{1.446470in}{1.980354in}}%
\pgfpathcurveto{\pgfqpoint{1.457521in}{1.980354in}}{\pgfqpoint{1.468120in}{1.984745in}}{\pgfqpoint{1.475933in}{1.992558in}}%
\pgfpathcurveto{\pgfqpoint{1.483747in}{2.000372in}}{\pgfqpoint{1.488137in}{2.010971in}}{\pgfqpoint{1.488137in}{2.022021in}}%
\pgfpathcurveto{\pgfqpoint{1.488137in}{2.033071in}}{\pgfqpoint{1.483747in}{2.043670in}}{\pgfqpoint{1.475933in}{2.051484in}}%
\pgfpathcurveto{\pgfqpoint{1.468120in}{2.059297in}}{\pgfqpoint{1.457521in}{2.063688in}}{\pgfqpoint{1.446470in}{2.063688in}}%
\pgfpathcurveto{\pgfqpoint{1.435420in}{2.063688in}}{\pgfqpoint{1.424821in}{2.059297in}}{\pgfqpoint{1.417008in}{2.051484in}}%
\pgfpathcurveto{\pgfqpoint{1.409194in}{2.043670in}}{\pgfqpoint{1.404804in}{2.033071in}}{\pgfqpoint{1.404804in}{2.022021in}}%
\pgfpathcurveto{\pgfqpoint{1.404804in}{2.010971in}}{\pgfqpoint{1.409194in}{2.000372in}}{\pgfqpoint{1.417008in}{1.992558in}}%
\pgfpathcurveto{\pgfqpoint{1.424821in}{1.984745in}}{\pgfqpoint{1.435420in}{1.980354in}}{\pgfqpoint{1.446470in}{1.980354in}}%
\pgfpathclose%
\pgfusepath{stroke,fill}%
\end{pgfscope}%
\begin{pgfscope}%
\pgfpathrectangle{\pgfqpoint{0.787074in}{0.548769in}}{\pgfqpoint{5.062926in}{3.102590in}}%
\pgfusepath{clip}%
\pgfsetbuttcap%
\pgfsetroundjoin%
\definecolor{currentfill}{rgb}{1.000000,0.498039,0.054902}%
\pgfsetfillcolor{currentfill}%
\pgfsetlinewidth{1.003750pt}%
\definecolor{currentstroke}{rgb}{1.000000,0.498039,0.054902}%
\pgfsetstrokecolor{currentstroke}%
\pgfsetdash{}{0pt}%
\pgfpathmoveto{\pgfqpoint{1.747162in}{1.870879in}}%
\pgfpathcurveto{\pgfqpoint{1.758213in}{1.870879in}}{\pgfqpoint{1.768812in}{1.875269in}}{\pgfqpoint{1.776625in}{1.883083in}}%
\pgfpathcurveto{\pgfqpoint{1.784439in}{1.890896in}}{\pgfqpoint{1.788829in}{1.901496in}}{\pgfqpoint{1.788829in}{1.912546in}}%
\pgfpathcurveto{\pgfqpoint{1.788829in}{1.923596in}}{\pgfqpoint{1.784439in}{1.934195in}}{\pgfqpoint{1.776625in}{1.942008in}}%
\pgfpathcurveto{\pgfqpoint{1.768812in}{1.949822in}}{\pgfqpoint{1.758213in}{1.954212in}}{\pgfqpoint{1.747162in}{1.954212in}}%
\pgfpathcurveto{\pgfqpoint{1.736112in}{1.954212in}}{\pgfqpoint{1.725513in}{1.949822in}}{\pgfqpoint{1.717700in}{1.942008in}}%
\pgfpathcurveto{\pgfqpoint{1.709886in}{1.934195in}}{\pgfqpoint{1.705496in}{1.923596in}}{\pgfqpoint{1.705496in}{1.912546in}}%
\pgfpathcurveto{\pgfqpoint{1.705496in}{1.901496in}}{\pgfqpoint{1.709886in}{1.890896in}}{\pgfqpoint{1.717700in}{1.883083in}}%
\pgfpathcurveto{\pgfqpoint{1.725513in}{1.875269in}}{\pgfqpoint{1.736112in}{1.870879in}}{\pgfqpoint{1.747162in}{1.870879in}}%
\pgfpathclose%
\pgfusepath{stroke,fill}%
\end{pgfscope}%
\begin{pgfscope}%
\pgfpathrectangle{\pgfqpoint{0.787074in}{0.548769in}}{\pgfqpoint{5.062926in}{3.102590in}}%
\pgfusepath{clip}%
\pgfsetbuttcap%
\pgfsetroundjoin%
\definecolor{currentfill}{rgb}{1.000000,0.498039,0.054902}%
\pgfsetfillcolor{currentfill}%
\pgfsetlinewidth{1.003750pt}%
\definecolor{currentstroke}{rgb}{1.000000,0.498039,0.054902}%
\pgfsetstrokecolor{currentstroke}%
\pgfsetdash{}{0pt}%
\pgfpathmoveto{\pgfqpoint{1.717470in}{1.599536in}}%
\pgfpathcurveto{\pgfqpoint{1.728520in}{1.599536in}}{\pgfqpoint{1.739119in}{1.603926in}}{\pgfqpoint{1.746933in}{1.611740in}}%
\pgfpathcurveto{\pgfqpoint{1.754747in}{1.619554in}}{\pgfqpoint{1.759137in}{1.630153in}}{\pgfqpoint{1.759137in}{1.641203in}}%
\pgfpathcurveto{\pgfqpoint{1.759137in}{1.652253in}}{\pgfqpoint{1.754747in}{1.662852in}}{\pgfqpoint{1.746933in}{1.670666in}}%
\pgfpathcurveto{\pgfqpoint{1.739119in}{1.678479in}}{\pgfqpoint{1.728520in}{1.682870in}}{\pgfqpoint{1.717470in}{1.682870in}}%
\pgfpathcurveto{\pgfqpoint{1.706420in}{1.682870in}}{\pgfqpoint{1.695821in}{1.678479in}}{\pgfqpoint{1.688007in}{1.670666in}}%
\pgfpathcurveto{\pgfqpoint{1.680194in}{1.662852in}}{\pgfqpoint{1.675804in}{1.652253in}}{\pgfqpoint{1.675804in}{1.641203in}}%
\pgfpathcurveto{\pgfqpoint{1.675804in}{1.630153in}}{\pgfqpoint{1.680194in}{1.619554in}}{\pgfqpoint{1.688007in}{1.611740in}}%
\pgfpathcurveto{\pgfqpoint{1.695821in}{1.603926in}}{\pgfqpoint{1.706420in}{1.599536in}}{\pgfqpoint{1.717470in}{1.599536in}}%
\pgfpathclose%
\pgfusepath{stroke,fill}%
\end{pgfscope}%
\begin{pgfscope}%
\pgfpathrectangle{\pgfqpoint{0.787074in}{0.548769in}}{\pgfqpoint{5.062926in}{3.102590in}}%
\pgfusepath{clip}%
\pgfsetbuttcap%
\pgfsetroundjoin%
\definecolor{currentfill}{rgb}{1.000000,0.498039,0.054902}%
\pgfsetfillcolor{currentfill}%
\pgfsetlinewidth{1.003750pt}%
\definecolor{currentstroke}{rgb}{1.000000,0.498039,0.054902}%
\pgfsetstrokecolor{currentstroke}%
\pgfsetdash{}{0pt}%
\pgfpathmoveto{\pgfqpoint{1.935354in}{1.739034in}}%
\pgfpathcurveto{\pgfqpoint{1.946404in}{1.739034in}}{\pgfqpoint{1.957003in}{1.743425in}}{\pgfqpoint{1.964817in}{1.751238in}}%
\pgfpathcurveto{\pgfqpoint{1.972630in}{1.759052in}}{\pgfqpoint{1.977021in}{1.769651in}}{\pgfqpoint{1.977021in}{1.780701in}}%
\pgfpathcurveto{\pgfqpoint{1.977021in}{1.791751in}}{\pgfqpoint{1.972630in}{1.802350in}}{\pgfqpoint{1.964817in}{1.810164in}}%
\pgfpathcurveto{\pgfqpoint{1.957003in}{1.817977in}}{\pgfqpoint{1.946404in}{1.822368in}}{\pgfqpoint{1.935354in}{1.822368in}}%
\pgfpathcurveto{\pgfqpoint{1.924304in}{1.822368in}}{\pgfqpoint{1.913705in}{1.817977in}}{\pgfqpoint{1.905891in}{1.810164in}}%
\pgfpathcurveto{\pgfqpoint{1.898078in}{1.802350in}}{\pgfqpoint{1.893687in}{1.791751in}}{\pgfqpoint{1.893687in}{1.780701in}}%
\pgfpathcurveto{\pgfqpoint{1.893687in}{1.769651in}}{\pgfqpoint{1.898078in}{1.759052in}}{\pgfqpoint{1.905891in}{1.751238in}}%
\pgfpathcurveto{\pgfqpoint{1.913705in}{1.743425in}}{\pgfqpoint{1.924304in}{1.739034in}}{\pgfqpoint{1.935354in}{1.739034in}}%
\pgfpathclose%
\pgfusepath{stroke,fill}%
\end{pgfscope}%
\begin{pgfscope}%
\pgfpathrectangle{\pgfqpoint{0.787074in}{0.548769in}}{\pgfqpoint{5.062926in}{3.102590in}}%
\pgfusepath{clip}%
\pgfsetbuttcap%
\pgfsetroundjoin%
\definecolor{currentfill}{rgb}{1.000000,0.498039,0.054902}%
\pgfsetfillcolor{currentfill}%
\pgfsetlinewidth{1.003750pt}%
\definecolor{currentstroke}{rgb}{1.000000,0.498039,0.054902}%
\pgfsetstrokecolor{currentstroke}%
\pgfsetdash{}{0pt}%
\pgfpathmoveto{\pgfqpoint{1.088751in}{2.553699in}}%
\pgfpathcurveto{\pgfqpoint{1.099801in}{2.553699in}}{\pgfqpoint{1.110400in}{2.558090in}}{\pgfqpoint{1.118214in}{2.565903in}}%
\pgfpathcurveto{\pgfqpoint{1.126027in}{2.573717in}}{\pgfqpoint{1.130417in}{2.584316in}}{\pgfqpoint{1.130417in}{2.595366in}}%
\pgfpathcurveto{\pgfqpoint{1.130417in}{2.606416in}}{\pgfqpoint{1.126027in}{2.617015in}}{\pgfqpoint{1.118214in}{2.624829in}}%
\pgfpathcurveto{\pgfqpoint{1.110400in}{2.632642in}}{\pgfqpoint{1.099801in}{2.637033in}}{\pgfqpoint{1.088751in}{2.637033in}}%
\pgfpathcurveto{\pgfqpoint{1.077701in}{2.637033in}}{\pgfqpoint{1.067102in}{2.632642in}}{\pgfqpoint{1.059288in}{2.624829in}}%
\pgfpathcurveto{\pgfqpoint{1.051474in}{2.617015in}}{\pgfqpoint{1.047084in}{2.606416in}}{\pgfqpoint{1.047084in}{2.595366in}}%
\pgfpathcurveto{\pgfqpoint{1.047084in}{2.584316in}}{\pgfqpoint{1.051474in}{2.573717in}}{\pgfqpoint{1.059288in}{2.565903in}}%
\pgfpathcurveto{\pgfqpoint{1.067102in}{2.558090in}}{\pgfqpoint{1.077701in}{2.553699in}}{\pgfqpoint{1.088751in}{2.553699in}}%
\pgfpathclose%
\pgfusepath{stroke,fill}%
\end{pgfscope}%
\begin{pgfscope}%
\pgfpathrectangle{\pgfqpoint{0.787074in}{0.548769in}}{\pgfqpoint{5.062926in}{3.102590in}}%
\pgfusepath{clip}%
\pgfsetbuttcap%
\pgfsetroundjoin%
\definecolor{currentfill}{rgb}{0.121569,0.466667,0.705882}%
\pgfsetfillcolor{currentfill}%
\pgfsetlinewidth{1.003750pt}%
\definecolor{currentstroke}{rgb}{0.121569,0.466667,0.705882}%
\pgfsetstrokecolor{currentstroke}%
\pgfsetdash{}{0pt}%
\pgfpathmoveto{\pgfqpoint{1.207991in}{2.326040in}}%
\pgfpathcurveto{\pgfqpoint{1.219041in}{2.326040in}}{\pgfqpoint{1.229640in}{2.330430in}}{\pgfqpoint{1.237453in}{2.338243in}}%
\pgfpathcurveto{\pgfqpoint{1.245267in}{2.346057in}}{\pgfqpoint{1.249657in}{2.356656in}}{\pgfqpoint{1.249657in}{2.367706in}}%
\pgfpathcurveto{\pgfqpoint{1.249657in}{2.378756in}}{\pgfqpoint{1.245267in}{2.389355in}}{\pgfqpoint{1.237453in}{2.397169in}}%
\pgfpathcurveto{\pgfqpoint{1.229640in}{2.404983in}}{\pgfqpoint{1.219041in}{2.409373in}}{\pgfqpoint{1.207991in}{2.409373in}}%
\pgfpathcurveto{\pgfqpoint{1.196941in}{2.409373in}}{\pgfqpoint{1.186342in}{2.404983in}}{\pgfqpoint{1.178528in}{2.397169in}}%
\pgfpathcurveto{\pgfqpoint{1.170714in}{2.389355in}}{\pgfqpoint{1.166324in}{2.378756in}}{\pgfqpoint{1.166324in}{2.367706in}}%
\pgfpathcurveto{\pgfqpoint{1.166324in}{2.356656in}}{\pgfqpoint{1.170714in}{2.346057in}}{\pgfqpoint{1.178528in}{2.338243in}}%
\pgfpathcurveto{\pgfqpoint{1.186342in}{2.330430in}}{\pgfqpoint{1.196941in}{2.326040in}}{\pgfqpoint{1.207991in}{2.326040in}}%
\pgfpathclose%
\pgfusepath{stroke,fill}%
\end{pgfscope}%
\begin{pgfscope}%
\pgfpathrectangle{\pgfqpoint{0.787074in}{0.548769in}}{\pgfqpoint{5.062926in}{3.102590in}}%
\pgfusepath{clip}%
\pgfsetbuttcap%
\pgfsetroundjoin%
\definecolor{currentfill}{rgb}{1.000000,0.498039,0.054902}%
\pgfsetfillcolor{currentfill}%
\pgfsetlinewidth{1.003750pt}%
\definecolor{currentstroke}{rgb}{1.000000,0.498039,0.054902}%
\pgfsetstrokecolor{currentstroke}%
\pgfsetdash{}{0pt}%
\pgfpathmoveto{\pgfqpoint{2.686565in}{2.769003in}}%
\pgfpathcurveto{\pgfqpoint{2.697616in}{2.769003in}}{\pgfqpoint{2.708215in}{2.773394in}}{\pgfqpoint{2.716028in}{2.781207in}}%
\pgfpathcurveto{\pgfqpoint{2.723842in}{2.789021in}}{\pgfqpoint{2.728232in}{2.799620in}}{\pgfqpoint{2.728232in}{2.810670in}}%
\pgfpathcurveto{\pgfqpoint{2.728232in}{2.821720in}}{\pgfqpoint{2.723842in}{2.832319in}}{\pgfqpoint{2.716028in}{2.840133in}}%
\pgfpathcurveto{\pgfqpoint{2.708215in}{2.847946in}}{\pgfqpoint{2.697616in}{2.852337in}}{\pgfqpoint{2.686565in}{2.852337in}}%
\pgfpathcurveto{\pgfqpoint{2.675515in}{2.852337in}}{\pgfqpoint{2.664916in}{2.847946in}}{\pgfqpoint{2.657103in}{2.840133in}}%
\pgfpathcurveto{\pgfqpoint{2.649289in}{2.832319in}}{\pgfqpoint{2.644899in}{2.821720in}}{\pgfqpoint{2.644899in}{2.810670in}}%
\pgfpathcurveto{\pgfqpoint{2.644899in}{2.799620in}}{\pgfqpoint{2.649289in}{2.789021in}}{\pgfqpoint{2.657103in}{2.781207in}}%
\pgfpathcurveto{\pgfqpoint{2.664916in}{2.773394in}}{\pgfqpoint{2.675515in}{2.769003in}}{\pgfqpoint{2.686565in}{2.769003in}}%
\pgfpathclose%
\pgfusepath{stroke,fill}%
\end{pgfscope}%
\begin{pgfscope}%
\pgfpathrectangle{\pgfqpoint{0.787074in}{0.548769in}}{\pgfqpoint{5.062926in}{3.102590in}}%
\pgfusepath{clip}%
\pgfsetbuttcap%
\pgfsetroundjoin%
\definecolor{currentfill}{rgb}{0.121569,0.466667,0.705882}%
\pgfsetfillcolor{currentfill}%
\pgfsetlinewidth{1.003750pt}%
\definecolor{currentstroke}{rgb}{0.121569,0.466667,0.705882}%
\pgfsetstrokecolor{currentstroke}%
\pgfsetdash{}{0pt}%
\pgfpathmoveto{\pgfqpoint{1.971126in}{0.659320in}}%
\pgfpathcurveto{\pgfqpoint{1.982176in}{0.659320in}}{\pgfqpoint{1.992775in}{0.663711in}}{\pgfqpoint{2.000589in}{0.671524in}}%
\pgfpathcurveto{\pgfqpoint{2.008402in}{0.679338in}}{\pgfqpoint{2.012793in}{0.689937in}}{\pgfqpoint{2.012793in}{0.700987in}}%
\pgfpathcurveto{\pgfqpoint{2.012793in}{0.712037in}}{\pgfqpoint{2.008402in}{0.722636in}}{\pgfqpoint{2.000589in}{0.730450in}}%
\pgfpathcurveto{\pgfqpoint{1.992775in}{0.738263in}}{\pgfqpoint{1.982176in}{0.742654in}}{\pgfqpoint{1.971126in}{0.742654in}}%
\pgfpathcurveto{\pgfqpoint{1.960076in}{0.742654in}}{\pgfqpoint{1.949477in}{0.738263in}}{\pgfqpoint{1.941663in}{0.730450in}}%
\pgfpathcurveto{\pgfqpoint{1.933850in}{0.722636in}}{\pgfqpoint{1.929459in}{0.712037in}}{\pgfqpoint{1.929459in}{0.700987in}}%
\pgfpathcurveto{\pgfqpoint{1.929459in}{0.689937in}}{\pgfqpoint{1.933850in}{0.679338in}}{\pgfqpoint{1.941663in}{0.671524in}}%
\pgfpathcurveto{\pgfqpoint{1.949477in}{0.663711in}}{\pgfqpoint{1.960076in}{0.659320in}}{\pgfqpoint{1.971126in}{0.659320in}}%
\pgfpathclose%
\pgfusepath{stroke,fill}%
\end{pgfscope}%
\begin{pgfscope}%
\pgfpathrectangle{\pgfqpoint{0.787074in}{0.548769in}}{\pgfqpoint{5.062926in}{3.102590in}}%
\pgfusepath{clip}%
\pgfsetbuttcap%
\pgfsetroundjoin%
\definecolor{currentfill}{rgb}{1.000000,0.498039,0.054902}%
\pgfsetfillcolor{currentfill}%
\pgfsetlinewidth{1.003750pt}%
\definecolor{currentstroke}{rgb}{1.000000,0.498039,0.054902}%
\pgfsetstrokecolor{currentstroke}%
\pgfsetdash{}{0pt}%
\pgfpathmoveto{\pgfqpoint{1.595520in}{1.946771in}}%
\pgfpathcurveto{\pgfqpoint{1.606570in}{1.946771in}}{\pgfqpoint{1.617170in}{1.951162in}}{\pgfqpoint{1.624983in}{1.958975in}}%
\pgfpathcurveto{\pgfqpoint{1.632797in}{1.966789in}}{\pgfqpoint{1.637187in}{1.977388in}}{\pgfqpoint{1.637187in}{1.988438in}}%
\pgfpathcurveto{\pgfqpoint{1.637187in}{1.999488in}}{\pgfqpoint{1.632797in}{2.010087in}}{\pgfqpoint{1.624983in}{2.017901in}}%
\pgfpathcurveto{\pgfqpoint{1.617170in}{2.025714in}}{\pgfqpoint{1.606570in}{2.030105in}}{\pgfqpoint{1.595520in}{2.030105in}}%
\pgfpathcurveto{\pgfqpoint{1.584470in}{2.030105in}}{\pgfqpoint{1.573871in}{2.025714in}}{\pgfqpoint{1.566058in}{2.017901in}}%
\pgfpathcurveto{\pgfqpoint{1.558244in}{2.010087in}}{\pgfqpoint{1.553854in}{1.999488in}}{\pgfqpoint{1.553854in}{1.988438in}}%
\pgfpathcurveto{\pgfqpoint{1.553854in}{1.977388in}}{\pgfqpoint{1.558244in}{1.966789in}}{\pgfqpoint{1.566058in}{1.958975in}}%
\pgfpathcurveto{\pgfqpoint{1.573871in}{1.951162in}}{\pgfqpoint{1.584470in}{1.946771in}}{\pgfqpoint{1.595520in}{1.946771in}}%
\pgfpathclose%
\pgfusepath{stroke,fill}%
\end{pgfscope}%
\begin{pgfscope}%
\pgfpathrectangle{\pgfqpoint{0.787074in}{0.548769in}}{\pgfqpoint{5.062926in}{3.102590in}}%
\pgfusepath{clip}%
\pgfsetbuttcap%
\pgfsetroundjoin%
\definecolor{currentfill}{rgb}{0.121569,0.466667,0.705882}%
\pgfsetfillcolor{currentfill}%
\pgfsetlinewidth{1.003750pt}%
\definecolor{currentstroke}{rgb}{0.121569,0.466667,0.705882}%
\pgfsetstrokecolor{currentstroke}%
\pgfsetdash{}{0pt}%
\pgfpathmoveto{\pgfqpoint{1.565710in}{0.648235in}}%
\pgfpathcurveto{\pgfqpoint{1.576761in}{0.648235in}}{\pgfqpoint{1.587360in}{0.652625in}}{\pgfqpoint{1.595173in}{0.660439in}}%
\pgfpathcurveto{\pgfqpoint{1.602987in}{0.668252in}}{\pgfqpoint{1.607377in}{0.678851in}}{\pgfqpoint{1.607377in}{0.689901in}}%
\pgfpathcurveto{\pgfqpoint{1.607377in}{0.700952in}}{\pgfqpoint{1.602987in}{0.711551in}}{\pgfqpoint{1.595173in}{0.719364in}}%
\pgfpathcurveto{\pgfqpoint{1.587360in}{0.727178in}}{\pgfqpoint{1.576761in}{0.731568in}}{\pgfqpoint{1.565710in}{0.731568in}}%
\pgfpathcurveto{\pgfqpoint{1.554660in}{0.731568in}}{\pgfqpoint{1.544061in}{0.727178in}}{\pgfqpoint{1.536248in}{0.719364in}}%
\pgfpathcurveto{\pgfqpoint{1.528434in}{0.711551in}}{\pgfqpoint{1.524044in}{0.700952in}}{\pgfqpoint{1.524044in}{0.689901in}}%
\pgfpathcurveto{\pgfqpoint{1.524044in}{0.678851in}}{\pgfqpoint{1.528434in}{0.668252in}}{\pgfqpoint{1.536248in}{0.660439in}}%
\pgfpathcurveto{\pgfqpoint{1.544061in}{0.652625in}}{\pgfqpoint{1.554660in}{0.648235in}}{\pgfqpoint{1.565710in}{0.648235in}}%
\pgfpathclose%
\pgfusepath{stroke,fill}%
\end{pgfscope}%
\begin{pgfscope}%
\pgfpathrectangle{\pgfqpoint{0.787074in}{0.548769in}}{\pgfqpoint{5.062926in}{3.102590in}}%
\pgfusepath{clip}%
\pgfsetbuttcap%
\pgfsetroundjoin%
\definecolor{currentfill}{rgb}{1.000000,0.498039,0.054902}%
\pgfsetfillcolor{currentfill}%
\pgfsetlinewidth{1.003750pt}%
\definecolor{currentstroke}{rgb}{1.000000,0.498039,0.054902}%
\pgfsetstrokecolor{currentstroke}%
\pgfsetdash{}{0pt}%
\pgfpathmoveto{\pgfqpoint{2.877349in}{3.081873in}}%
\pgfpathcurveto{\pgfqpoint{2.888399in}{3.081873in}}{\pgfqpoint{2.898998in}{3.086263in}}{\pgfqpoint{2.906812in}{3.094077in}}%
\pgfpathcurveto{\pgfqpoint{2.914626in}{3.101890in}}{\pgfqpoint{2.919016in}{3.112489in}}{\pgfqpoint{2.919016in}{3.123539in}}%
\pgfpathcurveto{\pgfqpoint{2.919016in}{3.134589in}}{\pgfqpoint{2.914626in}{3.145188in}}{\pgfqpoint{2.906812in}{3.153002in}}%
\pgfpathcurveto{\pgfqpoint{2.898998in}{3.160816in}}{\pgfqpoint{2.888399in}{3.165206in}}{\pgfqpoint{2.877349in}{3.165206in}}%
\pgfpathcurveto{\pgfqpoint{2.866299in}{3.165206in}}{\pgfqpoint{2.855700in}{3.160816in}}{\pgfqpoint{2.847887in}{3.153002in}}%
\pgfpathcurveto{\pgfqpoint{2.840073in}{3.145188in}}{\pgfqpoint{2.835683in}{3.134589in}}{\pgfqpoint{2.835683in}{3.123539in}}%
\pgfpathcurveto{\pgfqpoint{2.835683in}{3.112489in}}{\pgfqpoint{2.840073in}{3.101890in}}{\pgfqpoint{2.847887in}{3.094077in}}%
\pgfpathcurveto{\pgfqpoint{2.855700in}{3.086263in}}{\pgfqpoint{2.866299in}{3.081873in}}{\pgfqpoint{2.877349in}{3.081873in}}%
\pgfpathclose%
\pgfusepath{stroke,fill}%
\end{pgfscope}%
\begin{pgfscope}%
\pgfpathrectangle{\pgfqpoint{0.787074in}{0.548769in}}{\pgfqpoint{5.062926in}{3.102590in}}%
\pgfusepath{clip}%
\pgfsetbuttcap%
\pgfsetroundjoin%
\definecolor{currentfill}{rgb}{1.000000,0.498039,0.054902}%
\pgfsetfillcolor{currentfill}%
\pgfsetlinewidth{1.003750pt}%
\definecolor{currentstroke}{rgb}{1.000000,0.498039,0.054902}%
\pgfsetstrokecolor{currentstroke}%
\pgfsetdash{}{0pt}%
\pgfpathmoveto{\pgfqpoint{1.088751in}{2.902728in}}%
\pgfpathcurveto{\pgfqpoint{1.099801in}{2.902728in}}{\pgfqpoint{1.110400in}{2.907118in}}{\pgfqpoint{1.118214in}{2.914932in}}%
\pgfpathcurveto{\pgfqpoint{1.126027in}{2.922746in}}{\pgfqpoint{1.130417in}{2.933345in}}{\pgfqpoint{1.130417in}{2.944395in}}%
\pgfpathcurveto{\pgfqpoint{1.130417in}{2.955445in}}{\pgfqpoint{1.126027in}{2.966044in}}{\pgfqpoint{1.118214in}{2.973858in}}%
\pgfpathcurveto{\pgfqpoint{1.110400in}{2.981671in}}{\pgfqpoint{1.099801in}{2.986061in}}{\pgfqpoint{1.088751in}{2.986061in}}%
\pgfpathcurveto{\pgfqpoint{1.077701in}{2.986061in}}{\pgfqpoint{1.067102in}{2.981671in}}{\pgfqpoint{1.059288in}{2.973858in}}%
\pgfpathcurveto{\pgfqpoint{1.051474in}{2.966044in}}{\pgfqpoint{1.047084in}{2.955445in}}{\pgfqpoint{1.047084in}{2.944395in}}%
\pgfpathcurveto{\pgfqpoint{1.047084in}{2.933345in}}{\pgfqpoint{1.051474in}{2.922746in}}{\pgfqpoint{1.059288in}{2.914932in}}%
\pgfpathcurveto{\pgfqpoint{1.067102in}{2.907118in}}{\pgfqpoint{1.077701in}{2.902728in}}{\pgfqpoint{1.088751in}{2.902728in}}%
\pgfpathclose%
\pgfusepath{stroke,fill}%
\end{pgfscope}%
\begin{pgfscope}%
\pgfpathrectangle{\pgfqpoint{0.787074in}{0.548769in}}{\pgfqpoint{5.062926in}{3.102590in}}%
\pgfusepath{clip}%
\pgfsetbuttcap%
\pgfsetroundjoin%
\definecolor{currentfill}{rgb}{1.000000,0.498039,0.054902}%
\pgfsetfillcolor{currentfill}%
\pgfsetlinewidth{1.003750pt}%
\definecolor{currentstroke}{rgb}{1.000000,0.498039,0.054902}%
\pgfsetstrokecolor{currentstroke}%
\pgfsetdash{}{0pt}%
\pgfpathmoveto{\pgfqpoint{1.267611in}{2.766568in}}%
\pgfpathcurveto{\pgfqpoint{1.278661in}{2.766568in}}{\pgfqpoint{1.289260in}{2.770959in}}{\pgfqpoint{1.297073in}{2.778772in}}%
\pgfpathcurveto{\pgfqpoint{1.304887in}{2.786586in}}{\pgfqpoint{1.309277in}{2.797185in}}{\pgfqpoint{1.309277in}{2.808235in}}%
\pgfpathcurveto{\pgfqpoint{1.309277in}{2.819285in}}{\pgfqpoint{1.304887in}{2.829884in}}{\pgfqpoint{1.297073in}{2.837698in}}%
\pgfpathcurveto{\pgfqpoint{1.289260in}{2.845511in}}{\pgfqpoint{1.278661in}{2.849902in}}{\pgfqpoint{1.267611in}{2.849902in}}%
\pgfpathcurveto{\pgfqpoint{1.256561in}{2.849902in}}{\pgfqpoint{1.245961in}{2.845511in}}{\pgfqpoint{1.238148in}{2.837698in}}%
\pgfpathcurveto{\pgfqpoint{1.230334in}{2.829884in}}{\pgfqpoint{1.225944in}{2.819285in}}{\pgfqpoint{1.225944in}{2.808235in}}%
\pgfpathcurveto{\pgfqpoint{1.225944in}{2.797185in}}{\pgfqpoint{1.230334in}{2.786586in}}{\pgfqpoint{1.238148in}{2.778772in}}%
\pgfpathcurveto{\pgfqpoint{1.245961in}{2.770959in}}{\pgfqpoint{1.256561in}{2.766568in}}{\pgfqpoint{1.267611in}{2.766568in}}%
\pgfpathclose%
\pgfusepath{stroke,fill}%
\end{pgfscope}%
\begin{pgfscope}%
\pgfpathrectangle{\pgfqpoint{0.787074in}{0.548769in}}{\pgfqpoint{5.062926in}{3.102590in}}%
\pgfusepath{clip}%
\pgfsetbuttcap%
\pgfsetroundjoin%
\definecolor{currentfill}{rgb}{1.000000,0.498039,0.054902}%
\pgfsetfillcolor{currentfill}%
\pgfsetlinewidth{1.003750pt}%
\definecolor{currentstroke}{rgb}{1.000000,0.498039,0.054902}%
\pgfsetstrokecolor{currentstroke}%
\pgfsetdash{}{0pt}%
\pgfpathmoveto{\pgfqpoint{1.099591in}{1.775111in}}%
\pgfpathcurveto{\pgfqpoint{1.110641in}{1.775111in}}{\pgfqpoint{1.121240in}{1.779501in}}{\pgfqpoint{1.129054in}{1.787315in}}%
\pgfpathcurveto{\pgfqpoint{1.136867in}{1.795129in}}{\pgfqpoint{1.141257in}{1.805728in}}{\pgfqpoint{1.141257in}{1.816778in}}%
\pgfpathcurveto{\pgfqpoint{1.141257in}{1.827828in}}{\pgfqpoint{1.136867in}{1.838427in}}{\pgfqpoint{1.129054in}{1.846241in}}%
\pgfpathcurveto{\pgfqpoint{1.121240in}{1.854054in}}{\pgfqpoint{1.110641in}{1.858444in}}{\pgfqpoint{1.099591in}{1.858444in}}%
\pgfpathcurveto{\pgfqpoint{1.088541in}{1.858444in}}{\pgfqpoint{1.077942in}{1.854054in}}{\pgfqpoint{1.070128in}{1.846241in}}%
\pgfpathcurveto{\pgfqpoint{1.062314in}{1.838427in}}{\pgfqpoint{1.057924in}{1.827828in}}{\pgfqpoint{1.057924in}{1.816778in}}%
\pgfpathcurveto{\pgfqpoint{1.057924in}{1.805728in}}{\pgfqpoint{1.062314in}{1.795129in}}{\pgfqpoint{1.070128in}{1.787315in}}%
\pgfpathcurveto{\pgfqpoint{1.077942in}{1.779501in}}{\pgfqpoint{1.088541in}{1.775111in}}{\pgfqpoint{1.099591in}{1.775111in}}%
\pgfpathclose%
\pgfusepath{stroke,fill}%
\end{pgfscope}%
\begin{pgfscope}%
\pgfpathrectangle{\pgfqpoint{0.787074in}{0.548769in}}{\pgfqpoint{5.062926in}{3.102590in}}%
\pgfusepath{clip}%
\pgfsetbuttcap%
\pgfsetroundjoin%
\definecolor{currentfill}{rgb}{1.000000,0.498039,0.054902}%
\pgfsetfillcolor{currentfill}%
\pgfsetlinewidth{1.003750pt}%
\definecolor{currentstroke}{rgb}{1.000000,0.498039,0.054902}%
\pgfsetstrokecolor{currentstroke}%
\pgfsetdash{}{0pt}%
\pgfpathmoveto{\pgfqpoint{1.255687in}{1.545971in}}%
\pgfpathcurveto{\pgfqpoint{1.266737in}{1.545971in}}{\pgfqpoint{1.277336in}{1.550361in}}{\pgfqpoint{1.285149in}{1.558175in}}%
\pgfpathcurveto{\pgfqpoint{1.292963in}{1.565989in}}{\pgfqpoint{1.297353in}{1.576588in}}{\pgfqpoint{1.297353in}{1.587638in}}%
\pgfpathcurveto{\pgfqpoint{1.297353in}{1.598688in}}{\pgfqpoint{1.292963in}{1.609287in}}{\pgfqpoint{1.285149in}{1.617101in}}%
\pgfpathcurveto{\pgfqpoint{1.277336in}{1.624914in}}{\pgfqpoint{1.266737in}{1.629305in}}{\pgfqpoint{1.255687in}{1.629305in}}%
\pgfpathcurveto{\pgfqpoint{1.244637in}{1.629305in}}{\pgfqpoint{1.234037in}{1.624914in}}{\pgfqpoint{1.226224in}{1.617101in}}%
\pgfpathcurveto{\pgfqpoint{1.218410in}{1.609287in}}{\pgfqpoint{1.214020in}{1.598688in}}{\pgfqpoint{1.214020in}{1.587638in}}%
\pgfpathcurveto{\pgfqpoint{1.214020in}{1.576588in}}{\pgfqpoint{1.218410in}{1.565989in}}{\pgfqpoint{1.226224in}{1.558175in}}%
\pgfpathcurveto{\pgfqpoint{1.234037in}{1.550361in}}{\pgfqpoint{1.244637in}{1.545971in}}{\pgfqpoint{1.255687in}{1.545971in}}%
\pgfpathclose%
\pgfusepath{stroke,fill}%
\end{pgfscope}%
\begin{pgfscope}%
\pgfpathrectangle{\pgfqpoint{0.787074in}{0.548769in}}{\pgfqpoint{5.062926in}{3.102590in}}%
\pgfusepath{clip}%
\pgfsetbuttcap%
\pgfsetroundjoin%
\definecolor{currentfill}{rgb}{1.000000,0.498039,0.054902}%
\pgfsetfillcolor{currentfill}%
\pgfsetlinewidth{1.003750pt}%
\definecolor{currentstroke}{rgb}{1.000000,0.498039,0.054902}%
\pgfsetstrokecolor{currentstroke}%
\pgfsetdash{}{0pt}%
\pgfpathmoveto{\pgfqpoint{1.112599in}{2.232840in}}%
\pgfpathcurveto{\pgfqpoint{1.123649in}{2.232840in}}{\pgfqpoint{1.134248in}{2.237230in}}{\pgfqpoint{1.142062in}{2.245044in}}%
\pgfpathcurveto{\pgfqpoint{1.149875in}{2.252857in}}{\pgfqpoint{1.154265in}{2.263456in}}{\pgfqpoint{1.154265in}{2.274506in}}%
\pgfpathcurveto{\pgfqpoint{1.154265in}{2.285556in}}{\pgfqpoint{1.149875in}{2.296156in}}{\pgfqpoint{1.142062in}{2.303969in}}%
\pgfpathcurveto{\pgfqpoint{1.134248in}{2.311783in}}{\pgfqpoint{1.123649in}{2.316173in}}{\pgfqpoint{1.112599in}{2.316173in}}%
\pgfpathcurveto{\pgfqpoint{1.101549in}{2.316173in}}{\pgfqpoint{1.090950in}{2.311783in}}{\pgfqpoint{1.083136in}{2.303969in}}%
\pgfpathcurveto{\pgfqpoint{1.075322in}{2.296156in}}{\pgfqpoint{1.070932in}{2.285556in}}{\pgfqpoint{1.070932in}{2.274506in}}%
\pgfpathcurveto{\pgfqpoint{1.070932in}{2.263456in}}{\pgfqpoint{1.075322in}{2.252857in}}{\pgfqpoint{1.083136in}{2.245044in}}%
\pgfpathcurveto{\pgfqpoint{1.090950in}{2.237230in}}{\pgfqpoint{1.101549in}{2.232840in}}{\pgfqpoint{1.112599in}{2.232840in}}%
\pgfpathclose%
\pgfusepath{stroke,fill}%
\end{pgfscope}%
\begin{pgfscope}%
\pgfpathrectangle{\pgfqpoint{0.787074in}{0.548769in}}{\pgfqpoint{5.062926in}{3.102590in}}%
\pgfusepath{clip}%
\pgfsetbuttcap%
\pgfsetroundjoin%
\definecolor{currentfill}{rgb}{0.121569,0.466667,0.705882}%
\pgfsetfillcolor{currentfill}%
\pgfsetlinewidth{1.003750pt}%
\definecolor{currentstroke}{rgb}{0.121569,0.466667,0.705882}%
\pgfsetstrokecolor{currentstroke}%
\pgfsetdash{}{0pt}%
\pgfpathmoveto{\pgfqpoint{1.774380in}{0.676149in}}%
\pgfpathcurveto{\pgfqpoint{1.785430in}{0.676149in}}{\pgfqpoint{1.796029in}{0.680539in}}{\pgfqpoint{1.803843in}{0.688353in}}%
\pgfpathcurveto{\pgfqpoint{1.811657in}{0.696167in}}{\pgfqpoint{1.816047in}{0.706766in}}{\pgfqpoint{1.816047in}{0.717816in}}%
\pgfpathcurveto{\pgfqpoint{1.816047in}{0.728866in}}{\pgfqpoint{1.811657in}{0.739465in}}{\pgfqpoint{1.803843in}{0.747279in}}%
\pgfpathcurveto{\pgfqpoint{1.796029in}{0.755092in}}{\pgfqpoint{1.785430in}{0.759482in}}{\pgfqpoint{1.774380in}{0.759482in}}%
\pgfpathcurveto{\pgfqpoint{1.763330in}{0.759482in}}{\pgfqpoint{1.752731in}{0.755092in}}{\pgfqpoint{1.744917in}{0.747279in}}%
\pgfpathcurveto{\pgfqpoint{1.737104in}{0.739465in}}{\pgfqpoint{1.732714in}{0.728866in}}{\pgfqpoint{1.732714in}{0.717816in}}%
\pgfpathcurveto{\pgfqpoint{1.732714in}{0.706766in}}{\pgfqpoint{1.737104in}{0.696167in}}{\pgfqpoint{1.744917in}{0.688353in}}%
\pgfpathcurveto{\pgfqpoint{1.752731in}{0.680539in}}{\pgfqpoint{1.763330in}{0.676149in}}{\pgfqpoint{1.774380in}{0.676149in}}%
\pgfpathclose%
\pgfusepath{stroke,fill}%
\end{pgfscope}%
\begin{pgfscope}%
\pgfpathrectangle{\pgfqpoint{0.787074in}{0.548769in}}{\pgfqpoint{5.062926in}{3.102590in}}%
\pgfusepath{clip}%
\pgfsetbuttcap%
\pgfsetroundjoin%
\definecolor{currentfill}{rgb}{1.000000,0.498039,0.054902}%
\pgfsetfillcolor{currentfill}%
\pgfsetlinewidth{1.003750pt}%
\definecolor{currentstroke}{rgb}{1.000000,0.498039,0.054902}%
\pgfsetstrokecolor{currentstroke}%
\pgfsetdash{}{0pt}%
\pgfpathmoveto{\pgfqpoint{1.237801in}{1.907282in}}%
\pgfpathcurveto{\pgfqpoint{1.248851in}{1.907282in}}{\pgfqpoint{1.259450in}{1.911673in}}{\pgfqpoint{1.267263in}{1.919486in}}%
\pgfpathcurveto{\pgfqpoint{1.275077in}{1.927300in}}{\pgfqpoint{1.279467in}{1.937899in}}{\pgfqpoint{1.279467in}{1.948949in}}%
\pgfpathcurveto{\pgfqpoint{1.279467in}{1.959999in}}{\pgfqpoint{1.275077in}{1.970598in}}{\pgfqpoint{1.267263in}{1.978412in}}%
\pgfpathcurveto{\pgfqpoint{1.259450in}{1.986225in}}{\pgfqpoint{1.248851in}{1.990616in}}{\pgfqpoint{1.237801in}{1.990616in}}%
\pgfpathcurveto{\pgfqpoint{1.226751in}{1.990616in}}{\pgfqpoint{1.216151in}{1.986225in}}{\pgfqpoint{1.208338in}{1.978412in}}%
\pgfpathcurveto{\pgfqpoint{1.200524in}{1.970598in}}{\pgfqpoint{1.196134in}{1.959999in}}{\pgfqpoint{1.196134in}{1.948949in}}%
\pgfpathcurveto{\pgfqpoint{1.196134in}{1.937899in}}{\pgfqpoint{1.200524in}{1.927300in}}{\pgfqpoint{1.208338in}{1.919486in}}%
\pgfpathcurveto{\pgfqpoint{1.216151in}{1.911673in}}{\pgfqpoint{1.226751in}{1.907282in}}{\pgfqpoint{1.237801in}{1.907282in}}%
\pgfpathclose%
\pgfusepath{stroke,fill}%
\end{pgfscope}%
\begin{pgfscope}%
\pgfpathrectangle{\pgfqpoint{0.787074in}{0.548769in}}{\pgfqpoint{5.062926in}{3.102590in}}%
\pgfusepath{clip}%
\pgfsetbuttcap%
\pgfsetroundjoin%
\definecolor{currentfill}{rgb}{1.000000,0.498039,0.054902}%
\pgfsetfillcolor{currentfill}%
\pgfsetlinewidth{1.003750pt}%
\definecolor{currentstroke}{rgb}{1.000000,0.498039,0.054902}%
\pgfsetstrokecolor{currentstroke}%
\pgfsetdash{}{0pt}%
\pgfpathmoveto{\pgfqpoint{1.582745in}{2.839399in}}%
\pgfpathcurveto{\pgfqpoint{1.593795in}{2.839399in}}{\pgfqpoint{1.604394in}{2.843789in}}{\pgfqpoint{1.612207in}{2.851603in}}%
\pgfpathcurveto{\pgfqpoint{1.620021in}{2.859416in}}{\pgfqpoint{1.624411in}{2.870015in}}{\pgfqpoint{1.624411in}{2.881065in}}%
\pgfpathcurveto{\pgfqpoint{1.624411in}{2.892115in}}{\pgfqpoint{1.620021in}{2.902715in}}{\pgfqpoint{1.612207in}{2.910528in}}%
\pgfpathcurveto{\pgfqpoint{1.604394in}{2.918342in}}{\pgfqpoint{1.593795in}{2.922732in}}{\pgfqpoint{1.582745in}{2.922732in}}%
\pgfpathcurveto{\pgfqpoint{1.571695in}{2.922732in}}{\pgfqpoint{1.561095in}{2.918342in}}{\pgfqpoint{1.553282in}{2.910528in}}%
\pgfpathcurveto{\pgfqpoint{1.545468in}{2.902715in}}{\pgfqpoint{1.541078in}{2.892115in}}{\pgfqpoint{1.541078in}{2.881065in}}%
\pgfpathcurveto{\pgfqpoint{1.541078in}{2.870015in}}{\pgfqpoint{1.545468in}{2.859416in}}{\pgfqpoint{1.553282in}{2.851603in}}%
\pgfpathcurveto{\pgfqpoint{1.561095in}{2.843789in}}{\pgfqpoint{1.571695in}{2.839399in}}{\pgfqpoint{1.582745in}{2.839399in}}%
\pgfpathclose%
\pgfusepath{stroke,fill}%
\end{pgfscope}%
\begin{pgfscope}%
\pgfpathrectangle{\pgfqpoint{0.787074in}{0.548769in}}{\pgfqpoint{5.062926in}{3.102590in}}%
\pgfusepath{clip}%
\pgfsetbuttcap%
\pgfsetroundjoin%
\definecolor{currentfill}{rgb}{0.121569,0.466667,0.705882}%
\pgfsetfillcolor{currentfill}%
\pgfsetlinewidth{1.003750pt}%
\definecolor{currentstroke}{rgb}{0.121569,0.466667,0.705882}%
\pgfsetstrokecolor{currentstroke}%
\pgfsetdash{}{0pt}%
\pgfpathmoveto{\pgfqpoint{3.235069in}{0.648138in}}%
\pgfpathcurveto{\pgfqpoint{3.246119in}{0.648138in}}{\pgfqpoint{3.256718in}{0.652528in}}{\pgfqpoint{3.264532in}{0.660342in}}%
\pgfpathcurveto{\pgfqpoint{3.272345in}{0.668156in}}{\pgfqpoint{3.276736in}{0.678755in}}{\pgfqpoint{3.276736in}{0.689805in}}%
\pgfpathcurveto{\pgfqpoint{3.276736in}{0.700855in}}{\pgfqpoint{3.272345in}{0.711454in}}{\pgfqpoint{3.264532in}{0.719268in}}%
\pgfpathcurveto{\pgfqpoint{3.256718in}{0.727081in}}{\pgfqpoint{3.246119in}{0.731472in}}{\pgfqpoint{3.235069in}{0.731472in}}%
\pgfpathcurveto{\pgfqpoint{3.224019in}{0.731472in}}{\pgfqpoint{3.213420in}{0.727081in}}{\pgfqpoint{3.205606in}{0.719268in}}%
\pgfpathcurveto{\pgfqpoint{3.197793in}{0.711454in}}{\pgfqpoint{3.193402in}{0.700855in}}{\pgfqpoint{3.193402in}{0.689805in}}%
\pgfpathcurveto{\pgfqpoint{3.193402in}{0.678755in}}{\pgfqpoint{3.197793in}{0.668156in}}{\pgfqpoint{3.205606in}{0.660342in}}%
\pgfpathcurveto{\pgfqpoint{3.213420in}{0.652528in}}{\pgfqpoint{3.224019in}{0.648138in}}{\pgfqpoint{3.235069in}{0.648138in}}%
\pgfpathclose%
\pgfusepath{stroke,fill}%
\end{pgfscope}%
\begin{pgfscope}%
\pgfpathrectangle{\pgfqpoint{0.787074in}{0.548769in}}{\pgfqpoint{5.062926in}{3.102590in}}%
\pgfusepath{clip}%
\pgfsetbuttcap%
\pgfsetroundjoin%
\definecolor{currentfill}{rgb}{0.121569,0.466667,0.705882}%
\pgfsetfillcolor{currentfill}%
\pgfsetlinewidth{1.003750pt}%
\definecolor{currentstroke}{rgb}{0.121569,0.466667,0.705882}%
\pgfsetstrokecolor{currentstroke}%
\pgfsetdash{}{0pt}%
\pgfpathmoveto{\pgfqpoint{2.489820in}{0.649828in}}%
\pgfpathcurveto{\pgfqpoint{2.500870in}{0.649828in}}{\pgfqpoint{2.511469in}{0.654218in}}{\pgfqpoint{2.519282in}{0.662032in}}%
\pgfpathcurveto{\pgfqpoint{2.527096in}{0.669845in}}{\pgfqpoint{2.531486in}{0.680444in}}{\pgfqpoint{2.531486in}{0.691495in}}%
\pgfpathcurveto{\pgfqpoint{2.531486in}{0.702545in}}{\pgfqpoint{2.527096in}{0.713144in}}{\pgfqpoint{2.519282in}{0.720957in}}%
\pgfpathcurveto{\pgfqpoint{2.511469in}{0.728771in}}{\pgfqpoint{2.500870in}{0.733161in}}{\pgfqpoint{2.489820in}{0.733161in}}%
\pgfpathcurveto{\pgfqpoint{2.478769in}{0.733161in}}{\pgfqpoint{2.468170in}{0.728771in}}{\pgfqpoint{2.460357in}{0.720957in}}%
\pgfpathcurveto{\pgfqpoint{2.452543in}{0.713144in}}{\pgfqpoint{2.448153in}{0.702545in}}{\pgfqpoint{2.448153in}{0.691495in}}%
\pgfpathcurveto{\pgfqpoint{2.448153in}{0.680444in}}{\pgfqpoint{2.452543in}{0.669845in}}{\pgfqpoint{2.460357in}{0.662032in}}%
\pgfpathcurveto{\pgfqpoint{2.468170in}{0.654218in}}{\pgfqpoint{2.478769in}{0.649828in}}{\pgfqpoint{2.489820in}{0.649828in}}%
\pgfpathclose%
\pgfusepath{stroke,fill}%
\end{pgfscope}%
\begin{pgfscope}%
\pgfpathrectangle{\pgfqpoint{0.787074in}{0.548769in}}{\pgfqpoint{5.062926in}{3.102590in}}%
\pgfusepath{clip}%
\pgfsetbuttcap%
\pgfsetroundjoin%
\definecolor{currentfill}{rgb}{1.000000,0.498039,0.054902}%
\pgfsetfillcolor{currentfill}%
\pgfsetlinewidth{1.003750pt}%
\definecolor{currentstroke}{rgb}{1.000000,0.498039,0.054902}%
\pgfsetstrokecolor{currentstroke}%
\pgfsetdash{}{0pt}%
\pgfpathmoveto{\pgfqpoint{1.732646in}{2.575639in}}%
\pgfpathcurveto{\pgfqpoint{1.743696in}{2.575639in}}{\pgfqpoint{1.754295in}{2.580029in}}{\pgfqpoint{1.762109in}{2.587842in}}%
\pgfpathcurveto{\pgfqpoint{1.769923in}{2.595656in}}{\pgfqpoint{1.774313in}{2.606255in}}{\pgfqpoint{1.774313in}{2.617305in}}%
\pgfpathcurveto{\pgfqpoint{1.774313in}{2.628355in}}{\pgfqpoint{1.769923in}{2.638954in}}{\pgfqpoint{1.762109in}{2.646768in}}%
\pgfpathcurveto{\pgfqpoint{1.754295in}{2.654582in}}{\pgfqpoint{1.743696in}{2.658972in}}{\pgfqpoint{1.732646in}{2.658972in}}%
\pgfpathcurveto{\pgfqpoint{1.721596in}{2.658972in}}{\pgfqpoint{1.710997in}{2.654582in}}{\pgfqpoint{1.703183in}{2.646768in}}%
\pgfpathcurveto{\pgfqpoint{1.695370in}{2.638954in}}{\pgfqpoint{1.690980in}{2.628355in}}{\pgfqpoint{1.690980in}{2.617305in}}%
\pgfpathcurveto{\pgfqpoint{1.690980in}{2.606255in}}{\pgfqpoint{1.695370in}{2.595656in}}{\pgfqpoint{1.703183in}{2.587842in}}%
\pgfpathcurveto{\pgfqpoint{1.710997in}{2.580029in}}{\pgfqpoint{1.721596in}{2.575639in}}{\pgfqpoint{1.732646in}{2.575639in}}%
\pgfpathclose%
\pgfusepath{stroke,fill}%
\end{pgfscope}%
\begin{pgfscope}%
\pgfpathrectangle{\pgfqpoint{0.787074in}{0.548769in}}{\pgfqpoint{5.062926in}{3.102590in}}%
\pgfusepath{clip}%
\pgfsetbuttcap%
\pgfsetroundjoin%
\definecolor{currentfill}{rgb}{1.000000,0.498039,0.054902}%
\pgfsetfillcolor{currentfill}%
\pgfsetlinewidth{1.003750pt}%
\definecolor{currentstroke}{rgb}{1.000000,0.498039,0.054902}%
\pgfsetstrokecolor{currentstroke}%
\pgfsetdash{}{0pt}%
\pgfpathmoveto{\pgfqpoint{1.997955in}{2.218400in}}%
\pgfpathcurveto{\pgfqpoint{2.009005in}{2.218400in}}{\pgfqpoint{2.019604in}{2.222790in}}{\pgfqpoint{2.027418in}{2.230604in}}%
\pgfpathcurveto{\pgfqpoint{2.035231in}{2.238417in}}{\pgfqpoint{2.039622in}{2.249016in}}{\pgfqpoint{2.039622in}{2.260066in}}%
\pgfpathcurveto{\pgfqpoint{2.039622in}{2.271116in}}{\pgfqpoint{2.035231in}{2.281715in}}{\pgfqpoint{2.027418in}{2.289529in}}%
\pgfpathcurveto{\pgfqpoint{2.019604in}{2.297343in}}{\pgfqpoint{2.009005in}{2.301733in}}{\pgfqpoint{1.997955in}{2.301733in}}%
\pgfpathcurveto{\pgfqpoint{1.986905in}{2.301733in}}{\pgfqpoint{1.976306in}{2.297343in}}{\pgfqpoint{1.968492in}{2.289529in}}%
\pgfpathcurveto{\pgfqpoint{1.960679in}{2.281715in}}{\pgfqpoint{1.956288in}{2.271116in}}{\pgfqpoint{1.956288in}{2.260066in}}%
\pgfpathcurveto{\pgfqpoint{1.956288in}{2.249016in}}{\pgfqpoint{1.960679in}{2.238417in}}{\pgfqpoint{1.968492in}{2.230604in}}%
\pgfpathcurveto{\pgfqpoint{1.976306in}{2.222790in}}{\pgfqpoint{1.986905in}{2.218400in}}{\pgfqpoint{1.997955in}{2.218400in}}%
\pgfpathclose%
\pgfusepath{stroke,fill}%
\end{pgfscope}%
\begin{pgfscope}%
\pgfpathrectangle{\pgfqpoint{0.787074in}{0.548769in}}{\pgfqpoint{5.062926in}{3.102590in}}%
\pgfusepath{clip}%
\pgfsetbuttcap%
\pgfsetroundjoin%
\definecolor{currentfill}{rgb}{1.000000,0.498039,0.054902}%
\pgfsetfillcolor{currentfill}%
\pgfsetlinewidth{1.003750pt}%
\definecolor{currentstroke}{rgb}{1.000000,0.498039,0.054902}%
\pgfsetstrokecolor{currentstroke}%
\pgfsetdash{}{0pt}%
\pgfpathmoveto{\pgfqpoint{1.640235in}{2.443019in}}%
\pgfpathcurveto{\pgfqpoint{1.651285in}{2.443019in}}{\pgfqpoint{1.661884in}{2.447409in}}{\pgfqpoint{1.669698in}{2.455223in}}%
\pgfpathcurveto{\pgfqpoint{1.677512in}{2.463037in}}{\pgfqpoint{1.681902in}{2.473636in}}{\pgfqpoint{1.681902in}{2.484686in}}%
\pgfpathcurveto{\pgfqpoint{1.681902in}{2.495736in}}{\pgfqpoint{1.677512in}{2.506335in}}{\pgfqpoint{1.669698in}{2.514149in}}%
\pgfpathcurveto{\pgfqpoint{1.661884in}{2.521962in}}{\pgfqpoint{1.651285in}{2.526352in}}{\pgfqpoint{1.640235in}{2.526352in}}%
\pgfpathcurveto{\pgfqpoint{1.629185in}{2.526352in}}{\pgfqpoint{1.618586in}{2.521962in}}{\pgfqpoint{1.610773in}{2.514149in}}%
\pgfpathcurveto{\pgfqpoint{1.602959in}{2.506335in}}{\pgfqpoint{1.598569in}{2.495736in}}{\pgfqpoint{1.598569in}{2.484686in}}%
\pgfpathcurveto{\pgfqpoint{1.598569in}{2.473636in}}{\pgfqpoint{1.602959in}{2.463037in}}{\pgfqpoint{1.610773in}{2.455223in}}%
\pgfpathcurveto{\pgfqpoint{1.618586in}{2.447409in}}{\pgfqpoint{1.629185in}{2.443019in}}{\pgfqpoint{1.640235in}{2.443019in}}%
\pgfpathclose%
\pgfusepath{stroke,fill}%
\end{pgfscope}%
\begin{pgfscope}%
\pgfpathrectangle{\pgfqpoint{0.787074in}{0.548769in}}{\pgfqpoint{5.062926in}{3.102590in}}%
\pgfusepath{clip}%
\pgfsetbuttcap%
\pgfsetroundjoin%
\definecolor{currentfill}{rgb}{1.000000,0.498039,0.054902}%
\pgfsetfillcolor{currentfill}%
\pgfsetlinewidth{1.003750pt}%
\definecolor{currentstroke}{rgb}{1.000000,0.498039,0.054902}%
\pgfsetstrokecolor{currentstroke}%
\pgfsetdash{}{0pt}%
\pgfpathmoveto{\pgfqpoint{1.550805in}{1.631630in}}%
\pgfpathcurveto{\pgfqpoint{1.561856in}{1.631630in}}{\pgfqpoint{1.572455in}{1.636020in}}{\pgfqpoint{1.580268in}{1.643834in}}%
\pgfpathcurveto{\pgfqpoint{1.588082in}{1.651647in}}{\pgfqpoint{1.592472in}{1.662246in}}{\pgfqpoint{1.592472in}{1.673296in}}%
\pgfpathcurveto{\pgfqpoint{1.592472in}{1.684346in}}{\pgfqpoint{1.588082in}{1.694945in}}{\pgfqpoint{1.580268in}{1.702759in}}%
\pgfpathcurveto{\pgfqpoint{1.572455in}{1.710573in}}{\pgfqpoint{1.561856in}{1.714963in}}{\pgfqpoint{1.550805in}{1.714963in}}%
\pgfpathcurveto{\pgfqpoint{1.539755in}{1.714963in}}{\pgfqpoint{1.529156in}{1.710573in}}{\pgfqpoint{1.521343in}{1.702759in}}%
\pgfpathcurveto{\pgfqpoint{1.513529in}{1.694945in}}{\pgfqpoint{1.509139in}{1.684346in}}{\pgfqpoint{1.509139in}{1.673296in}}%
\pgfpathcurveto{\pgfqpoint{1.509139in}{1.662246in}}{\pgfqpoint{1.513529in}{1.651647in}}{\pgfqpoint{1.521343in}{1.643834in}}%
\pgfpathcurveto{\pgfqpoint{1.529156in}{1.636020in}}{\pgfqpoint{1.539755in}{1.631630in}}{\pgfqpoint{1.550805in}{1.631630in}}%
\pgfpathclose%
\pgfusepath{stroke,fill}%
\end{pgfscope}%
\begin{pgfscope}%
\pgfpathrectangle{\pgfqpoint{0.787074in}{0.548769in}}{\pgfqpoint{5.062926in}{3.102590in}}%
\pgfusepath{clip}%
\pgfsetbuttcap%
\pgfsetroundjoin%
\definecolor{currentfill}{rgb}{1.000000,0.498039,0.054902}%
\pgfsetfillcolor{currentfill}%
\pgfsetlinewidth{1.003750pt}%
\definecolor{currentstroke}{rgb}{1.000000,0.498039,0.054902}%
\pgfsetstrokecolor{currentstroke}%
\pgfsetdash{}{0pt}%
\pgfpathmoveto{\pgfqpoint{2.195978in}{1.681005in}}%
\pgfpathcurveto{\pgfqpoint{2.207029in}{1.681005in}}{\pgfqpoint{2.217628in}{1.685396in}}{\pgfqpoint{2.225441in}{1.693209in}}%
\pgfpathcurveto{\pgfqpoint{2.233255in}{1.701023in}}{\pgfqpoint{2.237645in}{1.711622in}}{\pgfqpoint{2.237645in}{1.722672in}}%
\pgfpathcurveto{\pgfqpoint{2.237645in}{1.733722in}}{\pgfqpoint{2.233255in}{1.744321in}}{\pgfqpoint{2.225441in}{1.752135in}}%
\pgfpathcurveto{\pgfqpoint{2.217628in}{1.759948in}}{\pgfqpoint{2.207029in}{1.764339in}}{\pgfqpoint{2.195978in}{1.764339in}}%
\pgfpathcurveto{\pgfqpoint{2.184928in}{1.764339in}}{\pgfqpoint{2.174329in}{1.759948in}}{\pgfqpoint{2.166516in}{1.752135in}}%
\pgfpathcurveto{\pgfqpoint{2.158702in}{1.744321in}}{\pgfqpoint{2.154312in}{1.733722in}}{\pgfqpoint{2.154312in}{1.722672in}}%
\pgfpathcurveto{\pgfqpoint{2.154312in}{1.711622in}}{\pgfqpoint{2.158702in}{1.701023in}}{\pgfqpoint{2.166516in}{1.693209in}}%
\pgfpathcurveto{\pgfqpoint{2.174329in}{1.685396in}}{\pgfqpoint{2.184928in}{1.681005in}}{\pgfqpoint{2.195978in}{1.681005in}}%
\pgfpathclose%
\pgfusepath{stroke,fill}%
\end{pgfscope}%
\begin{pgfscope}%
\pgfpathrectangle{\pgfqpoint{0.787074in}{0.548769in}}{\pgfqpoint{5.062926in}{3.102590in}}%
\pgfusepath{clip}%
\pgfsetbuttcap%
\pgfsetroundjoin%
\definecolor{currentfill}{rgb}{0.121569,0.466667,0.705882}%
\pgfsetfillcolor{currentfill}%
\pgfsetlinewidth{1.003750pt}%
\definecolor{currentstroke}{rgb}{0.121569,0.466667,0.705882}%
\pgfsetstrokecolor{currentstroke}%
\pgfsetdash{}{0pt}%
\pgfpathmoveto{\pgfqpoint{4.904428in}{0.648134in}}%
\pgfpathcurveto{\pgfqpoint{4.915478in}{0.648134in}}{\pgfqpoint{4.926077in}{0.652524in}}{\pgfqpoint{4.933890in}{0.660338in}}%
\pgfpathcurveto{\pgfqpoint{4.941704in}{0.668151in}}{\pgfqpoint{4.946094in}{0.678750in}}{\pgfqpoint{4.946094in}{0.689800in}}%
\pgfpathcurveto{\pgfqpoint{4.946094in}{0.700850in}}{\pgfqpoint{4.941704in}{0.711450in}}{\pgfqpoint{4.933890in}{0.719263in}}%
\pgfpathcurveto{\pgfqpoint{4.926077in}{0.727077in}}{\pgfqpoint{4.915478in}{0.731467in}}{\pgfqpoint{4.904428in}{0.731467in}}%
\pgfpathcurveto{\pgfqpoint{4.893377in}{0.731467in}}{\pgfqpoint{4.882778in}{0.727077in}}{\pgfqpoint{4.874965in}{0.719263in}}%
\pgfpathcurveto{\pgfqpoint{4.867151in}{0.711450in}}{\pgfqpoint{4.862761in}{0.700850in}}{\pgfqpoint{4.862761in}{0.689800in}}%
\pgfpathcurveto{\pgfqpoint{4.862761in}{0.678750in}}{\pgfqpoint{4.867151in}{0.668151in}}{\pgfqpoint{4.874965in}{0.660338in}}%
\pgfpathcurveto{\pgfqpoint{4.882778in}{0.652524in}}{\pgfqpoint{4.893377in}{0.648134in}}{\pgfqpoint{4.904428in}{0.648134in}}%
\pgfpathclose%
\pgfusepath{stroke,fill}%
\end{pgfscope}%
\begin{pgfscope}%
\pgfpathrectangle{\pgfqpoint{0.787074in}{0.548769in}}{\pgfqpoint{5.062926in}{3.102590in}}%
\pgfusepath{clip}%
\pgfsetbuttcap%
\pgfsetroundjoin%
\definecolor{currentfill}{rgb}{1.000000,0.498039,0.054902}%
\pgfsetfillcolor{currentfill}%
\pgfsetlinewidth{1.003750pt}%
\definecolor{currentstroke}{rgb}{1.000000,0.498039,0.054902}%
\pgfsetstrokecolor{currentstroke}%
\pgfsetdash{}{0pt}%
\pgfpathmoveto{\pgfqpoint{1.629916in}{2.734356in}}%
\pgfpathcurveto{\pgfqpoint{1.640967in}{2.734356in}}{\pgfqpoint{1.651566in}{2.738746in}}{\pgfqpoint{1.659379in}{2.746560in}}%
\pgfpathcurveto{\pgfqpoint{1.667193in}{2.754373in}}{\pgfqpoint{1.671583in}{2.764972in}}{\pgfqpoint{1.671583in}{2.776022in}}%
\pgfpathcurveto{\pgfqpoint{1.671583in}{2.787073in}}{\pgfqpoint{1.667193in}{2.797672in}}{\pgfqpoint{1.659379in}{2.805485in}}%
\pgfpathcurveto{\pgfqpoint{1.651566in}{2.813299in}}{\pgfqpoint{1.640967in}{2.817689in}}{\pgfqpoint{1.629916in}{2.817689in}}%
\pgfpathcurveto{\pgfqpoint{1.618866in}{2.817689in}}{\pgfqpoint{1.608267in}{2.813299in}}{\pgfqpoint{1.600454in}{2.805485in}}%
\pgfpathcurveto{\pgfqpoint{1.592640in}{2.797672in}}{\pgfqpoint{1.588250in}{2.787073in}}{\pgfqpoint{1.588250in}{2.776022in}}%
\pgfpathcurveto{\pgfqpoint{1.588250in}{2.764972in}}{\pgfqpoint{1.592640in}{2.754373in}}{\pgfqpoint{1.600454in}{2.746560in}}%
\pgfpathcurveto{\pgfqpoint{1.608267in}{2.738746in}}{\pgfqpoint{1.618866in}{2.734356in}}{\pgfqpoint{1.629916in}{2.734356in}}%
\pgfpathclose%
\pgfusepath{stroke,fill}%
\end{pgfscope}%
\begin{pgfscope}%
\pgfpathrectangle{\pgfqpoint{0.787074in}{0.548769in}}{\pgfqpoint{5.062926in}{3.102590in}}%
\pgfusepath{clip}%
\pgfsetbuttcap%
\pgfsetroundjoin%
\definecolor{currentfill}{rgb}{1.000000,0.498039,0.054902}%
\pgfsetfillcolor{currentfill}%
\pgfsetlinewidth{1.003750pt}%
\definecolor{currentstroke}{rgb}{1.000000,0.498039,0.054902}%
\pgfsetstrokecolor{currentstroke}%
\pgfsetdash{}{0pt}%
\pgfpathmoveto{\pgfqpoint{1.446470in}{2.144210in}}%
\pgfpathcurveto{\pgfqpoint{1.457521in}{2.144210in}}{\pgfqpoint{1.468120in}{2.148600in}}{\pgfqpoint{1.475933in}{2.156414in}}%
\pgfpathcurveto{\pgfqpoint{1.483747in}{2.164228in}}{\pgfqpoint{1.488137in}{2.174827in}}{\pgfqpoint{1.488137in}{2.185877in}}%
\pgfpathcurveto{\pgfqpoint{1.488137in}{2.196927in}}{\pgfqpoint{1.483747in}{2.207526in}}{\pgfqpoint{1.475933in}{2.215340in}}%
\pgfpathcurveto{\pgfqpoint{1.468120in}{2.223153in}}{\pgfqpoint{1.457521in}{2.227544in}}{\pgfqpoint{1.446470in}{2.227544in}}%
\pgfpathcurveto{\pgfqpoint{1.435420in}{2.227544in}}{\pgfqpoint{1.424821in}{2.223153in}}{\pgfqpoint{1.417008in}{2.215340in}}%
\pgfpathcurveto{\pgfqpoint{1.409194in}{2.207526in}}{\pgfqpoint{1.404804in}{2.196927in}}{\pgfqpoint{1.404804in}{2.185877in}}%
\pgfpathcurveto{\pgfqpoint{1.404804in}{2.174827in}}{\pgfqpoint{1.409194in}{2.164228in}}{\pgfqpoint{1.417008in}{2.156414in}}%
\pgfpathcurveto{\pgfqpoint{1.424821in}{2.148600in}}{\pgfqpoint{1.435420in}{2.144210in}}{\pgfqpoint{1.446470in}{2.144210in}}%
\pgfpathclose%
\pgfusepath{stroke,fill}%
\end{pgfscope}%
\begin{pgfscope}%
\pgfpathrectangle{\pgfqpoint{0.787074in}{0.548769in}}{\pgfqpoint{5.062926in}{3.102590in}}%
\pgfusepath{clip}%
\pgfsetbuttcap%
\pgfsetroundjoin%
\definecolor{currentfill}{rgb}{1.000000,0.498039,0.054902}%
\pgfsetfillcolor{currentfill}%
\pgfsetlinewidth{1.003750pt}%
\definecolor{currentstroke}{rgb}{1.000000,0.498039,0.054902}%
\pgfsetstrokecolor{currentstroke}%
\pgfsetdash{}{0pt}%
\pgfpathmoveto{\pgfqpoint{1.128497in}{2.799180in}}%
\pgfpathcurveto{\pgfqpoint{1.139548in}{2.799180in}}{\pgfqpoint{1.150147in}{2.803570in}}{\pgfqpoint{1.157960in}{2.811384in}}%
\pgfpathcurveto{\pgfqpoint{1.165774in}{2.819197in}}{\pgfqpoint{1.170164in}{2.829796in}}{\pgfqpoint{1.170164in}{2.840846in}}%
\pgfpathcurveto{\pgfqpoint{1.170164in}{2.851897in}}{\pgfqpoint{1.165774in}{2.862496in}}{\pgfqpoint{1.157960in}{2.870309in}}%
\pgfpathcurveto{\pgfqpoint{1.150147in}{2.878123in}}{\pgfqpoint{1.139548in}{2.882513in}}{\pgfqpoint{1.128497in}{2.882513in}}%
\pgfpathcurveto{\pgfqpoint{1.117447in}{2.882513in}}{\pgfqpoint{1.106848in}{2.878123in}}{\pgfqpoint{1.099035in}{2.870309in}}%
\pgfpathcurveto{\pgfqpoint{1.091221in}{2.862496in}}{\pgfqpoint{1.086831in}{2.851897in}}{\pgfqpoint{1.086831in}{2.840846in}}%
\pgfpathcurveto{\pgfqpoint{1.086831in}{2.829796in}}{\pgfqpoint{1.091221in}{2.819197in}}{\pgfqpoint{1.099035in}{2.811384in}}%
\pgfpathcurveto{\pgfqpoint{1.106848in}{2.803570in}}{\pgfqpoint{1.117447in}{2.799180in}}{\pgfqpoint{1.128497in}{2.799180in}}%
\pgfpathclose%
\pgfusepath{stroke,fill}%
\end{pgfscope}%
\begin{pgfscope}%
\pgfpathrectangle{\pgfqpoint{0.787074in}{0.548769in}}{\pgfqpoint{5.062926in}{3.102590in}}%
\pgfusepath{clip}%
\pgfsetbuttcap%
\pgfsetroundjoin%
\definecolor{currentfill}{rgb}{1.000000,0.498039,0.054902}%
\pgfsetfillcolor{currentfill}%
\pgfsetlinewidth{1.003750pt}%
\definecolor{currentstroke}{rgb}{1.000000,0.498039,0.054902}%
\pgfsetstrokecolor{currentstroke}%
\pgfsetdash{}{0pt}%
\pgfpathmoveto{\pgfqpoint{1.787156in}{2.526596in}}%
\pgfpathcurveto{\pgfqpoint{1.798206in}{2.526596in}}{\pgfqpoint{1.808805in}{2.530987in}}{\pgfqpoint{1.816619in}{2.538800in}}%
\pgfpathcurveto{\pgfqpoint{1.824432in}{2.546614in}}{\pgfqpoint{1.828823in}{2.557213in}}{\pgfqpoint{1.828823in}{2.568263in}}%
\pgfpathcurveto{\pgfqpoint{1.828823in}{2.579313in}}{\pgfqpoint{1.824432in}{2.589912in}}{\pgfqpoint{1.816619in}{2.597726in}}%
\pgfpathcurveto{\pgfqpoint{1.808805in}{2.605540in}}{\pgfqpoint{1.798206in}{2.609930in}}{\pgfqpoint{1.787156in}{2.609930in}}%
\pgfpathcurveto{\pgfqpoint{1.776106in}{2.609930in}}{\pgfqpoint{1.765507in}{2.605540in}}{\pgfqpoint{1.757693in}{2.597726in}}%
\pgfpathcurveto{\pgfqpoint{1.749880in}{2.589912in}}{\pgfqpoint{1.745489in}{2.579313in}}{\pgfqpoint{1.745489in}{2.568263in}}%
\pgfpathcurveto{\pgfqpoint{1.745489in}{2.557213in}}{\pgfqpoint{1.749880in}{2.546614in}}{\pgfqpoint{1.757693in}{2.538800in}}%
\pgfpathcurveto{\pgfqpoint{1.765507in}{2.530987in}}{\pgfqpoint{1.776106in}{2.526596in}}{\pgfqpoint{1.787156in}{2.526596in}}%
\pgfpathclose%
\pgfusepath{stroke,fill}%
\end{pgfscope}%
\begin{pgfscope}%
\pgfpathrectangle{\pgfqpoint{0.787074in}{0.548769in}}{\pgfqpoint{5.062926in}{3.102590in}}%
\pgfusepath{clip}%
\pgfsetbuttcap%
\pgfsetroundjoin%
\definecolor{currentfill}{rgb}{1.000000,0.498039,0.054902}%
\pgfsetfillcolor{currentfill}%
\pgfsetlinewidth{1.003750pt}%
\definecolor{currentstroke}{rgb}{1.000000,0.498039,0.054902}%
\pgfsetstrokecolor{currentstroke}%
\pgfsetdash{}{0pt}%
\pgfpathmoveto{\pgfqpoint{1.244680in}{2.883497in}}%
\pgfpathcurveto{\pgfqpoint{1.255730in}{2.883497in}}{\pgfqpoint{1.266329in}{2.887888in}}{\pgfqpoint{1.274143in}{2.895701in}}%
\pgfpathcurveto{\pgfqpoint{1.281956in}{2.903515in}}{\pgfqpoint{1.286347in}{2.914114in}}{\pgfqpoint{1.286347in}{2.925164in}}%
\pgfpathcurveto{\pgfqpoint{1.286347in}{2.936214in}}{\pgfqpoint{1.281956in}{2.946813in}}{\pgfqpoint{1.274143in}{2.954627in}}%
\pgfpathcurveto{\pgfqpoint{1.266329in}{2.962440in}}{\pgfqpoint{1.255730in}{2.966831in}}{\pgfqpoint{1.244680in}{2.966831in}}%
\pgfpathcurveto{\pgfqpoint{1.233630in}{2.966831in}}{\pgfqpoint{1.223031in}{2.962440in}}{\pgfqpoint{1.215217in}{2.954627in}}%
\pgfpathcurveto{\pgfqpoint{1.207403in}{2.946813in}}{\pgfqpoint{1.203013in}{2.936214in}}{\pgfqpoint{1.203013in}{2.925164in}}%
\pgfpathcurveto{\pgfqpoint{1.203013in}{2.914114in}}{\pgfqpoint{1.207403in}{2.903515in}}{\pgfqpoint{1.215217in}{2.895701in}}%
\pgfpathcurveto{\pgfqpoint{1.223031in}{2.887888in}}{\pgfqpoint{1.233630in}{2.883497in}}{\pgfqpoint{1.244680in}{2.883497in}}%
\pgfpathclose%
\pgfusepath{stroke,fill}%
\end{pgfscope}%
\begin{pgfscope}%
\pgfpathrectangle{\pgfqpoint{0.787074in}{0.548769in}}{\pgfqpoint{5.062926in}{3.102590in}}%
\pgfusepath{clip}%
\pgfsetbuttcap%
\pgfsetroundjoin%
\definecolor{currentfill}{rgb}{1.000000,0.498039,0.054902}%
\pgfsetfillcolor{currentfill}%
\pgfsetlinewidth{1.003750pt}%
\definecolor{currentstroke}{rgb}{1.000000,0.498039,0.054902}%
\pgfsetstrokecolor{currentstroke}%
\pgfsetdash{}{0pt}%
\pgfpathmoveto{\pgfqpoint{1.642866in}{2.042862in}}%
\pgfpathcurveto{\pgfqpoint{1.653916in}{2.042862in}}{\pgfqpoint{1.664515in}{2.047252in}}{\pgfqpoint{1.672328in}{2.055065in}}%
\pgfpathcurveto{\pgfqpoint{1.680142in}{2.062879in}}{\pgfqpoint{1.684532in}{2.073478in}}{\pgfqpoint{1.684532in}{2.084528in}}%
\pgfpathcurveto{\pgfqpoint{1.684532in}{2.095578in}}{\pgfqpoint{1.680142in}{2.106177in}}{\pgfqpoint{1.672328in}{2.113991in}}%
\pgfpathcurveto{\pgfqpoint{1.664515in}{2.121805in}}{\pgfqpoint{1.653916in}{2.126195in}}{\pgfqpoint{1.642866in}{2.126195in}}%
\pgfpathcurveto{\pgfqpoint{1.631815in}{2.126195in}}{\pgfqpoint{1.621216in}{2.121805in}}{\pgfqpoint{1.613403in}{2.113991in}}%
\pgfpathcurveto{\pgfqpoint{1.605589in}{2.106177in}}{\pgfqpoint{1.601199in}{2.095578in}}{\pgfqpoint{1.601199in}{2.084528in}}%
\pgfpathcurveto{\pgfqpoint{1.601199in}{2.073478in}}{\pgfqpoint{1.605589in}{2.062879in}}{\pgfqpoint{1.613403in}{2.055065in}}%
\pgfpathcurveto{\pgfqpoint{1.621216in}{2.047252in}}{\pgfqpoint{1.631815in}{2.042862in}}{\pgfqpoint{1.642866in}{2.042862in}}%
\pgfpathclose%
\pgfusepath{stroke,fill}%
\end{pgfscope}%
\begin{pgfscope}%
\pgfpathrectangle{\pgfqpoint{0.787074in}{0.548769in}}{\pgfqpoint{5.062926in}{3.102590in}}%
\pgfusepath{clip}%
\pgfsetbuttcap%
\pgfsetroundjoin%
\definecolor{currentfill}{rgb}{1.000000,0.498039,0.054902}%
\pgfsetfillcolor{currentfill}%
\pgfsetlinewidth{1.003750pt}%
\definecolor{currentstroke}{rgb}{1.000000,0.498039,0.054902}%
\pgfsetstrokecolor{currentstroke}%
\pgfsetdash{}{0pt}%
\pgfpathmoveto{\pgfqpoint{1.354747in}{2.760534in}}%
\pgfpathcurveto{\pgfqpoint{1.365798in}{2.760534in}}{\pgfqpoint{1.376397in}{2.764924in}}{\pgfqpoint{1.384210in}{2.772738in}}%
\pgfpathcurveto{\pgfqpoint{1.392024in}{2.780552in}}{\pgfqpoint{1.396414in}{2.791151in}}{\pgfqpoint{1.396414in}{2.802201in}}%
\pgfpathcurveto{\pgfqpoint{1.396414in}{2.813251in}}{\pgfqpoint{1.392024in}{2.823850in}}{\pgfqpoint{1.384210in}{2.831664in}}%
\pgfpathcurveto{\pgfqpoint{1.376397in}{2.839477in}}{\pgfqpoint{1.365798in}{2.843868in}}{\pgfqpoint{1.354747in}{2.843868in}}%
\pgfpathcurveto{\pgfqpoint{1.343697in}{2.843868in}}{\pgfqpoint{1.333098in}{2.839477in}}{\pgfqpoint{1.325285in}{2.831664in}}%
\pgfpathcurveto{\pgfqpoint{1.317471in}{2.823850in}}{\pgfqpoint{1.313081in}{2.813251in}}{\pgfqpoint{1.313081in}{2.802201in}}%
\pgfpathcurveto{\pgfqpoint{1.313081in}{2.791151in}}{\pgfqpoint{1.317471in}{2.780552in}}{\pgfqpoint{1.325285in}{2.772738in}}%
\pgfpathcurveto{\pgfqpoint{1.333098in}{2.764924in}}{\pgfqpoint{1.343697in}{2.760534in}}{\pgfqpoint{1.354747in}{2.760534in}}%
\pgfpathclose%
\pgfusepath{stroke,fill}%
\end{pgfscope}%
\begin{pgfscope}%
\pgfpathrectangle{\pgfqpoint{0.787074in}{0.548769in}}{\pgfqpoint{5.062926in}{3.102590in}}%
\pgfusepath{clip}%
\pgfsetbuttcap%
\pgfsetroundjoin%
\definecolor{currentfill}{rgb}{0.121569,0.466667,0.705882}%
\pgfsetfillcolor{currentfill}%
\pgfsetlinewidth{1.003750pt}%
\definecolor{currentstroke}{rgb}{0.121569,0.466667,0.705882}%
\pgfsetstrokecolor{currentstroke}%
\pgfsetdash{}{0pt}%
\pgfpathmoveto{\pgfqpoint{2.400390in}{0.648129in}}%
\pgfpathcurveto{\pgfqpoint{2.411440in}{0.648129in}}{\pgfqpoint{2.422039in}{0.652519in}}{\pgfqpoint{2.429852in}{0.660333in}}%
\pgfpathcurveto{\pgfqpoint{2.437666in}{0.668146in}}{\pgfqpoint{2.442056in}{0.678745in}}{\pgfqpoint{2.442056in}{0.689796in}}%
\pgfpathcurveto{\pgfqpoint{2.442056in}{0.700846in}}{\pgfqpoint{2.437666in}{0.711445in}}{\pgfqpoint{2.429852in}{0.719258in}}%
\pgfpathcurveto{\pgfqpoint{2.422039in}{0.727072in}}{\pgfqpoint{2.411440in}{0.731462in}}{\pgfqpoint{2.400390in}{0.731462in}}%
\pgfpathcurveto{\pgfqpoint{2.389340in}{0.731462in}}{\pgfqpoint{2.378741in}{0.727072in}}{\pgfqpoint{2.370927in}{0.719258in}}%
\pgfpathcurveto{\pgfqpoint{2.363113in}{0.711445in}}{\pgfqpoint{2.358723in}{0.700846in}}{\pgfqpoint{2.358723in}{0.689796in}}%
\pgfpathcurveto{\pgfqpoint{2.358723in}{0.678745in}}{\pgfqpoint{2.363113in}{0.668146in}}{\pgfqpoint{2.370927in}{0.660333in}}%
\pgfpathcurveto{\pgfqpoint{2.378741in}{0.652519in}}{\pgfqpoint{2.389340in}{0.648129in}}{\pgfqpoint{2.400390in}{0.648129in}}%
\pgfpathclose%
\pgfusepath{stroke,fill}%
\end{pgfscope}%
\begin{pgfscope}%
\pgfpathrectangle{\pgfqpoint{0.787074in}{0.548769in}}{\pgfqpoint{5.062926in}{3.102590in}}%
\pgfusepath{clip}%
\pgfsetbuttcap%
\pgfsetroundjoin%
\definecolor{currentfill}{rgb}{0.121569,0.466667,0.705882}%
\pgfsetfillcolor{currentfill}%
\pgfsetlinewidth{1.003750pt}%
\definecolor{currentstroke}{rgb}{0.121569,0.466667,0.705882}%
\pgfsetstrokecolor{currentstroke}%
\pgfsetdash{}{0pt}%
\pgfpathmoveto{\pgfqpoint{1.971126in}{0.658321in}}%
\pgfpathcurveto{\pgfqpoint{1.982176in}{0.658321in}}{\pgfqpoint{1.992775in}{0.662711in}}{\pgfqpoint{2.000589in}{0.670525in}}%
\pgfpathcurveto{\pgfqpoint{2.008402in}{0.678338in}}{\pgfqpoint{2.012793in}{0.688937in}}{\pgfqpoint{2.012793in}{0.699987in}}%
\pgfpathcurveto{\pgfqpoint{2.012793in}{0.711038in}}{\pgfqpoint{2.008402in}{0.721637in}}{\pgfqpoint{2.000589in}{0.729450in}}%
\pgfpathcurveto{\pgfqpoint{1.992775in}{0.737264in}}{\pgfqpoint{1.982176in}{0.741654in}}{\pgfqpoint{1.971126in}{0.741654in}}%
\pgfpathcurveto{\pgfqpoint{1.960076in}{0.741654in}}{\pgfqpoint{1.949477in}{0.737264in}}{\pgfqpoint{1.941663in}{0.729450in}}%
\pgfpathcurveto{\pgfqpoint{1.933850in}{0.721637in}}{\pgfqpoint{1.929459in}{0.711038in}}{\pgfqpoint{1.929459in}{0.699987in}}%
\pgfpathcurveto{\pgfqpoint{1.929459in}{0.688937in}}{\pgfqpoint{1.933850in}{0.678338in}}{\pgfqpoint{1.941663in}{0.670525in}}%
\pgfpathcurveto{\pgfqpoint{1.949477in}{0.662711in}}{\pgfqpoint{1.960076in}{0.658321in}}{\pgfqpoint{1.971126in}{0.658321in}}%
\pgfpathclose%
\pgfusepath{stroke,fill}%
\end{pgfscope}%
\begin{pgfscope}%
\pgfpathrectangle{\pgfqpoint{0.787074in}{0.548769in}}{\pgfqpoint{5.062926in}{3.102590in}}%
\pgfusepath{clip}%
\pgfsetbuttcap%
\pgfsetroundjoin%
\definecolor{currentfill}{rgb}{1.000000,0.498039,0.054902}%
\pgfsetfillcolor{currentfill}%
\pgfsetlinewidth{1.003750pt}%
\definecolor{currentstroke}{rgb}{1.000000,0.498039,0.054902}%
\pgfsetstrokecolor{currentstroke}%
\pgfsetdash{}{0pt}%
\pgfpathmoveto{\pgfqpoint{1.327231in}{1.208140in}}%
\pgfpathcurveto{\pgfqpoint{1.338281in}{1.208140in}}{\pgfqpoint{1.348880in}{1.212530in}}{\pgfqpoint{1.356693in}{1.220344in}}%
\pgfpathcurveto{\pgfqpoint{1.364507in}{1.228157in}}{\pgfqpoint{1.368897in}{1.238756in}}{\pgfqpoint{1.368897in}{1.249807in}}%
\pgfpathcurveto{\pgfqpoint{1.368897in}{1.260857in}}{\pgfqpoint{1.364507in}{1.271456in}}{\pgfqpoint{1.356693in}{1.279269in}}%
\pgfpathcurveto{\pgfqpoint{1.348880in}{1.287083in}}{\pgfqpoint{1.338281in}{1.291473in}}{\pgfqpoint{1.327231in}{1.291473in}}%
\pgfpathcurveto{\pgfqpoint{1.316180in}{1.291473in}}{\pgfqpoint{1.305581in}{1.287083in}}{\pgfqpoint{1.297768in}{1.279269in}}%
\pgfpathcurveto{\pgfqpoint{1.289954in}{1.271456in}}{\pgfqpoint{1.285564in}{1.260857in}}{\pgfqpoint{1.285564in}{1.249807in}}%
\pgfpathcurveto{\pgfqpoint{1.285564in}{1.238756in}}{\pgfqpoint{1.289954in}{1.228157in}}{\pgfqpoint{1.297768in}{1.220344in}}%
\pgfpathcurveto{\pgfqpoint{1.305581in}{1.212530in}}{\pgfqpoint{1.316180in}{1.208140in}}{\pgfqpoint{1.327231in}{1.208140in}}%
\pgfpathclose%
\pgfusepath{stroke,fill}%
\end{pgfscope}%
\begin{pgfscope}%
\pgfpathrectangle{\pgfqpoint{0.787074in}{0.548769in}}{\pgfqpoint{5.062926in}{3.102590in}}%
\pgfusepath{clip}%
\pgfsetbuttcap%
\pgfsetroundjoin%
\definecolor{currentfill}{rgb}{1.000000,0.498039,0.054902}%
\pgfsetfillcolor{currentfill}%
\pgfsetlinewidth{1.003750pt}%
\definecolor{currentstroke}{rgb}{1.000000,0.498039,0.054902}%
\pgfsetstrokecolor{currentstroke}%
\pgfsetdash{}{0pt}%
\pgfpathmoveto{\pgfqpoint{1.582745in}{2.220680in}}%
\pgfpathcurveto{\pgfqpoint{1.593795in}{2.220680in}}{\pgfqpoint{1.604394in}{2.225070in}}{\pgfqpoint{1.612207in}{2.232884in}}%
\pgfpathcurveto{\pgfqpoint{1.620021in}{2.240698in}}{\pgfqpoint{1.624411in}{2.251297in}}{\pgfqpoint{1.624411in}{2.262347in}}%
\pgfpathcurveto{\pgfqpoint{1.624411in}{2.273397in}}{\pgfqpoint{1.620021in}{2.283996in}}{\pgfqpoint{1.612207in}{2.291810in}}%
\pgfpathcurveto{\pgfqpoint{1.604394in}{2.299623in}}{\pgfqpoint{1.593795in}{2.304013in}}{\pgfqpoint{1.582745in}{2.304013in}}%
\pgfpathcurveto{\pgfqpoint{1.571695in}{2.304013in}}{\pgfqpoint{1.561095in}{2.299623in}}{\pgfqpoint{1.553282in}{2.291810in}}%
\pgfpathcurveto{\pgfqpoint{1.545468in}{2.283996in}}{\pgfqpoint{1.541078in}{2.273397in}}{\pgfqpoint{1.541078in}{2.262347in}}%
\pgfpathcurveto{\pgfqpoint{1.541078in}{2.251297in}}{\pgfqpoint{1.545468in}{2.240698in}}{\pgfqpoint{1.553282in}{2.232884in}}%
\pgfpathcurveto{\pgfqpoint{1.561095in}{2.225070in}}{\pgfqpoint{1.571695in}{2.220680in}}{\pgfqpoint{1.582745in}{2.220680in}}%
\pgfpathclose%
\pgfusepath{stroke,fill}%
\end{pgfscope}%
\begin{pgfscope}%
\pgfpathrectangle{\pgfqpoint{0.787074in}{0.548769in}}{\pgfqpoint{5.062926in}{3.102590in}}%
\pgfusepath{clip}%
\pgfsetbuttcap%
\pgfsetroundjoin%
\definecolor{currentfill}{rgb}{0.121569,0.466667,0.705882}%
\pgfsetfillcolor{currentfill}%
\pgfsetlinewidth{1.003750pt}%
\definecolor{currentstroke}{rgb}{0.121569,0.466667,0.705882}%
\pgfsetstrokecolor{currentstroke}%
\pgfsetdash{}{0pt}%
\pgfpathmoveto{\pgfqpoint{5.262147in}{0.648130in}}%
\pgfpathcurveto{\pgfqpoint{5.273197in}{0.648130in}}{\pgfqpoint{5.283796in}{0.652520in}}{\pgfqpoint{5.291610in}{0.660334in}}%
\pgfpathcurveto{\pgfqpoint{5.299424in}{0.668148in}}{\pgfqpoint{5.303814in}{0.678747in}}{\pgfqpoint{5.303814in}{0.689797in}}%
\pgfpathcurveto{\pgfqpoint{5.303814in}{0.700847in}}{\pgfqpoint{5.299424in}{0.711446in}}{\pgfqpoint{5.291610in}{0.719260in}}%
\pgfpathcurveto{\pgfqpoint{5.283796in}{0.727073in}}{\pgfqpoint{5.273197in}{0.731464in}}{\pgfqpoint{5.262147in}{0.731464in}}%
\pgfpathcurveto{\pgfqpoint{5.251097in}{0.731464in}}{\pgfqpoint{5.240498in}{0.727073in}}{\pgfqpoint{5.232685in}{0.719260in}}%
\pgfpathcurveto{\pgfqpoint{5.224871in}{0.711446in}}{\pgfqpoint{5.220481in}{0.700847in}}{\pgfqpoint{5.220481in}{0.689797in}}%
\pgfpathcurveto{\pgfqpoint{5.220481in}{0.678747in}}{\pgfqpoint{5.224871in}{0.668148in}}{\pgfqpoint{5.232685in}{0.660334in}}%
\pgfpathcurveto{\pgfqpoint{5.240498in}{0.652520in}}{\pgfqpoint{5.251097in}{0.648130in}}{\pgfqpoint{5.262147in}{0.648130in}}%
\pgfpathclose%
\pgfusepath{stroke,fill}%
\end{pgfscope}%
\begin{pgfscope}%
\pgfpathrectangle{\pgfqpoint{0.787074in}{0.548769in}}{\pgfqpoint{5.062926in}{3.102590in}}%
\pgfusepath{clip}%
\pgfsetbuttcap%
\pgfsetroundjoin%
\definecolor{currentfill}{rgb}{0.121569,0.466667,0.705882}%
\pgfsetfillcolor{currentfill}%
\pgfsetlinewidth{1.003750pt}%
\definecolor{currentstroke}{rgb}{0.121569,0.466667,0.705882}%
\pgfsetstrokecolor{currentstroke}%
\pgfsetdash{}{0pt}%
\pgfpathmoveto{\pgfqpoint{1.684950in}{0.648134in}}%
\pgfpathcurveto{\pgfqpoint{1.696000in}{0.648134in}}{\pgfqpoint{1.706599in}{0.652524in}}{\pgfqpoint{1.714413in}{0.660338in}}%
\pgfpathcurveto{\pgfqpoint{1.722227in}{0.668151in}}{\pgfqpoint{1.726617in}{0.678750in}}{\pgfqpoint{1.726617in}{0.689800in}}%
\pgfpathcurveto{\pgfqpoint{1.726617in}{0.700850in}}{\pgfqpoint{1.722227in}{0.711450in}}{\pgfqpoint{1.714413in}{0.719263in}}%
\pgfpathcurveto{\pgfqpoint{1.706599in}{0.727077in}}{\pgfqpoint{1.696000in}{0.731467in}}{\pgfqpoint{1.684950in}{0.731467in}}%
\pgfpathcurveto{\pgfqpoint{1.673900in}{0.731467in}}{\pgfqpoint{1.663301in}{0.727077in}}{\pgfqpoint{1.655488in}{0.719263in}}%
\pgfpathcurveto{\pgfqpoint{1.647674in}{0.711450in}}{\pgfqpoint{1.643284in}{0.700850in}}{\pgfqpoint{1.643284in}{0.689800in}}%
\pgfpathcurveto{\pgfqpoint{1.643284in}{0.678750in}}{\pgfqpoint{1.647674in}{0.668151in}}{\pgfqpoint{1.655488in}{0.660338in}}%
\pgfpathcurveto{\pgfqpoint{1.663301in}{0.652524in}}{\pgfqpoint{1.673900in}{0.648134in}}{\pgfqpoint{1.684950in}{0.648134in}}%
\pgfpathclose%
\pgfusepath{stroke,fill}%
\end{pgfscope}%
\begin{pgfscope}%
\pgfpathrectangle{\pgfqpoint{0.787074in}{0.548769in}}{\pgfqpoint{5.062926in}{3.102590in}}%
\pgfusepath{clip}%
\pgfsetbuttcap%
\pgfsetroundjoin%
\definecolor{currentfill}{rgb}{1.000000,0.498039,0.054902}%
\pgfsetfillcolor{currentfill}%
\pgfsetlinewidth{1.003750pt}%
\definecolor{currentstroke}{rgb}{1.000000,0.498039,0.054902}%
\pgfsetstrokecolor{currentstroke}%
\pgfsetdash{}{0pt}%
\pgfpathmoveto{\pgfqpoint{1.017207in}{3.468665in}}%
\pgfpathcurveto{\pgfqpoint{1.028257in}{3.468665in}}{\pgfqpoint{1.038856in}{3.473055in}}{\pgfqpoint{1.046670in}{3.480869in}}%
\pgfpathcurveto{\pgfqpoint{1.054483in}{3.488683in}}{\pgfqpoint{1.058874in}{3.499282in}}{\pgfqpoint{1.058874in}{3.510332in}}%
\pgfpathcurveto{\pgfqpoint{1.058874in}{3.521382in}}{\pgfqpoint{1.054483in}{3.531981in}}{\pgfqpoint{1.046670in}{3.539795in}}%
\pgfpathcurveto{\pgfqpoint{1.038856in}{3.547608in}}{\pgfqpoint{1.028257in}{3.551998in}}{\pgfqpoint{1.017207in}{3.551998in}}%
\pgfpathcurveto{\pgfqpoint{1.006157in}{3.551998in}}{\pgfqpoint{0.995558in}{3.547608in}}{\pgfqpoint{0.987744in}{3.539795in}}%
\pgfpathcurveto{\pgfqpoint{0.979930in}{3.531981in}}{\pgfqpoint{0.975540in}{3.521382in}}{\pgfqpoint{0.975540in}{3.510332in}}%
\pgfpathcurveto{\pgfqpoint{0.975540in}{3.499282in}}{\pgfqpoint{0.979930in}{3.488683in}}{\pgfqpoint{0.987744in}{3.480869in}}%
\pgfpathcurveto{\pgfqpoint{0.995558in}{3.473055in}}{\pgfqpoint{1.006157in}{3.468665in}}{\pgfqpoint{1.017207in}{3.468665in}}%
\pgfpathclose%
\pgfusepath{stroke,fill}%
\end{pgfscope}%
\begin{pgfscope}%
\pgfpathrectangle{\pgfqpoint{0.787074in}{0.548769in}}{\pgfqpoint{5.062926in}{3.102590in}}%
\pgfusepath{clip}%
\pgfsetbuttcap%
\pgfsetroundjoin%
\definecolor{currentfill}{rgb}{1.000000,0.498039,0.054902}%
\pgfsetfillcolor{currentfill}%
\pgfsetlinewidth{1.003750pt}%
\definecolor{currentstroke}{rgb}{1.000000,0.498039,0.054902}%
\pgfsetstrokecolor{currentstroke}%
\pgfsetdash{}{0pt}%
\pgfpathmoveto{\pgfqpoint{1.267611in}{2.370534in}}%
\pgfpathcurveto{\pgfqpoint{1.278661in}{2.370534in}}{\pgfqpoint{1.289260in}{2.374924in}}{\pgfqpoint{1.297073in}{2.382738in}}%
\pgfpathcurveto{\pgfqpoint{1.304887in}{2.390551in}}{\pgfqpoint{1.309277in}{2.401150in}}{\pgfqpoint{1.309277in}{2.412201in}}%
\pgfpathcurveto{\pgfqpoint{1.309277in}{2.423251in}}{\pgfqpoint{1.304887in}{2.433850in}}{\pgfqpoint{1.297073in}{2.441663in}}%
\pgfpathcurveto{\pgfqpoint{1.289260in}{2.449477in}}{\pgfqpoint{1.278661in}{2.453867in}}{\pgfqpoint{1.267611in}{2.453867in}}%
\pgfpathcurveto{\pgfqpoint{1.256561in}{2.453867in}}{\pgfqpoint{1.245961in}{2.449477in}}{\pgfqpoint{1.238148in}{2.441663in}}%
\pgfpathcurveto{\pgfqpoint{1.230334in}{2.433850in}}{\pgfqpoint{1.225944in}{2.423251in}}{\pgfqpoint{1.225944in}{2.412201in}}%
\pgfpathcurveto{\pgfqpoint{1.225944in}{2.401150in}}{\pgfqpoint{1.230334in}{2.390551in}}{\pgfqpoint{1.238148in}{2.382738in}}%
\pgfpathcurveto{\pgfqpoint{1.245961in}{2.374924in}}{\pgfqpoint{1.256561in}{2.370534in}}{\pgfqpoint{1.267611in}{2.370534in}}%
\pgfpathclose%
\pgfusepath{stroke,fill}%
\end{pgfscope}%
\begin{pgfscope}%
\pgfpathrectangle{\pgfqpoint{0.787074in}{0.548769in}}{\pgfqpoint{5.062926in}{3.102590in}}%
\pgfusepath{clip}%
\pgfsetbuttcap%
\pgfsetroundjoin%
\definecolor{currentfill}{rgb}{1.000000,0.498039,0.054902}%
\pgfsetfillcolor{currentfill}%
\pgfsetlinewidth{1.003750pt}%
\definecolor{currentstroke}{rgb}{1.000000,0.498039,0.054902}%
\pgfsetstrokecolor{currentstroke}%
\pgfsetdash{}{0pt}%
\pgfpathmoveto{\pgfqpoint{2.785626in}{2.417203in}}%
\pgfpathcurveto{\pgfqpoint{2.796676in}{2.417203in}}{\pgfqpoint{2.807275in}{2.421593in}}{\pgfqpoint{2.815089in}{2.429407in}}%
\pgfpathcurveto{\pgfqpoint{2.822903in}{2.437220in}}{\pgfqpoint{2.827293in}{2.447819in}}{\pgfqpoint{2.827293in}{2.458870in}}%
\pgfpathcurveto{\pgfqpoint{2.827293in}{2.469920in}}{\pgfqpoint{2.822903in}{2.480519in}}{\pgfqpoint{2.815089in}{2.488332in}}%
\pgfpathcurveto{\pgfqpoint{2.807275in}{2.496146in}}{\pgfqpoint{2.796676in}{2.500536in}}{\pgfqpoint{2.785626in}{2.500536in}}%
\pgfpathcurveto{\pgfqpoint{2.774576in}{2.500536in}}{\pgfqpoint{2.763977in}{2.496146in}}{\pgfqpoint{2.756164in}{2.488332in}}%
\pgfpathcurveto{\pgfqpoint{2.748350in}{2.480519in}}{\pgfqpoint{2.743960in}{2.469920in}}{\pgfqpoint{2.743960in}{2.458870in}}%
\pgfpathcurveto{\pgfqpoint{2.743960in}{2.447819in}}{\pgfqpoint{2.748350in}{2.437220in}}{\pgfqpoint{2.756164in}{2.429407in}}%
\pgfpathcurveto{\pgfqpoint{2.763977in}{2.421593in}}{\pgfqpoint{2.774576in}{2.417203in}}{\pgfqpoint{2.785626in}{2.417203in}}%
\pgfpathclose%
\pgfusepath{stroke,fill}%
\end{pgfscope}%
\begin{pgfscope}%
\pgfpathrectangle{\pgfqpoint{0.787074in}{0.548769in}}{\pgfqpoint{5.062926in}{3.102590in}}%
\pgfusepath{clip}%
\pgfsetbuttcap%
\pgfsetroundjoin%
\definecolor{currentfill}{rgb}{1.000000,0.498039,0.054902}%
\pgfsetfillcolor{currentfill}%
\pgfsetlinewidth{1.003750pt}%
\definecolor{currentstroke}{rgb}{1.000000,0.498039,0.054902}%
\pgfsetstrokecolor{currentstroke}%
\pgfsetdash{}{0pt}%
\pgfpathmoveto{\pgfqpoint{1.971126in}{2.967508in}}%
\pgfpathcurveto{\pgfqpoint{1.982176in}{2.967508in}}{\pgfqpoint{1.992775in}{2.971899in}}{\pgfqpoint{2.000589in}{2.979712in}}%
\pgfpathcurveto{\pgfqpoint{2.008402in}{2.987526in}}{\pgfqpoint{2.012793in}{2.998125in}}{\pgfqpoint{2.012793in}{3.009175in}}%
\pgfpathcurveto{\pgfqpoint{2.012793in}{3.020225in}}{\pgfqpoint{2.008402in}{3.030824in}}{\pgfqpoint{2.000589in}{3.038638in}}%
\pgfpathcurveto{\pgfqpoint{1.992775in}{3.046451in}}{\pgfqpoint{1.982176in}{3.050842in}}{\pgfqpoint{1.971126in}{3.050842in}}%
\pgfpathcurveto{\pgfqpoint{1.960076in}{3.050842in}}{\pgfqpoint{1.949477in}{3.046451in}}{\pgfqpoint{1.941663in}{3.038638in}}%
\pgfpathcurveto{\pgfqpoint{1.933850in}{3.030824in}}{\pgfqpoint{1.929459in}{3.020225in}}{\pgfqpoint{1.929459in}{3.009175in}}%
\pgfpathcurveto{\pgfqpoint{1.929459in}{2.998125in}}{\pgfqpoint{1.933850in}{2.987526in}}{\pgfqpoint{1.941663in}{2.979712in}}%
\pgfpathcurveto{\pgfqpoint{1.949477in}{2.971899in}}{\pgfqpoint{1.960076in}{2.967508in}}{\pgfqpoint{1.971126in}{2.967508in}}%
\pgfpathclose%
\pgfusepath{stroke,fill}%
\end{pgfscope}%
\begin{pgfscope}%
\pgfpathrectangle{\pgfqpoint{0.787074in}{0.548769in}}{\pgfqpoint{5.062926in}{3.102590in}}%
\pgfusepath{clip}%
\pgfsetbuttcap%
\pgfsetroundjoin%
\definecolor{currentfill}{rgb}{1.000000,0.498039,0.054902}%
\pgfsetfillcolor{currentfill}%
\pgfsetlinewidth{1.003750pt}%
\definecolor{currentstroke}{rgb}{1.000000,0.498039,0.054902}%
\pgfsetstrokecolor{currentstroke}%
\pgfsetdash{}{0pt}%
\pgfpathmoveto{\pgfqpoint{1.299714in}{2.421885in}}%
\pgfpathcurveto{\pgfqpoint{1.310764in}{2.421885in}}{\pgfqpoint{1.321363in}{2.426275in}}{\pgfqpoint{1.329176in}{2.434089in}}%
\pgfpathcurveto{\pgfqpoint{1.336990in}{2.441902in}}{\pgfqpoint{1.341380in}{2.452501in}}{\pgfqpoint{1.341380in}{2.463551in}}%
\pgfpathcurveto{\pgfqpoint{1.341380in}{2.474601in}}{\pgfqpoint{1.336990in}{2.485201in}}{\pgfqpoint{1.329176in}{2.493014in}}%
\pgfpathcurveto{\pgfqpoint{1.321363in}{2.500828in}}{\pgfqpoint{1.310764in}{2.505218in}}{\pgfqpoint{1.299714in}{2.505218in}}%
\pgfpathcurveto{\pgfqpoint{1.288664in}{2.505218in}}{\pgfqpoint{1.278065in}{2.500828in}}{\pgfqpoint{1.270251in}{2.493014in}}%
\pgfpathcurveto{\pgfqpoint{1.262437in}{2.485201in}}{\pgfqpoint{1.258047in}{2.474601in}}{\pgfqpoint{1.258047in}{2.463551in}}%
\pgfpathcurveto{\pgfqpoint{1.258047in}{2.452501in}}{\pgfqpoint{1.262437in}{2.441902in}}{\pgfqpoint{1.270251in}{2.434089in}}%
\pgfpathcurveto{\pgfqpoint{1.278065in}{2.426275in}}{\pgfqpoint{1.288664in}{2.421885in}}{\pgfqpoint{1.299714in}{2.421885in}}%
\pgfpathclose%
\pgfusepath{stroke,fill}%
\end{pgfscope}%
\begin{pgfscope}%
\pgfpathrectangle{\pgfqpoint{0.787074in}{0.548769in}}{\pgfqpoint{5.062926in}{3.102590in}}%
\pgfusepath{clip}%
\pgfsetbuttcap%
\pgfsetroundjoin%
\definecolor{currentfill}{rgb}{1.000000,0.498039,0.054902}%
\pgfsetfillcolor{currentfill}%
\pgfsetlinewidth{1.003750pt}%
\definecolor{currentstroke}{rgb}{1.000000,0.498039,0.054902}%
\pgfsetstrokecolor{currentstroke}%
\pgfsetdash{}{0pt}%
\pgfpathmoveto{\pgfqpoint{1.770122in}{2.463960in}}%
\pgfpathcurveto{\pgfqpoint{1.781172in}{2.463960in}}{\pgfqpoint{1.791771in}{2.468350in}}{\pgfqpoint{1.799584in}{2.476164in}}%
\pgfpathcurveto{\pgfqpoint{1.807398in}{2.483977in}}{\pgfqpoint{1.811788in}{2.494576in}}{\pgfqpoint{1.811788in}{2.505626in}}%
\pgfpathcurveto{\pgfqpoint{1.811788in}{2.516677in}}{\pgfqpoint{1.807398in}{2.527276in}}{\pgfqpoint{1.799584in}{2.535089in}}%
\pgfpathcurveto{\pgfqpoint{1.791771in}{2.542903in}}{\pgfqpoint{1.781172in}{2.547293in}}{\pgfqpoint{1.770122in}{2.547293in}}%
\pgfpathcurveto{\pgfqpoint{1.759072in}{2.547293in}}{\pgfqpoint{1.748472in}{2.542903in}}{\pgfqpoint{1.740659in}{2.535089in}}%
\pgfpathcurveto{\pgfqpoint{1.732845in}{2.527276in}}{\pgfqpoint{1.728455in}{2.516677in}}{\pgfqpoint{1.728455in}{2.505626in}}%
\pgfpathcurveto{\pgfqpoint{1.728455in}{2.494576in}}{\pgfqpoint{1.732845in}{2.483977in}}{\pgfqpoint{1.740659in}{2.476164in}}%
\pgfpathcurveto{\pgfqpoint{1.748472in}{2.468350in}}{\pgfqpoint{1.759072in}{2.463960in}}{\pgfqpoint{1.770122in}{2.463960in}}%
\pgfpathclose%
\pgfusepath{stroke,fill}%
\end{pgfscope}%
\begin{pgfscope}%
\pgfpathrectangle{\pgfqpoint{0.787074in}{0.548769in}}{\pgfqpoint{5.062926in}{3.102590in}}%
\pgfusepath{clip}%
\pgfsetbuttcap%
\pgfsetroundjoin%
\definecolor{currentfill}{rgb}{1.000000,0.498039,0.054902}%
\pgfsetfillcolor{currentfill}%
\pgfsetlinewidth{1.003750pt}%
\definecolor{currentstroke}{rgb}{1.000000,0.498039,0.054902}%
\pgfsetstrokecolor{currentstroke}%
\pgfsetdash{}{0pt}%
\pgfpathmoveto{\pgfqpoint{2.188408in}{2.012662in}}%
\pgfpathcurveto{\pgfqpoint{2.199458in}{2.012662in}}{\pgfqpoint{2.210057in}{2.017052in}}{\pgfqpoint{2.217870in}{2.024866in}}%
\pgfpathcurveto{\pgfqpoint{2.225684in}{2.032679in}}{\pgfqpoint{2.230074in}{2.043278in}}{\pgfqpoint{2.230074in}{2.054328in}}%
\pgfpathcurveto{\pgfqpoint{2.230074in}{2.065378in}}{\pgfqpoint{2.225684in}{2.075977in}}{\pgfqpoint{2.217870in}{2.083791in}}%
\pgfpathcurveto{\pgfqpoint{2.210057in}{2.091605in}}{\pgfqpoint{2.199458in}{2.095995in}}{\pgfqpoint{2.188408in}{2.095995in}}%
\pgfpathcurveto{\pgfqpoint{2.177358in}{2.095995in}}{\pgfqpoint{2.166758in}{2.091605in}}{\pgfqpoint{2.158945in}{2.083791in}}%
\pgfpathcurveto{\pgfqpoint{2.151131in}{2.075977in}}{\pgfqpoint{2.146741in}{2.065378in}}{\pgfqpoint{2.146741in}{2.054328in}}%
\pgfpathcurveto{\pgfqpoint{2.146741in}{2.043278in}}{\pgfqpoint{2.151131in}{2.032679in}}{\pgfqpoint{2.158945in}{2.024866in}}%
\pgfpathcurveto{\pgfqpoint{2.166758in}{2.017052in}}{\pgfqpoint{2.177358in}{2.012662in}}{\pgfqpoint{2.188408in}{2.012662in}}%
\pgfpathclose%
\pgfusepath{stroke,fill}%
\end{pgfscope}%
\begin{pgfscope}%
\pgfpathrectangle{\pgfqpoint{0.787074in}{0.548769in}}{\pgfqpoint{5.062926in}{3.102590in}}%
\pgfusepath{clip}%
\pgfsetbuttcap%
\pgfsetroundjoin%
\definecolor{currentfill}{rgb}{1.000000,0.498039,0.054902}%
\pgfsetfillcolor{currentfill}%
\pgfsetlinewidth{1.003750pt}%
\definecolor{currentstroke}{rgb}{1.000000,0.498039,0.054902}%
\pgfsetstrokecolor{currentstroke}%
\pgfsetdash{}{0pt}%
\pgfpathmoveto{\pgfqpoint{1.371946in}{1.794353in}}%
\pgfpathcurveto{\pgfqpoint{1.382996in}{1.794353in}}{\pgfqpoint{1.393595in}{1.798743in}}{\pgfqpoint{1.401408in}{1.806557in}}%
\pgfpathcurveto{\pgfqpoint{1.409222in}{1.814370in}}{\pgfqpoint{1.413612in}{1.824969in}}{\pgfqpoint{1.413612in}{1.836019in}}%
\pgfpathcurveto{\pgfqpoint{1.413612in}{1.847070in}}{\pgfqpoint{1.409222in}{1.857669in}}{\pgfqpoint{1.401408in}{1.865482in}}%
\pgfpathcurveto{\pgfqpoint{1.393595in}{1.873296in}}{\pgfqpoint{1.382996in}{1.877686in}}{\pgfqpoint{1.371946in}{1.877686in}}%
\pgfpathcurveto{\pgfqpoint{1.360895in}{1.877686in}}{\pgfqpoint{1.350296in}{1.873296in}}{\pgfqpoint{1.342483in}{1.865482in}}%
\pgfpathcurveto{\pgfqpoint{1.334669in}{1.857669in}}{\pgfqpoint{1.330279in}{1.847070in}}{\pgfqpoint{1.330279in}{1.836019in}}%
\pgfpathcurveto{\pgfqpoint{1.330279in}{1.824969in}}{\pgfqpoint{1.334669in}{1.814370in}}{\pgfqpoint{1.342483in}{1.806557in}}%
\pgfpathcurveto{\pgfqpoint{1.350296in}{1.798743in}}{\pgfqpoint{1.360895in}{1.794353in}}{\pgfqpoint{1.371946in}{1.794353in}}%
\pgfpathclose%
\pgfusepath{stroke,fill}%
\end{pgfscope}%
\begin{pgfscope}%
\pgfpathrectangle{\pgfqpoint{0.787074in}{0.548769in}}{\pgfqpoint{5.062926in}{3.102590in}}%
\pgfusepath{clip}%
\pgfsetbuttcap%
\pgfsetroundjoin%
\definecolor{currentfill}{rgb}{1.000000,0.498039,0.054902}%
\pgfsetfillcolor{currentfill}%
\pgfsetlinewidth{1.003750pt}%
\definecolor{currentstroke}{rgb}{1.000000,0.498039,0.054902}%
\pgfsetstrokecolor{currentstroke}%
\pgfsetdash{}{0pt}%
\pgfpathmoveto{\pgfqpoint{2.996589in}{2.149173in}}%
\pgfpathcurveto{\pgfqpoint{3.007639in}{2.149173in}}{\pgfqpoint{3.018238in}{2.153563in}}{\pgfqpoint{3.026052in}{2.161376in}}%
\pgfpathcurveto{\pgfqpoint{3.033866in}{2.169190in}}{\pgfqpoint{3.038256in}{2.179789in}}{\pgfqpoint{3.038256in}{2.190839in}}%
\pgfpathcurveto{\pgfqpoint{3.038256in}{2.201889in}}{\pgfqpoint{3.033866in}{2.212488in}}{\pgfqpoint{3.026052in}{2.220302in}}%
\pgfpathcurveto{\pgfqpoint{3.018238in}{2.228116in}}{\pgfqpoint{3.007639in}{2.232506in}}{\pgfqpoint{2.996589in}{2.232506in}}%
\pgfpathcurveto{\pgfqpoint{2.985539in}{2.232506in}}{\pgfqpoint{2.974940in}{2.228116in}}{\pgfqpoint{2.967126in}{2.220302in}}%
\pgfpathcurveto{\pgfqpoint{2.959313in}{2.212488in}}{\pgfqpoint{2.954923in}{2.201889in}}{\pgfqpoint{2.954923in}{2.190839in}}%
\pgfpathcurveto{\pgfqpoint{2.954923in}{2.179789in}}{\pgfqpoint{2.959313in}{2.169190in}}{\pgfqpoint{2.967126in}{2.161376in}}%
\pgfpathcurveto{\pgfqpoint{2.974940in}{2.153563in}}{\pgfqpoint{2.985539in}{2.149173in}}{\pgfqpoint{2.996589in}{2.149173in}}%
\pgfpathclose%
\pgfusepath{stroke,fill}%
\end{pgfscope}%
\begin{pgfscope}%
\pgfpathrectangle{\pgfqpoint{0.787074in}{0.548769in}}{\pgfqpoint{5.062926in}{3.102590in}}%
\pgfusepath{clip}%
\pgfsetbuttcap%
\pgfsetroundjoin%
\definecolor{currentfill}{rgb}{1.000000,0.498039,0.054902}%
\pgfsetfillcolor{currentfill}%
\pgfsetlinewidth{1.003750pt}%
\definecolor{currentstroke}{rgb}{1.000000,0.498039,0.054902}%
\pgfsetstrokecolor{currentstroke}%
\pgfsetdash{}{0pt}%
\pgfpathmoveto{\pgfqpoint{2.082417in}{2.551840in}}%
\pgfpathcurveto{\pgfqpoint{2.093467in}{2.551840in}}{\pgfqpoint{2.104066in}{2.556230in}}{\pgfqpoint{2.111879in}{2.564044in}}%
\pgfpathcurveto{\pgfqpoint{2.119693in}{2.571857in}}{\pgfqpoint{2.124083in}{2.582456in}}{\pgfqpoint{2.124083in}{2.593506in}}%
\pgfpathcurveto{\pgfqpoint{2.124083in}{2.604557in}}{\pgfqpoint{2.119693in}{2.615156in}}{\pgfqpoint{2.111879in}{2.622969in}}%
\pgfpathcurveto{\pgfqpoint{2.104066in}{2.630783in}}{\pgfqpoint{2.093467in}{2.635173in}}{\pgfqpoint{2.082417in}{2.635173in}}%
\pgfpathcurveto{\pgfqpoint{2.071366in}{2.635173in}}{\pgfqpoint{2.060767in}{2.630783in}}{\pgfqpoint{2.052954in}{2.622969in}}%
\pgfpathcurveto{\pgfqpoint{2.045140in}{2.615156in}}{\pgfqpoint{2.040750in}{2.604557in}}{\pgfqpoint{2.040750in}{2.593506in}}%
\pgfpathcurveto{\pgfqpoint{2.040750in}{2.582456in}}{\pgfqpoint{2.045140in}{2.571857in}}{\pgfqpoint{2.052954in}{2.564044in}}%
\pgfpathcurveto{\pgfqpoint{2.060767in}{2.556230in}}{\pgfqpoint{2.071366in}{2.551840in}}{\pgfqpoint{2.082417in}{2.551840in}}%
\pgfpathclose%
\pgfusepath{stroke,fill}%
\end{pgfscope}%
\begin{pgfscope}%
\pgfpathrectangle{\pgfqpoint{0.787074in}{0.548769in}}{\pgfqpoint{5.062926in}{3.102590in}}%
\pgfusepath{clip}%
\pgfsetbuttcap%
\pgfsetroundjoin%
\definecolor{currentfill}{rgb}{1.000000,0.498039,0.054902}%
\pgfsetfillcolor{currentfill}%
\pgfsetlinewidth{1.003750pt}%
\definecolor{currentstroke}{rgb}{1.000000,0.498039,0.054902}%
\pgfsetstrokecolor{currentstroke}%
\pgfsetdash{}{0pt}%
\pgfpathmoveto{\pgfqpoint{1.134612in}{2.187660in}}%
\pgfpathcurveto{\pgfqpoint{1.145662in}{2.187660in}}{\pgfqpoint{1.156261in}{2.192050in}}{\pgfqpoint{1.164075in}{2.199864in}}%
\pgfpathcurveto{\pgfqpoint{1.171889in}{2.207677in}}{\pgfqpoint{1.176279in}{2.218276in}}{\pgfqpoint{1.176279in}{2.229326in}}%
\pgfpathcurveto{\pgfqpoint{1.176279in}{2.240377in}}{\pgfqpoint{1.171889in}{2.250976in}}{\pgfqpoint{1.164075in}{2.258789in}}%
\pgfpathcurveto{\pgfqpoint{1.156261in}{2.266603in}}{\pgfqpoint{1.145662in}{2.270993in}}{\pgfqpoint{1.134612in}{2.270993in}}%
\pgfpathcurveto{\pgfqpoint{1.123562in}{2.270993in}}{\pgfqpoint{1.112963in}{2.266603in}}{\pgfqpoint{1.105150in}{2.258789in}}%
\pgfpathcurveto{\pgfqpoint{1.097336in}{2.250976in}}{\pgfqpoint{1.092946in}{2.240377in}}{\pgfqpoint{1.092946in}{2.229326in}}%
\pgfpathcurveto{\pgfqpoint{1.092946in}{2.218276in}}{\pgfqpoint{1.097336in}{2.207677in}}{\pgfqpoint{1.105150in}{2.199864in}}%
\pgfpathcurveto{\pgfqpoint{1.112963in}{2.192050in}}{\pgfqpoint{1.123562in}{2.187660in}}{\pgfqpoint{1.134612in}{2.187660in}}%
\pgfpathclose%
\pgfusepath{stroke,fill}%
\end{pgfscope}%
\begin{pgfscope}%
\pgfpathrectangle{\pgfqpoint{0.787074in}{0.548769in}}{\pgfqpoint{5.062926in}{3.102590in}}%
\pgfusepath{clip}%
\pgfsetbuttcap%
\pgfsetroundjoin%
\definecolor{currentfill}{rgb}{1.000000,0.498039,0.054902}%
\pgfsetfillcolor{currentfill}%
\pgfsetlinewidth{1.003750pt}%
\definecolor{currentstroke}{rgb}{1.000000,0.498039,0.054902}%
\pgfsetstrokecolor{currentstroke}%
\pgfsetdash{}{0pt}%
\pgfpathmoveto{\pgfqpoint{2.247081in}{2.846483in}}%
\pgfpathcurveto{\pgfqpoint{2.258131in}{2.846483in}}{\pgfqpoint{2.268730in}{2.850874in}}{\pgfqpoint{2.276544in}{2.858687in}}%
\pgfpathcurveto{\pgfqpoint{2.284358in}{2.866501in}}{\pgfqpoint{2.288748in}{2.877100in}}{\pgfqpoint{2.288748in}{2.888150in}}%
\pgfpathcurveto{\pgfqpoint{2.288748in}{2.899200in}}{\pgfqpoint{2.284358in}{2.909799in}}{\pgfqpoint{2.276544in}{2.917613in}}%
\pgfpathcurveto{\pgfqpoint{2.268730in}{2.925426in}}{\pgfqpoint{2.258131in}{2.929817in}}{\pgfqpoint{2.247081in}{2.929817in}}%
\pgfpathcurveto{\pgfqpoint{2.236031in}{2.929817in}}{\pgfqpoint{2.225432in}{2.925426in}}{\pgfqpoint{2.217618in}{2.917613in}}%
\pgfpathcurveto{\pgfqpoint{2.209805in}{2.909799in}}{\pgfqpoint{2.205415in}{2.899200in}}{\pgfqpoint{2.205415in}{2.888150in}}%
\pgfpathcurveto{\pgfqpoint{2.205415in}{2.877100in}}{\pgfqpoint{2.209805in}{2.866501in}}{\pgfqpoint{2.217618in}{2.858687in}}%
\pgfpathcurveto{\pgfqpoint{2.225432in}{2.850874in}}{\pgfqpoint{2.236031in}{2.846483in}}{\pgfqpoint{2.247081in}{2.846483in}}%
\pgfpathclose%
\pgfusepath{stroke,fill}%
\end{pgfscope}%
\begin{pgfscope}%
\pgfpathrectangle{\pgfqpoint{0.787074in}{0.548769in}}{\pgfqpoint{5.062926in}{3.102590in}}%
\pgfusepath{clip}%
\pgfsetbuttcap%
\pgfsetroundjoin%
\definecolor{currentfill}{rgb}{1.000000,0.498039,0.054902}%
\pgfsetfillcolor{currentfill}%
\pgfsetlinewidth{1.003750pt}%
\definecolor{currentstroke}{rgb}{1.000000,0.498039,0.054902}%
\pgfsetstrokecolor{currentstroke}%
\pgfsetdash{}{0pt}%
\pgfpathmoveto{\pgfqpoint{1.554870in}{2.316063in}}%
\pgfpathcurveto{\pgfqpoint{1.565921in}{2.316063in}}{\pgfqpoint{1.576520in}{2.320453in}}{\pgfqpoint{1.584333in}{2.328267in}}%
\pgfpathcurveto{\pgfqpoint{1.592147in}{2.336080in}}{\pgfqpoint{1.596537in}{2.346679in}}{\pgfqpoint{1.596537in}{2.357729in}}%
\pgfpathcurveto{\pgfqpoint{1.596537in}{2.368779in}}{\pgfqpoint{1.592147in}{2.379379in}}{\pgfqpoint{1.584333in}{2.387192in}}%
\pgfpathcurveto{\pgfqpoint{1.576520in}{2.395006in}}{\pgfqpoint{1.565921in}{2.399396in}}{\pgfqpoint{1.554870in}{2.399396in}}%
\pgfpathcurveto{\pgfqpoint{1.543820in}{2.399396in}}{\pgfqpoint{1.533221in}{2.395006in}}{\pgfqpoint{1.525408in}{2.387192in}}%
\pgfpathcurveto{\pgfqpoint{1.517594in}{2.379379in}}{\pgfqpoint{1.513204in}{2.368779in}}{\pgfqpoint{1.513204in}{2.357729in}}%
\pgfpathcurveto{\pgfqpoint{1.513204in}{2.346679in}}{\pgfqpoint{1.517594in}{2.336080in}}{\pgfqpoint{1.525408in}{2.328267in}}%
\pgfpathcurveto{\pgfqpoint{1.533221in}{2.320453in}}{\pgfqpoint{1.543820in}{2.316063in}}{\pgfqpoint{1.554870in}{2.316063in}}%
\pgfpathclose%
\pgfusepath{stroke,fill}%
\end{pgfscope}%
\begin{pgfscope}%
\pgfpathrectangle{\pgfqpoint{0.787074in}{0.548769in}}{\pgfqpoint{5.062926in}{3.102590in}}%
\pgfusepath{clip}%
\pgfsetbuttcap%
\pgfsetroundjoin%
\definecolor{currentfill}{rgb}{1.000000,0.498039,0.054902}%
\pgfsetfillcolor{currentfill}%
\pgfsetlinewidth{1.003750pt}%
\definecolor{currentstroke}{rgb}{1.000000,0.498039,0.054902}%
\pgfsetstrokecolor{currentstroke}%
\pgfsetdash{}{0pt}%
\pgfpathmoveto{\pgfqpoint{1.522350in}{1.585094in}}%
\pgfpathcurveto{\pgfqpoint{1.533401in}{1.585094in}}{\pgfqpoint{1.544000in}{1.589485in}}{\pgfqpoint{1.551813in}{1.597298in}}%
\pgfpathcurveto{\pgfqpoint{1.559627in}{1.605112in}}{\pgfqpoint{1.564017in}{1.615711in}}{\pgfqpoint{1.564017in}{1.626761in}}%
\pgfpathcurveto{\pgfqpoint{1.564017in}{1.637811in}}{\pgfqpoint{1.559627in}{1.648410in}}{\pgfqpoint{1.551813in}{1.656224in}}%
\pgfpathcurveto{\pgfqpoint{1.544000in}{1.664037in}}{\pgfqpoint{1.533401in}{1.668428in}}{\pgfqpoint{1.522350in}{1.668428in}}%
\pgfpathcurveto{\pgfqpoint{1.511300in}{1.668428in}}{\pgfqpoint{1.500701in}{1.664037in}}{\pgfqpoint{1.492888in}{1.656224in}}%
\pgfpathcurveto{\pgfqpoint{1.485074in}{1.648410in}}{\pgfqpoint{1.480684in}{1.637811in}}{\pgfqpoint{1.480684in}{1.626761in}}%
\pgfpathcurveto{\pgfqpoint{1.480684in}{1.615711in}}{\pgfqpoint{1.485074in}{1.605112in}}{\pgfqpoint{1.492888in}{1.597298in}}%
\pgfpathcurveto{\pgfqpoint{1.500701in}{1.589485in}}{\pgfqpoint{1.511300in}{1.585094in}}{\pgfqpoint{1.522350in}{1.585094in}}%
\pgfpathclose%
\pgfusepath{stroke,fill}%
\end{pgfscope}%
\begin{pgfscope}%
\pgfpathrectangle{\pgfqpoint{0.787074in}{0.548769in}}{\pgfqpoint{5.062926in}{3.102590in}}%
\pgfusepath{clip}%
\pgfsetbuttcap%
\pgfsetroundjoin%
\definecolor{currentfill}{rgb}{1.000000,0.498039,0.054902}%
\pgfsetfillcolor{currentfill}%
\pgfsetlinewidth{1.003750pt}%
\definecolor{currentstroke}{rgb}{1.000000,0.498039,0.054902}%
\pgfsetstrokecolor{currentstroke}%
\pgfsetdash{}{0pt}%
\pgfpathmoveto{\pgfqpoint{1.338071in}{1.803770in}}%
\pgfpathcurveto{\pgfqpoint{1.349121in}{1.803770in}}{\pgfqpoint{1.359720in}{1.808160in}}{\pgfqpoint{1.367533in}{1.815974in}}%
\pgfpathcurveto{\pgfqpoint{1.375347in}{1.823787in}}{\pgfqpoint{1.379737in}{1.834386in}}{\pgfqpoint{1.379737in}{1.845437in}}%
\pgfpathcurveto{\pgfqpoint{1.379737in}{1.856487in}}{\pgfqpoint{1.375347in}{1.867086in}}{\pgfqpoint{1.367533in}{1.874899in}}%
\pgfpathcurveto{\pgfqpoint{1.359720in}{1.882713in}}{\pgfqpoint{1.349121in}{1.887103in}}{\pgfqpoint{1.338071in}{1.887103in}}%
\pgfpathcurveto{\pgfqpoint{1.327020in}{1.887103in}}{\pgfqpoint{1.316421in}{1.882713in}}{\pgfqpoint{1.308608in}{1.874899in}}%
\pgfpathcurveto{\pgfqpoint{1.300794in}{1.867086in}}{\pgfqpoint{1.296404in}{1.856487in}}{\pgfqpoint{1.296404in}{1.845437in}}%
\pgfpathcurveto{\pgfqpoint{1.296404in}{1.834386in}}{\pgfqpoint{1.300794in}{1.823787in}}{\pgfqpoint{1.308608in}{1.815974in}}%
\pgfpathcurveto{\pgfqpoint{1.316421in}{1.808160in}}{\pgfqpoint{1.327020in}{1.803770in}}{\pgfqpoint{1.338071in}{1.803770in}}%
\pgfpathclose%
\pgfusepath{stroke,fill}%
\end{pgfscope}%
\begin{pgfscope}%
\pgfpathrectangle{\pgfqpoint{0.787074in}{0.548769in}}{\pgfqpoint{5.062926in}{3.102590in}}%
\pgfusepath{clip}%
\pgfsetbuttcap%
\pgfsetroundjoin%
\definecolor{currentfill}{rgb}{1.000000,0.498039,0.054902}%
\pgfsetfillcolor{currentfill}%
\pgfsetlinewidth{1.003750pt}%
\definecolor{currentstroke}{rgb}{1.000000,0.498039,0.054902}%
\pgfsetstrokecolor{currentstroke}%
\pgfsetdash{}{0pt}%
\pgfpathmoveto{\pgfqpoint{1.574883in}{2.162419in}}%
\pgfpathcurveto{\pgfqpoint{1.585933in}{2.162419in}}{\pgfqpoint{1.596532in}{2.166809in}}{\pgfqpoint{1.604345in}{2.174623in}}%
\pgfpathcurveto{\pgfqpoint{1.612159in}{2.182437in}}{\pgfqpoint{1.616549in}{2.193036in}}{\pgfqpoint{1.616549in}{2.204086in}}%
\pgfpathcurveto{\pgfqpoint{1.616549in}{2.215136in}}{\pgfqpoint{1.612159in}{2.225735in}}{\pgfqpoint{1.604345in}{2.233549in}}%
\pgfpathcurveto{\pgfqpoint{1.596532in}{2.241362in}}{\pgfqpoint{1.585933in}{2.245753in}}{\pgfqpoint{1.574883in}{2.245753in}}%
\pgfpathcurveto{\pgfqpoint{1.563833in}{2.245753in}}{\pgfqpoint{1.553234in}{2.241362in}}{\pgfqpoint{1.545420in}{2.233549in}}%
\pgfpathcurveto{\pgfqpoint{1.537606in}{2.225735in}}{\pgfqpoint{1.533216in}{2.215136in}}{\pgfqpoint{1.533216in}{2.204086in}}%
\pgfpathcurveto{\pgfqpoint{1.533216in}{2.193036in}}{\pgfqpoint{1.537606in}{2.182437in}}{\pgfqpoint{1.545420in}{2.174623in}}%
\pgfpathcurveto{\pgfqpoint{1.553234in}{2.166809in}}{\pgfqpoint{1.563833in}{2.162419in}}{\pgfqpoint{1.574883in}{2.162419in}}%
\pgfpathclose%
\pgfusepath{stroke,fill}%
\end{pgfscope}%
\begin{pgfscope}%
\pgfpathrectangle{\pgfqpoint{0.787074in}{0.548769in}}{\pgfqpoint{5.062926in}{3.102590in}}%
\pgfusepath{clip}%
\pgfsetbuttcap%
\pgfsetroundjoin%
\definecolor{currentfill}{rgb}{1.000000,0.498039,0.054902}%
\pgfsetfillcolor{currentfill}%
\pgfsetlinewidth{1.003750pt}%
\definecolor{currentstroke}{rgb}{1.000000,0.498039,0.054902}%
\pgfsetstrokecolor{currentstroke}%
\pgfsetdash{}{0pt}%
\pgfpathmoveto{\pgfqpoint{3.490583in}{3.057735in}}%
\pgfpathcurveto{\pgfqpoint{3.501633in}{3.057735in}}{\pgfqpoint{3.512232in}{3.062125in}}{\pgfqpoint{3.520046in}{3.069939in}}%
\pgfpathcurveto{\pgfqpoint{3.527859in}{3.077752in}}{\pgfqpoint{3.532250in}{3.088351in}}{\pgfqpoint{3.532250in}{3.099401in}}%
\pgfpathcurveto{\pgfqpoint{3.532250in}{3.110451in}}{\pgfqpoint{3.527859in}{3.121050in}}{\pgfqpoint{3.520046in}{3.128864in}}%
\pgfpathcurveto{\pgfqpoint{3.512232in}{3.136678in}}{\pgfqpoint{3.501633in}{3.141068in}}{\pgfqpoint{3.490583in}{3.141068in}}%
\pgfpathcurveto{\pgfqpoint{3.479533in}{3.141068in}}{\pgfqpoint{3.468934in}{3.136678in}}{\pgfqpoint{3.461120in}{3.128864in}}%
\pgfpathcurveto{\pgfqpoint{3.453307in}{3.121050in}}{\pgfqpoint{3.448916in}{3.110451in}}{\pgfqpoint{3.448916in}{3.099401in}}%
\pgfpathcurveto{\pgfqpoint{3.448916in}{3.088351in}}{\pgfqpoint{3.453307in}{3.077752in}}{\pgfqpoint{3.461120in}{3.069939in}}%
\pgfpathcurveto{\pgfqpoint{3.468934in}{3.062125in}}{\pgfqpoint{3.479533in}{3.057735in}}{\pgfqpoint{3.490583in}{3.057735in}}%
\pgfpathclose%
\pgfusepath{stroke,fill}%
\end{pgfscope}%
\begin{pgfscope}%
\pgfpathrectangle{\pgfqpoint{0.787074in}{0.548769in}}{\pgfqpoint{5.062926in}{3.102590in}}%
\pgfusepath{clip}%
\pgfsetbuttcap%
\pgfsetroundjoin%
\definecolor{currentfill}{rgb}{1.000000,0.498039,0.054902}%
\pgfsetfillcolor{currentfill}%
\pgfsetlinewidth{1.003750pt}%
\definecolor{currentstroke}{rgb}{1.000000,0.498039,0.054902}%
\pgfsetstrokecolor{currentstroke}%
\pgfsetdash{}{0pt}%
\pgfpathmoveto{\pgfqpoint{1.712467in}{2.195132in}}%
\pgfpathcurveto{\pgfqpoint{1.723517in}{2.195132in}}{\pgfqpoint{1.734116in}{2.199523in}}{\pgfqpoint{1.741930in}{2.207336in}}%
\pgfpathcurveto{\pgfqpoint{1.749744in}{2.215150in}}{\pgfqpoint{1.754134in}{2.225749in}}{\pgfqpoint{1.754134in}{2.236799in}}%
\pgfpathcurveto{\pgfqpoint{1.754134in}{2.247849in}}{\pgfqpoint{1.749744in}{2.258448in}}{\pgfqpoint{1.741930in}{2.266262in}}%
\pgfpathcurveto{\pgfqpoint{1.734116in}{2.274076in}}{\pgfqpoint{1.723517in}{2.278466in}}{\pgfqpoint{1.712467in}{2.278466in}}%
\pgfpathcurveto{\pgfqpoint{1.701417in}{2.278466in}}{\pgfqpoint{1.690818in}{2.274076in}}{\pgfqpoint{1.683004in}{2.266262in}}%
\pgfpathcurveto{\pgfqpoint{1.675191in}{2.258448in}}{\pgfqpoint{1.670801in}{2.247849in}}{\pgfqpoint{1.670801in}{2.236799in}}%
\pgfpathcurveto{\pgfqpoint{1.670801in}{2.225749in}}{\pgfqpoint{1.675191in}{2.215150in}}{\pgfqpoint{1.683004in}{2.207336in}}%
\pgfpathcurveto{\pgfqpoint{1.690818in}{2.199523in}}{\pgfqpoint{1.701417in}{2.195132in}}{\pgfqpoint{1.712467in}{2.195132in}}%
\pgfpathclose%
\pgfusepath{stroke,fill}%
\end{pgfscope}%
\begin{pgfscope}%
\pgfpathrectangle{\pgfqpoint{0.787074in}{0.548769in}}{\pgfqpoint{5.062926in}{3.102590in}}%
\pgfusepath{clip}%
\pgfsetbuttcap%
\pgfsetroundjoin%
\definecolor{currentfill}{rgb}{1.000000,0.498039,0.054902}%
\pgfsetfillcolor{currentfill}%
\pgfsetlinewidth{1.003750pt}%
\definecolor{currentstroke}{rgb}{1.000000,0.498039,0.054902}%
\pgfsetstrokecolor{currentstroke}%
\pgfsetdash{}{0pt}%
\pgfpathmoveto{\pgfqpoint{2.042670in}{2.538741in}}%
\pgfpathcurveto{\pgfqpoint{2.053720in}{2.538741in}}{\pgfqpoint{2.064319in}{2.543132in}}{\pgfqpoint{2.072133in}{2.550945in}}%
\pgfpathcurveto{\pgfqpoint{2.079946in}{2.558759in}}{\pgfqpoint{2.084337in}{2.569358in}}{\pgfqpoint{2.084337in}{2.580408in}}%
\pgfpathcurveto{\pgfqpoint{2.084337in}{2.591458in}}{\pgfqpoint{2.079946in}{2.602057in}}{\pgfqpoint{2.072133in}{2.609871in}}%
\pgfpathcurveto{\pgfqpoint{2.064319in}{2.617685in}}{\pgfqpoint{2.053720in}{2.622075in}}{\pgfqpoint{2.042670in}{2.622075in}}%
\pgfpathcurveto{\pgfqpoint{2.031620in}{2.622075in}}{\pgfqpoint{2.021021in}{2.617685in}}{\pgfqpoint{2.013207in}{2.609871in}}%
\pgfpathcurveto{\pgfqpoint{2.005394in}{2.602057in}}{\pgfqpoint{2.001003in}{2.591458in}}{\pgfqpoint{2.001003in}{2.580408in}}%
\pgfpathcurveto{\pgfqpoint{2.001003in}{2.569358in}}{\pgfqpoint{2.005394in}{2.558759in}}{\pgfqpoint{2.013207in}{2.550945in}}%
\pgfpathcurveto{\pgfqpoint{2.021021in}{2.543132in}}{\pgfqpoint{2.031620in}{2.538741in}}{\pgfqpoint{2.042670in}{2.538741in}}%
\pgfpathclose%
\pgfusepath{stroke,fill}%
\end{pgfscope}%
\begin{pgfscope}%
\pgfpathrectangle{\pgfqpoint{0.787074in}{0.548769in}}{\pgfqpoint{5.062926in}{3.102590in}}%
\pgfusepath{clip}%
\pgfsetbuttcap%
\pgfsetroundjoin%
\definecolor{currentfill}{rgb}{1.000000,0.498039,0.054902}%
\pgfsetfillcolor{currentfill}%
\pgfsetlinewidth{1.003750pt}%
\definecolor{currentstroke}{rgb}{1.000000,0.498039,0.054902}%
\pgfsetstrokecolor{currentstroke}%
\pgfsetdash{}{0pt}%
\pgfpathmoveto{\pgfqpoint{1.605457in}{2.246302in}}%
\pgfpathcurveto{\pgfqpoint{1.616507in}{2.246302in}}{\pgfqpoint{1.627106in}{2.250692in}}{\pgfqpoint{1.634920in}{2.258505in}}%
\pgfpathcurveto{\pgfqpoint{1.642733in}{2.266319in}}{\pgfqpoint{1.647124in}{2.276918in}}{\pgfqpoint{1.647124in}{2.287968in}}%
\pgfpathcurveto{\pgfqpoint{1.647124in}{2.299018in}}{\pgfqpoint{1.642733in}{2.309617in}}{\pgfqpoint{1.634920in}{2.317431in}}%
\pgfpathcurveto{\pgfqpoint{1.627106in}{2.325245in}}{\pgfqpoint{1.616507in}{2.329635in}}{\pgfqpoint{1.605457in}{2.329635in}}%
\pgfpathcurveto{\pgfqpoint{1.594407in}{2.329635in}}{\pgfqpoint{1.583808in}{2.325245in}}{\pgfqpoint{1.575994in}{2.317431in}}%
\pgfpathcurveto{\pgfqpoint{1.568181in}{2.309617in}}{\pgfqpoint{1.563790in}{2.299018in}}{\pgfqpoint{1.563790in}{2.287968in}}%
\pgfpathcurveto{\pgfqpoint{1.563790in}{2.276918in}}{\pgfqpoint{1.568181in}{2.266319in}}{\pgfqpoint{1.575994in}{2.258505in}}%
\pgfpathcurveto{\pgfqpoint{1.583808in}{2.250692in}}{\pgfqpoint{1.594407in}{2.246302in}}{\pgfqpoint{1.605457in}{2.246302in}}%
\pgfpathclose%
\pgfusepath{stroke,fill}%
\end{pgfscope}%
\begin{pgfscope}%
\pgfpathrectangle{\pgfqpoint{0.787074in}{0.548769in}}{\pgfqpoint{5.062926in}{3.102590in}}%
\pgfusepath{clip}%
\pgfsetbuttcap%
\pgfsetroundjoin%
\definecolor{currentfill}{rgb}{1.000000,0.498039,0.054902}%
\pgfsetfillcolor{currentfill}%
\pgfsetlinewidth{1.003750pt}%
\definecolor{currentstroke}{rgb}{1.000000,0.498039,0.054902}%
\pgfsetstrokecolor{currentstroke}%
\pgfsetdash{}{0pt}%
\pgfpathmoveto{\pgfqpoint{1.487588in}{2.224849in}}%
\pgfpathcurveto{\pgfqpoint{1.498638in}{2.224849in}}{\pgfqpoint{1.509237in}{2.229239in}}{\pgfqpoint{1.517050in}{2.237053in}}%
\pgfpathcurveto{\pgfqpoint{1.524864in}{2.244866in}}{\pgfqpoint{1.529254in}{2.255465in}}{\pgfqpoint{1.529254in}{2.266515in}}%
\pgfpathcurveto{\pgfqpoint{1.529254in}{2.277565in}}{\pgfqpoint{1.524864in}{2.288165in}}{\pgfqpoint{1.517050in}{2.295978in}}%
\pgfpathcurveto{\pgfqpoint{1.509237in}{2.303792in}}{\pgfqpoint{1.498638in}{2.308182in}}{\pgfqpoint{1.487588in}{2.308182in}}%
\pgfpathcurveto{\pgfqpoint{1.476538in}{2.308182in}}{\pgfqpoint{1.465939in}{2.303792in}}{\pgfqpoint{1.458125in}{2.295978in}}%
\pgfpathcurveto{\pgfqpoint{1.450311in}{2.288165in}}{\pgfqpoint{1.445921in}{2.277565in}}{\pgfqpoint{1.445921in}{2.266515in}}%
\pgfpathcurveto{\pgfqpoint{1.445921in}{2.255465in}}{\pgfqpoint{1.450311in}{2.244866in}}{\pgfqpoint{1.458125in}{2.237053in}}%
\pgfpathcurveto{\pgfqpoint{1.465939in}{2.229239in}}{\pgfqpoint{1.476538in}{2.224849in}}{\pgfqpoint{1.487588in}{2.224849in}}%
\pgfpathclose%
\pgfusepath{stroke,fill}%
\end{pgfscope}%
\begin{pgfscope}%
\pgfpathrectangle{\pgfqpoint{0.787074in}{0.548769in}}{\pgfqpoint{5.062926in}{3.102590in}}%
\pgfusepath{clip}%
\pgfsetbuttcap%
\pgfsetroundjoin%
\definecolor{currentfill}{rgb}{1.000000,0.498039,0.054902}%
\pgfsetfillcolor{currentfill}%
\pgfsetlinewidth{1.003750pt}%
\definecolor{currentstroke}{rgb}{1.000000,0.498039,0.054902}%
\pgfsetstrokecolor{currentstroke}%
\pgfsetdash{}{0pt}%
\pgfpathmoveto{\pgfqpoint{1.719019in}{2.370615in}}%
\pgfpathcurveto{\pgfqpoint{1.730069in}{2.370615in}}{\pgfqpoint{1.740668in}{2.375005in}}{\pgfqpoint{1.748482in}{2.382818in}}%
\pgfpathcurveto{\pgfqpoint{1.756295in}{2.390632in}}{\pgfqpoint{1.760685in}{2.401231in}}{\pgfqpoint{1.760685in}{2.412281in}}%
\pgfpathcurveto{\pgfqpoint{1.760685in}{2.423331in}}{\pgfqpoint{1.756295in}{2.433930in}}{\pgfqpoint{1.748482in}{2.441744in}}%
\pgfpathcurveto{\pgfqpoint{1.740668in}{2.449558in}}{\pgfqpoint{1.730069in}{2.453948in}}{\pgfqpoint{1.719019in}{2.453948in}}%
\pgfpathcurveto{\pgfqpoint{1.707969in}{2.453948in}}{\pgfqpoint{1.697370in}{2.449558in}}{\pgfqpoint{1.689556in}{2.441744in}}%
\pgfpathcurveto{\pgfqpoint{1.681742in}{2.433930in}}{\pgfqpoint{1.677352in}{2.423331in}}{\pgfqpoint{1.677352in}{2.412281in}}%
\pgfpathcurveto{\pgfqpoint{1.677352in}{2.401231in}}{\pgfqpoint{1.681742in}{2.390632in}}{\pgfqpoint{1.689556in}{2.382818in}}%
\pgfpathcurveto{\pgfqpoint{1.697370in}{2.375005in}}{\pgfqpoint{1.707969in}{2.370615in}}{\pgfqpoint{1.719019in}{2.370615in}}%
\pgfpathclose%
\pgfusepath{stroke,fill}%
\end{pgfscope}%
\begin{pgfscope}%
\pgfpathrectangle{\pgfqpoint{0.787074in}{0.548769in}}{\pgfqpoint{5.062926in}{3.102590in}}%
\pgfusepath{clip}%
\pgfsetbuttcap%
\pgfsetroundjoin%
\definecolor{currentfill}{rgb}{1.000000,0.498039,0.054902}%
\pgfsetfillcolor{currentfill}%
\pgfsetlinewidth{1.003750pt}%
\definecolor{currentstroke}{rgb}{1.000000,0.498039,0.054902}%
\pgfsetstrokecolor{currentstroke}%
\pgfsetdash{}{0pt}%
\pgfpathmoveto{\pgfqpoint{1.729665in}{1.974526in}}%
\pgfpathcurveto{\pgfqpoint{1.740715in}{1.974526in}}{\pgfqpoint{1.751314in}{1.978916in}}{\pgfqpoint{1.759128in}{1.986730in}}%
\pgfpathcurveto{\pgfqpoint{1.766942in}{1.994543in}}{\pgfqpoint{1.771332in}{2.005142in}}{\pgfqpoint{1.771332in}{2.016192in}}%
\pgfpathcurveto{\pgfqpoint{1.771332in}{2.027242in}}{\pgfqpoint{1.766942in}{2.037842in}}{\pgfqpoint{1.759128in}{2.045655in}}%
\pgfpathcurveto{\pgfqpoint{1.751314in}{2.053469in}}{\pgfqpoint{1.740715in}{2.057859in}}{\pgfqpoint{1.729665in}{2.057859in}}%
\pgfpathcurveto{\pgfqpoint{1.718615in}{2.057859in}}{\pgfqpoint{1.708016in}{2.053469in}}{\pgfqpoint{1.700202in}{2.045655in}}%
\pgfpathcurveto{\pgfqpoint{1.692389in}{2.037842in}}{\pgfqpoint{1.687999in}{2.027242in}}{\pgfqpoint{1.687999in}{2.016192in}}%
\pgfpathcurveto{\pgfqpoint{1.687999in}{2.005142in}}{\pgfqpoint{1.692389in}{1.994543in}}{\pgfqpoint{1.700202in}{1.986730in}}%
\pgfpathcurveto{\pgfqpoint{1.708016in}{1.978916in}}{\pgfqpoint{1.718615in}{1.974526in}}{\pgfqpoint{1.729665in}{1.974526in}}%
\pgfpathclose%
\pgfusepath{stroke,fill}%
\end{pgfscope}%
\begin{pgfscope}%
\pgfpathrectangle{\pgfqpoint{0.787074in}{0.548769in}}{\pgfqpoint{5.062926in}{3.102590in}}%
\pgfusepath{clip}%
\pgfsetbuttcap%
\pgfsetroundjoin%
\definecolor{currentfill}{rgb}{1.000000,0.498039,0.054902}%
\pgfsetfillcolor{currentfill}%
\pgfsetlinewidth{1.003750pt}%
\definecolor{currentstroke}{rgb}{1.000000,0.498039,0.054902}%
\pgfsetstrokecolor{currentstroke}%
\pgfsetdash{}{0pt}%
\pgfpathmoveto{\pgfqpoint{2.718363in}{2.689957in}}%
\pgfpathcurveto{\pgfqpoint{2.729413in}{2.689957in}}{\pgfqpoint{2.740012in}{2.694347in}}{\pgfqpoint{2.747826in}{2.702161in}}%
\pgfpathcurveto{\pgfqpoint{2.755639in}{2.709974in}}{\pgfqpoint{2.760029in}{2.720573in}}{\pgfqpoint{2.760029in}{2.731623in}}%
\pgfpathcurveto{\pgfqpoint{2.760029in}{2.742674in}}{\pgfqpoint{2.755639in}{2.753273in}}{\pgfqpoint{2.747826in}{2.761086in}}%
\pgfpathcurveto{\pgfqpoint{2.740012in}{2.768900in}}{\pgfqpoint{2.729413in}{2.773290in}}{\pgfqpoint{2.718363in}{2.773290in}}%
\pgfpathcurveto{\pgfqpoint{2.707313in}{2.773290in}}{\pgfqpoint{2.696714in}{2.768900in}}{\pgfqpoint{2.688900in}{2.761086in}}%
\pgfpathcurveto{\pgfqpoint{2.681086in}{2.753273in}}{\pgfqpoint{2.676696in}{2.742674in}}{\pgfqpoint{2.676696in}{2.731623in}}%
\pgfpathcurveto{\pgfqpoint{2.676696in}{2.720573in}}{\pgfqpoint{2.681086in}{2.709974in}}{\pgfqpoint{2.688900in}{2.702161in}}%
\pgfpathcurveto{\pgfqpoint{2.696714in}{2.694347in}}{\pgfqpoint{2.707313in}{2.689957in}}{\pgfqpoint{2.718363in}{2.689957in}}%
\pgfpathclose%
\pgfusepath{stroke,fill}%
\end{pgfscope}%
\begin{pgfscope}%
\pgfpathrectangle{\pgfqpoint{0.787074in}{0.548769in}}{\pgfqpoint{5.062926in}{3.102590in}}%
\pgfusepath{clip}%
\pgfsetbuttcap%
\pgfsetroundjoin%
\definecolor{currentfill}{rgb}{1.000000,0.498039,0.054902}%
\pgfsetfillcolor{currentfill}%
\pgfsetlinewidth{1.003750pt}%
\definecolor{currentstroke}{rgb}{1.000000,0.498039,0.054902}%
\pgfsetstrokecolor{currentstroke}%
\pgfsetdash{}{0pt}%
\pgfpathmoveto{\pgfqpoint{2.051395in}{2.569739in}}%
\pgfpathcurveto{\pgfqpoint{2.062445in}{2.569739in}}{\pgfqpoint{2.073044in}{2.574129in}}{\pgfqpoint{2.080858in}{2.581943in}}%
\pgfpathcurveto{\pgfqpoint{2.088671in}{2.589756in}}{\pgfqpoint{2.093062in}{2.600355in}}{\pgfqpoint{2.093062in}{2.611406in}}%
\pgfpathcurveto{\pgfqpoint{2.093062in}{2.622456in}}{\pgfqpoint{2.088671in}{2.633055in}}{\pgfqpoint{2.080858in}{2.640868in}}%
\pgfpathcurveto{\pgfqpoint{2.073044in}{2.648682in}}{\pgfqpoint{2.062445in}{2.653072in}}{\pgfqpoint{2.051395in}{2.653072in}}%
\pgfpathcurveto{\pgfqpoint{2.040345in}{2.653072in}}{\pgfqpoint{2.029746in}{2.648682in}}{\pgfqpoint{2.021932in}{2.640868in}}%
\pgfpathcurveto{\pgfqpoint{2.014118in}{2.633055in}}{\pgfqpoint{2.009728in}{2.622456in}}{\pgfqpoint{2.009728in}{2.611406in}}%
\pgfpathcurveto{\pgfqpoint{2.009728in}{2.600355in}}{\pgfqpoint{2.014118in}{2.589756in}}{\pgfqpoint{2.021932in}{2.581943in}}%
\pgfpathcurveto{\pgfqpoint{2.029746in}{2.574129in}}{\pgfqpoint{2.040345in}{2.569739in}}{\pgfqpoint{2.051395in}{2.569739in}}%
\pgfpathclose%
\pgfusepath{stroke,fill}%
\end{pgfscope}%
\begin{pgfscope}%
\pgfpathrectangle{\pgfqpoint{0.787074in}{0.548769in}}{\pgfqpoint{5.062926in}{3.102590in}}%
\pgfusepath{clip}%
\pgfsetbuttcap%
\pgfsetroundjoin%
\definecolor{currentfill}{rgb}{1.000000,0.498039,0.054902}%
\pgfsetfillcolor{currentfill}%
\pgfsetlinewidth{1.003750pt}%
\definecolor{currentstroke}{rgb}{1.000000,0.498039,0.054902}%
\pgfsetstrokecolor{currentstroke}%
\pgfsetdash{}{0pt}%
\pgfpathmoveto{\pgfqpoint{1.541862in}{2.197035in}}%
\pgfpathcurveto{\pgfqpoint{1.552913in}{2.197035in}}{\pgfqpoint{1.563512in}{2.201425in}}{\pgfqpoint{1.571325in}{2.209239in}}%
\pgfpathcurveto{\pgfqpoint{1.579139in}{2.217052in}}{\pgfqpoint{1.583529in}{2.227651in}}{\pgfqpoint{1.583529in}{2.238701in}}%
\pgfpathcurveto{\pgfqpoint{1.583529in}{2.249751in}}{\pgfqpoint{1.579139in}{2.260350in}}{\pgfqpoint{1.571325in}{2.268164in}}%
\pgfpathcurveto{\pgfqpoint{1.563512in}{2.275978in}}{\pgfqpoint{1.552913in}{2.280368in}}{\pgfqpoint{1.541862in}{2.280368in}}%
\pgfpathcurveto{\pgfqpoint{1.530812in}{2.280368in}}{\pgfqpoint{1.520213in}{2.275978in}}{\pgfqpoint{1.512400in}{2.268164in}}%
\pgfpathcurveto{\pgfqpoint{1.504586in}{2.260350in}}{\pgfqpoint{1.500196in}{2.249751in}}{\pgfqpoint{1.500196in}{2.238701in}}%
\pgfpathcurveto{\pgfqpoint{1.500196in}{2.227651in}}{\pgfqpoint{1.504586in}{2.217052in}}{\pgfqpoint{1.512400in}{2.209239in}}%
\pgfpathcurveto{\pgfqpoint{1.520213in}{2.201425in}}{\pgfqpoint{1.530812in}{2.197035in}}{\pgfqpoint{1.541862in}{2.197035in}}%
\pgfpathclose%
\pgfusepath{stroke,fill}%
\end{pgfscope}%
\begin{pgfscope}%
\pgfpathrectangle{\pgfqpoint{0.787074in}{0.548769in}}{\pgfqpoint{5.062926in}{3.102590in}}%
\pgfusepath{clip}%
\pgfsetbuttcap%
\pgfsetroundjoin%
\definecolor{currentfill}{rgb}{0.121569,0.466667,0.705882}%
\pgfsetfillcolor{currentfill}%
\pgfsetlinewidth{1.003750pt}%
\definecolor{currentstroke}{rgb}{0.121569,0.466667,0.705882}%
\pgfsetstrokecolor{currentstroke}%
\pgfsetdash{}{0pt}%
\pgfpathmoveto{\pgfqpoint{2.758109in}{0.648134in}}%
\pgfpathcurveto{\pgfqpoint{2.769160in}{0.648134in}}{\pgfqpoint{2.779759in}{0.652524in}}{\pgfqpoint{2.787572in}{0.660338in}}%
\pgfpathcurveto{\pgfqpoint{2.795386in}{0.668151in}}{\pgfqpoint{2.799776in}{0.678750in}}{\pgfqpoint{2.799776in}{0.689801in}}%
\pgfpathcurveto{\pgfqpoint{2.799776in}{0.700851in}}{\pgfqpoint{2.795386in}{0.711450in}}{\pgfqpoint{2.787572in}{0.719263in}}%
\pgfpathcurveto{\pgfqpoint{2.779759in}{0.727077in}}{\pgfqpoint{2.769160in}{0.731467in}}{\pgfqpoint{2.758109in}{0.731467in}}%
\pgfpathcurveto{\pgfqpoint{2.747059in}{0.731467in}}{\pgfqpoint{2.736460in}{0.727077in}}{\pgfqpoint{2.728647in}{0.719263in}}%
\pgfpathcurveto{\pgfqpoint{2.720833in}{0.711450in}}{\pgfqpoint{2.716443in}{0.700851in}}{\pgfqpoint{2.716443in}{0.689801in}}%
\pgfpathcurveto{\pgfqpoint{2.716443in}{0.678750in}}{\pgfqpoint{2.720833in}{0.668151in}}{\pgfqpoint{2.728647in}{0.660338in}}%
\pgfpathcurveto{\pgfqpoint{2.736460in}{0.652524in}}{\pgfqpoint{2.747059in}{0.648134in}}{\pgfqpoint{2.758109in}{0.648134in}}%
\pgfpathclose%
\pgfusepath{stroke,fill}%
\end{pgfscope}%
\begin{pgfscope}%
\pgfpathrectangle{\pgfqpoint{0.787074in}{0.548769in}}{\pgfqpoint{5.062926in}{3.102590in}}%
\pgfusepath{clip}%
\pgfsetbuttcap%
\pgfsetroundjoin%
\definecolor{currentfill}{rgb}{1.000000,0.498039,0.054902}%
\pgfsetfillcolor{currentfill}%
\pgfsetlinewidth{1.003750pt}%
\definecolor{currentstroke}{rgb}{1.000000,0.498039,0.054902}%
\pgfsetstrokecolor{currentstroke}%
\pgfsetdash{}{0pt}%
\pgfpathmoveto{\pgfqpoint{1.767501in}{1.410798in}}%
\pgfpathcurveto{\pgfqpoint{1.778551in}{1.410798in}}{\pgfqpoint{1.789150in}{1.415189in}}{\pgfqpoint{1.796964in}{1.423002in}}%
\pgfpathcurveto{\pgfqpoint{1.804777in}{1.430816in}}{\pgfqpoint{1.809168in}{1.441415in}}{\pgfqpoint{1.809168in}{1.452465in}}%
\pgfpathcurveto{\pgfqpoint{1.809168in}{1.463515in}}{\pgfqpoint{1.804777in}{1.474114in}}{\pgfqpoint{1.796964in}{1.481928in}}%
\pgfpathcurveto{\pgfqpoint{1.789150in}{1.489741in}}{\pgfqpoint{1.778551in}{1.494132in}}{\pgfqpoint{1.767501in}{1.494132in}}%
\pgfpathcurveto{\pgfqpoint{1.756451in}{1.494132in}}{\pgfqpoint{1.745852in}{1.489741in}}{\pgfqpoint{1.738038in}{1.481928in}}%
\pgfpathcurveto{\pgfqpoint{1.730225in}{1.474114in}}{\pgfqpoint{1.725834in}{1.463515in}}{\pgfqpoint{1.725834in}{1.452465in}}%
\pgfpathcurveto{\pgfqpoint{1.725834in}{1.441415in}}{\pgfqpoint{1.730225in}{1.430816in}}{\pgfqpoint{1.738038in}{1.423002in}}%
\pgfpathcurveto{\pgfqpoint{1.745852in}{1.415189in}}{\pgfqpoint{1.756451in}{1.410798in}}{\pgfqpoint{1.767501in}{1.410798in}}%
\pgfpathclose%
\pgfusepath{stroke,fill}%
\end{pgfscope}%
\begin{pgfscope}%
\pgfpathrectangle{\pgfqpoint{0.787074in}{0.548769in}}{\pgfqpoint{5.062926in}{3.102590in}}%
\pgfusepath{clip}%
\pgfsetbuttcap%
\pgfsetroundjoin%
\definecolor{currentfill}{rgb}{1.000000,0.498039,0.054902}%
\pgfsetfillcolor{currentfill}%
\pgfsetlinewidth{1.003750pt}%
\definecolor{currentstroke}{rgb}{1.000000,0.498039,0.054902}%
\pgfsetstrokecolor{currentstroke}%
\pgfsetdash{}{0pt}%
\pgfpathmoveto{\pgfqpoint{1.880070in}{2.681418in}}%
\pgfpathcurveto{\pgfqpoint{1.891120in}{2.681418in}}{\pgfqpoint{1.901719in}{2.685808in}}{\pgfqpoint{1.909533in}{2.693622in}}%
\pgfpathcurveto{\pgfqpoint{1.917347in}{2.701435in}}{\pgfqpoint{1.921737in}{2.712034in}}{\pgfqpoint{1.921737in}{2.723084in}}%
\pgfpathcurveto{\pgfqpoint{1.921737in}{2.734135in}}{\pgfqpoint{1.917347in}{2.744734in}}{\pgfqpoint{1.909533in}{2.752547in}}%
\pgfpathcurveto{\pgfqpoint{1.901719in}{2.760361in}}{\pgfqpoint{1.891120in}{2.764751in}}{\pgfqpoint{1.880070in}{2.764751in}}%
\pgfpathcurveto{\pgfqpoint{1.869020in}{2.764751in}}{\pgfqpoint{1.858421in}{2.760361in}}{\pgfqpoint{1.850607in}{2.752547in}}%
\pgfpathcurveto{\pgfqpoint{1.842794in}{2.744734in}}{\pgfqpoint{1.838403in}{2.734135in}}{\pgfqpoint{1.838403in}{2.723084in}}%
\pgfpathcurveto{\pgfqpoint{1.838403in}{2.712034in}}{\pgfqpoint{1.842794in}{2.701435in}}{\pgfqpoint{1.850607in}{2.693622in}}%
\pgfpathcurveto{\pgfqpoint{1.858421in}{2.685808in}}{\pgfqpoint{1.869020in}{2.681418in}}{\pgfqpoint{1.880070in}{2.681418in}}%
\pgfpathclose%
\pgfusepath{stroke,fill}%
\end{pgfscope}%
\begin{pgfscope}%
\pgfpathrectangle{\pgfqpoint{0.787074in}{0.548769in}}{\pgfqpoint{5.062926in}{3.102590in}}%
\pgfusepath{clip}%
\pgfsetbuttcap%
\pgfsetroundjoin%
\definecolor{currentfill}{rgb}{1.000000,0.498039,0.054902}%
\pgfsetfillcolor{currentfill}%
\pgfsetlinewidth{1.003750pt}%
\definecolor{currentstroke}{rgb}{1.000000,0.498039,0.054902}%
\pgfsetstrokecolor{currentstroke}%
\pgfsetdash{}{0pt}%
\pgfpathmoveto{\pgfqpoint{1.923430in}{2.256332in}}%
\pgfpathcurveto{\pgfqpoint{1.934480in}{2.256332in}}{\pgfqpoint{1.945079in}{2.260722in}}{\pgfqpoint{1.952893in}{2.268536in}}%
\pgfpathcurveto{\pgfqpoint{1.960706in}{2.276349in}}{\pgfqpoint{1.965097in}{2.286948in}}{\pgfqpoint{1.965097in}{2.297998in}}%
\pgfpathcurveto{\pgfqpoint{1.965097in}{2.309048in}}{\pgfqpoint{1.960706in}{2.319648in}}{\pgfqpoint{1.952893in}{2.327461in}}%
\pgfpathcurveto{\pgfqpoint{1.945079in}{2.335275in}}{\pgfqpoint{1.934480in}{2.339665in}}{\pgfqpoint{1.923430in}{2.339665in}}%
\pgfpathcurveto{\pgfqpoint{1.912380in}{2.339665in}}{\pgfqpoint{1.901781in}{2.335275in}}{\pgfqpoint{1.893967in}{2.327461in}}%
\pgfpathcurveto{\pgfqpoint{1.886154in}{2.319648in}}{\pgfqpoint{1.881763in}{2.309048in}}{\pgfqpoint{1.881763in}{2.297998in}}%
\pgfpathcurveto{\pgfqpoint{1.881763in}{2.286948in}}{\pgfqpoint{1.886154in}{2.276349in}}{\pgfqpoint{1.893967in}{2.268536in}}%
\pgfpathcurveto{\pgfqpoint{1.901781in}{2.260722in}}{\pgfqpoint{1.912380in}{2.256332in}}{\pgfqpoint{1.923430in}{2.256332in}}%
\pgfpathclose%
\pgfusepath{stroke,fill}%
\end{pgfscope}%
\begin{pgfscope}%
\pgfpathrectangle{\pgfqpoint{0.787074in}{0.548769in}}{\pgfqpoint{5.062926in}{3.102590in}}%
\pgfusepath{clip}%
\pgfsetbuttcap%
\pgfsetroundjoin%
\definecolor{currentfill}{rgb}{0.121569,0.466667,0.705882}%
\pgfsetfillcolor{currentfill}%
\pgfsetlinewidth{1.003750pt}%
\definecolor{currentstroke}{rgb}{0.121569,0.466667,0.705882}%
\pgfsetstrokecolor{currentstroke}%
\pgfsetdash{}{0pt}%
\pgfpathmoveto{\pgfqpoint{1.327231in}{0.775326in}}%
\pgfpathcurveto{\pgfqpoint{1.338281in}{0.775326in}}{\pgfqpoint{1.348880in}{0.779716in}}{\pgfqpoint{1.356693in}{0.787530in}}%
\pgfpathcurveto{\pgfqpoint{1.364507in}{0.795343in}}{\pgfqpoint{1.368897in}{0.805942in}}{\pgfqpoint{1.368897in}{0.816993in}}%
\pgfpathcurveto{\pgfqpoint{1.368897in}{0.828043in}}{\pgfqpoint{1.364507in}{0.838642in}}{\pgfqpoint{1.356693in}{0.846455in}}%
\pgfpathcurveto{\pgfqpoint{1.348880in}{0.854269in}}{\pgfqpoint{1.338281in}{0.858659in}}{\pgfqpoint{1.327231in}{0.858659in}}%
\pgfpathcurveto{\pgfqpoint{1.316180in}{0.858659in}}{\pgfqpoint{1.305581in}{0.854269in}}{\pgfqpoint{1.297768in}{0.846455in}}%
\pgfpathcurveto{\pgfqpoint{1.289954in}{0.838642in}}{\pgfqpoint{1.285564in}{0.828043in}}{\pgfqpoint{1.285564in}{0.816993in}}%
\pgfpathcurveto{\pgfqpoint{1.285564in}{0.805942in}}{\pgfqpoint{1.289954in}{0.795343in}}{\pgfqpoint{1.297768in}{0.787530in}}%
\pgfpathcurveto{\pgfqpoint{1.305581in}{0.779716in}}{\pgfqpoint{1.316180in}{0.775326in}}{\pgfqpoint{1.327231in}{0.775326in}}%
\pgfpathclose%
\pgfusepath{stroke,fill}%
\end{pgfscope}%
\begin{pgfscope}%
\pgfpathrectangle{\pgfqpoint{0.787074in}{0.548769in}}{\pgfqpoint{5.062926in}{3.102590in}}%
\pgfusepath{clip}%
\pgfsetbuttcap%
\pgfsetroundjoin%
\definecolor{currentfill}{rgb}{1.000000,0.498039,0.054902}%
\pgfsetfillcolor{currentfill}%
\pgfsetlinewidth{1.003750pt}%
\definecolor{currentstroke}{rgb}{1.000000,0.498039,0.054902}%
\pgfsetstrokecolor{currentstroke}%
\pgfsetdash{}{0pt}%
\pgfpathmoveto{\pgfqpoint{3.195322in}{1.583415in}}%
\pgfpathcurveto{\pgfqpoint{3.206372in}{1.583415in}}{\pgfqpoint{3.216972in}{1.587805in}}{\pgfqpoint{3.224785in}{1.595619in}}%
\pgfpathcurveto{\pgfqpoint{3.232599in}{1.603432in}}{\pgfqpoint{3.236989in}{1.614031in}}{\pgfqpoint{3.236989in}{1.625081in}}%
\pgfpathcurveto{\pgfqpoint{3.236989in}{1.636131in}}{\pgfqpoint{3.232599in}{1.646730in}}{\pgfqpoint{3.224785in}{1.654544in}}%
\pgfpathcurveto{\pgfqpoint{3.216972in}{1.662358in}}{\pgfqpoint{3.206372in}{1.666748in}}{\pgfqpoint{3.195322in}{1.666748in}}%
\pgfpathcurveto{\pgfqpoint{3.184272in}{1.666748in}}{\pgfqpoint{3.173673in}{1.662358in}}{\pgfqpoint{3.165860in}{1.654544in}}%
\pgfpathcurveto{\pgfqpoint{3.158046in}{1.646730in}}{\pgfqpoint{3.153656in}{1.636131in}}{\pgfqpoint{3.153656in}{1.625081in}}%
\pgfpathcurveto{\pgfqpoint{3.153656in}{1.614031in}}{\pgfqpoint{3.158046in}{1.603432in}}{\pgfqpoint{3.165860in}{1.595619in}}%
\pgfpathcurveto{\pgfqpoint{3.173673in}{1.587805in}}{\pgfqpoint{3.184272in}{1.583415in}}{\pgfqpoint{3.195322in}{1.583415in}}%
\pgfpathclose%
\pgfusepath{stroke,fill}%
\end{pgfscope}%
\begin{pgfscope}%
\pgfpathrectangle{\pgfqpoint{0.787074in}{0.548769in}}{\pgfqpoint{5.062926in}{3.102590in}}%
\pgfusepath{clip}%
\pgfsetbuttcap%
\pgfsetroundjoin%
\definecolor{currentfill}{rgb}{1.000000,0.498039,0.054902}%
\pgfsetfillcolor{currentfill}%
\pgfsetlinewidth{1.003750pt}%
\definecolor{currentstroke}{rgb}{1.000000,0.498039,0.054902}%
\pgfsetstrokecolor{currentstroke}%
\pgfsetdash{}{0pt}%
\pgfpathmoveto{\pgfqpoint{1.910879in}{2.261976in}}%
\pgfpathcurveto{\pgfqpoint{1.921929in}{2.261976in}}{\pgfqpoint{1.932528in}{2.266366in}}{\pgfqpoint{1.940341in}{2.274180in}}%
\pgfpathcurveto{\pgfqpoint{1.948155in}{2.281994in}}{\pgfqpoint{1.952545in}{2.292593in}}{\pgfqpoint{1.952545in}{2.303643in}}%
\pgfpathcurveto{\pgfqpoint{1.952545in}{2.314693in}}{\pgfqpoint{1.948155in}{2.325292in}}{\pgfqpoint{1.940341in}{2.333105in}}%
\pgfpathcurveto{\pgfqpoint{1.932528in}{2.340919in}}{\pgfqpoint{1.921929in}{2.345309in}}{\pgfqpoint{1.910879in}{2.345309in}}%
\pgfpathcurveto{\pgfqpoint{1.899828in}{2.345309in}}{\pgfqpoint{1.889229in}{2.340919in}}{\pgfqpoint{1.881416in}{2.333105in}}%
\pgfpathcurveto{\pgfqpoint{1.873602in}{2.325292in}}{\pgfqpoint{1.869212in}{2.314693in}}{\pgfqpoint{1.869212in}{2.303643in}}%
\pgfpathcurveto{\pgfqpoint{1.869212in}{2.292593in}}{\pgfqpoint{1.873602in}{2.281994in}}{\pgfqpoint{1.881416in}{2.274180in}}%
\pgfpathcurveto{\pgfqpoint{1.889229in}{2.266366in}}{\pgfqpoint{1.899828in}{2.261976in}}{\pgfqpoint{1.910879in}{2.261976in}}%
\pgfpathclose%
\pgfusepath{stroke,fill}%
\end{pgfscope}%
\begin{pgfscope}%
\pgfpathrectangle{\pgfqpoint{0.787074in}{0.548769in}}{\pgfqpoint{5.062926in}{3.102590in}}%
\pgfusepath{clip}%
\pgfsetbuttcap%
\pgfsetroundjoin%
\definecolor{currentfill}{rgb}{1.000000,0.498039,0.054902}%
\pgfsetfillcolor{currentfill}%
\pgfsetlinewidth{1.003750pt}%
\definecolor{currentstroke}{rgb}{1.000000,0.498039,0.054902}%
\pgfsetstrokecolor{currentstroke}%
\pgfsetdash{}{0pt}%
\pgfpathmoveto{\pgfqpoint{2.161910in}{3.078940in}}%
\pgfpathcurveto{\pgfqpoint{2.172960in}{3.078940in}}{\pgfqpoint{2.183559in}{3.083331in}}{\pgfqpoint{2.191373in}{3.091144in}}%
\pgfpathcurveto{\pgfqpoint{2.199186in}{3.098958in}}{\pgfqpoint{2.203577in}{3.109557in}}{\pgfqpoint{2.203577in}{3.120607in}}%
\pgfpathcurveto{\pgfqpoint{2.203577in}{3.131657in}}{\pgfqpoint{2.199186in}{3.142256in}}{\pgfqpoint{2.191373in}{3.150070in}}%
\pgfpathcurveto{\pgfqpoint{2.183559in}{3.157883in}}{\pgfqpoint{2.172960in}{3.162274in}}{\pgfqpoint{2.161910in}{3.162274in}}%
\pgfpathcurveto{\pgfqpoint{2.150860in}{3.162274in}}{\pgfqpoint{2.140261in}{3.157883in}}{\pgfqpoint{2.132447in}{3.150070in}}%
\pgfpathcurveto{\pgfqpoint{2.124633in}{3.142256in}}{\pgfqpoint{2.120243in}{3.131657in}}{\pgfqpoint{2.120243in}{3.120607in}}%
\pgfpathcurveto{\pgfqpoint{2.120243in}{3.109557in}}{\pgfqpoint{2.124633in}{3.098958in}}{\pgfqpoint{2.132447in}{3.091144in}}%
\pgfpathcurveto{\pgfqpoint{2.140261in}{3.083331in}}{\pgfqpoint{2.150860in}{3.078940in}}{\pgfqpoint{2.161910in}{3.078940in}}%
\pgfpathclose%
\pgfusepath{stroke,fill}%
\end{pgfscope}%
\begin{pgfscope}%
\pgfpathrectangle{\pgfqpoint{0.787074in}{0.548769in}}{\pgfqpoint{5.062926in}{3.102590in}}%
\pgfusepath{clip}%
\pgfsetbuttcap%
\pgfsetroundjoin%
\definecolor{currentfill}{rgb}{1.000000,0.498039,0.054902}%
\pgfsetfillcolor{currentfill}%
\pgfsetlinewidth{1.003750pt}%
\definecolor{currentstroke}{rgb}{1.000000,0.498039,0.054902}%
\pgfsetstrokecolor{currentstroke}%
\pgfsetdash{}{0pt}%
\pgfpathmoveto{\pgfqpoint{1.496677in}{1.210818in}}%
\pgfpathcurveto{\pgfqpoint{1.507727in}{1.210818in}}{\pgfqpoint{1.518326in}{1.215208in}}{\pgfqpoint{1.526140in}{1.223021in}}%
\pgfpathcurveto{\pgfqpoint{1.533953in}{1.230835in}}{\pgfqpoint{1.538343in}{1.241434in}}{\pgfqpoint{1.538343in}{1.252484in}}%
\pgfpathcurveto{\pgfqpoint{1.538343in}{1.263534in}}{\pgfqpoint{1.533953in}{1.274133in}}{\pgfqpoint{1.526140in}{1.281947in}}%
\pgfpathcurveto{\pgfqpoint{1.518326in}{1.289761in}}{\pgfqpoint{1.507727in}{1.294151in}}{\pgfqpoint{1.496677in}{1.294151in}}%
\pgfpathcurveto{\pgfqpoint{1.485627in}{1.294151in}}{\pgfqpoint{1.475028in}{1.289761in}}{\pgfqpoint{1.467214in}{1.281947in}}%
\pgfpathcurveto{\pgfqpoint{1.459400in}{1.274133in}}{\pgfqpoint{1.455010in}{1.263534in}}{\pgfqpoint{1.455010in}{1.252484in}}%
\pgfpathcurveto{\pgfqpoint{1.455010in}{1.241434in}}{\pgfqpoint{1.459400in}{1.230835in}}{\pgfqpoint{1.467214in}{1.223021in}}%
\pgfpathcurveto{\pgfqpoint{1.475028in}{1.215208in}}{\pgfqpoint{1.485627in}{1.210818in}}{\pgfqpoint{1.496677in}{1.210818in}}%
\pgfpathclose%
\pgfusepath{stroke,fill}%
\end{pgfscope}%
\begin{pgfscope}%
\pgfpathrectangle{\pgfqpoint{0.787074in}{0.548769in}}{\pgfqpoint{5.062926in}{3.102590in}}%
\pgfusepath{clip}%
\pgfsetbuttcap%
\pgfsetroundjoin%
\definecolor{currentfill}{rgb}{1.000000,0.498039,0.054902}%
\pgfsetfillcolor{currentfill}%
\pgfsetlinewidth{1.003750pt}%
\definecolor{currentstroke}{rgb}{1.000000,0.498039,0.054902}%
\pgfsetstrokecolor{currentstroke}%
\pgfsetdash{}{0pt}%
\pgfpathmoveto{\pgfqpoint{1.804190in}{1.923637in}}%
\pgfpathcurveto{\pgfqpoint{1.815240in}{1.923637in}}{\pgfqpoint{1.825839in}{1.928027in}}{\pgfqpoint{1.833653in}{1.935841in}}%
\pgfpathcurveto{\pgfqpoint{1.841467in}{1.943654in}}{\pgfqpoint{1.845857in}{1.954253in}}{\pgfqpoint{1.845857in}{1.965303in}}%
\pgfpathcurveto{\pgfqpoint{1.845857in}{1.976354in}}{\pgfqpoint{1.841467in}{1.986953in}}{\pgfqpoint{1.833653in}{1.994766in}}%
\pgfpathcurveto{\pgfqpoint{1.825839in}{2.002580in}}{\pgfqpoint{1.815240in}{2.006970in}}{\pgfqpoint{1.804190in}{2.006970in}}%
\pgfpathcurveto{\pgfqpoint{1.793140in}{2.006970in}}{\pgfqpoint{1.782541in}{2.002580in}}{\pgfqpoint{1.774727in}{1.994766in}}%
\pgfpathcurveto{\pgfqpoint{1.766914in}{1.986953in}}{\pgfqpoint{1.762524in}{1.976354in}}{\pgfqpoint{1.762524in}{1.965303in}}%
\pgfpathcurveto{\pgfqpoint{1.762524in}{1.954253in}}{\pgfqpoint{1.766914in}{1.943654in}}{\pgfqpoint{1.774727in}{1.935841in}}%
\pgfpathcurveto{\pgfqpoint{1.782541in}{1.928027in}}{\pgfqpoint{1.793140in}{1.923637in}}{\pgfqpoint{1.804190in}{1.923637in}}%
\pgfpathclose%
\pgfusepath{stroke,fill}%
\end{pgfscope}%
\begin{pgfscope}%
\pgfpathrectangle{\pgfqpoint{0.787074in}{0.548769in}}{\pgfqpoint{5.062926in}{3.102590in}}%
\pgfusepath{clip}%
\pgfsetbuttcap%
\pgfsetroundjoin%
\definecolor{currentfill}{rgb}{1.000000,0.498039,0.054902}%
\pgfsetfillcolor{currentfill}%
\pgfsetlinewidth{1.003750pt}%
\definecolor{currentstroke}{rgb}{1.000000,0.498039,0.054902}%
\pgfsetstrokecolor{currentstroke}%
\pgfsetdash{}{0pt}%
\pgfpathmoveto{\pgfqpoint{1.327231in}{2.880292in}}%
\pgfpathcurveto{\pgfqpoint{1.338281in}{2.880292in}}{\pgfqpoint{1.348880in}{2.884682in}}{\pgfqpoint{1.356693in}{2.892495in}}%
\pgfpathcurveto{\pgfqpoint{1.364507in}{2.900309in}}{\pgfqpoint{1.368897in}{2.910908in}}{\pgfqpoint{1.368897in}{2.921958in}}%
\pgfpathcurveto{\pgfqpoint{1.368897in}{2.933008in}}{\pgfqpoint{1.364507in}{2.943607in}}{\pgfqpoint{1.356693in}{2.951421in}}%
\pgfpathcurveto{\pgfqpoint{1.348880in}{2.959235in}}{\pgfqpoint{1.338281in}{2.963625in}}{\pgfqpoint{1.327231in}{2.963625in}}%
\pgfpathcurveto{\pgfqpoint{1.316180in}{2.963625in}}{\pgfqpoint{1.305581in}{2.959235in}}{\pgfqpoint{1.297768in}{2.951421in}}%
\pgfpathcurveto{\pgfqpoint{1.289954in}{2.943607in}}{\pgfqpoint{1.285564in}{2.933008in}}{\pgfqpoint{1.285564in}{2.921958in}}%
\pgfpathcurveto{\pgfqpoint{1.285564in}{2.910908in}}{\pgfqpoint{1.289954in}{2.900309in}}{\pgfqpoint{1.297768in}{2.892495in}}%
\pgfpathcurveto{\pgfqpoint{1.305581in}{2.884682in}}{\pgfqpoint{1.316180in}{2.880292in}}{\pgfqpoint{1.327231in}{2.880292in}}%
\pgfpathclose%
\pgfusepath{stroke,fill}%
\end{pgfscope}%
\begin{pgfscope}%
\pgfpathrectangle{\pgfqpoint{0.787074in}{0.548769in}}{\pgfqpoint{5.062926in}{3.102590in}}%
\pgfusepath{clip}%
\pgfsetbuttcap%
\pgfsetroundjoin%
\definecolor{currentfill}{rgb}{1.000000,0.498039,0.054902}%
\pgfsetfillcolor{currentfill}%
\pgfsetlinewidth{1.003750pt}%
\definecolor{currentstroke}{rgb}{1.000000,0.498039,0.054902}%
\pgfsetstrokecolor{currentstroke}%
\pgfsetdash{}{0pt}%
\pgfpathmoveto{\pgfqpoint{1.500321in}{2.018767in}}%
\pgfpathcurveto{\pgfqpoint{1.511371in}{2.018767in}}{\pgfqpoint{1.521970in}{2.023158in}}{\pgfqpoint{1.529784in}{2.030971in}}%
\pgfpathcurveto{\pgfqpoint{1.537597in}{2.038785in}}{\pgfqpoint{1.541987in}{2.049384in}}{\pgfqpoint{1.541987in}{2.060434in}}%
\pgfpathcurveto{\pgfqpoint{1.541987in}{2.071484in}}{\pgfqpoint{1.537597in}{2.082083in}}{\pgfqpoint{1.529784in}{2.089897in}}%
\pgfpathcurveto{\pgfqpoint{1.521970in}{2.097710in}}{\pgfqpoint{1.511371in}{2.102101in}}{\pgfqpoint{1.500321in}{2.102101in}}%
\pgfpathcurveto{\pgfqpoint{1.489271in}{2.102101in}}{\pgfqpoint{1.478672in}{2.097710in}}{\pgfqpoint{1.470858in}{2.089897in}}%
\pgfpathcurveto{\pgfqpoint{1.463044in}{2.082083in}}{\pgfqpoint{1.458654in}{2.071484in}}{\pgfqpoint{1.458654in}{2.060434in}}%
\pgfpathcurveto{\pgfqpoint{1.458654in}{2.049384in}}{\pgfqpoint{1.463044in}{2.038785in}}{\pgfqpoint{1.470858in}{2.030971in}}%
\pgfpathcurveto{\pgfqpoint{1.478672in}{2.023158in}}{\pgfqpoint{1.489271in}{2.018767in}}{\pgfqpoint{1.500321in}{2.018767in}}%
\pgfpathclose%
\pgfusepath{stroke,fill}%
\end{pgfscope}%
\begin{pgfscope}%
\pgfpathrectangle{\pgfqpoint{0.787074in}{0.548769in}}{\pgfqpoint{5.062926in}{3.102590in}}%
\pgfusepath{clip}%
\pgfsetbuttcap%
\pgfsetroundjoin%
\definecolor{currentfill}{rgb}{1.000000,0.498039,0.054902}%
\pgfsetfillcolor{currentfill}%
\pgfsetlinewidth{1.003750pt}%
\definecolor{currentstroke}{rgb}{1.000000,0.498039,0.054902}%
\pgfsetstrokecolor{currentstroke}%
\pgfsetdash{}{0pt}%
\pgfpathmoveto{\pgfqpoint{1.362301in}{0.875725in}}%
\pgfpathcurveto{\pgfqpoint{1.373351in}{0.875725in}}{\pgfqpoint{1.383950in}{0.880115in}}{\pgfqpoint{1.391764in}{0.887929in}}%
\pgfpathcurveto{\pgfqpoint{1.399578in}{0.895743in}}{\pgfqpoint{1.403968in}{0.906342in}}{\pgfqpoint{1.403968in}{0.917392in}}%
\pgfpathcurveto{\pgfqpoint{1.403968in}{0.928442in}}{\pgfqpoint{1.399578in}{0.939041in}}{\pgfqpoint{1.391764in}{0.946855in}}%
\pgfpathcurveto{\pgfqpoint{1.383950in}{0.954668in}}{\pgfqpoint{1.373351in}{0.959058in}}{\pgfqpoint{1.362301in}{0.959058in}}%
\pgfpathcurveto{\pgfqpoint{1.351251in}{0.959058in}}{\pgfqpoint{1.340652in}{0.954668in}}{\pgfqpoint{1.332838in}{0.946855in}}%
\pgfpathcurveto{\pgfqpoint{1.325025in}{0.939041in}}{\pgfqpoint{1.320634in}{0.928442in}}{\pgfqpoint{1.320634in}{0.917392in}}%
\pgfpathcurveto{\pgfqpoint{1.320634in}{0.906342in}}{\pgfqpoint{1.325025in}{0.895743in}}{\pgfqpoint{1.332838in}{0.887929in}}%
\pgfpathcurveto{\pgfqpoint{1.340652in}{0.880115in}}{\pgfqpoint{1.351251in}{0.875725in}}{\pgfqpoint{1.362301in}{0.875725in}}%
\pgfpathclose%
\pgfusepath{stroke,fill}%
\end{pgfscope}%
\begin{pgfscope}%
\pgfpathrectangle{\pgfqpoint{0.787074in}{0.548769in}}{\pgfqpoint{5.062926in}{3.102590in}}%
\pgfusepath{clip}%
\pgfsetbuttcap%
\pgfsetroundjoin%
\definecolor{currentfill}{rgb}{1.000000,0.498039,0.054902}%
\pgfsetfillcolor{currentfill}%
\pgfsetlinewidth{1.003750pt}%
\definecolor{currentstroke}{rgb}{1.000000,0.498039,0.054902}%
\pgfsetstrokecolor{currentstroke}%
\pgfsetdash{}{0pt}%
\pgfpathmoveto{\pgfqpoint{2.042670in}{1.792377in}}%
\pgfpathcurveto{\pgfqpoint{2.053720in}{1.792377in}}{\pgfqpoint{2.064319in}{1.796767in}}{\pgfqpoint{2.072133in}{1.804581in}}%
\pgfpathcurveto{\pgfqpoint{2.079946in}{1.812394in}}{\pgfqpoint{2.084337in}{1.822993in}}{\pgfqpoint{2.084337in}{1.834043in}}%
\pgfpathcurveto{\pgfqpoint{2.084337in}{1.845093in}}{\pgfqpoint{2.079946in}{1.855693in}}{\pgfqpoint{2.072133in}{1.863506in}}%
\pgfpathcurveto{\pgfqpoint{2.064319in}{1.871320in}}{\pgfqpoint{2.053720in}{1.875710in}}{\pgfqpoint{2.042670in}{1.875710in}}%
\pgfpathcurveto{\pgfqpoint{2.031620in}{1.875710in}}{\pgfqpoint{2.021021in}{1.871320in}}{\pgfqpoint{2.013207in}{1.863506in}}%
\pgfpathcurveto{\pgfqpoint{2.005394in}{1.855693in}}{\pgfqpoint{2.001003in}{1.845093in}}{\pgfqpoint{2.001003in}{1.834043in}}%
\pgfpathcurveto{\pgfqpoint{2.001003in}{1.822993in}}{\pgfqpoint{2.005394in}{1.812394in}}{\pgfqpoint{2.013207in}{1.804581in}}%
\pgfpathcurveto{\pgfqpoint{2.021021in}{1.796767in}}{\pgfqpoint{2.031620in}{1.792377in}}{\pgfqpoint{2.042670in}{1.792377in}}%
\pgfpathclose%
\pgfusepath{stroke,fill}%
\end{pgfscope}%
\begin{pgfscope}%
\pgfpathrectangle{\pgfqpoint{0.787074in}{0.548769in}}{\pgfqpoint{5.062926in}{3.102590in}}%
\pgfusepath{clip}%
\pgfsetbuttcap%
\pgfsetroundjoin%
\definecolor{currentfill}{rgb}{1.000000,0.498039,0.054902}%
\pgfsetfillcolor{currentfill}%
\pgfsetlinewidth{1.003750pt}%
\definecolor{currentstroke}{rgb}{1.000000,0.498039,0.054902}%
\pgfsetstrokecolor{currentstroke}%
\pgfsetdash{}{0pt}%
\pgfpathmoveto{\pgfqpoint{1.494166in}{1.860728in}}%
\pgfpathcurveto{\pgfqpoint{1.505217in}{1.860728in}}{\pgfqpoint{1.515816in}{1.865118in}}{\pgfqpoint{1.523629in}{1.872931in}}%
\pgfpathcurveto{\pgfqpoint{1.531443in}{1.880745in}}{\pgfqpoint{1.535833in}{1.891344in}}{\pgfqpoint{1.535833in}{1.902394in}}%
\pgfpathcurveto{\pgfqpoint{1.535833in}{1.913444in}}{\pgfqpoint{1.531443in}{1.924043in}}{\pgfqpoint{1.523629in}{1.931857in}}%
\pgfpathcurveto{\pgfqpoint{1.515816in}{1.939671in}}{\pgfqpoint{1.505217in}{1.944061in}}{\pgfqpoint{1.494166in}{1.944061in}}%
\pgfpathcurveto{\pgfqpoint{1.483116in}{1.944061in}}{\pgfqpoint{1.472517in}{1.939671in}}{\pgfqpoint{1.464704in}{1.931857in}}%
\pgfpathcurveto{\pgfqpoint{1.456890in}{1.924043in}}{\pgfqpoint{1.452500in}{1.913444in}}{\pgfqpoint{1.452500in}{1.902394in}}%
\pgfpathcurveto{\pgfqpoint{1.452500in}{1.891344in}}{\pgfqpoint{1.456890in}{1.880745in}}{\pgfqpoint{1.464704in}{1.872931in}}%
\pgfpathcurveto{\pgfqpoint{1.472517in}{1.865118in}}{\pgfqpoint{1.483116in}{1.860728in}}{\pgfqpoint{1.494166in}{1.860728in}}%
\pgfpathclose%
\pgfusepath{stroke,fill}%
\end{pgfscope}%
\begin{pgfscope}%
\pgfpathrectangle{\pgfqpoint{0.787074in}{0.548769in}}{\pgfqpoint{5.062926in}{3.102590in}}%
\pgfusepath{clip}%
\pgfsetbuttcap%
\pgfsetroundjoin%
\definecolor{currentfill}{rgb}{1.000000,0.498039,0.054902}%
\pgfsetfillcolor{currentfill}%
\pgfsetlinewidth{1.003750pt}%
\definecolor{currentstroke}{rgb}{1.000000,0.498039,0.054902}%
\pgfsetstrokecolor{currentstroke}%
\pgfsetdash{}{0pt}%
\pgfpathmoveto{\pgfqpoint{3.330461in}{1.531547in}}%
\pgfpathcurveto{\pgfqpoint{3.341511in}{1.531547in}}{\pgfqpoint{3.352110in}{1.535938in}}{\pgfqpoint{3.359924in}{1.543751in}}%
\pgfpathcurveto{\pgfqpoint{3.367737in}{1.551565in}}{\pgfqpoint{3.372128in}{1.562164in}}{\pgfqpoint{3.372128in}{1.573214in}}%
\pgfpathcurveto{\pgfqpoint{3.372128in}{1.584264in}}{\pgfqpoint{3.367737in}{1.594863in}}{\pgfqpoint{3.359924in}{1.602677in}}%
\pgfpathcurveto{\pgfqpoint{3.352110in}{1.610490in}}{\pgfqpoint{3.341511in}{1.614881in}}{\pgfqpoint{3.330461in}{1.614881in}}%
\pgfpathcurveto{\pgfqpoint{3.319411in}{1.614881in}}{\pgfqpoint{3.308812in}{1.610490in}}{\pgfqpoint{3.300998in}{1.602677in}}%
\pgfpathcurveto{\pgfqpoint{3.293185in}{1.594863in}}{\pgfqpoint{3.288794in}{1.584264in}}{\pgfqpoint{3.288794in}{1.573214in}}%
\pgfpathcurveto{\pgfqpoint{3.288794in}{1.562164in}}{\pgfqpoint{3.293185in}{1.551565in}}{\pgfqpoint{3.300998in}{1.543751in}}%
\pgfpathcurveto{\pgfqpoint{3.308812in}{1.535938in}}{\pgfqpoint{3.319411in}{1.531547in}}{\pgfqpoint{3.330461in}{1.531547in}}%
\pgfpathclose%
\pgfusepath{stroke,fill}%
\end{pgfscope}%
\begin{pgfscope}%
\pgfpathrectangle{\pgfqpoint{0.787074in}{0.548769in}}{\pgfqpoint{5.062926in}{3.102590in}}%
\pgfusepath{clip}%
\pgfsetbuttcap%
\pgfsetroundjoin%
\definecolor{currentfill}{rgb}{1.000000,0.498039,0.054902}%
\pgfsetfillcolor{currentfill}%
\pgfsetlinewidth{1.003750pt}%
\definecolor{currentstroke}{rgb}{1.000000,0.498039,0.054902}%
\pgfsetstrokecolor{currentstroke}%
\pgfsetdash{}{0pt}%
\pgfpathmoveto{\pgfqpoint{1.262191in}{2.472823in}}%
\pgfpathcurveto{\pgfqpoint{1.273241in}{2.472823in}}{\pgfqpoint{1.283840in}{2.477213in}}{\pgfqpoint{1.291653in}{2.485027in}}%
\pgfpathcurveto{\pgfqpoint{1.299467in}{2.492840in}}{\pgfqpoint{1.303857in}{2.503439in}}{\pgfqpoint{1.303857in}{2.514489in}}%
\pgfpathcurveto{\pgfqpoint{1.303857in}{2.525539in}}{\pgfqpoint{1.299467in}{2.536138in}}{\pgfqpoint{1.291653in}{2.543952in}}%
\pgfpathcurveto{\pgfqpoint{1.283840in}{2.551766in}}{\pgfqpoint{1.273241in}{2.556156in}}{\pgfqpoint{1.262191in}{2.556156in}}%
\pgfpathcurveto{\pgfqpoint{1.251141in}{2.556156in}}{\pgfqpoint{1.240541in}{2.551766in}}{\pgfqpoint{1.232728in}{2.543952in}}%
\pgfpathcurveto{\pgfqpoint{1.224914in}{2.536138in}}{\pgfqpoint{1.220524in}{2.525539in}}{\pgfqpoint{1.220524in}{2.514489in}}%
\pgfpathcurveto{\pgfqpoint{1.220524in}{2.503439in}}{\pgfqpoint{1.224914in}{2.492840in}}{\pgfqpoint{1.232728in}{2.485027in}}%
\pgfpathcurveto{\pgfqpoint{1.240541in}{2.477213in}}{\pgfqpoint{1.251141in}{2.472823in}}{\pgfqpoint{1.262191in}{2.472823in}}%
\pgfpathclose%
\pgfusepath{stroke,fill}%
\end{pgfscope}%
\begin{pgfscope}%
\pgfpathrectangle{\pgfqpoint{0.787074in}{0.548769in}}{\pgfqpoint{5.062926in}{3.102590in}}%
\pgfusepath{clip}%
\pgfsetbuttcap%
\pgfsetroundjoin%
\definecolor{currentfill}{rgb}{1.000000,0.498039,0.054902}%
\pgfsetfillcolor{currentfill}%
\pgfsetlinewidth{1.003750pt}%
\definecolor{currentstroke}{rgb}{1.000000,0.498039,0.054902}%
\pgfsetstrokecolor{currentstroke}%
\pgfsetdash{}{0pt}%
\pgfpathmoveto{\pgfqpoint{1.963177in}{2.690255in}}%
\pgfpathcurveto{\pgfqpoint{1.974227in}{2.690255in}}{\pgfqpoint{1.984826in}{2.694645in}}{\pgfqpoint{1.992640in}{2.702459in}}%
\pgfpathcurveto{\pgfqpoint{2.000453in}{2.710273in}}{\pgfqpoint{2.004843in}{2.720872in}}{\pgfqpoint{2.004843in}{2.731922in}}%
\pgfpathcurveto{\pgfqpoint{2.004843in}{2.742972in}}{\pgfqpoint{2.000453in}{2.753571in}}{\pgfqpoint{1.992640in}{2.761384in}}%
\pgfpathcurveto{\pgfqpoint{1.984826in}{2.769198in}}{\pgfqpoint{1.974227in}{2.773588in}}{\pgfqpoint{1.963177in}{2.773588in}}%
\pgfpathcurveto{\pgfqpoint{1.952127in}{2.773588in}}{\pgfqpoint{1.941528in}{2.769198in}}{\pgfqpoint{1.933714in}{2.761384in}}%
\pgfpathcurveto{\pgfqpoint{1.925900in}{2.753571in}}{\pgfqpoint{1.921510in}{2.742972in}}{\pgfqpoint{1.921510in}{2.731922in}}%
\pgfpathcurveto{\pgfqpoint{1.921510in}{2.720872in}}{\pgfqpoint{1.925900in}{2.710273in}}{\pgfqpoint{1.933714in}{2.702459in}}%
\pgfpathcurveto{\pgfqpoint{1.941528in}{2.694645in}}{\pgfqpoint{1.952127in}{2.690255in}}{\pgfqpoint{1.963177in}{2.690255in}}%
\pgfpathclose%
\pgfusepath{stroke,fill}%
\end{pgfscope}%
\begin{pgfscope}%
\pgfpathrectangle{\pgfqpoint{0.787074in}{0.548769in}}{\pgfqpoint{5.062926in}{3.102590in}}%
\pgfusepath{clip}%
\pgfsetbuttcap%
\pgfsetroundjoin%
\definecolor{currentfill}{rgb}{1.000000,0.498039,0.054902}%
\pgfsetfillcolor{currentfill}%
\pgfsetlinewidth{1.003750pt}%
\definecolor{currentstroke}{rgb}{1.000000,0.498039,0.054902}%
\pgfsetstrokecolor{currentstroke}%
\pgfsetdash{}{0pt}%
\pgfpathmoveto{\pgfqpoint{1.164631in}{3.301211in}}%
\pgfpathcurveto{\pgfqpoint{1.175681in}{3.301211in}}{\pgfqpoint{1.186280in}{3.305601in}}{\pgfqpoint{1.194094in}{3.313415in}}%
\pgfpathcurveto{\pgfqpoint{1.201907in}{3.321228in}}{\pgfqpoint{1.206297in}{3.331828in}}{\pgfqpoint{1.206297in}{3.342878in}}%
\pgfpathcurveto{\pgfqpoint{1.206297in}{3.353928in}}{\pgfqpoint{1.201907in}{3.364527in}}{\pgfqpoint{1.194094in}{3.372340in}}%
\pgfpathcurveto{\pgfqpoint{1.186280in}{3.380154in}}{\pgfqpoint{1.175681in}{3.384544in}}{\pgfqpoint{1.164631in}{3.384544in}}%
\pgfpathcurveto{\pgfqpoint{1.153581in}{3.384544in}}{\pgfqpoint{1.142982in}{3.380154in}}{\pgfqpoint{1.135168in}{3.372340in}}%
\pgfpathcurveto{\pgfqpoint{1.127354in}{3.364527in}}{\pgfqpoint{1.122964in}{3.353928in}}{\pgfqpoint{1.122964in}{3.342878in}}%
\pgfpathcurveto{\pgfqpoint{1.122964in}{3.331828in}}{\pgfqpoint{1.127354in}{3.321228in}}{\pgfqpoint{1.135168in}{3.313415in}}%
\pgfpathcurveto{\pgfqpoint{1.142982in}{3.305601in}}{\pgfqpoint{1.153581in}{3.301211in}}{\pgfqpoint{1.164631in}{3.301211in}}%
\pgfpathclose%
\pgfusepath{stroke,fill}%
\end{pgfscope}%
\begin{pgfscope}%
\pgfpathrectangle{\pgfqpoint{0.787074in}{0.548769in}}{\pgfqpoint{5.062926in}{3.102590in}}%
\pgfusepath{clip}%
\pgfsetbuttcap%
\pgfsetroundjoin%
\definecolor{currentfill}{rgb}{1.000000,0.498039,0.054902}%
\pgfsetfillcolor{currentfill}%
\pgfsetlinewidth{1.003750pt}%
\definecolor{currentstroke}{rgb}{1.000000,0.498039,0.054902}%
\pgfsetstrokecolor{currentstroke}%
\pgfsetdash{}{0pt}%
\pgfpathmoveto{\pgfqpoint{1.684950in}{3.101406in}}%
\pgfpathcurveto{\pgfqpoint{1.696000in}{3.101406in}}{\pgfqpoint{1.706599in}{3.105796in}}{\pgfqpoint{1.714413in}{3.113610in}}%
\pgfpathcurveto{\pgfqpoint{1.722227in}{3.121423in}}{\pgfqpoint{1.726617in}{3.132022in}}{\pgfqpoint{1.726617in}{3.143073in}}%
\pgfpathcurveto{\pgfqpoint{1.726617in}{3.154123in}}{\pgfqpoint{1.722227in}{3.164722in}}{\pgfqpoint{1.714413in}{3.172535in}}%
\pgfpathcurveto{\pgfqpoint{1.706599in}{3.180349in}}{\pgfqpoint{1.696000in}{3.184739in}}{\pgfqpoint{1.684950in}{3.184739in}}%
\pgfpathcurveto{\pgfqpoint{1.673900in}{3.184739in}}{\pgfqpoint{1.663301in}{3.180349in}}{\pgfqpoint{1.655488in}{3.172535in}}%
\pgfpathcurveto{\pgfqpoint{1.647674in}{3.164722in}}{\pgfqpoint{1.643284in}{3.154123in}}{\pgfqpoint{1.643284in}{3.143073in}}%
\pgfpathcurveto{\pgfqpoint{1.643284in}{3.132022in}}{\pgfqpoint{1.647674in}{3.121423in}}{\pgfqpoint{1.655488in}{3.113610in}}%
\pgfpathcurveto{\pgfqpoint{1.663301in}{3.105796in}}{\pgfqpoint{1.673900in}{3.101406in}}{\pgfqpoint{1.684950in}{3.101406in}}%
\pgfpathclose%
\pgfusepath{stroke,fill}%
\end{pgfscope}%
\begin{pgfscope}%
\pgfpathrectangle{\pgfqpoint{0.787074in}{0.548769in}}{\pgfqpoint{5.062926in}{3.102590in}}%
\pgfusepath{clip}%
\pgfsetbuttcap%
\pgfsetroundjoin%
\definecolor{currentfill}{rgb}{1.000000,0.498039,0.054902}%
\pgfsetfillcolor{currentfill}%
\pgfsetlinewidth{1.003750pt}%
\definecolor{currentstroke}{rgb}{1.000000,0.498039,0.054902}%
\pgfsetstrokecolor{currentstroke}%
\pgfsetdash{}{0pt}%
\pgfpathmoveto{\pgfqpoint{1.684950in}{1.878073in}}%
\pgfpathcurveto{\pgfqpoint{1.696000in}{1.878073in}}{\pgfqpoint{1.706599in}{1.882464in}}{\pgfqpoint{1.714413in}{1.890277in}}%
\pgfpathcurveto{\pgfqpoint{1.722227in}{1.898091in}}{\pgfqpoint{1.726617in}{1.908690in}}{\pgfqpoint{1.726617in}{1.919740in}}%
\pgfpathcurveto{\pgfqpoint{1.726617in}{1.930790in}}{\pgfqpoint{1.722227in}{1.941389in}}{\pgfqpoint{1.714413in}{1.949203in}}%
\pgfpathcurveto{\pgfqpoint{1.706599in}{1.957016in}}{\pgfqpoint{1.696000in}{1.961407in}}{\pgfqpoint{1.684950in}{1.961407in}}%
\pgfpathcurveto{\pgfqpoint{1.673900in}{1.961407in}}{\pgfqpoint{1.663301in}{1.957016in}}{\pgfqpoint{1.655488in}{1.949203in}}%
\pgfpathcurveto{\pgfqpoint{1.647674in}{1.941389in}}{\pgfqpoint{1.643284in}{1.930790in}}{\pgfqpoint{1.643284in}{1.919740in}}%
\pgfpathcurveto{\pgfqpoint{1.643284in}{1.908690in}}{\pgfqpoint{1.647674in}{1.898091in}}{\pgfqpoint{1.655488in}{1.890277in}}%
\pgfpathcurveto{\pgfqpoint{1.663301in}{1.882464in}}{\pgfqpoint{1.673900in}{1.878073in}}{\pgfqpoint{1.684950in}{1.878073in}}%
\pgfpathclose%
\pgfusepath{stroke,fill}%
\end{pgfscope}%
\begin{pgfscope}%
\pgfpathrectangle{\pgfqpoint{0.787074in}{0.548769in}}{\pgfqpoint{5.062926in}{3.102590in}}%
\pgfusepath{clip}%
\pgfsetbuttcap%
\pgfsetroundjoin%
\definecolor{currentfill}{rgb}{1.000000,0.498039,0.054902}%
\pgfsetfillcolor{currentfill}%
\pgfsetlinewidth{1.003750pt}%
\definecolor{currentstroke}{rgb}{1.000000,0.498039,0.054902}%
\pgfsetstrokecolor{currentstroke}%
\pgfsetdash{}{0pt}%
\pgfpathmoveto{\pgfqpoint{1.207991in}{2.761817in}}%
\pgfpathcurveto{\pgfqpoint{1.219041in}{2.761817in}}{\pgfqpoint{1.229640in}{2.766208in}}{\pgfqpoint{1.237453in}{2.774021in}}%
\pgfpathcurveto{\pgfqpoint{1.245267in}{2.781835in}}{\pgfqpoint{1.249657in}{2.792434in}}{\pgfqpoint{1.249657in}{2.803484in}}%
\pgfpathcurveto{\pgfqpoint{1.249657in}{2.814534in}}{\pgfqpoint{1.245267in}{2.825133in}}{\pgfqpoint{1.237453in}{2.832947in}}%
\pgfpathcurveto{\pgfqpoint{1.229640in}{2.840760in}}{\pgfqpoint{1.219041in}{2.845151in}}{\pgfqpoint{1.207991in}{2.845151in}}%
\pgfpathcurveto{\pgfqpoint{1.196941in}{2.845151in}}{\pgfqpoint{1.186342in}{2.840760in}}{\pgfqpoint{1.178528in}{2.832947in}}%
\pgfpathcurveto{\pgfqpoint{1.170714in}{2.825133in}}{\pgfqpoint{1.166324in}{2.814534in}}{\pgfqpoint{1.166324in}{2.803484in}}%
\pgfpathcurveto{\pgfqpoint{1.166324in}{2.792434in}}{\pgfqpoint{1.170714in}{2.781835in}}{\pgfqpoint{1.178528in}{2.774021in}}%
\pgfpathcurveto{\pgfqpoint{1.186342in}{2.766208in}}{\pgfqpoint{1.196941in}{2.761817in}}{\pgfqpoint{1.207991in}{2.761817in}}%
\pgfpathclose%
\pgfusepath{stroke,fill}%
\end{pgfscope}%
\begin{pgfscope}%
\pgfpathrectangle{\pgfqpoint{0.787074in}{0.548769in}}{\pgfqpoint{5.062926in}{3.102590in}}%
\pgfusepath{clip}%
\pgfsetbuttcap%
\pgfsetroundjoin%
\definecolor{currentfill}{rgb}{1.000000,0.498039,0.054902}%
\pgfsetfillcolor{currentfill}%
\pgfsetlinewidth{1.003750pt}%
\definecolor{currentstroke}{rgb}{1.000000,0.498039,0.054902}%
\pgfsetstrokecolor{currentstroke}%
\pgfsetdash{}{0pt}%
\pgfpathmoveto{\pgfqpoint{1.457310in}{2.075959in}}%
\pgfpathcurveto{\pgfqpoint{1.468361in}{2.075959in}}{\pgfqpoint{1.478960in}{2.080350in}}{\pgfqpoint{1.486773in}{2.088163in}}%
\pgfpathcurveto{\pgfqpoint{1.494587in}{2.095977in}}{\pgfqpoint{1.498977in}{2.106576in}}{\pgfqpoint{1.498977in}{2.117626in}}%
\pgfpathcurveto{\pgfqpoint{1.498977in}{2.128676in}}{\pgfqpoint{1.494587in}{2.139275in}}{\pgfqpoint{1.486773in}{2.147089in}}%
\pgfpathcurveto{\pgfqpoint{1.478960in}{2.154902in}}{\pgfqpoint{1.468361in}{2.159293in}}{\pgfqpoint{1.457310in}{2.159293in}}%
\pgfpathcurveto{\pgfqpoint{1.446260in}{2.159293in}}{\pgfqpoint{1.435661in}{2.154902in}}{\pgfqpoint{1.427848in}{2.147089in}}%
\pgfpathcurveto{\pgfqpoint{1.420034in}{2.139275in}}{\pgfqpoint{1.415644in}{2.128676in}}{\pgfqpoint{1.415644in}{2.117626in}}%
\pgfpathcurveto{\pgfqpoint{1.415644in}{2.106576in}}{\pgfqpoint{1.420034in}{2.095977in}}{\pgfqpoint{1.427848in}{2.088163in}}%
\pgfpathcurveto{\pgfqpoint{1.435661in}{2.080350in}}{\pgfqpoint{1.446260in}{2.075959in}}{\pgfqpoint{1.457310in}{2.075959in}}%
\pgfpathclose%
\pgfusepath{stroke,fill}%
\end{pgfscope}%
\begin{pgfscope}%
\pgfpathrectangle{\pgfqpoint{0.787074in}{0.548769in}}{\pgfqpoint{5.062926in}{3.102590in}}%
\pgfusepath{clip}%
\pgfsetbuttcap%
\pgfsetroundjoin%
\definecolor{currentfill}{rgb}{1.000000,0.498039,0.054902}%
\pgfsetfillcolor{currentfill}%
\pgfsetlinewidth{1.003750pt}%
\definecolor{currentstroke}{rgb}{1.000000,0.498039,0.054902}%
\pgfsetstrokecolor{currentstroke}%
\pgfsetdash{}{0pt}%
\pgfpathmoveto{\pgfqpoint{1.933799in}{2.369584in}}%
\pgfpathcurveto{\pgfqpoint{1.944849in}{2.369584in}}{\pgfqpoint{1.955448in}{2.373975in}}{\pgfqpoint{1.963262in}{2.381788in}}%
\pgfpathcurveto{\pgfqpoint{1.971075in}{2.389602in}}{\pgfqpoint{1.975465in}{2.400201in}}{\pgfqpoint{1.975465in}{2.411251in}}%
\pgfpathcurveto{\pgfqpoint{1.975465in}{2.422301in}}{\pgfqpoint{1.971075in}{2.432900in}}{\pgfqpoint{1.963262in}{2.440714in}}%
\pgfpathcurveto{\pgfqpoint{1.955448in}{2.448527in}}{\pgfqpoint{1.944849in}{2.452918in}}{\pgfqpoint{1.933799in}{2.452918in}}%
\pgfpathcurveto{\pgfqpoint{1.922749in}{2.452918in}}{\pgfqpoint{1.912150in}{2.448527in}}{\pgfqpoint{1.904336in}{2.440714in}}%
\pgfpathcurveto{\pgfqpoint{1.896522in}{2.432900in}}{\pgfqpoint{1.892132in}{2.422301in}}{\pgfqpoint{1.892132in}{2.411251in}}%
\pgfpathcurveto{\pgfqpoint{1.892132in}{2.400201in}}{\pgfqpoint{1.896522in}{2.389602in}}{\pgfqpoint{1.904336in}{2.381788in}}%
\pgfpathcurveto{\pgfqpoint{1.912150in}{2.373975in}}{\pgfqpoint{1.922749in}{2.369584in}}{\pgfqpoint{1.933799in}{2.369584in}}%
\pgfpathclose%
\pgfusepath{stroke,fill}%
\end{pgfscope}%
\begin{pgfscope}%
\pgfpathrectangle{\pgfqpoint{0.787074in}{0.548769in}}{\pgfqpoint{5.062926in}{3.102590in}}%
\pgfusepath{clip}%
\pgfsetbuttcap%
\pgfsetroundjoin%
\definecolor{currentfill}{rgb}{0.121569,0.466667,0.705882}%
\pgfsetfillcolor{currentfill}%
\pgfsetlinewidth{1.003750pt}%
\definecolor{currentstroke}{rgb}{0.121569,0.466667,0.705882}%
\pgfsetstrokecolor{currentstroke}%
\pgfsetdash{}{0pt}%
\pgfpathmoveto{\pgfqpoint{5.619867in}{0.648131in}}%
\pgfpathcurveto{\pgfqpoint{5.630917in}{0.648131in}}{\pgfqpoint{5.641516in}{0.652521in}}{\pgfqpoint{5.649330in}{0.660335in}}%
\pgfpathcurveto{\pgfqpoint{5.657143in}{0.668148in}}{\pgfqpoint{5.661534in}{0.678747in}}{\pgfqpoint{5.661534in}{0.689798in}}%
\pgfpathcurveto{\pgfqpoint{5.661534in}{0.700848in}}{\pgfqpoint{5.657143in}{0.711447in}}{\pgfqpoint{5.649330in}{0.719260in}}%
\pgfpathcurveto{\pgfqpoint{5.641516in}{0.727074in}}{\pgfqpoint{5.630917in}{0.731464in}}{\pgfqpoint{5.619867in}{0.731464in}}%
\pgfpathcurveto{\pgfqpoint{5.608817in}{0.731464in}}{\pgfqpoint{5.598218in}{0.727074in}}{\pgfqpoint{5.590404in}{0.719260in}}%
\pgfpathcurveto{\pgfqpoint{5.582591in}{0.711447in}}{\pgfqpoint{5.578200in}{0.700848in}}{\pgfqpoint{5.578200in}{0.689798in}}%
\pgfpathcurveto{\pgfqpoint{5.578200in}{0.678747in}}{\pgfqpoint{5.582591in}{0.668148in}}{\pgfqpoint{5.590404in}{0.660335in}}%
\pgfpathcurveto{\pgfqpoint{5.598218in}{0.652521in}}{\pgfqpoint{5.608817in}{0.648131in}}{\pgfqpoint{5.619867in}{0.648131in}}%
\pgfpathclose%
\pgfusepath{stroke,fill}%
\end{pgfscope}%
\begin{pgfscope}%
\pgfpathrectangle{\pgfqpoint{0.787074in}{0.548769in}}{\pgfqpoint{5.062926in}{3.102590in}}%
\pgfusepath{clip}%
\pgfsetbuttcap%
\pgfsetroundjoin%
\definecolor{currentfill}{rgb}{1.000000,0.498039,0.054902}%
\pgfsetfillcolor{currentfill}%
\pgfsetlinewidth{1.003750pt}%
\definecolor{currentstroke}{rgb}{1.000000,0.498039,0.054902}%
\pgfsetstrokecolor{currentstroke}%
\pgfsetdash{}{0pt}%
\pgfpathmoveto{\pgfqpoint{1.178181in}{1.884932in}}%
\pgfpathcurveto{\pgfqpoint{1.189231in}{1.884932in}}{\pgfqpoint{1.199830in}{1.889322in}}{\pgfqpoint{1.207643in}{1.897136in}}%
\pgfpathcurveto{\pgfqpoint{1.215457in}{1.904950in}}{\pgfqpoint{1.219847in}{1.915549in}}{\pgfqpoint{1.219847in}{1.926599in}}%
\pgfpathcurveto{\pgfqpoint{1.219847in}{1.937649in}}{\pgfqpoint{1.215457in}{1.948248in}}{\pgfqpoint{1.207643in}{1.956061in}}%
\pgfpathcurveto{\pgfqpoint{1.199830in}{1.963875in}}{\pgfqpoint{1.189231in}{1.968265in}}{\pgfqpoint{1.178181in}{1.968265in}}%
\pgfpathcurveto{\pgfqpoint{1.167131in}{1.968265in}}{\pgfqpoint{1.156532in}{1.963875in}}{\pgfqpoint{1.148718in}{1.956061in}}%
\pgfpathcurveto{\pgfqpoint{1.140904in}{1.948248in}}{\pgfqpoint{1.136514in}{1.937649in}}{\pgfqpoint{1.136514in}{1.926599in}}%
\pgfpathcurveto{\pgfqpoint{1.136514in}{1.915549in}}{\pgfqpoint{1.140904in}{1.904950in}}{\pgfqpoint{1.148718in}{1.897136in}}%
\pgfpathcurveto{\pgfqpoint{1.156532in}{1.889322in}}{\pgfqpoint{1.167131in}{1.884932in}}{\pgfqpoint{1.178181in}{1.884932in}}%
\pgfpathclose%
\pgfusepath{stroke,fill}%
\end{pgfscope}%
\begin{pgfscope}%
\pgfpathrectangle{\pgfqpoint{0.787074in}{0.548769in}}{\pgfqpoint{5.062926in}{3.102590in}}%
\pgfusepath{clip}%
\pgfsetbuttcap%
\pgfsetroundjoin%
\definecolor{currentfill}{rgb}{1.000000,0.498039,0.054902}%
\pgfsetfillcolor{currentfill}%
\pgfsetlinewidth{1.003750pt}%
\definecolor{currentstroke}{rgb}{1.000000,0.498039,0.054902}%
\pgfsetstrokecolor{currentstroke}%
\pgfsetdash{}{0pt}%
\pgfpathmoveto{\pgfqpoint{1.049004in}{1.779524in}}%
\pgfpathcurveto{\pgfqpoint{1.060054in}{1.779524in}}{\pgfqpoint{1.070653in}{1.783914in}}{\pgfqpoint{1.078467in}{1.791728in}}%
\pgfpathcurveto{\pgfqpoint{1.086281in}{1.799541in}}{\pgfqpoint{1.090671in}{1.810140in}}{\pgfqpoint{1.090671in}{1.821190in}}%
\pgfpathcurveto{\pgfqpoint{1.090671in}{1.832241in}}{\pgfqpoint{1.086281in}{1.842840in}}{\pgfqpoint{1.078467in}{1.850653in}}%
\pgfpathcurveto{\pgfqpoint{1.070653in}{1.858467in}}{\pgfqpoint{1.060054in}{1.862857in}}{\pgfqpoint{1.049004in}{1.862857in}}%
\pgfpathcurveto{\pgfqpoint{1.037954in}{1.862857in}}{\pgfqpoint{1.027355in}{1.858467in}}{\pgfqpoint{1.019541in}{1.850653in}}%
\pgfpathcurveto{\pgfqpoint{1.011728in}{1.842840in}}{\pgfqpoint{1.007337in}{1.832241in}}{\pgfqpoint{1.007337in}{1.821190in}}%
\pgfpathcurveto{\pgfqpoint{1.007337in}{1.810140in}}{\pgfqpoint{1.011728in}{1.799541in}}{\pgfqpoint{1.019541in}{1.791728in}}%
\pgfpathcurveto{\pgfqpoint{1.027355in}{1.783914in}}{\pgfqpoint{1.037954in}{1.779524in}}{\pgfqpoint{1.049004in}{1.779524in}}%
\pgfpathclose%
\pgfusepath{stroke,fill}%
\end{pgfscope}%
\begin{pgfscope}%
\pgfpathrectangle{\pgfqpoint{0.787074in}{0.548769in}}{\pgfqpoint{5.062926in}{3.102590in}}%
\pgfusepath{clip}%
\pgfsetbuttcap%
\pgfsetroundjoin%
\definecolor{currentfill}{rgb}{1.000000,0.498039,0.054902}%
\pgfsetfillcolor{currentfill}%
\pgfsetlinewidth{1.003750pt}%
\definecolor{currentstroke}{rgb}{1.000000,0.498039,0.054902}%
\pgfsetstrokecolor{currentstroke}%
\pgfsetdash{}{0pt}%
\pgfpathmoveto{\pgfqpoint{1.416661in}{2.435775in}}%
\pgfpathcurveto{\pgfqpoint{1.427711in}{2.435775in}}{\pgfqpoint{1.438310in}{2.440165in}}{\pgfqpoint{1.446123in}{2.447979in}}%
\pgfpathcurveto{\pgfqpoint{1.453937in}{2.455792in}}{\pgfqpoint{1.458327in}{2.466391in}}{\pgfqpoint{1.458327in}{2.477441in}}%
\pgfpathcurveto{\pgfqpoint{1.458327in}{2.488492in}}{\pgfqpoint{1.453937in}{2.499091in}}{\pgfqpoint{1.446123in}{2.506904in}}%
\pgfpathcurveto{\pgfqpoint{1.438310in}{2.514718in}}{\pgfqpoint{1.427711in}{2.519108in}}{\pgfqpoint{1.416661in}{2.519108in}}%
\pgfpathcurveto{\pgfqpoint{1.405610in}{2.519108in}}{\pgfqpoint{1.395011in}{2.514718in}}{\pgfqpoint{1.387198in}{2.506904in}}%
\pgfpathcurveto{\pgfqpoint{1.379384in}{2.499091in}}{\pgfqpoint{1.374994in}{2.488492in}}{\pgfqpoint{1.374994in}{2.477441in}}%
\pgfpathcurveto{\pgfqpoint{1.374994in}{2.466391in}}{\pgfqpoint{1.379384in}{2.455792in}}{\pgfqpoint{1.387198in}{2.447979in}}%
\pgfpathcurveto{\pgfqpoint{1.395011in}{2.440165in}}{\pgfqpoint{1.405610in}{2.435775in}}{\pgfqpoint{1.416661in}{2.435775in}}%
\pgfpathclose%
\pgfusepath{stroke,fill}%
\end{pgfscope}%
\begin{pgfscope}%
\pgfpathrectangle{\pgfqpoint{0.787074in}{0.548769in}}{\pgfqpoint{5.062926in}{3.102590in}}%
\pgfusepath{clip}%
\pgfsetbuttcap%
\pgfsetroundjoin%
\definecolor{currentfill}{rgb}{1.000000,0.498039,0.054902}%
\pgfsetfillcolor{currentfill}%
\pgfsetlinewidth{1.003750pt}%
\definecolor{currentstroke}{rgb}{1.000000,0.498039,0.054902}%
\pgfsetstrokecolor{currentstroke}%
\pgfsetdash{}{0pt}%
\pgfpathmoveto{\pgfqpoint{1.619910in}{1.639934in}}%
\pgfpathcurveto{\pgfqpoint{1.630960in}{1.639934in}}{\pgfqpoint{1.641560in}{1.644324in}}{\pgfqpoint{1.649373in}{1.652138in}}%
\pgfpathcurveto{\pgfqpoint{1.657187in}{1.659951in}}{\pgfqpoint{1.661577in}{1.670550in}}{\pgfqpoint{1.661577in}{1.681600in}}%
\pgfpathcurveto{\pgfqpoint{1.661577in}{1.692650in}}{\pgfqpoint{1.657187in}{1.703249in}}{\pgfqpoint{1.649373in}{1.711063in}}%
\pgfpathcurveto{\pgfqpoint{1.641560in}{1.718877in}}{\pgfqpoint{1.630960in}{1.723267in}}{\pgfqpoint{1.619910in}{1.723267in}}%
\pgfpathcurveto{\pgfqpoint{1.608860in}{1.723267in}}{\pgfqpoint{1.598261in}{1.718877in}}{\pgfqpoint{1.590448in}{1.711063in}}%
\pgfpathcurveto{\pgfqpoint{1.582634in}{1.703249in}}{\pgfqpoint{1.578244in}{1.692650in}}{\pgfqpoint{1.578244in}{1.681600in}}%
\pgfpathcurveto{\pgfqpoint{1.578244in}{1.670550in}}{\pgfqpoint{1.582634in}{1.659951in}}{\pgfqpoint{1.590448in}{1.652138in}}%
\pgfpathcurveto{\pgfqpoint{1.598261in}{1.644324in}}{\pgfqpoint{1.608860in}{1.639934in}}{\pgfqpoint{1.619910in}{1.639934in}}%
\pgfpathclose%
\pgfusepath{stroke,fill}%
\end{pgfscope}%
\begin{pgfscope}%
\pgfpathrectangle{\pgfqpoint{0.787074in}{0.548769in}}{\pgfqpoint{5.062926in}{3.102590in}}%
\pgfusepath{clip}%
\pgfsetbuttcap%
\pgfsetroundjoin%
\definecolor{currentfill}{rgb}{1.000000,0.498039,0.054902}%
\pgfsetfillcolor{currentfill}%
\pgfsetlinewidth{1.003750pt}%
\definecolor{currentstroke}{rgb}{1.000000,0.498039,0.054902}%
\pgfsetstrokecolor{currentstroke}%
\pgfsetdash{}{0pt}%
\pgfpathmoveto{\pgfqpoint{2.102290in}{1.770785in}}%
\pgfpathcurveto{\pgfqpoint{2.113340in}{1.770785in}}{\pgfqpoint{2.123939in}{1.775176in}}{\pgfqpoint{2.131753in}{1.782989in}}%
\pgfpathcurveto{\pgfqpoint{2.139566in}{1.790803in}}{\pgfqpoint{2.143957in}{1.801402in}}{\pgfqpoint{2.143957in}{1.812452in}}%
\pgfpathcurveto{\pgfqpoint{2.143957in}{1.823502in}}{\pgfqpoint{2.139566in}{1.834101in}}{\pgfqpoint{2.131753in}{1.841915in}}%
\pgfpathcurveto{\pgfqpoint{2.123939in}{1.849728in}}{\pgfqpoint{2.113340in}{1.854119in}}{\pgfqpoint{2.102290in}{1.854119in}}%
\pgfpathcurveto{\pgfqpoint{2.091240in}{1.854119in}}{\pgfqpoint{2.080641in}{1.849728in}}{\pgfqpoint{2.072827in}{1.841915in}}%
\pgfpathcurveto{\pgfqpoint{2.065014in}{1.834101in}}{\pgfqpoint{2.060623in}{1.823502in}}{\pgfqpoint{2.060623in}{1.812452in}}%
\pgfpathcurveto{\pgfqpoint{2.060623in}{1.801402in}}{\pgfqpoint{2.065014in}{1.790803in}}{\pgfqpoint{2.072827in}{1.782989in}}%
\pgfpathcurveto{\pgfqpoint{2.080641in}{1.775176in}}{\pgfqpoint{2.091240in}{1.770785in}}{\pgfqpoint{2.102290in}{1.770785in}}%
\pgfpathclose%
\pgfusepath{stroke,fill}%
\end{pgfscope}%
\begin{pgfscope}%
\pgfpathrectangle{\pgfqpoint{0.787074in}{0.548769in}}{\pgfqpoint{5.062926in}{3.102590in}}%
\pgfusepath{clip}%
\pgfsetbuttcap%
\pgfsetroundjoin%
\definecolor{currentfill}{rgb}{1.000000,0.498039,0.054902}%
\pgfsetfillcolor{currentfill}%
\pgfsetlinewidth{1.003750pt}%
\definecolor{currentstroke}{rgb}{1.000000,0.498039,0.054902}%
\pgfsetstrokecolor{currentstroke}%
\pgfsetdash{}{0pt}%
\pgfpathmoveto{\pgfqpoint{1.769120in}{2.028373in}}%
\pgfpathcurveto{\pgfqpoint{1.780170in}{2.028373in}}{\pgfqpoint{1.790769in}{2.032763in}}{\pgfqpoint{1.798582in}{2.040577in}}%
\pgfpathcurveto{\pgfqpoint{1.806396in}{2.048390in}}{\pgfqpoint{1.810786in}{2.058989in}}{\pgfqpoint{1.810786in}{2.070039in}}%
\pgfpathcurveto{\pgfqpoint{1.810786in}{2.081090in}}{\pgfqpoint{1.806396in}{2.091689in}}{\pgfqpoint{1.798582in}{2.099502in}}%
\pgfpathcurveto{\pgfqpoint{1.790769in}{2.107316in}}{\pgfqpoint{1.780170in}{2.111706in}}{\pgfqpoint{1.769120in}{2.111706in}}%
\pgfpathcurveto{\pgfqpoint{1.758069in}{2.111706in}}{\pgfqpoint{1.747470in}{2.107316in}}{\pgfqpoint{1.739657in}{2.099502in}}%
\pgfpathcurveto{\pgfqpoint{1.731843in}{2.091689in}}{\pgfqpoint{1.727453in}{2.081090in}}{\pgfqpoint{1.727453in}{2.070039in}}%
\pgfpathcurveto{\pgfqpoint{1.727453in}{2.058989in}}{\pgfqpoint{1.731843in}{2.048390in}}{\pgfqpoint{1.739657in}{2.040577in}}%
\pgfpathcurveto{\pgfqpoint{1.747470in}{2.032763in}}{\pgfqpoint{1.758069in}{2.028373in}}{\pgfqpoint{1.769120in}{2.028373in}}%
\pgfpathclose%
\pgfusepath{stroke,fill}%
\end{pgfscope}%
\begin{pgfscope}%
\pgfpathrectangle{\pgfqpoint{0.787074in}{0.548769in}}{\pgfqpoint{5.062926in}{3.102590in}}%
\pgfusepath{clip}%
\pgfsetbuttcap%
\pgfsetroundjoin%
\definecolor{currentfill}{rgb}{1.000000,0.498039,0.054902}%
\pgfsetfillcolor{currentfill}%
\pgfsetlinewidth{1.003750pt}%
\definecolor{currentstroke}{rgb}{1.000000,0.498039,0.054902}%
\pgfsetstrokecolor{currentstroke}%
\pgfsetdash{}{0pt}%
\pgfpathmoveto{\pgfqpoint{1.664509in}{2.904196in}}%
\pgfpathcurveto{\pgfqpoint{1.675559in}{2.904196in}}{\pgfqpoint{1.686158in}{2.908586in}}{\pgfqpoint{1.693972in}{2.916399in}}%
\pgfpathcurveto{\pgfqpoint{1.701786in}{2.924213in}}{\pgfqpoint{1.706176in}{2.934812in}}{\pgfqpoint{1.706176in}{2.945862in}}%
\pgfpathcurveto{\pgfqpoint{1.706176in}{2.956912in}}{\pgfqpoint{1.701786in}{2.967511in}}{\pgfqpoint{1.693972in}{2.975325in}}%
\pgfpathcurveto{\pgfqpoint{1.686158in}{2.983139in}}{\pgfqpoint{1.675559in}{2.987529in}}{\pgfqpoint{1.664509in}{2.987529in}}%
\pgfpathcurveto{\pgfqpoint{1.653459in}{2.987529in}}{\pgfqpoint{1.642860in}{2.983139in}}{\pgfqpoint{1.635046in}{2.975325in}}%
\pgfpathcurveto{\pgfqpoint{1.627233in}{2.967511in}}{\pgfqpoint{1.622842in}{2.956912in}}{\pgfqpoint{1.622842in}{2.945862in}}%
\pgfpathcurveto{\pgfqpoint{1.622842in}{2.934812in}}{\pgfqpoint{1.627233in}{2.924213in}}{\pgfqpoint{1.635046in}{2.916399in}}%
\pgfpathcurveto{\pgfqpoint{1.642860in}{2.908586in}}{\pgfqpoint{1.653459in}{2.904196in}}{\pgfqpoint{1.664509in}{2.904196in}}%
\pgfpathclose%
\pgfusepath{stroke,fill}%
\end{pgfscope}%
\begin{pgfscope}%
\pgfpathrectangle{\pgfqpoint{0.787074in}{0.548769in}}{\pgfqpoint{5.062926in}{3.102590in}}%
\pgfusepath{clip}%
\pgfsetbuttcap%
\pgfsetroundjoin%
\definecolor{currentfill}{rgb}{1.000000,0.498039,0.054902}%
\pgfsetfillcolor{currentfill}%
\pgfsetlinewidth{1.003750pt}%
\definecolor{currentstroke}{rgb}{1.000000,0.498039,0.054902}%
\pgfsetstrokecolor{currentstroke}%
\pgfsetdash{}{0pt}%
\pgfpathmoveto{\pgfqpoint{1.354747in}{3.239789in}}%
\pgfpathcurveto{\pgfqpoint{1.365798in}{3.239789in}}{\pgfqpoint{1.376397in}{3.244179in}}{\pgfqpoint{1.384210in}{3.251992in}}%
\pgfpathcurveto{\pgfqpoint{1.392024in}{3.259806in}}{\pgfqpoint{1.396414in}{3.270405in}}{\pgfqpoint{1.396414in}{3.281455in}}%
\pgfpathcurveto{\pgfqpoint{1.396414in}{3.292505in}}{\pgfqpoint{1.392024in}{3.303104in}}{\pgfqpoint{1.384210in}{3.310918in}}%
\pgfpathcurveto{\pgfqpoint{1.376397in}{3.318732in}}{\pgfqpoint{1.365798in}{3.323122in}}{\pgfqpoint{1.354747in}{3.323122in}}%
\pgfpathcurveto{\pgfqpoint{1.343697in}{3.323122in}}{\pgfqpoint{1.333098in}{3.318732in}}{\pgfqpoint{1.325285in}{3.310918in}}%
\pgfpathcurveto{\pgfqpoint{1.317471in}{3.303104in}}{\pgfqpoint{1.313081in}{3.292505in}}{\pgfqpoint{1.313081in}{3.281455in}}%
\pgfpathcurveto{\pgfqpoint{1.313081in}{3.270405in}}{\pgfqpoint{1.317471in}{3.259806in}}{\pgfqpoint{1.325285in}{3.251992in}}%
\pgfpathcurveto{\pgfqpoint{1.333098in}{3.244179in}}{\pgfqpoint{1.343697in}{3.239789in}}{\pgfqpoint{1.354747in}{3.239789in}}%
\pgfpathclose%
\pgfusepath{stroke,fill}%
\end{pgfscope}%
\begin{pgfscope}%
\pgfsetbuttcap%
\pgfsetroundjoin%
\definecolor{currentfill}{rgb}{0.000000,0.000000,0.000000}%
\pgfsetfillcolor{currentfill}%
\pgfsetlinewidth{0.803000pt}%
\definecolor{currentstroke}{rgb}{0.000000,0.000000,0.000000}%
\pgfsetstrokecolor{currentstroke}%
\pgfsetdash{}{0pt}%
\pgfsys@defobject{currentmarker}{\pgfqpoint{0.000000in}{-0.048611in}}{\pgfqpoint{0.000000in}{0.000000in}}{%
\pgfpathmoveto{\pgfqpoint{0.000000in}{0.000000in}}%
\pgfpathlineto{\pgfqpoint{0.000000in}{-0.048611in}}%
\pgfusepath{stroke,fill}%
}%
\begin{pgfscope}%
\pgfsys@transformshift{0.969511in}{0.548769in}%
\pgfsys@useobject{currentmarker}{}%
\end{pgfscope}%
\end{pgfscope}%
\begin{pgfscope}%
\definecolor{textcolor}{rgb}{0.000000,0.000000,0.000000}%
\pgfsetstrokecolor{textcolor}%
\pgfsetfillcolor{textcolor}%
\pgftext[x=0.969511in,y=0.451547in,,top]{\color{textcolor}\sffamily\fontsize{10.000000}{12.000000}\selectfont \(\displaystyle {-1}\)}%
\end{pgfscope}%
\begin{pgfscope}%
\pgfsetbuttcap%
\pgfsetroundjoin%
\definecolor{currentfill}{rgb}{0.000000,0.000000,0.000000}%
\pgfsetfillcolor{currentfill}%
\pgfsetlinewidth{0.803000pt}%
\definecolor{currentstroke}{rgb}{0.000000,0.000000,0.000000}%
\pgfsetstrokecolor{currentstroke}%
\pgfsetdash{}{0pt}%
\pgfsys@defobject{currentmarker}{\pgfqpoint{0.000000in}{-0.048611in}}{\pgfqpoint{0.000000in}{0.000000in}}{%
\pgfpathmoveto{\pgfqpoint{0.000000in}{0.000000in}}%
\pgfpathlineto{\pgfqpoint{0.000000in}{-0.048611in}}%
\pgfusepath{stroke,fill}%
}%
\begin{pgfscope}%
\pgfsys@transformshift{1.684950in}{0.548769in}%
\pgfsys@useobject{currentmarker}{}%
\end{pgfscope}%
\end{pgfscope}%
\begin{pgfscope}%
\definecolor{textcolor}{rgb}{0.000000,0.000000,0.000000}%
\pgfsetstrokecolor{textcolor}%
\pgfsetfillcolor{textcolor}%
\pgftext[x=1.684950in,y=0.451547in,,top]{\color{textcolor}\sffamily\fontsize{10.000000}{12.000000}\selectfont \(\displaystyle {0}\)}%
\end{pgfscope}%
\begin{pgfscope}%
\pgfsetbuttcap%
\pgfsetroundjoin%
\definecolor{currentfill}{rgb}{0.000000,0.000000,0.000000}%
\pgfsetfillcolor{currentfill}%
\pgfsetlinewidth{0.803000pt}%
\definecolor{currentstroke}{rgb}{0.000000,0.000000,0.000000}%
\pgfsetstrokecolor{currentstroke}%
\pgfsetdash{}{0pt}%
\pgfsys@defobject{currentmarker}{\pgfqpoint{0.000000in}{-0.048611in}}{\pgfqpoint{0.000000in}{0.000000in}}{%
\pgfpathmoveto{\pgfqpoint{0.000000in}{0.000000in}}%
\pgfpathlineto{\pgfqpoint{0.000000in}{-0.048611in}}%
\pgfusepath{stroke,fill}%
}%
\begin{pgfscope}%
\pgfsys@transformshift{2.400390in}{0.548769in}%
\pgfsys@useobject{currentmarker}{}%
\end{pgfscope}%
\end{pgfscope}%
\begin{pgfscope}%
\definecolor{textcolor}{rgb}{0.000000,0.000000,0.000000}%
\pgfsetstrokecolor{textcolor}%
\pgfsetfillcolor{textcolor}%
\pgftext[x=2.400390in,y=0.451547in,,top]{\color{textcolor}\sffamily\fontsize{10.000000}{12.000000}\selectfont \(\displaystyle {1}\)}%
\end{pgfscope}%
\begin{pgfscope}%
\pgfsetbuttcap%
\pgfsetroundjoin%
\definecolor{currentfill}{rgb}{0.000000,0.000000,0.000000}%
\pgfsetfillcolor{currentfill}%
\pgfsetlinewidth{0.803000pt}%
\definecolor{currentstroke}{rgb}{0.000000,0.000000,0.000000}%
\pgfsetstrokecolor{currentstroke}%
\pgfsetdash{}{0pt}%
\pgfsys@defobject{currentmarker}{\pgfqpoint{0.000000in}{-0.048611in}}{\pgfqpoint{0.000000in}{0.000000in}}{%
\pgfpathmoveto{\pgfqpoint{0.000000in}{0.000000in}}%
\pgfpathlineto{\pgfqpoint{0.000000in}{-0.048611in}}%
\pgfusepath{stroke,fill}%
}%
\begin{pgfscope}%
\pgfsys@transformshift{3.115829in}{0.548769in}%
\pgfsys@useobject{currentmarker}{}%
\end{pgfscope}%
\end{pgfscope}%
\begin{pgfscope}%
\definecolor{textcolor}{rgb}{0.000000,0.000000,0.000000}%
\pgfsetstrokecolor{textcolor}%
\pgfsetfillcolor{textcolor}%
\pgftext[x=3.115829in,y=0.451547in,,top]{\color{textcolor}\sffamily\fontsize{10.000000}{12.000000}\selectfont \(\displaystyle {2}\)}%
\end{pgfscope}%
\begin{pgfscope}%
\pgfsetbuttcap%
\pgfsetroundjoin%
\definecolor{currentfill}{rgb}{0.000000,0.000000,0.000000}%
\pgfsetfillcolor{currentfill}%
\pgfsetlinewidth{0.803000pt}%
\definecolor{currentstroke}{rgb}{0.000000,0.000000,0.000000}%
\pgfsetstrokecolor{currentstroke}%
\pgfsetdash{}{0pt}%
\pgfsys@defobject{currentmarker}{\pgfqpoint{0.000000in}{-0.048611in}}{\pgfqpoint{0.000000in}{0.000000in}}{%
\pgfpathmoveto{\pgfqpoint{0.000000in}{0.000000in}}%
\pgfpathlineto{\pgfqpoint{0.000000in}{-0.048611in}}%
\pgfusepath{stroke,fill}%
}%
\begin{pgfscope}%
\pgfsys@transformshift{3.831268in}{0.548769in}%
\pgfsys@useobject{currentmarker}{}%
\end{pgfscope}%
\end{pgfscope}%
\begin{pgfscope}%
\definecolor{textcolor}{rgb}{0.000000,0.000000,0.000000}%
\pgfsetstrokecolor{textcolor}%
\pgfsetfillcolor{textcolor}%
\pgftext[x=3.831268in,y=0.451547in,,top]{\color{textcolor}\sffamily\fontsize{10.000000}{12.000000}\selectfont \(\displaystyle {3}\)}%
\end{pgfscope}%
\begin{pgfscope}%
\pgfsetbuttcap%
\pgfsetroundjoin%
\definecolor{currentfill}{rgb}{0.000000,0.000000,0.000000}%
\pgfsetfillcolor{currentfill}%
\pgfsetlinewidth{0.803000pt}%
\definecolor{currentstroke}{rgb}{0.000000,0.000000,0.000000}%
\pgfsetstrokecolor{currentstroke}%
\pgfsetdash{}{0pt}%
\pgfsys@defobject{currentmarker}{\pgfqpoint{0.000000in}{-0.048611in}}{\pgfqpoint{0.000000in}{0.000000in}}{%
\pgfpathmoveto{\pgfqpoint{0.000000in}{0.000000in}}%
\pgfpathlineto{\pgfqpoint{0.000000in}{-0.048611in}}%
\pgfusepath{stroke,fill}%
}%
\begin{pgfscope}%
\pgfsys@transformshift{4.546708in}{0.548769in}%
\pgfsys@useobject{currentmarker}{}%
\end{pgfscope}%
\end{pgfscope}%
\begin{pgfscope}%
\definecolor{textcolor}{rgb}{0.000000,0.000000,0.000000}%
\pgfsetstrokecolor{textcolor}%
\pgfsetfillcolor{textcolor}%
\pgftext[x=4.546708in,y=0.451547in,,top]{\color{textcolor}\sffamily\fontsize{10.000000}{12.000000}\selectfont \(\displaystyle {4}\)}%
\end{pgfscope}%
\begin{pgfscope}%
\pgfsetbuttcap%
\pgfsetroundjoin%
\definecolor{currentfill}{rgb}{0.000000,0.000000,0.000000}%
\pgfsetfillcolor{currentfill}%
\pgfsetlinewidth{0.803000pt}%
\definecolor{currentstroke}{rgb}{0.000000,0.000000,0.000000}%
\pgfsetstrokecolor{currentstroke}%
\pgfsetdash{}{0pt}%
\pgfsys@defobject{currentmarker}{\pgfqpoint{0.000000in}{-0.048611in}}{\pgfqpoint{0.000000in}{0.000000in}}{%
\pgfpathmoveto{\pgfqpoint{0.000000in}{0.000000in}}%
\pgfpathlineto{\pgfqpoint{0.000000in}{-0.048611in}}%
\pgfusepath{stroke,fill}%
}%
\begin{pgfscope}%
\pgfsys@transformshift{5.262147in}{0.548769in}%
\pgfsys@useobject{currentmarker}{}%
\end{pgfscope}%
\end{pgfscope}%
\begin{pgfscope}%
\definecolor{textcolor}{rgb}{0.000000,0.000000,0.000000}%
\pgfsetstrokecolor{textcolor}%
\pgfsetfillcolor{textcolor}%
\pgftext[x=5.262147in,y=0.451547in,,top]{\color{textcolor}\sffamily\fontsize{10.000000}{12.000000}\selectfont \(\displaystyle {5}\)}%
\end{pgfscope}%
\begin{pgfscope}%
\definecolor{textcolor}{rgb}{0.000000,0.000000,0.000000}%
\pgfsetstrokecolor{textcolor}%
\pgfsetfillcolor{textcolor}%
\pgftext[x=3.318537in,y=0.272658in,,top]{\color{textcolor}\sffamily\fontsize{10.000000}{12.000000}\selectfont Ratio of Sources to Sinks}%
\end{pgfscope}%
\begin{pgfscope}%
\pgfsetbuttcap%
\pgfsetroundjoin%
\definecolor{currentfill}{rgb}{0.000000,0.000000,0.000000}%
\pgfsetfillcolor{currentfill}%
\pgfsetlinewidth{0.803000pt}%
\definecolor{currentstroke}{rgb}{0.000000,0.000000,0.000000}%
\pgfsetstrokecolor{currentstroke}%
\pgfsetdash{}{0pt}%
\pgfsys@defobject{currentmarker}{\pgfqpoint{-0.048611in}{0.000000in}}{\pgfqpoint{0.000000in}{0.000000in}}{%
\pgfpathmoveto{\pgfqpoint{0.000000in}{0.000000in}}%
\pgfpathlineto{\pgfqpoint{-0.048611in}{0.000000in}}%
\pgfusepath{stroke,fill}%
}%
\begin{pgfscope}%
\pgfsys@transformshift{0.787074in}{0.689795in}%
\pgfsys@useobject{currentmarker}{}%
\end{pgfscope}%
\end{pgfscope}%
\begin{pgfscope}%
\definecolor{textcolor}{rgb}{0.000000,0.000000,0.000000}%
\pgfsetstrokecolor{textcolor}%
\pgfsetfillcolor{textcolor}%
\pgftext[x=0.620407in, y=0.641601in, left, base]{\color{textcolor}\sffamily\fontsize{10.000000}{12.000000}\selectfont \(\displaystyle {0}\)}%
\end{pgfscope}%
\begin{pgfscope}%
\pgfsetbuttcap%
\pgfsetroundjoin%
\definecolor{currentfill}{rgb}{0.000000,0.000000,0.000000}%
\pgfsetfillcolor{currentfill}%
\pgfsetlinewidth{0.803000pt}%
\definecolor{currentstroke}{rgb}{0.000000,0.000000,0.000000}%
\pgfsetstrokecolor{currentstroke}%
\pgfsetdash{}{0pt}%
\pgfsys@defobject{currentmarker}{\pgfqpoint{-0.048611in}{0.000000in}}{\pgfqpoint{0.000000in}{0.000000in}}{%
\pgfpathmoveto{\pgfqpoint{0.000000in}{0.000000in}}%
\pgfpathlineto{\pgfqpoint{-0.048611in}{0.000000in}}%
\pgfusepath{stroke,fill}%
}%
\begin{pgfscope}%
\pgfsys@transformshift{0.787074in}{1.373761in}%
\pgfsys@useobject{currentmarker}{}%
\end{pgfscope}%
\end{pgfscope}%
\begin{pgfscope}%
\definecolor{textcolor}{rgb}{0.000000,0.000000,0.000000}%
\pgfsetstrokecolor{textcolor}%
\pgfsetfillcolor{textcolor}%
\pgftext[x=0.412073in, y=1.325566in, left, base]{\color{textcolor}\sffamily\fontsize{10.000000}{12.000000}\selectfont \(\displaystyle {5000}\)}%
\end{pgfscope}%
\begin{pgfscope}%
\pgfsetbuttcap%
\pgfsetroundjoin%
\definecolor{currentfill}{rgb}{0.000000,0.000000,0.000000}%
\pgfsetfillcolor{currentfill}%
\pgfsetlinewidth{0.803000pt}%
\definecolor{currentstroke}{rgb}{0.000000,0.000000,0.000000}%
\pgfsetstrokecolor{currentstroke}%
\pgfsetdash{}{0pt}%
\pgfsys@defobject{currentmarker}{\pgfqpoint{-0.048611in}{0.000000in}}{\pgfqpoint{0.000000in}{0.000000in}}{%
\pgfpathmoveto{\pgfqpoint{0.000000in}{0.000000in}}%
\pgfpathlineto{\pgfqpoint{-0.048611in}{0.000000in}}%
\pgfusepath{stroke,fill}%
}%
\begin{pgfscope}%
\pgfsys@transformshift{0.787074in}{2.057726in}%
\pgfsys@useobject{currentmarker}{}%
\end{pgfscope}%
\end{pgfscope}%
\begin{pgfscope}%
\definecolor{textcolor}{rgb}{0.000000,0.000000,0.000000}%
\pgfsetstrokecolor{textcolor}%
\pgfsetfillcolor{textcolor}%
\pgftext[x=0.342628in, y=2.009532in, left, base]{\color{textcolor}\sffamily\fontsize{10.000000}{12.000000}\selectfont \(\displaystyle {10000}\)}%
\end{pgfscope}%
\begin{pgfscope}%
\pgfsetbuttcap%
\pgfsetroundjoin%
\definecolor{currentfill}{rgb}{0.000000,0.000000,0.000000}%
\pgfsetfillcolor{currentfill}%
\pgfsetlinewidth{0.803000pt}%
\definecolor{currentstroke}{rgb}{0.000000,0.000000,0.000000}%
\pgfsetstrokecolor{currentstroke}%
\pgfsetdash{}{0pt}%
\pgfsys@defobject{currentmarker}{\pgfqpoint{-0.048611in}{0.000000in}}{\pgfqpoint{0.000000in}{0.000000in}}{%
\pgfpathmoveto{\pgfqpoint{0.000000in}{0.000000in}}%
\pgfpathlineto{\pgfqpoint{-0.048611in}{0.000000in}}%
\pgfusepath{stroke,fill}%
}%
\begin{pgfscope}%
\pgfsys@transformshift{0.787074in}{2.741692in}%
\pgfsys@useobject{currentmarker}{}%
\end{pgfscope}%
\end{pgfscope}%
\begin{pgfscope}%
\definecolor{textcolor}{rgb}{0.000000,0.000000,0.000000}%
\pgfsetstrokecolor{textcolor}%
\pgfsetfillcolor{textcolor}%
\pgftext[x=0.342628in, y=2.693498in, left, base]{\color{textcolor}\sffamily\fontsize{10.000000}{12.000000}\selectfont \(\displaystyle {15000}\)}%
\end{pgfscope}%
\begin{pgfscope}%
\pgfsetbuttcap%
\pgfsetroundjoin%
\definecolor{currentfill}{rgb}{0.000000,0.000000,0.000000}%
\pgfsetfillcolor{currentfill}%
\pgfsetlinewidth{0.803000pt}%
\definecolor{currentstroke}{rgb}{0.000000,0.000000,0.000000}%
\pgfsetstrokecolor{currentstroke}%
\pgfsetdash{}{0pt}%
\pgfsys@defobject{currentmarker}{\pgfqpoint{-0.048611in}{0.000000in}}{\pgfqpoint{0.000000in}{0.000000in}}{%
\pgfpathmoveto{\pgfqpoint{0.000000in}{0.000000in}}%
\pgfpathlineto{\pgfqpoint{-0.048611in}{0.000000in}}%
\pgfusepath{stroke,fill}%
}%
\begin{pgfscope}%
\pgfsys@transformshift{0.787074in}{3.425658in}%
\pgfsys@useobject{currentmarker}{}%
\end{pgfscope}%
\end{pgfscope}%
\begin{pgfscope}%
\definecolor{textcolor}{rgb}{0.000000,0.000000,0.000000}%
\pgfsetstrokecolor{textcolor}%
\pgfsetfillcolor{textcolor}%
\pgftext[x=0.342628in, y=3.377463in, left, base]{\color{textcolor}\sffamily\fontsize{10.000000}{12.000000}\selectfont \(\displaystyle {20000}\)}%
\end{pgfscope}%
\begin{pgfscope}%
\definecolor{textcolor}{rgb}{0.000000,0.000000,0.000000}%
\pgfsetstrokecolor{textcolor}%
\pgfsetfillcolor{textcolor}%
\pgftext[x=0.287073in,y=2.100064in,,bottom,rotate=90.000000]{\color{textcolor}\sffamily\fontsize{10.000000}{12.000000}\selectfont Maximum Memory Consumption (MB)}%
\end{pgfscope}%
\begin{pgfscope}%
\pgfsetrectcap%
\pgfsetmiterjoin%
\pgfsetlinewidth{0.803000pt}%
\definecolor{currentstroke}{rgb}{0.000000,0.000000,0.000000}%
\pgfsetstrokecolor{currentstroke}%
\pgfsetdash{}{0pt}%
\pgfpathmoveto{\pgfqpoint{0.787074in}{0.548769in}}%
\pgfpathlineto{\pgfqpoint{0.787074in}{3.651359in}}%
\pgfusepath{stroke}%
\end{pgfscope}%
\begin{pgfscope}%
\pgfsetrectcap%
\pgfsetmiterjoin%
\pgfsetlinewidth{0.803000pt}%
\definecolor{currentstroke}{rgb}{0.000000,0.000000,0.000000}%
\pgfsetstrokecolor{currentstroke}%
\pgfsetdash{}{0pt}%
\pgfpathmoveto{\pgfqpoint{5.850000in}{0.548769in}}%
\pgfpathlineto{\pgfqpoint{5.850000in}{3.651359in}}%
\pgfusepath{stroke}%
\end{pgfscope}%
\begin{pgfscope}%
\pgfsetrectcap%
\pgfsetmiterjoin%
\pgfsetlinewidth{0.803000pt}%
\definecolor{currentstroke}{rgb}{0.000000,0.000000,0.000000}%
\pgfsetstrokecolor{currentstroke}%
\pgfsetdash{}{0pt}%
\pgfpathmoveto{\pgfqpoint{0.787074in}{0.548769in}}%
\pgfpathlineto{\pgfqpoint{5.850000in}{0.548769in}}%
\pgfusepath{stroke}%
\end{pgfscope}%
\begin{pgfscope}%
\pgfsetrectcap%
\pgfsetmiterjoin%
\pgfsetlinewidth{0.803000pt}%
\definecolor{currentstroke}{rgb}{0.000000,0.000000,0.000000}%
\pgfsetstrokecolor{currentstroke}%
\pgfsetdash{}{0pt}%
\pgfpathmoveto{\pgfqpoint{0.787074in}{3.651359in}}%
\pgfpathlineto{\pgfqpoint{5.850000in}{3.651359in}}%
\pgfusepath{stroke}%
\end{pgfscope}%
\begin{pgfscope}%
\definecolor{textcolor}{rgb}{0.000000,0.000000,0.000000}%
\pgfsetstrokecolor{textcolor}%
\pgfsetfillcolor{textcolor}%
\pgftext[x=3.318537in,y=3.734692in,,base]{\color{textcolor}\sffamily\fontsize{12.000000}{14.400000}\selectfont Forward}%
\end{pgfscope}%
\begin{pgfscope}%
\pgfsetbuttcap%
\pgfsetmiterjoin%
\definecolor{currentfill}{rgb}{1.000000,1.000000,1.000000}%
\pgfsetfillcolor{currentfill}%
\pgfsetfillopacity{0.800000}%
\pgfsetlinewidth{1.003750pt}%
\definecolor{currentstroke}{rgb}{0.800000,0.800000,0.800000}%
\pgfsetstrokecolor{currentstroke}%
\pgfsetstrokeopacity{0.800000}%
\pgfsetdash{}{0pt}%
\pgfpathmoveto{\pgfqpoint{4.300417in}{2.957886in}}%
\pgfpathlineto{\pgfqpoint{5.752778in}{2.957886in}}%
\pgfpathquadraticcurveto{\pgfqpoint{5.780556in}{2.957886in}}{\pgfqpoint{5.780556in}{2.985664in}}%
\pgfpathlineto{\pgfqpoint{5.780556in}{3.554136in}}%
\pgfpathquadraticcurveto{\pgfqpoint{5.780556in}{3.581914in}}{\pgfqpoint{5.752778in}{3.581914in}}%
\pgfpathlineto{\pgfqpoint{4.300417in}{3.581914in}}%
\pgfpathquadraticcurveto{\pgfqpoint{4.272639in}{3.581914in}}{\pgfqpoint{4.272639in}{3.554136in}}%
\pgfpathlineto{\pgfqpoint{4.272639in}{2.985664in}}%
\pgfpathquadraticcurveto{\pgfqpoint{4.272639in}{2.957886in}}{\pgfqpoint{4.300417in}{2.957886in}}%
\pgfpathclose%
\pgfusepath{stroke,fill}%
\end{pgfscope}%
\begin{pgfscope}%
\pgfsetbuttcap%
\pgfsetroundjoin%
\definecolor{currentfill}{rgb}{0.121569,0.466667,0.705882}%
\pgfsetfillcolor{currentfill}%
\pgfsetlinewidth{1.003750pt}%
\definecolor{currentstroke}{rgb}{0.121569,0.466667,0.705882}%
\pgfsetstrokecolor{currentstroke}%
\pgfsetdash{}{0pt}%
\pgfsys@defobject{currentmarker}{\pgfqpoint{-0.034722in}{-0.034722in}}{\pgfqpoint{0.034722in}{0.034722in}}{%
\pgfpathmoveto{\pgfqpoint{0.000000in}{-0.034722in}}%
\pgfpathcurveto{\pgfqpoint{0.009208in}{-0.034722in}}{\pgfqpoint{0.018041in}{-0.031064in}}{\pgfqpoint{0.024552in}{-0.024552in}}%
\pgfpathcurveto{\pgfqpoint{0.031064in}{-0.018041in}}{\pgfqpoint{0.034722in}{-0.009208in}}{\pgfqpoint{0.034722in}{0.000000in}}%
\pgfpathcurveto{\pgfqpoint{0.034722in}{0.009208in}}{\pgfqpoint{0.031064in}{0.018041in}}{\pgfqpoint{0.024552in}{0.024552in}}%
\pgfpathcurveto{\pgfqpoint{0.018041in}{0.031064in}}{\pgfqpoint{0.009208in}{0.034722in}}{\pgfqpoint{0.000000in}{0.034722in}}%
\pgfpathcurveto{\pgfqpoint{-0.009208in}{0.034722in}}{\pgfqpoint{-0.018041in}{0.031064in}}{\pgfqpoint{-0.024552in}{0.024552in}}%
\pgfpathcurveto{\pgfqpoint{-0.031064in}{0.018041in}}{\pgfqpoint{-0.034722in}{0.009208in}}{\pgfqpoint{-0.034722in}{0.000000in}}%
\pgfpathcurveto{\pgfqpoint{-0.034722in}{-0.009208in}}{\pgfqpoint{-0.031064in}{-0.018041in}}{\pgfqpoint{-0.024552in}{-0.024552in}}%
\pgfpathcurveto{\pgfqpoint{-0.018041in}{-0.031064in}}{\pgfqpoint{-0.009208in}{-0.034722in}}{\pgfqpoint{0.000000in}{-0.034722in}}%
\pgfpathclose%
\pgfusepath{stroke,fill}%
}%
\begin{pgfscope}%
\pgfsys@transformshift{4.467083in}{3.477748in}%
\pgfsys@useobject{currentmarker}{}%
\end{pgfscope}%
\end{pgfscope}%
\begin{pgfscope}%
\definecolor{textcolor}{rgb}{0.000000,0.000000,0.000000}%
\pgfsetstrokecolor{textcolor}%
\pgfsetfillcolor{textcolor}%
\pgftext[x=4.717083in,y=3.429136in,left,base]{\color{textcolor}\sffamily\fontsize{10.000000}{12.000000}\selectfont No Timeout}%
\end{pgfscope}%
\begin{pgfscope}%
\pgfsetbuttcap%
\pgfsetroundjoin%
\definecolor{currentfill}{rgb}{1.000000,0.498039,0.054902}%
\pgfsetfillcolor{currentfill}%
\pgfsetlinewidth{1.003750pt}%
\definecolor{currentstroke}{rgb}{1.000000,0.498039,0.054902}%
\pgfsetstrokecolor{currentstroke}%
\pgfsetdash{}{0pt}%
\pgfsys@defobject{currentmarker}{\pgfqpoint{-0.034722in}{-0.034722in}}{\pgfqpoint{0.034722in}{0.034722in}}{%
\pgfpathmoveto{\pgfqpoint{0.000000in}{-0.034722in}}%
\pgfpathcurveto{\pgfqpoint{0.009208in}{-0.034722in}}{\pgfqpoint{0.018041in}{-0.031064in}}{\pgfqpoint{0.024552in}{-0.024552in}}%
\pgfpathcurveto{\pgfqpoint{0.031064in}{-0.018041in}}{\pgfqpoint{0.034722in}{-0.009208in}}{\pgfqpoint{0.034722in}{0.000000in}}%
\pgfpathcurveto{\pgfqpoint{0.034722in}{0.009208in}}{\pgfqpoint{0.031064in}{0.018041in}}{\pgfqpoint{0.024552in}{0.024552in}}%
\pgfpathcurveto{\pgfqpoint{0.018041in}{0.031064in}}{\pgfqpoint{0.009208in}{0.034722in}}{\pgfqpoint{0.000000in}{0.034722in}}%
\pgfpathcurveto{\pgfqpoint{-0.009208in}{0.034722in}}{\pgfqpoint{-0.018041in}{0.031064in}}{\pgfqpoint{-0.024552in}{0.024552in}}%
\pgfpathcurveto{\pgfqpoint{-0.031064in}{0.018041in}}{\pgfqpoint{-0.034722in}{0.009208in}}{\pgfqpoint{-0.034722in}{0.000000in}}%
\pgfpathcurveto{\pgfqpoint{-0.034722in}{-0.009208in}}{\pgfqpoint{-0.031064in}{-0.018041in}}{\pgfqpoint{-0.024552in}{-0.024552in}}%
\pgfpathcurveto{\pgfqpoint{-0.018041in}{-0.031064in}}{\pgfqpoint{-0.009208in}{-0.034722in}}{\pgfqpoint{0.000000in}{-0.034722in}}%
\pgfpathclose%
\pgfusepath{stroke,fill}%
}%
\begin{pgfscope}%
\pgfsys@transformshift{4.467083in}{3.284136in}%
\pgfsys@useobject{currentmarker}{}%
\end{pgfscope}%
\end{pgfscope}%
\begin{pgfscope}%
\definecolor{textcolor}{rgb}{0.000000,0.000000,0.000000}%
\pgfsetstrokecolor{textcolor}%
\pgfsetfillcolor{textcolor}%
\pgftext[x=4.717083in,y=3.235525in,left,base]{\color{textcolor}\sffamily\fontsize{10.000000}{12.000000}\selectfont Time Timeout}%
\end{pgfscope}%
\begin{pgfscope}%
\pgfsetbuttcap%
\pgfsetroundjoin%
\definecolor{currentfill}{rgb}{0.839216,0.152941,0.156863}%
\pgfsetfillcolor{currentfill}%
\pgfsetlinewidth{1.003750pt}%
\definecolor{currentstroke}{rgb}{0.839216,0.152941,0.156863}%
\pgfsetstrokecolor{currentstroke}%
\pgfsetdash{}{0pt}%
\pgfsys@defobject{currentmarker}{\pgfqpoint{-0.034722in}{-0.034722in}}{\pgfqpoint{0.034722in}{0.034722in}}{%
\pgfpathmoveto{\pgfqpoint{0.000000in}{-0.034722in}}%
\pgfpathcurveto{\pgfqpoint{0.009208in}{-0.034722in}}{\pgfqpoint{0.018041in}{-0.031064in}}{\pgfqpoint{0.024552in}{-0.024552in}}%
\pgfpathcurveto{\pgfqpoint{0.031064in}{-0.018041in}}{\pgfqpoint{0.034722in}{-0.009208in}}{\pgfqpoint{0.034722in}{0.000000in}}%
\pgfpathcurveto{\pgfqpoint{0.034722in}{0.009208in}}{\pgfqpoint{0.031064in}{0.018041in}}{\pgfqpoint{0.024552in}{0.024552in}}%
\pgfpathcurveto{\pgfqpoint{0.018041in}{0.031064in}}{\pgfqpoint{0.009208in}{0.034722in}}{\pgfqpoint{0.000000in}{0.034722in}}%
\pgfpathcurveto{\pgfqpoint{-0.009208in}{0.034722in}}{\pgfqpoint{-0.018041in}{0.031064in}}{\pgfqpoint{-0.024552in}{0.024552in}}%
\pgfpathcurveto{\pgfqpoint{-0.031064in}{0.018041in}}{\pgfqpoint{-0.034722in}{0.009208in}}{\pgfqpoint{-0.034722in}{0.000000in}}%
\pgfpathcurveto{\pgfqpoint{-0.034722in}{-0.009208in}}{\pgfqpoint{-0.031064in}{-0.018041in}}{\pgfqpoint{-0.024552in}{-0.024552in}}%
\pgfpathcurveto{\pgfqpoint{-0.018041in}{-0.031064in}}{\pgfqpoint{-0.009208in}{-0.034722in}}{\pgfqpoint{0.000000in}{-0.034722in}}%
\pgfpathclose%
\pgfusepath{stroke,fill}%
}%
\begin{pgfscope}%
\pgfsys@transformshift{4.467083in}{3.090525in}%
\pgfsys@useobject{currentmarker}{}%
\end{pgfscope}%
\end{pgfscope}%
\begin{pgfscope}%
\definecolor{textcolor}{rgb}{0.000000,0.000000,0.000000}%
\pgfsetstrokecolor{textcolor}%
\pgfsetfillcolor{textcolor}%
\pgftext[x=4.717083in,y=3.041914in,left,base]{\color{textcolor}\sffamily\fontsize{10.000000}{12.000000}\selectfont Memory Timeout}%
\end{pgfscope}%
\end{pgfpicture}%
\makeatother%
\endgroup%

                }
            \end{subfigure}
            \qquad
            \begin{subfigure}[]{0.45\textwidth}
                \centering
                \resizebox{\columnwidth}{!}{
                    %% Creator: Matplotlib, PGF backend
%%
%% To include the figure in your LaTeX document, write
%%   \input{<filename>.pgf}
%%
%% Make sure the required packages are loaded in your preamble
%%   \usepackage{pgf}
%%
%% and, on pdftex
%%   \usepackage[utf8]{inputenc}\DeclareUnicodeCharacter{2212}{-}
%%
%% or, on luatex and xetex
%%   \usepackage{unicode-math}
%%
%% Figures using additional raster images can only be included by \input if
%% they are in the same directory as the main LaTeX file. For loading figures
%% from other directories you can use the `import` package
%%   \usepackage{import}
%%
%% and then include the figures with
%%   \import{<path to file>}{<filename>.pgf}
%%
%% Matplotlib used the following preamble
%%   \usepackage{amsmath}
%%   \usepackage{fontspec}
%%
\begingroup%
\makeatletter%
\begin{pgfpicture}%
\pgfpathrectangle{\pgfpointorigin}{\pgfqpoint{6.000000in}{4.000000in}}%
\pgfusepath{use as bounding box, clip}%
\begin{pgfscope}%
\pgfsetbuttcap%
\pgfsetmiterjoin%
\definecolor{currentfill}{rgb}{1.000000,1.000000,1.000000}%
\pgfsetfillcolor{currentfill}%
\pgfsetlinewidth{0.000000pt}%
\definecolor{currentstroke}{rgb}{1.000000,1.000000,1.000000}%
\pgfsetstrokecolor{currentstroke}%
\pgfsetdash{}{0pt}%
\pgfpathmoveto{\pgfqpoint{0.000000in}{0.000000in}}%
\pgfpathlineto{\pgfqpoint{6.000000in}{0.000000in}}%
\pgfpathlineto{\pgfqpoint{6.000000in}{4.000000in}}%
\pgfpathlineto{\pgfqpoint{0.000000in}{4.000000in}}%
\pgfpathclose%
\pgfusepath{fill}%
\end{pgfscope}%
\begin{pgfscope}%
\pgfsetbuttcap%
\pgfsetmiterjoin%
\definecolor{currentfill}{rgb}{1.000000,1.000000,1.000000}%
\pgfsetfillcolor{currentfill}%
\pgfsetlinewidth{0.000000pt}%
\definecolor{currentstroke}{rgb}{0.000000,0.000000,0.000000}%
\pgfsetstrokecolor{currentstroke}%
\pgfsetstrokeopacity{0.000000}%
\pgfsetdash{}{0pt}%
\pgfpathmoveto{\pgfqpoint{0.787074in}{0.548769in}}%
\pgfpathlineto{\pgfqpoint{5.850000in}{0.548769in}}%
\pgfpathlineto{\pgfqpoint{5.850000in}{3.651359in}}%
\pgfpathlineto{\pgfqpoint{0.787074in}{3.651359in}}%
\pgfpathclose%
\pgfusepath{fill}%
\end{pgfscope}%
\begin{pgfscope}%
\pgfpathrectangle{\pgfqpoint{0.787074in}{0.548769in}}{\pgfqpoint{5.062926in}{3.102590in}}%
\pgfusepath{clip}%
\pgfsetbuttcap%
\pgfsetroundjoin%
\definecolor{currentfill}{rgb}{0.121569,0.466667,0.705882}%
\pgfsetfillcolor{currentfill}%
\pgfsetlinewidth{1.003750pt}%
\definecolor{currentstroke}{rgb}{0.121569,0.466667,0.705882}%
\pgfsetstrokecolor{currentstroke}%
\pgfsetdash{}{0pt}%
\pgfpathmoveto{\pgfqpoint{3.357543in}{0.676730in}}%
\pgfpathcurveto{\pgfqpoint{3.368593in}{0.676730in}}{\pgfqpoint{3.379192in}{0.681121in}}{\pgfqpoint{3.387005in}{0.688934in}}%
\pgfpathcurveto{\pgfqpoint{3.394819in}{0.696748in}}{\pgfqpoint{3.399209in}{0.707347in}}{\pgfqpoint{3.399209in}{0.718397in}}%
\pgfpathcurveto{\pgfqpoint{3.399209in}{0.729447in}}{\pgfqpoint{3.394819in}{0.740046in}}{\pgfqpoint{3.387005in}{0.747860in}}%
\pgfpathcurveto{\pgfqpoint{3.379192in}{0.755673in}}{\pgfqpoint{3.368593in}{0.760064in}}{\pgfqpoint{3.357543in}{0.760064in}}%
\pgfpathcurveto{\pgfqpoint{3.346492in}{0.760064in}}{\pgfqpoint{3.335893in}{0.755673in}}{\pgfqpoint{3.328080in}{0.747860in}}%
\pgfpathcurveto{\pgfqpoint{3.320266in}{0.740046in}}{\pgfqpoint{3.315876in}{0.729447in}}{\pgfqpoint{3.315876in}{0.718397in}}%
\pgfpathcurveto{\pgfqpoint{3.315876in}{0.707347in}}{\pgfqpoint{3.320266in}{0.696748in}}{\pgfqpoint{3.328080in}{0.688934in}}%
\pgfpathcurveto{\pgfqpoint{3.335893in}{0.681121in}}{\pgfqpoint{3.346492in}{0.676730in}}{\pgfqpoint{3.357543in}{0.676730in}}%
\pgfpathclose%
\pgfusepath{stroke,fill}%
\end{pgfscope}%
\begin{pgfscope}%
\pgfpathrectangle{\pgfqpoint{0.787074in}{0.548769in}}{\pgfqpoint{5.062926in}{3.102590in}}%
\pgfusepath{clip}%
\pgfsetbuttcap%
\pgfsetroundjoin%
\definecolor{currentfill}{rgb}{1.000000,0.498039,0.054902}%
\pgfsetfillcolor{currentfill}%
\pgfsetlinewidth{1.003750pt}%
\definecolor{currentstroke}{rgb}{1.000000,0.498039,0.054902}%
\pgfsetstrokecolor{currentstroke}%
\pgfsetdash{}{0pt}%
\pgfpathmoveto{\pgfqpoint{3.162515in}{1.501451in}}%
\pgfpathcurveto{\pgfqpoint{3.173565in}{1.501451in}}{\pgfqpoint{3.184164in}{1.505841in}}{\pgfqpoint{3.191977in}{1.513655in}}%
\pgfpathcurveto{\pgfqpoint{3.199791in}{1.521468in}}{\pgfqpoint{3.204181in}{1.532067in}}{\pgfqpoint{3.204181in}{1.543118in}}%
\pgfpathcurveto{\pgfqpoint{3.204181in}{1.554168in}}{\pgfqpoint{3.199791in}{1.564767in}}{\pgfqpoint{3.191977in}{1.572580in}}%
\pgfpathcurveto{\pgfqpoint{3.184164in}{1.580394in}}{\pgfqpoint{3.173565in}{1.584784in}}{\pgfqpoint{3.162515in}{1.584784in}}%
\pgfpathcurveto{\pgfqpoint{3.151464in}{1.584784in}}{\pgfqpoint{3.140865in}{1.580394in}}{\pgfqpoint{3.133052in}{1.572580in}}%
\pgfpathcurveto{\pgfqpoint{3.125238in}{1.564767in}}{\pgfqpoint{3.120848in}{1.554168in}}{\pgfqpoint{3.120848in}{1.543118in}}%
\pgfpathcurveto{\pgfqpoint{3.120848in}{1.532067in}}{\pgfqpoint{3.125238in}{1.521468in}}{\pgfqpoint{3.133052in}{1.513655in}}%
\pgfpathcurveto{\pgfqpoint{3.140865in}{1.505841in}}{\pgfqpoint{3.151464in}{1.501451in}}{\pgfqpoint{3.162515in}{1.501451in}}%
\pgfpathclose%
\pgfusepath{stroke,fill}%
\end{pgfscope}%
\begin{pgfscope}%
\pgfpathrectangle{\pgfqpoint{0.787074in}{0.548769in}}{\pgfqpoint{5.062926in}{3.102590in}}%
\pgfusepath{clip}%
\pgfsetbuttcap%
\pgfsetroundjoin%
\definecolor{currentfill}{rgb}{1.000000,0.498039,0.054902}%
\pgfsetfillcolor{currentfill}%
\pgfsetlinewidth{1.003750pt}%
\definecolor{currentstroke}{rgb}{1.000000,0.498039,0.054902}%
\pgfsetstrokecolor{currentstroke}%
\pgfsetdash{}{0pt}%
\pgfpathmoveto{\pgfqpoint{3.201520in}{2.589003in}}%
\pgfpathcurveto{\pgfqpoint{3.212570in}{2.589003in}}{\pgfqpoint{3.223169in}{2.593393in}}{\pgfqpoint{3.230983in}{2.601207in}}%
\pgfpathcurveto{\pgfqpoint{3.238797in}{2.609020in}}{\pgfqpoint{3.243187in}{2.619619in}}{\pgfqpoint{3.243187in}{2.630670in}}%
\pgfpathcurveto{\pgfqpoint{3.243187in}{2.641720in}}{\pgfqpoint{3.238797in}{2.652319in}}{\pgfqpoint{3.230983in}{2.660132in}}%
\pgfpathcurveto{\pgfqpoint{3.223169in}{2.667946in}}{\pgfqpoint{3.212570in}{2.672336in}}{\pgfqpoint{3.201520in}{2.672336in}}%
\pgfpathcurveto{\pgfqpoint{3.190470in}{2.672336in}}{\pgfqpoint{3.179871in}{2.667946in}}{\pgfqpoint{3.172057in}{2.660132in}}%
\pgfpathcurveto{\pgfqpoint{3.164244in}{2.652319in}}{\pgfqpoint{3.159853in}{2.641720in}}{\pgfqpoint{3.159853in}{2.630670in}}%
\pgfpathcurveto{\pgfqpoint{3.159853in}{2.619619in}}{\pgfqpoint{3.164244in}{2.609020in}}{\pgfqpoint{3.172057in}{2.601207in}}%
\pgfpathcurveto{\pgfqpoint{3.179871in}{2.593393in}}{\pgfqpoint{3.190470in}{2.589003in}}{\pgfqpoint{3.201520in}{2.589003in}}%
\pgfpathclose%
\pgfusepath{stroke,fill}%
\end{pgfscope}%
\begin{pgfscope}%
\pgfpathrectangle{\pgfqpoint{0.787074in}{0.548769in}}{\pgfqpoint{5.062926in}{3.102590in}}%
\pgfusepath{clip}%
\pgfsetbuttcap%
\pgfsetroundjoin%
\definecolor{currentfill}{rgb}{1.000000,0.498039,0.054902}%
\pgfsetfillcolor{currentfill}%
\pgfsetlinewidth{1.003750pt}%
\definecolor{currentstroke}{rgb}{1.000000,0.498039,0.054902}%
\pgfsetstrokecolor{currentstroke}%
\pgfsetdash{}{0pt}%
\pgfpathmoveto{\pgfqpoint{3.474559in}{1.800918in}}%
\pgfpathcurveto{\pgfqpoint{3.485609in}{1.800918in}}{\pgfqpoint{3.496208in}{1.805309in}}{\pgfqpoint{3.504022in}{1.813122in}}%
\pgfpathcurveto{\pgfqpoint{3.511836in}{1.820936in}}{\pgfqpoint{3.516226in}{1.831535in}}{\pgfqpoint{3.516226in}{1.842585in}}%
\pgfpathcurveto{\pgfqpoint{3.516226in}{1.853635in}}{\pgfqpoint{3.511836in}{1.864234in}}{\pgfqpoint{3.504022in}{1.872048in}}%
\pgfpathcurveto{\pgfqpoint{3.496208in}{1.879862in}}{\pgfqpoint{3.485609in}{1.884252in}}{\pgfqpoint{3.474559in}{1.884252in}}%
\pgfpathcurveto{\pgfqpoint{3.463509in}{1.884252in}}{\pgfqpoint{3.452910in}{1.879862in}}{\pgfqpoint{3.445097in}{1.872048in}}%
\pgfpathcurveto{\pgfqpoint{3.437283in}{1.864234in}}{\pgfqpoint{3.432893in}{1.853635in}}{\pgfqpoint{3.432893in}{1.842585in}}%
\pgfpathcurveto{\pgfqpoint{3.432893in}{1.831535in}}{\pgfqpoint{3.437283in}{1.820936in}}{\pgfqpoint{3.445097in}{1.813122in}}%
\pgfpathcurveto{\pgfqpoint{3.452910in}{1.805309in}}{\pgfqpoint{3.463509in}{1.800918in}}{\pgfqpoint{3.474559in}{1.800918in}}%
\pgfpathclose%
\pgfusepath{stroke,fill}%
\end{pgfscope}%
\begin{pgfscope}%
\pgfpathrectangle{\pgfqpoint{0.787074in}{0.548769in}}{\pgfqpoint{5.062926in}{3.102590in}}%
\pgfusepath{clip}%
\pgfsetbuttcap%
\pgfsetroundjoin%
\definecolor{currentfill}{rgb}{0.121569,0.466667,0.705882}%
\pgfsetfillcolor{currentfill}%
\pgfsetlinewidth{1.003750pt}%
\definecolor{currentstroke}{rgb}{0.121569,0.466667,0.705882}%
\pgfsetstrokecolor{currentstroke}%
\pgfsetdash{}{0pt}%
\pgfpathmoveto{\pgfqpoint{3.084503in}{0.650081in}}%
\pgfpathcurveto{\pgfqpoint{3.095553in}{0.650081in}}{\pgfqpoint{3.106153in}{0.654472in}}{\pgfqpoint{3.113966in}{0.662285in}}%
\pgfpathcurveto{\pgfqpoint{3.121780in}{0.670099in}}{\pgfqpoint{3.126170in}{0.680698in}}{\pgfqpoint{3.126170in}{0.691748in}}%
\pgfpathcurveto{\pgfqpoint{3.126170in}{0.702798in}}{\pgfqpoint{3.121780in}{0.713397in}}{\pgfqpoint{3.113966in}{0.721211in}}%
\pgfpathcurveto{\pgfqpoint{3.106153in}{0.729024in}}{\pgfqpoint{3.095553in}{0.733415in}}{\pgfqpoint{3.084503in}{0.733415in}}%
\pgfpathcurveto{\pgfqpoint{3.073453in}{0.733415in}}{\pgfqpoint{3.062854in}{0.729024in}}{\pgfqpoint{3.055041in}{0.721211in}}%
\pgfpathcurveto{\pgfqpoint{3.047227in}{0.713397in}}{\pgfqpoint{3.042837in}{0.702798in}}{\pgfqpoint{3.042837in}{0.691748in}}%
\pgfpathcurveto{\pgfqpoint{3.042837in}{0.680698in}}{\pgfqpoint{3.047227in}{0.670099in}}{\pgfqpoint{3.055041in}{0.662285in}}%
\pgfpathcurveto{\pgfqpoint{3.062854in}{0.654472in}}{\pgfqpoint{3.073453in}{0.650081in}}{\pgfqpoint{3.084503in}{0.650081in}}%
\pgfpathclose%
\pgfusepath{stroke,fill}%
\end{pgfscope}%
\begin{pgfscope}%
\pgfpathrectangle{\pgfqpoint{0.787074in}{0.548769in}}{\pgfqpoint{5.062926in}{3.102590in}}%
\pgfusepath{clip}%
\pgfsetbuttcap%
\pgfsetroundjoin%
\definecolor{currentfill}{rgb}{1.000000,0.498039,0.054902}%
\pgfsetfillcolor{currentfill}%
\pgfsetlinewidth{1.003750pt}%
\definecolor{currentstroke}{rgb}{1.000000,0.498039,0.054902}%
\pgfsetstrokecolor{currentstroke}%
\pgfsetdash{}{0pt}%
\pgfpathmoveto{\pgfqpoint{4.605722in}{2.499787in}}%
\pgfpathcurveto{\pgfqpoint{4.616772in}{2.499787in}}{\pgfqpoint{4.627371in}{2.504177in}}{\pgfqpoint{4.635184in}{2.511991in}}%
\pgfpathcurveto{\pgfqpoint{4.642998in}{2.519804in}}{\pgfqpoint{4.647388in}{2.530403in}}{\pgfqpoint{4.647388in}{2.541454in}}%
\pgfpathcurveto{\pgfqpoint{4.647388in}{2.552504in}}{\pgfqpoint{4.642998in}{2.563103in}}{\pgfqpoint{4.635184in}{2.570916in}}%
\pgfpathcurveto{\pgfqpoint{4.627371in}{2.578730in}}{\pgfqpoint{4.616772in}{2.583120in}}{\pgfqpoint{4.605722in}{2.583120in}}%
\pgfpathcurveto{\pgfqpoint{4.594671in}{2.583120in}}{\pgfqpoint{4.584072in}{2.578730in}}{\pgfqpoint{4.576259in}{2.570916in}}%
\pgfpathcurveto{\pgfqpoint{4.568445in}{2.563103in}}{\pgfqpoint{4.564055in}{2.552504in}}{\pgfqpoint{4.564055in}{2.541454in}}%
\pgfpathcurveto{\pgfqpoint{4.564055in}{2.530403in}}{\pgfqpoint{4.568445in}{2.519804in}}{\pgfqpoint{4.576259in}{2.511991in}}%
\pgfpathcurveto{\pgfqpoint{4.584072in}{2.504177in}}{\pgfqpoint{4.594671in}{2.499787in}}{\pgfqpoint{4.605722in}{2.499787in}}%
\pgfpathclose%
\pgfusepath{stroke,fill}%
\end{pgfscope}%
\begin{pgfscope}%
\pgfpathrectangle{\pgfqpoint{0.787074in}{0.548769in}}{\pgfqpoint{5.062926in}{3.102590in}}%
\pgfusepath{clip}%
\pgfsetbuttcap%
\pgfsetroundjoin%
\definecolor{currentfill}{rgb}{1.000000,0.498039,0.054902}%
\pgfsetfillcolor{currentfill}%
\pgfsetlinewidth{1.003750pt}%
\definecolor{currentstroke}{rgb}{1.000000,0.498039,0.054902}%
\pgfsetstrokecolor{currentstroke}%
\pgfsetdash{}{0pt}%
\pgfpathmoveto{\pgfqpoint{3.513565in}{2.338512in}}%
\pgfpathcurveto{\pgfqpoint{3.524615in}{2.338512in}}{\pgfqpoint{3.535214in}{2.342903in}}{\pgfqpoint{3.543028in}{2.350716in}}%
\pgfpathcurveto{\pgfqpoint{3.550841in}{2.358530in}}{\pgfqpoint{3.555232in}{2.369129in}}{\pgfqpoint{3.555232in}{2.380179in}}%
\pgfpathcurveto{\pgfqpoint{3.555232in}{2.391229in}}{\pgfqpoint{3.550841in}{2.401828in}}{\pgfqpoint{3.543028in}{2.409642in}}%
\pgfpathcurveto{\pgfqpoint{3.535214in}{2.417455in}}{\pgfqpoint{3.524615in}{2.421846in}}{\pgfqpoint{3.513565in}{2.421846in}}%
\pgfpathcurveto{\pgfqpoint{3.502515in}{2.421846in}}{\pgfqpoint{3.491916in}{2.417455in}}{\pgfqpoint{3.484102in}{2.409642in}}%
\pgfpathcurveto{\pgfqpoint{3.476288in}{2.401828in}}{\pgfqpoint{3.471898in}{2.391229in}}{\pgfqpoint{3.471898in}{2.380179in}}%
\pgfpathcurveto{\pgfqpoint{3.471898in}{2.369129in}}{\pgfqpoint{3.476288in}{2.358530in}}{\pgfqpoint{3.484102in}{2.350716in}}%
\pgfpathcurveto{\pgfqpoint{3.491916in}{2.342903in}}{\pgfqpoint{3.502515in}{2.338512in}}{\pgfqpoint{3.513565in}{2.338512in}}%
\pgfpathclose%
\pgfusepath{stroke,fill}%
\end{pgfscope}%
\begin{pgfscope}%
\pgfpathrectangle{\pgfqpoint{0.787074in}{0.548769in}}{\pgfqpoint{5.062926in}{3.102590in}}%
\pgfusepath{clip}%
\pgfsetbuttcap%
\pgfsetroundjoin%
\definecolor{currentfill}{rgb}{0.121569,0.466667,0.705882}%
\pgfsetfillcolor{currentfill}%
\pgfsetlinewidth{1.003750pt}%
\definecolor{currentstroke}{rgb}{0.121569,0.466667,0.705882}%
\pgfsetstrokecolor{currentstroke}%
\pgfsetdash{}{0pt}%
\pgfpathmoveto{\pgfqpoint{3.747598in}{0.648162in}}%
\pgfpathcurveto{\pgfqpoint{3.758649in}{0.648162in}}{\pgfqpoint{3.769248in}{0.652552in}}{\pgfqpoint{3.777061in}{0.660365in}}%
\pgfpathcurveto{\pgfqpoint{3.784875in}{0.668179in}}{\pgfqpoint{3.789265in}{0.678778in}}{\pgfqpoint{3.789265in}{0.689828in}}%
\pgfpathcurveto{\pgfqpoint{3.789265in}{0.700878in}}{\pgfqpoint{3.784875in}{0.711477in}}{\pgfqpoint{3.777061in}{0.719291in}}%
\pgfpathcurveto{\pgfqpoint{3.769248in}{0.727105in}}{\pgfqpoint{3.758649in}{0.731495in}}{\pgfqpoint{3.747598in}{0.731495in}}%
\pgfpathcurveto{\pgfqpoint{3.736548in}{0.731495in}}{\pgfqpoint{3.725949in}{0.727105in}}{\pgfqpoint{3.718136in}{0.719291in}}%
\pgfpathcurveto{\pgfqpoint{3.710322in}{0.711477in}}{\pgfqpoint{3.705932in}{0.700878in}}{\pgfqpoint{3.705932in}{0.689828in}}%
\pgfpathcurveto{\pgfqpoint{3.705932in}{0.678778in}}{\pgfqpoint{3.710322in}{0.668179in}}{\pgfqpoint{3.718136in}{0.660365in}}%
\pgfpathcurveto{\pgfqpoint{3.725949in}{0.652552in}}{\pgfqpoint{3.736548in}{0.648162in}}{\pgfqpoint{3.747598in}{0.648162in}}%
\pgfpathclose%
\pgfusepath{stroke,fill}%
\end{pgfscope}%
\begin{pgfscope}%
\pgfpathrectangle{\pgfqpoint{0.787074in}{0.548769in}}{\pgfqpoint{5.062926in}{3.102590in}}%
\pgfusepath{clip}%
\pgfsetbuttcap%
\pgfsetroundjoin%
\definecolor{currentfill}{rgb}{0.121569,0.466667,0.705882}%
\pgfsetfillcolor{currentfill}%
\pgfsetlinewidth{1.003750pt}%
\definecolor{currentstroke}{rgb}{0.121569,0.466667,0.705882}%
\pgfsetstrokecolor{currentstroke}%
\pgfsetdash{}{0pt}%
\pgfpathmoveto{\pgfqpoint{3.474559in}{0.840903in}}%
\pgfpathcurveto{\pgfqpoint{3.485609in}{0.840903in}}{\pgfqpoint{3.496208in}{0.845293in}}{\pgfqpoint{3.504022in}{0.853107in}}%
\pgfpathcurveto{\pgfqpoint{3.511836in}{0.860920in}}{\pgfqpoint{3.516226in}{0.871520in}}{\pgfqpoint{3.516226in}{0.882570in}}%
\pgfpathcurveto{\pgfqpoint{3.516226in}{0.893620in}}{\pgfqpoint{3.511836in}{0.904219in}}{\pgfqpoint{3.504022in}{0.912032in}}%
\pgfpathcurveto{\pgfqpoint{3.496208in}{0.919846in}}{\pgfqpoint{3.485609in}{0.924236in}}{\pgfqpoint{3.474559in}{0.924236in}}%
\pgfpathcurveto{\pgfqpoint{3.463509in}{0.924236in}}{\pgfqpoint{3.452910in}{0.919846in}}{\pgfqpoint{3.445097in}{0.912032in}}%
\pgfpathcurveto{\pgfqpoint{3.437283in}{0.904219in}}{\pgfqpoint{3.432893in}{0.893620in}}{\pgfqpoint{3.432893in}{0.882570in}}%
\pgfpathcurveto{\pgfqpoint{3.432893in}{0.871520in}}{\pgfqpoint{3.437283in}{0.860920in}}{\pgfqpoint{3.445097in}{0.853107in}}%
\pgfpathcurveto{\pgfqpoint{3.452910in}{0.845293in}}{\pgfqpoint{3.463509in}{0.840903in}}{\pgfqpoint{3.474559in}{0.840903in}}%
\pgfpathclose%
\pgfusepath{stroke,fill}%
\end{pgfscope}%
\begin{pgfscope}%
\pgfpathrectangle{\pgfqpoint{0.787074in}{0.548769in}}{\pgfqpoint{5.062926in}{3.102590in}}%
\pgfusepath{clip}%
\pgfsetbuttcap%
\pgfsetroundjoin%
\definecolor{currentfill}{rgb}{0.121569,0.466667,0.705882}%
\pgfsetfillcolor{currentfill}%
\pgfsetlinewidth{1.003750pt}%
\definecolor{currentstroke}{rgb}{0.121569,0.466667,0.705882}%
\pgfsetstrokecolor{currentstroke}%
\pgfsetdash{}{0pt}%
\pgfpathmoveto{\pgfqpoint{3.630582in}{0.787071in}}%
\pgfpathcurveto{\pgfqpoint{3.641632in}{0.787071in}}{\pgfqpoint{3.652231in}{0.791461in}}{\pgfqpoint{3.660044in}{0.799275in}}%
\pgfpathcurveto{\pgfqpoint{3.667858in}{0.807089in}}{\pgfqpoint{3.672248in}{0.817688in}}{\pgfqpoint{3.672248in}{0.828738in}}%
\pgfpathcurveto{\pgfqpoint{3.672248in}{0.839788in}}{\pgfqpoint{3.667858in}{0.850387in}}{\pgfqpoint{3.660044in}{0.858201in}}%
\pgfpathcurveto{\pgfqpoint{3.652231in}{0.866014in}}{\pgfqpoint{3.641632in}{0.870404in}}{\pgfqpoint{3.630582in}{0.870404in}}%
\pgfpathcurveto{\pgfqpoint{3.619532in}{0.870404in}}{\pgfqpoint{3.608933in}{0.866014in}}{\pgfqpoint{3.601119in}{0.858201in}}%
\pgfpathcurveto{\pgfqpoint{3.593305in}{0.850387in}}{\pgfqpoint{3.588915in}{0.839788in}}{\pgfqpoint{3.588915in}{0.828738in}}%
\pgfpathcurveto{\pgfqpoint{3.588915in}{0.817688in}}{\pgfqpoint{3.593305in}{0.807089in}}{\pgfqpoint{3.601119in}{0.799275in}}%
\pgfpathcurveto{\pgfqpoint{3.608933in}{0.791461in}}{\pgfqpoint{3.619532in}{0.787071in}}{\pgfqpoint{3.630582in}{0.787071in}}%
\pgfpathclose%
\pgfusepath{stroke,fill}%
\end{pgfscope}%
\begin{pgfscope}%
\pgfpathrectangle{\pgfqpoint{0.787074in}{0.548769in}}{\pgfqpoint{5.062926in}{3.102590in}}%
\pgfusepath{clip}%
\pgfsetbuttcap%
\pgfsetroundjoin%
\definecolor{currentfill}{rgb}{0.121569,0.466667,0.705882}%
\pgfsetfillcolor{currentfill}%
\pgfsetlinewidth{1.003750pt}%
\definecolor{currentstroke}{rgb}{0.121569,0.466667,0.705882}%
\pgfsetstrokecolor{currentstroke}%
\pgfsetdash{}{0pt}%
\pgfpathmoveto{\pgfqpoint{3.279531in}{0.660658in}}%
\pgfpathcurveto{\pgfqpoint{3.290581in}{0.660658in}}{\pgfqpoint{3.301180in}{0.665048in}}{\pgfqpoint{3.308994in}{0.672862in}}%
\pgfpathcurveto{\pgfqpoint{3.316808in}{0.680676in}}{\pgfqpoint{3.321198in}{0.691275in}}{\pgfqpoint{3.321198in}{0.702325in}}%
\pgfpathcurveto{\pgfqpoint{3.321198in}{0.713375in}}{\pgfqpoint{3.316808in}{0.723974in}}{\pgfqpoint{3.308994in}{0.731788in}}%
\pgfpathcurveto{\pgfqpoint{3.301180in}{0.739601in}}{\pgfqpoint{3.290581in}{0.743991in}}{\pgfqpoint{3.279531in}{0.743991in}}%
\pgfpathcurveto{\pgfqpoint{3.268481in}{0.743991in}}{\pgfqpoint{3.257882in}{0.739601in}}{\pgfqpoint{3.250069in}{0.731788in}}%
\pgfpathcurveto{\pgfqpoint{3.242255in}{0.723974in}}{\pgfqpoint{3.237865in}{0.713375in}}{\pgfqpoint{3.237865in}{0.702325in}}%
\pgfpathcurveto{\pgfqpoint{3.237865in}{0.691275in}}{\pgfqpoint{3.242255in}{0.680676in}}{\pgfqpoint{3.250069in}{0.672862in}}%
\pgfpathcurveto{\pgfqpoint{3.257882in}{0.665048in}}{\pgfqpoint{3.268481in}{0.660658in}}{\pgfqpoint{3.279531in}{0.660658in}}%
\pgfpathclose%
\pgfusepath{stroke,fill}%
\end{pgfscope}%
\begin{pgfscope}%
\pgfpathrectangle{\pgfqpoint{0.787074in}{0.548769in}}{\pgfqpoint{5.062926in}{3.102590in}}%
\pgfusepath{clip}%
\pgfsetbuttcap%
\pgfsetroundjoin%
\definecolor{currentfill}{rgb}{0.121569,0.466667,0.705882}%
\pgfsetfillcolor{currentfill}%
\pgfsetlinewidth{1.003750pt}%
\definecolor{currentstroke}{rgb}{0.121569,0.466667,0.705882}%
\pgfsetstrokecolor{currentstroke}%
\pgfsetdash{}{0pt}%
\pgfpathmoveto{\pgfqpoint{2.655442in}{0.648155in}}%
\pgfpathcurveto{\pgfqpoint{2.666492in}{0.648155in}}{\pgfqpoint{2.677091in}{0.652546in}}{\pgfqpoint{2.684905in}{0.660359in}}%
\pgfpathcurveto{\pgfqpoint{2.692718in}{0.668173in}}{\pgfqpoint{2.697108in}{0.678772in}}{\pgfqpoint{2.697108in}{0.689822in}}%
\pgfpathcurveto{\pgfqpoint{2.697108in}{0.700872in}}{\pgfqpoint{2.692718in}{0.711471in}}{\pgfqpoint{2.684905in}{0.719285in}}%
\pgfpathcurveto{\pgfqpoint{2.677091in}{0.727098in}}{\pgfqpoint{2.666492in}{0.731489in}}{\pgfqpoint{2.655442in}{0.731489in}}%
\pgfpathcurveto{\pgfqpoint{2.644392in}{0.731489in}}{\pgfqpoint{2.633793in}{0.727098in}}{\pgfqpoint{2.625979in}{0.719285in}}%
\pgfpathcurveto{\pgfqpoint{2.618165in}{0.711471in}}{\pgfqpoint{2.613775in}{0.700872in}}{\pgfqpoint{2.613775in}{0.689822in}}%
\pgfpathcurveto{\pgfqpoint{2.613775in}{0.678772in}}{\pgfqpoint{2.618165in}{0.668173in}}{\pgfqpoint{2.625979in}{0.660359in}}%
\pgfpathcurveto{\pgfqpoint{2.633793in}{0.652546in}}{\pgfqpoint{2.644392in}{0.648155in}}{\pgfqpoint{2.655442in}{0.648155in}}%
\pgfpathclose%
\pgfusepath{stroke,fill}%
\end{pgfscope}%
\begin{pgfscope}%
\pgfpathrectangle{\pgfqpoint{0.787074in}{0.548769in}}{\pgfqpoint{5.062926in}{3.102590in}}%
\pgfusepath{clip}%
\pgfsetbuttcap%
\pgfsetroundjoin%
\definecolor{currentfill}{rgb}{0.121569,0.466667,0.705882}%
\pgfsetfillcolor{currentfill}%
\pgfsetlinewidth{1.003750pt}%
\definecolor{currentstroke}{rgb}{0.121569,0.466667,0.705882}%
\pgfsetstrokecolor{currentstroke}%
\pgfsetdash{}{0pt}%
\pgfpathmoveto{\pgfqpoint{3.435554in}{0.648148in}}%
\pgfpathcurveto{\pgfqpoint{3.446604in}{0.648148in}}{\pgfqpoint{3.457203in}{0.652539in}}{\pgfqpoint{3.465016in}{0.660352in}}%
\pgfpathcurveto{\pgfqpoint{3.472830in}{0.668166in}}{\pgfqpoint{3.477220in}{0.678765in}}{\pgfqpoint{3.477220in}{0.689815in}}%
\pgfpathcurveto{\pgfqpoint{3.477220in}{0.700865in}}{\pgfqpoint{3.472830in}{0.711464in}}{\pgfqpoint{3.465016in}{0.719278in}}%
\pgfpathcurveto{\pgfqpoint{3.457203in}{0.727091in}}{\pgfqpoint{3.446604in}{0.731482in}}{\pgfqpoint{3.435554in}{0.731482in}}%
\pgfpathcurveto{\pgfqpoint{3.424504in}{0.731482in}}{\pgfqpoint{3.413905in}{0.727091in}}{\pgfqpoint{3.406091in}{0.719278in}}%
\pgfpathcurveto{\pgfqpoint{3.398277in}{0.711464in}}{\pgfqpoint{3.393887in}{0.700865in}}{\pgfqpoint{3.393887in}{0.689815in}}%
\pgfpathcurveto{\pgfqpoint{3.393887in}{0.678765in}}{\pgfqpoint{3.398277in}{0.668166in}}{\pgfqpoint{3.406091in}{0.660352in}}%
\pgfpathcurveto{\pgfqpoint{3.413905in}{0.652539in}}{\pgfqpoint{3.424504in}{0.648148in}}{\pgfqpoint{3.435554in}{0.648148in}}%
\pgfpathclose%
\pgfusepath{stroke,fill}%
\end{pgfscope}%
\begin{pgfscope}%
\pgfpathrectangle{\pgfqpoint{0.787074in}{0.548769in}}{\pgfqpoint{5.062926in}{3.102590in}}%
\pgfusepath{clip}%
\pgfsetbuttcap%
\pgfsetroundjoin%
\definecolor{currentfill}{rgb}{1.000000,0.498039,0.054902}%
\pgfsetfillcolor{currentfill}%
\pgfsetlinewidth{1.003750pt}%
\definecolor{currentstroke}{rgb}{1.000000,0.498039,0.054902}%
\pgfsetstrokecolor{currentstroke}%
\pgfsetdash{}{0pt}%
\pgfpathmoveto{\pgfqpoint{3.318537in}{1.766469in}}%
\pgfpathcurveto{\pgfqpoint{3.329587in}{1.766469in}}{\pgfqpoint{3.340186in}{1.770859in}}{\pgfqpoint{3.348000in}{1.778672in}}%
\pgfpathcurveto{\pgfqpoint{3.355813in}{1.786486in}}{\pgfqpoint{3.360204in}{1.797085in}}{\pgfqpoint{3.360204in}{1.808135in}}%
\pgfpathcurveto{\pgfqpoint{3.360204in}{1.819185in}}{\pgfqpoint{3.355813in}{1.829784in}}{\pgfqpoint{3.348000in}{1.837598in}}%
\pgfpathcurveto{\pgfqpoint{3.340186in}{1.845412in}}{\pgfqpoint{3.329587in}{1.849802in}}{\pgfqpoint{3.318537in}{1.849802in}}%
\pgfpathcurveto{\pgfqpoint{3.307487in}{1.849802in}}{\pgfqpoint{3.296888in}{1.845412in}}{\pgfqpoint{3.289074in}{1.837598in}}%
\pgfpathcurveto{\pgfqpoint{3.281261in}{1.829784in}}{\pgfqpoint{3.276870in}{1.819185in}}{\pgfqpoint{3.276870in}{1.808135in}}%
\pgfpathcurveto{\pgfqpoint{3.276870in}{1.797085in}}{\pgfqpoint{3.281261in}{1.786486in}}{\pgfqpoint{3.289074in}{1.778672in}}%
\pgfpathcurveto{\pgfqpoint{3.296888in}{1.770859in}}{\pgfqpoint{3.307487in}{1.766469in}}{\pgfqpoint{3.318537in}{1.766469in}}%
\pgfpathclose%
\pgfusepath{stroke,fill}%
\end{pgfscope}%
\begin{pgfscope}%
\pgfpathrectangle{\pgfqpoint{0.787074in}{0.548769in}}{\pgfqpoint{5.062926in}{3.102590in}}%
\pgfusepath{clip}%
\pgfsetbuttcap%
\pgfsetroundjoin%
\definecolor{currentfill}{rgb}{1.000000,0.498039,0.054902}%
\pgfsetfillcolor{currentfill}%
\pgfsetlinewidth{1.003750pt}%
\definecolor{currentstroke}{rgb}{1.000000,0.498039,0.054902}%
\pgfsetstrokecolor{currentstroke}%
\pgfsetdash{}{0pt}%
\pgfpathmoveto{\pgfqpoint{3.942626in}{2.575149in}}%
\pgfpathcurveto{\pgfqpoint{3.953677in}{2.575149in}}{\pgfqpoint{3.964276in}{2.579539in}}{\pgfqpoint{3.972089in}{2.587353in}}%
\pgfpathcurveto{\pgfqpoint{3.979903in}{2.595166in}}{\pgfqpoint{3.984293in}{2.605765in}}{\pgfqpoint{3.984293in}{2.616816in}}%
\pgfpathcurveto{\pgfqpoint{3.984293in}{2.627866in}}{\pgfqpoint{3.979903in}{2.638465in}}{\pgfqpoint{3.972089in}{2.646278in}}%
\pgfpathcurveto{\pgfqpoint{3.964276in}{2.654092in}}{\pgfqpoint{3.953677in}{2.658482in}}{\pgfqpoint{3.942626in}{2.658482in}}%
\pgfpathcurveto{\pgfqpoint{3.931576in}{2.658482in}}{\pgfqpoint{3.920977in}{2.654092in}}{\pgfqpoint{3.913164in}{2.646278in}}%
\pgfpathcurveto{\pgfqpoint{3.905350in}{2.638465in}}{\pgfqpoint{3.900960in}{2.627866in}}{\pgfqpoint{3.900960in}{2.616816in}}%
\pgfpathcurveto{\pgfqpoint{3.900960in}{2.605765in}}{\pgfqpoint{3.905350in}{2.595166in}}{\pgfqpoint{3.913164in}{2.587353in}}%
\pgfpathcurveto{\pgfqpoint{3.920977in}{2.579539in}}{\pgfqpoint{3.931576in}{2.575149in}}{\pgfqpoint{3.942626in}{2.575149in}}%
\pgfpathclose%
\pgfusepath{stroke,fill}%
\end{pgfscope}%
\begin{pgfscope}%
\pgfpathrectangle{\pgfqpoint{0.787074in}{0.548769in}}{\pgfqpoint{5.062926in}{3.102590in}}%
\pgfusepath{clip}%
\pgfsetbuttcap%
\pgfsetroundjoin%
\definecolor{currentfill}{rgb}{0.121569,0.466667,0.705882}%
\pgfsetfillcolor{currentfill}%
\pgfsetlinewidth{1.003750pt}%
\definecolor{currentstroke}{rgb}{0.121569,0.466667,0.705882}%
\pgfsetstrokecolor{currentstroke}%
\pgfsetdash{}{0pt}%
\pgfpathmoveto{\pgfqpoint{3.513565in}{0.825330in}}%
\pgfpathcurveto{\pgfqpoint{3.524615in}{0.825330in}}{\pgfqpoint{3.535214in}{0.829720in}}{\pgfqpoint{3.543028in}{0.837534in}}%
\pgfpathcurveto{\pgfqpoint{3.550841in}{0.845347in}}{\pgfqpoint{3.555232in}{0.855946in}}{\pgfqpoint{3.555232in}{0.866996in}}%
\pgfpathcurveto{\pgfqpoint{3.555232in}{0.878046in}}{\pgfqpoint{3.550841in}{0.888646in}}{\pgfqpoint{3.543028in}{0.896459in}}%
\pgfpathcurveto{\pgfqpoint{3.535214in}{0.904273in}}{\pgfqpoint{3.524615in}{0.908663in}}{\pgfqpoint{3.513565in}{0.908663in}}%
\pgfpathcurveto{\pgfqpoint{3.502515in}{0.908663in}}{\pgfqpoint{3.491916in}{0.904273in}}{\pgfqpoint{3.484102in}{0.896459in}}%
\pgfpathcurveto{\pgfqpoint{3.476288in}{0.888646in}}{\pgfqpoint{3.471898in}{0.878046in}}{\pgfqpoint{3.471898in}{0.866996in}}%
\pgfpathcurveto{\pgfqpoint{3.471898in}{0.855946in}}{\pgfqpoint{3.476288in}{0.845347in}}{\pgfqpoint{3.484102in}{0.837534in}}%
\pgfpathcurveto{\pgfqpoint{3.491916in}{0.829720in}}{\pgfqpoint{3.502515in}{0.825330in}}{\pgfqpoint{3.513565in}{0.825330in}}%
\pgfpathclose%
\pgfusepath{stroke,fill}%
\end{pgfscope}%
\begin{pgfscope}%
\pgfpathrectangle{\pgfqpoint{0.787074in}{0.548769in}}{\pgfqpoint{5.062926in}{3.102590in}}%
\pgfusepath{clip}%
\pgfsetbuttcap%
\pgfsetroundjoin%
\definecolor{currentfill}{rgb}{1.000000,0.498039,0.054902}%
\pgfsetfillcolor{currentfill}%
\pgfsetlinewidth{1.003750pt}%
\definecolor{currentstroke}{rgb}{1.000000,0.498039,0.054902}%
\pgfsetstrokecolor{currentstroke}%
\pgfsetdash{}{0pt}%
\pgfpathmoveto{\pgfqpoint{3.201520in}{1.508166in}}%
\pgfpathcurveto{\pgfqpoint{3.212570in}{1.508166in}}{\pgfqpoint{3.223169in}{1.512556in}}{\pgfqpoint{3.230983in}{1.520370in}}%
\pgfpathcurveto{\pgfqpoint{3.238797in}{1.528184in}}{\pgfqpoint{3.243187in}{1.538783in}}{\pgfqpoint{3.243187in}{1.549833in}}%
\pgfpathcurveto{\pgfqpoint{3.243187in}{1.560883in}}{\pgfqpoint{3.238797in}{1.571482in}}{\pgfqpoint{3.230983in}{1.579296in}}%
\pgfpathcurveto{\pgfqpoint{3.223169in}{1.587109in}}{\pgfqpoint{3.212570in}{1.591499in}}{\pgfqpoint{3.201520in}{1.591499in}}%
\pgfpathcurveto{\pgfqpoint{3.190470in}{1.591499in}}{\pgfqpoint{3.179871in}{1.587109in}}{\pgfqpoint{3.172057in}{1.579296in}}%
\pgfpathcurveto{\pgfqpoint{3.164244in}{1.571482in}}{\pgfqpoint{3.159853in}{1.560883in}}{\pgfqpoint{3.159853in}{1.549833in}}%
\pgfpathcurveto{\pgfqpoint{3.159853in}{1.538783in}}{\pgfqpoint{3.164244in}{1.528184in}}{\pgfqpoint{3.172057in}{1.520370in}}%
\pgfpathcurveto{\pgfqpoint{3.179871in}{1.512556in}}{\pgfqpoint{3.190470in}{1.508166in}}{\pgfqpoint{3.201520in}{1.508166in}}%
\pgfpathclose%
\pgfusepath{stroke,fill}%
\end{pgfscope}%
\begin{pgfscope}%
\pgfpathrectangle{\pgfqpoint{0.787074in}{0.548769in}}{\pgfqpoint{5.062926in}{3.102590in}}%
\pgfusepath{clip}%
\pgfsetbuttcap%
\pgfsetroundjoin%
\definecolor{currentfill}{rgb}{1.000000,0.498039,0.054902}%
\pgfsetfillcolor{currentfill}%
\pgfsetlinewidth{1.003750pt}%
\definecolor{currentstroke}{rgb}{1.000000,0.498039,0.054902}%
\pgfsetstrokecolor{currentstroke}%
\pgfsetdash{}{0pt}%
\pgfpathmoveto{\pgfqpoint{3.435554in}{2.241293in}}%
\pgfpathcurveto{\pgfqpoint{3.446604in}{2.241293in}}{\pgfqpoint{3.457203in}{2.245684in}}{\pgfqpoint{3.465016in}{2.253497in}}%
\pgfpathcurveto{\pgfqpoint{3.472830in}{2.261311in}}{\pgfqpoint{3.477220in}{2.271910in}}{\pgfqpoint{3.477220in}{2.282960in}}%
\pgfpathcurveto{\pgfqpoint{3.477220in}{2.294010in}}{\pgfqpoint{3.472830in}{2.304609in}}{\pgfqpoint{3.465016in}{2.312423in}}%
\pgfpathcurveto{\pgfqpoint{3.457203in}{2.320236in}}{\pgfqpoint{3.446604in}{2.324627in}}{\pgfqpoint{3.435554in}{2.324627in}}%
\pgfpathcurveto{\pgfqpoint{3.424504in}{2.324627in}}{\pgfqpoint{3.413905in}{2.320236in}}{\pgfqpoint{3.406091in}{2.312423in}}%
\pgfpathcurveto{\pgfqpoint{3.398277in}{2.304609in}}{\pgfqpoint{3.393887in}{2.294010in}}{\pgfqpoint{3.393887in}{2.282960in}}%
\pgfpathcurveto{\pgfqpoint{3.393887in}{2.271910in}}{\pgfqpoint{3.398277in}{2.261311in}}{\pgfqpoint{3.406091in}{2.253497in}}%
\pgfpathcurveto{\pgfqpoint{3.413905in}{2.245684in}}{\pgfqpoint{3.424504in}{2.241293in}}{\pgfqpoint{3.435554in}{2.241293in}}%
\pgfpathclose%
\pgfusepath{stroke,fill}%
\end{pgfscope}%
\begin{pgfscope}%
\pgfpathrectangle{\pgfqpoint{0.787074in}{0.548769in}}{\pgfqpoint{5.062926in}{3.102590in}}%
\pgfusepath{clip}%
\pgfsetbuttcap%
\pgfsetroundjoin%
\definecolor{currentfill}{rgb}{0.121569,0.466667,0.705882}%
\pgfsetfillcolor{currentfill}%
\pgfsetlinewidth{1.003750pt}%
\definecolor{currentstroke}{rgb}{0.121569,0.466667,0.705882}%
\pgfsetstrokecolor{currentstroke}%
\pgfsetdash{}{0pt}%
\pgfpathmoveto{\pgfqpoint{1.017207in}{0.648150in}}%
\pgfpathcurveto{\pgfqpoint{1.028257in}{0.648150in}}{\pgfqpoint{1.038856in}{0.652540in}}{\pgfqpoint{1.046670in}{0.660353in}}%
\pgfpathcurveto{\pgfqpoint{1.054483in}{0.668167in}}{\pgfqpoint{1.058874in}{0.678766in}}{\pgfqpoint{1.058874in}{0.689816in}}%
\pgfpathcurveto{\pgfqpoint{1.058874in}{0.700866in}}{\pgfqpoint{1.054483in}{0.711465in}}{\pgfqpoint{1.046670in}{0.719279in}}%
\pgfpathcurveto{\pgfqpoint{1.038856in}{0.727093in}}{\pgfqpoint{1.028257in}{0.731483in}}{\pgfqpoint{1.017207in}{0.731483in}}%
\pgfpathcurveto{\pgfqpoint{1.006157in}{0.731483in}}{\pgfqpoint{0.995558in}{0.727093in}}{\pgfqpoint{0.987744in}{0.719279in}}%
\pgfpathcurveto{\pgfqpoint{0.979930in}{0.711465in}}{\pgfqpoint{0.975540in}{0.700866in}}{\pgfqpoint{0.975540in}{0.689816in}}%
\pgfpathcurveto{\pgfqpoint{0.975540in}{0.678766in}}{\pgfqpoint{0.979930in}{0.668167in}}{\pgfqpoint{0.987744in}{0.660353in}}%
\pgfpathcurveto{\pgfqpoint{0.995558in}{0.652540in}}{\pgfqpoint{1.006157in}{0.648150in}}{\pgfqpoint{1.017207in}{0.648150in}}%
\pgfpathclose%
\pgfusepath{stroke,fill}%
\end{pgfscope}%
\begin{pgfscope}%
\pgfpathrectangle{\pgfqpoint{0.787074in}{0.548769in}}{\pgfqpoint{5.062926in}{3.102590in}}%
\pgfusepath{clip}%
\pgfsetbuttcap%
\pgfsetroundjoin%
\definecolor{currentfill}{rgb}{1.000000,0.498039,0.054902}%
\pgfsetfillcolor{currentfill}%
\pgfsetlinewidth{1.003750pt}%
\definecolor{currentstroke}{rgb}{1.000000,0.498039,0.054902}%
\pgfsetstrokecolor{currentstroke}%
\pgfsetdash{}{0pt}%
\pgfpathmoveto{\pgfqpoint{3.942626in}{2.050079in}}%
\pgfpathcurveto{\pgfqpoint{3.953677in}{2.050079in}}{\pgfqpoint{3.964276in}{2.054470in}}{\pgfqpoint{3.972089in}{2.062283in}}%
\pgfpathcurveto{\pgfqpoint{3.979903in}{2.070097in}}{\pgfqpoint{3.984293in}{2.080696in}}{\pgfqpoint{3.984293in}{2.091746in}}%
\pgfpathcurveto{\pgfqpoint{3.984293in}{2.102796in}}{\pgfqpoint{3.979903in}{2.113395in}}{\pgfqpoint{3.972089in}{2.121209in}}%
\pgfpathcurveto{\pgfqpoint{3.964276in}{2.129022in}}{\pgfqpoint{3.953677in}{2.133413in}}{\pgfqpoint{3.942626in}{2.133413in}}%
\pgfpathcurveto{\pgfqpoint{3.931576in}{2.133413in}}{\pgfqpoint{3.920977in}{2.129022in}}{\pgfqpoint{3.913164in}{2.121209in}}%
\pgfpathcurveto{\pgfqpoint{3.905350in}{2.113395in}}{\pgfqpoint{3.900960in}{2.102796in}}{\pgfqpoint{3.900960in}{2.091746in}}%
\pgfpathcurveto{\pgfqpoint{3.900960in}{2.080696in}}{\pgfqpoint{3.905350in}{2.070097in}}{\pgfqpoint{3.913164in}{2.062283in}}%
\pgfpathcurveto{\pgfqpoint{3.920977in}{2.054470in}}{\pgfqpoint{3.931576in}{2.050079in}}{\pgfqpoint{3.942626in}{2.050079in}}%
\pgfpathclose%
\pgfusepath{stroke,fill}%
\end{pgfscope}%
\begin{pgfscope}%
\pgfpathrectangle{\pgfqpoint{0.787074in}{0.548769in}}{\pgfqpoint{5.062926in}{3.102590in}}%
\pgfusepath{clip}%
\pgfsetbuttcap%
\pgfsetroundjoin%
\definecolor{currentfill}{rgb}{0.121569,0.466667,0.705882}%
\pgfsetfillcolor{currentfill}%
\pgfsetlinewidth{1.003750pt}%
\definecolor{currentstroke}{rgb}{0.121569,0.466667,0.705882}%
\pgfsetstrokecolor{currentstroke}%
\pgfsetdash{}{0pt}%
\pgfpathmoveto{\pgfqpoint{3.708593in}{0.648148in}}%
\pgfpathcurveto{\pgfqpoint{3.719643in}{0.648148in}}{\pgfqpoint{3.730242in}{0.652539in}}{\pgfqpoint{3.738056in}{0.660352in}}%
\pgfpathcurveto{\pgfqpoint{3.745869in}{0.668166in}}{\pgfqpoint{3.750260in}{0.678765in}}{\pgfqpoint{3.750260in}{0.689815in}}%
\pgfpathcurveto{\pgfqpoint{3.750260in}{0.700865in}}{\pgfqpoint{3.745869in}{0.711464in}}{\pgfqpoint{3.738056in}{0.719278in}}%
\pgfpathcurveto{\pgfqpoint{3.730242in}{0.727091in}}{\pgfqpoint{3.719643in}{0.731482in}}{\pgfqpoint{3.708593in}{0.731482in}}%
\pgfpathcurveto{\pgfqpoint{3.697543in}{0.731482in}}{\pgfqpoint{3.686944in}{0.727091in}}{\pgfqpoint{3.679130in}{0.719278in}}%
\pgfpathcurveto{\pgfqpoint{3.671316in}{0.711464in}}{\pgfqpoint{3.666926in}{0.700865in}}{\pgfqpoint{3.666926in}{0.689815in}}%
\pgfpathcurveto{\pgfqpoint{3.666926in}{0.678765in}}{\pgfqpoint{3.671316in}{0.668166in}}{\pgfqpoint{3.679130in}{0.660352in}}%
\pgfpathcurveto{\pgfqpoint{3.686944in}{0.652539in}}{\pgfqpoint{3.697543in}{0.648148in}}{\pgfqpoint{3.708593in}{0.648148in}}%
\pgfpathclose%
\pgfusepath{stroke,fill}%
\end{pgfscope}%
\begin{pgfscope}%
\pgfpathrectangle{\pgfqpoint{0.787074in}{0.548769in}}{\pgfqpoint{5.062926in}{3.102590in}}%
\pgfusepath{clip}%
\pgfsetbuttcap%
\pgfsetroundjoin%
\definecolor{currentfill}{rgb}{0.121569,0.466667,0.705882}%
\pgfsetfillcolor{currentfill}%
\pgfsetlinewidth{1.003750pt}%
\definecolor{currentstroke}{rgb}{0.121569,0.466667,0.705882}%
\pgfsetstrokecolor{currentstroke}%
\pgfsetdash{}{0pt}%
\pgfpathmoveto{\pgfqpoint{3.123509in}{0.648168in}}%
\pgfpathcurveto{\pgfqpoint{3.134559in}{0.648168in}}{\pgfqpoint{3.145158in}{0.652558in}}{\pgfqpoint{3.152972in}{0.660372in}}%
\pgfpathcurveto{\pgfqpoint{3.160785in}{0.668185in}}{\pgfqpoint{3.165176in}{0.678784in}}{\pgfqpoint{3.165176in}{0.689834in}}%
\pgfpathcurveto{\pgfqpoint{3.165176in}{0.700885in}}{\pgfqpoint{3.160785in}{0.711484in}}{\pgfqpoint{3.152972in}{0.719297in}}%
\pgfpathcurveto{\pgfqpoint{3.145158in}{0.727111in}}{\pgfqpoint{3.134559in}{0.731501in}}{\pgfqpoint{3.123509in}{0.731501in}}%
\pgfpathcurveto{\pgfqpoint{3.112459in}{0.731501in}}{\pgfqpoint{3.101860in}{0.727111in}}{\pgfqpoint{3.094046in}{0.719297in}}%
\pgfpathcurveto{\pgfqpoint{3.086233in}{0.711484in}}{\pgfqpoint{3.081842in}{0.700885in}}{\pgfqpoint{3.081842in}{0.689834in}}%
\pgfpathcurveto{\pgfqpoint{3.081842in}{0.678784in}}{\pgfqpoint{3.086233in}{0.668185in}}{\pgfqpoint{3.094046in}{0.660372in}}%
\pgfpathcurveto{\pgfqpoint{3.101860in}{0.652558in}}{\pgfqpoint{3.112459in}{0.648168in}}{\pgfqpoint{3.123509in}{0.648168in}}%
\pgfpathclose%
\pgfusepath{stroke,fill}%
\end{pgfscope}%
\begin{pgfscope}%
\pgfpathrectangle{\pgfqpoint{0.787074in}{0.548769in}}{\pgfqpoint{5.062926in}{3.102590in}}%
\pgfusepath{clip}%
\pgfsetbuttcap%
\pgfsetroundjoin%
\definecolor{currentfill}{rgb}{0.121569,0.466667,0.705882}%
\pgfsetfillcolor{currentfill}%
\pgfsetlinewidth{1.003750pt}%
\definecolor{currentstroke}{rgb}{0.121569,0.466667,0.705882}%
\pgfsetstrokecolor{currentstroke}%
\pgfsetdash{}{0pt}%
\pgfpathmoveto{\pgfqpoint{3.630582in}{0.861755in}}%
\pgfpathcurveto{\pgfqpoint{3.641632in}{0.861755in}}{\pgfqpoint{3.652231in}{0.866145in}}{\pgfqpoint{3.660044in}{0.873958in}}%
\pgfpathcurveto{\pgfqpoint{3.667858in}{0.881772in}}{\pgfqpoint{3.672248in}{0.892371in}}{\pgfqpoint{3.672248in}{0.903421in}}%
\pgfpathcurveto{\pgfqpoint{3.672248in}{0.914471in}}{\pgfqpoint{3.667858in}{0.925070in}}{\pgfqpoint{3.660044in}{0.932884in}}%
\pgfpathcurveto{\pgfqpoint{3.652231in}{0.940698in}}{\pgfqpoint{3.641632in}{0.945088in}}{\pgfqpoint{3.630582in}{0.945088in}}%
\pgfpathcurveto{\pgfqpoint{3.619532in}{0.945088in}}{\pgfqpoint{3.608933in}{0.940698in}}{\pgfqpoint{3.601119in}{0.932884in}}%
\pgfpathcurveto{\pgfqpoint{3.593305in}{0.925070in}}{\pgfqpoint{3.588915in}{0.914471in}}{\pgfqpoint{3.588915in}{0.903421in}}%
\pgfpathcurveto{\pgfqpoint{3.588915in}{0.892371in}}{\pgfqpoint{3.593305in}{0.881772in}}{\pgfqpoint{3.601119in}{0.873958in}}%
\pgfpathcurveto{\pgfqpoint{3.608933in}{0.866145in}}{\pgfqpoint{3.619532in}{0.861755in}}{\pgfqpoint{3.630582in}{0.861755in}}%
\pgfpathclose%
\pgfusepath{stroke,fill}%
\end{pgfscope}%
\begin{pgfscope}%
\pgfpathrectangle{\pgfqpoint{0.787074in}{0.548769in}}{\pgfqpoint{5.062926in}{3.102590in}}%
\pgfusepath{clip}%
\pgfsetbuttcap%
\pgfsetroundjoin%
\definecolor{currentfill}{rgb}{0.121569,0.466667,0.705882}%
\pgfsetfillcolor{currentfill}%
\pgfsetlinewidth{1.003750pt}%
\definecolor{currentstroke}{rgb}{0.121569,0.466667,0.705882}%
\pgfsetstrokecolor{currentstroke}%
\pgfsetdash{}{0pt}%
\pgfpathmoveto{\pgfqpoint{3.708593in}{1.953744in}}%
\pgfpathcurveto{\pgfqpoint{3.719643in}{1.953744in}}{\pgfqpoint{3.730242in}{1.958135in}}{\pgfqpoint{3.738056in}{1.965948in}}%
\pgfpathcurveto{\pgfqpoint{3.745869in}{1.973762in}}{\pgfqpoint{3.750260in}{1.984361in}}{\pgfqpoint{3.750260in}{1.995411in}}%
\pgfpathcurveto{\pgfqpoint{3.750260in}{2.006461in}}{\pgfqpoint{3.745869in}{2.017060in}}{\pgfqpoint{3.738056in}{2.024874in}}%
\pgfpathcurveto{\pgfqpoint{3.730242in}{2.032687in}}{\pgfqpoint{3.719643in}{2.037078in}}{\pgfqpoint{3.708593in}{2.037078in}}%
\pgfpathcurveto{\pgfqpoint{3.697543in}{2.037078in}}{\pgfqpoint{3.686944in}{2.032687in}}{\pgfqpoint{3.679130in}{2.024874in}}%
\pgfpathcurveto{\pgfqpoint{3.671316in}{2.017060in}}{\pgfqpoint{3.666926in}{2.006461in}}{\pgfqpoint{3.666926in}{1.995411in}}%
\pgfpathcurveto{\pgfqpoint{3.666926in}{1.984361in}}{\pgfqpoint{3.671316in}{1.973762in}}{\pgfqpoint{3.679130in}{1.965948in}}%
\pgfpathcurveto{\pgfqpoint{3.686944in}{1.958135in}}{\pgfqpoint{3.697543in}{1.953744in}}{\pgfqpoint{3.708593in}{1.953744in}}%
\pgfpathclose%
\pgfusepath{stroke,fill}%
\end{pgfscope}%
\begin{pgfscope}%
\pgfpathrectangle{\pgfqpoint{0.787074in}{0.548769in}}{\pgfqpoint{5.062926in}{3.102590in}}%
\pgfusepath{clip}%
\pgfsetbuttcap%
\pgfsetroundjoin%
\definecolor{currentfill}{rgb}{1.000000,0.498039,0.054902}%
\pgfsetfillcolor{currentfill}%
\pgfsetlinewidth{1.003750pt}%
\definecolor{currentstroke}{rgb}{1.000000,0.498039,0.054902}%
\pgfsetstrokecolor{currentstroke}%
\pgfsetdash{}{0pt}%
\pgfpathmoveto{\pgfqpoint{3.045498in}{1.401411in}}%
\pgfpathcurveto{\pgfqpoint{3.056548in}{1.401411in}}{\pgfqpoint{3.067147in}{1.405801in}}{\pgfqpoint{3.074961in}{1.413615in}}%
\pgfpathcurveto{\pgfqpoint{3.082774in}{1.421429in}}{\pgfqpoint{3.087164in}{1.432028in}}{\pgfqpoint{3.087164in}{1.443078in}}%
\pgfpathcurveto{\pgfqpoint{3.087164in}{1.454128in}}{\pgfqpoint{3.082774in}{1.464727in}}{\pgfqpoint{3.074961in}{1.472540in}}%
\pgfpathcurveto{\pgfqpoint{3.067147in}{1.480354in}}{\pgfqpoint{3.056548in}{1.484744in}}{\pgfqpoint{3.045498in}{1.484744in}}%
\pgfpathcurveto{\pgfqpoint{3.034448in}{1.484744in}}{\pgfqpoint{3.023849in}{1.480354in}}{\pgfqpoint{3.016035in}{1.472540in}}%
\pgfpathcurveto{\pgfqpoint{3.008221in}{1.464727in}}{\pgfqpoint{3.003831in}{1.454128in}}{\pgfqpoint{3.003831in}{1.443078in}}%
\pgfpathcurveto{\pgfqpoint{3.003831in}{1.432028in}}{\pgfqpoint{3.008221in}{1.421429in}}{\pgfqpoint{3.016035in}{1.413615in}}%
\pgfpathcurveto{\pgfqpoint{3.023849in}{1.405801in}}{\pgfqpoint{3.034448in}{1.401411in}}{\pgfqpoint{3.045498in}{1.401411in}}%
\pgfpathclose%
\pgfusepath{stroke,fill}%
\end{pgfscope}%
\begin{pgfscope}%
\pgfpathrectangle{\pgfqpoint{0.787074in}{0.548769in}}{\pgfqpoint{5.062926in}{3.102590in}}%
\pgfusepath{clip}%
\pgfsetbuttcap%
\pgfsetroundjoin%
\definecolor{currentfill}{rgb}{0.121569,0.466667,0.705882}%
\pgfsetfillcolor{currentfill}%
\pgfsetlinewidth{1.003750pt}%
\definecolor{currentstroke}{rgb}{0.121569,0.466667,0.705882}%
\pgfsetstrokecolor{currentstroke}%
\pgfsetdash{}{0pt}%
\pgfpathmoveto{\pgfqpoint{3.357543in}{0.648141in}}%
\pgfpathcurveto{\pgfqpoint{3.368593in}{0.648141in}}{\pgfqpoint{3.379192in}{0.652531in}}{\pgfqpoint{3.387005in}{0.660344in}}%
\pgfpathcurveto{\pgfqpoint{3.394819in}{0.668158in}}{\pgfqpoint{3.399209in}{0.678757in}}{\pgfqpoint{3.399209in}{0.689807in}}%
\pgfpathcurveto{\pgfqpoint{3.399209in}{0.700857in}}{\pgfqpoint{3.394819in}{0.711456in}}{\pgfqpoint{3.387005in}{0.719270in}}%
\pgfpathcurveto{\pgfqpoint{3.379192in}{0.727084in}}{\pgfqpoint{3.368593in}{0.731474in}}{\pgfqpoint{3.357543in}{0.731474in}}%
\pgfpathcurveto{\pgfqpoint{3.346492in}{0.731474in}}{\pgfqpoint{3.335893in}{0.727084in}}{\pgfqpoint{3.328080in}{0.719270in}}%
\pgfpathcurveto{\pgfqpoint{3.320266in}{0.711456in}}{\pgfqpoint{3.315876in}{0.700857in}}{\pgfqpoint{3.315876in}{0.689807in}}%
\pgfpathcurveto{\pgfqpoint{3.315876in}{0.678757in}}{\pgfqpoint{3.320266in}{0.668158in}}{\pgfqpoint{3.328080in}{0.660344in}}%
\pgfpathcurveto{\pgfqpoint{3.335893in}{0.652531in}}{\pgfqpoint{3.346492in}{0.648141in}}{\pgfqpoint{3.357543in}{0.648141in}}%
\pgfpathclose%
\pgfusepath{stroke,fill}%
\end{pgfscope}%
\begin{pgfscope}%
\pgfpathrectangle{\pgfqpoint{0.787074in}{0.548769in}}{\pgfqpoint{5.062926in}{3.102590in}}%
\pgfusepath{clip}%
\pgfsetbuttcap%
\pgfsetroundjoin%
\definecolor{currentfill}{rgb}{0.121569,0.466667,0.705882}%
\pgfsetfillcolor{currentfill}%
\pgfsetlinewidth{1.003750pt}%
\definecolor{currentstroke}{rgb}{0.121569,0.466667,0.705882}%
\pgfsetstrokecolor{currentstroke}%
\pgfsetdash{}{0pt}%
\pgfpathmoveto{\pgfqpoint{3.630582in}{0.787322in}}%
\pgfpathcurveto{\pgfqpoint{3.641632in}{0.787322in}}{\pgfqpoint{3.652231in}{0.791712in}}{\pgfqpoint{3.660044in}{0.799526in}}%
\pgfpathcurveto{\pgfqpoint{3.667858in}{0.807339in}}{\pgfqpoint{3.672248in}{0.817938in}}{\pgfqpoint{3.672248in}{0.828988in}}%
\pgfpathcurveto{\pgfqpoint{3.672248in}{0.840039in}}{\pgfqpoint{3.667858in}{0.850638in}}{\pgfqpoint{3.660044in}{0.858451in}}%
\pgfpathcurveto{\pgfqpoint{3.652231in}{0.866265in}}{\pgfqpoint{3.641632in}{0.870655in}}{\pgfqpoint{3.630582in}{0.870655in}}%
\pgfpathcurveto{\pgfqpoint{3.619532in}{0.870655in}}{\pgfqpoint{3.608933in}{0.866265in}}{\pgfqpoint{3.601119in}{0.858451in}}%
\pgfpathcurveto{\pgfqpoint{3.593305in}{0.850638in}}{\pgfqpoint{3.588915in}{0.840039in}}{\pgfqpoint{3.588915in}{0.828988in}}%
\pgfpathcurveto{\pgfqpoint{3.588915in}{0.817938in}}{\pgfqpoint{3.593305in}{0.807339in}}{\pgfqpoint{3.601119in}{0.799526in}}%
\pgfpathcurveto{\pgfqpoint{3.608933in}{0.791712in}}{\pgfqpoint{3.619532in}{0.787322in}}{\pgfqpoint{3.630582in}{0.787322in}}%
\pgfpathclose%
\pgfusepath{stroke,fill}%
\end{pgfscope}%
\begin{pgfscope}%
\pgfpathrectangle{\pgfqpoint{0.787074in}{0.548769in}}{\pgfqpoint{5.062926in}{3.102590in}}%
\pgfusepath{clip}%
\pgfsetbuttcap%
\pgfsetroundjoin%
\definecolor{currentfill}{rgb}{0.121569,0.466667,0.705882}%
\pgfsetfillcolor{currentfill}%
\pgfsetlinewidth{1.003750pt}%
\definecolor{currentstroke}{rgb}{0.121569,0.466667,0.705882}%
\pgfsetstrokecolor{currentstroke}%
\pgfsetdash{}{0pt}%
\pgfpathmoveto{\pgfqpoint{3.708593in}{1.729931in}}%
\pgfpathcurveto{\pgfqpoint{3.719643in}{1.729931in}}{\pgfqpoint{3.730242in}{1.734321in}}{\pgfqpoint{3.738056in}{1.742135in}}%
\pgfpathcurveto{\pgfqpoint{3.745869in}{1.749948in}}{\pgfqpoint{3.750260in}{1.760547in}}{\pgfqpoint{3.750260in}{1.771597in}}%
\pgfpathcurveto{\pgfqpoint{3.750260in}{1.782647in}}{\pgfqpoint{3.745869in}{1.793246in}}{\pgfqpoint{3.738056in}{1.801060in}}%
\pgfpathcurveto{\pgfqpoint{3.730242in}{1.808874in}}{\pgfqpoint{3.719643in}{1.813264in}}{\pgfqpoint{3.708593in}{1.813264in}}%
\pgfpathcurveto{\pgfqpoint{3.697543in}{1.813264in}}{\pgfqpoint{3.686944in}{1.808874in}}{\pgfqpoint{3.679130in}{1.801060in}}%
\pgfpathcurveto{\pgfqpoint{3.671316in}{1.793246in}}{\pgfqpoint{3.666926in}{1.782647in}}{\pgfqpoint{3.666926in}{1.771597in}}%
\pgfpathcurveto{\pgfqpoint{3.666926in}{1.760547in}}{\pgfqpoint{3.671316in}{1.749948in}}{\pgfqpoint{3.679130in}{1.742135in}}%
\pgfpathcurveto{\pgfqpoint{3.686944in}{1.734321in}}{\pgfqpoint{3.697543in}{1.729931in}}{\pgfqpoint{3.708593in}{1.729931in}}%
\pgfpathclose%
\pgfusepath{stroke,fill}%
\end{pgfscope}%
\begin{pgfscope}%
\pgfpathrectangle{\pgfqpoint{0.787074in}{0.548769in}}{\pgfqpoint{5.062926in}{3.102590in}}%
\pgfusepath{clip}%
\pgfsetbuttcap%
\pgfsetroundjoin%
\definecolor{currentfill}{rgb}{0.121569,0.466667,0.705882}%
\pgfsetfillcolor{currentfill}%
\pgfsetlinewidth{1.003750pt}%
\definecolor{currentstroke}{rgb}{0.121569,0.466667,0.705882}%
\pgfsetstrokecolor{currentstroke}%
\pgfsetdash{}{0pt}%
\pgfpathmoveto{\pgfqpoint{3.396548in}{0.648129in}}%
\pgfpathcurveto{\pgfqpoint{3.407598in}{0.648129in}}{\pgfqpoint{3.418197in}{0.652519in}}{\pgfqpoint{3.426011in}{0.660333in}}%
\pgfpathcurveto{\pgfqpoint{3.433825in}{0.668146in}}{\pgfqpoint{3.438215in}{0.678745in}}{\pgfqpoint{3.438215in}{0.689796in}}%
\pgfpathcurveto{\pgfqpoint{3.438215in}{0.700846in}}{\pgfqpoint{3.433825in}{0.711445in}}{\pgfqpoint{3.426011in}{0.719258in}}%
\pgfpathcurveto{\pgfqpoint{3.418197in}{0.727072in}}{\pgfqpoint{3.407598in}{0.731462in}}{\pgfqpoint{3.396548in}{0.731462in}}%
\pgfpathcurveto{\pgfqpoint{3.385498in}{0.731462in}}{\pgfqpoint{3.374899in}{0.727072in}}{\pgfqpoint{3.367085in}{0.719258in}}%
\pgfpathcurveto{\pgfqpoint{3.359272in}{0.711445in}}{\pgfqpoint{3.354881in}{0.700846in}}{\pgfqpoint{3.354881in}{0.689796in}}%
\pgfpathcurveto{\pgfqpoint{3.354881in}{0.678745in}}{\pgfqpoint{3.359272in}{0.668146in}}{\pgfqpoint{3.367085in}{0.660333in}}%
\pgfpathcurveto{\pgfqpoint{3.374899in}{0.652519in}}{\pgfqpoint{3.385498in}{0.648129in}}{\pgfqpoint{3.396548in}{0.648129in}}%
\pgfpathclose%
\pgfusepath{stroke,fill}%
\end{pgfscope}%
\begin{pgfscope}%
\pgfpathrectangle{\pgfqpoint{0.787074in}{0.548769in}}{\pgfqpoint{5.062926in}{3.102590in}}%
\pgfusepath{clip}%
\pgfsetbuttcap%
\pgfsetroundjoin%
\definecolor{currentfill}{rgb}{1.000000,0.498039,0.054902}%
\pgfsetfillcolor{currentfill}%
\pgfsetlinewidth{1.003750pt}%
\definecolor{currentstroke}{rgb}{1.000000,0.498039,0.054902}%
\pgfsetstrokecolor{currentstroke}%
\pgfsetdash{}{0pt}%
\pgfpathmoveto{\pgfqpoint{3.435554in}{1.417750in}}%
\pgfpathcurveto{\pgfqpoint{3.446604in}{1.417750in}}{\pgfqpoint{3.457203in}{1.422140in}}{\pgfqpoint{3.465016in}{1.429954in}}%
\pgfpathcurveto{\pgfqpoint{3.472830in}{1.437768in}}{\pgfqpoint{3.477220in}{1.448367in}}{\pgfqpoint{3.477220in}{1.459417in}}%
\pgfpathcurveto{\pgfqpoint{3.477220in}{1.470467in}}{\pgfqpoint{3.472830in}{1.481066in}}{\pgfqpoint{3.465016in}{1.488880in}}%
\pgfpathcurveto{\pgfqpoint{3.457203in}{1.496693in}}{\pgfqpoint{3.446604in}{1.501083in}}{\pgfqpoint{3.435554in}{1.501083in}}%
\pgfpathcurveto{\pgfqpoint{3.424504in}{1.501083in}}{\pgfqpoint{3.413905in}{1.496693in}}{\pgfqpoint{3.406091in}{1.488880in}}%
\pgfpathcurveto{\pgfqpoint{3.398277in}{1.481066in}}{\pgfqpoint{3.393887in}{1.470467in}}{\pgfqpoint{3.393887in}{1.459417in}}%
\pgfpathcurveto{\pgfqpoint{3.393887in}{1.448367in}}{\pgfqpoint{3.398277in}{1.437768in}}{\pgfqpoint{3.406091in}{1.429954in}}%
\pgfpathcurveto{\pgfqpoint{3.413905in}{1.422140in}}{\pgfqpoint{3.424504in}{1.417750in}}{\pgfqpoint{3.435554in}{1.417750in}}%
\pgfpathclose%
\pgfusepath{stroke,fill}%
\end{pgfscope}%
\begin{pgfscope}%
\pgfpathrectangle{\pgfqpoint{0.787074in}{0.548769in}}{\pgfqpoint{5.062926in}{3.102590in}}%
\pgfusepath{clip}%
\pgfsetbuttcap%
\pgfsetroundjoin%
\definecolor{currentfill}{rgb}{1.000000,0.498039,0.054902}%
\pgfsetfillcolor{currentfill}%
\pgfsetlinewidth{1.003750pt}%
\definecolor{currentstroke}{rgb}{1.000000,0.498039,0.054902}%
\pgfsetstrokecolor{currentstroke}%
\pgfsetdash{}{0pt}%
\pgfpathmoveto{\pgfqpoint{3.435554in}{2.301800in}}%
\pgfpathcurveto{\pgfqpoint{3.446604in}{2.301800in}}{\pgfqpoint{3.457203in}{2.306191in}}{\pgfqpoint{3.465016in}{2.314004in}}%
\pgfpathcurveto{\pgfqpoint{3.472830in}{2.321818in}}{\pgfqpoint{3.477220in}{2.332417in}}{\pgfqpoint{3.477220in}{2.343467in}}%
\pgfpathcurveto{\pgfqpoint{3.477220in}{2.354517in}}{\pgfqpoint{3.472830in}{2.365116in}}{\pgfqpoint{3.465016in}{2.372930in}}%
\pgfpathcurveto{\pgfqpoint{3.457203in}{2.380743in}}{\pgfqpoint{3.446604in}{2.385134in}}{\pgfqpoint{3.435554in}{2.385134in}}%
\pgfpathcurveto{\pgfqpoint{3.424504in}{2.385134in}}{\pgfqpoint{3.413905in}{2.380743in}}{\pgfqpoint{3.406091in}{2.372930in}}%
\pgfpathcurveto{\pgfqpoint{3.398277in}{2.365116in}}{\pgfqpoint{3.393887in}{2.354517in}}{\pgfqpoint{3.393887in}{2.343467in}}%
\pgfpathcurveto{\pgfqpoint{3.393887in}{2.332417in}}{\pgfqpoint{3.398277in}{2.321818in}}{\pgfqpoint{3.406091in}{2.314004in}}%
\pgfpathcurveto{\pgfqpoint{3.413905in}{2.306191in}}{\pgfqpoint{3.424504in}{2.301800in}}{\pgfqpoint{3.435554in}{2.301800in}}%
\pgfpathclose%
\pgfusepath{stroke,fill}%
\end{pgfscope}%
\begin{pgfscope}%
\pgfpathrectangle{\pgfqpoint{0.787074in}{0.548769in}}{\pgfqpoint{5.062926in}{3.102590in}}%
\pgfusepath{clip}%
\pgfsetbuttcap%
\pgfsetroundjoin%
\definecolor{currentfill}{rgb}{0.121569,0.466667,0.705882}%
\pgfsetfillcolor{currentfill}%
\pgfsetlinewidth{1.003750pt}%
\definecolor{currentstroke}{rgb}{0.121569,0.466667,0.705882}%
\pgfsetstrokecolor{currentstroke}%
\pgfsetdash{}{0pt}%
\pgfpathmoveto{\pgfqpoint{1.992347in}{0.652202in}}%
\pgfpathcurveto{\pgfqpoint{2.003397in}{0.652202in}}{\pgfqpoint{2.013996in}{0.656593in}}{\pgfqpoint{2.021809in}{0.664406in}}%
\pgfpathcurveto{\pgfqpoint{2.029623in}{0.672220in}}{\pgfqpoint{2.034013in}{0.682819in}}{\pgfqpoint{2.034013in}{0.693869in}}%
\pgfpathcurveto{\pgfqpoint{2.034013in}{0.704919in}}{\pgfqpoint{2.029623in}{0.715518in}}{\pgfqpoint{2.021809in}{0.723332in}}%
\pgfpathcurveto{\pgfqpoint{2.013996in}{0.731145in}}{\pgfqpoint{2.003397in}{0.735536in}}{\pgfqpoint{1.992347in}{0.735536in}}%
\pgfpathcurveto{\pgfqpoint{1.981297in}{0.735536in}}{\pgfqpoint{1.970698in}{0.731145in}}{\pgfqpoint{1.962884in}{0.723332in}}%
\pgfpathcurveto{\pgfqpoint{1.955070in}{0.715518in}}{\pgfqpoint{1.950680in}{0.704919in}}{\pgfqpoint{1.950680in}{0.693869in}}%
\pgfpathcurveto{\pgfqpoint{1.950680in}{0.682819in}}{\pgfqpoint{1.955070in}{0.672220in}}{\pgfqpoint{1.962884in}{0.664406in}}%
\pgfpathcurveto{\pgfqpoint{1.970698in}{0.656593in}}{\pgfqpoint{1.981297in}{0.652202in}}{\pgfqpoint{1.992347in}{0.652202in}}%
\pgfpathclose%
\pgfusepath{stroke,fill}%
\end{pgfscope}%
\begin{pgfscope}%
\pgfpathrectangle{\pgfqpoint{0.787074in}{0.548769in}}{\pgfqpoint{5.062926in}{3.102590in}}%
\pgfusepath{clip}%
\pgfsetbuttcap%
\pgfsetroundjoin%
\definecolor{currentfill}{rgb}{0.121569,0.466667,0.705882}%
\pgfsetfillcolor{currentfill}%
\pgfsetlinewidth{1.003750pt}%
\definecolor{currentstroke}{rgb}{0.121569,0.466667,0.705882}%
\pgfsetstrokecolor{currentstroke}%
\pgfsetdash{}{0pt}%
\pgfpathmoveto{\pgfqpoint{3.318537in}{0.648180in}}%
\pgfpathcurveto{\pgfqpoint{3.329587in}{0.648180in}}{\pgfqpoint{3.340186in}{0.652571in}}{\pgfqpoint{3.348000in}{0.660384in}}%
\pgfpathcurveto{\pgfqpoint{3.355813in}{0.668198in}}{\pgfqpoint{3.360204in}{0.678797in}}{\pgfqpoint{3.360204in}{0.689847in}}%
\pgfpathcurveto{\pgfqpoint{3.360204in}{0.700897in}}{\pgfqpoint{3.355813in}{0.711496in}}{\pgfqpoint{3.348000in}{0.719310in}}%
\pgfpathcurveto{\pgfqpoint{3.340186in}{0.727123in}}{\pgfqpoint{3.329587in}{0.731514in}}{\pgfqpoint{3.318537in}{0.731514in}}%
\pgfpathcurveto{\pgfqpoint{3.307487in}{0.731514in}}{\pgfqpoint{3.296888in}{0.727123in}}{\pgfqpoint{3.289074in}{0.719310in}}%
\pgfpathcurveto{\pgfqpoint{3.281261in}{0.711496in}}{\pgfqpoint{3.276870in}{0.700897in}}{\pgfqpoint{3.276870in}{0.689847in}}%
\pgfpathcurveto{\pgfqpoint{3.276870in}{0.678797in}}{\pgfqpoint{3.281261in}{0.668198in}}{\pgfqpoint{3.289074in}{0.660384in}}%
\pgfpathcurveto{\pgfqpoint{3.296888in}{0.652571in}}{\pgfqpoint{3.307487in}{0.648180in}}{\pgfqpoint{3.318537in}{0.648180in}}%
\pgfpathclose%
\pgfusepath{stroke,fill}%
\end{pgfscope}%
\begin{pgfscope}%
\pgfpathrectangle{\pgfqpoint{0.787074in}{0.548769in}}{\pgfqpoint{5.062926in}{3.102590in}}%
\pgfusepath{clip}%
\pgfsetbuttcap%
\pgfsetroundjoin%
\definecolor{currentfill}{rgb}{0.121569,0.466667,0.705882}%
\pgfsetfillcolor{currentfill}%
\pgfsetlinewidth{1.003750pt}%
\definecolor{currentstroke}{rgb}{0.121569,0.466667,0.705882}%
\pgfsetstrokecolor{currentstroke}%
\pgfsetdash{}{0pt}%
\pgfpathmoveto{\pgfqpoint{3.396548in}{0.648153in}}%
\pgfpathcurveto{\pgfqpoint{3.407598in}{0.648153in}}{\pgfqpoint{3.418197in}{0.652543in}}{\pgfqpoint{3.426011in}{0.660357in}}%
\pgfpathcurveto{\pgfqpoint{3.433825in}{0.668170in}}{\pgfqpoint{3.438215in}{0.678769in}}{\pgfqpoint{3.438215in}{0.689819in}}%
\pgfpathcurveto{\pgfqpoint{3.438215in}{0.700870in}}{\pgfqpoint{3.433825in}{0.711469in}}{\pgfqpoint{3.426011in}{0.719282in}}%
\pgfpathcurveto{\pgfqpoint{3.418197in}{0.727096in}}{\pgfqpoint{3.407598in}{0.731486in}}{\pgfqpoint{3.396548in}{0.731486in}}%
\pgfpathcurveto{\pgfqpoint{3.385498in}{0.731486in}}{\pgfqpoint{3.374899in}{0.727096in}}{\pgfqpoint{3.367085in}{0.719282in}}%
\pgfpathcurveto{\pgfqpoint{3.359272in}{0.711469in}}{\pgfqpoint{3.354881in}{0.700870in}}{\pgfqpoint{3.354881in}{0.689819in}}%
\pgfpathcurveto{\pgfqpoint{3.354881in}{0.678769in}}{\pgfqpoint{3.359272in}{0.668170in}}{\pgfqpoint{3.367085in}{0.660357in}}%
\pgfpathcurveto{\pgfqpoint{3.374899in}{0.652543in}}{\pgfqpoint{3.385498in}{0.648153in}}{\pgfqpoint{3.396548in}{0.648153in}}%
\pgfpathclose%
\pgfusepath{stroke,fill}%
\end{pgfscope}%
\begin{pgfscope}%
\pgfpathrectangle{\pgfqpoint{0.787074in}{0.548769in}}{\pgfqpoint{5.062926in}{3.102590in}}%
\pgfusepath{clip}%
\pgfsetbuttcap%
\pgfsetroundjoin%
\definecolor{currentfill}{rgb}{0.121569,0.466667,0.705882}%
\pgfsetfillcolor{currentfill}%
\pgfsetlinewidth{1.003750pt}%
\definecolor{currentstroke}{rgb}{0.121569,0.466667,0.705882}%
\pgfsetstrokecolor{currentstroke}%
\pgfsetdash{}{0pt}%
\pgfpathmoveto{\pgfqpoint{3.708593in}{0.648456in}}%
\pgfpathcurveto{\pgfqpoint{3.719643in}{0.648456in}}{\pgfqpoint{3.730242in}{0.652846in}}{\pgfqpoint{3.738056in}{0.660660in}}%
\pgfpathcurveto{\pgfqpoint{3.745869in}{0.668474in}}{\pgfqpoint{3.750260in}{0.679073in}}{\pgfqpoint{3.750260in}{0.690123in}}%
\pgfpathcurveto{\pgfqpoint{3.750260in}{0.701173in}}{\pgfqpoint{3.745869in}{0.711772in}}{\pgfqpoint{3.738056in}{0.719586in}}%
\pgfpathcurveto{\pgfqpoint{3.730242in}{0.727399in}}{\pgfqpoint{3.719643in}{0.731789in}}{\pgfqpoint{3.708593in}{0.731789in}}%
\pgfpathcurveto{\pgfqpoint{3.697543in}{0.731789in}}{\pgfqpoint{3.686944in}{0.727399in}}{\pgfqpoint{3.679130in}{0.719586in}}%
\pgfpathcurveto{\pgfqpoint{3.671316in}{0.711772in}}{\pgfqpoint{3.666926in}{0.701173in}}{\pgfqpoint{3.666926in}{0.690123in}}%
\pgfpathcurveto{\pgfqpoint{3.666926in}{0.679073in}}{\pgfqpoint{3.671316in}{0.668474in}}{\pgfqpoint{3.679130in}{0.660660in}}%
\pgfpathcurveto{\pgfqpoint{3.686944in}{0.652846in}}{\pgfqpoint{3.697543in}{0.648456in}}{\pgfqpoint{3.708593in}{0.648456in}}%
\pgfpathclose%
\pgfusepath{stroke,fill}%
\end{pgfscope}%
\begin{pgfscope}%
\pgfpathrectangle{\pgfqpoint{0.787074in}{0.548769in}}{\pgfqpoint{5.062926in}{3.102590in}}%
\pgfusepath{clip}%
\pgfsetbuttcap%
\pgfsetroundjoin%
\definecolor{currentfill}{rgb}{0.121569,0.466667,0.705882}%
\pgfsetfillcolor{currentfill}%
\pgfsetlinewidth{1.003750pt}%
\definecolor{currentstroke}{rgb}{0.121569,0.466667,0.705882}%
\pgfsetstrokecolor{currentstroke}%
\pgfsetdash{}{0pt}%
\pgfpathmoveto{\pgfqpoint{3.357543in}{0.648200in}}%
\pgfpathcurveto{\pgfqpoint{3.368593in}{0.648200in}}{\pgfqpoint{3.379192in}{0.652590in}}{\pgfqpoint{3.387005in}{0.660404in}}%
\pgfpathcurveto{\pgfqpoint{3.394819in}{0.668217in}}{\pgfqpoint{3.399209in}{0.678816in}}{\pgfqpoint{3.399209in}{0.689866in}}%
\pgfpathcurveto{\pgfqpoint{3.399209in}{0.700917in}}{\pgfqpoint{3.394819in}{0.711516in}}{\pgfqpoint{3.387005in}{0.719329in}}%
\pgfpathcurveto{\pgfqpoint{3.379192in}{0.727143in}}{\pgfqpoint{3.368593in}{0.731533in}}{\pgfqpoint{3.357543in}{0.731533in}}%
\pgfpathcurveto{\pgfqpoint{3.346492in}{0.731533in}}{\pgfqpoint{3.335893in}{0.727143in}}{\pgfqpoint{3.328080in}{0.719329in}}%
\pgfpathcurveto{\pgfqpoint{3.320266in}{0.711516in}}{\pgfqpoint{3.315876in}{0.700917in}}{\pgfqpoint{3.315876in}{0.689866in}}%
\pgfpathcurveto{\pgfqpoint{3.315876in}{0.678816in}}{\pgfqpoint{3.320266in}{0.668217in}}{\pgfqpoint{3.328080in}{0.660404in}}%
\pgfpathcurveto{\pgfqpoint{3.335893in}{0.652590in}}{\pgfqpoint{3.346492in}{0.648200in}}{\pgfqpoint{3.357543in}{0.648200in}}%
\pgfpathclose%
\pgfusepath{stroke,fill}%
\end{pgfscope}%
\begin{pgfscope}%
\pgfpathrectangle{\pgfqpoint{0.787074in}{0.548769in}}{\pgfqpoint{5.062926in}{3.102590in}}%
\pgfusepath{clip}%
\pgfsetbuttcap%
\pgfsetroundjoin%
\definecolor{currentfill}{rgb}{0.121569,0.466667,0.705882}%
\pgfsetfillcolor{currentfill}%
\pgfsetlinewidth{1.003750pt}%
\definecolor{currentstroke}{rgb}{0.121569,0.466667,0.705882}%
\pgfsetstrokecolor{currentstroke}%
\pgfsetdash{}{0pt}%
\pgfpathmoveto{\pgfqpoint{3.435554in}{0.648150in}}%
\pgfpathcurveto{\pgfqpoint{3.446604in}{0.648150in}}{\pgfqpoint{3.457203in}{0.652540in}}{\pgfqpoint{3.465016in}{0.660354in}}%
\pgfpathcurveto{\pgfqpoint{3.472830in}{0.668167in}}{\pgfqpoint{3.477220in}{0.678766in}}{\pgfqpoint{3.477220in}{0.689816in}}%
\pgfpathcurveto{\pgfqpoint{3.477220in}{0.700866in}}{\pgfqpoint{3.472830in}{0.711465in}}{\pgfqpoint{3.465016in}{0.719279in}}%
\pgfpathcurveto{\pgfqpoint{3.457203in}{0.727093in}}{\pgfqpoint{3.446604in}{0.731483in}}{\pgfqpoint{3.435554in}{0.731483in}}%
\pgfpathcurveto{\pgfqpoint{3.424504in}{0.731483in}}{\pgfqpoint{3.413905in}{0.727093in}}{\pgfqpoint{3.406091in}{0.719279in}}%
\pgfpathcurveto{\pgfqpoint{3.398277in}{0.711465in}}{\pgfqpoint{3.393887in}{0.700866in}}{\pgfqpoint{3.393887in}{0.689816in}}%
\pgfpathcurveto{\pgfqpoint{3.393887in}{0.678766in}}{\pgfqpoint{3.398277in}{0.668167in}}{\pgfqpoint{3.406091in}{0.660354in}}%
\pgfpathcurveto{\pgfqpoint{3.413905in}{0.652540in}}{\pgfqpoint{3.424504in}{0.648150in}}{\pgfqpoint{3.435554in}{0.648150in}}%
\pgfpathclose%
\pgfusepath{stroke,fill}%
\end{pgfscope}%
\begin{pgfscope}%
\pgfpathrectangle{\pgfqpoint{0.787074in}{0.548769in}}{\pgfqpoint{5.062926in}{3.102590in}}%
\pgfusepath{clip}%
\pgfsetbuttcap%
\pgfsetroundjoin%
\definecolor{currentfill}{rgb}{1.000000,0.498039,0.054902}%
\pgfsetfillcolor{currentfill}%
\pgfsetlinewidth{1.003750pt}%
\definecolor{currentstroke}{rgb}{1.000000,0.498039,0.054902}%
\pgfsetstrokecolor{currentstroke}%
\pgfsetdash{}{0pt}%
\pgfpathmoveto{\pgfqpoint{3.630582in}{3.030917in}}%
\pgfpathcurveto{\pgfqpoint{3.641632in}{3.030917in}}{\pgfqpoint{3.652231in}{3.035308in}}{\pgfqpoint{3.660044in}{3.043121in}}%
\pgfpathcurveto{\pgfqpoint{3.667858in}{3.050935in}}{\pgfqpoint{3.672248in}{3.061534in}}{\pgfqpoint{3.672248in}{3.072584in}}%
\pgfpathcurveto{\pgfqpoint{3.672248in}{3.083634in}}{\pgfqpoint{3.667858in}{3.094233in}}{\pgfqpoint{3.660044in}{3.102047in}}%
\pgfpathcurveto{\pgfqpoint{3.652231in}{3.109861in}}{\pgfqpoint{3.641632in}{3.114251in}}{\pgfqpoint{3.630582in}{3.114251in}}%
\pgfpathcurveto{\pgfqpoint{3.619532in}{3.114251in}}{\pgfqpoint{3.608933in}{3.109861in}}{\pgfqpoint{3.601119in}{3.102047in}}%
\pgfpathcurveto{\pgfqpoint{3.593305in}{3.094233in}}{\pgfqpoint{3.588915in}{3.083634in}}{\pgfqpoint{3.588915in}{3.072584in}}%
\pgfpathcurveto{\pgfqpoint{3.588915in}{3.061534in}}{\pgfqpoint{3.593305in}{3.050935in}}{\pgfqpoint{3.601119in}{3.043121in}}%
\pgfpathcurveto{\pgfqpoint{3.608933in}{3.035308in}}{\pgfqpoint{3.619532in}{3.030917in}}{\pgfqpoint{3.630582in}{3.030917in}}%
\pgfpathclose%
\pgfusepath{stroke,fill}%
\end{pgfscope}%
\begin{pgfscope}%
\pgfpathrectangle{\pgfqpoint{0.787074in}{0.548769in}}{\pgfqpoint{5.062926in}{3.102590in}}%
\pgfusepath{clip}%
\pgfsetbuttcap%
\pgfsetroundjoin%
\definecolor{currentfill}{rgb}{1.000000,0.498039,0.054902}%
\pgfsetfillcolor{currentfill}%
\pgfsetlinewidth{1.003750pt}%
\definecolor{currentstroke}{rgb}{1.000000,0.498039,0.054902}%
\pgfsetstrokecolor{currentstroke}%
\pgfsetdash{}{0pt}%
\pgfpathmoveto{\pgfqpoint{3.591576in}{2.938746in}}%
\pgfpathcurveto{\pgfqpoint{3.602626in}{2.938746in}}{\pgfqpoint{3.613225in}{2.943136in}}{\pgfqpoint{3.621039in}{2.950950in}}%
\pgfpathcurveto{\pgfqpoint{3.628852in}{2.958764in}}{\pgfqpoint{3.633243in}{2.969363in}}{\pgfqpoint{3.633243in}{2.980413in}}%
\pgfpathcurveto{\pgfqpoint{3.633243in}{2.991463in}}{\pgfqpoint{3.628852in}{3.002062in}}{\pgfqpoint{3.621039in}{3.009876in}}%
\pgfpathcurveto{\pgfqpoint{3.613225in}{3.017689in}}{\pgfqpoint{3.602626in}{3.022079in}}{\pgfqpoint{3.591576in}{3.022079in}}%
\pgfpathcurveto{\pgfqpoint{3.580526in}{3.022079in}}{\pgfqpoint{3.569927in}{3.017689in}}{\pgfqpoint{3.562113in}{3.009876in}}%
\pgfpathcurveto{\pgfqpoint{3.554300in}{3.002062in}}{\pgfqpoint{3.549909in}{2.991463in}}{\pgfqpoint{3.549909in}{2.980413in}}%
\pgfpathcurveto{\pgfqpoint{3.549909in}{2.969363in}}{\pgfqpoint{3.554300in}{2.958764in}}{\pgfqpoint{3.562113in}{2.950950in}}%
\pgfpathcurveto{\pgfqpoint{3.569927in}{2.943136in}}{\pgfqpoint{3.580526in}{2.938746in}}{\pgfqpoint{3.591576in}{2.938746in}}%
\pgfpathclose%
\pgfusepath{stroke,fill}%
\end{pgfscope}%
\begin{pgfscope}%
\pgfpathrectangle{\pgfqpoint{0.787074in}{0.548769in}}{\pgfqpoint{5.062926in}{3.102590in}}%
\pgfusepath{clip}%
\pgfsetbuttcap%
\pgfsetroundjoin%
\definecolor{currentfill}{rgb}{1.000000,0.498039,0.054902}%
\pgfsetfillcolor{currentfill}%
\pgfsetlinewidth{1.003750pt}%
\definecolor{currentstroke}{rgb}{1.000000,0.498039,0.054902}%
\pgfsetstrokecolor{currentstroke}%
\pgfsetdash{}{0pt}%
\pgfpathmoveto{\pgfqpoint{3.825610in}{2.706441in}}%
\pgfpathcurveto{\pgfqpoint{3.836660in}{2.706441in}}{\pgfqpoint{3.847259in}{2.710831in}}{\pgfqpoint{3.855072in}{2.718645in}}%
\pgfpathcurveto{\pgfqpoint{3.862886in}{2.726459in}}{\pgfqpoint{3.867276in}{2.737058in}}{\pgfqpoint{3.867276in}{2.748108in}}%
\pgfpathcurveto{\pgfqpoint{3.867276in}{2.759158in}}{\pgfqpoint{3.862886in}{2.769757in}}{\pgfqpoint{3.855072in}{2.777571in}}%
\pgfpathcurveto{\pgfqpoint{3.847259in}{2.785384in}}{\pgfqpoint{3.836660in}{2.789774in}}{\pgfqpoint{3.825610in}{2.789774in}}%
\pgfpathcurveto{\pgfqpoint{3.814560in}{2.789774in}}{\pgfqpoint{3.803960in}{2.785384in}}{\pgfqpoint{3.796147in}{2.777571in}}%
\pgfpathcurveto{\pgfqpoint{3.788333in}{2.769757in}}{\pgfqpoint{3.783943in}{2.759158in}}{\pgfqpoint{3.783943in}{2.748108in}}%
\pgfpathcurveto{\pgfqpoint{3.783943in}{2.737058in}}{\pgfqpoint{3.788333in}{2.726459in}}{\pgfqpoint{3.796147in}{2.718645in}}%
\pgfpathcurveto{\pgfqpoint{3.803960in}{2.710831in}}{\pgfqpoint{3.814560in}{2.706441in}}{\pgfqpoint{3.825610in}{2.706441in}}%
\pgfpathclose%
\pgfusepath{stroke,fill}%
\end{pgfscope}%
\begin{pgfscope}%
\pgfpathrectangle{\pgfqpoint{0.787074in}{0.548769in}}{\pgfqpoint{5.062926in}{3.102590in}}%
\pgfusepath{clip}%
\pgfsetbuttcap%
\pgfsetroundjoin%
\definecolor{currentfill}{rgb}{0.121569,0.466667,0.705882}%
\pgfsetfillcolor{currentfill}%
\pgfsetlinewidth{1.003750pt}%
\definecolor{currentstroke}{rgb}{0.121569,0.466667,0.705882}%
\pgfsetstrokecolor{currentstroke}%
\pgfsetdash{}{0pt}%
\pgfpathmoveto{\pgfqpoint{3.162515in}{0.648133in}}%
\pgfpathcurveto{\pgfqpoint{3.173565in}{0.648133in}}{\pgfqpoint{3.184164in}{0.652523in}}{\pgfqpoint{3.191977in}{0.660337in}}%
\pgfpathcurveto{\pgfqpoint{3.199791in}{0.668150in}}{\pgfqpoint{3.204181in}{0.678749in}}{\pgfqpoint{3.204181in}{0.689799in}}%
\pgfpathcurveto{\pgfqpoint{3.204181in}{0.700849in}}{\pgfqpoint{3.199791in}{0.711448in}}{\pgfqpoint{3.191977in}{0.719262in}}%
\pgfpathcurveto{\pgfqpoint{3.184164in}{0.727076in}}{\pgfqpoint{3.173565in}{0.731466in}}{\pgfqpoint{3.162515in}{0.731466in}}%
\pgfpathcurveto{\pgfqpoint{3.151464in}{0.731466in}}{\pgfqpoint{3.140865in}{0.727076in}}{\pgfqpoint{3.133052in}{0.719262in}}%
\pgfpathcurveto{\pgfqpoint{3.125238in}{0.711448in}}{\pgfqpoint{3.120848in}{0.700849in}}{\pgfqpoint{3.120848in}{0.689799in}}%
\pgfpathcurveto{\pgfqpoint{3.120848in}{0.678749in}}{\pgfqpoint{3.125238in}{0.668150in}}{\pgfqpoint{3.133052in}{0.660337in}}%
\pgfpathcurveto{\pgfqpoint{3.140865in}{0.652523in}}{\pgfqpoint{3.151464in}{0.648133in}}{\pgfqpoint{3.162515in}{0.648133in}}%
\pgfpathclose%
\pgfusepath{stroke,fill}%
\end{pgfscope}%
\begin{pgfscope}%
\pgfpathrectangle{\pgfqpoint{0.787074in}{0.548769in}}{\pgfqpoint{5.062926in}{3.102590in}}%
\pgfusepath{clip}%
\pgfsetbuttcap%
\pgfsetroundjoin%
\definecolor{currentfill}{rgb}{0.121569,0.466667,0.705882}%
\pgfsetfillcolor{currentfill}%
\pgfsetlinewidth{1.003750pt}%
\definecolor{currentstroke}{rgb}{0.121569,0.466667,0.705882}%
\pgfsetstrokecolor{currentstroke}%
\pgfsetdash{}{0pt}%
\pgfpathmoveto{\pgfqpoint{3.318537in}{0.678236in}}%
\pgfpathcurveto{\pgfqpoint{3.329587in}{0.678236in}}{\pgfqpoint{3.340186in}{0.682626in}}{\pgfqpoint{3.348000in}{0.690439in}}%
\pgfpathcurveto{\pgfqpoint{3.355813in}{0.698253in}}{\pgfqpoint{3.360204in}{0.708852in}}{\pgfqpoint{3.360204in}{0.719902in}}%
\pgfpathcurveto{\pgfqpoint{3.360204in}{0.730952in}}{\pgfqpoint{3.355813in}{0.741551in}}{\pgfqpoint{3.348000in}{0.749365in}}%
\pgfpathcurveto{\pgfqpoint{3.340186in}{0.757179in}}{\pgfqpoint{3.329587in}{0.761569in}}{\pgfqpoint{3.318537in}{0.761569in}}%
\pgfpathcurveto{\pgfqpoint{3.307487in}{0.761569in}}{\pgfqpoint{3.296888in}{0.757179in}}{\pgfqpoint{3.289074in}{0.749365in}}%
\pgfpathcurveto{\pgfqpoint{3.281261in}{0.741551in}}{\pgfqpoint{3.276870in}{0.730952in}}{\pgfqpoint{3.276870in}{0.719902in}}%
\pgfpathcurveto{\pgfqpoint{3.276870in}{0.708852in}}{\pgfqpoint{3.281261in}{0.698253in}}{\pgfqpoint{3.289074in}{0.690439in}}%
\pgfpathcurveto{\pgfqpoint{3.296888in}{0.682626in}}{\pgfqpoint{3.307487in}{0.678236in}}{\pgfqpoint{3.318537in}{0.678236in}}%
\pgfpathclose%
\pgfusepath{stroke,fill}%
\end{pgfscope}%
\begin{pgfscope}%
\pgfpathrectangle{\pgfqpoint{0.787074in}{0.548769in}}{\pgfqpoint{5.062926in}{3.102590in}}%
\pgfusepath{clip}%
\pgfsetbuttcap%
\pgfsetroundjoin%
\definecolor{currentfill}{rgb}{1.000000,0.498039,0.054902}%
\pgfsetfillcolor{currentfill}%
\pgfsetlinewidth{1.003750pt}%
\definecolor{currentstroke}{rgb}{1.000000,0.498039,0.054902}%
\pgfsetstrokecolor{currentstroke}%
\pgfsetdash{}{0pt}%
\pgfpathmoveto{\pgfqpoint{4.098649in}{3.011191in}}%
\pgfpathcurveto{\pgfqpoint{4.109699in}{3.011191in}}{\pgfqpoint{4.120298in}{3.015581in}}{\pgfqpoint{4.128112in}{3.023394in}}%
\pgfpathcurveto{\pgfqpoint{4.135925in}{3.031208in}}{\pgfqpoint{4.140315in}{3.041807in}}{\pgfqpoint{4.140315in}{3.052857in}}%
\pgfpathcurveto{\pgfqpoint{4.140315in}{3.063907in}}{\pgfqpoint{4.135925in}{3.074506in}}{\pgfqpoint{4.128112in}{3.082320in}}%
\pgfpathcurveto{\pgfqpoint{4.120298in}{3.090134in}}{\pgfqpoint{4.109699in}{3.094524in}}{\pgfqpoint{4.098649in}{3.094524in}}%
\pgfpathcurveto{\pgfqpoint{4.087599in}{3.094524in}}{\pgfqpoint{4.077000in}{3.090134in}}{\pgfqpoint{4.069186in}{3.082320in}}%
\pgfpathcurveto{\pgfqpoint{4.061372in}{3.074506in}}{\pgfqpoint{4.056982in}{3.063907in}}{\pgfqpoint{4.056982in}{3.052857in}}%
\pgfpathcurveto{\pgfqpoint{4.056982in}{3.041807in}}{\pgfqpoint{4.061372in}{3.031208in}}{\pgfqpoint{4.069186in}{3.023394in}}%
\pgfpathcurveto{\pgfqpoint{4.077000in}{3.015581in}}{\pgfqpoint{4.087599in}{3.011191in}}{\pgfqpoint{4.098649in}{3.011191in}}%
\pgfpathclose%
\pgfusepath{stroke,fill}%
\end{pgfscope}%
\begin{pgfscope}%
\pgfpathrectangle{\pgfqpoint{0.787074in}{0.548769in}}{\pgfqpoint{5.062926in}{3.102590in}}%
\pgfusepath{clip}%
\pgfsetbuttcap%
\pgfsetroundjoin%
\definecolor{currentfill}{rgb}{0.121569,0.466667,0.705882}%
\pgfsetfillcolor{currentfill}%
\pgfsetlinewidth{1.003750pt}%
\definecolor{currentstroke}{rgb}{0.121569,0.466667,0.705882}%
\pgfsetstrokecolor{currentstroke}%
\pgfsetdash{}{0pt}%
\pgfpathmoveto{\pgfqpoint{3.630582in}{0.787280in}}%
\pgfpathcurveto{\pgfqpoint{3.641632in}{0.787280in}}{\pgfqpoint{3.652231in}{0.791670in}}{\pgfqpoint{3.660044in}{0.799484in}}%
\pgfpathcurveto{\pgfqpoint{3.667858in}{0.807297in}}{\pgfqpoint{3.672248in}{0.817896in}}{\pgfqpoint{3.672248in}{0.828946in}}%
\pgfpathcurveto{\pgfqpoint{3.672248in}{0.839997in}}{\pgfqpoint{3.667858in}{0.850596in}}{\pgfqpoint{3.660044in}{0.858409in}}%
\pgfpathcurveto{\pgfqpoint{3.652231in}{0.866223in}}{\pgfqpoint{3.641632in}{0.870613in}}{\pgfqpoint{3.630582in}{0.870613in}}%
\pgfpathcurveto{\pgfqpoint{3.619532in}{0.870613in}}{\pgfqpoint{3.608933in}{0.866223in}}{\pgfqpoint{3.601119in}{0.858409in}}%
\pgfpathcurveto{\pgfqpoint{3.593305in}{0.850596in}}{\pgfqpoint{3.588915in}{0.839997in}}{\pgfqpoint{3.588915in}{0.828946in}}%
\pgfpathcurveto{\pgfqpoint{3.588915in}{0.817896in}}{\pgfqpoint{3.593305in}{0.807297in}}{\pgfqpoint{3.601119in}{0.799484in}}%
\pgfpathcurveto{\pgfqpoint{3.608933in}{0.791670in}}{\pgfqpoint{3.619532in}{0.787280in}}{\pgfqpoint{3.630582in}{0.787280in}}%
\pgfpathclose%
\pgfusepath{stroke,fill}%
\end{pgfscope}%
\begin{pgfscope}%
\pgfpathrectangle{\pgfqpoint{0.787074in}{0.548769in}}{\pgfqpoint{5.062926in}{3.102590in}}%
\pgfusepath{clip}%
\pgfsetbuttcap%
\pgfsetroundjoin%
\definecolor{currentfill}{rgb}{1.000000,0.498039,0.054902}%
\pgfsetfillcolor{currentfill}%
\pgfsetlinewidth{1.003750pt}%
\definecolor{currentstroke}{rgb}{1.000000,0.498039,0.054902}%
\pgfsetstrokecolor{currentstroke}%
\pgfsetdash{}{0pt}%
\pgfpathmoveto{\pgfqpoint{3.474559in}{3.468665in}}%
\pgfpathcurveto{\pgfqpoint{3.485609in}{3.468665in}}{\pgfqpoint{3.496208in}{3.473055in}}{\pgfqpoint{3.504022in}{3.480869in}}%
\pgfpathcurveto{\pgfqpoint{3.511836in}{3.488683in}}{\pgfqpoint{3.516226in}{3.499282in}}{\pgfqpoint{3.516226in}{3.510332in}}%
\pgfpathcurveto{\pgfqpoint{3.516226in}{3.521382in}}{\pgfqpoint{3.511836in}{3.531981in}}{\pgfqpoint{3.504022in}{3.539795in}}%
\pgfpathcurveto{\pgfqpoint{3.496208in}{3.547608in}}{\pgfqpoint{3.485609in}{3.551998in}}{\pgfqpoint{3.474559in}{3.551998in}}%
\pgfpathcurveto{\pgfqpoint{3.463509in}{3.551998in}}{\pgfqpoint{3.452910in}{3.547608in}}{\pgfqpoint{3.445097in}{3.539795in}}%
\pgfpathcurveto{\pgfqpoint{3.437283in}{3.531981in}}{\pgfqpoint{3.432893in}{3.521382in}}{\pgfqpoint{3.432893in}{3.510332in}}%
\pgfpathcurveto{\pgfqpoint{3.432893in}{3.499282in}}{\pgfqpoint{3.437283in}{3.488683in}}{\pgfqpoint{3.445097in}{3.480869in}}%
\pgfpathcurveto{\pgfqpoint{3.452910in}{3.473055in}}{\pgfqpoint{3.463509in}{3.468665in}}{\pgfqpoint{3.474559in}{3.468665in}}%
\pgfpathclose%
\pgfusepath{stroke,fill}%
\end{pgfscope}%
\begin{pgfscope}%
\pgfpathrectangle{\pgfqpoint{0.787074in}{0.548769in}}{\pgfqpoint{5.062926in}{3.102590in}}%
\pgfusepath{clip}%
\pgfsetbuttcap%
\pgfsetroundjoin%
\definecolor{currentfill}{rgb}{0.121569,0.466667,0.705882}%
\pgfsetfillcolor{currentfill}%
\pgfsetlinewidth{1.003750pt}%
\definecolor{currentstroke}{rgb}{0.121569,0.466667,0.705882}%
\pgfsetstrokecolor{currentstroke}%
\pgfsetdash{}{0pt}%
\pgfpathmoveto{\pgfqpoint{3.513565in}{0.649989in}}%
\pgfpathcurveto{\pgfqpoint{3.524615in}{0.649989in}}{\pgfqpoint{3.535214in}{0.654379in}}{\pgfqpoint{3.543028in}{0.662193in}}%
\pgfpathcurveto{\pgfqpoint{3.550841in}{0.670006in}}{\pgfqpoint{3.555232in}{0.680605in}}{\pgfqpoint{3.555232in}{0.691655in}}%
\pgfpathcurveto{\pgfqpoint{3.555232in}{0.702706in}}{\pgfqpoint{3.550841in}{0.713305in}}{\pgfqpoint{3.543028in}{0.721118in}}%
\pgfpathcurveto{\pgfqpoint{3.535214in}{0.728932in}}{\pgfqpoint{3.524615in}{0.733322in}}{\pgfqpoint{3.513565in}{0.733322in}}%
\pgfpathcurveto{\pgfqpoint{3.502515in}{0.733322in}}{\pgfqpoint{3.491916in}{0.728932in}}{\pgfqpoint{3.484102in}{0.721118in}}%
\pgfpathcurveto{\pgfqpoint{3.476288in}{0.713305in}}{\pgfqpoint{3.471898in}{0.702706in}}{\pgfqpoint{3.471898in}{0.691655in}}%
\pgfpathcurveto{\pgfqpoint{3.471898in}{0.680605in}}{\pgfqpoint{3.476288in}{0.670006in}}{\pgfqpoint{3.484102in}{0.662193in}}%
\pgfpathcurveto{\pgfqpoint{3.491916in}{0.654379in}}{\pgfqpoint{3.502515in}{0.649989in}}{\pgfqpoint{3.513565in}{0.649989in}}%
\pgfpathclose%
\pgfusepath{stroke,fill}%
\end{pgfscope}%
\begin{pgfscope}%
\pgfpathrectangle{\pgfqpoint{0.787074in}{0.548769in}}{\pgfqpoint{5.062926in}{3.102590in}}%
\pgfusepath{clip}%
\pgfsetbuttcap%
\pgfsetroundjoin%
\definecolor{currentfill}{rgb}{1.000000,0.498039,0.054902}%
\pgfsetfillcolor{currentfill}%
\pgfsetlinewidth{1.003750pt}%
\definecolor{currentstroke}{rgb}{1.000000,0.498039,0.054902}%
\pgfsetstrokecolor{currentstroke}%
\pgfsetdash{}{0pt}%
\pgfpathmoveto{\pgfqpoint{4.332682in}{2.212517in}}%
\pgfpathcurveto{\pgfqpoint{4.343733in}{2.212517in}}{\pgfqpoint{4.354332in}{2.216907in}}{\pgfqpoint{4.362145in}{2.224721in}}%
\pgfpathcurveto{\pgfqpoint{4.369959in}{2.232535in}}{\pgfqpoint{4.374349in}{2.243134in}}{\pgfqpoint{4.374349in}{2.254184in}}%
\pgfpathcurveto{\pgfqpoint{4.374349in}{2.265234in}}{\pgfqpoint{4.369959in}{2.275833in}}{\pgfqpoint{4.362145in}{2.283647in}}%
\pgfpathcurveto{\pgfqpoint{4.354332in}{2.291460in}}{\pgfqpoint{4.343733in}{2.295850in}}{\pgfqpoint{4.332682in}{2.295850in}}%
\pgfpathcurveto{\pgfqpoint{4.321632in}{2.295850in}}{\pgfqpoint{4.311033in}{2.291460in}}{\pgfqpoint{4.303220in}{2.283647in}}%
\pgfpathcurveto{\pgfqpoint{4.295406in}{2.275833in}}{\pgfqpoint{4.291016in}{2.265234in}}{\pgfqpoint{4.291016in}{2.254184in}}%
\pgfpathcurveto{\pgfqpoint{4.291016in}{2.243134in}}{\pgfqpoint{4.295406in}{2.232535in}}{\pgfqpoint{4.303220in}{2.224721in}}%
\pgfpathcurveto{\pgfqpoint{4.311033in}{2.216907in}}{\pgfqpoint{4.321632in}{2.212517in}}{\pgfqpoint{4.332682in}{2.212517in}}%
\pgfpathclose%
\pgfusepath{stroke,fill}%
\end{pgfscope}%
\begin{pgfscope}%
\pgfpathrectangle{\pgfqpoint{0.787074in}{0.548769in}}{\pgfqpoint{5.062926in}{3.102590in}}%
\pgfusepath{clip}%
\pgfsetbuttcap%
\pgfsetroundjoin%
\definecolor{currentfill}{rgb}{0.121569,0.466667,0.705882}%
\pgfsetfillcolor{currentfill}%
\pgfsetlinewidth{1.003750pt}%
\definecolor{currentstroke}{rgb}{0.121569,0.466667,0.705882}%
\pgfsetstrokecolor{currentstroke}%
\pgfsetdash{}{0pt}%
\pgfpathmoveto{\pgfqpoint{3.162515in}{0.651678in}}%
\pgfpathcurveto{\pgfqpoint{3.173565in}{0.651678in}}{\pgfqpoint{3.184164in}{0.656068in}}{\pgfqpoint{3.191977in}{0.663882in}}%
\pgfpathcurveto{\pgfqpoint{3.199791in}{0.671695in}}{\pgfqpoint{3.204181in}{0.682295in}}{\pgfqpoint{3.204181in}{0.693345in}}%
\pgfpathcurveto{\pgfqpoint{3.204181in}{0.704395in}}{\pgfqpoint{3.199791in}{0.714994in}}{\pgfqpoint{3.191977in}{0.722807in}}%
\pgfpathcurveto{\pgfqpoint{3.184164in}{0.730621in}}{\pgfqpoint{3.173565in}{0.735011in}}{\pgfqpoint{3.162515in}{0.735011in}}%
\pgfpathcurveto{\pgfqpoint{3.151464in}{0.735011in}}{\pgfqpoint{3.140865in}{0.730621in}}{\pgfqpoint{3.133052in}{0.722807in}}%
\pgfpathcurveto{\pgfqpoint{3.125238in}{0.714994in}}{\pgfqpoint{3.120848in}{0.704395in}}{\pgfqpoint{3.120848in}{0.693345in}}%
\pgfpathcurveto{\pgfqpoint{3.120848in}{0.682295in}}{\pgfqpoint{3.125238in}{0.671695in}}{\pgfqpoint{3.133052in}{0.663882in}}%
\pgfpathcurveto{\pgfqpoint{3.140865in}{0.656068in}}{\pgfqpoint{3.151464in}{0.651678in}}{\pgfqpoint{3.162515in}{0.651678in}}%
\pgfpathclose%
\pgfusepath{stroke,fill}%
\end{pgfscope}%
\begin{pgfscope}%
\pgfpathrectangle{\pgfqpoint{0.787074in}{0.548769in}}{\pgfqpoint{5.062926in}{3.102590in}}%
\pgfusepath{clip}%
\pgfsetbuttcap%
\pgfsetroundjoin%
\definecolor{currentfill}{rgb}{1.000000,0.498039,0.054902}%
\pgfsetfillcolor{currentfill}%
\pgfsetlinewidth{1.003750pt}%
\definecolor{currentstroke}{rgb}{1.000000,0.498039,0.054902}%
\pgfsetstrokecolor{currentstroke}%
\pgfsetdash{}{0pt}%
\pgfpathmoveto{\pgfqpoint{3.279531in}{2.584775in}}%
\pgfpathcurveto{\pgfqpoint{3.290581in}{2.584775in}}{\pgfqpoint{3.301180in}{2.589166in}}{\pgfqpoint{3.308994in}{2.596979in}}%
\pgfpathcurveto{\pgfqpoint{3.316808in}{2.604793in}}{\pgfqpoint{3.321198in}{2.615392in}}{\pgfqpoint{3.321198in}{2.626442in}}%
\pgfpathcurveto{\pgfqpoint{3.321198in}{2.637492in}}{\pgfqpoint{3.316808in}{2.648091in}}{\pgfqpoint{3.308994in}{2.655905in}}%
\pgfpathcurveto{\pgfqpoint{3.301180in}{2.663718in}}{\pgfqpoint{3.290581in}{2.668109in}}{\pgfqpoint{3.279531in}{2.668109in}}%
\pgfpathcurveto{\pgfqpoint{3.268481in}{2.668109in}}{\pgfqpoint{3.257882in}{2.663718in}}{\pgfqpoint{3.250069in}{2.655905in}}%
\pgfpathcurveto{\pgfqpoint{3.242255in}{2.648091in}}{\pgfqpoint{3.237865in}{2.637492in}}{\pgfqpoint{3.237865in}{2.626442in}}%
\pgfpathcurveto{\pgfqpoint{3.237865in}{2.615392in}}{\pgfqpoint{3.242255in}{2.604793in}}{\pgfqpoint{3.250069in}{2.596979in}}%
\pgfpathcurveto{\pgfqpoint{3.257882in}{2.589166in}}{\pgfqpoint{3.268481in}{2.584775in}}{\pgfqpoint{3.279531in}{2.584775in}}%
\pgfpathclose%
\pgfusepath{stroke,fill}%
\end{pgfscope}%
\begin{pgfscope}%
\pgfpathrectangle{\pgfqpoint{0.787074in}{0.548769in}}{\pgfqpoint{5.062926in}{3.102590in}}%
\pgfusepath{clip}%
\pgfsetbuttcap%
\pgfsetroundjoin%
\definecolor{currentfill}{rgb}{0.121569,0.466667,0.705882}%
\pgfsetfillcolor{currentfill}%
\pgfsetlinewidth{1.003750pt}%
\definecolor{currentstroke}{rgb}{0.121569,0.466667,0.705882}%
\pgfsetstrokecolor{currentstroke}%
\pgfsetdash{}{0pt}%
\pgfpathmoveto{\pgfqpoint{3.201520in}{0.648158in}}%
\pgfpathcurveto{\pgfqpoint{3.212570in}{0.648158in}}{\pgfqpoint{3.223169in}{0.652548in}}{\pgfqpoint{3.230983in}{0.660362in}}%
\pgfpathcurveto{\pgfqpoint{3.238797in}{0.668176in}}{\pgfqpoint{3.243187in}{0.678775in}}{\pgfqpoint{3.243187in}{0.689825in}}%
\pgfpathcurveto{\pgfqpoint{3.243187in}{0.700875in}}{\pgfqpoint{3.238797in}{0.711474in}}{\pgfqpoint{3.230983in}{0.719288in}}%
\pgfpathcurveto{\pgfqpoint{3.223169in}{0.727101in}}{\pgfqpoint{3.212570in}{0.731492in}}{\pgfqpoint{3.201520in}{0.731492in}}%
\pgfpathcurveto{\pgfqpoint{3.190470in}{0.731492in}}{\pgfqpoint{3.179871in}{0.727101in}}{\pgfqpoint{3.172057in}{0.719288in}}%
\pgfpathcurveto{\pgfqpoint{3.164244in}{0.711474in}}{\pgfqpoint{3.159853in}{0.700875in}}{\pgfqpoint{3.159853in}{0.689825in}}%
\pgfpathcurveto{\pgfqpoint{3.159853in}{0.678775in}}{\pgfqpoint{3.164244in}{0.668176in}}{\pgfqpoint{3.172057in}{0.660362in}}%
\pgfpathcurveto{\pgfqpoint{3.179871in}{0.652548in}}{\pgfqpoint{3.190470in}{0.648158in}}{\pgfqpoint{3.201520in}{0.648158in}}%
\pgfpathclose%
\pgfusepath{stroke,fill}%
\end{pgfscope}%
\begin{pgfscope}%
\pgfpathrectangle{\pgfqpoint{0.787074in}{0.548769in}}{\pgfqpoint{5.062926in}{3.102590in}}%
\pgfusepath{clip}%
\pgfsetbuttcap%
\pgfsetroundjoin%
\definecolor{currentfill}{rgb}{1.000000,0.498039,0.054902}%
\pgfsetfillcolor{currentfill}%
\pgfsetlinewidth{1.003750pt}%
\definecolor{currentstroke}{rgb}{1.000000,0.498039,0.054902}%
\pgfsetstrokecolor{currentstroke}%
\pgfsetdash{}{0pt}%
\pgfpathmoveto{\pgfqpoint{3.201520in}{2.981281in}}%
\pgfpathcurveto{\pgfqpoint{3.212570in}{2.981281in}}{\pgfqpoint{3.223169in}{2.985671in}}{\pgfqpoint{3.230983in}{2.993484in}}%
\pgfpathcurveto{\pgfqpoint{3.238797in}{3.001298in}}{\pgfqpoint{3.243187in}{3.011897in}}{\pgfqpoint{3.243187in}{3.022947in}}%
\pgfpathcurveto{\pgfqpoint{3.243187in}{3.033997in}}{\pgfqpoint{3.238797in}{3.044596in}}{\pgfqpoint{3.230983in}{3.052410in}}%
\pgfpathcurveto{\pgfqpoint{3.223169in}{3.060224in}}{\pgfqpoint{3.212570in}{3.064614in}}{\pgfqpoint{3.201520in}{3.064614in}}%
\pgfpathcurveto{\pgfqpoint{3.190470in}{3.064614in}}{\pgfqpoint{3.179871in}{3.060224in}}{\pgfqpoint{3.172057in}{3.052410in}}%
\pgfpathcurveto{\pgfqpoint{3.164244in}{3.044596in}}{\pgfqpoint{3.159853in}{3.033997in}}{\pgfqpoint{3.159853in}{3.022947in}}%
\pgfpathcurveto{\pgfqpoint{3.159853in}{3.011897in}}{\pgfqpoint{3.164244in}{3.001298in}}{\pgfqpoint{3.172057in}{2.993484in}}%
\pgfpathcurveto{\pgfqpoint{3.179871in}{2.985671in}}{\pgfqpoint{3.190470in}{2.981281in}}{\pgfqpoint{3.201520in}{2.981281in}}%
\pgfpathclose%
\pgfusepath{stroke,fill}%
\end{pgfscope}%
\begin{pgfscope}%
\pgfpathrectangle{\pgfqpoint{0.787074in}{0.548769in}}{\pgfqpoint{5.062926in}{3.102590in}}%
\pgfusepath{clip}%
\pgfsetbuttcap%
\pgfsetroundjoin%
\definecolor{currentfill}{rgb}{0.121569,0.466667,0.705882}%
\pgfsetfillcolor{currentfill}%
\pgfsetlinewidth{1.003750pt}%
\definecolor{currentstroke}{rgb}{0.121569,0.466667,0.705882}%
\pgfsetstrokecolor{currentstroke}%
\pgfsetdash{}{0pt}%
\pgfpathmoveto{\pgfqpoint{3.240526in}{0.648132in}}%
\pgfpathcurveto{\pgfqpoint{3.251576in}{0.648132in}}{\pgfqpoint{3.262175in}{0.652522in}}{\pgfqpoint{3.269989in}{0.660336in}}%
\pgfpathcurveto{\pgfqpoint{3.277802in}{0.668149in}}{\pgfqpoint{3.282192in}{0.678749in}}{\pgfqpoint{3.282192in}{0.689799in}}%
\pgfpathcurveto{\pgfqpoint{3.282192in}{0.700849in}}{\pgfqpoint{3.277802in}{0.711448in}}{\pgfqpoint{3.269989in}{0.719261in}}%
\pgfpathcurveto{\pgfqpoint{3.262175in}{0.727075in}}{\pgfqpoint{3.251576in}{0.731465in}}{\pgfqpoint{3.240526in}{0.731465in}}%
\pgfpathcurveto{\pgfqpoint{3.229476in}{0.731465in}}{\pgfqpoint{3.218877in}{0.727075in}}{\pgfqpoint{3.211063in}{0.719261in}}%
\pgfpathcurveto{\pgfqpoint{3.203249in}{0.711448in}}{\pgfqpoint{3.198859in}{0.700849in}}{\pgfqpoint{3.198859in}{0.689799in}}%
\pgfpathcurveto{\pgfqpoint{3.198859in}{0.678749in}}{\pgfqpoint{3.203249in}{0.668149in}}{\pgfqpoint{3.211063in}{0.660336in}}%
\pgfpathcurveto{\pgfqpoint{3.218877in}{0.652522in}}{\pgfqpoint{3.229476in}{0.648132in}}{\pgfqpoint{3.240526in}{0.648132in}}%
\pgfpathclose%
\pgfusepath{stroke,fill}%
\end{pgfscope}%
\begin{pgfscope}%
\pgfpathrectangle{\pgfqpoint{0.787074in}{0.548769in}}{\pgfqpoint{5.062926in}{3.102590in}}%
\pgfusepath{clip}%
\pgfsetbuttcap%
\pgfsetroundjoin%
\definecolor{currentfill}{rgb}{1.000000,0.498039,0.054902}%
\pgfsetfillcolor{currentfill}%
\pgfsetlinewidth{1.003750pt}%
\definecolor{currentstroke}{rgb}{1.000000,0.498039,0.054902}%
\pgfsetstrokecolor{currentstroke}%
\pgfsetdash{}{0pt}%
\pgfpathmoveto{\pgfqpoint{3.591576in}{2.254043in}}%
\pgfpathcurveto{\pgfqpoint{3.602626in}{2.254043in}}{\pgfqpoint{3.613225in}{2.258433in}}{\pgfqpoint{3.621039in}{2.266247in}}%
\pgfpathcurveto{\pgfqpoint{3.628852in}{2.274060in}}{\pgfqpoint{3.633243in}{2.284659in}}{\pgfqpoint{3.633243in}{2.295709in}}%
\pgfpathcurveto{\pgfqpoint{3.633243in}{2.306759in}}{\pgfqpoint{3.628852in}{2.317359in}}{\pgfqpoint{3.621039in}{2.325172in}}%
\pgfpathcurveto{\pgfqpoint{3.613225in}{2.332986in}}{\pgfqpoint{3.602626in}{2.337376in}}{\pgfqpoint{3.591576in}{2.337376in}}%
\pgfpathcurveto{\pgfqpoint{3.580526in}{2.337376in}}{\pgfqpoint{3.569927in}{2.332986in}}{\pgfqpoint{3.562113in}{2.325172in}}%
\pgfpathcurveto{\pgfqpoint{3.554300in}{2.317359in}}{\pgfqpoint{3.549909in}{2.306759in}}{\pgfqpoint{3.549909in}{2.295709in}}%
\pgfpathcurveto{\pgfqpoint{3.549909in}{2.284659in}}{\pgfqpoint{3.554300in}{2.274060in}}{\pgfqpoint{3.562113in}{2.266247in}}%
\pgfpathcurveto{\pgfqpoint{3.569927in}{2.258433in}}{\pgfqpoint{3.580526in}{2.254043in}}{\pgfqpoint{3.591576in}{2.254043in}}%
\pgfpathclose%
\pgfusepath{stroke,fill}%
\end{pgfscope}%
\begin{pgfscope}%
\pgfpathrectangle{\pgfqpoint{0.787074in}{0.548769in}}{\pgfqpoint{5.062926in}{3.102590in}}%
\pgfusepath{clip}%
\pgfsetbuttcap%
\pgfsetroundjoin%
\definecolor{currentfill}{rgb}{1.000000,0.498039,0.054902}%
\pgfsetfillcolor{currentfill}%
\pgfsetlinewidth{1.003750pt}%
\definecolor{currentstroke}{rgb}{1.000000,0.498039,0.054902}%
\pgfsetstrokecolor{currentstroke}%
\pgfsetdash{}{0pt}%
\pgfpathmoveto{\pgfqpoint{3.357543in}{2.386212in}}%
\pgfpathcurveto{\pgfqpoint{3.368593in}{2.386212in}}{\pgfqpoint{3.379192in}{2.390603in}}{\pgfqpoint{3.387005in}{2.398416in}}%
\pgfpathcurveto{\pgfqpoint{3.394819in}{2.406230in}}{\pgfqpoint{3.399209in}{2.416829in}}{\pgfqpoint{3.399209in}{2.427879in}}%
\pgfpathcurveto{\pgfqpoint{3.399209in}{2.438929in}}{\pgfqpoint{3.394819in}{2.449528in}}{\pgfqpoint{3.387005in}{2.457342in}}%
\pgfpathcurveto{\pgfqpoint{3.379192in}{2.465155in}}{\pgfqpoint{3.368593in}{2.469546in}}{\pgfqpoint{3.357543in}{2.469546in}}%
\pgfpathcurveto{\pgfqpoint{3.346492in}{2.469546in}}{\pgfqpoint{3.335893in}{2.465155in}}{\pgfqpoint{3.328080in}{2.457342in}}%
\pgfpathcurveto{\pgfqpoint{3.320266in}{2.449528in}}{\pgfqpoint{3.315876in}{2.438929in}}{\pgfqpoint{3.315876in}{2.427879in}}%
\pgfpathcurveto{\pgfqpoint{3.315876in}{2.416829in}}{\pgfqpoint{3.320266in}{2.406230in}}{\pgfqpoint{3.328080in}{2.398416in}}%
\pgfpathcurveto{\pgfqpoint{3.335893in}{2.390603in}}{\pgfqpoint{3.346492in}{2.386212in}}{\pgfqpoint{3.357543in}{2.386212in}}%
\pgfpathclose%
\pgfusepath{stroke,fill}%
\end{pgfscope}%
\begin{pgfscope}%
\pgfpathrectangle{\pgfqpoint{0.787074in}{0.548769in}}{\pgfqpoint{5.062926in}{3.102590in}}%
\pgfusepath{clip}%
\pgfsetbuttcap%
\pgfsetroundjoin%
\definecolor{currentfill}{rgb}{0.121569,0.466667,0.705882}%
\pgfsetfillcolor{currentfill}%
\pgfsetlinewidth{1.003750pt}%
\definecolor{currentstroke}{rgb}{0.121569,0.466667,0.705882}%
\pgfsetstrokecolor{currentstroke}%
\pgfsetdash{}{0pt}%
\pgfpathmoveto{\pgfqpoint{3.513565in}{0.648147in}}%
\pgfpathcurveto{\pgfqpoint{3.524615in}{0.648147in}}{\pgfqpoint{3.535214in}{0.652537in}}{\pgfqpoint{3.543028in}{0.660351in}}%
\pgfpathcurveto{\pgfqpoint{3.550841in}{0.668165in}}{\pgfqpoint{3.555232in}{0.678764in}}{\pgfqpoint{3.555232in}{0.689814in}}%
\pgfpathcurveto{\pgfqpoint{3.555232in}{0.700864in}}{\pgfqpoint{3.550841in}{0.711463in}}{\pgfqpoint{3.543028in}{0.719277in}}%
\pgfpathcurveto{\pgfqpoint{3.535214in}{0.727090in}}{\pgfqpoint{3.524615in}{0.731480in}}{\pgfqpoint{3.513565in}{0.731480in}}%
\pgfpathcurveto{\pgfqpoint{3.502515in}{0.731480in}}{\pgfqpoint{3.491916in}{0.727090in}}{\pgfqpoint{3.484102in}{0.719277in}}%
\pgfpathcurveto{\pgfqpoint{3.476288in}{0.711463in}}{\pgfqpoint{3.471898in}{0.700864in}}{\pgfqpoint{3.471898in}{0.689814in}}%
\pgfpathcurveto{\pgfqpoint{3.471898in}{0.678764in}}{\pgfqpoint{3.476288in}{0.668165in}}{\pgfqpoint{3.484102in}{0.660351in}}%
\pgfpathcurveto{\pgfqpoint{3.491916in}{0.652537in}}{\pgfqpoint{3.502515in}{0.648147in}}{\pgfqpoint{3.513565in}{0.648147in}}%
\pgfpathclose%
\pgfusepath{stroke,fill}%
\end{pgfscope}%
\begin{pgfscope}%
\pgfpathrectangle{\pgfqpoint{0.787074in}{0.548769in}}{\pgfqpoint{5.062926in}{3.102590in}}%
\pgfusepath{clip}%
\pgfsetbuttcap%
\pgfsetroundjoin%
\definecolor{currentfill}{rgb}{1.000000,0.498039,0.054902}%
\pgfsetfillcolor{currentfill}%
\pgfsetlinewidth{1.003750pt}%
\definecolor{currentstroke}{rgb}{1.000000,0.498039,0.054902}%
\pgfsetstrokecolor{currentstroke}%
\pgfsetdash{}{0pt}%
\pgfpathmoveto{\pgfqpoint{4.059643in}{2.590063in}}%
\pgfpathcurveto{\pgfqpoint{4.070693in}{2.590063in}}{\pgfqpoint{4.081292in}{2.594453in}}{\pgfqpoint{4.089106in}{2.602267in}}%
\pgfpathcurveto{\pgfqpoint{4.096920in}{2.610081in}}{\pgfqpoint{4.101310in}{2.620680in}}{\pgfqpoint{4.101310in}{2.631730in}}%
\pgfpathcurveto{\pgfqpoint{4.101310in}{2.642780in}}{\pgfqpoint{4.096920in}{2.653379in}}{\pgfqpoint{4.089106in}{2.661193in}}%
\pgfpathcurveto{\pgfqpoint{4.081292in}{2.669006in}}{\pgfqpoint{4.070693in}{2.673397in}}{\pgfqpoint{4.059643in}{2.673397in}}%
\pgfpathcurveto{\pgfqpoint{4.048593in}{2.673397in}}{\pgfqpoint{4.037994in}{2.669006in}}{\pgfqpoint{4.030180in}{2.661193in}}%
\pgfpathcurveto{\pgfqpoint{4.022367in}{2.653379in}}{\pgfqpoint{4.017977in}{2.642780in}}{\pgfqpoint{4.017977in}{2.631730in}}%
\pgfpathcurveto{\pgfqpoint{4.017977in}{2.620680in}}{\pgfqpoint{4.022367in}{2.610081in}}{\pgfqpoint{4.030180in}{2.602267in}}%
\pgfpathcurveto{\pgfqpoint{4.037994in}{2.594453in}}{\pgfqpoint{4.048593in}{2.590063in}}{\pgfqpoint{4.059643in}{2.590063in}}%
\pgfpathclose%
\pgfusepath{stroke,fill}%
\end{pgfscope}%
\begin{pgfscope}%
\pgfpathrectangle{\pgfqpoint{0.787074in}{0.548769in}}{\pgfqpoint{5.062926in}{3.102590in}}%
\pgfusepath{clip}%
\pgfsetbuttcap%
\pgfsetroundjoin%
\definecolor{currentfill}{rgb}{1.000000,0.498039,0.054902}%
\pgfsetfillcolor{currentfill}%
\pgfsetlinewidth{1.003750pt}%
\definecolor{currentstroke}{rgb}{1.000000,0.498039,0.054902}%
\pgfsetstrokecolor{currentstroke}%
\pgfsetdash{}{0pt}%
\pgfpathmoveto{\pgfqpoint{3.903621in}{2.389366in}}%
\pgfpathcurveto{\pgfqpoint{3.914671in}{2.389366in}}{\pgfqpoint{3.925270in}{2.393756in}}{\pgfqpoint{3.933084in}{2.401570in}}%
\pgfpathcurveto{\pgfqpoint{3.940897in}{2.409383in}}{\pgfqpoint{3.945288in}{2.419982in}}{\pgfqpoint{3.945288in}{2.431032in}}%
\pgfpathcurveto{\pgfqpoint{3.945288in}{2.442082in}}{\pgfqpoint{3.940897in}{2.452682in}}{\pgfqpoint{3.933084in}{2.460495in}}%
\pgfpathcurveto{\pgfqpoint{3.925270in}{2.468309in}}{\pgfqpoint{3.914671in}{2.472699in}}{\pgfqpoint{3.903621in}{2.472699in}}%
\pgfpathcurveto{\pgfqpoint{3.892571in}{2.472699in}}{\pgfqpoint{3.881972in}{2.468309in}}{\pgfqpoint{3.874158in}{2.460495in}}%
\pgfpathcurveto{\pgfqpoint{3.866344in}{2.452682in}}{\pgfqpoint{3.861954in}{2.442082in}}{\pgfqpoint{3.861954in}{2.431032in}}%
\pgfpathcurveto{\pgfqpoint{3.861954in}{2.419982in}}{\pgfqpoint{3.866344in}{2.409383in}}{\pgfqpoint{3.874158in}{2.401570in}}%
\pgfpathcurveto{\pgfqpoint{3.881972in}{2.393756in}}{\pgfqpoint{3.892571in}{2.389366in}}{\pgfqpoint{3.903621in}{2.389366in}}%
\pgfpathclose%
\pgfusepath{stroke,fill}%
\end{pgfscope}%
\begin{pgfscope}%
\pgfpathrectangle{\pgfqpoint{0.787074in}{0.548769in}}{\pgfqpoint{5.062926in}{3.102590in}}%
\pgfusepath{clip}%
\pgfsetbuttcap%
\pgfsetroundjoin%
\definecolor{currentfill}{rgb}{0.121569,0.466667,0.705882}%
\pgfsetfillcolor{currentfill}%
\pgfsetlinewidth{1.003750pt}%
\definecolor{currentstroke}{rgb}{0.121569,0.466667,0.705882}%
\pgfsetstrokecolor{currentstroke}%
\pgfsetdash{}{0pt}%
\pgfpathmoveto{\pgfqpoint{3.591576in}{0.658967in}}%
\pgfpathcurveto{\pgfqpoint{3.602626in}{0.658967in}}{\pgfqpoint{3.613225in}{0.663357in}}{\pgfqpoint{3.621039in}{0.671170in}}%
\pgfpathcurveto{\pgfqpoint{3.628852in}{0.678984in}}{\pgfqpoint{3.633243in}{0.689583in}}{\pgfqpoint{3.633243in}{0.700633in}}%
\pgfpathcurveto{\pgfqpoint{3.633243in}{0.711683in}}{\pgfqpoint{3.628852in}{0.722282in}}{\pgfqpoint{3.621039in}{0.730096in}}%
\pgfpathcurveto{\pgfqpoint{3.613225in}{0.737910in}}{\pgfqpoint{3.602626in}{0.742300in}}{\pgfqpoint{3.591576in}{0.742300in}}%
\pgfpathcurveto{\pgfqpoint{3.580526in}{0.742300in}}{\pgfqpoint{3.569927in}{0.737910in}}{\pgfqpoint{3.562113in}{0.730096in}}%
\pgfpathcurveto{\pgfqpoint{3.554300in}{0.722282in}}{\pgfqpoint{3.549909in}{0.711683in}}{\pgfqpoint{3.549909in}{0.700633in}}%
\pgfpathcurveto{\pgfqpoint{3.549909in}{0.689583in}}{\pgfqpoint{3.554300in}{0.678984in}}{\pgfqpoint{3.562113in}{0.671170in}}%
\pgfpathcurveto{\pgfqpoint{3.569927in}{0.663357in}}{\pgfqpoint{3.580526in}{0.658967in}}{\pgfqpoint{3.591576in}{0.658967in}}%
\pgfpathclose%
\pgfusepath{stroke,fill}%
\end{pgfscope}%
\begin{pgfscope}%
\pgfpathrectangle{\pgfqpoint{0.787074in}{0.548769in}}{\pgfqpoint{5.062926in}{3.102590in}}%
\pgfusepath{clip}%
\pgfsetbuttcap%
\pgfsetroundjoin%
\definecolor{currentfill}{rgb}{1.000000,0.498039,0.054902}%
\pgfsetfillcolor{currentfill}%
\pgfsetlinewidth{1.003750pt}%
\definecolor{currentstroke}{rgb}{1.000000,0.498039,0.054902}%
\pgfsetstrokecolor{currentstroke}%
\pgfsetdash{}{0pt}%
\pgfpathmoveto{\pgfqpoint{3.825610in}{2.623396in}}%
\pgfpathcurveto{\pgfqpoint{3.836660in}{2.623396in}}{\pgfqpoint{3.847259in}{2.627787in}}{\pgfqpoint{3.855072in}{2.635600in}}%
\pgfpathcurveto{\pgfqpoint{3.862886in}{2.643414in}}{\pgfqpoint{3.867276in}{2.654013in}}{\pgfqpoint{3.867276in}{2.665063in}}%
\pgfpathcurveto{\pgfqpoint{3.867276in}{2.676113in}}{\pgfqpoint{3.862886in}{2.686712in}}{\pgfqpoint{3.855072in}{2.694526in}}%
\pgfpathcurveto{\pgfqpoint{3.847259in}{2.702339in}}{\pgfqpoint{3.836660in}{2.706730in}}{\pgfqpoint{3.825610in}{2.706730in}}%
\pgfpathcurveto{\pgfqpoint{3.814560in}{2.706730in}}{\pgfqpoint{3.803960in}{2.702339in}}{\pgfqpoint{3.796147in}{2.694526in}}%
\pgfpathcurveto{\pgfqpoint{3.788333in}{2.686712in}}{\pgfqpoint{3.783943in}{2.676113in}}{\pgfqpoint{3.783943in}{2.665063in}}%
\pgfpathcurveto{\pgfqpoint{3.783943in}{2.654013in}}{\pgfqpoint{3.788333in}{2.643414in}}{\pgfqpoint{3.796147in}{2.635600in}}%
\pgfpathcurveto{\pgfqpoint{3.803960in}{2.627787in}}{\pgfqpoint{3.814560in}{2.623396in}}{\pgfqpoint{3.825610in}{2.623396in}}%
\pgfpathclose%
\pgfusepath{stroke,fill}%
\end{pgfscope}%
\begin{pgfscope}%
\pgfpathrectangle{\pgfqpoint{0.787074in}{0.548769in}}{\pgfqpoint{5.062926in}{3.102590in}}%
\pgfusepath{clip}%
\pgfsetbuttcap%
\pgfsetroundjoin%
\definecolor{currentfill}{rgb}{1.000000,0.498039,0.054902}%
\pgfsetfillcolor{currentfill}%
\pgfsetlinewidth{1.003750pt}%
\definecolor{currentstroke}{rgb}{1.000000,0.498039,0.054902}%
\pgfsetstrokecolor{currentstroke}%
\pgfsetdash{}{0pt}%
\pgfpathmoveto{\pgfqpoint{3.513565in}{2.816554in}}%
\pgfpathcurveto{\pgfqpoint{3.524615in}{2.816554in}}{\pgfqpoint{3.535214in}{2.820944in}}{\pgfqpoint{3.543028in}{2.828758in}}%
\pgfpathcurveto{\pgfqpoint{3.550841in}{2.836572in}}{\pgfqpoint{3.555232in}{2.847171in}}{\pgfqpoint{3.555232in}{2.858221in}}%
\pgfpathcurveto{\pgfqpoint{3.555232in}{2.869271in}}{\pgfqpoint{3.550841in}{2.879870in}}{\pgfqpoint{3.543028in}{2.887684in}}%
\pgfpathcurveto{\pgfqpoint{3.535214in}{2.895497in}}{\pgfqpoint{3.524615in}{2.899887in}}{\pgfqpoint{3.513565in}{2.899887in}}%
\pgfpathcurveto{\pgfqpoint{3.502515in}{2.899887in}}{\pgfqpoint{3.491916in}{2.895497in}}{\pgfqpoint{3.484102in}{2.887684in}}%
\pgfpathcurveto{\pgfqpoint{3.476288in}{2.879870in}}{\pgfqpoint{3.471898in}{2.869271in}}{\pgfqpoint{3.471898in}{2.858221in}}%
\pgfpathcurveto{\pgfqpoint{3.471898in}{2.847171in}}{\pgfqpoint{3.476288in}{2.836572in}}{\pgfqpoint{3.484102in}{2.828758in}}%
\pgfpathcurveto{\pgfqpoint{3.491916in}{2.820944in}}{\pgfqpoint{3.502515in}{2.816554in}}{\pgfqpoint{3.513565in}{2.816554in}}%
\pgfpathclose%
\pgfusepath{stroke,fill}%
\end{pgfscope}%
\begin{pgfscope}%
\pgfpathrectangle{\pgfqpoint{0.787074in}{0.548769in}}{\pgfqpoint{5.062926in}{3.102590in}}%
\pgfusepath{clip}%
\pgfsetbuttcap%
\pgfsetroundjoin%
\definecolor{currentfill}{rgb}{1.000000,0.498039,0.054902}%
\pgfsetfillcolor{currentfill}%
\pgfsetlinewidth{1.003750pt}%
\definecolor{currentstroke}{rgb}{1.000000,0.498039,0.054902}%
\pgfsetstrokecolor{currentstroke}%
\pgfsetdash{}{0pt}%
\pgfpathmoveto{\pgfqpoint{3.435554in}{2.998783in}}%
\pgfpathcurveto{\pgfqpoint{3.446604in}{2.998783in}}{\pgfqpoint{3.457203in}{3.003173in}}{\pgfqpoint{3.465016in}{3.010987in}}%
\pgfpathcurveto{\pgfqpoint{3.472830in}{3.018800in}}{\pgfqpoint{3.477220in}{3.029399in}}{\pgfqpoint{3.477220in}{3.040450in}}%
\pgfpathcurveto{\pgfqpoint{3.477220in}{3.051500in}}{\pgfqpoint{3.472830in}{3.062099in}}{\pgfqpoint{3.465016in}{3.069912in}}%
\pgfpathcurveto{\pgfqpoint{3.457203in}{3.077726in}}{\pgfqpoint{3.446604in}{3.082116in}}{\pgfqpoint{3.435554in}{3.082116in}}%
\pgfpathcurveto{\pgfqpoint{3.424504in}{3.082116in}}{\pgfqpoint{3.413905in}{3.077726in}}{\pgfqpoint{3.406091in}{3.069912in}}%
\pgfpathcurveto{\pgfqpoint{3.398277in}{3.062099in}}{\pgfqpoint{3.393887in}{3.051500in}}{\pgfqpoint{3.393887in}{3.040450in}}%
\pgfpathcurveto{\pgfqpoint{3.393887in}{3.029399in}}{\pgfqpoint{3.398277in}{3.018800in}}{\pgfqpoint{3.406091in}{3.010987in}}%
\pgfpathcurveto{\pgfqpoint{3.413905in}{3.003173in}}{\pgfqpoint{3.424504in}{2.998783in}}{\pgfqpoint{3.435554in}{2.998783in}}%
\pgfpathclose%
\pgfusepath{stroke,fill}%
\end{pgfscope}%
\begin{pgfscope}%
\pgfpathrectangle{\pgfqpoint{0.787074in}{0.548769in}}{\pgfqpoint{5.062926in}{3.102590in}}%
\pgfusepath{clip}%
\pgfsetbuttcap%
\pgfsetroundjoin%
\definecolor{currentfill}{rgb}{1.000000,0.498039,0.054902}%
\pgfsetfillcolor{currentfill}%
\pgfsetlinewidth{1.003750pt}%
\definecolor{currentstroke}{rgb}{1.000000,0.498039,0.054902}%
\pgfsetstrokecolor{currentstroke}%
\pgfsetdash{}{0pt}%
\pgfpathmoveto{\pgfqpoint{3.747598in}{2.057042in}}%
\pgfpathcurveto{\pgfqpoint{3.758649in}{2.057042in}}{\pgfqpoint{3.769248in}{2.061432in}}{\pgfqpoint{3.777061in}{2.069246in}}%
\pgfpathcurveto{\pgfqpoint{3.784875in}{2.077059in}}{\pgfqpoint{3.789265in}{2.087659in}}{\pgfqpoint{3.789265in}{2.098709in}}%
\pgfpathcurveto{\pgfqpoint{3.789265in}{2.109759in}}{\pgfqpoint{3.784875in}{2.120358in}}{\pgfqpoint{3.777061in}{2.128171in}}%
\pgfpathcurveto{\pgfqpoint{3.769248in}{2.135985in}}{\pgfqpoint{3.758649in}{2.140375in}}{\pgfqpoint{3.747598in}{2.140375in}}%
\pgfpathcurveto{\pgfqpoint{3.736548in}{2.140375in}}{\pgfqpoint{3.725949in}{2.135985in}}{\pgfqpoint{3.718136in}{2.128171in}}%
\pgfpathcurveto{\pgfqpoint{3.710322in}{2.120358in}}{\pgfqpoint{3.705932in}{2.109759in}}{\pgfqpoint{3.705932in}{2.098709in}}%
\pgfpathcurveto{\pgfqpoint{3.705932in}{2.087659in}}{\pgfqpoint{3.710322in}{2.077059in}}{\pgfqpoint{3.718136in}{2.069246in}}%
\pgfpathcurveto{\pgfqpoint{3.725949in}{2.061432in}}{\pgfqpoint{3.736548in}{2.057042in}}{\pgfqpoint{3.747598in}{2.057042in}}%
\pgfpathclose%
\pgfusepath{stroke,fill}%
\end{pgfscope}%
\begin{pgfscope}%
\pgfpathrectangle{\pgfqpoint{0.787074in}{0.548769in}}{\pgfqpoint{5.062926in}{3.102590in}}%
\pgfusepath{clip}%
\pgfsetbuttcap%
\pgfsetroundjoin%
\definecolor{currentfill}{rgb}{1.000000,0.498039,0.054902}%
\pgfsetfillcolor{currentfill}%
\pgfsetlinewidth{1.003750pt}%
\definecolor{currentstroke}{rgb}{1.000000,0.498039,0.054902}%
\pgfsetstrokecolor{currentstroke}%
\pgfsetdash{}{0pt}%
\pgfpathmoveto{\pgfqpoint{3.903621in}{3.349495in}}%
\pgfpathcurveto{\pgfqpoint{3.914671in}{3.349495in}}{\pgfqpoint{3.925270in}{3.353886in}}{\pgfqpoint{3.933084in}{3.361699in}}%
\pgfpathcurveto{\pgfqpoint{3.940897in}{3.369513in}}{\pgfqpoint{3.945288in}{3.380112in}}{\pgfqpoint{3.945288in}{3.391162in}}%
\pgfpathcurveto{\pgfqpoint{3.945288in}{3.402212in}}{\pgfqpoint{3.940897in}{3.412811in}}{\pgfqpoint{3.933084in}{3.420625in}}%
\pgfpathcurveto{\pgfqpoint{3.925270in}{3.428438in}}{\pgfqpoint{3.914671in}{3.432829in}}{\pgfqpoint{3.903621in}{3.432829in}}%
\pgfpathcurveto{\pgfqpoint{3.892571in}{3.432829in}}{\pgfqpoint{3.881972in}{3.428438in}}{\pgfqpoint{3.874158in}{3.420625in}}%
\pgfpathcurveto{\pgfqpoint{3.866344in}{3.412811in}}{\pgfqpoint{3.861954in}{3.402212in}}{\pgfqpoint{3.861954in}{3.391162in}}%
\pgfpathcurveto{\pgfqpoint{3.861954in}{3.380112in}}{\pgfqpoint{3.866344in}{3.369513in}}{\pgfqpoint{3.874158in}{3.361699in}}%
\pgfpathcurveto{\pgfqpoint{3.881972in}{3.353886in}}{\pgfqpoint{3.892571in}{3.349495in}}{\pgfqpoint{3.903621in}{3.349495in}}%
\pgfpathclose%
\pgfusepath{stroke,fill}%
\end{pgfscope}%
\begin{pgfscope}%
\pgfpathrectangle{\pgfqpoint{0.787074in}{0.548769in}}{\pgfqpoint{5.062926in}{3.102590in}}%
\pgfusepath{clip}%
\pgfsetbuttcap%
\pgfsetroundjoin%
\definecolor{currentfill}{rgb}{1.000000,0.498039,0.054902}%
\pgfsetfillcolor{currentfill}%
\pgfsetlinewidth{1.003750pt}%
\definecolor{currentstroke}{rgb}{1.000000,0.498039,0.054902}%
\pgfsetstrokecolor{currentstroke}%
\pgfsetdash{}{0pt}%
\pgfpathmoveto{\pgfqpoint{3.513565in}{2.864404in}}%
\pgfpathcurveto{\pgfqpoint{3.524615in}{2.864404in}}{\pgfqpoint{3.535214in}{2.868794in}}{\pgfqpoint{3.543028in}{2.876608in}}%
\pgfpathcurveto{\pgfqpoint{3.550841in}{2.884422in}}{\pgfqpoint{3.555232in}{2.895021in}}{\pgfqpoint{3.555232in}{2.906071in}}%
\pgfpathcurveto{\pgfqpoint{3.555232in}{2.917121in}}{\pgfqpoint{3.550841in}{2.927720in}}{\pgfqpoint{3.543028in}{2.935534in}}%
\pgfpathcurveto{\pgfqpoint{3.535214in}{2.943347in}}{\pgfqpoint{3.524615in}{2.947737in}}{\pgfqpoint{3.513565in}{2.947737in}}%
\pgfpathcurveto{\pgfqpoint{3.502515in}{2.947737in}}{\pgfqpoint{3.491916in}{2.943347in}}{\pgfqpoint{3.484102in}{2.935534in}}%
\pgfpathcurveto{\pgfqpoint{3.476288in}{2.927720in}}{\pgfqpoint{3.471898in}{2.917121in}}{\pgfqpoint{3.471898in}{2.906071in}}%
\pgfpathcurveto{\pgfqpoint{3.471898in}{2.895021in}}{\pgfqpoint{3.476288in}{2.884422in}}{\pgfqpoint{3.484102in}{2.876608in}}%
\pgfpathcurveto{\pgfqpoint{3.491916in}{2.868794in}}{\pgfqpoint{3.502515in}{2.864404in}}{\pgfqpoint{3.513565in}{2.864404in}}%
\pgfpathclose%
\pgfusepath{stroke,fill}%
\end{pgfscope}%
\begin{pgfscope}%
\pgfpathrectangle{\pgfqpoint{0.787074in}{0.548769in}}{\pgfqpoint{5.062926in}{3.102590in}}%
\pgfusepath{clip}%
\pgfsetbuttcap%
\pgfsetroundjoin%
\definecolor{currentfill}{rgb}{0.121569,0.466667,0.705882}%
\pgfsetfillcolor{currentfill}%
\pgfsetlinewidth{1.003750pt}%
\definecolor{currentstroke}{rgb}{0.121569,0.466667,0.705882}%
\pgfsetstrokecolor{currentstroke}%
\pgfsetdash{}{0pt}%
\pgfpathmoveto{\pgfqpoint{3.435554in}{0.648143in}}%
\pgfpathcurveto{\pgfqpoint{3.446604in}{0.648143in}}{\pgfqpoint{3.457203in}{0.652534in}}{\pgfqpoint{3.465016in}{0.660347in}}%
\pgfpathcurveto{\pgfqpoint{3.472830in}{0.668161in}}{\pgfqpoint{3.477220in}{0.678760in}}{\pgfqpoint{3.477220in}{0.689810in}}%
\pgfpathcurveto{\pgfqpoint{3.477220in}{0.700860in}}{\pgfqpoint{3.472830in}{0.711459in}}{\pgfqpoint{3.465016in}{0.719273in}}%
\pgfpathcurveto{\pgfqpoint{3.457203in}{0.727086in}}{\pgfqpoint{3.446604in}{0.731477in}}{\pgfqpoint{3.435554in}{0.731477in}}%
\pgfpathcurveto{\pgfqpoint{3.424504in}{0.731477in}}{\pgfqpoint{3.413905in}{0.727086in}}{\pgfqpoint{3.406091in}{0.719273in}}%
\pgfpathcurveto{\pgfqpoint{3.398277in}{0.711459in}}{\pgfqpoint{3.393887in}{0.700860in}}{\pgfqpoint{3.393887in}{0.689810in}}%
\pgfpathcurveto{\pgfqpoint{3.393887in}{0.678760in}}{\pgfqpoint{3.398277in}{0.668161in}}{\pgfqpoint{3.406091in}{0.660347in}}%
\pgfpathcurveto{\pgfqpoint{3.413905in}{0.652534in}}{\pgfqpoint{3.424504in}{0.648143in}}{\pgfqpoint{3.435554in}{0.648143in}}%
\pgfpathclose%
\pgfusepath{stroke,fill}%
\end{pgfscope}%
\begin{pgfscope}%
\pgfpathrectangle{\pgfqpoint{0.787074in}{0.548769in}}{\pgfqpoint{5.062926in}{3.102590in}}%
\pgfusepath{clip}%
\pgfsetbuttcap%
\pgfsetroundjoin%
\definecolor{currentfill}{rgb}{0.121569,0.466667,0.705882}%
\pgfsetfillcolor{currentfill}%
\pgfsetlinewidth{1.003750pt}%
\definecolor{currentstroke}{rgb}{0.121569,0.466667,0.705882}%
\pgfsetstrokecolor{currentstroke}%
\pgfsetdash{}{0pt}%
\pgfpathmoveto{\pgfqpoint{3.279531in}{0.648148in}}%
\pgfpathcurveto{\pgfqpoint{3.290581in}{0.648148in}}{\pgfqpoint{3.301180in}{0.652538in}}{\pgfqpoint{3.308994in}{0.660352in}}%
\pgfpathcurveto{\pgfqpoint{3.316808in}{0.668165in}}{\pgfqpoint{3.321198in}{0.678764in}}{\pgfqpoint{3.321198in}{0.689815in}}%
\pgfpathcurveto{\pgfqpoint{3.321198in}{0.700865in}}{\pgfqpoint{3.316808in}{0.711464in}}{\pgfqpoint{3.308994in}{0.719277in}}%
\pgfpathcurveto{\pgfqpoint{3.301180in}{0.727091in}}{\pgfqpoint{3.290581in}{0.731481in}}{\pgfqpoint{3.279531in}{0.731481in}}%
\pgfpathcurveto{\pgfqpoint{3.268481in}{0.731481in}}{\pgfqpoint{3.257882in}{0.727091in}}{\pgfqpoint{3.250069in}{0.719277in}}%
\pgfpathcurveto{\pgfqpoint{3.242255in}{0.711464in}}{\pgfqpoint{3.237865in}{0.700865in}}{\pgfqpoint{3.237865in}{0.689815in}}%
\pgfpathcurveto{\pgfqpoint{3.237865in}{0.678764in}}{\pgfqpoint{3.242255in}{0.668165in}}{\pgfqpoint{3.250069in}{0.660352in}}%
\pgfpathcurveto{\pgfqpoint{3.257882in}{0.652538in}}{\pgfqpoint{3.268481in}{0.648148in}}{\pgfqpoint{3.279531in}{0.648148in}}%
\pgfpathclose%
\pgfusepath{stroke,fill}%
\end{pgfscope}%
\begin{pgfscope}%
\pgfpathrectangle{\pgfqpoint{0.787074in}{0.548769in}}{\pgfqpoint{5.062926in}{3.102590in}}%
\pgfusepath{clip}%
\pgfsetbuttcap%
\pgfsetroundjoin%
\definecolor{currentfill}{rgb}{0.121569,0.466667,0.705882}%
\pgfsetfillcolor{currentfill}%
\pgfsetlinewidth{1.003750pt}%
\definecolor{currentstroke}{rgb}{0.121569,0.466667,0.705882}%
\pgfsetstrokecolor{currentstroke}%
\pgfsetdash{}{0pt}%
\pgfpathmoveto{\pgfqpoint{3.591576in}{0.648152in}}%
\pgfpathcurveto{\pgfqpoint{3.602626in}{0.648152in}}{\pgfqpoint{3.613225in}{0.652542in}}{\pgfqpoint{3.621039in}{0.660355in}}%
\pgfpathcurveto{\pgfqpoint{3.628852in}{0.668169in}}{\pgfqpoint{3.633243in}{0.678768in}}{\pgfqpoint{3.633243in}{0.689818in}}%
\pgfpathcurveto{\pgfqpoint{3.633243in}{0.700868in}}{\pgfqpoint{3.628852in}{0.711467in}}{\pgfqpoint{3.621039in}{0.719281in}}%
\pgfpathcurveto{\pgfqpoint{3.613225in}{0.727095in}}{\pgfqpoint{3.602626in}{0.731485in}}{\pgfqpoint{3.591576in}{0.731485in}}%
\pgfpathcurveto{\pgfqpoint{3.580526in}{0.731485in}}{\pgfqpoint{3.569927in}{0.727095in}}{\pgfqpoint{3.562113in}{0.719281in}}%
\pgfpathcurveto{\pgfqpoint{3.554300in}{0.711467in}}{\pgfqpoint{3.549909in}{0.700868in}}{\pgfqpoint{3.549909in}{0.689818in}}%
\pgfpathcurveto{\pgfqpoint{3.549909in}{0.678768in}}{\pgfqpoint{3.554300in}{0.668169in}}{\pgfqpoint{3.562113in}{0.660355in}}%
\pgfpathcurveto{\pgfqpoint{3.569927in}{0.652542in}}{\pgfqpoint{3.580526in}{0.648152in}}{\pgfqpoint{3.591576in}{0.648152in}}%
\pgfpathclose%
\pgfusepath{stroke,fill}%
\end{pgfscope}%
\begin{pgfscope}%
\pgfpathrectangle{\pgfqpoint{0.787074in}{0.548769in}}{\pgfqpoint{5.062926in}{3.102590in}}%
\pgfusepath{clip}%
\pgfsetbuttcap%
\pgfsetroundjoin%
\definecolor{currentfill}{rgb}{1.000000,0.498039,0.054902}%
\pgfsetfillcolor{currentfill}%
\pgfsetlinewidth{1.003750pt}%
\definecolor{currentstroke}{rgb}{1.000000,0.498039,0.054902}%
\pgfsetstrokecolor{currentstroke}%
\pgfsetdash{}{0pt}%
\pgfpathmoveto{\pgfqpoint{3.240526in}{2.884914in}}%
\pgfpathcurveto{\pgfqpoint{3.251576in}{2.884914in}}{\pgfqpoint{3.262175in}{2.889305in}}{\pgfqpoint{3.269989in}{2.897118in}}%
\pgfpathcurveto{\pgfqpoint{3.277802in}{2.904932in}}{\pgfqpoint{3.282192in}{2.915531in}}{\pgfqpoint{3.282192in}{2.926581in}}%
\pgfpathcurveto{\pgfqpoint{3.282192in}{2.937631in}}{\pgfqpoint{3.277802in}{2.948230in}}{\pgfqpoint{3.269989in}{2.956044in}}%
\pgfpathcurveto{\pgfqpoint{3.262175in}{2.963857in}}{\pgfqpoint{3.251576in}{2.968248in}}{\pgfqpoint{3.240526in}{2.968248in}}%
\pgfpathcurveto{\pgfqpoint{3.229476in}{2.968248in}}{\pgfqpoint{3.218877in}{2.963857in}}{\pgfqpoint{3.211063in}{2.956044in}}%
\pgfpathcurveto{\pgfqpoint{3.203249in}{2.948230in}}{\pgfqpoint{3.198859in}{2.937631in}}{\pgfqpoint{3.198859in}{2.926581in}}%
\pgfpathcurveto{\pgfqpoint{3.198859in}{2.915531in}}{\pgfqpoint{3.203249in}{2.904932in}}{\pgfqpoint{3.211063in}{2.897118in}}%
\pgfpathcurveto{\pgfqpoint{3.218877in}{2.889305in}}{\pgfqpoint{3.229476in}{2.884914in}}{\pgfqpoint{3.240526in}{2.884914in}}%
\pgfpathclose%
\pgfusepath{stroke,fill}%
\end{pgfscope}%
\begin{pgfscope}%
\pgfpathrectangle{\pgfqpoint{0.787074in}{0.548769in}}{\pgfqpoint{5.062926in}{3.102590in}}%
\pgfusepath{clip}%
\pgfsetbuttcap%
\pgfsetroundjoin%
\definecolor{currentfill}{rgb}{1.000000,0.498039,0.054902}%
\pgfsetfillcolor{currentfill}%
\pgfsetlinewidth{1.003750pt}%
\definecolor{currentstroke}{rgb}{1.000000,0.498039,0.054902}%
\pgfsetstrokecolor{currentstroke}%
\pgfsetdash{}{0pt}%
\pgfpathmoveto{\pgfqpoint{3.513565in}{2.875764in}}%
\pgfpathcurveto{\pgfqpoint{3.524615in}{2.875764in}}{\pgfqpoint{3.535214in}{2.880154in}}{\pgfqpoint{3.543028in}{2.887968in}}%
\pgfpathcurveto{\pgfqpoint{3.550841in}{2.895781in}}{\pgfqpoint{3.555232in}{2.906380in}}{\pgfqpoint{3.555232in}{2.917430in}}%
\pgfpathcurveto{\pgfqpoint{3.555232in}{2.928481in}}{\pgfqpoint{3.550841in}{2.939080in}}{\pgfqpoint{3.543028in}{2.946893in}}%
\pgfpathcurveto{\pgfqpoint{3.535214in}{2.954707in}}{\pgfqpoint{3.524615in}{2.959097in}}{\pgfqpoint{3.513565in}{2.959097in}}%
\pgfpathcurveto{\pgfqpoint{3.502515in}{2.959097in}}{\pgfqpoint{3.491916in}{2.954707in}}{\pgfqpoint{3.484102in}{2.946893in}}%
\pgfpathcurveto{\pgfqpoint{3.476288in}{2.939080in}}{\pgfqpoint{3.471898in}{2.928481in}}{\pgfqpoint{3.471898in}{2.917430in}}%
\pgfpathcurveto{\pgfqpoint{3.471898in}{2.906380in}}{\pgfqpoint{3.476288in}{2.895781in}}{\pgfqpoint{3.484102in}{2.887968in}}%
\pgfpathcurveto{\pgfqpoint{3.491916in}{2.880154in}}{\pgfqpoint{3.502515in}{2.875764in}}{\pgfqpoint{3.513565in}{2.875764in}}%
\pgfpathclose%
\pgfusepath{stroke,fill}%
\end{pgfscope}%
\begin{pgfscope}%
\pgfpathrectangle{\pgfqpoint{0.787074in}{0.548769in}}{\pgfqpoint{5.062926in}{3.102590in}}%
\pgfusepath{clip}%
\pgfsetbuttcap%
\pgfsetroundjoin%
\definecolor{currentfill}{rgb}{1.000000,0.498039,0.054902}%
\pgfsetfillcolor{currentfill}%
\pgfsetlinewidth{1.003750pt}%
\definecolor{currentstroke}{rgb}{1.000000,0.498039,0.054902}%
\pgfsetstrokecolor{currentstroke}%
\pgfsetdash{}{0pt}%
\pgfpathmoveto{\pgfqpoint{3.318537in}{2.086876in}}%
\pgfpathcurveto{\pgfqpoint{3.329587in}{2.086876in}}{\pgfqpoint{3.340186in}{2.091266in}}{\pgfqpoint{3.348000in}{2.099080in}}%
\pgfpathcurveto{\pgfqpoint{3.355813in}{2.106893in}}{\pgfqpoint{3.360204in}{2.117492in}}{\pgfqpoint{3.360204in}{2.128542in}}%
\pgfpathcurveto{\pgfqpoint{3.360204in}{2.139593in}}{\pgfqpoint{3.355813in}{2.150192in}}{\pgfqpoint{3.348000in}{2.158005in}}%
\pgfpathcurveto{\pgfqpoint{3.340186in}{2.165819in}}{\pgfqpoint{3.329587in}{2.170209in}}{\pgfqpoint{3.318537in}{2.170209in}}%
\pgfpathcurveto{\pgfqpoint{3.307487in}{2.170209in}}{\pgfqpoint{3.296888in}{2.165819in}}{\pgfqpoint{3.289074in}{2.158005in}}%
\pgfpathcurveto{\pgfqpoint{3.281261in}{2.150192in}}{\pgfqpoint{3.276870in}{2.139593in}}{\pgfqpoint{3.276870in}{2.128542in}}%
\pgfpathcurveto{\pgfqpoint{3.276870in}{2.117492in}}{\pgfqpoint{3.281261in}{2.106893in}}{\pgfqpoint{3.289074in}{2.099080in}}%
\pgfpathcurveto{\pgfqpoint{3.296888in}{2.091266in}}{\pgfqpoint{3.307487in}{2.086876in}}{\pgfqpoint{3.318537in}{2.086876in}}%
\pgfpathclose%
\pgfusepath{stroke,fill}%
\end{pgfscope}%
\begin{pgfscope}%
\pgfpathrectangle{\pgfqpoint{0.787074in}{0.548769in}}{\pgfqpoint{5.062926in}{3.102590in}}%
\pgfusepath{clip}%
\pgfsetbuttcap%
\pgfsetroundjoin%
\definecolor{currentfill}{rgb}{0.121569,0.466667,0.705882}%
\pgfsetfillcolor{currentfill}%
\pgfsetlinewidth{1.003750pt}%
\definecolor{currentstroke}{rgb}{0.121569,0.466667,0.705882}%
\pgfsetstrokecolor{currentstroke}%
\pgfsetdash{}{0pt}%
\pgfpathmoveto{\pgfqpoint{3.591576in}{0.648132in}}%
\pgfpathcurveto{\pgfqpoint{3.602626in}{0.648132in}}{\pgfqpoint{3.613225in}{0.652523in}}{\pgfqpoint{3.621039in}{0.660336in}}%
\pgfpathcurveto{\pgfqpoint{3.628852in}{0.668150in}}{\pgfqpoint{3.633243in}{0.678749in}}{\pgfqpoint{3.633243in}{0.689799in}}%
\pgfpathcurveto{\pgfqpoint{3.633243in}{0.700849in}}{\pgfqpoint{3.628852in}{0.711448in}}{\pgfqpoint{3.621039in}{0.719262in}}%
\pgfpathcurveto{\pgfqpoint{3.613225in}{0.727075in}}{\pgfqpoint{3.602626in}{0.731466in}}{\pgfqpoint{3.591576in}{0.731466in}}%
\pgfpathcurveto{\pgfqpoint{3.580526in}{0.731466in}}{\pgfqpoint{3.569927in}{0.727075in}}{\pgfqpoint{3.562113in}{0.719262in}}%
\pgfpathcurveto{\pgfqpoint{3.554300in}{0.711448in}}{\pgfqpoint{3.549909in}{0.700849in}}{\pgfqpoint{3.549909in}{0.689799in}}%
\pgfpathcurveto{\pgfqpoint{3.549909in}{0.678749in}}{\pgfqpoint{3.554300in}{0.668150in}}{\pgfqpoint{3.562113in}{0.660336in}}%
\pgfpathcurveto{\pgfqpoint{3.569927in}{0.652523in}}{\pgfqpoint{3.580526in}{0.648132in}}{\pgfqpoint{3.591576in}{0.648132in}}%
\pgfpathclose%
\pgfusepath{stroke,fill}%
\end{pgfscope}%
\begin{pgfscope}%
\pgfpathrectangle{\pgfqpoint{0.787074in}{0.548769in}}{\pgfqpoint{5.062926in}{3.102590in}}%
\pgfusepath{clip}%
\pgfsetbuttcap%
\pgfsetroundjoin%
\definecolor{currentfill}{rgb}{0.121569,0.466667,0.705882}%
\pgfsetfillcolor{currentfill}%
\pgfsetlinewidth{1.003750pt}%
\definecolor{currentstroke}{rgb}{0.121569,0.466667,0.705882}%
\pgfsetstrokecolor{currentstroke}%
\pgfsetdash{}{0pt}%
\pgfpathmoveto{\pgfqpoint{3.630582in}{0.787529in}}%
\pgfpathcurveto{\pgfqpoint{3.641632in}{0.787529in}}{\pgfqpoint{3.652231in}{0.791919in}}{\pgfqpoint{3.660044in}{0.799733in}}%
\pgfpathcurveto{\pgfqpoint{3.667858in}{0.807547in}}{\pgfqpoint{3.672248in}{0.818146in}}{\pgfqpoint{3.672248in}{0.829196in}}%
\pgfpathcurveto{\pgfqpoint{3.672248in}{0.840246in}}{\pgfqpoint{3.667858in}{0.850845in}}{\pgfqpoint{3.660044in}{0.858658in}}%
\pgfpathcurveto{\pgfqpoint{3.652231in}{0.866472in}}{\pgfqpoint{3.641632in}{0.870862in}}{\pgfqpoint{3.630582in}{0.870862in}}%
\pgfpathcurveto{\pgfqpoint{3.619532in}{0.870862in}}{\pgfqpoint{3.608933in}{0.866472in}}{\pgfqpoint{3.601119in}{0.858658in}}%
\pgfpathcurveto{\pgfqpoint{3.593305in}{0.850845in}}{\pgfqpoint{3.588915in}{0.840246in}}{\pgfqpoint{3.588915in}{0.829196in}}%
\pgfpathcurveto{\pgfqpoint{3.588915in}{0.818146in}}{\pgfqpoint{3.593305in}{0.807547in}}{\pgfqpoint{3.601119in}{0.799733in}}%
\pgfpathcurveto{\pgfqpoint{3.608933in}{0.791919in}}{\pgfqpoint{3.619532in}{0.787529in}}{\pgfqpoint{3.630582in}{0.787529in}}%
\pgfpathclose%
\pgfusepath{stroke,fill}%
\end{pgfscope}%
\begin{pgfscope}%
\pgfpathrectangle{\pgfqpoint{0.787074in}{0.548769in}}{\pgfqpoint{5.062926in}{3.102590in}}%
\pgfusepath{clip}%
\pgfsetbuttcap%
\pgfsetroundjoin%
\definecolor{currentfill}{rgb}{0.121569,0.466667,0.705882}%
\pgfsetfillcolor{currentfill}%
\pgfsetlinewidth{1.003750pt}%
\definecolor{currentstroke}{rgb}{0.121569,0.466667,0.705882}%
\pgfsetstrokecolor{currentstroke}%
\pgfsetdash{}{0pt}%
\pgfpathmoveto{\pgfqpoint{3.357543in}{2.073153in}}%
\pgfpathcurveto{\pgfqpoint{3.368593in}{2.073153in}}{\pgfqpoint{3.379192in}{2.077543in}}{\pgfqpoint{3.387005in}{2.085357in}}%
\pgfpathcurveto{\pgfqpoint{3.394819in}{2.093170in}}{\pgfqpoint{3.399209in}{2.103769in}}{\pgfqpoint{3.399209in}{2.114820in}}%
\pgfpathcurveto{\pgfqpoint{3.399209in}{2.125870in}}{\pgfqpoint{3.394819in}{2.136469in}}{\pgfqpoint{3.387005in}{2.144282in}}%
\pgfpathcurveto{\pgfqpoint{3.379192in}{2.152096in}}{\pgfqpoint{3.368593in}{2.156486in}}{\pgfqpoint{3.357543in}{2.156486in}}%
\pgfpathcurveto{\pgfqpoint{3.346492in}{2.156486in}}{\pgfqpoint{3.335893in}{2.152096in}}{\pgfqpoint{3.328080in}{2.144282in}}%
\pgfpathcurveto{\pgfqpoint{3.320266in}{2.136469in}}{\pgfqpoint{3.315876in}{2.125870in}}{\pgfqpoint{3.315876in}{2.114820in}}%
\pgfpathcurveto{\pgfqpoint{3.315876in}{2.103769in}}{\pgfqpoint{3.320266in}{2.093170in}}{\pgfqpoint{3.328080in}{2.085357in}}%
\pgfpathcurveto{\pgfqpoint{3.335893in}{2.077543in}}{\pgfqpoint{3.346492in}{2.073153in}}{\pgfqpoint{3.357543in}{2.073153in}}%
\pgfpathclose%
\pgfusepath{stroke,fill}%
\end{pgfscope}%
\begin{pgfscope}%
\pgfpathrectangle{\pgfqpoint{0.787074in}{0.548769in}}{\pgfqpoint{5.062926in}{3.102590in}}%
\pgfusepath{clip}%
\pgfsetbuttcap%
\pgfsetroundjoin%
\definecolor{currentfill}{rgb}{0.121569,0.466667,0.705882}%
\pgfsetfillcolor{currentfill}%
\pgfsetlinewidth{1.003750pt}%
\definecolor{currentstroke}{rgb}{0.121569,0.466667,0.705882}%
\pgfsetstrokecolor{currentstroke}%
\pgfsetdash{}{0pt}%
\pgfpathmoveto{\pgfqpoint{3.513565in}{0.648148in}}%
\pgfpathcurveto{\pgfqpoint{3.524615in}{0.648148in}}{\pgfqpoint{3.535214in}{0.652539in}}{\pgfqpoint{3.543028in}{0.660352in}}%
\pgfpathcurveto{\pgfqpoint{3.550841in}{0.668166in}}{\pgfqpoint{3.555232in}{0.678765in}}{\pgfqpoint{3.555232in}{0.689815in}}%
\pgfpathcurveto{\pgfqpoint{3.555232in}{0.700865in}}{\pgfqpoint{3.550841in}{0.711464in}}{\pgfqpoint{3.543028in}{0.719278in}}%
\pgfpathcurveto{\pgfqpoint{3.535214in}{0.727091in}}{\pgfqpoint{3.524615in}{0.731482in}}{\pgfqpoint{3.513565in}{0.731482in}}%
\pgfpathcurveto{\pgfqpoint{3.502515in}{0.731482in}}{\pgfqpoint{3.491916in}{0.727091in}}{\pgfqpoint{3.484102in}{0.719278in}}%
\pgfpathcurveto{\pgfqpoint{3.476288in}{0.711464in}}{\pgfqpoint{3.471898in}{0.700865in}}{\pgfqpoint{3.471898in}{0.689815in}}%
\pgfpathcurveto{\pgfqpoint{3.471898in}{0.678765in}}{\pgfqpoint{3.476288in}{0.668166in}}{\pgfqpoint{3.484102in}{0.660352in}}%
\pgfpathcurveto{\pgfqpoint{3.491916in}{0.652539in}}{\pgfqpoint{3.502515in}{0.648148in}}{\pgfqpoint{3.513565in}{0.648148in}}%
\pgfpathclose%
\pgfusepath{stroke,fill}%
\end{pgfscope}%
\begin{pgfscope}%
\pgfpathrectangle{\pgfqpoint{0.787074in}{0.548769in}}{\pgfqpoint{5.062926in}{3.102590in}}%
\pgfusepath{clip}%
\pgfsetbuttcap%
\pgfsetroundjoin%
\definecolor{currentfill}{rgb}{0.121569,0.466667,0.705882}%
\pgfsetfillcolor{currentfill}%
\pgfsetlinewidth{1.003750pt}%
\definecolor{currentstroke}{rgb}{0.121569,0.466667,0.705882}%
\pgfsetstrokecolor{currentstroke}%
\pgfsetdash{}{0pt}%
\pgfpathmoveto{\pgfqpoint{3.747598in}{0.648150in}}%
\pgfpathcurveto{\pgfqpoint{3.758649in}{0.648150in}}{\pgfqpoint{3.769248in}{0.652540in}}{\pgfqpoint{3.777061in}{0.660353in}}%
\pgfpathcurveto{\pgfqpoint{3.784875in}{0.668167in}}{\pgfqpoint{3.789265in}{0.678766in}}{\pgfqpoint{3.789265in}{0.689816in}}%
\pgfpathcurveto{\pgfqpoint{3.789265in}{0.700866in}}{\pgfqpoint{3.784875in}{0.711465in}}{\pgfqpoint{3.777061in}{0.719279in}}%
\pgfpathcurveto{\pgfqpoint{3.769248in}{0.727093in}}{\pgfqpoint{3.758649in}{0.731483in}}{\pgfqpoint{3.747598in}{0.731483in}}%
\pgfpathcurveto{\pgfqpoint{3.736548in}{0.731483in}}{\pgfqpoint{3.725949in}{0.727093in}}{\pgfqpoint{3.718136in}{0.719279in}}%
\pgfpathcurveto{\pgfqpoint{3.710322in}{0.711465in}}{\pgfqpoint{3.705932in}{0.700866in}}{\pgfqpoint{3.705932in}{0.689816in}}%
\pgfpathcurveto{\pgfqpoint{3.705932in}{0.678766in}}{\pgfqpoint{3.710322in}{0.668167in}}{\pgfqpoint{3.718136in}{0.660353in}}%
\pgfpathcurveto{\pgfqpoint{3.725949in}{0.652540in}}{\pgfqpoint{3.736548in}{0.648150in}}{\pgfqpoint{3.747598in}{0.648150in}}%
\pgfpathclose%
\pgfusepath{stroke,fill}%
\end{pgfscope}%
\begin{pgfscope}%
\pgfpathrectangle{\pgfqpoint{0.787074in}{0.548769in}}{\pgfqpoint{5.062926in}{3.102590in}}%
\pgfusepath{clip}%
\pgfsetbuttcap%
\pgfsetroundjoin%
\definecolor{currentfill}{rgb}{0.121569,0.466667,0.705882}%
\pgfsetfillcolor{currentfill}%
\pgfsetlinewidth{1.003750pt}%
\definecolor{currentstroke}{rgb}{0.121569,0.466667,0.705882}%
\pgfsetstrokecolor{currentstroke}%
\pgfsetdash{}{0pt}%
\pgfpathmoveto{\pgfqpoint{3.552570in}{2.326735in}}%
\pgfpathcurveto{\pgfqpoint{3.563621in}{2.326735in}}{\pgfqpoint{3.574220in}{2.331125in}}{\pgfqpoint{3.582033in}{2.338939in}}%
\pgfpathcurveto{\pgfqpoint{3.589847in}{2.346753in}}{\pgfqpoint{3.594237in}{2.357352in}}{\pgfqpoint{3.594237in}{2.368402in}}%
\pgfpathcurveto{\pgfqpoint{3.594237in}{2.379452in}}{\pgfqpoint{3.589847in}{2.390051in}}{\pgfqpoint{3.582033in}{2.397865in}}%
\pgfpathcurveto{\pgfqpoint{3.574220in}{2.405678in}}{\pgfqpoint{3.563621in}{2.410068in}}{\pgfqpoint{3.552570in}{2.410068in}}%
\pgfpathcurveto{\pgfqpoint{3.541520in}{2.410068in}}{\pgfqpoint{3.530921in}{2.405678in}}{\pgfqpoint{3.523108in}{2.397865in}}%
\pgfpathcurveto{\pgfqpoint{3.515294in}{2.390051in}}{\pgfqpoint{3.510904in}{2.379452in}}{\pgfqpoint{3.510904in}{2.368402in}}%
\pgfpathcurveto{\pgfqpoint{3.510904in}{2.357352in}}{\pgfqpoint{3.515294in}{2.346753in}}{\pgfqpoint{3.523108in}{2.338939in}}%
\pgfpathcurveto{\pgfqpoint{3.530921in}{2.331125in}}{\pgfqpoint{3.541520in}{2.326735in}}{\pgfqpoint{3.552570in}{2.326735in}}%
\pgfpathclose%
\pgfusepath{stroke,fill}%
\end{pgfscope}%
\begin{pgfscope}%
\pgfpathrectangle{\pgfqpoint{0.787074in}{0.548769in}}{\pgfqpoint{5.062926in}{3.102590in}}%
\pgfusepath{clip}%
\pgfsetbuttcap%
\pgfsetroundjoin%
\definecolor{currentfill}{rgb}{1.000000,0.498039,0.054902}%
\pgfsetfillcolor{currentfill}%
\pgfsetlinewidth{1.003750pt}%
\definecolor{currentstroke}{rgb}{1.000000,0.498039,0.054902}%
\pgfsetstrokecolor{currentstroke}%
\pgfsetdash{}{0pt}%
\pgfpathmoveto{\pgfqpoint{3.006492in}{2.612427in}}%
\pgfpathcurveto{\pgfqpoint{3.017542in}{2.612427in}}{\pgfqpoint{3.028141in}{2.616817in}}{\pgfqpoint{3.035955in}{2.624631in}}%
\pgfpathcurveto{\pgfqpoint{3.043769in}{2.632444in}}{\pgfqpoint{3.048159in}{2.643043in}}{\pgfqpoint{3.048159in}{2.654094in}}%
\pgfpathcurveto{\pgfqpoint{3.048159in}{2.665144in}}{\pgfqpoint{3.043769in}{2.675743in}}{\pgfqpoint{3.035955in}{2.683556in}}%
\pgfpathcurveto{\pgfqpoint{3.028141in}{2.691370in}}{\pgfqpoint{3.017542in}{2.695760in}}{\pgfqpoint{3.006492in}{2.695760in}}%
\pgfpathcurveto{\pgfqpoint{2.995442in}{2.695760in}}{\pgfqpoint{2.984843in}{2.691370in}}{\pgfqpoint{2.977029in}{2.683556in}}%
\pgfpathcurveto{\pgfqpoint{2.969216in}{2.675743in}}{\pgfqpoint{2.964825in}{2.665144in}}{\pgfqpoint{2.964825in}{2.654094in}}%
\pgfpathcurveto{\pgfqpoint{2.964825in}{2.643043in}}{\pgfqpoint{2.969216in}{2.632444in}}{\pgfqpoint{2.977029in}{2.624631in}}%
\pgfpathcurveto{\pgfqpoint{2.984843in}{2.616817in}}{\pgfqpoint{2.995442in}{2.612427in}}{\pgfqpoint{3.006492in}{2.612427in}}%
\pgfpathclose%
\pgfusepath{stroke,fill}%
\end{pgfscope}%
\begin{pgfscope}%
\pgfpathrectangle{\pgfqpoint{0.787074in}{0.548769in}}{\pgfqpoint{5.062926in}{3.102590in}}%
\pgfusepath{clip}%
\pgfsetbuttcap%
\pgfsetroundjoin%
\definecolor{currentfill}{rgb}{1.000000,0.498039,0.054902}%
\pgfsetfillcolor{currentfill}%
\pgfsetlinewidth{1.003750pt}%
\definecolor{currentstroke}{rgb}{1.000000,0.498039,0.054902}%
\pgfsetstrokecolor{currentstroke}%
\pgfsetdash{}{0pt}%
\pgfpathmoveto{\pgfqpoint{3.279531in}{2.467225in}}%
\pgfpathcurveto{\pgfqpoint{3.290581in}{2.467225in}}{\pgfqpoint{3.301180in}{2.471616in}}{\pgfqpoint{3.308994in}{2.479429in}}%
\pgfpathcurveto{\pgfqpoint{3.316808in}{2.487243in}}{\pgfqpoint{3.321198in}{2.497842in}}{\pgfqpoint{3.321198in}{2.508892in}}%
\pgfpathcurveto{\pgfqpoint{3.321198in}{2.519942in}}{\pgfqpoint{3.316808in}{2.530541in}}{\pgfqpoint{3.308994in}{2.538355in}}%
\pgfpathcurveto{\pgfqpoint{3.301180in}{2.546168in}}{\pgfqpoint{3.290581in}{2.550559in}}{\pgfqpoint{3.279531in}{2.550559in}}%
\pgfpathcurveto{\pgfqpoint{3.268481in}{2.550559in}}{\pgfqpoint{3.257882in}{2.546168in}}{\pgfqpoint{3.250069in}{2.538355in}}%
\pgfpathcurveto{\pgfqpoint{3.242255in}{2.530541in}}{\pgfqpoint{3.237865in}{2.519942in}}{\pgfqpoint{3.237865in}{2.508892in}}%
\pgfpathcurveto{\pgfqpoint{3.237865in}{2.497842in}}{\pgfqpoint{3.242255in}{2.487243in}}{\pgfqpoint{3.250069in}{2.479429in}}%
\pgfpathcurveto{\pgfqpoint{3.257882in}{2.471616in}}{\pgfqpoint{3.268481in}{2.467225in}}{\pgfqpoint{3.279531in}{2.467225in}}%
\pgfpathclose%
\pgfusepath{stroke,fill}%
\end{pgfscope}%
\begin{pgfscope}%
\pgfpathrectangle{\pgfqpoint{0.787074in}{0.548769in}}{\pgfqpoint{5.062926in}{3.102590in}}%
\pgfusepath{clip}%
\pgfsetbuttcap%
\pgfsetroundjoin%
\definecolor{currentfill}{rgb}{1.000000,0.498039,0.054902}%
\pgfsetfillcolor{currentfill}%
\pgfsetlinewidth{1.003750pt}%
\definecolor{currentstroke}{rgb}{1.000000,0.498039,0.054902}%
\pgfsetstrokecolor{currentstroke}%
\pgfsetdash{}{0pt}%
\pgfpathmoveto{\pgfqpoint{4.332682in}{2.185530in}}%
\pgfpathcurveto{\pgfqpoint{4.343733in}{2.185530in}}{\pgfqpoint{4.354332in}{2.189921in}}{\pgfqpoint{4.362145in}{2.197734in}}%
\pgfpathcurveto{\pgfqpoint{4.369959in}{2.205548in}}{\pgfqpoint{4.374349in}{2.216147in}}{\pgfqpoint{4.374349in}{2.227197in}}%
\pgfpathcurveto{\pgfqpoint{4.374349in}{2.238247in}}{\pgfqpoint{4.369959in}{2.248846in}}{\pgfqpoint{4.362145in}{2.256660in}}%
\pgfpathcurveto{\pgfqpoint{4.354332in}{2.264473in}}{\pgfqpoint{4.343733in}{2.268864in}}{\pgfqpoint{4.332682in}{2.268864in}}%
\pgfpathcurveto{\pgfqpoint{4.321632in}{2.268864in}}{\pgfqpoint{4.311033in}{2.264473in}}{\pgfqpoint{4.303220in}{2.256660in}}%
\pgfpathcurveto{\pgfqpoint{4.295406in}{2.248846in}}{\pgfqpoint{4.291016in}{2.238247in}}{\pgfqpoint{4.291016in}{2.227197in}}%
\pgfpathcurveto{\pgfqpoint{4.291016in}{2.216147in}}{\pgfqpoint{4.295406in}{2.205548in}}{\pgfqpoint{4.303220in}{2.197734in}}%
\pgfpathcurveto{\pgfqpoint{4.311033in}{2.189921in}}{\pgfqpoint{4.321632in}{2.185530in}}{\pgfqpoint{4.332682in}{2.185530in}}%
\pgfpathclose%
\pgfusepath{stroke,fill}%
\end{pgfscope}%
\begin{pgfscope}%
\pgfpathrectangle{\pgfqpoint{0.787074in}{0.548769in}}{\pgfqpoint{5.062926in}{3.102590in}}%
\pgfusepath{clip}%
\pgfsetbuttcap%
\pgfsetroundjoin%
\definecolor{currentfill}{rgb}{0.121569,0.466667,0.705882}%
\pgfsetfillcolor{currentfill}%
\pgfsetlinewidth{1.003750pt}%
\definecolor{currentstroke}{rgb}{0.121569,0.466667,0.705882}%
\pgfsetstrokecolor{currentstroke}%
\pgfsetdash{}{0pt}%
\pgfpathmoveto{\pgfqpoint{3.708593in}{2.666690in}}%
\pgfpathcurveto{\pgfqpoint{3.719643in}{2.666690in}}{\pgfqpoint{3.730242in}{2.671080in}}{\pgfqpoint{3.738056in}{2.678894in}}%
\pgfpathcurveto{\pgfqpoint{3.745869in}{2.686707in}}{\pgfqpoint{3.750260in}{2.697306in}}{\pgfqpoint{3.750260in}{2.708356in}}%
\pgfpathcurveto{\pgfqpoint{3.750260in}{2.719407in}}{\pgfqpoint{3.745869in}{2.730006in}}{\pgfqpoint{3.738056in}{2.737819in}}%
\pgfpathcurveto{\pgfqpoint{3.730242in}{2.745633in}}{\pgfqpoint{3.719643in}{2.750023in}}{\pgfqpoint{3.708593in}{2.750023in}}%
\pgfpathcurveto{\pgfqpoint{3.697543in}{2.750023in}}{\pgfqpoint{3.686944in}{2.745633in}}{\pgfqpoint{3.679130in}{2.737819in}}%
\pgfpathcurveto{\pgfqpoint{3.671316in}{2.730006in}}{\pgfqpoint{3.666926in}{2.719407in}}{\pgfqpoint{3.666926in}{2.708356in}}%
\pgfpathcurveto{\pgfqpoint{3.666926in}{2.697306in}}{\pgfqpoint{3.671316in}{2.686707in}}{\pgfqpoint{3.679130in}{2.678894in}}%
\pgfpathcurveto{\pgfqpoint{3.686944in}{2.671080in}}{\pgfqpoint{3.697543in}{2.666690in}}{\pgfqpoint{3.708593in}{2.666690in}}%
\pgfpathclose%
\pgfusepath{stroke,fill}%
\end{pgfscope}%
\begin{pgfscope}%
\pgfpathrectangle{\pgfqpoint{0.787074in}{0.548769in}}{\pgfqpoint{5.062926in}{3.102590in}}%
\pgfusepath{clip}%
\pgfsetbuttcap%
\pgfsetroundjoin%
\definecolor{currentfill}{rgb}{1.000000,0.498039,0.054902}%
\pgfsetfillcolor{currentfill}%
\pgfsetlinewidth{1.003750pt}%
\definecolor{currentstroke}{rgb}{1.000000,0.498039,0.054902}%
\pgfsetstrokecolor{currentstroke}%
\pgfsetdash{}{0pt}%
\pgfpathmoveto{\pgfqpoint{3.279531in}{2.209697in}}%
\pgfpathcurveto{\pgfqpoint{3.290581in}{2.209697in}}{\pgfqpoint{3.301180in}{2.214087in}}{\pgfqpoint{3.308994in}{2.221901in}}%
\pgfpathcurveto{\pgfqpoint{3.316808in}{2.229714in}}{\pgfqpoint{3.321198in}{2.240313in}}{\pgfqpoint{3.321198in}{2.251363in}}%
\pgfpathcurveto{\pgfqpoint{3.321198in}{2.262414in}}{\pgfqpoint{3.316808in}{2.273013in}}{\pgfqpoint{3.308994in}{2.280826in}}%
\pgfpathcurveto{\pgfqpoint{3.301180in}{2.288640in}}{\pgfqpoint{3.290581in}{2.293030in}}{\pgfqpoint{3.279531in}{2.293030in}}%
\pgfpathcurveto{\pgfqpoint{3.268481in}{2.293030in}}{\pgfqpoint{3.257882in}{2.288640in}}{\pgfqpoint{3.250069in}{2.280826in}}%
\pgfpathcurveto{\pgfqpoint{3.242255in}{2.273013in}}{\pgfqpoint{3.237865in}{2.262414in}}{\pgfqpoint{3.237865in}{2.251363in}}%
\pgfpathcurveto{\pgfqpoint{3.237865in}{2.240313in}}{\pgfqpoint{3.242255in}{2.229714in}}{\pgfqpoint{3.250069in}{2.221901in}}%
\pgfpathcurveto{\pgfqpoint{3.257882in}{2.214087in}}{\pgfqpoint{3.268481in}{2.209697in}}{\pgfqpoint{3.279531in}{2.209697in}}%
\pgfpathclose%
\pgfusepath{stroke,fill}%
\end{pgfscope}%
\begin{pgfscope}%
\pgfpathrectangle{\pgfqpoint{0.787074in}{0.548769in}}{\pgfqpoint{5.062926in}{3.102590in}}%
\pgfusepath{clip}%
\pgfsetbuttcap%
\pgfsetroundjoin%
\definecolor{currentfill}{rgb}{1.000000,0.498039,0.054902}%
\pgfsetfillcolor{currentfill}%
\pgfsetlinewidth{1.003750pt}%
\definecolor{currentstroke}{rgb}{1.000000,0.498039,0.054902}%
\pgfsetstrokecolor{currentstroke}%
\pgfsetdash{}{0pt}%
\pgfpathmoveto{\pgfqpoint{3.747598in}{1.987916in}}%
\pgfpathcurveto{\pgfqpoint{3.758649in}{1.987916in}}{\pgfqpoint{3.769248in}{1.992306in}}{\pgfqpoint{3.777061in}{2.000120in}}%
\pgfpathcurveto{\pgfqpoint{3.784875in}{2.007934in}}{\pgfqpoint{3.789265in}{2.018533in}}{\pgfqpoint{3.789265in}{2.029583in}}%
\pgfpathcurveto{\pgfqpoint{3.789265in}{2.040633in}}{\pgfqpoint{3.784875in}{2.051232in}}{\pgfqpoint{3.777061in}{2.059045in}}%
\pgfpathcurveto{\pgfqpoint{3.769248in}{2.066859in}}{\pgfqpoint{3.758649in}{2.071249in}}{\pgfqpoint{3.747598in}{2.071249in}}%
\pgfpathcurveto{\pgfqpoint{3.736548in}{2.071249in}}{\pgfqpoint{3.725949in}{2.066859in}}{\pgfqpoint{3.718136in}{2.059045in}}%
\pgfpathcurveto{\pgfqpoint{3.710322in}{2.051232in}}{\pgfqpoint{3.705932in}{2.040633in}}{\pgfqpoint{3.705932in}{2.029583in}}%
\pgfpathcurveto{\pgfqpoint{3.705932in}{2.018533in}}{\pgfqpoint{3.710322in}{2.007934in}}{\pgfqpoint{3.718136in}{2.000120in}}%
\pgfpathcurveto{\pgfqpoint{3.725949in}{1.992306in}}{\pgfqpoint{3.736548in}{1.987916in}}{\pgfqpoint{3.747598in}{1.987916in}}%
\pgfpathclose%
\pgfusepath{stroke,fill}%
\end{pgfscope}%
\begin{pgfscope}%
\pgfpathrectangle{\pgfqpoint{0.787074in}{0.548769in}}{\pgfqpoint{5.062926in}{3.102590in}}%
\pgfusepath{clip}%
\pgfsetbuttcap%
\pgfsetroundjoin%
\definecolor{currentfill}{rgb}{1.000000,0.498039,0.054902}%
\pgfsetfillcolor{currentfill}%
\pgfsetlinewidth{1.003750pt}%
\definecolor{currentstroke}{rgb}{1.000000,0.498039,0.054902}%
\pgfsetstrokecolor{currentstroke}%
\pgfsetdash{}{0pt}%
\pgfpathmoveto{\pgfqpoint{4.293677in}{2.610441in}}%
\pgfpathcurveto{\pgfqpoint{4.304727in}{2.610441in}}{\pgfqpoint{4.315326in}{2.614831in}}{\pgfqpoint{4.323140in}{2.622645in}}%
\pgfpathcurveto{\pgfqpoint{4.330953in}{2.630459in}}{\pgfqpoint{4.335343in}{2.641058in}}{\pgfqpoint{4.335343in}{2.652108in}}%
\pgfpathcurveto{\pgfqpoint{4.335343in}{2.663158in}}{\pgfqpoint{4.330953in}{2.673757in}}{\pgfqpoint{4.323140in}{2.681571in}}%
\pgfpathcurveto{\pgfqpoint{4.315326in}{2.689384in}}{\pgfqpoint{4.304727in}{2.693774in}}{\pgfqpoint{4.293677in}{2.693774in}}%
\pgfpathcurveto{\pgfqpoint{4.282627in}{2.693774in}}{\pgfqpoint{4.272028in}{2.689384in}}{\pgfqpoint{4.264214in}{2.681571in}}%
\pgfpathcurveto{\pgfqpoint{4.256400in}{2.673757in}}{\pgfqpoint{4.252010in}{2.663158in}}{\pgfqpoint{4.252010in}{2.652108in}}%
\pgfpathcurveto{\pgfqpoint{4.252010in}{2.641058in}}{\pgfqpoint{4.256400in}{2.630459in}}{\pgfqpoint{4.264214in}{2.622645in}}%
\pgfpathcurveto{\pgfqpoint{4.272028in}{2.614831in}}{\pgfqpoint{4.282627in}{2.610441in}}{\pgfqpoint{4.293677in}{2.610441in}}%
\pgfpathclose%
\pgfusepath{stroke,fill}%
\end{pgfscope}%
\begin{pgfscope}%
\pgfpathrectangle{\pgfqpoint{0.787074in}{0.548769in}}{\pgfqpoint{5.062926in}{3.102590in}}%
\pgfusepath{clip}%
\pgfsetbuttcap%
\pgfsetroundjoin%
\definecolor{currentfill}{rgb}{1.000000,0.498039,0.054902}%
\pgfsetfillcolor{currentfill}%
\pgfsetlinewidth{1.003750pt}%
\definecolor{currentstroke}{rgb}{1.000000,0.498039,0.054902}%
\pgfsetstrokecolor{currentstroke}%
\pgfsetdash{}{0pt}%
\pgfpathmoveto{\pgfqpoint{3.279531in}{1.913221in}}%
\pgfpathcurveto{\pgfqpoint{3.290581in}{1.913221in}}{\pgfqpoint{3.301180in}{1.917612in}}{\pgfqpoint{3.308994in}{1.925425in}}%
\pgfpathcurveto{\pgfqpoint{3.316808in}{1.933239in}}{\pgfqpoint{3.321198in}{1.943838in}}{\pgfqpoint{3.321198in}{1.954888in}}%
\pgfpathcurveto{\pgfqpoint{3.321198in}{1.965938in}}{\pgfqpoint{3.316808in}{1.976537in}}{\pgfqpoint{3.308994in}{1.984351in}}%
\pgfpathcurveto{\pgfqpoint{3.301180in}{1.992164in}}{\pgfqpoint{3.290581in}{1.996555in}}{\pgfqpoint{3.279531in}{1.996555in}}%
\pgfpathcurveto{\pgfqpoint{3.268481in}{1.996555in}}{\pgfqpoint{3.257882in}{1.992164in}}{\pgfqpoint{3.250069in}{1.984351in}}%
\pgfpathcurveto{\pgfqpoint{3.242255in}{1.976537in}}{\pgfqpoint{3.237865in}{1.965938in}}{\pgfqpoint{3.237865in}{1.954888in}}%
\pgfpathcurveto{\pgfqpoint{3.237865in}{1.943838in}}{\pgfqpoint{3.242255in}{1.933239in}}{\pgfqpoint{3.250069in}{1.925425in}}%
\pgfpathcurveto{\pgfqpoint{3.257882in}{1.917612in}}{\pgfqpoint{3.268481in}{1.913221in}}{\pgfqpoint{3.279531in}{1.913221in}}%
\pgfpathclose%
\pgfusepath{stroke,fill}%
\end{pgfscope}%
\begin{pgfscope}%
\pgfpathrectangle{\pgfqpoint{0.787074in}{0.548769in}}{\pgfqpoint{5.062926in}{3.102590in}}%
\pgfusepath{clip}%
\pgfsetbuttcap%
\pgfsetroundjoin%
\definecolor{currentfill}{rgb}{1.000000,0.498039,0.054902}%
\pgfsetfillcolor{currentfill}%
\pgfsetlinewidth{1.003750pt}%
\definecolor{currentstroke}{rgb}{1.000000,0.498039,0.054902}%
\pgfsetstrokecolor{currentstroke}%
\pgfsetdash{}{0pt}%
\pgfpathmoveto{\pgfqpoint{4.410694in}{2.130198in}}%
\pgfpathcurveto{\pgfqpoint{4.421744in}{2.130198in}}{\pgfqpoint{4.432343in}{2.134588in}}{\pgfqpoint{4.440156in}{2.142402in}}%
\pgfpathcurveto{\pgfqpoint{4.447970in}{2.150216in}}{\pgfqpoint{4.452360in}{2.160815in}}{\pgfqpoint{4.452360in}{2.171865in}}%
\pgfpathcurveto{\pgfqpoint{4.452360in}{2.182915in}}{\pgfqpoint{4.447970in}{2.193514in}}{\pgfqpoint{4.440156in}{2.201327in}}%
\pgfpathcurveto{\pgfqpoint{4.432343in}{2.209141in}}{\pgfqpoint{4.421744in}{2.213531in}}{\pgfqpoint{4.410694in}{2.213531in}}%
\pgfpathcurveto{\pgfqpoint{4.399643in}{2.213531in}}{\pgfqpoint{4.389044in}{2.209141in}}{\pgfqpoint{4.381231in}{2.201327in}}%
\pgfpathcurveto{\pgfqpoint{4.373417in}{2.193514in}}{\pgfqpoint{4.369027in}{2.182915in}}{\pgfqpoint{4.369027in}{2.171865in}}%
\pgfpathcurveto{\pgfqpoint{4.369027in}{2.160815in}}{\pgfqpoint{4.373417in}{2.150216in}}{\pgfqpoint{4.381231in}{2.142402in}}%
\pgfpathcurveto{\pgfqpoint{4.389044in}{2.134588in}}{\pgfqpoint{4.399643in}{2.130198in}}{\pgfqpoint{4.410694in}{2.130198in}}%
\pgfpathclose%
\pgfusepath{stroke,fill}%
\end{pgfscope}%
\begin{pgfscope}%
\pgfpathrectangle{\pgfqpoint{0.787074in}{0.548769in}}{\pgfqpoint{5.062926in}{3.102590in}}%
\pgfusepath{clip}%
\pgfsetbuttcap%
\pgfsetroundjoin%
\definecolor{currentfill}{rgb}{1.000000,0.498039,0.054902}%
\pgfsetfillcolor{currentfill}%
\pgfsetlinewidth{1.003750pt}%
\definecolor{currentstroke}{rgb}{1.000000,0.498039,0.054902}%
\pgfsetstrokecolor{currentstroke}%
\pgfsetdash{}{0pt}%
\pgfpathmoveto{\pgfqpoint{3.942626in}{1.747432in}}%
\pgfpathcurveto{\pgfqpoint{3.953677in}{1.747432in}}{\pgfqpoint{3.964276in}{1.751822in}}{\pgfqpoint{3.972089in}{1.759636in}}%
\pgfpathcurveto{\pgfqpoint{3.979903in}{1.767449in}}{\pgfqpoint{3.984293in}{1.778048in}}{\pgfqpoint{3.984293in}{1.789098in}}%
\pgfpathcurveto{\pgfqpoint{3.984293in}{1.800148in}}{\pgfqpoint{3.979903in}{1.810748in}}{\pgfqpoint{3.972089in}{1.818561in}}%
\pgfpathcurveto{\pgfqpoint{3.964276in}{1.826375in}}{\pgfqpoint{3.953677in}{1.830765in}}{\pgfqpoint{3.942626in}{1.830765in}}%
\pgfpathcurveto{\pgfqpoint{3.931576in}{1.830765in}}{\pgfqpoint{3.920977in}{1.826375in}}{\pgfqpoint{3.913164in}{1.818561in}}%
\pgfpathcurveto{\pgfqpoint{3.905350in}{1.810748in}}{\pgfqpoint{3.900960in}{1.800148in}}{\pgfqpoint{3.900960in}{1.789098in}}%
\pgfpathcurveto{\pgfqpoint{3.900960in}{1.778048in}}{\pgfqpoint{3.905350in}{1.767449in}}{\pgfqpoint{3.913164in}{1.759636in}}%
\pgfpathcurveto{\pgfqpoint{3.920977in}{1.751822in}}{\pgfqpoint{3.931576in}{1.747432in}}{\pgfqpoint{3.942626in}{1.747432in}}%
\pgfpathclose%
\pgfusepath{stroke,fill}%
\end{pgfscope}%
\begin{pgfscope}%
\pgfpathrectangle{\pgfqpoint{0.787074in}{0.548769in}}{\pgfqpoint{5.062926in}{3.102590in}}%
\pgfusepath{clip}%
\pgfsetbuttcap%
\pgfsetroundjoin%
\definecolor{currentfill}{rgb}{1.000000,0.498039,0.054902}%
\pgfsetfillcolor{currentfill}%
\pgfsetlinewidth{1.003750pt}%
\definecolor{currentstroke}{rgb}{1.000000,0.498039,0.054902}%
\pgfsetstrokecolor{currentstroke}%
\pgfsetdash{}{0pt}%
\pgfpathmoveto{\pgfqpoint{3.162515in}{2.135660in}}%
\pgfpathcurveto{\pgfqpoint{3.173565in}{2.135660in}}{\pgfqpoint{3.184164in}{2.140050in}}{\pgfqpoint{3.191977in}{2.147864in}}%
\pgfpathcurveto{\pgfqpoint{3.199791in}{2.155678in}}{\pgfqpoint{3.204181in}{2.166277in}}{\pgfqpoint{3.204181in}{2.177327in}}%
\pgfpathcurveto{\pgfqpoint{3.204181in}{2.188377in}}{\pgfqpoint{3.199791in}{2.198976in}}{\pgfqpoint{3.191977in}{2.206789in}}%
\pgfpathcurveto{\pgfqpoint{3.184164in}{2.214603in}}{\pgfqpoint{3.173565in}{2.218993in}}{\pgfqpoint{3.162515in}{2.218993in}}%
\pgfpathcurveto{\pgfqpoint{3.151464in}{2.218993in}}{\pgfqpoint{3.140865in}{2.214603in}}{\pgfqpoint{3.133052in}{2.206789in}}%
\pgfpathcurveto{\pgfqpoint{3.125238in}{2.198976in}}{\pgfqpoint{3.120848in}{2.188377in}}{\pgfqpoint{3.120848in}{2.177327in}}%
\pgfpathcurveto{\pgfqpoint{3.120848in}{2.166277in}}{\pgfqpoint{3.125238in}{2.155678in}}{\pgfqpoint{3.133052in}{2.147864in}}%
\pgfpathcurveto{\pgfqpoint{3.140865in}{2.140050in}}{\pgfqpoint{3.151464in}{2.135660in}}{\pgfqpoint{3.162515in}{2.135660in}}%
\pgfpathclose%
\pgfusepath{stroke,fill}%
\end{pgfscope}%
\begin{pgfscope}%
\pgfpathrectangle{\pgfqpoint{0.787074in}{0.548769in}}{\pgfqpoint{5.062926in}{3.102590in}}%
\pgfusepath{clip}%
\pgfsetbuttcap%
\pgfsetroundjoin%
\definecolor{currentfill}{rgb}{1.000000,0.498039,0.054902}%
\pgfsetfillcolor{currentfill}%
\pgfsetlinewidth{1.003750pt}%
\definecolor{currentstroke}{rgb}{1.000000,0.498039,0.054902}%
\pgfsetstrokecolor{currentstroke}%
\pgfsetdash{}{0pt}%
\pgfpathmoveto{\pgfqpoint{3.981632in}{1.976501in}}%
\pgfpathcurveto{\pgfqpoint{3.992682in}{1.976501in}}{\pgfqpoint{4.003281in}{1.980891in}}{\pgfqpoint{4.011095in}{1.988704in}}%
\pgfpathcurveto{\pgfqpoint{4.018908in}{1.996518in}}{\pgfqpoint{4.023299in}{2.007117in}}{\pgfqpoint{4.023299in}{2.018167in}}%
\pgfpathcurveto{\pgfqpoint{4.023299in}{2.029217in}}{\pgfqpoint{4.018908in}{2.039816in}}{\pgfqpoint{4.011095in}{2.047630in}}%
\pgfpathcurveto{\pgfqpoint{4.003281in}{2.055444in}}{\pgfqpoint{3.992682in}{2.059834in}}{\pgfqpoint{3.981632in}{2.059834in}}%
\pgfpathcurveto{\pgfqpoint{3.970582in}{2.059834in}}{\pgfqpoint{3.959983in}{2.055444in}}{\pgfqpoint{3.952169in}{2.047630in}}%
\pgfpathcurveto{\pgfqpoint{3.944356in}{2.039816in}}{\pgfqpoint{3.939965in}{2.029217in}}{\pgfqpoint{3.939965in}{2.018167in}}%
\pgfpathcurveto{\pgfqpoint{3.939965in}{2.007117in}}{\pgfqpoint{3.944356in}{1.996518in}}{\pgfqpoint{3.952169in}{1.988704in}}%
\pgfpathcurveto{\pgfqpoint{3.959983in}{1.980891in}}{\pgfqpoint{3.970582in}{1.976501in}}{\pgfqpoint{3.981632in}{1.976501in}}%
\pgfpathclose%
\pgfusepath{stroke,fill}%
\end{pgfscope}%
\begin{pgfscope}%
\pgfpathrectangle{\pgfqpoint{0.787074in}{0.548769in}}{\pgfqpoint{5.062926in}{3.102590in}}%
\pgfusepath{clip}%
\pgfsetbuttcap%
\pgfsetroundjoin%
\definecolor{currentfill}{rgb}{1.000000,0.498039,0.054902}%
\pgfsetfillcolor{currentfill}%
\pgfsetlinewidth{1.003750pt}%
\definecolor{currentstroke}{rgb}{1.000000,0.498039,0.054902}%
\pgfsetstrokecolor{currentstroke}%
\pgfsetdash{}{0pt}%
\pgfpathmoveto{\pgfqpoint{3.396548in}{2.141633in}}%
\pgfpathcurveto{\pgfqpoint{3.407598in}{2.141633in}}{\pgfqpoint{3.418197in}{2.146024in}}{\pgfqpoint{3.426011in}{2.153837in}}%
\pgfpathcurveto{\pgfqpoint{3.433825in}{2.161651in}}{\pgfqpoint{3.438215in}{2.172250in}}{\pgfqpoint{3.438215in}{2.183300in}}%
\pgfpathcurveto{\pgfqpoint{3.438215in}{2.194350in}}{\pgfqpoint{3.433825in}{2.204949in}}{\pgfqpoint{3.426011in}{2.212763in}}%
\pgfpathcurveto{\pgfqpoint{3.418197in}{2.220577in}}{\pgfqpoint{3.407598in}{2.224967in}}{\pgfqpoint{3.396548in}{2.224967in}}%
\pgfpathcurveto{\pgfqpoint{3.385498in}{2.224967in}}{\pgfqpoint{3.374899in}{2.220577in}}{\pgfqpoint{3.367085in}{2.212763in}}%
\pgfpathcurveto{\pgfqpoint{3.359272in}{2.204949in}}{\pgfqpoint{3.354881in}{2.194350in}}{\pgfqpoint{3.354881in}{2.183300in}}%
\pgfpathcurveto{\pgfqpoint{3.354881in}{2.172250in}}{\pgfqpoint{3.359272in}{2.161651in}}{\pgfqpoint{3.367085in}{2.153837in}}%
\pgfpathcurveto{\pgfqpoint{3.374899in}{2.146024in}}{\pgfqpoint{3.385498in}{2.141633in}}{\pgfqpoint{3.396548in}{2.141633in}}%
\pgfpathclose%
\pgfusepath{stroke,fill}%
\end{pgfscope}%
\begin{pgfscope}%
\pgfpathrectangle{\pgfqpoint{0.787074in}{0.548769in}}{\pgfqpoint{5.062926in}{3.102590in}}%
\pgfusepath{clip}%
\pgfsetbuttcap%
\pgfsetroundjoin%
\definecolor{currentfill}{rgb}{1.000000,0.498039,0.054902}%
\pgfsetfillcolor{currentfill}%
\pgfsetlinewidth{1.003750pt}%
\definecolor{currentstroke}{rgb}{1.000000,0.498039,0.054902}%
\pgfsetstrokecolor{currentstroke}%
\pgfsetdash{}{0pt}%
\pgfpathmoveto{\pgfqpoint{3.357543in}{2.039016in}}%
\pgfpathcurveto{\pgfqpoint{3.368593in}{2.039016in}}{\pgfqpoint{3.379192in}{2.043407in}}{\pgfqpoint{3.387005in}{2.051220in}}%
\pgfpathcurveto{\pgfqpoint{3.394819in}{2.059034in}}{\pgfqpoint{3.399209in}{2.069633in}}{\pgfqpoint{3.399209in}{2.080683in}}%
\pgfpathcurveto{\pgfqpoint{3.399209in}{2.091733in}}{\pgfqpoint{3.394819in}{2.102332in}}{\pgfqpoint{3.387005in}{2.110146in}}%
\pgfpathcurveto{\pgfqpoint{3.379192in}{2.117960in}}{\pgfqpoint{3.368593in}{2.122350in}}{\pgfqpoint{3.357543in}{2.122350in}}%
\pgfpathcurveto{\pgfqpoint{3.346492in}{2.122350in}}{\pgfqpoint{3.335893in}{2.117960in}}{\pgfqpoint{3.328080in}{2.110146in}}%
\pgfpathcurveto{\pgfqpoint{3.320266in}{2.102332in}}{\pgfqpoint{3.315876in}{2.091733in}}{\pgfqpoint{3.315876in}{2.080683in}}%
\pgfpathcurveto{\pgfqpoint{3.315876in}{2.069633in}}{\pgfqpoint{3.320266in}{2.059034in}}{\pgfqpoint{3.328080in}{2.051220in}}%
\pgfpathcurveto{\pgfqpoint{3.335893in}{2.043407in}}{\pgfqpoint{3.346492in}{2.039016in}}{\pgfqpoint{3.357543in}{2.039016in}}%
\pgfpathclose%
\pgfusepath{stroke,fill}%
\end{pgfscope}%
\begin{pgfscope}%
\pgfpathrectangle{\pgfqpoint{0.787074in}{0.548769in}}{\pgfqpoint{5.062926in}{3.102590in}}%
\pgfusepath{clip}%
\pgfsetbuttcap%
\pgfsetroundjoin%
\definecolor{currentfill}{rgb}{0.121569,0.466667,0.705882}%
\pgfsetfillcolor{currentfill}%
\pgfsetlinewidth{1.003750pt}%
\definecolor{currentstroke}{rgb}{0.121569,0.466667,0.705882}%
\pgfsetstrokecolor{currentstroke}%
\pgfsetdash{}{0pt}%
\pgfpathmoveto{\pgfqpoint{2.928481in}{1.981410in}}%
\pgfpathcurveto{\pgfqpoint{2.939531in}{1.981410in}}{\pgfqpoint{2.950130in}{1.985801in}}{\pgfqpoint{2.957944in}{1.993614in}}%
\pgfpathcurveto{\pgfqpoint{2.965757in}{2.001428in}}{\pgfqpoint{2.970148in}{2.012027in}}{\pgfqpoint{2.970148in}{2.023077in}}%
\pgfpathcurveto{\pgfqpoint{2.970148in}{2.034127in}}{\pgfqpoint{2.965757in}{2.044726in}}{\pgfqpoint{2.957944in}{2.052540in}}%
\pgfpathcurveto{\pgfqpoint{2.950130in}{2.060353in}}{\pgfqpoint{2.939531in}{2.064744in}}{\pgfqpoint{2.928481in}{2.064744in}}%
\pgfpathcurveto{\pgfqpoint{2.917431in}{2.064744in}}{\pgfqpoint{2.906832in}{2.060353in}}{\pgfqpoint{2.899018in}{2.052540in}}%
\pgfpathcurveto{\pgfqpoint{2.891205in}{2.044726in}}{\pgfqpoint{2.886814in}{2.034127in}}{\pgfqpoint{2.886814in}{2.023077in}}%
\pgfpathcurveto{\pgfqpoint{2.886814in}{2.012027in}}{\pgfqpoint{2.891205in}{2.001428in}}{\pgfqpoint{2.899018in}{1.993614in}}%
\pgfpathcurveto{\pgfqpoint{2.906832in}{1.985801in}}{\pgfqpoint{2.917431in}{1.981410in}}{\pgfqpoint{2.928481in}{1.981410in}}%
\pgfpathclose%
\pgfusepath{stroke,fill}%
\end{pgfscope}%
\begin{pgfscope}%
\pgfpathrectangle{\pgfqpoint{0.787074in}{0.548769in}}{\pgfqpoint{5.062926in}{3.102590in}}%
\pgfusepath{clip}%
\pgfsetbuttcap%
\pgfsetroundjoin%
\definecolor{currentfill}{rgb}{1.000000,0.498039,0.054902}%
\pgfsetfillcolor{currentfill}%
\pgfsetlinewidth{1.003750pt}%
\definecolor{currentstroke}{rgb}{1.000000,0.498039,0.054902}%
\pgfsetstrokecolor{currentstroke}%
\pgfsetdash{}{0pt}%
\pgfpathmoveto{\pgfqpoint{3.474559in}{2.386079in}}%
\pgfpathcurveto{\pgfqpoint{3.485609in}{2.386079in}}{\pgfqpoint{3.496208in}{2.390469in}}{\pgfqpoint{3.504022in}{2.398283in}}%
\pgfpathcurveto{\pgfqpoint{3.511836in}{2.406097in}}{\pgfqpoint{3.516226in}{2.416696in}}{\pgfqpoint{3.516226in}{2.427746in}}%
\pgfpathcurveto{\pgfqpoint{3.516226in}{2.438796in}}{\pgfqpoint{3.511836in}{2.449395in}}{\pgfqpoint{3.504022in}{2.457209in}}%
\pgfpathcurveto{\pgfqpoint{3.496208in}{2.465022in}}{\pgfqpoint{3.485609in}{2.469412in}}{\pgfqpoint{3.474559in}{2.469412in}}%
\pgfpathcurveto{\pgfqpoint{3.463509in}{2.469412in}}{\pgfqpoint{3.452910in}{2.465022in}}{\pgfqpoint{3.445097in}{2.457209in}}%
\pgfpathcurveto{\pgfqpoint{3.437283in}{2.449395in}}{\pgfqpoint{3.432893in}{2.438796in}}{\pgfqpoint{3.432893in}{2.427746in}}%
\pgfpathcurveto{\pgfqpoint{3.432893in}{2.416696in}}{\pgfqpoint{3.437283in}{2.406097in}}{\pgfqpoint{3.445097in}{2.398283in}}%
\pgfpathcurveto{\pgfqpoint{3.452910in}{2.390469in}}{\pgfqpoint{3.463509in}{2.386079in}}{\pgfqpoint{3.474559in}{2.386079in}}%
\pgfpathclose%
\pgfusepath{stroke,fill}%
\end{pgfscope}%
\begin{pgfscope}%
\pgfpathrectangle{\pgfqpoint{0.787074in}{0.548769in}}{\pgfqpoint{5.062926in}{3.102590in}}%
\pgfusepath{clip}%
\pgfsetbuttcap%
\pgfsetroundjoin%
\definecolor{currentfill}{rgb}{1.000000,0.498039,0.054902}%
\pgfsetfillcolor{currentfill}%
\pgfsetlinewidth{1.003750pt}%
\definecolor{currentstroke}{rgb}{1.000000,0.498039,0.054902}%
\pgfsetstrokecolor{currentstroke}%
\pgfsetdash{}{0pt}%
\pgfpathmoveto{\pgfqpoint{5.619867in}{2.446539in}}%
\pgfpathcurveto{\pgfqpoint{5.630917in}{2.446539in}}{\pgfqpoint{5.641516in}{2.450929in}}{\pgfqpoint{5.649330in}{2.458743in}}%
\pgfpathcurveto{\pgfqpoint{5.657143in}{2.466556in}}{\pgfqpoint{5.661534in}{2.477155in}}{\pgfqpoint{5.661534in}{2.488206in}}%
\pgfpathcurveto{\pgfqpoint{5.661534in}{2.499256in}}{\pgfqpoint{5.657143in}{2.509855in}}{\pgfqpoint{5.649330in}{2.517668in}}%
\pgfpathcurveto{\pgfqpoint{5.641516in}{2.525482in}}{\pgfqpoint{5.630917in}{2.529872in}}{\pgfqpoint{5.619867in}{2.529872in}}%
\pgfpathcurveto{\pgfqpoint{5.608817in}{2.529872in}}{\pgfqpoint{5.598218in}{2.525482in}}{\pgfqpoint{5.590404in}{2.517668in}}%
\pgfpathcurveto{\pgfqpoint{5.582591in}{2.509855in}}{\pgfqpoint{5.578200in}{2.499256in}}{\pgfqpoint{5.578200in}{2.488206in}}%
\pgfpathcurveto{\pgfqpoint{5.578200in}{2.477155in}}{\pgfqpoint{5.582591in}{2.466556in}}{\pgfqpoint{5.590404in}{2.458743in}}%
\pgfpathcurveto{\pgfqpoint{5.598218in}{2.450929in}}{\pgfqpoint{5.608817in}{2.446539in}}{\pgfqpoint{5.619867in}{2.446539in}}%
\pgfpathclose%
\pgfusepath{stroke,fill}%
\end{pgfscope}%
\begin{pgfscope}%
\pgfpathrectangle{\pgfqpoint{0.787074in}{0.548769in}}{\pgfqpoint{5.062926in}{3.102590in}}%
\pgfusepath{clip}%
\pgfsetbuttcap%
\pgfsetroundjoin%
\definecolor{currentfill}{rgb}{1.000000,0.498039,0.054902}%
\pgfsetfillcolor{currentfill}%
\pgfsetlinewidth{1.003750pt}%
\definecolor{currentstroke}{rgb}{1.000000,0.498039,0.054902}%
\pgfsetstrokecolor{currentstroke}%
\pgfsetdash{}{0pt}%
\pgfpathmoveto{\pgfqpoint{3.591576in}{2.604778in}}%
\pgfpathcurveto{\pgfqpoint{3.602626in}{2.604778in}}{\pgfqpoint{3.613225in}{2.609168in}}{\pgfqpoint{3.621039in}{2.616982in}}%
\pgfpathcurveto{\pgfqpoint{3.628852in}{2.624795in}}{\pgfqpoint{3.633243in}{2.635394in}}{\pgfqpoint{3.633243in}{2.646444in}}%
\pgfpathcurveto{\pgfqpoint{3.633243in}{2.657495in}}{\pgfqpoint{3.628852in}{2.668094in}}{\pgfqpoint{3.621039in}{2.675907in}}%
\pgfpathcurveto{\pgfqpoint{3.613225in}{2.683721in}}{\pgfqpoint{3.602626in}{2.688111in}}{\pgfqpoint{3.591576in}{2.688111in}}%
\pgfpathcurveto{\pgfqpoint{3.580526in}{2.688111in}}{\pgfqpoint{3.569927in}{2.683721in}}{\pgfqpoint{3.562113in}{2.675907in}}%
\pgfpathcurveto{\pgfqpoint{3.554300in}{2.668094in}}{\pgfqpoint{3.549909in}{2.657495in}}{\pgfqpoint{3.549909in}{2.646444in}}%
\pgfpathcurveto{\pgfqpoint{3.549909in}{2.635394in}}{\pgfqpoint{3.554300in}{2.624795in}}{\pgfqpoint{3.562113in}{2.616982in}}%
\pgfpathcurveto{\pgfqpoint{3.569927in}{2.609168in}}{\pgfqpoint{3.580526in}{2.604778in}}{\pgfqpoint{3.591576in}{2.604778in}}%
\pgfpathclose%
\pgfusepath{stroke,fill}%
\end{pgfscope}%
\begin{pgfscope}%
\pgfpathrectangle{\pgfqpoint{0.787074in}{0.548769in}}{\pgfqpoint{5.062926in}{3.102590in}}%
\pgfusepath{clip}%
\pgfsetbuttcap%
\pgfsetroundjoin%
\definecolor{currentfill}{rgb}{1.000000,0.498039,0.054902}%
\pgfsetfillcolor{currentfill}%
\pgfsetlinewidth{1.003750pt}%
\definecolor{currentstroke}{rgb}{1.000000,0.498039,0.054902}%
\pgfsetstrokecolor{currentstroke}%
\pgfsetdash{}{0pt}%
\pgfpathmoveto{\pgfqpoint{3.747598in}{2.291595in}}%
\pgfpathcurveto{\pgfqpoint{3.758649in}{2.291595in}}{\pgfqpoint{3.769248in}{2.295985in}}{\pgfqpoint{3.777061in}{2.303799in}}%
\pgfpathcurveto{\pgfqpoint{3.784875in}{2.311612in}}{\pgfqpoint{3.789265in}{2.322211in}}{\pgfqpoint{3.789265in}{2.333261in}}%
\pgfpathcurveto{\pgfqpoint{3.789265in}{2.344312in}}{\pgfqpoint{3.784875in}{2.354911in}}{\pgfqpoint{3.777061in}{2.362724in}}%
\pgfpathcurveto{\pgfqpoint{3.769248in}{2.370538in}}{\pgfqpoint{3.758649in}{2.374928in}}{\pgfqpoint{3.747598in}{2.374928in}}%
\pgfpathcurveto{\pgfqpoint{3.736548in}{2.374928in}}{\pgfqpoint{3.725949in}{2.370538in}}{\pgfqpoint{3.718136in}{2.362724in}}%
\pgfpathcurveto{\pgfqpoint{3.710322in}{2.354911in}}{\pgfqpoint{3.705932in}{2.344312in}}{\pgfqpoint{3.705932in}{2.333261in}}%
\pgfpathcurveto{\pgfqpoint{3.705932in}{2.322211in}}{\pgfqpoint{3.710322in}{2.311612in}}{\pgfqpoint{3.718136in}{2.303799in}}%
\pgfpathcurveto{\pgfqpoint{3.725949in}{2.295985in}}{\pgfqpoint{3.736548in}{2.291595in}}{\pgfqpoint{3.747598in}{2.291595in}}%
\pgfpathclose%
\pgfusepath{stroke,fill}%
\end{pgfscope}%
\begin{pgfscope}%
\pgfpathrectangle{\pgfqpoint{0.787074in}{0.548769in}}{\pgfqpoint{5.062926in}{3.102590in}}%
\pgfusepath{clip}%
\pgfsetbuttcap%
\pgfsetroundjoin%
\definecolor{currentfill}{rgb}{0.121569,0.466667,0.705882}%
\pgfsetfillcolor{currentfill}%
\pgfsetlinewidth{1.003750pt}%
\definecolor{currentstroke}{rgb}{0.121569,0.466667,0.705882}%
\pgfsetstrokecolor{currentstroke}%
\pgfsetdash{}{0pt}%
\pgfpathmoveto{\pgfqpoint{3.435554in}{1.765305in}}%
\pgfpathcurveto{\pgfqpoint{3.446604in}{1.765305in}}{\pgfqpoint{3.457203in}{1.769696in}}{\pgfqpoint{3.465016in}{1.777509in}}%
\pgfpathcurveto{\pgfqpoint{3.472830in}{1.785323in}}{\pgfqpoint{3.477220in}{1.795922in}}{\pgfqpoint{3.477220in}{1.806972in}}%
\pgfpathcurveto{\pgfqpoint{3.477220in}{1.818022in}}{\pgfqpoint{3.472830in}{1.828621in}}{\pgfqpoint{3.465016in}{1.836435in}}%
\pgfpathcurveto{\pgfqpoint{3.457203in}{1.844249in}}{\pgfqpoint{3.446604in}{1.848639in}}{\pgfqpoint{3.435554in}{1.848639in}}%
\pgfpathcurveto{\pgfqpoint{3.424504in}{1.848639in}}{\pgfqpoint{3.413905in}{1.844249in}}{\pgfqpoint{3.406091in}{1.836435in}}%
\pgfpathcurveto{\pgfqpoint{3.398277in}{1.828621in}}{\pgfqpoint{3.393887in}{1.818022in}}{\pgfqpoint{3.393887in}{1.806972in}}%
\pgfpathcurveto{\pgfqpoint{3.393887in}{1.795922in}}{\pgfqpoint{3.398277in}{1.785323in}}{\pgfqpoint{3.406091in}{1.777509in}}%
\pgfpathcurveto{\pgfqpoint{3.413905in}{1.769696in}}{\pgfqpoint{3.424504in}{1.765305in}}{\pgfqpoint{3.435554in}{1.765305in}}%
\pgfpathclose%
\pgfusepath{stroke,fill}%
\end{pgfscope}%
\begin{pgfscope}%
\pgfpathrectangle{\pgfqpoint{0.787074in}{0.548769in}}{\pgfqpoint{5.062926in}{3.102590in}}%
\pgfusepath{clip}%
\pgfsetbuttcap%
\pgfsetroundjoin%
\definecolor{currentfill}{rgb}{1.000000,0.498039,0.054902}%
\pgfsetfillcolor{currentfill}%
\pgfsetlinewidth{1.003750pt}%
\definecolor{currentstroke}{rgb}{1.000000,0.498039,0.054902}%
\pgfsetstrokecolor{currentstroke}%
\pgfsetdash{}{0pt}%
\pgfpathmoveto{\pgfqpoint{3.240526in}{2.168495in}}%
\pgfpathcurveto{\pgfqpoint{3.251576in}{2.168495in}}{\pgfqpoint{3.262175in}{2.172885in}}{\pgfqpoint{3.269989in}{2.180698in}}%
\pgfpathcurveto{\pgfqpoint{3.277802in}{2.188512in}}{\pgfqpoint{3.282192in}{2.199111in}}{\pgfqpoint{3.282192in}{2.210161in}}%
\pgfpathcurveto{\pgfqpoint{3.282192in}{2.221211in}}{\pgfqpoint{3.277802in}{2.231810in}}{\pgfqpoint{3.269989in}{2.239624in}}%
\pgfpathcurveto{\pgfqpoint{3.262175in}{2.247438in}}{\pgfqpoint{3.251576in}{2.251828in}}{\pgfqpoint{3.240526in}{2.251828in}}%
\pgfpathcurveto{\pgfqpoint{3.229476in}{2.251828in}}{\pgfqpoint{3.218877in}{2.247438in}}{\pgfqpoint{3.211063in}{2.239624in}}%
\pgfpathcurveto{\pgfqpoint{3.203249in}{2.231810in}}{\pgfqpoint{3.198859in}{2.221211in}}{\pgfqpoint{3.198859in}{2.210161in}}%
\pgfpathcurveto{\pgfqpoint{3.198859in}{2.199111in}}{\pgfqpoint{3.203249in}{2.188512in}}{\pgfqpoint{3.211063in}{2.180698in}}%
\pgfpathcurveto{\pgfqpoint{3.218877in}{2.172885in}}{\pgfqpoint{3.229476in}{2.168495in}}{\pgfqpoint{3.240526in}{2.168495in}}%
\pgfpathclose%
\pgfusepath{stroke,fill}%
\end{pgfscope}%
\begin{pgfscope}%
\pgfpathrectangle{\pgfqpoint{0.787074in}{0.548769in}}{\pgfqpoint{5.062926in}{3.102590in}}%
\pgfusepath{clip}%
\pgfsetbuttcap%
\pgfsetroundjoin%
\definecolor{currentfill}{rgb}{1.000000,0.498039,0.054902}%
\pgfsetfillcolor{currentfill}%
\pgfsetlinewidth{1.003750pt}%
\definecolor{currentstroke}{rgb}{1.000000,0.498039,0.054902}%
\pgfsetstrokecolor{currentstroke}%
\pgfsetdash{}{0pt}%
\pgfpathmoveto{\pgfqpoint{3.591576in}{2.330781in}}%
\pgfpathcurveto{\pgfqpoint{3.602626in}{2.330781in}}{\pgfqpoint{3.613225in}{2.335171in}}{\pgfqpoint{3.621039in}{2.342985in}}%
\pgfpathcurveto{\pgfqpoint{3.628852in}{2.350799in}}{\pgfqpoint{3.633243in}{2.361398in}}{\pgfqpoint{3.633243in}{2.372448in}}%
\pgfpathcurveto{\pgfqpoint{3.633243in}{2.383498in}}{\pgfqpoint{3.628852in}{2.394097in}}{\pgfqpoint{3.621039in}{2.401910in}}%
\pgfpathcurveto{\pgfqpoint{3.613225in}{2.409724in}}{\pgfqpoint{3.602626in}{2.414114in}}{\pgfqpoint{3.591576in}{2.414114in}}%
\pgfpathcurveto{\pgfqpoint{3.580526in}{2.414114in}}{\pgfqpoint{3.569927in}{2.409724in}}{\pgfqpoint{3.562113in}{2.401910in}}%
\pgfpathcurveto{\pgfqpoint{3.554300in}{2.394097in}}{\pgfqpoint{3.549909in}{2.383498in}}{\pgfqpoint{3.549909in}{2.372448in}}%
\pgfpathcurveto{\pgfqpoint{3.549909in}{2.361398in}}{\pgfqpoint{3.554300in}{2.350799in}}{\pgfqpoint{3.562113in}{2.342985in}}%
\pgfpathcurveto{\pgfqpoint{3.569927in}{2.335171in}}{\pgfqpoint{3.580526in}{2.330781in}}{\pgfqpoint{3.591576in}{2.330781in}}%
\pgfpathclose%
\pgfusepath{stroke,fill}%
\end{pgfscope}%
\begin{pgfscope}%
\pgfpathrectangle{\pgfqpoint{0.787074in}{0.548769in}}{\pgfqpoint{5.062926in}{3.102590in}}%
\pgfusepath{clip}%
\pgfsetbuttcap%
\pgfsetroundjoin%
\definecolor{currentfill}{rgb}{1.000000,0.498039,0.054902}%
\pgfsetfillcolor{currentfill}%
\pgfsetlinewidth{1.003750pt}%
\definecolor{currentstroke}{rgb}{1.000000,0.498039,0.054902}%
\pgfsetstrokecolor{currentstroke}%
\pgfsetdash{}{0pt}%
\pgfpathmoveto{\pgfqpoint{3.591576in}{2.341609in}}%
\pgfpathcurveto{\pgfqpoint{3.602626in}{2.341609in}}{\pgfqpoint{3.613225in}{2.345999in}}{\pgfqpoint{3.621039in}{2.353813in}}%
\pgfpathcurveto{\pgfqpoint{3.628852in}{2.361626in}}{\pgfqpoint{3.633243in}{2.372225in}}{\pgfqpoint{3.633243in}{2.383275in}}%
\pgfpathcurveto{\pgfqpoint{3.633243in}{2.394326in}}{\pgfqpoint{3.628852in}{2.404925in}}{\pgfqpoint{3.621039in}{2.412738in}}%
\pgfpathcurveto{\pgfqpoint{3.613225in}{2.420552in}}{\pgfqpoint{3.602626in}{2.424942in}}{\pgfqpoint{3.591576in}{2.424942in}}%
\pgfpathcurveto{\pgfqpoint{3.580526in}{2.424942in}}{\pgfqpoint{3.569927in}{2.420552in}}{\pgfqpoint{3.562113in}{2.412738in}}%
\pgfpathcurveto{\pgfqpoint{3.554300in}{2.404925in}}{\pgfqpoint{3.549909in}{2.394326in}}{\pgfqpoint{3.549909in}{2.383275in}}%
\pgfpathcurveto{\pgfqpoint{3.549909in}{2.372225in}}{\pgfqpoint{3.554300in}{2.361626in}}{\pgfqpoint{3.562113in}{2.353813in}}%
\pgfpathcurveto{\pgfqpoint{3.569927in}{2.345999in}}{\pgfqpoint{3.580526in}{2.341609in}}{\pgfqpoint{3.591576in}{2.341609in}}%
\pgfpathclose%
\pgfusepath{stroke,fill}%
\end{pgfscope}%
\begin{pgfscope}%
\pgfpathrectangle{\pgfqpoint{0.787074in}{0.548769in}}{\pgfqpoint{5.062926in}{3.102590in}}%
\pgfusepath{clip}%
\pgfsetbuttcap%
\pgfsetroundjoin%
\definecolor{currentfill}{rgb}{1.000000,0.498039,0.054902}%
\pgfsetfillcolor{currentfill}%
\pgfsetlinewidth{1.003750pt}%
\definecolor{currentstroke}{rgb}{1.000000,0.498039,0.054902}%
\pgfsetstrokecolor{currentstroke}%
\pgfsetdash{}{0pt}%
\pgfpathmoveto{\pgfqpoint{4.059643in}{2.708305in}}%
\pgfpathcurveto{\pgfqpoint{4.070693in}{2.708305in}}{\pgfqpoint{4.081292in}{2.712695in}}{\pgfqpoint{4.089106in}{2.720509in}}%
\pgfpathcurveto{\pgfqpoint{4.096920in}{2.728322in}}{\pgfqpoint{4.101310in}{2.738921in}}{\pgfqpoint{4.101310in}{2.749972in}}%
\pgfpathcurveto{\pgfqpoint{4.101310in}{2.761022in}}{\pgfqpoint{4.096920in}{2.771621in}}{\pgfqpoint{4.089106in}{2.779434in}}%
\pgfpathcurveto{\pgfqpoint{4.081292in}{2.787248in}}{\pgfqpoint{4.070693in}{2.791638in}}{\pgfqpoint{4.059643in}{2.791638in}}%
\pgfpathcurveto{\pgfqpoint{4.048593in}{2.791638in}}{\pgfqpoint{4.037994in}{2.787248in}}{\pgfqpoint{4.030180in}{2.779434in}}%
\pgfpathcurveto{\pgfqpoint{4.022367in}{2.771621in}}{\pgfqpoint{4.017977in}{2.761022in}}{\pgfqpoint{4.017977in}{2.749972in}}%
\pgfpathcurveto{\pgfqpoint{4.017977in}{2.738921in}}{\pgfqpoint{4.022367in}{2.728322in}}{\pgfqpoint{4.030180in}{2.720509in}}%
\pgfpathcurveto{\pgfqpoint{4.037994in}{2.712695in}}{\pgfqpoint{4.048593in}{2.708305in}}{\pgfqpoint{4.059643in}{2.708305in}}%
\pgfpathclose%
\pgfusepath{stroke,fill}%
\end{pgfscope}%
\begin{pgfscope}%
\pgfpathrectangle{\pgfqpoint{0.787074in}{0.548769in}}{\pgfqpoint{5.062926in}{3.102590in}}%
\pgfusepath{clip}%
\pgfsetbuttcap%
\pgfsetroundjoin%
\definecolor{currentfill}{rgb}{1.000000,0.498039,0.054902}%
\pgfsetfillcolor{currentfill}%
\pgfsetlinewidth{1.003750pt}%
\definecolor{currentstroke}{rgb}{1.000000,0.498039,0.054902}%
\pgfsetstrokecolor{currentstroke}%
\pgfsetdash{}{0pt}%
\pgfpathmoveto{\pgfqpoint{4.371688in}{2.886489in}}%
\pgfpathcurveto{\pgfqpoint{4.382738in}{2.886489in}}{\pgfqpoint{4.393337in}{2.890880in}}{\pgfqpoint{4.401151in}{2.898693in}}%
\pgfpathcurveto{\pgfqpoint{4.408964in}{2.906507in}}{\pgfqpoint{4.413355in}{2.917106in}}{\pgfqpoint{4.413355in}{2.928156in}}%
\pgfpathcurveto{\pgfqpoint{4.413355in}{2.939206in}}{\pgfqpoint{4.408964in}{2.949805in}}{\pgfqpoint{4.401151in}{2.957619in}}%
\pgfpathcurveto{\pgfqpoint{4.393337in}{2.965433in}}{\pgfqpoint{4.382738in}{2.969823in}}{\pgfqpoint{4.371688in}{2.969823in}}%
\pgfpathcurveto{\pgfqpoint{4.360638in}{2.969823in}}{\pgfqpoint{4.350039in}{2.965433in}}{\pgfqpoint{4.342225in}{2.957619in}}%
\pgfpathcurveto{\pgfqpoint{4.334412in}{2.949805in}}{\pgfqpoint{4.330021in}{2.939206in}}{\pgfqpoint{4.330021in}{2.928156in}}%
\pgfpathcurveto{\pgfqpoint{4.330021in}{2.917106in}}{\pgfqpoint{4.334412in}{2.906507in}}{\pgfqpoint{4.342225in}{2.898693in}}%
\pgfpathcurveto{\pgfqpoint{4.350039in}{2.890880in}}{\pgfqpoint{4.360638in}{2.886489in}}{\pgfqpoint{4.371688in}{2.886489in}}%
\pgfpathclose%
\pgfusepath{stroke,fill}%
\end{pgfscope}%
\begin{pgfscope}%
\pgfpathrectangle{\pgfqpoint{0.787074in}{0.548769in}}{\pgfqpoint{5.062926in}{3.102590in}}%
\pgfusepath{clip}%
\pgfsetbuttcap%
\pgfsetroundjoin%
\definecolor{currentfill}{rgb}{1.000000,0.498039,0.054902}%
\pgfsetfillcolor{currentfill}%
\pgfsetlinewidth{1.003750pt}%
\definecolor{currentstroke}{rgb}{1.000000,0.498039,0.054902}%
\pgfsetstrokecolor{currentstroke}%
\pgfsetdash{}{0pt}%
\pgfpathmoveto{\pgfqpoint{3.279531in}{2.003186in}}%
\pgfpathcurveto{\pgfqpoint{3.290581in}{2.003186in}}{\pgfqpoint{3.301180in}{2.007576in}}{\pgfqpoint{3.308994in}{2.015390in}}%
\pgfpathcurveto{\pgfqpoint{3.316808in}{2.023204in}}{\pgfqpoint{3.321198in}{2.033803in}}{\pgfqpoint{3.321198in}{2.044853in}}%
\pgfpathcurveto{\pgfqpoint{3.321198in}{2.055903in}}{\pgfqpoint{3.316808in}{2.066502in}}{\pgfqpoint{3.308994in}{2.074316in}}%
\pgfpathcurveto{\pgfqpoint{3.301180in}{2.082129in}}{\pgfqpoint{3.290581in}{2.086520in}}{\pgfqpoint{3.279531in}{2.086520in}}%
\pgfpathcurveto{\pgfqpoint{3.268481in}{2.086520in}}{\pgfqpoint{3.257882in}{2.082129in}}{\pgfqpoint{3.250069in}{2.074316in}}%
\pgfpathcurveto{\pgfqpoint{3.242255in}{2.066502in}}{\pgfqpoint{3.237865in}{2.055903in}}{\pgfqpoint{3.237865in}{2.044853in}}%
\pgfpathcurveto{\pgfqpoint{3.237865in}{2.033803in}}{\pgfqpoint{3.242255in}{2.023204in}}{\pgfqpoint{3.250069in}{2.015390in}}%
\pgfpathcurveto{\pgfqpoint{3.257882in}{2.007576in}}{\pgfqpoint{3.268481in}{2.003186in}}{\pgfqpoint{3.279531in}{2.003186in}}%
\pgfpathclose%
\pgfusepath{stroke,fill}%
\end{pgfscope}%
\begin{pgfscope}%
\pgfpathrectangle{\pgfqpoint{0.787074in}{0.548769in}}{\pgfqpoint{5.062926in}{3.102590in}}%
\pgfusepath{clip}%
\pgfsetbuttcap%
\pgfsetroundjoin%
\definecolor{currentfill}{rgb}{0.121569,0.466667,0.705882}%
\pgfsetfillcolor{currentfill}%
\pgfsetlinewidth{1.003750pt}%
\definecolor{currentstroke}{rgb}{0.121569,0.466667,0.705882}%
\pgfsetstrokecolor{currentstroke}%
\pgfsetdash{}{0pt}%
\pgfpathmoveto{\pgfqpoint{3.669587in}{0.942348in}}%
\pgfpathcurveto{\pgfqpoint{3.680637in}{0.942348in}}{\pgfqpoint{3.691236in}{0.946738in}}{\pgfqpoint{3.699050in}{0.954552in}}%
\pgfpathcurveto{\pgfqpoint{3.706864in}{0.962365in}}{\pgfqpoint{3.711254in}{0.972964in}}{\pgfqpoint{3.711254in}{0.984014in}}%
\pgfpathcurveto{\pgfqpoint{3.711254in}{0.995065in}}{\pgfqpoint{3.706864in}{1.005664in}}{\pgfqpoint{3.699050in}{1.013477in}}%
\pgfpathcurveto{\pgfqpoint{3.691236in}{1.021291in}}{\pgfqpoint{3.680637in}{1.025681in}}{\pgfqpoint{3.669587in}{1.025681in}}%
\pgfpathcurveto{\pgfqpoint{3.658537in}{1.025681in}}{\pgfqpoint{3.647938in}{1.021291in}}{\pgfqpoint{3.640124in}{1.013477in}}%
\pgfpathcurveto{\pgfqpoint{3.632311in}{1.005664in}}{\pgfqpoint{3.627921in}{0.995065in}}{\pgfqpoint{3.627921in}{0.984014in}}%
\pgfpathcurveto{\pgfqpoint{3.627921in}{0.972964in}}{\pgfqpoint{3.632311in}{0.962365in}}{\pgfqpoint{3.640124in}{0.954552in}}%
\pgfpathcurveto{\pgfqpoint{3.647938in}{0.946738in}}{\pgfqpoint{3.658537in}{0.942348in}}{\pgfqpoint{3.669587in}{0.942348in}}%
\pgfpathclose%
\pgfusepath{stroke,fill}%
\end{pgfscope}%
\begin{pgfscope}%
\pgfpathrectangle{\pgfqpoint{0.787074in}{0.548769in}}{\pgfqpoint{5.062926in}{3.102590in}}%
\pgfusepath{clip}%
\pgfsetbuttcap%
\pgfsetroundjoin%
\definecolor{currentfill}{rgb}{1.000000,0.498039,0.054902}%
\pgfsetfillcolor{currentfill}%
\pgfsetlinewidth{1.003750pt}%
\definecolor{currentstroke}{rgb}{1.000000,0.498039,0.054902}%
\pgfsetstrokecolor{currentstroke}%
\pgfsetdash{}{0pt}%
\pgfpathmoveto{\pgfqpoint{3.669587in}{2.620166in}}%
\pgfpathcurveto{\pgfqpoint{3.680637in}{2.620166in}}{\pgfqpoint{3.691236in}{2.624556in}}{\pgfqpoint{3.699050in}{2.632370in}}%
\pgfpathcurveto{\pgfqpoint{3.706864in}{2.640183in}}{\pgfqpoint{3.711254in}{2.650782in}}{\pgfqpoint{3.711254in}{2.661832in}}%
\pgfpathcurveto{\pgfqpoint{3.711254in}{2.672883in}}{\pgfqpoint{3.706864in}{2.683482in}}{\pgfqpoint{3.699050in}{2.691295in}}%
\pgfpathcurveto{\pgfqpoint{3.691236in}{2.699109in}}{\pgfqpoint{3.680637in}{2.703499in}}{\pgfqpoint{3.669587in}{2.703499in}}%
\pgfpathcurveto{\pgfqpoint{3.658537in}{2.703499in}}{\pgfqpoint{3.647938in}{2.699109in}}{\pgfqpoint{3.640124in}{2.691295in}}%
\pgfpathcurveto{\pgfqpoint{3.632311in}{2.683482in}}{\pgfqpoint{3.627921in}{2.672883in}}{\pgfqpoint{3.627921in}{2.661832in}}%
\pgfpathcurveto{\pgfqpoint{3.627921in}{2.650782in}}{\pgfqpoint{3.632311in}{2.640183in}}{\pgfqpoint{3.640124in}{2.632370in}}%
\pgfpathcurveto{\pgfqpoint{3.647938in}{2.624556in}}{\pgfqpoint{3.658537in}{2.620166in}}{\pgfqpoint{3.669587in}{2.620166in}}%
\pgfpathclose%
\pgfusepath{stroke,fill}%
\end{pgfscope}%
\begin{pgfscope}%
\pgfpathrectangle{\pgfqpoint{0.787074in}{0.548769in}}{\pgfqpoint{5.062926in}{3.102590in}}%
\pgfusepath{clip}%
\pgfsetbuttcap%
\pgfsetroundjoin%
\definecolor{currentfill}{rgb}{1.000000,0.498039,0.054902}%
\pgfsetfillcolor{currentfill}%
\pgfsetlinewidth{1.003750pt}%
\definecolor{currentstroke}{rgb}{1.000000,0.498039,0.054902}%
\pgfsetstrokecolor{currentstroke}%
\pgfsetdash{}{0pt}%
\pgfpathmoveto{\pgfqpoint{3.669587in}{2.991999in}}%
\pgfpathcurveto{\pgfqpoint{3.680637in}{2.991999in}}{\pgfqpoint{3.691236in}{2.996389in}}{\pgfqpoint{3.699050in}{3.004203in}}%
\pgfpathcurveto{\pgfqpoint{3.706864in}{3.012016in}}{\pgfqpoint{3.711254in}{3.022615in}}{\pgfqpoint{3.711254in}{3.033665in}}%
\pgfpathcurveto{\pgfqpoint{3.711254in}{3.044716in}}{\pgfqpoint{3.706864in}{3.055315in}}{\pgfqpoint{3.699050in}{3.063128in}}%
\pgfpathcurveto{\pgfqpoint{3.691236in}{3.070942in}}{\pgfqpoint{3.680637in}{3.075332in}}{\pgfqpoint{3.669587in}{3.075332in}}%
\pgfpathcurveto{\pgfqpoint{3.658537in}{3.075332in}}{\pgfqpoint{3.647938in}{3.070942in}}{\pgfqpoint{3.640124in}{3.063128in}}%
\pgfpathcurveto{\pgfqpoint{3.632311in}{3.055315in}}{\pgfqpoint{3.627921in}{3.044716in}}{\pgfqpoint{3.627921in}{3.033665in}}%
\pgfpathcurveto{\pgfqpoint{3.627921in}{3.022615in}}{\pgfqpoint{3.632311in}{3.012016in}}{\pgfqpoint{3.640124in}{3.004203in}}%
\pgfpathcurveto{\pgfqpoint{3.647938in}{2.996389in}}{\pgfqpoint{3.658537in}{2.991999in}}{\pgfqpoint{3.669587in}{2.991999in}}%
\pgfpathclose%
\pgfusepath{stroke,fill}%
\end{pgfscope}%
\begin{pgfscope}%
\pgfpathrectangle{\pgfqpoint{0.787074in}{0.548769in}}{\pgfqpoint{5.062926in}{3.102590in}}%
\pgfusepath{clip}%
\pgfsetbuttcap%
\pgfsetroundjoin%
\definecolor{currentfill}{rgb}{1.000000,0.498039,0.054902}%
\pgfsetfillcolor{currentfill}%
\pgfsetlinewidth{1.003750pt}%
\definecolor{currentstroke}{rgb}{1.000000,0.498039,0.054902}%
\pgfsetstrokecolor{currentstroke}%
\pgfsetdash{}{0pt}%
\pgfpathmoveto{\pgfqpoint{3.708593in}{3.136678in}}%
\pgfpathcurveto{\pgfqpoint{3.719643in}{3.136678in}}{\pgfqpoint{3.730242in}{3.141068in}}{\pgfqpoint{3.738056in}{3.148882in}}%
\pgfpathcurveto{\pgfqpoint{3.745869in}{3.156695in}}{\pgfqpoint{3.750260in}{3.167294in}}{\pgfqpoint{3.750260in}{3.178345in}}%
\pgfpathcurveto{\pgfqpoint{3.750260in}{3.189395in}}{\pgfqpoint{3.745869in}{3.199994in}}{\pgfqpoint{3.738056in}{3.207807in}}%
\pgfpathcurveto{\pgfqpoint{3.730242in}{3.215621in}}{\pgfqpoint{3.719643in}{3.220011in}}{\pgfqpoint{3.708593in}{3.220011in}}%
\pgfpathcurveto{\pgfqpoint{3.697543in}{3.220011in}}{\pgfqpoint{3.686944in}{3.215621in}}{\pgfqpoint{3.679130in}{3.207807in}}%
\pgfpathcurveto{\pgfqpoint{3.671316in}{3.199994in}}{\pgfqpoint{3.666926in}{3.189395in}}{\pgfqpoint{3.666926in}{3.178345in}}%
\pgfpathcurveto{\pgfqpoint{3.666926in}{3.167294in}}{\pgfqpoint{3.671316in}{3.156695in}}{\pgfqpoint{3.679130in}{3.148882in}}%
\pgfpathcurveto{\pgfqpoint{3.686944in}{3.141068in}}{\pgfqpoint{3.697543in}{3.136678in}}{\pgfqpoint{3.708593in}{3.136678in}}%
\pgfpathclose%
\pgfusepath{stroke,fill}%
\end{pgfscope}%
\begin{pgfscope}%
\pgfpathrectangle{\pgfqpoint{0.787074in}{0.548769in}}{\pgfqpoint{5.062926in}{3.102590in}}%
\pgfusepath{clip}%
\pgfsetbuttcap%
\pgfsetroundjoin%
\definecolor{currentfill}{rgb}{1.000000,0.498039,0.054902}%
\pgfsetfillcolor{currentfill}%
\pgfsetlinewidth{1.003750pt}%
\definecolor{currentstroke}{rgb}{1.000000,0.498039,0.054902}%
\pgfsetstrokecolor{currentstroke}%
\pgfsetdash{}{0pt}%
\pgfpathmoveto{\pgfqpoint{3.240526in}{2.684169in}}%
\pgfpathcurveto{\pgfqpoint{3.251576in}{2.684169in}}{\pgfqpoint{3.262175in}{2.688559in}}{\pgfqpoint{3.269989in}{2.696373in}}%
\pgfpathcurveto{\pgfqpoint{3.277802in}{2.704186in}}{\pgfqpoint{3.282192in}{2.714785in}}{\pgfqpoint{3.282192in}{2.725835in}}%
\pgfpathcurveto{\pgfqpoint{3.282192in}{2.736886in}}{\pgfqpoint{3.277802in}{2.747485in}}{\pgfqpoint{3.269989in}{2.755298in}}%
\pgfpathcurveto{\pgfqpoint{3.262175in}{2.763112in}}{\pgfqpoint{3.251576in}{2.767502in}}{\pgfqpoint{3.240526in}{2.767502in}}%
\pgfpathcurveto{\pgfqpoint{3.229476in}{2.767502in}}{\pgfqpoint{3.218877in}{2.763112in}}{\pgfqpoint{3.211063in}{2.755298in}}%
\pgfpathcurveto{\pgfqpoint{3.203249in}{2.747485in}}{\pgfqpoint{3.198859in}{2.736886in}}{\pgfqpoint{3.198859in}{2.725835in}}%
\pgfpathcurveto{\pgfqpoint{3.198859in}{2.714785in}}{\pgfqpoint{3.203249in}{2.704186in}}{\pgfqpoint{3.211063in}{2.696373in}}%
\pgfpathcurveto{\pgfqpoint{3.218877in}{2.688559in}}{\pgfqpoint{3.229476in}{2.684169in}}{\pgfqpoint{3.240526in}{2.684169in}}%
\pgfpathclose%
\pgfusepath{stroke,fill}%
\end{pgfscope}%
\begin{pgfscope}%
\pgfpathrectangle{\pgfqpoint{0.787074in}{0.548769in}}{\pgfqpoint{5.062926in}{3.102590in}}%
\pgfusepath{clip}%
\pgfsetbuttcap%
\pgfsetroundjoin%
\definecolor{currentfill}{rgb}{1.000000,0.498039,0.054902}%
\pgfsetfillcolor{currentfill}%
\pgfsetlinewidth{1.003750pt}%
\definecolor{currentstroke}{rgb}{1.000000,0.498039,0.054902}%
\pgfsetstrokecolor{currentstroke}%
\pgfsetdash{}{0pt}%
\pgfpathmoveto{\pgfqpoint{4.293677in}{2.925050in}}%
\pgfpathcurveto{\pgfqpoint{4.304727in}{2.925050in}}{\pgfqpoint{4.315326in}{2.929440in}}{\pgfqpoint{4.323140in}{2.937253in}}%
\pgfpathcurveto{\pgfqpoint{4.330953in}{2.945067in}}{\pgfqpoint{4.335343in}{2.955666in}}{\pgfqpoint{4.335343in}{2.966716in}}%
\pgfpathcurveto{\pgfqpoint{4.335343in}{2.977766in}}{\pgfqpoint{4.330953in}{2.988365in}}{\pgfqpoint{4.323140in}{2.996179in}}%
\pgfpathcurveto{\pgfqpoint{4.315326in}{3.003993in}}{\pgfqpoint{4.304727in}{3.008383in}}{\pgfqpoint{4.293677in}{3.008383in}}%
\pgfpathcurveto{\pgfqpoint{4.282627in}{3.008383in}}{\pgfqpoint{4.272028in}{3.003993in}}{\pgfqpoint{4.264214in}{2.996179in}}%
\pgfpathcurveto{\pgfqpoint{4.256400in}{2.988365in}}{\pgfqpoint{4.252010in}{2.977766in}}{\pgfqpoint{4.252010in}{2.966716in}}%
\pgfpathcurveto{\pgfqpoint{4.252010in}{2.955666in}}{\pgfqpoint{4.256400in}{2.945067in}}{\pgfqpoint{4.264214in}{2.937253in}}%
\pgfpathcurveto{\pgfqpoint{4.272028in}{2.929440in}}{\pgfqpoint{4.282627in}{2.925050in}}{\pgfqpoint{4.293677in}{2.925050in}}%
\pgfpathclose%
\pgfusepath{stroke,fill}%
\end{pgfscope}%
\begin{pgfscope}%
\pgfpathrectangle{\pgfqpoint{0.787074in}{0.548769in}}{\pgfqpoint{5.062926in}{3.102590in}}%
\pgfusepath{clip}%
\pgfsetbuttcap%
\pgfsetroundjoin%
\definecolor{currentfill}{rgb}{0.121569,0.466667,0.705882}%
\pgfsetfillcolor{currentfill}%
\pgfsetlinewidth{1.003750pt}%
\definecolor{currentstroke}{rgb}{0.121569,0.466667,0.705882}%
\pgfsetstrokecolor{currentstroke}%
\pgfsetdash{}{0pt}%
\pgfpathmoveto{\pgfqpoint{3.786604in}{2.772349in}}%
\pgfpathcurveto{\pgfqpoint{3.797654in}{2.772349in}}{\pgfqpoint{3.808253in}{2.776739in}}{\pgfqpoint{3.816067in}{2.784553in}}%
\pgfpathcurveto{\pgfqpoint{3.823880in}{2.792367in}}{\pgfqpoint{3.828271in}{2.802966in}}{\pgfqpoint{3.828271in}{2.814016in}}%
\pgfpathcurveto{\pgfqpoint{3.828271in}{2.825066in}}{\pgfqpoint{3.823880in}{2.835665in}}{\pgfqpoint{3.816067in}{2.843479in}}%
\pgfpathcurveto{\pgfqpoint{3.808253in}{2.851292in}}{\pgfqpoint{3.797654in}{2.855683in}}{\pgfqpoint{3.786604in}{2.855683in}}%
\pgfpathcurveto{\pgfqpoint{3.775554in}{2.855683in}}{\pgfqpoint{3.764955in}{2.851292in}}{\pgfqpoint{3.757141in}{2.843479in}}%
\pgfpathcurveto{\pgfqpoint{3.749328in}{2.835665in}}{\pgfqpoint{3.744937in}{2.825066in}}{\pgfqpoint{3.744937in}{2.814016in}}%
\pgfpathcurveto{\pgfqpoint{3.744937in}{2.802966in}}{\pgfqpoint{3.749328in}{2.792367in}}{\pgfqpoint{3.757141in}{2.784553in}}%
\pgfpathcurveto{\pgfqpoint{3.764955in}{2.776739in}}{\pgfqpoint{3.775554in}{2.772349in}}{\pgfqpoint{3.786604in}{2.772349in}}%
\pgfpathclose%
\pgfusepath{stroke,fill}%
\end{pgfscope}%
\begin{pgfscope}%
\pgfpathrectangle{\pgfqpoint{0.787074in}{0.548769in}}{\pgfqpoint{5.062926in}{3.102590in}}%
\pgfusepath{clip}%
\pgfsetbuttcap%
\pgfsetroundjoin%
\definecolor{currentfill}{rgb}{1.000000,0.498039,0.054902}%
\pgfsetfillcolor{currentfill}%
\pgfsetlinewidth{1.003750pt}%
\definecolor{currentstroke}{rgb}{1.000000,0.498039,0.054902}%
\pgfsetstrokecolor{currentstroke}%
\pgfsetdash{}{0pt}%
\pgfpathmoveto{\pgfqpoint{4.098649in}{3.273091in}}%
\pgfpathcurveto{\pgfqpoint{4.109699in}{3.273091in}}{\pgfqpoint{4.120298in}{3.277482in}}{\pgfqpoint{4.128112in}{3.285295in}}%
\pgfpathcurveto{\pgfqpoint{4.135925in}{3.293109in}}{\pgfqpoint{4.140315in}{3.303708in}}{\pgfqpoint{4.140315in}{3.314758in}}%
\pgfpathcurveto{\pgfqpoint{4.140315in}{3.325808in}}{\pgfqpoint{4.135925in}{3.336407in}}{\pgfqpoint{4.128112in}{3.344221in}}%
\pgfpathcurveto{\pgfqpoint{4.120298in}{3.352035in}}{\pgfqpoint{4.109699in}{3.356425in}}{\pgfqpoint{4.098649in}{3.356425in}}%
\pgfpathcurveto{\pgfqpoint{4.087599in}{3.356425in}}{\pgfqpoint{4.077000in}{3.352035in}}{\pgfqpoint{4.069186in}{3.344221in}}%
\pgfpathcurveto{\pgfqpoint{4.061372in}{3.336407in}}{\pgfqpoint{4.056982in}{3.325808in}}{\pgfqpoint{4.056982in}{3.314758in}}%
\pgfpathcurveto{\pgfqpoint{4.056982in}{3.303708in}}{\pgfqpoint{4.061372in}{3.293109in}}{\pgfqpoint{4.069186in}{3.285295in}}%
\pgfpathcurveto{\pgfqpoint{4.077000in}{3.277482in}}{\pgfqpoint{4.087599in}{3.273091in}}{\pgfqpoint{4.098649in}{3.273091in}}%
\pgfpathclose%
\pgfusepath{stroke,fill}%
\end{pgfscope}%
\begin{pgfscope}%
\pgfpathrectangle{\pgfqpoint{0.787074in}{0.548769in}}{\pgfqpoint{5.062926in}{3.102590in}}%
\pgfusepath{clip}%
\pgfsetbuttcap%
\pgfsetroundjoin%
\definecolor{currentfill}{rgb}{1.000000,0.498039,0.054902}%
\pgfsetfillcolor{currentfill}%
\pgfsetlinewidth{1.003750pt}%
\definecolor{currentstroke}{rgb}{1.000000,0.498039,0.054902}%
\pgfsetstrokecolor{currentstroke}%
\pgfsetdash{}{0pt}%
\pgfpathmoveto{\pgfqpoint{3.357543in}{2.433412in}}%
\pgfpathcurveto{\pgfqpoint{3.368593in}{2.433412in}}{\pgfqpoint{3.379192in}{2.437803in}}{\pgfqpoint{3.387005in}{2.445616in}}%
\pgfpathcurveto{\pgfqpoint{3.394819in}{2.453430in}}{\pgfqpoint{3.399209in}{2.464029in}}{\pgfqpoint{3.399209in}{2.475079in}}%
\pgfpathcurveto{\pgfqpoint{3.399209in}{2.486129in}}{\pgfqpoint{3.394819in}{2.496728in}}{\pgfqpoint{3.387005in}{2.504542in}}%
\pgfpathcurveto{\pgfqpoint{3.379192in}{2.512356in}}{\pgfqpoint{3.368593in}{2.516746in}}{\pgfqpoint{3.357543in}{2.516746in}}%
\pgfpathcurveto{\pgfqpoint{3.346492in}{2.516746in}}{\pgfqpoint{3.335893in}{2.512356in}}{\pgfqpoint{3.328080in}{2.504542in}}%
\pgfpathcurveto{\pgfqpoint{3.320266in}{2.496728in}}{\pgfqpoint{3.315876in}{2.486129in}}{\pgfqpoint{3.315876in}{2.475079in}}%
\pgfpathcurveto{\pgfqpoint{3.315876in}{2.464029in}}{\pgfqpoint{3.320266in}{2.453430in}}{\pgfqpoint{3.328080in}{2.445616in}}%
\pgfpathcurveto{\pgfqpoint{3.335893in}{2.437803in}}{\pgfqpoint{3.346492in}{2.433412in}}{\pgfqpoint{3.357543in}{2.433412in}}%
\pgfpathclose%
\pgfusepath{stroke,fill}%
\end{pgfscope}%
\begin{pgfscope}%
\pgfpathrectangle{\pgfqpoint{0.787074in}{0.548769in}}{\pgfqpoint{5.062926in}{3.102590in}}%
\pgfusepath{clip}%
\pgfsetbuttcap%
\pgfsetroundjoin%
\definecolor{currentfill}{rgb}{1.000000,0.498039,0.054902}%
\pgfsetfillcolor{currentfill}%
\pgfsetlinewidth{1.003750pt}%
\definecolor{currentstroke}{rgb}{1.000000,0.498039,0.054902}%
\pgfsetstrokecolor{currentstroke}%
\pgfsetdash{}{0pt}%
\pgfpathmoveto{\pgfqpoint{3.669587in}{2.784629in}}%
\pgfpathcurveto{\pgfqpoint{3.680637in}{2.784629in}}{\pgfqpoint{3.691236in}{2.789019in}}{\pgfqpoint{3.699050in}{2.796833in}}%
\pgfpathcurveto{\pgfqpoint{3.706864in}{2.804646in}}{\pgfqpoint{3.711254in}{2.815245in}}{\pgfqpoint{3.711254in}{2.826296in}}%
\pgfpathcurveto{\pgfqpoint{3.711254in}{2.837346in}}{\pgfqpoint{3.706864in}{2.847945in}}{\pgfqpoint{3.699050in}{2.855758in}}%
\pgfpathcurveto{\pgfqpoint{3.691236in}{2.863572in}}{\pgfqpoint{3.680637in}{2.867962in}}{\pgfqpoint{3.669587in}{2.867962in}}%
\pgfpathcurveto{\pgfqpoint{3.658537in}{2.867962in}}{\pgfqpoint{3.647938in}{2.863572in}}{\pgfqpoint{3.640124in}{2.855758in}}%
\pgfpathcurveto{\pgfqpoint{3.632311in}{2.847945in}}{\pgfqpoint{3.627921in}{2.837346in}}{\pgfqpoint{3.627921in}{2.826296in}}%
\pgfpathcurveto{\pgfqpoint{3.627921in}{2.815245in}}{\pgfqpoint{3.632311in}{2.804646in}}{\pgfqpoint{3.640124in}{2.796833in}}%
\pgfpathcurveto{\pgfqpoint{3.647938in}{2.789019in}}{\pgfqpoint{3.658537in}{2.784629in}}{\pgfqpoint{3.669587in}{2.784629in}}%
\pgfpathclose%
\pgfusepath{stroke,fill}%
\end{pgfscope}%
\begin{pgfscope}%
\pgfpathrectangle{\pgfqpoint{0.787074in}{0.548769in}}{\pgfqpoint{5.062926in}{3.102590in}}%
\pgfusepath{clip}%
\pgfsetbuttcap%
\pgfsetroundjoin%
\definecolor{currentfill}{rgb}{1.000000,0.498039,0.054902}%
\pgfsetfillcolor{currentfill}%
\pgfsetlinewidth{1.003750pt}%
\definecolor{currentstroke}{rgb}{1.000000,0.498039,0.054902}%
\pgfsetstrokecolor{currentstroke}%
\pgfsetdash{}{0pt}%
\pgfpathmoveto{\pgfqpoint{3.240526in}{2.406321in}}%
\pgfpathcurveto{\pgfqpoint{3.251576in}{2.406321in}}{\pgfqpoint{3.262175in}{2.410712in}}{\pgfqpoint{3.269989in}{2.418525in}}%
\pgfpathcurveto{\pgfqpoint{3.277802in}{2.426339in}}{\pgfqpoint{3.282192in}{2.436938in}}{\pgfqpoint{3.282192in}{2.447988in}}%
\pgfpathcurveto{\pgfqpoint{3.282192in}{2.459038in}}{\pgfqpoint{3.277802in}{2.469637in}}{\pgfqpoint{3.269989in}{2.477451in}}%
\pgfpathcurveto{\pgfqpoint{3.262175in}{2.485264in}}{\pgfqpoint{3.251576in}{2.489655in}}{\pgfqpoint{3.240526in}{2.489655in}}%
\pgfpathcurveto{\pgfqpoint{3.229476in}{2.489655in}}{\pgfqpoint{3.218877in}{2.485264in}}{\pgfqpoint{3.211063in}{2.477451in}}%
\pgfpathcurveto{\pgfqpoint{3.203249in}{2.469637in}}{\pgfqpoint{3.198859in}{2.459038in}}{\pgfqpoint{3.198859in}{2.447988in}}%
\pgfpathcurveto{\pgfqpoint{3.198859in}{2.436938in}}{\pgfqpoint{3.203249in}{2.426339in}}{\pgfqpoint{3.211063in}{2.418525in}}%
\pgfpathcurveto{\pgfqpoint{3.218877in}{2.410712in}}{\pgfqpoint{3.229476in}{2.406321in}}{\pgfqpoint{3.240526in}{2.406321in}}%
\pgfpathclose%
\pgfusepath{stroke,fill}%
\end{pgfscope}%
\begin{pgfscope}%
\pgfpathrectangle{\pgfqpoint{0.787074in}{0.548769in}}{\pgfqpoint{5.062926in}{3.102590in}}%
\pgfusepath{clip}%
\pgfsetbuttcap%
\pgfsetroundjoin%
\definecolor{currentfill}{rgb}{1.000000,0.498039,0.054902}%
\pgfsetfillcolor{currentfill}%
\pgfsetlinewidth{1.003750pt}%
\definecolor{currentstroke}{rgb}{1.000000,0.498039,0.054902}%
\pgfsetstrokecolor{currentstroke}%
\pgfsetdash{}{0pt}%
\pgfpathmoveto{\pgfqpoint{3.240526in}{2.705375in}}%
\pgfpathcurveto{\pgfqpoint{3.251576in}{2.705375in}}{\pgfqpoint{3.262175in}{2.709765in}}{\pgfqpoint{3.269989in}{2.717578in}}%
\pgfpathcurveto{\pgfqpoint{3.277802in}{2.725392in}}{\pgfqpoint{3.282192in}{2.735991in}}{\pgfqpoint{3.282192in}{2.747041in}}%
\pgfpathcurveto{\pgfqpoint{3.282192in}{2.758091in}}{\pgfqpoint{3.277802in}{2.768690in}}{\pgfqpoint{3.269989in}{2.776504in}}%
\pgfpathcurveto{\pgfqpoint{3.262175in}{2.784318in}}{\pgfqpoint{3.251576in}{2.788708in}}{\pgfqpoint{3.240526in}{2.788708in}}%
\pgfpathcurveto{\pgfqpoint{3.229476in}{2.788708in}}{\pgfqpoint{3.218877in}{2.784318in}}{\pgfqpoint{3.211063in}{2.776504in}}%
\pgfpathcurveto{\pgfqpoint{3.203249in}{2.768690in}}{\pgfqpoint{3.198859in}{2.758091in}}{\pgfqpoint{3.198859in}{2.747041in}}%
\pgfpathcurveto{\pgfqpoint{3.198859in}{2.735991in}}{\pgfqpoint{3.203249in}{2.725392in}}{\pgfqpoint{3.211063in}{2.717578in}}%
\pgfpathcurveto{\pgfqpoint{3.218877in}{2.709765in}}{\pgfqpoint{3.229476in}{2.705375in}}{\pgfqpoint{3.240526in}{2.705375in}}%
\pgfpathclose%
\pgfusepath{stroke,fill}%
\end{pgfscope}%
\begin{pgfscope}%
\pgfpathrectangle{\pgfqpoint{0.787074in}{0.548769in}}{\pgfqpoint{5.062926in}{3.102590in}}%
\pgfusepath{clip}%
\pgfsetbuttcap%
\pgfsetroundjoin%
\definecolor{currentfill}{rgb}{1.000000,0.498039,0.054902}%
\pgfsetfillcolor{currentfill}%
\pgfsetlinewidth{1.003750pt}%
\definecolor{currentstroke}{rgb}{1.000000,0.498039,0.054902}%
\pgfsetstrokecolor{currentstroke}%
\pgfsetdash{}{0pt}%
\pgfpathmoveto{\pgfqpoint{2.655442in}{2.307212in}}%
\pgfpathcurveto{\pgfqpoint{2.666492in}{2.307212in}}{\pgfqpoint{2.677091in}{2.311602in}}{\pgfqpoint{2.684905in}{2.319416in}}%
\pgfpathcurveto{\pgfqpoint{2.692718in}{2.327230in}}{\pgfqpoint{2.697108in}{2.337829in}}{\pgfqpoint{2.697108in}{2.348879in}}%
\pgfpathcurveto{\pgfqpoint{2.697108in}{2.359929in}}{\pgfqpoint{2.692718in}{2.370528in}}{\pgfqpoint{2.684905in}{2.378342in}}%
\pgfpathcurveto{\pgfqpoint{2.677091in}{2.386155in}}{\pgfqpoint{2.666492in}{2.390545in}}{\pgfqpoint{2.655442in}{2.390545in}}%
\pgfpathcurveto{\pgfqpoint{2.644392in}{2.390545in}}{\pgfqpoint{2.633793in}{2.386155in}}{\pgfqpoint{2.625979in}{2.378342in}}%
\pgfpathcurveto{\pgfqpoint{2.618165in}{2.370528in}}{\pgfqpoint{2.613775in}{2.359929in}}{\pgfqpoint{2.613775in}{2.348879in}}%
\pgfpathcurveto{\pgfqpoint{2.613775in}{2.337829in}}{\pgfqpoint{2.618165in}{2.327230in}}{\pgfqpoint{2.625979in}{2.319416in}}%
\pgfpathcurveto{\pgfqpoint{2.633793in}{2.311602in}}{\pgfqpoint{2.644392in}{2.307212in}}{\pgfqpoint{2.655442in}{2.307212in}}%
\pgfpathclose%
\pgfusepath{stroke,fill}%
\end{pgfscope}%
\begin{pgfscope}%
\pgfpathrectangle{\pgfqpoint{0.787074in}{0.548769in}}{\pgfqpoint{5.062926in}{3.102590in}}%
\pgfusepath{clip}%
\pgfsetbuttcap%
\pgfsetroundjoin%
\definecolor{currentfill}{rgb}{1.000000,0.498039,0.054902}%
\pgfsetfillcolor{currentfill}%
\pgfsetlinewidth{1.003750pt}%
\definecolor{currentstroke}{rgb}{1.000000,0.498039,0.054902}%
\pgfsetstrokecolor{currentstroke}%
\pgfsetdash{}{0pt}%
\pgfpathmoveto{\pgfqpoint{3.825610in}{3.001341in}}%
\pgfpathcurveto{\pgfqpoint{3.836660in}{3.001341in}}{\pgfqpoint{3.847259in}{3.005732in}}{\pgfqpoint{3.855072in}{3.013545in}}%
\pgfpathcurveto{\pgfqpoint{3.862886in}{3.021359in}}{\pgfqpoint{3.867276in}{3.031958in}}{\pgfqpoint{3.867276in}{3.043008in}}%
\pgfpathcurveto{\pgfqpoint{3.867276in}{3.054058in}}{\pgfqpoint{3.862886in}{3.064657in}}{\pgfqpoint{3.855072in}{3.072471in}}%
\pgfpathcurveto{\pgfqpoint{3.847259in}{3.080284in}}{\pgfqpoint{3.836660in}{3.084675in}}{\pgfqpoint{3.825610in}{3.084675in}}%
\pgfpathcurveto{\pgfqpoint{3.814560in}{3.084675in}}{\pgfqpoint{3.803960in}{3.080284in}}{\pgfqpoint{3.796147in}{3.072471in}}%
\pgfpathcurveto{\pgfqpoint{3.788333in}{3.064657in}}{\pgfqpoint{3.783943in}{3.054058in}}{\pgfqpoint{3.783943in}{3.043008in}}%
\pgfpathcurveto{\pgfqpoint{3.783943in}{3.031958in}}{\pgfqpoint{3.788333in}{3.021359in}}{\pgfqpoint{3.796147in}{3.013545in}}%
\pgfpathcurveto{\pgfqpoint{3.803960in}{3.005732in}}{\pgfqpoint{3.814560in}{3.001341in}}{\pgfqpoint{3.825610in}{3.001341in}}%
\pgfpathclose%
\pgfusepath{stroke,fill}%
\end{pgfscope}%
\begin{pgfscope}%
\pgfpathrectangle{\pgfqpoint{0.787074in}{0.548769in}}{\pgfqpoint{5.062926in}{3.102590in}}%
\pgfusepath{clip}%
\pgfsetbuttcap%
\pgfsetroundjoin%
\definecolor{currentfill}{rgb}{1.000000,0.498039,0.054902}%
\pgfsetfillcolor{currentfill}%
\pgfsetlinewidth{1.003750pt}%
\definecolor{currentstroke}{rgb}{1.000000,0.498039,0.054902}%
\pgfsetstrokecolor{currentstroke}%
\pgfsetdash{}{0pt}%
\pgfpathmoveto{\pgfqpoint{3.396548in}{2.401858in}}%
\pgfpathcurveto{\pgfqpoint{3.407598in}{2.401858in}}{\pgfqpoint{3.418197in}{2.406248in}}{\pgfqpoint{3.426011in}{2.414062in}}%
\pgfpathcurveto{\pgfqpoint{3.433825in}{2.421875in}}{\pgfqpoint{3.438215in}{2.432474in}}{\pgfqpoint{3.438215in}{2.443524in}}%
\pgfpathcurveto{\pgfqpoint{3.438215in}{2.454575in}}{\pgfqpoint{3.433825in}{2.465174in}}{\pgfqpoint{3.426011in}{2.472987in}}%
\pgfpathcurveto{\pgfqpoint{3.418197in}{2.480801in}}{\pgfqpoint{3.407598in}{2.485191in}}{\pgfqpoint{3.396548in}{2.485191in}}%
\pgfpathcurveto{\pgfqpoint{3.385498in}{2.485191in}}{\pgfqpoint{3.374899in}{2.480801in}}{\pgfqpoint{3.367085in}{2.472987in}}%
\pgfpathcurveto{\pgfqpoint{3.359272in}{2.465174in}}{\pgfqpoint{3.354881in}{2.454575in}}{\pgfqpoint{3.354881in}{2.443524in}}%
\pgfpathcurveto{\pgfqpoint{3.354881in}{2.432474in}}{\pgfqpoint{3.359272in}{2.421875in}}{\pgfqpoint{3.367085in}{2.414062in}}%
\pgfpathcurveto{\pgfqpoint{3.374899in}{2.406248in}}{\pgfqpoint{3.385498in}{2.401858in}}{\pgfqpoint{3.396548in}{2.401858in}}%
\pgfpathclose%
\pgfusepath{stroke,fill}%
\end{pgfscope}%
\begin{pgfscope}%
\pgfpathrectangle{\pgfqpoint{0.787074in}{0.548769in}}{\pgfqpoint{5.062926in}{3.102590in}}%
\pgfusepath{clip}%
\pgfsetbuttcap%
\pgfsetroundjoin%
\definecolor{currentfill}{rgb}{0.121569,0.466667,0.705882}%
\pgfsetfillcolor{currentfill}%
\pgfsetlinewidth{1.003750pt}%
\definecolor{currentstroke}{rgb}{0.121569,0.466667,0.705882}%
\pgfsetstrokecolor{currentstroke}%
\pgfsetdash{}{0pt}%
\pgfpathmoveto{\pgfqpoint{4.449699in}{2.416209in}}%
\pgfpathcurveto{\pgfqpoint{4.460749in}{2.416209in}}{\pgfqpoint{4.471348in}{2.420599in}}{\pgfqpoint{4.479162in}{2.428413in}}%
\pgfpathcurveto{\pgfqpoint{4.486976in}{2.436227in}}{\pgfqpoint{4.491366in}{2.446826in}}{\pgfqpoint{4.491366in}{2.457876in}}%
\pgfpathcurveto{\pgfqpoint{4.491366in}{2.468926in}}{\pgfqpoint{4.486976in}{2.479525in}}{\pgfqpoint{4.479162in}{2.487339in}}%
\pgfpathcurveto{\pgfqpoint{4.471348in}{2.495152in}}{\pgfqpoint{4.460749in}{2.499543in}}{\pgfqpoint{4.449699in}{2.499543in}}%
\pgfpathcurveto{\pgfqpoint{4.438649in}{2.499543in}}{\pgfqpoint{4.428050in}{2.495152in}}{\pgfqpoint{4.420236in}{2.487339in}}%
\pgfpathcurveto{\pgfqpoint{4.412423in}{2.479525in}}{\pgfqpoint{4.408032in}{2.468926in}}{\pgfqpoint{4.408032in}{2.457876in}}%
\pgfpathcurveto{\pgfqpoint{4.408032in}{2.446826in}}{\pgfqpoint{4.412423in}{2.436227in}}{\pgfqpoint{4.420236in}{2.428413in}}%
\pgfpathcurveto{\pgfqpoint{4.428050in}{2.420599in}}{\pgfqpoint{4.438649in}{2.416209in}}{\pgfqpoint{4.449699in}{2.416209in}}%
\pgfpathclose%
\pgfusepath{stroke,fill}%
\end{pgfscope}%
\begin{pgfscope}%
\pgfpathrectangle{\pgfqpoint{0.787074in}{0.548769in}}{\pgfqpoint{5.062926in}{3.102590in}}%
\pgfusepath{clip}%
\pgfsetbuttcap%
\pgfsetroundjoin%
\definecolor{currentfill}{rgb}{1.000000,0.498039,0.054902}%
\pgfsetfillcolor{currentfill}%
\pgfsetlinewidth{1.003750pt}%
\definecolor{currentstroke}{rgb}{1.000000,0.498039,0.054902}%
\pgfsetstrokecolor{currentstroke}%
\pgfsetdash{}{0pt}%
\pgfpathmoveto{\pgfqpoint{3.045498in}{3.029407in}}%
\pgfpathcurveto{\pgfqpoint{3.056548in}{3.029407in}}{\pgfqpoint{3.067147in}{3.033797in}}{\pgfqpoint{3.074961in}{3.041611in}}%
\pgfpathcurveto{\pgfqpoint{3.082774in}{3.049424in}}{\pgfqpoint{3.087164in}{3.060023in}}{\pgfqpoint{3.087164in}{3.071073in}}%
\pgfpathcurveto{\pgfqpoint{3.087164in}{3.082124in}}{\pgfqpoint{3.082774in}{3.092723in}}{\pgfqpoint{3.074961in}{3.100536in}}%
\pgfpathcurveto{\pgfqpoint{3.067147in}{3.108350in}}{\pgfqpoint{3.056548in}{3.112740in}}{\pgfqpoint{3.045498in}{3.112740in}}%
\pgfpathcurveto{\pgfqpoint{3.034448in}{3.112740in}}{\pgfqpoint{3.023849in}{3.108350in}}{\pgfqpoint{3.016035in}{3.100536in}}%
\pgfpathcurveto{\pgfqpoint{3.008221in}{3.092723in}}{\pgfqpoint{3.003831in}{3.082124in}}{\pgfqpoint{3.003831in}{3.071073in}}%
\pgfpathcurveto{\pgfqpoint{3.003831in}{3.060023in}}{\pgfqpoint{3.008221in}{3.049424in}}{\pgfqpoint{3.016035in}{3.041611in}}%
\pgfpathcurveto{\pgfqpoint{3.023849in}{3.033797in}}{\pgfqpoint{3.034448in}{3.029407in}}{\pgfqpoint{3.045498in}{3.029407in}}%
\pgfpathclose%
\pgfusepath{stroke,fill}%
\end{pgfscope}%
\begin{pgfscope}%
\pgfpathrectangle{\pgfqpoint{0.787074in}{0.548769in}}{\pgfqpoint{5.062926in}{3.102590in}}%
\pgfusepath{clip}%
\pgfsetbuttcap%
\pgfsetroundjoin%
\definecolor{currentfill}{rgb}{1.000000,0.498039,0.054902}%
\pgfsetfillcolor{currentfill}%
\pgfsetlinewidth{1.003750pt}%
\definecolor{currentstroke}{rgb}{1.000000,0.498039,0.054902}%
\pgfsetstrokecolor{currentstroke}%
\pgfsetdash{}{0pt}%
\pgfpathmoveto{\pgfqpoint{3.825610in}{2.943038in}}%
\pgfpathcurveto{\pgfqpoint{3.836660in}{2.943038in}}{\pgfqpoint{3.847259in}{2.947429in}}{\pgfqpoint{3.855072in}{2.955242in}}%
\pgfpathcurveto{\pgfqpoint{3.862886in}{2.963056in}}{\pgfqpoint{3.867276in}{2.973655in}}{\pgfqpoint{3.867276in}{2.984705in}}%
\pgfpathcurveto{\pgfqpoint{3.867276in}{2.995755in}}{\pgfqpoint{3.862886in}{3.006354in}}{\pgfqpoint{3.855072in}{3.014168in}}%
\pgfpathcurveto{\pgfqpoint{3.847259in}{3.021981in}}{\pgfqpoint{3.836660in}{3.026372in}}{\pgfqpoint{3.825610in}{3.026372in}}%
\pgfpathcurveto{\pgfqpoint{3.814560in}{3.026372in}}{\pgfqpoint{3.803960in}{3.021981in}}{\pgfqpoint{3.796147in}{3.014168in}}%
\pgfpathcurveto{\pgfqpoint{3.788333in}{3.006354in}}{\pgfqpoint{3.783943in}{2.995755in}}{\pgfqpoint{3.783943in}{2.984705in}}%
\pgfpathcurveto{\pgfqpoint{3.783943in}{2.973655in}}{\pgfqpoint{3.788333in}{2.963056in}}{\pgfqpoint{3.796147in}{2.955242in}}%
\pgfpathcurveto{\pgfqpoint{3.803960in}{2.947429in}}{\pgfqpoint{3.814560in}{2.943038in}}{\pgfqpoint{3.825610in}{2.943038in}}%
\pgfpathclose%
\pgfusepath{stroke,fill}%
\end{pgfscope}%
\begin{pgfscope}%
\pgfpathrectangle{\pgfqpoint{0.787074in}{0.548769in}}{\pgfqpoint{5.062926in}{3.102590in}}%
\pgfusepath{clip}%
\pgfsetbuttcap%
\pgfsetroundjoin%
\definecolor{currentfill}{rgb}{1.000000,0.498039,0.054902}%
\pgfsetfillcolor{currentfill}%
\pgfsetlinewidth{1.003750pt}%
\definecolor{currentstroke}{rgb}{1.000000,0.498039,0.054902}%
\pgfsetstrokecolor{currentstroke}%
\pgfsetdash{}{0pt}%
\pgfpathmoveto{\pgfqpoint{3.240526in}{2.587341in}}%
\pgfpathcurveto{\pgfqpoint{3.251576in}{2.587341in}}{\pgfqpoint{3.262175in}{2.591732in}}{\pgfqpoint{3.269989in}{2.599545in}}%
\pgfpathcurveto{\pgfqpoint{3.277802in}{2.607359in}}{\pgfqpoint{3.282192in}{2.617958in}}{\pgfqpoint{3.282192in}{2.629008in}}%
\pgfpathcurveto{\pgfqpoint{3.282192in}{2.640058in}}{\pgfqpoint{3.277802in}{2.650657in}}{\pgfqpoint{3.269989in}{2.658471in}}%
\pgfpathcurveto{\pgfqpoint{3.262175in}{2.666284in}}{\pgfqpoint{3.251576in}{2.670675in}}{\pgfqpoint{3.240526in}{2.670675in}}%
\pgfpathcurveto{\pgfqpoint{3.229476in}{2.670675in}}{\pgfqpoint{3.218877in}{2.666284in}}{\pgfqpoint{3.211063in}{2.658471in}}%
\pgfpathcurveto{\pgfqpoint{3.203249in}{2.650657in}}{\pgfqpoint{3.198859in}{2.640058in}}{\pgfqpoint{3.198859in}{2.629008in}}%
\pgfpathcurveto{\pgfqpoint{3.198859in}{2.617958in}}{\pgfqpoint{3.203249in}{2.607359in}}{\pgfqpoint{3.211063in}{2.599545in}}%
\pgfpathcurveto{\pgfqpoint{3.218877in}{2.591732in}}{\pgfqpoint{3.229476in}{2.587341in}}{\pgfqpoint{3.240526in}{2.587341in}}%
\pgfpathclose%
\pgfusepath{stroke,fill}%
\end{pgfscope}%
\begin{pgfscope}%
\pgfpathrectangle{\pgfqpoint{0.787074in}{0.548769in}}{\pgfqpoint{5.062926in}{3.102590in}}%
\pgfusepath{clip}%
\pgfsetbuttcap%
\pgfsetroundjoin%
\definecolor{currentfill}{rgb}{1.000000,0.498039,0.054902}%
\pgfsetfillcolor{currentfill}%
\pgfsetlinewidth{1.003750pt}%
\definecolor{currentstroke}{rgb}{1.000000,0.498039,0.054902}%
\pgfsetstrokecolor{currentstroke}%
\pgfsetdash{}{0pt}%
\pgfpathmoveto{\pgfqpoint{3.552570in}{2.817920in}}%
\pgfpathcurveto{\pgfqpoint{3.563621in}{2.817920in}}{\pgfqpoint{3.574220in}{2.822310in}}{\pgfqpoint{3.582033in}{2.830124in}}%
\pgfpathcurveto{\pgfqpoint{3.589847in}{2.837937in}}{\pgfqpoint{3.594237in}{2.848536in}}{\pgfqpoint{3.594237in}{2.859586in}}%
\pgfpathcurveto{\pgfqpoint{3.594237in}{2.870637in}}{\pgfqpoint{3.589847in}{2.881236in}}{\pgfqpoint{3.582033in}{2.889049in}}%
\pgfpathcurveto{\pgfqpoint{3.574220in}{2.896863in}}{\pgfqpoint{3.563621in}{2.901253in}}{\pgfqpoint{3.552570in}{2.901253in}}%
\pgfpathcurveto{\pgfqpoint{3.541520in}{2.901253in}}{\pgfqpoint{3.530921in}{2.896863in}}{\pgfqpoint{3.523108in}{2.889049in}}%
\pgfpathcurveto{\pgfqpoint{3.515294in}{2.881236in}}{\pgfqpoint{3.510904in}{2.870637in}}{\pgfqpoint{3.510904in}{2.859586in}}%
\pgfpathcurveto{\pgfqpoint{3.510904in}{2.848536in}}{\pgfqpoint{3.515294in}{2.837937in}}{\pgfqpoint{3.523108in}{2.830124in}}%
\pgfpathcurveto{\pgfqpoint{3.530921in}{2.822310in}}{\pgfqpoint{3.541520in}{2.817920in}}{\pgfqpoint{3.552570in}{2.817920in}}%
\pgfpathclose%
\pgfusepath{stroke,fill}%
\end{pgfscope}%
\begin{pgfscope}%
\pgfpathrectangle{\pgfqpoint{0.787074in}{0.548769in}}{\pgfqpoint{5.062926in}{3.102590in}}%
\pgfusepath{clip}%
\pgfsetbuttcap%
\pgfsetroundjoin%
\definecolor{currentfill}{rgb}{1.000000,0.498039,0.054902}%
\pgfsetfillcolor{currentfill}%
\pgfsetlinewidth{1.003750pt}%
\definecolor{currentstroke}{rgb}{1.000000,0.498039,0.054902}%
\pgfsetstrokecolor{currentstroke}%
\pgfsetdash{}{0pt}%
\pgfpathmoveto{\pgfqpoint{3.552570in}{2.246871in}}%
\pgfpathcurveto{\pgfqpoint{3.563621in}{2.246871in}}{\pgfqpoint{3.574220in}{2.251261in}}{\pgfqpoint{3.582033in}{2.259075in}}%
\pgfpathcurveto{\pgfqpoint{3.589847in}{2.266888in}}{\pgfqpoint{3.594237in}{2.277487in}}{\pgfqpoint{3.594237in}{2.288537in}}%
\pgfpathcurveto{\pgfqpoint{3.594237in}{2.299587in}}{\pgfqpoint{3.589847in}{2.310186in}}{\pgfqpoint{3.582033in}{2.318000in}}%
\pgfpathcurveto{\pgfqpoint{3.574220in}{2.325814in}}{\pgfqpoint{3.563621in}{2.330204in}}{\pgfqpoint{3.552570in}{2.330204in}}%
\pgfpathcurveto{\pgfqpoint{3.541520in}{2.330204in}}{\pgfqpoint{3.530921in}{2.325814in}}{\pgfqpoint{3.523108in}{2.318000in}}%
\pgfpathcurveto{\pgfqpoint{3.515294in}{2.310186in}}{\pgfqpoint{3.510904in}{2.299587in}}{\pgfqpoint{3.510904in}{2.288537in}}%
\pgfpathcurveto{\pgfqpoint{3.510904in}{2.277487in}}{\pgfqpoint{3.515294in}{2.266888in}}{\pgfqpoint{3.523108in}{2.259075in}}%
\pgfpathcurveto{\pgfqpoint{3.530921in}{2.251261in}}{\pgfqpoint{3.541520in}{2.246871in}}{\pgfqpoint{3.552570in}{2.246871in}}%
\pgfpathclose%
\pgfusepath{stroke,fill}%
\end{pgfscope}%
\begin{pgfscope}%
\pgfpathrectangle{\pgfqpoint{0.787074in}{0.548769in}}{\pgfqpoint{5.062926in}{3.102590in}}%
\pgfusepath{clip}%
\pgfsetbuttcap%
\pgfsetroundjoin%
\definecolor{currentfill}{rgb}{1.000000,0.498039,0.054902}%
\pgfsetfillcolor{currentfill}%
\pgfsetlinewidth{1.003750pt}%
\definecolor{currentstroke}{rgb}{1.000000,0.498039,0.054902}%
\pgfsetstrokecolor{currentstroke}%
\pgfsetdash{}{0pt}%
\pgfpathmoveto{\pgfqpoint{3.006492in}{1.776148in}}%
\pgfpathcurveto{\pgfqpoint{3.017542in}{1.776148in}}{\pgfqpoint{3.028141in}{1.780538in}}{\pgfqpoint{3.035955in}{1.788352in}}%
\pgfpathcurveto{\pgfqpoint{3.043769in}{1.796166in}}{\pgfqpoint{3.048159in}{1.806765in}}{\pgfqpoint{3.048159in}{1.817815in}}%
\pgfpathcurveto{\pgfqpoint{3.048159in}{1.828865in}}{\pgfqpoint{3.043769in}{1.839464in}}{\pgfqpoint{3.035955in}{1.847278in}}%
\pgfpathcurveto{\pgfqpoint{3.028141in}{1.855091in}}{\pgfqpoint{3.017542in}{1.859481in}}{\pgfqpoint{3.006492in}{1.859481in}}%
\pgfpathcurveto{\pgfqpoint{2.995442in}{1.859481in}}{\pgfqpoint{2.984843in}{1.855091in}}{\pgfqpoint{2.977029in}{1.847278in}}%
\pgfpathcurveto{\pgfqpoint{2.969216in}{1.839464in}}{\pgfqpoint{2.964825in}{1.828865in}}{\pgfqpoint{2.964825in}{1.817815in}}%
\pgfpathcurveto{\pgfqpoint{2.964825in}{1.806765in}}{\pgfqpoint{2.969216in}{1.796166in}}{\pgfqpoint{2.977029in}{1.788352in}}%
\pgfpathcurveto{\pgfqpoint{2.984843in}{1.780538in}}{\pgfqpoint{2.995442in}{1.776148in}}{\pgfqpoint{3.006492in}{1.776148in}}%
\pgfpathclose%
\pgfusepath{stroke,fill}%
\end{pgfscope}%
\begin{pgfscope}%
\pgfpathrectangle{\pgfqpoint{0.787074in}{0.548769in}}{\pgfqpoint{5.062926in}{3.102590in}}%
\pgfusepath{clip}%
\pgfsetbuttcap%
\pgfsetroundjoin%
\definecolor{currentfill}{rgb}{1.000000,0.498039,0.054902}%
\pgfsetfillcolor{currentfill}%
\pgfsetlinewidth{1.003750pt}%
\definecolor{currentstroke}{rgb}{1.000000,0.498039,0.054902}%
\pgfsetstrokecolor{currentstroke}%
\pgfsetdash{}{0pt}%
\pgfpathmoveto{\pgfqpoint{3.006492in}{2.992883in}}%
\pgfpathcurveto{\pgfqpoint{3.017542in}{2.992883in}}{\pgfqpoint{3.028141in}{2.997273in}}{\pgfqpoint{3.035955in}{3.005087in}}%
\pgfpathcurveto{\pgfqpoint{3.043769in}{3.012900in}}{\pgfqpoint{3.048159in}{3.023499in}}{\pgfqpoint{3.048159in}{3.034549in}}%
\pgfpathcurveto{\pgfqpoint{3.048159in}{3.045599in}}{\pgfqpoint{3.043769in}{3.056199in}}{\pgfqpoint{3.035955in}{3.064012in}}%
\pgfpathcurveto{\pgfqpoint{3.028141in}{3.071826in}}{\pgfqpoint{3.017542in}{3.076216in}}{\pgfqpoint{3.006492in}{3.076216in}}%
\pgfpathcurveto{\pgfqpoint{2.995442in}{3.076216in}}{\pgfqpoint{2.984843in}{3.071826in}}{\pgfqpoint{2.977029in}{3.064012in}}%
\pgfpathcurveto{\pgfqpoint{2.969216in}{3.056199in}}{\pgfqpoint{2.964825in}{3.045599in}}{\pgfqpoint{2.964825in}{3.034549in}}%
\pgfpathcurveto{\pgfqpoint{2.964825in}{3.023499in}}{\pgfqpoint{2.969216in}{3.012900in}}{\pgfqpoint{2.977029in}{3.005087in}}%
\pgfpathcurveto{\pgfqpoint{2.984843in}{2.997273in}}{\pgfqpoint{2.995442in}{2.992883in}}{\pgfqpoint{3.006492in}{2.992883in}}%
\pgfpathclose%
\pgfusepath{stroke,fill}%
\end{pgfscope}%
\begin{pgfscope}%
\pgfpathrectangle{\pgfqpoint{0.787074in}{0.548769in}}{\pgfqpoint{5.062926in}{3.102590in}}%
\pgfusepath{clip}%
\pgfsetbuttcap%
\pgfsetroundjoin%
\definecolor{currentfill}{rgb}{1.000000,0.498039,0.054902}%
\pgfsetfillcolor{currentfill}%
\pgfsetlinewidth{1.003750pt}%
\definecolor{currentstroke}{rgb}{1.000000,0.498039,0.054902}%
\pgfsetstrokecolor{currentstroke}%
\pgfsetdash{}{0pt}%
\pgfpathmoveto{\pgfqpoint{3.864615in}{1.537669in}}%
\pgfpathcurveto{\pgfqpoint{3.875665in}{1.537669in}}{\pgfqpoint{3.886264in}{1.542060in}}{\pgfqpoint{3.894078in}{1.549873in}}%
\pgfpathcurveto{\pgfqpoint{3.901892in}{1.557687in}}{\pgfqpoint{3.906282in}{1.568286in}}{\pgfqpoint{3.906282in}{1.579336in}}%
\pgfpathcurveto{\pgfqpoint{3.906282in}{1.590386in}}{\pgfqpoint{3.901892in}{1.600985in}}{\pgfqpoint{3.894078in}{1.608799in}}%
\pgfpathcurveto{\pgfqpoint{3.886264in}{1.616612in}}{\pgfqpoint{3.875665in}{1.621003in}}{\pgfqpoint{3.864615in}{1.621003in}}%
\pgfpathcurveto{\pgfqpoint{3.853565in}{1.621003in}}{\pgfqpoint{3.842966in}{1.616612in}}{\pgfqpoint{3.835152in}{1.608799in}}%
\pgfpathcurveto{\pgfqpoint{3.827339in}{1.600985in}}{\pgfqpoint{3.822949in}{1.590386in}}{\pgfqpoint{3.822949in}{1.579336in}}%
\pgfpathcurveto{\pgfqpoint{3.822949in}{1.568286in}}{\pgfqpoint{3.827339in}{1.557687in}}{\pgfqpoint{3.835152in}{1.549873in}}%
\pgfpathcurveto{\pgfqpoint{3.842966in}{1.542060in}}{\pgfqpoint{3.853565in}{1.537669in}}{\pgfqpoint{3.864615in}{1.537669in}}%
\pgfpathclose%
\pgfusepath{stroke,fill}%
\end{pgfscope}%
\begin{pgfscope}%
\pgfpathrectangle{\pgfqpoint{0.787074in}{0.548769in}}{\pgfqpoint{5.062926in}{3.102590in}}%
\pgfusepath{clip}%
\pgfsetbuttcap%
\pgfsetroundjoin%
\definecolor{currentfill}{rgb}{0.121569,0.466667,0.705882}%
\pgfsetfillcolor{currentfill}%
\pgfsetlinewidth{1.003750pt}%
\definecolor{currentstroke}{rgb}{0.121569,0.466667,0.705882}%
\pgfsetstrokecolor{currentstroke}%
\pgfsetdash{}{0pt}%
\pgfpathmoveto{\pgfqpoint{3.981632in}{0.681479in}}%
\pgfpathcurveto{\pgfqpoint{3.992682in}{0.681479in}}{\pgfqpoint{4.003281in}{0.685869in}}{\pgfqpoint{4.011095in}{0.693683in}}%
\pgfpathcurveto{\pgfqpoint{4.018908in}{0.701497in}}{\pgfqpoint{4.023299in}{0.712096in}}{\pgfqpoint{4.023299in}{0.723146in}}%
\pgfpathcurveto{\pgfqpoint{4.023299in}{0.734196in}}{\pgfqpoint{4.018908in}{0.744795in}}{\pgfqpoint{4.011095in}{0.752609in}}%
\pgfpathcurveto{\pgfqpoint{4.003281in}{0.760422in}}{\pgfqpoint{3.992682in}{0.764812in}}{\pgfqpoint{3.981632in}{0.764812in}}%
\pgfpathcurveto{\pgfqpoint{3.970582in}{0.764812in}}{\pgfqpoint{3.959983in}{0.760422in}}{\pgfqpoint{3.952169in}{0.752609in}}%
\pgfpathcurveto{\pgfqpoint{3.944356in}{0.744795in}}{\pgfqpoint{3.939965in}{0.734196in}}{\pgfqpoint{3.939965in}{0.723146in}}%
\pgfpathcurveto{\pgfqpoint{3.939965in}{0.712096in}}{\pgfqpoint{3.944356in}{0.701497in}}{\pgfqpoint{3.952169in}{0.693683in}}%
\pgfpathcurveto{\pgfqpoint{3.959983in}{0.685869in}}{\pgfqpoint{3.970582in}{0.681479in}}{\pgfqpoint{3.981632in}{0.681479in}}%
\pgfpathclose%
\pgfusepath{stroke,fill}%
\end{pgfscope}%
\begin{pgfscope}%
\pgfpathrectangle{\pgfqpoint{0.787074in}{0.548769in}}{\pgfqpoint{5.062926in}{3.102590in}}%
\pgfusepath{clip}%
\pgfsetbuttcap%
\pgfsetroundjoin%
\definecolor{currentfill}{rgb}{1.000000,0.498039,0.054902}%
\pgfsetfillcolor{currentfill}%
\pgfsetlinewidth{1.003750pt}%
\definecolor{currentstroke}{rgb}{1.000000,0.498039,0.054902}%
\pgfsetstrokecolor{currentstroke}%
\pgfsetdash{}{0pt}%
\pgfpathmoveto{\pgfqpoint{2.226380in}{2.885059in}}%
\pgfpathcurveto{\pgfqpoint{2.237430in}{2.885059in}}{\pgfqpoint{2.248029in}{2.889449in}}{\pgfqpoint{2.255843in}{2.897263in}}%
\pgfpathcurveto{\pgfqpoint{2.263657in}{2.905076in}}{\pgfqpoint{2.268047in}{2.915675in}}{\pgfqpoint{2.268047in}{2.926726in}}%
\pgfpathcurveto{\pgfqpoint{2.268047in}{2.937776in}}{\pgfqpoint{2.263657in}{2.948375in}}{\pgfqpoint{2.255843in}{2.956188in}}%
\pgfpathcurveto{\pgfqpoint{2.248029in}{2.964002in}}{\pgfqpoint{2.237430in}{2.968392in}}{\pgfqpoint{2.226380in}{2.968392in}}%
\pgfpathcurveto{\pgfqpoint{2.215330in}{2.968392in}}{\pgfqpoint{2.204731in}{2.964002in}}{\pgfqpoint{2.196917in}{2.956188in}}%
\pgfpathcurveto{\pgfqpoint{2.189104in}{2.948375in}}{\pgfqpoint{2.184714in}{2.937776in}}{\pgfqpoint{2.184714in}{2.926726in}}%
\pgfpathcurveto{\pgfqpoint{2.184714in}{2.915675in}}{\pgfqpoint{2.189104in}{2.905076in}}{\pgfqpoint{2.196917in}{2.897263in}}%
\pgfpathcurveto{\pgfqpoint{2.204731in}{2.889449in}}{\pgfqpoint{2.215330in}{2.885059in}}{\pgfqpoint{2.226380in}{2.885059in}}%
\pgfpathclose%
\pgfusepath{stroke,fill}%
\end{pgfscope}%
\begin{pgfscope}%
\pgfpathrectangle{\pgfqpoint{0.787074in}{0.548769in}}{\pgfqpoint{5.062926in}{3.102590in}}%
\pgfusepath{clip}%
\pgfsetbuttcap%
\pgfsetroundjoin%
\definecolor{currentfill}{rgb}{1.000000,0.498039,0.054902}%
\pgfsetfillcolor{currentfill}%
\pgfsetlinewidth{1.003750pt}%
\definecolor{currentstroke}{rgb}{1.000000,0.498039,0.054902}%
\pgfsetstrokecolor{currentstroke}%
\pgfsetdash{}{0pt}%
\pgfpathmoveto{\pgfqpoint{3.240526in}{2.017445in}}%
\pgfpathcurveto{\pgfqpoint{3.251576in}{2.017445in}}{\pgfqpoint{3.262175in}{2.021836in}}{\pgfqpoint{3.269989in}{2.029649in}}%
\pgfpathcurveto{\pgfqpoint{3.277802in}{2.037463in}}{\pgfqpoint{3.282192in}{2.048062in}}{\pgfqpoint{3.282192in}{2.059112in}}%
\pgfpathcurveto{\pgfqpoint{3.282192in}{2.070162in}}{\pgfqpoint{3.277802in}{2.080761in}}{\pgfqpoint{3.269989in}{2.088575in}}%
\pgfpathcurveto{\pgfqpoint{3.262175in}{2.096388in}}{\pgfqpoint{3.251576in}{2.100779in}}{\pgfqpoint{3.240526in}{2.100779in}}%
\pgfpathcurveto{\pgfqpoint{3.229476in}{2.100779in}}{\pgfqpoint{3.218877in}{2.096388in}}{\pgfqpoint{3.211063in}{2.088575in}}%
\pgfpathcurveto{\pgfqpoint{3.203249in}{2.080761in}}{\pgfqpoint{3.198859in}{2.070162in}}{\pgfqpoint{3.198859in}{2.059112in}}%
\pgfpathcurveto{\pgfqpoint{3.198859in}{2.048062in}}{\pgfqpoint{3.203249in}{2.037463in}}{\pgfqpoint{3.211063in}{2.029649in}}%
\pgfpathcurveto{\pgfqpoint{3.218877in}{2.021836in}}{\pgfqpoint{3.229476in}{2.017445in}}{\pgfqpoint{3.240526in}{2.017445in}}%
\pgfpathclose%
\pgfusepath{stroke,fill}%
\end{pgfscope}%
\begin{pgfscope}%
\pgfpathrectangle{\pgfqpoint{0.787074in}{0.548769in}}{\pgfqpoint{5.062926in}{3.102590in}}%
\pgfusepath{clip}%
\pgfsetbuttcap%
\pgfsetroundjoin%
\definecolor{currentfill}{rgb}{0.121569,0.466667,0.705882}%
\pgfsetfillcolor{currentfill}%
\pgfsetlinewidth{1.003750pt}%
\definecolor{currentstroke}{rgb}{0.121569,0.466667,0.705882}%
\pgfsetstrokecolor{currentstroke}%
\pgfsetdash{}{0pt}%
\pgfpathmoveto{\pgfqpoint{3.435554in}{2.239976in}}%
\pgfpathcurveto{\pgfqpoint{3.446604in}{2.239976in}}{\pgfqpoint{3.457203in}{2.244367in}}{\pgfqpoint{3.465016in}{2.252180in}}%
\pgfpathcurveto{\pgfqpoint{3.472830in}{2.259994in}}{\pgfqpoint{3.477220in}{2.270593in}}{\pgfqpoint{3.477220in}{2.281643in}}%
\pgfpathcurveto{\pgfqpoint{3.477220in}{2.292693in}}{\pgfqpoint{3.472830in}{2.303292in}}{\pgfqpoint{3.465016in}{2.311106in}}%
\pgfpathcurveto{\pgfqpoint{3.457203in}{2.318919in}}{\pgfqpoint{3.446604in}{2.323310in}}{\pgfqpoint{3.435554in}{2.323310in}}%
\pgfpathcurveto{\pgfqpoint{3.424504in}{2.323310in}}{\pgfqpoint{3.413905in}{2.318919in}}{\pgfqpoint{3.406091in}{2.311106in}}%
\pgfpathcurveto{\pgfqpoint{3.398277in}{2.303292in}}{\pgfqpoint{3.393887in}{2.292693in}}{\pgfqpoint{3.393887in}{2.281643in}}%
\pgfpathcurveto{\pgfqpoint{3.393887in}{2.270593in}}{\pgfqpoint{3.398277in}{2.259994in}}{\pgfqpoint{3.406091in}{2.252180in}}%
\pgfpathcurveto{\pgfqpoint{3.413905in}{2.244367in}}{\pgfqpoint{3.424504in}{2.239976in}}{\pgfqpoint{3.435554in}{2.239976in}}%
\pgfpathclose%
\pgfusepath{stroke,fill}%
\end{pgfscope}%
\begin{pgfscope}%
\pgfpathrectangle{\pgfqpoint{0.787074in}{0.548769in}}{\pgfqpoint{5.062926in}{3.102590in}}%
\pgfusepath{clip}%
\pgfsetbuttcap%
\pgfsetroundjoin%
\definecolor{currentfill}{rgb}{0.121569,0.466667,0.705882}%
\pgfsetfillcolor{currentfill}%
\pgfsetlinewidth{1.003750pt}%
\definecolor{currentstroke}{rgb}{0.121569,0.466667,0.705882}%
\pgfsetstrokecolor{currentstroke}%
\pgfsetdash{}{0pt}%
\pgfpathmoveto{\pgfqpoint{3.474559in}{2.686510in}}%
\pgfpathcurveto{\pgfqpoint{3.485609in}{2.686510in}}{\pgfqpoint{3.496208in}{2.690900in}}{\pgfqpoint{3.504022in}{2.698714in}}%
\pgfpathcurveto{\pgfqpoint{3.511836in}{2.706527in}}{\pgfqpoint{3.516226in}{2.717126in}}{\pgfqpoint{3.516226in}{2.728177in}}%
\pgfpathcurveto{\pgfqpoint{3.516226in}{2.739227in}}{\pgfqpoint{3.511836in}{2.749826in}}{\pgfqpoint{3.504022in}{2.757639in}}%
\pgfpathcurveto{\pgfqpoint{3.496208in}{2.765453in}}{\pgfqpoint{3.485609in}{2.769843in}}{\pgfqpoint{3.474559in}{2.769843in}}%
\pgfpathcurveto{\pgfqpoint{3.463509in}{2.769843in}}{\pgfqpoint{3.452910in}{2.765453in}}{\pgfqpoint{3.445097in}{2.757639in}}%
\pgfpathcurveto{\pgfqpoint{3.437283in}{2.749826in}}{\pgfqpoint{3.432893in}{2.739227in}}{\pgfqpoint{3.432893in}{2.728177in}}%
\pgfpathcurveto{\pgfqpoint{3.432893in}{2.717126in}}{\pgfqpoint{3.437283in}{2.706527in}}{\pgfqpoint{3.445097in}{2.698714in}}%
\pgfpathcurveto{\pgfqpoint{3.452910in}{2.690900in}}{\pgfqpoint{3.463509in}{2.686510in}}{\pgfqpoint{3.474559in}{2.686510in}}%
\pgfpathclose%
\pgfusepath{stroke,fill}%
\end{pgfscope}%
\begin{pgfscope}%
\pgfpathrectangle{\pgfqpoint{0.787074in}{0.548769in}}{\pgfqpoint{5.062926in}{3.102590in}}%
\pgfusepath{clip}%
\pgfsetbuttcap%
\pgfsetroundjoin%
\definecolor{currentfill}{rgb}{1.000000,0.498039,0.054902}%
\pgfsetfillcolor{currentfill}%
\pgfsetlinewidth{1.003750pt}%
\definecolor{currentstroke}{rgb}{1.000000,0.498039,0.054902}%
\pgfsetstrokecolor{currentstroke}%
\pgfsetdash{}{0pt}%
\pgfpathmoveto{\pgfqpoint{3.825610in}{2.173816in}}%
\pgfpathcurveto{\pgfqpoint{3.836660in}{2.173816in}}{\pgfqpoint{3.847259in}{2.178206in}}{\pgfqpoint{3.855072in}{2.186020in}}%
\pgfpathcurveto{\pgfqpoint{3.862886in}{2.193834in}}{\pgfqpoint{3.867276in}{2.204433in}}{\pgfqpoint{3.867276in}{2.215483in}}%
\pgfpathcurveto{\pgfqpoint{3.867276in}{2.226533in}}{\pgfqpoint{3.862886in}{2.237132in}}{\pgfqpoint{3.855072in}{2.244946in}}%
\pgfpathcurveto{\pgfqpoint{3.847259in}{2.252759in}}{\pgfqpoint{3.836660in}{2.257150in}}{\pgfqpoint{3.825610in}{2.257150in}}%
\pgfpathcurveto{\pgfqpoint{3.814560in}{2.257150in}}{\pgfqpoint{3.803960in}{2.252759in}}{\pgfqpoint{3.796147in}{2.244946in}}%
\pgfpathcurveto{\pgfqpoint{3.788333in}{2.237132in}}{\pgfqpoint{3.783943in}{2.226533in}}{\pgfqpoint{3.783943in}{2.215483in}}%
\pgfpathcurveto{\pgfqpoint{3.783943in}{2.204433in}}{\pgfqpoint{3.788333in}{2.193834in}}{\pgfqpoint{3.796147in}{2.186020in}}%
\pgfpathcurveto{\pgfqpoint{3.803960in}{2.178206in}}{\pgfqpoint{3.814560in}{2.173816in}}{\pgfqpoint{3.825610in}{2.173816in}}%
\pgfpathclose%
\pgfusepath{stroke,fill}%
\end{pgfscope}%
\begin{pgfscope}%
\pgfpathrectangle{\pgfqpoint{0.787074in}{0.548769in}}{\pgfqpoint{5.062926in}{3.102590in}}%
\pgfusepath{clip}%
\pgfsetbuttcap%
\pgfsetroundjoin%
\definecolor{currentfill}{rgb}{0.121569,0.466667,0.705882}%
\pgfsetfillcolor{currentfill}%
\pgfsetlinewidth{1.003750pt}%
\definecolor{currentstroke}{rgb}{0.121569,0.466667,0.705882}%
\pgfsetstrokecolor{currentstroke}%
\pgfsetdash{}{0pt}%
\pgfpathmoveto{\pgfqpoint{3.630582in}{1.676903in}}%
\pgfpathcurveto{\pgfqpoint{3.641632in}{1.676903in}}{\pgfqpoint{3.652231in}{1.681293in}}{\pgfqpoint{3.660044in}{1.689107in}}%
\pgfpathcurveto{\pgfqpoint{3.667858in}{1.696921in}}{\pgfqpoint{3.672248in}{1.707520in}}{\pgfqpoint{3.672248in}{1.718570in}}%
\pgfpathcurveto{\pgfqpoint{3.672248in}{1.729620in}}{\pgfqpoint{3.667858in}{1.740219in}}{\pgfqpoint{3.660044in}{1.748032in}}%
\pgfpathcurveto{\pgfqpoint{3.652231in}{1.755846in}}{\pgfqpoint{3.641632in}{1.760236in}}{\pgfqpoint{3.630582in}{1.760236in}}%
\pgfpathcurveto{\pgfqpoint{3.619532in}{1.760236in}}{\pgfqpoint{3.608933in}{1.755846in}}{\pgfqpoint{3.601119in}{1.748032in}}%
\pgfpathcurveto{\pgfqpoint{3.593305in}{1.740219in}}{\pgfqpoint{3.588915in}{1.729620in}}{\pgfqpoint{3.588915in}{1.718570in}}%
\pgfpathcurveto{\pgfqpoint{3.588915in}{1.707520in}}{\pgfqpoint{3.593305in}{1.696921in}}{\pgfqpoint{3.601119in}{1.689107in}}%
\pgfpathcurveto{\pgfqpoint{3.608933in}{1.681293in}}{\pgfqpoint{3.619532in}{1.676903in}}{\pgfqpoint{3.630582in}{1.676903in}}%
\pgfpathclose%
\pgfusepath{stroke,fill}%
\end{pgfscope}%
\begin{pgfscope}%
\pgfpathrectangle{\pgfqpoint{0.787074in}{0.548769in}}{\pgfqpoint{5.062926in}{3.102590in}}%
\pgfusepath{clip}%
\pgfsetbuttcap%
\pgfsetroundjoin%
\definecolor{currentfill}{rgb}{0.121569,0.466667,0.705882}%
\pgfsetfillcolor{currentfill}%
\pgfsetlinewidth{1.003750pt}%
\definecolor{currentstroke}{rgb}{0.121569,0.466667,0.705882}%
\pgfsetstrokecolor{currentstroke}%
\pgfsetdash{}{0pt}%
\pgfpathmoveto{\pgfqpoint{3.513565in}{1.092741in}}%
\pgfpathcurveto{\pgfqpoint{3.524615in}{1.092741in}}{\pgfqpoint{3.535214in}{1.097131in}}{\pgfqpoint{3.543028in}{1.104945in}}%
\pgfpathcurveto{\pgfqpoint{3.550841in}{1.112758in}}{\pgfqpoint{3.555232in}{1.123357in}}{\pgfqpoint{3.555232in}{1.134408in}}%
\pgfpathcurveto{\pgfqpoint{3.555232in}{1.145458in}}{\pgfqpoint{3.550841in}{1.156057in}}{\pgfqpoint{3.543028in}{1.163870in}}%
\pgfpathcurveto{\pgfqpoint{3.535214in}{1.171684in}}{\pgfqpoint{3.524615in}{1.176074in}}{\pgfqpoint{3.513565in}{1.176074in}}%
\pgfpathcurveto{\pgfqpoint{3.502515in}{1.176074in}}{\pgfqpoint{3.491916in}{1.171684in}}{\pgfqpoint{3.484102in}{1.163870in}}%
\pgfpathcurveto{\pgfqpoint{3.476288in}{1.156057in}}{\pgfqpoint{3.471898in}{1.145458in}}{\pgfqpoint{3.471898in}{1.134408in}}%
\pgfpathcurveto{\pgfqpoint{3.471898in}{1.123357in}}{\pgfqpoint{3.476288in}{1.112758in}}{\pgfqpoint{3.484102in}{1.104945in}}%
\pgfpathcurveto{\pgfqpoint{3.491916in}{1.097131in}}{\pgfqpoint{3.502515in}{1.092741in}}{\pgfqpoint{3.513565in}{1.092741in}}%
\pgfpathclose%
\pgfusepath{stroke,fill}%
\end{pgfscope}%
\begin{pgfscope}%
\pgfpathrectangle{\pgfqpoint{0.787074in}{0.548769in}}{\pgfqpoint{5.062926in}{3.102590in}}%
\pgfusepath{clip}%
\pgfsetbuttcap%
\pgfsetroundjoin%
\definecolor{currentfill}{rgb}{1.000000,0.498039,0.054902}%
\pgfsetfillcolor{currentfill}%
\pgfsetlinewidth{1.003750pt}%
\definecolor{currentstroke}{rgb}{1.000000,0.498039,0.054902}%
\pgfsetstrokecolor{currentstroke}%
\pgfsetdash{}{0pt}%
\pgfpathmoveto{\pgfqpoint{3.318537in}{2.255576in}}%
\pgfpathcurveto{\pgfqpoint{3.329587in}{2.255576in}}{\pgfqpoint{3.340186in}{2.259966in}}{\pgfqpoint{3.348000in}{2.267780in}}%
\pgfpathcurveto{\pgfqpoint{3.355813in}{2.275593in}}{\pgfqpoint{3.360204in}{2.286192in}}{\pgfqpoint{3.360204in}{2.297243in}}%
\pgfpathcurveto{\pgfqpoint{3.360204in}{2.308293in}}{\pgfqpoint{3.355813in}{2.318892in}}{\pgfqpoint{3.348000in}{2.326705in}}%
\pgfpathcurveto{\pgfqpoint{3.340186in}{2.334519in}}{\pgfqpoint{3.329587in}{2.338909in}}{\pgfqpoint{3.318537in}{2.338909in}}%
\pgfpathcurveto{\pgfqpoint{3.307487in}{2.338909in}}{\pgfqpoint{3.296888in}{2.334519in}}{\pgfqpoint{3.289074in}{2.326705in}}%
\pgfpathcurveto{\pgfqpoint{3.281261in}{2.318892in}}{\pgfqpoint{3.276870in}{2.308293in}}{\pgfqpoint{3.276870in}{2.297243in}}%
\pgfpathcurveto{\pgfqpoint{3.276870in}{2.286192in}}{\pgfqpoint{3.281261in}{2.275593in}}{\pgfqpoint{3.289074in}{2.267780in}}%
\pgfpathcurveto{\pgfqpoint{3.296888in}{2.259966in}}{\pgfqpoint{3.307487in}{2.255576in}}{\pgfqpoint{3.318537in}{2.255576in}}%
\pgfpathclose%
\pgfusepath{stroke,fill}%
\end{pgfscope}%
\begin{pgfscope}%
\pgfsetbuttcap%
\pgfsetroundjoin%
\definecolor{currentfill}{rgb}{0.000000,0.000000,0.000000}%
\pgfsetfillcolor{currentfill}%
\pgfsetlinewidth{0.803000pt}%
\definecolor{currentstroke}{rgb}{0.000000,0.000000,0.000000}%
\pgfsetstrokecolor{currentstroke}%
\pgfsetdash{}{0pt}%
\pgfsys@defobject{currentmarker}{\pgfqpoint{0.000000in}{-0.048611in}}{\pgfqpoint{0.000000in}{0.000000in}}{%
\pgfpathmoveto{\pgfqpoint{0.000000in}{0.000000in}}%
\pgfpathlineto{\pgfqpoint{0.000000in}{-0.048611in}}%
\pgfusepath{stroke,fill}%
}%
\begin{pgfscope}%
\pgfsys@transformshift{1.212235in}{0.548769in}%
\pgfsys@useobject{currentmarker}{}%
\end{pgfscope}%
\end{pgfscope}%
\begin{pgfscope}%
\definecolor{textcolor}{rgb}{0.000000,0.000000,0.000000}%
\pgfsetstrokecolor{textcolor}%
\pgfsetfillcolor{textcolor}%
\pgftext[x=1.212235in,y=0.451547in,,top]{\color{textcolor}\sffamily\fontsize{10.000000}{12.000000}\selectfont \(\displaystyle {-60}\)}%
\end{pgfscope}%
\begin{pgfscope}%
\pgfsetbuttcap%
\pgfsetroundjoin%
\definecolor{currentfill}{rgb}{0.000000,0.000000,0.000000}%
\pgfsetfillcolor{currentfill}%
\pgfsetlinewidth{0.803000pt}%
\definecolor{currentstroke}{rgb}{0.000000,0.000000,0.000000}%
\pgfsetstrokecolor{currentstroke}%
\pgfsetdash{}{0pt}%
\pgfsys@defobject{currentmarker}{\pgfqpoint{0.000000in}{-0.048611in}}{\pgfqpoint{0.000000in}{0.000000in}}{%
\pgfpathmoveto{\pgfqpoint{0.000000in}{0.000000in}}%
\pgfpathlineto{\pgfqpoint{0.000000in}{-0.048611in}}%
\pgfusepath{stroke,fill}%
}%
\begin{pgfscope}%
\pgfsys@transformshift{1.992347in}{0.548769in}%
\pgfsys@useobject{currentmarker}{}%
\end{pgfscope}%
\end{pgfscope}%
\begin{pgfscope}%
\definecolor{textcolor}{rgb}{0.000000,0.000000,0.000000}%
\pgfsetstrokecolor{textcolor}%
\pgfsetfillcolor{textcolor}%
\pgftext[x=1.992347in,y=0.451547in,,top]{\color{textcolor}\sffamily\fontsize{10.000000}{12.000000}\selectfont \(\displaystyle {-40}\)}%
\end{pgfscope}%
\begin{pgfscope}%
\pgfsetbuttcap%
\pgfsetroundjoin%
\definecolor{currentfill}{rgb}{0.000000,0.000000,0.000000}%
\pgfsetfillcolor{currentfill}%
\pgfsetlinewidth{0.803000pt}%
\definecolor{currentstroke}{rgb}{0.000000,0.000000,0.000000}%
\pgfsetstrokecolor{currentstroke}%
\pgfsetdash{}{0pt}%
\pgfsys@defobject{currentmarker}{\pgfqpoint{0.000000in}{-0.048611in}}{\pgfqpoint{0.000000in}{0.000000in}}{%
\pgfpathmoveto{\pgfqpoint{0.000000in}{0.000000in}}%
\pgfpathlineto{\pgfqpoint{0.000000in}{-0.048611in}}%
\pgfusepath{stroke,fill}%
}%
\begin{pgfscope}%
\pgfsys@transformshift{2.772459in}{0.548769in}%
\pgfsys@useobject{currentmarker}{}%
\end{pgfscope}%
\end{pgfscope}%
\begin{pgfscope}%
\definecolor{textcolor}{rgb}{0.000000,0.000000,0.000000}%
\pgfsetstrokecolor{textcolor}%
\pgfsetfillcolor{textcolor}%
\pgftext[x=2.772459in,y=0.451547in,,top]{\color{textcolor}\sffamily\fontsize{10.000000}{12.000000}\selectfont \(\displaystyle {-20}\)}%
\end{pgfscope}%
\begin{pgfscope}%
\pgfsetbuttcap%
\pgfsetroundjoin%
\definecolor{currentfill}{rgb}{0.000000,0.000000,0.000000}%
\pgfsetfillcolor{currentfill}%
\pgfsetlinewidth{0.803000pt}%
\definecolor{currentstroke}{rgb}{0.000000,0.000000,0.000000}%
\pgfsetstrokecolor{currentstroke}%
\pgfsetdash{}{0pt}%
\pgfsys@defobject{currentmarker}{\pgfqpoint{0.000000in}{-0.048611in}}{\pgfqpoint{0.000000in}{0.000000in}}{%
\pgfpathmoveto{\pgfqpoint{0.000000in}{0.000000in}}%
\pgfpathlineto{\pgfqpoint{0.000000in}{-0.048611in}}%
\pgfusepath{stroke,fill}%
}%
\begin{pgfscope}%
\pgfsys@transformshift{3.552570in}{0.548769in}%
\pgfsys@useobject{currentmarker}{}%
\end{pgfscope}%
\end{pgfscope}%
\begin{pgfscope}%
\definecolor{textcolor}{rgb}{0.000000,0.000000,0.000000}%
\pgfsetstrokecolor{textcolor}%
\pgfsetfillcolor{textcolor}%
\pgftext[x=3.552570in,y=0.451547in,,top]{\color{textcolor}\sffamily\fontsize{10.000000}{12.000000}\selectfont \(\displaystyle {0}\)}%
\end{pgfscope}%
\begin{pgfscope}%
\pgfsetbuttcap%
\pgfsetroundjoin%
\definecolor{currentfill}{rgb}{0.000000,0.000000,0.000000}%
\pgfsetfillcolor{currentfill}%
\pgfsetlinewidth{0.803000pt}%
\definecolor{currentstroke}{rgb}{0.000000,0.000000,0.000000}%
\pgfsetstrokecolor{currentstroke}%
\pgfsetdash{}{0pt}%
\pgfsys@defobject{currentmarker}{\pgfqpoint{0.000000in}{-0.048611in}}{\pgfqpoint{0.000000in}{0.000000in}}{%
\pgfpathmoveto{\pgfqpoint{0.000000in}{0.000000in}}%
\pgfpathlineto{\pgfqpoint{0.000000in}{-0.048611in}}%
\pgfusepath{stroke,fill}%
}%
\begin{pgfscope}%
\pgfsys@transformshift{4.332682in}{0.548769in}%
\pgfsys@useobject{currentmarker}{}%
\end{pgfscope}%
\end{pgfscope}%
\begin{pgfscope}%
\definecolor{textcolor}{rgb}{0.000000,0.000000,0.000000}%
\pgfsetstrokecolor{textcolor}%
\pgfsetfillcolor{textcolor}%
\pgftext[x=4.332682in,y=0.451547in,,top]{\color{textcolor}\sffamily\fontsize{10.000000}{12.000000}\selectfont \(\displaystyle {20}\)}%
\end{pgfscope}%
\begin{pgfscope}%
\pgfsetbuttcap%
\pgfsetroundjoin%
\definecolor{currentfill}{rgb}{0.000000,0.000000,0.000000}%
\pgfsetfillcolor{currentfill}%
\pgfsetlinewidth{0.803000pt}%
\definecolor{currentstroke}{rgb}{0.000000,0.000000,0.000000}%
\pgfsetstrokecolor{currentstroke}%
\pgfsetdash{}{0pt}%
\pgfsys@defobject{currentmarker}{\pgfqpoint{0.000000in}{-0.048611in}}{\pgfqpoint{0.000000in}{0.000000in}}{%
\pgfpathmoveto{\pgfqpoint{0.000000in}{0.000000in}}%
\pgfpathlineto{\pgfqpoint{0.000000in}{-0.048611in}}%
\pgfusepath{stroke,fill}%
}%
\begin{pgfscope}%
\pgfsys@transformshift{5.112794in}{0.548769in}%
\pgfsys@useobject{currentmarker}{}%
\end{pgfscope}%
\end{pgfscope}%
\begin{pgfscope}%
\definecolor{textcolor}{rgb}{0.000000,0.000000,0.000000}%
\pgfsetstrokecolor{textcolor}%
\pgfsetfillcolor{textcolor}%
\pgftext[x=5.112794in,y=0.451547in,,top]{\color{textcolor}\sffamily\fontsize{10.000000}{12.000000}\selectfont \(\displaystyle {40}\)}%
\end{pgfscope}%
\begin{pgfscope}%
\definecolor{textcolor}{rgb}{0.000000,0.000000,0.000000}%
\pgfsetstrokecolor{textcolor}%
\pgfsetfillcolor{textcolor}%
\pgftext[x=3.318537in,y=0.272658in,,top]{\color{textcolor}\sffamily\fontsize{10.000000}{12.000000}\selectfont Ratio of Sources to Sinks}%
\end{pgfscope}%
\begin{pgfscope}%
\pgfsetbuttcap%
\pgfsetroundjoin%
\definecolor{currentfill}{rgb}{0.000000,0.000000,0.000000}%
\pgfsetfillcolor{currentfill}%
\pgfsetlinewidth{0.803000pt}%
\definecolor{currentstroke}{rgb}{0.000000,0.000000,0.000000}%
\pgfsetstrokecolor{currentstroke}%
\pgfsetdash{}{0pt}%
\pgfsys@defobject{currentmarker}{\pgfqpoint{-0.048611in}{0.000000in}}{\pgfqpoint{0.000000in}{0.000000in}}{%
\pgfpathmoveto{\pgfqpoint{0.000000in}{0.000000in}}%
\pgfpathlineto{\pgfqpoint{-0.048611in}{0.000000in}}%
\pgfusepath{stroke,fill}%
}%
\begin{pgfscope}%
\pgfsys@transformshift{0.787074in}{0.689795in}%
\pgfsys@useobject{currentmarker}{}%
\end{pgfscope}%
\end{pgfscope}%
\begin{pgfscope}%
\definecolor{textcolor}{rgb}{0.000000,0.000000,0.000000}%
\pgfsetstrokecolor{textcolor}%
\pgfsetfillcolor{textcolor}%
\pgftext[x=0.620407in, y=0.641601in, left, base]{\color{textcolor}\sffamily\fontsize{10.000000}{12.000000}\selectfont \(\displaystyle {0}\)}%
\end{pgfscope}%
\begin{pgfscope}%
\pgfsetbuttcap%
\pgfsetroundjoin%
\definecolor{currentfill}{rgb}{0.000000,0.000000,0.000000}%
\pgfsetfillcolor{currentfill}%
\pgfsetlinewidth{0.803000pt}%
\definecolor{currentstroke}{rgb}{0.000000,0.000000,0.000000}%
\pgfsetstrokecolor{currentstroke}%
\pgfsetdash{}{0pt}%
\pgfsys@defobject{currentmarker}{\pgfqpoint{-0.048611in}{0.000000in}}{\pgfqpoint{0.000000in}{0.000000in}}{%
\pgfpathmoveto{\pgfqpoint{0.000000in}{0.000000in}}%
\pgfpathlineto{\pgfqpoint{-0.048611in}{0.000000in}}%
\pgfusepath{stroke,fill}%
}%
\begin{pgfscope}%
\pgfsys@transformshift{0.787074in}{1.062854in}%
\pgfsys@useobject{currentmarker}{}%
\end{pgfscope}%
\end{pgfscope}%
\begin{pgfscope}%
\definecolor{textcolor}{rgb}{0.000000,0.000000,0.000000}%
\pgfsetstrokecolor{textcolor}%
\pgfsetfillcolor{textcolor}%
\pgftext[x=0.412073in, y=1.014659in, left, base]{\color{textcolor}\sffamily\fontsize{10.000000}{12.000000}\selectfont \(\displaystyle {2500}\)}%
\end{pgfscope}%
\begin{pgfscope}%
\pgfsetbuttcap%
\pgfsetroundjoin%
\definecolor{currentfill}{rgb}{0.000000,0.000000,0.000000}%
\pgfsetfillcolor{currentfill}%
\pgfsetlinewidth{0.803000pt}%
\definecolor{currentstroke}{rgb}{0.000000,0.000000,0.000000}%
\pgfsetstrokecolor{currentstroke}%
\pgfsetdash{}{0pt}%
\pgfsys@defobject{currentmarker}{\pgfqpoint{-0.048611in}{0.000000in}}{\pgfqpoint{0.000000in}{0.000000in}}{%
\pgfpathmoveto{\pgfqpoint{0.000000in}{0.000000in}}%
\pgfpathlineto{\pgfqpoint{-0.048611in}{0.000000in}}%
\pgfusepath{stroke,fill}%
}%
\begin{pgfscope}%
\pgfsys@transformshift{0.787074in}{1.435912in}%
\pgfsys@useobject{currentmarker}{}%
\end{pgfscope}%
\end{pgfscope}%
\begin{pgfscope}%
\definecolor{textcolor}{rgb}{0.000000,0.000000,0.000000}%
\pgfsetstrokecolor{textcolor}%
\pgfsetfillcolor{textcolor}%
\pgftext[x=0.412073in, y=1.387718in, left, base]{\color{textcolor}\sffamily\fontsize{10.000000}{12.000000}\selectfont \(\displaystyle {5000}\)}%
\end{pgfscope}%
\begin{pgfscope}%
\pgfsetbuttcap%
\pgfsetroundjoin%
\definecolor{currentfill}{rgb}{0.000000,0.000000,0.000000}%
\pgfsetfillcolor{currentfill}%
\pgfsetlinewidth{0.803000pt}%
\definecolor{currentstroke}{rgb}{0.000000,0.000000,0.000000}%
\pgfsetstrokecolor{currentstroke}%
\pgfsetdash{}{0pt}%
\pgfsys@defobject{currentmarker}{\pgfqpoint{-0.048611in}{0.000000in}}{\pgfqpoint{0.000000in}{0.000000in}}{%
\pgfpathmoveto{\pgfqpoint{0.000000in}{0.000000in}}%
\pgfpathlineto{\pgfqpoint{-0.048611in}{0.000000in}}%
\pgfusepath{stroke,fill}%
}%
\begin{pgfscope}%
\pgfsys@transformshift{0.787074in}{1.808971in}%
\pgfsys@useobject{currentmarker}{}%
\end{pgfscope}%
\end{pgfscope}%
\begin{pgfscope}%
\definecolor{textcolor}{rgb}{0.000000,0.000000,0.000000}%
\pgfsetstrokecolor{textcolor}%
\pgfsetfillcolor{textcolor}%
\pgftext[x=0.412073in, y=1.760776in, left, base]{\color{textcolor}\sffamily\fontsize{10.000000}{12.000000}\selectfont \(\displaystyle {7500}\)}%
\end{pgfscope}%
\begin{pgfscope}%
\pgfsetbuttcap%
\pgfsetroundjoin%
\definecolor{currentfill}{rgb}{0.000000,0.000000,0.000000}%
\pgfsetfillcolor{currentfill}%
\pgfsetlinewidth{0.803000pt}%
\definecolor{currentstroke}{rgb}{0.000000,0.000000,0.000000}%
\pgfsetstrokecolor{currentstroke}%
\pgfsetdash{}{0pt}%
\pgfsys@defobject{currentmarker}{\pgfqpoint{-0.048611in}{0.000000in}}{\pgfqpoint{0.000000in}{0.000000in}}{%
\pgfpathmoveto{\pgfqpoint{0.000000in}{0.000000in}}%
\pgfpathlineto{\pgfqpoint{-0.048611in}{0.000000in}}%
\pgfusepath{stroke,fill}%
}%
\begin{pgfscope}%
\pgfsys@transformshift{0.787074in}{2.182029in}%
\pgfsys@useobject{currentmarker}{}%
\end{pgfscope}%
\end{pgfscope}%
\begin{pgfscope}%
\definecolor{textcolor}{rgb}{0.000000,0.000000,0.000000}%
\pgfsetstrokecolor{textcolor}%
\pgfsetfillcolor{textcolor}%
\pgftext[x=0.342628in, y=2.133835in, left, base]{\color{textcolor}\sffamily\fontsize{10.000000}{12.000000}\selectfont \(\displaystyle {10000}\)}%
\end{pgfscope}%
\begin{pgfscope}%
\pgfsetbuttcap%
\pgfsetroundjoin%
\definecolor{currentfill}{rgb}{0.000000,0.000000,0.000000}%
\pgfsetfillcolor{currentfill}%
\pgfsetlinewidth{0.803000pt}%
\definecolor{currentstroke}{rgb}{0.000000,0.000000,0.000000}%
\pgfsetstrokecolor{currentstroke}%
\pgfsetdash{}{0pt}%
\pgfsys@defobject{currentmarker}{\pgfqpoint{-0.048611in}{0.000000in}}{\pgfqpoint{0.000000in}{0.000000in}}{%
\pgfpathmoveto{\pgfqpoint{0.000000in}{0.000000in}}%
\pgfpathlineto{\pgfqpoint{-0.048611in}{0.000000in}}%
\pgfusepath{stroke,fill}%
}%
\begin{pgfscope}%
\pgfsys@transformshift{0.787074in}{2.555088in}%
\pgfsys@useobject{currentmarker}{}%
\end{pgfscope}%
\end{pgfscope}%
\begin{pgfscope}%
\definecolor{textcolor}{rgb}{0.000000,0.000000,0.000000}%
\pgfsetstrokecolor{textcolor}%
\pgfsetfillcolor{textcolor}%
\pgftext[x=0.342628in, y=2.506893in, left, base]{\color{textcolor}\sffamily\fontsize{10.000000}{12.000000}\selectfont \(\displaystyle {12500}\)}%
\end{pgfscope}%
\begin{pgfscope}%
\pgfsetbuttcap%
\pgfsetroundjoin%
\definecolor{currentfill}{rgb}{0.000000,0.000000,0.000000}%
\pgfsetfillcolor{currentfill}%
\pgfsetlinewidth{0.803000pt}%
\definecolor{currentstroke}{rgb}{0.000000,0.000000,0.000000}%
\pgfsetstrokecolor{currentstroke}%
\pgfsetdash{}{0pt}%
\pgfsys@defobject{currentmarker}{\pgfqpoint{-0.048611in}{0.000000in}}{\pgfqpoint{0.000000in}{0.000000in}}{%
\pgfpathmoveto{\pgfqpoint{0.000000in}{0.000000in}}%
\pgfpathlineto{\pgfqpoint{-0.048611in}{0.000000in}}%
\pgfusepath{stroke,fill}%
}%
\begin{pgfscope}%
\pgfsys@transformshift{0.787074in}{2.928146in}%
\pgfsys@useobject{currentmarker}{}%
\end{pgfscope}%
\end{pgfscope}%
\begin{pgfscope}%
\definecolor{textcolor}{rgb}{0.000000,0.000000,0.000000}%
\pgfsetstrokecolor{textcolor}%
\pgfsetfillcolor{textcolor}%
\pgftext[x=0.342628in, y=2.879952in, left, base]{\color{textcolor}\sffamily\fontsize{10.000000}{12.000000}\selectfont \(\displaystyle {15000}\)}%
\end{pgfscope}%
\begin{pgfscope}%
\pgfsetbuttcap%
\pgfsetroundjoin%
\definecolor{currentfill}{rgb}{0.000000,0.000000,0.000000}%
\pgfsetfillcolor{currentfill}%
\pgfsetlinewidth{0.803000pt}%
\definecolor{currentstroke}{rgb}{0.000000,0.000000,0.000000}%
\pgfsetstrokecolor{currentstroke}%
\pgfsetdash{}{0pt}%
\pgfsys@defobject{currentmarker}{\pgfqpoint{-0.048611in}{0.000000in}}{\pgfqpoint{0.000000in}{0.000000in}}{%
\pgfpathmoveto{\pgfqpoint{0.000000in}{0.000000in}}%
\pgfpathlineto{\pgfqpoint{-0.048611in}{0.000000in}}%
\pgfusepath{stroke,fill}%
}%
\begin{pgfscope}%
\pgfsys@transformshift{0.787074in}{3.301204in}%
\pgfsys@useobject{currentmarker}{}%
\end{pgfscope}%
\end{pgfscope}%
\begin{pgfscope}%
\definecolor{textcolor}{rgb}{0.000000,0.000000,0.000000}%
\pgfsetstrokecolor{textcolor}%
\pgfsetfillcolor{textcolor}%
\pgftext[x=0.342628in, y=3.253010in, left, base]{\color{textcolor}\sffamily\fontsize{10.000000}{12.000000}\selectfont \(\displaystyle {17500}\)}%
\end{pgfscope}%
\begin{pgfscope}%
\definecolor{textcolor}{rgb}{0.000000,0.000000,0.000000}%
\pgfsetstrokecolor{textcolor}%
\pgfsetfillcolor{textcolor}%
\pgftext[x=0.287073in,y=2.100064in,,bottom,rotate=90.000000]{\color{textcolor}\sffamily\fontsize{10.000000}{12.000000}\selectfont Maximum Memory Consumption (MB)}%
\end{pgfscope}%
\begin{pgfscope}%
\pgfsetrectcap%
\pgfsetmiterjoin%
\pgfsetlinewidth{0.803000pt}%
\definecolor{currentstroke}{rgb}{0.000000,0.000000,0.000000}%
\pgfsetstrokecolor{currentstroke}%
\pgfsetdash{}{0pt}%
\pgfpathmoveto{\pgfqpoint{0.787074in}{0.548769in}}%
\pgfpathlineto{\pgfqpoint{0.787074in}{3.651359in}}%
\pgfusepath{stroke}%
\end{pgfscope}%
\begin{pgfscope}%
\pgfsetrectcap%
\pgfsetmiterjoin%
\pgfsetlinewidth{0.803000pt}%
\definecolor{currentstroke}{rgb}{0.000000,0.000000,0.000000}%
\pgfsetstrokecolor{currentstroke}%
\pgfsetdash{}{0pt}%
\pgfpathmoveto{\pgfqpoint{5.850000in}{0.548769in}}%
\pgfpathlineto{\pgfqpoint{5.850000in}{3.651359in}}%
\pgfusepath{stroke}%
\end{pgfscope}%
\begin{pgfscope}%
\pgfsetrectcap%
\pgfsetmiterjoin%
\pgfsetlinewidth{0.803000pt}%
\definecolor{currentstroke}{rgb}{0.000000,0.000000,0.000000}%
\pgfsetstrokecolor{currentstroke}%
\pgfsetdash{}{0pt}%
\pgfpathmoveto{\pgfqpoint{0.787074in}{0.548769in}}%
\pgfpathlineto{\pgfqpoint{5.850000in}{0.548769in}}%
\pgfusepath{stroke}%
\end{pgfscope}%
\begin{pgfscope}%
\pgfsetrectcap%
\pgfsetmiterjoin%
\pgfsetlinewidth{0.803000pt}%
\definecolor{currentstroke}{rgb}{0.000000,0.000000,0.000000}%
\pgfsetstrokecolor{currentstroke}%
\pgfsetdash{}{0pt}%
\pgfpathmoveto{\pgfqpoint{0.787074in}{3.651359in}}%
\pgfpathlineto{\pgfqpoint{5.850000in}{3.651359in}}%
\pgfusepath{stroke}%
\end{pgfscope}%
\begin{pgfscope}%
\definecolor{textcolor}{rgb}{0.000000,0.000000,0.000000}%
\pgfsetstrokecolor{textcolor}%
\pgfsetfillcolor{textcolor}%
\pgftext[x=3.318537in,y=3.734692in,,base]{\color{textcolor}\sffamily\fontsize{12.000000}{14.400000}\selectfont Forward}%
\end{pgfscope}%
\begin{pgfscope}%
\pgfsetbuttcap%
\pgfsetmiterjoin%
\definecolor{currentfill}{rgb}{1.000000,1.000000,1.000000}%
\pgfsetfillcolor{currentfill}%
\pgfsetfillopacity{0.800000}%
\pgfsetlinewidth{1.003750pt}%
\definecolor{currentstroke}{rgb}{0.800000,0.800000,0.800000}%
\pgfsetstrokecolor{currentstroke}%
\pgfsetstrokeopacity{0.800000}%
\pgfsetdash{}{0pt}%
\pgfpathmoveto{\pgfqpoint{0.884296in}{2.957886in}}%
\pgfpathlineto{\pgfqpoint{2.336657in}{2.957886in}}%
\pgfpathquadraticcurveto{\pgfqpoint{2.364435in}{2.957886in}}{\pgfqpoint{2.364435in}{2.985664in}}%
\pgfpathlineto{\pgfqpoint{2.364435in}{3.554136in}}%
\pgfpathquadraticcurveto{\pgfqpoint{2.364435in}{3.581914in}}{\pgfqpoint{2.336657in}{3.581914in}}%
\pgfpathlineto{\pgfqpoint{0.884296in}{3.581914in}}%
\pgfpathquadraticcurveto{\pgfqpoint{0.856518in}{3.581914in}}{\pgfqpoint{0.856518in}{3.554136in}}%
\pgfpathlineto{\pgfqpoint{0.856518in}{2.985664in}}%
\pgfpathquadraticcurveto{\pgfqpoint{0.856518in}{2.957886in}}{\pgfqpoint{0.884296in}{2.957886in}}%
\pgfpathclose%
\pgfusepath{stroke,fill}%
\end{pgfscope}%
\begin{pgfscope}%
\pgfsetbuttcap%
\pgfsetroundjoin%
\definecolor{currentfill}{rgb}{0.121569,0.466667,0.705882}%
\pgfsetfillcolor{currentfill}%
\pgfsetlinewidth{1.003750pt}%
\definecolor{currentstroke}{rgb}{0.121569,0.466667,0.705882}%
\pgfsetstrokecolor{currentstroke}%
\pgfsetdash{}{0pt}%
\pgfsys@defobject{currentmarker}{\pgfqpoint{-0.034722in}{-0.034722in}}{\pgfqpoint{0.034722in}{0.034722in}}{%
\pgfpathmoveto{\pgfqpoint{0.000000in}{-0.034722in}}%
\pgfpathcurveto{\pgfqpoint{0.009208in}{-0.034722in}}{\pgfqpoint{0.018041in}{-0.031064in}}{\pgfqpoint{0.024552in}{-0.024552in}}%
\pgfpathcurveto{\pgfqpoint{0.031064in}{-0.018041in}}{\pgfqpoint{0.034722in}{-0.009208in}}{\pgfqpoint{0.034722in}{0.000000in}}%
\pgfpathcurveto{\pgfqpoint{0.034722in}{0.009208in}}{\pgfqpoint{0.031064in}{0.018041in}}{\pgfqpoint{0.024552in}{0.024552in}}%
\pgfpathcurveto{\pgfqpoint{0.018041in}{0.031064in}}{\pgfqpoint{0.009208in}{0.034722in}}{\pgfqpoint{0.000000in}{0.034722in}}%
\pgfpathcurveto{\pgfqpoint{-0.009208in}{0.034722in}}{\pgfqpoint{-0.018041in}{0.031064in}}{\pgfqpoint{-0.024552in}{0.024552in}}%
\pgfpathcurveto{\pgfqpoint{-0.031064in}{0.018041in}}{\pgfqpoint{-0.034722in}{0.009208in}}{\pgfqpoint{-0.034722in}{0.000000in}}%
\pgfpathcurveto{\pgfqpoint{-0.034722in}{-0.009208in}}{\pgfqpoint{-0.031064in}{-0.018041in}}{\pgfqpoint{-0.024552in}{-0.024552in}}%
\pgfpathcurveto{\pgfqpoint{-0.018041in}{-0.031064in}}{\pgfqpoint{-0.009208in}{-0.034722in}}{\pgfqpoint{0.000000in}{-0.034722in}}%
\pgfpathclose%
\pgfusepath{stroke,fill}%
}%
\begin{pgfscope}%
\pgfsys@transformshift{1.050963in}{3.477748in}%
\pgfsys@useobject{currentmarker}{}%
\end{pgfscope}%
\end{pgfscope}%
\begin{pgfscope}%
\definecolor{textcolor}{rgb}{0.000000,0.000000,0.000000}%
\pgfsetstrokecolor{textcolor}%
\pgfsetfillcolor{textcolor}%
\pgftext[x=1.300963in,y=3.429136in,left,base]{\color{textcolor}\sffamily\fontsize{10.000000}{12.000000}\selectfont No Timeout}%
\end{pgfscope}%
\begin{pgfscope}%
\pgfsetbuttcap%
\pgfsetroundjoin%
\definecolor{currentfill}{rgb}{1.000000,0.498039,0.054902}%
\pgfsetfillcolor{currentfill}%
\pgfsetlinewidth{1.003750pt}%
\definecolor{currentstroke}{rgb}{1.000000,0.498039,0.054902}%
\pgfsetstrokecolor{currentstroke}%
\pgfsetdash{}{0pt}%
\pgfsys@defobject{currentmarker}{\pgfqpoint{-0.034722in}{-0.034722in}}{\pgfqpoint{0.034722in}{0.034722in}}{%
\pgfpathmoveto{\pgfqpoint{0.000000in}{-0.034722in}}%
\pgfpathcurveto{\pgfqpoint{0.009208in}{-0.034722in}}{\pgfqpoint{0.018041in}{-0.031064in}}{\pgfqpoint{0.024552in}{-0.024552in}}%
\pgfpathcurveto{\pgfqpoint{0.031064in}{-0.018041in}}{\pgfqpoint{0.034722in}{-0.009208in}}{\pgfqpoint{0.034722in}{0.000000in}}%
\pgfpathcurveto{\pgfqpoint{0.034722in}{0.009208in}}{\pgfqpoint{0.031064in}{0.018041in}}{\pgfqpoint{0.024552in}{0.024552in}}%
\pgfpathcurveto{\pgfqpoint{0.018041in}{0.031064in}}{\pgfqpoint{0.009208in}{0.034722in}}{\pgfqpoint{0.000000in}{0.034722in}}%
\pgfpathcurveto{\pgfqpoint{-0.009208in}{0.034722in}}{\pgfqpoint{-0.018041in}{0.031064in}}{\pgfqpoint{-0.024552in}{0.024552in}}%
\pgfpathcurveto{\pgfqpoint{-0.031064in}{0.018041in}}{\pgfqpoint{-0.034722in}{0.009208in}}{\pgfqpoint{-0.034722in}{0.000000in}}%
\pgfpathcurveto{\pgfqpoint{-0.034722in}{-0.009208in}}{\pgfqpoint{-0.031064in}{-0.018041in}}{\pgfqpoint{-0.024552in}{-0.024552in}}%
\pgfpathcurveto{\pgfqpoint{-0.018041in}{-0.031064in}}{\pgfqpoint{-0.009208in}{-0.034722in}}{\pgfqpoint{0.000000in}{-0.034722in}}%
\pgfpathclose%
\pgfusepath{stroke,fill}%
}%
\begin{pgfscope}%
\pgfsys@transformshift{1.050963in}{3.284136in}%
\pgfsys@useobject{currentmarker}{}%
\end{pgfscope}%
\end{pgfscope}%
\begin{pgfscope}%
\definecolor{textcolor}{rgb}{0.000000,0.000000,0.000000}%
\pgfsetstrokecolor{textcolor}%
\pgfsetfillcolor{textcolor}%
\pgftext[x=1.300963in,y=3.235525in,left,base]{\color{textcolor}\sffamily\fontsize{10.000000}{12.000000}\selectfont Time Timeout}%
\end{pgfscope}%
\begin{pgfscope}%
\pgfsetbuttcap%
\pgfsetroundjoin%
\definecolor{currentfill}{rgb}{0.839216,0.152941,0.156863}%
\pgfsetfillcolor{currentfill}%
\pgfsetlinewidth{1.003750pt}%
\definecolor{currentstroke}{rgb}{0.839216,0.152941,0.156863}%
\pgfsetstrokecolor{currentstroke}%
\pgfsetdash{}{0pt}%
\pgfsys@defobject{currentmarker}{\pgfqpoint{-0.034722in}{-0.034722in}}{\pgfqpoint{0.034722in}{0.034722in}}{%
\pgfpathmoveto{\pgfqpoint{0.000000in}{-0.034722in}}%
\pgfpathcurveto{\pgfqpoint{0.009208in}{-0.034722in}}{\pgfqpoint{0.018041in}{-0.031064in}}{\pgfqpoint{0.024552in}{-0.024552in}}%
\pgfpathcurveto{\pgfqpoint{0.031064in}{-0.018041in}}{\pgfqpoint{0.034722in}{-0.009208in}}{\pgfqpoint{0.034722in}{0.000000in}}%
\pgfpathcurveto{\pgfqpoint{0.034722in}{0.009208in}}{\pgfqpoint{0.031064in}{0.018041in}}{\pgfqpoint{0.024552in}{0.024552in}}%
\pgfpathcurveto{\pgfqpoint{0.018041in}{0.031064in}}{\pgfqpoint{0.009208in}{0.034722in}}{\pgfqpoint{0.000000in}{0.034722in}}%
\pgfpathcurveto{\pgfqpoint{-0.009208in}{0.034722in}}{\pgfqpoint{-0.018041in}{0.031064in}}{\pgfqpoint{-0.024552in}{0.024552in}}%
\pgfpathcurveto{\pgfqpoint{-0.031064in}{0.018041in}}{\pgfqpoint{-0.034722in}{0.009208in}}{\pgfqpoint{-0.034722in}{0.000000in}}%
\pgfpathcurveto{\pgfqpoint{-0.034722in}{-0.009208in}}{\pgfqpoint{-0.031064in}{-0.018041in}}{\pgfqpoint{-0.024552in}{-0.024552in}}%
\pgfpathcurveto{\pgfqpoint{-0.018041in}{-0.031064in}}{\pgfqpoint{-0.009208in}{-0.034722in}}{\pgfqpoint{0.000000in}{-0.034722in}}%
\pgfpathclose%
\pgfusepath{stroke,fill}%
}%
\begin{pgfscope}%
\pgfsys@transformshift{1.050963in}{3.090525in}%
\pgfsys@useobject{currentmarker}{}%
\end{pgfscope}%
\end{pgfscope}%
\begin{pgfscope}%
\definecolor{textcolor}{rgb}{0.000000,0.000000,0.000000}%
\pgfsetstrokecolor{textcolor}%
\pgfsetfillcolor{textcolor}%
\pgftext[x=1.300963in,y=3.041914in,left,base]{\color{textcolor}\sffamily\fontsize{10.000000}{12.000000}\selectfont Memory Timeout}%
\end{pgfscope}%
\end{pgfpicture}%
\makeatother%
\endgroup%

                }
            \end{subfigure}
            \caption{Ratio}
        \end{subfigure}
        \caption{Maximum Memory Consumption In Comparison To Source, Sink And Edge Count}
        \label{f:maxmemtoss}
    \end{figure}

    \begin{figure}[tbp]
        \centering
        \begin{subfigure}[b]{\textwidth}
            \centering
            \begin{subfigure}[]{0.45\textwidth}
                \centering
                \resizebox{\columnwidth}{!}{
                    %% Creator: Matplotlib, PGF backend
%%
%% To include the figure in your LaTeX document, write
%%   \input{<filename>.pgf}
%%
%% Make sure the required packages are loaded in your preamble
%%   \usepackage{pgf}
%%
%% and, on pdftex
%%   \usepackage[utf8]{inputenc}\DeclareUnicodeCharacter{2212}{-}
%%
%% or, on luatex and xetex
%%   \usepackage{unicode-math}
%%
%% Figures using additional raster images can only be included by \input if
%% they are in the same directory as the main LaTeX file. For loading figures
%% from other directories you can use the `import` package
%%   \usepackage{import}
%%
%% and then include the figures with
%%   \import{<path to file>}{<filename>.pgf}
%%
%% Matplotlib used the following preamble
%%   \usepackage{amsmath}
%%   \usepackage{fontspec}
%%
\begingroup%
\makeatletter%
\begin{pgfpicture}%
\pgfpathrectangle{\pgfpointorigin}{\pgfqpoint{6.000000in}{4.000000in}}%
\pgfusepath{use as bounding box, clip}%
\begin{pgfscope}%
\pgfsetbuttcap%
\pgfsetmiterjoin%
\definecolor{currentfill}{rgb}{1.000000,1.000000,1.000000}%
\pgfsetfillcolor{currentfill}%
\pgfsetlinewidth{0.000000pt}%
\definecolor{currentstroke}{rgb}{1.000000,1.000000,1.000000}%
\pgfsetstrokecolor{currentstroke}%
\pgfsetdash{}{0pt}%
\pgfpathmoveto{\pgfqpoint{0.000000in}{0.000000in}}%
\pgfpathlineto{\pgfqpoint{6.000000in}{0.000000in}}%
\pgfpathlineto{\pgfqpoint{6.000000in}{4.000000in}}%
\pgfpathlineto{\pgfqpoint{0.000000in}{4.000000in}}%
\pgfpathclose%
\pgfusepath{fill}%
\end{pgfscope}%
\begin{pgfscope}%
\pgfsetbuttcap%
\pgfsetmiterjoin%
\definecolor{currentfill}{rgb}{1.000000,1.000000,1.000000}%
\pgfsetfillcolor{currentfill}%
\pgfsetlinewidth{0.000000pt}%
\definecolor{currentstroke}{rgb}{0.000000,0.000000,0.000000}%
\pgfsetstrokecolor{currentstroke}%
\pgfsetstrokeopacity{0.000000}%
\pgfsetdash{}{0pt}%
\pgfpathmoveto{\pgfqpoint{0.787074in}{0.548769in}}%
\pgfpathlineto{\pgfqpoint{5.850000in}{0.548769in}}%
\pgfpathlineto{\pgfqpoint{5.850000in}{3.651359in}}%
\pgfpathlineto{\pgfqpoint{0.787074in}{3.651359in}}%
\pgfpathclose%
\pgfusepath{fill}%
\end{pgfscope}%
\begin{pgfscope}%
\pgfpathrectangle{\pgfqpoint{0.787074in}{0.548769in}}{\pgfqpoint{5.062926in}{3.102590in}}%
\pgfusepath{clip}%
\pgfsetbuttcap%
\pgfsetroundjoin%
\definecolor{currentfill}{rgb}{0.121569,0.466667,0.705882}%
\pgfsetfillcolor{currentfill}%
\pgfsetlinewidth{1.003750pt}%
\definecolor{currentstroke}{rgb}{0.121569,0.466667,0.705882}%
\pgfsetstrokecolor{currentstroke}%
\pgfsetdash{}{0pt}%
\pgfpathmoveto{\pgfqpoint{1.239778in}{0.648198in}}%
\pgfpathcurveto{\pgfqpoint{1.250828in}{0.648198in}}{\pgfqpoint{1.261427in}{0.652588in}}{\pgfqpoint{1.269241in}{0.660402in}}%
\pgfpathcurveto{\pgfqpoint{1.277054in}{0.668215in}}{\pgfqpoint{1.281445in}{0.678814in}}{\pgfqpoint{1.281445in}{0.689865in}}%
\pgfpathcurveto{\pgfqpoint{1.281445in}{0.700915in}}{\pgfqpoint{1.277054in}{0.711514in}}{\pgfqpoint{1.269241in}{0.719327in}}%
\pgfpathcurveto{\pgfqpoint{1.261427in}{0.727141in}}{\pgfqpoint{1.250828in}{0.731531in}}{\pgfqpoint{1.239778in}{0.731531in}}%
\pgfpathcurveto{\pgfqpoint{1.228728in}{0.731531in}}{\pgfqpoint{1.218129in}{0.727141in}}{\pgfqpoint{1.210315in}{0.719327in}}%
\pgfpathcurveto{\pgfqpoint{1.202502in}{0.711514in}}{\pgfqpoint{1.198111in}{0.700915in}}{\pgfqpoint{1.198111in}{0.689865in}}%
\pgfpathcurveto{\pgfqpoint{1.198111in}{0.678814in}}{\pgfqpoint{1.202502in}{0.668215in}}{\pgfqpoint{1.210315in}{0.660402in}}%
\pgfpathcurveto{\pgfqpoint{1.218129in}{0.652588in}}{\pgfqpoint{1.228728in}{0.648198in}}{\pgfqpoint{1.239778in}{0.648198in}}%
\pgfpathclose%
\pgfusepath{stroke,fill}%
\end{pgfscope}%
\begin{pgfscope}%
\pgfpathrectangle{\pgfqpoint{0.787074in}{0.548769in}}{\pgfqpoint{5.062926in}{3.102590in}}%
\pgfusepath{clip}%
\pgfsetbuttcap%
\pgfsetroundjoin%
\definecolor{currentfill}{rgb}{0.121569,0.466667,0.705882}%
\pgfsetfillcolor{currentfill}%
\pgfsetlinewidth{1.003750pt}%
\definecolor{currentstroke}{rgb}{0.121569,0.466667,0.705882}%
\pgfsetstrokecolor{currentstroke}%
\pgfsetdash{}{0pt}%
\pgfpathmoveto{\pgfqpoint{2.900535in}{1.887135in}}%
\pgfpathcurveto{\pgfqpoint{2.911585in}{1.887135in}}{\pgfqpoint{2.922184in}{1.891526in}}{\pgfqpoint{2.929998in}{1.899339in}}%
\pgfpathcurveto{\pgfqpoint{2.937812in}{1.907153in}}{\pgfqpoint{2.942202in}{1.917752in}}{\pgfqpoint{2.942202in}{1.928802in}}%
\pgfpathcurveto{\pgfqpoint{2.942202in}{1.939852in}}{\pgfqpoint{2.937812in}{1.950451in}}{\pgfqpoint{2.929998in}{1.958265in}}%
\pgfpathcurveto{\pgfqpoint{2.922184in}{1.966079in}}{\pgfqpoint{2.911585in}{1.970469in}}{\pgfqpoint{2.900535in}{1.970469in}}%
\pgfpathcurveto{\pgfqpoint{2.889485in}{1.970469in}}{\pgfqpoint{2.878886in}{1.966079in}}{\pgfqpoint{2.871072in}{1.958265in}}%
\pgfpathcurveto{\pgfqpoint{2.863259in}{1.950451in}}{\pgfqpoint{2.858868in}{1.939852in}}{\pgfqpoint{2.858868in}{1.928802in}}%
\pgfpathcurveto{\pgfqpoint{2.858868in}{1.917752in}}{\pgfqpoint{2.863259in}{1.907153in}}{\pgfqpoint{2.871072in}{1.899339in}}%
\pgfpathcurveto{\pgfqpoint{2.878886in}{1.891526in}}{\pgfqpoint{2.889485in}{1.887135in}}{\pgfqpoint{2.900535in}{1.887135in}}%
\pgfpathclose%
\pgfusepath{stroke,fill}%
\end{pgfscope}%
\begin{pgfscope}%
\pgfpathrectangle{\pgfqpoint{0.787074in}{0.548769in}}{\pgfqpoint{5.062926in}{3.102590in}}%
\pgfusepath{clip}%
\pgfsetbuttcap%
\pgfsetroundjoin%
\definecolor{currentfill}{rgb}{1.000000,0.498039,0.054902}%
\pgfsetfillcolor{currentfill}%
\pgfsetlinewidth{1.003750pt}%
\definecolor{currentstroke}{rgb}{1.000000,0.498039,0.054902}%
\pgfsetstrokecolor{currentstroke}%
\pgfsetdash{}{0pt}%
\pgfpathmoveto{\pgfqpoint{1.388404in}{2.790510in}}%
\pgfpathcurveto{\pgfqpoint{1.399454in}{2.790510in}}{\pgfqpoint{1.410053in}{2.794900in}}{\pgfqpoint{1.417867in}{2.802714in}}%
\pgfpathcurveto{\pgfqpoint{1.425680in}{2.810527in}}{\pgfqpoint{1.430071in}{2.821126in}}{\pgfqpoint{1.430071in}{2.832176in}}%
\pgfpathcurveto{\pgfqpoint{1.430071in}{2.843227in}}{\pgfqpoint{1.425680in}{2.853826in}}{\pgfqpoint{1.417867in}{2.861639in}}%
\pgfpathcurveto{\pgfqpoint{1.410053in}{2.869453in}}{\pgfqpoint{1.399454in}{2.873843in}}{\pgfqpoint{1.388404in}{2.873843in}}%
\pgfpathcurveto{\pgfqpoint{1.377354in}{2.873843in}}{\pgfqpoint{1.366755in}{2.869453in}}{\pgfqpoint{1.358941in}{2.861639in}}%
\pgfpathcurveto{\pgfqpoint{1.351128in}{2.853826in}}{\pgfqpoint{1.346737in}{2.843227in}}{\pgfqpoint{1.346737in}{2.832176in}}%
\pgfpathcurveto{\pgfqpoint{1.346737in}{2.821126in}}{\pgfqpoint{1.351128in}{2.810527in}}{\pgfqpoint{1.358941in}{2.802714in}}%
\pgfpathcurveto{\pgfqpoint{1.366755in}{2.794900in}}{\pgfqpoint{1.377354in}{2.790510in}}{\pgfqpoint{1.388404in}{2.790510in}}%
\pgfpathclose%
\pgfusepath{stroke,fill}%
\end{pgfscope}%
\begin{pgfscope}%
\pgfpathrectangle{\pgfqpoint{0.787074in}{0.548769in}}{\pgfqpoint{5.062926in}{3.102590in}}%
\pgfusepath{clip}%
\pgfsetbuttcap%
\pgfsetroundjoin%
\definecolor{currentfill}{rgb}{0.121569,0.466667,0.705882}%
\pgfsetfillcolor{currentfill}%
\pgfsetlinewidth{1.003750pt}%
\definecolor{currentstroke}{rgb}{0.121569,0.466667,0.705882}%
\pgfsetstrokecolor{currentstroke}%
\pgfsetdash{}{0pt}%
\pgfpathmoveto{\pgfqpoint{1.638133in}{2.680271in}}%
\pgfpathcurveto{\pgfqpoint{1.649183in}{2.680271in}}{\pgfqpoint{1.659782in}{2.684661in}}{\pgfqpoint{1.667596in}{2.692475in}}%
\pgfpathcurveto{\pgfqpoint{1.675409in}{2.700289in}}{\pgfqpoint{1.679799in}{2.710888in}}{\pgfqpoint{1.679799in}{2.721938in}}%
\pgfpathcurveto{\pgfqpoint{1.679799in}{2.732988in}}{\pgfqpoint{1.675409in}{2.743587in}}{\pgfqpoint{1.667596in}{2.751401in}}%
\pgfpathcurveto{\pgfqpoint{1.659782in}{2.759214in}}{\pgfqpoint{1.649183in}{2.763604in}}{\pgfqpoint{1.638133in}{2.763604in}}%
\pgfpathcurveto{\pgfqpoint{1.627083in}{2.763604in}}{\pgfqpoint{1.616484in}{2.759214in}}{\pgfqpoint{1.608670in}{2.751401in}}%
\pgfpathcurveto{\pgfqpoint{1.600856in}{2.743587in}}{\pgfqpoint{1.596466in}{2.732988in}}{\pgfqpoint{1.596466in}{2.721938in}}%
\pgfpathcurveto{\pgfqpoint{1.596466in}{2.710888in}}{\pgfqpoint{1.600856in}{2.700289in}}{\pgfqpoint{1.608670in}{2.692475in}}%
\pgfpathcurveto{\pgfqpoint{1.616484in}{2.684661in}}{\pgfqpoint{1.627083in}{2.680271in}}{\pgfqpoint{1.638133in}{2.680271in}}%
\pgfpathclose%
\pgfusepath{stroke,fill}%
\end{pgfscope}%
\begin{pgfscope}%
\pgfpathrectangle{\pgfqpoint{0.787074in}{0.548769in}}{\pgfqpoint{5.062926in}{3.102590in}}%
\pgfusepath{clip}%
\pgfsetbuttcap%
\pgfsetroundjoin%
\definecolor{currentfill}{rgb}{1.000000,0.498039,0.054902}%
\pgfsetfillcolor{currentfill}%
\pgfsetlinewidth{1.003750pt}%
\definecolor{currentstroke}{rgb}{1.000000,0.498039,0.054902}%
\pgfsetstrokecolor{currentstroke}%
\pgfsetdash{}{0pt}%
\pgfpathmoveto{\pgfqpoint{1.474701in}{2.647277in}}%
\pgfpathcurveto{\pgfqpoint{1.485751in}{2.647277in}}{\pgfqpoint{1.496350in}{2.651668in}}{\pgfqpoint{1.504164in}{2.659481in}}%
\pgfpathcurveto{\pgfqpoint{1.511977in}{2.667295in}}{\pgfqpoint{1.516368in}{2.677894in}}{\pgfqpoint{1.516368in}{2.688944in}}%
\pgfpathcurveto{\pgfqpoint{1.516368in}{2.699994in}}{\pgfqpoint{1.511977in}{2.710593in}}{\pgfqpoint{1.504164in}{2.718407in}}%
\pgfpathcurveto{\pgfqpoint{1.496350in}{2.726221in}}{\pgfqpoint{1.485751in}{2.730611in}}{\pgfqpoint{1.474701in}{2.730611in}}%
\pgfpathcurveto{\pgfqpoint{1.463651in}{2.730611in}}{\pgfqpoint{1.453052in}{2.726221in}}{\pgfqpoint{1.445238in}{2.718407in}}%
\pgfpathcurveto{\pgfqpoint{1.437424in}{2.710593in}}{\pgfqpoint{1.433034in}{2.699994in}}{\pgfqpoint{1.433034in}{2.688944in}}%
\pgfpathcurveto{\pgfqpoint{1.433034in}{2.677894in}}{\pgfqpoint{1.437424in}{2.667295in}}{\pgfqpoint{1.445238in}{2.659481in}}%
\pgfpathcurveto{\pgfqpoint{1.453052in}{2.651668in}}{\pgfqpoint{1.463651in}{2.647277in}}{\pgfqpoint{1.474701in}{2.647277in}}%
\pgfpathclose%
\pgfusepath{stroke,fill}%
\end{pgfscope}%
\begin{pgfscope}%
\pgfpathrectangle{\pgfqpoint{0.787074in}{0.548769in}}{\pgfqpoint{5.062926in}{3.102590in}}%
\pgfusepath{clip}%
\pgfsetbuttcap%
\pgfsetroundjoin%
\definecolor{currentfill}{rgb}{0.121569,0.466667,0.705882}%
\pgfsetfillcolor{currentfill}%
\pgfsetlinewidth{1.003750pt}%
\definecolor{currentstroke}{rgb}{0.121569,0.466667,0.705882}%
\pgfsetstrokecolor{currentstroke}%
\pgfsetdash{}{0pt}%
\pgfpathmoveto{\pgfqpoint{1.788011in}{2.057905in}}%
\pgfpathcurveto{\pgfqpoint{1.799061in}{2.057905in}}{\pgfqpoint{1.809660in}{2.062295in}}{\pgfqpoint{1.817474in}{2.070109in}}%
\pgfpathcurveto{\pgfqpoint{1.825287in}{2.077922in}}{\pgfqpoint{1.829678in}{2.088521in}}{\pgfqpoint{1.829678in}{2.099572in}}%
\pgfpathcurveto{\pgfqpoint{1.829678in}{2.110622in}}{\pgfqpoint{1.825287in}{2.121221in}}{\pgfqpoint{1.817474in}{2.129034in}}%
\pgfpathcurveto{\pgfqpoint{1.809660in}{2.136848in}}{\pgfqpoint{1.799061in}{2.141238in}}{\pgfqpoint{1.788011in}{2.141238in}}%
\pgfpathcurveto{\pgfqpoint{1.776961in}{2.141238in}}{\pgfqpoint{1.766362in}{2.136848in}}{\pgfqpoint{1.758548in}{2.129034in}}%
\pgfpathcurveto{\pgfqpoint{1.750735in}{2.121221in}}{\pgfqpoint{1.746344in}{2.110622in}}{\pgfqpoint{1.746344in}{2.099572in}}%
\pgfpathcurveto{\pgfqpoint{1.746344in}{2.088521in}}{\pgfqpoint{1.750735in}{2.077922in}}{\pgfqpoint{1.758548in}{2.070109in}}%
\pgfpathcurveto{\pgfqpoint{1.766362in}{2.062295in}}{\pgfqpoint{1.776961in}{2.057905in}}{\pgfqpoint{1.788011in}{2.057905in}}%
\pgfpathclose%
\pgfusepath{stroke,fill}%
\end{pgfscope}%
\begin{pgfscope}%
\pgfpathrectangle{\pgfqpoint{0.787074in}{0.548769in}}{\pgfqpoint{5.062926in}{3.102590in}}%
\pgfusepath{clip}%
\pgfsetbuttcap%
\pgfsetroundjoin%
\definecolor{currentfill}{rgb}{0.121569,0.466667,0.705882}%
\pgfsetfillcolor{currentfill}%
\pgfsetlinewidth{1.003750pt}%
\definecolor{currentstroke}{rgb}{0.121569,0.466667,0.705882}%
\pgfsetstrokecolor{currentstroke}%
\pgfsetdash{}{0pt}%
\pgfpathmoveto{\pgfqpoint{2.894286in}{2.153691in}}%
\pgfpathcurveto{\pgfqpoint{2.905336in}{2.153691in}}{\pgfqpoint{2.915935in}{2.158081in}}{\pgfqpoint{2.923749in}{2.165895in}}%
\pgfpathcurveto{\pgfqpoint{2.931562in}{2.173708in}}{\pgfqpoint{2.935953in}{2.184307in}}{\pgfqpoint{2.935953in}{2.195357in}}%
\pgfpathcurveto{\pgfqpoint{2.935953in}{2.206408in}}{\pgfqpoint{2.931562in}{2.217007in}}{\pgfqpoint{2.923749in}{2.224820in}}%
\pgfpathcurveto{\pgfqpoint{2.915935in}{2.232634in}}{\pgfqpoint{2.905336in}{2.237024in}}{\pgfqpoint{2.894286in}{2.237024in}}%
\pgfpathcurveto{\pgfqpoint{2.883236in}{2.237024in}}{\pgfqpoint{2.872637in}{2.232634in}}{\pgfqpoint{2.864823in}{2.224820in}}%
\pgfpathcurveto{\pgfqpoint{2.857010in}{2.217007in}}{\pgfqpoint{2.852619in}{2.206408in}}{\pgfqpoint{2.852619in}{2.195357in}}%
\pgfpathcurveto{\pgfqpoint{2.852619in}{2.184307in}}{\pgfqpoint{2.857010in}{2.173708in}}{\pgfqpoint{2.864823in}{2.165895in}}%
\pgfpathcurveto{\pgfqpoint{2.872637in}{2.158081in}}{\pgfqpoint{2.883236in}{2.153691in}}{\pgfqpoint{2.894286in}{2.153691in}}%
\pgfpathclose%
\pgfusepath{stroke,fill}%
\end{pgfscope}%
\begin{pgfscope}%
\pgfpathrectangle{\pgfqpoint{0.787074in}{0.548769in}}{\pgfqpoint{5.062926in}{3.102590in}}%
\pgfusepath{clip}%
\pgfsetbuttcap%
\pgfsetroundjoin%
\definecolor{currentfill}{rgb}{1.000000,0.498039,0.054902}%
\pgfsetfillcolor{currentfill}%
\pgfsetlinewidth{1.003750pt}%
\definecolor{currentstroke}{rgb}{1.000000,0.498039,0.054902}%
\pgfsetstrokecolor{currentstroke}%
\pgfsetdash{}{0pt}%
\pgfpathmoveto{\pgfqpoint{1.591184in}{2.714270in}}%
\pgfpathcurveto{\pgfqpoint{1.602234in}{2.714270in}}{\pgfqpoint{1.612833in}{2.718660in}}{\pgfqpoint{1.620646in}{2.726474in}}%
\pgfpathcurveto{\pgfqpoint{1.628460in}{2.734288in}}{\pgfqpoint{1.632850in}{2.744887in}}{\pgfqpoint{1.632850in}{2.755937in}}%
\pgfpathcurveto{\pgfqpoint{1.632850in}{2.766987in}}{\pgfqpoint{1.628460in}{2.777586in}}{\pgfqpoint{1.620646in}{2.785399in}}%
\pgfpathcurveto{\pgfqpoint{1.612833in}{2.793213in}}{\pgfqpoint{1.602234in}{2.797603in}}{\pgfqpoint{1.591184in}{2.797603in}}%
\pgfpathcurveto{\pgfqpoint{1.580134in}{2.797603in}}{\pgfqpoint{1.569535in}{2.793213in}}{\pgfqpoint{1.561721in}{2.785399in}}%
\pgfpathcurveto{\pgfqpoint{1.553907in}{2.777586in}}{\pgfqpoint{1.549517in}{2.766987in}}{\pgfqpoint{1.549517in}{2.755937in}}%
\pgfpathcurveto{\pgfqpoint{1.549517in}{2.744887in}}{\pgfqpoint{1.553907in}{2.734288in}}{\pgfqpoint{1.561721in}{2.726474in}}%
\pgfpathcurveto{\pgfqpoint{1.569535in}{2.718660in}}{\pgfqpoint{1.580134in}{2.714270in}}{\pgfqpoint{1.591184in}{2.714270in}}%
\pgfpathclose%
\pgfusepath{stroke,fill}%
\end{pgfscope}%
\begin{pgfscope}%
\pgfpathrectangle{\pgfqpoint{0.787074in}{0.548769in}}{\pgfqpoint{5.062926in}{3.102590in}}%
\pgfusepath{clip}%
\pgfsetbuttcap%
\pgfsetroundjoin%
\definecolor{currentfill}{rgb}{1.000000,0.498039,0.054902}%
\pgfsetfillcolor{currentfill}%
\pgfsetlinewidth{1.003750pt}%
\definecolor{currentstroke}{rgb}{1.000000,0.498039,0.054902}%
\pgfsetstrokecolor{currentstroke}%
\pgfsetdash{}{0pt}%
\pgfpathmoveto{\pgfqpoint{1.801199in}{2.861842in}}%
\pgfpathcurveto{\pgfqpoint{1.812249in}{2.861842in}}{\pgfqpoint{1.822848in}{2.866233in}}{\pgfqpoint{1.830662in}{2.874046in}}%
\pgfpathcurveto{\pgfqpoint{1.838475in}{2.881860in}}{\pgfqpoint{1.842865in}{2.892459in}}{\pgfqpoint{1.842865in}{2.903509in}}%
\pgfpathcurveto{\pgfqpoint{1.842865in}{2.914559in}}{\pgfqpoint{1.838475in}{2.925158in}}{\pgfqpoint{1.830662in}{2.932972in}}%
\pgfpathcurveto{\pgfqpoint{1.822848in}{2.940786in}}{\pgfqpoint{1.812249in}{2.945176in}}{\pgfqpoint{1.801199in}{2.945176in}}%
\pgfpathcurveto{\pgfqpoint{1.790149in}{2.945176in}}{\pgfqpoint{1.779550in}{2.940786in}}{\pgfqpoint{1.771736in}{2.932972in}}%
\pgfpathcurveto{\pgfqpoint{1.763922in}{2.925158in}}{\pgfqpoint{1.759532in}{2.914559in}}{\pgfqpoint{1.759532in}{2.903509in}}%
\pgfpathcurveto{\pgfqpoint{1.759532in}{2.892459in}}{\pgfqpoint{1.763922in}{2.881860in}}{\pgfqpoint{1.771736in}{2.874046in}}%
\pgfpathcurveto{\pgfqpoint{1.779550in}{2.866233in}}{\pgfqpoint{1.790149in}{2.861842in}}{\pgfqpoint{1.801199in}{2.861842in}}%
\pgfpathclose%
\pgfusepath{stroke,fill}%
\end{pgfscope}%
\begin{pgfscope}%
\pgfpathrectangle{\pgfqpoint{0.787074in}{0.548769in}}{\pgfqpoint{5.062926in}{3.102590in}}%
\pgfusepath{clip}%
\pgfsetbuttcap%
\pgfsetroundjoin%
\definecolor{currentfill}{rgb}{0.121569,0.466667,0.705882}%
\pgfsetfillcolor{currentfill}%
\pgfsetlinewidth{1.003750pt}%
\definecolor{currentstroke}{rgb}{0.121569,0.466667,0.705882}%
\pgfsetstrokecolor{currentstroke}%
\pgfsetdash{}{0pt}%
\pgfpathmoveto{\pgfqpoint{1.061082in}{0.668329in}}%
\pgfpathcurveto{\pgfqpoint{1.072132in}{0.668329in}}{\pgfqpoint{1.082731in}{0.672720in}}{\pgfqpoint{1.090544in}{0.680533in}}%
\pgfpathcurveto{\pgfqpoint{1.098358in}{0.688347in}}{\pgfqpoint{1.102748in}{0.698946in}}{\pgfqpoint{1.102748in}{0.709996in}}%
\pgfpathcurveto{\pgfqpoint{1.102748in}{0.721046in}}{\pgfqpoint{1.098358in}{0.731645in}}{\pgfqpoint{1.090544in}{0.739459in}}%
\pgfpathcurveto{\pgfqpoint{1.082731in}{0.747272in}}{\pgfqpoint{1.072132in}{0.751663in}}{\pgfqpoint{1.061082in}{0.751663in}}%
\pgfpathcurveto{\pgfqpoint{1.050031in}{0.751663in}}{\pgfqpoint{1.039432in}{0.747272in}}{\pgfqpoint{1.031619in}{0.739459in}}%
\pgfpathcurveto{\pgfqpoint{1.023805in}{0.731645in}}{\pgfqpoint{1.019415in}{0.721046in}}{\pgfqpoint{1.019415in}{0.709996in}}%
\pgfpathcurveto{\pgfqpoint{1.019415in}{0.698946in}}{\pgfqpoint{1.023805in}{0.688347in}}{\pgfqpoint{1.031619in}{0.680533in}}%
\pgfpathcurveto{\pgfqpoint{1.039432in}{0.672720in}}{\pgfqpoint{1.050031in}{0.668329in}}{\pgfqpoint{1.061082in}{0.668329in}}%
\pgfpathclose%
\pgfusepath{stroke,fill}%
\end{pgfscope}%
\begin{pgfscope}%
\pgfpathrectangle{\pgfqpoint{0.787074in}{0.548769in}}{\pgfqpoint{5.062926in}{3.102590in}}%
\pgfusepath{clip}%
\pgfsetbuttcap%
\pgfsetroundjoin%
\definecolor{currentfill}{rgb}{1.000000,0.498039,0.054902}%
\pgfsetfillcolor{currentfill}%
\pgfsetlinewidth{1.003750pt}%
\definecolor{currentstroke}{rgb}{1.000000,0.498039,0.054902}%
\pgfsetstrokecolor{currentstroke}%
\pgfsetdash{}{0pt}%
\pgfpathmoveto{\pgfqpoint{1.362136in}{3.008532in}}%
\pgfpathcurveto{\pgfqpoint{1.373186in}{3.008532in}}{\pgfqpoint{1.383785in}{3.012923in}}{\pgfqpoint{1.391599in}{3.020736in}}%
\pgfpathcurveto{\pgfqpoint{1.399413in}{3.028550in}}{\pgfqpoint{1.403803in}{3.039149in}}{\pgfqpoint{1.403803in}{3.050199in}}%
\pgfpathcurveto{\pgfqpoint{1.403803in}{3.061249in}}{\pgfqpoint{1.399413in}{3.071848in}}{\pgfqpoint{1.391599in}{3.079662in}}%
\pgfpathcurveto{\pgfqpoint{1.383785in}{3.087476in}}{\pgfqpoint{1.373186in}{3.091866in}}{\pgfqpoint{1.362136in}{3.091866in}}%
\pgfpathcurveto{\pgfqpoint{1.351086in}{3.091866in}}{\pgfqpoint{1.340487in}{3.087476in}}{\pgfqpoint{1.332673in}{3.079662in}}%
\pgfpathcurveto{\pgfqpoint{1.324860in}{3.071848in}}{\pgfqpoint{1.320470in}{3.061249in}}{\pgfqpoint{1.320470in}{3.050199in}}%
\pgfpathcurveto{\pgfqpoint{1.320470in}{3.039149in}}{\pgfqpoint{1.324860in}{3.028550in}}{\pgfqpoint{1.332673in}{3.020736in}}%
\pgfpathcurveto{\pgfqpoint{1.340487in}{3.012923in}}{\pgfqpoint{1.351086in}{3.008532in}}{\pgfqpoint{1.362136in}{3.008532in}}%
\pgfpathclose%
\pgfusepath{stroke,fill}%
\end{pgfscope}%
\begin{pgfscope}%
\pgfpathrectangle{\pgfqpoint{0.787074in}{0.548769in}}{\pgfqpoint{5.062926in}{3.102590in}}%
\pgfusepath{clip}%
\pgfsetbuttcap%
\pgfsetroundjoin%
\definecolor{currentfill}{rgb}{1.000000,0.498039,0.054902}%
\pgfsetfillcolor{currentfill}%
\pgfsetlinewidth{1.003750pt}%
\definecolor{currentstroke}{rgb}{1.000000,0.498039,0.054902}%
\pgfsetstrokecolor{currentstroke}%
\pgfsetdash{}{0pt}%
\pgfpathmoveto{\pgfqpoint{1.396317in}{2.645063in}}%
\pgfpathcurveto{\pgfqpoint{1.407368in}{2.645063in}}{\pgfqpoint{1.417967in}{2.649453in}}{\pgfqpoint{1.425780in}{2.657266in}}%
\pgfpathcurveto{\pgfqpoint{1.433594in}{2.665080in}}{\pgfqpoint{1.437984in}{2.675679in}}{\pgfqpoint{1.437984in}{2.686729in}}%
\pgfpathcurveto{\pgfqpoint{1.437984in}{2.697779in}}{\pgfqpoint{1.433594in}{2.708378in}}{\pgfqpoint{1.425780in}{2.716192in}}%
\pgfpathcurveto{\pgfqpoint{1.417967in}{2.724006in}}{\pgfqpoint{1.407368in}{2.728396in}}{\pgfqpoint{1.396317in}{2.728396in}}%
\pgfpathcurveto{\pgfqpoint{1.385267in}{2.728396in}}{\pgfqpoint{1.374668in}{2.724006in}}{\pgfqpoint{1.366855in}{2.716192in}}%
\pgfpathcurveto{\pgfqpoint{1.359041in}{2.708378in}}{\pgfqpoint{1.354651in}{2.697779in}}{\pgfqpoint{1.354651in}{2.686729in}}%
\pgfpathcurveto{\pgfqpoint{1.354651in}{2.675679in}}{\pgfqpoint{1.359041in}{2.665080in}}{\pgfqpoint{1.366855in}{2.657266in}}%
\pgfpathcurveto{\pgfqpoint{1.374668in}{2.649453in}}{\pgfqpoint{1.385267in}{2.645063in}}{\pgfqpoint{1.396317in}{2.645063in}}%
\pgfpathclose%
\pgfusepath{stroke,fill}%
\end{pgfscope}%
\begin{pgfscope}%
\pgfpathrectangle{\pgfqpoint{0.787074in}{0.548769in}}{\pgfqpoint{5.062926in}{3.102590in}}%
\pgfusepath{clip}%
\pgfsetbuttcap%
\pgfsetroundjoin%
\definecolor{currentfill}{rgb}{1.000000,0.498039,0.054902}%
\pgfsetfillcolor{currentfill}%
\pgfsetlinewidth{1.003750pt}%
\definecolor{currentstroke}{rgb}{1.000000,0.498039,0.054902}%
\pgfsetstrokecolor{currentstroke}%
\pgfsetdash{}{0pt}%
\pgfpathmoveto{\pgfqpoint{1.746302in}{3.140866in}}%
\pgfpathcurveto{\pgfqpoint{1.757352in}{3.140866in}}{\pgfqpoint{1.767951in}{3.145257in}}{\pgfqpoint{1.775764in}{3.153070in}}%
\pgfpathcurveto{\pgfqpoint{1.783578in}{3.160884in}}{\pgfqpoint{1.787968in}{3.171483in}}{\pgfqpoint{1.787968in}{3.182533in}}%
\pgfpathcurveto{\pgfqpoint{1.787968in}{3.193583in}}{\pgfqpoint{1.783578in}{3.204182in}}{\pgfqpoint{1.775764in}{3.211996in}}%
\pgfpathcurveto{\pgfqpoint{1.767951in}{3.219809in}}{\pgfqpoint{1.757352in}{3.224200in}}{\pgfqpoint{1.746302in}{3.224200in}}%
\pgfpathcurveto{\pgfqpoint{1.735251in}{3.224200in}}{\pgfqpoint{1.724652in}{3.219809in}}{\pgfqpoint{1.716839in}{3.211996in}}%
\pgfpathcurveto{\pgfqpoint{1.709025in}{3.204182in}}{\pgfqpoint{1.704635in}{3.193583in}}{\pgfqpoint{1.704635in}{3.182533in}}%
\pgfpathcurveto{\pgfqpoint{1.704635in}{3.171483in}}{\pgfqpoint{1.709025in}{3.160884in}}{\pgfqpoint{1.716839in}{3.153070in}}%
\pgfpathcurveto{\pgfqpoint{1.724652in}{3.145257in}}{\pgfqpoint{1.735251in}{3.140866in}}{\pgfqpoint{1.746302in}{3.140866in}}%
\pgfpathclose%
\pgfusepath{stroke,fill}%
\end{pgfscope}%
\begin{pgfscope}%
\pgfpathrectangle{\pgfqpoint{0.787074in}{0.548769in}}{\pgfqpoint{5.062926in}{3.102590in}}%
\pgfusepath{clip}%
\pgfsetbuttcap%
\pgfsetroundjoin%
\definecolor{currentfill}{rgb}{1.000000,0.498039,0.054902}%
\pgfsetfillcolor{currentfill}%
\pgfsetlinewidth{1.003750pt}%
\definecolor{currentstroke}{rgb}{1.000000,0.498039,0.054902}%
\pgfsetstrokecolor{currentstroke}%
\pgfsetdash{}{0pt}%
\pgfpathmoveto{\pgfqpoint{1.637358in}{2.974499in}}%
\pgfpathcurveto{\pgfqpoint{1.648409in}{2.974499in}}{\pgfqpoint{1.659008in}{2.978889in}}{\pgfqpoint{1.666821in}{2.986703in}}%
\pgfpathcurveto{\pgfqpoint{1.674635in}{2.994517in}}{\pgfqpoint{1.679025in}{3.005116in}}{\pgfqpoint{1.679025in}{3.016166in}}%
\pgfpathcurveto{\pgfqpoint{1.679025in}{3.027216in}}{\pgfqpoint{1.674635in}{3.037815in}}{\pgfqpoint{1.666821in}{3.045629in}}%
\pgfpathcurveto{\pgfqpoint{1.659008in}{3.053442in}}{\pgfqpoint{1.648409in}{3.057833in}}{\pgfqpoint{1.637358in}{3.057833in}}%
\pgfpathcurveto{\pgfqpoint{1.626308in}{3.057833in}}{\pgfqpoint{1.615709in}{3.053442in}}{\pgfqpoint{1.607896in}{3.045629in}}%
\pgfpathcurveto{\pgfqpoint{1.600082in}{3.037815in}}{\pgfqpoint{1.595692in}{3.027216in}}{\pgfqpoint{1.595692in}{3.016166in}}%
\pgfpathcurveto{\pgfqpoint{1.595692in}{3.005116in}}{\pgfqpoint{1.600082in}{2.994517in}}{\pgfqpoint{1.607896in}{2.986703in}}%
\pgfpathcurveto{\pgfqpoint{1.615709in}{2.978889in}}{\pgfqpoint{1.626308in}{2.974499in}}{\pgfqpoint{1.637358in}{2.974499in}}%
\pgfpathclose%
\pgfusepath{stroke,fill}%
\end{pgfscope}%
\begin{pgfscope}%
\pgfpathrectangle{\pgfqpoint{0.787074in}{0.548769in}}{\pgfqpoint{5.062926in}{3.102590in}}%
\pgfusepath{clip}%
\pgfsetbuttcap%
\pgfsetroundjoin%
\definecolor{currentfill}{rgb}{1.000000,0.498039,0.054902}%
\pgfsetfillcolor{currentfill}%
\pgfsetlinewidth{1.003750pt}%
\definecolor{currentstroke}{rgb}{1.000000,0.498039,0.054902}%
\pgfsetstrokecolor{currentstroke}%
\pgfsetdash{}{0pt}%
\pgfpathmoveto{\pgfqpoint{1.443128in}{1.902652in}}%
\pgfpathcurveto{\pgfqpoint{1.454178in}{1.902652in}}{\pgfqpoint{1.464777in}{1.907042in}}{\pgfqpoint{1.472591in}{1.914856in}}%
\pgfpathcurveto{\pgfqpoint{1.480404in}{1.922669in}}{\pgfqpoint{1.484795in}{1.933268in}}{\pgfqpoint{1.484795in}{1.944318in}}%
\pgfpathcurveto{\pgfqpoint{1.484795in}{1.955368in}}{\pgfqpoint{1.480404in}{1.965967in}}{\pgfqpoint{1.472591in}{1.973781in}}%
\pgfpathcurveto{\pgfqpoint{1.464777in}{1.981595in}}{\pgfqpoint{1.454178in}{1.985985in}}{\pgfqpoint{1.443128in}{1.985985in}}%
\pgfpathcurveto{\pgfqpoint{1.432078in}{1.985985in}}{\pgfqpoint{1.421479in}{1.981595in}}{\pgfqpoint{1.413665in}{1.973781in}}%
\pgfpathcurveto{\pgfqpoint{1.405851in}{1.965967in}}{\pgfqpoint{1.401461in}{1.955368in}}{\pgfqpoint{1.401461in}{1.944318in}}%
\pgfpathcurveto{\pgfqpoint{1.401461in}{1.933268in}}{\pgfqpoint{1.405851in}{1.922669in}}{\pgfqpoint{1.413665in}{1.914856in}}%
\pgfpathcurveto{\pgfqpoint{1.421479in}{1.907042in}}{\pgfqpoint{1.432078in}{1.902652in}}{\pgfqpoint{1.443128in}{1.902652in}}%
\pgfpathclose%
\pgfusepath{stroke,fill}%
\end{pgfscope}%
\begin{pgfscope}%
\pgfpathrectangle{\pgfqpoint{0.787074in}{0.548769in}}{\pgfqpoint{5.062926in}{3.102590in}}%
\pgfusepath{clip}%
\pgfsetbuttcap%
\pgfsetroundjoin%
\definecolor{currentfill}{rgb}{1.000000,0.498039,0.054902}%
\pgfsetfillcolor{currentfill}%
\pgfsetlinewidth{1.003750pt}%
\definecolor{currentstroke}{rgb}{1.000000,0.498039,0.054902}%
\pgfsetstrokecolor{currentstroke}%
\pgfsetdash{}{0pt}%
\pgfpathmoveto{\pgfqpoint{1.425151in}{1.942448in}}%
\pgfpathcurveto{\pgfqpoint{1.436201in}{1.942448in}}{\pgfqpoint{1.446800in}{1.946838in}}{\pgfqpoint{1.454614in}{1.954652in}}%
\pgfpathcurveto{\pgfqpoint{1.462428in}{1.962466in}}{\pgfqpoint{1.466818in}{1.973065in}}{\pgfqpoint{1.466818in}{1.984115in}}%
\pgfpathcurveto{\pgfqpoint{1.466818in}{1.995165in}}{\pgfqpoint{1.462428in}{2.005764in}}{\pgfqpoint{1.454614in}{2.013578in}}%
\pgfpathcurveto{\pgfqpoint{1.446800in}{2.021391in}}{\pgfqpoint{1.436201in}{2.025781in}}{\pgfqpoint{1.425151in}{2.025781in}}%
\pgfpathcurveto{\pgfqpoint{1.414101in}{2.025781in}}{\pgfqpoint{1.403502in}{2.021391in}}{\pgfqpoint{1.395688in}{2.013578in}}%
\pgfpathcurveto{\pgfqpoint{1.387875in}{2.005764in}}{\pgfqpoint{1.383484in}{1.995165in}}{\pgfqpoint{1.383484in}{1.984115in}}%
\pgfpathcurveto{\pgfqpoint{1.383484in}{1.973065in}}{\pgfqpoint{1.387875in}{1.962466in}}{\pgfqpoint{1.395688in}{1.954652in}}%
\pgfpathcurveto{\pgfqpoint{1.403502in}{1.946838in}}{\pgfqpoint{1.414101in}{1.942448in}}{\pgfqpoint{1.425151in}{1.942448in}}%
\pgfpathclose%
\pgfusepath{stroke,fill}%
\end{pgfscope}%
\begin{pgfscope}%
\pgfpathrectangle{\pgfqpoint{0.787074in}{0.548769in}}{\pgfqpoint{5.062926in}{3.102590in}}%
\pgfusepath{clip}%
\pgfsetbuttcap%
\pgfsetroundjoin%
\definecolor{currentfill}{rgb}{1.000000,0.498039,0.054902}%
\pgfsetfillcolor{currentfill}%
\pgfsetlinewidth{1.003750pt}%
\definecolor{currentstroke}{rgb}{1.000000,0.498039,0.054902}%
\pgfsetstrokecolor{currentstroke}%
\pgfsetdash{}{0pt}%
\pgfpathmoveto{\pgfqpoint{1.466491in}{2.924069in}}%
\pgfpathcurveto{\pgfqpoint{1.477541in}{2.924069in}}{\pgfqpoint{1.488140in}{2.928459in}}{\pgfqpoint{1.495953in}{2.936273in}}%
\pgfpathcurveto{\pgfqpoint{1.503767in}{2.944086in}}{\pgfqpoint{1.508157in}{2.954685in}}{\pgfqpoint{1.508157in}{2.965736in}}%
\pgfpathcurveto{\pgfqpoint{1.508157in}{2.976786in}}{\pgfqpoint{1.503767in}{2.987385in}}{\pgfqpoint{1.495953in}{2.995198in}}%
\pgfpathcurveto{\pgfqpoint{1.488140in}{3.003012in}}{\pgfqpoint{1.477541in}{3.007402in}}{\pgfqpoint{1.466491in}{3.007402in}}%
\pgfpathcurveto{\pgfqpoint{1.455441in}{3.007402in}}{\pgfqpoint{1.444842in}{3.003012in}}{\pgfqpoint{1.437028in}{2.995198in}}%
\pgfpathcurveto{\pgfqpoint{1.429214in}{2.987385in}}{\pgfqpoint{1.424824in}{2.976786in}}{\pgfqpoint{1.424824in}{2.965736in}}%
\pgfpathcurveto{\pgfqpoint{1.424824in}{2.954685in}}{\pgfqpoint{1.429214in}{2.944086in}}{\pgfqpoint{1.437028in}{2.936273in}}%
\pgfpathcurveto{\pgfqpoint{1.444842in}{2.928459in}}{\pgfqpoint{1.455441in}{2.924069in}}{\pgfqpoint{1.466491in}{2.924069in}}%
\pgfpathclose%
\pgfusepath{stroke,fill}%
\end{pgfscope}%
\begin{pgfscope}%
\pgfpathrectangle{\pgfqpoint{0.787074in}{0.548769in}}{\pgfqpoint{5.062926in}{3.102590in}}%
\pgfusepath{clip}%
\pgfsetbuttcap%
\pgfsetroundjoin%
\definecolor{currentfill}{rgb}{0.121569,0.466667,0.705882}%
\pgfsetfillcolor{currentfill}%
\pgfsetlinewidth{1.003750pt}%
\definecolor{currentstroke}{rgb}{0.121569,0.466667,0.705882}%
\pgfsetstrokecolor{currentstroke}%
\pgfsetdash{}{0pt}%
\pgfpathmoveto{\pgfqpoint{1.260606in}{2.510157in}}%
\pgfpathcurveto{\pgfqpoint{1.271656in}{2.510157in}}{\pgfqpoint{1.282255in}{2.514547in}}{\pgfqpoint{1.290068in}{2.522361in}}%
\pgfpathcurveto{\pgfqpoint{1.297882in}{2.530174in}}{\pgfqpoint{1.302272in}{2.540773in}}{\pgfqpoint{1.302272in}{2.551824in}}%
\pgfpathcurveto{\pgfqpoint{1.302272in}{2.562874in}}{\pgfqpoint{1.297882in}{2.573473in}}{\pgfqpoint{1.290068in}{2.581286in}}%
\pgfpathcurveto{\pgfqpoint{1.282255in}{2.589100in}}{\pgfqpoint{1.271656in}{2.593490in}}{\pgfqpoint{1.260606in}{2.593490in}}%
\pgfpathcurveto{\pgfqpoint{1.249556in}{2.593490in}}{\pgfqpoint{1.238957in}{2.589100in}}{\pgfqpoint{1.231143in}{2.581286in}}%
\pgfpathcurveto{\pgfqpoint{1.223329in}{2.573473in}}{\pgfqpoint{1.218939in}{2.562874in}}{\pgfqpoint{1.218939in}{2.551824in}}%
\pgfpathcurveto{\pgfqpoint{1.218939in}{2.540773in}}{\pgfqpoint{1.223329in}{2.530174in}}{\pgfqpoint{1.231143in}{2.522361in}}%
\pgfpathcurveto{\pgfqpoint{1.238957in}{2.514547in}}{\pgfqpoint{1.249556in}{2.510157in}}{\pgfqpoint{1.260606in}{2.510157in}}%
\pgfpathclose%
\pgfusepath{stroke,fill}%
\end{pgfscope}%
\begin{pgfscope}%
\pgfpathrectangle{\pgfqpoint{0.787074in}{0.548769in}}{\pgfqpoint{5.062926in}{3.102590in}}%
\pgfusepath{clip}%
\pgfsetbuttcap%
\pgfsetroundjoin%
\definecolor{currentfill}{rgb}{0.121569,0.466667,0.705882}%
\pgfsetfillcolor{currentfill}%
\pgfsetlinewidth{1.003750pt}%
\definecolor{currentstroke}{rgb}{0.121569,0.466667,0.705882}%
\pgfsetstrokecolor{currentstroke}%
\pgfsetdash{}{0pt}%
\pgfpathmoveto{\pgfqpoint{5.619867in}{1.380136in}}%
\pgfpathcurveto{\pgfqpoint{5.630917in}{1.380136in}}{\pgfqpoint{5.641516in}{1.384527in}}{\pgfqpoint{5.649330in}{1.392340in}}%
\pgfpathcurveto{\pgfqpoint{5.657143in}{1.400154in}}{\pgfqpoint{5.661534in}{1.410753in}}{\pgfqpoint{5.661534in}{1.421803in}}%
\pgfpathcurveto{\pgfqpoint{5.661534in}{1.432853in}}{\pgfqpoint{5.657143in}{1.443452in}}{\pgfqpoint{5.649330in}{1.451266in}}%
\pgfpathcurveto{\pgfqpoint{5.641516in}{1.459079in}}{\pgfqpoint{5.630917in}{1.463470in}}{\pgfqpoint{5.619867in}{1.463470in}}%
\pgfpathcurveto{\pgfqpoint{5.608817in}{1.463470in}}{\pgfqpoint{5.598218in}{1.459079in}}{\pgfqpoint{5.590404in}{1.451266in}}%
\pgfpathcurveto{\pgfqpoint{5.582591in}{1.443452in}}{\pgfqpoint{5.578200in}{1.432853in}}{\pgfqpoint{5.578200in}{1.421803in}}%
\pgfpathcurveto{\pgfqpoint{5.578200in}{1.410753in}}{\pgfqpoint{5.582591in}{1.400154in}}{\pgfqpoint{5.590404in}{1.392340in}}%
\pgfpathcurveto{\pgfqpoint{5.598218in}{1.384527in}}{\pgfqpoint{5.608817in}{1.380136in}}{\pgfqpoint{5.619867in}{1.380136in}}%
\pgfpathclose%
\pgfusepath{stroke,fill}%
\end{pgfscope}%
\begin{pgfscope}%
\pgfpathrectangle{\pgfqpoint{0.787074in}{0.548769in}}{\pgfqpoint{5.062926in}{3.102590in}}%
\pgfusepath{clip}%
\pgfsetbuttcap%
\pgfsetroundjoin%
\definecolor{currentfill}{rgb}{1.000000,0.498039,0.054902}%
\pgfsetfillcolor{currentfill}%
\pgfsetlinewidth{1.003750pt}%
\definecolor{currentstroke}{rgb}{1.000000,0.498039,0.054902}%
\pgfsetstrokecolor{currentstroke}%
\pgfsetdash{}{0pt}%
\pgfpathmoveto{\pgfqpoint{1.421876in}{2.854158in}}%
\pgfpathcurveto{\pgfqpoint{1.432926in}{2.854158in}}{\pgfqpoint{1.443525in}{2.858548in}}{\pgfqpoint{1.451339in}{2.866362in}}%
\pgfpathcurveto{\pgfqpoint{1.459153in}{2.874175in}}{\pgfqpoint{1.463543in}{2.884774in}}{\pgfqpoint{1.463543in}{2.895825in}}%
\pgfpathcurveto{\pgfqpoint{1.463543in}{2.906875in}}{\pgfqpoint{1.459153in}{2.917474in}}{\pgfqpoint{1.451339in}{2.925287in}}%
\pgfpathcurveto{\pgfqpoint{1.443525in}{2.933101in}}{\pgfqpoint{1.432926in}{2.937491in}}{\pgfqpoint{1.421876in}{2.937491in}}%
\pgfpathcurveto{\pgfqpoint{1.410826in}{2.937491in}}{\pgfqpoint{1.400227in}{2.933101in}}{\pgfqpoint{1.392414in}{2.925287in}}%
\pgfpathcurveto{\pgfqpoint{1.384600in}{2.917474in}}{\pgfqpoint{1.380210in}{2.906875in}}{\pgfqpoint{1.380210in}{2.895825in}}%
\pgfpathcurveto{\pgfqpoint{1.380210in}{2.884774in}}{\pgfqpoint{1.384600in}{2.874175in}}{\pgfqpoint{1.392414in}{2.866362in}}%
\pgfpathcurveto{\pgfqpoint{1.400227in}{2.858548in}}{\pgfqpoint{1.410826in}{2.854158in}}{\pgfqpoint{1.421876in}{2.854158in}}%
\pgfpathclose%
\pgfusepath{stroke,fill}%
\end{pgfscope}%
\begin{pgfscope}%
\pgfpathrectangle{\pgfqpoint{0.787074in}{0.548769in}}{\pgfqpoint{5.062926in}{3.102590in}}%
\pgfusepath{clip}%
\pgfsetbuttcap%
\pgfsetroundjoin%
\definecolor{currentfill}{rgb}{0.121569,0.466667,0.705882}%
\pgfsetfillcolor{currentfill}%
\pgfsetlinewidth{1.003750pt}%
\definecolor{currentstroke}{rgb}{0.121569,0.466667,0.705882}%
\pgfsetstrokecolor{currentstroke}%
\pgfsetdash{}{0pt}%
\pgfpathmoveto{\pgfqpoint{1.226925in}{0.648318in}}%
\pgfpathcurveto{\pgfqpoint{1.237975in}{0.648318in}}{\pgfqpoint{1.248574in}{0.652709in}}{\pgfqpoint{1.256388in}{0.660522in}}%
\pgfpathcurveto{\pgfqpoint{1.264202in}{0.668336in}}{\pgfqpoint{1.268592in}{0.678935in}}{\pgfqpoint{1.268592in}{0.689985in}}%
\pgfpathcurveto{\pgfqpoint{1.268592in}{0.701035in}}{\pgfqpoint{1.264202in}{0.711634in}}{\pgfqpoint{1.256388in}{0.719448in}}%
\pgfpathcurveto{\pgfqpoint{1.248574in}{0.727262in}}{\pgfqpoint{1.237975in}{0.731652in}}{\pgfqpoint{1.226925in}{0.731652in}}%
\pgfpathcurveto{\pgfqpoint{1.215875in}{0.731652in}}{\pgfqpoint{1.205276in}{0.727262in}}{\pgfqpoint{1.197463in}{0.719448in}}%
\pgfpathcurveto{\pgfqpoint{1.189649in}{0.711634in}}{\pgfqpoint{1.185259in}{0.701035in}}{\pgfqpoint{1.185259in}{0.689985in}}%
\pgfpathcurveto{\pgfqpoint{1.185259in}{0.678935in}}{\pgfqpoint{1.189649in}{0.668336in}}{\pgfqpoint{1.197463in}{0.660522in}}%
\pgfpathcurveto{\pgfqpoint{1.205276in}{0.652709in}}{\pgfqpoint{1.215875in}{0.648318in}}{\pgfqpoint{1.226925in}{0.648318in}}%
\pgfpathclose%
\pgfusepath{stroke,fill}%
\end{pgfscope}%
\begin{pgfscope}%
\pgfpathrectangle{\pgfqpoint{0.787074in}{0.548769in}}{\pgfqpoint{5.062926in}{3.102590in}}%
\pgfusepath{clip}%
\pgfsetbuttcap%
\pgfsetroundjoin%
\definecolor{currentfill}{rgb}{0.121569,0.466667,0.705882}%
\pgfsetfillcolor{currentfill}%
\pgfsetlinewidth{1.003750pt}%
\definecolor{currentstroke}{rgb}{0.121569,0.466667,0.705882}%
\pgfsetstrokecolor{currentstroke}%
\pgfsetdash{}{0pt}%
\pgfpathmoveto{\pgfqpoint{1.621377in}{1.283193in}}%
\pgfpathcurveto{\pgfqpoint{1.632428in}{1.283193in}}{\pgfqpoint{1.643027in}{1.287584in}}{\pgfqpoint{1.650840in}{1.295397in}}%
\pgfpathcurveto{\pgfqpoint{1.658654in}{1.303211in}}{\pgfqpoint{1.663044in}{1.313810in}}{\pgfqpoint{1.663044in}{1.324860in}}%
\pgfpathcurveto{\pgfqpoint{1.663044in}{1.335910in}}{\pgfqpoint{1.658654in}{1.346509in}}{\pgfqpoint{1.650840in}{1.354323in}}%
\pgfpathcurveto{\pgfqpoint{1.643027in}{1.362137in}}{\pgfqpoint{1.632428in}{1.366527in}}{\pgfqpoint{1.621377in}{1.366527in}}%
\pgfpathcurveto{\pgfqpoint{1.610327in}{1.366527in}}{\pgfqpoint{1.599728in}{1.362137in}}{\pgfqpoint{1.591915in}{1.354323in}}%
\pgfpathcurveto{\pgfqpoint{1.584101in}{1.346509in}}{\pgfqpoint{1.579711in}{1.335910in}}{\pgfqpoint{1.579711in}{1.324860in}}%
\pgfpathcurveto{\pgfqpoint{1.579711in}{1.313810in}}{\pgfqpoint{1.584101in}{1.303211in}}{\pgfqpoint{1.591915in}{1.295397in}}%
\pgfpathcurveto{\pgfqpoint{1.599728in}{1.287584in}}{\pgfqpoint{1.610327in}{1.283193in}}{\pgfqpoint{1.621377in}{1.283193in}}%
\pgfpathclose%
\pgfusepath{stroke,fill}%
\end{pgfscope}%
\begin{pgfscope}%
\pgfpathrectangle{\pgfqpoint{0.787074in}{0.548769in}}{\pgfqpoint{5.062926in}{3.102590in}}%
\pgfusepath{clip}%
\pgfsetbuttcap%
\pgfsetroundjoin%
\definecolor{currentfill}{rgb}{1.000000,0.498039,0.054902}%
\pgfsetfillcolor{currentfill}%
\pgfsetlinewidth{1.003750pt}%
\definecolor{currentstroke}{rgb}{1.000000,0.498039,0.054902}%
\pgfsetstrokecolor{currentstroke}%
\pgfsetdash{}{0pt}%
\pgfpathmoveto{\pgfqpoint{1.224502in}{2.443414in}}%
\pgfpathcurveto{\pgfqpoint{1.235552in}{2.443414in}}{\pgfqpoint{1.246151in}{2.447804in}}{\pgfqpoint{1.253965in}{2.455618in}}%
\pgfpathcurveto{\pgfqpoint{1.261778in}{2.463432in}}{\pgfqpoint{1.266169in}{2.474031in}}{\pgfqpoint{1.266169in}{2.485081in}}%
\pgfpathcurveto{\pgfqpoint{1.266169in}{2.496131in}}{\pgfqpoint{1.261778in}{2.506730in}}{\pgfqpoint{1.253965in}{2.514544in}}%
\pgfpathcurveto{\pgfqpoint{1.246151in}{2.522357in}}{\pgfqpoint{1.235552in}{2.526747in}}{\pgfqpoint{1.224502in}{2.526747in}}%
\pgfpathcurveto{\pgfqpoint{1.213452in}{2.526747in}}{\pgfqpoint{1.202853in}{2.522357in}}{\pgfqpoint{1.195039in}{2.514544in}}%
\pgfpathcurveto{\pgfqpoint{1.187226in}{2.506730in}}{\pgfqpoint{1.182835in}{2.496131in}}{\pgfqpoint{1.182835in}{2.485081in}}%
\pgfpathcurveto{\pgfqpoint{1.182835in}{2.474031in}}{\pgfqpoint{1.187226in}{2.463432in}}{\pgfqpoint{1.195039in}{2.455618in}}%
\pgfpathcurveto{\pgfqpoint{1.202853in}{2.447804in}}{\pgfqpoint{1.213452in}{2.443414in}}{\pgfqpoint{1.224502in}{2.443414in}}%
\pgfpathclose%
\pgfusepath{stroke,fill}%
\end{pgfscope}%
\begin{pgfscope}%
\pgfpathrectangle{\pgfqpoint{0.787074in}{0.548769in}}{\pgfqpoint{5.062926in}{3.102590in}}%
\pgfusepath{clip}%
\pgfsetbuttcap%
\pgfsetroundjoin%
\definecolor{currentfill}{rgb}{0.121569,0.466667,0.705882}%
\pgfsetfillcolor{currentfill}%
\pgfsetlinewidth{1.003750pt}%
\definecolor{currentstroke}{rgb}{0.121569,0.466667,0.705882}%
\pgfsetstrokecolor{currentstroke}%
\pgfsetdash{}{0pt}%
\pgfpathmoveto{\pgfqpoint{2.560468in}{1.745269in}}%
\pgfpathcurveto{\pgfqpoint{2.571518in}{1.745269in}}{\pgfqpoint{2.582117in}{1.749660in}}{\pgfqpoint{2.589931in}{1.757473in}}%
\pgfpathcurveto{\pgfqpoint{2.597744in}{1.765287in}}{\pgfqpoint{2.602135in}{1.775886in}}{\pgfqpoint{2.602135in}{1.786936in}}%
\pgfpathcurveto{\pgfqpoint{2.602135in}{1.797986in}}{\pgfqpoint{2.597744in}{1.808585in}}{\pgfqpoint{2.589931in}{1.816399in}}%
\pgfpathcurveto{\pgfqpoint{2.582117in}{1.824212in}}{\pgfqpoint{2.571518in}{1.828603in}}{\pgfqpoint{2.560468in}{1.828603in}}%
\pgfpathcurveto{\pgfqpoint{2.549418in}{1.828603in}}{\pgfqpoint{2.538819in}{1.824212in}}{\pgfqpoint{2.531005in}{1.816399in}}%
\pgfpathcurveto{\pgfqpoint{2.523192in}{1.808585in}}{\pgfqpoint{2.518801in}{1.797986in}}{\pgfqpoint{2.518801in}{1.786936in}}%
\pgfpathcurveto{\pgfqpoint{2.518801in}{1.775886in}}{\pgfqpoint{2.523192in}{1.765287in}}{\pgfqpoint{2.531005in}{1.757473in}}%
\pgfpathcurveto{\pgfqpoint{2.538819in}{1.749660in}}{\pgfqpoint{2.549418in}{1.745269in}}{\pgfqpoint{2.560468in}{1.745269in}}%
\pgfpathclose%
\pgfusepath{stroke,fill}%
\end{pgfscope}%
\begin{pgfscope}%
\pgfpathrectangle{\pgfqpoint{0.787074in}{0.548769in}}{\pgfqpoint{5.062926in}{3.102590in}}%
\pgfusepath{clip}%
\pgfsetbuttcap%
\pgfsetroundjoin%
\definecolor{currentfill}{rgb}{0.839216,0.152941,0.156863}%
\pgfsetfillcolor{currentfill}%
\pgfsetlinewidth{1.003750pt}%
\definecolor{currentstroke}{rgb}{0.839216,0.152941,0.156863}%
\pgfsetstrokecolor{currentstroke}%
\pgfsetdash{}{0pt}%
\pgfpathmoveto{\pgfqpoint{2.119240in}{3.066498in}}%
\pgfpathcurveto{\pgfqpoint{2.130290in}{3.066498in}}{\pgfqpoint{2.140889in}{3.070888in}}{\pgfqpoint{2.148703in}{3.078701in}}%
\pgfpathcurveto{\pgfqpoint{2.156516in}{3.086515in}}{\pgfqpoint{2.160907in}{3.097114in}}{\pgfqpoint{2.160907in}{3.108164in}}%
\pgfpathcurveto{\pgfqpoint{2.160907in}{3.119214in}}{\pgfqpoint{2.156516in}{3.129813in}}{\pgfqpoint{2.148703in}{3.137627in}}%
\pgfpathcurveto{\pgfqpoint{2.140889in}{3.145441in}}{\pgfqpoint{2.130290in}{3.149831in}}{\pgfqpoint{2.119240in}{3.149831in}}%
\pgfpathcurveto{\pgfqpoint{2.108190in}{3.149831in}}{\pgfqpoint{2.097591in}{3.145441in}}{\pgfqpoint{2.089777in}{3.137627in}}%
\pgfpathcurveto{\pgfqpoint{2.081964in}{3.129813in}}{\pgfqpoint{2.077573in}{3.119214in}}{\pgfqpoint{2.077573in}{3.108164in}}%
\pgfpathcurveto{\pgfqpoint{2.077573in}{3.097114in}}{\pgfqpoint{2.081964in}{3.086515in}}{\pgfqpoint{2.089777in}{3.078701in}}%
\pgfpathcurveto{\pgfqpoint{2.097591in}{3.070888in}}{\pgfqpoint{2.108190in}{3.066498in}}{\pgfqpoint{2.119240in}{3.066498in}}%
\pgfpathclose%
\pgfusepath{stroke,fill}%
\end{pgfscope}%
\begin{pgfscope}%
\pgfpathrectangle{\pgfqpoint{0.787074in}{0.548769in}}{\pgfqpoint{5.062926in}{3.102590in}}%
\pgfusepath{clip}%
\pgfsetbuttcap%
\pgfsetroundjoin%
\definecolor{currentfill}{rgb}{1.000000,0.498039,0.054902}%
\pgfsetfillcolor{currentfill}%
\pgfsetlinewidth{1.003750pt}%
\definecolor{currentstroke}{rgb}{1.000000,0.498039,0.054902}%
\pgfsetstrokecolor{currentstroke}%
\pgfsetdash{}{0pt}%
\pgfpathmoveto{\pgfqpoint{1.439483in}{3.012840in}}%
\pgfpathcurveto{\pgfqpoint{1.450533in}{3.012840in}}{\pgfqpoint{1.461132in}{3.017231in}}{\pgfqpoint{1.468946in}{3.025044in}}%
\pgfpathcurveto{\pgfqpoint{1.476760in}{3.032858in}}{\pgfqpoint{1.481150in}{3.043457in}}{\pgfqpoint{1.481150in}{3.054507in}}%
\pgfpathcurveto{\pgfqpoint{1.481150in}{3.065557in}}{\pgfqpoint{1.476760in}{3.076156in}}{\pgfqpoint{1.468946in}{3.083970in}}%
\pgfpathcurveto{\pgfqpoint{1.461132in}{3.091783in}}{\pgfqpoint{1.450533in}{3.096174in}}{\pgfqpoint{1.439483in}{3.096174in}}%
\pgfpathcurveto{\pgfqpoint{1.428433in}{3.096174in}}{\pgfqpoint{1.417834in}{3.091783in}}{\pgfqpoint{1.410020in}{3.083970in}}%
\pgfpathcurveto{\pgfqpoint{1.402207in}{3.076156in}}{\pgfqpoint{1.397817in}{3.065557in}}{\pgfqpoint{1.397817in}{3.054507in}}%
\pgfpathcurveto{\pgfqpoint{1.397817in}{3.043457in}}{\pgfqpoint{1.402207in}{3.032858in}}{\pgfqpoint{1.410020in}{3.025044in}}%
\pgfpathcurveto{\pgfqpoint{1.417834in}{3.017231in}}{\pgfqpoint{1.428433in}{3.012840in}}{\pgfqpoint{1.439483in}{3.012840in}}%
\pgfpathclose%
\pgfusepath{stroke,fill}%
\end{pgfscope}%
\begin{pgfscope}%
\pgfpathrectangle{\pgfqpoint{0.787074in}{0.548769in}}{\pgfqpoint{5.062926in}{3.102590in}}%
\pgfusepath{clip}%
\pgfsetbuttcap%
\pgfsetroundjoin%
\definecolor{currentfill}{rgb}{0.121569,0.466667,0.705882}%
\pgfsetfillcolor{currentfill}%
\pgfsetlinewidth{1.003750pt}%
\definecolor{currentstroke}{rgb}{0.121569,0.466667,0.705882}%
\pgfsetstrokecolor{currentstroke}%
\pgfsetdash{}{0pt}%
\pgfpathmoveto{\pgfqpoint{1.066452in}{0.670241in}}%
\pgfpathcurveto{\pgfqpoint{1.077502in}{0.670241in}}{\pgfqpoint{1.088101in}{0.674631in}}{\pgfqpoint{1.095915in}{0.682445in}}%
\pgfpathcurveto{\pgfqpoint{1.103729in}{0.690258in}}{\pgfqpoint{1.108119in}{0.700857in}}{\pgfqpoint{1.108119in}{0.711907in}}%
\pgfpathcurveto{\pgfqpoint{1.108119in}{0.722958in}}{\pgfqpoint{1.103729in}{0.733557in}}{\pgfqpoint{1.095915in}{0.741370in}}%
\pgfpathcurveto{\pgfqpoint{1.088101in}{0.749184in}}{\pgfqpoint{1.077502in}{0.753574in}}{\pgfqpoint{1.066452in}{0.753574in}}%
\pgfpathcurveto{\pgfqpoint{1.055402in}{0.753574in}}{\pgfqpoint{1.044803in}{0.749184in}}{\pgfqpoint{1.036989in}{0.741370in}}%
\pgfpathcurveto{\pgfqpoint{1.029176in}{0.733557in}}{\pgfqpoint{1.024786in}{0.722958in}}{\pgfqpoint{1.024786in}{0.711907in}}%
\pgfpathcurveto{\pgfqpoint{1.024786in}{0.700857in}}{\pgfqpoint{1.029176in}{0.690258in}}{\pgfqpoint{1.036989in}{0.682445in}}%
\pgfpathcurveto{\pgfqpoint{1.044803in}{0.674631in}}{\pgfqpoint{1.055402in}{0.670241in}}{\pgfqpoint{1.066452in}{0.670241in}}%
\pgfpathclose%
\pgfusepath{stroke,fill}%
\end{pgfscope}%
\begin{pgfscope}%
\pgfpathrectangle{\pgfqpoint{0.787074in}{0.548769in}}{\pgfqpoint{5.062926in}{3.102590in}}%
\pgfusepath{clip}%
\pgfsetbuttcap%
\pgfsetroundjoin%
\definecolor{currentfill}{rgb}{0.121569,0.466667,0.705882}%
\pgfsetfillcolor{currentfill}%
\pgfsetlinewidth{1.003750pt}%
\definecolor{currentstroke}{rgb}{0.121569,0.466667,0.705882}%
\pgfsetstrokecolor{currentstroke}%
\pgfsetdash{}{0pt}%
\pgfpathmoveto{\pgfqpoint{2.483460in}{1.395275in}}%
\pgfpathcurveto{\pgfqpoint{2.494510in}{1.395275in}}{\pgfqpoint{2.505109in}{1.399666in}}{\pgfqpoint{2.512923in}{1.407479in}}%
\pgfpathcurveto{\pgfqpoint{2.520736in}{1.415293in}}{\pgfqpoint{2.525127in}{1.425892in}}{\pgfqpoint{2.525127in}{1.436942in}}%
\pgfpathcurveto{\pgfqpoint{2.525127in}{1.447992in}}{\pgfqpoint{2.520736in}{1.458591in}}{\pgfqpoint{2.512923in}{1.466405in}}%
\pgfpathcurveto{\pgfqpoint{2.505109in}{1.474219in}}{\pgfqpoint{2.494510in}{1.478609in}}{\pgfqpoint{2.483460in}{1.478609in}}%
\pgfpathcurveto{\pgfqpoint{2.472410in}{1.478609in}}{\pgfqpoint{2.461811in}{1.474219in}}{\pgfqpoint{2.453997in}{1.466405in}}%
\pgfpathcurveto{\pgfqpoint{2.446184in}{1.458591in}}{\pgfqpoint{2.441793in}{1.447992in}}{\pgfqpoint{2.441793in}{1.436942in}}%
\pgfpathcurveto{\pgfqpoint{2.441793in}{1.425892in}}{\pgfqpoint{2.446184in}{1.415293in}}{\pgfqpoint{2.453997in}{1.407479in}}%
\pgfpathcurveto{\pgfqpoint{2.461811in}{1.399666in}}{\pgfqpoint{2.472410in}{1.395275in}}{\pgfqpoint{2.483460in}{1.395275in}}%
\pgfpathclose%
\pgfusepath{stroke,fill}%
\end{pgfscope}%
\begin{pgfscope}%
\pgfpathrectangle{\pgfqpoint{0.787074in}{0.548769in}}{\pgfqpoint{5.062926in}{3.102590in}}%
\pgfusepath{clip}%
\pgfsetbuttcap%
\pgfsetroundjoin%
\definecolor{currentfill}{rgb}{0.121569,0.466667,0.705882}%
\pgfsetfillcolor{currentfill}%
\pgfsetlinewidth{1.003750pt}%
\definecolor{currentstroke}{rgb}{0.121569,0.466667,0.705882}%
\pgfsetstrokecolor{currentstroke}%
\pgfsetdash{}{0pt}%
\pgfpathmoveto{\pgfqpoint{1.079813in}{0.648350in}}%
\pgfpathcurveto{\pgfqpoint{1.090864in}{0.648350in}}{\pgfqpoint{1.101463in}{0.652741in}}{\pgfqpoint{1.109276in}{0.660554in}}%
\pgfpathcurveto{\pgfqpoint{1.117090in}{0.668368in}}{\pgfqpoint{1.121480in}{0.678967in}}{\pgfqpoint{1.121480in}{0.690017in}}%
\pgfpathcurveto{\pgfqpoint{1.121480in}{0.701067in}}{\pgfqpoint{1.117090in}{0.711666in}}{\pgfqpoint{1.109276in}{0.719480in}}%
\pgfpathcurveto{\pgfqpoint{1.101463in}{0.727293in}}{\pgfqpoint{1.090864in}{0.731684in}}{\pgfqpoint{1.079813in}{0.731684in}}%
\pgfpathcurveto{\pgfqpoint{1.068763in}{0.731684in}}{\pgfqpoint{1.058164in}{0.727293in}}{\pgfqpoint{1.050351in}{0.719480in}}%
\pgfpathcurveto{\pgfqpoint{1.042537in}{0.711666in}}{\pgfqpoint{1.038147in}{0.701067in}}{\pgfqpoint{1.038147in}{0.690017in}}%
\pgfpathcurveto{\pgfqpoint{1.038147in}{0.678967in}}{\pgfqpoint{1.042537in}{0.668368in}}{\pgfqpoint{1.050351in}{0.660554in}}%
\pgfpathcurveto{\pgfqpoint{1.058164in}{0.652741in}}{\pgfqpoint{1.068763in}{0.648350in}}{\pgfqpoint{1.079813in}{0.648350in}}%
\pgfpathclose%
\pgfusepath{stroke,fill}%
\end{pgfscope}%
\begin{pgfscope}%
\pgfpathrectangle{\pgfqpoint{0.787074in}{0.548769in}}{\pgfqpoint{5.062926in}{3.102590in}}%
\pgfusepath{clip}%
\pgfsetbuttcap%
\pgfsetroundjoin%
\definecolor{currentfill}{rgb}{1.000000,0.498039,0.054902}%
\pgfsetfillcolor{currentfill}%
\pgfsetlinewidth{1.003750pt}%
\definecolor{currentstroke}{rgb}{1.000000,0.498039,0.054902}%
\pgfsetstrokecolor{currentstroke}%
\pgfsetdash{}{0pt}%
\pgfpathmoveto{\pgfqpoint{1.586491in}{1.744372in}}%
\pgfpathcurveto{\pgfqpoint{1.597541in}{1.744372in}}{\pgfqpoint{1.608140in}{1.748762in}}{\pgfqpoint{1.615954in}{1.756576in}}%
\pgfpathcurveto{\pgfqpoint{1.623767in}{1.764389in}}{\pgfqpoint{1.628158in}{1.774988in}}{\pgfqpoint{1.628158in}{1.786038in}}%
\pgfpathcurveto{\pgfqpoint{1.628158in}{1.797089in}}{\pgfqpoint{1.623767in}{1.807688in}}{\pgfqpoint{1.615954in}{1.815501in}}%
\pgfpathcurveto{\pgfqpoint{1.608140in}{1.823315in}}{\pgfqpoint{1.597541in}{1.827705in}}{\pgfqpoint{1.586491in}{1.827705in}}%
\pgfpathcurveto{\pgfqpoint{1.575441in}{1.827705in}}{\pgfqpoint{1.564842in}{1.823315in}}{\pgfqpoint{1.557028in}{1.815501in}}%
\pgfpathcurveto{\pgfqpoint{1.549215in}{1.807688in}}{\pgfqpoint{1.544824in}{1.797089in}}{\pgfqpoint{1.544824in}{1.786038in}}%
\pgfpathcurveto{\pgfqpoint{1.544824in}{1.774988in}}{\pgfqpoint{1.549215in}{1.764389in}}{\pgfqpoint{1.557028in}{1.756576in}}%
\pgfpathcurveto{\pgfqpoint{1.564842in}{1.748762in}}{\pgfqpoint{1.575441in}{1.744372in}}{\pgfqpoint{1.586491in}{1.744372in}}%
\pgfpathclose%
\pgfusepath{stroke,fill}%
\end{pgfscope}%
\begin{pgfscope}%
\pgfpathrectangle{\pgfqpoint{0.787074in}{0.548769in}}{\pgfqpoint{5.062926in}{3.102590in}}%
\pgfusepath{clip}%
\pgfsetbuttcap%
\pgfsetroundjoin%
\definecolor{currentfill}{rgb}{1.000000,0.498039,0.054902}%
\pgfsetfillcolor{currentfill}%
\pgfsetlinewidth{1.003750pt}%
\definecolor{currentstroke}{rgb}{1.000000,0.498039,0.054902}%
\pgfsetstrokecolor{currentstroke}%
\pgfsetdash{}{0pt}%
\pgfpathmoveto{\pgfqpoint{1.403876in}{2.684286in}}%
\pgfpathcurveto{\pgfqpoint{1.414927in}{2.684286in}}{\pgfqpoint{1.425526in}{2.688676in}}{\pgfqpoint{1.433339in}{2.696490in}}%
\pgfpathcurveto{\pgfqpoint{1.441153in}{2.704304in}}{\pgfqpoint{1.445543in}{2.714903in}}{\pgfqpoint{1.445543in}{2.725953in}}%
\pgfpathcurveto{\pgfqpoint{1.445543in}{2.737003in}}{\pgfqpoint{1.441153in}{2.747602in}}{\pgfqpoint{1.433339in}{2.755416in}}%
\pgfpathcurveto{\pgfqpoint{1.425526in}{2.763229in}}{\pgfqpoint{1.414927in}{2.767619in}}{\pgfqpoint{1.403876in}{2.767619in}}%
\pgfpathcurveto{\pgfqpoint{1.392826in}{2.767619in}}{\pgfqpoint{1.382227in}{2.763229in}}{\pgfqpoint{1.374414in}{2.755416in}}%
\pgfpathcurveto{\pgfqpoint{1.366600in}{2.747602in}}{\pgfqpoint{1.362210in}{2.737003in}}{\pgfqpoint{1.362210in}{2.725953in}}%
\pgfpathcurveto{\pgfqpoint{1.362210in}{2.714903in}}{\pgfqpoint{1.366600in}{2.704304in}}{\pgfqpoint{1.374414in}{2.696490in}}%
\pgfpathcurveto{\pgfqpoint{1.382227in}{2.688676in}}{\pgfqpoint{1.392826in}{2.684286in}}{\pgfqpoint{1.403876in}{2.684286in}}%
\pgfpathclose%
\pgfusepath{stroke,fill}%
\end{pgfscope}%
\begin{pgfscope}%
\pgfpathrectangle{\pgfqpoint{0.787074in}{0.548769in}}{\pgfqpoint{5.062926in}{3.102590in}}%
\pgfusepath{clip}%
\pgfsetbuttcap%
\pgfsetroundjoin%
\definecolor{currentfill}{rgb}{1.000000,0.498039,0.054902}%
\pgfsetfillcolor{currentfill}%
\pgfsetlinewidth{1.003750pt}%
\definecolor{currentstroke}{rgb}{1.000000,0.498039,0.054902}%
\pgfsetstrokecolor{currentstroke}%
\pgfsetdash{}{0pt}%
\pgfpathmoveto{\pgfqpoint{1.993095in}{2.189411in}}%
\pgfpathcurveto{\pgfqpoint{2.004145in}{2.189411in}}{\pgfqpoint{2.014744in}{2.193801in}}{\pgfqpoint{2.022557in}{2.201614in}}%
\pgfpathcurveto{\pgfqpoint{2.030371in}{2.209428in}}{\pgfqpoint{2.034761in}{2.220027in}}{\pgfqpoint{2.034761in}{2.231077in}}%
\pgfpathcurveto{\pgfqpoint{2.034761in}{2.242127in}}{\pgfqpoint{2.030371in}{2.252726in}}{\pgfqpoint{2.022557in}{2.260540in}}%
\pgfpathcurveto{\pgfqpoint{2.014744in}{2.268354in}}{\pgfqpoint{2.004145in}{2.272744in}}{\pgfqpoint{1.993095in}{2.272744in}}%
\pgfpathcurveto{\pgfqpoint{1.982044in}{2.272744in}}{\pgfqpoint{1.971445in}{2.268354in}}{\pgfqpoint{1.963632in}{2.260540in}}%
\pgfpathcurveto{\pgfqpoint{1.955818in}{2.252726in}}{\pgfqpoint{1.951428in}{2.242127in}}{\pgfqpoint{1.951428in}{2.231077in}}%
\pgfpathcurveto{\pgfqpoint{1.951428in}{2.220027in}}{\pgfqpoint{1.955818in}{2.209428in}}{\pgfqpoint{1.963632in}{2.201614in}}%
\pgfpathcurveto{\pgfqpoint{1.971445in}{2.193801in}}{\pgfqpoint{1.982044in}{2.189411in}}{\pgfqpoint{1.993095in}{2.189411in}}%
\pgfpathclose%
\pgfusepath{stroke,fill}%
\end{pgfscope}%
\begin{pgfscope}%
\pgfpathrectangle{\pgfqpoint{0.787074in}{0.548769in}}{\pgfqpoint{5.062926in}{3.102590in}}%
\pgfusepath{clip}%
\pgfsetbuttcap%
\pgfsetroundjoin%
\definecolor{currentfill}{rgb}{1.000000,0.498039,0.054902}%
\pgfsetfillcolor{currentfill}%
\pgfsetlinewidth{1.003750pt}%
\definecolor{currentstroke}{rgb}{1.000000,0.498039,0.054902}%
\pgfsetstrokecolor{currentstroke}%
\pgfsetdash{}{0pt}%
\pgfpathmoveto{\pgfqpoint{1.530896in}{2.951689in}}%
\pgfpathcurveto{\pgfqpoint{1.541947in}{2.951689in}}{\pgfqpoint{1.552546in}{2.956079in}}{\pgfqpoint{1.560359in}{2.963893in}}%
\pgfpathcurveto{\pgfqpoint{1.568173in}{2.971707in}}{\pgfqpoint{1.572563in}{2.982306in}}{\pgfqpoint{1.572563in}{2.993356in}}%
\pgfpathcurveto{\pgfqpoint{1.572563in}{3.004406in}}{\pgfqpoint{1.568173in}{3.015005in}}{\pgfqpoint{1.560359in}{3.022818in}}%
\pgfpathcurveto{\pgfqpoint{1.552546in}{3.030632in}}{\pgfqpoint{1.541947in}{3.035022in}}{\pgfqpoint{1.530896in}{3.035022in}}%
\pgfpathcurveto{\pgfqpoint{1.519846in}{3.035022in}}{\pgfqpoint{1.509247in}{3.030632in}}{\pgfqpoint{1.501434in}{3.022818in}}%
\pgfpathcurveto{\pgfqpoint{1.493620in}{3.015005in}}{\pgfqpoint{1.489230in}{3.004406in}}{\pgfqpoint{1.489230in}{2.993356in}}%
\pgfpathcurveto{\pgfqpoint{1.489230in}{2.982306in}}{\pgfqpoint{1.493620in}{2.971707in}}{\pgfqpoint{1.501434in}{2.963893in}}%
\pgfpathcurveto{\pgfqpoint{1.509247in}{2.956079in}}{\pgfqpoint{1.519846in}{2.951689in}}{\pgfqpoint{1.530896in}{2.951689in}}%
\pgfpathclose%
\pgfusepath{stroke,fill}%
\end{pgfscope}%
\begin{pgfscope}%
\pgfpathrectangle{\pgfqpoint{0.787074in}{0.548769in}}{\pgfqpoint{5.062926in}{3.102590in}}%
\pgfusepath{clip}%
\pgfsetbuttcap%
\pgfsetroundjoin%
\definecolor{currentfill}{rgb}{1.000000,0.498039,0.054902}%
\pgfsetfillcolor{currentfill}%
\pgfsetlinewidth{1.003750pt}%
\definecolor{currentstroke}{rgb}{1.000000,0.498039,0.054902}%
\pgfsetstrokecolor{currentstroke}%
\pgfsetdash{}{0pt}%
\pgfpathmoveto{\pgfqpoint{1.539792in}{2.956805in}}%
\pgfpathcurveto{\pgfqpoint{1.550843in}{2.956805in}}{\pgfqpoint{1.561442in}{2.961195in}}{\pgfqpoint{1.569255in}{2.969009in}}%
\pgfpathcurveto{\pgfqpoint{1.577069in}{2.976822in}}{\pgfqpoint{1.581459in}{2.987421in}}{\pgfqpoint{1.581459in}{2.998472in}}%
\pgfpathcurveto{\pgfqpoint{1.581459in}{3.009522in}}{\pgfqpoint{1.577069in}{3.020121in}}{\pgfqpoint{1.569255in}{3.027934in}}%
\pgfpathcurveto{\pgfqpoint{1.561442in}{3.035748in}}{\pgfqpoint{1.550843in}{3.040138in}}{\pgfqpoint{1.539792in}{3.040138in}}%
\pgfpathcurveto{\pgfqpoint{1.528742in}{3.040138in}}{\pgfqpoint{1.518143in}{3.035748in}}{\pgfqpoint{1.510330in}{3.027934in}}%
\pgfpathcurveto{\pgfqpoint{1.502516in}{3.020121in}}{\pgfqpoint{1.498126in}{3.009522in}}{\pgfqpoint{1.498126in}{2.998472in}}%
\pgfpathcurveto{\pgfqpoint{1.498126in}{2.987421in}}{\pgfqpoint{1.502516in}{2.976822in}}{\pgfqpoint{1.510330in}{2.969009in}}%
\pgfpathcurveto{\pgfqpoint{1.518143in}{2.961195in}}{\pgfqpoint{1.528742in}{2.956805in}}{\pgfqpoint{1.539792in}{2.956805in}}%
\pgfpathclose%
\pgfusepath{stroke,fill}%
\end{pgfscope}%
\begin{pgfscope}%
\pgfpathrectangle{\pgfqpoint{0.787074in}{0.548769in}}{\pgfqpoint{5.062926in}{3.102590in}}%
\pgfusepath{clip}%
\pgfsetbuttcap%
\pgfsetroundjoin%
\definecolor{currentfill}{rgb}{1.000000,0.498039,0.054902}%
\pgfsetfillcolor{currentfill}%
\pgfsetlinewidth{1.003750pt}%
\definecolor{currentstroke}{rgb}{1.000000,0.498039,0.054902}%
\pgfsetstrokecolor{currentstroke}%
\pgfsetdash{}{0pt}%
\pgfpathmoveto{\pgfqpoint{1.568896in}{2.890755in}}%
\pgfpathcurveto{\pgfqpoint{1.579946in}{2.890755in}}{\pgfqpoint{1.590545in}{2.895145in}}{\pgfqpoint{1.598359in}{2.902959in}}%
\pgfpathcurveto{\pgfqpoint{1.606172in}{2.910773in}}{\pgfqpoint{1.610562in}{2.921372in}}{\pgfqpoint{1.610562in}{2.932422in}}%
\pgfpathcurveto{\pgfqpoint{1.610562in}{2.943472in}}{\pgfqpoint{1.606172in}{2.954071in}}{\pgfqpoint{1.598359in}{2.961885in}}%
\pgfpathcurveto{\pgfqpoint{1.590545in}{2.969698in}}{\pgfqpoint{1.579946in}{2.974089in}}{\pgfqpoint{1.568896in}{2.974089in}}%
\pgfpathcurveto{\pgfqpoint{1.557846in}{2.974089in}}{\pgfqpoint{1.547247in}{2.969698in}}{\pgfqpoint{1.539433in}{2.961885in}}%
\pgfpathcurveto{\pgfqpoint{1.531619in}{2.954071in}}{\pgfqpoint{1.527229in}{2.943472in}}{\pgfqpoint{1.527229in}{2.932422in}}%
\pgfpathcurveto{\pgfqpoint{1.527229in}{2.921372in}}{\pgfqpoint{1.531619in}{2.910773in}}{\pgfqpoint{1.539433in}{2.902959in}}%
\pgfpathcurveto{\pgfqpoint{1.547247in}{2.895145in}}{\pgfqpoint{1.557846in}{2.890755in}}{\pgfqpoint{1.568896in}{2.890755in}}%
\pgfpathclose%
\pgfusepath{stroke,fill}%
\end{pgfscope}%
\begin{pgfscope}%
\pgfpathrectangle{\pgfqpoint{0.787074in}{0.548769in}}{\pgfqpoint{5.062926in}{3.102590in}}%
\pgfusepath{clip}%
\pgfsetbuttcap%
\pgfsetroundjoin%
\definecolor{currentfill}{rgb}{0.121569,0.466667,0.705882}%
\pgfsetfillcolor{currentfill}%
\pgfsetlinewidth{1.003750pt}%
\definecolor{currentstroke}{rgb}{0.121569,0.466667,0.705882}%
\pgfsetstrokecolor{currentstroke}%
\pgfsetdash{}{0pt}%
\pgfpathmoveto{\pgfqpoint{1.157187in}{2.074998in}}%
\pgfpathcurveto{\pgfqpoint{1.168238in}{2.074998in}}{\pgfqpoint{1.178837in}{2.079388in}}{\pgfqpoint{1.186650in}{2.087202in}}%
\pgfpathcurveto{\pgfqpoint{1.194464in}{2.095016in}}{\pgfqpoint{1.198854in}{2.105615in}}{\pgfqpoint{1.198854in}{2.116665in}}%
\pgfpathcurveto{\pgfqpoint{1.198854in}{2.127715in}}{\pgfqpoint{1.194464in}{2.138314in}}{\pgfqpoint{1.186650in}{2.146127in}}%
\pgfpathcurveto{\pgfqpoint{1.178837in}{2.153941in}}{\pgfqpoint{1.168238in}{2.158331in}}{\pgfqpoint{1.157187in}{2.158331in}}%
\pgfpathcurveto{\pgfqpoint{1.146137in}{2.158331in}}{\pgfqpoint{1.135538in}{2.153941in}}{\pgfqpoint{1.127725in}{2.146127in}}%
\pgfpathcurveto{\pgfqpoint{1.119911in}{2.138314in}}{\pgfqpoint{1.115521in}{2.127715in}}{\pgfqpoint{1.115521in}{2.116665in}}%
\pgfpathcurveto{\pgfqpoint{1.115521in}{2.105615in}}{\pgfqpoint{1.119911in}{2.095016in}}{\pgfqpoint{1.127725in}{2.087202in}}%
\pgfpathcurveto{\pgfqpoint{1.135538in}{2.079388in}}{\pgfqpoint{1.146137in}{2.074998in}}{\pgfqpoint{1.157187in}{2.074998in}}%
\pgfpathclose%
\pgfusepath{stroke,fill}%
\end{pgfscope}%
\begin{pgfscope}%
\pgfpathrectangle{\pgfqpoint{0.787074in}{0.548769in}}{\pgfqpoint{5.062926in}{3.102590in}}%
\pgfusepath{clip}%
\pgfsetbuttcap%
\pgfsetroundjoin%
\definecolor{currentfill}{rgb}{1.000000,0.498039,0.054902}%
\pgfsetfillcolor{currentfill}%
\pgfsetlinewidth{1.003750pt}%
\definecolor{currentstroke}{rgb}{1.000000,0.498039,0.054902}%
\pgfsetstrokecolor{currentstroke}%
\pgfsetdash{}{0pt}%
\pgfpathmoveto{\pgfqpoint{1.245029in}{2.746524in}}%
\pgfpathcurveto{\pgfqpoint{1.256079in}{2.746524in}}{\pgfqpoint{1.266678in}{2.750915in}}{\pgfqpoint{1.274492in}{2.758728in}}%
\pgfpathcurveto{\pgfqpoint{1.282306in}{2.766542in}}{\pgfqpoint{1.286696in}{2.777141in}}{\pgfqpoint{1.286696in}{2.788191in}}%
\pgfpathcurveto{\pgfqpoint{1.286696in}{2.799241in}}{\pgfqpoint{1.282306in}{2.809840in}}{\pgfqpoint{1.274492in}{2.817654in}}%
\pgfpathcurveto{\pgfqpoint{1.266678in}{2.825468in}}{\pgfqpoint{1.256079in}{2.829858in}}{\pgfqpoint{1.245029in}{2.829858in}}%
\pgfpathcurveto{\pgfqpoint{1.233979in}{2.829858in}}{\pgfqpoint{1.223380in}{2.825468in}}{\pgfqpoint{1.215566in}{2.817654in}}%
\pgfpathcurveto{\pgfqpoint{1.207753in}{2.809840in}}{\pgfqpoint{1.203363in}{2.799241in}}{\pgfqpoint{1.203363in}{2.788191in}}%
\pgfpathcurveto{\pgfqpoint{1.203363in}{2.777141in}}{\pgfqpoint{1.207753in}{2.766542in}}{\pgfqpoint{1.215566in}{2.758728in}}%
\pgfpathcurveto{\pgfqpoint{1.223380in}{2.750915in}}{\pgfqpoint{1.233979in}{2.746524in}}{\pgfqpoint{1.245029in}{2.746524in}}%
\pgfpathclose%
\pgfusepath{stroke,fill}%
\end{pgfscope}%
\begin{pgfscope}%
\pgfpathrectangle{\pgfqpoint{0.787074in}{0.548769in}}{\pgfqpoint{5.062926in}{3.102590in}}%
\pgfusepath{clip}%
\pgfsetbuttcap%
\pgfsetroundjoin%
\definecolor{currentfill}{rgb}{0.121569,0.466667,0.705882}%
\pgfsetfillcolor{currentfill}%
\pgfsetlinewidth{1.003750pt}%
\definecolor{currentstroke}{rgb}{0.121569,0.466667,0.705882}%
\pgfsetstrokecolor{currentstroke}%
\pgfsetdash{}{0pt}%
\pgfpathmoveto{\pgfqpoint{2.840240in}{1.732307in}}%
\pgfpathcurveto{\pgfqpoint{2.851290in}{1.732307in}}{\pgfqpoint{2.861889in}{1.736697in}}{\pgfqpoint{2.869703in}{1.744511in}}%
\pgfpathcurveto{\pgfqpoint{2.877517in}{1.752325in}}{\pgfqpoint{2.881907in}{1.762924in}}{\pgfqpoint{2.881907in}{1.773974in}}%
\pgfpathcurveto{\pgfqpoint{2.881907in}{1.785024in}}{\pgfqpoint{2.877517in}{1.795623in}}{\pgfqpoint{2.869703in}{1.803437in}}%
\pgfpathcurveto{\pgfqpoint{2.861889in}{1.811250in}}{\pgfqpoint{2.851290in}{1.815640in}}{\pgfqpoint{2.840240in}{1.815640in}}%
\pgfpathcurveto{\pgfqpoint{2.829190in}{1.815640in}}{\pgfqpoint{2.818591in}{1.811250in}}{\pgfqpoint{2.810777in}{1.803437in}}%
\pgfpathcurveto{\pgfqpoint{2.802964in}{1.795623in}}{\pgfqpoint{2.798574in}{1.785024in}}{\pgfqpoint{2.798574in}{1.773974in}}%
\pgfpathcurveto{\pgfqpoint{2.798574in}{1.762924in}}{\pgfqpoint{2.802964in}{1.752325in}}{\pgfqpoint{2.810777in}{1.744511in}}%
\pgfpathcurveto{\pgfqpoint{2.818591in}{1.736697in}}{\pgfqpoint{2.829190in}{1.732307in}}{\pgfqpoint{2.840240in}{1.732307in}}%
\pgfpathclose%
\pgfusepath{stroke,fill}%
\end{pgfscope}%
\begin{pgfscope}%
\pgfpathrectangle{\pgfqpoint{0.787074in}{0.548769in}}{\pgfqpoint{5.062926in}{3.102590in}}%
\pgfusepath{clip}%
\pgfsetbuttcap%
\pgfsetroundjoin%
\definecolor{currentfill}{rgb}{0.121569,0.466667,0.705882}%
\pgfsetfillcolor{currentfill}%
\pgfsetlinewidth{1.003750pt}%
\definecolor{currentstroke}{rgb}{0.121569,0.466667,0.705882}%
\pgfsetstrokecolor{currentstroke}%
\pgfsetdash{}{0pt}%
\pgfpathmoveto{\pgfqpoint{2.349278in}{2.443186in}}%
\pgfpathcurveto{\pgfqpoint{2.360328in}{2.443186in}}{\pgfqpoint{2.370927in}{2.447576in}}{\pgfqpoint{2.378741in}{2.455390in}}%
\pgfpathcurveto{\pgfqpoint{2.386554in}{2.463203in}}{\pgfqpoint{2.390944in}{2.473802in}}{\pgfqpoint{2.390944in}{2.484852in}}%
\pgfpathcurveto{\pgfqpoint{2.390944in}{2.495903in}}{\pgfqpoint{2.386554in}{2.506502in}}{\pgfqpoint{2.378741in}{2.514315in}}%
\pgfpathcurveto{\pgfqpoint{2.370927in}{2.522129in}}{\pgfqpoint{2.360328in}{2.526519in}}{\pgfqpoint{2.349278in}{2.526519in}}%
\pgfpathcurveto{\pgfqpoint{2.338228in}{2.526519in}}{\pgfqpoint{2.327629in}{2.522129in}}{\pgfqpoint{2.319815in}{2.514315in}}%
\pgfpathcurveto{\pgfqpoint{2.312001in}{2.506502in}}{\pgfqpoint{2.307611in}{2.495903in}}{\pgfqpoint{2.307611in}{2.484852in}}%
\pgfpathcurveto{\pgfqpoint{2.307611in}{2.473802in}}{\pgfqpoint{2.312001in}{2.463203in}}{\pgfqpoint{2.319815in}{2.455390in}}%
\pgfpathcurveto{\pgfqpoint{2.327629in}{2.447576in}}{\pgfqpoint{2.338228in}{2.443186in}}{\pgfqpoint{2.349278in}{2.443186in}}%
\pgfpathclose%
\pgfusepath{stroke,fill}%
\end{pgfscope}%
\begin{pgfscope}%
\pgfpathrectangle{\pgfqpoint{0.787074in}{0.548769in}}{\pgfqpoint{5.062926in}{3.102590in}}%
\pgfusepath{clip}%
\pgfsetbuttcap%
\pgfsetroundjoin%
\definecolor{currentfill}{rgb}{0.121569,0.466667,0.705882}%
\pgfsetfillcolor{currentfill}%
\pgfsetlinewidth{1.003750pt}%
\definecolor{currentstroke}{rgb}{0.121569,0.466667,0.705882}%
\pgfsetstrokecolor{currentstroke}%
\pgfsetdash{}{0pt}%
\pgfpathmoveto{\pgfqpoint{1.783407in}{1.419741in}}%
\pgfpathcurveto{\pgfqpoint{1.794457in}{1.419741in}}{\pgfqpoint{1.805056in}{1.424131in}}{\pgfqpoint{1.812870in}{1.431945in}}%
\pgfpathcurveto{\pgfqpoint{1.820683in}{1.439759in}}{\pgfqpoint{1.825074in}{1.450358in}}{\pgfqpoint{1.825074in}{1.461408in}}%
\pgfpathcurveto{\pgfqpoint{1.825074in}{1.472458in}}{\pgfqpoint{1.820683in}{1.483057in}}{\pgfqpoint{1.812870in}{1.490870in}}%
\pgfpathcurveto{\pgfqpoint{1.805056in}{1.498684in}}{\pgfqpoint{1.794457in}{1.503074in}}{\pgfqpoint{1.783407in}{1.503074in}}%
\pgfpathcurveto{\pgfqpoint{1.772357in}{1.503074in}}{\pgfqpoint{1.761758in}{1.498684in}}{\pgfqpoint{1.753944in}{1.490870in}}%
\pgfpathcurveto{\pgfqpoint{1.746131in}{1.483057in}}{\pgfqpoint{1.741740in}{1.472458in}}{\pgfqpoint{1.741740in}{1.461408in}}%
\pgfpathcurveto{\pgfqpoint{1.741740in}{1.450358in}}{\pgfqpoint{1.746131in}{1.439759in}}{\pgfqpoint{1.753944in}{1.431945in}}%
\pgfpathcurveto{\pgfqpoint{1.761758in}{1.424131in}}{\pgfqpoint{1.772357in}{1.419741in}}{\pgfqpoint{1.783407in}{1.419741in}}%
\pgfpathclose%
\pgfusepath{stroke,fill}%
\end{pgfscope}%
\begin{pgfscope}%
\pgfpathrectangle{\pgfqpoint{0.787074in}{0.548769in}}{\pgfqpoint{5.062926in}{3.102590in}}%
\pgfusepath{clip}%
\pgfsetbuttcap%
\pgfsetroundjoin%
\definecolor{currentfill}{rgb}{1.000000,0.498039,0.054902}%
\pgfsetfillcolor{currentfill}%
\pgfsetlinewidth{1.003750pt}%
\definecolor{currentstroke}{rgb}{1.000000,0.498039,0.054902}%
\pgfsetstrokecolor{currentstroke}%
\pgfsetdash{}{0pt}%
\pgfpathmoveto{\pgfqpoint{1.752235in}{2.589074in}}%
\pgfpathcurveto{\pgfqpoint{1.763285in}{2.589074in}}{\pgfqpoint{1.773884in}{2.593464in}}{\pgfqpoint{1.781697in}{2.601278in}}%
\pgfpathcurveto{\pgfqpoint{1.789511in}{2.609092in}}{\pgfqpoint{1.793901in}{2.619691in}}{\pgfqpoint{1.793901in}{2.630741in}}%
\pgfpathcurveto{\pgfqpoint{1.793901in}{2.641791in}}{\pgfqpoint{1.789511in}{2.652390in}}{\pgfqpoint{1.781697in}{2.660204in}}%
\pgfpathcurveto{\pgfqpoint{1.773884in}{2.668017in}}{\pgfqpoint{1.763285in}{2.672407in}}{\pgfqpoint{1.752235in}{2.672407in}}%
\pgfpathcurveto{\pgfqpoint{1.741185in}{2.672407in}}{\pgfqpoint{1.730586in}{2.668017in}}{\pgfqpoint{1.722772in}{2.660204in}}%
\pgfpathcurveto{\pgfqpoint{1.714958in}{2.652390in}}{\pgfqpoint{1.710568in}{2.641791in}}{\pgfqpoint{1.710568in}{2.630741in}}%
\pgfpathcurveto{\pgfqpoint{1.710568in}{2.619691in}}{\pgfqpoint{1.714958in}{2.609092in}}{\pgfqpoint{1.722772in}{2.601278in}}%
\pgfpathcurveto{\pgfqpoint{1.730586in}{2.593464in}}{\pgfqpoint{1.741185in}{2.589074in}}{\pgfqpoint{1.752235in}{2.589074in}}%
\pgfpathclose%
\pgfusepath{stroke,fill}%
\end{pgfscope}%
\begin{pgfscope}%
\pgfpathrectangle{\pgfqpoint{0.787074in}{0.548769in}}{\pgfqpoint{5.062926in}{3.102590in}}%
\pgfusepath{clip}%
\pgfsetbuttcap%
\pgfsetroundjoin%
\definecolor{currentfill}{rgb}{1.000000,0.498039,0.054902}%
\pgfsetfillcolor{currentfill}%
\pgfsetlinewidth{1.003750pt}%
\definecolor{currentstroke}{rgb}{1.000000,0.498039,0.054902}%
\pgfsetstrokecolor{currentstroke}%
\pgfsetdash{}{0pt}%
\pgfpathmoveto{\pgfqpoint{1.209634in}{2.590352in}}%
\pgfpathcurveto{\pgfqpoint{1.220684in}{2.590352in}}{\pgfqpoint{1.231283in}{2.594742in}}{\pgfqpoint{1.239097in}{2.602556in}}%
\pgfpathcurveto{\pgfqpoint{1.246911in}{2.610369in}}{\pgfqpoint{1.251301in}{2.620968in}}{\pgfqpoint{1.251301in}{2.632018in}}%
\pgfpathcurveto{\pgfqpoint{1.251301in}{2.643068in}}{\pgfqpoint{1.246911in}{2.653668in}}{\pgfqpoint{1.239097in}{2.661481in}}%
\pgfpathcurveto{\pgfqpoint{1.231283in}{2.669295in}}{\pgfqpoint{1.220684in}{2.673685in}}{\pgfqpoint{1.209634in}{2.673685in}}%
\pgfpathcurveto{\pgfqpoint{1.198584in}{2.673685in}}{\pgfqpoint{1.187985in}{2.669295in}}{\pgfqpoint{1.180172in}{2.661481in}}%
\pgfpathcurveto{\pgfqpoint{1.172358in}{2.653668in}}{\pgfqpoint{1.167968in}{2.643068in}}{\pgfqpoint{1.167968in}{2.632018in}}%
\pgfpathcurveto{\pgfqpoint{1.167968in}{2.620968in}}{\pgfqpoint{1.172358in}{2.610369in}}{\pgfqpoint{1.180172in}{2.602556in}}%
\pgfpathcurveto{\pgfqpoint{1.187985in}{2.594742in}}{\pgfqpoint{1.198584in}{2.590352in}}{\pgfqpoint{1.209634in}{2.590352in}}%
\pgfpathclose%
\pgfusepath{stroke,fill}%
\end{pgfscope}%
\begin{pgfscope}%
\pgfpathrectangle{\pgfqpoint{0.787074in}{0.548769in}}{\pgfqpoint{5.062926in}{3.102590in}}%
\pgfusepath{clip}%
\pgfsetbuttcap%
\pgfsetroundjoin%
\definecolor{currentfill}{rgb}{1.000000,0.498039,0.054902}%
\pgfsetfillcolor{currentfill}%
\pgfsetlinewidth{1.003750pt}%
\definecolor{currentstroke}{rgb}{1.000000,0.498039,0.054902}%
\pgfsetstrokecolor{currentstroke}%
\pgfsetdash{}{0pt}%
\pgfpathmoveto{\pgfqpoint{1.490643in}{2.993235in}}%
\pgfpathcurveto{\pgfqpoint{1.501693in}{2.993235in}}{\pgfqpoint{1.512293in}{2.997625in}}{\pgfqpoint{1.520106in}{3.005439in}}%
\pgfpathcurveto{\pgfqpoint{1.527920in}{3.013252in}}{\pgfqpoint{1.532310in}{3.023851in}}{\pgfqpoint{1.532310in}{3.034902in}}%
\pgfpathcurveto{\pgfqpoint{1.532310in}{3.045952in}}{\pgfqpoint{1.527920in}{3.056551in}}{\pgfqpoint{1.520106in}{3.064364in}}%
\pgfpathcurveto{\pgfqpoint{1.512293in}{3.072178in}}{\pgfqpoint{1.501693in}{3.076568in}}{\pgfqpoint{1.490643in}{3.076568in}}%
\pgfpathcurveto{\pgfqpoint{1.479593in}{3.076568in}}{\pgfqpoint{1.468994in}{3.072178in}}{\pgfqpoint{1.461181in}{3.064364in}}%
\pgfpathcurveto{\pgfqpoint{1.453367in}{3.056551in}}{\pgfqpoint{1.448977in}{3.045952in}}{\pgfqpoint{1.448977in}{3.034902in}}%
\pgfpathcurveto{\pgfqpoint{1.448977in}{3.023851in}}{\pgfqpoint{1.453367in}{3.013252in}}{\pgfqpoint{1.461181in}{3.005439in}}%
\pgfpathcurveto{\pgfqpoint{1.468994in}{2.997625in}}{\pgfqpoint{1.479593in}{2.993235in}}{\pgfqpoint{1.490643in}{2.993235in}}%
\pgfpathclose%
\pgfusepath{stroke,fill}%
\end{pgfscope}%
\begin{pgfscope}%
\pgfpathrectangle{\pgfqpoint{0.787074in}{0.548769in}}{\pgfqpoint{5.062926in}{3.102590in}}%
\pgfusepath{clip}%
\pgfsetbuttcap%
\pgfsetroundjoin%
\definecolor{currentfill}{rgb}{0.121569,0.466667,0.705882}%
\pgfsetfillcolor{currentfill}%
\pgfsetlinewidth{1.003750pt}%
\definecolor{currentstroke}{rgb}{0.121569,0.466667,0.705882}%
\pgfsetstrokecolor{currentstroke}%
\pgfsetdash{}{0pt}%
\pgfpathmoveto{\pgfqpoint{1.066452in}{0.666848in}}%
\pgfpathcurveto{\pgfqpoint{1.077502in}{0.666848in}}{\pgfqpoint{1.088101in}{0.671238in}}{\pgfqpoint{1.095915in}{0.679052in}}%
\pgfpathcurveto{\pgfqpoint{1.103729in}{0.686866in}}{\pgfqpoint{1.108119in}{0.697465in}}{\pgfqpoint{1.108119in}{0.708515in}}%
\pgfpathcurveto{\pgfqpoint{1.108119in}{0.719565in}}{\pgfqpoint{1.103729in}{0.730164in}}{\pgfqpoint{1.095915in}{0.737978in}}%
\pgfpathcurveto{\pgfqpoint{1.088101in}{0.745791in}}{\pgfqpoint{1.077502in}{0.750182in}}{\pgfqpoint{1.066452in}{0.750182in}}%
\pgfpathcurveto{\pgfqpoint{1.055402in}{0.750182in}}{\pgfqpoint{1.044803in}{0.745791in}}{\pgfqpoint{1.036989in}{0.737978in}}%
\pgfpathcurveto{\pgfqpoint{1.029176in}{0.730164in}}{\pgfqpoint{1.024786in}{0.719565in}}{\pgfqpoint{1.024786in}{0.708515in}}%
\pgfpathcurveto{\pgfqpoint{1.024786in}{0.697465in}}{\pgfqpoint{1.029176in}{0.686866in}}{\pgfqpoint{1.036989in}{0.679052in}}%
\pgfpathcurveto{\pgfqpoint{1.044803in}{0.671238in}}{\pgfqpoint{1.055402in}{0.666848in}}{\pgfqpoint{1.066452in}{0.666848in}}%
\pgfpathclose%
\pgfusepath{stroke,fill}%
\end{pgfscope}%
\begin{pgfscope}%
\pgfpathrectangle{\pgfqpoint{0.787074in}{0.548769in}}{\pgfqpoint{5.062926in}{3.102590in}}%
\pgfusepath{clip}%
\pgfsetbuttcap%
\pgfsetroundjoin%
\definecolor{currentfill}{rgb}{1.000000,0.498039,0.054902}%
\pgfsetfillcolor{currentfill}%
\pgfsetlinewidth{1.003750pt}%
\definecolor{currentstroke}{rgb}{1.000000,0.498039,0.054902}%
\pgfsetstrokecolor{currentstroke}%
\pgfsetdash{}{0pt}%
\pgfpathmoveto{\pgfqpoint{1.583081in}{2.802044in}}%
\pgfpathcurveto{\pgfqpoint{1.594132in}{2.802044in}}{\pgfqpoint{1.604731in}{2.806435in}}{\pgfqpoint{1.612544in}{2.814248in}}%
\pgfpathcurveto{\pgfqpoint{1.620358in}{2.822062in}}{\pgfqpoint{1.624748in}{2.832661in}}{\pgfqpoint{1.624748in}{2.843711in}}%
\pgfpathcurveto{\pgfqpoint{1.624748in}{2.854761in}}{\pgfqpoint{1.620358in}{2.865360in}}{\pgfqpoint{1.612544in}{2.873174in}}%
\pgfpathcurveto{\pgfqpoint{1.604731in}{2.880987in}}{\pgfqpoint{1.594132in}{2.885378in}}{\pgfqpoint{1.583081in}{2.885378in}}%
\pgfpathcurveto{\pgfqpoint{1.572031in}{2.885378in}}{\pgfqpoint{1.561432in}{2.880987in}}{\pgfqpoint{1.553619in}{2.873174in}}%
\pgfpathcurveto{\pgfqpoint{1.545805in}{2.865360in}}{\pgfqpoint{1.541415in}{2.854761in}}{\pgfqpoint{1.541415in}{2.843711in}}%
\pgfpathcurveto{\pgfqpoint{1.541415in}{2.832661in}}{\pgfqpoint{1.545805in}{2.822062in}}{\pgfqpoint{1.553619in}{2.814248in}}%
\pgfpathcurveto{\pgfqpoint{1.561432in}{2.806435in}}{\pgfqpoint{1.572031in}{2.802044in}}{\pgfqpoint{1.583081in}{2.802044in}}%
\pgfpathclose%
\pgfusepath{stroke,fill}%
\end{pgfscope}%
\begin{pgfscope}%
\pgfpathrectangle{\pgfqpoint{0.787074in}{0.548769in}}{\pgfqpoint{5.062926in}{3.102590in}}%
\pgfusepath{clip}%
\pgfsetbuttcap%
\pgfsetroundjoin%
\definecolor{currentfill}{rgb}{0.121569,0.466667,0.705882}%
\pgfsetfillcolor{currentfill}%
\pgfsetlinewidth{1.003750pt}%
\definecolor{currentstroke}{rgb}{0.121569,0.466667,0.705882}%
\pgfsetstrokecolor{currentstroke}%
\pgfsetdash{}{0pt}%
\pgfpathmoveto{\pgfqpoint{1.095297in}{0.648244in}}%
\pgfpathcurveto{\pgfqpoint{1.106348in}{0.648244in}}{\pgfqpoint{1.116947in}{0.652634in}}{\pgfqpoint{1.124760in}{0.660448in}}%
\pgfpathcurveto{\pgfqpoint{1.132574in}{0.668261in}}{\pgfqpoint{1.136964in}{0.678860in}}{\pgfqpoint{1.136964in}{0.689911in}}%
\pgfpathcurveto{\pgfqpoint{1.136964in}{0.700961in}}{\pgfqpoint{1.132574in}{0.711560in}}{\pgfqpoint{1.124760in}{0.719373in}}%
\pgfpathcurveto{\pgfqpoint{1.116947in}{0.727187in}}{\pgfqpoint{1.106348in}{0.731577in}}{\pgfqpoint{1.095297in}{0.731577in}}%
\pgfpathcurveto{\pgfqpoint{1.084247in}{0.731577in}}{\pgfqpoint{1.073648in}{0.727187in}}{\pgfqpoint{1.065835in}{0.719373in}}%
\pgfpathcurveto{\pgfqpoint{1.058021in}{0.711560in}}{\pgfqpoint{1.053631in}{0.700961in}}{\pgfqpoint{1.053631in}{0.689911in}}%
\pgfpathcurveto{\pgfqpoint{1.053631in}{0.678860in}}{\pgfqpoint{1.058021in}{0.668261in}}{\pgfqpoint{1.065835in}{0.660448in}}%
\pgfpathcurveto{\pgfqpoint{1.073648in}{0.652634in}}{\pgfqpoint{1.084247in}{0.648244in}}{\pgfqpoint{1.095297in}{0.648244in}}%
\pgfpathclose%
\pgfusepath{stroke,fill}%
\end{pgfscope}%
\begin{pgfscope}%
\pgfpathrectangle{\pgfqpoint{0.787074in}{0.548769in}}{\pgfqpoint{5.062926in}{3.102590in}}%
\pgfusepath{clip}%
\pgfsetbuttcap%
\pgfsetroundjoin%
\definecolor{currentfill}{rgb}{1.000000,0.498039,0.054902}%
\pgfsetfillcolor{currentfill}%
\pgfsetlinewidth{1.003750pt}%
\definecolor{currentstroke}{rgb}{1.000000,0.498039,0.054902}%
\pgfsetstrokecolor{currentstroke}%
\pgfsetdash{}{0pt}%
\pgfpathmoveto{\pgfqpoint{1.922078in}{3.229318in}}%
\pgfpathcurveto{\pgfqpoint{1.933128in}{3.229318in}}{\pgfqpoint{1.943727in}{3.233708in}}{\pgfqpoint{1.951540in}{3.241522in}}%
\pgfpathcurveto{\pgfqpoint{1.959354in}{3.249335in}}{\pgfqpoint{1.963744in}{3.259934in}}{\pgfqpoint{1.963744in}{3.270984in}}%
\pgfpathcurveto{\pgfqpoint{1.963744in}{3.282035in}}{\pgfqpoint{1.959354in}{3.292634in}}{\pgfqpoint{1.951540in}{3.300447in}}%
\pgfpathcurveto{\pgfqpoint{1.943727in}{3.308261in}}{\pgfqpoint{1.933128in}{3.312651in}}{\pgfqpoint{1.922078in}{3.312651in}}%
\pgfpathcurveto{\pgfqpoint{1.911027in}{3.312651in}}{\pgfqpoint{1.900428in}{3.308261in}}{\pgfqpoint{1.892615in}{3.300447in}}%
\pgfpathcurveto{\pgfqpoint{1.884801in}{3.292634in}}{\pgfqpoint{1.880411in}{3.282035in}}{\pgfqpoint{1.880411in}{3.270984in}}%
\pgfpathcurveto{\pgfqpoint{1.880411in}{3.259934in}}{\pgfqpoint{1.884801in}{3.249335in}}{\pgfqpoint{1.892615in}{3.241522in}}%
\pgfpathcurveto{\pgfqpoint{1.900428in}{3.233708in}}{\pgfqpoint{1.911027in}{3.229318in}}{\pgfqpoint{1.922078in}{3.229318in}}%
\pgfpathclose%
\pgfusepath{stroke,fill}%
\end{pgfscope}%
\begin{pgfscope}%
\pgfpathrectangle{\pgfqpoint{0.787074in}{0.548769in}}{\pgfqpoint{5.062926in}{3.102590in}}%
\pgfusepath{clip}%
\pgfsetbuttcap%
\pgfsetroundjoin%
\definecolor{currentfill}{rgb}{0.121569,0.466667,0.705882}%
\pgfsetfillcolor{currentfill}%
\pgfsetlinewidth{1.003750pt}%
\definecolor{currentstroke}{rgb}{0.121569,0.466667,0.705882}%
\pgfsetstrokecolor{currentstroke}%
\pgfsetdash{}{0pt}%
\pgfpathmoveto{\pgfqpoint{2.414766in}{2.961491in}}%
\pgfpathcurveto{\pgfqpoint{2.425816in}{2.961491in}}{\pgfqpoint{2.436415in}{2.965882in}}{\pgfqpoint{2.444229in}{2.973695in}}%
\pgfpathcurveto{\pgfqpoint{2.452043in}{2.981509in}}{\pgfqpoint{2.456433in}{2.992108in}}{\pgfqpoint{2.456433in}{3.003158in}}%
\pgfpathcurveto{\pgfqpoint{2.456433in}{3.014208in}}{\pgfqpoint{2.452043in}{3.024807in}}{\pgfqpoint{2.444229in}{3.032621in}}%
\pgfpathcurveto{\pgfqpoint{2.436415in}{3.040434in}}{\pgfqpoint{2.425816in}{3.044825in}}{\pgfqpoint{2.414766in}{3.044825in}}%
\pgfpathcurveto{\pgfqpoint{2.403716in}{3.044825in}}{\pgfqpoint{2.393117in}{3.040434in}}{\pgfqpoint{2.385303in}{3.032621in}}%
\pgfpathcurveto{\pgfqpoint{2.377490in}{3.024807in}}{\pgfqpoint{2.373099in}{3.014208in}}{\pgfqpoint{2.373099in}{3.003158in}}%
\pgfpathcurveto{\pgfqpoint{2.373099in}{2.992108in}}{\pgfqpoint{2.377490in}{2.981509in}}{\pgfqpoint{2.385303in}{2.973695in}}%
\pgfpathcurveto{\pgfqpoint{2.393117in}{2.965882in}}{\pgfqpoint{2.403716in}{2.961491in}}{\pgfqpoint{2.414766in}{2.961491in}}%
\pgfpathclose%
\pgfusepath{stroke,fill}%
\end{pgfscope}%
\begin{pgfscope}%
\pgfpathrectangle{\pgfqpoint{0.787074in}{0.548769in}}{\pgfqpoint{5.062926in}{3.102590in}}%
\pgfusepath{clip}%
\pgfsetbuttcap%
\pgfsetroundjoin%
\definecolor{currentfill}{rgb}{1.000000,0.498039,0.054902}%
\pgfsetfillcolor{currentfill}%
\pgfsetlinewidth{1.003750pt}%
\definecolor{currentstroke}{rgb}{1.000000,0.498039,0.054902}%
\pgfsetstrokecolor{currentstroke}%
\pgfsetdash{}{0pt}%
\pgfpathmoveto{\pgfqpoint{1.817958in}{2.689616in}}%
\pgfpathcurveto{\pgfqpoint{1.829008in}{2.689616in}}{\pgfqpoint{1.839607in}{2.694006in}}{\pgfqpoint{1.847421in}{2.701820in}}%
\pgfpathcurveto{\pgfqpoint{1.855235in}{2.709634in}}{\pgfqpoint{1.859625in}{2.720233in}}{\pgfqpoint{1.859625in}{2.731283in}}%
\pgfpathcurveto{\pgfqpoint{1.859625in}{2.742333in}}{\pgfqpoint{1.855235in}{2.752932in}}{\pgfqpoint{1.847421in}{2.760746in}}%
\pgfpathcurveto{\pgfqpoint{1.839607in}{2.768559in}}{\pgfqpoint{1.829008in}{2.772950in}}{\pgfqpoint{1.817958in}{2.772950in}}%
\pgfpathcurveto{\pgfqpoint{1.806908in}{2.772950in}}{\pgfqpoint{1.796309in}{2.768559in}}{\pgfqpoint{1.788495in}{2.760746in}}%
\pgfpathcurveto{\pgfqpoint{1.780682in}{2.752932in}}{\pgfqpoint{1.776291in}{2.742333in}}{\pgfqpoint{1.776291in}{2.731283in}}%
\pgfpathcurveto{\pgfqpoint{1.776291in}{2.720233in}}{\pgfqpoint{1.780682in}{2.709634in}}{\pgfqpoint{1.788495in}{2.701820in}}%
\pgfpathcurveto{\pgfqpoint{1.796309in}{2.694006in}}{\pgfqpoint{1.806908in}{2.689616in}}{\pgfqpoint{1.817958in}{2.689616in}}%
\pgfpathclose%
\pgfusepath{stroke,fill}%
\end{pgfscope}%
\begin{pgfscope}%
\pgfpathrectangle{\pgfqpoint{0.787074in}{0.548769in}}{\pgfqpoint{5.062926in}{3.102590in}}%
\pgfusepath{clip}%
\pgfsetbuttcap%
\pgfsetroundjoin%
\definecolor{currentfill}{rgb}{1.000000,0.498039,0.054902}%
\pgfsetfillcolor{currentfill}%
\pgfsetlinewidth{1.003750pt}%
\definecolor{currentstroke}{rgb}{1.000000,0.498039,0.054902}%
\pgfsetstrokecolor{currentstroke}%
\pgfsetdash{}{0pt}%
\pgfpathmoveto{\pgfqpoint{1.605543in}{1.853687in}}%
\pgfpathcurveto{\pgfqpoint{1.616593in}{1.853687in}}{\pgfqpoint{1.627192in}{1.858077in}}{\pgfqpoint{1.635006in}{1.865891in}}%
\pgfpathcurveto{\pgfqpoint{1.642819in}{1.873705in}}{\pgfqpoint{1.647209in}{1.884304in}}{\pgfqpoint{1.647209in}{1.895354in}}%
\pgfpathcurveto{\pgfqpoint{1.647209in}{1.906404in}}{\pgfqpoint{1.642819in}{1.917003in}}{\pgfqpoint{1.635006in}{1.924817in}}%
\pgfpathcurveto{\pgfqpoint{1.627192in}{1.932630in}}{\pgfqpoint{1.616593in}{1.937021in}}{\pgfqpoint{1.605543in}{1.937021in}}%
\pgfpathcurveto{\pgfqpoint{1.594493in}{1.937021in}}{\pgfqpoint{1.583894in}{1.932630in}}{\pgfqpoint{1.576080in}{1.924817in}}%
\pgfpathcurveto{\pgfqpoint{1.568266in}{1.917003in}}{\pgfqpoint{1.563876in}{1.906404in}}{\pgfqpoint{1.563876in}{1.895354in}}%
\pgfpathcurveto{\pgfqpoint{1.563876in}{1.884304in}}{\pgfqpoint{1.568266in}{1.873705in}}{\pgfqpoint{1.576080in}{1.865891in}}%
\pgfpathcurveto{\pgfqpoint{1.583894in}{1.858077in}}{\pgfqpoint{1.594493in}{1.853687in}}{\pgfqpoint{1.605543in}{1.853687in}}%
\pgfpathclose%
\pgfusepath{stroke,fill}%
\end{pgfscope}%
\begin{pgfscope}%
\pgfpathrectangle{\pgfqpoint{0.787074in}{0.548769in}}{\pgfqpoint{5.062926in}{3.102590in}}%
\pgfusepath{clip}%
\pgfsetbuttcap%
\pgfsetroundjoin%
\definecolor{currentfill}{rgb}{1.000000,0.498039,0.054902}%
\pgfsetfillcolor{currentfill}%
\pgfsetlinewidth{1.003750pt}%
\definecolor{currentstroke}{rgb}{1.000000,0.498039,0.054902}%
\pgfsetstrokecolor{currentstroke}%
\pgfsetdash{}{0pt}%
\pgfpathmoveto{\pgfqpoint{1.315314in}{2.416725in}}%
\pgfpathcurveto{\pgfqpoint{1.326364in}{2.416725in}}{\pgfqpoint{1.336963in}{2.421116in}}{\pgfqpoint{1.344777in}{2.428929in}}%
\pgfpathcurveto{\pgfqpoint{1.352591in}{2.436743in}}{\pgfqpoint{1.356981in}{2.447342in}}{\pgfqpoint{1.356981in}{2.458392in}}%
\pgfpathcurveto{\pgfqpoint{1.356981in}{2.469442in}}{\pgfqpoint{1.352591in}{2.480041in}}{\pgfqpoint{1.344777in}{2.487855in}}%
\pgfpathcurveto{\pgfqpoint{1.336963in}{2.495669in}}{\pgfqpoint{1.326364in}{2.500059in}}{\pgfqpoint{1.315314in}{2.500059in}}%
\pgfpathcurveto{\pgfqpoint{1.304264in}{2.500059in}}{\pgfqpoint{1.293665in}{2.495669in}}{\pgfqpoint{1.285851in}{2.487855in}}%
\pgfpathcurveto{\pgfqpoint{1.278038in}{2.480041in}}{\pgfqpoint{1.273648in}{2.469442in}}{\pgfqpoint{1.273648in}{2.458392in}}%
\pgfpathcurveto{\pgfqpoint{1.273648in}{2.447342in}}{\pgfqpoint{1.278038in}{2.436743in}}{\pgfqpoint{1.285851in}{2.428929in}}%
\pgfpathcurveto{\pgfqpoint{1.293665in}{2.421116in}}{\pgfqpoint{1.304264in}{2.416725in}}{\pgfqpoint{1.315314in}{2.416725in}}%
\pgfpathclose%
\pgfusepath{stroke,fill}%
\end{pgfscope}%
\begin{pgfscope}%
\pgfpathrectangle{\pgfqpoint{0.787074in}{0.548769in}}{\pgfqpoint{5.062926in}{3.102590in}}%
\pgfusepath{clip}%
\pgfsetbuttcap%
\pgfsetroundjoin%
\definecolor{currentfill}{rgb}{0.121569,0.466667,0.705882}%
\pgfsetfillcolor{currentfill}%
\pgfsetlinewidth{1.003750pt}%
\definecolor{currentstroke}{rgb}{0.121569,0.466667,0.705882}%
\pgfsetstrokecolor{currentstroke}%
\pgfsetdash{}{0pt}%
\pgfpathmoveto{\pgfqpoint{1.562454in}{2.410505in}}%
\pgfpathcurveto{\pgfqpoint{1.573504in}{2.410505in}}{\pgfqpoint{1.584103in}{2.414896in}}{\pgfqpoint{1.591917in}{2.422709in}}%
\pgfpathcurveto{\pgfqpoint{1.599730in}{2.430523in}}{\pgfqpoint{1.604121in}{2.441122in}}{\pgfqpoint{1.604121in}{2.452172in}}%
\pgfpathcurveto{\pgfqpoint{1.604121in}{2.463222in}}{\pgfqpoint{1.599730in}{2.473821in}}{\pgfqpoint{1.591917in}{2.481635in}}%
\pgfpathcurveto{\pgfqpoint{1.584103in}{2.489449in}}{\pgfqpoint{1.573504in}{2.493839in}}{\pgfqpoint{1.562454in}{2.493839in}}%
\pgfpathcurveto{\pgfqpoint{1.551404in}{2.493839in}}{\pgfqpoint{1.540805in}{2.489449in}}{\pgfqpoint{1.532991in}{2.481635in}}%
\pgfpathcurveto{\pgfqpoint{1.525178in}{2.473821in}}{\pgfqpoint{1.520787in}{2.463222in}}{\pgfqpoint{1.520787in}{2.452172in}}%
\pgfpathcurveto{\pgfqpoint{1.520787in}{2.441122in}}{\pgfqpoint{1.525178in}{2.430523in}}{\pgfqpoint{1.532991in}{2.422709in}}%
\pgfpathcurveto{\pgfqpoint{1.540805in}{2.414896in}}{\pgfqpoint{1.551404in}{2.410505in}}{\pgfqpoint{1.562454in}{2.410505in}}%
\pgfpathclose%
\pgfusepath{stroke,fill}%
\end{pgfscope}%
\begin{pgfscope}%
\pgfpathrectangle{\pgfqpoint{0.787074in}{0.548769in}}{\pgfqpoint{5.062926in}{3.102590in}}%
\pgfusepath{clip}%
\pgfsetbuttcap%
\pgfsetroundjoin%
\definecolor{currentfill}{rgb}{0.121569,0.466667,0.705882}%
\pgfsetfillcolor{currentfill}%
\pgfsetlinewidth{1.003750pt}%
\definecolor{currentstroke}{rgb}{0.121569,0.466667,0.705882}%
\pgfsetstrokecolor{currentstroke}%
\pgfsetdash{}{0pt}%
\pgfpathmoveto{\pgfqpoint{1.192482in}{0.678421in}}%
\pgfpathcurveto{\pgfqpoint{1.203532in}{0.678421in}}{\pgfqpoint{1.214131in}{0.682812in}}{\pgfqpoint{1.221945in}{0.690625in}}%
\pgfpathcurveto{\pgfqpoint{1.229758in}{0.698439in}}{\pgfqpoint{1.234149in}{0.709038in}}{\pgfqpoint{1.234149in}{0.720088in}}%
\pgfpathcurveto{\pgfqpoint{1.234149in}{0.731138in}}{\pgfqpoint{1.229758in}{0.741737in}}{\pgfqpoint{1.221945in}{0.749551in}}%
\pgfpathcurveto{\pgfqpoint{1.214131in}{0.757364in}}{\pgfqpoint{1.203532in}{0.761755in}}{\pgfqpoint{1.192482in}{0.761755in}}%
\pgfpathcurveto{\pgfqpoint{1.181432in}{0.761755in}}{\pgfqpoint{1.170833in}{0.757364in}}{\pgfqpoint{1.163019in}{0.749551in}}%
\pgfpathcurveto{\pgfqpoint{1.155206in}{0.741737in}}{\pgfqpoint{1.150815in}{0.731138in}}{\pgfqpoint{1.150815in}{0.720088in}}%
\pgfpathcurveto{\pgfqpoint{1.150815in}{0.709038in}}{\pgfqpoint{1.155206in}{0.698439in}}{\pgfqpoint{1.163019in}{0.690625in}}%
\pgfpathcurveto{\pgfqpoint{1.170833in}{0.682812in}}{\pgfqpoint{1.181432in}{0.678421in}}{\pgfqpoint{1.192482in}{0.678421in}}%
\pgfpathclose%
\pgfusepath{stroke,fill}%
\end{pgfscope}%
\begin{pgfscope}%
\pgfpathrectangle{\pgfqpoint{0.787074in}{0.548769in}}{\pgfqpoint{5.062926in}{3.102590in}}%
\pgfusepath{clip}%
\pgfsetbuttcap%
\pgfsetroundjoin%
\definecolor{currentfill}{rgb}{1.000000,0.498039,0.054902}%
\pgfsetfillcolor{currentfill}%
\pgfsetlinewidth{1.003750pt}%
\definecolor{currentstroke}{rgb}{1.000000,0.498039,0.054902}%
\pgfsetstrokecolor{currentstroke}%
\pgfsetdash{}{0pt}%
\pgfpathmoveto{\pgfqpoint{1.263002in}{2.587414in}}%
\pgfpathcurveto{\pgfqpoint{1.274052in}{2.587414in}}{\pgfqpoint{1.284651in}{2.591804in}}{\pgfqpoint{1.292465in}{2.599618in}}%
\pgfpathcurveto{\pgfqpoint{1.300278in}{2.607432in}}{\pgfqpoint{1.304669in}{2.618031in}}{\pgfqpoint{1.304669in}{2.629081in}}%
\pgfpathcurveto{\pgfqpoint{1.304669in}{2.640131in}}{\pgfqpoint{1.300278in}{2.650730in}}{\pgfqpoint{1.292465in}{2.658544in}}%
\pgfpathcurveto{\pgfqpoint{1.284651in}{2.666357in}}{\pgfqpoint{1.274052in}{2.670747in}}{\pgfqpoint{1.263002in}{2.670747in}}%
\pgfpathcurveto{\pgfqpoint{1.251952in}{2.670747in}}{\pgfqpoint{1.241353in}{2.666357in}}{\pgfqpoint{1.233539in}{2.658544in}}%
\pgfpathcurveto{\pgfqpoint{1.225726in}{2.650730in}}{\pgfqpoint{1.221335in}{2.640131in}}{\pgfqpoint{1.221335in}{2.629081in}}%
\pgfpathcurveto{\pgfqpoint{1.221335in}{2.618031in}}{\pgfqpoint{1.225726in}{2.607432in}}{\pgfqpoint{1.233539in}{2.599618in}}%
\pgfpathcurveto{\pgfqpoint{1.241353in}{2.591804in}}{\pgfqpoint{1.251952in}{2.587414in}}{\pgfqpoint{1.263002in}{2.587414in}}%
\pgfpathclose%
\pgfusepath{stroke,fill}%
\end{pgfscope}%
\begin{pgfscope}%
\pgfpathrectangle{\pgfqpoint{0.787074in}{0.548769in}}{\pgfqpoint{5.062926in}{3.102590in}}%
\pgfusepath{clip}%
\pgfsetbuttcap%
\pgfsetroundjoin%
\definecolor{currentfill}{rgb}{1.000000,0.498039,0.054902}%
\pgfsetfillcolor{currentfill}%
\pgfsetlinewidth{1.003750pt}%
\definecolor{currentstroke}{rgb}{1.000000,0.498039,0.054902}%
\pgfsetstrokecolor{currentstroke}%
\pgfsetdash{}{0pt}%
\pgfpathmoveto{\pgfqpoint{1.487465in}{3.097620in}}%
\pgfpathcurveto{\pgfqpoint{1.498515in}{3.097620in}}{\pgfqpoint{1.509114in}{3.102011in}}{\pgfqpoint{1.516928in}{3.109824in}}%
\pgfpathcurveto{\pgfqpoint{1.524741in}{3.117638in}}{\pgfqpoint{1.529132in}{3.128237in}}{\pgfqpoint{1.529132in}{3.139287in}}%
\pgfpathcurveto{\pgfqpoint{1.529132in}{3.150337in}}{\pgfqpoint{1.524741in}{3.160936in}}{\pgfqpoint{1.516928in}{3.168750in}}%
\pgfpathcurveto{\pgfqpoint{1.509114in}{3.176564in}}{\pgfqpoint{1.498515in}{3.180954in}}{\pgfqpoint{1.487465in}{3.180954in}}%
\pgfpathcurveto{\pgfqpoint{1.476415in}{3.180954in}}{\pgfqpoint{1.465816in}{3.176564in}}{\pgfqpoint{1.458002in}{3.168750in}}%
\pgfpathcurveto{\pgfqpoint{1.450188in}{3.160936in}}{\pgfqpoint{1.445798in}{3.150337in}}{\pgfqpoint{1.445798in}{3.139287in}}%
\pgfpathcurveto{\pgfqpoint{1.445798in}{3.128237in}}{\pgfqpoint{1.450188in}{3.117638in}}{\pgfqpoint{1.458002in}{3.109824in}}%
\pgfpathcurveto{\pgfqpoint{1.465816in}{3.102011in}}{\pgfqpoint{1.476415in}{3.097620in}}{\pgfqpoint{1.487465in}{3.097620in}}%
\pgfpathclose%
\pgfusepath{stroke,fill}%
\end{pgfscope}%
\begin{pgfscope}%
\pgfpathrectangle{\pgfqpoint{0.787074in}{0.548769in}}{\pgfqpoint{5.062926in}{3.102590in}}%
\pgfusepath{clip}%
\pgfsetbuttcap%
\pgfsetroundjoin%
\definecolor{currentfill}{rgb}{0.121569,0.466667,0.705882}%
\pgfsetfillcolor{currentfill}%
\pgfsetlinewidth{1.003750pt}%
\definecolor{currentstroke}{rgb}{0.121569,0.466667,0.705882}%
\pgfsetstrokecolor{currentstroke}%
\pgfsetdash{}{0pt}%
\pgfpathmoveto{\pgfqpoint{1.215063in}{0.648139in}}%
\pgfpathcurveto{\pgfqpoint{1.226113in}{0.648139in}}{\pgfqpoint{1.236712in}{0.652529in}}{\pgfqpoint{1.244526in}{0.660343in}}%
\pgfpathcurveto{\pgfqpoint{1.252339in}{0.668156in}}{\pgfqpoint{1.256729in}{0.678755in}}{\pgfqpoint{1.256729in}{0.689806in}}%
\pgfpathcurveto{\pgfqpoint{1.256729in}{0.700856in}}{\pgfqpoint{1.252339in}{0.711455in}}{\pgfqpoint{1.244526in}{0.719268in}}%
\pgfpathcurveto{\pgfqpoint{1.236712in}{0.727082in}}{\pgfqpoint{1.226113in}{0.731472in}}{\pgfqpoint{1.215063in}{0.731472in}}%
\pgfpathcurveto{\pgfqpoint{1.204013in}{0.731472in}}{\pgfqpoint{1.193414in}{0.727082in}}{\pgfqpoint{1.185600in}{0.719268in}}%
\pgfpathcurveto{\pgfqpoint{1.177786in}{0.711455in}}{\pgfqpoint{1.173396in}{0.700856in}}{\pgfqpoint{1.173396in}{0.689806in}}%
\pgfpathcurveto{\pgfqpoint{1.173396in}{0.678755in}}{\pgfqpoint{1.177786in}{0.668156in}}{\pgfqpoint{1.185600in}{0.660343in}}%
\pgfpathcurveto{\pgfqpoint{1.193414in}{0.652529in}}{\pgfqpoint{1.204013in}{0.648139in}}{\pgfqpoint{1.215063in}{0.648139in}}%
\pgfpathclose%
\pgfusepath{stroke,fill}%
\end{pgfscope}%
\begin{pgfscope}%
\pgfpathrectangle{\pgfqpoint{0.787074in}{0.548769in}}{\pgfqpoint{5.062926in}{3.102590in}}%
\pgfusepath{clip}%
\pgfsetbuttcap%
\pgfsetroundjoin%
\definecolor{currentfill}{rgb}{0.121569,0.466667,0.705882}%
\pgfsetfillcolor{currentfill}%
\pgfsetlinewidth{1.003750pt}%
\definecolor{currentstroke}{rgb}{0.121569,0.466667,0.705882}%
\pgfsetstrokecolor{currentstroke}%
\pgfsetdash{}{0pt}%
\pgfpathmoveto{\pgfqpoint{1.071931in}{0.649969in}}%
\pgfpathcurveto{\pgfqpoint{1.082981in}{0.649969in}}{\pgfqpoint{1.093580in}{0.654360in}}{\pgfqpoint{1.101394in}{0.662173in}}%
\pgfpathcurveto{\pgfqpoint{1.109207in}{0.669987in}}{\pgfqpoint{1.113597in}{0.680586in}}{\pgfqpoint{1.113597in}{0.691636in}}%
\pgfpathcurveto{\pgfqpoint{1.113597in}{0.702686in}}{\pgfqpoint{1.109207in}{0.713285in}}{\pgfqpoint{1.101394in}{0.721099in}}%
\pgfpathcurveto{\pgfqpoint{1.093580in}{0.728913in}}{\pgfqpoint{1.082981in}{0.733303in}}{\pgfqpoint{1.071931in}{0.733303in}}%
\pgfpathcurveto{\pgfqpoint{1.060881in}{0.733303in}}{\pgfqpoint{1.050282in}{0.728913in}}{\pgfqpoint{1.042468in}{0.721099in}}%
\pgfpathcurveto{\pgfqpoint{1.034654in}{0.713285in}}{\pgfqpoint{1.030264in}{0.702686in}}{\pgfqpoint{1.030264in}{0.691636in}}%
\pgfpathcurveto{\pgfqpoint{1.030264in}{0.680586in}}{\pgfqpoint{1.034654in}{0.669987in}}{\pgfqpoint{1.042468in}{0.662173in}}%
\pgfpathcurveto{\pgfqpoint{1.050282in}{0.654360in}}{\pgfqpoint{1.060881in}{0.649969in}}{\pgfqpoint{1.071931in}{0.649969in}}%
\pgfpathclose%
\pgfusepath{stroke,fill}%
\end{pgfscope}%
\begin{pgfscope}%
\pgfpathrectangle{\pgfqpoint{0.787074in}{0.548769in}}{\pgfqpoint{5.062926in}{3.102590in}}%
\pgfusepath{clip}%
\pgfsetbuttcap%
\pgfsetroundjoin%
\definecolor{currentfill}{rgb}{0.121569,0.466667,0.705882}%
\pgfsetfillcolor{currentfill}%
\pgfsetlinewidth{1.003750pt}%
\definecolor{currentstroke}{rgb}{0.121569,0.466667,0.705882}%
\pgfsetstrokecolor{currentstroke}%
\pgfsetdash{}{0pt}%
\pgfpathmoveto{\pgfqpoint{2.253126in}{2.881001in}}%
\pgfpathcurveto{\pgfqpoint{2.264176in}{2.881001in}}{\pgfqpoint{2.274775in}{2.885391in}}{\pgfqpoint{2.282588in}{2.893205in}}%
\pgfpathcurveto{\pgfqpoint{2.290402in}{2.901018in}}{\pgfqpoint{2.294792in}{2.911617in}}{\pgfqpoint{2.294792in}{2.922668in}}%
\pgfpathcurveto{\pgfqpoint{2.294792in}{2.933718in}}{\pgfqpoint{2.290402in}{2.944317in}}{\pgfqpoint{2.282588in}{2.952130in}}%
\pgfpathcurveto{\pgfqpoint{2.274775in}{2.959944in}}{\pgfqpoint{2.264176in}{2.964334in}}{\pgfqpoint{2.253126in}{2.964334in}}%
\pgfpathcurveto{\pgfqpoint{2.242075in}{2.964334in}}{\pgfqpoint{2.231476in}{2.959944in}}{\pgfqpoint{2.223663in}{2.952130in}}%
\pgfpathcurveto{\pgfqpoint{2.215849in}{2.944317in}}{\pgfqpoint{2.211459in}{2.933718in}}{\pgfqpoint{2.211459in}{2.922668in}}%
\pgfpathcurveto{\pgfqpoint{2.211459in}{2.911617in}}{\pgfqpoint{2.215849in}{2.901018in}}{\pgfqpoint{2.223663in}{2.893205in}}%
\pgfpathcurveto{\pgfqpoint{2.231476in}{2.885391in}}{\pgfqpoint{2.242075in}{2.881001in}}{\pgfqpoint{2.253126in}{2.881001in}}%
\pgfpathclose%
\pgfusepath{stroke,fill}%
\end{pgfscope}%
\begin{pgfscope}%
\pgfpathrectangle{\pgfqpoint{0.787074in}{0.548769in}}{\pgfqpoint{5.062926in}{3.102590in}}%
\pgfusepath{clip}%
\pgfsetbuttcap%
\pgfsetroundjoin%
\definecolor{currentfill}{rgb}{1.000000,0.498039,0.054902}%
\pgfsetfillcolor{currentfill}%
\pgfsetlinewidth{1.003750pt}%
\definecolor{currentstroke}{rgb}{1.000000,0.498039,0.054902}%
\pgfsetstrokecolor{currentstroke}%
\pgfsetdash{}{0pt}%
\pgfpathmoveto{\pgfqpoint{2.035544in}{3.082877in}}%
\pgfpathcurveto{\pgfqpoint{2.046594in}{3.082877in}}{\pgfqpoint{2.057193in}{3.087267in}}{\pgfqpoint{2.065007in}{3.095080in}}%
\pgfpathcurveto{\pgfqpoint{2.072820in}{3.102894in}}{\pgfqpoint{2.077210in}{3.113493in}}{\pgfqpoint{2.077210in}{3.124543in}}%
\pgfpathcurveto{\pgfqpoint{2.077210in}{3.135593in}}{\pgfqpoint{2.072820in}{3.146192in}}{\pgfqpoint{2.065007in}{3.154006in}}%
\pgfpathcurveto{\pgfqpoint{2.057193in}{3.161820in}}{\pgfqpoint{2.046594in}{3.166210in}}{\pgfqpoint{2.035544in}{3.166210in}}%
\pgfpathcurveto{\pgfqpoint{2.024494in}{3.166210in}}{\pgfqpoint{2.013895in}{3.161820in}}{\pgfqpoint{2.006081in}{3.154006in}}%
\pgfpathcurveto{\pgfqpoint{1.998267in}{3.146192in}}{\pgfqpoint{1.993877in}{3.135593in}}{\pgfqpoint{1.993877in}{3.124543in}}%
\pgfpathcurveto{\pgfqpoint{1.993877in}{3.113493in}}{\pgfqpoint{1.998267in}{3.102894in}}{\pgfqpoint{2.006081in}{3.095080in}}%
\pgfpathcurveto{\pgfqpoint{2.013895in}{3.087267in}}{\pgfqpoint{2.024494in}{3.082877in}}{\pgfqpoint{2.035544in}{3.082877in}}%
\pgfpathclose%
\pgfusepath{stroke,fill}%
\end{pgfscope}%
\begin{pgfscope}%
\pgfpathrectangle{\pgfqpoint{0.787074in}{0.548769in}}{\pgfqpoint{5.062926in}{3.102590in}}%
\pgfusepath{clip}%
\pgfsetbuttcap%
\pgfsetroundjoin%
\definecolor{currentfill}{rgb}{1.000000,0.498039,0.054902}%
\pgfsetfillcolor{currentfill}%
\pgfsetlinewidth{1.003750pt}%
\definecolor{currentstroke}{rgb}{1.000000,0.498039,0.054902}%
\pgfsetstrokecolor{currentstroke}%
\pgfsetdash{}{0pt}%
\pgfpathmoveto{\pgfqpoint{1.497455in}{2.563602in}}%
\pgfpathcurveto{\pgfqpoint{1.508505in}{2.563602in}}{\pgfqpoint{1.519104in}{2.567992in}}{\pgfqpoint{1.526918in}{2.575806in}}%
\pgfpathcurveto{\pgfqpoint{1.534731in}{2.583619in}}{\pgfqpoint{1.539122in}{2.594218in}}{\pgfqpoint{1.539122in}{2.605268in}}%
\pgfpathcurveto{\pgfqpoint{1.539122in}{2.616318in}}{\pgfqpoint{1.534731in}{2.626917in}}{\pgfqpoint{1.526918in}{2.634731in}}%
\pgfpathcurveto{\pgfqpoint{1.519104in}{2.642545in}}{\pgfqpoint{1.508505in}{2.646935in}}{\pgfqpoint{1.497455in}{2.646935in}}%
\pgfpathcurveto{\pgfqpoint{1.486405in}{2.646935in}}{\pgfqpoint{1.475806in}{2.642545in}}{\pgfqpoint{1.467992in}{2.634731in}}%
\pgfpathcurveto{\pgfqpoint{1.460179in}{2.626917in}}{\pgfqpoint{1.455788in}{2.616318in}}{\pgfqpoint{1.455788in}{2.605268in}}%
\pgfpathcurveto{\pgfqpoint{1.455788in}{2.594218in}}{\pgfqpoint{1.460179in}{2.583619in}}{\pgfqpoint{1.467992in}{2.575806in}}%
\pgfpathcurveto{\pgfqpoint{1.475806in}{2.567992in}}{\pgfqpoint{1.486405in}{2.563602in}}{\pgfqpoint{1.497455in}{2.563602in}}%
\pgfpathclose%
\pgfusepath{stroke,fill}%
\end{pgfscope}%
\begin{pgfscope}%
\pgfpathrectangle{\pgfqpoint{0.787074in}{0.548769in}}{\pgfqpoint{5.062926in}{3.102590in}}%
\pgfusepath{clip}%
\pgfsetbuttcap%
\pgfsetroundjoin%
\definecolor{currentfill}{rgb}{1.000000,0.498039,0.054902}%
\pgfsetfillcolor{currentfill}%
\pgfsetlinewidth{1.003750pt}%
\definecolor{currentstroke}{rgb}{1.000000,0.498039,0.054902}%
\pgfsetstrokecolor{currentstroke}%
\pgfsetdash{}{0pt}%
\pgfpathmoveto{\pgfqpoint{1.603570in}{2.277203in}}%
\pgfpathcurveto{\pgfqpoint{1.614620in}{2.277203in}}{\pgfqpoint{1.625219in}{2.281594in}}{\pgfqpoint{1.633033in}{2.289407in}}%
\pgfpathcurveto{\pgfqpoint{1.640847in}{2.297221in}}{\pgfqpoint{1.645237in}{2.307820in}}{\pgfqpoint{1.645237in}{2.318870in}}%
\pgfpathcurveto{\pgfqpoint{1.645237in}{2.329920in}}{\pgfqpoint{1.640847in}{2.340519in}}{\pgfqpoint{1.633033in}{2.348333in}}%
\pgfpathcurveto{\pgfqpoint{1.625219in}{2.356146in}}{\pgfqpoint{1.614620in}{2.360537in}}{\pgfqpoint{1.603570in}{2.360537in}}%
\pgfpathcurveto{\pgfqpoint{1.592520in}{2.360537in}}{\pgfqpoint{1.581921in}{2.356146in}}{\pgfqpoint{1.574107in}{2.348333in}}%
\pgfpathcurveto{\pgfqpoint{1.566294in}{2.340519in}}{\pgfqpoint{1.561903in}{2.329920in}}{\pgfqpoint{1.561903in}{2.318870in}}%
\pgfpathcurveto{\pgfqpoint{1.561903in}{2.307820in}}{\pgfqpoint{1.566294in}{2.297221in}}{\pgfqpoint{1.574107in}{2.289407in}}%
\pgfpathcurveto{\pgfqpoint{1.581921in}{2.281594in}}{\pgfqpoint{1.592520in}{2.277203in}}{\pgfqpoint{1.603570in}{2.277203in}}%
\pgfpathclose%
\pgfusepath{stroke,fill}%
\end{pgfscope}%
\begin{pgfscope}%
\pgfpathrectangle{\pgfqpoint{0.787074in}{0.548769in}}{\pgfqpoint{5.062926in}{3.102590in}}%
\pgfusepath{clip}%
\pgfsetbuttcap%
\pgfsetroundjoin%
\definecolor{currentfill}{rgb}{1.000000,0.498039,0.054902}%
\pgfsetfillcolor{currentfill}%
\pgfsetlinewidth{1.003750pt}%
\definecolor{currentstroke}{rgb}{1.000000,0.498039,0.054902}%
\pgfsetstrokecolor{currentstroke}%
\pgfsetdash{}{0pt}%
\pgfpathmoveto{\pgfqpoint{1.486717in}{2.871516in}}%
\pgfpathcurveto{\pgfqpoint{1.497768in}{2.871516in}}{\pgfqpoint{1.508367in}{2.875907in}}{\pgfqpoint{1.516180in}{2.883720in}}%
\pgfpathcurveto{\pgfqpoint{1.523994in}{2.891534in}}{\pgfqpoint{1.528384in}{2.902133in}}{\pgfqpoint{1.528384in}{2.913183in}}%
\pgfpathcurveto{\pgfqpoint{1.528384in}{2.924233in}}{\pgfqpoint{1.523994in}{2.934832in}}{\pgfqpoint{1.516180in}{2.942646in}}%
\pgfpathcurveto{\pgfqpoint{1.508367in}{2.950459in}}{\pgfqpoint{1.497768in}{2.954850in}}{\pgfqpoint{1.486717in}{2.954850in}}%
\pgfpathcurveto{\pgfqpoint{1.475667in}{2.954850in}}{\pgfqpoint{1.465068in}{2.950459in}}{\pgfqpoint{1.457255in}{2.942646in}}%
\pgfpathcurveto{\pgfqpoint{1.449441in}{2.934832in}}{\pgfqpoint{1.445051in}{2.924233in}}{\pgfqpoint{1.445051in}{2.913183in}}%
\pgfpathcurveto{\pgfqpoint{1.445051in}{2.902133in}}{\pgfqpoint{1.449441in}{2.891534in}}{\pgfqpoint{1.457255in}{2.883720in}}%
\pgfpathcurveto{\pgfqpoint{1.465068in}{2.875907in}}{\pgfqpoint{1.475667in}{2.871516in}}{\pgfqpoint{1.486717in}{2.871516in}}%
\pgfpathclose%
\pgfusepath{stroke,fill}%
\end{pgfscope}%
\begin{pgfscope}%
\pgfpathrectangle{\pgfqpoint{0.787074in}{0.548769in}}{\pgfqpoint{5.062926in}{3.102590in}}%
\pgfusepath{clip}%
\pgfsetbuttcap%
\pgfsetroundjoin%
\definecolor{currentfill}{rgb}{0.121569,0.466667,0.705882}%
\pgfsetfillcolor{currentfill}%
\pgfsetlinewidth{1.003750pt}%
\definecolor{currentstroke}{rgb}{0.121569,0.466667,0.705882}%
\pgfsetstrokecolor{currentstroke}%
\pgfsetdash{}{0pt}%
\pgfpathmoveto{\pgfqpoint{1.070070in}{0.648134in}}%
\pgfpathcurveto{\pgfqpoint{1.081120in}{0.648134in}}{\pgfqpoint{1.091719in}{0.652524in}}{\pgfqpoint{1.099533in}{0.660338in}}%
\pgfpathcurveto{\pgfqpoint{1.107346in}{0.668152in}}{\pgfqpoint{1.111737in}{0.678751in}}{\pgfqpoint{1.111737in}{0.689801in}}%
\pgfpathcurveto{\pgfqpoint{1.111737in}{0.700851in}}{\pgfqpoint{1.107346in}{0.711450in}}{\pgfqpoint{1.099533in}{0.719264in}}%
\pgfpathcurveto{\pgfqpoint{1.091719in}{0.727077in}}{\pgfqpoint{1.081120in}{0.731467in}}{\pgfqpoint{1.070070in}{0.731467in}}%
\pgfpathcurveto{\pgfqpoint{1.059020in}{0.731467in}}{\pgfqpoint{1.048421in}{0.727077in}}{\pgfqpoint{1.040607in}{0.719264in}}%
\pgfpathcurveto{\pgfqpoint{1.032793in}{0.711450in}}{\pgfqpoint{1.028403in}{0.700851in}}{\pgfqpoint{1.028403in}{0.689801in}}%
\pgfpathcurveto{\pgfqpoint{1.028403in}{0.678751in}}{\pgfqpoint{1.032793in}{0.668152in}}{\pgfqpoint{1.040607in}{0.660338in}}%
\pgfpathcurveto{\pgfqpoint{1.048421in}{0.652524in}}{\pgfqpoint{1.059020in}{0.648134in}}{\pgfqpoint{1.070070in}{0.648134in}}%
\pgfpathclose%
\pgfusepath{stroke,fill}%
\end{pgfscope}%
\begin{pgfscope}%
\pgfpathrectangle{\pgfqpoint{0.787074in}{0.548769in}}{\pgfqpoint{5.062926in}{3.102590in}}%
\pgfusepath{clip}%
\pgfsetbuttcap%
\pgfsetroundjoin%
\definecolor{currentfill}{rgb}{0.121569,0.466667,0.705882}%
\pgfsetfillcolor{currentfill}%
\pgfsetlinewidth{1.003750pt}%
\definecolor{currentstroke}{rgb}{0.121569,0.466667,0.705882}%
\pgfsetstrokecolor{currentstroke}%
\pgfsetdash{}{0pt}%
\pgfpathmoveto{\pgfqpoint{1.986152in}{3.215961in}}%
\pgfpathcurveto{\pgfqpoint{1.997202in}{3.215961in}}{\pgfqpoint{2.007801in}{3.220351in}}{\pgfqpoint{2.015615in}{3.228165in}}%
\pgfpathcurveto{\pgfqpoint{2.023428in}{3.235978in}}{\pgfqpoint{2.027819in}{3.246577in}}{\pgfqpoint{2.027819in}{3.257627in}}%
\pgfpathcurveto{\pgfqpoint{2.027819in}{3.268677in}}{\pgfqpoint{2.023428in}{3.279276in}}{\pgfqpoint{2.015615in}{3.287090in}}%
\pgfpathcurveto{\pgfqpoint{2.007801in}{3.294904in}}{\pgfqpoint{1.997202in}{3.299294in}}{\pgfqpoint{1.986152in}{3.299294in}}%
\pgfpathcurveto{\pgfqpoint{1.975102in}{3.299294in}}{\pgfqpoint{1.964503in}{3.294904in}}{\pgfqpoint{1.956689in}{3.287090in}}%
\pgfpathcurveto{\pgfqpoint{1.948876in}{3.279276in}}{\pgfqpoint{1.944485in}{3.268677in}}{\pgfqpoint{1.944485in}{3.257627in}}%
\pgfpathcurveto{\pgfqpoint{1.944485in}{3.246577in}}{\pgfqpoint{1.948876in}{3.235978in}}{\pgfqpoint{1.956689in}{3.228165in}}%
\pgfpathcurveto{\pgfqpoint{1.964503in}{3.220351in}}{\pgfqpoint{1.975102in}{3.215961in}}{\pgfqpoint{1.986152in}{3.215961in}}%
\pgfpathclose%
\pgfusepath{stroke,fill}%
\end{pgfscope}%
\begin{pgfscope}%
\pgfpathrectangle{\pgfqpoint{0.787074in}{0.548769in}}{\pgfqpoint{5.062926in}{3.102590in}}%
\pgfusepath{clip}%
\pgfsetbuttcap%
\pgfsetroundjoin%
\definecolor{currentfill}{rgb}{1.000000,0.498039,0.054902}%
\pgfsetfillcolor{currentfill}%
\pgfsetlinewidth{1.003750pt}%
\definecolor{currentstroke}{rgb}{1.000000,0.498039,0.054902}%
\pgfsetstrokecolor{currentstroke}%
\pgfsetdash{}{0pt}%
\pgfpathmoveto{\pgfqpoint{1.460635in}{3.082101in}}%
\pgfpathcurveto{\pgfqpoint{1.471685in}{3.082101in}}{\pgfqpoint{1.482284in}{3.086492in}}{\pgfqpoint{1.490097in}{3.094305in}}%
\pgfpathcurveto{\pgfqpoint{1.497911in}{3.102119in}}{\pgfqpoint{1.502301in}{3.112718in}}{\pgfqpoint{1.502301in}{3.123768in}}%
\pgfpathcurveto{\pgfqpoint{1.502301in}{3.134818in}}{\pgfqpoint{1.497911in}{3.145417in}}{\pgfqpoint{1.490097in}{3.153231in}}%
\pgfpathcurveto{\pgfqpoint{1.482284in}{3.161044in}}{\pgfqpoint{1.471685in}{3.165435in}}{\pgfqpoint{1.460635in}{3.165435in}}%
\pgfpathcurveto{\pgfqpoint{1.449584in}{3.165435in}}{\pgfqpoint{1.438985in}{3.161044in}}{\pgfqpoint{1.431172in}{3.153231in}}%
\pgfpathcurveto{\pgfqpoint{1.423358in}{3.145417in}}{\pgfqpoint{1.418968in}{3.134818in}}{\pgfqpoint{1.418968in}{3.123768in}}%
\pgfpathcurveto{\pgfqpoint{1.418968in}{3.112718in}}{\pgfqpoint{1.423358in}{3.102119in}}{\pgfqpoint{1.431172in}{3.094305in}}%
\pgfpathcurveto{\pgfqpoint{1.438985in}{3.086492in}}{\pgfqpoint{1.449584in}{3.082101in}}{\pgfqpoint{1.460635in}{3.082101in}}%
\pgfpathclose%
\pgfusepath{stroke,fill}%
\end{pgfscope}%
\begin{pgfscope}%
\pgfpathrectangle{\pgfqpoint{0.787074in}{0.548769in}}{\pgfqpoint{5.062926in}{3.102590in}}%
\pgfusepath{clip}%
\pgfsetbuttcap%
\pgfsetroundjoin%
\definecolor{currentfill}{rgb}{0.121569,0.466667,0.705882}%
\pgfsetfillcolor{currentfill}%
\pgfsetlinewidth{1.003750pt}%
\definecolor{currentstroke}{rgb}{0.121569,0.466667,0.705882}%
\pgfsetstrokecolor{currentstroke}%
\pgfsetdash{}{0pt}%
\pgfpathmoveto{\pgfqpoint{1.692190in}{1.896734in}}%
\pgfpathcurveto{\pgfqpoint{1.703240in}{1.896734in}}{\pgfqpoint{1.713839in}{1.901124in}}{\pgfqpoint{1.721653in}{1.908938in}}%
\pgfpathcurveto{\pgfqpoint{1.729467in}{1.916751in}}{\pgfqpoint{1.733857in}{1.927350in}}{\pgfqpoint{1.733857in}{1.938401in}}%
\pgfpathcurveto{\pgfqpoint{1.733857in}{1.949451in}}{\pgfqpoint{1.729467in}{1.960050in}}{\pgfqpoint{1.721653in}{1.967863in}}%
\pgfpathcurveto{\pgfqpoint{1.713839in}{1.975677in}}{\pgfqpoint{1.703240in}{1.980067in}}{\pgfqpoint{1.692190in}{1.980067in}}%
\pgfpathcurveto{\pgfqpoint{1.681140in}{1.980067in}}{\pgfqpoint{1.670541in}{1.975677in}}{\pgfqpoint{1.662727in}{1.967863in}}%
\pgfpathcurveto{\pgfqpoint{1.654914in}{1.960050in}}{\pgfqpoint{1.650524in}{1.949451in}}{\pgfqpoint{1.650524in}{1.938401in}}%
\pgfpathcurveto{\pgfqpoint{1.650524in}{1.927350in}}{\pgfqpoint{1.654914in}{1.916751in}}{\pgfqpoint{1.662727in}{1.908938in}}%
\pgfpathcurveto{\pgfqpoint{1.670541in}{1.901124in}}{\pgfqpoint{1.681140in}{1.896734in}}{\pgfqpoint{1.692190in}{1.896734in}}%
\pgfpathclose%
\pgfusepath{stroke,fill}%
\end{pgfscope}%
\begin{pgfscope}%
\pgfpathrectangle{\pgfqpoint{0.787074in}{0.548769in}}{\pgfqpoint{5.062926in}{3.102590in}}%
\pgfusepath{clip}%
\pgfsetbuttcap%
\pgfsetroundjoin%
\definecolor{currentfill}{rgb}{1.000000,0.498039,0.054902}%
\pgfsetfillcolor{currentfill}%
\pgfsetlinewidth{1.003750pt}%
\definecolor{currentstroke}{rgb}{1.000000,0.498039,0.054902}%
\pgfsetstrokecolor{currentstroke}%
\pgfsetdash{}{0pt}%
\pgfpathmoveto{\pgfqpoint{1.222341in}{3.048752in}}%
\pgfpathcurveto{\pgfqpoint{1.233391in}{3.048752in}}{\pgfqpoint{1.243990in}{3.053142in}}{\pgfqpoint{1.251803in}{3.060956in}}%
\pgfpathcurveto{\pgfqpoint{1.259617in}{3.068769in}}{\pgfqpoint{1.264007in}{3.079368in}}{\pgfqpoint{1.264007in}{3.090419in}}%
\pgfpathcurveto{\pgfqpoint{1.264007in}{3.101469in}}{\pgfqpoint{1.259617in}{3.112068in}}{\pgfqpoint{1.251803in}{3.119881in}}%
\pgfpathcurveto{\pgfqpoint{1.243990in}{3.127695in}}{\pgfqpoint{1.233391in}{3.132085in}}{\pgfqpoint{1.222341in}{3.132085in}}%
\pgfpathcurveto{\pgfqpoint{1.211290in}{3.132085in}}{\pgfqpoint{1.200691in}{3.127695in}}{\pgfqpoint{1.192878in}{3.119881in}}%
\pgfpathcurveto{\pgfqpoint{1.185064in}{3.112068in}}{\pgfqpoint{1.180674in}{3.101469in}}{\pgfqpoint{1.180674in}{3.090419in}}%
\pgfpathcurveto{\pgfqpoint{1.180674in}{3.079368in}}{\pgfqpoint{1.185064in}{3.068769in}}{\pgfqpoint{1.192878in}{3.060956in}}%
\pgfpathcurveto{\pgfqpoint{1.200691in}{3.053142in}}{\pgfqpoint{1.211290in}{3.048752in}}{\pgfqpoint{1.222341in}{3.048752in}}%
\pgfpathclose%
\pgfusepath{stroke,fill}%
\end{pgfscope}%
\begin{pgfscope}%
\pgfpathrectangle{\pgfqpoint{0.787074in}{0.548769in}}{\pgfqpoint{5.062926in}{3.102590in}}%
\pgfusepath{clip}%
\pgfsetbuttcap%
\pgfsetroundjoin%
\definecolor{currentfill}{rgb}{1.000000,0.498039,0.054902}%
\pgfsetfillcolor{currentfill}%
\pgfsetlinewidth{1.003750pt}%
\definecolor{currentstroke}{rgb}{1.000000,0.498039,0.054902}%
\pgfsetstrokecolor{currentstroke}%
\pgfsetdash{}{0pt}%
\pgfpathmoveto{\pgfqpoint{1.491915in}{2.964483in}}%
\pgfpathcurveto{\pgfqpoint{1.502965in}{2.964483in}}{\pgfqpoint{1.513564in}{2.968874in}}{\pgfqpoint{1.521378in}{2.976687in}}%
\pgfpathcurveto{\pgfqpoint{1.529191in}{2.984501in}}{\pgfqpoint{1.533581in}{2.995100in}}{\pgfqpoint{1.533581in}{3.006150in}}%
\pgfpathcurveto{\pgfqpoint{1.533581in}{3.017200in}}{\pgfqpoint{1.529191in}{3.027799in}}{\pgfqpoint{1.521378in}{3.035613in}}%
\pgfpathcurveto{\pgfqpoint{1.513564in}{3.043426in}}{\pgfqpoint{1.502965in}{3.047817in}}{\pgfqpoint{1.491915in}{3.047817in}}%
\pgfpathcurveto{\pgfqpoint{1.480865in}{3.047817in}}{\pgfqpoint{1.470266in}{3.043426in}}{\pgfqpoint{1.462452in}{3.035613in}}%
\pgfpathcurveto{\pgfqpoint{1.454638in}{3.027799in}}{\pgfqpoint{1.450248in}{3.017200in}}{\pgfqpoint{1.450248in}{3.006150in}}%
\pgfpathcurveto{\pgfqpoint{1.450248in}{2.995100in}}{\pgfqpoint{1.454638in}{2.984501in}}{\pgfqpoint{1.462452in}{2.976687in}}%
\pgfpathcurveto{\pgfqpoint{1.470266in}{2.968874in}}{\pgfqpoint{1.480865in}{2.964483in}}{\pgfqpoint{1.491915in}{2.964483in}}%
\pgfpathclose%
\pgfusepath{stroke,fill}%
\end{pgfscope}%
\begin{pgfscope}%
\pgfpathrectangle{\pgfqpoint{0.787074in}{0.548769in}}{\pgfqpoint{5.062926in}{3.102590in}}%
\pgfusepath{clip}%
\pgfsetbuttcap%
\pgfsetroundjoin%
\definecolor{currentfill}{rgb}{1.000000,0.498039,0.054902}%
\pgfsetfillcolor{currentfill}%
\pgfsetlinewidth{1.003750pt}%
\definecolor{currentstroke}{rgb}{1.000000,0.498039,0.054902}%
\pgfsetstrokecolor{currentstroke}%
\pgfsetdash{}{0pt}%
\pgfpathmoveto{\pgfqpoint{1.396487in}{2.843196in}}%
\pgfpathcurveto{\pgfqpoint{1.407537in}{2.843196in}}{\pgfqpoint{1.418136in}{2.847587in}}{\pgfqpoint{1.425950in}{2.855400in}}%
\pgfpathcurveto{\pgfqpoint{1.433763in}{2.863214in}}{\pgfqpoint{1.438154in}{2.873813in}}{\pgfqpoint{1.438154in}{2.884863in}}%
\pgfpathcurveto{\pgfqpoint{1.438154in}{2.895913in}}{\pgfqpoint{1.433763in}{2.906512in}}{\pgfqpoint{1.425950in}{2.914326in}}%
\pgfpathcurveto{\pgfqpoint{1.418136in}{2.922140in}}{\pgfqpoint{1.407537in}{2.926530in}}{\pgfqpoint{1.396487in}{2.926530in}}%
\pgfpathcurveto{\pgfqpoint{1.385437in}{2.926530in}}{\pgfqpoint{1.374838in}{2.922140in}}{\pgfqpoint{1.367024in}{2.914326in}}%
\pgfpathcurveto{\pgfqpoint{1.359211in}{2.906512in}}{\pgfqpoint{1.354820in}{2.895913in}}{\pgfqpoint{1.354820in}{2.884863in}}%
\pgfpathcurveto{\pgfqpoint{1.354820in}{2.873813in}}{\pgfqpoint{1.359211in}{2.863214in}}{\pgfqpoint{1.367024in}{2.855400in}}%
\pgfpathcurveto{\pgfqpoint{1.374838in}{2.847587in}}{\pgfqpoint{1.385437in}{2.843196in}}{\pgfqpoint{1.396487in}{2.843196in}}%
\pgfpathclose%
\pgfusepath{stroke,fill}%
\end{pgfscope}%
\begin{pgfscope}%
\pgfpathrectangle{\pgfqpoint{0.787074in}{0.548769in}}{\pgfqpoint{5.062926in}{3.102590in}}%
\pgfusepath{clip}%
\pgfsetbuttcap%
\pgfsetroundjoin%
\definecolor{currentfill}{rgb}{1.000000,0.498039,0.054902}%
\pgfsetfillcolor{currentfill}%
\pgfsetlinewidth{1.003750pt}%
\definecolor{currentstroke}{rgb}{1.000000,0.498039,0.054902}%
\pgfsetstrokecolor{currentstroke}%
\pgfsetdash{}{0pt}%
\pgfpathmoveto{\pgfqpoint{1.615232in}{2.715747in}}%
\pgfpathcurveto{\pgfqpoint{1.626282in}{2.715747in}}{\pgfqpoint{1.636881in}{2.720137in}}{\pgfqpoint{1.644695in}{2.727951in}}%
\pgfpathcurveto{\pgfqpoint{1.652509in}{2.735764in}}{\pgfqpoint{1.656899in}{2.746363in}}{\pgfqpoint{1.656899in}{2.757413in}}%
\pgfpathcurveto{\pgfqpoint{1.656899in}{2.768463in}}{\pgfqpoint{1.652509in}{2.779062in}}{\pgfqpoint{1.644695in}{2.786876in}}%
\pgfpathcurveto{\pgfqpoint{1.636881in}{2.794690in}}{\pgfqpoint{1.626282in}{2.799080in}}{\pgfqpoint{1.615232in}{2.799080in}}%
\pgfpathcurveto{\pgfqpoint{1.604182in}{2.799080in}}{\pgfqpoint{1.593583in}{2.794690in}}{\pgfqpoint{1.585770in}{2.786876in}}%
\pgfpathcurveto{\pgfqpoint{1.577956in}{2.779062in}}{\pgfqpoint{1.573566in}{2.768463in}}{\pgfqpoint{1.573566in}{2.757413in}}%
\pgfpathcurveto{\pgfqpoint{1.573566in}{2.746363in}}{\pgfqpoint{1.577956in}{2.735764in}}{\pgfqpoint{1.585770in}{2.727951in}}%
\pgfpathcurveto{\pgfqpoint{1.593583in}{2.720137in}}{\pgfqpoint{1.604182in}{2.715747in}}{\pgfqpoint{1.615232in}{2.715747in}}%
\pgfpathclose%
\pgfusepath{stroke,fill}%
\end{pgfscope}%
\begin{pgfscope}%
\pgfpathrectangle{\pgfqpoint{0.787074in}{0.548769in}}{\pgfqpoint{5.062926in}{3.102590in}}%
\pgfusepath{clip}%
\pgfsetbuttcap%
\pgfsetroundjoin%
\definecolor{currentfill}{rgb}{0.121569,0.466667,0.705882}%
\pgfsetfillcolor{currentfill}%
\pgfsetlinewidth{1.003750pt}%
\definecolor{currentstroke}{rgb}{0.121569,0.466667,0.705882}%
\pgfsetstrokecolor{currentstroke}%
\pgfsetdash{}{0pt}%
\pgfpathmoveto{\pgfqpoint{1.017207in}{0.648129in}}%
\pgfpathcurveto{\pgfqpoint{1.028257in}{0.648129in}}{\pgfqpoint{1.038856in}{0.652519in}}{\pgfqpoint{1.046670in}{0.660333in}}%
\pgfpathcurveto{\pgfqpoint{1.054483in}{0.668146in}}{\pgfqpoint{1.058874in}{0.678745in}}{\pgfqpoint{1.058874in}{0.689796in}}%
\pgfpathcurveto{\pgfqpoint{1.058874in}{0.700846in}}{\pgfqpoint{1.054483in}{0.711445in}}{\pgfqpoint{1.046670in}{0.719258in}}%
\pgfpathcurveto{\pgfqpoint{1.038856in}{0.727072in}}{\pgfqpoint{1.028257in}{0.731462in}}{\pgfqpoint{1.017207in}{0.731462in}}%
\pgfpathcurveto{\pgfqpoint{1.006157in}{0.731462in}}{\pgfqpoint{0.995558in}{0.727072in}}{\pgfqpoint{0.987744in}{0.719258in}}%
\pgfpathcurveto{\pgfqpoint{0.979930in}{0.711445in}}{\pgfqpoint{0.975540in}{0.700846in}}{\pgfqpoint{0.975540in}{0.689796in}}%
\pgfpathcurveto{\pgfqpoint{0.975540in}{0.678745in}}{\pgfqpoint{0.979930in}{0.668146in}}{\pgfqpoint{0.987744in}{0.660333in}}%
\pgfpathcurveto{\pgfqpoint{0.995558in}{0.652519in}}{\pgfqpoint{1.006157in}{0.648129in}}{\pgfqpoint{1.017207in}{0.648129in}}%
\pgfpathclose%
\pgfusepath{stroke,fill}%
\end{pgfscope}%
\begin{pgfscope}%
\pgfpathrectangle{\pgfqpoint{0.787074in}{0.548769in}}{\pgfqpoint{5.062926in}{3.102590in}}%
\pgfusepath{clip}%
\pgfsetbuttcap%
\pgfsetroundjoin%
\definecolor{currentfill}{rgb}{0.121569,0.466667,0.705882}%
\pgfsetfillcolor{currentfill}%
\pgfsetlinewidth{1.003750pt}%
\definecolor{currentstroke}{rgb}{0.121569,0.466667,0.705882}%
\pgfsetstrokecolor{currentstroke}%
\pgfsetdash{}{0pt}%
\pgfpathmoveto{\pgfqpoint{1.061082in}{0.665742in}}%
\pgfpathcurveto{\pgfqpoint{1.072132in}{0.665742in}}{\pgfqpoint{1.082731in}{0.670132in}}{\pgfqpoint{1.090544in}{0.677946in}}%
\pgfpathcurveto{\pgfqpoint{1.098358in}{0.685760in}}{\pgfqpoint{1.102748in}{0.696359in}}{\pgfqpoint{1.102748in}{0.707409in}}%
\pgfpathcurveto{\pgfqpoint{1.102748in}{0.718459in}}{\pgfqpoint{1.098358in}{0.729058in}}{\pgfqpoint{1.090544in}{0.736872in}}%
\pgfpathcurveto{\pgfqpoint{1.082731in}{0.744685in}}{\pgfqpoint{1.072132in}{0.749076in}}{\pgfqpoint{1.061082in}{0.749076in}}%
\pgfpathcurveto{\pgfqpoint{1.050031in}{0.749076in}}{\pgfqpoint{1.039432in}{0.744685in}}{\pgfqpoint{1.031619in}{0.736872in}}%
\pgfpathcurveto{\pgfqpoint{1.023805in}{0.729058in}}{\pgfqpoint{1.019415in}{0.718459in}}{\pgfqpoint{1.019415in}{0.707409in}}%
\pgfpathcurveto{\pgfqpoint{1.019415in}{0.696359in}}{\pgfqpoint{1.023805in}{0.685760in}}{\pgfqpoint{1.031619in}{0.677946in}}%
\pgfpathcurveto{\pgfqpoint{1.039432in}{0.670132in}}{\pgfqpoint{1.050031in}{0.665742in}}{\pgfqpoint{1.061082in}{0.665742in}}%
\pgfpathclose%
\pgfusepath{stroke,fill}%
\end{pgfscope}%
\begin{pgfscope}%
\pgfpathrectangle{\pgfqpoint{0.787074in}{0.548769in}}{\pgfqpoint{5.062926in}{3.102590in}}%
\pgfusepath{clip}%
\pgfsetbuttcap%
\pgfsetroundjoin%
\definecolor{currentfill}{rgb}{0.121569,0.466667,0.705882}%
\pgfsetfillcolor{currentfill}%
\pgfsetlinewidth{1.003750pt}%
\definecolor{currentstroke}{rgb}{0.121569,0.466667,0.705882}%
\pgfsetstrokecolor{currentstroke}%
\pgfsetdash{}{0pt}%
\pgfpathmoveto{\pgfqpoint{4.834234in}{1.038852in}}%
\pgfpathcurveto{\pgfqpoint{4.845284in}{1.038852in}}{\pgfqpoint{4.855883in}{1.043242in}}{\pgfqpoint{4.863697in}{1.051055in}}%
\pgfpathcurveto{\pgfqpoint{4.871510in}{1.058869in}}{\pgfqpoint{4.875900in}{1.069468in}}{\pgfqpoint{4.875900in}{1.080518in}}%
\pgfpathcurveto{\pgfqpoint{4.875900in}{1.091568in}}{\pgfqpoint{4.871510in}{1.102167in}}{\pgfqpoint{4.863697in}{1.109981in}}%
\pgfpathcurveto{\pgfqpoint{4.855883in}{1.117795in}}{\pgfqpoint{4.845284in}{1.122185in}}{\pgfqpoint{4.834234in}{1.122185in}}%
\pgfpathcurveto{\pgfqpoint{4.823184in}{1.122185in}}{\pgfqpoint{4.812585in}{1.117795in}}{\pgfqpoint{4.804771in}{1.109981in}}%
\pgfpathcurveto{\pgfqpoint{4.796957in}{1.102167in}}{\pgfqpoint{4.792567in}{1.091568in}}{\pgfqpoint{4.792567in}{1.080518in}}%
\pgfpathcurveto{\pgfqpoint{4.792567in}{1.069468in}}{\pgfqpoint{4.796957in}{1.058869in}}{\pgfqpoint{4.804771in}{1.051055in}}%
\pgfpathcurveto{\pgfqpoint{4.812585in}{1.043242in}}{\pgfqpoint{4.823184in}{1.038852in}}{\pgfqpoint{4.834234in}{1.038852in}}%
\pgfpathclose%
\pgfusepath{stroke,fill}%
\end{pgfscope}%
\begin{pgfscope}%
\pgfpathrectangle{\pgfqpoint{0.787074in}{0.548769in}}{\pgfqpoint{5.062926in}{3.102590in}}%
\pgfusepath{clip}%
\pgfsetbuttcap%
\pgfsetroundjoin%
\definecolor{currentfill}{rgb}{1.000000,0.498039,0.054902}%
\pgfsetfillcolor{currentfill}%
\pgfsetlinewidth{1.003750pt}%
\definecolor{currentstroke}{rgb}{1.000000,0.498039,0.054902}%
\pgfsetstrokecolor{currentstroke}%
\pgfsetdash{}{0pt}%
\pgfpathmoveto{\pgfqpoint{1.498638in}{2.996726in}}%
\pgfpathcurveto{\pgfqpoint{1.509688in}{2.996726in}}{\pgfqpoint{1.520287in}{3.001116in}}{\pgfqpoint{1.528101in}{3.008930in}}%
\pgfpathcurveto{\pgfqpoint{1.535914in}{3.016743in}}{\pgfqpoint{1.540304in}{3.027342in}}{\pgfqpoint{1.540304in}{3.038392in}}%
\pgfpathcurveto{\pgfqpoint{1.540304in}{3.049443in}}{\pgfqpoint{1.535914in}{3.060042in}}{\pgfqpoint{1.528101in}{3.067855in}}%
\pgfpathcurveto{\pgfqpoint{1.520287in}{3.075669in}}{\pgfqpoint{1.509688in}{3.080059in}}{\pgfqpoint{1.498638in}{3.080059in}}%
\pgfpathcurveto{\pgfqpoint{1.487588in}{3.080059in}}{\pgfqpoint{1.476989in}{3.075669in}}{\pgfqpoint{1.469175in}{3.067855in}}%
\pgfpathcurveto{\pgfqpoint{1.461361in}{3.060042in}}{\pgfqpoint{1.456971in}{3.049443in}}{\pgfqpoint{1.456971in}{3.038392in}}%
\pgfpathcurveto{\pgfqpoint{1.456971in}{3.027342in}}{\pgfqpoint{1.461361in}{3.016743in}}{\pgfqpoint{1.469175in}{3.008930in}}%
\pgfpathcurveto{\pgfqpoint{1.476989in}{3.001116in}}{\pgfqpoint{1.487588in}{2.996726in}}{\pgfqpoint{1.498638in}{2.996726in}}%
\pgfpathclose%
\pgfusepath{stroke,fill}%
\end{pgfscope}%
\begin{pgfscope}%
\pgfpathrectangle{\pgfqpoint{0.787074in}{0.548769in}}{\pgfqpoint{5.062926in}{3.102590in}}%
\pgfusepath{clip}%
\pgfsetbuttcap%
\pgfsetroundjoin%
\definecolor{currentfill}{rgb}{0.121569,0.466667,0.705882}%
\pgfsetfillcolor{currentfill}%
\pgfsetlinewidth{1.003750pt}%
\definecolor{currentstroke}{rgb}{0.121569,0.466667,0.705882}%
\pgfsetstrokecolor{currentstroke}%
\pgfsetdash{}{0pt}%
\pgfpathmoveto{\pgfqpoint{1.096184in}{0.648130in}}%
\pgfpathcurveto{\pgfqpoint{1.107234in}{0.648130in}}{\pgfqpoint{1.117833in}{0.652521in}}{\pgfqpoint{1.125646in}{0.660334in}}%
\pgfpathcurveto{\pgfqpoint{1.133460in}{0.668148in}}{\pgfqpoint{1.137850in}{0.678747in}}{\pgfqpoint{1.137850in}{0.689797in}}%
\pgfpathcurveto{\pgfqpoint{1.137850in}{0.700847in}}{\pgfqpoint{1.133460in}{0.711446in}}{\pgfqpoint{1.125646in}{0.719260in}}%
\pgfpathcurveto{\pgfqpoint{1.117833in}{0.727073in}}{\pgfqpoint{1.107234in}{0.731464in}}{\pgfqpoint{1.096184in}{0.731464in}}%
\pgfpathcurveto{\pgfqpoint{1.085133in}{0.731464in}}{\pgfqpoint{1.074534in}{0.727073in}}{\pgfqpoint{1.066721in}{0.719260in}}%
\pgfpathcurveto{\pgfqpoint{1.058907in}{0.711446in}}{\pgfqpoint{1.054517in}{0.700847in}}{\pgfqpoint{1.054517in}{0.689797in}}%
\pgfpathcurveto{\pgfqpoint{1.054517in}{0.678747in}}{\pgfqpoint{1.058907in}{0.668148in}}{\pgfqpoint{1.066721in}{0.660334in}}%
\pgfpathcurveto{\pgfqpoint{1.074534in}{0.652521in}}{\pgfqpoint{1.085133in}{0.648130in}}{\pgfqpoint{1.096184in}{0.648130in}}%
\pgfpathclose%
\pgfusepath{stroke,fill}%
\end{pgfscope}%
\begin{pgfscope}%
\pgfpathrectangle{\pgfqpoint{0.787074in}{0.548769in}}{\pgfqpoint{5.062926in}{3.102590in}}%
\pgfusepath{clip}%
\pgfsetbuttcap%
\pgfsetroundjoin%
\definecolor{currentfill}{rgb}{0.121569,0.466667,0.705882}%
\pgfsetfillcolor{currentfill}%
\pgfsetlinewidth{1.003750pt}%
\definecolor{currentstroke}{rgb}{0.121569,0.466667,0.705882}%
\pgfsetstrokecolor{currentstroke}%
\pgfsetdash{}{0pt}%
\pgfpathmoveto{\pgfqpoint{1.023884in}{0.648134in}}%
\pgfpathcurveto{\pgfqpoint{1.034934in}{0.648134in}}{\pgfqpoint{1.045533in}{0.652524in}}{\pgfqpoint{1.053346in}{0.660338in}}%
\pgfpathcurveto{\pgfqpoint{1.061160in}{0.668152in}}{\pgfqpoint{1.065550in}{0.678751in}}{\pgfqpoint{1.065550in}{0.689801in}}%
\pgfpathcurveto{\pgfqpoint{1.065550in}{0.700851in}}{\pgfqpoint{1.061160in}{0.711450in}}{\pgfqpoint{1.053346in}{0.719264in}}%
\pgfpathcurveto{\pgfqpoint{1.045533in}{0.727077in}}{\pgfqpoint{1.034934in}{0.731467in}}{\pgfqpoint{1.023884in}{0.731467in}}%
\pgfpathcurveto{\pgfqpoint{1.012833in}{0.731467in}}{\pgfqpoint{1.002234in}{0.727077in}}{\pgfqpoint{0.994421in}{0.719264in}}%
\pgfpathcurveto{\pgfqpoint{0.986607in}{0.711450in}}{\pgfqpoint{0.982217in}{0.700851in}}{\pgfqpoint{0.982217in}{0.689801in}}%
\pgfpathcurveto{\pgfqpoint{0.982217in}{0.678751in}}{\pgfqpoint{0.986607in}{0.668152in}}{\pgfqpoint{0.994421in}{0.660338in}}%
\pgfpathcurveto{\pgfqpoint{1.002234in}{0.652524in}}{\pgfqpoint{1.012833in}{0.648134in}}{\pgfqpoint{1.023884in}{0.648134in}}%
\pgfpathclose%
\pgfusepath{stroke,fill}%
\end{pgfscope}%
\begin{pgfscope}%
\pgfpathrectangle{\pgfqpoint{0.787074in}{0.548769in}}{\pgfqpoint{5.062926in}{3.102590in}}%
\pgfusepath{clip}%
\pgfsetbuttcap%
\pgfsetroundjoin%
\definecolor{currentfill}{rgb}{0.839216,0.152941,0.156863}%
\pgfsetfillcolor{currentfill}%
\pgfsetlinewidth{1.003750pt}%
\definecolor{currentstroke}{rgb}{0.839216,0.152941,0.156863}%
\pgfsetstrokecolor{currentstroke}%
\pgfsetdash{}{0pt}%
\pgfpathmoveto{\pgfqpoint{1.241635in}{3.468665in}}%
\pgfpathcurveto{\pgfqpoint{1.252685in}{3.468665in}}{\pgfqpoint{1.263284in}{3.473055in}}{\pgfqpoint{1.271098in}{3.480869in}}%
\pgfpathcurveto{\pgfqpoint{1.278911in}{3.488683in}}{\pgfqpoint{1.283302in}{3.499282in}}{\pgfqpoint{1.283302in}{3.510332in}}%
\pgfpathcurveto{\pgfqpoint{1.283302in}{3.521382in}}{\pgfqpoint{1.278911in}{3.531981in}}{\pgfqpoint{1.271098in}{3.539795in}}%
\pgfpathcurveto{\pgfqpoint{1.263284in}{3.547608in}}{\pgfqpoint{1.252685in}{3.551998in}}{\pgfqpoint{1.241635in}{3.551998in}}%
\pgfpathcurveto{\pgfqpoint{1.230585in}{3.551998in}}{\pgfqpoint{1.219986in}{3.547608in}}{\pgfqpoint{1.212172in}{3.539795in}}%
\pgfpathcurveto{\pgfqpoint{1.204359in}{3.531981in}}{\pgfqpoint{1.199968in}{3.521382in}}{\pgfqpoint{1.199968in}{3.510332in}}%
\pgfpathcurveto{\pgfqpoint{1.199968in}{3.499282in}}{\pgfqpoint{1.204359in}{3.488683in}}{\pgfqpoint{1.212172in}{3.480869in}}%
\pgfpathcurveto{\pgfqpoint{1.219986in}{3.473055in}}{\pgfqpoint{1.230585in}{3.468665in}}{\pgfqpoint{1.241635in}{3.468665in}}%
\pgfpathclose%
\pgfusepath{stroke,fill}%
\end{pgfscope}%
\begin{pgfscope}%
\pgfpathrectangle{\pgfqpoint{0.787074in}{0.548769in}}{\pgfqpoint{5.062926in}{3.102590in}}%
\pgfusepath{clip}%
\pgfsetbuttcap%
\pgfsetroundjoin%
\definecolor{currentfill}{rgb}{0.121569,0.466667,0.705882}%
\pgfsetfillcolor{currentfill}%
\pgfsetlinewidth{1.003750pt}%
\definecolor{currentstroke}{rgb}{0.121569,0.466667,0.705882}%
\pgfsetstrokecolor{currentstroke}%
\pgfsetdash{}{0pt}%
\pgfpathmoveto{\pgfqpoint{2.161682in}{3.009766in}}%
\pgfpathcurveto{\pgfqpoint{2.172732in}{3.009766in}}{\pgfqpoint{2.183331in}{3.014156in}}{\pgfqpoint{2.191144in}{3.021970in}}%
\pgfpathcurveto{\pgfqpoint{2.198958in}{3.029783in}}{\pgfqpoint{2.203348in}{3.040383in}}{\pgfqpoint{2.203348in}{3.051433in}}%
\pgfpathcurveto{\pgfqpoint{2.203348in}{3.062483in}}{\pgfqpoint{2.198958in}{3.073082in}}{\pgfqpoint{2.191144in}{3.080895in}}%
\pgfpathcurveto{\pgfqpoint{2.183331in}{3.088709in}}{\pgfqpoint{2.172732in}{3.093099in}}{\pgfqpoint{2.161682in}{3.093099in}}%
\pgfpathcurveto{\pgfqpoint{2.150631in}{3.093099in}}{\pgfqpoint{2.140032in}{3.088709in}}{\pgfqpoint{2.132219in}{3.080895in}}%
\pgfpathcurveto{\pgfqpoint{2.124405in}{3.073082in}}{\pgfqpoint{2.120015in}{3.062483in}}{\pgfqpoint{2.120015in}{3.051433in}}%
\pgfpathcurveto{\pgfqpoint{2.120015in}{3.040383in}}{\pgfqpoint{2.124405in}{3.029783in}}{\pgfqpoint{2.132219in}{3.021970in}}%
\pgfpathcurveto{\pgfqpoint{2.140032in}{3.014156in}}{\pgfqpoint{2.150631in}{3.009766in}}{\pgfqpoint{2.161682in}{3.009766in}}%
\pgfpathclose%
\pgfusepath{stroke,fill}%
\end{pgfscope}%
\begin{pgfscope}%
\pgfpathrectangle{\pgfqpoint{0.787074in}{0.548769in}}{\pgfqpoint{5.062926in}{3.102590in}}%
\pgfusepath{clip}%
\pgfsetbuttcap%
\pgfsetroundjoin%
\definecolor{currentfill}{rgb}{1.000000,0.498039,0.054902}%
\pgfsetfillcolor{currentfill}%
\pgfsetlinewidth{1.003750pt}%
\definecolor{currentstroke}{rgb}{1.000000,0.498039,0.054902}%
\pgfsetstrokecolor{currentstroke}%
\pgfsetdash{}{0pt}%
\pgfpathmoveto{\pgfqpoint{1.835839in}{3.140942in}}%
\pgfpathcurveto{\pgfqpoint{1.846889in}{3.140942in}}{\pgfqpoint{1.857488in}{3.145332in}}{\pgfqpoint{1.865301in}{3.153146in}}%
\pgfpathcurveto{\pgfqpoint{1.873115in}{3.160959in}}{\pgfqpoint{1.877505in}{3.171558in}}{\pgfqpoint{1.877505in}{3.182609in}}%
\pgfpathcurveto{\pgfqpoint{1.877505in}{3.193659in}}{\pgfqpoint{1.873115in}{3.204258in}}{\pgfqpoint{1.865301in}{3.212071in}}%
\pgfpathcurveto{\pgfqpoint{1.857488in}{3.219885in}}{\pgfqpoint{1.846889in}{3.224275in}}{\pgfqpoint{1.835839in}{3.224275in}}%
\pgfpathcurveto{\pgfqpoint{1.824788in}{3.224275in}}{\pgfqpoint{1.814189in}{3.219885in}}{\pgfqpoint{1.806376in}{3.212071in}}%
\pgfpathcurveto{\pgfqpoint{1.798562in}{3.204258in}}{\pgfqpoint{1.794172in}{3.193659in}}{\pgfqpoint{1.794172in}{3.182609in}}%
\pgfpathcurveto{\pgfqpoint{1.794172in}{3.171558in}}{\pgfqpoint{1.798562in}{3.160959in}}{\pgfqpoint{1.806376in}{3.153146in}}%
\pgfpathcurveto{\pgfqpoint{1.814189in}{3.145332in}}{\pgfqpoint{1.824788in}{3.140942in}}{\pgfqpoint{1.835839in}{3.140942in}}%
\pgfpathclose%
\pgfusepath{stroke,fill}%
\end{pgfscope}%
\begin{pgfscope}%
\pgfpathrectangle{\pgfqpoint{0.787074in}{0.548769in}}{\pgfqpoint{5.062926in}{3.102590in}}%
\pgfusepath{clip}%
\pgfsetbuttcap%
\pgfsetroundjoin%
\definecolor{currentfill}{rgb}{1.000000,0.498039,0.054902}%
\pgfsetfillcolor{currentfill}%
\pgfsetlinewidth{1.003750pt}%
\definecolor{currentstroke}{rgb}{1.000000,0.498039,0.054902}%
\pgfsetstrokecolor{currentstroke}%
\pgfsetdash{}{0pt}%
\pgfpathmoveto{\pgfqpoint{1.455846in}{2.861427in}}%
\pgfpathcurveto{\pgfqpoint{1.466896in}{2.861427in}}{\pgfqpoint{1.477495in}{2.865817in}}{\pgfqpoint{1.485308in}{2.873631in}}%
\pgfpathcurveto{\pgfqpoint{1.493122in}{2.881444in}}{\pgfqpoint{1.497512in}{2.892043in}}{\pgfqpoint{1.497512in}{2.903094in}}%
\pgfpathcurveto{\pgfqpoint{1.497512in}{2.914144in}}{\pgfqpoint{1.493122in}{2.924743in}}{\pgfqpoint{1.485308in}{2.932556in}}%
\pgfpathcurveto{\pgfqpoint{1.477495in}{2.940370in}}{\pgfqpoint{1.466896in}{2.944760in}}{\pgfqpoint{1.455846in}{2.944760in}}%
\pgfpathcurveto{\pgfqpoint{1.444796in}{2.944760in}}{\pgfqpoint{1.434197in}{2.940370in}}{\pgfqpoint{1.426383in}{2.932556in}}%
\pgfpathcurveto{\pgfqpoint{1.418569in}{2.924743in}}{\pgfqpoint{1.414179in}{2.914144in}}{\pgfqpoint{1.414179in}{2.903094in}}%
\pgfpathcurveto{\pgfqpoint{1.414179in}{2.892043in}}{\pgfqpoint{1.418569in}{2.881444in}}{\pgfqpoint{1.426383in}{2.873631in}}%
\pgfpathcurveto{\pgfqpoint{1.434197in}{2.865817in}}{\pgfqpoint{1.444796in}{2.861427in}}{\pgfqpoint{1.455846in}{2.861427in}}%
\pgfpathclose%
\pgfusepath{stroke,fill}%
\end{pgfscope}%
\begin{pgfscope}%
\pgfpathrectangle{\pgfqpoint{0.787074in}{0.548769in}}{\pgfqpoint{5.062926in}{3.102590in}}%
\pgfusepath{clip}%
\pgfsetbuttcap%
\pgfsetroundjoin%
\definecolor{currentfill}{rgb}{1.000000,0.498039,0.054902}%
\pgfsetfillcolor{currentfill}%
\pgfsetlinewidth{1.003750pt}%
\definecolor{currentstroke}{rgb}{1.000000,0.498039,0.054902}%
\pgfsetstrokecolor{currentstroke}%
\pgfsetdash{}{0pt}%
\pgfpathmoveto{\pgfqpoint{1.906131in}{3.035300in}}%
\pgfpathcurveto{\pgfqpoint{1.917181in}{3.035300in}}{\pgfqpoint{1.927780in}{3.039690in}}{\pgfqpoint{1.935594in}{3.047504in}}%
\pgfpathcurveto{\pgfqpoint{1.943408in}{3.055317in}}{\pgfqpoint{1.947798in}{3.065916in}}{\pgfqpoint{1.947798in}{3.076967in}}%
\pgfpathcurveto{\pgfqpoint{1.947798in}{3.088017in}}{\pgfqpoint{1.943408in}{3.098616in}}{\pgfqpoint{1.935594in}{3.106429in}}%
\pgfpathcurveto{\pgfqpoint{1.927780in}{3.114243in}}{\pgfqpoint{1.917181in}{3.118633in}}{\pgfqpoint{1.906131in}{3.118633in}}%
\pgfpathcurveto{\pgfqpoint{1.895081in}{3.118633in}}{\pgfqpoint{1.884482in}{3.114243in}}{\pgfqpoint{1.876668in}{3.106429in}}%
\pgfpathcurveto{\pgfqpoint{1.868855in}{3.098616in}}{\pgfqpoint{1.864465in}{3.088017in}}{\pgfqpoint{1.864465in}{3.076967in}}%
\pgfpathcurveto{\pgfqpoint{1.864465in}{3.065916in}}{\pgfqpoint{1.868855in}{3.055317in}}{\pgfqpoint{1.876668in}{3.047504in}}%
\pgfpathcurveto{\pgfqpoint{1.884482in}{3.039690in}}{\pgfqpoint{1.895081in}{3.035300in}}{\pgfqpoint{1.906131in}{3.035300in}}%
\pgfpathclose%
\pgfusepath{stroke,fill}%
\end{pgfscope}%
\begin{pgfscope}%
\pgfpathrectangle{\pgfqpoint{0.787074in}{0.548769in}}{\pgfqpoint{5.062926in}{3.102590in}}%
\pgfusepath{clip}%
\pgfsetbuttcap%
\pgfsetroundjoin%
\definecolor{currentfill}{rgb}{1.000000,0.498039,0.054902}%
\pgfsetfillcolor{currentfill}%
\pgfsetlinewidth{1.003750pt}%
\definecolor{currentstroke}{rgb}{1.000000,0.498039,0.054902}%
\pgfsetstrokecolor{currentstroke}%
\pgfsetdash{}{0pt}%
\pgfpathmoveto{\pgfqpoint{2.002549in}{2.247058in}}%
\pgfpathcurveto{\pgfqpoint{2.013599in}{2.247058in}}{\pgfqpoint{2.024198in}{2.251449in}}{\pgfqpoint{2.032012in}{2.259262in}}%
\pgfpathcurveto{\pgfqpoint{2.039826in}{2.267076in}}{\pgfqpoint{2.044216in}{2.277675in}}{\pgfqpoint{2.044216in}{2.288725in}}%
\pgfpathcurveto{\pgfqpoint{2.044216in}{2.299775in}}{\pgfqpoint{2.039826in}{2.310374in}}{\pgfqpoint{2.032012in}{2.318188in}}%
\pgfpathcurveto{\pgfqpoint{2.024198in}{2.326002in}}{\pgfqpoint{2.013599in}{2.330392in}}{\pgfqpoint{2.002549in}{2.330392in}}%
\pgfpathcurveto{\pgfqpoint{1.991499in}{2.330392in}}{\pgfqpoint{1.980900in}{2.326002in}}{\pgfqpoint{1.973086in}{2.318188in}}%
\pgfpathcurveto{\pgfqpoint{1.965273in}{2.310374in}}{\pgfqpoint{1.960883in}{2.299775in}}{\pgfqpoint{1.960883in}{2.288725in}}%
\pgfpathcurveto{\pgfqpoint{1.960883in}{2.277675in}}{\pgfqpoint{1.965273in}{2.267076in}}{\pgfqpoint{1.973086in}{2.259262in}}%
\pgfpathcurveto{\pgfqpoint{1.980900in}{2.251449in}}{\pgfqpoint{1.991499in}{2.247058in}}{\pgfqpoint{2.002549in}{2.247058in}}%
\pgfpathclose%
\pgfusepath{stroke,fill}%
\end{pgfscope}%
\begin{pgfscope}%
\pgfpathrectangle{\pgfqpoint{0.787074in}{0.548769in}}{\pgfqpoint{5.062926in}{3.102590in}}%
\pgfusepath{clip}%
\pgfsetbuttcap%
\pgfsetroundjoin%
\definecolor{currentfill}{rgb}{1.000000,0.498039,0.054902}%
\pgfsetfillcolor{currentfill}%
\pgfsetlinewidth{1.003750pt}%
\definecolor{currentstroke}{rgb}{1.000000,0.498039,0.054902}%
\pgfsetstrokecolor{currentstroke}%
\pgfsetdash{}{0pt}%
\pgfpathmoveto{\pgfqpoint{1.610474in}{2.894377in}}%
\pgfpathcurveto{\pgfqpoint{1.621524in}{2.894377in}}{\pgfqpoint{1.632123in}{2.898767in}}{\pgfqpoint{1.639937in}{2.906581in}}%
\pgfpathcurveto{\pgfqpoint{1.647751in}{2.914394in}}{\pgfqpoint{1.652141in}{2.924993in}}{\pgfqpoint{1.652141in}{2.936044in}}%
\pgfpathcurveto{\pgfqpoint{1.652141in}{2.947094in}}{\pgfqpoint{1.647751in}{2.957693in}}{\pgfqpoint{1.639937in}{2.965506in}}%
\pgfpathcurveto{\pgfqpoint{1.632123in}{2.973320in}}{\pgfqpoint{1.621524in}{2.977710in}}{\pgfqpoint{1.610474in}{2.977710in}}%
\pgfpathcurveto{\pgfqpoint{1.599424in}{2.977710in}}{\pgfqpoint{1.588825in}{2.973320in}}{\pgfqpoint{1.581011in}{2.965506in}}%
\pgfpathcurveto{\pgfqpoint{1.573198in}{2.957693in}}{\pgfqpoint{1.568808in}{2.947094in}}{\pgfqpoint{1.568808in}{2.936044in}}%
\pgfpathcurveto{\pgfqpoint{1.568808in}{2.924993in}}{\pgfqpoint{1.573198in}{2.914394in}}{\pgfqpoint{1.581011in}{2.906581in}}%
\pgfpathcurveto{\pgfqpoint{1.588825in}{2.898767in}}{\pgfqpoint{1.599424in}{2.894377in}}{\pgfqpoint{1.610474in}{2.894377in}}%
\pgfpathclose%
\pgfusepath{stroke,fill}%
\end{pgfscope}%
\begin{pgfscope}%
\pgfpathrectangle{\pgfqpoint{0.787074in}{0.548769in}}{\pgfqpoint{5.062926in}{3.102590in}}%
\pgfusepath{clip}%
\pgfsetbuttcap%
\pgfsetroundjoin%
\definecolor{currentfill}{rgb}{1.000000,0.498039,0.054902}%
\pgfsetfillcolor{currentfill}%
\pgfsetlinewidth{1.003750pt}%
\definecolor{currentstroke}{rgb}{1.000000,0.498039,0.054902}%
\pgfsetstrokecolor{currentstroke}%
\pgfsetdash{}{0pt}%
\pgfpathmoveto{\pgfqpoint{1.740484in}{3.298118in}}%
\pgfpathcurveto{\pgfqpoint{1.751534in}{3.298118in}}{\pgfqpoint{1.762133in}{3.302508in}}{\pgfqpoint{1.769947in}{3.310321in}}%
\pgfpathcurveto{\pgfqpoint{1.777760in}{3.318135in}}{\pgfqpoint{1.782151in}{3.328734in}}{\pgfqpoint{1.782151in}{3.339784in}}%
\pgfpathcurveto{\pgfqpoint{1.782151in}{3.350834in}}{\pgfqpoint{1.777760in}{3.361433in}}{\pgfqpoint{1.769947in}{3.369247in}}%
\pgfpathcurveto{\pgfqpoint{1.762133in}{3.377061in}}{\pgfqpoint{1.751534in}{3.381451in}}{\pgfqpoint{1.740484in}{3.381451in}}%
\pgfpathcurveto{\pgfqpoint{1.729434in}{3.381451in}}{\pgfqpoint{1.718835in}{3.377061in}}{\pgfqpoint{1.711021in}{3.369247in}}%
\pgfpathcurveto{\pgfqpoint{1.703208in}{3.361433in}}{\pgfqpoint{1.698817in}{3.350834in}}{\pgfqpoint{1.698817in}{3.339784in}}%
\pgfpathcurveto{\pgfqpoint{1.698817in}{3.328734in}}{\pgfqpoint{1.703208in}{3.318135in}}{\pgfqpoint{1.711021in}{3.310321in}}%
\pgfpathcurveto{\pgfqpoint{1.718835in}{3.302508in}}{\pgfqpoint{1.729434in}{3.298118in}}{\pgfqpoint{1.740484in}{3.298118in}}%
\pgfpathclose%
\pgfusepath{stroke,fill}%
\end{pgfscope}%
\begin{pgfscope}%
\pgfpathrectangle{\pgfqpoint{0.787074in}{0.548769in}}{\pgfqpoint{5.062926in}{3.102590in}}%
\pgfusepath{clip}%
\pgfsetbuttcap%
\pgfsetroundjoin%
\definecolor{currentfill}{rgb}{1.000000,0.498039,0.054902}%
\pgfsetfillcolor{currentfill}%
\pgfsetlinewidth{1.003750pt}%
\definecolor{currentstroke}{rgb}{1.000000,0.498039,0.054902}%
\pgfsetstrokecolor{currentstroke}%
\pgfsetdash{}{0pt}%
\pgfpathmoveto{\pgfqpoint{2.240916in}{2.566130in}}%
\pgfpathcurveto{\pgfqpoint{2.251967in}{2.566130in}}{\pgfqpoint{2.262566in}{2.570520in}}{\pgfqpoint{2.270379in}{2.578334in}}%
\pgfpathcurveto{\pgfqpoint{2.278193in}{2.586148in}}{\pgfqpoint{2.282583in}{2.596747in}}{\pgfqpoint{2.282583in}{2.607797in}}%
\pgfpathcurveto{\pgfqpoint{2.282583in}{2.618847in}}{\pgfqpoint{2.278193in}{2.629446in}}{\pgfqpoint{2.270379in}{2.637260in}}%
\pgfpathcurveto{\pgfqpoint{2.262566in}{2.645073in}}{\pgfqpoint{2.251967in}{2.649463in}}{\pgfqpoint{2.240916in}{2.649463in}}%
\pgfpathcurveto{\pgfqpoint{2.229866in}{2.649463in}}{\pgfqpoint{2.219267in}{2.645073in}}{\pgfqpoint{2.211454in}{2.637260in}}%
\pgfpathcurveto{\pgfqpoint{2.203640in}{2.629446in}}{\pgfqpoint{2.199250in}{2.618847in}}{\pgfqpoint{2.199250in}{2.607797in}}%
\pgfpathcurveto{\pgfqpoint{2.199250in}{2.596747in}}{\pgfqpoint{2.203640in}{2.586148in}}{\pgfqpoint{2.211454in}{2.578334in}}%
\pgfpathcurveto{\pgfqpoint{2.219267in}{2.570520in}}{\pgfqpoint{2.229866in}{2.566130in}}{\pgfqpoint{2.240916in}{2.566130in}}%
\pgfpathclose%
\pgfusepath{stroke,fill}%
\end{pgfscope}%
\begin{pgfscope}%
\pgfpathrectangle{\pgfqpoint{0.787074in}{0.548769in}}{\pgfqpoint{5.062926in}{3.102590in}}%
\pgfusepath{clip}%
\pgfsetbuttcap%
\pgfsetroundjoin%
\definecolor{currentfill}{rgb}{0.121569,0.466667,0.705882}%
\pgfsetfillcolor{currentfill}%
\pgfsetlinewidth{1.003750pt}%
\definecolor{currentstroke}{rgb}{0.121569,0.466667,0.705882}%
\pgfsetstrokecolor{currentstroke}%
\pgfsetdash{}{0pt}%
\pgfpathmoveto{\pgfqpoint{1.933759in}{2.795473in}}%
\pgfpathcurveto{\pgfqpoint{1.944809in}{2.795473in}}{\pgfqpoint{1.955408in}{2.799863in}}{\pgfqpoint{1.963222in}{2.807676in}}%
\pgfpathcurveto{\pgfqpoint{1.971035in}{2.815490in}}{\pgfqpoint{1.975426in}{2.826089in}}{\pgfqpoint{1.975426in}{2.837139in}}%
\pgfpathcurveto{\pgfqpoint{1.975426in}{2.848189in}}{\pgfqpoint{1.971035in}{2.858788in}}{\pgfqpoint{1.963222in}{2.866602in}}%
\pgfpathcurveto{\pgfqpoint{1.955408in}{2.874416in}}{\pgfqpoint{1.944809in}{2.878806in}}{\pgfqpoint{1.933759in}{2.878806in}}%
\pgfpathcurveto{\pgfqpoint{1.922709in}{2.878806in}}{\pgfqpoint{1.912110in}{2.874416in}}{\pgfqpoint{1.904296in}{2.866602in}}%
\pgfpathcurveto{\pgfqpoint{1.896483in}{2.858788in}}{\pgfqpoint{1.892092in}{2.848189in}}{\pgfqpoint{1.892092in}{2.837139in}}%
\pgfpathcurveto{\pgfqpoint{1.892092in}{2.826089in}}{\pgfqpoint{1.896483in}{2.815490in}}{\pgfqpoint{1.904296in}{2.807676in}}%
\pgfpathcurveto{\pgfqpoint{1.912110in}{2.799863in}}{\pgfqpoint{1.922709in}{2.795473in}}{\pgfqpoint{1.933759in}{2.795473in}}%
\pgfpathclose%
\pgfusepath{stroke,fill}%
\end{pgfscope}%
\begin{pgfscope}%
\pgfpathrectangle{\pgfqpoint{0.787074in}{0.548769in}}{\pgfqpoint{5.062926in}{3.102590in}}%
\pgfusepath{clip}%
\pgfsetbuttcap%
\pgfsetroundjoin%
\definecolor{currentfill}{rgb}{1.000000,0.498039,0.054902}%
\pgfsetfillcolor{currentfill}%
\pgfsetlinewidth{1.003750pt}%
\definecolor{currentstroke}{rgb}{1.000000,0.498039,0.054902}%
\pgfsetstrokecolor{currentstroke}%
\pgfsetdash{}{0pt}%
\pgfpathmoveto{\pgfqpoint{1.628906in}{3.027648in}}%
\pgfpathcurveto{\pgfqpoint{1.639956in}{3.027648in}}{\pgfqpoint{1.650555in}{3.032039in}}{\pgfqpoint{1.658368in}{3.039852in}}%
\pgfpathcurveto{\pgfqpoint{1.666182in}{3.047666in}}{\pgfqpoint{1.670572in}{3.058265in}}{\pgfqpoint{1.670572in}{3.069315in}}%
\pgfpathcurveto{\pgfqpoint{1.670572in}{3.080365in}}{\pgfqpoint{1.666182in}{3.090964in}}{\pgfqpoint{1.658368in}{3.098778in}}%
\pgfpathcurveto{\pgfqpoint{1.650555in}{3.106591in}}{\pgfqpoint{1.639956in}{3.110982in}}{\pgfqpoint{1.628906in}{3.110982in}}%
\pgfpathcurveto{\pgfqpoint{1.617855in}{3.110982in}}{\pgfqpoint{1.607256in}{3.106591in}}{\pgfqpoint{1.599443in}{3.098778in}}%
\pgfpathcurveto{\pgfqpoint{1.591629in}{3.090964in}}{\pgfqpoint{1.587239in}{3.080365in}}{\pgfqpoint{1.587239in}{3.069315in}}%
\pgfpathcurveto{\pgfqpoint{1.587239in}{3.058265in}}{\pgfqpoint{1.591629in}{3.047666in}}{\pgfqpoint{1.599443in}{3.039852in}}%
\pgfpathcurveto{\pgfqpoint{1.607256in}{3.032039in}}{\pgfqpoint{1.617855in}{3.027648in}}{\pgfqpoint{1.628906in}{3.027648in}}%
\pgfpathclose%
\pgfusepath{stroke,fill}%
\end{pgfscope}%
\begin{pgfscope}%
\pgfpathrectangle{\pgfqpoint{0.787074in}{0.548769in}}{\pgfqpoint{5.062926in}{3.102590in}}%
\pgfusepath{clip}%
\pgfsetbuttcap%
\pgfsetroundjoin%
\definecolor{currentfill}{rgb}{1.000000,0.498039,0.054902}%
\pgfsetfillcolor{currentfill}%
\pgfsetlinewidth{1.003750pt}%
\definecolor{currentstroke}{rgb}{1.000000,0.498039,0.054902}%
\pgfsetstrokecolor{currentstroke}%
\pgfsetdash{}{0pt}%
\pgfpathmoveto{\pgfqpoint{1.619725in}{2.815221in}}%
\pgfpathcurveto{\pgfqpoint{1.630775in}{2.815221in}}{\pgfqpoint{1.641374in}{2.819611in}}{\pgfqpoint{1.649187in}{2.827425in}}%
\pgfpathcurveto{\pgfqpoint{1.657001in}{2.835238in}}{\pgfqpoint{1.661391in}{2.845837in}}{\pgfqpoint{1.661391in}{2.856887in}}%
\pgfpathcurveto{\pgfqpoint{1.661391in}{2.867937in}}{\pgfqpoint{1.657001in}{2.878537in}}{\pgfqpoint{1.649187in}{2.886350in}}%
\pgfpathcurveto{\pgfqpoint{1.641374in}{2.894164in}}{\pgfqpoint{1.630775in}{2.898554in}}{\pgfqpoint{1.619725in}{2.898554in}}%
\pgfpathcurveto{\pgfqpoint{1.608674in}{2.898554in}}{\pgfqpoint{1.598075in}{2.894164in}}{\pgfqpoint{1.590262in}{2.886350in}}%
\pgfpathcurveto{\pgfqpoint{1.582448in}{2.878537in}}{\pgfqpoint{1.578058in}{2.867937in}}{\pgfqpoint{1.578058in}{2.856887in}}%
\pgfpathcurveto{\pgfqpoint{1.578058in}{2.845837in}}{\pgfqpoint{1.582448in}{2.835238in}}{\pgfqpoint{1.590262in}{2.827425in}}%
\pgfpathcurveto{\pgfqpoint{1.598075in}{2.819611in}}{\pgfqpoint{1.608674in}{2.815221in}}{\pgfqpoint{1.619725in}{2.815221in}}%
\pgfpathclose%
\pgfusepath{stroke,fill}%
\end{pgfscope}%
\begin{pgfscope}%
\pgfpathrectangle{\pgfqpoint{0.787074in}{0.548769in}}{\pgfqpoint{5.062926in}{3.102590in}}%
\pgfusepath{clip}%
\pgfsetbuttcap%
\pgfsetroundjoin%
\definecolor{currentfill}{rgb}{1.000000,0.498039,0.054902}%
\pgfsetfillcolor{currentfill}%
\pgfsetlinewidth{1.003750pt}%
\definecolor{currentstroke}{rgb}{1.000000,0.498039,0.054902}%
\pgfsetstrokecolor{currentstroke}%
\pgfsetdash{}{0pt}%
\pgfpathmoveto{\pgfqpoint{1.878014in}{1.272567in}}%
\pgfpathcurveto{\pgfqpoint{1.889064in}{1.272567in}}{\pgfqpoint{1.899663in}{1.276958in}}{\pgfqpoint{1.907477in}{1.284771in}}%
\pgfpathcurveto{\pgfqpoint{1.915291in}{1.292585in}}{\pgfqpoint{1.919681in}{1.303184in}}{\pgfqpoint{1.919681in}{1.314234in}}%
\pgfpathcurveto{\pgfqpoint{1.919681in}{1.325284in}}{\pgfqpoint{1.915291in}{1.335883in}}{\pgfqpoint{1.907477in}{1.343697in}}%
\pgfpathcurveto{\pgfqpoint{1.899663in}{1.351510in}}{\pgfqpoint{1.889064in}{1.355901in}}{\pgfqpoint{1.878014in}{1.355901in}}%
\pgfpathcurveto{\pgfqpoint{1.866964in}{1.355901in}}{\pgfqpoint{1.856365in}{1.351510in}}{\pgfqpoint{1.848551in}{1.343697in}}%
\pgfpathcurveto{\pgfqpoint{1.840738in}{1.335883in}}{\pgfqpoint{1.836347in}{1.325284in}}{\pgfqpoint{1.836347in}{1.314234in}}%
\pgfpathcurveto{\pgfqpoint{1.836347in}{1.303184in}}{\pgfqpoint{1.840738in}{1.292585in}}{\pgfqpoint{1.848551in}{1.284771in}}%
\pgfpathcurveto{\pgfqpoint{1.856365in}{1.276958in}}{\pgfqpoint{1.866964in}{1.272567in}}{\pgfqpoint{1.878014in}{1.272567in}}%
\pgfpathclose%
\pgfusepath{stroke,fill}%
\end{pgfscope}%
\begin{pgfscope}%
\pgfpathrectangle{\pgfqpoint{0.787074in}{0.548769in}}{\pgfqpoint{5.062926in}{3.102590in}}%
\pgfusepath{clip}%
\pgfsetbuttcap%
\pgfsetroundjoin%
\definecolor{currentfill}{rgb}{0.121569,0.466667,0.705882}%
\pgfsetfillcolor{currentfill}%
\pgfsetlinewidth{1.003750pt}%
\definecolor{currentstroke}{rgb}{0.121569,0.466667,0.705882}%
\pgfsetstrokecolor{currentstroke}%
\pgfsetdash{}{0pt}%
\pgfpathmoveto{\pgfqpoint{2.544514in}{1.853887in}}%
\pgfpathcurveto{\pgfqpoint{2.555564in}{1.853887in}}{\pgfqpoint{2.566163in}{1.858277in}}{\pgfqpoint{2.573977in}{1.866090in}}%
\pgfpathcurveto{\pgfqpoint{2.581790in}{1.873904in}}{\pgfqpoint{2.586181in}{1.884503in}}{\pgfqpoint{2.586181in}{1.895553in}}%
\pgfpathcurveto{\pgfqpoint{2.586181in}{1.906603in}}{\pgfqpoint{2.581790in}{1.917202in}}{\pgfqpoint{2.573977in}{1.925016in}}%
\pgfpathcurveto{\pgfqpoint{2.566163in}{1.932830in}}{\pgfqpoint{2.555564in}{1.937220in}}{\pgfqpoint{2.544514in}{1.937220in}}%
\pgfpathcurveto{\pgfqpoint{2.533464in}{1.937220in}}{\pgfqpoint{2.522865in}{1.932830in}}{\pgfqpoint{2.515051in}{1.925016in}}%
\pgfpathcurveto{\pgfqpoint{2.507237in}{1.917202in}}{\pgfqpoint{2.502847in}{1.906603in}}{\pgfqpoint{2.502847in}{1.895553in}}%
\pgfpathcurveto{\pgfqpoint{2.502847in}{1.884503in}}{\pgfqpoint{2.507237in}{1.873904in}}{\pgfqpoint{2.515051in}{1.866090in}}%
\pgfpathcurveto{\pgfqpoint{2.522865in}{1.858277in}}{\pgfqpoint{2.533464in}{1.853887in}}{\pgfqpoint{2.544514in}{1.853887in}}%
\pgfpathclose%
\pgfusepath{stroke,fill}%
\end{pgfscope}%
\begin{pgfscope}%
\pgfpathrectangle{\pgfqpoint{0.787074in}{0.548769in}}{\pgfqpoint{5.062926in}{3.102590in}}%
\pgfusepath{clip}%
\pgfsetbuttcap%
\pgfsetroundjoin%
\definecolor{currentfill}{rgb}{0.121569,0.466667,0.705882}%
\pgfsetfillcolor{currentfill}%
\pgfsetlinewidth{1.003750pt}%
\definecolor{currentstroke}{rgb}{0.121569,0.466667,0.705882}%
\pgfsetstrokecolor{currentstroke}%
\pgfsetdash{}{0pt}%
\pgfpathmoveto{\pgfqpoint{2.044324in}{2.914142in}}%
\pgfpathcurveto{\pgfqpoint{2.055374in}{2.914142in}}{\pgfqpoint{2.065973in}{2.918533in}}{\pgfqpoint{2.073787in}{2.926346in}}%
\pgfpathcurveto{\pgfqpoint{2.081601in}{2.934160in}}{\pgfqpoint{2.085991in}{2.944759in}}{\pgfqpoint{2.085991in}{2.955809in}}%
\pgfpathcurveto{\pgfqpoint{2.085991in}{2.966859in}}{\pgfqpoint{2.081601in}{2.977458in}}{\pgfqpoint{2.073787in}{2.985272in}}%
\pgfpathcurveto{\pgfqpoint{2.065973in}{2.993085in}}{\pgfqpoint{2.055374in}{2.997476in}}{\pgfqpoint{2.044324in}{2.997476in}}%
\pgfpathcurveto{\pgfqpoint{2.033274in}{2.997476in}}{\pgfqpoint{2.022675in}{2.993085in}}{\pgfqpoint{2.014861in}{2.985272in}}%
\pgfpathcurveto{\pgfqpoint{2.007048in}{2.977458in}}{\pgfqpoint{2.002657in}{2.966859in}}{\pgfqpoint{2.002657in}{2.955809in}}%
\pgfpathcurveto{\pgfqpoint{2.002657in}{2.944759in}}{\pgfqpoint{2.007048in}{2.934160in}}{\pgfqpoint{2.014861in}{2.926346in}}%
\pgfpathcurveto{\pgfqpoint{2.022675in}{2.918533in}}{\pgfqpoint{2.033274in}{2.914142in}}{\pgfqpoint{2.044324in}{2.914142in}}%
\pgfpathclose%
\pgfusepath{stroke,fill}%
\end{pgfscope}%
\begin{pgfscope}%
\pgfpathrectangle{\pgfqpoint{0.787074in}{0.548769in}}{\pgfqpoint{5.062926in}{3.102590in}}%
\pgfusepath{clip}%
\pgfsetbuttcap%
\pgfsetroundjoin%
\definecolor{currentfill}{rgb}{1.000000,0.498039,0.054902}%
\pgfsetfillcolor{currentfill}%
\pgfsetlinewidth{1.003750pt}%
\definecolor{currentstroke}{rgb}{1.000000,0.498039,0.054902}%
\pgfsetstrokecolor{currentstroke}%
\pgfsetdash{}{0pt}%
\pgfpathmoveto{\pgfqpoint{2.228615in}{2.606637in}}%
\pgfpathcurveto{\pgfqpoint{2.239665in}{2.606637in}}{\pgfqpoint{2.250264in}{2.611027in}}{\pgfqpoint{2.258077in}{2.618841in}}%
\pgfpathcurveto{\pgfqpoint{2.265891in}{2.626654in}}{\pgfqpoint{2.270281in}{2.637253in}}{\pgfqpoint{2.270281in}{2.648303in}}%
\pgfpathcurveto{\pgfqpoint{2.270281in}{2.659353in}}{\pgfqpoint{2.265891in}{2.669952in}}{\pgfqpoint{2.258077in}{2.677766in}}%
\pgfpathcurveto{\pgfqpoint{2.250264in}{2.685580in}}{\pgfqpoint{2.239665in}{2.689970in}}{\pgfqpoint{2.228615in}{2.689970in}}%
\pgfpathcurveto{\pgfqpoint{2.217565in}{2.689970in}}{\pgfqpoint{2.206966in}{2.685580in}}{\pgfqpoint{2.199152in}{2.677766in}}%
\pgfpathcurveto{\pgfqpoint{2.191338in}{2.669952in}}{\pgfqpoint{2.186948in}{2.659353in}}{\pgfqpoint{2.186948in}{2.648303in}}%
\pgfpathcurveto{\pgfqpoint{2.186948in}{2.637253in}}{\pgfqpoint{2.191338in}{2.626654in}}{\pgfqpoint{2.199152in}{2.618841in}}%
\pgfpathcurveto{\pgfqpoint{2.206966in}{2.611027in}}{\pgfqpoint{2.217565in}{2.606637in}}{\pgfqpoint{2.228615in}{2.606637in}}%
\pgfpathclose%
\pgfusepath{stroke,fill}%
\end{pgfscope}%
\begin{pgfscope}%
\pgfpathrectangle{\pgfqpoint{0.787074in}{0.548769in}}{\pgfqpoint{5.062926in}{3.102590in}}%
\pgfusepath{clip}%
\pgfsetbuttcap%
\pgfsetroundjoin%
\definecolor{currentfill}{rgb}{0.839216,0.152941,0.156863}%
\pgfsetfillcolor{currentfill}%
\pgfsetlinewidth{1.003750pt}%
\definecolor{currentstroke}{rgb}{0.839216,0.152941,0.156863}%
\pgfsetstrokecolor{currentstroke}%
\pgfsetdash{}{0pt}%
\pgfpathmoveto{\pgfqpoint{2.406171in}{2.324482in}}%
\pgfpathcurveto{\pgfqpoint{2.417221in}{2.324482in}}{\pgfqpoint{2.427820in}{2.328873in}}{\pgfqpoint{2.435633in}{2.336686in}}%
\pgfpathcurveto{\pgfqpoint{2.443447in}{2.344500in}}{\pgfqpoint{2.447837in}{2.355099in}}{\pgfqpoint{2.447837in}{2.366149in}}%
\pgfpathcurveto{\pgfqpoint{2.447837in}{2.377199in}}{\pgfqpoint{2.443447in}{2.387798in}}{\pgfqpoint{2.435633in}{2.395612in}}%
\pgfpathcurveto{\pgfqpoint{2.427820in}{2.403426in}}{\pgfqpoint{2.417221in}{2.407816in}}{\pgfqpoint{2.406171in}{2.407816in}}%
\pgfpathcurveto{\pgfqpoint{2.395121in}{2.407816in}}{\pgfqpoint{2.384522in}{2.403426in}}{\pgfqpoint{2.376708in}{2.395612in}}%
\pgfpathcurveto{\pgfqpoint{2.368894in}{2.387798in}}{\pgfqpoint{2.364504in}{2.377199in}}{\pgfqpoint{2.364504in}{2.366149in}}%
\pgfpathcurveto{\pgfqpoint{2.364504in}{2.355099in}}{\pgfqpoint{2.368894in}{2.344500in}}{\pgfqpoint{2.376708in}{2.336686in}}%
\pgfpathcurveto{\pgfqpoint{2.384522in}{2.328873in}}{\pgfqpoint{2.395121in}{2.324482in}}{\pgfqpoint{2.406171in}{2.324482in}}%
\pgfpathclose%
\pgfusepath{stroke,fill}%
\end{pgfscope}%
\begin{pgfscope}%
\pgfpathrectangle{\pgfqpoint{0.787074in}{0.548769in}}{\pgfqpoint{5.062926in}{3.102590in}}%
\pgfusepath{clip}%
\pgfsetbuttcap%
\pgfsetroundjoin%
\definecolor{currentfill}{rgb}{1.000000,0.498039,0.054902}%
\pgfsetfillcolor{currentfill}%
\pgfsetlinewidth{1.003750pt}%
\definecolor{currentstroke}{rgb}{1.000000,0.498039,0.054902}%
\pgfsetstrokecolor{currentstroke}%
\pgfsetdash{}{0pt}%
\pgfpathmoveto{\pgfqpoint{1.514954in}{2.951278in}}%
\pgfpathcurveto{\pgfqpoint{1.526004in}{2.951278in}}{\pgfqpoint{1.536603in}{2.955668in}}{\pgfqpoint{1.544417in}{2.963482in}}%
\pgfpathcurveto{\pgfqpoint{1.552230in}{2.971295in}}{\pgfqpoint{1.556621in}{2.981894in}}{\pgfqpoint{1.556621in}{2.992945in}}%
\pgfpathcurveto{\pgfqpoint{1.556621in}{3.003995in}}{\pgfqpoint{1.552230in}{3.014594in}}{\pgfqpoint{1.544417in}{3.022407in}}%
\pgfpathcurveto{\pgfqpoint{1.536603in}{3.030221in}}{\pgfqpoint{1.526004in}{3.034611in}}{\pgfqpoint{1.514954in}{3.034611in}}%
\pgfpathcurveto{\pgfqpoint{1.503904in}{3.034611in}}{\pgfqpoint{1.493305in}{3.030221in}}{\pgfqpoint{1.485491in}{3.022407in}}%
\pgfpathcurveto{\pgfqpoint{1.477678in}{3.014594in}}{\pgfqpoint{1.473287in}{3.003995in}}{\pgfqpoint{1.473287in}{2.992945in}}%
\pgfpathcurveto{\pgfqpoint{1.473287in}{2.981894in}}{\pgfqpoint{1.477678in}{2.971295in}}{\pgfqpoint{1.485491in}{2.963482in}}%
\pgfpathcurveto{\pgfqpoint{1.493305in}{2.955668in}}{\pgfqpoint{1.503904in}{2.951278in}}{\pgfqpoint{1.514954in}{2.951278in}}%
\pgfpathclose%
\pgfusepath{stroke,fill}%
\end{pgfscope}%
\begin{pgfscope}%
\pgfpathrectangle{\pgfqpoint{0.787074in}{0.548769in}}{\pgfqpoint{5.062926in}{3.102590in}}%
\pgfusepath{clip}%
\pgfsetbuttcap%
\pgfsetroundjoin%
\definecolor{currentfill}{rgb}{0.121569,0.466667,0.705882}%
\pgfsetfillcolor{currentfill}%
\pgfsetlinewidth{1.003750pt}%
\definecolor{currentstroke}{rgb}{0.121569,0.466667,0.705882}%
\pgfsetstrokecolor{currentstroke}%
\pgfsetdash{}{0pt}%
\pgfpathmoveto{\pgfqpoint{2.842825in}{2.536597in}}%
\pgfpathcurveto{\pgfqpoint{2.853876in}{2.536597in}}{\pgfqpoint{2.864475in}{2.540987in}}{\pgfqpoint{2.872288in}{2.548801in}}%
\pgfpathcurveto{\pgfqpoint{2.880102in}{2.556614in}}{\pgfqpoint{2.884492in}{2.567213in}}{\pgfqpoint{2.884492in}{2.578263in}}%
\pgfpathcurveto{\pgfqpoint{2.884492in}{2.589314in}}{\pgfqpoint{2.880102in}{2.599913in}}{\pgfqpoint{2.872288in}{2.607726in}}%
\pgfpathcurveto{\pgfqpoint{2.864475in}{2.615540in}}{\pgfqpoint{2.853876in}{2.619930in}}{\pgfqpoint{2.842825in}{2.619930in}}%
\pgfpathcurveto{\pgfqpoint{2.831775in}{2.619930in}}{\pgfqpoint{2.821176in}{2.615540in}}{\pgfqpoint{2.813363in}{2.607726in}}%
\pgfpathcurveto{\pgfqpoint{2.805549in}{2.599913in}}{\pgfqpoint{2.801159in}{2.589314in}}{\pgfqpoint{2.801159in}{2.578263in}}%
\pgfpathcurveto{\pgfqpoint{2.801159in}{2.567213in}}{\pgfqpoint{2.805549in}{2.556614in}}{\pgfqpoint{2.813363in}{2.548801in}}%
\pgfpathcurveto{\pgfqpoint{2.821176in}{2.540987in}}{\pgfqpoint{2.831775in}{2.536597in}}{\pgfqpoint{2.842825in}{2.536597in}}%
\pgfpathclose%
\pgfusepath{stroke,fill}%
\end{pgfscope}%
\begin{pgfscope}%
\pgfpathrectangle{\pgfqpoint{0.787074in}{0.548769in}}{\pgfqpoint{5.062926in}{3.102590in}}%
\pgfusepath{clip}%
\pgfsetbuttcap%
\pgfsetroundjoin%
\definecolor{currentfill}{rgb}{1.000000,0.498039,0.054902}%
\pgfsetfillcolor{currentfill}%
\pgfsetlinewidth{1.003750pt}%
\definecolor{currentstroke}{rgb}{1.000000,0.498039,0.054902}%
\pgfsetstrokecolor{currentstroke}%
\pgfsetdash{}{0pt}%
\pgfpathmoveto{\pgfqpoint{2.046354in}{2.094657in}}%
\pgfpathcurveto{\pgfqpoint{2.057405in}{2.094657in}}{\pgfqpoint{2.068004in}{2.099047in}}{\pgfqpoint{2.075817in}{2.106861in}}%
\pgfpathcurveto{\pgfqpoint{2.083631in}{2.114675in}}{\pgfqpoint{2.088021in}{2.125274in}}{\pgfqpoint{2.088021in}{2.136324in}}%
\pgfpathcurveto{\pgfqpoint{2.088021in}{2.147374in}}{\pgfqpoint{2.083631in}{2.157973in}}{\pgfqpoint{2.075817in}{2.165787in}}%
\pgfpathcurveto{\pgfqpoint{2.068004in}{2.173600in}}{\pgfqpoint{2.057405in}{2.177990in}}{\pgfqpoint{2.046354in}{2.177990in}}%
\pgfpathcurveto{\pgfqpoint{2.035304in}{2.177990in}}{\pgfqpoint{2.024705in}{2.173600in}}{\pgfqpoint{2.016892in}{2.165787in}}%
\pgfpathcurveto{\pgfqpoint{2.009078in}{2.157973in}}{\pgfqpoint{2.004688in}{2.147374in}}{\pgfqpoint{2.004688in}{2.136324in}}%
\pgfpathcurveto{\pgfqpoint{2.004688in}{2.125274in}}{\pgfqpoint{2.009078in}{2.114675in}}{\pgfqpoint{2.016892in}{2.106861in}}%
\pgfpathcurveto{\pgfqpoint{2.024705in}{2.099047in}}{\pgfqpoint{2.035304in}{2.094657in}}{\pgfqpoint{2.046354in}{2.094657in}}%
\pgfpathclose%
\pgfusepath{stroke,fill}%
\end{pgfscope}%
\begin{pgfscope}%
\pgfpathrectangle{\pgfqpoint{0.787074in}{0.548769in}}{\pgfqpoint{5.062926in}{3.102590in}}%
\pgfusepath{clip}%
\pgfsetbuttcap%
\pgfsetroundjoin%
\definecolor{currentfill}{rgb}{1.000000,0.498039,0.054902}%
\pgfsetfillcolor{currentfill}%
\pgfsetlinewidth{1.003750pt}%
\definecolor{currentstroke}{rgb}{1.000000,0.498039,0.054902}%
\pgfsetstrokecolor{currentstroke}%
\pgfsetdash{}{0pt}%
\pgfpathmoveto{\pgfqpoint{2.046243in}{2.224129in}}%
\pgfpathcurveto{\pgfqpoint{2.057293in}{2.224129in}}{\pgfqpoint{2.067892in}{2.228519in}}{\pgfqpoint{2.075706in}{2.236332in}}%
\pgfpathcurveto{\pgfqpoint{2.083519in}{2.244146in}}{\pgfqpoint{2.087909in}{2.254745in}}{\pgfqpoint{2.087909in}{2.265795in}}%
\pgfpathcurveto{\pgfqpoint{2.087909in}{2.276845in}}{\pgfqpoint{2.083519in}{2.287444in}}{\pgfqpoint{2.075706in}{2.295258in}}%
\pgfpathcurveto{\pgfqpoint{2.067892in}{2.303072in}}{\pgfqpoint{2.057293in}{2.307462in}}{\pgfqpoint{2.046243in}{2.307462in}}%
\pgfpathcurveto{\pgfqpoint{2.035193in}{2.307462in}}{\pgfqpoint{2.024594in}{2.303072in}}{\pgfqpoint{2.016780in}{2.295258in}}%
\pgfpathcurveto{\pgfqpoint{2.008966in}{2.287444in}}{\pgfqpoint{2.004576in}{2.276845in}}{\pgfqpoint{2.004576in}{2.265795in}}%
\pgfpathcurveto{\pgfqpoint{2.004576in}{2.254745in}}{\pgfqpoint{2.008966in}{2.244146in}}{\pgfqpoint{2.016780in}{2.236332in}}%
\pgfpathcurveto{\pgfqpoint{2.024594in}{2.228519in}}{\pgfqpoint{2.035193in}{2.224129in}}{\pgfqpoint{2.046243in}{2.224129in}}%
\pgfpathclose%
\pgfusepath{stroke,fill}%
\end{pgfscope}%
\begin{pgfscope}%
\pgfpathrectangle{\pgfqpoint{0.787074in}{0.548769in}}{\pgfqpoint{5.062926in}{3.102590in}}%
\pgfusepath{clip}%
\pgfsetbuttcap%
\pgfsetroundjoin%
\definecolor{currentfill}{rgb}{1.000000,0.498039,0.054902}%
\pgfsetfillcolor{currentfill}%
\pgfsetlinewidth{1.003750pt}%
\definecolor{currentstroke}{rgb}{1.000000,0.498039,0.054902}%
\pgfsetstrokecolor{currentstroke}%
\pgfsetdash{}{0pt}%
\pgfpathmoveto{\pgfqpoint{2.525416in}{2.050679in}}%
\pgfpathcurveto{\pgfqpoint{2.536466in}{2.050679in}}{\pgfqpoint{2.547065in}{2.055070in}}{\pgfqpoint{2.554879in}{2.062883in}}%
\pgfpathcurveto{\pgfqpoint{2.562692in}{2.070697in}}{\pgfqpoint{2.567083in}{2.081296in}}{\pgfqpoint{2.567083in}{2.092346in}}%
\pgfpathcurveto{\pgfqpoint{2.567083in}{2.103396in}}{\pgfqpoint{2.562692in}{2.113995in}}{\pgfqpoint{2.554879in}{2.121809in}}%
\pgfpathcurveto{\pgfqpoint{2.547065in}{2.129622in}}{\pgfqpoint{2.536466in}{2.134013in}}{\pgfqpoint{2.525416in}{2.134013in}}%
\pgfpathcurveto{\pgfqpoint{2.514366in}{2.134013in}}{\pgfqpoint{2.503767in}{2.129622in}}{\pgfqpoint{2.495953in}{2.121809in}}%
\pgfpathcurveto{\pgfqpoint{2.488140in}{2.113995in}}{\pgfqpoint{2.483749in}{2.103396in}}{\pgfqpoint{2.483749in}{2.092346in}}%
\pgfpathcurveto{\pgfqpoint{2.483749in}{2.081296in}}{\pgfqpoint{2.488140in}{2.070697in}}{\pgfqpoint{2.495953in}{2.062883in}}%
\pgfpathcurveto{\pgfqpoint{2.503767in}{2.055070in}}{\pgfqpoint{2.514366in}{2.050679in}}{\pgfqpoint{2.525416in}{2.050679in}}%
\pgfpathclose%
\pgfusepath{stroke,fill}%
\end{pgfscope}%
\begin{pgfscope}%
\pgfpathrectangle{\pgfqpoint{0.787074in}{0.548769in}}{\pgfqpoint{5.062926in}{3.102590in}}%
\pgfusepath{clip}%
\pgfsetbuttcap%
\pgfsetroundjoin%
\definecolor{currentfill}{rgb}{0.121569,0.466667,0.705882}%
\pgfsetfillcolor{currentfill}%
\pgfsetlinewidth{1.003750pt}%
\definecolor{currentstroke}{rgb}{0.121569,0.466667,0.705882}%
\pgfsetstrokecolor{currentstroke}%
\pgfsetdash{}{0pt}%
\pgfpathmoveto{\pgfqpoint{2.180182in}{2.613502in}}%
\pgfpathcurveto{\pgfqpoint{2.191232in}{2.613502in}}{\pgfqpoint{2.201831in}{2.617892in}}{\pgfqpoint{2.209645in}{2.625706in}}%
\pgfpathcurveto{\pgfqpoint{2.217459in}{2.633519in}}{\pgfqpoint{2.221849in}{2.644118in}}{\pgfqpoint{2.221849in}{2.655169in}}%
\pgfpathcurveto{\pgfqpoint{2.221849in}{2.666219in}}{\pgfqpoint{2.217459in}{2.676818in}}{\pgfqpoint{2.209645in}{2.684631in}}%
\pgfpathcurveto{\pgfqpoint{2.201831in}{2.692445in}}{\pgfqpoint{2.191232in}{2.696835in}}{\pgfqpoint{2.180182in}{2.696835in}}%
\pgfpathcurveto{\pgfqpoint{2.169132in}{2.696835in}}{\pgfqpoint{2.158533in}{2.692445in}}{\pgfqpoint{2.150719in}{2.684631in}}%
\pgfpathcurveto{\pgfqpoint{2.142906in}{2.676818in}}{\pgfqpoint{2.138516in}{2.666219in}}{\pgfqpoint{2.138516in}{2.655169in}}%
\pgfpathcurveto{\pgfqpoint{2.138516in}{2.644118in}}{\pgfqpoint{2.142906in}{2.633519in}}{\pgfqpoint{2.150719in}{2.625706in}}%
\pgfpathcurveto{\pgfqpoint{2.158533in}{2.617892in}}{\pgfqpoint{2.169132in}{2.613502in}}{\pgfqpoint{2.180182in}{2.613502in}}%
\pgfpathclose%
\pgfusepath{stroke,fill}%
\end{pgfscope}%
\begin{pgfscope}%
\pgfpathrectangle{\pgfqpoint{0.787074in}{0.548769in}}{\pgfqpoint{5.062926in}{3.102590in}}%
\pgfusepath{clip}%
\pgfsetbuttcap%
\pgfsetroundjoin%
\definecolor{currentfill}{rgb}{0.121569,0.466667,0.705882}%
\pgfsetfillcolor{currentfill}%
\pgfsetlinewidth{1.003750pt}%
\definecolor{currentstroke}{rgb}{0.121569,0.466667,0.705882}%
\pgfsetstrokecolor{currentstroke}%
\pgfsetdash{}{0pt}%
\pgfpathmoveto{\pgfqpoint{2.242280in}{2.701999in}}%
\pgfpathcurveto{\pgfqpoint{2.253330in}{2.701999in}}{\pgfqpoint{2.263929in}{2.706389in}}{\pgfqpoint{2.271743in}{2.714203in}}%
\pgfpathcurveto{\pgfqpoint{2.279557in}{2.722016in}}{\pgfqpoint{2.283947in}{2.732616in}}{\pgfqpoint{2.283947in}{2.743666in}}%
\pgfpathcurveto{\pgfqpoint{2.283947in}{2.754716in}}{\pgfqpoint{2.279557in}{2.765315in}}{\pgfqpoint{2.271743in}{2.773128in}}%
\pgfpathcurveto{\pgfqpoint{2.263929in}{2.780942in}}{\pgfqpoint{2.253330in}{2.785332in}}{\pgfqpoint{2.242280in}{2.785332in}}%
\pgfpathcurveto{\pgfqpoint{2.231230in}{2.785332in}}{\pgfqpoint{2.220631in}{2.780942in}}{\pgfqpoint{2.212817in}{2.773128in}}%
\pgfpathcurveto{\pgfqpoint{2.205004in}{2.765315in}}{\pgfqpoint{2.200614in}{2.754716in}}{\pgfqpoint{2.200614in}{2.743666in}}%
\pgfpathcurveto{\pgfqpoint{2.200614in}{2.732616in}}{\pgfqpoint{2.205004in}{2.722016in}}{\pgfqpoint{2.212817in}{2.714203in}}%
\pgfpathcurveto{\pgfqpoint{2.220631in}{2.706389in}}{\pgfqpoint{2.231230in}{2.701999in}}{\pgfqpoint{2.242280in}{2.701999in}}%
\pgfpathclose%
\pgfusepath{stroke,fill}%
\end{pgfscope}%
\begin{pgfscope}%
\pgfpathrectangle{\pgfqpoint{0.787074in}{0.548769in}}{\pgfqpoint{5.062926in}{3.102590in}}%
\pgfusepath{clip}%
\pgfsetbuttcap%
\pgfsetroundjoin%
\definecolor{currentfill}{rgb}{0.121569,0.466667,0.705882}%
\pgfsetfillcolor{currentfill}%
\pgfsetlinewidth{1.003750pt}%
\definecolor{currentstroke}{rgb}{0.121569,0.466667,0.705882}%
\pgfsetstrokecolor{currentstroke}%
\pgfsetdash{}{0pt}%
\pgfpathmoveto{\pgfqpoint{2.158792in}{2.930649in}}%
\pgfpathcurveto{\pgfqpoint{2.169842in}{2.930649in}}{\pgfqpoint{2.180441in}{2.935039in}}{\pgfqpoint{2.188255in}{2.942853in}}%
\pgfpathcurveto{\pgfqpoint{2.196068in}{2.950666in}}{\pgfqpoint{2.200459in}{2.961265in}}{\pgfqpoint{2.200459in}{2.972316in}}%
\pgfpathcurveto{\pgfqpoint{2.200459in}{2.983366in}}{\pgfqpoint{2.196068in}{2.993965in}}{\pgfqpoint{2.188255in}{3.001778in}}%
\pgfpathcurveto{\pgfqpoint{2.180441in}{3.009592in}}{\pgfqpoint{2.169842in}{3.013982in}}{\pgfqpoint{2.158792in}{3.013982in}}%
\pgfpathcurveto{\pgfqpoint{2.147742in}{3.013982in}}{\pgfqpoint{2.137143in}{3.009592in}}{\pgfqpoint{2.129329in}{3.001778in}}%
\pgfpathcurveto{\pgfqpoint{2.121516in}{2.993965in}}{\pgfqpoint{2.117125in}{2.983366in}}{\pgfqpoint{2.117125in}{2.972316in}}%
\pgfpathcurveto{\pgfqpoint{2.117125in}{2.961265in}}{\pgfqpoint{2.121516in}{2.950666in}}{\pgfqpoint{2.129329in}{2.942853in}}%
\pgfpathcurveto{\pgfqpoint{2.137143in}{2.935039in}}{\pgfqpoint{2.147742in}{2.930649in}}{\pgfqpoint{2.158792in}{2.930649in}}%
\pgfpathclose%
\pgfusepath{stroke,fill}%
\end{pgfscope}%
\begin{pgfscope}%
\pgfpathrectangle{\pgfqpoint{0.787074in}{0.548769in}}{\pgfqpoint{5.062926in}{3.102590in}}%
\pgfusepath{clip}%
\pgfsetbuttcap%
\pgfsetroundjoin%
\definecolor{currentfill}{rgb}{0.121569,0.466667,0.705882}%
\pgfsetfillcolor{currentfill}%
\pgfsetlinewidth{1.003750pt}%
\definecolor{currentstroke}{rgb}{0.121569,0.466667,0.705882}%
\pgfsetstrokecolor{currentstroke}%
\pgfsetdash{}{0pt}%
\pgfpathmoveto{\pgfqpoint{1.180438in}{0.648134in}}%
\pgfpathcurveto{\pgfqpoint{1.191489in}{0.648134in}}{\pgfqpoint{1.202088in}{0.652525in}}{\pgfqpoint{1.209901in}{0.660338in}}%
\pgfpathcurveto{\pgfqpoint{1.217715in}{0.668152in}}{\pgfqpoint{1.222105in}{0.678751in}}{\pgfqpoint{1.222105in}{0.689801in}}%
\pgfpathcurveto{\pgfqpoint{1.222105in}{0.700851in}}{\pgfqpoint{1.217715in}{0.711450in}}{\pgfqpoint{1.209901in}{0.719264in}}%
\pgfpathcurveto{\pgfqpoint{1.202088in}{0.727077in}}{\pgfqpoint{1.191489in}{0.731468in}}{\pgfqpoint{1.180438in}{0.731468in}}%
\pgfpathcurveto{\pgfqpoint{1.169388in}{0.731468in}}{\pgfqpoint{1.158789in}{0.727077in}}{\pgfqpoint{1.150976in}{0.719264in}}%
\pgfpathcurveto{\pgfqpoint{1.143162in}{0.711450in}}{\pgfqpoint{1.138772in}{0.700851in}}{\pgfqpoint{1.138772in}{0.689801in}}%
\pgfpathcurveto{\pgfqpoint{1.138772in}{0.678751in}}{\pgfqpoint{1.143162in}{0.668152in}}{\pgfqpoint{1.150976in}{0.660338in}}%
\pgfpathcurveto{\pgfqpoint{1.158789in}{0.652525in}}{\pgfqpoint{1.169388in}{0.648134in}}{\pgfqpoint{1.180438in}{0.648134in}}%
\pgfpathclose%
\pgfusepath{stroke,fill}%
\end{pgfscope}%
\begin{pgfscope}%
\pgfpathrectangle{\pgfqpoint{0.787074in}{0.548769in}}{\pgfqpoint{5.062926in}{3.102590in}}%
\pgfusepath{clip}%
\pgfsetbuttcap%
\pgfsetroundjoin%
\definecolor{currentfill}{rgb}{1.000000,0.498039,0.054902}%
\pgfsetfillcolor{currentfill}%
\pgfsetlinewidth{1.003750pt}%
\definecolor{currentstroke}{rgb}{1.000000,0.498039,0.054902}%
\pgfsetstrokecolor{currentstroke}%
\pgfsetdash{}{0pt}%
\pgfpathmoveto{\pgfqpoint{2.050188in}{1.583814in}}%
\pgfpathcurveto{\pgfqpoint{2.061238in}{1.583814in}}{\pgfqpoint{2.071837in}{1.588205in}}{\pgfqpoint{2.079651in}{1.596018in}}%
\pgfpathcurveto{\pgfqpoint{2.087464in}{1.603832in}}{\pgfqpoint{2.091855in}{1.614431in}}{\pgfqpoint{2.091855in}{1.625481in}}%
\pgfpathcurveto{\pgfqpoint{2.091855in}{1.636531in}}{\pgfqpoint{2.087464in}{1.647130in}}{\pgfqpoint{2.079651in}{1.654944in}}%
\pgfpathcurveto{\pgfqpoint{2.071837in}{1.662758in}}{\pgfqpoint{2.061238in}{1.667148in}}{\pgfqpoint{2.050188in}{1.667148in}}%
\pgfpathcurveto{\pgfqpoint{2.039138in}{1.667148in}}{\pgfqpoint{2.028539in}{1.662758in}}{\pgfqpoint{2.020725in}{1.654944in}}%
\pgfpathcurveto{\pgfqpoint{2.012912in}{1.647130in}}{\pgfqpoint{2.008521in}{1.636531in}}{\pgfqpoint{2.008521in}{1.625481in}}%
\pgfpathcurveto{\pgfqpoint{2.008521in}{1.614431in}}{\pgfqpoint{2.012912in}{1.603832in}}{\pgfqpoint{2.020725in}{1.596018in}}%
\pgfpathcurveto{\pgfqpoint{2.028539in}{1.588205in}}{\pgfqpoint{2.039138in}{1.583814in}}{\pgfqpoint{2.050188in}{1.583814in}}%
\pgfpathclose%
\pgfusepath{stroke,fill}%
\end{pgfscope}%
\begin{pgfscope}%
\pgfpathrectangle{\pgfqpoint{0.787074in}{0.548769in}}{\pgfqpoint{5.062926in}{3.102590in}}%
\pgfusepath{clip}%
\pgfsetbuttcap%
\pgfsetroundjoin%
\definecolor{currentfill}{rgb}{1.000000,0.498039,0.054902}%
\pgfsetfillcolor{currentfill}%
\pgfsetlinewidth{1.003750pt}%
\definecolor{currentstroke}{rgb}{1.000000,0.498039,0.054902}%
\pgfsetstrokecolor{currentstroke}%
\pgfsetdash{}{0pt}%
\pgfpathmoveto{\pgfqpoint{1.600045in}{3.152602in}}%
\pgfpathcurveto{\pgfqpoint{1.611095in}{3.152602in}}{\pgfqpoint{1.621694in}{3.156992in}}{\pgfqpoint{1.629508in}{3.164806in}}%
\pgfpathcurveto{\pgfqpoint{1.637321in}{3.172619in}}{\pgfqpoint{1.641712in}{3.183218in}}{\pgfqpoint{1.641712in}{3.194268in}}%
\pgfpathcurveto{\pgfqpoint{1.641712in}{3.205319in}}{\pgfqpoint{1.637321in}{3.215918in}}{\pgfqpoint{1.629508in}{3.223731in}}%
\pgfpathcurveto{\pgfqpoint{1.621694in}{3.231545in}}{\pgfqpoint{1.611095in}{3.235935in}}{\pgfqpoint{1.600045in}{3.235935in}}%
\pgfpathcurveto{\pgfqpoint{1.588995in}{3.235935in}}{\pgfqpoint{1.578396in}{3.231545in}}{\pgfqpoint{1.570582in}{3.223731in}}%
\pgfpathcurveto{\pgfqpoint{1.562769in}{3.215918in}}{\pgfqpoint{1.558378in}{3.205319in}}{\pgfqpoint{1.558378in}{3.194268in}}%
\pgfpathcurveto{\pgfqpoint{1.558378in}{3.183218in}}{\pgfqpoint{1.562769in}{3.172619in}}{\pgfqpoint{1.570582in}{3.164806in}}%
\pgfpathcurveto{\pgfqpoint{1.578396in}{3.156992in}}{\pgfqpoint{1.588995in}{3.152602in}}{\pgfqpoint{1.600045in}{3.152602in}}%
\pgfpathclose%
\pgfusepath{stroke,fill}%
\end{pgfscope}%
\begin{pgfscope}%
\pgfpathrectangle{\pgfqpoint{0.787074in}{0.548769in}}{\pgfqpoint{5.062926in}{3.102590in}}%
\pgfusepath{clip}%
\pgfsetbuttcap%
\pgfsetroundjoin%
\definecolor{currentfill}{rgb}{1.000000,0.498039,0.054902}%
\pgfsetfillcolor{currentfill}%
\pgfsetlinewidth{1.003750pt}%
\definecolor{currentstroke}{rgb}{1.000000,0.498039,0.054902}%
\pgfsetstrokecolor{currentstroke}%
\pgfsetdash{}{0pt}%
\pgfpathmoveto{\pgfqpoint{1.526867in}{2.882713in}}%
\pgfpathcurveto{\pgfqpoint{1.537917in}{2.882713in}}{\pgfqpoint{1.548516in}{2.887104in}}{\pgfqpoint{1.556329in}{2.894917in}}%
\pgfpathcurveto{\pgfqpoint{1.564143in}{2.902731in}}{\pgfqpoint{1.568533in}{2.913330in}}{\pgfqpoint{1.568533in}{2.924380in}}%
\pgfpathcurveto{\pgfqpoint{1.568533in}{2.935430in}}{\pgfqpoint{1.564143in}{2.946029in}}{\pgfqpoint{1.556329in}{2.953843in}}%
\pgfpathcurveto{\pgfqpoint{1.548516in}{2.961656in}}{\pgfqpoint{1.537917in}{2.966047in}}{\pgfqpoint{1.526867in}{2.966047in}}%
\pgfpathcurveto{\pgfqpoint{1.515816in}{2.966047in}}{\pgfqpoint{1.505217in}{2.961656in}}{\pgfqpoint{1.497404in}{2.953843in}}%
\pgfpathcurveto{\pgfqpoint{1.489590in}{2.946029in}}{\pgfqpoint{1.485200in}{2.935430in}}{\pgfqpoint{1.485200in}{2.924380in}}%
\pgfpathcurveto{\pgfqpoint{1.485200in}{2.913330in}}{\pgfqpoint{1.489590in}{2.902731in}}{\pgfqpoint{1.497404in}{2.894917in}}%
\pgfpathcurveto{\pgfqpoint{1.505217in}{2.887104in}}{\pgfqpoint{1.515816in}{2.882713in}}{\pgfqpoint{1.526867in}{2.882713in}}%
\pgfpathclose%
\pgfusepath{stroke,fill}%
\end{pgfscope}%
\begin{pgfscope}%
\pgfpathrectangle{\pgfqpoint{0.787074in}{0.548769in}}{\pgfqpoint{5.062926in}{3.102590in}}%
\pgfusepath{clip}%
\pgfsetbuttcap%
\pgfsetroundjoin%
\definecolor{currentfill}{rgb}{0.121569,0.466667,0.705882}%
\pgfsetfillcolor{currentfill}%
\pgfsetlinewidth{1.003750pt}%
\definecolor{currentstroke}{rgb}{0.121569,0.466667,0.705882}%
\pgfsetstrokecolor{currentstroke}%
\pgfsetdash{}{0pt}%
\pgfpathmoveto{\pgfqpoint{1.342692in}{0.749337in}}%
\pgfpathcurveto{\pgfqpoint{1.353742in}{0.749337in}}{\pgfqpoint{1.364341in}{0.753727in}}{\pgfqpoint{1.372154in}{0.761541in}}%
\pgfpathcurveto{\pgfqpoint{1.379968in}{0.769355in}}{\pgfqpoint{1.384358in}{0.779954in}}{\pgfqpoint{1.384358in}{0.791004in}}%
\pgfpathcurveto{\pgfqpoint{1.384358in}{0.802054in}}{\pgfqpoint{1.379968in}{0.812653in}}{\pgfqpoint{1.372154in}{0.820467in}}%
\pgfpathcurveto{\pgfqpoint{1.364341in}{0.828280in}}{\pgfqpoint{1.353742in}{0.832670in}}{\pgfqpoint{1.342692in}{0.832670in}}%
\pgfpathcurveto{\pgfqpoint{1.331641in}{0.832670in}}{\pgfqpoint{1.321042in}{0.828280in}}{\pgfqpoint{1.313229in}{0.820467in}}%
\pgfpathcurveto{\pgfqpoint{1.305415in}{0.812653in}}{\pgfqpoint{1.301025in}{0.802054in}}{\pgfqpoint{1.301025in}{0.791004in}}%
\pgfpathcurveto{\pgfqpoint{1.301025in}{0.779954in}}{\pgfqpoint{1.305415in}{0.769355in}}{\pgfqpoint{1.313229in}{0.761541in}}%
\pgfpathcurveto{\pgfqpoint{1.321042in}{0.753727in}}{\pgfqpoint{1.331641in}{0.749337in}}{\pgfqpoint{1.342692in}{0.749337in}}%
\pgfpathclose%
\pgfusepath{stroke,fill}%
\end{pgfscope}%
\begin{pgfscope}%
\pgfpathrectangle{\pgfqpoint{0.787074in}{0.548769in}}{\pgfqpoint{5.062926in}{3.102590in}}%
\pgfusepath{clip}%
\pgfsetbuttcap%
\pgfsetroundjoin%
\definecolor{currentfill}{rgb}{1.000000,0.498039,0.054902}%
\pgfsetfillcolor{currentfill}%
\pgfsetlinewidth{1.003750pt}%
\definecolor{currentstroke}{rgb}{1.000000,0.498039,0.054902}%
\pgfsetstrokecolor{currentstroke}%
\pgfsetdash{}{0pt}%
\pgfpathmoveto{\pgfqpoint{1.671528in}{1.712229in}}%
\pgfpathcurveto{\pgfqpoint{1.682578in}{1.712229in}}{\pgfqpoint{1.693177in}{1.716619in}}{\pgfqpoint{1.700991in}{1.724432in}}%
\pgfpathcurveto{\pgfqpoint{1.708805in}{1.732246in}}{\pgfqpoint{1.713195in}{1.742845in}}{\pgfqpoint{1.713195in}{1.753895in}}%
\pgfpathcurveto{\pgfqpoint{1.713195in}{1.764945in}}{\pgfqpoint{1.708805in}{1.775544in}}{\pgfqpoint{1.700991in}{1.783358in}}%
\pgfpathcurveto{\pgfqpoint{1.693177in}{1.791172in}}{\pgfqpoint{1.682578in}{1.795562in}}{\pgfqpoint{1.671528in}{1.795562in}}%
\pgfpathcurveto{\pgfqpoint{1.660478in}{1.795562in}}{\pgfqpoint{1.649879in}{1.791172in}}{\pgfqpoint{1.642065in}{1.783358in}}%
\pgfpathcurveto{\pgfqpoint{1.634252in}{1.775544in}}{\pgfqpoint{1.629861in}{1.764945in}}{\pgfqpoint{1.629861in}{1.753895in}}%
\pgfpathcurveto{\pgfqpoint{1.629861in}{1.742845in}}{\pgfqpoint{1.634252in}{1.732246in}}{\pgfqpoint{1.642065in}{1.724432in}}%
\pgfpathcurveto{\pgfqpoint{1.649879in}{1.716619in}}{\pgfqpoint{1.660478in}{1.712229in}}{\pgfqpoint{1.671528in}{1.712229in}}%
\pgfpathclose%
\pgfusepath{stroke,fill}%
\end{pgfscope}%
\begin{pgfscope}%
\pgfpathrectangle{\pgfqpoint{0.787074in}{0.548769in}}{\pgfqpoint{5.062926in}{3.102590in}}%
\pgfusepath{clip}%
\pgfsetbuttcap%
\pgfsetroundjoin%
\definecolor{currentfill}{rgb}{0.121569,0.466667,0.705882}%
\pgfsetfillcolor{currentfill}%
\pgfsetlinewidth{1.003750pt}%
\definecolor{currentstroke}{rgb}{0.121569,0.466667,0.705882}%
\pgfsetstrokecolor{currentstroke}%
\pgfsetdash{}{0pt}%
\pgfpathmoveto{\pgfqpoint{2.319003in}{2.635102in}}%
\pgfpathcurveto{\pgfqpoint{2.330053in}{2.635102in}}{\pgfqpoint{2.340652in}{2.639493in}}{\pgfqpoint{2.348466in}{2.647306in}}%
\pgfpathcurveto{\pgfqpoint{2.356280in}{2.655120in}}{\pgfqpoint{2.360670in}{2.665719in}}{\pgfqpoint{2.360670in}{2.676769in}}%
\pgfpathcurveto{\pgfqpoint{2.360670in}{2.687819in}}{\pgfqpoint{2.356280in}{2.698418in}}{\pgfqpoint{2.348466in}{2.706232in}}%
\pgfpathcurveto{\pgfqpoint{2.340652in}{2.714045in}}{\pgfqpoint{2.330053in}{2.718436in}}{\pgfqpoint{2.319003in}{2.718436in}}%
\pgfpathcurveto{\pgfqpoint{2.307953in}{2.718436in}}{\pgfqpoint{2.297354in}{2.714045in}}{\pgfqpoint{2.289540in}{2.706232in}}%
\pgfpathcurveto{\pgfqpoint{2.281727in}{2.698418in}}{\pgfqpoint{2.277336in}{2.687819in}}{\pgfqpoint{2.277336in}{2.676769in}}%
\pgfpathcurveto{\pgfqpoint{2.277336in}{2.665719in}}{\pgfqpoint{2.281727in}{2.655120in}}{\pgfqpoint{2.289540in}{2.647306in}}%
\pgfpathcurveto{\pgfqpoint{2.297354in}{2.639493in}}{\pgfqpoint{2.307953in}{2.635102in}}{\pgfqpoint{2.319003in}{2.635102in}}%
\pgfpathclose%
\pgfusepath{stroke,fill}%
\end{pgfscope}%
\begin{pgfscope}%
\pgfpathrectangle{\pgfqpoint{0.787074in}{0.548769in}}{\pgfqpoint{5.062926in}{3.102590in}}%
\pgfusepath{clip}%
\pgfsetbuttcap%
\pgfsetroundjoin%
\definecolor{currentfill}{rgb}{1.000000,0.498039,0.054902}%
\pgfsetfillcolor{currentfill}%
\pgfsetlinewidth{1.003750pt}%
\definecolor{currentstroke}{rgb}{1.000000,0.498039,0.054902}%
\pgfsetstrokecolor{currentstroke}%
\pgfsetdash{}{0pt}%
\pgfpathmoveto{\pgfqpoint{2.167179in}{2.957127in}}%
\pgfpathcurveto{\pgfqpoint{2.178229in}{2.957127in}}{\pgfqpoint{2.188829in}{2.961517in}}{\pgfqpoint{2.196642in}{2.969331in}}%
\pgfpathcurveto{\pgfqpoint{2.204456in}{2.977144in}}{\pgfqpoint{2.208846in}{2.987743in}}{\pgfqpoint{2.208846in}{2.998793in}}%
\pgfpathcurveto{\pgfqpoint{2.208846in}{3.009844in}}{\pgfqpoint{2.204456in}{3.020443in}}{\pgfqpoint{2.196642in}{3.028256in}}%
\pgfpathcurveto{\pgfqpoint{2.188829in}{3.036070in}}{\pgfqpoint{2.178229in}{3.040460in}}{\pgfqpoint{2.167179in}{3.040460in}}%
\pgfpathcurveto{\pgfqpoint{2.156129in}{3.040460in}}{\pgfqpoint{2.145530in}{3.036070in}}{\pgfqpoint{2.137717in}{3.028256in}}%
\pgfpathcurveto{\pgfqpoint{2.129903in}{3.020443in}}{\pgfqpoint{2.125513in}{3.009844in}}{\pgfqpoint{2.125513in}{2.998793in}}%
\pgfpathcurveto{\pgfqpoint{2.125513in}{2.987743in}}{\pgfqpoint{2.129903in}{2.977144in}}{\pgfqpoint{2.137717in}{2.969331in}}%
\pgfpathcurveto{\pgfqpoint{2.145530in}{2.961517in}}{\pgfqpoint{2.156129in}{2.957127in}}{\pgfqpoint{2.167179in}{2.957127in}}%
\pgfpathclose%
\pgfusepath{stroke,fill}%
\end{pgfscope}%
\begin{pgfscope}%
\pgfpathrectangle{\pgfqpoint{0.787074in}{0.548769in}}{\pgfqpoint{5.062926in}{3.102590in}}%
\pgfusepath{clip}%
\pgfsetbuttcap%
\pgfsetroundjoin%
\definecolor{currentfill}{rgb}{0.121569,0.466667,0.705882}%
\pgfsetfillcolor{currentfill}%
\pgfsetlinewidth{1.003750pt}%
\definecolor{currentstroke}{rgb}{0.121569,0.466667,0.705882}%
\pgfsetstrokecolor{currentstroke}%
\pgfsetdash{}{0pt}%
\pgfpathmoveto{\pgfqpoint{1.894962in}{1.266126in}}%
\pgfpathcurveto{\pgfqpoint{1.906012in}{1.266126in}}{\pgfqpoint{1.916611in}{1.270516in}}{\pgfqpoint{1.924425in}{1.278330in}}%
\pgfpathcurveto{\pgfqpoint{1.932239in}{1.286143in}}{\pgfqpoint{1.936629in}{1.296742in}}{\pgfqpoint{1.936629in}{1.307792in}}%
\pgfpathcurveto{\pgfqpoint{1.936629in}{1.318842in}}{\pgfqpoint{1.932239in}{1.329441in}}{\pgfqpoint{1.924425in}{1.337255in}}%
\pgfpathcurveto{\pgfqpoint{1.916611in}{1.345069in}}{\pgfqpoint{1.906012in}{1.349459in}}{\pgfqpoint{1.894962in}{1.349459in}}%
\pgfpathcurveto{\pgfqpoint{1.883912in}{1.349459in}}{\pgfqpoint{1.873313in}{1.345069in}}{\pgfqpoint{1.865499in}{1.337255in}}%
\pgfpathcurveto{\pgfqpoint{1.857686in}{1.329441in}}{\pgfqpoint{1.853296in}{1.318842in}}{\pgfqpoint{1.853296in}{1.307792in}}%
\pgfpathcurveto{\pgfqpoint{1.853296in}{1.296742in}}{\pgfqpoint{1.857686in}{1.286143in}}{\pgfqpoint{1.865499in}{1.278330in}}%
\pgfpathcurveto{\pgfqpoint{1.873313in}{1.270516in}}{\pgfqpoint{1.883912in}{1.266126in}}{\pgfqpoint{1.894962in}{1.266126in}}%
\pgfpathclose%
\pgfusepath{stroke,fill}%
\end{pgfscope}%
\begin{pgfscope}%
\pgfpathrectangle{\pgfqpoint{0.787074in}{0.548769in}}{\pgfqpoint{5.062926in}{3.102590in}}%
\pgfusepath{clip}%
\pgfsetbuttcap%
\pgfsetroundjoin%
\definecolor{currentfill}{rgb}{1.000000,0.498039,0.054902}%
\pgfsetfillcolor{currentfill}%
\pgfsetlinewidth{1.003750pt}%
\definecolor{currentstroke}{rgb}{1.000000,0.498039,0.054902}%
\pgfsetstrokecolor{currentstroke}%
\pgfsetdash{}{0pt}%
\pgfpathmoveto{\pgfqpoint{1.540902in}{2.376418in}}%
\pgfpathcurveto{\pgfqpoint{1.551952in}{2.376418in}}{\pgfqpoint{1.562551in}{2.380808in}}{\pgfqpoint{1.570365in}{2.388622in}}%
\pgfpathcurveto{\pgfqpoint{1.578178in}{2.396436in}}{\pgfqpoint{1.582569in}{2.407035in}}{\pgfqpoint{1.582569in}{2.418085in}}%
\pgfpathcurveto{\pgfqpoint{1.582569in}{2.429135in}}{\pgfqpoint{1.578178in}{2.439734in}}{\pgfqpoint{1.570365in}{2.447548in}}%
\pgfpathcurveto{\pgfqpoint{1.562551in}{2.455361in}}{\pgfqpoint{1.551952in}{2.459751in}}{\pgfqpoint{1.540902in}{2.459751in}}%
\pgfpathcurveto{\pgfqpoint{1.529852in}{2.459751in}}{\pgfqpoint{1.519253in}{2.455361in}}{\pgfqpoint{1.511439in}{2.447548in}}%
\pgfpathcurveto{\pgfqpoint{1.503626in}{2.439734in}}{\pgfqpoint{1.499235in}{2.429135in}}{\pgfqpoint{1.499235in}{2.418085in}}%
\pgfpathcurveto{\pgfqpoint{1.499235in}{2.407035in}}{\pgfqpoint{1.503626in}{2.396436in}}{\pgfqpoint{1.511439in}{2.388622in}}%
\pgfpathcurveto{\pgfqpoint{1.519253in}{2.380808in}}{\pgfqpoint{1.529852in}{2.376418in}}{\pgfqpoint{1.540902in}{2.376418in}}%
\pgfpathclose%
\pgfusepath{stroke,fill}%
\end{pgfscope}%
\begin{pgfscope}%
\pgfpathrectangle{\pgfqpoint{0.787074in}{0.548769in}}{\pgfqpoint{5.062926in}{3.102590in}}%
\pgfusepath{clip}%
\pgfsetbuttcap%
\pgfsetroundjoin%
\definecolor{currentfill}{rgb}{1.000000,0.498039,0.054902}%
\pgfsetfillcolor{currentfill}%
\pgfsetlinewidth{1.003750pt}%
\definecolor{currentstroke}{rgb}{1.000000,0.498039,0.054902}%
\pgfsetstrokecolor{currentstroke}%
\pgfsetdash{}{0pt}%
\pgfpathmoveto{\pgfqpoint{1.838728in}{3.006227in}}%
\pgfpathcurveto{\pgfqpoint{1.849778in}{3.006227in}}{\pgfqpoint{1.860377in}{3.010617in}}{\pgfqpoint{1.868191in}{3.018431in}}%
\pgfpathcurveto{\pgfqpoint{1.876004in}{3.026244in}}{\pgfqpoint{1.880395in}{3.036844in}}{\pgfqpoint{1.880395in}{3.047894in}}%
\pgfpathcurveto{\pgfqpoint{1.880395in}{3.058944in}}{\pgfqpoint{1.876004in}{3.069543in}}{\pgfqpoint{1.868191in}{3.077356in}}%
\pgfpathcurveto{\pgfqpoint{1.860377in}{3.085170in}}{\pgfqpoint{1.849778in}{3.089560in}}{\pgfqpoint{1.838728in}{3.089560in}}%
\pgfpathcurveto{\pgfqpoint{1.827678in}{3.089560in}}{\pgfqpoint{1.817079in}{3.085170in}}{\pgfqpoint{1.809265in}{3.077356in}}%
\pgfpathcurveto{\pgfqpoint{1.801452in}{3.069543in}}{\pgfqpoint{1.797061in}{3.058944in}}{\pgfqpoint{1.797061in}{3.047894in}}%
\pgfpathcurveto{\pgfqpoint{1.797061in}{3.036844in}}{\pgfqpoint{1.801452in}{3.026244in}}{\pgfqpoint{1.809265in}{3.018431in}}%
\pgfpathcurveto{\pgfqpoint{1.817079in}{3.010617in}}{\pgfqpoint{1.827678in}{3.006227in}}{\pgfqpoint{1.838728in}{3.006227in}}%
\pgfpathclose%
\pgfusepath{stroke,fill}%
\end{pgfscope}%
\begin{pgfscope}%
\pgfpathrectangle{\pgfqpoint{0.787074in}{0.548769in}}{\pgfqpoint{5.062926in}{3.102590in}}%
\pgfusepath{clip}%
\pgfsetbuttcap%
\pgfsetroundjoin%
\definecolor{currentfill}{rgb}{1.000000,0.498039,0.054902}%
\pgfsetfillcolor{currentfill}%
\pgfsetlinewidth{1.003750pt}%
\definecolor{currentstroke}{rgb}{1.000000,0.498039,0.054902}%
\pgfsetstrokecolor{currentstroke}%
\pgfsetdash{}{0pt}%
\pgfpathmoveto{\pgfqpoint{1.897004in}{2.381259in}}%
\pgfpathcurveto{\pgfqpoint{1.908054in}{2.381259in}}{\pgfqpoint{1.918653in}{2.385649in}}{\pgfqpoint{1.926467in}{2.393463in}}%
\pgfpathcurveto{\pgfqpoint{1.934281in}{2.401277in}}{\pgfqpoint{1.938671in}{2.411876in}}{\pgfqpoint{1.938671in}{2.422926in}}%
\pgfpathcurveto{\pgfqpoint{1.938671in}{2.433976in}}{\pgfqpoint{1.934281in}{2.444575in}}{\pgfqpoint{1.926467in}{2.452388in}}%
\pgfpathcurveto{\pgfqpoint{1.918653in}{2.460202in}}{\pgfqpoint{1.908054in}{2.464592in}}{\pgfqpoint{1.897004in}{2.464592in}}%
\pgfpathcurveto{\pgfqpoint{1.885954in}{2.464592in}}{\pgfqpoint{1.875355in}{2.460202in}}{\pgfqpoint{1.867541in}{2.452388in}}%
\pgfpathcurveto{\pgfqpoint{1.859728in}{2.444575in}}{\pgfqpoint{1.855337in}{2.433976in}}{\pgfqpoint{1.855337in}{2.422926in}}%
\pgfpathcurveto{\pgfqpoint{1.855337in}{2.411876in}}{\pgfqpoint{1.859728in}{2.401277in}}{\pgfqpoint{1.867541in}{2.393463in}}%
\pgfpathcurveto{\pgfqpoint{1.875355in}{2.385649in}}{\pgfqpoint{1.885954in}{2.381259in}}{\pgfqpoint{1.897004in}{2.381259in}}%
\pgfpathclose%
\pgfusepath{stroke,fill}%
\end{pgfscope}%
\begin{pgfscope}%
\pgfpathrectangle{\pgfqpoint{0.787074in}{0.548769in}}{\pgfqpoint{5.062926in}{3.102590in}}%
\pgfusepath{clip}%
\pgfsetbuttcap%
\pgfsetroundjoin%
\definecolor{currentfill}{rgb}{1.000000,0.498039,0.054902}%
\pgfsetfillcolor{currentfill}%
\pgfsetlinewidth{1.003750pt}%
\definecolor{currentstroke}{rgb}{1.000000,0.498039,0.054902}%
\pgfsetstrokecolor{currentstroke}%
\pgfsetdash{}{0pt}%
\pgfpathmoveto{\pgfqpoint{2.093781in}{1.619069in}}%
\pgfpathcurveto{\pgfqpoint{2.104831in}{1.619069in}}{\pgfqpoint{2.115431in}{1.623460in}}{\pgfqpoint{2.123244in}{1.631273in}}%
\pgfpathcurveto{\pgfqpoint{2.131058in}{1.639087in}}{\pgfqpoint{2.135448in}{1.649686in}}{\pgfqpoint{2.135448in}{1.660736in}}%
\pgfpathcurveto{\pgfqpoint{2.135448in}{1.671786in}}{\pgfqpoint{2.131058in}{1.682385in}}{\pgfqpoint{2.123244in}{1.690199in}}%
\pgfpathcurveto{\pgfqpoint{2.115431in}{1.698012in}}{\pgfqpoint{2.104831in}{1.702403in}}{\pgfqpoint{2.093781in}{1.702403in}}%
\pgfpathcurveto{\pgfqpoint{2.082731in}{1.702403in}}{\pgfqpoint{2.072132in}{1.698012in}}{\pgfqpoint{2.064319in}{1.690199in}}%
\pgfpathcurveto{\pgfqpoint{2.056505in}{1.682385in}}{\pgfqpoint{2.052115in}{1.671786in}}{\pgfqpoint{2.052115in}{1.660736in}}%
\pgfpathcurveto{\pgfqpoint{2.052115in}{1.649686in}}{\pgfqpoint{2.056505in}{1.639087in}}{\pgfqpoint{2.064319in}{1.631273in}}%
\pgfpathcurveto{\pgfqpoint{2.072132in}{1.623460in}}{\pgfqpoint{2.082731in}{1.619069in}}{\pgfqpoint{2.093781in}{1.619069in}}%
\pgfpathclose%
\pgfusepath{stroke,fill}%
\end{pgfscope}%
\begin{pgfscope}%
\pgfpathrectangle{\pgfqpoint{0.787074in}{0.548769in}}{\pgfqpoint{5.062926in}{3.102590in}}%
\pgfusepath{clip}%
\pgfsetbuttcap%
\pgfsetroundjoin%
\definecolor{currentfill}{rgb}{1.000000,0.498039,0.054902}%
\pgfsetfillcolor{currentfill}%
\pgfsetlinewidth{1.003750pt}%
\definecolor{currentstroke}{rgb}{1.000000,0.498039,0.054902}%
\pgfsetstrokecolor{currentstroke}%
\pgfsetdash{}{0pt}%
\pgfpathmoveto{\pgfqpoint{1.609068in}{2.847775in}}%
\pgfpathcurveto{\pgfqpoint{1.620118in}{2.847775in}}{\pgfqpoint{1.630717in}{2.852165in}}{\pgfqpoint{1.638531in}{2.859979in}}%
\pgfpathcurveto{\pgfqpoint{1.646344in}{2.867793in}}{\pgfqpoint{1.650735in}{2.878392in}}{\pgfqpoint{1.650735in}{2.889442in}}%
\pgfpathcurveto{\pgfqpoint{1.650735in}{2.900492in}}{\pgfqpoint{1.646344in}{2.911091in}}{\pgfqpoint{1.638531in}{2.918904in}}%
\pgfpathcurveto{\pgfqpoint{1.630717in}{2.926718in}}{\pgfqpoint{1.620118in}{2.931108in}}{\pgfqpoint{1.609068in}{2.931108in}}%
\pgfpathcurveto{\pgfqpoint{1.598018in}{2.931108in}}{\pgfqpoint{1.587419in}{2.926718in}}{\pgfqpoint{1.579605in}{2.918904in}}%
\pgfpathcurveto{\pgfqpoint{1.571792in}{2.911091in}}{\pgfqpoint{1.567401in}{2.900492in}}{\pgfqpoint{1.567401in}{2.889442in}}%
\pgfpathcurveto{\pgfqpoint{1.567401in}{2.878392in}}{\pgfqpoint{1.571792in}{2.867793in}}{\pgfqpoint{1.579605in}{2.859979in}}%
\pgfpathcurveto{\pgfqpoint{1.587419in}{2.852165in}}{\pgfqpoint{1.598018in}{2.847775in}}{\pgfqpoint{1.609068in}{2.847775in}}%
\pgfpathclose%
\pgfusepath{stroke,fill}%
\end{pgfscope}%
\begin{pgfscope}%
\pgfpathrectangle{\pgfqpoint{0.787074in}{0.548769in}}{\pgfqpoint{5.062926in}{3.102590in}}%
\pgfusepath{clip}%
\pgfsetbuttcap%
\pgfsetroundjoin%
\definecolor{currentfill}{rgb}{1.000000,0.498039,0.054902}%
\pgfsetfillcolor{currentfill}%
\pgfsetlinewidth{1.003750pt}%
\definecolor{currentstroke}{rgb}{1.000000,0.498039,0.054902}%
\pgfsetstrokecolor{currentstroke}%
\pgfsetdash{}{0pt}%
\pgfpathmoveto{\pgfqpoint{1.928018in}{2.241897in}}%
\pgfpathcurveto{\pgfqpoint{1.939069in}{2.241897in}}{\pgfqpoint{1.949668in}{2.246287in}}{\pgfqpoint{1.957481in}{2.254101in}}%
\pgfpathcurveto{\pgfqpoint{1.965295in}{2.261914in}}{\pgfqpoint{1.969685in}{2.272513in}}{\pgfqpoint{1.969685in}{2.283563in}}%
\pgfpathcurveto{\pgfqpoint{1.969685in}{2.294613in}}{\pgfqpoint{1.965295in}{2.305212in}}{\pgfqpoint{1.957481in}{2.313026in}}%
\pgfpathcurveto{\pgfqpoint{1.949668in}{2.320840in}}{\pgfqpoint{1.939069in}{2.325230in}}{\pgfqpoint{1.928018in}{2.325230in}}%
\pgfpathcurveto{\pgfqpoint{1.916968in}{2.325230in}}{\pgfqpoint{1.906369in}{2.320840in}}{\pgfqpoint{1.898556in}{2.313026in}}%
\pgfpathcurveto{\pgfqpoint{1.890742in}{2.305212in}}{\pgfqpoint{1.886352in}{2.294613in}}{\pgfqpoint{1.886352in}{2.283563in}}%
\pgfpathcurveto{\pgfqpoint{1.886352in}{2.272513in}}{\pgfqpoint{1.890742in}{2.261914in}}{\pgfqpoint{1.898556in}{2.254101in}}%
\pgfpathcurveto{\pgfqpoint{1.906369in}{2.246287in}}{\pgfqpoint{1.916968in}{2.241897in}}{\pgfqpoint{1.928018in}{2.241897in}}%
\pgfpathclose%
\pgfusepath{stroke,fill}%
\end{pgfscope}%
\begin{pgfscope}%
\pgfpathrectangle{\pgfqpoint{0.787074in}{0.548769in}}{\pgfqpoint{5.062926in}{3.102590in}}%
\pgfusepath{clip}%
\pgfsetbuttcap%
\pgfsetroundjoin%
\definecolor{currentfill}{rgb}{1.000000,0.498039,0.054902}%
\pgfsetfillcolor{currentfill}%
\pgfsetlinewidth{1.003750pt}%
\definecolor{currentstroke}{rgb}{1.000000,0.498039,0.054902}%
\pgfsetstrokecolor{currentstroke}%
\pgfsetdash{}{0pt}%
\pgfpathmoveto{\pgfqpoint{2.202586in}{1.903634in}}%
\pgfpathcurveto{\pgfqpoint{2.213636in}{1.903634in}}{\pgfqpoint{2.224235in}{1.908024in}}{\pgfqpoint{2.232049in}{1.915838in}}%
\pgfpathcurveto{\pgfqpoint{2.239862in}{1.923652in}}{\pgfqpoint{2.244252in}{1.934251in}}{\pgfqpoint{2.244252in}{1.945301in}}%
\pgfpathcurveto{\pgfqpoint{2.244252in}{1.956351in}}{\pgfqpoint{2.239862in}{1.966950in}}{\pgfqpoint{2.232049in}{1.974764in}}%
\pgfpathcurveto{\pgfqpoint{2.224235in}{1.982577in}}{\pgfqpoint{2.213636in}{1.986968in}}{\pgfqpoint{2.202586in}{1.986968in}}%
\pgfpathcurveto{\pgfqpoint{2.191536in}{1.986968in}}{\pgfqpoint{2.180937in}{1.982577in}}{\pgfqpoint{2.173123in}{1.974764in}}%
\pgfpathcurveto{\pgfqpoint{2.165309in}{1.966950in}}{\pgfqpoint{2.160919in}{1.956351in}}{\pgfqpoint{2.160919in}{1.945301in}}%
\pgfpathcurveto{\pgfqpoint{2.160919in}{1.934251in}}{\pgfqpoint{2.165309in}{1.923652in}}{\pgfqpoint{2.173123in}{1.915838in}}%
\pgfpathcurveto{\pgfqpoint{2.180937in}{1.908024in}}{\pgfqpoint{2.191536in}{1.903634in}}{\pgfqpoint{2.202586in}{1.903634in}}%
\pgfpathclose%
\pgfusepath{stroke,fill}%
\end{pgfscope}%
\begin{pgfscope}%
\pgfpathrectangle{\pgfqpoint{0.787074in}{0.548769in}}{\pgfqpoint{5.062926in}{3.102590in}}%
\pgfusepath{clip}%
\pgfsetbuttcap%
\pgfsetroundjoin%
\definecolor{currentfill}{rgb}{1.000000,0.498039,0.054902}%
\pgfsetfillcolor{currentfill}%
\pgfsetlinewidth{1.003750pt}%
\definecolor{currentstroke}{rgb}{1.000000,0.498039,0.054902}%
\pgfsetstrokecolor{currentstroke}%
\pgfsetdash{}{0pt}%
\pgfpathmoveto{\pgfqpoint{2.014928in}{2.451696in}}%
\pgfpathcurveto{\pgfqpoint{2.025978in}{2.451696in}}{\pgfqpoint{2.036577in}{2.456086in}}{\pgfqpoint{2.044391in}{2.463900in}}%
\pgfpathcurveto{\pgfqpoint{2.052204in}{2.471713in}}{\pgfqpoint{2.056595in}{2.482312in}}{\pgfqpoint{2.056595in}{2.493362in}}%
\pgfpathcurveto{\pgfqpoint{2.056595in}{2.504412in}}{\pgfqpoint{2.052204in}{2.515012in}}{\pgfqpoint{2.044391in}{2.522825in}}%
\pgfpathcurveto{\pgfqpoint{2.036577in}{2.530639in}}{\pgfqpoint{2.025978in}{2.535029in}}{\pgfqpoint{2.014928in}{2.535029in}}%
\pgfpathcurveto{\pgfqpoint{2.003878in}{2.535029in}}{\pgfqpoint{1.993279in}{2.530639in}}{\pgfqpoint{1.985465in}{2.522825in}}%
\pgfpathcurveto{\pgfqpoint{1.977652in}{2.515012in}}{\pgfqpoint{1.973261in}{2.504412in}}{\pgfqpoint{1.973261in}{2.493362in}}%
\pgfpathcurveto{\pgfqpoint{1.973261in}{2.482312in}}{\pgfqpoint{1.977652in}{2.471713in}}{\pgfqpoint{1.985465in}{2.463900in}}%
\pgfpathcurveto{\pgfqpoint{1.993279in}{2.456086in}}{\pgfqpoint{2.003878in}{2.451696in}}{\pgfqpoint{2.014928in}{2.451696in}}%
\pgfpathclose%
\pgfusepath{stroke,fill}%
\end{pgfscope}%
\begin{pgfscope}%
\pgfpathrectangle{\pgfqpoint{0.787074in}{0.548769in}}{\pgfqpoint{5.062926in}{3.102590in}}%
\pgfusepath{clip}%
\pgfsetbuttcap%
\pgfsetroundjoin%
\definecolor{currentfill}{rgb}{0.839216,0.152941,0.156863}%
\pgfsetfillcolor{currentfill}%
\pgfsetlinewidth{1.003750pt}%
\definecolor{currentstroke}{rgb}{0.839216,0.152941,0.156863}%
\pgfsetstrokecolor{currentstroke}%
\pgfsetdash{}{0pt}%
\pgfpathmoveto{\pgfqpoint{1.787922in}{3.176910in}}%
\pgfpathcurveto{\pgfqpoint{1.798972in}{3.176910in}}{\pgfqpoint{1.809572in}{3.181301in}}{\pgfqpoint{1.817385in}{3.189114in}}%
\pgfpathcurveto{\pgfqpoint{1.825199in}{3.196928in}}{\pgfqpoint{1.829589in}{3.207527in}}{\pgfqpoint{1.829589in}{3.218577in}}%
\pgfpathcurveto{\pgfqpoint{1.829589in}{3.229627in}}{\pgfqpoint{1.825199in}{3.240226in}}{\pgfqpoint{1.817385in}{3.248040in}}%
\pgfpathcurveto{\pgfqpoint{1.809572in}{3.255853in}}{\pgfqpoint{1.798972in}{3.260244in}}{\pgfqpoint{1.787922in}{3.260244in}}%
\pgfpathcurveto{\pgfqpoint{1.776872in}{3.260244in}}{\pgfqpoint{1.766273in}{3.255853in}}{\pgfqpoint{1.758460in}{3.248040in}}%
\pgfpathcurveto{\pgfqpoint{1.750646in}{3.240226in}}{\pgfqpoint{1.746256in}{3.229627in}}{\pgfqpoint{1.746256in}{3.218577in}}%
\pgfpathcurveto{\pgfqpoint{1.746256in}{3.207527in}}{\pgfqpoint{1.750646in}{3.196928in}}{\pgfqpoint{1.758460in}{3.189114in}}%
\pgfpathcurveto{\pgfqpoint{1.766273in}{3.181301in}}{\pgfqpoint{1.776872in}{3.176910in}}{\pgfqpoint{1.787922in}{3.176910in}}%
\pgfpathclose%
\pgfusepath{stroke,fill}%
\end{pgfscope}%
\begin{pgfscope}%
\pgfpathrectangle{\pgfqpoint{0.787074in}{0.548769in}}{\pgfqpoint{5.062926in}{3.102590in}}%
\pgfusepath{clip}%
\pgfsetbuttcap%
\pgfsetroundjoin%
\definecolor{currentfill}{rgb}{1.000000,0.498039,0.054902}%
\pgfsetfillcolor{currentfill}%
\pgfsetlinewidth{1.003750pt}%
\definecolor{currentstroke}{rgb}{1.000000,0.498039,0.054902}%
\pgfsetstrokecolor{currentstroke}%
\pgfsetdash{}{0pt}%
\pgfpathmoveto{\pgfqpoint{1.225034in}{2.972026in}}%
\pgfpathcurveto{\pgfqpoint{1.236084in}{2.972026in}}{\pgfqpoint{1.246683in}{2.976417in}}{\pgfqpoint{1.254496in}{2.984230in}}%
\pgfpathcurveto{\pgfqpoint{1.262310in}{2.992044in}}{\pgfqpoint{1.266700in}{3.002643in}}{\pgfqpoint{1.266700in}{3.013693in}}%
\pgfpathcurveto{\pgfqpoint{1.266700in}{3.024743in}}{\pgfqpoint{1.262310in}{3.035342in}}{\pgfqpoint{1.254496in}{3.043156in}}%
\pgfpathcurveto{\pgfqpoint{1.246683in}{3.050969in}}{\pgfqpoint{1.236084in}{3.055360in}}{\pgfqpoint{1.225034in}{3.055360in}}%
\pgfpathcurveto{\pgfqpoint{1.213983in}{3.055360in}}{\pgfqpoint{1.203384in}{3.050969in}}{\pgfqpoint{1.195571in}{3.043156in}}%
\pgfpathcurveto{\pgfqpoint{1.187757in}{3.035342in}}{\pgfqpoint{1.183367in}{3.024743in}}{\pgfqpoint{1.183367in}{3.013693in}}%
\pgfpathcurveto{\pgfqpoint{1.183367in}{3.002643in}}{\pgfqpoint{1.187757in}{2.992044in}}{\pgfqpoint{1.195571in}{2.984230in}}%
\pgfpathcurveto{\pgfqpoint{1.203384in}{2.976417in}}{\pgfqpoint{1.213983in}{2.972026in}}{\pgfqpoint{1.225034in}{2.972026in}}%
\pgfpathclose%
\pgfusepath{stroke,fill}%
\end{pgfscope}%
\begin{pgfscope}%
\pgfpathrectangle{\pgfqpoint{0.787074in}{0.548769in}}{\pgfqpoint{5.062926in}{3.102590in}}%
\pgfusepath{clip}%
\pgfsetbuttcap%
\pgfsetroundjoin%
\definecolor{currentfill}{rgb}{1.000000,0.498039,0.054902}%
\pgfsetfillcolor{currentfill}%
\pgfsetlinewidth{1.003750pt}%
\definecolor{currentstroke}{rgb}{1.000000,0.498039,0.054902}%
\pgfsetstrokecolor{currentstroke}%
\pgfsetdash{}{0pt}%
\pgfpathmoveto{\pgfqpoint{2.275911in}{3.386124in}}%
\pgfpathcurveto{\pgfqpoint{2.286961in}{3.386124in}}{\pgfqpoint{2.297560in}{3.390514in}}{\pgfqpoint{2.305373in}{3.398328in}}%
\pgfpathcurveto{\pgfqpoint{2.313187in}{3.406142in}}{\pgfqpoint{2.317577in}{3.416741in}}{\pgfqpoint{2.317577in}{3.427791in}}%
\pgfpathcurveto{\pgfqpoint{2.317577in}{3.438841in}}{\pgfqpoint{2.313187in}{3.449440in}}{\pgfqpoint{2.305373in}{3.457254in}}%
\pgfpathcurveto{\pgfqpoint{2.297560in}{3.465067in}}{\pgfqpoint{2.286961in}{3.469458in}}{\pgfqpoint{2.275911in}{3.469458in}}%
\pgfpathcurveto{\pgfqpoint{2.264860in}{3.469458in}}{\pgfqpoint{2.254261in}{3.465067in}}{\pgfqpoint{2.246448in}{3.457254in}}%
\pgfpathcurveto{\pgfqpoint{2.238634in}{3.449440in}}{\pgfqpoint{2.234244in}{3.438841in}}{\pgfqpoint{2.234244in}{3.427791in}}%
\pgfpathcurveto{\pgfqpoint{2.234244in}{3.416741in}}{\pgfqpoint{2.238634in}{3.406142in}}{\pgfqpoint{2.246448in}{3.398328in}}%
\pgfpathcurveto{\pgfqpoint{2.254261in}{3.390514in}}{\pgfqpoint{2.264860in}{3.386124in}}{\pgfqpoint{2.275911in}{3.386124in}}%
\pgfpathclose%
\pgfusepath{stroke,fill}%
\end{pgfscope}%
\begin{pgfscope}%
\pgfpathrectangle{\pgfqpoint{0.787074in}{0.548769in}}{\pgfqpoint{5.062926in}{3.102590in}}%
\pgfusepath{clip}%
\pgfsetbuttcap%
\pgfsetroundjoin%
\definecolor{currentfill}{rgb}{1.000000,0.498039,0.054902}%
\pgfsetfillcolor{currentfill}%
\pgfsetlinewidth{1.003750pt}%
\definecolor{currentstroke}{rgb}{1.000000,0.498039,0.054902}%
\pgfsetstrokecolor{currentstroke}%
\pgfsetdash{}{0pt}%
\pgfpathmoveto{\pgfqpoint{1.484695in}{2.328792in}}%
\pgfpathcurveto{\pgfqpoint{1.495745in}{2.328792in}}{\pgfqpoint{1.506344in}{2.333182in}}{\pgfqpoint{1.514158in}{2.340995in}}%
\pgfpathcurveto{\pgfqpoint{1.521971in}{2.348809in}}{\pgfqpoint{1.526361in}{2.359408in}}{\pgfqpoint{1.526361in}{2.370458in}}%
\pgfpathcurveto{\pgfqpoint{1.526361in}{2.381508in}}{\pgfqpoint{1.521971in}{2.392107in}}{\pgfqpoint{1.514158in}{2.399921in}}%
\pgfpathcurveto{\pgfqpoint{1.506344in}{2.407735in}}{\pgfqpoint{1.495745in}{2.412125in}}{\pgfqpoint{1.484695in}{2.412125in}}%
\pgfpathcurveto{\pgfqpoint{1.473645in}{2.412125in}}{\pgfqpoint{1.463046in}{2.407735in}}{\pgfqpoint{1.455232in}{2.399921in}}%
\pgfpathcurveto{\pgfqpoint{1.447418in}{2.392107in}}{\pgfqpoint{1.443028in}{2.381508in}}{\pgfqpoint{1.443028in}{2.370458in}}%
\pgfpathcurveto{\pgfqpoint{1.443028in}{2.359408in}}{\pgfqpoint{1.447418in}{2.348809in}}{\pgfqpoint{1.455232in}{2.340995in}}%
\pgfpathcurveto{\pgfqpoint{1.463046in}{2.333182in}}{\pgfqpoint{1.473645in}{2.328792in}}{\pgfqpoint{1.484695in}{2.328792in}}%
\pgfpathclose%
\pgfusepath{stroke,fill}%
\end{pgfscope}%
\begin{pgfscope}%
\pgfpathrectangle{\pgfqpoint{0.787074in}{0.548769in}}{\pgfqpoint{5.062926in}{3.102590in}}%
\pgfusepath{clip}%
\pgfsetbuttcap%
\pgfsetroundjoin%
\definecolor{currentfill}{rgb}{1.000000,0.498039,0.054902}%
\pgfsetfillcolor{currentfill}%
\pgfsetlinewidth{1.003750pt}%
\definecolor{currentstroke}{rgb}{1.000000,0.498039,0.054902}%
\pgfsetstrokecolor{currentstroke}%
\pgfsetdash{}{0pt}%
\pgfpathmoveto{\pgfqpoint{1.977144in}{3.073267in}}%
\pgfpathcurveto{\pgfqpoint{1.988195in}{3.073267in}}{\pgfqpoint{1.998794in}{3.077657in}}{\pgfqpoint{2.006607in}{3.085471in}}%
\pgfpathcurveto{\pgfqpoint{2.014421in}{3.093285in}}{\pgfqpoint{2.018811in}{3.103884in}}{\pgfqpoint{2.018811in}{3.114934in}}%
\pgfpathcurveto{\pgfqpoint{2.018811in}{3.125984in}}{\pgfqpoint{2.014421in}{3.136583in}}{\pgfqpoint{2.006607in}{3.144396in}}%
\pgfpathcurveto{\pgfqpoint{1.998794in}{3.152210in}}{\pgfqpoint{1.988195in}{3.156600in}}{\pgfqpoint{1.977144in}{3.156600in}}%
\pgfpathcurveto{\pgfqpoint{1.966094in}{3.156600in}}{\pgfqpoint{1.955495in}{3.152210in}}{\pgfqpoint{1.947682in}{3.144396in}}%
\pgfpathcurveto{\pgfqpoint{1.939868in}{3.136583in}}{\pgfqpoint{1.935478in}{3.125984in}}{\pgfqpoint{1.935478in}{3.114934in}}%
\pgfpathcurveto{\pgfqpoint{1.935478in}{3.103884in}}{\pgfqpoint{1.939868in}{3.093285in}}{\pgfqpoint{1.947682in}{3.085471in}}%
\pgfpathcurveto{\pgfqpoint{1.955495in}{3.077657in}}{\pgfqpoint{1.966094in}{3.073267in}}{\pgfqpoint{1.977144in}{3.073267in}}%
\pgfpathclose%
\pgfusepath{stroke,fill}%
\end{pgfscope}%
\begin{pgfscope}%
\pgfpathrectangle{\pgfqpoint{0.787074in}{0.548769in}}{\pgfqpoint{5.062926in}{3.102590in}}%
\pgfusepath{clip}%
\pgfsetbuttcap%
\pgfsetroundjoin%
\definecolor{currentfill}{rgb}{0.121569,0.466667,0.705882}%
\pgfsetfillcolor{currentfill}%
\pgfsetlinewidth{1.003750pt}%
\definecolor{currentstroke}{rgb}{0.121569,0.466667,0.705882}%
\pgfsetstrokecolor{currentstroke}%
\pgfsetdash{}{0pt}%
\pgfpathmoveto{\pgfqpoint{1.350227in}{0.648131in}}%
\pgfpathcurveto{\pgfqpoint{1.361278in}{0.648131in}}{\pgfqpoint{1.371877in}{0.652521in}}{\pgfqpoint{1.379690in}{0.660335in}}%
\pgfpathcurveto{\pgfqpoint{1.387504in}{0.668148in}}{\pgfqpoint{1.391894in}{0.678748in}}{\pgfqpoint{1.391894in}{0.689798in}}%
\pgfpathcurveto{\pgfqpoint{1.391894in}{0.700848in}}{\pgfqpoint{1.387504in}{0.711447in}}{\pgfqpoint{1.379690in}{0.719260in}}%
\pgfpathcurveto{\pgfqpoint{1.371877in}{0.727074in}}{\pgfqpoint{1.361278in}{0.731464in}}{\pgfqpoint{1.350227in}{0.731464in}}%
\pgfpathcurveto{\pgfqpoint{1.339177in}{0.731464in}}{\pgfqpoint{1.328578in}{0.727074in}}{\pgfqpoint{1.320765in}{0.719260in}}%
\pgfpathcurveto{\pgfqpoint{1.312951in}{0.711447in}}{\pgfqpoint{1.308561in}{0.700848in}}{\pgfqpoint{1.308561in}{0.689798in}}%
\pgfpathcurveto{\pgfqpoint{1.308561in}{0.678748in}}{\pgfqpoint{1.312951in}{0.668148in}}{\pgfqpoint{1.320765in}{0.660335in}}%
\pgfpathcurveto{\pgfqpoint{1.328578in}{0.652521in}}{\pgfqpoint{1.339177in}{0.648131in}}{\pgfqpoint{1.350227in}{0.648131in}}%
\pgfpathclose%
\pgfusepath{stroke,fill}%
\end{pgfscope}%
\begin{pgfscope}%
\pgfpathrectangle{\pgfqpoint{0.787074in}{0.548769in}}{\pgfqpoint{5.062926in}{3.102590in}}%
\pgfusepath{clip}%
\pgfsetbuttcap%
\pgfsetroundjoin%
\definecolor{currentfill}{rgb}{1.000000,0.498039,0.054902}%
\pgfsetfillcolor{currentfill}%
\pgfsetlinewidth{1.003750pt}%
\definecolor{currentstroke}{rgb}{1.000000,0.498039,0.054902}%
\pgfsetstrokecolor{currentstroke}%
\pgfsetdash{}{0pt}%
\pgfpathmoveto{\pgfqpoint{2.270906in}{1.987192in}}%
\pgfpathcurveto{\pgfqpoint{2.281956in}{1.987192in}}{\pgfqpoint{2.292555in}{1.991582in}}{\pgfqpoint{2.300369in}{1.999395in}}%
\pgfpathcurveto{\pgfqpoint{2.308182in}{2.007209in}}{\pgfqpoint{2.312573in}{2.017808in}}{\pgfqpoint{2.312573in}{2.028858in}}%
\pgfpathcurveto{\pgfqpoint{2.312573in}{2.039908in}}{\pgfqpoint{2.308182in}{2.050507in}}{\pgfqpoint{2.300369in}{2.058321in}}%
\pgfpathcurveto{\pgfqpoint{2.292555in}{2.066135in}}{\pgfqpoint{2.281956in}{2.070525in}}{\pgfqpoint{2.270906in}{2.070525in}}%
\pgfpathcurveto{\pgfqpoint{2.259856in}{2.070525in}}{\pgfqpoint{2.249257in}{2.066135in}}{\pgfqpoint{2.241443in}{2.058321in}}%
\pgfpathcurveto{\pgfqpoint{2.233629in}{2.050507in}}{\pgfqpoint{2.229239in}{2.039908in}}{\pgfqpoint{2.229239in}{2.028858in}}%
\pgfpathcurveto{\pgfqpoint{2.229239in}{2.017808in}}{\pgfqpoint{2.233629in}{2.007209in}}{\pgfqpoint{2.241443in}{1.999395in}}%
\pgfpathcurveto{\pgfqpoint{2.249257in}{1.991582in}}{\pgfqpoint{2.259856in}{1.987192in}}{\pgfqpoint{2.270906in}{1.987192in}}%
\pgfpathclose%
\pgfusepath{stroke,fill}%
\end{pgfscope}%
\begin{pgfscope}%
\pgfpathrectangle{\pgfqpoint{0.787074in}{0.548769in}}{\pgfqpoint{5.062926in}{3.102590in}}%
\pgfusepath{clip}%
\pgfsetbuttcap%
\pgfsetroundjoin%
\definecolor{currentfill}{rgb}{1.000000,0.498039,0.054902}%
\pgfsetfillcolor{currentfill}%
\pgfsetlinewidth{1.003750pt}%
\definecolor{currentstroke}{rgb}{1.000000,0.498039,0.054902}%
\pgfsetstrokecolor{currentstroke}%
\pgfsetdash{}{0pt}%
\pgfpathmoveto{\pgfqpoint{1.313083in}{2.820143in}}%
\pgfpathcurveto{\pgfqpoint{1.324134in}{2.820143in}}{\pgfqpoint{1.334733in}{2.824533in}}{\pgfqpoint{1.342546in}{2.832347in}}%
\pgfpathcurveto{\pgfqpoint{1.350360in}{2.840160in}}{\pgfqpoint{1.354750in}{2.850759in}}{\pgfqpoint{1.354750in}{2.861810in}}%
\pgfpathcurveto{\pgfqpoint{1.354750in}{2.872860in}}{\pgfqpoint{1.350360in}{2.883459in}}{\pgfqpoint{1.342546in}{2.891272in}}%
\pgfpathcurveto{\pgfqpoint{1.334733in}{2.899086in}}{\pgfqpoint{1.324134in}{2.903476in}}{\pgfqpoint{1.313083in}{2.903476in}}%
\pgfpathcurveto{\pgfqpoint{1.302033in}{2.903476in}}{\pgfqpoint{1.291434in}{2.899086in}}{\pgfqpoint{1.283621in}{2.891272in}}%
\pgfpathcurveto{\pgfqpoint{1.275807in}{2.883459in}}{\pgfqpoint{1.271417in}{2.872860in}}{\pgfqpoint{1.271417in}{2.861810in}}%
\pgfpathcurveto{\pgfqpoint{1.271417in}{2.850759in}}{\pgfqpoint{1.275807in}{2.840160in}}{\pgfqpoint{1.283621in}{2.832347in}}%
\pgfpathcurveto{\pgfqpoint{1.291434in}{2.824533in}}{\pgfqpoint{1.302033in}{2.820143in}}{\pgfqpoint{1.313083in}{2.820143in}}%
\pgfpathclose%
\pgfusepath{stroke,fill}%
\end{pgfscope}%
\begin{pgfscope}%
\pgfpathrectangle{\pgfqpoint{0.787074in}{0.548769in}}{\pgfqpoint{5.062926in}{3.102590in}}%
\pgfusepath{clip}%
\pgfsetbuttcap%
\pgfsetroundjoin%
\definecolor{currentfill}{rgb}{1.000000,0.498039,0.054902}%
\pgfsetfillcolor{currentfill}%
\pgfsetlinewidth{1.003750pt}%
\definecolor{currentstroke}{rgb}{1.000000,0.498039,0.054902}%
\pgfsetstrokecolor{currentstroke}%
\pgfsetdash{}{0pt}%
\pgfpathmoveto{\pgfqpoint{1.294849in}{2.523468in}}%
\pgfpathcurveto{\pgfqpoint{1.305899in}{2.523468in}}{\pgfqpoint{1.316498in}{2.527858in}}{\pgfqpoint{1.324311in}{2.535672in}}%
\pgfpathcurveto{\pgfqpoint{1.332125in}{2.543485in}}{\pgfqpoint{1.336515in}{2.554084in}}{\pgfqpoint{1.336515in}{2.565134in}}%
\pgfpathcurveto{\pgfqpoint{1.336515in}{2.576185in}}{\pgfqpoint{1.332125in}{2.586784in}}{\pgfqpoint{1.324311in}{2.594597in}}%
\pgfpathcurveto{\pgfqpoint{1.316498in}{2.602411in}}{\pgfqpoint{1.305899in}{2.606801in}}{\pgfqpoint{1.294849in}{2.606801in}}%
\pgfpathcurveto{\pgfqpoint{1.283798in}{2.606801in}}{\pgfqpoint{1.273199in}{2.602411in}}{\pgfqpoint{1.265386in}{2.594597in}}%
\pgfpathcurveto{\pgfqpoint{1.257572in}{2.586784in}}{\pgfqpoint{1.253182in}{2.576185in}}{\pgfqpoint{1.253182in}{2.565134in}}%
\pgfpathcurveto{\pgfqpoint{1.253182in}{2.554084in}}{\pgfqpoint{1.257572in}{2.543485in}}{\pgfqpoint{1.265386in}{2.535672in}}%
\pgfpathcurveto{\pgfqpoint{1.273199in}{2.527858in}}{\pgfqpoint{1.283798in}{2.523468in}}{\pgfqpoint{1.294849in}{2.523468in}}%
\pgfpathclose%
\pgfusepath{stroke,fill}%
\end{pgfscope}%
\begin{pgfscope}%
\pgfpathrectangle{\pgfqpoint{0.787074in}{0.548769in}}{\pgfqpoint{5.062926in}{3.102590in}}%
\pgfusepath{clip}%
\pgfsetbuttcap%
\pgfsetroundjoin%
\definecolor{currentfill}{rgb}{1.000000,0.498039,0.054902}%
\pgfsetfillcolor{currentfill}%
\pgfsetlinewidth{1.003750pt}%
\definecolor{currentstroke}{rgb}{1.000000,0.498039,0.054902}%
\pgfsetstrokecolor{currentstroke}%
\pgfsetdash{}{0pt}%
\pgfpathmoveto{\pgfqpoint{2.166370in}{1.569832in}}%
\pgfpathcurveto{\pgfqpoint{2.177420in}{1.569832in}}{\pgfqpoint{2.188019in}{1.574223in}}{\pgfqpoint{2.195833in}{1.582036in}}%
\pgfpathcurveto{\pgfqpoint{2.203647in}{1.589850in}}{\pgfqpoint{2.208037in}{1.600449in}}{\pgfqpoint{2.208037in}{1.611499in}}%
\pgfpathcurveto{\pgfqpoint{2.208037in}{1.622549in}}{\pgfqpoint{2.203647in}{1.633148in}}{\pgfqpoint{2.195833in}{1.640962in}}%
\pgfpathcurveto{\pgfqpoint{2.188019in}{1.648775in}}{\pgfqpoint{2.177420in}{1.653166in}}{\pgfqpoint{2.166370in}{1.653166in}}%
\pgfpathcurveto{\pgfqpoint{2.155320in}{1.653166in}}{\pgfqpoint{2.144721in}{1.648775in}}{\pgfqpoint{2.136908in}{1.640962in}}%
\pgfpathcurveto{\pgfqpoint{2.129094in}{1.633148in}}{\pgfqpoint{2.124704in}{1.622549in}}{\pgfqpoint{2.124704in}{1.611499in}}%
\pgfpathcurveto{\pgfqpoint{2.124704in}{1.600449in}}{\pgfqpoint{2.129094in}{1.589850in}}{\pgfqpoint{2.136908in}{1.582036in}}%
\pgfpathcurveto{\pgfqpoint{2.144721in}{1.574223in}}{\pgfqpoint{2.155320in}{1.569832in}}{\pgfqpoint{2.166370in}{1.569832in}}%
\pgfpathclose%
\pgfusepath{stroke,fill}%
\end{pgfscope}%
\begin{pgfscope}%
\pgfpathrectangle{\pgfqpoint{0.787074in}{0.548769in}}{\pgfqpoint{5.062926in}{3.102590in}}%
\pgfusepath{clip}%
\pgfsetbuttcap%
\pgfsetroundjoin%
\definecolor{currentfill}{rgb}{1.000000,0.498039,0.054902}%
\pgfsetfillcolor{currentfill}%
\pgfsetlinewidth{1.003750pt}%
\definecolor{currentstroke}{rgb}{1.000000,0.498039,0.054902}%
\pgfsetstrokecolor{currentstroke}%
\pgfsetdash{}{0pt}%
\pgfpathmoveto{\pgfqpoint{1.775640in}{2.228984in}}%
\pgfpathcurveto{\pgfqpoint{1.786690in}{2.228984in}}{\pgfqpoint{1.797289in}{2.233374in}}{\pgfqpoint{1.805103in}{2.241188in}}%
\pgfpathcurveto{\pgfqpoint{1.812916in}{2.249002in}}{\pgfqpoint{1.817307in}{2.259601in}}{\pgfqpoint{1.817307in}{2.270651in}}%
\pgfpathcurveto{\pgfqpoint{1.817307in}{2.281701in}}{\pgfqpoint{1.812916in}{2.292300in}}{\pgfqpoint{1.805103in}{2.300114in}}%
\pgfpathcurveto{\pgfqpoint{1.797289in}{2.307927in}}{\pgfqpoint{1.786690in}{2.312318in}}{\pgfqpoint{1.775640in}{2.312318in}}%
\pgfpathcurveto{\pgfqpoint{1.764590in}{2.312318in}}{\pgfqpoint{1.753991in}{2.307927in}}{\pgfqpoint{1.746177in}{2.300114in}}%
\pgfpathcurveto{\pgfqpoint{1.738364in}{2.292300in}}{\pgfqpoint{1.733973in}{2.281701in}}{\pgfqpoint{1.733973in}{2.270651in}}%
\pgfpathcurveto{\pgfqpoint{1.733973in}{2.259601in}}{\pgfqpoint{1.738364in}{2.249002in}}{\pgfqpoint{1.746177in}{2.241188in}}%
\pgfpathcurveto{\pgfqpoint{1.753991in}{2.233374in}}{\pgfqpoint{1.764590in}{2.228984in}}{\pgfqpoint{1.775640in}{2.228984in}}%
\pgfpathclose%
\pgfusepath{stroke,fill}%
\end{pgfscope}%
\begin{pgfscope}%
\pgfpathrectangle{\pgfqpoint{0.787074in}{0.548769in}}{\pgfqpoint{5.062926in}{3.102590in}}%
\pgfusepath{clip}%
\pgfsetbuttcap%
\pgfsetroundjoin%
\definecolor{currentfill}{rgb}{0.121569,0.466667,0.705882}%
\pgfsetfillcolor{currentfill}%
\pgfsetlinewidth{1.003750pt}%
\definecolor{currentstroke}{rgb}{0.121569,0.466667,0.705882}%
\pgfsetstrokecolor{currentstroke}%
\pgfsetdash{}{0pt}%
\pgfpathmoveto{\pgfqpoint{2.247478in}{2.843750in}}%
\pgfpathcurveto{\pgfqpoint{2.258528in}{2.843750in}}{\pgfqpoint{2.269127in}{2.848141in}}{\pgfqpoint{2.276940in}{2.855954in}}%
\pgfpathcurveto{\pgfqpoint{2.284754in}{2.863768in}}{\pgfqpoint{2.289144in}{2.874367in}}{\pgfqpoint{2.289144in}{2.885417in}}%
\pgfpathcurveto{\pgfqpoint{2.289144in}{2.896467in}}{\pgfqpoint{2.284754in}{2.907066in}}{\pgfqpoint{2.276940in}{2.914880in}}%
\pgfpathcurveto{\pgfqpoint{2.269127in}{2.922693in}}{\pgfqpoint{2.258528in}{2.927084in}}{\pgfqpoint{2.247478in}{2.927084in}}%
\pgfpathcurveto{\pgfqpoint{2.236427in}{2.927084in}}{\pgfqpoint{2.225828in}{2.922693in}}{\pgfqpoint{2.218015in}{2.914880in}}%
\pgfpathcurveto{\pgfqpoint{2.210201in}{2.907066in}}{\pgfqpoint{2.205811in}{2.896467in}}{\pgfqpoint{2.205811in}{2.885417in}}%
\pgfpathcurveto{\pgfqpoint{2.205811in}{2.874367in}}{\pgfqpoint{2.210201in}{2.863768in}}{\pgfqpoint{2.218015in}{2.855954in}}%
\pgfpathcurveto{\pgfqpoint{2.225828in}{2.848141in}}{\pgfqpoint{2.236427in}{2.843750in}}{\pgfqpoint{2.247478in}{2.843750in}}%
\pgfpathclose%
\pgfusepath{stroke,fill}%
\end{pgfscope}%
\begin{pgfscope}%
\pgfpathrectangle{\pgfqpoint{0.787074in}{0.548769in}}{\pgfqpoint{5.062926in}{3.102590in}}%
\pgfusepath{clip}%
\pgfsetbuttcap%
\pgfsetroundjoin%
\definecolor{currentfill}{rgb}{1.000000,0.498039,0.054902}%
\pgfsetfillcolor{currentfill}%
\pgfsetlinewidth{1.003750pt}%
\definecolor{currentstroke}{rgb}{1.000000,0.498039,0.054902}%
\pgfsetstrokecolor{currentstroke}%
\pgfsetdash{}{0pt}%
\pgfpathmoveto{\pgfqpoint{2.154824in}{3.155929in}}%
\pgfpathcurveto{\pgfqpoint{2.165874in}{3.155929in}}{\pgfqpoint{2.176473in}{3.160320in}}{\pgfqpoint{2.184286in}{3.168133in}}%
\pgfpathcurveto{\pgfqpoint{2.192100in}{3.175947in}}{\pgfqpoint{2.196490in}{3.186546in}}{\pgfqpoint{2.196490in}{3.197596in}}%
\pgfpathcurveto{\pgfqpoint{2.196490in}{3.208646in}}{\pgfqpoint{2.192100in}{3.219245in}}{\pgfqpoint{2.184286in}{3.227059in}}%
\pgfpathcurveto{\pgfqpoint{2.176473in}{3.234872in}}{\pgfqpoint{2.165874in}{3.239263in}}{\pgfqpoint{2.154824in}{3.239263in}}%
\pgfpathcurveto{\pgfqpoint{2.143774in}{3.239263in}}{\pgfqpoint{2.133175in}{3.234872in}}{\pgfqpoint{2.125361in}{3.227059in}}%
\pgfpathcurveto{\pgfqpoint{2.117547in}{3.219245in}}{\pgfqpoint{2.113157in}{3.208646in}}{\pgfqpoint{2.113157in}{3.197596in}}%
\pgfpathcurveto{\pgfqpoint{2.113157in}{3.186546in}}{\pgfqpoint{2.117547in}{3.175947in}}{\pgfqpoint{2.125361in}{3.168133in}}%
\pgfpathcurveto{\pgfqpoint{2.133175in}{3.160320in}}{\pgfqpoint{2.143774in}{3.155929in}}{\pgfqpoint{2.154824in}{3.155929in}}%
\pgfpathclose%
\pgfusepath{stroke,fill}%
\end{pgfscope}%
\begin{pgfscope}%
\pgfsetbuttcap%
\pgfsetroundjoin%
\definecolor{currentfill}{rgb}{0.000000,0.000000,0.000000}%
\pgfsetfillcolor{currentfill}%
\pgfsetlinewidth{0.803000pt}%
\definecolor{currentstroke}{rgb}{0.000000,0.000000,0.000000}%
\pgfsetstrokecolor{currentstroke}%
\pgfsetdash{}{0pt}%
\pgfsys@defobject{currentmarker}{\pgfqpoint{0.000000in}{-0.048611in}}{\pgfqpoint{0.000000in}{0.000000in}}{%
\pgfpathmoveto{\pgfqpoint{0.000000in}{0.000000in}}%
\pgfpathlineto{\pgfqpoint{0.000000in}{-0.048611in}}%
\pgfusepath{stroke,fill}%
}%
\begin{pgfscope}%
\pgfsys@transformshift{1.005367in}{0.548769in}%
\pgfsys@useobject{currentmarker}{}%
\end{pgfscope}%
\end{pgfscope}%
\begin{pgfscope}%
\definecolor{textcolor}{rgb}{0.000000,0.000000,0.000000}%
\pgfsetstrokecolor{textcolor}%
\pgfsetfillcolor{textcolor}%
\pgftext[x=1.005367in,y=0.451547in,,top]{\color{textcolor}\sffamily\fontsize{10.000000}{12.000000}\selectfont \(\displaystyle {0.0}\)}%
\end{pgfscope}%
\begin{pgfscope}%
\pgfsetbuttcap%
\pgfsetroundjoin%
\definecolor{currentfill}{rgb}{0.000000,0.000000,0.000000}%
\pgfsetfillcolor{currentfill}%
\pgfsetlinewidth{0.803000pt}%
\definecolor{currentstroke}{rgb}{0.000000,0.000000,0.000000}%
\pgfsetstrokecolor{currentstroke}%
\pgfsetdash{}{0pt}%
\pgfsys@defobject{currentmarker}{\pgfqpoint{0.000000in}{-0.048611in}}{\pgfqpoint{0.000000in}{0.000000in}}{%
\pgfpathmoveto{\pgfqpoint{0.000000in}{0.000000in}}%
\pgfpathlineto{\pgfqpoint{0.000000in}{-0.048611in}}%
\pgfusepath{stroke,fill}%
}%
\begin{pgfscope}%
\pgfsys@transformshift{1.775910in}{0.548769in}%
\pgfsys@useobject{currentmarker}{}%
\end{pgfscope}%
\end{pgfscope}%
\begin{pgfscope}%
\definecolor{textcolor}{rgb}{0.000000,0.000000,0.000000}%
\pgfsetstrokecolor{textcolor}%
\pgfsetfillcolor{textcolor}%
\pgftext[x=1.775910in,y=0.451547in,,top]{\color{textcolor}\sffamily\fontsize{10.000000}{12.000000}\selectfont \(\displaystyle {0.2}\)}%
\end{pgfscope}%
\begin{pgfscope}%
\pgfsetbuttcap%
\pgfsetroundjoin%
\definecolor{currentfill}{rgb}{0.000000,0.000000,0.000000}%
\pgfsetfillcolor{currentfill}%
\pgfsetlinewidth{0.803000pt}%
\definecolor{currentstroke}{rgb}{0.000000,0.000000,0.000000}%
\pgfsetstrokecolor{currentstroke}%
\pgfsetdash{}{0pt}%
\pgfsys@defobject{currentmarker}{\pgfqpoint{0.000000in}{-0.048611in}}{\pgfqpoint{0.000000in}{0.000000in}}{%
\pgfpathmoveto{\pgfqpoint{0.000000in}{0.000000in}}%
\pgfpathlineto{\pgfqpoint{0.000000in}{-0.048611in}}%
\pgfusepath{stroke,fill}%
}%
\begin{pgfscope}%
\pgfsys@transformshift{2.546452in}{0.548769in}%
\pgfsys@useobject{currentmarker}{}%
\end{pgfscope}%
\end{pgfscope}%
\begin{pgfscope}%
\definecolor{textcolor}{rgb}{0.000000,0.000000,0.000000}%
\pgfsetstrokecolor{textcolor}%
\pgfsetfillcolor{textcolor}%
\pgftext[x=2.546452in,y=0.451547in,,top]{\color{textcolor}\sffamily\fontsize{10.000000}{12.000000}\selectfont \(\displaystyle {0.4}\)}%
\end{pgfscope}%
\begin{pgfscope}%
\pgfsetbuttcap%
\pgfsetroundjoin%
\definecolor{currentfill}{rgb}{0.000000,0.000000,0.000000}%
\pgfsetfillcolor{currentfill}%
\pgfsetlinewidth{0.803000pt}%
\definecolor{currentstroke}{rgb}{0.000000,0.000000,0.000000}%
\pgfsetstrokecolor{currentstroke}%
\pgfsetdash{}{0pt}%
\pgfsys@defobject{currentmarker}{\pgfqpoint{0.000000in}{-0.048611in}}{\pgfqpoint{0.000000in}{0.000000in}}{%
\pgfpathmoveto{\pgfqpoint{0.000000in}{0.000000in}}%
\pgfpathlineto{\pgfqpoint{0.000000in}{-0.048611in}}%
\pgfusepath{stroke,fill}%
}%
\begin{pgfscope}%
\pgfsys@transformshift{3.316994in}{0.548769in}%
\pgfsys@useobject{currentmarker}{}%
\end{pgfscope}%
\end{pgfscope}%
\begin{pgfscope}%
\definecolor{textcolor}{rgb}{0.000000,0.000000,0.000000}%
\pgfsetstrokecolor{textcolor}%
\pgfsetfillcolor{textcolor}%
\pgftext[x=3.316994in,y=0.451547in,,top]{\color{textcolor}\sffamily\fontsize{10.000000}{12.000000}\selectfont \(\displaystyle {0.6}\)}%
\end{pgfscope}%
\begin{pgfscope}%
\pgfsetbuttcap%
\pgfsetroundjoin%
\definecolor{currentfill}{rgb}{0.000000,0.000000,0.000000}%
\pgfsetfillcolor{currentfill}%
\pgfsetlinewidth{0.803000pt}%
\definecolor{currentstroke}{rgb}{0.000000,0.000000,0.000000}%
\pgfsetstrokecolor{currentstroke}%
\pgfsetdash{}{0pt}%
\pgfsys@defobject{currentmarker}{\pgfqpoint{0.000000in}{-0.048611in}}{\pgfqpoint{0.000000in}{0.000000in}}{%
\pgfpathmoveto{\pgfqpoint{0.000000in}{0.000000in}}%
\pgfpathlineto{\pgfqpoint{0.000000in}{-0.048611in}}%
\pgfusepath{stroke,fill}%
}%
\begin{pgfscope}%
\pgfsys@transformshift{4.087536in}{0.548769in}%
\pgfsys@useobject{currentmarker}{}%
\end{pgfscope}%
\end{pgfscope}%
\begin{pgfscope}%
\definecolor{textcolor}{rgb}{0.000000,0.000000,0.000000}%
\pgfsetstrokecolor{textcolor}%
\pgfsetfillcolor{textcolor}%
\pgftext[x=4.087536in,y=0.451547in,,top]{\color{textcolor}\sffamily\fontsize{10.000000}{12.000000}\selectfont \(\displaystyle {0.8}\)}%
\end{pgfscope}%
\begin{pgfscope}%
\pgfsetbuttcap%
\pgfsetroundjoin%
\definecolor{currentfill}{rgb}{0.000000,0.000000,0.000000}%
\pgfsetfillcolor{currentfill}%
\pgfsetlinewidth{0.803000pt}%
\definecolor{currentstroke}{rgb}{0.000000,0.000000,0.000000}%
\pgfsetstrokecolor{currentstroke}%
\pgfsetdash{}{0pt}%
\pgfsys@defobject{currentmarker}{\pgfqpoint{0.000000in}{-0.048611in}}{\pgfqpoint{0.000000in}{0.000000in}}{%
\pgfpathmoveto{\pgfqpoint{0.000000in}{0.000000in}}%
\pgfpathlineto{\pgfqpoint{0.000000in}{-0.048611in}}%
\pgfusepath{stroke,fill}%
}%
\begin{pgfscope}%
\pgfsys@transformshift{4.858078in}{0.548769in}%
\pgfsys@useobject{currentmarker}{}%
\end{pgfscope}%
\end{pgfscope}%
\begin{pgfscope}%
\definecolor{textcolor}{rgb}{0.000000,0.000000,0.000000}%
\pgfsetstrokecolor{textcolor}%
\pgfsetfillcolor{textcolor}%
\pgftext[x=4.858078in,y=0.451547in,,top]{\color{textcolor}\sffamily\fontsize{10.000000}{12.000000}\selectfont \(\displaystyle {1.0}\)}%
\end{pgfscope}%
\begin{pgfscope}%
\pgfsetbuttcap%
\pgfsetroundjoin%
\definecolor{currentfill}{rgb}{0.000000,0.000000,0.000000}%
\pgfsetfillcolor{currentfill}%
\pgfsetlinewidth{0.803000pt}%
\definecolor{currentstroke}{rgb}{0.000000,0.000000,0.000000}%
\pgfsetstrokecolor{currentstroke}%
\pgfsetdash{}{0pt}%
\pgfsys@defobject{currentmarker}{\pgfqpoint{0.000000in}{-0.048611in}}{\pgfqpoint{0.000000in}{0.000000in}}{%
\pgfpathmoveto{\pgfqpoint{0.000000in}{0.000000in}}%
\pgfpathlineto{\pgfqpoint{0.000000in}{-0.048611in}}%
\pgfusepath{stroke,fill}%
}%
\begin{pgfscope}%
\pgfsys@transformshift{5.628620in}{0.548769in}%
\pgfsys@useobject{currentmarker}{}%
\end{pgfscope}%
\end{pgfscope}%
\begin{pgfscope}%
\definecolor{textcolor}{rgb}{0.000000,0.000000,0.000000}%
\pgfsetstrokecolor{textcolor}%
\pgfsetfillcolor{textcolor}%
\pgftext[x=5.628620in,y=0.451547in,,top]{\color{textcolor}\sffamily\fontsize{10.000000}{12.000000}\selectfont \(\displaystyle {1.2}\)}%
\end{pgfscope}%
\begin{pgfscope}%
\definecolor{textcolor}{rgb}{0.000000,0.000000,0.000000}%
\pgfsetstrokecolor{textcolor}%
\pgfsetfillcolor{textcolor}%
\pgftext[x=3.318537in,y=0.272658in,,top]{\color{textcolor}\sffamily\fontsize{10.000000}{12.000000}\selectfont Statements}%
\end{pgfscope}%
\begin{pgfscope}%
\definecolor{textcolor}{rgb}{0.000000,0.000000,0.000000}%
\pgfsetstrokecolor{textcolor}%
\pgfsetfillcolor{textcolor}%
\pgftext[x=5.850000in,y=0.286547in,right,top]{\color{textcolor}\sffamily\fontsize{10.000000}{12.000000}\selectfont \(\displaystyle \times{10^{6}}{}\)}%
\end{pgfscope}%
\begin{pgfscope}%
\pgfsetbuttcap%
\pgfsetroundjoin%
\definecolor{currentfill}{rgb}{0.000000,0.000000,0.000000}%
\pgfsetfillcolor{currentfill}%
\pgfsetlinewidth{0.803000pt}%
\definecolor{currentstroke}{rgb}{0.000000,0.000000,0.000000}%
\pgfsetstrokecolor{currentstroke}%
\pgfsetdash{}{0pt}%
\pgfsys@defobject{currentmarker}{\pgfqpoint{-0.048611in}{0.000000in}}{\pgfqpoint{0.000000in}{0.000000in}}{%
\pgfpathmoveto{\pgfqpoint{0.000000in}{0.000000in}}%
\pgfpathlineto{\pgfqpoint{-0.048611in}{0.000000in}}%
\pgfusepath{stroke,fill}%
}%
\begin{pgfscope}%
\pgfsys@transformshift{0.787074in}{0.689795in}%
\pgfsys@useobject{currentmarker}{}%
\end{pgfscope}%
\end{pgfscope}%
\begin{pgfscope}%
\definecolor{textcolor}{rgb}{0.000000,0.000000,0.000000}%
\pgfsetstrokecolor{textcolor}%
\pgfsetfillcolor{textcolor}%
\pgftext[x=0.620407in, y=0.641601in, left, base]{\color{textcolor}\sffamily\fontsize{10.000000}{12.000000}\selectfont \(\displaystyle {0}\)}%
\end{pgfscope}%
\begin{pgfscope}%
\pgfsetbuttcap%
\pgfsetroundjoin%
\definecolor{currentfill}{rgb}{0.000000,0.000000,0.000000}%
\pgfsetfillcolor{currentfill}%
\pgfsetlinewidth{0.803000pt}%
\definecolor{currentstroke}{rgb}{0.000000,0.000000,0.000000}%
\pgfsetstrokecolor{currentstroke}%
\pgfsetdash{}{0pt}%
\pgfsys@defobject{currentmarker}{\pgfqpoint{-0.048611in}{0.000000in}}{\pgfqpoint{0.000000in}{0.000000in}}{%
\pgfpathmoveto{\pgfqpoint{0.000000in}{0.000000in}}%
\pgfpathlineto{\pgfqpoint{-0.048611in}{0.000000in}}%
\pgfusepath{stroke,fill}%
}%
\begin{pgfscope}%
\pgfsys@transformshift{0.787074in}{1.059666in}%
\pgfsys@useobject{currentmarker}{}%
\end{pgfscope}%
\end{pgfscope}%
\begin{pgfscope}%
\definecolor{textcolor}{rgb}{0.000000,0.000000,0.000000}%
\pgfsetstrokecolor{textcolor}%
\pgfsetfillcolor{textcolor}%
\pgftext[x=0.412073in, y=1.011471in, left, base]{\color{textcolor}\sffamily\fontsize{10.000000}{12.000000}\selectfont \(\displaystyle {2500}\)}%
\end{pgfscope}%
\begin{pgfscope}%
\pgfsetbuttcap%
\pgfsetroundjoin%
\definecolor{currentfill}{rgb}{0.000000,0.000000,0.000000}%
\pgfsetfillcolor{currentfill}%
\pgfsetlinewidth{0.803000pt}%
\definecolor{currentstroke}{rgb}{0.000000,0.000000,0.000000}%
\pgfsetstrokecolor{currentstroke}%
\pgfsetdash{}{0pt}%
\pgfsys@defobject{currentmarker}{\pgfqpoint{-0.048611in}{0.000000in}}{\pgfqpoint{0.000000in}{0.000000in}}{%
\pgfpathmoveto{\pgfqpoint{0.000000in}{0.000000in}}%
\pgfpathlineto{\pgfqpoint{-0.048611in}{0.000000in}}%
\pgfusepath{stroke,fill}%
}%
\begin{pgfscope}%
\pgfsys@transformshift{0.787074in}{1.429536in}%
\pgfsys@useobject{currentmarker}{}%
\end{pgfscope}%
\end{pgfscope}%
\begin{pgfscope}%
\definecolor{textcolor}{rgb}{0.000000,0.000000,0.000000}%
\pgfsetstrokecolor{textcolor}%
\pgfsetfillcolor{textcolor}%
\pgftext[x=0.412073in, y=1.381342in, left, base]{\color{textcolor}\sffamily\fontsize{10.000000}{12.000000}\selectfont \(\displaystyle {5000}\)}%
\end{pgfscope}%
\begin{pgfscope}%
\pgfsetbuttcap%
\pgfsetroundjoin%
\definecolor{currentfill}{rgb}{0.000000,0.000000,0.000000}%
\pgfsetfillcolor{currentfill}%
\pgfsetlinewidth{0.803000pt}%
\definecolor{currentstroke}{rgb}{0.000000,0.000000,0.000000}%
\pgfsetstrokecolor{currentstroke}%
\pgfsetdash{}{0pt}%
\pgfsys@defobject{currentmarker}{\pgfqpoint{-0.048611in}{0.000000in}}{\pgfqpoint{0.000000in}{0.000000in}}{%
\pgfpathmoveto{\pgfqpoint{0.000000in}{0.000000in}}%
\pgfpathlineto{\pgfqpoint{-0.048611in}{0.000000in}}%
\pgfusepath{stroke,fill}%
}%
\begin{pgfscope}%
\pgfsys@transformshift{0.787074in}{1.799407in}%
\pgfsys@useobject{currentmarker}{}%
\end{pgfscope}%
\end{pgfscope}%
\begin{pgfscope}%
\definecolor{textcolor}{rgb}{0.000000,0.000000,0.000000}%
\pgfsetstrokecolor{textcolor}%
\pgfsetfillcolor{textcolor}%
\pgftext[x=0.412073in, y=1.751213in, left, base]{\color{textcolor}\sffamily\fontsize{10.000000}{12.000000}\selectfont \(\displaystyle {7500}\)}%
\end{pgfscope}%
\begin{pgfscope}%
\pgfsetbuttcap%
\pgfsetroundjoin%
\definecolor{currentfill}{rgb}{0.000000,0.000000,0.000000}%
\pgfsetfillcolor{currentfill}%
\pgfsetlinewidth{0.803000pt}%
\definecolor{currentstroke}{rgb}{0.000000,0.000000,0.000000}%
\pgfsetstrokecolor{currentstroke}%
\pgfsetdash{}{0pt}%
\pgfsys@defobject{currentmarker}{\pgfqpoint{-0.048611in}{0.000000in}}{\pgfqpoint{0.000000in}{0.000000in}}{%
\pgfpathmoveto{\pgfqpoint{0.000000in}{0.000000in}}%
\pgfpathlineto{\pgfqpoint{-0.048611in}{0.000000in}}%
\pgfusepath{stroke,fill}%
}%
\begin{pgfscope}%
\pgfsys@transformshift{0.787074in}{2.169278in}%
\pgfsys@useobject{currentmarker}{}%
\end{pgfscope}%
\end{pgfscope}%
\begin{pgfscope}%
\definecolor{textcolor}{rgb}{0.000000,0.000000,0.000000}%
\pgfsetstrokecolor{textcolor}%
\pgfsetfillcolor{textcolor}%
\pgftext[x=0.342628in, y=2.121083in, left, base]{\color{textcolor}\sffamily\fontsize{10.000000}{12.000000}\selectfont \(\displaystyle {10000}\)}%
\end{pgfscope}%
\begin{pgfscope}%
\pgfsetbuttcap%
\pgfsetroundjoin%
\definecolor{currentfill}{rgb}{0.000000,0.000000,0.000000}%
\pgfsetfillcolor{currentfill}%
\pgfsetlinewidth{0.803000pt}%
\definecolor{currentstroke}{rgb}{0.000000,0.000000,0.000000}%
\pgfsetstrokecolor{currentstroke}%
\pgfsetdash{}{0pt}%
\pgfsys@defobject{currentmarker}{\pgfqpoint{-0.048611in}{0.000000in}}{\pgfqpoint{0.000000in}{0.000000in}}{%
\pgfpathmoveto{\pgfqpoint{0.000000in}{0.000000in}}%
\pgfpathlineto{\pgfqpoint{-0.048611in}{0.000000in}}%
\pgfusepath{stroke,fill}%
}%
\begin{pgfscope}%
\pgfsys@transformshift{0.787074in}{2.539148in}%
\pgfsys@useobject{currentmarker}{}%
\end{pgfscope}%
\end{pgfscope}%
\begin{pgfscope}%
\definecolor{textcolor}{rgb}{0.000000,0.000000,0.000000}%
\pgfsetstrokecolor{textcolor}%
\pgfsetfillcolor{textcolor}%
\pgftext[x=0.342628in, y=2.490954in, left, base]{\color{textcolor}\sffamily\fontsize{10.000000}{12.000000}\selectfont \(\displaystyle {12500}\)}%
\end{pgfscope}%
\begin{pgfscope}%
\pgfsetbuttcap%
\pgfsetroundjoin%
\definecolor{currentfill}{rgb}{0.000000,0.000000,0.000000}%
\pgfsetfillcolor{currentfill}%
\pgfsetlinewidth{0.803000pt}%
\definecolor{currentstroke}{rgb}{0.000000,0.000000,0.000000}%
\pgfsetstrokecolor{currentstroke}%
\pgfsetdash{}{0pt}%
\pgfsys@defobject{currentmarker}{\pgfqpoint{-0.048611in}{0.000000in}}{\pgfqpoint{0.000000in}{0.000000in}}{%
\pgfpathmoveto{\pgfqpoint{0.000000in}{0.000000in}}%
\pgfpathlineto{\pgfqpoint{-0.048611in}{0.000000in}}%
\pgfusepath{stroke,fill}%
}%
\begin{pgfscope}%
\pgfsys@transformshift{0.787074in}{2.909019in}%
\pgfsys@useobject{currentmarker}{}%
\end{pgfscope}%
\end{pgfscope}%
\begin{pgfscope}%
\definecolor{textcolor}{rgb}{0.000000,0.000000,0.000000}%
\pgfsetstrokecolor{textcolor}%
\pgfsetfillcolor{textcolor}%
\pgftext[x=0.342628in, y=2.860824in, left, base]{\color{textcolor}\sffamily\fontsize{10.000000}{12.000000}\selectfont \(\displaystyle {15000}\)}%
\end{pgfscope}%
\begin{pgfscope}%
\pgfsetbuttcap%
\pgfsetroundjoin%
\definecolor{currentfill}{rgb}{0.000000,0.000000,0.000000}%
\pgfsetfillcolor{currentfill}%
\pgfsetlinewidth{0.803000pt}%
\definecolor{currentstroke}{rgb}{0.000000,0.000000,0.000000}%
\pgfsetstrokecolor{currentstroke}%
\pgfsetdash{}{0pt}%
\pgfsys@defobject{currentmarker}{\pgfqpoint{-0.048611in}{0.000000in}}{\pgfqpoint{0.000000in}{0.000000in}}{%
\pgfpathmoveto{\pgfqpoint{0.000000in}{0.000000in}}%
\pgfpathlineto{\pgfqpoint{-0.048611in}{0.000000in}}%
\pgfusepath{stroke,fill}%
}%
\begin{pgfscope}%
\pgfsys@transformshift{0.787074in}{3.278889in}%
\pgfsys@useobject{currentmarker}{}%
\end{pgfscope}%
\end{pgfscope}%
\begin{pgfscope}%
\definecolor{textcolor}{rgb}{0.000000,0.000000,0.000000}%
\pgfsetstrokecolor{textcolor}%
\pgfsetfillcolor{textcolor}%
\pgftext[x=0.342628in, y=3.230695in, left, base]{\color{textcolor}\sffamily\fontsize{10.000000}{12.000000}\selectfont \(\displaystyle {17500}\)}%
\end{pgfscope}%
\begin{pgfscope}%
\pgfsetbuttcap%
\pgfsetroundjoin%
\definecolor{currentfill}{rgb}{0.000000,0.000000,0.000000}%
\pgfsetfillcolor{currentfill}%
\pgfsetlinewidth{0.803000pt}%
\definecolor{currentstroke}{rgb}{0.000000,0.000000,0.000000}%
\pgfsetstrokecolor{currentstroke}%
\pgfsetdash{}{0pt}%
\pgfsys@defobject{currentmarker}{\pgfqpoint{-0.048611in}{0.000000in}}{\pgfqpoint{0.000000in}{0.000000in}}{%
\pgfpathmoveto{\pgfqpoint{0.000000in}{0.000000in}}%
\pgfpathlineto{\pgfqpoint{-0.048611in}{0.000000in}}%
\pgfusepath{stroke,fill}%
}%
\begin{pgfscope}%
\pgfsys@transformshift{0.787074in}{3.648760in}%
\pgfsys@useobject{currentmarker}{}%
\end{pgfscope}%
\end{pgfscope}%
\begin{pgfscope}%
\definecolor{textcolor}{rgb}{0.000000,0.000000,0.000000}%
\pgfsetstrokecolor{textcolor}%
\pgfsetfillcolor{textcolor}%
\pgftext[x=0.342628in, y=3.600565in, left, base]{\color{textcolor}\sffamily\fontsize{10.000000}{12.000000}\selectfont \(\displaystyle {20000}\)}%
\end{pgfscope}%
\begin{pgfscope}%
\definecolor{textcolor}{rgb}{0.000000,0.000000,0.000000}%
\pgfsetstrokecolor{textcolor}%
\pgfsetfillcolor{textcolor}%
\pgftext[x=0.287073in,y=2.100064in,,bottom,rotate=90.000000]{\color{textcolor}\sffamily\fontsize{10.000000}{12.000000}\selectfont Maximum Memory Usage (MB)}%
\end{pgfscope}%
\begin{pgfscope}%
\pgfsetrectcap%
\pgfsetmiterjoin%
\pgfsetlinewidth{0.803000pt}%
\definecolor{currentstroke}{rgb}{0.000000,0.000000,0.000000}%
\pgfsetstrokecolor{currentstroke}%
\pgfsetdash{}{0pt}%
\pgfpathmoveto{\pgfqpoint{0.787074in}{0.548769in}}%
\pgfpathlineto{\pgfqpoint{0.787074in}{3.651359in}}%
\pgfusepath{stroke}%
\end{pgfscope}%
\begin{pgfscope}%
\pgfsetrectcap%
\pgfsetmiterjoin%
\pgfsetlinewidth{0.803000pt}%
\definecolor{currentstroke}{rgb}{0.000000,0.000000,0.000000}%
\pgfsetstrokecolor{currentstroke}%
\pgfsetdash{}{0pt}%
\pgfpathmoveto{\pgfqpoint{5.850000in}{0.548769in}}%
\pgfpathlineto{\pgfqpoint{5.850000in}{3.651359in}}%
\pgfusepath{stroke}%
\end{pgfscope}%
\begin{pgfscope}%
\pgfsetrectcap%
\pgfsetmiterjoin%
\pgfsetlinewidth{0.803000pt}%
\definecolor{currentstroke}{rgb}{0.000000,0.000000,0.000000}%
\pgfsetstrokecolor{currentstroke}%
\pgfsetdash{}{0pt}%
\pgfpathmoveto{\pgfqpoint{0.787074in}{0.548769in}}%
\pgfpathlineto{\pgfqpoint{5.850000in}{0.548769in}}%
\pgfusepath{stroke}%
\end{pgfscope}%
\begin{pgfscope}%
\pgfsetrectcap%
\pgfsetmiterjoin%
\pgfsetlinewidth{0.803000pt}%
\definecolor{currentstroke}{rgb}{0.000000,0.000000,0.000000}%
\pgfsetstrokecolor{currentstroke}%
\pgfsetdash{}{0pt}%
\pgfpathmoveto{\pgfqpoint{0.787074in}{3.651359in}}%
\pgfpathlineto{\pgfqpoint{5.850000in}{3.651359in}}%
\pgfusepath{stroke}%
\end{pgfscope}%
\begin{pgfscope}%
\definecolor{textcolor}{rgb}{0.000000,0.000000,0.000000}%
\pgfsetstrokecolor{textcolor}%
\pgfsetfillcolor{textcolor}%
\pgftext[x=3.318537in,y=3.734692in,,base]{\color{textcolor}\sffamily\fontsize{12.000000}{14.400000}\selectfont Forwards}%
\end{pgfscope}%
\begin{pgfscope}%
\pgfsetbuttcap%
\pgfsetmiterjoin%
\definecolor{currentfill}{rgb}{1.000000,1.000000,1.000000}%
\pgfsetfillcolor{currentfill}%
\pgfsetfillopacity{0.800000}%
\pgfsetlinewidth{1.003750pt}%
\definecolor{currentstroke}{rgb}{0.800000,0.800000,0.800000}%
\pgfsetstrokecolor{currentstroke}%
\pgfsetstrokeopacity{0.800000}%
\pgfsetdash{}{0pt}%
\pgfpathmoveto{\pgfqpoint{4.300417in}{2.957886in}}%
\pgfpathlineto{\pgfqpoint{5.752778in}{2.957886in}}%
\pgfpathquadraticcurveto{\pgfqpoint{5.780556in}{2.957886in}}{\pgfqpoint{5.780556in}{2.985664in}}%
\pgfpathlineto{\pgfqpoint{5.780556in}{3.554136in}}%
\pgfpathquadraticcurveto{\pgfqpoint{5.780556in}{3.581914in}}{\pgfqpoint{5.752778in}{3.581914in}}%
\pgfpathlineto{\pgfqpoint{4.300417in}{3.581914in}}%
\pgfpathquadraticcurveto{\pgfqpoint{4.272639in}{3.581914in}}{\pgfqpoint{4.272639in}{3.554136in}}%
\pgfpathlineto{\pgfqpoint{4.272639in}{2.985664in}}%
\pgfpathquadraticcurveto{\pgfqpoint{4.272639in}{2.957886in}}{\pgfqpoint{4.300417in}{2.957886in}}%
\pgfpathclose%
\pgfusepath{stroke,fill}%
\end{pgfscope}%
\begin{pgfscope}%
\pgfsetbuttcap%
\pgfsetroundjoin%
\definecolor{currentfill}{rgb}{0.121569,0.466667,0.705882}%
\pgfsetfillcolor{currentfill}%
\pgfsetlinewidth{1.003750pt}%
\definecolor{currentstroke}{rgb}{0.121569,0.466667,0.705882}%
\pgfsetstrokecolor{currentstroke}%
\pgfsetdash{}{0pt}%
\pgfsys@defobject{currentmarker}{\pgfqpoint{-0.034722in}{-0.034722in}}{\pgfqpoint{0.034722in}{0.034722in}}{%
\pgfpathmoveto{\pgfqpoint{0.000000in}{-0.034722in}}%
\pgfpathcurveto{\pgfqpoint{0.009208in}{-0.034722in}}{\pgfqpoint{0.018041in}{-0.031064in}}{\pgfqpoint{0.024552in}{-0.024552in}}%
\pgfpathcurveto{\pgfqpoint{0.031064in}{-0.018041in}}{\pgfqpoint{0.034722in}{-0.009208in}}{\pgfqpoint{0.034722in}{0.000000in}}%
\pgfpathcurveto{\pgfqpoint{0.034722in}{0.009208in}}{\pgfqpoint{0.031064in}{0.018041in}}{\pgfqpoint{0.024552in}{0.024552in}}%
\pgfpathcurveto{\pgfqpoint{0.018041in}{0.031064in}}{\pgfqpoint{0.009208in}{0.034722in}}{\pgfqpoint{0.000000in}{0.034722in}}%
\pgfpathcurveto{\pgfqpoint{-0.009208in}{0.034722in}}{\pgfqpoint{-0.018041in}{0.031064in}}{\pgfqpoint{-0.024552in}{0.024552in}}%
\pgfpathcurveto{\pgfqpoint{-0.031064in}{0.018041in}}{\pgfqpoint{-0.034722in}{0.009208in}}{\pgfqpoint{-0.034722in}{0.000000in}}%
\pgfpathcurveto{\pgfqpoint{-0.034722in}{-0.009208in}}{\pgfqpoint{-0.031064in}{-0.018041in}}{\pgfqpoint{-0.024552in}{-0.024552in}}%
\pgfpathcurveto{\pgfqpoint{-0.018041in}{-0.031064in}}{\pgfqpoint{-0.009208in}{-0.034722in}}{\pgfqpoint{0.000000in}{-0.034722in}}%
\pgfpathclose%
\pgfusepath{stroke,fill}%
}%
\begin{pgfscope}%
\pgfsys@transformshift{4.467083in}{3.477748in}%
\pgfsys@useobject{currentmarker}{}%
\end{pgfscope}%
\end{pgfscope}%
\begin{pgfscope}%
\definecolor{textcolor}{rgb}{0.000000,0.000000,0.000000}%
\pgfsetstrokecolor{textcolor}%
\pgfsetfillcolor{textcolor}%
\pgftext[x=4.717083in,y=3.429136in,left,base]{\color{textcolor}\sffamily\fontsize{10.000000}{12.000000}\selectfont No Timeout}%
\end{pgfscope}%
\begin{pgfscope}%
\pgfsetbuttcap%
\pgfsetroundjoin%
\definecolor{currentfill}{rgb}{1.000000,0.498039,0.054902}%
\pgfsetfillcolor{currentfill}%
\pgfsetlinewidth{1.003750pt}%
\definecolor{currentstroke}{rgb}{1.000000,0.498039,0.054902}%
\pgfsetstrokecolor{currentstroke}%
\pgfsetdash{}{0pt}%
\pgfsys@defobject{currentmarker}{\pgfqpoint{-0.034722in}{-0.034722in}}{\pgfqpoint{0.034722in}{0.034722in}}{%
\pgfpathmoveto{\pgfqpoint{0.000000in}{-0.034722in}}%
\pgfpathcurveto{\pgfqpoint{0.009208in}{-0.034722in}}{\pgfqpoint{0.018041in}{-0.031064in}}{\pgfqpoint{0.024552in}{-0.024552in}}%
\pgfpathcurveto{\pgfqpoint{0.031064in}{-0.018041in}}{\pgfqpoint{0.034722in}{-0.009208in}}{\pgfqpoint{0.034722in}{0.000000in}}%
\pgfpathcurveto{\pgfqpoint{0.034722in}{0.009208in}}{\pgfqpoint{0.031064in}{0.018041in}}{\pgfqpoint{0.024552in}{0.024552in}}%
\pgfpathcurveto{\pgfqpoint{0.018041in}{0.031064in}}{\pgfqpoint{0.009208in}{0.034722in}}{\pgfqpoint{0.000000in}{0.034722in}}%
\pgfpathcurveto{\pgfqpoint{-0.009208in}{0.034722in}}{\pgfqpoint{-0.018041in}{0.031064in}}{\pgfqpoint{-0.024552in}{0.024552in}}%
\pgfpathcurveto{\pgfqpoint{-0.031064in}{0.018041in}}{\pgfqpoint{-0.034722in}{0.009208in}}{\pgfqpoint{-0.034722in}{0.000000in}}%
\pgfpathcurveto{\pgfqpoint{-0.034722in}{-0.009208in}}{\pgfqpoint{-0.031064in}{-0.018041in}}{\pgfqpoint{-0.024552in}{-0.024552in}}%
\pgfpathcurveto{\pgfqpoint{-0.018041in}{-0.031064in}}{\pgfqpoint{-0.009208in}{-0.034722in}}{\pgfqpoint{0.000000in}{-0.034722in}}%
\pgfpathclose%
\pgfusepath{stroke,fill}%
}%
\begin{pgfscope}%
\pgfsys@transformshift{4.467083in}{3.284136in}%
\pgfsys@useobject{currentmarker}{}%
\end{pgfscope}%
\end{pgfscope}%
\begin{pgfscope}%
\definecolor{textcolor}{rgb}{0.000000,0.000000,0.000000}%
\pgfsetstrokecolor{textcolor}%
\pgfsetfillcolor{textcolor}%
\pgftext[x=4.717083in,y=3.235525in,left,base]{\color{textcolor}\sffamily\fontsize{10.000000}{12.000000}\selectfont Time Timeout}%
\end{pgfscope}%
\begin{pgfscope}%
\pgfsetbuttcap%
\pgfsetroundjoin%
\definecolor{currentfill}{rgb}{0.839216,0.152941,0.156863}%
\pgfsetfillcolor{currentfill}%
\pgfsetlinewidth{1.003750pt}%
\definecolor{currentstroke}{rgb}{0.839216,0.152941,0.156863}%
\pgfsetstrokecolor{currentstroke}%
\pgfsetdash{}{0pt}%
\pgfsys@defobject{currentmarker}{\pgfqpoint{-0.034722in}{-0.034722in}}{\pgfqpoint{0.034722in}{0.034722in}}{%
\pgfpathmoveto{\pgfqpoint{0.000000in}{-0.034722in}}%
\pgfpathcurveto{\pgfqpoint{0.009208in}{-0.034722in}}{\pgfqpoint{0.018041in}{-0.031064in}}{\pgfqpoint{0.024552in}{-0.024552in}}%
\pgfpathcurveto{\pgfqpoint{0.031064in}{-0.018041in}}{\pgfqpoint{0.034722in}{-0.009208in}}{\pgfqpoint{0.034722in}{0.000000in}}%
\pgfpathcurveto{\pgfqpoint{0.034722in}{0.009208in}}{\pgfqpoint{0.031064in}{0.018041in}}{\pgfqpoint{0.024552in}{0.024552in}}%
\pgfpathcurveto{\pgfqpoint{0.018041in}{0.031064in}}{\pgfqpoint{0.009208in}{0.034722in}}{\pgfqpoint{0.000000in}{0.034722in}}%
\pgfpathcurveto{\pgfqpoint{-0.009208in}{0.034722in}}{\pgfqpoint{-0.018041in}{0.031064in}}{\pgfqpoint{-0.024552in}{0.024552in}}%
\pgfpathcurveto{\pgfqpoint{-0.031064in}{0.018041in}}{\pgfqpoint{-0.034722in}{0.009208in}}{\pgfqpoint{-0.034722in}{0.000000in}}%
\pgfpathcurveto{\pgfqpoint{-0.034722in}{-0.009208in}}{\pgfqpoint{-0.031064in}{-0.018041in}}{\pgfqpoint{-0.024552in}{-0.024552in}}%
\pgfpathcurveto{\pgfqpoint{-0.018041in}{-0.031064in}}{\pgfqpoint{-0.009208in}{-0.034722in}}{\pgfqpoint{0.000000in}{-0.034722in}}%
\pgfpathclose%
\pgfusepath{stroke,fill}%
}%
\begin{pgfscope}%
\pgfsys@transformshift{4.467083in}{3.090525in}%
\pgfsys@useobject{currentmarker}{}%
\end{pgfscope}%
\end{pgfscope}%
\begin{pgfscope}%
\definecolor{textcolor}{rgb}{0.000000,0.000000,0.000000}%
\pgfsetstrokecolor{textcolor}%
\pgfsetfillcolor{textcolor}%
\pgftext[x=4.717083in,y=3.041914in,left,base]{\color{textcolor}\sffamily\fontsize{10.000000}{12.000000}\selectfont Memory Timeout}%
\end{pgfscope}%
\end{pgfpicture}%
\makeatother%
\endgroup%

                }
            \end{subfigure}
            \qquad
            \begin{subfigure}[]{0.45\textwidth}
                \centering
                \resizebox{\columnwidth}{!}{
                    %% Creator: Matplotlib, PGF backend
%%
%% To include the figure in your LaTeX document, write
%%   \input{<filename>.pgf}
%%
%% Make sure the required packages are loaded in your preamble
%%   \usepackage{pgf}
%%
%% and, on pdftex
%%   \usepackage[utf8]{inputenc}\DeclareUnicodeCharacter{2212}{-}
%%
%% or, on luatex and xetex
%%   \usepackage{unicode-math}
%%
%% Figures using additional raster images can only be included by \input if
%% they are in the same directory as the main LaTeX file. For loading figures
%% from other directories you can use the `import` package
%%   \usepackage{import}
%%
%% and then include the figures with
%%   \import{<path to file>}{<filename>.pgf}
%%
%% Matplotlib used the following preamble
%%   \usepackage{amsmath}
%%   \usepackage{fontspec}
%%
\begingroup%
\makeatletter%
\begin{pgfpicture}%
\pgfpathrectangle{\pgfpointorigin}{\pgfqpoint{6.000000in}{4.000000in}}%
\pgfusepath{use as bounding box, clip}%
\begin{pgfscope}%
\pgfsetbuttcap%
\pgfsetmiterjoin%
\definecolor{currentfill}{rgb}{1.000000,1.000000,1.000000}%
\pgfsetfillcolor{currentfill}%
\pgfsetlinewidth{0.000000pt}%
\definecolor{currentstroke}{rgb}{1.000000,1.000000,1.000000}%
\pgfsetstrokecolor{currentstroke}%
\pgfsetdash{}{0pt}%
\pgfpathmoveto{\pgfqpoint{0.000000in}{0.000000in}}%
\pgfpathlineto{\pgfqpoint{6.000000in}{0.000000in}}%
\pgfpathlineto{\pgfqpoint{6.000000in}{4.000000in}}%
\pgfpathlineto{\pgfqpoint{0.000000in}{4.000000in}}%
\pgfpathclose%
\pgfusepath{fill}%
\end{pgfscope}%
\begin{pgfscope}%
\pgfsetbuttcap%
\pgfsetmiterjoin%
\definecolor{currentfill}{rgb}{1.000000,1.000000,1.000000}%
\pgfsetfillcolor{currentfill}%
\pgfsetlinewidth{0.000000pt}%
\definecolor{currentstroke}{rgb}{0.000000,0.000000,0.000000}%
\pgfsetstrokecolor{currentstroke}%
\pgfsetstrokeopacity{0.000000}%
\pgfsetdash{}{0pt}%
\pgfpathmoveto{\pgfqpoint{0.787074in}{0.548769in}}%
\pgfpathlineto{\pgfqpoint{5.850000in}{0.548769in}}%
\pgfpathlineto{\pgfqpoint{5.850000in}{3.651359in}}%
\pgfpathlineto{\pgfqpoint{0.787074in}{3.651359in}}%
\pgfpathclose%
\pgfusepath{fill}%
\end{pgfscope}%
\begin{pgfscope}%
\pgfpathrectangle{\pgfqpoint{0.787074in}{0.548769in}}{\pgfqpoint{5.062926in}{3.102590in}}%
\pgfusepath{clip}%
\pgfsetbuttcap%
\pgfsetroundjoin%
\definecolor{currentfill}{rgb}{0.121569,0.466667,0.705882}%
\pgfsetfillcolor{currentfill}%
\pgfsetlinewidth{1.003750pt}%
\definecolor{currentstroke}{rgb}{0.121569,0.466667,0.705882}%
\pgfsetstrokecolor{currentstroke}%
\pgfsetdash{}{0pt}%
\pgfpathmoveto{\pgfqpoint{1.239778in}{0.676616in}}%
\pgfpathcurveto{\pgfqpoint{1.250828in}{0.676616in}}{\pgfqpoint{1.261427in}{0.681006in}}{\pgfqpoint{1.269241in}{0.688820in}}%
\pgfpathcurveto{\pgfqpoint{1.277054in}{0.696633in}}{\pgfqpoint{1.281445in}{0.707232in}}{\pgfqpoint{1.281445in}{0.718283in}}%
\pgfpathcurveto{\pgfqpoint{1.281445in}{0.729333in}}{\pgfqpoint{1.277054in}{0.739932in}}{\pgfqpoint{1.269241in}{0.747745in}}%
\pgfpathcurveto{\pgfqpoint{1.261427in}{0.755559in}}{\pgfqpoint{1.250828in}{0.759949in}}{\pgfqpoint{1.239778in}{0.759949in}}%
\pgfpathcurveto{\pgfqpoint{1.228728in}{0.759949in}}{\pgfqpoint{1.218129in}{0.755559in}}{\pgfqpoint{1.210315in}{0.747745in}}%
\pgfpathcurveto{\pgfqpoint{1.202502in}{0.739932in}}{\pgfqpoint{1.198111in}{0.729333in}}{\pgfqpoint{1.198111in}{0.718283in}}%
\pgfpathcurveto{\pgfqpoint{1.198111in}{0.707232in}}{\pgfqpoint{1.202502in}{0.696633in}}{\pgfqpoint{1.210315in}{0.688820in}}%
\pgfpathcurveto{\pgfqpoint{1.218129in}{0.681006in}}{\pgfqpoint{1.228728in}{0.676616in}}{\pgfqpoint{1.239778in}{0.676616in}}%
\pgfpathclose%
\pgfusepath{stroke,fill}%
\end{pgfscope}%
\begin{pgfscope}%
\pgfpathrectangle{\pgfqpoint{0.787074in}{0.548769in}}{\pgfqpoint{5.062926in}{3.102590in}}%
\pgfusepath{clip}%
\pgfsetbuttcap%
\pgfsetroundjoin%
\definecolor{currentfill}{rgb}{0.121569,0.466667,0.705882}%
\pgfsetfillcolor{currentfill}%
\pgfsetlinewidth{1.003750pt}%
\definecolor{currentstroke}{rgb}{0.121569,0.466667,0.705882}%
\pgfsetstrokecolor{currentstroke}%
\pgfsetdash{}{0pt}%
\pgfpathmoveto{\pgfqpoint{2.900535in}{2.052247in}}%
\pgfpathcurveto{\pgfqpoint{2.911585in}{2.052247in}}{\pgfqpoint{2.922184in}{2.056638in}}{\pgfqpoint{2.929998in}{2.064451in}}%
\pgfpathcurveto{\pgfqpoint{2.937812in}{2.072265in}}{\pgfqpoint{2.942202in}{2.082864in}}{\pgfqpoint{2.942202in}{2.093914in}}%
\pgfpathcurveto{\pgfqpoint{2.942202in}{2.104964in}}{\pgfqpoint{2.937812in}{2.115563in}}{\pgfqpoint{2.929998in}{2.123377in}}%
\pgfpathcurveto{\pgfqpoint{2.922184in}{2.131190in}}{\pgfqpoint{2.911585in}{2.135581in}}{\pgfqpoint{2.900535in}{2.135581in}}%
\pgfpathcurveto{\pgfqpoint{2.889485in}{2.135581in}}{\pgfqpoint{2.878886in}{2.131190in}}{\pgfqpoint{2.871072in}{2.123377in}}%
\pgfpathcurveto{\pgfqpoint{2.863259in}{2.115563in}}{\pgfqpoint{2.858868in}{2.104964in}}{\pgfqpoint{2.858868in}{2.093914in}}%
\pgfpathcurveto{\pgfqpoint{2.858868in}{2.082864in}}{\pgfqpoint{2.863259in}{2.072265in}}{\pgfqpoint{2.871072in}{2.064451in}}%
\pgfpathcurveto{\pgfqpoint{2.878886in}{2.056638in}}{\pgfqpoint{2.889485in}{2.052247in}}{\pgfqpoint{2.900535in}{2.052247in}}%
\pgfpathclose%
\pgfusepath{stroke,fill}%
\end{pgfscope}%
\begin{pgfscope}%
\pgfpathrectangle{\pgfqpoint{0.787074in}{0.548769in}}{\pgfqpoint{5.062926in}{3.102590in}}%
\pgfusepath{clip}%
\pgfsetbuttcap%
\pgfsetroundjoin%
\definecolor{currentfill}{rgb}{1.000000,0.498039,0.054902}%
\pgfsetfillcolor{currentfill}%
\pgfsetlinewidth{1.003750pt}%
\definecolor{currentstroke}{rgb}{1.000000,0.498039,0.054902}%
\pgfsetstrokecolor{currentstroke}%
\pgfsetdash{}{0pt}%
\pgfpathmoveto{\pgfqpoint{1.388404in}{2.417739in}}%
\pgfpathcurveto{\pgfqpoint{1.399454in}{2.417739in}}{\pgfqpoint{1.410053in}{2.422129in}}{\pgfqpoint{1.417867in}{2.429943in}}%
\pgfpathcurveto{\pgfqpoint{1.425680in}{2.437757in}}{\pgfqpoint{1.430071in}{2.448356in}}{\pgfqpoint{1.430071in}{2.459406in}}%
\pgfpathcurveto{\pgfqpoint{1.430071in}{2.470456in}}{\pgfqpoint{1.425680in}{2.481055in}}{\pgfqpoint{1.417867in}{2.488869in}}%
\pgfpathcurveto{\pgfqpoint{1.410053in}{2.496682in}}{\pgfqpoint{1.399454in}{2.501072in}}{\pgfqpoint{1.388404in}{2.501072in}}%
\pgfpathcurveto{\pgfqpoint{1.377354in}{2.501072in}}{\pgfqpoint{1.366755in}{2.496682in}}{\pgfqpoint{1.358941in}{2.488869in}}%
\pgfpathcurveto{\pgfqpoint{1.351128in}{2.481055in}}{\pgfqpoint{1.346737in}{2.470456in}}{\pgfqpoint{1.346737in}{2.459406in}}%
\pgfpathcurveto{\pgfqpoint{1.346737in}{2.448356in}}{\pgfqpoint{1.351128in}{2.437757in}}{\pgfqpoint{1.358941in}{2.429943in}}%
\pgfpathcurveto{\pgfqpoint{1.366755in}{2.422129in}}{\pgfqpoint{1.377354in}{2.417739in}}{\pgfqpoint{1.388404in}{2.417739in}}%
\pgfpathclose%
\pgfusepath{stroke,fill}%
\end{pgfscope}%
\begin{pgfscope}%
\pgfpathrectangle{\pgfqpoint{0.787074in}{0.548769in}}{\pgfqpoint{5.062926in}{3.102590in}}%
\pgfusepath{clip}%
\pgfsetbuttcap%
\pgfsetroundjoin%
\definecolor{currentfill}{rgb}{1.000000,0.498039,0.054902}%
\pgfsetfillcolor{currentfill}%
\pgfsetlinewidth{1.003750pt}%
\definecolor{currentstroke}{rgb}{1.000000,0.498039,0.054902}%
\pgfsetstrokecolor{currentstroke}%
\pgfsetdash{}{0pt}%
\pgfpathmoveto{\pgfqpoint{1.638133in}{2.242926in}}%
\pgfpathcurveto{\pgfqpoint{1.649183in}{2.242926in}}{\pgfqpoint{1.659782in}{2.247317in}}{\pgfqpoint{1.667596in}{2.255130in}}%
\pgfpathcurveto{\pgfqpoint{1.675409in}{2.262944in}}{\pgfqpoint{1.679799in}{2.273543in}}{\pgfqpoint{1.679799in}{2.284593in}}%
\pgfpathcurveto{\pgfqpoint{1.679799in}{2.295643in}}{\pgfqpoint{1.675409in}{2.306242in}}{\pgfqpoint{1.667596in}{2.314056in}}%
\pgfpathcurveto{\pgfqpoint{1.659782in}{2.321869in}}{\pgfqpoint{1.649183in}{2.326260in}}{\pgfqpoint{1.638133in}{2.326260in}}%
\pgfpathcurveto{\pgfqpoint{1.627083in}{2.326260in}}{\pgfqpoint{1.616484in}{2.321869in}}{\pgfqpoint{1.608670in}{2.314056in}}%
\pgfpathcurveto{\pgfqpoint{1.600856in}{2.306242in}}{\pgfqpoint{1.596466in}{2.295643in}}{\pgfqpoint{1.596466in}{2.284593in}}%
\pgfpathcurveto{\pgfqpoint{1.596466in}{2.273543in}}{\pgfqpoint{1.600856in}{2.262944in}}{\pgfqpoint{1.608670in}{2.255130in}}%
\pgfpathcurveto{\pgfqpoint{1.616484in}{2.247317in}}{\pgfqpoint{1.627083in}{2.242926in}}{\pgfqpoint{1.638133in}{2.242926in}}%
\pgfpathclose%
\pgfusepath{stroke,fill}%
\end{pgfscope}%
\begin{pgfscope}%
\pgfpathrectangle{\pgfqpoint{0.787074in}{0.548769in}}{\pgfqpoint{5.062926in}{3.102590in}}%
\pgfusepath{clip}%
\pgfsetbuttcap%
\pgfsetroundjoin%
\definecolor{currentfill}{rgb}{0.121569,0.466667,0.705882}%
\pgfsetfillcolor{currentfill}%
\pgfsetlinewidth{1.003750pt}%
\definecolor{currentstroke}{rgb}{0.121569,0.466667,0.705882}%
\pgfsetstrokecolor{currentstroke}%
\pgfsetdash{}{0pt}%
\pgfpathmoveto{\pgfqpoint{1.474701in}{0.650065in}}%
\pgfpathcurveto{\pgfqpoint{1.485751in}{0.650065in}}{\pgfqpoint{1.496350in}{0.654455in}}{\pgfqpoint{1.504164in}{0.662269in}}%
\pgfpathcurveto{\pgfqpoint{1.511977in}{0.670082in}}{\pgfqpoint{1.516368in}{0.680681in}}{\pgfqpoint{1.516368in}{0.691731in}}%
\pgfpathcurveto{\pgfqpoint{1.516368in}{0.702781in}}{\pgfqpoint{1.511977in}{0.713380in}}{\pgfqpoint{1.504164in}{0.721194in}}%
\pgfpathcurveto{\pgfqpoint{1.496350in}{0.729008in}}{\pgfqpoint{1.485751in}{0.733398in}}{\pgfqpoint{1.474701in}{0.733398in}}%
\pgfpathcurveto{\pgfqpoint{1.463651in}{0.733398in}}{\pgfqpoint{1.453052in}{0.729008in}}{\pgfqpoint{1.445238in}{0.721194in}}%
\pgfpathcurveto{\pgfqpoint{1.437424in}{0.713380in}}{\pgfqpoint{1.433034in}{0.702781in}}{\pgfqpoint{1.433034in}{0.691731in}}%
\pgfpathcurveto{\pgfqpoint{1.433034in}{0.680681in}}{\pgfqpoint{1.437424in}{0.670082in}}{\pgfqpoint{1.445238in}{0.662269in}}%
\pgfpathcurveto{\pgfqpoint{1.453052in}{0.654455in}}{\pgfqpoint{1.463651in}{0.650065in}}{\pgfqpoint{1.474701in}{0.650065in}}%
\pgfpathclose%
\pgfusepath{stroke,fill}%
\end{pgfscope}%
\begin{pgfscope}%
\pgfpathrectangle{\pgfqpoint{0.787074in}{0.548769in}}{\pgfqpoint{5.062926in}{3.102590in}}%
\pgfusepath{clip}%
\pgfsetbuttcap%
\pgfsetroundjoin%
\definecolor{currentfill}{rgb}{0.121569,0.466667,0.705882}%
\pgfsetfillcolor{currentfill}%
\pgfsetlinewidth{1.003750pt}%
\definecolor{currentstroke}{rgb}{0.121569,0.466667,0.705882}%
\pgfsetstrokecolor{currentstroke}%
\pgfsetdash{}{0pt}%
\pgfpathmoveto{\pgfqpoint{1.788011in}{2.492055in}}%
\pgfpathcurveto{\pgfqpoint{1.799061in}{2.492055in}}{\pgfqpoint{1.809660in}{2.496445in}}{\pgfqpoint{1.817474in}{2.504258in}}%
\pgfpathcurveto{\pgfqpoint{1.825287in}{2.512072in}}{\pgfqpoint{1.829678in}{2.522671in}}{\pgfqpoint{1.829678in}{2.533721in}}%
\pgfpathcurveto{\pgfqpoint{1.829678in}{2.544771in}}{\pgfqpoint{1.825287in}{2.555370in}}{\pgfqpoint{1.817474in}{2.563184in}}%
\pgfpathcurveto{\pgfqpoint{1.809660in}{2.570998in}}{\pgfqpoint{1.799061in}{2.575388in}}{\pgfqpoint{1.788011in}{2.575388in}}%
\pgfpathcurveto{\pgfqpoint{1.776961in}{2.575388in}}{\pgfqpoint{1.766362in}{2.570998in}}{\pgfqpoint{1.758548in}{2.563184in}}%
\pgfpathcurveto{\pgfqpoint{1.750735in}{2.555370in}}{\pgfqpoint{1.746344in}{2.544771in}}{\pgfqpoint{1.746344in}{2.533721in}}%
\pgfpathcurveto{\pgfqpoint{1.746344in}{2.522671in}}{\pgfqpoint{1.750735in}{2.512072in}}{\pgfqpoint{1.758548in}{2.504258in}}%
\pgfpathcurveto{\pgfqpoint{1.766362in}{2.496445in}}{\pgfqpoint{1.776961in}{2.492055in}}{\pgfqpoint{1.788011in}{2.492055in}}%
\pgfpathclose%
\pgfusepath{stroke,fill}%
\end{pgfscope}%
\begin{pgfscope}%
\pgfpathrectangle{\pgfqpoint{0.787074in}{0.548769in}}{\pgfqpoint{5.062926in}{3.102590in}}%
\pgfusepath{clip}%
\pgfsetbuttcap%
\pgfsetroundjoin%
\definecolor{currentfill}{rgb}{1.000000,0.498039,0.054902}%
\pgfsetfillcolor{currentfill}%
\pgfsetlinewidth{1.003750pt}%
\definecolor{currentstroke}{rgb}{1.000000,0.498039,0.054902}%
\pgfsetstrokecolor{currentstroke}%
\pgfsetdash{}{0pt}%
\pgfpathmoveto{\pgfqpoint{2.894286in}{2.495853in}}%
\pgfpathcurveto{\pgfqpoint{2.905336in}{2.495853in}}{\pgfqpoint{2.915935in}{2.500244in}}{\pgfqpoint{2.923749in}{2.508057in}}%
\pgfpathcurveto{\pgfqpoint{2.931562in}{2.515871in}}{\pgfqpoint{2.935953in}{2.526470in}}{\pgfqpoint{2.935953in}{2.537520in}}%
\pgfpathcurveto{\pgfqpoint{2.935953in}{2.548570in}}{\pgfqpoint{2.931562in}{2.559169in}}{\pgfqpoint{2.923749in}{2.566983in}}%
\pgfpathcurveto{\pgfqpoint{2.915935in}{2.574797in}}{\pgfqpoint{2.905336in}{2.579187in}}{\pgfqpoint{2.894286in}{2.579187in}}%
\pgfpathcurveto{\pgfqpoint{2.883236in}{2.579187in}}{\pgfqpoint{2.872637in}{2.574797in}}{\pgfqpoint{2.864823in}{2.566983in}}%
\pgfpathcurveto{\pgfqpoint{2.857010in}{2.559169in}}{\pgfqpoint{2.852619in}{2.548570in}}{\pgfqpoint{2.852619in}{2.537520in}}%
\pgfpathcurveto{\pgfqpoint{2.852619in}{2.526470in}}{\pgfqpoint{2.857010in}{2.515871in}}{\pgfqpoint{2.864823in}{2.508057in}}%
\pgfpathcurveto{\pgfqpoint{2.872637in}{2.500244in}}{\pgfqpoint{2.883236in}{2.495853in}}{\pgfqpoint{2.894286in}{2.495853in}}%
\pgfpathclose%
\pgfusepath{stroke,fill}%
\end{pgfscope}%
\begin{pgfscope}%
\pgfpathrectangle{\pgfqpoint{0.787074in}{0.548769in}}{\pgfqpoint{5.062926in}{3.102590in}}%
\pgfusepath{clip}%
\pgfsetbuttcap%
\pgfsetroundjoin%
\definecolor{currentfill}{rgb}{0.121569,0.466667,0.705882}%
\pgfsetfillcolor{currentfill}%
\pgfsetlinewidth{1.003750pt}%
\definecolor{currentstroke}{rgb}{0.121569,0.466667,0.705882}%
\pgfsetstrokecolor{currentstroke}%
\pgfsetdash{}{0pt}%
\pgfpathmoveto{\pgfqpoint{1.591184in}{0.648161in}}%
\pgfpathcurveto{\pgfqpoint{1.602234in}{0.648161in}}{\pgfqpoint{1.612833in}{0.652552in}}{\pgfqpoint{1.620646in}{0.660365in}}%
\pgfpathcurveto{\pgfqpoint{1.628460in}{0.668179in}}{\pgfqpoint{1.632850in}{0.678778in}}{\pgfqpoint{1.632850in}{0.689828in}}%
\pgfpathcurveto{\pgfqpoint{1.632850in}{0.700878in}}{\pgfqpoint{1.628460in}{0.711477in}}{\pgfqpoint{1.620646in}{0.719291in}}%
\pgfpathcurveto{\pgfqpoint{1.612833in}{0.727104in}}{\pgfqpoint{1.602234in}{0.731495in}}{\pgfqpoint{1.591184in}{0.731495in}}%
\pgfpathcurveto{\pgfqpoint{1.580134in}{0.731495in}}{\pgfqpoint{1.569535in}{0.727104in}}{\pgfqpoint{1.561721in}{0.719291in}}%
\pgfpathcurveto{\pgfqpoint{1.553907in}{0.711477in}}{\pgfqpoint{1.549517in}{0.700878in}}{\pgfqpoint{1.549517in}{0.689828in}}%
\pgfpathcurveto{\pgfqpoint{1.549517in}{0.678778in}}{\pgfqpoint{1.553907in}{0.668179in}}{\pgfqpoint{1.561721in}{0.660365in}}%
\pgfpathcurveto{\pgfqpoint{1.569535in}{0.652552in}}{\pgfqpoint{1.580134in}{0.648161in}}{\pgfqpoint{1.591184in}{0.648161in}}%
\pgfpathclose%
\pgfusepath{stroke,fill}%
\end{pgfscope}%
\begin{pgfscope}%
\pgfpathrectangle{\pgfqpoint{0.787074in}{0.548769in}}{\pgfqpoint{5.062926in}{3.102590in}}%
\pgfusepath{clip}%
\pgfsetbuttcap%
\pgfsetroundjoin%
\definecolor{currentfill}{rgb}{0.121569,0.466667,0.705882}%
\pgfsetfillcolor{currentfill}%
\pgfsetlinewidth{1.003750pt}%
\definecolor{currentstroke}{rgb}{0.121569,0.466667,0.705882}%
\pgfsetstrokecolor{currentstroke}%
\pgfsetdash{}{0pt}%
\pgfpathmoveto{\pgfqpoint{1.801199in}{0.839743in}}%
\pgfpathcurveto{\pgfqpoint{1.812249in}{0.839743in}}{\pgfqpoint{1.822848in}{0.844134in}}{\pgfqpoint{1.830662in}{0.851947in}}%
\pgfpathcurveto{\pgfqpoint{1.838475in}{0.859761in}}{\pgfqpoint{1.842865in}{0.870360in}}{\pgfqpoint{1.842865in}{0.881410in}}%
\pgfpathcurveto{\pgfqpoint{1.842865in}{0.892460in}}{\pgfqpoint{1.838475in}{0.903059in}}{\pgfqpoint{1.830662in}{0.910873in}}%
\pgfpathcurveto{\pgfqpoint{1.822848in}{0.918687in}}{\pgfqpoint{1.812249in}{0.923077in}}{\pgfqpoint{1.801199in}{0.923077in}}%
\pgfpathcurveto{\pgfqpoint{1.790149in}{0.923077in}}{\pgfqpoint{1.779550in}{0.918687in}}{\pgfqpoint{1.771736in}{0.910873in}}%
\pgfpathcurveto{\pgfqpoint{1.763922in}{0.903059in}}{\pgfqpoint{1.759532in}{0.892460in}}{\pgfqpoint{1.759532in}{0.881410in}}%
\pgfpathcurveto{\pgfqpoint{1.759532in}{0.870360in}}{\pgfqpoint{1.763922in}{0.859761in}}{\pgfqpoint{1.771736in}{0.851947in}}%
\pgfpathcurveto{\pgfqpoint{1.779550in}{0.844134in}}{\pgfqpoint{1.790149in}{0.839743in}}{\pgfqpoint{1.801199in}{0.839743in}}%
\pgfpathclose%
\pgfusepath{stroke,fill}%
\end{pgfscope}%
\begin{pgfscope}%
\pgfpathrectangle{\pgfqpoint{0.787074in}{0.548769in}}{\pgfqpoint{5.062926in}{3.102590in}}%
\pgfusepath{clip}%
\pgfsetbuttcap%
\pgfsetroundjoin%
\definecolor{currentfill}{rgb}{0.121569,0.466667,0.705882}%
\pgfsetfillcolor{currentfill}%
\pgfsetlinewidth{1.003750pt}%
\definecolor{currentstroke}{rgb}{0.121569,0.466667,0.705882}%
\pgfsetstrokecolor{currentstroke}%
\pgfsetdash{}{0pt}%
\pgfpathmoveto{\pgfqpoint{1.061082in}{0.786120in}}%
\pgfpathcurveto{\pgfqpoint{1.072132in}{0.786120in}}{\pgfqpoint{1.082731in}{0.790510in}}{\pgfqpoint{1.090544in}{0.798324in}}%
\pgfpathcurveto{\pgfqpoint{1.098358in}{0.806137in}}{\pgfqpoint{1.102748in}{0.816736in}}{\pgfqpoint{1.102748in}{0.827787in}}%
\pgfpathcurveto{\pgfqpoint{1.102748in}{0.838837in}}{\pgfqpoint{1.098358in}{0.849436in}}{\pgfqpoint{1.090544in}{0.857249in}}%
\pgfpathcurveto{\pgfqpoint{1.082731in}{0.865063in}}{\pgfqpoint{1.072132in}{0.869453in}}{\pgfqpoint{1.061082in}{0.869453in}}%
\pgfpathcurveto{\pgfqpoint{1.050031in}{0.869453in}}{\pgfqpoint{1.039432in}{0.865063in}}{\pgfqpoint{1.031619in}{0.857249in}}%
\pgfpathcurveto{\pgfqpoint{1.023805in}{0.849436in}}{\pgfqpoint{1.019415in}{0.838837in}}{\pgfqpoint{1.019415in}{0.827787in}}%
\pgfpathcurveto{\pgfqpoint{1.019415in}{0.816736in}}{\pgfqpoint{1.023805in}{0.806137in}}{\pgfqpoint{1.031619in}{0.798324in}}%
\pgfpathcurveto{\pgfqpoint{1.039432in}{0.790510in}}{\pgfqpoint{1.050031in}{0.786120in}}{\pgfqpoint{1.061082in}{0.786120in}}%
\pgfpathclose%
\pgfusepath{stroke,fill}%
\end{pgfscope}%
\begin{pgfscope}%
\pgfpathrectangle{\pgfqpoint{0.787074in}{0.548769in}}{\pgfqpoint{5.062926in}{3.102590in}}%
\pgfusepath{clip}%
\pgfsetbuttcap%
\pgfsetroundjoin%
\definecolor{currentfill}{rgb}{0.121569,0.466667,0.705882}%
\pgfsetfillcolor{currentfill}%
\pgfsetlinewidth{1.003750pt}%
\definecolor{currentstroke}{rgb}{0.121569,0.466667,0.705882}%
\pgfsetstrokecolor{currentstroke}%
\pgfsetdash{}{0pt}%
\pgfpathmoveto{\pgfqpoint{1.362136in}{0.660513in}}%
\pgfpathcurveto{\pgfqpoint{1.373186in}{0.660513in}}{\pgfqpoint{1.383785in}{0.664903in}}{\pgfqpoint{1.391599in}{0.672717in}}%
\pgfpathcurveto{\pgfqpoint{1.399413in}{0.680531in}}{\pgfqpoint{1.403803in}{0.691130in}}{\pgfqpoint{1.403803in}{0.702180in}}%
\pgfpathcurveto{\pgfqpoint{1.403803in}{0.713230in}}{\pgfqpoint{1.399413in}{0.723829in}}{\pgfqpoint{1.391599in}{0.731643in}}%
\pgfpathcurveto{\pgfqpoint{1.383785in}{0.739456in}}{\pgfqpoint{1.373186in}{0.743847in}}{\pgfqpoint{1.362136in}{0.743847in}}%
\pgfpathcurveto{\pgfqpoint{1.351086in}{0.743847in}}{\pgfqpoint{1.340487in}{0.739456in}}{\pgfqpoint{1.332673in}{0.731643in}}%
\pgfpathcurveto{\pgfqpoint{1.324860in}{0.723829in}}{\pgfqpoint{1.320470in}{0.713230in}}{\pgfqpoint{1.320470in}{0.702180in}}%
\pgfpathcurveto{\pgfqpoint{1.320470in}{0.691130in}}{\pgfqpoint{1.324860in}{0.680531in}}{\pgfqpoint{1.332673in}{0.672717in}}%
\pgfpathcurveto{\pgfqpoint{1.340487in}{0.664903in}}{\pgfqpoint{1.351086in}{0.660513in}}{\pgfqpoint{1.362136in}{0.660513in}}%
\pgfpathclose%
\pgfusepath{stroke,fill}%
\end{pgfscope}%
\begin{pgfscope}%
\pgfpathrectangle{\pgfqpoint{0.787074in}{0.548769in}}{\pgfqpoint{5.062926in}{3.102590in}}%
\pgfusepath{clip}%
\pgfsetbuttcap%
\pgfsetroundjoin%
\definecolor{currentfill}{rgb}{0.121569,0.466667,0.705882}%
\pgfsetfillcolor{currentfill}%
\pgfsetlinewidth{1.003750pt}%
\definecolor{currentstroke}{rgb}{0.121569,0.466667,0.705882}%
\pgfsetstrokecolor{currentstroke}%
\pgfsetdash{}{0pt}%
\pgfpathmoveto{\pgfqpoint{1.396317in}{0.648155in}}%
\pgfpathcurveto{\pgfqpoint{1.407368in}{0.648155in}}{\pgfqpoint{1.417967in}{0.652545in}}{\pgfqpoint{1.425780in}{0.660359in}}%
\pgfpathcurveto{\pgfqpoint{1.433594in}{0.668173in}}{\pgfqpoint{1.437984in}{0.678772in}}{\pgfqpoint{1.437984in}{0.689822in}}%
\pgfpathcurveto{\pgfqpoint{1.437984in}{0.700872in}}{\pgfqpoint{1.433594in}{0.711471in}}{\pgfqpoint{1.425780in}{0.719285in}}%
\pgfpathcurveto{\pgfqpoint{1.417967in}{0.727098in}}{\pgfqpoint{1.407368in}{0.731489in}}{\pgfqpoint{1.396317in}{0.731489in}}%
\pgfpathcurveto{\pgfqpoint{1.385267in}{0.731489in}}{\pgfqpoint{1.374668in}{0.727098in}}{\pgfqpoint{1.366855in}{0.719285in}}%
\pgfpathcurveto{\pgfqpoint{1.359041in}{0.711471in}}{\pgfqpoint{1.354651in}{0.700872in}}{\pgfqpoint{1.354651in}{0.689822in}}%
\pgfpathcurveto{\pgfqpoint{1.354651in}{0.678772in}}{\pgfqpoint{1.359041in}{0.668173in}}{\pgfqpoint{1.366855in}{0.660359in}}%
\pgfpathcurveto{\pgfqpoint{1.374668in}{0.652545in}}{\pgfqpoint{1.385267in}{0.648155in}}{\pgfqpoint{1.396317in}{0.648155in}}%
\pgfpathclose%
\pgfusepath{stroke,fill}%
\end{pgfscope}%
\begin{pgfscope}%
\pgfpathrectangle{\pgfqpoint{0.787074in}{0.548769in}}{\pgfqpoint{5.062926in}{3.102590in}}%
\pgfusepath{clip}%
\pgfsetbuttcap%
\pgfsetroundjoin%
\definecolor{currentfill}{rgb}{0.121569,0.466667,0.705882}%
\pgfsetfillcolor{currentfill}%
\pgfsetlinewidth{1.003750pt}%
\definecolor{currentstroke}{rgb}{0.121569,0.466667,0.705882}%
\pgfsetstrokecolor{currentstroke}%
\pgfsetdash{}{0pt}%
\pgfpathmoveto{\pgfqpoint{1.746302in}{0.648148in}}%
\pgfpathcurveto{\pgfqpoint{1.757352in}{0.648148in}}{\pgfqpoint{1.767951in}{0.652538in}}{\pgfqpoint{1.775764in}{0.660352in}}%
\pgfpathcurveto{\pgfqpoint{1.783578in}{0.668166in}}{\pgfqpoint{1.787968in}{0.678765in}}{\pgfqpoint{1.787968in}{0.689815in}}%
\pgfpathcurveto{\pgfqpoint{1.787968in}{0.700865in}}{\pgfqpoint{1.783578in}{0.711464in}}{\pgfqpoint{1.775764in}{0.719278in}}%
\pgfpathcurveto{\pgfqpoint{1.767951in}{0.727091in}}{\pgfqpoint{1.757352in}{0.731482in}}{\pgfqpoint{1.746302in}{0.731482in}}%
\pgfpathcurveto{\pgfqpoint{1.735251in}{0.731482in}}{\pgfqpoint{1.724652in}{0.727091in}}{\pgfqpoint{1.716839in}{0.719278in}}%
\pgfpathcurveto{\pgfqpoint{1.709025in}{0.711464in}}{\pgfqpoint{1.704635in}{0.700865in}}{\pgfqpoint{1.704635in}{0.689815in}}%
\pgfpathcurveto{\pgfqpoint{1.704635in}{0.678765in}}{\pgfqpoint{1.709025in}{0.668166in}}{\pgfqpoint{1.716839in}{0.660352in}}%
\pgfpathcurveto{\pgfqpoint{1.724652in}{0.652538in}}{\pgfqpoint{1.735251in}{0.648148in}}{\pgfqpoint{1.746302in}{0.648148in}}%
\pgfpathclose%
\pgfusepath{stroke,fill}%
\end{pgfscope}%
\begin{pgfscope}%
\pgfpathrectangle{\pgfqpoint{0.787074in}{0.548769in}}{\pgfqpoint{5.062926in}{3.102590in}}%
\pgfusepath{clip}%
\pgfsetbuttcap%
\pgfsetroundjoin%
\definecolor{currentfill}{rgb}{1.000000,0.498039,0.054902}%
\pgfsetfillcolor{currentfill}%
\pgfsetlinewidth{1.003750pt}%
\definecolor{currentstroke}{rgb}{1.000000,0.498039,0.054902}%
\pgfsetstrokecolor{currentstroke}%
\pgfsetdash{}{0pt}%
\pgfpathmoveto{\pgfqpoint{1.637358in}{2.337155in}}%
\pgfpathcurveto{\pgfqpoint{1.648409in}{2.337155in}}{\pgfqpoint{1.659008in}{2.341545in}}{\pgfqpoint{1.666821in}{2.349358in}}%
\pgfpathcurveto{\pgfqpoint{1.674635in}{2.357172in}}{\pgfqpoint{1.679025in}{2.367771in}}{\pgfqpoint{1.679025in}{2.378821in}}%
\pgfpathcurveto{\pgfqpoint{1.679025in}{2.389871in}}{\pgfqpoint{1.674635in}{2.400470in}}{\pgfqpoint{1.666821in}{2.408284in}}%
\pgfpathcurveto{\pgfqpoint{1.659008in}{2.416098in}}{\pgfqpoint{1.648409in}{2.420488in}}{\pgfqpoint{1.637358in}{2.420488in}}%
\pgfpathcurveto{\pgfqpoint{1.626308in}{2.420488in}}{\pgfqpoint{1.615709in}{2.416098in}}{\pgfqpoint{1.607896in}{2.408284in}}%
\pgfpathcurveto{\pgfqpoint{1.600082in}{2.400470in}}{\pgfqpoint{1.595692in}{2.389871in}}{\pgfqpoint{1.595692in}{2.378821in}}%
\pgfpathcurveto{\pgfqpoint{1.595692in}{2.367771in}}{\pgfqpoint{1.600082in}{2.357172in}}{\pgfqpoint{1.607896in}{2.349358in}}%
\pgfpathcurveto{\pgfqpoint{1.615709in}{2.341545in}}{\pgfqpoint{1.626308in}{2.337155in}}{\pgfqpoint{1.637358in}{2.337155in}}%
\pgfpathclose%
\pgfusepath{stroke,fill}%
\end{pgfscope}%
\begin{pgfscope}%
\pgfpathrectangle{\pgfqpoint{0.787074in}{0.548769in}}{\pgfqpoint{5.062926in}{3.102590in}}%
\pgfusepath{clip}%
\pgfsetbuttcap%
\pgfsetroundjoin%
\definecolor{currentfill}{rgb}{1.000000,0.498039,0.054902}%
\pgfsetfillcolor{currentfill}%
\pgfsetlinewidth{1.003750pt}%
\definecolor{currentstroke}{rgb}{1.000000,0.498039,0.054902}%
\pgfsetstrokecolor{currentstroke}%
\pgfsetdash{}{0pt}%
\pgfpathmoveto{\pgfqpoint{1.443128in}{2.913319in}}%
\pgfpathcurveto{\pgfqpoint{1.454178in}{2.913319in}}{\pgfqpoint{1.464777in}{2.917709in}}{\pgfqpoint{1.472591in}{2.925523in}}%
\pgfpathcurveto{\pgfqpoint{1.480404in}{2.933336in}}{\pgfqpoint{1.484795in}{2.943935in}}{\pgfqpoint{1.484795in}{2.954985in}}%
\pgfpathcurveto{\pgfqpoint{1.484795in}{2.966036in}}{\pgfqpoint{1.480404in}{2.976635in}}{\pgfqpoint{1.472591in}{2.984448in}}%
\pgfpathcurveto{\pgfqpoint{1.464777in}{2.992262in}}{\pgfqpoint{1.454178in}{2.996652in}}{\pgfqpoint{1.443128in}{2.996652in}}%
\pgfpathcurveto{\pgfqpoint{1.432078in}{2.996652in}}{\pgfqpoint{1.421479in}{2.992262in}}{\pgfqpoint{1.413665in}{2.984448in}}%
\pgfpathcurveto{\pgfqpoint{1.405851in}{2.976635in}}{\pgfqpoint{1.401461in}{2.966036in}}{\pgfqpoint{1.401461in}{2.954985in}}%
\pgfpathcurveto{\pgfqpoint{1.401461in}{2.943935in}}{\pgfqpoint{1.405851in}{2.933336in}}{\pgfqpoint{1.413665in}{2.925523in}}%
\pgfpathcurveto{\pgfqpoint{1.421479in}{2.917709in}}{\pgfqpoint{1.432078in}{2.913319in}}{\pgfqpoint{1.443128in}{2.913319in}}%
\pgfpathclose%
\pgfusepath{stroke,fill}%
\end{pgfscope}%
\begin{pgfscope}%
\pgfpathrectangle{\pgfqpoint{0.787074in}{0.548769in}}{\pgfqpoint{5.062926in}{3.102590in}}%
\pgfusepath{clip}%
\pgfsetbuttcap%
\pgfsetroundjoin%
\definecolor{currentfill}{rgb}{0.121569,0.466667,0.705882}%
\pgfsetfillcolor{currentfill}%
\pgfsetlinewidth{1.003750pt}%
\definecolor{currentstroke}{rgb}{0.121569,0.466667,0.705882}%
\pgfsetstrokecolor{currentstroke}%
\pgfsetdash{}{0pt}%
\pgfpathmoveto{\pgfqpoint{1.425151in}{0.825123in}}%
\pgfpathcurveto{\pgfqpoint{1.436201in}{0.825123in}}{\pgfqpoint{1.446800in}{0.829514in}}{\pgfqpoint{1.454614in}{0.837327in}}%
\pgfpathcurveto{\pgfqpoint{1.462428in}{0.845141in}}{\pgfqpoint{1.466818in}{0.855740in}}{\pgfqpoint{1.466818in}{0.866790in}}%
\pgfpathcurveto{\pgfqpoint{1.466818in}{0.877840in}}{\pgfqpoint{1.462428in}{0.888439in}}{\pgfqpoint{1.454614in}{0.896253in}}%
\pgfpathcurveto{\pgfqpoint{1.446800in}{0.904067in}}{\pgfqpoint{1.436201in}{0.908457in}}{\pgfqpoint{1.425151in}{0.908457in}}%
\pgfpathcurveto{\pgfqpoint{1.414101in}{0.908457in}}{\pgfqpoint{1.403502in}{0.904067in}}{\pgfqpoint{1.395688in}{0.896253in}}%
\pgfpathcurveto{\pgfqpoint{1.387875in}{0.888439in}}{\pgfqpoint{1.383484in}{0.877840in}}{\pgfqpoint{1.383484in}{0.866790in}}%
\pgfpathcurveto{\pgfqpoint{1.383484in}{0.855740in}}{\pgfqpoint{1.387875in}{0.845141in}}{\pgfqpoint{1.395688in}{0.837327in}}%
\pgfpathcurveto{\pgfqpoint{1.403502in}{0.829514in}}{\pgfqpoint{1.414101in}{0.825123in}}{\pgfqpoint{1.425151in}{0.825123in}}%
\pgfpathclose%
\pgfusepath{stroke,fill}%
\end{pgfscope}%
\begin{pgfscope}%
\pgfpathrectangle{\pgfqpoint{0.787074in}{0.548769in}}{\pgfqpoint{5.062926in}{3.102590in}}%
\pgfusepath{clip}%
\pgfsetbuttcap%
\pgfsetroundjoin%
\definecolor{currentfill}{rgb}{1.000000,0.498039,0.054902}%
\pgfsetfillcolor{currentfill}%
\pgfsetlinewidth{1.003750pt}%
\definecolor{currentstroke}{rgb}{1.000000,0.498039,0.054902}%
\pgfsetstrokecolor{currentstroke}%
\pgfsetdash{}{0pt}%
\pgfpathmoveto{\pgfqpoint{1.466491in}{1.776870in}}%
\pgfpathcurveto{\pgfqpoint{1.477541in}{1.776870in}}{\pgfqpoint{1.488140in}{1.781260in}}{\pgfqpoint{1.495953in}{1.789074in}}%
\pgfpathcurveto{\pgfqpoint{1.503767in}{1.796887in}}{\pgfqpoint{1.508157in}{1.807486in}}{\pgfqpoint{1.508157in}{1.818537in}}%
\pgfpathcurveto{\pgfqpoint{1.508157in}{1.829587in}}{\pgfqpoint{1.503767in}{1.840186in}}{\pgfqpoint{1.495953in}{1.847999in}}%
\pgfpathcurveto{\pgfqpoint{1.488140in}{1.855813in}}{\pgfqpoint{1.477541in}{1.860203in}}{\pgfqpoint{1.466491in}{1.860203in}}%
\pgfpathcurveto{\pgfqpoint{1.455441in}{1.860203in}}{\pgfqpoint{1.444842in}{1.855813in}}{\pgfqpoint{1.437028in}{1.847999in}}%
\pgfpathcurveto{\pgfqpoint{1.429214in}{1.840186in}}{\pgfqpoint{1.424824in}{1.829587in}}{\pgfqpoint{1.424824in}{1.818537in}}%
\pgfpathcurveto{\pgfqpoint{1.424824in}{1.807486in}}{\pgfqpoint{1.429214in}{1.796887in}}{\pgfqpoint{1.437028in}{1.789074in}}%
\pgfpathcurveto{\pgfqpoint{1.444842in}{1.781260in}}{\pgfqpoint{1.455441in}{1.776870in}}{\pgfqpoint{1.466491in}{1.776870in}}%
\pgfpathclose%
\pgfusepath{stroke,fill}%
\end{pgfscope}%
\begin{pgfscope}%
\pgfpathrectangle{\pgfqpoint{0.787074in}{0.548769in}}{\pgfqpoint{5.062926in}{3.102590in}}%
\pgfusepath{clip}%
\pgfsetbuttcap%
\pgfsetroundjoin%
\definecolor{currentfill}{rgb}{1.000000,0.498039,0.054902}%
\pgfsetfillcolor{currentfill}%
\pgfsetlinewidth{1.003750pt}%
\definecolor{currentstroke}{rgb}{1.000000,0.498039,0.054902}%
\pgfsetstrokecolor{currentstroke}%
\pgfsetdash{}{0pt}%
\pgfpathmoveto{\pgfqpoint{1.260606in}{2.789062in}}%
\pgfpathcurveto{\pgfqpoint{1.271656in}{2.789062in}}{\pgfqpoint{1.282255in}{2.793453in}}{\pgfqpoint{1.290068in}{2.801266in}}%
\pgfpathcurveto{\pgfqpoint{1.297882in}{2.809080in}}{\pgfqpoint{1.302272in}{2.819679in}}{\pgfqpoint{1.302272in}{2.830729in}}%
\pgfpathcurveto{\pgfqpoint{1.302272in}{2.841779in}}{\pgfqpoint{1.297882in}{2.852378in}}{\pgfqpoint{1.290068in}{2.860192in}}%
\pgfpathcurveto{\pgfqpoint{1.282255in}{2.868005in}}{\pgfqpoint{1.271656in}{2.872396in}}{\pgfqpoint{1.260606in}{2.872396in}}%
\pgfpathcurveto{\pgfqpoint{1.249556in}{2.872396in}}{\pgfqpoint{1.238957in}{2.868005in}}{\pgfqpoint{1.231143in}{2.860192in}}%
\pgfpathcurveto{\pgfqpoint{1.223329in}{2.852378in}}{\pgfqpoint{1.218939in}{2.841779in}}{\pgfqpoint{1.218939in}{2.830729in}}%
\pgfpathcurveto{\pgfqpoint{1.218939in}{2.819679in}}{\pgfqpoint{1.223329in}{2.809080in}}{\pgfqpoint{1.231143in}{2.801266in}}%
\pgfpathcurveto{\pgfqpoint{1.238957in}{2.793453in}}{\pgfqpoint{1.249556in}{2.789062in}}{\pgfqpoint{1.260606in}{2.789062in}}%
\pgfpathclose%
\pgfusepath{stroke,fill}%
\end{pgfscope}%
\begin{pgfscope}%
\pgfpathrectangle{\pgfqpoint{0.787074in}{0.548769in}}{\pgfqpoint{5.062926in}{3.102590in}}%
\pgfusepath{clip}%
\pgfsetbuttcap%
\pgfsetroundjoin%
\definecolor{currentfill}{rgb}{0.121569,0.466667,0.705882}%
\pgfsetfillcolor{currentfill}%
\pgfsetlinewidth{1.003750pt}%
\definecolor{currentstroke}{rgb}{0.121569,0.466667,0.705882}%
\pgfsetstrokecolor{currentstroke}%
\pgfsetdash{}{0pt}%
\pgfpathmoveto{\pgfqpoint{5.619867in}{0.648149in}}%
\pgfpathcurveto{\pgfqpoint{5.630917in}{0.648149in}}{\pgfqpoint{5.641516in}{0.652540in}}{\pgfqpoint{5.649330in}{0.660353in}}%
\pgfpathcurveto{\pgfqpoint{5.657143in}{0.668167in}}{\pgfqpoint{5.661534in}{0.678766in}}{\pgfqpoint{5.661534in}{0.689816in}}%
\pgfpathcurveto{\pgfqpoint{5.661534in}{0.700866in}}{\pgfqpoint{5.657143in}{0.711465in}}{\pgfqpoint{5.649330in}{0.719279in}}%
\pgfpathcurveto{\pgfqpoint{5.641516in}{0.727093in}}{\pgfqpoint{5.630917in}{0.731483in}}{\pgfqpoint{5.619867in}{0.731483in}}%
\pgfpathcurveto{\pgfqpoint{5.608817in}{0.731483in}}{\pgfqpoint{5.598218in}{0.727093in}}{\pgfqpoint{5.590404in}{0.719279in}}%
\pgfpathcurveto{\pgfqpoint{5.582591in}{0.711465in}}{\pgfqpoint{5.578200in}{0.700866in}}{\pgfqpoint{5.578200in}{0.689816in}}%
\pgfpathcurveto{\pgfqpoint{5.578200in}{0.678766in}}{\pgfqpoint{5.582591in}{0.668167in}}{\pgfqpoint{5.590404in}{0.660353in}}%
\pgfpathcurveto{\pgfqpoint{5.598218in}{0.652540in}}{\pgfqpoint{5.608817in}{0.648149in}}{\pgfqpoint{5.619867in}{0.648149in}}%
\pgfpathclose%
\pgfusepath{stroke,fill}%
\end{pgfscope}%
\begin{pgfscope}%
\pgfpathrectangle{\pgfqpoint{0.787074in}{0.548769in}}{\pgfqpoint{5.062926in}{3.102590in}}%
\pgfusepath{clip}%
\pgfsetbuttcap%
\pgfsetroundjoin%
\definecolor{currentfill}{rgb}{1.000000,0.498039,0.054902}%
\pgfsetfillcolor{currentfill}%
\pgfsetlinewidth{1.003750pt}%
\definecolor{currentstroke}{rgb}{1.000000,0.498039,0.054902}%
\pgfsetstrokecolor{currentstroke}%
\pgfsetdash{}{0pt}%
\pgfpathmoveto{\pgfqpoint{1.421876in}{2.789836in}}%
\pgfpathcurveto{\pgfqpoint{1.432926in}{2.789836in}}{\pgfqpoint{1.443525in}{2.794226in}}{\pgfqpoint{1.451339in}{2.802040in}}%
\pgfpathcurveto{\pgfqpoint{1.459153in}{2.809854in}}{\pgfqpoint{1.463543in}{2.820453in}}{\pgfqpoint{1.463543in}{2.831503in}}%
\pgfpathcurveto{\pgfqpoint{1.463543in}{2.842553in}}{\pgfqpoint{1.459153in}{2.853152in}}{\pgfqpoint{1.451339in}{2.860966in}}%
\pgfpathcurveto{\pgfqpoint{1.443525in}{2.868779in}}{\pgfqpoint{1.432926in}{2.873169in}}{\pgfqpoint{1.421876in}{2.873169in}}%
\pgfpathcurveto{\pgfqpoint{1.410826in}{2.873169in}}{\pgfqpoint{1.400227in}{2.868779in}}{\pgfqpoint{1.392414in}{2.860966in}}%
\pgfpathcurveto{\pgfqpoint{1.384600in}{2.853152in}}{\pgfqpoint{1.380210in}{2.842553in}}{\pgfqpoint{1.380210in}{2.831503in}}%
\pgfpathcurveto{\pgfqpoint{1.380210in}{2.820453in}}{\pgfqpoint{1.384600in}{2.809854in}}{\pgfqpoint{1.392414in}{2.802040in}}%
\pgfpathcurveto{\pgfqpoint{1.400227in}{2.794226in}}{\pgfqpoint{1.410826in}{2.789836in}}{\pgfqpoint{1.421876in}{2.789836in}}%
\pgfpathclose%
\pgfusepath{stroke,fill}%
\end{pgfscope}%
\begin{pgfscope}%
\pgfpathrectangle{\pgfqpoint{0.787074in}{0.548769in}}{\pgfqpoint{5.062926in}{3.102590in}}%
\pgfusepath{clip}%
\pgfsetbuttcap%
\pgfsetroundjoin%
\definecolor{currentfill}{rgb}{0.121569,0.466667,0.705882}%
\pgfsetfillcolor{currentfill}%
\pgfsetlinewidth{1.003750pt}%
\definecolor{currentstroke}{rgb}{0.121569,0.466667,0.705882}%
\pgfsetstrokecolor{currentstroke}%
\pgfsetdash{}{0pt}%
\pgfpathmoveto{\pgfqpoint{1.226925in}{0.648148in}}%
\pgfpathcurveto{\pgfqpoint{1.237975in}{0.648148in}}{\pgfqpoint{1.248574in}{0.652539in}}{\pgfqpoint{1.256388in}{0.660352in}}%
\pgfpathcurveto{\pgfqpoint{1.264202in}{0.668166in}}{\pgfqpoint{1.268592in}{0.678765in}}{\pgfqpoint{1.268592in}{0.689815in}}%
\pgfpathcurveto{\pgfqpoint{1.268592in}{0.700865in}}{\pgfqpoint{1.264202in}{0.711464in}}{\pgfqpoint{1.256388in}{0.719278in}}%
\pgfpathcurveto{\pgfqpoint{1.248574in}{0.727091in}}{\pgfqpoint{1.237975in}{0.731482in}}{\pgfqpoint{1.226925in}{0.731482in}}%
\pgfpathcurveto{\pgfqpoint{1.215875in}{0.731482in}}{\pgfqpoint{1.205276in}{0.727091in}}{\pgfqpoint{1.197463in}{0.719278in}}%
\pgfpathcurveto{\pgfqpoint{1.189649in}{0.711464in}}{\pgfqpoint{1.185259in}{0.700865in}}{\pgfqpoint{1.185259in}{0.689815in}}%
\pgfpathcurveto{\pgfqpoint{1.185259in}{0.678765in}}{\pgfqpoint{1.189649in}{0.668166in}}{\pgfqpoint{1.197463in}{0.660352in}}%
\pgfpathcurveto{\pgfqpoint{1.205276in}{0.652539in}}{\pgfqpoint{1.215875in}{0.648148in}}{\pgfqpoint{1.226925in}{0.648148in}}%
\pgfpathclose%
\pgfusepath{stroke,fill}%
\end{pgfscope}%
\begin{pgfscope}%
\pgfpathrectangle{\pgfqpoint{0.787074in}{0.548769in}}{\pgfqpoint{5.062926in}{3.102590in}}%
\pgfusepath{clip}%
\pgfsetbuttcap%
\pgfsetroundjoin%
\definecolor{currentfill}{rgb}{0.121569,0.466667,0.705882}%
\pgfsetfillcolor{currentfill}%
\pgfsetlinewidth{1.003750pt}%
\definecolor{currentstroke}{rgb}{0.121569,0.466667,0.705882}%
\pgfsetstrokecolor{currentstroke}%
\pgfsetdash{}{0pt}%
\pgfpathmoveto{\pgfqpoint{1.621377in}{0.648168in}}%
\pgfpathcurveto{\pgfqpoint{1.632428in}{0.648168in}}{\pgfqpoint{1.643027in}{0.652558in}}{\pgfqpoint{1.650840in}{0.660371in}}%
\pgfpathcurveto{\pgfqpoint{1.658654in}{0.668185in}}{\pgfqpoint{1.663044in}{0.678784in}}{\pgfqpoint{1.663044in}{0.689834in}}%
\pgfpathcurveto{\pgfqpoint{1.663044in}{0.700884in}}{\pgfqpoint{1.658654in}{0.711483in}}{\pgfqpoint{1.650840in}{0.719297in}}%
\pgfpathcurveto{\pgfqpoint{1.643027in}{0.727111in}}{\pgfqpoint{1.632428in}{0.731501in}}{\pgfqpoint{1.621377in}{0.731501in}}%
\pgfpathcurveto{\pgfqpoint{1.610327in}{0.731501in}}{\pgfqpoint{1.599728in}{0.727111in}}{\pgfqpoint{1.591915in}{0.719297in}}%
\pgfpathcurveto{\pgfqpoint{1.584101in}{0.711483in}}{\pgfqpoint{1.579711in}{0.700884in}}{\pgfqpoint{1.579711in}{0.689834in}}%
\pgfpathcurveto{\pgfqpoint{1.579711in}{0.678784in}}{\pgfqpoint{1.584101in}{0.668185in}}{\pgfqpoint{1.591915in}{0.660371in}}%
\pgfpathcurveto{\pgfqpoint{1.599728in}{0.652558in}}{\pgfqpoint{1.610327in}{0.648168in}}{\pgfqpoint{1.621377in}{0.648168in}}%
\pgfpathclose%
\pgfusepath{stroke,fill}%
\end{pgfscope}%
\begin{pgfscope}%
\pgfpathrectangle{\pgfqpoint{0.787074in}{0.548769in}}{\pgfqpoint{5.062926in}{3.102590in}}%
\pgfusepath{clip}%
\pgfsetbuttcap%
\pgfsetroundjoin%
\definecolor{currentfill}{rgb}{0.121569,0.466667,0.705882}%
\pgfsetfillcolor{currentfill}%
\pgfsetlinewidth{1.003750pt}%
\definecolor{currentstroke}{rgb}{0.121569,0.466667,0.705882}%
\pgfsetstrokecolor{currentstroke}%
\pgfsetdash{}{0pt}%
\pgfpathmoveto{\pgfqpoint{1.224502in}{0.858168in}}%
\pgfpathcurveto{\pgfqpoint{1.235552in}{0.858168in}}{\pgfqpoint{1.246151in}{0.862558in}}{\pgfqpoint{1.253965in}{0.870372in}}%
\pgfpathcurveto{\pgfqpoint{1.261778in}{0.878185in}}{\pgfqpoint{1.266169in}{0.888784in}}{\pgfqpoint{1.266169in}{0.899835in}}%
\pgfpathcurveto{\pgfqpoint{1.266169in}{0.910885in}}{\pgfqpoint{1.261778in}{0.921484in}}{\pgfqpoint{1.253965in}{0.929297in}}%
\pgfpathcurveto{\pgfqpoint{1.246151in}{0.937111in}}{\pgfqpoint{1.235552in}{0.941501in}}{\pgfqpoint{1.224502in}{0.941501in}}%
\pgfpathcurveto{\pgfqpoint{1.213452in}{0.941501in}}{\pgfqpoint{1.202853in}{0.937111in}}{\pgfqpoint{1.195039in}{0.929297in}}%
\pgfpathcurveto{\pgfqpoint{1.187226in}{0.921484in}}{\pgfqpoint{1.182835in}{0.910885in}}{\pgfqpoint{1.182835in}{0.899835in}}%
\pgfpathcurveto{\pgfqpoint{1.182835in}{0.888784in}}{\pgfqpoint{1.187226in}{0.878185in}}{\pgfqpoint{1.195039in}{0.870372in}}%
\pgfpathcurveto{\pgfqpoint{1.202853in}{0.862558in}}{\pgfqpoint{1.213452in}{0.858168in}}{\pgfqpoint{1.224502in}{0.858168in}}%
\pgfpathclose%
\pgfusepath{stroke,fill}%
\end{pgfscope}%
\begin{pgfscope}%
\pgfpathrectangle{\pgfqpoint{0.787074in}{0.548769in}}{\pgfqpoint{5.062926in}{3.102590in}}%
\pgfusepath{clip}%
\pgfsetbuttcap%
\pgfsetroundjoin%
\definecolor{currentfill}{rgb}{1.000000,0.498039,0.054902}%
\pgfsetfillcolor{currentfill}%
\pgfsetlinewidth{1.003750pt}%
\definecolor{currentstroke}{rgb}{1.000000,0.498039,0.054902}%
\pgfsetstrokecolor{currentstroke}%
\pgfsetdash{}{0pt}%
\pgfpathmoveto{\pgfqpoint{2.560468in}{2.478569in}}%
\pgfpathcurveto{\pgfqpoint{2.571518in}{2.478569in}}{\pgfqpoint{2.582117in}{2.482959in}}{\pgfqpoint{2.589931in}{2.490773in}}%
\pgfpathcurveto{\pgfqpoint{2.597744in}{2.498586in}}{\pgfqpoint{2.602135in}{2.509185in}}{\pgfqpoint{2.602135in}{2.520235in}}%
\pgfpathcurveto{\pgfqpoint{2.602135in}{2.531286in}}{\pgfqpoint{2.597744in}{2.541885in}}{\pgfqpoint{2.589931in}{2.549698in}}%
\pgfpathcurveto{\pgfqpoint{2.582117in}{2.557512in}}{\pgfqpoint{2.571518in}{2.561902in}}{\pgfqpoint{2.560468in}{2.561902in}}%
\pgfpathcurveto{\pgfqpoint{2.549418in}{2.561902in}}{\pgfqpoint{2.538819in}{2.557512in}}{\pgfqpoint{2.531005in}{2.549698in}}%
\pgfpathcurveto{\pgfqpoint{2.523192in}{2.541885in}}{\pgfqpoint{2.518801in}{2.531286in}}{\pgfqpoint{2.518801in}{2.520235in}}%
\pgfpathcurveto{\pgfqpoint{2.518801in}{2.509185in}}{\pgfqpoint{2.523192in}{2.498586in}}{\pgfqpoint{2.531005in}{2.490773in}}%
\pgfpathcurveto{\pgfqpoint{2.538819in}{2.482959in}}{\pgfqpoint{2.549418in}{2.478569in}}{\pgfqpoint{2.560468in}{2.478569in}}%
\pgfpathclose%
\pgfusepath{stroke,fill}%
\end{pgfscope}%
\begin{pgfscope}%
\pgfpathrectangle{\pgfqpoint{0.787074in}{0.548769in}}{\pgfqpoint{5.062926in}{3.102590in}}%
\pgfusepath{clip}%
\pgfsetbuttcap%
\pgfsetroundjoin%
\definecolor{currentfill}{rgb}{1.000000,0.498039,0.054902}%
\pgfsetfillcolor{currentfill}%
\pgfsetlinewidth{1.003750pt}%
\definecolor{currentstroke}{rgb}{1.000000,0.498039,0.054902}%
\pgfsetstrokecolor{currentstroke}%
\pgfsetdash{}{0pt}%
\pgfpathmoveto{\pgfqpoint{2.119240in}{1.627903in}}%
\pgfpathcurveto{\pgfqpoint{2.130290in}{1.627903in}}{\pgfqpoint{2.140889in}{1.632294in}}{\pgfqpoint{2.148703in}{1.640107in}}%
\pgfpathcurveto{\pgfqpoint{2.156516in}{1.647921in}}{\pgfqpoint{2.160907in}{1.658520in}}{\pgfqpoint{2.160907in}{1.669570in}}%
\pgfpathcurveto{\pgfqpoint{2.160907in}{1.680620in}}{\pgfqpoint{2.156516in}{1.691219in}}{\pgfqpoint{2.148703in}{1.699033in}}%
\pgfpathcurveto{\pgfqpoint{2.140889in}{1.706846in}}{\pgfqpoint{2.130290in}{1.711237in}}{\pgfqpoint{2.119240in}{1.711237in}}%
\pgfpathcurveto{\pgfqpoint{2.108190in}{1.711237in}}{\pgfqpoint{2.097591in}{1.706846in}}{\pgfqpoint{2.089777in}{1.699033in}}%
\pgfpathcurveto{\pgfqpoint{2.081964in}{1.691219in}}{\pgfqpoint{2.077573in}{1.680620in}}{\pgfqpoint{2.077573in}{1.669570in}}%
\pgfpathcurveto{\pgfqpoint{2.077573in}{1.658520in}}{\pgfqpoint{2.081964in}{1.647921in}}{\pgfqpoint{2.089777in}{1.640107in}}%
\pgfpathcurveto{\pgfqpoint{2.097591in}{1.632294in}}{\pgfqpoint{2.108190in}{1.627903in}}{\pgfqpoint{2.119240in}{1.627903in}}%
\pgfpathclose%
\pgfusepath{stroke,fill}%
\end{pgfscope}%
\begin{pgfscope}%
\pgfpathrectangle{\pgfqpoint{0.787074in}{0.548769in}}{\pgfqpoint{5.062926in}{3.102590in}}%
\pgfusepath{clip}%
\pgfsetbuttcap%
\pgfsetroundjoin%
\definecolor{currentfill}{rgb}{0.121569,0.466667,0.705882}%
\pgfsetfillcolor{currentfill}%
\pgfsetlinewidth{1.003750pt}%
\definecolor{currentstroke}{rgb}{0.121569,0.466667,0.705882}%
\pgfsetstrokecolor{currentstroke}%
\pgfsetdash{}{0pt}%
\pgfpathmoveto{\pgfqpoint{1.439483in}{0.648140in}}%
\pgfpathcurveto{\pgfqpoint{1.450533in}{0.648140in}}{\pgfqpoint{1.461132in}{0.652531in}}{\pgfqpoint{1.468946in}{0.660344in}}%
\pgfpathcurveto{\pgfqpoint{1.476760in}{0.668158in}}{\pgfqpoint{1.481150in}{0.678757in}}{\pgfqpoint{1.481150in}{0.689807in}}%
\pgfpathcurveto{\pgfqpoint{1.481150in}{0.700857in}}{\pgfqpoint{1.476760in}{0.711456in}}{\pgfqpoint{1.468946in}{0.719270in}}%
\pgfpathcurveto{\pgfqpoint{1.461132in}{0.727084in}}{\pgfqpoint{1.450533in}{0.731474in}}{\pgfqpoint{1.439483in}{0.731474in}}%
\pgfpathcurveto{\pgfqpoint{1.428433in}{0.731474in}}{\pgfqpoint{1.417834in}{0.727084in}}{\pgfqpoint{1.410020in}{0.719270in}}%
\pgfpathcurveto{\pgfqpoint{1.402207in}{0.711456in}}{\pgfqpoint{1.397817in}{0.700857in}}{\pgfqpoint{1.397817in}{0.689807in}}%
\pgfpathcurveto{\pgfqpoint{1.397817in}{0.678757in}}{\pgfqpoint{1.402207in}{0.668158in}}{\pgfqpoint{1.410020in}{0.660344in}}%
\pgfpathcurveto{\pgfqpoint{1.417834in}{0.652531in}}{\pgfqpoint{1.428433in}{0.648140in}}{\pgfqpoint{1.439483in}{0.648140in}}%
\pgfpathclose%
\pgfusepath{stroke,fill}%
\end{pgfscope}%
\begin{pgfscope}%
\pgfpathrectangle{\pgfqpoint{0.787074in}{0.548769in}}{\pgfqpoint{5.062926in}{3.102590in}}%
\pgfusepath{clip}%
\pgfsetbuttcap%
\pgfsetroundjoin%
\definecolor{currentfill}{rgb}{0.121569,0.466667,0.705882}%
\pgfsetfillcolor{currentfill}%
\pgfsetlinewidth{1.003750pt}%
\definecolor{currentstroke}{rgb}{0.121569,0.466667,0.705882}%
\pgfsetstrokecolor{currentstroke}%
\pgfsetdash{}{0pt}%
\pgfpathmoveto{\pgfqpoint{1.066452in}{0.786141in}}%
\pgfpathcurveto{\pgfqpoint{1.077502in}{0.786141in}}{\pgfqpoint{1.088101in}{0.790531in}}{\pgfqpoint{1.095915in}{0.798345in}}%
\pgfpathcurveto{\pgfqpoint{1.103729in}{0.806159in}}{\pgfqpoint{1.108119in}{0.816758in}}{\pgfqpoint{1.108119in}{0.827808in}}%
\pgfpathcurveto{\pgfqpoint{1.108119in}{0.838858in}}{\pgfqpoint{1.103729in}{0.849457in}}{\pgfqpoint{1.095915in}{0.857271in}}%
\pgfpathcurveto{\pgfqpoint{1.088101in}{0.865084in}}{\pgfqpoint{1.077502in}{0.869474in}}{\pgfqpoint{1.066452in}{0.869474in}}%
\pgfpathcurveto{\pgfqpoint{1.055402in}{0.869474in}}{\pgfqpoint{1.044803in}{0.865084in}}{\pgfqpoint{1.036989in}{0.857271in}}%
\pgfpathcurveto{\pgfqpoint{1.029176in}{0.849457in}}{\pgfqpoint{1.024786in}{0.838858in}}{\pgfqpoint{1.024786in}{0.827808in}}%
\pgfpathcurveto{\pgfqpoint{1.024786in}{0.816758in}}{\pgfqpoint{1.029176in}{0.806159in}}{\pgfqpoint{1.036989in}{0.798345in}}%
\pgfpathcurveto{\pgfqpoint{1.044803in}{0.790531in}}{\pgfqpoint{1.055402in}{0.786141in}}{\pgfqpoint{1.066452in}{0.786141in}}%
\pgfpathclose%
\pgfusepath{stroke,fill}%
\end{pgfscope}%
\begin{pgfscope}%
\pgfpathrectangle{\pgfqpoint{0.787074in}{0.548769in}}{\pgfqpoint{5.062926in}{3.102590in}}%
\pgfusepath{clip}%
\pgfsetbuttcap%
\pgfsetroundjoin%
\definecolor{currentfill}{rgb}{1.000000,0.498039,0.054902}%
\pgfsetfillcolor{currentfill}%
\pgfsetlinewidth{1.003750pt}%
\definecolor{currentstroke}{rgb}{1.000000,0.498039,0.054902}%
\pgfsetstrokecolor{currentstroke}%
\pgfsetdash{}{0pt}%
\pgfpathmoveto{\pgfqpoint{2.483460in}{2.612758in}}%
\pgfpathcurveto{\pgfqpoint{2.494510in}{2.612758in}}{\pgfqpoint{2.505109in}{2.617148in}}{\pgfqpoint{2.512923in}{2.624962in}}%
\pgfpathcurveto{\pgfqpoint{2.520736in}{2.632775in}}{\pgfqpoint{2.525127in}{2.643374in}}{\pgfqpoint{2.525127in}{2.654424in}}%
\pgfpathcurveto{\pgfqpoint{2.525127in}{2.665474in}}{\pgfqpoint{2.520736in}{2.676074in}}{\pgfqpoint{2.512923in}{2.683887in}}%
\pgfpathcurveto{\pgfqpoint{2.505109in}{2.691701in}}{\pgfqpoint{2.494510in}{2.696091in}}{\pgfqpoint{2.483460in}{2.696091in}}%
\pgfpathcurveto{\pgfqpoint{2.472410in}{2.696091in}}{\pgfqpoint{2.461811in}{2.691701in}}{\pgfqpoint{2.453997in}{2.683887in}}%
\pgfpathcurveto{\pgfqpoint{2.446184in}{2.676074in}}{\pgfqpoint{2.441793in}{2.665474in}}{\pgfqpoint{2.441793in}{2.654424in}}%
\pgfpathcurveto{\pgfqpoint{2.441793in}{2.643374in}}{\pgfqpoint{2.446184in}{2.632775in}}{\pgfqpoint{2.453997in}{2.624962in}}%
\pgfpathcurveto{\pgfqpoint{2.461811in}{2.617148in}}{\pgfqpoint{2.472410in}{2.612758in}}{\pgfqpoint{2.483460in}{2.612758in}}%
\pgfpathclose%
\pgfusepath{stroke,fill}%
\end{pgfscope}%
\begin{pgfscope}%
\pgfpathrectangle{\pgfqpoint{0.787074in}{0.548769in}}{\pgfqpoint{5.062926in}{3.102590in}}%
\pgfusepath{clip}%
\pgfsetbuttcap%
\pgfsetroundjoin%
\definecolor{currentfill}{rgb}{0.121569,0.466667,0.705882}%
\pgfsetfillcolor{currentfill}%
\pgfsetlinewidth{1.003750pt}%
\definecolor{currentstroke}{rgb}{0.121569,0.466667,0.705882}%
\pgfsetstrokecolor{currentstroke}%
\pgfsetdash{}{0pt}%
\pgfpathmoveto{\pgfqpoint{1.079813in}{0.648129in}}%
\pgfpathcurveto{\pgfqpoint{1.090864in}{0.648129in}}{\pgfqpoint{1.101463in}{0.652519in}}{\pgfqpoint{1.109276in}{0.660333in}}%
\pgfpathcurveto{\pgfqpoint{1.117090in}{0.668146in}}{\pgfqpoint{1.121480in}{0.678745in}}{\pgfqpoint{1.121480in}{0.689796in}}%
\pgfpathcurveto{\pgfqpoint{1.121480in}{0.700846in}}{\pgfqpoint{1.117090in}{0.711445in}}{\pgfqpoint{1.109276in}{0.719258in}}%
\pgfpathcurveto{\pgfqpoint{1.101463in}{0.727072in}}{\pgfqpoint{1.090864in}{0.731462in}}{\pgfqpoint{1.079813in}{0.731462in}}%
\pgfpathcurveto{\pgfqpoint{1.068763in}{0.731462in}}{\pgfqpoint{1.058164in}{0.727072in}}{\pgfqpoint{1.050351in}{0.719258in}}%
\pgfpathcurveto{\pgfqpoint{1.042537in}{0.711445in}}{\pgfqpoint{1.038147in}{0.700846in}}{\pgfqpoint{1.038147in}{0.689796in}}%
\pgfpathcurveto{\pgfqpoint{1.038147in}{0.678745in}}{\pgfqpoint{1.042537in}{0.668146in}}{\pgfqpoint{1.050351in}{0.660333in}}%
\pgfpathcurveto{\pgfqpoint{1.058164in}{0.652519in}}{\pgfqpoint{1.068763in}{0.648129in}}{\pgfqpoint{1.079813in}{0.648129in}}%
\pgfpathclose%
\pgfusepath{stroke,fill}%
\end{pgfscope}%
\begin{pgfscope}%
\pgfpathrectangle{\pgfqpoint{0.787074in}{0.548769in}}{\pgfqpoint{5.062926in}{3.102590in}}%
\pgfusepath{clip}%
\pgfsetbuttcap%
\pgfsetroundjoin%
\definecolor{currentfill}{rgb}{1.000000,0.498039,0.054902}%
\pgfsetfillcolor{currentfill}%
\pgfsetlinewidth{1.003750pt}%
\definecolor{currentstroke}{rgb}{1.000000,0.498039,0.054902}%
\pgfsetstrokecolor{currentstroke}%
\pgfsetdash{}{0pt}%
\pgfpathmoveto{\pgfqpoint{1.586491in}{2.514801in}}%
\pgfpathcurveto{\pgfqpoint{1.597541in}{2.514801in}}{\pgfqpoint{1.608140in}{2.519191in}}{\pgfqpoint{1.615954in}{2.527004in}}%
\pgfpathcurveto{\pgfqpoint{1.623767in}{2.534818in}}{\pgfqpoint{1.628158in}{2.545417in}}{\pgfqpoint{1.628158in}{2.556467in}}%
\pgfpathcurveto{\pgfqpoint{1.628158in}{2.567517in}}{\pgfqpoint{1.623767in}{2.578116in}}{\pgfqpoint{1.615954in}{2.585930in}}%
\pgfpathcurveto{\pgfqpoint{1.608140in}{2.593744in}}{\pgfqpoint{1.597541in}{2.598134in}}{\pgfqpoint{1.586491in}{2.598134in}}%
\pgfpathcurveto{\pgfqpoint{1.575441in}{2.598134in}}{\pgfqpoint{1.564842in}{2.593744in}}{\pgfqpoint{1.557028in}{2.585930in}}%
\pgfpathcurveto{\pgfqpoint{1.549215in}{2.578116in}}{\pgfqpoint{1.544824in}{2.567517in}}{\pgfqpoint{1.544824in}{2.556467in}}%
\pgfpathcurveto{\pgfqpoint{1.544824in}{2.545417in}}{\pgfqpoint{1.549215in}{2.534818in}}{\pgfqpoint{1.557028in}{2.527004in}}%
\pgfpathcurveto{\pgfqpoint{1.564842in}{2.519191in}}{\pgfqpoint{1.575441in}{2.514801in}}{\pgfqpoint{1.586491in}{2.514801in}}%
\pgfpathclose%
\pgfusepath{stroke,fill}%
\end{pgfscope}%
\begin{pgfscope}%
\pgfpathrectangle{\pgfqpoint{0.787074in}{0.548769in}}{\pgfqpoint{5.062926in}{3.102590in}}%
\pgfusepath{clip}%
\pgfsetbuttcap%
\pgfsetroundjoin%
\definecolor{currentfill}{rgb}{1.000000,0.498039,0.054902}%
\pgfsetfillcolor{currentfill}%
\pgfsetlinewidth{1.003750pt}%
\definecolor{currentstroke}{rgb}{1.000000,0.498039,0.054902}%
\pgfsetstrokecolor{currentstroke}%
\pgfsetdash{}{0pt}%
\pgfpathmoveto{\pgfqpoint{1.403876in}{2.647988in}}%
\pgfpathcurveto{\pgfqpoint{1.414927in}{2.647988in}}{\pgfqpoint{1.425526in}{2.652378in}}{\pgfqpoint{1.433339in}{2.660191in}}%
\pgfpathcurveto{\pgfqpoint{1.441153in}{2.668005in}}{\pgfqpoint{1.445543in}{2.678604in}}{\pgfqpoint{1.445543in}{2.689654in}}%
\pgfpathcurveto{\pgfqpoint{1.445543in}{2.700704in}}{\pgfqpoint{1.441153in}{2.711303in}}{\pgfqpoint{1.433339in}{2.719117in}}%
\pgfpathcurveto{\pgfqpoint{1.425526in}{2.726931in}}{\pgfqpoint{1.414927in}{2.731321in}}{\pgfqpoint{1.403876in}{2.731321in}}%
\pgfpathcurveto{\pgfqpoint{1.392826in}{2.731321in}}{\pgfqpoint{1.382227in}{2.726931in}}{\pgfqpoint{1.374414in}{2.719117in}}%
\pgfpathcurveto{\pgfqpoint{1.366600in}{2.711303in}}{\pgfqpoint{1.362210in}{2.700704in}}{\pgfqpoint{1.362210in}{2.689654in}}%
\pgfpathcurveto{\pgfqpoint{1.362210in}{2.678604in}}{\pgfqpoint{1.366600in}{2.668005in}}{\pgfqpoint{1.374414in}{2.660191in}}%
\pgfpathcurveto{\pgfqpoint{1.382227in}{2.652378in}}{\pgfqpoint{1.392826in}{2.647988in}}{\pgfqpoint{1.403876in}{2.647988in}}%
\pgfpathclose%
\pgfusepath{stroke,fill}%
\end{pgfscope}%
\begin{pgfscope}%
\pgfpathrectangle{\pgfqpoint{0.787074in}{0.548769in}}{\pgfqpoint{5.062926in}{3.102590in}}%
\pgfusepath{clip}%
\pgfsetbuttcap%
\pgfsetroundjoin%
\definecolor{currentfill}{rgb}{0.121569,0.466667,0.705882}%
\pgfsetfillcolor{currentfill}%
\pgfsetlinewidth{1.003750pt}%
\definecolor{currentstroke}{rgb}{0.121569,0.466667,0.705882}%
\pgfsetstrokecolor{currentstroke}%
\pgfsetdash{}{0pt}%
\pgfpathmoveto{\pgfqpoint{1.993095in}{0.652182in}}%
\pgfpathcurveto{\pgfqpoint{2.004145in}{0.652182in}}{\pgfqpoint{2.014744in}{0.656572in}}{\pgfqpoint{2.022557in}{0.664386in}}%
\pgfpathcurveto{\pgfqpoint{2.030371in}{0.672199in}}{\pgfqpoint{2.034761in}{0.682798in}}{\pgfqpoint{2.034761in}{0.693848in}}%
\pgfpathcurveto{\pgfqpoint{2.034761in}{0.704899in}}{\pgfqpoint{2.030371in}{0.715498in}}{\pgfqpoint{2.022557in}{0.723311in}}%
\pgfpathcurveto{\pgfqpoint{2.014744in}{0.731125in}}{\pgfqpoint{2.004145in}{0.735515in}}{\pgfqpoint{1.993095in}{0.735515in}}%
\pgfpathcurveto{\pgfqpoint{1.982044in}{0.735515in}}{\pgfqpoint{1.971445in}{0.731125in}}{\pgfqpoint{1.963632in}{0.723311in}}%
\pgfpathcurveto{\pgfqpoint{1.955818in}{0.715498in}}{\pgfqpoint{1.951428in}{0.704899in}}{\pgfqpoint{1.951428in}{0.693848in}}%
\pgfpathcurveto{\pgfqpoint{1.951428in}{0.682798in}}{\pgfqpoint{1.955818in}{0.672199in}}{\pgfqpoint{1.963632in}{0.664386in}}%
\pgfpathcurveto{\pgfqpoint{1.971445in}{0.656572in}}{\pgfqpoint{1.982044in}{0.652182in}}{\pgfqpoint{1.993095in}{0.652182in}}%
\pgfpathclose%
\pgfusepath{stroke,fill}%
\end{pgfscope}%
\begin{pgfscope}%
\pgfpathrectangle{\pgfqpoint{0.787074in}{0.548769in}}{\pgfqpoint{5.062926in}{3.102590in}}%
\pgfusepath{clip}%
\pgfsetbuttcap%
\pgfsetroundjoin%
\definecolor{currentfill}{rgb}{0.121569,0.466667,0.705882}%
\pgfsetfillcolor{currentfill}%
\pgfsetlinewidth{1.003750pt}%
\definecolor{currentstroke}{rgb}{0.121569,0.466667,0.705882}%
\pgfsetstrokecolor{currentstroke}%
\pgfsetdash{}{0pt}%
\pgfpathmoveto{\pgfqpoint{1.530896in}{0.648180in}}%
\pgfpathcurveto{\pgfqpoint{1.541947in}{0.648180in}}{\pgfqpoint{1.552546in}{0.652570in}}{\pgfqpoint{1.560359in}{0.660384in}}%
\pgfpathcurveto{\pgfqpoint{1.568173in}{0.668197in}}{\pgfqpoint{1.572563in}{0.678797in}}{\pgfqpoint{1.572563in}{0.689847in}}%
\pgfpathcurveto{\pgfqpoint{1.572563in}{0.700897in}}{\pgfqpoint{1.568173in}{0.711496in}}{\pgfqpoint{1.560359in}{0.719309in}}%
\pgfpathcurveto{\pgfqpoint{1.552546in}{0.727123in}}{\pgfqpoint{1.541947in}{0.731513in}}{\pgfqpoint{1.530896in}{0.731513in}}%
\pgfpathcurveto{\pgfqpoint{1.519846in}{0.731513in}}{\pgfqpoint{1.509247in}{0.727123in}}{\pgfqpoint{1.501434in}{0.719309in}}%
\pgfpathcurveto{\pgfqpoint{1.493620in}{0.711496in}}{\pgfqpoint{1.489230in}{0.700897in}}{\pgfqpoint{1.489230in}{0.689847in}}%
\pgfpathcurveto{\pgfqpoint{1.489230in}{0.678797in}}{\pgfqpoint{1.493620in}{0.668197in}}{\pgfqpoint{1.501434in}{0.660384in}}%
\pgfpathcurveto{\pgfqpoint{1.509247in}{0.652570in}}{\pgfqpoint{1.519846in}{0.648180in}}{\pgfqpoint{1.530896in}{0.648180in}}%
\pgfpathclose%
\pgfusepath{stroke,fill}%
\end{pgfscope}%
\begin{pgfscope}%
\pgfpathrectangle{\pgfqpoint{0.787074in}{0.548769in}}{\pgfqpoint{5.062926in}{3.102590in}}%
\pgfusepath{clip}%
\pgfsetbuttcap%
\pgfsetroundjoin%
\definecolor{currentfill}{rgb}{0.121569,0.466667,0.705882}%
\pgfsetfillcolor{currentfill}%
\pgfsetlinewidth{1.003750pt}%
\definecolor{currentstroke}{rgb}{0.121569,0.466667,0.705882}%
\pgfsetstrokecolor{currentstroke}%
\pgfsetdash{}{0pt}%
\pgfpathmoveto{\pgfqpoint{1.539792in}{0.648153in}}%
\pgfpathcurveto{\pgfqpoint{1.550843in}{0.648153in}}{\pgfqpoint{1.561442in}{0.652543in}}{\pgfqpoint{1.569255in}{0.660356in}}%
\pgfpathcurveto{\pgfqpoint{1.577069in}{0.668170in}}{\pgfqpoint{1.581459in}{0.678769in}}{\pgfqpoint{1.581459in}{0.689819in}}%
\pgfpathcurveto{\pgfqpoint{1.581459in}{0.700869in}}{\pgfqpoint{1.577069in}{0.711468in}}{\pgfqpoint{1.569255in}{0.719282in}}%
\pgfpathcurveto{\pgfqpoint{1.561442in}{0.727096in}}{\pgfqpoint{1.550843in}{0.731486in}}{\pgfqpoint{1.539792in}{0.731486in}}%
\pgfpathcurveto{\pgfqpoint{1.528742in}{0.731486in}}{\pgfqpoint{1.518143in}{0.727096in}}{\pgfqpoint{1.510330in}{0.719282in}}%
\pgfpathcurveto{\pgfqpoint{1.502516in}{0.711468in}}{\pgfqpoint{1.498126in}{0.700869in}}{\pgfqpoint{1.498126in}{0.689819in}}%
\pgfpathcurveto{\pgfqpoint{1.498126in}{0.678769in}}{\pgfqpoint{1.502516in}{0.668170in}}{\pgfqpoint{1.510330in}{0.660356in}}%
\pgfpathcurveto{\pgfqpoint{1.518143in}{0.652543in}}{\pgfqpoint{1.528742in}{0.648153in}}{\pgfqpoint{1.539792in}{0.648153in}}%
\pgfpathclose%
\pgfusepath{stroke,fill}%
\end{pgfscope}%
\begin{pgfscope}%
\pgfpathrectangle{\pgfqpoint{0.787074in}{0.548769in}}{\pgfqpoint{5.062926in}{3.102590in}}%
\pgfusepath{clip}%
\pgfsetbuttcap%
\pgfsetroundjoin%
\definecolor{currentfill}{rgb}{0.121569,0.466667,0.705882}%
\pgfsetfillcolor{currentfill}%
\pgfsetlinewidth{1.003750pt}%
\definecolor{currentstroke}{rgb}{0.121569,0.466667,0.705882}%
\pgfsetstrokecolor{currentstroke}%
\pgfsetdash{}{0pt}%
\pgfpathmoveto{\pgfqpoint{1.568896in}{0.648454in}}%
\pgfpathcurveto{\pgfqpoint{1.579946in}{0.648454in}}{\pgfqpoint{1.590545in}{0.652844in}}{\pgfqpoint{1.598359in}{0.660658in}}%
\pgfpathcurveto{\pgfqpoint{1.606172in}{0.668471in}}{\pgfqpoint{1.610562in}{0.679070in}}{\pgfqpoint{1.610562in}{0.690121in}}%
\pgfpathcurveto{\pgfqpoint{1.610562in}{0.701171in}}{\pgfqpoint{1.606172in}{0.711770in}}{\pgfqpoint{1.598359in}{0.719583in}}%
\pgfpathcurveto{\pgfqpoint{1.590545in}{0.727397in}}{\pgfqpoint{1.579946in}{0.731787in}}{\pgfqpoint{1.568896in}{0.731787in}}%
\pgfpathcurveto{\pgfqpoint{1.557846in}{0.731787in}}{\pgfqpoint{1.547247in}{0.727397in}}{\pgfqpoint{1.539433in}{0.719583in}}%
\pgfpathcurveto{\pgfqpoint{1.531619in}{0.711770in}}{\pgfqpoint{1.527229in}{0.701171in}}{\pgfqpoint{1.527229in}{0.690121in}}%
\pgfpathcurveto{\pgfqpoint{1.527229in}{0.679070in}}{\pgfqpoint{1.531619in}{0.668471in}}{\pgfqpoint{1.539433in}{0.660658in}}%
\pgfpathcurveto{\pgfqpoint{1.547247in}{0.652844in}}{\pgfqpoint{1.557846in}{0.648454in}}{\pgfqpoint{1.568896in}{0.648454in}}%
\pgfpathclose%
\pgfusepath{stroke,fill}%
\end{pgfscope}%
\begin{pgfscope}%
\pgfpathrectangle{\pgfqpoint{0.787074in}{0.548769in}}{\pgfqpoint{5.062926in}{3.102590in}}%
\pgfusepath{clip}%
\pgfsetbuttcap%
\pgfsetroundjoin%
\definecolor{currentfill}{rgb}{0.121569,0.466667,0.705882}%
\pgfsetfillcolor{currentfill}%
\pgfsetlinewidth{1.003750pt}%
\definecolor{currentstroke}{rgb}{0.121569,0.466667,0.705882}%
\pgfsetstrokecolor{currentstroke}%
\pgfsetdash{}{0pt}%
\pgfpathmoveto{\pgfqpoint{1.157187in}{0.648199in}}%
\pgfpathcurveto{\pgfqpoint{1.168238in}{0.648199in}}{\pgfqpoint{1.178837in}{0.652589in}}{\pgfqpoint{1.186650in}{0.660403in}}%
\pgfpathcurveto{\pgfqpoint{1.194464in}{0.668217in}}{\pgfqpoint{1.198854in}{0.678816in}}{\pgfqpoint{1.198854in}{0.689866in}}%
\pgfpathcurveto{\pgfqpoint{1.198854in}{0.700916in}}{\pgfqpoint{1.194464in}{0.711515in}}{\pgfqpoint{1.186650in}{0.719329in}}%
\pgfpathcurveto{\pgfqpoint{1.178837in}{0.727142in}}{\pgfqpoint{1.168238in}{0.731533in}}{\pgfqpoint{1.157187in}{0.731533in}}%
\pgfpathcurveto{\pgfqpoint{1.146137in}{0.731533in}}{\pgfqpoint{1.135538in}{0.727142in}}{\pgfqpoint{1.127725in}{0.719329in}}%
\pgfpathcurveto{\pgfqpoint{1.119911in}{0.711515in}}{\pgfqpoint{1.115521in}{0.700916in}}{\pgfqpoint{1.115521in}{0.689866in}}%
\pgfpathcurveto{\pgfqpoint{1.115521in}{0.678816in}}{\pgfqpoint{1.119911in}{0.668217in}}{\pgfqpoint{1.127725in}{0.660403in}}%
\pgfpathcurveto{\pgfqpoint{1.135538in}{0.652589in}}{\pgfqpoint{1.146137in}{0.648199in}}{\pgfqpoint{1.157187in}{0.648199in}}%
\pgfpathclose%
\pgfusepath{stroke,fill}%
\end{pgfscope}%
\begin{pgfscope}%
\pgfpathrectangle{\pgfqpoint{0.787074in}{0.548769in}}{\pgfqpoint{5.062926in}{3.102590in}}%
\pgfusepath{clip}%
\pgfsetbuttcap%
\pgfsetroundjoin%
\definecolor{currentfill}{rgb}{0.121569,0.466667,0.705882}%
\pgfsetfillcolor{currentfill}%
\pgfsetlinewidth{1.003750pt}%
\definecolor{currentstroke}{rgb}{0.121569,0.466667,0.705882}%
\pgfsetstrokecolor{currentstroke}%
\pgfsetdash{}{0pt}%
\pgfpathmoveto{\pgfqpoint{1.245029in}{0.648149in}}%
\pgfpathcurveto{\pgfqpoint{1.256079in}{0.648149in}}{\pgfqpoint{1.266678in}{0.652540in}}{\pgfqpoint{1.274492in}{0.660353in}}%
\pgfpathcurveto{\pgfqpoint{1.282306in}{0.668167in}}{\pgfqpoint{1.286696in}{0.678766in}}{\pgfqpoint{1.286696in}{0.689816in}}%
\pgfpathcurveto{\pgfqpoint{1.286696in}{0.700866in}}{\pgfqpoint{1.282306in}{0.711465in}}{\pgfqpoint{1.274492in}{0.719279in}}%
\pgfpathcurveto{\pgfqpoint{1.266678in}{0.727092in}}{\pgfqpoint{1.256079in}{0.731483in}}{\pgfqpoint{1.245029in}{0.731483in}}%
\pgfpathcurveto{\pgfqpoint{1.233979in}{0.731483in}}{\pgfqpoint{1.223380in}{0.727092in}}{\pgfqpoint{1.215566in}{0.719279in}}%
\pgfpathcurveto{\pgfqpoint{1.207753in}{0.711465in}}{\pgfqpoint{1.203363in}{0.700866in}}{\pgfqpoint{1.203363in}{0.689816in}}%
\pgfpathcurveto{\pgfqpoint{1.203363in}{0.678766in}}{\pgfqpoint{1.207753in}{0.668167in}}{\pgfqpoint{1.215566in}{0.660353in}}%
\pgfpathcurveto{\pgfqpoint{1.223380in}{0.652540in}}{\pgfqpoint{1.233979in}{0.648149in}}{\pgfqpoint{1.245029in}{0.648149in}}%
\pgfpathclose%
\pgfusepath{stroke,fill}%
\end{pgfscope}%
\begin{pgfscope}%
\pgfpathrectangle{\pgfqpoint{0.787074in}{0.548769in}}{\pgfqpoint{5.062926in}{3.102590in}}%
\pgfusepath{clip}%
\pgfsetbuttcap%
\pgfsetroundjoin%
\definecolor{currentfill}{rgb}{0.121569,0.466667,0.705882}%
\pgfsetfillcolor{currentfill}%
\pgfsetlinewidth{1.003750pt}%
\definecolor{currentstroke}{rgb}{0.121569,0.466667,0.705882}%
\pgfsetstrokecolor{currentstroke}%
\pgfsetdash{}{0pt}%
\pgfpathmoveto{\pgfqpoint{2.840240in}{3.153720in}}%
\pgfpathcurveto{\pgfqpoint{2.851290in}{3.153720in}}{\pgfqpoint{2.861889in}{3.158111in}}{\pgfqpoint{2.869703in}{3.165924in}}%
\pgfpathcurveto{\pgfqpoint{2.877517in}{3.173738in}}{\pgfqpoint{2.881907in}{3.184337in}}{\pgfqpoint{2.881907in}{3.195387in}}%
\pgfpathcurveto{\pgfqpoint{2.881907in}{3.206437in}}{\pgfqpoint{2.877517in}{3.217036in}}{\pgfqpoint{2.869703in}{3.224850in}}%
\pgfpathcurveto{\pgfqpoint{2.861889in}{3.232663in}}{\pgfqpoint{2.851290in}{3.237054in}}{\pgfqpoint{2.840240in}{3.237054in}}%
\pgfpathcurveto{\pgfqpoint{2.829190in}{3.237054in}}{\pgfqpoint{2.818591in}{3.232663in}}{\pgfqpoint{2.810777in}{3.224850in}}%
\pgfpathcurveto{\pgfqpoint{2.802964in}{3.217036in}}{\pgfqpoint{2.798574in}{3.206437in}}{\pgfqpoint{2.798574in}{3.195387in}}%
\pgfpathcurveto{\pgfqpoint{2.798574in}{3.184337in}}{\pgfqpoint{2.802964in}{3.173738in}}{\pgfqpoint{2.810777in}{3.165924in}}%
\pgfpathcurveto{\pgfqpoint{2.818591in}{3.158111in}}{\pgfqpoint{2.829190in}{3.153720in}}{\pgfqpoint{2.840240in}{3.153720in}}%
\pgfpathclose%
\pgfusepath{stroke,fill}%
\end{pgfscope}%
\begin{pgfscope}%
\pgfpathrectangle{\pgfqpoint{0.787074in}{0.548769in}}{\pgfqpoint{5.062926in}{3.102590in}}%
\pgfusepath{clip}%
\pgfsetbuttcap%
\pgfsetroundjoin%
\definecolor{currentfill}{rgb}{1.000000,0.498039,0.054902}%
\pgfsetfillcolor{currentfill}%
\pgfsetlinewidth{1.003750pt}%
\definecolor{currentstroke}{rgb}{1.000000,0.498039,0.054902}%
\pgfsetstrokecolor{currentstroke}%
\pgfsetdash{}{0pt}%
\pgfpathmoveto{\pgfqpoint{2.349278in}{2.857398in}}%
\pgfpathcurveto{\pgfqpoint{2.360328in}{2.857398in}}{\pgfqpoint{2.370927in}{2.861789in}}{\pgfqpoint{2.378741in}{2.869602in}}%
\pgfpathcurveto{\pgfqpoint{2.386554in}{2.877416in}}{\pgfqpoint{2.390944in}{2.888015in}}{\pgfqpoint{2.390944in}{2.899065in}}%
\pgfpathcurveto{\pgfqpoint{2.390944in}{2.910115in}}{\pgfqpoint{2.386554in}{2.920714in}}{\pgfqpoint{2.378741in}{2.928528in}}%
\pgfpathcurveto{\pgfqpoint{2.370927in}{2.936341in}}{\pgfqpoint{2.360328in}{2.940732in}}{\pgfqpoint{2.349278in}{2.940732in}}%
\pgfpathcurveto{\pgfqpoint{2.338228in}{2.940732in}}{\pgfqpoint{2.327629in}{2.936341in}}{\pgfqpoint{2.319815in}{2.928528in}}%
\pgfpathcurveto{\pgfqpoint{2.312001in}{2.920714in}}{\pgfqpoint{2.307611in}{2.910115in}}{\pgfqpoint{2.307611in}{2.899065in}}%
\pgfpathcurveto{\pgfqpoint{2.307611in}{2.888015in}}{\pgfqpoint{2.312001in}{2.877416in}}{\pgfqpoint{2.319815in}{2.869602in}}%
\pgfpathcurveto{\pgfqpoint{2.327629in}{2.861789in}}{\pgfqpoint{2.338228in}{2.857398in}}{\pgfqpoint{2.349278in}{2.857398in}}%
\pgfpathclose%
\pgfusepath{stroke,fill}%
\end{pgfscope}%
\begin{pgfscope}%
\pgfpathrectangle{\pgfqpoint{0.787074in}{0.548769in}}{\pgfqpoint{5.062926in}{3.102590in}}%
\pgfusepath{clip}%
\pgfsetbuttcap%
\pgfsetroundjoin%
\definecolor{currentfill}{rgb}{1.000000,0.498039,0.054902}%
\pgfsetfillcolor{currentfill}%
\pgfsetlinewidth{1.003750pt}%
\definecolor{currentstroke}{rgb}{1.000000,0.498039,0.054902}%
\pgfsetstrokecolor{currentstroke}%
\pgfsetdash{}{0pt}%
\pgfpathmoveto{\pgfqpoint{1.783407in}{3.104611in}}%
\pgfpathcurveto{\pgfqpoint{1.794457in}{3.104611in}}{\pgfqpoint{1.805056in}{3.109002in}}{\pgfqpoint{1.812870in}{3.116815in}}%
\pgfpathcurveto{\pgfqpoint{1.820683in}{3.124629in}}{\pgfqpoint{1.825074in}{3.135228in}}{\pgfqpoint{1.825074in}{3.146278in}}%
\pgfpathcurveto{\pgfqpoint{1.825074in}{3.157328in}}{\pgfqpoint{1.820683in}{3.167927in}}{\pgfqpoint{1.812870in}{3.175741in}}%
\pgfpathcurveto{\pgfqpoint{1.805056in}{3.183554in}}{\pgfqpoint{1.794457in}{3.187945in}}{\pgfqpoint{1.783407in}{3.187945in}}%
\pgfpathcurveto{\pgfqpoint{1.772357in}{3.187945in}}{\pgfqpoint{1.761758in}{3.183554in}}{\pgfqpoint{1.753944in}{3.175741in}}%
\pgfpathcurveto{\pgfqpoint{1.746131in}{3.167927in}}{\pgfqpoint{1.741740in}{3.157328in}}{\pgfqpoint{1.741740in}{3.146278in}}%
\pgfpathcurveto{\pgfqpoint{1.741740in}{3.135228in}}{\pgfqpoint{1.746131in}{3.124629in}}{\pgfqpoint{1.753944in}{3.116815in}}%
\pgfpathcurveto{\pgfqpoint{1.761758in}{3.109002in}}{\pgfqpoint{1.772357in}{3.104611in}}{\pgfqpoint{1.783407in}{3.104611in}}%
\pgfpathclose%
\pgfusepath{stroke,fill}%
\end{pgfscope}%
\begin{pgfscope}%
\pgfpathrectangle{\pgfqpoint{0.787074in}{0.548769in}}{\pgfqpoint{5.062926in}{3.102590in}}%
\pgfusepath{clip}%
\pgfsetbuttcap%
\pgfsetroundjoin%
\definecolor{currentfill}{rgb}{0.121569,0.466667,0.705882}%
\pgfsetfillcolor{currentfill}%
\pgfsetlinewidth{1.003750pt}%
\definecolor{currentstroke}{rgb}{0.121569,0.466667,0.705882}%
\pgfsetstrokecolor{currentstroke}%
\pgfsetdash{}{0pt}%
\pgfpathmoveto{\pgfqpoint{1.752235in}{0.648133in}}%
\pgfpathcurveto{\pgfqpoint{1.763285in}{0.648133in}}{\pgfqpoint{1.773884in}{0.652523in}}{\pgfqpoint{1.781697in}{0.660337in}}%
\pgfpathcurveto{\pgfqpoint{1.789511in}{0.668150in}}{\pgfqpoint{1.793901in}{0.678749in}}{\pgfqpoint{1.793901in}{0.689799in}}%
\pgfpathcurveto{\pgfqpoint{1.793901in}{0.700849in}}{\pgfqpoint{1.789511in}{0.711448in}}{\pgfqpoint{1.781697in}{0.719262in}}%
\pgfpathcurveto{\pgfqpoint{1.773884in}{0.727076in}}{\pgfqpoint{1.763285in}{0.731466in}}{\pgfqpoint{1.752235in}{0.731466in}}%
\pgfpathcurveto{\pgfqpoint{1.741185in}{0.731466in}}{\pgfqpoint{1.730586in}{0.727076in}}{\pgfqpoint{1.722772in}{0.719262in}}%
\pgfpathcurveto{\pgfqpoint{1.714958in}{0.711448in}}{\pgfqpoint{1.710568in}{0.700849in}}{\pgfqpoint{1.710568in}{0.689799in}}%
\pgfpathcurveto{\pgfqpoint{1.710568in}{0.678749in}}{\pgfqpoint{1.714958in}{0.668150in}}{\pgfqpoint{1.722772in}{0.660337in}}%
\pgfpathcurveto{\pgfqpoint{1.730586in}{0.652523in}}{\pgfqpoint{1.741185in}{0.648133in}}{\pgfqpoint{1.752235in}{0.648133in}}%
\pgfpathclose%
\pgfusepath{stroke,fill}%
\end{pgfscope}%
\begin{pgfscope}%
\pgfpathrectangle{\pgfqpoint{0.787074in}{0.548769in}}{\pgfqpoint{5.062926in}{3.102590in}}%
\pgfusepath{clip}%
\pgfsetbuttcap%
\pgfsetroundjoin%
\definecolor{currentfill}{rgb}{0.121569,0.466667,0.705882}%
\pgfsetfillcolor{currentfill}%
\pgfsetlinewidth{1.003750pt}%
\definecolor{currentstroke}{rgb}{0.121569,0.466667,0.705882}%
\pgfsetstrokecolor{currentstroke}%
\pgfsetdash{}{0pt}%
\pgfpathmoveto{\pgfqpoint{1.209634in}{0.679648in}}%
\pgfpathcurveto{\pgfqpoint{1.220684in}{0.679648in}}{\pgfqpoint{1.231283in}{0.684039in}}{\pgfqpoint{1.239097in}{0.691852in}}%
\pgfpathcurveto{\pgfqpoint{1.246911in}{0.699666in}}{\pgfqpoint{1.251301in}{0.710265in}}{\pgfqpoint{1.251301in}{0.721315in}}%
\pgfpathcurveto{\pgfqpoint{1.251301in}{0.732365in}}{\pgfqpoint{1.246911in}{0.742964in}}{\pgfqpoint{1.239097in}{0.750778in}}%
\pgfpathcurveto{\pgfqpoint{1.231283in}{0.758591in}}{\pgfqpoint{1.220684in}{0.762982in}}{\pgfqpoint{1.209634in}{0.762982in}}%
\pgfpathcurveto{\pgfqpoint{1.198584in}{0.762982in}}{\pgfqpoint{1.187985in}{0.758591in}}{\pgfqpoint{1.180172in}{0.750778in}}%
\pgfpathcurveto{\pgfqpoint{1.172358in}{0.742964in}}{\pgfqpoint{1.167968in}{0.732365in}}{\pgfqpoint{1.167968in}{0.721315in}}%
\pgfpathcurveto{\pgfqpoint{1.167968in}{0.710265in}}{\pgfqpoint{1.172358in}{0.699666in}}{\pgfqpoint{1.180172in}{0.691852in}}%
\pgfpathcurveto{\pgfqpoint{1.187985in}{0.684039in}}{\pgfqpoint{1.198584in}{0.679648in}}{\pgfqpoint{1.209634in}{0.679648in}}%
\pgfpathclose%
\pgfusepath{stroke,fill}%
\end{pgfscope}%
\begin{pgfscope}%
\pgfpathrectangle{\pgfqpoint{0.787074in}{0.548769in}}{\pgfqpoint{5.062926in}{3.102590in}}%
\pgfusepath{clip}%
\pgfsetbuttcap%
\pgfsetroundjoin%
\definecolor{currentfill}{rgb}{1.000000,0.498039,0.054902}%
\pgfsetfillcolor{currentfill}%
\pgfsetlinewidth{1.003750pt}%
\definecolor{currentstroke}{rgb}{1.000000,0.498039,0.054902}%
\pgfsetstrokecolor{currentstroke}%
\pgfsetdash{}{0pt}%
\pgfpathmoveto{\pgfqpoint{1.490643in}{2.890450in}}%
\pgfpathcurveto{\pgfqpoint{1.501693in}{2.890450in}}{\pgfqpoint{1.512293in}{2.894841in}}{\pgfqpoint{1.520106in}{2.902654in}}%
\pgfpathcurveto{\pgfqpoint{1.527920in}{2.910468in}}{\pgfqpoint{1.532310in}{2.921067in}}{\pgfqpoint{1.532310in}{2.932117in}}%
\pgfpathcurveto{\pgfqpoint{1.532310in}{2.943167in}}{\pgfqpoint{1.527920in}{2.953766in}}{\pgfqpoint{1.520106in}{2.961580in}}%
\pgfpathcurveto{\pgfqpoint{1.512293in}{2.969393in}}{\pgfqpoint{1.501693in}{2.973784in}}{\pgfqpoint{1.490643in}{2.973784in}}%
\pgfpathcurveto{\pgfqpoint{1.479593in}{2.973784in}}{\pgfqpoint{1.468994in}{2.969393in}}{\pgfqpoint{1.461181in}{2.961580in}}%
\pgfpathcurveto{\pgfqpoint{1.453367in}{2.953766in}}{\pgfqpoint{1.448977in}{2.943167in}}{\pgfqpoint{1.448977in}{2.932117in}}%
\pgfpathcurveto{\pgfqpoint{1.448977in}{2.921067in}}{\pgfqpoint{1.453367in}{2.910468in}}{\pgfqpoint{1.461181in}{2.902654in}}%
\pgfpathcurveto{\pgfqpoint{1.468994in}{2.894841in}}{\pgfqpoint{1.479593in}{2.890450in}}{\pgfqpoint{1.490643in}{2.890450in}}%
\pgfpathclose%
\pgfusepath{stroke,fill}%
\end{pgfscope}%
\begin{pgfscope}%
\pgfpathrectangle{\pgfqpoint{0.787074in}{0.548769in}}{\pgfqpoint{5.062926in}{3.102590in}}%
\pgfusepath{clip}%
\pgfsetbuttcap%
\pgfsetroundjoin%
\definecolor{currentfill}{rgb}{0.121569,0.466667,0.705882}%
\pgfsetfillcolor{currentfill}%
\pgfsetlinewidth{1.003750pt}%
\definecolor{currentstroke}{rgb}{0.121569,0.466667,0.705882}%
\pgfsetstrokecolor{currentstroke}%
\pgfsetdash{}{0pt}%
\pgfpathmoveto{\pgfqpoint{1.066452in}{0.786731in}}%
\pgfpathcurveto{\pgfqpoint{1.077502in}{0.786731in}}{\pgfqpoint{1.088101in}{0.791121in}}{\pgfqpoint{1.095915in}{0.798934in}}%
\pgfpathcurveto{\pgfqpoint{1.103729in}{0.806748in}}{\pgfqpoint{1.108119in}{0.817347in}}{\pgfqpoint{1.108119in}{0.828397in}}%
\pgfpathcurveto{\pgfqpoint{1.108119in}{0.839447in}}{\pgfqpoint{1.103729in}{0.850046in}}{\pgfqpoint{1.095915in}{0.857860in}}%
\pgfpathcurveto{\pgfqpoint{1.088101in}{0.865674in}}{\pgfqpoint{1.077502in}{0.870064in}}{\pgfqpoint{1.066452in}{0.870064in}}%
\pgfpathcurveto{\pgfqpoint{1.055402in}{0.870064in}}{\pgfqpoint{1.044803in}{0.865674in}}{\pgfqpoint{1.036989in}{0.857860in}}%
\pgfpathcurveto{\pgfqpoint{1.029176in}{0.850046in}}{\pgfqpoint{1.024786in}{0.839447in}}{\pgfqpoint{1.024786in}{0.828397in}}%
\pgfpathcurveto{\pgfqpoint{1.024786in}{0.817347in}}{\pgfqpoint{1.029176in}{0.806748in}}{\pgfqpoint{1.036989in}{0.798934in}}%
\pgfpathcurveto{\pgfqpoint{1.044803in}{0.791121in}}{\pgfqpoint{1.055402in}{0.786731in}}{\pgfqpoint{1.066452in}{0.786731in}}%
\pgfpathclose%
\pgfusepath{stroke,fill}%
\end{pgfscope}%
\begin{pgfscope}%
\pgfpathrectangle{\pgfqpoint{0.787074in}{0.548769in}}{\pgfqpoint{5.062926in}{3.102590in}}%
\pgfusepath{clip}%
\pgfsetbuttcap%
\pgfsetroundjoin%
\definecolor{currentfill}{rgb}{0.839216,0.152941,0.156863}%
\pgfsetfillcolor{currentfill}%
\pgfsetlinewidth{1.003750pt}%
\definecolor{currentstroke}{rgb}{0.839216,0.152941,0.156863}%
\pgfsetstrokecolor{currentstroke}%
\pgfsetdash{}{0pt}%
\pgfpathmoveto{\pgfqpoint{1.583081in}{2.746144in}}%
\pgfpathcurveto{\pgfqpoint{1.594132in}{2.746144in}}{\pgfqpoint{1.604731in}{2.750534in}}{\pgfqpoint{1.612544in}{2.758348in}}%
\pgfpathcurveto{\pgfqpoint{1.620358in}{2.766161in}}{\pgfqpoint{1.624748in}{2.776760in}}{\pgfqpoint{1.624748in}{2.787810in}}%
\pgfpathcurveto{\pgfqpoint{1.624748in}{2.798860in}}{\pgfqpoint{1.620358in}{2.809460in}}{\pgfqpoint{1.612544in}{2.817273in}}%
\pgfpathcurveto{\pgfqpoint{1.604731in}{2.825087in}}{\pgfqpoint{1.594132in}{2.829477in}}{\pgfqpoint{1.583081in}{2.829477in}}%
\pgfpathcurveto{\pgfqpoint{1.572031in}{2.829477in}}{\pgfqpoint{1.561432in}{2.825087in}}{\pgfqpoint{1.553619in}{2.817273in}}%
\pgfpathcurveto{\pgfqpoint{1.545805in}{2.809460in}}{\pgfqpoint{1.541415in}{2.798860in}}{\pgfqpoint{1.541415in}{2.787810in}}%
\pgfpathcurveto{\pgfqpoint{1.541415in}{2.776760in}}{\pgfqpoint{1.545805in}{2.766161in}}{\pgfqpoint{1.553619in}{2.758348in}}%
\pgfpathcurveto{\pgfqpoint{1.561432in}{2.750534in}}{\pgfqpoint{1.572031in}{2.746144in}}{\pgfqpoint{1.583081in}{2.746144in}}%
\pgfpathclose%
\pgfusepath{stroke,fill}%
\end{pgfscope}%
\begin{pgfscope}%
\pgfpathrectangle{\pgfqpoint{0.787074in}{0.548769in}}{\pgfqpoint{5.062926in}{3.102590in}}%
\pgfusepath{clip}%
\pgfsetbuttcap%
\pgfsetroundjoin%
\definecolor{currentfill}{rgb}{0.121569,0.466667,0.705882}%
\pgfsetfillcolor{currentfill}%
\pgfsetlinewidth{1.003750pt}%
\definecolor{currentstroke}{rgb}{0.121569,0.466667,0.705882}%
\pgfsetstrokecolor{currentstroke}%
\pgfsetdash{}{0pt}%
\pgfpathmoveto{\pgfqpoint{1.095297in}{0.649982in}}%
\pgfpathcurveto{\pgfqpoint{1.106348in}{0.649982in}}{\pgfqpoint{1.116947in}{0.654372in}}{\pgfqpoint{1.124760in}{0.662186in}}%
\pgfpathcurveto{\pgfqpoint{1.132574in}{0.669999in}}{\pgfqpoint{1.136964in}{0.680598in}}{\pgfqpoint{1.136964in}{0.691648in}}%
\pgfpathcurveto{\pgfqpoint{1.136964in}{0.702699in}}{\pgfqpoint{1.132574in}{0.713298in}}{\pgfqpoint{1.124760in}{0.721111in}}%
\pgfpathcurveto{\pgfqpoint{1.116947in}{0.728925in}}{\pgfqpoint{1.106348in}{0.733315in}}{\pgfqpoint{1.095297in}{0.733315in}}%
\pgfpathcurveto{\pgfqpoint{1.084247in}{0.733315in}}{\pgfqpoint{1.073648in}{0.728925in}}{\pgfqpoint{1.065835in}{0.721111in}}%
\pgfpathcurveto{\pgfqpoint{1.058021in}{0.713298in}}{\pgfqpoint{1.053631in}{0.702699in}}{\pgfqpoint{1.053631in}{0.691648in}}%
\pgfpathcurveto{\pgfqpoint{1.053631in}{0.680598in}}{\pgfqpoint{1.058021in}{0.669999in}}{\pgfqpoint{1.065835in}{0.662186in}}%
\pgfpathcurveto{\pgfqpoint{1.073648in}{0.654372in}}{\pgfqpoint{1.084247in}{0.649982in}}{\pgfqpoint{1.095297in}{0.649982in}}%
\pgfpathclose%
\pgfusepath{stroke,fill}%
\end{pgfscope}%
\begin{pgfscope}%
\pgfpathrectangle{\pgfqpoint{0.787074in}{0.548769in}}{\pgfqpoint{5.062926in}{3.102590in}}%
\pgfusepath{clip}%
\pgfsetbuttcap%
\pgfsetroundjoin%
\definecolor{currentfill}{rgb}{1.000000,0.498039,0.054902}%
\pgfsetfillcolor{currentfill}%
\pgfsetlinewidth{1.003750pt}%
\definecolor{currentstroke}{rgb}{1.000000,0.498039,0.054902}%
\pgfsetstrokecolor{currentstroke}%
\pgfsetdash{}{0pt}%
\pgfpathmoveto{\pgfqpoint{1.922078in}{2.345681in}}%
\pgfpathcurveto{\pgfqpoint{1.933128in}{2.345681in}}{\pgfqpoint{1.943727in}{2.350071in}}{\pgfqpoint{1.951540in}{2.357885in}}%
\pgfpathcurveto{\pgfqpoint{1.959354in}{2.365698in}}{\pgfqpoint{1.963744in}{2.376297in}}{\pgfqpoint{1.963744in}{2.387348in}}%
\pgfpathcurveto{\pgfqpoint{1.963744in}{2.398398in}}{\pgfqpoint{1.959354in}{2.408997in}}{\pgfqpoint{1.951540in}{2.416810in}}%
\pgfpathcurveto{\pgfqpoint{1.943727in}{2.424624in}}{\pgfqpoint{1.933128in}{2.429014in}}{\pgfqpoint{1.922078in}{2.429014in}}%
\pgfpathcurveto{\pgfqpoint{1.911027in}{2.429014in}}{\pgfqpoint{1.900428in}{2.424624in}}{\pgfqpoint{1.892615in}{2.416810in}}%
\pgfpathcurveto{\pgfqpoint{1.884801in}{2.408997in}}{\pgfqpoint{1.880411in}{2.398398in}}{\pgfqpoint{1.880411in}{2.387348in}}%
\pgfpathcurveto{\pgfqpoint{1.880411in}{2.376297in}}{\pgfqpoint{1.884801in}{2.365698in}}{\pgfqpoint{1.892615in}{2.357885in}}%
\pgfpathcurveto{\pgfqpoint{1.900428in}{2.350071in}}{\pgfqpoint{1.911027in}{2.345681in}}{\pgfqpoint{1.922078in}{2.345681in}}%
\pgfpathclose%
\pgfusepath{stroke,fill}%
\end{pgfscope}%
\begin{pgfscope}%
\pgfpathrectangle{\pgfqpoint{0.787074in}{0.548769in}}{\pgfqpoint{5.062926in}{3.102590in}}%
\pgfusepath{clip}%
\pgfsetbuttcap%
\pgfsetroundjoin%
\definecolor{currentfill}{rgb}{0.121569,0.466667,0.705882}%
\pgfsetfillcolor{currentfill}%
\pgfsetlinewidth{1.003750pt}%
\definecolor{currentstroke}{rgb}{0.121569,0.466667,0.705882}%
\pgfsetstrokecolor{currentstroke}%
\pgfsetdash{}{0pt}%
\pgfpathmoveto{\pgfqpoint{2.414766in}{0.651543in}}%
\pgfpathcurveto{\pgfqpoint{2.425816in}{0.651543in}}{\pgfqpoint{2.436415in}{0.655933in}}{\pgfqpoint{2.444229in}{0.663747in}}%
\pgfpathcurveto{\pgfqpoint{2.452043in}{0.671560in}}{\pgfqpoint{2.456433in}{0.682159in}}{\pgfqpoint{2.456433in}{0.693209in}}%
\pgfpathcurveto{\pgfqpoint{2.456433in}{0.704259in}}{\pgfqpoint{2.452043in}{0.714859in}}{\pgfqpoint{2.444229in}{0.722672in}}%
\pgfpathcurveto{\pgfqpoint{2.436415in}{0.730486in}}{\pgfqpoint{2.425816in}{0.734876in}}{\pgfqpoint{2.414766in}{0.734876in}}%
\pgfpathcurveto{\pgfqpoint{2.403716in}{0.734876in}}{\pgfqpoint{2.393117in}{0.730486in}}{\pgfqpoint{2.385303in}{0.722672in}}%
\pgfpathcurveto{\pgfqpoint{2.377490in}{0.714859in}}{\pgfqpoint{2.373099in}{0.704259in}}{\pgfqpoint{2.373099in}{0.693209in}}%
\pgfpathcurveto{\pgfqpoint{2.373099in}{0.682159in}}{\pgfqpoint{2.377490in}{0.671560in}}{\pgfqpoint{2.385303in}{0.663747in}}%
\pgfpathcurveto{\pgfqpoint{2.393117in}{0.655933in}}{\pgfqpoint{2.403716in}{0.651543in}}{\pgfqpoint{2.414766in}{0.651543in}}%
\pgfpathclose%
\pgfusepath{stroke,fill}%
\end{pgfscope}%
\begin{pgfscope}%
\pgfpathrectangle{\pgfqpoint{0.787074in}{0.548769in}}{\pgfqpoint{5.062926in}{3.102590in}}%
\pgfusepath{clip}%
\pgfsetbuttcap%
\pgfsetroundjoin%
\definecolor{currentfill}{rgb}{1.000000,0.498039,0.054902}%
\pgfsetfillcolor{currentfill}%
\pgfsetlinewidth{1.003750pt}%
\definecolor{currentstroke}{rgb}{1.000000,0.498039,0.054902}%
\pgfsetstrokecolor{currentstroke}%
\pgfsetdash{}{0pt}%
\pgfpathmoveto{\pgfqpoint{1.817958in}{2.708956in}}%
\pgfpathcurveto{\pgfqpoint{1.829008in}{2.708956in}}{\pgfqpoint{1.839607in}{2.713346in}}{\pgfqpoint{1.847421in}{2.721160in}}%
\pgfpathcurveto{\pgfqpoint{1.855235in}{2.728974in}}{\pgfqpoint{1.859625in}{2.739573in}}{\pgfqpoint{1.859625in}{2.750623in}}%
\pgfpathcurveto{\pgfqpoint{1.859625in}{2.761673in}}{\pgfqpoint{1.855235in}{2.772272in}}{\pgfqpoint{1.847421in}{2.780086in}}%
\pgfpathcurveto{\pgfqpoint{1.839607in}{2.787899in}}{\pgfqpoint{1.829008in}{2.792289in}}{\pgfqpoint{1.817958in}{2.792289in}}%
\pgfpathcurveto{\pgfqpoint{1.806908in}{2.792289in}}{\pgfqpoint{1.796309in}{2.787899in}}{\pgfqpoint{1.788495in}{2.780086in}}%
\pgfpathcurveto{\pgfqpoint{1.780682in}{2.772272in}}{\pgfqpoint{1.776291in}{2.761673in}}{\pgfqpoint{1.776291in}{2.750623in}}%
\pgfpathcurveto{\pgfqpoint{1.776291in}{2.739573in}}{\pgfqpoint{1.780682in}{2.728974in}}{\pgfqpoint{1.788495in}{2.721160in}}%
\pgfpathcurveto{\pgfqpoint{1.796309in}{2.713346in}}{\pgfqpoint{1.806908in}{2.708956in}}{\pgfqpoint{1.817958in}{2.708956in}}%
\pgfpathclose%
\pgfusepath{stroke,fill}%
\end{pgfscope}%
\begin{pgfscope}%
\pgfpathrectangle{\pgfqpoint{0.787074in}{0.548769in}}{\pgfqpoint{5.062926in}{3.102590in}}%
\pgfusepath{clip}%
\pgfsetbuttcap%
\pgfsetroundjoin%
\definecolor{currentfill}{rgb}{0.121569,0.466667,0.705882}%
\pgfsetfillcolor{currentfill}%
\pgfsetlinewidth{1.003750pt}%
\definecolor{currentstroke}{rgb}{0.121569,0.466667,0.705882}%
\pgfsetstrokecolor{currentstroke}%
\pgfsetdash{}{0pt}%
\pgfpathmoveto{\pgfqpoint{1.605543in}{0.648158in}}%
\pgfpathcurveto{\pgfqpoint{1.616593in}{0.648158in}}{\pgfqpoint{1.627192in}{0.652548in}}{\pgfqpoint{1.635006in}{0.660362in}}%
\pgfpathcurveto{\pgfqpoint{1.642819in}{0.668176in}}{\pgfqpoint{1.647209in}{0.678775in}}{\pgfqpoint{1.647209in}{0.689825in}}%
\pgfpathcurveto{\pgfqpoint{1.647209in}{0.700875in}}{\pgfqpoint{1.642819in}{0.711474in}}{\pgfqpoint{1.635006in}{0.719287in}}%
\pgfpathcurveto{\pgfqpoint{1.627192in}{0.727101in}}{\pgfqpoint{1.616593in}{0.731491in}}{\pgfqpoint{1.605543in}{0.731491in}}%
\pgfpathcurveto{\pgfqpoint{1.594493in}{0.731491in}}{\pgfqpoint{1.583894in}{0.727101in}}{\pgfqpoint{1.576080in}{0.719287in}}%
\pgfpathcurveto{\pgfqpoint{1.568266in}{0.711474in}}{\pgfqpoint{1.563876in}{0.700875in}}{\pgfqpoint{1.563876in}{0.689825in}}%
\pgfpathcurveto{\pgfqpoint{1.563876in}{0.678775in}}{\pgfqpoint{1.568266in}{0.668176in}}{\pgfqpoint{1.576080in}{0.660362in}}%
\pgfpathcurveto{\pgfqpoint{1.583894in}{0.652548in}}{\pgfqpoint{1.594493in}{0.648158in}}{\pgfqpoint{1.605543in}{0.648158in}}%
\pgfpathclose%
\pgfusepath{stroke,fill}%
\end{pgfscope}%
\begin{pgfscope}%
\pgfpathrectangle{\pgfqpoint{0.787074in}{0.548769in}}{\pgfqpoint{5.062926in}{3.102590in}}%
\pgfusepath{clip}%
\pgfsetbuttcap%
\pgfsetroundjoin%
\definecolor{currentfill}{rgb}{1.000000,0.498039,0.054902}%
\pgfsetfillcolor{currentfill}%
\pgfsetlinewidth{1.003750pt}%
\definecolor{currentstroke}{rgb}{1.000000,0.498039,0.054902}%
\pgfsetstrokecolor{currentstroke}%
\pgfsetdash{}{0pt}%
\pgfpathmoveto{\pgfqpoint{1.315314in}{2.868283in}}%
\pgfpathcurveto{\pgfqpoint{1.326364in}{2.868283in}}{\pgfqpoint{1.336963in}{2.872673in}}{\pgfqpoint{1.344777in}{2.880486in}}%
\pgfpathcurveto{\pgfqpoint{1.352591in}{2.888300in}}{\pgfqpoint{1.356981in}{2.898899in}}{\pgfqpoint{1.356981in}{2.909949in}}%
\pgfpathcurveto{\pgfqpoint{1.356981in}{2.920999in}}{\pgfqpoint{1.352591in}{2.931598in}}{\pgfqpoint{1.344777in}{2.939412in}}%
\pgfpathcurveto{\pgfqpoint{1.336963in}{2.947226in}}{\pgfqpoint{1.326364in}{2.951616in}}{\pgfqpoint{1.315314in}{2.951616in}}%
\pgfpathcurveto{\pgfqpoint{1.304264in}{2.951616in}}{\pgfqpoint{1.293665in}{2.947226in}}{\pgfqpoint{1.285851in}{2.939412in}}%
\pgfpathcurveto{\pgfqpoint{1.278038in}{2.931598in}}{\pgfqpoint{1.273648in}{2.920999in}}{\pgfqpoint{1.273648in}{2.909949in}}%
\pgfpathcurveto{\pgfqpoint{1.273648in}{2.898899in}}{\pgfqpoint{1.278038in}{2.888300in}}{\pgfqpoint{1.285851in}{2.880486in}}%
\pgfpathcurveto{\pgfqpoint{1.293665in}{2.872673in}}{\pgfqpoint{1.304264in}{2.868283in}}{\pgfqpoint{1.315314in}{2.868283in}}%
\pgfpathclose%
\pgfusepath{stroke,fill}%
\end{pgfscope}%
\begin{pgfscope}%
\pgfpathrectangle{\pgfqpoint{0.787074in}{0.548769in}}{\pgfqpoint{5.062926in}{3.102590in}}%
\pgfusepath{clip}%
\pgfsetbuttcap%
\pgfsetroundjoin%
\definecolor{currentfill}{rgb}{0.121569,0.466667,0.705882}%
\pgfsetfillcolor{currentfill}%
\pgfsetlinewidth{1.003750pt}%
\definecolor{currentstroke}{rgb}{0.121569,0.466667,0.705882}%
\pgfsetstrokecolor{currentstroke}%
\pgfsetdash{}{0pt}%
\pgfpathmoveto{\pgfqpoint{1.562454in}{0.648132in}}%
\pgfpathcurveto{\pgfqpoint{1.573504in}{0.648132in}}{\pgfqpoint{1.584103in}{0.652522in}}{\pgfqpoint{1.591917in}{0.660336in}}%
\pgfpathcurveto{\pgfqpoint{1.599730in}{0.668149in}}{\pgfqpoint{1.604121in}{0.678748in}}{\pgfqpoint{1.604121in}{0.689799in}}%
\pgfpathcurveto{\pgfqpoint{1.604121in}{0.700849in}}{\pgfqpoint{1.599730in}{0.711448in}}{\pgfqpoint{1.591917in}{0.719261in}}%
\pgfpathcurveto{\pgfqpoint{1.584103in}{0.727075in}}{\pgfqpoint{1.573504in}{0.731465in}}{\pgfqpoint{1.562454in}{0.731465in}}%
\pgfpathcurveto{\pgfqpoint{1.551404in}{0.731465in}}{\pgfqpoint{1.540805in}{0.727075in}}{\pgfqpoint{1.532991in}{0.719261in}}%
\pgfpathcurveto{\pgfqpoint{1.525178in}{0.711448in}}{\pgfqpoint{1.520787in}{0.700849in}}{\pgfqpoint{1.520787in}{0.689799in}}%
\pgfpathcurveto{\pgfqpoint{1.520787in}{0.678748in}}{\pgfqpoint{1.525178in}{0.668149in}}{\pgfqpoint{1.532991in}{0.660336in}}%
\pgfpathcurveto{\pgfqpoint{1.540805in}{0.652522in}}{\pgfqpoint{1.551404in}{0.648132in}}{\pgfqpoint{1.562454in}{0.648132in}}%
\pgfpathclose%
\pgfusepath{stroke,fill}%
\end{pgfscope}%
\begin{pgfscope}%
\pgfpathrectangle{\pgfqpoint{0.787074in}{0.548769in}}{\pgfqpoint{5.062926in}{3.102590in}}%
\pgfusepath{clip}%
\pgfsetbuttcap%
\pgfsetroundjoin%
\definecolor{currentfill}{rgb}{1.000000,0.498039,0.054902}%
\pgfsetfillcolor{currentfill}%
\pgfsetlinewidth{1.003750pt}%
\definecolor{currentstroke}{rgb}{1.000000,0.498039,0.054902}%
\pgfsetstrokecolor{currentstroke}%
\pgfsetdash{}{0pt}%
\pgfpathmoveto{\pgfqpoint{1.192482in}{2.305856in}}%
\pgfpathcurveto{\pgfqpoint{1.203532in}{2.305856in}}{\pgfqpoint{1.214131in}{2.310246in}}{\pgfqpoint{1.221945in}{2.318060in}}%
\pgfpathcurveto{\pgfqpoint{1.229758in}{2.325873in}}{\pgfqpoint{1.234149in}{2.336472in}}{\pgfqpoint{1.234149in}{2.347523in}}%
\pgfpathcurveto{\pgfqpoint{1.234149in}{2.358573in}}{\pgfqpoint{1.229758in}{2.369172in}}{\pgfqpoint{1.221945in}{2.376985in}}%
\pgfpathcurveto{\pgfqpoint{1.214131in}{2.384799in}}{\pgfqpoint{1.203532in}{2.389189in}}{\pgfqpoint{1.192482in}{2.389189in}}%
\pgfpathcurveto{\pgfqpoint{1.181432in}{2.389189in}}{\pgfqpoint{1.170833in}{2.384799in}}{\pgfqpoint{1.163019in}{2.376985in}}%
\pgfpathcurveto{\pgfqpoint{1.155206in}{2.369172in}}{\pgfqpoint{1.150815in}{2.358573in}}{\pgfqpoint{1.150815in}{2.347523in}}%
\pgfpathcurveto{\pgfqpoint{1.150815in}{2.336472in}}{\pgfqpoint{1.155206in}{2.325873in}}{\pgfqpoint{1.163019in}{2.318060in}}%
\pgfpathcurveto{\pgfqpoint{1.170833in}{2.310246in}}{\pgfqpoint{1.181432in}{2.305856in}}{\pgfqpoint{1.192482in}{2.305856in}}%
\pgfpathclose%
\pgfusepath{stroke,fill}%
\end{pgfscope}%
\begin{pgfscope}%
\pgfpathrectangle{\pgfqpoint{0.787074in}{0.548769in}}{\pgfqpoint{5.062926in}{3.102590in}}%
\pgfusepath{clip}%
\pgfsetbuttcap%
\pgfsetroundjoin%
\definecolor{currentfill}{rgb}{1.000000,0.498039,0.054902}%
\pgfsetfillcolor{currentfill}%
\pgfsetlinewidth{1.003750pt}%
\definecolor{currentstroke}{rgb}{1.000000,0.498039,0.054902}%
\pgfsetstrokecolor{currentstroke}%
\pgfsetdash{}{0pt}%
\pgfpathmoveto{\pgfqpoint{1.263002in}{2.492823in}}%
\pgfpathcurveto{\pgfqpoint{1.274052in}{2.492823in}}{\pgfqpoint{1.284651in}{2.497213in}}{\pgfqpoint{1.292465in}{2.505027in}}%
\pgfpathcurveto{\pgfqpoint{1.300278in}{2.512840in}}{\pgfqpoint{1.304669in}{2.523439in}}{\pgfqpoint{1.304669in}{2.534490in}}%
\pgfpathcurveto{\pgfqpoint{1.304669in}{2.545540in}}{\pgfqpoint{1.300278in}{2.556139in}}{\pgfqpoint{1.292465in}{2.563952in}}%
\pgfpathcurveto{\pgfqpoint{1.284651in}{2.571766in}}{\pgfqpoint{1.274052in}{2.576156in}}{\pgfqpoint{1.263002in}{2.576156in}}%
\pgfpathcurveto{\pgfqpoint{1.251952in}{2.576156in}}{\pgfqpoint{1.241353in}{2.571766in}}{\pgfqpoint{1.233539in}{2.563952in}}%
\pgfpathcurveto{\pgfqpoint{1.225726in}{2.556139in}}{\pgfqpoint{1.221335in}{2.545540in}}{\pgfqpoint{1.221335in}{2.534490in}}%
\pgfpathcurveto{\pgfqpoint{1.221335in}{2.523439in}}{\pgfqpoint{1.225726in}{2.512840in}}{\pgfqpoint{1.233539in}{2.505027in}}%
\pgfpathcurveto{\pgfqpoint{1.241353in}{2.497213in}}{\pgfqpoint{1.251952in}{2.492823in}}{\pgfqpoint{1.263002in}{2.492823in}}%
\pgfpathclose%
\pgfusepath{stroke,fill}%
\end{pgfscope}%
\begin{pgfscope}%
\pgfpathrectangle{\pgfqpoint{0.787074in}{0.548769in}}{\pgfqpoint{5.062926in}{3.102590in}}%
\pgfusepath{clip}%
\pgfsetbuttcap%
\pgfsetroundjoin%
\definecolor{currentfill}{rgb}{0.121569,0.466667,0.705882}%
\pgfsetfillcolor{currentfill}%
\pgfsetlinewidth{1.003750pt}%
\definecolor{currentstroke}{rgb}{0.121569,0.466667,0.705882}%
\pgfsetstrokecolor{currentstroke}%
\pgfsetdash{}{0pt}%
\pgfpathmoveto{\pgfqpoint{1.487465in}{0.648147in}}%
\pgfpathcurveto{\pgfqpoint{1.498515in}{0.648147in}}{\pgfqpoint{1.509114in}{0.652537in}}{\pgfqpoint{1.516928in}{0.660351in}}%
\pgfpathcurveto{\pgfqpoint{1.524741in}{0.668165in}}{\pgfqpoint{1.529132in}{0.678764in}}{\pgfqpoint{1.529132in}{0.689814in}}%
\pgfpathcurveto{\pgfqpoint{1.529132in}{0.700864in}}{\pgfqpoint{1.524741in}{0.711463in}}{\pgfqpoint{1.516928in}{0.719276in}}%
\pgfpathcurveto{\pgfqpoint{1.509114in}{0.727090in}}{\pgfqpoint{1.498515in}{0.731480in}}{\pgfqpoint{1.487465in}{0.731480in}}%
\pgfpathcurveto{\pgfqpoint{1.476415in}{0.731480in}}{\pgfqpoint{1.465816in}{0.727090in}}{\pgfqpoint{1.458002in}{0.719276in}}%
\pgfpathcurveto{\pgfqpoint{1.450188in}{0.711463in}}{\pgfqpoint{1.445798in}{0.700864in}}{\pgfqpoint{1.445798in}{0.689814in}}%
\pgfpathcurveto{\pgfqpoint{1.445798in}{0.678764in}}{\pgfqpoint{1.450188in}{0.668165in}}{\pgfqpoint{1.458002in}{0.660351in}}%
\pgfpathcurveto{\pgfqpoint{1.465816in}{0.652537in}}{\pgfqpoint{1.476415in}{0.648147in}}{\pgfqpoint{1.487465in}{0.648147in}}%
\pgfpathclose%
\pgfusepath{stroke,fill}%
\end{pgfscope}%
\begin{pgfscope}%
\pgfpathrectangle{\pgfqpoint{0.787074in}{0.548769in}}{\pgfqpoint{5.062926in}{3.102590in}}%
\pgfusepath{clip}%
\pgfsetbuttcap%
\pgfsetroundjoin%
\definecolor{currentfill}{rgb}{1.000000,0.498039,0.054902}%
\pgfsetfillcolor{currentfill}%
\pgfsetlinewidth{1.003750pt}%
\definecolor{currentstroke}{rgb}{1.000000,0.498039,0.054902}%
\pgfsetstrokecolor{currentstroke}%
\pgfsetdash{}{0pt}%
\pgfpathmoveto{\pgfqpoint{1.215063in}{2.754884in}}%
\pgfpathcurveto{\pgfqpoint{1.226113in}{2.754884in}}{\pgfqpoint{1.236712in}{2.759274in}}{\pgfqpoint{1.244526in}{2.767088in}}%
\pgfpathcurveto{\pgfqpoint{1.252339in}{2.774902in}}{\pgfqpoint{1.256729in}{2.785501in}}{\pgfqpoint{1.256729in}{2.796551in}}%
\pgfpathcurveto{\pgfqpoint{1.256729in}{2.807601in}}{\pgfqpoint{1.252339in}{2.818200in}}{\pgfqpoint{1.244526in}{2.826014in}}%
\pgfpathcurveto{\pgfqpoint{1.236712in}{2.833827in}}{\pgfqpoint{1.226113in}{2.838218in}}{\pgfqpoint{1.215063in}{2.838218in}}%
\pgfpathcurveto{\pgfqpoint{1.204013in}{2.838218in}}{\pgfqpoint{1.193414in}{2.833827in}}{\pgfqpoint{1.185600in}{2.826014in}}%
\pgfpathcurveto{\pgfqpoint{1.177786in}{2.818200in}}{\pgfqpoint{1.173396in}{2.807601in}}{\pgfqpoint{1.173396in}{2.796551in}}%
\pgfpathcurveto{\pgfqpoint{1.173396in}{2.785501in}}{\pgfqpoint{1.177786in}{2.774902in}}{\pgfqpoint{1.185600in}{2.767088in}}%
\pgfpathcurveto{\pgfqpoint{1.193414in}{2.759274in}}{\pgfqpoint{1.204013in}{2.754884in}}{\pgfqpoint{1.215063in}{2.754884in}}%
\pgfpathclose%
\pgfusepath{stroke,fill}%
\end{pgfscope}%
\begin{pgfscope}%
\pgfpathrectangle{\pgfqpoint{0.787074in}{0.548769in}}{\pgfqpoint{5.062926in}{3.102590in}}%
\pgfusepath{clip}%
\pgfsetbuttcap%
\pgfsetroundjoin%
\definecolor{currentfill}{rgb}{1.000000,0.498039,0.054902}%
\pgfsetfillcolor{currentfill}%
\pgfsetlinewidth{1.003750pt}%
\definecolor{currentstroke}{rgb}{1.000000,0.498039,0.054902}%
\pgfsetstrokecolor{currentstroke}%
\pgfsetdash{}{0pt}%
\pgfpathmoveto{\pgfqpoint{1.071931in}{2.596250in}}%
\pgfpathcurveto{\pgfqpoint{1.082981in}{2.596250in}}{\pgfqpoint{1.093580in}{2.600640in}}{\pgfqpoint{1.101394in}{2.608454in}}%
\pgfpathcurveto{\pgfqpoint{1.109207in}{2.616267in}}{\pgfqpoint{1.113597in}{2.626867in}}{\pgfqpoint{1.113597in}{2.637917in}}%
\pgfpathcurveto{\pgfqpoint{1.113597in}{2.648967in}}{\pgfqpoint{1.109207in}{2.659566in}}{\pgfqpoint{1.101394in}{2.667379in}}%
\pgfpathcurveto{\pgfqpoint{1.093580in}{2.675193in}}{\pgfqpoint{1.082981in}{2.679583in}}{\pgfqpoint{1.071931in}{2.679583in}}%
\pgfpathcurveto{\pgfqpoint{1.060881in}{2.679583in}}{\pgfqpoint{1.050282in}{2.675193in}}{\pgfqpoint{1.042468in}{2.667379in}}%
\pgfpathcurveto{\pgfqpoint{1.034654in}{2.659566in}}{\pgfqpoint{1.030264in}{2.648967in}}{\pgfqpoint{1.030264in}{2.637917in}}%
\pgfpathcurveto{\pgfqpoint{1.030264in}{2.626867in}}{\pgfqpoint{1.034654in}{2.616267in}}{\pgfqpoint{1.042468in}{2.608454in}}%
\pgfpathcurveto{\pgfqpoint{1.050282in}{2.600640in}}{\pgfqpoint{1.060881in}{2.596250in}}{\pgfqpoint{1.071931in}{2.596250in}}%
\pgfpathclose%
\pgfusepath{stroke,fill}%
\end{pgfscope}%
\begin{pgfscope}%
\pgfpathrectangle{\pgfqpoint{0.787074in}{0.548769in}}{\pgfqpoint{5.062926in}{3.102590in}}%
\pgfusepath{clip}%
\pgfsetbuttcap%
\pgfsetroundjoin%
\definecolor{currentfill}{rgb}{0.121569,0.466667,0.705882}%
\pgfsetfillcolor{currentfill}%
\pgfsetlinewidth{1.003750pt}%
\definecolor{currentstroke}{rgb}{0.121569,0.466667,0.705882}%
\pgfsetstrokecolor{currentstroke}%
\pgfsetdash{}{0pt}%
\pgfpathmoveto{\pgfqpoint{2.253126in}{0.659080in}}%
\pgfpathcurveto{\pgfqpoint{2.264176in}{0.659080in}}{\pgfqpoint{2.274775in}{0.663470in}}{\pgfqpoint{2.282588in}{0.671284in}}%
\pgfpathcurveto{\pgfqpoint{2.290402in}{0.679098in}}{\pgfqpoint{2.294792in}{0.689697in}}{\pgfqpoint{2.294792in}{0.700747in}}%
\pgfpathcurveto{\pgfqpoint{2.294792in}{0.711797in}}{\pgfqpoint{2.290402in}{0.722396in}}{\pgfqpoint{2.282588in}{0.730210in}}%
\pgfpathcurveto{\pgfqpoint{2.274775in}{0.738023in}}{\pgfqpoint{2.264176in}{0.742413in}}{\pgfqpoint{2.253126in}{0.742413in}}%
\pgfpathcurveto{\pgfqpoint{2.242075in}{0.742413in}}{\pgfqpoint{2.231476in}{0.738023in}}{\pgfqpoint{2.223663in}{0.730210in}}%
\pgfpathcurveto{\pgfqpoint{2.215849in}{0.722396in}}{\pgfqpoint{2.211459in}{0.711797in}}{\pgfqpoint{2.211459in}{0.700747in}}%
\pgfpathcurveto{\pgfqpoint{2.211459in}{0.689697in}}{\pgfqpoint{2.215849in}{0.679098in}}{\pgfqpoint{2.223663in}{0.671284in}}%
\pgfpathcurveto{\pgfqpoint{2.231476in}{0.663470in}}{\pgfqpoint{2.242075in}{0.659080in}}{\pgfqpoint{2.253126in}{0.659080in}}%
\pgfpathclose%
\pgfusepath{stroke,fill}%
\end{pgfscope}%
\begin{pgfscope}%
\pgfpathrectangle{\pgfqpoint{0.787074in}{0.548769in}}{\pgfqpoint{5.062926in}{3.102590in}}%
\pgfusepath{clip}%
\pgfsetbuttcap%
\pgfsetroundjoin%
\definecolor{currentfill}{rgb}{1.000000,0.498039,0.054902}%
\pgfsetfillcolor{currentfill}%
\pgfsetlinewidth{1.003750pt}%
\definecolor{currentstroke}{rgb}{1.000000,0.498039,0.054902}%
\pgfsetstrokecolor{currentstroke}%
\pgfsetdash{}{0pt}%
\pgfpathmoveto{\pgfqpoint{2.035544in}{2.981796in}}%
\pgfpathcurveto{\pgfqpoint{2.046594in}{2.981796in}}{\pgfqpoint{2.057193in}{2.986186in}}{\pgfqpoint{2.065007in}{2.994000in}}%
\pgfpathcurveto{\pgfqpoint{2.072820in}{3.001813in}}{\pgfqpoint{2.077210in}{3.012412in}}{\pgfqpoint{2.077210in}{3.023462in}}%
\pgfpathcurveto{\pgfqpoint{2.077210in}{3.034512in}}{\pgfqpoint{2.072820in}{3.045112in}}{\pgfqpoint{2.065007in}{3.052925in}}%
\pgfpathcurveto{\pgfqpoint{2.057193in}{3.060739in}}{\pgfqpoint{2.046594in}{3.065129in}}{\pgfqpoint{2.035544in}{3.065129in}}%
\pgfpathcurveto{\pgfqpoint{2.024494in}{3.065129in}}{\pgfqpoint{2.013895in}{3.060739in}}{\pgfqpoint{2.006081in}{3.052925in}}%
\pgfpathcurveto{\pgfqpoint{1.998267in}{3.045112in}}{\pgfqpoint{1.993877in}{3.034512in}}{\pgfqpoint{1.993877in}{3.023462in}}%
\pgfpathcurveto{\pgfqpoint{1.993877in}{3.012412in}}{\pgfqpoint{1.998267in}{3.001813in}}{\pgfqpoint{2.006081in}{2.994000in}}%
\pgfpathcurveto{\pgfqpoint{2.013895in}{2.986186in}}{\pgfqpoint{2.024494in}{2.981796in}}{\pgfqpoint{2.035544in}{2.981796in}}%
\pgfpathclose%
\pgfusepath{stroke,fill}%
\end{pgfscope}%
\begin{pgfscope}%
\pgfpathrectangle{\pgfqpoint{0.787074in}{0.548769in}}{\pgfqpoint{5.062926in}{3.102590in}}%
\pgfusepath{clip}%
\pgfsetbuttcap%
\pgfsetroundjoin%
\definecolor{currentfill}{rgb}{1.000000,0.498039,0.054902}%
\pgfsetfillcolor{currentfill}%
\pgfsetlinewidth{1.003750pt}%
\definecolor{currentstroke}{rgb}{1.000000,0.498039,0.054902}%
\pgfsetstrokecolor{currentstroke}%
\pgfsetdash{}{0pt}%
\pgfpathmoveto{\pgfqpoint{1.497455in}{2.923368in}}%
\pgfpathcurveto{\pgfqpoint{1.508505in}{2.923368in}}{\pgfqpoint{1.519104in}{2.927758in}}{\pgfqpoint{1.526918in}{2.935572in}}%
\pgfpathcurveto{\pgfqpoint{1.534731in}{2.943385in}}{\pgfqpoint{1.539122in}{2.953984in}}{\pgfqpoint{1.539122in}{2.965034in}}%
\pgfpathcurveto{\pgfqpoint{1.539122in}{2.976084in}}{\pgfqpoint{1.534731in}{2.986683in}}{\pgfqpoint{1.526918in}{2.994497in}}%
\pgfpathcurveto{\pgfqpoint{1.519104in}{3.002311in}}{\pgfqpoint{1.508505in}{3.006701in}}{\pgfqpoint{1.497455in}{3.006701in}}%
\pgfpathcurveto{\pgfqpoint{1.486405in}{3.006701in}}{\pgfqpoint{1.475806in}{3.002311in}}{\pgfqpoint{1.467992in}{2.994497in}}%
\pgfpathcurveto{\pgfqpoint{1.460179in}{2.986683in}}{\pgfqpoint{1.455788in}{2.976084in}}{\pgfqpoint{1.455788in}{2.965034in}}%
\pgfpathcurveto{\pgfqpoint{1.455788in}{2.953984in}}{\pgfqpoint{1.460179in}{2.943385in}}{\pgfqpoint{1.467992in}{2.935572in}}%
\pgfpathcurveto{\pgfqpoint{1.475806in}{2.927758in}}{\pgfqpoint{1.486405in}{2.923368in}}{\pgfqpoint{1.497455in}{2.923368in}}%
\pgfpathclose%
\pgfusepath{stroke,fill}%
\end{pgfscope}%
\begin{pgfscope}%
\pgfpathrectangle{\pgfqpoint{0.787074in}{0.548769in}}{\pgfqpoint{5.062926in}{3.102590in}}%
\pgfusepath{clip}%
\pgfsetbuttcap%
\pgfsetroundjoin%
\definecolor{currentfill}{rgb}{1.000000,0.498039,0.054902}%
\pgfsetfillcolor{currentfill}%
\pgfsetlinewidth{1.003750pt}%
\definecolor{currentstroke}{rgb}{1.000000,0.498039,0.054902}%
\pgfsetstrokecolor{currentstroke}%
\pgfsetdash{}{0pt}%
\pgfpathmoveto{\pgfqpoint{1.603570in}{2.279400in}}%
\pgfpathcurveto{\pgfqpoint{1.614620in}{2.279400in}}{\pgfqpoint{1.625219in}{2.283790in}}{\pgfqpoint{1.633033in}{2.291603in}}%
\pgfpathcurveto{\pgfqpoint{1.640847in}{2.299417in}}{\pgfqpoint{1.645237in}{2.310016in}}{\pgfqpoint{1.645237in}{2.321066in}}%
\pgfpathcurveto{\pgfqpoint{1.645237in}{2.332116in}}{\pgfqpoint{1.640847in}{2.342715in}}{\pgfqpoint{1.633033in}{2.350529in}}%
\pgfpathcurveto{\pgfqpoint{1.625219in}{2.358343in}}{\pgfqpoint{1.614620in}{2.362733in}}{\pgfqpoint{1.603570in}{2.362733in}}%
\pgfpathcurveto{\pgfqpoint{1.592520in}{2.362733in}}{\pgfqpoint{1.581921in}{2.358343in}}{\pgfqpoint{1.574107in}{2.350529in}}%
\pgfpathcurveto{\pgfqpoint{1.566294in}{2.342715in}}{\pgfqpoint{1.561903in}{2.332116in}}{\pgfqpoint{1.561903in}{2.321066in}}%
\pgfpathcurveto{\pgfqpoint{1.561903in}{2.310016in}}{\pgfqpoint{1.566294in}{2.299417in}}{\pgfqpoint{1.574107in}{2.291603in}}%
\pgfpathcurveto{\pgfqpoint{1.581921in}{2.283790in}}{\pgfqpoint{1.592520in}{2.279400in}}{\pgfqpoint{1.603570in}{2.279400in}}%
\pgfpathclose%
\pgfusepath{stroke,fill}%
\end{pgfscope}%
\begin{pgfscope}%
\pgfpathrectangle{\pgfqpoint{0.787074in}{0.548769in}}{\pgfqpoint{5.062926in}{3.102590in}}%
\pgfusepath{clip}%
\pgfsetbuttcap%
\pgfsetroundjoin%
\definecolor{currentfill}{rgb}{1.000000,0.498039,0.054902}%
\pgfsetfillcolor{currentfill}%
\pgfsetlinewidth{1.003750pt}%
\definecolor{currentstroke}{rgb}{1.000000,0.498039,0.054902}%
\pgfsetstrokecolor{currentstroke}%
\pgfsetdash{}{0pt}%
\pgfpathmoveto{\pgfqpoint{1.486717in}{2.414739in}}%
\pgfpathcurveto{\pgfqpoint{1.497768in}{2.414739in}}{\pgfqpoint{1.508367in}{2.419129in}}{\pgfqpoint{1.516180in}{2.426942in}}%
\pgfpathcurveto{\pgfqpoint{1.523994in}{2.434756in}}{\pgfqpoint{1.528384in}{2.445355in}}{\pgfqpoint{1.528384in}{2.456405in}}%
\pgfpathcurveto{\pgfqpoint{1.528384in}{2.467455in}}{\pgfqpoint{1.523994in}{2.478054in}}{\pgfqpoint{1.516180in}{2.485868in}}%
\pgfpathcurveto{\pgfqpoint{1.508367in}{2.493682in}}{\pgfqpoint{1.497768in}{2.498072in}}{\pgfqpoint{1.486717in}{2.498072in}}%
\pgfpathcurveto{\pgfqpoint{1.475667in}{2.498072in}}{\pgfqpoint{1.465068in}{2.493682in}}{\pgfqpoint{1.457255in}{2.485868in}}%
\pgfpathcurveto{\pgfqpoint{1.449441in}{2.478054in}}{\pgfqpoint{1.445051in}{2.467455in}}{\pgfqpoint{1.445051in}{2.456405in}}%
\pgfpathcurveto{\pgfqpoint{1.445051in}{2.445355in}}{\pgfqpoint{1.449441in}{2.434756in}}{\pgfqpoint{1.457255in}{2.426942in}}%
\pgfpathcurveto{\pgfqpoint{1.465068in}{2.419129in}}{\pgfqpoint{1.475667in}{2.414739in}}{\pgfqpoint{1.486717in}{2.414739in}}%
\pgfpathclose%
\pgfusepath{stroke,fill}%
\end{pgfscope}%
\begin{pgfscope}%
\pgfpathrectangle{\pgfqpoint{0.787074in}{0.548769in}}{\pgfqpoint{5.062926in}{3.102590in}}%
\pgfusepath{clip}%
\pgfsetbuttcap%
\pgfsetroundjoin%
\definecolor{currentfill}{rgb}{1.000000,0.498039,0.054902}%
\pgfsetfillcolor{currentfill}%
\pgfsetlinewidth{1.003750pt}%
\definecolor{currentstroke}{rgb}{1.000000,0.498039,0.054902}%
\pgfsetstrokecolor{currentstroke}%
\pgfsetdash{}{0pt}%
\pgfpathmoveto{\pgfqpoint{1.070070in}{3.155405in}}%
\pgfpathcurveto{\pgfqpoint{1.081120in}{3.155405in}}{\pgfqpoint{1.091719in}{3.159795in}}{\pgfqpoint{1.099533in}{3.167609in}}%
\pgfpathcurveto{\pgfqpoint{1.107346in}{3.175422in}}{\pgfqpoint{1.111737in}{3.186021in}}{\pgfqpoint{1.111737in}{3.197072in}}%
\pgfpathcurveto{\pgfqpoint{1.111737in}{3.208122in}}{\pgfqpoint{1.107346in}{3.218721in}}{\pgfqpoint{1.099533in}{3.226534in}}%
\pgfpathcurveto{\pgfqpoint{1.091719in}{3.234348in}}{\pgfqpoint{1.081120in}{3.238738in}}{\pgfqpoint{1.070070in}{3.238738in}}%
\pgfpathcurveto{\pgfqpoint{1.059020in}{3.238738in}}{\pgfqpoint{1.048421in}{3.234348in}}{\pgfqpoint{1.040607in}{3.226534in}}%
\pgfpathcurveto{\pgfqpoint{1.032793in}{3.218721in}}{\pgfqpoint{1.028403in}{3.208122in}}{\pgfqpoint{1.028403in}{3.197072in}}%
\pgfpathcurveto{\pgfqpoint{1.028403in}{3.186021in}}{\pgfqpoint{1.032793in}{3.175422in}}{\pgfqpoint{1.040607in}{3.167609in}}%
\pgfpathcurveto{\pgfqpoint{1.048421in}{3.159795in}}{\pgfqpoint{1.059020in}{3.155405in}}{\pgfqpoint{1.070070in}{3.155405in}}%
\pgfpathclose%
\pgfusepath{stroke,fill}%
\end{pgfscope}%
\begin{pgfscope}%
\pgfpathrectangle{\pgfqpoint{0.787074in}{0.548769in}}{\pgfqpoint{5.062926in}{3.102590in}}%
\pgfusepath{clip}%
\pgfsetbuttcap%
\pgfsetroundjoin%
\definecolor{currentfill}{rgb}{1.000000,0.498039,0.054902}%
\pgfsetfillcolor{currentfill}%
\pgfsetlinewidth{1.003750pt}%
\definecolor{currentstroke}{rgb}{1.000000,0.498039,0.054902}%
\pgfsetstrokecolor{currentstroke}%
\pgfsetdash{}{0pt}%
\pgfpathmoveto{\pgfqpoint{1.986152in}{2.889472in}}%
\pgfpathcurveto{\pgfqpoint{1.997202in}{2.889472in}}{\pgfqpoint{2.007801in}{2.893862in}}{\pgfqpoint{2.015615in}{2.901676in}}%
\pgfpathcurveto{\pgfqpoint{2.023428in}{2.909489in}}{\pgfqpoint{2.027819in}{2.920088in}}{\pgfqpoint{2.027819in}{2.931138in}}%
\pgfpathcurveto{\pgfqpoint{2.027819in}{2.942189in}}{\pgfqpoint{2.023428in}{2.952788in}}{\pgfqpoint{2.015615in}{2.960601in}}%
\pgfpathcurveto{\pgfqpoint{2.007801in}{2.968415in}}{\pgfqpoint{1.997202in}{2.972805in}}{\pgfqpoint{1.986152in}{2.972805in}}%
\pgfpathcurveto{\pgfqpoint{1.975102in}{2.972805in}}{\pgfqpoint{1.964503in}{2.968415in}}{\pgfqpoint{1.956689in}{2.960601in}}%
\pgfpathcurveto{\pgfqpoint{1.948876in}{2.952788in}}{\pgfqpoint{1.944485in}{2.942189in}}{\pgfqpoint{1.944485in}{2.931138in}}%
\pgfpathcurveto{\pgfqpoint{1.944485in}{2.920088in}}{\pgfqpoint{1.948876in}{2.909489in}}{\pgfqpoint{1.956689in}{2.901676in}}%
\pgfpathcurveto{\pgfqpoint{1.964503in}{2.893862in}}{\pgfqpoint{1.975102in}{2.889472in}}{\pgfqpoint{1.986152in}{2.889472in}}%
\pgfpathclose%
\pgfusepath{stroke,fill}%
\end{pgfscope}%
\begin{pgfscope}%
\pgfpathrectangle{\pgfqpoint{0.787074in}{0.548769in}}{\pgfqpoint{5.062926in}{3.102590in}}%
\pgfusepath{clip}%
\pgfsetbuttcap%
\pgfsetroundjoin%
\definecolor{currentfill}{rgb}{0.121569,0.466667,0.705882}%
\pgfsetfillcolor{currentfill}%
\pgfsetlinewidth{1.003750pt}%
\definecolor{currentstroke}{rgb}{0.121569,0.466667,0.705882}%
\pgfsetstrokecolor{currentstroke}%
\pgfsetdash{}{0pt}%
\pgfpathmoveto{\pgfqpoint{1.460635in}{0.648143in}}%
\pgfpathcurveto{\pgfqpoint{1.471685in}{0.648143in}}{\pgfqpoint{1.482284in}{0.652534in}}{\pgfqpoint{1.490097in}{0.660347in}}%
\pgfpathcurveto{\pgfqpoint{1.497911in}{0.668161in}}{\pgfqpoint{1.502301in}{0.678760in}}{\pgfqpoint{1.502301in}{0.689810in}}%
\pgfpathcurveto{\pgfqpoint{1.502301in}{0.700860in}}{\pgfqpoint{1.497911in}{0.711459in}}{\pgfqpoint{1.490097in}{0.719273in}}%
\pgfpathcurveto{\pgfqpoint{1.482284in}{0.727086in}}{\pgfqpoint{1.471685in}{0.731477in}}{\pgfqpoint{1.460635in}{0.731477in}}%
\pgfpathcurveto{\pgfqpoint{1.449584in}{0.731477in}}{\pgfqpoint{1.438985in}{0.727086in}}{\pgfqpoint{1.431172in}{0.719273in}}%
\pgfpathcurveto{\pgfqpoint{1.423358in}{0.711459in}}{\pgfqpoint{1.418968in}{0.700860in}}{\pgfqpoint{1.418968in}{0.689810in}}%
\pgfpathcurveto{\pgfqpoint{1.418968in}{0.678760in}}{\pgfqpoint{1.423358in}{0.668161in}}{\pgfqpoint{1.431172in}{0.660347in}}%
\pgfpathcurveto{\pgfqpoint{1.438985in}{0.652534in}}{\pgfqpoint{1.449584in}{0.648143in}}{\pgfqpoint{1.460635in}{0.648143in}}%
\pgfpathclose%
\pgfusepath{stroke,fill}%
\end{pgfscope}%
\begin{pgfscope}%
\pgfpathrectangle{\pgfqpoint{0.787074in}{0.548769in}}{\pgfqpoint{5.062926in}{3.102590in}}%
\pgfusepath{clip}%
\pgfsetbuttcap%
\pgfsetroundjoin%
\definecolor{currentfill}{rgb}{0.121569,0.466667,0.705882}%
\pgfsetfillcolor{currentfill}%
\pgfsetlinewidth{1.003750pt}%
\definecolor{currentstroke}{rgb}{0.121569,0.466667,0.705882}%
\pgfsetstrokecolor{currentstroke}%
\pgfsetdash{}{0pt}%
\pgfpathmoveto{\pgfqpoint{1.692190in}{0.648148in}}%
\pgfpathcurveto{\pgfqpoint{1.703240in}{0.648148in}}{\pgfqpoint{1.713839in}{0.652538in}}{\pgfqpoint{1.721653in}{0.660352in}}%
\pgfpathcurveto{\pgfqpoint{1.729467in}{0.668165in}}{\pgfqpoint{1.733857in}{0.678764in}}{\pgfqpoint{1.733857in}{0.689814in}}%
\pgfpathcurveto{\pgfqpoint{1.733857in}{0.700865in}}{\pgfqpoint{1.729467in}{0.711464in}}{\pgfqpoint{1.721653in}{0.719277in}}%
\pgfpathcurveto{\pgfqpoint{1.713839in}{0.727091in}}{\pgfqpoint{1.703240in}{0.731481in}}{\pgfqpoint{1.692190in}{0.731481in}}%
\pgfpathcurveto{\pgfqpoint{1.681140in}{0.731481in}}{\pgfqpoint{1.670541in}{0.727091in}}{\pgfqpoint{1.662727in}{0.719277in}}%
\pgfpathcurveto{\pgfqpoint{1.654914in}{0.711464in}}{\pgfqpoint{1.650524in}{0.700865in}}{\pgfqpoint{1.650524in}{0.689814in}}%
\pgfpathcurveto{\pgfqpoint{1.650524in}{0.678764in}}{\pgfqpoint{1.654914in}{0.668165in}}{\pgfqpoint{1.662727in}{0.660352in}}%
\pgfpathcurveto{\pgfqpoint{1.670541in}{0.652538in}}{\pgfqpoint{1.681140in}{0.648148in}}{\pgfqpoint{1.692190in}{0.648148in}}%
\pgfpathclose%
\pgfusepath{stroke,fill}%
\end{pgfscope}%
\begin{pgfscope}%
\pgfpathrectangle{\pgfqpoint{0.787074in}{0.548769in}}{\pgfqpoint{5.062926in}{3.102590in}}%
\pgfusepath{clip}%
\pgfsetbuttcap%
\pgfsetroundjoin%
\definecolor{currentfill}{rgb}{0.121569,0.466667,0.705882}%
\pgfsetfillcolor{currentfill}%
\pgfsetlinewidth{1.003750pt}%
\definecolor{currentstroke}{rgb}{0.121569,0.466667,0.705882}%
\pgfsetstrokecolor{currentstroke}%
\pgfsetdash{}{0pt}%
\pgfpathmoveto{\pgfqpoint{1.222341in}{0.648151in}}%
\pgfpathcurveto{\pgfqpoint{1.233391in}{0.648151in}}{\pgfqpoint{1.243990in}{0.652542in}}{\pgfqpoint{1.251803in}{0.660355in}}%
\pgfpathcurveto{\pgfqpoint{1.259617in}{0.668169in}}{\pgfqpoint{1.264007in}{0.678768in}}{\pgfqpoint{1.264007in}{0.689818in}}%
\pgfpathcurveto{\pgfqpoint{1.264007in}{0.700868in}}{\pgfqpoint{1.259617in}{0.711467in}}{\pgfqpoint{1.251803in}{0.719281in}}%
\pgfpathcurveto{\pgfqpoint{1.243990in}{0.727094in}}{\pgfqpoint{1.233391in}{0.731485in}}{\pgfqpoint{1.222341in}{0.731485in}}%
\pgfpathcurveto{\pgfqpoint{1.211290in}{0.731485in}}{\pgfqpoint{1.200691in}{0.727094in}}{\pgfqpoint{1.192878in}{0.719281in}}%
\pgfpathcurveto{\pgfqpoint{1.185064in}{0.711467in}}{\pgfqpoint{1.180674in}{0.700868in}}{\pgfqpoint{1.180674in}{0.689818in}}%
\pgfpathcurveto{\pgfqpoint{1.180674in}{0.678768in}}{\pgfqpoint{1.185064in}{0.668169in}}{\pgfqpoint{1.192878in}{0.660355in}}%
\pgfpathcurveto{\pgfqpoint{1.200691in}{0.652542in}}{\pgfqpoint{1.211290in}{0.648151in}}{\pgfqpoint{1.222341in}{0.648151in}}%
\pgfpathclose%
\pgfusepath{stroke,fill}%
\end{pgfscope}%
\begin{pgfscope}%
\pgfpathrectangle{\pgfqpoint{0.787074in}{0.548769in}}{\pgfqpoint{5.062926in}{3.102590in}}%
\pgfusepath{clip}%
\pgfsetbuttcap%
\pgfsetroundjoin%
\definecolor{currentfill}{rgb}{1.000000,0.498039,0.054902}%
\pgfsetfillcolor{currentfill}%
\pgfsetlinewidth{1.003750pt}%
\definecolor{currentstroke}{rgb}{1.000000,0.498039,0.054902}%
\pgfsetstrokecolor{currentstroke}%
\pgfsetdash{}{0pt}%
\pgfpathmoveto{\pgfqpoint{1.491915in}{2.770182in}}%
\pgfpathcurveto{\pgfqpoint{1.502965in}{2.770182in}}{\pgfqpoint{1.513564in}{2.774572in}}{\pgfqpoint{1.521378in}{2.782386in}}%
\pgfpathcurveto{\pgfqpoint{1.529191in}{2.790199in}}{\pgfqpoint{1.533581in}{2.800798in}}{\pgfqpoint{1.533581in}{2.811848in}}%
\pgfpathcurveto{\pgfqpoint{1.533581in}{2.822898in}}{\pgfqpoint{1.529191in}{2.833497in}}{\pgfqpoint{1.521378in}{2.841311in}}%
\pgfpathcurveto{\pgfqpoint{1.513564in}{2.849125in}}{\pgfqpoint{1.502965in}{2.853515in}}{\pgfqpoint{1.491915in}{2.853515in}}%
\pgfpathcurveto{\pgfqpoint{1.480865in}{2.853515in}}{\pgfqpoint{1.470266in}{2.849125in}}{\pgfqpoint{1.462452in}{2.841311in}}%
\pgfpathcurveto{\pgfqpoint{1.454638in}{2.833497in}}{\pgfqpoint{1.450248in}{2.822898in}}{\pgfqpoint{1.450248in}{2.811848in}}%
\pgfpathcurveto{\pgfqpoint{1.450248in}{2.800798in}}{\pgfqpoint{1.454638in}{2.790199in}}{\pgfqpoint{1.462452in}{2.782386in}}%
\pgfpathcurveto{\pgfqpoint{1.470266in}{2.774572in}}{\pgfqpoint{1.480865in}{2.770182in}}{\pgfqpoint{1.491915in}{2.770182in}}%
\pgfpathclose%
\pgfusepath{stroke,fill}%
\end{pgfscope}%
\begin{pgfscope}%
\pgfpathrectangle{\pgfqpoint{0.787074in}{0.548769in}}{\pgfqpoint{5.062926in}{3.102590in}}%
\pgfusepath{clip}%
\pgfsetbuttcap%
\pgfsetroundjoin%
\definecolor{currentfill}{rgb}{1.000000,0.498039,0.054902}%
\pgfsetfillcolor{currentfill}%
\pgfsetlinewidth{1.003750pt}%
\definecolor{currentstroke}{rgb}{1.000000,0.498039,0.054902}%
\pgfsetstrokecolor{currentstroke}%
\pgfsetdash{}{0pt}%
\pgfpathmoveto{\pgfqpoint{1.396487in}{3.154057in}}%
\pgfpathcurveto{\pgfqpoint{1.407537in}{3.154057in}}{\pgfqpoint{1.418136in}{3.158447in}}{\pgfqpoint{1.425950in}{3.166261in}}%
\pgfpathcurveto{\pgfqpoint{1.433763in}{3.174075in}}{\pgfqpoint{1.438154in}{3.184674in}}{\pgfqpoint{1.438154in}{3.195724in}}%
\pgfpathcurveto{\pgfqpoint{1.438154in}{3.206774in}}{\pgfqpoint{1.433763in}{3.217373in}}{\pgfqpoint{1.425950in}{3.225187in}}%
\pgfpathcurveto{\pgfqpoint{1.418136in}{3.233000in}}{\pgfqpoint{1.407537in}{3.237391in}}{\pgfqpoint{1.396487in}{3.237391in}}%
\pgfpathcurveto{\pgfqpoint{1.385437in}{3.237391in}}{\pgfqpoint{1.374838in}{3.233000in}}{\pgfqpoint{1.367024in}{3.225187in}}%
\pgfpathcurveto{\pgfqpoint{1.359211in}{3.217373in}}{\pgfqpoint{1.354820in}{3.206774in}}{\pgfqpoint{1.354820in}{3.195724in}}%
\pgfpathcurveto{\pgfqpoint{1.354820in}{3.184674in}}{\pgfqpoint{1.359211in}{3.174075in}}{\pgfqpoint{1.367024in}{3.166261in}}%
\pgfpathcurveto{\pgfqpoint{1.374838in}{3.158447in}}{\pgfqpoint{1.385437in}{3.154057in}}{\pgfqpoint{1.396487in}{3.154057in}}%
\pgfpathclose%
\pgfusepath{stroke,fill}%
\end{pgfscope}%
\begin{pgfscope}%
\pgfpathrectangle{\pgfqpoint{0.787074in}{0.548769in}}{\pgfqpoint{5.062926in}{3.102590in}}%
\pgfusepath{clip}%
\pgfsetbuttcap%
\pgfsetroundjoin%
\definecolor{currentfill}{rgb}{1.000000,0.498039,0.054902}%
\pgfsetfillcolor{currentfill}%
\pgfsetlinewidth{1.003750pt}%
\definecolor{currentstroke}{rgb}{1.000000,0.498039,0.054902}%
\pgfsetstrokecolor{currentstroke}%
\pgfsetdash{}{0pt}%
\pgfpathmoveto{\pgfqpoint{1.615232in}{2.232087in}}%
\pgfpathcurveto{\pgfqpoint{1.626282in}{2.232087in}}{\pgfqpoint{1.636881in}{2.236477in}}{\pgfqpoint{1.644695in}{2.244291in}}%
\pgfpathcurveto{\pgfqpoint{1.652509in}{2.252104in}}{\pgfqpoint{1.656899in}{2.262703in}}{\pgfqpoint{1.656899in}{2.273753in}}%
\pgfpathcurveto{\pgfqpoint{1.656899in}{2.284804in}}{\pgfqpoint{1.652509in}{2.295403in}}{\pgfqpoint{1.644695in}{2.303216in}}%
\pgfpathcurveto{\pgfqpoint{1.636881in}{2.311030in}}{\pgfqpoint{1.626282in}{2.315420in}}{\pgfqpoint{1.615232in}{2.315420in}}%
\pgfpathcurveto{\pgfqpoint{1.604182in}{2.315420in}}{\pgfqpoint{1.593583in}{2.311030in}}{\pgfqpoint{1.585770in}{2.303216in}}%
\pgfpathcurveto{\pgfqpoint{1.577956in}{2.295403in}}{\pgfqpoint{1.573566in}{2.284804in}}{\pgfqpoint{1.573566in}{2.273753in}}%
\pgfpathcurveto{\pgfqpoint{1.573566in}{2.262703in}}{\pgfqpoint{1.577956in}{2.252104in}}{\pgfqpoint{1.585770in}{2.244291in}}%
\pgfpathcurveto{\pgfqpoint{1.593583in}{2.236477in}}{\pgfqpoint{1.604182in}{2.232087in}}{\pgfqpoint{1.615232in}{2.232087in}}%
\pgfpathclose%
\pgfusepath{stroke,fill}%
\end{pgfscope}%
\begin{pgfscope}%
\pgfpathrectangle{\pgfqpoint{0.787074in}{0.548769in}}{\pgfqpoint{5.062926in}{3.102590in}}%
\pgfusepath{clip}%
\pgfsetbuttcap%
\pgfsetroundjoin%
\definecolor{currentfill}{rgb}{0.121569,0.466667,0.705882}%
\pgfsetfillcolor{currentfill}%
\pgfsetlinewidth{1.003750pt}%
\definecolor{currentstroke}{rgb}{0.121569,0.466667,0.705882}%
\pgfsetstrokecolor{currentstroke}%
\pgfsetdash{}{0pt}%
\pgfpathmoveto{\pgfqpoint{1.017207in}{0.648132in}}%
\pgfpathcurveto{\pgfqpoint{1.028257in}{0.648132in}}{\pgfqpoint{1.038856in}{0.652523in}}{\pgfqpoint{1.046670in}{0.660336in}}%
\pgfpathcurveto{\pgfqpoint{1.054483in}{0.668150in}}{\pgfqpoint{1.058874in}{0.678749in}}{\pgfqpoint{1.058874in}{0.689799in}}%
\pgfpathcurveto{\pgfqpoint{1.058874in}{0.700849in}}{\pgfqpoint{1.054483in}{0.711448in}}{\pgfqpoint{1.046670in}{0.719262in}}%
\pgfpathcurveto{\pgfqpoint{1.038856in}{0.727075in}}{\pgfqpoint{1.028257in}{0.731466in}}{\pgfqpoint{1.017207in}{0.731466in}}%
\pgfpathcurveto{\pgfqpoint{1.006157in}{0.731466in}}{\pgfqpoint{0.995558in}{0.727075in}}{\pgfqpoint{0.987744in}{0.719262in}}%
\pgfpathcurveto{\pgfqpoint{0.979930in}{0.711448in}}{\pgfqpoint{0.975540in}{0.700849in}}{\pgfqpoint{0.975540in}{0.689799in}}%
\pgfpathcurveto{\pgfqpoint{0.975540in}{0.678749in}}{\pgfqpoint{0.979930in}{0.668150in}}{\pgfqpoint{0.987744in}{0.660336in}}%
\pgfpathcurveto{\pgfqpoint{0.995558in}{0.652523in}}{\pgfqpoint{1.006157in}{0.648132in}}{\pgfqpoint{1.017207in}{0.648132in}}%
\pgfpathclose%
\pgfusepath{stroke,fill}%
\end{pgfscope}%
\begin{pgfscope}%
\pgfpathrectangle{\pgfqpoint{0.787074in}{0.548769in}}{\pgfqpoint{5.062926in}{3.102590in}}%
\pgfusepath{clip}%
\pgfsetbuttcap%
\pgfsetroundjoin%
\definecolor{currentfill}{rgb}{0.121569,0.466667,0.705882}%
\pgfsetfillcolor{currentfill}%
\pgfsetlinewidth{1.003750pt}%
\definecolor{currentstroke}{rgb}{0.121569,0.466667,0.705882}%
\pgfsetstrokecolor{currentstroke}%
\pgfsetdash{}{0pt}%
\pgfpathmoveto{\pgfqpoint{1.061082in}{0.784785in}}%
\pgfpathcurveto{\pgfqpoint{1.072132in}{0.784785in}}{\pgfqpoint{1.082731in}{0.789175in}}{\pgfqpoint{1.090544in}{0.796989in}}%
\pgfpathcurveto{\pgfqpoint{1.098358in}{0.804802in}}{\pgfqpoint{1.102748in}{0.815401in}}{\pgfqpoint{1.102748in}{0.826451in}}%
\pgfpathcurveto{\pgfqpoint{1.102748in}{0.837502in}}{\pgfqpoint{1.098358in}{0.848101in}}{\pgfqpoint{1.090544in}{0.855914in}}%
\pgfpathcurveto{\pgfqpoint{1.082731in}{0.863728in}}{\pgfqpoint{1.072132in}{0.868118in}}{\pgfqpoint{1.061082in}{0.868118in}}%
\pgfpathcurveto{\pgfqpoint{1.050031in}{0.868118in}}{\pgfqpoint{1.039432in}{0.863728in}}{\pgfqpoint{1.031619in}{0.855914in}}%
\pgfpathcurveto{\pgfqpoint{1.023805in}{0.848101in}}{\pgfqpoint{1.019415in}{0.837502in}}{\pgfqpoint{1.019415in}{0.826451in}}%
\pgfpathcurveto{\pgfqpoint{1.019415in}{0.815401in}}{\pgfqpoint{1.023805in}{0.804802in}}{\pgfqpoint{1.031619in}{0.796989in}}%
\pgfpathcurveto{\pgfqpoint{1.039432in}{0.789175in}}{\pgfqpoint{1.050031in}{0.784785in}}{\pgfqpoint{1.061082in}{0.784785in}}%
\pgfpathclose%
\pgfusepath{stroke,fill}%
\end{pgfscope}%
\begin{pgfscope}%
\pgfpathrectangle{\pgfqpoint{0.787074in}{0.548769in}}{\pgfqpoint{5.062926in}{3.102590in}}%
\pgfusepath{clip}%
\pgfsetbuttcap%
\pgfsetroundjoin%
\definecolor{currentfill}{rgb}{0.121569,0.466667,0.705882}%
\pgfsetfillcolor{currentfill}%
\pgfsetlinewidth{1.003750pt}%
\definecolor{currentstroke}{rgb}{0.121569,0.466667,0.705882}%
\pgfsetstrokecolor{currentstroke}%
\pgfsetdash{}{0pt}%
\pgfpathmoveto{\pgfqpoint{4.834234in}{2.639480in}}%
\pgfpathcurveto{\pgfqpoint{4.845284in}{2.639480in}}{\pgfqpoint{4.855883in}{2.643870in}}{\pgfqpoint{4.863697in}{2.651684in}}%
\pgfpathcurveto{\pgfqpoint{4.871510in}{2.659498in}}{\pgfqpoint{4.875900in}{2.670097in}}{\pgfqpoint{4.875900in}{2.681147in}}%
\pgfpathcurveto{\pgfqpoint{4.875900in}{2.692197in}}{\pgfqpoint{4.871510in}{2.702796in}}{\pgfqpoint{4.863697in}{2.710610in}}%
\pgfpathcurveto{\pgfqpoint{4.855883in}{2.718423in}}{\pgfqpoint{4.845284in}{2.722813in}}{\pgfqpoint{4.834234in}{2.722813in}}%
\pgfpathcurveto{\pgfqpoint{4.823184in}{2.722813in}}{\pgfqpoint{4.812585in}{2.718423in}}{\pgfqpoint{4.804771in}{2.710610in}}%
\pgfpathcurveto{\pgfqpoint{4.796957in}{2.702796in}}{\pgfqpoint{4.792567in}{2.692197in}}{\pgfqpoint{4.792567in}{2.681147in}}%
\pgfpathcurveto{\pgfqpoint{4.792567in}{2.670097in}}{\pgfqpoint{4.796957in}{2.659498in}}{\pgfqpoint{4.804771in}{2.651684in}}%
\pgfpathcurveto{\pgfqpoint{4.812585in}{2.643870in}}{\pgfqpoint{4.823184in}{2.639480in}}{\pgfqpoint{4.834234in}{2.639480in}}%
\pgfpathclose%
\pgfusepath{stroke,fill}%
\end{pgfscope}%
\begin{pgfscope}%
\pgfpathrectangle{\pgfqpoint{0.787074in}{0.548769in}}{\pgfqpoint{5.062926in}{3.102590in}}%
\pgfusepath{clip}%
\pgfsetbuttcap%
\pgfsetroundjoin%
\definecolor{currentfill}{rgb}{0.121569,0.466667,0.705882}%
\pgfsetfillcolor{currentfill}%
\pgfsetlinewidth{1.003750pt}%
\definecolor{currentstroke}{rgb}{0.121569,0.466667,0.705882}%
\pgfsetstrokecolor{currentstroke}%
\pgfsetdash{}{0pt}%
\pgfpathmoveto{\pgfqpoint{1.498638in}{0.648148in}}%
\pgfpathcurveto{\pgfqpoint{1.509688in}{0.648148in}}{\pgfqpoint{1.520287in}{0.652539in}}{\pgfqpoint{1.528101in}{0.660352in}}%
\pgfpathcurveto{\pgfqpoint{1.535914in}{0.668166in}}{\pgfqpoint{1.540304in}{0.678765in}}{\pgfqpoint{1.540304in}{0.689815in}}%
\pgfpathcurveto{\pgfqpoint{1.540304in}{0.700865in}}{\pgfqpoint{1.535914in}{0.711464in}}{\pgfqpoint{1.528101in}{0.719278in}}%
\pgfpathcurveto{\pgfqpoint{1.520287in}{0.727091in}}{\pgfqpoint{1.509688in}{0.731482in}}{\pgfqpoint{1.498638in}{0.731482in}}%
\pgfpathcurveto{\pgfqpoint{1.487588in}{0.731482in}}{\pgfqpoint{1.476989in}{0.727091in}}{\pgfqpoint{1.469175in}{0.719278in}}%
\pgfpathcurveto{\pgfqpoint{1.461361in}{0.711464in}}{\pgfqpoint{1.456971in}{0.700865in}}{\pgfqpoint{1.456971in}{0.689815in}}%
\pgfpathcurveto{\pgfqpoint{1.456971in}{0.678765in}}{\pgfqpoint{1.461361in}{0.668166in}}{\pgfqpoint{1.469175in}{0.660352in}}%
\pgfpathcurveto{\pgfqpoint{1.476989in}{0.652539in}}{\pgfqpoint{1.487588in}{0.648148in}}{\pgfqpoint{1.498638in}{0.648148in}}%
\pgfpathclose%
\pgfusepath{stroke,fill}%
\end{pgfscope}%
\begin{pgfscope}%
\pgfpathrectangle{\pgfqpoint{0.787074in}{0.548769in}}{\pgfqpoint{5.062926in}{3.102590in}}%
\pgfusepath{clip}%
\pgfsetbuttcap%
\pgfsetroundjoin%
\definecolor{currentfill}{rgb}{0.121569,0.466667,0.705882}%
\pgfsetfillcolor{currentfill}%
\pgfsetlinewidth{1.003750pt}%
\definecolor{currentstroke}{rgb}{0.121569,0.466667,0.705882}%
\pgfsetstrokecolor{currentstroke}%
\pgfsetdash{}{0pt}%
\pgfpathmoveto{\pgfqpoint{1.096184in}{0.648149in}}%
\pgfpathcurveto{\pgfqpoint{1.107234in}{0.648149in}}{\pgfqpoint{1.117833in}{0.652540in}}{\pgfqpoint{1.125646in}{0.660353in}}%
\pgfpathcurveto{\pgfqpoint{1.133460in}{0.668167in}}{\pgfqpoint{1.137850in}{0.678766in}}{\pgfqpoint{1.137850in}{0.689816in}}%
\pgfpathcurveto{\pgfqpoint{1.137850in}{0.700866in}}{\pgfqpoint{1.133460in}{0.711465in}}{\pgfqpoint{1.125646in}{0.719279in}}%
\pgfpathcurveto{\pgfqpoint{1.117833in}{0.727093in}}{\pgfqpoint{1.107234in}{0.731483in}}{\pgfqpoint{1.096184in}{0.731483in}}%
\pgfpathcurveto{\pgfqpoint{1.085133in}{0.731483in}}{\pgfqpoint{1.074534in}{0.727093in}}{\pgfqpoint{1.066721in}{0.719279in}}%
\pgfpathcurveto{\pgfqpoint{1.058907in}{0.711465in}}{\pgfqpoint{1.054517in}{0.700866in}}{\pgfqpoint{1.054517in}{0.689816in}}%
\pgfpathcurveto{\pgfqpoint{1.054517in}{0.678766in}}{\pgfqpoint{1.058907in}{0.668167in}}{\pgfqpoint{1.066721in}{0.660353in}}%
\pgfpathcurveto{\pgfqpoint{1.074534in}{0.652540in}}{\pgfqpoint{1.085133in}{0.648149in}}{\pgfqpoint{1.096184in}{0.648149in}}%
\pgfpathclose%
\pgfusepath{stroke,fill}%
\end{pgfscope}%
\begin{pgfscope}%
\pgfpathrectangle{\pgfqpoint{0.787074in}{0.548769in}}{\pgfqpoint{5.062926in}{3.102590in}}%
\pgfusepath{clip}%
\pgfsetbuttcap%
\pgfsetroundjoin%
\definecolor{currentfill}{rgb}{0.121569,0.466667,0.705882}%
\pgfsetfillcolor{currentfill}%
\pgfsetlinewidth{1.003750pt}%
\definecolor{currentstroke}{rgb}{0.121569,0.466667,0.705882}%
\pgfsetstrokecolor{currentstroke}%
\pgfsetdash{}{0pt}%
\pgfpathmoveto{\pgfqpoint{1.023884in}{2.298244in}}%
\pgfpathcurveto{\pgfqpoint{1.034934in}{2.298244in}}{\pgfqpoint{1.045533in}{2.302634in}}{\pgfqpoint{1.053346in}{2.310448in}}%
\pgfpathcurveto{\pgfqpoint{1.061160in}{2.318261in}}{\pgfqpoint{1.065550in}{2.328860in}}{\pgfqpoint{1.065550in}{2.339911in}}%
\pgfpathcurveto{\pgfqpoint{1.065550in}{2.350961in}}{\pgfqpoint{1.061160in}{2.361560in}}{\pgfqpoint{1.053346in}{2.369373in}}%
\pgfpathcurveto{\pgfqpoint{1.045533in}{2.377187in}}{\pgfqpoint{1.034934in}{2.381577in}}{\pgfqpoint{1.023884in}{2.381577in}}%
\pgfpathcurveto{\pgfqpoint{1.012833in}{2.381577in}}{\pgfqpoint{1.002234in}{2.377187in}}{\pgfqpoint{0.994421in}{2.369373in}}%
\pgfpathcurveto{\pgfqpoint{0.986607in}{2.361560in}}{\pgfqpoint{0.982217in}{2.350961in}}{\pgfqpoint{0.982217in}{2.339911in}}%
\pgfpathcurveto{\pgfqpoint{0.982217in}{2.328860in}}{\pgfqpoint{0.986607in}{2.318261in}}{\pgfqpoint{0.994421in}{2.310448in}}%
\pgfpathcurveto{\pgfqpoint{1.002234in}{2.302634in}}{\pgfqpoint{1.012833in}{2.298244in}}{\pgfqpoint{1.023884in}{2.298244in}}%
\pgfpathclose%
\pgfusepath{stroke,fill}%
\end{pgfscope}%
\begin{pgfscope}%
\pgfpathrectangle{\pgfqpoint{0.787074in}{0.548769in}}{\pgfqpoint{5.062926in}{3.102590in}}%
\pgfusepath{clip}%
\pgfsetbuttcap%
\pgfsetroundjoin%
\definecolor{currentfill}{rgb}{1.000000,0.498039,0.054902}%
\pgfsetfillcolor{currentfill}%
\pgfsetlinewidth{1.003750pt}%
\definecolor{currentstroke}{rgb}{1.000000,0.498039,0.054902}%
\pgfsetstrokecolor{currentstroke}%
\pgfsetdash{}{0pt}%
\pgfpathmoveto{\pgfqpoint{1.241635in}{2.885396in}}%
\pgfpathcurveto{\pgfqpoint{1.252685in}{2.885396in}}{\pgfqpoint{1.263284in}{2.889786in}}{\pgfqpoint{1.271098in}{2.897600in}}%
\pgfpathcurveto{\pgfqpoint{1.278911in}{2.905413in}}{\pgfqpoint{1.283302in}{2.916012in}}{\pgfqpoint{1.283302in}{2.927063in}}%
\pgfpathcurveto{\pgfqpoint{1.283302in}{2.938113in}}{\pgfqpoint{1.278911in}{2.948712in}}{\pgfqpoint{1.271098in}{2.956525in}}%
\pgfpathcurveto{\pgfqpoint{1.263284in}{2.964339in}}{\pgfqpoint{1.252685in}{2.968729in}}{\pgfqpoint{1.241635in}{2.968729in}}%
\pgfpathcurveto{\pgfqpoint{1.230585in}{2.968729in}}{\pgfqpoint{1.219986in}{2.964339in}}{\pgfqpoint{1.212172in}{2.956525in}}%
\pgfpathcurveto{\pgfqpoint{1.204359in}{2.948712in}}{\pgfqpoint{1.199968in}{2.938113in}}{\pgfqpoint{1.199968in}{2.927063in}}%
\pgfpathcurveto{\pgfqpoint{1.199968in}{2.916012in}}{\pgfqpoint{1.204359in}{2.905413in}}{\pgfqpoint{1.212172in}{2.897600in}}%
\pgfpathcurveto{\pgfqpoint{1.219986in}{2.889786in}}{\pgfqpoint{1.230585in}{2.885396in}}{\pgfqpoint{1.241635in}{2.885396in}}%
\pgfpathclose%
\pgfusepath{stroke,fill}%
\end{pgfscope}%
\begin{pgfscope}%
\pgfpathrectangle{\pgfqpoint{0.787074in}{0.548769in}}{\pgfqpoint{5.062926in}{3.102590in}}%
\pgfusepath{clip}%
\pgfsetbuttcap%
\pgfsetroundjoin%
\definecolor{currentfill}{rgb}{1.000000,0.498039,0.054902}%
\pgfsetfillcolor{currentfill}%
\pgfsetlinewidth{1.003750pt}%
\definecolor{currentstroke}{rgb}{1.000000,0.498039,0.054902}%
\pgfsetstrokecolor{currentstroke}%
\pgfsetdash{}{0pt}%
\pgfpathmoveto{\pgfqpoint{2.161682in}{2.303844in}}%
\pgfpathcurveto{\pgfqpoint{2.172732in}{2.303844in}}{\pgfqpoint{2.183331in}{2.308235in}}{\pgfqpoint{2.191144in}{2.316048in}}%
\pgfpathcurveto{\pgfqpoint{2.198958in}{2.323862in}}{\pgfqpoint{2.203348in}{2.334461in}}{\pgfqpoint{2.203348in}{2.345511in}}%
\pgfpathcurveto{\pgfqpoint{2.203348in}{2.356561in}}{\pgfqpoint{2.198958in}{2.367160in}}{\pgfqpoint{2.191144in}{2.374974in}}%
\pgfpathcurveto{\pgfqpoint{2.183331in}{2.382787in}}{\pgfqpoint{2.172732in}{2.387178in}}{\pgfqpoint{2.161682in}{2.387178in}}%
\pgfpathcurveto{\pgfqpoint{2.150631in}{2.387178in}}{\pgfqpoint{2.140032in}{2.382787in}}{\pgfqpoint{2.132219in}{2.374974in}}%
\pgfpathcurveto{\pgfqpoint{2.124405in}{2.367160in}}{\pgfqpoint{2.120015in}{2.356561in}}{\pgfqpoint{2.120015in}{2.345511in}}%
\pgfpathcurveto{\pgfqpoint{2.120015in}{2.334461in}}{\pgfqpoint{2.124405in}{2.323862in}}{\pgfqpoint{2.132219in}{2.316048in}}%
\pgfpathcurveto{\pgfqpoint{2.140032in}{2.308235in}}{\pgfqpoint{2.150631in}{2.303844in}}{\pgfqpoint{2.161682in}{2.303844in}}%
\pgfpathclose%
\pgfusepath{stroke,fill}%
\end{pgfscope}%
\begin{pgfscope}%
\pgfpathrectangle{\pgfqpoint{0.787074in}{0.548769in}}{\pgfqpoint{5.062926in}{3.102590in}}%
\pgfusepath{clip}%
\pgfsetbuttcap%
\pgfsetroundjoin%
\definecolor{currentfill}{rgb}{1.000000,0.498039,0.054902}%
\pgfsetfillcolor{currentfill}%
\pgfsetlinewidth{1.003750pt}%
\definecolor{currentstroke}{rgb}{1.000000,0.498039,0.054902}%
\pgfsetstrokecolor{currentstroke}%
\pgfsetdash{}{0pt}%
\pgfpathmoveto{\pgfqpoint{1.835839in}{2.924913in}}%
\pgfpathcurveto{\pgfqpoint{1.846889in}{2.924913in}}{\pgfqpoint{1.857488in}{2.929304in}}{\pgfqpoint{1.865301in}{2.937117in}}%
\pgfpathcurveto{\pgfqpoint{1.873115in}{2.944931in}}{\pgfqpoint{1.877505in}{2.955530in}}{\pgfqpoint{1.877505in}{2.966580in}}%
\pgfpathcurveto{\pgfqpoint{1.877505in}{2.977630in}}{\pgfqpoint{1.873115in}{2.988229in}}{\pgfqpoint{1.865301in}{2.996043in}}%
\pgfpathcurveto{\pgfqpoint{1.857488in}{3.003856in}}{\pgfqpoint{1.846889in}{3.008247in}}{\pgfqpoint{1.835839in}{3.008247in}}%
\pgfpathcurveto{\pgfqpoint{1.824788in}{3.008247in}}{\pgfqpoint{1.814189in}{3.003856in}}{\pgfqpoint{1.806376in}{2.996043in}}%
\pgfpathcurveto{\pgfqpoint{1.798562in}{2.988229in}}{\pgfqpoint{1.794172in}{2.977630in}}{\pgfqpoint{1.794172in}{2.966580in}}%
\pgfpathcurveto{\pgfqpoint{1.794172in}{2.955530in}}{\pgfqpoint{1.798562in}{2.944931in}}{\pgfqpoint{1.806376in}{2.937117in}}%
\pgfpathcurveto{\pgfqpoint{1.814189in}{2.929304in}}{\pgfqpoint{1.824788in}{2.924913in}}{\pgfqpoint{1.835839in}{2.924913in}}%
\pgfpathclose%
\pgfusepath{stroke,fill}%
\end{pgfscope}%
\begin{pgfscope}%
\pgfpathrectangle{\pgfqpoint{0.787074in}{0.548769in}}{\pgfqpoint{5.062926in}{3.102590in}}%
\pgfusepath{clip}%
\pgfsetbuttcap%
\pgfsetroundjoin%
\definecolor{currentfill}{rgb}{1.000000,0.498039,0.054902}%
\pgfsetfillcolor{currentfill}%
\pgfsetlinewidth{1.003750pt}%
\definecolor{currentstroke}{rgb}{1.000000,0.498039,0.054902}%
\pgfsetstrokecolor{currentstroke}%
\pgfsetdash{}{0pt}%
\pgfpathmoveto{\pgfqpoint{1.455846in}{2.712046in}}%
\pgfpathcurveto{\pgfqpoint{1.466896in}{2.712046in}}{\pgfqpoint{1.477495in}{2.716436in}}{\pgfqpoint{1.485308in}{2.724249in}}%
\pgfpathcurveto{\pgfqpoint{1.493122in}{2.732063in}}{\pgfqpoint{1.497512in}{2.742662in}}{\pgfqpoint{1.497512in}{2.753712in}}%
\pgfpathcurveto{\pgfqpoint{1.497512in}{2.764762in}}{\pgfqpoint{1.493122in}{2.775361in}}{\pgfqpoint{1.485308in}{2.783175in}}%
\pgfpathcurveto{\pgfqpoint{1.477495in}{2.790989in}}{\pgfqpoint{1.466896in}{2.795379in}}{\pgfqpoint{1.455846in}{2.795379in}}%
\pgfpathcurveto{\pgfqpoint{1.444796in}{2.795379in}}{\pgfqpoint{1.434197in}{2.790989in}}{\pgfqpoint{1.426383in}{2.783175in}}%
\pgfpathcurveto{\pgfqpoint{1.418569in}{2.775361in}}{\pgfqpoint{1.414179in}{2.764762in}}{\pgfqpoint{1.414179in}{2.753712in}}%
\pgfpathcurveto{\pgfqpoint{1.414179in}{2.742662in}}{\pgfqpoint{1.418569in}{2.732063in}}{\pgfqpoint{1.426383in}{2.724249in}}%
\pgfpathcurveto{\pgfqpoint{1.434197in}{2.716436in}}{\pgfqpoint{1.444796in}{2.712046in}}{\pgfqpoint{1.455846in}{2.712046in}}%
\pgfpathclose%
\pgfusepath{stroke,fill}%
\end{pgfscope}%
\begin{pgfscope}%
\pgfpathrectangle{\pgfqpoint{0.787074in}{0.548769in}}{\pgfqpoint{5.062926in}{3.102590in}}%
\pgfusepath{clip}%
\pgfsetbuttcap%
\pgfsetroundjoin%
\definecolor{currentfill}{rgb}{0.121569,0.466667,0.705882}%
\pgfsetfillcolor{currentfill}%
\pgfsetlinewidth{1.003750pt}%
\definecolor{currentstroke}{rgb}{0.121569,0.466667,0.705882}%
\pgfsetstrokecolor{currentstroke}%
\pgfsetdash{}{0pt}%
\pgfpathmoveto{\pgfqpoint{1.906131in}{3.116121in}}%
\pgfpathcurveto{\pgfqpoint{1.917181in}{3.116121in}}{\pgfqpoint{1.927780in}{3.120511in}}{\pgfqpoint{1.935594in}{3.128325in}}%
\pgfpathcurveto{\pgfqpoint{1.943408in}{3.136138in}}{\pgfqpoint{1.947798in}{3.146737in}}{\pgfqpoint{1.947798in}{3.157787in}}%
\pgfpathcurveto{\pgfqpoint{1.947798in}{3.168837in}}{\pgfqpoint{1.943408in}{3.179436in}}{\pgfqpoint{1.935594in}{3.187250in}}%
\pgfpathcurveto{\pgfqpoint{1.927780in}{3.195064in}}{\pgfqpoint{1.917181in}{3.199454in}}{\pgfqpoint{1.906131in}{3.199454in}}%
\pgfpathcurveto{\pgfqpoint{1.895081in}{3.199454in}}{\pgfqpoint{1.884482in}{3.195064in}}{\pgfqpoint{1.876668in}{3.187250in}}%
\pgfpathcurveto{\pgfqpoint{1.868855in}{3.179436in}}{\pgfqpoint{1.864465in}{3.168837in}}{\pgfqpoint{1.864465in}{3.157787in}}%
\pgfpathcurveto{\pgfqpoint{1.864465in}{3.146737in}}{\pgfqpoint{1.868855in}{3.136138in}}{\pgfqpoint{1.876668in}{3.128325in}}%
\pgfpathcurveto{\pgfqpoint{1.884482in}{3.120511in}}{\pgfqpoint{1.895081in}{3.116121in}}{\pgfqpoint{1.906131in}{3.116121in}}%
\pgfpathclose%
\pgfusepath{stroke,fill}%
\end{pgfscope}%
\begin{pgfscope}%
\pgfpathrectangle{\pgfqpoint{0.787074in}{0.548769in}}{\pgfqpoint{5.062926in}{3.102590in}}%
\pgfusepath{clip}%
\pgfsetbuttcap%
\pgfsetroundjoin%
\definecolor{currentfill}{rgb}{0.839216,0.152941,0.156863}%
\pgfsetfillcolor{currentfill}%
\pgfsetlinewidth{1.003750pt}%
\definecolor{currentstroke}{rgb}{0.839216,0.152941,0.156863}%
\pgfsetstrokecolor{currentstroke}%
\pgfsetdash{}{0pt}%
\pgfpathmoveto{\pgfqpoint{2.002549in}{3.468665in}}%
\pgfpathcurveto{\pgfqpoint{2.013599in}{3.468665in}}{\pgfqpoint{2.024198in}{3.473055in}}{\pgfqpoint{2.032012in}{3.480869in}}%
\pgfpathcurveto{\pgfqpoint{2.039826in}{3.488683in}}{\pgfqpoint{2.044216in}{3.499282in}}{\pgfqpoint{2.044216in}{3.510332in}}%
\pgfpathcurveto{\pgfqpoint{2.044216in}{3.521382in}}{\pgfqpoint{2.039826in}{3.531981in}}{\pgfqpoint{2.032012in}{3.539795in}}%
\pgfpathcurveto{\pgfqpoint{2.024198in}{3.547608in}}{\pgfqpoint{2.013599in}{3.551998in}}{\pgfqpoint{2.002549in}{3.551998in}}%
\pgfpathcurveto{\pgfqpoint{1.991499in}{3.551998in}}{\pgfqpoint{1.980900in}{3.547608in}}{\pgfqpoint{1.973086in}{3.539795in}}%
\pgfpathcurveto{\pgfqpoint{1.965273in}{3.531981in}}{\pgfqpoint{1.960883in}{3.521382in}}{\pgfqpoint{1.960883in}{3.510332in}}%
\pgfpathcurveto{\pgfqpoint{1.960883in}{3.499282in}}{\pgfqpoint{1.965273in}{3.488683in}}{\pgfqpoint{1.973086in}{3.480869in}}%
\pgfpathcurveto{\pgfqpoint{1.980900in}{3.473055in}}{\pgfqpoint{1.991499in}{3.468665in}}{\pgfqpoint{2.002549in}{3.468665in}}%
\pgfpathclose%
\pgfusepath{stroke,fill}%
\end{pgfscope}%
\begin{pgfscope}%
\pgfpathrectangle{\pgfqpoint{0.787074in}{0.548769in}}{\pgfqpoint{5.062926in}{3.102590in}}%
\pgfusepath{clip}%
\pgfsetbuttcap%
\pgfsetroundjoin%
\definecolor{currentfill}{rgb}{1.000000,0.498039,0.054902}%
\pgfsetfillcolor{currentfill}%
\pgfsetlinewidth{1.003750pt}%
\definecolor{currentstroke}{rgb}{1.000000,0.498039,0.054902}%
\pgfsetstrokecolor{currentstroke}%
\pgfsetdash{}{0pt}%
\pgfpathmoveto{\pgfqpoint{1.610474in}{3.121989in}}%
\pgfpathcurveto{\pgfqpoint{1.621524in}{3.121989in}}{\pgfqpoint{1.632123in}{3.126380in}}{\pgfqpoint{1.639937in}{3.134193in}}%
\pgfpathcurveto{\pgfqpoint{1.647751in}{3.142007in}}{\pgfqpoint{1.652141in}{3.152606in}}{\pgfqpoint{1.652141in}{3.163656in}}%
\pgfpathcurveto{\pgfqpoint{1.652141in}{3.174706in}}{\pgfqpoint{1.647751in}{3.185305in}}{\pgfqpoint{1.639937in}{3.193119in}}%
\pgfpathcurveto{\pgfqpoint{1.632123in}{3.200932in}}{\pgfqpoint{1.621524in}{3.205323in}}{\pgfqpoint{1.610474in}{3.205323in}}%
\pgfpathcurveto{\pgfqpoint{1.599424in}{3.205323in}}{\pgfqpoint{1.588825in}{3.200932in}}{\pgfqpoint{1.581011in}{3.193119in}}%
\pgfpathcurveto{\pgfqpoint{1.573198in}{3.185305in}}{\pgfqpoint{1.568808in}{3.174706in}}{\pgfqpoint{1.568808in}{3.163656in}}%
\pgfpathcurveto{\pgfqpoint{1.568808in}{3.152606in}}{\pgfqpoint{1.573198in}{3.142007in}}{\pgfqpoint{1.581011in}{3.134193in}}%
\pgfpathcurveto{\pgfqpoint{1.588825in}{3.126380in}}{\pgfqpoint{1.599424in}{3.121989in}}{\pgfqpoint{1.610474in}{3.121989in}}%
\pgfpathclose%
\pgfusepath{stroke,fill}%
\end{pgfscope}%
\begin{pgfscope}%
\pgfpathrectangle{\pgfqpoint{0.787074in}{0.548769in}}{\pgfqpoint{5.062926in}{3.102590in}}%
\pgfusepath{clip}%
\pgfsetbuttcap%
\pgfsetroundjoin%
\definecolor{currentfill}{rgb}{1.000000,0.498039,0.054902}%
\pgfsetfillcolor{currentfill}%
\pgfsetlinewidth{1.003750pt}%
\definecolor{currentstroke}{rgb}{1.000000,0.498039,0.054902}%
\pgfsetstrokecolor{currentstroke}%
\pgfsetdash{}{0pt}%
\pgfpathmoveto{\pgfqpoint{1.740484in}{2.796310in}}%
\pgfpathcurveto{\pgfqpoint{1.751534in}{2.796310in}}{\pgfqpoint{1.762133in}{2.800700in}}{\pgfqpoint{1.769947in}{2.808514in}}%
\pgfpathcurveto{\pgfqpoint{1.777760in}{2.816328in}}{\pgfqpoint{1.782151in}{2.826927in}}{\pgfqpoint{1.782151in}{2.837977in}}%
\pgfpathcurveto{\pgfqpoint{1.782151in}{2.849027in}}{\pgfqpoint{1.777760in}{2.859626in}}{\pgfqpoint{1.769947in}{2.867440in}}%
\pgfpathcurveto{\pgfqpoint{1.762133in}{2.875253in}}{\pgfqpoint{1.751534in}{2.879644in}}{\pgfqpoint{1.740484in}{2.879644in}}%
\pgfpathcurveto{\pgfqpoint{1.729434in}{2.879644in}}{\pgfqpoint{1.718835in}{2.875253in}}{\pgfqpoint{1.711021in}{2.867440in}}%
\pgfpathcurveto{\pgfqpoint{1.703208in}{2.859626in}}{\pgfqpoint{1.698817in}{2.849027in}}{\pgfqpoint{1.698817in}{2.837977in}}%
\pgfpathcurveto{\pgfqpoint{1.698817in}{2.826927in}}{\pgfqpoint{1.703208in}{2.816328in}}{\pgfqpoint{1.711021in}{2.808514in}}%
\pgfpathcurveto{\pgfqpoint{1.718835in}{2.800700in}}{\pgfqpoint{1.729434in}{2.796310in}}{\pgfqpoint{1.740484in}{2.796310in}}%
\pgfpathclose%
\pgfusepath{stroke,fill}%
\end{pgfscope}%
\begin{pgfscope}%
\pgfpathrectangle{\pgfqpoint{0.787074in}{0.548769in}}{\pgfqpoint{5.062926in}{3.102590in}}%
\pgfusepath{clip}%
\pgfsetbuttcap%
\pgfsetroundjoin%
\definecolor{currentfill}{rgb}{0.121569,0.466667,0.705882}%
\pgfsetfillcolor{currentfill}%
\pgfsetlinewidth{1.003750pt}%
\definecolor{currentstroke}{rgb}{0.121569,0.466667,0.705882}%
\pgfsetstrokecolor{currentstroke}%
\pgfsetdash{}{0pt}%
\pgfpathmoveto{\pgfqpoint{2.240916in}{2.391680in}}%
\pgfpathcurveto{\pgfqpoint{2.251967in}{2.391680in}}{\pgfqpoint{2.262566in}{2.396070in}}{\pgfqpoint{2.270379in}{2.403884in}}%
\pgfpathcurveto{\pgfqpoint{2.278193in}{2.411697in}}{\pgfqpoint{2.282583in}{2.422296in}}{\pgfqpoint{2.282583in}{2.433347in}}%
\pgfpathcurveto{\pgfqpoint{2.282583in}{2.444397in}}{\pgfqpoint{2.278193in}{2.454996in}}{\pgfqpoint{2.270379in}{2.462809in}}%
\pgfpathcurveto{\pgfqpoint{2.262566in}{2.470623in}}{\pgfqpoint{2.251967in}{2.475013in}}{\pgfqpoint{2.240916in}{2.475013in}}%
\pgfpathcurveto{\pgfqpoint{2.229866in}{2.475013in}}{\pgfqpoint{2.219267in}{2.470623in}}{\pgfqpoint{2.211454in}{2.462809in}}%
\pgfpathcurveto{\pgfqpoint{2.203640in}{2.454996in}}{\pgfqpoint{2.199250in}{2.444397in}}{\pgfqpoint{2.199250in}{2.433347in}}%
\pgfpathcurveto{\pgfqpoint{2.199250in}{2.422296in}}{\pgfqpoint{2.203640in}{2.411697in}}{\pgfqpoint{2.211454in}{2.403884in}}%
\pgfpathcurveto{\pgfqpoint{2.219267in}{2.396070in}}{\pgfqpoint{2.229866in}{2.391680in}}{\pgfqpoint{2.240916in}{2.391680in}}%
\pgfpathclose%
\pgfusepath{stroke,fill}%
\end{pgfscope}%
\begin{pgfscope}%
\pgfpathrectangle{\pgfqpoint{0.787074in}{0.548769in}}{\pgfqpoint{5.062926in}{3.102590in}}%
\pgfusepath{clip}%
\pgfsetbuttcap%
\pgfsetroundjoin%
\definecolor{currentfill}{rgb}{1.000000,0.498039,0.054902}%
\pgfsetfillcolor{currentfill}%
\pgfsetlinewidth{1.003750pt}%
\definecolor{currentstroke}{rgb}{1.000000,0.498039,0.054902}%
\pgfsetstrokecolor{currentstroke}%
\pgfsetdash{}{0pt}%
\pgfpathmoveto{\pgfqpoint{1.933759in}{2.448155in}}%
\pgfpathcurveto{\pgfqpoint{1.944809in}{2.448155in}}{\pgfqpoint{1.955408in}{2.452545in}}{\pgfqpoint{1.963222in}{2.460359in}}%
\pgfpathcurveto{\pgfqpoint{1.971035in}{2.468173in}}{\pgfqpoint{1.975426in}{2.478772in}}{\pgfqpoint{1.975426in}{2.489822in}}%
\pgfpathcurveto{\pgfqpoint{1.975426in}{2.500872in}}{\pgfqpoint{1.971035in}{2.511471in}}{\pgfqpoint{1.963222in}{2.519285in}}%
\pgfpathcurveto{\pgfqpoint{1.955408in}{2.527098in}}{\pgfqpoint{1.944809in}{2.531489in}}{\pgfqpoint{1.933759in}{2.531489in}}%
\pgfpathcurveto{\pgfqpoint{1.922709in}{2.531489in}}{\pgfqpoint{1.912110in}{2.527098in}}{\pgfqpoint{1.904296in}{2.519285in}}%
\pgfpathcurveto{\pgfqpoint{1.896483in}{2.511471in}}{\pgfqpoint{1.892092in}{2.500872in}}{\pgfqpoint{1.892092in}{2.489822in}}%
\pgfpathcurveto{\pgfqpoint{1.892092in}{2.478772in}}{\pgfqpoint{1.896483in}{2.468173in}}{\pgfqpoint{1.904296in}{2.460359in}}%
\pgfpathcurveto{\pgfqpoint{1.912110in}{2.452545in}}{\pgfqpoint{1.922709in}{2.448155in}}{\pgfqpoint{1.933759in}{2.448155in}}%
\pgfpathclose%
\pgfusepath{stroke,fill}%
\end{pgfscope}%
\begin{pgfscope}%
\pgfpathrectangle{\pgfqpoint{0.787074in}{0.548769in}}{\pgfqpoint{5.062926in}{3.102590in}}%
\pgfusepath{clip}%
\pgfsetbuttcap%
\pgfsetroundjoin%
\definecolor{currentfill}{rgb}{1.000000,0.498039,0.054902}%
\pgfsetfillcolor{currentfill}%
\pgfsetlinewidth{1.003750pt}%
\definecolor{currentstroke}{rgb}{1.000000,0.498039,0.054902}%
\pgfsetstrokecolor{currentstroke}%
\pgfsetdash{}{0pt}%
\pgfpathmoveto{\pgfqpoint{1.628906in}{3.336404in}}%
\pgfpathcurveto{\pgfqpoint{1.639956in}{3.336404in}}{\pgfqpoint{1.650555in}{3.340794in}}{\pgfqpoint{1.658368in}{3.348607in}}%
\pgfpathcurveto{\pgfqpoint{1.666182in}{3.356421in}}{\pgfqpoint{1.670572in}{3.367020in}}{\pgfqpoint{1.670572in}{3.378070in}}%
\pgfpathcurveto{\pgfqpoint{1.670572in}{3.389120in}}{\pgfqpoint{1.666182in}{3.399719in}}{\pgfqpoint{1.658368in}{3.407533in}}%
\pgfpathcurveto{\pgfqpoint{1.650555in}{3.415347in}}{\pgfqpoint{1.639956in}{3.419737in}}{\pgfqpoint{1.628906in}{3.419737in}}%
\pgfpathcurveto{\pgfqpoint{1.617855in}{3.419737in}}{\pgfqpoint{1.607256in}{3.415347in}}{\pgfqpoint{1.599443in}{3.407533in}}%
\pgfpathcurveto{\pgfqpoint{1.591629in}{3.399719in}}{\pgfqpoint{1.587239in}{3.389120in}}{\pgfqpoint{1.587239in}{3.378070in}}%
\pgfpathcurveto{\pgfqpoint{1.587239in}{3.367020in}}{\pgfqpoint{1.591629in}{3.356421in}}{\pgfqpoint{1.599443in}{3.348607in}}%
\pgfpathcurveto{\pgfqpoint{1.607256in}{3.340794in}}{\pgfqpoint{1.617855in}{3.336404in}}{\pgfqpoint{1.628906in}{3.336404in}}%
\pgfpathclose%
\pgfusepath{stroke,fill}%
\end{pgfscope}%
\begin{pgfscope}%
\pgfpathrectangle{\pgfqpoint{0.787074in}{0.548769in}}{\pgfqpoint{5.062926in}{3.102590in}}%
\pgfusepath{clip}%
\pgfsetbuttcap%
\pgfsetroundjoin%
\definecolor{currentfill}{rgb}{1.000000,0.498039,0.054902}%
\pgfsetfillcolor{currentfill}%
\pgfsetlinewidth{1.003750pt}%
\definecolor{currentstroke}{rgb}{1.000000,0.498039,0.054902}%
\pgfsetstrokecolor{currentstroke}%
\pgfsetdash{}{0pt}%
\pgfpathmoveto{\pgfqpoint{1.619725in}{2.728731in}}%
\pgfpathcurveto{\pgfqpoint{1.630775in}{2.728731in}}{\pgfqpoint{1.641374in}{2.733121in}}{\pgfqpoint{1.649187in}{2.740935in}}%
\pgfpathcurveto{\pgfqpoint{1.657001in}{2.748749in}}{\pgfqpoint{1.661391in}{2.759348in}}{\pgfqpoint{1.661391in}{2.770398in}}%
\pgfpathcurveto{\pgfqpoint{1.661391in}{2.781448in}}{\pgfqpoint{1.657001in}{2.792047in}}{\pgfqpoint{1.649187in}{2.799861in}}%
\pgfpathcurveto{\pgfqpoint{1.641374in}{2.807674in}}{\pgfqpoint{1.630775in}{2.812064in}}{\pgfqpoint{1.619725in}{2.812064in}}%
\pgfpathcurveto{\pgfqpoint{1.608674in}{2.812064in}}{\pgfqpoint{1.598075in}{2.807674in}}{\pgfqpoint{1.590262in}{2.799861in}}%
\pgfpathcurveto{\pgfqpoint{1.582448in}{2.792047in}}{\pgfqpoint{1.578058in}{2.781448in}}{\pgfqpoint{1.578058in}{2.770398in}}%
\pgfpathcurveto{\pgfqpoint{1.578058in}{2.759348in}}{\pgfqpoint{1.582448in}{2.748749in}}{\pgfqpoint{1.590262in}{2.740935in}}%
\pgfpathcurveto{\pgfqpoint{1.598075in}{2.733121in}}{\pgfqpoint{1.608674in}{2.728731in}}{\pgfqpoint{1.619725in}{2.728731in}}%
\pgfpathclose%
\pgfusepath{stroke,fill}%
\end{pgfscope}%
\begin{pgfscope}%
\pgfpathrectangle{\pgfqpoint{0.787074in}{0.548769in}}{\pgfqpoint{5.062926in}{3.102590in}}%
\pgfusepath{clip}%
\pgfsetbuttcap%
\pgfsetroundjoin%
\definecolor{currentfill}{rgb}{1.000000,0.498039,0.054902}%
\pgfsetfillcolor{currentfill}%
\pgfsetlinewidth{1.003750pt}%
\definecolor{currentstroke}{rgb}{1.000000,0.498039,0.054902}%
\pgfsetstrokecolor{currentstroke}%
\pgfsetdash{}{0pt}%
\pgfpathmoveto{\pgfqpoint{1.878014in}{2.523921in}}%
\pgfpathcurveto{\pgfqpoint{1.889064in}{2.523921in}}{\pgfqpoint{1.899663in}{2.528311in}}{\pgfqpoint{1.907477in}{2.536125in}}%
\pgfpathcurveto{\pgfqpoint{1.915291in}{2.543939in}}{\pgfqpoint{1.919681in}{2.554538in}}{\pgfqpoint{1.919681in}{2.565588in}}%
\pgfpathcurveto{\pgfqpoint{1.919681in}{2.576638in}}{\pgfqpoint{1.915291in}{2.587237in}}{\pgfqpoint{1.907477in}{2.595051in}}%
\pgfpathcurveto{\pgfqpoint{1.899663in}{2.602864in}}{\pgfqpoint{1.889064in}{2.607254in}}{\pgfqpoint{1.878014in}{2.607254in}}%
\pgfpathcurveto{\pgfqpoint{1.866964in}{2.607254in}}{\pgfqpoint{1.856365in}{2.602864in}}{\pgfqpoint{1.848551in}{2.595051in}}%
\pgfpathcurveto{\pgfqpoint{1.840738in}{2.587237in}}{\pgfqpoint{1.836347in}{2.576638in}}{\pgfqpoint{1.836347in}{2.565588in}}%
\pgfpathcurveto{\pgfqpoint{1.836347in}{2.554538in}}{\pgfqpoint{1.840738in}{2.543939in}}{\pgfqpoint{1.848551in}{2.536125in}}%
\pgfpathcurveto{\pgfqpoint{1.856365in}{2.528311in}}{\pgfqpoint{1.866964in}{2.523921in}}{\pgfqpoint{1.878014in}{2.523921in}}%
\pgfpathclose%
\pgfusepath{stroke,fill}%
\end{pgfscope}%
\begin{pgfscope}%
\pgfpathrectangle{\pgfqpoint{0.787074in}{0.548769in}}{\pgfqpoint{5.062926in}{3.102590in}}%
\pgfusepath{clip}%
\pgfsetbuttcap%
\pgfsetroundjoin%
\definecolor{currentfill}{rgb}{0.121569,0.466667,0.705882}%
\pgfsetfillcolor{currentfill}%
\pgfsetlinewidth{1.003750pt}%
\definecolor{currentstroke}{rgb}{0.121569,0.466667,0.705882}%
\pgfsetstrokecolor{currentstroke}%
\pgfsetdash{}{0pt}%
\pgfpathmoveto{\pgfqpoint{2.544514in}{2.939541in}}%
\pgfpathcurveto{\pgfqpoint{2.555564in}{2.939541in}}{\pgfqpoint{2.566163in}{2.943931in}}{\pgfqpoint{2.573977in}{2.951745in}}%
\pgfpathcurveto{\pgfqpoint{2.581790in}{2.959559in}}{\pgfqpoint{2.586181in}{2.970158in}}{\pgfqpoint{2.586181in}{2.981208in}}%
\pgfpathcurveto{\pgfqpoint{2.586181in}{2.992258in}}{\pgfqpoint{2.581790in}{3.002857in}}{\pgfqpoint{2.573977in}{3.010671in}}%
\pgfpathcurveto{\pgfqpoint{2.566163in}{3.018484in}}{\pgfqpoint{2.555564in}{3.022874in}}{\pgfqpoint{2.544514in}{3.022874in}}%
\pgfpathcurveto{\pgfqpoint{2.533464in}{3.022874in}}{\pgfqpoint{2.522865in}{3.018484in}}{\pgfqpoint{2.515051in}{3.010671in}}%
\pgfpathcurveto{\pgfqpoint{2.507237in}{3.002857in}}{\pgfqpoint{2.502847in}{2.992258in}}{\pgfqpoint{2.502847in}{2.981208in}}%
\pgfpathcurveto{\pgfqpoint{2.502847in}{2.970158in}}{\pgfqpoint{2.507237in}{2.959559in}}{\pgfqpoint{2.515051in}{2.951745in}}%
\pgfpathcurveto{\pgfqpoint{2.522865in}{2.943931in}}{\pgfqpoint{2.533464in}{2.939541in}}{\pgfqpoint{2.544514in}{2.939541in}}%
\pgfpathclose%
\pgfusepath{stroke,fill}%
\end{pgfscope}%
\begin{pgfscope}%
\pgfpathrectangle{\pgfqpoint{0.787074in}{0.548769in}}{\pgfqpoint{5.062926in}{3.102590in}}%
\pgfusepath{clip}%
\pgfsetbuttcap%
\pgfsetroundjoin%
\definecolor{currentfill}{rgb}{1.000000,0.498039,0.054902}%
\pgfsetfillcolor{currentfill}%
\pgfsetlinewidth{1.003750pt}%
\definecolor{currentstroke}{rgb}{1.000000,0.498039,0.054902}%
\pgfsetstrokecolor{currentstroke}%
\pgfsetdash{}{0pt}%
\pgfpathmoveto{\pgfqpoint{2.044324in}{2.934357in}}%
\pgfpathcurveto{\pgfqpoint{2.055374in}{2.934357in}}{\pgfqpoint{2.065973in}{2.938747in}}{\pgfqpoint{2.073787in}{2.946561in}}%
\pgfpathcurveto{\pgfqpoint{2.081601in}{2.954375in}}{\pgfqpoint{2.085991in}{2.964974in}}{\pgfqpoint{2.085991in}{2.976024in}}%
\pgfpathcurveto{\pgfqpoint{2.085991in}{2.987074in}}{\pgfqpoint{2.081601in}{2.997673in}}{\pgfqpoint{2.073787in}{3.005487in}}%
\pgfpathcurveto{\pgfqpoint{2.065973in}{3.013300in}}{\pgfqpoint{2.055374in}{3.017690in}}{\pgfqpoint{2.044324in}{3.017690in}}%
\pgfpathcurveto{\pgfqpoint{2.033274in}{3.017690in}}{\pgfqpoint{2.022675in}{3.013300in}}{\pgfqpoint{2.014861in}{3.005487in}}%
\pgfpathcurveto{\pgfqpoint{2.007048in}{2.997673in}}{\pgfqpoint{2.002657in}{2.987074in}}{\pgfqpoint{2.002657in}{2.976024in}}%
\pgfpathcurveto{\pgfqpoint{2.002657in}{2.964974in}}{\pgfqpoint{2.007048in}{2.954375in}}{\pgfqpoint{2.014861in}{2.946561in}}%
\pgfpathcurveto{\pgfqpoint{2.022675in}{2.938747in}}{\pgfqpoint{2.033274in}{2.934357in}}{\pgfqpoint{2.044324in}{2.934357in}}%
\pgfpathclose%
\pgfusepath{stroke,fill}%
\end{pgfscope}%
\begin{pgfscope}%
\pgfpathrectangle{\pgfqpoint{0.787074in}{0.548769in}}{\pgfqpoint{5.062926in}{3.102590in}}%
\pgfusepath{clip}%
\pgfsetbuttcap%
\pgfsetroundjoin%
\definecolor{currentfill}{rgb}{1.000000,0.498039,0.054902}%
\pgfsetfillcolor{currentfill}%
\pgfsetlinewidth{1.003750pt}%
\definecolor{currentstroke}{rgb}{1.000000,0.498039,0.054902}%
\pgfsetstrokecolor{currentstroke}%
\pgfsetdash{}{0pt}%
\pgfpathmoveto{\pgfqpoint{2.228615in}{3.327747in}}%
\pgfpathcurveto{\pgfqpoint{2.239665in}{3.327747in}}{\pgfqpoint{2.250264in}{3.332137in}}{\pgfqpoint{2.258077in}{3.339951in}}%
\pgfpathcurveto{\pgfqpoint{2.265891in}{3.347764in}}{\pgfqpoint{2.270281in}{3.358363in}}{\pgfqpoint{2.270281in}{3.369413in}}%
\pgfpathcurveto{\pgfqpoint{2.270281in}{3.380463in}}{\pgfqpoint{2.265891in}{3.391063in}}{\pgfqpoint{2.258077in}{3.398876in}}%
\pgfpathcurveto{\pgfqpoint{2.250264in}{3.406690in}}{\pgfqpoint{2.239665in}{3.411080in}}{\pgfqpoint{2.228615in}{3.411080in}}%
\pgfpathcurveto{\pgfqpoint{2.217565in}{3.411080in}}{\pgfqpoint{2.206966in}{3.406690in}}{\pgfqpoint{2.199152in}{3.398876in}}%
\pgfpathcurveto{\pgfqpoint{2.191338in}{3.391063in}}{\pgfqpoint{2.186948in}{3.380463in}}{\pgfqpoint{2.186948in}{3.369413in}}%
\pgfpathcurveto{\pgfqpoint{2.186948in}{3.358363in}}{\pgfqpoint{2.191338in}{3.347764in}}{\pgfqpoint{2.199152in}{3.339951in}}%
\pgfpathcurveto{\pgfqpoint{2.206966in}{3.332137in}}{\pgfqpoint{2.217565in}{3.327747in}}{\pgfqpoint{2.228615in}{3.327747in}}%
\pgfpathclose%
\pgfusepath{stroke,fill}%
\end{pgfscope}%
\begin{pgfscope}%
\pgfpathrectangle{\pgfqpoint{0.787074in}{0.548769in}}{\pgfqpoint{5.062926in}{3.102590in}}%
\pgfusepath{clip}%
\pgfsetbuttcap%
\pgfsetroundjoin%
\definecolor{currentfill}{rgb}{1.000000,0.498039,0.054902}%
\pgfsetfillcolor{currentfill}%
\pgfsetlinewidth{1.003750pt}%
\definecolor{currentstroke}{rgb}{1.000000,0.498039,0.054902}%
\pgfsetstrokecolor{currentstroke}%
\pgfsetdash{}{0pt}%
\pgfpathmoveto{\pgfqpoint{2.406171in}{2.872580in}}%
\pgfpathcurveto{\pgfqpoint{2.417221in}{2.872580in}}{\pgfqpoint{2.427820in}{2.876970in}}{\pgfqpoint{2.435633in}{2.884784in}}%
\pgfpathcurveto{\pgfqpoint{2.443447in}{2.892597in}}{\pgfqpoint{2.447837in}{2.903196in}}{\pgfqpoint{2.447837in}{2.914246in}}%
\pgfpathcurveto{\pgfqpoint{2.447837in}{2.925297in}}{\pgfqpoint{2.443447in}{2.935896in}}{\pgfqpoint{2.435633in}{2.943709in}}%
\pgfpathcurveto{\pgfqpoint{2.427820in}{2.951523in}}{\pgfqpoint{2.417221in}{2.955913in}}{\pgfqpoint{2.406171in}{2.955913in}}%
\pgfpathcurveto{\pgfqpoint{2.395121in}{2.955913in}}{\pgfqpoint{2.384522in}{2.951523in}}{\pgfqpoint{2.376708in}{2.943709in}}%
\pgfpathcurveto{\pgfqpoint{2.368894in}{2.935896in}}{\pgfqpoint{2.364504in}{2.925297in}}{\pgfqpoint{2.364504in}{2.914246in}}%
\pgfpathcurveto{\pgfqpoint{2.364504in}{2.903196in}}{\pgfqpoint{2.368894in}{2.892597in}}{\pgfqpoint{2.376708in}{2.884784in}}%
\pgfpathcurveto{\pgfqpoint{2.384522in}{2.876970in}}{\pgfqpoint{2.395121in}{2.872580in}}{\pgfqpoint{2.406171in}{2.872580in}}%
\pgfpathclose%
\pgfusepath{stroke,fill}%
\end{pgfscope}%
\begin{pgfscope}%
\pgfpathrectangle{\pgfqpoint{0.787074in}{0.548769in}}{\pgfqpoint{5.062926in}{3.102590in}}%
\pgfusepath{clip}%
\pgfsetbuttcap%
\pgfsetroundjoin%
\definecolor{currentfill}{rgb}{0.839216,0.152941,0.156863}%
\pgfsetfillcolor{currentfill}%
\pgfsetlinewidth{1.003750pt}%
\definecolor{currentstroke}{rgb}{0.839216,0.152941,0.156863}%
\pgfsetstrokecolor{currentstroke}%
\pgfsetdash{}{0pt}%
\pgfpathmoveto{\pgfqpoint{1.514954in}{2.752142in}}%
\pgfpathcurveto{\pgfqpoint{1.526004in}{2.752142in}}{\pgfqpoint{1.536603in}{2.756532in}}{\pgfqpoint{1.544417in}{2.764346in}}%
\pgfpathcurveto{\pgfqpoint{1.552230in}{2.772160in}}{\pgfqpoint{1.556621in}{2.782759in}}{\pgfqpoint{1.556621in}{2.793809in}}%
\pgfpathcurveto{\pgfqpoint{1.556621in}{2.804859in}}{\pgfqpoint{1.552230in}{2.815458in}}{\pgfqpoint{1.544417in}{2.823271in}}%
\pgfpathcurveto{\pgfqpoint{1.536603in}{2.831085in}}{\pgfqpoint{1.526004in}{2.835475in}}{\pgfqpoint{1.514954in}{2.835475in}}%
\pgfpathcurveto{\pgfqpoint{1.503904in}{2.835475in}}{\pgfqpoint{1.493305in}{2.831085in}}{\pgfqpoint{1.485491in}{2.823271in}}%
\pgfpathcurveto{\pgfqpoint{1.477678in}{2.815458in}}{\pgfqpoint{1.473287in}{2.804859in}}{\pgfqpoint{1.473287in}{2.793809in}}%
\pgfpathcurveto{\pgfqpoint{1.473287in}{2.782759in}}{\pgfqpoint{1.477678in}{2.772160in}}{\pgfqpoint{1.485491in}{2.764346in}}%
\pgfpathcurveto{\pgfqpoint{1.493305in}{2.756532in}}{\pgfqpoint{1.503904in}{2.752142in}}{\pgfqpoint{1.514954in}{2.752142in}}%
\pgfpathclose%
\pgfusepath{stroke,fill}%
\end{pgfscope}%
\begin{pgfscope}%
\pgfpathrectangle{\pgfqpoint{0.787074in}{0.548769in}}{\pgfqpoint{5.062926in}{3.102590in}}%
\pgfusepath{clip}%
\pgfsetbuttcap%
\pgfsetroundjoin%
\definecolor{currentfill}{rgb}{1.000000,0.498039,0.054902}%
\pgfsetfillcolor{currentfill}%
\pgfsetlinewidth{1.003750pt}%
\definecolor{currentstroke}{rgb}{1.000000,0.498039,0.054902}%
\pgfsetstrokecolor{currentstroke}%
\pgfsetdash{}{0pt}%
\pgfpathmoveto{\pgfqpoint{2.842825in}{2.067284in}}%
\pgfpathcurveto{\pgfqpoint{2.853876in}{2.067284in}}{\pgfqpoint{2.864475in}{2.071675in}}{\pgfqpoint{2.872288in}{2.079488in}}%
\pgfpathcurveto{\pgfqpoint{2.880102in}{2.087302in}}{\pgfqpoint{2.884492in}{2.097901in}}{\pgfqpoint{2.884492in}{2.108951in}}%
\pgfpathcurveto{\pgfqpoint{2.884492in}{2.120001in}}{\pgfqpoint{2.880102in}{2.130600in}}{\pgfqpoint{2.872288in}{2.138414in}}%
\pgfpathcurveto{\pgfqpoint{2.864475in}{2.146227in}}{\pgfqpoint{2.853876in}{2.150618in}}{\pgfqpoint{2.842825in}{2.150618in}}%
\pgfpathcurveto{\pgfqpoint{2.831775in}{2.150618in}}{\pgfqpoint{2.821176in}{2.146227in}}{\pgfqpoint{2.813363in}{2.138414in}}%
\pgfpathcurveto{\pgfqpoint{2.805549in}{2.130600in}}{\pgfqpoint{2.801159in}{2.120001in}}{\pgfqpoint{2.801159in}{2.108951in}}%
\pgfpathcurveto{\pgfqpoint{2.801159in}{2.097901in}}{\pgfqpoint{2.805549in}{2.087302in}}{\pgfqpoint{2.813363in}{2.079488in}}%
\pgfpathcurveto{\pgfqpoint{2.821176in}{2.071675in}}{\pgfqpoint{2.831775in}{2.067284in}}{\pgfqpoint{2.842825in}{2.067284in}}%
\pgfpathclose%
\pgfusepath{stroke,fill}%
\end{pgfscope}%
\begin{pgfscope}%
\pgfpathrectangle{\pgfqpoint{0.787074in}{0.548769in}}{\pgfqpoint{5.062926in}{3.102590in}}%
\pgfusepath{clip}%
\pgfsetbuttcap%
\pgfsetroundjoin%
\definecolor{currentfill}{rgb}{1.000000,0.498039,0.054902}%
\pgfsetfillcolor{currentfill}%
\pgfsetlinewidth{1.003750pt}%
\definecolor{currentstroke}{rgb}{1.000000,0.498039,0.054902}%
\pgfsetstrokecolor{currentstroke}%
\pgfsetdash{}{0pt}%
\pgfpathmoveto{\pgfqpoint{2.046354in}{3.097151in}}%
\pgfpathcurveto{\pgfqpoint{2.057405in}{3.097151in}}{\pgfqpoint{2.068004in}{3.101541in}}{\pgfqpoint{2.075817in}{3.109355in}}%
\pgfpathcurveto{\pgfqpoint{2.083631in}{3.117169in}}{\pgfqpoint{2.088021in}{3.127768in}}{\pgfqpoint{2.088021in}{3.138818in}}%
\pgfpathcurveto{\pgfqpoint{2.088021in}{3.149868in}}{\pgfqpoint{2.083631in}{3.160467in}}{\pgfqpoint{2.075817in}{3.168281in}}%
\pgfpathcurveto{\pgfqpoint{2.068004in}{3.176094in}}{\pgfqpoint{2.057405in}{3.180484in}}{\pgfqpoint{2.046354in}{3.180484in}}%
\pgfpathcurveto{\pgfqpoint{2.035304in}{3.180484in}}{\pgfqpoint{2.024705in}{3.176094in}}{\pgfqpoint{2.016892in}{3.168281in}}%
\pgfpathcurveto{\pgfqpoint{2.009078in}{3.160467in}}{\pgfqpoint{2.004688in}{3.149868in}}{\pgfqpoint{2.004688in}{3.138818in}}%
\pgfpathcurveto{\pgfqpoint{2.004688in}{3.127768in}}{\pgfqpoint{2.009078in}{3.117169in}}{\pgfqpoint{2.016892in}{3.109355in}}%
\pgfpathcurveto{\pgfqpoint{2.024705in}{3.101541in}}{\pgfqpoint{2.035304in}{3.097151in}}{\pgfqpoint{2.046354in}{3.097151in}}%
\pgfpathclose%
\pgfusepath{stroke,fill}%
\end{pgfscope}%
\begin{pgfscope}%
\pgfpathrectangle{\pgfqpoint{0.787074in}{0.548769in}}{\pgfqpoint{5.062926in}{3.102590in}}%
\pgfusepath{clip}%
\pgfsetbuttcap%
\pgfsetroundjoin%
\definecolor{currentfill}{rgb}{1.000000,0.498039,0.054902}%
\pgfsetfillcolor{currentfill}%
\pgfsetlinewidth{1.003750pt}%
\definecolor{currentstroke}{rgb}{1.000000,0.498039,0.054902}%
\pgfsetstrokecolor{currentstroke}%
\pgfsetdash{}{0pt}%
\pgfpathmoveto{\pgfqpoint{2.046243in}{2.976134in}}%
\pgfpathcurveto{\pgfqpoint{2.057293in}{2.976134in}}{\pgfqpoint{2.067892in}{2.980524in}}{\pgfqpoint{2.075706in}{2.988338in}}%
\pgfpathcurveto{\pgfqpoint{2.083519in}{2.996152in}}{\pgfqpoint{2.087909in}{3.006751in}}{\pgfqpoint{2.087909in}{3.017801in}}%
\pgfpathcurveto{\pgfqpoint{2.087909in}{3.028851in}}{\pgfqpoint{2.083519in}{3.039450in}}{\pgfqpoint{2.075706in}{3.047264in}}%
\pgfpathcurveto{\pgfqpoint{2.067892in}{3.055077in}}{\pgfqpoint{2.057293in}{3.059468in}}{\pgfqpoint{2.046243in}{3.059468in}}%
\pgfpathcurveto{\pgfqpoint{2.035193in}{3.059468in}}{\pgfqpoint{2.024594in}{3.055077in}}{\pgfqpoint{2.016780in}{3.047264in}}%
\pgfpathcurveto{\pgfqpoint{2.008966in}{3.039450in}}{\pgfqpoint{2.004576in}{3.028851in}}{\pgfqpoint{2.004576in}{3.017801in}}%
\pgfpathcurveto{\pgfqpoint{2.004576in}{3.006751in}}{\pgfqpoint{2.008966in}{2.996152in}}{\pgfqpoint{2.016780in}{2.988338in}}%
\pgfpathcurveto{\pgfqpoint{2.024594in}{2.980524in}}{\pgfqpoint{2.035193in}{2.976134in}}{\pgfqpoint{2.046243in}{2.976134in}}%
\pgfpathclose%
\pgfusepath{stroke,fill}%
\end{pgfscope}%
\begin{pgfscope}%
\pgfpathrectangle{\pgfqpoint{0.787074in}{0.548769in}}{\pgfqpoint{5.062926in}{3.102590in}}%
\pgfusepath{clip}%
\pgfsetbuttcap%
\pgfsetroundjoin%
\definecolor{currentfill}{rgb}{1.000000,0.498039,0.054902}%
\pgfsetfillcolor{currentfill}%
\pgfsetlinewidth{1.003750pt}%
\definecolor{currentstroke}{rgb}{1.000000,0.498039,0.054902}%
\pgfsetstrokecolor{currentstroke}%
\pgfsetdash{}{0pt}%
\pgfpathmoveto{\pgfqpoint{2.525416in}{2.882792in}}%
\pgfpathcurveto{\pgfqpoint{2.536466in}{2.882792in}}{\pgfqpoint{2.547065in}{2.887182in}}{\pgfqpoint{2.554879in}{2.894996in}}%
\pgfpathcurveto{\pgfqpoint{2.562692in}{2.902810in}}{\pgfqpoint{2.567083in}{2.913409in}}{\pgfqpoint{2.567083in}{2.924459in}}%
\pgfpathcurveto{\pgfqpoint{2.567083in}{2.935509in}}{\pgfqpoint{2.562692in}{2.946108in}}{\pgfqpoint{2.554879in}{2.953922in}}%
\pgfpathcurveto{\pgfqpoint{2.547065in}{2.961735in}}{\pgfqpoint{2.536466in}{2.966125in}}{\pgfqpoint{2.525416in}{2.966125in}}%
\pgfpathcurveto{\pgfqpoint{2.514366in}{2.966125in}}{\pgfqpoint{2.503767in}{2.961735in}}{\pgfqpoint{2.495953in}{2.953922in}}%
\pgfpathcurveto{\pgfqpoint{2.488140in}{2.946108in}}{\pgfqpoint{2.483749in}{2.935509in}}{\pgfqpoint{2.483749in}{2.924459in}}%
\pgfpathcurveto{\pgfqpoint{2.483749in}{2.913409in}}{\pgfqpoint{2.488140in}{2.902810in}}{\pgfqpoint{2.495953in}{2.894996in}}%
\pgfpathcurveto{\pgfqpoint{2.503767in}{2.887182in}}{\pgfqpoint{2.514366in}{2.882792in}}{\pgfqpoint{2.525416in}{2.882792in}}%
\pgfpathclose%
\pgfusepath{stroke,fill}%
\end{pgfscope}%
\begin{pgfscope}%
\pgfpathrectangle{\pgfqpoint{0.787074in}{0.548769in}}{\pgfqpoint{5.062926in}{3.102590in}}%
\pgfusepath{clip}%
\pgfsetbuttcap%
\pgfsetroundjoin%
\definecolor{currentfill}{rgb}{0.839216,0.152941,0.156863}%
\pgfsetfillcolor{currentfill}%
\pgfsetlinewidth{1.003750pt}%
\definecolor{currentstroke}{rgb}{0.839216,0.152941,0.156863}%
\pgfsetstrokecolor{currentstroke}%
\pgfsetdash{}{0pt}%
\pgfpathmoveto{\pgfqpoint{2.180182in}{3.334884in}}%
\pgfpathcurveto{\pgfqpoint{2.191232in}{3.334884in}}{\pgfqpoint{2.201831in}{3.339274in}}{\pgfqpoint{2.209645in}{3.347088in}}%
\pgfpathcurveto{\pgfqpoint{2.217459in}{3.354902in}}{\pgfqpoint{2.221849in}{3.365501in}}{\pgfqpoint{2.221849in}{3.376551in}}%
\pgfpathcurveto{\pgfqpoint{2.221849in}{3.387601in}}{\pgfqpoint{2.217459in}{3.398200in}}{\pgfqpoint{2.209645in}{3.406014in}}%
\pgfpathcurveto{\pgfqpoint{2.201831in}{3.413827in}}{\pgfqpoint{2.191232in}{3.418218in}}{\pgfqpoint{2.180182in}{3.418218in}}%
\pgfpathcurveto{\pgfqpoint{2.169132in}{3.418218in}}{\pgfqpoint{2.158533in}{3.413827in}}{\pgfqpoint{2.150719in}{3.406014in}}%
\pgfpathcurveto{\pgfqpoint{2.142906in}{3.398200in}}{\pgfqpoint{2.138516in}{3.387601in}}{\pgfqpoint{2.138516in}{3.376551in}}%
\pgfpathcurveto{\pgfqpoint{2.138516in}{3.365501in}}{\pgfqpoint{2.142906in}{3.354902in}}{\pgfqpoint{2.150719in}{3.347088in}}%
\pgfpathcurveto{\pgfqpoint{2.158533in}{3.339274in}}{\pgfqpoint{2.169132in}{3.334884in}}{\pgfqpoint{2.180182in}{3.334884in}}%
\pgfpathclose%
\pgfusepath{stroke,fill}%
\end{pgfscope}%
\begin{pgfscope}%
\pgfpathrectangle{\pgfqpoint{0.787074in}{0.548769in}}{\pgfqpoint{5.062926in}{3.102590in}}%
\pgfusepath{clip}%
\pgfsetbuttcap%
\pgfsetroundjoin%
\definecolor{currentfill}{rgb}{1.000000,0.498039,0.054902}%
\pgfsetfillcolor{currentfill}%
\pgfsetlinewidth{1.003750pt}%
\definecolor{currentstroke}{rgb}{1.000000,0.498039,0.054902}%
\pgfsetstrokecolor{currentstroke}%
\pgfsetdash{}{0pt}%
\pgfpathmoveto{\pgfqpoint{2.242280in}{3.000517in}}%
\pgfpathcurveto{\pgfqpoint{2.253330in}{3.000517in}}{\pgfqpoint{2.263929in}{3.004907in}}{\pgfqpoint{2.271743in}{3.012721in}}%
\pgfpathcurveto{\pgfqpoint{2.279557in}{3.020535in}}{\pgfqpoint{2.283947in}{3.031134in}}{\pgfqpoint{2.283947in}{3.042184in}}%
\pgfpathcurveto{\pgfqpoint{2.283947in}{3.053234in}}{\pgfqpoint{2.279557in}{3.063833in}}{\pgfqpoint{2.271743in}{3.071646in}}%
\pgfpathcurveto{\pgfqpoint{2.263929in}{3.079460in}}{\pgfqpoint{2.253330in}{3.083850in}}{\pgfqpoint{2.242280in}{3.083850in}}%
\pgfpathcurveto{\pgfqpoint{2.231230in}{3.083850in}}{\pgfqpoint{2.220631in}{3.079460in}}{\pgfqpoint{2.212817in}{3.071646in}}%
\pgfpathcurveto{\pgfqpoint{2.205004in}{3.063833in}}{\pgfqpoint{2.200614in}{3.053234in}}{\pgfqpoint{2.200614in}{3.042184in}}%
\pgfpathcurveto{\pgfqpoint{2.200614in}{3.031134in}}{\pgfqpoint{2.205004in}{3.020535in}}{\pgfqpoint{2.212817in}{3.012721in}}%
\pgfpathcurveto{\pgfqpoint{2.220631in}{3.004907in}}{\pgfqpoint{2.231230in}{3.000517in}}{\pgfqpoint{2.242280in}{3.000517in}}%
\pgfpathclose%
\pgfusepath{stroke,fill}%
\end{pgfscope}%
\begin{pgfscope}%
\pgfpathrectangle{\pgfqpoint{0.787074in}{0.548769in}}{\pgfqpoint{5.062926in}{3.102590in}}%
\pgfusepath{clip}%
\pgfsetbuttcap%
\pgfsetroundjoin%
\definecolor{currentfill}{rgb}{0.121569,0.466667,0.705882}%
\pgfsetfillcolor{currentfill}%
\pgfsetlinewidth{1.003750pt}%
\definecolor{currentstroke}{rgb}{0.121569,0.466667,0.705882}%
\pgfsetstrokecolor{currentstroke}%
\pgfsetdash{}{0pt}%
\pgfpathmoveto{\pgfqpoint{2.158792in}{2.441957in}}%
\pgfpathcurveto{\pgfqpoint{2.169842in}{2.441957in}}{\pgfqpoint{2.180441in}{2.446347in}}{\pgfqpoint{2.188255in}{2.454161in}}%
\pgfpathcurveto{\pgfqpoint{2.196068in}{2.461974in}}{\pgfqpoint{2.200459in}{2.472573in}}{\pgfqpoint{2.200459in}{2.483623in}}%
\pgfpathcurveto{\pgfqpoint{2.200459in}{2.494673in}}{\pgfqpoint{2.196068in}{2.505273in}}{\pgfqpoint{2.188255in}{2.513086in}}%
\pgfpathcurveto{\pgfqpoint{2.180441in}{2.520900in}}{\pgfqpoint{2.169842in}{2.525290in}}{\pgfqpoint{2.158792in}{2.525290in}}%
\pgfpathcurveto{\pgfqpoint{2.147742in}{2.525290in}}{\pgfqpoint{2.137143in}{2.520900in}}{\pgfqpoint{2.129329in}{2.513086in}}%
\pgfpathcurveto{\pgfqpoint{2.121516in}{2.505273in}}{\pgfqpoint{2.117125in}{2.494673in}}{\pgfqpoint{2.117125in}{2.483623in}}%
\pgfpathcurveto{\pgfqpoint{2.117125in}{2.472573in}}{\pgfqpoint{2.121516in}{2.461974in}}{\pgfqpoint{2.129329in}{2.454161in}}%
\pgfpathcurveto{\pgfqpoint{2.137143in}{2.446347in}}{\pgfqpoint{2.147742in}{2.441957in}}{\pgfqpoint{2.158792in}{2.441957in}}%
\pgfpathclose%
\pgfusepath{stroke,fill}%
\end{pgfscope}%
\begin{pgfscope}%
\pgfpathrectangle{\pgfqpoint{0.787074in}{0.548769in}}{\pgfqpoint{5.062926in}{3.102590in}}%
\pgfusepath{clip}%
\pgfsetbuttcap%
\pgfsetroundjoin%
\definecolor{currentfill}{rgb}{0.121569,0.466667,0.705882}%
\pgfsetfillcolor{currentfill}%
\pgfsetlinewidth{1.003750pt}%
\definecolor{currentstroke}{rgb}{0.121569,0.466667,0.705882}%
\pgfsetstrokecolor{currentstroke}%
\pgfsetdash{}{0pt}%
\pgfpathmoveto{\pgfqpoint{1.180438in}{0.919027in}}%
\pgfpathcurveto{\pgfqpoint{1.191489in}{0.919027in}}{\pgfqpoint{1.202088in}{0.923417in}}{\pgfqpoint{1.209901in}{0.931231in}}%
\pgfpathcurveto{\pgfqpoint{1.217715in}{0.939044in}}{\pgfqpoint{1.222105in}{0.949643in}}{\pgfqpoint{1.222105in}{0.960694in}}%
\pgfpathcurveto{\pgfqpoint{1.222105in}{0.971744in}}{\pgfqpoint{1.217715in}{0.982343in}}{\pgfqpoint{1.209901in}{0.990156in}}%
\pgfpathcurveto{\pgfqpoint{1.202088in}{0.997970in}}{\pgfqpoint{1.191489in}{1.002360in}}{\pgfqpoint{1.180438in}{1.002360in}}%
\pgfpathcurveto{\pgfqpoint{1.169388in}{1.002360in}}{\pgfqpoint{1.158789in}{0.997970in}}{\pgfqpoint{1.150976in}{0.990156in}}%
\pgfpathcurveto{\pgfqpoint{1.143162in}{0.982343in}}{\pgfqpoint{1.138772in}{0.971744in}}{\pgfqpoint{1.138772in}{0.960694in}}%
\pgfpathcurveto{\pgfqpoint{1.138772in}{0.949643in}}{\pgfqpoint{1.143162in}{0.939044in}}{\pgfqpoint{1.150976in}{0.931231in}}%
\pgfpathcurveto{\pgfqpoint{1.158789in}{0.923417in}}{\pgfqpoint{1.169388in}{0.919027in}}{\pgfqpoint{1.180438in}{0.919027in}}%
\pgfpathclose%
\pgfusepath{stroke,fill}%
\end{pgfscope}%
\begin{pgfscope}%
\pgfpathrectangle{\pgfqpoint{0.787074in}{0.548769in}}{\pgfqpoint{5.062926in}{3.102590in}}%
\pgfusepath{clip}%
\pgfsetbuttcap%
\pgfsetroundjoin%
\definecolor{currentfill}{rgb}{0.839216,0.152941,0.156863}%
\pgfsetfillcolor{currentfill}%
\pgfsetlinewidth{1.003750pt}%
\definecolor{currentstroke}{rgb}{0.839216,0.152941,0.156863}%
\pgfsetstrokecolor{currentstroke}%
\pgfsetdash{}{0pt}%
\pgfpathmoveto{\pgfqpoint{2.050188in}{3.248678in}}%
\pgfpathcurveto{\pgfqpoint{2.061238in}{3.248678in}}{\pgfqpoint{2.071837in}{3.253068in}}{\pgfqpoint{2.079651in}{3.260882in}}%
\pgfpathcurveto{\pgfqpoint{2.087464in}{3.268695in}}{\pgfqpoint{2.091855in}{3.279294in}}{\pgfqpoint{2.091855in}{3.290344in}}%
\pgfpathcurveto{\pgfqpoint{2.091855in}{3.301395in}}{\pgfqpoint{2.087464in}{3.311994in}}{\pgfqpoint{2.079651in}{3.319807in}}%
\pgfpathcurveto{\pgfqpoint{2.071837in}{3.327621in}}{\pgfqpoint{2.061238in}{3.332011in}}{\pgfqpoint{2.050188in}{3.332011in}}%
\pgfpathcurveto{\pgfqpoint{2.039138in}{3.332011in}}{\pgfqpoint{2.028539in}{3.327621in}}{\pgfqpoint{2.020725in}{3.319807in}}%
\pgfpathcurveto{\pgfqpoint{2.012912in}{3.311994in}}{\pgfqpoint{2.008521in}{3.301395in}}{\pgfqpoint{2.008521in}{3.290344in}}%
\pgfpathcurveto{\pgfqpoint{2.008521in}{3.279294in}}{\pgfqpoint{2.012912in}{3.268695in}}{\pgfqpoint{2.020725in}{3.260882in}}%
\pgfpathcurveto{\pgfqpoint{2.028539in}{3.253068in}}{\pgfqpoint{2.039138in}{3.248678in}}{\pgfqpoint{2.050188in}{3.248678in}}%
\pgfpathclose%
\pgfusepath{stroke,fill}%
\end{pgfscope}%
\begin{pgfscope}%
\pgfpathrectangle{\pgfqpoint{0.787074in}{0.548769in}}{\pgfqpoint{5.062926in}{3.102590in}}%
\pgfusepath{clip}%
\pgfsetbuttcap%
\pgfsetroundjoin%
\definecolor{currentfill}{rgb}{1.000000,0.498039,0.054902}%
\pgfsetfillcolor{currentfill}%
\pgfsetlinewidth{1.003750pt}%
\definecolor{currentstroke}{rgb}{1.000000,0.498039,0.054902}%
\pgfsetstrokecolor{currentstroke}%
\pgfsetdash{}{0pt}%
\pgfpathmoveto{\pgfqpoint{1.600045in}{2.631493in}}%
\pgfpathcurveto{\pgfqpoint{1.611095in}{2.631493in}}{\pgfqpoint{1.621694in}{2.635883in}}{\pgfqpoint{1.629508in}{2.643697in}}%
\pgfpathcurveto{\pgfqpoint{1.637321in}{2.651510in}}{\pgfqpoint{1.641712in}{2.662109in}}{\pgfqpoint{1.641712in}{2.673159in}}%
\pgfpathcurveto{\pgfqpoint{1.641712in}{2.684210in}}{\pgfqpoint{1.637321in}{2.694809in}}{\pgfqpoint{1.629508in}{2.702622in}}%
\pgfpathcurveto{\pgfqpoint{1.621694in}{2.710436in}}{\pgfqpoint{1.611095in}{2.714826in}}{\pgfqpoint{1.600045in}{2.714826in}}%
\pgfpathcurveto{\pgfqpoint{1.588995in}{2.714826in}}{\pgfqpoint{1.578396in}{2.710436in}}{\pgfqpoint{1.570582in}{2.702622in}}%
\pgfpathcurveto{\pgfqpoint{1.562769in}{2.694809in}}{\pgfqpoint{1.558378in}{2.684210in}}{\pgfqpoint{1.558378in}{2.673159in}}%
\pgfpathcurveto{\pgfqpoint{1.558378in}{2.662109in}}{\pgfqpoint{1.562769in}{2.651510in}}{\pgfqpoint{1.570582in}{2.643697in}}%
\pgfpathcurveto{\pgfqpoint{1.578396in}{2.635883in}}{\pgfqpoint{1.588995in}{2.631493in}}{\pgfqpoint{1.600045in}{2.631493in}}%
\pgfpathclose%
\pgfusepath{stroke,fill}%
\end{pgfscope}%
\begin{pgfscope}%
\pgfpathrectangle{\pgfqpoint{0.787074in}{0.548769in}}{\pgfqpoint{5.062926in}{3.102590in}}%
\pgfusepath{clip}%
\pgfsetbuttcap%
\pgfsetroundjoin%
\definecolor{currentfill}{rgb}{1.000000,0.498039,0.054902}%
\pgfsetfillcolor{currentfill}%
\pgfsetlinewidth{1.003750pt}%
\definecolor{currentstroke}{rgb}{1.000000,0.498039,0.054902}%
\pgfsetstrokecolor{currentstroke}%
\pgfsetdash{}{0pt}%
\pgfpathmoveto{\pgfqpoint{1.526867in}{2.514237in}}%
\pgfpathcurveto{\pgfqpoint{1.537917in}{2.514237in}}{\pgfqpoint{1.548516in}{2.518627in}}{\pgfqpoint{1.556329in}{2.526441in}}%
\pgfpathcurveto{\pgfqpoint{1.564143in}{2.534255in}}{\pgfqpoint{1.568533in}{2.544854in}}{\pgfqpoint{1.568533in}{2.555904in}}%
\pgfpathcurveto{\pgfqpoint{1.568533in}{2.566954in}}{\pgfqpoint{1.564143in}{2.577553in}}{\pgfqpoint{1.556329in}{2.585367in}}%
\pgfpathcurveto{\pgfqpoint{1.548516in}{2.593180in}}{\pgfqpoint{1.537917in}{2.597570in}}{\pgfqpoint{1.526867in}{2.597570in}}%
\pgfpathcurveto{\pgfqpoint{1.515816in}{2.597570in}}{\pgfqpoint{1.505217in}{2.593180in}}{\pgfqpoint{1.497404in}{2.585367in}}%
\pgfpathcurveto{\pgfqpoint{1.489590in}{2.577553in}}{\pgfqpoint{1.485200in}{2.566954in}}{\pgfqpoint{1.485200in}{2.555904in}}%
\pgfpathcurveto{\pgfqpoint{1.485200in}{2.544854in}}{\pgfqpoint{1.489590in}{2.534255in}}{\pgfqpoint{1.497404in}{2.526441in}}%
\pgfpathcurveto{\pgfqpoint{1.505217in}{2.518627in}}{\pgfqpoint{1.515816in}{2.514237in}}{\pgfqpoint{1.526867in}{2.514237in}}%
\pgfpathclose%
\pgfusepath{stroke,fill}%
\end{pgfscope}%
\begin{pgfscope}%
\pgfpathrectangle{\pgfqpoint{0.787074in}{0.548769in}}{\pgfqpoint{5.062926in}{3.102590in}}%
\pgfusepath{clip}%
\pgfsetbuttcap%
\pgfsetroundjoin%
\definecolor{currentfill}{rgb}{1.000000,0.498039,0.054902}%
\pgfsetfillcolor{currentfill}%
\pgfsetlinewidth{1.003750pt}%
\definecolor{currentstroke}{rgb}{1.000000,0.498039,0.054902}%
\pgfsetstrokecolor{currentstroke}%
\pgfsetdash{}{0pt}%
\pgfpathmoveto{\pgfqpoint{1.342692in}{2.630257in}}%
\pgfpathcurveto{\pgfqpoint{1.353742in}{2.630257in}}{\pgfqpoint{1.364341in}{2.634648in}}{\pgfqpoint{1.372154in}{2.642461in}}%
\pgfpathcurveto{\pgfqpoint{1.379968in}{2.650275in}}{\pgfqpoint{1.384358in}{2.660874in}}{\pgfqpoint{1.384358in}{2.671924in}}%
\pgfpathcurveto{\pgfqpoint{1.384358in}{2.682974in}}{\pgfqpoint{1.379968in}{2.693573in}}{\pgfqpoint{1.372154in}{2.701387in}}%
\pgfpathcurveto{\pgfqpoint{1.364341in}{2.709200in}}{\pgfqpoint{1.353742in}{2.713591in}}{\pgfqpoint{1.342692in}{2.713591in}}%
\pgfpathcurveto{\pgfqpoint{1.331641in}{2.713591in}}{\pgfqpoint{1.321042in}{2.709200in}}{\pgfqpoint{1.313229in}{2.701387in}}%
\pgfpathcurveto{\pgfqpoint{1.305415in}{2.693573in}}{\pgfqpoint{1.301025in}{2.682974in}}{\pgfqpoint{1.301025in}{2.671924in}}%
\pgfpathcurveto{\pgfqpoint{1.301025in}{2.660874in}}{\pgfqpoint{1.305415in}{2.650275in}}{\pgfqpoint{1.313229in}{2.642461in}}%
\pgfpathcurveto{\pgfqpoint{1.321042in}{2.634648in}}{\pgfqpoint{1.331641in}{2.630257in}}{\pgfqpoint{1.342692in}{2.630257in}}%
\pgfpathclose%
\pgfusepath{stroke,fill}%
\end{pgfscope}%
\begin{pgfscope}%
\pgfpathrectangle{\pgfqpoint{0.787074in}{0.548769in}}{\pgfqpoint{5.062926in}{3.102590in}}%
\pgfusepath{clip}%
\pgfsetbuttcap%
\pgfsetroundjoin%
\definecolor{currentfill}{rgb}{1.000000,0.498039,0.054902}%
\pgfsetfillcolor{currentfill}%
\pgfsetlinewidth{1.003750pt}%
\definecolor{currentstroke}{rgb}{1.000000,0.498039,0.054902}%
\pgfsetstrokecolor{currentstroke}%
\pgfsetdash{}{0pt}%
\pgfpathmoveto{\pgfqpoint{1.671528in}{3.109003in}}%
\pgfpathcurveto{\pgfqpoint{1.682578in}{3.109003in}}{\pgfqpoint{1.693177in}{3.113393in}}{\pgfqpoint{1.700991in}{3.121207in}}%
\pgfpathcurveto{\pgfqpoint{1.708805in}{3.129020in}}{\pgfqpoint{1.713195in}{3.139619in}}{\pgfqpoint{1.713195in}{3.150669in}}%
\pgfpathcurveto{\pgfqpoint{1.713195in}{3.161719in}}{\pgfqpoint{1.708805in}{3.172319in}}{\pgfqpoint{1.700991in}{3.180132in}}%
\pgfpathcurveto{\pgfqpoint{1.693177in}{3.187946in}}{\pgfqpoint{1.682578in}{3.192336in}}{\pgfqpoint{1.671528in}{3.192336in}}%
\pgfpathcurveto{\pgfqpoint{1.660478in}{3.192336in}}{\pgfqpoint{1.649879in}{3.187946in}}{\pgfqpoint{1.642065in}{3.180132in}}%
\pgfpathcurveto{\pgfqpoint{1.634252in}{3.172319in}}{\pgfqpoint{1.629861in}{3.161719in}}{\pgfqpoint{1.629861in}{3.150669in}}%
\pgfpathcurveto{\pgfqpoint{1.629861in}{3.139619in}}{\pgfqpoint{1.634252in}{3.129020in}}{\pgfqpoint{1.642065in}{3.121207in}}%
\pgfpathcurveto{\pgfqpoint{1.649879in}{3.113393in}}{\pgfqpoint{1.660478in}{3.109003in}}{\pgfqpoint{1.671528in}{3.109003in}}%
\pgfpathclose%
\pgfusepath{stroke,fill}%
\end{pgfscope}%
\begin{pgfscope}%
\pgfpathrectangle{\pgfqpoint{0.787074in}{0.548769in}}{\pgfqpoint{5.062926in}{3.102590in}}%
\pgfusepath{clip}%
\pgfsetbuttcap%
\pgfsetroundjoin%
\definecolor{currentfill}{rgb}{1.000000,0.498039,0.054902}%
\pgfsetfillcolor{currentfill}%
\pgfsetlinewidth{1.003750pt}%
\definecolor{currentstroke}{rgb}{1.000000,0.498039,0.054902}%
\pgfsetstrokecolor{currentstroke}%
\pgfsetdash{}{0pt}%
\pgfpathmoveto{\pgfqpoint{2.319003in}{2.413769in}}%
\pgfpathcurveto{\pgfqpoint{2.330053in}{2.413769in}}{\pgfqpoint{2.340652in}{2.418159in}}{\pgfqpoint{2.348466in}{2.425973in}}%
\pgfpathcurveto{\pgfqpoint{2.356280in}{2.433787in}}{\pgfqpoint{2.360670in}{2.444386in}}{\pgfqpoint{2.360670in}{2.455436in}}%
\pgfpathcurveto{\pgfqpoint{2.360670in}{2.466486in}}{\pgfqpoint{2.356280in}{2.477085in}}{\pgfqpoint{2.348466in}{2.484899in}}%
\pgfpathcurveto{\pgfqpoint{2.340652in}{2.492712in}}{\pgfqpoint{2.330053in}{2.497102in}}{\pgfqpoint{2.319003in}{2.497102in}}%
\pgfpathcurveto{\pgfqpoint{2.307953in}{2.497102in}}{\pgfqpoint{2.297354in}{2.492712in}}{\pgfqpoint{2.289540in}{2.484899in}}%
\pgfpathcurveto{\pgfqpoint{2.281727in}{2.477085in}}{\pgfqpoint{2.277336in}{2.466486in}}{\pgfqpoint{2.277336in}{2.455436in}}%
\pgfpathcurveto{\pgfqpoint{2.277336in}{2.444386in}}{\pgfqpoint{2.281727in}{2.433787in}}{\pgfqpoint{2.289540in}{2.425973in}}%
\pgfpathcurveto{\pgfqpoint{2.297354in}{2.418159in}}{\pgfqpoint{2.307953in}{2.413769in}}{\pgfqpoint{2.319003in}{2.413769in}}%
\pgfpathclose%
\pgfusepath{stroke,fill}%
\end{pgfscope}%
\begin{pgfscope}%
\pgfpathrectangle{\pgfqpoint{0.787074in}{0.548769in}}{\pgfqpoint{5.062926in}{3.102590in}}%
\pgfusepath{clip}%
\pgfsetbuttcap%
\pgfsetroundjoin%
\definecolor{currentfill}{rgb}{0.121569,0.466667,0.705882}%
\pgfsetfillcolor{currentfill}%
\pgfsetlinewidth{1.003750pt}%
\definecolor{currentstroke}{rgb}{0.121569,0.466667,0.705882}%
\pgfsetstrokecolor{currentstroke}%
\pgfsetdash{}{0pt}%
\pgfpathmoveto{\pgfqpoint{2.167179in}{2.714693in}}%
\pgfpathcurveto{\pgfqpoint{2.178229in}{2.714693in}}{\pgfqpoint{2.188829in}{2.719083in}}{\pgfqpoint{2.196642in}{2.726897in}}%
\pgfpathcurveto{\pgfqpoint{2.204456in}{2.734711in}}{\pgfqpoint{2.208846in}{2.745310in}}{\pgfqpoint{2.208846in}{2.756360in}}%
\pgfpathcurveto{\pgfqpoint{2.208846in}{2.767410in}}{\pgfqpoint{2.204456in}{2.778009in}}{\pgfqpoint{2.196642in}{2.785822in}}%
\pgfpathcurveto{\pgfqpoint{2.188829in}{2.793636in}}{\pgfqpoint{2.178229in}{2.798026in}}{\pgfqpoint{2.167179in}{2.798026in}}%
\pgfpathcurveto{\pgfqpoint{2.156129in}{2.798026in}}{\pgfqpoint{2.145530in}{2.793636in}}{\pgfqpoint{2.137717in}{2.785822in}}%
\pgfpathcurveto{\pgfqpoint{2.129903in}{2.778009in}}{\pgfqpoint{2.125513in}{2.767410in}}{\pgfqpoint{2.125513in}{2.756360in}}%
\pgfpathcurveto{\pgfqpoint{2.125513in}{2.745310in}}{\pgfqpoint{2.129903in}{2.734711in}}{\pgfqpoint{2.137717in}{2.726897in}}%
\pgfpathcurveto{\pgfqpoint{2.145530in}{2.719083in}}{\pgfqpoint{2.156129in}{2.714693in}}{\pgfqpoint{2.167179in}{2.714693in}}%
\pgfpathclose%
\pgfusepath{stroke,fill}%
\end{pgfscope}%
\begin{pgfscope}%
\pgfpathrectangle{\pgfqpoint{0.787074in}{0.548769in}}{\pgfqpoint{5.062926in}{3.102590in}}%
\pgfusepath{clip}%
\pgfsetbuttcap%
\pgfsetroundjoin%
\definecolor{currentfill}{rgb}{1.000000,0.498039,0.054902}%
\pgfsetfillcolor{currentfill}%
\pgfsetlinewidth{1.003750pt}%
\definecolor{currentstroke}{rgb}{1.000000,0.498039,0.054902}%
\pgfsetstrokecolor{currentstroke}%
\pgfsetdash{}{0pt}%
\pgfpathmoveto{\pgfqpoint{1.894962in}{2.455826in}}%
\pgfpathcurveto{\pgfqpoint{1.906012in}{2.455826in}}{\pgfqpoint{1.916611in}{2.460216in}}{\pgfqpoint{1.924425in}{2.468030in}}%
\pgfpathcurveto{\pgfqpoint{1.932239in}{2.475843in}}{\pgfqpoint{1.936629in}{2.486442in}}{\pgfqpoint{1.936629in}{2.497492in}}%
\pgfpathcurveto{\pgfqpoint{1.936629in}{2.508542in}}{\pgfqpoint{1.932239in}{2.519141in}}{\pgfqpoint{1.924425in}{2.526955in}}%
\pgfpathcurveto{\pgfqpoint{1.916611in}{2.534769in}}{\pgfqpoint{1.906012in}{2.539159in}}{\pgfqpoint{1.894962in}{2.539159in}}%
\pgfpathcurveto{\pgfqpoint{1.883912in}{2.539159in}}{\pgfqpoint{1.873313in}{2.534769in}}{\pgfqpoint{1.865499in}{2.526955in}}%
\pgfpathcurveto{\pgfqpoint{1.857686in}{2.519141in}}{\pgfqpoint{1.853296in}{2.508542in}}{\pgfqpoint{1.853296in}{2.497492in}}%
\pgfpathcurveto{\pgfqpoint{1.853296in}{2.486442in}}{\pgfqpoint{1.857686in}{2.475843in}}{\pgfqpoint{1.865499in}{2.468030in}}%
\pgfpathcurveto{\pgfqpoint{1.873313in}{2.460216in}}{\pgfqpoint{1.883912in}{2.455826in}}{\pgfqpoint{1.894962in}{2.455826in}}%
\pgfpathclose%
\pgfusepath{stroke,fill}%
\end{pgfscope}%
\begin{pgfscope}%
\pgfpathrectangle{\pgfqpoint{0.787074in}{0.548769in}}{\pgfqpoint{5.062926in}{3.102590in}}%
\pgfusepath{clip}%
\pgfsetbuttcap%
\pgfsetroundjoin%
\definecolor{currentfill}{rgb}{1.000000,0.498039,0.054902}%
\pgfsetfillcolor{currentfill}%
\pgfsetlinewidth{1.003750pt}%
\definecolor{currentstroke}{rgb}{1.000000,0.498039,0.054902}%
\pgfsetstrokecolor{currentstroke}%
\pgfsetdash{}{0pt}%
\pgfpathmoveto{\pgfqpoint{1.540902in}{2.530243in}}%
\pgfpathcurveto{\pgfqpoint{1.551952in}{2.530243in}}{\pgfqpoint{1.562551in}{2.534634in}}{\pgfqpoint{1.570365in}{2.542447in}}%
\pgfpathcurveto{\pgfqpoint{1.578178in}{2.550261in}}{\pgfqpoint{1.582569in}{2.560860in}}{\pgfqpoint{1.582569in}{2.571910in}}%
\pgfpathcurveto{\pgfqpoint{1.582569in}{2.582960in}}{\pgfqpoint{1.578178in}{2.593559in}}{\pgfqpoint{1.570365in}{2.601373in}}%
\pgfpathcurveto{\pgfqpoint{1.562551in}{2.609186in}}{\pgfqpoint{1.551952in}{2.613577in}}{\pgfqpoint{1.540902in}{2.613577in}}%
\pgfpathcurveto{\pgfqpoint{1.529852in}{2.613577in}}{\pgfqpoint{1.519253in}{2.609186in}}{\pgfqpoint{1.511439in}{2.601373in}}%
\pgfpathcurveto{\pgfqpoint{1.503626in}{2.593559in}}{\pgfqpoint{1.499235in}{2.582960in}}{\pgfqpoint{1.499235in}{2.571910in}}%
\pgfpathcurveto{\pgfqpoint{1.499235in}{2.560860in}}{\pgfqpoint{1.503626in}{2.550261in}}{\pgfqpoint{1.511439in}{2.542447in}}%
\pgfpathcurveto{\pgfqpoint{1.519253in}{2.534634in}}{\pgfqpoint{1.529852in}{2.530243in}}{\pgfqpoint{1.540902in}{2.530243in}}%
\pgfpathclose%
\pgfusepath{stroke,fill}%
\end{pgfscope}%
\begin{pgfscope}%
\pgfpathrectangle{\pgfqpoint{0.787074in}{0.548769in}}{\pgfqpoint{5.062926in}{3.102590in}}%
\pgfusepath{clip}%
\pgfsetbuttcap%
\pgfsetroundjoin%
\definecolor{currentfill}{rgb}{1.000000,0.498039,0.054902}%
\pgfsetfillcolor{currentfill}%
\pgfsetlinewidth{1.003750pt}%
\definecolor{currentstroke}{rgb}{1.000000,0.498039,0.054902}%
\pgfsetstrokecolor{currentstroke}%
\pgfsetdash{}{0pt}%
\pgfpathmoveto{\pgfqpoint{1.838728in}{2.652528in}}%
\pgfpathcurveto{\pgfqpoint{1.849778in}{2.652528in}}{\pgfqpoint{1.860377in}{2.656919in}}{\pgfqpoint{1.868191in}{2.664732in}}%
\pgfpathcurveto{\pgfqpoint{1.876004in}{2.672546in}}{\pgfqpoint{1.880395in}{2.683145in}}{\pgfqpoint{1.880395in}{2.694195in}}%
\pgfpathcurveto{\pgfqpoint{1.880395in}{2.705245in}}{\pgfqpoint{1.876004in}{2.715844in}}{\pgfqpoint{1.868191in}{2.723658in}}%
\pgfpathcurveto{\pgfqpoint{1.860377in}{2.731471in}}{\pgfqpoint{1.849778in}{2.735862in}}{\pgfqpoint{1.838728in}{2.735862in}}%
\pgfpathcurveto{\pgfqpoint{1.827678in}{2.735862in}}{\pgfqpoint{1.817079in}{2.731471in}}{\pgfqpoint{1.809265in}{2.723658in}}%
\pgfpathcurveto{\pgfqpoint{1.801452in}{2.715844in}}{\pgfqpoint{1.797061in}{2.705245in}}{\pgfqpoint{1.797061in}{2.694195in}}%
\pgfpathcurveto{\pgfqpoint{1.797061in}{2.683145in}}{\pgfqpoint{1.801452in}{2.672546in}}{\pgfqpoint{1.809265in}{2.664732in}}%
\pgfpathcurveto{\pgfqpoint{1.817079in}{2.656919in}}{\pgfqpoint{1.827678in}{2.652528in}}{\pgfqpoint{1.838728in}{2.652528in}}%
\pgfpathclose%
\pgfusepath{stroke,fill}%
\end{pgfscope}%
\begin{pgfscope}%
\pgfpathrectangle{\pgfqpoint{0.787074in}{0.548769in}}{\pgfqpoint{5.062926in}{3.102590in}}%
\pgfusepath{clip}%
\pgfsetbuttcap%
\pgfsetroundjoin%
\definecolor{currentfill}{rgb}{1.000000,0.498039,0.054902}%
\pgfsetfillcolor{currentfill}%
\pgfsetlinewidth{1.003750pt}%
\definecolor{currentstroke}{rgb}{1.000000,0.498039,0.054902}%
\pgfsetstrokecolor{currentstroke}%
\pgfsetdash{}{0pt}%
\pgfpathmoveto{\pgfqpoint{1.897004in}{2.518244in}}%
\pgfpathcurveto{\pgfqpoint{1.908054in}{2.518244in}}{\pgfqpoint{1.918653in}{2.522634in}}{\pgfqpoint{1.926467in}{2.530448in}}%
\pgfpathcurveto{\pgfqpoint{1.934281in}{2.538261in}}{\pgfqpoint{1.938671in}{2.548860in}}{\pgfqpoint{1.938671in}{2.559910in}}%
\pgfpathcurveto{\pgfqpoint{1.938671in}{2.570961in}}{\pgfqpoint{1.934281in}{2.581560in}}{\pgfqpoint{1.926467in}{2.589373in}}%
\pgfpathcurveto{\pgfqpoint{1.918653in}{2.597187in}}{\pgfqpoint{1.908054in}{2.601577in}}{\pgfqpoint{1.897004in}{2.601577in}}%
\pgfpathcurveto{\pgfqpoint{1.885954in}{2.601577in}}{\pgfqpoint{1.875355in}{2.597187in}}{\pgfqpoint{1.867541in}{2.589373in}}%
\pgfpathcurveto{\pgfqpoint{1.859728in}{2.581560in}}{\pgfqpoint{1.855337in}{2.570961in}}{\pgfqpoint{1.855337in}{2.559910in}}%
\pgfpathcurveto{\pgfqpoint{1.855337in}{2.548860in}}{\pgfqpoint{1.859728in}{2.538261in}}{\pgfqpoint{1.867541in}{2.530448in}}%
\pgfpathcurveto{\pgfqpoint{1.875355in}{2.522634in}}{\pgfqpoint{1.885954in}{2.518244in}}{\pgfqpoint{1.897004in}{2.518244in}}%
\pgfpathclose%
\pgfusepath{stroke,fill}%
\end{pgfscope}%
\begin{pgfscope}%
\pgfpathrectangle{\pgfqpoint{0.787074in}{0.548769in}}{\pgfqpoint{5.062926in}{3.102590in}}%
\pgfusepath{clip}%
\pgfsetbuttcap%
\pgfsetroundjoin%
\definecolor{currentfill}{rgb}{0.839216,0.152941,0.156863}%
\pgfsetfillcolor{currentfill}%
\pgfsetlinewidth{1.003750pt}%
\definecolor{currentstroke}{rgb}{0.839216,0.152941,0.156863}%
\pgfsetstrokecolor{currentstroke}%
\pgfsetdash{}{0pt}%
\pgfpathmoveto{\pgfqpoint{2.093781in}{2.503837in}}%
\pgfpathcurveto{\pgfqpoint{2.104831in}{2.503837in}}{\pgfqpoint{2.115431in}{2.508227in}}{\pgfqpoint{2.123244in}{2.516041in}}%
\pgfpathcurveto{\pgfqpoint{2.131058in}{2.523854in}}{\pgfqpoint{2.135448in}{2.534453in}}{\pgfqpoint{2.135448in}{2.545503in}}%
\pgfpathcurveto{\pgfqpoint{2.135448in}{2.556554in}}{\pgfqpoint{2.131058in}{2.567153in}}{\pgfqpoint{2.123244in}{2.574966in}}%
\pgfpathcurveto{\pgfqpoint{2.115431in}{2.582780in}}{\pgfqpoint{2.104831in}{2.587170in}}{\pgfqpoint{2.093781in}{2.587170in}}%
\pgfpathcurveto{\pgfqpoint{2.082731in}{2.587170in}}{\pgfqpoint{2.072132in}{2.582780in}}{\pgfqpoint{2.064319in}{2.574966in}}%
\pgfpathcurveto{\pgfqpoint{2.056505in}{2.567153in}}{\pgfqpoint{2.052115in}{2.556554in}}{\pgfqpoint{2.052115in}{2.545503in}}%
\pgfpathcurveto{\pgfqpoint{2.052115in}{2.534453in}}{\pgfqpoint{2.056505in}{2.523854in}}{\pgfqpoint{2.064319in}{2.516041in}}%
\pgfpathcurveto{\pgfqpoint{2.072132in}{2.508227in}}{\pgfqpoint{2.082731in}{2.503837in}}{\pgfqpoint{2.093781in}{2.503837in}}%
\pgfpathclose%
\pgfusepath{stroke,fill}%
\end{pgfscope}%
\begin{pgfscope}%
\pgfpathrectangle{\pgfqpoint{0.787074in}{0.548769in}}{\pgfqpoint{5.062926in}{3.102590in}}%
\pgfusepath{clip}%
\pgfsetbuttcap%
\pgfsetroundjoin%
\definecolor{currentfill}{rgb}{1.000000,0.498039,0.054902}%
\pgfsetfillcolor{currentfill}%
\pgfsetlinewidth{1.003750pt}%
\definecolor{currentstroke}{rgb}{1.000000,0.498039,0.054902}%
\pgfsetstrokecolor{currentstroke}%
\pgfsetdash{}{0pt}%
\pgfpathmoveto{\pgfqpoint{1.609068in}{2.836137in}}%
\pgfpathcurveto{\pgfqpoint{1.620118in}{2.836137in}}{\pgfqpoint{1.630717in}{2.840528in}}{\pgfqpoint{1.638531in}{2.848341in}}%
\pgfpathcurveto{\pgfqpoint{1.646344in}{2.856155in}}{\pgfqpoint{1.650735in}{2.866754in}}{\pgfqpoint{1.650735in}{2.877804in}}%
\pgfpathcurveto{\pgfqpoint{1.650735in}{2.888854in}}{\pgfqpoint{1.646344in}{2.899453in}}{\pgfqpoint{1.638531in}{2.907267in}}%
\pgfpathcurveto{\pgfqpoint{1.630717in}{2.915080in}}{\pgfqpoint{1.620118in}{2.919471in}}{\pgfqpoint{1.609068in}{2.919471in}}%
\pgfpathcurveto{\pgfqpoint{1.598018in}{2.919471in}}{\pgfqpoint{1.587419in}{2.915080in}}{\pgfqpoint{1.579605in}{2.907267in}}%
\pgfpathcurveto{\pgfqpoint{1.571792in}{2.899453in}}{\pgfqpoint{1.567401in}{2.888854in}}{\pgfqpoint{1.567401in}{2.877804in}}%
\pgfpathcurveto{\pgfqpoint{1.567401in}{2.866754in}}{\pgfqpoint{1.571792in}{2.856155in}}{\pgfqpoint{1.579605in}{2.848341in}}%
\pgfpathcurveto{\pgfqpoint{1.587419in}{2.840528in}}{\pgfqpoint{1.598018in}{2.836137in}}{\pgfqpoint{1.609068in}{2.836137in}}%
\pgfpathclose%
\pgfusepath{stroke,fill}%
\end{pgfscope}%
\begin{pgfscope}%
\pgfpathrectangle{\pgfqpoint{0.787074in}{0.548769in}}{\pgfqpoint{5.062926in}{3.102590in}}%
\pgfusepath{clip}%
\pgfsetbuttcap%
\pgfsetroundjoin%
\definecolor{currentfill}{rgb}{0.839216,0.152941,0.156863}%
\pgfsetfillcolor{currentfill}%
\pgfsetlinewidth{1.003750pt}%
\definecolor{currentstroke}{rgb}{0.839216,0.152941,0.156863}%
\pgfsetstrokecolor{currentstroke}%
\pgfsetdash{}{0pt}%
\pgfpathmoveto{\pgfqpoint{1.928018in}{2.834106in}}%
\pgfpathcurveto{\pgfqpoint{1.939069in}{2.834106in}}{\pgfqpoint{1.949668in}{2.838496in}}{\pgfqpoint{1.957481in}{2.846310in}}%
\pgfpathcurveto{\pgfqpoint{1.965295in}{2.854123in}}{\pgfqpoint{1.969685in}{2.864722in}}{\pgfqpoint{1.969685in}{2.875772in}}%
\pgfpathcurveto{\pgfqpoint{1.969685in}{2.886823in}}{\pgfqpoint{1.965295in}{2.897422in}}{\pgfqpoint{1.957481in}{2.905235in}}%
\pgfpathcurveto{\pgfqpoint{1.949668in}{2.913049in}}{\pgfqpoint{1.939069in}{2.917439in}}{\pgfqpoint{1.928018in}{2.917439in}}%
\pgfpathcurveto{\pgfqpoint{1.916968in}{2.917439in}}{\pgfqpoint{1.906369in}{2.913049in}}{\pgfqpoint{1.898556in}{2.905235in}}%
\pgfpathcurveto{\pgfqpoint{1.890742in}{2.897422in}}{\pgfqpoint{1.886352in}{2.886823in}}{\pgfqpoint{1.886352in}{2.875772in}}%
\pgfpathcurveto{\pgfqpoint{1.886352in}{2.864722in}}{\pgfqpoint{1.890742in}{2.854123in}}{\pgfqpoint{1.898556in}{2.846310in}}%
\pgfpathcurveto{\pgfqpoint{1.906369in}{2.838496in}}{\pgfqpoint{1.916968in}{2.834106in}}{\pgfqpoint{1.928018in}{2.834106in}}%
\pgfpathclose%
\pgfusepath{stroke,fill}%
\end{pgfscope}%
\begin{pgfscope}%
\pgfpathrectangle{\pgfqpoint{0.787074in}{0.548769in}}{\pgfqpoint{5.062926in}{3.102590in}}%
\pgfusepath{clip}%
\pgfsetbuttcap%
\pgfsetroundjoin%
\definecolor{currentfill}{rgb}{0.121569,0.466667,0.705882}%
\pgfsetfillcolor{currentfill}%
\pgfsetlinewidth{1.003750pt}%
\definecolor{currentstroke}{rgb}{0.121569,0.466667,0.705882}%
\pgfsetstrokecolor{currentstroke}%
\pgfsetdash{}{0pt}%
\pgfpathmoveto{\pgfqpoint{2.202586in}{2.400262in}}%
\pgfpathcurveto{\pgfqpoint{2.213636in}{2.400262in}}{\pgfqpoint{2.224235in}{2.404652in}}{\pgfqpoint{2.232049in}{2.412465in}}%
\pgfpathcurveto{\pgfqpoint{2.239862in}{2.420279in}}{\pgfqpoint{2.244252in}{2.430878in}}{\pgfqpoint{2.244252in}{2.441928in}}%
\pgfpathcurveto{\pgfqpoint{2.244252in}{2.452978in}}{\pgfqpoint{2.239862in}{2.463577in}}{\pgfqpoint{2.232049in}{2.471391in}}%
\pgfpathcurveto{\pgfqpoint{2.224235in}{2.479205in}}{\pgfqpoint{2.213636in}{2.483595in}}{\pgfqpoint{2.202586in}{2.483595in}}%
\pgfpathcurveto{\pgfqpoint{2.191536in}{2.483595in}}{\pgfqpoint{2.180937in}{2.479205in}}{\pgfqpoint{2.173123in}{2.471391in}}%
\pgfpathcurveto{\pgfqpoint{2.165309in}{2.463577in}}{\pgfqpoint{2.160919in}{2.452978in}}{\pgfqpoint{2.160919in}{2.441928in}}%
\pgfpathcurveto{\pgfqpoint{2.160919in}{2.430878in}}{\pgfqpoint{2.165309in}{2.420279in}}{\pgfqpoint{2.173123in}{2.412465in}}%
\pgfpathcurveto{\pgfqpoint{2.180937in}{2.404652in}}{\pgfqpoint{2.191536in}{2.400262in}}{\pgfqpoint{2.202586in}{2.400262in}}%
\pgfpathclose%
\pgfusepath{stroke,fill}%
\end{pgfscope}%
\begin{pgfscope}%
\pgfpathrectangle{\pgfqpoint{0.787074in}{0.548769in}}{\pgfqpoint{5.062926in}{3.102590in}}%
\pgfusepath{clip}%
\pgfsetbuttcap%
\pgfsetroundjoin%
\definecolor{currentfill}{rgb}{1.000000,0.498039,0.054902}%
\pgfsetfillcolor{currentfill}%
\pgfsetlinewidth{1.003750pt}%
\definecolor{currentstroke}{rgb}{1.000000,0.498039,0.054902}%
\pgfsetstrokecolor{currentstroke}%
\pgfsetdash{}{0pt}%
\pgfpathmoveto{\pgfqpoint{2.014928in}{2.515312in}}%
\pgfpathcurveto{\pgfqpoint{2.025978in}{2.515312in}}{\pgfqpoint{2.036577in}{2.519702in}}{\pgfqpoint{2.044391in}{2.527516in}}%
\pgfpathcurveto{\pgfqpoint{2.052204in}{2.535330in}}{\pgfqpoint{2.056595in}{2.545929in}}{\pgfqpoint{2.056595in}{2.556979in}}%
\pgfpathcurveto{\pgfqpoint{2.056595in}{2.568029in}}{\pgfqpoint{2.052204in}{2.578628in}}{\pgfqpoint{2.044391in}{2.586442in}}%
\pgfpathcurveto{\pgfqpoint{2.036577in}{2.594255in}}{\pgfqpoint{2.025978in}{2.598645in}}{\pgfqpoint{2.014928in}{2.598645in}}%
\pgfpathcurveto{\pgfqpoint{2.003878in}{2.598645in}}{\pgfqpoint{1.993279in}{2.594255in}}{\pgfqpoint{1.985465in}{2.586442in}}%
\pgfpathcurveto{\pgfqpoint{1.977652in}{2.578628in}}{\pgfqpoint{1.973261in}{2.568029in}}{\pgfqpoint{1.973261in}{2.556979in}}%
\pgfpathcurveto{\pgfqpoint{1.973261in}{2.545929in}}{\pgfqpoint{1.977652in}{2.535330in}}{\pgfqpoint{1.985465in}{2.527516in}}%
\pgfpathcurveto{\pgfqpoint{1.993279in}{2.519702in}}{\pgfqpoint{2.003878in}{2.515312in}}{\pgfqpoint{2.014928in}{2.515312in}}%
\pgfpathclose%
\pgfusepath{stroke,fill}%
\end{pgfscope}%
\begin{pgfscope}%
\pgfpathrectangle{\pgfqpoint{0.787074in}{0.548769in}}{\pgfqpoint{5.062926in}{3.102590in}}%
\pgfusepath{clip}%
\pgfsetbuttcap%
\pgfsetroundjoin%
\definecolor{currentfill}{rgb}{1.000000,0.498039,0.054902}%
\pgfsetfillcolor{currentfill}%
\pgfsetlinewidth{1.003750pt}%
\definecolor{currentstroke}{rgb}{1.000000,0.498039,0.054902}%
\pgfsetstrokecolor{currentstroke}%
\pgfsetdash{}{0pt}%
\pgfpathmoveto{\pgfqpoint{1.787922in}{2.940138in}}%
\pgfpathcurveto{\pgfqpoint{1.798972in}{2.940138in}}{\pgfqpoint{1.809572in}{2.944529in}}{\pgfqpoint{1.817385in}{2.952342in}}%
\pgfpathcurveto{\pgfqpoint{1.825199in}{2.960156in}}{\pgfqpoint{1.829589in}{2.970755in}}{\pgfqpoint{1.829589in}{2.981805in}}%
\pgfpathcurveto{\pgfqpoint{1.829589in}{2.992855in}}{\pgfqpoint{1.825199in}{3.003454in}}{\pgfqpoint{1.817385in}{3.011268in}}%
\pgfpathcurveto{\pgfqpoint{1.809572in}{3.019082in}}{\pgfqpoint{1.798972in}{3.023472in}}{\pgfqpoint{1.787922in}{3.023472in}}%
\pgfpathcurveto{\pgfqpoint{1.776872in}{3.023472in}}{\pgfqpoint{1.766273in}{3.019082in}}{\pgfqpoint{1.758460in}{3.011268in}}%
\pgfpathcurveto{\pgfqpoint{1.750646in}{3.003454in}}{\pgfqpoint{1.746256in}{2.992855in}}{\pgfqpoint{1.746256in}{2.981805in}}%
\pgfpathcurveto{\pgfqpoint{1.746256in}{2.970755in}}{\pgfqpoint{1.750646in}{2.960156in}}{\pgfqpoint{1.758460in}{2.952342in}}%
\pgfpathcurveto{\pgfqpoint{1.766273in}{2.944529in}}{\pgfqpoint{1.776872in}{2.940138in}}{\pgfqpoint{1.787922in}{2.940138in}}%
\pgfpathclose%
\pgfusepath{stroke,fill}%
\end{pgfscope}%
\begin{pgfscope}%
\pgfpathrectangle{\pgfqpoint{0.787074in}{0.548769in}}{\pgfqpoint{5.062926in}{3.102590in}}%
\pgfusepath{clip}%
\pgfsetbuttcap%
\pgfsetroundjoin%
\definecolor{currentfill}{rgb}{1.000000,0.498039,0.054902}%
\pgfsetfillcolor{currentfill}%
\pgfsetlinewidth{1.003750pt}%
\definecolor{currentstroke}{rgb}{1.000000,0.498039,0.054902}%
\pgfsetstrokecolor{currentstroke}%
\pgfsetdash{}{0pt}%
\pgfpathmoveto{\pgfqpoint{1.225034in}{3.117335in}}%
\pgfpathcurveto{\pgfqpoint{1.236084in}{3.117335in}}{\pgfqpoint{1.246683in}{3.121725in}}{\pgfqpoint{1.254496in}{3.129539in}}%
\pgfpathcurveto{\pgfqpoint{1.262310in}{3.137352in}}{\pgfqpoint{1.266700in}{3.147951in}}{\pgfqpoint{1.266700in}{3.159001in}}%
\pgfpathcurveto{\pgfqpoint{1.266700in}{3.170052in}}{\pgfqpoint{1.262310in}{3.180651in}}{\pgfqpoint{1.254496in}{3.188464in}}%
\pgfpathcurveto{\pgfqpoint{1.246683in}{3.196278in}}{\pgfqpoint{1.236084in}{3.200668in}}{\pgfqpoint{1.225034in}{3.200668in}}%
\pgfpathcurveto{\pgfqpoint{1.213983in}{3.200668in}}{\pgfqpoint{1.203384in}{3.196278in}}{\pgfqpoint{1.195571in}{3.188464in}}%
\pgfpathcurveto{\pgfqpoint{1.187757in}{3.180651in}}{\pgfqpoint{1.183367in}{3.170052in}}{\pgfqpoint{1.183367in}{3.159001in}}%
\pgfpathcurveto{\pgfqpoint{1.183367in}{3.147951in}}{\pgfqpoint{1.187757in}{3.137352in}}{\pgfqpoint{1.195571in}{3.129539in}}%
\pgfpathcurveto{\pgfqpoint{1.203384in}{3.121725in}}{\pgfqpoint{1.213983in}{3.117335in}}{\pgfqpoint{1.225034in}{3.117335in}}%
\pgfpathclose%
\pgfusepath{stroke,fill}%
\end{pgfscope}%
\begin{pgfscope}%
\pgfpathrectangle{\pgfqpoint{0.787074in}{0.548769in}}{\pgfqpoint{5.062926in}{3.102590in}}%
\pgfusepath{clip}%
\pgfsetbuttcap%
\pgfsetroundjoin%
\definecolor{currentfill}{rgb}{1.000000,0.498039,0.054902}%
\pgfsetfillcolor{currentfill}%
\pgfsetlinewidth{1.003750pt}%
\definecolor{currentstroke}{rgb}{1.000000,0.498039,0.054902}%
\pgfsetstrokecolor{currentstroke}%
\pgfsetdash{}{0pt}%
\pgfpathmoveto{\pgfqpoint{2.275911in}{3.017356in}}%
\pgfpathcurveto{\pgfqpoint{2.286961in}{3.017356in}}{\pgfqpoint{2.297560in}{3.021746in}}{\pgfqpoint{2.305373in}{3.029559in}}%
\pgfpathcurveto{\pgfqpoint{2.313187in}{3.037373in}}{\pgfqpoint{2.317577in}{3.047972in}}{\pgfqpoint{2.317577in}{3.059022in}}%
\pgfpathcurveto{\pgfqpoint{2.317577in}{3.070072in}}{\pgfqpoint{2.313187in}{3.080671in}}{\pgfqpoint{2.305373in}{3.088485in}}%
\pgfpathcurveto{\pgfqpoint{2.297560in}{3.096299in}}{\pgfqpoint{2.286961in}{3.100689in}}{\pgfqpoint{2.275911in}{3.100689in}}%
\pgfpathcurveto{\pgfqpoint{2.264860in}{3.100689in}}{\pgfqpoint{2.254261in}{3.096299in}}{\pgfqpoint{2.246448in}{3.088485in}}%
\pgfpathcurveto{\pgfqpoint{2.238634in}{3.080671in}}{\pgfqpoint{2.234244in}{3.070072in}}{\pgfqpoint{2.234244in}{3.059022in}}%
\pgfpathcurveto{\pgfqpoint{2.234244in}{3.047972in}}{\pgfqpoint{2.238634in}{3.037373in}}{\pgfqpoint{2.246448in}{3.029559in}}%
\pgfpathcurveto{\pgfqpoint{2.254261in}{3.021746in}}{\pgfqpoint{2.264860in}{3.017356in}}{\pgfqpoint{2.275911in}{3.017356in}}%
\pgfpathclose%
\pgfusepath{stroke,fill}%
\end{pgfscope}%
\begin{pgfscope}%
\pgfpathrectangle{\pgfqpoint{0.787074in}{0.548769in}}{\pgfqpoint{5.062926in}{3.102590in}}%
\pgfusepath{clip}%
\pgfsetbuttcap%
\pgfsetroundjoin%
\definecolor{currentfill}{rgb}{1.000000,0.498039,0.054902}%
\pgfsetfillcolor{currentfill}%
\pgfsetlinewidth{1.003750pt}%
\definecolor{currentstroke}{rgb}{1.000000,0.498039,0.054902}%
\pgfsetstrokecolor{currentstroke}%
\pgfsetdash{}{0pt}%
\pgfpathmoveto{\pgfqpoint{1.484695in}{1.775158in}}%
\pgfpathcurveto{\pgfqpoint{1.495745in}{1.775158in}}{\pgfqpoint{1.506344in}{1.779548in}}{\pgfqpoint{1.514158in}{1.787362in}}%
\pgfpathcurveto{\pgfqpoint{1.521971in}{1.795175in}}{\pgfqpoint{1.526361in}{1.805774in}}{\pgfqpoint{1.526361in}{1.816825in}}%
\pgfpathcurveto{\pgfqpoint{1.526361in}{1.827875in}}{\pgfqpoint{1.521971in}{1.838474in}}{\pgfqpoint{1.514158in}{1.846287in}}%
\pgfpathcurveto{\pgfqpoint{1.506344in}{1.854101in}}{\pgfqpoint{1.495745in}{1.858491in}}{\pgfqpoint{1.484695in}{1.858491in}}%
\pgfpathcurveto{\pgfqpoint{1.473645in}{1.858491in}}{\pgfqpoint{1.463046in}{1.854101in}}{\pgfqpoint{1.455232in}{1.846287in}}%
\pgfpathcurveto{\pgfqpoint{1.447418in}{1.838474in}}{\pgfqpoint{1.443028in}{1.827875in}}{\pgfqpoint{1.443028in}{1.816825in}}%
\pgfpathcurveto{\pgfqpoint{1.443028in}{1.805774in}}{\pgfqpoint{1.447418in}{1.795175in}}{\pgfqpoint{1.455232in}{1.787362in}}%
\pgfpathcurveto{\pgfqpoint{1.463046in}{1.779548in}}{\pgfqpoint{1.473645in}{1.775158in}}{\pgfqpoint{1.484695in}{1.775158in}}%
\pgfpathclose%
\pgfusepath{stroke,fill}%
\end{pgfscope}%
\begin{pgfscope}%
\pgfpathrectangle{\pgfqpoint{0.787074in}{0.548769in}}{\pgfqpoint{5.062926in}{3.102590in}}%
\pgfusepath{clip}%
\pgfsetbuttcap%
\pgfsetroundjoin%
\definecolor{currentfill}{rgb}{1.000000,0.498039,0.054902}%
\pgfsetfillcolor{currentfill}%
\pgfsetlinewidth{1.003750pt}%
\definecolor{currentstroke}{rgb}{1.000000,0.498039,0.054902}%
\pgfsetstrokecolor{currentstroke}%
\pgfsetdash{}{0pt}%
\pgfpathmoveto{\pgfqpoint{1.977144in}{3.272382in}}%
\pgfpathcurveto{\pgfqpoint{1.988195in}{3.272382in}}{\pgfqpoint{1.998794in}{3.276772in}}{\pgfqpoint{2.006607in}{3.284586in}}%
\pgfpathcurveto{\pgfqpoint{2.014421in}{3.292399in}}{\pgfqpoint{2.018811in}{3.302998in}}{\pgfqpoint{2.018811in}{3.314049in}}%
\pgfpathcurveto{\pgfqpoint{2.018811in}{3.325099in}}{\pgfqpoint{2.014421in}{3.335698in}}{\pgfqpoint{2.006607in}{3.343511in}}%
\pgfpathcurveto{\pgfqpoint{1.998794in}{3.351325in}}{\pgfqpoint{1.988195in}{3.355715in}}{\pgfqpoint{1.977144in}{3.355715in}}%
\pgfpathcurveto{\pgfqpoint{1.966094in}{3.355715in}}{\pgfqpoint{1.955495in}{3.351325in}}{\pgfqpoint{1.947682in}{3.343511in}}%
\pgfpathcurveto{\pgfqpoint{1.939868in}{3.335698in}}{\pgfqpoint{1.935478in}{3.325099in}}{\pgfqpoint{1.935478in}{3.314049in}}%
\pgfpathcurveto{\pgfqpoint{1.935478in}{3.302998in}}{\pgfqpoint{1.939868in}{3.292399in}}{\pgfqpoint{1.947682in}{3.284586in}}%
\pgfpathcurveto{\pgfqpoint{1.955495in}{3.276772in}}{\pgfqpoint{1.966094in}{3.272382in}}{\pgfqpoint{1.977144in}{3.272382in}}%
\pgfpathclose%
\pgfusepath{stroke,fill}%
\end{pgfscope}%
\begin{pgfscope}%
\pgfpathrectangle{\pgfqpoint{0.787074in}{0.548769in}}{\pgfqpoint{5.062926in}{3.102590in}}%
\pgfusepath{clip}%
\pgfsetbuttcap%
\pgfsetroundjoin%
\definecolor{currentfill}{rgb}{0.121569,0.466667,0.705882}%
\pgfsetfillcolor{currentfill}%
\pgfsetlinewidth{1.003750pt}%
\definecolor{currentstroke}{rgb}{0.121569,0.466667,0.705882}%
\pgfsetstrokecolor{currentstroke}%
\pgfsetdash{}{0pt}%
\pgfpathmoveto{\pgfqpoint{1.350227in}{0.676982in}}%
\pgfpathcurveto{\pgfqpoint{1.361278in}{0.676982in}}{\pgfqpoint{1.371877in}{0.681372in}}{\pgfqpoint{1.379690in}{0.689186in}}%
\pgfpathcurveto{\pgfqpoint{1.387504in}{0.696999in}}{\pgfqpoint{1.391894in}{0.707598in}}{\pgfqpoint{1.391894in}{0.718649in}}%
\pgfpathcurveto{\pgfqpoint{1.391894in}{0.729699in}}{\pgfqpoint{1.387504in}{0.740298in}}{\pgfqpoint{1.379690in}{0.748111in}}%
\pgfpathcurveto{\pgfqpoint{1.371877in}{0.755925in}}{\pgfqpoint{1.361278in}{0.760315in}}{\pgfqpoint{1.350227in}{0.760315in}}%
\pgfpathcurveto{\pgfqpoint{1.339177in}{0.760315in}}{\pgfqpoint{1.328578in}{0.755925in}}{\pgfqpoint{1.320765in}{0.748111in}}%
\pgfpathcurveto{\pgfqpoint{1.312951in}{0.740298in}}{\pgfqpoint{1.308561in}{0.729699in}}{\pgfqpoint{1.308561in}{0.718649in}}%
\pgfpathcurveto{\pgfqpoint{1.308561in}{0.707598in}}{\pgfqpoint{1.312951in}{0.696999in}}{\pgfqpoint{1.320765in}{0.689186in}}%
\pgfpathcurveto{\pgfqpoint{1.328578in}{0.681372in}}{\pgfqpoint{1.339177in}{0.676982in}}{\pgfqpoint{1.350227in}{0.676982in}}%
\pgfpathclose%
\pgfusepath{stroke,fill}%
\end{pgfscope}%
\begin{pgfscope}%
\pgfpathrectangle{\pgfqpoint{0.787074in}{0.548769in}}{\pgfqpoint{5.062926in}{3.102590in}}%
\pgfusepath{clip}%
\pgfsetbuttcap%
\pgfsetroundjoin%
\definecolor{currentfill}{rgb}{0.839216,0.152941,0.156863}%
\pgfsetfillcolor{currentfill}%
\pgfsetlinewidth{1.003750pt}%
\definecolor{currentstroke}{rgb}{0.839216,0.152941,0.156863}%
\pgfsetstrokecolor{currentstroke}%
\pgfsetdash{}{0pt}%
\pgfpathmoveto{\pgfqpoint{2.270906in}{3.256315in}}%
\pgfpathcurveto{\pgfqpoint{2.281956in}{3.256315in}}{\pgfqpoint{2.292555in}{3.260705in}}{\pgfqpoint{2.300369in}{3.268518in}}%
\pgfpathcurveto{\pgfqpoint{2.308182in}{3.276332in}}{\pgfqpoint{2.312573in}{3.286931in}}{\pgfqpoint{2.312573in}{3.297981in}}%
\pgfpathcurveto{\pgfqpoint{2.312573in}{3.309031in}}{\pgfqpoint{2.308182in}{3.319630in}}{\pgfqpoint{2.300369in}{3.327444in}}%
\pgfpathcurveto{\pgfqpoint{2.292555in}{3.335258in}}{\pgfqpoint{2.281956in}{3.339648in}}{\pgfqpoint{2.270906in}{3.339648in}}%
\pgfpathcurveto{\pgfqpoint{2.259856in}{3.339648in}}{\pgfqpoint{2.249257in}{3.335258in}}{\pgfqpoint{2.241443in}{3.327444in}}%
\pgfpathcurveto{\pgfqpoint{2.233629in}{3.319630in}}{\pgfqpoint{2.229239in}{3.309031in}}{\pgfqpoint{2.229239in}{3.297981in}}%
\pgfpathcurveto{\pgfqpoint{2.229239in}{3.286931in}}{\pgfqpoint{2.233629in}{3.276332in}}{\pgfqpoint{2.241443in}{3.268518in}}%
\pgfpathcurveto{\pgfqpoint{2.249257in}{3.260705in}}{\pgfqpoint{2.259856in}{3.256315in}}{\pgfqpoint{2.270906in}{3.256315in}}%
\pgfpathclose%
\pgfusepath{stroke,fill}%
\end{pgfscope}%
\begin{pgfscope}%
\pgfpathrectangle{\pgfqpoint{0.787074in}{0.548769in}}{\pgfqpoint{5.062926in}{3.102590in}}%
\pgfusepath{clip}%
\pgfsetbuttcap%
\pgfsetroundjoin%
\definecolor{currentfill}{rgb}{1.000000,0.498039,0.054902}%
\pgfsetfillcolor{currentfill}%
\pgfsetlinewidth{1.003750pt}%
\definecolor{currentstroke}{rgb}{1.000000,0.498039,0.054902}%
\pgfsetstrokecolor{currentstroke}%
\pgfsetdash{}{0pt}%
\pgfpathmoveto{\pgfqpoint{1.313083in}{2.697852in}}%
\pgfpathcurveto{\pgfqpoint{1.324134in}{2.697852in}}{\pgfqpoint{1.334733in}{2.702242in}}{\pgfqpoint{1.342546in}{2.710056in}}%
\pgfpathcurveto{\pgfqpoint{1.350360in}{2.717869in}}{\pgfqpoint{1.354750in}{2.728468in}}{\pgfqpoint{1.354750in}{2.739518in}}%
\pgfpathcurveto{\pgfqpoint{1.354750in}{2.750568in}}{\pgfqpoint{1.350360in}{2.761167in}}{\pgfqpoint{1.342546in}{2.768981in}}%
\pgfpathcurveto{\pgfqpoint{1.334733in}{2.776795in}}{\pgfqpoint{1.324134in}{2.781185in}}{\pgfqpoint{1.313083in}{2.781185in}}%
\pgfpathcurveto{\pgfqpoint{1.302033in}{2.781185in}}{\pgfqpoint{1.291434in}{2.776795in}}{\pgfqpoint{1.283621in}{2.768981in}}%
\pgfpathcurveto{\pgfqpoint{1.275807in}{2.761167in}}{\pgfqpoint{1.271417in}{2.750568in}}{\pgfqpoint{1.271417in}{2.739518in}}%
\pgfpathcurveto{\pgfqpoint{1.271417in}{2.728468in}}{\pgfqpoint{1.275807in}{2.717869in}}{\pgfqpoint{1.283621in}{2.710056in}}%
\pgfpathcurveto{\pgfqpoint{1.291434in}{2.702242in}}{\pgfqpoint{1.302033in}{2.697852in}}{\pgfqpoint{1.313083in}{2.697852in}}%
\pgfpathclose%
\pgfusepath{stroke,fill}%
\end{pgfscope}%
\begin{pgfscope}%
\pgfpathrectangle{\pgfqpoint{0.787074in}{0.548769in}}{\pgfqpoint{5.062926in}{3.102590in}}%
\pgfusepath{clip}%
\pgfsetbuttcap%
\pgfsetroundjoin%
\definecolor{currentfill}{rgb}{0.121569,0.466667,0.705882}%
\pgfsetfillcolor{currentfill}%
\pgfsetlinewidth{1.003750pt}%
\definecolor{currentstroke}{rgb}{0.121569,0.466667,0.705882}%
\pgfsetstrokecolor{currentstroke}%
\pgfsetdash{}{0pt}%
\pgfpathmoveto{\pgfqpoint{1.294849in}{1.421237in}}%
\pgfpathcurveto{\pgfqpoint{1.305899in}{1.421237in}}{\pgfqpoint{1.316498in}{1.425628in}}{\pgfqpoint{1.324311in}{1.433441in}}%
\pgfpathcurveto{\pgfqpoint{1.332125in}{1.441255in}}{\pgfqpoint{1.336515in}{1.451854in}}{\pgfqpoint{1.336515in}{1.462904in}}%
\pgfpathcurveto{\pgfqpoint{1.336515in}{1.473954in}}{\pgfqpoint{1.332125in}{1.484553in}}{\pgfqpoint{1.324311in}{1.492367in}}%
\pgfpathcurveto{\pgfqpoint{1.316498in}{1.500180in}}{\pgfqpoint{1.305899in}{1.504571in}}{\pgfqpoint{1.294849in}{1.504571in}}%
\pgfpathcurveto{\pgfqpoint{1.283798in}{1.504571in}}{\pgfqpoint{1.273199in}{1.500180in}}{\pgfqpoint{1.265386in}{1.492367in}}%
\pgfpathcurveto{\pgfqpoint{1.257572in}{1.484553in}}{\pgfqpoint{1.253182in}{1.473954in}}{\pgfqpoint{1.253182in}{1.462904in}}%
\pgfpathcurveto{\pgfqpoint{1.253182in}{1.451854in}}{\pgfqpoint{1.257572in}{1.441255in}}{\pgfqpoint{1.265386in}{1.433441in}}%
\pgfpathcurveto{\pgfqpoint{1.273199in}{1.425628in}}{\pgfqpoint{1.283798in}{1.421237in}}{\pgfqpoint{1.294849in}{1.421237in}}%
\pgfpathclose%
\pgfusepath{stroke,fill}%
\end{pgfscope}%
\begin{pgfscope}%
\pgfpathrectangle{\pgfqpoint{0.787074in}{0.548769in}}{\pgfqpoint{5.062926in}{3.102590in}}%
\pgfusepath{clip}%
\pgfsetbuttcap%
\pgfsetroundjoin%
\definecolor{currentfill}{rgb}{0.121569,0.466667,0.705882}%
\pgfsetfillcolor{currentfill}%
\pgfsetlinewidth{1.003750pt}%
\definecolor{currentstroke}{rgb}{0.121569,0.466667,0.705882}%
\pgfsetstrokecolor{currentstroke}%
\pgfsetdash{}{0pt}%
\pgfpathmoveto{\pgfqpoint{2.166370in}{1.410498in}}%
\pgfpathcurveto{\pgfqpoint{2.177420in}{1.410498in}}{\pgfqpoint{2.188019in}{1.414888in}}{\pgfqpoint{2.195833in}{1.422702in}}%
\pgfpathcurveto{\pgfqpoint{2.203647in}{1.430515in}}{\pgfqpoint{2.208037in}{1.441114in}}{\pgfqpoint{2.208037in}{1.452164in}}%
\pgfpathcurveto{\pgfqpoint{2.208037in}{1.463214in}}{\pgfqpoint{2.203647in}{1.473813in}}{\pgfqpoint{2.195833in}{1.481627in}}%
\pgfpathcurveto{\pgfqpoint{2.188019in}{1.489441in}}{\pgfqpoint{2.177420in}{1.493831in}}{\pgfqpoint{2.166370in}{1.493831in}}%
\pgfpathcurveto{\pgfqpoint{2.155320in}{1.493831in}}{\pgfqpoint{2.144721in}{1.489441in}}{\pgfqpoint{2.136908in}{1.481627in}}%
\pgfpathcurveto{\pgfqpoint{2.129094in}{1.473813in}}{\pgfqpoint{2.124704in}{1.463214in}}{\pgfqpoint{2.124704in}{1.452164in}}%
\pgfpathcurveto{\pgfqpoint{2.124704in}{1.441114in}}{\pgfqpoint{2.129094in}{1.430515in}}{\pgfqpoint{2.136908in}{1.422702in}}%
\pgfpathcurveto{\pgfqpoint{2.144721in}{1.414888in}}{\pgfqpoint{2.155320in}{1.410498in}}{\pgfqpoint{2.166370in}{1.410498in}}%
\pgfpathclose%
\pgfusepath{stroke,fill}%
\end{pgfscope}%
\begin{pgfscope}%
\pgfpathrectangle{\pgfqpoint{0.787074in}{0.548769in}}{\pgfqpoint{5.062926in}{3.102590in}}%
\pgfusepath{clip}%
\pgfsetbuttcap%
\pgfsetroundjoin%
\definecolor{currentfill}{rgb}{0.121569,0.466667,0.705882}%
\pgfsetfillcolor{currentfill}%
\pgfsetlinewidth{1.003750pt}%
\definecolor{currentstroke}{rgb}{0.121569,0.466667,0.705882}%
\pgfsetstrokecolor{currentstroke}%
\pgfsetdash{}{0pt}%
\pgfpathmoveto{\pgfqpoint{1.775640in}{1.127209in}}%
\pgfpathcurveto{\pgfqpoint{1.786690in}{1.127209in}}{\pgfqpoint{1.797289in}{1.131599in}}{\pgfqpoint{1.805103in}{1.139413in}}%
\pgfpathcurveto{\pgfqpoint{1.812916in}{1.147226in}}{\pgfqpoint{1.817307in}{1.157825in}}{\pgfqpoint{1.817307in}{1.168876in}}%
\pgfpathcurveto{\pgfqpoint{1.817307in}{1.179926in}}{\pgfqpoint{1.812916in}{1.190525in}}{\pgfqpoint{1.805103in}{1.198338in}}%
\pgfpathcurveto{\pgfqpoint{1.797289in}{1.206152in}}{\pgfqpoint{1.786690in}{1.210542in}}{\pgfqpoint{1.775640in}{1.210542in}}%
\pgfpathcurveto{\pgfqpoint{1.764590in}{1.210542in}}{\pgfqpoint{1.753991in}{1.206152in}}{\pgfqpoint{1.746177in}{1.198338in}}%
\pgfpathcurveto{\pgfqpoint{1.738364in}{1.190525in}}{\pgfqpoint{1.733973in}{1.179926in}}{\pgfqpoint{1.733973in}{1.168876in}}%
\pgfpathcurveto{\pgfqpoint{1.733973in}{1.157825in}}{\pgfqpoint{1.738364in}{1.147226in}}{\pgfqpoint{1.746177in}{1.139413in}}%
\pgfpathcurveto{\pgfqpoint{1.753991in}{1.131599in}}{\pgfqpoint{1.764590in}{1.127209in}}{\pgfqpoint{1.775640in}{1.127209in}}%
\pgfpathclose%
\pgfusepath{stroke,fill}%
\end{pgfscope}%
\begin{pgfscope}%
\pgfpathrectangle{\pgfqpoint{0.787074in}{0.548769in}}{\pgfqpoint{5.062926in}{3.102590in}}%
\pgfusepath{clip}%
\pgfsetbuttcap%
\pgfsetroundjoin%
\definecolor{currentfill}{rgb}{0.121569,0.466667,0.705882}%
\pgfsetfillcolor{currentfill}%
\pgfsetlinewidth{1.003750pt}%
\definecolor{currentstroke}{rgb}{0.121569,0.466667,0.705882}%
\pgfsetstrokecolor{currentstroke}%
\pgfsetdash{}{0pt}%
\pgfpathmoveto{\pgfqpoint{2.247478in}{0.894035in}}%
\pgfpathcurveto{\pgfqpoint{2.258528in}{0.894035in}}{\pgfqpoint{2.269127in}{0.898425in}}{\pgfqpoint{2.276940in}{0.906239in}}%
\pgfpathcurveto{\pgfqpoint{2.284754in}{0.914052in}}{\pgfqpoint{2.289144in}{0.924651in}}{\pgfqpoint{2.289144in}{0.935701in}}%
\pgfpathcurveto{\pgfqpoint{2.289144in}{0.946752in}}{\pgfqpoint{2.284754in}{0.957351in}}{\pgfqpoint{2.276940in}{0.965164in}}%
\pgfpathcurveto{\pgfqpoint{2.269127in}{0.972978in}}{\pgfqpoint{2.258528in}{0.977368in}}{\pgfqpoint{2.247478in}{0.977368in}}%
\pgfpathcurveto{\pgfqpoint{2.236427in}{0.977368in}}{\pgfqpoint{2.225828in}{0.972978in}}{\pgfqpoint{2.218015in}{0.965164in}}%
\pgfpathcurveto{\pgfqpoint{2.210201in}{0.957351in}}{\pgfqpoint{2.205811in}{0.946752in}}{\pgfqpoint{2.205811in}{0.935701in}}%
\pgfpathcurveto{\pgfqpoint{2.205811in}{0.924651in}}{\pgfqpoint{2.210201in}{0.914052in}}{\pgfqpoint{2.218015in}{0.906239in}}%
\pgfpathcurveto{\pgfqpoint{2.225828in}{0.898425in}}{\pgfqpoint{2.236427in}{0.894035in}}{\pgfqpoint{2.247478in}{0.894035in}}%
\pgfpathclose%
\pgfusepath{stroke,fill}%
\end{pgfscope}%
\begin{pgfscope}%
\pgfpathrectangle{\pgfqpoint{0.787074in}{0.548769in}}{\pgfqpoint{5.062926in}{3.102590in}}%
\pgfusepath{clip}%
\pgfsetbuttcap%
\pgfsetroundjoin%
\definecolor{currentfill}{rgb}{0.839216,0.152941,0.156863}%
\pgfsetfillcolor{currentfill}%
\pgfsetlinewidth{1.003750pt}%
\definecolor{currentstroke}{rgb}{0.839216,0.152941,0.156863}%
\pgfsetstrokecolor{currentstroke}%
\pgfsetdash{}{0pt}%
\pgfpathmoveto{\pgfqpoint{2.154824in}{3.187559in}}%
\pgfpathcurveto{\pgfqpoint{2.165874in}{3.187559in}}{\pgfqpoint{2.176473in}{3.191949in}}{\pgfqpoint{2.184286in}{3.199763in}}%
\pgfpathcurveto{\pgfqpoint{2.192100in}{3.207576in}}{\pgfqpoint{2.196490in}{3.218175in}}{\pgfqpoint{2.196490in}{3.229226in}}%
\pgfpathcurveto{\pgfqpoint{2.196490in}{3.240276in}}{\pgfqpoint{2.192100in}{3.250875in}}{\pgfqpoint{2.184286in}{3.258688in}}%
\pgfpathcurveto{\pgfqpoint{2.176473in}{3.266502in}}{\pgfqpoint{2.165874in}{3.270892in}}{\pgfqpoint{2.154824in}{3.270892in}}%
\pgfpathcurveto{\pgfqpoint{2.143774in}{3.270892in}}{\pgfqpoint{2.133175in}{3.266502in}}{\pgfqpoint{2.125361in}{3.258688in}}%
\pgfpathcurveto{\pgfqpoint{2.117547in}{3.250875in}}{\pgfqpoint{2.113157in}{3.240276in}}{\pgfqpoint{2.113157in}{3.229226in}}%
\pgfpathcurveto{\pgfqpoint{2.113157in}{3.218175in}}{\pgfqpoint{2.117547in}{3.207576in}}{\pgfqpoint{2.125361in}{3.199763in}}%
\pgfpathcurveto{\pgfqpoint{2.133175in}{3.191949in}}{\pgfqpoint{2.143774in}{3.187559in}}{\pgfqpoint{2.154824in}{3.187559in}}%
\pgfpathclose%
\pgfusepath{stroke,fill}%
\end{pgfscope}%
\begin{pgfscope}%
\pgfsetbuttcap%
\pgfsetroundjoin%
\definecolor{currentfill}{rgb}{0.000000,0.000000,0.000000}%
\pgfsetfillcolor{currentfill}%
\pgfsetlinewidth{0.803000pt}%
\definecolor{currentstroke}{rgb}{0.000000,0.000000,0.000000}%
\pgfsetstrokecolor{currentstroke}%
\pgfsetdash{}{0pt}%
\pgfsys@defobject{currentmarker}{\pgfqpoint{0.000000in}{-0.048611in}}{\pgfqpoint{0.000000in}{0.000000in}}{%
\pgfpathmoveto{\pgfqpoint{0.000000in}{0.000000in}}%
\pgfpathlineto{\pgfqpoint{0.000000in}{-0.048611in}}%
\pgfusepath{stroke,fill}%
}%
\begin{pgfscope}%
\pgfsys@transformshift{1.005367in}{0.548769in}%
\pgfsys@useobject{currentmarker}{}%
\end{pgfscope}%
\end{pgfscope}%
\begin{pgfscope}%
\definecolor{textcolor}{rgb}{0.000000,0.000000,0.000000}%
\pgfsetstrokecolor{textcolor}%
\pgfsetfillcolor{textcolor}%
\pgftext[x=1.005367in,y=0.451547in,,top]{\color{textcolor}\sffamily\fontsize{10.000000}{12.000000}\selectfont \(\displaystyle {0.0}\)}%
\end{pgfscope}%
\begin{pgfscope}%
\pgfsetbuttcap%
\pgfsetroundjoin%
\definecolor{currentfill}{rgb}{0.000000,0.000000,0.000000}%
\pgfsetfillcolor{currentfill}%
\pgfsetlinewidth{0.803000pt}%
\definecolor{currentstroke}{rgb}{0.000000,0.000000,0.000000}%
\pgfsetstrokecolor{currentstroke}%
\pgfsetdash{}{0pt}%
\pgfsys@defobject{currentmarker}{\pgfqpoint{0.000000in}{-0.048611in}}{\pgfqpoint{0.000000in}{0.000000in}}{%
\pgfpathmoveto{\pgfqpoint{0.000000in}{0.000000in}}%
\pgfpathlineto{\pgfqpoint{0.000000in}{-0.048611in}}%
\pgfusepath{stroke,fill}%
}%
\begin{pgfscope}%
\pgfsys@transformshift{1.775910in}{0.548769in}%
\pgfsys@useobject{currentmarker}{}%
\end{pgfscope}%
\end{pgfscope}%
\begin{pgfscope}%
\definecolor{textcolor}{rgb}{0.000000,0.000000,0.000000}%
\pgfsetstrokecolor{textcolor}%
\pgfsetfillcolor{textcolor}%
\pgftext[x=1.775910in,y=0.451547in,,top]{\color{textcolor}\sffamily\fontsize{10.000000}{12.000000}\selectfont \(\displaystyle {0.2}\)}%
\end{pgfscope}%
\begin{pgfscope}%
\pgfsetbuttcap%
\pgfsetroundjoin%
\definecolor{currentfill}{rgb}{0.000000,0.000000,0.000000}%
\pgfsetfillcolor{currentfill}%
\pgfsetlinewidth{0.803000pt}%
\definecolor{currentstroke}{rgb}{0.000000,0.000000,0.000000}%
\pgfsetstrokecolor{currentstroke}%
\pgfsetdash{}{0pt}%
\pgfsys@defobject{currentmarker}{\pgfqpoint{0.000000in}{-0.048611in}}{\pgfqpoint{0.000000in}{0.000000in}}{%
\pgfpathmoveto{\pgfqpoint{0.000000in}{0.000000in}}%
\pgfpathlineto{\pgfqpoint{0.000000in}{-0.048611in}}%
\pgfusepath{stroke,fill}%
}%
\begin{pgfscope}%
\pgfsys@transformshift{2.546452in}{0.548769in}%
\pgfsys@useobject{currentmarker}{}%
\end{pgfscope}%
\end{pgfscope}%
\begin{pgfscope}%
\definecolor{textcolor}{rgb}{0.000000,0.000000,0.000000}%
\pgfsetstrokecolor{textcolor}%
\pgfsetfillcolor{textcolor}%
\pgftext[x=2.546452in,y=0.451547in,,top]{\color{textcolor}\sffamily\fontsize{10.000000}{12.000000}\selectfont \(\displaystyle {0.4}\)}%
\end{pgfscope}%
\begin{pgfscope}%
\pgfsetbuttcap%
\pgfsetroundjoin%
\definecolor{currentfill}{rgb}{0.000000,0.000000,0.000000}%
\pgfsetfillcolor{currentfill}%
\pgfsetlinewidth{0.803000pt}%
\definecolor{currentstroke}{rgb}{0.000000,0.000000,0.000000}%
\pgfsetstrokecolor{currentstroke}%
\pgfsetdash{}{0pt}%
\pgfsys@defobject{currentmarker}{\pgfqpoint{0.000000in}{-0.048611in}}{\pgfqpoint{0.000000in}{0.000000in}}{%
\pgfpathmoveto{\pgfqpoint{0.000000in}{0.000000in}}%
\pgfpathlineto{\pgfqpoint{0.000000in}{-0.048611in}}%
\pgfusepath{stroke,fill}%
}%
\begin{pgfscope}%
\pgfsys@transformshift{3.316994in}{0.548769in}%
\pgfsys@useobject{currentmarker}{}%
\end{pgfscope}%
\end{pgfscope}%
\begin{pgfscope}%
\definecolor{textcolor}{rgb}{0.000000,0.000000,0.000000}%
\pgfsetstrokecolor{textcolor}%
\pgfsetfillcolor{textcolor}%
\pgftext[x=3.316994in,y=0.451547in,,top]{\color{textcolor}\sffamily\fontsize{10.000000}{12.000000}\selectfont \(\displaystyle {0.6}\)}%
\end{pgfscope}%
\begin{pgfscope}%
\pgfsetbuttcap%
\pgfsetroundjoin%
\definecolor{currentfill}{rgb}{0.000000,0.000000,0.000000}%
\pgfsetfillcolor{currentfill}%
\pgfsetlinewidth{0.803000pt}%
\definecolor{currentstroke}{rgb}{0.000000,0.000000,0.000000}%
\pgfsetstrokecolor{currentstroke}%
\pgfsetdash{}{0pt}%
\pgfsys@defobject{currentmarker}{\pgfqpoint{0.000000in}{-0.048611in}}{\pgfqpoint{0.000000in}{0.000000in}}{%
\pgfpathmoveto{\pgfqpoint{0.000000in}{0.000000in}}%
\pgfpathlineto{\pgfqpoint{0.000000in}{-0.048611in}}%
\pgfusepath{stroke,fill}%
}%
\begin{pgfscope}%
\pgfsys@transformshift{4.087536in}{0.548769in}%
\pgfsys@useobject{currentmarker}{}%
\end{pgfscope}%
\end{pgfscope}%
\begin{pgfscope}%
\definecolor{textcolor}{rgb}{0.000000,0.000000,0.000000}%
\pgfsetstrokecolor{textcolor}%
\pgfsetfillcolor{textcolor}%
\pgftext[x=4.087536in,y=0.451547in,,top]{\color{textcolor}\sffamily\fontsize{10.000000}{12.000000}\selectfont \(\displaystyle {0.8}\)}%
\end{pgfscope}%
\begin{pgfscope}%
\pgfsetbuttcap%
\pgfsetroundjoin%
\definecolor{currentfill}{rgb}{0.000000,0.000000,0.000000}%
\pgfsetfillcolor{currentfill}%
\pgfsetlinewidth{0.803000pt}%
\definecolor{currentstroke}{rgb}{0.000000,0.000000,0.000000}%
\pgfsetstrokecolor{currentstroke}%
\pgfsetdash{}{0pt}%
\pgfsys@defobject{currentmarker}{\pgfqpoint{0.000000in}{-0.048611in}}{\pgfqpoint{0.000000in}{0.000000in}}{%
\pgfpathmoveto{\pgfqpoint{0.000000in}{0.000000in}}%
\pgfpathlineto{\pgfqpoint{0.000000in}{-0.048611in}}%
\pgfusepath{stroke,fill}%
}%
\begin{pgfscope}%
\pgfsys@transformshift{4.858078in}{0.548769in}%
\pgfsys@useobject{currentmarker}{}%
\end{pgfscope}%
\end{pgfscope}%
\begin{pgfscope}%
\definecolor{textcolor}{rgb}{0.000000,0.000000,0.000000}%
\pgfsetstrokecolor{textcolor}%
\pgfsetfillcolor{textcolor}%
\pgftext[x=4.858078in,y=0.451547in,,top]{\color{textcolor}\sffamily\fontsize{10.000000}{12.000000}\selectfont \(\displaystyle {1.0}\)}%
\end{pgfscope}%
\begin{pgfscope}%
\pgfsetbuttcap%
\pgfsetroundjoin%
\definecolor{currentfill}{rgb}{0.000000,0.000000,0.000000}%
\pgfsetfillcolor{currentfill}%
\pgfsetlinewidth{0.803000pt}%
\definecolor{currentstroke}{rgb}{0.000000,0.000000,0.000000}%
\pgfsetstrokecolor{currentstroke}%
\pgfsetdash{}{0pt}%
\pgfsys@defobject{currentmarker}{\pgfqpoint{0.000000in}{-0.048611in}}{\pgfqpoint{0.000000in}{0.000000in}}{%
\pgfpathmoveto{\pgfqpoint{0.000000in}{0.000000in}}%
\pgfpathlineto{\pgfqpoint{0.000000in}{-0.048611in}}%
\pgfusepath{stroke,fill}%
}%
\begin{pgfscope}%
\pgfsys@transformshift{5.628620in}{0.548769in}%
\pgfsys@useobject{currentmarker}{}%
\end{pgfscope}%
\end{pgfscope}%
\begin{pgfscope}%
\definecolor{textcolor}{rgb}{0.000000,0.000000,0.000000}%
\pgfsetstrokecolor{textcolor}%
\pgfsetfillcolor{textcolor}%
\pgftext[x=5.628620in,y=0.451547in,,top]{\color{textcolor}\sffamily\fontsize{10.000000}{12.000000}\selectfont \(\displaystyle {1.2}\)}%
\end{pgfscope}%
\begin{pgfscope}%
\definecolor{textcolor}{rgb}{0.000000,0.000000,0.000000}%
\pgfsetstrokecolor{textcolor}%
\pgfsetfillcolor{textcolor}%
\pgftext[x=3.318537in,y=0.272658in,,top]{\color{textcolor}\sffamily\fontsize{10.000000}{12.000000}\selectfont Statements}%
\end{pgfscope}%
\begin{pgfscope}%
\definecolor{textcolor}{rgb}{0.000000,0.000000,0.000000}%
\pgfsetstrokecolor{textcolor}%
\pgfsetfillcolor{textcolor}%
\pgftext[x=5.850000in,y=0.286547in,right,top]{\color{textcolor}\sffamily\fontsize{10.000000}{12.000000}\selectfont \(\displaystyle \times{10^{6}}{}\)}%
\end{pgfscope}%
\begin{pgfscope}%
\pgfsetbuttcap%
\pgfsetroundjoin%
\definecolor{currentfill}{rgb}{0.000000,0.000000,0.000000}%
\pgfsetfillcolor{currentfill}%
\pgfsetlinewidth{0.803000pt}%
\definecolor{currentstroke}{rgb}{0.000000,0.000000,0.000000}%
\pgfsetstrokecolor{currentstroke}%
\pgfsetdash{}{0pt}%
\pgfsys@defobject{currentmarker}{\pgfqpoint{-0.048611in}{0.000000in}}{\pgfqpoint{0.000000in}{0.000000in}}{%
\pgfpathmoveto{\pgfqpoint{0.000000in}{0.000000in}}%
\pgfpathlineto{\pgfqpoint{-0.048611in}{0.000000in}}%
\pgfusepath{stroke,fill}%
}%
\begin{pgfscope}%
\pgfsys@transformshift{0.787074in}{0.689795in}%
\pgfsys@useobject{currentmarker}{}%
\end{pgfscope}%
\end{pgfscope}%
\begin{pgfscope}%
\definecolor{textcolor}{rgb}{0.000000,0.000000,0.000000}%
\pgfsetstrokecolor{textcolor}%
\pgfsetfillcolor{textcolor}%
\pgftext[x=0.620407in, y=0.641601in, left, base]{\color{textcolor}\sffamily\fontsize{10.000000}{12.000000}\selectfont \(\displaystyle {0}\)}%
\end{pgfscope}%
\begin{pgfscope}%
\pgfsetbuttcap%
\pgfsetroundjoin%
\definecolor{currentfill}{rgb}{0.000000,0.000000,0.000000}%
\pgfsetfillcolor{currentfill}%
\pgfsetlinewidth{0.803000pt}%
\definecolor{currentstroke}{rgb}{0.000000,0.000000,0.000000}%
\pgfsetstrokecolor{currentstroke}%
\pgfsetdash{}{0pt}%
\pgfsys@defobject{currentmarker}{\pgfqpoint{-0.048611in}{0.000000in}}{\pgfqpoint{0.000000in}{0.000000in}}{%
\pgfpathmoveto{\pgfqpoint{0.000000in}{0.000000in}}%
\pgfpathlineto{\pgfqpoint{-0.048611in}{0.000000in}}%
\pgfusepath{stroke,fill}%
}%
\begin{pgfscope}%
\pgfsys@transformshift{0.787074in}{1.060560in}%
\pgfsys@useobject{currentmarker}{}%
\end{pgfscope}%
\end{pgfscope}%
\begin{pgfscope}%
\definecolor{textcolor}{rgb}{0.000000,0.000000,0.000000}%
\pgfsetstrokecolor{textcolor}%
\pgfsetfillcolor{textcolor}%
\pgftext[x=0.412073in, y=1.012365in, left, base]{\color{textcolor}\sffamily\fontsize{10.000000}{12.000000}\selectfont \(\displaystyle {2500}\)}%
\end{pgfscope}%
\begin{pgfscope}%
\pgfsetbuttcap%
\pgfsetroundjoin%
\definecolor{currentfill}{rgb}{0.000000,0.000000,0.000000}%
\pgfsetfillcolor{currentfill}%
\pgfsetlinewidth{0.803000pt}%
\definecolor{currentstroke}{rgb}{0.000000,0.000000,0.000000}%
\pgfsetstrokecolor{currentstroke}%
\pgfsetdash{}{0pt}%
\pgfsys@defobject{currentmarker}{\pgfqpoint{-0.048611in}{0.000000in}}{\pgfqpoint{0.000000in}{0.000000in}}{%
\pgfpathmoveto{\pgfqpoint{0.000000in}{0.000000in}}%
\pgfpathlineto{\pgfqpoint{-0.048611in}{0.000000in}}%
\pgfusepath{stroke,fill}%
}%
\begin{pgfscope}%
\pgfsys@transformshift{0.787074in}{1.431324in}%
\pgfsys@useobject{currentmarker}{}%
\end{pgfscope}%
\end{pgfscope}%
\begin{pgfscope}%
\definecolor{textcolor}{rgb}{0.000000,0.000000,0.000000}%
\pgfsetstrokecolor{textcolor}%
\pgfsetfillcolor{textcolor}%
\pgftext[x=0.412073in, y=1.383129in, left, base]{\color{textcolor}\sffamily\fontsize{10.000000}{12.000000}\selectfont \(\displaystyle {5000}\)}%
\end{pgfscope}%
\begin{pgfscope}%
\pgfsetbuttcap%
\pgfsetroundjoin%
\definecolor{currentfill}{rgb}{0.000000,0.000000,0.000000}%
\pgfsetfillcolor{currentfill}%
\pgfsetlinewidth{0.803000pt}%
\definecolor{currentstroke}{rgb}{0.000000,0.000000,0.000000}%
\pgfsetstrokecolor{currentstroke}%
\pgfsetdash{}{0pt}%
\pgfsys@defobject{currentmarker}{\pgfqpoint{-0.048611in}{0.000000in}}{\pgfqpoint{0.000000in}{0.000000in}}{%
\pgfpathmoveto{\pgfqpoint{0.000000in}{0.000000in}}%
\pgfpathlineto{\pgfqpoint{-0.048611in}{0.000000in}}%
\pgfusepath{stroke,fill}%
}%
\begin{pgfscope}%
\pgfsys@transformshift{0.787074in}{1.802088in}%
\pgfsys@useobject{currentmarker}{}%
\end{pgfscope}%
\end{pgfscope}%
\begin{pgfscope}%
\definecolor{textcolor}{rgb}{0.000000,0.000000,0.000000}%
\pgfsetstrokecolor{textcolor}%
\pgfsetfillcolor{textcolor}%
\pgftext[x=0.412073in, y=1.753894in, left, base]{\color{textcolor}\sffamily\fontsize{10.000000}{12.000000}\selectfont \(\displaystyle {7500}\)}%
\end{pgfscope}%
\begin{pgfscope}%
\pgfsetbuttcap%
\pgfsetroundjoin%
\definecolor{currentfill}{rgb}{0.000000,0.000000,0.000000}%
\pgfsetfillcolor{currentfill}%
\pgfsetlinewidth{0.803000pt}%
\definecolor{currentstroke}{rgb}{0.000000,0.000000,0.000000}%
\pgfsetstrokecolor{currentstroke}%
\pgfsetdash{}{0pt}%
\pgfsys@defobject{currentmarker}{\pgfqpoint{-0.048611in}{0.000000in}}{\pgfqpoint{0.000000in}{0.000000in}}{%
\pgfpathmoveto{\pgfqpoint{0.000000in}{0.000000in}}%
\pgfpathlineto{\pgfqpoint{-0.048611in}{0.000000in}}%
\pgfusepath{stroke,fill}%
}%
\begin{pgfscope}%
\pgfsys@transformshift{0.787074in}{2.172852in}%
\pgfsys@useobject{currentmarker}{}%
\end{pgfscope}%
\end{pgfscope}%
\begin{pgfscope}%
\definecolor{textcolor}{rgb}{0.000000,0.000000,0.000000}%
\pgfsetstrokecolor{textcolor}%
\pgfsetfillcolor{textcolor}%
\pgftext[x=0.342628in, y=2.124658in, left, base]{\color{textcolor}\sffamily\fontsize{10.000000}{12.000000}\selectfont \(\displaystyle {10000}\)}%
\end{pgfscope}%
\begin{pgfscope}%
\pgfsetbuttcap%
\pgfsetroundjoin%
\definecolor{currentfill}{rgb}{0.000000,0.000000,0.000000}%
\pgfsetfillcolor{currentfill}%
\pgfsetlinewidth{0.803000pt}%
\definecolor{currentstroke}{rgb}{0.000000,0.000000,0.000000}%
\pgfsetstrokecolor{currentstroke}%
\pgfsetdash{}{0pt}%
\pgfsys@defobject{currentmarker}{\pgfqpoint{-0.048611in}{0.000000in}}{\pgfqpoint{0.000000in}{0.000000in}}{%
\pgfpathmoveto{\pgfqpoint{0.000000in}{0.000000in}}%
\pgfpathlineto{\pgfqpoint{-0.048611in}{0.000000in}}%
\pgfusepath{stroke,fill}%
}%
\begin{pgfscope}%
\pgfsys@transformshift{0.787074in}{2.543617in}%
\pgfsys@useobject{currentmarker}{}%
\end{pgfscope}%
\end{pgfscope}%
\begin{pgfscope}%
\definecolor{textcolor}{rgb}{0.000000,0.000000,0.000000}%
\pgfsetstrokecolor{textcolor}%
\pgfsetfillcolor{textcolor}%
\pgftext[x=0.342628in, y=2.495422in, left, base]{\color{textcolor}\sffamily\fontsize{10.000000}{12.000000}\selectfont \(\displaystyle {12500}\)}%
\end{pgfscope}%
\begin{pgfscope}%
\pgfsetbuttcap%
\pgfsetroundjoin%
\definecolor{currentfill}{rgb}{0.000000,0.000000,0.000000}%
\pgfsetfillcolor{currentfill}%
\pgfsetlinewidth{0.803000pt}%
\definecolor{currentstroke}{rgb}{0.000000,0.000000,0.000000}%
\pgfsetstrokecolor{currentstroke}%
\pgfsetdash{}{0pt}%
\pgfsys@defobject{currentmarker}{\pgfqpoint{-0.048611in}{0.000000in}}{\pgfqpoint{0.000000in}{0.000000in}}{%
\pgfpathmoveto{\pgfqpoint{0.000000in}{0.000000in}}%
\pgfpathlineto{\pgfqpoint{-0.048611in}{0.000000in}}%
\pgfusepath{stroke,fill}%
}%
\begin{pgfscope}%
\pgfsys@transformshift{0.787074in}{2.914381in}%
\pgfsys@useobject{currentmarker}{}%
\end{pgfscope}%
\end{pgfscope}%
\begin{pgfscope}%
\definecolor{textcolor}{rgb}{0.000000,0.000000,0.000000}%
\pgfsetstrokecolor{textcolor}%
\pgfsetfillcolor{textcolor}%
\pgftext[x=0.342628in, y=2.866187in, left, base]{\color{textcolor}\sffamily\fontsize{10.000000}{12.000000}\selectfont \(\displaystyle {15000}\)}%
\end{pgfscope}%
\begin{pgfscope}%
\pgfsetbuttcap%
\pgfsetroundjoin%
\definecolor{currentfill}{rgb}{0.000000,0.000000,0.000000}%
\pgfsetfillcolor{currentfill}%
\pgfsetlinewidth{0.803000pt}%
\definecolor{currentstroke}{rgb}{0.000000,0.000000,0.000000}%
\pgfsetstrokecolor{currentstroke}%
\pgfsetdash{}{0pt}%
\pgfsys@defobject{currentmarker}{\pgfqpoint{-0.048611in}{0.000000in}}{\pgfqpoint{0.000000in}{0.000000in}}{%
\pgfpathmoveto{\pgfqpoint{0.000000in}{0.000000in}}%
\pgfpathlineto{\pgfqpoint{-0.048611in}{0.000000in}}%
\pgfusepath{stroke,fill}%
}%
\begin{pgfscope}%
\pgfsys@transformshift{0.787074in}{3.285145in}%
\pgfsys@useobject{currentmarker}{}%
\end{pgfscope}%
\end{pgfscope}%
\begin{pgfscope}%
\definecolor{textcolor}{rgb}{0.000000,0.000000,0.000000}%
\pgfsetstrokecolor{textcolor}%
\pgfsetfillcolor{textcolor}%
\pgftext[x=0.342628in, y=3.236951in, left, base]{\color{textcolor}\sffamily\fontsize{10.000000}{12.000000}\selectfont \(\displaystyle {17500}\)}%
\end{pgfscope}%
\begin{pgfscope}%
\definecolor{textcolor}{rgb}{0.000000,0.000000,0.000000}%
\pgfsetstrokecolor{textcolor}%
\pgfsetfillcolor{textcolor}%
\pgftext[x=0.287073in,y=2.100064in,,bottom,rotate=90.000000]{\color{textcolor}\sffamily\fontsize{10.000000}{12.000000}\selectfont Maximum Memory Usage (MB)}%
\end{pgfscope}%
\begin{pgfscope}%
\pgfsetrectcap%
\pgfsetmiterjoin%
\pgfsetlinewidth{0.803000pt}%
\definecolor{currentstroke}{rgb}{0.000000,0.000000,0.000000}%
\pgfsetstrokecolor{currentstroke}%
\pgfsetdash{}{0pt}%
\pgfpathmoveto{\pgfqpoint{0.787074in}{0.548769in}}%
\pgfpathlineto{\pgfqpoint{0.787074in}{3.651359in}}%
\pgfusepath{stroke}%
\end{pgfscope}%
\begin{pgfscope}%
\pgfsetrectcap%
\pgfsetmiterjoin%
\pgfsetlinewidth{0.803000pt}%
\definecolor{currentstroke}{rgb}{0.000000,0.000000,0.000000}%
\pgfsetstrokecolor{currentstroke}%
\pgfsetdash{}{0pt}%
\pgfpathmoveto{\pgfqpoint{5.850000in}{0.548769in}}%
\pgfpathlineto{\pgfqpoint{5.850000in}{3.651359in}}%
\pgfusepath{stroke}%
\end{pgfscope}%
\begin{pgfscope}%
\pgfsetrectcap%
\pgfsetmiterjoin%
\pgfsetlinewidth{0.803000pt}%
\definecolor{currentstroke}{rgb}{0.000000,0.000000,0.000000}%
\pgfsetstrokecolor{currentstroke}%
\pgfsetdash{}{0pt}%
\pgfpathmoveto{\pgfqpoint{0.787074in}{0.548769in}}%
\pgfpathlineto{\pgfqpoint{5.850000in}{0.548769in}}%
\pgfusepath{stroke}%
\end{pgfscope}%
\begin{pgfscope}%
\pgfsetrectcap%
\pgfsetmiterjoin%
\pgfsetlinewidth{0.803000pt}%
\definecolor{currentstroke}{rgb}{0.000000,0.000000,0.000000}%
\pgfsetstrokecolor{currentstroke}%
\pgfsetdash{}{0pt}%
\pgfpathmoveto{\pgfqpoint{0.787074in}{3.651359in}}%
\pgfpathlineto{\pgfqpoint{5.850000in}{3.651359in}}%
\pgfusepath{stroke}%
\end{pgfscope}%
\begin{pgfscope}%
\definecolor{textcolor}{rgb}{0.000000,0.000000,0.000000}%
\pgfsetstrokecolor{textcolor}%
\pgfsetfillcolor{textcolor}%
\pgftext[x=3.318537in,y=3.734692in,,base]{\color{textcolor}\sffamily\fontsize{12.000000}{14.400000}\selectfont Backwards}%
\end{pgfscope}%
\begin{pgfscope}%
\pgfsetbuttcap%
\pgfsetmiterjoin%
\definecolor{currentfill}{rgb}{1.000000,1.000000,1.000000}%
\pgfsetfillcolor{currentfill}%
\pgfsetfillopacity{0.800000}%
\pgfsetlinewidth{1.003750pt}%
\definecolor{currentstroke}{rgb}{0.800000,0.800000,0.800000}%
\pgfsetstrokecolor{currentstroke}%
\pgfsetstrokeopacity{0.800000}%
\pgfsetdash{}{0pt}%
\pgfpathmoveto{\pgfqpoint{4.300417in}{2.957886in}}%
\pgfpathlineto{\pgfqpoint{5.752778in}{2.957886in}}%
\pgfpathquadraticcurveto{\pgfqpoint{5.780556in}{2.957886in}}{\pgfqpoint{5.780556in}{2.985664in}}%
\pgfpathlineto{\pgfqpoint{5.780556in}{3.554136in}}%
\pgfpathquadraticcurveto{\pgfqpoint{5.780556in}{3.581914in}}{\pgfqpoint{5.752778in}{3.581914in}}%
\pgfpathlineto{\pgfqpoint{4.300417in}{3.581914in}}%
\pgfpathquadraticcurveto{\pgfqpoint{4.272639in}{3.581914in}}{\pgfqpoint{4.272639in}{3.554136in}}%
\pgfpathlineto{\pgfqpoint{4.272639in}{2.985664in}}%
\pgfpathquadraticcurveto{\pgfqpoint{4.272639in}{2.957886in}}{\pgfqpoint{4.300417in}{2.957886in}}%
\pgfpathclose%
\pgfusepath{stroke,fill}%
\end{pgfscope}%
\begin{pgfscope}%
\pgfsetbuttcap%
\pgfsetroundjoin%
\definecolor{currentfill}{rgb}{0.121569,0.466667,0.705882}%
\pgfsetfillcolor{currentfill}%
\pgfsetlinewidth{1.003750pt}%
\definecolor{currentstroke}{rgb}{0.121569,0.466667,0.705882}%
\pgfsetstrokecolor{currentstroke}%
\pgfsetdash{}{0pt}%
\pgfsys@defobject{currentmarker}{\pgfqpoint{-0.034722in}{-0.034722in}}{\pgfqpoint{0.034722in}{0.034722in}}{%
\pgfpathmoveto{\pgfqpoint{0.000000in}{-0.034722in}}%
\pgfpathcurveto{\pgfqpoint{0.009208in}{-0.034722in}}{\pgfqpoint{0.018041in}{-0.031064in}}{\pgfqpoint{0.024552in}{-0.024552in}}%
\pgfpathcurveto{\pgfqpoint{0.031064in}{-0.018041in}}{\pgfqpoint{0.034722in}{-0.009208in}}{\pgfqpoint{0.034722in}{0.000000in}}%
\pgfpathcurveto{\pgfqpoint{0.034722in}{0.009208in}}{\pgfqpoint{0.031064in}{0.018041in}}{\pgfqpoint{0.024552in}{0.024552in}}%
\pgfpathcurveto{\pgfqpoint{0.018041in}{0.031064in}}{\pgfqpoint{0.009208in}{0.034722in}}{\pgfqpoint{0.000000in}{0.034722in}}%
\pgfpathcurveto{\pgfqpoint{-0.009208in}{0.034722in}}{\pgfqpoint{-0.018041in}{0.031064in}}{\pgfqpoint{-0.024552in}{0.024552in}}%
\pgfpathcurveto{\pgfqpoint{-0.031064in}{0.018041in}}{\pgfqpoint{-0.034722in}{0.009208in}}{\pgfqpoint{-0.034722in}{0.000000in}}%
\pgfpathcurveto{\pgfqpoint{-0.034722in}{-0.009208in}}{\pgfqpoint{-0.031064in}{-0.018041in}}{\pgfqpoint{-0.024552in}{-0.024552in}}%
\pgfpathcurveto{\pgfqpoint{-0.018041in}{-0.031064in}}{\pgfqpoint{-0.009208in}{-0.034722in}}{\pgfqpoint{0.000000in}{-0.034722in}}%
\pgfpathclose%
\pgfusepath{stroke,fill}%
}%
\begin{pgfscope}%
\pgfsys@transformshift{4.467083in}{3.477748in}%
\pgfsys@useobject{currentmarker}{}%
\end{pgfscope}%
\end{pgfscope}%
\begin{pgfscope}%
\definecolor{textcolor}{rgb}{0.000000,0.000000,0.000000}%
\pgfsetstrokecolor{textcolor}%
\pgfsetfillcolor{textcolor}%
\pgftext[x=4.717083in,y=3.429136in,left,base]{\color{textcolor}\sffamily\fontsize{10.000000}{12.000000}\selectfont No Timeout}%
\end{pgfscope}%
\begin{pgfscope}%
\pgfsetbuttcap%
\pgfsetroundjoin%
\definecolor{currentfill}{rgb}{1.000000,0.498039,0.054902}%
\pgfsetfillcolor{currentfill}%
\pgfsetlinewidth{1.003750pt}%
\definecolor{currentstroke}{rgb}{1.000000,0.498039,0.054902}%
\pgfsetstrokecolor{currentstroke}%
\pgfsetdash{}{0pt}%
\pgfsys@defobject{currentmarker}{\pgfqpoint{-0.034722in}{-0.034722in}}{\pgfqpoint{0.034722in}{0.034722in}}{%
\pgfpathmoveto{\pgfqpoint{0.000000in}{-0.034722in}}%
\pgfpathcurveto{\pgfqpoint{0.009208in}{-0.034722in}}{\pgfqpoint{0.018041in}{-0.031064in}}{\pgfqpoint{0.024552in}{-0.024552in}}%
\pgfpathcurveto{\pgfqpoint{0.031064in}{-0.018041in}}{\pgfqpoint{0.034722in}{-0.009208in}}{\pgfqpoint{0.034722in}{0.000000in}}%
\pgfpathcurveto{\pgfqpoint{0.034722in}{0.009208in}}{\pgfqpoint{0.031064in}{0.018041in}}{\pgfqpoint{0.024552in}{0.024552in}}%
\pgfpathcurveto{\pgfqpoint{0.018041in}{0.031064in}}{\pgfqpoint{0.009208in}{0.034722in}}{\pgfqpoint{0.000000in}{0.034722in}}%
\pgfpathcurveto{\pgfqpoint{-0.009208in}{0.034722in}}{\pgfqpoint{-0.018041in}{0.031064in}}{\pgfqpoint{-0.024552in}{0.024552in}}%
\pgfpathcurveto{\pgfqpoint{-0.031064in}{0.018041in}}{\pgfqpoint{-0.034722in}{0.009208in}}{\pgfqpoint{-0.034722in}{0.000000in}}%
\pgfpathcurveto{\pgfqpoint{-0.034722in}{-0.009208in}}{\pgfqpoint{-0.031064in}{-0.018041in}}{\pgfqpoint{-0.024552in}{-0.024552in}}%
\pgfpathcurveto{\pgfqpoint{-0.018041in}{-0.031064in}}{\pgfqpoint{-0.009208in}{-0.034722in}}{\pgfqpoint{0.000000in}{-0.034722in}}%
\pgfpathclose%
\pgfusepath{stroke,fill}%
}%
\begin{pgfscope}%
\pgfsys@transformshift{4.467083in}{3.284136in}%
\pgfsys@useobject{currentmarker}{}%
\end{pgfscope}%
\end{pgfscope}%
\begin{pgfscope}%
\definecolor{textcolor}{rgb}{0.000000,0.000000,0.000000}%
\pgfsetstrokecolor{textcolor}%
\pgfsetfillcolor{textcolor}%
\pgftext[x=4.717083in,y=3.235525in,left,base]{\color{textcolor}\sffamily\fontsize{10.000000}{12.000000}\selectfont Time Timeout}%
\end{pgfscope}%
\begin{pgfscope}%
\pgfsetbuttcap%
\pgfsetroundjoin%
\definecolor{currentfill}{rgb}{0.839216,0.152941,0.156863}%
\pgfsetfillcolor{currentfill}%
\pgfsetlinewidth{1.003750pt}%
\definecolor{currentstroke}{rgb}{0.839216,0.152941,0.156863}%
\pgfsetstrokecolor{currentstroke}%
\pgfsetdash{}{0pt}%
\pgfsys@defobject{currentmarker}{\pgfqpoint{-0.034722in}{-0.034722in}}{\pgfqpoint{0.034722in}{0.034722in}}{%
\pgfpathmoveto{\pgfqpoint{0.000000in}{-0.034722in}}%
\pgfpathcurveto{\pgfqpoint{0.009208in}{-0.034722in}}{\pgfqpoint{0.018041in}{-0.031064in}}{\pgfqpoint{0.024552in}{-0.024552in}}%
\pgfpathcurveto{\pgfqpoint{0.031064in}{-0.018041in}}{\pgfqpoint{0.034722in}{-0.009208in}}{\pgfqpoint{0.034722in}{0.000000in}}%
\pgfpathcurveto{\pgfqpoint{0.034722in}{0.009208in}}{\pgfqpoint{0.031064in}{0.018041in}}{\pgfqpoint{0.024552in}{0.024552in}}%
\pgfpathcurveto{\pgfqpoint{0.018041in}{0.031064in}}{\pgfqpoint{0.009208in}{0.034722in}}{\pgfqpoint{0.000000in}{0.034722in}}%
\pgfpathcurveto{\pgfqpoint{-0.009208in}{0.034722in}}{\pgfqpoint{-0.018041in}{0.031064in}}{\pgfqpoint{-0.024552in}{0.024552in}}%
\pgfpathcurveto{\pgfqpoint{-0.031064in}{0.018041in}}{\pgfqpoint{-0.034722in}{0.009208in}}{\pgfqpoint{-0.034722in}{0.000000in}}%
\pgfpathcurveto{\pgfqpoint{-0.034722in}{-0.009208in}}{\pgfqpoint{-0.031064in}{-0.018041in}}{\pgfqpoint{-0.024552in}{-0.024552in}}%
\pgfpathcurveto{\pgfqpoint{-0.018041in}{-0.031064in}}{\pgfqpoint{-0.009208in}{-0.034722in}}{\pgfqpoint{0.000000in}{-0.034722in}}%
\pgfpathclose%
\pgfusepath{stroke,fill}%
}%
\begin{pgfscope}%
\pgfsys@transformshift{4.467083in}{3.090525in}%
\pgfsys@useobject{currentmarker}{}%
\end{pgfscope}%
\end{pgfscope}%
\begin{pgfscope}%
\definecolor{textcolor}{rgb}{0.000000,0.000000,0.000000}%
\pgfsetstrokecolor{textcolor}%
\pgfsetfillcolor{textcolor}%
\pgftext[x=4.717083in,y=3.041914in,left,base]{\color{textcolor}\sffamily\fontsize{10.000000}{12.000000}\selectfont Memory Timeout}%
\end{pgfscope}%
\end{pgfpicture}%
\makeatother%
\endgroup%

                }
            \end{subfigure}
            \caption{Statements}
        \end{subfigure}
        \bigbreak
        \begin{subfigure}[b]{\textwidth}
            \centering
            \begin{subfigure}[]{0.45\textwidth}
                \centering
                \resizebox{\columnwidth}{!}{
                    %% Creator: Matplotlib, PGF backend
%%
%% To include the figure in your LaTeX document, write
%%   \input{<filename>.pgf}
%%
%% Make sure the required packages are loaded in your preamble
%%   \usepackage{pgf}
%%
%% and, on pdftex
%%   \usepackage[utf8]{inputenc}\DeclareUnicodeCharacter{2212}{-}
%%
%% or, on luatex and xetex
%%   \usepackage{unicode-math}
%%
%% Figures using additional raster images can only be included by \input if
%% they are in the same directory as the main LaTeX file. For loading figures
%% from other directories you can use the `import` package
%%   \usepackage{import}
%%
%% and then include the figures with
%%   \import{<path to file>}{<filename>.pgf}
%%
%% Matplotlib used the following preamble
%%   \usepackage{amsmath}
%%   \usepackage{fontspec}
%%
\begingroup%
\makeatletter%
\begin{pgfpicture}%
\pgfpathrectangle{\pgfpointorigin}{\pgfqpoint{6.000000in}{4.000000in}}%
\pgfusepath{use as bounding box, clip}%
\begin{pgfscope}%
\pgfsetbuttcap%
\pgfsetmiterjoin%
\definecolor{currentfill}{rgb}{1.000000,1.000000,1.000000}%
\pgfsetfillcolor{currentfill}%
\pgfsetlinewidth{0.000000pt}%
\definecolor{currentstroke}{rgb}{1.000000,1.000000,1.000000}%
\pgfsetstrokecolor{currentstroke}%
\pgfsetdash{}{0pt}%
\pgfpathmoveto{\pgfqpoint{0.000000in}{0.000000in}}%
\pgfpathlineto{\pgfqpoint{6.000000in}{0.000000in}}%
\pgfpathlineto{\pgfqpoint{6.000000in}{4.000000in}}%
\pgfpathlineto{\pgfqpoint{0.000000in}{4.000000in}}%
\pgfpathclose%
\pgfusepath{fill}%
\end{pgfscope}%
\begin{pgfscope}%
\pgfsetbuttcap%
\pgfsetmiterjoin%
\definecolor{currentfill}{rgb}{1.000000,1.000000,1.000000}%
\pgfsetfillcolor{currentfill}%
\pgfsetlinewidth{0.000000pt}%
\definecolor{currentstroke}{rgb}{0.000000,0.000000,0.000000}%
\pgfsetstrokecolor{currentstroke}%
\pgfsetstrokeopacity{0.000000}%
\pgfsetdash{}{0pt}%
\pgfpathmoveto{\pgfqpoint{0.787074in}{0.548769in}}%
\pgfpathlineto{\pgfqpoint{5.850000in}{0.548769in}}%
\pgfpathlineto{\pgfqpoint{5.850000in}{3.651359in}}%
\pgfpathlineto{\pgfqpoint{0.787074in}{3.651359in}}%
\pgfpathclose%
\pgfusepath{fill}%
\end{pgfscope}%
\begin{pgfscope}%
\pgfpathrectangle{\pgfqpoint{0.787074in}{0.548769in}}{\pgfqpoint{5.062926in}{3.102590in}}%
\pgfusepath{clip}%
\pgfsetbuttcap%
\pgfsetroundjoin%
\definecolor{currentfill}{rgb}{0.121569,0.466667,0.705882}%
\pgfsetfillcolor{currentfill}%
\pgfsetlinewidth{1.003750pt}%
\definecolor{currentstroke}{rgb}{0.121569,0.466667,0.705882}%
\pgfsetstrokecolor{currentstroke}%
\pgfsetdash{}{0pt}%
\pgfpathmoveto{\pgfqpoint{1.453919in}{0.648193in}}%
\pgfpathcurveto{\pgfqpoint{1.464969in}{0.648193in}}{\pgfqpoint{1.475568in}{0.652583in}}{\pgfqpoint{1.483381in}{0.660397in}}%
\pgfpathcurveto{\pgfqpoint{1.491195in}{0.668210in}}{\pgfqpoint{1.495585in}{0.678809in}}{\pgfqpoint{1.495585in}{0.689859in}}%
\pgfpathcurveto{\pgfqpoint{1.495585in}{0.700910in}}{\pgfqpoint{1.491195in}{0.711509in}}{\pgfqpoint{1.483381in}{0.719322in}}%
\pgfpathcurveto{\pgfqpoint{1.475568in}{0.727136in}}{\pgfqpoint{1.464969in}{0.731526in}}{\pgfqpoint{1.453919in}{0.731526in}}%
\pgfpathcurveto{\pgfqpoint{1.442868in}{0.731526in}}{\pgfqpoint{1.432269in}{0.727136in}}{\pgfqpoint{1.424456in}{0.719322in}}%
\pgfpathcurveto{\pgfqpoint{1.416642in}{0.711509in}}{\pgfqpoint{1.412252in}{0.700910in}}{\pgfqpoint{1.412252in}{0.689859in}}%
\pgfpathcurveto{\pgfqpoint{1.412252in}{0.678809in}}{\pgfqpoint{1.416642in}{0.668210in}}{\pgfqpoint{1.424456in}{0.660397in}}%
\pgfpathcurveto{\pgfqpoint{1.432269in}{0.652583in}}{\pgfqpoint{1.442868in}{0.648193in}}{\pgfqpoint{1.453919in}{0.648193in}}%
\pgfpathclose%
\pgfusepath{stroke,fill}%
\end{pgfscope}%
\begin{pgfscope}%
\pgfpathrectangle{\pgfqpoint{0.787074in}{0.548769in}}{\pgfqpoint{5.062926in}{3.102590in}}%
\pgfusepath{clip}%
\pgfsetbuttcap%
\pgfsetroundjoin%
\definecolor{currentfill}{rgb}{1.000000,0.498039,0.054902}%
\pgfsetfillcolor{currentfill}%
\pgfsetlinewidth{1.003750pt}%
\definecolor{currentstroke}{rgb}{1.000000,0.498039,0.054902}%
\pgfsetstrokecolor{currentstroke}%
\pgfsetdash{}{0pt}%
\pgfpathmoveto{\pgfqpoint{4.757873in}{1.859307in}}%
\pgfpathcurveto{\pgfqpoint{4.768923in}{1.859307in}}{\pgfqpoint{4.779522in}{1.863697in}}{\pgfqpoint{4.787336in}{1.871511in}}%
\pgfpathcurveto{\pgfqpoint{4.795149in}{1.879324in}}{\pgfqpoint{4.799540in}{1.889923in}}{\pgfqpoint{4.799540in}{1.900973in}}%
\pgfpathcurveto{\pgfqpoint{4.799540in}{1.912024in}}{\pgfqpoint{4.795149in}{1.922623in}}{\pgfqpoint{4.787336in}{1.930436in}}%
\pgfpathcurveto{\pgfqpoint{4.779522in}{1.938250in}}{\pgfqpoint{4.768923in}{1.942640in}}{\pgfqpoint{4.757873in}{1.942640in}}%
\pgfpathcurveto{\pgfqpoint{4.746823in}{1.942640in}}{\pgfqpoint{4.736224in}{1.938250in}}{\pgfqpoint{4.728410in}{1.930436in}}%
\pgfpathcurveto{\pgfqpoint{4.720597in}{1.922623in}}{\pgfqpoint{4.716206in}{1.912024in}}{\pgfqpoint{4.716206in}{1.900973in}}%
\pgfpathcurveto{\pgfqpoint{4.716206in}{1.889923in}}{\pgfqpoint{4.720597in}{1.879324in}}{\pgfqpoint{4.728410in}{1.871511in}}%
\pgfpathcurveto{\pgfqpoint{4.736224in}{1.863697in}}{\pgfqpoint{4.746823in}{1.859307in}}{\pgfqpoint{4.757873in}{1.859307in}}%
\pgfpathclose%
\pgfusepath{stroke,fill}%
\end{pgfscope}%
\begin{pgfscope}%
\pgfpathrectangle{\pgfqpoint{0.787074in}{0.548769in}}{\pgfqpoint{5.062926in}{3.102590in}}%
\pgfusepath{clip}%
\pgfsetbuttcap%
\pgfsetroundjoin%
\definecolor{currentfill}{rgb}{1.000000,0.498039,0.054902}%
\pgfsetfillcolor{currentfill}%
\pgfsetlinewidth{1.003750pt}%
\definecolor{currentstroke}{rgb}{1.000000,0.498039,0.054902}%
\pgfsetstrokecolor{currentstroke}%
\pgfsetdash{}{0pt}%
\pgfpathmoveto{\pgfqpoint{1.385621in}{2.662812in}}%
\pgfpathcurveto{\pgfqpoint{1.396672in}{2.662812in}}{\pgfqpoint{1.407271in}{2.667202in}}{\pgfqpoint{1.415084in}{2.675016in}}%
\pgfpathcurveto{\pgfqpoint{1.422898in}{2.682830in}}{\pgfqpoint{1.427288in}{2.693429in}}{\pgfqpoint{1.427288in}{2.704479in}}%
\pgfpathcurveto{\pgfqpoint{1.427288in}{2.715529in}}{\pgfqpoint{1.422898in}{2.726128in}}{\pgfqpoint{1.415084in}{2.733942in}}%
\pgfpathcurveto{\pgfqpoint{1.407271in}{2.741755in}}{\pgfqpoint{1.396672in}{2.746145in}}{\pgfqpoint{1.385621in}{2.746145in}}%
\pgfpathcurveto{\pgfqpoint{1.374571in}{2.746145in}}{\pgfqpoint{1.363972in}{2.741755in}}{\pgfqpoint{1.356159in}{2.733942in}}%
\pgfpathcurveto{\pgfqpoint{1.348345in}{2.726128in}}{\pgfqpoint{1.343955in}{2.715529in}}{\pgfqpoint{1.343955in}{2.704479in}}%
\pgfpathcurveto{\pgfqpoint{1.343955in}{2.693429in}}{\pgfqpoint{1.348345in}{2.682830in}}{\pgfqpoint{1.356159in}{2.675016in}}%
\pgfpathcurveto{\pgfqpoint{1.363972in}{2.667202in}}{\pgfqpoint{1.374571in}{2.662812in}}{\pgfqpoint{1.385621in}{2.662812in}}%
\pgfpathclose%
\pgfusepath{stroke,fill}%
\end{pgfscope}%
\begin{pgfscope}%
\pgfpathrectangle{\pgfqpoint{0.787074in}{0.548769in}}{\pgfqpoint{5.062926in}{3.102590in}}%
\pgfusepath{clip}%
\pgfsetbuttcap%
\pgfsetroundjoin%
\definecolor{currentfill}{rgb}{1.000000,0.498039,0.054902}%
\pgfsetfillcolor{currentfill}%
\pgfsetlinewidth{1.003750pt}%
\definecolor{currentstroke}{rgb}{1.000000,0.498039,0.054902}%
\pgfsetstrokecolor{currentstroke}%
\pgfsetdash{}{0pt}%
\pgfpathmoveto{\pgfqpoint{3.029578in}{2.336454in}}%
\pgfpathcurveto{\pgfqpoint{3.040628in}{2.336454in}}{\pgfqpoint{3.051227in}{2.340845in}}{\pgfqpoint{3.059040in}{2.348658in}}%
\pgfpathcurveto{\pgfqpoint{3.066854in}{2.356472in}}{\pgfqpoint{3.071244in}{2.367071in}}{\pgfqpoint{3.071244in}{2.378121in}}%
\pgfpathcurveto{\pgfqpoint{3.071244in}{2.389171in}}{\pgfqpoint{3.066854in}{2.399770in}}{\pgfqpoint{3.059040in}{2.407584in}}%
\pgfpathcurveto{\pgfqpoint{3.051227in}{2.415397in}}{\pgfqpoint{3.040628in}{2.419788in}}{\pgfqpoint{3.029578in}{2.419788in}}%
\pgfpathcurveto{\pgfqpoint{3.018528in}{2.419788in}}{\pgfqpoint{3.007928in}{2.415397in}}{\pgfqpoint{3.000115in}{2.407584in}}%
\pgfpathcurveto{\pgfqpoint{2.992301in}{2.399770in}}{\pgfqpoint{2.987911in}{2.389171in}}{\pgfqpoint{2.987911in}{2.378121in}}%
\pgfpathcurveto{\pgfqpoint{2.987911in}{2.367071in}}{\pgfqpoint{2.992301in}{2.356472in}}{\pgfqpoint{3.000115in}{2.348658in}}%
\pgfpathcurveto{\pgfqpoint{3.007928in}{2.340845in}}{\pgfqpoint{3.018528in}{2.336454in}}{\pgfqpoint{3.029578in}{2.336454in}}%
\pgfpathclose%
\pgfusepath{stroke,fill}%
\end{pgfscope}%
\begin{pgfscope}%
\pgfpathrectangle{\pgfqpoint{0.787074in}{0.548769in}}{\pgfqpoint{5.062926in}{3.102590in}}%
\pgfusepath{clip}%
\pgfsetbuttcap%
\pgfsetroundjoin%
\definecolor{currentfill}{rgb}{0.121569,0.466667,0.705882}%
\pgfsetfillcolor{currentfill}%
\pgfsetlinewidth{1.003750pt}%
\definecolor{currentstroke}{rgb}{0.121569,0.466667,0.705882}%
\pgfsetstrokecolor{currentstroke}%
\pgfsetdash{}{0pt}%
\pgfpathmoveto{\pgfqpoint{2.933936in}{2.322912in}}%
\pgfpathcurveto{\pgfqpoint{2.944986in}{2.322912in}}{\pgfqpoint{2.955585in}{2.327302in}}{\pgfqpoint{2.963399in}{2.335116in}}%
\pgfpathcurveto{\pgfqpoint{2.971213in}{2.342929in}}{\pgfqpoint{2.975603in}{2.353528in}}{\pgfqpoint{2.975603in}{2.364578in}}%
\pgfpathcurveto{\pgfqpoint{2.975603in}{2.375629in}}{\pgfqpoint{2.971213in}{2.386228in}}{\pgfqpoint{2.963399in}{2.394041in}}%
\pgfpathcurveto{\pgfqpoint{2.955585in}{2.401855in}}{\pgfqpoint{2.944986in}{2.406245in}}{\pgfqpoint{2.933936in}{2.406245in}}%
\pgfpathcurveto{\pgfqpoint{2.922886in}{2.406245in}}{\pgfqpoint{2.912287in}{2.401855in}}{\pgfqpoint{2.904473in}{2.394041in}}%
\pgfpathcurveto{\pgfqpoint{2.896660in}{2.386228in}}{\pgfqpoint{2.892270in}{2.375629in}}{\pgfqpoint{2.892270in}{2.364578in}}%
\pgfpathcurveto{\pgfqpoint{2.892270in}{2.353528in}}{\pgfqpoint{2.896660in}{2.342929in}}{\pgfqpoint{2.904473in}{2.335116in}}%
\pgfpathcurveto{\pgfqpoint{2.912287in}{2.327302in}}{\pgfqpoint{2.922886in}{2.322912in}}{\pgfqpoint{2.933936in}{2.322912in}}%
\pgfpathclose%
\pgfusepath{stroke,fill}%
\end{pgfscope}%
\begin{pgfscope}%
\pgfpathrectangle{\pgfqpoint{0.787074in}{0.548769in}}{\pgfqpoint{5.062926in}{3.102590in}}%
\pgfusepath{clip}%
\pgfsetbuttcap%
\pgfsetroundjoin%
\definecolor{currentfill}{rgb}{1.000000,0.498039,0.054902}%
\pgfsetfillcolor{currentfill}%
\pgfsetlinewidth{1.003750pt}%
\definecolor{currentstroke}{rgb}{1.000000,0.498039,0.054902}%
\pgfsetstrokecolor{currentstroke}%
\pgfsetdash{}{0pt}%
\pgfpathmoveto{\pgfqpoint{2.647063in}{1.553412in}}%
\pgfpathcurveto{\pgfqpoint{2.658113in}{1.553412in}}{\pgfqpoint{2.668712in}{1.557802in}}{\pgfqpoint{2.676525in}{1.565616in}}%
\pgfpathcurveto{\pgfqpoint{2.684339in}{1.573429in}}{\pgfqpoint{2.688729in}{1.584029in}}{\pgfqpoint{2.688729in}{1.595079in}}%
\pgfpathcurveto{\pgfqpoint{2.688729in}{1.606129in}}{\pgfqpoint{2.684339in}{1.616728in}}{\pgfqpoint{2.676525in}{1.624541in}}%
\pgfpathcurveto{\pgfqpoint{2.668712in}{1.632355in}}{\pgfqpoint{2.658113in}{1.636745in}}{\pgfqpoint{2.647063in}{1.636745in}}%
\pgfpathcurveto{\pgfqpoint{2.636013in}{1.636745in}}{\pgfqpoint{2.625414in}{1.632355in}}{\pgfqpoint{2.617600in}{1.624541in}}%
\pgfpathcurveto{\pgfqpoint{2.609786in}{1.616728in}}{\pgfqpoint{2.605396in}{1.606129in}}{\pgfqpoint{2.605396in}{1.595079in}}%
\pgfpathcurveto{\pgfqpoint{2.605396in}{1.584029in}}{\pgfqpoint{2.609786in}{1.573429in}}{\pgfqpoint{2.617600in}{1.565616in}}%
\pgfpathcurveto{\pgfqpoint{2.625414in}{1.557802in}}{\pgfqpoint{2.636013in}{1.553412in}}{\pgfqpoint{2.647063in}{1.553412in}}%
\pgfpathclose%
\pgfusepath{stroke,fill}%
\end{pgfscope}%
\begin{pgfscope}%
\pgfpathrectangle{\pgfqpoint{0.787074in}{0.548769in}}{\pgfqpoint{5.062926in}{3.102590in}}%
\pgfusepath{clip}%
\pgfsetbuttcap%
\pgfsetroundjoin%
\definecolor{currentfill}{rgb}{1.000000,0.498039,0.054902}%
\pgfsetfillcolor{currentfill}%
\pgfsetlinewidth{1.003750pt}%
\definecolor{currentstroke}{rgb}{1.000000,0.498039,0.054902}%
\pgfsetstrokecolor{currentstroke}%
\pgfsetdash{}{0pt}%
\pgfpathmoveto{\pgfqpoint{2.561367in}{2.076652in}}%
\pgfpathcurveto{\pgfqpoint{2.572417in}{2.076652in}}{\pgfqpoint{2.583016in}{2.081042in}}{\pgfqpoint{2.590830in}{2.088856in}}%
\pgfpathcurveto{\pgfqpoint{2.598643in}{2.096669in}}{\pgfqpoint{2.603034in}{2.107268in}}{\pgfqpoint{2.603034in}{2.118319in}}%
\pgfpathcurveto{\pgfqpoint{2.603034in}{2.129369in}}{\pgfqpoint{2.598643in}{2.139968in}}{\pgfqpoint{2.590830in}{2.147781in}}%
\pgfpathcurveto{\pgfqpoint{2.583016in}{2.155595in}}{\pgfqpoint{2.572417in}{2.159985in}}{\pgfqpoint{2.561367in}{2.159985in}}%
\pgfpathcurveto{\pgfqpoint{2.550317in}{2.159985in}}{\pgfqpoint{2.539718in}{2.155595in}}{\pgfqpoint{2.531904in}{2.147781in}}%
\pgfpathcurveto{\pgfqpoint{2.524091in}{2.139968in}}{\pgfqpoint{2.519700in}{2.129369in}}{\pgfqpoint{2.519700in}{2.118319in}}%
\pgfpathcurveto{\pgfqpoint{2.519700in}{2.107268in}}{\pgfqpoint{2.524091in}{2.096669in}}{\pgfqpoint{2.531904in}{2.088856in}}%
\pgfpathcurveto{\pgfqpoint{2.539718in}{2.081042in}}{\pgfqpoint{2.550317in}{2.076652in}}{\pgfqpoint{2.561367in}{2.076652in}}%
\pgfpathclose%
\pgfusepath{stroke,fill}%
\end{pgfscope}%
\begin{pgfscope}%
\pgfpathrectangle{\pgfqpoint{0.787074in}{0.548769in}}{\pgfqpoint{5.062926in}{3.102590in}}%
\pgfusepath{clip}%
\pgfsetbuttcap%
\pgfsetroundjoin%
\definecolor{currentfill}{rgb}{1.000000,0.498039,0.054902}%
\pgfsetfillcolor{currentfill}%
\pgfsetlinewidth{1.003750pt}%
\definecolor{currentstroke}{rgb}{1.000000,0.498039,0.054902}%
\pgfsetstrokecolor{currentstroke}%
\pgfsetdash{}{0pt}%
\pgfpathmoveto{\pgfqpoint{1.573657in}{2.475226in}}%
\pgfpathcurveto{\pgfqpoint{1.584707in}{2.475226in}}{\pgfqpoint{1.595306in}{2.479616in}}{\pgfqpoint{1.603120in}{2.487430in}}%
\pgfpathcurveto{\pgfqpoint{1.610933in}{2.495243in}}{\pgfqpoint{1.615324in}{2.505842in}}{\pgfqpoint{1.615324in}{2.516893in}}%
\pgfpathcurveto{\pgfqpoint{1.615324in}{2.527943in}}{\pgfqpoint{1.610933in}{2.538542in}}{\pgfqpoint{1.603120in}{2.546355in}}%
\pgfpathcurveto{\pgfqpoint{1.595306in}{2.554169in}}{\pgfqpoint{1.584707in}{2.558559in}}{\pgfqpoint{1.573657in}{2.558559in}}%
\pgfpathcurveto{\pgfqpoint{1.562607in}{2.558559in}}{\pgfqpoint{1.552008in}{2.554169in}}{\pgfqpoint{1.544194in}{2.546355in}}%
\pgfpathcurveto{\pgfqpoint{1.536381in}{2.538542in}}{\pgfqpoint{1.531990in}{2.527943in}}{\pgfqpoint{1.531990in}{2.516893in}}%
\pgfpathcurveto{\pgfqpoint{1.531990in}{2.505842in}}{\pgfqpoint{1.536381in}{2.495243in}}{\pgfqpoint{1.544194in}{2.487430in}}%
\pgfpathcurveto{\pgfqpoint{1.552008in}{2.479616in}}{\pgfqpoint{1.562607in}{2.475226in}}{\pgfqpoint{1.573657in}{2.475226in}}%
\pgfpathclose%
\pgfusepath{stroke,fill}%
\end{pgfscope}%
\begin{pgfscope}%
\pgfpathrectangle{\pgfqpoint{0.787074in}{0.548769in}}{\pgfqpoint{5.062926in}{3.102590in}}%
\pgfusepath{clip}%
\pgfsetbuttcap%
\pgfsetroundjoin%
\definecolor{currentfill}{rgb}{1.000000,0.498039,0.054902}%
\pgfsetfillcolor{currentfill}%
\pgfsetlinewidth{1.003750pt}%
\definecolor{currentstroke}{rgb}{1.000000,0.498039,0.054902}%
\pgfsetstrokecolor{currentstroke}%
\pgfsetdash{}{0pt}%
\pgfpathmoveto{\pgfqpoint{2.339365in}{2.253097in}}%
\pgfpathcurveto{\pgfqpoint{2.350415in}{2.253097in}}{\pgfqpoint{2.361014in}{2.257487in}}{\pgfqpoint{2.368828in}{2.265300in}}%
\pgfpathcurveto{\pgfqpoint{2.376642in}{2.273114in}}{\pgfqpoint{2.381032in}{2.283713in}}{\pgfqpoint{2.381032in}{2.294763in}}%
\pgfpathcurveto{\pgfqpoint{2.381032in}{2.305813in}}{\pgfqpoint{2.376642in}{2.316412in}}{\pgfqpoint{2.368828in}{2.324226in}}%
\pgfpathcurveto{\pgfqpoint{2.361014in}{2.332040in}}{\pgfqpoint{2.350415in}{2.336430in}}{\pgfqpoint{2.339365in}{2.336430in}}%
\pgfpathcurveto{\pgfqpoint{2.328315in}{2.336430in}}{\pgfqpoint{2.317716in}{2.332040in}}{\pgfqpoint{2.309902in}{2.324226in}}%
\pgfpathcurveto{\pgfqpoint{2.302089in}{2.316412in}}{\pgfqpoint{2.297698in}{2.305813in}}{\pgfqpoint{2.297698in}{2.294763in}}%
\pgfpathcurveto{\pgfqpoint{2.297698in}{2.283713in}}{\pgfqpoint{2.302089in}{2.273114in}}{\pgfqpoint{2.309902in}{2.265300in}}%
\pgfpathcurveto{\pgfqpoint{2.317716in}{2.257487in}}{\pgfqpoint{2.328315in}{2.253097in}}{\pgfqpoint{2.339365in}{2.253097in}}%
\pgfpathclose%
\pgfusepath{stroke,fill}%
\end{pgfscope}%
\begin{pgfscope}%
\pgfpathrectangle{\pgfqpoint{0.787074in}{0.548769in}}{\pgfqpoint{5.062926in}{3.102590in}}%
\pgfusepath{clip}%
\pgfsetbuttcap%
\pgfsetroundjoin%
\definecolor{currentfill}{rgb}{0.121569,0.466667,0.705882}%
\pgfsetfillcolor{currentfill}%
\pgfsetlinewidth{1.003750pt}%
\definecolor{currentstroke}{rgb}{0.121569,0.466667,0.705882}%
\pgfsetstrokecolor{currentstroke}%
\pgfsetdash{}{0pt}%
\pgfpathmoveto{\pgfqpoint{1.017258in}{0.659516in}}%
\pgfpathcurveto{\pgfqpoint{1.028308in}{0.659516in}}{\pgfqpoint{1.038907in}{0.663906in}}{\pgfqpoint{1.046721in}{0.671719in}}%
\pgfpathcurveto{\pgfqpoint{1.054534in}{0.679533in}}{\pgfqpoint{1.058924in}{0.690132in}}{\pgfqpoint{1.058924in}{0.701182in}}%
\pgfpathcurveto{\pgfqpoint{1.058924in}{0.712232in}}{\pgfqpoint{1.054534in}{0.722831in}}{\pgfqpoint{1.046721in}{0.730645in}}%
\pgfpathcurveto{\pgfqpoint{1.038907in}{0.738459in}}{\pgfqpoint{1.028308in}{0.742849in}}{\pgfqpoint{1.017258in}{0.742849in}}%
\pgfpathcurveto{\pgfqpoint{1.006208in}{0.742849in}}{\pgfqpoint{0.995609in}{0.738459in}}{\pgfqpoint{0.987795in}{0.730645in}}%
\pgfpathcurveto{\pgfqpoint{0.979981in}{0.722831in}}{\pgfqpoint{0.975591in}{0.712232in}}{\pgfqpoint{0.975591in}{0.701182in}}%
\pgfpathcurveto{\pgfqpoint{0.975591in}{0.690132in}}{\pgfqpoint{0.979981in}{0.679533in}}{\pgfqpoint{0.987795in}{0.671719in}}%
\pgfpathcurveto{\pgfqpoint{0.995609in}{0.663906in}}{\pgfqpoint{1.006208in}{0.659516in}}{\pgfqpoint{1.017258in}{0.659516in}}%
\pgfpathclose%
\pgfusepath{stroke,fill}%
\end{pgfscope}%
\begin{pgfscope}%
\pgfpathrectangle{\pgfqpoint{0.787074in}{0.548769in}}{\pgfqpoint{5.062926in}{3.102590in}}%
\pgfusepath{clip}%
\pgfsetbuttcap%
\pgfsetroundjoin%
\definecolor{currentfill}{rgb}{1.000000,0.498039,0.054902}%
\pgfsetfillcolor{currentfill}%
\pgfsetlinewidth{1.003750pt}%
\definecolor{currentstroke}{rgb}{1.000000,0.498039,0.054902}%
\pgfsetstrokecolor{currentstroke}%
\pgfsetdash{}{0pt}%
\pgfpathmoveto{\pgfqpoint{2.185220in}{2.129288in}}%
\pgfpathcurveto{\pgfqpoint{2.196270in}{2.129288in}}{\pgfqpoint{2.206869in}{2.133678in}}{\pgfqpoint{2.214682in}{2.141492in}}%
\pgfpathcurveto{\pgfqpoint{2.222496in}{2.149306in}}{\pgfqpoint{2.226886in}{2.159905in}}{\pgfqpoint{2.226886in}{2.170955in}}%
\pgfpathcurveto{\pgfqpoint{2.226886in}{2.182005in}}{\pgfqpoint{2.222496in}{2.192604in}}{\pgfqpoint{2.214682in}{2.200418in}}%
\pgfpathcurveto{\pgfqpoint{2.206869in}{2.208231in}}{\pgfqpoint{2.196270in}{2.212621in}}{\pgfqpoint{2.185220in}{2.212621in}}%
\pgfpathcurveto{\pgfqpoint{2.174169in}{2.212621in}}{\pgfqpoint{2.163570in}{2.208231in}}{\pgfqpoint{2.155757in}{2.200418in}}%
\pgfpathcurveto{\pgfqpoint{2.147943in}{2.192604in}}{\pgfqpoint{2.143553in}{2.182005in}}{\pgfqpoint{2.143553in}{2.170955in}}%
\pgfpathcurveto{\pgfqpoint{2.143553in}{2.159905in}}{\pgfqpoint{2.147943in}{2.149306in}}{\pgfqpoint{2.155757in}{2.141492in}}%
\pgfpathcurveto{\pgfqpoint{2.163570in}{2.133678in}}{\pgfqpoint{2.174169in}{2.129288in}}{\pgfqpoint{2.185220in}{2.129288in}}%
\pgfpathclose%
\pgfusepath{stroke,fill}%
\end{pgfscope}%
\begin{pgfscope}%
\pgfpathrectangle{\pgfqpoint{0.787074in}{0.548769in}}{\pgfqpoint{5.062926in}{3.102590in}}%
\pgfusepath{clip}%
\pgfsetbuttcap%
\pgfsetroundjoin%
\definecolor{currentfill}{rgb}{1.000000,0.498039,0.054902}%
\pgfsetfillcolor{currentfill}%
\pgfsetlinewidth{1.003750pt}%
\definecolor{currentstroke}{rgb}{1.000000,0.498039,0.054902}%
\pgfsetstrokecolor{currentstroke}%
\pgfsetdash{}{0pt}%
\pgfpathmoveto{\pgfqpoint{1.618595in}{2.778374in}}%
\pgfpathcurveto{\pgfqpoint{1.629645in}{2.778374in}}{\pgfqpoint{1.640244in}{2.782764in}}{\pgfqpoint{1.648058in}{2.790577in}}%
\pgfpathcurveto{\pgfqpoint{1.655871in}{2.798391in}}{\pgfqpoint{1.660262in}{2.808990in}}{\pgfqpoint{1.660262in}{2.820040in}}%
\pgfpathcurveto{\pgfqpoint{1.660262in}{2.831090in}}{\pgfqpoint{1.655871in}{2.841689in}}{\pgfqpoint{1.648058in}{2.849503in}}%
\pgfpathcurveto{\pgfqpoint{1.640244in}{2.857317in}}{\pgfqpoint{1.629645in}{2.861707in}}{\pgfqpoint{1.618595in}{2.861707in}}%
\pgfpathcurveto{\pgfqpoint{1.607545in}{2.861707in}}{\pgfqpoint{1.596946in}{2.857317in}}{\pgfqpoint{1.589132in}{2.849503in}}%
\pgfpathcurveto{\pgfqpoint{1.581318in}{2.841689in}}{\pgfqpoint{1.576928in}{2.831090in}}{\pgfqpoint{1.576928in}{2.820040in}}%
\pgfpathcurveto{\pgfqpoint{1.576928in}{2.808990in}}{\pgfqpoint{1.581318in}{2.798391in}}{\pgfqpoint{1.589132in}{2.790577in}}%
\pgfpathcurveto{\pgfqpoint{1.596946in}{2.782764in}}{\pgfqpoint{1.607545in}{2.778374in}}{\pgfqpoint{1.618595in}{2.778374in}}%
\pgfpathclose%
\pgfusepath{stroke,fill}%
\end{pgfscope}%
\begin{pgfscope}%
\pgfpathrectangle{\pgfqpoint{0.787074in}{0.548769in}}{\pgfqpoint{5.062926in}{3.102590in}}%
\pgfusepath{clip}%
\pgfsetbuttcap%
\pgfsetroundjoin%
\definecolor{currentfill}{rgb}{0.121569,0.466667,0.705882}%
\pgfsetfillcolor{currentfill}%
\pgfsetlinewidth{1.003750pt}%
\definecolor{currentstroke}{rgb}{0.121569,0.466667,0.705882}%
\pgfsetstrokecolor{currentstroke}%
\pgfsetdash{}{0pt}%
\pgfpathmoveto{\pgfqpoint{1.925063in}{2.577708in}}%
\pgfpathcurveto{\pgfqpoint{1.936113in}{2.577708in}}{\pgfqpoint{1.946712in}{2.582098in}}{\pgfqpoint{1.954526in}{2.589912in}}%
\pgfpathcurveto{\pgfqpoint{1.962339in}{2.597725in}}{\pgfqpoint{1.966730in}{2.608324in}}{\pgfqpoint{1.966730in}{2.619374in}}%
\pgfpathcurveto{\pgfqpoint{1.966730in}{2.630425in}}{\pgfqpoint{1.962339in}{2.641024in}}{\pgfqpoint{1.954526in}{2.648837in}}%
\pgfpathcurveto{\pgfqpoint{1.946712in}{2.656651in}}{\pgfqpoint{1.936113in}{2.661041in}}{\pgfqpoint{1.925063in}{2.661041in}}%
\pgfpathcurveto{\pgfqpoint{1.914013in}{2.661041in}}{\pgfqpoint{1.903414in}{2.656651in}}{\pgfqpoint{1.895600in}{2.648837in}}%
\pgfpathcurveto{\pgfqpoint{1.887787in}{2.641024in}}{\pgfqpoint{1.883396in}{2.630425in}}{\pgfqpoint{1.883396in}{2.619374in}}%
\pgfpathcurveto{\pgfqpoint{1.883396in}{2.608324in}}{\pgfqpoint{1.887787in}{2.597725in}}{\pgfqpoint{1.895600in}{2.589912in}}%
\pgfpathcurveto{\pgfqpoint{1.903414in}{2.582098in}}{\pgfqpoint{1.914013in}{2.577708in}}{\pgfqpoint{1.925063in}{2.577708in}}%
\pgfpathclose%
\pgfusepath{stroke,fill}%
\end{pgfscope}%
\begin{pgfscope}%
\pgfpathrectangle{\pgfqpoint{0.787074in}{0.548769in}}{\pgfqpoint{5.062926in}{3.102590in}}%
\pgfusepath{clip}%
\pgfsetbuttcap%
\pgfsetroundjoin%
\definecolor{currentfill}{rgb}{1.000000,0.498039,0.054902}%
\pgfsetfillcolor{currentfill}%
\pgfsetlinewidth{1.003750pt}%
\definecolor{currentstroke}{rgb}{1.000000,0.498039,0.054902}%
\pgfsetstrokecolor{currentstroke}%
\pgfsetdash{}{0pt}%
\pgfpathmoveto{\pgfqpoint{1.495499in}{2.595506in}}%
\pgfpathcurveto{\pgfqpoint{1.506549in}{2.595506in}}{\pgfqpoint{1.517148in}{2.599897in}}{\pgfqpoint{1.524962in}{2.607710in}}%
\pgfpathcurveto{\pgfqpoint{1.532775in}{2.615524in}}{\pgfqpoint{1.537166in}{2.626123in}}{\pgfqpoint{1.537166in}{2.637173in}}%
\pgfpathcurveto{\pgfqpoint{1.537166in}{2.648223in}}{\pgfqpoint{1.532775in}{2.658822in}}{\pgfqpoint{1.524962in}{2.666636in}}%
\pgfpathcurveto{\pgfqpoint{1.517148in}{2.674449in}}{\pgfqpoint{1.506549in}{2.678840in}}{\pgfqpoint{1.495499in}{2.678840in}}%
\pgfpathcurveto{\pgfqpoint{1.484449in}{2.678840in}}{\pgfqpoint{1.473850in}{2.674449in}}{\pgfqpoint{1.466036in}{2.666636in}}%
\pgfpathcurveto{\pgfqpoint{1.458222in}{2.658822in}}{\pgfqpoint{1.453832in}{2.648223in}}{\pgfqpoint{1.453832in}{2.637173in}}%
\pgfpathcurveto{\pgfqpoint{1.453832in}{2.626123in}}{\pgfqpoint{1.458222in}{2.615524in}}{\pgfqpoint{1.466036in}{2.607710in}}%
\pgfpathcurveto{\pgfqpoint{1.473850in}{2.599897in}}{\pgfqpoint{1.484449in}{2.595506in}}{\pgfqpoint{1.495499in}{2.595506in}}%
\pgfpathclose%
\pgfusepath{stroke,fill}%
\end{pgfscope}%
\begin{pgfscope}%
\pgfpathrectangle{\pgfqpoint{0.787074in}{0.548769in}}{\pgfqpoint{5.062926in}{3.102590in}}%
\pgfusepath{clip}%
\pgfsetbuttcap%
\pgfsetroundjoin%
\definecolor{currentfill}{rgb}{1.000000,0.498039,0.054902}%
\pgfsetfillcolor{currentfill}%
\pgfsetlinewidth{1.003750pt}%
\definecolor{currentstroke}{rgb}{1.000000,0.498039,0.054902}%
\pgfsetstrokecolor{currentstroke}%
\pgfsetdash{}{0pt}%
\pgfpathmoveto{\pgfqpoint{1.590174in}{1.910607in}}%
\pgfpathcurveto{\pgfqpoint{1.601224in}{1.910607in}}{\pgfqpoint{1.611823in}{1.914997in}}{\pgfqpoint{1.619637in}{1.922810in}}%
\pgfpathcurveto{\pgfqpoint{1.627450in}{1.930624in}}{\pgfqpoint{1.631840in}{1.941223in}}{\pgfqpoint{1.631840in}{1.952273in}}%
\pgfpathcurveto{\pgfqpoint{1.631840in}{1.963323in}}{\pgfqpoint{1.627450in}{1.973922in}}{\pgfqpoint{1.619637in}{1.981736in}}%
\pgfpathcurveto{\pgfqpoint{1.611823in}{1.989550in}}{\pgfqpoint{1.601224in}{1.993940in}}{\pgfqpoint{1.590174in}{1.993940in}}%
\pgfpathcurveto{\pgfqpoint{1.579124in}{1.993940in}}{\pgfqpoint{1.568525in}{1.989550in}}{\pgfqpoint{1.560711in}{1.981736in}}%
\pgfpathcurveto{\pgfqpoint{1.552897in}{1.973922in}}{\pgfqpoint{1.548507in}{1.963323in}}{\pgfqpoint{1.548507in}{1.952273in}}%
\pgfpathcurveto{\pgfqpoint{1.548507in}{1.941223in}}{\pgfqpoint{1.552897in}{1.930624in}}{\pgfqpoint{1.560711in}{1.922810in}}%
\pgfpathcurveto{\pgfqpoint{1.568525in}{1.914997in}}{\pgfqpoint{1.579124in}{1.910607in}}{\pgfqpoint{1.590174in}{1.910607in}}%
\pgfpathclose%
\pgfusepath{stroke,fill}%
\end{pgfscope}%
\begin{pgfscope}%
\pgfpathrectangle{\pgfqpoint{0.787074in}{0.548769in}}{\pgfqpoint{5.062926in}{3.102590in}}%
\pgfusepath{clip}%
\pgfsetbuttcap%
\pgfsetroundjoin%
\definecolor{currentfill}{rgb}{1.000000,0.498039,0.054902}%
\pgfsetfillcolor{currentfill}%
\pgfsetlinewidth{1.003750pt}%
\definecolor{currentstroke}{rgb}{1.000000,0.498039,0.054902}%
\pgfsetstrokecolor{currentstroke}%
\pgfsetdash{}{0pt}%
\pgfpathmoveto{\pgfqpoint{1.892479in}{1.406572in}}%
\pgfpathcurveto{\pgfqpoint{1.903529in}{1.406572in}}{\pgfqpoint{1.914128in}{1.410962in}}{\pgfqpoint{1.921942in}{1.418775in}}%
\pgfpathcurveto{\pgfqpoint{1.929755in}{1.426589in}}{\pgfqpoint{1.934145in}{1.437188in}}{\pgfqpoint{1.934145in}{1.448238in}}%
\pgfpathcurveto{\pgfqpoint{1.934145in}{1.459288in}}{\pgfqpoint{1.929755in}{1.469887in}}{\pgfqpoint{1.921942in}{1.477701in}}%
\pgfpathcurveto{\pgfqpoint{1.914128in}{1.485515in}}{\pgfqpoint{1.903529in}{1.489905in}}{\pgfqpoint{1.892479in}{1.489905in}}%
\pgfpathcurveto{\pgfqpoint{1.881429in}{1.489905in}}{\pgfqpoint{1.870830in}{1.485515in}}{\pgfqpoint{1.863016in}{1.477701in}}%
\pgfpathcurveto{\pgfqpoint{1.855202in}{1.469887in}}{\pgfqpoint{1.850812in}{1.459288in}}{\pgfqpoint{1.850812in}{1.448238in}}%
\pgfpathcurveto{\pgfqpoint{1.850812in}{1.437188in}}{\pgfqpoint{1.855202in}{1.426589in}}{\pgfqpoint{1.863016in}{1.418775in}}%
\pgfpathcurveto{\pgfqpoint{1.870830in}{1.410962in}}{\pgfqpoint{1.881429in}{1.406572in}}{\pgfqpoint{1.892479in}{1.406572in}}%
\pgfpathclose%
\pgfusepath{stroke,fill}%
\end{pgfscope}%
\begin{pgfscope}%
\pgfpathrectangle{\pgfqpoint{0.787074in}{0.548769in}}{\pgfqpoint{5.062926in}{3.102590in}}%
\pgfusepath{clip}%
\pgfsetbuttcap%
\pgfsetroundjoin%
\definecolor{currentfill}{rgb}{1.000000,0.498039,0.054902}%
\pgfsetfillcolor{currentfill}%
\pgfsetlinewidth{1.003750pt}%
\definecolor{currentstroke}{rgb}{1.000000,0.498039,0.054902}%
\pgfsetstrokecolor{currentstroke}%
\pgfsetdash{}{0pt}%
\pgfpathmoveto{\pgfqpoint{1.693633in}{2.078914in}}%
\pgfpathcurveto{\pgfqpoint{1.704683in}{2.078914in}}{\pgfqpoint{1.715282in}{2.083304in}}{\pgfqpoint{1.723096in}{2.091118in}}%
\pgfpathcurveto{\pgfqpoint{1.730909in}{2.098931in}}{\pgfqpoint{1.735299in}{2.109530in}}{\pgfqpoint{1.735299in}{2.120580in}}%
\pgfpathcurveto{\pgfqpoint{1.735299in}{2.131630in}}{\pgfqpoint{1.730909in}{2.142229in}}{\pgfqpoint{1.723096in}{2.150043in}}%
\pgfpathcurveto{\pgfqpoint{1.715282in}{2.157857in}}{\pgfqpoint{1.704683in}{2.162247in}}{\pgfqpoint{1.693633in}{2.162247in}}%
\pgfpathcurveto{\pgfqpoint{1.682583in}{2.162247in}}{\pgfqpoint{1.671984in}{2.157857in}}{\pgfqpoint{1.664170in}{2.150043in}}%
\pgfpathcurveto{\pgfqpoint{1.656356in}{2.142229in}}{\pgfqpoint{1.651966in}{2.131630in}}{\pgfqpoint{1.651966in}{2.120580in}}%
\pgfpathcurveto{\pgfqpoint{1.651966in}{2.109530in}}{\pgfqpoint{1.656356in}{2.098931in}}{\pgfqpoint{1.664170in}{2.091118in}}%
\pgfpathcurveto{\pgfqpoint{1.671984in}{2.083304in}}{\pgfqpoint{1.682583in}{2.078914in}}{\pgfqpoint{1.693633in}{2.078914in}}%
\pgfpathclose%
\pgfusepath{stroke,fill}%
\end{pgfscope}%
\begin{pgfscope}%
\pgfpathrectangle{\pgfqpoint{0.787074in}{0.548769in}}{\pgfqpoint{5.062926in}{3.102590in}}%
\pgfusepath{clip}%
\pgfsetbuttcap%
\pgfsetroundjoin%
\definecolor{currentfill}{rgb}{1.000000,0.498039,0.054902}%
\pgfsetfillcolor{currentfill}%
\pgfsetlinewidth{1.003750pt}%
\definecolor{currentstroke}{rgb}{1.000000,0.498039,0.054902}%
\pgfsetstrokecolor{currentstroke}%
\pgfsetdash{}{0pt}%
\pgfpathmoveto{\pgfqpoint{1.874402in}{1.659203in}}%
\pgfpathcurveto{\pgfqpoint{1.885452in}{1.659203in}}{\pgfqpoint{1.896051in}{1.663593in}}{\pgfqpoint{1.903865in}{1.671407in}}%
\pgfpathcurveto{\pgfqpoint{1.911678in}{1.679220in}}{\pgfqpoint{1.916069in}{1.689820in}}{\pgfqpoint{1.916069in}{1.700870in}}%
\pgfpathcurveto{\pgfqpoint{1.916069in}{1.711920in}}{\pgfqpoint{1.911678in}{1.722519in}}{\pgfqpoint{1.903865in}{1.730332in}}%
\pgfpathcurveto{\pgfqpoint{1.896051in}{1.738146in}}{\pgfqpoint{1.885452in}{1.742536in}}{\pgfqpoint{1.874402in}{1.742536in}}%
\pgfpathcurveto{\pgfqpoint{1.863352in}{1.742536in}}{\pgfqpoint{1.852753in}{1.738146in}}{\pgfqpoint{1.844939in}{1.730332in}}%
\pgfpathcurveto{\pgfqpoint{1.837125in}{1.722519in}}{\pgfqpoint{1.832735in}{1.711920in}}{\pgfqpoint{1.832735in}{1.700870in}}%
\pgfpathcurveto{\pgfqpoint{1.832735in}{1.689820in}}{\pgfqpoint{1.837125in}{1.679220in}}{\pgfqpoint{1.844939in}{1.671407in}}%
\pgfpathcurveto{\pgfqpoint{1.852753in}{1.663593in}}{\pgfqpoint{1.863352in}{1.659203in}}{\pgfqpoint{1.874402in}{1.659203in}}%
\pgfpathclose%
\pgfusepath{stroke,fill}%
\end{pgfscope}%
\begin{pgfscope}%
\pgfpathrectangle{\pgfqpoint{0.787074in}{0.548769in}}{\pgfqpoint{5.062926in}{3.102590in}}%
\pgfusepath{clip}%
\pgfsetbuttcap%
\pgfsetroundjoin%
\definecolor{currentfill}{rgb}{0.121569,0.466667,0.705882}%
\pgfsetfillcolor{currentfill}%
\pgfsetlinewidth{1.003750pt}%
\definecolor{currentstroke}{rgb}{0.121569,0.466667,0.705882}%
\pgfsetstrokecolor{currentstroke}%
\pgfsetdash{}{0pt}%
\pgfpathmoveto{\pgfqpoint{5.619867in}{1.472684in}}%
\pgfpathcurveto{\pgfqpoint{5.630917in}{1.472684in}}{\pgfqpoint{5.641516in}{1.477075in}}{\pgfqpoint{5.649330in}{1.484888in}}%
\pgfpathcurveto{\pgfqpoint{5.657143in}{1.492702in}}{\pgfqpoint{5.661534in}{1.503301in}}{\pgfqpoint{5.661534in}{1.514351in}}%
\pgfpathcurveto{\pgfqpoint{5.661534in}{1.525401in}}{\pgfqpoint{5.657143in}{1.536000in}}{\pgfqpoint{5.649330in}{1.543814in}}%
\pgfpathcurveto{\pgfqpoint{5.641516in}{1.551627in}}{\pgfqpoint{5.630917in}{1.556018in}}{\pgfqpoint{5.619867in}{1.556018in}}%
\pgfpathcurveto{\pgfqpoint{5.608817in}{1.556018in}}{\pgfqpoint{5.598218in}{1.551627in}}{\pgfqpoint{5.590404in}{1.543814in}}%
\pgfpathcurveto{\pgfqpoint{5.582591in}{1.536000in}}{\pgfqpoint{5.578200in}{1.525401in}}{\pgfqpoint{5.578200in}{1.514351in}}%
\pgfpathcurveto{\pgfqpoint{5.578200in}{1.503301in}}{\pgfqpoint{5.582591in}{1.492702in}}{\pgfqpoint{5.590404in}{1.484888in}}%
\pgfpathcurveto{\pgfqpoint{5.598218in}{1.477075in}}{\pgfqpoint{5.608817in}{1.472684in}}{\pgfqpoint{5.619867in}{1.472684in}}%
\pgfpathclose%
\pgfusepath{stroke,fill}%
\end{pgfscope}%
\begin{pgfscope}%
\pgfpathrectangle{\pgfqpoint{0.787074in}{0.548769in}}{\pgfqpoint{5.062926in}{3.102590in}}%
\pgfusepath{clip}%
\pgfsetbuttcap%
\pgfsetroundjoin%
\definecolor{currentfill}{rgb}{1.000000,0.498039,0.054902}%
\pgfsetfillcolor{currentfill}%
\pgfsetlinewidth{1.003750pt}%
\definecolor{currentstroke}{rgb}{1.000000,0.498039,0.054902}%
\pgfsetstrokecolor{currentstroke}%
\pgfsetdash{}{0pt}%
\pgfpathmoveto{\pgfqpoint{2.200108in}{2.453351in}}%
\pgfpathcurveto{\pgfqpoint{2.211159in}{2.453351in}}{\pgfqpoint{2.221758in}{2.457742in}}{\pgfqpoint{2.229571in}{2.465555in}}%
\pgfpathcurveto{\pgfqpoint{2.237385in}{2.473369in}}{\pgfqpoint{2.241775in}{2.483968in}}{\pgfqpoint{2.241775in}{2.495018in}}%
\pgfpathcurveto{\pgfqpoint{2.241775in}{2.506068in}}{\pgfqpoint{2.237385in}{2.516667in}}{\pgfqpoint{2.229571in}{2.524481in}}%
\pgfpathcurveto{\pgfqpoint{2.221758in}{2.532294in}}{\pgfqpoint{2.211159in}{2.536685in}}{\pgfqpoint{2.200108in}{2.536685in}}%
\pgfpathcurveto{\pgfqpoint{2.189058in}{2.536685in}}{\pgfqpoint{2.178459in}{2.532294in}}{\pgfqpoint{2.170646in}{2.524481in}}%
\pgfpathcurveto{\pgfqpoint{2.162832in}{2.516667in}}{\pgfqpoint{2.158442in}{2.506068in}}{\pgfqpoint{2.158442in}{2.495018in}}%
\pgfpathcurveto{\pgfqpoint{2.158442in}{2.483968in}}{\pgfqpoint{2.162832in}{2.473369in}}{\pgfqpoint{2.170646in}{2.465555in}}%
\pgfpathcurveto{\pgfqpoint{2.178459in}{2.457742in}}{\pgfqpoint{2.189058in}{2.453351in}}{\pgfqpoint{2.200108in}{2.453351in}}%
\pgfpathclose%
\pgfusepath{stroke,fill}%
\end{pgfscope}%
\begin{pgfscope}%
\pgfpathrectangle{\pgfqpoint{0.787074in}{0.548769in}}{\pgfqpoint{5.062926in}{3.102590in}}%
\pgfusepath{clip}%
\pgfsetbuttcap%
\pgfsetroundjoin%
\definecolor{currentfill}{rgb}{0.121569,0.466667,0.705882}%
\pgfsetfillcolor{currentfill}%
\pgfsetlinewidth{1.003750pt}%
\definecolor{currentstroke}{rgb}{0.121569,0.466667,0.705882}%
\pgfsetstrokecolor{currentstroke}%
\pgfsetdash{}{0pt}%
\pgfpathmoveto{\pgfqpoint{1.481534in}{0.648305in}}%
\pgfpathcurveto{\pgfqpoint{1.492584in}{0.648305in}}{\pgfqpoint{1.503183in}{0.652695in}}{\pgfqpoint{1.510997in}{0.660509in}}%
\pgfpathcurveto{\pgfqpoint{1.518811in}{0.668322in}}{\pgfqpoint{1.523201in}{0.678921in}}{\pgfqpoint{1.523201in}{0.689972in}}%
\pgfpathcurveto{\pgfqpoint{1.523201in}{0.701022in}}{\pgfqpoint{1.518811in}{0.711621in}}{\pgfqpoint{1.510997in}{0.719434in}}%
\pgfpathcurveto{\pgfqpoint{1.503183in}{0.727248in}}{\pgfqpoint{1.492584in}{0.731638in}}{\pgfqpoint{1.481534in}{0.731638in}}%
\pgfpathcurveto{\pgfqpoint{1.470484in}{0.731638in}}{\pgfqpoint{1.459885in}{0.727248in}}{\pgfqpoint{1.452071in}{0.719434in}}%
\pgfpathcurveto{\pgfqpoint{1.444258in}{0.711621in}}{\pgfqpoint{1.439868in}{0.701022in}}{\pgfqpoint{1.439868in}{0.689972in}}%
\pgfpathcurveto{\pgfqpoint{1.439868in}{0.678921in}}{\pgfqpoint{1.444258in}{0.668322in}}{\pgfqpoint{1.452071in}{0.660509in}}%
\pgfpathcurveto{\pgfqpoint{1.459885in}{0.652695in}}{\pgfqpoint{1.470484in}{0.648305in}}{\pgfqpoint{1.481534in}{0.648305in}}%
\pgfpathclose%
\pgfusepath{stroke,fill}%
\end{pgfscope}%
\begin{pgfscope}%
\pgfpathrectangle{\pgfqpoint{0.787074in}{0.548769in}}{\pgfqpoint{5.062926in}{3.102590in}}%
\pgfusepath{clip}%
\pgfsetbuttcap%
\pgfsetroundjoin%
\definecolor{currentfill}{rgb}{0.121569,0.466667,0.705882}%
\pgfsetfillcolor{currentfill}%
\pgfsetlinewidth{1.003750pt}%
\definecolor{currentstroke}{rgb}{0.121569,0.466667,0.705882}%
\pgfsetstrokecolor{currentstroke}%
\pgfsetdash{}{0pt}%
\pgfpathmoveto{\pgfqpoint{1.269368in}{1.422655in}}%
\pgfpathcurveto{\pgfqpoint{1.280418in}{1.422655in}}{\pgfqpoint{1.291017in}{1.427045in}}{\pgfqpoint{1.298831in}{1.434858in}}%
\pgfpathcurveto{\pgfqpoint{1.306644in}{1.442672in}}{\pgfqpoint{1.311035in}{1.453271in}}{\pgfqpoint{1.311035in}{1.464321in}}%
\pgfpathcurveto{\pgfqpoint{1.311035in}{1.475371in}}{\pgfqpoint{1.306644in}{1.485970in}}{\pgfqpoint{1.298831in}{1.493784in}}%
\pgfpathcurveto{\pgfqpoint{1.291017in}{1.501598in}}{\pgfqpoint{1.280418in}{1.505988in}}{\pgfqpoint{1.269368in}{1.505988in}}%
\pgfpathcurveto{\pgfqpoint{1.258318in}{1.505988in}}{\pgfqpoint{1.247719in}{1.501598in}}{\pgfqpoint{1.239905in}{1.493784in}}%
\pgfpathcurveto{\pgfqpoint{1.232092in}{1.485970in}}{\pgfqpoint{1.227701in}{1.475371in}}{\pgfqpoint{1.227701in}{1.464321in}}%
\pgfpathcurveto{\pgfqpoint{1.227701in}{1.453271in}}{\pgfqpoint{1.232092in}{1.442672in}}{\pgfqpoint{1.239905in}{1.434858in}}%
\pgfpathcurveto{\pgfqpoint{1.247719in}{1.427045in}}{\pgfqpoint{1.258318in}{1.422655in}}{\pgfqpoint{1.269368in}{1.422655in}}%
\pgfpathclose%
\pgfusepath{stroke,fill}%
\end{pgfscope}%
\begin{pgfscope}%
\pgfpathrectangle{\pgfqpoint{0.787074in}{0.548769in}}{\pgfqpoint{5.062926in}{3.102590in}}%
\pgfusepath{clip}%
\pgfsetbuttcap%
\pgfsetroundjoin%
\definecolor{currentfill}{rgb}{1.000000,0.498039,0.054902}%
\pgfsetfillcolor{currentfill}%
\pgfsetlinewidth{1.003750pt}%
\definecolor{currentstroke}{rgb}{1.000000,0.498039,0.054902}%
\pgfsetstrokecolor{currentstroke}%
\pgfsetdash{}{0pt}%
\pgfpathmoveto{\pgfqpoint{1.491361in}{2.833777in}}%
\pgfpathcurveto{\pgfqpoint{1.502411in}{2.833777in}}{\pgfqpoint{1.513010in}{2.838167in}}{\pgfqpoint{1.520824in}{2.845981in}}%
\pgfpathcurveto{\pgfqpoint{1.528638in}{2.853795in}}{\pgfqpoint{1.533028in}{2.864394in}}{\pgfqpoint{1.533028in}{2.875444in}}%
\pgfpathcurveto{\pgfqpoint{1.533028in}{2.886494in}}{\pgfqpoint{1.528638in}{2.897093in}}{\pgfqpoint{1.520824in}{2.904907in}}%
\pgfpathcurveto{\pgfqpoint{1.513010in}{2.912720in}}{\pgfqpoint{1.502411in}{2.917111in}}{\pgfqpoint{1.491361in}{2.917111in}}%
\pgfpathcurveto{\pgfqpoint{1.480311in}{2.917111in}}{\pgfqpoint{1.469712in}{2.912720in}}{\pgfqpoint{1.461898in}{2.904907in}}%
\pgfpathcurveto{\pgfqpoint{1.454085in}{2.897093in}}{\pgfqpoint{1.449695in}{2.886494in}}{\pgfqpoint{1.449695in}{2.875444in}}%
\pgfpathcurveto{\pgfqpoint{1.449695in}{2.864394in}}{\pgfqpoint{1.454085in}{2.853795in}}{\pgfqpoint{1.461898in}{2.845981in}}%
\pgfpathcurveto{\pgfqpoint{1.469712in}{2.838167in}}{\pgfqpoint{1.480311in}{2.833777in}}{\pgfqpoint{1.491361in}{2.833777in}}%
\pgfpathclose%
\pgfusepath{stroke,fill}%
\end{pgfscope}%
\begin{pgfscope}%
\pgfpathrectangle{\pgfqpoint{0.787074in}{0.548769in}}{\pgfqpoint{5.062926in}{3.102590in}}%
\pgfusepath{clip}%
\pgfsetbuttcap%
\pgfsetroundjoin%
\definecolor{currentfill}{rgb}{0.121569,0.466667,0.705882}%
\pgfsetfillcolor{currentfill}%
\pgfsetlinewidth{1.003750pt}%
\definecolor{currentstroke}{rgb}{0.121569,0.466667,0.705882}%
\pgfsetstrokecolor{currentstroke}%
\pgfsetdash{}{0pt}%
\pgfpathmoveto{\pgfqpoint{2.366192in}{1.540711in}}%
\pgfpathcurveto{\pgfqpoint{2.377242in}{1.540711in}}{\pgfqpoint{2.387841in}{1.545101in}}{\pgfqpoint{2.395655in}{1.552915in}}%
\pgfpathcurveto{\pgfqpoint{2.403469in}{1.560729in}}{\pgfqpoint{2.407859in}{1.571328in}}{\pgfqpoint{2.407859in}{1.582378in}}%
\pgfpathcurveto{\pgfqpoint{2.407859in}{1.593428in}}{\pgfqpoint{2.403469in}{1.604027in}}{\pgfqpoint{2.395655in}{1.611841in}}%
\pgfpathcurveto{\pgfqpoint{2.387841in}{1.619654in}}{\pgfqpoint{2.377242in}{1.624044in}}{\pgfqpoint{2.366192in}{1.624044in}}%
\pgfpathcurveto{\pgfqpoint{2.355142in}{1.624044in}}{\pgfqpoint{2.344543in}{1.619654in}}{\pgfqpoint{2.336729in}{1.611841in}}%
\pgfpathcurveto{\pgfqpoint{2.328916in}{1.604027in}}{\pgfqpoint{2.324526in}{1.593428in}}{\pgfqpoint{2.324526in}{1.582378in}}%
\pgfpathcurveto{\pgfqpoint{2.324526in}{1.571328in}}{\pgfqpoint{2.328916in}{1.560729in}}{\pgfqpoint{2.336729in}{1.552915in}}%
\pgfpathcurveto{\pgfqpoint{2.344543in}{1.545101in}}{\pgfqpoint{2.355142in}{1.540711in}}{\pgfqpoint{2.366192in}{1.540711in}}%
\pgfpathclose%
\pgfusepath{stroke,fill}%
\end{pgfscope}%
\begin{pgfscope}%
\pgfpathrectangle{\pgfqpoint{0.787074in}{0.548769in}}{\pgfqpoint{5.062926in}{3.102590in}}%
\pgfusepath{clip}%
\pgfsetbuttcap%
\pgfsetroundjoin%
\definecolor{currentfill}{rgb}{0.121569,0.466667,0.705882}%
\pgfsetfillcolor{currentfill}%
\pgfsetlinewidth{1.003750pt}%
\definecolor{currentstroke}{rgb}{0.121569,0.466667,0.705882}%
\pgfsetstrokecolor{currentstroke}%
\pgfsetdash{}{0pt}%
\pgfpathmoveto{\pgfqpoint{1.452511in}{1.951746in}}%
\pgfpathcurveto{\pgfqpoint{1.463561in}{1.951746in}}{\pgfqpoint{1.474160in}{1.956136in}}{\pgfqpoint{1.481974in}{1.963950in}}%
\pgfpathcurveto{\pgfqpoint{1.489788in}{1.971763in}}{\pgfqpoint{1.494178in}{1.982362in}}{\pgfqpoint{1.494178in}{1.993412in}}%
\pgfpathcurveto{\pgfqpoint{1.494178in}{2.004463in}}{\pgfqpoint{1.489788in}{2.015062in}}{\pgfqpoint{1.481974in}{2.022875in}}%
\pgfpathcurveto{\pgfqpoint{1.474160in}{2.030689in}}{\pgfqpoint{1.463561in}{2.035079in}}{\pgfqpoint{1.452511in}{2.035079in}}%
\pgfpathcurveto{\pgfqpoint{1.441461in}{2.035079in}}{\pgfqpoint{1.430862in}{2.030689in}}{\pgfqpoint{1.423048in}{2.022875in}}%
\pgfpathcurveto{\pgfqpoint{1.415235in}{2.015062in}}{\pgfqpoint{1.410844in}{2.004463in}}{\pgfqpoint{1.410844in}{1.993412in}}%
\pgfpathcurveto{\pgfqpoint{1.410844in}{1.982362in}}{\pgfqpoint{1.415235in}{1.971763in}}{\pgfqpoint{1.423048in}{1.963950in}}%
\pgfpathcurveto{\pgfqpoint{1.430862in}{1.956136in}}{\pgfqpoint{1.441461in}{1.951746in}}{\pgfqpoint{1.452511in}{1.951746in}}%
\pgfpathclose%
\pgfusepath{stroke,fill}%
\end{pgfscope}%
\begin{pgfscope}%
\pgfpathrectangle{\pgfqpoint{0.787074in}{0.548769in}}{\pgfqpoint{5.062926in}{3.102590in}}%
\pgfusepath{clip}%
\pgfsetbuttcap%
\pgfsetroundjoin%
\definecolor{currentfill}{rgb}{1.000000,0.498039,0.054902}%
\pgfsetfillcolor{currentfill}%
\pgfsetlinewidth{1.003750pt}%
\definecolor{currentstroke}{rgb}{1.000000,0.498039,0.054902}%
\pgfsetstrokecolor{currentstroke}%
\pgfsetdash{}{0pt}%
\pgfpathmoveto{\pgfqpoint{1.400197in}{1.644565in}}%
\pgfpathcurveto{\pgfqpoint{1.411247in}{1.644565in}}{\pgfqpoint{1.421846in}{1.648955in}}{\pgfqpoint{1.429659in}{1.656769in}}%
\pgfpathcurveto{\pgfqpoint{1.437473in}{1.664582in}}{\pgfqpoint{1.441863in}{1.675181in}}{\pgfqpoint{1.441863in}{1.686232in}}%
\pgfpathcurveto{\pgfqpoint{1.441863in}{1.697282in}}{\pgfqpoint{1.437473in}{1.707881in}}{\pgfqpoint{1.429659in}{1.715694in}}%
\pgfpathcurveto{\pgfqpoint{1.421846in}{1.723508in}}{\pgfqpoint{1.411247in}{1.727898in}}{\pgfqpoint{1.400197in}{1.727898in}}%
\pgfpathcurveto{\pgfqpoint{1.389146in}{1.727898in}}{\pgfqpoint{1.378547in}{1.723508in}}{\pgfqpoint{1.370734in}{1.715694in}}%
\pgfpathcurveto{\pgfqpoint{1.362920in}{1.707881in}}{\pgfqpoint{1.358530in}{1.697282in}}{\pgfqpoint{1.358530in}{1.686232in}}%
\pgfpathcurveto{\pgfqpoint{1.358530in}{1.675181in}}{\pgfqpoint{1.362920in}{1.664582in}}{\pgfqpoint{1.370734in}{1.656769in}}%
\pgfpathcurveto{\pgfqpoint{1.378547in}{1.648955in}}{\pgfqpoint{1.389146in}{1.644565in}}{\pgfqpoint{1.400197in}{1.644565in}}%
\pgfpathclose%
\pgfusepath{stroke,fill}%
\end{pgfscope}%
\begin{pgfscope}%
\pgfpathrectangle{\pgfqpoint{0.787074in}{0.548769in}}{\pgfqpoint{5.062926in}{3.102590in}}%
\pgfusepath{clip}%
\pgfsetbuttcap%
\pgfsetroundjoin%
\definecolor{currentfill}{rgb}{0.121569,0.466667,0.705882}%
\pgfsetfillcolor{currentfill}%
\pgfsetlinewidth{1.003750pt}%
\definecolor{currentstroke}{rgb}{0.121569,0.466667,0.705882}%
\pgfsetstrokecolor{currentstroke}%
\pgfsetdash{}{0pt}%
\pgfpathmoveto{\pgfqpoint{1.293363in}{0.664836in}}%
\pgfpathcurveto{\pgfqpoint{1.304413in}{0.664836in}}{\pgfqpoint{1.315012in}{0.669227in}}{\pgfqpoint{1.322826in}{0.677040in}}%
\pgfpathcurveto{\pgfqpoint{1.330639in}{0.684854in}}{\pgfqpoint{1.335030in}{0.695453in}}{\pgfqpoint{1.335030in}{0.706503in}}%
\pgfpathcurveto{\pgfqpoint{1.335030in}{0.717553in}}{\pgfqpoint{1.330639in}{0.728152in}}{\pgfqpoint{1.322826in}{0.735966in}}%
\pgfpathcurveto{\pgfqpoint{1.315012in}{0.743779in}}{\pgfqpoint{1.304413in}{0.748170in}}{\pgfqpoint{1.293363in}{0.748170in}}%
\pgfpathcurveto{\pgfqpoint{1.282313in}{0.748170in}}{\pgfqpoint{1.271714in}{0.743779in}}{\pgfqpoint{1.263900in}{0.735966in}}%
\pgfpathcurveto{\pgfqpoint{1.256087in}{0.728152in}}{\pgfqpoint{1.251696in}{0.717553in}}{\pgfqpoint{1.251696in}{0.706503in}}%
\pgfpathcurveto{\pgfqpoint{1.251696in}{0.695453in}}{\pgfqpoint{1.256087in}{0.684854in}}{\pgfqpoint{1.263900in}{0.677040in}}%
\pgfpathcurveto{\pgfqpoint{1.271714in}{0.669227in}}{\pgfqpoint{1.282313in}{0.664836in}}{\pgfqpoint{1.293363in}{0.664836in}}%
\pgfpathclose%
\pgfusepath{stroke,fill}%
\end{pgfscope}%
\begin{pgfscope}%
\pgfpathrectangle{\pgfqpoint{0.787074in}{0.548769in}}{\pgfqpoint{5.062926in}{3.102590in}}%
\pgfusepath{clip}%
\pgfsetbuttcap%
\pgfsetroundjoin%
\definecolor{currentfill}{rgb}{0.121569,0.466667,0.705882}%
\pgfsetfillcolor{currentfill}%
\pgfsetlinewidth{1.003750pt}%
\definecolor{currentstroke}{rgb}{0.121569,0.466667,0.705882}%
\pgfsetstrokecolor{currentstroke}%
\pgfsetdash{}{0pt}%
\pgfpathmoveto{\pgfqpoint{2.064557in}{1.428628in}}%
\pgfpathcurveto{\pgfqpoint{2.075607in}{1.428628in}}{\pgfqpoint{2.086206in}{1.433018in}}{\pgfqpoint{2.094020in}{1.440832in}}%
\pgfpathcurveto{\pgfqpoint{2.101833in}{1.448646in}}{\pgfqpoint{2.106224in}{1.459245in}}{\pgfqpoint{2.106224in}{1.470295in}}%
\pgfpathcurveto{\pgfqpoint{2.106224in}{1.481345in}}{\pgfqpoint{2.101833in}{1.491944in}}{\pgfqpoint{2.094020in}{1.499758in}}%
\pgfpathcurveto{\pgfqpoint{2.086206in}{1.507571in}}{\pgfqpoint{2.075607in}{1.511961in}}{\pgfqpoint{2.064557in}{1.511961in}}%
\pgfpathcurveto{\pgfqpoint{2.053507in}{1.511961in}}{\pgfqpoint{2.042908in}{1.507571in}}{\pgfqpoint{2.035094in}{1.499758in}}%
\pgfpathcurveto{\pgfqpoint{2.027281in}{1.491944in}}{\pgfqpoint{2.022890in}{1.481345in}}{\pgfqpoint{2.022890in}{1.470295in}}%
\pgfpathcurveto{\pgfqpoint{2.022890in}{1.459245in}}{\pgfqpoint{2.027281in}{1.448646in}}{\pgfqpoint{2.035094in}{1.440832in}}%
\pgfpathcurveto{\pgfqpoint{2.042908in}{1.433018in}}{\pgfqpoint{2.053507in}{1.428628in}}{\pgfqpoint{2.064557in}{1.428628in}}%
\pgfpathclose%
\pgfusepath{stroke,fill}%
\end{pgfscope}%
\begin{pgfscope}%
\pgfpathrectangle{\pgfqpoint{0.787074in}{0.548769in}}{\pgfqpoint{5.062926in}{3.102590in}}%
\pgfusepath{clip}%
\pgfsetbuttcap%
\pgfsetroundjoin%
\definecolor{currentfill}{rgb}{0.121569,0.466667,0.705882}%
\pgfsetfillcolor{currentfill}%
\pgfsetlinewidth{1.003750pt}%
\definecolor{currentstroke}{rgb}{0.121569,0.466667,0.705882}%
\pgfsetstrokecolor{currentstroke}%
\pgfsetdash{}{0pt}%
\pgfpathmoveto{\pgfqpoint{1.113086in}{0.648339in}}%
\pgfpathcurveto{\pgfqpoint{1.124136in}{0.648339in}}{\pgfqpoint{1.134735in}{0.652729in}}{\pgfqpoint{1.142548in}{0.660543in}}%
\pgfpathcurveto{\pgfqpoint{1.150362in}{0.668356in}}{\pgfqpoint{1.154752in}{0.678955in}}{\pgfqpoint{1.154752in}{0.690005in}}%
\pgfpathcurveto{\pgfqpoint{1.154752in}{0.701056in}}{\pgfqpoint{1.150362in}{0.711655in}}{\pgfqpoint{1.142548in}{0.719468in}}%
\pgfpathcurveto{\pgfqpoint{1.134735in}{0.727282in}}{\pgfqpoint{1.124136in}{0.731672in}}{\pgfqpoint{1.113086in}{0.731672in}}%
\pgfpathcurveto{\pgfqpoint{1.102036in}{0.731672in}}{\pgfqpoint{1.091437in}{0.727282in}}{\pgfqpoint{1.083623in}{0.719468in}}%
\pgfpathcurveto{\pgfqpoint{1.075809in}{0.711655in}}{\pgfqpoint{1.071419in}{0.701056in}}{\pgfqpoint{1.071419in}{0.690005in}}%
\pgfpathcurveto{\pgfqpoint{1.071419in}{0.678955in}}{\pgfqpoint{1.075809in}{0.668356in}}{\pgfqpoint{1.083623in}{0.660543in}}%
\pgfpathcurveto{\pgfqpoint{1.091437in}{0.652729in}}{\pgfqpoint{1.102036in}{0.648339in}}{\pgfqpoint{1.113086in}{0.648339in}}%
\pgfpathclose%
\pgfusepath{stroke,fill}%
\end{pgfscope}%
\begin{pgfscope}%
\pgfpathrectangle{\pgfqpoint{0.787074in}{0.548769in}}{\pgfqpoint{5.062926in}{3.102590in}}%
\pgfusepath{clip}%
\pgfsetbuttcap%
\pgfsetroundjoin%
\definecolor{currentfill}{rgb}{1.000000,0.498039,0.054902}%
\pgfsetfillcolor{currentfill}%
\pgfsetlinewidth{1.003750pt}%
\definecolor{currentstroke}{rgb}{1.000000,0.498039,0.054902}%
\pgfsetstrokecolor{currentstroke}%
\pgfsetdash{}{0pt}%
\pgfpathmoveto{\pgfqpoint{1.823376in}{1.507887in}}%
\pgfpathcurveto{\pgfqpoint{1.834426in}{1.507887in}}{\pgfqpoint{1.845025in}{1.512277in}}{\pgfqpoint{1.852839in}{1.520091in}}%
\pgfpathcurveto{\pgfqpoint{1.860653in}{1.527904in}}{\pgfqpoint{1.865043in}{1.538503in}}{\pgfqpoint{1.865043in}{1.549554in}}%
\pgfpathcurveto{\pgfqpoint{1.865043in}{1.560604in}}{\pgfqpoint{1.860653in}{1.571203in}}{\pgfqpoint{1.852839in}{1.579016in}}%
\pgfpathcurveto{\pgfqpoint{1.845025in}{1.586830in}}{\pgfqpoint{1.834426in}{1.591220in}}{\pgfqpoint{1.823376in}{1.591220in}}%
\pgfpathcurveto{\pgfqpoint{1.812326in}{1.591220in}}{\pgfqpoint{1.801727in}{1.586830in}}{\pgfqpoint{1.793913in}{1.579016in}}%
\pgfpathcurveto{\pgfqpoint{1.786100in}{1.571203in}}{\pgfqpoint{1.781709in}{1.560604in}}{\pgfqpoint{1.781709in}{1.549554in}}%
\pgfpathcurveto{\pgfqpoint{1.781709in}{1.538503in}}{\pgfqpoint{1.786100in}{1.527904in}}{\pgfqpoint{1.793913in}{1.520091in}}%
\pgfpathcurveto{\pgfqpoint{1.801727in}{1.512277in}}{\pgfqpoint{1.812326in}{1.507887in}}{\pgfqpoint{1.823376in}{1.507887in}}%
\pgfpathclose%
\pgfusepath{stroke,fill}%
\end{pgfscope}%
\begin{pgfscope}%
\pgfpathrectangle{\pgfqpoint{0.787074in}{0.548769in}}{\pgfqpoint{5.062926in}{3.102590in}}%
\pgfusepath{clip}%
\pgfsetbuttcap%
\pgfsetroundjoin%
\definecolor{currentfill}{rgb}{1.000000,0.498039,0.054902}%
\pgfsetfillcolor{currentfill}%
\pgfsetlinewidth{1.003750pt}%
\definecolor{currentstroke}{rgb}{1.000000,0.498039,0.054902}%
\pgfsetstrokecolor{currentstroke}%
\pgfsetdash{}{0pt}%
\pgfpathmoveto{\pgfqpoint{1.699526in}{2.621734in}}%
\pgfpathcurveto{\pgfqpoint{1.710576in}{2.621734in}}{\pgfqpoint{1.721175in}{2.626124in}}{\pgfqpoint{1.728988in}{2.633938in}}%
\pgfpathcurveto{\pgfqpoint{1.736802in}{2.641751in}}{\pgfqpoint{1.741192in}{2.652350in}}{\pgfqpoint{1.741192in}{2.663401in}}%
\pgfpathcurveto{\pgfqpoint{1.741192in}{2.674451in}}{\pgfqpoint{1.736802in}{2.685050in}}{\pgfqpoint{1.728988in}{2.692863in}}%
\pgfpathcurveto{\pgfqpoint{1.721175in}{2.700677in}}{\pgfqpoint{1.710576in}{2.705067in}}{\pgfqpoint{1.699526in}{2.705067in}}%
\pgfpathcurveto{\pgfqpoint{1.688475in}{2.705067in}}{\pgfqpoint{1.677876in}{2.700677in}}{\pgfqpoint{1.670063in}{2.692863in}}%
\pgfpathcurveto{\pgfqpoint{1.662249in}{2.685050in}}{\pgfqpoint{1.657859in}{2.674451in}}{\pgfqpoint{1.657859in}{2.663401in}}%
\pgfpathcurveto{\pgfqpoint{1.657859in}{2.652350in}}{\pgfqpoint{1.662249in}{2.641751in}}{\pgfqpoint{1.670063in}{2.633938in}}%
\pgfpathcurveto{\pgfqpoint{1.677876in}{2.626124in}}{\pgfqpoint{1.688475in}{2.621734in}}{\pgfqpoint{1.699526in}{2.621734in}}%
\pgfpathclose%
\pgfusepath{stroke,fill}%
\end{pgfscope}%
\begin{pgfscope}%
\pgfpathrectangle{\pgfqpoint{0.787074in}{0.548769in}}{\pgfqpoint{5.062926in}{3.102590in}}%
\pgfusepath{clip}%
\pgfsetbuttcap%
\pgfsetroundjoin%
\definecolor{currentfill}{rgb}{1.000000,0.498039,0.054902}%
\pgfsetfillcolor{currentfill}%
\pgfsetlinewidth{1.003750pt}%
\definecolor{currentstroke}{rgb}{1.000000,0.498039,0.054902}%
\pgfsetstrokecolor{currentstroke}%
\pgfsetdash{}{0pt}%
\pgfpathmoveto{\pgfqpoint{2.079802in}{1.832304in}}%
\pgfpathcurveto{\pgfqpoint{2.090852in}{1.832304in}}{\pgfqpoint{2.101451in}{1.836694in}}{\pgfqpoint{2.109265in}{1.844508in}}%
\pgfpathcurveto{\pgfqpoint{2.117078in}{1.852322in}}{\pgfqpoint{2.121469in}{1.862921in}}{\pgfqpoint{2.121469in}{1.873971in}}%
\pgfpathcurveto{\pgfqpoint{2.121469in}{1.885021in}}{\pgfqpoint{2.117078in}{1.895620in}}{\pgfqpoint{2.109265in}{1.903433in}}%
\pgfpathcurveto{\pgfqpoint{2.101451in}{1.911247in}}{\pgfqpoint{2.090852in}{1.915637in}}{\pgfqpoint{2.079802in}{1.915637in}}%
\pgfpathcurveto{\pgfqpoint{2.068752in}{1.915637in}}{\pgfqpoint{2.058153in}{1.911247in}}{\pgfqpoint{2.050339in}{1.903433in}}%
\pgfpathcurveto{\pgfqpoint{2.042526in}{1.895620in}}{\pgfqpoint{2.038135in}{1.885021in}}{\pgfqpoint{2.038135in}{1.873971in}}%
\pgfpathcurveto{\pgfqpoint{2.038135in}{1.862921in}}{\pgfqpoint{2.042526in}{1.852322in}}{\pgfqpoint{2.050339in}{1.844508in}}%
\pgfpathcurveto{\pgfqpoint{2.058153in}{1.836694in}}{\pgfqpoint{2.068752in}{1.832304in}}{\pgfqpoint{2.079802in}{1.832304in}}%
\pgfpathclose%
\pgfusepath{stroke,fill}%
\end{pgfscope}%
\begin{pgfscope}%
\pgfpathrectangle{\pgfqpoint{0.787074in}{0.548769in}}{\pgfqpoint{5.062926in}{3.102590in}}%
\pgfusepath{clip}%
\pgfsetbuttcap%
\pgfsetroundjoin%
\definecolor{currentfill}{rgb}{1.000000,0.498039,0.054902}%
\pgfsetfillcolor{currentfill}%
\pgfsetlinewidth{1.003750pt}%
\definecolor{currentstroke}{rgb}{1.000000,0.498039,0.054902}%
\pgfsetstrokecolor{currentstroke}%
\pgfsetdash{}{0pt}%
\pgfpathmoveto{\pgfqpoint{2.485388in}{3.055568in}}%
\pgfpathcurveto{\pgfqpoint{2.496438in}{3.055568in}}{\pgfqpoint{2.507037in}{3.059958in}}{\pgfqpoint{2.514851in}{3.067772in}}%
\pgfpathcurveto{\pgfqpoint{2.522664in}{3.075585in}}{\pgfqpoint{2.527055in}{3.086184in}}{\pgfqpoint{2.527055in}{3.097235in}}%
\pgfpathcurveto{\pgfqpoint{2.527055in}{3.108285in}}{\pgfqpoint{2.522664in}{3.118884in}}{\pgfqpoint{2.514851in}{3.126697in}}%
\pgfpathcurveto{\pgfqpoint{2.507037in}{3.134511in}}{\pgfqpoint{2.496438in}{3.138901in}}{\pgfqpoint{2.485388in}{3.138901in}}%
\pgfpathcurveto{\pgfqpoint{2.474338in}{3.138901in}}{\pgfqpoint{2.463739in}{3.134511in}}{\pgfqpoint{2.455925in}{3.126697in}}%
\pgfpathcurveto{\pgfqpoint{2.448112in}{3.118884in}}{\pgfqpoint{2.443721in}{3.108285in}}{\pgfqpoint{2.443721in}{3.097235in}}%
\pgfpathcurveto{\pgfqpoint{2.443721in}{3.086184in}}{\pgfqpoint{2.448112in}{3.075585in}}{\pgfqpoint{2.455925in}{3.067772in}}%
\pgfpathcurveto{\pgfqpoint{2.463739in}{3.059958in}}{\pgfqpoint{2.474338in}{3.055568in}}{\pgfqpoint{2.485388in}{3.055568in}}%
\pgfpathclose%
\pgfusepath{stroke,fill}%
\end{pgfscope}%
\begin{pgfscope}%
\pgfpathrectangle{\pgfqpoint{0.787074in}{0.548769in}}{\pgfqpoint{5.062926in}{3.102590in}}%
\pgfusepath{clip}%
\pgfsetbuttcap%
\pgfsetroundjoin%
\definecolor{currentfill}{rgb}{1.000000,0.498039,0.054902}%
\pgfsetfillcolor{currentfill}%
\pgfsetlinewidth{1.003750pt}%
\definecolor{currentstroke}{rgb}{1.000000,0.498039,0.054902}%
\pgfsetstrokecolor{currentstroke}%
\pgfsetdash{}{0pt}%
\pgfpathmoveto{\pgfqpoint{1.408777in}{3.291673in}}%
\pgfpathcurveto{\pgfqpoint{1.419827in}{3.291673in}}{\pgfqpoint{1.430426in}{3.296063in}}{\pgfqpoint{1.438240in}{3.303876in}}%
\pgfpathcurveto{\pgfqpoint{1.446054in}{3.311690in}}{\pgfqpoint{1.450444in}{3.322289in}}{\pgfqpoint{1.450444in}{3.333339in}}%
\pgfpathcurveto{\pgfqpoint{1.450444in}{3.344389in}}{\pgfqpoint{1.446054in}{3.354988in}}{\pgfqpoint{1.438240in}{3.362802in}}%
\pgfpathcurveto{\pgfqpoint{1.430426in}{3.370616in}}{\pgfqpoint{1.419827in}{3.375006in}}{\pgfqpoint{1.408777in}{3.375006in}}%
\pgfpathcurveto{\pgfqpoint{1.397727in}{3.375006in}}{\pgfqpoint{1.387128in}{3.370616in}}{\pgfqpoint{1.379314in}{3.362802in}}%
\pgfpathcurveto{\pgfqpoint{1.371501in}{3.354988in}}{\pgfqpoint{1.367111in}{3.344389in}}{\pgfqpoint{1.367111in}{3.333339in}}%
\pgfpathcurveto{\pgfqpoint{1.367111in}{3.322289in}}{\pgfqpoint{1.371501in}{3.311690in}}{\pgfqpoint{1.379314in}{3.303876in}}%
\pgfpathcurveto{\pgfqpoint{1.387128in}{3.296063in}}{\pgfqpoint{1.397727in}{3.291673in}}{\pgfqpoint{1.408777in}{3.291673in}}%
\pgfpathclose%
\pgfusepath{stroke,fill}%
\end{pgfscope}%
\begin{pgfscope}%
\pgfpathrectangle{\pgfqpoint{0.787074in}{0.548769in}}{\pgfqpoint{5.062926in}{3.102590in}}%
\pgfusepath{clip}%
\pgfsetbuttcap%
\pgfsetroundjoin%
\definecolor{currentfill}{rgb}{1.000000,0.498039,0.054902}%
\pgfsetfillcolor{currentfill}%
\pgfsetlinewidth{1.003750pt}%
\definecolor{currentstroke}{rgb}{1.000000,0.498039,0.054902}%
\pgfsetstrokecolor{currentstroke}%
\pgfsetdash{}{0pt}%
\pgfpathmoveto{\pgfqpoint{1.569367in}{1.890154in}}%
\pgfpathcurveto{\pgfqpoint{1.580417in}{1.890154in}}{\pgfqpoint{1.591016in}{1.894544in}}{\pgfqpoint{1.598829in}{1.902358in}}%
\pgfpathcurveto{\pgfqpoint{1.606643in}{1.910171in}}{\pgfqpoint{1.611033in}{1.920770in}}{\pgfqpoint{1.611033in}{1.931820in}}%
\pgfpathcurveto{\pgfqpoint{1.611033in}{1.942870in}}{\pgfqpoint{1.606643in}{1.953469in}}{\pgfqpoint{1.598829in}{1.961283in}}%
\pgfpathcurveto{\pgfqpoint{1.591016in}{1.969097in}}{\pgfqpoint{1.580417in}{1.973487in}}{\pgfqpoint{1.569367in}{1.973487in}}%
\pgfpathcurveto{\pgfqpoint{1.558317in}{1.973487in}}{\pgfqpoint{1.547717in}{1.969097in}}{\pgfqpoint{1.539904in}{1.961283in}}%
\pgfpathcurveto{\pgfqpoint{1.532090in}{1.953469in}}{\pgfqpoint{1.527700in}{1.942870in}}{\pgfqpoint{1.527700in}{1.931820in}}%
\pgfpathcurveto{\pgfqpoint{1.527700in}{1.920770in}}{\pgfqpoint{1.532090in}{1.910171in}}{\pgfqpoint{1.539904in}{1.902358in}}%
\pgfpathcurveto{\pgfqpoint{1.547717in}{1.894544in}}{\pgfqpoint{1.558317in}{1.890154in}}{\pgfqpoint{1.569367in}{1.890154in}}%
\pgfpathclose%
\pgfusepath{stroke,fill}%
\end{pgfscope}%
\begin{pgfscope}%
\pgfpathrectangle{\pgfqpoint{0.787074in}{0.548769in}}{\pgfqpoint{5.062926in}{3.102590in}}%
\pgfusepath{clip}%
\pgfsetbuttcap%
\pgfsetroundjoin%
\definecolor{currentfill}{rgb}{0.121569,0.466667,0.705882}%
\pgfsetfillcolor{currentfill}%
\pgfsetlinewidth{1.003750pt}%
\definecolor{currentstroke}{rgb}{0.121569,0.466667,0.705882}%
\pgfsetstrokecolor{currentstroke}%
\pgfsetdash{}{0pt}%
\pgfpathmoveto{\pgfqpoint{1.352088in}{1.784271in}}%
\pgfpathcurveto{\pgfqpoint{1.363138in}{1.784271in}}{\pgfqpoint{1.373737in}{1.788661in}}{\pgfqpoint{1.381550in}{1.796475in}}%
\pgfpathcurveto{\pgfqpoint{1.389364in}{1.804289in}}{\pgfqpoint{1.393754in}{1.814888in}}{\pgfqpoint{1.393754in}{1.825938in}}%
\pgfpathcurveto{\pgfqpoint{1.393754in}{1.836988in}}{\pgfqpoint{1.389364in}{1.847587in}}{\pgfqpoint{1.381550in}{1.855401in}}%
\pgfpathcurveto{\pgfqpoint{1.373737in}{1.863214in}}{\pgfqpoint{1.363138in}{1.867604in}}{\pgfqpoint{1.352088in}{1.867604in}}%
\pgfpathcurveto{\pgfqpoint{1.341037in}{1.867604in}}{\pgfqpoint{1.330438in}{1.863214in}}{\pgfqpoint{1.322625in}{1.855401in}}%
\pgfpathcurveto{\pgfqpoint{1.314811in}{1.847587in}}{\pgfqpoint{1.310421in}{1.836988in}}{\pgfqpoint{1.310421in}{1.825938in}}%
\pgfpathcurveto{\pgfqpoint{1.310421in}{1.814888in}}{\pgfqpoint{1.314811in}{1.804289in}}{\pgfqpoint{1.322625in}{1.796475in}}%
\pgfpathcurveto{\pgfqpoint{1.330438in}{1.788661in}}{\pgfqpoint{1.341037in}{1.784271in}}{\pgfqpoint{1.352088in}{1.784271in}}%
\pgfpathclose%
\pgfusepath{stroke,fill}%
\end{pgfscope}%
\begin{pgfscope}%
\pgfpathrectangle{\pgfqpoint{0.787074in}{0.548769in}}{\pgfqpoint{5.062926in}{3.102590in}}%
\pgfusepath{clip}%
\pgfsetbuttcap%
\pgfsetroundjoin%
\definecolor{currentfill}{rgb}{1.000000,0.498039,0.054902}%
\pgfsetfillcolor{currentfill}%
\pgfsetlinewidth{1.003750pt}%
\definecolor{currentstroke}{rgb}{1.000000,0.498039,0.054902}%
\pgfsetstrokecolor{currentstroke}%
\pgfsetdash{}{0pt}%
\pgfpathmoveto{\pgfqpoint{1.516603in}{1.980354in}}%
\pgfpathcurveto{\pgfqpoint{1.527653in}{1.980354in}}{\pgfqpoint{1.538252in}{1.984745in}}{\pgfqpoint{1.546066in}{1.992558in}}%
\pgfpathcurveto{\pgfqpoint{1.553879in}{2.000372in}}{\pgfqpoint{1.558269in}{2.010971in}}{\pgfqpoint{1.558269in}{2.022021in}}%
\pgfpathcurveto{\pgfqpoint{1.558269in}{2.033071in}}{\pgfqpoint{1.553879in}{2.043670in}}{\pgfqpoint{1.546066in}{2.051484in}}%
\pgfpathcurveto{\pgfqpoint{1.538252in}{2.059297in}}{\pgfqpoint{1.527653in}{2.063688in}}{\pgfqpoint{1.516603in}{2.063688in}}%
\pgfpathcurveto{\pgfqpoint{1.505553in}{2.063688in}}{\pgfqpoint{1.494954in}{2.059297in}}{\pgfqpoint{1.487140in}{2.051484in}}%
\pgfpathcurveto{\pgfqpoint{1.479326in}{2.043670in}}{\pgfqpoint{1.474936in}{2.033071in}}{\pgfqpoint{1.474936in}{2.022021in}}%
\pgfpathcurveto{\pgfqpoint{1.474936in}{2.010971in}}{\pgfqpoint{1.479326in}{2.000372in}}{\pgfqpoint{1.487140in}{1.992558in}}%
\pgfpathcurveto{\pgfqpoint{1.494954in}{1.984745in}}{\pgfqpoint{1.505553in}{1.980354in}}{\pgfqpoint{1.516603in}{1.980354in}}%
\pgfpathclose%
\pgfusepath{stroke,fill}%
\end{pgfscope}%
\begin{pgfscope}%
\pgfpathrectangle{\pgfqpoint{0.787074in}{0.548769in}}{\pgfqpoint{5.062926in}{3.102590in}}%
\pgfusepath{clip}%
\pgfsetbuttcap%
\pgfsetroundjoin%
\definecolor{currentfill}{rgb}{0.121569,0.466667,0.705882}%
\pgfsetfillcolor{currentfill}%
\pgfsetlinewidth{1.003750pt}%
\definecolor{currentstroke}{rgb}{0.121569,0.466667,0.705882}%
\pgfsetstrokecolor{currentstroke}%
\pgfsetdash{}{0pt}%
\pgfpathmoveto{\pgfqpoint{2.976144in}{1.870879in}}%
\pgfpathcurveto{\pgfqpoint{2.987194in}{1.870879in}}{\pgfqpoint{2.997793in}{1.875269in}}{\pgfqpoint{3.005607in}{1.883083in}}%
\pgfpathcurveto{\pgfqpoint{3.013420in}{1.890896in}}{\pgfqpoint{3.017811in}{1.901496in}}{\pgfqpoint{3.017811in}{1.912546in}}%
\pgfpathcurveto{\pgfqpoint{3.017811in}{1.923596in}}{\pgfqpoint{3.013420in}{1.934195in}}{\pgfqpoint{3.005607in}{1.942008in}}%
\pgfpathcurveto{\pgfqpoint{2.997793in}{1.949822in}}{\pgfqpoint{2.987194in}{1.954212in}}{\pgfqpoint{2.976144in}{1.954212in}}%
\pgfpathcurveto{\pgfqpoint{2.965094in}{1.954212in}}{\pgfqpoint{2.954495in}{1.949822in}}{\pgfqpoint{2.946681in}{1.942008in}}%
\pgfpathcurveto{\pgfqpoint{2.938868in}{1.934195in}}{\pgfqpoint{2.934477in}{1.923596in}}{\pgfqpoint{2.934477in}{1.912546in}}%
\pgfpathcurveto{\pgfqpoint{2.934477in}{1.901496in}}{\pgfqpoint{2.938868in}{1.890896in}}{\pgfqpoint{2.946681in}{1.883083in}}%
\pgfpathcurveto{\pgfqpoint{2.954495in}{1.875269in}}{\pgfqpoint{2.965094in}{1.870879in}}{\pgfqpoint{2.976144in}{1.870879in}}%
\pgfpathclose%
\pgfusepath{stroke,fill}%
\end{pgfscope}%
\begin{pgfscope}%
\pgfpathrectangle{\pgfqpoint{0.787074in}{0.548769in}}{\pgfqpoint{5.062926in}{3.102590in}}%
\pgfusepath{clip}%
\pgfsetbuttcap%
\pgfsetroundjoin%
\definecolor{currentfill}{rgb}{1.000000,0.498039,0.054902}%
\pgfsetfillcolor{currentfill}%
\pgfsetlinewidth{1.003750pt}%
\definecolor{currentstroke}{rgb}{1.000000,0.498039,0.054902}%
\pgfsetstrokecolor{currentstroke}%
\pgfsetdash{}{0pt}%
\pgfpathmoveto{\pgfqpoint{2.187916in}{1.599536in}}%
\pgfpathcurveto{\pgfqpoint{2.198966in}{1.599536in}}{\pgfqpoint{2.209565in}{1.603926in}}{\pgfqpoint{2.217379in}{1.611740in}}%
\pgfpathcurveto{\pgfqpoint{2.225192in}{1.619554in}}{\pgfqpoint{2.229583in}{1.630153in}}{\pgfqpoint{2.229583in}{1.641203in}}%
\pgfpathcurveto{\pgfqpoint{2.229583in}{1.652253in}}{\pgfqpoint{2.225192in}{1.662852in}}{\pgfqpoint{2.217379in}{1.670666in}}%
\pgfpathcurveto{\pgfqpoint{2.209565in}{1.678479in}}{\pgfqpoint{2.198966in}{1.682870in}}{\pgfqpoint{2.187916in}{1.682870in}}%
\pgfpathcurveto{\pgfqpoint{2.176866in}{1.682870in}}{\pgfqpoint{2.166267in}{1.678479in}}{\pgfqpoint{2.158453in}{1.670666in}}%
\pgfpathcurveto{\pgfqpoint{2.150639in}{1.662852in}}{\pgfqpoint{2.146249in}{1.652253in}}{\pgfqpoint{2.146249in}{1.641203in}}%
\pgfpathcurveto{\pgfqpoint{2.146249in}{1.630153in}}{\pgfqpoint{2.150639in}{1.619554in}}{\pgfqpoint{2.158453in}{1.611740in}}%
\pgfpathcurveto{\pgfqpoint{2.166267in}{1.603926in}}{\pgfqpoint{2.176866in}{1.599536in}}{\pgfqpoint{2.187916in}{1.599536in}}%
\pgfpathclose%
\pgfusepath{stroke,fill}%
\end{pgfscope}%
\begin{pgfscope}%
\pgfpathrectangle{\pgfqpoint{0.787074in}{0.548769in}}{\pgfqpoint{5.062926in}{3.102590in}}%
\pgfusepath{clip}%
\pgfsetbuttcap%
\pgfsetroundjoin%
\definecolor{currentfill}{rgb}{0.121569,0.466667,0.705882}%
\pgfsetfillcolor{currentfill}%
\pgfsetlinewidth{1.003750pt}%
\definecolor{currentstroke}{rgb}{0.121569,0.466667,0.705882}%
\pgfsetstrokecolor{currentstroke}%
\pgfsetdash{}{0pt}%
\pgfpathmoveto{\pgfqpoint{1.789741in}{1.739034in}}%
\pgfpathcurveto{\pgfqpoint{1.800791in}{1.739034in}}{\pgfqpoint{1.811390in}{1.743425in}}{\pgfqpoint{1.819203in}{1.751238in}}%
\pgfpathcurveto{\pgfqpoint{1.827017in}{1.759052in}}{\pgfqpoint{1.831407in}{1.769651in}}{\pgfqpoint{1.831407in}{1.780701in}}%
\pgfpathcurveto{\pgfqpoint{1.831407in}{1.791751in}}{\pgfqpoint{1.827017in}{1.802350in}}{\pgfqpoint{1.819203in}{1.810164in}}%
\pgfpathcurveto{\pgfqpoint{1.811390in}{1.817977in}}{\pgfqpoint{1.800791in}{1.822368in}}{\pgfqpoint{1.789741in}{1.822368in}}%
\pgfpathcurveto{\pgfqpoint{1.778690in}{1.822368in}}{\pgfqpoint{1.768091in}{1.817977in}}{\pgfqpoint{1.760278in}{1.810164in}}%
\pgfpathcurveto{\pgfqpoint{1.752464in}{1.802350in}}{\pgfqpoint{1.748074in}{1.791751in}}{\pgfqpoint{1.748074in}{1.780701in}}%
\pgfpathcurveto{\pgfqpoint{1.748074in}{1.769651in}}{\pgfqpoint{1.752464in}{1.759052in}}{\pgfqpoint{1.760278in}{1.751238in}}%
\pgfpathcurveto{\pgfqpoint{1.768091in}{1.743425in}}{\pgfqpoint{1.778690in}{1.739034in}}{\pgfqpoint{1.789741in}{1.739034in}}%
\pgfpathclose%
\pgfusepath{stroke,fill}%
\end{pgfscope}%
\begin{pgfscope}%
\pgfpathrectangle{\pgfqpoint{0.787074in}{0.548769in}}{\pgfqpoint{5.062926in}{3.102590in}}%
\pgfusepath{clip}%
\pgfsetbuttcap%
\pgfsetroundjoin%
\definecolor{currentfill}{rgb}{1.000000,0.498039,0.054902}%
\pgfsetfillcolor{currentfill}%
\pgfsetlinewidth{1.003750pt}%
\definecolor{currentstroke}{rgb}{1.000000,0.498039,0.054902}%
\pgfsetstrokecolor{currentstroke}%
\pgfsetdash{}{0pt}%
\pgfpathmoveto{\pgfqpoint{1.566721in}{2.553699in}}%
\pgfpathcurveto{\pgfqpoint{1.577771in}{2.553699in}}{\pgfqpoint{1.588370in}{2.558090in}}{\pgfqpoint{1.596184in}{2.565903in}}%
\pgfpathcurveto{\pgfqpoint{1.603998in}{2.573717in}}{\pgfqpoint{1.608388in}{2.584316in}}{\pgfqpoint{1.608388in}{2.595366in}}%
\pgfpathcurveto{\pgfqpoint{1.608388in}{2.606416in}}{\pgfqpoint{1.603998in}{2.617015in}}{\pgfqpoint{1.596184in}{2.624829in}}%
\pgfpathcurveto{\pgfqpoint{1.588370in}{2.632642in}}{\pgfqpoint{1.577771in}{2.637033in}}{\pgfqpoint{1.566721in}{2.637033in}}%
\pgfpathcurveto{\pgfqpoint{1.555671in}{2.637033in}}{\pgfqpoint{1.545072in}{2.632642in}}{\pgfqpoint{1.537258in}{2.624829in}}%
\pgfpathcurveto{\pgfqpoint{1.529445in}{2.617015in}}{\pgfqpoint{1.525055in}{2.606416in}}{\pgfqpoint{1.525055in}{2.595366in}}%
\pgfpathcurveto{\pgfqpoint{1.525055in}{2.584316in}}{\pgfqpoint{1.529445in}{2.573717in}}{\pgfqpoint{1.537258in}{2.565903in}}%
\pgfpathcurveto{\pgfqpoint{1.545072in}{2.558090in}}{\pgfqpoint{1.555671in}{2.553699in}}{\pgfqpoint{1.566721in}{2.553699in}}%
\pgfpathclose%
\pgfusepath{stroke,fill}%
\end{pgfscope}%
\begin{pgfscope}%
\pgfpathrectangle{\pgfqpoint{0.787074in}{0.548769in}}{\pgfqpoint{5.062926in}{3.102590in}}%
\pgfusepath{clip}%
\pgfsetbuttcap%
\pgfsetroundjoin%
\definecolor{currentfill}{rgb}{0.121569,0.466667,0.705882}%
\pgfsetfillcolor{currentfill}%
\pgfsetlinewidth{1.003750pt}%
\definecolor{currentstroke}{rgb}{0.121569,0.466667,0.705882}%
\pgfsetstrokecolor{currentstroke}%
\pgfsetdash{}{0pt}%
\pgfpathmoveto{\pgfqpoint{1.387775in}{2.326040in}}%
\pgfpathcurveto{\pgfqpoint{1.398825in}{2.326040in}}{\pgfqpoint{1.409424in}{2.330430in}}{\pgfqpoint{1.417238in}{2.338243in}}%
\pgfpathcurveto{\pgfqpoint{1.425051in}{2.346057in}}{\pgfqpoint{1.429442in}{2.356656in}}{\pgfqpoint{1.429442in}{2.367706in}}%
\pgfpathcurveto{\pgfqpoint{1.429442in}{2.378756in}}{\pgfqpoint{1.425051in}{2.389355in}}{\pgfqpoint{1.417238in}{2.397169in}}%
\pgfpathcurveto{\pgfqpoint{1.409424in}{2.404983in}}{\pgfqpoint{1.398825in}{2.409373in}}{\pgfqpoint{1.387775in}{2.409373in}}%
\pgfpathcurveto{\pgfqpoint{1.376725in}{2.409373in}}{\pgfqpoint{1.366126in}{2.404983in}}{\pgfqpoint{1.358312in}{2.397169in}}%
\pgfpathcurveto{\pgfqpoint{1.350499in}{2.389355in}}{\pgfqpoint{1.346108in}{2.378756in}}{\pgfqpoint{1.346108in}{2.367706in}}%
\pgfpathcurveto{\pgfqpoint{1.346108in}{2.356656in}}{\pgfqpoint{1.350499in}{2.346057in}}{\pgfqpoint{1.358312in}{2.338243in}}%
\pgfpathcurveto{\pgfqpoint{1.366126in}{2.330430in}}{\pgfqpoint{1.376725in}{2.326040in}}{\pgfqpoint{1.387775in}{2.326040in}}%
\pgfpathclose%
\pgfusepath{stroke,fill}%
\end{pgfscope}%
\begin{pgfscope}%
\pgfpathrectangle{\pgfqpoint{0.787074in}{0.548769in}}{\pgfqpoint{5.062926in}{3.102590in}}%
\pgfusepath{clip}%
\pgfsetbuttcap%
\pgfsetroundjoin%
\definecolor{currentfill}{rgb}{1.000000,0.498039,0.054902}%
\pgfsetfillcolor{currentfill}%
\pgfsetlinewidth{1.003750pt}%
\definecolor{currentstroke}{rgb}{1.000000,0.498039,0.054902}%
\pgfsetstrokecolor{currentstroke}%
\pgfsetdash{}{0pt}%
\pgfpathmoveto{\pgfqpoint{1.483247in}{2.769003in}}%
\pgfpathcurveto{\pgfqpoint{1.494297in}{2.769003in}}{\pgfqpoint{1.504896in}{2.773394in}}{\pgfqpoint{1.512710in}{2.781207in}}%
\pgfpathcurveto{\pgfqpoint{1.520523in}{2.789021in}}{\pgfqpoint{1.524914in}{2.799620in}}{\pgfqpoint{1.524914in}{2.810670in}}%
\pgfpathcurveto{\pgfqpoint{1.524914in}{2.821720in}}{\pgfqpoint{1.520523in}{2.832319in}}{\pgfqpoint{1.512710in}{2.840133in}}%
\pgfpathcurveto{\pgfqpoint{1.504896in}{2.847946in}}{\pgfqpoint{1.494297in}{2.852337in}}{\pgfqpoint{1.483247in}{2.852337in}}%
\pgfpathcurveto{\pgfqpoint{1.472197in}{2.852337in}}{\pgfqpoint{1.461598in}{2.847946in}}{\pgfqpoint{1.453784in}{2.840133in}}%
\pgfpathcurveto{\pgfqpoint{1.445971in}{2.832319in}}{\pgfqpoint{1.441580in}{2.821720in}}{\pgfqpoint{1.441580in}{2.810670in}}%
\pgfpathcurveto{\pgfqpoint{1.441580in}{2.799620in}}{\pgfqpoint{1.445971in}{2.789021in}}{\pgfqpoint{1.453784in}{2.781207in}}%
\pgfpathcurveto{\pgfqpoint{1.461598in}{2.773394in}}{\pgfqpoint{1.472197in}{2.769003in}}{\pgfqpoint{1.483247in}{2.769003in}}%
\pgfpathclose%
\pgfusepath{stroke,fill}%
\end{pgfscope}%
\begin{pgfscope}%
\pgfpathrectangle{\pgfqpoint{0.787074in}{0.548769in}}{\pgfqpoint{5.062926in}{3.102590in}}%
\pgfusepath{clip}%
\pgfsetbuttcap%
\pgfsetroundjoin%
\definecolor{currentfill}{rgb}{0.121569,0.466667,0.705882}%
\pgfsetfillcolor{currentfill}%
\pgfsetlinewidth{1.003750pt}%
\definecolor{currentstroke}{rgb}{0.121569,0.466667,0.705882}%
\pgfsetstrokecolor{currentstroke}%
\pgfsetdash{}{0pt}%
\pgfpathmoveto{\pgfqpoint{1.020183in}{0.659320in}}%
\pgfpathcurveto{\pgfqpoint{1.031233in}{0.659320in}}{\pgfqpoint{1.041832in}{0.663711in}}{\pgfqpoint{1.049646in}{0.671524in}}%
\pgfpathcurveto{\pgfqpoint{1.057459in}{0.679338in}}{\pgfqpoint{1.061850in}{0.689937in}}{\pgfqpoint{1.061850in}{0.700987in}}%
\pgfpathcurveto{\pgfqpoint{1.061850in}{0.712037in}}{\pgfqpoint{1.057459in}{0.722636in}}{\pgfqpoint{1.049646in}{0.730450in}}%
\pgfpathcurveto{\pgfqpoint{1.041832in}{0.738263in}}{\pgfqpoint{1.031233in}{0.742654in}}{\pgfqpoint{1.020183in}{0.742654in}}%
\pgfpathcurveto{\pgfqpoint{1.009133in}{0.742654in}}{\pgfqpoint{0.998534in}{0.738263in}}{\pgfqpoint{0.990720in}{0.730450in}}%
\pgfpathcurveto{\pgfqpoint{0.982907in}{0.722636in}}{\pgfqpoint{0.978516in}{0.712037in}}{\pgfqpoint{0.978516in}{0.700987in}}%
\pgfpathcurveto{\pgfqpoint{0.978516in}{0.689937in}}{\pgfqpoint{0.982907in}{0.679338in}}{\pgfqpoint{0.990720in}{0.671524in}}%
\pgfpathcurveto{\pgfqpoint{0.998534in}{0.663711in}}{\pgfqpoint{1.009133in}{0.659320in}}{\pgfqpoint{1.020183in}{0.659320in}}%
\pgfpathclose%
\pgfusepath{stroke,fill}%
\end{pgfscope}%
\begin{pgfscope}%
\pgfpathrectangle{\pgfqpoint{0.787074in}{0.548769in}}{\pgfqpoint{5.062926in}{3.102590in}}%
\pgfusepath{clip}%
\pgfsetbuttcap%
\pgfsetroundjoin%
\definecolor{currentfill}{rgb}{1.000000,0.498039,0.054902}%
\pgfsetfillcolor{currentfill}%
\pgfsetlinewidth{1.003750pt}%
\definecolor{currentstroke}{rgb}{1.000000,0.498039,0.054902}%
\pgfsetstrokecolor{currentstroke}%
\pgfsetdash{}{0pt}%
\pgfpathmoveto{\pgfqpoint{2.039926in}{1.946771in}}%
\pgfpathcurveto{\pgfqpoint{2.050976in}{1.946771in}}{\pgfqpoint{2.061575in}{1.951162in}}{\pgfqpoint{2.069389in}{1.958975in}}%
\pgfpathcurveto{\pgfqpoint{2.077202in}{1.966789in}}{\pgfqpoint{2.081593in}{1.977388in}}{\pgfqpoint{2.081593in}{1.988438in}}%
\pgfpathcurveto{\pgfqpoint{2.081593in}{1.999488in}}{\pgfqpoint{2.077202in}{2.010087in}}{\pgfqpoint{2.069389in}{2.017901in}}%
\pgfpathcurveto{\pgfqpoint{2.061575in}{2.025714in}}{\pgfqpoint{2.050976in}{2.030105in}}{\pgfqpoint{2.039926in}{2.030105in}}%
\pgfpathcurveto{\pgfqpoint{2.028876in}{2.030105in}}{\pgfqpoint{2.018277in}{2.025714in}}{\pgfqpoint{2.010463in}{2.017901in}}%
\pgfpathcurveto{\pgfqpoint{2.002650in}{2.010087in}}{\pgfqpoint{1.998259in}{1.999488in}}{\pgfqpoint{1.998259in}{1.988438in}}%
\pgfpathcurveto{\pgfqpoint{1.998259in}{1.977388in}}{\pgfqpoint{2.002650in}{1.966789in}}{\pgfqpoint{2.010463in}{1.958975in}}%
\pgfpathcurveto{\pgfqpoint{2.018277in}{1.951162in}}{\pgfqpoint{2.028876in}{1.946771in}}{\pgfqpoint{2.039926in}{1.946771in}}%
\pgfpathclose%
\pgfusepath{stroke,fill}%
\end{pgfscope}%
\begin{pgfscope}%
\pgfpathrectangle{\pgfqpoint{0.787074in}{0.548769in}}{\pgfqpoint{5.062926in}{3.102590in}}%
\pgfusepath{clip}%
\pgfsetbuttcap%
\pgfsetroundjoin%
\definecolor{currentfill}{rgb}{0.121569,0.466667,0.705882}%
\pgfsetfillcolor{currentfill}%
\pgfsetlinewidth{1.003750pt}%
\definecolor{currentstroke}{rgb}{0.121569,0.466667,0.705882}%
\pgfsetstrokecolor{currentstroke}%
\pgfsetdash{}{0pt}%
\pgfpathmoveto{\pgfqpoint{1.220208in}{0.648235in}}%
\pgfpathcurveto{\pgfqpoint{1.231258in}{0.648235in}}{\pgfqpoint{1.241857in}{0.652625in}}{\pgfqpoint{1.249670in}{0.660439in}}%
\pgfpathcurveto{\pgfqpoint{1.257484in}{0.668252in}}{\pgfqpoint{1.261874in}{0.678851in}}{\pgfqpoint{1.261874in}{0.689901in}}%
\pgfpathcurveto{\pgfqpoint{1.261874in}{0.700952in}}{\pgfqpoint{1.257484in}{0.711551in}}{\pgfqpoint{1.249670in}{0.719364in}}%
\pgfpathcurveto{\pgfqpoint{1.241857in}{0.727178in}}{\pgfqpoint{1.231258in}{0.731568in}}{\pgfqpoint{1.220208in}{0.731568in}}%
\pgfpathcurveto{\pgfqpoint{1.209157in}{0.731568in}}{\pgfqpoint{1.198558in}{0.727178in}}{\pgfqpoint{1.190745in}{0.719364in}}%
\pgfpathcurveto{\pgfqpoint{1.182931in}{0.711551in}}{\pgfqpoint{1.178541in}{0.700952in}}{\pgfqpoint{1.178541in}{0.689901in}}%
\pgfpathcurveto{\pgfqpoint{1.178541in}{0.678851in}}{\pgfqpoint{1.182931in}{0.668252in}}{\pgfqpoint{1.190745in}{0.660439in}}%
\pgfpathcurveto{\pgfqpoint{1.198558in}{0.652625in}}{\pgfqpoint{1.209157in}{0.648235in}}{\pgfqpoint{1.220208in}{0.648235in}}%
\pgfpathclose%
\pgfusepath{stroke,fill}%
\end{pgfscope}%
\begin{pgfscope}%
\pgfpathrectangle{\pgfqpoint{0.787074in}{0.548769in}}{\pgfqpoint{5.062926in}{3.102590in}}%
\pgfusepath{clip}%
\pgfsetbuttcap%
\pgfsetroundjoin%
\definecolor{currentfill}{rgb}{0.121569,0.466667,0.705882}%
\pgfsetfillcolor{currentfill}%
\pgfsetlinewidth{1.003750pt}%
\definecolor{currentstroke}{rgb}{0.121569,0.466667,0.705882}%
\pgfsetstrokecolor{currentstroke}%
\pgfsetdash{}{0pt}%
\pgfpathmoveto{\pgfqpoint{1.981575in}{3.081873in}}%
\pgfpathcurveto{\pgfqpoint{1.992625in}{3.081873in}}{\pgfqpoint{2.003224in}{3.086263in}}{\pgfqpoint{2.011037in}{3.094077in}}%
\pgfpathcurveto{\pgfqpoint{2.018851in}{3.101890in}}{\pgfqpoint{2.023241in}{3.112489in}}{\pgfqpoint{2.023241in}{3.123539in}}%
\pgfpathcurveto{\pgfqpoint{2.023241in}{3.134589in}}{\pgfqpoint{2.018851in}{3.145188in}}{\pgfqpoint{2.011037in}{3.153002in}}%
\pgfpathcurveto{\pgfqpoint{2.003224in}{3.160816in}}{\pgfqpoint{1.992625in}{3.165206in}}{\pgfqpoint{1.981575in}{3.165206in}}%
\pgfpathcurveto{\pgfqpoint{1.970524in}{3.165206in}}{\pgfqpoint{1.959925in}{3.160816in}}{\pgfqpoint{1.952112in}{3.153002in}}%
\pgfpathcurveto{\pgfqpoint{1.944298in}{3.145188in}}{\pgfqpoint{1.939908in}{3.134589in}}{\pgfqpoint{1.939908in}{3.123539in}}%
\pgfpathcurveto{\pgfqpoint{1.939908in}{3.112489in}}{\pgfqpoint{1.944298in}{3.101890in}}{\pgfqpoint{1.952112in}{3.094077in}}%
\pgfpathcurveto{\pgfqpoint{1.959925in}{3.086263in}}{\pgfqpoint{1.970524in}{3.081873in}}{\pgfqpoint{1.981575in}{3.081873in}}%
\pgfpathclose%
\pgfusepath{stroke,fill}%
\end{pgfscope}%
\begin{pgfscope}%
\pgfpathrectangle{\pgfqpoint{0.787074in}{0.548769in}}{\pgfqpoint{5.062926in}{3.102590in}}%
\pgfusepath{clip}%
\pgfsetbuttcap%
\pgfsetroundjoin%
\definecolor{currentfill}{rgb}{1.000000,0.498039,0.054902}%
\pgfsetfillcolor{currentfill}%
\pgfsetlinewidth{1.003750pt}%
\definecolor{currentstroke}{rgb}{1.000000,0.498039,0.054902}%
\pgfsetstrokecolor{currentstroke}%
\pgfsetdash{}{0pt}%
\pgfpathmoveto{\pgfqpoint{2.747817in}{2.902728in}}%
\pgfpathcurveto{\pgfqpoint{2.758867in}{2.902728in}}{\pgfqpoint{2.769466in}{2.907118in}}{\pgfqpoint{2.777280in}{2.914932in}}%
\pgfpathcurveto{\pgfqpoint{2.785093in}{2.922746in}}{\pgfqpoint{2.789484in}{2.933345in}}{\pgfqpoint{2.789484in}{2.944395in}}%
\pgfpathcurveto{\pgfqpoint{2.789484in}{2.955445in}}{\pgfqpoint{2.785093in}{2.966044in}}{\pgfqpoint{2.777280in}{2.973858in}}%
\pgfpathcurveto{\pgfqpoint{2.769466in}{2.981671in}}{\pgfqpoint{2.758867in}{2.986061in}}{\pgfqpoint{2.747817in}{2.986061in}}%
\pgfpathcurveto{\pgfqpoint{2.736767in}{2.986061in}}{\pgfqpoint{2.726168in}{2.981671in}}{\pgfqpoint{2.718354in}{2.973858in}}%
\pgfpathcurveto{\pgfqpoint{2.710540in}{2.966044in}}{\pgfqpoint{2.706150in}{2.955445in}}{\pgfqpoint{2.706150in}{2.944395in}}%
\pgfpathcurveto{\pgfqpoint{2.706150in}{2.933345in}}{\pgfqpoint{2.710540in}{2.922746in}}{\pgfqpoint{2.718354in}{2.914932in}}%
\pgfpathcurveto{\pgfqpoint{2.726168in}{2.907118in}}{\pgfqpoint{2.736767in}{2.902728in}}{\pgfqpoint{2.747817in}{2.902728in}}%
\pgfpathclose%
\pgfusepath{stroke,fill}%
\end{pgfscope}%
\begin{pgfscope}%
\pgfpathrectangle{\pgfqpoint{0.787074in}{0.548769in}}{\pgfqpoint{5.062926in}{3.102590in}}%
\pgfusepath{clip}%
\pgfsetbuttcap%
\pgfsetroundjoin%
\definecolor{currentfill}{rgb}{1.000000,0.498039,0.054902}%
\pgfsetfillcolor{currentfill}%
\pgfsetlinewidth{1.003750pt}%
\definecolor{currentstroke}{rgb}{1.000000,0.498039,0.054902}%
\pgfsetstrokecolor{currentstroke}%
\pgfsetdash{}{0pt}%
\pgfpathmoveto{\pgfqpoint{1.317011in}{2.766568in}}%
\pgfpathcurveto{\pgfqpoint{1.328061in}{2.766568in}}{\pgfqpoint{1.338660in}{2.770959in}}{\pgfqpoint{1.346473in}{2.778772in}}%
\pgfpathcurveto{\pgfqpoint{1.354287in}{2.786586in}}{\pgfqpoint{1.358677in}{2.797185in}}{\pgfqpoint{1.358677in}{2.808235in}}%
\pgfpathcurveto{\pgfqpoint{1.358677in}{2.819285in}}{\pgfqpoint{1.354287in}{2.829884in}}{\pgfqpoint{1.346473in}{2.837698in}}%
\pgfpathcurveto{\pgfqpoint{1.338660in}{2.845511in}}{\pgfqpoint{1.328061in}{2.849902in}}{\pgfqpoint{1.317011in}{2.849902in}}%
\pgfpathcurveto{\pgfqpoint{1.305960in}{2.849902in}}{\pgfqpoint{1.295361in}{2.845511in}}{\pgfqpoint{1.287548in}{2.837698in}}%
\pgfpathcurveto{\pgfqpoint{1.279734in}{2.829884in}}{\pgfqpoint{1.275344in}{2.819285in}}{\pgfqpoint{1.275344in}{2.808235in}}%
\pgfpathcurveto{\pgfqpoint{1.275344in}{2.797185in}}{\pgfqpoint{1.279734in}{2.786586in}}{\pgfqpoint{1.287548in}{2.778772in}}%
\pgfpathcurveto{\pgfqpoint{1.295361in}{2.770959in}}{\pgfqpoint{1.305960in}{2.766568in}}{\pgfqpoint{1.317011in}{2.766568in}}%
\pgfpathclose%
\pgfusepath{stroke,fill}%
\end{pgfscope}%
\begin{pgfscope}%
\pgfpathrectangle{\pgfqpoint{0.787074in}{0.548769in}}{\pgfqpoint{5.062926in}{3.102590in}}%
\pgfusepath{clip}%
\pgfsetbuttcap%
\pgfsetroundjoin%
\definecolor{currentfill}{rgb}{1.000000,0.498039,0.054902}%
\pgfsetfillcolor{currentfill}%
\pgfsetlinewidth{1.003750pt}%
\definecolor{currentstroke}{rgb}{1.000000,0.498039,0.054902}%
\pgfsetstrokecolor{currentstroke}%
\pgfsetdash{}{0pt}%
\pgfpathmoveto{\pgfqpoint{1.925996in}{1.775111in}}%
\pgfpathcurveto{\pgfqpoint{1.937046in}{1.775111in}}{\pgfqpoint{1.947645in}{1.779501in}}{\pgfqpoint{1.955458in}{1.787315in}}%
\pgfpathcurveto{\pgfqpoint{1.963272in}{1.795129in}}{\pgfqpoint{1.967662in}{1.805728in}}{\pgfqpoint{1.967662in}{1.816778in}}%
\pgfpathcurveto{\pgfqpoint{1.967662in}{1.827828in}}{\pgfqpoint{1.963272in}{1.838427in}}{\pgfqpoint{1.955458in}{1.846241in}}%
\pgfpathcurveto{\pgfqpoint{1.947645in}{1.854054in}}{\pgfqpoint{1.937046in}{1.858444in}}{\pgfqpoint{1.925996in}{1.858444in}}%
\pgfpathcurveto{\pgfqpoint{1.914946in}{1.858444in}}{\pgfqpoint{1.904347in}{1.854054in}}{\pgfqpoint{1.896533in}{1.846241in}}%
\pgfpathcurveto{\pgfqpoint{1.888719in}{1.838427in}}{\pgfqpoint{1.884329in}{1.827828in}}{\pgfqpoint{1.884329in}{1.816778in}}%
\pgfpathcurveto{\pgfqpoint{1.884329in}{1.805728in}}{\pgfqpoint{1.888719in}{1.795129in}}{\pgfqpoint{1.896533in}{1.787315in}}%
\pgfpathcurveto{\pgfqpoint{1.904347in}{1.779501in}}{\pgfqpoint{1.914946in}{1.775111in}}{\pgfqpoint{1.925996in}{1.775111in}}%
\pgfpathclose%
\pgfusepath{stroke,fill}%
\end{pgfscope}%
\begin{pgfscope}%
\pgfpathrectangle{\pgfqpoint{0.787074in}{0.548769in}}{\pgfqpoint{5.062926in}{3.102590in}}%
\pgfusepath{clip}%
\pgfsetbuttcap%
\pgfsetroundjoin%
\definecolor{currentfill}{rgb}{1.000000,0.498039,0.054902}%
\pgfsetfillcolor{currentfill}%
\pgfsetlinewidth{1.003750pt}%
\definecolor{currentstroke}{rgb}{1.000000,0.498039,0.054902}%
\pgfsetstrokecolor{currentstroke}%
\pgfsetdash{}{0pt}%
\pgfpathmoveto{\pgfqpoint{1.154980in}{1.545971in}}%
\pgfpathcurveto{\pgfqpoint{1.166030in}{1.545971in}}{\pgfqpoint{1.176629in}{1.550361in}}{\pgfqpoint{1.184442in}{1.558175in}}%
\pgfpathcurveto{\pgfqpoint{1.192256in}{1.565989in}}{\pgfqpoint{1.196646in}{1.576588in}}{\pgfqpoint{1.196646in}{1.587638in}}%
\pgfpathcurveto{\pgfqpoint{1.196646in}{1.598688in}}{\pgfqpoint{1.192256in}{1.609287in}}{\pgfqpoint{1.184442in}{1.617101in}}%
\pgfpathcurveto{\pgfqpoint{1.176629in}{1.624914in}}{\pgfqpoint{1.166030in}{1.629305in}}{\pgfqpoint{1.154980in}{1.629305in}}%
\pgfpathcurveto{\pgfqpoint{1.143930in}{1.629305in}}{\pgfqpoint{1.133331in}{1.624914in}}{\pgfqpoint{1.125517in}{1.617101in}}%
\pgfpathcurveto{\pgfqpoint{1.117703in}{1.609287in}}{\pgfqpoint{1.113313in}{1.598688in}}{\pgfqpoint{1.113313in}{1.587638in}}%
\pgfpathcurveto{\pgfqpoint{1.113313in}{1.576588in}}{\pgfqpoint{1.117703in}{1.565989in}}{\pgfqpoint{1.125517in}{1.558175in}}%
\pgfpathcurveto{\pgfqpoint{1.133331in}{1.550361in}}{\pgfqpoint{1.143930in}{1.545971in}}{\pgfqpoint{1.154980in}{1.545971in}}%
\pgfpathclose%
\pgfusepath{stroke,fill}%
\end{pgfscope}%
\begin{pgfscope}%
\pgfpathrectangle{\pgfqpoint{0.787074in}{0.548769in}}{\pgfqpoint{5.062926in}{3.102590in}}%
\pgfusepath{clip}%
\pgfsetbuttcap%
\pgfsetroundjoin%
\definecolor{currentfill}{rgb}{0.121569,0.466667,0.705882}%
\pgfsetfillcolor{currentfill}%
\pgfsetlinewidth{1.003750pt}%
\definecolor{currentstroke}{rgb}{0.121569,0.466667,0.705882}%
\pgfsetstrokecolor{currentstroke}%
\pgfsetdash{}{0pt}%
\pgfpathmoveto{\pgfqpoint{1.564771in}{2.232840in}}%
\pgfpathcurveto{\pgfqpoint{1.575821in}{2.232840in}}{\pgfqpoint{1.586420in}{2.237230in}}{\pgfqpoint{1.594234in}{2.245044in}}%
\pgfpathcurveto{\pgfqpoint{1.602048in}{2.252857in}}{\pgfqpoint{1.606438in}{2.263456in}}{\pgfqpoint{1.606438in}{2.274506in}}%
\pgfpathcurveto{\pgfqpoint{1.606438in}{2.285556in}}{\pgfqpoint{1.602048in}{2.296156in}}{\pgfqpoint{1.594234in}{2.303969in}}%
\pgfpathcurveto{\pgfqpoint{1.586420in}{2.311783in}}{\pgfqpoint{1.575821in}{2.316173in}}{\pgfqpoint{1.564771in}{2.316173in}}%
\pgfpathcurveto{\pgfqpoint{1.553721in}{2.316173in}}{\pgfqpoint{1.543122in}{2.311783in}}{\pgfqpoint{1.535308in}{2.303969in}}%
\pgfpathcurveto{\pgfqpoint{1.527495in}{2.296156in}}{\pgfqpoint{1.523104in}{2.285556in}}{\pgfqpoint{1.523104in}{2.274506in}}%
\pgfpathcurveto{\pgfqpoint{1.523104in}{2.263456in}}{\pgfqpoint{1.527495in}{2.252857in}}{\pgfqpoint{1.535308in}{2.245044in}}%
\pgfpathcurveto{\pgfqpoint{1.543122in}{2.237230in}}{\pgfqpoint{1.553721in}{2.232840in}}{\pgfqpoint{1.564771in}{2.232840in}}%
\pgfpathclose%
\pgfusepath{stroke,fill}%
\end{pgfscope}%
\begin{pgfscope}%
\pgfpathrectangle{\pgfqpoint{0.787074in}{0.548769in}}{\pgfqpoint{5.062926in}{3.102590in}}%
\pgfusepath{clip}%
\pgfsetbuttcap%
\pgfsetroundjoin%
\definecolor{currentfill}{rgb}{0.121569,0.466667,0.705882}%
\pgfsetfillcolor{currentfill}%
\pgfsetlinewidth{1.003750pt}%
\definecolor{currentstroke}{rgb}{0.121569,0.466667,0.705882}%
\pgfsetstrokecolor{currentstroke}%
\pgfsetdash{}{0pt}%
\pgfpathmoveto{\pgfqpoint{1.466806in}{0.676149in}}%
\pgfpathcurveto{\pgfqpoint{1.477857in}{0.676149in}}{\pgfqpoint{1.488456in}{0.680539in}}{\pgfqpoint{1.496269in}{0.688353in}}%
\pgfpathcurveto{\pgfqpoint{1.504083in}{0.696167in}}{\pgfqpoint{1.508473in}{0.706766in}}{\pgfqpoint{1.508473in}{0.717816in}}%
\pgfpathcurveto{\pgfqpoint{1.508473in}{0.728866in}}{\pgfqpoint{1.504083in}{0.739465in}}{\pgfqpoint{1.496269in}{0.747279in}}%
\pgfpathcurveto{\pgfqpoint{1.488456in}{0.755092in}}{\pgfqpoint{1.477857in}{0.759482in}}{\pgfqpoint{1.466806in}{0.759482in}}%
\pgfpathcurveto{\pgfqpoint{1.455756in}{0.759482in}}{\pgfqpoint{1.445157in}{0.755092in}}{\pgfqpoint{1.437344in}{0.747279in}}%
\pgfpathcurveto{\pgfqpoint{1.429530in}{0.739465in}}{\pgfqpoint{1.425140in}{0.728866in}}{\pgfqpoint{1.425140in}{0.717816in}}%
\pgfpathcurveto{\pgfqpoint{1.425140in}{0.706766in}}{\pgfqpoint{1.429530in}{0.696167in}}{\pgfqpoint{1.437344in}{0.688353in}}%
\pgfpathcurveto{\pgfqpoint{1.445157in}{0.680539in}}{\pgfqpoint{1.455756in}{0.676149in}}{\pgfqpoint{1.466806in}{0.676149in}}%
\pgfpathclose%
\pgfusepath{stroke,fill}%
\end{pgfscope}%
\begin{pgfscope}%
\pgfpathrectangle{\pgfqpoint{0.787074in}{0.548769in}}{\pgfqpoint{5.062926in}{3.102590in}}%
\pgfusepath{clip}%
\pgfsetbuttcap%
\pgfsetroundjoin%
\definecolor{currentfill}{rgb}{1.000000,0.498039,0.054902}%
\pgfsetfillcolor{currentfill}%
\pgfsetlinewidth{1.003750pt}%
\definecolor{currentstroke}{rgb}{1.000000,0.498039,0.054902}%
\pgfsetstrokecolor{currentstroke}%
\pgfsetdash{}{0pt}%
\pgfpathmoveto{\pgfqpoint{2.465005in}{1.907282in}}%
\pgfpathcurveto{\pgfqpoint{2.476055in}{1.907282in}}{\pgfqpoint{2.486654in}{1.911673in}}{\pgfqpoint{2.494468in}{1.919486in}}%
\pgfpathcurveto{\pgfqpoint{2.502281in}{1.927300in}}{\pgfqpoint{2.506671in}{1.937899in}}{\pgfqpoint{2.506671in}{1.948949in}}%
\pgfpathcurveto{\pgfqpoint{2.506671in}{1.959999in}}{\pgfqpoint{2.502281in}{1.970598in}}{\pgfqpoint{2.494468in}{1.978412in}}%
\pgfpathcurveto{\pgfqpoint{2.486654in}{1.986225in}}{\pgfqpoint{2.476055in}{1.990616in}}{\pgfqpoint{2.465005in}{1.990616in}}%
\pgfpathcurveto{\pgfqpoint{2.453955in}{1.990616in}}{\pgfqpoint{2.443356in}{1.986225in}}{\pgfqpoint{2.435542in}{1.978412in}}%
\pgfpathcurveto{\pgfqpoint{2.427728in}{1.970598in}}{\pgfqpoint{2.423338in}{1.959999in}}{\pgfqpoint{2.423338in}{1.948949in}}%
\pgfpathcurveto{\pgfqpoint{2.423338in}{1.937899in}}{\pgfqpoint{2.427728in}{1.927300in}}{\pgfqpoint{2.435542in}{1.919486in}}%
\pgfpathcurveto{\pgfqpoint{2.443356in}{1.911673in}}{\pgfqpoint{2.453955in}{1.907282in}}{\pgfqpoint{2.465005in}{1.907282in}}%
\pgfpathclose%
\pgfusepath{stroke,fill}%
\end{pgfscope}%
\begin{pgfscope}%
\pgfpathrectangle{\pgfqpoint{0.787074in}{0.548769in}}{\pgfqpoint{5.062926in}{3.102590in}}%
\pgfusepath{clip}%
\pgfsetbuttcap%
\pgfsetroundjoin%
\definecolor{currentfill}{rgb}{1.000000,0.498039,0.054902}%
\pgfsetfillcolor{currentfill}%
\pgfsetlinewidth{1.003750pt}%
\definecolor{currentstroke}{rgb}{1.000000,0.498039,0.054902}%
\pgfsetstrokecolor{currentstroke}%
\pgfsetdash{}{0pt}%
\pgfpathmoveto{\pgfqpoint{1.913896in}{2.839399in}}%
\pgfpathcurveto{\pgfqpoint{1.924946in}{2.839399in}}{\pgfqpoint{1.935546in}{2.843789in}}{\pgfqpoint{1.943359in}{2.851603in}}%
\pgfpathcurveto{\pgfqpoint{1.951173in}{2.859416in}}{\pgfqpoint{1.955563in}{2.870015in}}{\pgfqpoint{1.955563in}{2.881065in}}%
\pgfpathcurveto{\pgfqpoint{1.955563in}{2.892115in}}{\pgfqpoint{1.951173in}{2.902715in}}{\pgfqpoint{1.943359in}{2.910528in}}%
\pgfpathcurveto{\pgfqpoint{1.935546in}{2.918342in}}{\pgfqpoint{1.924946in}{2.922732in}}{\pgfqpoint{1.913896in}{2.922732in}}%
\pgfpathcurveto{\pgfqpoint{1.902846in}{2.922732in}}{\pgfqpoint{1.892247in}{2.918342in}}{\pgfqpoint{1.884434in}{2.910528in}}%
\pgfpathcurveto{\pgfqpoint{1.876620in}{2.902715in}}{\pgfqpoint{1.872230in}{2.892115in}}{\pgfqpoint{1.872230in}{2.881065in}}%
\pgfpathcurveto{\pgfqpoint{1.872230in}{2.870015in}}{\pgfqpoint{1.876620in}{2.859416in}}{\pgfqpoint{1.884434in}{2.851603in}}%
\pgfpathcurveto{\pgfqpoint{1.892247in}{2.843789in}}{\pgfqpoint{1.902846in}{2.839399in}}{\pgfqpoint{1.913896in}{2.839399in}}%
\pgfpathclose%
\pgfusepath{stroke,fill}%
\end{pgfscope}%
\begin{pgfscope}%
\pgfpathrectangle{\pgfqpoint{0.787074in}{0.548769in}}{\pgfqpoint{5.062926in}{3.102590in}}%
\pgfusepath{clip}%
\pgfsetbuttcap%
\pgfsetroundjoin%
\definecolor{currentfill}{rgb}{0.121569,0.466667,0.705882}%
\pgfsetfillcolor{currentfill}%
\pgfsetlinewidth{1.003750pt}%
\definecolor{currentstroke}{rgb}{0.121569,0.466667,0.705882}%
\pgfsetstrokecolor{currentstroke}%
\pgfsetdash{}{0pt}%
\pgfpathmoveto{\pgfqpoint{1.415840in}{0.648138in}}%
\pgfpathcurveto{\pgfqpoint{1.426890in}{0.648138in}}{\pgfqpoint{1.437489in}{0.652528in}}{\pgfqpoint{1.445303in}{0.660342in}}%
\pgfpathcurveto{\pgfqpoint{1.453116in}{0.668156in}}{\pgfqpoint{1.457507in}{0.678755in}}{\pgfqpoint{1.457507in}{0.689805in}}%
\pgfpathcurveto{\pgfqpoint{1.457507in}{0.700855in}}{\pgfqpoint{1.453116in}{0.711454in}}{\pgfqpoint{1.445303in}{0.719268in}}%
\pgfpathcurveto{\pgfqpoint{1.437489in}{0.727081in}}{\pgfqpoint{1.426890in}{0.731472in}}{\pgfqpoint{1.415840in}{0.731472in}}%
\pgfpathcurveto{\pgfqpoint{1.404790in}{0.731472in}}{\pgfqpoint{1.394191in}{0.727081in}}{\pgfqpoint{1.386377in}{0.719268in}}%
\pgfpathcurveto{\pgfqpoint{1.378564in}{0.711454in}}{\pgfqpoint{1.374173in}{0.700855in}}{\pgfqpoint{1.374173in}{0.689805in}}%
\pgfpathcurveto{\pgfqpoint{1.374173in}{0.678755in}}{\pgfqpoint{1.378564in}{0.668156in}}{\pgfqpoint{1.386377in}{0.660342in}}%
\pgfpathcurveto{\pgfqpoint{1.394191in}{0.652528in}}{\pgfqpoint{1.404790in}{0.648138in}}{\pgfqpoint{1.415840in}{0.648138in}}%
\pgfpathclose%
\pgfusepath{stroke,fill}%
\end{pgfscope}%
\begin{pgfscope}%
\pgfpathrectangle{\pgfqpoint{0.787074in}{0.548769in}}{\pgfqpoint{5.062926in}{3.102590in}}%
\pgfusepath{clip}%
\pgfsetbuttcap%
\pgfsetroundjoin%
\definecolor{currentfill}{rgb}{0.121569,0.466667,0.705882}%
\pgfsetfillcolor{currentfill}%
\pgfsetlinewidth{1.003750pt}%
\definecolor{currentstroke}{rgb}{0.121569,0.466667,0.705882}%
\pgfsetstrokecolor{currentstroke}%
\pgfsetdash{}{0pt}%
\pgfpathmoveto{\pgfqpoint{1.073617in}{0.649828in}}%
\pgfpathcurveto{\pgfqpoint{1.084667in}{0.649828in}}{\pgfqpoint{1.095266in}{0.654218in}}{\pgfqpoint{1.103079in}{0.662032in}}%
\pgfpathcurveto{\pgfqpoint{1.110893in}{0.669845in}}{\pgfqpoint{1.115283in}{0.680444in}}{\pgfqpoint{1.115283in}{0.691495in}}%
\pgfpathcurveto{\pgfqpoint{1.115283in}{0.702545in}}{\pgfqpoint{1.110893in}{0.713144in}}{\pgfqpoint{1.103079in}{0.720957in}}%
\pgfpathcurveto{\pgfqpoint{1.095266in}{0.728771in}}{\pgfqpoint{1.084667in}{0.733161in}}{\pgfqpoint{1.073617in}{0.733161in}}%
\pgfpathcurveto{\pgfqpoint{1.062567in}{0.733161in}}{\pgfqpoint{1.051967in}{0.728771in}}{\pgfqpoint{1.044154in}{0.720957in}}%
\pgfpathcurveto{\pgfqpoint{1.036340in}{0.713144in}}{\pgfqpoint{1.031950in}{0.702545in}}{\pgfqpoint{1.031950in}{0.691495in}}%
\pgfpathcurveto{\pgfqpoint{1.031950in}{0.680444in}}{\pgfqpoint{1.036340in}{0.669845in}}{\pgfqpoint{1.044154in}{0.662032in}}%
\pgfpathcurveto{\pgfqpoint{1.051967in}{0.654218in}}{\pgfqpoint{1.062567in}{0.649828in}}{\pgfqpoint{1.073617in}{0.649828in}}%
\pgfpathclose%
\pgfusepath{stroke,fill}%
\end{pgfscope}%
\begin{pgfscope}%
\pgfpathrectangle{\pgfqpoint{0.787074in}{0.548769in}}{\pgfqpoint{5.062926in}{3.102590in}}%
\pgfusepath{clip}%
\pgfsetbuttcap%
\pgfsetroundjoin%
\definecolor{currentfill}{rgb}{0.121569,0.466667,0.705882}%
\pgfsetfillcolor{currentfill}%
\pgfsetlinewidth{1.003750pt}%
\definecolor{currentstroke}{rgb}{0.121569,0.466667,0.705882}%
\pgfsetstrokecolor{currentstroke}%
\pgfsetdash{}{0pt}%
\pgfpathmoveto{\pgfqpoint{1.915516in}{2.575639in}}%
\pgfpathcurveto{\pgfqpoint{1.926566in}{2.575639in}}{\pgfqpoint{1.937165in}{2.580029in}}{\pgfqpoint{1.944979in}{2.587842in}}%
\pgfpathcurveto{\pgfqpoint{1.952792in}{2.595656in}}{\pgfqpoint{1.957182in}{2.606255in}}{\pgfqpoint{1.957182in}{2.617305in}}%
\pgfpathcurveto{\pgfqpoint{1.957182in}{2.628355in}}{\pgfqpoint{1.952792in}{2.638954in}}{\pgfqpoint{1.944979in}{2.646768in}}%
\pgfpathcurveto{\pgfqpoint{1.937165in}{2.654582in}}{\pgfqpoint{1.926566in}{2.658972in}}{\pgfqpoint{1.915516in}{2.658972in}}%
\pgfpathcurveto{\pgfqpoint{1.904466in}{2.658972in}}{\pgfqpoint{1.893867in}{2.654582in}}{\pgfqpoint{1.886053in}{2.646768in}}%
\pgfpathcurveto{\pgfqpoint{1.878239in}{2.638954in}}{\pgfqpoint{1.873849in}{2.628355in}}{\pgfqpoint{1.873849in}{2.617305in}}%
\pgfpathcurveto{\pgfqpoint{1.873849in}{2.606255in}}{\pgfqpoint{1.878239in}{2.595656in}}{\pgfqpoint{1.886053in}{2.587842in}}%
\pgfpathcurveto{\pgfqpoint{1.893867in}{2.580029in}}{\pgfqpoint{1.904466in}{2.575639in}}{\pgfqpoint{1.915516in}{2.575639in}}%
\pgfpathclose%
\pgfusepath{stroke,fill}%
\end{pgfscope}%
\begin{pgfscope}%
\pgfpathrectangle{\pgfqpoint{0.787074in}{0.548769in}}{\pgfqpoint{5.062926in}{3.102590in}}%
\pgfusepath{clip}%
\pgfsetbuttcap%
\pgfsetroundjoin%
\definecolor{currentfill}{rgb}{1.000000,0.498039,0.054902}%
\pgfsetfillcolor{currentfill}%
\pgfsetlinewidth{1.003750pt}%
\definecolor{currentstroke}{rgb}{1.000000,0.498039,0.054902}%
\pgfsetstrokecolor{currentstroke}%
\pgfsetdash{}{0pt}%
\pgfpathmoveto{\pgfqpoint{3.051199in}{2.218400in}}%
\pgfpathcurveto{\pgfqpoint{3.062249in}{2.218400in}}{\pgfqpoint{3.072848in}{2.222790in}}{\pgfqpoint{3.080661in}{2.230604in}}%
\pgfpathcurveto{\pgfqpoint{3.088475in}{2.238417in}}{\pgfqpoint{3.092865in}{2.249016in}}{\pgfqpoint{3.092865in}{2.260066in}}%
\pgfpathcurveto{\pgfqpoint{3.092865in}{2.271116in}}{\pgfqpoint{3.088475in}{2.281715in}}{\pgfqpoint{3.080661in}{2.289529in}}%
\pgfpathcurveto{\pgfqpoint{3.072848in}{2.297343in}}{\pgfqpoint{3.062249in}{2.301733in}}{\pgfqpoint{3.051199in}{2.301733in}}%
\pgfpathcurveto{\pgfqpoint{3.040149in}{2.301733in}}{\pgfqpoint{3.029550in}{2.297343in}}{\pgfqpoint{3.021736in}{2.289529in}}%
\pgfpathcurveto{\pgfqpoint{3.013922in}{2.281715in}}{\pgfqpoint{3.009532in}{2.271116in}}{\pgfqpoint{3.009532in}{2.260066in}}%
\pgfpathcurveto{\pgfqpoint{3.009532in}{2.249016in}}{\pgfqpoint{3.013922in}{2.238417in}}{\pgfqpoint{3.021736in}{2.230604in}}%
\pgfpathcurveto{\pgfqpoint{3.029550in}{2.222790in}}{\pgfqpoint{3.040149in}{2.218400in}}{\pgfqpoint{3.051199in}{2.218400in}}%
\pgfpathclose%
\pgfusepath{stroke,fill}%
\end{pgfscope}%
\begin{pgfscope}%
\pgfpathrectangle{\pgfqpoint{0.787074in}{0.548769in}}{\pgfqpoint{5.062926in}{3.102590in}}%
\pgfusepath{clip}%
\pgfsetbuttcap%
\pgfsetroundjoin%
\definecolor{currentfill}{rgb}{1.000000,0.498039,0.054902}%
\pgfsetfillcolor{currentfill}%
\pgfsetlinewidth{1.003750pt}%
\definecolor{currentstroke}{rgb}{1.000000,0.498039,0.054902}%
\pgfsetstrokecolor{currentstroke}%
\pgfsetdash{}{0pt}%
\pgfpathmoveto{\pgfqpoint{1.845955in}{2.443019in}}%
\pgfpathcurveto{\pgfqpoint{1.857005in}{2.443019in}}{\pgfqpoint{1.867604in}{2.447409in}}{\pgfqpoint{1.875418in}{2.455223in}}%
\pgfpathcurveto{\pgfqpoint{1.883232in}{2.463037in}}{\pgfqpoint{1.887622in}{2.473636in}}{\pgfqpoint{1.887622in}{2.484686in}}%
\pgfpathcurveto{\pgfqpoint{1.887622in}{2.495736in}}{\pgfqpoint{1.883232in}{2.506335in}}{\pgfqpoint{1.875418in}{2.514149in}}%
\pgfpathcurveto{\pgfqpoint{1.867604in}{2.521962in}}{\pgfqpoint{1.857005in}{2.526352in}}{\pgfqpoint{1.845955in}{2.526352in}}%
\pgfpathcurveto{\pgfqpoint{1.834905in}{2.526352in}}{\pgfqpoint{1.824306in}{2.521962in}}{\pgfqpoint{1.816493in}{2.514149in}}%
\pgfpathcurveto{\pgfqpoint{1.808679in}{2.506335in}}{\pgfqpoint{1.804289in}{2.495736in}}{\pgfqpoint{1.804289in}{2.484686in}}%
\pgfpathcurveto{\pgfqpoint{1.804289in}{2.473636in}}{\pgfqpoint{1.808679in}{2.463037in}}{\pgfqpoint{1.816493in}{2.455223in}}%
\pgfpathcurveto{\pgfqpoint{1.824306in}{2.447409in}}{\pgfqpoint{1.834905in}{2.443019in}}{\pgfqpoint{1.845955in}{2.443019in}}%
\pgfpathclose%
\pgfusepath{stroke,fill}%
\end{pgfscope}%
\begin{pgfscope}%
\pgfpathrectangle{\pgfqpoint{0.787074in}{0.548769in}}{\pgfqpoint{5.062926in}{3.102590in}}%
\pgfusepath{clip}%
\pgfsetbuttcap%
\pgfsetroundjoin%
\definecolor{currentfill}{rgb}{0.121569,0.466667,0.705882}%
\pgfsetfillcolor{currentfill}%
\pgfsetlinewidth{1.003750pt}%
\definecolor{currentstroke}{rgb}{0.121569,0.466667,0.705882}%
\pgfsetstrokecolor{currentstroke}%
\pgfsetdash{}{0pt}%
\pgfpathmoveto{\pgfqpoint{1.985110in}{1.631630in}}%
\pgfpathcurveto{\pgfqpoint{1.996160in}{1.631630in}}{\pgfqpoint{2.006759in}{1.636020in}}{\pgfqpoint{2.014573in}{1.643834in}}%
\pgfpathcurveto{\pgfqpoint{2.022387in}{1.651647in}}{\pgfqpoint{2.026777in}{1.662246in}}{\pgfqpoint{2.026777in}{1.673296in}}%
\pgfpathcurveto{\pgfqpoint{2.026777in}{1.684346in}}{\pgfqpoint{2.022387in}{1.694945in}}{\pgfqpoint{2.014573in}{1.702759in}}%
\pgfpathcurveto{\pgfqpoint{2.006759in}{1.710573in}}{\pgfqpoint{1.996160in}{1.714963in}}{\pgfqpoint{1.985110in}{1.714963in}}%
\pgfpathcurveto{\pgfqpoint{1.974060in}{1.714963in}}{\pgfqpoint{1.963461in}{1.710573in}}{\pgfqpoint{1.955647in}{1.702759in}}%
\pgfpathcurveto{\pgfqpoint{1.947834in}{1.694945in}}{\pgfqpoint{1.943444in}{1.684346in}}{\pgfqpoint{1.943444in}{1.673296in}}%
\pgfpathcurveto{\pgfqpoint{1.943444in}{1.662246in}}{\pgfqpoint{1.947834in}{1.651647in}}{\pgfqpoint{1.955647in}{1.643834in}}%
\pgfpathcurveto{\pgfqpoint{1.963461in}{1.636020in}}{\pgfqpoint{1.974060in}{1.631630in}}{\pgfqpoint{1.985110in}{1.631630in}}%
\pgfpathclose%
\pgfusepath{stroke,fill}%
\end{pgfscope}%
\begin{pgfscope}%
\pgfpathrectangle{\pgfqpoint{0.787074in}{0.548769in}}{\pgfqpoint{5.062926in}{3.102590in}}%
\pgfusepath{clip}%
\pgfsetbuttcap%
\pgfsetroundjoin%
\definecolor{currentfill}{rgb}{1.000000,0.498039,0.054902}%
\pgfsetfillcolor{currentfill}%
\pgfsetlinewidth{1.003750pt}%
\definecolor{currentstroke}{rgb}{1.000000,0.498039,0.054902}%
\pgfsetstrokecolor{currentstroke}%
\pgfsetdash{}{0pt}%
\pgfpathmoveto{\pgfqpoint{1.692734in}{1.681005in}}%
\pgfpathcurveto{\pgfqpoint{1.703784in}{1.681005in}}{\pgfqpoint{1.714383in}{1.685396in}}{\pgfqpoint{1.722197in}{1.693209in}}%
\pgfpathcurveto{\pgfqpoint{1.730010in}{1.701023in}}{\pgfqpoint{1.734401in}{1.711622in}}{\pgfqpoint{1.734401in}{1.722672in}}%
\pgfpathcurveto{\pgfqpoint{1.734401in}{1.733722in}}{\pgfqpoint{1.730010in}{1.744321in}}{\pgfqpoint{1.722197in}{1.752135in}}%
\pgfpathcurveto{\pgfqpoint{1.714383in}{1.759948in}}{\pgfqpoint{1.703784in}{1.764339in}}{\pgfqpoint{1.692734in}{1.764339in}}%
\pgfpathcurveto{\pgfqpoint{1.681684in}{1.764339in}}{\pgfqpoint{1.671085in}{1.759948in}}{\pgfqpoint{1.663271in}{1.752135in}}%
\pgfpathcurveto{\pgfqpoint{1.655458in}{1.744321in}}{\pgfqpoint{1.651067in}{1.733722in}}{\pgfqpoint{1.651067in}{1.722672in}}%
\pgfpathcurveto{\pgfqpoint{1.651067in}{1.711622in}}{\pgfqpoint{1.655458in}{1.701023in}}{\pgfqpoint{1.663271in}{1.693209in}}%
\pgfpathcurveto{\pgfqpoint{1.671085in}{1.685396in}}{\pgfqpoint{1.681684in}{1.681005in}}{\pgfqpoint{1.692734in}{1.681005in}}%
\pgfpathclose%
\pgfusepath{stroke,fill}%
\end{pgfscope}%
\begin{pgfscope}%
\pgfpathrectangle{\pgfqpoint{0.787074in}{0.548769in}}{\pgfqpoint{5.062926in}{3.102590in}}%
\pgfusepath{clip}%
\pgfsetbuttcap%
\pgfsetroundjoin%
\definecolor{currentfill}{rgb}{0.121569,0.466667,0.705882}%
\pgfsetfillcolor{currentfill}%
\pgfsetlinewidth{1.003750pt}%
\definecolor{currentstroke}{rgb}{0.121569,0.466667,0.705882}%
\pgfsetstrokecolor{currentstroke}%
\pgfsetdash{}{0pt}%
\pgfpathmoveto{\pgfqpoint{1.150842in}{0.648134in}}%
\pgfpathcurveto{\pgfqpoint{1.161892in}{0.648134in}}{\pgfqpoint{1.172491in}{0.652524in}}{\pgfqpoint{1.180305in}{0.660338in}}%
\pgfpathcurveto{\pgfqpoint{1.188118in}{0.668151in}}{\pgfqpoint{1.192509in}{0.678750in}}{\pgfqpoint{1.192509in}{0.689800in}}%
\pgfpathcurveto{\pgfqpoint{1.192509in}{0.700850in}}{\pgfqpoint{1.188118in}{0.711450in}}{\pgfqpoint{1.180305in}{0.719263in}}%
\pgfpathcurveto{\pgfqpoint{1.172491in}{0.727077in}}{\pgfqpoint{1.161892in}{0.731467in}}{\pgfqpoint{1.150842in}{0.731467in}}%
\pgfpathcurveto{\pgfqpoint{1.139792in}{0.731467in}}{\pgfqpoint{1.129193in}{0.727077in}}{\pgfqpoint{1.121379in}{0.719263in}}%
\pgfpathcurveto{\pgfqpoint{1.113566in}{0.711450in}}{\pgfqpoint{1.109175in}{0.700850in}}{\pgfqpoint{1.109175in}{0.689800in}}%
\pgfpathcurveto{\pgfqpoint{1.109175in}{0.678750in}}{\pgfqpoint{1.113566in}{0.668151in}}{\pgfqpoint{1.121379in}{0.660338in}}%
\pgfpathcurveto{\pgfqpoint{1.129193in}{0.652524in}}{\pgfqpoint{1.139792in}{0.648134in}}{\pgfqpoint{1.150842in}{0.648134in}}%
\pgfpathclose%
\pgfusepath{stroke,fill}%
\end{pgfscope}%
\begin{pgfscope}%
\pgfpathrectangle{\pgfqpoint{0.787074in}{0.548769in}}{\pgfqpoint{5.062926in}{3.102590in}}%
\pgfusepath{clip}%
\pgfsetbuttcap%
\pgfsetroundjoin%
\definecolor{currentfill}{rgb}{1.000000,0.498039,0.054902}%
\pgfsetfillcolor{currentfill}%
\pgfsetlinewidth{1.003750pt}%
\definecolor{currentstroke}{rgb}{1.000000,0.498039,0.054902}%
\pgfsetstrokecolor{currentstroke}%
\pgfsetdash{}{0pt}%
\pgfpathmoveto{\pgfqpoint{1.883042in}{2.734356in}}%
\pgfpathcurveto{\pgfqpoint{1.894092in}{2.734356in}}{\pgfqpoint{1.904691in}{2.738746in}}{\pgfqpoint{1.912505in}{2.746560in}}%
\pgfpathcurveto{\pgfqpoint{1.920318in}{2.754373in}}{\pgfqpoint{1.924708in}{2.764972in}}{\pgfqpoint{1.924708in}{2.776022in}}%
\pgfpathcurveto{\pgfqpoint{1.924708in}{2.787073in}}{\pgfqpoint{1.920318in}{2.797672in}}{\pgfqpoint{1.912505in}{2.805485in}}%
\pgfpathcurveto{\pgfqpoint{1.904691in}{2.813299in}}{\pgfqpoint{1.894092in}{2.817689in}}{\pgfqpoint{1.883042in}{2.817689in}}%
\pgfpathcurveto{\pgfqpoint{1.871992in}{2.817689in}}{\pgfqpoint{1.861393in}{2.813299in}}{\pgfqpoint{1.853579in}{2.805485in}}%
\pgfpathcurveto{\pgfqpoint{1.845765in}{2.797672in}}{\pgfqpoint{1.841375in}{2.787073in}}{\pgfqpoint{1.841375in}{2.776022in}}%
\pgfpathcurveto{\pgfqpoint{1.841375in}{2.764972in}}{\pgfqpoint{1.845765in}{2.754373in}}{\pgfqpoint{1.853579in}{2.746560in}}%
\pgfpathcurveto{\pgfqpoint{1.861393in}{2.738746in}}{\pgfqpoint{1.871992in}{2.734356in}}{\pgfqpoint{1.883042in}{2.734356in}}%
\pgfpathclose%
\pgfusepath{stroke,fill}%
\end{pgfscope}%
\begin{pgfscope}%
\pgfpathrectangle{\pgfqpoint{0.787074in}{0.548769in}}{\pgfqpoint{5.062926in}{3.102590in}}%
\pgfusepath{clip}%
\pgfsetbuttcap%
\pgfsetroundjoin%
\definecolor{currentfill}{rgb}{1.000000,0.498039,0.054902}%
\pgfsetfillcolor{currentfill}%
\pgfsetlinewidth{1.003750pt}%
\definecolor{currentstroke}{rgb}{1.000000,0.498039,0.054902}%
\pgfsetstrokecolor{currentstroke}%
\pgfsetdash{}{0pt}%
\pgfpathmoveto{\pgfqpoint{1.873130in}{2.144210in}}%
\pgfpathcurveto{\pgfqpoint{1.884180in}{2.144210in}}{\pgfqpoint{1.894779in}{2.148600in}}{\pgfqpoint{1.902593in}{2.156414in}}%
\pgfpathcurveto{\pgfqpoint{1.910406in}{2.164228in}}{\pgfqpoint{1.914797in}{2.174827in}}{\pgfqpoint{1.914797in}{2.185877in}}%
\pgfpathcurveto{\pgfqpoint{1.914797in}{2.196927in}}{\pgfqpoint{1.910406in}{2.207526in}}{\pgfqpoint{1.902593in}{2.215340in}}%
\pgfpathcurveto{\pgfqpoint{1.894779in}{2.223153in}}{\pgfqpoint{1.884180in}{2.227544in}}{\pgfqpoint{1.873130in}{2.227544in}}%
\pgfpathcurveto{\pgfqpoint{1.862080in}{2.227544in}}{\pgfqpoint{1.851481in}{2.223153in}}{\pgfqpoint{1.843667in}{2.215340in}}%
\pgfpathcurveto{\pgfqpoint{1.835854in}{2.207526in}}{\pgfqpoint{1.831463in}{2.196927in}}{\pgfqpoint{1.831463in}{2.185877in}}%
\pgfpathcurveto{\pgfqpoint{1.831463in}{2.174827in}}{\pgfqpoint{1.835854in}{2.164228in}}{\pgfqpoint{1.843667in}{2.156414in}}%
\pgfpathcurveto{\pgfqpoint{1.851481in}{2.148600in}}{\pgfqpoint{1.862080in}{2.144210in}}{\pgfqpoint{1.873130in}{2.144210in}}%
\pgfpathclose%
\pgfusepath{stroke,fill}%
\end{pgfscope}%
\begin{pgfscope}%
\pgfpathrectangle{\pgfqpoint{0.787074in}{0.548769in}}{\pgfqpoint{5.062926in}{3.102590in}}%
\pgfusepath{clip}%
\pgfsetbuttcap%
\pgfsetroundjoin%
\definecolor{currentfill}{rgb}{1.000000,0.498039,0.054902}%
\pgfsetfillcolor{currentfill}%
\pgfsetlinewidth{1.003750pt}%
\definecolor{currentstroke}{rgb}{1.000000,0.498039,0.054902}%
\pgfsetstrokecolor{currentstroke}%
\pgfsetdash{}{0pt}%
\pgfpathmoveto{\pgfqpoint{1.789189in}{2.799180in}}%
\pgfpathcurveto{\pgfqpoint{1.800240in}{2.799180in}}{\pgfqpoint{1.810839in}{2.803570in}}{\pgfqpoint{1.818652in}{2.811384in}}%
\pgfpathcurveto{\pgfqpoint{1.826466in}{2.819197in}}{\pgfqpoint{1.830856in}{2.829796in}}{\pgfqpoint{1.830856in}{2.840846in}}%
\pgfpathcurveto{\pgfqpoint{1.830856in}{2.851897in}}{\pgfqpoint{1.826466in}{2.862496in}}{\pgfqpoint{1.818652in}{2.870309in}}%
\pgfpathcurveto{\pgfqpoint{1.810839in}{2.878123in}}{\pgfqpoint{1.800240in}{2.882513in}}{\pgfqpoint{1.789189in}{2.882513in}}%
\pgfpathcurveto{\pgfqpoint{1.778139in}{2.882513in}}{\pgfqpoint{1.767540in}{2.878123in}}{\pgfqpoint{1.759727in}{2.870309in}}%
\pgfpathcurveto{\pgfqpoint{1.751913in}{2.862496in}}{\pgfqpoint{1.747523in}{2.851897in}}{\pgfqpoint{1.747523in}{2.840846in}}%
\pgfpathcurveto{\pgfqpoint{1.747523in}{2.829796in}}{\pgfqpoint{1.751913in}{2.819197in}}{\pgfqpoint{1.759727in}{2.811384in}}%
\pgfpathcurveto{\pgfqpoint{1.767540in}{2.803570in}}{\pgfqpoint{1.778139in}{2.799180in}}{\pgfqpoint{1.789189in}{2.799180in}}%
\pgfpathclose%
\pgfusepath{stroke,fill}%
\end{pgfscope}%
\begin{pgfscope}%
\pgfpathrectangle{\pgfqpoint{0.787074in}{0.548769in}}{\pgfqpoint{5.062926in}{3.102590in}}%
\pgfusepath{clip}%
\pgfsetbuttcap%
\pgfsetroundjoin%
\definecolor{currentfill}{rgb}{1.000000,0.498039,0.054902}%
\pgfsetfillcolor{currentfill}%
\pgfsetlinewidth{1.003750pt}%
\definecolor{currentstroke}{rgb}{1.000000,0.498039,0.054902}%
\pgfsetstrokecolor{currentstroke}%
\pgfsetdash{}{0pt}%
\pgfpathmoveto{\pgfqpoint{1.705469in}{2.526596in}}%
\pgfpathcurveto{\pgfqpoint{1.716519in}{2.526596in}}{\pgfqpoint{1.727118in}{2.530987in}}{\pgfqpoint{1.734932in}{2.538800in}}%
\pgfpathcurveto{\pgfqpoint{1.742746in}{2.546614in}}{\pgfqpoint{1.747136in}{2.557213in}}{\pgfqpoint{1.747136in}{2.568263in}}%
\pgfpathcurveto{\pgfqpoint{1.747136in}{2.579313in}}{\pgfqpoint{1.742746in}{2.589912in}}{\pgfqpoint{1.734932in}{2.597726in}}%
\pgfpathcurveto{\pgfqpoint{1.727118in}{2.605540in}}{\pgfqpoint{1.716519in}{2.609930in}}{\pgfqpoint{1.705469in}{2.609930in}}%
\pgfpathcurveto{\pgfqpoint{1.694419in}{2.609930in}}{\pgfqpoint{1.683820in}{2.605540in}}{\pgfqpoint{1.676006in}{2.597726in}}%
\pgfpathcurveto{\pgfqpoint{1.668193in}{2.589912in}}{\pgfqpoint{1.663803in}{2.579313in}}{\pgfqpoint{1.663803in}{2.568263in}}%
\pgfpathcurveto{\pgfqpoint{1.663803in}{2.557213in}}{\pgfqpoint{1.668193in}{2.546614in}}{\pgfqpoint{1.676006in}{2.538800in}}%
\pgfpathcurveto{\pgfqpoint{1.683820in}{2.530987in}}{\pgfqpoint{1.694419in}{2.526596in}}{\pgfqpoint{1.705469in}{2.526596in}}%
\pgfpathclose%
\pgfusepath{stroke,fill}%
\end{pgfscope}%
\begin{pgfscope}%
\pgfpathrectangle{\pgfqpoint{0.787074in}{0.548769in}}{\pgfqpoint{5.062926in}{3.102590in}}%
\pgfusepath{clip}%
\pgfsetbuttcap%
\pgfsetroundjoin%
\definecolor{currentfill}{rgb}{1.000000,0.498039,0.054902}%
\pgfsetfillcolor{currentfill}%
\pgfsetlinewidth{1.003750pt}%
\definecolor{currentstroke}{rgb}{1.000000,0.498039,0.054902}%
\pgfsetstrokecolor{currentstroke}%
\pgfsetdash{}{0pt}%
\pgfpathmoveto{\pgfqpoint{1.435876in}{2.883497in}}%
\pgfpathcurveto{\pgfqpoint{1.446926in}{2.883497in}}{\pgfqpoint{1.457525in}{2.887888in}}{\pgfqpoint{1.465338in}{2.895701in}}%
\pgfpathcurveto{\pgfqpoint{1.473152in}{2.903515in}}{\pgfqpoint{1.477542in}{2.914114in}}{\pgfqpoint{1.477542in}{2.925164in}}%
\pgfpathcurveto{\pgfqpoint{1.477542in}{2.936214in}}{\pgfqpoint{1.473152in}{2.946813in}}{\pgfqpoint{1.465338in}{2.954627in}}%
\pgfpathcurveto{\pgfqpoint{1.457525in}{2.962440in}}{\pgfqpoint{1.446926in}{2.966831in}}{\pgfqpoint{1.435876in}{2.966831in}}%
\pgfpathcurveto{\pgfqpoint{1.424825in}{2.966831in}}{\pgfqpoint{1.414226in}{2.962440in}}{\pgfqpoint{1.406413in}{2.954627in}}%
\pgfpathcurveto{\pgfqpoint{1.398599in}{2.946813in}}{\pgfqpoint{1.394209in}{2.936214in}}{\pgfqpoint{1.394209in}{2.925164in}}%
\pgfpathcurveto{\pgfqpoint{1.394209in}{2.914114in}}{\pgfqpoint{1.398599in}{2.903515in}}{\pgfqpoint{1.406413in}{2.895701in}}%
\pgfpathcurveto{\pgfqpoint{1.414226in}{2.887888in}}{\pgfqpoint{1.424825in}{2.883497in}}{\pgfqpoint{1.435876in}{2.883497in}}%
\pgfpathclose%
\pgfusepath{stroke,fill}%
\end{pgfscope}%
\begin{pgfscope}%
\pgfpathrectangle{\pgfqpoint{0.787074in}{0.548769in}}{\pgfqpoint{5.062926in}{3.102590in}}%
\pgfusepath{clip}%
\pgfsetbuttcap%
\pgfsetroundjoin%
\definecolor{currentfill}{rgb}{1.000000,0.498039,0.054902}%
\pgfsetfillcolor{currentfill}%
\pgfsetlinewidth{1.003750pt}%
\definecolor{currentstroke}{rgb}{1.000000,0.498039,0.054902}%
\pgfsetstrokecolor{currentstroke}%
\pgfsetdash{}{0pt}%
\pgfpathmoveto{\pgfqpoint{2.198709in}{2.042862in}}%
\pgfpathcurveto{\pgfqpoint{2.209760in}{2.042862in}}{\pgfqpoint{2.220359in}{2.047252in}}{\pgfqpoint{2.228172in}{2.055065in}}%
\pgfpathcurveto{\pgfqpoint{2.235986in}{2.062879in}}{\pgfqpoint{2.240376in}{2.073478in}}{\pgfqpoint{2.240376in}{2.084528in}}%
\pgfpathcurveto{\pgfqpoint{2.240376in}{2.095578in}}{\pgfqpoint{2.235986in}{2.106177in}}{\pgfqpoint{2.228172in}{2.113991in}}%
\pgfpathcurveto{\pgfqpoint{2.220359in}{2.121805in}}{\pgfqpoint{2.209760in}{2.126195in}}{\pgfqpoint{2.198709in}{2.126195in}}%
\pgfpathcurveto{\pgfqpoint{2.187659in}{2.126195in}}{\pgfqpoint{2.177060in}{2.121805in}}{\pgfqpoint{2.169247in}{2.113991in}}%
\pgfpathcurveto{\pgfqpoint{2.161433in}{2.106177in}}{\pgfqpoint{2.157043in}{2.095578in}}{\pgfqpoint{2.157043in}{2.084528in}}%
\pgfpathcurveto{\pgfqpoint{2.157043in}{2.073478in}}{\pgfqpoint{2.161433in}{2.062879in}}{\pgfqpoint{2.169247in}{2.055065in}}%
\pgfpathcurveto{\pgfqpoint{2.177060in}{2.047252in}}{\pgfqpoint{2.187659in}{2.042862in}}{\pgfqpoint{2.198709in}{2.042862in}}%
\pgfpathclose%
\pgfusepath{stroke,fill}%
\end{pgfscope}%
\begin{pgfscope}%
\pgfpathrectangle{\pgfqpoint{0.787074in}{0.548769in}}{\pgfqpoint{5.062926in}{3.102590in}}%
\pgfusepath{clip}%
\pgfsetbuttcap%
\pgfsetroundjoin%
\definecolor{currentfill}{rgb}{1.000000,0.498039,0.054902}%
\pgfsetfillcolor{currentfill}%
\pgfsetlinewidth{1.003750pt}%
\definecolor{currentstroke}{rgb}{1.000000,0.498039,0.054902}%
\pgfsetstrokecolor{currentstroke}%
\pgfsetdash{}{0pt}%
\pgfpathmoveto{\pgfqpoint{1.937620in}{2.760534in}}%
\pgfpathcurveto{\pgfqpoint{1.948670in}{2.760534in}}{\pgfqpoint{1.959269in}{2.764924in}}{\pgfqpoint{1.967083in}{2.772738in}}%
\pgfpathcurveto{\pgfqpoint{1.974897in}{2.780552in}}{\pgfqpoint{1.979287in}{2.791151in}}{\pgfqpoint{1.979287in}{2.802201in}}%
\pgfpathcurveto{\pgfqpoint{1.979287in}{2.813251in}}{\pgfqpoint{1.974897in}{2.823850in}}{\pgfqpoint{1.967083in}{2.831664in}}%
\pgfpathcurveto{\pgfqpoint{1.959269in}{2.839477in}}{\pgfqpoint{1.948670in}{2.843868in}}{\pgfqpoint{1.937620in}{2.843868in}}%
\pgfpathcurveto{\pgfqpoint{1.926570in}{2.843868in}}{\pgfqpoint{1.915971in}{2.839477in}}{\pgfqpoint{1.908157in}{2.831664in}}%
\pgfpathcurveto{\pgfqpoint{1.900344in}{2.823850in}}{\pgfqpoint{1.895954in}{2.813251in}}{\pgfqpoint{1.895954in}{2.802201in}}%
\pgfpathcurveto{\pgfqpoint{1.895954in}{2.791151in}}{\pgfqpoint{1.900344in}{2.780552in}}{\pgfqpoint{1.908157in}{2.772738in}}%
\pgfpathcurveto{\pgfqpoint{1.915971in}{2.764924in}}{\pgfqpoint{1.926570in}{2.760534in}}{\pgfqpoint{1.937620in}{2.760534in}}%
\pgfpathclose%
\pgfusepath{stroke,fill}%
\end{pgfscope}%
\begin{pgfscope}%
\pgfpathrectangle{\pgfqpoint{0.787074in}{0.548769in}}{\pgfqpoint{5.062926in}{3.102590in}}%
\pgfusepath{clip}%
\pgfsetbuttcap%
\pgfsetroundjoin%
\definecolor{currentfill}{rgb}{0.121569,0.466667,0.705882}%
\pgfsetfillcolor{currentfill}%
\pgfsetlinewidth{1.003750pt}%
\definecolor{currentstroke}{rgb}{0.121569,0.466667,0.705882}%
\pgfsetstrokecolor{currentstroke}%
\pgfsetdash{}{0pt}%
\pgfpathmoveto{\pgfqpoint{1.099231in}{0.648129in}}%
\pgfpathcurveto{\pgfqpoint{1.110281in}{0.648129in}}{\pgfqpoint{1.120880in}{0.652519in}}{\pgfqpoint{1.128694in}{0.660333in}}%
\pgfpathcurveto{\pgfqpoint{1.136508in}{0.668146in}}{\pgfqpoint{1.140898in}{0.678745in}}{\pgfqpoint{1.140898in}{0.689796in}}%
\pgfpathcurveto{\pgfqpoint{1.140898in}{0.700846in}}{\pgfqpoint{1.136508in}{0.711445in}}{\pgfqpoint{1.128694in}{0.719258in}}%
\pgfpathcurveto{\pgfqpoint{1.120880in}{0.727072in}}{\pgfqpoint{1.110281in}{0.731462in}}{\pgfqpoint{1.099231in}{0.731462in}}%
\pgfpathcurveto{\pgfqpoint{1.088181in}{0.731462in}}{\pgfqpoint{1.077582in}{0.727072in}}{\pgfqpoint{1.069768in}{0.719258in}}%
\pgfpathcurveto{\pgfqpoint{1.061955in}{0.711445in}}{\pgfqpoint{1.057565in}{0.700846in}}{\pgfqpoint{1.057565in}{0.689796in}}%
\pgfpathcurveto{\pgfqpoint{1.057565in}{0.678745in}}{\pgfqpoint{1.061955in}{0.668146in}}{\pgfqpoint{1.069768in}{0.660333in}}%
\pgfpathcurveto{\pgfqpoint{1.077582in}{0.652519in}}{\pgfqpoint{1.088181in}{0.648129in}}{\pgfqpoint{1.099231in}{0.648129in}}%
\pgfpathclose%
\pgfusepath{stroke,fill}%
\end{pgfscope}%
\begin{pgfscope}%
\pgfpathrectangle{\pgfqpoint{0.787074in}{0.548769in}}{\pgfqpoint{5.062926in}{3.102590in}}%
\pgfusepath{clip}%
\pgfsetbuttcap%
\pgfsetroundjoin%
\definecolor{currentfill}{rgb}{0.121569,0.466667,0.705882}%
\pgfsetfillcolor{currentfill}%
\pgfsetlinewidth{1.003750pt}%
\definecolor{currentstroke}{rgb}{0.121569,0.466667,0.705882}%
\pgfsetstrokecolor{currentstroke}%
\pgfsetdash{}{0pt}%
\pgfpathmoveto{\pgfqpoint{1.017207in}{0.658321in}}%
\pgfpathcurveto{\pgfqpoint{1.028257in}{0.658321in}}{\pgfqpoint{1.038856in}{0.662711in}}{\pgfqpoint{1.046670in}{0.670525in}}%
\pgfpathcurveto{\pgfqpoint{1.054483in}{0.678338in}}{\pgfqpoint{1.058874in}{0.688937in}}{\pgfqpoint{1.058874in}{0.699987in}}%
\pgfpathcurveto{\pgfqpoint{1.058874in}{0.711038in}}{\pgfqpoint{1.054483in}{0.721637in}}{\pgfqpoint{1.046670in}{0.729450in}}%
\pgfpathcurveto{\pgfqpoint{1.038856in}{0.737264in}}{\pgfqpoint{1.028257in}{0.741654in}}{\pgfqpoint{1.017207in}{0.741654in}}%
\pgfpathcurveto{\pgfqpoint{1.006157in}{0.741654in}}{\pgfqpoint{0.995558in}{0.737264in}}{\pgfqpoint{0.987744in}{0.729450in}}%
\pgfpathcurveto{\pgfqpoint{0.979930in}{0.721637in}}{\pgfqpoint{0.975540in}{0.711038in}}{\pgfqpoint{0.975540in}{0.699987in}}%
\pgfpathcurveto{\pgfqpoint{0.975540in}{0.688937in}}{\pgfqpoint{0.979930in}{0.678338in}}{\pgfqpoint{0.987744in}{0.670525in}}%
\pgfpathcurveto{\pgfqpoint{0.995558in}{0.662711in}}{\pgfqpoint{1.006157in}{0.658321in}}{\pgfqpoint{1.017207in}{0.658321in}}%
\pgfpathclose%
\pgfusepath{stroke,fill}%
\end{pgfscope}%
\begin{pgfscope}%
\pgfpathrectangle{\pgfqpoint{0.787074in}{0.548769in}}{\pgfqpoint{5.062926in}{3.102590in}}%
\pgfusepath{clip}%
\pgfsetbuttcap%
\pgfsetroundjoin%
\definecolor{currentfill}{rgb}{0.121569,0.466667,0.705882}%
\pgfsetfillcolor{currentfill}%
\pgfsetlinewidth{1.003750pt}%
\definecolor{currentstroke}{rgb}{0.121569,0.466667,0.705882}%
\pgfsetstrokecolor{currentstroke}%
\pgfsetdash{}{0pt}%
\pgfpathmoveto{\pgfqpoint{1.851297in}{1.208140in}}%
\pgfpathcurveto{\pgfqpoint{1.862347in}{1.208140in}}{\pgfqpoint{1.872946in}{1.212530in}}{\pgfqpoint{1.880760in}{1.220344in}}%
\pgfpathcurveto{\pgfqpoint{1.888573in}{1.228157in}}{\pgfqpoint{1.892964in}{1.238756in}}{\pgfqpoint{1.892964in}{1.249807in}}%
\pgfpathcurveto{\pgfqpoint{1.892964in}{1.260857in}}{\pgfqpoint{1.888573in}{1.271456in}}{\pgfqpoint{1.880760in}{1.279269in}}%
\pgfpathcurveto{\pgfqpoint{1.872946in}{1.287083in}}{\pgfqpoint{1.862347in}{1.291473in}}{\pgfqpoint{1.851297in}{1.291473in}}%
\pgfpathcurveto{\pgfqpoint{1.840247in}{1.291473in}}{\pgfqpoint{1.829648in}{1.287083in}}{\pgfqpoint{1.821834in}{1.279269in}}%
\pgfpathcurveto{\pgfqpoint{1.814021in}{1.271456in}}{\pgfqpoint{1.809630in}{1.260857in}}{\pgfqpoint{1.809630in}{1.249807in}}%
\pgfpathcurveto{\pgfqpoint{1.809630in}{1.238756in}}{\pgfqpoint{1.814021in}{1.228157in}}{\pgfqpoint{1.821834in}{1.220344in}}%
\pgfpathcurveto{\pgfqpoint{1.829648in}{1.212530in}}{\pgfqpoint{1.840247in}{1.208140in}}{\pgfqpoint{1.851297in}{1.208140in}}%
\pgfpathclose%
\pgfusepath{stroke,fill}%
\end{pgfscope}%
\begin{pgfscope}%
\pgfpathrectangle{\pgfqpoint{0.787074in}{0.548769in}}{\pgfqpoint{5.062926in}{3.102590in}}%
\pgfusepath{clip}%
\pgfsetbuttcap%
\pgfsetroundjoin%
\definecolor{currentfill}{rgb}{1.000000,0.498039,0.054902}%
\pgfsetfillcolor{currentfill}%
\pgfsetlinewidth{1.003750pt}%
\definecolor{currentstroke}{rgb}{1.000000,0.498039,0.054902}%
\pgfsetstrokecolor{currentstroke}%
\pgfsetdash{}{0pt}%
\pgfpathmoveto{\pgfqpoint{1.814609in}{2.220680in}}%
\pgfpathcurveto{\pgfqpoint{1.825659in}{2.220680in}}{\pgfqpoint{1.836258in}{2.225070in}}{\pgfqpoint{1.844072in}{2.232884in}}%
\pgfpathcurveto{\pgfqpoint{1.851885in}{2.240698in}}{\pgfqpoint{1.856276in}{2.251297in}}{\pgfqpoint{1.856276in}{2.262347in}}%
\pgfpathcurveto{\pgfqpoint{1.856276in}{2.273397in}}{\pgfqpoint{1.851885in}{2.283996in}}{\pgfqpoint{1.844072in}{2.291810in}}%
\pgfpathcurveto{\pgfqpoint{1.836258in}{2.299623in}}{\pgfqpoint{1.825659in}{2.304013in}}{\pgfqpoint{1.814609in}{2.304013in}}%
\pgfpathcurveto{\pgfqpoint{1.803559in}{2.304013in}}{\pgfqpoint{1.792960in}{2.299623in}}{\pgfqpoint{1.785146in}{2.291810in}}%
\pgfpathcurveto{\pgfqpoint{1.777333in}{2.283996in}}{\pgfqpoint{1.772942in}{2.273397in}}{\pgfqpoint{1.772942in}{2.262347in}}%
\pgfpathcurveto{\pgfqpoint{1.772942in}{2.251297in}}{\pgfqpoint{1.777333in}{2.240698in}}{\pgfqpoint{1.785146in}{2.232884in}}%
\pgfpathcurveto{\pgfqpoint{1.792960in}{2.225070in}}{\pgfqpoint{1.803559in}{2.220680in}}{\pgfqpoint{1.814609in}{2.220680in}}%
\pgfpathclose%
\pgfusepath{stroke,fill}%
\end{pgfscope}%
\begin{pgfscope}%
\pgfpathrectangle{\pgfqpoint{0.787074in}{0.548769in}}{\pgfqpoint{5.062926in}{3.102590in}}%
\pgfusepath{clip}%
\pgfsetbuttcap%
\pgfsetroundjoin%
\definecolor{currentfill}{rgb}{0.121569,0.466667,0.705882}%
\pgfsetfillcolor{currentfill}%
\pgfsetlinewidth{1.003750pt}%
\definecolor{currentstroke}{rgb}{0.121569,0.466667,0.705882}%
\pgfsetstrokecolor{currentstroke}%
\pgfsetdash{}{0pt}%
\pgfpathmoveto{\pgfqpoint{1.345533in}{0.648130in}}%
\pgfpathcurveto{\pgfqpoint{1.356584in}{0.648130in}}{\pgfqpoint{1.367183in}{0.652520in}}{\pgfqpoint{1.374996in}{0.660334in}}%
\pgfpathcurveto{\pgfqpoint{1.382810in}{0.668148in}}{\pgfqpoint{1.387200in}{0.678747in}}{\pgfqpoint{1.387200in}{0.689797in}}%
\pgfpathcurveto{\pgfqpoint{1.387200in}{0.700847in}}{\pgfqpoint{1.382810in}{0.711446in}}{\pgfqpoint{1.374996in}{0.719260in}}%
\pgfpathcurveto{\pgfqpoint{1.367183in}{0.727073in}}{\pgfqpoint{1.356584in}{0.731464in}}{\pgfqpoint{1.345533in}{0.731464in}}%
\pgfpathcurveto{\pgfqpoint{1.334483in}{0.731464in}}{\pgfqpoint{1.323884in}{0.727073in}}{\pgfqpoint{1.316071in}{0.719260in}}%
\pgfpathcurveto{\pgfqpoint{1.308257in}{0.711446in}}{\pgfqpoint{1.303867in}{0.700847in}}{\pgfqpoint{1.303867in}{0.689797in}}%
\pgfpathcurveto{\pgfqpoint{1.303867in}{0.678747in}}{\pgfqpoint{1.308257in}{0.668148in}}{\pgfqpoint{1.316071in}{0.660334in}}%
\pgfpathcurveto{\pgfqpoint{1.323884in}{0.652520in}}{\pgfqpoint{1.334483in}{0.648130in}}{\pgfqpoint{1.345533in}{0.648130in}}%
\pgfpathclose%
\pgfusepath{stroke,fill}%
\end{pgfscope}%
\begin{pgfscope}%
\pgfpathrectangle{\pgfqpoint{0.787074in}{0.548769in}}{\pgfqpoint{5.062926in}{3.102590in}}%
\pgfusepath{clip}%
\pgfsetbuttcap%
\pgfsetroundjoin%
\definecolor{currentfill}{rgb}{0.121569,0.466667,0.705882}%
\pgfsetfillcolor{currentfill}%
\pgfsetlinewidth{1.003750pt}%
\definecolor{currentstroke}{rgb}{0.121569,0.466667,0.705882}%
\pgfsetstrokecolor{currentstroke}%
\pgfsetdash{}{0pt}%
\pgfpathmoveto{\pgfqpoint{1.030078in}{0.648134in}}%
\pgfpathcurveto{\pgfqpoint{1.041128in}{0.648134in}}{\pgfqpoint{1.051727in}{0.652524in}}{\pgfqpoint{1.059541in}{0.660338in}}%
\pgfpathcurveto{\pgfqpoint{1.067354in}{0.668151in}}{\pgfqpoint{1.071744in}{0.678750in}}{\pgfqpoint{1.071744in}{0.689800in}}%
\pgfpathcurveto{\pgfqpoint{1.071744in}{0.700850in}}{\pgfqpoint{1.067354in}{0.711450in}}{\pgfqpoint{1.059541in}{0.719263in}}%
\pgfpathcurveto{\pgfqpoint{1.051727in}{0.727077in}}{\pgfqpoint{1.041128in}{0.731467in}}{\pgfqpoint{1.030078in}{0.731467in}}%
\pgfpathcurveto{\pgfqpoint{1.019028in}{0.731467in}}{\pgfqpoint{1.008429in}{0.727077in}}{\pgfqpoint{1.000615in}{0.719263in}}%
\pgfpathcurveto{\pgfqpoint{0.992801in}{0.711450in}}{\pgfqpoint{0.988411in}{0.700850in}}{\pgfqpoint{0.988411in}{0.689800in}}%
\pgfpathcurveto{\pgfqpoint{0.988411in}{0.678750in}}{\pgfqpoint{0.992801in}{0.668151in}}{\pgfqpoint{1.000615in}{0.660338in}}%
\pgfpathcurveto{\pgfqpoint{1.008429in}{0.652524in}}{\pgfqpoint{1.019028in}{0.648134in}}{\pgfqpoint{1.030078in}{0.648134in}}%
\pgfpathclose%
\pgfusepath{stroke,fill}%
\end{pgfscope}%
\begin{pgfscope}%
\pgfpathrectangle{\pgfqpoint{0.787074in}{0.548769in}}{\pgfqpoint{5.062926in}{3.102590in}}%
\pgfusepath{clip}%
\pgfsetbuttcap%
\pgfsetroundjoin%
\definecolor{currentfill}{rgb}{1.000000,0.498039,0.054902}%
\pgfsetfillcolor{currentfill}%
\pgfsetlinewidth{1.003750pt}%
\definecolor{currentstroke}{rgb}{1.000000,0.498039,0.054902}%
\pgfsetstrokecolor{currentstroke}%
\pgfsetdash{}{0pt}%
\pgfpathmoveto{\pgfqpoint{1.339318in}{3.468665in}}%
\pgfpathcurveto{\pgfqpoint{1.350369in}{3.468665in}}{\pgfqpoint{1.360968in}{3.473055in}}{\pgfqpoint{1.368781in}{3.480869in}}%
\pgfpathcurveto{\pgfqpoint{1.376595in}{3.488683in}}{\pgfqpoint{1.380985in}{3.499282in}}{\pgfqpoint{1.380985in}{3.510332in}}%
\pgfpathcurveto{\pgfqpoint{1.380985in}{3.521382in}}{\pgfqpoint{1.376595in}{3.531981in}}{\pgfqpoint{1.368781in}{3.539795in}}%
\pgfpathcurveto{\pgfqpoint{1.360968in}{3.547608in}}{\pgfqpoint{1.350369in}{3.551998in}}{\pgfqpoint{1.339318in}{3.551998in}}%
\pgfpathcurveto{\pgfqpoint{1.328268in}{3.551998in}}{\pgfqpoint{1.317669in}{3.547608in}}{\pgfqpoint{1.309856in}{3.539795in}}%
\pgfpathcurveto{\pgfqpoint{1.302042in}{3.531981in}}{\pgfqpoint{1.297652in}{3.521382in}}{\pgfqpoint{1.297652in}{3.510332in}}%
\pgfpathcurveto{\pgfqpoint{1.297652in}{3.499282in}}{\pgfqpoint{1.302042in}{3.488683in}}{\pgfqpoint{1.309856in}{3.480869in}}%
\pgfpathcurveto{\pgfqpoint{1.317669in}{3.473055in}}{\pgfqpoint{1.328268in}{3.468665in}}{\pgfqpoint{1.339318in}{3.468665in}}%
\pgfpathclose%
\pgfusepath{stroke,fill}%
\end{pgfscope}%
\begin{pgfscope}%
\pgfpathrectangle{\pgfqpoint{0.787074in}{0.548769in}}{\pgfqpoint{5.062926in}{3.102590in}}%
\pgfusepath{clip}%
\pgfsetbuttcap%
\pgfsetroundjoin%
\definecolor{currentfill}{rgb}{0.121569,0.466667,0.705882}%
\pgfsetfillcolor{currentfill}%
\pgfsetlinewidth{1.003750pt}%
\definecolor{currentstroke}{rgb}{0.121569,0.466667,0.705882}%
\pgfsetstrokecolor{currentstroke}%
\pgfsetdash{}{0pt}%
\pgfpathmoveto{\pgfqpoint{1.292778in}{2.370534in}}%
\pgfpathcurveto{\pgfqpoint{1.303828in}{2.370534in}}{\pgfqpoint{1.314427in}{2.374924in}}{\pgfqpoint{1.322241in}{2.382738in}}%
\pgfpathcurveto{\pgfqpoint{1.330054in}{2.390551in}}{\pgfqpoint{1.334445in}{2.401150in}}{\pgfqpoint{1.334445in}{2.412201in}}%
\pgfpathcurveto{\pgfqpoint{1.334445in}{2.423251in}}{\pgfqpoint{1.330054in}{2.433850in}}{\pgfqpoint{1.322241in}{2.441663in}}%
\pgfpathcurveto{\pgfqpoint{1.314427in}{2.449477in}}{\pgfqpoint{1.303828in}{2.453867in}}{\pgfqpoint{1.292778in}{2.453867in}}%
\pgfpathcurveto{\pgfqpoint{1.281728in}{2.453867in}}{\pgfqpoint{1.271129in}{2.449477in}}{\pgfqpoint{1.263315in}{2.441663in}}%
\pgfpathcurveto{\pgfqpoint{1.255502in}{2.433850in}}{\pgfqpoint{1.251111in}{2.423251in}}{\pgfqpoint{1.251111in}{2.412201in}}%
\pgfpathcurveto{\pgfqpoint{1.251111in}{2.401150in}}{\pgfqpoint{1.255502in}{2.390551in}}{\pgfqpoint{1.263315in}{2.382738in}}%
\pgfpathcurveto{\pgfqpoint{1.271129in}{2.374924in}}{\pgfqpoint{1.281728in}{2.370534in}}{\pgfqpoint{1.292778in}{2.370534in}}%
\pgfpathclose%
\pgfusepath{stroke,fill}%
\end{pgfscope}%
\begin{pgfscope}%
\pgfpathrectangle{\pgfqpoint{0.787074in}{0.548769in}}{\pgfqpoint{5.062926in}{3.102590in}}%
\pgfusepath{clip}%
\pgfsetbuttcap%
\pgfsetroundjoin%
\definecolor{currentfill}{rgb}{1.000000,0.498039,0.054902}%
\pgfsetfillcolor{currentfill}%
\pgfsetlinewidth{1.003750pt}%
\definecolor{currentstroke}{rgb}{1.000000,0.498039,0.054902}%
\pgfsetstrokecolor{currentstroke}%
\pgfsetdash{}{0pt}%
\pgfpathmoveto{\pgfqpoint{2.309799in}{2.417203in}}%
\pgfpathcurveto{\pgfqpoint{2.320850in}{2.417203in}}{\pgfqpoint{2.331449in}{2.421593in}}{\pgfqpoint{2.339262in}{2.429407in}}%
\pgfpathcurveto{\pgfqpoint{2.347076in}{2.437220in}}{\pgfqpoint{2.351466in}{2.447819in}}{\pgfqpoint{2.351466in}{2.458870in}}%
\pgfpathcurveto{\pgfqpoint{2.351466in}{2.469920in}}{\pgfqpoint{2.347076in}{2.480519in}}{\pgfqpoint{2.339262in}{2.488332in}}%
\pgfpathcurveto{\pgfqpoint{2.331449in}{2.496146in}}{\pgfqpoint{2.320850in}{2.500536in}}{\pgfqpoint{2.309799in}{2.500536in}}%
\pgfpathcurveto{\pgfqpoint{2.298749in}{2.500536in}}{\pgfqpoint{2.288150in}{2.496146in}}{\pgfqpoint{2.280337in}{2.488332in}}%
\pgfpathcurveto{\pgfqpoint{2.272523in}{2.480519in}}{\pgfqpoint{2.268133in}{2.469920in}}{\pgfqpoint{2.268133in}{2.458870in}}%
\pgfpathcurveto{\pgfqpoint{2.268133in}{2.447819in}}{\pgfqpoint{2.272523in}{2.437220in}}{\pgfqpoint{2.280337in}{2.429407in}}%
\pgfpathcurveto{\pgfqpoint{2.288150in}{2.421593in}}{\pgfqpoint{2.298749in}{2.417203in}}{\pgfqpoint{2.309799in}{2.417203in}}%
\pgfpathclose%
\pgfusepath{stroke,fill}%
\end{pgfscope}%
\begin{pgfscope}%
\pgfpathrectangle{\pgfqpoint{0.787074in}{0.548769in}}{\pgfqpoint{5.062926in}{3.102590in}}%
\pgfusepath{clip}%
\pgfsetbuttcap%
\pgfsetroundjoin%
\definecolor{currentfill}{rgb}{1.000000,0.498039,0.054902}%
\pgfsetfillcolor{currentfill}%
\pgfsetlinewidth{1.003750pt}%
\definecolor{currentstroke}{rgb}{1.000000,0.498039,0.054902}%
\pgfsetstrokecolor{currentstroke}%
\pgfsetdash{}{0pt}%
\pgfpathmoveto{\pgfqpoint{2.800047in}{2.967508in}}%
\pgfpathcurveto{\pgfqpoint{2.811097in}{2.967508in}}{\pgfqpoint{2.821696in}{2.971899in}}{\pgfqpoint{2.829509in}{2.979712in}}%
\pgfpathcurveto{\pgfqpoint{2.837323in}{2.987526in}}{\pgfqpoint{2.841713in}{2.998125in}}{\pgfqpoint{2.841713in}{3.009175in}}%
\pgfpathcurveto{\pgfqpoint{2.841713in}{3.020225in}}{\pgfqpoint{2.837323in}{3.030824in}}{\pgfqpoint{2.829509in}{3.038638in}}%
\pgfpathcurveto{\pgfqpoint{2.821696in}{3.046451in}}{\pgfqpoint{2.811097in}{3.050842in}}{\pgfqpoint{2.800047in}{3.050842in}}%
\pgfpathcurveto{\pgfqpoint{2.788996in}{3.050842in}}{\pgfqpoint{2.778397in}{3.046451in}}{\pgfqpoint{2.770584in}{3.038638in}}%
\pgfpathcurveto{\pgfqpoint{2.762770in}{3.030824in}}{\pgfqpoint{2.758380in}{3.020225in}}{\pgfqpoint{2.758380in}{3.009175in}}%
\pgfpathcurveto{\pgfqpoint{2.758380in}{2.998125in}}{\pgfqpoint{2.762770in}{2.987526in}}{\pgfqpoint{2.770584in}{2.979712in}}%
\pgfpathcurveto{\pgfqpoint{2.778397in}{2.971899in}}{\pgfqpoint{2.788996in}{2.967508in}}{\pgfqpoint{2.800047in}{2.967508in}}%
\pgfpathclose%
\pgfusepath{stroke,fill}%
\end{pgfscope}%
\begin{pgfscope}%
\pgfpathrectangle{\pgfqpoint{0.787074in}{0.548769in}}{\pgfqpoint{5.062926in}{3.102590in}}%
\pgfusepath{clip}%
\pgfsetbuttcap%
\pgfsetroundjoin%
\definecolor{currentfill}{rgb}{1.000000,0.498039,0.054902}%
\pgfsetfillcolor{currentfill}%
\pgfsetlinewidth{1.003750pt}%
\definecolor{currentstroke}{rgb}{1.000000,0.498039,0.054902}%
\pgfsetstrokecolor{currentstroke}%
\pgfsetdash{}{0pt}%
\pgfpathmoveto{\pgfqpoint{1.939375in}{2.421885in}}%
\pgfpathcurveto{\pgfqpoint{1.950425in}{2.421885in}}{\pgfqpoint{1.961024in}{2.426275in}}{\pgfqpoint{1.968838in}{2.434089in}}%
\pgfpathcurveto{\pgfqpoint{1.976652in}{2.441902in}}{\pgfqpoint{1.981042in}{2.452501in}}{\pgfqpoint{1.981042in}{2.463551in}}%
\pgfpathcurveto{\pgfqpoint{1.981042in}{2.474601in}}{\pgfqpoint{1.976652in}{2.485201in}}{\pgfqpoint{1.968838in}{2.493014in}}%
\pgfpathcurveto{\pgfqpoint{1.961024in}{2.500828in}}{\pgfqpoint{1.950425in}{2.505218in}}{\pgfqpoint{1.939375in}{2.505218in}}%
\pgfpathcurveto{\pgfqpoint{1.928325in}{2.505218in}}{\pgfqpoint{1.917726in}{2.500828in}}{\pgfqpoint{1.909913in}{2.493014in}}%
\pgfpathcurveto{\pgfqpoint{1.902099in}{2.485201in}}{\pgfqpoint{1.897709in}{2.474601in}}{\pgfqpoint{1.897709in}{2.463551in}}%
\pgfpathcurveto{\pgfqpoint{1.897709in}{2.452501in}}{\pgfqpoint{1.902099in}{2.441902in}}{\pgfqpoint{1.909913in}{2.434089in}}%
\pgfpathcurveto{\pgfqpoint{1.917726in}{2.426275in}}{\pgfqpoint{1.928325in}{2.421885in}}{\pgfqpoint{1.939375in}{2.421885in}}%
\pgfpathclose%
\pgfusepath{stroke,fill}%
\end{pgfscope}%
\begin{pgfscope}%
\pgfpathrectangle{\pgfqpoint{0.787074in}{0.548769in}}{\pgfqpoint{5.062926in}{3.102590in}}%
\pgfusepath{clip}%
\pgfsetbuttcap%
\pgfsetroundjoin%
\definecolor{currentfill}{rgb}{0.121569,0.466667,0.705882}%
\pgfsetfillcolor{currentfill}%
\pgfsetlinewidth{1.003750pt}%
\definecolor{currentstroke}{rgb}{0.121569,0.466667,0.705882}%
\pgfsetstrokecolor{currentstroke}%
\pgfsetdash{}{0pt}%
\pgfpathmoveto{\pgfqpoint{1.913438in}{2.463960in}}%
\pgfpathcurveto{\pgfqpoint{1.924489in}{2.463960in}}{\pgfqpoint{1.935088in}{2.468350in}}{\pgfqpoint{1.942901in}{2.476164in}}%
\pgfpathcurveto{\pgfqpoint{1.950715in}{2.483977in}}{\pgfqpoint{1.955105in}{2.494576in}}{\pgfqpoint{1.955105in}{2.505626in}}%
\pgfpathcurveto{\pgfqpoint{1.955105in}{2.516677in}}{\pgfqpoint{1.950715in}{2.527276in}}{\pgfqpoint{1.942901in}{2.535089in}}%
\pgfpathcurveto{\pgfqpoint{1.935088in}{2.542903in}}{\pgfqpoint{1.924489in}{2.547293in}}{\pgfqpoint{1.913438in}{2.547293in}}%
\pgfpathcurveto{\pgfqpoint{1.902388in}{2.547293in}}{\pgfqpoint{1.891789in}{2.542903in}}{\pgfqpoint{1.883976in}{2.535089in}}%
\pgfpathcurveto{\pgfqpoint{1.876162in}{2.527276in}}{\pgfqpoint{1.871772in}{2.516677in}}{\pgfqpoint{1.871772in}{2.505626in}}%
\pgfpathcurveto{\pgfqpoint{1.871772in}{2.494576in}}{\pgfqpoint{1.876162in}{2.483977in}}{\pgfqpoint{1.883976in}{2.476164in}}%
\pgfpathcurveto{\pgfqpoint{1.891789in}{2.468350in}}{\pgfqpoint{1.902388in}{2.463960in}}{\pgfqpoint{1.913438in}{2.463960in}}%
\pgfpathclose%
\pgfusepath{stroke,fill}%
\end{pgfscope}%
\begin{pgfscope}%
\pgfpathrectangle{\pgfqpoint{0.787074in}{0.548769in}}{\pgfqpoint{5.062926in}{3.102590in}}%
\pgfusepath{clip}%
\pgfsetbuttcap%
\pgfsetroundjoin%
\definecolor{currentfill}{rgb}{1.000000,0.498039,0.054902}%
\pgfsetfillcolor{currentfill}%
\pgfsetlinewidth{1.003750pt}%
\definecolor{currentstroke}{rgb}{1.000000,0.498039,0.054902}%
\pgfsetstrokecolor{currentstroke}%
\pgfsetdash{}{0pt}%
\pgfpathmoveto{\pgfqpoint{1.610650in}{2.012662in}}%
\pgfpathcurveto{\pgfqpoint{1.621700in}{2.012662in}}{\pgfqpoint{1.632299in}{2.017052in}}{\pgfqpoint{1.640113in}{2.024866in}}%
\pgfpathcurveto{\pgfqpoint{1.647927in}{2.032679in}}{\pgfqpoint{1.652317in}{2.043278in}}{\pgfqpoint{1.652317in}{2.054328in}}%
\pgfpathcurveto{\pgfqpoint{1.652317in}{2.065378in}}{\pgfqpoint{1.647927in}{2.075977in}}{\pgfqpoint{1.640113in}{2.083791in}}%
\pgfpathcurveto{\pgfqpoint{1.632299in}{2.091605in}}{\pgfqpoint{1.621700in}{2.095995in}}{\pgfqpoint{1.610650in}{2.095995in}}%
\pgfpathcurveto{\pgfqpoint{1.599600in}{2.095995in}}{\pgfqpoint{1.589001in}{2.091605in}}{\pgfqpoint{1.581187in}{2.083791in}}%
\pgfpathcurveto{\pgfqpoint{1.573374in}{2.075977in}}{\pgfqpoint{1.568984in}{2.065378in}}{\pgfqpoint{1.568984in}{2.054328in}}%
\pgfpathcurveto{\pgfqpoint{1.568984in}{2.043278in}}{\pgfqpoint{1.573374in}{2.032679in}}{\pgfqpoint{1.581187in}{2.024866in}}%
\pgfpathcurveto{\pgfqpoint{1.589001in}{2.017052in}}{\pgfqpoint{1.599600in}{2.012662in}}{\pgfqpoint{1.610650in}{2.012662in}}%
\pgfpathclose%
\pgfusepath{stroke,fill}%
\end{pgfscope}%
\begin{pgfscope}%
\pgfpathrectangle{\pgfqpoint{0.787074in}{0.548769in}}{\pgfqpoint{5.062926in}{3.102590in}}%
\pgfusepath{clip}%
\pgfsetbuttcap%
\pgfsetroundjoin%
\definecolor{currentfill}{rgb}{1.000000,0.498039,0.054902}%
\pgfsetfillcolor{currentfill}%
\pgfsetlinewidth{1.003750pt}%
\definecolor{currentstroke}{rgb}{1.000000,0.498039,0.054902}%
\pgfsetstrokecolor{currentstroke}%
\pgfsetdash{}{0pt}%
\pgfpathmoveto{\pgfqpoint{1.489844in}{1.794353in}}%
\pgfpathcurveto{\pgfqpoint{1.500894in}{1.794353in}}{\pgfqpoint{1.511493in}{1.798743in}}{\pgfqpoint{1.519306in}{1.806557in}}%
\pgfpathcurveto{\pgfqpoint{1.527120in}{1.814370in}}{\pgfqpoint{1.531510in}{1.824969in}}{\pgfqpoint{1.531510in}{1.836019in}}%
\pgfpathcurveto{\pgfqpoint{1.531510in}{1.847070in}}{\pgfqpoint{1.527120in}{1.857669in}}{\pgfqpoint{1.519306in}{1.865482in}}%
\pgfpathcurveto{\pgfqpoint{1.511493in}{1.873296in}}{\pgfqpoint{1.500894in}{1.877686in}}{\pgfqpoint{1.489844in}{1.877686in}}%
\pgfpathcurveto{\pgfqpoint{1.478793in}{1.877686in}}{\pgfqpoint{1.468194in}{1.873296in}}{\pgfqpoint{1.460381in}{1.865482in}}%
\pgfpathcurveto{\pgfqpoint{1.452567in}{1.857669in}}{\pgfqpoint{1.448177in}{1.847070in}}{\pgfqpoint{1.448177in}{1.836019in}}%
\pgfpathcurveto{\pgfqpoint{1.448177in}{1.824969in}}{\pgfqpoint{1.452567in}{1.814370in}}{\pgfqpoint{1.460381in}{1.806557in}}%
\pgfpathcurveto{\pgfqpoint{1.468194in}{1.798743in}}{\pgfqpoint{1.478793in}{1.794353in}}{\pgfqpoint{1.489844in}{1.794353in}}%
\pgfpathclose%
\pgfusepath{stroke,fill}%
\end{pgfscope}%
\begin{pgfscope}%
\pgfpathrectangle{\pgfqpoint{0.787074in}{0.548769in}}{\pgfqpoint{5.062926in}{3.102590in}}%
\pgfusepath{clip}%
\pgfsetbuttcap%
\pgfsetroundjoin%
\definecolor{currentfill}{rgb}{0.121569,0.466667,0.705882}%
\pgfsetfillcolor{currentfill}%
\pgfsetlinewidth{1.003750pt}%
\definecolor{currentstroke}{rgb}{0.121569,0.466667,0.705882}%
\pgfsetstrokecolor{currentstroke}%
\pgfsetdash{}{0pt}%
\pgfpathmoveto{\pgfqpoint{2.303483in}{2.149173in}}%
\pgfpathcurveto{\pgfqpoint{2.314533in}{2.149173in}}{\pgfqpoint{2.325132in}{2.153563in}}{\pgfqpoint{2.332945in}{2.161376in}}%
\pgfpathcurveto{\pgfqpoint{2.340759in}{2.169190in}}{\pgfqpoint{2.345149in}{2.179789in}}{\pgfqpoint{2.345149in}{2.190839in}}%
\pgfpathcurveto{\pgfqpoint{2.345149in}{2.201889in}}{\pgfqpoint{2.340759in}{2.212488in}}{\pgfqpoint{2.332945in}{2.220302in}}%
\pgfpathcurveto{\pgfqpoint{2.325132in}{2.228116in}}{\pgfqpoint{2.314533in}{2.232506in}}{\pgfqpoint{2.303483in}{2.232506in}}%
\pgfpathcurveto{\pgfqpoint{2.292433in}{2.232506in}}{\pgfqpoint{2.281833in}{2.228116in}}{\pgfqpoint{2.274020in}{2.220302in}}%
\pgfpathcurveto{\pgfqpoint{2.266206in}{2.212488in}}{\pgfqpoint{2.261816in}{2.201889in}}{\pgfqpoint{2.261816in}{2.190839in}}%
\pgfpathcurveto{\pgfqpoint{2.261816in}{2.179789in}}{\pgfqpoint{2.266206in}{2.169190in}}{\pgfqpoint{2.274020in}{2.161376in}}%
\pgfpathcurveto{\pgfqpoint{2.281833in}{2.153563in}}{\pgfqpoint{2.292433in}{2.149173in}}{\pgfqpoint{2.303483in}{2.149173in}}%
\pgfpathclose%
\pgfusepath{stroke,fill}%
\end{pgfscope}%
\begin{pgfscope}%
\pgfpathrectangle{\pgfqpoint{0.787074in}{0.548769in}}{\pgfqpoint{5.062926in}{3.102590in}}%
\pgfusepath{clip}%
\pgfsetbuttcap%
\pgfsetroundjoin%
\definecolor{currentfill}{rgb}{1.000000,0.498039,0.054902}%
\pgfsetfillcolor{currentfill}%
\pgfsetlinewidth{1.003750pt}%
\definecolor{currentstroke}{rgb}{1.000000,0.498039,0.054902}%
\pgfsetstrokecolor{currentstroke}%
\pgfsetdash{}{0pt}%
\pgfpathmoveto{\pgfqpoint{2.755134in}{2.551840in}}%
\pgfpathcurveto{\pgfqpoint{2.766184in}{2.551840in}}{\pgfqpoint{2.776783in}{2.556230in}}{\pgfqpoint{2.784597in}{2.564044in}}%
\pgfpathcurveto{\pgfqpoint{2.792411in}{2.571857in}}{\pgfqpoint{2.796801in}{2.582456in}}{\pgfqpoint{2.796801in}{2.593506in}}%
\pgfpathcurveto{\pgfqpoint{2.796801in}{2.604557in}}{\pgfqpoint{2.792411in}{2.615156in}}{\pgfqpoint{2.784597in}{2.622969in}}%
\pgfpathcurveto{\pgfqpoint{2.776783in}{2.630783in}}{\pgfqpoint{2.766184in}{2.635173in}}{\pgfqpoint{2.755134in}{2.635173in}}%
\pgfpathcurveto{\pgfqpoint{2.744084in}{2.635173in}}{\pgfqpoint{2.733485in}{2.630783in}}{\pgfqpoint{2.725671in}{2.622969in}}%
\pgfpathcurveto{\pgfqpoint{2.717858in}{2.615156in}}{\pgfqpoint{2.713467in}{2.604557in}}{\pgfqpoint{2.713467in}{2.593506in}}%
\pgfpathcurveto{\pgfqpoint{2.713467in}{2.582456in}}{\pgfqpoint{2.717858in}{2.571857in}}{\pgfqpoint{2.725671in}{2.564044in}}%
\pgfpathcurveto{\pgfqpoint{2.733485in}{2.556230in}}{\pgfqpoint{2.744084in}{2.551840in}}{\pgfqpoint{2.755134in}{2.551840in}}%
\pgfpathclose%
\pgfusepath{stroke,fill}%
\end{pgfscope}%
\begin{pgfscope}%
\pgfpathrectangle{\pgfqpoint{0.787074in}{0.548769in}}{\pgfqpoint{5.062926in}{3.102590in}}%
\pgfusepath{clip}%
\pgfsetbuttcap%
\pgfsetroundjoin%
\definecolor{currentfill}{rgb}{1.000000,0.498039,0.054902}%
\pgfsetfillcolor{currentfill}%
\pgfsetlinewidth{1.003750pt}%
\definecolor{currentstroke}{rgb}{1.000000,0.498039,0.054902}%
\pgfsetstrokecolor{currentstroke}%
\pgfsetdash{}{0pt}%
\pgfpathmoveto{\pgfqpoint{4.643078in}{2.187660in}}%
\pgfpathcurveto{\pgfqpoint{4.654128in}{2.187660in}}{\pgfqpoint{4.664727in}{2.192050in}}{\pgfqpoint{4.672541in}{2.199864in}}%
\pgfpathcurveto{\pgfqpoint{4.680354in}{2.207677in}}{\pgfqpoint{4.684744in}{2.218276in}}{\pgfqpoint{4.684744in}{2.229326in}}%
\pgfpathcurveto{\pgfqpoint{4.684744in}{2.240377in}}{\pgfqpoint{4.680354in}{2.250976in}}{\pgfqpoint{4.672541in}{2.258789in}}%
\pgfpathcurveto{\pgfqpoint{4.664727in}{2.266603in}}{\pgfqpoint{4.654128in}{2.270993in}}{\pgfqpoint{4.643078in}{2.270993in}}%
\pgfpathcurveto{\pgfqpoint{4.632028in}{2.270993in}}{\pgfqpoint{4.621429in}{2.266603in}}{\pgfqpoint{4.613615in}{2.258789in}}%
\pgfpathcurveto{\pgfqpoint{4.605801in}{2.250976in}}{\pgfqpoint{4.601411in}{2.240377in}}{\pgfqpoint{4.601411in}{2.229326in}}%
\pgfpathcurveto{\pgfqpoint{4.601411in}{2.218276in}}{\pgfqpoint{4.605801in}{2.207677in}}{\pgfqpoint{4.613615in}{2.199864in}}%
\pgfpathcurveto{\pgfqpoint{4.621429in}{2.192050in}}{\pgfqpoint{4.632028in}{2.187660in}}{\pgfqpoint{4.643078in}{2.187660in}}%
\pgfpathclose%
\pgfusepath{stroke,fill}%
\end{pgfscope}%
\begin{pgfscope}%
\pgfpathrectangle{\pgfqpoint{0.787074in}{0.548769in}}{\pgfqpoint{5.062926in}{3.102590in}}%
\pgfusepath{clip}%
\pgfsetbuttcap%
\pgfsetroundjoin%
\definecolor{currentfill}{rgb}{1.000000,0.498039,0.054902}%
\pgfsetfillcolor{currentfill}%
\pgfsetlinewidth{1.003750pt}%
\definecolor{currentstroke}{rgb}{1.000000,0.498039,0.054902}%
\pgfsetstrokecolor{currentstroke}%
\pgfsetdash{}{0pt}%
\pgfpathmoveto{\pgfqpoint{1.779964in}{2.846483in}}%
\pgfpathcurveto{\pgfqpoint{1.791015in}{2.846483in}}{\pgfqpoint{1.801614in}{2.850874in}}{\pgfqpoint{1.809427in}{2.858687in}}%
\pgfpathcurveto{\pgfqpoint{1.817241in}{2.866501in}}{\pgfqpoint{1.821631in}{2.877100in}}{\pgfqpoint{1.821631in}{2.888150in}}%
\pgfpathcurveto{\pgfqpoint{1.821631in}{2.899200in}}{\pgfqpoint{1.817241in}{2.909799in}}{\pgfqpoint{1.809427in}{2.917613in}}%
\pgfpathcurveto{\pgfqpoint{1.801614in}{2.925426in}}{\pgfqpoint{1.791015in}{2.929817in}}{\pgfqpoint{1.779964in}{2.929817in}}%
\pgfpathcurveto{\pgfqpoint{1.768914in}{2.929817in}}{\pgfqpoint{1.758315in}{2.925426in}}{\pgfqpoint{1.750502in}{2.917613in}}%
\pgfpathcurveto{\pgfqpoint{1.742688in}{2.909799in}}{\pgfqpoint{1.738298in}{2.899200in}}{\pgfqpoint{1.738298in}{2.888150in}}%
\pgfpathcurveto{\pgfqpoint{1.738298in}{2.877100in}}{\pgfqpoint{1.742688in}{2.866501in}}{\pgfqpoint{1.750502in}{2.858687in}}%
\pgfpathcurveto{\pgfqpoint{1.758315in}{2.850874in}}{\pgfqpoint{1.768914in}{2.846483in}}{\pgfqpoint{1.779964in}{2.846483in}}%
\pgfpathclose%
\pgfusepath{stroke,fill}%
\end{pgfscope}%
\begin{pgfscope}%
\pgfpathrectangle{\pgfqpoint{0.787074in}{0.548769in}}{\pgfqpoint{5.062926in}{3.102590in}}%
\pgfusepath{clip}%
\pgfsetbuttcap%
\pgfsetroundjoin%
\definecolor{currentfill}{rgb}{1.000000,0.498039,0.054902}%
\pgfsetfillcolor{currentfill}%
\pgfsetlinewidth{1.003750pt}%
\definecolor{currentstroke}{rgb}{1.000000,0.498039,0.054902}%
\pgfsetstrokecolor{currentstroke}%
\pgfsetdash{}{0pt}%
\pgfpathmoveto{\pgfqpoint{2.229734in}{2.316063in}}%
\pgfpathcurveto{\pgfqpoint{2.240784in}{2.316063in}}{\pgfqpoint{2.251383in}{2.320453in}}{\pgfqpoint{2.259196in}{2.328267in}}%
\pgfpathcurveto{\pgfqpoint{2.267010in}{2.336080in}}{\pgfqpoint{2.271400in}{2.346679in}}{\pgfqpoint{2.271400in}{2.357729in}}%
\pgfpathcurveto{\pgfqpoint{2.271400in}{2.368779in}}{\pgfqpoint{2.267010in}{2.379379in}}{\pgfqpoint{2.259196in}{2.387192in}}%
\pgfpathcurveto{\pgfqpoint{2.251383in}{2.395006in}}{\pgfqpoint{2.240784in}{2.399396in}}{\pgfqpoint{2.229734in}{2.399396in}}%
\pgfpathcurveto{\pgfqpoint{2.218683in}{2.399396in}}{\pgfqpoint{2.208084in}{2.395006in}}{\pgfqpoint{2.200271in}{2.387192in}}%
\pgfpathcurveto{\pgfqpoint{2.192457in}{2.379379in}}{\pgfqpoint{2.188067in}{2.368779in}}{\pgfqpoint{2.188067in}{2.357729in}}%
\pgfpathcurveto{\pgfqpoint{2.188067in}{2.346679in}}{\pgfqpoint{2.192457in}{2.336080in}}{\pgfqpoint{2.200271in}{2.328267in}}%
\pgfpathcurveto{\pgfqpoint{2.208084in}{2.320453in}}{\pgfqpoint{2.218683in}{2.316063in}}{\pgfqpoint{2.229734in}{2.316063in}}%
\pgfpathclose%
\pgfusepath{stroke,fill}%
\end{pgfscope}%
\begin{pgfscope}%
\pgfpathrectangle{\pgfqpoint{0.787074in}{0.548769in}}{\pgfqpoint{5.062926in}{3.102590in}}%
\pgfusepath{clip}%
\pgfsetbuttcap%
\pgfsetroundjoin%
\definecolor{currentfill}{rgb}{1.000000,0.498039,0.054902}%
\pgfsetfillcolor{currentfill}%
\pgfsetlinewidth{1.003750pt}%
\definecolor{currentstroke}{rgb}{1.000000,0.498039,0.054902}%
\pgfsetstrokecolor{currentstroke}%
\pgfsetdash{}{0pt}%
\pgfpathmoveto{\pgfqpoint{2.066414in}{1.585094in}}%
\pgfpathcurveto{\pgfqpoint{2.077464in}{1.585094in}}{\pgfqpoint{2.088063in}{1.589485in}}{\pgfqpoint{2.095877in}{1.597298in}}%
\pgfpathcurveto{\pgfqpoint{2.103690in}{1.605112in}}{\pgfqpoint{2.108081in}{1.615711in}}{\pgfqpoint{2.108081in}{1.626761in}}%
\pgfpathcurveto{\pgfqpoint{2.108081in}{1.637811in}}{\pgfqpoint{2.103690in}{1.648410in}}{\pgfqpoint{2.095877in}{1.656224in}}%
\pgfpathcurveto{\pgfqpoint{2.088063in}{1.664037in}}{\pgfqpoint{2.077464in}{1.668428in}}{\pgfqpoint{2.066414in}{1.668428in}}%
\pgfpathcurveto{\pgfqpoint{2.055364in}{1.668428in}}{\pgfqpoint{2.044765in}{1.664037in}}{\pgfqpoint{2.036951in}{1.656224in}}%
\pgfpathcurveto{\pgfqpoint{2.029138in}{1.648410in}}{\pgfqpoint{2.024747in}{1.637811in}}{\pgfqpoint{2.024747in}{1.626761in}}%
\pgfpathcurveto{\pgfqpoint{2.024747in}{1.615711in}}{\pgfqpoint{2.029138in}{1.605112in}}{\pgfqpoint{2.036951in}{1.597298in}}%
\pgfpathcurveto{\pgfqpoint{2.044765in}{1.589485in}}{\pgfqpoint{2.055364in}{1.585094in}}{\pgfqpoint{2.066414in}{1.585094in}}%
\pgfpathclose%
\pgfusepath{stroke,fill}%
\end{pgfscope}%
\begin{pgfscope}%
\pgfpathrectangle{\pgfqpoint{0.787074in}{0.548769in}}{\pgfqpoint{5.062926in}{3.102590in}}%
\pgfusepath{clip}%
\pgfsetbuttcap%
\pgfsetroundjoin%
\definecolor{currentfill}{rgb}{0.121569,0.466667,0.705882}%
\pgfsetfillcolor{currentfill}%
\pgfsetlinewidth{1.003750pt}%
\definecolor{currentstroke}{rgb}{0.121569,0.466667,0.705882}%
\pgfsetstrokecolor{currentstroke}%
\pgfsetdash{}{0pt}%
\pgfpathmoveto{\pgfqpoint{2.802717in}{1.803770in}}%
\pgfpathcurveto{\pgfqpoint{2.813768in}{1.803770in}}{\pgfqpoint{2.824367in}{1.808160in}}{\pgfqpoint{2.832180in}{1.815974in}}%
\pgfpathcurveto{\pgfqpoint{2.839994in}{1.823787in}}{\pgfqpoint{2.844384in}{1.834386in}}{\pgfqpoint{2.844384in}{1.845437in}}%
\pgfpathcurveto{\pgfqpoint{2.844384in}{1.856487in}}{\pgfqpoint{2.839994in}{1.867086in}}{\pgfqpoint{2.832180in}{1.874899in}}%
\pgfpathcurveto{\pgfqpoint{2.824367in}{1.882713in}}{\pgfqpoint{2.813768in}{1.887103in}}{\pgfqpoint{2.802717in}{1.887103in}}%
\pgfpathcurveto{\pgfqpoint{2.791667in}{1.887103in}}{\pgfqpoint{2.781068in}{1.882713in}}{\pgfqpoint{2.773255in}{1.874899in}}%
\pgfpathcurveto{\pgfqpoint{2.765441in}{1.867086in}}{\pgfqpoint{2.761051in}{1.856487in}}{\pgfqpoint{2.761051in}{1.845437in}}%
\pgfpathcurveto{\pgfqpoint{2.761051in}{1.834386in}}{\pgfqpoint{2.765441in}{1.823787in}}{\pgfqpoint{2.773255in}{1.815974in}}%
\pgfpathcurveto{\pgfqpoint{2.781068in}{1.808160in}}{\pgfqpoint{2.791667in}{1.803770in}}{\pgfqpoint{2.802717in}{1.803770in}}%
\pgfpathclose%
\pgfusepath{stroke,fill}%
\end{pgfscope}%
\begin{pgfscope}%
\pgfpathrectangle{\pgfqpoint{0.787074in}{0.548769in}}{\pgfqpoint{5.062926in}{3.102590in}}%
\pgfusepath{clip}%
\pgfsetbuttcap%
\pgfsetroundjoin%
\definecolor{currentfill}{rgb}{1.000000,0.498039,0.054902}%
\pgfsetfillcolor{currentfill}%
\pgfsetlinewidth{1.003750pt}%
\definecolor{currentstroke}{rgb}{1.000000,0.498039,0.054902}%
\pgfsetstrokecolor{currentstroke}%
\pgfsetdash{}{0pt}%
\pgfpathmoveto{\pgfqpoint{2.345563in}{2.162419in}}%
\pgfpathcurveto{\pgfqpoint{2.356613in}{2.162419in}}{\pgfqpoint{2.367212in}{2.166809in}}{\pgfqpoint{2.375026in}{2.174623in}}%
\pgfpathcurveto{\pgfqpoint{2.382840in}{2.182437in}}{\pgfqpoint{2.387230in}{2.193036in}}{\pgfqpoint{2.387230in}{2.204086in}}%
\pgfpathcurveto{\pgfqpoint{2.387230in}{2.215136in}}{\pgfqpoint{2.382840in}{2.225735in}}{\pgfqpoint{2.375026in}{2.233549in}}%
\pgfpathcurveto{\pgfqpoint{2.367212in}{2.241362in}}{\pgfqpoint{2.356613in}{2.245753in}}{\pgfqpoint{2.345563in}{2.245753in}}%
\pgfpathcurveto{\pgfqpoint{2.334513in}{2.245753in}}{\pgfqpoint{2.323914in}{2.241362in}}{\pgfqpoint{2.316100in}{2.233549in}}%
\pgfpathcurveto{\pgfqpoint{2.308287in}{2.225735in}}{\pgfqpoint{2.303897in}{2.215136in}}{\pgfqpoint{2.303897in}{2.204086in}}%
\pgfpathcurveto{\pgfqpoint{2.303897in}{2.193036in}}{\pgfqpoint{2.308287in}{2.182437in}}{\pgfqpoint{2.316100in}{2.174623in}}%
\pgfpathcurveto{\pgfqpoint{2.323914in}{2.166809in}}{\pgfqpoint{2.334513in}{2.162419in}}{\pgfqpoint{2.345563in}{2.162419in}}%
\pgfpathclose%
\pgfusepath{stroke,fill}%
\end{pgfscope}%
\begin{pgfscope}%
\pgfpathrectangle{\pgfqpoint{0.787074in}{0.548769in}}{\pgfqpoint{5.062926in}{3.102590in}}%
\pgfusepath{clip}%
\pgfsetbuttcap%
\pgfsetroundjoin%
\definecolor{currentfill}{rgb}{1.000000,0.498039,0.054902}%
\pgfsetfillcolor{currentfill}%
\pgfsetlinewidth{1.003750pt}%
\definecolor{currentstroke}{rgb}{1.000000,0.498039,0.054902}%
\pgfsetstrokecolor{currentstroke}%
\pgfsetdash{}{0pt}%
\pgfpathmoveto{\pgfqpoint{2.203186in}{3.057735in}}%
\pgfpathcurveto{\pgfqpoint{2.214236in}{3.057735in}}{\pgfqpoint{2.224835in}{3.062125in}}{\pgfqpoint{2.232649in}{3.069939in}}%
\pgfpathcurveto{\pgfqpoint{2.240463in}{3.077752in}}{\pgfqpoint{2.244853in}{3.088351in}}{\pgfqpoint{2.244853in}{3.099401in}}%
\pgfpathcurveto{\pgfqpoint{2.244853in}{3.110451in}}{\pgfqpoint{2.240463in}{3.121050in}}{\pgfqpoint{2.232649in}{3.128864in}}%
\pgfpathcurveto{\pgfqpoint{2.224835in}{3.136678in}}{\pgfqpoint{2.214236in}{3.141068in}}{\pgfqpoint{2.203186in}{3.141068in}}%
\pgfpathcurveto{\pgfqpoint{2.192136in}{3.141068in}}{\pgfqpoint{2.181537in}{3.136678in}}{\pgfqpoint{2.173724in}{3.128864in}}%
\pgfpathcurveto{\pgfqpoint{2.165910in}{3.121050in}}{\pgfqpoint{2.161520in}{3.110451in}}{\pgfqpoint{2.161520in}{3.099401in}}%
\pgfpathcurveto{\pgfqpoint{2.161520in}{3.088351in}}{\pgfqpoint{2.165910in}{3.077752in}}{\pgfqpoint{2.173724in}{3.069939in}}%
\pgfpathcurveto{\pgfqpoint{2.181537in}{3.062125in}}{\pgfqpoint{2.192136in}{3.057735in}}{\pgfqpoint{2.203186in}{3.057735in}}%
\pgfpathclose%
\pgfusepath{stroke,fill}%
\end{pgfscope}%
\begin{pgfscope}%
\pgfpathrectangle{\pgfqpoint{0.787074in}{0.548769in}}{\pgfqpoint{5.062926in}{3.102590in}}%
\pgfusepath{clip}%
\pgfsetbuttcap%
\pgfsetroundjoin%
\definecolor{currentfill}{rgb}{0.121569,0.466667,0.705882}%
\pgfsetfillcolor{currentfill}%
\pgfsetlinewidth{1.003750pt}%
\definecolor{currentstroke}{rgb}{0.121569,0.466667,0.705882}%
\pgfsetstrokecolor{currentstroke}%
\pgfsetdash{}{0pt}%
\pgfpathmoveto{\pgfqpoint{2.707161in}{2.195132in}}%
\pgfpathcurveto{\pgfqpoint{2.718211in}{2.195132in}}{\pgfqpoint{2.728810in}{2.199523in}}{\pgfqpoint{2.736624in}{2.207336in}}%
\pgfpathcurveto{\pgfqpoint{2.744437in}{2.215150in}}{\pgfqpoint{2.748827in}{2.225749in}}{\pgfqpoint{2.748827in}{2.236799in}}%
\pgfpathcurveto{\pgfqpoint{2.748827in}{2.247849in}}{\pgfqpoint{2.744437in}{2.258448in}}{\pgfqpoint{2.736624in}{2.266262in}}%
\pgfpathcurveto{\pgfqpoint{2.728810in}{2.274076in}}{\pgfqpoint{2.718211in}{2.278466in}}{\pgfqpoint{2.707161in}{2.278466in}}%
\pgfpathcurveto{\pgfqpoint{2.696111in}{2.278466in}}{\pgfqpoint{2.685512in}{2.274076in}}{\pgfqpoint{2.677698in}{2.266262in}}%
\pgfpathcurveto{\pgfqpoint{2.669884in}{2.258448in}}{\pgfqpoint{2.665494in}{2.247849in}}{\pgfqpoint{2.665494in}{2.236799in}}%
\pgfpathcurveto{\pgfqpoint{2.665494in}{2.225749in}}{\pgfqpoint{2.669884in}{2.215150in}}{\pgfqpoint{2.677698in}{2.207336in}}%
\pgfpathcurveto{\pgfqpoint{2.685512in}{2.199523in}}{\pgfqpoint{2.696111in}{2.195132in}}{\pgfqpoint{2.707161in}{2.195132in}}%
\pgfpathclose%
\pgfusepath{stroke,fill}%
\end{pgfscope}%
\begin{pgfscope}%
\pgfpathrectangle{\pgfqpoint{0.787074in}{0.548769in}}{\pgfqpoint{5.062926in}{3.102590in}}%
\pgfusepath{clip}%
\pgfsetbuttcap%
\pgfsetroundjoin%
\definecolor{currentfill}{rgb}{1.000000,0.498039,0.054902}%
\pgfsetfillcolor{currentfill}%
\pgfsetlinewidth{1.003750pt}%
\definecolor{currentstroke}{rgb}{1.000000,0.498039,0.054902}%
\pgfsetstrokecolor{currentstroke}%
\pgfsetdash{}{0pt}%
\pgfpathmoveto{\pgfqpoint{1.656911in}{2.538741in}}%
\pgfpathcurveto{\pgfqpoint{1.667961in}{2.538741in}}{\pgfqpoint{1.678560in}{2.543132in}}{\pgfqpoint{1.686374in}{2.550945in}}%
\pgfpathcurveto{\pgfqpoint{1.694187in}{2.558759in}}{\pgfqpoint{1.698577in}{2.569358in}}{\pgfqpoint{1.698577in}{2.580408in}}%
\pgfpathcurveto{\pgfqpoint{1.698577in}{2.591458in}}{\pgfqpoint{1.694187in}{2.602057in}}{\pgfqpoint{1.686374in}{2.609871in}}%
\pgfpathcurveto{\pgfqpoint{1.678560in}{2.617685in}}{\pgfqpoint{1.667961in}{2.622075in}}{\pgfqpoint{1.656911in}{2.622075in}}%
\pgfpathcurveto{\pgfqpoint{1.645861in}{2.622075in}}{\pgfqpoint{1.635262in}{2.617685in}}{\pgfqpoint{1.627448in}{2.609871in}}%
\pgfpathcurveto{\pgfqpoint{1.619634in}{2.602057in}}{\pgfqpoint{1.615244in}{2.591458in}}{\pgfqpoint{1.615244in}{2.580408in}}%
\pgfpathcurveto{\pgfqpoint{1.615244in}{2.569358in}}{\pgfqpoint{1.619634in}{2.558759in}}{\pgfqpoint{1.627448in}{2.550945in}}%
\pgfpathcurveto{\pgfqpoint{1.635262in}{2.543132in}}{\pgfqpoint{1.645861in}{2.538741in}}{\pgfqpoint{1.656911in}{2.538741in}}%
\pgfpathclose%
\pgfusepath{stroke,fill}%
\end{pgfscope}%
\begin{pgfscope}%
\pgfpathrectangle{\pgfqpoint{0.787074in}{0.548769in}}{\pgfqpoint{5.062926in}{3.102590in}}%
\pgfusepath{clip}%
\pgfsetbuttcap%
\pgfsetroundjoin%
\definecolor{currentfill}{rgb}{1.000000,0.498039,0.054902}%
\pgfsetfillcolor{currentfill}%
\pgfsetlinewidth{1.003750pt}%
\definecolor{currentstroke}{rgb}{1.000000,0.498039,0.054902}%
\pgfsetstrokecolor{currentstroke}%
\pgfsetdash{}{0pt}%
\pgfpathmoveto{\pgfqpoint{1.988731in}{2.246302in}}%
\pgfpathcurveto{\pgfqpoint{1.999781in}{2.246302in}}{\pgfqpoint{2.010380in}{2.250692in}}{\pgfqpoint{2.018193in}{2.258505in}}%
\pgfpathcurveto{\pgfqpoint{2.026007in}{2.266319in}}{\pgfqpoint{2.030397in}{2.276918in}}{\pgfqpoint{2.030397in}{2.287968in}}%
\pgfpathcurveto{\pgfqpoint{2.030397in}{2.299018in}}{\pgfqpoint{2.026007in}{2.309617in}}{\pgfqpoint{2.018193in}{2.317431in}}%
\pgfpathcurveto{\pgfqpoint{2.010380in}{2.325245in}}{\pgfqpoint{1.999781in}{2.329635in}}{\pgfqpoint{1.988731in}{2.329635in}}%
\pgfpathcurveto{\pgfqpoint{1.977681in}{2.329635in}}{\pgfqpoint{1.967082in}{2.325245in}}{\pgfqpoint{1.959268in}{2.317431in}}%
\pgfpathcurveto{\pgfqpoint{1.951454in}{2.309617in}}{\pgfqpoint{1.947064in}{2.299018in}}{\pgfqpoint{1.947064in}{2.287968in}}%
\pgfpathcurveto{\pgfqpoint{1.947064in}{2.276918in}}{\pgfqpoint{1.951454in}{2.266319in}}{\pgfqpoint{1.959268in}{2.258505in}}%
\pgfpathcurveto{\pgfqpoint{1.967082in}{2.250692in}}{\pgfqpoint{1.977681in}{2.246302in}}{\pgfqpoint{1.988731in}{2.246302in}}%
\pgfpathclose%
\pgfusepath{stroke,fill}%
\end{pgfscope}%
\begin{pgfscope}%
\pgfpathrectangle{\pgfqpoint{0.787074in}{0.548769in}}{\pgfqpoint{5.062926in}{3.102590in}}%
\pgfusepath{clip}%
\pgfsetbuttcap%
\pgfsetroundjoin%
\definecolor{currentfill}{rgb}{0.121569,0.466667,0.705882}%
\pgfsetfillcolor{currentfill}%
\pgfsetlinewidth{1.003750pt}%
\definecolor{currentstroke}{rgb}{0.121569,0.466667,0.705882}%
\pgfsetstrokecolor{currentstroke}%
\pgfsetdash{}{0pt}%
\pgfpathmoveto{\pgfqpoint{2.814469in}{2.224849in}}%
\pgfpathcurveto{\pgfqpoint{2.825519in}{2.224849in}}{\pgfqpoint{2.836118in}{2.229239in}}{\pgfqpoint{2.843932in}{2.237053in}}%
\pgfpathcurveto{\pgfqpoint{2.851746in}{2.244866in}}{\pgfqpoint{2.856136in}{2.255465in}}{\pgfqpoint{2.856136in}{2.266515in}}%
\pgfpathcurveto{\pgfqpoint{2.856136in}{2.277565in}}{\pgfqpoint{2.851746in}{2.288165in}}{\pgfqpoint{2.843932in}{2.295978in}}%
\pgfpathcurveto{\pgfqpoint{2.836118in}{2.303792in}}{\pgfqpoint{2.825519in}{2.308182in}}{\pgfqpoint{2.814469in}{2.308182in}}%
\pgfpathcurveto{\pgfqpoint{2.803419in}{2.308182in}}{\pgfqpoint{2.792820in}{2.303792in}}{\pgfqpoint{2.785006in}{2.295978in}}%
\pgfpathcurveto{\pgfqpoint{2.777193in}{2.288165in}}{\pgfqpoint{2.772802in}{2.277565in}}{\pgfqpoint{2.772802in}{2.266515in}}%
\pgfpathcurveto{\pgfqpoint{2.772802in}{2.255465in}}{\pgfqpoint{2.777193in}{2.244866in}}{\pgfqpoint{2.785006in}{2.237053in}}%
\pgfpathcurveto{\pgfqpoint{2.792820in}{2.229239in}}{\pgfqpoint{2.803419in}{2.224849in}}{\pgfqpoint{2.814469in}{2.224849in}}%
\pgfpathclose%
\pgfusepath{stroke,fill}%
\end{pgfscope}%
\begin{pgfscope}%
\pgfpathrectangle{\pgfqpoint{0.787074in}{0.548769in}}{\pgfqpoint{5.062926in}{3.102590in}}%
\pgfusepath{clip}%
\pgfsetbuttcap%
\pgfsetroundjoin%
\definecolor{currentfill}{rgb}{0.121569,0.466667,0.705882}%
\pgfsetfillcolor{currentfill}%
\pgfsetlinewidth{1.003750pt}%
\definecolor{currentstroke}{rgb}{0.121569,0.466667,0.705882}%
\pgfsetstrokecolor{currentstroke}%
\pgfsetdash{}{0pt}%
\pgfpathmoveto{\pgfqpoint{1.690945in}{2.370615in}}%
\pgfpathcurveto{\pgfqpoint{1.701995in}{2.370615in}}{\pgfqpoint{1.712594in}{2.375005in}}{\pgfqpoint{1.720408in}{2.382818in}}%
\pgfpathcurveto{\pgfqpoint{1.728221in}{2.390632in}}{\pgfqpoint{1.732612in}{2.401231in}}{\pgfqpoint{1.732612in}{2.412281in}}%
\pgfpathcurveto{\pgfqpoint{1.732612in}{2.423331in}}{\pgfqpoint{1.728221in}{2.433930in}}{\pgfqpoint{1.720408in}{2.441744in}}%
\pgfpathcurveto{\pgfqpoint{1.712594in}{2.449558in}}{\pgfqpoint{1.701995in}{2.453948in}}{\pgfqpoint{1.690945in}{2.453948in}}%
\pgfpathcurveto{\pgfqpoint{1.679895in}{2.453948in}}{\pgfqpoint{1.669296in}{2.449558in}}{\pgfqpoint{1.661482in}{2.441744in}}%
\pgfpathcurveto{\pgfqpoint{1.653669in}{2.433930in}}{\pgfqpoint{1.649278in}{2.423331in}}{\pgfqpoint{1.649278in}{2.412281in}}%
\pgfpathcurveto{\pgfqpoint{1.649278in}{2.401231in}}{\pgfqpoint{1.653669in}{2.390632in}}{\pgfqpoint{1.661482in}{2.382818in}}%
\pgfpathcurveto{\pgfqpoint{1.669296in}{2.375005in}}{\pgfqpoint{1.679895in}{2.370615in}}{\pgfqpoint{1.690945in}{2.370615in}}%
\pgfpathclose%
\pgfusepath{stroke,fill}%
\end{pgfscope}%
\begin{pgfscope}%
\pgfpathrectangle{\pgfqpoint{0.787074in}{0.548769in}}{\pgfqpoint{5.062926in}{3.102590in}}%
\pgfusepath{clip}%
\pgfsetbuttcap%
\pgfsetroundjoin%
\definecolor{currentfill}{rgb}{0.121569,0.466667,0.705882}%
\pgfsetfillcolor{currentfill}%
\pgfsetlinewidth{1.003750pt}%
\definecolor{currentstroke}{rgb}{0.121569,0.466667,0.705882}%
\pgfsetstrokecolor{currentstroke}%
\pgfsetdash{}{0pt}%
\pgfpathmoveto{\pgfqpoint{2.448284in}{1.974526in}}%
\pgfpathcurveto{\pgfqpoint{2.459335in}{1.974526in}}{\pgfqpoint{2.469934in}{1.978916in}}{\pgfqpoint{2.477747in}{1.986730in}}%
\pgfpathcurveto{\pgfqpoint{2.485561in}{1.994543in}}{\pgfqpoint{2.489951in}{2.005142in}}{\pgfqpoint{2.489951in}{2.016192in}}%
\pgfpathcurveto{\pgfqpoint{2.489951in}{2.027242in}}{\pgfqpoint{2.485561in}{2.037842in}}{\pgfqpoint{2.477747in}{2.045655in}}%
\pgfpathcurveto{\pgfqpoint{2.469934in}{2.053469in}}{\pgfqpoint{2.459335in}{2.057859in}}{\pgfqpoint{2.448284in}{2.057859in}}%
\pgfpathcurveto{\pgfqpoint{2.437234in}{2.057859in}}{\pgfqpoint{2.426635in}{2.053469in}}{\pgfqpoint{2.418822in}{2.045655in}}%
\pgfpathcurveto{\pgfqpoint{2.411008in}{2.037842in}}{\pgfqpoint{2.406618in}{2.027242in}}{\pgfqpoint{2.406618in}{2.016192in}}%
\pgfpathcurveto{\pgfqpoint{2.406618in}{2.005142in}}{\pgfqpoint{2.411008in}{1.994543in}}{\pgfqpoint{2.418822in}{1.986730in}}%
\pgfpathcurveto{\pgfqpoint{2.426635in}{1.978916in}}{\pgfqpoint{2.437234in}{1.974526in}}{\pgfqpoint{2.448284in}{1.974526in}}%
\pgfpathclose%
\pgfusepath{stroke,fill}%
\end{pgfscope}%
\begin{pgfscope}%
\pgfpathrectangle{\pgfqpoint{0.787074in}{0.548769in}}{\pgfqpoint{5.062926in}{3.102590in}}%
\pgfusepath{clip}%
\pgfsetbuttcap%
\pgfsetroundjoin%
\definecolor{currentfill}{rgb}{1.000000,0.498039,0.054902}%
\pgfsetfillcolor{currentfill}%
\pgfsetlinewidth{1.003750pt}%
\definecolor{currentstroke}{rgb}{1.000000,0.498039,0.054902}%
\pgfsetstrokecolor{currentstroke}%
\pgfsetdash{}{0pt}%
\pgfpathmoveto{\pgfqpoint{2.477808in}{2.689957in}}%
\pgfpathcurveto{\pgfqpoint{2.488858in}{2.689957in}}{\pgfqpoint{2.499457in}{2.694347in}}{\pgfqpoint{2.507271in}{2.702161in}}%
\pgfpathcurveto{\pgfqpoint{2.515084in}{2.709974in}}{\pgfqpoint{2.519475in}{2.720573in}}{\pgfqpoint{2.519475in}{2.731623in}}%
\pgfpathcurveto{\pgfqpoint{2.519475in}{2.742674in}}{\pgfqpoint{2.515084in}{2.753273in}}{\pgfqpoint{2.507271in}{2.761086in}}%
\pgfpathcurveto{\pgfqpoint{2.499457in}{2.768900in}}{\pgfqpoint{2.488858in}{2.773290in}}{\pgfqpoint{2.477808in}{2.773290in}}%
\pgfpathcurveto{\pgfqpoint{2.466758in}{2.773290in}}{\pgfqpoint{2.456159in}{2.768900in}}{\pgfqpoint{2.448345in}{2.761086in}}%
\pgfpathcurveto{\pgfqpoint{2.440531in}{2.753273in}}{\pgfqpoint{2.436141in}{2.742674in}}{\pgfqpoint{2.436141in}{2.731623in}}%
\pgfpathcurveto{\pgfqpoint{2.436141in}{2.720573in}}{\pgfqpoint{2.440531in}{2.709974in}}{\pgfqpoint{2.448345in}{2.702161in}}%
\pgfpathcurveto{\pgfqpoint{2.456159in}{2.694347in}}{\pgfqpoint{2.466758in}{2.689957in}}{\pgfqpoint{2.477808in}{2.689957in}}%
\pgfpathclose%
\pgfusepath{stroke,fill}%
\end{pgfscope}%
\begin{pgfscope}%
\pgfpathrectangle{\pgfqpoint{0.787074in}{0.548769in}}{\pgfqpoint{5.062926in}{3.102590in}}%
\pgfusepath{clip}%
\pgfsetbuttcap%
\pgfsetroundjoin%
\definecolor{currentfill}{rgb}{0.121569,0.466667,0.705882}%
\pgfsetfillcolor{currentfill}%
\pgfsetlinewidth{1.003750pt}%
\definecolor{currentstroke}{rgb}{0.121569,0.466667,0.705882}%
\pgfsetstrokecolor{currentstroke}%
\pgfsetdash{}{0pt}%
\pgfpathmoveto{\pgfqpoint{4.683615in}{2.569739in}}%
\pgfpathcurveto{\pgfqpoint{4.694665in}{2.569739in}}{\pgfqpoint{4.705264in}{2.574129in}}{\pgfqpoint{4.713078in}{2.581943in}}%
\pgfpathcurveto{\pgfqpoint{4.720892in}{2.589756in}}{\pgfqpoint{4.725282in}{2.600355in}}{\pgfqpoint{4.725282in}{2.611406in}}%
\pgfpathcurveto{\pgfqpoint{4.725282in}{2.622456in}}{\pgfqpoint{4.720892in}{2.633055in}}{\pgfqpoint{4.713078in}{2.640868in}}%
\pgfpathcurveto{\pgfqpoint{4.705264in}{2.648682in}}{\pgfqpoint{4.694665in}{2.653072in}}{\pgfqpoint{4.683615in}{2.653072in}}%
\pgfpathcurveto{\pgfqpoint{4.672565in}{2.653072in}}{\pgfqpoint{4.661966in}{2.648682in}}{\pgfqpoint{4.654152in}{2.640868in}}%
\pgfpathcurveto{\pgfqpoint{4.646339in}{2.633055in}}{\pgfqpoint{4.641949in}{2.622456in}}{\pgfqpoint{4.641949in}{2.611406in}}%
\pgfpathcurveto{\pgfqpoint{4.641949in}{2.600355in}}{\pgfqpoint{4.646339in}{2.589756in}}{\pgfqpoint{4.654152in}{2.581943in}}%
\pgfpathcurveto{\pgfqpoint{4.661966in}{2.574129in}}{\pgfqpoint{4.672565in}{2.569739in}}{\pgfqpoint{4.683615in}{2.569739in}}%
\pgfpathclose%
\pgfusepath{stroke,fill}%
\end{pgfscope}%
\begin{pgfscope}%
\pgfpathrectangle{\pgfqpoint{0.787074in}{0.548769in}}{\pgfqpoint{5.062926in}{3.102590in}}%
\pgfusepath{clip}%
\pgfsetbuttcap%
\pgfsetroundjoin%
\definecolor{currentfill}{rgb}{0.121569,0.466667,0.705882}%
\pgfsetfillcolor{currentfill}%
\pgfsetlinewidth{1.003750pt}%
\definecolor{currentstroke}{rgb}{0.121569,0.466667,0.705882}%
\pgfsetstrokecolor{currentstroke}%
\pgfsetdash{}{0pt}%
\pgfpathmoveto{\pgfqpoint{2.148065in}{2.197035in}}%
\pgfpathcurveto{\pgfqpoint{2.159115in}{2.197035in}}{\pgfqpoint{2.169714in}{2.201425in}}{\pgfqpoint{2.177528in}{2.209239in}}%
\pgfpathcurveto{\pgfqpoint{2.185342in}{2.217052in}}{\pgfqpoint{2.189732in}{2.227651in}}{\pgfqpoint{2.189732in}{2.238701in}}%
\pgfpathcurveto{\pgfqpoint{2.189732in}{2.249751in}}{\pgfqpoint{2.185342in}{2.260350in}}{\pgfqpoint{2.177528in}{2.268164in}}%
\pgfpathcurveto{\pgfqpoint{2.169714in}{2.275978in}}{\pgfqpoint{2.159115in}{2.280368in}}{\pgfqpoint{2.148065in}{2.280368in}}%
\pgfpathcurveto{\pgfqpoint{2.137015in}{2.280368in}}{\pgfqpoint{2.126416in}{2.275978in}}{\pgfqpoint{2.118603in}{2.268164in}}%
\pgfpathcurveto{\pgfqpoint{2.110789in}{2.260350in}}{\pgfqpoint{2.106399in}{2.249751in}}{\pgfqpoint{2.106399in}{2.238701in}}%
\pgfpathcurveto{\pgfqpoint{2.106399in}{2.227651in}}{\pgfqpoint{2.110789in}{2.217052in}}{\pgfqpoint{2.118603in}{2.209239in}}%
\pgfpathcurveto{\pgfqpoint{2.126416in}{2.201425in}}{\pgfqpoint{2.137015in}{2.197035in}}{\pgfqpoint{2.148065in}{2.197035in}}%
\pgfpathclose%
\pgfusepath{stroke,fill}%
\end{pgfscope}%
\begin{pgfscope}%
\pgfpathrectangle{\pgfqpoint{0.787074in}{0.548769in}}{\pgfqpoint{5.062926in}{3.102590in}}%
\pgfusepath{clip}%
\pgfsetbuttcap%
\pgfsetroundjoin%
\definecolor{currentfill}{rgb}{0.121569,0.466667,0.705882}%
\pgfsetfillcolor{currentfill}%
\pgfsetlinewidth{1.003750pt}%
\definecolor{currentstroke}{rgb}{0.121569,0.466667,0.705882}%
\pgfsetstrokecolor{currentstroke}%
\pgfsetdash{}{0pt}%
\pgfpathmoveto{\pgfqpoint{1.170878in}{0.648134in}}%
\pgfpathcurveto{\pgfqpoint{1.181928in}{0.648134in}}{\pgfqpoint{1.192527in}{0.652524in}}{\pgfqpoint{1.200340in}{0.660338in}}%
\pgfpathcurveto{\pgfqpoint{1.208154in}{0.668151in}}{\pgfqpoint{1.212544in}{0.678750in}}{\pgfqpoint{1.212544in}{0.689801in}}%
\pgfpathcurveto{\pgfqpoint{1.212544in}{0.700851in}}{\pgfqpoint{1.208154in}{0.711450in}}{\pgfqpoint{1.200340in}{0.719263in}}%
\pgfpathcurveto{\pgfqpoint{1.192527in}{0.727077in}}{\pgfqpoint{1.181928in}{0.731467in}}{\pgfqpoint{1.170878in}{0.731467in}}%
\pgfpathcurveto{\pgfqpoint{1.159827in}{0.731467in}}{\pgfqpoint{1.149228in}{0.727077in}}{\pgfqpoint{1.141415in}{0.719263in}}%
\pgfpathcurveto{\pgfqpoint{1.133601in}{0.711450in}}{\pgfqpoint{1.129211in}{0.700851in}}{\pgfqpoint{1.129211in}{0.689801in}}%
\pgfpathcurveto{\pgfqpoint{1.129211in}{0.678750in}}{\pgfqpoint{1.133601in}{0.668151in}}{\pgfqpoint{1.141415in}{0.660338in}}%
\pgfpathcurveto{\pgfqpoint{1.149228in}{0.652524in}}{\pgfqpoint{1.159827in}{0.648134in}}{\pgfqpoint{1.170878in}{0.648134in}}%
\pgfpathclose%
\pgfusepath{stroke,fill}%
\end{pgfscope}%
\begin{pgfscope}%
\pgfpathrectangle{\pgfqpoint{0.787074in}{0.548769in}}{\pgfqpoint{5.062926in}{3.102590in}}%
\pgfusepath{clip}%
\pgfsetbuttcap%
\pgfsetroundjoin%
\definecolor{currentfill}{rgb}{1.000000,0.498039,0.054902}%
\pgfsetfillcolor{currentfill}%
\pgfsetlinewidth{1.003750pt}%
\definecolor{currentstroke}{rgb}{1.000000,0.498039,0.054902}%
\pgfsetstrokecolor{currentstroke}%
\pgfsetdash{}{0pt}%
\pgfpathmoveto{\pgfqpoint{2.036958in}{1.410798in}}%
\pgfpathcurveto{\pgfqpoint{2.048009in}{1.410798in}}{\pgfqpoint{2.058608in}{1.415189in}}{\pgfqpoint{2.066421in}{1.423002in}}%
\pgfpathcurveto{\pgfqpoint{2.074235in}{1.430816in}}{\pgfqpoint{2.078625in}{1.441415in}}{\pgfqpoint{2.078625in}{1.452465in}}%
\pgfpathcurveto{\pgfqpoint{2.078625in}{1.463515in}}{\pgfqpoint{2.074235in}{1.474114in}}{\pgfqpoint{2.066421in}{1.481928in}}%
\pgfpathcurveto{\pgfqpoint{2.058608in}{1.489741in}}{\pgfqpoint{2.048009in}{1.494132in}}{\pgfqpoint{2.036958in}{1.494132in}}%
\pgfpathcurveto{\pgfqpoint{2.025908in}{1.494132in}}{\pgfqpoint{2.015309in}{1.489741in}}{\pgfqpoint{2.007496in}{1.481928in}}%
\pgfpathcurveto{\pgfqpoint{1.999682in}{1.474114in}}{\pgfqpoint{1.995292in}{1.463515in}}{\pgfqpoint{1.995292in}{1.452465in}}%
\pgfpathcurveto{\pgfqpoint{1.995292in}{1.441415in}}{\pgfqpoint{1.999682in}{1.430816in}}{\pgfqpoint{2.007496in}{1.423002in}}%
\pgfpathcurveto{\pgfqpoint{2.015309in}{1.415189in}}{\pgfqpoint{2.025908in}{1.410798in}}{\pgfqpoint{2.036958in}{1.410798in}}%
\pgfpathclose%
\pgfusepath{stroke,fill}%
\end{pgfscope}%
\begin{pgfscope}%
\pgfpathrectangle{\pgfqpoint{0.787074in}{0.548769in}}{\pgfqpoint{5.062926in}{3.102590in}}%
\pgfusepath{clip}%
\pgfsetbuttcap%
\pgfsetroundjoin%
\definecolor{currentfill}{rgb}{1.000000,0.498039,0.054902}%
\pgfsetfillcolor{currentfill}%
\pgfsetlinewidth{1.003750pt}%
\definecolor{currentstroke}{rgb}{1.000000,0.498039,0.054902}%
\pgfsetstrokecolor{currentstroke}%
\pgfsetdash{}{0pt}%
\pgfpathmoveto{\pgfqpoint{2.290052in}{2.681418in}}%
\pgfpathcurveto{\pgfqpoint{2.301102in}{2.681418in}}{\pgfqpoint{2.311701in}{2.685808in}}{\pgfqpoint{2.319515in}{2.693622in}}%
\pgfpathcurveto{\pgfqpoint{2.327329in}{2.701435in}}{\pgfqpoint{2.331719in}{2.712034in}}{\pgfqpoint{2.331719in}{2.723084in}}%
\pgfpathcurveto{\pgfqpoint{2.331719in}{2.734135in}}{\pgfqpoint{2.327329in}{2.744734in}}{\pgfqpoint{2.319515in}{2.752547in}}%
\pgfpathcurveto{\pgfqpoint{2.311701in}{2.760361in}}{\pgfqpoint{2.301102in}{2.764751in}}{\pgfqpoint{2.290052in}{2.764751in}}%
\pgfpathcurveto{\pgfqpoint{2.279002in}{2.764751in}}{\pgfqpoint{2.268403in}{2.760361in}}{\pgfqpoint{2.260589in}{2.752547in}}%
\pgfpathcurveto{\pgfqpoint{2.252776in}{2.744734in}}{\pgfqpoint{2.248385in}{2.734135in}}{\pgfqpoint{2.248385in}{2.723084in}}%
\pgfpathcurveto{\pgfqpoint{2.248385in}{2.712034in}}{\pgfqpoint{2.252776in}{2.701435in}}{\pgfqpoint{2.260589in}{2.693622in}}%
\pgfpathcurveto{\pgfqpoint{2.268403in}{2.685808in}}{\pgfqpoint{2.279002in}{2.681418in}}{\pgfqpoint{2.290052in}{2.681418in}}%
\pgfpathclose%
\pgfusepath{stroke,fill}%
\end{pgfscope}%
\begin{pgfscope}%
\pgfpathrectangle{\pgfqpoint{0.787074in}{0.548769in}}{\pgfqpoint{5.062926in}{3.102590in}}%
\pgfusepath{clip}%
\pgfsetbuttcap%
\pgfsetroundjoin%
\definecolor{currentfill}{rgb}{1.000000,0.498039,0.054902}%
\pgfsetfillcolor{currentfill}%
\pgfsetlinewidth{1.003750pt}%
\definecolor{currentstroke}{rgb}{1.000000,0.498039,0.054902}%
\pgfsetstrokecolor{currentstroke}%
\pgfsetdash{}{0pt}%
\pgfpathmoveto{\pgfqpoint{1.392769in}{2.256332in}}%
\pgfpathcurveto{\pgfqpoint{1.403819in}{2.256332in}}{\pgfqpoint{1.414418in}{2.260722in}}{\pgfqpoint{1.422232in}{2.268536in}}%
\pgfpathcurveto{\pgfqpoint{1.430046in}{2.276349in}}{\pgfqpoint{1.434436in}{2.286948in}}{\pgfqpoint{1.434436in}{2.297998in}}%
\pgfpathcurveto{\pgfqpoint{1.434436in}{2.309048in}}{\pgfqpoint{1.430046in}{2.319648in}}{\pgfqpoint{1.422232in}{2.327461in}}%
\pgfpathcurveto{\pgfqpoint{1.414418in}{2.335275in}}{\pgfqpoint{1.403819in}{2.339665in}}{\pgfqpoint{1.392769in}{2.339665in}}%
\pgfpathcurveto{\pgfqpoint{1.381719in}{2.339665in}}{\pgfqpoint{1.371120in}{2.335275in}}{\pgfqpoint{1.363306in}{2.327461in}}%
\pgfpathcurveto{\pgfqpoint{1.355493in}{2.319648in}}{\pgfqpoint{1.351102in}{2.309048in}}{\pgfqpoint{1.351102in}{2.297998in}}%
\pgfpathcurveto{\pgfqpoint{1.351102in}{2.286948in}}{\pgfqpoint{1.355493in}{2.276349in}}{\pgfqpoint{1.363306in}{2.268536in}}%
\pgfpathcurveto{\pgfqpoint{1.371120in}{2.260722in}}{\pgfqpoint{1.381719in}{2.256332in}}{\pgfqpoint{1.392769in}{2.256332in}}%
\pgfpathclose%
\pgfusepath{stroke,fill}%
\end{pgfscope}%
\begin{pgfscope}%
\pgfpathrectangle{\pgfqpoint{0.787074in}{0.548769in}}{\pgfqpoint{5.062926in}{3.102590in}}%
\pgfusepath{clip}%
\pgfsetbuttcap%
\pgfsetroundjoin%
\definecolor{currentfill}{rgb}{0.121569,0.466667,0.705882}%
\pgfsetfillcolor{currentfill}%
\pgfsetlinewidth{1.003750pt}%
\definecolor{currentstroke}{rgb}{0.121569,0.466667,0.705882}%
\pgfsetstrokecolor{currentstroke}%
\pgfsetdash{}{0pt}%
\pgfpathmoveto{\pgfqpoint{1.354852in}{0.775326in}}%
\pgfpathcurveto{\pgfqpoint{1.365902in}{0.775326in}}{\pgfqpoint{1.376501in}{0.779716in}}{\pgfqpoint{1.384314in}{0.787530in}}%
\pgfpathcurveto{\pgfqpoint{1.392128in}{0.795343in}}{\pgfqpoint{1.396518in}{0.805942in}}{\pgfqpoint{1.396518in}{0.816993in}}%
\pgfpathcurveto{\pgfqpoint{1.396518in}{0.828043in}}{\pgfqpoint{1.392128in}{0.838642in}}{\pgfqpoint{1.384314in}{0.846455in}}%
\pgfpathcurveto{\pgfqpoint{1.376501in}{0.854269in}}{\pgfqpoint{1.365902in}{0.858659in}}{\pgfqpoint{1.354852in}{0.858659in}}%
\pgfpathcurveto{\pgfqpoint{1.343802in}{0.858659in}}{\pgfqpoint{1.333203in}{0.854269in}}{\pgfqpoint{1.325389in}{0.846455in}}%
\pgfpathcurveto{\pgfqpoint{1.317575in}{0.838642in}}{\pgfqpoint{1.313185in}{0.828043in}}{\pgfqpoint{1.313185in}{0.816993in}}%
\pgfpathcurveto{\pgfqpoint{1.313185in}{0.805942in}}{\pgfqpoint{1.317575in}{0.795343in}}{\pgfqpoint{1.325389in}{0.787530in}}%
\pgfpathcurveto{\pgfqpoint{1.333203in}{0.779716in}}{\pgfqpoint{1.343802in}{0.775326in}}{\pgfqpoint{1.354852in}{0.775326in}}%
\pgfpathclose%
\pgfusepath{stroke,fill}%
\end{pgfscope}%
\begin{pgfscope}%
\pgfpathrectangle{\pgfqpoint{0.787074in}{0.548769in}}{\pgfqpoint{5.062926in}{3.102590in}}%
\pgfusepath{clip}%
\pgfsetbuttcap%
\pgfsetroundjoin%
\definecolor{currentfill}{rgb}{1.000000,0.498039,0.054902}%
\pgfsetfillcolor{currentfill}%
\pgfsetlinewidth{1.003750pt}%
\definecolor{currentstroke}{rgb}{1.000000,0.498039,0.054902}%
\pgfsetstrokecolor{currentstroke}%
\pgfsetdash{}{0pt}%
\pgfpathmoveto{\pgfqpoint{2.498700in}{1.583415in}}%
\pgfpathcurveto{\pgfqpoint{2.509750in}{1.583415in}}{\pgfqpoint{2.520349in}{1.587805in}}{\pgfqpoint{2.528163in}{1.595619in}}%
\pgfpathcurveto{\pgfqpoint{2.535976in}{1.603432in}}{\pgfqpoint{2.540366in}{1.614031in}}{\pgfqpoint{2.540366in}{1.625081in}}%
\pgfpathcurveto{\pgfqpoint{2.540366in}{1.636131in}}{\pgfqpoint{2.535976in}{1.646730in}}{\pgfqpoint{2.528163in}{1.654544in}}%
\pgfpathcurveto{\pgfqpoint{2.520349in}{1.662358in}}{\pgfqpoint{2.509750in}{1.666748in}}{\pgfqpoint{2.498700in}{1.666748in}}%
\pgfpathcurveto{\pgfqpoint{2.487650in}{1.666748in}}{\pgfqpoint{2.477051in}{1.662358in}}{\pgfqpoint{2.469237in}{1.654544in}}%
\pgfpathcurveto{\pgfqpoint{2.461423in}{1.646730in}}{\pgfqpoint{2.457033in}{1.636131in}}{\pgfqpoint{2.457033in}{1.625081in}}%
\pgfpathcurveto{\pgfqpoint{2.457033in}{1.614031in}}{\pgfqpoint{2.461423in}{1.603432in}}{\pgfqpoint{2.469237in}{1.595619in}}%
\pgfpathcurveto{\pgfqpoint{2.477051in}{1.587805in}}{\pgfqpoint{2.487650in}{1.583415in}}{\pgfqpoint{2.498700in}{1.583415in}}%
\pgfpathclose%
\pgfusepath{stroke,fill}%
\end{pgfscope}%
\begin{pgfscope}%
\pgfpathrectangle{\pgfqpoint{0.787074in}{0.548769in}}{\pgfqpoint{5.062926in}{3.102590in}}%
\pgfusepath{clip}%
\pgfsetbuttcap%
\pgfsetroundjoin%
\definecolor{currentfill}{rgb}{0.121569,0.466667,0.705882}%
\pgfsetfillcolor{currentfill}%
\pgfsetlinewidth{1.003750pt}%
\definecolor{currentstroke}{rgb}{0.121569,0.466667,0.705882}%
\pgfsetstrokecolor{currentstroke}%
\pgfsetdash{}{0pt}%
\pgfpathmoveto{\pgfqpoint{2.237958in}{2.261976in}}%
\pgfpathcurveto{\pgfqpoint{2.249008in}{2.261976in}}{\pgfqpoint{2.259607in}{2.266366in}}{\pgfqpoint{2.267421in}{2.274180in}}%
\pgfpathcurveto{\pgfqpoint{2.275234in}{2.281994in}}{\pgfqpoint{2.279625in}{2.292593in}}{\pgfqpoint{2.279625in}{2.303643in}}%
\pgfpathcurveto{\pgfqpoint{2.279625in}{2.314693in}}{\pgfqpoint{2.275234in}{2.325292in}}{\pgfqpoint{2.267421in}{2.333105in}}%
\pgfpathcurveto{\pgfqpoint{2.259607in}{2.340919in}}{\pgfqpoint{2.249008in}{2.345309in}}{\pgfqpoint{2.237958in}{2.345309in}}%
\pgfpathcurveto{\pgfqpoint{2.226908in}{2.345309in}}{\pgfqpoint{2.216309in}{2.340919in}}{\pgfqpoint{2.208495in}{2.333105in}}%
\pgfpathcurveto{\pgfqpoint{2.200682in}{2.325292in}}{\pgfqpoint{2.196291in}{2.314693in}}{\pgfqpoint{2.196291in}{2.303643in}}%
\pgfpathcurveto{\pgfqpoint{2.196291in}{2.292593in}}{\pgfqpoint{2.200682in}{2.281994in}}{\pgfqpoint{2.208495in}{2.274180in}}%
\pgfpathcurveto{\pgfqpoint{2.216309in}{2.266366in}}{\pgfqpoint{2.226908in}{2.261976in}}{\pgfqpoint{2.237958in}{2.261976in}}%
\pgfpathclose%
\pgfusepath{stroke,fill}%
\end{pgfscope}%
\begin{pgfscope}%
\pgfpathrectangle{\pgfqpoint{0.787074in}{0.548769in}}{\pgfqpoint{5.062926in}{3.102590in}}%
\pgfusepath{clip}%
\pgfsetbuttcap%
\pgfsetroundjoin%
\definecolor{currentfill}{rgb}{0.121569,0.466667,0.705882}%
\pgfsetfillcolor{currentfill}%
\pgfsetlinewidth{1.003750pt}%
\definecolor{currentstroke}{rgb}{0.121569,0.466667,0.705882}%
\pgfsetstrokecolor{currentstroke}%
\pgfsetdash{}{0pt}%
\pgfpathmoveto{\pgfqpoint{3.201156in}{3.078940in}}%
\pgfpathcurveto{\pgfqpoint{3.212206in}{3.078940in}}{\pgfqpoint{3.222805in}{3.083331in}}{\pgfqpoint{3.230618in}{3.091144in}}%
\pgfpathcurveto{\pgfqpoint{3.238432in}{3.098958in}}{\pgfqpoint{3.242822in}{3.109557in}}{\pgfqpoint{3.242822in}{3.120607in}}%
\pgfpathcurveto{\pgfqpoint{3.242822in}{3.131657in}}{\pgfqpoint{3.238432in}{3.142256in}}{\pgfqpoint{3.230618in}{3.150070in}}%
\pgfpathcurveto{\pgfqpoint{3.222805in}{3.157883in}}{\pgfqpoint{3.212206in}{3.162274in}}{\pgfqpoint{3.201156in}{3.162274in}}%
\pgfpathcurveto{\pgfqpoint{3.190106in}{3.162274in}}{\pgfqpoint{3.179507in}{3.157883in}}{\pgfqpoint{3.171693in}{3.150070in}}%
\pgfpathcurveto{\pgfqpoint{3.163879in}{3.142256in}}{\pgfqpoint{3.159489in}{3.131657in}}{\pgfqpoint{3.159489in}{3.120607in}}%
\pgfpathcurveto{\pgfqpoint{3.159489in}{3.109557in}}{\pgfqpoint{3.163879in}{3.098958in}}{\pgfqpoint{3.171693in}{3.091144in}}%
\pgfpathcurveto{\pgfqpoint{3.179507in}{3.083331in}}{\pgfqpoint{3.190106in}{3.078940in}}{\pgfqpoint{3.201156in}{3.078940in}}%
\pgfpathclose%
\pgfusepath{stroke,fill}%
\end{pgfscope}%
\begin{pgfscope}%
\pgfpathrectangle{\pgfqpoint{0.787074in}{0.548769in}}{\pgfqpoint{5.062926in}{3.102590in}}%
\pgfusepath{clip}%
\pgfsetbuttcap%
\pgfsetroundjoin%
\definecolor{currentfill}{rgb}{1.000000,0.498039,0.054902}%
\pgfsetfillcolor{currentfill}%
\pgfsetlinewidth{1.003750pt}%
\definecolor{currentstroke}{rgb}{1.000000,0.498039,0.054902}%
\pgfsetstrokecolor{currentstroke}%
\pgfsetdash{}{0pt}%
\pgfpathmoveto{\pgfqpoint{2.400777in}{1.210818in}}%
\pgfpathcurveto{\pgfqpoint{2.411828in}{1.210818in}}{\pgfqpoint{2.422427in}{1.215208in}}{\pgfqpoint{2.430240in}{1.223021in}}%
\pgfpathcurveto{\pgfqpoint{2.438054in}{1.230835in}}{\pgfqpoint{2.442444in}{1.241434in}}{\pgfqpoint{2.442444in}{1.252484in}}%
\pgfpathcurveto{\pgfqpoint{2.442444in}{1.263534in}}{\pgfqpoint{2.438054in}{1.274133in}}{\pgfqpoint{2.430240in}{1.281947in}}%
\pgfpathcurveto{\pgfqpoint{2.422427in}{1.289761in}}{\pgfqpoint{2.411828in}{1.294151in}}{\pgfqpoint{2.400777in}{1.294151in}}%
\pgfpathcurveto{\pgfqpoint{2.389727in}{1.294151in}}{\pgfqpoint{2.379128in}{1.289761in}}{\pgfqpoint{2.371315in}{1.281947in}}%
\pgfpathcurveto{\pgfqpoint{2.363501in}{1.274133in}}{\pgfqpoint{2.359111in}{1.263534in}}{\pgfqpoint{2.359111in}{1.252484in}}%
\pgfpathcurveto{\pgfqpoint{2.359111in}{1.241434in}}{\pgfqpoint{2.363501in}{1.230835in}}{\pgfqpoint{2.371315in}{1.223021in}}%
\pgfpathcurveto{\pgfqpoint{2.379128in}{1.215208in}}{\pgfqpoint{2.389727in}{1.210818in}}{\pgfqpoint{2.400777in}{1.210818in}}%
\pgfpathclose%
\pgfusepath{stroke,fill}%
\end{pgfscope}%
\begin{pgfscope}%
\pgfpathrectangle{\pgfqpoint{0.787074in}{0.548769in}}{\pgfqpoint{5.062926in}{3.102590in}}%
\pgfusepath{clip}%
\pgfsetbuttcap%
\pgfsetroundjoin%
\definecolor{currentfill}{rgb}{1.000000,0.498039,0.054902}%
\pgfsetfillcolor{currentfill}%
\pgfsetlinewidth{1.003750pt}%
\definecolor{currentstroke}{rgb}{1.000000,0.498039,0.054902}%
\pgfsetstrokecolor{currentstroke}%
\pgfsetdash{}{0pt}%
\pgfpathmoveto{\pgfqpoint{2.514165in}{1.923637in}}%
\pgfpathcurveto{\pgfqpoint{2.525215in}{1.923637in}}{\pgfqpoint{2.535814in}{1.928027in}}{\pgfqpoint{2.543628in}{1.935841in}}%
\pgfpathcurveto{\pgfqpoint{2.551442in}{1.943654in}}{\pgfqpoint{2.555832in}{1.954253in}}{\pgfqpoint{2.555832in}{1.965303in}}%
\pgfpathcurveto{\pgfqpoint{2.555832in}{1.976354in}}{\pgfqpoint{2.551442in}{1.986953in}}{\pgfqpoint{2.543628in}{1.994766in}}%
\pgfpathcurveto{\pgfqpoint{2.535814in}{2.002580in}}{\pgfqpoint{2.525215in}{2.006970in}}{\pgfqpoint{2.514165in}{2.006970in}}%
\pgfpathcurveto{\pgfqpoint{2.503115in}{2.006970in}}{\pgfqpoint{2.492516in}{2.002580in}}{\pgfqpoint{2.484702in}{1.994766in}}%
\pgfpathcurveto{\pgfqpoint{2.476889in}{1.986953in}}{\pgfqpoint{2.472498in}{1.976354in}}{\pgfqpoint{2.472498in}{1.965303in}}%
\pgfpathcurveto{\pgfqpoint{2.472498in}{1.954253in}}{\pgfqpoint{2.476889in}{1.943654in}}{\pgfqpoint{2.484702in}{1.935841in}}%
\pgfpathcurveto{\pgfqpoint{2.492516in}{1.928027in}}{\pgfqpoint{2.503115in}{1.923637in}}{\pgfqpoint{2.514165in}{1.923637in}}%
\pgfpathclose%
\pgfusepath{stroke,fill}%
\end{pgfscope}%
\begin{pgfscope}%
\pgfpathrectangle{\pgfqpoint{0.787074in}{0.548769in}}{\pgfqpoint{5.062926in}{3.102590in}}%
\pgfusepath{clip}%
\pgfsetbuttcap%
\pgfsetroundjoin%
\definecolor{currentfill}{rgb}{0.121569,0.466667,0.705882}%
\pgfsetfillcolor{currentfill}%
\pgfsetlinewidth{1.003750pt}%
\definecolor{currentstroke}{rgb}{0.121569,0.466667,0.705882}%
\pgfsetstrokecolor{currentstroke}%
\pgfsetdash{}{0pt}%
\pgfpathmoveto{\pgfqpoint{1.738689in}{2.880292in}}%
\pgfpathcurveto{\pgfqpoint{1.749739in}{2.880292in}}{\pgfqpoint{1.760339in}{2.884682in}}{\pgfqpoint{1.768152in}{2.892495in}}%
\pgfpathcurveto{\pgfqpoint{1.775966in}{2.900309in}}{\pgfqpoint{1.780356in}{2.910908in}}{\pgfqpoint{1.780356in}{2.921958in}}%
\pgfpathcurveto{\pgfqpoint{1.780356in}{2.933008in}}{\pgfqpoint{1.775966in}{2.943607in}}{\pgfqpoint{1.768152in}{2.951421in}}%
\pgfpathcurveto{\pgfqpoint{1.760339in}{2.959235in}}{\pgfqpoint{1.749739in}{2.963625in}}{\pgfqpoint{1.738689in}{2.963625in}}%
\pgfpathcurveto{\pgfqpoint{1.727639in}{2.963625in}}{\pgfqpoint{1.717040in}{2.959235in}}{\pgfqpoint{1.709227in}{2.951421in}}%
\pgfpathcurveto{\pgfqpoint{1.701413in}{2.943607in}}{\pgfqpoint{1.697023in}{2.933008in}}{\pgfqpoint{1.697023in}{2.921958in}}%
\pgfpathcurveto{\pgfqpoint{1.697023in}{2.910908in}}{\pgfqpoint{1.701413in}{2.900309in}}{\pgfqpoint{1.709227in}{2.892495in}}%
\pgfpathcurveto{\pgfqpoint{1.717040in}{2.884682in}}{\pgfqpoint{1.727639in}{2.880292in}}{\pgfqpoint{1.738689in}{2.880292in}}%
\pgfpathclose%
\pgfusepath{stroke,fill}%
\end{pgfscope}%
\begin{pgfscope}%
\pgfpathrectangle{\pgfqpoint{0.787074in}{0.548769in}}{\pgfqpoint{5.062926in}{3.102590in}}%
\pgfusepath{clip}%
\pgfsetbuttcap%
\pgfsetroundjoin%
\definecolor{currentfill}{rgb}{0.121569,0.466667,0.705882}%
\pgfsetfillcolor{currentfill}%
\pgfsetlinewidth{1.003750pt}%
\definecolor{currentstroke}{rgb}{0.121569,0.466667,0.705882}%
\pgfsetstrokecolor{currentstroke}%
\pgfsetdash{}{0pt}%
\pgfpathmoveto{\pgfqpoint{2.066982in}{2.018767in}}%
\pgfpathcurveto{\pgfqpoint{2.078032in}{2.018767in}}{\pgfqpoint{2.088631in}{2.023158in}}{\pgfqpoint{2.096445in}{2.030971in}}%
\pgfpathcurveto{\pgfqpoint{2.104258in}{2.038785in}}{\pgfqpoint{2.108649in}{2.049384in}}{\pgfqpoint{2.108649in}{2.060434in}}%
\pgfpathcurveto{\pgfqpoint{2.108649in}{2.071484in}}{\pgfqpoint{2.104258in}{2.082083in}}{\pgfqpoint{2.096445in}{2.089897in}}%
\pgfpathcurveto{\pgfqpoint{2.088631in}{2.097710in}}{\pgfqpoint{2.078032in}{2.102101in}}{\pgfqpoint{2.066982in}{2.102101in}}%
\pgfpathcurveto{\pgfqpoint{2.055932in}{2.102101in}}{\pgfqpoint{2.045333in}{2.097710in}}{\pgfqpoint{2.037519in}{2.089897in}}%
\pgfpathcurveto{\pgfqpoint{2.029706in}{2.082083in}}{\pgfqpoint{2.025315in}{2.071484in}}{\pgfqpoint{2.025315in}{2.060434in}}%
\pgfpathcurveto{\pgfqpoint{2.025315in}{2.049384in}}{\pgfqpoint{2.029706in}{2.038785in}}{\pgfqpoint{2.037519in}{2.030971in}}%
\pgfpathcurveto{\pgfqpoint{2.045333in}{2.023158in}}{\pgfqpoint{2.055932in}{2.018767in}}{\pgfqpoint{2.066982in}{2.018767in}}%
\pgfpathclose%
\pgfusepath{stroke,fill}%
\end{pgfscope}%
\begin{pgfscope}%
\pgfpathrectangle{\pgfqpoint{0.787074in}{0.548769in}}{\pgfqpoint{5.062926in}{3.102590in}}%
\pgfusepath{clip}%
\pgfsetbuttcap%
\pgfsetroundjoin%
\definecolor{currentfill}{rgb}{1.000000,0.498039,0.054902}%
\pgfsetfillcolor{currentfill}%
\pgfsetlinewidth{1.003750pt}%
\definecolor{currentstroke}{rgb}{1.000000,0.498039,0.054902}%
\pgfsetstrokecolor{currentstroke}%
\pgfsetdash{}{0pt}%
\pgfpathmoveto{\pgfqpoint{2.681419in}{0.875725in}}%
\pgfpathcurveto{\pgfqpoint{2.692469in}{0.875725in}}{\pgfqpoint{2.703068in}{0.880115in}}{\pgfqpoint{2.710882in}{0.887929in}}%
\pgfpathcurveto{\pgfqpoint{2.718695in}{0.895743in}}{\pgfqpoint{2.723086in}{0.906342in}}{\pgfqpoint{2.723086in}{0.917392in}}%
\pgfpathcurveto{\pgfqpoint{2.723086in}{0.928442in}}{\pgfqpoint{2.718695in}{0.939041in}}{\pgfqpoint{2.710882in}{0.946855in}}%
\pgfpathcurveto{\pgfqpoint{2.703068in}{0.954668in}}{\pgfqpoint{2.692469in}{0.959058in}}{\pgfqpoint{2.681419in}{0.959058in}}%
\pgfpathcurveto{\pgfqpoint{2.670369in}{0.959058in}}{\pgfqpoint{2.659770in}{0.954668in}}{\pgfqpoint{2.651956in}{0.946855in}}%
\pgfpathcurveto{\pgfqpoint{2.644143in}{0.939041in}}{\pgfqpoint{2.639752in}{0.928442in}}{\pgfqpoint{2.639752in}{0.917392in}}%
\pgfpathcurveto{\pgfqpoint{2.639752in}{0.906342in}}{\pgfqpoint{2.644143in}{0.895743in}}{\pgfqpoint{2.651956in}{0.887929in}}%
\pgfpathcurveto{\pgfqpoint{2.659770in}{0.880115in}}{\pgfqpoint{2.670369in}{0.875725in}}{\pgfqpoint{2.681419in}{0.875725in}}%
\pgfpathclose%
\pgfusepath{stroke,fill}%
\end{pgfscope}%
\begin{pgfscope}%
\pgfpathrectangle{\pgfqpoint{0.787074in}{0.548769in}}{\pgfqpoint{5.062926in}{3.102590in}}%
\pgfusepath{clip}%
\pgfsetbuttcap%
\pgfsetroundjoin%
\definecolor{currentfill}{rgb}{1.000000,0.498039,0.054902}%
\pgfsetfillcolor{currentfill}%
\pgfsetlinewidth{1.003750pt}%
\definecolor{currentstroke}{rgb}{1.000000,0.498039,0.054902}%
\pgfsetstrokecolor{currentstroke}%
\pgfsetdash{}{0pt}%
\pgfpathmoveto{\pgfqpoint{1.608107in}{1.792377in}}%
\pgfpathcurveto{\pgfqpoint{1.619157in}{1.792377in}}{\pgfqpoint{1.629756in}{1.796767in}}{\pgfqpoint{1.637569in}{1.804581in}}%
\pgfpathcurveto{\pgfqpoint{1.645383in}{1.812394in}}{\pgfqpoint{1.649773in}{1.822993in}}{\pgfqpoint{1.649773in}{1.834043in}}%
\pgfpathcurveto{\pgfqpoint{1.649773in}{1.845093in}}{\pgfqpoint{1.645383in}{1.855693in}}{\pgfqpoint{1.637569in}{1.863506in}}%
\pgfpathcurveto{\pgfqpoint{1.629756in}{1.871320in}}{\pgfqpoint{1.619157in}{1.875710in}}{\pgfqpoint{1.608107in}{1.875710in}}%
\pgfpathcurveto{\pgfqpoint{1.597056in}{1.875710in}}{\pgfqpoint{1.586457in}{1.871320in}}{\pgfqpoint{1.578644in}{1.863506in}}%
\pgfpathcurveto{\pgfqpoint{1.570830in}{1.855693in}}{\pgfqpoint{1.566440in}{1.845093in}}{\pgfqpoint{1.566440in}{1.834043in}}%
\pgfpathcurveto{\pgfqpoint{1.566440in}{1.822993in}}{\pgfqpoint{1.570830in}{1.812394in}}{\pgfqpoint{1.578644in}{1.804581in}}%
\pgfpathcurveto{\pgfqpoint{1.586457in}{1.796767in}}{\pgfqpoint{1.597056in}{1.792377in}}{\pgfqpoint{1.608107in}{1.792377in}}%
\pgfpathclose%
\pgfusepath{stroke,fill}%
\end{pgfscope}%
\begin{pgfscope}%
\pgfpathrectangle{\pgfqpoint{0.787074in}{0.548769in}}{\pgfqpoint{5.062926in}{3.102590in}}%
\pgfusepath{clip}%
\pgfsetbuttcap%
\pgfsetroundjoin%
\definecolor{currentfill}{rgb}{1.000000,0.498039,0.054902}%
\pgfsetfillcolor{currentfill}%
\pgfsetlinewidth{1.003750pt}%
\definecolor{currentstroke}{rgb}{1.000000,0.498039,0.054902}%
\pgfsetstrokecolor{currentstroke}%
\pgfsetdash{}{0pt}%
\pgfpathmoveto{\pgfqpoint{1.879303in}{1.860728in}}%
\pgfpathcurveto{\pgfqpoint{1.890353in}{1.860728in}}{\pgfqpoint{1.900952in}{1.865118in}}{\pgfqpoint{1.908765in}{1.872931in}}%
\pgfpathcurveto{\pgfqpoint{1.916579in}{1.880745in}}{\pgfqpoint{1.920969in}{1.891344in}}{\pgfqpoint{1.920969in}{1.902394in}}%
\pgfpathcurveto{\pgfqpoint{1.920969in}{1.913444in}}{\pgfqpoint{1.916579in}{1.924043in}}{\pgfqpoint{1.908765in}{1.931857in}}%
\pgfpathcurveto{\pgfqpoint{1.900952in}{1.939671in}}{\pgfqpoint{1.890353in}{1.944061in}}{\pgfqpoint{1.879303in}{1.944061in}}%
\pgfpathcurveto{\pgfqpoint{1.868252in}{1.944061in}}{\pgfqpoint{1.857653in}{1.939671in}}{\pgfqpoint{1.849840in}{1.931857in}}%
\pgfpathcurveto{\pgfqpoint{1.842026in}{1.924043in}}{\pgfqpoint{1.837636in}{1.913444in}}{\pgfqpoint{1.837636in}{1.902394in}}%
\pgfpathcurveto{\pgfqpoint{1.837636in}{1.891344in}}{\pgfqpoint{1.842026in}{1.880745in}}{\pgfqpoint{1.849840in}{1.872931in}}%
\pgfpathcurveto{\pgfqpoint{1.857653in}{1.865118in}}{\pgfqpoint{1.868252in}{1.860728in}}{\pgfqpoint{1.879303in}{1.860728in}}%
\pgfpathclose%
\pgfusepath{stroke,fill}%
\end{pgfscope}%
\begin{pgfscope}%
\pgfpathrectangle{\pgfqpoint{0.787074in}{0.548769in}}{\pgfqpoint{5.062926in}{3.102590in}}%
\pgfusepath{clip}%
\pgfsetbuttcap%
\pgfsetroundjoin%
\definecolor{currentfill}{rgb}{0.121569,0.466667,0.705882}%
\pgfsetfillcolor{currentfill}%
\pgfsetlinewidth{1.003750pt}%
\definecolor{currentstroke}{rgb}{0.121569,0.466667,0.705882}%
\pgfsetstrokecolor{currentstroke}%
\pgfsetdash{}{0pt}%
\pgfpathmoveto{\pgfqpoint{2.431403in}{1.531547in}}%
\pgfpathcurveto{\pgfqpoint{2.442453in}{1.531547in}}{\pgfqpoint{2.453052in}{1.535938in}}{\pgfqpoint{2.460866in}{1.543751in}}%
\pgfpathcurveto{\pgfqpoint{2.468679in}{1.551565in}}{\pgfqpoint{2.473070in}{1.562164in}}{\pgfqpoint{2.473070in}{1.573214in}}%
\pgfpathcurveto{\pgfqpoint{2.473070in}{1.584264in}}{\pgfqpoint{2.468679in}{1.594863in}}{\pgfqpoint{2.460866in}{1.602677in}}%
\pgfpathcurveto{\pgfqpoint{2.453052in}{1.610490in}}{\pgfqpoint{2.442453in}{1.614881in}}{\pgfqpoint{2.431403in}{1.614881in}}%
\pgfpathcurveto{\pgfqpoint{2.420353in}{1.614881in}}{\pgfqpoint{2.409754in}{1.610490in}}{\pgfqpoint{2.401940in}{1.602677in}}%
\pgfpathcurveto{\pgfqpoint{2.394127in}{1.594863in}}{\pgfqpoint{2.389736in}{1.584264in}}{\pgfqpoint{2.389736in}{1.573214in}}%
\pgfpathcurveto{\pgfqpoint{2.389736in}{1.562164in}}{\pgfqpoint{2.394127in}{1.551565in}}{\pgfqpoint{2.401940in}{1.543751in}}%
\pgfpathcurveto{\pgfqpoint{2.409754in}{1.535938in}}{\pgfqpoint{2.420353in}{1.531547in}}{\pgfqpoint{2.431403in}{1.531547in}}%
\pgfpathclose%
\pgfusepath{stroke,fill}%
\end{pgfscope}%
\begin{pgfscope}%
\pgfpathrectangle{\pgfqpoint{0.787074in}{0.548769in}}{\pgfqpoint{5.062926in}{3.102590in}}%
\pgfusepath{clip}%
\pgfsetbuttcap%
\pgfsetroundjoin%
\definecolor{currentfill}{rgb}{1.000000,0.498039,0.054902}%
\pgfsetfillcolor{currentfill}%
\pgfsetlinewidth{1.003750pt}%
\definecolor{currentstroke}{rgb}{1.000000,0.498039,0.054902}%
\pgfsetstrokecolor{currentstroke}%
\pgfsetdash{}{0pt}%
\pgfpathmoveto{\pgfqpoint{2.270313in}{2.472823in}}%
\pgfpathcurveto{\pgfqpoint{2.281364in}{2.472823in}}{\pgfqpoint{2.291963in}{2.477213in}}{\pgfqpoint{2.299776in}{2.485027in}}%
\pgfpathcurveto{\pgfqpoint{2.307590in}{2.492840in}}{\pgfqpoint{2.311980in}{2.503439in}}{\pgfqpoint{2.311980in}{2.514489in}}%
\pgfpathcurveto{\pgfqpoint{2.311980in}{2.525539in}}{\pgfqpoint{2.307590in}{2.536138in}}{\pgfqpoint{2.299776in}{2.543952in}}%
\pgfpathcurveto{\pgfqpoint{2.291963in}{2.551766in}}{\pgfqpoint{2.281364in}{2.556156in}}{\pgfqpoint{2.270313in}{2.556156in}}%
\pgfpathcurveto{\pgfqpoint{2.259263in}{2.556156in}}{\pgfqpoint{2.248664in}{2.551766in}}{\pgfqpoint{2.240851in}{2.543952in}}%
\pgfpathcurveto{\pgfqpoint{2.233037in}{2.536138in}}{\pgfqpoint{2.228647in}{2.525539in}}{\pgfqpoint{2.228647in}{2.514489in}}%
\pgfpathcurveto{\pgfqpoint{2.228647in}{2.503439in}}{\pgfqpoint{2.233037in}{2.492840in}}{\pgfqpoint{2.240851in}{2.485027in}}%
\pgfpathcurveto{\pgfqpoint{2.248664in}{2.477213in}}{\pgfqpoint{2.259263in}{2.472823in}}{\pgfqpoint{2.270313in}{2.472823in}}%
\pgfpathclose%
\pgfusepath{stroke,fill}%
\end{pgfscope}%
\begin{pgfscope}%
\pgfpathrectangle{\pgfqpoint{0.787074in}{0.548769in}}{\pgfqpoint{5.062926in}{3.102590in}}%
\pgfusepath{clip}%
\pgfsetbuttcap%
\pgfsetroundjoin%
\definecolor{currentfill}{rgb}{0.121569,0.466667,0.705882}%
\pgfsetfillcolor{currentfill}%
\pgfsetlinewidth{1.003750pt}%
\definecolor{currentstroke}{rgb}{0.121569,0.466667,0.705882}%
\pgfsetstrokecolor{currentstroke}%
\pgfsetdash{}{0pt}%
\pgfpathmoveto{\pgfqpoint{2.059148in}{2.690255in}}%
\pgfpathcurveto{\pgfqpoint{2.070198in}{2.690255in}}{\pgfqpoint{2.080797in}{2.694645in}}{\pgfqpoint{2.088610in}{2.702459in}}%
\pgfpathcurveto{\pgfqpoint{2.096424in}{2.710273in}}{\pgfqpoint{2.100814in}{2.720872in}}{\pgfqpoint{2.100814in}{2.731922in}}%
\pgfpathcurveto{\pgfqpoint{2.100814in}{2.742972in}}{\pgfqpoint{2.096424in}{2.753571in}}{\pgfqpoint{2.088610in}{2.761384in}}%
\pgfpathcurveto{\pgfqpoint{2.080797in}{2.769198in}}{\pgfqpoint{2.070198in}{2.773588in}}{\pgfqpoint{2.059148in}{2.773588in}}%
\pgfpathcurveto{\pgfqpoint{2.048097in}{2.773588in}}{\pgfqpoint{2.037498in}{2.769198in}}{\pgfqpoint{2.029685in}{2.761384in}}%
\pgfpathcurveto{\pgfqpoint{2.021871in}{2.753571in}}{\pgfqpoint{2.017481in}{2.742972in}}{\pgfqpoint{2.017481in}{2.731922in}}%
\pgfpathcurveto{\pgfqpoint{2.017481in}{2.720872in}}{\pgfqpoint{2.021871in}{2.710273in}}{\pgfqpoint{2.029685in}{2.702459in}}%
\pgfpathcurveto{\pgfqpoint{2.037498in}{2.694645in}}{\pgfqpoint{2.048097in}{2.690255in}}{\pgfqpoint{2.059148in}{2.690255in}}%
\pgfpathclose%
\pgfusepath{stroke,fill}%
\end{pgfscope}%
\begin{pgfscope}%
\pgfpathrectangle{\pgfqpoint{0.787074in}{0.548769in}}{\pgfqpoint{5.062926in}{3.102590in}}%
\pgfusepath{clip}%
\pgfsetbuttcap%
\pgfsetroundjoin%
\definecolor{currentfill}{rgb}{1.000000,0.498039,0.054902}%
\pgfsetfillcolor{currentfill}%
\pgfsetlinewidth{1.003750pt}%
\definecolor{currentstroke}{rgb}{1.000000,0.498039,0.054902}%
\pgfsetstrokecolor{currentstroke}%
\pgfsetdash{}{0pt}%
\pgfpathmoveto{\pgfqpoint{1.806791in}{3.301211in}}%
\pgfpathcurveto{\pgfqpoint{1.817842in}{3.301211in}}{\pgfqpoint{1.828441in}{3.305601in}}{\pgfqpoint{1.836254in}{3.313415in}}%
\pgfpathcurveto{\pgfqpoint{1.844068in}{3.321228in}}{\pgfqpoint{1.848458in}{3.331828in}}{\pgfqpoint{1.848458in}{3.342878in}}%
\pgfpathcurveto{\pgfqpoint{1.848458in}{3.353928in}}{\pgfqpoint{1.844068in}{3.364527in}}{\pgfqpoint{1.836254in}{3.372340in}}%
\pgfpathcurveto{\pgfqpoint{1.828441in}{3.380154in}}{\pgfqpoint{1.817842in}{3.384544in}}{\pgfqpoint{1.806791in}{3.384544in}}%
\pgfpathcurveto{\pgfqpoint{1.795741in}{3.384544in}}{\pgfqpoint{1.785142in}{3.380154in}}{\pgfqpoint{1.777329in}{3.372340in}}%
\pgfpathcurveto{\pgfqpoint{1.769515in}{3.364527in}}{\pgfqpoint{1.765125in}{3.353928in}}{\pgfqpoint{1.765125in}{3.342878in}}%
\pgfpathcurveto{\pgfqpoint{1.765125in}{3.331828in}}{\pgfqpoint{1.769515in}{3.321228in}}{\pgfqpoint{1.777329in}{3.313415in}}%
\pgfpathcurveto{\pgfqpoint{1.785142in}{3.305601in}}{\pgfqpoint{1.795741in}{3.301211in}}{\pgfqpoint{1.806791in}{3.301211in}}%
\pgfpathclose%
\pgfusepath{stroke,fill}%
\end{pgfscope}%
\begin{pgfscope}%
\pgfpathrectangle{\pgfqpoint{0.787074in}{0.548769in}}{\pgfqpoint{5.062926in}{3.102590in}}%
\pgfusepath{clip}%
\pgfsetbuttcap%
\pgfsetroundjoin%
\definecolor{currentfill}{rgb}{0.121569,0.466667,0.705882}%
\pgfsetfillcolor{currentfill}%
\pgfsetlinewidth{1.003750pt}%
\definecolor{currentstroke}{rgb}{0.121569,0.466667,0.705882}%
\pgfsetstrokecolor{currentstroke}%
\pgfsetdash{}{0pt}%
\pgfpathmoveto{\pgfqpoint{1.909776in}{3.101406in}}%
\pgfpathcurveto{\pgfqpoint{1.920826in}{3.101406in}}{\pgfqpoint{1.931425in}{3.105796in}}{\pgfqpoint{1.939238in}{3.113610in}}%
\pgfpathcurveto{\pgfqpoint{1.947052in}{3.121423in}}{\pgfqpoint{1.951442in}{3.132022in}}{\pgfqpoint{1.951442in}{3.143073in}}%
\pgfpathcurveto{\pgfqpoint{1.951442in}{3.154123in}}{\pgfqpoint{1.947052in}{3.164722in}}{\pgfqpoint{1.939238in}{3.172535in}}%
\pgfpathcurveto{\pgfqpoint{1.931425in}{3.180349in}}{\pgfqpoint{1.920826in}{3.184739in}}{\pgfqpoint{1.909776in}{3.184739in}}%
\pgfpathcurveto{\pgfqpoint{1.898725in}{3.184739in}}{\pgfqpoint{1.888126in}{3.180349in}}{\pgfqpoint{1.880313in}{3.172535in}}%
\pgfpathcurveto{\pgfqpoint{1.872499in}{3.164722in}}{\pgfqpoint{1.868109in}{3.154123in}}{\pgfqpoint{1.868109in}{3.143073in}}%
\pgfpathcurveto{\pgfqpoint{1.868109in}{3.132022in}}{\pgfqpoint{1.872499in}{3.121423in}}{\pgfqpoint{1.880313in}{3.113610in}}%
\pgfpathcurveto{\pgfqpoint{1.888126in}{3.105796in}}{\pgfqpoint{1.898725in}{3.101406in}}{\pgfqpoint{1.909776in}{3.101406in}}%
\pgfpathclose%
\pgfusepath{stroke,fill}%
\end{pgfscope}%
\begin{pgfscope}%
\pgfpathrectangle{\pgfqpoint{0.787074in}{0.548769in}}{\pgfqpoint{5.062926in}{3.102590in}}%
\pgfusepath{clip}%
\pgfsetbuttcap%
\pgfsetroundjoin%
\definecolor{currentfill}{rgb}{1.000000,0.498039,0.054902}%
\pgfsetfillcolor{currentfill}%
\pgfsetlinewidth{1.003750pt}%
\definecolor{currentstroke}{rgb}{1.000000,0.498039,0.054902}%
\pgfsetstrokecolor{currentstroke}%
\pgfsetdash{}{0pt}%
\pgfpathmoveto{\pgfqpoint{1.735595in}{1.878073in}}%
\pgfpathcurveto{\pgfqpoint{1.746645in}{1.878073in}}{\pgfqpoint{1.757244in}{1.882464in}}{\pgfqpoint{1.765057in}{1.890277in}}%
\pgfpathcurveto{\pgfqpoint{1.772871in}{1.898091in}}{\pgfqpoint{1.777261in}{1.908690in}}{\pgfqpoint{1.777261in}{1.919740in}}%
\pgfpathcurveto{\pgfqpoint{1.777261in}{1.930790in}}{\pgfqpoint{1.772871in}{1.941389in}}{\pgfqpoint{1.765057in}{1.949203in}}%
\pgfpathcurveto{\pgfqpoint{1.757244in}{1.957016in}}{\pgfqpoint{1.746645in}{1.961407in}}{\pgfqpoint{1.735595in}{1.961407in}}%
\pgfpathcurveto{\pgfqpoint{1.724544in}{1.961407in}}{\pgfqpoint{1.713945in}{1.957016in}}{\pgfqpoint{1.706132in}{1.949203in}}%
\pgfpathcurveto{\pgfqpoint{1.698318in}{1.941389in}}{\pgfqpoint{1.693928in}{1.930790in}}{\pgfqpoint{1.693928in}{1.919740in}}%
\pgfpathcurveto{\pgfqpoint{1.693928in}{1.908690in}}{\pgfqpoint{1.698318in}{1.898091in}}{\pgfqpoint{1.706132in}{1.890277in}}%
\pgfpathcurveto{\pgfqpoint{1.713945in}{1.882464in}}{\pgfqpoint{1.724544in}{1.878073in}}{\pgfqpoint{1.735595in}{1.878073in}}%
\pgfpathclose%
\pgfusepath{stroke,fill}%
\end{pgfscope}%
\begin{pgfscope}%
\pgfpathrectangle{\pgfqpoint{0.787074in}{0.548769in}}{\pgfqpoint{5.062926in}{3.102590in}}%
\pgfusepath{clip}%
\pgfsetbuttcap%
\pgfsetroundjoin%
\definecolor{currentfill}{rgb}{1.000000,0.498039,0.054902}%
\pgfsetfillcolor{currentfill}%
\pgfsetlinewidth{1.003750pt}%
\definecolor{currentstroke}{rgb}{1.000000,0.498039,0.054902}%
\pgfsetstrokecolor{currentstroke}%
\pgfsetdash{}{0pt}%
\pgfpathmoveto{\pgfqpoint{2.670049in}{2.761817in}}%
\pgfpathcurveto{\pgfqpoint{2.681099in}{2.761817in}}{\pgfqpoint{2.691698in}{2.766208in}}{\pgfqpoint{2.699512in}{2.774021in}}%
\pgfpathcurveto{\pgfqpoint{2.707325in}{2.781835in}}{\pgfqpoint{2.711716in}{2.792434in}}{\pgfqpoint{2.711716in}{2.803484in}}%
\pgfpathcurveto{\pgfqpoint{2.711716in}{2.814534in}}{\pgfqpoint{2.707325in}{2.825133in}}{\pgfqpoint{2.699512in}{2.832947in}}%
\pgfpathcurveto{\pgfqpoint{2.691698in}{2.840760in}}{\pgfqpoint{2.681099in}{2.845151in}}{\pgfqpoint{2.670049in}{2.845151in}}%
\pgfpathcurveto{\pgfqpoint{2.658999in}{2.845151in}}{\pgfqpoint{2.648400in}{2.840760in}}{\pgfqpoint{2.640586in}{2.832947in}}%
\pgfpathcurveto{\pgfqpoint{2.632772in}{2.825133in}}{\pgfqpoint{2.628382in}{2.814534in}}{\pgfqpoint{2.628382in}{2.803484in}}%
\pgfpathcurveto{\pgfqpoint{2.628382in}{2.792434in}}{\pgfqpoint{2.632772in}{2.781835in}}{\pgfqpoint{2.640586in}{2.774021in}}%
\pgfpathcurveto{\pgfqpoint{2.648400in}{2.766208in}}{\pgfqpoint{2.658999in}{2.761817in}}{\pgfqpoint{2.670049in}{2.761817in}}%
\pgfpathclose%
\pgfusepath{stroke,fill}%
\end{pgfscope}%
\begin{pgfscope}%
\pgfpathrectangle{\pgfqpoint{0.787074in}{0.548769in}}{\pgfqpoint{5.062926in}{3.102590in}}%
\pgfusepath{clip}%
\pgfsetbuttcap%
\pgfsetroundjoin%
\definecolor{currentfill}{rgb}{1.000000,0.498039,0.054902}%
\pgfsetfillcolor{currentfill}%
\pgfsetlinewidth{1.003750pt}%
\definecolor{currentstroke}{rgb}{1.000000,0.498039,0.054902}%
\pgfsetstrokecolor{currentstroke}%
\pgfsetdash{}{0pt}%
\pgfpathmoveto{\pgfqpoint{3.039506in}{2.075959in}}%
\pgfpathcurveto{\pgfqpoint{3.050557in}{2.075959in}}{\pgfqpoint{3.061156in}{2.080350in}}{\pgfqpoint{3.068969in}{2.088163in}}%
\pgfpathcurveto{\pgfqpoint{3.076783in}{2.095977in}}{\pgfqpoint{3.081173in}{2.106576in}}{\pgfqpoint{3.081173in}{2.117626in}}%
\pgfpathcurveto{\pgfqpoint{3.081173in}{2.128676in}}{\pgfqpoint{3.076783in}{2.139275in}}{\pgfqpoint{3.068969in}{2.147089in}}%
\pgfpathcurveto{\pgfqpoint{3.061156in}{2.154902in}}{\pgfqpoint{3.050557in}{2.159293in}}{\pgfqpoint{3.039506in}{2.159293in}}%
\pgfpathcurveto{\pgfqpoint{3.028456in}{2.159293in}}{\pgfqpoint{3.017857in}{2.154902in}}{\pgfqpoint{3.010044in}{2.147089in}}%
\pgfpathcurveto{\pgfqpoint{3.002230in}{2.139275in}}{\pgfqpoint{2.997840in}{2.128676in}}{\pgfqpoint{2.997840in}{2.117626in}}%
\pgfpathcurveto{\pgfqpoint{2.997840in}{2.106576in}}{\pgfqpoint{3.002230in}{2.095977in}}{\pgfqpoint{3.010044in}{2.088163in}}%
\pgfpathcurveto{\pgfqpoint{3.017857in}{2.080350in}}{\pgfqpoint{3.028456in}{2.075959in}}{\pgfqpoint{3.039506in}{2.075959in}}%
\pgfpathclose%
\pgfusepath{stroke,fill}%
\end{pgfscope}%
\begin{pgfscope}%
\pgfpathrectangle{\pgfqpoint{0.787074in}{0.548769in}}{\pgfqpoint{5.062926in}{3.102590in}}%
\pgfusepath{clip}%
\pgfsetbuttcap%
\pgfsetroundjoin%
\definecolor{currentfill}{rgb}{1.000000,0.498039,0.054902}%
\pgfsetfillcolor{currentfill}%
\pgfsetlinewidth{1.003750pt}%
\definecolor{currentstroke}{rgb}{1.000000,0.498039,0.054902}%
\pgfsetstrokecolor{currentstroke}%
\pgfsetdash{}{0pt}%
\pgfpathmoveto{\pgfqpoint{2.377452in}{2.369584in}}%
\pgfpathcurveto{\pgfqpoint{2.388502in}{2.369584in}}{\pgfqpoint{2.399101in}{2.373975in}}{\pgfqpoint{2.406915in}{2.381788in}}%
\pgfpathcurveto{\pgfqpoint{2.414729in}{2.389602in}}{\pgfqpoint{2.419119in}{2.400201in}}{\pgfqpoint{2.419119in}{2.411251in}}%
\pgfpathcurveto{\pgfqpoint{2.419119in}{2.422301in}}{\pgfqpoint{2.414729in}{2.432900in}}{\pgfqpoint{2.406915in}{2.440714in}}%
\pgfpathcurveto{\pgfqpoint{2.399101in}{2.448527in}}{\pgfqpoint{2.388502in}{2.452918in}}{\pgfqpoint{2.377452in}{2.452918in}}%
\pgfpathcurveto{\pgfqpoint{2.366402in}{2.452918in}}{\pgfqpoint{2.355803in}{2.448527in}}{\pgfqpoint{2.347989in}{2.440714in}}%
\pgfpathcurveto{\pgfqpoint{2.340176in}{2.432900in}}{\pgfqpoint{2.335785in}{2.422301in}}{\pgfqpoint{2.335785in}{2.411251in}}%
\pgfpathcurveto{\pgfqpoint{2.335785in}{2.400201in}}{\pgfqpoint{2.340176in}{2.389602in}}{\pgfqpoint{2.347989in}{2.381788in}}%
\pgfpathcurveto{\pgfqpoint{2.355803in}{2.373975in}}{\pgfqpoint{2.366402in}{2.369584in}}{\pgfqpoint{2.377452in}{2.369584in}}%
\pgfpathclose%
\pgfusepath{stroke,fill}%
\end{pgfscope}%
\begin{pgfscope}%
\pgfpathrectangle{\pgfqpoint{0.787074in}{0.548769in}}{\pgfqpoint{5.062926in}{3.102590in}}%
\pgfusepath{clip}%
\pgfsetbuttcap%
\pgfsetroundjoin%
\definecolor{currentfill}{rgb}{0.121569,0.466667,0.705882}%
\pgfsetfillcolor{currentfill}%
\pgfsetlinewidth{1.003750pt}%
\definecolor{currentstroke}{rgb}{0.121569,0.466667,0.705882}%
\pgfsetstrokecolor{currentstroke}%
\pgfsetdash{}{0pt}%
\pgfpathmoveto{\pgfqpoint{1.716517in}{0.648131in}}%
\pgfpathcurveto{\pgfqpoint{1.727567in}{0.648131in}}{\pgfqpoint{1.738166in}{0.652521in}}{\pgfqpoint{1.745980in}{0.660335in}}%
\pgfpathcurveto{\pgfqpoint{1.753794in}{0.668148in}}{\pgfqpoint{1.758184in}{0.678747in}}{\pgfqpoint{1.758184in}{0.689798in}}%
\pgfpathcurveto{\pgfqpoint{1.758184in}{0.700848in}}{\pgfqpoint{1.753794in}{0.711447in}}{\pgfqpoint{1.745980in}{0.719260in}}%
\pgfpathcurveto{\pgfqpoint{1.738166in}{0.727074in}}{\pgfqpoint{1.727567in}{0.731464in}}{\pgfqpoint{1.716517in}{0.731464in}}%
\pgfpathcurveto{\pgfqpoint{1.705467in}{0.731464in}}{\pgfqpoint{1.694868in}{0.727074in}}{\pgfqpoint{1.687054in}{0.719260in}}%
\pgfpathcurveto{\pgfqpoint{1.679241in}{0.711447in}}{\pgfqpoint{1.674850in}{0.700848in}}{\pgfqpoint{1.674850in}{0.689798in}}%
\pgfpathcurveto{\pgfqpoint{1.674850in}{0.678747in}}{\pgfqpoint{1.679241in}{0.668148in}}{\pgfqpoint{1.687054in}{0.660335in}}%
\pgfpathcurveto{\pgfqpoint{1.694868in}{0.652521in}}{\pgfqpoint{1.705467in}{0.648131in}}{\pgfqpoint{1.716517in}{0.648131in}}%
\pgfpathclose%
\pgfusepath{stroke,fill}%
\end{pgfscope}%
\begin{pgfscope}%
\pgfpathrectangle{\pgfqpoint{0.787074in}{0.548769in}}{\pgfqpoint{5.062926in}{3.102590in}}%
\pgfusepath{clip}%
\pgfsetbuttcap%
\pgfsetroundjoin%
\definecolor{currentfill}{rgb}{1.000000,0.498039,0.054902}%
\pgfsetfillcolor{currentfill}%
\pgfsetlinewidth{1.003750pt}%
\definecolor{currentstroke}{rgb}{1.000000,0.498039,0.054902}%
\pgfsetstrokecolor{currentstroke}%
\pgfsetdash{}{0pt}%
\pgfpathmoveto{\pgfqpoint{2.230615in}{1.884932in}}%
\pgfpathcurveto{\pgfqpoint{2.241666in}{1.884932in}}{\pgfqpoint{2.252265in}{1.889322in}}{\pgfqpoint{2.260078in}{1.897136in}}%
\pgfpathcurveto{\pgfqpoint{2.267892in}{1.904950in}}{\pgfqpoint{2.272282in}{1.915549in}}{\pgfqpoint{2.272282in}{1.926599in}}%
\pgfpathcurveto{\pgfqpoint{2.272282in}{1.937649in}}{\pgfqpoint{2.267892in}{1.948248in}}{\pgfqpoint{2.260078in}{1.956061in}}%
\pgfpathcurveto{\pgfqpoint{2.252265in}{1.963875in}}{\pgfqpoint{2.241666in}{1.968265in}}{\pgfqpoint{2.230615in}{1.968265in}}%
\pgfpathcurveto{\pgfqpoint{2.219565in}{1.968265in}}{\pgfqpoint{2.208966in}{1.963875in}}{\pgfqpoint{2.201153in}{1.956061in}}%
\pgfpathcurveto{\pgfqpoint{2.193339in}{1.948248in}}{\pgfqpoint{2.188949in}{1.937649in}}{\pgfqpoint{2.188949in}{1.926599in}}%
\pgfpathcurveto{\pgfqpoint{2.188949in}{1.915549in}}{\pgfqpoint{2.193339in}{1.904950in}}{\pgfqpoint{2.201153in}{1.897136in}}%
\pgfpathcurveto{\pgfqpoint{2.208966in}{1.889322in}}{\pgfqpoint{2.219565in}{1.884932in}}{\pgfqpoint{2.230615in}{1.884932in}}%
\pgfpathclose%
\pgfusepath{stroke,fill}%
\end{pgfscope}%
\begin{pgfscope}%
\pgfpathrectangle{\pgfqpoint{0.787074in}{0.548769in}}{\pgfqpoint{5.062926in}{3.102590in}}%
\pgfusepath{clip}%
\pgfsetbuttcap%
\pgfsetroundjoin%
\definecolor{currentfill}{rgb}{1.000000,0.498039,0.054902}%
\pgfsetfillcolor{currentfill}%
\pgfsetlinewidth{1.003750pt}%
\definecolor{currentstroke}{rgb}{1.000000,0.498039,0.054902}%
\pgfsetstrokecolor{currentstroke}%
\pgfsetdash{}{0pt}%
\pgfpathmoveto{\pgfqpoint{1.627574in}{1.779524in}}%
\pgfpathcurveto{\pgfqpoint{1.638624in}{1.779524in}}{\pgfqpoint{1.649223in}{1.783914in}}{\pgfqpoint{1.657037in}{1.791728in}}%
\pgfpathcurveto{\pgfqpoint{1.664850in}{1.799541in}}{\pgfqpoint{1.669241in}{1.810140in}}{\pgfqpoint{1.669241in}{1.821190in}}%
\pgfpathcurveto{\pgfqpoint{1.669241in}{1.832241in}}{\pgfqpoint{1.664850in}{1.842840in}}{\pgfqpoint{1.657037in}{1.850653in}}%
\pgfpathcurveto{\pgfqpoint{1.649223in}{1.858467in}}{\pgfqpoint{1.638624in}{1.862857in}}{\pgfqpoint{1.627574in}{1.862857in}}%
\pgfpathcurveto{\pgfqpoint{1.616524in}{1.862857in}}{\pgfqpoint{1.605925in}{1.858467in}}{\pgfqpoint{1.598111in}{1.850653in}}%
\pgfpathcurveto{\pgfqpoint{1.590298in}{1.842840in}}{\pgfqpoint{1.585907in}{1.832241in}}{\pgfqpoint{1.585907in}{1.821190in}}%
\pgfpathcurveto{\pgfqpoint{1.585907in}{1.810140in}}{\pgfqpoint{1.590298in}{1.799541in}}{\pgfqpoint{1.598111in}{1.791728in}}%
\pgfpathcurveto{\pgfqpoint{1.605925in}{1.783914in}}{\pgfqpoint{1.616524in}{1.779524in}}{\pgfqpoint{1.627574in}{1.779524in}}%
\pgfpathclose%
\pgfusepath{stroke,fill}%
\end{pgfscope}%
\begin{pgfscope}%
\pgfpathrectangle{\pgfqpoint{0.787074in}{0.548769in}}{\pgfqpoint{5.062926in}{3.102590in}}%
\pgfusepath{clip}%
\pgfsetbuttcap%
\pgfsetroundjoin%
\definecolor{currentfill}{rgb}{1.000000,0.498039,0.054902}%
\pgfsetfillcolor{currentfill}%
\pgfsetlinewidth{1.003750pt}%
\definecolor{currentstroke}{rgb}{1.000000,0.498039,0.054902}%
\pgfsetstrokecolor{currentstroke}%
\pgfsetdash{}{0pt}%
\pgfpathmoveto{\pgfqpoint{1.860700in}{2.435775in}}%
\pgfpathcurveto{\pgfqpoint{1.871750in}{2.435775in}}{\pgfqpoint{1.882349in}{2.440165in}}{\pgfqpoint{1.890163in}{2.447979in}}%
\pgfpathcurveto{\pgfqpoint{1.897976in}{2.455792in}}{\pgfqpoint{1.902367in}{2.466391in}}{\pgfqpoint{1.902367in}{2.477441in}}%
\pgfpathcurveto{\pgfqpoint{1.902367in}{2.488492in}}{\pgfqpoint{1.897976in}{2.499091in}}{\pgfqpoint{1.890163in}{2.506904in}}%
\pgfpathcurveto{\pgfqpoint{1.882349in}{2.514718in}}{\pgfqpoint{1.871750in}{2.519108in}}{\pgfqpoint{1.860700in}{2.519108in}}%
\pgfpathcurveto{\pgfqpoint{1.849650in}{2.519108in}}{\pgfqpoint{1.839051in}{2.514718in}}{\pgfqpoint{1.831237in}{2.506904in}}%
\pgfpathcurveto{\pgfqpoint{1.823424in}{2.499091in}}{\pgfqpoint{1.819033in}{2.488492in}}{\pgfqpoint{1.819033in}{2.477441in}}%
\pgfpathcurveto{\pgfqpoint{1.819033in}{2.466391in}}{\pgfqpoint{1.823424in}{2.455792in}}{\pgfqpoint{1.831237in}{2.447979in}}%
\pgfpathcurveto{\pgfqpoint{1.839051in}{2.440165in}}{\pgfqpoint{1.849650in}{2.435775in}}{\pgfqpoint{1.860700in}{2.435775in}}%
\pgfpathclose%
\pgfusepath{stroke,fill}%
\end{pgfscope}%
\begin{pgfscope}%
\pgfpathrectangle{\pgfqpoint{0.787074in}{0.548769in}}{\pgfqpoint{5.062926in}{3.102590in}}%
\pgfusepath{clip}%
\pgfsetbuttcap%
\pgfsetroundjoin%
\definecolor{currentfill}{rgb}{1.000000,0.498039,0.054902}%
\pgfsetfillcolor{currentfill}%
\pgfsetlinewidth{1.003750pt}%
\definecolor{currentstroke}{rgb}{1.000000,0.498039,0.054902}%
\pgfsetstrokecolor{currentstroke}%
\pgfsetdash{}{0pt}%
\pgfpathmoveto{\pgfqpoint{2.083414in}{1.639934in}}%
\pgfpathcurveto{\pgfqpoint{2.094464in}{1.639934in}}{\pgfqpoint{2.105063in}{1.644324in}}{\pgfqpoint{2.112877in}{1.652138in}}%
\pgfpathcurveto{\pgfqpoint{2.120690in}{1.659951in}}{\pgfqpoint{2.125081in}{1.670550in}}{\pgfqpoint{2.125081in}{1.681600in}}%
\pgfpathcurveto{\pgfqpoint{2.125081in}{1.692650in}}{\pgfqpoint{2.120690in}{1.703249in}}{\pgfqpoint{2.112877in}{1.711063in}}%
\pgfpathcurveto{\pgfqpoint{2.105063in}{1.718877in}}{\pgfqpoint{2.094464in}{1.723267in}}{\pgfqpoint{2.083414in}{1.723267in}}%
\pgfpathcurveto{\pgfqpoint{2.072364in}{1.723267in}}{\pgfqpoint{2.061765in}{1.718877in}}{\pgfqpoint{2.053951in}{1.711063in}}%
\pgfpathcurveto{\pgfqpoint{2.046138in}{1.703249in}}{\pgfqpoint{2.041747in}{1.692650in}}{\pgfqpoint{2.041747in}{1.681600in}}%
\pgfpathcurveto{\pgfqpoint{2.041747in}{1.670550in}}{\pgfqpoint{2.046138in}{1.659951in}}{\pgfqpoint{2.053951in}{1.652138in}}%
\pgfpathcurveto{\pgfqpoint{2.061765in}{1.644324in}}{\pgfqpoint{2.072364in}{1.639934in}}{\pgfqpoint{2.083414in}{1.639934in}}%
\pgfpathclose%
\pgfusepath{stroke,fill}%
\end{pgfscope}%
\begin{pgfscope}%
\pgfpathrectangle{\pgfqpoint{0.787074in}{0.548769in}}{\pgfqpoint{5.062926in}{3.102590in}}%
\pgfusepath{clip}%
\pgfsetbuttcap%
\pgfsetroundjoin%
\definecolor{currentfill}{rgb}{0.121569,0.466667,0.705882}%
\pgfsetfillcolor{currentfill}%
\pgfsetlinewidth{1.003750pt}%
\definecolor{currentstroke}{rgb}{0.121569,0.466667,0.705882}%
\pgfsetstrokecolor{currentstroke}%
\pgfsetdash{}{0pt}%
\pgfpathmoveto{\pgfqpoint{1.875784in}{1.770785in}}%
\pgfpathcurveto{\pgfqpoint{1.886834in}{1.770785in}}{\pgfqpoint{1.897433in}{1.775176in}}{\pgfqpoint{1.905247in}{1.782989in}}%
\pgfpathcurveto{\pgfqpoint{1.913060in}{1.790803in}}{\pgfqpoint{1.917451in}{1.801402in}}{\pgfqpoint{1.917451in}{1.812452in}}%
\pgfpathcurveto{\pgfqpoint{1.917451in}{1.823502in}}{\pgfqpoint{1.913060in}{1.834101in}}{\pgfqpoint{1.905247in}{1.841915in}}%
\pgfpathcurveto{\pgfqpoint{1.897433in}{1.849728in}}{\pgfqpoint{1.886834in}{1.854119in}}{\pgfqpoint{1.875784in}{1.854119in}}%
\pgfpathcurveto{\pgfqpoint{1.864734in}{1.854119in}}{\pgfqpoint{1.854135in}{1.849728in}}{\pgfqpoint{1.846321in}{1.841915in}}%
\pgfpathcurveto{\pgfqpoint{1.838507in}{1.834101in}}{\pgfqpoint{1.834117in}{1.823502in}}{\pgfqpoint{1.834117in}{1.812452in}}%
\pgfpathcurveto{\pgfqpoint{1.834117in}{1.801402in}}{\pgfqpoint{1.838507in}{1.790803in}}{\pgfqpoint{1.846321in}{1.782989in}}%
\pgfpathcurveto{\pgfqpoint{1.854135in}{1.775176in}}{\pgfqpoint{1.864734in}{1.770785in}}{\pgfqpoint{1.875784in}{1.770785in}}%
\pgfpathclose%
\pgfusepath{stroke,fill}%
\end{pgfscope}%
\begin{pgfscope}%
\pgfpathrectangle{\pgfqpoint{0.787074in}{0.548769in}}{\pgfqpoint{5.062926in}{3.102590in}}%
\pgfusepath{clip}%
\pgfsetbuttcap%
\pgfsetroundjoin%
\definecolor{currentfill}{rgb}{1.000000,0.498039,0.054902}%
\pgfsetfillcolor{currentfill}%
\pgfsetlinewidth{1.003750pt}%
\definecolor{currentstroke}{rgb}{1.000000,0.498039,0.054902}%
\pgfsetstrokecolor{currentstroke}%
\pgfsetdash{}{0pt}%
\pgfpathmoveto{\pgfqpoint{1.709191in}{2.028373in}}%
\pgfpathcurveto{\pgfqpoint{1.720242in}{2.028373in}}{\pgfqpoint{1.730841in}{2.032763in}}{\pgfqpoint{1.738654in}{2.040577in}}%
\pgfpathcurveto{\pgfqpoint{1.746468in}{2.048390in}}{\pgfqpoint{1.750858in}{2.058989in}}{\pgfqpoint{1.750858in}{2.070039in}}%
\pgfpathcurveto{\pgfqpoint{1.750858in}{2.081090in}}{\pgfqpoint{1.746468in}{2.091689in}}{\pgfqpoint{1.738654in}{2.099502in}}%
\pgfpathcurveto{\pgfqpoint{1.730841in}{2.107316in}}{\pgfqpoint{1.720242in}{2.111706in}}{\pgfqpoint{1.709191in}{2.111706in}}%
\pgfpathcurveto{\pgfqpoint{1.698141in}{2.111706in}}{\pgfqpoint{1.687542in}{2.107316in}}{\pgfqpoint{1.679729in}{2.099502in}}%
\pgfpathcurveto{\pgfqpoint{1.671915in}{2.091689in}}{\pgfqpoint{1.667525in}{2.081090in}}{\pgfqpoint{1.667525in}{2.070039in}}%
\pgfpathcurveto{\pgfqpoint{1.667525in}{2.058989in}}{\pgfqpoint{1.671915in}{2.048390in}}{\pgfqpoint{1.679729in}{2.040577in}}%
\pgfpathcurveto{\pgfqpoint{1.687542in}{2.032763in}}{\pgfqpoint{1.698141in}{2.028373in}}{\pgfqpoint{1.709191in}{2.028373in}}%
\pgfpathclose%
\pgfusepath{stroke,fill}%
\end{pgfscope}%
\begin{pgfscope}%
\pgfpathrectangle{\pgfqpoint{0.787074in}{0.548769in}}{\pgfqpoint{5.062926in}{3.102590in}}%
\pgfusepath{clip}%
\pgfsetbuttcap%
\pgfsetroundjoin%
\definecolor{currentfill}{rgb}{1.000000,0.498039,0.054902}%
\pgfsetfillcolor{currentfill}%
\pgfsetlinewidth{1.003750pt}%
\definecolor{currentstroke}{rgb}{1.000000,0.498039,0.054902}%
\pgfsetstrokecolor{currentstroke}%
\pgfsetdash{}{0pt}%
\pgfpathmoveto{\pgfqpoint{2.676696in}{2.904196in}}%
\pgfpathcurveto{\pgfqpoint{2.687746in}{2.904196in}}{\pgfqpoint{2.698345in}{2.908586in}}{\pgfqpoint{2.706159in}{2.916399in}}%
\pgfpathcurveto{\pgfqpoint{2.713973in}{2.924213in}}{\pgfqpoint{2.718363in}{2.934812in}}{\pgfqpoint{2.718363in}{2.945862in}}%
\pgfpathcurveto{\pgfqpoint{2.718363in}{2.956912in}}{\pgfqpoint{2.713973in}{2.967511in}}{\pgfqpoint{2.706159in}{2.975325in}}%
\pgfpathcurveto{\pgfqpoint{2.698345in}{2.983139in}}{\pgfqpoint{2.687746in}{2.987529in}}{\pgfqpoint{2.676696in}{2.987529in}}%
\pgfpathcurveto{\pgfqpoint{2.665646in}{2.987529in}}{\pgfqpoint{2.655047in}{2.983139in}}{\pgfqpoint{2.647234in}{2.975325in}}%
\pgfpathcurveto{\pgfqpoint{2.639420in}{2.967511in}}{\pgfqpoint{2.635030in}{2.956912in}}{\pgfqpoint{2.635030in}{2.945862in}}%
\pgfpathcurveto{\pgfqpoint{2.635030in}{2.934812in}}{\pgfqpoint{2.639420in}{2.924213in}}{\pgfqpoint{2.647234in}{2.916399in}}%
\pgfpathcurveto{\pgfqpoint{2.655047in}{2.908586in}}{\pgfqpoint{2.665646in}{2.904196in}}{\pgfqpoint{2.676696in}{2.904196in}}%
\pgfpathclose%
\pgfusepath{stroke,fill}%
\end{pgfscope}%
\begin{pgfscope}%
\pgfpathrectangle{\pgfqpoint{0.787074in}{0.548769in}}{\pgfqpoint{5.062926in}{3.102590in}}%
\pgfusepath{clip}%
\pgfsetbuttcap%
\pgfsetroundjoin%
\definecolor{currentfill}{rgb}{1.000000,0.498039,0.054902}%
\pgfsetfillcolor{currentfill}%
\pgfsetlinewidth{1.003750pt}%
\definecolor{currentstroke}{rgb}{1.000000,0.498039,0.054902}%
\pgfsetstrokecolor{currentstroke}%
\pgfsetdash{}{0pt}%
\pgfpathmoveto{\pgfqpoint{1.848016in}{3.239789in}}%
\pgfpathcurveto{\pgfqpoint{1.859066in}{3.239789in}}{\pgfqpoint{1.869665in}{3.244179in}}{\pgfqpoint{1.877478in}{3.251992in}}%
\pgfpathcurveto{\pgfqpoint{1.885292in}{3.259806in}}{\pgfqpoint{1.889682in}{3.270405in}}{\pgfqpoint{1.889682in}{3.281455in}}%
\pgfpathcurveto{\pgfqpoint{1.889682in}{3.292505in}}{\pgfqpoint{1.885292in}{3.303104in}}{\pgfqpoint{1.877478in}{3.310918in}}%
\pgfpathcurveto{\pgfqpoint{1.869665in}{3.318732in}}{\pgfqpoint{1.859066in}{3.323122in}}{\pgfqpoint{1.848016in}{3.323122in}}%
\pgfpathcurveto{\pgfqpoint{1.836966in}{3.323122in}}{\pgfqpoint{1.826366in}{3.318732in}}{\pgfqpoint{1.818553in}{3.310918in}}%
\pgfpathcurveto{\pgfqpoint{1.810739in}{3.303104in}}{\pgfqpoint{1.806349in}{3.292505in}}{\pgfqpoint{1.806349in}{3.281455in}}%
\pgfpathcurveto{\pgfqpoint{1.806349in}{3.270405in}}{\pgfqpoint{1.810739in}{3.259806in}}{\pgfqpoint{1.818553in}{3.251992in}}%
\pgfpathcurveto{\pgfqpoint{1.826366in}{3.244179in}}{\pgfqpoint{1.836966in}{3.239789in}}{\pgfqpoint{1.848016in}{3.239789in}}%
\pgfpathclose%
\pgfusepath{stroke,fill}%
\end{pgfscope}%
\begin{pgfscope}%
\pgfsetbuttcap%
\pgfsetroundjoin%
\definecolor{currentfill}{rgb}{0.000000,0.000000,0.000000}%
\pgfsetfillcolor{currentfill}%
\pgfsetlinewidth{0.803000pt}%
\definecolor{currentstroke}{rgb}{0.000000,0.000000,0.000000}%
\pgfsetstrokecolor{currentstroke}%
\pgfsetdash{}{0pt}%
\pgfsys@defobject{currentmarker}{\pgfqpoint{0.000000in}{-0.048611in}}{\pgfqpoint{0.000000in}{0.000000in}}{%
\pgfpathmoveto{\pgfqpoint{0.000000in}{0.000000in}}%
\pgfpathlineto{\pgfqpoint{0.000000in}{-0.048611in}}%
\pgfusepath{stroke,fill}%
}%
\begin{pgfscope}%
\pgfsys@transformshift{0.981384in}{0.548769in}%
\pgfsys@useobject{currentmarker}{}%
\end{pgfscope}%
\end{pgfscope}%
\begin{pgfscope}%
\definecolor{textcolor}{rgb}{0.000000,0.000000,0.000000}%
\pgfsetstrokecolor{textcolor}%
\pgfsetfillcolor{textcolor}%
\pgftext[x=0.981384in,y=0.451547in,,top]{\color{textcolor}\sffamily\fontsize{10.000000}{12.000000}\selectfont \(\displaystyle {0}\)}%
\end{pgfscope}%
\begin{pgfscope}%
\pgfsetbuttcap%
\pgfsetroundjoin%
\definecolor{currentfill}{rgb}{0.000000,0.000000,0.000000}%
\pgfsetfillcolor{currentfill}%
\pgfsetlinewidth{0.803000pt}%
\definecolor{currentstroke}{rgb}{0.000000,0.000000,0.000000}%
\pgfsetstrokecolor{currentstroke}%
\pgfsetdash{}{0pt}%
\pgfsys@defobject{currentmarker}{\pgfqpoint{0.000000in}{-0.048611in}}{\pgfqpoint{0.000000in}{0.000000in}}{%
\pgfpathmoveto{\pgfqpoint{0.000000in}{0.000000in}}%
\pgfpathlineto{\pgfqpoint{0.000000in}{-0.048611in}}%
\pgfusepath{stroke,fill}%
}%
\begin{pgfscope}%
\pgfsys@transformshift{1.829269in}{0.548769in}%
\pgfsys@useobject{currentmarker}{}%
\end{pgfscope}%
\end{pgfscope}%
\begin{pgfscope}%
\definecolor{textcolor}{rgb}{0.000000,0.000000,0.000000}%
\pgfsetstrokecolor{textcolor}%
\pgfsetfillcolor{textcolor}%
\pgftext[x=1.829269in,y=0.451547in,,top]{\color{textcolor}\sffamily\fontsize{10.000000}{12.000000}\selectfont \(\displaystyle {100000}\)}%
\end{pgfscope}%
\begin{pgfscope}%
\pgfsetbuttcap%
\pgfsetroundjoin%
\definecolor{currentfill}{rgb}{0.000000,0.000000,0.000000}%
\pgfsetfillcolor{currentfill}%
\pgfsetlinewidth{0.803000pt}%
\definecolor{currentstroke}{rgb}{0.000000,0.000000,0.000000}%
\pgfsetstrokecolor{currentstroke}%
\pgfsetdash{}{0pt}%
\pgfsys@defobject{currentmarker}{\pgfqpoint{0.000000in}{-0.048611in}}{\pgfqpoint{0.000000in}{0.000000in}}{%
\pgfpathmoveto{\pgfqpoint{0.000000in}{0.000000in}}%
\pgfpathlineto{\pgfqpoint{0.000000in}{-0.048611in}}%
\pgfusepath{stroke,fill}%
}%
\begin{pgfscope}%
\pgfsys@transformshift{2.677154in}{0.548769in}%
\pgfsys@useobject{currentmarker}{}%
\end{pgfscope}%
\end{pgfscope}%
\begin{pgfscope}%
\definecolor{textcolor}{rgb}{0.000000,0.000000,0.000000}%
\pgfsetstrokecolor{textcolor}%
\pgfsetfillcolor{textcolor}%
\pgftext[x=2.677154in,y=0.451547in,,top]{\color{textcolor}\sffamily\fontsize{10.000000}{12.000000}\selectfont \(\displaystyle {200000}\)}%
\end{pgfscope}%
\begin{pgfscope}%
\pgfsetbuttcap%
\pgfsetroundjoin%
\definecolor{currentfill}{rgb}{0.000000,0.000000,0.000000}%
\pgfsetfillcolor{currentfill}%
\pgfsetlinewidth{0.803000pt}%
\definecolor{currentstroke}{rgb}{0.000000,0.000000,0.000000}%
\pgfsetstrokecolor{currentstroke}%
\pgfsetdash{}{0pt}%
\pgfsys@defobject{currentmarker}{\pgfqpoint{0.000000in}{-0.048611in}}{\pgfqpoint{0.000000in}{0.000000in}}{%
\pgfpathmoveto{\pgfqpoint{0.000000in}{0.000000in}}%
\pgfpathlineto{\pgfqpoint{0.000000in}{-0.048611in}}%
\pgfusepath{stroke,fill}%
}%
\begin{pgfscope}%
\pgfsys@transformshift{3.525039in}{0.548769in}%
\pgfsys@useobject{currentmarker}{}%
\end{pgfscope}%
\end{pgfscope}%
\begin{pgfscope}%
\definecolor{textcolor}{rgb}{0.000000,0.000000,0.000000}%
\pgfsetstrokecolor{textcolor}%
\pgfsetfillcolor{textcolor}%
\pgftext[x=3.525039in,y=0.451547in,,top]{\color{textcolor}\sffamily\fontsize{10.000000}{12.000000}\selectfont \(\displaystyle {300000}\)}%
\end{pgfscope}%
\begin{pgfscope}%
\pgfsetbuttcap%
\pgfsetroundjoin%
\definecolor{currentfill}{rgb}{0.000000,0.000000,0.000000}%
\pgfsetfillcolor{currentfill}%
\pgfsetlinewidth{0.803000pt}%
\definecolor{currentstroke}{rgb}{0.000000,0.000000,0.000000}%
\pgfsetstrokecolor{currentstroke}%
\pgfsetdash{}{0pt}%
\pgfsys@defobject{currentmarker}{\pgfqpoint{0.000000in}{-0.048611in}}{\pgfqpoint{0.000000in}{0.000000in}}{%
\pgfpathmoveto{\pgfqpoint{0.000000in}{0.000000in}}%
\pgfpathlineto{\pgfqpoint{0.000000in}{-0.048611in}}%
\pgfusepath{stroke,fill}%
}%
\begin{pgfscope}%
\pgfsys@transformshift{4.372925in}{0.548769in}%
\pgfsys@useobject{currentmarker}{}%
\end{pgfscope}%
\end{pgfscope}%
\begin{pgfscope}%
\definecolor{textcolor}{rgb}{0.000000,0.000000,0.000000}%
\pgfsetstrokecolor{textcolor}%
\pgfsetfillcolor{textcolor}%
\pgftext[x=4.372925in,y=0.451547in,,top]{\color{textcolor}\sffamily\fontsize{10.000000}{12.000000}\selectfont \(\displaystyle {400000}\)}%
\end{pgfscope}%
\begin{pgfscope}%
\pgfsetbuttcap%
\pgfsetroundjoin%
\definecolor{currentfill}{rgb}{0.000000,0.000000,0.000000}%
\pgfsetfillcolor{currentfill}%
\pgfsetlinewidth{0.803000pt}%
\definecolor{currentstroke}{rgb}{0.000000,0.000000,0.000000}%
\pgfsetstrokecolor{currentstroke}%
\pgfsetdash{}{0pt}%
\pgfsys@defobject{currentmarker}{\pgfqpoint{0.000000in}{-0.048611in}}{\pgfqpoint{0.000000in}{0.000000in}}{%
\pgfpathmoveto{\pgfqpoint{0.000000in}{0.000000in}}%
\pgfpathlineto{\pgfqpoint{0.000000in}{-0.048611in}}%
\pgfusepath{stroke,fill}%
}%
\begin{pgfscope}%
\pgfsys@transformshift{5.220810in}{0.548769in}%
\pgfsys@useobject{currentmarker}{}%
\end{pgfscope}%
\end{pgfscope}%
\begin{pgfscope}%
\definecolor{textcolor}{rgb}{0.000000,0.000000,0.000000}%
\pgfsetstrokecolor{textcolor}%
\pgfsetfillcolor{textcolor}%
\pgftext[x=5.220810in,y=0.451547in,,top]{\color{textcolor}\sffamily\fontsize{10.000000}{12.000000}\selectfont \(\displaystyle {500000}\)}%
\end{pgfscope}%
\begin{pgfscope}%
\definecolor{textcolor}{rgb}{0.000000,0.000000,0.000000}%
\pgfsetstrokecolor{textcolor}%
\pgfsetfillcolor{textcolor}%
\pgftext[x=3.318537in,y=0.272658in,,top]{\color{textcolor}\sffamily\fontsize{10.000000}{12.000000}\selectfont Methods}%
\end{pgfscope}%
\begin{pgfscope}%
\pgfsetbuttcap%
\pgfsetroundjoin%
\definecolor{currentfill}{rgb}{0.000000,0.000000,0.000000}%
\pgfsetfillcolor{currentfill}%
\pgfsetlinewidth{0.803000pt}%
\definecolor{currentstroke}{rgb}{0.000000,0.000000,0.000000}%
\pgfsetstrokecolor{currentstroke}%
\pgfsetdash{}{0pt}%
\pgfsys@defobject{currentmarker}{\pgfqpoint{-0.048611in}{0.000000in}}{\pgfqpoint{0.000000in}{0.000000in}}{%
\pgfpathmoveto{\pgfqpoint{0.000000in}{0.000000in}}%
\pgfpathlineto{\pgfqpoint{-0.048611in}{0.000000in}}%
\pgfusepath{stroke,fill}%
}%
\begin{pgfscope}%
\pgfsys@transformshift{0.787074in}{0.689795in}%
\pgfsys@useobject{currentmarker}{}%
\end{pgfscope}%
\end{pgfscope}%
\begin{pgfscope}%
\definecolor{textcolor}{rgb}{0.000000,0.000000,0.000000}%
\pgfsetstrokecolor{textcolor}%
\pgfsetfillcolor{textcolor}%
\pgftext[x=0.620407in, y=0.641601in, left, base]{\color{textcolor}\sffamily\fontsize{10.000000}{12.000000}\selectfont \(\displaystyle {0}\)}%
\end{pgfscope}%
\begin{pgfscope}%
\pgfsetbuttcap%
\pgfsetroundjoin%
\definecolor{currentfill}{rgb}{0.000000,0.000000,0.000000}%
\pgfsetfillcolor{currentfill}%
\pgfsetlinewidth{0.803000pt}%
\definecolor{currentstroke}{rgb}{0.000000,0.000000,0.000000}%
\pgfsetstrokecolor{currentstroke}%
\pgfsetdash{}{0pt}%
\pgfsys@defobject{currentmarker}{\pgfqpoint{-0.048611in}{0.000000in}}{\pgfqpoint{0.000000in}{0.000000in}}{%
\pgfpathmoveto{\pgfqpoint{0.000000in}{0.000000in}}%
\pgfpathlineto{\pgfqpoint{-0.048611in}{0.000000in}}%
\pgfusepath{stroke,fill}%
}%
\begin{pgfscope}%
\pgfsys@transformshift{0.787074in}{1.373761in}%
\pgfsys@useobject{currentmarker}{}%
\end{pgfscope}%
\end{pgfscope}%
\begin{pgfscope}%
\definecolor{textcolor}{rgb}{0.000000,0.000000,0.000000}%
\pgfsetstrokecolor{textcolor}%
\pgfsetfillcolor{textcolor}%
\pgftext[x=0.412073in, y=1.325566in, left, base]{\color{textcolor}\sffamily\fontsize{10.000000}{12.000000}\selectfont \(\displaystyle {5000}\)}%
\end{pgfscope}%
\begin{pgfscope}%
\pgfsetbuttcap%
\pgfsetroundjoin%
\definecolor{currentfill}{rgb}{0.000000,0.000000,0.000000}%
\pgfsetfillcolor{currentfill}%
\pgfsetlinewidth{0.803000pt}%
\definecolor{currentstroke}{rgb}{0.000000,0.000000,0.000000}%
\pgfsetstrokecolor{currentstroke}%
\pgfsetdash{}{0pt}%
\pgfsys@defobject{currentmarker}{\pgfqpoint{-0.048611in}{0.000000in}}{\pgfqpoint{0.000000in}{0.000000in}}{%
\pgfpathmoveto{\pgfqpoint{0.000000in}{0.000000in}}%
\pgfpathlineto{\pgfqpoint{-0.048611in}{0.000000in}}%
\pgfusepath{stroke,fill}%
}%
\begin{pgfscope}%
\pgfsys@transformshift{0.787074in}{2.057726in}%
\pgfsys@useobject{currentmarker}{}%
\end{pgfscope}%
\end{pgfscope}%
\begin{pgfscope}%
\definecolor{textcolor}{rgb}{0.000000,0.000000,0.000000}%
\pgfsetstrokecolor{textcolor}%
\pgfsetfillcolor{textcolor}%
\pgftext[x=0.342628in, y=2.009532in, left, base]{\color{textcolor}\sffamily\fontsize{10.000000}{12.000000}\selectfont \(\displaystyle {10000}\)}%
\end{pgfscope}%
\begin{pgfscope}%
\pgfsetbuttcap%
\pgfsetroundjoin%
\definecolor{currentfill}{rgb}{0.000000,0.000000,0.000000}%
\pgfsetfillcolor{currentfill}%
\pgfsetlinewidth{0.803000pt}%
\definecolor{currentstroke}{rgb}{0.000000,0.000000,0.000000}%
\pgfsetstrokecolor{currentstroke}%
\pgfsetdash{}{0pt}%
\pgfsys@defobject{currentmarker}{\pgfqpoint{-0.048611in}{0.000000in}}{\pgfqpoint{0.000000in}{0.000000in}}{%
\pgfpathmoveto{\pgfqpoint{0.000000in}{0.000000in}}%
\pgfpathlineto{\pgfqpoint{-0.048611in}{0.000000in}}%
\pgfusepath{stroke,fill}%
}%
\begin{pgfscope}%
\pgfsys@transformshift{0.787074in}{2.741692in}%
\pgfsys@useobject{currentmarker}{}%
\end{pgfscope}%
\end{pgfscope}%
\begin{pgfscope}%
\definecolor{textcolor}{rgb}{0.000000,0.000000,0.000000}%
\pgfsetstrokecolor{textcolor}%
\pgfsetfillcolor{textcolor}%
\pgftext[x=0.342628in, y=2.693498in, left, base]{\color{textcolor}\sffamily\fontsize{10.000000}{12.000000}\selectfont \(\displaystyle {15000}\)}%
\end{pgfscope}%
\begin{pgfscope}%
\pgfsetbuttcap%
\pgfsetroundjoin%
\definecolor{currentfill}{rgb}{0.000000,0.000000,0.000000}%
\pgfsetfillcolor{currentfill}%
\pgfsetlinewidth{0.803000pt}%
\definecolor{currentstroke}{rgb}{0.000000,0.000000,0.000000}%
\pgfsetstrokecolor{currentstroke}%
\pgfsetdash{}{0pt}%
\pgfsys@defobject{currentmarker}{\pgfqpoint{-0.048611in}{0.000000in}}{\pgfqpoint{0.000000in}{0.000000in}}{%
\pgfpathmoveto{\pgfqpoint{0.000000in}{0.000000in}}%
\pgfpathlineto{\pgfqpoint{-0.048611in}{0.000000in}}%
\pgfusepath{stroke,fill}%
}%
\begin{pgfscope}%
\pgfsys@transformshift{0.787074in}{3.425658in}%
\pgfsys@useobject{currentmarker}{}%
\end{pgfscope}%
\end{pgfscope}%
\begin{pgfscope}%
\definecolor{textcolor}{rgb}{0.000000,0.000000,0.000000}%
\pgfsetstrokecolor{textcolor}%
\pgfsetfillcolor{textcolor}%
\pgftext[x=0.342628in, y=3.377463in, left, base]{\color{textcolor}\sffamily\fontsize{10.000000}{12.000000}\selectfont \(\displaystyle {20000}\)}%
\end{pgfscope}%
\begin{pgfscope}%
\definecolor{textcolor}{rgb}{0.000000,0.000000,0.000000}%
\pgfsetstrokecolor{textcolor}%
\pgfsetfillcolor{textcolor}%
\pgftext[x=0.287073in,y=2.100064in,,bottom,rotate=90.000000]{\color{textcolor}\sffamily\fontsize{10.000000}{12.000000}\selectfont Maximum Memory Consumption (MB)}%
\end{pgfscope}%
\begin{pgfscope}%
\pgfsetrectcap%
\pgfsetmiterjoin%
\pgfsetlinewidth{0.803000pt}%
\definecolor{currentstroke}{rgb}{0.000000,0.000000,0.000000}%
\pgfsetstrokecolor{currentstroke}%
\pgfsetdash{}{0pt}%
\pgfpathmoveto{\pgfqpoint{0.787074in}{0.548769in}}%
\pgfpathlineto{\pgfqpoint{0.787074in}{3.651359in}}%
\pgfusepath{stroke}%
\end{pgfscope}%
\begin{pgfscope}%
\pgfsetrectcap%
\pgfsetmiterjoin%
\pgfsetlinewidth{0.803000pt}%
\definecolor{currentstroke}{rgb}{0.000000,0.000000,0.000000}%
\pgfsetstrokecolor{currentstroke}%
\pgfsetdash{}{0pt}%
\pgfpathmoveto{\pgfqpoint{5.850000in}{0.548769in}}%
\pgfpathlineto{\pgfqpoint{5.850000in}{3.651359in}}%
\pgfusepath{stroke}%
\end{pgfscope}%
\begin{pgfscope}%
\pgfsetrectcap%
\pgfsetmiterjoin%
\pgfsetlinewidth{0.803000pt}%
\definecolor{currentstroke}{rgb}{0.000000,0.000000,0.000000}%
\pgfsetstrokecolor{currentstroke}%
\pgfsetdash{}{0pt}%
\pgfpathmoveto{\pgfqpoint{0.787074in}{0.548769in}}%
\pgfpathlineto{\pgfqpoint{5.850000in}{0.548769in}}%
\pgfusepath{stroke}%
\end{pgfscope}%
\begin{pgfscope}%
\pgfsetrectcap%
\pgfsetmiterjoin%
\pgfsetlinewidth{0.803000pt}%
\definecolor{currentstroke}{rgb}{0.000000,0.000000,0.000000}%
\pgfsetstrokecolor{currentstroke}%
\pgfsetdash{}{0pt}%
\pgfpathmoveto{\pgfqpoint{0.787074in}{3.651359in}}%
\pgfpathlineto{\pgfqpoint{5.850000in}{3.651359in}}%
\pgfusepath{stroke}%
\end{pgfscope}%
\begin{pgfscope}%
\definecolor{textcolor}{rgb}{0.000000,0.000000,0.000000}%
\pgfsetstrokecolor{textcolor}%
\pgfsetfillcolor{textcolor}%
\pgftext[x=3.318537in,y=3.734692in,,base]{\color{textcolor}\sffamily\fontsize{12.000000}{14.400000}\selectfont Forward}%
\end{pgfscope}%
\begin{pgfscope}%
\pgfsetbuttcap%
\pgfsetmiterjoin%
\definecolor{currentfill}{rgb}{1.000000,1.000000,1.000000}%
\pgfsetfillcolor{currentfill}%
\pgfsetfillopacity{0.800000}%
\pgfsetlinewidth{1.003750pt}%
\definecolor{currentstroke}{rgb}{0.800000,0.800000,0.800000}%
\pgfsetstrokecolor{currentstroke}%
\pgfsetstrokeopacity{0.800000}%
\pgfsetdash{}{0pt}%
\pgfpathmoveto{\pgfqpoint{4.300417in}{2.957886in}}%
\pgfpathlineto{\pgfqpoint{5.752778in}{2.957886in}}%
\pgfpathquadraticcurveto{\pgfqpoint{5.780556in}{2.957886in}}{\pgfqpoint{5.780556in}{2.985664in}}%
\pgfpathlineto{\pgfqpoint{5.780556in}{3.554136in}}%
\pgfpathquadraticcurveto{\pgfqpoint{5.780556in}{3.581914in}}{\pgfqpoint{5.752778in}{3.581914in}}%
\pgfpathlineto{\pgfqpoint{4.300417in}{3.581914in}}%
\pgfpathquadraticcurveto{\pgfqpoint{4.272639in}{3.581914in}}{\pgfqpoint{4.272639in}{3.554136in}}%
\pgfpathlineto{\pgfqpoint{4.272639in}{2.985664in}}%
\pgfpathquadraticcurveto{\pgfqpoint{4.272639in}{2.957886in}}{\pgfqpoint{4.300417in}{2.957886in}}%
\pgfpathclose%
\pgfusepath{stroke,fill}%
\end{pgfscope}%
\begin{pgfscope}%
\pgfsetbuttcap%
\pgfsetroundjoin%
\definecolor{currentfill}{rgb}{0.121569,0.466667,0.705882}%
\pgfsetfillcolor{currentfill}%
\pgfsetlinewidth{1.003750pt}%
\definecolor{currentstroke}{rgb}{0.121569,0.466667,0.705882}%
\pgfsetstrokecolor{currentstroke}%
\pgfsetdash{}{0pt}%
\pgfsys@defobject{currentmarker}{\pgfqpoint{-0.034722in}{-0.034722in}}{\pgfqpoint{0.034722in}{0.034722in}}{%
\pgfpathmoveto{\pgfqpoint{0.000000in}{-0.034722in}}%
\pgfpathcurveto{\pgfqpoint{0.009208in}{-0.034722in}}{\pgfqpoint{0.018041in}{-0.031064in}}{\pgfqpoint{0.024552in}{-0.024552in}}%
\pgfpathcurveto{\pgfqpoint{0.031064in}{-0.018041in}}{\pgfqpoint{0.034722in}{-0.009208in}}{\pgfqpoint{0.034722in}{0.000000in}}%
\pgfpathcurveto{\pgfqpoint{0.034722in}{0.009208in}}{\pgfqpoint{0.031064in}{0.018041in}}{\pgfqpoint{0.024552in}{0.024552in}}%
\pgfpathcurveto{\pgfqpoint{0.018041in}{0.031064in}}{\pgfqpoint{0.009208in}{0.034722in}}{\pgfqpoint{0.000000in}{0.034722in}}%
\pgfpathcurveto{\pgfqpoint{-0.009208in}{0.034722in}}{\pgfqpoint{-0.018041in}{0.031064in}}{\pgfqpoint{-0.024552in}{0.024552in}}%
\pgfpathcurveto{\pgfqpoint{-0.031064in}{0.018041in}}{\pgfqpoint{-0.034722in}{0.009208in}}{\pgfqpoint{-0.034722in}{0.000000in}}%
\pgfpathcurveto{\pgfqpoint{-0.034722in}{-0.009208in}}{\pgfqpoint{-0.031064in}{-0.018041in}}{\pgfqpoint{-0.024552in}{-0.024552in}}%
\pgfpathcurveto{\pgfqpoint{-0.018041in}{-0.031064in}}{\pgfqpoint{-0.009208in}{-0.034722in}}{\pgfqpoint{0.000000in}{-0.034722in}}%
\pgfpathclose%
\pgfusepath{stroke,fill}%
}%
\begin{pgfscope}%
\pgfsys@transformshift{4.467083in}{3.477748in}%
\pgfsys@useobject{currentmarker}{}%
\end{pgfscope}%
\end{pgfscope}%
\begin{pgfscope}%
\definecolor{textcolor}{rgb}{0.000000,0.000000,0.000000}%
\pgfsetstrokecolor{textcolor}%
\pgfsetfillcolor{textcolor}%
\pgftext[x=4.717083in,y=3.429136in,left,base]{\color{textcolor}\sffamily\fontsize{10.000000}{12.000000}\selectfont No Timeout}%
\end{pgfscope}%
\begin{pgfscope}%
\pgfsetbuttcap%
\pgfsetroundjoin%
\definecolor{currentfill}{rgb}{1.000000,0.498039,0.054902}%
\pgfsetfillcolor{currentfill}%
\pgfsetlinewidth{1.003750pt}%
\definecolor{currentstroke}{rgb}{1.000000,0.498039,0.054902}%
\pgfsetstrokecolor{currentstroke}%
\pgfsetdash{}{0pt}%
\pgfsys@defobject{currentmarker}{\pgfqpoint{-0.034722in}{-0.034722in}}{\pgfqpoint{0.034722in}{0.034722in}}{%
\pgfpathmoveto{\pgfqpoint{0.000000in}{-0.034722in}}%
\pgfpathcurveto{\pgfqpoint{0.009208in}{-0.034722in}}{\pgfqpoint{0.018041in}{-0.031064in}}{\pgfqpoint{0.024552in}{-0.024552in}}%
\pgfpathcurveto{\pgfqpoint{0.031064in}{-0.018041in}}{\pgfqpoint{0.034722in}{-0.009208in}}{\pgfqpoint{0.034722in}{0.000000in}}%
\pgfpathcurveto{\pgfqpoint{0.034722in}{0.009208in}}{\pgfqpoint{0.031064in}{0.018041in}}{\pgfqpoint{0.024552in}{0.024552in}}%
\pgfpathcurveto{\pgfqpoint{0.018041in}{0.031064in}}{\pgfqpoint{0.009208in}{0.034722in}}{\pgfqpoint{0.000000in}{0.034722in}}%
\pgfpathcurveto{\pgfqpoint{-0.009208in}{0.034722in}}{\pgfqpoint{-0.018041in}{0.031064in}}{\pgfqpoint{-0.024552in}{0.024552in}}%
\pgfpathcurveto{\pgfqpoint{-0.031064in}{0.018041in}}{\pgfqpoint{-0.034722in}{0.009208in}}{\pgfqpoint{-0.034722in}{0.000000in}}%
\pgfpathcurveto{\pgfqpoint{-0.034722in}{-0.009208in}}{\pgfqpoint{-0.031064in}{-0.018041in}}{\pgfqpoint{-0.024552in}{-0.024552in}}%
\pgfpathcurveto{\pgfqpoint{-0.018041in}{-0.031064in}}{\pgfqpoint{-0.009208in}{-0.034722in}}{\pgfqpoint{0.000000in}{-0.034722in}}%
\pgfpathclose%
\pgfusepath{stroke,fill}%
}%
\begin{pgfscope}%
\pgfsys@transformshift{4.467083in}{3.284136in}%
\pgfsys@useobject{currentmarker}{}%
\end{pgfscope}%
\end{pgfscope}%
\begin{pgfscope}%
\definecolor{textcolor}{rgb}{0.000000,0.000000,0.000000}%
\pgfsetstrokecolor{textcolor}%
\pgfsetfillcolor{textcolor}%
\pgftext[x=4.717083in,y=3.235525in,left,base]{\color{textcolor}\sffamily\fontsize{10.000000}{12.000000}\selectfont Time Timeout}%
\end{pgfscope}%
\begin{pgfscope}%
\pgfsetbuttcap%
\pgfsetroundjoin%
\definecolor{currentfill}{rgb}{0.839216,0.152941,0.156863}%
\pgfsetfillcolor{currentfill}%
\pgfsetlinewidth{1.003750pt}%
\definecolor{currentstroke}{rgb}{0.839216,0.152941,0.156863}%
\pgfsetstrokecolor{currentstroke}%
\pgfsetdash{}{0pt}%
\pgfsys@defobject{currentmarker}{\pgfqpoint{-0.034722in}{-0.034722in}}{\pgfqpoint{0.034722in}{0.034722in}}{%
\pgfpathmoveto{\pgfqpoint{0.000000in}{-0.034722in}}%
\pgfpathcurveto{\pgfqpoint{0.009208in}{-0.034722in}}{\pgfqpoint{0.018041in}{-0.031064in}}{\pgfqpoint{0.024552in}{-0.024552in}}%
\pgfpathcurveto{\pgfqpoint{0.031064in}{-0.018041in}}{\pgfqpoint{0.034722in}{-0.009208in}}{\pgfqpoint{0.034722in}{0.000000in}}%
\pgfpathcurveto{\pgfqpoint{0.034722in}{0.009208in}}{\pgfqpoint{0.031064in}{0.018041in}}{\pgfqpoint{0.024552in}{0.024552in}}%
\pgfpathcurveto{\pgfqpoint{0.018041in}{0.031064in}}{\pgfqpoint{0.009208in}{0.034722in}}{\pgfqpoint{0.000000in}{0.034722in}}%
\pgfpathcurveto{\pgfqpoint{-0.009208in}{0.034722in}}{\pgfqpoint{-0.018041in}{0.031064in}}{\pgfqpoint{-0.024552in}{0.024552in}}%
\pgfpathcurveto{\pgfqpoint{-0.031064in}{0.018041in}}{\pgfqpoint{-0.034722in}{0.009208in}}{\pgfqpoint{-0.034722in}{0.000000in}}%
\pgfpathcurveto{\pgfqpoint{-0.034722in}{-0.009208in}}{\pgfqpoint{-0.031064in}{-0.018041in}}{\pgfqpoint{-0.024552in}{-0.024552in}}%
\pgfpathcurveto{\pgfqpoint{-0.018041in}{-0.031064in}}{\pgfqpoint{-0.009208in}{-0.034722in}}{\pgfqpoint{0.000000in}{-0.034722in}}%
\pgfpathclose%
\pgfusepath{stroke,fill}%
}%
\begin{pgfscope}%
\pgfsys@transformshift{4.467083in}{3.090525in}%
\pgfsys@useobject{currentmarker}{}%
\end{pgfscope}%
\end{pgfscope}%
\begin{pgfscope}%
\definecolor{textcolor}{rgb}{0.000000,0.000000,0.000000}%
\pgfsetstrokecolor{textcolor}%
\pgfsetfillcolor{textcolor}%
\pgftext[x=4.717083in,y=3.041914in,left,base]{\color{textcolor}\sffamily\fontsize{10.000000}{12.000000}\selectfont Memory Timeout}%
\end{pgfscope}%
\end{pgfpicture}%
\makeatother%
\endgroup%

                }
            \end{subfigure}
            \qquad
            \begin{subfigure}[]{0.45\textwidth}
                \centering
                \resizebox{\columnwidth}{!}{
                    %% Creator: Matplotlib, PGF backend
%%
%% To include the figure in your LaTeX document, write
%%   \input{<filename>.pgf}
%%
%% Make sure the required packages are loaded in your preamble
%%   \usepackage{pgf}
%%
%% and, on pdftex
%%   \usepackage[utf8]{inputenc}\DeclareUnicodeCharacter{2212}{-}
%%
%% or, on luatex and xetex
%%   \usepackage{unicode-math}
%%
%% Figures using additional raster images can only be included by \input if
%% they are in the same directory as the main LaTeX file. For loading figures
%% from other directories you can use the `import` package
%%   \usepackage{import}
%%
%% and then include the figures with
%%   \import{<path to file>}{<filename>.pgf}
%%
%% Matplotlib used the following preamble
%%   \usepackage{amsmath}
%%   \usepackage{fontspec}
%%
\begingroup%
\makeatletter%
\begin{pgfpicture}%
\pgfpathrectangle{\pgfpointorigin}{\pgfqpoint{6.000000in}{4.000000in}}%
\pgfusepath{use as bounding box, clip}%
\begin{pgfscope}%
\pgfsetbuttcap%
\pgfsetmiterjoin%
\definecolor{currentfill}{rgb}{1.000000,1.000000,1.000000}%
\pgfsetfillcolor{currentfill}%
\pgfsetlinewidth{0.000000pt}%
\definecolor{currentstroke}{rgb}{1.000000,1.000000,1.000000}%
\pgfsetstrokecolor{currentstroke}%
\pgfsetdash{}{0pt}%
\pgfpathmoveto{\pgfqpoint{0.000000in}{0.000000in}}%
\pgfpathlineto{\pgfqpoint{6.000000in}{0.000000in}}%
\pgfpathlineto{\pgfqpoint{6.000000in}{4.000000in}}%
\pgfpathlineto{\pgfqpoint{0.000000in}{4.000000in}}%
\pgfpathclose%
\pgfusepath{fill}%
\end{pgfscope}%
\begin{pgfscope}%
\pgfsetbuttcap%
\pgfsetmiterjoin%
\definecolor{currentfill}{rgb}{1.000000,1.000000,1.000000}%
\pgfsetfillcolor{currentfill}%
\pgfsetlinewidth{0.000000pt}%
\definecolor{currentstroke}{rgb}{0.000000,0.000000,0.000000}%
\pgfsetstrokecolor{currentstroke}%
\pgfsetstrokeopacity{0.000000}%
\pgfsetdash{}{0pt}%
\pgfpathmoveto{\pgfqpoint{0.787074in}{0.548769in}}%
\pgfpathlineto{\pgfqpoint{5.850000in}{0.548769in}}%
\pgfpathlineto{\pgfqpoint{5.850000in}{3.651359in}}%
\pgfpathlineto{\pgfqpoint{0.787074in}{3.651359in}}%
\pgfpathclose%
\pgfusepath{fill}%
\end{pgfscope}%
\begin{pgfscope}%
\pgfpathrectangle{\pgfqpoint{0.787074in}{0.548769in}}{\pgfqpoint{5.062926in}{3.102590in}}%
\pgfusepath{clip}%
\pgfsetbuttcap%
\pgfsetroundjoin%
\definecolor{currentfill}{rgb}{0.121569,0.466667,0.705882}%
\pgfsetfillcolor{currentfill}%
\pgfsetlinewidth{1.003750pt}%
\definecolor{currentstroke}{rgb}{0.121569,0.466667,0.705882}%
\pgfsetstrokecolor{currentstroke}%
\pgfsetdash{}{0pt}%
\pgfpathmoveto{\pgfqpoint{1.453919in}{0.676730in}}%
\pgfpathcurveto{\pgfqpoint{1.464969in}{0.676730in}}{\pgfqpoint{1.475568in}{0.681121in}}{\pgfqpoint{1.483381in}{0.688934in}}%
\pgfpathcurveto{\pgfqpoint{1.491195in}{0.696748in}}{\pgfqpoint{1.495585in}{0.707347in}}{\pgfqpoint{1.495585in}{0.718397in}}%
\pgfpathcurveto{\pgfqpoint{1.495585in}{0.729447in}}{\pgfqpoint{1.491195in}{0.740046in}}{\pgfqpoint{1.483381in}{0.747860in}}%
\pgfpathcurveto{\pgfqpoint{1.475568in}{0.755673in}}{\pgfqpoint{1.464969in}{0.760064in}}{\pgfqpoint{1.453919in}{0.760064in}}%
\pgfpathcurveto{\pgfqpoint{1.442868in}{0.760064in}}{\pgfqpoint{1.432269in}{0.755673in}}{\pgfqpoint{1.424456in}{0.747860in}}%
\pgfpathcurveto{\pgfqpoint{1.416642in}{0.740046in}}{\pgfqpoint{1.412252in}{0.729447in}}{\pgfqpoint{1.412252in}{0.718397in}}%
\pgfpathcurveto{\pgfqpoint{1.412252in}{0.707347in}}{\pgfqpoint{1.416642in}{0.696748in}}{\pgfqpoint{1.424456in}{0.688934in}}%
\pgfpathcurveto{\pgfqpoint{1.432269in}{0.681121in}}{\pgfqpoint{1.442868in}{0.676730in}}{\pgfqpoint{1.453919in}{0.676730in}}%
\pgfpathclose%
\pgfusepath{stroke,fill}%
\end{pgfscope}%
\begin{pgfscope}%
\pgfpathrectangle{\pgfqpoint{0.787074in}{0.548769in}}{\pgfqpoint{5.062926in}{3.102590in}}%
\pgfusepath{clip}%
\pgfsetbuttcap%
\pgfsetroundjoin%
\definecolor{currentfill}{rgb}{1.000000,0.498039,0.054902}%
\pgfsetfillcolor{currentfill}%
\pgfsetlinewidth{1.003750pt}%
\definecolor{currentstroke}{rgb}{1.000000,0.498039,0.054902}%
\pgfsetstrokecolor{currentstroke}%
\pgfsetdash{}{0pt}%
\pgfpathmoveto{\pgfqpoint{4.757873in}{1.501451in}}%
\pgfpathcurveto{\pgfqpoint{4.768923in}{1.501451in}}{\pgfqpoint{4.779522in}{1.505841in}}{\pgfqpoint{4.787336in}{1.513655in}}%
\pgfpathcurveto{\pgfqpoint{4.795149in}{1.521468in}}{\pgfqpoint{4.799540in}{1.532067in}}{\pgfqpoint{4.799540in}{1.543118in}}%
\pgfpathcurveto{\pgfqpoint{4.799540in}{1.554168in}}{\pgfqpoint{4.795149in}{1.564767in}}{\pgfqpoint{4.787336in}{1.572580in}}%
\pgfpathcurveto{\pgfqpoint{4.779522in}{1.580394in}}{\pgfqpoint{4.768923in}{1.584784in}}{\pgfqpoint{4.757873in}{1.584784in}}%
\pgfpathcurveto{\pgfqpoint{4.746823in}{1.584784in}}{\pgfqpoint{4.736224in}{1.580394in}}{\pgfqpoint{4.728410in}{1.572580in}}%
\pgfpathcurveto{\pgfqpoint{4.720597in}{1.564767in}}{\pgfqpoint{4.716206in}{1.554168in}}{\pgfqpoint{4.716206in}{1.543118in}}%
\pgfpathcurveto{\pgfqpoint{4.716206in}{1.532067in}}{\pgfqpoint{4.720597in}{1.521468in}}{\pgfqpoint{4.728410in}{1.513655in}}%
\pgfpathcurveto{\pgfqpoint{4.736224in}{1.505841in}}{\pgfqpoint{4.746823in}{1.501451in}}{\pgfqpoint{4.757873in}{1.501451in}}%
\pgfpathclose%
\pgfusepath{stroke,fill}%
\end{pgfscope}%
\begin{pgfscope}%
\pgfpathrectangle{\pgfqpoint{0.787074in}{0.548769in}}{\pgfqpoint{5.062926in}{3.102590in}}%
\pgfusepath{clip}%
\pgfsetbuttcap%
\pgfsetroundjoin%
\definecolor{currentfill}{rgb}{1.000000,0.498039,0.054902}%
\pgfsetfillcolor{currentfill}%
\pgfsetlinewidth{1.003750pt}%
\definecolor{currentstroke}{rgb}{1.000000,0.498039,0.054902}%
\pgfsetstrokecolor{currentstroke}%
\pgfsetdash{}{0pt}%
\pgfpathmoveto{\pgfqpoint{1.385621in}{2.589003in}}%
\pgfpathcurveto{\pgfqpoint{1.396672in}{2.589003in}}{\pgfqpoint{1.407271in}{2.593393in}}{\pgfqpoint{1.415084in}{2.601207in}}%
\pgfpathcurveto{\pgfqpoint{1.422898in}{2.609020in}}{\pgfqpoint{1.427288in}{2.619619in}}{\pgfqpoint{1.427288in}{2.630670in}}%
\pgfpathcurveto{\pgfqpoint{1.427288in}{2.641720in}}{\pgfqpoint{1.422898in}{2.652319in}}{\pgfqpoint{1.415084in}{2.660132in}}%
\pgfpathcurveto{\pgfqpoint{1.407271in}{2.667946in}}{\pgfqpoint{1.396672in}{2.672336in}}{\pgfqpoint{1.385621in}{2.672336in}}%
\pgfpathcurveto{\pgfqpoint{1.374571in}{2.672336in}}{\pgfqpoint{1.363972in}{2.667946in}}{\pgfqpoint{1.356159in}{2.660132in}}%
\pgfpathcurveto{\pgfqpoint{1.348345in}{2.652319in}}{\pgfqpoint{1.343955in}{2.641720in}}{\pgfqpoint{1.343955in}{2.630670in}}%
\pgfpathcurveto{\pgfqpoint{1.343955in}{2.619619in}}{\pgfqpoint{1.348345in}{2.609020in}}{\pgfqpoint{1.356159in}{2.601207in}}%
\pgfpathcurveto{\pgfqpoint{1.363972in}{2.593393in}}{\pgfqpoint{1.374571in}{2.589003in}}{\pgfqpoint{1.385621in}{2.589003in}}%
\pgfpathclose%
\pgfusepath{stroke,fill}%
\end{pgfscope}%
\begin{pgfscope}%
\pgfpathrectangle{\pgfqpoint{0.787074in}{0.548769in}}{\pgfqpoint{5.062926in}{3.102590in}}%
\pgfusepath{clip}%
\pgfsetbuttcap%
\pgfsetroundjoin%
\definecolor{currentfill}{rgb}{1.000000,0.498039,0.054902}%
\pgfsetfillcolor{currentfill}%
\pgfsetlinewidth{1.003750pt}%
\definecolor{currentstroke}{rgb}{1.000000,0.498039,0.054902}%
\pgfsetstrokecolor{currentstroke}%
\pgfsetdash{}{0pt}%
\pgfpathmoveto{\pgfqpoint{3.029578in}{1.800918in}}%
\pgfpathcurveto{\pgfqpoint{3.040628in}{1.800918in}}{\pgfqpoint{3.051227in}{1.805309in}}{\pgfqpoint{3.059040in}{1.813122in}}%
\pgfpathcurveto{\pgfqpoint{3.066854in}{1.820936in}}{\pgfqpoint{3.071244in}{1.831535in}}{\pgfqpoint{3.071244in}{1.842585in}}%
\pgfpathcurveto{\pgfqpoint{3.071244in}{1.853635in}}{\pgfqpoint{3.066854in}{1.864234in}}{\pgfqpoint{3.059040in}{1.872048in}}%
\pgfpathcurveto{\pgfqpoint{3.051227in}{1.879862in}}{\pgfqpoint{3.040628in}{1.884252in}}{\pgfqpoint{3.029578in}{1.884252in}}%
\pgfpathcurveto{\pgfqpoint{3.018528in}{1.884252in}}{\pgfqpoint{3.007928in}{1.879862in}}{\pgfqpoint{3.000115in}{1.872048in}}%
\pgfpathcurveto{\pgfqpoint{2.992301in}{1.864234in}}{\pgfqpoint{2.987911in}{1.853635in}}{\pgfqpoint{2.987911in}{1.842585in}}%
\pgfpathcurveto{\pgfqpoint{2.987911in}{1.831535in}}{\pgfqpoint{2.992301in}{1.820936in}}{\pgfqpoint{3.000115in}{1.813122in}}%
\pgfpathcurveto{\pgfqpoint{3.007928in}{1.805309in}}{\pgfqpoint{3.018528in}{1.800918in}}{\pgfqpoint{3.029578in}{1.800918in}}%
\pgfpathclose%
\pgfusepath{stroke,fill}%
\end{pgfscope}%
\begin{pgfscope}%
\pgfpathrectangle{\pgfqpoint{0.787074in}{0.548769in}}{\pgfqpoint{5.062926in}{3.102590in}}%
\pgfusepath{clip}%
\pgfsetbuttcap%
\pgfsetroundjoin%
\definecolor{currentfill}{rgb}{0.121569,0.466667,0.705882}%
\pgfsetfillcolor{currentfill}%
\pgfsetlinewidth{1.003750pt}%
\definecolor{currentstroke}{rgb}{0.121569,0.466667,0.705882}%
\pgfsetstrokecolor{currentstroke}%
\pgfsetdash{}{0pt}%
\pgfpathmoveto{\pgfqpoint{2.933936in}{0.650081in}}%
\pgfpathcurveto{\pgfqpoint{2.944986in}{0.650081in}}{\pgfqpoint{2.955585in}{0.654472in}}{\pgfqpoint{2.963399in}{0.662285in}}%
\pgfpathcurveto{\pgfqpoint{2.971213in}{0.670099in}}{\pgfqpoint{2.975603in}{0.680698in}}{\pgfqpoint{2.975603in}{0.691748in}}%
\pgfpathcurveto{\pgfqpoint{2.975603in}{0.702798in}}{\pgfqpoint{2.971213in}{0.713397in}}{\pgfqpoint{2.963399in}{0.721211in}}%
\pgfpathcurveto{\pgfqpoint{2.955585in}{0.729024in}}{\pgfqpoint{2.944986in}{0.733415in}}{\pgfqpoint{2.933936in}{0.733415in}}%
\pgfpathcurveto{\pgfqpoint{2.922886in}{0.733415in}}{\pgfqpoint{2.912287in}{0.729024in}}{\pgfqpoint{2.904473in}{0.721211in}}%
\pgfpathcurveto{\pgfqpoint{2.896660in}{0.713397in}}{\pgfqpoint{2.892270in}{0.702798in}}{\pgfqpoint{2.892270in}{0.691748in}}%
\pgfpathcurveto{\pgfqpoint{2.892270in}{0.680698in}}{\pgfqpoint{2.896660in}{0.670099in}}{\pgfqpoint{2.904473in}{0.662285in}}%
\pgfpathcurveto{\pgfqpoint{2.912287in}{0.654472in}}{\pgfqpoint{2.922886in}{0.650081in}}{\pgfqpoint{2.933936in}{0.650081in}}%
\pgfpathclose%
\pgfusepath{stroke,fill}%
\end{pgfscope}%
\begin{pgfscope}%
\pgfpathrectangle{\pgfqpoint{0.787074in}{0.548769in}}{\pgfqpoint{5.062926in}{3.102590in}}%
\pgfusepath{clip}%
\pgfsetbuttcap%
\pgfsetroundjoin%
\definecolor{currentfill}{rgb}{1.000000,0.498039,0.054902}%
\pgfsetfillcolor{currentfill}%
\pgfsetlinewidth{1.003750pt}%
\definecolor{currentstroke}{rgb}{1.000000,0.498039,0.054902}%
\pgfsetstrokecolor{currentstroke}%
\pgfsetdash{}{0pt}%
\pgfpathmoveto{\pgfqpoint{2.647063in}{2.499787in}}%
\pgfpathcurveto{\pgfqpoint{2.658113in}{2.499787in}}{\pgfqpoint{2.668712in}{2.504177in}}{\pgfqpoint{2.676525in}{2.511991in}}%
\pgfpathcurveto{\pgfqpoint{2.684339in}{2.519804in}}{\pgfqpoint{2.688729in}{2.530403in}}{\pgfqpoint{2.688729in}{2.541454in}}%
\pgfpathcurveto{\pgfqpoint{2.688729in}{2.552504in}}{\pgfqpoint{2.684339in}{2.563103in}}{\pgfqpoint{2.676525in}{2.570916in}}%
\pgfpathcurveto{\pgfqpoint{2.668712in}{2.578730in}}{\pgfqpoint{2.658113in}{2.583120in}}{\pgfqpoint{2.647063in}{2.583120in}}%
\pgfpathcurveto{\pgfqpoint{2.636013in}{2.583120in}}{\pgfqpoint{2.625414in}{2.578730in}}{\pgfqpoint{2.617600in}{2.570916in}}%
\pgfpathcurveto{\pgfqpoint{2.609786in}{2.563103in}}{\pgfqpoint{2.605396in}{2.552504in}}{\pgfqpoint{2.605396in}{2.541454in}}%
\pgfpathcurveto{\pgfqpoint{2.605396in}{2.530403in}}{\pgfqpoint{2.609786in}{2.519804in}}{\pgfqpoint{2.617600in}{2.511991in}}%
\pgfpathcurveto{\pgfqpoint{2.625414in}{2.504177in}}{\pgfqpoint{2.636013in}{2.499787in}}{\pgfqpoint{2.647063in}{2.499787in}}%
\pgfpathclose%
\pgfusepath{stroke,fill}%
\end{pgfscope}%
\begin{pgfscope}%
\pgfpathrectangle{\pgfqpoint{0.787074in}{0.548769in}}{\pgfqpoint{5.062926in}{3.102590in}}%
\pgfusepath{clip}%
\pgfsetbuttcap%
\pgfsetroundjoin%
\definecolor{currentfill}{rgb}{1.000000,0.498039,0.054902}%
\pgfsetfillcolor{currentfill}%
\pgfsetlinewidth{1.003750pt}%
\definecolor{currentstroke}{rgb}{1.000000,0.498039,0.054902}%
\pgfsetstrokecolor{currentstroke}%
\pgfsetdash{}{0pt}%
\pgfpathmoveto{\pgfqpoint{2.561367in}{2.338512in}}%
\pgfpathcurveto{\pgfqpoint{2.572417in}{2.338512in}}{\pgfqpoint{2.583016in}{2.342903in}}{\pgfqpoint{2.590830in}{2.350716in}}%
\pgfpathcurveto{\pgfqpoint{2.598643in}{2.358530in}}{\pgfqpoint{2.603034in}{2.369129in}}{\pgfqpoint{2.603034in}{2.380179in}}%
\pgfpathcurveto{\pgfqpoint{2.603034in}{2.391229in}}{\pgfqpoint{2.598643in}{2.401828in}}{\pgfqpoint{2.590830in}{2.409642in}}%
\pgfpathcurveto{\pgfqpoint{2.583016in}{2.417455in}}{\pgfqpoint{2.572417in}{2.421846in}}{\pgfqpoint{2.561367in}{2.421846in}}%
\pgfpathcurveto{\pgfqpoint{2.550317in}{2.421846in}}{\pgfqpoint{2.539718in}{2.417455in}}{\pgfqpoint{2.531904in}{2.409642in}}%
\pgfpathcurveto{\pgfqpoint{2.524091in}{2.401828in}}{\pgfqpoint{2.519700in}{2.391229in}}{\pgfqpoint{2.519700in}{2.380179in}}%
\pgfpathcurveto{\pgfqpoint{2.519700in}{2.369129in}}{\pgfqpoint{2.524091in}{2.358530in}}{\pgfqpoint{2.531904in}{2.350716in}}%
\pgfpathcurveto{\pgfqpoint{2.539718in}{2.342903in}}{\pgfqpoint{2.550317in}{2.338512in}}{\pgfqpoint{2.561367in}{2.338512in}}%
\pgfpathclose%
\pgfusepath{stroke,fill}%
\end{pgfscope}%
\begin{pgfscope}%
\pgfpathrectangle{\pgfqpoint{0.787074in}{0.548769in}}{\pgfqpoint{5.062926in}{3.102590in}}%
\pgfusepath{clip}%
\pgfsetbuttcap%
\pgfsetroundjoin%
\definecolor{currentfill}{rgb}{0.121569,0.466667,0.705882}%
\pgfsetfillcolor{currentfill}%
\pgfsetlinewidth{1.003750pt}%
\definecolor{currentstroke}{rgb}{0.121569,0.466667,0.705882}%
\pgfsetstrokecolor{currentstroke}%
\pgfsetdash{}{0pt}%
\pgfpathmoveto{\pgfqpoint{1.573657in}{0.648162in}}%
\pgfpathcurveto{\pgfqpoint{1.584707in}{0.648162in}}{\pgfqpoint{1.595306in}{0.652552in}}{\pgfqpoint{1.603120in}{0.660365in}}%
\pgfpathcurveto{\pgfqpoint{1.610933in}{0.668179in}}{\pgfqpoint{1.615324in}{0.678778in}}{\pgfqpoint{1.615324in}{0.689828in}}%
\pgfpathcurveto{\pgfqpoint{1.615324in}{0.700878in}}{\pgfqpoint{1.610933in}{0.711477in}}{\pgfqpoint{1.603120in}{0.719291in}}%
\pgfpathcurveto{\pgfqpoint{1.595306in}{0.727105in}}{\pgfqpoint{1.584707in}{0.731495in}}{\pgfqpoint{1.573657in}{0.731495in}}%
\pgfpathcurveto{\pgfqpoint{1.562607in}{0.731495in}}{\pgfqpoint{1.552008in}{0.727105in}}{\pgfqpoint{1.544194in}{0.719291in}}%
\pgfpathcurveto{\pgfqpoint{1.536381in}{0.711477in}}{\pgfqpoint{1.531990in}{0.700878in}}{\pgfqpoint{1.531990in}{0.689828in}}%
\pgfpathcurveto{\pgfqpoint{1.531990in}{0.678778in}}{\pgfqpoint{1.536381in}{0.668179in}}{\pgfqpoint{1.544194in}{0.660365in}}%
\pgfpathcurveto{\pgfqpoint{1.552008in}{0.652552in}}{\pgfqpoint{1.562607in}{0.648162in}}{\pgfqpoint{1.573657in}{0.648162in}}%
\pgfpathclose%
\pgfusepath{stroke,fill}%
\end{pgfscope}%
\begin{pgfscope}%
\pgfpathrectangle{\pgfqpoint{0.787074in}{0.548769in}}{\pgfqpoint{5.062926in}{3.102590in}}%
\pgfusepath{clip}%
\pgfsetbuttcap%
\pgfsetroundjoin%
\definecolor{currentfill}{rgb}{0.121569,0.466667,0.705882}%
\pgfsetfillcolor{currentfill}%
\pgfsetlinewidth{1.003750pt}%
\definecolor{currentstroke}{rgb}{0.121569,0.466667,0.705882}%
\pgfsetstrokecolor{currentstroke}%
\pgfsetdash{}{0pt}%
\pgfpathmoveto{\pgfqpoint{2.339365in}{0.840903in}}%
\pgfpathcurveto{\pgfqpoint{2.350415in}{0.840903in}}{\pgfqpoint{2.361014in}{0.845293in}}{\pgfqpoint{2.368828in}{0.853107in}}%
\pgfpathcurveto{\pgfqpoint{2.376642in}{0.860920in}}{\pgfqpoint{2.381032in}{0.871520in}}{\pgfqpoint{2.381032in}{0.882570in}}%
\pgfpathcurveto{\pgfqpoint{2.381032in}{0.893620in}}{\pgfqpoint{2.376642in}{0.904219in}}{\pgfqpoint{2.368828in}{0.912032in}}%
\pgfpathcurveto{\pgfqpoint{2.361014in}{0.919846in}}{\pgfqpoint{2.350415in}{0.924236in}}{\pgfqpoint{2.339365in}{0.924236in}}%
\pgfpathcurveto{\pgfqpoint{2.328315in}{0.924236in}}{\pgfqpoint{2.317716in}{0.919846in}}{\pgfqpoint{2.309902in}{0.912032in}}%
\pgfpathcurveto{\pgfqpoint{2.302089in}{0.904219in}}{\pgfqpoint{2.297698in}{0.893620in}}{\pgfqpoint{2.297698in}{0.882570in}}%
\pgfpathcurveto{\pgfqpoint{2.297698in}{0.871520in}}{\pgfqpoint{2.302089in}{0.860920in}}{\pgfqpoint{2.309902in}{0.853107in}}%
\pgfpathcurveto{\pgfqpoint{2.317716in}{0.845293in}}{\pgfqpoint{2.328315in}{0.840903in}}{\pgfqpoint{2.339365in}{0.840903in}}%
\pgfpathclose%
\pgfusepath{stroke,fill}%
\end{pgfscope}%
\begin{pgfscope}%
\pgfpathrectangle{\pgfqpoint{0.787074in}{0.548769in}}{\pgfqpoint{5.062926in}{3.102590in}}%
\pgfusepath{clip}%
\pgfsetbuttcap%
\pgfsetroundjoin%
\definecolor{currentfill}{rgb}{0.121569,0.466667,0.705882}%
\pgfsetfillcolor{currentfill}%
\pgfsetlinewidth{1.003750pt}%
\definecolor{currentstroke}{rgb}{0.121569,0.466667,0.705882}%
\pgfsetstrokecolor{currentstroke}%
\pgfsetdash{}{0pt}%
\pgfpathmoveto{\pgfqpoint{1.017258in}{0.787071in}}%
\pgfpathcurveto{\pgfqpoint{1.028308in}{0.787071in}}{\pgfqpoint{1.038907in}{0.791461in}}{\pgfqpoint{1.046721in}{0.799275in}}%
\pgfpathcurveto{\pgfqpoint{1.054534in}{0.807089in}}{\pgfqpoint{1.058924in}{0.817688in}}{\pgfqpoint{1.058924in}{0.828738in}}%
\pgfpathcurveto{\pgfqpoint{1.058924in}{0.839788in}}{\pgfqpoint{1.054534in}{0.850387in}}{\pgfqpoint{1.046721in}{0.858201in}}%
\pgfpathcurveto{\pgfqpoint{1.038907in}{0.866014in}}{\pgfqpoint{1.028308in}{0.870404in}}{\pgfqpoint{1.017258in}{0.870404in}}%
\pgfpathcurveto{\pgfqpoint{1.006208in}{0.870404in}}{\pgfqpoint{0.995609in}{0.866014in}}{\pgfqpoint{0.987795in}{0.858201in}}%
\pgfpathcurveto{\pgfqpoint{0.979981in}{0.850387in}}{\pgfqpoint{0.975591in}{0.839788in}}{\pgfqpoint{0.975591in}{0.828738in}}%
\pgfpathcurveto{\pgfqpoint{0.975591in}{0.817688in}}{\pgfqpoint{0.979981in}{0.807089in}}{\pgfqpoint{0.987795in}{0.799275in}}%
\pgfpathcurveto{\pgfqpoint{0.995609in}{0.791461in}}{\pgfqpoint{1.006208in}{0.787071in}}{\pgfqpoint{1.017258in}{0.787071in}}%
\pgfpathclose%
\pgfusepath{stroke,fill}%
\end{pgfscope}%
\begin{pgfscope}%
\pgfpathrectangle{\pgfqpoint{0.787074in}{0.548769in}}{\pgfqpoint{5.062926in}{3.102590in}}%
\pgfusepath{clip}%
\pgfsetbuttcap%
\pgfsetroundjoin%
\definecolor{currentfill}{rgb}{0.121569,0.466667,0.705882}%
\pgfsetfillcolor{currentfill}%
\pgfsetlinewidth{1.003750pt}%
\definecolor{currentstroke}{rgb}{0.121569,0.466667,0.705882}%
\pgfsetstrokecolor{currentstroke}%
\pgfsetdash{}{0pt}%
\pgfpathmoveto{\pgfqpoint{2.185220in}{0.660658in}}%
\pgfpathcurveto{\pgfqpoint{2.196270in}{0.660658in}}{\pgfqpoint{2.206869in}{0.665048in}}{\pgfqpoint{2.214682in}{0.672862in}}%
\pgfpathcurveto{\pgfqpoint{2.222496in}{0.680676in}}{\pgfqpoint{2.226886in}{0.691275in}}{\pgfqpoint{2.226886in}{0.702325in}}%
\pgfpathcurveto{\pgfqpoint{2.226886in}{0.713375in}}{\pgfqpoint{2.222496in}{0.723974in}}{\pgfqpoint{2.214682in}{0.731788in}}%
\pgfpathcurveto{\pgfqpoint{2.206869in}{0.739601in}}{\pgfqpoint{2.196270in}{0.743991in}}{\pgfqpoint{2.185220in}{0.743991in}}%
\pgfpathcurveto{\pgfqpoint{2.174169in}{0.743991in}}{\pgfqpoint{2.163570in}{0.739601in}}{\pgfqpoint{2.155757in}{0.731788in}}%
\pgfpathcurveto{\pgfqpoint{2.147943in}{0.723974in}}{\pgfqpoint{2.143553in}{0.713375in}}{\pgfqpoint{2.143553in}{0.702325in}}%
\pgfpathcurveto{\pgfqpoint{2.143553in}{0.691275in}}{\pgfqpoint{2.147943in}{0.680676in}}{\pgfqpoint{2.155757in}{0.672862in}}%
\pgfpathcurveto{\pgfqpoint{2.163570in}{0.665048in}}{\pgfqpoint{2.174169in}{0.660658in}}{\pgfqpoint{2.185220in}{0.660658in}}%
\pgfpathclose%
\pgfusepath{stroke,fill}%
\end{pgfscope}%
\begin{pgfscope}%
\pgfpathrectangle{\pgfqpoint{0.787074in}{0.548769in}}{\pgfqpoint{5.062926in}{3.102590in}}%
\pgfusepath{clip}%
\pgfsetbuttcap%
\pgfsetroundjoin%
\definecolor{currentfill}{rgb}{0.121569,0.466667,0.705882}%
\pgfsetfillcolor{currentfill}%
\pgfsetlinewidth{1.003750pt}%
\definecolor{currentstroke}{rgb}{0.121569,0.466667,0.705882}%
\pgfsetstrokecolor{currentstroke}%
\pgfsetdash{}{0pt}%
\pgfpathmoveto{\pgfqpoint{1.618595in}{0.648155in}}%
\pgfpathcurveto{\pgfqpoint{1.629645in}{0.648155in}}{\pgfqpoint{1.640244in}{0.652546in}}{\pgfqpoint{1.648058in}{0.660359in}}%
\pgfpathcurveto{\pgfqpoint{1.655871in}{0.668173in}}{\pgfqpoint{1.660262in}{0.678772in}}{\pgfqpoint{1.660262in}{0.689822in}}%
\pgfpathcurveto{\pgfqpoint{1.660262in}{0.700872in}}{\pgfqpoint{1.655871in}{0.711471in}}{\pgfqpoint{1.648058in}{0.719285in}}%
\pgfpathcurveto{\pgfqpoint{1.640244in}{0.727098in}}{\pgfqpoint{1.629645in}{0.731489in}}{\pgfqpoint{1.618595in}{0.731489in}}%
\pgfpathcurveto{\pgfqpoint{1.607545in}{0.731489in}}{\pgfqpoint{1.596946in}{0.727098in}}{\pgfqpoint{1.589132in}{0.719285in}}%
\pgfpathcurveto{\pgfqpoint{1.581318in}{0.711471in}}{\pgfqpoint{1.576928in}{0.700872in}}{\pgfqpoint{1.576928in}{0.689822in}}%
\pgfpathcurveto{\pgfqpoint{1.576928in}{0.678772in}}{\pgfqpoint{1.581318in}{0.668173in}}{\pgfqpoint{1.589132in}{0.660359in}}%
\pgfpathcurveto{\pgfqpoint{1.596946in}{0.652546in}}{\pgfqpoint{1.607545in}{0.648155in}}{\pgfqpoint{1.618595in}{0.648155in}}%
\pgfpathclose%
\pgfusepath{stroke,fill}%
\end{pgfscope}%
\begin{pgfscope}%
\pgfpathrectangle{\pgfqpoint{0.787074in}{0.548769in}}{\pgfqpoint{5.062926in}{3.102590in}}%
\pgfusepath{clip}%
\pgfsetbuttcap%
\pgfsetroundjoin%
\definecolor{currentfill}{rgb}{0.121569,0.466667,0.705882}%
\pgfsetfillcolor{currentfill}%
\pgfsetlinewidth{1.003750pt}%
\definecolor{currentstroke}{rgb}{0.121569,0.466667,0.705882}%
\pgfsetstrokecolor{currentstroke}%
\pgfsetdash{}{0pt}%
\pgfpathmoveto{\pgfqpoint{1.925063in}{0.648148in}}%
\pgfpathcurveto{\pgfqpoint{1.936113in}{0.648148in}}{\pgfqpoint{1.946712in}{0.652539in}}{\pgfqpoint{1.954526in}{0.660352in}}%
\pgfpathcurveto{\pgfqpoint{1.962339in}{0.668166in}}{\pgfqpoint{1.966730in}{0.678765in}}{\pgfqpoint{1.966730in}{0.689815in}}%
\pgfpathcurveto{\pgfqpoint{1.966730in}{0.700865in}}{\pgfqpoint{1.962339in}{0.711464in}}{\pgfqpoint{1.954526in}{0.719278in}}%
\pgfpathcurveto{\pgfqpoint{1.946712in}{0.727091in}}{\pgfqpoint{1.936113in}{0.731482in}}{\pgfqpoint{1.925063in}{0.731482in}}%
\pgfpathcurveto{\pgfqpoint{1.914013in}{0.731482in}}{\pgfqpoint{1.903414in}{0.727091in}}{\pgfqpoint{1.895600in}{0.719278in}}%
\pgfpathcurveto{\pgfqpoint{1.887787in}{0.711464in}}{\pgfqpoint{1.883396in}{0.700865in}}{\pgfqpoint{1.883396in}{0.689815in}}%
\pgfpathcurveto{\pgfqpoint{1.883396in}{0.678765in}}{\pgfqpoint{1.887787in}{0.668166in}}{\pgfqpoint{1.895600in}{0.660352in}}%
\pgfpathcurveto{\pgfqpoint{1.903414in}{0.652539in}}{\pgfqpoint{1.914013in}{0.648148in}}{\pgfqpoint{1.925063in}{0.648148in}}%
\pgfpathclose%
\pgfusepath{stroke,fill}%
\end{pgfscope}%
\begin{pgfscope}%
\pgfpathrectangle{\pgfqpoint{0.787074in}{0.548769in}}{\pgfqpoint{5.062926in}{3.102590in}}%
\pgfusepath{clip}%
\pgfsetbuttcap%
\pgfsetroundjoin%
\definecolor{currentfill}{rgb}{1.000000,0.498039,0.054902}%
\pgfsetfillcolor{currentfill}%
\pgfsetlinewidth{1.003750pt}%
\definecolor{currentstroke}{rgb}{1.000000,0.498039,0.054902}%
\pgfsetstrokecolor{currentstroke}%
\pgfsetdash{}{0pt}%
\pgfpathmoveto{\pgfqpoint{1.495499in}{1.766469in}}%
\pgfpathcurveto{\pgfqpoint{1.506549in}{1.766469in}}{\pgfqpoint{1.517148in}{1.770859in}}{\pgfqpoint{1.524962in}{1.778672in}}%
\pgfpathcurveto{\pgfqpoint{1.532775in}{1.786486in}}{\pgfqpoint{1.537166in}{1.797085in}}{\pgfqpoint{1.537166in}{1.808135in}}%
\pgfpathcurveto{\pgfqpoint{1.537166in}{1.819185in}}{\pgfqpoint{1.532775in}{1.829784in}}{\pgfqpoint{1.524962in}{1.837598in}}%
\pgfpathcurveto{\pgfqpoint{1.517148in}{1.845412in}}{\pgfqpoint{1.506549in}{1.849802in}}{\pgfqpoint{1.495499in}{1.849802in}}%
\pgfpathcurveto{\pgfqpoint{1.484449in}{1.849802in}}{\pgfqpoint{1.473850in}{1.845412in}}{\pgfqpoint{1.466036in}{1.837598in}}%
\pgfpathcurveto{\pgfqpoint{1.458222in}{1.829784in}}{\pgfqpoint{1.453832in}{1.819185in}}{\pgfqpoint{1.453832in}{1.808135in}}%
\pgfpathcurveto{\pgfqpoint{1.453832in}{1.797085in}}{\pgfqpoint{1.458222in}{1.786486in}}{\pgfqpoint{1.466036in}{1.778672in}}%
\pgfpathcurveto{\pgfqpoint{1.473850in}{1.770859in}}{\pgfqpoint{1.484449in}{1.766469in}}{\pgfqpoint{1.495499in}{1.766469in}}%
\pgfpathclose%
\pgfusepath{stroke,fill}%
\end{pgfscope}%
\begin{pgfscope}%
\pgfpathrectangle{\pgfqpoint{0.787074in}{0.548769in}}{\pgfqpoint{5.062926in}{3.102590in}}%
\pgfusepath{clip}%
\pgfsetbuttcap%
\pgfsetroundjoin%
\definecolor{currentfill}{rgb}{1.000000,0.498039,0.054902}%
\pgfsetfillcolor{currentfill}%
\pgfsetlinewidth{1.003750pt}%
\definecolor{currentstroke}{rgb}{1.000000,0.498039,0.054902}%
\pgfsetstrokecolor{currentstroke}%
\pgfsetdash{}{0pt}%
\pgfpathmoveto{\pgfqpoint{1.590174in}{2.575149in}}%
\pgfpathcurveto{\pgfqpoint{1.601224in}{2.575149in}}{\pgfqpoint{1.611823in}{2.579539in}}{\pgfqpoint{1.619637in}{2.587353in}}%
\pgfpathcurveto{\pgfqpoint{1.627450in}{2.595166in}}{\pgfqpoint{1.631840in}{2.605765in}}{\pgfqpoint{1.631840in}{2.616816in}}%
\pgfpathcurveto{\pgfqpoint{1.631840in}{2.627866in}}{\pgfqpoint{1.627450in}{2.638465in}}{\pgfqpoint{1.619637in}{2.646278in}}%
\pgfpathcurveto{\pgfqpoint{1.611823in}{2.654092in}}{\pgfqpoint{1.601224in}{2.658482in}}{\pgfqpoint{1.590174in}{2.658482in}}%
\pgfpathcurveto{\pgfqpoint{1.579124in}{2.658482in}}{\pgfqpoint{1.568525in}{2.654092in}}{\pgfqpoint{1.560711in}{2.646278in}}%
\pgfpathcurveto{\pgfqpoint{1.552897in}{2.638465in}}{\pgfqpoint{1.548507in}{2.627866in}}{\pgfqpoint{1.548507in}{2.616816in}}%
\pgfpathcurveto{\pgfqpoint{1.548507in}{2.605765in}}{\pgfqpoint{1.552897in}{2.595166in}}{\pgfqpoint{1.560711in}{2.587353in}}%
\pgfpathcurveto{\pgfqpoint{1.568525in}{2.579539in}}{\pgfqpoint{1.579124in}{2.575149in}}{\pgfqpoint{1.590174in}{2.575149in}}%
\pgfpathclose%
\pgfusepath{stroke,fill}%
\end{pgfscope}%
\begin{pgfscope}%
\pgfpathrectangle{\pgfqpoint{0.787074in}{0.548769in}}{\pgfqpoint{5.062926in}{3.102590in}}%
\pgfusepath{clip}%
\pgfsetbuttcap%
\pgfsetroundjoin%
\definecolor{currentfill}{rgb}{0.121569,0.466667,0.705882}%
\pgfsetfillcolor{currentfill}%
\pgfsetlinewidth{1.003750pt}%
\definecolor{currentstroke}{rgb}{0.121569,0.466667,0.705882}%
\pgfsetstrokecolor{currentstroke}%
\pgfsetdash{}{0pt}%
\pgfpathmoveto{\pgfqpoint{1.892479in}{0.825330in}}%
\pgfpathcurveto{\pgfqpoint{1.903529in}{0.825330in}}{\pgfqpoint{1.914128in}{0.829720in}}{\pgfqpoint{1.921942in}{0.837534in}}%
\pgfpathcurveto{\pgfqpoint{1.929755in}{0.845347in}}{\pgfqpoint{1.934145in}{0.855946in}}{\pgfqpoint{1.934145in}{0.866996in}}%
\pgfpathcurveto{\pgfqpoint{1.934145in}{0.878046in}}{\pgfqpoint{1.929755in}{0.888646in}}{\pgfqpoint{1.921942in}{0.896459in}}%
\pgfpathcurveto{\pgfqpoint{1.914128in}{0.904273in}}{\pgfqpoint{1.903529in}{0.908663in}}{\pgfqpoint{1.892479in}{0.908663in}}%
\pgfpathcurveto{\pgfqpoint{1.881429in}{0.908663in}}{\pgfqpoint{1.870830in}{0.904273in}}{\pgfqpoint{1.863016in}{0.896459in}}%
\pgfpathcurveto{\pgfqpoint{1.855202in}{0.888646in}}{\pgfqpoint{1.850812in}{0.878046in}}{\pgfqpoint{1.850812in}{0.866996in}}%
\pgfpathcurveto{\pgfqpoint{1.850812in}{0.855946in}}{\pgfqpoint{1.855202in}{0.845347in}}{\pgfqpoint{1.863016in}{0.837534in}}%
\pgfpathcurveto{\pgfqpoint{1.870830in}{0.829720in}}{\pgfqpoint{1.881429in}{0.825330in}}{\pgfqpoint{1.892479in}{0.825330in}}%
\pgfpathclose%
\pgfusepath{stroke,fill}%
\end{pgfscope}%
\begin{pgfscope}%
\pgfpathrectangle{\pgfqpoint{0.787074in}{0.548769in}}{\pgfqpoint{5.062926in}{3.102590in}}%
\pgfusepath{clip}%
\pgfsetbuttcap%
\pgfsetroundjoin%
\definecolor{currentfill}{rgb}{1.000000,0.498039,0.054902}%
\pgfsetfillcolor{currentfill}%
\pgfsetlinewidth{1.003750pt}%
\definecolor{currentstroke}{rgb}{1.000000,0.498039,0.054902}%
\pgfsetstrokecolor{currentstroke}%
\pgfsetdash{}{0pt}%
\pgfpathmoveto{\pgfqpoint{1.693633in}{1.508166in}}%
\pgfpathcurveto{\pgfqpoint{1.704683in}{1.508166in}}{\pgfqpoint{1.715282in}{1.512556in}}{\pgfqpoint{1.723096in}{1.520370in}}%
\pgfpathcurveto{\pgfqpoint{1.730909in}{1.528184in}}{\pgfqpoint{1.735299in}{1.538783in}}{\pgfqpoint{1.735299in}{1.549833in}}%
\pgfpathcurveto{\pgfqpoint{1.735299in}{1.560883in}}{\pgfqpoint{1.730909in}{1.571482in}}{\pgfqpoint{1.723096in}{1.579296in}}%
\pgfpathcurveto{\pgfqpoint{1.715282in}{1.587109in}}{\pgfqpoint{1.704683in}{1.591499in}}{\pgfqpoint{1.693633in}{1.591499in}}%
\pgfpathcurveto{\pgfqpoint{1.682583in}{1.591499in}}{\pgfqpoint{1.671984in}{1.587109in}}{\pgfqpoint{1.664170in}{1.579296in}}%
\pgfpathcurveto{\pgfqpoint{1.656356in}{1.571482in}}{\pgfqpoint{1.651966in}{1.560883in}}{\pgfqpoint{1.651966in}{1.549833in}}%
\pgfpathcurveto{\pgfqpoint{1.651966in}{1.538783in}}{\pgfqpoint{1.656356in}{1.528184in}}{\pgfqpoint{1.664170in}{1.520370in}}%
\pgfpathcurveto{\pgfqpoint{1.671984in}{1.512556in}}{\pgfqpoint{1.682583in}{1.508166in}}{\pgfqpoint{1.693633in}{1.508166in}}%
\pgfpathclose%
\pgfusepath{stroke,fill}%
\end{pgfscope}%
\begin{pgfscope}%
\pgfpathrectangle{\pgfqpoint{0.787074in}{0.548769in}}{\pgfqpoint{5.062926in}{3.102590in}}%
\pgfusepath{clip}%
\pgfsetbuttcap%
\pgfsetroundjoin%
\definecolor{currentfill}{rgb}{1.000000,0.498039,0.054902}%
\pgfsetfillcolor{currentfill}%
\pgfsetlinewidth{1.003750pt}%
\definecolor{currentstroke}{rgb}{1.000000,0.498039,0.054902}%
\pgfsetstrokecolor{currentstroke}%
\pgfsetdash{}{0pt}%
\pgfpathmoveto{\pgfqpoint{1.874402in}{2.241293in}}%
\pgfpathcurveto{\pgfqpoint{1.885452in}{2.241293in}}{\pgfqpoint{1.896051in}{2.245684in}}{\pgfqpoint{1.903865in}{2.253497in}}%
\pgfpathcurveto{\pgfqpoint{1.911678in}{2.261311in}}{\pgfqpoint{1.916069in}{2.271910in}}{\pgfqpoint{1.916069in}{2.282960in}}%
\pgfpathcurveto{\pgfqpoint{1.916069in}{2.294010in}}{\pgfqpoint{1.911678in}{2.304609in}}{\pgfqpoint{1.903865in}{2.312423in}}%
\pgfpathcurveto{\pgfqpoint{1.896051in}{2.320236in}}{\pgfqpoint{1.885452in}{2.324627in}}{\pgfqpoint{1.874402in}{2.324627in}}%
\pgfpathcurveto{\pgfqpoint{1.863352in}{2.324627in}}{\pgfqpoint{1.852753in}{2.320236in}}{\pgfqpoint{1.844939in}{2.312423in}}%
\pgfpathcurveto{\pgfqpoint{1.837125in}{2.304609in}}{\pgfqpoint{1.832735in}{2.294010in}}{\pgfqpoint{1.832735in}{2.282960in}}%
\pgfpathcurveto{\pgfqpoint{1.832735in}{2.271910in}}{\pgfqpoint{1.837125in}{2.261311in}}{\pgfqpoint{1.844939in}{2.253497in}}%
\pgfpathcurveto{\pgfqpoint{1.852753in}{2.245684in}}{\pgfqpoint{1.863352in}{2.241293in}}{\pgfqpoint{1.874402in}{2.241293in}}%
\pgfpathclose%
\pgfusepath{stroke,fill}%
\end{pgfscope}%
\begin{pgfscope}%
\pgfpathrectangle{\pgfqpoint{0.787074in}{0.548769in}}{\pgfqpoint{5.062926in}{3.102590in}}%
\pgfusepath{clip}%
\pgfsetbuttcap%
\pgfsetroundjoin%
\definecolor{currentfill}{rgb}{0.121569,0.466667,0.705882}%
\pgfsetfillcolor{currentfill}%
\pgfsetlinewidth{1.003750pt}%
\definecolor{currentstroke}{rgb}{0.121569,0.466667,0.705882}%
\pgfsetstrokecolor{currentstroke}%
\pgfsetdash{}{0pt}%
\pgfpathmoveto{\pgfqpoint{5.619867in}{0.648150in}}%
\pgfpathcurveto{\pgfqpoint{5.630917in}{0.648150in}}{\pgfqpoint{5.641516in}{0.652540in}}{\pgfqpoint{5.649330in}{0.660353in}}%
\pgfpathcurveto{\pgfqpoint{5.657143in}{0.668167in}}{\pgfqpoint{5.661534in}{0.678766in}}{\pgfqpoint{5.661534in}{0.689816in}}%
\pgfpathcurveto{\pgfqpoint{5.661534in}{0.700866in}}{\pgfqpoint{5.657143in}{0.711465in}}{\pgfqpoint{5.649330in}{0.719279in}}%
\pgfpathcurveto{\pgfqpoint{5.641516in}{0.727093in}}{\pgfqpoint{5.630917in}{0.731483in}}{\pgfqpoint{5.619867in}{0.731483in}}%
\pgfpathcurveto{\pgfqpoint{5.608817in}{0.731483in}}{\pgfqpoint{5.598218in}{0.727093in}}{\pgfqpoint{5.590404in}{0.719279in}}%
\pgfpathcurveto{\pgfqpoint{5.582591in}{0.711465in}}{\pgfqpoint{5.578200in}{0.700866in}}{\pgfqpoint{5.578200in}{0.689816in}}%
\pgfpathcurveto{\pgfqpoint{5.578200in}{0.678766in}}{\pgfqpoint{5.582591in}{0.668167in}}{\pgfqpoint{5.590404in}{0.660353in}}%
\pgfpathcurveto{\pgfqpoint{5.598218in}{0.652540in}}{\pgfqpoint{5.608817in}{0.648150in}}{\pgfqpoint{5.619867in}{0.648150in}}%
\pgfpathclose%
\pgfusepath{stroke,fill}%
\end{pgfscope}%
\begin{pgfscope}%
\pgfpathrectangle{\pgfqpoint{0.787074in}{0.548769in}}{\pgfqpoint{5.062926in}{3.102590in}}%
\pgfusepath{clip}%
\pgfsetbuttcap%
\pgfsetroundjoin%
\definecolor{currentfill}{rgb}{1.000000,0.498039,0.054902}%
\pgfsetfillcolor{currentfill}%
\pgfsetlinewidth{1.003750pt}%
\definecolor{currentstroke}{rgb}{1.000000,0.498039,0.054902}%
\pgfsetstrokecolor{currentstroke}%
\pgfsetdash{}{0pt}%
\pgfpathmoveto{\pgfqpoint{2.200108in}{2.050079in}}%
\pgfpathcurveto{\pgfqpoint{2.211159in}{2.050079in}}{\pgfqpoint{2.221758in}{2.054470in}}{\pgfqpoint{2.229571in}{2.062283in}}%
\pgfpathcurveto{\pgfqpoint{2.237385in}{2.070097in}}{\pgfqpoint{2.241775in}{2.080696in}}{\pgfqpoint{2.241775in}{2.091746in}}%
\pgfpathcurveto{\pgfqpoint{2.241775in}{2.102796in}}{\pgfqpoint{2.237385in}{2.113395in}}{\pgfqpoint{2.229571in}{2.121209in}}%
\pgfpathcurveto{\pgfqpoint{2.221758in}{2.129022in}}{\pgfqpoint{2.211159in}{2.133413in}}{\pgfqpoint{2.200108in}{2.133413in}}%
\pgfpathcurveto{\pgfqpoint{2.189058in}{2.133413in}}{\pgfqpoint{2.178459in}{2.129022in}}{\pgfqpoint{2.170646in}{2.121209in}}%
\pgfpathcurveto{\pgfqpoint{2.162832in}{2.113395in}}{\pgfqpoint{2.158442in}{2.102796in}}{\pgfqpoint{2.158442in}{2.091746in}}%
\pgfpathcurveto{\pgfqpoint{2.158442in}{2.080696in}}{\pgfqpoint{2.162832in}{2.070097in}}{\pgfqpoint{2.170646in}{2.062283in}}%
\pgfpathcurveto{\pgfqpoint{2.178459in}{2.054470in}}{\pgfqpoint{2.189058in}{2.050079in}}{\pgfqpoint{2.200108in}{2.050079in}}%
\pgfpathclose%
\pgfusepath{stroke,fill}%
\end{pgfscope}%
\begin{pgfscope}%
\pgfpathrectangle{\pgfqpoint{0.787074in}{0.548769in}}{\pgfqpoint{5.062926in}{3.102590in}}%
\pgfusepath{clip}%
\pgfsetbuttcap%
\pgfsetroundjoin%
\definecolor{currentfill}{rgb}{0.121569,0.466667,0.705882}%
\pgfsetfillcolor{currentfill}%
\pgfsetlinewidth{1.003750pt}%
\definecolor{currentstroke}{rgb}{0.121569,0.466667,0.705882}%
\pgfsetstrokecolor{currentstroke}%
\pgfsetdash{}{0pt}%
\pgfpathmoveto{\pgfqpoint{1.481534in}{0.648148in}}%
\pgfpathcurveto{\pgfqpoint{1.492584in}{0.648148in}}{\pgfqpoint{1.503183in}{0.652539in}}{\pgfqpoint{1.510997in}{0.660352in}}%
\pgfpathcurveto{\pgfqpoint{1.518811in}{0.668166in}}{\pgfqpoint{1.523201in}{0.678765in}}{\pgfqpoint{1.523201in}{0.689815in}}%
\pgfpathcurveto{\pgfqpoint{1.523201in}{0.700865in}}{\pgfqpoint{1.518811in}{0.711464in}}{\pgfqpoint{1.510997in}{0.719278in}}%
\pgfpathcurveto{\pgfqpoint{1.503183in}{0.727091in}}{\pgfqpoint{1.492584in}{0.731482in}}{\pgfqpoint{1.481534in}{0.731482in}}%
\pgfpathcurveto{\pgfqpoint{1.470484in}{0.731482in}}{\pgfqpoint{1.459885in}{0.727091in}}{\pgfqpoint{1.452071in}{0.719278in}}%
\pgfpathcurveto{\pgfqpoint{1.444258in}{0.711464in}}{\pgfqpoint{1.439868in}{0.700865in}}{\pgfqpoint{1.439868in}{0.689815in}}%
\pgfpathcurveto{\pgfqpoint{1.439868in}{0.678765in}}{\pgfqpoint{1.444258in}{0.668166in}}{\pgfqpoint{1.452071in}{0.660352in}}%
\pgfpathcurveto{\pgfqpoint{1.459885in}{0.652539in}}{\pgfqpoint{1.470484in}{0.648148in}}{\pgfqpoint{1.481534in}{0.648148in}}%
\pgfpathclose%
\pgfusepath{stroke,fill}%
\end{pgfscope}%
\begin{pgfscope}%
\pgfpathrectangle{\pgfqpoint{0.787074in}{0.548769in}}{\pgfqpoint{5.062926in}{3.102590in}}%
\pgfusepath{clip}%
\pgfsetbuttcap%
\pgfsetroundjoin%
\definecolor{currentfill}{rgb}{0.121569,0.466667,0.705882}%
\pgfsetfillcolor{currentfill}%
\pgfsetlinewidth{1.003750pt}%
\definecolor{currentstroke}{rgb}{0.121569,0.466667,0.705882}%
\pgfsetstrokecolor{currentstroke}%
\pgfsetdash{}{0pt}%
\pgfpathmoveto{\pgfqpoint{1.269368in}{0.648168in}}%
\pgfpathcurveto{\pgfqpoint{1.280418in}{0.648168in}}{\pgfqpoint{1.291017in}{0.652558in}}{\pgfqpoint{1.298831in}{0.660372in}}%
\pgfpathcurveto{\pgfqpoint{1.306644in}{0.668185in}}{\pgfqpoint{1.311035in}{0.678784in}}{\pgfqpoint{1.311035in}{0.689834in}}%
\pgfpathcurveto{\pgfqpoint{1.311035in}{0.700885in}}{\pgfqpoint{1.306644in}{0.711484in}}{\pgfqpoint{1.298831in}{0.719297in}}%
\pgfpathcurveto{\pgfqpoint{1.291017in}{0.727111in}}{\pgfqpoint{1.280418in}{0.731501in}}{\pgfqpoint{1.269368in}{0.731501in}}%
\pgfpathcurveto{\pgfqpoint{1.258318in}{0.731501in}}{\pgfqpoint{1.247719in}{0.727111in}}{\pgfqpoint{1.239905in}{0.719297in}}%
\pgfpathcurveto{\pgfqpoint{1.232092in}{0.711484in}}{\pgfqpoint{1.227701in}{0.700885in}}{\pgfqpoint{1.227701in}{0.689834in}}%
\pgfpathcurveto{\pgfqpoint{1.227701in}{0.678784in}}{\pgfqpoint{1.232092in}{0.668185in}}{\pgfqpoint{1.239905in}{0.660372in}}%
\pgfpathcurveto{\pgfqpoint{1.247719in}{0.652558in}}{\pgfqpoint{1.258318in}{0.648168in}}{\pgfqpoint{1.269368in}{0.648168in}}%
\pgfpathclose%
\pgfusepath{stroke,fill}%
\end{pgfscope}%
\begin{pgfscope}%
\pgfpathrectangle{\pgfqpoint{0.787074in}{0.548769in}}{\pgfqpoint{5.062926in}{3.102590in}}%
\pgfusepath{clip}%
\pgfsetbuttcap%
\pgfsetroundjoin%
\definecolor{currentfill}{rgb}{0.121569,0.466667,0.705882}%
\pgfsetfillcolor{currentfill}%
\pgfsetlinewidth{1.003750pt}%
\definecolor{currentstroke}{rgb}{0.121569,0.466667,0.705882}%
\pgfsetstrokecolor{currentstroke}%
\pgfsetdash{}{0pt}%
\pgfpathmoveto{\pgfqpoint{1.491361in}{0.861755in}}%
\pgfpathcurveto{\pgfqpoint{1.502411in}{0.861755in}}{\pgfqpoint{1.513010in}{0.866145in}}{\pgfqpoint{1.520824in}{0.873958in}}%
\pgfpathcurveto{\pgfqpoint{1.528638in}{0.881772in}}{\pgfqpoint{1.533028in}{0.892371in}}{\pgfqpoint{1.533028in}{0.903421in}}%
\pgfpathcurveto{\pgfqpoint{1.533028in}{0.914471in}}{\pgfqpoint{1.528638in}{0.925070in}}{\pgfqpoint{1.520824in}{0.932884in}}%
\pgfpathcurveto{\pgfqpoint{1.513010in}{0.940698in}}{\pgfqpoint{1.502411in}{0.945088in}}{\pgfqpoint{1.491361in}{0.945088in}}%
\pgfpathcurveto{\pgfqpoint{1.480311in}{0.945088in}}{\pgfqpoint{1.469712in}{0.940698in}}{\pgfqpoint{1.461898in}{0.932884in}}%
\pgfpathcurveto{\pgfqpoint{1.454085in}{0.925070in}}{\pgfqpoint{1.449695in}{0.914471in}}{\pgfqpoint{1.449695in}{0.903421in}}%
\pgfpathcurveto{\pgfqpoint{1.449695in}{0.892371in}}{\pgfqpoint{1.454085in}{0.881772in}}{\pgfqpoint{1.461898in}{0.873958in}}%
\pgfpathcurveto{\pgfqpoint{1.469712in}{0.866145in}}{\pgfqpoint{1.480311in}{0.861755in}}{\pgfqpoint{1.491361in}{0.861755in}}%
\pgfpathclose%
\pgfusepath{stroke,fill}%
\end{pgfscope}%
\begin{pgfscope}%
\pgfpathrectangle{\pgfqpoint{0.787074in}{0.548769in}}{\pgfqpoint{5.062926in}{3.102590in}}%
\pgfusepath{clip}%
\pgfsetbuttcap%
\pgfsetroundjoin%
\definecolor{currentfill}{rgb}{1.000000,0.498039,0.054902}%
\pgfsetfillcolor{currentfill}%
\pgfsetlinewidth{1.003750pt}%
\definecolor{currentstroke}{rgb}{1.000000,0.498039,0.054902}%
\pgfsetstrokecolor{currentstroke}%
\pgfsetdash{}{0pt}%
\pgfpathmoveto{\pgfqpoint{2.366192in}{1.953744in}}%
\pgfpathcurveto{\pgfqpoint{2.377242in}{1.953744in}}{\pgfqpoint{2.387841in}{1.958135in}}{\pgfqpoint{2.395655in}{1.965948in}}%
\pgfpathcurveto{\pgfqpoint{2.403469in}{1.973762in}}{\pgfqpoint{2.407859in}{1.984361in}}{\pgfqpoint{2.407859in}{1.995411in}}%
\pgfpathcurveto{\pgfqpoint{2.407859in}{2.006461in}}{\pgfqpoint{2.403469in}{2.017060in}}{\pgfqpoint{2.395655in}{2.024874in}}%
\pgfpathcurveto{\pgfqpoint{2.387841in}{2.032687in}}{\pgfqpoint{2.377242in}{2.037078in}}{\pgfqpoint{2.366192in}{2.037078in}}%
\pgfpathcurveto{\pgfqpoint{2.355142in}{2.037078in}}{\pgfqpoint{2.344543in}{2.032687in}}{\pgfqpoint{2.336729in}{2.024874in}}%
\pgfpathcurveto{\pgfqpoint{2.328916in}{2.017060in}}{\pgfqpoint{2.324526in}{2.006461in}}{\pgfqpoint{2.324526in}{1.995411in}}%
\pgfpathcurveto{\pgfqpoint{2.324526in}{1.984361in}}{\pgfqpoint{2.328916in}{1.973762in}}{\pgfqpoint{2.336729in}{1.965948in}}%
\pgfpathcurveto{\pgfqpoint{2.344543in}{1.958135in}}{\pgfqpoint{2.355142in}{1.953744in}}{\pgfqpoint{2.366192in}{1.953744in}}%
\pgfpathclose%
\pgfusepath{stroke,fill}%
\end{pgfscope}%
\begin{pgfscope}%
\pgfpathrectangle{\pgfqpoint{0.787074in}{0.548769in}}{\pgfqpoint{5.062926in}{3.102590in}}%
\pgfusepath{clip}%
\pgfsetbuttcap%
\pgfsetroundjoin%
\definecolor{currentfill}{rgb}{1.000000,0.498039,0.054902}%
\pgfsetfillcolor{currentfill}%
\pgfsetlinewidth{1.003750pt}%
\definecolor{currentstroke}{rgb}{1.000000,0.498039,0.054902}%
\pgfsetstrokecolor{currentstroke}%
\pgfsetdash{}{0pt}%
\pgfpathmoveto{\pgfqpoint{1.452511in}{1.401411in}}%
\pgfpathcurveto{\pgfqpoint{1.463561in}{1.401411in}}{\pgfqpoint{1.474160in}{1.405801in}}{\pgfqpoint{1.481974in}{1.413615in}}%
\pgfpathcurveto{\pgfqpoint{1.489788in}{1.421429in}}{\pgfqpoint{1.494178in}{1.432028in}}{\pgfqpoint{1.494178in}{1.443078in}}%
\pgfpathcurveto{\pgfqpoint{1.494178in}{1.454128in}}{\pgfqpoint{1.489788in}{1.464727in}}{\pgfqpoint{1.481974in}{1.472540in}}%
\pgfpathcurveto{\pgfqpoint{1.474160in}{1.480354in}}{\pgfqpoint{1.463561in}{1.484744in}}{\pgfqpoint{1.452511in}{1.484744in}}%
\pgfpathcurveto{\pgfqpoint{1.441461in}{1.484744in}}{\pgfqpoint{1.430862in}{1.480354in}}{\pgfqpoint{1.423048in}{1.472540in}}%
\pgfpathcurveto{\pgfqpoint{1.415235in}{1.464727in}}{\pgfqpoint{1.410844in}{1.454128in}}{\pgfqpoint{1.410844in}{1.443078in}}%
\pgfpathcurveto{\pgfqpoint{1.410844in}{1.432028in}}{\pgfqpoint{1.415235in}{1.421429in}}{\pgfqpoint{1.423048in}{1.413615in}}%
\pgfpathcurveto{\pgfqpoint{1.430862in}{1.405801in}}{\pgfqpoint{1.441461in}{1.401411in}}{\pgfqpoint{1.452511in}{1.401411in}}%
\pgfpathclose%
\pgfusepath{stroke,fill}%
\end{pgfscope}%
\begin{pgfscope}%
\pgfpathrectangle{\pgfqpoint{0.787074in}{0.548769in}}{\pgfqpoint{5.062926in}{3.102590in}}%
\pgfusepath{clip}%
\pgfsetbuttcap%
\pgfsetroundjoin%
\definecolor{currentfill}{rgb}{0.121569,0.466667,0.705882}%
\pgfsetfillcolor{currentfill}%
\pgfsetlinewidth{1.003750pt}%
\definecolor{currentstroke}{rgb}{0.121569,0.466667,0.705882}%
\pgfsetstrokecolor{currentstroke}%
\pgfsetdash{}{0pt}%
\pgfpathmoveto{\pgfqpoint{1.400197in}{0.648141in}}%
\pgfpathcurveto{\pgfqpoint{1.411247in}{0.648141in}}{\pgfqpoint{1.421846in}{0.652531in}}{\pgfqpoint{1.429659in}{0.660344in}}%
\pgfpathcurveto{\pgfqpoint{1.437473in}{0.668158in}}{\pgfqpoint{1.441863in}{0.678757in}}{\pgfqpoint{1.441863in}{0.689807in}}%
\pgfpathcurveto{\pgfqpoint{1.441863in}{0.700857in}}{\pgfqpoint{1.437473in}{0.711456in}}{\pgfqpoint{1.429659in}{0.719270in}}%
\pgfpathcurveto{\pgfqpoint{1.421846in}{0.727084in}}{\pgfqpoint{1.411247in}{0.731474in}}{\pgfqpoint{1.400197in}{0.731474in}}%
\pgfpathcurveto{\pgfqpoint{1.389146in}{0.731474in}}{\pgfqpoint{1.378547in}{0.727084in}}{\pgfqpoint{1.370734in}{0.719270in}}%
\pgfpathcurveto{\pgfqpoint{1.362920in}{0.711456in}}{\pgfqpoint{1.358530in}{0.700857in}}{\pgfqpoint{1.358530in}{0.689807in}}%
\pgfpathcurveto{\pgfqpoint{1.358530in}{0.678757in}}{\pgfqpoint{1.362920in}{0.668158in}}{\pgfqpoint{1.370734in}{0.660344in}}%
\pgfpathcurveto{\pgfqpoint{1.378547in}{0.652531in}}{\pgfqpoint{1.389146in}{0.648141in}}{\pgfqpoint{1.400197in}{0.648141in}}%
\pgfpathclose%
\pgfusepath{stroke,fill}%
\end{pgfscope}%
\begin{pgfscope}%
\pgfpathrectangle{\pgfqpoint{0.787074in}{0.548769in}}{\pgfqpoint{5.062926in}{3.102590in}}%
\pgfusepath{clip}%
\pgfsetbuttcap%
\pgfsetroundjoin%
\definecolor{currentfill}{rgb}{0.121569,0.466667,0.705882}%
\pgfsetfillcolor{currentfill}%
\pgfsetlinewidth{1.003750pt}%
\definecolor{currentstroke}{rgb}{0.121569,0.466667,0.705882}%
\pgfsetstrokecolor{currentstroke}%
\pgfsetdash{}{0pt}%
\pgfpathmoveto{\pgfqpoint{1.293363in}{0.787322in}}%
\pgfpathcurveto{\pgfqpoint{1.304413in}{0.787322in}}{\pgfqpoint{1.315012in}{0.791712in}}{\pgfqpoint{1.322826in}{0.799526in}}%
\pgfpathcurveto{\pgfqpoint{1.330639in}{0.807339in}}{\pgfqpoint{1.335030in}{0.817938in}}{\pgfqpoint{1.335030in}{0.828988in}}%
\pgfpathcurveto{\pgfqpoint{1.335030in}{0.840039in}}{\pgfqpoint{1.330639in}{0.850638in}}{\pgfqpoint{1.322826in}{0.858451in}}%
\pgfpathcurveto{\pgfqpoint{1.315012in}{0.866265in}}{\pgfqpoint{1.304413in}{0.870655in}}{\pgfqpoint{1.293363in}{0.870655in}}%
\pgfpathcurveto{\pgfqpoint{1.282313in}{0.870655in}}{\pgfqpoint{1.271714in}{0.866265in}}{\pgfqpoint{1.263900in}{0.858451in}}%
\pgfpathcurveto{\pgfqpoint{1.256087in}{0.850638in}}{\pgfqpoint{1.251696in}{0.840039in}}{\pgfqpoint{1.251696in}{0.828988in}}%
\pgfpathcurveto{\pgfqpoint{1.251696in}{0.817938in}}{\pgfqpoint{1.256087in}{0.807339in}}{\pgfqpoint{1.263900in}{0.799526in}}%
\pgfpathcurveto{\pgfqpoint{1.271714in}{0.791712in}}{\pgfqpoint{1.282313in}{0.787322in}}{\pgfqpoint{1.293363in}{0.787322in}}%
\pgfpathclose%
\pgfusepath{stroke,fill}%
\end{pgfscope}%
\begin{pgfscope}%
\pgfpathrectangle{\pgfqpoint{0.787074in}{0.548769in}}{\pgfqpoint{5.062926in}{3.102590in}}%
\pgfusepath{clip}%
\pgfsetbuttcap%
\pgfsetroundjoin%
\definecolor{currentfill}{rgb}{1.000000,0.498039,0.054902}%
\pgfsetfillcolor{currentfill}%
\pgfsetlinewidth{1.003750pt}%
\definecolor{currentstroke}{rgb}{1.000000,0.498039,0.054902}%
\pgfsetstrokecolor{currentstroke}%
\pgfsetdash{}{0pt}%
\pgfpathmoveto{\pgfqpoint{2.064557in}{1.729931in}}%
\pgfpathcurveto{\pgfqpoint{2.075607in}{1.729931in}}{\pgfqpoint{2.086206in}{1.734321in}}{\pgfqpoint{2.094020in}{1.742135in}}%
\pgfpathcurveto{\pgfqpoint{2.101833in}{1.749948in}}{\pgfqpoint{2.106224in}{1.760547in}}{\pgfqpoint{2.106224in}{1.771597in}}%
\pgfpathcurveto{\pgfqpoint{2.106224in}{1.782647in}}{\pgfqpoint{2.101833in}{1.793246in}}{\pgfqpoint{2.094020in}{1.801060in}}%
\pgfpathcurveto{\pgfqpoint{2.086206in}{1.808874in}}{\pgfqpoint{2.075607in}{1.813264in}}{\pgfqpoint{2.064557in}{1.813264in}}%
\pgfpathcurveto{\pgfqpoint{2.053507in}{1.813264in}}{\pgfqpoint{2.042908in}{1.808874in}}{\pgfqpoint{2.035094in}{1.801060in}}%
\pgfpathcurveto{\pgfqpoint{2.027281in}{1.793246in}}{\pgfqpoint{2.022890in}{1.782647in}}{\pgfqpoint{2.022890in}{1.771597in}}%
\pgfpathcurveto{\pgfqpoint{2.022890in}{1.760547in}}{\pgfqpoint{2.027281in}{1.749948in}}{\pgfqpoint{2.035094in}{1.742135in}}%
\pgfpathcurveto{\pgfqpoint{2.042908in}{1.734321in}}{\pgfqpoint{2.053507in}{1.729931in}}{\pgfqpoint{2.064557in}{1.729931in}}%
\pgfpathclose%
\pgfusepath{stroke,fill}%
\end{pgfscope}%
\begin{pgfscope}%
\pgfpathrectangle{\pgfqpoint{0.787074in}{0.548769in}}{\pgfqpoint{5.062926in}{3.102590in}}%
\pgfusepath{clip}%
\pgfsetbuttcap%
\pgfsetroundjoin%
\definecolor{currentfill}{rgb}{0.121569,0.466667,0.705882}%
\pgfsetfillcolor{currentfill}%
\pgfsetlinewidth{1.003750pt}%
\definecolor{currentstroke}{rgb}{0.121569,0.466667,0.705882}%
\pgfsetstrokecolor{currentstroke}%
\pgfsetdash{}{0pt}%
\pgfpathmoveto{\pgfqpoint{1.113086in}{0.648129in}}%
\pgfpathcurveto{\pgfqpoint{1.124136in}{0.648129in}}{\pgfqpoint{1.134735in}{0.652519in}}{\pgfqpoint{1.142548in}{0.660333in}}%
\pgfpathcurveto{\pgfqpoint{1.150362in}{0.668146in}}{\pgfqpoint{1.154752in}{0.678745in}}{\pgfqpoint{1.154752in}{0.689796in}}%
\pgfpathcurveto{\pgfqpoint{1.154752in}{0.700846in}}{\pgfqpoint{1.150362in}{0.711445in}}{\pgfqpoint{1.142548in}{0.719258in}}%
\pgfpathcurveto{\pgfqpoint{1.134735in}{0.727072in}}{\pgfqpoint{1.124136in}{0.731462in}}{\pgfqpoint{1.113086in}{0.731462in}}%
\pgfpathcurveto{\pgfqpoint{1.102036in}{0.731462in}}{\pgfqpoint{1.091437in}{0.727072in}}{\pgfqpoint{1.083623in}{0.719258in}}%
\pgfpathcurveto{\pgfqpoint{1.075809in}{0.711445in}}{\pgfqpoint{1.071419in}{0.700846in}}{\pgfqpoint{1.071419in}{0.689796in}}%
\pgfpathcurveto{\pgfqpoint{1.071419in}{0.678745in}}{\pgfqpoint{1.075809in}{0.668146in}}{\pgfqpoint{1.083623in}{0.660333in}}%
\pgfpathcurveto{\pgfqpoint{1.091437in}{0.652519in}}{\pgfqpoint{1.102036in}{0.648129in}}{\pgfqpoint{1.113086in}{0.648129in}}%
\pgfpathclose%
\pgfusepath{stroke,fill}%
\end{pgfscope}%
\begin{pgfscope}%
\pgfpathrectangle{\pgfqpoint{0.787074in}{0.548769in}}{\pgfqpoint{5.062926in}{3.102590in}}%
\pgfusepath{clip}%
\pgfsetbuttcap%
\pgfsetroundjoin%
\definecolor{currentfill}{rgb}{1.000000,0.498039,0.054902}%
\pgfsetfillcolor{currentfill}%
\pgfsetlinewidth{1.003750pt}%
\definecolor{currentstroke}{rgb}{1.000000,0.498039,0.054902}%
\pgfsetstrokecolor{currentstroke}%
\pgfsetdash{}{0pt}%
\pgfpathmoveto{\pgfqpoint{1.823376in}{1.417750in}}%
\pgfpathcurveto{\pgfqpoint{1.834426in}{1.417750in}}{\pgfqpoint{1.845025in}{1.422140in}}{\pgfqpoint{1.852839in}{1.429954in}}%
\pgfpathcurveto{\pgfqpoint{1.860653in}{1.437768in}}{\pgfqpoint{1.865043in}{1.448367in}}{\pgfqpoint{1.865043in}{1.459417in}}%
\pgfpathcurveto{\pgfqpoint{1.865043in}{1.470467in}}{\pgfqpoint{1.860653in}{1.481066in}}{\pgfqpoint{1.852839in}{1.488880in}}%
\pgfpathcurveto{\pgfqpoint{1.845025in}{1.496693in}}{\pgfqpoint{1.834426in}{1.501083in}}{\pgfqpoint{1.823376in}{1.501083in}}%
\pgfpathcurveto{\pgfqpoint{1.812326in}{1.501083in}}{\pgfqpoint{1.801727in}{1.496693in}}{\pgfqpoint{1.793913in}{1.488880in}}%
\pgfpathcurveto{\pgfqpoint{1.786100in}{1.481066in}}{\pgfqpoint{1.781709in}{1.470467in}}{\pgfqpoint{1.781709in}{1.459417in}}%
\pgfpathcurveto{\pgfqpoint{1.781709in}{1.448367in}}{\pgfqpoint{1.786100in}{1.437768in}}{\pgfqpoint{1.793913in}{1.429954in}}%
\pgfpathcurveto{\pgfqpoint{1.801727in}{1.422140in}}{\pgfqpoint{1.812326in}{1.417750in}}{\pgfqpoint{1.823376in}{1.417750in}}%
\pgfpathclose%
\pgfusepath{stroke,fill}%
\end{pgfscope}%
\begin{pgfscope}%
\pgfpathrectangle{\pgfqpoint{0.787074in}{0.548769in}}{\pgfqpoint{5.062926in}{3.102590in}}%
\pgfusepath{clip}%
\pgfsetbuttcap%
\pgfsetroundjoin%
\definecolor{currentfill}{rgb}{1.000000,0.498039,0.054902}%
\pgfsetfillcolor{currentfill}%
\pgfsetlinewidth{1.003750pt}%
\definecolor{currentstroke}{rgb}{1.000000,0.498039,0.054902}%
\pgfsetstrokecolor{currentstroke}%
\pgfsetdash{}{0pt}%
\pgfpathmoveto{\pgfqpoint{1.699526in}{2.301800in}}%
\pgfpathcurveto{\pgfqpoint{1.710576in}{2.301800in}}{\pgfqpoint{1.721175in}{2.306191in}}{\pgfqpoint{1.728988in}{2.314004in}}%
\pgfpathcurveto{\pgfqpoint{1.736802in}{2.321818in}}{\pgfqpoint{1.741192in}{2.332417in}}{\pgfqpoint{1.741192in}{2.343467in}}%
\pgfpathcurveto{\pgfqpoint{1.741192in}{2.354517in}}{\pgfqpoint{1.736802in}{2.365116in}}{\pgfqpoint{1.728988in}{2.372930in}}%
\pgfpathcurveto{\pgfqpoint{1.721175in}{2.380743in}}{\pgfqpoint{1.710576in}{2.385134in}}{\pgfqpoint{1.699526in}{2.385134in}}%
\pgfpathcurveto{\pgfqpoint{1.688475in}{2.385134in}}{\pgfqpoint{1.677876in}{2.380743in}}{\pgfqpoint{1.670063in}{2.372930in}}%
\pgfpathcurveto{\pgfqpoint{1.662249in}{2.365116in}}{\pgfqpoint{1.657859in}{2.354517in}}{\pgfqpoint{1.657859in}{2.343467in}}%
\pgfpathcurveto{\pgfqpoint{1.657859in}{2.332417in}}{\pgfqpoint{1.662249in}{2.321818in}}{\pgfqpoint{1.670063in}{2.314004in}}%
\pgfpathcurveto{\pgfqpoint{1.677876in}{2.306191in}}{\pgfqpoint{1.688475in}{2.301800in}}{\pgfqpoint{1.699526in}{2.301800in}}%
\pgfpathclose%
\pgfusepath{stroke,fill}%
\end{pgfscope}%
\begin{pgfscope}%
\pgfpathrectangle{\pgfqpoint{0.787074in}{0.548769in}}{\pgfqpoint{5.062926in}{3.102590in}}%
\pgfusepath{clip}%
\pgfsetbuttcap%
\pgfsetroundjoin%
\definecolor{currentfill}{rgb}{0.121569,0.466667,0.705882}%
\pgfsetfillcolor{currentfill}%
\pgfsetlinewidth{1.003750pt}%
\definecolor{currentstroke}{rgb}{0.121569,0.466667,0.705882}%
\pgfsetstrokecolor{currentstroke}%
\pgfsetdash{}{0pt}%
\pgfpathmoveto{\pgfqpoint{2.079802in}{0.652202in}}%
\pgfpathcurveto{\pgfqpoint{2.090852in}{0.652202in}}{\pgfqpoint{2.101451in}{0.656593in}}{\pgfqpoint{2.109265in}{0.664406in}}%
\pgfpathcurveto{\pgfqpoint{2.117078in}{0.672220in}}{\pgfqpoint{2.121469in}{0.682819in}}{\pgfqpoint{2.121469in}{0.693869in}}%
\pgfpathcurveto{\pgfqpoint{2.121469in}{0.704919in}}{\pgfqpoint{2.117078in}{0.715518in}}{\pgfqpoint{2.109265in}{0.723332in}}%
\pgfpathcurveto{\pgfqpoint{2.101451in}{0.731145in}}{\pgfqpoint{2.090852in}{0.735536in}}{\pgfqpoint{2.079802in}{0.735536in}}%
\pgfpathcurveto{\pgfqpoint{2.068752in}{0.735536in}}{\pgfqpoint{2.058153in}{0.731145in}}{\pgfqpoint{2.050339in}{0.723332in}}%
\pgfpathcurveto{\pgfqpoint{2.042526in}{0.715518in}}{\pgfqpoint{2.038135in}{0.704919in}}{\pgfqpoint{2.038135in}{0.693869in}}%
\pgfpathcurveto{\pgfqpoint{2.038135in}{0.682819in}}{\pgfqpoint{2.042526in}{0.672220in}}{\pgfqpoint{2.050339in}{0.664406in}}%
\pgfpathcurveto{\pgfqpoint{2.058153in}{0.656593in}}{\pgfqpoint{2.068752in}{0.652202in}}{\pgfqpoint{2.079802in}{0.652202in}}%
\pgfpathclose%
\pgfusepath{stroke,fill}%
\end{pgfscope}%
\begin{pgfscope}%
\pgfpathrectangle{\pgfqpoint{0.787074in}{0.548769in}}{\pgfqpoint{5.062926in}{3.102590in}}%
\pgfusepath{clip}%
\pgfsetbuttcap%
\pgfsetroundjoin%
\definecolor{currentfill}{rgb}{0.121569,0.466667,0.705882}%
\pgfsetfillcolor{currentfill}%
\pgfsetlinewidth{1.003750pt}%
\definecolor{currentstroke}{rgb}{0.121569,0.466667,0.705882}%
\pgfsetstrokecolor{currentstroke}%
\pgfsetdash{}{0pt}%
\pgfpathmoveto{\pgfqpoint{2.485388in}{0.648180in}}%
\pgfpathcurveto{\pgfqpoint{2.496438in}{0.648180in}}{\pgfqpoint{2.507037in}{0.652571in}}{\pgfqpoint{2.514851in}{0.660384in}}%
\pgfpathcurveto{\pgfqpoint{2.522664in}{0.668198in}}{\pgfqpoint{2.527055in}{0.678797in}}{\pgfqpoint{2.527055in}{0.689847in}}%
\pgfpathcurveto{\pgfqpoint{2.527055in}{0.700897in}}{\pgfqpoint{2.522664in}{0.711496in}}{\pgfqpoint{2.514851in}{0.719310in}}%
\pgfpathcurveto{\pgfqpoint{2.507037in}{0.727123in}}{\pgfqpoint{2.496438in}{0.731514in}}{\pgfqpoint{2.485388in}{0.731514in}}%
\pgfpathcurveto{\pgfqpoint{2.474338in}{0.731514in}}{\pgfqpoint{2.463739in}{0.727123in}}{\pgfqpoint{2.455925in}{0.719310in}}%
\pgfpathcurveto{\pgfqpoint{2.448112in}{0.711496in}}{\pgfqpoint{2.443721in}{0.700897in}}{\pgfqpoint{2.443721in}{0.689847in}}%
\pgfpathcurveto{\pgfqpoint{2.443721in}{0.678797in}}{\pgfqpoint{2.448112in}{0.668198in}}{\pgfqpoint{2.455925in}{0.660384in}}%
\pgfpathcurveto{\pgfqpoint{2.463739in}{0.652571in}}{\pgfqpoint{2.474338in}{0.648180in}}{\pgfqpoint{2.485388in}{0.648180in}}%
\pgfpathclose%
\pgfusepath{stroke,fill}%
\end{pgfscope}%
\begin{pgfscope}%
\pgfpathrectangle{\pgfqpoint{0.787074in}{0.548769in}}{\pgfqpoint{5.062926in}{3.102590in}}%
\pgfusepath{clip}%
\pgfsetbuttcap%
\pgfsetroundjoin%
\definecolor{currentfill}{rgb}{0.121569,0.466667,0.705882}%
\pgfsetfillcolor{currentfill}%
\pgfsetlinewidth{1.003750pt}%
\definecolor{currentstroke}{rgb}{0.121569,0.466667,0.705882}%
\pgfsetstrokecolor{currentstroke}%
\pgfsetdash{}{0pt}%
\pgfpathmoveto{\pgfqpoint{1.408777in}{0.648153in}}%
\pgfpathcurveto{\pgfqpoint{1.419827in}{0.648153in}}{\pgfqpoint{1.430426in}{0.652543in}}{\pgfqpoint{1.438240in}{0.660357in}}%
\pgfpathcurveto{\pgfqpoint{1.446054in}{0.668170in}}{\pgfqpoint{1.450444in}{0.678769in}}{\pgfqpoint{1.450444in}{0.689819in}}%
\pgfpathcurveto{\pgfqpoint{1.450444in}{0.700870in}}{\pgfqpoint{1.446054in}{0.711469in}}{\pgfqpoint{1.438240in}{0.719282in}}%
\pgfpathcurveto{\pgfqpoint{1.430426in}{0.727096in}}{\pgfqpoint{1.419827in}{0.731486in}}{\pgfqpoint{1.408777in}{0.731486in}}%
\pgfpathcurveto{\pgfqpoint{1.397727in}{0.731486in}}{\pgfqpoint{1.387128in}{0.727096in}}{\pgfqpoint{1.379314in}{0.719282in}}%
\pgfpathcurveto{\pgfqpoint{1.371501in}{0.711469in}}{\pgfqpoint{1.367111in}{0.700870in}}{\pgfqpoint{1.367111in}{0.689819in}}%
\pgfpathcurveto{\pgfqpoint{1.367111in}{0.678769in}}{\pgfqpoint{1.371501in}{0.668170in}}{\pgfqpoint{1.379314in}{0.660357in}}%
\pgfpathcurveto{\pgfqpoint{1.387128in}{0.652543in}}{\pgfqpoint{1.397727in}{0.648153in}}{\pgfqpoint{1.408777in}{0.648153in}}%
\pgfpathclose%
\pgfusepath{stroke,fill}%
\end{pgfscope}%
\begin{pgfscope}%
\pgfpathrectangle{\pgfqpoint{0.787074in}{0.548769in}}{\pgfqpoint{5.062926in}{3.102590in}}%
\pgfusepath{clip}%
\pgfsetbuttcap%
\pgfsetroundjoin%
\definecolor{currentfill}{rgb}{0.121569,0.466667,0.705882}%
\pgfsetfillcolor{currentfill}%
\pgfsetlinewidth{1.003750pt}%
\definecolor{currentstroke}{rgb}{0.121569,0.466667,0.705882}%
\pgfsetstrokecolor{currentstroke}%
\pgfsetdash{}{0pt}%
\pgfpathmoveto{\pgfqpoint{1.569367in}{0.648456in}}%
\pgfpathcurveto{\pgfqpoint{1.580417in}{0.648456in}}{\pgfqpoint{1.591016in}{0.652846in}}{\pgfqpoint{1.598829in}{0.660660in}}%
\pgfpathcurveto{\pgfqpoint{1.606643in}{0.668474in}}{\pgfqpoint{1.611033in}{0.679073in}}{\pgfqpoint{1.611033in}{0.690123in}}%
\pgfpathcurveto{\pgfqpoint{1.611033in}{0.701173in}}{\pgfqpoint{1.606643in}{0.711772in}}{\pgfqpoint{1.598829in}{0.719586in}}%
\pgfpathcurveto{\pgfqpoint{1.591016in}{0.727399in}}{\pgfqpoint{1.580417in}{0.731789in}}{\pgfqpoint{1.569367in}{0.731789in}}%
\pgfpathcurveto{\pgfqpoint{1.558317in}{0.731789in}}{\pgfqpoint{1.547717in}{0.727399in}}{\pgfqpoint{1.539904in}{0.719586in}}%
\pgfpathcurveto{\pgfqpoint{1.532090in}{0.711772in}}{\pgfqpoint{1.527700in}{0.701173in}}{\pgfqpoint{1.527700in}{0.690123in}}%
\pgfpathcurveto{\pgfqpoint{1.527700in}{0.679073in}}{\pgfqpoint{1.532090in}{0.668474in}}{\pgfqpoint{1.539904in}{0.660660in}}%
\pgfpathcurveto{\pgfqpoint{1.547717in}{0.652846in}}{\pgfqpoint{1.558317in}{0.648456in}}{\pgfqpoint{1.569367in}{0.648456in}}%
\pgfpathclose%
\pgfusepath{stroke,fill}%
\end{pgfscope}%
\begin{pgfscope}%
\pgfpathrectangle{\pgfqpoint{0.787074in}{0.548769in}}{\pgfqpoint{5.062926in}{3.102590in}}%
\pgfusepath{clip}%
\pgfsetbuttcap%
\pgfsetroundjoin%
\definecolor{currentfill}{rgb}{0.121569,0.466667,0.705882}%
\pgfsetfillcolor{currentfill}%
\pgfsetlinewidth{1.003750pt}%
\definecolor{currentstroke}{rgb}{0.121569,0.466667,0.705882}%
\pgfsetstrokecolor{currentstroke}%
\pgfsetdash{}{0pt}%
\pgfpathmoveto{\pgfqpoint{1.352088in}{0.648200in}}%
\pgfpathcurveto{\pgfqpoint{1.363138in}{0.648200in}}{\pgfqpoint{1.373737in}{0.652590in}}{\pgfqpoint{1.381550in}{0.660404in}}%
\pgfpathcurveto{\pgfqpoint{1.389364in}{0.668217in}}{\pgfqpoint{1.393754in}{0.678816in}}{\pgfqpoint{1.393754in}{0.689866in}}%
\pgfpathcurveto{\pgfqpoint{1.393754in}{0.700917in}}{\pgfqpoint{1.389364in}{0.711516in}}{\pgfqpoint{1.381550in}{0.719329in}}%
\pgfpathcurveto{\pgfqpoint{1.373737in}{0.727143in}}{\pgfqpoint{1.363138in}{0.731533in}}{\pgfqpoint{1.352088in}{0.731533in}}%
\pgfpathcurveto{\pgfqpoint{1.341037in}{0.731533in}}{\pgfqpoint{1.330438in}{0.727143in}}{\pgfqpoint{1.322625in}{0.719329in}}%
\pgfpathcurveto{\pgfqpoint{1.314811in}{0.711516in}}{\pgfqpoint{1.310421in}{0.700917in}}{\pgfqpoint{1.310421in}{0.689866in}}%
\pgfpathcurveto{\pgfqpoint{1.310421in}{0.678816in}}{\pgfqpoint{1.314811in}{0.668217in}}{\pgfqpoint{1.322625in}{0.660404in}}%
\pgfpathcurveto{\pgfqpoint{1.330438in}{0.652590in}}{\pgfqpoint{1.341037in}{0.648200in}}{\pgfqpoint{1.352088in}{0.648200in}}%
\pgfpathclose%
\pgfusepath{stroke,fill}%
\end{pgfscope}%
\begin{pgfscope}%
\pgfpathrectangle{\pgfqpoint{0.787074in}{0.548769in}}{\pgfqpoint{5.062926in}{3.102590in}}%
\pgfusepath{clip}%
\pgfsetbuttcap%
\pgfsetroundjoin%
\definecolor{currentfill}{rgb}{0.121569,0.466667,0.705882}%
\pgfsetfillcolor{currentfill}%
\pgfsetlinewidth{1.003750pt}%
\definecolor{currentstroke}{rgb}{0.121569,0.466667,0.705882}%
\pgfsetstrokecolor{currentstroke}%
\pgfsetdash{}{0pt}%
\pgfpathmoveto{\pgfqpoint{1.516603in}{0.648150in}}%
\pgfpathcurveto{\pgfqpoint{1.527653in}{0.648150in}}{\pgfqpoint{1.538252in}{0.652540in}}{\pgfqpoint{1.546066in}{0.660354in}}%
\pgfpathcurveto{\pgfqpoint{1.553879in}{0.668167in}}{\pgfqpoint{1.558269in}{0.678766in}}{\pgfqpoint{1.558269in}{0.689816in}}%
\pgfpathcurveto{\pgfqpoint{1.558269in}{0.700866in}}{\pgfqpoint{1.553879in}{0.711465in}}{\pgfqpoint{1.546066in}{0.719279in}}%
\pgfpathcurveto{\pgfqpoint{1.538252in}{0.727093in}}{\pgfqpoint{1.527653in}{0.731483in}}{\pgfqpoint{1.516603in}{0.731483in}}%
\pgfpathcurveto{\pgfqpoint{1.505553in}{0.731483in}}{\pgfqpoint{1.494954in}{0.727093in}}{\pgfqpoint{1.487140in}{0.719279in}}%
\pgfpathcurveto{\pgfqpoint{1.479326in}{0.711465in}}{\pgfqpoint{1.474936in}{0.700866in}}{\pgfqpoint{1.474936in}{0.689816in}}%
\pgfpathcurveto{\pgfqpoint{1.474936in}{0.678766in}}{\pgfqpoint{1.479326in}{0.668167in}}{\pgfqpoint{1.487140in}{0.660354in}}%
\pgfpathcurveto{\pgfqpoint{1.494954in}{0.652540in}}{\pgfqpoint{1.505553in}{0.648150in}}{\pgfqpoint{1.516603in}{0.648150in}}%
\pgfpathclose%
\pgfusepath{stroke,fill}%
\end{pgfscope}%
\begin{pgfscope}%
\pgfpathrectangle{\pgfqpoint{0.787074in}{0.548769in}}{\pgfqpoint{5.062926in}{3.102590in}}%
\pgfusepath{clip}%
\pgfsetbuttcap%
\pgfsetroundjoin%
\definecolor{currentfill}{rgb}{1.000000,0.498039,0.054902}%
\pgfsetfillcolor{currentfill}%
\pgfsetlinewidth{1.003750pt}%
\definecolor{currentstroke}{rgb}{1.000000,0.498039,0.054902}%
\pgfsetstrokecolor{currentstroke}%
\pgfsetdash{}{0pt}%
\pgfpathmoveto{\pgfqpoint{2.976144in}{3.030917in}}%
\pgfpathcurveto{\pgfqpoint{2.987194in}{3.030917in}}{\pgfqpoint{2.997793in}{3.035308in}}{\pgfqpoint{3.005607in}{3.043121in}}%
\pgfpathcurveto{\pgfqpoint{3.013420in}{3.050935in}}{\pgfqpoint{3.017811in}{3.061534in}}{\pgfqpoint{3.017811in}{3.072584in}}%
\pgfpathcurveto{\pgfqpoint{3.017811in}{3.083634in}}{\pgfqpoint{3.013420in}{3.094233in}}{\pgfqpoint{3.005607in}{3.102047in}}%
\pgfpathcurveto{\pgfqpoint{2.997793in}{3.109861in}}{\pgfqpoint{2.987194in}{3.114251in}}{\pgfqpoint{2.976144in}{3.114251in}}%
\pgfpathcurveto{\pgfqpoint{2.965094in}{3.114251in}}{\pgfqpoint{2.954495in}{3.109861in}}{\pgfqpoint{2.946681in}{3.102047in}}%
\pgfpathcurveto{\pgfqpoint{2.938868in}{3.094233in}}{\pgfqpoint{2.934477in}{3.083634in}}{\pgfqpoint{2.934477in}{3.072584in}}%
\pgfpathcurveto{\pgfqpoint{2.934477in}{3.061534in}}{\pgfqpoint{2.938868in}{3.050935in}}{\pgfqpoint{2.946681in}{3.043121in}}%
\pgfpathcurveto{\pgfqpoint{2.954495in}{3.035308in}}{\pgfqpoint{2.965094in}{3.030917in}}{\pgfqpoint{2.976144in}{3.030917in}}%
\pgfpathclose%
\pgfusepath{stroke,fill}%
\end{pgfscope}%
\begin{pgfscope}%
\pgfpathrectangle{\pgfqpoint{0.787074in}{0.548769in}}{\pgfqpoint{5.062926in}{3.102590in}}%
\pgfusepath{clip}%
\pgfsetbuttcap%
\pgfsetroundjoin%
\definecolor{currentfill}{rgb}{1.000000,0.498039,0.054902}%
\pgfsetfillcolor{currentfill}%
\pgfsetlinewidth{1.003750pt}%
\definecolor{currentstroke}{rgb}{1.000000,0.498039,0.054902}%
\pgfsetstrokecolor{currentstroke}%
\pgfsetdash{}{0pt}%
\pgfpathmoveto{\pgfqpoint{2.187916in}{2.938746in}}%
\pgfpathcurveto{\pgfqpoint{2.198966in}{2.938746in}}{\pgfqpoint{2.209565in}{2.943136in}}{\pgfqpoint{2.217379in}{2.950950in}}%
\pgfpathcurveto{\pgfqpoint{2.225192in}{2.958764in}}{\pgfqpoint{2.229583in}{2.969363in}}{\pgfqpoint{2.229583in}{2.980413in}}%
\pgfpathcurveto{\pgfqpoint{2.229583in}{2.991463in}}{\pgfqpoint{2.225192in}{3.002062in}}{\pgfqpoint{2.217379in}{3.009876in}}%
\pgfpathcurveto{\pgfqpoint{2.209565in}{3.017689in}}{\pgfqpoint{2.198966in}{3.022079in}}{\pgfqpoint{2.187916in}{3.022079in}}%
\pgfpathcurveto{\pgfqpoint{2.176866in}{3.022079in}}{\pgfqpoint{2.166267in}{3.017689in}}{\pgfqpoint{2.158453in}{3.009876in}}%
\pgfpathcurveto{\pgfqpoint{2.150639in}{3.002062in}}{\pgfqpoint{2.146249in}{2.991463in}}{\pgfqpoint{2.146249in}{2.980413in}}%
\pgfpathcurveto{\pgfqpoint{2.146249in}{2.969363in}}{\pgfqpoint{2.150639in}{2.958764in}}{\pgfqpoint{2.158453in}{2.950950in}}%
\pgfpathcurveto{\pgfqpoint{2.166267in}{2.943136in}}{\pgfqpoint{2.176866in}{2.938746in}}{\pgfqpoint{2.187916in}{2.938746in}}%
\pgfpathclose%
\pgfusepath{stroke,fill}%
\end{pgfscope}%
\begin{pgfscope}%
\pgfpathrectangle{\pgfqpoint{0.787074in}{0.548769in}}{\pgfqpoint{5.062926in}{3.102590in}}%
\pgfusepath{clip}%
\pgfsetbuttcap%
\pgfsetroundjoin%
\definecolor{currentfill}{rgb}{1.000000,0.498039,0.054902}%
\pgfsetfillcolor{currentfill}%
\pgfsetlinewidth{1.003750pt}%
\definecolor{currentstroke}{rgb}{1.000000,0.498039,0.054902}%
\pgfsetstrokecolor{currentstroke}%
\pgfsetdash{}{0pt}%
\pgfpathmoveto{\pgfqpoint{1.789741in}{2.706441in}}%
\pgfpathcurveto{\pgfqpoint{1.800791in}{2.706441in}}{\pgfqpoint{1.811390in}{2.710831in}}{\pgfqpoint{1.819203in}{2.718645in}}%
\pgfpathcurveto{\pgfqpoint{1.827017in}{2.726459in}}{\pgfqpoint{1.831407in}{2.737058in}}{\pgfqpoint{1.831407in}{2.748108in}}%
\pgfpathcurveto{\pgfqpoint{1.831407in}{2.759158in}}{\pgfqpoint{1.827017in}{2.769757in}}{\pgfqpoint{1.819203in}{2.777571in}}%
\pgfpathcurveto{\pgfqpoint{1.811390in}{2.785384in}}{\pgfqpoint{1.800791in}{2.789774in}}{\pgfqpoint{1.789741in}{2.789774in}}%
\pgfpathcurveto{\pgfqpoint{1.778690in}{2.789774in}}{\pgfqpoint{1.768091in}{2.785384in}}{\pgfqpoint{1.760278in}{2.777571in}}%
\pgfpathcurveto{\pgfqpoint{1.752464in}{2.769757in}}{\pgfqpoint{1.748074in}{2.759158in}}{\pgfqpoint{1.748074in}{2.748108in}}%
\pgfpathcurveto{\pgfqpoint{1.748074in}{2.737058in}}{\pgfqpoint{1.752464in}{2.726459in}}{\pgfqpoint{1.760278in}{2.718645in}}%
\pgfpathcurveto{\pgfqpoint{1.768091in}{2.710831in}}{\pgfqpoint{1.778690in}{2.706441in}}{\pgfqpoint{1.789741in}{2.706441in}}%
\pgfpathclose%
\pgfusepath{stroke,fill}%
\end{pgfscope}%
\begin{pgfscope}%
\pgfpathrectangle{\pgfqpoint{0.787074in}{0.548769in}}{\pgfqpoint{5.062926in}{3.102590in}}%
\pgfusepath{clip}%
\pgfsetbuttcap%
\pgfsetroundjoin%
\definecolor{currentfill}{rgb}{0.121569,0.466667,0.705882}%
\pgfsetfillcolor{currentfill}%
\pgfsetlinewidth{1.003750pt}%
\definecolor{currentstroke}{rgb}{0.121569,0.466667,0.705882}%
\pgfsetstrokecolor{currentstroke}%
\pgfsetdash{}{0pt}%
\pgfpathmoveto{\pgfqpoint{1.566721in}{0.648133in}}%
\pgfpathcurveto{\pgfqpoint{1.577771in}{0.648133in}}{\pgfqpoint{1.588370in}{0.652523in}}{\pgfqpoint{1.596184in}{0.660337in}}%
\pgfpathcurveto{\pgfqpoint{1.603998in}{0.668150in}}{\pgfqpoint{1.608388in}{0.678749in}}{\pgfqpoint{1.608388in}{0.689799in}}%
\pgfpathcurveto{\pgfqpoint{1.608388in}{0.700849in}}{\pgfqpoint{1.603998in}{0.711448in}}{\pgfqpoint{1.596184in}{0.719262in}}%
\pgfpathcurveto{\pgfqpoint{1.588370in}{0.727076in}}{\pgfqpoint{1.577771in}{0.731466in}}{\pgfqpoint{1.566721in}{0.731466in}}%
\pgfpathcurveto{\pgfqpoint{1.555671in}{0.731466in}}{\pgfqpoint{1.545072in}{0.727076in}}{\pgfqpoint{1.537258in}{0.719262in}}%
\pgfpathcurveto{\pgfqpoint{1.529445in}{0.711448in}}{\pgfqpoint{1.525055in}{0.700849in}}{\pgfqpoint{1.525055in}{0.689799in}}%
\pgfpathcurveto{\pgfqpoint{1.525055in}{0.678749in}}{\pgfqpoint{1.529445in}{0.668150in}}{\pgfqpoint{1.537258in}{0.660337in}}%
\pgfpathcurveto{\pgfqpoint{1.545072in}{0.652523in}}{\pgfqpoint{1.555671in}{0.648133in}}{\pgfqpoint{1.566721in}{0.648133in}}%
\pgfpathclose%
\pgfusepath{stroke,fill}%
\end{pgfscope}%
\begin{pgfscope}%
\pgfpathrectangle{\pgfqpoint{0.787074in}{0.548769in}}{\pgfqpoint{5.062926in}{3.102590in}}%
\pgfusepath{clip}%
\pgfsetbuttcap%
\pgfsetroundjoin%
\definecolor{currentfill}{rgb}{0.121569,0.466667,0.705882}%
\pgfsetfillcolor{currentfill}%
\pgfsetlinewidth{1.003750pt}%
\definecolor{currentstroke}{rgb}{0.121569,0.466667,0.705882}%
\pgfsetstrokecolor{currentstroke}%
\pgfsetdash{}{0pt}%
\pgfpathmoveto{\pgfqpoint{1.387775in}{0.678236in}}%
\pgfpathcurveto{\pgfqpoint{1.398825in}{0.678236in}}{\pgfqpoint{1.409424in}{0.682626in}}{\pgfqpoint{1.417238in}{0.690439in}}%
\pgfpathcurveto{\pgfqpoint{1.425051in}{0.698253in}}{\pgfqpoint{1.429442in}{0.708852in}}{\pgfqpoint{1.429442in}{0.719902in}}%
\pgfpathcurveto{\pgfqpoint{1.429442in}{0.730952in}}{\pgfqpoint{1.425051in}{0.741551in}}{\pgfqpoint{1.417238in}{0.749365in}}%
\pgfpathcurveto{\pgfqpoint{1.409424in}{0.757179in}}{\pgfqpoint{1.398825in}{0.761569in}}{\pgfqpoint{1.387775in}{0.761569in}}%
\pgfpathcurveto{\pgfqpoint{1.376725in}{0.761569in}}{\pgfqpoint{1.366126in}{0.757179in}}{\pgfqpoint{1.358312in}{0.749365in}}%
\pgfpathcurveto{\pgfqpoint{1.350499in}{0.741551in}}{\pgfqpoint{1.346108in}{0.730952in}}{\pgfqpoint{1.346108in}{0.719902in}}%
\pgfpathcurveto{\pgfqpoint{1.346108in}{0.708852in}}{\pgfqpoint{1.350499in}{0.698253in}}{\pgfqpoint{1.358312in}{0.690439in}}%
\pgfpathcurveto{\pgfqpoint{1.366126in}{0.682626in}}{\pgfqpoint{1.376725in}{0.678236in}}{\pgfqpoint{1.387775in}{0.678236in}}%
\pgfpathclose%
\pgfusepath{stroke,fill}%
\end{pgfscope}%
\begin{pgfscope}%
\pgfpathrectangle{\pgfqpoint{0.787074in}{0.548769in}}{\pgfqpoint{5.062926in}{3.102590in}}%
\pgfusepath{clip}%
\pgfsetbuttcap%
\pgfsetroundjoin%
\definecolor{currentfill}{rgb}{1.000000,0.498039,0.054902}%
\pgfsetfillcolor{currentfill}%
\pgfsetlinewidth{1.003750pt}%
\definecolor{currentstroke}{rgb}{1.000000,0.498039,0.054902}%
\pgfsetstrokecolor{currentstroke}%
\pgfsetdash{}{0pt}%
\pgfpathmoveto{\pgfqpoint{1.483247in}{3.011191in}}%
\pgfpathcurveto{\pgfqpoint{1.494297in}{3.011191in}}{\pgfqpoint{1.504896in}{3.015581in}}{\pgfqpoint{1.512710in}{3.023394in}}%
\pgfpathcurveto{\pgfqpoint{1.520523in}{3.031208in}}{\pgfqpoint{1.524914in}{3.041807in}}{\pgfqpoint{1.524914in}{3.052857in}}%
\pgfpathcurveto{\pgfqpoint{1.524914in}{3.063907in}}{\pgfqpoint{1.520523in}{3.074506in}}{\pgfqpoint{1.512710in}{3.082320in}}%
\pgfpathcurveto{\pgfqpoint{1.504896in}{3.090134in}}{\pgfqpoint{1.494297in}{3.094524in}}{\pgfqpoint{1.483247in}{3.094524in}}%
\pgfpathcurveto{\pgfqpoint{1.472197in}{3.094524in}}{\pgfqpoint{1.461598in}{3.090134in}}{\pgfqpoint{1.453784in}{3.082320in}}%
\pgfpathcurveto{\pgfqpoint{1.445971in}{3.074506in}}{\pgfqpoint{1.441580in}{3.063907in}}{\pgfqpoint{1.441580in}{3.052857in}}%
\pgfpathcurveto{\pgfqpoint{1.441580in}{3.041807in}}{\pgfqpoint{1.445971in}{3.031208in}}{\pgfqpoint{1.453784in}{3.023394in}}%
\pgfpathcurveto{\pgfqpoint{1.461598in}{3.015581in}}{\pgfqpoint{1.472197in}{3.011191in}}{\pgfqpoint{1.483247in}{3.011191in}}%
\pgfpathclose%
\pgfusepath{stroke,fill}%
\end{pgfscope}%
\begin{pgfscope}%
\pgfpathrectangle{\pgfqpoint{0.787074in}{0.548769in}}{\pgfqpoint{5.062926in}{3.102590in}}%
\pgfusepath{clip}%
\pgfsetbuttcap%
\pgfsetroundjoin%
\definecolor{currentfill}{rgb}{0.121569,0.466667,0.705882}%
\pgfsetfillcolor{currentfill}%
\pgfsetlinewidth{1.003750pt}%
\definecolor{currentstroke}{rgb}{0.121569,0.466667,0.705882}%
\pgfsetstrokecolor{currentstroke}%
\pgfsetdash{}{0pt}%
\pgfpathmoveto{\pgfqpoint{1.020183in}{0.787280in}}%
\pgfpathcurveto{\pgfqpoint{1.031233in}{0.787280in}}{\pgfqpoint{1.041832in}{0.791670in}}{\pgfqpoint{1.049646in}{0.799484in}}%
\pgfpathcurveto{\pgfqpoint{1.057459in}{0.807297in}}{\pgfqpoint{1.061850in}{0.817896in}}{\pgfqpoint{1.061850in}{0.828946in}}%
\pgfpathcurveto{\pgfqpoint{1.061850in}{0.839997in}}{\pgfqpoint{1.057459in}{0.850596in}}{\pgfqpoint{1.049646in}{0.858409in}}%
\pgfpathcurveto{\pgfqpoint{1.041832in}{0.866223in}}{\pgfqpoint{1.031233in}{0.870613in}}{\pgfqpoint{1.020183in}{0.870613in}}%
\pgfpathcurveto{\pgfqpoint{1.009133in}{0.870613in}}{\pgfqpoint{0.998534in}{0.866223in}}{\pgfqpoint{0.990720in}{0.858409in}}%
\pgfpathcurveto{\pgfqpoint{0.982907in}{0.850596in}}{\pgfqpoint{0.978516in}{0.839997in}}{\pgfqpoint{0.978516in}{0.828946in}}%
\pgfpathcurveto{\pgfqpoint{0.978516in}{0.817896in}}{\pgfqpoint{0.982907in}{0.807297in}}{\pgfqpoint{0.990720in}{0.799484in}}%
\pgfpathcurveto{\pgfqpoint{0.998534in}{0.791670in}}{\pgfqpoint{1.009133in}{0.787280in}}{\pgfqpoint{1.020183in}{0.787280in}}%
\pgfpathclose%
\pgfusepath{stroke,fill}%
\end{pgfscope}%
\begin{pgfscope}%
\pgfpathrectangle{\pgfqpoint{0.787074in}{0.548769in}}{\pgfqpoint{5.062926in}{3.102590in}}%
\pgfusepath{clip}%
\pgfsetbuttcap%
\pgfsetroundjoin%
\definecolor{currentfill}{rgb}{1.000000,0.498039,0.054902}%
\pgfsetfillcolor{currentfill}%
\pgfsetlinewidth{1.003750pt}%
\definecolor{currentstroke}{rgb}{1.000000,0.498039,0.054902}%
\pgfsetstrokecolor{currentstroke}%
\pgfsetdash{}{0pt}%
\pgfpathmoveto{\pgfqpoint{2.039926in}{3.468665in}}%
\pgfpathcurveto{\pgfqpoint{2.050976in}{3.468665in}}{\pgfqpoint{2.061575in}{3.473055in}}{\pgfqpoint{2.069389in}{3.480869in}}%
\pgfpathcurveto{\pgfqpoint{2.077202in}{3.488683in}}{\pgfqpoint{2.081593in}{3.499282in}}{\pgfqpoint{2.081593in}{3.510332in}}%
\pgfpathcurveto{\pgfqpoint{2.081593in}{3.521382in}}{\pgfqpoint{2.077202in}{3.531981in}}{\pgfqpoint{2.069389in}{3.539795in}}%
\pgfpathcurveto{\pgfqpoint{2.061575in}{3.547608in}}{\pgfqpoint{2.050976in}{3.551998in}}{\pgfqpoint{2.039926in}{3.551998in}}%
\pgfpathcurveto{\pgfqpoint{2.028876in}{3.551998in}}{\pgfqpoint{2.018277in}{3.547608in}}{\pgfqpoint{2.010463in}{3.539795in}}%
\pgfpathcurveto{\pgfqpoint{2.002650in}{3.531981in}}{\pgfqpoint{1.998259in}{3.521382in}}{\pgfqpoint{1.998259in}{3.510332in}}%
\pgfpathcurveto{\pgfqpoint{1.998259in}{3.499282in}}{\pgfqpoint{2.002650in}{3.488683in}}{\pgfqpoint{2.010463in}{3.480869in}}%
\pgfpathcurveto{\pgfqpoint{2.018277in}{3.473055in}}{\pgfqpoint{2.028876in}{3.468665in}}{\pgfqpoint{2.039926in}{3.468665in}}%
\pgfpathclose%
\pgfusepath{stroke,fill}%
\end{pgfscope}%
\begin{pgfscope}%
\pgfpathrectangle{\pgfqpoint{0.787074in}{0.548769in}}{\pgfqpoint{5.062926in}{3.102590in}}%
\pgfusepath{clip}%
\pgfsetbuttcap%
\pgfsetroundjoin%
\definecolor{currentfill}{rgb}{0.121569,0.466667,0.705882}%
\pgfsetfillcolor{currentfill}%
\pgfsetlinewidth{1.003750pt}%
\definecolor{currentstroke}{rgb}{0.121569,0.466667,0.705882}%
\pgfsetstrokecolor{currentstroke}%
\pgfsetdash{}{0pt}%
\pgfpathmoveto{\pgfqpoint{1.220208in}{0.649989in}}%
\pgfpathcurveto{\pgfqpoint{1.231258in}{0.649989in}}{\pgfqpoint{1.241857in}{0.654379in}}{\pgfqpoint{1.249670in}{0.662193in}}%
\pgfpathcurveto{\pgfqpoint{1.257484in}{0.670006in}}{\pgfqpoint{1.261874in}{0.680605in}}{\pgfqpoint{1.261874in}{0.691655in}}%
\pgfpathcurveto{\pgfqpoint{1.261874in}{0.702706in}}{\pgfqpoint{1.257484in}{0.713305in}}{\pgfqpoint{1.249670in}{0.721118in}}%
\pgfpathcurveto{\pgfqpoint{1.241857in}{0.728932in}}{\pgfqpoint{1.231258in}{0.733322in}}{\pgfqpoint{1.220208in}{0.733322in}}%
\pgfpathcurveto{\pgfqpoint{1.209157in}{0.733322in}}{\pgfqpoint{1.198558in}{0.728932in}}{\pgfqpoint{1.190745in}{0.721118in}}%
\pgfpathcurveto{\pgfqpoint{1.182931in}{0.713305in}}{\pgfqpoint{1.178541in}{0.702706in}}{\pgfqpoint{1.178541in}{0.691655in}}%
\pgfpathcurveto{\pgfqpoint{1.178541in}{0.680605in}}{\pgfqpoint{1.182931in}{0.670006in}}{\pgfqpoint{1.190745in}{0.662193in}}%
\pgfpathcurveto{\pgfqpoint{1.198558in}{0.654379in}}{\pgfqpoint{1.209157in}{0.649989in}}{\pgfqpoint{1.220208in}{0.649989in}}%
\pgfpathclose%
\pgfusepath{stroke,fill}%
\end{pgfscope}%
\begin{pgfscope}%
\pgfpathrectangle{\pgfqpoint{0.787074in}{0.548769in}}{\pgfqpoint{5.062926in}{3.102590in}}%
\pgfusepath{clip}%
\pgfsetbuttcap%
\pgfsetroundjoin%
\definecolor{currentfill}{rgb}{1.000000,0.498039,0.054902}%
\pgfsetfillcolor{currentfill}%
\pgfsetlinewidth{1.003750pt}%
\definecolor{currentstroke}{rgb}{1.000000,0.498039,0.054902}%
\pgfsetstrokecolor{currentstroke}%
\pgfsetdash{}{0pt}%
\pgfpathmoveto{\pgfqpoint{1.981575in}{2.212517in}}%
\pgfpathcurveto{\pgfqpoint{1.992625in}{2.212517in}}{\pgfqpoint{2.003224in}{2.216907in}}{\pgfqpoint{2.011037in}{2.224721in}}%
\pgfpathcurveto{\pgfqpoint{2.018851in}{2.232535in}}{\pgfqpoint{2.023241in}{2.243134in}}{\pgfqpoint{2.023241in}{2.254184in}}%
\pgfpathcurveto{\pgfqpoint{2.023241in}{2.265234in}}{\pgfqpoint{2.018851in}{2.275833in}}{\pgfqpoint{2.011037in}{2.283647in}}%
\pgfpathcurveto{\pgfqpoint{2.003224in}{2.291460in}}{\pgfqpoint{1.992625in}{2.295850in}}{\pgfqpoint{1.981575in}{2.295850in}}%
\pgfpathcurveto{\pgfqpoint{1.970524in}{2.295850in}}{\pgfqpoint{1.959925in}{2.291460in}}{\pgfqpoint{1.952112in}{2.283647in}}%
\pgfpathcurveto{\pgfqpoint{1.944298in}{2.275833in}}{\pgfqpoint{1.939908in}{2.265234in}}{\pgfqpoint{1.939908in}{2.254184in}}%
\pgfpathcurveto{\pgfqpoint{1.939908in}{2.243134in}}{\pgfqpoint{1.944298in}{2.232535in}}{\pgfqpoint{1.952112in}{2.224721in}}%
\pgfpathcurveto{\pgfqpoint{1.959925in}{2.216907in}}{\pgfqpoint{1.970524in}{2.212517in}}{\pgfqpoint{1.981575in}{2.212517in}}%
\pgfpathclose%
\pgfusepath{stroke,fill}%
\end{pgfscope}%
\begin{pgfscope}%
\pgfpathrectangle{\pgfqpoint{0.787074in}{0.548769in}}{\pgfqpoint{5.062926in}{3.102590in}}%
\pgfusepath{clip}%
\pgfsetbuttcap%
\pgfsetroundjoin%
\definecolor{currentfill}{rgb}{0.121569,0.466667,0.705882}%
\pgfsetfillcolor{currentfill}%
\pgfsetlinewidth{1.003750pt}%
\definecolor{currentstroke}{rgb}{0.121569,0.466667,0.705882}%
\pgfsetstrokecolor{currentstroke}%
\pgfsetdash{}{0pt}%
\pgfpathmoveto{\pgfqpoint{2.747817in}{0.651678in}}%
\pgfpathcurveto{\pgfqpoint{2.758867in}{0.651678in}}{\pgfqpoint{2.769466in}{0.656068in}}{\pgfqpoint{2.777280in}{0.663882in}}%
\pgfpathcurveto{\pgfqpoint{2.785093in}{0.671695in}}{\pgfqpoint{2.789484in}{0.682295in}}{\pgfqpoint{2.789484in}{0.693345in}}%
\pgfpathcurveto{\pgfqpoint{2.789484in}{0.704395in}}{\pgfqpoint{2.785093in}{0.714994in}}{\pgfqpoint{2.777280in}{0.722807in}}%
\pgfpathcurveto{\pgfqpoint{2.769466in}{0.730621in}}{\pgfqpoint{2.758867in}{0.735011in}}{\pgfqpoint{2.747817in}{0.735011in}}%
\pgfpathcurveto{\pgfqpoint{2.736767in}{0.735011in}}{\pgfqpoint{2.726168in}{0.730621in}}{\pgfqpoint{2.718354in}{0.722807in}}%
\pgfpathcurveto{\pgfqpoint{2.710540in}{0.714994in}}{\pgfqpoint{2.706150in}{0.704395in}}{\pgfqpoint{2.706150in}{0.693345in}}%
\pgfpathcurveto{\pgfqpoint{2.706150in}{0.682295in}}{\pgfqpoint{2.710540in}{0.671695in}}{\pgfqpoint{2.718354in}{0.663882in}}%
\pgfpathcurveto{\pgfqpoint{2.726168in}{0.656068in}}{\pgfqpoint{2.736767in}{0.651678in}}{\pgfqpoint{2.747817in}{0.651678in}}%
\pgfpathclose%
\pgfusepath{stroke,fill}%
\end{pgfscope}%
\begin{pgfscope}%
\pgfpathrectangle{\pgfqpoint{0.787074in}{0.548769in}}{\pgfqpoint{5.062926in}{3.102590in}}%
\pgfusepath{clip}%
\pgfsetbuttcap%
\pgfsetroundjoin%
\definecolor{currentfill}{rgb}{1.000000,0.498039,0.054902}%
\pgfsetfillcolor{currentfill}%
\pgfsetlinewidth{1.003750pt}%
\definecolor{currentstroke}{rgb}{1.000000,0.498039,0.054902}%
\pgfsetstrokecolor{currentstroke}%
\pgfsetdash{}{0pt}%
\pgfpathmoveto{\pgfqpoint{1.317011in}{2.584775in}}%
\pgfpathcurveto{\pgfqpoint{1.328061in}{2.584775in}}{\pgfqpoint{1.338660in}{2.589166in}}{\pgfqpoint{1.346473in}{2.596979in}}%
\pgfpathcurveto{\pgfqpoint{1.354287in}{2.604793in}}{\pgfqpoint{1.358677in}{2.615392in}}{\pgfqpoint{1.358677in}{2.626442in}}%
\pgfpathcurveto{\pgfqpoint{1.358677in}{2.637492in}}{\pgfqpoint{1.354287in}{2.648091in}}{\pgfqpoint{1.346473in}{2.655905in}}%
\pgfpathcurveto{\pgfqpoint{1.338660in}{2.663718in}}{\pgfqpoint{1.328061in}{2.668109in}}{\pgfqpoint{1.317011in}{2.668109in}}%
\pgfpathcurveto{\pgfqpoint{1.305960in}{2.668109in}}{\pgfqpoint{1.295361in}{2.663718in}}{\pgfqpoint{1.287548in}{2.655905in}}%
\pgfpathcurveto{\pgfqpoint{1.279734in}{2.648091in}}{\pgfqpoint{1.275344in}{2.637492in}}{\pgfqpoint{1.275344in}{2.626442in}}%
\pgfpathcurveto{\pgfqpoint{1.275344in}{2.615392in}}{\pgfqpoint{1.279734in}{2.604793in}}{\pgfqpoint{1.287548in}{2.596979in}}%
\pgfpathcurveto{\pgfqpoint{1.295361in}{2.589166in}}{\pgfqpoint{1.305960in}{2.584775in}}{\pgfqpoint{1.317011in}{2.584775in}}%
\pgfpathclose%
\pgfusepath{stroke,fill}%
\end{pgfscope}%
\begin{pgfscope}%
\pgfpathrectangle{\pgfqpoint{0.787074in}{0.548769in}}{\pgfqpoint{5.062926in}{3.102590in}}%
\pgfusepath{clip}%
\pgfsetbuttcap%
\pgfsetroundjoin%
\definecolor{currentfill}{rgb}{0.121569,0.466667,0.705882}%
\pgfsetfillcolor{currentfill}%
\pgfsetlinewidth{1.003750pt}%
\definecolor{currentstroke}{rgb}{0.121569,0.466667,0.705882}%
\pgfsetstrokecolor{currentstroke}%
\pgfsetdash{}{0pt}%
\pgfpathmoveto{\pgfqpoint{1.925996in}{0.648158in}}%
\pgfpathcurveto{\pgfqpoint{1.937046in}{0.648158in}}{\pgfqpoint{1.947645in}{0.652548in}}{\pgfqpoint{1.955458in}{0.660362in}}%
\pgfpathcurveto{\pgfqpoint{1.963272in}{0.668176in}}{\pgfqpoint{1.967662in}{0.678775in}}{\pgfqpoint{1.967662in}{0.689825in}}%
\pgfpathcurveto{\pgfqpoint{1.967662in}{0.700875in}}{\pgfqpoint{1.963272in}{0.711474in}}{\pgfqpoint{1.955458in}{0.719288in}}%
\pgfpathcurveto{\pgfqpoint{1.947645in}{0.727101in}}{\pgfqpoint{1.937046in}{0.731492in}}{\pgfqpoint{1.925996in}{0.731492in}}%
\pgfpathcurveto{\pgfqpoint{1.914946in}{0.731492in}}{\pgfqpoint{1.904347in}{0.727101in}}{\pgfqpoint{1.896533in}{0.719288in}}%
\pgfpathcurveto{\pgfqpoint{1.888719in}{0.711474in}}{\pgfqpoint{1.884329in}{0.700875in}}{\pgfqpoint{1.884329in}{0.689825in}}%
\pgfpathcurveto{\pgfqpoint{1.884329in}{0.678775in}}{\pgfqpoint{1.888719in}{0.668176in}}{\pgfqpoint{1.896533in}{0.660362in}}%
\pgfpathcurveto{\pgfqpoint{1.904347in}{0.652548in}}{\pgfqpoint{1.914946in}{0.648158in}}{\pgfqpoint{1.925996in}{0.648158in}}%
\pgfpathclose%
\pgfusepath{stroke,fill}%
\end{pgfscope}%
\begin{pgfscope}%
\pgfpathrectangle{\pgfqpoint{0.787074in}{0.548769in}}{\pgfqpoint{5.062926in}{3.102590in}}%
\pgfusepath{clip}%
\pgfsetbuttcap%
\pgfsetroundjoin%
\definecolor{currentfill}{rgb}{1.000000,0.498039,0.054902}%
\pgfsetfillcolor{currentfill}%
\pgfsetlinewidth{1.003750pt}%
\definecolor{currentstroke}{rgb}{1.000000,0.498039,0.054902}%
\pgfsetstrokecolor{currentstroke}%
\pgfsetdash{}{0pt}%
\pgfpathmoveto{\pgfqpoint{1.154980in}{2.981281in}}%
\pgfpathcurveto{\pgfqpoint{1.166030in}{2.981281in}}{\pgfqpoint{1.176629in}{2.985671in}}{\pgfqpoint{1.184442in}{2.993484in}}%
\pgfpathcurveto{\pgfqpoint{1.192256in}{3.001298in}}{\pgfqpoint{1.196646in}{3.011897in}}{\pgfqpoint{1.196646in}{3.022947in}}%
\pgfpathcurveto{\pgfqpoint{1.196646in}{3.033997in}}{\pgfqpoint{1.192256in}{3.044596in}}{\pgfqpoint{1.184442in}{3.052410in}}%
\pgfpathcurveto{\pgfqpoint{1.176629in}{3.060224in}}{\pgfqpoint{1.166030in}{3.064614in}}{\pgfqpoint{1.154980in}{3.064614in}}%
\pgfpathcurveto{\pgfqpoint{1.143930in}{3.064614in}}{\pgfqpoint{1.133331in}{3.060224in}}{\pgfqpoint{1.125517in}{3.052410in}}%
\pgfpathcurveto{\pgfqpoint{1.117703in}{3.044596in}}{\pgfqpoint{1.113313in}{3.033997in}}{\pgfqpoint{1.113313in}{3.022947in}}%
\pgfpathcurveto{\pgfqpoint{1.113313in}{3.011897in}}{\pgfqpoint{1.117703in}{3.001298in}}{\pgfqpoint{1.125517in}{2.993484in}}%
\pgfpathcurveto{\pgfqpoint{1.133331in}{2.985671in}}{\pgfqpoint{1.143930in}{2.981281in}}{\pgfqpoint{1.154980in}{2.981281in}}%
\pgfpathclose%
\pgfusepath{stroke,fill}%
\end{pgfscope}%
\begin{pgfscope}%
\pgfpathrectangle{\pgfqpoint{0.787074in}{0.548769in}}{\pgfqpoint{5.062926in}{3.102590in}}%
\pgfusepath{clip}%
\pgfsetbuttcap%
\pgfsetroundjoin%
\definecolor{currentfill}{rgb}{0.121569,0.466667,0.705882}%
\pgfsetfillcolor{currentfill}%
\pgfsetlinewidth{1.003750pt}%
\definecolor{currentstroke}{rgb}{0.121569,0.466667,0.705882}%
\pgfsetstrokecolor{currentstroke}%
\pgfsetdash{}{0pt}%
\pgfpathmoveto{\pgfqpoint{1.564771in}{0.648132in}}%
\pgfpathcurveto{\pgfqpoint{1.575821in}{0.648132in}}{\pgfqpoint{1.586420in}{0.652522in}}{\pgfqpoint{1.594234in}{0.660336in}}%
\pgfpathcurveto{\pgfqpoint{1.602048in}{0.668149in}}{\pgfqpoint{1.606438in}{0.678749in}}{\pgfqpoint{1.606438in}{0.689799in}}%
\pgfpathcurveto{\pgfqpoint{1.606438in}{0.700849in}}{\pgfqpoint{1.602048in}{0.711448in}}{\pgfqpoint{1.594234in}{0.719261in}}%
\pgfpathcurveto{\pgfqpoint{1.586420in}{0.727075in}}{\pgfqpoint{1.575821in}{0.731465in}}{\pgfqpoint{1.564771in}{0.731465in}}%
\pgfpathcurveto{\pgfqpoint{1.553721in}{0.731465in}}{\pgfqpoint{1.543122in}{0.727075in}}{\pgfqpoint{1.535308in}{0.719261in}}%
\pgfpathcurveto{\pgfqpoint{1.527495in}{0.711448in}}{\pgfqpoint{1.523104in}{0.700849in}}{\pgfqpoint{1.523104in}{0.689799in}}%
\pgfpathcurveto{\pgfqpoint{1.523104in}{0.678749in}}{\pgfqpoint{1.527495in}{0.668149in}}{\pgfqpoint{1.535308in}{0.660336in}}%
\pgfpathcurveto{\pgfqpoint{1.543122in}{0.652522in}}{\pgfqpoint{1.553721in}{0.648132in}}{\pgfqpoint{1.564771in}{0.648132in}}%
\pgfpathclose%
\pgfusepath{stroke,fill}%
\end{pgfscope}%
\begin{pgfscope}%
\pgfpathrectangle{\pgfqpoint{0.787074in}{0.548769in}}{\pgfqpoint{5.062926in}{3.102590in}}%
\pgfusepath{clip}%
\pgfsetbuttcap%
\pgfsetroundjoin%
\definecolor{currentfill}{rgb}{1.000000,0.498039,0.054902}%
\pgfsetfillcolor{currentfill}%
\pgfsetlinewidth{1.003750pt}%
\definecolor{currentstroke}{rgb}{1.000000,0.498039,0.054902}%
\pgfsetstrokecolor{currentstroke}%
\pgfsetdash{}{0pt}%
\pgfpathmoveto{\pgfqpoint{1.466806in}{2.254043in}}%
\pgfpathcurveto{\pgfqpoint{1.477857in}{2.254043in}}{\pgfqpoint{1.488456in}{2.258433in}}{\pgfqpoint{1.496269in}{2.266247in}}%
\pgfpathcurveto{\pgfqpoint{1.504083in}{2.274060in}}{\pgfqpoint{1.508473in}{2.284659in}}{\pgfqpoint{1.508473in}{2.295709in}}%
\pgfpathcurveto{\pgfqpoint{1.508473in}{2.306759in}}{\pgfqpoint{1.504083in}{2.317359in}}{\pgfqpoint{1.496269in}{2.325172in}}%
\pgfpathcurveto{\pgfqpoint{1.488456in}{2.332986in}}{\pgfqpoint{1.477857in}{2.337376in}}{\pgfqpoint{1.466806in}{2.337376in}}%
\pgfpathcurveto{\pgfqpoint{1.455756in}{2.337376in}}{\pgfqpoint{1.445157in}{2.332986in}}{\pgfqpoint{1.437344in}{2.325172in}}%
\pgfpathcurveto{\pgfqpoint{1.429530in}{2.317359in}}{\pgfqpoint{1.425140in}{2.306759in}}{\pgfqpoint{1.425140in}{2.295709in}}%
\pgfpathcurveto{\pgfqpoint{1.425140in}{2.284659in}}{\pgfqpoint{1.429530in}{2.274060in}}{\pgfqpoint{1.437344in}{2.266247in}}%
\pgfpathcurveto{\pgfqpoint{1.445157in}{2.258433in}}{\pgfqpoint{1.455756in}{2.254043in}}{\pgfqpoint{1.466806in}{2.254043in}}%
\pgfpathclose%
\pgfusepath{stroke,fill}%
\end{pgfscope}%
\begin{pgfscope}%
\pgfpathrectangle{\pgfqpoint{0.787074in}{0.548769in}}{\pgfqpoint{5.062926in}{3.102590in}}%
\pgfusepath{clip}%
\pgfsetbuttcap%
\pgfsetroundjoin%
\definecolor{currentfill}{rgb}{1.000000,0.498039,0.054902}%
\pgfsetfillcolor{currentfill}%
\pgfsetlinewidth{1.003750pt}%
\definecolor{currentstroke}{rgb}{1.000000,0.498039,0.054902}%
\pgfsetstrokecolor{currentstroke}%
\pgfsetdash{}{0pt}%
\pgfpathmoveto{\pgfqpoint{2.465005in}{2.386212in}}%
\pgfpathcurveto{\pgfqpoint{2.476055in}{2.386212in}}{\pgfqpoint{2.486654in}{2.390603in}}{\pgfqpoint{2.494468in}{2.398416in}}%
\pgfpathcurveto{\pgfqpoint{2.502281in}{2.406230in}}{\pgfqpoint{2.506671in}{2.416829in}}{\pgfqpoint{2.506671in}{2.427879in}}%
\pgfpathcurveto{\pgfqpoint{2.506671in}{2.438929in}}{\pgfqpoint{2.502281in}{2.449528in}}{\pgfqpoint{2.494468in}{2.457342in}}%
\pgfpathcurveto{\pgfqpoint{2.486654in}{2.465155in}}{\pgfqpoint{2.476055in}{2.469546in}}{\pgfqpoint{2.465005in}{2.469546in}}%
\pgfpathcurveto{\pgfqpoint{2.453955in}{2.469546in}}{\pgfqpoint{2.443356in}{2.465155in}}{\pgfqpoint{2.435542in}{2.457342in}}%
\pgfpathcurveto{\pgfqpoint{2.427728in}{2.449528in}}{\pgfqpoint{2.423338in}{2.438929in}}{\pgfqpoint{2.423338in}{2.427879in}}%
\pgfpathcurveto{\pgfqpoint{2.423338in}{2.416829in}}{\pgfqpoint{2.427728in}{2.406230in}}{\pgfqpoint{2.435542in}{2.398416in}}%
\pgfpathcurveto{\pgfqpoint{2.443356in}{2.390603in}}{\pgfqpoint{2.453955in}{2.386212in}}{\pgfqpoint{2.465005in}{2.386212in}}%
\pgfpathclose%
\pgfusepath{stroke,fill}%
\end{pgfscope}%
\begin{pgfscope}%
\pgfpathrectangle{\pgfqpoint{0.787074in}{0.548769in}}{\pgfqpoint{5.062926in}{3.102590in}}%
\pgfusepath{clip}%
\pgfsetbuttcap%
\pgfsetroundjoin%
\definecolor{currentfill}{rgb}{0.121569,0.466667,0.705882}%
\pgfsetfillcolor{currentfill}%
\pgfsetlinewidth{1.003750pt}%
\definecolor{currentstroke}{rgb}{0.121569,0.466667,0.705882}%
\pgfsetstrokecolor{currentstroke}%
\pgfsetdash{}{0pt}%
\pgfpathmoveto{\pgfqpoint{1.913896in}{0.648147in}}%
\pgfpathcurveto{\pgfqpoint{1.924946in}{0.648147in}}{\pgfqpoint{1.935546in}{0.652537in}}{\pgfqpoint{1.943359in}{0.660351in}}%
\pgfpathcurveto{\pgfqpoint{1.951173in}{0.668165in}}{\pgfqpoint{1.955563in}{0.678764in}}{\pgfqpoint{1.955563in}{0.689814in}}%
\pgfpathcurveto{\pgfqpoint{1.955563in}{0.700864in}}{\pgfqpoint{1.951173in}{0.711463in}}{\pgfqpoint{1.943359in}{0.719277in}}%
\pgfpathcurveto{\pgfqpoint{1.935546in}{0.727090in}}{\pgfqpoint{1.924946in}{0.731480in}}{\pgfqpoint{1.913896in}{0.731480in}}%
\pgfpathcurveto{\pgfqpoint{1.902846in}{0.731480in}}{\pgfqpoint{1.892247in}{0.727090in}}{\pgfqpoint{1.884434in}{0.719277in}}%
\pgfpathcurveto{\pgfqpoint{1.876620in}{0.711463in}}{\pgfqpoint{1.872230in}{0.700864in}}{\pgfqpoint{1.872230in}{0.689814in}}%
\pgfpathcurveto{\pgfqpoint{1.872230in}{0.678764in}}{\pgfqpoint{1.876620in}{0.668165in}}{\pgfqpoint{1.884434in}{0.660351in}}%
\pgfpathcurveto{\pgfqpoint{1.892247in}{0.652537in}}{\pgfqpoint{1.902846in}{0.648147in}}{\pgfqpoint{1.913896in}{0.648147in}}%
\pgfpathclose%
\pgfusepath{stroke,fill}%
\end{pgfscope}%
\begin{pgfscope}%
\pgfpathrectangle{\pgfqpoint{0.787074in}{0.548769in}}{\pgfqpoint{5.062926in}{3.102590in}}%
\pgfusepath{clip}%
\pgfsetbuttcap%
\pgfsetroundjoin%
\definecolor{currentfill}{rgb}{1.000000,0.498039,0.054902}%
\pgfsetfillcolor{currentfill}%
\pgfsetlinewidth{1.003750pt}%
\definecolor{currentstroke}{rgb}{1.000000,0.498039,0.054902}%
\pgfsetstrokecolor{currentstroke}%
\pgfsetdash{}{0pt}%
\pgfpathmoveto{\pgfqpoint{1.415840in}{2.590063in}}%
\pgfpathcurveto{\pgfqpoint{1.426890in}{2.590063in}}{\pgfqpoint{1.437489in}{2.594453in}}{\pgfqpoint{1.445303in}{2.602267in}}%
\pgfpathcurveto{\pgfqpoint{1.453116in}{2.610081in}}{\pgfqpoint{1.457507in}{2.620680in}}{\pgfqpoint{1.457507in}{2.631730in}}%
\pgfpathcurveto{\pgfqpoint{1.457507in}{2.642780in}}{\pgfqpoint{1.453116in}{2.653379in}}{\pgfqpoint{1.445303in}{2.661193in}}%
\pgfpathcurveto{\pgfqpoint{1.437489in}{2.669006in}}{\pgfqpoint{1.426890in}{2.673397in}}{\pgfqpoint{1.415840in}{2.673397in}}%
\pgfpathcurveto{\pgfqpoint{1.404790in}{2.673397in}}{\pgfqpoint{1.394191in}{2.669006in}}{\pgfqpoint{1.386377in}{2.661193in}}%
\pgfpathcurveto{\pgfqpoint{1.378564in}{2.653379in}}{\pgfqpoint{1.374173in}{2.642780in}}{\pgfqpoint{1.374173in}{2.631730in}}%
\pgfpathcurveto{\pgfqpoint{1.374173in}{2.620680in}}{\pgfqpoint{1.378564in}{2.610081in}}{\pgfqpoint{1.386377in}{2.602267in}}%
\pgfpathcurveto{\pgfqpoint{1.394191in}{2.594453in}}{\pgfqpoint{1.404790in}{2.590063in}}{\pgfqpoint{1.415840in}{2.590063in}}%
\pgfpathclose%
\pgfusepath{stroke,fill}%
\end{pgfscope}%
\begin{pgfscope}%
\pgfpathrectangle{\pgfqpoint{0.787074in}{0.548769in}}{\pgfqpoint{5.062926in}{3.102590in}}%
\pgfusepath{clip}%
\pgfsetbuttcap%
\pgfsetroundjoin%
\definecolor{currentfill}{rgb}{1.000000,0.498039,0.054902}%
\pgfsetfillcolor{currentfill}%
\pgfsetlinewidth{1.003750pt}%
\definecolor{currentstroke}{rgb}{1.000000,0.498039,0.054902}%
\pgfsetstrokecolor{currentstroke}%
\pgfsetdash{}{0pt}%
\pgfpathmoveto{\pgfqpoint{1.073617in}{2.389366in}}%
\pgfpathcurveto{\pgfqpoint{1.084667in}{2.389366in}}{\pgfqpoint{1.095266in}{2.393756in}}{\pgfqpoint{1.103079in}{2.401570in}}%
\pgfpathcurveto{\pgfqpoint{1.110893in}{2.409383in}}{\pgfqpoint{1.115283in}{2.419982in}}{\pgfqpoint{1.115283in}{2.431032in}}%
\pgfpathcurveto{\pgfqpoint{1.115283in}{2.442082in}}{\pgfqpoint{1.110893in}{2.452682in}}{\pgfqpoint{1.103079in}{2.460495in}}%
\pgfpathcurveto{\pgfqpoint{1.095266in}{2.468309in}}{\pgfqpoint{1.084667in}{2.472699in}}{\pgfqpoint{1.073617in}{2.472699in}}%
\pgfpathcurveto{\pgfqpoint{1.062567in}{2.472699in}}{\pgfqpoint{1.051967in}{2.468309in}}{\pgfqpoint{1.044154in}{2.460495in}}%
\pgfpathcurveto{\pgfqpoint{1.036340in}{2.452682in}}{\pgfqpoint{1.031950in}{2.442082in}}{\pgfqpoint{1.031950in}{2.431032in}}%
\pgfpathcurveto{\pgfqpoint{1.031950in}{2.419982in}}{\pgfqpoint{1.036340in}{2.409383in}}{\pgfqpoint{1.044154in}{2.401570in}}%
\pgfpathcurveto{\pgfqpoint{1.051967in}{2.393756in}}{\pgfqpoint{1.062567in}{2.389366in}}{\pgfqpoint{1.073617in}{2.389366in}}%
\pgfpathclose%
\pgfusepath{stroke,fill}%
\end{pgfscope}%
\begin{pgfscope}%
\pgfpathrectangle{\pgfqpoint{0.787074in}{0.548769in}}{\pgfqpoint{5.062926in}{3.102590in}}%
\pgfusepath{clip}%
\pgfsetbuttcap%
\pgfsetroundjoin%
\definecolor{currentfill}{rgb}{0.121569,0.466667,0.705882}%
\pgfsetfillcolor{currentfill}%
\pgfsetlinewidth{1.003750pt}%
\definecolor{currentstroke}{rgb}{0.121569,0.466667,0.705882}%
\pgfsetstrokecolor{currentstroke}%
\pgfsetdash{}{0pt}%
\pgfpathmoveto{\pgfqpoint{1.915516in}{0.658967in}}%
\pgfpathcurveto{\pgfqpoint{1.926566in}{0.658967in}}{\pgfqpoint{1.937165in}{0.663357in}}{\pgfqpoint{1.944979in}{0.671170in}}%
\pgfpathcurveto{\pgfqpoint{1.952792in}{0.678984in}}{\pgfqpoint{1.957182in}{0.689583in}}{\pgfqpoint{1.957182in}{0.700633in}}%
\pgfpathcurveto{\pgfqpoint{1.957182in}{0.711683in}}{\pgfqpoint{1.952792in}{0.722282in}}{\pgfqpoint{1.944979in}{0.730096in}}%
\pgfpathcurveto{\pgfqpoint{1.937165in}{0.737910in}}{\pgfqpoint{1.926566in}{0.742300in}}{\pgfqpoint{1.915516in}{0.742300in}}%
\pgfpathcurveto{\pgfqpoint{1.904466in}{0.742300in}}{\pgfqpoint{1.893867in}{0.737910in}}{\pgfqpoint{1.886053in}{0.730096in}}%
\pgfpathcurveto{\pgfqpoint{1.878239in}{0.722282in}}{\pgfqpoint{1.873849in}{0.711683in}}{\pgfqpoint{1.873849in}{0.700633in}}%
\pgfpathcurveto{\pgfqpoint{1.873849in}{0.689583in}}{\pgfqpoint{1.878239in}{0.678984in}}{\pgfqpoint{1.886053in}{0.671170in}}%
\pgfpathcurveto{\pgfqpoint{1.893867in}{0.663357in}}{\pgfqpoint{1.904466in}{0.658967in}}{\pgfqpoint{1.915516in}{0.658967in}}%
\pgfpathclose%
\pgfusepath{stroke,fill}%
\end{pgfscope}%
\begin{pgfscope}%
\pgfpathrectangle{\pgfqpoint{0.787074in}{0.548769in}}{\pgfqpoint{5.062926in}{3.102590in}}%
\pgfusepath{clip}%
\pgfsetbuttcap%
\pgfsetroundjoin%
\definecolor{currentfill}{rgb}{1.000000,0.498039,0.054902}%
\pgfsetfillcolor{currentfill}%
\pgfsetlinewidth{1.003750pt}%
\definecolor{currentstroke}{rgb}{1.000000,0.498039,0.054902}%
\pgfsetstrokecolor{currentstroke}%
\pgfsetdash{}{0pt}%
\pgfpathmoveto{\pgfqpoint{3.051199in}{2.623396in}}%
\pgfpathcurveto{\pgfqpoint{3.062249in}{2.623396in}}{\pgfqpoint{3.072848in}{2.627787in}}{\pgfqpoint{3.080661in}{2.635600in}}%
\pgfpathcurveto{\pgfqpoint{3.088475in}{2.643414in}}{\pgfqpoint{3.092865in}{2.654013in}}{\pgfqpoint{3.092865in}{2.665063in}}%
\pgfpathcurveto{\pgfqpoint{3.092865in}{2.676113in}}{\pgfqpoint{3.088475in}{2.686712in}}{\pgfqpoint{3.080661in}{2.694526in}}%
\pgfpathcurveto{\pgfqpoint{3.072848in}{2.702339in}}{\pgfqpoint{3.062249in}{2.706730in}}{\pgfqpoint{3.051199in}{2.706730in}}%
\pgfpathcurveto{\pgfqpoint{3.040149in}{2.706730in}}{\pgfqpoint{3.029550in}{2.702339in}}{\pgfqpoint{3.021736in}{2.694526in}}%
\pgfpathcurveto{\pgfqpoint{3.013922in}{2.686712in}}{\pgfqpoint{3.009532in}{2.676113in}}{\pgfqpoint{3.009532in}{2.665063in}}%
\pgfpathcurveto{\pgfqpoint{3.009532in}{2.654013in}}{\pgfqpoint{3.013922in}{2.643414in}}{\pgfqpoint{3.021736in}{2.635600in}}%
\pgfpathcurveto{\pgfqpoint{3.029550in}{2.627787in}}{\pgfqpoint{3.040149in}{2.623396in}}{\pgfqpoint{3.051199in}{2.623396in}}%
\pgfpathclose%
\pgfusepath{stroke,fill}%
\end{pgfscope}%
\begin{pgfscope}%
\pgfpathrectangle{\pgfqpoint{0.787074in}{0.548769in}}{\pgfqpoint{5.062926in}{3.102590in}}%
\pgfusepath{clip}%
\pgfsetbuttcap%
\pgfsetroundjoin%
\definecolor{currentfill}{rgb}{1.000000,0.498039,0.054902}%
\pgfsetfillcolor{currentfill}%
\pgfsetlinewidth{1.003750pt}%
\definecolor{currentstroke}{rgb}{1.000000,0.498039,0.054902}%
\pgfsetstrokecolor{currentstroke}%
\pgfsetdash{}{0pt}%
\pgfpathmoveto{\pgfqpoint{1.845955in}{2.816554in}}%
\pgfpathcurveto{\pgfqpoint{1.857005in}{2.816554in}}{\pgfqpoint{1.867604in}{2.820944in}}{\pgfqpoint{1.875418in}{2.828758in}}%
\pgfpathcurveto{\pgfqpoint{1.883232in}{2.836572in}}{\pgfqpoint{1.887622in}{2.847171in}}{\pgfqpoint{1.887622in}{2.858221in}}%
\pgfpathcurveto{\pgfqpoint{1.887622in}{2.869271in}}{\pgfqpoint{1.883232in}{2.879870in}}{\pgfqpoint{1.875418in}{2.887684in}}%
\pgfpathcurveto{\pgfqpoint{1.867604in}{2.895497in}}{\pgfqpoint{1.857005in}{2.899887in}}{\pgfqpoint{1.845955in}{2.899887in}}%
\pgfpathcurveto{\pgfqpoint{1.834905in}{2.899887in}}{\pgfqpoint{1.824306in}{2.895497in}}{\pgfqpoint{1.816493in}{2.887684in}}%
\pgfpathcurveto{\pgfqpoint{1.808679in}{2.879870in}}{\pgfqpoint{1.804289in}{2.869271in}}{\pgfqpoint{1.804289in}{2.858221in}}%
\pgfpathcurveto{\pgfqpoint{1.804289in}{2.847171in}}{\pgfqpoint{1.808679in}{2.836572in}}{\pgfqpoint{1.816493in}{2.828758in}}%
\pgfpathcurveto{\pgfqpoint{1.824306in}{2.820944in}}{\pgfqpoint{1.834905in}{2.816554in}}{\pgfqpoint{1.845955in}{2.816554in}}%
\pgfpathclose%
\pgfusepath{stroke,fill}%
\end{pgfscope}%
\begin{pgfscope}%
\pgfpathrectangle{\pgfqpoint{0.787074in}{0.548769in}}{\pgfqpoint{5.062926in}{3.102590in}}%
\pgfusepath{clip}%
\pgfsetbuttcap%
\pgfsetroundjoin%
\definecolor{currentfill}{rgb}{1.000000,0.498039,0.054902}%
\pgfsetfillcolor{currentfill}%
\pgfsetlinewidth{1.003750pt}%
\definecolor{currentstroke}{rgb}{1.000000,0.498039,0.054902}%
\pgfsetstrokecolor{currentstroke}%
\pgfsetdash{}{0pt}%
\pgfpathmoveto{\pgfqpoint{1.985110in}{2.998783in}}%
\pgfpathcurveto{\pgfqpoint{1.996160in}{2.998783in}}{\pgfqpoint{2.006759in}{3.003173in}}{\pgfqpoint{2.014573in}{3.010987in}}%
\pgfpathcurveto{\pgfqpoint{2.022387in}{3.018800in}}{\pgfqpoint{2.026777in}{3.029399in}}{\pgfqpoint{2.026777in}{3.040450in}}%
\pgfpathcurveto{\pgfqpoint{2.026777in}{3.051500in}}{\pgfqpoint{2.022387in}{3.062099in}}{\pgfqpoint{2.014573in}{3.069912in}}%
\pgfpathcurveto{\pgfqpoint{2.006759in}{3.077726in}}{\pgfqpoint{1.996160in}{3.082116in}}{\pgfqpoint{1.985110in}{3.082116in}}%
\pgfpathcurveto{\pgfqpoint{1.974060in}{3.082116in}}{\pgfqpoint{1.963461in}{3.077726in}}{\pgfqpoint{1.955647in}{3.069912in}}%
\pgfpathcurveto{\pgfqpoint{1.947834in}{3.062099in}}{\pgfqpoint{1.943444in}{3.051500in}}{\pgfqpoint{1.943444in}{3.040450in}}%
\pgfpathcurveto{\pgfqpoint{1.943444in}{3.029399in}}{\pgfqpoint{1.947834in}{3.018800in}}{\pgfqpoint{1.955647in}{3.010987in}}%
\pgfpathcurveto{\pgfqpoint{1.963461in}{3.003173in}}{\pgfqpoint{1.974060in}{2.998783in}}{\pgfqpoint{1.985110in}{2.998783in}}%
\pgfpathclose%
\pgfusepath{stroke,fill}%
\end{pgfscope}%
\begin{pgfscope}%
\pgfpathrectangle{\pgfqpoint{0.787074in}{0.548769in}}{\pgfqpoint{5.062926in}{3.102590in}}%
\pgfusepath{clip}%
\pgfsetbuttcap%
\pgfsetroundjoin%
\definecolor{currentfill}{rgb}{1.000000,0.498039,0.054902}%
\pgfsetfillcolor{currentfill}%
\pgfsetlinewidth{1.003750pt}%
\definecolor{currentstroke}{rgb}{1.000000,0.498039,0.054902}%
\pgfsetstrokecolor{currentstroke}%
\pgfsetdash{}{0pt}%
\pgfpathmoveto{\pgfqpoint{1.692734in}{2.057042in}}%
\pgfpathcurveto{\pgfqpoint{1.703784in}{2.057042in}}{\pgfqpoint{1.714383in}{2.061432in}}{\pgfqpoint{1.722197in}{2.069246in}}%
\pgfpathcurveto{\pgfqpoint{1.730010in}{2.077059in}}{\pgfqpoint{1.734401in}{2.087659in}}{\pgfqpoint{1.734401in}{2.098709in}}%
\pgfpathcurveto{\pgfqpoint{1.734401in}{2.109759in}}{\pgfqpoint{1.730010in}{2.120358in}}{\pgfqpoint{1.722197in}{2.128171in}}%
\pgfpathcurveto{\pgfqpoint{1.714383in}{2.135985in}}{\pgfqpoint{1.703784in}{2.140375in}}{\pgfqpoint{1.692734in}{2.140375in}}%
\pgfpathcurveto{\pgfqpoint{1.681684in}{2.140375in}}{\pgfqpoint{1.671085in}{2.135985in}}{\pgfqpoint{1.663271in}{2.128171in}}%
\pgfpathcurveto{\pgfqpoint{1.655458in}{2.120358in}}{\pgfqpoint{1.651067in}{2.109759in}}{\pgfqpoint{1.651067in}{2.098709in}}%
\pgfpathcurveto{\pgfqpoint{1.651067in}{2.087659in}}{\pgfqpoint{1.655458in}{2.077059in}}{\pgfqpoint{1.663271in}{2.069246in}}%
\pgfpathcurveto{\pgfqpoint{1.671085in}{2.061432in}}{\pgfqpoint{1.681684in}{2.057042in}}{\pgfqpoint{1.692734in}{2.057042in}}%
\pgfpathclose%
\pgfusepath{stroke,fill}%
\end{pgfscope}%
\begin{pgfscope}%
\pgfpathrectangle{\pgfqpoint{0.787074in}{0.548769in}}{\pgfqpoint{5.062926in}{3.102590in}}%
\pgfusepath{clip}%
\pgfsetbuttcap%
\pgfsetroundjoin%
\definecolor{currentfill}{rgb}{1.000000,0.498039,0.054902}%
\pgfsetfillcolor{currentfill}%
\pgfsetlinewidth{1.003750pt}%
\definecolor{currentstroke}{rgb}{1.000000,0.498039,0.054902}%
\pgfsetstrokecolor{currentstroke}%
\pgfsetdash{}{0pt}%
\pgfpathmoveto{\pgfqpoint{1.150842in}{3.349495in}}%
\pgfpathcurveto{\pgfqpoint{1.161892in}{3.349495in}}{\pgfqpoint{1.172491in}{3.353886in}}{\pgfqpoint{1.180305in}{3.361699in}}%
\pgfpathcurveto{\pgfqpoint{1.188118in}{3.369513in}}{\pgfqpoint{1.192509in}{3.380112in}}{\pgfqpoint{1.192509in}{3.391162in}}%
\pgfpathcurveto{\pgfqpoint{1.192509in}{3.402212in}}{\pgfqpoint{1.188118in}{3.412811in}}{\pgfqpoint{1.180305in}{3.420625in}}%
\pgfpathcurveto{\pgfqpoint{1.172491in}{3.428438in}}{\pgfqpoint{1.161892in}{3.432829in}}{\pgfqpoint{1.150842in}{3.432829in}}%
\pgfpathcurveto{\pgfqpoint{1.139792in}{3.432829in}}{\pgfqpoint{1.129193in}{3.428438in}}{\pgfqpoint{1.121379in}{3.420625in}}%
\pgfpathcurveto{\pgfqpoint{1.113566in}{3.412811in}}{\pgfqpoint{1.109175in}{3.402212in}}{\pgfqpoint{1.109175in}{3.391162in}}%
\pgfpathcurveto{\pgfqpoint{1.109175in}{3.380112in}}{\pgfqpoint{1.113566in}{3.369513in}}{\pgfqpoint{1.121379in}{3.361699in}}%
\pgfpathcurveto{\pgfqpoint{1.129193in}{3.353886in}}{\pgfqpoint{1.139792in}{3.349495in}}{\pgfqpoint{1.150842in}{3.349495in}}%
\pgfpathclose%
\pgfusepath{stroke,fill}%
\end{pgfscope}%
\begin{pgfscope}%
\pgfpathrectangle{\pgfqpoint{0.787074in}{0.548769in}}{\pgfqpoint{5.062926in}{3.102590in}}%
\pgfusepath{clip}%
\pgfsetbuttcap%
\pgfsetroundjoin%
\definecolor{currentfill}{rgb}{1.000000,0.498039,0.054902}%
\pgfsetfillcolor{currentfill}%
\pgfsetlinewidth{1.003750pt}%
\definecolor{currentstroke}{rgb}{1.000000,0.498039,0.054902}%
\pgfsetstrokecolor{currentstroke}%
\pgfsetdash{}{0pt}%
\pgfpathmoveto{\pgfqpoint{1.883042in}{2.864404in}}%
\pgfpathcurveto{\pgfqpoint{1.894092in}{2.864404in}}{\pgfqpoint{1.904691in}{2.868794in}}{\pgfqpoint{1.912505in}{2.876608in}}%
\pgfpathcurveto{\pgfqpoint{1.920318in}{2.884422in}}{\pgfqpoint{1.924708in}{2.895021in}}{\pgfqpoint{1.924708in}{2.906071in}}%
\pgfpathcurveto{\pgfqpoint{1.924708in}{2.917121in}}{\pgfqpoint{1.920318in}{2.927720in}}{\pgfqpoint{1.912505in}{2.935534in}}%
\pgfpathcurveto{\pgfqpoint{1.904691in}{2.943347in}}{\pgfqpoint{1.894092in}{2.947737in}}{\pgfqpoint{1.883042in}{2.947737in}}%
\pgfpathcurveto{\pgfqpoint{1.871992in}{2.947737in}}{\pgfqpoint{1.861393in}{2.943347in}}{\pgfqpoint{1.853579in}{2.935534in}}%
\pgfpathcurveto{\pgfqpoint{1.845765in}{2.927720in}}{\pgfqpoint{1.841375in}{2.917121in}}{\pgfqpoint{1.841375in}{2.906071in}}%
\pgfpathcurveto{\pgfqpoint{1.841375in}{2.895021in}}{\pgfqpoint{1.845765in}{2.884422in}}{\pgfqpoint{1.853579in}{2.876608in}}%
\pgfpathcurveto{\pgfqpoint{1.861393in}{2.868794in}}{\pgfqpoint{1.871992in}{2.864404in}}{\pgfqpoint{1.883042in}{2.864404in}}%
\pgfpathclose%
\pgfusepath{stroke,fill}%
\end{pgfscope}%
\begin{pgfscope}%
\pgfpathrectangle{\pgfqpoint{0.787074in}{0.548769in}}{\pgfqpoint{5.062926in}{3.102590in}}%
\pgfusepath{clip}%
\pgfsetbuttcap%
\pgfsetroundjoin%
\definecolor{currentfill}{rgb}{0.121569,0.466667,0.705882}%
\pgfsetfillcolor{currentfill}%
\pgfsetlinewidth{1.003750pt}%
\definecolor{currentstroke}{rgb}{0.121569,0.466667,0.705882}%
\pgfsetstrokecolor{currentstroke}%
\pgfsetdash{}{0pt}%
\pgfpathmoveto{\pgfqpoint{1.873130in}{0.648143in}}%
\pgfpathcurveto{\pgfqpoint{1.884180in}{0.648143in}}{\pgfqpoint{1.894779in}{0.652534in}}{\pgfqpoint{1.902593in}{0.660347in}}%
\pgfpathcurveto{\pgfqpoint{1.910406in}{0.668161in}}{\pgfqpoint{1.914797in}{0.678760in}}{\pgfqpoint{1.914797in}{0.689810in}}%
\pgfpathcurveto{\pgfqpoint{1.914797in}{0.700860in}}{\pgfqpoint{1.910406in}{0.711459in}}{\pgfqpoint{1.902593in}{0.719273in}}%
\pgfpathcurveto{\pgfqpoint{1.894779in}{0.727086in}}{\pgfqpoint{1.884180in}{0.731477in}}{\pgfqpoint{1.873130in}{0.731477in}}%
\pgfpathcurveto{\pgfqpoint{1.862080in}{0.731477in}}{\pgfqpoint{1.851481in}{0.727086in}}{\pgfqpoint{1.843667in}{0.719273in}}%
\pgfpathcurveto{\pgfqpoint{1.835854in}{0.711459in}}{\pgfqpoint{1.831463in}{0.700860in}}{\pgfqpoint{1.831463in}{0.689810in}}%
\pgfpathcurveto{\pgfqpoint{1.831463in}{0.678760in}}{\pgfqpoint{1.835854in}{0.668161in}}{\pgfqpoint{1.843667in}{0.660347in}}%
\pgfpathcurveto{\pgfqpoint{1.851481in}{0.652534in}}{\pgfqpoint{1.862080in}{0.648143in}}{\pgfqpoint{1.873130in}{0.648143in}}%
\pgfpathclose%
\pgfusepath{stroke,fill}%
\end{pgfscope}%
\begin{pgfscope}%
\pgfpathrectangle{\pgfqpoint{0.787074in}{0.548769in}}{\pgfqpoint{5.062926in}{3.102590in}}%
\pgfusepath{clip}%
\pgfsetbuttcap%
\pgfsetroundjoin%
\definecolor{currentfill}{rgb}{0.121569,0.466667,0.705882}%
\pgfsetfillcolor{currentfill}%
\pgfsetlinewidth{1.003750pt}%
\definecolor{currentstroke}{rgb}{0.121569,0.466667,0.705882}%
\pgfsetstrokecolor{currentstroke}%
\pgfsetdash{}{0pt}%
\pgfpathmoveto{\pgfqpoint{1.789189in}{0.648148in}}%
\pgfpathcurveto{\pgfqpoint{1.800240in}{0.648148in}}{\pgfqpoint{1.810839in}{0.652538in}}{\pgfqpoint{1.818652in}{0.660352in}}%
\pgfpathcurveto{\pgfqpoint{1.826466in}{0.668165in}}{\pgfqpoint{1.830856in}{0.678764in}}{\pgfqpoint{1.830856in}{0.689815in}}%
\pgfpathcurveto{\pgfqpoint{1.830856in}{0.700865in}}{\pgfqpoint{1.826466in}{0.711464in}}{\pgfqpoint{1.818652in}{0.719277in}}%
\pgfpathcurveto{\pgfqpoint{1.810839in}{0.727091in}}{\pgfqpoint{1.800240in}{0.731481in}}{\pgfqpoint{1.789189in}{0.731481in}}%
\pgfpathcurveto{\pgfqpoint{1.778139in}{0.731481in}}{\pgfqpoint{1.767540in}{0.727091in}}{\pgfqpoint{1.759727in}{0.719277in}}%
\pgfpathcurveto{\pgfqpoint{1.751913in}{0.711464in}}{\pgfqpoint{1.747523in}{0.700865in}}{\pgfqpoint{1.747523in}{0.689815in}}%
\pgfpathcurveto{\pgfqpoint{1.747523in}{0.678764in}}{\pgfqpoint{1.751913in}{0.668165in}}{\pgfqpoint{1.759727in}{0.660352in}}%
\pgfpathcurveto{\pgfqpoint{1.767540in}{0.652538in}}{\pgfqpoint{1.778139in}{0.648148in}}{\pgfqpoint{1.789189in}{0.648148in}}%
\pgfpathclose%
\pgfusepath{stroke,fill}%
\end{pgfscope}%
\begin{pgfscope}%
\pgfpathrectangle{\pgfqpoint{0.787074in}{0.548769in}}{\pgfqpoint{5.062926in}{3.102590in}}%
\pgfusepath{clip}%
\pgfsetbuttcap%
\pgfsetroundjoin%
\definecolor{currentfill}{rgb}{0.121569,0.466667,0.705882}%
\pgfsetfillcolor{currentfill}%
\pgfsetlinewidth{1.003750pt}%
\definecolor{currentstroke}{rgb}{0.121569,0.466667,0.705882}%
\pgfsetstrokecolor{currentstroke}%
\pgfsetdash{}{0pt}%
\pgfpathmoveto{\pgfqpoint{1.705469in}{0.648152in}}%
\pgfpathcurveto{\pgfqpoint{1.716519in}{0.648152in}}{\pgfqpoint{1.727118in}{0.652542in}}{\pgfqpoint{1.734932in}{0.660355in}}%
\pgfpathcurveto{\pgfqpoint{1.742746in}{0.668169in}}{\pgfqpoint{1.747136in}{0.678768in}}{\pgfqpoint{1.747136in}{0.689818in}}%
\pgfpathcurveto{\pgfqpoint{1.747136in}{0.700868in}}{\pgfqpoint{1.742746in}{0.711467in}}{\pgfqpoint{1.734932in}{0.719281in}}%
\pgfpathcurveto{\pgfqpoint{1.727118in}{0.727095in}}{\pgfqpoint{1.716519in}{0.731485in}}{\pgfqpoint{1.705469in}{0.731485in}}%
\pgfpathcurveto{\pgfqpoint{1.694419in}{0.731485in}}{\pgfqpoint{1.683820in}{0.727095in}}{\pgfqpoint{1.676006in}{0.719281in}}%
\pgfpathcurveto{\pgfqpoint{1.668193in}{0.711467in}}{\pgfqpoint{1.663803in}{0.700868in}}{\pgfqpoint{1.663803in}{0.689818in}}%
\pgfpathcurveto{\pgfqpoint{1.663803in}{0.678768in}}{\pgfqpoint{1.668193in}{0.668169in}}{\pgfqpoint{1.676006in}{0.660355in}}%
\pgfpathcurveto{\pgfqpoint{1.683820in}{0.652542in}}{\pgfqpoint{1.694419in}{0.648152in}}{\pgfqpoint{1.705469in}{0.648152in}}%
\pgfpathclose%
\pgfusepath{stroke,fill}%
\end{pgfscope}%
\begin{pgfscope}%
\pgfpathrectangle{\pgfqpoint{0.787074in}{0.548769in}}{\pgfqpoint{5.062926in}{3.102590in}}%
\pgfusepath{clip}%
\pgfsetbuttcap%
\pgfsetroundjoin%
\definecolor{currentfill}{rgb}{1.000000,0.498039,0.054902}%
\pgfsetfillcolor{currentfill}%
\pgfsetlinewidth{1.003750pt}%
\definecolor{currentstroke}{rgb}{1.000000,0.498039,0.054902}%
\pgfsetstrokecolor{currentstroke}%
\pgfsetdash{}{0pt}%
\pgfpathmoveto{\pgfqpoint{1.435876in}{2.884914in}}%
\pgfpathcurveto{\pgfqpoint{1.446926in}{2.884914in}}{\pgfqpoint{1.457525in}{2.889305in}}{\pgfqpoint{1.465338in}{2.897118in}}%
\pgfpathcurveto{\pgfqpoint{1.473152in}{2.904932in}}{\pgfqpoint{1.477542in}{2.915531in}}{\pgfqpoint{1.477542in}{2.926581in}}%
\pgfpathcurveto{\pgfqpoint{1.477542in}{2.937631in}}{\pgfqpoint{1.473152in}{2.948230in}}{\pgfqpoint{1.465338in}{2.956044in}}%
\pgfpathcurveto{\pgfqpoint{1.457525in}{2.963857in}}{\pgfqpoint{1.446926in}{2.968248in}}{\pgfqpoint{1.435876in}{2.968248in}}%
\pgfpathcurveto{\pgfqpoint{1.424825in}{2.968248in}}{\pgfqpoint{1.414226in}{2.963857in}}{\pgfqpoint{1.406413in}{2.956044in}}%
\pgfpathcurveto{\pgfqpoint{1.398599in}{2.948230in}}{\pgfqpoint{1.394209in}{2.937631in}}{\pgfqpoint{1.394209in}{2.926581in}}%
\pgfpathcurveto{\pgfqpoint{1.394209in}{2.915531in}}{\pgfqpoint{1.398599in}{2.904932in}}{\pgfqpoint{1.406413in}{2.897118in}}%
\pgfpathcurveto{\pgfqpoint{1.414226in}{2.889305in}}{\pgfqpoint{1.424825in}{2.884914in}}{\pgfqpoint{1.435876in}{2.884914in}}%
\pgfpathclose%
\pgfusepath{stroke,fill}%
\end{pgfscope}%
\begin{pgfscope}%
\pgfpathrectangle{\pgfqpoint{0.787074in}{0.548769in}}{\pgfqpoint{5.062926in}{3.102590in}}%
\pgfusepath{clip}%
\pgfsetbuttcap%
\pgfsetroundjoin%
\definecolor{currentfill}{rgb}{1.000000,0.498039,0.054902}%
\pgfsetfillcolor{currentfill}%
\pgfsetlinewidth{1.003750pt}%
\definecolor{currentstroke}{rgb}{1.000000,0.498039,0.054902}%
\pgfsetstrokecolor{currentstroke}%
\pgfsetdash{}{0pt}%
\pgfpathmoveto{\pgfqpoint{2.198709in}{2.875764in}}%
\pgfpathcurveto{\pgfqpoint{2.209760in}{2.875764in}}{\pgfqpoint{2.220359in}{2.880154in}}{\pgfqpoint{2.228172in}{2.887968in}}%
\pgfpathcurveto{\pgfqpoint{2.235986in}{2.895781in}}{\pgfqpoint{2.240376in}{2.906380in}}{\pgfqpoint{2.240376in}{2.917430in}}%
\pgfpathcurveto{\pgfqpoint{2.240376in}{2.928481in}}{\pgfqpoint{2.235986in}{2.939080in}}{\pgfqpoint{2.228172in}{2.946893in}}%
\pgfpathcurveto{\pgfqpoint{2.220359in}{2.954707in}}{\pgfqpoint{2.209760in}{2.959097in}}{\pgfqpoint{2.198709in}{2.959097in}}%
\pgfpathcurveto{\pgfqpoint{2.187659in}{2.959097in}}{\pgfqpoint{2.177060in}{2.954707in}}{\pgfqpoint{2.169247in}{2.946893in}}%
\pgfpathcurveto{\pgfqpoint{2.161433in}{2.939080in}}{\pgfqpoint{2.157043in}{2.928481in}}{\pgfqpoint{2.157043in}{2.917430in}}%
\pgfpathcurveto{\pgfqpoint{2.157043in}{2.906380in}}{\pgfqpoint{2.161433in}{2.895781in}}{\pgfqpoint{2.169247in}{2.887968in}}%
\pgfpathcurveto{\pgfqpoint{2.177060in}{2.880154in}}{\pgfqpoint{2.187659in}{2.875764in}}{\pgfqpoint{2.198709in}{2.875764in}}%
\pgfpathclose%
\pgfusepath{stroke,fill}%
\end{pgfscope}%
\begin{pgfscope}%
\pgfpathrectangle{\pgfqpoint{0.787074in}{0.548769in}}{\pgfqpoint{5.062926in}{3.102590in}}%
\pgfusepath{clip}%
\pgfsetbuttcap%
\pgfsetroundjoin%
\definecolor{currentfill}{rgb}{1.000000,0.498039,0.054902}%
\pgfsetfillcolor{currentfill}%
\pgfsetlinewidth{1.003750pt}%
\definecolor{currentstroke}{rgb}{1.000000,0.498039,0.054902}%
\pgfsetstrokecolor{currentstroke}%
\pgfsetdash{}{0pt}%
\pgfpathmoveto{\pgfqpoint{1.937620in}{2.086876in}}%
\pgfpathcurveto{\pgfqpoint{1.948670in}{2.086876in}}{\pgfqpoint{1.959269in}{2.091266in}}{\pgfqpoint{1.967083in}{2.099080in}}%
\pgfpathcurveto{\pgfqpoint{1.974897in}{2.106893in}}{\pgfqpoint{1.979287in}{2.117492in}}{\pgfqpoint{1.979287in}{2.128542in}}%
\pgfpathcurveto{\pgfqpoint{1.979287in}{2.139593in}}{\pgfqpoint{1.974897in}{2.150192in}}{\pgfqpoint{1.967083in}{2.158005in}}%
\pgfpathcurveto{\pgfqpoint{1.959269in}{2.165819in}}{\pgfqpoint{1.948670in}{2.170209in}}{\pgfqpoint{1.937620in}{2.170209in}}%
\pgfpathcurveto{\pgfqpoint{1.926570in}{2.170209in}}{\pgfqpoint{1.915971in}{2.165819in}}{\pgfqpoint{1.908157in}{2.158005in}}%
\pgfpathcurveto{\pgfqpoint{1.900344in}{2.150192in}}{\pgfqpoint{1.895954in}{2.139593in}}{\pgfqpoint{1.895954in}{2.128542in}}%
\pgfpathcurveto{\pgfqpoint{1.895954in}{2.117492in}}{\pgfqpoint{1.900344in}{2.106893in}}{\pgfqpoint{1.908157in}{2.099080in}}%
\pgfpathcurveto{\pgfqpoint{1.915971in}{2.091266in}}{\pgfqpoint{1.926570in}{2.086876in}}{\pgfqpoint{1.937620in}{2.086876in}}%
\pgfpathclose%
\pgfusepath{stroke,fill}%
\end{pgfscope}%
\begin{pgfscope}%
\pgfpathrectangle{\pgfqpoint{0.787074in}{0.548769in}}{\pgfqpoint{5.062926in}{3.102590in}}%
\pgfusepath{clip}%
\pgfsetbuttcap%
\pgfsetroundjoin%
\definecolor{currentfill}{rgb}{0.121569,0.466667,0.705882}%
\pgfsetfillcolor{currentfill}%
\pgfsetlinewidth{1.003750pt}%
\definecolor{currentstroke}{rgb}{0.121569,0.466667,0.705882}%
\pgfsetstrokecolor{currentstroke}%
\pgfsetdash{}{0pt}%
\pgfpathmoveto{\pgfqpoint{1.099231in}{0.648132in}}%
\pgfpathcurveto{\pgfqpoint{1.110281in}{0.648132in}}{\pgfqpoint{1.120880in}{0.652523in}}{\pgfqpoint{1.128694in}{0.660336in}}%
\pgfpathcurveto{\pgfqpoint{1.136508in}{0.668150in}}{\pgfqpoint{1.140898in}{0.678749in}}{\pgfqpoint{1.140898in}{0.689799in}}%
\pgfpathcurveto{\pgfqpoint{1.140898in}{0.700849in}}{\pgfqpoint{1.136508in}{0.711448in}}{\pgfqpoint{1.128694in}{0.719262in}}%
\pgfpathcurveto{\pgfqpoint{1.120880in}{0.727075in}}{\pgfqpoint{1.110281in}{0.731466in}}{\pgfqpoint{1.099231in}{0.731466in}}%
\pgfpathcurveto{\pgfqpoint{1.088181in}{0.731466in}}{\pgfqpoint{1.077582in}{0.727075in}}{\pgfqpoint{1.069768in}{0.719262in}}%
\pgfpathcurveto{\pgfqpoint{1.061955in}{0.711448in}}{\pgfqpoint{1.057565in}{0.700849in}}{\pgfqpoint{1.057565in}{0.689799in}}%
\pgfpathcurveto{\pgfqpoint{1.057565in}{0.678749in}}{\pgfqpoint{1.061955in}{0.668150in}}{\pgfqpoint{1.069768in}{0.660336in}}%
\pgfpathcurveto{\pgfqpoint{1.077582in}{0.652523in}}{\pgfqpoint{1.088181in}{0.648132in}}{\pgfqpoint{1.099231in}{0.648132in}}%
\pgfpathclose%
\pgfusepath{stroke,fill}%
\end{pgfscope}%
\begin{pgfscope}%
\pgfpathrectangle{\pgfqpoint{0.787074in}{0.548769in}}{\pgfqpoint{5.062926in}{3.102590in}}%
\pgfusepath{clip}%
\pgfsetbuttcap%
\pgfsetroundjoin%
\definecolor{currentfill}{rgb}{0.121569,0.466667,0.705882}%
\pgfsetfillcolor{currentfill}%
\pgfsetlinewidth{1.003750pt}%
\definecolor{currentstroke}{rgb}{0.121569,0.466667,0.705882}%
\pgfsetstrokecolor{currentstroke}%
\pgfsetdash{}{0pt}%
\pgfpathmoveto{\pgfqpoint{1.017207in}{0.787529in}}%
\pgfpathcurveto{\pgfqpoint{1.028257in}{0.787529in}}{\pgfqpoint{1.038856in}{0.791919in}}{\pgfqpoint{1.046670in}{0.799733in}}%
\pgfpathcurveto{\pgfqpoint{1.054483in}{0.807547in}}{\pgfqpoint{1.058874in}{0.818146in}}{\pgfqpoint{1.058874in}{0.829196in}}%
\pgfpathcurveto{\pgfqpoint{1.058874in}{0.840246in}}{\pgfqpoint{1.054483in}{0.850845in}}{\pgfqpoint{1.046670in}{0.858658in}}%
\pgfpathcurveto{\pgfqpoint{1.038856in}{0.866472in}}{\pgfqpoint{1.028257in}{0.870862in}}{\pgfqpoint{1.017207in}{0.870862in}}%
\pgfpathcurveto{\pgfqpoint{1.006157in}{0.870862in}}{\pgfqpoint{0.995558in}{0.866472in}}{\pgfqpoint{0.987744in}{0.858658in}}%
\pgfpathcurveto{\pgfqpoint{0.979930in}{0.850845in}}{\pgfqpoint{0.975540in}{0.840246in}}{\pgfqpoint{0.975540in}{0.829196in}}%
\pgfpathcurveto{\pgfqpoint{0.975540in}{0.818146in}}{\pgfqpoint{0.979930in}{0.807547in}}{\pgfqpoint{0.987744in}{0.799733in}}%
\pgfpathcurveto{\pgfqpoint{0.995558in}{0.791919in}}{\pgfqpoint{1.006157in}{0.787529in}}{\pgfqpoint{1.017207in}{0.787529in}}%
\pgfpathclose%
\pgfusepath{stroke,fill}%
\end{pgfscope}%
\begin{pgfscope}%
\pgfpathrectangle{\pgfqpoint{0.787074in}{0.548769in}}{\pgfqpoint{5.062926in}{3.102590in}}%
\pgfusepath{clip}%
\pgfsetbuttcap%
\pgfsetroundjoin%
\definecolor{currentfill}{rgb}{1.000000,0.498039,0.054902}%
\pgfsetfillcolor{currentfill}%
\pgfsetlinewidth{1.003750pt}%
\definecolor{currentstroke}{rgb}{1.000000,0.498039,0.054902}%
\pgfsetstrokecolor{currentstroke}%
\pgfsetdash{}{0pt}%
\pgfpathmoveto{\pgfqpoint{1.851297in}{2.073153in}}%
\pgfpathcurveto{\pgfqpoint{1.862347in}{2.073153in}}{\pgfqpoint{1.872946in}{2.077543in}}{\pgfqpoint{1.880760in}{2.085357in}}%
\pgfpathcurveto{\pgfqpoint{1.888573in}{2.093170in}}{\pgfqpoint{1.892964in}{2.103769in}}{\pgfqpoint{1.892964in}{2.114820in}}%
\pgfpathcurveto{\pgfqpoint{1.892964in}{2.125870in}}{\pgfqpoint{1.888573in}{2.136469in}}{\pgfqpoint{1.880760in}{2.144282in}}%
\pgfpathcurveto{\pgfqpoint{1.872946in}{2.152096in}}{\pgfqpoint{1.862347in}{2.156486in}}{\pgfqpoint{1.851297in}{2.156486in}}%
\pgfpathcurveto{\pgfqpoint{1.840247in}{2.156486in}}{\pgfqpoint{1.829648in}{2.152096in}}{\pgfqpoint{1.821834in}{2.144282in}}%
\pgfpathcurveto{\pgfqpoint{1.814021in}{2.136469in}}{\pgfqpoint{1.809630in}{2.125870in}}{\pgfqpoint{1.809630in}{2.114820in}}%
\pgfpathcurveto{\pgfqpoint{1.809630in}{2.103769in}}{\pgfqpoint{1.814021in}{2.093170in}}{\pgfqpoint{1.821834in}{2.085357in}}%
\pgfpathcurveto{\pgfqpoint{1.829648in}{2.077543in}}{\pgfqpoint{1.840247in}{2.073153in}}{\pgfqpoint{1.851297in}{2.073153in}}%
\pgfpathclose%
\pgfusepath{stroke,fill}%
\end{pgfscope}%
\begin{pgfscope}%
\pgfpathrectangle{\pgfqpoint{0.787074in}{0.548769in}}{\pgfqpoint{5.062926in}{3.102590in}}%
\pgfusepath{clip}%
\pgfsetbuttcap%
\pgfsetroundjoin%
\definecolor{currentfill}{rgb}{0.121569,0.466667,0.705882}%
\pgfsetfillcolor{currentfill}%
\pgfsetlinewidth{1.003750pt}%
\definecolor{currentstroke}{rgb}{0.121569,0.466667,0.705882}%
\pgfsetstrokecolor{currentstroke}%
\pgfsetdash{}{0pt}%
\pgfpathmoveto{\pgfqpoint{1.814609in}{0.648148in}}%
\pgfpathcurveto{\pgfqpoint{1.825659in}{0.648148in}}{\pgfqpoint{1.836258in}{0.652539in}}{\pgfqpoint{1.844072in}{0.660352in}}%
\pgfpathcurveto{\pgfqpoint{1.851885in}{0.668166in}}{\pgfqpoint{1.856276in}{0.678765in}}{\pgfqpoint{1.856276in}{0.689815in}}%
\pgfpathcurveto{\pgfqpoint{1.856276in}{0.700865in}}{\pgfqpoint{1.851885in}{0.711464in}}{\pgfqpoint{1.844072in}{0.719278in}}%
\pgfpathcurveto{\pgfqpoint{1.836258in}{0.727091in}}{\pgfqpoint{1.825659in}{0.731482in}}{\pgfqpoint{1.814609in}{0.731482in}}%
\pgfpathcurveto{\pgfqpoint{1.803559in}{0.731482in}}{\pgfqpoint{1.792960in}{0.727091in}}{\pgfqpoint{1.785146in}{0.719278in}}%
\pgfpathcurveto{\pgfqpoint{1.777333in}{0.711464in}}{\pgfqpoint{1.772942in}{0.700865in}}{\pgfqpoint{1.772942in}{0.689815in}}%
\pgfpathcurveto{\pgfqpoint{1.772942in}{0.678765in}}{\pgfqpoint{1.777333in}{0.668166in}}{\pgfqpoint{1.785146in}{0.660352in}}%
\pgfpathcurveto{\pgfqpoint{1.792960in}{0.652539in}}{\pgfqpoint{1.803559in}{0.648148in}}{\pgfqpoint{1.814609in}{0.648148in}}%
\pgfpathclose%
\pgfusepath{stroke,fill}%
\end{pgfscope}%
\begin{pgfscope}%
\pgfpathrectangle{\pgfqpoint{0.787074in}{0.548769in}}{\pgfqpoint{5.062926in}{3.102590in}}%
\pgfusepath{clip}%
\pgfsetbuttcap%
\pgfsetroundjoin%
\definecolor{currentfill}{rgb}{0.121569,0.466667,0.705882}%
\pgfsetfillcolor{currentfill}%
\pgfsetlinewidth{1.003750pt}%
\definecolor{currentstroke}{rgb}{0.121569,0.466667,0.705882}%
\pgfsetstrokecolor{currentstroke}%
\pgfsetdash{}{0pt}%
\pgfpathmoveto{\pgfqpoint{1.345533in}{0.648150in}}%
\pgfpathcurveto{\pgfqpoint{1.356584in}{0.648150in}}{\pgfqpoint{1.367183in}{0.652540in}}{\pgfqpoint{1.374996in}{0.660353in}}%
\pgfpathcurveto{\pgfqpoint{1.382810in}{0.668167in}}{\pgfqpoint{1.387200in}{0.678766in}}{\pgfqpoint{1.387200in}{0.689816in}}%
\pgfpathcurveto{\pgfqpoint{1.387200in}{0.700866in}}{\pgfqpoint{1.382810in}{0.711465in}}{\pgfqpoint{1.374996in}{0.719279in}}%
\pgfpathcurveto{\pgfqpoint{1.367183in}{0.727093in}}{\pgfqpoint{1.356584in}{0.731483in}}{\pgfqpoint{1.345533in}{0.731483in}}%
\pgfpathcurveto{\pgfqpoint{1.334483in}{0.731483in}}{\pgfqpoint{1.323884in}{0.727093in}}{\pgfqpoint{1.316071in}{0.719279in}}%
\pgfpathcurveto{\pgfqpoint{1.308257in}{0.711465in}}{\pgfqpoint{1.303867in}{0.700866in}}{\pgfqpoint{1.303867in}{0.689816in}}%
\pgfpathcurveto{\pgfqpoint{1.303867in}{0.678766in}}{\pgfqpoint{1.308257in}{0.668167in}}{\pgfqpoint{1.316071in}{0.660353in}}%
\pgfpathcurveto{\pgfqpoint{1.323884in}{0.652540in}}{\pgfqpoint{1.334483in}{0.648150in}}{\pgfqpoint{1.345533in}{0.648150in}}%
\pgfpathclose%
\pgfusepath{stroke,fill}%
\end{pgfscope}%
\begin{pgfscope}%
\pgfpathrectangle{\pgfqpoint{0.787074in}{0.548769in}}{\pgfqpoint{5.062926in}{3.102590in}}%
\pgfusepath{clip}%
\pgfsetbuttcap%
\pgfsetroundjoin%
\definecolor{currentfill}{rgb}{0.121569,0.466667,0.705882}%
\pgfsetfillcolor{currentfill}%
\pgfsetlinewidth{1.003750pt}%
\definecolor{currentstroke}{rgb}{0.121569,0.466667,0.705882}%
\pgfsetstrokecolor{currentstroke}%
\pgfsetdash{}{0pt}%
\pgfpathmoveto{\pgfqpoint{1.030078in}{2.326735in}}%
\pgfpathcurveto{\pgfqpoint{1.041128in}{2.326735in}}{\pgfqpoint{1.051727in}{2.331125in}}{\pgfqpoint{1.059541in}{2.338939in}}%
\pgfpathcurveto{\pgfqpoint{1.067354in}{2.346753in}}{\pgfqpoint{1.071744in}{2.357352in}}{\pgfqpoint{1.071744in}{2.368402in}}%
\pgfpathcurveto{\pgfqpoint{1.071744in}{2.379452in}}{\pgfqpoint{1.067354in}{2.390051in}}{\pgfqpoint{1.059541in}{2.397865in}}%
\pgfpathcurveto{\pgfqpoint{1.051727in}{2.405678in}}{\pgfqpoint{1.041128in}{2.410068in}}{\pgfqpoint{1.030078in}{2.410068in}}%
\pgfpathcurveto{\pgfqpoint{1.019028in}{2.410068in}}{\pgfqpoint{1.008429in}{2.405678in}}{\pgfqpoint{1.000615in}{2.397865in}}%
\pgfpathcurveto{\pgfqpoint{0.992801in}{2.390051in}}{\pgfqpoint{0.988411in}{2.379452in}}{\pgfqpoint{0.988411in}{2.368402in}}%
\pgfpathcurveto{\pgfqpoint{0.988411in}{2.357352in}}{\pgfqpoint{0.992801in}{2.346753in}}{\pgfqpoint{1.000615in}{2.338939in}}%
\pgfpathcurveto{\pgfqpoint{1.008429in}{2.331125in}}{\pgfqpoint{1.019028in}{2.326735in}}{\pgfqpoint{1.030078in}{2.326735in}}%
\pgfpathclose%
\pgfusepath{stroke,fill}%
\end{pgfscope}%
\begin{pgfscope}%
\pgfpathrectangle{\pgfqpoint{0.787074in}{0.548769in}}{\pgfqpoint{5.062926in}{3.102590in}}%
\pgfusepath{clip}%
\pgfsetbuttcap%
\pgfsetroundjoin%
\definecolor{currentfill}{rgb}{1.000000,0.498039,0.054902}%
\pgfsetfillcolor{currentfill}%
\pgfsetlinewidth{1.003750pt}%
\definecolor{currentstroke}{rgb}{1.000000,0.498039,0.054902}%
\pgfsetstrokecolor{currentstroke}%
\pgfsetdash{}{0pt}%
\pgfpathmoveto{\pgfqpoint{1.339318in}{2.612427in}}%
\pgfpathcurveto{\pgfqpoint{1.350369in}{2.612427in}}{\pgfqpoint{1.360968in}{2.616817in}}{\pgfqpoint{1.368781in}{2.624631in}}%
\pgfpathcurveto{\pgfqpoint{1.376595in}{2.632444in}}{\pgfqpoint{1.380985in}{2.643043in}}{\pgfqpoint{1.380985in}{2.654094in}}%
\pgfpathcurveto{\pgfqpoint{1.380985in}{2.665144in}}{\pgfqpoint{1.376595in}{2.675743in}}{\pgfqpoint{1.368781in}{2.683556in}}%
\pgfpathcurveto{\pgfqpoint{1.360968in}{2.691370in}}{\pgfqpoint{1.350369in}{2.695760in}}{\pgfqpoint{1.339318in}{2.695760in}}%
\pgfpathcurveto{\pgfqpoint{1.328268in}{2.695760in}}{\pgfqpoint{1.317669in}{2.691370in}}{\pgfqpoint{1.309856in}{2.683556in}}%
\pgfpathcurveto{\pgfqpoint{1.302042in}{2.675743in}}{\pgfqpoint{1.297652in}{2.665144in}}{\pgfqpoint{1.297652in}{2.654094in}}%
\pgfpathcurveto{\pgfqpoint{1.297652in}{2.643043in}}{\pgfqpoint{1.302042in}{2.632444in}}{\pgfqpoint{1.309856in}{2.624631in}}%
\pgfpathcurveto{\pgfqpoint{1.317669in}{2.616817in}}{\pgfqpoint{1.328268in}{2.612427in}}{\pgfqpoint{1.339318in}{2.612427in}}%
\pgfpathclose%
\pgfusepath{stroke,fill}%
\end{pgfscope}%
\begin{pgfscope}%
\pgfpathrectangle{\pgfqpoint{0.787074in}{0.548769in}}{\pgfqpoint{5.062926in}{3.102590in}}%
\pgfusepath{clip}%
\pgfsetbuttcap%
\pgfsetroundjoin%
\definecolor{currentfill}{rgb}{1.000000,0.498039,0.054902}%
\pgfsetfillcolor{currentfill}%
\pgfsetlinewidth{1.003750pt}%
\definecolor{currentstroke}{rgb}{1.000000,0.498039,0.054902}%
\pgfsetstrokecolor{currentstroke}%
\pgfsetdash{}{0pt}%
\pgfpathmoveto{\pgfqpoint{1.292778in}{2.467225in}}%
\pgfpathcurveto{\pgfqpoint{1.303828in}{2.467225in}}{\pgfqpoint{1.314427in}{2.471616in}}{\pgfqpoint{1.322241in}{2.479429in}}%
\pgfpathcurveto{\pgfqpoint{1.330054in}{2.487243in}}{\pgfqpoint{1.334445in}{2.497842in}}{\pgfqpoint{1.334445in}{2.508892in}}%
\pgfpathcurveto{\pgfqpoint{1.334445in}{2.519942in}}{\pgfqpoint{1.330054in}{2.530541in}}{\pgfqpoint{1.322241in}{2.538355in}}%
\pgfpathcurveto{\pgfqpoint{1.314427in}{2.546168in}}{\pgfqpoint{1.303828in}{2.550559in}}{\pgfqpoint{1.292778in}{2.550559in}}%
\pgfpathcurveto{\pgfqpoint{1.281728in}{2.550559in}}{\pgfqpoint{1.271129in}{2.546168in}}{\pgfqpoint{1.263315in}{2.538355in}}%
\pgfpathcurveto{\pgfqpoint{1.255502in}{2.530541in}}{\pgfqpoint{1.251111in}{2.519942in}}{\pgfqpoint{1.251111in}{2.508892in}}%
\pgfpathcurveto{\pgfqpoint{1.251111in}{2.497842in}}{\pgfqpoint{1.255502in}{2.487243in}}{\pgfqpoint{1.263315in}{2.479429in}}%
\pgfpathcurveto{\pgfqpoint{1.271129in}{2.471616in}}{\pgfqpoint{1.281728in}{2.467225in}}{\pgfqpoint{1.292778in}{2.467225in}}%
\pgfpathclose%
\pgfusepath{stroke,fill}%
\end{pgfscope}%
\begin{pgfscope}%
\pgfpathrectangle{\pgfqpoint{0.787074in}{0.548769in}}{\pgfqpoint{5.062926in}{3.102590in}}%
\pgfusepath{clip}%
\pgfsetbuttcap%
\pgfsetroundjoin%
\definecolor{currentfill}{rgb}{1.000000,0.498039,0.054902}%
\pgfsetfillcolor{currentfill}%
\pgfsetlinewidth{1.003750pt}%
\definecolor{currentstroke}{rgb}{1.000000,0.498039,0.054902}%
\pgfsetstrokecolor{currentstroke}%
\pgfsetdash{}{0pt}%
\pgfpathmoveto{\pgfqpoint{2.309799in}{2.185530in}}%
\pgfpathcurveto{\pgfqpoint{2.320850in}{2.185530in}}{\pgfqpoint{2.331449in}{2.189921in}}{\pgfqpoint{2.339262in}{2.197734in}}%
\pgfpathcurveto{\pgfqpoint{2.347076in}{2.205548in}}{\pgfqpoint{2.351466in}{2.216147in}}{\pgfqpoint{2.351466in}{2.227197in}}%
\pgfpathcurveto{\pgfqpoint{2.351466in}{2.238247in}}{\pgfqpoint{2.347076in}{2.248846in}}{\pgfqpoint{2.339262in}{2.256660in}}%
\pgfpathcurveto{\pgfqpoint{2.331449in}{2.264473in}}{\pgfqpoint{2.320850in}{2.268864in}}{\pgfqpoint{2.309799in}{2.268864in}}%
\pgfpathcurveto{\pgfqpoint{2.298749in}{2.268864in}}{\pgfqpoint{2.288150in}{2.264473in}}{\pgfqpoint{2.280337in}{2.256660in}}%
\pgfpathcurveto{\pgfqpoint{2.272523in}{2.248846in}}{\pgfqpoint{2.268133in}{2.238247in}}{\pgfqpoint{2.268133in}{2.227197in}}%
\pgfpathcurveto{\pgfqpoint{2.268133in}{2.216147in}}{\pgfqpoint{2.272523in}{2.205548in}}{\pgfqpoint{2.280337in}{2.197734in}}%
\pgfpathcurveto{\pgfqpoint{2.288150in}{2.189921in}}{\pgfqpoint{2.298749in}{2.185530in}}{\pgfqpoint{2.309799in}{2.185530in}}%
\pgfpathclose%
\pgfusepath{stroke,fill}%
\end{pgfscope}%
\begin{pgfscope}%
\pgfpathrectangle{\pgfqpoint{0.787074in}{0.548769in}}{\pgfqpoint{5.062926in}{3.102590in}}%
\pgfusepath{clip}%
\pgfsetbuttcap%
\pgfsetroundjoin%
\definecolor{currentfill}{rgb}{1.000000,0.498039,0.054902}%
\pgfsetfillcolor{currentfill}%
\pgfsetlinewidth{1.003750pt}%
\definecolor{currentstroke}{rgb}{1.000000,0.498039,0.054902}%
\pgfsetstrokecolor{currentstroke}%
\pgfsetdash{}{0pt}%
\pgfpathmoveto{\pgfqpoint{2.800047in}{2.666690in}}%
\pgfpathcurveto{\pgfqpoint{2.811097in}{2.666690in}}{\pgfqpoint{2.821696in}{2.671080in}}{\pgfqpoint{2.829509in}{2.678894in}}%
\pgfpathcurveto{\pgfqpoint{2.837323in}{2.686707in}}{\pgfqpoint{2.841713in}{2.697306in}}{\pgfqpoint{2.841713in}{2.708356in}}%
\pgfpathcurveto{\pgfqpoint{2.841713in}{2.719407in}}{\pgfqpoint{2.837323in}{2.730006in}}{\pgfqpoint{2.829509in}{2.737819in}}%
\pgfpathcurveto{\pgfqpoint{2.821696in}{2.745633in}}{\pgfqpoint{2.811097in}{2.750023in}}{\pgfqpoint{2.800047in}{2.750023in}}%
\pgfpathcurveto{\pgfqpoint{2.788996in}{2.750023in}}{\pgfqpoint{2.778397in}{2.745633in}}{\pgfqpoint{2.770584in}{2.737819in}}%
\pgfpathcurveto{\pgfqpoint{2.762770in}{2.730006in}}{\pgfqpoint{2.758380in}{2.719407in}}{\pgfqpoint{2.758380in}{2.708356in}}%
\pgfpathcurveto{\pgfqpoint{2.758380in}{2.697306in}}{\pgfqpoint{2.762770in}{2.686707in}}{\pgfqpoint{2.770584in}{2.678894in}}%
\pgfpathcurveto{\pgfqpoint{2.778397in}{2.671080in}}{\pgfqpoint{2.788996in}{2.666690in}}{\pgfqpoint{2.800047in}{2.666690in}}%
\pgfpathclose%
\pgfusepath{stroke,fill}%
\end{pgfscope}%
\begin{pgfscope}%
\pgfpathrectangle{\pgfqpoint{0.787074in}{0.548769in}}{\pgfqpoint{5.062926in}{3.102590in}}%
\pgfusepath{clip}%
\pgfsetbuttcap%
\pgfsetroundjoin%
\definecolor{currentfill}{rgb}{1.000000,0.498039,0.054902}%
\pgfsetfillcolor{currentfill}%
\pgfsetlinewidth{1.003750pt}%
\definecolor{currentstroke}{rgb}{1.000000,0.498039,0.054902}%
\pgfsetstrokecolor{currentstroke}%
\pgfsetdash{}{0pt}%
\pgfpathmoveto{\pgfqpoint{1.939375in}{2.209697in}}%
\pgfpathcurveto{\pgfqpoint{1.950425in}{2.209697in}}{\pgfqpoint{1.961024in}{2.214087in}}{\pgfqpoint{1.968838in}{2.221901in}}%
\pgfpathcurveto{\pgfqpoint{1.976652in}{2.229714in}}{\pgfqpoint{1.981042in}{2.240313in}}{\pgfqpoint{1.981042in}{2.251363in}}%
\pgfpathcurveto{\pgfqpoint{1.981042in}{2.262414in}}{\pgfqpoint{1.976652in}{2.273013in}}{\pgfqpoint{1.968838in}{2.280826in}}%
\pgfpathcurveto{\pgfqpoint{1.961024in}{2.288640in}}{\pgfqpoint{1.950425in}{2.293030in}}{\pgfqpoint{1.939375in}{2.293030in}}%
\pgfpathcurveto{\pgfqpoint{1.928325in}{2.293030in}}{\pgfqpoint{1.917726in}{2.288640in}}{\pgfqpoint{1.909913in}{2.280826in}}%
\pgfpathcurveto{\pgfqpoint{1.902099in}{2.273013in}}{\pgfqpoint{1.897709in}{2.262414in}}{\pgfqpoint{1.897709in}{2.251363in}}%
\pgfpathcurveto{\pgfqpoint{1.897709in}{2.240313in}}{\pgfqpoint{1.902099in}{2.229714in}}{\pgfqpoint{1.909913in}{2.221901in}}%
\pgfpathcurveto{\pgfqpoint{1.917726in}{2.214087in}}{\pgfqpoint{1.928325in}{2.209697in}}{\pgfqpoint{1.939375in}{2.209697in}}%
\pgfpathclose%
\pgfusepath{stroke,fill}%
\end{pgfscope}%
\begin{pgfscope}%
\pgfpathrectangle{\pgfqpoint{0.787074in}{0.548769in}}{\pgfqpoint{5.062926in}{3.102590in}}%
\pgfusepath{clip}%
\pgfsetbuttcap%
\pgfsetroundjoin%
\definecolor{currentfill}{rgb}{1.000000,0.498039,0.054902}%
\pgfsetfillcolor{currentfill}%
\pgfsetlinewidth{1.003750pt}%
\definecolor{currentstroke}{rgb}{1.000000,0.498039,0.054902}%
\pgfsetstrokecolor{currentstroke}%
\pgfsetdash{}{0pt}%
\pgfpathmoveto{\pgfqpoint{1.913438in}{1.987916in}}%
\pgfpathcurveto{\pgfqpoint{1.924489in}{1.987916in}}{\pgfqpoint{1.935088in}{1.992306in}}{\pgfqpoint{1.942901in}{2.000120in}}%
\pgfpathcurveto{\pgfqpoint{1.950715in}{2.007934in}}{\pgfqpoint{1.955105in}{2.018533in}}{\pgfqpoint{1.955105in}{2.029583in}}%
\pgfpathcurveto{\pgfqpoint{1.955105in}{2.040633in}}{\pgfqpoint{1.950715in}{2.051232in}}{\pgfqpoint{1.942901in}{2.059045in}}%
\pgfpathcurveto{\pgfqpoint{1.935088in}{2.066859in}}{\pgfqpoint{1.924489in}{2.071249in}}{\pgfqpoint{1.913438in}{2.071249in}}%
\pgfpathcurveto{\pgfqpoint{1.902388in}{2.071249in}}{\pgfqpoint{1.891789in}{2.066859in}}{\pgfqpoint{1.883976in}{2.059045in}}%
\pgfpathcurveto{\pgfqpoint{1.876162in}{2.051232in}}{\pgfqpoint{1.871772in}{2.040633in}}{\pgfqpoint{1.871772in}{2.029583in}}%
\pgfpathcurveto{\pgfqpoint{1.871772in}{2.018533in}}{\pgfqpoint{1.876162in}{2.007934in}}{\pgfqpoint{1.883976in}{2.000120in}}%
\pgfpathcurveto{\pgfqpoint{1.891789in}{1.992306in}}{\pgfqpoint{1.902388in}{1.987916in}}{\pgfqpoint{1.913438in}{1.987916in}}%
\pgfpathclose%
\pgfusepath{stroke,fill}%
\end{pgfscope}%
\begin{pgfscope}%
\pgfpathrectangle{\pgfqpoint{0.787074in}{0.548769in}}{\pgfqpoint{5.062926in}{3.102590in}}%
\pgfusepath{clip}%
\pgfsetbuttcap%
\pgfsetroundjoin%
\definecolor{currentfill}{rgb}{1.000000,0.498039,0.054902}%
\pgfsetfillcolor{currentfill}%
\pgfsetlinewidth{1.003750pt}%
\definecolor{currentstroke}{rgb}{1.000000,0.498039,0.054902}%
\pgfsetstrokecolor{currentstroke}%
\pgfsetdash{}{0pt}%
\pgfpathmoveto{\pgfqpoint{1.610650in}{2.610441in}}%
\pgfpathcurveto{\pgfqpoint{1.621700in}{2.610441in}}{\pgfqpoint{1.632299in}{2.614831in}}{\pgfqpoint{1.640113in}{2.622645in}}%
\pgfpathcurveto{\pgfqpoint{1.647927in}{2.630459in}}{\pgfqpoint{1.652317in}{2.641058in}}{\pgfqpoint{1.652317in}{2.652108in}}%
\pgfpathcurveto{\pgfqpoint{1.652317in}{2.663158in}}{\pgfqpoint{1.647927in}{2.673757in}}{\pgfqpoint{1.640113in}{2.681571in}}%
\pgfpathcurveto{\pgfqpoint{1.632299in}{2.689384in}}{\pgfqpoint{1.621700in}{2.693774in}}{\pgfqpoint{1.610650in}{2.693774in}}%
\pgfpathcurveto{\pgfqpoint{1.599600in}{2.693774in}}{\pgfqpoint{1.589001in}{2.689384in}}{\pgfqpoint{1.581187in}{2.681571in}}%
\pgfpathcurveto{\pgfqpoint{1.573374in}{2.673757in}}{\pgfqpoint{1.568984in}{2.663158in}}{\pgfqpoint{1.568984in}{2.652108in}}%
\pgfpathcurveto{\pgfqpoint{1.568984in}{2.641058in}}{\pgfqpoint{1.573374in}{2.630459in}}{\pgfqpoint{1.581187in}{2.622645in}}%
\pgfpathcurveto{\pgfqpoint{1.589001in}{2.614831in}}{\pgfqpoint{1.599600in}{2.610441in}}{\pgfqpoint{1.610650in}{2.610441in}}%
\pgfpathclose%
\pgfusepath{stroke,fill}%
\end{pgfscope}%
\begin{pgfscope}%
\pgfpathrectangle{\pgfqpoint{0.787074in}{0.548769in}}{\pgfqpoint{5.062926in}{3.102590in}}%
\pgfusepath{clip}%
\pgfsetbuttcap%
\pgfsetroundjoin%
\definecolor{currentfill}{rgb}{1.000000,0.498039,0.054902}%
\pgfsetfillcolor{currentfill}%
\pgfsetlinewidth{1.003750pt}%
\definecolor{currentstroke}{rgb}{1.000000,0.498039,0.054902}%
\pgfsetstrokecolor{currentstroke}%
\pgfsetdash{}{0pt}%
\pgfpathmoveto{\pgfqpoint{1.489844in}{1.913221in}}%
\pgfpathcurveto{\pgfqpoint{1.500894in}{1.913221in}}{\pgfqpoint{1.511493in}{1.917612in}}{\pgfqpoint{1.519306in}{1.925425in}}%
\pgfpathcurveto{\pgfqpoint{1.527120in}{1.933239in}}{\pgfqpoint{1.531510in}{1.943838in}}{\pgfqpoint{1.531510in}{1.954888in}}%
\pgfpathcurveto{\pgfqpoint{1.531510in}{1.965938in}}{\pgfqpoint{1.527120in}{1.976537in}}{\pgfqpoint{1.519306in}{1.984351in}}%
\pgfpathcurveto{\pgfqpoint{1.511493in}{1.992164in}}{\pgfqpoint{1.500894in}{1.996555in}}{\pgfqpoint{1.489844in}{1.996555in}}%
\pgfpathcurveto{\pgfqpoint{1.478793in}{1.996555in}}{\pgfqpoint{1.468194in}{1.992164in}}{\pgfqpoint{1.460381in}{1.984351in}}%
\pgfpathcurveto{\pgfqpoint{1.452567in}{1.976537in}}{\pgfqpoint{1.448177in}{1.965938in}}{\pgfqpoint{1.448177in}{1.954888in}}%
\pgfpathcurveto{\pgfqpoint{1.448177in}{1.943838in}}{\pgfqpoint{1.452567in}{1.933239in}}{\pgfqpoint{1.460381in}{1.925425in}}%
\pgfpathcurveto{\pgfqpoint{1.468194in}{1.917612in}}{\pgfqpoint{1.478793in}{1.913221in}}{\pgfqpoint{1.489844in}{1.913221in}}%
\pgfpathclose%
\pgfusepath{stroke,fill}%
\end{pgfscope}%
\begin{pgfscope}%
\pgfpathrectangle{\pgfqpoint{0.787074in}{0.548769in}}{\pgfqpoint{5.062926in}{3.102590in}}%
\pgfusepath{clip}%
\pgfsetbuttcap%
\pgfsetroundjoin%
\definecolor{currentfill}{rgb}{1.000000,0.498039,0.054902}%
\pgfsetfillcolor{currentfill}%
\pgfsetlinewidth{1.003750pt}%
\definecolor{currentstroke}{rgb}{1.000000,0.498039,0.054902}%
\pgfsetstrokecolor{currentstroke}%
\pgfsetdash{}{0pt}%
\pgfpathmoveto{\pgfqpoint{2.303483in}{2.130198in}}%
\pgfpathcurveto{\pgfqpoint{2.314533in}{2.130198in}}{\pgfqpoint{2.325132in}{2.134588in}}{\pgfqpoint{2.332945in}{2.142402in}}%
\pgfpathcurveto{\pgfqpoint{2.340759in}{2.150216in}}{\pgfqpoint{2.345149in}{2.160815in}}{\pgfqpoint{2.345149in}{2.171865in}}%
\pgfpathcurveto{\pgfqpoint{2.345149in}{2.182915in}}{\pgfqpoint{2.340759in}{2.193514in}}{\pgfqpoint{2.332945in}{2.201327in}}%
\pgfpathcurveto{\pgfqpoint{2.325132in}{2.209141in}}{\pgfqpoint{2.314533in}{2.213531in}}{\pgfqpoint{2.303483in}{2.213531in}}%
\pgfpathcurveto{\pgfqpoint{2.292433in}{2.213531in}}{\pgfqpoint{2.281833in}{2.209141in}}{\pgfqpoint{2.274020in}{2.201327in}}%
\pgfpathcurveto{\pgfqpoint{2.266206in}{2.193514in}}{\pgfqpoint{2.261816in}{2.182915in}}{\pgfqpoint{2.261816in}{2.171865in}}%
\pgfpathcurveto{\pgfqpoint{2.261816in}{2.160815in}}{\pgfqpoint{2.266206in}{2.150216in}}{\pgfqpoint{2.274020in}{2.142402in}}%
\pgfpathcurveto{\pgfqpoint{2.281833in}{2.134588in}}{\pgfqpoint{2.292433in}{2.130198in}}{\pgfqpoint{2.303483in}{2.130198in}}%
\pgfpathclose%
\pgfusepath{stroke,fill}%
\end{pgfscope}%
\begin{pgfscope}%
\pgfpathrectangle{\pgfqpoint{0.787074in}{0.548769in}}{\pgfqpoint{5.062926in}{3.102590in}}%
\pgfusepath{clip}%
\pgfsetbuttcap%
\pgfsetroundjoin%
\definecolor{currentfill}{rgb}{1.000000,0.498039,0.054902}%
\pgfsetfillcolor{currentfill}%
\pgfsetlinewidth{1.003750pt}%
\definecolor{currentstroke}{rgb}{1.000000,0.498039,0.054902}%
\pgfsetstrokecolor{currentstroke}%
\pgfsetdash{}{0pt}%
\pgfpathmoveto{\pgfqpoint{2.755134in}{1.747432in}}%
\pgfpathcurveto{\pgfqpoint{2.766184in}{1.747432in}}{\pgfqpoint{2.776783in}{1.751822in}}{\pgfqpoint{2.784597in}{1.759636in}}%
\pgfpathcurveto{\pgfqpoint{2.792411in}{1.767449in}}{\pgfqpoint{2.796801in}{1.778048in}}{\pgfqpoint{2.796801in}{1.789098in}}%
\pgfpathcurveto{\pgfqpoint{2.796801in}{1.800148in}}{\pgfqpoint{2.792411in}{1.810748in}}{\pgfqpoint{2.784597in}{1.818561in}}%
\pgfpathcurveto{\pgfqpoint{2.776783in}{1.826375in}}{\pgfqpoint{2.766184in}{1.830765in}}{\pgfqpoint{2.755134in}{1.830765in}}%
\pgfpathcurveto{\pgfqpoint{2.744084in}{1.830765in}}{\pgfqpoint{2.733485in}{1.826375in}}{\pgfqpoint{2.725671in}{1.818561in}}%
\pgfpathcurveto{\pgfqpoint{2.717858in}{1.810748in}}{\pgfqpoint{2.713467in}{1.800148in}}{\pgfqpoint{2.713467in}{1.789098in}}%
\pgfpathcurveto{\pgfqpoint{2.713467in}{1.778048in}}{\pgfqpoint{2.717858in}{1.767449in}}{\pgfqpoint{2.725671in}{1.759636in}}%
\pgfpathcurveto{\pgfqpoint{2.733485in}{1.751822in}}{\pgfqpoint{2.744084in}{1.747432in}}{\pgfqpoint{2.755134in}{1.747432in}}%
\pgfpathclose%
\pgfusepath{stroke,fill}%
\end{pgfscope}%
\begin{pgfscope}%
\pgfpathrectangle{\pgfqpoint{0.787074in}{0.548769in}}{\pgfqpoint{5.062926in}{3.102590in}}%
\pgfusepath{clip}%
\pgfsetbuttcap%
\pgfsetroundjoin%
\definecolor{currentfill}{rgb}{1.000000,0.498039,0.054902}%
\pgfsetfillcolor{currentfill}%
\pgfsetlinewidth{1.003750pt}%
\definecolor{currentstroke}{rgb}{1.000000,0.498039,0.054902}%
\pgfsetstrokecolor{currentstroke}%
\pgfsetdash{}{0pt}%
\pgfpathmoveto{\pgfqpoint{4.643078in}{2.135660in}}%
\pgfpathcurveto{\pgfqpoint{4.654128in}{2.135660in}}{\pgfqpoint{4.664727in}{2.140050in}}{\pgfqpoint{4.672541in}{2.147864in}}%
\pgfpathcurveto{\pgfqpoint{4.680354in}{2.155678in}}{\pgfqpoint{4.684744in}{2.166277in}}{\pgfqpoint{4.684744in}{2.177327in}}%
\pgfpathcurveto{\pgfqpoint{4.684744in}{2.188377in}}{\pgfqpoint{4.680354in}{2.198976in}}{\pgfqpoint{4.672541in}{2.206789in}}%
\pgfpathcurveto{\pgfqpoint{4.664727in}{2.214603in}}{\pgfqpoint{4.654128in}{2.218993in}}{\pgfqpoint{4.643078in}{2.218993in}}%
\pgfpathcurveto{\pgfqpoint{4.632028in}{2.218993in}}{\pgfqpoint{4.621429in}{2.214603in}}{\pgfqpoint{4.613615in}{2.206789in}}%
\pgfpathcurveto{\pgfqpoint{4.605801in}{2.198976in}}{\pgfqpoint{4.601411in}{2.188377in}}{\pgfqpoint{4.601411in}{2.177327in}}%
\pgfpathcurveto{\pgfqpoint{4.601411in}{2.166277in}}{\pgfqpoint{4.605801in}{2.155678in}}{\pgfqpoint{4.613615in}{2.147864in}}%
\pgfpathcurveto{\pgfqpoint{4.621429in}{2.140050in}}{\pgfqpoint{4.632028in}{2.135660in}}{\pgfqpoint{4.643078in}{2.135660in}}%
\pgfpathclose%
\pgfusepath{stroke,fill}%
\end{pgfscope}%
\begin{pgfscope}%
\pgfpathrectangle{\pgfqpoint{0.787074in}{0.548769in}}{\pgfqpoint{5.062926in}{3.102590in}}%
\pgfusepath{clip}%
\pgfsetbuttcap%
\pgfsetroundjoin%
\definecolor{currentfill}{rgb}{1.000000,0.498039,0.054902}%
\pgfsetfillcolor{currentfill}%
\pgfsetlinewidth{1.003750pt}%
\definecolor{currentstroke}{rgb}{1.000000,0.498039,0.054902}%
\pgfsetstrokecolor{currentstroke}%
\pgfsetdash{}{0pt}%
\pgfpathmoveto{\pgfqpoint{1.779964in}{1.976501in}}%
\pgfpathcurveto{\pgfqpoint{1.791015in}{1.976501in}}{\pgfqpoint{1.801614in}{1.980891in}}{\pgfqpoint{1.809427in}{1.988704in}}%
\pgfpathcurveto{\pgfqpoint{1.817241in}{1.996518in}}{\pgfqpoint{1.821631in}{2.007117in}}{\pgfqpoint{1.821631in}{2.018167in}}%
\pgfpathcurveto{\pgfqpoint{1.821631in}{2.029217in}}{\pgfqpoint{1.817241in}{2.039816in}}{\pgfqpoint{1.809427in}{2.047630in}}%
\pgfpathcurveto{\pgfqpoint{1.801614in}{2.055444in}}{\pgfqpoint{1.791015in}{2.059834in}}{\pgfqpoint{1.779964in}{2.059834in}}%
\pgfpathcurveto{\pgfqpoint{1.768914in}{2.059834in}}{\pgfqpoint{1.758315in}{2.055444in}}{\pgfqpoint{1.750502in}{2.047630in}}%
\pgfpathcurveto{\pgfqpoint{1.742688in}{2.039816in}}{\pgfqpoint{1.738298in}{2.029217in}}{\pgfqpoint{1.738298in}{2.018167in}}%
\pgfpathcurveto{\pgfqpoint{1.738298in}{2.007117in}}{\pgfqpoint{1.742688in}{1.996518in}}{\pgfqpoint{1.750502in}{1.988704in}}%
\pgfpathcurveto{\pgfqpoint{1.758315in}{1.980891in}}{\pgfqpoint{1.768914in}{1.976501in}}{\pgfqpoint{1.779964in}{1.976501in}}%
\pgfpathclose%
\pgfusepath{stroke,fill}%
\end{pgfscope}%
\begin{pgfscope}%
\pgfpathrectangle{\pgfqpoint{0.787074in}{0.548769in}}{\pgfqpoint{5.062926in}{3.102590in}}%
\pgfusepath{clip}%
\pgfsetbuttcap%
\pgfsetroundjoin%
\definecolor{currentfill}{rgb}{1.000000,0.498039,0.054902}%
\pgfsetfillcolor{currentfill}%
\pgfsetlinewidth{1.003750pt}%
\definecolor{currentstroke}{rgb}{1.000000,0.498039,0.054902}%
\pgfsetstrokecolor{currentstroke}%
\pgfsetdash{}{0pt}%
\pgfpathmoveto{\pgfqpoint{2.229734in}{2.141633in}}%
\pgfpathcurveto{\pgfqpoint{2.240784in}{2.141633in}}{\pgfqpoint{2.251383in}{2.146024in}}{\pgfqpoint{2.259196in}{2.153837in}}%
\pgfpathcurveto{\pgfqpoint{2.267010in}{2.161651in}}{\pgfqpoint{2.271400in}{2.172250in}}{\pgfqpoint{2.271400in}{2.183300in}}%
\pgfpathcurveto{\pgfqpoint{2.271400in}{2.194350in}}{\pgfqpoint{2.267010in}{2.204949in}}{\pgfqpoint{2.259196in}{2.212763in}}%
\pgfpathcurveto{\pgfqpoint{2.251383in}{2.220577in}}{\pgfqpoint{2.240784in}{2.224967in}}{\pgfqpoint{2.229734in}{2.224967in}}%
\pgfpathcurveto{\pgfqpoint{2.218683in}{2.224967in}}{\pgfqpoint{2.208084in}{2.220577in}}{\pgfqpoint{2.200271in}{2.212763in}}%
\pgfpathcurveto{\pgfqpoint{2.192457in}{2.204949in}}{\pgfqpoint{2.188067in}{2.194350in}}{\pgfqpoint{2.188067in}{2.183300in}}%
\pgfpathcurveto{\pgfqpoint{2.188067in}{2.172250in}}{\pgfqpoint{2.192457in}{2.161651in}}{\pgfqpoint{2.200271in}{2.153837in}}%
\pgfpathcurveto{\pgfqpoint{2.208084in}{2.146024in}}{\pgfqpoint{2.218683in}{2.141633in}}{\pgfqpoint{2.229734in}{2.141633in}}%
\pgfpathclose%
\pgfusepath{stroke,fill}%
\end{pgfscope}%
\begin{pgfscope}%
\pgfpathrectangle{\pgfqpoint{0.787074in}{0.548769in}}{\pgfqpoint{5.062926in}{3.102590in}}%
\pgfusepath{clip}%
\pgfsetbuttcap%
\pgfsetroundjoin%
\definecolor{currentfill}{rgb}{1.000000,0.498039,0.054902}%
\pgfsetfillcolor{currentfill}%
\pgfsetlinewidth{1.003750pt}%
\definecolor{currentstroke}{rgb}{1.000000,0.498039,0.054902}%
\pgfsetstrokecolor{currentstroke}%
\pgfsetdash{}{0pt}%
\pgfpathmoveto{\pgfqpoint{2.066414in}{2.039016in}}%
\pgfpathcurveto{\pgfqpoint{2.077464in}{2.039016in}}{\pgfqpoint{2.088063in}{2.043407in}}{\pgfqpoint{2.095877in}{2.051220in}}%
\pgfpathcurveto{\pgfqpoint{2.103690in}{2.059034in}}{\pgfqpoint{2.108081in}{2.069633in}}{\pgfqpoint{2.108081in}{2.080683in}}%
\pgfpathcurveto{\pgfqpoint{2.108081in}{2.091733in}}{\pgfqpoint{2.103690in}{2.102332in}}{\pgfqpoint{2.095877in}{2.110146in}}%
\pgfpathcurveto{\pgfqpoint{2.088063in}{2.117960in}}{\pgfqpoint{2.077464in}{2.122350in}}{\pgfqpoint{2.066414in}{2.122350in}}%
\pgfpathcurveto{\pgfqpoint{2.055364in}{2.122350in}}{\pgfqpoint{2.044765in}{2.117960in}}{\pgfqpoint{2.036951in}{2.110146in}}%
\pgfpathcurveto{\pgfqpoint{2.029138in}{2.102332in}}{\pgfqpoint{2.024747in}{2.091733in}}{\pgfqpoint{2.024747in}{2.080683in}}%
\pgfpathcurveto{\pgfqpoint{2.024747in}{2.069633in}}{\pgfqpoint{2.029138in}{2.059034in}}{\pgfqpoint{2.036951in}{2.051220in}}%
\pgfpathcurveto{\pgfqpoint{2.044765in}{2.043407in}}{\pgfqpoint{2.055364in}{2.039016in}}{\pgfqpoint{2.066414in}{2.039016in}}%
\pgfpathclose%
\pgfusepath{stroke,fill}%
\end{pgfscope}%
\begin{pgfscope}%
\pgfpathrectangle{\pgfqpoint{0.787074in}{0.548769in}}{\pgfqpoint{5.062926in}{3.102590in}}%
\pgfusepath{clip}%
\pgfsetbuttcap%
\pgfsetroundjoin%
\definecolor{currentfill}{rgb}{1.000000,0.498039,0.054902}%
\pgfsetfillcolor{currentfill}%
\pgfsetlinewidth{1.003750pt}%
\definecolor{currentstroke}{rgb}{1.000000,0.498039,0.054902}%
\pgfsetstrokecolor{currentstroke}%
\pgfsetdash{}{0pt}%
\pgfpathmoveto{\pgfqpoint{2.802717in}{1.981410in}}%
\pgfpathcurveto{\pgfqpoint{2.813768in}{1.981410in}}{\pgfqpoint{2.824367in}{1.985801in}}{\pgfqpoint{2.832180in}{1.993614in}}%
\pgfpathcurveto{\pgfqpoint{2.839994in}{2.001428in}}{\pgfqpoint{2.844384in}{2.012027in}}{\pgfqpoint{2.844384in}{2.023077in}}%
\pgfpathcurveto{\pgfqpoint{2.844384in}{2.034127in}}{\pgfqpoint{2.839994in}{2.044726in}}{\pgfqpoint{2.832180in}{2.052540in}}%
\pgfpathcurveto{\pgfqpoint{2.824367in}{2.060353in}}{\pgfqpoint{2.813768in}{2.064744in}}{\pgfqpoint{2.802717in}{2.064744in}}%
\pgfpathcurveto{\pgfqpoint{2.791667in}{2.064744in}}{\pgfqpoint{2.781068in}{2.060353in}}{\pgfqpoint{2.773255in}{2.052540in}}%
\pgfpathcurveto{\pgfqpoint{2.765441in}{2.044726in}}{\pgfqpoint{2.761051in}{2.034127in}}{\pgfqpoint{2.761051in}{2.023077in}}%
\pgfpathcurveto{\pgfqpoint{2.761051in}{2.012027in}}{\pgfqpoint{2.765441in}{2.001428in}}{\pgfqpoint{2.773255in}{1.993614in}}%
\pgfpathcurveto{\pgfqpoint{2.781068in}{1.985801in}}{\pgfqpoint{2.791667in}{1.981410in}}{\pgfqpoint{2.802717in}{1.981410in}}%
\pgfpathclose%
\pgfusepath{stroke,fill}%
\end{pgfscope}%
\begin{pgfscope}%
\pgfpathrectangle{\pgfqpoint{0.787074in}{0.548769in}}{\pgfqpoint{5.062926in}{3.102590in}}%
\pgfusepath{clip}%
\pgfsetbuttcap%
\pgfsetroundjoin%
\definecolor{currentfill}{rgb}{1.000000,0.498039,0.054902}%
\pgfsetfillcolor{currentfill}%
\pgfsetlinewidth{1.003750pt}%
\definecolor{currentstroke}{rgb}{1.000000,0.498039,0.054902}%
\pgfsetstrokecolor{currentstroke}%
\pgfsetdash{}{0pt}%
\pgfpathmoveto{\pgfqpoint{2.345563in}{2.386079in}}%
\pgfpathcurveto{\pgfqpoint{2.356613in}{2.386079in}}{\pgfqpoint{2.367212in}{2.390469in}}{\pgfqpoint{2.375026in}{2.398283in}}%
\pgfpathcurveto{\pgfqpoint{2.382840in}{2.406097in}}{\pgfqpoint{2.387230in}{2.416696in}}{\pgfqpoint{2.387230in}{2.427746in}}%
\pgfpathcurveto{\pgfqpoint{2.387230in}{2.438796in}}{\pgfqpoint{2.382840in}{2.449395in}}{\pgfqpoint{2.375026in}{2.457209in}}%
\pgfpathcurveto{\pgfqpoint{2.367212in}{2.465022in}}{\pgfqpoint{2.356613in}{2.469412in}}{\pgfqpoint{2.345563in}{2.469412in}}%
\pgfpathcurveto{\pgfqpoint{2.334513in}{2.469412in}}{\pgfqpoint{2.323914in}{2.465022in}}{\pgfqpoint{2.316100in}{2.457209in}}%
\pgfpathcurveto{\pgfqpoint{2.308287in}{2.449395in}}{\pgfqpoint{2.303897in}{2.438796in}}{\pgfqpoint{2.303897in}{2.427746in}}%
\pgfpathcurveto{\pgfqpoint{2.303897in}{2.416696in}}{\pgfqpoint{2.308287in}{2.406097in}}{\pgfqpoint{2.316100in}{2.398283in}}%
\pgfpathcurveto{\pgfqpoint{2.323914in}{2.390469in}}{\pgfqpoint{2.334513in}{2.386079in}}{\pgfqpoint{2.345563in}{2.386079in}}%
\pgfpathclose%
\pgfusepath{stroke,fill}%
\end{pgfscope}%
\begin{pgfscope}%
\pgfpathrectangle{\pgfqpoint{0.787074in}{0.548769in}}{\pgfqpoint{5.062926in}{3.102590in}}%
\pgfusepath{clip}%
\pgfsetbuttcap%
\pgfsetroundjoin%
\definecolor{currentfill}{rgb}{1.000000,0.498039,0.054902}%
\pgfsetfillcolor{currentfill}%
\pgfsetlinewidth{1.003750pt}%
\definecolor{currentstroke}{rgb}{1.000000,0.498039,0.054902}%
\pgfsetstrokecolor{currentstroke}%
\pgfsetdash{}{0pt}%
\pgfpathmoveto{\pgfqpoint{2.203186in}{2.446539in}}%
\pgfpathcurveto{\pgfqpoint{2.214236in}{2.446539in}}{\pgfqpoint{2.224835in}{2.450929in}}{\pgfqpoint{2.232649in}{2.458743in}}%
\pgfpathcurveto{\pgfqpoint{2.240463in}{2.466556in}}{\pgfqpoint{2.244853in}{2.477155in}}{\pgfqpoint{2.244853in}{2.488206in}}%
\pgfpathcurveto{\pgfqpoint{2.244853in}{2.499256in}}{\pgfqpoint{2.240463in}{2.509855in}}{\pgfqpoint{2.232649in}{2.517668in}}%
\pgfpathcurveto{\pgfqpoint{2.224835in}{2.525482in}}{\pgfqpoint{2.214236in}{2.529872in}}{\pgfqpoint{2.203186in}{2.529872in}}%
\pgfpathcurveto{\pgfqpoint{2.192136in}{2.529872in}}{\pgfqpoint{2.181537in}{2.525482in}}{\pgfqpoint{2.173724in}{2.517668in}}%
\pgfpathcurveto{\pgfqpoint{2.165910in}{2.509855in}}{\pgfqpoint{2.161520in}{2.499256in}}{\pgfqpoint{2.161520in}{2.488206in}}%
\pgfpathcurveto{\pgfqpoint{2.161520in}{2.477155in}}{\pgfqpoint{2.165910in}{2.466556in}}{\pgfqpoint{2.173724in}{2.458743in}}%
\pgfpathcurveto{\pgfqpoint{2.181537in}{2.450929in}}{\pgfqpoint{2.192136in}{2.446539in}}{\pgfqpoint{2.203186in}{2.446539in}}%
\pgfpathclose%
\pgfusepath{stroke,fill}%
\end{pgfscope}%
\begin{pgfscope}%
\pgfpathrectangle{\pgfqpoint{0.787074in}{0.548769in}}{\pgfqpoint{5.062926in}{3.102590in}}%
\pgfusepath{clip}%
\pgfsetbuttcap%
\pgfsetroundjoin%
\definecolor{currentfill}{rgb}{1.000000,0.498039,0.054902}%
\pgfsetfillcolor{currentfill}%
\pgfsetlinewidth{1.003750pt}%
\definecolor{currentstroke}{rgb}{1.000000,0.498039,0.054902}%
\pgfsetstrokecolor{currentstroke}%
\pgfsetdash{}{0pt}%
\pgfpathmoveto{\pgfqpoint{2.707161in}{2.604778in}}%
\pgfpathcurveto{\pgfqpoint{2.718211in}{2.604778in}}{\pgfqpoint{2.728810in}{2.609168in}}{\pgfqpoint{2.736624in}{2.616982in}}%
\pgfpathcurveto{\pgfqpoint{2.744437in}{2.624795in}}{\pgfqpoint{2.748827in}{2.635394in}}{\pgfqpoint{2.748827in}{2.646444in}}%
\pgfpathcurveto{\pgfqpoint{2.748827in}{2.657495in}}{\pgfqpoint{2.744437in}{2.668094in}}{\pgfqpoint{2.736624in}{2.675907in}}%
\pgfpathcurveto{\pgfqpoint{2.728810in}{2.683721in}}{\pgfqpoint{2.718211in}{2.688111in}}{\pgfqpoint{2.707161in}{2.688111in}}%
\pgfpathcurveto{\pgfqpoint{2.696111in}{2.688111in}}{\pgfqpoint{2.685512in}{2.683721in}}{\pgfqpoint{2.677698in}{2.675907in}}%
\pgfpathcurveto{\pgfqpoint{2.669884in}{2.668094in}}{\pgfqpoint{2.665494in}{2.657495in}}{\pgfqpoint{2.665494in}{2.646444in}}%
\pgfpathcurveto{\pgfqpoint{2.665494in}{2.635394in}}{\pgfqpoint{2.669884in}{2.624795in}}{\pgfqpoint{2.677698in}{2.616982in}}%
\pgfpathcurveto{\pgfqpoint{2.685512in}{2.609168in}}{\pgfqpoint{2.696111in}{2.604778in}}{\pgfqpoint{2.707161in}{2.604778in}}%
\pgfpathclose%
\pgfusepath{stroke,fill}%
\end{pgfscope}%
\begin{pgfscope}%
\pgfpathrectangle{\pgfqpoint{0.787074in}{0.548769in}}{\pgfqpoint{5.062926in}{3.102590in}}%
\pgfusepath{clip}%
\pgfsetbuttcap%
\pgfsetroundjoin%
\definecolor{currentfill}{rgb}{1.000000,0.498039,0.054902}%
\pgfsetfillcolor{currentfill}%
\pgfsetlinewidth{1.003750pt}%
\definecolor{currentstroke}{rgb}{1.000000,0.498039,0.054902}%
\pgfsetstrokecolor{currentstroke}%
\pgfsetdash{}{0pt}%
\pgfpathmoveto{\pgfqpoint{1.656911in}{2.291595in}}%
\pgfpathcurveto{\pgfqpoint{1.667961in}{2.291595in}}{\pgfqpoint{1.678560in}{2.295985in}}{\pgfqpoint{1.686374in}{2.303799in}}%
\pgfpathcurveto{\pgfqpoint{1.694187in}{2.311612in}}{\pgfqpoint{1.698577in}{2.322211in}}{\pgfqpoint{1.698577in}{2.333261in}}%
\pgfpathcurveto{\pgfqpoint{1.698577in}{2.344312in}}{\pgfqpoint{1.694187in}{2.354911in}}{\pgfqpoint{1.686374in}{2.362724in}}%
\pgfpathcurveto{\pgfqpoint{1.678560in}{2.370538in}}{\pgfqpoint{1.667961in}{2.374928in}}{\pgfqpoint{1.656911in}{2.374928in}}%
\pgfpathcurveto{\pgfqpoint{1.645861in}{2.374928in}}{\pgfqpoint{1.635262in}{2.370538in}}{\pgfqpoint{1.627448in}{2.362724in}}%
\pgfpathcurveto{\pgfqpoint{1.619634in}{2.354911in}}{\pgfqpoint{1.615244in}{2.344312in}}{\pgfqpoint{1.615244in}{2.333261in}}%
\pgfpathcurveto{\pgfqpoint{1.615244in}{2.322211in}}{\pgfqpoint{1.619634in}{2.311612in}}{\pgfqpoint{1.627448in}{2.303799in}}%
\pgfpathcurveto{\pgfqpoint{1.635262in}{2.295985in}}{\pgfqpoint{1.645861in}{2.291595in}}{\pgfqpoint{1.656911in}{2.291595in}}%
\pgfpathclose%
\pgfusepath{stroke,fill}%
\end{pgfscope}%
\begin{pgfscope}%
\pgfpathrectangle{\pgfqpoint{0.787074in}{0.548769in}}{\pgfqpoint{5.062926in}{3.102590in}}%
\pgfusepath{clip}%
\pgfsetbuttcap%
\pgfsetroundjoin%
\definecolor{currentfill}{rgb}{1.000000,0.498039,0.054902}%
\pgfsetfillcolor{currentfill}%
\pgfsetlinewidth{1.003750pt}%
\definecolor{currentstroke}{rgb}{1.000000,0.498039,0.054902}%
\pgfsetstrokecolor{currentstroke}%
\pgfsetdash{}{0pt}%
\pgfpathmoveto{\pgfqpoint{1.988731in}{1.765305in}}%
\pgfpathcurveto{\pgfqpoint{1.999781in}{1.765305in}}{\pgfqpoint{2.010380in}{1.769696in}}{\pgfqpoint{2.018193in}{1.777509in}}%
\pgfpathcurveto{\pgfqpoint{2.026007in}{1.785323in}}{\pgfqpoint{2.030397in}{1.795922in}}{\pgfqpoint{2.030397in}{1.806972in}}%
\pgfpathcurveto{\pgfqpoint{2.030397in}{1.818022in}}{\pgfqpoint{2.026007in}{1.828621in}}{\pgfqpoint{2.018193in}{1.836435in}}%
\pgfpathcurveto{\pgfqpoint{2.010380in}{1.844249in}}{\pgfqpoint{1.999781in}{1.848639in}}{\pgfqpoint{1.988731in}{1.848639in}}%
\pgfpathcurveto{\pgfqpoint{1.977681in}{1.848639in}}{\pgfqpoint{1.967082in}{1.844249in}}{\pgfqpoint{1.959268in}{1.836435in}}%
\pgfpathcurveto{\pgfqpoint{1.951454in}{1.828621in}}{\pgfqpoint{1.947064in}{1.818022in}}{\pgfqpoint{1.947064in}{1.806972in}}%
\pgfpathcurveto{\pgfqpoint{1.947064in}{1.795922in}}{\pgfqpoint{1.951454in}{1.785323in}}{\pgfqpoint{1.959268in}{1.777509in}}%
\pgfpathcurveto{\pgfqpoint{1.967082in}{1.769696in}}{\pgfqpoint{1.977681in}{1.765305in}}{\pgfqpoint{1.988731in}{1.765305in}}%
\pgfpathclose%
\pgfusepath{stroke,fill}%
\end{pgfscope}%
\begin{pgfscope}%
\pgfpathrectangle{\pgfqpoint{0.787074in}{0.548769in}}{\pgfqpoint{5.062926in}{3.102590in}}%
\pgfusepath{clip}%
\pgfsetbuttcap%
\pgfsetroundjoin%
\definecolor{currentfill}{rgb}{1.000000,0.498039,0.054902}%
\pgfsetfillcolor{currentfill}%
\pgfsetlinewidth{1.003750pt}%
\definecolor{currentstroke}{rgb}{1.000000,0.498039,0.054902}%
\pgfsetstrokecolor{currentstroke}%
\pgfsetdash{}{0pt}%
\pgfpathmoveto{\pgfqpoint{2.814469in}{2.168495in}}%
\pgfpathcurveto{\pgfqpoint{2.825519in}{2.168495in}}{\pgfqpoint{2.836118in}{2.172885in}}{\pgfqpoint{2.843932in}{2.180698in}}%
\pgfpathcurveto{\pgfqpoint{2.851746in}{2.188512in}}{\pgfqpoint{2.856136in}{2.199111in}}{\pgfqpoint{2.856136in}{2.210161in}}%
\pgfpathcurveto{\pgfqpoint{2.856136in}{2.221211in}}{\pgfqpoint{2.851746in}{2.231810in}}{\pgfqpoint{2.843932in}{2.239624in}}%
\pgfpathcurveto{\pgfqpoint{2.836118in}{2.247438in}}{\pgfqpoint{2.825519in}{2.251828in}}{\pgfqpoint{2.814469in}{2.251828in}}%
\pgfpathcurveto{\pgfqpoint{2.803419in}{2.251828in}}{\pgfqpoint{2.792820in}{2.247438in}}{\pgfqpoint{2.785006in}{2.239624in}}%
\pgfpathcurveto{\pgfqpoint{2.777193in}{2.231810in}}{\pgfqpoint{2.772802in}{2.221211in}}{\pgfqpoint{2.772802in}{2.210161in}}%
\pgfpathcurveto{\pgfqpoint{2.772802in}{2.199111in}}{\pgfqpoint{2.777193in}{2.188512in}}{\pgfqpoint{2.785006in}{2.180698in}}%
\pgfpathcurveto{\pgfqpoint{2.792820in}{2.172885in}}{\pgfqpoint{2.803419in}{2.168495in}}{\pgfqpoint{2.814469in}{2.168495in}}%
\pgfpathclose%
\pgfusepath{stroke,fill}%
\end{pgfscope}%
\begin{pgfscope}%
\pgfpathrectangle{\pgfqpoint{0.787074in}{0.548769in}}{\pgfqpoint{5.062926in}{3.102590in}}%
\pgfusepath{clip}%
\pgfsetbuttcap%
\pgfsetroundjoin%
\definecolor{currentfill}{rgb}{1.000000,0.498039,0.054902}%
\pgfsetfillcolor{currentfill}%
\pgfsetlinewidth{1.003750pt}%
\definecolor{currentstroke}{rgb}{1.000000,0.498039,0.054902}%
\pgfsetstrokecolor{currentstroke}%
\pgfsetdash{}{0pt}%
\pgfpathmoveto{\pgfqpoint{1.690945in}{2.330781in}}%
\pgfpathcurveto{\pgfqpoint{1.701995in}{2.330781in}}{\pgfqpoint{1.712594in}{2.335171in}}{\pgfqpoint{1.720408in}{2.342985in}}%
\pgfpathcurveto{\pgfqpoint{1.728221in}{2.350799in}}{\pgfqpoint{1.732612in}{2.361398in}}{\pgfqpoint{1.732612in}{2.372448in}}%
\pgfpathcurveto{\pgfqpoint{1.732612in}{2.383498in}}{\pgfqpoint{1.728221in}{2.394097in}}{\pgfqpoint{1.720408in}{2.401910in}}%
\pgfpathcurveto{\pgfqpoint{1.712594in}{2.409724in}}{\pgfqpoint{1.701995in}{2.414114in}}{\pgfqpoint{1.690945in}{2.414114in}}%
\pgfpathcurveto{\pgfqpoint{1.679895in}{2.414114in}}{\pgfqpoint{1.669296in}{2.409724in}}{\pgfqpoint{1.661482in}{2.401910in}}%
\pgfpathcurveto{\pgfqpoint{1.653669in}{2.394097in}}{\pgfqpoint{1.649278in}{2.383498in}}{\pgfqpoint{1.649278in}{2.372448in}}%
\pgfpathcurveto{\pgfqpoint{1.649278in}{2.361398in}}{\pgfqpoint{1.653669in}{2.350799in}}{\pgfqpoint{1.661482in}{2.342985in}}%
\pgfpathcurveto{\pgfqpoint{1.669296in}{2.335171in}}{\pgfqpoint{1.679895in}{2.330781in}}{\pgfqpoint{1.690945in}{2.330781in}}%
\pgfpathclose%
\pgfusepath{stroke,fill}%
\end{pgfscope}%
\begin{pgfscope}%
\pgfpathrectangle{\pgfqpoint{0.787074in}{0.548769in}}{\pgfqpoint{5.062926in}{3.102590in}}%
\pgfusepath{clip}%
\pgfsetbuttcap%
\pgfsetroundjoin%
\definecolor{currentfill}{rgb}{1.000000,0.498039,0.054902}%
\pgfsetfillcolor{currentfill}%
\pgfsetlinewidth{1.003750pt}%
\definecolor{currentstroke}{rgb}{1.000000,0.498039,0.054902}%
\pgfsetstrokecolor{currentstroke}%
\pgfsetdash{}{0pt}%
\pgfpathmoveto{\pgfqpoint{2.448284in}{2.341609in}}%
\pgfpathcurveto{\pgfqpoint{2.459335in}{2.341609in}}{\pgfqpoint{2.469934in}{2.345999in}}{\pgfqpoint{2.477747in}{2.353813in}}%
\pgfpathcurveto{\pgfqpoint{2.485561in}{2.361626in}}{\pgfqpoint{2.489951in}{2.372225in}}{\pgfqpoint{2.489951in}{2.383275in}}%
\pgfpathcurveto{\pgfqpoint{2.489951in}{2.394326in}}{\pgfqpoint{2.485561in}{2.404925in}}{\pgfqpoint{2.477747in}{2.412738in}}%
\pgfpathcurveto{\pgfqpoint{2.469934in}{2.420552in}}{\pgfqpoint{2.459335in}{2.424942in}}{\pgfqpoint{2.448284in}{2.424942in}}%
\pgfpathcurveto{\pgfqpoint{2.437234in}{2.424942in}}{\pgfqpoint{2.426635in}{2.420552in}}{\pgfqpoint{2.418822in}{2.412738in}}%
\pgfpathcurveto{\pgfqpoint{2.411008in}{2.404925in}}{\pgfqpoint{2.406618in}{2.394326in}}{\pgfqpoint{2.406618in}{2.383275in}}%
\pgfpathcurveto{\pgfqpoint{2.406618in}{2.372225in}}{\pgfqpoint{2.411008in}{2.361626in}}{\pgfqpoint{2.418822in}{2.353813in}}%
\pgfpathcurveto{\pgfqpoint{2.426635in}{2.345999in}}{\pgfqpoint{2.437234in}{2.341609in}}{\pgfqpoint{2.448284in}{2.341609in}}%
\pgfpathclose%
\pgfusepath{stroke,fill}%
\end{pgfscope}%
\begin{pgfscope}%
\pgfpathrectangle{\pgfqpoint{0.787074in}{0.548769in}}{\pgfqpoint{5.062926in}{3.102590in}}%
\pgfusepath{clip}%
\pgfsetbuttcap%
\pgfsetroundjoin%
\definecolor{currentfill}{rgb}{1.000000,0.498039,0.054902}%
\pgfsetfillcolor{currentfill}%
\pgfsetlinewidth{1.003750pt}%
\definecolor{currentstroke}{rgb}{1.000000,0.498039,0.054902}%
\pgfsetstrokecolor{currentstroke}%
\pgfsetdash{}{0pt}%
\pgfpathmoveto{\pgfqpoint{2.477808in}{2.708305in}}%
\pgfpathcurveto{\pgfqpoint{2.488858in}{2.708305in}}{\pgfqpoint{2.499457in}{2.712695in}}{\pgfqpoint{2.507271in}{2.720509in}}%
\pgfpathcurveto{\pgfqpoint{2.515084in}{2.728322in}}{\pgfqpoint{2.519475in}{2.738921in}}{\pgfqpoint{2.519475in}{2.749972in}}%
\pgfpathcurveto{\pgfqpoint{2.519475in}{2.761022in}}{\pgfqpoint{2.515084in}{2.771621in}}{\pgfqpoint{2.507271in}{2.779434in}}%
\pgfpathcurveto{\pgfqpoint{2.499457in}{2.787248in}}{\pgfqpoint{2.488858in}{2.791638in}}{\pgfqpoint{2.477808in}{2.791638in}}%
\pgfpathcurveto{\pgfqpoint{2.466758in}{2.791638in}}{\pgfqpoint{2.456159in}{2.787248in}}{\pgfqpoint{2.448345in}{2.779434in}}%
\pgfpathcurveto{\pgfqpoint{2.440531in}{2.771621in}}{\pgfqpoint{2.436141in}{2.761022in}}{\pgfqpoint{2.436141in}{2.749972in}}%
\pgfpathcurveto{\pgfqpoint{2.436141in}{2.738921in}}{\pgfqpoint{2.440531in}{2.728322in}}{\pgfqpoint{2.448345in}{2.720509in}}%
\pgfpathcurveto{\pgfqpoint{2.456159in}{2.712695in}}{\pgfqpoint{2.466758in}{2.708305in}}{\pgfqpoint{2.477808in}{2.708305in}}%
\pgfpathclose%
\pgfusepath{stroke,fill}%
\end{pgfscope}%
\begin{pgfscope}%
\pgfpathrectangle{\pgfqpoint{0.787074in}{0.548769in}}{\pgfqpoint{5.062926in}{3.102590in}}%
\pgfusepath{clip}%
\pgfsetbuttcap%
\pgfsetroundjoin%
\definecolor{currentfill}{rgb}{1.000000,0.498039,0.054902}%
\pgfsetfillcolor{currentfill}%
\pgfsetlinewidth{1.003750pt}%
\definecolor{currentstroke}{rgb}{1.000000,0.498039,0.054902}%
\pgfsetstrokecolor{currentstroke}%
\pgfsetdash{}{0pt}%
\pgfpathmoveto{\pgfqpoint{4.683615in}{2.886489in}}%
\pgfpathcurveto{\pgfqpoint{4.694665in}{2.886489in}}{\pgfqpoint{4.705264in}{2.890880in}}{\pgfqpoint{4.713078in}{2.898693in}}%
\pgfpathcurveto{\pgfqpoint{4.720892in}{2.906507in}}{\pgfqpoint{4.725282in}{2.917106in}}{\pgfqpoint{4.725282in}{2.928156in}}%
\pgfpathcurveto{\pgfqpoint{4.725282in}{2.939206in}}{\pgfqpoint{4.720892in}{2.949805in}}{\pgfqpoint{4.713078in}{2.957619in}}%
\pgfpathcurveto{\pgfqpoint{4.705264in}{2.965433in}}{\pgfqpoint{4.694665in}{2.969823in}}{\pgfqpoint{4.683615in}{2.969823in}}%
\pgfpathcurveto{\pgfqpoint{4.672565in}{2.969823in}}{\pgfqpoint{4.661966in}{2.965433in}}{\pgfqpoint{4.654152in}{2.957619in}}%
\pgfpathcurveto{\pgfqpoint{4.646339in}{2.949805in}}{\pgfqpoint{4.641949in}{2.939206in}}{\pgfqpoint{4.641949in}{2.928156in}}%
\pgfpathcurveto{\pgfqpoint{4.641949in}{2.917106in}}{\pgfqpoint{4.646339in}{2.906507in}}{\pgfqpoint{4.654152in}{2.898693in}}%
\pgfpathcurveto{\pgfqpoint{4.661966in}{2.890880in}}{\pgfqpoint{4.672565in}{2.886489in}}{\pgfqpoint{4.683615in}{2.886489in}}%
\pgfpathclose%
\pgfusepath{stroke,fill}%
\end{pgfscope}%
\begin{pgfscope}%
\pgfpathrectangle{\pgfqpoint{0.787074in}{0.548769in}}{\pgfqpoint{5.062926in}{3.102590in}}%
\pgfusepath{clip}%
\pgfsetbuttcap%
\pgfsetroundjoin%
\definecolor{currentfill}{rgb}{1.000000,0.498039,0.054902}%
\pgfsetfillcolor{currentfill}%
\pgfsetlinewidth{1.003750pt}%
\definecolor{currentstroke}{rgb}{1.000000,0.498039,0.054902}%
\pgfsetstrokecolor{currentstroke}%
\pgfsetdash{}{0pt}%
\pgfpathmoveto{\pgfqpoint{2.148065in}{2.003186in}}%
\pgfpathcurveto{\pgfqpoint{2.159115in}{2.003186in}}{\pgfqpoint{2.169714in}{2.007576in}}{\pgfqpoint{2.177528in}{2.015390in}}%
\pgfpathcurveto{\pgfqpoint{2.185342in}{2.023204in}}{\pgfqpoint{2.189732in}{2.033803in}}{\pgfqpoint{2.189732in}{2.044853in}}%
\pgfpathcurveto{\pgfqpoint{2.189732in}{2.055903in}}{\pgfqpoint{2.185342in}{2.066502in}}{\pgfqpoint{2.177528in}{2.074316in}}%
\pgfpathcurveto{\pgfqpoint{2.169714in}{2.082129in}}{\pgfqpoint{2.159115in}{2.086520in}}{\pgfqpoint{2.148065in}{2.086520in}}%
\pgfpathcurveto{\pgfqpoint{2.137015in}{2.086520in}}{\pgfqpoint{2.126416in}{2.082129in}}{\pgfqpoint{2.118603in}{2.074316in}}%
\pgfpathcurveto{\pgfqpoint{2.110789in}{2.066502in}}{\pgfqpoint{2.106399in}{2.055903in}}{\pgfqpoint{2.106399in}{2.044853in}}%
\pgfpathcurveto{\pgfqpoint{2.106399in}{2.033803in}}{\pgfqpoint{2.110789in}{2.023204in}}{\pgfqpoint{2.118603in}{2.015390in}}%
\pgfpathcurveto{\pgfqpoint{2.126416in}{2.007576in}}{\pgfqpoint{2.137015in}{2.003186in}}{\pgfqpoint{2.148065in}{2.003186in}}%
\pgfpathclose%
\pgfusepath{stroke,fill}%
\end{pgfscope}%
\begin{pgfscope}%
\pgfpathrectangle{\pgfqpoint{0.787074in}{0.548769in}}{\pgfqpoint{5.062926in}{3.102590in}}%
\pgfusepath{clip}%
\pgfsetbuttcap%
\pgfsetroundjoin%
\definecolor{currentfill}{rgb}{0.121569,0.466667,0.705882}%
\pgfsetfillcolor{currentfill}%
\pgfsetlinewidth{1.003750pt}%
\definecolor{currentstroke}{rgb}{0.121569,0.466667,0.705882}%
\pgfsetstrokecolor{currentstroke}%
\pgfsetdash{}{0pt}%
\pgfpathmoveto{\pgfqpoint{1.170878in}{0.942348in}}%
\pgfpathcurveto{\pgfqpoint{1.181928in}{0.942348in}}{\pgfqpoint{1.192527in}{0.946738in}}{\pgfqpoint{1.200340in}{0.954552in}}%
\pgfpathcurveto{\pgfqpoint{1.208154in}{0.962365in}}{\pgfqpoint{1.212544in}{0.972964in}}{\pgfqpoint{1.212544in}{0.984014in}}%
\pgfpathcurveto{\pgfqpoint{1.212544in}{0.995065in}}{\pgfqpoint{1.208154in}{1.005664in}}{\pgfqpoint{1.200340in}{1.013477in}}%
\pgfpathcurveto{\pgfqpoint{1.192527in}{1.021291in}}{\pgfqpoint{1.181928in}{1.025681in}}{\pgfqpoint{1.170878in}{1.025681in}}%
\pgfpathcurveto{\pgfqpoint{1.159827in}{1.025681in}}{\pgfqpoint{1.149228in}{1.021291in}}{\pgfqpoint{1.141415in}{1.013477in}}%
\pgfpathcurveto{\pgfqpoint{1.133601in}{1.005664in}}{\pgfqpoint{1.129211in}{0.995065in}}{\pgfqpoint{1.129211in}{0.984014in}}%
\pgfpathcurveto{\pgfqpoint{1.129211in}{0.972964in}}{\pgfqpoint{1.133601in}{0.962365in}}{\pgfqpoint{1.141415in}{0.954552in}}%
\pgfpathcurveto{\pgfqpoint{1.149228in}{0.946738in}}{\pgfqpoint{1.159827in}{0.942348in}}{\pgfqpoint{1.170878in}{0.942348in}}%
\pgfpathclose%
\pgfusepath{stroke,fill}%
\end{pgfscope}%
\begin{pgfscope}%
\pgfpathrectangle{\pgfqpoint{0.787074in}{0.548769in}}{\pgfqpoint{5.062926in}{3.102590in}}%
\pgfusepath{clip}%
\pgfsetbuttcap%
\pgfsetroundjoin%
\definecolor{currentfill}{rgb}{1.000000,0.498039,0.054902}%
\pgfsetfillcolor{currentfill}%
\pgfsetlinewidth{1.003750pt}%
\definecolor{currentstroke}{rgb}{1.000000,0.498039,0.054902}%
\pgfsetstrokecolor{currentstroke}%
\pgfsetdash{}{0pt}%
\pgfpathmoveto{\pgfqpoint{2.036958in}{2.620166in}}%
\pgfpathcurveto{\pgfqpoint{2.048009in}{2.620166in}}{\pgfqpoint{2.058608in}{2.624556in}}{\pgfqpoint{2.066421in}{2.632370in}}%
\pgfpathcurveto{\pgfqpoint{2.074235in}{2.640183in}}{\pgfqpoint{2.078625in}{2.650782in}}{\pgfqpoint{2.078625in}{2.661832in}}%
\pgfpathcurveto{\pgfqpoint{2.078625in}{2.672883in}}{\pgfqpoint{2.074235in}{2.683482in}}{\pgfqpoint{2.066421in}{2.691295in}}%
\pgfpathcurveto{\pgfqpoint{2.058608in}{2.699109in}}{\pgfqpoint{2.048009in}{2.703499in}}{\pgfqpoint{2.036958in}{2.703499in}}%
\pgfpathcurveto{\pgfqpoint{2.025908in}{2.703499in}}{\pgfqpoint{2.015309in}{2.699109in}}{\pgfqpoint{2.007496in}{2.691295in}}%
\pgfpathcurveto{\pgfqpoint{1.999682in}{2.683482in}}{\pgfqpoint{1.995292in}{2.672883in}}{\pgfqpoint{1.995292in}{2.661832in}}%
\pgfpathcurveto{\pgfqpoint{1.995292in}{2.650782in}}{\pgfqpoint{1.999682in}{2.640183in}}{\pgfqpoint{2.007496in}{2.632370in}}%
\pgfpathcurveto{\pgfqpoint{2.015309in}{2.624556in}}{\pgfqpoint{2.025908in}{2.620166in}}{\pgfqpoint{2.036958in}{2.620166in}}%
\pgfpathclose%
\pgfusepath{stroke,fill}%
\end{pgfscope}%
\begin{pgfscope}%
\pgfpathrectangle{\pgfqpoint{0.787074in}{0.548769in}}{\pgfqpoint{5.062926in}{3.102590in}}%
\pgfusepath{clip}%
\pgfsetbuttcap%
\pgfsetroundjoin%
\definecolor{currentfill}{rgb}{1.000000,0.498039,0.054902}%
\pgfsetfillcolor{currentfill}%
\pgfsetlinewidth{1.003750pt}%
\definecolor{currentstroke}{rgb}{1.000000,0.498039,0.054902}%
\pgfsetstrokecolor{currentstroke}%
\pgfsetdash{}{0pt}%
\pgfpathmoveto{\pgfqpoint{2.290052in}{2.991999in}}%
\pgfpathcurveto{\pgfqpoint{2.301102in}{2.991999in}}{\pgfqpoint{2.311701in}{2.996389in}}{\pgfqpoint{2.319515in}{3.004203in}}%
\pgfpathcurveto{\pgfqpoint{2.327329in}{3.012016in}}{\pgfqpoint{2.331719in}{3.022615in}}{\pgfqpoint{2.331719in}{3.033665in}}%
\pgfpathcurveto{\pgfqpoint{2.331719in}{3.044716in}}{\pgfqpoint{2.327329in}{3.055315in}}{\pgfqpoint{2.319515in}{3.063128in}}%
\pgfpathcurveto{\pgfqpoint{2.311701in}{3.070942in}}{\pgfqpoint{2.301102in}{3.075332in}}{\pgfqpoint{2.290052in}{3.075332in}}%
\pgfpathcurveto{\pgfqpoint{2.279002in}{3.075332in}}{\pgfqpoint{2.268403in}{3.070942in}}{\pgfqpoint{2.260589in}{3.063128in}}%
\pgfpathcurveto{\pgfqpoint{2.252776in}{3.055315in}}{\pgfqpoint{2.248385in}{3.044716in}}{\pgfqpoint{2.248385in}{3.033665in}}%
\pgfpathcurveto{\pgfqpoint{2.248385in}{3.022615in}}{\pgfqpoint{2.252776in}{3.012016in}}{\pgfqpoint{2.260589in}{3.004203in}}%
\pgfpathcurveto{\pgfqpoint{2.268403in}{2.996389in}}{\pgfqpoint{2.279002in}{2.991999in}}{\pgfqpoint{2.290052in}{2.991999in}}%
\pgfpathclose%
\pgfusepath{stroke,fill}%
\end{pgfscope}%
\begin{pgfscope}%
\pgfpathrectangle{\pgfqpoint{0.787074in}{0.548769in}}{\pgfqpoint{5.062926in}{3.102590in}}%
\pgfusepath{clip}%
\pgfsetbuttcap%
\pgfsetroundjoin%
\definecolor{currentfill}{rgb}{1.000000,0.498039,0.054902}%
\pgfsetfillcolor{currentfill}%
\pgfsetlinewidth{1.003750pt}%
\definecolor{currentstroke}{rgb}{1.000000,0.498039,0.054902}%
\pgfsetstrokecolor{currentstroke}%
\pgfsetdash{}{0pt}%
\pgfpathmoveto{\pgfqpoint{1.392769in}{3.136678in}}%
\pgfpathcurveto{\pgfqpoint{1.403819in}{3.136678in}}{\pgfqpoint{1.414418in}{3.141068in}}{\pgfqpoint{1.422232in}{3.148882in}}%
\pgfpathcurveto{\pgfqpoint{1.430046in}{3.156695in}}{\pgfqpoint{1.434436in}{3.167294in}}{\pgfqpoint{1.434436in}{3.178345in}}%
\pgfpathcurveto{\pgfqpoint{1.434436in}{3.189395in}}{\pgfqpoint{1.430046in}{3.199994in}}{\pgfqpoint{1.422232in}{3.207807in}}%
\pgfpathcurveto{\pgfqpoint{1.414418in}{3.215621in}}{\pgfqpoint{1.403819in}{3.220011in}}{\pgfqpoint{1.392769in}{3.220011in}}%
\pgfpathcurveto{\pgfqpoint{1.381719in}{3.220011in}}{\pgfqpoint{1.371120in}{3.215621in}}{\pgfqpoint{1.363306in}{3.207807in}}%
\pgfpathcurveto{\pgfqpoint{1.355493in}{3.199994in}}{\pgfqpoint{1.351102in}{3.189395in}}{\pgfqpoint{1.351102in}{3.178345in}}%
\pgfpathcurveto{\pgfqpoint{1.351102in}{3.167294in}}{\pgfqpoint{1.355493in}{3.156695in}}{\pgfqpoint{1.363306in}{3.148882in}}%
\pgfpathcurveto{\pgfqpoint{1.371120in}{3.141068in}}{\pgfqpoint{1.381719in}{3.136678in}}{\pgfqpoint{1.392769in}{3.136678in}}%
\pgfpathclose%
\pgfusepath{stroke,fill}%
\end{pgfscope}%
\begin{pgfscope}%
\pgfpathrectangle{\pgfqpoint{0.787074in}{0.548769in}}{\pgfqpoint{5.062926in}{3.102590in}}%
\pgfusepath{clip}%
\pgfsetbuttcap%
\pgfsetroundjoin%
\definecolor{currentfill}{rgb}{1.000000,0.498039,0.054902}%
\pgfsetfillcolor{currentfill}%
\pgfsetlinewidth{1.003750pt}%
\definecolor{currentstroke}{rgb}{1.000000,0.498039,0.054902}%
\pgfsetstrokecolor{currentstroke}%
\pgfsetdash{}{0pt}%
\pgfpathmoveto{\pgfqpoint{1.354852in}{2.684169in}}%
\pgfpathcurveto{\pgfqpoint{1.365902in}{2.684169in}}{\pgfqpoint{1.376501in}{2.688559in}}{\pgfqpoint{1.384314in}{2.696373in}}%
\pgfpathcurveto{\pgfqpoint{1.392128in}{2.704186in}}{\pgfqpoint{1.396518in}{2.714785in}}{\pgfqpoint{1.396518in}{2.725835in}}%
\pgfpathcurveto{\pgfqpoint{1.396518in}{2.736886in}}{\pgfqpoint{1.392128in}{2.747485in}}{\pgfqpoint{1.384314in}{2.755298in}}%
\pgfpathcurveto{\pgfqpoint{1.376501in}{2.763112in}}{\pgfqpoint{1.365902in}{2.767502in}}{\pgfqpoint{1.354852in}{2.767502in}}%
\pgfpathcurveto{\pgfqpoint{1.343802in}{2.767502in}}{\pgfqpoint{1.333203in}{2.763112in}}{\pgfqpoint{1.325389in}{2.755298in}}%
\pgfpathcurveto{\pgfqpoint{1.317575in}{2.747485in}}{\pgfqpoint{1.313185in}{2.736886in}}{\pgfqpoint{1.313185in}{2.725835in}}%
\pgfpathcurveto{\pgfqpoint{1.313185in}{2.714785in}}{\pgfqpoint{1.317575in}{2.704186in}}{\pgfqpoint{1.325389in}{2.696373in}}%
\pgfpathcurveto{\pgfqpoint{1.333203in}{2.688559in}}{\pgfqpoint{1.343802in}{2.684169in}}{\pgfqpoint{1.354852in}{2.684169in}}%
\pgfpathclose%
\pgfusepath{stroke,fill}%
\end{pgfscope}%
\begin{pgfscope}%
\pgfpathrectangle{\pgfqpoint{0.787074in}{0.548769in}}{\pgfqpoint{5.062926in}{3.102590in}}%
\pgfusepath{clip}%
\pgfsetbuttcap%
\pgfsetroundjoin%
\definecolor{currentfill}{rgb}{1.000000,0.498039,0.054902}%
\pgfsetfillcolor{currentfill}%
\pgfsetlinewidth{1.003750pt}%
\definecolor{currentstroke}{rgb}{1.000000,0.498039,0.054902}%
\pgfsetstrokecolor{currentstroke}%
\pgfsetdash{}{0pt}%
\pgfpathmoveto{\pgfqpoint{2.498700in}{2.925050in}}%
\pgfpathcurveto{\pgfqpoint{2.509750in}{2.925050in}}{\pgfqpoint{2.520349in}{2.929440in}}{\pgfqpoint{2.528163in}{2.937253in}}%
\pgfpathcurveto{\pgfqpoint{2.535976in}{2.945067in}}{\pgfqpoint{2.540366in}{2.955666in}}{\pgfqpoint{2.540366in}{2.966716in}}%
\pgfpathcurveto{\pgfqpoint{2.540366in}{2.977766in}}{\pgfqpoint{2.535976in}{2.988365in}}{\pgfqpoint{2.528163in}{2.996179in}}%
\pgfpathcurveto{\pgfqpoint{2.520349in}{3.003993in}}{\pgfqpoint{2.509750in}{3.008383in}}{\pgfqpoint{2.498700in}{3.008383in}}%
\pgfpathcurveto{\pgfqpoint{2.487650in}{3.008383in}}{\pgfqpoint{2.477051in}{3.003993in}}{\pgfqpoint{2.469237in}{2.996179in}}%
\pgfpathcurveto{\pgfqpoint{2.461423in}{2.988365in}}{\pgfqpoint{2.457033in}{2.977766in}}{\pgfqpoint{2.457033in}{2.966716in}}%
\pgfpathcurveto{\pgfqpoint{2.457033in}{2.955666in}}{\pgfqpoint{2.461423in}{2.945067in}}{\pgfqpoint{2.469237in}{2.937253in}}%
\pgfpathcurveto{\pgfqpoint{2.477051in}{2.929440in}}{\pgfqpoint{2.487650in}{2.925050in}}{\pgfqpoint{2.498700in}{2.925050in}}%
\pgfpathclose%
\pgfusepath{stroke,fill}%
\end{pgfscope}%
\begin{pgfscope}%
\pgfpathrectangle{\pgfqpoint{0.787074in}{0.548769in}}{\pgfqpoint{5.062926in}{3.102590in}}%
\pgfusepath{clip}%
\pgfsetbuttcap%
\pgfsetroundjoin%
\definecolor{currentfill}{rgb}{1.000000,0.498039,0.054902}%
\pgfsetfillcolor{currentfill}%
\pgfsetlinewidth{1.003750pt}%
\definecolor{currentstroke}{rgb}{1.000000,0.498039,0.054902}%
\pgfsetstrokecolor{currentstroke}%
\pgfsetdash{}{0pt}%
\pgfpathmoveto{\pgfqpoint{2.237958in}{2.772349in}}%
\pgfpathcurveto{\pgfqpoint{2.249008in}{2.772349in}}{\pgfqpoint{2.259607in}{2.776739in}}{\pgfqpoint{2.267421in}{2.784553in}}%
\pgfpathcurveto{\pgfqpoint{2.275234in}{2.792367in}}{\pgfqpoint{2.279625in}{2.802966in}}{\pgfqpoint{2.279625in}{2.814016in}}%
\pgfpathcurveto{\pgfqpoint{2.279625in}{2.825066in}}{\pgfqpoint{2.275234in}{2.835665in}}{\pgfqpoint{2.267421in}{2.843479in}}%
\pgfpathcurveto{\pgfqpoint{2.259607in}{2.851292in}}{\pgfqpoint{2.249008in}{2.855683in}}{\pgfqpoint{2.237958in}{2.855683in}}%
\pgfpathcurveto{\pgfqpoint{2.226908in}{2.855683in}}{\pgfqpoint{2.216309in}{2.851292in}}{\pgfqpoint{2.208495in}{2.843479in}}%
\pgfpathcurveto{\pgfqpoint{2.200682in}{2.835665in}}{\pgfqpoint{2.196291in}{2.825066in}}{\pgfqpoint{2.196291in}{2.814016in}}%
\pgfpathcurveto{\pgfqpoint{2.196291in}{2.802966in}}{\pgfqpoint{2.200682in}{2.792367in}}{\pgfqpoint{2.208495in}{2.784553in}}%
\pgfpathcurveto{\pgfqpoint{2.216309in}{2.776739in}}{\pgfqpoint{2.226908in}{2.772349in}}{\pgfqpoint{2.237958in}{2.772349in}}%
\pgfpathclose%
\pgfusepath{stroke,fill}%
\end{pgfscope}%
\begin{pgfscope}%
\pgfpathrectangle{\pgfqpoint{0.787074in}{0.548769in}}{\pgfqpoint{5.062926in}{3.102590in}}%
\pgfusepath{clip}%
\pgfsetbuttcap%
\pgfsetroundjoin%
\definecolor{currentfill}{rgb}{1.000000,0.498039,0.054902}%
\pgfsetfillcolor{currentfill}%
\pgfsetlinewidth{1.003750pt}%
\definecolor{currentstroke}{rgb}{1.000000,0.498039,0.054902}%
\pgfsetstrokecolor{currentstroke}%
\pgfsetdash{}{0pt}%
\pgfpathmoveto{\pgfqpoint{3.201156in}{3.273091in}}%
\pgfpathcurveto{\pgfqpoint{3.212206in}{3.273091in}}{\pgfqpoint{3.222805in}{3.277482in}}{\pgfqpoint{3.230618in}{3.285295in}}%
\pgfpathcurveto{\pgfqpoint{3.238432in}{3.293109in}}{\pgfqpoint{3.242822in}{3.303708in}}{\pgfqpoint{3.242822in}{3.314758in}}%
\pgfpathcurveto{\pgfqpoint{3.242822in}{3.325808in}}{\pgfqpoint{3.238432in}{3.336407in}}{\pgfqpoint{3.230618in}{3.344221in}}%
\pgfpathcurveto{\pgfqpoint{3.222805in}{3.352035in}}{\pgfqpoint{3.212206in}{3.356425in}}{\pgfqpoint{3.201156in}{3.356425in}}%
\pgfpathcurveto{\pgfqpoint{3.190106in}{3.356425in}}{\pgfqpoint{3.179507in}{3.352035in}}{\pgfqpoint{3.171693in}{3.344221in}}%
\pgfpathcurveto{\pgfqpoint{3.163879in}{3.336407in}}{\pgfqpoint{3.159489in}{3.325808in}}{\pgfqpoint{3.159489in}{3.314758in}}%
\pgfpathcurveto{\pgfqpoint{3.159489in}{3.303708in}}{\pgfqpoint{3.163879in}{3.293109in}}{\pgfqpoint{3.171693in}{3.285295in}}%
\pgfpathcurveto{\pgfqpoint{3.179507in}{3.277482in}}{\pgfqpoint{3.190106in}{3.273091in}}{\pgfqpoint{3.201156in}{3.273091in}}%
\pgfpathclose%
\pgfusepath{stroke,fill}%
\end{pgfscope}%
\begin{pgfscope}%
\pgfpathrectangle{\pgfqpoint{0.787074in}{0.548769in}}{\pgfqpoint{5.062926in}{3.102590in}}%
\pgfusepath{clip}%
\pgfsetbuttcap%
\pgfsetroundjoin%
\definecolor{currentfill}{rgb}{1.000000,0.498039,0.054902}%
\pgfsetfillcolor{currentfill}%
\pgfsetlinewidth{1.003750pt}%
\definecolor{currentstroke}{rgb}{1.000000,0.498039,0.054902}%
\pgfsetstrokecolor{currentstroke}%
\pgfsetdash{}{0pt}%
\pgfpathmoveto{\pgfqpoint{2.400777in}{2.433412in}}%
\pgfpathcurveto{\pgfqpoint{2.411828in}{2.433412in}}{\pgfqpoint{2.422427in}{2.437803in}}{\pgfqpoint{2.430240in}{2.445616in}}%
\pgfpathcurveto{\pgfqpoint{2.438054in}{2.453430in}}{\pgfqpoint{2.442444in}{2.464029in}}{\pgfqpoint{2.442444in}{2.475079in}}%
\pgfpathcurveto{\pgfqpoint{2.442444in}{2.486129in}}{\pgfqpoint{2.438054in}{2.496728in}}{\pgfqpoint{2.430240in}{2.504542in}}%
\pgfpathcurveto{\pgfqpoint{2.422427in}{2.512356in}}{\pgfqpoint{2.411828in}{2.516746in}}{\pgfqpoint{2.400777in}{2.516746in}}%
\pgfpathcurveto{\pgfqpoint{2.389727in}{2.516746in}}{\pgfqpoint{2.379128in}{2.512356in}}{\pgfqpoint{2.371315in}{2.504542in}}%
\pgfpathcurveto{\pgfqpoint{2.363501in}{2.496728in}}{\pgfqpoint{2.359111in}{2.486129in}}{\pgfqpoint{2.359111in}{2.475079in}}%
\pgfpathcurveto{\pgfqpoint{2.359111in}{2.464029in}}{\pgfqpoint{2.363501in}{2.453430in}}{\pgfqpoint{2.371315in}{2.445616in}}%
\pgfpathcurveto{\pgfqpoint{2.379128in}{2.437803in}}{\pgfqpoint{2.389727in}{2.433412in}}{\pgfqpoint{2.400777in}{2.433412in}}%
\pgfpathclose%
\pgfusepath{stroke,fill}%
\end{pgfscope}%
\begin{pgfscope}%
\pgfpathrectangle{\pgfqpoint{0.787074in}{0.548769in}}{\pgfqpoint{5.062926in}{3.102590in}}%
\pgfusepath{clip}%
\pgfsetbuttcap%
\pgfsetroundjoin%
\definecolor{currentfill}{rgb}{1.000000,0.498039,0.054902}%
\pgfsetfillcolor{currentfill}%
\pgfsetlinewidth{1.003750pt}%
\definecolor{currentstroke}{rgb}{1.000000,0.498039,0.054902}%
\pgfsetstrokecolor{currentstroke}%
\pgfsetdash{}{0pt}%
\pgfpathmoveto{\pgfqpoint{2.514165in}{2.784629in}}%
\pgfpathcurveto{\pgfqpoint{2.525215in}{2.784629in}}{\pgfqpoint{2.535814in}{2.789019in}}{\pgfqpoint{2.543628in}{2.796833in}}%
\pgfpathcurveto{\pgfqpoint{2.551442in}{2.804646in}}{\pgfqpoint{2.555832in}{2.815245in}}{\pgfqpoint{2.555832in}{2.826296in}}%
\pgfpathcurveto{\pgfqpoint{2.555832in}{2.837346in}}{\pgfqpoint{2.551442in}{2.847945in}}{\pgfqpoint{2.543628in}{2.855758in}}%
\pgfpathcurveto{\pgfqpoint{2.535814in}{2.863572in}}{\pgfqpoint{2.525215in}{2.867962in}}{\pgfqpoint{2.514165in}{2.867962in}}%
\pgfpathcurveto{\pgfqpoint{2.503115in}{2.867962in}}{\pgfqpoint{2.492516in}{2.863572in}}{\pgfqpoint{2.484702in}{2.855758in}}%
\pgfpathcurveto{\pgfqpoint{2.476889in}{2.847945in}}{\pgfqpoint{2.472498in}{2.837346in}}{\pgfqpoint{2.472498in}{2.826296in}}%
\pgfpathcurveto{\pgfqpoint{2.472498in}{2.815245in}}{\pgfqpoint{2.476889in}{2.804646in}}{\pgfqpoint{2.484702in}{2.796833in}}%
\pgfpathcurveto{\pgfqpoint{2.492516in}{2.789019in}}{\pgfqpoint{2.503115in}{2.784629in}}{\pgfqpoint{2.514165in}{2.784629in}}%
\pgfpathclose%
\pgfusepath{stroke,fill}%
\end{pgfscope}%
\begin{pgfscope}%
\pgfpathrectangle{\pgfqpoint{0.787074in}{0.548769in}}{\pgfqpoint{5.062926in}{3.102590in}}%
\pgfusepath{clip}%
\pgfsetbuttcap%
\pgfsetroundjoin%
\definecolor{currentfill}{rgb}{1.000000,0.498039,0.054902}%
\pgfsetfillcolor{currentfill}%
\pgfsetlinewidth{1.003750pt}%
\definecolor{currentstroke}{rgb}{1.000000,0.498039,0.054902}%
\pgfsetstrokecolor{currentstroke}%
\pgfsetdash{}{0pt}%
\pgfpathmoveto{\pgfqpoint{1.738689in}{2.406321in}}%
\pgfpathcurveto{\pgfqpoint{1.749739in}{2.406321in}}{\pgfqpoint{1.760339in}{2.410712in}}{\pgfqpoint{1.768152in}{2.418525in}}%
\pgfpathcurveto{\pgfqpoint{1.775966in}{2.426339in}}{\pgfqpoint{1.780356in}{2.436938in}}{\pgfqpoint{1.780356in}{2.447988in}}%
\pgfpathcurveto{\pgfqpoint{1.780356in}{2.459038in}}{\pgfqpoint{1.775966in}{2.469637in}}{\pgfqpoint{1.768152in}{2.477451in}}%
\pgfpathcurveto{\pgfqpoint{1.760339in}{2.485264in}}{\pgfqpoint{1.749739in}{2.489655in}}{\pgfqpoint{1.738689in}{2.489655in}}%
\pgfpathcurveto{\pgfqpoint{1.727639in}{2.489655in}}{\pgfqpoint{1.717040in}{2.485264in}}{\pgfqpoint{1.709227in}{2.477451in}}%
\pgfpathcurveto{\pgfqpoint{1.701413in}{2.469637in}}{\pgfqpoint{1.697023in}{2.459038in}}{\pgfqpoint{1.697023in}{2.447988in}}%
\pgfpathcurveto{\pgfqpoint{1.697023in}{2.436938in}}{\pgfqpoint{1.701413in}{2.426339in}}{\pgfqpoint{1.709227in}{2.418525in}}%
\pgfpathcurveto{\pgfqpoint{1.717040in}{2.410712in}}{\pgfqpoint{1.727639in}{2.406321in}}{\pgfqpoint{1.738689in}{2.406321in}}%
\pgfpathclose%
\pgfusepath{stroke,fill}%
\end{pgfscope}%
\begin{pgfscope}%
\pgfpathrectangle{\pgfqpoint{0.787074in}{0.548769in}}{\pgfqpoint{5.062926in}{3.102590in}}%
\pgfusepath{clip}%
\pgfsetbuttcap%
\pgfsetroundjoin%
\definecolor{currentfill}{rgb}{1.000000,0.498039,0.054902}%
\pgfsetfillcolor{currentfill}%
\pgfsetlinewidth{1.003750pt}%
\definecolor{currentstroke}{rgb}{1.000000,0.498039,0.054902}%
\pgfsetstrokecolor{currentstroke}%
\pgfsetdash{}{0pt}%
\pgfpathmoveto{\pgfqpoint{2.066982in}{2.705375in}}%
\pgfpathcurveto{\pgfqpoint{2.078032in}{2.705375in}}{\pgfqpoint{2.088631in}{2.709765in}}{\pgfqpoint{2.096445in}{2.717578in}}%
\pgfpathcurveto{\pgfqpoint{2.104258in}{2.725392in}}{\pgfqpoint{2.108649in}{2.735991in}}{\pgfqpoint{2.108649in}{2.747041in}}%
\pgfpathcurveto{\pgfqpoint{2.108649in}{2.758091in}}{\pgfqpoint{2.104258in}{2.768690in}}{\pgfqpoint{2.096445in}{2.776504in}}%
\pgfpathcurveto{\pgfqpoint{2.088631in}{2.784318in}}{\pgfqpoint{2.078032in}{2.788708in}}{\pgfqpoint{2.066982in}{2.788708in}}%
\pgfpathcurveto{\pgfqpoint{2.055932in}{2.788708in}}{\pgfqpoint{2.045333in}{2.784318in}}{\pgfqpoint{2.037519in}{2.776504in}}%
\pgfpathcurveto{\pgfqpoint{2.029706in}{2.768690in}}{\pgfqpoint{2.025315in}{2.758091in}}{\pgfqpoint{2.025315in}{2.747041in}}%
\pgfpathcurveto{\pgfqpoint{2.025315in}{2.735991in}}{\pgfqpoint{2.029706in}{2.725392in}}{\pgfqpoint{2.037519in}{2.717578in}}%
\pgfpathcurveto{\pgfqpoint{2.045333in}{2.709765in}}{\pgfqpoint{2.055932in}{2.705375in}}{\pgfqpoint{2.066982in}{2.705375in}}%
\pgfpathclose%
\pgfusepath{stroke,fill}%
\end{pgfscope}%
\begin{pgfscope}%
\pgfpathrectangle{\pgfqpoint{0.787074in}{0.548769in}}{\pgfqpoint{5.062926in}{3.102590in}}%
\pgfusepath{clip}%
\pgfsetbuttcap%
\pgfsetroundjoin%
\definecolor{currentfill}{rgb}{1.000000,0.498039,0.054902}%
\pgfsetfillcolor{currentfill}%
\pgfsetlinewidth{1.003750pt}%
\definecolor{currentstroke}{rgb}{1.000000,0.498039,0.054902}%
\pgfsetstrokecolor{currentstroke}%
\pgfsetdash{}{0pt}%
\pgfpathmoveto{\pgfqpoint{2.681419in}{2.307212in}}%
\pgfpathcurveto{\pgfqpoint{2.692469in}{2.307212in}}{\pgfqpoint{2.703068in}{2.311602in}}{\pgfqpoint{2.710882in}{2.319416in}}%
\pgfpathcurveto{\pgfqpoint{2.718695in}{2.327230in}}{\pgfqpoint{2.723086in}{2.337829in}}{\pgfqpoint{2.723086in}{2.348879in}}%
\pgfpathcurveto{\pgfqpoint{2.723086in}{2.359929in}}{\pgfqpoint{2.718695in}{2.370528in}}{\pgfqpoint{2.710882in}{2.378342in}}%
\pgfpathcurveto{\pgfqpoint{2.703068in}{2.386155in}}{\pgfqpoint{2.692469in}{2.390545in}}{\pgfqpoint{2.681419in}{2.390545in}}%
\pgfpathcurveto{\pgfqpoint{2.670369in}{2.390545in}}{\pgfqpoint{2.659770in}{2.386155in}}{\pgfqpoint{2.651956in}{2.378342in}}%
\pgfpathcurveto{\pgfqpoint{2.644143in}{2.370528in}}{\pgfqpoint{2.639752in}{2.359929in}}{\pgfqpoint{2.639752in}{2.348879in}}%
\pgfpathcurveto{\pgfqpoint{2.639752in}{2.337829in}}{\pgfqpoint{2.644143in}{2.327230in}}{\pgfqpoint{2.651956in}{2.319416in}}%
\pgfpathcurveto{\pgfqpoint{2.659770in}{2.311602in}}{\pgfqpoint{2.670369in}{2.307212in}}{\pgfqpoint{2.681419in}{2.307212in}}%
\pgfpathclose%
\pgfusepath{stroke,fill}%
\end{pgfscope}%
\begin{pgfscope}%
\pgfpathrectangle{\pgfqpoint{0.787074in}{0.548769in}}{\pgfqpoint{5.062926in}{3.102590in}}%
\pgfusepath{clip}%
\pgfsetbuttcap%
\pgfsetroundjoin%
\definecolor{currentfill}{rgb}{1.000000,0.498039,0.054902}%
\pgfsetfillcolor{currentfill}%
\pgfsetlinewidth{1.003750pt}%
\definecolor{currentstroke}{rgb}{1.000000,0.498039,0.054902}%
\pgfsetstrokecolor{currentstroke}%
\pgfsetdash{}{0pt}%
\pgfpathmoveto{\pgfqpoint{1.608107in}{3.001341in}}%
\pgfpathcurveto{\pgfqpoint{1.619157in}{3.001341in}}{\pgfqpoint{1.629756in}{3.005732in}}{\pgfqpoint{1.637569in}{3.013545in}}%
\pgfpathcurveto{\pgfqpoint{1.645383in}{3.021359in}}{\pgfqpoint{1.649773in}{3.031958in}}{\pgfqpoint{1.649773in}{3.043008in}}%
\pgfpathcurveto{\pgfqpoint{1.649773in}{3.054058in}}{\pgfqpoint{1.645383in}{3.064657in}}{\pgfqpoint{1.637569in}{3.072471in}}%
\pgfpathcurveto{\pgfqpoint{1.629756in}{3.080284in}}{\pgfqpoint{1.619157in}{3.084675in}}{\pgfqpoint{1.608107in}{3.084675in}}%
\pgfpathcurveto{\pgfqpoint{1.597056in}{3.084675in}}{\pgfqpoint{1.586457in}{3.080284in}}{\pgfqpoint{1.578644in}{3.072471in}}%
\pgfpathcurveto{\pgfqpoint{1.570830in}{3.064657in}}{\pgfqpoint{1.566440in}{3.054058in}}{\pgfqpoint{1.566440in}{3.043008in}}%
\pgfpathcurveto{\pgfqpoint{1.566440in}{3.031958in}}{\pgfqpoint{1.570830in}{3.021359in}}{\pgfqpoint{1.578644in}{3.013545in}}%
\pgfpathcurveto{\pgfqpoint{1.586457in}{3.005732in}}{\pgfqpoint{1.597056in}{3.001341in}}{\pgfqpoint{1.608107in}{3.001341in}}%
\pgfpathclose%
\pgfusepath{stroke,fill}%
\end{pgfscope}%
\begin{pgfscope}%
\pgfpathrectangle{\pgfqpoint{0.787074in}{0.548769in}}{\pgfqpoint{5.062926in}{3.102590in}}%
\pgfusepath{clip}%
\pgfsetbuttcap%
\pgfsetroundjoin%
\definecolor{currentfill}{rgb}{1.000000,0.498039,0.054902}%
\pgfsetfillcolor{currentfill}%
\pgfsetlinewidth{1.003750pt}%
\definecolor{currentstroke}{rgb}{1.000000,0.498039,0.054902}%
\pgfsetstrokecolor{currentstroke}%
\pgfsetdash{}{0pt}%
\pgfpathmoveto{\pgfqpoint{1.879303in}{2.401858in}}%
\pgfpathcurveto{\pgfqpoint{1.890353in}{2.401858in}}{\pgfqpoint{1.900952in}{2.406248in}}{\pgfqpoint{1.908765in}{2.414062in}}%
\pgfpathcurveto{\pgfqpoint{1.916579in}{2.421875in}}{\pgfqpoint{1.920969in}{2.432474in}}{\pgfqpoint{1.920969in}{2.443524in}}%
\pgfpathcurveto{\pgfqpoint{1.920969in}{2.454575in}}{\pgfqpoint{1.916579in}{2.465174in}}{\pgfqpoint{1.908765in}{2.472987in}}%
\pgfpathcurveto{\pgfqpoint{1.900952in}{2.480801in}}{\pgfqpoint{1.890353in}{2.485191in}}{\pgfqpoint{1.879303in}{2.485191in}}%
\pgfpathcurveto{\pgfqpoint{1.868252in}{2.485191in}}{\pgfqpoint{1.857653in}{2.480801in}}{\pgfqpoint{1.849840in}{2.472987in}}%
\pgfpathcurveto{\pgfqpoint{1.842026in}{2.465174in}}{\pgfqpoint{1.837636in}{2.454575in}}{\pgfqpoint{1.837636in}{2.443524in}}%
\pgfpathcurveto{\pgfqpoint{1.837636in}{2.432474in}}{\pgfqpoint{1.842026in}{2.421875in}}{\pgfqpoint{1.849840in}{2.414062in}}%
\pgfpathcurveto{\pgfqpoint{1.857653in}{2.406248in}}{\pgfqpoint{1.868252in}{2.401858in}}{\pgfqpoint{1.879303in}{2.401858in}}%
\pgfpathclose%
\pgfusepath{stroke,fill}%
\end{pgfscope}%
\begin{pgfscope}%
\pgfpathrectangle{\pgfqpoint{0.787074in}{0.548769in}}{\pgfqpoint{5.062926in}{3.102590in}}%
\pgfusepath{clip}%
\pgfsetbuttcap%
\pgfsetroundjoin%
\definecolor{currentfill}{rgb}{0.121569,0.466667,0.705882}%
\pgfsetfillcolor{currentfill}%
\pgfsetlinewidth{1.003750pt}%
\definecolor{currentstroke}{rgb}{0.121569,0.466667,0.705882}%
\pgfsetstrokecolor{currentstroke}%
\pgfsetdash{}{0pt}%
\pgfpathmoveto{\pgfqpoint{2.431403in}{2.416209in}}%
\pgfpathcurveto{\pgfqpoint{2.442453in}{2.416209in}}{\pgfqpoint{2.453052in}{2.420599in}}{\pgfqpoint{2.460866in}{2.428413in}}%
\pgfpathcurveto{\pgfqpoint{2.468679in}{2.436227in}}{\pgfqpoint{2.473070in}{2.446826in}}{\pgfqpoint{2.473070in}{2.457876in}}%
\pgfpathcurveto{\pgfqpoint{2.473070in}{2.468926in}}{\pgfqpoint{2.468679in}{2.479525in}}{\pgfqpoint{2.460866in}{2.487339in}}%
\pgfpathcurveto{\pgfqpoint{2.453052in}{2.495152in}}{\pgfqpoint{2.442453in}{2.499543in}}{\pgfqpoint{2.431403in}{2.499543in}}%
\pgfpathcurveto{\pgfqpoint{2.420353in}{2.499543in}}{\pgfqpoint{2.409754in}{2.495152in}}{\pgfqpoint{2.401940in}{2.487339in}}%
\pgfpathcurveto{\pgfqpoint{2.394127in}{2.479525in}}{\pgfqpoint{2.389736in}{2.468926in}}{\pgfqpoint{2.389736in}{2.457876in}}%
\pgfpathcurveto{\pgfqpoint{2.389736in}{2.446826in}}{\pgfqpoint{2.394127in}{2.436227in}}{\pgfqpoint{2.401940in}{2.428413in}}%
\pgfpathcurveto{\pgfqpoint{2.409754in}{2.420599in}}{\pgfqpoint{2.420353in}{2.416209in}}{\pgfqpoint{2.431403in}{2.416209in}}%
\pgfpathclose%
\pgfusepath{stroke,fill}%
\end{pgfscope}%
\begin{pgfscope}%
\pgfpathrectangle{\pgfqpoint{0.787074in}{0.548769in}}{\pgfqpoint{5.062926in}{3.102590in}}%
\pgfusepath{clip}%
\pgfsetbuttcap%
\pgfsetroundjoin%
\definecolor{currentfill}{rgb}{1.000000,0.498039,0.054902}%
\pgfsetfillcolor{currentfill}%
\pgfsetlinewidth{1.003750pt}%
\definecolor{currentstroke}{rgb}{1.000000,0.498039,0.054902}%
\pgfsetstrokecolor{currentstroke}%
\pgfsetdash{}{0pt}%
\pgfpathmoveto{\pgfqpoint{2.270313in}{3.029407in}}%
\pgfpathcurveto{\pgfqpoint{2.281364in}{3.029407in}}{\pgfqpoint{2.291963in}{3.033797in}}{\pgfqpoint{2.299776in}{3.041611in}}%
\pgfpathcurveto{\pgfqpoint{2.307590in}{3.049424in}}{\pgfqpoint{2.311980in}{3.060023in}}{\pgfqpoint{2.311980in}{3.071073in}}%
\pgfpathcurveto{\pgfqpoint{2.311980in}{3.082124in}}{\pgfqpoint{2.307590in}{3.092723in}}{\pgfqpoint{2.299776in}{3.100536in}}%
\pgfpathcurveto{\pgfqpoint{2.291963in}{3.108350in}}{\pgfqpoint{2.281364in}{3.112740in}}{\pgfqpoint{2.270313in}{3.112740in}}%
\pgfpathcurveto{\pgfqpoint{2.259263in}{3.112740in}}{\pgfqpoint{2.248664in}{3.108350in}}{\pgfqpoint{2.240851in}{3.100536in}}%
\pgfpathcurveto{\pgfqpoint{2.233037in}{3.092723in}}{\pgfqpoint{2.228647in}{3.082124in}}{\pgfqpoint{2.228647in}{3.071073in}}%
\pgfpathcurveto{\pgfqpoint{2.228647in}{3.060023in}}{\pgfqpoint{2.233037in}{3.049424in}}{\pgfqpoint{2.240851in}{3.041611in}}%
\pgfpathcurveto{\pgfqpoint{2.248664in}{3.033797in}}{\pgfqpoint{2.259263in}{3.029407in}}{\pgfqpoint{2.270313in}{3.029407in}}%
\pgfpathclose%
\pgfusepath{stroke,fill}%
\end{pgfscope}%
\begin{pgfscope}%
\pgfpathrectangle{\pgfqpoint{0.787074in}{0.548769in}}{\pgfqpoint{5.062926in}{3.102590in}}%
\pgfusepath{clip}%
\pgfsetbuttcap%
\pgfsetroundjoin%
\definecolor{currentfill}{rgb}{1.000000,0.498039,0.054902}%
\pgfsetfillcolor{currentfill}%
\pgfsetlinewidth{1.003750pt}%
\definecolor{currentstroke}{rgb}{1.000000,0.498039,0.054902}%
\pgfsetstrokecolor{currentstroke}%
\pgfsetdash{}{0pt}%
\pgfpathmoveto{\pgfqpoint{2.059148in}{2.943038in}}%
\pgfpathcurveto{\pgfqpoint{2.070198in}{2.943038in}}{\pgfqpoint{2.080797in}{2.947429in}}{\pgfqpoint{2.088610in}{2.955242in}}%
\pgfpathcurveto{\pgfqpoint{2.096424in}{2.963056in}}{\pgfqpoint{2.100814in}{2.973655in}}{\pgfqpoint{2.100814in}{2.984705in}}%
\pgfpathcurveto{\pgfqpoint{2.100814in}{2.995755in}}{\pgfqpoint{2.096424in}{3.006354in}}{\pgfqpoint{2.088610in}{3.014168in}}%
\pgfpathcurveto{\pgfqpoint{2.080797in}{3.021981in}}{\pgfqpoint{2.070198in}{3.026372in}}{\pgfqpoint{2.059148in}{3.026372in}}%
\pgfpathcurveto{\pgfqpoint{2.048097in}{3.026372in}}{\pgfqpoint{2.037498in}{3.021981in}}{\pgfqpoint{2.029685in}{3.014168in}}%
\pgfpathcurveto{\pgfqpoint{2.021871in}{3.006354in}}{\pgfqpoint{2.017481in}{2.995755in}}{\pgfqpoint{2.017481in}{2.984705in}}%
\pgfpathcurveto{\pgfqpoint{2.017481in}{2.973655in}}{\pgfqpoint{2.021871in}{2.963056in}}{\pgfqpoint{2.029685in}{2.955242in}}%
\pgfpathcurveto{\pgfqpoint{2.037498in}{2.947429in}}{\pgfqpoint{2.048097in}{2.943038in}}{\pgfqpoint{2.059148in}{2.943038in}}%
\pgfpathclose%
\pgfusepath{stroke,fill}%
\end{pgfscope}%
\begin{pgfscope}%
\pgfpathrectangle{\pgfqpoint{0.787074in}{0.548769in}}{\pgfqpoint{5.062926in}{3.102590in}}%
\pgfusepath{clip}%
\pgfsetbuttcap%
\pgfsetroundjoin%
\definecolor{currentfill}{rgb}{1.000000,0.498039,0.054902}%
\pgfsetfillcolor{currentfill}%
\pgfsetlinewidth{1.003750pt}%
\definecolor{currentstroke}{rgb}{1.000000,0.498039,0.054902}%
\pgfsetstrokecolor{currentstroke}%
\pgfsetdash{}{0pt}%
\pgfpathmoveto{\pgfqpoint{1.806791in}{2.587341in}}%
\pgfpathcurveto{\pgfqpoint{1.817842in}{2.587341in}}{\pgfqpoint{1.828441in}{2.591732in}}{\pgfqpoint{1.836254in}{2.599545in}}%
\pgfpathcurveto{\pgfqpoint{1.844068in}{2.607359in}}{\pgfqpoint{1.848458in}{2.617958in}}{\pgfqpoint{1.848458in}{2.629008in}}%
\pgfpathcurveto{\pgfqpoint{1.848458in}{2.640058in}}{\pgfqpoint{1.844068in}{2.650657in}}{\pgfqpoint{1.836254in}{2.658471in}}%
\pgfpathcurveto{\pgfqpoint{1.828441in}{2.666284in}}{\pgfqpoint{1.817842in}{2.670675in}}{\pgfqpoint{1.806791in}{2.670675in}}%
\pgfpathcurveto{\pgfqpoint{1.795741in}{2.670675in}}{\pgfqpoint{1.785142in}{2.666284in}}{\pgfqpoint{1.777329in}{2.658471in}}%
\pgfpathcurveto{\pgfqpoint{1.769515in}{2.650657in}}{\pgfqpoint{1.765125in}{2.640058in}}{\pgfqpoint{1.765125in}{2.629008in}}%
\pgfpathcurveto{\pgfqpoint{1.765125in}{2.617958in}}{\pgfqpoint{1.769515in}{2.607359in}}{\pgfqpoint{1.777329in}{2.599545in}}%
\pgfpathcurveto{\pgfqpoint{1.785142in}{2.591732in}}{\pgfqpoint{1.795741in}{2.587341in}}{\pgfqpoint{1.806791in}{2.587341in}}%
\pgfpathclose%
\pgfusepath{stroke,fill}%
\end{pgfscope}%
\begin{pgfscope}%
\pgfpathrectangle{\pgfqpoint{0.787074in}{0.548769in}}{\pgfqpoint{5.062926in}{3.102590in}}%
\pgfusepath{clip}%
\pgfsetbuttcap%
\pgfsetroundjoin%
\definecolor{currentfill}{rgb}{1.000000,0.498039,0.054902}%
\pgfsetfillcolor{currentfill}%
\pgfsetlinewidth{1.003750pt}%
\definecolor{currentstroke}{rgb}{1.000000,0.498039,0.054902}%
\pgfsetstrokecolor{currentstroke}%
\pgfsetdash{}{0pt}%
\pgfpathmoveto{\pgfqpoint{1.909776in}{2.817920in}}%
\pgfpathcurveto{\pgfqpoint{1.920826in}{2.817920in}}{\pgfqpoint{1.931425in}{2.822310in}}{\pgfqpoint{1.939238in}{2.830124in}}%
\pgfpathcurveto{\pgfqpoint{1.947052in}{2.837937in}}{\pgfqpoint{1.951442in}{2.848536in}}{\pgfqpoint{1.951442in}{2.859586in}}%
\pgfpathcurveto{\pgfqpoint{1.951442in}{2.870637in}}{\pgfqpoint{1.947052in}{2.881236in}}{\pgfqpoint{1.939238in}{2.889049in}}%
\pgfpathcurveto{\pgfqpoint{1.931425in}{2.896863in}}{\pgfqpoint{1.920826in}{2.901253in}}{\pgfqpoint{1.909776in}{2.901253in}}%
\pgfpathcurveto{\pgfqpoint{1.898725in}{2.901253in}}{\pgfqpoint{1.888126in}{2.896863in}}{\pgfqpoint{1.880313in}{2.889049in}}%
\pgfpathcurveto{\pgfqpoint{1.872499in}{2.881236in}}{\pgfqpoint{1.868109in}{2.870637in}}{\pgfqpoint{1.868109in}{2.859586in}}%
\pgfpathcurveto{\pgfqpoint{1.868109in}{2.848536in}}{\pgfqpoint{1.872499in}{2.837937in}}{\pgfqpoint{1.880313in}{2.830124in}}%
\pgfpathcurveto{\pgfqpoint{1.888126in}{2.822310in}}{\pgfqpoint{1.898725in}{2.817920in}}{\pgfqpoint{1.909776in}{2.817920in}}%
\pgfpathclose%
\pgfusepath{stroke,fill}%
\end{pgfscope}%
\begin{pgfscope}%
\pgfpathrectangle{\pgfqpoint{0.787074in}{0.548769in}}{\pgfqpoint{5.062926in}{3.102590in}}%
\pgfusepath{clip}%
\pgfsetbuttcap%
\pgfsetroundjoin%
\definecolor{currentfill}{rgb}{1.000000,0.498039,0.054902}%
\pgfsetfillcolor{currentfill}%
\pgfsetlinewidth{1.003750pt}%
\definecolor{currentstroke}{rgb}{1.000000,0.498039,0.054902}%
\pgfsetstrokecolor{currentstroke}%
\pgfsetdash{}{0pt}%
\pgfpathmoveto{\pgfqpoint{1.735595in}{2.246871in}}%
\pgfpathcurveto{\pgfqpoint{1.746645in}{2.246871in}}{\pgfqpoint{1.757244in}{2.251261in}}{\pgfqpoint{1.765057in}{2.259075in}}%
\pgfpathcurveto{\pgfqpoint{1.772871in}{2.266888in}}{\pgfqpoint{1.777261in}{2.277487in}}{\pgfqpoint{1.777261in}{2.288537in}}%
\pgfpathcurveto{\pgfqpoint{1.777261in}{2.299587in}}{\pgfqpoint{1.772871in}{2.310186in}}{\pgfqpoint{1.765057in}{2.318000in}}%
\pgfpathcurveto{\pgfqpoint{1.757244in}{2.325814in}}{\pgfqpoint{1.746645in}{2.330204in}}{\pgfqpoint{1.735595in}{2.330204in}}%
\pgfpathcurveto{\pgfqpoint{1.724544in}{2.330204in}}{\pgfqpoint{1.713945in}{2.325814in}}{\pgfqpoint{1.706132in}{2.318000in}}%
\pgfpathcurveto{\pgfqpoint{1.698318in}{2.310186in}}{\pgfqpoint{1.693928in}{2.299587in}}{\pgfqpoint{1.693928in}{2.288537in}}%
\pgfpathcurveto{\pgfqpoint{1.693928in}{2.277487in}}{\pgfqpoint{1.698318in}{2.266888in}}{\pgfqpoint{1.706132in}{2.259075in}}%
\pgfpathcurveto{\pgfqpoint{1.713945in}{2.251261in}}{\pgfqpoint{1.724544in}{2.246871in}}{\pgfqpoint{1.735595in}{2.246871in}}%
\pgfpathclose%
\pgfusepath{stroke,fill}%
\end{pgfscope}%
\begin{pgfscope}%
\pgfpathrectangle{\pgfqpoint{0.787074in}{0.548769in}}{\pgfqpoint{5.062926in}{3.102590in}}%
\pgfusepath{clip}%
\pgfsetbuttcap%
\pgfsetroundjoin%
\definecolor{currentfill}{rgb}{1.000000,0.498039,0.054902}%
\pgfsetfillcolor{currentfill}%
\pgfsetlinewidth{1.003750pt}%
\definecolor{currentstroke}{rgb}{1.000000,0.498039,0.054902}%
\pgfsetstrokecolor{currentstroke}%
\pgfsetdash{}{0pt}%
\pgfpathmoveto{\pgfqpoint{2.670049in}{1.776148in}}%
\pgfpathcurveto{\pgfqpoint{2.681099in}{1.776148in}}{\pgfqpoint{2.691698in}{1.780538in}}{\pgfqpoint{2.699512in}{1.788352in}}%
\pgfpathcurveto{\pgfqpoint{2.707325in}{1.796166in}}{\pgfqpoint{2.711716in}{1.806765in}}{\pgfqpoint{2.711716in}{1.817815in}}%
\pgfpathcurveto{\pgfqpoint{2.711716in}{1.828865in}}{\pgfqpoint{2.707325in}{1.839464in}}{\pgfqpoint{2.699512in}{1.847278in}}%
\pgfpathcurveto{\pgfqpoint{2.691698in}{1.855091in}}{\pgfqpoint{2.681099in}{1.859481in}}{\pgfqpoint{2.670049in}{1.859481in}}%
\pgfpathcurveto{\pgfqpoint{2.658999in}{1.859481in}}{\pgfqpoint{2.648400in}{1.855091in}}{\pgfqpoint{2.640586in}{1.847278in}}%
\pgfpathcurveto{\pgfqpoint{2.632772in}{1.839464in}}{\pgfqpoint{2.628382in}{1.828865in}}{\pgfqpoint{2.628382in}{1.817815in}}%
\pgfpathcurveto{\pgfqpoint{2.628382in}{1.806765in}}{\pgfqpoint{2.632772in}{1.796166in}}{\pgfqpoint{2.640586in}{1.788352in}}%
\pgfpathcurveto{\pgfqpoint{2.648400in}{1.780538in}}{\pgfqpoint{2.658999in}{1.776148in}}{\pgfqpoint{2.670049in}{1.776148in}}%
\pgfpathclose%
\pgfusepath{stroke,fill}%
\end{pgfscope}%
\begin{pgfscope}%
\pgfpathrectangle{\pgfqpoint{0.787074in}{0.548769in}}{\pgfqpoint{5.062926in}{3.102590in}}%
\pgfusepath{clip}%
\pgfsetbuttcap%
\pgfsetroundjoin%
\definecolor{currentfill}{rgb}{1.000000,0.498039,0.054902}%
\pgfsetfillcolor{currentfill}%
\pgfsetlinewidth{1.003750pt}%
\definecolor{currentstroke}{rgb}{1.000000,0.498039,0.054902}%
\pgfsetstrokecolor{currentstroke}%
\pgfsetdash{}{0pt}%
\pgfpathmoveto{\pgfqpoint{3.039506in}{2.992883in}}%
\pgfpathcurveto{\pgfqpoint{3.050557in}{2.992883in}}{\pgfqpoint{3.061156in}{2.997273in}}{\pgfqpoint{3.068969in}{3.005087in}}%
\pgfpathcurveto{\pgfqpoint{3.076783in}{3.012900in}}{\pgfqpoint{3.081173in}{3.023499in}}{\pgfqpoint{3.081173in}{3.034549in}}%
\pgfpathcurveto{\pgfqpoint{3.081173in}{3.045599in}}{\pgfqpoint{3.076783in}{3.056199in}}{\pgfqpoint{3.068969in}{3.064012in}}%
\pgfpathcurveto{\pgfqpoint{3.061156in}{3.071826in}}{\pgfqpoint{3.050557in}{3.076216in}}{\pgfqpoint{3.039506in}{3.076216in}}%
\pgfpathcurveto{\pgfqpoint{3.028456in}{3.076216in}}{\pgfqpoint{3.017857in}{3.071826in}}{\pgfqpoint{3.010044in}{3.064012in}}%
\pgfpathcurveto{\pgfqpoint{3.002230in}{3.056199in}}{\pgfqpoint{2.997840in}{3.045599in}}{\pgfqpoint{2.997840in}{3.034549in}}%
\pgfpathcurveto{\pgfqpoint{2.997840in}{3.023499in}}{\pgfqpoint{3.002230in}{3.012900in}}{\pgfqpoint{3.010044in}{3.005087in}}%
\pgfpathcurveto{\pgfqpoint{3.017857in}{2.997273in}}{\pgfqpoint{3.028456in}{2.992883in}}{\pgfqpoint{3.039506in}{2.992883in}}%
\pgfpathclose%
\pgfusepath{stroke,fill}%
\end{pgfscope}%
\begin{pgfscope}%
\pgfpathrectangle{\pgfqpoint{0.787074in}{0.548769in}}{\pgfqpoint{5.062926in}{3.102590in}}%
\pgfusepath{clip}%
\pgfsetbuttcap%
\pgfsetroundjoin%
\definecolor{currentfill}{rgb}{1.000000,0.498039,0.054902}%
\pgfsetfillcolor{currentfill}%
\pgfsetlinewidth{1.003750pt}%
\definecolor{currentstroke}{rgb}{1.000000,0.498039,0.054902}%
\pgfsetstrokecolor{currentstroke}%
\pgfsetdash{}{0pt}%
\pgfpathmoveto{\pgfqpoint{2.377452in}{1.537669in}}%
\pgfpathcurveto{\pgfqpoint{2.388502in}{1.537669in}}{\pgfqpoint{2.399101in}{1.542060in}}{\pgfqpoint{2.406915in}{1.549873in}}%
\pgfpathcurveto{\pgfqpoint{2.414729in}{1.557687in}}{\pgfqpoint{2.419119in}{1.568286in}}{\pgfqpoint{2.419119in}{1.579336in}}%
\pgfpathcurveto{\pgfqpoint{2.419119in}{1.590386in}}{\pgfqpoint{2.414729in}{1.600985in}}{\pgfqpoint{2.406915in}{1.608799in}}%
\pgfpathcurveto{\pgfqpoint{2.399101in}{1.616612in}}{\pgfqpoint{2.388502in}{1.621003in}}{\pgfqpoint{2.377452in}{1.621003in}}%
\pgfpathcurveto{\pgfqpoint{2.366402in}{1.621003in}}{\pgfqpoint{2.355803in}{1.616612in}}{\pgfqpoint{2.347989in}{1.608799in}}%
\pgfpathcurveto{\pgfqpoint{2.340176in}{1.600985in}}{\pgfqpoint{2.335785in}{1.590386in}}{\pgfqpoint{2.335785in}{1.579336in}}%
\pgfpathcurveto{\pgfqpoint{2.335785in}{1.568286in}}{\pgfqpoint{2.340176in}{1.557687in}}{\pgfqpoint{2.347989in}{1.549873in}}%
\pgfpathcurveto{\pgfqpoint{2.355803in}{1.542060in}}{\pgfqpoint{2.366402in}{1.537669in}}{\pgfqpoint{2.377452in}{1.537669in}}%
\pgfpathclose%
\pgfusepath{stroke,fill}%
\end{pgfscope}%
\begin{pgfscope}%
\pgfpathrectangle{\pgfqpoint{0.787074in}{0.548769in}}{\pgfqpoint{5.062926in}{3.102590in}}%
\pgfusepath{clip}%
\pgfsetbuttcap%
\pgfsetroundjoin%
\definecolor{currentfill}{rgb}{0.121569,0.466667,0.705882}%
\pgfsetfillcolor{currentfill}%
\pgfsetlinewidth{1.003750pt}%
\definecolor{currentstroke}{rgb}{0.121569,0.466667,0.705882}%
\pgfsetstrokecolor{currentstroke}%
\pgfsetdash{}{0pt}%
\pgfpathmoveto{\pgfqpoint{1.716517in}{0.681479in}}%
\pgfpathcurveto{\pgfqpoint{1.727567in}{0.681479in}}{\pgfqpoint{1.738166in}{0.685869in}}{\pgfqpoint{1.745980in}{0.693683in}}%
\pgfpathcurveto{\pgfqpoint{1.753794in}{0.701497in}}{\pgfqpoint{1.758184in}{0.712096in}}{\pgfqpoint{1.758184in}{0.723146in}}%
\pgfpathcurveto{\pgfqpoint{1.758184in}{0.734196in}}{\pgfqpoint{1.753794in}{0.744795in}}{\pgfqpoint{1.745980in}{0.752609in}}%
\pgfpathcurveto{\pgfqpoint{1.738166in}{0.760422in}}{\pgfqpoint{1.727567in}{0.764812in}}{\pgfqpoint{1.716517in}{0.764812in}}%
\pgfpathcurveto{\pgfqpoint{1.705467in}{0.764812in}}{\pgfqpoint{1.694868in}{0.760422in}}{\pgfqpoint{1.687054in}{0.752609in}}%
\pgfpathcurveto{\pgfqpoint{1.679241in}{0.744795in}}{\pgfqpoint{1.674850in}{0.734196in}}{\pgfqpoint{1.674850in}{0.723146in}}%
\pgfpathcurveto{\pgfqpoint{1.674850in}{0.712096in}}{\pgfqpoint{1.679241in}{0.701497in}}{\pgfqpoint{1.687054in}{0.693683in}}%
\pgfpathcurveto{\pgfqpoint{1.694868in}{0.685869in}}{\pgfqpoint{1.705467in}{0.681479in}}{\pgfqpoint{1.716517in}{0.681479in}}%
\pgfpathclose%
\pgfusepath{stroke,fill}%
\end{pgfscope}%
\begin{pgfscope}%
\pgfpathrectangle{\pgfqpoint{0.787074in}{0.548769in}}{\pgfqpoint{5.062926in}{3.102590in}}%
\pgfusepath{clip}%
\pgfsetbuttcap%
\pgfsetroundjoin%
\definecolor{currentfill}{rgb}{1.000000,0.498039,0.054902}%
\pgfsetfillcolor{currentfill}%
\pgfsetlinewidth{1.003750pt}%
\definecolor{currentstroke}{rgb}{1.000000,0.498039,0.054902}%
\pgfsetstrokecolor{currentstroke}%
\pgfsetdash{}{0pt}%
\pgfpathmoveto{\pgfqpoint{2.230615in}{2.885059in}}%
\pgfpathcurveto{\pgfqpoint{2.241666in}{2.885059in}}{\pgfqpoint{2.252265in}{2.889449in}}{\pgfqpoint{2.260078in}{2.897263in}}%
\pgfpathcurveto{\pgfqpoint{2.267892in}{2.905076in}}{\pgfqpoint{2.272282in}{2.915675in}}{\pgfqpoint{2.272282in}{2.926726in}}%
\pgfpathcurveto{\pgfqpoint{2.272282in}{2.937776in}}{\pgfqpoint{2.267892in}{2.948375in}}{\pgfqpoint{2.260078in}{2.956188in}}%
\pgfpathcurveto{\pgfqpoint{2.252265in}{2.964002in}}{\pgfqpoint{2.241666in}{2.968392in}}{\pgfqpoint{2.230615in}{2.968392in}}%
\pgfpathcurveto{\pgfqpoint{2.219565in}{2.968392in}}{\pgfqpoint{2.208966in}{2.964002in}}{\pgfqpoint{2.201153in}{2.956188in}}%
\pgfpathcurveto{\pgfqpoint{2.193339in}{2.948375in}}{\pgfqpoint{2.188949in}{2.937776in}}{\pgfqpoint{2.188949in}{2.926726in}}%
\pgfpathcurveto{\pgfqpoint{2.188949in}{2.915675in}}{\pgfqpoint{2.193339in}{2.905076in}}{\pgfqpoint{2.201153in}{2.897263in}}%
\pgfpathcurveto{\pgfqpoint{2.208966in}{2.889449in}}{\pgfqpoint{2.219565in}{2.885059in}}{\pgfqpoint{2.230615in}{2.885059in}}%
\pgfpathclose%
\pgfusepath{stroke,fill}%
\end{pgfscope}%
\begin{pgfscope}%
\pgfpathrectangle{\pgfqpoint{0.787074in}{0.548769in}}{\pgfqpoint{5.062926in}{3.102590in}}%
\pgfusepath{clip}%
\pgfsetbuttcap%
\pgfsetroundjoin%
\definecolor{currentfill}{rgb}{1.000000,0.498039,0.054902}%
\pgfsetfillcolor{currentfill}%
\pgfsetlinewidth{1.003750pt}%
\definecolor{currentstroke}{rgb}{1.000000,0.498039,0.054902}%
\pgfsetstrokecolor{currentstroke}%
\pgfsetdash{}{0pt}%
\pgfpathmoveto{\pgfqpoint{1.627574in}{2.017445in}}%
\pgfpathcurveto{\pgfqpoint{1.638624in}{2.017445in}}{\pgfqpoint{1.649223in}{2.021836in}}{\pgfqpoint{1.657037in}{2.029649in}}%
\pgfpathcurveto{\pgfqpoint{1.664850in}{2.037463in}}{\pgfqpoint{1.669241in}{2.048062in}}{\pgfqpoint{1.669241in}{2.059112in}}%
\pgfpathcurveto{\pgfqpoint{1.669241in}{2.070162in}}{\pgfqpoint{1.664850in}{2.080761in}}{\pgfqpoint{1.657037in}{2.088575in}}%
\pgfpathcurveto{\pgfqpoint{1.649223in}{2.096388in}}{\pgfqpoint{1.638624in}{2.100779in}}{\pgfqpoint{1.627574in}{2.100779in}}%
\pgfpathcurveto{\pgfqpoint{1.616524in}{2.100779in}}{\pgfqpoint{1.605925in}{2.096388in}}{\pgfqpoint{1.598111in}{2.088575in}}%
\pgfpathcurveto{\pgfqpoint{1.590298in}{2.080761in}}{\pgfqpoint{1.585907in}{2.070162in}}{\pgfqpoint{1.585907in}{2.059112in}}%
\pgfpathcurveto{\pgfqpoint{1.585907in}{2.048062in}}{\pgfqpoint{1.590298in}{2.037463in}}{\pgfqpoint{1.598111in}{2.029649in}}%
\pgfpathcurveto{\pgfqpoint{1.605925in}{2.021836in}}{\pgfqpoint{1.616524in}{2.017445in}}{\pgfqpoint{1.627574in}{2.017445in}}%
\pgfpathclose%
\pgfusepath{stroke,fill}%
\end{pgfscope}%
\begin{pgfscope}%
\pgfpathrectangle{\pgfqpoint{0.787074in}{0.548769in}}{\pgfqpoint{5.062926in}{3.102590in}}%
\pgfusepath{clip}%
\pgfsetbuttcap%
\pgfsetroundjoin%
\definecolor{currentfill}{rgb}{0.121569,0.466667,0.705882}%
\pgfsetfillcolor{currentfill}%
\pgfsetlinewidth{1.003750pt}%
\definecolor{currentstroke}{rgb}{0.121569,0.466667,0.705882}%
\pgfsetstrokecolor{currentstroke}%
\pgfsetdash{}{0pt}%
\pgfpathmoveto{\pgfqpoint{1.860700in}{2.239976in}}%
\pgfpathcurveto{\pgfqpoint{1.871750in}{2.239976in}}{\pgfqpoint{1.882349in}{2.244367in}}{\pgfqpoint{1.890163in}{2.252180in}}%
\pgfpathcurveto{\pgfqpoint{1.897976in}{2.259994in}}{\pgfqpoint{1.902367in}{2.270593in}}{\pgfqpoint{1.902367in}{2.281643in}}%
\pgfpathcurveto{\pgfqpoint{1.902367in}{2.292693in}}{\pgfqpoint{1.897976in}{2.303292in}}{\pgfqpoint{1.890163in}{2.311106in}}%
\pgfpathcurveto{\pgfqpoint{1.882349in}{2.318919in}}{\pgfqpoint{1.871750in}{2.323310in}}{\pgfqpoint{1.860700in}{2.323310in}}%
\pgfpathcurveto{\pgfqpoint{1.849650in}{2.323310in}}{\pgfqpoint{1.839051in}{2.318919in}}{\pgfqpoint{1.831237in}{2.311106in}}%
\pgfpathcurveto{\pgfqpoint{1.823424in}{2.303292in}}{\pgfqpoint{1.819033in}{2.292693in}}{\pgfqpoint{1.819033in}{2.281643in}}%
\pgfpathcurveto{\pgfqpoint{1.819033in}{2.270593in}}{\pgfqpoint{1.823424in}{2.259994in}}{\pgfqpoint{1.831237in}{2.252180in}}%
\pgfpathcurveto{\pgfqpoint{1.839051in}{2.244367in}}{\pgfqpoint{1.849650in}{2.239976in}}{\pgfqpoint{1.860700in}{2.239976in}}%
\pgfpathclose%
\pgfusepath{stroke,fill}%
\end{pgfscope}%
\begin{pgfscope}%
\pgfpathrectangle{\pgfqpoint{0.787074in}{0.548769in}}{\pgfqpoint{5.062926in}{3.102590in}}%
\pgfusepath{clip}%
\pgfsetbuttcap%
\pgfsetroundjoin%
\definecolor{currentfill}{rgb}{0.121569,0.466667,0.705882}%
\pgfsetfillcolor{currentfill}%
\pgfsetlinewidth{1.003750pt}%
\definecolor{currentstroke}{rgb}{0.121569,0.466667,0.705882}%
\pgfsetstrokecolor{currentstroke}%
\pgfsetdash{}{0pt}%
\pgfpathmoveto{\pgfqpoint{2.083414in}{2.686510in}}%
\pgfpathcurveto{\pgfqpoint{2.094464in}{2.686510in}}{\pgfqpoint{2.105063in}{2.690900in}}{\pgfqpoint{2.112877in}{2.698714in}}%
\pgfpathcurveto{\pgfqpoint{2.120690in}{2.706527in}}{\pgfqpoint{2.125081in}{2.717126in}}{\pgfqpoint{2.125081in}{2.728177in}}%
\pgfpathcurveto{\pgfqpoint{2.125081in}{2.739227in}}{\pgfqpoint{2.120690in}{2.749826in}}{\pgfqpoint{2.112877in}{2.757639in}}%
\pgfpathcurveto{\pgfqpoint{2.105063in}{2.765453in}}{\pgfqpoint{2.094464in}{2.769843in}}{\pgfqpoint{2.083414in}{2.769843in}}%
\pgfpathcurveto{\pgfqpoint{2.072364in}{2.769843in}}{\pgfqpoint{2.061765in}{2.765453in}}{\pgfqpoint{2.053951in}{2.757639in}}%
\pgfpathcurveto{\pgfqpoint{2.046138in}{2.749826in}}{\pgfqpoint{2.041747in}{2.739227in}}{\pgfqpoint{2.041747in}{2.728177in}}%
\pgfpathcurveto{\pgfqpoint{2.041747in}{2.717126in}}{\pgfqpoint{2.046138in}{2.706527in}}{\pgfqpoint{2.053951in}{2.698714in}}%
\pgfpathcurveto{\pgfqpoint{2.061765in}{2.690900in}}{\pgfqpoint{2.072364in}{2.686510in}}{\pgfqpoint{2.083414in}{2.686510in}}%
\pgfpathclose%
\pgfusepath{stroke,fill}%
\end{pgfscope}%
\begin{pgfscope}%
\pgfpathrectangle{\pgfqpoint{0.787074in}{0.548769in}}{\pgfqpoint{5.062926in}{3.102590in}}%
\pgfusepath{clip}%
\pgfsetbuttcap%
\pgfsetroundjoin%
\definecolor{currentfill}{rgb}{1.000000,0.498039,0.054902}%
\pgfsetfillcolor{currentfill}%
\pgfsetlinewidth{1.003750pt}%
\definecolor{currentstroke}{rgb}{1.000000,0.498039,0.054902}%
\pgfsetstrokecolor{currentstroke}%
\pgfsetdash{}{0pt}%
\pgfpathmoveto{\pgfqpoint{1.875784in}{2.173816in}}%
\pgfpathcurveto{\pgfqpoint{1.886834in}{2.173816in}}{\pgfqpoint{1.897433in}{2.178206in}}{\pgfqpoint{1.905247in}{2.186020in}}%
\pgfpathcurveto{\pgfqpoint{1.913060in}{2.193834in}}{\pgfqpoint{1.917451in}{2.204433in}}{\pgfqpoint{1.917451in}{2.215483in}}%
\pgfpathcurveto{\pgfqpoint{1.917451in}{2.226533in}}{\pgfqpoint{1.913060in}{2.237132in}}{\pgfqpoint{1.905247in}{2.244946in}}%
\pgfpathcurveto{\pgfqpoint{1.897433in}{2.252759in}}{\pgfqpoint{1.886834in}{2.257150in}}{\pgfqpoint{1.875784in}{2.257150in}}%
\pgfpathcurveto{\pgfqpoint{1.864734in}{2.257150in}}{\pgfqpoint{1.854135in}{2.252759in}}{\pgfqpoint{1.846321in}{2.244946in}}%
\pgfpathcurveto{\pgfqpoint{1.838507in}{2.237132in}}{\pgfqpoint{1.834117in}{2.226533in}}{\pgfqpoint{1.834117in}{2.215483in}}%
\pgfpathcurveto{\pgfqpoint{1.834117in}{2.204433in}}{\pgfqpoint{1.838507in}{2.193834in}}{\pgfqpoint{1.846321in}{2.186020in}}%
\pgfpathcurveto{\pgfqpoint{1.854135in}{2.178206in}}{\pgfqpoint{1.864734in}{2.173816in}}{\pgfqpoint{1.875784in}{2.173816in}}%
\pgfpathclose%
\pgfusepath{stroke,fill}%
\end{pgfscope}%
\begin{pgfscope}%
\pgfpathrectangle{\pgfqpoint{0.787074in}{0.548769in}}{\pgfqpoint{5.062926in}{3.102590in}}%
\pgfusepath{clip}%
\pgfsetbuttcap%
\pgfsetroundjoin%
\definecolor{currentfill}{rgb}{0.121569,0.466667,0.705882}%
\pgfsetfillcolor{currentfill}%
\pgfsetlinewidth{1.003750pt}%
\definecolor{currentstroke}{rgb}{0.121569,0.466667,0.705882}%
\pgfsetstrokecolor{currentstroke}%
\pgfsetdash{}{0pt}%
\pgfpathmoveto{\pgfqpoint{1.709191in}{1.676903in}}%
\pgfpathcurveto{\pgfqpoint{1.720242in}{1.676903in}}{\pgfqpoint{1.730841in}{1.681293in}}{\pgfqpoint{1.738654in}{1.689107in}}%
\pgfpathcurveto{\pgfqpoint{1.746468in}{1.696921in}}{\pgfqpoint{1.750858in}{1.707520in}}{\pgfqpoint{1.750858in}{1.718570in}}%
\pgfpathcurveto{\pgfqpoint{1.750858in}{1.729620in}}{\pgfqpoint{1.746468in}{1.740219in}}{\pgfqpoint{1.738654in}{1.748032in}}%
\pgfpathcurveto{\pgfqpoint{1.730841in}{1.755846in}}{\pgfqpoint{1.720242in}{1.760236in}}{\pgfqpoint{1.709191in}{1.760236in}}%
\pgfpathcurveto{\pgfqpoint{1.698141in}{1.760236in}}{\pgfqpoint{1.687542in}{1.755846in}}{\pgfqpoint{1.679729in}{1.748032in}}%
\pgfpathcurveto{\pgfqpoint{1.671915in}{1.740219in}}{\pgfqpoint{1.667525in}{1.729620in}}{\pgfqpoint{1.667525in}{1.718570in}}%
\pgfpathcurveto{\pgfqpoint{1.667525in}{1.707520in}}{\pgfqpoint{1.671915in}{1.696921in}}{\pgfqpoint{1.679729in}{1.689107in}}%
\pgfpathcurveto{\pgfqpoint{1.687542in}{1.681293in}}{\pgfqpoint{1.698141in}{1.676903in}}{\pgfqpoint{1.709191in}{1.676903in}}%
\pgfpathclose%
\pgfusepath{stroke,fill}%
\end{pgfscope}%
\begin{pgfscope}%
\pgfpathrectangle{\pgfqpoint{0.787074in}{0.548769in}}{\pgfqpoint{5.062926in}{3.102590in}}%
\pgfusepath{clip}%
\pgfsetbuttcap%
\pgfsetroundjoin%
\definecolor{currentfill}{rgb}{0.121569,0.466667,0.705882}%
\pgfsetfillcolor{currentfill}%
\pgfsetlinewidth{1.003750pt}%
\definecolor{currentstroke}{rgb}{0.121569,0.466667,0.705882}%
\pgfsetstrokecolor{currentstroke}%
\pgfsetdash{}{0pt}%
\pgfpathmoveto{\pgfqpoint{2.676696in}{1.092741in}}%
\pgfpathcurveto{\pgfqpoint{2.687746in}{1.092741in}}{\pgfqpoint{2.698345in}{1.097131in}}{\pgfqpoint{2.706159in}{1.104945in}}%
\pgfpathcurveto{\pgfqpoint{2.713973in}{1.112758in}}{\pgfqpoint{2.718363in}{1.123357in}}{\pgfqpoint{2.718363in}{1.134408in}}%
\pgfpathcurveto{\pgfqpoint{2.718363in}{1.145458in}}{\pgfqpoint{2.713973in}{1.156057in}}{\pgfqpoint{2.706159in}{1.163870in}}%
\pgfpathcurveto{\pgfqpoint{2.698345in}{1.171684in}}{\pgfqpoint{2.687746in}{1.176074in}}{\pgfqpoint{2.676696in}{1.176074in}}%
\pgfpathcurveto{\pgfqpoint{2.665646in}{1.176074in}}{\pgfqpoint{2.655047in}{1.171684in}}{\pgfqpoint{2.647234in}{1.163870in}}%
\pgfpathcurveto{\pgfqpoint{2.639420in}{1.156057in}}{\pgfqpoint{2.635030in}{1.145458in}}{\pgfqpoint{2.635030in}{1.134408in}}%
\pgfpathcurveto{\pgfqpoint{2.635030in}{1.123357in}}{\pgfqpoint{2.639420in}{1.112758in}}{\pgfqpoint{2.647234in}{1.104945in}}%
\pgfpathcurveto{\pgfqpoint{2.655047in}{1.097131in}}{\pgfqpoint{2.665646in}{1.092741in}}{\pgfqpoint{2.676696in}{1.092741in}}%
\pgfpathclose%
\pgfusepath{stroke,fill}%
\end{pgfscope}%
\begin{pgfscope}%
\pgfpathrectangle{\pgfqpoint{0.787074in}{0.548769in}}{\pgfqpoint{5.062926in}{3.102590in}}%
\pgfusepath{clip}%
\pgfsetbuttcap%
\pgfsetroundjoin%
\definecolor{currentfill}{rgb}{1.000000,0.498039,0.054902}%
\pgfsetfillcolor{currentfill}%
\pgfsetlinewidth{1.003750pt}%
\definecolor{currentstroke}{rgb}{1.000000,0.498039,0.054902}%
\pgfsetstrokecolor{currentstroke}%
\pgfsetdash{}{0pt}%
\pgfpathmoveto{\pgfqpoint{1.848016in}{2.255576in}}%
\pgfpathcurveto{\pgfqpoint{1.859066in}{2.255576in}}{\pgfqpoint{1.869665in}{2.259966in}}{\pgfqpoint{1.877478in}{2.267780in}}%
\pgfpathcurveto{\pgfqpoint{1.885292in}{2.275593in}}{\pgfqpoint{1.889682in}{2.286192in}}{\pgfqpoint{1.889682in}{2.297243in}}%
\pgfpathcurveto{\pgfqpoint{1.889682in}{2.308293in}}{\pgfqpoint{1.885292in}{2.318892in}}{\pgfqpoint{1.877478in}{2.326705in}}%
\pgfpathcurveto{\pgfqpoint{1.869665in}{2.334519in}}{\pgfqpoint{1.859066in}{2.338909in}}{\pgfqpoint{1.848016in}{2.338909in}}%
\pgfpathcurveto{\pgfqpoint{1.836966in}{2.338909in}}{\pgfqpoint{1.826366in}{2.334519in}}{\pgfqpoint{1.818553in}{2.326705in}}%
\pgfpathcurveto{\pgfqpoint{1.810739in}{2.318892in}}{\pgfqpoint{1.806349in}{2.308293in}}{\pgfqpoint{1.806349in}{2.297243in}}%
\pgfpathcurveto{\pgfqpoint{1.806349in}{2.286192in}}{\pgfqpoint{1.810739in}{2.275593in}}{\pgfqpoint{1.818553in}{2.267780in}}%
\pgfpathcurveto{\pgfqpoint{1.826366in}{2.259966in}}{\pgfqpoint{1.836966in}{2.255576in}}{\pgfqpoint{1.848016in}{2.255576in}}%
\pgfpathclose%
\pgfusepath{stroke,fill}%
\end{pgfscope}%
\begin{pgfscope}%
\pgfsetbuttcap%
\pgfsetroundjoin%
\definecolor{currentfill}{rgb}{0.000000,0.000000,0.000000}%
\pgfsetfillcolor{currentfill}%
\pgfsetlinewidth{0.803000pt}%
\definecolor{currentstroke}{rgb}{0.000000,0.000000,0.000000}%
\pgfsetstrokecolor{currentstroke}%
\pgfsetdash{}{0pt}%
\pgfsys@defobject{currentmarker}{\pgfqpoint{0.000000in}{-0.048611in}}{\pgfqpoint{0.000000in}{0.000000in}}{%
\pgfpathmoveto{\pgfqpoint{0.000000in}{0.000000in}}%
\pgfpathlineto{\pgfqpoint{0.000000in}{-0.048611in}}%
\pgfusepath{stroke,fill}%
}%
\begin{pgfscope}%
\pgfsys@transformshift{0.981384in}{0.548769in}%
\pgfsys@useobject{currentmarker}{}%
\end{pgfscope}%
\end{pgfscope}%
\begin{pgfscope}%
\definecolor{textcolor}{rgb}{0.000000,0.000000,0.000000}%
\pgfsetstrokecolor{textcolor}%
\pgfsetfillcolor{textcolor}%
\pgftext[x=0.981384in,y=0.451547in,,top]{\color{textcolor}\sffamily\fontsize{10.000000}{12.000000}\selectfont \(\displaystyle {0}\)}%
\end{pgfscope}%
\begin{pgfscope}%
\pgfsetbuttcap%
\pgfsetroundjoin%
\definecolor{currentfill}{rgb}{0.000000,0.000000,0.000000}%
\pgfsetfillcolor{currentfill}%
\pgfsetlinewidth{0.803000pt}%
\definecolor{currentstroke}{rgb}{0.000000,0.000000,0.000000}%
\pgfsetstrokecolor{currentstroke}%
\pgfsetdash{}{0pt}%
\pgfsys@defobject{currentmarker}{\pgfqpoint{0.000000in}{-0.048611in}}{\pgfqpoint{0.000000in}{0.000000in}}{%
\pgfpathmoveto{\pgfqpoint{0.000000in}{0.000000in}}%
\pgfpathlineto{\pgfqpoint{0.000000in}{-0.048611in}}%
\pgfusepath{stroke,fill}%
}%
\begin{pgfscope}%
\pgfsys@transformshift{1.829269in}{0.548769in}%
\pgfsys@useobject{currentmarker}{}%
\end{pgfscope}%
\end{pgfscope}%
\begin{pgfscope}%
\definecolor{textcolor}{rgb}{0.000000,0.000000,0.000000}%
\pgfsetstrokecolor{textcolor}%
\pgfsetfillcolor{textcolor}%
\pgftext[x=1.829269in,y=0.451547in,,top]{\color{textcolor}\sffamily\fontsize{10.000000}{12.000000}\selectfont \(\displaystyle {100000}\)}%
\end{pgfscope}%
\begin{pgfscope}%
\pgfsetbuttcap%
\pgfsetroundjoin%
\definecolor{currentfill}{rgb}{0.000000,0.000000,0.000000}%
\pgfsetfillcolor{currentfill}%
\pgfsetlinewidth{0.803000pt}%
\definecolor{currentstroke}{rgb}{0.000000,0.000000,0.000000}%
\pgfsetstrokecolor{currentstroke}%
\pgfsetdash{}{0pt}%
\pgfsys@defobject{currentmarker}{\pgfqpoint{0.000000in}{-0.048611in}}{\pgfqpoint{0.000000in}{0.000000in}}{%
\pgfpathmoveto{\pgfqpoint{0.000000in}{0.000000in}}%
\pgfpathlineto{\pgfqpoint{0.000000in}{-0.048611in}}%
\pgfusepath{stroke,fill}%
}%
\begin{pgfscope}%
\pgfsys@transformshift{2.677154in}{0.548769in}%
\pgfsys@useobject{currentmarker}{}%
\end{pgfscope}%
\end{pgfscope}%
\begin{pgfscope}%
\definecolor{textcolor}{rgb}{0.000000,0.000000,0.000000}%
\pgfsetstrokecolor{textcolor}%
\pgfsetfillcolor{textcolor}%
\pgftext[x=2.677154in,y=0.451547in,,top]{\color{textcolor}\sffamily\fontsize{10.000000}{12.000000}\selectfont \(\displaystyle {200000}\)}%
\end{pgfscope}%
\begin{pgfscope}%
\pgfsetbuttcap%
\pgfsetroundjoin%
\definecolor{currentfill}{rgb}{0.000000,0.000000,0.000000}%
\pgfsetfillcolor{currentfill}%
\pgfsetlinewidth{0.803000pt}%
\definecolor{currentstroke}{rgb}{0.000000,0.000000,0.000000}%
\pgfsetstrokecolor{currentstroke}%
\pgfsetdash{}{0pt}%
\pgfsys@defobject{currentmarker}{\pgfqpoint{0.000000in}{-0.048611in}}{\pgfqpoint{0.000000in}{0.000000in}}{%
\pgfpathmoveto{\pgfqpoint{0.000000in}{0.000000in}}%
\pgfpathlineto{\pgfqpoint{0.000000in}{-0.048611in}}%
\pgfusepath{stroke,fill}%
}%
\begin{pgfscope}%
\pgfsys@transformshift{3.525039in}{0.548769in}%
\pgfsys@useobject{currentmarker}{}%
\end{pgfscope}%
\end{pgfscope}%
\begin{pgfscope}%
\definecolor{textcolor}{rgb}{0.000000,0.000000,0.000000}%
\pgfsetstrokecolor{textcolor}%
\pgfsetfillcolor{textcolor}%
\pgftext[x=3.525039in,y=0.451547in,,top]{\color{textcolor}\sffamily\fontsize{10.000000}{12.000000}\selectfont \(\displaystyle {300000}\)}%
\end{pgfscope}%
\begin{pgfscope}%
\pgfsetbuttcap%
\pgfsetroundjoin%
\definecolor{currentfill}{rgb}{0.000000,0.000000,0.000000}%
\pgfsetfillcolor{currentfill}%
\pgfsetlinewidth{0.803000pt}%
\definecolor{currentstroke}{rgb}{0.000000,0.000000,0.000000}%
\pgfsetstrokecolor{currentstroke}%
\pgfsetdash{}{0pt}%
\pgfsys@defobject{currentmarker}{\pgfqpoint{0.000000in}{-0.048611in}}{\pgfqpoint{0.000000in}{0.000000in}}{%
\pgfpathmoveto{\pgfqpoint{0.000000in}{0.000000in}}%
\pgfpathlineto{\pgfqpoint{0.000000in}{-0.048611in}}%
\pgfusepath{stroke,fill}%
}%
\begin{pgfscope}%
\pgfsys@transformshift{4.372925in}{0.548769in}%
\pgfsys@useobject{currentmarker}{}%
\end{pgfscope}%
\end{pgfscope}%
\begin{pgfscope}%
\definecolor{textcolor}{rgb}{0.000000,0.000000,0.000000}%
\pgfsetstrokecolor{textcolor}%
\pgfsetfillcolor{textcolor}%
\pgftext[x=4.372925in,y=0.451547in,,top]{\color{textcolor}\sffamily\fontsize{10.000000}{12.000000}\selectfont \(\displaystyle {400000}\)}%
\end{pgfscope}%
\begin{pgfscope}%
\pgfsetbuttcap%
\pgfsetroundjoin%
\definecolor{currentfill}{rgb}{0.000000,0.000000,0.000000}%
\pgfsetfillcolor{currentfill}%
\pgfsetlinewidth{0.803000pt}%
\definecolor{currentstroke}{rgb}{0.000000,0.000000,0.000000}%
\pgfsetstrokecolor{currentstroke}%
\pgfsetdash{}{0pt}%
\pgfsys@defobject{currentmarker}{\pgfqpoint{0.000000in}{-0.048611in}}{\pgfqpoint{0.000000in}{0.000000in}}{%
\pgfpathmoveto{\pgfqpoint{0.000000in}{0.000000in}}%
\pgfpathlineto{\pgfqpoint{0.000000in}{-0.048611in}}%
\pgfusepath{stroke,fill}%
}%
\begin{pgfscope}%
\pgfsys@transformshift{5.220810in}{0.548769in}%
\pgfsys@useobject{currentmarker}{}%
\end{pgfscope}%
\end{pgfscope}%
\begin{pgfscope}%
\definecolor{textcolor}{rgb}{0.000000,0.000000,0.000000}%
\pgfsetstrokecolor{textcolor}%
\pgfsetfillcolor{textcolor}%
\pgftext[x=5.220810in,y=0.451547in,,top]{\color{textcolor}\sffamily\fontsize{10.000000}{12.000000}\selectfont \(\displaystyle {500000}\)}%
\end{pgfscope}%
\begin{pgfscope}%
\definecolor{textcolor}{rgb}{0.000000,0.000000,0.000000}%
\pgfsetstrokecolor{textcolor}%
\pgfsetfillcolor{textcolor}%
\pgftext[x=3.318537in,y=0.272658in,,top]{\color{textcolor}\sffamily\fontsize{10.000000}{12.000000}\selectfont Methods}%
\end{pgfscope}%
\begin{pgfscope}%
\pgfsetbuttcap%
\pgfsetroundjoin%
\definecolor{currentfill}{rgb}{0.000000,0.000000,0.000000}%
\pgfsetfillcolor{currentfill}%
\pgfsetlinewidth{0.803000pt}%
\definecolor{currentstroke}{rgb}{0.000000,0.000000,0.000000}%
\pgfsetstrokecolor{currentstroke}%
\pgfsetdash{}{0pt}%
\pgfsys@defobject{currentmarker}{\pgfqpoint{-0.048611in}{0.000000in}}{\pgfqpoint{0.000000in}{0.000000in}}{%
\pgfpathmoveto{\pgfqpoint{0.000000in}{0.000000in}}%
\pgfpathlineto{\pgfqpoint{-0.048611in}{0.000000in}}%
\pgfusepath{stroke,fill}%
}%
\begin{pgfscope}%
\pgfsys@transformshift{0.787074in}{0.689795in}%
\pgfsys@useobject{currentmarker}{}%
\end{pgfscope}%
\end{pgfscope}%
\begin{pgfscope}%
\definecolor{textcolor}{rgb}{0.000000,0.000000,0.000000}%
\pgfsetstrokecolor{textcolor}%
\pgfsetfillcolor{textcolor}%
\pgftext[x=0.620407in, y=0.641601in, left, base]{\color{textcolor}\sffamily\fontsize{10.000000}{12.000000}\selectfont \(\displaystyle {0}\)}%
\end{pgfscope}%
\begin{pgfscope}%
\pgfsetbuttcap%
\pgfsetroundjoin%
\definecolor{currentfill}{rgb}{0.000000,0.000000,0.000000}%
\pgfsetfillcolor{currentfill}%
\pgfsetlinewidth{0.803000pt}%
\definecolor{currentstroke}{rgb}{0.000000,0.000000,0.000000}%
\pgfsetstrokecolor{currentstroke}%
\pgfsetdash{}{0pt}%
\pgfsys@defobject{currentmarker}{\pgfqpoint{-0.048611in}{0.000000in}}{\pgfqpoint{0.000000in}{0.000000in}}{%
\pgfpathmoveto{\pgfqpoint{0.000000in}{0.000000in}}%
\pgfpathlineto{\pgfqpoint{-0.048611in}{0.000000in}}%
\pgfusepath{stroke,fill}%
}%
\begin{pgfscope}%
\pgfsys@transformshift{0.787074in}{1.062854in}%
\pgfsys@useobject{currentmarker}{}%
\end{pgfscope}%
\end{pgfscope}%
\begin{pgfscope}%
\definecolor{textcolor}{rgb}{0.000000,0.000000,0.000000}%
\pgfsetstrokecolor{textcolor}%
\pgfsetfillcolor{textcolor}%
\pgftext[x=0.412073in, y=1.014659in, left, base]{\color{textcolor}\sffamily\fontsize{10.000000}{12.000000}\selectfont \(\displaystyle {2500}\)}%
\end{pgfscope}%
\begin{pgfscope}%
\pgfsetbuttcap%
\pgfsetroundjoin%
\definecolor{currentfill}{rgb}{0.000000,0.000000,0.000000}%
\pgfsetfillcolor{currentfill}%
\pgfsetlinewidth{0.803000pt}%
\definecolor{currentstroke}{rgb}{0.000000,0.000000,0.000000}%
\pgfsetstrokecolor{currentstroke}%
\pgfsetdash{}{0pt}%
\pgfsys@defobject{currentmarker}{\pgfqpoint{-0.048611in}{0.000000in}}{\pgfqpoint{0.000000in}{0.000000in}}{%
\pgfpathmoveto{\pgfqpoint{0.000000in}{0.000000in}}%
\pgfpathlineto{\pgfqpoint{-0.048611in}{0.000000in}}%
\pgfusepath{stroke,fill}%
}%
\begin{pgfscope}%
\pgfsys@transformshift{0.787074in}{1.435912in}%
\pgfsys@useobject{currentmarker}{}%
\end{pgfscope}%
\end{pgfscope}%
\begin{pgfscope}%
\definecolor{textcolor}{rgb}{0.000000,0.000000,0.000000}%
\pgfsetstrokecolor{textcolor}%
\pgfsetfillcolor{textcolor}%
\pgftext[x=0.412073in, y=1.387718in, left, base]{\color{textcolor}\sffamily\fontsize{10.000000}{12.000000}\selectfont \(\displaystyle {5000}\)}%
\end{pgfscope}%
\begin{pgfscope}%
\pgfsetbuttcap%
\pgfsetroundjoin%
\definecolor{currentfill}{rgb}{0.000000,0.000000,0.000000}%
\pgfsetfillcolor{currentfill}%
\pgfsetlinewidth{0.803000pt}%
\definecolor{currentstroke}{rgb}{0.000000,0.000000,0.000000}%
\pgfsetstrokecolor{currentstroke}%
\pgfsetdash{}{0pt}%
\pgfsys@defobject{currentmarker}{\pgfqpoint{-0.048611in}{0.000000in}}{\pgfqpoint{0.000000in}{0.000000in}}{%
\pgfpathmoveto{\pgfqpoint{0.000000in}{0.000000in}}%
\pgfpathlineto{\pgfqpoint{-0.048611in}{0.000000in}}%
\pgfusepath{stroke,fill}%
}%
\begin{pgfscope}%
\pgfsys@transformshift{0.787074in}{1.808971in}%
\pgfsys@useobject{currentmarker}{}%
\end{pgfscope}%
\end{pgfscope}%
\begin{pgfscope}%
\definecolor{textcolor}{rgb}{0.000000,0.000000,0.000000}%
\pgfsetstrokecolor{textcolor}%
\pgfsetfillcolor{textcolor}%
\pgftext[x=0.412073in, y=1.760776in, left, base]{\color{textcolor}\sffamily\fontsize{10.000000}{12.000000}\selectfont \(\displaystyle {7500}\)}%
\end{pgfscope}%
\begin{pgfscope}%
\pgfsetbuttcap%
\pgfsetroundjoin%
\definecolor{currentfill}{rgb}{0.000000,0.000000,0.000000}%
\pgfsetfillcolor{currentfill}%
\pgfsetlinewidth{0.803000pt}%
\definecolor{currentstroke}{rgb}{0.000000,0.000000,0.000000}%
\pgfsetstrokecolor{currentstroke}%
\pgfsetdash{}{0pt}%
\pgfsys@defobject{currentmarker}{\pgfqpoint{-0.048611in}{0.000000in}}{\pgfqpoint{0.000000in}{0.000000in}}{%
\pgfpathmoveto{\pgfqpoint{0.000000in}{0.000000in}}%
\pgfpathlineto{\pgfqpoint{-0.048611in}{0.000000in}}%
\pgfusepath{stroke,fill}%
}%
\begin{pgfscope}%
\pgfsys@transformshift{0.787074in}{2.182029in}%
\pgfsys@useobject{currentmarker}{}%
\end{pgfscope}%
\end{pgfscope}%
\begin{pgfscope}%
\definecolor{textcolor}{rgb}{0.000000,0.000000,0.000000}%
\pgfsetstrokecolor{textcolor}%
\pgfsetfillcolor{textcolor}%
\pgftext[x=0.342628in, y=2.133835in, left, base]{\color{textcolor}\sffamily\fontsize{10.000000}{12.000000}\selectfont \(\displaystyle {10000}\)}%
\end{pgfscope}%
\begin{pgfscope}%
\pgfsetbuttcap%
\pgfsetroundjoin%
\definecolor{currentfill}{rgb}{0.000000,0.000000,0.000000}%
\pgfsetfillcolor{currentfill}%
\pgfsetlinewidth{0.803000pt}%
\definecolor{currentstroke}{rgb}{0.000000,0.000000,0.000000}%
\pgfsetstrokecolor{currentstroke}%
\pgfsetdash{}{0pt}%
\pgfsys@defobject{currentmarker}{\pgfqpoint{-0.048611in}{0.000000in}}{\pgfqpoint{0.000000in}{0.000000in}}{%
\pgfpathmoveto{\pgfqpoint{0.000000in}{0.000000in}}%
\pgfpathlineto{\pgfqpoint{-0.048611in}{0.000000in}}%
\pgfusepath{stroke,fill}%
}%
\begin{pgfscope}%
\pgfsys@transformshift{0.787074in}{2.555088in}%
\pgfsys@useobject{currentmarker}{}%
\end{pgfscope}%
\end{pgfscope}%
\begin{pgfscope}%
\definecolor{textcolor}{rgb}{0.000000,0.000000,0.000000}%
\pgfsetstrokecolor{textcolor}%
\pgfsetfillcolor{textcolor}%
\pgftext[x=0.342628in, y=2.506893in, left, base]{\color{textcolor}\sffamily\fontsize{10.000000}{12.000000}\selectfont \(\displaystyle {12500}\)}%
\end{pgfscope}%
\begin{pgfscope}%
\pgfsetbuttcap%
\pgfsetroundjoin%
\definecolor{currentfill}{rgb}{0.000000,0.000000,0.000000}%
\pgfsetfillcolor{currentfill}%
\pgfsetlinewidth{0.803000pt}%
\definecolor{currentstroke}{rgb}{0.000000,0.000000,0.000000}%
\pgfsetstrokecolor{currentstroke}%
\pgfsetdash{}{0pt}%
\pgfsys@defobject{currentmarker}{\pgfqpoint{-0.048611in}{0.000000in}}{\pgfqpoint{0.000000in}{0.000000in}}{%
\pgfpathmoveto{\pgfqpoint{0.000000in}{0.000000in}}%
\pgfpathlineto{\pgfqpoint{-0.048611in}{0.000000in}}%
\pgfusepath{stroke,fill}%
}%
\begin{pgfscope}%
\pgfsys@transformshift{0.787074in}{2.928146in}%
\pgfsys@useobject{currentmarker}{}%
\end{pgfscope}%
\end{pgfscope}%
\begin{pgfscope}%
\definecolor{textcolor}{rgb}{0.000000,0.000000,0.000000}%
\pgfsetstrokecolor{textcolor}%
\pgfsetfillcolor{textcolor}%
\pgftext[x=0.342628in, y=2.879952in, left, base]{\color{textcolor}\sffamily\fontsize{10.000000}{12.000000}\selectfont \(\displaystyle {15000}\)}%
\end{pgfscope}%
\begin{pgfscope}%
\pgfsetbuttcap%
\pgfsetroundjoin%
\definecolor{currentfill}{rgb}{0.000000,0.000000,0.000000}%
\pgfsetfillcolor{currentfill}%
\pgfsetlinewidth{0.803000pt}%
\definecolor{currentstroke}{rgb}{0.000000,0.000000,0.000000}%
\pgfsetstrokecolor{currentstroke}%
\pgfsetdash{}{0pt}%
\pgfsys@defobject{currentmarker}{\pgfqpoint{-0.048611in}{0.000000in}}{\pgfqpoint{0.000000in}{0.000000in}}{%
\pgfpathmoveto{\pgfqpoint{0.000000in}{0.000000in}}%
\pgfpathlineto{\pgfqpoint{-0.048611in}{0.000000in}}%
\pgfusepath{stroke,fill}%
}%
\begin{pgfscope}%
\pgfsys@transformshift{0.787074in}{3.301204in}%
\pgfsys@useobject{currentmarker}{}%
\end{pgfscope}%
\end{pgfscope}%
\begin{pgfscope}%
\definecolor{textcolor}{rgb}{0.000000,0.000000,0.000000}%
\pgfsetstrokecolor{textcolor}%
\pgfsetfillcolor{textcolor}%
\pgftext[x=0.342628in, y=3.253010in, left, base]{\color{textcolor}\sffamily\fontsize{10.000000}{12.000000}\selectfont \(\displaystyle {17500}\)}%
\end{pgfscope}%
\begin{pgfscope}%
\definecolor{textcolor}{rgb}{0.000000,0.000000,0.000000}%
\pgfsetstrokecolor{textcolor}%
\pgfsetfillcolor{textcolor}%
\pgftext[x=0.287073in,y=2.100064in,,bottom,rotate=90.000000]{\color{textcolor}\sffamily\fontsize{10.000000}{12.000000}\selectfont Maximum Memory Consumption (MB)}%
\end{pgfscope}%
\begin{pgfscope}%
\pgfsetrectcap%
\pgfsetmiterjoin%
\pgfsetlinewidth{0.803000pt}%
\definecolor{currentstroke}{rgb}{0.000000,0.000000,0.000000}%
\pgfsetstrokecolor{currentstroke}%
\pgfsetdash{}{0pt}%
\pgfpathmoveto{\pgfqpoint{0.787074in}{0.548769in}}%
\pgfpathlineto{\pgfqpoint{0.787074in}{3.651359in}}%
\pgfusepath{stroke}%
\end{pgfscope}%
\begin{pgfscope}%
\pgfsetrectcap%
\pgfsetmiterjoin%
\pgfsetlinewidth{0.803000pt}%
\definecolor{currentstroke}{rgb}{0.000000,0.000000,0.000000}%
\pgfsetstrokecolor{currentstroke}%
\pgfsetdash{}{0pt}%
\pgfpathmoveto{\pgfqpoint{5.850000in}{0.548769in}}%
\pgfpathlineto{\pgfqpoint{5.850000in}{3.651359in}}%
\pgfusepath{stroke}%
\end{pgfscope}%
\begin{pgfscope}%
\pgfsetrectcap%
\pgfsetmiterjoin%
\pgfsetlinewidth{0.803000pt}%
\definecolor{currentstroke}{rgb}{0.000000,0.000000,0.000000}%
\pgfsetstrokecolor{currentstroke}%
\pgfsetdash{}{0pt}%
\pgfpathmoveto{\pgfqpoint{0.787074in}{0.548769in}}%
\pgfpathlineto{\pgfqpoint{5.850000in}{0.548769in}}%
\pgfusepath{stroke}%
\end{pgfscope}%
\begin{pgfscope}%
\pgfsetrectcap%
\pgfsetmiterjoin%
\pgfsetlinewidth{0.803000pt}%
\definecolor{currentstroke}{rgb}{0.000000,0.000000,0.000000}%
\pgfsetstrokecolor{currentstroke}%
\pgfsetdash{}{0pt}%
\pgfpathmoveto{\pgfqpoint{0.787074in}{3.651359in}}%
\pgfpathlineto{\pgfqpoint{5.850000in}{3.651359in}}%
\pgfusepath{stroke}%
\end{pgfscope}%
\begin{pgfscope}%
\definecolor{textcolor}{rgb}{0.000000,0.000000,0.000000}%
\pgfsetstrokecolor{textcolor}%
\pgfsetfillcolor{textcolor}%
\pgftext[x=3.318537in,y=3.734692in,,base]{\color{textcolor}\sffamily\fontsize{12.000000}{14.400000}\selectfont Backward}%
\end{pgfscope}%
\begin{pgfscope}%
\pgfsetbuttcap%
\pgfsetmiterjoin%
\definecolor{currentfill}{rgb}{1.000000,1.000000,1.000000}%
\pgfsetfillcolor{currentfill}%
\pgfsetfillopacity{0.800000}%
\pgfsetlinewidth{1.003750pt}%
\definecolor{currentstroke}{rgb}{0.800000,0.800000,0.800000}%
\pgfsetstrokecolor{currentstroke}%
\pgfsetstrokeopacity{0.800000}%
\pgfsetdash{}{0pt}%
\pgfpathmoveto{\pgfqpoint{4.300417in}{2.957886in}}%
\pgfpathlineto{\pgfqpoint{5.752778in}{2.957886in}}%
\pgfpathquadraticcurveto{\pgfqpoint{5.780556in}{2.957886in}}{\pgfqpoint{5.780556in}{2.985664in}}%
\pgfpathlineto{\pgfqpoint{5.780556in}{3.554136in}}%
\pgfpathquadraticcurveto{\pgfqpoint{5.780556in}{3.581914in}}{\pgfqpoint{5.752778in}{3.581914in}}%
\pgfpathlineto{\pgfqpoint{4.300417in}{3.581914in}}%
\pgfpathquadraticcurveto{\pgfqpoint{4.272639in}{3.581914in}}{\pgfqpoint{4.272639in}{3.554136in}}%
\pgfpathlineto{\pgfqpoint{4.272639in}{2.985664in}}%
\pgfpathquadraticcurveto{\pgfqpoint{4.272639in}{2.957886in}}{\pgfqpoint{4.300417in}{2.957886in}}%
\pgfpathclose%
\pgfusepath{stroke,fill}%
\end{pgfscope}%
\begin{pgfscope}%
\pgfsetbuttcap%
\pgfsetroundjoin%
\definecolor{currentfill}{rgb}{0.121569,0.466667,0.705882}%
\pgfsetfillcolor{currentfill}%
\pgfsetlinewidth{1.003750pt}%
\definecolor{currentstroke}{rgb}{0.121569,0.466667,0.705882}%
\pgfsetstrokecolor{currentstroke}%
\pgfsetdash{}{0pt}%
\pgfsys@defobject{currentmarker}{\pgfqpoint{-0.034722in}{-0.034722in}}{\pgfqpoint{0.034722in}{0.034722in}}{%
\pgfpathmoveto{\pgfqpoint{0.000000in}{-0.034722in}}%
\pgfpathcurveto{\pgfqpoint{0.009208in}{-0.034722in}}{\pgfqpoint{0.018041in}{-0.031064in}}{\pgfqpoint{0.024552in}{-0.024552in}}%
\pgfpathcurveto{\pgfqpoint{0.031064in}{-0.018041in}}{\pgfqpoint{0.034722in}{-0.009208in}}{\pgfqpoint{0.034722in}{0.000000in}}%
\pgfpathcurveto{\pgfqpoint{0.034722in}{0.009208in}}{\pgfqpoint{0.031064in}{0.018041in}}{\pgfqpoint{0.024552in}{0.024552in}}%
\pgfpathcurveto{\pgfqpoint{0.018041in}{0.031064in}}{\pgfqpoint{0.009208in}{0.034722in}}{\pgfqpoint{0.000000in}{0.034722in}}%
\pgfpathcurveto{\pgfqpoint{-0.009208in}{0.034722in}}{\pgfqpoint{-0.018041in}{0.031064in}}{\pgfqpoint{-0.024552in}{0.024552in}}%
\pgfpathcurveto{\pgfqpoint{-0.031064in}{0.018041in}}{\pgfqpoint{-0.034722in}{0.009208in}}{\pgfqpoint{-0.034722in}{0.000000in}}%
\pgfpathcurveto{\pgfqpoint{-0.034722in}{-0.009208in}}{\pgfqpoint{-0.031064in}{-0.018041in}}{\pgfqpoint{-0.024552in}{-0.024552in}}%
\pgfpathcurveto{\pgfqpoint{-0.018041in}{-0.031064in}}{\pgfqpoint{-0.009208in}{-0.034722in}}{\pgfqpoint{0.000000in}{-0.034722in}}%
\pgfpathclose%
\pgfusepath{stroke,fill}%
}%
\begin{pgfscope}%
\pgfsys@transformshift{4.467083in}{3.477748in}%
\pgfsys@useobject{currentmarker}{}%
\end{pgfscope}%
\end{pgfscope}%
\begin{pgfscope}%
\definecolor{textcolor}{rgb}{0.000000,0.000000,0.000000}%
\pgfsetstrokecolor{textcolor}%
\pgfsetfillcolor{textcolor}%
\pgftext[x=4.717083in,y=3.429136in,left,base]{\color{textcolor}\sffamily\fontsize{10.000000}{12.000000}\selectfont No Timeout}%
\end{pgfscope}%
\begin{pgfscope}%
\pgfsetbuttcap%
\pgfsetroundjoin%
\definecolor{currentfill}{rgb}{1.000000,0.498039,0.054902}%
\pgfsetfillcolor{currentfill}%
\pgfsetlinewidth{1.003750pt}%
\definecolor{currentstroke}{rgb}{1.000000,0.498039,0.054902}%
\pgfsetstrokecolor{currentstroke}%
\pgfsetdash{}{0pt}%
\pgfsys@defobject{currentmarker}{\pgfqpoint{-0.034722in}{-0.034722in}}{\pgfqpoint{0.034722in}{0.034722in}}{%
\pgfpathmoveto{\pgfqpoint{0.000000in}{-0.034722in}}%
\pgfpathcurveto{\pgfqpoint{0.009208in}{-0.034722in}}{\pgfqpoint{0.018041in}{-0.031064in}}{\pgfqpoint{0.024552in}{-0.024552in}}%
\pgfpathcurveto{\pgfqpoint{0.031064in}{-0.018041in}}{\pgfqpoint{0.034722in}{-0.009208in}}{\pgfqpoint{0.034722in}{0.000000in}}%
\pgfpathcurveto{\pgfqpoint{0.034722in}{0.009208in}}{\pgfqpoint{0.031064in}{0.018041in}}{\pgfqpoint{0.024552in}{0.024552in}}%
\pgfpathcurveto{\pgfqpoint{0.018041in}{0.031064in}}{\pgfqpoint{0.009208in}{0.034722in}}{\pgfqpoint{0.000000in}{0.034722in}}%
\pgfpathcurveto{\pgfqpoint{-0.009208in}{0.034722in}}{\pgfqpoint{-0.018041in}{0.031064in}}{\pgfqpoint{-0.024552in}{0.024552in}}%
\pgfpathcurveto{\pgfqpoint{-0.031064in}{0.018041in}}{\pgfqpoint{-0.034722in}{0.009208in}}{\pgfqpoint{-0.034722in}{0.000000in}}%
\pgfpathcurveto{\pgfqpoint{-0.034722in}{-0.009208in}}{\pgfqpoint{-0.031064in}{-0.018041in}}{\pgfqpoint{-0.024552in}{-0.024552in}}%
\pgfpathcurveto{\pgfqpoint{-0.018041in}{-0.031064in}}{\pgfqpoint{-0.009208in}{-0.034722in}}{\pgfqpoint{0.000000in}{-0.034722in}}%
\pgfpathclose%
\pgfusepath{stroke,fill}%
}%
\begin{pgfscope}%
\pgfsys@transformshift{4.467083in}{3.284136in}%
\pgfsys@useobject{currentmarker}{}%
\end{pgfscope}%
\end{pgfscope}%
\begin{pgfscope}%
\definecolor{textcolor}{rgb}{0.000000,0.000000,0.000000}%
\pgfsetstrokecolor{textcolor}%
\pgfsetfillcolor{textcolor}%
\pgftext[x=4.717083in,y=3.235525in,left,base]{\color{textcolor}\sffamily\fontsize{10.000000}{12.000000}\selectfont Time Timeout}%
\end{pgfscope}%
\begin{pgfscope}%
\pgfsetbuttcap%
\pgfsetroundjoin%
\definecolor{currentfill}{rgb}{0.839216,0.152941,0.156863}%
\pgfsetfillcolor{currentfill}%
\pgfsetlinewidth{1.003750pt}%
\definecolor{currentstroke}{rgb}{0.839216,0.152941,0.156863}%
\pgfsetstrokecolor{currentstroke}%
\pgfsetdash{}{0pt}%
\pgfsys@defobject{currentmarker}{\pgfqpoint{-0.034722in}{-0.034722in}}{\pgfqpoint{0.034722in}{0.034722in}}{%
\pgfpathmoveto{\pgfqpoint{0.000000in}{-0.034722in}}%
\pgfpathcurveto{\pgfqpoint{0.009208in}{-0.034722in}}{\pgfqpoint{0.018041in}{-0.031064in}}{\pgfqpoint{0.024552in}{-0.024552in}}%
\pgfpathcurveto{\pgfqpoint{0.031064in}{-0.018041in}}{\pgfqpoint{0.034722in}{-0.009208in}}{\pgfqpoint{0.034722in}{0.000000in}}%
\pgfpathcurveto{\pgfqpoint{0.034722in}{0.009208in}}{\pgfqpoint{0.031064in}{0.018041in}}{\pgfqpoint{0.024552in}{0.024552in}}%
\pgfpathcurveto{\pgfqpoint{0.018041in}{0.031064in}}{\pgfqpoint{0.009208in}{0.034722in}}{\pgfqpoint{0.000000in}{0.034722in}}%
\pgfpathcurveto{\pgfqpoint{-0.009208in}{0.034722in}}{\pgfqpoint{-0.018041in}{0.031064in}}{\pgfqpoint{-0.024552in}{0.024552in}}%
\pgfpathcurveto{\pgfqpoint{-0.031064in}{0.018041in}}{\pgfqpoint{-0.034722in}{0.009208in}}{\pgfqpoint{-0.034722in}{0.000000in}}%
\pgfpathcurveto{\pgfqpoint{-0.034722in}{-0.009208in}}{\pgfqpoint{-0.031064in}{-0.018041in}}{\pgfqpoint{-0.024552in}{-0.024552in}}%
\pgfpathcurveto{\pgfqpoint{-0.018041in}{-0.031064in}}{\pgfqpoint{-0.009208in}{-0.034722in}}{\pgfqpoint{0.000000in}{-0.034722in}}%
\pgfpathclose%
\pgfusepath{stroke,fill}%
}%
\begin{pgfscope}%
\pgfsys@transformshift{4.467083in}{3.090525in}%
\pgfsys@useobject{currentmarker}{}%
\end{pgfscope}%
\end{pgfscope}%
\begin{pgfscope}%
\definecolor{textcolor}{rgb}{0.000000,0.000000,0.000000}%
\pgfsetstrokecolor{textcolor}%
\pgfsetfillcolor{textcolor}%
\pgftext[x=4.717083in,y=3.041914in,left,base]{\color{textcolor}\sffamily\fontsize{10.000000}{12.000000}\selectfont Memory Timeout}%
\end{pgfscope}%
\end{pgfpicture}%
\makeatother%
\endgroup%

                }
            \end{subfigure}
            \caption{Methods}
        \end{subfigure}
        \bigbreak
        \begin{subfigure}[b]{\textwidth}
            \centering
            \begin{subfigure}[]{0.45\textwidth}
                \centering
                \resizebox{\columnwidth}{!}{
                    %% Creator: Matplotlib, PGF backend
%%
%% To include the figure in your LaTeX document, write
%%   \input{<filename>.pgf}
%%
%% Make sure the required packages are loaded in your preamble
%%   \usepackage{pgf}
%%
%% and, on pdftex
%%   \usepackage[utf8]{inputenc}\DeclareUnicodeCharacter{2212}{-}
%%
%% or, on luatex and xetex
%%   \usepackage{unicode-math}
%%
%% Figures using additional raster images can only be included by \input if
%% they are in the same directory as the main LaTeX file. For loading figures
%% from other directories you can use the `import` package
%%   \usepackage{import}
%%
%% and then include the figures with
%%   \import{<path to file>}{<filename>.pgf}
%%
%% Matplotlib used the following preamble
%%   \usepackage{amsmath}
%%   \usepackage{fontspec}
%%
\begingroup%
\makeatletter%
\begin{pgfpicture}%
\pgfpathrectangle{\pgfpointorigin}{\pgfqpoint{6.000000in}{4.000000in}}%
\pgfusepath{use as bounding box, clip}%
\begin{pgfscope}%
\pgfsetbuttcap%
\pgfsetmiterjoin%
\definecolor{currentfill}{rgb}{1.000000,1.000000,1.000000}%
\pgfsetfillcolor{currentfill}%
\pgfsetlinewidth{0.000000pt}%
\definecolor{currentstroke}{rgb}{1.000000,1.000000,1.000000}%
\pgfsetstrokecolor{currentstroke}%
\pgfsetdash{}{0pt}%
\pgfpathmoveto{\pgfqpoint{0.000000in}{0.000000in}}%
\pgfpathlineto{\pgfqpoint{6.000000in}{0.000000in}}%
\pgfpathlineto{\pgfqpoint{6.000000in}{4.000000in}}%
\pgfpathlineto{\pgfqpoint{0.000000in}{4.000000in}}%
\pgfpathclose%
\pgfusepath{fill}%
\end{pgfscope}%
\begin{pgfscope}%
\pgfsetbuttcap%
\pgfsetmiterjoin%
\definecolor{currentfill}{rgb}{1.000000,1.000000,1.000000}%
\pgfsetfillcolor{currentfill}%
\pgfsetlinewidth{0.000000pt}%
\definecolor{currentstroke}{rgb}{0.000000,0.000000,0.000000}%
\pgfsetstrokecolor{currentstroke}%
\pgfsetstrokeopacity{0.000000}%
\pgfsetdash{}{0pt}%
\pgfpathmoveto{\pgfqpoint{0.787074in}{0.548769in}}%
\pgfpathlineto{\pgfqpoint{5.850000in}{0.548769in}}%
\pgfpathlineto{\pgfqpoint{5.850000in}{3.651359in}}%
\pgfpathlineto{\pgfqpoint{0.787074in}{3.651359in}}%
\pgfpathclose%
\pgfusepath{fill}%
\end{pgfscope}%
\begin{pgfscope}%
\pgfpathrectangle{\pgfqpoint{0.787074in}{0.548769in}}{\pgfqpoint{5.062926in}{3.102590in}}%
\pgfusepath{clip}%
\pgfsetbuttcap%
\pgfsetroundjoin%
\definecolor{currentfill}{rgb}{0.121569,0.466667,0.705882}%
\pgfsetfillcolor{currentfill}%
\pgfsetlinewidth{1.003750pt}%
\definecolor{currentstroke}{rgb}{0.121569,0.466667,0.705882}%
\pgfsetstrokecolor{currentstroke}%
\pgfsetdash{}{0pt}%
\pgfpathmoveto{\pgfqpoint{1.338854in}{0.648198in}}%
\pgfpathcurveto{\pgfqpoint{1.349904in}{0.648198in}}{\pgfqpoint{1.360503in}{0.652588in}}{\pgfqpoint{1.368316in}{0.660402in}}%
\pgfpathcurveto{\pgfqpoint{1.376130in}{0.668215in}}{\pgfqpoint{1.380520in}{0.678814in}}{\pgfqpoint{1.380520in}{0.689865in}}%
\pgfpathcurveto{\pgfqpoint{1.380520in}{0.700915in}}{\pgfqpoint{1.376130in}{0.711514in}}{\pgfqpoint{1.368316in}{0.719327in}}%
\pgfpathcurveto{\pgfqpoint{1.360503in}{0.727141in}}{\pgfqpoint{1.349904in}{0.731531in}}{\pgfqpoint{1.338854in}{0.731531in}}%
\pgfpathcurveto{\pgfqpoint{1.327803in}{0.731531in}}{\pgfqpoint{1.317204in}{0.727141in}}{\pgfqpoint{1.309391in}{0.719327in}}%
\pgfpathcurveto{\pgfqpoint{1.301577in}{0.711514in}}{\pgfqpoint{1.297187in}{0.700915in}}{\pgfqpoint{1.297187in}{0.689865in}}%
\pgfpathcurveto{\pgfqpoint{1.297187in}{0.678814in}}{\pgfqpoint{1.301577in}{0.668215in}}{\pgfqpoint{1.309391in}{0.660402in}}%
\pgfpathcurveto{\pgfqpoint{1.317204in}{0.652588in}}{\pgfqpoint{1.327803in}{0.648198in}}{\pgfqpoint{1.338854in}{0.648198in}}%
\pgfpathclose%
\pgfusepath{stroke,fill}%
\end{pgfscope}%
\begin{pgfscope}%
\pgfpathrectangle{\pgfqpoint{0.787074in}{0.548769in}}{\pgfqpoint{5.062926in}{3.102590in}}%
\pgfusepath{clip}%
\pgfsetbuttcap%
\pgfsetroundjoin%
\definecolor{currentfill}{rgb}{0.121569,0.466667,0.705882}%
\pgfsetfillcolor{currentfill}%
\pgfsetlinewidth{1.003750pt}%
\definecolor{currentstroke}{rgb}{0.121569,0.466667,0.705882}%
\pgfsetstrokecolor{currentstroke}%
\pgfsetdash{}{0pt}%
\pgfpathmoveto{\pgfqpoint{3.946792in}{1.887135in}}%
\pgfpathcurveto{\pgfqpoint{3.957842in}{1.887135in}}{\pgfqpoint{3.968441in}{1.891526in}}{\pgfqpoint{3.976255in}{1.899339in}}%
\pgfpathcurveto{\pgfqpoint{3.984068in}{1.907153in}}{\pgfqpoint{3.988459in}{1.917752in}}{\pgfqpoint{3.988459in}{1.928802in}}%
\pgfpathcurveto{\pgfqpoint{3.988459in}{1.939852in}}{\pgfqpoint{3.984068in}{1.950451in}}{\pgfqpoint{3.976255in}{1.958265in}}%
\pgfpathcurveto{\pgfqpoint{3.968441in}{1.966079in}}{\pgfqpoint{3.957842in}{1.970469in}}{\pgfqpoint{3.946792in}{1.970469in}}%
\pgfpathcurveto{\pgfqpoint{3.935742in}{1.970469in}}{\pgfqpoint{3.925143in}{1.966079in}}{\pgfqpoint{3.917329in}{1.958265in}}%
\pgfpathcurveto{\pgfqpoint{3.909515in}{1.950451in}}{\pgfqpoint{3.905125in}{1.939852in}}{\pgfqpoint{3.905125in}{1.928802in}}%
\pgfpathcurveto{\pgfqpoint{3.905125in}{1.917752in}}{\pgfqpoint{3.909515in}{1.907153in}}{\pgfqpoint{3.917329in}{1.899339in}}%
\pgfpathcurveto{\pgfqpoint{3.925143in}{1.891526in}}{\pgfqpoint{3.935742in}{1.887135in}}{\pgfqpoint{3.946792in}{1.887135in}}%
\pgfpathclose%
\pgfusepath{stroke,fill}%
\end{pgfscope}%
\begin{pgfscope}%
\pgfpathrectangle{\pgfqpoint{0.787074in}{0.548769in}}{\pgfqpoint{5.062926in}{3.102590in}}%
\pgfusepath{clip}%
\pgfsetbuttcap%
\pgfsetroundjoin%
\definecolor{currentfill}{rgb}{1.000000,0.498039,0.054902}%
\pgfsetfillcolor{currentfill}%
\pgfsetlinewidth{1.003750pt}%
\definecolor{currentstroke}{rgb}{1.000000,0.498039,0.054902}%
\pgfsetstrokecolor{currentstroke}%
\pgfsetdash{}{0pt}%
\pgfpathmoveto{\pgfqpoint{1.347143in}{2.790510in}}%
\pgfpathcurveto{\pgfqpoint{1.358193in}{2.790510in}}{\pgfqpoint{1.368792in}{2.794900in}}{\pgfqpoint{1.376606in}{2.802714in}}%
\pgfpathcurveto{\pgfqpoint{1.384420in}{2.810527in}}{\pgfqpoint{1.388810in}{2.821126in}}{\pgfqpoint{1.388810in}{2.832176in}}%
\pgfpathcurveto{\pgfqpoint{1.388810in}{2.843227in}}{\pgfqpoint{1.384420in}{2.853826in}}{\pgfqpoint{1.376606in}{2.861639in}}%
\pgfpathcurveto{\pgfqpoint{1.368792in}{2.869453in}}{\pgfqpoint{1.358193in}{2.873843in}}{\pgfqpoint{1.347143in}{2.873843in}}%
\pgfpathcurveto{\pgfqpoint{1.336093in}{2.873843in}}{\pgfqpoint{1.325494in}{2.869453in}}{\pgfqpoint{1.317680in}{2.861639in}}%
\pgfpathcurveto{\pgfqpoint{1.309867in}{2.853826in}}{\pgfqpoint{1.305477in}{2.843227in}}{\pgfqpoint{1.305477in}{2.832176in}}%
\pgfpathcurveto{\pgfqpoint{1.305477in}{2.821126in}}{\pgfqpoint{1.309867in}{2.810527in}}{\pgfqpoint{1.317680in}{2.802714in}}%
\pgfpathcurveto{\pgfqpoint{1.325494in}{2.794900in}}{\pgfqpoint{1.336093in}{2.790510in}}{\pgfqpoint{1.347143in}{2.790510in}}%
\pgfpathclose%
\pgfusepath{stroke,fill}%
\end{pgfscope}%
\begin{pgfscope}%
\pgfpathrectangle{\pgfqpoint{0.787074in}{0.548769in}}{\pgfqpoint{5.062926in}{3.102590in}}%
\pgfusepath{clip}%
\pgfsetbuttcap%
\pgfsetroundjoin%
\definecolor{currentfill}{rgb}{0.121569,0.466667,0.705882}%
\pgfsetfillcolor{currentfill}%
\pgfsetlinewidth{1.003750pt}%
\definecolor{currentstroke}{rgb}{0.121569,0.466667,0.705882}%
\pgfsetstrokecolor{currentstroke}%
\pgfsetdash{}{0pt}%
\pgfpathmoveto{\pgfqpoint{2.585121in}{2.680271in}}%
\pgfpathcurveto{\pgfqpoint{2.596171in}{2.680271in}}{\pgfqpoint{2.606770in}{2.684661in}}{\pgfqpoint{2.614584in}{2.692475in}}%
\pgfpathcurveto{\pgfqpoint{2.622397in}{2.700289in}}{\pgfqpoint{2.626787in}{2.710888in}}{\pgfqpoint{2.626787in}{2.721938in}}%
\pgfpathcurveto{\pgfqpoint{2.626787in}{2.732988in}}{\pgfqpoint{2.622397in}{2.743587in}}{\pgfqpoint{2.614584in}{2.751401in}}%
\pgfpathcurveto{\pgfqpoint{2.606770in}{2.759214in}}{\pgfqpoint{2.596171in}{2.763604in}}{\pgfqpoint{2.585121in}{2.763604in}}%
\pgfpathcurveto{\pgfqpoint{2.574071in}{2.763604in}}{\pgfqpoint{2.563472in}{2.759214in}}{\pgfqpoint{2.555658in}{2.751401in}}%
\pgfpathcurveto{\pgfqpoint{2.547844in}{2.743587in}}{\pgfqpoint{2.543454in}{2.732988in}}{\pgfqpoint{2.543454in}{2.721938in}}%
\pgfpathcurveto{\pgfqpoint{2.543454in}{2.710888in}}{\pgfqpoint{2.547844in}{2.700289in}}{\pgfqpoint{2.555658in}{2.692475in}}%
\pgfpathcurveto{\pgfqpoint{2.563472in}{2.684661in}}{\pgfqpoint{2.574071in}{2.680271in}}{\pgfqpoint{2.585121in}{2.680271in}}%
\pgfpathclose%
\pgfusepath{stroke,fill}%
\end{pgfscope}%
\begin{pgfscope}%
\pgfpathrectangle{\pgfqpoint{0.787074in}{0.548769in}}{\pgfqpoint{5.062926in}{3.102590in}}%
\pgfusepath{clip}%
\pgfsetbuttcap%
\pgfsetroundjoin%
\definecolor{currentfill}{rgb}{1.000000,0.498039,0.054902}%
\pgfsetfillcolor{currentfill}%
\pgfsetlinewidth{1.003750pt}%
\definecolor{currentstroke}{rgb}{1.000000,0.498039,0.054902}%
\pgfsetstrokecolor{currentstroke}%
\pgfsetdash{}{0pt}%
\pgfpathmoveto{\pgfqpoint{2.729604in}{2.647277in}}%
\pgfpathcurveto{\pgfqpoint{2.740654in}{2.647277in}}{\pgfqpoint{2.751253in}{2.651668in}}{\pgfqpoint{2.759066in}{2.659481in}}%
\pgfpathcurveto{\pgfqpoint{2.766880in}{2.667295in}}{\pgfqpoint{2.771270in}{2.677894in}}{\pgfqpoint{2.771270in}{2.688944in}}%
\pgfpathcurveto{\pgfqpoint{2.771270in}{2.699994in}}{\pgfqpoint{2.766880in}{2.710593in}}{\pgfqpoint{2.759066in}{2.718407in}}%
\pgfpathcurveto{\pgfqpoint{2.751253in}{2.726221in}}{\pgfqpoint{2.740654in}{2.730611in}}{\pgfqpoint{2.729604in}{2.730611in}}%
\pgfpathcurveto{\pgfqpoint{2.718553in}{2.730611in}}{\pgfqpoint{2.707954in}{2.726221in}}{\pgfqpoint{2.700141in}{2.718407in}}%
\pgfpathcurveto{\pgfqpoint{2.692327in}{2.710593in}}{\pgfqpoint{2.687937in}{2.699994in}}{\pgfqpoint{2.687937in}{2.688944in}}%
\pgfpathcurveto{\pgfqpoint{2.687937in}{2.677894in}}{\pgfqpoint{2.692327in}{2.667295in}}{\pgfqpoint{2.700141in}{2.659481in}}%
\pgfpathcurveto{\pgfqpoint{2.707954in}{2.651668in}}{\pgfqpoint{2.718553in}{2.647277in}}{\pgfqpoint{2.729604in}{2.647277in}}%
\pgfpathclose%
\pgfusepath{stroke,fill}%
\end{pgfscope}%
\begin{pgfscope}%
\pgfpathrectangle{\pgfqpoint{0.787074in}{0.548769in}}{\pgfqpoint{5.062926in}{3.102590in}}%
\pgfusepath{clip}%
\pgfsetbuttcap%
\pgfsetroundjoin%
\definecolor{currentfill}{rgb}{0.121569,0.466667,0.705882}%
\pgfsetfillcolor{currentfill}%
\pgfsetlinewidth{1.003750pt}%
\definecolor{currentstroke}{rgb}{0.121569,0.466667,0.705882}%
\pgfsetstrokecolor{currentstroke}%
\pgfsetdash{}{0pt}%
\pgfpathmoveto{\pgfqpoint{2.210394in}{2.057905in}}%
\pgfpathcurveto{\pgfqpoint{2.221444in}{2.057905in}}{\pgfqpoint{2.232043in}{2.062295in}}{\pgfqpoint{2.239857in}{2.070109in}}%
\pgfpathcurveto{\pgfqpoint{2.247671in}{2.077922in}}{\pgfqpoint{2.252061in}{2.088521in}}{\pgfqpoint{2.252061in}{2.099572in}}%
\pgfpathcurveto{\pgfqpoint{2.252061in}{2.110622in}}{\pgfqpoint{2.247671in}{2.121221in}}{\pgfqpoint{2.239857in}{2.129034in}}%
\pgfpathcurveto{\pgfqpoint{2.232043in}{2.136848in}}{\pgfqpoint{2.221444in}{2.141238in}}{\pgfqpoint{2.210394in}{2.141238in}}%
\pgfpathcurveto{\pgfqpoint{2.199344in}{2.141238in}}{\pgfqpoint{2.188745in}{2.136848in}}{\pgfqpoint{2.180931in}{2.129034in}}%
\pgfpathcurveto{\pgfqpoint{2.173118in}{2.121221in}}{\pgfqpoint{2.168728in}{2.110622in}}{\pgfqpoint{2.168728in}{2.099572in}}%
\pgfpathcurveto{\pgfqpoint{2.168728in}{2.088521in}}{\pgfqpoint{2.173118in}{2.077922in}}{\pgfqpoint{2.180931in}{2.070109in}}%
\pgfpathcurveto{\pgfqpoint{2.188745in}{2.062295in}}{\pgfqpoint{2.199344in}{2.057905in}}{\pgfqpoint{2.210394in}{2.057905in}}%
\pgfpathclose%
\pgfusepath{stroke,fill}%
\end{pgfscope}%
\begin{pgfscope}%
\pgfpathrectangle{\pgfqpoint{0.787074in}{0.548769in}}{\pgfqpoint{5.062926in}{3.102590in}}%
\pgfusepath{clip}%
\pgfsetbuttcap%
\pgfsetroundjoin%
\definecolor{currentfill}{rgb}{0.121569,0.466667,0.705882}%
\pgfsetfillcolor{currentfill}%
\pgfsetlinewidth{1.003750pt}%
\definecolor{currentstroke}{rgb}{0.121569,0.466667,0.705882}%
\pgfsetstrokecolor{currentstroke}%
\pgfsetdash{}{0pt}%
\pgfpathmoveto{\pgfqpoint{2.349452in}{2.153691in}}%
\pgfpathcurveto{\pgfqpoint{2.360502in}{2.153691in}}{\pgfqpoint{2.371101in}{2.158081in}}{\pgfqpoint{2.378915in}{2.165895in}}%
\pgfpathcurveto{\pgfqpoint{2.386728in}{2.173708in}}{\pgfqpoint{2.391119in}{2.184307in}}{\pgfqpoint{2.391119in}{2.195357in}}%
\pgfpathcurveto{\pgfqpoint{2.391119in}{2.206408in}}{\pgfqpoint{2.386728in}{2.217007in}}{\pgfqpoint{2.378915in}{2.224820in}}%
\pgfpathcurveto{\pgfqpoint{2.371101in}{2.232634in}}{\pgfqpoint{2.360502in}{2.237024in}}{\pgfqpoint{2.349452in}{2.237024in}}%
\pgfpathcurveto{\pgfqpoint{2.338402in}{2.237024in}}{\pgfqpoint{2.327803in}{2.232634in}}{\pgfqpoint{2.319989in}{2.224820in}}%
\pgfpathcurveto{\pgfqpoint{2.312175in}{2.217007in}}{\pgfqpoint{2.307785in}{2.206408in}}{\pgfqpoint{2.307785in}{2.195357in}}%
\pgfpathcurveto{\pgfqpoint{2.307785in}{2.184307in}}{\pgfqpoint{2.312175in}{2.173708in}}{\pgfqpoint{2.319989in}{2.165895in}}%
\pgfpathcurveto{\pgfqpoint{2.327803in}{2.158081in}}{\pgfqpoint{2.338402in}{2.153691in}}{\pgfqpoint{2.349452in}{2.153691in}}%
\pgfpathclose%
\pgfusepath{stroke,fill}%
\end{pgfscope}%
\begin{pgfscope}%
\pgfpathrectangle{\pgfqpoint{0.787074in}{0.548769in}}{\pgfqpoint{5.062926in}{3.102590in}}%
\pgfusepath{clip}%
\pgfsetbuttcap%
\pgfsetroundjoin%
\definecolor{currentfill}{rgb}{1.000000,0.498039,0.054902}%
\pgfsetfillcolor{currentfill}%
\pgfsetlinewidth{1.003750pt}%
\definecolor{currentstroke}{rgb}{1.000000,0.498039,0.054902}%
\pgfsetstrokecolor{currentstroke}%
\pgfsetdash{}{0pt}%
\pgfpathmoveto{\pgfqpoint{1.461202in}{2.714270in}}%
\pgfpathcurveto{\pgfqpoint{1.472252in}{2.714270in}}{\pgfqpoint{1.482851in}{2.718660in}}{\pgfqpoint{1.490665in}{2.726474in}}%
\pgfpathcurveto{\pgfqpoint{1.498478in}{2.734288in}}{\pgfqpoint{1.502868in}{2.744887in}}{\pgfqpoint{1.502868in}{2.755937in}}%
\pgfpathcurveto{\pgfqpoint{1.502868in}{2.766987in}}{\pgfqpoint{1.498478in}{2.777586in}}{\pgfqpoint{1.490665in}{2.785399in}}%
\pgfpathcurveto{\pgfqpoint{1.482851in}{2.793213in}}{\pgfqpoint{1.472252in}{2.797603in}}{\pgfqpoint{1.461202in}{2.797603in}}%
\pgfpathcurveto{\pgfqpoint{1.450152in}{2.797603in}}{\pgfqpoint{1.439553in}{2.793213in}}{\pgfqpoint{1.431739in}{2.785399in}}%
\pgfpathcurveto{\pgfqpoint{1.423925in}{2.777586in}}{\pgfqpoint{1.419535in}{2.766987in}}{\pgfqpoint{1.419535in}{2.755937in}}%
\pgfpathcurveto{\pgfqpoint{1.419535in}{2.744887in}}{\pgfqpoint{1.423925in}{2.734288in}}{\pgfqpoint{1.431739in}{2.726474in}}%
\pgfpathcurveto{\pgfqpoint{1.439553in}{2.718660in}}{\pgfqpoint{1.450152in}{2.714270in}}{\pgfqpoint{1.461202in}{2.714270in}}%
\pgfpathclose%
\pgfusepath{stroke,fill}%
\end{pgfscope}%
\begin{pgfscope}%
\pgfpathrectangle{\pgfqpoint{0.787074in}{0.548769in}}{\pgfqpoint{5.062926in}{3.102590in}}%
\pgfusepath{clip}%
\pgfsetbuttcap%
\pgfsetroundjoin%
\definecolor{currentfill}{rgb}{1.000000,0.498039,0.054902}%
\pgfsetfillcolor{currentfill}%
\pgfsetlinewidth{1.003750pt}%
\definecolor{currentstroke}{rgb}{1.000000,0.498039,0.054902}%
\pgfsetstrokecolor{currentstroke}%
\pgfsetdash{}{0pt}%
\pgfpathmoveto{\pgfqpoint{2.057969in}{2.861842in}}%
\pgfpathcurveto{\pgfqpoint{2.069019in}{2.861842in}}{\pgfqpoint{2.079618in}{2.866233in}}{\pgfqpoint{2.087432in}{2.874046in}}%
\pgfpathcurveto{\pgfqpoint{2.095245in}{2.881860in}}{\pgfqpoint{2.099636in}{2.892459in}}{\pgfqpoint{2.099636in}{2.903509in}}%
\pgfpathcurveto{\pgfqpoint{2.099636in}{2.914559in}}{\pgfqpoint{2.095245in}{2.925158in}}{\pgfqpoint{2.087432in}{2.932972in}}%
\pgfpathcurveto{\pgfqpoint{2.079618in}{2.940786in}}{\pgfqpoint{2.069019in}{2.945176in}}{\pgfqpoint{2.057969in}{2.945176in}}%
\pgfpathcurveto{\pgfqpoint{2.046919in}{2.945176in}}{\pgfqpoint{2.036320in}{2.940786in}}{\pgfqpoint{2.028506in}{2.932972in}}%
\pgfpathcurveto{\pgfqpoint{2.020693in}{2.925158in}}{\pgfqpoint{2.016302in}{2.914559in}}{\pgfqpoint{2.016302in}{2.903509in}}%
\pgfpathcurveto{\pgfqpoint{2.016302in}{2.892459in}}{\pgfqpoint{2.020693in}{2.881860in}}{\pgfqpoint{2.028506in}{2.874046in}}%
\pgfpathcurveto{\pgfqpoint{2.036320in}{2.866233in}}{\pgfqpoint{2.046919in}{2.861842in}}{\pgfqpoint{2.057969in}{2.861842in}}%
\pgfpathclose%
\pgfusepath{stroke,fill}%
\end{pgfscope}%
\begin{pgfscope}%
\pgfpathrectangle{\pgfqpoint{0.787074in}{0.548769in}}{\pgfqpoint{5.062926in}{3.102590in}}%
\pgfusepath{clip}%
\pgfsetbuttcap%
\pgfsetroundjoin%
\definecolor{currentfill}{rgb}{0.121569,0.466667,0.705882}%
\pgfsetfillcolor{currentfill}%
\pgfsetlinewidth{1.003750pt}%
\definecolor{currentstroke}{rgb}{0.121569,0.466667,0.705882}%
\pgfsetstrokecolor{currentstroke}%
\pgfsetdash{}{0pt}%
\pgfpathmoveto{\pgfqpoint{1.017294in}{0.668329in}}%
\pgfpathcurveto{\pgfqpoint{1.028344in}{0.668329in}}{\pgfqpoint{1.038943in}{0.672720in}}{\pgfqpoint{1.046756in}{0.680533in}}%
\pgfpathcurveto{\pgfqpoint{1.054570in}{0.688347in}}{\pgfqpoint{1.058960in}{0.698946in}}{\pgfqpoint{1.058960in}{0.709996in}}%
\pgfpathcurveto{\pgfqpoint{1.058960in}{0.721046in}}{\pgfqpoint{1.054570in}{0.731645in}}{\pgfqpoint{1.046756in}{0.739459in}}%
\pgfpathcurveto{\pgfqpoint{1.038943in}{0.747272in}}{\pgfqpoint{1.028344in}{0.751663in}}{\pgfqpoint{1.017294in}{0.751663in}}%
\pgfpathcurveto{\pgfqpoint{1.006244in}{0.751663in}}{\pgfqpoint{0.995644in}{0.747272in}}{\pgfqpoint{0.987831in}{0.739459in}}%
\pgfpathcurveto{\pgfqpoint{0.980017in}{0.731645in}}{\pgfqpoint{0.975627in}{0.721046in}}{\pgfqpoint{0.975627in}{0.709996in}}%
\pgfpathcurveto{\pgfqpoint{0.975627in}{0.698946in}}{\pgfqpoint{0.980017in}{0.688347in}}{\pgfqpoint{0.987831in}{0.680533in}}%
\pgfpathcurveto{\pgfqpoint{0.995644in}{0.672720in}}{\pgfqpoint{1.006244in}{0.668329in}}{\pgfqpoint{1.017294in}{0.668329in}}%
\pgfpathclose%
\pgfusepath{stroke,fill}%
\end{pgfscope}%
\begin{pgfscope}%
\pgfpathrectangle{\pgfqpoint{0.787074in}{0.548769in}}{\pgfqpoint{5.062926in}{3.102590in}}%
\pgfusepath{clip}%
\pgfsetbuttcap%
\pgfsetroundjoin%
\definecolor{currentfill}{rgb}{1.000000,0.498039,0.054902}%
\pgfsetfillcolor{currentfill}%
\pgfsetlinewidth{1.003750pt}%
\definecolor{currentstroke}{rgb}{1.000000,0.498039,0.054902}%
\pgfsetstrokecolor{currentstroke}%
\pgfsetdash{}{0pt}%
\pgfpathmoveto{\pgfqpoint{2.115085in}{3.008532in}}%
\pgfpathcurveto{\pgfqpoint{2.126135in}{3.008532in}}{\pgfqpoint{2.136734in}{3.012923in}}{\pgfqpoint{2.144548in}{3.020736in}}%
\pgfpathcurveto{\pgfqpoint{2.152362in}{3.028550in}}{\pgfqpoint{2.156752in}{3.039149in}}{\pgfqpoint{2.156752in}{3.050199in}}%
\pgfpathcurveto{\pgfqpoint{2.156752in}{3.061249in}}{\pgfqpoint{2.152362in}{3.071848in}}{\pgfqpoint{2.144548in}{3.079662in}}%
\pgfpathcurveto{\pgfqpoint{2.136734in}{3.087476in}}{\pgfqpoint{2.126135in}{3.091866in}}{\pgfqpoint{2.115085in}{3.091866in}}%
\pgfpathcurveto{\pgfqpoint{2.104035in}{3.091866in}}{\pgfqpoint{2.093436in}{3.087476in}}{\pgfqpoint{2.085622in}{3.079662in}}%
\pgfpathcurveto{\pgfqpoint{2.077809in}{3.071848in}}{\pgfqpoint{2.073418in}{3.061249in}}{\pgfqpoint{2.073418in}{3.050199in}}%
\pgfpathcurveto{\pgfqpoint{2.073418in}{3.039149in}}{\pgfqpoint{2.077809in}{3.028550in}}{\pgfqpoint{2.085622in}{3.020736in}}%
\pgfpathcurveto{\pgfqpoint{2.093436in}{3.012923in}}{\pgfqpoint{2.104035in}{3.008532in}}{\pgfqpoint{2.115085in}{3.008532in}}%
\pgfpathclose%
\pgfusepath{stroke,fill}%
\end{pgfscope}%
\begin{pgfscope}%
\pgfpathrectangle{\pgfqpoint{0.787074in}{0.548769in}}{\pgfqpoint{5.062926in}{3.102590in}}%
\pgfusepath{clip}%
\pgfsetbuttcap%
\pgfsetroundjoin%
\definecolor{currentfill}{rgb}{1.000000,0.498039,0.054902}%
\pgfsetfillcolor{currentfill}%
\pgfsetlinewidth{1.003750pt}%
\definecolor{currentstroke}{rgb}{1.000000,0.498039,0.054902}%
\pgfsetstrokecolor{currentstroke}%
\pgfsetdash{}{0pt}%
\pgfpathmoveto{\pgfqpoint{1.549133in}{2.645063in}}%
\pgfpathcurveto{\pgfqpoint{1.560183in}{2.645063in}}{\pgfqpoint{1.570782in}{2.649453in}}{\pgfqpoint{1.578595in}{2.657266in}}%
\pgfpathcurveto{\pgfqpoint{1.586409in}{2.665080in}}{\pgfqpoint{1.590799in}{2.675679in}}{\pgfqpoint{1.590799in}{2.686729in}}%
\pgfpathcurveto{\pgfqpoint{1.590799in}{2.697779in}}{\pgfqpoint{1.586409in}{2.708378in}}{\pgfqpoint{1.578595in}{2.716192in}}%
\pgfpathcurveto{\pgfqpoint{1.570782in}{2.724006in}}{\pgfqpoint{1.560183in}{2.728396in}}{\pgfqpoint{1.549133in}{2.728396in}}%
\pgfpathcurveto{\pgfqpoint{1.538083in}{2.728396in}}{\pgfqpoint{1.527484in}{2.724006in}}{\pgfqpoint{1.519670in}{2.716192in}}%
\pgfpathcurveto{\pgfqpoint{1.511856in}{2.708378in}}{\pgfqpoint{1.507466in}{2.697779in}}{\pgfqpoint{1.507466in}{2.686729in}}%
\pgfpathcurveto{\pgfqpoint{1.507466in}{2.675679in}}{\pgfqpoint{1.511856in}{2.665080in}}{\pgfqpoint{1.519670in}{2.657266in}}%
\pgfpathcurveto{\pgfqpoint{1.527484in}{2.649453in}}{\pgfqpoint{1.538083in}{2.645063in}}{\pgfqpoint{1.549133in}{2.645063in}}%
\pgfpathclose%
\pgfusepath{stroke,fill}%
\end{pgfscope}%
\begin{pgfscope}%
\pgfpathrectangle{\pgfqpoint{0.787074in}{0.548769in}}{\pgfqpoint{5.062926in}{3.102590in}}%
\pgfusepath{clip}%
\pgfsetbuttcap%
\pgfsetroundjoin%
\definecolor{currentfill}{rgb}{1.000000,0.498039,0.054902}%
\pgfsetfillcolor{currentfill}%
\pgfsetlinewidth{1.003750pt}%
\definecolor{currentstroke}{rgb}{1.000000,0.498039,0.054902}%
\pgfsetstrokecolor{currentstroke}%
\pgfsetdash{}{0pt}%
\pgfpathmoveto{\pgfqpoint{1.728770in}{3.140866in}}%
\pgfpathcurveto{\pgfqpoint{1.739821in}{3.140866in}}{\pgfqpoint{1.750420in}{3.145257in}}{\pgfqpoint{1.758233in}{3.153070in}}%
\pgfpathcurveto{\pgfqpoint{1.766047in}{3.160884in}}{\pgfqpoint{1.770437in}{3.171483in}}{\pgfqpoint{1.770437in}{3.182533in}}%
\pgfpathcurveto{\pgfqpoint{1.770437in}{3.193583in}}{\pgfqpoint{1.766047in}{3.204182in}}{\pgfqpoint{1.758233in}{3.211996in}}%
\pgfpathcurveto{\pgfqpoint{1.750420in}{3.219809in}}{\pgfqpoint{1.739821in}{3.224200in}}{\pgfqpoint{1.728770in}{3.224200in}}%
\pgfpathcurveto{\pgfqpoint{1.717720in}{3.224200in}}{\pgfqpoint{1.707121in}{3.219809in}}{\pgfqpoint{1.699308in}{3.211996in}}%
\pgfpathcurveto{\pgfqpoint{1.691494in}{3.204182in}}{\pgfqpoint{1.687104in}{3.193583in}}{\pgfqpoint{1.687104in}{3.182533in}}%
\pgfpathcurveto{\pgfqpoint{1.687104in}{3.171483in}}{\pgfqpoint{1.691494in}{3.160884in}}{\pgfqpoint{1.699308in}{3.153070in}}%
\pgfpathcurveto{\pgfqpoint{1.707121in}{3.145257in}}{\pgfqpoint{1.717720in}{3.140866in}}{\pgfqpoint{1.728770in}{3.140866in}}%
\pgfpathclose%
\pgfusepath{stroke,fill}%
\end{pgfscope}%
\begin{pgfscope}%
\pgfpathrectangle{\pgfqpoint{0.787074in}{0.548769in}}{\pgfqpoint{5.062926in}{3.102590in}}%
\pgfusepath{clip}%
\pgfsetbuttcap%
\pgfsetroundjoin%
\definecolor{currentfill}{rgb}{1.000000,0.498039,0.054902}%
\pgfsetfillcolor{currentfill}%
\pgfsetlinewidth{1.003750pt}%
\definecolor{currentstroke}{rgb}{1.000000,0.498039,0.054902}%
\pgfsetstrokecolor{currentstroke}%
\pgfsetdash{}{0pt}%
\pgfpathmoveto{\pgfqpoint{1.498700in}{2.974499in}}%
\pgfpathcurveto{\pgfqpoint{1.509751in}{2.974499in}}{\pgfqpoint{1.520350in}{2.978889in}}{\pgfqpoint{1.528163in}{2.986703in}}%
\pgfpathcurveto{\pgfqpoint{1.535977in}{2.994517in}}{\pgfqpoint{1.540367in}{3.005116in}}{\pgfqpoint{1.540367in}{3.016166in}}%
\pgfpathcurveto{\pgfqpoint{1.540367in}{3.027216in}}{\pgfqpoint{1.535977in}{3.037815in}}{\pgfqpoint{1.528163in}{3.045629in}}%
\pgfpathcurveto{\pgfqpoint{1.520350in}{3.053442in}}{\pgfqpoint{1.509751in}{3.057833in}}{\pgfqpoint{1.498700in}{3.057833in}}%
\pgfpathcurveto{\pgfqpoint{1.487650in}{3.057833in}}{\pgfqpoint{1.477051in}{3.053442in}}{\pgfqpoint{1.469238in}{3.045629in}}%
\pgfpathcurveto{\pgfqpoint{1.461424in}{3.037815in}}{\pgfqpoint{1.457034in}{3.027216in}}{\pgfqpoint{1.457034in}{3.016166in}}%
\pgfpathcurveto{\pgfqpoint{1.457034in}{3.005116in}}{\pgfqpoint{1.461424in}{2.994517in}}{\pgfqpoint{1.469238in}{2.986703in}}%
\pgfpathcurveto{\pgfqpoint{1.477051in}{2.978889in}}{\pgfqpoint{1.487650in}{2.974499in}}{\pgfqpoint{1.498700in}{2.974499in}}%
\pgfpathclose%
\pgfusepath{stroke,fill}%
\end{pgfscope}%
\begin{pgfscope}%
\pgfpathrectangle{\pgfqpoint{0.787074in}{0.548769in}}{\pgfqpoint{5.062926in}{3.102590in}}%
\pgfusepath{clip}%
\pgfsetbuttcap%
\pgfsetroundjoin%
\definecolor{currentfill}{rgb}{1.000000,0.498039,0.054902}%
\pgfsetfillcolor{currentfill}%
\pgfsetlinewidth{1.003750pt}%
\definecolor{currentstroke}{rgb}{1.000000,0.498039,0.054902}%
\pgfsetstrokecolor{currentstroke}%
\pgfsetdash{}{0pt}%
\pgfpathmoveto{\pgfqpoint{1.509681in}{1.902652in}}%
\pgfpathcurveto{\pgfqpoint{1.520731in}{1.902652in}}{\pgfqpoint{1.531330in}{1.907042in}}{\pgfqpoint{1.539144in}{1.914856in}}%
\pgfpathcurveto{\pgfqpoint{1.546957in}{1.922669in}}{\pgfqpoint{1.551348in}{1.933268in}}{\pgfqpoint{1.551348in}{1.944318in}}%
\pgfpathcurveto{\pgfqpoint{1.551348in}{1.955368in}}{\pgfqpoint{1.546957in}{1.965967in}}{\pgfqpoint{1.539144in}{1.973781in}}%
\pgfpathcurveto{\pgfqpoint{1.531330in}{1.981595in}}{\pgfqpoint{1.520731in}{1.985985in}}{\pgfqpoint{1.509681in}{1.985985in}}%
\pgfpathcurveto{\pgfqpoint{1.498631in}{1.985985in}}{\pgfqpoint{1.488032in}{1.981595in}}{\pgfqpoint{1.480218in}{1.973781in}}%
\pgfpathcurveto{\pgfqpoint{1.472405in}{1.965967in}}{\pgfqpoint{1.468014in}{1.955368in}}{\pgfqpoint{1.468014in}{1.944318in}}%
\pgfpathcurveto{\pgfqpoint{1.468014in}{1.933268in}}{\pgfqpoint{1.472405in}{1.922669in}}{\pgfqpoint{1.480218in}{1.914856in}}%
\pgfpathcurveto{\pgfqpoint{1.488032in}{1.907042in}}{\pgfqpoint{1.498631in}{1.902652in}}{\pgfqpoint{1.509681in}{1.902652in}}%
\pgfpathclose%
\pgfusepath{stroke,fill}%
\end{pgfscope}%
\begin{pgfscope}%
\pgfpathrectangle{\pgfqpoint{0.787074in}{0.548769in}}{\pgfqpoint{5.062926in}{3.102590in}}%
\pgfusepath{clip}%
\pgfsetbuttcap%
\pgfsetroundjoin%
\definecolor{currentfill}{rgb}{1.000000,0.498039,0.054902}%
\pgfsetfillcolor{currentfill}%
\pgfsetlinewidth{1.003750pt}%
\definecolor{currentstroke}{rgb}{1.000000,0.498039,0.054902}%
\pgfsetstrokecolor{currentstroke}%
\pgfsetdash{}{0pt}%
\pgfpathmoveto{\pgfqpoint{1.736496in}{1.942448in}}%
\pgfpathcurveto{\pgfqpoint{1.747546in}{1.942448in}}{\pgfqpoint{1.758145in}{1.946838in}}{\pgfqpoint{1.765959in}{1.954652in}}%
\pgfpathcurveto{\pgfqpoint{1.773772in}{1.962466in}}{\pgfqpoint{1.778163in}{1.973065in}}{\pgfqpoint{1.778163in}{1.984115in}}%
\pgfpathcurveto{\pgfqpoint{1.778163in}{1.995165in}}{\pgfqpoint{1.773772in}{2.005764in}}{\pgfqpoint{1.765959in}{2.013578in}}%
\pgfpathcurveto{\pgfqpoint{1.758145in}{2.021391in}}{\pgfqpoint{1.747546in}{2.025781in}}{\pgfqpoint{1.736496in}{2.025781in}}%
\pgfpathcurveto{\pgfqpoint{1.725446in}{2.025781in}}{\pgfqpoint{1.714847in}{2.021391in}}{\pgfqpoint{1.707033in}{2.013578in}}%
\pgfpathcurveto{\pgfqpoint{1.699220in}{2.005764in}}{\pgfqpoint{1.694829in}{1.995165in}}{\pgfqpoint{1.694829in}{1.984115in}}%
\pgfpathcurveto{\pgfqpoint{1.694829in}{1.973065in}}{\pgfqpoint{1.699220in}{1.962466in}}{\pgfqpoint{1.707033in}{1.954652in}}%
\pgfpathcurveto{\pgfqpoint{1.714847in}{1.946838in}}{\pgfqpoint{1.725446in}{1.942448in}}{\pgfqpoint{1.736496in}{1.942448in}}%
\pgfpathclose%
\pgfusepath{stroke,fill}%
\end{pgfscope}%
\begin{pgfscope}%
\pgfpathrectangle{\pgfqpoint{0.787074in}{0.548769in}}{\pgfqpoint{5.062926in}{3.102590in}}%
\pgfusepath{clip}%
\pgfsetbuttcap%
\pgfsetroundjoin%
\definecolor{currentfill}{rgb}{1.000000,0.498039,0.054902}%
\pgfsetfillcolor{currentfill}%
\pgfsetlinewidth{1.003750pt}%
\definecolor{currentstroke}{rgb}{1.000000,0.498039,0.054902}%
\pgfsetstrokecolor{currentstroke}%
\pgfsetdash{}{0pt}%
\pgfpathmoveto{\pgfqpoint{1.555252in}{2.924069in}}%
\pgfpathcurveto{\pgfqpoint{1.566302in}{2.924069in}}{\pgfqpoint{1.576901in}{2.928459in}}{\pgfqpoint{1.584715in}{2.936273in}}%
\pgfpathcurveto{\pgfqpoint{1.592529in}{2.944086in}}{\pgfqpoint{1.596919in}{2.954685in}}{\pgfqpoint{1.596919in}{2.965736in}}%
\pgfpathcurveto{\pgfqpoint{1.596919in}{2.976786in}}{\pgfqpoint{1.592529in}{2.987385in}}{\pgfqpoint{1.584715in}{2.995198in}}%
\pgfpathcurveto{\pgfqpoint{1.576901in}{3.003012in}}{\pgfqpoint{1.566302in}{3.007402in}}{\pgfqpoint{1.555252in}{3.007402in}}%
\pgfpathcurveto{\pgfqpoint{1.544202in}{3.007402in}}{\pgfqpoint{1.533603in}{3.003012in}}{\pgfqpoint{1.525789in}{2.995198in}}%
\pgfpathcurveto{\pgfqpoint{1.517976in}{2.987385in}}{\pgfqpoint{1.513586in}{2.976786in}}{\pgfqpoint{1.513586in}{2.965736in}}%
\pgfpathcurveto{\pgfqpoint{1.513586in}{2.954685in}}{\pgfqpoint{1.517976in}{2.944086in}}{\pgfqpoint{1.525789in}{2.936273in}}%
\pgfpathcurveto{\pgfqpoint{1.533603in}{2.928459in}}{\pgfqpoint{1.544202in}{2.924069in}}{\pgfqpoint{1.555252in}{2.924069in}}%
\pgfpathclose%
\pgfusepath{stroke,fill}%
\end{pgfscope}%
\begin{pgfscope}%
\pgfpathrectangle{\pgfqpoint{0.787074in}{0.548769in}}{\pgfqpoint{5.062926in}{3.102590in}}%
\pgfusepath{clip}%
\pgfsetbuttcap%
\pgfsetroundjoin%
\definecolor{currentfill}{rgb}{0.121569,0.466667,0.705882}%
\pgfsetfillcolor{currentfill}%
\pgfsetlinewidth{1.003750pt}%
\definecolor{currentstroke}{rgb}{0.121569,0.466667,0.705882}%
\pgfsetstrokecolor{currentstroke}%
\pgfsetdash{}{0pt}%
\pgfpathmoveto{\pgfqpoint{1.707070in}{2.510157in}}%
\pgfpathcurveto{\pgfqpoint{1.718120in}{2.510157in}}{\pgfqpoint{1.728719in}{2.514547in}}{\pgfqpoint{1.736533in}{2.522361in}}%
\pgfpathcurveto{\pgfqpoint{1.744346in}{2.530174in}}{\pgfqpoint{1.748737in}{2.540773in}}{\pgfqpoint{1.748737in}{2.551824in}}%
\pgfpathcurveto{\pgfqpoint{1.748737in}{2.562874in}}{\pgfqpoint{1.744346in}{2.573473in}}{\pgfqpoint{1.736533in}{2.581286in}}%
\pgfpathcurveto{\pgfqpoint{1.728719in}{2.589100in}}{\pgfqpoint{1.718120in}{2.593490in}}{\pgfqpoint{1.707070in}{2.593490in}}%
\pgfpathcurveto{\pgfqpoint{1.696020in}{2.593490in}}{\pgfqpoint{1.685421in}{2.589100in}}{\pgfqpoint{1.677607in}{2.581286in}}%
\pgfpathcurveto{\pgfqpoint{1.669793in}{2.573473in}}{\pgfqpoint{1.665403in}{2.562874in}}{\pgfqpoint{1.665403in}{2.551824in}}%
\pgfpathcurveto{\pgfqpoint{1.665403in}{2.540773in}}{\pgfqpoint{1.669793in}{2.530174in}}{\pgfqpoint{1.677607in}{2.522361in}}%
\pgfpathcurveto{\pgfqpoint{1.685421in}{2.514547in}}{\pgfqpoint{1.696020in}{2.510157in}}{\pgfqpoint{1.707070in}{2.510157in}}%
\pgfpathclose%
\pgfusepath{stroke,fill}%
\end{pgfscope}%
\begin{pgfscope}%
\pgfpathrectangle{\pgfqpoint{0.787074in}{0.548769in}}{\pgfqpoint{5.062926in}{3.102590in}}%
\pgfusepath{clip}%
\pgfsetbuttcap%
\pgfsetroundjoin%
\definecolor{currentfill}{rgb}{0.121569,0.466667,0.705882}%
\pgfsetfillcolor{currentfill}%
\pgfsetlinewidth{1.003750pt}%
\definecolor{currentstroke}{rgb}{0.121569,0.466667,0.705882}%
\pgfsetstrokecolor{currentstroke}%
\pgfsetdash{}{0pt}%
\pgfpathmoveto{\pgfqpoint{5.619867in}{1.380136in}}%
\pgfpathcurveto{\pgfqpoint{5.630917in}{1.380136in}}{\pgfqpoint{5.641516in}{1.384527in}}{\pgfqpoint{5.649330in}{1.392340in}}%
\pgfpathcurveto{\pgfqpoint{5.657143in}{1.400154in}}{\pgfqpoint{5.661534in}{1.410753in}}{\pgfqpoint{5.661534in}{1.421803in}}%
\pgfpathcurveto{\pgfqpoint{5.661534in}{1.432853in}}{\pgfqpoint{5.657143in}{1.443452in}}{\pgfqpoint{5.649330in}{1.451266in}}%
\pgfpathcurveto{\pgfqpoint{5.641516in}{1.459079in}}{\pgfqpoint{5.630917in}{1.463470in}}{\pgfqpoint{5.619867in}{1.463470in}}%
\pgfpathcurveto{\pgfqpoint{5.608817in}{1.463470in}}{\pgfqpoint{5.598218in}{1.459079in}}{\pgfqpoint{5.590404in}{1.451266in}}%
\pgfpathcurveto{\pgfqpoint{5.582591in}{1.443452in}}{\pgfqpoint{5.578200in}{1.432853in}}{\pgfqpoint{5.578200in}{1.421803in}}%
\pgfpathcurveto{\pgfqpoint{5.578200in}{1.410753in}}{\pgfqpoint{5.582591in}{1.400154in}}{\pgfqpoint{5.590404in}{1.392340in}}%
\pgfpathcurveto{\pgfqpoint{5.598218in}{1.384527in}}{\pgfqpoint{5.608817in}{1.380136in}}{\pgfqpoint{5.619867in}{1.380136in}}%
\pgfpathclose%
\pgfusepath{stroke,fill}%
\end{pgfscope}%
\begin{pgfscope}%
\pgfpathrectangle{\pgfqpoint{0.787074in}{0.548769in}}{\pgfqpoint{5.062926in}{3.102590in}}%
\pgfusepath{clip}%
\pgfsetbuttcap%
\pgfsetroundjoin%
\definecolor{currentfill}{rgb}{1.000000,0.498039,0.054902}%
\pgfsetfillcolor{currentfill}%
\pgfsetlinewidth{1.003750pt}%
\definecolor{currentstroke}{rgb}{1.000000,0.498039,0.054902}%
\pgfsetstrokecolor{currentstroke}%
\pgfsetdash{}{0pt}%
\pgfpathmoveto{\pgfqpoint{1.990654in}{2.854158in}}%
\pgfpathcurveto{\pgfqpoint{2.001704in}{2.854158in}}{\pgfqpoint{2.012303in}{2.858548in}}{\pgfqpoint{2.020116in}{2.866362in}}%
\pgfpathcurveto{\pgfqpoint{2.027930in}{2.874175in}}{\pgfqpoint{2.032320in}{2.884774in}}{\pgfqpoint{2.032320in}{2.895825in}}%
\pgfpathcurveto{\pgfqpoint{2.032320in}{2.906875in}}{\pgfqpoint{2.027930in}{2.917474in}}{\pgfqpoint{2.020116in}{2.925287in}}%
\pgfpathcurveto{\pgfqpoint{2.012303in}{2.933101in}}{\pgfqpoint{2.001704in}{2.937491in}}{\pgfqpoint{1.990654in}{2.937491in}}%
\pgfpathcurveto{\pgfqpoint{1.979604in}{2.937491in}}{\pgfqpoint{1.969005in}{2.933101in}}{\pgfqpoint{1.961191in}{2.925287in}}%
\pgfpathcurveto{\pgfqpoint{1.953377in}{2.917474in}}{\pgfqpoint{1.948987in}{2.906875in}}{\pgfqpoint{1.948987in}{2.895825in}}%
\pgfpathcurveto{\pgfqpoint{1.948987in}{2.884774in}}{\pgfqpoint{1.953377in}{2.874175in}}{\pgfqpoint{1.961191in}{2.866362in}}%
\pgfpathcurveto{\pgfqpoint{1.969005in}{2.858548in}}{\pgfqpoint{1.979604in}{2.854158in}}{\pgfqpoint{1.990654in}{2.854158in}}%
\pgfpathclose%
\pgfusepath{stroke,fill}%
\end{pgfscope}%
\begin{pgfscope}%
\pgfpathrectangle{\pgfqpoint{0.787074in}{0.548769in}}{\pgfqpoint{5.062926in}{3.102590in}}%
\pgfusepath{clip}%
\pgfsetbuttcap%
\pgfsetroundjoin%
\definecolor{currentfill}{rgb}{0.121569,0.466667,0.705882}%
\pgfsetfillcolor{currentfill}%
\pgfsetlinewidth{1.003750pt}%
\definecolor{currentstroke}{rgb}{0.121569,0.466667,0.705882}%
\pgfsetstrokecolor{currentstroke}%
\pgfsetdash{}{0pt}%
\pgfpathmoveto{\pgfqpoint{1.361596in}{0.648318in}}%
\pgfpathcurveto{\pgfqpoint{1.372646in}{0.648318in}}{\pgfqpoint{1.383245in}{0.652709in}}{\pgfqpoint{1.391059in}{0.660522in}}%
\pgfpathcurveto{\pgfqpoint{1.398872in}{0.668336in}}{\pgfqpoint{1.403263in}{0.678935in}}{\pgfqpoint{1.403263in}{0.689985in}}%
\pgfpathcurveto{\pgfqpoint{1.403263in}{0.701035in}}{\pgfqpoint{1.398872in}{0.711634in}}{\pgfqpoint{1.391059in}{0.719448in}}%
\pgfpathcurveto{\pgfqpoint{1.383245in}{0.727262in}}{\pgfqpoint{1.372646in}{0.731652in}}{\pgfqpoint{1.361596in}{0.731652in}}%
\pgfpathcurveto{\pgfqpoint{1.350546in}{0.731652in}}{\pgfqpoint{1.339947in}{0.727262in}}{\pgfqpoint{1.332133in}{0.719448in}}%
\pgfpathcurveto{\pgfqpoint{1.324319in}{0.711634in}}{\pgfqpoint{1.319929in}{0.701035in}}{\pgfqpoint{1.319929in}{0.689985in}}%
\pgfpathcurveto{\pgfqpoint{1.319929in}{0.678935in}}{\pgfqpoint{1.324319in}{0.668336in}}{\pgfqpoint{1.332133in}{0.660522in}}%
\pgfpathcurveto{\pgfqpoint{1.339947in}{0.652709in}}{\pgfqpoint{1.350546in}{0.648318in}}{\pgfqpoint{1.361596in}{0.648318in}}%
\pgfpathclose%
\pgfusepath{stroke,fill}%
\end{pgfscope}%
\begin{pgfscope}%
\pgfpathrectangle{\pgfqpoint{0.787074in}{0.548769in}}{\pgfqpoint{5.062926in}{3.102590in}}%
\pgfusepath{clip}%
\pgfsetbuttcap%
\pgfsetroundjoin%
\definecolor{currentfill}{rgb}{0.121569,0.466667,0.705882}%
\pgfsetfillcolor{currentfill}%
\pgfsetlinewidth{1.003750pt}%
\definecolor{currentstroke}{rgb}{0.121569,0.466667,0.705882}%
\pgfsetstrokecolor{currentstroke}%
\pgfsetdash{}{0pt}%
\pgfpathmoveto{\pgfqpoint{1.317934in}{1.283193in}}%
\pgfpathcurveto{\pgfqpoint{1.328984in}{1.283193in}}{\pgfqpoint{1.339583in}{1.287584in}}{\pgfqpoint{1.347397in}{1.295397in}}%
\pgfpathcurveto{\pgfqpoint{1.355211in}{1.303211in}}{\pgfqpoint{1.359601in}{1.313810in}}{\pgfqpoint{1.359601in}{1.324860in}}%
\pgfpathcurveto{\pgfqpoint{1.359601in}{1.335910in}}{\pgfqpoint{1.355211in}{1.346509in}}{\pgfqpoint{1.347397in}{1.354323in}}%
\pgfpathcurveto{\pgfqpoint{1.339583in}{1.362137in}}{\pgfqpoint{1.328984in}{1.366527in}}{\pgfqpoint{1.317934in}{1.366527in}}%
\pgfpathcurveto{\pgfqpoint{1.306884in}{1.366527in}}{\pgfqpoint{1.296285in}{1.362137in}}{\pgfqpoint{1.288471in}{1.354323in}}%
\pgfpathcurveto{\pgfqpoint{1.280658in}{1.346509in}}{\pgfqpoint{1.276268in}{1.335910in}}{\pgfqpoint{1.276268in}{1.324860in}}%
\pgfpathcurveto{\pgfqpoint{1.276268in}{1.313810in}}{\pgfqpoint{1.280658in}{1.303211in}}{\pgfqpoint{1.288471in}{1.295397in}}%
\pgfpathcurveto{\pgfqpoint{1.296285in}{1.287584in}}{\pgfqpoint{1.306884in}{1.283193in}}{\pgfqpoint{1.317934in}{1.283193in}}%
\pgfpathclose%
\pgfusepath{stroke,fill}%
\end{pgfscope}%
\begin{pgfscope}%
\pgfpathrectangle{\pgfqpoint{0.787074in}{0.548769in}}{\pgfqpoint{5.062926in}{3.102590in}}%
\pgfusepath{clip}%
\pgfsetbuttcap%
\pgfsetroundjoin%
\definecolor{currentfill}{rgb}{1.000000,0.498039,0.054902}%
\pgfsetfillcolor{currentfill}%
\pgfsetlinewidth{1.003750pt}%
\definecolor{currentstroke}{rgb}{1.000000,0.498039,0.054902}%
\pgfsetstrokecolor{currentstroke}%
\pgfsetdash{}{0pt}%
\pgfpathmoveto{\pgfqpoint{1.379347in}{2.443414in}}%
\pgfpathcurveto{\pgfqpoint{1.390397in}{2.443414in}}{\pgfqpoint{1.400996in}{2.447804in}}{\pgfqpoint{1.408810in}{2.455618in}}%
\pgfpathcurveto{\pgfqpoint{1.416623in}{2.463432in}}{\pgfqpoint{1.421014in}{2.474031in}}{\pgfqpoint{1.421014in}{2.485081in}}%
\pgfpathcurveto{\pgfqpoint{1.421014in}{2.496131in}}{\pgfqpoint{1.416623in}{2.506730in}}{\pgfqpoint{1.408810in}{2.514544in}}%
\pgfpathcurveto{\pgfqpoint{1.400996in}{2.522357in}}{\pgfqpoint{1.390397in}{2.526747in}}{\pgfqpoint{1.379347in}{2.526747in}}%
\pgfpathcurveto{\pgfqpoint{1.368297in}{2.526747in}}{\pgfqpoint{1.357698in}{2.522357in}}{\pgfqpoint{1.349884in}{2.514544in}}%
\pgfpathcurveto{\pgfqpoint{1.342071in}{2.506730in}}{\pgfqpoint{1.337680in}{2.496131in}}{\pgfqpoint{1.337680in}{2.485081in}}%
\pgfpathcurveto{\pgfqpoint{1.337680in}{2.474031in}}{\pgfqpoint{1.342071in}{2.463432in}}{\pgfqpoint{1.349884in}{2.455618in}}%
\pgfpathcurveto{\pgfqpoint{1.357698in}{2.447804in}}{\pgfqpoint{1.368297in}{2.443414in}}{\pgfqpoint{1.379347in}{2.443414in}}%
\pgfpathclose%
\pgfusepath{stroke,fill}%
\end{pgfscope}%
\begin{pgfscope}%
\pgfpathrectangle{\pgfqpoint{0.787074in}{0.548769in}}{\pgfqpoint{5.062926in}{3.102590in}}%
\pgfusepath{clip}%
\pgfsetbuttcap%
\pgfsetroundjoin%
\definecolor{currentfill}{rgb}{0.121569,0.466667,0.705882}%
\pgfsetfillcolor{currentfill}%
\pgfsetlinewidth{1.003750pt}%
\definecolor{currentstroke}{rgb}{0.121569,0.466667,0.705882}%
\pgfsetstrokecolor{currentstroke}%
\pgfsetdash{}{0pt}%
\pgfpathmoveto{\pgfqpoint{2.181055in}{1.745269in}}%
\pgfpathcurveto{\pgfqpoint{2.192105in}{1.745269in}}{\pgfqpoint{2.202704in}{1.749660in}}{\pgfqpoint{2.210518in}{1.757473in}}%
\pgfpathcurveto{\pgfqpoint{2.218331in}{1.765287in}}{\pgfqpoint{2.222722in}{1.775886in}}{\pgfqpoint{2.222722in}{1.786936in}}%
\pgfpathcurveto{\pgfqpoint{2.222722in}{1.797986in}}{\pgfqpoint{2.218331in}{1.808585in}}{\pgfqpoint{2.210518in}{1.816399in}}%
\pgfpathcurveto{\pgfqpoint{2.202704in}{1.824212in}}{\pgfqpoint{2.192105in}{1.828603in}}{\pgfqpoint{2.181055in}{1.828603in}}%
\pgfpathcurveto{\pgfqpoint{2.170005in}{1.828603in}}{\pgfqpoint{2.159406in}{1.824212in}}{\pgfqpoint{2.151592in}{1.816399in}}%
\pgfpathcurveto{\pgfqpoint{2.143779in}{1.808585in}}{\pgfqpoint{2.139388in}{1.797986in}}{\pgfqpoint{2.139388in}{1.786936in}}%
\pgfpathcurveto{\pgfqpoint{2.139388in}{1.775886in}}{\pgfqpoint{2.143779in}{1.765287in}}{\pgfqpoint{2.151592in}{1.757473in}}%
\pgfpathcurveto{\pgfqpoint{2.159406in}{1.749660in}}{\pgfqpoint{2.170005in}{1.745269in}}{\pgfqpoint{2.181055in}{1.745269in}}%
\pgfpathclose%
\pgfusepath{stroke,fill}%
\end{pgfscope}%
\begin{pgfscope}%
\pgfpathrectangle{\pgfqpoint{0.787074in}{0.548769in}}{\pgfqpoint{5.062926in}{3.102590in}}%
\pgfusepath{clip}%
\pgfsetbuttcap%
\pgfsetroundjoin%
\definecolor{currentfill}{rgb}{0.839216,0.152941,0.156863}%
\pgfsetfillcolor{currentfill}%
\pgfsetlinewidth{1.003750pt}%
\definecolor{currentstroke}{rgb}{0.839216,0.152941,0.156863}%
\pgfsetstrokecolor{currentstroke}%
\pgfsetdash{}{0pt}%
\pgfpathmoveto{\pgfqpoint{1.551650in}{3.066498in}}%
\pgfpathcurveto{\pgfqpoint{1.562700in}{3.066498in}}{\pgfqpoint{1.573299in}{3.070888in}}{\pgfqpoint{1.581113in}{3.078701in}}%
\pgfpathcurveto{\pgfqpoint{1.588926in}{3.086515in}}{\pgfqpoint{1.593317in}{3.097114in}}{\pgfqpoint{1.593317in}{3.108164in}}%
\pgfpathcurveto{\pgfqpoint{1.593317in}{3.119214in}}{\pgfqpoint{1.588926in}{3.129813in}}{\pgfqpoint{1.581113in}{3.137627in}}%
\pgfpathcurveto{\pgfqpoint{1.573299in}{3.145441in}}{\pgfqpoint{1.562700in}{3.149831in}}{\pgfqpoint{1.551650in}{3.149831in}}%
\pgfpathcurveto{\pgfqpoint{1.540600in}{3.149831in}}{\pgfqpoint{1.530001in}{3.145441in}}{\pgfqpoint{1.522187in}{3.137627in}}%
\pgfpathcurveto{\pgfqpoint{1.514374in}{3.129813in}}{\pgfqpoint{1.509983in}{3.119214in}}{\pgfqpoint{1.509983in}{3.108164in}}%
\pgfpathcurveto{\pgfqpoint{1.509983in}{3.097114in}}{\pgfqpoint{1.514374in}{3.086515in}}{\pgfqpoint{1.522187in}{3.078701in}}%
\pgfpathcurveto{\pgfqpoint{1.530001in}{3.070888in}}{\pgfqpoint{1.540600in}{3.066498in}}{\pgfqpoint{1.551650in}{3.066498in}}%
\pgfpathclose%
\pgfusepath{stroke,fill}%
\end{pgfscope}%
\begin{pgfscope}%
\pgfpathrectangle{\pgfqpoint{0.787074in}{0.548769in}}{\pgfqpoint{5.062926in}{3.102590in}}%
\pgfusepath{clip}%
\pgfsetbuttcap%
\pgfsetroundjoin%
\definecolor{currentfill}{rgb}{1.000000,0.498039,0.054902}%
\pgfsetfillcolor{currentfill}%
\pgfsetlinewidth{1.003750pt}%
\definecolor{currentstroke}{rgb}{1.000000,0.498039,0.054902}%
\pgfsetstrokecolor{currentstroke}%
\pgfsetdash{}{0pt}%
\pgfpathmoveto{\pgfqpoint{1.340850in}{3.012840in}}%
\pgfpathcurveto{\pgfqpoint{1.351900in}{3.012840in}}{\pgfqpoint{1.362499in}{3.017231in}}{\pgfqpoint{1.370313in}{3.025044in}}%
\pgfpathcurveto{\pgfqpoint{1.378126in}{3.032858in}}{\pgfqpoint{1.382517in}{3.043457in}}{\pgfqpoint{1.382517in}{3.054507in}}%
\pgfpathcurveto{\pgfqpoint{1.382517in}{3.065557in}}{\pgfqpoint{1.378126in}{3.076156in}}{\pgfqpoint{1.370313in}{3.083970in}}%
\pgfpathcurveto{\pgfqpoint{1.362499in}{3.091783in}}{\pgfqpoint{1.351900in}{3.096174in}}{\pgfqpoint{1.340850in}{3.096174in}}%
\pgfpathcurveto{\pgfqpoint{1.329800in}{3.096174in}}{\pgfqpoint{1.319201in}{3.091783in}}{\pgfqpoint{1.311387in}{3.083970in}}%
\pgfpathcurveto{\pgfqpoint{1.303574in}{3.076156in}}{\pgfqpoint{1.299183in}{3.065557in}}{\pgfqpoint{1.299183in}{3.054507in}}%
\pgfpathcurveto{\pgfqpoint{1.299183in}{3.043457in}}{\pgfqpoint{1.303574in}{3.032858in}}{\pgfqpoint{1.311387in}{3.025044in}}%
\pgfpathcurveto{\pgfqpoint{1.319201in}{3.017231in}}{\pgfqpoint{1.329800in}{3.012840in}}{\pgfqpoint{1.340850in}{3.012840in}}%
\pgfpathclose%
\pgfusepath{stroke,fill}%
\end{pgfscope}%
\begin{pgfscope}%
\pgfpathrectangle{\pgfqpoint{0.787074in}{0.548769in}}{\pgfqpoint{5.062926in}{3.102590in}}%
\pgfusepath{clip}%
\pgfsetbuttcap%
\pgfsetroundjoin%
\definecolor{currentfill}{rgb}{0.121569,0.466667,0.705882}%
\pgfsetfillcolor{currentfill}%
\pgfsetlinewidth{1.003750pt}%
\definecolor{currentstroke}{rgb}{0.121569,0.466667,0.705882}%
\pgfsetstrokecolor{currentstroke}%
\pgfsetdash{}{0pt}%
\pgfpathmoveto{\pgfqpoint{1.290331in}{0.670241in}}%
\pgfpathcurveto{\pgfqpoint{1.301381in}{0.670241in}}{\pgfqpoint{1.311980in}{0.674631in}}{\pgfqpoint{1.319794in}{0.682445in}}%
\pgfpathcurveto{\pgfqpoint{1.327607in}{0.690258in}}{\pgfqpoint{1.331998in}{0.700857in}}{\pgfqpoint{1.331998in}{0.711907in}}%
\pgfpathcurveto{\pgfqpoint{1.331998in}{0.722958in}}{\pgfqpoint{1.327607in}{0.733557in}}{\pgfqpoint{1.319794in}{0.741370in}}%
\pgfpathcurveto{\pgfqpoint{1.311980in}{0.749184in}}{\pgfqpoint{1.301381in}{0.753574in}}{\pgfqpoint{1.290331in}{0.753574in}}%
\pgfpathcurveto{\pgfqpoint{1.279281in}{0.753574in}}{\pgfqpoint{1.268682in}{0.749184in}}{\pgfqpoint{1.260868in}{0.741370in}}%
\pgfpathcurveto{\pgfqpoint{1.253055in}{0.733557in}}{\pgfqpoint{1.248664in}{0.722958in}}{\pgfqpoint{1.248664in}{0.711907in}}%
\pgfpathcurveto{\pgfqpoint{1.248664in}{0.700857in}}{\pgfqpoint{1.253055in}{0.690258in}}{\pgfqpoint{1.260868in}{0.682445in}}%
\pgfpathcurveto{\pgfqpoint{1.268682in}{0.674631in}}{\pgfqpoint{1.279281in}{0.670241in}}{\pgfqpoint{1.290331in}{0.670241in}}%
\pgfpathclose%
\pgfusepath{stroke,fill}%
\end{pgfscope}%
\begin{pgfscope}%
\pgfpathrectangle{\pgfqpoint{0.787074in}{0.548769in}}{\pgfqpoint{5.062926in}{3.102590in}}%
\pgfusepath{clip}%
\pgfsetbuttcap%
\pgfsetroundjoin%
\definecolor{currentfill}{rgb}{0.121569,0.466667,0.705882}%
\pgfsetfillcolor{currentfill}%
\pgfsetlinewidth{1.003750pt}%
\definecolor{currentstroke}{rgb}{0.121569,0.466667,0.705882}%
\pgfsetstrokecolor{currentstroke}%
\pgfsetdash{}{0pt}%
\pgfpathmoveto{\pgfqpoint{2.477486in}{1.395275in}}%
\pgfpathcurveto{\pgfqpoint{2.488536in}{1.395275in}}{\pgfqpoint{2.499135in}{1.399666in}}{\pgfqpoint{2.506948in}{1.407479in}}%
\pgfpathcurveto{\pgfqpoint{2.514762in}{1.415293in}}{\pgfqpoint{2.519152in}{1.425892in}}{\pgfqpoint{2.519152in}{1.436942in}}%
\pgfpathcurveto{\pgfqpoint{2.519152in}{1.447992in}}{\pgfqpoint{2.514762in}{1.458591in}}{\pgfqpoint{2.506948in}{1.466405in}}%
\pgfpathcurveto{\pgfqpoint{2.499135in}{1.474219in}}{\pgfqpoint{2.488536in}{1.478609in}}{\pgfqpoint{2.477486in}{1.478609in}}%
\pgfpathcurveto{\pgfqpoint{2.466435in}{1.478609in}}{\pgfqpoint{2.455836in}{1.474219in}}{\pgfqpoint{2.448023in}{1.466405in}}%
\pgfpathcurveto{\pgfqpoint{2.440209in}{1.458591in}}{\pgfqpoint{2.435819in}{1.447992in}}{\pgfqpoint{2.435819in}{1.436942in}}%
\pgfpathcurveto{\pgfqpoint{2.435819in}{1.425892in}}{\pgfqpoint{2.440209in}{1.415293in}}{\pgfqpoint{2.448023in}{1.407479in}}%
\pgfpathcurveto{\pgfqpoint{2.455836in}{1.399666in}}{\pgfqpoint{2.466435in}{1.395275in}}{\pgfqpoint{2.477486in}{1.395275in}}%
\pgfpathclose%
\pgfusepath{stroke,fill}%
\end{pgfscope}%
\begin{pgfscope}%
\pgfpathrectangle{\pgfqpoint{0.787074in}{0.548769in}}{\pgfqpoint{5.062926in}{3.102590in}}%
\pgfusepath{clip}%
\pgfsetbuttcap%
\pgfsetroundjoin%
\definecolor{currentfill}{rgb}{0.121569,0.466667,0.705882}%
\pgfsetfillcolor{currentfill}%
\pgfsetlinewidth{1.003750pt}%
\definecolor{currentstroke}{rgb}{0.121569,0.466667,0.705882}%
\pgfsetstrokecolor{currentstroke}%
\pgfsetdash{}{0pt}%
\pgfpathmoveto{\pgfqpoint{1.078359in}{0.648350in}}%
\pgfpathcurveto{\pgfqpoint{1.089409in}{0.648350in}}{\pgfqpoint{1.100008in}{0.652741in}}{\pgfqpoint{1.107822in}{0.660554in}}%
\pgfpathcurveto{\pgfqpoint{1.115636in}{0.668368in}}{\pgfqpoint{1.120026in}{0.678967in}}{\pgfqpoint{1.120026in}{0.690017in}}%
\pgfpathcurveto{\pgfqpoint{1.120026in}{0.701067in}}{\pgfqpoint{1.115636in}{0.711666in}}{\pgfqpoint{1.107822in}{0.719480in}}%
\pgfpathcurveto{\pgfqpoint{1.100008in}{0.727293in}}{\pgfqpoint{1.089409in}{0.731684in}}{\pgfqpoint{1.078359in}{0.731684in}}%
\pgfpathcurveto{\pgfqpoint{1.067309in}{0.731684in}}{\pgfqpoint{1.056710in}{0.727293in}}{\pgfqpoint{1.048896in}{0.719480in}}%
\pgfpathcurveto{\pgfqpoint{1.041083in}{0.711666in}}{\pgfqpoint{1.036693in}{0.701067in}}{\pgfqpoint{1.036693in}{0.690017in}}%
\pgfpathcurveto{\pgfqpoint{1.036693in}{0.678967in}}{\pgfqpoint{1.041083in}{0.668368in}}{\pgfqpoint{1.048896in}{0.660554in}}%
\pgfpathcurveto{\pgfqpoint{1.056710in}{0.652741in}}{\pgfqpoint{1.067309in}{0.648350in}}{\pgfqpoint{1.078359in}{0.648350in}}%
\pgfpathclose%
\pgfusepath{stroke,fill}%
\end{pgfscope}%
\begin{pgfscope}%
\pgfpathrectangle{\pgfqpoint{0.787074in}{0.548769in}}{\pgfqpoint{5.062926in}{3.102590in}}%
\pgfusepath{clip}%
\pgfsetbuttcap%
\pgfsetroundjoin%
\definecolor{currentfill}{rgb}{1.000000,0.498039,0.054902}%
\pgfsetfillcolor{currentfill}%
\pgfsetlinewidth{1.003750pt}%
\definecolor{currentstroke}{rgb}{1.000000,0.498039,0.054902}%
\pgfsetstrokecolor{currentstroke}%
\pgfsetdash{}{0pt}%
\pgfpathmoveto{\pgfqpoint{1.785019in}{1.744372in}}%
\pgfpathcurveto{\pgfqpoint{1.796069in}{1.744372in}}{\pgfqpoint{1.806668in}{1.748762in}}{\pgfqpoint{1.814481in}{1.756576in}}%
\pgfpathcurveto{\pgfqpoint{1.822295in}{1.764389in}}{\pgfqpoint{1.826685in}{1.774988in}}{\pgfqpoint{1.826685in}{1.786038in}}%
\pgfpathcurveto{\pgfqpoint{1.826685in}{1.797089in}}{\pgfqpoint{1.822295in}{1.807688in}}{\pgfqpoint{1.814481in}{1.815501in}}%
\pgfpathcurveto{\pgfqpoint{1.806668in}{1.823315in}}{\pgfqpoint{1.796069in}{1.827705in}}{\pgfqpoint{1.785019in}{1.827705in}}%
\pgfpathcurveto{\pgfqpoint{1.773968in}{1.827705in}}{\pgfqpoint{1.763369in}{1.823315in}}{\pgfqpoint{1.755556in}{1.815501in}}%
\pgfpathcurveto{\pgfqpoint{1.747742in}{1.807688in}}{\pgfqpoint{1.743352in}{1.797089in}}{\pgfqpoint{1.743352in}{1.786038in}}%
\pgfpathcurveto{\pgfqpoint{1.743352in}{1.774988in}}{\pgfqpoint{1.747742in}{1.764389in}}{\pgfqpoint{1.755556in}{1.756576in}}%
\pgfpathcurveto{\pgfqpoint{1.763369in}{1.748762in}}{\pgfqpoint{1.773968in}{1.744372in}}{\pgfqpoint{1.785019in}{1.744372in}}%
\pgfpathclose%
\pgfusepath{stroke,fill}%
\end{pgfscope}%
\begin{pgfscope}%
\pgfpathrectangle{\pgfqpoint{0.787074in}{0.548769in}}{\pgfqpoint{5.062926in}{3.102590in}}%
\pgfusepath{clip}%
\pgfsetbuttcap%
\pgfsetroundjoin%
\definecolor{currentfill}{rgb}{1.000000,0.498039,0.054902}%
\pgfsetfillcolor{currentfill}%
\pgfsetlinewidth{1.003750pt}%
\definecolor{currentstroke}{rgb}{1.000000,0.498039,0.054902}%
\pgfsetstrokecolor{currentstroke}%
\pgfsetdash{}{0pt}%
\pgfpathmoveto{\pgfqpoint{1.587847in}{2.684286in}}%
\pgfpathcurveto{\pgfqpoint{1.598897in}{2.684286in}}{\pgfqpoint{1.609496in}{2.688676in}}{\pgfqpoint{1.617309in}{2.696490in}}%
\pgfpathcurveto{\pgfqpoint{1.625123in}{2.704304in}}{\pgfqpoint{1.629513in}{2.714903in}}{\pgfqpoint{1.629513in}{2.725953in}}%
\pgfpathcurveto{\pgfqpoint{1.629513in}{2.737003in}}{\pgfqpoint{1.625123in}{2.747602in}}{\pgfqpoint{1.617309in}{2.755416in}}%
\pgfpathcurveto{\pgfqpoint{1.609496in}{2.763229in}}{\pgfqpoint{1.598897in}{2.767619in}}{\pgfqpoint{1.587847in}{2.767619in}}%
\pgfpathcurveto{\pgfqpoint{1.576796in}{2.767619in}}{\pgfqpoint{1.566197in}{2.763229in}}{\pgfqpoint{1.558384in}{2.755416in}}%
\pgfpathcurveto{\pgfqpoint{1.550570in}{2.747602in}}{\pgfqpoint{1.546180in}{2.737003in}}{\pgfqpoint{1.546180in}{2.725953in}}%
\pgfpathcurveto{\pgfqpoint{1.546180in}{2.714903in}}{\pgfqpoint{1.550570in}{2.704304in}}{\pgfqpoint{1.558384in}{2.696490in}}%
\pgfpathcurveto{\pgfqpoint{1.566197in}{2.688676in}}{\pgfqpoint{1.576796in}{2.684286in}}{\pgfqpoint{1.587847in}{2.684286in}}%
\pgfpathclose%
\pgfusepath{stroke,fill}%
\end{pgfscope}%
\begin{pgfscope}%
\pgfpathrectangle{\pgfqpoint{0.787074in}{0.548769in}}{\pgfqpoint{5.062926in}{3.102590in}}%
\pgfusepath{clip}%
\pgfsetbuttcap%
\pgfsetroundjoin%
\definecolor{currentfill}{rgb}{1.000000,0.498039,0.054902}%
\pgfsetfillcolor{currentfill}%
\pgfsetlinewidth{1.003750pt}%
\definecolor{currentstroke}{rgb}{1.000000,0.498039,0.054902}%
\pgfsetstrokecolor{currentstroke}%
\pgfsetdash{}{0pt}%
\pgfpathmoveto{\pgfqpoint{1.909580in}{2.189411in}}%
\pgfpathcurveto{\pgfqpoint{1.920630in}{2.189411in}}{\pgfqpoint{1.931229in}{2.193801in}}{\pgfqpoint{1.939043in}{2.201614in}}%
\pgfpathcurveto{\pgfqpoint{1.946857in}{2.209428in}}{\pgfqpoint{1.951247in}{2.220027in}}{\pgfqpoint{1.951247in}{2.231077in}}%
\pgfpathcurveto{\pgfqpoint{1.951247in}{2.242127in}}{\pgfqpoint{1.946857in}{2.252726in}}{\pgfqpoint{1.939043in}{2.260540in}}%
\pgfpathcurveto{\pgfqpoint{1.931229in}{2.268354in}}{\pgfqpoint{1.920630in}{2.272744in}}{\pgfqpoint{1.909580in}{2.272744in}}%
\pgfpathcurveto{\pgfqpoint{1.898530in}{2.272744in}}{\pgfqpoint{1.887931in}{2.268354in}}{\pgfqpoint{1.880117in}{2.260540in}}%
\pgfpathcurveto{\pgfqpoint{1.872304in}{2.252726in}}{\pgfqpoint{1.867913in}{2.242127in}}{\pgfqpoint{1.867913in}{2.231077in}}%
\pgfpathcurveto{\pgfqpoint{1.867913in}{2.220027in}}{\pgfqpoint{1.872304in}{2.209428in}}{\pgfqpoint{1.880117in}{2.201614in}}%
\pgfpathcurveto{\pgfqpoint{1.887931in}{2.193801in}}{\pgfqpoint{1.898530in}{2.189411in}}{\pgfqpoint{1.909580in}{2.189411in}}%
\pgfpathclose%
\pgfusepath{stroke,fill}%
\end{pgfscope}%
\begin{pgfscope}%
\pgfpathrectangle{\pgfqpoint{0.787074in}{0.548769in}}{\pgfqpoint{5.062926in}{3.102590in}}%
\pgfusepath{clip}%
\pgfsetbuttcap%
\pgfsetroundjoin%
\definecolor{currentfill}{rgb}{1.000000,0.498039,0.054902}%
\pgfsetfillcolor{currentfill}%
\pgfsetlinewidth{1.003750pt}%
\definecolor{currentstroke}{rgb}{1.000000,0.498039,0.054902}%
\pgfsetstrokecolor{currentstroke}%
\pgfsetdash{}{0pt}%
\pgfpathmoveto{\pgfqpoint{2.067517in}{2.951689in}}%
\pgfpathcurveto{\pgfqpoint{2.078567in}{2.951689in}}{\pgfqpoint{2.089166in}{2.956079in}}{\pgfqpoint{2.096980in}{2.963893in}}%
\pgfpathcurveto{\pgfqpoint{2.104794in}{2.971707in}}{\pgfqpoint{2.109184in}{2.982306in}}{\pgfqpoint{2.109184in}{2.993356in}}%
\pgfpathcurveto{\pgfqpoint{2.109184in}{3.004406in}}{\pgfqpoint{2.104794in}{3.015005in}}{\pgfqpoint{2.096980in}{3.022818in}}%
\pgfpathcurveto{\pgfqpoint{2.089166in}{3.030632in}}{\pgfqpoint{2.078567in}{3.035022in}}{\pgfqpoint{2.067517in}{3.035022in}}%
\pgfpathcurveto{\pgfqpoint{2.056467in}{3.035022in}}{\pgfqpoint{2.045868in}{3.030632in}}{\pgfqpoint{2.038055in}{3.022818in}}%
\pgfpathcurveto{\pgfqpoint{2.030241in}{3.015005in}}{\pgfqpoint{2.025851in}{3.004406in}}{\pgfqpoint{2.025851in}{2.993356in}}%
\pgfpathcurveto{\pgfqpoint{2.025851in}{2.982306in}}{\pgfqpoint{2.030241in}{2.971707in}}{\pgfqpoint{2.038055in}{2.963893in}}%
\pgfpathcurveto{\pgfqpoint{2.045868in}{2.956079in}}{\pgfqpoint{2.056467in}{2.951689in}}{\pgfqpoint{2.067517in}{2.951689in}}%
\pgfpathclose%
\pgfusepath{stroke,fill}%
\end{pgfscope}%
\begin{pgfscope}%
\pgfpathrectangle{\pgfqpoint{0.787074in}{0.548769in}}{\pgfqpoint{5.062926in}{3.102590in}}%
\pgfusepath{clip}%
\pgfsetbuttcap%
\pgfsetroundjoin%
\definecolor{currentfill}{rgb}{1.000000,0.498039,0.054902}%
\pgfsetfillcolor{currentfill}%
\pgfsetlinewidth{1.003750pt}%
\definecolor{currentstroke}{rgb}{1.000000,0.498039,0.054902}%
\pgfsetstrokecolor{currentstroke}%
\pgfsetdash{}{0pt}%
\pgfpathmoveto{\pgfqpoint{1.349487in}{2.956805in}}%
\pgfpathcurveto{\pgfqpoint{1.360537in}{2.956805in}}{\pgfqpoint{1.371136in}{2.961195in}}{\pgfqpoint{1.378950in}{2.969009in}}%
\pgfpathcurveto{\pgfqpoint{1.386763in}{2.976822in}}{\pgfqpoint{1.391154in}{2.987421in}}{\pgfqpoint{1.391154in}{2.998472in}}%
\pgfpathcurveto{\pgfqpoint{1.391154in}{3.009522in}}{\pgfqpoint{1.386763in}{3.020121in}}{\pgfqpoint{1.378950in}{3.027934in}}%
\pgfpathcurveto{\pgfqpoint{1.371136in}{3.035748in}}{\pgfqpoint{1.360537in}{3.040138in}}{\pgfqpoint{1.349487in}{3.040138in}}%
\pgfpathcurveto{\pgfqpoint{1.338437in}{3.040138in}}{\pgfqpoint{1.327838in}{3.035748in}}{\pgfqpoint{1.320024in}{3.027934in}}%
\pgfpathcurveto{\pgfqpoint{1.312210in}{3.020121in}}{\pgfqpoint{1.307820in}{3.009522in}}{\pgfqpoint{1.307820in}{2.998472in}}%
\pgfpathcurveto{\pgfqpoint{1.307820in}{2.987421in}}{\pgfqpoint{1.312210in}{2.976822in}}{\pgfqpoint{1.320024in}{2.969009in}}%
\pgfpathcurveto{\pgfqpoint{1.327838in}{2.961195in}}{\pgfqpoint{1.338437in}{2.956805in}}{\pgfqpoint{1.349487in}{2.956805in}}%
\pgfpathclose%
\pgfusepath{stroke,fill}%
\end{pgfscope}%
\begin{pgfscope}%
\pgfpathrectangle{\pgfqpoint{0.787074in}{0.548769in}}{\pgfqpoint{5.062926in}{3.102590in}}%
\pgfusepath{clip}%
\pgfsetbuttcap%
\pgfsetroundjoin%
\definecolor{currentfill}{rgb}{1.000000,0.498039,0.054902}%
\pgfsetfillcolor{currentfill}%
\pgfsetlinewidth{1.003750pt}%
\definecolor{currentstroke}{rgb}{1.000000,0.498039,0.054902}%
\pgfsetstrokecolor{currentstroke}%
\pgfsetdash{}{0pt}%
\pgfpathmoveto{\pgfqpoint{1.618531in}{2.890755in}}%
\pgfpathcurveto{\pgfqpoint{1.629581in}{2.890755in}}{\pgfqpoint{1.640180in}{2.895145in}}{\pgfqpoint{1.647994in}{2.902959in}}%
\pgfpathcurveto{\pgfqpoint{1.655808in}{2.910773in}}{\pgfqpoint{1.660198in}{2.921372in}}{\pgfqpoint{1.660198in}{2.932422in}}%
\pgfpathcurveto{\pgfqpoint{1.660198in}{2.943472in}}{\pgfqpoint{1.655808in}{2.954071in}}{\pgfqpoint{1.647994in}{2.961885in}}%
\pgfpathcurveto{\pgfqpoint{1.640180in}{2.969698in}}{\pgfqpoint{1.629581in}{2.974089in}}{\pgfqpoint{1.618531in}{2.974089in}}%
\pgfpathcurveto{\pgfqpoint{1.607481in}{2.974089in}}{\pgfqpoint{1.596882in}{2.969698in}}{\pgfqpoint{1.589069in}{2.961885in}}%
\pgfpathcurveto{\pgfqpoint{1.581255in}{2.954071in}}{\pgfqpoint{1.576865in}{2.943472in}}{\pgfqpoint{1.576865in}{2.932422in}}%
\pgfpathcurveto{\pgfqpoint{1.576865in}{2.921372in}}{\pgfqpoint{1.581255in}{2.910773in}}{\pgfqpoint{1.589069in}{2.902959in}}%
\pgfpathcurveto{\pgfqpoint{1.596882in}{2.895145in}}{\pgfqpoint{1.607481in}{2.890755in}}{\pgfqpoint{1.618531in}{2.890755in}}%
\pgfpathclose%
\pgfusepath{stroke,fill}%
\end{pgfscope}%
\begin{pgfscope}%
\pgfpathrectangle{\pgfqpoint{0.787074in}{0.548769in}}{\pgfqpoint{5.062926in}{3.102590in}}%
\pgfusepath{clip}%
\pgfsetbuttcap%
\pgfsetroundjoin%
\definecolor{currentfill}{rgb}{0.121569,0.466667,0.705882}%
\pgfsetfillcolor{currentfill}%
\pgfsetlinewidth{1.003750pt}%
\definecolor{currentstroke}{rgb}{0.121569,0.466667,0.705882}%
\pgfsetstrokecolor{currentstroke}%
\pgfsetdash{}{0pt}%
\pgfpathmoveto{\pgfqpoint{1.322665in}{2.074998in}}%
\pgfpathcurveto{\pgfqpoint{1.333715in}{2.074998in}}{\pgfqpoint{1.344314in}{2.079388in}}{\pgfqpoint{1.352128in}{2.087202in}}%
\pgfpathcurveto{\pgfqpoint{1.359941in}{2.095016in}}{\pgfqpoint{1.364332in}{2.105615in}}{\pgfqpoint{1.364332in}{2.116665in}}%
\pgfpathcurveto{\pgfqpoint{1.364332in}{2.127715in}}{\pgfqpoint{1.359941in}{2.138314in}}{\pgfqpoint{1.352128in}{2.146127in}}%
\pgfpathcurveto{\pgfqpoint{1.344314in}{2.153941in}}{\pgfqpoint{1.333715in}{2.158331in}}{\pgfqpoint{1.322665in}{2.158331in}}%
\pgfpathcurveto{\pgfqpoint{1.311615in}{2.158331in}}{\pgfqpoint{1.301016in}{2.153941in}}{\pgfqpoint{1.293202in}{2.146127in}}%
\pgfpathcurveto{\pgfqpoint{1.285389in}{2.138314in}}{\pgfqpoint{1.280998in}{2.127715in}}{\pgfqpoint{1.280998in}{2.116665in}}%
\pgfpathcurveto{\pgfqpoint{1.280998in}{2.105615in}}{\pgfqpoint{1.285389in}{2.095016in}}{\pgfqpoint{1.293202in}{2.087202in}}%
\pgfpathcurveto{\pgfqpoint{1.301016in}{2.079388in}}{\pgfqpoint{1.311615in}{2.074998in}}{\pgfqpoint{1.322665in}{2.074998in}}%
\pgfpathclose%
\pgfusepath{stroke,fill}%
\end{pgfscope}%
\begin{pgfscope}%
\pgfpathrectangle{\pgfqpoint{0.787074in}{0.548769in}}{\pgfqpoint{5.062926in}{3.102590in}}%
\pgfusepath{clip}%
\pgfsetbuttcap%
\pgfsetroundjoin%
\definecolor{currentfill}{rgb}{1.000000,0.498039,0.054902}%
\pgfsetfillcolor{currentfill}%
\pgfsetlinewidth{1.003750pt}%
\definecolor{currentstroke}{rgb}{1.000000,0.498039,0.054902}%
\pgfsetstrokecolor{currentstroke}%
\pgfsetdash{}{0pt}%
\pgfpathmoveto{\pgfqpoint{1.398530in}{2.746524in}}%
\pgfpathcurveto{\pgfqpoint{1.409580in}{2.746524in}}{\pgfqpoint{1.420179in}{2.750915in}}{\pgfqpoint{1.427993in}{2.758728in}}%
\pgfpathcurveto{\pgfqpoint{1.435807in}{2.766542in}}{\pgfqpoint{1.440197in}{2.777141in}}{\pgfqpoint{1.440197in}{2.788191in}}%
\pgfpathcurveto{\pgfqpoint{1.440197in}{2.799241in}}{\pgfqpoint{1.435807in}{2.809840in}}{\pgfqpoint{1.427993in}{2.817654in}}%
\pgfpathcurveto{\pgfqpoint{1.420179in}{2.825468in}}{\pgfqpoint{1.409580in}{2.829858in}}{\pgfqpoint{1.398530in}{2.829858in}}%
\pgfpathcurveto{\pgfqpoint{1.387480in}{2.829858in}}{\pgfqpoint{1.376881in}{2.825468in}}{\pgfqpoint{1.369068in}{2.817654in}}%
\pgfpathcurveto{\pgfqpoint{1.361254in}{2.809840in}}{\pgfqpoint{1.356864in}{2.799241in}}{\pgfqpoint{1.356864in}{2.788191in}}%
\pgfpathcurveto{\pgfqpoint{1.356864in}{2.777141in}}{\pgfqpoint{1.361254in}{2.766542in}}{\pgfqpoint{1.369068in}{2.758728in}}%
\pgfpathcurveto{\pgfqpoint{1.376881in}{2.750915in}}{\pgfqpoint{1.387480in}{2.746524in}}{\pgfqpoint{1.398530in}{2.746524in}}%
\pgfpathclose%
\pgfusepath{stroke,fill}%
\end{pgfscope}%
\begin{pgfscope}%
\pgfpathrectangle{\pgfqpoint{0.787074in}{0.548769in}}{\pgfqpoint{5.062926in}{3.102590in}}%
\pgfusepath{clip}%
\pgfsetbuttcap%
\pgfsetroundjoin%
\definecolor{currentfill}{rgb}{0.121569,0.466667,0.705882}%
\pgfsetfillcolor{currentfill}%
\pgfsetlinewidth{1.003750pt}%
\definecolor{currentstroke}{rgb}{0.121569,0.466667,0.705882}%
\pgfsetstrokecolor{currentstroke}%
\pgfsetdash{}{0pt}%
\pgfpathmoveto{\pgfqpoint{2.632906in}{1.732307in}}%
\pgfpathcurveto{\pgfqpoint{2.643956in}{1.732307in}}{\pgfqpoint{2.654555in}{1.736697in}}{\pgfqpoint{2.662368in}{1.744511in}}%
\pgfpathcurveto{\pgfqpoint{2.670182in}{1.752325in}}{\pgfqpoint{2.674572in}{1.762924in}}{\pgfqpoint{2.674572in}{1.773974in}}%
\pgfpathcurveto{\pgfqpoint{2.674572in}{1.785024in}}{\pgfqpoint{2.670182in}{1.795623in}}{\pgfqpoint{2.662368in}{1.803437in}}%
\pgfpathcurveto{\pgfqpoint{2.654555in}{1.811250in}}{\pgfqpoint{2.643956in}{1.815640in}}{\pgfqpoint{2.632906in}{1.815640in}}%
\pgfpathcurveto{\pgfqpoint{2.621855in}{1.815640in}}{\pgfqpoint{2.611256in}{1.811250in}}{\pgfqpoint{2.603443in}{1.803437in}}%
\pgfpathcurveto{\pgfqpoint{2.595629in}{1.795623in}}{\pgfqpoint{2.591239in}{1.785024in}}{\pgfqpoint{2.591239in}{1.773974in}}%
\pgfpathcurveto{\pgfqpoint{2.591239in}{1.762924in}}{\pgfqpoint{2.595629in}{1.752325in}}{\pgfqpoint{2.603443in}{1.744511in}}%
\pgfpathcurveto{\pgfqpoint{2.611256in}{1.736697in}}{\pgfqpoint{2.621855in}{1.732307in}}{\pgfqpoint{2.632906in}{1.732307in}}%
\pgfpathclose%
\pgfusepath{stroke,fill}%
\end{pgfscope}%
\begin{pgfscope}%
\pgfpathrectangle{\pgfqpoint{0.787074in}{0.548769in}}{\pgfqpoint{5.062926in}{3.102590in}}%
\pgfusepath{clip}%
\pgfsetbuttcap%
\pgfsetroundjoin%
\definecolor{currentfill}{rgb}{0.121569,0.466667,0.705882}%
\pgfsetfillcolor{currentfill}%
\pgfsetlinewidth{1.003750pt}%
\definecolor{currentstroke}{rgb}{0.121569,0.466667,0.705882}%
\pgfsetstrokecolor{currentstroke}%
\pgfsetdash{}{0pt}%
\pgfpathmoveto{\pgfqpoint{2.108792in}{2.443186in}}%
\pgfpathcurveto{\pgfqpoint{2.119842in}{2.443186in}}{\pgfqpoint{2.130441in}{2.447576in}}{\pgfqpoint{2.138255in}{2.455390in}}%
\pgfpathcurveto{\pgfqpoint{2.146068in}{2.463203in}}{\pgfqpoint{2.150459in}{2.473802in}}{\pgfqpoint{2.150459in}{2.484852in}}%
\pgfpathcurveto{\pgfqpoint{2.150459in}{2.495903in}}{\pgfqpoint{2.146068in}{2.506502in}}{\pgfqpoint{2.138255in}{2.514315in}}%
\pgfpathcurveto{\pgfqpoint{2.130441in}{2.522129in}}{\pgfqpoint{2.119842in}{2.526519in}}{\pgfqpoint{2.108792in}{2.526519in}}%
\pgfpathcurveto{\pgfqpoint{2.097742in}{2.526519in}}{\pgfqpoint{2.087143in}{2.522129in}}{\pgfqpoint{2.079329in}{2.514315in}}%
\pgfpathcurveto{\pgfqpoint{2.071516in}{2.506502in}}{\pgfqpoint{2.067125in}{2.495903in}}{\pgfqpoint{2.067125in}{2.484852in}}%
\pgfpathcurveto{\pgfqpoint{2.067125in}{2.473802in}}{\pgfqpoint{2.071516in}{2.463203in}}{\pgfqpoint{2.079329in}{2.455390in}}%
\pgfpathcurveto{\pgfqpoint{2.087143in}{2.447576in}}{\pgfqpoint{2.097742in}{2.443186in}}{\pgfqpoint{2.108792in}{2.443186in}}%
\pgfpathclose%
\pgfusepath{stroke,fill}%
\end{pgfscope}%
\begin{pgfscope}%
\pgfpathrectangle{\pgfqpoint{0.787074in}{0.548769in}}{\pgfqpoint{5.062926in}{3.102590in}}%
\pgfusepath{clip}%
\pgfsetbuttcap%
\pgfsetroundjoin%
\definecolor{currentfill}{rgb}{0.121569,0.466667,0.705882}%
\pgfsetfillcolor{currentfill}%
\pgfsetlinewidth{1.003750pt}%
\definecolor{currentstroke}{rgb}{0.121569,0.466667,0.705882}%
\pgfsetstrokecolor{currentstroke}%
\pgfsetdash{}{0pt}%
\pgfpathmoveto{\pgfqpoint{1.706332in}{1.419741in}}%
\pgfpathcurveto{\pgfqpoint{1.717382in}{1.419741in}}{\pgfqpoint{1.727981in}{1.424131in}}{\pgfqpoint{1.735795in}{1.431945in}}%
\pgfpathcurveto{\pgfqpoint{1.743608in}{1.439759in}}{\pgfqpoint{1.747999in}{1.450358in}}{\pgfqpoint{1.747999in}{1.461408in}}%
\pgfpathcurveto{\pgfqpoint{1.747999in}{1.472458in}}{\pgfqpoint{1.743608in}{1.483057in}}{\pgfqpoint{1.735795in}{1.490870in}}%
\pgfpathcurveto{\pgfqpoint{1.727981in}{1.498684in}}{\pgfqpoint{1.717382in}{1.503074in}}{\pgfqpoint{1.706332in}{1.503074in}}%
\pgfpathcurveto{\pgfqpoint{1.695282in}{1.503074in}}{\pgfqpoint{1.684683in}{1.498684in}}{\pgfqpoint{1.676869in}{1.490870in}}%
\pgfpathcurveto{\pgfqpoint{1.669056in}{1.483057in}}{\pgfqpoint{1.664665in}{1.472458in}}{\pgfqpoint{1.664665in}{1.461408in}}%
\pgfpathcurveto{\pgfqpoint{1.664665in}{1.450358in}}{\pgfqpoint{1.669056in}{1.439759in}}{\pgfqpoint{1.676869in}{1.431945in}}%
\pgfpathcurveto{\pgfqpoint{1.684683in}{1.424131in}}{\pgfqpoint{1.695282in}{1.419741in}}{\pgfqpoint{1.706332in}{1.419741in}}%
\pgfpathclose%
\pgfusepath{stroke,fill}%
\end{pgfscope}%
\begin{pgfscope}%
\pgfpathrectangle{\pgfqpoint{0.787074in}{0.548769in}}{\pgfqpoint{5.062926in}{3.102590in}}%
\pgfusepath{clip}%
\pgfsetbuttcap%
\pgfsetroundjoin%
\definecolor{currentfill}{rgb}{1.000000,0.498039,0.054902}%
\pgfsetfillcolor{currentfill}%
\pgfsetlinewidth{1.003750pt}%
\definecolor{currentstroke}{rgb}{1.000000,0.498039,0.054902}%
\pgfsetstrokecolor{currentstroke}%
\pgfsetdash{}{0pt}%
\pgfpathmoveto{\pgfqpoint{1.506339in}{2.589074in}}%
\pgfpathcurveto{\pgfqpoint{1.517389in}{2.589074in}}{\pgfqpoint{1.527988in}{2.593464in}}{\pgfqpoint{1.535802in}{2.601278in}}%
\pgfpathcurveto{\pgfqpoint{1.543615in}{2.609092in}}{\pgfqpoint{1.548006in}{2.619691in}}{\pgfqpoint{1.548006in}{2.630741in}}%
\pgfpathcurveto{\pgfqpoint{1.548006in}{2.641791in}}{\pgfqpoint{1.543615in}{2.652390in}}{\pgfqpoint{1.535802in}{2.660204in}}%
\pgfpathcurveto{\pgfqpoint{1.527988in}{2.668017in}}{\pgfqpoint{1.517389in}{2.672407in}}{\pgfqpoint{1.506339in}{2.672407in}}%
\pgfpathcurveto{\pgfqpoint{1.495289in}{2.672407in}}{\pgfqpoint{1.484690in}{2.668017in}}{\pgfqpoint{1.476876in}{2.660204in}}%
\pgfpathcurveto{\pgfqpoint{1.469063in}{2.652390in}}{\pgfqpoint{1.464672in}{2.641791in}}{\pgfqpoint{1.464672in}{2.630741in}}%
\pgfpathcurveto{\pgfqpoint{1.464672in}{2.619691in}}{\pgfqpoint{1.469063in}{2.609092in}}{\pgfqpoint{1.476876in}{2.601278in}}%
\pgfpathcurveto{\pgfqpoint{1.484690in}{2.593464in}}{\pgfqpoint{1.495289in}{2.589074in}}{\pgfqpoint{1.506339in}{2.589074in}}%
\pgfpathclose%
\pgfusepath{stroke,fill}%
\end{pgfscope}%
\begin{pgfscope}%
\pgfpathrectangle{\pgfqpoint{0.787074in}{0.548769in}}{\pgfqpoint{5.062926in}{3.102590in}}%
\pgfusepath{clip}%
\pgfsetbuttcap%
\pgfsetroundjoin%
\definecolor{currentfill}{rgb}{1.000000,0.498039,0.054902}%
\pgfsetfillcolor{currentfill}%
\pgfsetlinewidth{1.003750pt}%
\definecolor{currentstroke}{rgb}{1.000000,0.498039,0.054902}%
\pgfsetstrokecolor{currentstroke}%
\pgfsetdash{}{0pt}%
\pgfpathmoveto{\pgfqpoint{1.383210in}{2.590352in}}%
\pgfpathcurveto{\pgfqpoint{1.394260in}{2.590352in}}{\pgfqpoint{1.404859in}{2.594742in}}{\pgfqpoint{1.412672in}{2.602556in}}%
\pgfpathcurveto{\pgfqpoint{1.420486in}{2.610369in}}{\pgfqpoint{1.424876in}{2.620968in}}{\pgfqpoint{1.424876in}{2.632018in}}%
\pgfpathcurveto{\pgfqpoint{1.424876in}{2.643068in}}{\pgfqpoint{1.420486in}{2.653668in}}{\pgfqpoint{1.412672in}{2.661481in}}%
\pgfpathcurveto{\pgfqpoint{1.404859in}{2.669295in}}{\pgfqpoint{1.394260in}{2.673685in}}{\pgfqpoint{1.383210in}{2.673685in}}%
\pgfpathcurveto{\pgfqpoint{1.372160in}{2.673685in}}{\pgfqpoint{1.361561in}{2.669295in}}{\pgfqpoint{1.353747in}{2.661481in}}%
\pgfpathcurveto{\pgfqpoint{1.345933in}{2.653668in}}{\pgfqpoint{1.341543in}{2.643068in}}{\pgfqpoint{1.341543in}{2.632018in}}%
\pgfpathcurveto{\pgfqpoint{1.341543in}{2.620968in}}{\pgfqpoint{1.345933in}{2.610369in}}{\pgfqpoint{1.353747in}{2.602556in}}%
\pgfpathcurveto{\pgfqpoint{1.361561in}{2.594742in}}{\pgfqpoint{1.372160in}{2.590352in}}{\pgfqpoint{1.383210in}{2.590352in}}%
\pgfpathclose%
\pgfusepath{stroke,fill}%
\end{pgfscope}%
\begin{pgfscope}%
\pgfpathrectangle{\pgfqpoint{0.787074in}{0.548769in}}{\pgfqpoint{5.062926in}{3.102590in}}%
\pgfusepath{clip}%
\pgfsetbuttcap%
\pgfsetroundjoin%
\definecolor{currentfill}{rgb}{1.000000,0.498039,0.054902}%
\pgfsetfillcolor{currentfill}%
\pgfsetlinewidth{1.003750pt}%
\definecolor{currentstroke}{rgb}{1.000000,0.498039,0.054902}%
\pgfsetstrokecolor{currentstroke}%
\pgfsetdash{}{0pt}%
\pgfpathmoveto{\pgfqpoint{1.398530in}{2.993235in}}%
\pgfpathcurveto{\pgfqpoint{1.409580in}{2.993235in}}{\pgfqpoint{1.420179in}{2.997625in}}{\pgfqpoint{1.427993in}{3.005439in}}%
\pgfpathcurveto{\pgfqpoint{1.435807in}{3.013252in}}{\pgfqpoint{1.440197in}{3.023851in}}{\pgfqpoint{1.440197in}{3.034902in}}%
\pgfpathcurveto{\pgfqpoint{1.440197in}{3.045952in}}{\pgfqpoint{1.435807in}{3.056551in}}{\pgfqpoint{1.427993in}{3.064364in}}%
\pgfpathcurveto{\pgfqpoint{1.420179in}{3.072178in}}{\pgfqpoint{1.409580in}{3.076568in}}{\pgfqpoint{1.398530in}{3.076568in}}%
\pgfpathcurveto{\pgfqpoint{1.387480in}{3.076568in}}{\pgfqpoint{1.376881in}{3.072178in}}{\pgfqpoint{1.369068in}{3.064364in}}%
\pgfpathcurveto{\pgfqpoint{1.361254in}{3.056551in}}{\pgfqpoint{1.356864in}{3.045952in}}{\pgfqpoint{1.356864in}{3.034902in}}%
\pgfpathcurveto{\pgfqpoint{1.356864in}{3.023851in}}{\pgfqpoint{1.361254in}{3.013252in}}{\pgfqpoint{1.369068in}{3.005439in}}%
\pgfpathcurveto{\pgfqpoint{1.376881in}{2.997625in}}{\pgfqpoint{1.387480in}{2.993235in}}{\pgfqpoint{1.398530in}{2.993235in}}%
\pgfpathclose%
\pgfusepath{stroke,fill}%
\end{pgfscope}%
\begin{pgfscope}%
\pgfpathrectangle{\pgfqpoint{0.787074in}{0.548769in}}{\pgfqpoint{5.062926in}{3.102590in}}%
\pgfusepath{clip}%
\pgfsetbuttcap%
\pgfsetroundjoin%
\definecolor{currentfill}{rgb}{0.121569,0.466667,0.705882}%
\pgfsetfillcolor{currentfill}%
\pgfsetlinewidth{1.003750pt}%
\definecolor{currentstroke}{rgb}{0.121569,0.466667,0.705882}%
\pgfsetstrokecolor{currentstroke}%
\pgfsetdash{}{0pt}%
\pgfpathmoveto{\pgfqpoint{1.021113in}{0.666848in}}%
\pgfpathcurveto{\pgfqpoint{1.032163in}{0.666848in}}{\pgfqpoint{1.042762in}{0.671238in}}{\pgfqpoint{1.050576in}{0.679052in}}%
\pgfpathcurveto{\pgfqpoint{1.058389in}{0.686866in}}{\pgfqpoint{1.062780in}{0.697465in}}{\pgfqpoint{1.062780in}{0.708515in}}%
\pgfpathcurveto{\pgfqpoint{1.062780in}{0.719565in}}{\pgfqpoint{1.058389in}{0.730164in}}{\pgfqpoint{1.050576in}{0.737978in}}%
\pgfpathcurveto{\pgfqpoint{1.042762in}{0.745791in}}{\pgfqpoint{1.032163in}{0.750182in}}{\pgfqpoint{1.021113in}{0.750182in}}%
\pgfpathcurveto{\pgfqpoint{1.010063in}{0.750182in}}{\pgfqpoint{0.999464in}{0.745791in}}{\pgfqpoint{0.991650in}{0.737978in}}%
\pgfpathcurveto{\pgfqpoint{0.983837in}{0.730164in}}{\pgfqpoint{0.979446in}{0.719565in}}{\pgfqpoint{0.979446in}{0.708515in}}%
\pgfpathcurveto{\pgfqpoint{0.979446in}{0.697465in}}{\pgfqpoint{0.983837in}{0.686866in}}{\pgfqpoint{0.991650in}{0.679052in}}%
\pgfpathcurveto{\pgfqpoint{0.999464in}{0.671238in}}{\pgfqpoint{1.010063in}{0.666848in}}{\pgfqpoint{1.021113in}{0.666848in}}%
\pgfpathclose%
\pgfusepath{stroke,fill}%
\end{pgfscope}%
\begin{pgfscope}%
\pgfpathrectangle{\pgfqpoint{0.787074in}{0.548769in}}{\pgfqpoint{5.062926in}{3.102590in}}%
\pgfusepath{clip}%
\pgfsetbuttcap%
\pgfsetroundjoin%
\definecolor{currentfill}{rgb}{1.000000,0.498039,0.054902}%
\pgfsetfillcolor{currentfill}%
\pgfsetlinewidth{1.003750pt}%
\definecolor{currentstroke}{rgb}{1.000000,0.498039,0.054902}%
\pgfsetstrokecolor{currentstroke}%
\pgfsetdash{}{0pt}%
\pgfpathmoveto{\pgfqpoint{1.883279in}{2.802044in}}%
\pgfpathcurveto{\pgfqpoint{1.894329in}{2.802044in}}{\pgfqpoint{1.904928in}{2.806435in}}{\pgfqpoint{1.912742in}{2.814248in}}%
\pgfpathcurveto{\pgfqpoint{1.920555in}{2.822062in}}{\pgfqpoint{1.924946in}{2.832661in}}{\pgfqpoint{1.924946in}{2.843711in}}%
\pgfpathcurveto{\pgfqpoint{1.924946in}{2.854761in}}{\pgfqpoint{1.920555in}{2.865360in}}{\pgfqpoint{1.912742in}{2.873174in}}%
\pgfpathcurveto{\pgfqpoint{1.904928in}{2.880987in}}{\pgfqpoint{1.894329in}{2.885378in}}{\pgfqpoint{1.883279in}{2.885378in}}%
\pgfpathcurveto{\pgfqpoint{1.872229in}{2.885378in}}{\pgfqpoint{1.861630in}{2.880987in}}{\pgfqpoint{1.853816in}{2.873174in}}%
\pgfpathcurveto{\pgfqpoint{1.846003in}{2.865360in}}{\pgfqpoint{1.841612in}{2.854761in}}{\pgfqpoint{1.841612in}{2.843711in}}%
\pgfpathcurveto{\pgfqpoint{1.841612in}{2.832661in}}{\pgfqpoint{1.846003in}{2.822062in}}{\pgfqpoint{1.853816in}{2.814248in}}%
\pgfpathcurveto{\pgfqpoint{1.861630in}{2.806435in}}{\pgfqpoint{1.872229in}{2.802044in}}{\pgfqpoint{1.883279in}{2.802044in}}%
\pgfpathclose%
\pgfusepath{stroke,fill}%
\end{pgfscope}%
\begin{pgfscope}%
\pgfpathrectangle{\pgfqpoint{0.787074in}{0.548769in}}{\pgfqpoint{5.062926in}{3.102590in}}%
\pgfusepath{clip}%
\pgfsetbuttcap%
\pgfsetroundjoin%
\definecolor{currentfill}{rgb}{0.121569,0.466667,0.705882}%
\pgfsetfillcolor{currentfill}%
\pgfsetlinewidth{1.003750pt}%
\definecolor{currentstroke}{rgb}{0.121569,0.466667,0.705882}%
\pgfsetstrokecolor{currentstroke}%
\pgfsetdash{}{0pt}%
\pgfpathmoveto{\pgfqpoint{1.188468in}{0.648244in}}%
\pgfpathcurveto{\pgfqpoint{1.199518in}{0.648244in}}{\pgfqpoint{1.210117in}{0.652634in}}{\pgfqpoint{1.217931in}{0.660448in}}%
\pgfpathcurveto{\pgfqpoint{1.225745in}{0.668261in}}{\pgfqpoint{1.230135in}{0.678860in}}{\pgfqpoint{1.230135in}{0.689911in}}%
\pgfpathcurveto{\pgfqpoint{1.230135in}{0.700961in}}{\pgfqpoint{1.225745in}{0.711560in}}{\pgfqpoint{1.217931in}{0.719373in}}%
\pgfpathcurveto{\pgfqpoint{1.210117in}{0.727187in}}{\pgfqpoint{1.199518in}{0.731577in}}{\pgfqpoint{1.188468in}{0.731577in}}%
\pgfpathcurveto{\pgfqpoint{1.177418in}{0.731577in}}{\pgfqpoint{1.166819in}{0.727187in}}{\pgfqpoint{1.159005in}{0.719373in}}%
\pgfpathcurveto{\pgfqpoint{1.151192in}{0.711560in}}{\pgfqpoint{1.146802in}{0.700961in}}{\pgfqpoint{1.146802in}{0.689911in}}%
\pgfpathcurveto{\pgfqpoint{1.146802in}{0.678860in}}{\pgfqpoint{1.151192in}{0.668261in}}{\pgfqpoint{1.159005in}{0.660448in}}%
\pgfpathcurveto{\pgfqpoint{1.166819in}{0.652634in}}{\pgfqpoint{1.177418in}{0.648244in}}{\pgfqpoint{1.188468in}{0.648244in}}%
\pgfpathclose%
\pgfusepath{stroke,fill}%
\end{pgfscope}%
\begin{pgfscope}%
\pgfpathrectangle{\pgfqpoint{0.787074in}{0.548769in}}{\pgfqpoint{5.062926in}{3.102590in}}%
\pgfusepath{clip}%
\pgfsetbuttcap%
\pgfsetroundjoin%
\definecolor{currentfill}{rgb}{1.000000,0.498039,0.054902}%
\pgfsetfillcolor{currentfill}%
\pgfsetlinewidth{1.003750pt}%
\definecolor{currentstroke}{rgb}{1.000000,0.498039,0.054902}%
\pgfsetstrokecolor{currentstroke}%
\pgfsetdash{}{0pt}%
\pgfpathmoveto{\pgfqpoint{1.853679in}{3.229318in}}%
\pgfpathcurveto{\pgfqpoint{1.864729in}{3.229318in}}{\pgfqpoint{1.875328in}{3.233708in}}{\pgfqpoint{1.883142in}{3.241522in}}%
\pgfpathcurveto{\pgfqpoint{1.890956in}{3.249335in}}{\pgfqpoint{1.895346in}{3.259934in}}{\pgfqpoint{1.895346in}{3.270984in}}%
\pgfpathcurveto{\pgfqpoint{1.895346in}{3.282035in}}{\pgfqpoint{1.890956in}{3.292634in}}{\pgfqpoint{1.883142in}{3.300447in}}%
\pgfpathcurveto{\pgfqpoint{1.875328in}{3.308261in}}{\pgfqpoint{1.864729in}{3.312651in}}{\pgfqpoint{1.853679in}{3.312651in}}%
\pgfpathcurveto{\pgfqpoint{1.842629in}{3.312651in}}{\pgfqpoint{1.832030in}{3.308261in}}{\pgfqpoint{1.824217in}{3.300447in}}%
\pgfpathcurveto{\pgfqpoint{1.816403in}{3.292634in}}{\pgfqpoint{1.812013in}{3.282035in}}{\pgfqpoint{1.812013in}{3.270984in}}%
\pgfpathcurveto{\pgfqpoint{1.812013in}{3.259934in}}{\pgfqpoint{1.816403in}{3.249335in}}{\pgfqpoint{1.824217in}{3.241522in}}%
\pgfpathcurveto{\pgfqpoint{1.832030in}{3.233708in}}{\pgfqpoint{1.842629in}{3.229318in}}{\pgfqpoint{1.853679in}{3.229318in}}%
\pgfpathclose%
\pgfusepath{stroke,fill}%
\end{pgfscope}%
\begin{pgfscope}%
\pgfpathrectangle{\pgfqpoint{0.787074in}{0.548769in}}{\pgfqpoint{5.062926in}{3.102590in}}%
\pgfusepath{clip}%
\pgfsetbuttcap%
\pgfsetroundjoin%
\definecolor{currentfill}{rgb}{0.121569,0.466667,0.705882}%
\pgfsetfillcolor{currentfill}%
\pgfsetlinewidth{1.003750pt}%
\definecolor{currentstroke}{rgb}{0.121569,0.466667,0.705882}%
\pgfsetstrokecolor{currentstroke}%
\pgfsetdash{}{0pt}%
\pgfpathmoveto{\pgfqpoint{2.339773in}{2.961491in}}%
\pgfpathcurveto{\pgfqpoint{2.350824in}{2.961491in}}{\pgfqpoint{2.361423in}{2.965882in}}{\pgfqpoint{2.369236in}{2.973695in}}%
\pgfpathcurveto{\pgfqpoint{2.377050in}{2.981509in}}{\pgfqpoint{2.381440in}{2.992108in}}{\pgfqpoint{2.381440in}{3.003158in}}%
\pgfpathcurveto{\pgfqpoint{2.381440in}{3.014208in}}{\pgfqpoint{2.377050in}{3.024807in}}{\pgfqpoint{2.369236in}{3.032621in}}%
\pgfpathcurveto{\pgfqpoint{2.361423in}{3.040434in}}{\pgfqpoint{2.350824in}{3.044825in}}{\pgfqpoint{2.339773in}{3.044825in}}%
\pgfpathcurveto{\pgfqpoint{2.328723in}{3.044825in}}{\pgfqpoint{2.318124in}{3.040434in}}{\pgfqpoint{2.310311in}{3.032621in}}%
\pgfpathcurveto{\pgfqpoint{2.302497in}{3.024807in}}{\pgfqpoint{2.298107in}{3.014208in}}{\pgfqpoint{2.298107in}{3.003158in}}%
\pgfpathcurveto{\pgfqpoint{2.298107in}{2.992108in}}{\pgfqpoint{2.302497in}{2.981509in}}{\pgfqpoint{2.310311in}{2.973695in}}%
\pgfpathcurveto{\pgfqpoint{2.318124in}{2.965882in}}{\pgfqpoint{2.328723in}{2.961491in}}{\pgfqpoint{2.339773in}{2.961491in}}%
\pgfpathclose%
\pgfusepath{stroke,fill}%
\end{pgfscope}%
\begin{pgfscope}%
\pgfpathrectangle{\pgfqpoint{0.787074in}{0.548769in}}{\pgfqpoint{5.062926in}{3.102590in}}%
\pgfusepath{clip}%
\pgfsetbuttcap%
\pgfsetroundjoin%
\definecolor{currentfill}{rgb}{1.000000,0.498039,0.054902}%
\pgfsetfillcolor{currentfill}%
\pgfsetlinewidth{1.003750pt}%
\definecolor{currentstroke}{rgb}{1.000000,0.498039,0.054902}%
\pgfsetstrokecolor{currentstroke}%
\pgfsetdash{}{0pt}%
\pgfpathmoveto{\pgfqpoint{1.468884in}{2.689616in}}%
\pgfpathcurveto{\pgfqpoint{1.479934in}{2.689616in}}{\pgfqpoint{1.490533in}{2.694006in}}{\pgfqpoint{1.498347in}{2.701820in}}%
\pgfpathcurveto{\pgfqpoint{1.506160in}{2.709634in}}{\pgfqpoint{1.510550in}{2.720233in}}{\pgfqpoint{1.510550in}{2.731283in}}%
\pgfpathcurveto{\pgfqpoint{1.510550in}{2.742333in}}{\pgfqpoint{1.506160in}{2.752932in}}{\pgfqpoint{1.498347in}{2.760746in}}%
\pgfpathcurveto{\pgfqpoint{1.490533in}{2.768559in}}{\pgfqpoint{1.479934in}{2.772950in}}{\pgfqpoint{1.468884in}{2.772950in}}%
\pgfpathcurveto{\pgfqpoint{1.457834in}{2.772950in}}{\pgfqpoint{1.447235in}{2.768559in}}{\pgfqpoint{1.439421in}{2.760746in}}%
\pgfpathcurveto{\pgfqpoint{1.431607in}{2.752932in}}{\pgfqpoint{1.427217in}{2.742333in}}{\pgfqpoint{1.427217in}{2.731283in}}%
\pgfpathcurveto{\pgfqpoint{1.427217in}{2.720233in}}{\pgfqpoint{1.431607in}{2.709634in}}{\pgfqpoint{1.439421in}{2.701820in}}%
\pgfpathcurveto{\pgfqpoint{1.447235in}{2.694006in}}{\pgfqpoint{1.457834in}{2.689616in}}{\pgfqpoint{1.468884in}{2.689616in}}%
\pgfpathclose%
\pgfusepath{stroke,fill}%
\end{pgfscope}%
\begin{pgfscope}%
\pgfpathrectangle{\pgfqpoint{0.787074in}{0.548769in}}{\pgfqpoint{5.062926in}{3.102590in}}%
\pgfusepath{clip}%
\pgfsetbuttcap%
\pgfsetroundjoin%
\definecolor{currentfill}{rgb}{1.000000,0.498039,0.054902}%
\pgfsetfillcolor{currentfill}%
\pgfsetlinewidth{1.003750pt}%
\definecolor{currentstroke}{rgb}{1.000000,0.498039,0.054902}%
\pgfsetstrokecolor{currentstroke}%
\pgfsetdash{}{0pt}%
\pgfpathmoveto{\pgfqpoint{1.923425in}{1.853687in}}%
\pgfpathcurveto{\pgfqpoint{1.934475in}{1.853687in}}{\pgfqpoint{1.945074in}{1.858077in}}{\pgfqpoint{1.952888in}{1.865891in}}%
\pgfpathcurveto{\pgfqpoint{1.960702in}{1.873705in}}{\pgfqpoint{1.965092in}{1.884304in}}{\pgfqpoint{1.965092in}{1.895354in}}%
\pgfpathcurveto{\pgfqpoint{1.965092in}{1.906404in}}{\pgfqpoint{1.960702in}{1.917003in}}{\pgfqpoint{1.952888in}{1.924817in}}%
\pgfpathcurveto{\pgfqpoint{1.945074in}{1.932630in}}{\pgfqpoint{1.934475in}{1.937021in}}{\pgfqpoint{1.923425in}{1.937021in}}%
\pgfpathcurveto{\pgfqpoint{1.912375in}{1.937021in}}{\pgfqpoint{1.901776in}{1.932630in}}{\pgfqpoint{1.893962in}{1.924817in}}%
\pgfpathcurveto{\pgfqpoint{1.886149in}{1.917003in}}{\pgfqpoint{1.881758in}{1.906404in}}{\pgfqpoint{1.881758in}{1.895354in}}%
\pgfpathcurveto{\pgfqpoint{1.881758in}{1.884304in}}{\pgfqpoint{1.886149in}{1.873705in}}{\pgfqpoint{1.893962in}{1.865891in}}%
\pgfpathcurveto{\pgfqpoint{1.901776in}{1.858077in}}{\pgfqpoint{1.912375in}{1.853687in}}{\pgfqpoint{1.923425in}{1.853687in}}%
\pgfpathclose%
\pgfusepath{stroke,fill}%
\end{pgfscope}%
\begin{pgfscope}%
\pgfpathrectangle{\pgfqpoint{0.787074in}{0.548769in}}{\pgfqpoint{5.062926in}{3.102590in}}%
\pgfusepath{clip}%
\pgfsetbuttcap%
\pgfsetroundjoin%
\definecolor{currentfill}{rgb}{1.000000,0.498039,0.054902}%
\pgfsetfillcolor{currentfill}%
\pgfsetlinewidth{1.003750pt}%
\definecolor{currentstroke}{rgb}{1.000000,0.498039,0.054902}%
\pgfsetstrokecolor{currentstroke}%
\pgfsetdash{}{0pt}%
\pgfpathmoveto{\pgfqpoint{1.165335in}{2.416725in}}%
\pgfpathcurveto{\pgfqpoint{1.176385in}{2.416725in}}{\pgfqpoint{1.186985in}{2.421116in}}{\pgfqpoint{1.194798in}{2.428929in}}%
\pgfpathcurveto{\pgfqpoint{1.202612in}{2.436743in}}{\pgfqpoint{1.207002in}{2.447342in}}{\pgfqpoint{1.207002in}{2.458392in}}%
\pgfpathcurveto{\pgfqpoint{1.207002in}{2.469442in}}{\pgfqpoint{1.202612in}{2.480041in}}{\pgfqpoint{1.194798in}{2.487855in}}%
\pgfpathcurveto{\pgfqpoint{1.186985in}{2.495669in}}{\pgfqpoint{1.176385in}{2.500059in}}{\pgfqpoint{1.165335in}{2.500059in}}%
\pgfpathcurveto{\pgfqpoint{1.154285in}{2.500059in}}{\pgfqpoint{1.143686in}{2.495669in}}{\pgfqpoint{1.135873in}{2.487855in}}%
\pgfpathcurveto{\pgfqpoint{1.128059in}{2.480041in}}{\pgfqpoint{1.123669in}{2.469442in}}{\pgfqpoint{1.123669in}{2.458392in}}%
\pgfpathcurveto{\pgfqpoint{1.123669in}{2.447342in}}{\pgfqpoint{1.128059in}{2.436743in}}{\pgfqpoint{1.135873in}{2.428929in}}%
\pgfpathcurveto{\pgfqpoint{1.143686in}{2.421116in}}{\pgfqpoint{1.154285in}{2.416725in}}{\pgfqpoint{1.165335in}{2.416725in}}%
\pgfpathclose%
\pgfusepath{stroke,fill}%
\end{pgfscope}%
\begin{pgfscope}%
\pgfpathrectangle{\pgfqpoint{0.787074in}{0.548769in}}{\pgfqpoint{5.062926in}{3.102590in}}%
\pgfusepath{clip}%
\pgfsetbuttcap%
\pgfsetroundjoin%
\definecolor{currentfill}{rgb}{0.121569,0.466667,0.705882}%
\pgfsetfillcolor{currentfill}%
\pgfsetlinewidth{1.003750pt}%
\definecolor{currentstroke}{rgb}{0.121569,0.466667,0.705882}%
\pgfsetstrokecolor{currentstroke}%
\pgfsetdash{}{0pt}%
\pgfpathmoveto{\pgfqpoint{1.527345in}{2.410505in}}%
\pgfpathcurveto{\pgfqpoint{1.538395in}{2.410505in}}{\pgfqpoint{1.548994in}{2.414896in}}{\pgfqpoint{1.556808in}{2.422709in}}%
\pgfpathcurveto{\pgfqpoint{1.564622in}{2.430523in}}{\pgfqpoint{1.569012in}{2.441122in}}{\pgfqpoint{1.569012in}{2.452172in}}%
\pgfpathcurveto{\pgfqpoint{1.569012in}{2.463222in}}{\pgfqpoint{1.564622in}{2.473821in}}{\pgfqpoint{1.556808in}{2.481635in}}%
\pgfpathcurveto{\pgfqpoint{1.548994in}{2.489449in}}{\pgfqpoint{1.538395in}{2.493839in}}{\pgfqpoint{1.527345in}{2.493839in}}%
\pgfpathcurveto{\pgfqpoint{1.516295in}{2.493839in}}{\pgfqpoint{1.505696in}{2.489449in}}{\pgfqpoint{1.497882in}{2.481635in}}%
\pgfpathcurveto{\pgfqpoint{1.490069in}{2.473821in}}{\pgfqpoint{1.485679in}{2.463222in}}{\pgfqpoint{1.485679in}{2.452172in}}%
\pgfpathcurveto{\pgfqpoint{1.485679in}{2.441122in}}{\pgfqpoint{1.490069in}{2.430523in}}{\pgfqpoint{1.497882in}{2.422709in}}%
\pgfpathcurveto{\pgfqpoint{1.505696in}{2.414896in}}{\pgfqpoint{1.516295in}{2.410505in}}{\pgfqpoint{1.527345in}{2.410505in}}%
\pgfpathclose%
\pgfusepath{stroke,fill}%
\end{pgfscope}%
\begin{pgfscope}%
\pgfpathrectangle{\pgfqpoint{0.787074in}{0.548769in}}{\pgfqpoint{5.062926in}{3.102590in}}%
\pgfusepath{clip}%
\pgfsetbuttcap%
\pgfsetroundjoin%
\definecolor{currentfill}{rgb}{0.121569,0.466667,0.705882}%
\pgfsetfillcolor{currentfill}%
\pgfsetlinewidth{1.003750pt}%
\definecolor{currentstroke}{rgb}{0.121569,0.466667,0.705882}%
\pgfsetstrokecolor{currentstroke}%
\pgfsetdash{}{0pt}%
\pgfpathmoveto{\pgfqpoint{1.524481in}{0.678421in}}%
\pgfpathcurveto{\pgfqpoint{1.535531in}{0.678421in}}{\pgfqpoint{1.546130in}{0.682812in}}{\pgfqpoint{1.553944in}{0.690625in}}%
\pgfpathcurveto{\pgfqpoint{1.561757in}{0.698439in}}{\pgfqpoint{1.566147in}{0.709038in}}{\pgfqpoint{1.566147in}{0.720088in}}%
\pgfpathcurveto{\pgfqpoint{1.566147in}{0.731138in}}{\pgfqpoint{1.561757in}{0.741737in}}{\pgfqpoint{1.553944in}{0.749551in}}%
\pgfpathcurveto{\pgfqpoint{1.546130in}{0.757364in}}{\pgfqpoint{1.535531in}{0.761755in}}{\pgfqpoint{1.524481in}{0.761755in}}%
\pgfpathcurveto{\pgfqpoint{1.513431in}{0.761755in}}{\pgfqpoint{1.502832in}{0.757364in}}{\pgfqpoint{1.495018in}{0.749551in}}%
\pgfpathcurveto{\pgfqpoint{1.487204in}{0.741737in}}{\pgfqpoint{1.482814in}{0.731138in}}{\pgfqpoint{1.482814in}{0.720088in}}%
\pgfpathcurveto{\pgfqpoint{1.482814in}{0.709038in}}{\pgfqpoint{1.487204in}{0.698439in}}{\pgfqpoint{1.495018in}{0.690625in}}%
\pgfpathcurveto{\pgfqpoint{1.502832in}{0.682812in}}{\pgfqpoint{1.513431in}{0.678421in}}{\pgfqpoint{1.524481in}{0.678421in}}%
\pgfpathclose%
\pgfusepath{stroke,fill}%
\end{pgfscope}%
\begin{pgfscope}%
\pgfpathrectangle{\pgfqpoint{0.787074in}{0.548769in}}{\pgfqpoint{5.062926in}{3.102590in}}%
\pgfusepath{clip}%
\pgfsetbuttcap%
\pgfsetroundjoin%
\definecolor{currentfill}{rgb}{1.000000,0.498039,0.054902}%
\pgfsetfillcolor{currentfill}%
\pgfsetlinewidth{1.003750pt}%
\definecolor{currentstroke}{rgb}{1.000000,0.498039,0.054902}%
\pgfsetstrokecolor{currentstroke}%
\pgfsetdash{}{0pt}%
\pgfpathmoveto{\pgfqpoint{2.096640in}{2.587414in}}%
\pgfpathcurveto{\pgfqpoint{2.107690in}{2.587414in}}{\pgfqpoint{2.118289in}{2.591804in}}{\pgfqpoint{2.126102in}{2.599618in}}%
\pgfpathcurveto{\pgfqpoint{2.133916in}{2.607432in}}{\pgfqpoint{2.138306in}{2.618031in}}{\pgfqpoint{2.138306in}{2.629081in}}%
\pgfpathcurveto{\pgfqpoint{2.138306in}{2.640131in}}{\pgfqpoint{2.133916in}{2.650730in}}{\pgfqpoint{2.126102in}{2.658544in}}%
\pgfpathcurveto{\pgfqpoint{2.118289in}{2.666357in}}{\pgfqpoint{2.107690in}{2.670747in}}{\pgfqpoint{2.096640in}{2.670747in}}%
\pgfpathcurveto{\pgfqpoint{2.085589in}{2.670747in}}{\pgfqpoint{2.074990in}{2.666357in}}{\pgfqpoint{2.067177in}{2.658544in}}%
\pgfpathcurveto{\pgfqpoint{2.059363in}{2.650730in}}{\pgfqpoint{2.054973in}{2.640131in}}{\pgfqpoint{2.054973in}{2.629081in}}%
\pgfpathcurveto{\pgfqpoint{2.054973in}{2.618031in}}{\pgfqpoint{2.059363in}{2.607432in}}{\pgfqpoint{2.067177in}{2.599618in}}%
\pgfpathcurveto{\pgfqpoint{2.074990in}{2.591804in}}{\pgfqpoint{2.085589in}{2.587414in}}{\pgfqpoint{2.096640in}{2.587414in}}%
\pgfpathclose%
\pgfusepath{stroke,fill}%
\end{pgfscope}%
\begin{pgfscope}%
\pgfpathrectangle{\pgfqpoint{0.787074in}{0.548769in}}{\pgfqpoint{5.062926in}{3.102590in}}%
\pgfusepath{clip}%
\pgfsetbuttcap%
\pgfsetroundjoin%
\definecolor{currentfill}{rgb}{1.000000,0.498039,0.054902}%
\pgfsetfillcolor{currentfill}%
\pgfsetlinewidth{1.003750pt}%
\definecolor{currentstroke}{rgb}{1.000000,0.498039,0.054902}%
\pgfsetstrokecolor{currentstroke}%
\pgfsetdash{}{0pt}%
\pgfpathmoveto{\pgfqpoint{1.740011in}{3.097620in}}%
\pgfpathcurveto{\pgfqpoint{1.751062in}{3.097620in}}{\pgfqpoint{1.761661in}{3.102011in}}{\pgfqpoint{1.769474in}{3.109824in}}%
\pgfpathcurveto{\pgfqpoint{1.777288in}{3.117638in}}{\pgfqpoint{1.781678in}{3.128237in}}{\pgfqpoint{1.781678in}{3.139287in}}%
\pgfpathcurveto{\pgfqpoint{1.781678in}{3.150337in}}{\pgfqpoint{1.777288in}{3.160936in}}{\pgfqpoint{1.769474in}{3.168750in}}%
\pgfpathcurveto{\pgfqpoint{1.761661in}{3.176564in}}{\pgfqpoint{1.751062in}{3.180954in}}{\pgfqpoint{1.740011in}{3.180954in}}%
\pgfpathcurveto{\pgfqpoint{1.728961in}{3.180954in}}{\pgfqpoint{1.718362in}{3.176564in}}{\pgfqpoint{1.710549in}{3.168750in}}%
\pgfpathcurveto{\pgfqpoint{1.702735in}{3.160936in}}{\pgfqpoint{1.698345in}{3.150337in}}{\pgfqpoint{1.698345in}{3.139287in}}%
\pgfpathcurveto{\pgfqpoint{1.698345in}{3.128237in}}{\pgfqpoint{1.702735in}{3.117638in}}{\pgfqpoint{1.710549in}{3.109824in}}%
\pgfpathcurveto{\pgfqpoint{1.718362in}{3.102011in}}{\pgfqpoint{1.728961in}{3.097620in}}{\pgfqpoint{1.740011in}{3.097620in}}%
\pgfpathclose%
\pgfusepath{stroke,fill}%
\end{pgfscope}%
\begin{pgfscope}%
\pgfpathrectangle{\pgfqpoint{0.787074in}{0.548769in}}{\pgfqpoint{5.062926in}{3.102590in}}%
\pgfusepath{clip}%
\pgfsetbuttcap%
\pgfsetroundjoin%
\definecolor{currentfill}{rgb}{0.121569,0.466667,0.705882}%
\pgfsetfillcolor{currentfill}%
\pgfsetlinewidth{1.003750pt}%
\definecolor{currentstroke}{rgb}{0.121569,0.466667,0.705882}%
\pgfsetstrokecolor{currentstroke}%
\pgfsetdash{}{0pt}%
\pgfpathmoveto{\pgfqpoint{1.433034in}{0.648139in}}%
\pgfpathcurveto{\pgfqpoint{1.444084in}{0.648139in}}{\pgfqpoint{1.454683in}{0.652529in}}{\pgfqpoint{1.462497in}{0.660343in}}%
\pgfpathcurveto{\pgfqpoint{1.470311in}{0.668156in}}{\pgfqpoint{1.474701in}{0.678755in}}{\pgfqpoint{1.474701in}{0.689806in}}%
\pgfpathcurveto{\pgfqpoint{1.474701in}{0.700856in}}{\pgfqpoint{1.470311in}{0.711455in}}{\pgfqpoint{1.462497in}{0.719268in}}%
\pgfpathcurveto{\pgfqpoint{1.454683in}{0.727082in}}{\pgfqpoint{1.444084in}{0.731472in}}{\pgfqpoint{1.433034in}{0.731472in}}%
\pgfpathcurveto{\pgfqpoint{1.421984in}{0.731472in}}{\pgfqpoint{1.411385in}{0.727082in}}{\pgfqpoint{1.403572in}{0.719268in}}%
\pgfpathcurveto{\pgfqpoint{1.395758in}{0.711455in}}{\pgfqpoint{1.391368in}{0.700856in}}{\pgfqpoint{1.391368in}{0.689806in}}%
\pgfpathcurveto{\pgfqpoint{1.391368in}{0.678755in}}{\pgfqpoint{1.395758in}{0.668156in}}{\pgfqpoint{1.403572in}{0.660343in}}%
\pgfpathcurveto{\pgfqpoint{1.411385in}{0.652529in}}{\pgfqpoint{1.421984in}{0.648139in}}{\pgfqpoint{1.433034in}{0.648139in}}%
\pgfpathclose%
\pgfusepath{stroke,fill}%
\end{pgfscope}%
\begin{pgfscope}%
\pgfpathrectangle{\pgfqpoint{0.787074in}{0.548769in}}{\pgfqpoint{5.062926in}{3.102590in}}%
\pgfusepath{clip}%
\pgfsetbuttcap%
\pgfsetroundjoin%
\definecolor{currentfill}{rgb}{0.121569,0.466667,0.705882}%
\pgfsetfillcolor{currentfill}%
\pgfsetlinewidth{1.003750pt}%
\definecolor{currentstroke}{rgb}{0.121569,0.466667,0.705882}%
\pgfsetstrokecolor{currentstroke}%
\pgfsetdash{}{0pt}%
\pgfpathmoveto{\pgfqpoint{1.077057in}{0.649969in}}%
\pgfpathcurveto{\pgfqpoint{1.088107in}{0.649969in}}{\pgfqpoint{1.098706in}{0.654360in}}{\pgfqpoint{1.106520in}{0.662173in}}%
\pgfpathcurveto{\pgfqpoint{1.114334in}{0.669987in}}{\pgfqpoint{1.118724in}{0.680586in}}{\pgfqpoint{1.118724in}{0.691636in}}%
\pgfpathcurveto{\pgfqpoint{1.118724in}{0.702686in}}{\pgfqpoint{1.114334in}{0.713285in}}{\pgfqpoint{1.106520in}{0.721099in}}%
\pgfpathcurveto{\pgfqpoint{1.098706in}{0.728913in}}{\pgfqpoint{1.088107in}{0.733303in}}{\pgfqpoint{1.077057in}{0.733303in}}%
\pgfpathcurveto{\pgfqpoint{1.066007in}{0.733303in}}{\pgfqpoint{1.055408in}{0.728913in}}{\pgfqpoint{1.047594in}{0.721099in}}%
\pgfpathcurveto{\pgfqpoint{1.039781in}{0.713285in}}{\pgfqpoint{1.035391in}{0.702686in}}{\pgfqpoint{1.035391in}{0.691636in}}%
\pgfpathcurveto{\pgfqpoint{1.035391in}{0.680586in}}{\pgfqpoint{1.039781in}{0.669987in}}{\pgfqpoint{1.047594in}{0.662173in}}%
\pgfpathcurveto{\pgfqpoint{1.055408in}{0.654360in}}{\pgfqpoint{1.066007in}{0.649969in}}{\pgfqpoint{1.077057in}{0.649969in}}%
\pgfpathclose%
\pgfusepath{stroke,fill}%
\end{pgfscope}%
\begin{pgfscope}%
\pgfpathrectangle{\pgfqpoint{0.787074in}{0.548769in}}{\pgfqpoint{5.062926in}{3.102590in}}%
\pgfusepath{clip}%
\pgfsetbuttcap%
\pgfsetroundjoin%
\definecolor{currentfill}{rgb}{0.121569,0.466667,0.705882}%
\pgfsetfillcolor{currentfill}%
\pgfsetlinewidth{1.003750pt}%
\definecolor{currentstroke}{rgb}{0.121569,0.466667,0.705882}%
\pgfsetstrokecolor{currentstroke}%
\pgfsetdash{}{0pt}%
\pgfpathmoveto{\pgfqpoint{1.841136in}{2.881001in}}%
\pgfpathcurveto{\pgfqpoint{1.852186in}{2.881001in}}{\pgfqpoint{1.862786in}{2.885391in}}{\pgfqpoint{1.870599in}{2.893205in}}%
\pgfpathcurveto{\pgfqpoint{1.878413in}{2.901018in}}{\pgfqpoint{1.882803in}{2.911617in}}{\pgfqpoint{1.882803in}{2.922668in}}%
\pgfpathcurveto{\pgfqpoint{1.882803in}{2.933718in}}{\pgfqpoint{1.878413in}{2.944317in}}{\pgfqpoint{1.870599in}{2.952130in}}%
\pgfpathcurveto{\pgfqpoint{1.862786in}{2.959944in}}{\pgfqpoint{1.852186in}{2.964334in}}{\pgfqpoint{1.841136in}{2.964334in}}%
\pgfpathcurveto{\pgfqpoint{1.830086in}{2.964334in}}{\pgfqpoint{1.819487in}{2.959944in}}{\pgfqpoint{1.811674in}{2.952130in}}%
\pgfpathcurveto{\pgfqpoint{1.803860in}{2.944317in}}{\pgfqpoint{1.799470in}{2.933718in}}{\pgfqpoint{1.799470in}{2.922668in}}%
\pgfpathcurveto{\pgfqpoint{1.799470in}{2.911617in}}{\pgfqpoint{1.803860in}{2.901018in}}{\pgfqpoint{1.811674in}{2.893205in}}%
\pgfpathcurveto{\pgfqpoint{1.819487in}{2.885391in}}{\pgfqpoint{1.830086in}{2.881001in}}{\pgfqpoint{1.841136in}{2.881001in}}%
\pgfpathclose%
\pgfusepath{stroke,fill}%
\end{pgfscope}%
\begin{pgfscope}%
\pgfpathrectangle{\pgfqpoint{0.787074in}{0.548769in}}{\pgfqpoint{5.062926in}{3.102590in}}%
\pgfusepath{clip}%
\pgfsetbuttcap%
\pgfsetroundjoin%
\definecolor{currentfill}{rgb}{1.000000,0.498039,0.054902}%
\pgfsetfillcolor{currentfill}%
\pgfsetlinewidth{1.003750pt}%
\definecolor{currentstroke}{rgb}{1.000000,0.498039,0.054902}%
\pgfsetstrokecolor{currentstroke}%
\pgfsetdash{}{0pt}%
\pgfpathmoveto{\pgfqpoint{2.560729in}{3.082877in}}%
\pgfpathcurveto{\pgfqpoint{2.571779in}{3.082877in}}{\pgfqpoint{2.582378in}{3.087267in}}{\pgfqpoint{2.590192in}{3.095080in}}%
\pgfpathcurveto{\pgfqpoint{2.598006in}{3.102894in}}{\pgfqpoint{2.602396in}{3.113493in}}{\pgfqpoint{2.602396in}{3.124543in}}%
\pgfpathcurveto{\pgfqpoint{2.602396in}{3.135593in}}{\pgfqpoint{2.598006in}{3.146192in}}{\pgfqpoint{2.590192in}{3.154006in}}%
\pgfpathcurveto{\pgfqpoint{2.582378in}{3.161820in}}{\pgfqpoint{2.571779in}{3.166210in}}{\pgfqpoint{2.560729in}{3.166210in}}%
\pgfpathcurveto{\pgfqpoint{2.549679in}{3.166210in}}{\pgfqpoint{2.539080in}{3.161820in}}{\pgfqpoint{2.531266in}{3.154006in}}%
\pgfpathcurveto{\pgfqpoint{2.523453in}{3.146192in}}{\pgfqpoint{2.519063in}{3.135593in}}{\pgfqpoint{2.519063in}{3.124543in}}%
\pgfpathcurveto{\pgfqpoint{2.519063in}{3.113493in}}{\pgfqpoint{2.523453in}{3.102894in}}{\pgfqpoint{2.531266in}{3.095080in}}%
\pgfpathcurveto{\pgfqpoint{2.539080in}{3.087267in}}{\pgfqpoint{2.549679in}{3.082877in}}{\pgfqpoint{2.560729in}{3.082877in}}%
\pgfpathclose%
\pgfusepath{stroke,fill}%
\end{pgfscope}%
\begin{pgfscope}%
\pgfpathrectangle{\pgfqpoint{0.787074in}{0.548769in}}{\pgfqpoint{5.062926in}{3.102590in}}%
\pgfusepath{clip}%
\pgfsetbuttcap%
\pgfsetroundjoin%
\definecolor{currentfill}{rgb}{1.000000,0.498039,0.054902}%
\pgfsetfillcolor{currentfill}%
\pgfsetlinewidth{1.003750pt}%
\definecolor{currentstroke}{rgb}{1.000000,0.498039,0.054902}%
\pgfsetstrokecolor{currentstroke}%
\pgfsetdash{}{0pt}%
\pgfpathmoveto{\pgfqpoint{1.747607in}{2.563602in}}%
\pgfpathcurveto{\pgfqpoint{1.758657in}{2.563602in}}{\pgfqpoint{1.769256in}{2.567992in}}{\pgfqpoint{1.777069in}{2.575806in}}%
\pgfpathcurveto{\pgfqpoint{1.784883in}{2.583619in}}{\pgfqpoint{1.789273in}{2.594218in}}{\pgfqpoint{1.789273in}{2.605268in}}%
\pgfpathcurveto{\pgfqpoint{1.789273in}{2.616318in}}{\pgfqpoint{1.784883in}{2.626917in}}{\pgfqpoint{1.777069in}{2.634731in}}%
\pgfpathcurveto{\pgfqpoint{1.769256in}{2.642545in}}{\pgfqpoint{1.758657in}{2.646935in}}{\pgfqpoint{1.747607in}{2.646935in}}%
\pgfpathcurveto{\pgfqpoint{1.736557in}{2.646935in}}{\pgfqpoint{1.725957in}{2.642545in}}{\pgfqpoint{1.718144in}{2.634731in}}%
\pgfpathcurveto{\pgfqpoint{1.710330in}{2.626917in}}{\pgfqpoint{1.705940in}{2.616318in}}{\pgfqpoint{1.705940in}{2.605268in}}%
\pgfpathcurveto{\pgfqpoint{1.705940in}{2.594218in}}{\pgfqpoint{1.710330in}{2.583619in}}{\pgfqpoint{1.718144in}{2.575806in}}%
\pgfpathcurveto{\pgfqpoint{1.725957in}{2.567992in}}{\pgfqpoint{1.736557in}{2.563602in}}{\pgfqpoint{1.747607in}{2.563602in}}%
\pgfpathclose%
\pgfusepath{stroke,fill}%
\end{pgfscope}%
\begin{pgfscope}%
\pgfpathrectangle{\pgfqpoint{0.787074in}{0.548769in}}{\pgfqpoint{5.062926in}{3.102590in}}%
\pgfusepath{clip}%
\pgfsetbuttcap%
\pgfsetroundjoin%
\definecolor{currentfill}{rgb}{1.000000,0.498039,0.054902}%
\pgfsetfillcolor{currentfill}%
\pgfsetlinewidth{1.003750pt}%
\definecolor{currentstroke}{rgb}{1.000000,0.498039,0.054902}%
\pgfsetstrokecolor{currentstroke}%
\pgfsetdash{}{0pt}%
\pgfpathmoveto{\pgfqpoint{1.879980in}{2.277203in}}%
\pgfpathcurveto{\pgfqpoint{1.891031in}{2.277203in}}{\pgfqpoint{1.901630in}{2.281594in}}{\pgfqpoint{1.909443in}{2.289407in}}%
\pgfpathcurveto{\pgfqpoint{1.917257in}{2.297221in}}{\pgfqpoint{1.921647in}{2.307820in}}{\pgfqpoint{1.921647in}{2.318870in}}%
\pgfpathcurveto{\pgfqpoint{1.921647in}{2.329920in}}{\pgfqpoint{1.917257in}{2.340519in}}{\pgfqpoint{1.909443in}{2.348333in}}%
\pgfpathcurveto{\pgfqpoint{1.901630in}{2.356146in}}{\pgfqpoint{1.891031in}{2.360537in}}{\pgfqpoint{1.879980in}{2.360537in}}%
\pgfpathcurveto{\pgfqpoint{1.868930in}{2.360537in}}{\pgfqpoint{1.858331in}{2.356146in}}{\pgfqpoint{1.850518in}{2.348333in}}%
\pgfpathcurveto{\pgfqpoint{1.842704in}{2.340519in}}{\pgfqpoint{1.838314in}{2.329920in}}{\pgfqpoint{1.838314in}{2.318870in}}%
\pgfpathcurveto{\pgfqpoint{1.838314in}{2.307820in}}{\pgfqpoint{1.842704in}{2.297221in}}{\pgfqpoint{1.850518in}{2.289407in}}%
\pgfpathcurveto{\pgfqpoint{1.858331in}{2.281594in}}{\pgfqpoint{1.868930in}{2.277203in}}{\pgfqpoint{1.879980in}{2.277203in}}%
\pgfpathclose%
\pgfusepath{stroke,fill}%
\end{pgfscope}%
\begin{pgfscope}%
\pgfpathrectangle{\pgfqpoint{0.787074in}{0.548769in}}{\pgfqpoint{5.062926in}{3.102590in}}%
\pgfusepath{clip}%
\pgfsetbuttcap%
\pgfsetroundjoin%
\definecolor{currentfill}{rgb}{1.000000,0.498039,0.054902}%
\pgfsetfillcolor{currentfill}%
\pgfsetlinewidth{1.003750pt}%
\definecolor{currentstroke}{rgb}{1.000000,0.498039,0.054902}%
\pgfsetstrokecolor{currentstroke}%
\pgfsetdash{}{0pt}%
\pgfpathmoveto{\pgfqpoint{1.627689in}{2.871516in}}%
\pgfpathcurveto{\pgfqpoint{1.638739in}{2.871516in}}{\pgfqpoint{1.649338in}{2.875907in}}{\pgfqpoint{1.657152in}{2.883720in}}%
\pgfpathcurveto{\pgfqpoint{1.664965in}{2.891534in}}{\pgfqpoint{1.669356in}{2.902133in}}{\pgfqpoint{1.669356in}{2.913183in}}%
\pgfpathcurveto{\pgfqpoint{1.669356in}{2.924233in}}{\pgfqpoint{1.664965in}{2.934832in}}{\pgfqpoint{1.657152in}{2.942646in}}%
\pgfpathcurveto{\pgfqpoint{1.649338in}{2.950459in}}{\pgfqpoint{1.638739in}{2.954850in}}{\pgfqpoint{1.627689in}{2.954850in}}%
\pgfpathcurveto{\pgfqpoint{1.616639in}{2.954850in}}{\pgfqpoint{1.606040in}{2.950459in}}{\pgfqpoint{1.598226in}{2.942646in}}%
\pgfpathcurveto{\pgfqpoint{1.590413in}{2.934832in}}{\pgfqpoint{1.586022in}{2.924233in}}{\pgfqpoint{1.586022in}{2.913183in}}%
\pgfpathcurveto{\pgfqpoint{1.586022in}{2.902133in}}{\pgfqpoint{1.590413in}{2.891534in}}{\pgfqpoint{1.598226in}{2.883720in}}%
\pgfpathcurveto{\pgfqpoint{1.606040in}{2.875907in}}{\pgfqpoint{1.616639in}{2.871516in}}{\pgfqpoint{1.627689in}{2.871516in}}%
\pgfpathclose%
\pgfusepath{stroke,fill}%
\end{pgfscope}%
\begin{pgfscope}%
\pgfpathrectangle{\pgfqpoint{0.787074in}{0.548769in}}{\pgfqpoint{5.062926in}{3.102590in}}%
\pgfusepath{clip}%
\pgfsetbuttcap%
\pgfsetroundjoin%
\definecolor{currentfill}{rgb}{0.121569,0.466667,0.705882}%
\pgfsetfillcolor{currentfill}%
\pgfsetlinewidth{1.003750pt}%
\definecolor{currentstroke}{rgb}{0.121569,0.466667,0.705882}%
\pgfsetstrokecolor{currentstroke}%
\pgfsetdash{}{0pt}%
\pgfpathmoveto{\pgfqpoint{1.126318in}{0.648134in}}%
\pgfpathcurveto{\pgfqpoint{1.137368in}{0.648134in}}{\pgfqpoint{1.147967in}{0.652524in}}{\pgfqpoint{1.155780in}{0.660338in}}%
\pgfpathcurveto{\pgfqpoint{1.163594in}{0.668152in}}{\pgfqpoint{1.167984in}{0.678751in}}{\pgfqpoint{1.167984in}{0.689801in}}%
\pgfpathcurveto{\pgfqpoint{1.167984in}{0.700851in}}{\pgfqpoint{1.163594in}{0.711450in}}{\pgfqpoint{1.155780in}{0.719264in}}%
\pgfpathcurveto{\pgfqpoint{1.147967in}{0.727077in}}{\pgfqpoint{1.137368in}{0.731467in}}{\pgfqpoint{1.126318in}{0.731467in}}%
\pgfpathcurveto{\pgfqpoint{1.115267in}{0.731467in}}{\pgfqpoint{1.104668in}{0.727077in}}{\pgfqpoint{1.096855in}{0.719264in}}%
\pgfpathcurveto{\pgfqpoint{1.089041in}{0.711450in}}{\pgfqpoint{1.084651in}{0.700851in}}{\pgfqpoint{1.084651in}{0.689801in}}%
\pgfpathcurveto{\pgfqpoint{1.084651in}{0.678751in}}{\pgfqpoint{1.089041in}{0.668152in}}{\pgfqpoint{1.096855in}{0.660338in}}%
\pgfpathcurveto{\pgfqpoint{1.104668in}{0.652524in}}{\pgfqpoint{1.115267in}{0.648134in}}{\pgfqpoint{1.126318in}{0.648134in}}%
\pgfpathclose%
\pgfusepath{stroke,fill}%
\end{pgfscope}%
\begin{pgfscope}%
\pgfpathrectangle{\pgfqpoint{0.787074in}{0.548769in}}{\pgfqpoint{5.062926in}{3.102590in}}%
\pgfusepath{clip}%
\pgfsetbuttcap%
\pgfsetroundjoin%
\definecolor{currentfill}{rgb}{0.121569,0.466667,0.705882}%
\pgfsetfillcolor{currentfill}%
\pgfsetlinewidth{1.003750pt}%
\definecolor{currentstroke}{rgb}{0.121569,0.466667,0.705882}%
\pgfsetstrokecolor{currentstroke}%
\pgfsetdash{}{0pt}%
\pgfpathmoveto{\pgfqpoint{1.727425in}{3.215961in}}%
\pgfpathcurveto{\pgfqpoint{1.738475in}{3.215961in}}{\pgfqpoint{1.749074in}{3.220351in}}{\pgfqpoint{1.756888in}{3.228165in}}%
\pgfpathcurveto{\pgfqpoint{1.764701in}{3.235978in}}{\pgfqpoint{1.769092in}{3.246577in}}{\pgfqpoint{1.769092in}{3.257627in}}%
\pgfpathcurveto{\pgfqpoint{1.769092in}{3.268677in}}{\pgfqpoint{1.764701in}{3.279276in}}{\pgfqpoint{1.756888in}{3.287090in}}%
\pgfpathcurveto{\pgfqpoint{1.749074in}{3.294904in}}{\pgfqpoint{1.738475in}{3.299294in}}{\pgfqpoint{1.727425in}{3.299294in}}%
\pgfpathcurveto{\pgfqpoint{1.716375in}{3.299294in}}{\pgfqpoint{1.705776in}{3.294904in}}{\pgfqpoint{1.697962in}{3.287090in}}%
\pgfpathcurveto{\pgfqpoint{1.690149in}{3.279276in}}{\pgfqpoint{1.685758in}{3.268677in}}{\pgfqpoint{1.685758in}{3.257627in}}%
\pgfpathcurveto{\pgfqpoint{1.685758in}{3.246577in}}{\pgfqpoint{1.690149in}{3.235978in}}{\pgfqpoint{1.697962in}{3.228165in}}%
\pgfpathcurveto{\pgfqpoint{1.705776in}{3.220351in}}{\pgfqpoint{1.716375in}{3.215961in}}{\pgfqpoint{1.727425in}{3.215961in}}%
\pgfpathclose%
\pgfusepath{stroke,fill}%
\end{pgfscope}%
\begin{pgfscope}%
\pgfpathrectangle{\pgfqpoint{0.787074in}{0.548769in}}{\pgfqpoint{5.062926in}{3.102590in}}%
\pgfusepath{clip}%
\pgfsetbuttcap%
\pgfsetroundjoin%
\definecolor{currentfill}{rgb}{1.000000,0.498039,0.054902}%
\pgfsetfillcolor{currentfill}%
\pgfsetlinewidth{1.003750pt}%
\definecolor{currentstroke}{rgb}{1.000000,0.498039,0.054902}%
\pgfsetstrokecolor{currentstroke}%
\pgfsetdash{}{0pt}%
\pgfpathmoveto{\pgfqpoint{1.721913in}{3.082101in}}%
\pgfpathcurveto{\pgfqpoint{1.732963in}{3.082101in}}{\pgfqpoint{1.743562in}{3.086492in}}{\pgfqpoint{1.751376in}{3.094305in}}%
\pgfpathcurveto{\pgfqpoint{1.759189in}{3.102119in}}{\pgfqpoint{1.763580in}{3.112718in}}{\pgfqpoint{1.763580in}{3.123768in}}%
\pgfpathcurveto{\pgfqpoint{1.763580in}{3.134818in}}{\pgfqpoint{1.759189in}{3.145417in}}{\pgfqpoint{1.751376in}{3.153231in}}%
\pgfpathcurveto{\pgfqpoint{1.743562in}{3.161044in}}{\pgfqpoint{1.732963in}{3.165435in}}{\pgfqpoint{1.721913in}{3.165435in}}%
\pgfpathcurveto{\pgfqpoint{1.710863in}{3.165435in}}{\pgfqpoint{1.700264in}{3.161044in}}{\pgfqpoint{1.692450in}{3.153231in}}%
\pgfpathcurveto{\pgfqpoint{1.684637in}{3.145417in}}{\pgfqpoint{1.680246in}{3.134818in}}{\pgfqpoint{1.680246in}{3.123768in}}%
\pgfpathcurveto{\pgfqpoint{1.680246in}{3.112718in}}{\pgfqpoint{1.684637in}{3.102119in}}{\pgfqpoint{1.692450in}{3.094305in}}%
\pgfpathcurveto{\pgfqpoint{1.700264in}{3.086492in}}{\pgfqpoint{1.710863in}{3.082101in}}{\pgfqpoint{1.721913in}{3.082101in}}%
\pgfpathclose%
\pgfusepath{stroke,fill}%
\end{pgfscope}%
\begin{pgfscope}%
\pgfpathrectangle{\pgfqpoint{0.787074in}{0.548769in}}{\pgfqpoint{5.062926in}{3.102590in}}%
\pgfusepath{clip}%
\pgfsetbuttcap%
\pgfsetroundjoin%
\definecolor{currentfill}{rgb}{0.121569,0.466667,0.705882}%
\pgfsetfillcolor{currentfill}%
\pgfsetlinewidth{1.003750pt}%
\definecolor{currentstroke}{rgb}{0.121569,0.466667,0.705882}%
\pgfsetstrokecolor{currentstroke}%
\pgfsetdash{}{0pt}%
\pgfpathmoveto{\pgfqpoint{1.577213in}{1.896734in}}%
\pgfpathcurveto{\pgfqpoint{1.588263in}{1.896734in}}{\pgfqpoint{1.598862in}{1.901124in}}{\pgfqpoint{1.606676in}{1.908938in}}%
\pgfpathcurveto{\pgfqpoint{1.614490in}{1.916751in}}{\pgfqpoint{1.618880in}{1.927350in}}{\pgfqpoint{1.618880in}{1.938401in}}%
\pgfpathcurveto{\pgfqpoint{1.618880in}{1.949451in}}{\pgfqpoint{1.614490in}{1.960050in}}{\pgfqpoint{1.606676in}{1.967863in}}%
\pgfpathcurveto{\pgfqpoint{1.598862in}{1.975677in}}{\pgfqpoint{1.588263in}{1.980067in}}{\pgfqpoint{1.577213in}{1.980067in}}%
\pgfpathcurveto{\pgfqpoint{1.566163in}{1.980067in}}{\pgfqpoint{1.555564in}{1.975677in}}{\pgfqpoint{1.547751in}{1.967863in}}%
\pgfpathcurveto{\pgfqpoint{1.539937in}{1.960050in}}{\pgfqpoint{1.535547in}{1.949451in}}{\pgfqpoint{1.535547in}{1.938401in}}%
\pgfpathcurveto{\pgfqpoint{1.535547in}{1.927350in}}{\pgfqpoint{1.539937in}{1.916751in}}{\pgfqpoint{1.547751in}{1.908938in}}%
\pgfpathcurveto{\pgfqpoint{1.555564in}{1.901124in}}{\pgfqpoint{1.566163in}{1.896734in}}{\pgfqpoint{1.577213in}{1.896734in}}%
\pgfpathclose%
\pgfusepath{stroke,fill}%
\end{pgfscope}%
\begin{pgfscope}%
\pgfpathrectangle{\pgfqpoint{0.787074in}{0.548769in}}{\pgfqpoint{5.062926in}{3.102590in}}%
\pgfusepath{clip}%
\pgfsetbuttcap%
\pgfsetroundjoin%
\definecolor{currentfill}{rgb}{1.000000,0.498039,0.054902}%
\pgfsetfillcolor{currentfill}%
\pgfsetlinewidth{1.003750pt}%
\definecolor{currentstroke}{rgb}{1.000000,0.498039,0.054902}%
\pgfsetstrokecolor{currentstroke}%
\pgfsetdash{}{0pt}%
\pgfpathmoveto{\pgfqpoint{1.573828in}{3.048752in}}%
\pgfpathcurveto{\pgfqpoint{1.584878in}{3.048752in}}{\pgfqpoint{1.595477in}{3.053142in}}{\pgfqpoint{1.603291in}{3.060956in}}%
\pgfpathcurveto{\pgfqpoint{1.611104in}{3.068769in}}{\pgfqpoint{1.615495in}{3.079368in}}{\pgfqpoint{1.615495in}{3.090419in}}%
\pgfpathcurveto{\pgfqpoint{1.615495in}{3.101469in}}{\pgfqpoint{1.611104in}{3.112068in}}{\pgfqpoint{1.603291in}{3.119881in}}%
\pgfpathcurveto{\pgfqpoint{1.595477in}{3.127695in}}{\pgfqpoint{1.584878in}{3.132085in}}{\pgfqpoint{1.573828in}{3.132085in}}%
\pgfpathcurveto{\pgfqpoint{1.562778in}{3.132085in}}{\pgfqpoint{1.552179in}{3.127695in}}{\pgfqpoint{1.544365in}{3.119881in}}%
\pgfpathcurveto{\pgfqpoint{1.536552in}{3.112068in}}{\pgfqpoint{1.532161in}{3.101469in}}{\pgfqpoint{1.532161in}{3.090419in}}%
\pgfpathcurveto{\pgfqpoint{1.532161in}{3.079368in}}{\pgfqpoint{1.536552in}{3.068769in}}{\pgfqpoint{1.544365in}{3.060956in}}%
\pgfpathcurveto{\pgfqpoint{1.552179in}{3.053142in}}{\pgfqpoint{1.562778in}{3.048752in}}{\pgfqpoint{1.573828in}{3.048752in}}%
\pgfpathclose%
\pgfusepath{stroke,fill}%
\end{pgfscope}%
\begin{pgfscope}%
\pgfpathrectangle{\pgfqpoint{0.787074in}{0.548769in}}{\pgfqpoint{5.062926in}{3.102590in}}%
\pgfusepath{clip}%
\pgfsetbuttcap%
\pgfsetroundjoin%
\definecolor{currentfill}{rgb}{1.000000,0.498039,0.054902}%
\pgfsetfillcolor{currentfill}%
\pgfsetlinewidth{1.003750pt}%
\definecolor{currentstroke}{rgb}{1.000000,0.498039,0.054902}%
\pgfsetstrokecolor{currentstroke}%
\pgfsetdash{}{0pt}%
\pgfpathmoveto{\pgfqpoint{1.433816in}{2.964483in}}%
\pgfpathcurveto{\pgfqpoint{1.444866in}{2.964483in}}{\pgfqpoint{1.455465in}{2.968874in}}{\pgfqpoint{1.463278in}{2.976687in}}%
\pgfpathcurveto{\pgfqpoint{1.471092in}{2.984501in}}{\pgfqpoint{1.475482in}{2.995100in}}{\pgfqpoint{1.475482in}{3.006150in}}%
\pgfpathcurveto{\pgfqpoint{1.475482in}{3.017200in}}{\pgfqpoint{1.471092in}{3.027799in}}{\pgfqpoint{1.463278in}{3.035613in}}%
\pgfpathcurveto{\pgfqpoint{1.455465in}{3.043426in}}{\pgfqpoint{1.444866in}{3.047817in}}{\pgfqpoint{1.433816in}{3.047817in}}%
\pgfpathcurveto{\pgfqpoint{1.422765in}{3.047817in}}{\pgfqpoint{1.412166in}{3.043426in}}{\pgfqpoint{1.404353in}{3.035613in}}%
\pgfpathcurveto{\pgfqpoint{1.396539in}{3.027799in}}{\pgfqpoint{1.392149in}{3.017200in}}{\pgfqpoint{1.392149in}{3.006150in}}%
\pgfpathcurveto{\pgfqpoint{1.392149in}{2.995100in}}{\pgfqpoint{1.396539in}{2.984501in}}{\pgfqpoint{1.404353in}{2.976687in}}%
\pgfpathcurveto{\pgfqpoint{1.412166in}{2.968874in}}{\pgfqpoint{1.422765in}{2.964483in}}{\pgfqpoint{1.433816in}{2.964483in}}%
\pgfpathclose%
\pgfusepath{stroke,fill}%
\end{pgfscope}%
\begin{pgfscope}%
\pgfpathrectangle{\pgfqpoint{0.787074in}{0.548769in}}{\pgfqpoint{5.062926in}{3.102590in}}%
\pgfusepath{clip}%
\pgfsetbuttcap%
\pgfsetroundjoin%
\definecolor{currentfill}{rgb}{1.000000,0.498039,0.054902}%
\pgfsetfillcolor{currentfill}%
\pgfsetlinewidth{1.003750pt}%
\definecolor{currentstroke}{rgb}{1.000000,0.498039,0.054902}%
\pgfsetstrokecolor{currentstroke}%
\pgfsetdash{}{0pt}%
\pgfpathmoveto{\pgfqpoint{1.853983in}{2.843196in}}%
\pgfpathcurveto{\pgfqpoint{1.865033in}{2.843196in}}{\pgfqpoint{1.875632in}{2.847587in}}{\pgfqpoint{1.883446in}{2.855400in}}%
\pgfpathcurveto{\pgfqpoint{1.891260in}{2.863214in}}{\pgfqpoint{1.895650in}{2.873813in}}{\pgfqpoint{1.895650in}{2.884863in}}%
\pgfpathcurveto{\pgfqpoint{1.895650in}{2.895913in}}{\pgfqpoint{1.891260in}{2.906512in}}{\pgfqpoint{1.883446in}{2.914326in}}%
\pgfpathcurveto{\pgfqpoint{1.875632in}{2.922140in}}{\pgfqpoint{1.865033in}{2.926530in}}{\pgfqpoint{1.853983in}{2.926530in}}%
\pgfpathcurveto{\pgfqpoint{1.842933in}{2.926530in}}{\pgfqpoint{1.832334in}{2.922140in}}{\pgfqpoint{1.824520in}{2.914326in}}%
\pgfpathcurveto{\pgfqpoint{1.816707in}{2.906512in}}{\pgfqpoint{1.812316in}{2.895913in}}{\pgfqpoint{1.812316in}{2.884863in}}%
\pgfpathcurveto{\pgfqpoint{1.812316in}{2.873813in}}{\pgfqpoint{1.816707in}{2.863214in}}{\pgfqpoint{1.824520in}{2.855400in}}%
\pgfpathcurveto{\pgfqpoint{1.832334in}{2.847587in}}{\pgfqpoint{1.842933in}{2.843196in}}{\pgfqpoint{1.853983in}{2.843196in}}%
\pgfpathclose%
\pgfusepath{stroke,fill}%
\end{pgfscope}%
\begin{pgfscope}%
\pgfpathrectangle{\pgfqpoint{0.787074in}{0.548769in}}{\pgfqpoint{5.062926in}{3.102590in}}%
\pgfusepath{clip}%
\pgfsetbuttcap%
\pgfsetroundjoin%
\definecolor{currentfill}{rgb}{1.000000,0.498039,0.054902}%
\pgfsetfillcolor{currentfill}%
\pgfsetlinewidth{1.003750pt}%
\definecolor{currentstroke}{rgb}{1.000000,0.498039,0.054902}%
\pgfsetstrokecolor{currentstroke}%
\pgfsetdash{}{0pt}%
\pgfpathmoveto{\pgfqpoint{1.734456in}{2.715747in}}%
\pgfpathcurveto{\pgfqpoint{1.745506in}{2.715747in}}{\pgfqpoint{1.756105in}{2.720137in}}{\pgfqpoint{1.763919in}{2.727951in}}%
\pgfpathcurveto{\pgfqpoint{1.771732in}{2.735764in}}{\pgfqpoint{1.776123in}{2.746363in}}{\pgfqpoint{1.776123in}{2.757413in}}%
\pgfpathcurveto{\pgfqpoint{1.776123in}{2.768463in}}{\pgfqpoint{1.771732in}{2.779062in}}{\pgfqpoint{1.763919in}{2.786876in}}%
\pgfpathcurveto{\pgfqpoint{1.756105in}{2.794690in}}{\pgfqpoint{1.745506in}{2.799080in}}{\pgfqpoint{1.734456in}{2.799080in}}%
\pgfpathcurveto{\pgfqpoint{1.723406in}{2.799080in}}{\pgfqpoint{1.712807in}{2.794690in}}{\pgfqpoint{1.704993in}{2.786876in}}%
\pgfpathcurveto{\pgfqpoint{1.697180in}{2.779062in}}{\pgfqpoint{1.692789in}{2.768463in}}{\pgfqpoint{1.692789in}{2.757413in}}%
\pgfpathcurveto{\pgfqpoint{1.692789in}{2.746363in}}{\pgfqpoint{1.697180in}{2.735764in}}{\pgfqpoint{1.704993in}{2.727951in}}%
\pgfpathcurveto{\pgfqpoint{1.712807in}{2.720137in}}{\pgfqpoint{1.723406in}{2.715747in}}{\pgfqpoint{1.734456in}{2.715747in}}%
\pgfpathclose%
\pgfusepath{stroke,fill}%
\end{pgfscope}%
\begin{pgfscope}%
\pgfpathrectangle{\pgfqpoint{0.787074in}{0.548769in}}{\pgfqpoint{5.062926in}{3.102590in}}%
\pgfusepath{clip}%
\pgfsetbuttcap%
\pgfsetroundjoin%
\definecolor{currentfill}{rgb}{0.121569,0.466667,0.705882}%
\pgfsetfillcolor{currentfill}%
\pgfsetlinewidth{1.003750pt}%
\definecolor{currentstroke}{rgb}{0.121569,0.466667,0.705882}%
\pgfsetstrokecolor{currentstroke}%
\pgfsetdash{}{0pt}%
\pgfpathmoveto{\pgfqpoint{1.076493in}{0.648129in}}%
\pgfpathcurveto{\pgfqpoint{1.087543in}{0.648129in}}{\pgfqpoint{1.098142in}{0.652519in}}{\pgfqpoint{1.105956in}{0.660333in}}%
\pgfpathcurveto{\pgfqpoint{1.113769in}{0.668146in}}{\pgfqpoint{1.118160in}{0.678745in}}{\pgfqpoint{1.118160in}{0.689796in}}%
\pgfpathcurveto{\pgfqpoint{1.118160in}{0.700846in}}{\pgfqpoint{1.113769in}{0.711445in}}{\pgfqpoint{1.105956in}{0.719258in}}%
\pgfpathcurveto{\pgfqpoint{1.098142in}{0.727072in}}{\pgfqpoint{1.087543in}{0.731462in}}{\pgfqpoint{1.076493in}{0.731462in}}%
\pgfpathcurveto{\pgfqpoint{1.065443in}{0.731462in}}{\pgfqpoint{1.054844in}{0.727072in}}{\pgfqpoint{1.047030in}{0.719258in}}%
\pgfpathcurveto{\pgfqpoint{1.039217in}{0.711445in}}{\pgfqpoint{1.034826in}{0.700846in}}{\pgfqpoint{1.034826in}{0.689796in}}%
\pgfpathcurveto{\pgfqpoint{1.034826in}{0.678745in}}{\pgfqpoint{1.039217in}{0.668146in}}{\pgfqpoint{1.047030in}{0.660333in}}%
\pgfpathcurveto{\pgfqpoint{1.054844in}{0.652519in}}{\pgfqpoint{1.065443in}{0.648129in}}{\pgfqpoint{1.076493in}{0.648129in}}%
\pgfpathclose%
\pgfusepath{stroke,fill}%
\end{pgfscope}%
\begin{pgfscope}%
\pgfpathrectangle{\pgfqpoint{0.787074in}{0.548769in}}{\pgfqpoint{5.062926in}{3.102590in}}%
\pgfusepath{clip}%
\pgfsetbuttcap%
\pgfsetroundjoin%
\definecolor{currentfill}{rgb}{0.121569,0.466667,0.705882}%
\pgfsetfillcolor{currentfill}%
\pgfsetlinewidth{1.003750pt}%
\definecolor{currentstroke}{rgb}{0.121569,0.466667,0.705882}%
\pgfsetstrokecolor{currentstroke}%
\pgfsetdash{}{0pt}%
\pgfpathmoveto{\pgfqpoint{1.017207in}{0.665742in}}%
\pgfpathcurveto{\pgfqpoint{1.028257in}{0.665742in}}{\pgfqpoint{1.038856in}{0.670132in}}{\pgfqpoint{1.046670in}{0.677946in}}%
\pgfpathcurveto{\pgfqpoint{1.054483in}{0.685760in}}{\pgfqpoint{1.058874in}{0.696359in}}{\pgfqpoint{1.058874in}{0.707409in}}%
\pgfpathcurveto{\pgfqpoint{1.058874in}{0.718459in}}{\pgfqpoint{1.054483in}{0.729058in}}{\pgfqpoint{1.046670in}{0.736872in}}%
\pgfpathcurveto{\pgfqpoint{1.038856in}{0.744685in}}{\pgfqpoint{1.028257in}{0.749076in}}{\pgfqpoint{1.017207in}{0.749076in}}%
\pgfpathcurveto{\pgfqpoint{1.006157in}{0.749076in}}{\pgfqpoint{0.995558in}{0.744685in}}{\pgfqpoint{0.987744in}{0.736872in}}%
\pgfpathcurveto{\pgfqpoint{0.979930in}{0.729058in}}{\pgfqpoint{0.975540in}{0.718459in}}{\pgfqpoint{0.975540in}{0.707409in}}%
\pgfpathcurveto{\pgfqpoint{0.975540in}{0.696359in}}{\pgfqpoint{0.979930in}{0.685760in}}{\pgfqpoint{0.987744in}{0.677946in}}%
\pgfpathcurveto{\pgfqpoint{0.995558in}{0.670132in}}{\pgfqpoint{1.006157in}{0.665742in}}{\pgfqpoint{1.017207in}{0.665742in}}%
\pgfpathclose%
\pgfusepath{stroke,fill}%
\end{pgfscope}%
\begin{pgfscope}%
\pgfpathrectangle{\pgfqpoint{0.787074in}{0.548769in}}{\pgfqpoint{5.062926in}{3.102590in}}%
\pgfusepath{clip}%
\pgfsetbuttcap%
\pgfsetroundjoin%
\definecolor{currentfill}{rgb}{0.121569,0.466667,0.705882}%
\pgfsetfillcolor{currentfill}%
\pgfsetlinewidth{1.003750pt}%
\definecolor{currentstroke}{rgb}{0.121569,0.466667,0.705882}%
\pgfsetstrokecolor{currentstroke}%
\pgfsetdash{}{0pt}%
\pgfpathmoveto{\pgfqpoint{2.061181in}{1.038852in}}%
\pgfpathcurveto{\pgfqpoint{2.072231in}{1.038852in}}{\pgfqpoint{2.082830in}{1.043242in}}{\pgfqpoint{2.090644in}{1.051055in}}%
\pgfpathcurveto{\pgfqpoint{2.098457in}{1.058869in}}{\pgfqpoint{2.102847in}{1.069468in}}{\pgfqpoint{2.102847in}{1.080518in}}%
\pgfpathcurveto{\pgfqpoint{2.102847in}{1.091568in}}{\pgfqpoint{2.098457in}{1.102167in}}{\pgfqpoint{2.090644in}{1.109981in}}%
\pgfpathcurveto{\pgfqpoint{2.082830in}{1.117795in}}{\pgfqpoint{2.072231in}{1.122185in}}{\pgfqpoint{2.061181in}{1.122185in}}%
\pgfpathcurveto{\pgfqpoint{2.050131in}{1.122185in}}{\pgfqpoint{2.039532in}{1.117795in}}{\pgfqpoint{2.031718in}{1.109981in}}%
\pgfpathcurveto{\pgfqpoint{2.023904in}{1.102167in}}{\pgfqpoint{2.019514in}{1.091568in}}{\pgfqpoint{2.019514in}{1.080518in}}%
\pgfpathcurveto{\pgfqpoint{2.019514in}{1.069468in}}{\pgfqpoint{2.023904in}{1.058869in}}{\pgfqpoint{2.031718in}{1.051055in}}%
\pgfpathcurveto{\pgfqpoint{2.039532in}{1.043242in}}{\pgfqpoint{2.050131in}{1.038852in}}{\pgfqpoint{2.061181in}{1.038852in}}%
\pgfpathclose%
\pgfusepath{stroke,fill}%
\end{pgfscope}%
\begin{pgfscope}%
\pgfpathrectangle{\pgfqpoint{0.787074in}{0.548769in}}{\pgfqpoint{5.062926in}{3.102590in}}%
\pgfusepath{clip}%
\pgfsetbuttcap%
\pgfsetroundjoin%
\definecolor{currentfill}{rgb}{1.000000,0.498039,0.054902}%
\pgfsetfillcolor{currentfill}%
\pgfsetlinewidth{1.003750pt}%
\definecolor{currentstroke}{rgb}{1.000000,0.498039,0.054902}%
\pgfsetstrokecolor{currentstroke}%
\pgfsetdash{}{0pt}%
\pgfpathmoveto{\pgfqpoint{1.689623in}{2.996726in}}%
\pgfpathcurveto{\pgfqpoint{1.700673in}{2.996726in}}{\pgfqpoint{1.711272in}{3.001116in}}{\pgfqpoint{1.719085in}{3.008930in}}%
\pgfpathcurveto{\pgfqpoint{1.726899in}{3.016743in}}{\pgfqpoint{1.731289in}{3.027342in}}{\pgfqpoint{1.731289in}{3.038392in}}%
\pgfpathcurveto{\pgfqpoint{1.731289in}{3.049443in}}{\pgfqpoint{1.726899in}{3.060042in}}{\pgfqpoint{1.719085in}{3.067855in}}%
\pgfpathcurveto{\pgfqpoint{1.711272in}{3.075669in}}{\pgfqpoint{1.700673in}{3.080059in}}{\pgfqpoint{1.689623in}{3.080059in}}%
\pgfpathcurveto{\pgfqpoint{1.678572in}{3.080059in}}{\pgfqpoint{1.667973in}{3.075669in}}{\pgfqpoint{1.660160in}{3.067855in}}%
\pgfpathcurveto{\pgfqpoint{1.652346in}{3.060042in}}{\pgfqpoint{1.647956in}{3.049443in}}{\pgfqpoint{1.647956in}{3.038392in}}%
\pgfpathcurveto{\pgfqpoint{1.647956in}{3.027342in}}{\pgfqpoint{1.652346in}{3.016743in}}{\pgfqpoint{1.660160in}{3.008930in}}%
\pgfpathcurveto{\pgfqpoint{1.667973in}{3.001116in}}{\pgfqpoint{1.678572in}{2.996726in}}{\pgfqpoint{1.689623in}{2.996726in}}%
\pgfpathclose%
\pgfusepath{stroke,fill}%
\end{pgfscope}%
\begin{pgfscope}%
\pgfpathrectangle{\pgfqpoint{0.787074in}{0.548769in}}{\pgfqpoint{5.062926in}{3.102590in}}%
\pgfusepath{clip}%
\pgfsetbuttcap%
\pgfsetroundjoin%
\definecolor{currentfill}{rgb}{0.121569,0.466667,0.705882}%
\pgfsetfillcolor{currentfill}%
\pgfsetlinewidth{1.003750pt}%
\definecolor{currentstroke}{rgb}{0.121569,0.466667,0.705882}%
\pgfsetstrokecolor{currentstroke}%
\pgfsetdash{}{0pt}%
\pgfpathmoveto{\pgfqpoint{1.285644in}{0.648130in}}%
\pgfpathcurveto{\pgfqpoint{1.296694in}{0.648130in}}{\pgfqpoint{1.307293in}{0.652521in}}{\pgfqpoint{1.315106in}{0.660334in}}%
\pgfpathcurveto{\pgfqpoint{1.322920in}{0.668148in}}{\pgfqpoint{1.327310in}{0.678747in}}{\pgfqpoint{1.327310in}{0.689797in}}%
\pgfpathcurveto{\pgfqpoint{1.327310in}{0.700847in}}{\pgfqpoint{1.322920in}{0.711446in}}{\pgfqpoint{1.315106in}{0.719260in}}%
\pgfpathcurveto{\pgfqpoint{1.307293in}{0.727073in}}{\pgfqpoint{1.296694in}{0.731464in}}{\pgfqpoint{1.285644in}{0.731464in}}%
\pgfpathcurveto{\pgfqpoint{1.274594in}{0.731464in}}{\pgfqpoint{1.263994in}{0.727073in}}{\pgfqpoint{1.256181in}{0.719260in}}%
\pgfpathcurveto{\pgfqpoint{1.248367in}{0.711446in}}{\pgfqpoint{1.243977in}{0.700847in}}{\pgfqpoint{1.243977in}{0.689797in}}%
\pgfpathcurveto{\pgfqpoint{1.243977in}{0.678747in}}{\pgfqpoint{1.248367in}{0.668148in}}{\pgfqpoint{1.256181in}{0.660334in}}%
\pgfpathcurveto{\pgfqpoint{1.263994in}{0.652521in}}{\pgfqpoint{1.274594in}{0.648130in}}{\pgfqpoint{1.285644in}{0.648130in}}%
\pgfpathclose%
\pgfusepath{stroke,fill}%
\end{pgfscope}%
\begin{pgfscope}%
\pgfpathrectangle{\pgfqpoint{0.787074in}{0.548769in}}{\pgfqpoint{5.062926in}{3.102590in}}%
\pgfusepath{clip}%
\pgfsetbuttcap%
\pgfsetroundjoin%
\definecolor{currentfill}{rgb}{0.121569,0.466667,0.705882}%
\pgfsetfillcolor{currentfill}%
\pgfsetlinewidth{1.003750pt}%
\definecolor{currentstroke}{rgb}{0.121569,0.466667,0.705882}%
\pgfsetstrokecolor{currentstroke}%
\pgfsetdash{}{0pt}%
\pgfpathmoveto{\pgfqpoint{1.032310in}{0.648134in}}%
\pgfpathcurveto{\pgfqpoint{1.043361in}{0.648134in}}{\pgfqpoint{1.053960in}{0.652524in}}{\pgfqpoint{1.061773in}{0.660338in}}%
\pgfpathcurveto{\pgfqpoint{1.069587in}{0.668152in}}{\pgfqpoint{1.073977in}{0.678751in}}{\pgfqpoint{1.073977in}{0.689801in}}%
\pgfpathcurveto{\pgfqpoint{1.073977in}{0.700851in}}{\pgfqpoint{1.069587in}{0.711450in}}{\pgfqpoint{1.061773in}{0.719264in}}%
\pgfpathcurveto{\pgfqpoint{1.053960in}{0.727077in}}{\pgfqpoint{1.043361in}{0.731467in}}{\pgfqpoint{1.032310in}{0.731467in}}%
\pgfpathcurveto{\pgfqpoint{1.021260in}{0.731467in}}{\pgfqpoint{1.010661in}{0.727077in}}{\pgfqpoint{1.002848in}{0.719264in}}%
\pgfpathcurveto{\pgfqpoint{0.995034in}{0.711450in}}{\pgfqpoint{0.990644in}{0.700851in}}{\pgfqpoint{0.990644in}{0.689801in}}%
\pgfpathcurveto{\pgfqpoint{0.990644in}{0.678751in}}{\pgfqpoint{0.995034in}{0.668152in}}{\pgfqpoint{1.002848in}{0.660338in}}%
\pgfpathcurveto{\pgfqpoint{1.010661in}{0.652524in}}{\pgfqpoint{1.021260in}{0.648134in}}{\pgfqpoint{1.032310in}{0.648134in}}%
\pgfpathclose%
\pgfusepath{stroke,fill}%
\end{pgfscope}%
\begin{pgfscope}%
\pgfpathrectangle{\pgfqpoint{0.787074in}{0.548769in}}{\pgfqpoint{5.062926in}{3.102590in}}%
\pgfusepath{clip}%
\pgfsetbuttcap%
\pgfsetroundjoin%
\definecolor{currentfill}{rgb}{0.839216,0.152941,0.156863}%
\pgfsetfillcolor{currentfill}%
\pgfsetlinewidth{1.003750pt}%
\definecolor{currentstroke}{rgb}{0.839216,0.152941,0.156863}%
\pgfsetstrokecolor{currentstroke}%
\pgfsetdash{}{0pt}%
\pgfpathmoveto{\pgfqpoint{1.318846in}{3.468665in}}%
\pgfpathcurveto{\pgfqpoint{1.329896in}{3.468665in}}{\pgfqpoint{1.340495in}{3.473055in}}{\pgfqpoint{1.348308in}{3.480869in}}%
\pgfpathcurveto{\pgfqpoint{1.356122in}{3.488683in}}{\pgfqpoint{1.360512in}{3.499282in}}{\pgfqpoint{1.360512in}{3.510332in}}%
\pgfpathcurveto{\pgfqpoint{1.360512in}{3.521382in}}{\pgfqpoint{1.356122in}{3.531981in}}{\pgfqpoint{1.348308in}{3.539795in}}%
\pgfpathcurveto{\pgfqpoint{1.340495in}{3.547608in}}{\pgfqpoint{1.329896in}{3.551998in}}{\pgfqpoint{1.318846in}{3.551998in}}%
\pgfpathcurveto{\pgfqpoint{1.307795in}{3.551998in}}{\pgfqpoint{1.297196in}{3.547608in}}{\pgfqpoint{1.289383in}{3.539795in}}%
\pgfpathcurveto{\pgfqpoint{1.281569in}{3.531981in}}{\pgfqpoint{1.277179in}{3.521382in}}{\pgfqpoint{1.277179in}{3.510332in}}%
\pgfpathcurveto{\pgfqpoint{1.277179in}{3.499282in}}{\pgfqpoint{1.281569in}{3.488683in}}{\pgfqpoint{1.289383in}{3.480869in}}%
\pgfpathcurveto{\pgfqpoint{1.297196in}{3.473055in}}{\pgfqpoint{1.307795in}{3.468665in}}{\pgfqpoint{1.318846in}{3.468665in}}%
\pgfpathclose%
\pgfusepath{stroke,fill}%
\end{pgfscope}%
\begin{pgfscope}%
\pgfpathrectangle{\pgfqpoint{0.787074in}{0.548769in}}{\pgfqpoint{5.062926in}{3.102590in}}%
\pgfusepath{clip}%
\pgfsetbuttcap%
\pgfsetroundjoin%
\definecolor{currentfill}{rgb}{0.121569,0.466667,0.705882}%
\pgfsetfillcolor{currentfill}%
\pgfsetlinewidth{1.003750pt}%
\definecolor{currentstroke}{rgb}{0.121569,0.466667,0.705882}%
\pgfsetstrokecolor{currentstroke}%
\pgfsetdash{}{0pt}%
\pgfpathmoveto{\pgfqpoint{1.991044in}{3.009766in}}%
\pgfpathcurveto{\pgfqpoint{2.002094in}{3.009766in}}{\pgfqpoint{2.012693in}{3.014156in}}{\pgfqpoint{2.020507in}{3.021970in}}%
\pgfpathcurveto{\pgfqpoint{2.028321in}{3.029783in}}{\pgfqpoint{2.032711in}{3.040383in}}{\pgfqpoint{2.032711in}{3.051433in}}%
\pgfpathcurveto{\pgfqpoint{2.032711in}{3.062483in}}{\pgfqpoint{2.028321in}{3.073082in}}{\pgfqpoint{2.020507in}{3.080895in}}%
\pgfpathcurveto{\pgfqpoint{2.012693in}{3.088709in}}{\pgfqpoint{2.002094in}{3.093099in}}{\pgfqpoint{1.991044in}{3.093099in}}%
\pgfpathcurveto{\pgfqpoint{1.979994in}{3.093099in}}{\pgfqpoint{1.969395in}{3.088709in}}{\pgfqpoint{1.961582in}{3.080895in}}%
\pgfpathcurveto{\pgfqpoint{1.953768in}{3.073082in}}{\pgfqpoint{1.949378in}{3.062483in}}{\pgfqpoint{1.949378in}{3.051433in}}%
\pgfpathcurveto{\pgfqpoint{1.949378in}{3.040383in}}{\pgfqpoint{1.953768in}{3.029783in}}{\pgfqpoint{1.961582in}{3.021970in}}%
\pgfpathcurveto{\pgfqpoint{1.969395in}{3.014156in}}{\pgfqpoint{1.979994in}{3.009766in}}{\pgfqpoint{1.991044in}{3.009766in}}%
\pgfpathclose%
\pgfusepath{stroke,fill}%
\end{pgfscope}%
\begin{pgfscope}%
\pgfpathrectangle{\pgfqpoint{0.787074in}{0.548769in}}{\pgfqpoint{5.062926in}{3.102590in}}%
\pgfusepath{clip}%
\pgfsetbuttcap%
\pgfsetroundjoin%
\definecolor{currentfill}{rgb}{1.000000,0.498039,0.054902}%
\pgfsetfillcolor{currentfill}%
\pgfsetlinewidth{1.003750pt}%
\definecolor{currentstroke}{rgb}{1.000000,0.498039,0.054902}%
\pgfsetstrokecolor{currentstroke}%
\pgfsetdash{}{0pt}%
\pgfpathmoveto{\pgfqpoint{2.367724in}{3.140942in}}%
\pgfpathcurveto{\pgfqpoint{2.378774in}{3.140942in}}{\pgfqpoint{2.389373in}{3.145332in}}{\pgfqpoint{2.397187in}{3.153146in}}%
\pgfpathcurveto{\pgfqpoint{2.405000in}{3.160959in}}{\pgfqpoint{2.409390in}{3.171558in}}{\pgfqpoint{2.409390in}{3.182609in}}%
\pgfpathcurveto{\pgfqpoint{2.409390in}{3.193659in}}{\pgfqpoint{2.405000in}{3.204258in}}{\pgfqpoint{2.397187in}{3.212071in}}%
\pgfpathcurveto{\pgfqpoint{2.389373in}{3.219885in}}{\pgfqpoint{2.378774in}{3.224275in}}{\pgfqpoint{2.367724in}{3.224275in}}%
\pgfpathcurveto{\pgfqpoint{2.356674in}{3.224275in}}{\pgfqpoint{2.346075in}{3.219885in}}{\pgfqpoint{2.338261in}{3.212071in}}%
\pgfpathcurveto{\pgfqpoint{2.330447in}{3.204258in}}{\pgfqpoint{2.326057in}{3.193659in}}{\pgfqpoint{2.326057in}{3.182609in}}%
\pgfpathcurveto{\pgfqpoint{2.326057in}{3.171558in}}{\pgfqpoint{2.330447in}{3.160959in}}{\pgfqpoint{2.338261in}{3.153146in}}%
\pgfpathcurveto{\pgfqpoint{2.346075in}{3.145332in}}{\pgfqpoint{2.356674in}{3.140942in}}{\pgfqpoint{2.367724in}{3.140942in}}%
\pgfpathclose%
\pgfusepath{stroke,fill}%
\end{pgfscope}%
\begin{pgfscope}%
\pgfpathrectangle{\pgfqpoint{0.787074in}{0.548769in}}{\pgfqpoint{5.062926in}{3.102590in}}%
\pgfusepath{clip}%
\pgfsetbuttcap%
\pgfsetroundjoin%
\definecolor{currentfill}{rgb}{1.000000,0.498039,0.054902}%
\pgfsetfillcolor{currentfill}%
\pgfsetlinewidth{1.003750pt}%
\definecolor{currentstroke}{rgb}{1.000000,0.498039,0.054902}%
\pgfsetstrokecolor{currentstroke}%
\pgfsetdash{}{0pt}%
\pgfpathmoveto{\pgfqpoint{1.878765in}{2.861427in}}%
\pgfpathcurveto{\pgfqpoint{1.889815in}{2.861427in}}{\pgfqpoint{1.900414in}{2.865817in}}{\pgfqpoint{1.908228in}{2.873631in}}%
\pgfpathcurveto{\pgfqpoint{1.916042in}{2.881444in}}{\pgfqpoint{1.920432in}{2.892043in}}{\pgfqpoint{1.920432in}{2.903094in}}%
\pgfpathcurveto{\pgfqpoint{1.920432in}{2.914144in}}{\pgfqpoint{1.916042in}{2.924743in}}{\pgfqpoint{1.908228in}{2.932556in}}%
\pgfpathcurveto{\pgfqpoint{1.900414in}{2.940370in}}{\pgfqpoint{1.889815in}{2.944760in}}{\pgfqpoint{1.878765in}{2.944760in}}%
\pgfpathcurveto{\pgfqpoint{1.867715in}{2.944760in}}{\pgfqpoint{1.857116in}{2.940370in}}{\pgfqpoint{1.849302in}{2.932556in}}%
\pgfpathcurveto{\pgfqpoint{1.841489in}{2.924743in}}{\pgfqpoint{1.837099in}{2.914144in}}{\pgfqpoint{1.837099in}{2.903094in}}%
\pgfpathcurveto{\pgfqpoint{1.837099in}{2.892043in}}{\pgfqpoint{1.841489in}{2.881444in}}{\pgfqpoint{1.849302in}{2.873631in}}%
\pgfpathcurveto{\pgfqpoint{1.857116in}{2.865817in}}{\pgfqpoint{1.867715in}{2.861427in}}{\pgfqpoint{1.878765in}{2.861427in}}%
\pgfpathclose%
\pgfusepath{stroke,fill}%
\end{pgfscope}%
\begin{pgfscope}%
\pgfpathrectangle{\pgfqpoint{0.787074in}{0.548769in}}{\pgfqpoint{5.062926in}{3.102590in}}%
\pgfusepath{clip}%
\pgfsetbuttcap%
\pgfsetroundjoin%
\definecolor{currentfill}{rgb}{1.000000,0.498039,0.054902}%
\pgfsetfillcolor{currentfill}%
\pgfsetlinewidth{1.003750pt}%
\definecolor{currentstroke}{rgb}{1.000000,0.498039,0.054902}%
\pgfsetstrokecolor{currentstroke}%
\pgfsetdash{}{0pt}%
\pgfpathmoveto{\pgfqpoint{1.786538in}{3.035300in}}%
\pgfpathcurveto{\pgfqpoint{1.797588in}{3.035300in}}{\pgfqpoint{1.808187in}{3.039690in}}{\pgfqpoint{1.816000in}{3.047504in}}%
\pgfpathcurveto{\pgfqpoint{1.823814in}{3.055317in}}{\pgfqpoint{1.828204in}{3.065916in}}{\pgfqpoint{1.828204in}{3.076967in}}%
\pgfpathcurveto{\pgfqpoint{1.828204in}{3.088017in}}{\pgfqpoint{1.823814in}{3.098616in}}{\pgfqpoint{1.816000in}{3.106429in}}%
\pgfpathcurveto{\pgfqpoint{1.808187in}{3.114243in}}{\pgfqpoint{1.797588in}{3.118633in}}{\pgfqpoint{1.786538in}{3.118633in}}%
\pgfpathcurveto{\pgfqpoint{1.775487in}{3.118633in}}{\pgfqpoint{1.764888in}{3.114243in}}{\pgfqpoint{1.757075in}{3.106429in}}%
\pgfpathcurveto{\pgfqpoint{1.749261in}{3.098616in}}{\pgfqpoint{1.744871in}{3.088017in}}{\pgfqpoint{1.744871in}{3.076967in}}%
\pgfpathcurveto{\pgfqpoint{1.744871in}{3.065916in}}{\pgfqpoint{1.749261in}{3.055317in}}{\pgfqpoint{1.757075in}{3.047504in}}%
\pgfpathcurveto{\pgfqpoint{1.764888in}{3.039690in}}{\pgfqpoint{1.775487in}{3.035300in}}{\pgfqpoint{1.786538in}{3.035300in}}%
\pgfpathclose%
\pgfusepath{stroke,fill}%
\end{pgfscope}%
\begin{pgfscope}%
\pgfpathrectangle{\pgfqpoint{0.787074in}{0.548769in}}{\pgfqpoint{5.062926in}{3.102590in}}%
\pgfusepath{clip}%
\pgfsetbuttcap%
\pgfsetroundjoin%
\definecolor{currentfill}{rgb}{1.000000,0.498039,0.054902}%
\pgfsetfillcolor{currentfill}%
\pgfsetlinewidth{1.003750pt}%
\definecolor{currentstroke}{rgb}{1.000000,0.498039,0.054902}%
\pgfsetstrokecolor{currentstroke}%
\pgfsetdash{}{0pt}%
\pgfpathmoveto{\pgfqpoint{1.731765in}{2.247058in}}%
\pgfpathcurveto{\pgfqpoint{1.742815in}{2.247058in}}{\pgfqpoint{1.753414in}{2.251449in}}{\pgfqpoint{1.761228in}{2.259262in}}%
\pgfpathcurveto{\pgfqpoint{1.769042in}{2.267076in}}{\pgfqpoint{1.773432in}{2.277675in}}{\pgfqpoint{1.773432in}{2.288725in}}%
\pgfpathcurveto{\pgfqpoint{1.773432in}{2.299775in}}{\pgfqpoint{1.769042in}{2.310374in}}{\pgfqpoint{1.761228in}{2.318188in}}%
\pgfpathcurveto{\pgfqpoint{1.753414in}{2.326002in}}{\pgfqpoint{1.742815in}{2.330392in}}{\pgfqpoint{1.731765in}{2.330392in}}%
\pgfpathcurveto{\pgfqpoint{1.720715in}{2.330392in}}{\pgfqpoint{1.710116in}{2.326002in}}{\pgfqpoint{1.702302in}{2.318188in}}%
\pgfpathcurveto{\pgfqpoint{1.694489in}{2.310374in}}{\pgfqpoint{1.690099in}{2.299775in}}{\pgfqpoint{1.690099in}{2.288725in}}%
\pgfpathcurveto{\pgfqpoint{1.690099in}{2.277675in}}{\pgfqpoint{1.694489in}{2.267076in}}{\pgfqpoint{1.702302in}{2.259262in}}%
\pgfpathcurveto{\pgfqpoint{1.710116in}{2.251449in}}{\pgfqpoint{1.720715in}{2.247058in}}{\pgfqpoint{1.731765in}{2.247058in}}%
\pgfpathclose%
\pgfusepath{stroke,fill}%
\end{pgfscope}%
\begin{pgfscope}%
\pgfpathrectangle{\pgfqpoint{0.787074in}{0.548769in}}{\pgfqpoint{5.062926in}{3.102590in}}%
\pgfusepath{clip}%
\pgfsetbuttcap%
\pgfsetroundjoin%
\definecolor{currentfill}{rgb}{1.000000,0.498039,0.054902}%
\pgfsetfillcolor{currentfill}%
\pgfsetlinewidth{1.003750pt}%
\definecolor{currentstroke}{rgb}{1.000000,0.498039,0.054902}%
\pgfsetstrokecolor{currentstroke}%
\pgfsetdash{}{0pt}%
\pgfpathmoveto{\pgfqpoint{1.456384in}{2.894377in}}%
\pgfpathcurveto{\pgfqpoint{1.467434in}{2.894377in}}{\pgfqpoint{1.478033in}{2.898767in}}{\pgfqpoint{1.485847in}{2.906581in}}%
\pgfpathcurveto{\pgfqpoint{1.493661in}{2.914394in}}{\pgfqpoint{1.498051in}{2.924993in}}{\pgfqpoint{1.498051in}{2.936044in}}%
\pgfpathcurveto{\pgfqpoint{1.498051in}{2.947094in}}{\pgfqpoint{1.493661in}{2.957693in}}{\pgfqpoint{1.485847in}{2.965506in}}%
\pgfpathcurveto{\pgfqpoint{1.478033in}{2.973320in}}{\pgfqpoint{1.467434in}{2.977710in}}{\pgfqpoint{1.456384in}{2.977710in}}%
\pgfpathcurveto{\pgfqpoint{1.445334in}{2.977710in}}{\pgfqpoint{1.434735in}{2.973320in}}{\pgfqpoint{1.426921in}{2.965506in}}%
\pgfpathcurveto{\pgfqpoint{1.419108in}{2.957693in}}{\pgfqpoint{1.414718in}{2.947094in}}{\pgfqpoint{1.414718in}{2.936044in}}%
\pgfpathcurveto{\pgfqpoint{1.414718in}{2.924993in}}{\pgfqpoint{1.419108in}{2.914394in}}{\pgfqpoint{1.426921in}{2.906581in}}%
\pgfpathcurveto{\pgfqpoint{1.434735in}{2.898767in}}{\pgfqpoint{1.445334in}{2.894377in}}{\pgfqpoint{1.456384in}{2.894377in}}%
\pgfpathclose%
\pgfusepath{stroke,fill}%
\end{pgfscope}%
\begin{pgfscope}%
\pgfpathrectangle{\pgfqpoint{0.787074in}{0.548769in}}{\pgfqpoint{5.062926in}{3.102590in}}%
\pgfusepath{clip}%
\pgfsetbuttcap%
\pgfsetroundjoin%
\definecolor{currentfill}{rgb}{1.000000,0.498039,0.054902}%
\pgfsetfillcolor{currentfill}%
\pgfsetlinewidth{1.003750pt}%
\definecolor{currentstroke}{rgb}{1.000000,0.498039,0.054902}%
\pgfsetstrokecolor{currentstroke}%
\pgfsetdash{}{0pt}%
\pgfpathmoveto{\pgfqpoint{2.181012in}{3.298118in}}%
\pgfpathcurveto{\pgfqpoint{2.192062in}{3.298118in}}{\pgfqpoint{2.202661in}{3.302508in}}{\pgfqpoint{2.210474in}{3.310321in}}%
\pgfpathcurveto{\pgfqpoint{2.218288in}{3.318135in}}{\pgfqpoint{2.222678in}{3.328734in}}{\pgfqpoint{2.222678in}{3.339784in}}%
\pgfpathcurveto{\pgfqpoint{2.222678in}{3.350834in}}{\pgfqpoint{2.218288in}{3.361433in}}{\pgfqpoint{2.210474in}{3.369247in}}%
\pgfpathcurveto{\pgfqpoint{2.202661in}{3.377061in}}{\pgfqpoint{2.192062in}{3.381451in}}{\pgfqpoint{2.181012in}{3.381451in}}%
\pgfpathcurveto{\pgfqpoint{2.169961in}{3.381451in}}{\pgfqpoint{2.159362in}{3.377061in}}{\pgfqpoint{2.151549in}{3.369247in}}%
\pgfpathcurveto{\pgfqpoint{2.143735in}{3.361433in}}{\pgfqpoint{2.139345in}{3.350834in}}{\pgfqpoint{2.139345in}{3.339784in}}%
\pgfpathcurveto{\pgfqpoint{2.139345in}{3.328734in}}{\pgfqpoint{2.143735in}{3.318135in}}{\pgfqpoint{2.151549in}{3.310321in}}%
\pgfpathcurveto{\pgfqpoint{2.159362in}{3.302508in}}{\pgfqpoint{2.169961in}{3.298118in}}{\pgfqpoint{2.181012in}{3.298118in}}%
\pgfpathclose%
\pgfusepath{stroke,fill}%
\end{pgfscope}%
\begin{pgfscope}%
\pgfpathrectangle{\pgfqpoint{0.787074in}{0.548769in}}{\pgfqpoint{5.062926in}{3.102590in}}%
\pgfusepath{clip}%
\pgfsetbuttcap%
\pgfsetroundjoin%
\definecolor{currentfill}{rgb}{1.000000,0.498039,0.054902}%
\pgfsetfillcolor{currentfill}%
\pgfsetlinewidth{1.003750pt}%
\definecolor{currentstroke}{rgb}{1.000000,0.498039,0.054902}%
\pgfsetstrokecolor{currentstroke}%
\pgfsetdash{}{0pt}%
\pgfpathmoveto{\pgfqpoint{2.497537in}{2.566130in}}%
\pgfpathcurveto{\pgfqpoint{2.508587in}{2.566130in}}{\pgfqpoint{2.519186in}{2.570520in}}{\pgfqpoint{2.527000in}{2.578334in}}%
\pgfpathcurveto{\pgfqpoint{2.534813in}{2.586148in}}{\pgfqpoint{2.539204in}{2.596747in}}{\pgfqpoint{2.539204in}{2.607797in}}%
\pgfpathcurveto{\pgfqpoint{2.539204in}{2.618847in}}{\pgfqpoint{2.534813in}{2.629446in}}{\pgfqpoint{2.527000in}{2.637260in}}%
\pgfpathcurveto{\pgfqpoint{2.519186in}{2.645073in}}{\pgfqpoint{2.508587in}{2.649463in}}{\pgfqpoint{2.497537in}{2.649463in}}%
\pgfpathcurveto{\pgfqpoint{2.486487in}{2.649463in}}{\pgfqpoint{2.475888in}{2.645073in}}{\pgfqpoint{2.468074in}{2.637260in}}%
\pgfpathcurveto{\pgfqpoint{2.460261in}{2.629446in}}{\pgfqpoint{2.455870in}{2.618847in}}{\pgfqpoint{2.455870in}{2.607797in}}%
\pgfpathcurveto{\pgfqpoint{2.455870in}{2.596747in}}{\pgfqpoint{2.460261in}{2.586148in}}{\pgfqpoint{2.468074in}{2.578334in}}%
\pgfpathcurveto{\pgfqpoint{2.475888in}{2.570520in}}{\pgfqpoint{2.486487in}{2.566130in}}{\pgfqpoint{2.497537in}{2.566130in}}%
\pgfpathclose%
\pgfusepath{stroke,fill}%
\end{pgfscope}%
\begin{pgfscope}%
\pgfpathrectangle{\pgfqpoint{0.787074in}{0.548769in}}{\pgfqpoint{5.062926in}{3.102590in}}%
\pgfusepath{clip}%
\pgfsetbuttcap%
\pgfsetroundjoin%
\definecolor{currentfill}{rgb}{0.121569,0.466667,0.705882}%
\pgfsetfillcolor{currentfill}%
\pgfsetlinewidth{1.003750pt}%
\definecolor{currentstroke}{rgb}{0.121569,0.466667,0.705882}%
\pgfsetstrokecolor{currentstroke}%
\pgfsetdash{}{0pt}%
\pgfpathmoveto{\pgfqpoint{4.331457in}{2.795473in}}%
\pgfpathcurveto{\pgfqpoint{4.342507in}{2.795473in}}{\pgfqpoint{4.353106in}{2.799863in}}{\pgfqpoint{4.360920in}{2.807676in}}%
\pgfpathcurveto{\pgfqpoint{4.368734in}{2.815490in}}{\pgfqpoint{4.373124in}{2.826089in}}{\pgfqpoint{4.373124in}{2.837139in}}%
\pgfpathcurveto{\pgfqpoint{4.373124in}{2.848189in}}{\pgfqpoint{4.368734in}{2.858788in}}{\pgfqpoint{4.360920in}{2.866602in}}%
\pgfpathcurveto{\pgfqpoint{4.353106in}{2.874416in}}{\pgfqpoint{4.342507in}{2.878806in}}{\pgfqpoint{4.331457in}{2.878806in}}%
\pgfpathcurveto{\pgfqpoint{4.320407in}{2.878806in}}{\pgfqpoint{4.309808in}{2.874416in}}{\pgfqpoint{4.301994in}{2.866602in}}%
\pgfpathcurveto{\pgfqpoint{4.294181in}{2.858788in}}{\pgfqpoint{4.289791in}{2.848189in}}{\pgfqpoint{4.289791in}{2.837139in}}%
\pgfpathcurveto{\pgfqpoint{4.289791in}{2.826089in}}{\pgfqpoint{4.294181in}{2.815490in}}{\pgfqpoint{4.301994in}{2.807676in}}%
\pgfpathcurveto{\pgfqpoint{4.309808in}{2.799863in}}{\pgfqpoint{4.320407in}{2.795473in}}{\pgfqpoint{4.331457in}{2.795473in}}%
\pgfpathclose%
\pgfusepath{stroke,fill}%
\end{pgfscope}%
\begin{pgfscope}%
\pgfpathrectangle{\pgfqpoint{0.787074in}{0.548769in}}{\pgfqpoint{5.062926in}{3.102590in}}%
\pgfusepath{clip}%
\pgfsetbuttcap%
\pgfsetroundjoin%
\definecolor{currentfill}{rgb}{1.000000,0.498039,0.054902}%
\pgfsetfillcolor{currentfill}%
\pgfsetlinewidth{1.003750pt}%
\definecolor{currentstroke}{rgb}{1.000000,0.498039,0.054902}%
\pgfsetstrokecolor{currentstroke}%
\pgfsetdash{}{0pt}%
\pgfpathmoveto{\pgfqpoint{1.679597in}{3.027648in}}%
\pgfpathcurveto{\pgfqpoint{1.690647in}{3.027648in}}{\pgfqpoint{1.701246in}{3.032039in}}{\pgfqpoint{1.709060in}{3.039852in}}%
\pgfpathcurveto{\pgfqpoint{1.716873in}{3.047666in}}{\pgfqpoint{1.721264in}{3.058265in}}{\pgfqpoint{1.721264in}{3.069315in}}%
\pgfpathcurveto{\pgfqpoint{1.721264in}{3.080365in}}{\pgfqpoint{1.716873in}{3.090964in}}{\pgfqpoint{1.709060in}{3.098778in}}%
\pgfpathcurveto{\pgfqpoint{1.701246in}{3.106591in}}{\pgfqpoint{1.690647in}{3.110982in}}{\pgfqpoint{1.679597in}{3.110982in}}%
\pgfpathcurveto{\pgfqpoint{1.668547in}{3.110982in}}{\pgfqpoint{1.657948in}{3.106591in}}{\pgfqpoint{1.650134in}{3.098778in}}%
\pgfpathcurveto{\pgfqpoint{1.642320in}{3.090964in}}{\pgfqpoint{1.637930in}{3.080365in}}{\pgfqpoint{1.637930in}{3.069315in}}%
\pgfpathcurveto{\pgfqpoint{1.637930in}{3.058265in}}{\pgfqpoint{1.642320in}{3.047666in}}{\pgfqpoint{1.650134in}{3.039852in}}%
\pgfpathcurveto{\pgfqpoint{1.657948in}{3.032039in}}{\pgfqpoint{1.668547in}{3.027648in}}{\pgfqpoint{1.679597in}{3.027648in}}%
\pgfpathclose%
\pgfusepath{stroke,fill}%
\end{pgfscope}%
\begin{pgfscope}%
\pgfpathrectangle{\pgfqpoint{0.787074in}{0.548769in}}{\pgfqpoint{5.062926in}{3.102590in}}%
\pgfusepath{clip}%
\pgfsetbuttcap%
\pgfsetroundjoin%
\definecolor{currentfill}{rgb}{1.000000,0.498039,0.054902}%
\pgfsetfillcolor{currentfill}%
\pgfsetlinewidth{1.003750pt}%
\definecolor{currentstroke}{rgb}{1.000000,0.498039,0.054902}%
\pgfsetstrokecolor{currentstroke}%
\pgfsetdash{}{0pt}%
\pgfpathmoveto{\pgfqpoint{1.995992in}{2.815221in}}%
\pgfpathcurveto{\pgfqpoint{2.007042in}{2.815221in}}{\pgfqpoint{2.017641in}{2.819611in}}{\pgfqpoint{2.025455in}{2.827425in}}%
\pgfpathcurveto{\pgfqpoint{2.033268in}{2.835238in}}{\pgfqpoint{2.037659in}{2.845837in}}{\pgfqpoint{2.037659in}{2.856887in}}%
\pgfpathcurveto{\pgfqpoint{2.037659in}{2.867937in}}{\pgfqpoint{2.033268in}{2.878537in}}{\pgfqpoint{2.025455in}{2.886350in}}%
\pgfpathcurveto{\pgfqpoint{2.017641in}{2.894164in}}{\pgfqpoint{2.007042in}{2.898554in}}{\pgfqpoint{1.995992in}{2.898554in}}%
\pgfpathcurveto{\pgfqpoint{1.984942in}{2.898554in}}{\pgfqpoint{1.974343in}{2.894164in}}{\pgfqpoint{1.966529in}{2.886350in}}%
\pgfpathcurveto{\pgfqpoint{1.958716in}{2.878537in}}{\pgfqpoint{1.954325in}{2.867937in}}{\pgfqpoint{1.954325in}{2.856887in}}%
\pgfpathcurveto{\pgfqpoint{1.954325in}{2.845837in}}{\pgfqpoint{1.958716in}{2.835238in}}{\pgfqpoint{1.966529in}{2.827425in}}%
\pgfpathcurveto{\pgfqpoint{1.974343in}{2.819611in}}{\pgfqpoint{1.984942in}{2.815221in}}{\pgfqpoint{1.995992in}{2.815221in}}%
\pgfpathclose%
\pgfusepath{stroke,fill}%
\end{pgfscope}%
\begin{pgfscope}%
\pgfpathrectangle{\pgfqpoint{0.787074in}{0.548769in}}{\pgfqpoint{5.062926in}{3.102590in}}%
\pgfusepath{clip}%
\pgfsetbuttcap%
\pgfsetroundjoin%
\definecolor{currentfill}{rgb}{1.000000,0.498039,0.054902}%
\pgfsetfillcolor{currentfill}%
\pgfsetlinewidth{1.003750pt}%
\definecolor{currentstroke}{rgb}{1.000000,0.498039,0.054902}%
\pgfsetstrokecolor{currentstroke}%
\pgfsetdash{}{0pt}%
\pgfpathmoveto{\pgfqpoint{2.053672in}{1.272567in}}%
\pgfpathcurveto{\pgfqpoint{2.064722in}{1.272567in}}{\pgfqpoint{2.075321in}{1.276958in}}{\pgfqpoint{2.083135in}{1.284771in}}%
\pgfpathcurveto{\pgfqpoint{2.090949in}{1.292585in}}{\pgfqpoint{2.095339in}{1.303184in}}{\pgfqpoint{2.095339in}{1.314234in}}%
\pgfpathcurveto{\pgfqpoint{2.095339in}{1.325284in}}{\pgfqpoint{2.090949in}{1.335883in}}{\pgfqpoint{2.083135in}{1.343697in}}%
\pgfpathcurveto{\pgfqpoint{2.075321in}{1.351510in}}{\pgfqpoint{2.064722in}{1.355901in}}{\pgfqpoint{2.053672in}{1.355901in}}%
\pgfpathcurveto{\pgfqpoint{2.042622in}{1.355901in}}{\pgfqpoint{2.032023in}{1.351510in}}{\pgfqpoint{2.024210in}{1.343697in}}%
\pgfpathcurveto{\pgfqpoint{2.016396in}{1.335883in}}{\pgfqpoint{2.012006in}{1.325284in}}{\pgfqpoint{2.012006in}{1.314234in}}%
\pgfpathcurveto{\pgfqpoint{2.012006in}{1.303184in}}{\pgfqpoint{2.016396in}{1.292585in}}{\pgfqpoint{2.024210in}{1.284771in}}%
\pgfpathcurveto{\pgfqpoint{2.032023in}{1.276958in}}{\pgfqpoint{2.042622in}{1.272567in}}{\pgfqpoint{2.053672in}{1.272567in}}%
\pgfpathclose%
\pgfusepath{stroke,fill}%
\end{pgfscope}%
\begin{pgfscope}%
\pgfpathrectangle{\pgfqpoint{0.787074in}{0.548769in}}{\pgfqpoint{5.062926in}{3.102590in}}%
\pgfusepath{clip}%
\pgfsetbuttcap%
\pgfsetroundjoin%
\definecolor{currentfill}{rgb}{0.121569,0.466667,0.705882}%
\pgfsetfillcolor{currentfill}%
\pgfsetlinewidth{1.003750pt}%
\definecolor{currentstroke}{rgb}{0.121569,0.466667,0.705882}%
\pgfsetstrokecolor{currentstroke}%
\pgfsetdash{}{0pt}%
\pgfpathmoveto{\pgfqpoint{2.413990in}{1.853887in}}%
\pgfpathcurveto{\pgfqpoint{2.425040in}{1.853887in}}{\pgfqpoint{2.435639in}{1.858277in}}{\pgfqpoint{2.443452in}{1.866090in}}%
\pgfpathcurveto{\pgfqpoint{2.451266in}{1.873904in}}{\pgfqpoint{2.455656in}{1.884503in}}{\pgfqpoint{2.455656in}{1.895553in}}%
\pgfpathcurveto{\pgfqpoint{2.455656in}{1.906603in}}{\pgfqpoint{2.451266in}{1.917202in}}{\pgfqpoint{2.443452in}{1.925016in}}%
\pgfpathcurveto{\pgfqpoint{2.435639in}{1.932830in}}{\pgfqpoint{2.425040in}{1.937220in}}{\pgfqpoint{2.413990in}{1.937220in}}%
\pgfpathcurveto{\pgfqpoint{2.402939in}{1.937220in}}{\pgfqpoint{2.392340in}{1.932830in}}{\pgfqpoint{2.384527in}{1.925016in}}%
\pgfpathcurveto{\pgfqpoint{2.376713in}{1.917202in}}{\pgfqpoint{2.372323in}{1.906603in}}{\pgfqpoint{2.372323in}{1.895553in}}%
\pgfpathcurveto{\pgfqpoint{2.372323in}{1.884503in}}{\pgfqpoint{2.376713in}{1.873904in}}{\pgfqpoint{2.384527in}{1.866090in}}%
\pgfpathcurveto{\pgfqpoint{2.392340in}{1.858277in}}{\pgfqpoint{2.402939in}{1.853887in}}{\pgfqpoint{2.413990in}{1.853887in}}%
\pgfpathclose%
\pgfusepath{stroke,fill}%
\end{pgfscope}%
\begin{pgfscope}%
\pgfpathrectangle{\pgfqpoint{0.787074in}{0.548769in}}{\pgfqpoint{5.062926in}{3.102590in}}%
\pgfusepath{clip}%
\pgfsetbuttcap%
\pgfsetroundjoin%
\definecolor{currentfill}{rgb}{0.121569,0.466667,0.705882}%
\pgfsetfillcolor{currentfill}%
\pgfsetlinewidth{1.003750pt}%
\definecolor{currentstroke}{rgb}{0.121569,0.466667,0.705882}%
\pgfsetstrokecolor{currentstroke}%
\pgfsetdash{}{0pt}%
\pgfpathmoveto{\pgfqpoint{2.079236in}{2.914142in}}%
\pgfpathcurveto{\pgfqpoint{2.090286in}{2.914142in}}{\pgfqpoint{2.100885in}{2.918533in}}{\pgfqpoint{2.108698in}{2.926346in}}%
\pgfpathcurveto{\pgfqpoint{2.116512in}{2.934160in}}{\pgfqpoint{2.120902in}{2.944759in}}{\pgfqpoint{2.120902in}{2.955809in}}%
\pgfpathcurveto{\pgfqpoint{2.120902in}{2.966859in}}{\pgfqpoint{2.116512in}{2.977458in}}{\pgfqpoint{2.108698in}{2.985272in}}%
\pgfpathcurveto{\pgfqpoint{2.100885in}{2.993085in}}{\pgfqpoint{2.090286in}{2.997476in}}{\pgfqpoint{2.079236in}{2.997476in}}%
\pgfpathcurveto{\pgfqpoint{2.068186in}{2.997476in}}{\pgfqpoint{2.057586in}{2.993085in}}{\pgfqpoint{2.049773in}{2.985272in}}%
\pgfpathcurveto{\pgfqpoint{2.041959in}{2.977458in}}{\pgfqpoint{2.037569in}{2.966859in}}{\pgfqpoint{2.037569in}{2.955809in}}%
\pgfpathcurveto{\pgfqpoint{2.037569in}{2.944759in}}{\pgfqpoint{2.041959in}{2.934160in}}{\pgfqpoint{2.049773in}{2.926346in}}%
\pgfpathcurveto{\pgfqpoint{2.057586in}{2.918533in}}{\pgfqpoint{2.068186in}{2.914142in}}{\pgfqpoint{2.079236in}{2.914142in}}%
\pgfpathclose%
\pgfusepath{stroke,fill}%
\end{pgfscope}%
\begin{pgfscope}%
\pgfpathrectangle{\pgfqpoint{0.787074in}{0.548769in}}{\pgfqpoint{5.062926in}{3.102590in}}%
\pgfusepath{clip}%
\pgfsetbuttcap%
\pgfsetroundjoin%
\definecolor{currentfill}{rgb}{1.000000,0.498039,0.054902}%
\pgfsetfillcolor{currentfill}%
\pgfsetlinewidth{1.003750pt}%
\definecolor{currentstroke}{rgb}{1.000000,0.498039,0.054902}%
\pgfsetstrokecolor{currentstroke}%
\pgfsetdash{}{0pt}%
\pgfpathmoveto{\pgfqpoint{2.073376in}{2.606637in}}%
\pgfpathcurveto{\pgfqpoint{2.084427in}{2.606637in}}{\pgfqpoint{2.095026in}{2.611027in}}{\pgfqpoint{2.102839in}{2.618841in}}%
\pgfpathcurveto{\pgfqpoint{2.110653in}{2.626654in}}{\pgfqpoint{2.115043in}{2.637253in}}{\pgfqpoint{2.115043in}{2.648303in}}%
\pgfpathcurveto{\pgfqpoint{2.115043in}{2.659353in}}{\pgfqpoint{2.110653in}{2.669952in}}{\pgfqpoint{2.102839in}{2.677766in}}%
\pgfpathcurveto{\pgfqpoint{2.095026in}{2.685580in}}{\pgfqpoint{2.084427in}{2.689970in}}{\pgfqpoint{2.073376in}{2.689970in}}%
\pgfpathcurveto{\pgfqpoint{2.062326in}{2.689970in}}{\pgfqpoint{2.051727in}{2.685580in}}{\pgfqpoint{2.043914in}{2.677766in}}%
\pgfpathcurveto{\pgfqpoint{2.036100in}{2.669952in}}{\pgfqpoint{2.031710in}{2.659353in}}{\pgfqpoint{2.031710in}{2.648303in}}%
\pgfpathcurveto{\pgfqpoint{2.031710in}{2.637253in}}{\pgfqpoint{2.036100in}{2.626654in}}{\pgfqpoint{2.043914in}{2.618841in}}%
\pgfpathcurveto{\pgfqpoint{2.051727in}{2.611027in}}{\pgfqpoint{2.062326in}{2.606637in}}{\pgfqpoint{2.073376in}{2.606637in}}%
\pgfpathclose%
\pgfusepath{stroke,fill}%
\end{pgfscope}%
\begin{pgfscope}%
\pgfpathrectangle{\pgfqpoint{0.787074in}{0.548769in}}{\pgfqpoint{5.062926in}{3.102590in}}%
\pgfusepath{clip}%
\pgfsetbuttcap%
\pgfsetroundjoin%
\definecolor{currentfill}{rgb}{0.839216,0.152941,0.156863}%
\pgfsetfillcolor{currentfill}%
\pgfsetlinewidth{1.003750pt}%
\definecolor{currentstroke}{rgb}{0.839216,0.152941,0.156863}%
\pgfsetstrokecolor{currentstroke}%
\pgfsetdash{}{0pt}%
\pgfpathmoveto{\pgfqpoint{2.733553in}{2.324482in}}%
\pgfpathcurveto{\pgfqpoint{2.744603in}{2.324482in}}{\pgfqpoint{2.755202in}{2.328873in}}{\pgfqpoint{2.763016in}{2.336686in}}%
\pgfpathcurveto{\pgfqpoint{2.770829in}{2.344500in}}{\pgfqpoint{2.775220in}{2.355099in}}{\pgfqpoint{2.775220in}{2.366149in}}%
\pgfpathcurveto{\pgfqpoint{2.775220in}{2.377199in}}{\pgfqpoint{2.770829in}{2.387798in}}{\pgfqpoint{2.763016in}{2.395612in}}%
\pgfpathcurveto{\pgfqpoint{2.755202in}{2.403426in}}{\pgfqpoint{2.744603in}{2.407816in}}{\pgfqpoint{2.733553in}{2.407816in}}%
\pgfpathcurveto{\pgfqpoint{2.722503in}{2.407816in}}{\pgfqpoint{2.711904in}{2.403426in}}{\pgfqpoint{2.704090in}{2.395612in}}%
\pgfpathcurveto{\pgfqpoint{2.696277in}{2.387798in}}{\pgfqpoint{2.691886in}{2.377199in}}{\pgfqpoint{2.691886in}{2.366149in}}%
\pgfpathcurveto{\pgfqpoint{2.691886in}{2.355099in}}{\pgfqpoint{2.696277in}{2.344500in}}{\pgfqpoint{2.704090in}{2.336686in}}%
\pgfpathcurveto{\pgfqpoint{2.711904in}{2.328873in}}{\pgfqpoint{2.722503in}{2.324482in}}{\pgfqpoint{2.733553in}{2.324482in}}%
\pgfpathclose%
\pgfusepath{stroke,fill}%
\end{pgfscope}%
\begin{pgfscope}%
\pgfpathrectangle{\pgfqpoint{0.787074in}{0.548769in}}{\pgfqpoint{5.062926in}{3.102590in}}%
\pgfusepath{clip}%
\pgfsetbuttcap%
\pgfsetroundjoin%
\definecolor{currentfill}{rgb}{1.000000,0.498039,0.054902}%
\pgfsetfillcolor{currentfill}%
\pgfsetlinewidth{1.003750pt}%
\definecolor{currentstroke}{rgb}{1.000000,0.498039,0.054902}%
\pgfsetstrokecolor{currentstroke}%
\pgfsetdash{}{0pt}%
\pgfpathmoveto{\pgfqpoint{1.564844in}{2.951278in}}%
\pgfpathcurveto{\pgfqpoint{1.575894in}{2.951278in}}{\pgfqpoint{1.586493in}{2.955668in}}{\pgfqpoint{1.594307in}{2.963482in}}%
\pgfpathcurveto{\pgfqpoint{1.602120in}{2.971295in}}{\pgfqpoint{1.606511in}{2.981894in}}{\pgfqpoint{1.606511in}{2.992945in}}%
\pgfpathcurveto{\pgfqpoint{1.606511in}{3.003995in}}{\pgfqpoint{1.602120in}{3.014594in}}{\pgfqpoint{1.594307in}{3.022407in}}%
\pgfpathcurveto{\pgfqpoint{1.586493in}{3.030221in}}{\pgfqpoint{1.575894in}{3.034611in}}{\pgfqpoint{1.564844in}{3.034611in}}%
\pgfpathcurveto{\pgfqpoint{1.553794in}{3.034611in}}{\pgfqpoint{1.543195in}{3.030221in}}{\pgfqpoint{1.535381in}{3.022407in}}%
\pgfpathcurveto{\pgfqpoint{1.527568in}{3.014594in}}{\pgfqpoint{1.523177in}{3.003995in}}{\pgfqpoint{1.523177in}{2.992945in}}%
\pgfpathcurveto{\pgfqpoint{1.523177in}{2.981894in}}{\pgfqpoint{1.527568in}{2.971295in}}{\pgfqpoint{1.535381in}{2.963482in}}%
\pgfpathcurveto{\pgfqpoint{1.543195in}{2.955668in}}{\pgfqpoint{1.553794in}{2.951278in}}{\pgfqpoint{1.564844in}{2.951278in}}%
\pgfpathclose%
\pgfusepath{stroke,fill}%
\end{pgfscope}%
\begin{pgfscope}%
\pgfpathrectangle{\pgfqpoint{0.787074in}{0.548769in}}{\pgfqpoint{5.062926in}{3.102590in}}%
\pgfusepath{clip}%
\pgfsetbuttcap%
\pgfsetroundjoin%
\definecolor{currentfill}{rgb}{0.121569,0.466667,0.705882}%
\pgfsetfillcolor{currentfill}%
\pgfsetlinewidth{1.003750pt}%
\definecolor{currentstroke}{rgb}{0.121569,0.466667,0.705882}%
\pgfsetstrokecolor{currentstroke}%
\pgfsetdash{}{0pt}%
\pgfpathmoveto{\pgfqpoint{2.130710in}{2.536597in}}%
\pgfpathcurveto{\pgfqpoint{2.141760in}{2.536597in}}{\pgfqpoint{2.152359in}{2.540987in}}{\pgfqpoint{2.160172in}{2.548801in}}%
\pgfpathcurveto{\pgfqpoint{2.167986in}{2.556614in}}{\pgfqpoint{2.172376in}{2.567213in}}{\pgfqpoint{2.172376in}{2.578263in}}%
\pgfpathcurveto{\pgfqpoint{2.172376in}{2.589314in}}{\pgfqpoint{2.167986in}{2.599913in}}{\pgfqpoint{2.160172in}{2.607726in}}%
\pgfpathcurveto{\pgfqpoint{2.152359in}{2.615540in}}{\pgfqpoint{2.141760in}{2.619930in}}{\pgfqpoint{2.130710in}{2.619930in}}%
\pgfpathcurveto{\pgfqpoint{2.119659in}{2.619930in}}{\pgfqpoint{2.109060in}{2.615540in}}{\pgfqpoint{2.101247in}{2.607726in}}%
\pgfpathcurveto{\pgfqpoint{2.093433in}{2.599913in}}{\pgfqpoint{2.089043in}{2.589314in}}{\pgfqpoint{2.089043in}{2.578263in}}%
\pgfpathcurveto{\pgfqpoint{2.089043in}{2.567213in}}{\pgfqpoint{2.093433in}{2.556614in}}{\pgfqpoint{2.101247in}{2.548801in}}%
\pgfpathcurveto{\pgfqpoint{2.109060in}{2.540987in}}{\pgfqpoint{2.119659in}{2.536597in}}{\pgfqpoint{2.130710in}{2.536597in}}%
\pgfpathclose%
\pgfusepath{stroke,fill}%
\end{pgfscope}%
\begin{pgfscope}%
\pgfpathrectangle{\pgfqpoint{0.787074in}{0.548769in}}{\pgfqpoint{5.062926in}{3.102590in}}%
\pgfusepath{clip}%
\pgfsetbuttcap%
\pgfsetroundjoin%
\definecolor{currentfill}{rgb}{1.000000,0.498039,0.054902}%
\pgfsetfillcolor{currentfill}%
\pgfsetlinewidth{1.003750pt}%
\definecolor{currentstroke}{rgb}{1.000000,0.498039,0.054902}%
\pgfsetstrokecolor{currentstroke}%
\pgfsetdash{}{0pt}%
\pgfpathmoveto{\pgfqpoint{2.488076in}{2.094657in}}%
\pgfpathcurveto{\pgfqpoint{2.499126in}{2.094657in}}{\pgfqpoint{2.509725in}{2.099047in}}{\pgfqpoint{2.517538in}{2.106861in}}%
\pgfpathcurveto{\pgfqpoint{2.525352in}{2.114675in}}{\pgfqpoint{2.529742in}{2.125274in}}{\pgfqpoint{2.529742in}{2.136324in}}%
\pgfpathcurveto{\pgfqpoint{2.529742in}{2.147374in}}{\pgfqpoint{2.525352in}{2.157973in}}{\pgfqpoint{2.517538in}{2.165787in}}%
\pgfpathcurveto{\pgfqpoint{2.509725in}{2.173600in}}{\pgfqpoint{2.499126in}{2.177990in}}{\pgfqpoint{2.488076in}{2.177990in}}%
\pgfpathcurveto{\pgfqpoint{2.477025in}{2.177990in}}{\pgfqpoint{2.466426in}{2.173600in}}{\pgfqpoint{2.458613in}{2.165787in}}%
\pgfpathcurveto{\pgfqpoint{2.450799in}{2.157973in}}{\pgfqpoint{2.446409in}{2.147374in}}{\pgfqpoint{2.446409in}{2.136324in}}%
\pgfpathcurveto{\pgfqpoint{2.446409in}{2.125274in}}{\pgfqpoint{2.450799in}{2.114675in}}{\pgfqpoint{2.458613in}{2.106861in}}%
\pgfpathcurveto{\pgfqpoint{2.466426in}{2.099047in}}{\pgfqpoint{2.477025in}{2.094657in}}{\pgfqpoint{2.488076in}{2.094657in}}%
\pgfpathclose%
\pgfusepath{stroke,fill}%
\end{pgfscope}%
\begin{pgfscope}%
\pgfpathrectangle{\pgfqpoint{0.787074in}{0.548769in}}{\pgfqpoint{5.062926in}{3.102590in}}%
\pgfusepath{clip}%
\pgfsetbuttcap%
\pgfsetroundjoin%
\definecolor{currentfill}{rgb}{1.000000,0.498039,0.054902}%
\pgfsetfillcolor{currentfill}%
\pgfsetlinewidth{1.003750pt}%
\definecolor{currentstroke}{rgb}{1.000000,0.498039,0.054902}%
\pgfsetstrokecolor{currentstroke}%
\pgfsetdash{}{0pt}%
\pgfpathmoveto{\pgfqpoint{1.879416in}{2.224129in}}%
\pgfpathcurveto{\pgfqpoint{1.890466in}{2.224129in}}{\pgfqpoint{1.901065in}{2.228519in}}{\pgfqpoint{1.908879in}{2.236332in}}%
\pgfpathcurveto{\pgfqpoint{1.916693in}{2.244146in}}{\pgfqpoint{1.921083in}{2.254745in}}{\pgfqpoint{1.921083in}{2.265795in}}%
\pgfpathcurveto{\pgfqpoint{1.921083in}{2.276845in}}{\pgfqpoint{1.916693in}{2.287444in}}{\pgfqpoint{1.908879in}{2.295258in}}%
\pgfpathcurveto{\pgfqpoint{1.901065in}{2.303072in}}{\pgfqpoint{1.890466in}{2.307462in}}{\pgfqpoint{1.879416in}{2.307462in}}%
\pgfpathcurveto{\pgfqpoint{1.868366in}{2.307462in}}{\pgfqpoint{1.857767in}{2.303072in}}{\pgfqpoint{1.849953in}{2.295258in}}%
\pgfpathcurveto{\pgfqpoint{1.842140in}{2.287444in}}{\pgfqpoint{1.837750in}{2.276845in}}{\pgfqpoint{1.837750in}{2.265795in}}%
\pgfpathcurveto{\pgfqpoint{1.837750in}{2.254745in}}{\pgfqpoint{1.842140in}{2.244146in}}{\pgfqpoint{1.849953in}{2.236332in}}%
\pgfpathcurveto{\pgfqpoint{1.857767in}{2.228519in}}{\pgfqpoint{1.868366in}{2.224129in}}{\pgfqpoint{1.879416in}{2.224129in}}%
\pgfpathclose%
\pgfusepath{stroke,fill}%
\end{pgfscope}%
\begin{pgfscope}%
\pgfpathrectangle{\pgfqpoint{0.787074in}{0.548769in}}{\pgfqpoint{5.062926in}{3.102590in}}%
\pgfusepath{clip}%
\pgfsetbuttcap%
\pgfsetroundjoin%
\definecolor{currentfill}{rgb}{1.000000,0.498039,0.054902}%
\pgfsetfillcolor{currentfill}%
\pgfsetlinewidth{1.003750pt}%
\definecolor{currentstroke}{rgb}{1.000000,0.498039,0.054902}%
\pgfsetstrokecolor{currentstroke}%
\pgfsetdash{}{0pt}%
\pgfpathmoveto{\pgfqpoint{2.216687in}{2.050679in}}%
\pgfpathcurveto{\pgfqpoint{2.227738in}{2.050679in}}{\pgfqpoint{2.238337in}{2.055070in}}{\pgfqpoint{2.246150in}{2.062883in}}%
\pgfpathcurveto{\pgfqpoint{2.253964in}{2.070697in}}{\pgfqpoint{2.258354in}{2.081296in}}{\pgfqpoint{2.258354in}{2.092346in}}%
\pgfpathcurveto{\pgfqpoint{2.258354in}{2.103396in}}{\pgfqpoint{2.253964in}{2.113995in}}{\pgfqpoint{2.246150in}{2.121809in}}%
\pgfpathcurveto{\pgfqpoint{2.238337in}{2.129622in}}{\pgfqpoint{2.227738in}{2.134013in}}{\pgfqpoint{2.216687in}{2.134013in}}%
\pgfpathcurveto{\pgfqpoint{2.205637in}{2.134013in}}{\pgfqpoint{2.195038in}{2.129622in}}{\pgfqpoint{2.187225in}{2.121809in}}%
\pgfpathcurveto{\pgfqpoint{2.179411in}{2.113995in}}{\pgfqpoint{2.175021in}{2.103396in}}{\pgfqpoint{2.175021in}{2.092346in}}%
\pgfpathcurveto{\pgfqpoint{2.175021in}{2.081296in}}{\pgfqpoint{2.179411in}{2.070697in}}{\pgfqpoint{2.187225in}{2.062883in}}%
\pgfpathcurveto{\pgfqpoint{2.195038in}{2.055070in}}{\pgfqpoint{2.205637in}{2.050679in}}{\pgfqpoint{2.216687in}{2.050679in}}%
\pgfpathclose%
\pgfusepath{stroke,fill}%
\end{pgfscope}%
\begin{pgfscope}%
\pgfpathrectangle{\pgfqpoint{0.787074in}{0.548769in}}{\pgfqpoint{5.062926in}{3.102590in}}%
\pgfusepath{clip}%
\pgfsetbuttcap%
\pgfsetroundjoin%
\definecolor{currentfill}{rgb}{0.121569,0.466667,0.705882}%
\pgfsetfillcolor{currentfill}%
\pgfsetlinewidth{1.003750pt}%
\definecolor{currentstroke}{rgb}{0.121569,0.466667,0.705882}%
\pgfsetstrokecolor{currentstroke}%
\pgfsetdash{}{0pt}%
\pgfpathmoveto{\pgfqpoint{2.150414in}{2.613502in}}%
\pgfpathcurveto{\pgfqpoint{2.161464in}{2.613502in}}{\pgfqpoint{2.172063in}{2.617892in}}{\pgfqpoint{2.179877in}{2.625706in}}%
\pgfpathcurveto{\pgfqpoint{2.187690in}{2.633519in}}{\pgfqpoint{2.192080in}{2.644118in}}{\pgfqpoint{2.192080in}{2.655169in}}%
\pgfpathcurveto{\pgfqpoint{2.192080in}{2.666219in}}{\pgfqpoint{2.187690in}{2.676818in}}{\pgfqpoint{2.179877in}{2.684631in}}%
\pgfpathcurveto{\pgfqpoint{2.172063in}{2.692445in}}{\pgfqpoint{2.161464in}{2.696835in}}{\pgfqpoint{2.150414in}{2.696835in}}%
\pgfpathcurveto{\pgfqpoint{2.139364in}{2.696835in}}{\pgfqpoint{2.128765in}{2.692445in}}{\pgfqpoint{2.120951in}{2.684631in}}%
\pgfpathcurveto{\pgfqpoint{2.113137in}{2.676818in}}{\pgfqpoint{2.108747in}{2.666219in}}{\pgfqpoint{2.108747in}{2.655169in}}%
\pgfpathcurveto{\pgfqpoint{2.108747in}{2.644118in}}{\pgfqpoint{2.113137in}{2.633519in}}{\pgfqpoint{2.120951in}{2.625706in}}%
\pgfpathcurveto{\pgfqpoint{2.128765in}{2.617892in}}{\pgfqpoint{2.139364in}{2.613502in}}{\pgfqpoint{2.150414in}{2.613502in}}%
\pgfpathclose%
\pgfusepath{stroke,fill}%
\end{pgfscope}%
\begin{pgfscope}%
\pgfpathrectangle{\pgfqpoint{0.787074in}{0.548769in}}{\pgfqpoint{5.062926in}{3.102590in}}%
\pgfusepath{clip}%
\pgfsetbuttcap%
\pgfsetroundjoin%
\definecolor{currentfill}{rgb}{0.121569,0.466667,0.705882}%
\pgfsetfillcolor{currentfill}%
\pgfsetlinewidth{1.003750pt}%
\definecolor{currentstroke}{rgb}{0.121569,0.466667,0.705882}%
\pgfsetstrokecolor{currentstroke}%
\pgfsetdash{}{0pt}%
\pgfpathmoveto{\pgfqpoint{4.083593in}{2.701999in}}%
\pgfpathcurveto{\pgfqpoint{4.094643in}{2.701999in}}{\pgfqpoint{4.105242in}{2.706389in}}{\pgfqpoint{4.113055in}{2.714203in}}%
\pgfpathcurveto{\pgfqpoint{4.120869in}{2.722016in}}{\pgfqpoint{4.125259in}{2.732616in}}{\pgfqpoint{4.125259in}{2.743666in}}%
\pgfpathcurveto{\pgfqpoint{4.125259in}{2.754716in}}{\pgfqpoint{4.120869in}{2.765315in}}{\pgfqpoint{4.113055in}{2.773128in}}%
\pgfpathcurveto{\pgfqpoint{4.105242in}{2.780942in}}{\pgfqpoint{4.094643in}{2.785332in}}{\pgfqpoint{4.083593in}{2.785332in}}%
\pgfpathcurveto{\pgfqpoint{4.072542in}{2.785332in}}{\pgfqpoint{4.061943in}{2.780942in}}{\pgfqpoint{4.054130in}{2.773128in}}%
\pgfpathcurveto{\pgfqpoint{4.046316in}{2.765315in}}{\pgfqpoint{4.041926in}{2.754716in}}{\pgfqpoint{4.041926in}{2.743666in}}%
\pgfpathcurveto{\pgfqpoint{4.041926in}{2.732616in}}{\pgfqpoint{4.046316in}{2.722016in}}{\pgfqpoint{4.054130in}{2.714203in}}%
\pgfpathcurveto{\pgfqpoint{4.061943in}{2.706389in}}{\pgfqpoint{4.072542in}{2.701999in}}{\pgfqpoint{4.083593in}{2.701999in}}%
\pgfpathclose%
\pgfusepath{stroke,fill}%
\end{pgfscope}%
\begin{pgfscope}%
\pgfpathrectangle{\pgfqpoint{0.787074in}{0.548769in}}{\pgfqpoint{5.062926in}{3.102590in}}%
\pgfusepath{clip}%
\pgfsetbuttcap%
\pgfsetroundjoin%
\definecolor{currentfill}{rgb}{0.121569,0.466667,0.705882}%
\pgfsetfillcolor{currentfill}%
\pgfsetlinewidth{1.003750pt}%
\definecolor{currentstroke}{rgb}{0.121569,0.466667,0.705882}%
\pgfsetstrokecolor{currentstroke}%
\pgfsetdash{}{0pt}%
\pgfpathmoveto{\pgfqpoint{2.009707in}{2.930649in}}%
\pgfpathcurveto{\pgfqpoint{2.020757in}{2.930649in}}{\pgfqpoint{2.031356in}{2.935039in}}{\pgfqpoint{2.039170in}{2.942853in}}%
\pgfpathcurveto{\pgfqpoint{2.046983in}{2.950666in}}{\pgfqpoint{2.051374in}{2.961265in}}{\pgfqpoint{2.051374in}{2.972316in}}%
\pgfpathcurveto{\pgfqpoint{2.051374in}{2.983366in}}{\pgfqpoint{2.046983in}{2.993965in}}{\pgfqpoint{2.039170in}{3.001778in}}%
\pgfpathcurveto{\pgfqpoint{2.031356in}{3.009592in}}{\pgfqpoint{2.020757in}{3.013982in}}{\pgfqpoint{2.009707in}{3.013982in}}%
\pgfpathcurveto{\pgfqpoint{1.998657in}{3.013982in}}{\pgfqpoint{1.988058in}{3.009592in}}{\pgfqpoint{1.980244in}{3.001778in}}%
\pgfpathcurveto{\pgfqpoint{1.972430in}{2.993965in}}{\pgfqpoint{1.968040in}{2.983366in}}{\pgfqpoint{1.968040in}{2.972316in}}%
\pgfpathcurveto{\pgfqpoint{1.968040in}{2.961265in}}{\pgfqpoint{1.972430in}{2.950666in}}{\pgfqpoint{1.980244in}{2.942853in}}%
\pgfpathcurveto{\pgfqpoint{1.988058in}{2.935039in}}{\pgfqpoint{1.998657in}{2.930649in}}{\pgfqpoint{2.009707in}{2.930649in}}%
\pgfpathclose%
\pgfusepath{stroke,fill}%
\end{pgfscope}%
\begin{pgfscope}%
\pgfpathrectangle{\pgfqpoint{0.787074in}{0.548769in}}{\pgfqpoint{5.062926in}{3.102590in}}%
\pgfusepath{clip}%
\pgfsetbuttcap%
\pgfsetroundjoin%
\definecolor{currentfill}{rgb}{0.121569,0.466667,0.705882}%
\pgfsetfillcolor{currentfill}%
\pgfsetlinewidth{1.003750pt}%
\definecolor{currentstroke}{rgb}{0.121569,0.466667,0.705882}%
\pgfsetstrokecolor{currentstroke}%
\pgfsetdash{}{0pt}%
\pgfpathmoveto{\pgfqpoint{1.150883in}{0.648134in}}%
\pgfpathcurveto{\pgfqpoint{1.161933in}{0.648134in}}{\pgfqpoint{1.172532in}{0.652525in}}{\pgfqpoint{1.180346in}{0.660338in}}%
\pgfpathcurveto{\pgfqpoint{1.188159in}{0.668152in}}{\pgfqpoint{1.192549in}{0.678751in}}{\pgfqpoint{1.192549in}{0.689801in}}%
\pgfpathcurveto{\pgfqpoint{1.192549in}{0.700851in}}{\pgfqpoint{1.188159in}{0.711450in}}{\pgfqpoint{1.180346in}{0.719264in}}%
\pgfpathcurveto{\pgfqpoint{1.172532in}{0.727077in}}{\pgfqpoint{1.161933in}{0.731468in}}{\pgfqpoint{1.150883in}{0.731468in}}%
\pgfpathcurveto{\pgfqpoint{1.139833in}{0.731468in}}{\pgfqpoint{1.129234in}{0.727077in}}{\pgfqpoint{1.121420in}{0.719264in}}%
\pgfpathcurveto{\pgfqpoint{1.113606in}{0.711450in}}{\pgfqpoint{1.109216in}{0.700851in}}{\pgfqpoint{1.109216in}{0.689801in}}%
\pgfpathcurveto{\pgfqpoint{1.109216in}{0.678751in}}{\pgfqpoint{1.113606in}{0.668152in}}{\pgfqpoint{1.121420in}{0.660338in}}%
\pgfpathcurveto{\pgfqpoint{1.129234in}{0.652525in}}{\pgfqpoint{1.139833in}{0.648134in}}{\pgfqpoint{1.150883in}{0.648134in}}%
\pgfpathclose%
\pgfusepath{stroke,fill}%
\end{pgfscope}%
\begin{pgfscope}%
\pgfpathrectangle{\pgfqpoint{0.787074in}{0.548769in}}{\pgfqpoint{5.062926in}{3.102590in}}%
\pgfusepath{clip}%
\pgfsetbuttcap%
\pgfsetroundjoin%
\definecolor{currentfill}{rgb}{1.000000,0.498039,0.054902}%
\pgfsetfillcolor{currentfill}%
\pgfsetlinewidth{1.003750pt}%
\definecolor{currentstroke}{rgb}{1.000000,0.498039,0.054902}%
\pgfsetstrokecolor{currentstroke}%
\pgfsetdash{}{0pt}%
\pgfpathmoveto{\pgfqpoint{2.331701in}{1.583814in}}%
\pgfpathcurveto{\pgfqpoint{2.342751in}{1.583814in}}{\pgfqpoint{2.353350in}{1.588205in}}{\pgfqpoint{2.361164in}{1.596018in}}%
\pgfpathcurveto{\pgfqpoint{2.368977in}{1.603832in}}{\pgfqpoint{2.373367in}{1.614431in}}{\pgfqpoint{2.373367in}{1.625481in}}%
\pgfpathcurveto{\pgfqpoint{2.373367in}{1.636531in}}{\pgfqpoint{2.368977in}{1.647130in}}{\pgfqpoint{2.361164in}{1.654944in}}%
\pgfpathcurveto{\pgfqpoint{2.353350in}{1.662758in}}{\pgfqpoint{2.342751in}{1.667148in}}{\pgfqpoint{2.331701in}{1.667148in}}%
\pgfpathcurveto{\pgfqpoint{2.320651in}{1.667148in}}{\pgfqpoint{2.310052in}{1.662758in}}{\pgfqpoint{2.302238in}{1.654944in}}%
\pgfpathcurveto{\pgfqpoint{2.294424in}{1.647130in}}{\pgfqpoint{2.290034in}{1.636531in}}{\pgfqpoint{2.290034in}{1.625481in}}%
\pgfpathcurveto{\pgfqpoint{2.290034in}{1.614431in}}{\pgfqpoint{2.294424in}{1.603832in}}{\pgfqpoint{2.302238in}{1.596018in}}%
\pgfpathcurveto{\pgfqpoint{2.310052in}{1.588205in}}{\pgfqpoint{2.320651in}{1.583814in}}{\pgfqpoint{2.331701in}{1.583814in}}%
\pgfpathclose%
\pgfusepath{stroke,fill}%
\end{pgfscope}%
\begin{pgfscope}%
\pgfpathrectangle{\pgfqpoint{0.787074in}{0.548769in}}{\pgfqpoint{5.062926in}{3.102590in}}%
\pgfusepath{clip}%
\pgfsetbuttcap%
\pgfsetroundjoin%
\definecolor{currentfill}{rgb}{1.000000,0.498039,0.054902}%
\pgfsetfillcolor{currentfill}%
\pgfsetlinewidth{1.003750pt}%
\definecolor{currentstroke}{rgb}{1.000000,0.498039,0.054902}%
\pgfsetstrokecolor{currentstroke}%
\pgfsetdash{}{0pt}%
\pgfpathmoveto{\pgfqpoint{1.966262in}{3.152602in}}%
\pgfpathcurveto{\pgfqpoint{1.977312in}{3.152602in}}{\pgfqpoint{1.987911in}{3.156992in}}{\pgfqpoint{1.995725in}{3.164806in}}%
\pgfpathcurveto{\pgfqpoint{2.003539in}{3.172619in}}{\pgfqpoint{2.007929in}{3.183218in}}{\pgfqpoint{2.007929in}{3.194268in}}%
\pgfpathcurveto{\pgfqpoint{2.007929in}{3.205319in}}{\pgfqpoint{2.003539in}{3.215918in}}{\pgfqpoint{1.995725in}{3.223731in}}%
\pgfpathcurveto{\pgfqpoint{1.987911in}{3.231545in}}{\pgfqpoint{1.977312in}{3.235935in}}{\pgfqpoint{1.966262in}{3.235935in}}%
\pgfpathcurveto{\pgfqpoint{1.955212in}{3.235935in}}{\pgfqpoint{1.944613in}{3.231545in}}{\pgfqpoint{1.936799in}{3.223731in}}%
\pgfpathcurveto{\pgfqpoint{1.928986in}{3.215918in}}{\pgfqpoint{1.924596in}{3.205319in}}{\pgfqpoint{1.924596in}{3.194268in}}%
\pgfpathcurveto{\pgfqpoint{1.924596in}{3.183218in}}{\pgfqpoint{1.928986in}{3.172619in}}{\pgfqpoint{1.936799in}{3.164806in}}%
\pgfpathcurveto{\pgfqpoint{1.944613in}{3.156992in}}{\pgfqpoint{1.955212in}{3.152602in}}{\pgfqpoint{1.966262in}{3.152602in}}%
\pgfpathclose%
\pgfusepath{stroke,fill}%
\end{pgfscope}%
\begin{pgfscope}%
\pgfpathrectangle{\pgfqpoint{0.787074in}{0.548769in}}{\pgfqpoint{5.062926in}{3.102590in}}%
\pgfusepath{clip}%
\pgfsetbuttcap%
\pgfsetroundjoin%
\definecolor{currentfill}{rgb}{1.000000,0.498039,0.054902}%
\pgfsetfillcolor{currentfill}%
\pgfsetlinewidth{1.003750pt}%
\definecolor{currentstroke}{rgb}{1.000000,0.498039,0.054902}%
\pgfsetstrokecolor{currentstroke}%
\pgfsetdash{}{0pt}%
\pgfpathmoveto{\pgfqpoint{1.493709in}{2.882713in}}%
\pgfpathcurveto{\pgfqpoint{1.504759in}{2.882713in}}{\pgfqpoint{1.515358in}{2.887104in}}{\pgfqpoint{1.523172in}{2.894917in}}%
\pgfpathcurveto{\pgfqpoint{1.530986in}{2.902731in}}{\pgfqpoint{1.535376in}{2.913330in}}{\pgfqpoint{1.535376in}{2.924380in}}%
\pgfpathcurveto{\pgfqpoint{1.535376in}{2.935430in}}{\pgfqpoint{1.530986in}{2.946029in}}{\pgfqpoint{1.523172in}{2.953843in}}%
\pgfpathcurveto{\pgfqpoint{1.515358in}{2.961656in}}{\pgfqpoint{1.504759in}{2.966047in}}{\pgfqpoint{1.493709in}{2.966047in}}%
\pgfpathcurveto{\pgfqpoint{1.482659in}{2.966047in}}{\pgfqpoint{1.472060in}{2.961656in}}{\pgfqpoint{1.464246in}{2.953843in}}%
\pgfpathcurveto{\pgfqpoint{1.456433in}{2.946029in}}{\pgfqpoint{1.452043in}{2.935430in}}{\pgfqpoint{1.452043in}{2.924380in}}%
\pgfpathcurveto{\pgfqpoint{1.452043in}{2.913330in}}{\pgfqpoint{1.456433in}{2.902731in}}{\pgfqpoint{1.464246in}{2.894917in}}%
\pgfpathcurveto{\pgfqpoint{1.472060in}{2.887104in}}{\pgfqpoint{1.482659in}{2.882713in}}{\pgfqpoint{1.493709in}{2.882713in}}%
\pgfpathclose%
\pgfusepath{stroke,fill}%
\end{pgfscope}%
\begin{pgfscope}%
\pgfpathrectangle{\pgfqpoint{0.787074in}{0.548769in}}{\pgfqpoint{5.062926in}{3.102590in}}%
\pgfusepath{clip}%
\pgfsetbuttcap%
\pgfsetroundjoin%
\definecolor{currentfill}{rgb}{0.121569,0.466667,0.705882}%
\pgfsetfillcolor{currentfill}%
\pgfsetlinewidth{1.003750pt}%
\definecolor{currentstroke}{rgb}{0.121569,0.466667,0.705882}%
\pgfsetstrokecolor{currentstroke}%
\pgfsetdash{}{0pt}%
\pgfpathmoveto{\pgfqpoint{1.283040in}{0.749337in}}%
\pgfpathcurveto{\pgfqpoint{1.294090in}{0.749337in}}{\pgfqpoint{1.304689in}{0.753727in}}{\pgfqpoint{1.312502in}{0.761541in}}%
\pgfpathcurveto{\pgfqpoint{1.320316in}{0.769355in}}{\pgfqpoint{1.324706in}{0.779954in}}{\pgfqpoint{1.324706in}{0.791004in}}%
\pgfpathcurveto{\pgfqpoint{1.324706in}{0.802054in}}{\pgfqpoint{1.320316in}{0.812653in}}{\pgfqpoint{1.312502in}{0.820467in}}%
\pgfpathcurveto{\pgfqpoint{1.304689in}{0.828280in}}{\pgfqpoint{1.294090in}{0.832670in}}{\pgfqpoint{1.283040in}{0.832670in}}%
\pgfpathcurveto{\pgfqpoint{1.271989in}{0.832670in}}{\pgfqpoint{1.261390in}{0.828280in}}{\pgfqpoint{1.253577in}{0.820467in}}%
\pgfpathcurveto{\pgfqpoint{1.245763in}{0.812653in}}{\pgfqpoint{1.241373in}{0.802054in}}{\pgfqpoint{1.241373in}{0.791004in}}%
\pgfpathcurveto{\pgfqpoint{1.241373in}{0.779954in}}{\pgfqpoint{1.245763in}{0.769355in}}{\pgfqpoint{1.253577in}{0.761541in}}%
\pgfpathcurveto{\pgfqpoint{1.261390in}{0.753727in}}{\pgfqpoint{1.271989in}{0.749337in}}{\pgfqpoint{1.283040in}{0.749337in}}%
\pgfpathclose%
\pgfusepath{stroke,fill}%
\end{pgfscope}%
\begin{pgfscope}%
\pgfpathrectangle{\pgfqpoint{0.787074in}{0.548769in}}{\pgfqpoint{5.062926in}{3.102590in}}%
\pgfusepath{clip}%
\pgfsetbuttcap%
\pgfsetroundjoin%
\definecolor{currentfill}{rgb}{1.000000,0.498039,0.054902}%
\pgfsetfillcolor{currentfill}%
\pgfsetlinewidth{1.003750pt}%
\definecolor{currentstroke}{rgb}{1.000000,0.498039,0.054902}%
\pgfsetstrokecolor{currentstroke}%
\pgfsetdash{}{0pt}%
\pgfpathmoveto{\pgfqpoint{2.192947in}{1.712229in}}%
\pgfpathcurveto{\pgfqpoint{2.203997in}{1.712229in}}{\pgfqpoint{2.214596in}{1.716619in}}{\pgfqpoint{2.222410in}{1.724432in}}%
\pgfpathcurveto{\pgfqpoint{2.230223in}{1.732246in}}{\pgfqpoint{2.234614in}{1.742845in}}{\pgfqpoint{2.234614in}{1.753895in}}%
\pgfpathcurveto{\pgfqpoint{2.234614in}{1.764945in}}{\pgfqpoint{2.230223in}{1.775544in}}{\pgfqpoint{2.222410in}{1.783358in}}%
\pgfpathcurveto{\pgfqpoint{2.214596in}{1.791172in}}{\pgfqpoint{2.203997in}{1.795562in}}{\pgfqpoint{2.192947in}{1.795562in}}%
\pgfpathcurveto{\pgfqpoint{2.181897in}{1.795562in}}{\pgfqpoint{2.171298in}{1.791172in}}{\pgfqpoint{2.163484in}{1.783358in}}%
\pgfpathcurveto{\pgfqpoint{2.155671in}{1.775544in}}{\pgfqpoint{2.151280in}{1.764945in}}{\pgfqpoint{2.151280in}{1.753895in}}%
\pgfpathcurveto{\pgfqpoint{2.151280in}{1.742845in}}{\pgfqpoint{2.155671in}{1.732246in}}{\pgfqpoint{2.163484in}{1.724432in}}%
\pgfpathcurveto{\pgfqpoint{2.171298in}{1.716619in}}{\pgfqpoint{2.181897in}{1.712229in}}{\pgfqpoint{2.192947in}{1.712229in}}%
\pgfpathclose%
\pgfusepath{stroke,fill}%
\end{pgfscope}%
\begin{pgfscope}%
\pgfpathrectangle{\pgfqpoint{0.787074in}{0.548769in}}{\pgfqpoint{5.062926in}{3.102590in}}%
\pgfusepath{clip}%
\pgfsetbuttcap%
\pgfsetroundjoin%
\definecolor{currentfill}{rgb}{0.121569,0.466667,0.705882}%
\pgfsetfillcolor{currentfill}%
\pgfsetlinewidth{1.003750pt}%
\definecolor{currentstroke}{rgb}{0.121569,0.466667,0.705882}%
\pgfsetstrokecolor{currentstroke}%
\pgfsetdash{}{0pt}%
\pgfpathmoveto{\pgfqpoint{2.075850in}{2.635102in}}%
\pgfpathcurveto{\pgfqpoint{2.086900in}{2.635102in}}{\pgfqpoint{2.097500in}{2.639493in}}{\pgfqpoint{2.105313in}{2.647306in}}%
\pgfpathcurveto{\pgfqpoint{2.113127in}{2.655120in}}{\pgfqpoint{2.117517in}{2.665719in}}{\pgfqpoint{2.117517in}{2.676769in}}%
\pgfpathcurveto{\pgfqpoint{2.117517in}{2.687819in}}{\pgfqpoint{2.113127in}{2.698418in}}{\pgfqpoint{2.105313in}{2.706232in}}%
\pgfpathcurveto{\pgfqpoint{2.097500in}{2.714045in}}{\pgfqpoint{2.086900in}{2.718436in}}{\pgfqpoint{2.075850in}{2.718436in}}%
\pgfpathcurveto{\pgfqpoint{2.064800in}{2.718436in}}{\pgfqpoint{2.054201in}{2.714045in}}{\pgfqpoint{2.046388in}{2.706232in}}%
\pgfpathcurveto{\pgfqpoint{2.038574in}{2.698418in}}{\pgfqpoint{2.034184in}{2.687819in}}{\pgfqpoint{2.034184in}{2.676769in}}%
\pgfpathcurveto{\pgfqpoint{2.034184in}{2.665719in}}{\pgfqpoint{2.038574in}{2.655120in}}{\pgfqpoint{2.046388in}{2.647306in}}%
\pgfpathcurveto{\pgfqpoint{2.054201in}{2.639493in}}{\pgfqpoint{2.064800in}{2.635102in}}{\pgfqpoint{2.075850in}{2.635102in}}%
\pgfpathclose%
\pgfusepath{stroke,fill}%
\end{pgfscope}%
\begin{pgfscope}%
\pgfpathrectangle{\pgfqpoint{0.787074in}{0.548769in}}{\pgfqpoint{5.062926in}{3.102590in}}%
\pgfusepath{clip}%
\pgfsetbuttcap%
\pgfsetroundjoin%
\definecolor{currentfill}{rgb}{1.000000,0.498039,0.054902}%
\pgfsetfillcolor{currentfill}%
\pgfsetlinewidth{1.003750pt}%
\definecolor{currentstroke}{rgb}{1.000000,0.498039,0.054902}%
\pgfsetstrokecolor{currentstroke}%
\pgfsetdash{}{0pt}%
\pgfpathmoveto{\pgfqpoint{2.884503in}{2.957127in}}%
\pgfpathcurveto{\pgfqpoint{2.895553in}{2.957127in}}{\pgfqpoint{2.906152in}{2.961517in}}{\pgfqpoint{2.913965in}{2.969331in}}%
\pgfpathcurveto{\pgfqpoint{2.921779in}{2.977144in}}{\pgfqpoint{2.926169in}{2.987743in}}{\pgfqpoint{2.926169in}{2.998793in}}%
\pgfpathcurveto{\pgfqpoint{2.926169in}{3.009844in}}{\pgfqpoint{2.921779in}{3.020443in}}{\pgfqpoint{2.913965in}{3.028256in}}%
\pgfpathcurveto{\pgfqpoint{2.906152in}{3.036070in}}{\pgfqpoint{2.895553in}{3.040460in}}{\pgfqpoint{2.884503in}{3.040460in}}%
\pgfpathcurveto{\pgfqpoint{2.873452in}{3.040460in}}{\pgfqpoint{2.862853in}{3.036070in}}{\pgfqpoint{2.855040in}{3.028256in}}%
\pgfpathcurveto{\pgfqpoint{2.847226in}{3.020443in}}{\pgfqpoint{2.842836in}{3.009844in}}{\pgfqpoint{2.842836in}{2.998793in}}%
\pgfpathcurveto{\pgfqpoint{2.842836in}{2.987743in}}{\pgfqpoint{2.847226in}{2.977144in}}{\pgfqpoint{2.855040in}{2.969331in}}%
\pgfpathcurveto{\pgfqpoint{2.862853in}{2.961517in}}{\pgfqpoint{2.873452in}{2.957127in}}{\pgfqpoint{2.884503in}{2.957127in}}%
\pgfpathclose%
\pgfusepath{stroke,fill}%
\end{pgfscope}%
\begin{pgfscope}%
\pgfpathrectangle{\pgfqpoint{0.787074in}{0.548769in}}{\pgfqpoint{5.062926in}{3.102590in}}%
\pgfusepath{clip}%
\pgfsetbuttcap%
\pgfsetroundjoin%
\definecolor{currentfill}{rgb}{0.121569,0.466667,0.705882}%
\pgfsetfillcolor{currentfill}%
\pgfsetlinewidth{1.003750pt}%
\definecolor{currentstroke}{rgb}{0.121569,0.466667,0.705882}%
\pgfsetstrokecolor{currentstroke}%
\pgfsetdash{}{0pt}%
\pgfpathmoveto{\pgfqpoint{2.093384in}{1.266126in}}%
\pgfpathcurveto{\pgfqpoint{2.104435in}{1.266126in}}{\pgfqpoint{2.115034in}{1.270516in}}{\pgfqpoint{2.122847in}{1.278330in}}%
\pgfpathcurveto{\pgfqpoint{2.130661in}{1.286143in}}{\pgfqpoint{2.135051in}{1.296742in}}{\pgfqpoint{2.135051in}{1.307792in}}%
\pgfpathcurveto{\pgfqpoint{2.135051in}{1.318842in}}{\pgfqpoint{2.130661in}{1.329441in}}{\pgfqpoint{2.122847in}{1.337255in}}%
\pgfpathcurveto{\pgfqpoint{2.115034in}{1.345069in}}{\pgfqpoint{2.104435in}{1.349459in}}{\pgfqpoint{2.093384in}{1.349459in}}%
\pgfpathcurveto{\pgfqpoint{2.082334in}{1.349459in}}{\pgfqpoint{2.071735in}{1.345069in}}{\pgfqpoint{2.063922in}{1.337255in}}%
\pgfpathcurveto{\pgfqpoint{2.056108in}{1.329441in}}{\pgfqpoint{2.051718in}{1.318842in}}{\pgfqpoint{2.051718in}{1.307792in}}%
\pgfpathcurveto{\pgfqpoint{2.051718in}{1.296742in}}{\pgfqpoint{2.056108in}{1.286143in}}{\pgfqpoint{2.063922in}{1.278330in}}%
\pgfpathcurveto{\pgfqpoint{2.071735in}{1.270516in}}{\pgfqpoint{2.082334in}{1.266126in}}{\pgfqpoint{2.093384in}{1.266126in}}%
\pgfpathclose%
\pgfusepath{stroke,fill}%
\end{pgfscope}%
\begin{pgfscope}%
\pgfpathrectangle{\pgfqpoint{0.787074in}{0.548769in}}{\pgfqpoint{5.062926in}{3.102590in}}%
\pgfusepath{clip}%
\pgfsetbuttcap%
\pgfsetroundjoin%
\definecolor{currentfill}{rgb}{1.000000,0.498039,0.054902}%
\pgfsetfillcolor{currentfill}%
\pgfsetlinewidth{1.003750pt}%
\definecolor{currentstroke}{rgb}{1.000000,0.498039,0.054902}%
\pgfsetstrokecolor{currentstroke}%
\pgfsetdash{}{0pt}%
\pgfpathmoveto{\pgfqpoint{2.100459in}{2.376418in}}%
\pgfpathcurveto{\pgfqpoint{2.111509in}{2.376418in}}{\pgfqpoint{2.122108in}{2.380808in}}{\pgfqpoint{2.129922in}{2.388622in}}%
\pgfpathcurveto{\pgfqpoint{2.137735in}{2.396436in}}{\pgfqpoint{2.142126in}{2.407035in}}{\pgfqpoint{2.142126in}{2.418085in}}%
\pgfpathcurveto{\pgfqpoint{2.142126in}{2.429135in}}{\pgfqpoint{2.137735in}{2.439734in}}{\pgfqpoint{2.129922in}{2.447548in}}%
\pgfpathcurveto{\pgfqpoint{2.122108in}{2.455361in}}{\pgfqpoint{2.111509in}{2.459751in}}{\pgfqpoint{2.100459in}{2.459751in}}%
\pgfpathcurveto{\pgfqpoint{2.089409in}{2.459751in}}{\pgfqpoint{2.078810in}{2.455361in}}{\pgfqpoint{2.070996in}{2.447548in}}%
\pgfpathcurveto{\pgfqpoint{2.063182in}{2.439734in}}{\pgfqpoint{2.058792in}{2.429135in}}{\pgfqpoint{2.058792in}{2.418085in}}%
\pgfpathcurveto{\pgfqpoint{2.058792in}{2.407035in}}{\pgfqpoint{2.063182in}{2.396436in}}{\pgfqpoint{2.070996in}{2.388622in}}%
\pgfpathcurveto{\pgfqpoint{2.078810in}{2.380808in}}{\pgfqpoint{2.089409in}{2.376418in}}{\pgfqpoint{2.100459in}{2.376418in}}%
\pgfpathclose%
\pgfusepath{stroke,fill}%
\end{pgfscope}%
\begin{pgfscope}%
\pgfpathrectangle{\pgfqpoint{0.787074in}{0.548769in}}{\pgfqpoint{5.062926in}{3.102590in}}%
\pgfusepath{clip}%
\pgfsetbuttcap%
\pgfsetroundjoin%
\definecolor{currentfill}{rgb}{1.000000,0.498039,0.054902}%
\pgfsetfillcolor{currentfill}%
\pgfsetlinewidth{1.003750pt}%
\definecolor{currentstroke}{rgb}{1.000000,0.498039,0.054902}%
\pgfsetstrokecolor{currentstroke}%
\pgfsetdash{}{0pt}%
\pgfpathmoveto{\pgfqpoint{1.616448in}{3.006227in}}%
\pgfpathcurveto{\pgfqpoint{1.627498in}{3.006227in}}{\pgfqpoint{1.638097in}{3.010617in}}{\pgfqpoint{1.645911in}{3.018431in}}%
\pgfpathcurveto{\pgfqpoint{1.653724in}{3.026244in}}{\pgfqpoint{1.658115in}{3.036844in}}{\pgfqpoint{1.658115in}{3.047894in}}%
\pgfpathcurveto{\pgfqpoint{1.658115in}{3.058944in}}{\pgfqpoint{1.653724in}{3.069543in}}{\pgfqpoint{1.645911in}{3.077356in}}%
\pgfpathcurveto{\pgfqpoint{1.638097in}{3.085170in}}{\pgfqpoint{1.627498in}{3.089560in}}{\pgfqpoint{1.616448in}{3.089560in}}%
\pgfpathcurveto{\pgfqpoint{1.605398in}{3.089560in}}{\pgfqpoint{1.594799in}{3.085170in}}{\pgfqpoint{1.586985in}{3.077356in}}%
\pgfpathcurveto{\pgfqpoint{1.579172in}{3.069543in}}{\pgfqpoint{1.574781in}{3.058944in}}{\pgfqpoint{1.574781in}{3.047894in}}%
\pgfpathcurveto{\pgfqpoint{1.574781in}{3.036844in}}{\pgfqpoint{1.579172in}{3.026244in}}{\pgfqpoint{1.586985in}{3.018431in}}%
\pgfpathcurveto{\pgfqpoint{1.594799in}{3.010617in}}{\pgfqpoint{1.605398in}{3.006227in}}{\pgfqpoint{1.616448in}{3.006227in}}%
\pgfpathclose%
\pgfusepath{stroke,fill}%
\end{pgfscope}%
\begin{pgfscope}%
\pgfpathrectangle{\pgfqpoint{0.787074in}{0.548769in}}{\pgfqpoint{5.062926in}{3.102590in}}%
\pgfusepath{clip}%
\pgfsetbuttcap%
\pgfsetroundjoin%
\definecolor{currentfill}{rgb}{1.000000,0.498039,0.054902}%
\pgfsetfillcolor{currentfill}%
\pgfsetlinewidth{1.003750pt}%
\definecolor{currentstroke}{rgb}{1.000000,0.498039,0.054902}%
\pgfsetstrokecolor{currentstroke}%
\pgfsetdash{}{0pt}%
\pgfpathmoveto{\pgfqpoint{1.982147in}{2.381259in}}%
\pgfpathcurveto{\pgfqpoint{1.993197in}{2.381259in}}{\pgfqpoint{2.003796in}{2.385649in}}{\pgfqpoint{2.011610in}{2.393463in}}%
\pgfpathcurveto{\pgfqpoint{2.019423in}{2.401277in}}{\pgfqpoint{2.023814in}{2.411876in}}{\pgfqpoint{2.023814in}{2.422926in}}%
\pgfpathcurveto{\pgfqpoint{2.023814in}{2.433976in}}{\pgfqpoint{2.019423in}{2.444575in}}{\pgfqpoint{2.011610in}{2.452388in}}%
\pgfpathcurveto{\pgfqpoint{2.003796in}{2.460202in}}{\pgfqpoint{1.993197in}{2.464592in}}{\pgfqpoint{1.982147in}{2.464592in}}%
\pgfpathcurveto{\pgfqpoint{1.971097in}{2.464592in}}{\pgfqpoint{1.960498in}{2.460202in}}{\pgfqpoint{1.952684in}{2.452388in}}%
\pgfpathcurveto{\pgfqpoint{1.944871in}{2.444575in}}{\pgfqpoint{1.940480in}{2.433976in}}{\pgfqpoint{1.940480in}{2.422926in}}%
\pgfpathcurveto{\pgfqpoint{1.940480in}{2.411876in}}{\pgfqpoint{1.944871in}{2.401277in}}{\pgfqpoint{1.952684in}{2.393463in}}%
\pgfpathcurveto{\pgfqpoint{1.960498in}{2.385649in}}{\pgfqpoint{1.971097in}{2.381259in}}{\pgfqpoint{1.982147in}{2.381259in}}%
\pgfpathclose%
\pgfusepath{stroke,fill}%
\end{pgfscope}%
\begin{pgfscope}%
\pgfpathrectangle{\pgfqpoint{0.787074in}{0.548769in}}{\pgfqpoint{5.062926in}{3.102590in}}%
\pgfusepath{clip}%
\pgfsetbuttcap%
\pgfsetroundjoin%
\definecolor{currentfill}{rgb}{1.000000,0.498039,0.054902}%
\pgfsetfillcolor{currentfill}%
\pgfsetlinewidth{1.003750pt}%
\definecolor{currentstroke}{rgb}{1.000000,0.498039,0.054902}%
\pgfsetstrokecolor{currentstroke}%
\pgfsetdash{}{0pt}%
\pgfpathmoveto{\pgfqpoint{2.474144in}{1.619069in}}%
\pgfpathcurveto{\pgfqpoint{2.485194in}{1.619069in}}{\pgfqpoint{2.495793in}{1.623460in}}{\pgfqpoint{2.503607in}{1.631273in}}%
\pgfpathcurveto{\pgfqpoint{2.511420in}{1.639087in}}{\pgfqpoint{2.515810in}{1.649686in}}{\pgfqpoint{2.515810in}{1.660736in}}%
\pgfpathcurveto{\pgfqpoint{2.515810in}{1.671786in}}{\pgfqpoint{2.511420in}{1.682385in}}{\pgfqpoint{2.503607in}{1.690199in}}%
\pgfpathcurveto{\pgfqpoint{2.495793in}{1.698012in}}{\pgfqpoint{2.485194in}{1.702403in}}{\pgfqpoint{2.474144in}{1.702403in}}%
\pgfpathcurveto{\pgfqpoint{2.463094in}{1.702403in}}{\pgfqpoint{2.452495in}{1.698012in}}{\pgfqpoint{2.444681in}{1.690199in}}%
\pgfpathcurveto{\pgfqpoint{2.436867in}{1.682385in}}{\pgfqpoint{2.432477in}{1.671786in}}{\pgfqpoint{2.432477in}{1.660736in}}%
\pgfpathcurveto{\pgfqpoint{2.432477in}{1.649686in}}{\pgfqpoint{2.436867in}{1.639087in}}{\pgfqpoint{2.444681in}{1.631273in}}%
\pgfpathcurveto{\pgfqpoint{2.452495in}{1.623460in}}{\pgfqpoint{2.463094in}{1.619069in}}{\pgfqpoint{2.474144in}{1.619069in}}%
\pgfpathclose%
\pgfusepath{stroke,fill}%
\end{pgfscope}%
\begin{pgfscope}%
\pgfpathrectangle{\pgfqpoint{0.787074in}{0.548769in}}{\pgfqpoint{5.062926in}{3.102590in}}%
\pgfusepath{clip}%
\pgfsetbuttcap%
\pgfsetroundjoin%
\definecolor{currentfill}{rgb}{1.000000,0.498039,0.054902}%
\pgfsetfillcolor{currentfill}%
\pgfsetlinewidth{1.003750pt}%
\definecolor{currentstroke}{rgb}{1.000000,0.498039,0.054902}%
\pgfsetstrokecolor{currentstroke}%
\pgfsetdash{}{0pt}%
\pgfpathmoveto{\pgfqpoint{1.797041in}{2.847775in}}%
\pgfpathcurveto{\pgfqpoint{1.808091in}{2.847775in}}{\pgfqpoint{1.818690in}{2.852165in}}{\pgfqpoint{1.826503in}{2.859979in}}%
\pgfpathcurveto{\pgfqpoint{1.834317in}{2.867793in}}{\pgfqpoint{1.838707in}{2.878392in}}{\pgfqpoint{1.838707in}{2.889442in}}%
\pgfpathcurveto{\pgfqpoint{1.838707in}{2.900492in}}{\pgfqpoint{1.834317in}{2.911091in}}{\pgfqpoint{1.826503in}{2.918904in}}%
\pgfpathcurveto{\pgfqpoint{1.818690in}{2.926718in}}{\pgfqpoint{1.808091in}{2.931108in}}{\pgfqpoint{1.797041in}{2.931108in}}%
\pgfpathcurveto{\pgfqpoint{1.785991in}{2.931108in}}{\pgfqpoint{1.775392in}{2.926718in}}{\pgfqpoint{1.767578in}{2.918904in}}%
\pgfpathcurveto{\pgfqpoint{1.759764in}{2.911091in}}{\pgfqpoint{1.755374in}{2.900492in}}{\pgfqpoint{1.755374in}{2.889442in}}%
\pgfpathcurveto{\pgfqpoint{1.755374in}{2.878392in}}{\pgfqpoint{1.759764in}{2.867793in}}{\pgfqpoint{1.767578in}{2.859979in}}%
\pgfpathcurveto{\pgfqpoint{1.775392in}{2.852165in}}{\pgfqpoint{1.785991in}{2.847775in}}{\pgfqpoint{1.797041in}{2.847775in}}%
\pgfpathclose%
\pgfusepath{stroke,fill}%
\end{pgfscope}%
\begin{pgfscope}%
\pgfpathrectangle{\pgfqpoint{0.787074in}{0.548769in}}{\pgfqpoint{5.062926in}{3.102590in}}%
\pgfusepath{clip}%
\pgfsetbuttcap%
\pgfsetroundjoin%
\definecolor{currentfill}{rgb}{1.000000,0.498039,0.054902}%
\pgfsetfillcolor{currentfill}%
\pgfsetlinewidth{1.003750pt}%
\definecolor{currentstroke}{rgb}{1.000000,0.498039,0.054902}%
\pgfsetstrokecolor{currentstroke}%
\pgfsetdash{}{0pt}%
\pgfpathmoveto{\pgfqpoint{1.794957in}{2.241897in}}%
\pgfpathcurveto{\pgfqpoint{1.806008in}{2.241897in}}{\pgfqpoint{1.816607in}{2.246287in}}{\pgfqpoint{1.824420in}{2.254101in}}%
\pgfpathcurveto{\pgfqpoint{1.832234in}{2.261914in}}{\pgfqpoint{1.836624in}{2.272513in}}{\pgfqpoint{1.836624in}{2.283563in}}%
\pgfpathcurveto{\pgfqpoint{1.836624in}{2.294613in}}{\pgfqpoint{1.832234in}{2.305212in}}{\pgfqpoint{1.824420in}{2.313026in}}%
\pgfpathcurveto{\pgfqpoint{1.816607in}{2.320840in}}{\pgfqpoint{1.806008in}{2.325230in}}{\pgfqpoint{1.794957in}{2.325230in}}%
\pgfpathcurveto{\pgfqpoint{1.783907in}{2.325230in}}{\pgfqpoint{1.773308in}{2.320840in}}{\pgfqpoint{1.765495in}{2.313026in}}%
\pgfpathcurveto{\pgfqpoint{1.757681in}{2.305212in}}{\pgfqpoint{1.753291in}{2.294613in}}{\pgfqpoint{1.753291in}{2.283563in}}%
\pgfpathcurveto{\pgfqpoint{1.753291in}{2.272513in}}{\pgfqpoint{1.757681in}{2.261914in}}{\pgfqpoint{1.765495in}{2.254101in}}%
\pgfpathcurveto{\pgfqpoint{1.773308in}{2.246287in}}{\pgfqpoint{1.783907in}{2.241897in}}{\pgfqpoint{1.794957in}{2.241897in}}%
\pgfpathclose%
\pgfusepath{stroke,fill}%
\end{pgfscope}%
\begin{pgfscope}%
\pgfpathrectangle{\pgfqpoint{0.787074in}{0.548769in}}{\pgfqpoint{5.062926in}{3.102590in}}%
\pgfusepath{clip}%
\pgfsetbuttcap%
\pgfsetroundjoin%
\definecolor{currentfill}{rgb}{1.000000,0.498039,0.054902}%
\pgfsetfillcolor{currentfill}%
\pgfsetlinewidth{1.003750pt}%
\definecolor{currentstroke}{rgb}{1.000000,0.498039,0.054902}%
\pgfsetstrokecolor{currentstroke}%
\pgfsetdash{}{0pt}%
\pgfpathmoveto{\pgfqpoint{2.240341in}{1.903634in}}%
\pgfpathcurveto{\pgfqpoint{2.251391in}{1.903634in}}{\pgfqpoint{2.261990in}{1.908024in}}{\pgfqpoint{2.269804in}{1.915838in}}%
\pgfpathcurveto{\pgfqpoint{2.277618in}{1.923652in}}{\pgfqpoint{2.282008in}{1.934251in}}{\pgfqpoint{2.282008in}{1.945301in}}%
\pgfpathcurveto{\pgfqpoint{2.282008in}{1.956351in}}{\pgfqpoint{2.277618in}{1.966950in}}{\pgfqpoint{2.269804in}{1.974764in}}%
\pgfpathcurveto{\pgfqpoint{2.261990in}{1.982577in}}{\pgfqpoint{2.251391in}{1.986968in}}{\pgfqpoint{2.240341in}{1.986968in}}%
\pgfpathcurveto{\pgfqpoint{2.229291in}{1.986968in}}{\pgfqpoint{2.218692in}{1.982577in}}{\pgfqpoint{2.210878in}{1.974764in}}%
\pgfpathcurveto{\pgfqpoint{2.203065in}{1.966950in}}{\pgfqpoint{2.198674in}{1.956351in}}{\pgfqpoint{2.198674in}{1.945301in}}%
\pgfpathcurveto{\pgfqpoint{2.198674in}{1.934251in}}{\pgfqpoint{2.203065in}{1.923652in}}{\pgfqpoint{2.210878in}{1.915838in}}%
\pgfpathcurveto{\pgfqpoint{2.218692in}{1.908024in}}{\pgfqpoint{2.229291in}{1.903634in}}{\pgfqpoint{2.240341in}{1.903634in}}%
\pgfpathclose%
\pgfusepath{stroke,fill}%
\end{pgfscope}%
\begin{pgfscope}%
\pgfpathrectangle{\pgfqpoint{0.787074in}{0.548769in}}{\pgfqpoint{5.062926in}{3.102590in}}%
\pgfusepath{clip}%
\pgfsetbuttcap%
\pgfsetroundjoin%
\definecolor{currentfill}{rgb}{1.000000,0.498039,0.054902}%
\pgfsetfillcolor{currentfill}%
\pgfsetlinewidth{1.003750pt}%
\definecolor{currentstroke}{rgb}{1.000000,0.498039,0.054902}%
\pgfsetstrokecolor{currentstroke}%
\pgfsetdash{}{0pt}%
\pgfpathmoveto{\pgfqpoint{1.858887in}{2.451696in}}%
\pgfpathcurveto{\pgfqpoint{1.869938in}{2.451696in}}{\pgfqpoint{1.880537in}{2.456086in}}{\pgfqpoint{1.888350in}{2.463900in}}%
\pgfpathcurveto{\pgfqpoint{1.896164in}{2.471713in}}{\pgfqpoint{1.900554in}{2.482312in}}{\pgfqpoint{1.900554in}{2.493362in}}%
\pgfpathcurveto{\pgfqpoint{1.900554in}{2.504412in}}{\pgfqpoint{1.896164in}{2.515012in}}{\pgfqpoint{1.888350in}{2.522825in}}%
\pgfpathcurveto{\pgfqpoint{1.880537in}{2.530639in}}{\pgfqpoint{1.869938in}{2.535029in}}{\pgfqpoint{1.858887in}{2.535029in}}%
\pgfpathcurveto{\pgfqpoint{1.847837in}{2.535029in}}{\pgfqpoint{1.837238in}{2.530639in}}{\pgfqpoint{1.829425in}{2.522825in}}%
\pgfpathcurveto{\pgfqpoint{1.821611in}{2.515012in}}{\pgfqpoint{1.817221in}{2.504412in}}{\pgfqpoint{1.817221in}{2.493362in}}%
\pgfpathcurveto{\pgfqpoint{1.817221in}{2.482312in}}{\pgfqpoint{1.821611in}{2.471713in}}{\pgfqpoint{1.829425in}{2.463900in}}%
\pgfpathcurveto{\pgfqpoint{1.837238in}{2.456086in}}{\pgfqpoint{1.847837in}{2.451696in}}{\pgfqpoint{1.858887in}{2.451696in}}%
\pgfpathclose%
\pgfusepath{stroke,fill}%
\end{pgfscope}%
\begin{pgfscope}%
\pgfpathrectangle{\pgfqpoint{0.787074in}{0.548769in}}{\pgfqpoint{5.062926in}{3.102590in}}%
\pgfusepath{clip}%
\pgfsetbuttcap%
\pgfsetroundjoin%
\definecolor{currentfill}{rgb}{0.839216,0.152941,0.156863}%
\pgfsetfillcolor{currentfill}%
\pgfsetlinewidth{1.003750pt}%
\definecolor{currentstroke}{rgb}{0.839216,0.152941,0.156863}%
\pgfsetstrokecolor{currentstroke}%
\pgfsetdash{}{0pt}%
\pgfpathmoveto{\pgfqpoint{1.977590in}{3.176910in}}%
\pgfpathcurveto{\pgfqpoint{1.988640in}{3.176910in}}{\pgfqpoint{1.999239in}{3.181301in}}{\pgfqpoint{2.007053in}{3.189114in}}%
\pgfpathcurveto{\pgfqpoint{2.014866in}{3.196928in}}{\pgfqpoint{2.019257in}{3.207527in}}{\pgfqpoint{2.019257in}{3.218577in}}%
\pgfpathcurveto{\pgfqpoint{2.019257in}{3.229627in}}{\pgfqpoint{2.014866in}{3.240226in}}{\pgfqpoint{2.007053in}{3.248040in}}%
\pgfpathcurveto{\pgfqpoint{1.999239in}{3.255853in}}{\pgfqpoint{1.988640in}{3.260244in}}{\pgfqpoint{1.977590in}{3.260244in}}%
\pgfpathcurveto{\pgfqpoint{1.966540in}{3.260244in}}{\pgfqpoint{1.955941in}{3.255853in}}{\pgfqpoint{1.948127in}{3.248040in}}%
\pgfpathcurveto{\pgfqpoint{1.940314in}{3.240226in}}{\pgfqpoint{1.935923in}{3.229627in}}{\pgfqpoint{1.935923in}{3.218577in}}%
\pgfpathcurveto{\pgfqpoint{1.935923in}{3.207527in}}{\pgfqpoint{1.940314in}{3.196928in}}{\pgfqpoint{1.948127in}{3.189114in}}%
\pgfpathcurveto{\pgfqpoint{1.955941in}{3.181301in}}{\pgfqpoint{1.966540in}{3.176910in}}{\pgfqpoint{1.977590in}{3.176910in}}%
\pgfpathclose%
\pgfusepath{stroke,fill}%
\end{pgfscope}%
\begin{pgfscope}%
\pgfpathrectangle{\pgfqpoint{0.787074in}{0.548769in}}{\pgfqpoint{5.062926in}{3.102590in}}%
\pgfusepath{clip}%
\pgfsetbuttcap%
\pgfsetroundjoin%
\definecolor{currentfill}{rgb}{1.000000,0.498039,0.054902}%
\pgfsetfillcolor{currentfill}%
\pgfsetlinewidth{1.003750pt}%
\definecolor{currentstroke}{rgb}{1.000000,0.498039,0.054902}%
\pgfsetstrokecolor{currentstroke}%
\pgfsetdash{}{0pt}%
\pgfpathmoveto{\pgfqpoint{1.700039in}{2.972026in}}%
\pgfpathcurveto{\pgfqpoint{1.711089in}{2.972026in}}{\pgfqpoint{1.721688in}{2.976417in}}{\pgfqpoint{1.729502in}{2.984230in}}%
\pgfpathcurveto{\pgfqpoint{1.737315in}{2.992044in}}{\pgfqpoint{1.741706in}{3.002643in}}{\pgfqpoint{1.741706in}{3.013693in}}%
\pgfpathcurveto{\pgfqpoint{1.741706in}{3.024743in}}{\pgfqpoint{1.737315in}{3.035342in}}{\pgfqpoint{1.729502in}{3.043156in}}%
\pgfpathcurveto{\pgfqpoint{1.721688in}{3.050969in}}{\pgfqpoint{1.711089in}{3.055360in}}{\pgfqpoint{1.700039in}{3.055360in}}%
\pgfpathcurveto{\pgfqpoint{1.688989in}{3.055360in}}{\pgfqpoint{1.678390in}{3.050969in}}{\pgfqpoint{1.670576in}{3.043156in}}%
\pgfpathcurveto{\pgfqpoint{1.662762in}{3.035342in}}{\pgfqpoint{1.658372in}{3.024743in}}{\pgfqpoint{1.658372in}{3.013693in}}%
\pgfpathcurveto{\pgfqpoint{1.658372in}{3.002643in}}{\pgfqpoint{1.662762in}{2.992044in}}{\pgfqpoint{1.670576in}{2.984230in}}%
\pgfpathcurveto{\pgfqpoint{1.678390in}{2.976417in}}{\pgfqpoint{1.688989in}{2.972026in}}{\pgfqpoint{1.700039in}{2.972026in}}%
\pgfpathclose%
\pgfusepath{stroke,fill}%
\end{pgfscope}%
\begin{pgfscope}%
\pgfpathrectangle{\pgfqpoint{0.787074in}{0.548769in}}{\pgfqpoint{5.062926in}{3.102590in}}%
\pgfusepath{clip}%
\pgfsetbuttcap%
\pgfsetroundjoin%
\definecolor{currentfill}{rgb}{1.000000,0.498039,0.054902}%
\pgfsetfillcolor{currentfill}%
\pgfsetlinewidth{1.003750pt}%
\definecolor{currentstroke}{rgb}{1.000000,0.498039,0.054902}%
\pgfsetstrokecolor{currentstroke}%
\pgfsetdash{}{0pt}%
\pgfpathmoveto{\pgfqpoint{1.868479in}{3.386124in}}%
\pgfpathcurveto{\pgfqpoint{1.879529in}{3.386124in}}{\pgfqpoint{1.890128in}{3.390514in}}{\pgfqpoint{1.897942in}{3.398328in}}%
\pgfpathcurveto{\pgfqpoint{1.905756in}{3.406142in}}{\pgfqpoint{1.910146in}{3.416741in}}{\pgfqpoint{1.910146in}{3.427791in}}%
\pgfpathcurveto{\pgfqpoint{1.910146in}{3.438841in}}{\pgfqpoint{1.905756in}{3.449440in}}{\pgfqpoint{1.897942in}{3.457254in}}%
\pgfpathcurveto{\pgfqpoint{1.890128in}{3.465067in}}{\pgfqpoint{1.879529in}{3.469458in}}{\pgfqpoint{1.868479in}{3.469458in}}%
\pgfpathcurveto{\pgfqpoint{1.857429in}{3.469458in}}{\pgfqpoint{1.846830in}{3.465067in}}{\pgfqpoint{1.839016in}{3.457254in}}%
\pgfpathcurveto{\pgfqpoint{1.831203in}{3.449440in}}{\pgfqpoint{1.826812in}{3.438841in}}{\pgfqpoint{1.826812in}{3.427791in}}%
\pgfpathcurveto{\pgfqpoint{1.826812in}{3.416741in}}{\pgfqpoint{1.831203in}{3.406142in}}{\pgfqpoint{1.839016in}{3.398328in}}%
\pgfpathcurveto{\pgfqpoint{1.846830in}{3.390514in}}{\pgfqpoint{1.857429in}{3.386124in}}{\pgfqpoint{1.868479in}{3.386124in}}%
\pgfpathclose%
\pgfusepath{stroke,fill}%
\end{pgfscope}%
\begin{pgfscope}%
\pgfpathrectangle{\pgfqpoint{0.787074in}{0.548769in}}{\pgfqpoint{5.062926in}{3.102590in}}%
\pgfusepath{clip}%
\pgfsetbuttcap%
\pgfsetroundjoin%
\definecolor{currentfill}{rgb}{1.000000,0.498039,0.054902}%
\pgfsetfillcolor{currentfill}%
\pgfsetlinewidth{1.003750pt}%
\definecolor{currentstroke}{rgb}{1.000000,0.498039,0.054902}%
\pgfsetstrokecolor{currentstroke}%
\pgfsetdash{}{0pt}%
\pgfpathmoveto{\pgfqpoint{1.754724in}{2.328792in}}%
\pgfpathcurveto{\pgfqpoint{1.765775in}{2.328792in}}{\pgfqpoint{1.776374in}{2.333182in}}{\pgfqpoint{1.784187in}{2.340995in}}%
\pgfpathcurveto{\pgfqpoint{1.792001in}{2.348809in}}{\pgfqpoint{1.796391in}{2.359408in}}{\pgfqpoint{1.796391in}{2.370458in}}%
\pgfpathcurveto{\pgfqpoint{1.796391in}{2.381508in}}{\pgfqpoint{1.792001in}{2.392107in}}{\pgfqpoint{1.784187in}{2.399921in}}%
\pgfpathcurveto{\pgfqpoint{1.776374in}{2.407735in}}{\pgfqpoint{1.765775in}{2.412125in}}{\pgfqpoint{1.754724in}{2.412125in}}%
\pgfpathcurveto{\pgfqpoint{1.743674in}{2.412125in}}{\pgfqpoint{1.733075in}{2.407735in}}{\pgfqpoint{1.725262in}{2.399921in}}%
\pgfpathcurveto{\pgfqpoint{1.717448in}{2.392107in}}{\pgfqpoint{1.713058in}{2.381508in}}{\pgfqpoint{1.713058in}{2.370458in}}%
\pgfpathcurveto{\pgfqpoint{1.713058in}{2.359408in}}{\pgfqpoint{1.717448in}{2.348809in}}{\pgfqpoint{1.725262in}{2.340995in}}%
\pgfpathcurveto{\pgfqpoint{1.733075in}{2.333182in}}{\pgfqpoint{1.743674in}{2.328792in}}{\pgfqpoint{1.754724in}{2.328792in}}%
\pgfpathclose%
\pgfusepath{stroke,fill}%
\end{pgfscope}%
\begin{pgfscope}%
\pgfpathrectangle{\pgfqpoint{0.787074in}{0.548769in}}{\pgfqpoint{5.062926in}{3.102590in}}%
\pgfusepath{clip}%
\pgfsetbuttcap%
\pgfsetroundjoin%
\definecolor{currentfill}{rgb}{1.000000,0.498039,0.054902}%
\pgfsetfillcolor{currentfill}%
\pgfsetlinewidth{1.003750pt}%
\definecolor{currentstroke}{rgb}{1.000000,0.498039,0.054902}%
\pgfsetstrokecolor{currentstroke}%
\pgfsetdash{}{0pt}%
\pgfpathmoveto{\pgfqpoint{2.343029in}{3.073267in}}%
\pgfpathcurveto{\pgfqpoint{2.354079in}{3.073267in}}{\pgfqpoint{2.364678in}{3.077657in}}{\pgfqpoint{2.372491in}{3.085471in}}%
\pgfpathcurveto{\pgfqpoint{2.380305in}{3.093285in}}{\pgfqpoint{2.384695in}{3.103884in}}{\pgfqpoint{2.384695in}{3.114934in}}%
\pgfpathcurveto{\pgfqpoint{2.384695in}{3.125984in}}{\pgfqpoint{2.380305in}{3.136583in}}{\pgfqpoint{2.372491in}{3.144396in}}%
\pgfpathcurveto{\pgfqpoint{2.364678in}{3.152210in}}{\pgfqpoint{2.354079in}{3.156600in}}{\pgfqpoint{2.343029in}{3.156600in}}%
\pgfpathcurveto{\pgfqpoint{2.331978in}{3.156600in}}{\pgfqpoint{2.321379in}{3.152210in}}{\pgfqpoint{2.313566in}{3.144396in}}%
\pgfpathcurveto{\pgfqpoint{2.305752in}{3.136583in}}{\pgfqpoint{2.301362in}{3.125984in}}{\pgfqpoint{2.301362in}{3.114934in}}%
\pgfpathcurveto{\pgfqpoint{2.301362in}{3.103884in}}{\pgfqpoint{2.305752in}{3.093285in}}{\pgfqpoint{2.313566in}{3.085471in}}%
\pgfpathcurveto{\pgfqpoint{2.321379in}{3.077657in}}{\pgfqpoint{2.331978in}{3.073267in}}{\pgfqpoint{2.343029in}{3.073267in}}%
\pgfpathclose%
\pgfusepath{stroke,fill}%
\end{pgfscope}%
\begin{pgfscope}%
\pgfpathrectangle{\pgfqpoint{0.787074in}{0.548769in}}{\pgfqpoint{5.062926in}{3.102590in}}%
\pgfusepath{clip}%
\pgfsetbuttcap%
\pgfsetroundjoin%
\definecolor{currentfill}{rgb}{0.121569,0.466667,0.705882}%
\pgfsetfillcolor{currentfill}%
\pgfsetlinewidth{1.003750pt}%
\definecolor{currentstroke}{rgb}{0.121569,0.466667,0.705882}%
\pgfsetstrokecolor{currentstroke}%
\pgfsetdash{}{0pt}%
\pgfpathmoveto{\pgfqpoint{1.656941in}{0.648131in}}%
\pgfpathcurveto{\pgfqpoint{1.667992in}{0.648131in}}{\pgfqpoint{1.678591in}{0.652521in}}{\pgfqpoint{1.686404in}{0.660335in}}%
\pgfpathcurveto{\pgfqpoint{1.694218in}{0.668148in}}{\pgfqpoint{1.698608in}{0.678748in}}{\pgfqpoint{1.698608in}{0.689798in}}%
\pgfpathcurveto{\pgfqpoint{1.698608in}{0.700848in}}{\pgfqpoint{1.694218in}{0.711447in}}{\pgfqpoint{1.686404in}{0.719260in}}%
\pgfpathcurveto{\pgfqpoint{1.678591in}{0.727074in}}{\pgfqpoint{1.667992in}{0.731464in}}{\pgfqpoint{1.656941in}{0.731464in}}%
\pgfpathcurveto{\pgfqpoint{1.645891in}{0.731464in}}{\pgfqpoint{1.635292in}{0.727074in}}{\pgfqpoint{1.627479in}{0.719260in}}%
\pgfpathcurveto{\pgfqpoint{1.619665in}{0.711447in}}{\pgfqpoint{1.615275in}{0.700848in}}{\pgfqpoint{1.615275in}{0.689798in}}%
\pgfpathcurveto{\pgfqpoint{1.615275in}{0.678748in}}{\pgfqpoint{1.619665in}{0.668148in}}{\pgfqpoint{1.627479in}{0.660335in}}%
\pgfpathcurveto{\pgfqpoint{1.635292in}{0.652521in}}{\pgfqpoint{1.645891in}{0.648131in}}{\pgfqpoint{1.656941in}{0.648131in}}%
\pgfpathclose%
\pgfusepath{stroke,fill}%
\end{pgfscope}%
\begin{pgfscope}%
\pgfpathrectangle{\pgfqpoint{0.787074in}{0.548769in}}{\pgfqpoint{5.062926in}{3.102590in}}%
\pgfusepath{clip}%
\pgfsetbuttcap%
\pgfsetroundjoin%
\definecolor{currentfill}{rgb}{1.000000,0.498039,0.054902}%
\pgfsetfillcolor{currentfill}%
\pgfsetlinewidth{1.003750pt}%
\definecolor{currentstroke}{rgb}{1.000000,0.498039,0.054902}%
\pgfsetstrokecolor{currentstroke}%
\pgfsetdash{}{0pt}%
\pgfpathmoveto{\pgfqpoint{2.089044in}{1.987192in}}%
\pgfpathcurveto{\pgfqpoint{2.100094in}{1.987192in}}{\pgfqpoint{2.110694in}{1.991582in}}{\pgfqpoint{2.118507in}{1.999395in}}%
\pgfpathcurveto{\pgfqpoint{2.126321in}{2.007209in}}{\pgfqpoint{2.130711in}{2.017808in}}{\pgfqpoint{2.130711in}{2.028858in}}%
\pgfpathcurveto{\pgfqpoint{2.130711in}{2.039908in}}{\pgfqpoint{2.126321in}{2.050507in}}{\pgfqpoint{2.118507in}{2.058321in}}%
\pgfpathcurveto{\pgfqpoint{2.110694in}{2.066135in}}{\pgfqpoint{2.100094in}{2.070525in}}{\pgfqpoint{2.089044in}{2.070525in}}%
\pgfpathcurveto{\pgfqpoint{2.077994in}{2.070525in}}{\pgfqpoint{2.067395in}{2.066135in}}{\pgfqpoint{2.059582in}{2.058321in}}%
\pgfpathcurveto{\pgfqpoint{2.051768in}{2.050507in}}{\pgfqpoint{2.047378in}{2.039908in}}{\pgfqpoint{2.047378in}{2.028858in}}%
\pgfpathcurveto{\pgfqpoint{2.047378in}{2.017808in}}{\pgfqpoint{2.051768in}{2.007209in}}{\pgfqpoint{2.059582in}{1.999395in}}%
\pgfpathcurveto{\pgfqpoint{2.067395in}{1.991582in}}{\pgfqpoint{2.077994in}{1.987192in}}{\pgfqpoint{2.089044in}{1.987192in}}%
\pgfpathclose%
\pgfusepath{stroke,fill}%
\end{pgfscope}%
\begin{pgfscope}%
\pgfpathrectangle{\pgfqpoint{0.787074in}{0.548769in}}{\pgfqpoint{5.062926in}{3.102590in}}%
\pgfusepath{clip}%
\pgfsetbuttcap%
\pgfsetroundjoin%
\definecolor{currentfill}{rgb}{1.000000,0.498039,0.054902}%
\pgfsetfillcolor{currentfill}%
\pgfsetlinewidth{1.003750pt}%
\definecolor{currentstroke}{rgb}{1.000000,0.498039,0.054902}%
\pgfsetstrokecolor{currentstroke}%
\pgfsetdash{}{0pt}%
\pgfpathmoveto{\pgfqpoint{1.482034in}{2.820143in}}%
\pgfpathcurveto{\pgfqpoint{1.493084in}{2.820143in}}{\pgfqpoint{1.503683in}{2.824533in}}{\pgfqpoint{1.511497in}{2.832347in}}%
\pgfpathcurveto{\pgfqpoint{1.519311in}{2.840160in}}{\pgfqpoint{1.523701in}{2.850759in}}{\pgfqpoint{1.523701in}{2.861810in}}%
\pgfpathcurveto{\pgfqpoint{1.523701in}{2.872860in}}{\pgfqpoint{1.519311in}{2.883459in}}{\pgfqpoint{1.511497in}{2.891272in}}%
\pgfpathcurveto{\pgfqpoint{1.503683in}{2.899086in}}{\pgfqpoint{1.493084in}{2.903476in}}{\pgfqpoint{1.482034in}{2.903476in}}%
\pgfpathcurveto{\pgfqpoint{1.470984in}{2.903476in}}{\pgfqpoint{1.460385in}{2.899086in}}{\pgfqpoint{1.452572in}{2.891272in}}%
\pgfpathcurveto{\pgfqpoint{1.444758in}{2.883459in}}{\pgfqpoint{1.440368in}{2.872860in}}{\pgfqpoint{1.440368in}{2.861810in}}%
\pgfpathcurveto{\pgfqpoint{1.440368in}{2.850759in}}{\pgfqpoint{1.444758in}{2.840160in}}{\pgfqpoint{1.452572in}{2.832347in}}%
\pgfpathcurveto{\pgfqpoint{1.460385in}{2.824533in}}{\pgfqpoint{1.470984in}{2.820143in}}{\pgfqpoint{1.482034in}{2.820143in}}%
\pgfpathclose%
\pgfusepath{stroke,fill}%
\end{pgfscope}%
\begin{pgfscope}%
\pgfpathrectangle{\pgfqpoint{0.787074in}{0.548769in}}{\pgfqpoint{5.062926in}{3.102590in}}%
\pgfusepath{clip}%
\pgfsetbuttcap%
\pgfsetroundjoin%
\definecolor{currentfill}{rgb}{1.000000,0.498039,0.054902}%
\pgfsetfillcolor{currentfill}%
\pgfsetlinewidth{1.003750pt}%
\definecolor{currentstroke}{rgb}{1.000000,0.498039,0.054902}%
\pgfsetstrokecolor{currentstroke}%
\pgfsetdash{}{0pt}%
\pgfpathmoveto{\pgfqpoint{1.699431in}{2.523468in}}%
\pgfpathcurveto{\pgfqpoint{1.710481in}{2.523468in}}{\pgfqpoint{1.721080in}{2.527858in}}{\pgfqpoint{1.728894in}{2.535672in}}%
\pgfpathcurveto{\pgfqpoint{1.736708in}{2.543485in}}{\pgfqpoint{1.741098in}{2.554084in}}{\pgfqpoint{1.741098in}{2.565134in}}%
\pgfpathcurveto{\pgfqpoint{1.741098in}{2.576185in}}{\pgfqpoint{1.736708in}{2.586784in}}{\pgfqpoint{1.728894in}{2.594597in}}%
\pgfpathcurveto{\pgfqpoint{1.721080in}{2.602411in}}{\pgfqpoint{1.710481in}{2.606801in}}{\pgfqpoint{1.699431in}{2.606801in}}%
\pgfpathcurveto{\pgfqpoint{1.688381in}{2.606801in}}{\pgfqpoint{1.677782in}{2.602411in}}{\pgfqpoint{1.669968in}{2.594597in}}%
\pgfpathcurveto{\pgfqpoint{1.662155in}{2.586784in}}{\pgfqpoint{1.657765in}{2.576185in}}{\pgfqpoint{1.657765in}{2.565134in}}%
\pgfpathcurveto{\pgfqpoint{1.657765in}{2.554084in}}{\pgfqpoint{1.662155in}{2.543485in}}{\pgfqpoint{1.669968in}{2.535672in}}%
\pgfpathcurveto{\pgfqpoint{1.677782in}{2.527858in}}{\pgfqpoint{1.688381in}{2.523468in}}{\pgfqpoint{1.699431in}{2.523468in}}%
\pgfpathclose%
\pgfusepath{stroke,fill}%
\end{pgfscope}%
\begin{pgfscope}%
\pgfpathrectangle{\pgfqpoint{0.787074in}{0.548769in}}{\pgfqpoint{5.062926in}{3.102590in}}%
\pgfusepath{clip}%
\pgfsetbuttcap%
\pgfsetroundjoin%
\definecolor{currentfill}{rgb}{1.000000,0.498039,0.054902}%
\pgfsetfillcolor{currentfill}%
\pgfsetlinewidth{1.003750pt}%
\definecolor{currentstroke}{rgb}{1.000000,0.498039,0.054902}%
\pgfsetstrokecolor{currentstroke}%
\pgfsetdash{}{0pt}%
\pgfpathmoveto{\pgfqpoint{1.908452in}{1.569832in}}%
\pgfpathcurveto{\pgfqpoint{1.919502in}{1.569832in}}{\pgfqpoint{1.930101in}{1.574223in}}{\pgfqpoint{1.937914in}{1.582036in}}%
\pgfpathcurveto{\pgfqpoint{1.945728in}{1.589850in}}{\pgfqpoint{1.950118in}{1.600449in}}{\pgfqpoint{1.950118in}{1.611499in}}%
\pgfpathcurveto{\pgfqpoint{1.950118in}{1.622549in}}{\pgfqpoint{1.945728in}{1.633148in}}{\pgfqpoint{1.937914in}{1.640962in}}%
\pgfpathcurveto{\pgfqpoint{1.930101in}{1.648775in}}{\pgfqpoint{1.919502in}{1.653166in}}{\pgfqpoint{1.908452in}{1.653166in}}%
\pgfpathcurveto{\pgfqpoint{1.897402in}{1.653166in}}{\pgfqpoint{1.886803in}{1.648775in}}{\pgfqpoint{1.878989in}{1.640962in}}%
\pgfpathcurveto{\pgfqpoint{1.871175in}{1.633148in}}{\pgfqpoint{1.866785in}{1.622549in}}{\pgfqpoint{1.866785in}{1.611499in}}%
\pgfpathcurveto{\pgfqpoint{1.866785in}{1.600449in}}{\pgfqpoint{1.871175in}{1.589850in}}{\pgfqpoint{1.878989in}{1.582036in}}%
\pgfpathcurveto{\pgfqpoint{1.886803in}{1.574223in}}{\pgfqpoint{1.897402in}{1.569832in}}{\pgfqpoint{1.908452in}{1.569832in}}%
\pgfpathclose%
\pgfusepath{stroke,fill}%
\end{pgfscope}%
\begin{pgfscope}%
\pgfpathrectangle{\pgfqpoint{0.787074in}{0.548769in}}{\pgfqpoint{5.062926in}{3.102590in}}%
\pgfusepath{clip}%
\pgfsetbuttcap%
\pgfsetroundjoin%
\definecolor{currentfill}{rgb}{1.000000,0.498039,0.054902}%
\pgfsetfillcolor{currentfill}%
\pgfsetlinewidth{1.003750pt}%
\definecolor{currentstroke}{rgb}{1.000000,0.498039,0.054902}%
\pgfsetstrokecolor{currentstroke}%
\pgfsetdash{}{0pt}%
\pgfpathmoveto{\pgfqpoint{1.639798in}{2.228984in}}%
\pgfpathcurveto{\pgfqpoint{1.650848in}{2.228984in}}{\pgfqpoint{1.661447in}{2.233374in}}{\pgfqpoint{1.669261in}{2.241188in}}%
\pgfpathcurveto{\pgfqpoint{1.677074in}{2.249002in}}{\pgfqpoint{1.681465in}{2.259601in}}{\pgfqpoint{1.681465in}{2.270651in}}%
\pgfpathcurveto{\pgfqpoint{1.681465in}{2.281701in}}{\pgfqpoint{1.677074in}{2.292300in}}{\pgfqpoint{1.669261in}{2.300114in}}%
\pgfpathcurveto{\pgfqpoint{1.661447in}{2.307927in}}{\pgfqpoint{1.650848in}{2.312318in}}{\pgfqpoint{1.639798in}{2.312318in}}%
\pgfpathcurveto{\pgfqpoint{1.628748in}{2.312318in}}{\pgfqpoint{1.618149in}{2.307927in}}{\pgfqpoint{1.610335in}{2.300114in}}%
\pgfpathcurveto{\pgfqpoint{1.602522in}{2.292300in}}{\pgfqpoint{1.598131in}{2.281701in}}{\pgfqpoint{1.598131in}{2.270651in}}%
\pgfpathcurveto{\pgfqpoint{1.598131in}{2.259601in}}{\pgfqpoint{1.602522in}{2.249002in}}{\pgfqpoint{1.610335in}{2.241188in}}%
\pgfpathcurveto{\pgfqpoint{1.618149in}{2.233374in}}{\pgfqpoint{1.628748in}{2.228984in}}{\pgfqpoint{1.639798in}{2.228984in}}%
\pgfpathclose%
\pgfusepath{stroke,fill}%
\end{pgfscope}%
\begin{pgfscope}%
\pgfpathrectangle{\pgfqpoint{0.787074in}{0.548769in}}{\pgfqpoint{5.062926in}{3.102590in}}%
\pgfusepath{clip}%
\pgfsetbuttcap%
\pgfsetroundjoin%
\definecolor{currentfill}{rgb}{0.121569,0.466667,0.705882}%
\pgfsetfillcolor{currentfill}%
\pgfsetlinewidth{1.003750pt}%
\definecolor{currentstroke}{rgb}{0.121569,0.466667,0.705882}%
\pgfsetstrokecolor{currentstroke}%
\pgfsetdash{}{0pt}%
\pgfpathmoveto{\pgfqpoint{2.356222in}{2.843750in}}%
\pgfpathcurveto{\pgfqpoint{2.367273in}{2.843750in}}{\pgfqpoint{2.377872in}{2.848141in}}{\pgfqpoint{2.385685in}{2.855954in}}%
\pgfpathcurveto{\pgfqpoint{2.393499in}{2.863768in}}{\pgfqpoint{2.397889in}{2.874367in}}{\pgfqpoint{2.397889in}{2.885417in}}%
\pgfpathcurveto{\pgfqpoint{2.397889in}{2.896467in}}{\pgfqpoint{2.393499in}{2.907066in}}{\pgfqpoint{2.385685in}{2.914880in}}%
\pgfpathcurveto{\pgfqpoint{2.377872in}{2.922693in}}{\pgfqpoint{2.367273in}{2.927084in}}{\pgfqpoint{2.356222in}{2.927084in}}%
\pgfpathcurveto{\pgfqpoint{2.345172in}{2.927084in}}{\pgfqpoint{2.334573in}{2.922693in}}{\pgfqpoint{2.326760in}{2.914880in}}%
\pgfpathcurveto{\pgfqpoint{2.318946in}{2.907066in}}{\pgfqpoint{2.314556in}{2.896467in}}{\pgfqpoint{2.314556in}{2.885417in}}%
\pgfpathcurveto{\pgfqpoint{2.314556in}{2.874367in}}{\pgfqpoint{2.318946in}{2.863768in}}{\pgfqpoint{2.326760in}{2.855954in}}%
\pgfpathcurveto{\pgfqpoint{2.334573in}{2.848141in}}{\pgfqpoint{2.345172in}{2.843750in}}{\pgfqpoint{2.356222in}{2.843750in}}%
\pgfpathclose%
\pgfusepath{stroke,fill}%
\end{pgfscope}%
\begin{pgfscope}%
\pgfpathrectangle{\pgfqpoint{0.787074in}{0.548769in}}{\pgfqpoint{5.062926in}{3.102590in}}%
\pgfusepath{clip}%
\pgfsetbuttcap%
\pgfsetroundjoin%
\definecolor{currentfill}{rgb}{1.000000,0.498039,0.054902}%
\pgfsetfillcolor{currentfill}%
\pgfsetlinewidth{1.003750pt}%
\definecolor{currentstroke}{rgb}{1.000000,0.498039,0.054902}%
\pgfsetstrokecolor{currentstroke}%
\pgfsetdash{}{0pt}%
\pgfpathmoveto{\pgfqpoint{1.887923in}{3.155929in}}%
\pgfpathcurveto{\pgfqpoint{1.898973in}{3.155929in}}{\pgfqpoint{1.909572in}{3.160320in}}{\pgfqpoint{1.917386in}{3.168133in}}%
\pgfpathcurveto{\pgfqpoint{1.925199in}{3.175947in}}{\pgfqpoint{1.929590in}{3.186546in}}{\pgfqpoint{1.929590in}{3.197596in}}%
\pgfpathcurveto{\pgfqpoint{1.929590in}{3.208646in}}{\pgfqpoint{1.925199in}{3.219245in}}{\pgfqpoint{1.917386in}{3.227059in}}%
\pgfpathcurveto{\pgfqpoint{1.909572in}{3.234872in}}{\pgfqpoint{1.898973in}{3.239263in}}{\pgfqpoint{1.887923in}{3.239263in}}%
\pgfpathcurveto{\pgfqpoint{1.876873in}{3.239263in}}{\pgfqpoint{1.866274in}{3.234872in}}{\pgfqpoint{1.858460in}{3.227059in}}%
\pgfpathcurveto{\pgfqpoint{1.850647in}{3.219245in}}{\pgfqpoint{1.846256in}{3.208646in}}{\pgfqpoint{1.846256in}{3.197596in}}%
\pgfpathcurveto{\pgfqpoint{1.846256in}{3.186546in}}{\pgfqpoint{1.850647in}{3.175947in}}{\pgfqpoint{1.858460in}{3.168133in}}%
\pgfpathcurveto{\pgfqpoint{1.866274in}{3.160320in}}{\pgfqpoint{1.876873in}{3.155929in}}{\pgfqpoint{1.887923in}{3.155929in}}%
\pgfpathclose%
\pgfusepath{stroke,fill}%
\end{pgfscope}%
\begin{pgfscope}%
\pgfsetbuttcap%
\pgfsetroundjoin%
\definecolor{currentfill}{rgb}{0.000000,0.000000,0.000000}%
\pgfsetfillcolor{currentfill}%
\pgfsetlinewidth{0.803000pt}%
\definecolor{currentstroke}{rgb}{0.000000,0.000000,0.000000}%
\pgfsetstrokecolor{currentstroke}%
\pgfsetdash{}{0pt}%
\pgfsys@defobject{currentmarker}{\pgfqpoint{0.000000in}{-0.048611in}}{\pgfqpoint{0.000000in}{0.000000in}}{%
\pgfpathmoveto{\pgfqpoint{0.000000in}{0.000000in}}%
\pgfpathlineto{\pgfqpoint{0.000000in}{-0.048611in}}%
\pgfusepath{stroke,fill}%
}%
\begin{pgfscope}%
\pgfsys@transformshift{0.987520in}{0.548769in}%
\pgfsys@useobject{currentmarker}{}%
\end{pgfscope}%
\end{pgfscope}%
\begin{pgfscope}%
\definecolor{textcolor}{rgb}{0.000000,0.000000,0.000000}%
\pgfsetstrokecolor{textcolor}%
\pgfsetfillcolor{textcolor}%
\pgftext[x=0.987520in,y=0.451547in,,top]{\color{textcolor}\sffamily\fontsize{10.000000}{12.000000}\selectfont \(\displaystyle {0}\)}%
\end{pgfscope}%
\begin{pgfscope}%
\pgfsetbuttcap%
\pgfsetroundjoin%
\definecolor{currentfill}{rgb}{0.000000,0.000000,0.000000}%
\pgfsetfillcolor{currentfill}%
\pgfsetlinewidth{0.803000pt}%
\definecolor{currentstroke}{rgb}{0.000000,0.000000,0.000000}%
\pgfsetstrokecolor{currentstroke}%
\pgfsetdash{}{0pt}%
\pgfsys@defobject{currentmarker}{\pgfqpoint{0.000000in}{-0.048611in}}{\pgfqpoint{0.000000in}{0.000000in}}{%
\pgfpathmoveto{\pgfqpoint{0.000000in}{0.000000in}}%
\pgfpathlineto{\pgfqpoint{0.000000in}{-0.048611in}}%
\pgfusepath{stroke,fill}%
}%
\begin{pgfscope}%
\pgfsys@transformshift{1.855546in}{0.548769in}%
\pgfsys@useobject{currentmarker}{}%
\end{pgfscope}%
\end{pgfscope}%
\begin{pgfscope}%
\definecolor{textcolor}{rgb}{0.000000,0.000000,0.000000}%
\pgfsetstrokecolor{textcolor}%
\pgfsetfillcolor{textcolor}%
\pgftext[x=1.855546in,y=0.451547in,,top]{\color{textcolor}\sffamily\fontsize{10.000000}{12.000000}\selectfont \(\displaystyle {20000}\)}%
\end{pgfscope}%
\begin{pgfscope}%
\pgfsetbuttcap%
\pgfsetroundjoin%
\definecolor{currentfill}{rgb}{0.000000,0.000000,0.000000}%
\pgfsetfillcolor{currentfill}%
\pgfsetlinewidth{0.803000pt}%
\definecolor{currentstroke}{rgb}{0.000000,0.000000,0.000000}%
\pgfsetstrokecolor{currentstroke}%
\pgfsetdash{}{0pt}%
\pgfsys@defobject{currentmarker}{\pgfqpoint{0.000000in}{-0.048611in}}{\pgfqpoint{0.000000in}{0.000000in}}{%
\pgfpathmoveto{\pgfqpoint{0.000000in}{0.000000in}}%
\pgfpathlineto{\pgfqpoint{0.000000in}{-0.048611in}}%
\pgfusepath{stroke,fill}%
}%
\begin{pgfscope}%
\pgfsys@transformshift{2.723571in}{0.548769in}%
\pgfsys@useobject{currentmarker}{}%
\end{pgfscope}%
\end{pgfscope}%
\begin{pgfscope}%
\definecolor{textcolor}{rgb}{0.000000,0.000000,0.000000}%
\pgfsetstrokecolor{textcolor}%
\pgfsetfillcolor{textcolor}%
\pgftext[x=2.723571in,y=0.451547in,,top]{\color{textcolor}\sffamily\fontsize{10.000000}{12.000000}\selectfont \(\displaystyle {40000}\)}%
\end{pgfscope}%
\begin{pgfscope}%
\pgfsetbuttcap%
\pgfsetroundjoin%
\definecolor{currentfill}{rgb}{0.000000,0.000000,0.000000}%
\pgfsetfillcolor{currentfill}%
\pgfsetlinewidth{0.803000pt}%
\definecolor{currentstroke}{rgb}{0.000000,0.000000,0.000000}%
\pgfsetstrokecolor{currentstroke}%
\pgfsetdash{}{0pt}%
\pgfsys@defobject{currentmarker}{\pgfqpoint{0.000000in}{-0.048611in}}{\pgfqpoint{0.000000in}{0.000000in}}{%
\pgfpathmoveto{\pgfqpoint{0.000000in}{0.000000in}}%
\pgfpathlineto{\pgfqpoint{0.000000in}{-0.048611in}}%
\pgfusepath{stroke,fill}%
}%
\begin{pgfscope}%
\pgfsys@transformshift{3.591596in}{0.548769in}%
\pgfsys@useobject{currentmarker}{}%
\end{pgfscope}%
\end{pgfscope}%
\begin{pgfscope}%
\definecolor{textcolor}{rgb}{0.000000,0.000000,0.000000}%
\pgfsetstrokecolor{textcolor}%
\pgfsetfillcolor{textcolor}%
\pgftext[x=3.591596in,y=0.451547in,,top]{\color{textcolor}\sffamily\fontsize{10.000000}{12.000000}\selectfont \(\displaystyle {60000}\)}%
\end{pgfscope}%
\begin{pgfscope}%
\pgfsetbuttcap%
\pgfsetroundjoin%
\definecolor{currentfill}{rgb}{0.000000,0.000000,0.000000}%
\pgfsetfillcolor{currentfill}%
\pgfsetlinewidth{0.803000pt}%
\definecolor{currentstroke}{rgb}{0.000000,0.000000,0.000000}%
\pgfsetstrokecolor{currentstroke}%
\pgfsetdash{}{0pt}%
\pgfsys@defobject{currentmarker}{\pgfqpoint{0.000000in}{-0.048611in}}{\pgfqpoint{0.000000in}{0.000000in}}{%
\pgfpathmoveto{\pgfqpoint{0.000000in}{0.000000in}}%
\pgfpathlineto{\pgfqpoint{0.000000in}{-0.048611in}}%
\pgfusepath{stroke,fill}%
}%
\begin{pgfscope}%
\pgfsys@transformshift{4.459621in}{0.548769in}%
\pgfsys@useobject{currentmarker}{}%
\end{pgfscope}%
\end{pgfscope}%
\begin{pgfscope}%
\definecolor{textcolor}{rgb}{0.000000,0.000000,0.000000}%
\pgfsetstrokecolor{textcolor}%
\pgfsetfillcolor{textcolor}%
\pgftext[x=4.459621in,y=0.451547in,,top]{\color{textcolor}\sffamily\fontsize{10.000000}{12.000000}\selectfont \(\displaystyle {80000}\)}%
\end{pgfscope}%
\begin{pgfscope}%
\pgfsetbuttcap%
\pgfsetroundjoin%
\definecolor{currentfill}{rgb}{0.000000,0.000000,0.000000}%
\pgfsetfillcolor{currentfill}%
\pgfsetlinewidth{0.803000pt}%
\definecolor{currentstroke}{rgb}{0.000000,0.000000,0.000000}%
\pgfsetstrokecolor{currentstroke}%
\pgfsetdash{}{0pt}%
\pgfsys@defobject{currentmarker}{\pgfqpoint{0.000000in}{-0.048611in}}{\pgfqpoint{0.000000in}{0.000000in}}{%
\pgfpathmoveto{\pgfqpoint{0.000000in}{0.000000in}}%
\pgfpathlineto{\pgfqpoint{0.000000in}{-0.048611in}}%
\pgfusepath{stroke,fill}%
}%
\begin{pgfscope}%
\pgfsys@transformshift{5.327646in}{0.548769in}%
\pgfsys@useobject{currentmarker}{}%
\end{pgfscope}%
\end{pgfscope}%
\begin{pgfscope}%
\definecolor{textcolor}{rgb}{0.000000,0.000000,0.000000}%
\pgfsetstrokecolor{textcolor}%
\pgfsetfillcolor{textcolor}%
\pgftext[x=5.327646in,y=0.451547in,,top]{\color{textcolor}\sffamily\fontsize{10.000000}{12.000000}\selectfont \(\displaystyle {100000}\)}%
\end{pgfscope}%
\begin{pgfscope}%
\definecolor{textcolor}{rgb}{0.000000,0.000000,0.000000}%
\pgfsetstrokecolor{textcolor}%
\pgfsetfillcolor{textcolor}%
\pgftext[x=3.318537in,y=0.272658in,,top]{\color{textcolor}\sffamily\fontsize{10.000000}{12.000000}\selectfont Classes}%
\end{pgfscope}%
\begin{pgfscope}%
\pgfsetbuttcap%
\pgfsetroundjoin%
\definecolor{currentfill}{rgb}{0.000000,0.000000,0.000000}%
\pgfsetfillcolor{currentfill}%
\pgfsetlinewidth{0.803000pt}%
\definecolor{currentstroke}{rgb}{0.000000,0.000000,0.000000}%
\pgfsetstrokecolor{currentstroke}%
\pgfsetdash{}{0pt}%
\pgfsys@defobject{currentmarker}{\pgfqpoint{-0.048611in}{0.000000in}}{\pgfqpoint{0.000000in}{0.000000in}}{%
\pgfpathmoveto{\pgfqpoint{0.000000in}{0.000000in}}%
\pgfpathlineto{\pgfqpoint{-0.048611in}{0.000000in}}%
\pgfusepath{stroke,fill}%
}%
\begin{pgfscope}%
\pgfsys@transformshift{0.787074in}{0.689795in}%
\pgfsys@useobject{currentmarker}{}%
\end{pgfscope}%
\end{pgfscope}%
\begin{pgfscope}%
\definecolor{textcolor}{rgb}{0.000000,0.000000,0.000000}%
\pgfsetstrokecolor{textcolor}%
\pgfsetfillcolor{textcolor}%
\pgftext[x=0.620407in, y=0.641601in, left, base]{\color{textcolor}\sffamily\fontsize{10.000000}{12.000000}\selectfont \(\displaystyle {0}\)}%
\end{pgfscope}%
\begin{pgfscope}%
\pgfsetbuttcap%
\pgfsetroundjoin%
\definecolor{currentfill}{rgb}{0.000000,0.000000,0.000000}%
\pgfsetfillcolor{currentfill}%
\pgfsetlinewidth{0.803000pt}%
\definecolor{currentstroke}{rgb}{0.000000,0.000000,0.000000}%
\pgfsetstrokecolor{currentstroke}%
\pgfsetdash{}{0pt}%
\pgfsys@defobject{currentmarker}{\pgfqpoint{-0.048611in}{0.000000in}}{\pgfqpoint{0.000000in}{0.000000in}}{%
\pgfpathmoveto{\pgfqpoint{0.000000in}{0.000000in}}%
\pgfpathlineto{\pgfqpoint{-0.048611in}{0.000000in}}%
\pgfusepath{stroke,fill}%
}%
\begin{pgfscope}%
\pgfsys@transformshift{0.787074in}{1.059666in}%
\pgfsys@useobject{currentmarker}{}%
\end{pgfscope}%
\end{pgfscope}%
\begin{pgfscope}%
\definecolor{textcolor}{rgb}{0.000000,0.000000,0.000000}%
\pgfsetstrokecolor{textcolor}%
\pgfsetfillcolor{textcolor}%
\pgftext[x=0.412073in, y=1.011471in, left, base]{\color{textcolor}\sffamily\fontsize{10.000000}{12.000000}\selectfont \(\displaystyle {2500}\)}%
\end{pgfscope}%
\begin{pgfscope}%
\pgfsetbuttcap%
\pgfsetroundjoin%
\definecolor{currentfill}{rgb}{0.000000,0.000000,0.000000}%
\pgfsetfillcolor{currentfill}%
\pgfsetlinewidth{0.803000pt}%
\definecolor{currentstroke}{rgb}{0.000000,0.000000,0.000000}%
\pgfsetstrokecolor{currentstroke}%
\pgfsetdash{}{0pt}%
\pgfsys@defobject{currentmarker}{\pgfqpoint{-0.048611in}{0.000000in}}{\pgfqpoint{0.000000in}{0.000000in}}{%
\pgfpathmoveto{\pgfqpoint{0.000000in}{0.000000in}}%
\pgfpathlineto{\pgfqpoint{-0.048611in}{0.000000in}}%
\pgfusepath{stroke,fill}%
}%
\begin{pgfscope}%
\pgfsys@transformshift{0.787074in}{1.429536in}%
\pgfsys@useobject{currentmarker}{}%
\end{pgfscope}%
\end{pgfscope}%
\begin{pgfscope}%
\definecolor{textcolor}{rgb}{0.000000,0.000000,0.000000}%
\pgfsetstrokecolor{textcolor}%
\pgfsetfillcolor{textcolor}%
\pgftext[x=0.412073in, y=1.381342in, left, base]{\color{textcolor}\sffamily\fontsize{10.000000}{12.000000}\selectfont \(\displaystyle {5000}\)}%
\end{pgfscope}%
\begin{pgfscope}%
\pgfsetbuttcap%
\pgfsetroundjoin%
\definecolor{currentfill}{rgb}{0.000000,0.000000,0.000000}%
\pgfsetfillcolor{currentfill}%
\pgfsetlinewidth{0.803000pt}%
\definecolor{currentstroke}{rgb}{0.000000,0.000000,0.000000}%
\pgfsetstrokecolor{currentstroke}%
\pgfsetdash{}{0pt}%
\pgfsys@defobject{currentmarker}{\pgfqpoint{-0.048611in}{0.000000in}}{\pgfqpoint{0.000000in}{0.000000in}}{%
\pgfpathmoveto{\pgfqpoint{0.000000in}{0.000000in}}%
\pgfpathlineto{\pgfqpoint{-0.048611in}{0.000000in}}%
\pgfusepath{stroke,fill}%
}%
\begin{pgfscope}%
\pgfsys@transformshift{0.787074in}{1.799407in}%
\pgfsys@useobject{currentmarker}{}%
\end{pgfscope}%
\end{pgfscope}%
\begin{pgfscope}%
\definecolor{textcolor}{rgb}{0.000000,0.000000,0.000000}%
\pgfsetstrokecolor{textcolor}%
\pgfsetfillcolor{textcolor}%
\pgftext[x=0.412073in, y=1.751213in, left, base]{\color{textcolor}\sffamily\fontsize{10.000000}{12.000000}\selectfont \(\displaystyle {7500}\)}%
\end{pgfscope}%
\begin{pgfscope}%
\pgfsetbuttcap%
\pgfsetroundjoin%
\definecolor{currentfill}{rgb}{0.000000,0.000000,0.000000}%
\pgfsetfillcolor{currentfill}%
\pgfsetlinewidth{0.803000pt}%
\definecolor{currentstroke}{rgb}{0.000000,0.000000,0.000000}%
\pgfsetstrokecolor{currentstroke}%
\pgfsetdash{}{0pt}%
\pgfsys@defobject{currentmarker}{\pgfqpoint{-0.048611in}{0.000000in}}{\pgfqpoint{0.000000in}{0.000000in}}{%
\pgfpathmoveto{\pgfqpoint{0.000000in}{0.000000in}}%
\pgfpathlineto{\pgfqpoint{-0.048611in}{0.000000in}}%
\pgfusepath{stroke,fill}%
}%
\begin{pgfscope}%
\pgfsys@transformshift{0.787074in}{2.169278in}%
\pgfsys@useobject{currentmarker}{}%
\end{pgfscope}%
\end{pgfscope}%
\begin{pgfscope}%
\definecolor{textcolor}{rgb}{0.000000,0.000000,0.000000}%
\pgfsetstrokecolor{textcolor}%
\pgfsetfillcolor{textcolor}%
\pgftext[x=0.342628in, y=2.121083in, left, base]{\color{textcolor}\sffamily\fontsize{10.000000}{12.000000}\selectfont \(\displaystyle {10000}\)}%
\end{pgfscope}%
\begin{pgfscope}%
\pgfsetbuttcap%
\pgfsetroundjoin%
\definecolor{currentfill}{rgb}{0.000000,0.000000,0.000000}%
\pgfsetfillcolor{currentfill}%
\pgfsetlinewidth{0.803000pt}%
\definecolor{currentstroke}{rgb}{0.000000,0.000000,0.000000}%
\pgfsetstrokecolor{currentstroke}%
\pgfsetdash{}{0pt}%
\pgfsys@defobject{currentmarker}{\pgfqpoint{-0.048611in}{0.000000in}}{\pgfqpoint{0.000000in}{0.000000in}}{%
\pgfpathmoveto{\pgfqpoint{0.000000in}{0.000000in}}%
\pgfpathlineto{\pgfqpoint{-0.048611in}{0.000000in}}%
\pgfusepath{stroke,fill}%
}%
\begin{pgfscope}%
\pgfsys@transformshift{0.787074in}{2.539148in}%
\pgfsys@useobject{currentmarker}{}%
\end{pgfscope}%
\end{pgfscope}%
\begin{pgfscope}%
\definecolor{textcolor}{rgb}{0.000000,0.000000,0.000000}%
\pgfsetstrokecolor{textcolor}%
\pgfsetfillcolor{textcolor}%
\pgftext[x=0.342628in, y=2.490954in, left, base]{\color{textcolor}\sffamily\fontsize{10.000000}{12.000000}\selectfont \(\displaystyle {12500}\)}%
\end{pgfscope}%
\begin{pgfscope}%
\pgfsetbuttcap%
\pgfsetroundjoin%
\definecolor{currentfill}{rgb}{0.000000,0.000000,0.000000}%
\pgfsetfillcolor{currentfill}%
\pgfsetlinewidth{0.803000pt}%
\definecolor{currentstroke}{rgb}{0.000000,0.000000,0.000000}%
\pgfsetstrokecolor{currentstroke}%
\pgfsetdash{}{0pt}%
\pgfsys@defobject{currentmarker}{\pgfqpoint{-0.048611in}{0.000000in}}{\pgfqpoint{0.000000in}{0.000000in}}{%
\pgfpathmoveto{\pgfqpoint{0.000000in}{0.000000in}}%
\pgfpathlineto{\pgfqpoint{-0.048611in}{0.000000in}}%
\pgfusepath{stroke,fill}%
}%
\begin{pgfscope}%
\pgfsys@transformshift{0.787074in}{2.909019in}%
\pgfsys@useobject{currentmarker}{}%
\end{pgfscope}%
\end{pgfscope}%
\begin{pgfscope}%
\definecolor{textcolor}{rgb}{0.000000,0.000000,0.000000}%
\pgfsetstrokecolor{textcolor}%
\pgfsetfillcolor{textcolor}%
\pgftext[x=0.342628in, y=2.860824in, left, base]{\color{textcolor}\sffamily\fontsize{10.000000}{12.000000}\selectfont \(\displaystyle {15000}\)}%
\end{pgfscope}%
\begin{pgfscope}%
\pgfsetbuttcap%
\pgfsetroundjoin%
\definecolor{currentfill}{rgb}{0.000000,0.000000,0.000000}%
\pgfsetfillcolor{currentfill}%
\pgfsetlinewidth{0.803000pt}%
\definecolor{currentstroke}{rgb}{0.000000,0.000000,0.000000}%
\pgfsetstrokecolor{currentstroke}%
\pgfsetdash{}{0pt}%
\pgfsys@defobject{currentmarker}{\pgfqpoint{-0.048611in}{0.000000in}}{\pgfqpoint{0.000000in}{0.000000in}}{%
\pgfpathmoveto{\pgfqpoint{0.000000in}{0.000000in}}%
\pgfpathlineto{\pgfqpoint{-0.048611in}{0.000000in}}%
\pgfusepath{stroke,fill}%
}%
\begin{pgfscope}%
\pgfsys@transformshift{0.787074in}{3.278889in}%
\pgfsys@useobject{currentmarker}{}%
\end{pgfscope}%
\end{pgfscope}%
\begin{pgfscope}%
\definecolor{textcolor}{rgb}{0.000000,0.000000,0.000000}%
\pgfsetstrokecolor{textcolor}%
\pgfsetfillcolor{textcolor}%
\pgftext[x=0.342628in, y=3.230695in, left, base]{\color{textcolor}\sffamily\fontsize{10.000000}{12.000000}\selectfont \(\displaystyle {17500}\)}%
\end{pgfscope}%
\begin{pgfscope}%
\pgfsetbuttcap%
\pgfsetroundjoin%
\definecolor{currentfill}{rgb}{0.000000,0.000000,0.000000}%
\pgfsetfillcolor{currentfill}%
\pgfsetlinewidth{0.803000pt}%
\definecolor{currentstroke}{rgb}{0.000000,0.000000,0.000000}%
\pgfsetstrokecolor{currentstroke}%
\pgfsetdash{}{0pt}%
\pgfsys@defobject{currentmarker}{\pgfqpoint{-0.048611in}{0.000000in}}{\pgfqpoint{0.000000in}{0.000000in}}{%
\pgfpathmoveto{\pgfqpoint{0.000000in}{0.000000in}}%
\pgfpathlineto{\pgfqpoint{-0.048611in}{0.000000in}}%
\pgfusepath{stroke,fill}%
}%
\begin{pgfscope}%
\pgfsys@transformshift{0.787074in}{3.648760in}%
\pgfsys@useobject{currentmarker}{}%
\end{pgfscope}%
\end{pgfscope}%
\begin{pgfscope}%
\definecolor{textcolor}{rgb}{0.000000,0.000000,0.000000}%
\pgfsetstrokecolor{textcolor}%
\pgfsetfillcolor{textcolor}%
\pgftext[x=0.342628in, y=3.600565in, left, base]{\color{textcolor}\sffamily\fontsize{10.000000}{12.000000}\selectfont \(\displaystyle {20000}\)}%
\end{pgfscope}%
\begin{pgfscope}%
\definecolor{textcolor}{rgb}{0.000000,0.000000,0.000000}%
\pgfsetstrokecolor{textcolor}%
\pgfsetfillcolor{textcolor}%
\pgftext[x=0.287073in,y=2.100064in,,bottom,rotate=90.000000]{\color{textcolor}\sffamily\fontsize{10.000000}{12.000000}\selectfont Maximum Memory Usage (MB)}%
\end{pgfscope}%
\begin{pgfscope}%
\pgfsetrectcap%
\pgfsetmiterjoin%
\pgfsetlinewidth{0.803000pt}%
\definecolor{currentstroke}{rgb}{0.000000,0.000000,0.000000}%
\pgfsetstrokecolor{currentstroke}%
\pgfsetdash{}{0pt}%
\pgfpathmoveto{\pgfqpoint{0.787074in}{0.548769in}}%
\pgfpathlineto{\pgfqpoint{0.787074in}{3.651359in}}%
\pgfusepath{stroke}%
\end{pgfscope}%
\begin{pgfscope}%
\pgfsetrectcap%
\pgfsetmiterjoin%
\pgfsetlinewidth{0.803000pt}%
\definecolor{currentstroke}{rgb}{0.000000,0.000000,0.000000}%
\pgfsetstrokecolor{currentstroke}%
\pgfsetdash{}{0pt}%
\pgfpathmoveto{\pgfqpoint{5.850000in}{0.548769in}}%
\pgfpathlineto{\pgfqpoint{5.850000in}{3.651359in}}%
\pgfusepath{stroke}%
\end{pgfscope}%
\begin{pgfscope}%
\pgfsetrectcap%
\pgfsetmiterjoin%
\pgfsetlinewidth{0.803000pt}%
\definecolor{currentstroke}{rgb}{0.000000,0.000000,0.000000}%
\pgfsetstrokecolor{currentstroke}%
\pgfsetdash{}{0pt}%
\pgfpathmoveto{\pgfqpoint{0.787074in}{0.548769in}}%
\pgfpathlineto{\pgfqpoint{5.850000in}{0.548769in}}%
\pgfusepath{stroke}%
\end{pgfscope}%
\begin{pgfscope}%
\pgfsetrectcap%
\pgfsetmiterjoin%
\pgfsetlinewidth{0.803000pt}%
\definecolor{currentstroke}{rgb}{0.000000,0.000000,0.000000}%
\pgfsetstrokecolor{currentstroke}%
\pgfsetdash{}{0pt}%
\pgfpathmoveto{\pgfqpoint{0.787074in}{3.651359in}}%
\pgfpathlineto{\pgfqpoint{5.850000in}{3.651359in}}%
\pgfusepath{stroke}%
\end{pgfscope}%
\begin{pgfscope}%
\definecolor{textcolor}{rgb}{0.000000,0.000000,0.000000}%
\pgfsetstrokecolor{textcolor}%
\pgfsetfillcolor{textcolor}%
\pgftext[x=3.318537in,y=3.734692in,,base]{\color{textcolor}\sffamily\fontsize{12.000000}{14.400000}\selectfont Forwards}%
\end{pgfscope}%
\begin{pgfscope}%
\pgfsetbuttcap%
\pgfsetmiterjoin%
\definecolor{currentfill}{rgb}{1.000000,1.000000,1.000000}%
\pgfsetfillcolor{currentfill}%
\pgfsetfillopacity{0.800000}%
\pgfsetlinewidth{1.003750pt}%
\definecolor{currentstroke}{rgb}{0.800000,0.800000,0.800000}%
\pgfsetstrokecolor{currentstroke}%
\pgfsetstrokeopacity{0.800000}%
\pgfsetdash{}{0pt}%
\pgfpathmoveto{\pgfqpoint{4.300417in}{2.957886in}}%
\pgfpathlineto{\pgfqpoint{5.752778in}{2.957886in}}%
\pgfpathquadraticcurveto{\pgfqpoint{5.780556in}{2.957886in}}{\pgfqpoint{5.780556in}{2.985664in}}%
\pgfpathlineto{\pgfqpoint{5.780556in}{3.554136in}}%
\pgfpathquadraticcurveto{\pgfqpoint{5.780556in}{3.581914in}}{\pgfqpoint{5.752778in}{3.581914in}}%
\pgfpathlineto{\pgfqpoint{4.300417in}{3.581914in}}%
\pgfpathquadraticcurveto{\pgfqpoint{4.272639in}{3.581914in}}{\pgfqpoint{4.272639in}{3.554136in}}%
\pgfpathlineto{\pgfqpoint{4.272639in}{2.985664in}}%
\pgfpathquadraticcurveto{\pgfqpoint{4.272639in}{2.957886in}}{\pgfqpoint{4.300417in}{2.957886in}}%
\pgfpathclose%
\pgfusepath{stroke,fill}%
\end{pgfscope}%
\begin{pgfscope}%
\pgfsetbuttcap%
\pgfsetroundjoin%
\definecolor{currentfill}{rgb}{0.121569,0.466667,0.705882}%
\pgfsetfillcolor{currentfill}%
\pgfsetlinewidth{1.003750pt}%
\definecolor{currentstroke}{rgb}{0.121569,0.466667,0.705882}%
\pgfsetstrokecolor{currentstroke}%
\pgfsetdash{}{0pt}%
\pgfsys@defobject{currentmarker}{\pgfqpoint{-0.034722in}{-0.034722in}}{\pgfqpoint{0.034722in}{0.034722in}}{%
\pgfpathmoveto{\pgfqpoint{0.000000in}{-0.034722in}}%
\pgfpathcurveto{\pgfqpoint{0.009208in}{-0.034722in}}{\pgfqpoint{0.018041in}{-0.031064in}}{\pgfqpoint{0.024552in}{-0.024552in}}%
\pgfpathcurveto{\pgfqpoint{0.031064in}{-0.018041in}}{\pgfqpoint{0.034722in}{-0.009208in}}{\pgfqpoint{0.034722in}{0.000000in}}%
\pgfpathcurveto{\pgfqpoint{0.034722in}{0.009208in}}{\pgfqpoint{0.031064in}{0.018041in}}{\pgfqpoint{0.024552in}{0.024552in}}%
\pgfpathcurveto{\pgfqpoint{0.018041in}{0.031064in}}{\pgfqpoint{0.009208in}{0.034722in}}{\pgfqpoint{0.000000in}{0.034722in}}%
\pgfpathcurveto{\pgfqpoint{-0.009208in}{0.034722in}}{\pgfqpoint{-0.018041in}{0.031064in}}{\pgfqpoint{-0.024552in}{0.024552in}}%
\pgfpathcurveto{\pgfqpoint{-0.031064in}{0.018041in}}{\pgfqpoint{-0.034722in}{0.009208in}}{\pgfqpoint{-0.034722in}{0.000000in}}%
\pgfpathcurveto{\pgfqpoint{-0.034722in}{-0.009208in}}{\pgfqpoint{-0.031064in}{-0.018041in}}{\pgfqpoint{-0.024552in}{-0.024552in}}%
\pgfpathcurveto{\pgfqpoint{-0.018041in}{-0.031064in}}{\pgfqpoint{-0.009208in}{-0.034722in}}{\pgfqpoint{0.000000in}{-0.034722in}}%
\pgfpathclose%
\pgfusepath{stroke,fill}%
}%
\begin{pgfscope}%
\pgfsys@transformshift{4.467083in}{3.477748in}%
\pgfsys@useobject{currentmarker}{}%
\end{pgfscope}%
\end{pgfscope}%
\begin{pgfscope}%
\definecolor{textcolor}{rgb}{0.000000,0.000000,0.000000}%
\pgfsetstrokecolor{textcolor}%
\pgfsetfillcolor{textcolor}%
\pgftext[x=4.717083in,y=3.429136in,left,base]{\color{textcolor}\sffamily\fontsize{10.000000}{12.000000}\selectfont No Timeout}%
\end{pgfscope}%
\begin{pgfscope}%
\pgfsetbuttcap%
\pgfsetroundjoin%
\definecolor{currentfill}{rgb}{1.000000,0.498039,0.054902}%
\pgfsetfillcolor{currentfill}%
\pgfsetlinewidth{1.003750pt}%
\definecolor{currentstroke}{rgb}{1.000000,0.498039,0.054902}%
\pgfsetstrokecolor{currentstroke}%
\pgfsetdash{}{0pt}%
\pgfsys@defobject{currentmarker}{\pgfqpoint{-0.034722in}{-0.034722in}}{\pgfqpoint{0.034722in}{0.034722in}}{%
\pgfpathmoveto{\pgfqpoint{0.000000in}{-0.034722in}}%
\pgfpathcurveto{\pgfqpoint{0.009208in}{-0.034722in}}{\pgfqpoint{0.018041in}{-0.031064in}}{\pgfqpoint{0.024552in}{-0.024552in}}%
\pgfpathcurveto{\pgfqpoint{0.031064in}{-0.018041in}}{\pgfqpoint{0.034722in}{-0.009208in}}{\pgfqpoint{0.034722in}{0.000000in}}%
\pgfpathcurveto{\pgfqpoint{0.034722in}{0.009208in}}{\pgfqpoint{0.031064in}{0.018041in}}{\pgfqpoint{0.024552in}{0.024552in}}%
\pgfpathcurveto{\pgfqpoint{0.018041in}{0.031064in}}{\pgfqpoint{0.009208in}{0.034722in}}{\pgfqpoint{0.000000in}{0.034722in}}%
\pgfpathcurveto{\pgfqpoint{-0.009208in}{0.034722in}}{\pgfqpoint{-0.018041in}{0.031064in}}{\pgfqpoint{-0.024552in}{0.024552in}}%
\pgfpathcurveto{\pgfqpoint{-0.031064in}{0.018041in}}{\pgfqpoint{-0.034722in}{0.009208in}}{\pgfqpoint{-0.034722in}{0.000000in}}%
\pgfpathcurveto{\pgfqpoint{-0.034722in}{-0.009208in}}{\pgfqpoint{-0.031064in}{-0.018041in}}{\pgfqpoint{-0.024552in}{-0.024552in}}%
\pgfpathcurveto{\pgfqpoint{-0.018041in}{-0.031064in}}{\pgfqpoint{-0.009208in}{-0.034722in}}{\pgfqpoint{0.000000in}{-0.034722in}}%
\pgfpathclose%
\pgfusepath{stroke,fill}%
}%
\begin{pgfscope}%
\pgfsys@transformshift{4.467083in}{3.284136in}%
\pgfsys@useobject{currentmarker}{}%
\end{pgfscope}%
\end{pgfscope}%
\begin{pgfscope}%
\definecolor{textcolor}{rgb}{0.000000,0.000000,0.000000}%
\pgfsetstrokecolor{textcolor}%
\pgfsetfillcolor{textcolor}%
\pgftext[x=4.717083in,y=3.235525in,left,base]{\color{textcolor}\sffamily\fontsize{10.000000}{12.000000}\selectfont Time Timeout}%
\end{pgfscope}%
\begin{pgfscope}%
\pgfsetbuttcap%
\pgfsetroundjoin%
\definecolor{currentfill}{rgb}{0.839216,0.152941,0.156863}%
\pgfsetfillcolor{currentfill}%
\pgfsetlinewidth{1.003750pt}%
\definecolor{currentstroke}{rgb}{0.839216,0.152941,0.156863}%
\pgfsetstrokecolor{currentstroke}%
\pgfsetdash{}{0pt}%
\pgfsys@defobject{currentmarker}{\pgfqpoint{-0.034722in}{-0.034722in}}{\pgfqpoint{0.034722in}{0.034722in}}{%
\pgfpathmoveto{\pgfqpoint{0.000000in}{-0.034722in}}%
\pgfpathcurveto{\pgfqpoint{0.009208in}{-0.034722in}}{\pgfqpoint{0.018041in}{-0.031064in}}{\pgfqpoint{0.024552in}{-0.024552in}}%
\pgfpathcurveto{\pgfqpoint{0.031064in}{-0.018041in}}{\pgfqpoint{0.034722in}{-0.009208in}}{\pgfqpoint{0.034722in}{0.000000in}}%
\pgfpathcurveto{\pgfqpoint{0.034722in}{0.009208in}}{\pgfqpoint{0.031064in}{0.018041in}}{\pgfqpoint{0.024552in}{0.024552in}}%
\pgfpathcurveto{\pgfqpoint{0.018041in}{0.031064in}}{\pgfqpoint{0.009208in}{0.034722in}}{\pgfqpoint{0.000000in}{0.034722in}}%
\pgfpathcurveto{\pgfqpoint{-0.009208in}{0.034722in}}{\pgfqpoint{-0.018041in}{0.031064in}}{\pgfqpoint{-0.024552in}{0.024552in}}%
\pgfpathcurveto{\pgfqpoint{-0.031064in}{0.018041in}}{\pgfqpoint{-0.034722in}{0.009208in}}{\pgfqpoint{-0.034722in}{0.000000in}}%
\pgfpathcurveto{\pgfqpoint{-0.034722in}{-0.009208in}}{\pgfqpoint{-0.031064in}{-0.018041in}}{\pgfqpoint{-0.024552in}{-0.024552in}}%
\pgfpathcurveto{\pgfqpoint{-0.018041in}{-0.031064in}}{\pgfqpoint{-0.009208in}{-0.034722in}}{\pgfqpoint{0.000000in}{-0.034722in}}%
\pgfpathclose%
\pgfusepath{stroke,fill}%
}%
\begin{pgfscope}%
\pgfsys@transformshift{4.467083in}{3.090525in}%
\pgfsys@useobject{currentmarker}{}%
\end{pgfscope}%
\end{pgfscope}%
\begin{pgfscope}%
\definecolor{textcolor}{rgb}{0.000000,0.000000,0.000000}%
\pgfsetstrokecolor{textcolor}%
\pgfsetfillcolor{textcolor}%
\pgftext[x=4.717083in,y=3.041914in,left,base]{\color{textcolor}\sffamily\fontsize{10.000000}{12.000000}\selectfont Memory Timeout}%
\end{pgfscope}%
\end{pgfpicture}%
\makeatother%
\endgroup%

                }
            \end{subfigure}
            \qquad
            \begin{subfigure}[]{0.45\textwidth}
                \centering
                \resizebox{\columnwidth}{!}{
                    %% Creator: Matplotlib, PGF backend
%%
%% To include the figure in your LaTeX document, write
%%   \input{<filename>.pgf}
%%
%% Make sure the required packages are loaded in your preamble
%%   \usepackage{pgf}
%%
%% and, on pdftex
%%   \usepackage[utf8]{inputenc}\DeclareUnicodeCharacter{2212}{-}
%%
%% or, on luatex and xetex
%%   \usepackage{unicode-math}
%%
%% Figures using additional raster images can only be included by \input if
%% they are in the same directory as the main LaTeX file. For loading figures
%% from other directories you can use the `import` package
%%   \usepackage{import}
%%
%% and then include the figures with
%%   \import{<path to file>}{<filename>.pgf}
%%
%% Matplotlib used the following preamble
%%   \usepackage{amsmath}
%%   \usepackage{fontspec}
%%
\begingroup%
\makeatletter%
\begin{pgfpicture}%
\pgfpathrectangle{\pgfpointorigin}{\pgfqpoint{6.000000in}{4.000000in}}%
\pgfusepath{use as bounding box, clip}%
\begin{pgfscope}%
\pgfsetbuttcap%
\pgfsetmiterjoin%
\definecolor{currentfill}{rgb}{1.000000,1.000000,1.000000}%
\pgfsetfillcolor{currentfill}%
\pgfsetlinewidth{0.000000pt}%
\definecolor{currentstroke}{rgb}{1.000000,1.000000,1.000000}%
\pgfsetstrokecolor{currentstroke}%
\pgfsetdash{}{0pt}%
\pgfpathmoveto{\pgfqpoint{0.000000in}{0.000000in}}%
\pgfpathlineto{\pgfqpoint{6.000000in}{0.000000in}}%
\pgfpathlineto{\pgfqpoint{6.000000in}{4.000000in}}%
\pgfpathlineto{\pgfqpoint{0.000000in}{4.000000in}}%
\pgfpathclose%
\pgfusepath{fill}%
\end{pgfscope}%
\begin{pgfscope}%
\pgfsetbuttcap%
\pgfsetmiterjoin%
\definecolor{currentfill}{rgb}{1.000000,1.000000,1.000000}%
\pgfsetfillcolor{currentfill}%
\pgfsetlinewidth{0.000000pt}%
\definecolor{currentstroke}{rgb}{0.000000,0.000000,0.000000}%
\pgfsetstrokecolor{currentstroke}%
\pgfsetstrokeopacity{0.000000}%
\pgfsetdash{}{0pt}%
\pgfpathmoveto{\pgfqpoint{0.787074in}{0.548769in}}%
\pgfpathlineto{\pgfqpoint{5.850000in}{0.548769in}}%
\pgfpathlineto{\pgfqpoint{5.850000in}{3.651359in}}%
\pgfpathlineto{\pgfqpoint{0.787074in}{3.651359in}}%
\pgfpathclose%
\pgfusepath{fill}%
\end{pgfscope}%
\begin{pgfscope}%
\pgfpathrectangle{\pgfqpoint{0.787074in}{0.548769in}}{\pgfqpoint{5.062926in}{3.102590in}}%
\pgfusepath{clip}%
\pgfsetbuttcap%
\pgfsetroundjoin%
\definecolor{currentfill}{rgb}{0.121569,0.466667,0.705882}%
\pgfsetfillcolor{currentfill}%
\pgfsetlinewidth{1.003750pt}%
\definecolor{currentstroke}{rgb}{0.121569,0.466667,0.705882}%
\pgfsetstrokecolor{currentstroke}%
\pgfsetdash{}{0pt}%
\pgfpathmoveto{\pgfqpoint{1.338854in}{0.676730in}}%
\pgfpathcurveto{\pgfqpoint{1.349904in}{0.676730in}}{\pgfqpoint{1.360503in}{0.681121in}}{\pgfqpoint{1.368316in}{0.688934in}}%
\pgfpathcurveto{\pgfqpoint{1.376130in}{0.696748in}}{\pgfqpoint{1.380520in}{0.707347in}}{\pgfqpoint{1.380520in}{0.718397in}}%
\pgfpathcurveto{\pgfqpoint{1.380520in}{0.729447in}}{\pgfqpoint{1.376130in}{0.740046in}}{\pgfqpoint{1.368316in}{0.747860in}}%
\pgfpathcurveto{\pgfqpoint{1.360503in}{0.755673in}}{\pgfqpoint{1.349904in}{0.760064in}}{\pgfqpoint{1.338854in}{0.760064in}}%
\pgfpathcurveto{\pgfqpoint{1.327803in}{0.760064in}}{\pgfqpoint{1.317204in}{0.755673in}}{\pgfqpoint{1.309391in}{0.747860in}}%
\pgfpathcurveto{\pgfqpoint{1.301577in}{0.740046in}}{\pgfqpoint{1.297187in}{0.729447in}}{\pgfqpoint{1.297187in}{0.718397in}}%
\pgfpathcurveto{\pgfqpoint{1.297187in}{0.707347in}}{\pgfqpoint{1.301577in}{0.696748in}}{\pgfqpoint{1.309391in}{0.688934in}}%
\pgfpathcurveto{\pgfqpoint{1.317204in}{0.681121in}}{\pgfqpoint{1.327803in}{0.676730in}}{\pgfqpoint{1.338854in}{0.676730in}}%
\pgfpathclose%
\pgfusepath{stroke,fill}%
\end{pgfscope}%
\begin{pgfscope}%
\pgfpathrectangle{\pgfqpoint{0.787074in}{0.548769in}}{\pgfqpoint{5.062926in}{3.102590in}}%
\pgfusepath{clip}%
\pgfsetbuttcap%
\pgfsetroundjoin%
\definecolor{currentfill}{rgb}{1.000000,0.498039,0.054902}%
\pgfsetfillcolor{currentfill}%
\pgfsetlinewidth{1.003750pt}%
\definecolor{currentstroke}{rgb}{1.000000,0.498039,0.054902}%
\pgfsetstrokecolor{currentstroke}%
\pgfsetdash{}{0pt}%
\pgfpathmoveto{\pgfqpoint{3.946792in}{1.501451in}}%
\pgfpathcurveto{\pgfqpoint{3.957842in}{1.501451in}}{\pgfqpoint{3.968441in}{1.505841in}}{\pgfqpoint{3.976255in}{1.513655in}}%
\pgfpathcurveto{\pgfqpoint{3.984068in}{1.521468in}}{\pgfqpoint{3.988459in}{1.532067in}}{\pgfqpoint{3.988459in}{1.543118in}}%
\pgfpathcurveto{\pgfqpoint{3.988459in}{1.554168in}}{\pgfqpoint{3.984068in}{1.564767in}}{\pgfqpoint{3.976255in}{1.572580in}}%
\pgfpathcurveto{\pgfqpoint{3.968441in}{1.580394in}}{\pgfqpoint{3.957842in}{1.584784in}}{\pgfqpoint{3.946792in}{1.584784in}}%
\pgfpathcurveto{\pgfqpoint{3.935742in}{1.584784in}}{\pgfqpoint{3.925143in}{1.580394in}}{\pgfqpoint{3.917329in}{1.572580in}}%
\pgfpathcurveto{\pgfqpoint{3.909515in}{1.564767in}}{\pgfqpoint{3.905125in}{1.554168in}}{\pgfqpoint{3.905125in}{1.543118in}}%
\pgfpathcurveto{\pgfqpoint{3.905125in}{1.532067in}}{\pgfqpoint{3.909515in}{1.521468in}}{\pgfqpoint{3.917329in}{1.513655in}}%
\pgfpathcurveto{\pgfqpoint{3.925143in}{1.505841in}}{\pgfqpoint{3.935742in}{1.501451in}}{\pgfqpoint{3.946792in}{1.501451in}}%
\pgfpathclose%
\pgfusepath{stroke,fill}%
\end{pgfscope}%
\begin{pgfscope}%
\pgfpathrectangle{\pgfqpoint{0.787074in}{0.548769in}}{\pgfqpoint{5.062926in}{3.102590in}}%
\pgfusepath{clip}%
\pgfsetbuttcap%
\pgfsetroundjoin%
\definecolor{currentfill}{rgb}{1.000000,0.498039,0.054902}%
\pgfsetfillcolor{currentfill}%
\pgfsetlinewidth{1.003750pt}%
\definecolor{currentstroke}{rgb}{1.000000,0.498039,0.054902}%
\pgfsetstrokecolor{currentstroke}%
\pgfsetdash{}{0pt}%
\pgfpathmoveto{\pgfqpoint{1.347143in}{2.589003in}}%
\pgfpathcurveto{\pgfqpoint{1.358193in}{2.589003in}}{\pgfqpoint{1.368792in}{2.593393in}}{\pgfqpoint{1.376606in}{2.601207in}}%
\pgfpathcurveto{\pgfqpoint{1.384420in}{2.609020in}}{\pgfqpoint{1.388810in}{2.619619in}}{\pgfqpoint{1.388810in}{2.630670in}}%
\pgfpathcurveto{\pgfqpoint{1.388810in}{2.641720in}}{\pgfqpoint{1.384420in}{2.652319in}}{\pgfqpoint{1.376606in}{2.660132in}}%
\pgfpathcurveto{\pgfqpoint{1.368792in}{2.667946in}}{\pgfqpoint{1.358193in}{2.672336in}}{\pgfqpoint{1.347143in}{2.672336in}}%
\pgfpathcurveto{\pgfqpoint{1.336093in}{2.672336in}}{\pgfqpoint{1.325494in}{2.667946in}}{\pgfqpoint{1.317680in}{2.660132in}}%
\pgfpathcurveto{\pgfqpoint{1.309867in}{2.652319in}}{\pgfqpoint{1.305477in}{2.641720in}}{\pgfqpoint{1.305477in}{2.630670in}}%
\pgfpathcurveto{\pgfqpoint{1.305477in}{2.619619in}}{\pgfqpoint{1.309867in}{2.609020in}}{\pgfqpoint{1.317680in}{2.601207in}}%
\pgfpathcurveto{\pgfqpoint{1.325494in}{2.593393in}}{\pgfqpoint{1.336093in}{2.589003in}}{\pgfqpoint{1.347143in}{2.589003in}}%
\pgfpathclose%
\pgfusepath{stroke,fill}%
\end{pgfscope}%
\begin{pgfscope}%
\pgfpathrectangle{\pgfqpoint{0.787074in}{0.548769in}}{\pgfqpoint{5.062926in}{3.102590in}}%
\pgfusepath{clip}%
\pgfsetbuttcap%
\pgfsetroundjoin%
\definecolor{currentfill}{rgb}{1.000000,0.498039,0.054902}%
\pgfsetfillcolor{currentfill}%
\pgfsetlinewidth{1.003750pt}%
\definecolor{currentstroke}{rgb}{1.000000,0.498039,0.054902}%
\pgfsetstrokecolor{currentstroke}%
\pgfsetdash{}{0pt}%
\pgfpathmoveto{\pgfqpoint{2.585121in}{1.800918in}}%
\pgfpathcurveto{\pgfqpoint{2.596171in}{1.800918in}}{\pgfqpoint{2.606770in}{1.805309in}}{\pgfqpoint{2.614584in}{1.813122in}}%
\pgfpathcurveto{\pgfqpoint{2.622397in}{1.820936in}}{\pgfqpoint{2.626787in}{1.831535in}}{\pgfqpoint{2.626787in}{1.842585in}}%
\pgfpathcurveto{\pgfqpoint{2.626787in}{1.853635in}}{\pgfqpoint{2.622397in}{1.864234in}}{\pgfqpoint{2.614584in}{1.872048in}}%
\pgfpathcurveto{\pgfqpoint{2.606770in}{1.879862in}}{\pgfqpoint{2.596171in}{1.884252in}}{\pgfqpoint{2.585121in}{1.884252in}}%
\pgfpathcurveto{\pgfqpoint{2.574071in}{1.884252in}}{\pgfqpoint{2.563472in}{1.879862in}}{\pgfqpoint{2.555658in}{1.872048in}}%
\pgfpathcurveto{\pgfqpoint{2.547844in}{1.864234in}}{\pgfqpoint{2.543454in}{1.853635in}}{\pgfqpoint{2.543454in}{1.842585in}}%
\pgfpathcurveto{\pgfqpoint{2.543454in}{1.831535in}}{\pgfqpoint{2.547844in}{1.820936in}}{\pgfqpoint{2.555658in}{1.813122in}}%
\pgfpathcurveto{\pgfqpoint{2.563472in}{1.805309in}}{\pgfqpoint{2.574071in}{1.800918in}}{\pgfqpoint{2.585121in}{1.800918in}}%
\pgfpathclose%
\pgfusepath{stroke,fill}%
\end{pgfscope}%
\begin{pgfscope}%
\pgfpathrectangle{\pgfqpoint{0.787074in}{0.548769in}}{\pgfqpoint{5.062926in}{3.102590in}}%
\pgfusepath{clip}%
\pgfsetbuttcap%
\pgfsetroundjoin%
\definecolor{currentfill}{rgb}{0.121569,0.466667,0.705882}%
\pgfsetfillcolor{currentfill}%
\pgfsetlinewidth{1.003750pt}%
\definecolor{currentstroke}{rgb}{0.121569,0.466667,0.705882}%
\pgfsetstrokecolor{currentstroke}%
\pgfsetdash{}{0pt}%
\pgfpathmoveto{\pgfqpoint{2.729604in}{0.650081in}}%
\pgfpathcurveto{\pgfqpoint{2.740654in}{0.650081in}}{\pgfqpoint{2.751253in}{0.654472in}}{\pgfqpoint{2.759066in}{0.662285in}}%
\pgfpathcurveto{\pgfqpoint{2.766880in}{0.670099in}}{\pgfqpoint{2.771270in}{0.680698in}}{\pgfqpoint{2.771270in}{0.691748in}}%
\pgfpathcurveto{\pgfqpoint{2.771270in}{0.702798in}}{\pgfqpoint{2.766880in}{0.713397in}}{\pgfqpoint{2.759066in}{0.721211in}}%
\pgfpathcurveto{\pgfqpoint{2.751253in}{0.729024in}}{\pgfqpoint{2.740654in}{0.733415in}}{\pgfqpoint{2.729604in}{0.733415in}}%
\pgfpathcurveto{\pgfqpoint{2.718553in}{0.733415in}}{\pgfqpoint{2.707954in}{0.729024in}}{\pgfqpoint{2.700141in}{0.721211in}}%
\pgfpathcurveto{\pgfqpoint{2.692327in}{0.713397in}}{\pgfqpoint{2.687937in}{0.702798in}}{\pgfqpoint{2.687937in}{0.691748in}}%
\pgfpathcurveto{\pgfqpoint{2.687937in}{0.680698in}}{\pgfqpoint{2.692327in}{0.670099in}}{\pgfqpoint{2.700141in}{0.662285in}}%
\pgfpathcurveto{\pgfqpoint{2.707954in}{0.654472in}}{\pgfqpoint{2.718553in}{0.650081in}}{\pgfqpoint{2.729604in}{0.650081in}}%
\pgfpathclose%
\pgfusepath{stroke,fill}%
\end{pgfscope}%
\begin{pgfscope}%
\pgfpathrectangle{\pgfqpoint{0.787074in}{0.548769in}}{\pgfqpoint{5.062926in}{3.102590in}}%
\pgfusepath{clip}%
\pgfsetbuttcap%
\pgfsetroundjoin%
\definecolor{currentfill}{rgb}{1.000000,0.498039,0.054902}%
\pgfsetfillcolor{currentfill}%
\pgfsetlinewidth{1.003750pt}%
\definecolor{currentstroke}{rgb}{1.000000,0.498039,0.054902}%
\pgfsetstrokecolor{currentstroke}%
\pgfsetdash{}{0pt}%
\pgfpathmoveto{\pgfqpoint{2.210394in}{2.499787in}}%
\pgfpathcurveto{\pgfqpoint{2.221444in}{2.499787in}}{\pgfqpoint{2.232043in}{2.504177in}}{\pgfqpoint{2.239857in}{2.511991in}}%
\pgfpathcurveto{\pgfqpoint{2.247671in}{2.519804in}}{\pgfqpoint{2.252061in}{2.530403in}}{\pgfqpoint{2.252061in}{2.541454in}}%
\pgfpathcurveto{\pgfqpoint{2.252061in}{2.552504in}}{\pgfqpoint{2.247671in}{2.563103in}}{\pgfqpoint{2.239857in}{2.570916in}}%
\pgfpathcurveto{\pgfqpoint{2.232043in}{2.578730in}}{\pgfqpoint{2.221444in}{2.583120in}}{\pgfqpoint{2.210394in}{2.583120in}}%
\pgfpathcurveto{\pgfqpoint{2.199344in}{2.583120in}}{\pgfqpoint{2.188745in}{2.578730in}}{\pgfqpoint{2.180931in}{2.570916in}}%
\pgfpathcurveto{\pgfqpoint{2.173118in}{2.563103in}}{\pgfqpoint{2.168728in}{2.552504in}}{\pgfqpoint{2.168728in}{2.541454in}}%
\pgfpathcurveto{\pgfqpoint{2.168728in}{2.530403in}}{\pgfqpoint{2.173118in}{2.519804in}}{\pgfqpoint{2.180931in}{2.511991in}}%
\pgfpathcurveto{\pgfqpoint{2.188745in}{2.504177in}}{\pgfqpoint{2.199344in}{2.499787in}}{\pgfqpoint{2.210394in}{2.499787in}}%
\pgfpathclose%
\pgfusepath{stroke,fill}%
\end{pgfscope}%
\begin{pgfscope}%
\pgfpathrectangle{\pgfqpoint{0.787074in}{0.548769in}}{\pgfqpoint{5.062926in}{3.102590in}}%
\pgfusepath{clip}%
\pgfsetbuttcap%
\pgfsetroundjoin%
\definecolor{currentfill}{rgb}{1.000000,0.498039,0.054902}%
\pgfsetfillcolor{currentfill}%
\pgfsetlinewidth{1.003750pt}%
\definecolor{currentstroke}{rgb}{1.000000,0.498039,0.054902}%
\pgfsetstrokecolor{currentstroke}%
\pgfsetdash{}{0pt}%
\pgfpathmoveto{\pgfqpoint{2.349452in}{2.338512in}}%
\pgfpathcurveto{\pgfqpoint{2.360502in}{2.338512in}}{\pgfqpoint{2.371101in}{2.342903in}}{\pgfqpoint{2.378915in}{2.350716in}}%
\pgfpathcurveto{\pgfqpoint{2.386728in}{2.358530in}}{\pgfqpoint{2.391119in}{2.369129in}}{\pgfqpoint{2.391119in}{2.380179in}}%
\pgfpathcurveto{\pgfqpoint{2.391119in}{2.391229in}}{\pgfqpoint{2.386728in}{2.401828in}}{\pgfqpoint{2.378915in}{2.409642in}}%
\pgfpathcurveto{\pgfqpoint{2.371101in}{2.417455in}}{\pgfqpoint{2.360502in}{2.421846in}}{\pgfqpoint{2.349452in}{2.421846in}}%
\pgfpathcurveto{\pgfqpoint{2.338402in}{2.421846in}}{\pgfqpoint{2.327803in}{2.417455in}}{\pgfqpoint{2.319989in}{2.409642in}}%
\pgfpathcurveto{\pgfqpoint{2.312175in}{2.401828in}}{\pgfqpoint{2.307785in}{2.391229in}}{\pgfqpoint{2.307785in}{2.380179in}}%
\pgfpathcurveto{\pgfqpoint{2.307785in}{2.369129in}}{\pgfqpoint{2.312175in}{2.358530in}}{\pgfqpoint{2.319989in}{2.350716in}}%
\pgfpathcurveto{\pgfqpoint{2.327803in}{2.342903in}}{\pgfqpoint{2.338402in}{2.338512in}}{\pgfqpoint{2.349452in}{2.338512in}}%
\pgfpathclose%
\pgfusepath{stroke,fill}%
\end{pgfscope}%
\begin{pgfscope}%
\pgfpathrectangle{\pgfqpoint{0.787074in}{0.548769in}}{\pgfqpoint{5.062926in}{3.102590in}}%
\pgfusepath{clip}%
\pgfsetbuttcap%
\pgfsetroundjoin%
\definecolor{currentfill}{rgb}{0.121569,0.466667,0.705882}%
\pgfsetfillcolor{currentfill}%
\pgfsetlinewidth{1.003750pt}%
\definecolor{currentstroke}{rgb}{0.121569,0.466667,0.705882}%
\pgfsetstrokecolor{currentstroke}%
\pgfsetdash{}{0pt}%
\pgfpathmoveto{\pgfqpoint{1.461202in}{0.648162in}}%
\pgfpathcurveto{\pgfqpoint{1.472252in}{0.648162in}}{\pgfqpoint{1.482851in}{0.652552in}}{\pgfqpoint{1.490665in}{0.660365in}}%
\pgfpathcurveto{\pgfqpoint{1.498478in}{0.668179in}}{\pgfqpoint{1.502868in}{0.678778in}}{\pgfqpoint{1.502868in}{0.689828in}}%
\pgfpathcurveto{\pgfqpoint{1.502868in}{0.700878in}}{\pgfqpoint{1.498478in}{0.711477in}}{\pgfqpoint{1.490665in}{0.719291in}}%
\pgfpathcurveto{\pgfqpoint{1.482851in}{0.727105in}}{\pgfqpoint{1.472252in}{0.731495in}}{\pgfqpoint{1.461202in}{0.731495in}}%
\pgfpathcurveto{\pgfqpoint{1.450152in}{0.731495in}}{\pgfqpoint{1.439553in}{0.727105in}}{\pgfqpoint{1.431739in}{0.719291in}}%
\pgfpathcurveto{\pgfqpoint{1.423925in}{0.711477in}}{\pgfqpoint{1.419535in}{0.700878in}}{\pgfqpoint{1.419535in}{0.689828in}}%
\pgfpathcurveto{\pgfqpoint{1.419535in}{0.678778in}}{\pgfqpoint{1.423925in}{0.668179in}}{\pgfqpoint{1.431739in}{0.660365in}}%
\pgfpathcurveto{\pgfqpoint{1.439553in}{0.652552in}}{\pgfqpoint{1.450152in}{0.648162in}}{\pgfqpoint{1.461202in}{0.648162in}}%
\pgfpathclose%
\pgfusepath{stroke,fill}%
\end{pgfscope}%
\begin{pgfscope}%
\pgfpathrectangle{\pgfqpoint{0.787074in}{0.548769in}}{\pgfqpoint{5.062926in}{3.102590in}}%
\pgfusepath{clip}%
\pgfsetbuttcap%
\pgfsetroundjoin%
\definecolor{currentfill}{rgb}{0.121569,0.466667,0.705882}%
\pgfsetfillcolor{currentfill}%
\pgfsetlinewidth{1.003750pt}%
\definecolor{currentstroke}{rgb}{0.121569,0.466667,0.705882}%
\pgfsetstrokecolor{currentstroke}%
\pgfsetdash{}{0pt}%
\pgfpathmoveto{\pgfqpoint{2.057969in}{0.840903in}}%
\pgfpathcurveto{\pgfqpoint{2.069019in}{0.840903in}}{\pgfqpoint{2.079618in}{0.845293in}}{\pgfqpoint{2.087432in}{0.853107in}}%
\pgfpathcurveto{\pgfqpoint{2.095245in}{0.860920in}}{\pgfqpoint{2.099636in}{0.871520in}}{\pgfqpoint{2.099636in}{0.882570in}}%
\pgfpathcurveto{\pgfqpoint{2.099636in}{0.893620in}}{\pgfqpoint{2.095245in}{0.904219in}}{\pgfqpoint{2.087432in}{0.912032in}}%
\pgfpathcurveto{\pgfqpoint{2.079618in}{0.919846in}}{\pgfqpoint{2.069019in}{0.924236in}}{\pgfqpoint{2.057969in}{0.924236in}}%
\pgfpathcurveto{\pgfqpoint{2.046919in}{0.924236in}}{\pgfqpoint{2.036320in}{0.919846in}}{\pgfqpoint{2.028506in}{0.912032in}}%
\pgfpathcurveto{\pgfqpoint{2.020693in}{0.904219in}}{\pgfqpoint{2.016302in}{0.893620in}}{\pgfqpoint{2.016302in}{0.882570in}}%
\pgfpathcurveto{\pgfqpoint{2.016302in}{0.871520in}}{\pgfqpoint{2.020693in}{0.860920in}}{\pgfqpoint{2.028506in}{0.853107in}}%
\pgfpathcurveto{\pgfqpoint{2.036320in}{0.845293in}}{\pgfqpoint{2.046919in}{0.840903in}}{\pgfqpoint{2.057969in}{0.840903in}}%
\pgfpathclose%
\pgfusepath{stroke,fill}%
\end{pgfscope}%
\begin{pgfscope}%
\pgfpathrectangle{\pgfqpoint{0.787074in}{0.548769in}}{\pgfqpoint{5.062926in}{3.102590in}}%
\pgfusepath{clip}%
\pgfsetbuttcap%
\pgfsetroundjoin%
\definecolor{currentfill}{rgb}{0.121569,0.466667,0.705882}%
\pgfsetfillcolor{currentfill}%
\pgfsetlinewidth{1.003750pt}%
\definecolor{currentstroke}{rgb}{0.121569,0.466667,0.705882}%
\pgfsetstrokecolor{currentstroke}%
\pgfsetdash{}{0pt}%
\pgfpathmoveto{\pgfqpoint{1.017294in}{0.787071in}}%
\pgfpathcurveto{\pgfqpoint{1.028344in}{0.787071in}}{\pgfqpoint{1.038943in}{0.791461in}}{\pgfqpoint{1.046756in}{0.799275in}}%
\pgfpathcurveto{\pgfqpoint{1.054570in}{0.807089in}}{\pgfqpoint{1.058960in}{0.817688in}}{\pgfqpoint{1.058960in}{0.828738in}}%
\pgfpathcurveto{\pgfqpoint{1.058960in}{0.839788in}}{\pgfqpoint{1.054570in}{0.850387in}}{\pgfqpoint{1.046756in}{0.858201in}}%
\pgfpathcurveto{\pgfqpoint{1.038943in}{0.866014in}}{\pgfqpoint{1.028344in}{0.870404in}}{\pgfqpoint{1.017294in}{0.870404in}}%
\pgfpathcurveto{\pgfqpoint{1.006244in}{0.870404in}}{\pgfqpoint{0.995644in}{0.866014in}}{\pgfqpoint{0.987831in}{0.858201in}}%
\pgfpathcurveto{\pgfqpoint{0.980017in}{0.850387in}}{\pgfqpoint{0.975627in}{0.839788in}}{\pgfqpoint{0.975627in}{0.828738in}}%
\pgfpathcurveto{\pgfqpoint{0.975627in}{0.817688in}}{\pgfqpoint{0.980017in}{0.807089in}}{\pgfqpoint{0.987831in}{0.799275in}}%
\pgfpathcurveto{\pgfqpoint{0.995644in}{0.791461in}}{\pgfqpoint{1.006244in}{0.787071in}}{\pgfqpoint{1.017294in}{0.787071in}}%
\pgfpathclose%
\pgfusepath{stroke,fill}%
\end{pgfscope}%
\begin{pgfscope}%
\pgfpathrectangle{\pgfqpoint{0.787074in}{0.548769in}}{\pgfqpoint{5.062926in}{3.102590in}}%
\pgfusepath{clip}%
\pgfsetbuttcap%
\pgfsetroundjoin%
\definecolor{currentfill}{rgb}{0.121569,0.466667,0.705882}%
\pgfsetfillcolor{currentfill}%
\pgfsetlinewidth{1.003750pt}%
\definecolor{currentstroke}{rgb}{0.121569,0.466667,0.705882}%
\pgfsetstrokecolor{currentstroke}%
\pgfsetdash{}{0pt}%
\pgfpathmoveto{\pgfqpoint{2.115085in}{0.660658in}}%
\pgfpathcurveto{\pgfqpoint{2.126135in}{0.660658in}}{\pgfqpoint{2.136734in}{0.665048in}}{\pgfqpoint{2.144548in}{0.672862in}}%
\pgfpathcurveto{\pgfqpoint{2.152362in}{0.680676in}}{\pgfqpoint{2.156752in}{0.691275in}}{\pgfqpoint{2.156752in}{0.702325in}}%
\pgfpathcurveto{\pgfqpoint{2.156752in}{0.713375in}}{\pgfqpoint{2.152362in}{0.723974in}}{\pgfqpoint{2.144548in}{0.731788in}}%
\pgfpathcurveto{\pgfqpoint{2.136734in}{0.739601in}}{\pgfqpoint{2.126135in}{0.743991in}}{\pgfqpoint{2.115085in}{0.743991in}}%
\pgfpathcurveto{\pgfqpoint{2.104035in}{0.743991in}}{\pgfqpoint{2.093436in}{0.739601in}}{\pgfqpoint{2.085622in}{0.731788in}}%
\pgfpathcurveto{\pgfqpoint{2.077809in}{0.723974in}}{\pgfqpoint{2.073418in}{0.713375in}}{\pgfqpoint{2.073418in}{0.702325in}}%
\pgfpathcurveto{\pgfqpoint{2.073418in}{0.691275in}}{\pgfqpoint{2.077809in}{0.680676in}}{\pgfqpoint{2.085622in}{0.672862in}}%
\pgfpathcurveto{\pgfqpoint{2.093436in}{0.665048in}}{\pgfqpoint{2.104035in}{0.660658in}}{\pgfqpoint{2.115085in}{0.660658in}}%
\pgfpathclose%
\pgfusepath{stroke,fill}%
\end{pgfscope}%
\begin{pgfscope}%
\pgfpathrectangle{\pgfqpoint{0.787074in}{0.548769in}}{\pgfqpoint{5.062926in}{3.102590in}}%
\pgfusepath{clip}%
\pgfsetbuttcap%
\pgfsetroundjoin%
\definecolor{currentfill}{rgb}{0.121569,0.466667,0.705882}%
\pgfsetfillcolor{currentfill}%
\pgfsetlinewidth{1.003750pt}%
\definecolor{currentstroke}{rgb}{0.121569,0.466667,0.705882}%
\pgfsetstrokecolor{currentstroke}%
\pgfsetdash{}{0pt}%
\pgfpathmoveto{\pgfqpoint{1.549133in}{0.648155in}}%
\pgfpathcurveto{\pgfqpoint{1.560183in}{0.648155in}}{\pgfqpoint{1.570782in}{0.652546in}}{\pgfqpoint{1.578595in}{0.660359in}}%
\pgfpathcurveto{\pgfqpoint{1.586409in}{0.668173in}}{\pgfqpoint{1.590799in}{0.678772in}}{\pgfqpoint{1.590799in}{0.689822in}}%
\pgfpathcurveto{\pgfqpoint{1.590799in}{0.700872in}}{\pgfqpoint{1.586409in}{0.711471in}}{\pgfqpoint{1.578595in}{0.719285in}}%
\pgfpathcurveto{\pgfqpoint{1.570782in}{0.727098in}}{\pgfqpoint{1.560183in}{0.731489in}}{\pgfqpoint{1.549133in}{0.731489in}}%
\pgfpathcurveto{\pgfqpoint{1.538083in}{0.731489in}}{\pgfqpoint{1.527484in}{0.727098in}}{\pgfqpoint{1.519670in}{0.719285in}}%
\pgfpathcurveto{\pgfqpoint{1.511856in}{0.711471in}}{\pgfqpoint{1.507466in}{0.700872in}}{\pgfqpoint{1.507466in}{0.689822in}}%
\pgfpathcurveto{\pgfqpoint{1.507466in}{0.678772in}}{\pgfqpoint{1.511856in}{0.668173in}}{\pgfqpoint{1.519670in}{0.660359in}}%
\pgfpathcurveto{\pgfqpoint{1.527484in}{0.652546in}}{\pgfqpoint{1.538083in}{0.648155in}}{\pgfqpoint{1.549133in}{0.648155in}}%
\pgfpathclose%
\pgfusepath{stroke,fill}%
\end{pgfscope}%
\begin{pgfscope}%
\pgfpathrectangle{\pgfqpoint{0.787074in}{0.548769in}}{\pgfqpoint{5.062926in}{3.102590in}}%
\pgfusepath{clip}%
\pgfsetbuttcap%
\pgfsetroundjoin%
\definecolor{currentfill}{rgb}{0.121569,0.466667,0.705882}%
\pgfsetfillcolor{currentfill}%
\pgfsetlinewidth{1.003750pt}%
\definecolor{currentstroke}{rgb}{0.121569,0.466667,0.705882}%
\pgfsetstrokecolor{currentstroke}%
\pgfsetdash{}{0pt}%
\pgfpathmoveto{\pgfqpoint{1.728770in}{0.648148in}}%
\pgfpathcurveto{\pgfqpoint{1.739821in}{0.648148in}}{\pgfqpoint{1.750420in}{0.652539in}}{\pgfqpoint{1.758233in}{0.660352in}}%
\pgfpathcurveto{\pgfqpoint{1.766047in}{0.668166in}}{\pgfqpoint{1.770437in}{0.678765in}}{\pgfqpoint{1.770437in}{0.689815in}}%
\pgfpathcurveto{\pgfqpoint{1.770437in}{0.700865in}}{\pgfqpoint{1.766047in}{0.711464in}}{\pgfqpoint{1.758233in}{0.719278in}}%
\pgfpathcurveto{\pgfqpoint{1.750420in}{0.727091in}}{\pgfqpoint{1.739821in}{0.731482in}}{\pgfqpoint{1.728770in}{0.731482in}}%
\pgfpathcurveto{\pgfqpoint{1.717720in}{0.731482in}}{\pgfqpoint{1.707121in}{0.727091in}}{\pgfqpoint{1.699308in}{0.719278in}}%
\pgfpathcurveto{\pgfqpoint{1.691494in}{0.711464in}}{\pgfqpoint{1.687104in}{0.700865in}}{\pgfqpoint{1.687104in}{0.689815in}}%
\pgfpathcurveto{\pgfqpoint{1.687104in}{0.678765in}}{\pgfqpoint{1.691494in}{0.668166in}}{\pgfqpoint{1.699308in}{0.660352in}}%
\pgfpathcurveto{\pgfqpoint{1.707121in}{0.652539in}}{\pgfqpoint{1.717720in}{0.648148in}}{\pgfqpoint{1.728770in}{0.648148in}}%
\pgfpathclose%
\pgfusepath{stroke,fill}%
\end{pgfscope}%
\begin{pgfscope}%
\pgfpathrectangle{\pgfqpoint{0.787074in}{0.548769in}}{\pgfqpoint{5.062926in}{3.102590in}}%
\pgfusepath{clip}%
\pgfsetbuttcap%
\pgfsetroundjoin%
\definecolor{currentfill}{rgb}{1.000000,0.498039,0.054902}%
\pgfsetfillcolor{currentfill}%
\pgfsetlinewidth{1.003750pt}%
\definecolor{currentstroke}{rgb}{1.000000,0.498039,0.054902}%
\pgfsetstrokecolor{currentstroke}%
\pgfsetdash{}{0pt}%
\pgfpathmoveto{\pgfqpoint{1.498700in}{1.766469in}}%
\pgfpathcurveto{\pgfqpoint{1.509751in}{1.766469in}}{\pgfqpoint{1.520350in}{1.770859in}}{\pgfqpoint{1.528163in}{1.778672in}}%
\pgfpathcurveto{\pgfqpoint{1.535977in}{1.786486in}}{\pgfqpoint{1.540367in}{1.797085in}}{\pgfqpoint{1.540367in}{1.808135in}}%
\pgfpathcurveto{\pgfqpoint{1.540367in}{1.819185in}}{\pgfqpoint{1.535977in}{1.829784in}}{\pgfqpoint{1.528163in}{1.837598in}}%
\pgfpathcurveto{\pgfqpoint{1.520350in}{1.845412in}}{\pgfqpoint{1.509751in}{1.849802in}}{\pgfqpoint{1.498700in}{1.849802in}}%
\pgfpathcurveto{\pgfqpoint{1.487650in}{1.849802in}}{\pgfqpoint{1.477051in}{1.845412in}}{\pgfqpoint{1.469238in}{1.837598in}}%
\pgfpathcurveto{\pgfqpoint{1.461424in}{1.829784in}}{\pgfqpoint{1.457034in}{1.819185in}}{\pgfqpoint{1.457034in}{1.808135in}}%
\pgfpathcurveto{\pgfqpoint{1.457034in}{1.797085in}}{\pgfqpoint{1.461424in}{1.786486in}}{\pgfqpoint{1.469238in}{1.778672in}}%
\pgfpathcurveto{\pgfqpoint{1.477051in}{1.770859in}}{\pgfqpoint{1.487650in}{1.766469in}}{\pgfqpoint{1.498700in}{1.766469in}}%
\pgfpathclose%
\pgfusepath{stroke,fill}%
\end{pgfscope}%
\begin{pgfscope}%
\pgfpathrectangle{\pgfqpoint{0.787074in}{0.548769in}}{\pgfqpoint{5.062926in}{3.102590in}}%
\pgfusepath{clip}%
\pgfsetbuttcap%
\pgfsetroundjoin%
\definecolor{currentfill}{rgb}{1.000000,0.498039,0.054902}%
\pgfsetfillcolor{currentfill}%
\pgfsetlinewidth{1.003750pt}%
\definecolor{currentstroke}{rgb}{1.000000,0.498039,0.054902}%
\pgfsetstrokecolor{currentstroke}%
\pgfsetdash{}{0pt}%
\pgfpathmoveto{\pgfqpoint{1.509681in}{2.575149in}}%
\pgfpathcurveto{\pgfqpoint{1.520731in}{2.575149in}}{\pgfqpoint{1.531330in}{2.579539in}}{\pgfqpoint{1.539144in}{2.587353in}}%
\pgfpathcurveto{\pgfqpoint{1.546957in}{2.595166in}}{\pgfqpoint{1.551348in}{2.605765in}}{\pgfqpoint{1.551348in}{2.616816in}}%
\pgfpathcurveto{\pgfqpoint{1.551348in}{2.627866in}}{\pgfqpoint{1.546957in}{2.638465in}}{\pgfqpoint{1.539144in}{2.646278in}}%
\pgfpathcurveto{\pgfqpoint{1.531330in}{2.654092in}}{\pgfqpoint{1.520731in}{2.658482in}}{\pgfqpoint{1.509681in}{2.658482in}}%
\pgfpathcurveto{\pgfqpoint{1.498631in}{2.658482in}}{\pgfqpoint{1.488032in}{2.654092in}}{\pgfqpoint{1.480218in}{2.646278in}}%
\pgfpathcurveto{\pgfqpoint{1.472405in}{2.638465in}}{\pgfqpoint{1.468014in}{2.627866in}}{\pgfqpoint{1.468014in}{2.616816in}}%
\pgfpathcurveto{\pgfqpoint{1.468014in}{2.605765in}}{\pgfqpoint{1.472405in}{2.595166in}}{\pgfqpoint{1.480218in}{2.587353in}}%
\pgfpathcurveto{\pgfqpoint{1.488032in}{2.579539in}}{\pgfqpoint{1.498631in}{2.575149in}}{\pgfqpoint{1.509681in}{2.575149in}}%
\pgfpathclose%
\pgfusepath{stroke,fill}%
\end{pgfscope}%
\begin{pgfscope}%
\pgfpathrectangle{\pgfqpoint{0.787074in}{0.548769in}}{\pgfqpoint{5.062926in}{3.102590in}}%
\pgfusepath{clip}%
\pgfsetbuttcap%
\pgfsetroundjoin%
\definecolor{currentfill}{rgb}{0.121569,0.466667,0.705882}%
\pgfsetfillcolor{currentfill}%
\pgfsetlinewidth{1.003750pt}%
\definecolor{currentstroke}{rgb}{0.121569,0.466667,0.705882}%
\pgfsetstrokecolor{currentstroke}%
\pgfsetdash{}{0pt}%
\pgfpathmoveto{\pgfqpoint{1.736496in}{0.825330in}}%
\pgfpathcurveto{\pgfqpoint{1.747546in}{0.825330in}}{\pgfqpoint{1.758145in}{0.829720in}}{\pgfqpoint{1.765959in}{0.837534in}}%
\pgfpathcurveto{\pgfqpoint{1.773772in}{0.845347in}}{\pgfqpoint{1.778163in}{0.855946in}}{\pgfqpoint{1.778163in}{0.866996in}}%
\pgfpathcurveto{\pgfqpoint{1.778163in}{0.878046in}}{\pgfqpoint{1.773772in}{0.888646in}}{\pgfqpoint{1.765959in}{0.896459in}}%
\pgfpathcurveto{\pgfqpoint{1.758145in}{0.904273in}}{\pgfqpoint{1.747546in}{0.908663in}}{\pgfqpoint{1.736496in}{0.908663in}}%
\pgfpathcurveto{\pgfqpoint{1.725446in}{0.908663in}}{\pgfqpoint{1.714847in}{0.904273in}}{\pgfqpoint{1.707033in}{0.896459in}}%
\pgfpathcurveto{\pgfqpoint{1.699220in}{0.888646in}}{\pgfqpoint{1.694829in}{0.878046in}}{\pgfqpoint{1.694829in}{0.866996in}}%
\pgfpathcurveto{\pgfqpoint{1.694829in}{0.855946in}}{\pgfqpoint{1.699220in}{0.845347in}}{\pgfqpoint{1.707033in}{0.837534in}}%
\pgfpathcurveto{\pgfqpoint{1.714847in}{0.829720in}}{\pgfqpoint{1.725446in}{0.825330in}}{\pgfqpoint{1.736496in}{0.825330in}}%
\pgfpathclose%
\pgfusepath{stroke,fill}%
\end{pgfscope}%
\begin{pgfscope}%
\pgfpathrectangle{\pgfqpoint{0.787074in}{0.548769in}}{\pgfqpoint{5.062926in}{3.102590in}}%
\pgfusepath{clip}%
\pgfsetbuttcap%
\pgfsetroundjoin%
\definecolor{currentfill}{rgb}{1.000000,0.498039,0.054902}%
\pgfsetfillcolor{currentfill}%
\pgfsetlinewidth{1.003750pt}%
\definecolor{currentstroke}{rgb}{1.000000,0.498039,0.054902}%
\pgfsetstrokecolor{currentstroke}%
\pgfsetdash{}{0pt}%
\pgfpathmoveto{\pgfqpoint{1.555252in}{1.508166in}}%
\pgfpathcurveto{\pgfqpoint{1.566302in}{1.508166in}}{\pgfqpoint{1.576901in}{1.512556in}}{\pgfqpoint{1.584715in}{1.520370in}}%
\pgfpathcurveto{\pgfqpoint{1.592529in}{1.528184in}}{\pgfqpoint{1.596919in}{1.538783in}}{\pgfqpoint{1.596919in}{1.549833in}}%
\pgfpathcurveto{\pgfqpoint{1.596919in}{1.560883in}}{\pgfqpoint{1.592529in}{1.571482in}}{\pgfqpoint{1.584715in}{1.579296in}}%
\pgfpathcurveto{\pgfqpoint{1.576901in}{1.587109in}}{\pgfqpoint{1.566302in}{1.591499in}}{\pgfqpoint{1.555252in}{1.591499in}}%
\pgfpathcurveto{\pgfqpoint{1.544202in}{1.591499in}}{\pgfqpoint{1.533603in}{1.587109in}}{\pgfqpoint{1.525789in}{1.579296in}}%
\pgfpathcurveto{\pgfqpoint{1.517976in}{1.571482in}}{\pgfqpoint{1.513586in}{1.560883in}}{\pgfqpoint{1.513586in}{1.549833in}}%
\pgfpathcurveto{\pgfqpoint{1.513586in}{1.538783in}}{\pgfqpoint{1.517976in}{1.528184in}}{\pgfqpoint{1.525789in}{1.520370in}}%
\pgfpathcurveto{\pgfqpoint{1.533603in}{1.512556in}}{\pgfqpoint{1.544202in}{1.508166in}}{\pgfqpoint{1.555252in}{1.508166in}}%
\pgfpathclose%
\pgfusepath{stroke,fill}%
\end{pgfscope}%
\begin{pgfscope}%
\pgfpathrectangle{\pgfqpoint{0.787074in}{0.548769in}}{\pgfqpoint{5.062926in}{3.102590in}}%
\pgfusepath{clip}%
\pgfsetbuttcap%
\pgfsetroundjoin%
\definecolor{currentfill}{rgb}{1.000000,0.498039,0.054902}%
\pgfsetfillcolor{currentfill}%
\pgfsetlinewidth{1.003750pt}%
\definecolor{currentstroke}{rgb}{1.000000,0.498039,0.054902}%
\pgfsetstrokecolor{currentstroke}%
\pgfsetdash{}{0pt}%
\pgfpathmoveto{\pgfqpoint{1.707070in}{2.241293in}}%
\pgfpathcurveto{\pgfqpoint{1.718120in}{2.241293in}}{\pgfqpoint{1.728719in}{2.245684in}}{\pgfqpoint{1.736533in}{2.253497in}}%
\pgfpathcurveto{\pgfqpoint{1.744346in}{2.261311in}}{\pgfqpoint{1.748737in}{2.271910in}}{\pgfqpoint{1.748737in}{2.282960in}}%
\pgfpathcurveto{\pgfqpoint{1.748737in}{2.294010in}}{\pgfqpoint{1.744346in}{2.304609in}}{\pgfqpoint{1.736533in}{2.312423in}}%
\pgfpathcurveto{\pgfqpoint{1.728719in}{2.320236in}}{\pgfqpoint{1.718120in}{2.324627in}}{\pgfqpoint{1.707070in}{2.324627in}}%
\pgfpathcurveto{\pgfqpoint{1.696020in}{2.324627in}}{\pgfqpoint{1.685421in}{2.320236in}}{\pgfqpoint{1.677607in}{2.312423in}}%
\pgfpathcurveto{\pgfqpoint{1.669793in}{2.304609in}}{\pgfqpoint{1.665403in}{2.294010in}}{\pgfqpoint{1.665403in}{2.282960in}}%
\pgfpathcurveto{\pgfqpoint{1.665403in}{2.271910in}}{\pgfqpoint{1.669793in}{2.261311in}}{\pgfqpoint{1.677607in}{2.253497in}}%
\pgfpathcurveto{\pgfqpoint{1.685421in}{2.245684in}}{\pgfqpoint{1.696020in}{2.241293in}}{\pgfqpoint{1.707070in}{2.241293in}}%
\pgfpathclose%
\pgfusepath{stroke,fill}%
\end{pgfscope}%
\begin{pgfscope}%
\pgfpathrectangle{\pgfqpoint{0.787074in}{0.548769in}}{\pgfqpoint{5.062926in}{3.102590in}}%
\pgfusepath{clip}%
\pgfsetbuttcap%
\pgfsetroundjoin%
\definecolor{currentfill}{rgb}{0.121569,0.466667,0.705882}%
\pgfsetfillcolor{currentfill}%
\pgfsetlinewidth{1.003750pt}%
\definecolor{currentstroke}{rgb}{0.121569,0.466667,0.705882}%
\pgfsetstrokecolor{currentstroke}%
\pgfsetdash{}{0pt}%
\pgfpathmoveto{\pgfqpoint{5.619867in}{0.648150in}}%
\pgfpathcurveto{\pgfqpoint{5.630917in}{0.648150in}}{\pgfqpoint{5.641516in}{0.652540in}}{\pgfqpoint{5.649330in}{0.660353in}}%
\pgfpathcurveto{\pgfqpoint{5.657143in}{0.668167in}}{\pgfqpoint{5.661534in}{0.678766in}}{\pgfqpoint{5.661534in}{0.689816in}}%
\pgfpathcurveto{\pgfqpoint{5.661534in}{0.700866in}}{\pgfqpoint{5.657143in}{0.711465in}}{\pgfqpoint{5.649330in}{0.719279in}}%
\pgfpathcurveto{\pgfqpoint{5.641516in}{0.727093in}}{\pgfqpoint{5.630917in}{0.731483in}}{\pgfqpoint{5.619867in}{0.731483in}}%
\pgfpathcurveto{\pgfqpoint{5.608817in}{0.731483in}}{\pgfqpoint{5.598218in}{0.727093in}}{\pgfqpoint{5.590404in}{0.719279in}}%
\pgfpathcurveto{\pgfqpoint{5.582591in}{0.711465in}}{\pgfqpoint{5.578200in}{0.700866in}}{\pgfqpoint{5.578200in}{0.689816in}}%
\pgfpathcurveto{\pgfqpoint{5.578200in}{0.678766in}}{\pgfqpoint{5.582591in}{0.668167in}}{\pgfqpoint{5.590404in}{0.660353in}}%
\pgfpathcurveto{\pgfqpoint{5.598218in}{0.652540in}}{\pgfqpoint{5.608817in}{0.648150in}}{\pgfqpoint{5.619867in}{0.648150in}}%
\pgfpathclose%
\pgfusepath{stroke,fill}%
\end{pgfscope}%
\begin{pgfscope}%
\pgfpathrectangle{\pgfqpoint{0.787074in}{0.548769in}}{\pgfqpoint{5.062926in}{3.102590in}}%
\pgfusepath{clip}%
\pgfsetbuttcap%
\pgfsetroundjoin%
\definecolor{currentfill}{rgb}{1.000000,0.498039,0.054902}%
\pgfsetfillcolor{currentfill}%
\pgfsetlinewidth{1.003750pt}%
\definecolor{currentstroke}{rgb}{1.000000,0.498039,0.054902}%
\pgfsetstrokecolor{currentstroke}%
\pgfsetdash{}{0pt}%
\pgfpathmoveto{\pgfqpoint{1.990654in}{2.050079in}}%
\pgfpathcurveto{\pgfqpoint{2.001704in}{2.050079in}}{\pgfqpoint{2.012303in}{2.054470in}}{\pgfqpoint{2.020116in}{2.062283in}}%
\pgfpathcurveto{\pgfqpoint{2.027930in}{2.070097in}}{\pgfqpoint{2.032320in}{2.080696in}}{\pgfqpoint{2.032320in}{2.091746in}}%
\pgfpathcurveto{\pgfqpoint{2.032320in}{2.102796in}}{\pgfqpoint{2.027930in}{2.113395in}}{\pgfqpoint{2.020116in}{2.121209in}}%
\pgfpathcurveto{\pgfqpoint{2.012303in}{2.129022in}}{\pgfqpoint{2.001704in}{2.133413in}}{\pgfqpoint{1.990654in}{2.133413in}}%
\pgfpathcurveto{\pgfqpoint{1.979604in}{2.133413in}}{\pgfqpoint{1.969005in}{2.129022in}}{\pgfqpoint{1.961191in}{2.121209in}}%
\pgfpathcurveto{\pgfqpoint{1.953377in}{2.113395in}}{\pgfqpoint{1.948987in}{2.102796in}}{\pgfqpoint{1.948987in}{2.091746in}}%
\pgfpathcurveto{\pgfqpoint{1.948987in}{2.080696in}}{\pgfqpoint{1.953377in}{2.070097in}}{\pgfqpoint{1.961191in}{2.062283in}}%
\pgfpathcurveto{\pgfqpoint{1.969005in}{2.054470in}}{\pgfqpoint{1.979604in}{2.050079in}}{\pgfqpoint{1.990654in}{2.050079in}}%
\pgfpathclose%
\pgfusepath{stroke,fill}%
\end{pgfscope}%
\begin{pgfscope}%
\pgfpathrectangle{\pgfqpoint{0.787074in}{0.548769in}}{\pgfqpoint{5.062926in}{3.102590in}}%
\pgfusepath{clip}%
\pgfsetbuttcap%
\pgfsetroundjoin%
\definecolor{currentfill}{rgb}{0.121569,0.466667,0.705882}%
\pgfsetfillcolor{currentfill}%
\pgfsetlinewidth{1.003750pt}%
\definecolor{currentstroke}{rgb}{0.121569,0.466667,0.705882}%
\pgfsetstrokecolor{currentstroke}%
\pgfsetdash{}{0pt}%
\pgfpathmoveto{\pgfqpoint{1.361596in}{0.648148in}}%
\pgfpathcurveto{\pgfqpoint{1.372646in}{0.648148in}}{\pgfqpoint{1.383245in}{0.652539in}}{\pgfqpoint{1.391059in}{0.660352in}}%
\pgfpathcurveto{\pgfqpoint{1.398872in}{0.668166in}}{\pgfqpoint{1.403263in}{0.678765in}}{\pgfqpoint{1.403263in}{0.689815in}}%
\pgfpathcurveto{\pgfqpoint{1.403263in}{0.700865in}}{\pgfqpoint{1.398872in}{0.711464in}}{\pgfqpoint{1.391059in}{0.719278in}}%
\pgfpathcurveto{\pgfqpoint{1.383245in}{0.727091in}}{\pgfqpoint{1.372646in}{0.731482in}}{\pgfqpoint{1.361596in}{0.731482in}}%
\pgfpathcurveto{\pgfqpoint{1.350546in}{0.731482in}}{\pgfqpoint{1.339947in}{0.727091in}}{\pgfqpoint{1.332133in}{0.719278in}}%
\pgfpathcurveto{\pgfqpoint{1.324319in}{0.711464in}}{\pgfqpoint{1.319929in}{0.700865in}}{\pgfqpoint{1.319929in}{0.689815in}}%
\pgfpathcurveto{\pgfqpoint{1.319929in}{0.678765in}}{\pgfqpoint{1.324319in}{0.668166in}}{\pgfqpoint{1.332133in}{0.660352in}}%
\pgfpathcurveto{\pgfqpoint{1.339947in}{0.652539in}}{\pgfqpoint{1.350546in}{0.648148in}}{\pgfqpoint{1.361596in}{0.648148in}}%
\pgfpathclose%
\pgfusepath{stroke,fill}%
\end{pgfscope}%
\begin{pgfscope}%
\pgfpathrectangle{\pgfqpoint{0.787074in}{0.548769in}}{\pgfqpoint{5.062926in}{3.102590in}}%
\pgfusepath{clip}%
\pgfsetbuttcap%
\pgfsetroundjoin%
\definecolor{currentfill}{rgb}{0.121569,0.466667,0.705882}%
\pgfsetfillcolor{currentfill}%
\pgfsetlinewidth{1.003750pt}%
\definecolor{currentstroke}{rgb}{0.121569,0.466667,0.705882}%
\pgfsetstrokecolor{currentstroke}%
\pgfsetdash{}{0pt}%
\pgfpathmoveto{\pgfqpoint{1.317934in}{0.648168in}}%
\pgfpathcurveto{\pgfqpoint{1.328984in}{0.648168in}}{\pgfqpoint{1.339583in}{0.652558in}}{\pgfqpoint{1.347397in}{0.660372in}}%
\pgfpathcurveto{\pgfqpoint{1.355211in}{0.668185in}}{\pgfqpoint{1.359601in}{0.678784in}}{\pgfqpoint{1.359601in}{0.689834in}}%
\pgfpathcurveto{\pgfqpoint{1.359601in}{0.700885in}}{\pgfqpoint{1.355211in}{0.711484in}}{\pgfqpoint{1.347397in}{0.719297in}}%
\pgfpathcurveto{\pgfqpoint{1.339583in}{0.727111in}}{\pgfqpoint{1.328984in}{0.731501in}}{\pgfqpoint{1.317934in}{0.731501in}}%
\pgfpathcurveto{\pgfqpoint{1.306884in}{0.731501in}}{\pgfqpoint{1.296285in}{0.727111in}}{\pgfqpoint{1.288471in}{0.719297in}}%
\pgfpathcurveto{\pgfqpoint{1.280658in}{0.711484in}}{\pgfqpoint{1.276268in}{0.700885in}}{\pgfqpoint{1.276268in}{0.689834in}}%
\pgfpathcurveto{\pgfqpoint{1.276268in}{0.678784in}}{\pgfqpoint{1.280658in}{0.668185in}}{\pgfqpoint{1.288471in}{0.660372in}}%
\pgfpathcurveto{\pgfqpoint{1.296285in}{0.652558in}}{\pgfqpoint{1.306884in}{0.648168in}}{\pgfqpoint{1.317934in}{0.648168in}}%
\pgfpathclose%
\pgfusepath{stroke,fill}%
\end{pgfscope}%
\begin{pgfscope}%
\pgfpathrectangle{\pgfqpoint{0.787074in}{0.548769in}}{\pgfqpoint{5.062926in}{3.102590in}}%
\pgfusepath{clip}%
\pgfsetbuttcap%
\pgfsetroundjoin%
\definecolor{currentfill}{rgb}{0.121569,0.466667,0.705882}%
\pgfsetfillcolor{currentfill}%
\pgfsetlinewidth{1.003750pt}%
\definecolor{currentstroke}{rgb}{0.121569,0.466667,0.705882}%
\pgfsetstrokecolor{currentstroke}%
\pgfsetdash{}{0pt}%
\pgfpathmoveto{\pgfqpoint{1.379347in}{0.861755in}}%
\pgfpathcurveto{\pgfqpoint{1.390397in}{0.861755in}}{\pgfqpoint{1.400996in}{0.866145in}}{\pgfqpoint{1.408810in}{0.873958in}}%
\pgfpathcurveto{\pgfqpoint{1.416623in}{0.881772in}}{\pgfqpoint{1.421014in}{0.892371in}}{\pgfqpoint{1.421014in}{0.903421in}}%
\pgfpathcurveto{\pgfqpoint{1.421014in}{0.914471in}}{\pgfqpoint{1.416623in}{0.925070in}}{\pgfqpoint{1.408810in}{0.932884in}}%
\pgfpathcurveto{\pgfqpoint{1.400996in}{0.940698in}}{\pgfqpoint{1.390397in}{0.945088in}}{\pgfqpoint{1.379347in}{0.945088in}}%
\pgfpathcurveto{\pgfqpoint{1.368297in}{0.945088in}}{\pgfqpoint{1.357698in}{0.940698in}}{\pgfqpoint{1.349884in}{0.932884in}}%
\pgfpathcurveto{\pgfqpoint{1.342071in}{0.925070in}}{\pgfqpoint{1.337680in}{0.914471in}}{\pgfqpoint{1.337680in}{0.903421in}}%
\pgfpathcurveto{\pgfqpoint{1.337680in}{0.892371in}}{\pgfqpoint{1.342071in}{0.881772in}}{\pgfqpoint{1.349884in}{0.873958in}}%
\pgfpathcurveto{\pgfqpoint{1.357698in}{0.866145in}}{\pgfqpoint{1.368297in}{0.861755in}}{\pgfqpoint{1.379347in}{0.861755in}}%
\pgfpathclose%
\pgfusepath{stroke,fill}%
\end{pgfscope}%
\begin{pgfscope}%
\pgfpathrectangle{\pgfqpoint{0.787074in}{0.548769in}}{\pgfqpoint{5.062926in}{3.102590in}}%
\pgfusepath{clip}%
\pgfsetbuttcap%
\pgfsetroundjoin%
\definecolor{currentfill}{rgb}{0.121569,0.466667,0.705882}%
\pgfsetfillcolor{currentfill}%
\pgfsetlinewidth{1.003750pt}%
\definecolor{currentstroke}{rgb}{0.121569,0.466667,0.705882}%
\pgfsetstrokecolor{currentstroke}%
\pgfsetdash{}{0pt}%
\pgfpathmoveto{\pgfqpoint{2.181055in}{1.953744in}}%
\pgfpathcurveto{\pgfqpoint{2.192105in}{1.953744in}}{\pgfqpoint{2.202704in}{1.958135in}}{\pgfqpoint{2.210518in}{1.965948in}}%
\pgfpathcurveto{\pgfqpoint{2.218331in}{1.973762in}}{\pgfqpoint{2.222722in}{1.984361in}}{\pgfqpoint{2.222722in}{1.995411in}}%
\pgfpathcurveto{\pgfqpoint{2.222722in}{2.006461in}}{\pgfqpoint{2.218331in}{2.017060in}}{\pgfqpoint{2.210518in}{2.024874in}}%
\pgfpathcurveto{\pgfqpoint{2.202704in}{2.032687in}}{\pgfqpoint{2.192105in}{2.037078in}}{\pgfqpoint{2.181055in}{2.037078in}}%
\pgfpathcurveto{\pgfqpoint{2.170005in}{2.037078in}}{\pgfqpoint{2.159406in}{2.032687in}}{\pgfqpoint{2.151592in}{2.024874in}}%
\pgfpathcurveto{\pgfqpoint{2.143779in}{2.017060in}}{\pgfqpoint{2.139388in}{2.006461in}}{\pgfqpoint{2.139388in}{1.995411in}}%
\pgfpathcurveto{\pgfqpoint{2.139388in}{1.984361in}}{\pgfqpoint{2.143779in}{1.973762in}}{\pgfqpoint{2.151592in}{1.965948in}}%
\pgfpathcurveto{\pgfqpoint{2.159406in}{1.958135in}}{\pgfqpoint{2.170005in}{1.953744in}}{\pgfqpoint{2.181055in}{1.953744in}}%
\pgfpathclose%
\pgfusepath{stroke,fill}%
\end{pgfscope}%
\begin{pgfscope}%
\pgfpathrectangle{\pgfqpoint{0.787074in}{0.548769in}}{\pgfqpoint{5.062926in}{3.102590in}}%
\pgfusepath{clip}%
\pgfsetbuttcap%
\pgfsetroundjoin%
\definecolor{currentfill}{rgb}{1.000000,0.498039,0.054902}%
\pgfsetfillcolor{currentfill}%
\pgfsetlinewidth{1.003750pt}%
\definecolor{currentstroke}{rgb}{1.000000,0.498039,0.054902}%
\pgfsetstrokecolor{currentstroke}%
\pgfsetdash{}{0pt}%
\pgfpathmoveto{\pgfqpoint{1.551650in}{1.401411in}}%
\pgfpathcurveto{\pgfqpoint{1.562700in}{1.401411in}}{\pgfqpoint{1.573299in}{1.405801in}}{\pgfqpoint{1.581113in}{1.413615in}}%
\pgfpathcurveto{\pgfqpoint{1.588926in}{1.421429in}}{\pgfqpoint{1.593317in}{1.432028in}}{\pgfqpoint{1.593317in}{1.443078in}}%
\pgfpathcurveto{\pgfqpoint{1.593317in}{1.454128in}}{\pgfqpoint{1.588926in}{1.464727in}}{\pgfqpoint{1.581113in}{1.472540in}}%
\pgfpathcurveto{\pgfqpoint{1.573299in}{1.480354in}}{\pgfqpoint{1.562700in}{1.484744in}}{\pgfqpoint{1.551650in}{1.484744in}}%
\pgfpathcurveto{\pgfqpoint{1.540600in}{1.484744in}}{\pgfqpoint{1.530001in}{1.480354in}}{\pgfqpoint{1.522187in}{1.472540in}}%
\pgfpathcurveto{\pgfqpoint{1.514374in}{1.464727in}}{\pgfqpoint{1.509983in}{1.454128in}}{\pgfqpoint{1.509983in}{1.443078in}}%
\pgfpathcurveto{\pgfqpoint{1.509983in}{1.432028in}}{\pgfqpoint{1.514374in}{1.421429in}}{\pgfqpoint{1.522187in}{1.413615in}}%
\pgfpathcurveto{\pgfqpoint{1.530001in}{1.405801in}}{\pgfqpoint{1.540600in}{1.401411in}}{\pgfqpoint{1.551650in}{1.401411in}}%
\pgfpathclose%
\pgfusepath{stroke,fill}%
\end{pgfscope}%
\begin{pgfscope}%
\pgfpathrectangle{\pgfqpoint{0.787074in}{0.548769in}}{\pgfqpoint{5.062926in}{3.102590in}}%
\pgfusepath{clip}%
\pgfsetbuttcap%
\pgfsetroundjoin%
\definecolor{currentfill}{rgb}{0.121569,0.466667,0.705882}%
\pgfsetfillcolor{currentfill}%
\pgfsetlinewidth{1.003750pt}%
\definecolor{currentstroke}{rgb}{0.121569,0.466667,0.705882}%
\pgfsetstrokecolor{currentstroke}%
\pgfsetdash{}{0pt}%
\pgfpathmoveto{\pgfqpoint{1.340850in}{0.648141in}}%
\pgfpathcurveto{\pgfqpoint{1.351900in}{0.648141in}}{\pgfqpoint{1.362499in}{0.652531in}}{\pgfqpoint{1.370313in}{0.660344in}}%
\pgfpathcurveto{\pgfqpoint{1.378126in}{0.668158in}}{\pgfqpoint{1.382517in}{0.678757in}}{\pgfqpoint{1.382517in}{0.689807in}}%
\pgfpathcurveto{\pgfqpoint{1.382517in}{0.700857in}}{\pgfqpoint{1.378126in}{0.711456in}}{\pgfqpoint{1.370313in}{0.719270in}}%
\pgfpathcurveto{\pgfqpoint{1.362499in}{0.727084in}}{\pgfqpoint{1.351900in}{0.731474in}}{\pgfqpoint{1.340850in}{0.731474in}}%
\pgfpathcurveto{\pgfqpoint{1.329800in}{0.731474in}}{\pgfqpoint{1.319201in}{0.727084in}}{\pgfqpoint{1.311387in}{0.719270in}}%
\pgfpathcurveto{\pgfqpoint{1.303574in}{0.711456in}}{\pgfqpoint{1.299183in}{0.700857in}}{\pgfqpoint{1.299183in}{0.689807in}}%
\pgfpathcurveto{\pgfqpoint{1.299183in}{0.678757in}}{\pgfqpoint{1.303574in}{0.668158in}}{\pgfqpoint{1.311387in}{0.660344in}}%
\pgfpathcurveto{\pgfqpoint{1.319201in}{0.652531in}}{\pgfqpoint{1.329800in}{0.648141in}}{\pgfqpoint{1.340850in}{0.648141in}}%
\pgfpathclose%
\pgfusepath{stroke,fill}%
\end{pgfscope}%
\begin{pgfscope}%
\pgfpathrectangle{\pgfqpoint{0.787074in}{0.548769in}}{\pgfqpoint{5.062926in}{3.102590in}}%
\pgfusepath{clip}%
\pgfsetbuttcap%
\pgfsetroundjoin%
\definecolor{currentfill}{rgb}{0.121569,0.466667,0.705882}%
\pgfsetfillcolor{currentfill}%
\pgfsetlinewidth{1.003750pt}%
\definecolor{currentstroke}{rgb}{0.121569,0.466667,0.705882}%
\pgfsetstrokecolor{currentstroke}%
\pgfsetdash{}{0pt}%
\pgfpathmoveto{\pgfqpoint{1.290331in}{0.787322in}}%
\pgfpathcurveto{\pgfqpoint{1.301381in}{0.787322in}}{\pgfqpoint{1.311980in}{0.791712in}}{\pgfqpoint{1.319794in}{0.799526in}}%
\pgfpathcurveto{\pgfqpoint{1.327607in}{0.807339in}}{\pgfqpoint{1.331998in}{0.817938in}}{\pgfqpoint{1.331998in}{0.828988in}}%
\pgfpathcurveto{\pgfqpoint{1.331998in}{0.840039in}}{\pgfqpoint{1.327607in}{0.850638in}}{\pgfqpoint{1.319794in}{0.858451in}}%
\pgfpathcurveto{\pgfqpoint{1.311980in}{0.866265in}}{\pgfqpoint{1.301381in}{0.870655in}}{\pgfqpoint{1.290331in}{0.870655in}}%
\pgfpathcurveto{\pgfqpoint{1.279281in}{0.870655in}}{\pgfqpoint{1.268682in}{0.866265in}}{\pgfqpoint{1.260868in}{0.858451in}}%
\pgfpathcurveto{\pgfqpoint{1.253055in}{0.850638in}}{\pgfqpoint{1.248664in}{0.840039in}}{\pgfqpoint{1.248664in}{0.828988in}}%
\pgfpathcurveto{\pgfqpoint{1.248664in}{0.817938in}}{\pgfqpoint{1.253055in}{0.807339in}}{\pgfqpoint{1.260868in}{0.799526in}}%
\pgfpathcurveto{\pgfqpoint{1.268682in}{0.791712in}}{\pgfqpoint{1.279281in}{0.787322in}}{\pgfqpoint{1.290331in}{0.787322in}}%
\pgfpathclose%
\pgfusepath{stroke,fill}%
\end{pgfscope}%
\begin{pgfscope}%
\pgfpathrectangle{\pgfqpoint{0.787074in}{0.548769in}}{\pgfqpoint{5.062926in}{3.102590in}}%
\pgfusepath{clip}%
\pgfsetbuttcap%
\pgfsetroundjoin%
\definecolor{currentfill}{rgb}{0.121569,0.466667,0.705882}%
\pgfsetfillcolor{currentfill}%
\pgfsetlinewidth{1.003750pt}%
\definecolor{currentstroke}{rgb}{0.121569,0.466667,0.705882}%
\pgfsetstrokecolor{currentstroke}%
\pgfsetdash{}{0pt}%
\pgfpathmoveto{\pgfqpoint{2.477486in}{1.729931in}}%
\pgfpathcurveto{\pgfqpoint{2.488536in}{1.729931in}}{\pgfqpoint{2.499135in}{1.734321in}}{\pgfqpoint{2.506948in}{1.742135in}}%
\pgfpathcurveto{\pgfqpoint{2.514762in}{1.749948in}}{\pgfqpoint{2.519152in}{1.760547in}}{\pgfqpoint{2.519152in}{1.771597in}}%
\pgfpathcurveto{\pgfqpoint{2.519152in}{1.782647in}}{\pgfqpoint{2.514762in}{1.793246in}}{\pgfqpoint{2.506948in}{1.801060in}}%
\pgfpathcurveto{\pgfqpoint{2.499135in}{1.808874in}}{\pgfqpoint{2.488536in}{1.813264in}}{\pgfqpoint{2.477486in}{1.813264in}}%
\pgfpathcurveto{\pgfqpoint{2.466435in}{1.813264in}}{\pgfqpoint{2.455836in}{1.808874in}}{\pgfqpoint{2.448023in}{1.801060in}}%
\pgfpathcurveto{\pgfqpoint{2.440209in}{1.793246in}}{\pgfqpoint{2.435819in}{1.782647in}}{\pgfqpoint{2.435819in}{1.771597in}}%
\pgfpathcurveto{\pgfqpoint{2.435819in}{1.760547in}}{\pgfqpoint{2.440209in}{1.749948in}}{\pgfqpoint{2.448023in}{1.742135in}}%
\pgfpathcurveto{\pgfqpoint{2.455836in}{1.734321in}}{\pgfqpoint{2.466435in}{1.729931in}}{\pgfqpoint{2.477486in}{1.729931in}}%
\pgfpathclose%
\pgfusepath{stroke,fill}%
\end{pgfscope}%
\begin{pgfscope}%
\pgfpathrectangle{\pgfqpoint{0.787074in}{0.548769in}}{\pgfqpoint{5.062926in}{3.102590in}}%
\pgfusepath{clip}%
\pgfsetbuttcap%
\pgfsetroundjoin%
\definecolor{currentfill}{rgb}{0.121569,0.466667,0.705882}%
\pgfsetfillcolor{currentfill}%
\pgfsetlinewidth{1.003750pt}%
\definecolor{currentstroke}{rgb}{0.121569,0.466667,0.705882}%
\pgfsetstrokecolor{currentstroke}%
\pgfsetdash{}{0pt}%
\pgfpathmoveto{\pgfqpoint{1.078359in}{0.648129in}}%
\pgfpathcurveto{\pgfqpoint{1.089409in}{0.648129in}}{\pgfqpoint{1.100008in}{0.652519in}}{\pgfqpoint{1.107822in}{0.660333in}}%
\pgfpathcurveto{\pgfqpoint{1.115636in}{0.668146in}}{\pgfqpoint{1.120026in}{0.678745in}}{\pgfqpoint{1.120026in}{0.689796in}}%
\pgfpathcurveto{\pgfqpoint{1.120026in}{0.700846in}}{\pgfqpoint{1.115636in}{0.711445in}}{\pgfqpoint{1.107822in}{0.719258in}}%
\pgfpathcurveto{\pgfqpoint{1.100008in}{0.727072in}}{\pgfqpoint{1.089409in}{0.731462in}}{\pgfqpoint{1.078359in}{0.731462in}}%
\pgfpathcurveto{\pgfqpoint{1.067309in}{0.731462in}}{\pgfqpoint{1.056710in}{0.727072in}}{\pgfqpoint{1.048896in}{0.719258in}}%
\pgfpathcurveto{\pgfqpoint{1.041083in}{0.711445in}}{\pgfqpoint{1.036693in}{0.700846in}}{\pgfqpoint{1.036693in}{0.689796in}}%
\pgfpathcurveto{\pgfqpoint{1.036693in}{0.678745in}}{\pgfqpoint{1.041083in}{0.668146in}}{\pgfqpoint{1.048896in}{0.660333in}}%
\pgfpathcurveto{\pgfqpoint{1.056710in}{0.652519in}}{\pgfqpoint{1.067309in}{0.648129in}}{\pgfqpoint{1.078359in}{0.648129in}}%
\pgfpathclose%
\pgfusepath{stroke,fill}%
\end{pgfscope}%
\begin{pgfscope}%
\pgfpathrectangle{\pgfqpoint{0.787074in}{0.548769in}}{\pgfqpoint{5.062926in}{3.102590in}}%
\pgfusepath{clip}%
\pgfsetbuttcap%
\pgfsetroundjoin%
\definecolor{currentfill}{rgb}{1.000000,0.498039,0.054902}%
\pgfsetfillcolor{currentfill}%
\pgfsetlinewidth{1.003750pt}%
\definecolor{currentstroke}{rgb}{1.000000,0.498039,0.054902}%
\pgfsetstrokecolor{currentstroke}%
\pgfsetdash{}{0pt}%
\pgfpathmoveto{\pgfqpoint{1.785019in}{1.417750in}}%
\pgfpathcurveto{\pgfqpoint{1.796069in}{1.417750in}}{\pgfqpoint{1.806668in}{1.422140in}}{\pgfqpoint{1.814481in}{1.429954in}}%
\pgfpathcurveto{\pgfqpoint{1.822295in}{1.437768in}}{\pgfqpoint{1.826685in}{1.448367in}}{\pgfqpoint{1.826685in}{1.459417in}}%
\pgfpathcurveto{\pgfqpoint{1.826685in}{1.470467in}}{\pgfqpoint{1.822295in}{1.481066in}}{\pgfqpoint{1.814481in}{1.488880in}}%
\pgfpathcurveto{\pgfqpoint{1.806668in}{1.496693in}}{\pgfqpoint{1.796069in}{1.501083in}}{\pgfqpoint{1.785019in}{1.501083in}}%
\pgfpathcurveto{\pgfqpoint{1.773968in}{1.501083in}}{\pgfqpoint{1.763369in}{1.496693in}}{\pgfqpoint{1.755556in}{1.488880in}}%
\pgfpathcurveto{\pgfqpoint{1.747742in}{1.481066in}}{\pgfqpoint{1.743352in}{1.470467in}}{\pgfqpoint{1.743352in}{1.459417in}}%
\pgfpathcurveto{\pgfqpoint{1.743352in}{1.448367in}}{\pgfqpoint{1.747742in}{1.437768in}}{\pgfqpoint{1.755556in}{1.429954in}}%
\pgfpathcurveto{\pgfqpoint{1.763369in}{1.422140in}}{\pgfqpoint{1.773968in}{1.417750in}}{\pgfqpoint{1.785019in}{1.417750in}}%
\pgfpathclose%
\pgfusepath{stroke,fill}%
\end{pgfscope}%
\begin{pgfscope}%
\pgfpathrectangle{\pgfqpoint{0.787074in}{0.548769in}}{\pgfqpoint{5.062926in}{3.102590in}}%
\pgfusepath{clip}%
\pgfsetbuttcap%
\pgfsetroundjoin%
\definecolor{currentfill}{rgb}{1.000000,0.498039,0.054902}%
\pgfsetfillcolor{currentfill}%
\pgfsetlinewidth{1.003750pt}%
\definecolor{currentstroke}{rgb}{1.000000,0.498039,0.054902}%
\pgfsetstrokecolor{currentstroke}%
\pgfsetdash{}{0pt}%
\pgfpathmoveto{\pgfqpoint{1.587847in}{2.301800in}}%
\pgfpathcurveto{\pgfqpoint{1.598897in}{2.301800in}}{\pgfqpoint{1.609496in}{2.306191in}}{\pgfqpoint{1.617309in}{2.314004in}}%
\pgfpathcurveto{\pgfqpoint{1.625123in}{2.321818in}}{\pgfqpoint{1.629513in}{2.332417in}}{\pgfqpoint{1.629513in}{2.343467in}}%
\pgfpathcurveto{\pgfqpoint{1.629513in}{2.354517in}}{\pgfqpoint{1.625123in}{2.365116in}}{\pgfqpoint{1.617309in}{2.372930in}}%
\pgfpathcurveto{\pgfqpoint{1.609496in}{2.380743in}}{\pgfqpoint{1.598897in}{2.385134in}}{\pgfqpoint{1.587847in}{2.385134in}}%
\pgfpathcurveto{\pgfqpoint{1.576796in}{2.385134in}}{\pgfqpoint{1.566197in}{2.380743in}}{\pgfqpoint{1.558384in}{2.372930in}}%
\pgfpathcurveto{\pgfqpoint{1.550570in}{2.365116in}}{\pgfqpoint{1.546180in}{2.354517in}}{\pgfqpoint{1.546180in}{2.343467in}}%
\pgfpathcurveto{\pgfqpoint{1.546180in}{2.332417in}}{\pgfqpoint{1.550570in}{2.321818in}}{\pgfqpoint{1.558384in}{2.314004in}}%
\pgfpathcurveto{\pgfqpoint{1.566197in}{2.306191in}}{\pgfqpoint{1.576796in}{2.301800in}}{\pgfqpoint{1.587847in}{2.301800in}}%
\pgfpathclose%
\pgfusepath{stroke,fill}%
\end{pgfscope}%
\begin{pgfscope}%
\pgfpathrectangle{\pgfqpoint{0.787074in}{0.548769in}}{\pgfqpoint{5.062926in}{3.102590in}}%
\pgfusepath{clip}%
\pgfsetbuttcap%
\pgfsetroundjoin%
\definecolor{currentfill}{rgb}{0.121569,0.466667,0.705882}%
\pgfsetfillcolor{currentfill}%
\pgfsetlinewidth{1.003750pt}%
\definecolor{currentstroke}{rgb}{0.121569,0.466667,0.705882}%
\pgfsetstrokecolor{currentstroke}%
\pgfsetdash{}{0pt}%
\pgfpathmoveto{\pgfqpoint{1.909580in}{0.652202in}}%
\pgfpathcurveto{\pgfqpoint{1.920630in}{0.652202in}}{\pgfqpoint{1.931229in}{0.656593in}}{\pgfqpoint{1.939043in}{0.664406in}}%
\pgfpathcurveto{\pgfqpoint{1.946857in}{0.672220in}}{\pgfqpoint{1.951247in}{0.682819in}}{\pgfqpoint{1.951247in}{0.693869in}}%
\pgfpathcurveto{\pgfqpoint{1.951247in}{0.704919in}}{\pgfqpoint{1.946857in}{0.715518in}}{\pgfqpoint{1.939043in}{0.723332in}}%
\pgfpathcurveto{\pgfqpoint{1.931229in}{0.731145in}}{\pgfqpoint{1.920630in}{0.735536in}}{\pgfqpoint{1.909580in}{0.735536in}}%
\pgfpathcurveto{\pgfqpoint{1.898530in}{0.735536in}}{\pgfqpoint{1.887931in}{0.731145in}}{\pgfqpoint{1.880117in}{0.723332in}}%
\pgfpathcurveto{\pgfqpoint{1.872304in}{0.715518in}}{\pgfqpoint{1.867913in}{0.704919in}}{\pgfqpoint{1.867913in}{0.693869in}}%
\pgfpathcurveto{\pgfqpoint{1.867913in}{0.682819in}}{\pgfqpoint{1.872304in}{0.672220in}}{\pgfqpoint{1.880117in}{0.664406in}}%
\pgfpathcurveto{\pgfqpoint{1.887931in}{0.656593in}}{\pgfqpoint{1.898530in}{0.652202in}}{\pgfqpoint{1.909580in}{0.652202in}}%
\pgfpathclose%
\pgfusepath{stroke,fill}%
\end{pgfscope}%
\begin{pgfscope}%
\pgfpathrectangle{\pgfqpoint{0.787074in}{0.548769in}}{\pgfqpoint{5.062926in}{3.102590in}}%
\pgfusepath{clip}%
\pgfsetbuttcap%
\pgfsetroundjoin%
\definecolor{currentfill}{rgb}{0.121569,0.466667,0.705882}%
\pgfsetfillcolor{currentfill}%
\pgfsetlinewidth{1.003750pt}%
\definecolor{currentstroke}{rgb}{0.121569,0.466667,0.705882}%
\pgfsetstrokecolor{currentstroke}%
\pgfsetdash{}{0pt}%
\pgfpathmoveto{\pgfqpoint{2.067517in}{0.648180in}}%
\pgfpathcurveto{\pgfqpoint{2.078567in}{0.648180in}}{\pgfqpoint{2.089166in}{0.652571in}}{\pgfqpoint{2.096980in}{0.660384in}}%
\pgfpathcurveto{\pgfqpoint{2.104794in}{0.668198in}}{\pgfqpoint{2.109184in}{0.678797in}}{\pgfqpoint{2.109184in}{0.689847in}}%
\pgfpathcurveto{\pgfqpoint{2.109184in}{0.700897in}}{\pgfqpoint{2.104794in}{0.711496in}}{\pgfqpoint{2.096980in}{0.719310in}}%
\pgfpathcurveto{\pgfqpoint{2.089166in}{0.727123in}}{\pgfqpoint{2.078567in}{0.731514in}}{\pgfqpoint{2.067517in}{0.731514in}}%
\pgfpathcurveto{\pgfqpoint{2.056467in}{0.731514in}}{\pgfqpoint{2.045868in}{0.727123in}}{\pgfqpoint{2.038055in}{0.719310in}}%
\pgfpathcurveto{\pgfqpoint{2.030241in}{0.711496in}}{\pgfqpoint{2.025851in}{0.700897in}}{\pgfqpoint{2.025851in}{0.689847in}}%
\pgfpathcurveto{\pgfqpoint{2.025851in}{0.678797in}}{\pgfqpoint{2.030241in}{0.668198in}}{\pgfqpoint{2.038055in}{0.660384in}}%
\pgfpathcurveto{\pgfqpoint{2.045868in}{0.652571in}}{\pgfqpoint{2.056467in}{0.648180in}}{\pgfqpoint{2.067517in}{0.648180in}}%
\pgfpathclose%
\pgfusepath{stroke,fill}%
\end{pgfscope}%
\begin{pgfscope}%
\pgfpathrectangle{\pgfqpoint{0.787074in}{0.548769in}}{\pgfqpoint{5.062926in}{3.102590in}}%
\pgfusepath{clip}%
\pgfsetbuttcap%
\pgfsetroundjoin%
\definecolor{currentfill}{rgb}{0.121569,0.466667,0.705882}%
\pgfsetfillcolor{currentfill}%
\pgfsetlinewidth{1.003750pt}%
\definecolor{currentstroke}{rgb}{0.121569,0.466667,0.705882}%
\pgfsetstrokecolor{currentstroke}%
\pgfsetdash{}{0pt}%
\pgfpathmoveto{\pgfqpoint{1.349487in}{0.648153in}}%
\pgfpathcurveto{\pgfqpoint{1.360537in}{0.648153in}}{\pgfqpoint{1.371136in}{0.652543in}}{\pgfqpoint{1.378950in}{0.660357in}}%
\pgfpathcurveto{\pgfqpoint{1.386763in}{0.668170in}}{\pgfqpoint{1.391154in}{0.678769in}}{\pgfqpoint{1.391154in}{0.689819in}}%
\pgfpathcurveto{\pgfqpoint{1.391154in}{0.700870in}}{\pgfqpoint{1.386763in}{0.711469in}}{\pgfqpoint{1.378950in}{0.719282in}}%
\pgfpathcurveto{\pgfqpoint{1.371136in}{0.727096in}}{\pgfqpoint{1.360537in}{0.731486in}}{\pgfqpoint{1.349487in}{0.731486in}}%
\pgfpathcurveto{\pgfqpoint{1.338437in}{0.731486in}}{\pgfqpoint{1.327838in}{0.727096in}}{\pgfqpoint{1.320024in}{0.719282in}}%
\pgfpathcurveto{\pgfqpoint{1.312210in}{0.711469in}}{\pgfqpoint{1.307820in}{0.700870in}}{\pgfqpoint{1.307820in}{0.689819in}}%
\pgfpathcurveto{\pgfqpoint{1.307820in}{0.678769in}}{\pgfqpoint{1.312210in}{0.668170in}}{\pgfqpoint{1.320024in}{0.660357in}}%
\pgfpathcurveto{\pgfqpoint{1.327838in}{0.652543in}}{\pgfqpoint{1.338437in}{0.648153in}}{\pgfqpoint{1.349487in}{0.648153in}}%
\pgfpathclose%
\pgfusepath{stroke,fill}%
\end{pgfscope}%
\begin{pgfscope}%
\pgfpathrectangle{\pgfqpoint{0.787074in}{0.548769in}}{\pgfqpoint{5.062926in}{3.102590in}}%
\pgfusepath{clip}%
\pgfsetbuttcap%
\pgfsetroundjoin%
\definecolor{currentfill}{rgb}{0.121569,0.466667,0.705882}%
\pgfsetfillcolor{currentfill}%
\pgfsetlinewidth{1.003750pt}%
\definecolor{currentstroke}{rgb}{0.121569,0.466667,0.705882}%
\pgfsetstrokecolor{currentstroke}%
\pgfsetdash{}{0pt}%
\pgfpathmoveto{\pgfqpoint{1.618531in}{0.648456in}}%
\pgfpathcurveto{\pgfqpoint{1.629581in}{0.648456in}}{\pgfqpoint{1.640180in}{0.652846in}}{\pgfqpoint{1.647994in}{0.660660in}}%
\pgfpathcurveto{\pgfqpoint{1.655808in}{0.668474in}}{\pgfqpoint{1.660198in}{0.679073in}}{\pgfqpoint{1.660198in}{0.690123in}}%
\pgfpathcurveto{\pgfqpoint{1.660198in}{0.701173in}}{\pgfqpoint{1.655808in}{0.711772in}}{\pgfqpoint{1.647994in}{0.719586in}}%
\pgfpathcurveto{\pgfqpoint{1.640180in}{0.727399in}}{\pgfqpoint{1.629581in}{0.731789in}}{\pgfqpoint{1.618531in}{0.731789in}}%
\pgfpathcurveto{\pgfqpoint{1.607481in}{0.731789in}}{\pgfqpoint{1.596882in}{0.727399in}}{\pgfqpoint{1.589069in}{0.719586in}}%
\pgfpathcurveto{\pgfqpoint{1.581255in}{0.711772in}}{\pgfqpoint{1.576865in}{0.701173in}}{\pgfqpoint{1.576865in}{0.690123in}}%
\pgfpathcurveto{\pgfqpoint{1.576865in}{0.679073in}}{\pgfqpoint{1.581255in}{0.668474in}}{\pgfqpoint{1.589069in}{0.660660in}}%
\pgfpathcurveto{\pgfqpoint{1.596882in}{0.652846in}}{\pgfqpoint{1.607481in}{0.648456in}}{\pgfqpoint{1.618531in}{0.648456in}}%
\pgfpathclose%
\pgfusepath{stroke,fill}%
\end{pgfscope}%
\begin{pgfscope}%
\pgfpathrectangle{\pgfqpoint{0.787074in}{0.548769in}}{\pgfqpoint{5.062926in}{3.102590in}}%
\pgfusepath{clip}%
\pgfsetbuttcap%
\pgfsetroundjoin%
\definecolor{currentfill}{rgb}{0.121569,0.466667,0.705882}%
\pgfsetfillcolor{currentfill}%
\pgfsetlinewidth{1.003750pt}%
\definecolor{currentstroke}{rgb}{0.121569,0.466667,0.705882}%
\pgfsetstrokecolor{currentstroke}%
\pgfsetdash{}{0pt}%
\pgfpathmoveto{\pgfqpoint{1.322665in}{0.648200in}}%
\pgfpathcurveto{\pgfqpoint{1.333715in}{0.648200in}}{\pgfqpoint{1.344314in}{0.652590in}}{\pgfqpoint{1.352128in}{0.660404in}}%
\pgfpathcurveto{\pgfqpoint{1.359941in}{0.668217in}}{\pgfqpoint{1.364332in}{0.678816in}}{\pgfqpoint{1.364332in}{0.689866in}}%
\pgfpathcurveto{\pgfqpoint{1.364332in}{0.700917in}}{\pgfqpoint{1.359941in}{0.711516in}}{\pgfqpoint{1.352128in}{0.719329in}}%
\pgfpathcurveto{\pgfqpoint{1.344314in}{0.727143in}}{\pgfqpoint{1.333715in}{0.731533in}}{\pgfqpoint{1.322665in}{0.731533in}}%
\pgfpathcurveto{\pgfqpoint{1.311615in}{0.731533in}}{\pgfqpoint{1.301016in}{0.727143in}}{\pgfqpoint{1.293202in}{0.719329in}}%
\pgfpathcurveto{\pgfqpoint{1.285389in}{0.711516in}}{\pgfqpoint{1.280998in}{0.700917in}}{\pgfqpoint{1.280998in}{0.689866in}}%
\pgfpathcurveto{\pgfqpoint{1.280998in}{0.678816in}}{\pgfqpoint{1.285389in}{0.668217in}}{\pgfqpoint{1.293202in}{0.660404in}}%
\pgfpathcurveto{\pgfqpoint{1.301016in}{0.652590in}}{\pgfqpoint{1.311615in}{0.648200in}}{\pgfqpoint{1.322665in}{0.648200in}}%
\pgfpathclose%
\pgfusepath{stroke,fill}%
\end{pgfscope}%
\begin{pgfscope}%
\pgfpathrectangle{\pgfqpoint{0.787074in}{0.548769in}}{\pgfqpoint{5.062926in}{3.102590in}}%
\pgfusepath{clip}%
\pgfsetbuttcap%
\pgfsetroundjoin%
\definecolor{currentfill}{rgb}{0.121569,0.466667,0.705882}%
\pgfsetfillcolor{currentfill}%
\pgfsetlinewidth{1.003750pt}%
\definecolor{currentstroke}{rgb}{0.121569,0.466667,0.705882}%
\pgfsetstrokecolor{currentstroke}%
\pgfsetdash{}{0pt}%
\pgfpathmoveto{\pgfqpoint{1.398530in}{0.648150in}}%
\pgfpathcurveto{\pgfqpoint{1.409580in}{0.648150in}}{\pgfqpoint{1.420179in}{0.652540in}}{\pgfqpoint{1.427993in}{0.660354in}}%
\pgfpathcurveto{\pgfqpoint{1.435807in}{0.668167in}}{\pgfqpoint{1.440197in}{0.678766in}}{\pgfqpoint{1.440197in}{0.689816in}}%
\pgfpathcurveto{\pgfqpoint{1.440197in}{0.700866in}}{\pgfqpoint{1.435807in}{0.711465in}}{\pgfqpoint{1.427993in}{0.719279in}}%
\pgfpathcurveto{\pgfqpoint{1.420179in}{0.727093in}}{\pgfqpoint{1.409580in}{0.731483in}}{\pgfqpoint{1.398530in}{0.731483in}}%
\pgfpathcurveto{\pgfqpoint{1.387480in}{0.731483in}}{\pgfqpoint{1.376881in}{0.727093in}}{\pgfqpoint{1.369068in}{0.719279in}}%
\pgfpathcurveto{\pgfqpoint{1.361254in}{0.711465in}}{\pgfqpoint{1.356864in}{0.700866in}}{\pgfqpoint{1.356864in}{0.689816in}}%
\pgfpathcurveto{\pgfqpoint{1.356864in}{0.678766in}}{\pgfqpoint{1.361254in}{0.668167in}}{\pgfqpoint{1.369068in}{0.660354in}}%
\pgfpathcurveto{\pgfqpoint{1.376881in}{0.652540in}}{\pgfqpoint{1.387480in}{0.648150in}}{\pgfqpoint{1.398530in}{0.648150in}}%
\pgfpathclose%
\pgfusepath{stroke,fill}%
\end{pgfscope}%
\begin{pgfscope}%
\pgfpathrectangle{\pgfqpoint{0.787074in}{0.548769in}}{\pgfqpoint{5.062926in}{3.102590in}}%
\pgfusepath{clip}%
\pgfsetbuttcap%
\pgfsetroundjoin%
\definecolor{currentfill}{rgb}{1.000000,0.498039,0.054902}%
\pgfsetfillcolor{currentfill}%
\pgfsetlinewidth{1.003750pt}%
\definecolor{currentstroke}{rgb}{1.000000,0.498039,0.054902}%
\pgfsetstrokecolor{currentstroke}%
\pgfsetdash{}{0pt}%
\pgfpathmoveto{\pgfqpoint{2.632906in}{3.030917in}}%
\pgfpathcurveto{\pgfqpoint{2.643956in}{3.030917in}}{\pgfqpoint{2.654555in}{3.035308in}}{\pgfqpoint{2.662368in}{3.043121in}}%
\pgfpathcurveto{\pgfqpoint{2.670182in}{3.050935in}}{\pgfqpoint{2.674572in}{3.061534in}}{\pgfqpoint{2.674572in}{3.072584in}}%
\pgfpathcurveto{\pgfqpoint{2.674572in}{3.083634in}}{\pgfqpoint{2.670182in}{3.094233in}}{\pgfqpoint{2.662368in}{3.102047in}}%
\pgfpathcurveto{\pgfqpoint{2.654555in}{3.109861in}}{\pgfqpoint{2.643956in}{3.114251in}}{\pgfqpoint{2.632906in}{3.114251in}}%
\pgfpathcurveto{\pgfqpoint{2.621855in}{3.114251in}}{\pgfqpoint{2.611256in}{3.109861in}}{\pgfqpoint{2.603443in}{3.102047in}}%
\pgfpathcurveto{\pgfqpoint{2.595629in}{3.094233in}}{\pgfqpoint{2.591239in}{3.083634in}}{\pgfqpoint{2.591239in}{3.072584in}}%
\pgfpathcurveto{\pgfqpoint{2.591239in}{3.061534in}}{\pgfqpoint{2.595629in}{3.050935in}}{\pgfqpoint{2.603443in}{3.043121in}}%
\pgfpathcurveto{\pgfqpoint{2.611256in}{3.035308in}}{\pgfqpoint{2.621855in}{3.030917in}}{\pgfqpoint{2.632906in}{3.030917in}}%
\pgfpathclose%
\pgfusepath{stroke,fill}%
\end{pgfscope}%
\begin{pgfscope}%
\pgfpathrectangle{\pgfqpoint{0.787074in}{0.548769in}}{\pgfqpoint{5.062926in}{3.102590in}}%
\pgfusepath{clip}%
\pgfsetbuttcap%
\pgfsetroundjoin%
\definecolor{currentfill}{rgb}{1.000000,0.498039,0.054902}%
\pgfsetfillcolor{currentfill}%
\pgfsetlinewidth{1.003750pt}%
\definecolor{currentstroke}{rgb}{1.000000,0.498039,0.054902}%
\pgfsetstrokecolor{currentstroke}%
\pgfsetdash{}{0pt}%
\pgfpathmoveto{\pgfqpoint{2.108792in}{2.938746in}}%
\pgfpathcurveto{\pgfqpoint{2.119842in}{2.938746in}}{\pgfqpoint{2.130441in}{2.943136in}}{\pgfqpoint{2.138255in}{2.950950in}}%
\pgfpathcurveto{\pgfqpoint{2.146068in}{2.958764in}}{\pgfqpoint{2.150459in}{2.969363in}}{\pgfqpoint{2.150459in}{2.980413in}}%
\pgfpathcurveto{\pgfqpoint{2.150459in}{2.991463in}}{\pgfqpoint{2.146068in}{3.002062in}}{\pgfqpoint{2.138255in}{3.009876in}}%
\pgfpathcurveto{\pgfqpoint{2.130441in}{3.017689in}}{\pgfqpoint{2.119842in}{3.022079in}}{\pgfqpoint{2.108792in}{3.022079in}}%
\pgfpathcurveto{\pgfqpoint{2.097742in}{3.022079in}}{\pgfqpoint{2.087143in}{3.017689in}}{\pgfqpoint{2.079329in}{3.009876in}}%
\pgfpathcurveto{\pgfqpoint{2.071516in}{3.002062in}}{\pgfqpoint{2.067125in}{2.991463in}}{\pgfqpoint{2.067125in}{2.980413in}}%
\pgfpathcurveto{\pgfqpoint{2.067125in}{2.969363in}}{\pgfqpoint{2.071516in}{2.958764in}}{\pgfqpoint{2.079329in}{2.950950in}}%
\pgfpathcurveto{\pgfqpoint{2.087143in}{2.943136in}}{\pgfqpoint{2.097742in}{2.938746in}}{\pgfqpoint{2.108792in}{2.938746in}}%
\pgfpathclose%
\pgfusepath{stroke,fill}%
\end{pgfscope}%
\begin{pgfscope}%
\pgfpathrectangle{\pgfqpoint{0.787074in}{0.548769in}}{\pgfqpoint{5.062926in}{3.102590in}}%
\pgfusepath{clip}%
\pgfsetbuttcap%
\pgfsetroundjoin%
\definecolor{currentfill}{rgb}{1.000000,0.498039,0.054902}%
\pgfsetfillcolor{currentfill}%
\pgfsetlinewidth{1.003750pt}%
\definecolor{currentstroke}{rgb}{1.000000,0.498039,0.054902}%
\pgfsetstrokecolor{currentstroke}%
\pgfsetdash{}{0pt}%
\pgfpathmoveto{\pgfqpoint{1.706332in}{2.706441in}}%
\pgfpathcurveto{\pgfqpoint{1.717382in}{2.706441in}}{\pgfqpoint{1.727981in}{2.710831in}}{\pgfqpoint{1.735795in}{2.718645in}}%
\pgfpathcurveto{\pgfqpoint{1.743608in}{2.726459in}}{\pgfqpoint{1.747999in}{2.737058in}}{\pgfqpoint{1.747999in}{2.748108in}}%
\pgfpathcurveto{\pgfqpoint{1.747999in}{2.759158in}}{\pgfqpoint{1.743608in}{2.769757in}}{\pgfqpoint{1.735795in}{2.777571in}}%
\pgfpathcurveto{\pgfqpoint{1.727981in}{2.785384in}}{\pgfqpoint{1.717382in}{2.789774in}}{\pgfqpoint{1.706332in}{2.789774in}}%
\pgfpathcurveto{\pgfqpoint{1.695282in}{2.789774in}}{\pgfqpoint{1.684683in}{2.785384in}}{\pgfqpoint{1.676869in}{2.777571in}}%
\pgfpathcurveto{\pgfqpoint{1.669056in}{2.769757in}}{\pgfqpoint{1.664665in}{2.759158in}}{\pgfqpoint{1.664665in}{2.748108in}}%
\pgfpathcurveto{\pgfqpoint{1.664665in}{2.737058in}}{\pgfqpoint{1.669056in}{2.726459in}}{\pgfqpoint{1.676869in}{2.718645in}}%
\pgfpathcurveto{\pgfqpoint{1.684683in}{2.710831in}}{\pgfqpoint{1.695282in}{2.706441in}}{\pgfqpoint{1.706332in}{2.706441in}}%
\pgfpathclose%
\pgfusepath{stroke,fill}%
\end{pgfscope}%
\begin{pgfscope}%
\pgfpathrectangle{\pgfqpoint{0.787074in}{0.548769in}}{\pgfqpoint{5.062926in}{3.102590in}}%
\pgfusepath{clip}%
\pgfsetbuttcap%
\pgfsetroundjoin%
\definecolor{currentfill}{rgb}{0.121569,0.466667,0.705882}%
\pgfsetfillcolor{currentfill}%
\pgfsetlinewidth{1.003750pt}%
\definecolor{currentstroke}{rgb}{0.121569,0.466667,0.705882}%
\pgfsetstrokecolor{currentstroke}%
\pgfsetdash{}{0pt}%
\pgfpathmoveto{\pgfqpoint{1.506339in}{0.648133in}}%
\pgfpathcurveto{\pgfqpoint{1.517389in}{0.648133in}}{\pgfqpoint{1.527988in}{0.652523in}}{\pgfqpoint{1.535802in}{0.660337in}}%
\pgfpathcurveto{\pgfqpoint{1.543615in}{0.668150in}}{\pgfqpoint{1.548006in}{0.678749in}}{\pgfqpoint{1.548006in}{0.689799in}}%
\pgfpathcurveto{\pgfqpoint{1.548006in}{0.700849in}}{\pgfqpoint{1.543615in}{0.711448in}}{\pgfqpoint{1.535802in}{0.719262in}}%
\pgfpathcurveto{\pgfqpoint{1.527988in}{0.727076in}}{\pgfqpoint{1.517389in}{0.731466in}}{\pgfqpoint{1.506339in}{0.731466in}}%
\pgfpathcurveto{\pgfqpoint{1.495289in}{0.731466in}}{\pgfqpoint{1.484690in}{0.727076in}}{\pgfqpoint{1.476876in}{0.719262in}}%
\pgfpathcurveto{\pgfqpoint{1.469063in}{0.711448in}}{\pgfqpoint{1.464672in}{0.700849in}}{\pgfqpoint{1.464672in}{0.689799in}}%
\pgfpathcurveto{\pgfqpoint{1.464672in}{0.678749in}}{\pgfqpoint{1.469063in}{0.668150in}}{\pgfqpoint{1.476876in}{0.660337in}}%
\pgfpathcurveto{\pgfqpoint{1.484690in}{0.652523in}}{\pgfqpoint{1.495289in}{0.648133in}}{\pgfqpoint{1.506339in}{0.648133in}}%
\pgfpathclose%
\pgfusepath{stroke,fill}%
\end{pgfscope}%
\begin{pgfscope}%
\pgfpathrectangle{\pgfqpoint{0.787074in}{0.548769in}}{\pgfqpoint{5.062926in}{3.102590in}}%
\pgfusepath{clip}%
\pgfsetbuttcap%
\pgfsetroundjoin%
\definecolor{currentfill}{rgb}{0.121569,0.466667,0.705882}%
\pgfsetfillcolor{currentfill}%
\pgfsetlinewidth{1.003750pt}%
\definecolor{currentstroke}{rgb}{0.121569,0.466667,0.705882}%
\pgfsetstrokecolor{currentstroke}%
\pgfsetdash{}{0pt}%
\pgfpathmoveto{\pgfqpoint{1.383210in}{0.678236in}}%
\pgfpathcurveto{\pgfqpoint{1.394260in}{0.678236in}}{\pgfqpoint{1.404859in}{0.682626in}}{\pgfqpoint{1.412672in}{0.690439in}}%
\pgfpathcurveto{\pgfqpoint{1.420486in}{0.698253in}}{\pgfqpoint{1.424876in}{0.708852in}}{\pgfqpoint{1.424876in}{0.719902in}}%
\pgfpathcurveto{\pgfqpoint{1.424876in}{0.730952in}}{\pgfqpoint{1.420486in}{0.741551in}}{\pgfqpoint{1.412672in}{0.749365in}}%
\pgfpathcurveto{\pgfqpoint{1.404859in}{0.757179in}}{\pgfqpoint{1.394260in}{0.761569in}}{\pgfqpoint{1.383210in}{0.761569in}}%
\pgfpathcurveto{\pgfqpoint{1.372160in}{0.761569in}}{\pgfqpoint{1.361561in}{0.757179in}}{\pgfqpoint{1.353747in}{0.749365in}}%
\pgfpathcurveto{\pgfqpoint{1.345933in}{0.741551in}}{\pgfqpoint{1.341543in}{0.730952in}}{\pgfqpoint{1.341543in}{0.719902in}}%
\pgfpathcurveto{\pgfqpoint{1.341543in}{0.708852in}}{\pgfqpoint{1.345933in}{0.698253in}}{\pgfqpoint{1.353747in}{0.690439in}}%
\pgfpathcurveto{\pgfqpoint{1.361561in}{0.682626in}}{\pgfqpoint{1.372160in}{0.678236in}}{\pgfqpoint{1.383210in}{0.678236in}}%
\pgfpathclose%
\pgfusepath{stroke,fill}%
\end{pgfscope}%
\begin{pgfscope}%
\pgfpathrectangle{\pgfqpoint{0.787074in}{0.548769in}}{\pgfqpoint{5.062926in}{3.102590in}}%
\pgfusepath{clip}%
\pgfsetbuttcap%
\pgfsetroundjoin%
\definecolor{currentfill}{rgb}{1.000000,0.498039,0.054902}%
\pgfsetfillcolor{currentfill}%
\pgfsetlinewidth{1.003750pt}%
\definecolor{currentstroke}{rgb}{1.000000,0.498039,0.054902}%
\pgfsetstrokecolor{currentstroke}%
\pgfsetdash{}{0pt}%
\pgfpathmoveto{\pgfqpoint{1.398530in}{3.011191in}}%
\pgfpathcurveto{\pgfqpoint{1.409580in}{3.011191in}}{\pgfqpoint{1.420179in}{3.015581in}}{\pgfqpoint{1.427993in}{3.023394in}}%
\pgfpathcurveto{\pgfqpoint{1.435807in}{3.031208in}}{\pgfqpoint{1.440197in}{3.041807in}}{\pgfqpoint{1.440197in}{3.052857in}}%
\pgfpathcurveto{\pgfqpoint{1.440197in}{3.063907in}}{\pgfqpoint{1.435807in}{3.074506in}}{\pgfqpoint{1.427993in}{3.082320in}}%
\pgfpathcurveto{\pgfqpoint{1.420179in}{3.090134in}}{\pgfqpoint{1.409580in}{3.094524in}}{\pgfqpoint{1.398530in}{3.094524in}}%
\pgfpathcurveto{\pgfqpoint{1.387480in}{3.094524in}}{\pgfqpoint{1.376881in}{3.090134in}}{\pgfqpoint{1.369068in}{3.082320in}}%
\pgfpathcurveto{\pgfqpoint{1.361254in}{3.074506in}}{\pgfqpoint{1.356864in}{3.063907in}}{\pgfqpoint{1.356864in}{3.052857in}}%
\pgfpathcurveto{\pgfqpoint{1.356864in}{3.041807in}}{\pgfqpoint{1.361254in}{3.031208in}}{\pgfqpoint{1.369068in}{3.023394in}}%
\pgfpathcurveto{\pgfqpoint{1.376881in}{3.015581in}}{\pgfqpoint{1.387480in}{3.011191in}}{\pgfqpoint{1.398530in}{3.011191in}}%
\pgfpathclose%
\pgfusepath{stroke,fill}%
\end{pgfscope}%
\begin{pgfscope}%
\pgfpathrectangle{\pgfqpoint{0.787074in}{0.548769in}}{\pgfqpoint{5.062926in}{3.102590in}}%
\pgfusepath{clip}%
\pgfsetbuttcap%
\pgfsetroundjoin%
\definecolor{currentfill}{rgb}{0.121569,0.466667,0.705882}%
\pgfsetfillcolor{currentfill}%
\pgfsetlinewidth{1.003750pt}%
\definecolor{currentstroke}{rgb}{0.121569,0.466667,0.705882}%
\pgfsetstrokecolor{currentstroke}%
\pgfsetdash{}{0pt}%
\pgfpathmoveto{\pgfqpoint{1.021113in}{0.787280in}}%
\pgfpathcurveto{\pgfqpoint{1.032163in}{0.787280in}}{\pgfqpoint{1.042762in}{0.791670in}}{\pgfqpoint{1.050576in}{0.799484in}}%
\pgfpathcurveto{\pgfqpoint{1.058389in}{0.807297in}}{\pgfqpoint{1.062780in}{0.817896in}}{\pgfqpoint{1.062780in}{0.828946in}}%
\pgfpathcurveto{\pgfqpoint{1.062780in}{0.839997in}}{\pgfqpoint{1.058389in}{0.850596in}}{\pgfqpoint{1.050576in}{0.858409in}}%
\pgfpathcurveto{\pgfqpoint{1.042762in}{0.866223in}}{\pgfqpoint{1.032163in}{0.870613in}}{\pgfqpoint{1.021113in}{0.870613in}}%
\pgfpathcurveto{\pgfqpoint{1.010063in}{0.870613in}}{\pgfqpoint{0.999464in}{0.866223in}}{\pgfqpoint{0.991650in}{0.858409in}}%
\pgfpathcurveto{\pgfqpoint{0.983837in}{0.850596in}}{\pgfqpoint{0.979446in}{0.839997in}}{\pgfqpoint{0.979446in}{0.828946in}}%
\pgfpathcurveto{\pgfqpoint{0.979446in}{0.817896in}}{\pgfqpoint{0.983837in}{0.807297in}}{\pgfqpoint{0.991650in}{0.799484in}}%
\pgfpathcurveto{\pgfqpoint{0.999464in}{0.791670in}}{\pgfqpoint{1.010063in}{0.787280in}}{\pgfqpoint{1.021113in}{0.787280in}}%
\pgfpathclose%
\pgfusepath{stroke,fill}%
\end{pgfscope}%
\begin{pgfscope}%
\pgfpathrectangle{\pgfqpoint{0.787074in}{0.548769in}}{\pgfqpoint{5.062926in}{3.102590in}}%
\pgfusepath{clip}%
\pgfsetbuttcap%
\pgfsetroundjoin%
\definecolor{currentfill}{rgb}{1.000000,0.498039,0.054902}%
\pgfsetfillcolor{currentfill}%
\pgfsetlinewidth{1.003750pt}%
\definecolor{currentstroke}{rgb}{1.000000,0.498039,0.054902}%
\pgfsetstrokecolor{currentstroke}%
\pgfsetdash{}{0pt}%
\pgfpathmoveto{\pgfqpoint{1.883279in}{3.468665in}}%
\pgfpathcurveto{\pgfqpoint{1.894329in}{3.468665in}}{\pgfqpoint{1.904928in}{3.473055in}}{\pgfqpoint{1.912742in}{3.480869in}}%
\pgfpathcurveto{\pgfqpoint{1.920555in}{3.488683in}}{\pgfqpoint{1.924946in}{3.499282in}}{\pgfqpoint{1.924946in}{3.510332in}}%
\pgfpathcurveto{\pgfqpoint{1.924946in}{3.521382in}}{\pgfqpoint{1.920555in}{3.531981in}}{\pgfqpoint{1.912742in}{3.539795in}}%
\pgfpathcurveto{\pgfqpoint{1.904928in}{3.547608in}}{\pgfqpoint{1.894329in}{3.551998in}}{\pgfqpoint{1.883279in}{3.551998in}}%
\pgfpathcurveto{\pgfqpoint{1.872229in}{3.551998in}}{\pgfqpoint{1.861630in}{3.547608in}}{\pgfqpoint{1.853816in}{3.539795in}}%
\pgfpathcurveto{\pgfqpoint{1.846003in}{3.531981in}}{\pgfqpoint{1.841612in}{3.521382in}}{\pgfqpoint{1.841612in}{3.510332in}}%
\pgfpathcurveto{\pgfqpoint{1.841612in}{3.499282in}}{\pgfqpoint{1.846003in}{3.488683in}}{\pgfqpoint{1.853816in}{3.480869in}}%
\pgfpathcurveto{\pgfqpoint{1.861630in}{3.473055in}}{\pgfqpoint{1.872229in}{3.468665in}}{\pgfqpoint{1.883279in}{3.468665in}}%
\pgfpathclose%
\pgfusepath{stroke,fill}%
\end{pgfscope}%
\begin{pgfscope}%
\pgfpathrectangle{\pgfqpoint{0.787074in}{0.548769in}}{\pgfqpoint{5.062926in}{3.102590in}}%
\pgfusepath{clip}%
\pgfsetbuttcap%
\pgfsetroundjoin%
\definecolor{currentfill}{rgb}{0.121569,0.466667,0.705882}%
\pgfsetfillcolor{currentfill}%
\pgfsetlinewidth{1.003750pt}%
\definecolor{currentstroke}{rgb}{0.121569,0.466667,0.705882}%
\pgfsetstrokecolor{currentstroke}%
\pgfsetdash{}{0pt}%
\pgfpathmoveto{\pgfqpoint{1.188468in}{0.649989in}}%
\pgfpathcurveto{\pgfqpoint{1.199518in}{0.649989in}}{\pgfqpoint{1.210117in}{0.654379in}}{\pgfqpoint{1.217931in}{0.662193in}}%
\pgfpathcurveto{\pgfqpoint{1.225745in}{0.670006in}}{\pgfqpoint{1.230135in}{0.680605in}}{\pgfqpoint{1.230135in}{0.691655in}}%
\pgfpathcurveto{\pgfqpoint{1.230135in}{0.702706in}}{\pgfqpoint{1.225745in}{0.713305in}}{\pgfqpoint{1.217931in}{0.721118in}}%
\pgfpathcurveto{\pgfqpoint{1.210117in}{0.728932in}}{\pgfqpoint{1.199518in}{0.733322in}}{\pgfqpoint{1.188468in}{0.733322in}}%
\pgfpathcurveto{\pgfqpoint{1.177418in}{0.733322in}}{\pgfqpoint{1.166819in}{0.728932in}}{\pgfqpoint{1.159005in}{0.721118in}}%
\pgfpathcurveto{\pgfqpoint{1.151192in}{0.713305in}}{\pgfqpoint{1.146802in}{0.702706in}}{\pgfqpoint{1.146802in}{0.691655in}}%
\pgfpathcurveto{\pgfqpoint{1.146802in}{0.680605in}}{\pgfqpoint{1.151192in}{0.670006in}}{\pgfqpoint{1.159005in}{0.662193in}}%
\pgfpathcurveto{\pgfqpoint{1.166819in}{0.654379in}}{\pgfqpoint{1.177418in}{0.649989in}}{\pgfqpoint{1.188468in}{0.649989in}}%
\pgfpathclose%
\pgfusepath{stroke,fill}%
\end{pgfscope}%
\begin{pgfscope}%
\pgfpathrectangle{\pgfqpoint{0.787074in}{0.548769in}}{\pgfqpoint{5.062926in}{3.102590in}}%
\pgfusepath{clip}%
\pgfsetbuttcap%
\pgfsetroundjoin%
\definecolor{currentfill}{rgb}{1.000000,0.498039,0.054902}%
\pgfsetfillcolor{currentfill}%
\pgfsetlinewidth{1.003750pt}%
\definecolor{currentstroke}{rgb}{1.000000,0.498039,0.054902}%
\pgfsetstrokecolor{currentstroke}%
\pgfsetdash{}{0pt}%
\pgfpathmoveto{\pgfqpoint{1.853679in}{2.212517in}}%
\pgfpathcurveto{\pgfqpoint{1.864729in}{2.212517in}}{\pgfqpoint{1.875328in}{2.216907in}}{\pgfqpoint{1.883142in}{2.224721in}}%
\pgfpathcurveto{\pgfqpoint{1.890956in}{2.232535in}}{\pgfqpoint{1.895346in}{2.243134in}}{\pgfqpoint{1.895346in}{2.254184in}}%
\pgfpathcurveto{\pgfqpoint{1.895346in}{2.265234in}}{\pgfqpoint{1.890956in}{2.275833in}}{\pgfqpoint{1.883142in}{2.283647in}}%
\pgfpathcurveto{\pgfqpoint{1.875328in}{2.291460in}}{\pgfqpoint{1.864729in}{2.295850in}}{\pgfqpoint{1.853679in}{2.295850in}}%
\pgfpathcurveto{\pgfqpoint{1.842629in}{2.295850in}}{\pgfqpoint{1.832030in}{2.291460in}}{\pgfqpoint{1.824217in}{2.283647in}}%
\pgfpathcurveto{\pgfqpoint{1.816403in}{2.275833in}}{\pgfqpoint{1.812013in}{2.265234in}}{\pgfqpoint{1.812013in}{2.254184in}}%
\pgfpathcurveto{\pgfqpoint{1.812013in}{2.243134in}}{\pgfqpoint{1.816403in}{2.232535in}}{\pgfqpoint{1.824217in}{2.224721in}}%
\pgfpathcurveto{\pgfqpoint{1.832030in}{2.216907in}}{\pgfqpoint{1.842629in}{2.212517in}}{\pgfqpoint{1.853679in}{2.212517in}}%
\pgfpathclose%
\pgfusepath{stroke,fill}%
\end{pgfscope}%
\begin{pgfscope}%
\pgfpathrectangle{\pgfqpoint{0.787074in}{0.548769in}}{\pgfqpoint{5.062926in}{3.102590in}}%
\pgfusepath{clip}%
\pgfsetbuttcap%
\pgfsetroundjoin%
\definecolor{currentfill}{rgb}{0.121569,0.466667,0.705882}%
\pgfsetfillcolor{currentfill}%
\pgfsetlinewidth{1.003750pt}%
\definecolor{currentstroke}{rgb}{0.121569,0.466667,0.705882}%
\pgfsetstrokecolor{currentstroke}%
\pgfsetdash{}{0pt}%
\pgfpathmoveto{\pgfqpoint{2.339773in}{0.651678in}}%
\pgfpathcurveto{\pgfqpoint{2.350824in}{0.651678in}}{\pgfqpoint{2.361423in}{0.656068in}}{\pgfqpoint{2.369236in}{0.663882in}}%
\pgfpathcurveto{\pgfqpoint{2.377050in}{0.671695in}}{\pgfqpoint{2.381440in}{0.682295in}}{\pgfqpoint{2.381440in}{0.693345in}}%
\pgfpathcurveto{\pgfqpoint{2.381440in}{0.704395in}}{\pgfqpoint{2.377050in}{0.714994in}}{\pgfqpoint{2.369236in}{0.722807in}}%
\pgfpathcurveto{\pgfqpoint{2.361423in}{0.730621in}}{\pgfqpoint{2.350824in}{0.735011in}}{\pgfqpoint{2.339773in}{0.735011in}}%
\pgfpathcurveto{\pgfqpoint{2.328723in}{0.735011in}}{\pgfqpoint{2.318124in}{0.730621in}}{\pgfqpoint{2.310311in}{0.722807in}}%
\pgfpathcurveto{\pgfqpoint{2.302497in}{0.714994in}}{\pgfqpoint{2.298107in}{0.704395in}}{\pgfqpoint{2.298107in}{0.693345in}}%
\pgfpathcurveto{\pgfqpoint{2.298107in}{0.682295in}}{\pgfqpoint{2.302497in}{0.671695in}}{\pgfqpoint{2.310311in}{0.663882in}}%
\pgfpathcurveto{\pgfqpoint{2.318124in}{0.656068in}}{\pgfqpoint{2.328723in}{0.651678in}}{\pgfqpoint{2.339773in}{0.651678in}}%
\pgfpathclose%
\pgfusepath{stroke,fill}%
\end{pgfscope}%
\begin{pgfscope}%
\pgfpathrectangle{\pgfqpoint{0.787074in}{0.548769in}}{\pgfqpoint{5.062926in}{3.102590in}}%
\pgfusepath{clip}%
\pgfsetbuttcap%
\pgfsetroundjoin%
\definecolor{currentfill}{rgb}{1.000000,0.498039,0.054902}%
\pgfsetfillcolor{currentfill}%
\pgfsetlinewidth{1.003750pt}%
\definecolor{currentstroke}{rgb}{1.000000,0.498039,0.054902}%
\pgfsetstrokecolor{currentstroke}%
\pgfsetdash{}{0pt}%
\pgfpathmoveto{\pgfqpoint{1.468884in}{2.584775in}}%
\pgfpathcurveto{\pgfqpoint{1.479934in}{2.584775in}}{\pgfqpoint{1.490533in}{2.589166in}}{\pgfqpoint{1.498347in}{2.596979in}}%
\pgfpathcurveto{\pgfqpoint{1.506160in}{2.604793in}}{\pgfqpoint{1.510550in}{2.615392in}}{\pgfqpoint{1.510550in}{2.626442in}}%
\pgfpathcurveto{\pgfqpoint{1.510550in}{2.637492in}}{\pgfqpoint{1.506160in}{2.648091in}}{\pgfqpoint{1.498347in}{2.655905in}}%
\pgfpathcurveto{\pgfqpoint{1.490533in}{2.663718in}}{\pgfqpoint{1.479934in}{2.668109in}}{\pgfqpoint{1.468884in}{2.668109in}}%
\pgfpathcurveto{\pgfqpoint{1.457834in}{2.668109in}}{\pgfqpoint{1.447235in}{2.663718in}}{\pgfqpoint{1.439421in}{2.655905in}}%
\pgfpathcurveto{\pgfqpoint{1.431607in}{2.648091in}}{\pgfqpoint{1.427217in}{2.637492in}}{\pgfqpoint{1.427217in}{2.626442in}}%
\pgfpathcurveto{\pgfqpoint{1.427217in}{2.615392in}}{\pgfqpoint{1.431607in}{2.604793in}}{\pgfqpoint{1.439421in}{2.596979in}}%
\pgfpathcurveto{\pgfqpoint{1.447235in}{2.589166in}}{\pgfqpoint{1.457834in}{2.584775in}}{\pgfqpoint{1.468884in}{2.584775in}}%
\pgfpathclose%
\pgfusepath{stroke,fill}%
\end{pgfscope}%
\begin{pgfscope}%
\pgfpathrectangle{\pgfqpoint{0.787074in}{0.548769in}}{\pgfqpoint{5.062926in}{3.102590in}}%
\pgfusepath{clip}%
\pgfsetbuttcap%
\pgfsetroundjoin%
\definecolor{currentfill}{rgb}{0.121569,0.466667,0.705882}%
\pgfsetfillcolor{currentfill}%
\pgfsetlinewidth{1.003750pt}%
\definecolor{currentstroke}{rgb}{0.121569,0.466667,0.705882}%
\pgfsetstrokecolor{currentstroke}%
\pgfsetdash{}{0pt}%
\pgfpathmoveto{\pgfqpoint{1.923425in}{0.648158in}}%
\pgfpathcurveto{\pgfqpoint{1.934475in}{0.648158in}}{\pgfqpoint{1.945074in}{0.652548in}}{\pgfqpoint{1.952888in}{0.660362in}}%
\pgfpathcurveto{\pgfqpoint{1.960702in}{0.668176in}}{\pgfqpoint{1.965092in}{0.678775in}}{\pgfqpoint{1.965092in}{0.689825in}}%
\pgfpathcurveto{\pgfqpoint{1.965092in}{0.700875in}}{\pgfqpoint{1.960702in}{0.711474in}}{\pgfqpoint{1.952888in}{0.719288in}}%
\pgfpathcurveto{\pgfqpoint{1.945074in}{0.727101in}}{\pgfqpoint{1.934475in}{0.731492in}}{\pgfqpoint{1.923425in}{0.731492in}}%
\pgfpathcurveto{\pgfqpoint{1.912375in}{0.731492in}}{\pgfqpoint{1.901776in}{0.727101in}}{\pgfqpoint{1.893962in}{0.719288in}}%
\pgfpathcurveto{\pgfqpoint{1.886149in}{0.711474in}}{\pgfqpoint{1.881758in}{0.700875in}}{\pgfqpoint{1.881758in}{0.689825in}}%
\pgfpathcurveto{\pgfqpoint{1.881758in}{0.678775in}}{\pgfqpoint{1.886149in}{0.668176in}}{\pgfqpoint{1.893962in}{0.660362in}}%
\pgfpathcurveto{\pgfqpoint{1.901776in}{0.652548in}}{\pgfqpoint{1.912375in}{0.648158in}}{\pgfqpoint{1.923425in}{0.648158in}}%
\pgfpathclose%
\pgfusepath{stroke,fill}%
\end{pgfscope}%
\begin{pgfscope}%
\pgfpathrectangle{\pgfqpoint{0.787074in}{0.548769in}}{\pgfqpoint{5.062926in}{3.102590in}}%
\pgfusepath{clip}%
\pgfsetbuttcap%
\pgfsetroundjoin%
\definecolor{currentfill}{rgb}{1.000000,0.498039,0.054902}%
\pgfsetfillcolor{currentfill}%
\pgfsetlinewidth{1.003750pt}%
\definecolor{currentstroke}{rgb}{1.000000,0.498039,0.054902}%
\pgfsetstrokecolor{currentstroke}%
\pgfsetdash{}{0pt}%
\pgfpathmoveto{\pgfqpoint{1.165335in}{2.981281in}}%
\pgfpathcurveto{\pgfqpoint{1.176385in}{2.981281in}}{\pgfqpoint{1.186985in}{2.985671in}}{\pgfqpoint{1.194798in}{2.993484in}}%
\pgfpathcurveto{\pgfqpoint{1.202612in}{3.001298in}}{\pgfqpoint{1.207002in}{3.011897in}}{\pgfqpoint{1.207002in}{3.022947in}}%
\pgfpathcurveto{\pgfqpoint{1.207002in}{3.033997in}}{\pgfqpoint{1.202612in}{3.044596in}}{\pgfqpoint{1.194798in}{3.052410in}}%
\pgfpathcurveto{\pgfqpoint{1.186985in}{3.060224in}}{\pgfqpoint{1.176385in}{3.064614in}}{\pgfqpoint{1.165335in}{3.064614in}}%
\pgfpathcurveto{\pgfqpoint{1.154285in}{3.064614in}}{\pgfqpoint{1.143686in}{3.060224in}}{\pgfqpoint{1.135873in}{3.052410in}}%
\pgfpathcurveto{\pgfqpoint{1.128059in}{3.044596in}}{\pgfqpoint{1.123669in}{3.033997in}}{\pgfqpoint{1.123669in}{3.022947in}}%
\pgfpathcurveto{\pgfqpoint{1.123669in}{3.011897in}}{\pgfqpoint{1.128059in}{3.001298in}}{\pgfqpoint{1.135873in}{2.993484in}}%
\pgfpathcurveto{\pgfqpoint{1.143686in}{2.985671in}}{\pgfqpoint{1.154285in}{2.981281in}}{\pgfqpoint{1.165335in}{2.981281in}}%
\pgfpathclose%
\pgfusepath{stroke,fill}%
\end{pgfscope}%
\begin{pgfscope}%
\pgfpathrectangle{\pgfqpoint{0.787074in}{0.548769in}}{\pgfqpoint{5.062926in}{3.102590in}}%
\pgfusepath{clip}%
\pgfsetbuttcap%
\pgfsetroundjoin%
\definecolor{currentfill}{rgb}{0.121569,0.466667,0.705882}%
\pgfsetfillcolor{currentfill}%
\pgfsetlinewidth{1.003750pt}%
\definecolor{currentstroke}{rgb}{0.121569,0.466667,0.705882}%
\pgfsetstrokecolor{currentstroke}%
\pgfsetdash{}{0pt}%
\pgfpathmoveto{\pgfqpoint{1.527345in}{0.648132in}}%
\pgfpathcurveto{\pgfqpoint{1.538395in}{0.648132in}}{\pgfqpoint{1.548994in}{0.652522in}}{\pgfqpoint{1.556808in}{0.660336in}}%
\pgfpathcurveto{\pgfqpoint{1.564622in}{0.668149in}}{\pgfqpoint{1.569012in}{0.678749in}}{\pgfqpoint{1.569012in}{0.689799in}}%
\pgfpathcurveto{\pgfqpoint{1.569012in}{0.700849in}}{\pgfqpoint{1.564622in}{0.711448in}}{\pgfqpoint{1.556808in}{0.719261in}}%
\pgfpathcurveto{\pgfqpoint{1.548994in}{0.727075in}}{\pgfqpoint{1.538395in}{0.731465in}}{\pgfqpoint{1.527345in}{0.731465in}}%
\pgfpathcurveto{\pgfqpoint{1.516295in}{0.731465in}}{\pgfqpoint{1.505696in}{0.727075in}}{\pgfqpoint{1.497882in}{0.719261in}}%
\pgfpathcurveto{\pgfqpoint{1.490069in}{0.711448in}}{\pgfqpoint{1.485679in}{0.700849in}}{\pgfqpoint{1.485679in}{0.689799in}}%
\pgfpathcurveto{\pgfqpoint{1.485679in}{0.678749in}}{\pgfqpoint{1.490069in}{0.668149in}}{\pgfqpoint{1.497882in}{0.660336in}}%
\pgfpathcurveto{\pgfqpoint{1.505696in}{0.652522in}}{\pgfqpoint{1.516295in}{0.648132in}}{\pgfqpoint{1.527345in}{0.648132in}}%
\pgfpathclose%
\pgfusepath{stroke,fill}%
\end{pgfscope}%
\begin{pgfscope}%
\pgfpathrectangle{\pgfqpoint{0.787074in}{0.548769in}}{\pgfqpoint{5.062926in}{3.102590in}}%
\pgfusepath{clip}%
\pgfsetbuttcap%
\pgfsetroundjoin%
\definecolor{currentfill}{rgb}{1.000000,0.498039,0.054902}%
\pgfsetfillcolor{currentfill}%
\pgfsetlinewidth{1.003750pt}%
\definecolor{currentstroke}{rgb}{1.000000,0.498039,0.054902}%
\pgfsetstrokecolor{currentstroke}%
\pgfsetdash{}{0pt}%
\pgfpathmoveto{\pgfqpoint{1.524481in}{2.254043in}}%
\pgfpathcurveto{\pgfqpoint{1.535531in}{2.254043in}}{\pgfqpoint{1.546130in}{2.258433in}}{\pgfqpoint{1.553944in}{2.266247in}}%
\pgfpathcurveto{\pgfqpoint{1.561757in}{2.274060in}}{\pgfqpoint{1.566147in}{2.284659in}}{\pgfqpoint{1.566147in}{2.295709in}}%
\pgfpathcurveto{\pgfqpoint{1.566147in}{2.306759in}}{\pgfqpoint{1.561757in}{2.317359in}}{\pgfqpoint{1.553944in}{2.325172in}}%
\pgfpathcurveto{\pgfqpoint{1.546130in}{2.332986in}}{\pgfqpoint{1.535531in}{2.337376in}}{\pgfqpoint{1.524481in}{2.337376in}}%
\pgfpathcurveto{\pgfqpoint{1.513431in}{2.337376in}}{\pgfqpoint{1.502832in}{2.332986in}}{\pgfqpoint{1.495018in}{2.325172in}}%
\pgfpathcurveto{\pgfqpoint{1.487204in}{2.317359in}}{\pgfqpoint{1.482814in}{2.306759in}}{\pgfqpoint{1.482814in}{2.295709in}}%
\pgfpathcurveto{\pgfqpoint{1.482814in}{2.284659in}}{\pgfqpoint{1.487204in}{2.274060in}}{\pgfqpoint{1.495018in}{2.266247in}}%
\pgfpathcurveto{\pgfqpoint{1.502832in}{2.258433in}}{\pgfqpoint{1.513431in}{2.254043in}}{\pgfqpoint{1.524481in}{2.254043in}}%
\pgfpathclose%
\pgfusepath{stroke,fill}%
\end{pgfscope}%
\begin{pgfscope}%
\pgfpathrectangle{\pgfqpoint{0.787074in}{0.548769in}}{\pgfqpoint{5.062926in}{3.102590in}}%
\pgfusepath{clip}%
\pgfsetbuttcap%
\pgfsetroundjoin%
\definecolor{currentfill}{rgb}{1.000000,0.498039,0.054902}%
\pgfsetfillcolor{currentfill}%
\pgfsetlinewidth{1.003750pt}%
\definecolor{currentstroke}{rgb}{1.000000,0.498039,0.054902}%
\pgfsetstrokecolor{currentstroke}%
\pgfsetdash{}{0pt}%
\pgfpathmoveto{\pgfqpoint{2.096640in}{2.386212in}}%
\pgfpathcurveto{\pgfqpoint{2.107690in}{2.386212in}}{\pgfqpoint{2.118289in}{2.390603in}}{\pgfqpoint{2.126102in}{2.398416in}}%
\pgfpathcurveto{\pgfqpoint{2.133916in}{2.406230in}}{\pgfqpoint{2.138306in}{2.416829in}}{\pgfqpoint{2.138306in}{2.427879in}}%
\pgfpathcurveto{\pgfqpoint{2.138306in}{2.438929in}}{\pgfqpoint{2.133916in}{2.449528in}}{\pgfqpoint{2.126102in}{2.457342in}}%
\pgfpathcurveto{\pgfqpoint{2.118289in}{2.465155in}}{\pgfqpoint{2.107690in}{2.469546in}}{\pgfqpoint{2.096640in}{2.469546in}}%
\pgfpathcurveto{\pgfqpoint{2.085589in}{2.469546in}}{\pgfqpoint{2.074990in}{2.465155in}}{\pgfqpoint{2.067177in}{2.457342in}}%
\pgfpathcurveto{\pgfqpoint{2.059363in}{2.449528in}}{\pgfqpoint{2.054973in}{2.438929in}}{\pgfqpoint{2.054973in}{2.427879in}}%
\pgfpathcurveto{\pgfqpoint{2.054973in}{2.416829in}}{\pgfqpoint{2.059363in}{2.406230in}}{\pgfqpoint{2.067177in}{2.398416in}}%
\pgfpathcurveto{\pgfqpoint{2.074990in}{2.390603in}}{\pgfqpoint{2.085589in}{2.386212in}}{\pgfqpoint{2.096640in}{2.386212in}}%
\pgfpathclose%
\pgfusepath{stroke,fill}%
\end{pgfscope}%
\begin{pgfscope}%
\pgfpathrectangle{\pgfqpoint{0.787074in}{0.548769in}}{\pgfqpoint{5.062926in}{3.102590in}}%
\pgfusepath{clip}%
\pgfsetbuttcap%
\pgfsetroundjoin%
\definecolor{currentfill}{rgb}{0.121569,0.466667,0.705882}%
\pgfsetfillcolor{currentfill}%
\pgfsetlinewidth{1.003750pt}%
\definecolor{currentstroke}{rgb}{0.121569,0.466667,0.705882}%
\pgfsetstrokecolor{currentstroke}%
\pgfsetdash{}{0pt}%
\pgfpathmoveto{\pgfqpoint{1.740011in}{0.648147in}}%
\pgfpathcurveto{\pgfqpoint{1.751062in}{0.648147in}}{\pgfqpoint{1.761661in}{0.652537in}}{\pgfqpoint{1.769474in}{0.660351in}}%
\pgfpathcurveto{\pgfqpoint{1.777288in}{0.668165in}}{\pgfqpoint{1.781678in}{0.678764in}}{\pgfqpoint{1.781678in}{0.689814in}}%
\pgfpathcurveto{\pgfqpoint{1.781678in}{0.700864in}}{\pgfqpoint{1.777288in}{0.711463in}}{\pgfqpoint{1.769474in}{0.719277in}}%
\pgfpathcurveto{\pgfqpoint{1.761661in}{0.727090in}}{\pgfqpoint{1.751062in}{0.731480in}}{\pgfqpoint{1.740011in}{0.731480in}}%
\pgfpathcurveto{\pgfqpoint{1.728961in}{0.731480in}}{\pgfqpoint{1.718362in}{0.727090in}}{\pgfqpoint{1.710549in}{0.719277in}}%
\pgfpathcurveto{\pgfqpoint{1.702735in}{0.711463in}}{\pgfqpoint{1.698345in}{0.700864in}}{\pgfqpoint{1.698345in}{0.689814in}}%
\pgfpathcurveto{\pgfqpoint{1.698345in}{0.678764in}}{\pgfqpoint{1.702735in}{0.668165in}}{\pgfqpoint{1.710549in}{0.660351in}}%
\pgfpathcurveto{\pgfqpoint{1.718362in}{0.652537in}}{\pgfqpoint{1.728961in}{0.648147in}}{\pgfqpoint{1.740011in}{0.648147in}}%
\pgfpathclose%
\pgfusepath{stroke,fill}%
\end{pgfscope}%
\begin{pgfscope}%
\pgfpathrectangle{\pgfqpoint{0.787074in}{0.548769in}}{\pgfqpoint{5.062926in}{3.102590in}}%
\pgfusepath{clip}%
\pgfsetbuttcap%
\pgfsetroundjoin%
\definecolor{currentfill}{rgb}{1.000000,0.498039,0.054902}%
\pgfsetfillcolor{currentfill}%
\pgfsetlinewidth{1.003750pt}%
\definecolor{currentstroke}{rgb}{1.000000,0.498039,0.054902}%
\pgfsetstrokecolor{currentstroke}%
\pgfsetdash{}{0pt}%
\pgfpathmoveto{\pgfqpoint{1.433034in}{2.590063in}}%
\pgfpathcurveto{\pgfqpoint{1.444084in}{2.590063in}}{\pgfqpoint{1.454683in}{2.594453in}}{\pgfqpoint{1.462497in}{2.602267in}}%
\pgfpathcurveto{\pgfqpoint{1.470311in}{2.610081in}}{\pgfqpoint{1.474701in}{2.620680in}}{\pgfqpoint{1.474701in}{2.631730in}}%
\pgfpathcurveto{\pgfqpoint{1.474701in}{2.642780in}}{\pgfqpoint{1.470311in}{2.653379in}}{\pgfqpoint{1.462497in}{2.661193in}}%
\pgfpathcurveto{\pgfqpoint{1.454683in}{2.669006in}}{\pgfqpoint{1.444084in}{2.673397in}}{\pgfqpoint{1.433034in}{2.673397in}}%
\pgfpathcurveto{\pgfqpoint{1.421984in}{2.673397in}}{\pgfqpoint{1.411385in}{2.669006in}}{\pgfqpoint{1.403572in}{2.661193in}}%
\pgfpathcurveto{\pgfqpoint{1.395758in}{2.653379in}}{\pgfqpoint{1.391368in}{2.642780in}}{\pgfqpoint{1.391368in}{2.631730in}}%
\pgfpathcurveto{\pgfqpoint{1.391368in}{2.620680in}}{\pgfqpoint{1.395758in}{2.610081in}}{\pgfqpoint{1.403572in}{2.602267in}}%
\pgfpathcurveto{\pgfqpoint{1.411385in}{2.594453in}}{\pgfqpoint{1.421984in}{2.590063in}}{\pgfqpoint{1.433034in}{2.590063in}}%
\pgfpathclose%
\pgfusepath{stroke,fill}%
\end{pgfscope}%
\begin{pgfscope}%
\pgfpathrectangle{\pgfqpoint{0.787074in}{0.548769in}}{\pgfqpoint{5.062926in}{3.102590in}}%
\pgfusepath{clip}%
\pgfsetbuttcap%
\pgfsetroundjoin%
\definecolor{currentfill}{rgb}{1.000000,0.498039,0.054902}%
\pgfsetfillcolor{currentfill}%
\pgfsetlinewidth{1.003750pt}%
\definecolor{currentstroke}{rgb}{1.000000,0.498039,0.054902}%
\pgfsetstrokecolor{currentstroke}%
\pgfsetdash{}{0pt}%
\pgfpathmoveto{\pgfqpoint{1.077057in}{2.389366in}}%
\pgfpathcurveto{\pgfqpoint{1.088107in}{2.389366in}}{\pgfqpoint{1.098706in}{2.393756in}}{\pgfqpoint{1.106520in}{2.401570in}}%
\pgfpathcurveto{\pgfqpoint{1.114334in}{2.409383in}}{\pgfqpoint{1.118724in}{2.419982in}}{\pgfqpoint{1.118724in}{2.431032in}}%
\pgfpathcurveto{\pgfqpoint{1.118724in}{2.442082in}}{\pgfqpoint{1.114334in}{2.452682in}}{\pgfqpoint{1.106520in}{2.460495in}}%
\pgfpathcurveto{\pgfqpoint{1.098706in}{2.468309in}}{\pgfqpoint{1.088107in}{2.472699in}}{\pgfqpoint{1.077057in}{2.472699in}}%
\pgfpathcurveto{\pgfqpoint{1.066007in}{2.472699in}}{\pgfqpoint{1.055408in}{2.468309in}}{\pgfqpoint{1.047594in}{2.460495in}}%
\pgfpathcurveto{\pgfqpoint{1.039781in}{2.452682in}}{\pgfqpoint{1.035391in}{2.442082in}}{\pgfqpoint{1.035391in}{2.431032in}}%
\pgfpathcurveto{\pgfqpoint{1.035391in}{2.419982in}}{\pgfqpoint{1.039781in}{2.409383in}}{\pgfqpoint{1.047594in}{2.401570in}}%
\pgfpathcurveto{\pgfqpoint{1.055408in}{2.393756in}}{\pgfqpoint{1.066007in}{2.389366in}}{\pgfqpoint{1.077057in}{2.389366in}}%
\pgfpathclose%
\pgfusepath{stroke,fill}%
\end{pgfscope}%
\begin{pgfscope}%
\pgfpathrectangle{\pgfqpoint{0.787074in}{0.548769in}}{\pgfqpoint{5.062926in}{3.102590in}}%
\pgfusepath{clip}%
\pgfsetbuttcap%
\pgfsetroundjoin%
\definecolor{currentfill}{rgb}{0.121569,0.466667,0.705882}%
\pgfsetfillcolor{currentfill}%
\pgfsetlinewidth{1.003750pt}%
\definecolor{currentstroke}{rgb}{0.121569,0.466667,0.705882}%
\pgfsetstrokecolor{currentstroke}%
\pgfsetdash{}{0pt}%
\pgfpathmoveto{\pgfqpoint{1.841136in}{0.658967in}}%
\pgfpathcurveto{\pgfqpoint{1.852186in}{0.658967in}}{\pgfqpoint{1.862786in}{0.663357in}}{\pgfqpoint{1.870599in}{0.671170in}}%
\pgfpathcurveto{\pgfqpoint{1.878413in}{0.678984in}}{\pgfqpoint{1.882803in}{0.689583in}}{\pgfqpoint{1.882803in}{0.700633in}}%
\pgfpathcurveto{\pgfqpoint{1.882803in}{0.711683in}}{\pgfqpoint{1.878413in}{0.722282in}}{\pgfqpoint{1.870599in}{0.730096in}}%
\pgfpathcurveto{\pgfqpoint{1.862786in}{0.737910in}}{\pgfqpoint{1.852186in}{0.742300in}}{\pgfqpoint{1.841136in}{0.742300in}}%
\pgfpathcurveto{\pgfqpoint{1.830086in}{0.742300in}}{\pgfqpoint{1.819487in}{0.737910in}}{\pgfqpoint{1.811674in}{0.730096in}}%
\pgfpathcurveto{\pgfqpoint{1.803860in}{0.722282in}}{\pgfqpoint{1.799470in}{0.711683in}}{\pgfqpoint{1.799470in}{0.700633in}}%
\pgfpathcurveto{\pgfqpoint{1.799470in}{0.689583in}}{\pgfqpoint{1.803860in}{0.678984in}}{\pgfqpoint{1.811674in}{0.671170in}}%
\pgfpathcurveto{\pgfqpoint{1.819487in}{0.663357in}}{\pgfqpoint{1.830086in}{0.658967in}}{\pgfqpoint{1.841136in}{0.658967in}}%
\pgfpathclose%
\pgfusepath{stroke,fill}%
\end{pgfscope}%
\begin{pgfscope}%
\pgfpathrectangle{\pgfqpoint{0.787074in}{0.548769in}}{\pgfqpoint{5.062926in}{3.102590in}}%
\pgfusepath{clip}%
\pgfsetbuttcap%
\pgfsetroundjoin%
\definecolor{currentfill}{rgb}{1.000000,0.498039,0.054902}%
\pgfsetfillcolor{currentfill}%
\pgfsetlinewidth{1.003750pt}%
\definecolor{currentstroke}{rgb}{1.000000,0.498039,0.054902}%
\pgfsetstrokecolor{currentstroke}%
\pgfsetdash{}{0pt}%
\pgfpathmoveto{\pgfqpoint{2.560729in}{2.623396in}}%
\pgfpathcurveto{\pgfqpoint{2.571779in}{2.623396in}}{\pgfqpoint{2.582378in}{2.627787in}}{\pgfqpoint{2.590192in}{2.635600in}}%
\pgfpathcurveto{\pgfqpoint{2.598006in}{2.643414in}}{\pgfqpoint{2.602396in}{2.654013in}}{\pgfqpoint{2.602396in}{2.665063in}}%
\pgfpathcurveto{\pgfqpoint{2.602396in}{2.676113in}}{\pgfqpoint{2.598006in}{2.686712in}}{\pgfqpoint{2.590192in}{2.694526in}}%
\pgfpathcurveto{\pgfqpoint{2.582378in}{2.702339in}}{\pgfqpoint{2.571779in}{2.706730in}}{\pgfqpoint{2.560729in}{2.706730in}}%
\pgfpathcurveto{\pgfqpoint{2.549679in}{2.706730in}}{\pgfqpoint{2.539080in}{2.702339in}}{\pgfqpoint{2.531266in}{2.694526in}}%
\pgfpathcurveto{\pgfqpoint{2.523453in}{2.686712in}}{\pgfqpoint{2.519063in}{2.676113in}}{\pgfqpoint{2.519063in}{2.665063in}}%
\pgfpathcurveto{\pgfqpoint{2.519063in}{2.654013in}}{\pgfqpoint{2.523453in}{2.643414in}}{\pgfqpoint{2.531266in}{2.635600in}}%
\pgfpathcurveto{\pgfqpoint{2.539080in}{2.627787in}}{\pgfqpoint{2.549679in}{2.623396in}}{\pgfqpoint{2.560729in}{2.623396in}}%
\pgfpathclose%
\pgfusepath{stroke,fill}%
\end{pgfscope}%
\begin{pgfscope}%
\pgfpathrectangle{\pgfqpoint{0.787074in}{0.548769in}}{\pgfqpoint{5.062926in}{3.102590in}}%
\pgfusepath{clip}%
\pgfsetbuttcap%
\pgfsetroundjoin%
\definecolor{currentfill}{rgb}{1.000000,0.498039,0.054902}%
\pgfsetfillcolor{currentfill}%
\pgfsetlinewidth{1.003750pt}%
\definecolor{currentstroke}{rgb}{1.000000,0.498039,0.054902}%
\pgfsetstrokecolor{currentstroke}%
\pgfsetdash{}{0pt}%
\pgfpathmoveto{\pgfqpoint{1.747607in}{2.816554in}}%
\pgfpathcurveto{\pgfqpoint{1.758657in}{2.816554in}}{\pgfqpoint{1.769256in}{2.820944in}}{\pgfqpoint{1.777069in}{2.828758in}}%
\pgfpathcurveto{\pgfqpoint{1.784883in}{2.836572in}}{\pgfqpoint{1.789273in}{2.847171in}}{\pgfqpoint{1.789273in}{2.858221in}}%
\pgfpathcurveto{\pgfqpoint{1.789273in}{2.869271in}}{\pgfqpoint{1.784883in}{2.879870in}}{\pgfqpoint{1.777069in}{2.887684in}}%
\pgfpathcurveto{\pgfqpoint{1.769256in}{2.895497in}}{\pgfqpoint{1.758657in}{2.899887in}}{\pgfqpoint{1.747607in}{2.899887in}}%
\pgfpathcurveto{\pgfqpoint{1.736557in}{2.899887in}}{\pgfqpoint{1.725957in}{2.895497in}}{\pgfqpoint{1.718144in}{2.887684in}}%
\pgfpathcurveto{\pgfqpoint{1.710330in}{2.879870in}}{\pgfqpoint{1.705940in}{2.869271in}}{\pgfqpoint{1.705940in}{2.858221in}}%
\pgfpathcurveto{\pgfqpoint{1.705940in}{2.847171in}}{\pgfqpoint{1.710330in}{2.836572in}}{\pgfqpoint{1.718144in}{2.828758in}}%
\pgfpathcurveto{\pgfqpoint{1.725957in}{2.820944in}}{\pgfqpoint{1.736557in}{2.816554in}}{\pgfqpoint{1.747607in}{2.816554in}}%
\pgfpathclose%
\pgfusepath{stroke,fill}%
\end{pgfscope}%
\begin{pgfscope}%
\pgfpathrectangle{\pgfqpoint{0.787074in}{0.548769in}}{\pgfqpoint{5.062926in}{3.102590in}}%
\pgfusepath{clip}%
\pgfsetbuttcap%
\pgfsetroundjoin%
\definecolor{currentfill}{rgb}{1.000000,0.498039,0.054902}%
\pgfsetfillcolor{currentfill}%
\pgfsetlinewidth{1.003750pt}%
\definecolor{currentstroke}{rgb}{1.000000,0.498039,0.054902}%
\pgfsetstrokecolor{currentstroke}%
\pgfsetdash{}{0pt}%
\pgfpathmoveto{\pgfqpoint{1.879980in}{2.998783in}}%
\pgfpathcurveto{\pgfqpoint{1.891031in}{2.998783in}}{\pgfqpoint{1.901630in}{3.003173in}}{\pgfqpoint{1.909443in}{3.010987in}}%
\pgfpathcurveto{\pgfqpoint{1.917257in}{3.018800in}}{\pgfqpoint{1.921647in}{3.029399in}}{\pgfqpoint{1.921647in}{3.040450in}}%
\pgfpathcurveto{\pgfqpoint{1.921647in}{3.051500in}}{\pgfqpoint{1.917257in}{3.062099in}}{\pgfqpoint{1.909443in}{3.069912in}}%
\pgfpathcurveto{\pgfqpoint{1.901630in}{3.077726in}}{\pgfqpoint{1.891031in}{3.082116in}}{\pgfqpoint{1.879980in}{3.082116in}}%
\pgfpathcurveto{\pgfqpoint{1.868930in}{3.082116in}}{\pgfqpoint{1.858331in}{3.077726in}}{\pgfqpoint{1.850518in}{3.069912in}}%
\pgfpathcurveto{\pgfqpoint{1.842704in}{3.062099in}}{\pgfqpoint{1.838314in}{3.051500in}}{\pgfqpoint{1.838314in}{3.040450in}}%
\pgfpathcurveto{\pgfqpoint{1.838314in}{3.029399in}}{\pgfqpoint{1.842704in}{3.018800in}}{\pgfqpoint{1.850518in}{3.010987in}}%
\pgfpathcurveto{\pgfqpoint{1.858331in}{3.003173in}}{\pgfqpoint{1.868930in}{2.998783in}}{\pgfqpoint{1.879980in}{2.998783in}}%
\pgfpathclose%
\pgfusepath{stroke,fill}%
\end{pgfscope}%
\begin{pgfscope}%
\pgfpathrectangle{\pgfqpoint{0.787074in}{0.548769in}}{\pgfqpoint{5.062926in}{3.102590in}}%
\pgfusepath{clip}%
\pgfsetbuttcap%
\pgfsetroundjoin%
\definecolor{currentfill}{rgb}{1.000000,0.498039,0.054902}%
\pgfsetfillcolor{currentfill}%
\pgfsetlinewidth{1.003750pt}%
\definecolor{currentstroke}{rgb}{1.000000,0.498039,0.054902}%
\pgfsetstrokecolor{currentstroke}%
\pgfsetdash{}{0pt}%
\pgfpathmoveto{\pgfqpoint{1.627689in}{2.057042in}}%
\pgfpathcurveto{\pgfqpoint{1.638739in}{2.057042in}}{\pgfqpoint{1.649338in}{2.061432in}}{\pgfqpoint{1.657152in}{2.069246in}}%
\pgfpathcurveto{\pgfqpoint{1.664965in}{2.077059in}}{\pgfqpoint{1.669356in}{2.087659in}}{\pgfqpoint{1.669356in}{2.098709in}}%
\pgfpathcurveto{\pgfqpoint{1.669356in}{2.109759in}}{\pgfqpoint{1.664965in}{2.120358in}}{\pgfqpoint{1.657152in}{2.128171in}}%
\pgfpathcurveto{\pgfqpoint{1.649338in}{2.135985in}}{\pgfqpoint{1.638739in}{2.140375in}}{\pgfqpoint{1.627689in}{2.140375in}}%
\pgfpathcurveto{\pgfqpoint{1.616639in}{2.140375in}}{\pgfqpoint{1.606040in}{2.135985in}}{\pgfqpoint{1.598226in}{2.128171in}}%
\pgfpathcurveto{\pgfqpoint{1.590413in}{2.120358in}}{\pgfqpoint{1.586022in}{2.109759in}}{\pgfqpoint{1.586022in}{2.098709in}}%
\pgfpathcurveto{\pgfqpoint{1.586022in}{2.087659in}}{\pgfqpoint{1.590413in}{2.077059in}}{\pgfqpoint{1.598226in}{2.069246in}}%
\pgfpathcurveto{\pgfqpoint{1.606040in}{2.061432in}}{\pgfqpoint{1.616639in}{2.057042in}}{\pgfqpoint{1.627689in}{2.057042in}}%
\pgfpathclose%
\pgfusepath{stroke,fill}%
\end{pgfscope}%
\begin{pgfscope}%
\pgfpathrectangle{\pgfqpoint{0.787074in}{0.548769in}}{\pgfqpoint{5.062926in}{3.102590in}}%
\pgfusepath{clip}%
\pgfsetbuttcap%
\pgfsetroundjoin%
\definecolor{currentfill}{rgb}{1.000000,0.498039,0.054902}%
\pgfsetfillcolor{currentfill}%
\pgfsetlinewidth{1.003750pt}%
\definecolor{currentstroke}{rgb}{1.000000,0.498039,0.054902}%
\pgfsetstrokecolor{currentstroke}%
\pgfsetdash{}{0pt}%
\pgfpathmoveto{\pgfqpoint{1.126318in}{3.349495in}}%
\pgfpathcurveto{\pgfqpoint{1.137368in}{3.349495in}}{\pgfqpoint{1.147967in}{3.353886in}}{\pgfqpoint{1.155780in}{3.361699in}}%
\pgfpathcurveto{\pgfqpoint{1.163594in}{3.369513in}}{\pgfqpoint{1.167984in}{3.380112in}}{\pgfqpoint{1.167984in}{3.391162in}}%
\pgfpathcurveto{\pgfqpoint{1.167984in}{3.402212in}}{\pgfqpoint{1.163594in}{3.412811in}}{\pgfqpoint{1.155780in}{3.420625in}}%
\pgfpathcurveto{\pgfqpoint{1.147967in}{3.428438in}}{\pgfqpoint{1.137368in}{3.432829in}}{\pgfqpoint{1.126318in}{3.432829in}}%
\pgfpathcurveto{\pgfqpoint{1.115267in}{3.432829in}}{\pgfqpoint{1.104668in}{3.428438in}}{\pgfqpoint{1.096855in}{3.420625in}}%
\pgfpathcurveto{\pgfqpoint{1.089041in}{3.412811in}}{\pgfqpoint{1.084651in}{3.402212in}}{\pgfqpoint{1.084651in}{3.391162in}}%
\pgfpathcurveto{\pgfqpoint{1.084651in}{3.380112in}}{\pgfqpoint{1.089041in}{3.369513in}}{\pgfqpoint{1.096855in}{3.361699in}}%
\pgfpathcurveto{\pgfqpoint{1.104668in}{3.353886in}}{\pgfqpoint{1.115267in}{3.349495in}}{\pgfqpoint{1.126318in}{3.349495in}}%
\pgfpathclose%
\pgfusepath{stroke,fill}%
\end{pgfscope}%
\begin{pgfscope}%
\pgfpathrectangle{\pgfqpoint{0.787074in}{0.548769in}}{\pgfqpoint{5.062926in}{3.102590in}}%
\pgfusepath{clip}%
\pgfsetbuttcap%
\pgfsetroundjoin%
\definecolor{currentfill}{rgb}{1.000000,0.498039,0.054902}%
\pgfsetfillcolor{currentfill}%
\pgfsetlinewidth{1.003750pt}%
\definecolor{currentstroke}{rgb}{1.000000,0.498039,0.054902}%
\pgfsetstrokecolor{currentstroke}%
\pgfsetdash{}{0pt}%
\pgfpathmoveto{\pgfqpoint{1.727425in}{2.864404in}}%
\pgfpathcurveto{\pgfqpoint{1.738475in}{2.864404in}}{\pgfqpoint{1.749074in}{2.868794in}}{\pgfqpoint{1.756888in}{2.876608in}}%
\pgfpathcurveto{\pgfqpoint{1.764701in}{2.884422in}}{\pgfqpoint{1.769092in}{2.895021in}}{\pgfqpoint{1.769092in}{2.906071in}}%
\pgfpathcurveto{\pgfqpoint{1.769092in}{2.917121in}}{\pgfqpoint{1.764701in}{2.927720in}}{\pgfqpoint{1.756888in}{2.935534in}}%
\pgfpathcurveto{\pgfqpoint{1.749074in}{2.943347in}}{\pgfqpoint{1.738475in}{2.947737in}}{\pgfqpoint{1.727425in}{2.947737in}}%
\pgfpathcurveto{\pgfqpoint{1.716375in}{2.947737in}}{\pgfqpoint{1.705776in}{2.943347in}}{\pgfqpoint{1.697962in}{2.935534in}}%
\pgfpathcurveto{\pgfqpoint{1.690149in}{2.927720in}}{\pgfqpoint{1.685758in}{2.917121in}}{\pgfqpoint{1.685758in}{2.906071in}}%
\pgfpathcurveto{\pgfqpoint{1.685758in}{2.895021in}}{\pgfqpoint{1.690149in}{2.884422in}}{\pgfqpoint{1.697962in}{2.876608in}}%
\pgfpathcurveto{\pgfqpoint{1.705776in}{2.868794in}}{\pgfqpoint{1.716375in}{2.864404in}}{\pgfqpoint{1.727425in}{2.864404in}}%
\pgfpathclose%
\pgfusepath{stroke,fill}%
\end{pgfscope}%
\begin{pgfscope}%
\pgfpathrectangle{\pgfqpoint{0.787074in}{0.548769in}}{\pgfqpoint{5.062926in}{3.102590in}}%
\pgfusepath{clip}%
\pgfsetbuttcap%
\pgfsetroundjoin%
\definecolor{currentfill}{rgb}{0.121569,0.466667,0.705882}%
\pgfsetfillcolor{currentfill}%
\pgfsetlinewidth{1.003750pt}%
\definecolor{currentstroke}{rgb}{0.121569,0.466667,0.705882}%
\pgfsetstrokecolor{currentstroke}%
\pgfsetdash{}{0pt}%
\pgfpathmoveto{\pgfqpoint{1.721913in}{0.648143in}}%
\pgfpathcurveto{\pgfqpoint{1.732963in}{0.648143in}}{\pgfqpoint{1.743562in}{0.652534in}}{\pgfqpoint{1.751376in}{0.660347in}}%
\pgfpathcurveto{\pgfqpoint{1.759189in}{0.668161in}}{\pgfqpoint{1.763580in}{0.678760in}}{\pgfqpoint{1.763580in}{0.689810in}}%
\pgfpathcurveto{\pgfqpoint{1.763580in}{0.700860in}}{\pgfqpoint{1.759189in}{0.711459in}}{\pgfqpoint{1.751376in}{0.719273in}}%
\pgfpathcurveto{\pgfqpoint{1.743562in}{0.727086in}}{\pgfqpoint{1.732963in}{0.731477in}}{\pgfqpoint{1.721913in}{0.731477in}}%
\pgfpathcurveto{\pgfqpoint{1.710863in}{0.731477in}}{\pgfqpoint{1.700264in}{0.727086in}}{\pgfqpoint{1.692450in}{0.719273in}}%
\pgfpathcurveto{\pgfqpoint{1.684637in}{0.711459in}}{\pgfqpoint{1.680246in}{0.700860in}}{\pgfqpoint{1.680246in}{0.689810in}}%
\pgfpathcurveto{\pgfqpoint{1.680246in}{0.678760in}}{\pgfqpoint{1.684637in}{0.668161in}}{\pgfqpoint{1.692450in}{0.660347in}}%
\pgfpathcurveto{\pgfqpoint{1.700264in}{0.652534in}}{\pgfqpoint{1.710863in}{0.648143in}}{\pgfqpoint{1.721913in}{0.648143in}}%
\pgfpathclose%
\pgfusepath{stroke,fill}%
\end{pgfscope}%
\begin{pgfscope}%
\pgfpathrectangle{\pgfqpoint{0.787074in}{0.548769in}}{\pgfqpoint{5.062926in}{3.102590in}}%
\pgfusepath{clip}%
\pgfsetbuttcap%
\pgfsetroundjoin%
\definecolor{currentfill}{rgb}{0.121569,0.466667,0.705882}%
\pgfsetfillcolor{currentfill}%
\pgfsetlinewidth{1.003750pt}%
\definecolor{currentstroke}{rgb}{0.121569,0.466667,0.705882}%
\pgfsetstrokecolor{currentstroke}%
\pgfsetdash{}{0pt}%
\pgfpathmoveto{\pgfqpoint{1.577213in}{0.648148in}}%
\pgfpathcurveto{\pgfqpoint{1.588263in}{0.648148in}}{\pgfqpoint{1.598862in}{0.652538in}}{\pgfqpoint{1.606676in}{0.660352in}}%
\pgfpathcurveto{\pgfqpoint{1.614490in}{0.668165in}}{\pgfqpoint{1.618880in}{0.678764in}}{\pgfqpoint{1.618880in}{0.689815in}}%
\pgfpathcurveto{\pgfqpoint{1.618880in}{0.700865in}}{\pgfqpoint{1.614490in}{0.711464in}}{\pgfqpoint{1.606676in}{0.719277in}}%
\pgfpathcurveto{\pgfqpoint{1.598862in}{0.727091in}}{\pgfqpoint{1.588263in}{0.731481in}}{\pgfqpoint{1.577213in}{0.731481in}}%
\pgfpathcurveto{\pgfqpoint{1.566163in}{0.731481in}}{\pgfqpoint{1.555564in}{0.727091in}}{\pgfqpoint{1.547751in}{0.719277in}}%
\pgfpathcurveto{\pgfqpoint{1.539937in}{0.711464in}}{\pgfqpoint{1.535547in}{0.700865in}}{\pgfqpoint{1.535547in}{0.689815in}}%
\pgfpathcurveto{\pgfqpoint{1.535547in}{0.678764in}}{\pgfqpoint{1.539937in}{0.668165in}}{\pgfqpoint{1.547751in}{0.660352in}}%
\pgfpathcurveto{\pgfqpoint{1.555564in}{0.652538in}}{\pgfqpoint{1.566163in}{0.648148in}}{\pgfqpoint{1.577213in}{0.648148in}}%
\pgfpathclose%
\pgfusepath{stroke,fill}%
\end{pgfscope}%
\begin{pgfscope}%
\pgfpathrectangle{\pgfqpoint{0.787074in}{0.548769in}}{\pgfqpoint{5.062926in}{3.102590in}}%
\pgfusepath{clip}%
\pgfsetbuttcap%
\pgfsetroundjoin%
\definecolor{currentfill}{rgb}{0.121569,0.466667,0.705882}%
\pgfsetfillcolor{currentfill}%
\pgfsetlinewidth{1.003750pt}%
\definecolor{currentstroke}{rgb}{0.121569,0.466667,0.705882}%
\pgfsetstrokecolor{currentstroke}%
\pgfsetdash{}{0pt}%
\pgfpathmoveto{\pgfqpoint{1.573828in}{0.648152in}}%
\pgfpathcurveto{\pgfqpoint{1.584878in}{0.648152in}}{\pgfqpoint{1.595477in}{0.652542in}}{\pgfqpoint{1.603291in}{0.660355in}}%
\pgfpathcurveto{\pgfqpoint{1.611104in}{0.668169in}}{\pgfqpoint{1.615495in}{0.678768in}}{\pgfqpoint{1.615495in}{0.689818in}}%
\pgfpathcurveto{\pgfqpoint{1.615495in}{0.700868in}}{\pgfqpoint{1.611104in}{0.711467in}}{\pgfqpoint{1.603291in}{0.719281in}}%
\pgfpathcurveto{\pgfqpoint{1.595477in}{0.727095in}}{\pgfqpoint{1.584878in}{0.731485in}}{\pgfqpoint{1.573828in}{0.731485in}}%
\pgfpathcurveto{\pgfqpoint{1.562778in}{0.731485in}}{\pgfqpoint{1.552179in}{0.727095in}}{\pgfqpoint{1.544365in}{0.719281in}}%
\pgfpathcurveto{\pgfqpoint{1.536552in}{0.711467in}}{\pgfqpoint{1.532161in}{0.700868in}}{\pgfqpoint{1.532161in}{0.689818in}}%
\pgfpathcurveto{\pgfqpoint{1.532161in}{0.678768in}}{\pgfqpoint{1.536552in}{0.668169in}}{\pgfqpoint{1.544365in}{0.660355in}}%
\pgfpathcurveto{\pgfqpoint{1.552179in}{0.652542in}}{\pgfqpoint{1.562778in}{0.648152in}}{\pgfqpoint{1.573828in}{0.648152in}}%
\pgfpathclose%
\pgfusepath{stroke,fill}%
\end{pgfscope}%
\begin{pgfscope}%
\pgfpathrectangle{\pgfqpoint{0.787074in}{0.548769in}}{\pgfqpoint{5.062926in}{3.102590in}}%
\pgfusepath{clip}%
\pgfsetbuttcap%
\pgfsetroundjoin%
\definecolor{currentfill}{rgb}{1.000000,0.498039,0.054902}%
\pgfsetfillcolor{currentfill}%
\pgfsetlinewidth{1.003750pt}%
\definecolor{currentstroke}{rgb}{1.000000,0.498039,0.054902}%
\pgfsetstrokecolor{currentstroke}%
\pgfsetdash{}{0pt}%
\pgfpathmoveto{\pgfqpoint{1.433816in}{2.884914in}}%
\pgfpathcurveto{\pgfqpoint{1.444866in}{2.884914in}}{\pgfqpoint{1.455465in}{2.889305in}}{\pgfqpoint{1.463278in}{2.897118in}}%
\pgfpathcurveto{\pgfqpoint{1.471092in}{2.904932in}}{\pgfqpoint{1.475482in}{2.915531in}}{\pgfqpoint{1.475482in}{2.926581in}}%
\pgfpathcurveto{\pgfqpoint{1.475482in}{2.937631in}}{\pgfqpoint{1.471092in}{2.948230in}}{\pgfqpoint{1.463278in}{2.956044in}}%
\pgfpathcurveto{\pgfqpoint{1.455465in}{2.963857in}}{\pgfqpoint{1.444866in}{2.968248in}}{\pgfqpoint{1.433816in}{2.968248in}}%
\pgfpathcurveto{\pgfqpoint{1.422765in}{2.968248in}}{\pgfqpoint{1.412166in}{2.963857in}}{\pgfqpoint{1.404353in}{2.956044in}}%
\pgfpathcurveto{\pgfqpoint{1.396539in}{2.948230in}}{\pgfqpoint{1.392149in}{2.937631in}}{\pgfqpoint{1.392149in}{2.926581in}}%
\pgfpathcurveto{\pgfqpoint{1.392149in}{2.915531in}}{\pgfqpoint{1.396539in}{2.904932in}}{\pgfqpoint{1.404353in}{2.897118in}}%
\pgfpathcurveto{\pgfqpoint{1.412166in}{2.889305in}}{\pgfqpoint{1.422765in}{2.884914in}}{\pgfqpoint{1.433816in}{2.884914in}}%
\pgfpathclose%
\pgfusepath{stroke,fill}%
\end{pgfscope}%
\begin{pgfscope}%
\pgfpathrectangle{\pgfqpoint{0.787074in}{0.548769in}}{\pgfqpoint{5.062926in}{3.102590in}}%
\pgfusepath{clip}%
\pgfsetbuttcap%
\pgfsetroundjoin%
\definecolor{currentfill}{rgb}{1.000000,0.498039,0.054902}%
\pgfsetfillcolor{currentfill}%
\pgfsetlinewidth{1.003750pt}%
\definecolor{currentstroke}{rgb}{1.000000,0.498039,0.054902}%
\pgfsetstrokecolor{currentstroke}%
\pgfsetdash{}{0pt}%
\pgfpathmoveto{\pgfqpoint{1.853983in}{2.875764in}}%
\pgfpathcurveto{\pgfqpoint{1.865033in}{2.875764in}}{\pgfqpoint{1.875632in}{2.880154in}}{\pgfqpoint{1.883446in}{2.887968in}}%
\pgfpathcurveto{\pgfqpoint{1.891260in}{2.895781in}}{\pgfqpoint{1.895650in}{2.906380in}}{\pgfqpoint{1.895650in}{2.917430in}}%
\pgfpathcurveto{\pgfqpoint{1.895650in}{2.928481in}}{\pgfqpoint{1.891260in}{2.939080in}}{\pgfqpoint{1.883446in}{2.946893in}}%
\pgfpathcurveto{\pgfqpoint{1.875632in}{2.954707in}}{\pgfqpoint{1.865033in}{2.959097in}}{\pgfqpoint{1.853983in}{2.959097in}}%
\pgfpathcurveto{\pgfqpoint{1.842933in}{2.959097in}}{\pgfqpoint{1.832334in}{2.954707in}}{\pgfqpoint{1.824520in}{2.946893in}}%
\pgfpathcurveto{\pgfqpoint{1.816707in}{2.939080in}}{\pgfqpoint{1.812316in}{2.928481in}}{\pgfqpoint{1.812316in}{2.917430in}}%
\pgfpathcurveto{\pgfqpoint{1.812316in}{2.906380in}}{\pgfqpoint{1.816707in}{2.895781in}}{\pgfqpoint{1.824520in}{2.887968in}}%
\pgfpathcurveto{\pgfqpoint{1.832334in}{2.880154in}}{\pgfqpoint{1.842933in}{2.875764in}}{\pgfqpoint{1.853983in}{2.875764in}}%
\pgfpathclose%
\pgfusepath{stroke,fill}%
\end{pgfscope}%
\begin{pgfscope}%
\pgfpathrectangle{\pgfqpoint{0.787074in}{0.548769in}}{\pgfqpoint{5.062926in}{3.102590in}}%
\pgfusepath{clip}%
\pgfsetbuttcap%
\pgfsetroundjoin%
\definecolor{currentfill}{rgb}{1.000000,0.498039,0.054902}%
\pgfsetfillcolor{currentfill}%
\pgfsetlinewidth{1.003750pt}%
\definecolor{currentstroke}{rgb}{1.000000,0.498039,0.054902}%
\pgfsetstrokecolor{currentstroke}%
\pgfsetdash{}{0pt}%
\pgfpathmoveto{\pgfqpoint{1.734456in}{2.086876in}}%
\pgfpathcurveto{\pgfqpoint{1.745506in}{2.086876in}}{\pgfqpoint{1.756105in}{2.091266in}}{\pgfqpoint{1.763919in}{2.099080in}}%
\pgfpathcurveto{\pgfqpoint{1.771732in}{2.106893in}}{\pgfqpoint{1.776123in}{2.117492in}}{\pgfqpoint{1.776123in}{2.128542in}}%
\pgfpathcurveto{\pgfqpoint{1.776123in}{2.139593in}}{\pgfqpoint{1.771732in}{2.150192in}}{\pgfqpoint{1.763919in}{2.158005in}}%
\pgfpathcurveto{\pgfqpoint{1.756105in}{2.165819in}}{\pgfqpoint{1.745506in}{2.170209in}}{\pgfqpoint{1.734456in}{2.170209in}}%
\pgfpathcurveto{\pgfqpoint{1.723406in}{2.170209in}}{\pgfqpoint{1.712807in}{2.165819in}}{\pgfqpoint{1.704993in}{2.158005in}}%
\pgfpathcurveto{\pgfqpoint{1.697180in}{2.150192in}}{\pgfqpoint{1.692789in}{2.139593in}}{\pgfqpoint{1.692789in}{2.128542in}}%
\pgfpathcurveto{\pgfqpoint{1.692789in}{2.117492in}}{\pgfqpoint{1.697180in}{2.106893in}}{\pgfqpoint{1.704993in}{2.099080in}}%
\pgfpathcurveto{\pgfqpoint{1.712807in}{2.091266in}}{\pgfqpoint{1.723406in}{2.086876in}}{\pgfqpoint{1.734456in}{2.086876in}}%
\pgfpathclose%
\pgfusepath{stroke,fill}%
\end{pgfscope}%
\begin{pgfscope}%
\pgfpathrectangle{\pgfqpoint{0.787074in}{0.548769in}}{\pgfqpoint{5.062926in}{3.102590in}}%
\pgfusepath{clip}%
\pgfsetbuttcap%
\pgfsetroundjoin%
\definecolor{currentfill}{rgb}{0.121569,0.466667,0.705882}%
\pgfsetfillcolor{currentfill}%
\pgfsetlinewidth{1.003750pt}%
\definecolor{currentstroke}{rgb}{0.121569,0.466667,0.705882}%
\pgfsetstrokecolor{currentstroke}%
\pgfsetdash{}{0pt}%
\pgfpathmoveto{\pgfqpoint{1.076493in}{0.648132in}}%
\pgfpathcurveto{\pgfqpoint{1.087543in}{0.648132in}}{\pgfqpoint{1.098142in}{0.652523in}}{\pgfqpoint{1.105956in}{0.660336in}}%
\pgfpathcurveto{\pgfqpoint{1.113769in}{0.668150in}}{\pgfqpoint{1.118160in}{0.678749in}}{\pgfqpoint{1.118160in}{0.689799in}}%
\pgfpathcurveto{\pgfqpoint{1.118160in}{0.700849in}}{\pgfqpoint{1.113769in}{0.711448in}}{\pgfqpoint{1.105956in}{0.719262in}}%
\pgfpathcurveto{\pgfqpoint{1.098142in}{0.727075in}}{\pgfqpoint{1.087543in}{0.731466in}}{\pgfqpoint{1.076493in}{0.731466in}}%
\pgfpathcurveto{\pgfqpoint{1.065443in}{0.731466in}}{\pgfqpoint{1.054844in}{0.727075in}}{\pgfqpoint{1.047030in}{0.719262in}}%
\pgfpathcurveto{\pgfqpoint{1.039217in}{0.711448in}}{\pgfqpoint{1.034826in}{0.700849in}}{\pgfqpoint{1.034826in}{0.689799in}}%
\pgfpathcurveto{\pgfqpoint{1.034826in}{0.678749in}}{\pgfqpoint{1.039217in}{0.668150in}}{\pgfqpoint{1.047030in}{0.660336in}}%
\pgfpathcurveto{\pgfqpoint{1.054844in}{0.652523in}}{\pgfqpoint{1.065443in}{0.648132in}}{\pgfqpoint{1.076493in}{0.648132in}}%
\pgfpathclose%
\pgfusepath{stroke,fill}%
\end{pgfscope}%
\begin{pgfscope}%
\pgfpathrectangle{\pgfqpoint{0.787074in}{0.548769in}}{\pgfqpoint{5.062926in}{3.102590in}}%
\pgfusepath{clip}%
\pgfsetbuttcap%
\pgfsetroundjoin%
\definecolor{currentfill}{rgb}{0.121569,0.466667,0.705882}%
\pgfsetfillcolor{currentfill}%
\pgfsetlinewidth{1.003750pt}%
\definecolor{currentstroke}{rgb}{0.121569,0.466667,0.705882}%
\pgfsetstrokecolor{currentstroke}%
\pgfsetdash{}{0pt}%
\pgfpathmoveto{\pgfqpoint{1.017207in}{0.787529in}}%
\pgfpathcurveto{\pgfqpoint{1.028257in}{0.787529in}}{\pgfqpoint{1.038856in}{0.791919in}}{\pgfqpoint{1.046670in}{0.799733in}}%
\pgfpathcurveto{\pgfqpoint{1.054483in}{0.807547in}}{\pgfqpoint{1.058874in}{0.818146in}}{\pgfqpoint{1.058874in}{0.829196in}}%
\pgfpathcurveto{\pgfqpoint{1.058874in}{0.840246in}}{\pgfqpoint{1.054483in}{0.850845in}}{\pgfqpoint{1.046670in}{0.858658in}}%
\pgfpathcurveto{\pgfqpoint{1.038856in}{0.866472in}}{\pgfqpoint{1.028257in}{0.870862in}}{\pgfqpoint{1.017207in}{0.870862in}}%
\pgfpathcurveto{\pgfqpoint{1.006157in}{0.870862in}}{\pgfqpoint{0.995558in}{0.866472in}}{\pgfqpoint{0.987744in}{0.858658in}}%
\pgfpathcurveto{\pgfqpoint{0.979930in}{0.850845in}}{\pgfqpoint{0.975540in}{0.840246in}}{\pgfqpoint{0.975540in}{0.829196in}}%
\pgfpathcurveto{\pgfqpoint{0.975540in}{0.818146in}}{\pgfqpoint{0.979930in}{0.807547in}}{\pgfqpoint{0.987744in}{0.799733in}}%
\pgfpathcurveto{\pgfqpoint{0.995558in}{0.791919in}}{\pgfqpoint{1.006157in}{0.787529in}}{\pgfqpoint{1.017207in}{0.787529in}}%
\pgfpathclose%
\pgfusepath{stroke,fill}%
\end{pgfscope}%
\begin{pgfscope}%
\pgfpathrectangle{\pgfqpoint{0.787074in}{0.548769in}}{\pgfqpoint{5.062926in}{3.102590in}}%
\pgfusepath{clip}%
\pgfsetbuttcap%
\pgfsetroundjoin%
\definecolor{currentfill}{rgb}{0.121569,0.466667,0.705882}%
\pgfsetfillcolor{currentfill}%
\pgfsetlinewidth{1.003750pt}%
\definecolor{currentstroke}{rgb}{0.121569,0.466667,0.705882}%
\pgfsetstrokecolor{currentstroke}%
\pgfsetdash{}{0pt}%
\pgfpathmoveto{\pgfqpoint{2.061181in}{2.073153in}}%
\pgfpathcurveto{\pgfqpoint{2.072231in}{2.073153in}}{\pgfqpoint{2.082830in}{2.077543in}}{\pgfqpoint{2.090644in}{2.085357in}}%
\pgfpathcurveto{\pgfqpoint{2.098457in}{2.093170in}}{\pgfqpoint{2.102847in}{2.103769in}}{\pgfqpoint{2.102847in}{2.114820in}}%
\pgfpathcurveto{\pgfqpoint{2.102847in}{2.125870in}}{\pgfqpoint{2.098457in}{2.136469in}}{\pgfqpoint{2.090644in}{2.144282in}}%
\pgfpathcurveto{\pgfqpoint{2.082830in}{2.152096in}}{\pgfqpoint{2.072231in}{2.156486in}}{\pgfqpoint{2.061181in}{2.156486in}}%
\pgfpathcurveto{\pgfqpoint{2.050131in}{2.156486in}}{\pgfqpoint{2.039532in}{2.152096in}}{\pgfqpoint{2.031718in}{2.144282in}}%
\pgfpathcurveto{\pgfqpoint{2.023904in}{2.136469in}}{\pgfqpoint{2.019514in}{2.125870in}}{\pgfqpoint{2.019514in}{2.114820in}}%
\pgfpathcurveto{\pgfqpoint{2.019514in}{2.103769in}}{\pgfqpoint{2.023904in}{2.093170in}}{\pgfqpoint{2.031718in}{2.085357in}}%
\pgfpathcurveto{\pgfqpoint{2.039532in}{2.077543in}}{\pgfqpoint{2.050131in}{2.073153in}}{\pgfqpoint{2.061181in}{2.073153in}}%
\pgfpathclose%
\pgfusepath{stroke,fill}%
\end{pgfscope}%
\begin{pgfscope}%
\pgfpathrectangle{\pgfqpoint{0.787074in}{0.548769in}}{\pgfqpoint{5.062926in}{3.102590in}}%
\pgfusepath{clip}%
\pgfsetbuttcap%
\pgfsetroundjoin%
\definecolor{currentfill}{rgb}{0.121569,0.466667,0.705882}%
\pgfsetfillcolor{currentfill}%
\pgfsetlinewidth{1.003750pt}%
\definecolor{currentstroke}{rgb}{0.121569,0.466667,0.705882}%
\pgfsetstrokecolor{currentstroke}%
\pgfsetdash{}{0pt}%
\pgfpathmoveto{\pgfqpoint{1.689623in}{0.648148in}}%
\pgfpathcurveto{\pgfqpoint{1.700673in}{0.648148in}}{\pgfqpoint{1.711272in}{0.652539in}}{\pgfqpoint{1.719085in}{0.660352in}}%
\pgfpathcurveto{\pgfqpoint{1.726899in}{0.668166in}}{\pgfqpoint{1.731289in}{0.678765in}}{\pgfqpoint{1.731289in}{0.689815in}}%
\pgfpathcurveto{\pgfqpoint{1.731289in}{0.700865in}}{\pgfqpoint{1.726899in}{0.711464in}}{\pgfqpoint{1.719085in}{0.719278in}}%
\pgfpathcurveto{\pgfqpoint{1.711272in}{0.727091in}}{\pgfqpoint{1.700673in}{0.731482in}}{\pgfqpoint{1.689623in}{0.731482in}}%
\pgfpathcurveto{\pgfqpoint{1.678572in}{0.731482in}}{\pgfqpoint{1.667973in}{0.727091in}}{\pgfqpoint{1.660160in}{0.719278in}}%
\pgfpathcurveto{\pgfqpoint{1.652346in}{0.711464in}}{\pgfqpoint{1.647956in}{0.700865in}}{\pgfqpoint{1.647956in}{0.689815in}}%
\pgfpathcurveto{\pgfqpoint{1.647956in}{0.678765in}}{\pgfqpoint{1.652346in}{0.668166in}}{\pgfqpoint{1.660160in}{0.660352in}}%
\pgfpathcurveto{\pgfqpoint{1.667973in}{0.652539in}}{\pgfqpoint{1.678572in}{0.648148in}}{\pgfqpoint{1.689623in}{0.648148in}}%
\pgfpathclose%
\pgfusepath{stroke,fill}%
\end{pgfscope}%
\begin{pgfscope}%
\pgfpathrectangle{\pgfqpoint{0.787074in}{0.548769in}}{\pgfqpoint{5.062926in}{3.102590in}}%
\pgfusepath{clip}%
\pgfsetbuttcap%
\pgfsetroundjoin%
\definecolor{currentfill}{rgb}{0.121569,0.466667,0.705882}%
\pgfsetfillcolor{currentfill}%
\pgfsetlinewidth{1.003750pt}%
\definecolor{currentstroke}{rgb}{0.121569,0.466667,0.705882}%
\pgfsetstrokecolor{currentstroke}%
\pgfsetdash{}{0pt}%
\pgfpathmoveto{\pgfqpoint{1.285644in}{0.648150in}}%
\pgfpathcurveto{\pgfqpoint{1.296694in}{0.648150in}}{\pgfqpoint{1.307293in}{0.652540in}}{\pgfqpoint{1.315106in}{0.660353in}}%
\pgfpathcurveto{\pgfqpoint{1.322920in}{0.668167in}}{\pgfqpoint{1.327310in}{0.678766in}}{\pgfqpoint{1.327310in}{0.689816in}}%
\pgfpathcurveto{\pgfqpoint{1.327310in}{0.700866in}}{\pgfqpoint{1.322920in}{0.711465in}}{\pgfqpoint{1.315106in}{0.719279in}}%
\pgfpathcurveto{\pgfqpoint{1.307293in}{0.727093in}}{\pgfqpoint{1.296694in}{0.731483in}}{\pgfqpoint{1.285644in}{0.731483in}}%
\pgfpathcurveto{\pgfqpoint{1.274594in}{0.731483in}}{\pgfqpoint{1.263994in}{0.727093in}}{\pgfqpoint{1.256181in}{0.719279in}}%
\pgfpathcurveto{\pgfqpoint{1.248367in}{0.711465in}}{\pgfqpoint{1.243977in}{0.700866in}}{\pgfqpoint{1.243977in}{0.689816in}}%
\pgfpathcurveto{\pgfqpoint{1.243977in}{0.678766in}}{\pgfqpoint{1.248367in}{0.668167in}}{\pgfqpoint{1.256181in}{0.660353in}}%
\pgfpathcurveto{\pgfqpoint{1.263994in}{0.652540in}}{\pgfqpoint{1.274594in}{0.648150in}}{\pgfqpoint{1.285644in}{0.648150in}}%
\pgfpathclose%
\pgfusepath{stroke,fill}%
\end{pgfscope}%
\begin{pgfscope}%
\pgfpathrectangle{\pgfqpoint{0.787074in}{0.548769in}}{\pgfqpoint{5.062926in}{3.102590in}}%
\pgfusepath{clip}%
\pgfsetbuttcap%
\pgfsetroundjoin%
\definecolor{currentfill}{rgb}{0.121569,0.466667,0.705882}%
\pgfsetfillcolor{currentfill}%
\pgfsetlinewidth{1.003750pt}%
\definecolor{currentstroke}{rgb}{0.121569,0.466667,0.705882}%
\pgfsetstrokecolor{currentstroke}%
\pgfsetdash{}{0pt}%
\pgfpathmoveto{\pgfqpoint{1.032310in}{2.326735in}}%
\pgfpathcurveto{\pgfqpoint{1.043361in}{2.326735in}}{\pgfqpoint{1.053960in}{2.331125in}}{\pgfqpoint{1.061773in}{2.338939in}}%
\pgfpathcurveto{\pgfqpoint{1.069587in}{2.346753in}}{\pgfqpoint{1.073977in}{2.357352in}}{\pgfqpoint{1.073977in}{2.368402in}}%
\pgfpathcurveto{\pgfqpoint{1.073977in}{2.379452in}}{\pgfqpoint{1.069587in}{2.390051in}}{\pgfqpoint{1.061773in}{2.397865in}}%
\pgfpathcurveto{\pgfqpoint{1.053960in}{2.405678in}}{\pgfqpoint{1.043361in}{2.410068in}}{\pgfqpoint{1.032310in}{2.410068in}}%
\pgfpathcurveto{\pgfqpoint{1.021260in}{2.410068in}}{\pgfqpoint{1.010661in}{2.405678in}}{\pgfqpoint{1.002848in}{2.397865in}}%
\pgfpathcurveto{\pgfqpoint{0.995034in}{2.390051in}}{\pgfqpoint{0.990644in}{2.379452in}}{\pgfqpoint{0.990644in}{2.368402in}}%
\pgfpathcurveto{\pgfqpoint{0.990644in}{2.357352in}}{\pgfqpoint{0.995034in}{2.346753in}}{\pgfqpoint{1.002848in}{2.338939in}}%
\pgfpathcurveto{\pgfqpoint{1.010661in}{2.331125in}}{\pgfqpoint{1.021260in}{2.326735in}}{\pgfqpoint{1.032310in}{2.326735in}}%
\pgfpathclose%
\pgfusepath{stroke,fill}%
\end{pgfscope}%
\begin{pgfscope}%
\pgfpathrectangle{\pgfqpoint{0.787074in}{0.548769in}}{\pgfqpoint{5.062926in}{3.102590in}}%
\pgfusepath{clip}%
\pgfsetbuttcap%
\pgfsetroundjoin%
\definecolor{currentfill}{rgb}{1.000000,0.498039,0.054902}%
\pgfsetfillcolor{currentfill}%
\pgfsetlinewidth{1.003750pt}%
\definecolor{currentstroke}{rgb}{1.000000,0.498039,0.054902}%
\pgfsetstrokecolor{currentstroke}%
\pgfsetdash{}{0pt}%
\pgfpathmoveto{\pgfqpoint{1.318846in}{2.612427in}}%
\pgfpathcurveto{\pgfqpoint{1.329896in}{2.612427in}}{\pgfqpoint{1.340495in}{2.616817in}}{\pgfqpoint{1.348308in}{2.624631in}}%
\pgfpathcurveto{\pgfqpoint{1.356122in}{2.632444in}}{\pgfqpoint{1.360512in}{2.643043in}}{\pgfqpoint{1.360512in}{2.654094in}}%
\pgfpathcurveto{\pgfqpoint{1.360512in}{2.665144in}}{\pgfqpoint{1.356122in}{2.675743in}}{\pgfqpoint{1.348308in}{2.683556in}}%
\pgfpathcurveto{\pgfqpoint{1.340495in}{2.691370in}}{\pgfqpoint{1.329896in}{2.695760in}}{\pgfqpoint{1.318846in}{2.695760in}}%
\pgfpathcurveto{\pgfqpoint{1.307795in}{2.695760in}}{\pgfqpoint{1.297196in}{2.691370in}}{\pgfqpoint{1.289383in}{2.683556in}}%
\pgfpathcurveto{\pgfqpoint{1.281569in}{2.675743in}}{\pgfqpoint{1.277179in}{2.665144in}}{\pgfqpoint{1.277179in}{2.654094in}}%
\pgfpathcurveto{\pgfqpoint{1.277179in}{2.643043in}}{\pgfqpoint{1.281569in}{2.632444in}}{\pgfqpoint{1.289383in}{2.624631in}}%
\pgfpathcurveto{\pgfqpoint{1.297196in}{2.616817in}}{\pgfqpoint{1.307795in}{2.612427in}}{\pgfqpoint{1.318846in}{2.612427in}}%
\pgfpathclose%
\pgfusepath{stroke,fill}%
\end{pgfscope}%
\begin{pgfscope}%
\pgfpathrectangle{\pgfqpoint{0.787074in}{0.548769in}}{\pgfqpoint{5.062926in}{3.102590in}}%
\pgfusepath{clip}%
\pgfsetbuttcap%
\pgfsetroundjoin%
\definecolor{currentfill}{rgb}{1.000000,0.498039,0.054902}%
\pgfsetfillcolor{currentfill}%
\pgfsetlinewidth{1.003750pt}%
\definecolor{currentstroke}{rgb}{1.000000,0.498039,0.054902}%
\pgfsetstrokecolor{currentstroke}%
\pgfsetdash{}{0pt}%
\pgfpathmoveto{\pgfqpoint{1.245845in}{2.467225in}}%
\pgfpathcurveto{\pgfqpoint{1.256895in}{2.467225in}}{\pgfqpoint{1.267494in}{2.471616in}}{\pgfqpoint{1.275307in}{2.479429in}}%
\pgfpathcurveto{\pgfqpoint{1.283121in}{2.487243in}}{\pgfqpoint{1.287511in}{2.497842in}}{\pgfqpoint{1.287511in}{2.508892in}}%
\pgfpathcurveto{\pgfqpoint{1.287511in}{2.519942in}}{\pgfqpoint{1.283121in}{2.530541in}}{\pgfqpoint{1.275307in}{2.538355in}}%
\pgfpathcurveto{\pgfqpoint{1.267494in}{2.546168in}}{\pgfqpoint{1.256895in}{2.550559in}}{\pgfqpoint{1.245845in}{2.550559in}}%
\pgfpathcurveto{\pgfqpoint{1.234795in}{2.550559in}}{\pgfqpoint{1.224196in}{2.546168in}}{\pgfqpoint{1.216382in}{2.538355in}}%
\pgfpathcurveto{\pgfqpoint{1.208568in}{2.530541in}}{\pgfqpoint{1.204178in}{2.519942in}}{\pgfqpoint{1.204178in}{2.508892in}}%
\pgfpathcurveto{\pgfqpoint{1.204178in}{2.497842in}}{\pgfqpoint{1.208568in}{2.487243in}}{\pgfqpoint{1.216382in}{2.479429in}}%
\pgfpathcurveto{\pgfqpoint{1.224196in}{2.471616in}}{\pgfqpoint{1.234795in}{2.467225in}}{\pgfqpoint{1.245845in}{2.467225in}}%
\pgfpathclose%
\pgfusepath{stroke,fill}%
\end{pgfscope}%
\begin{pgfscope}%
\pgfpathrectangle{\pgfqpoint{0.787074in}{0.548769in}}{\pgfqpoint{5.062926in}{3.102590in}}%
\pgfusepath{clip}%
\pgfsetbuttcap%
\pgfsetroundjoin%
\definecolor{currentfill}{rgb}{1.000000,0.498039,0.054902}%
\pgfsetfillcolor{currentfill}%
\pgfsetlinewidth{1.003750pt}%
\definecolor{currentstroke}{rgb}{1.000000,0.498039,0.054902}%
\pgfsetstrokecolor{currentstroke}%
\pgfsetdash{}{0pt}%
\pgfpathmoveto{\pgfqpoint{1.991044in}{2.185530in}}%
\pgfpathcurveto{\pgfqpoint{2.002094in}{2.185530in}}{\pgfqpoint{2.012693in}{2.189921in}}{\pgfqpoint{2.020507in}{2.197734in}}%
\pgfpathcurveto{\pgfqpoint{2.028321in}{2.205548in}}{\pgfqpoint{2.032711in}{2.216147in}}{\pgfqpoint{2.032711in}{2.227197in}}%
\pgfpathcurveto{\pgfqpoint{2.032711in}{2.238247in}}{\pgfqpoint{2.028321in}{2.248846in}}{\pgfqpoint{2.020507in}{2.256660in}}%
\pgfpathcurveto{\pgfqpoint{2.012693in}{2.264473in}}{\pgfqpoint{2.002094in}{2.268864in}}{\pgfqpoint{1.991044in}{2.268864in}}%
\pgfpathcurveto{\pgfqpoint{1.979994in}{2.268864in}}{\pgfqpoint{1.969395in}{2.264473in}}{\pgfqpoint{1.961582in}{2.256660in}}%
\pgfpathcurveto{\pgfqpoint{1.953768in}{2.248846in}}{\pgfqpoint{1.949378in}{2.238247in}}{\pgfqpoint{1.949378in}{2.227197in}}%
\pgfpathcurveto{\pgfqpoint{1.949378in}{2.216147in}}{\pgfqpoint{1.953768in}{2.205548in}}{\pgfqpoint{1.961582in}{2.197734in}}%
\pgfpathcurveto{\pgfqpoint{1.969395in}{2.189921in}}{\pgfqpoint{1.979994in}{2.185530in}}{\pgfqpoint{1.991044in}{2.185530in}}%
\pgfpathclose%
\pgfusepath{stroke,fill}%
\end{pgfscope}%
\begin{pgfscope}%
\pgfpathrectangle{\pgfqpoint{0.787074in}{0.548769in}}{\pgfqpoint{5.062926in}{3.102590in}}%
\pgfusepath{clip}%
\pgfsetbuttcap%
\pgfsetroundjoin%
\definecolor{currentfill}{rgb}{0.121569,0.466667,0.705882}%
\pgfsetfillcolor{currentfill}%
\pgfsetlinewidth{1.003750pt}%
\definecolor{currentstroke}{rgb}{0.121569,0.466667,0.705882}%
\pgfsetstrokecolor{currentstroke}%
\pgfsetdash{}{0pt}%
\pgfpathmoveto{\pgfqpoint{2.367724in}{2.666690in}}%
\pgfpathcurveto{\pgfqpoint{2.378774in}{2.666690in}}{\pgfqpoint{2.389373in}{2.671080in}}{\pgfqpoint{2.397187in}{2.678894in}}%
\pgfpathcurveto{\pgfqpoint{2.405000in}{2.686707in}}{\pgfqpoint{2.409390in}{2.697306in}}{\pgfqpoint{2.409390in}{2.708356in}}%
\pgfpathcurveto{\pgfqpoint{2.409390in}{2.719407in}}{\pgfqpoint{2.405000in}{2.730006in}}{\pgfqpoint{2.397187in}{2.737819in}}%
\pgfpathcurveto{\pgfqpoint{2.389373in}{2.745633in}}{\pgfqpoint{2.378774in}{2.750023in}}{\pgfqpoint{2.367724in}{2.750023in}}%
\pgfpathcurveto{\pgfqpoint{2.356674in}{2.750023in}}{\pgfqpoint{2.346075in}{2.745633in}}{\pgfqpoint{2.338261in}{2.737819in}}%
\pgfpathcurveto{\pgfqpoint{2.330447in}{2.730006in}}{\pgfqpoint{2.326057in}{2.719407in}}{\pgfqpoint{2.326057in}{2.708356in}}%
\pgfpathcurveto{\pgfqpoint{2.326057in}{2.697306in}}{\pgfqpoint{2.330447in}{2.686707in}}{\pgfqpoint{2.338261in}{2.678894in}}%
\pgfpathcurveto{\pgfqpoint{2.346075in}{2.671080in}}{\pgfqpoint{2.356674in}{2.666690in}}{\pgfqpoint{2.367724in}{2.666690in}}%
\pgfpathclose%
\pgfusepath{stroke,fill}%
\end{pgfscope}%
\begin{pgfscope}%
\pgfpathrectangle{\pgfqpoint{0.787074in}{0.548769in}}{\pgfqpoint{5.062926in}{3.102590in}}%
\pgfusepath{clip}%
\pgfsetbuttcap%
\pgfsetroundjoin%
\definecolor{currentfill}{rgb}{1.000000,0.498039,0.054902}%
\pgfsetfillcolor{currentfill}%
\pgfsetlinewidth{1.003750pt}%
\definecolor{currentstroke}{rgb}{1.000000,0.498039,0.054902}%
\pgfsetstrokecolor{currentstroke}%
\pgfsetdash{}{0pt}%
\pgfpathmoveto{\pgfqpoint{1.878765in}{2.209697in}}%
\pgfpathcurveto{\pgfqpoint{1.889815in}{2.209697in}}{\pgfqpoint{1.900414in}{2.214087in}}{\pgfqpoint{1.908228in}{2.221901in}}%
\pgfpathcurveto{\pgfqpoint{1.916042in}{2.229714in}}{\pgfqpoint{1.920432in}{2.240313in}}{\pgfqpoint{1.920432in}{2.251363in}}%
\pgfpathcurveto{\pgfqpoint{1.920432in}{2.262414in}}{\pgfqpoint{1.916042in}{2.273013in}}{\pgfqpoint{1.908228in}{2.280826in}}%
\pgfpathcurveto{\pgfqpoint{1.900414in}{2.288640in}}{\pgfqpoint{1.889815in}{2.293030in}}{\pgfqpoint{1.878765in}{2.293030in}}%
\pgfpathcurveto{\pgfqpoint{1.867715in}{2.293030in}}{\pgfqpoint{1.857116in}{2.288640in}}{\pgfqpoint{1.849302in}{2.280826in}}%
\pgfpathcurveto{\pgfqpoint{1.841489in}{2.273013in}}{\pgfqpoint{1.837099in}{2.262414in}}{\pgfqpoint{1.837099in}{2.251363in}}%
\pgfpathcurveto{\pgfqpoint{1.837099in}{2.240313in}}{\pgfqpoint{1.841489in}{2.229714in}}{\pgfqpoint{1.849302in}{2.221901in}}%
\pgfpathcurveto{\pgfqpoint{1.857116in}{2.214087in}}{\pgfqpoint{1.867715in}{2.209697in}}{\pgfqpoint{1.878765in}{2.209697in}}%
\pgfpathclose%
\pgfusepath{stroke,fill}%
\end{pgfscope}%
\begin{pgfscope}%
\pgfpathrectangle{\pgfqpoint{0.787074in}{0.548769in}}{\pgfqpoint{5.062926in}{3.102590in}}%
\pgfusepath{clip}%
\pgfsetbuttcap%
\pgfsetroundjoin%
\definecolor{currentfill}{rgb}{1.000000,0.498039,0.054902}%
\pgfsetfillcolor{currentfill}%
\pgfsetlinewidth{1.003750pt}%
\definecolor{currentstroke}{rgb}{1.000000,0.498039,0.054902}%
\pgfsetstrokecolor{currentstroke}%
\pgfsetdash{}{0pt}%
\pgfpathmoveto{\pgfqpoint{1.786538in}{1.987916in}}%
\pgfpathcurveto{\pgfqpoint{1.797588in}{1.987916in}}{\pgfqpoint{1.808187in}{1.992306in}}{\pgfqpoint{1.816000in}{2.000120in}}%
\pgfpathcurveto{\pgfqpoint{1.823814in}{2.007934in}}{\pgfqpoint{1.828204in}{2.018533in}}{\pgfqpoint{1.828204in}{2.029583in}}%
\pgfpathcurveto{\pgfqpoint{1.828204in}{2.040633in}}{\pgfqpoint{1.823814in}{2.051232in}}{\pgfqpoint{1.816000in}{2.059045in}}%
\pgfpathcurveto{\pgfqpoint{1.808187in}{2.066859in}}{\pgfqpoint{1.797588in}{2.071249in}}{\pgfqpoint{1.786538in}{2.071249in}}%
\pgfpathcurveto{\pgfqpoint{1.775487in}{2.071249in}}{\pgfqpoint{1.764888in}{2.066859in}}{\pgfqpoint{1.757075in}{2.059045in}}%
\pgfpathcurveto{\pgfqpoint{1.749261in}{2.051232in}}{\pgfqpoint{1.744871in}{2.040633in}}{\pgfqpoint{1.744871in}{2.029583in}}%
\pgfpathcurveto{\pgfqpoint{1.744871in}{2.018533in}}{\pgfqpoint{1.749261in}{2.007934in}}{\pgfqpoint{1.757075in}{2.000120in}}%
\pgfpathcurveto{\pgfqpoint{1.764888in}{1.992306in}}{\pgfqpoint{1.775487in}{1.987916in}}{\pgfqpoint{1.786538in}{1.987916in}}%
\pgfpathclose%
\pgfusepath{stroke,fill}%
\end{pgfscope}%
\begin{pgfscope}%
\pgfpathrectangle{\pgfqpoint{0.787074in}{0.548769in}}{\pgfqpoint{5.062926in}{3.102590in}}%
\pgfusepath{clip}%
\pgfsetbuttcap%
\pgfsetroundjoin%
\definecolor{currentfill}{rgb}{1.000000,0.498039,0.054902}%
\pgfsetfillcolor{currentfill}%
\pgfsetlinewidth{1.003750pt}%
\definecolor{currentstroke}{rgb}{1.000000,0.498039,0.054902}%
\pgfsetstrokecolor{currentstroke}%
\pgfsetdash{}{0pt}%
\pgfpathmoveto{\pgfqpoint{1.731765in}{2.610441in}}%
\pgfpathcurveto{\pgfqpoint{1.742815in}{2.610441in}}{\pgfqpoint{1.753414in}{2.614831in}}{\pgfqpoint{1.761228in}{2.622645in}}%
\pgfpathcurveto{\pgfqpoint{1.769042in}{2.630459in}}{\pgfqpoint{1.773432in}{2.641058in}}{\pgfqpoint{1.773432in}{2.652108in}}%
\pgfpathcurveto{\pgfqpoint{1.773432in}{2.663158in}}{\pgfqpoint{1.769042in}{2.673757in}}{\pgfqpoint{1.761228in}{2.681571in}}%
\pgfpathcurveto{\pgfqpoint{1.753414in}{2.689384in}}{\pgfqpoint{1.742815in}{2.693774in}}{\pgfqpoint{1.731765in}{2.693774in}}%
\pgfpathcurveto{\pgfqpoint{1.720715in}{2.693774in}}{\pgfqpoint{1.710116in}{2.689384in}}{\pgfqpoint{1.702302in}{2.681571in}}%
\pgfpathcurveto{\pgfqpoint{1.694489in}{2.673757in}}{\pgfqpoint{1.690099in}{2.663158in}}{\pgfqpoint{1.690099in}{2.652108in}}%
\pgfpathcurveto{\pgfqpoint{1.690099in}{2.641058in}}{\pgfqpoint{1.694489in}{2.630459in}}{\pgfqpoint{1.702302in}{2.622645in}}%
\pgfpathcurveto{\pgfqpoint{1.710116in}{2.614831in}}{\pgfqpoint{1.720715in}{2.610441in}}{\pgfqpoint{1.731765in}{2.610441in}}%
\pgfpathclose%
\pgfusepath{stroke,fill}%
\end{pgfscope}%
\begin{pgfscope}%
\pgfpathrectangle{\pgfqpoint{0.787074in}{0.548769in}}{\pgfqpoint{5.062926in}{3.102590in}}%
\pgfusepath{clip}%
\pgfsetbuttcap%
\pgfsetroundjoin%
\definecolor{currentfill}{rgb}{1.000000,0.498039,0.054902}%
\pgfsetfillcolor{currentfill}%
\pgfsetlinewidth{1.003750pt}%
\definecolor{currentstroke}{rgb}{1.000000,0.498039,0.054902}%
\pgfsetstrokecolor{currentstroke}%
\pgfsetdash{}{0pt}%
\pgfpathmoveto{\pgfqpoint{1.456384in}{1.913221in}}%
\pgfpathcurveto{\pgfqpoint{1.467434in}{1.913221in}}{\pgfqpoint{1.478033in}{1.917612in}}{\pgfqpoint{1.485847in}{1.925425in}}%
\pgfpathcurveto{\pgfqpoint{1.493661in}{1.933239in}}{\pgfqpoint{1.498051in}{1.943838in}}{\pgfqpoint{1.498051in}{1.954888in}}%
\pgfpathcurveto{\pgfqpoint{1.498051in}{1.965938in}}{\pgfqpoint{1.493661in}{1.976537in}}{\pgfqpoint{1.485847in}{1.984351in}}%
\pgfpathcurveto{\pgfqpoint{1.478033in}{1.992164in}}{\pgfqpoint{1.467434in}{1.996555in}}{\pgfqpoint{1.456384in}{1.996555in}}%
\pgfpathcurveto{\pgfqpoint{1.445334in}{1.996555in}}{\pgfqpoint{1.434735in}{1.992164in}}{\pgfqpoint{1.426921in}{1.984351in}}%
\pgfpathcurveto{\pgfqpoint{1.419108in}{1.976537in}}{\pgfqpoint{1.414718in}{1.965938in}}{\pgfqpoint{1.414718in}{1.954888in}}%
\pgfpathcurveto{\pgfqpoint{1.414718in}{1.943838in}}{\pgfqpoint{1.419108in}{1.933239in}}{\pgfqpoint{1.426921in}{1.925425in}}%
\pgfpathcurveto{\pgfqpoint{1.434735in}{1.917612in}}{\pgfqpoint{1.445334in}{1.913221in}}{\pgfqpoint{1.456384in}{1.913221in}}%
\pgfpathclose%
\pgfusepath{stroke,fill}%
\end{pgfscope}%
\begin{pgfscope}%
\pgfpathrectangle{\pgfqpoint{0.787074in}{0.548769in}}{\pgfqpoint{5.062926in}{3.102590in}}%
\pgfusepath{clip}%
\pgfsetbuttcap%
\pgfsetroundjoin%
\definecolor{currentfill}{rgb}{1.000000,0.498039,0.054902}%
\pgfsetfillcolor{currentfill}%
\pgfsetlinewidth{1.003750pt}%
\definecolor{currentstroke}{rgb}{1.000000,0.498039,0.054902}%
\pgfsetstrokecolor{currentstroke}%
\pgfsetdash{}{0pt}%
\pgfpathmoveto{\pgfqpoint{2.181012in}{2.130198in}}%
\pgfpathcurveto{\pgfqpoint{2.192062in}{2.130198in}}{\pgfqpoint{2.202661in}{2.134588in}}{\pgfqpoint{2.210474in}{2.142402in}}%
\pgfpathcurveto{\pgfqpoint{2.218288in}{2.150216in}}{\pgfqpoint{2.222678in}{2.160815in}}{\pgfqpoint{2.222678in}{2.171865in}}%
\pgfpathcurveto{\pgfqpoint{2.222678in}{2.182915in}}{\pgfqpoint{2.218288in}{2.193514in}}{\pgfqpoint{2.210474in}{2.201327in}}%
\pgfpathcurveto{\pgfqpoint{2.202661in}{2.209141in}}{\pgfqpoint{2.192062in}{2.213531in}}{\pgfqpoint{2.181012in}{2.213531in}}%
\pgfpathcurveto{\pgfqpoint{2.169961in}{2.213531in}}{\pgfqpoint{2.159362in}{2.209141in}}{\pgfqpoint{2.151549in}{2.201327in}}%
\pgfpathcurveto{\pgfqpoint{2.143735in}{2.193514in}}{\pgfqpoint{2.139345in}{2.182915in}}{\pgfqpoint{2.139345in}{2.171865in}}%
\pgfpathcurveto{\pgfqpoint{2.139345in}{2.160815in}}{\pgfqpoint{2.143735in}{2.150216in}}{\pgfqpoint{2.151549in}{2.142402in}}%
\pgfpathcurveto{\pgfqpoint{2.159362in}{2.134588in}}{\pgfqpoint{2.169961in}{2.130198in}}{\pgfqpoint{2.181012in}{2.130198in}}%
\pgfpathclose%
\pgfusepath{stroke,fill}%
\end{pgfscope}%
\begin{pgfscope}%
\pgfpathrectangle{\pgfqpoint{0.787074in}{0.548769in}}{\pgfqpoint{5.062926in}{3.102590in}}%
\pgfusepath{clip}%
\pgfsetbuttcap%
\pgfsetroundjoin%
\definecolor{currentfill}{rgb}{1.000000,0.498039,0.054902}%
\pgfsetfillcolor{currentfill}%
\pgfsetlinewidth{1.003750pt}%
\definecolor{currentstroke}{rgb}{1.000000,0.498039,0.054902}%
\pgfsetstrokecolor{currentstroke}%
\pgfsetdash{}{0pt}%
\pgfpathmoveto{\pgfqpoint{2.497537in}{1.747432in}}%
\pgfpathcurveto{\pgfqpoint{2.508587in}{1.747432in}}{\pgfqpoint{2.519186in}{1.751822in}}{\pgfqpoint{2.527000in}{1.759636in}}%
\pgfpathcurveto{\pgfqpoint{2.534813in}{1.767449in}}{\pgfqpoint{2.539204in}{1.778048in}}{\pgfqpoint{2.539204in}{1.789098in}}%
\pgfpathcurveto{\pgfqpoint{2.539204in}{1.800148in}}{\pgfqpoint{2.534813in}{1.810748in}}{\pgfqpoint{2.527000in}{1.818561in}}%
\pgfpathcurveto{\pgfqpoint{2.519186in}{1.826375in}}{\pgfqpoint{2.508587in}{1.830765in}}{\pgfqpoint{2.497537in}{1.830765in}}%
\pgfpathcurveto{\pgfqpoint{2.486487in}{1.830765in}}{\pgfqpoint{2.475888in}{1.826375in}}{\pgfqpoint{2.468074in}{1.818561in}}%
\pgfpathcurveto{\pgfqpoint{2.460261in}{1.810748in}}{\pgfqpoint{2.455870in}{1.800148in}}{\pgfqpoint{2.455870in}{1.789098in}}%
\pgfpathcurveto{\pgfqpoint{2.455870in}{1.778048in}}{\pgfqpoint{2.460261in}{1.767449in}}{\pgfqpoint{2.468074in}{1.759636in}}%
\pgfpathcurveto{\pgfqpoint{2.475888in}{1.751822in}}{\pgfqpoint{2.486487in}{1.747432in}}{\pgfqpoint{2.497537in}{1.747432in}}%
\pgfpathclose%
\pgfusepath{stroke,fill}%
\end{pgfscope}%
\begin{pgfscope}%
\pgfpathrectangle{\pgfqpoint{0.787074in}{0.548769in}}{\pgfqpoint{5.062926in}{3.102590in}}%
\pgfusepath{clip}%
\pgfsetbuttcap%
\pgfsetroundjoin%
\definecolor{currentfill}{rgb}{1.000000,0.498039,0.054902}%
\pgfsetfillcolor{currentfill}%
\pgfsetlinewidth{1.003750pt}%
\definecolor{currentstroke}{rgb}{1.000000,0.498039,0.054902}%
\pgfsetstrokecolor{currentstroke}%
\pgfsetdash{}{0pt}%
\pgfpathmoveto{\pgfqpoint{4.331457in}{2.135660in}}%
\pgfpathcurveto{\pgfqpoint{4.342507in}{2.135660in}}{\pgfqpoint{4.353106in}{2.140050in}}{\pgfqpoint{4.360920in}{2.147864in}}%
\pgfpathcurveto{\pgfqpoint{4.368734in}{2.155678in}}{\pgfqpoint{4.373124in}{2.166277in}}{\pgfqpoint{4.373124in}{2.177327in}}%
\pgfpathcurveto{\pgfqpoint{4.373124in}{2.188377in}}{\pgfqpoint{4.368734in}{2.198976in}}{\pgfqpoint{4.360920in}{2.206789in}}%
\pgfpathcurveto{\pgfqpoint{4.353106in}{2.214603in}}{\pgfqpoint{4.342507in}{2.218993in}}{\pgfqpoint{4.331457in}{2.218993in}}%
\pgfpathcurveto{\pgfqpoint{4.320407in}{2.218993in}}{\pgfqpoint{4.309808in}{2.214603in}}{\pgfqpoint{4.301994in}{2.206789in}}%
\pgfpathcurveto{\pgfqpoint{4.294181in}{2.198976in}}{\pgfqpoint{4.289791in}{2.188377in}}{\pgfqpoint{4.289791in}{2.177327in}}%
\pgfpathcurveto{\pgfqpoint{4.289791in}{2.166277in}}{\pgfqpoint{4.294181in}{2.155678in}}{\pgfqpoint{4.301994in}{2.147864in}}%
\pgfpathcurveto{\pgfqpoint{4.309808in}{2.140050in}}{\pgfqpoint{4.320407in}{2.135660in}}{\pgfqpoint{4.331457in}{2.135660in}}%
\pgfpathclose%
\pgfusepath{stroke,fill}%
\end{pgfscope}%
\begin{pgfscope}%
\pgfpathrectangle{\pgfqpoint{0.787074in}{0.548769in}}{\pgfqpoint{5.062926in}{3.102590in}}%
\pgfusepath{clip}%
\pgfsetbuttcap%
\pgfsetroundjoin%
\definecolor{currentfill}{rgb}{1.000000,0.498039,0.054902}%
\pgfsetfillcolor{currentfill}%
\pgfsetlinewidth{1.003750pt}%
\definecolor{currentstroke}{rgb}{1.000000,0.498039,0.054902}%
\pgfsetstrokecolor{currentstroke}%
\pgfsetdash{}{0pt}%
\pgfpathmoveto{\pgfqpoint{1.679597in}{1.976501in}}%
\pgfpathcurveto{\pgfqpoint{1.690647in}{1.976501in}}{\pgfqpoint{1.701246in}{1.980891in}}{\pgfqpoint{1.709060in}{1.988704in}}%
\pgfpathcurveto{\pgfqpoint{1.716873in}{1.996518in}}{\pgfqpoint{1.721264in}{2.007117in}}{\pgfqpoint{1.721264in}{2.018167in}}%
\pgfpathcurveto{\pgfqpoint{1.721264in}{2.029217in}}{\pgfqpoint{1.716873in}{2.039816in}}{\pgfqpoint{1.709060in}{2.047630in}}%
\pgfpathcurveto{\pgfqpoint{1.701246in}{2.055444in}}{\pgfqpoint{1.690647in}{2.059834in}}{\pgfqpoint{1.679597in}{2.059834in}}%
\pgfpathcurveto{\pgfqpoint{1.668547in}{2.059834in}}{\pgfqpoint{1.657948in}{2.055444in}}{\pgfqpoint{1.650134in}{2.047630in}}%
\pgfpathcurveto{\pgfqpoint{1.642320in}{2.039816in}}{\pgfqpoint{1.637930in}{2.029217in}}{\pgfqpoint{1.637930in}{2.018167in}}%
\pgfpathcurveto{\pgfqpoint{1.637930in}{2.007117in}}{\pgfqpoint{1.642320in}{1.996518in}}{\pgfqpoint{1.650134in}{1.988704in}}%
\pgfpathcurveto{\pgfqpoint{1.657948in}{1.980891in}}{\pgfqpoint{1.668547in}{1.976501in}}{\pgfqpoint{1.679597in}{1.976501in}}%
\pgfpathclose%
\pgfusepath{stroke,fill}%
\end{pgfscope}%
\begin{pgfscope}%
\pgfpathrectangle{\pgfqpoint{0.787074in}{0.548769in}}{\pgfqpoint{5.062926in}{3.102590in}}%
\pgfusepath{clip}%
\pgfsetbuttcap%
\pgfsetroundjoin%
\definecolor{currentfill}{rgb}{1.000000,0.498039,0.054902}%
\pgfsetfillcolor{currentfill}%
\pgfsetlinewidth{1.003750pt}%
\definecolor{currentstroke}{rgb}{1.000000,0.498039,0.054902}%
\pgfsetstrokecolor{currentstroke}%
\pgfsetdash{}{0pt}%
\pgfpathmoveto{\pgfqpoint{1.995992in}{2.141633in}}%
\pgfpathcurveto{\pgfqpoint{2.007042in}{2.141633in}}{\pgfqpoint{2.017641in}{2.146024in}}{\pgfqpoint{2.025455in}{2.153837in}}%
\pgfpathcurveto{\pgfqpoint{2.033268in}{2.161651in}}{\pgfqpoint{2.037659in}{2.172250in}}{\pgfqpoint{2.037659in}{2.183300in}}%
\pgfpathcurveto{\pgfqpoint{2.037659in}{2.194350in}}{\pgfqpoint{2.033268in}{2.204949in}}{\pgfqpoint{2.025455in}{2.212763in}}%
\pgfpathcurveto{\pgfqpoint{2.017641in}{2.220577in}}{\pgfqpoint{2.007042in}{2.224967in}}{\pgfqpoint{1.995992in}{2.224967in}}%
\pgfpathcurveto{\pgfqpoint{1.984942in}{2.224967in}}{\pgfqpoint{1.974343in}{2.220577in}}{\pgfqpoint{1.966529in}{2.212763in}}%
\pgfpathcurveto{\pgfqpoint{1.958716in}{2.204949in}}{\pgfqpoint{1.954325in}{2.194350in}}{\pgfqpoint{1.954325in}{2.183300in}}%
\pgfpathcurveto{\pgfqpoint{1.954325in}{2.172250in}}{\pgfqpoint{1.958716in}{2.161651in}}{\pgfqpoint{1.966529in}{2.153837in}}%
\pgfpathcurveto{\pgfqpoint{1.974343in}{2.146024in}}{\pgfqpoint{1.984942in}{2.141633in}}{\pgfqpoint{1.995992in}{2.141633in}}%
\pgfpathclose%
\pgfusepath{stroke,fill}%
\end{pgfscope}%
\begin{pgfscope}%
\pgfpathrectangle{\pgfqpoint{0.787074in}{0.548769in}}{\pgfqpoint{5.062926in}{3.102590in}}%
\pgfusepath{clip}%
\pgfsetbuttcap%
\pgfsetroundjoin%
\definecolor{currentfill}{rgb}{1.000000,0.498039,0.054902}%
\pgfsetfillcolor{currentfill}%
\pgfsetlinewidth{1.003750pt}%
\definecolor{currentstroke}{rgb}{1.000000,0.498039,0.054902}%
\pgfsetstrokecolor{currentstroke}%
\pgfsetdash{}{0pt}%
\pgfpathmoveto{\pgfqpoint{2.053672in}{2.039016in}}%
\pgfpathcurveto{\pgfqpoint{2.064722in}{2.039016in}}{\pgfqpoint{2.075321in}{2.043407in}}{\pgfqpoint{2.083135in}{2.051220in}}%
\pgfpathcurveto{\pgfqpoint{2.090949in}{2.059034in}}{\pgfqpoint{2.095339in}{2.069633in}}{\pgfqpoint{2.095339in}{2.080683in}}%
\pgfpathcurveto{\pgfqpoint{2.095339in}{2.091733in}}{\pgfqpoint{2.090949in}{2.102332in}}{\pgfqpoint{2.083135in}{2.110146in}}%
\pgfpathcurveto{\pgfqpoint{2.075321in}{2.117960in}}{\pgfqpoint{2.064722in}{2.122350in}}{\pgfqpoint{2.053672in}{2.122350in}}%
\pgfpathcurveto{\pgfqpoint{2.042622in}{2.122350in}}{\pgfqpoint{2.032023in}{2.117960in}}{\pgfqpoint{2.024210in}{2.110146in}}%
\pgfpathcurveto{\pgfqpoint{2.016396in}{2.102332in}}{\pgfqpoint{2.012006in}{2.091733in}}{\pgfqpoint{2.012006in}{2.080683in}}%
\pgfpathcurveto{\pgfqpoint{2.012006in}{2.069633in}}{\pgfqpoint{2.016396in}{2.059034in}}{\pgfqpoint{2.024210in}{2.051220in}}%
\pgfpathcurveto{\pgfqpoint{2.032023in}{2.043407in}}{\pgfqpoint{2.042622in}{2.039016in}}{\pgfqpoint{2.053672in}{2.039016in}}%
\pgfpathclose%
\pgfusepath{stroke,fill}%
\end{pgfscope}%
\begin{pgfscope}%
\pgfpathrectangle{\pgfqpoint{0.787074in}{0.548769in}}{\pgfqpoint{5.062926in}{3.102590in}}%
\pgfusepath{clip}%
\pgfsetbuttcap%
\pgfsetroundjoin%
\definecolor{currentfill}{rgb}{0.121569,0.466667,0.705882}%
\pgfsetfillcolor{currentfill}%
\pgfsetlinewidth{1.003750pt}%
\definecolor{currentstroke}{rgb}{0.121569,0.466667,0.705882}%
\pgfsetstrokecolor{currentstroke}%
\pgfsetdash{}{0pt}%
\pgfpathmoveto{\pgfqpoint{2.413990in}{1.981410in}}%
\pgfpathcurveto{\pgfqpoint{2.425040in}{1.981410in}}{\pgfqpoint{2.435639in}{1.985801in}}{\pgfqpoint{2.443452in}{1.993614in}}%
\pgfpathcurveto{\pgfqpoint{2.451266in}{2.001428in}}{\pgfqpoint{2.455656in}{2.012027in}}{\pgfqpoint{2.455656in}{2.023077in}}%
\pgfpathcurveto{\pgfqpoint{2.455656in}{2.034127in}}{\pgfqpoint{2.451266in}{2.044726in}}{\pgfqpoint{2.443452in}{2.052540in}}%
\pgfpathcurveto{\pgfqpoint{2.435639in}{2.060353in}}{\pgfqpoint{2.425040in}{2.064744in}}{\pgfqpoint{2.413990in}{2.064744in}}%
\pgfpathcurveto{\pgfqpoint{2.402939in}{2.064744in}}{\pgfqpoint{2.392340in}{2.060353in}}{\pgfqpoint{2.384527in}{2.052540in}}%
\pgfpathcurveto{\pgfqpoint{2.376713in}{2.044726in}}{\pgfqpoint{2.372323in}{2.034127in}}{\pgfqpoint{2.372323in}{2.023077in}}%
\pgfpathcurveto{\pgfqpoint{2.372323in}{2.012027in}}{\pgfqpoint{2.376713in}{2.001428in}}{\pgfqpoint{2.384527in}{1.993614in}}%
\pgfpathcurveto{\pgfqpoint{2.392340in}{1.985801in}}{\pgfqpoint{2.402939in}{1.981410in}}{\pgfqpoint{2.413990in}{1.981410in}}%
\pgfpathclose%
\pgfusepath{stroke,fill}%
\end{pgfscope}%
\begin{pgfscope}%
\pgfpathrectangle{\pgfqpoint{0.787074in}{0.548769in}}{\pgfqpoint{5.062926in}{3.102590in}}%
\pgfusepath{clip}%
\pgfsetbuttcap%
\pgfsetroundjoin%
\definecolor{currentfill}{rgb}{1.000000,0.498039,0.054902}%
\pgfsetfillcolor{currentfill}%
\pgfsetlinewidth{1.003750pt}%
\definecolor{currentstroke}{rgb}{1.000000,0.498039,0.054902}%
\pgfsetstrokecolor{currentstroke}%
\pgfsetdash{}{0pt}%
\pgfpathmoveto{\pgfqpoint{2.079236in}{2.386079in}}%
\pgfpathcurveto{\pgfqpoint{2.090286in}{2.386079in}}{\pgfqpoint{2.100885in}{2.390469in}}{\pgfqpoint{2.108698in}{2.398283in}}%
\pgfpathcurveto{\pgfqpoint{2.116512in}{2.406097in}}{\pgfqpoint{2.120902in}{2.416696in}}{\pgfqpoint{2.120902in}{2.427746in}}%
\pgfpathcurveto{\pgfqpoint{2.120902in}{2.438796in}}{\pgfqpoint{2.116512in}{2.449395in}}{\pgfqpoint{2.108698in}{2.457209in}}%
\pgfpathcurveto{\pgfqpoint{2.100885in}{2.465022in}}{\pgfqpoint{2.090286in}{2.469412in}}{\pgfqpoint{2.079236in}{2.469412in}}%
\pgfpathcurveto{\pgfqpoint{2.068186in}{2.469412in}}{\pgfqpoint{2.057586in}{2.465022in}}{\pgfqpoint{2.049773in}{2.457209in}}%
\pgfpathcurveto{\pgfqpoint{2.041959in}{2.449395in}}{\pgfqpoint{2.037569in}{2.438796in}}{\pgfqpoint{2.037569in}{2.427746in}}%
\pgfpathcurveto{\pgfqpoint{2.037569in}{2.416696in}}{\pgfqpoint{2.041959in}{2.406097in}}{\pgfqpoint{2.049773in}{2.398283in}}%
\pgfpathcurveto{\pgfqpoint{2.057586in}{2.390469in}}{\pgfqpoint{2.068186in}{2.386079in}}{\pgfqpoint{2.079236in}{2.386079in}}%
\pgfpathclose%
\pgfusepath{stroke,fill}%
\end{pgfscope}%
\begin{pgfscope}%
\pgfpathrectangle{\pgfqpoint{0.787074in}{0.548769in}}{\pgfqpoint{5.062926in}{3.102590in}}%
\pgfusepath{clip}%
\pgfsetbuttcap%
\pgfsetroundjoin%
\definecolor{currentfill}{rgb}{1.000000,0.498039,0.054902}%
\pgfsetfillcolor{currentfill}%
\pgfsetlinewidth{1.003750pt}%
\definecolor{currentstroke}{rgb}{1.000000,0.498039,0.054902}%
\pgfsetstrokecolor{currentstroke}%
\pgfsetdash{}{0pt}%
\pgfpathmoveto{\pgfqpoint{2.073376in}{2.446539in}}%
\pgfpathcurveto{\pgfqpoint{2.084427in}{2.446539in}}{\pgfqpoint{2.095026in}{2.450929in}}{\pgfqpoint{2.102839in}{2.458743in}}%
\pgfpathcurveto{\pgfqpoint{2.110653in}{2.466556in}}{\pgfqpoint{2.115043in}{2.477155in}}{\pgfqpoint{2.115043in}{2.488206in}}%
\pgfpathcurveto{\pgfqpoint{2.115043in}{2.499256in}}{\pgfqpoint{2.110653in}{2.509855in}}{\pgfqpoint{2.102839in}{2.517668in}}%
\pgfpathcurveto{\pgfqpoint{2.095026in}{2.525482in}}{\pgfqpoint{2.084427in}{2.529872in}}{\pgfqpoint{2.073376in}{2.529872in}}%
\pgfpathcurveto{\pgfqpoint{2.062326in}{2.529872in}}{\pgfqpoint{2.051727in}{2.525482in}}{\pgfqpoint{2.043914in}{2.517668in}}%
\pgfpathcurveto{\pgfqpoint{2.036100in}{2.509855in}}{\pgfqpoint{2.031710in}{2.499256in}}{\pgfqpoint{2.031710in}{2.488206in}}%
\pgfpathcurveto{\pgfqpoint{2.031710in}{2.477155in}}{\pgfqpoint{2.036100in}{2.466556in}}{\pgfqpoint{2.043914in}{2.458743in}}%
\pgfpathcurveto{\pgfqpoint{2.051727in}{2.450929in}}{\pgfqpoint{2.062326in}{2.446539in}}{\pgfqpoint{2.073376in}{2.446539in}}%
\pgfpathclose%
\pgfusepath{stroke,fill}%
\end{pgfscope}%
\begin{pgfscope}%
\pgfpathrectangle{\pgfqpoint{0.787074in}{0.548769in}}{\pgfqpoint{5.062926in}{3.102590in}}%
\pgfusepath{clip}%
\pgfsetbuttcap%
\pgfsetroundjoin%
\definecolor{currentfill}{rgb}{1.000000,0.498039,0.054902}%
\pgfsetfillcolor{currentfill}%
\pgfsetlinewidth{1.003750pt}%
\definecolor{currentstroke}{rgb}{1.000000,0.498039,0.054902}%
\pgfsetstrokecolor{currentstroke}%
\pgfsetdash{}{0pt}%
\pgfpathmoveto{\pgfqpoint{2.733553in}{2.604778in}}%
\pgfpathcurveto{\pgfqpoint{2.744603in}{2.604778in}}{\pgfqpoint{2.755202in}{2.609168in}}{\pgfqpoint{2.763016in}{2.616982in}}%
\pgfpathcurveto{\pgfqpoint{2.770829in}{2.624795in}}{\pgfqpoint{2.775220in}{2.635394in}}{\pgfqpoint{2.775220in}{2.646444in}}%
\pgfpathcurveto{\pgfqpoint{2.775220in}{2.657495in}}{\pgfqpoint{2.770829in}{2.668094in}}{\pgfqpoint{2.763016in}{2.675907in}}%
\pgfpathcurveto{\pgfqpoint{2.755202in}{2.683721in}}{\pgfqpoint{2.744603in}{2.688111in}}{\pgfqpoint{2.733553in}{2.688111in}}%
\pgfpathcurveto{\pgfqpoint{2.722503in}{2.688111in}}{\pgfqpoint{2.711904in}{2.683721in}}{\pgfqpoint{2.704090in}{2.675907in}}%
\pgfpathcurveto{\pgfqpoint{2.696277in}{2.668094in}}{\pgfqpoint{2.691886in}{2.657495in}}{\pgfqpoint{2.691886in}{2.646444in}}%
\pgfpathcurveto{\pgfqpoint{2.691886in}{2.635394in}}{\pgfqpoint{2.696277in}{2.624795in}}{\pgfqpoint{2.704090in}{2.616982in}}%
\pgfpathcurveto{\pgfqpoint{2.711904in}{2.609168in}}{\pgfqpoint{2.722503in}{2.604778in}}{\pgfqpoint{2.733553in}{2.604778in}}%
\pgfpathclose%
\pgfusepath{stroke,fill}%
\end{pgfscope}%
\begin{pgfscope}%
\pgfpathrectangle{\pgfqpoint{0.787074in}{0.548769in}}{\pgfqpoint{5.062926in}{3.102590in}}%
\pgfusepath{clip}%
\pgfsetbuttcap%
\pgfsetroundjoin%
\definecolor{currentfill}{rgb}{1.000000,0.498039,0.054902}%
\pgfsetfillcolor{currentfill}%
\pgfsetlinewidth{1.003750pt}%
\definecolor{currentstroke}{rgb}{1.000000,0.498039,0.054902}%
\pgfsetstrokecolor{currentstroke}%
\pgfsetdash{}{0pt}%
\pgfpathmoveto{\pgfqpoint{1.564844in}{2.291595in}}%
\pgfpathcurveto{\pgfqpoint{1.575894in}{2.291595in}}{\pgfqpoint{1.586493in}{2.295985in}}{\pgfqpoint{1.594307in}{2.303799in}}%
\pgfpathcurveto{\pgfqpoint{1.602120in}{2.311612in}}{\pgfqpoint{1.606511in}{2.322211in}}{\pgfqpoint{1.606511in}{2.333261in}}%
\pgfpathcurveto{\pgfqpoint{1.606511in}{2.344312in}}{\pgfqpoint{1.602120in}{2.354911in}}{\pgfqpoint{1.594307in}{2.362724in}}%
\pgfpathcurveto{\pgfqpoint{1.586493in}{2.370538in}}{\pgfqpoint{1.575894in}{2.374928in}}{\pgfqpoint{1.564844in}{2.374928in}}%
\pgfpathcurveto{\pgfqpoint{1.553794in}{2.374928in}}{\pgfqpoint{1.543195in}{2.370538in}}{\pgfqpoint{1.535381in}{2.362724in}}%
\pgfpathcurveto{\pgfqpoint{1.527568in}{2.354911in}}{\pgfqpoint{1.523177in}{2.344312in}}{\pgfqpoint{1.523177in}{2.333261in}}%
\pgfpathcurveto{\pgfqpoint{1.523177in}{2.322211in}}{\pgfqpoint{1.527568in}{2.311612in}}{\pgfqpoint{1.535381in}{2.303799in}}%
\pgfpathcurveto{\pgfqpoint{1.543195in}{2.295985in}}{\pgfqpoint{1.553794in}{2.291595in}}{\pgfqpoint{1.564844in}{2.291595in}}%
\pgfpathclose%
\pgfusepath{stroke,fill}%
\end{pgfscope}%
\begin{pgfscope}%
\pgfpathrectangle{\pgfqpoint{0.787074in}{0.548769in}}{\pgfqpoint{5.062926in}{3.102590in}}%
\pgfusepath{clip}%
\pgfsetbuttcap%
\pgfsetroundjoin%
\definecolor{currentfill}{rgb}{0.121569,0.466667,0.705882}%
\pgfsetfillcolor{currentfill}%
\pgfsetlinewidth{1.003750pt}%
\definecolor{currentstroke}{rgb}{0.121569,0.466667,0.705882}%
\pgfsetstrokecolor{currentstroke}%
\pgfsetdash{}{0pt}%
\pgfpathmoveto{\pgfqpoint{2.130710in}{1.765305in}}%
\pgfpathcurveto{\pgfqpoint{2.141760in}{1.765305in}}{\pgfqpoint{2.152359in}{1.769696in}}{\pgfqpoint{2.160172in}{1.777509in}}%
\pgfpathcurveto{\pgfqpoint{2.167986in}{1.785323in}}{\pgfqpoint{2.172376in}{1.795922in}}{\pgfqpoint{2.172376in}{1.806972in}}%
\pgfpathcurveto{\pgfqpoint{2.172376in}{1.818022in}}{\pgfqpoint{2.167986in}{1.828621in}}{\pgfqpoint{2.160172in}{1.836435in}}%
\pgfpathcurveto{\pgfqpoint{2.152359in}{1.844249in}}{\pgfqpoint{2.141760in}{1.848639in}}{\pgfqpoint{2.130710in}{1.848639in}}%
\pgfpathcurveto{\pgfqpoint{2.119659in}{1.848639in}}{\pgfqpoint{2.109060in}{1.844249in}}{\pgfqpoint{2.101247in}{1.836435in}}%
\pgfpathcurveto{\pgfqpoint{2.093433in}{1.828621in}}{\pgfqpoint{2.089043in}{1.818022in}}{\pgfqpoint{2.089043in}{1.806972in}}%
\pgfpathcurveto{\pgfqpoint{2.089043in}{1.795922in}}{\pgfqpoint{2.093433in}{1.785323in}}{\pgfqpoint{2.101247in}{1.777509in}}%
\pgfpathcurveto{\pgfqpoint{2.109060in}{1.769696in}}{\pgfqpoint{2.119659in}{1.765305in}}{\pgfqpoint{2.130710in}{1.765305in}}%
\pgfpathclose%
\pgfusepath{stroke,fill}%
\end{pgfscope}%
\begin{pgfscope}%
\pgfpathrectangle{\pgfqpoint{0.787074in}{0.548769in}}{\pgfqpoint{5.062926in}{3.102590in}}%
\pgfusepath{clip}%
\pgfsetbuttcap%
\pgfsetroundjoin%
\definecolor{currentfill}{rgb}{1.000000,0.498039,0.054902}%
\pgfsetfillcolor{currentfill}%
\pgfsetlinewidth{1.003750pt}%
\definecolor{currentstroke}{rgb}{1.000000,0.498039,0.054902}%
\pgfsetstrokecolor{currentstroke}%
\pgfsetdash{}{0pt}%
\pgfpathmoveto{\pgfqpoint{2.488076in}{2.168495in}}%
\pgfpathcurveto{\pgfqpoint{2.499126in}{2.168495in}}{\pgfqpoint{2.509725in}{2.172885in}}{\pgfqpoint{2.517538in}{2.180698in}}%
\pgfpathcurveto{\pgfqpoint{2.525352in}{2.188512in}}{\pgfqpoint{2.529742in}{2.199111in}}{\pgfqpoint{2.529742in}{2.210161in}}%
\pgfpathcurveto{\pgfqpoint{2.529742in}{2.221211in}}{\pgfqpoint{2.525352in}{2.231810in}}{\pgfqpoint{2.517538in}{2.239624in}}%
\pgfpathcurveto{\pgfqpoint{2.509725in}{2.247438in}}{\pgfqpoint{2.499126in}{2.251828in}}{\pgfqpoint{2.488076in}{2.251828in}}%
\pgfpathcurveto{\pgfqpoint{2.477025in}{2.251828in}}{\pgfqpoint{2.466426in}{2.247438in}}{\pgfqpoint{2.458613in}{2.239624in}}%
\pgfpathcurveto{\pgfqpoint{2.450799in}{2.231810in}}{\pgfqpoint{2.446409in}{2.221211in}}{\pgfqpoint{2.446409in}{2.210161in}}%
\pgfpathcurveto{\pgfqpoint{2.446409in}{2.199111in}}{\pgfqpoint{2.450799in}{2.188512in}}{\pgfqpoint{2.458613in}{2.180698in}}%
\pgfpathcurveto{\pgfqpoint{2.466426in}{2.172885in}}{\pgfqpoint{2.477025in}{2.168495in}}{\pgfqpoint{2.488076in}{2.168495in}}%
\pgfpathclose%
\pgfusepath{stroke,fill}%
\end{pgfscope}%
\begin{pgfscope}%
\pgfpathrectangle{\pgfqpoint{0.787074in}{0.548769in}}{\pgfqpoint{5.062926in}{3.102590in}}%
\pgfusepath{clip}%
\pgfsetbuttcap%
\pgfsetroundjoin%
\definecolor{currentfill}{rgb}{1.000000,0.498039,0.054902}%
\pgfsetfillcolor{currentfill}%
\pgfsetlinewidth{1.003750pt}%
\definecolor{currentstroke}{rgb}{1.000000,0.498039,0.054902}%
\pgfsetstrokecolor{currentstroke}%
\pgfsetdash{}{0pt}%
\pgfpathmoveto{\pgfqpoint{1.879416in}{2.330781in}}%
\pgfpathcurveto{\pgfqpoint{1.890466in}{2.330781in}}{\pgfqpoint{1.901065in}{2.335171in}}{\pgfqpoint{1.908879in}{2.342985in}}%
\pgfpathcurveto{\pgfqpoint{1.916693in}{2.350799in}}{\pgfqpoint{1.921083in}{2.361398in}}{\pgfqpoint{1.921083in}{2.372448in}}%
\pgfpathcurveto{\pgfqpoint{1.921083in}{2.383498in}}{\pgfqpoint{1.916693in}{2.394097in}}{\pgfqpoint{1.908879in}{2.401910in}}%
\pgfpathcurveto{\pgfqpoint{1.901065in}{2.409724in}}{\pgfqpoint{1.890466in}{2.414114in}}{\pgfqpoint{1.879416in}{2.414114in}}%
\pgfpathcurveto{\pgfqpoint{1.868366in}{2.414114in}}{\pgfqpoint{1.857767in}{2.409724in}}{\pgfqpoint{1.849953in}{2.401910in}}%
\pgfpathcurveto{\pgfqpoint{1.842140in}{2.394097in}}{\pgfqpoint{1.837750in}{2.383498in}}{\pgfqpoint{1.837750in}{2.372448in}}%
\pgfpathcurveto{\pgfqpoint{1.837750in}{2.361398in}}{\pgfqpoint{1.842140in}{2.350799in}}{\pgfqpoint{1.849953in}{2.342985in}}%
\pgfpathcurveto{\pgfqpoint{1.857767in}{2.335171in}}{\pgfqpoint{1.868366in}{2.330781in}}{\pgfqpoint{1.879416in}{2.330781in}}%
\pgfpathclose%
\pgfusepath{stroke,fill}%
\end{pgfscope}%
\begin{pgfscope}%
\pgfpathrectangle{\pgfqpoint{0.787074in}{0.548769in}}{\pgfqpoint{5.062926in}{3.102590in}}%
\pgfusepath{clip}%
\pgfsetbuttcap%
\pgfsetroundjoin%
\definecolor{currentfill}{rgb}{1.000000,0.498039,0.054902}%
\pgfsetfillcolor{currentfill}%
\pgfsetlinewidth{1.003750pt}%
\definecolor{currentstroke}{rgb}{1.000000,0.498039,0.054902}%
\pgfsetstrokecolor{currentstroke}%
\pgfsetdash{}{0pt}%
\pgfpathmoveto{\pgfqpoint{2.216687in}{2.341609in}}%
\pgfpathcurveto{\pgfqpoint{2.227738in}{2.341609in}}{\pgfqpoint{2.238337in}{2.345999in}}{\pgfqpoint{2.246150in}{2.353813in}}%
\pgfpathcurveto{\pgfqpoint{2.253964in}{2.361626in}}{\pgfqpoint{2.258354in}{2.372225in}}{\pgfqpoint{2.258354in}{2.383275in}}%
\pgfpathcurveto{\pgfqpoint{2.258354in}{2.394326in}}{\pgfqpoint{2.253964in}{2.404925in}}{\pgfqpoint{2.246150in}{2.412738in}}%
\pgfpathcurveto{\pgfqpoint{2.238337in}{2.420552in}}{\pgfqpoint{2.227738in}{2.424942in}}{\pgfqpoint{2.216687in}{2.424942in}}%
\pgfpathcurveto{\pgfqpoint{2.205637in}{2.424942in}}{\pgfqpoint{2.195038in}{2.420552in}}{\pgfqpoint{2.187225in}{2.412738in}}%
\pgfpathcurveto{\pgfqpoint{2.179411in}{2.404925in}}{\pgfqpoint{2.175021in}{2.394326in}}{\pgfqpoint{2.175021in}{2.383275in}}%
\pgfpathcurveto{\pgfqpoint{2.175021in}{2.372225in}}{\pgfqpoint{2.179411in}{2.361626in}}{\pgfqpoint{2.187225in}{2.353813in}}%
\pgfpathcurveto{\pgfqpoint{2.195038in}{2.345999in}}{\pgfqpoint{2.205637in}{2.341609in}}{\pgfqpoint{2.216687in}{2.341609in}}%
\pgfpathclose%
\pgfusepath{stroke,fill}%
\end{pgfscope}%
\begin{pgfscope}%
\pgfpathrectangle{\pgfqpoint{0.787074in}{0.548769in}}{\pgfqpoint{5.062926in}{3.102590in}}%
\pgfusepath{clip}%
\pgfsetbuttcap%
\pgfsetroundjoin%
\definecolor{currentfill}{rgb}{1.000000,0.498039,0.054902}%
\pgfsetfillcolor{currentfill}%
\pgfsetlinewidth{1.003750pt}%
\definecolor{currentstroke}{rgb}{1.000000,0.498039,0.054902}%
\pgfsetstrokecolor{currentstroke}%
\pgfsetdash{}{0pt}%
\pgfpathmoveto{\pgfqpoint{2.150414in}{2.708305in}}%
\pgfpathcurveto{\pgfqpoint{2.161464in}{2.708305in}}{\pgfqpoint{2.172063in}{2.712695in}}{\pgfqpoint{2.179877in}{2.720509in}}%
\pgfpathcurveto{\pgfqpoint{2.187690in}{2.728322in}}{\pgfqpoint{2.192080in}{2.738921in}}{\pgfqpoint{2.192080in}{2.749972in}}%
\pgfpathcurveto{\pgfqpoint{2.192080in}{2.761022in}}{\pgfqpoint{2.187690in}{2.771621in}}{\pgfqpoint{2.179877in}{2.779434in}}%
\pgfpathcurveto{\pgfqpoint{2.172063in}{2.787248in}}{\pgfqpoint{2.161464in}{2.791638in}}{\pgfqpoint{2.150414in}{2.791638in}}%
\pgfpathcurveto{\pgfqpoint{2.139364in}{2.791638in}}{\pgfqpoint{2.128765in}{2.787248in}}{\pgfqpoint{2.120951in}{2.779434in}}%
\pgfpathcurveto{\pgfqpoint{2.113137in}{2.771621in}}{\pgfqpoint{2.108747in}{2.761022in}}{\pgfqpoint{2.108747in}{2.749972in}}%
\pgfpathcurveto{\pgfqpoint{2.108747in}{2.738921in}}{\pgfqpoint{2.113137in}{2.728322in}}{\pgfqpoint{2.120951in}{2.720509in}}%
\pgfpathcurveto{\pgfqpoint{2.128765in}{2.712695in}}{\pgfqpoint{2.139364in}{2.708305in}}{\pgfqpoint{2.150414in}{2.708305in}}%
\pgfpathclose%
\pgfusepath{stroke,fill}%
\end{pgfscope}%
\begin{pgfscope}%
\pgfpathrectangle{\pgfqpoint{0.787074in}{0.548769in}}{\pgfqpoint{5.062926in}{3.102590in}}%
\pgfusepath{clip}%
\pgfsetbuttcap%
\pgfsetroundjoin%
\definecolor{currentfill}{rgb}{1.000000,0.498039,0.054902}%
\pgfsetfillcolor{currentfill}%
\pgfsetlinewidth{1.003750pt}%
\definecolor{currentstroke}{rgb}{1.000000,0.498039,0.054902}%
\pgfsetstrokecolor{currentstroke}%
\pgfsetdash{}{0pt}%
\pgfpathmoveto{\pgfqpoint{4.083593in}{2.886489in}}%
\pgfpathcurveto{\pgfqpoint{4.094643in}{2.886489in}}{\pgfqpoint{4.105242in}{2.890880in}}{\pgfqpoint{4.113055in}{2.898693in}}%
\pgfpathcurveto{\pgfqpoint{4.120869in}{2.906507in}}{\pgfqpoint{4.125259in}{2.917106in}}{\pgfqpoint{4.125259in}{2.928156in}}%
\pgfpathcurveto{\pgfqpoint{4.125259in}{2.939206in}}{\pgfqpoint{4.120869in}{2.949805in}}{\pgfqpoint{4.113055in}{2.957619in}}%
\pgfpathcurveto{\pgfqpoint{4.105242in}{2.965433in}}{\pgfqpoint{4.094643in}{2.969823in}}{\pgfqpoint{4.083593in}{2.969823in}}%
\pgfpathcurveto{\pgfqpoint{4.072542in}{2.969823in}}{\pgfqpoint{4.061943in}{2.965433in}}{\pgfqpoint{4.054130in}{2.957619in}}%
\pgfpathcurveto{\pgfqpoint{4.046316in}{2.949805in}}{\pgfqpoint{4.041926in}{2.939206in}}{\pgfqpoint{4.041926in}{2.928156in}}%
\pgfpathcurveto{\pgfqpoint{4.041926in}{2.917106in}}{\pgfqpoint{4.046316in}{2.906507in}}{\pgfqpoint{4.054130in}{2.898693in}}%
\pgfpathcurveto{\pgfqpoint{4.061943in}{2.890880in}}{\pgfqpoint{4.072542in}{2.886489in}}{\pgfqpoint{4.083593in}{2.886489in}}%
\pgfpathclose%
\pgfusepath{stroke,fill}%
\end{pgfscope}%
\begin{pgfscope}%
\pgfpathrectangle{\pgfqpoint{0.787074in}{0.548769in}}{\pgfqpoint{5.062926in}{3.102590in}}%
\pgfusepath{clip}%
\pgfsetbuttcap%
\pgfsetroundjoin%
\definecolor{currentfill}{rgb}{1.000000,0.498039,0.054902}%
\pgfsetfillcolor{currentfill}%
\pgfsetlinewidth{1.003750pt}%
\definecolor{currentstroke}{rgb}{1.000000,0.498039,0.054902}%
\pgfsetstrokecolor{currentstroke}%
\pgfsetdash{}{0pt}%
\pgfpathmoveto{\pgfqpoint{2.009707in}{2.003186in}}%
\pgfpathcurveto{\pgfqpoint{2.020757in}{2.003186in}}{\pgfqpoint{2.031356in}{2.007576in}}{\pgfqpoint{2.039170in}{2.015390in}}%
\pgfpathcurveto{\pgfqpoint{2.046983in}{2.023204in}}{\pgfqpoint{2.051374in}{2.033803in}}{\pgfqpoint{2.051374in}{2.044853in}}%
\pgfpathcurveto{\pgfqpoint{2.051374in}{2.055903in}}{\pgfqpoint{2.046983in}{2.066502in}}{\pgfqpoint{2.039170in}{2.074316in}}%
\pgfpathcurveto{\pgfqpoint{2.031356in}{2.082129in}}{\pgfqpoint{2.020757in}{2.086520in}}{\pgfqpoint{2.009707in}{2.086520in}}%
\pgfpathcurveto{\pgfqpoint{1.998657in}{2.086520in}}{\pgfqpoint{1.988058in}{2.082129in}}{\pgfqpoint{1.980244in}{2.074316in}}%
\pgfpathcurveto{\pgfqpoint{1.972430in}{2.066502in}}{\pgfqpoint{1.968040in}{2.055903in}}{\pgfqpoint{1.968040in}{2.044853in}}%
\pgfpathcurveto{\pgfqpoint{1.968040in}{2.033803in}}{\pgfqpoint{1.972430in}{2.023204in}}{\pgfqpoint{1.980244in}{2.015390in}}%
\pgfpathcurveto{\pgfqpoint{1.988058in}{2.007576in}}{\pgfqpoint{1.998657in}{2.003186in}}{\pgfqpoint{2.009707in}{2.003186in}}%
\pgfpathclose%
\pgfusepath{stroke,fill}%
\end{pgfscope}%
\begin{pgfscope}%
\pgfpathrectangle{\pgfqpoint{0.787074in}{0.548769in}}{\pgfqpoint{5.062926in}{3.102590in}}%
\pgfusepath{clip}%
\pgfsetbuttcap%
\pgfsetroundjoin%
\definecolor{currentfill}{rgb}{0.121569,0.466667,0.705882}%
\pgfsetfillcolor{currentfill}%
\pgfsetlinewidth{1.003750pt}%
\definecolor{currentstroke}{rgb}{0.121569,0.466667,0.705882}%
\pgfsetstrokecolor{currentstroke}%
\pgfsetdash{}{0pt}%
\pgfpathmoveto{\pgfqpoint{1.150883in}{0.942348in}}%
\pgfpathcurveto{\pgfqpoint{1.161933in}{0.942348in}}{\pgfqpoint{1.172532in}{0.946738in}}{\pgfqpoint{1.180346in}{0.954552in}}%
\pgfpathcurveto{\pgfqpoint{1.188159in}{0.962365in}}{\pgfqpoint{1.192549in}{0.972964in}}{\pgfqpoint{1.192549in}{0.984014in}}%
\pgfpathcurveto{\pgfqpoint{1.192549in}{0.995065in}}{\pgfqpoint{1.188159in}{1.005664in}}{\pgfqpoint{1.180346in}{1.013477in}}%
\pgfpathcurveto{\pgfqpoint{1.172532in}{1.021291in}}{\pgfqpoint{1.161933in}{1.025681in}}{\pgfqpoint{1.150883in}{1.025681in}}%
\pgfpathcurveto{\pgfqpoint{1.139833in}{1.025681in}}{\pgfqpoint{1.129234in}{1.021291in}}{\pgfqpoint{1.121420in}{1.013477in}}%
\pgfpathcurveto{\pgfqpoint{1.113606in}{1.005664in}}{\pgfqpoint{1.109216in}{0.995065in}}{\pgfqpoint{1.109216in}{0.984014in}}%
\pgfpathcurveto{\pgfqpoint{1.109216in}{0.972964in}}{\pgfqpoint{1.113606in}{0.962365in}}{\pgfqpoint{1.121420in}{0.954552in}}%
\pgfpathcurveto{\pgfqpoint{1.129234in}{0.946738in}}{\pgfqpoint{1.139833in}{0.942348in}}{\pgfqpoint{1.150883in}{0.942348in}}%
\pgfpathclose%
\pgfusepath{stroke,fill}%
\end{pgfscope}%
\begin{pgfscope}%
\pgfpathrectangle{\pgfqpoint{0.787074in}{0.548769in}}{\pgfqpoint{5.062926in}{3.102590in}}%
\pgfusepath{clip}%
\pgfsetbuttcap%
\pgfsetroundjoin%
\definecolor{currentfill}{rgb}{1.000000,0.498039,0.054902}%
\pgfsetfillcolor{currentfill}%
\pgfsetlinewidth{1.003750pt}%
\definecolor{currentstroke}{rgb}{1.000000,0.498039,0.054902}%
\pgfsetstrokecolor{currentstroke}%
\pgfsetdash{}{0pt}%
\pgfpathmoveto{\pgfqpoint{2.331701in}{2.620166in}}%
\pgfpathcurveto{\pgfqpoint{2.342751in}{2.620166in}}{\pgfqpoint{2.353350in}{2.624556in}}{\pgfqpoint{2.361164in}{2.632370in}}%
\pgfpathcurveto{\pgfqpoint{2.368977in}{2.640183in}}{\pgfqpoint{2.373367in}{2.650782in}}{\pgfqpoint{2.373367in}{2.661832in}}%
\pgfpathcurveto{\pgfqpoint{2.373367in}{2.672883in}}{\pgfqpoint{2.368977in}{2.683482in}}{\pgfqpoint{2.361164in}{2.691295in}}%
\pgfpathcurveto{\pgfqpoint{2.353350in}{2.699109in}}{\pgfqpoint{2.342751in}{2.703499in}}{\pgfqpoint{2.331701in}{2.703499in}}%
\pgfpathcurveto{\pgfqpoint{2.320651in}{2.703499in}}{\pgfqpoint{2.310052in}{2.699109in}}{\pgfqpoint{2.302238in}{2.691295in}}%
\pgfpathcurveto{\pgfqpoint{2.294424in}{2.683482in}}{\pgfqpoint{2.290034in}{2.672883in}}{\pgfqpoint{2.290034in}{2.661832in}}%
\pgfpathcurveto{\pgfqpoint{2.290034in}{2.650782in}}{\pgfqpoint{2.294424in}{2.640183in}}{\pgfqpoint{2.302238in}{2.632370in}}%
\pgfpathcurveto{\pgfqpoint{2.310052in}{2.624556in}}{\pgfqpoint{2.320651in}{2.620166in}}{\pgfqpoint{2.331701in}{2.620166in}}%
\pgfpathclose%
\pgfusepath{stroke,fill}%
\end{pgfscope}%
\begin{pgfscope}%
\pgfpathrectangle{\pgfqpoint{0.787074in}{0.548769in}}{\pgfqpoint{5.062926in}{3.102590in}}%
\pgfusepath{clip}%
\pgfsetbuttcap%
\pgfsetroundjoin%
\definecolor{currentfill}{rgb}{1.000000,0.498039,0.054902}%
\pgfsetfillcolor{currentfill}%
\pgfsetlinewidth{1.003750pt}%
\definecolor{currentstroke}{rgb}{1.000000,0.498039,0.054902}%
\pgfsetstrokecolor{currentstroke}%
\pgfsetdash{}{0pt}%
\pgfpathmoveto{\pgfqpoint{1.966262in}{2.991999in}}%
\pgfpathcurveto{\pgfqpoint{1.977312in}{2.991999in}}{\pgfqpoint{1.987911in}{2.996389in}}{\pgfqpoint{1.995725in}{3.004203in}}%
\pgfpathcurveto{\pgfqpoint{2.003539in}{3.012016in}}{\pgfqpoint{2.007929in}{3.022615in}}{\pgfqpoint{2.007929in}{3.033665in}}%
\pgfpathcurveto{\pgfqpoint{2.007929in}{3.044716in}}{\pgfqpoint{2.003539in}{3.055315in}}{\pgfqpoint{1.995725in}{3.063128in}}%
\pgfpathcurveto{\pgfqpoint{1.987911in}{3.070942in}}{\pgfqpoint{1.977312in}{3.075332in}}{\pgfqpoint{1.966262in}{3.075332in}}%
\pgfpathcurveto{\pgfqpoint{1.955212in}{3.075332in}}{\pgfqpoint{1.944613in}{3.070942in}}{\pgfqpoint{1.936799in}{3.063128in}}%
\pgfpathcurveto{\pgfqpoint{1.928986in}{3.055315in}}{\pgfqpoint{1.924596in}{3.044716in}}{\pgfqpoint{1.924596in}{3.033665in}}%
\pgfpathcurveto{\pgfqpoint{1.924596in}{3.022615in}}{\pgfqpoint{1.928986in}{3.012016in}}{\pgfqpoint{1.936799in}{3.004203in}}%
\pgfpathcurveto{\pgfqpoint{1.944613in}{2.996389in}}{\pgfqpoint{1.955212in}{2.991999in}}{\pgfqpoint{1.966262in}{2.991999in}}%
\pgfpathclose%
\pgfusepath{stroke,fill}%
\end{pgfscope}%
\begin{pgfscope}%
\pgfpathrectangle{\pgfqpoint{0.787074in}{0.548769in}}{\pgfqpoint{5.062926in}{3.102590in}}%
\pgfusepath{clip}%
\pgfsetbuttcap%
\pgfsetroundjoin%
\definecolor{currentfill}{rgb}{1.000000,0.498039,0.054902}%
\pgfsetfillcolor{currentfill}%
\pgfsetlinewidth{1.003750pt}%
\definecolor{currentstroke}{rgb}{1.000000,0.498039,0.054902}%
\pgfsetstrokecolor{currentstroke}%
\pgfsetdash{}{0pt}%
\pgfpathmoveto{\pgfqpoint{1.493709in}{3.136678in}}%
\pgfpathcurveto{\pgfqpoint{1.504759in}{3.136678in}}{\pgfqpoint{1.515358in}{3.141068in}}{\pgfqpoint{1.523172in}{3.148882in}}%
\pgfpathcurveto{\pgfqpoint{1.530986in}{3.156695in}}{\pgfqpoint{1.535376in}{3.167294in}}{\pgfqpoint{1.535376in}{3.178345in}}%
\pgfpathcurveto{\pgfqpoint{1.535376in}{3.189395in}}{\pgfqpoint{1.530986in}{3.199994in}}{\pgfqpoint{1.523172in}{3.207807in}}%
\pgfpathcurveto{\pgfqpoint{1.515358in}{3.215621in}}{\pgfqpoint{1.504759in}{3.220011in}}{\pgfqpoint{1.493709in}{3.220011in}}%
\pgfpathcurveto{\pgfqpoint{1.482659in}{3.220011in}}{\pgfqpoint{1.472060in}{3.215621in}}{\pgfqpoint{1.464246in}{3.207807in}}%
\pgfpathcurveto{\pgfqpoint{1.456433in}{3.199994in}}{\pgfqpoint{1.452043in}{3.189395in}}{\pgfqpoint{1.452043in}{3.178345in}}%
\pgfpathcurveto{\pgfqpoint{1.452043in}{3.167294in}}{\pgfqpoint{1.456433in}{3.156695in}}{\pgfqpoint{1.464246in}{3.148882in}}%
\pgfpathcurveto{\pgfqpoint{1.472060in}{3.141068in}}{\pgfqpoint{1.482659in}{3.136678in}}{\pgfqpoint{1.493709in}{3.136678in}}%
\pgfpathclose%
\pgfusepath{stroke,fill}%
\end{pgfscope}%
\begin{pgfscope}%
\pgfpathrectangle{\pgfqpoint{0.787074in}{0.548769in}}{\pgfqpoint{5.062926in}{3.102590in}}%
\pgfusepath{clip}%
\pgfsetbuttcap%
\pgfsetroundjoin%
\definecolor{currentfill}{rgb}{1.000000,0.498039,0.054902}%
\pgfsetfillcolor{currentfill}%
\pgfsetlinewidth{1.003750pt}%
\definecolor{currentstroke}{rgb}{1.000000,0.498039,0.054902}%
\pgfsetstrokecolor{currentstroke}%
\pgfsetdash{}{0pt}%
\pgfpathmoveto{\pgfqpoint{1.283040in}{2.684169in}}%
\pgfpathcurveto{\pgfqpoint{1.294090in}{2.684169in}}{\pgfqpoint{1.304689in}{2.688559in}}{\pgfqpoint{1.312502in}{2.696373in}}%
\pgfpathcurveto{\pgfqpoint{1.320316in}{2.704186in}}{\pgfqpoint{1.324706in}{2.714785in}}{\pgfqpoint{1.324706in}{2.725835in}}%
\pgfpathcurveto{\pgfqpoint{1.324706in}{2.736886in}}{\pgfqpoint{1.320316in}{2.747485in}}{\pgfqpoint{1.312502in}{2.755298in}}%
\pgfpathcurveto{\pgfqpoint{1.304689in}{2.763112in}}{\pgfqpoint{1.294090in}{2.767502in}}{\pgfqpoint{1.283040in}{2.767502in}}%
\pgfpathcurveto{\pgfqpoint{1.271989in}{2.767502in}}{\pgfqpoint{1.261390in}{2.763112in}}{\pgfqpoint{1.253577in}{2.755298in}}%
\pgfpathcurveto{\pgfqpoint{1.245763in}{2.747485in}}{\pgfqpoint{1.241373in}{2.736886in}}{\pgfqpoint{1.241373in}{2.725835in}}%
\pgfpathcurveto{\pgfqpoint{1.241373in}{2.714785in}}{\pgfqpoint{1.245763in}{2.704186in}}{\pgfqpoint{1.253577in}{2.696373in}}%
\pgfpathcurveto{\pgfqpoint{1.261390in}{2.688559in}}{\pgfqpoint{1.271989in}{2.684169in}}{\pgfqpoint{1.283040in}{2.684169in}}%
\pgfpathclose%
\pgfusepath{stroke,fill}%
\end{pgfscope}%
\begin{pgfscope}%
\pgfpathrectangle{\pgfqpoint{0.787074in}{0.548769in}}{\pgfqpoint{5.062926in}{3.102590in}}%
\pgfusepath{clip}%
\pgfsetbuttcap%
\pgfsetroundjoin%
\definecolor{currentfill}{rgb}{1.000000,0.498039,0.054902}%
\pgfsetfillcolor{currentfill}%
\pgfsetlinewidth{1.003750pt}%
\definecolor{currentstroke}{rgb}{1.000000,0.498039,0.054902}%
\pgfsetstrokecolor{currentstroke}%
\pgfsetdash{}{0pt}%
\pgfpathmoveto{\pgfqpoint{2.192947in}{2.925050in}}%
\pgfpathcurveto{\pgfqpoint{2.203997in}{2.925050in}}{\pgfqpoint{2.214596in}{2.929440in}}{\pgfqpoint{2.222410in}{2.937253in}}%
\pgfpathcurveto{\pgfqpoint{2.230223in}{2.945067in}}{\pgfqpoint{2.234614in}{2.955666in}}{\pgfqpoint{2.234614in}{2.966716in}}%
\pgfpathcurveto{\pgfqpoint{2.234614in}{2.977766in}}{\pgfqpoint{2.230223in}{2.988365in}}{\pgfqpoint{2.222410in}{2.996179in}}%
\pgfpathcurveto{\pgfqpoint{2.214596in}{3.003993in}}{\pgfqpoint{2.203997in}{3.008383in}}{\pgfqpoint{2.192947in}{3.008383in}}%
\pgfpathcurveto{\pgfqpoint{2.181897in}{3.008383in}}{\pgfqpoint{2.171298in}{3.003993in}}{\pgfqpoint{2.163484in}{2.996179in}}%
\pgfpathcurveto{\pgfqpoint{2.155671in}{2.988365in}}{\pgfqpoint{2.151280in}{2.977766in}}{\pgfqpoint{2.151280in}{2.966716in}}%
\pgfpathcurveto{\pgfqpoint{2.151280in}{2.955666in}}{\pgfqpoint{2.155671in}{2.945067in}}{\pgfqpoint{2.163484in}{2.937253in}}%
\pgfpathcurveto{\pgfqpoint{2.171298in}{2.929440in}}{\pgfqpoint{2.181897in}{2.925050in}}{\pgfqpoint{2.192947in}{2.925050in}}%
\pgfpathclose%
\pgfusepath{stroke,fill}%
\end{pgfscope}%
\begin{pgfscope}%
\pgfpathrectangle{\pgfqpoint{0.787074in}{0.548769in}}{\pgfqpoint{5.062926in}{3.102590in}}%
\pgfusepath{clip}%
\pgfsetbuttcap%
\pgfsetroundjoin%
\definecolor{currentfill}{rgb}{0.121569,0.466667,0.705882}%
\pgfsetfillcolor{currentfill}%
\pgfsetlinewidth{1.003750pt}%
\definecolor{currentstroke}{rgb}{0.121569,0.466667,0.705882}%
\pgfsetstrokecolor{currentstroke}%
\pgfsetdash{}{0pt}%
\pgfpathmoveto{\pgfqpoint{2.075850in}{2.772349in}}%
\pgfpathcurveto{\pgfqpoint{2.086900in}{2.772349in}}{\pgfqpoint{2.097500in}{2.776739in}}{\pgfqpoint{2.105313in}{2.784553in}}%
\pgfpathcurveto{\pgfqpoint{2.113127in}{2.792367in}}{\pgfqpoint{2.117517in}{2.802966in}}{\pgfqpoint{2.117517in}{2.814016in}}%
\pgfpathcurveto{\pgfqpoint{2.117517in}{2.825066in}}{\pgfqpoint{2.113127in}{2.835665in}}{\pgfqpoint{2.105313in}{2.843479in}}%
\pgfpathcurveto{\pgfqpoint{2.097500in}{2.851292in}}{\pgfqpoint{2.086900in}{2.855683in}}{\pgfqpoint{2.075850in}{2.855683in}}%
\pgfpathcurveto{\pgfqpoint{2.064800in}{2.855683in}}{\pgfqpoint{2.054201in}{2.851292in}}{\pgfqpoint{2.046388in}{2.843479in}}%
\pgfpathcurveto{\pgfqpoint{2.038574in}{2.835665in}}{\pgfqpoint{2.034184in}{2.825066in}}{\pgfqpoint{2.034184in}{2.814016in}}%
\pgfpathcurveto{\pgfqpoint{2.034184in}{2.802966in}}{\pgfqpoint{2.038574in}{2.792367in}}{\pgfqpoint{2.046388in}{2.784553in}}%
\pgfpathcurveto{\pgfqpoint{2.054201in}{2.776739in}}{\pgfqpoint{2.064800in}{2.772349in}}{\pgfqpoint{2.075850in}{2.772349in}}%
\pgfpathclose%
\pgfusepath{stroke,fill}%
\end{pgfscope}%
\begin{pgfscope}%
\pgfpathrectangle{\pgfqpoint{0.787074in}{0.548769in}}{\pgfqpoint{5.062926in}{3.102590in}}%
\pgfusepath{clip}%
\pgfsetbuttcap%
\pgfsetroundjoin%
\definecolor{currentfill}{rgb}{1.000000,0.498039,0.054902}%
\pgfsetfillcolor{currentfill}%
\pgfsetlinewidth{1.003750pt}%
\definecolor{currentstroke}{rgb}{1.000000,0.498039,0.054902}%
\pgfsetstrokecolor{currentstroke}%
\pgfsetdash{}{0pt}%
\pgfpathmoveto{\pgfqpoint{2.884503in}{3.273091in}}%
\pgfpathcurveto{\pgfqpoint{2.895553in}{3.273091in}}{\pgfqpoint{2.906152in}{3.277482in}}{\pgfqpoint{2.913965in}{3.285295in}}%
\pgfpathcurveto{\pgfqpoint{2.921779in}{3.293109in}}{\pgfqpoint{2.926169in}{3.303708in}}{\pgfqpoint{2.926169in}{3.314758in}}%
\pgfpathcurveto{\pgfqpoint{2.926169in}{3.325808in}}{\pgfqpoint{2.921779in}{3.336407in}}{\pgfqpoint{2.913965in}{3.344221in}}%
\pgfpathcurveto{\pgfqpoint{2.906152in}{3.352035in}}{\pgfqpoint{2.895553in}{3.356425in}}{\pgfqpoint{2.884503in}{3.356425in}}%
\pgfpathcurveto{\pgfqpoint{2.873452in}{3.356425in}}{\pgfqpoint{2.862853in}{3.352035in}}{\pgfqpoint{2.855040in}{3.344221in}}%
\pgfpathcurveto{\pgfqpoint{2.847226in}{3.336407in}}{\pgfqpoint{2.842836in}{3.325808in}}{\pgfqpoint{2.842836in}{3.314758in}}%
\pgfpathcurveto{\pgfqpoint{2.842836in}{3.303708in}}{\pgfqpoint{2.847226in}{3.293109in}}{\pgfqpoint{2.855040in}{3.285295in}}%
\pgfpathcurveto{\pgfqpoint{2.862853in}{3.277482in}}{\pgfqpoint{2.873452in}{3.273091in}}{\pgfqpoint{2.884503in}{3.273091in}}%
\pgfpathclose%
\pgfusepath{stroke,fill}%
\end{pgfscope}%
\begin{pgfscope}%
\pgfpathrectangle{\pgfqpoint{0.787074in}{0.548769in}}{\pgfqpoint{5.062926in}{3.102590in}}%
\pgfusepath{clip}%
\pgfsetbuttcap%
\pgfsetroundjoin%
\definecolor{currentfill}{rgb}{1.000000,0.498039,0.054902}%
\pgfsetfillcolor{currentfill}%
\pgfsetlinewidth{1.003750pt}%
\definecolor{currentstroke}{rgb}{1.000000,0.498039,0.054902}%
\pgfsetstrokecolor{currentstroke}%
\pgfsetdash{}{0pt}%
\pgfpathmoveto{\pgfqpoint{2.093384in}{2.433412in}}%
\pgfpathcurveto{\pgfqpoint{2.104435in}{2.433412in}}{\pgfqpoint{2.115034in}{2.437803in}}{\pgfqpoint{2.122847in}{2.445616in}}%
\pgfpathcurveto{\pgfqpoint{2.130661in}{2.453430in}}{\pgfqpoint{2.135051in}{2.464029in}}{\pgfqpoint{2.135051in}{2.475079in}}%
\pgfpathcurveto{\pgfqpoint{2.135051in}{2.486129in}}{\pgfqpoint{2.130661in}{2.496728in}}{\pgfqpoint{2.122847in}{2.504542in}}%
\pgfpathcurveto{\pgfqpoint{2.115034in}{2.512356in}}{\pgfqpoint{2.104435in}{2.516746in}}{\pgfqpoint{2.093384in}{2.516746in}}%
\pgfpathcurveto{\pgfqpoint{2.082334in}{2.516746in}}{\pgfqpoint{2.071735in}{2.512356in}}{\pgfqpoint{2.063922in}{2.504542in}}%
\pgfpathcurveto{\pgfqpoint{2.056108in}{2.496728in}}{\pgfqpoint{2.051718in}{2.486129in}}{\pgfqpoint{2.051718in}{2.475079in}}%
\pgfpathcurveto{\pgfqpoint{2.051718in}{2.464029in}}{\pgfqpoint{2.056108in}{2.453430in}}{\pgfqpoint{2.063922in}{2.445616in}}%
\pgfpathcurveto{\pgfqpoint{2.071735in}{2.437803in}}{\pgfqpoint{2.082334in}{2.433412in}}{\pgfqpoint{2.093384in}{2.433412in}}%
\pgfpathclose%
\pgfusepath{stroke,fill}%
\end{pgfscope}%
\begin{pgfscope}%
\pgfpathrectangle{\pgfqpoint{0.787074in}{0.548769in}}{\pgfqpoint{5.062926in}{3.102590in}}%
\pgfusepath{clip}%
\pgfsetbuttcap%
\pgfsetroundjoin%
\definecolor{currentfill}{rgb}{1.000000,0.498039,0.054902}%
\pgfsetfillcolor{currentfill}%
\pgfsetlinewidth{1.003750pt}%
\definecolor{currentstroke}{rgb}{1.000000,0.498039,0.054902}%
\pgfsetstrokecolor{currentstroke}%
\pgfsetdash{}{0pt}%
\pgfpathmoveto{\pgfqpoint{2.100459in}{2.784629in}}%
\pgfpathcurveto{\pgfqpoint{2.111509in}{2.784629in}}{\pgfqpoint{2.122108in}{2.789019in}}{\pgfqpoint{2.129922in}{2.796833in}}%
\pgfpathcurveto{\pgfqpoint{2.137735in}{2.804646in}}{\pgfqpoint{2.142126in}{2.815245in}}{\pgfqpoint{2.142126in}{2.826296in}}%
\pgfpathcurveto{\pgfqpoint{2.142126in}{2.837346in}}{\pgfqpoint{2.137735in}{2.847945in}}{\pgfqpoint{2.129922in}{2.855758in}}%
\pgfpathcurveto{\pgfqpoint{2.122108in}{2.863572in}}{\pgfqpoint{2.111509in}{2.867962in}}{\pgfqpoint{2.100459in}{2.867962in}}%
\pgfpathcurveto{\pgfqpoint{2.089409in}{2.867962in}}{\pgfqpoint{2.078810in}{2.863572in}}{\pgfqpoint{2.070996in}{2.855758in}}%
\pgfpathcurveto{\pgfqpoint{2.063182in}{2.847945in}}{\pgfqpoint{2.058792in}{2.837346in}}{\pgfqpoint{2.058792in}{2.826296in}}%
\pgfpathcurveto{\pgfqpoint{2.058792in}{2.815245in}}{\pgfqpoint{2.063182in}{2.804646in}}{\pgfqpoint{2.070996in}{2.796833in}}%
\pgfpathcurveto{\pgfqpoint{2.078810in}{2.789019in}}{\pgfqpoint{2.089409in}{2.784629in}}{\pgfqpoint{2.100459in}{2.784629in}}%
\pgfpathclose%
\pgfusepath{stroke,fill}%
\end{pgfscope}%
\begin{pgfscope}%
\pgfpathrectangle{\pgfqpoint{0.787074in}{0.548769in}}{\pgfqpoint{5.062926in}{3.102590in}}%
\pgfusepath{clip}%
\pgfsetbuttcap%
\pgfsetroundjoin%
\definecolor{currentfill}{rgb}{1.000000,0.498039,0.054902}%
\pgfsetfillcolor{currentfill}%
\pgfsetlinewidth{1.003750pt}%
\definecolor{currentstroke}{rgb}{1.000000,0.498039,0.054902}%
\pgfsetstrokecolor{currentstroke}%
\pgfsetdash{}{0pt}%
\pgfpathmoveto{\pgfqpoint{1.616448in}{2.406321in}}%
\pgfpathcurveto{\pgfqpoint{1.627498in}{2.406321in}}{\pgfqpoint{1.638097in}{2.410712in}}{\pgfqpoint{1.645911in}{2.418525in}}%
\pgfpathcurveto{\pgfqpoint{1.653724in}{2.426339in}}{\pgfqpoint{1.658115in}{2.436938in}}{\pgfqpoint{1.658115in}{2.447988in}}%
\pgfpathcurveto{\pgfqpoint{1.658115in}{2.459038in}}{\pgfqpoint{1.653724in}{2.469637in}}{\pgfqpoint{1.645911in}{2.477451in}}%
\pgfpathcurveto{\pgfqpoint{1.638097in}{2.485264in}}{\pgfqpoint{1.627498in}{2.489655in}}{\pgfqpoint{1.616448in}{2.489655in}}%
\pgfpathcurveto{\pgfqpoint{1.605398in}{2.489655in}}{\pgfqpoint{1.594799in}{2.485264in}}{\pgfqpoint{1.586985in}{2.477451in}}%
\pgfpathcurveto{\pgfqpoint{1.579172in}{2.469637in}}{\pgfqpoint{1.574781in}{2.459038in}}{\pgfqpoint{1.574781in}{2.447988in}}%
\pgfpathcurveto{\pgfqpoint{1.574781in}{2.436938in}}{\pgfqpoint{1.579172in}{2.426339in}}{\pgfqpoint{1.586985in}{2.418525in}}%
\pgfpathcurveto{\pgfqpoint{1.594799in}{2.410712in}}{\pgfqpoint{1.605398in}{2.406321in}}{\pgfqpoint{1.616448in}{2.406321in}}%
\pgfpathclose%
\pgfusepath{stroke,fill}%
\end{pgfscope}%
\begin{pgfscope}%
\pgfpathrectangle{\pgfqpoint{0.787074in}{0.548769in}}{\pgfqpoint{5.062926in}{3.102590in}}%
\pgfusepath{clip}%
\pgfsetbuttcap%
\pgfsetroundjoin%
\definecolor{currentfill}{rgb}{1.000000,0.498039,0.054902}%
\pgfsetfillcolor{currentfill}%
\pgfsetlinewidth{1.003750pt}%
\definecolor{currentstroke}{rgb}{1.000000,0.498039,0.054902}%
\pgfsetstrokecolor{currentstroke}%
\pgfsetdash{}{0pt}%
\pgfpathmoveto{\pgfqpoint{1.982147in}{2.705375in}}%
\pgfpathcurveto{\pgfqpoint{1.993197in}{2.705375in}}{\pgfqpoint{2.003796in}{2.709765in}}{\pgfqpoint{2.011610in}{2.717578in}}%
\pgfpathcurveto{\pgfqpoint{2.019423in}{2.725392in}}{\pgfqpoint{2.023814in}{2.735991in}}{\pgfqpoint{2.023814in}{2.747041in}}%
\pgfpathcurveto{\pgfqpoint{2.023814in}{2.758091in}}{\pgfqpoint{2.019423in}{2.768690in}}{\pgfqpoint{2.011610in}{2.776504in}}%
\pgfpathcurveto{\pgfqpoint{2.003796in}{2.784318in}}{\pgfqpoint{1.993197in}{2.788708in}}{\pgfqpoint{1.982147in}{2.788708in}}%
\pgfpathcurveto{\pgfqpoint{1.971097in}{2.788708in}}{\pgfqpoint{1.960498in}{2.784318in}}{\pgfqpoint{1.952684in}{2.776504in}}%
\pgfpathcurveto{\pgfqpoint{1.944871in}{2.768690in}}{\pgfqpoint{1.940480in}{2.758091in}}{\pgfqpoint{1.940480in}{2.747041in}}%
\pgfpathcurveto{\pgfqpoint{1.940480in}{2.735991in}}{\pgfqpoint{1.944871in}{2.725392in}}{\pgfqpoint{1.952684in}{2.717578in}}%
\pgfpathcurveto{\pgfqpoint{1.960498in}{2.709765in}}{\pgfqpoint{1.971097in}{2.705375in}}{\pgfqpoint{1.982147in}{2.705375in}}%
\pgfpathclose%
\pgfusepath{stroke,fill}%
\end{pgfscope}%
\begin{pgfscope}%
\pgfpathrectangle{\pgfqpoint{0.787074in}{0.548769in}}{\pgfqpoint{5.062926in}{3.102590in}}%
\pgfusepath{clip}%
\pgfsetbuttcap%
\pgfsetroundjoin%
\definecolor{currentfill}{rgb}{1.000000,0.498039,0.054902}%
\pgfsetfillcolor{currentfill}%
\pgfsetlinewidth{1.003750pt}%
\definecolor{currentstroke}{rgb}{1.000000,0.498039,0.054902}%
\pgfsetstrokecolor{currentstroke}%
\pgfsetdash{}{0pt}%
\pgfpathmoveto{\pgfqpoint{2.474144in}{2.307212in}}%
\pgfpathcurveto{\pgfqpoint{2.485194in}{2.307212in}}{\pgfqpoint{2.495793in}{2.311602in}}{\pgfqpoint{2.503607in}{2.319416in}}%
\pgfpathcurveto{\pgfqpoint{2.511420in}{2.327230in}}{\pgfqpoint{2.515810in}{2.337829in}}{\pgfqpoint{2.515810in}{2.348879in}}%
\pgfpathcurveto{\pgfqpoint{2.515810in}{2.359929in}}{\pgfqpoint{2.511420in}{2.370528in}}{\pgfqpoint{2.503607in}{2.378342in}}%
\pgfpathcurveto{\pgfqpoint{2.495793in}{2.386155in}}{\pgfqpoint{2.485194in}{2.390545in}}{\pgfqpoint{2.474144in}{2.390545in}}%
\pgfpathcurveto{\pgfqpoint{2.463094in}{2.390545in}}{\pgfqpoint{2.452495in}{2.386155in}}{\pgfqpoint{2.444681in}{2.378342in}}%
\pgfpathcurveto{\pgfqpoint{2.436867in}{2.370528in}}{\pgfqpoint{2.432477in}{2.359929in}}{\pgfqpoint{2.432477in}{2.348879in}}%
\pgfpathcurveto{\pgfqpoint{2.432477in}{2.337829in}}{\pgfqpoint{2.436867in}{2.327230in}}{\pgfqpoint{2.444681in}{2.319416in}}%
\pgfpathcurveto{\pgfqpoint{2.452495in}{2.311602in}}{\pgfqpoint{2.463094in}{2.307212in}}{\pgfqpoint{2.474144in}{2.307212in}}%
\pgfpathclose%
\pgfusepath{stroke,fill}%
\end{pgfscope}%
\begin{pgfscope}%
\pgfpathrectangle{\pgfqpoint{0.787074in}{0.548769in}}{\pgfqpoint{5.062926in}{3.102590in}}%
\pgfusepath{clip}%
\pgfsetbuttcap%
\pgfsetroundjoin%
\definecolor{currentfill}{rgb}{1.000000,0.498039,0.054902}%
\pgfsetfillcolor{currentfill}%
\pgfsetlinewidth{1.003750pt}%
\definecolor{currentstroke}{rgb}{1.000000,0.498039,0.054902}%
\pgfsetstrokecolor{currentstroke}%
\pgfsetdash{}{0pt}%
\pgfpathmoveto{\pgfqpoint{1.797041in}{3.001341in}}%
\pgfpathcurveto{\pgfqpoint{1.808091in}{3.001341in}}{\pgfqpoint{1.818690in}{3.005732in}}{\pgfqpoint{1.826503in}{3.013545in}}%
\pgfpathcurveto{\pgfqpoint{1.834317in}{3.021359in}}{\pgfqpoint{1.838707in}{3.031958in}}{\pgfqpoint{1.838707in}{3.043008in}}%
\pgfpathcurveto{\pgfqpoint{1.838707in}{3.054058in}}{\pgfqpoint{1.834317in}{3.064657in}}{\pgfqpoint{1.826503in}{3.072471in}}%
\pgfpathcurveto{\pgfqpoint{1.818690in}{3.080284in}}{\pgfqpoint{1.808091in}{3.084675in}}{\pgfqpoint{1.797041in}{3.084675in}}%
\pgfpathcurveto{\pgfqpoint{1.785991in}{3.084675in}}{\pgfqpoint{1.775392in}{3.080284in}}{\pgfqpoint{1.767578in}{3.072471in}}%
\pgfpathcurveto{\pgfqpoint{1.759764in}{3.064657in}}{\pgfqpoint{1.755374in}{3.054058in}}{\pgfqpoint{1.755374in}{3.043008in}}%
\pgfpathcurveto{\pgfqpoint{1.755374in}{3.031958in}}{\pgfqpoint{1.759764in}{3.021359in}}{\pgfqpoint{1.767578in}{3.013545in}}%
\pgfpathcurveto{\pgfqpoint{1.775392in}{3.005732in}}{\pgfqpoint{1.785991in}{3.001341in}}{\pgfqpoint{1.797041in}{3.001341in}}%
\pgfpathclose%
\pgfusepath{stroke,fill}%
\end{pgfscope}%
\begin{pgfscope}%
\pgfpathrectangle{\pgfqpoint{0.787074in}{0.548769in}}{\pgfqpoint{5.062926in}{3.102590in}}%
\pgfusepath{clip}%
\pgfsetbuttcap%
\pgfsetroundjoin%
\definecolor{currentfill}{rgb}{1.000000,0.498039,0.054902}%
\pgfsetfillcolor{currentfill}%
\pgfsetlinewidth{1.003750pt}%
\definecolor{currentstroke}{rgb}{1.000000,0.498039,0.054902}%
\pgfsetstrokecolor{currentstroke}%
\pgfsetdash{}{0pt}%
\pgfpathmoveto{\pgfqpoint{1.794957in}{2.401858in}}%
\pgfpathcurveto{\pgfqpoint{1.806008in}{2.401858in}}{\pgfqpoint{1.816607in}{2.406248in}}{\pgfqpoint{1.824420in}{2.414062in}}%
\pgfpathcurveto{\pgfqpoint{1.832234in}{2.421875in}}{\pgfqpoint{1.836624in}{2.432474in}}{\pgfqpoint{1.836624in}{2.443524in}}%
\pgfpathcurveto{\pgfqpoint{1.836624in}{2.454575in}}{\pgfqpoint{1.832234in}{2.465174in}}{\pgfqpoint{1.824420in}{2.472987in}}%
\pgfpathcurveto{\pgfqpoint{1.816607in}{2.480801in}}{\pgfqpoint{1.806008in}{2.485191in}}{\pgfqpoint{1.794957in}{2.485191in}}%
\pgfpathcurveto{\pgfqpoint{1.783907in}{2.485191in}}{\pgfqpoint{1.773308in}{2.480801in}}{\pgfqpoint{1.765495in}{2.472987in}}%
\pgfpathcurveto{\pgfqpoint{1.757681in}{2.465174in}}{\pgfqpoint{1.753291in}{2.454575in}}{\pgfqpoint{1.753291in}{2.443524in}}%
\pgfpathcurveto{\pgfqpoint{1.753291in}{2.432474in}}{\pgfqpoint{1.757681in}{2.421875in}}{\pgfqpoint{1.765495in}{2.414062in}}%
\pgfpathcurveto{\pgfqpoint{1.773308in}{2.406248in}}{\pgfqpoint{1.783907in}{2.401858in}}{\pgfqpoint{1.794957in}{2.401858in}}%
\pgfpathclose%
\pgfusepath{stroke,fill}%
\end{pgfscope}%
\begin{pgfscope}%
\pgfpathrectangle{\pgfqpoint{0.787074in}{0.548769in}}{\pgfqpoint{5.062926in}{3.102590in}}%
\pgfusepath{clip}%
\pgfsetbuttcap%
\pgfsetroundjoin%
\definecolor{currentfill}{rgb}{0.121569,0.466667,0.705882}%
\pgfsetfillcolor{currentfill}%
\pgfsetlinewidth{1.003750pt}%
\definecolor{currentstroke}{rgb}{0.121569,0.466667,0.705882}%
\pgfsetstrokecolor{currentstroke}%
\pgfsetdash{}{0pt}%
\pgfpathmoveto{\pgfqpoint{2.240341in}{2.416209in}}%
\pgfpathcurveto{\pgfqpoint{2.251391in}{2.416209in}}{\pgfqpoint{2.261990in}{2.420599in}}{\pgfqpoint{2.269804in}{2.428413in}}%
\pgfpathcurveto{\pgfqpoint{2.277618in}{2.436227in}}{\pgfqpoint{2.282008in}{2.446826in}}{\pgfqpoint{2.282008in}{2.457876in}}%
\pgfpathcurveto{\pgfqpoint{2.282008in}{2.468926in}}{\pgfqpoint{2.277618in}{2.479525in}}{\pgfqpoint{2.269804in}{2.487339in}}%
\pgfpathcurveto{\pgfqpoint{2.261990in}{2.495152in}}{\pgfqpoint{2.251391in}{2.499543in}}{\pgfqpoint{2.240341in}{2.499543in}}%
\pgfpathcurveto{\pgfqpoint{2.229291in}{2.499543in}}{\pgfqpoint{2.218692in}{2.495152in}}{\pgfqpoint{2.210878in}{2.487339in}}%
\pgfpathcurveto{\pgfqpoint{2.203065in}{2.479525in}}{\pgfqpoint{2.198674in}{2.468926in}}{\pgfqpoint{2.198674in}{2.457876in}}%
\pgfpathcurveto{\pgfqpoint{2.198674in}{2.446826in}}{\pgfqpoint{2.203065in}{2.436227in}}{\pgfqpoint{2.210878in}{2.428413in}}%
\pgfpathcurveto{\pgfqpoint{2.218692in}{2.420599in}}{\pgfqpoint{2.229291in}{2.416209in}}{\pgfqpoint{2.240341in}{2.416209in}}%
\pgfpathclose%
\pgfusepath{stroke,fill}%
\end{pgfscope}%
\begin{pgfscope}%
\pgfpathrectangle{\pgfqpoint{0.787074in}{0.548769in}}{\pgfqpoint{5.062926in}{3.102590in}}%
\pgfusepath{clip}%
\pgfsetbuttcap%
\pgfsetroundjoin%
\definecolor{currentfill}{rgb}{1.000000,0.498039,0.054902}%
\pgfsetfillcolor{currentfill}%
\pgfsetlinewidth{1.003750pt}%
\definecolor{currentstroke}{rgb}{1.000000,0.498039,0.054902}%
\pgfsetstrokecolor{currentstroke}%
\pgfsetdash{}{0pt}%
\pgfpathmoveto{\pgfqpoint{1.858887in}{3.029407in}}%
\pgfpathcurveto{\pgfqpoint{1.869938in}{3.029407in}}{\pgfqpoint{1.880537in}{3.033797in}}{\pgfqpoint{1.888350in}{3.041611in}}%
\pgfpathcurveto{\pgfqpoint{1.896164in}{3.049424in}}{\pgfqpoint{1.900554in}{3.060023in}}{\pgfqpoint{1.900554in}{3.071073in}}%
\pgfpathcurveto{\pgfqpoint{1.900554in}{3.082124in}}{\pgfqpoint{1.896164in}{3.092723in}}{\pgfqpoint{1.888350in}{3.100536in}}%
\pgfpathcurveto{\pgfqpoint{1.880537in}{3.108350in}}{\pgfqpoint{1.869938in}{3.112740in}}{\pgfqpoint{1.858887in}{3.112740in}}%
\pgfpathcurveto{\pgfqpoint{1.847837in}{3.112740in}}{\pgfqpoint{1.837238in}{3.108350in}}{\pgfqpoint{1.829425in}{3.100536in}}%
\pgfpathcurveto{\pgfqpoint{1.821611in}{3.092723in}}{\pgfqpoint{1.817221in}{3.082124in}}{\pgfqpoint{1.817221in}{3.071073in}}%
\pgfpathcurveto{\pgfqpoint{1.817221in}{3.060023in}}{\pgfqpoint{1.821611in}{3.049424in}}{\pgfqpoint{1.829425in}{3.041611in}}%
\pgfpathcurveto{\pgfqpoint{1.837238in}{3.033797in}}{\pgfqpoint{1.847837in}{3.029407in}}{\pgfqpoint{1.858887in}{3.029407in}}%
\pgfpathclose%
\pgfusepath{stroke,fill}%
\end{pgfscope}%
\begin{pgfscope}%
\pgfpathrectangle{\pgfqpoint{0.787074in}{0.548769in}}{\pgfqpoint{5.062926in}{3.102590in}}%
\pgfusepath{clip}%
\pgfsetbuttcap%
\pgfsetroundjoin%
\definecolor{currentfill}{rgb}{1.000000,0.498039,0.054902}%
\pgfsetfillcolor{currentfill}%
\pgfsetlinewidth{1.003750pt}%
\definecolor{currentstroke}{rgb}{1.000000,0.498039,0.054902}%
\pgfsetstrokecolor{currentstroke}%
\pgfsetdash{}{0pt}%
\pgfpathmoveto{\pgfqpoint{1.977590in}{2.943038in}}%
\pgfpathcurveto{\pgfqpoint{1.988640in}{2.943038in}}{\pgfqpoint{1.999239in}{2.947429in}}{\pgfqpoint{2.007053in}{2.955242in}}%
\pgfpathcurveto{\pgfqpoint{2.014866in}{2.963056in}}{\pgfqpoint{2.019257in}{2.973655in}}{\pgfqpoint{2.019257in}{2.984705in}}%
\pgfpathcurveto{\pgfqpoint{2.019257in}{2.995755in}}{\pgfqpoint{2.014866in}{3.006354in}}{\pgfqpoint{2.007053in}{3.014168in}}%
\pgfpathcurveto{\pgfqpoint{1.999239in}{3.021981in}}{\pgfqpoint{1.988640in}{3.026372in}}{\pgfqpoint{1.977590in}{3.026372in}}%
\pgfpathcurveto{\pgfqpoint{1.966540in}{3.026372in}}{\pgfqpoint{1.955941in}{3.021981in}}{\pgfqpoint{1.948127in}{3.014168in}}%
\pgfpathcurveto{\pgfqpoint{1.940314in}{3.006354in}}{\pgfqpoint{1.935923in}{2.995755in}}{\pgfqpoint{1.935923in}{2.984705in}}%
\pgfpathcurveto{\pgfqpoint{1.935923in}{2.973655in}}{\pgfqpoint{1.940314in}{2.963056in}}{\pgfqpoint{1.948127in}{2.955242in}}%
\pgfpathcurveto{\pgfqpoint{1.955941in}{2.947429in}}{\pgfqpoint{1.966540in}{2.943038in}}{\pgfqpoint{1.977590in}{2.943038in}}%
\pgfpathclose%
\pgfusepath{stroke,fill}%
\end{pgfscope}%
\begin{pgfscope}%
\pgfpathrectangle{\pgfqpoint{0.787074in}{0.548769in}}{\pgfqpoint{5.062926in}{3.102590in}}%
\pgfusepath{clip}%
\pgfsetbuttcap%
\pgfsetroundjoin%
\definecolor{currentfill}{rgb}{1.000000,0.498039,0.054902}%
\pgfsetfillcolor{currentfill}%
\pgfsetlinewidth{1.003750pt}%
\definecolor{currentstroke}{rgb}{1.000000,0.498039,0.054902}%
\pgfsetstrokecolor{currentstroke}%
\pgfsetdash{}{0pt}%
\pgfpathmoveto{\pgfqpoint{1.700039in}{2.587341in}}%
\pgfpathcurveto{\pgfqpoint{1.711089in}{2.587341in}}{\pgfqpoint{1.721688in}{2.591732in}}{\pgfqpoint{1.729502in}{2.599545in}}%
\pgfpathcurveto{\pgfqpoint{1.737315in}{2.607359in}}{\pgfqpoint{1.741706in}{2.617958in}}{\pgfqpoint{1.741706in}{2.629008in}}%
\pgfpathcurveto{\pgfqpoint{1.741706in}{2.640058in}}{\pgfqpoint{1.737315in}{2.650657in}}{\pgfqpoint{1.729502in}{2.658471in}}%
\pgfpathcurveto{\pgfqpoint{1.721688in}{2.666284in}}{\pgfqpoint{1.711089in}{2.670675in}}{\pgfqpoint{1.700039in}{2.670675in}}%
\pgfpathcurveto{\pgfqpoint{1.688989in}{2.670675in}}{\pgfqpoint{1.678390in}{2.666284in}}{\pgfqpoint{1.670576in}{2.658471in}}%
\pgfpathcurveto{\pgfqpoint{1.662762in}{2.650657in}}{\pgfqpoint{1.658372in}{2.640058in}}{\pgfqpoint{1.658372in}{2.629008in}}%
\pgfpathcurveto{\pgfqpoint{1.658372in}{2.617958in}}{\pgfqpoint{1.662762in}{2.607359in}}{\pgfqpoint{1.670576in}{2.599545in}}%
\pgfpathcurveto{\pgfqpoint{1.678390in}{2.591732in}}{\pgfqpoint{1.688989in}{2.587341in}}{\pgfqpoint{1.700039in}{2.587341in}}%
\pgfpathclose%
\pgfusepath{stroke,fill}%
\end{pgfscope}%
\begin{pgfscope}%
\pgfpathrectangle{\pgfqpoint{0.787074in}{0.548769in}}{\pgfqpoint{5.062926in}{3.102590in}}%
\pgfusepath{clip}%
\pgfsetbuttcap%
\pgfsetroundjoin%
\definecolor{currentfill}{rgb}{1.000000,0.498039,0.054902}%
\pgfsetfillcolor{currentfill}%
\pgfsetlinewidth{1.003750pt}%
\definecolor{currentstroke}{rgb}{1.000000,0.498039,0.054902}%
\pgfsetstrokecolor{currentstroke}%
\pgfsetdash{}{0pt}%
\pgfpathmoveto{\pgfqpoint{1.868479in}{2.817920in}}%
\pgfpathcurveto{\pgfqpoint{1.879529in}{2.817920in}}{\pgfqpoint{1.890128in}{2.822310in}}{\pgfqpoint{1.897942in}{2.830124in}}%
\pgfpathcurveto{\pgfqpoint{1.905756in}{2.837937in}}{\pgfqpoint{1.910146in}{2.848536in}}{\pgfqpoint{1.910146in}{2.859586in}}%
\pgfpathcurveto{\pgfqpoint{1.910146in}{2.870637in}}{\pgfqpoint{1.905756in}{2.881236in}}{\pgfqpoint{1.897942in}{2.889049in}}%
\pgfpathcurveto{\pgfqpoint{1.890128in}{2.896863in}}{\pgfqpoint{1.879529in}{2.901253in}}{\pgfqpoint{1.868479in}{2.901253in}}%
\pgfpathcurveto{\pgfqpoint{1.857429in}{2.901253in}}{\pgfqpoint{1.846830in}{2.896863in}}{\pgfqpoint{1.839016in}{2.889049in}}%
\pgfpathcurveto{\pgfqpoint{1.831203in}{2.881236in}}{\pgfqpoint{1.826812in}{2.870637in}}{\pgfqpoint{1.826812in}{2.859586in}}%
\pgfpathcurveto{\pgfqpoint{1.826812in}{2.848536in}}{\pgfqpoint{1.831203in}{2.837937in}}{\pgfqpoint{1.839016in}{2.830124in}}%
\pgfpathcurveto{\pgfqpoint{1.846830in}{2.822310in}}{\pgfqpoint{1.857429in}{2.817920in}}{\pgfqpoint{1.868479in}{2.817920in}}%
\pgfpathclose%
\pgfusepath{stroke,fill}%
\end{pgfscope}%
\begin{pgfscope}%
\pgfpathrectangle{\pgfqpoint{0.787074in}{0.548769in}}{\pgfqpoint{5.062926in}{3.102590in}}%
\pgfusepath{clip}%
\pgfsetbuttcap%
\pgfsetroundjoin%
\definecolor{currentfill}{rgb}{1.000000,0.498039,0.054902}%
\pgfsetfillcolor{currentfill}%
\pgfsetlinewidth{1.003750pt}%
\definecolor{currentstroke}{rgb}{1.000000,0.498039,0.054902}%
\pgfsetstrokecolor{currentstroke}%
\pgfsetdash{}{0pt}%
\pgfpathmoveto{\pgfqpoint{1.754724in}{2.246871in}}%
\pgfpathcurveto{\pgfqpoint{1.765775in}{2.246871in}}{\pgfqpoint{1.776374in}{2.251261in}}{\pgfqpoint{1.784187in}{2.259075in}}%
\pgfpathcurveto{\pgfqpoint{1.792001in}{2.266888in}}{\pgfqpoint{1.796391in}{2.277487in}}{\pgfqpoint{1.796391in}{2.288537in}}%
\pgfpathcurveto{\pgfqpoint{1.796391in}{2.299587in}}{\pgfqpoint{1.792001in}{2.310186in}}{\pgfqpoint{1.784187in}{2.318000in}}%
\pgfpathcurveto{\pgfqpoint{1.776374in}{2.325814in}}{\pgfqpoint{1.765775in}{2.330204in}}{\pgfqpoint{1.754724in}{2.330204in}}%
\pgfpathcurveto{\pgfqpoint{1.743674in}{2.330204in}}{\pgfqpoint{1.733075in}{2.325814in}}{\pgfqpoint{1.725262in}{2.318000in}}%
\pgfpathcurveto{\pgfqpoint{1.717448in}{2.310186in}}{\pgfqpoint{1.713058in}{2.299587in}}{\pgfqpoint{1.713058in}{2.288537in}}%
\pgfpathcurveto{\pgfqpoint{1.713058in}{2.277487in}}{\pgfqpoint{1.717448in}{2.266888in}}{\pgfqpoint{1.725262in}{2.259075in}}%
\pgfpathcurveto{\pgfqpoint{1.733075in}{2.251261in}}{\pgfqpoint{1.743674in}{2.246871in}}{\pgfqpoint{1.754724in}{2.246871in}}%
\pgfpathclose%
\pgfusepath{stroke,fill}%
\end{pgfscope}%
\begin{pgfscope}%
\pgfpathrectangle{\pgfqpoint{0.787074in}{0.548769in}}{\pgfqpoint{5.062926in}{3.102590in}}%
\pgfusepath{clip}%
\pgfsetbuttcap%
\pgfsetroundjoin%
\definecolor{currentfill}{rgb}{1.000000,0.498039,0.054902}%
\pgfsetfillcolor{currentfill}%
\pgfsetlinewidth{1.003750pt}%
\definecolor{currentstroke}{rgb}{1.000000,0.498039,0.054902}%
\pgfsetstrokecolor{currentstroke}%
\pgfsetdash{}{0pt}%
\pgfpathmoveto{\pgfqpoint{2.226409in}{1.776148in}}%
\pgfpathcurveto{\pgfqpoint{2.237459in}{1.776148in}}{\pgfqpoint{2.248058in}{1.780538in}}{\pgfqpoint{2.255872in}{1.788352in}}%
\pgfpathcurveto{\pgfqpoint{2.263686in}{1.796166in}}{\pgfqpoint{2.268076in}{1.806765in}}{\pgfqpoint{2.268076in}{1.817815in}}%
\pgfpathcurveto{\pgfqpoint{2.268076in}{1.828865in}}{\pgfqpoint{2.263686in}{1.839464in}}{\pgfqpoint{2.255872in}{1.847278in}}%
\pgfpathcurveto{\pgfqpoint{2.248058in}{1.855091in}}{\pgfqpoint{2.237459in}{1.859481in}}{\pgfqpoint{2.226409in}{1.859481in}}%
\pgfpathcurveto{\pgfqpoint{2.215359in}{1.859481in}}{\pgfqpoint{2.204760in}{1.855091in}}{\pgfqpoint{2.196947in}{1.847278in}}%
\pgfpathcurveto{\pgfqpoint{2.189133in}{1.839464in}}{\pgfqpoint{2.184743in}{1.828865in}}{\pgfqpoint{2.184743in}{1.817815in}}%
\pgfpathcurveto{\pgfqpoint{2.184743in}{1.806765in}}{\pgfqpoint{2.189133in}{1.796166in}}{\pgfqpoint{2.196947in}{1.788352in}}%
\pgfpathcurveto{\pgfqpoint{2.204760in}{1.780538in}}{\pgfqpoint{2.215359in}{1.776148in}}{\pgfqpoint{2.226409in}{1.776148in}}%
\pgfpathclose%
\pgfusepath{stroke,fill}%
\end{pgfscope}%
\begin{pgfscope}%
\pgfpathrectangle{\pgfqpoint{0.787074in}{0.548769in}}{\pgfqpoint{5.062926in}{3.102590in}}%
\pgfusepath{clip}%
\pgfsetbuttcap%
\pgfsetroundjoin%
\definecolor{currentfill}{rgb}{1.000000,0.498039,0.054902}%
\pgfsetfillcolor{currentfill}%
\pgfsetlinewidth{1.003750pt}%
\definecolor{currentstroke}{rgb}{1.000000,0.498039,0.054902}%
\pgfsetstrokecolor{currentstroke}%
\pgfsetdash{}{0pt}%
\pgfpathmoveto{\pgfqpoint{2.533517in}{2.992883in}}%
\pgfpathcurveto{\pgfqpoint{2.544567in}{2.992883in}}{\pgfqpoint{2.555166in}{2.997273in}}{\pgfqpoint{2.562979in}{3.005087in}}%
\pgfpathcurveto{\pgfqpoint{2.570793in}{3.012900in}}{\pgfqpoint{2.575183in}{3.023499in}}{\pgfqpoint{2.575183in}{3.034549in}}%
\pgfpathcurveto{\pgfqpoint{2.575183in}{3.045599in}}{\pgfqpoint{2.570793in}{3.056199in}}{\pgfqpoint{2.562979in}{3.064012in}}%
\pgfpathcurveto{\pgfqpoint{2.555166in}{3.071826in}}{\pgfqpoint{2.544567in}{3.076216in}}{\pgfqpoint{2.533517in}{3.076216in}}%
\pgfpathcurveto{\pgfqpoint{2.522467in}{3.076216in}}{\pgfqpoint{2.511867in}{3.071826in}}{\pgfqpoint{2.504054in}{3.064012in}}%
\pgfpathcurveto{\pgfqpoint{2.496240in}{3.056199in}}{\pgfqpoint{2.491850in}{3.045599in}}{\pgfqpoint{2.491850in}{3.034549in}}%
\pgfpathcurveto{\pgfqpoint{2.491850in}{3.023499in}}{\pgfqpoint{2.496240in}{3.012900in}}{\pgfqpoint{2.504054in}{3.005087in}}%
\pgfpathcurveto{\pgfqpoint{2.511867in}{2.997273in}}{\pgfqpoint{2.522467in}{2.992883in}}{\pgfqpoint{2.533517in}{2.992883in}}%
\pgfpathclose%
\pgfusepath{stroke,fill}%
\end{pgfscope}%
\begin{pgfscope}%
\pgfpathrectangle{\pgfqpoint{0.787074in}{0.548769in}}{\pgfqpoint{5.062926in}{3.102590in}}%
\pgfusepath{clip}%
\pgfsetbuttcap%
\pgfsetroundjoin%
\definecolor{currentfill}{rgb}{1.000000,0.498039,0.054902}%
\pgfsetfillcolor{currentfill}%
\pgfsetlinewidth{1.003750pt}%
\definecolor{currentstroke}{rgb}{1.000000,0.498039,0.054902}%
\pgfsetstrokecolor{currentstroke}%
\pgfsetdash{}{0pt}%
\pgfpathmoveto{\pgfqpoint{2.343029in}{1.537669in}}%
\pgfpathcurveto{\pgfqpoint{2.354079in}{1.537669in}}{\pgfqpoint{2.364678in}{1.542060in}}{\pgfqpoint{2.372491in}{1.549873in}}%
\pgfpathcurveto{\pgfqpoint{2.380305in}{1.557687in}}{\pgfqpoint{2.384695in}{1.568286in}}{\pgfqpoint{2.384695in}{1.579336in}}%
\pgfpathcurveto{\pgfqpoint{2.384695in}{1.590386in}}{\pgfqpoint{2.380305in}{1.600985in}}{\pgfqpoint{2.372491in}{1.608799in}}%
\pgfpathcurveto{\pgfqpoint{2.364678in}{1.616612in}}{\pgfqpoint{2.354079in}{1.621003in}}{\pgfqpoint{2.343029in}{1.621003in}}%
\pgfpathcurveto{\pgfqpoint{2.331978in}{1.621003in}}{\pgfqpoint{2.321379in}{1.616612in}}{\pgfqpoint{2.313566in}{1.608799in}}%
\pgfpathcurveto{\pgfqpoint{2.305752in}{1.600985in}}{\pgfqpoint{2.301362in}{1.590386in}}{\pgfqpoint{2.301362in}{1.579336in}}%
\pgfpathcurveto{\pgfqpoint{2.301362in}{1.568286in}}{\pgfqpoint{2.305752in}{1.557687in}}{\pgfqpoint{2.313566in}{1.549873in}}%
\pgfpathcurveto{\pgfqpoint{2.321379in}{1.542060in}}{\pgfqpoint{2.331978in}{1.537669in}}{\pgfqpoint{2.343029in}{1.537669in}}%
\pgfpathclose%
\pgfusepath{stroke,fill}%
\end{pgfscope}%
\begin{pgfscope}%
\pgfpathrectangle{\pgfqpoint{0.787074in}{0.548769in}}{\pgfqpoint{5.062926in}{3.102590in}}%
\pgfusepath{clip}%
\pgfsetbuttcap%
\pgfsetroundjoin%
\definecolor{currentfill}{rgb}{0.121569,0.466667,0.705882}%
\pgfsetfillcolor{currentfill}%
\pgfsetlinewidth{1.003750pt}%
\definecolor{currentstroke}{rgb}{0.121569,0.466667,0.705882}%
\pgfsetstrokecolor{currentstroke}%
\pgfsetdash{}{0pt}%
\pgfpathmoveto{\pgfqpoint{1.656941in}{0.681479in}}%
\pgfpathcurveto{\pgfqpoint{1.667992in}{0.681479in}}{\pgfqpoint{1.678591in}{0.685869in}}{\pgfqpoint{1.686404in}{0.693683in}}%
\pgfpathcurveto{\pgfqpoint{1.694218in}{0.701497in}}{\pgfqpoint{1.698608in}{0.712096in}}{\pgfqpoint{1.698608in}{0.723146in}}%
\pgfpathcurveto{\pgfqpoint{1.698608in}{0.734196in}}{\pgfqpoint{1.694218in}{0.744795in}}{\pgfqpoint{1.686404in}{0.752609in}}%
\pgfpathcurveto{\pgfqpoint{1.678591in}{0.760422in}}{\pgfqpoint{1.667992in}{0.764812in}}{\pgfqpoint{1.656941in}{0.764812in}}%
\pgfpathcurveto{\pgfqpoint{1.645891in}{0.764812in}}{\pgfqpoint{1.635292in}{0.760422in}}{\pgfqpoint{1.627479in}{0.752609in}}%
\pgfpathcurveto{\pgfqpoint{1.619665in}{0.744795in}}{\pgfqpoint{1.615275in}{0.734196in}}{\pgfqpoint{1.615275in}{0.723146in}}%
\pgfpathcurveto{\pgfqpoint{1.615275in}{0.712096in}}{\pgfqpoint{1.619665in}{0.701497in}}{\pgfqpoint{1.627479in}{0.693683in}}%
\pgfpathcurveto{\pgfqpoint{1.635292in}{0.685869in}}{\pgfqpoint{1.645891in}{0.681479in}}{\pgfqpoint{1.656941in}{0.681479in}}%
\pgfpathclose%
\pgfusepath{stroke,fill}%
\end{pgfscope}%
\begin{pgfscope}%
\pgfpathrectangle{\pgfqpoint{0.787074in}{0.548769in}}{\pgfqpoint{5.062926in}{3.102590in}}%
\pgfusepath{clip}%
\pgfsetbuttcap%
\pgfsetroundjoin%
\definecolor{currentfill}{rgb}{1.000000,0.498039,0.054902}%
\pgfsetfillcolor{currentfill}%
\pgfsetlinewidth{1.003750pt}%
\definecolor{currentstroke}{rgb}{1.000000,0.498039,0.054902}%
\pgfsetstrokecolor{currentstroke}%
\pgfsetdash{}{0pt}%
\pgfpathmoveto{\pgfqpoint{2.089044in}{2.885059in}}%
\pgfpathcurveto{\pgfqpoint{2.100094in}{2.885059in}}{\pgfqpoint{2.110694in}{2.889449in}}{\pgfqpoint{2.118507in}{2.897263in}}%
\pgfpathcurveto{\pgfqpoint{2.126321in}{2.905076in}}{\pgfqpoint{2.130711in}{2.915675in}}{\pgfqpoint{2.130711in}{2.926726in}}%
\pgfpathcurveto{\pgfqpoint{2.130711in}{2.937776in}}{\pgfqpoint{2.126321in}{2.948375in}}{\pgfqpoint{2.118507in}{2.956188in}}%
\pgfpathcurveto{\pgfqpoint{2.110694in}{2.964002in}}{\pgfqpoint{2.100094in}{2.968392in}}{\pgfqpoint{2.089044in}{2.968392in}}%
\pgfpathcurveto{\pgfqpoint{2.077994in}{2.968392in}}{\pgfqpoint{2.067395in}{2.964002in}}{\pgfqpoint{2.059582in}{2.956188in}}%
\pgfpathcurveto{\pgfqpoint{2.051768in}{2.948375in}}{\pgfqpoint{2.047378in}{2.937776in}}{\pgfqpoint{2.047378in}{2.926726in}}%
\pgfpathcurveto{\pgfqpoint{2.047378in}{2.915675in}}{\pgfqpoint{2.051768in}{2.905076in}}{\pgfqpoint{2.059582in}{2.897263in}}%
\pgfpathcurveto{\pgfqpoint{2.067395in}{2.889449in}}{\pgfqpoint{2.077994in}{2.885059in}}{\pgfqpoint{2.089044in}{2.885059in}}%
\pgfpathclose%
\pgfusepath{stroke,fill}%
\end{pgfscope}%
\begin{pgfscope}%
\pgfpathrectangle{\pgfqpoint{0.787074in}{0.548769in}}{\pgfqpoint{5.062926in}{3.102590in}}%
\pgfusepath{clip}%
\pgfsetbuttcap%
\pgfsetroundjoin%
\definecolor{currentfill}{rgb}{1.000000,0.498039,0.054902}%
\pgfsetfillcolor{currentfill}%
\pgfsetlinewidth{1.003750pt}%
\definecolor{currentstroke}{rgb}{1.000000,0.498039,0.054902}%
\pgfsetstrokecolor{currentstroke}%
\pgfsetdash{}{0pt}%
\pgfpathmoveto{\pgfqpoint{1.482034in}{2.017445in}}%
\pgfpathcurveto{\pgfqpoint{1.493084in}{2.017445in}}{\pgfqpoint{1.503683in}{2.021836in}}{\pgfqpoint{1.511497in}{2.029649in}}%
\pgfpathcurveto{\pgfqpoint{1.519311in}{2.037463in}}{\pgfqpoint{1.523701in}{2.048062in}}{\pgfqpoint{1.523701in}{2.059112in}}%
\pgfpathcurveto{\pgfqpoint{1.523701in}{2.070162in}}{\pgfqpoint{1.519311in}{2.080761in}}{\pgfqpoint{1.511497in}{2.088575in}}%
\pgfpathcurveto{\pgfqpoint{1.503683in}{2.096388in}}{\pgfqpoint{1.493084in}{2.100779in}}{\pgfqpoint{1.482034in}{2.100779in}}%
\pgfpathcurveto{\pgfqpoint{1.470984in}{2.100779in}}{\pgfqpoint{1.460385in}{2.096388in}}{\pgfqpoint{1.452572in}{2.088575in}}%
\pgfpathcurveto{\pgfqpoint{1.444758in}{2.080761in}}{\pgfqpoint{1.440368in}{2.070162in}}{\pgfqpoint{1.440368in}{2.059112in}}%
\pgfpathcurveto{\pgfqpoint{1.440368in}{2.048062in}}{\pgfqpoint{1.444758in}{2.037463in}}{\pgfqpoint{1.452572in}{2.029649in}}%
\pgfpathcurveto{\pgfqpoint{1.460385in}{2.021836in}}{\pgfqpoint{1.470984in}{2.017445in}}{\pgfqpoint{1.482034in}{2.017445in}}%
\pgfpathclose%
\pgfusepath{stroke,fill}%
\end{pgfscope}%
\begin{pgfscope}%
\pgfpathrectangle{\pgfqpoint{0.787074in}{0.548769in}}{\pgfqpoint{5.062926in}{3.102590in}}%
\pgfusepath{clip}%
\pgfsetbuttcap%
\pgfsetroundjoin%
\definecolor{currentfill}{rgb}{0.121569,0.466667,0.705882}%
\pgfsetfillcolor{currentfill}%
\pgfsetlinewidth{1.003750pt}%
\definecolor{currentstroke}{rgb}{0.121569,0.466667,0.705882}%
\pgfsetstrokecolor{currentstroke}%
\pgfsetdash{}{0pt}%
\pgfpathmoveto{\pgfqpoint{1.699431in}{2.239976in}}%
\pgfpathcurveto{\pgfqpoint{1.710481in}{2.239976in}}{\pgfqpoint{1.721080in}{2.244367in}}{\pgfqpoint{1.728894in}{2.252180in}}%
\pgfpathcurveto{\pgfqpoint{1.736708in}{2.259994in}}{\pgfqpoint{1.741098in}{2.270593in}}{\pgfqpoint{1.741098in}{2.281643in}}%
\pgfpathcurveto{\pgfqpoint{1.741098in}{2.292693in}}{\pgfqpoint{1.736708in}{2.303292in}}{\pgfqpoint{1.728894in}{2.311106in}}%
\pgfpathcurveto{\pgfqpoint{1.721080in}{2.318919in}}{\pgfqpoint{1.710481in}{2.323310in}}{\pgfqpoint{1.699431in}{2.323310in}}%
\pgfpathcurveto{\pgfqpoint{1.688381in}{2.323310in}}{\pgfqpoint{1.677782in}{2.318919in}}{\pgfqpoint{1.669968in}{2.311106in}}%
\pgfpathcurveto{\pgfqpoint{1.662155in}{2.303292in}}{\pgfqpoint{1.657765in}{2.292693in}}{\pgfqpoint{1.657765in}{2.281643in}}%
\pgfpathcurveto{\pgfqpoint{1.657765in}{2.270593in}}{\pgfqpoint{1.662155in}{2.259994in}}{\pgfqpoint{1.669968in}{2.252180in}}%
\pgfpathcurveto{\pgfqpoint{1.677782in}{2.244367in}}{\pgfqpoint{1.688381in}{2.239976in}}{\pgfqpoint{1.699431in}{2.239976in}}%
\pgfpathclose%
\pgfusepath{stroke,fill}%
\end{pgfscope}%
\begin{pgfscope}%
\pgfpathrectangle{\pgfqpoint{0.787074in}{0.548769in}}{\pgfqpoint{5.062926in}{3.102590in}}%
\pgfusepath{clip}%
\pgfsetbuttcap%
\pgfsetroundjoin%
\definecolor{currentfill}{rgb}{0.121569,0.466667,0.705882}%
\pgfsetfillcolor{currentfill}%
\pgfsetlinewidth{1.003750pt}%
\definecolor{currentstroke}{rgb}{0.121569,0.466667,0.705882}%
\pgfsetstrokecolor{currentstroke}%
\pgfsetdash{}{0pt}%
\pgfpathmoveto{\pgfqpoint{1.908452in}{2.686510in}}%
\pgfpathcurveto{\pgfqpoint{1.919502in}{2.686510in}}{\pgfqpoint{1.930101in}{2.690900in}}{\pgfqpoint{1.937914in}{2.698714in}}%
\pgfpathcurveto{\pgfqpoint{1.945728in}{2.706527in}}{\pgfqpoint{1.950118in}{2.717126in}}{\pgfqpoint{1.950118in}{2.728177in}}%
\pgfpathcurveto{\pgfqpoint{1.950118in}{2.739227in}}{\pgfqpoint{1.945728in}{2.749826in}}{\pgfqpoint{1.937914in}{2.757639in}}%
\pgfpathcurveto{\pgfqpoint{1.930101in}{2.765453in}}{\pgfqpoint{1.919502in}{2.769843in}}{\pgfqpoint{1.908452in}{2.769843in}}%
\pgfpathcurveto{\pgfqpoint{1.897402in}{2.769843in}}{\pgfqpoint{1.886803in}{2.765453in}}{\pgfqpoint{1.878989in}{2.757639in}}%
\pgfpathcurveto{\pgfqpoint{1.871175in}{2.749826in}}{\pgfqpoint{1.866785in}{2.739227in}}{\pgfqpoint{1.866785in}{2.728177in}}%
\pgfpathcurveto{\pgfqpoint{1.866785in}{2.717126in}}{\pgfqpoint{1.871175in}{2.706527in}}{\pgfqpoint{1.878989in}{2.698714in}}%
\pgfpathcurveto{\pgfqpoint{1.886803in}{2.690900in}}{\pgfqpoint{1.897402in}{2.686510in}}{\pgfqpoint{1.908452in}{2.686510in}}%
\pgfpathclose%
\pgfusepath{stroke,fill}%
\end{pgfscope}%
\begin{pgfscope}%
\pgfpathrectangle{\pgfqpoint{0.787074in}{0.548769in}}{\pgfqpoint{5.062926in}{3.102590in}}%
\pgfusepath{clip}%
\pgfsetbuttcap%
\pgfsetroundjoin%
\definecolor{currentfill}{rgb}{1.000000,0.498039,0.054902}%
\pgfsetfillcolor{currentfill}%
\pgfsetlinewidth{1.003750pt}%
\definecolor{currentstroke}{rgb}{1.000000,0.498039,0.054902}%
\pgfsetstrokecolor{currentstroke}%
\pgfsetdash{}{0pt}%
\pgfpathmoveto{\pgfqpoint{1.755419in}{2.173816in}}%
\pgfpathcurveto{\pgfqpoint{1.766469in}{2.173816in}}{\pgfqpoint{1.777068in}{2.178206in}}{\pgfqpoint{1.784882in}{2.186020in}}%
\pgfpathcurveto{\pgfqpoint{1.792695in}{2.193834in}}{\pgfqpoint{1.797086in}{2.204433in}}{\pgfqpoint{1.797086in}{2.215483in}}%
\pgfpathcurveto{\pgfqpoint{1.797086in}{2.226533in}}{\pgfqpoint{1.792695in}{2.237132in}}{\pgfqpoint{1.784882in}{2.244946in}}%
\pgfpathcurveto{\pgfqpoint{1.777068in}{2.252759in}}{\pgfqpoint{1.766469in}{2.257150in}}{\pgfqpoint{1.755419in}{2.257150in}}%
\pgfpathcurveto{\pgfqpoint{1.744369in}{2.257150in}}{\pgfqpoint{1.733770in}{2.252759in}}{\pgfqpoint{1.725956in}{2.244946in}}%
\pgfpathcurveto{\pgfqpoint{1.718142in}{2.237132in}}{\pgfqpoint{1.713752in}{2.226533in}}{\pgfqpoint{1.713752in}{2.215483in}}%
\pgfpathcurveto{\pgfqpoint{1.713752in}{2.204433in}}{\pgfqpoint{1.718142in}{2.193834in}}{\pgfqpoint{1.725956in}{2.186020in}}%
\pgfpathcurveto{\pgfqpoint{1.733770in}{2.178206in}}{\pgfqpoint{1.744369in}{2.173816in}}{\pgfqpoint{1.755419in}{2.173816in}}%
\pgfpathclose%
\pgfusepath{stroke,fill}%
\end{pgfscope}%
\begin{pgfscope}%
\pgfpathrectangle{\pgfqpoint{0.787074in}{0.548769in}}{\pgfqpoint{5.062926in}{3.102590in}}%
\pgfusepath{clip}%
\pgfsetbuttcap%
\pgfsetroundjoin%
\definecolor{currentfill}{rgb}{0.121569,0.466667,0.705882}%
\pgfsetfillcolor{currentfill}%
\pgfsetlinewidth{1.003750pt}%
\definecolor{currentstroke}{rgb}{0.121569,0.466667,0.705882}%
\pgfsetstrokecolor{currentstroke}%
\pgfsetdash{}{0pt}%
\pgfpathmoveto{\pgfqpoint{1.639798in}{1.676903in}}%
\pgfpathcurveto{\pgfqpoint{1.650848in}{1.676903in}}{\pgfqpoint{1.661447in}{1.681293in}}{\pgfqpoint{1.669261in}{1.689107in}}%
\pgfpathcurveto{\pgfqpoint{1.677074in}{1.696921in}}{\pgfqpoint{1.681465in}{1.707520in}}{\pgfqpoint{1.681465in}{1.718570in}}%
\pgfpathcurveto{\pgfqpoint{1.681465in}{1.729620in}}{\pgfqpoint{1.677074in}{1.740219in}}{\pgfqpoint{1.669261in}{1.748032in}}%
\pgfpathcurveto{\pgfqpoint{1.661447in}{1.755846in}}{\pgfqpoint{1.650848in}{1.760236in}}{\pgfqpoint{1.639798in}{1.760236in}}%
\pgfpathcurveto{\pgfqpoint{1.628748in}{1.760236in}}{\pgfqpoint{1.618149in}{1.755846in}}{\pgfqpoint{1.610335in}{1.748032in}}%
\pgfpathcurveto{\pgfqpoint{1.602522in}{1.740219in}}{\pgfqpoint{1.598131in}{1.729620in}}{\pgfqpoint{1.598131in}{1.718570in}}%
\pgfpathcurveto{\pgfqpoint{1.598131in}{1.707520in}}{\pgfqpoint{1.602522in}{1.696921in}}{\pgfqpoint{1.610335in}{1.689107in}}%
\pgfpathcurveto{\pgfqpoint{1.618149in}{1.681293in}}{\pgfqpoint{1.628748in}{1.676903in}}{\pgfqpoint{1.639798in}{1.676903in}}%
\pgfpathclose%
\pgfusepath{stroke,fill}%
\end{pgfscope}%
\begin{pgfscope}%
\pgfpathrectangle{\pgfqpoint{0.787074in}{0.548769in}}{\pgfqpoint{5.062926in}{3.102590in}}%
\pgfusepath{clip}%
\pgfsetbuttcap%
\pgfsetroundjoin%
\definecolor{currentfill}{rgb}{0.121569,0.466667,0.705882}%
\pgfsetfillcolor{currentfill}%
\pgfsetlinewidth{1.003750pt}%
\definecolor{currentstroke}{rgb}{0.121569,0.466667,0.705882}%
\pgfsetstrokecolor{currentstroke}%
\pgfsetdash{}{0pt}%
\pgfpathmoveto{\pgfqpoint{2.356222in}{1.092741in}}%
\pgfpathcurveto{\pgfqpoint{2.367273in}{1.092741in}}{\pgfqpoint{2.377872in}{1.097131in}}{\pgfqpoint{2.385685in}{1.104945in}}%
\pgfpathcurveto{\pgfqpoint{2.393499in}{1.112758in}}{\pgfqpoint{2.397889in}{1.123357in}}{\pgfqpoint{2.397889in}{1.134408in}}%
\pgfpathcurveto{\pgfqpoint{2.397889in}{1.145458in}}{\pgfqpoint{2.393499in}{1.156057in}}{\pgfqpoint{2.385685in}{1.163870in}}%
\pgfpathcurveto{\pgfqpoint{2.377872in}{1.171684in}}{\pgfqpoint{2.367273in}{1.176074in}}{\pgfqpoint{2.356222in}{1.176074in}}%
\pgfpathcurveto{\pgfqpoint{2.345172in}{1.176074in}}{\pgfqpoint{2.334573in}{1.171684in}}{\pgfqpoint{2.326760in}{1.163870in}}%
\pgfpathcurveto{\pgfqpoint{2.318946in}{1.156057in}}{\pgfqpoint{2.314556in}{1.145458in}}{\pgfqpoint{2.314556in}{1.134408in}}%
\pgfpathcurveto{\pgfqpoint{2.314556in}{1.123357in}}{\pgfqpoint{2.318946in}{1.112758in}}{\pgfqpoint{2.326760in}{1.104945in}}%
\pgfpathcurveto{\pgfqpoint{2.334573in}{1.097131in}}{\pgfqpoint{2.345172in}{1.092741in}}{\pgfqpoint{2.356222in}{1.092741in}}%
\pgfpathclose%
\pgfusepath{stroke,fill}%
\end{pgfscope}%
\begin{pgfscope}%
\pgfpathrectangle{\pgfqpoint{0.787074in}{0.548769in}}{\pgfqpoint{5.062926in}{3.102590in}}%
\pgfusepath{clip}%
\pgfsetbuttcap%
\pgfsetroundjoin%
\definecolor{currentfill}{rgb}{1.000000,0.498039,0.054902}%
\pgfsetfillcolor{currentfill}%
\pgfsetlinewidth{1.003750pt}%
\definecolor{currentstroke}{rgb}{1.000000,0.498039,0.054902}%
\pgfsetstrokecolor{currentstroke}%
\pgfsetdash{}{0pt}%
\pgfpathmoveto{\pgfqpoint{1.887923in}{2.255576in}}%
\pgfpathcurveto{\pgfqpoint{1.898973in}{2.255576in}}{\pgfqpoint{1.909572in}{2.259966in}}{\pgfqpoint{1.917386in}{2.267780in}}%
\pgfpathcurveto{\pgfqpoint{1.925199in}{2.275593in}}{\pgfqpoint{1.929590in}{2.286192in}}{\pgfqpoint{1.929590in}{2.297243in}}%
\pgfpathcurveto{\pgfqpoint{1.929590in}{2.308293in}}{\pgfqpoint{1.925199in}{2.318892in}}{\pgfqpoint{1.917386in}{2.326705in}}%
\pgfpathcurveto{\pgfqpoint{1.909572in}{2.334519in}}{\pgfqpoint{1.898973in}{2.338909in}}{\pgfqpoint{1.887923in}{2.338909in}}%
\pgfpathcurveto{\pgfqpoint{1.876873in}{2.338909in}}{\pgfqpoint{1.866274in}{2.334519in}}{\pgfqpoint{1.858460in}{2.326705in}}%
\pgfpathcurveto{\pgfqpoint{1.850647in}{2.318892in}}{\pgfqpoint{1.846256in}{2.308293in}}{\pgfqpoint{1.846256in}{2.297243in}}%
\pgfpathcurveto{\pgfqpoint{1.846256in}{2.286192in}}{\pgfqpoint{1.850647in}{2.275593in}}{\pgfqpoint{1.858460in}{2.267780in}}%
\pgfpathcurveto{\pgfqpoint{1.866274in}{2.259966in}}{\pgfqpoint{1.876873in}{2.255576in}}{\pgfqpoint{1.887923in}{2.255576in}}%
\pgfpathclose%
\pgfusepath{stroke,fill}%
\end{pgfscope}%
\begin{pgfscope}%
\pgfsetbuttcap%
\pgfsetroundjoin%
\definecolor{currentfill}{rgb}{0.000000,0.000000,0.000000}%
\pgfsetfillcolor{currentfill}%
\pgfsetlinewidth{0.803000pt}%
\definecolor{currentstroke}{rgb}{0.000000,0.000000,0.000000}%
\pgfsetstrokecolor{currentstroke}%
\pgfsetdash{}{0pt}%
\pgfsys@defobject{currentmarker}{\pgfqpoint{0.000000in}{-0.048611in}}{\pgfqpoint{0.000000in}{0.000000in}}{%
\pgfpathmoveto{\pgfqpoint{0.000000in}{0.000000in}}%
\pgfpathlineto{\pgfqpoint{0.000000in}{-0.048611in}}%
\pgfusepath{stroke,fill}%
}%
\begin{pgfscope}%
\pgfsys@transformshift{0.987520in}{0.548769in}%
\pgfsys@useobject{currentmarker}{}%
\end{pgfscope}%
\end{pgfscope}%
\begin{pgfscope}%
\definecolor{textcolor}{rgb}{0.000000,0.000000,0.000000}%
\pgfsetstrokecolor{textcolor}%
\pgfsetfillcolor{textcolor}%
\pgftext[x=0.987520in,y=0.451547in,,top]{\color{textcolor}\sffamily\fontsize{10.000000}{12.000000}\selectfont \(\displaystyle {0}\)}%
\end{pgfscope}%
\begin{pgfscope}%
\pgfsetbuttcap%
\pgfsetroundjoin%
\definecolor{currentfill}{rgb}{0.000000,0.000000,0.000000}%
\pgfsetfillcolor{currentfill}%
\pgfsetlinewidth{0.803000pt}%
\definecolor{currentstroke}{rgb}{0.000000,0.000000,0.000000}%
\pgfsetstrokecolor{currentstroke}%
\pgfsetdash{}{0pt}%
\pgfsys@defobject{currentmarker}{\pgfqpoint{0.000000in}{-0.048611in}}{\pgfqpoint{0.000000in}{0.000000in}}{%
\pgfpathmoveto{\pgfqpoint{0.000000in}{0.000000in}}%
\pgfpathlineto{\pgfqpoint{0.000000in}{-0.048611in}}%
\pgfusepath{stroke,fill}%
}%
\begin{pgfscope}%
\pgfsys@transformshift{1.855546in}{0.548769in}%
\pgfsys@useobject{currentmarker}{}%
\end{pgfscope}%
\end{pgfscope}%
\begin{pgfscope}%
\definecolor{textcolor}{rgb}{0.000000,0.000000,0.000000}%
\pgfsetstrokecolor{textcolor}%
\pgfsetfillcolor{textcolor}%
\pgftext[x=1.855546in,y=0.451547in,,top]{\color{textcolor}\sffamily\fontsize{10.000000}{12.000000}\selectfont \(\displaystyle {20000}\)}%
\end{pgfscope}%
\begin{pgfscope}%
\pgfsetbuttcap%
\pgfsetroundjoin%
\definecolor{currentfill}{rgb}{0.000000,0.000000,0.000000}%
\pgfsetfillcolor{currentfill}%
\pgfsetlinewidth{0.803000pt}%
\definecolor{currentstroke}{rgb}{0.000000,0.000000,0.000000}%
\pgfsetstrokecolor{currentstroke}%
\pgfsetdash{}{0pt}%
\pgfsys@defobject{currentmarker}{\pgfqpoint{0.000000in}{-0.048611in}}{\pgfqpoint{0.000000in}{0.000000in}}{%
\pgfpathmoveto{\pgfqpoint{0.000000in}{0.000000in}}%
\pgfpathlineto{\pgfqpoint{0.000000in}{-0.048611in}}%
\pgfusepath{stroke,fill}%
}%
\begin{pgfscope}%
\pgfsys@transformshift{2.723571in}{0.548769in}%
\pgfsys@useobject{currentmarker}{}%
\end{pgfscope}%
\end{pgfscope}%
\begin{pgfscope}%
\definecolor{textcolor}{rgb}{0.000000,0.000000,0.000000}%
\pgfsetstrokecolor{textcolor}%
\pgfsetfillcolor{textcolor}%
\pgftext[x=2.723571in,y=0.451547in,,top]{\color{textcolor}\sffamily\fontsize{10.000000}{12.000000}\selectfont \(\displaystyle {40000}\)}%
\end{pgfscope}%
\begin{pgfscope}%
\pgfsetbuttcap%
\pgfsetroundjoin%
\definecolor{currentfill}{rgb}{0.000000,0.000000,0.000000}%
\pgfsetfillcolor{currentfill}%
\pgfsetlinewidth{0.803000pt}%
\definecolor{currentstroke}{rgb}{0.000000,0.000000,0.000000}%
\pgfsetstrokecolor{currentstroke}%
\pgfsetdash{}{0pt}%
\pgfsys@defobject{currentmarker}{\pgfqpoint{0.000000in}{-0.048611in}}{\pgfqpoint{0.000000in}{0.000000in}}{%
\pgfpathmoveto{\pgfqpoint{0.000000in}{0.000000in}}%
\pgfpathlineto{\pgfqpoint{0.000000in}{-0.048611in}}%
\pgfusepath{stroke,fill}%
}%
\begin{pgfscope}%
\pgfsys@transformshift{3.591596in}{0.548769in}%
\pgfsys@useobject{currentmarker}{}%
\end{pgfscope}%
\end{pgfscope}%
\begin{pgfscope}%
\definecolor{textcolor}{rgb}{0.000000,0.000000,0.000000}%
\pgfsetstrokecolor{textcolor}%
\pgfsetfillcolor{textcolor}%
\pgftext[x=3.591596in,y=0.451547in,,top]{\color{textcolor}\sffamily\fontsize{10.000000}{12.000000}\selectfont \(\displaystyle {60000}\)}%
\end{pgfscope}%
\begin{pgfscope}%
\pgfsetbuttcap%
\pgfsetroundjoin%
\definecolor{currentfill}{rgb}{0.000000,0.000000,0.000000}%
\pgfsetfillcolor{currentfill}%
\pgfsetlinewidth{0.803000pt}%
\definecolor{currentstroke}{rgb}{0.000000,0.000000,0.000000}%
\pgfsetstrokecolor{currentstroke}%
\pgfsetdash{}{0pt}%
\pgfsys@defobject{currentmarker}{\pgfqpoint{0.000000in}{-0.048611in}}{\pgfqpoint{0.000000in}{0.000000in}}{%
\pgfpathmoveto{\pgfqpoint{0.000000in}{0.000000in}}%
\pgfpathlineto{\pgfqpoint{0.000000in}{-0.048611in}}%
\pgfusepath{stroke,fill}%
}%
\begin{pgfscope}%
\pgfsys@transformshift{4.459621in}{0.548769in}%
\pgfsys@useobject{currentmarker}{}%
\end{pgfscope}%
\end{pgfscope}%
\begin{pgfscope}%
\definecolor{textcolor}{rgb}{0.000000,0.000000,0.000000}%
\pgfsetstrokecolor{textcolor}%
\pgfsetfillcolor{textcolor}%
\pgftext[x=4.459621in,y=0.451547in,,top]{\color{textcolor}\sffamily\fontsize{10.000000}{12.000000}\selectfont \(\displaystyle {80000}\)}%
\end{pgfscope}%
\begin{pgfscope}%
\pgfsetbuttcap%
\pgfsetroundjoin%
\definecolor{currentfill}{rgb}{0.000000,0.000000,0.000000}%
\pgfsetfillcolor{currentfill}%
\pgfsetlinewidth{0.803000pt}%
\definecolor{currentstroke}{rgb}{0.000000,0.000000,0.000000}%
\pgfsetstrokecolor{currentstroke}%
\pgfsetdash{}{0pt}%
\pgfsys@defobject{currentmarker}{\pgfqpoint{0.000000in}{-0.048611in}}{\pgfqpoint{0.000000in}{0.000000in}}{%
\pgfpathmoveto{\pgfqpoint{0.000000in}{0.000000in}}%
\pgfpathlineto{\pgfqpoint{0.000000in}{-0.048611in}}%
\pgfusepath{stroke,fill}%
}%
\begin{pgfscope}%
\pgfsys@transformshift{5.327646in}{0.548769in}%
\pgfsys@useobject{currentmarker}{}%
\end{pgfscope}%
\end{pgfscope}%
\begin{pgfscope}%
\definecolor{textcolor}{rgb}{0.000000,0.000000,0.000000}%
\pgfsetstrokecolor{textcolor}%
\pgfsetfillcolor{textcolor}%
\pgftext[x=5.327646in,y=0.451547in,,top]{\color{textcolor}\sffamily\fontsize{10.000000}{12.000000}\selectfont \(\displaystyle {100000}\)}%
\end{pgfscope}%
\begin{pgfscope}%
\definecolor{textcolor}{rgb}{0.000000,0.000000,0.000000}%
\pgfsetstrokecolor{textcolor}%
\pgfsetfillcolor{textcolor}%
\pgftext[x=3.318537in,y=0.272658in,,top]{\color{textcolor}\sffamily\fontsize{10.000000}{12.000000}\selectfont Classes}%
\end{pgfscope}%
\begin{pgfscope}%
\pgfsetbuttcap%
\pgfsetroundjoin%
\definecolor{currentfill}{rgb}{0.000000,0.000000,0.000000}%
\pgfsetfillcolor{currentfill}%
\pgfsetlinewidth{0.803000pt}%
\definecolor{currentstroke}{rgb}{0.000000,0.000000,0.000000}%
\pgfsetstrokecolor{currentstroke}%
\pgfsetdash{}{0pt}%
\pgfsys@defobject{currentmarker}{\pgfqpoint{-0.048611in}{0.000000in}}{\pgfqpoint{0.000000in}{0.000000in}}{%
\pgfpathmoveto{\pgfqpoint{0.000000in}{0.000000in}}%
\pgfpathlineto{\pgfqpoint{-0.048611in}{0.000000in}}%
\pgfusepath{stroke,fill}%
}%
\begin{pgfscope}%
\pgfsys@transformshift{0.787074in}{0.689795in}%
\pgfsys@useobject{currentmarker}{}%
\end{pgfscope}%
\end{pgfscope}%
\begin{pgfscope}%
\definecolor{textcolor}{rgb}{0.000000,0.000000,0.000000}%
\pgfsetstrokecolor{textcolor}%
\pgfsetfillcolor{textcolor}%
\pgftext[x=0.620407in, y=0.641601in, left, base]{\color{textcolor}\sffamily\fontsize{10.000000}{12.000000}\selectfont \(\displaystyle {0}\)}%
\end{pgfscope}%
\begin{pgfscope}%
\pgfsetbuttcap%
\pgfsetroundjoin%
\definecolor{currentfill}{rgb}{0.000000,0.000000,0.000000}%
\pgfsetfillcolor{currentfill}%
\pgfsetlinewidth{0.803000pt}%
\definecolor{currentstroke}{rgb}{0.000000,0.000000,0.000000}%
\pgfsetstrokecolor{currentstroke}%
\pgfsetdash{}{0pt}%
\pgfsys@defobject{currentmarker}{\pgfqpoint{-0.048611in}{0.000000in}}{\pgfqpoint{0.000000in}{0.000000in}}{%
\pgfpathmoveto{\pgfqpoint{0.000000in}{0.000000in}}%
\pgfpathlineto{\pgfqpoint{-0.048611in}{0.000000in}}%
\pgfusepath{stroke,fill}%
}%
\begin{pgfscope}%
\pgfsys@transformshift{0.787074in}{1.062854in}%
\pgfsys@useobject{currentmarker}{}%
\end{pgfscope}%
\end{pgfscope}%
\begin{pgfscope}%
\definecolor{textcolor}{rgb}{0.000000,0.000000,0.000000}%
\pgfsetstrokecolor{textcolor}%
\pgfsetfillcolor{textcolor}%
\pgftext[x=0.412073in, y=1.014659in, left, base]{\color{textcolor}\sffamily\fontsize{10.000000}{12.000000}\selectfont \(\displaystyle {2500}\)}%
\end{pgfscope}%
\begin{pgfscope}%
\pgfsetbuttcap%
\pgfsetroundjoin%
\definecolor{currentfill}{rgb}{0.000000,0.000000,0.000000}%
\pgfsetfillcolor{currentfill}%
\pgfsetlinewidth{0.803000pt}%
\definecolor{currentstroke}{rgb}{0.000000,0.000000,0.000000}%
\pgfsetstrokecolor{currentstroke}%
\pgfsetdash{}{0pt}%
\pgfsys@defobject{currentmarker}{\pgfqpoint{-0.048611in}{0.000000in}}{\pgfqpoint{0.000000in}{0.000000in}}{%
\pgfpathmoveto{\pgfqpoint{0.000000in}{0.000000in}}%
\pgfpathlineto{\pgfqpoint{-0.048611in}{0.000000in}}%
\pgfusepath{stroke,fill}%
}%
\begin{pgfscope}%
\pgfsys@transformshift{0.787074in}{1.435912in}%
\pgfsys@useobject{currentmarker}{}%
\end{pgfscope}%
\end{pgfscope}%
\begin{pgfscope}%
\definecolor{textcolor}{rgb}{0.000000,0.000000,0.000000}%
\pgfsetstrokecolor{textcolor}%
\pgfsetfillcolor{textcolor}%
\pgftext[x=0.412073in, y=1.387718in, left, base]{\color{textcolor}\sffamily\fontsize{10.000000}{12.000000}\selectfont \(\displaystyle {5000}\)}%
\end{pgfscope}%
\begin{pgfscope}%
\pgfsetbuttcap%
\pgfsetroundjoin%
\definecolor{currentfill}{rgb}{0.000000,0.000000,0.000000}%
\pgfsetfillcolor{currentfill}%
\pgfsetlinewidth{0.803000pt}%
\definecolor{currentstroke}{rgb}{0.000000,0.000000,0.000000}%
\pgfsetstrokecolor{currentstroke}%
\pgfsetdash{}{0pt}%
\pgfsys@defobject{currentmarker}{\pgfqpoint{-0.048611in}{0.000000in}}{\pgfqpoint{0.000000in}{0.000000in}}{%
\pgfpathmoveto{\pgfqpoint{0.000000in}{0.000000in}}%
\pgfpathlineto{\pgfqpoint{-0.048611in}{0.000000in}}%
\pgfusepath{stroke,fill}%
}%
\begin{pgfscope}%
\pgfsys@transformshift{0.787074in}{1.808971in}%
\pgfsys@useobject{currentmarker}{}%
\end{pgfscope}%
\end{pgfscope}%
\begin{pgfscope}%
\definecolor{textcolor}{rgb}{0.000000,0.000000,0.000000}%
\pgfsetstrokecolor{textcolor}%
\pgfsetfillcolor{textcolor}%
\pgftext[x=0.412073in, y=1.760776in, left, base]{\color{textcolor}\sffamily\fontsize{10.000000}{12.000000}\selectfont \(\displaystyle {7500}\)}%
\end{pgfscope}%
\begin{pgfscope}%
\pgfsetbuttcap%
\pgfsetroundjoin%
\definecolor{currentfill}{rgb}{0.000000,0.000000,0.000000}%
\pgfsetfillcolor{currentfill}%
\pgfsetlinewidth{0.803000pt}%
\definecolor{currentstroke}{rgb}{0.000000,0.000000,0.000000}%
\pgfsetstrokecolor{currentstroke}%
\pgfsetdash{}{0pt}%
\pgfsys@defobject{currentmarker}{\pgfqpoint{-0.048611in}{0.000000in}}{\pgfqpoint{0.000000in}{0.000000in}}{%
\pgfpathmoveto{\pgfqpoint{0.000000in}{0.000000in}}%
\pgfpathlineto{\pgfqpoint{-0.048611in}{0.000000in}}%
\pgfusepath{stroke,fill}%
}%
\begin{pgfscope}%
\pgfsys@transformshift{0.787074in}{2.182029in}%
\pgfsys@useobject{currentmarker}{}%
\end{pgfscope}%
\end{pgfscope}%
\begin{pgfscope}%
\definecolor{textcolor}{rgb}{0.000000,0.000000,0.000000}%
\pgfsetstrokecolor{textcolor}%
\pgfsetfillcolor{textcolor}%
\pgftext[x=0.342628in, y=2.133835in, left, base]{\color{textcolor}\sffamily\fontsize{10.000000}{12.000000}\selectfont \(\displaystyle {10000}\)}%
\end{pgfscope}%
\begin{pgfscope}%
\pgfsetbuttcap%
\pgfsetroundjoin%
\definecolor{currentfill}{rgb}{0.000000,0.000000,0.000000}%
\pgfsetfillcolor{currentfill}%
\pgfsetlinewidth{0.803000pt}%
\definecolor{currentstroke}{rgb}{0.000000,0.000000,0.000000}%
\pgfsetstrokecolor{currentstroke}%
\pgfsetdash{}{0pt}%
\pgfsys@defobject{currentmarker}{\pgfqpoint{-0.048611in}{0.000000in}}{\pgfqpoint{0.000000in}{0.000000in}}{%
\pgfpathmoveto{\pgfqpoint{0.000000in}{0.000000in}}%
\pgfpathlineto{\pgfqpoint{-0.048611in}{0.000000in}}%
\pgfusepath{stroke,fill}%
}%
\begin{pgfscope}%
\pgfsys@transformshift{0.787074in}{2.555088in}%
\pgfsys@useobject{currentmarker}{}%
\end{pgfscope}%
\end{pgfscope}%
\begin{pgfscope}%
\definecolor{textcolor}{rgb}{0.000000,0.000000,0.000000}%
\pgfsetstrokecolor{textcolor}%
\pgfsetfillcolor{textcolor}%
\pgftext[x=0.342628in, y=2.506893in, left, base]{\color{textcolor}\sffamily\fontsize{10.000000}{12.000000}\selectfont \(\displaystyle {12500}\)}%
\end{pgfscope}%
\begin{pgfscope}%
\pgfsetbuttcap%
\pgfsetroundjoin%
\definecolor{currentfill}{rgb}{0.000000,0.000000,0.000000}%
\pgfsetfillcolor{currentfill}%
\pgfsetlinewidth{0.803000pt}%
\definecolor{currentstroke}{rgb}{0.000000,0.000000,0.000000}%
\pgfsetstrokecolor{currentstroke}%
\pgfsetdash{}{0pt}%
\pgfsys@defobject{currentmarker}{\pgfqpoint{-0.048611in}{0.000000in}}{\pgfqpoint{0.000000in}{0.000000in}}{%
\pgfpathmoveto{\pgfqpoint{0.000000in}{0.000000in}}%
\pgfpathlineto{\pgfqpoint{-0.048611in}{0.000000in}}%
\pgfusepath{stroke,fill}%
}%
\begin{pgfscope}%
\pgfsys@transformshift{0.787074in}{2.928146in}%
\pgfsys@useobject{currentmarker}{}%
\end{pgfscope}%
\end{pgfscope}%
\begin{pgfscope}%
\definecolor{textcolor}{rgb}{0.000000,0.000000,0.000000}%
\pgfsetstrokecolor{textcolor}%
\pgfsetfillcolor{textcolor}%
\pgftext[x=0.342628in, y=2.879952in, left, base]{\color{textcolor}\sffamily\fontsize{10.000000}{12.000000}\selectfont \(\displaystyle {15000}\)}%
\end{pgfscope}%
\begin{pgfscope}%
\pgfsetbuttcap%
\pgfsetroundjoin%
\definecolor{currentfill}{rgb}{0.000000,0.000000,0.000000}%
\pgfsetfillcolor{currentfill}%
\pgfsetlinewidth{0.803000pt}%
\definecolor{currentstroke}{rgb}{0.000000,0.000000,0.000000}%
\pgfsetstrokecolor{currentstroke}%
\pgfsetdash{}{0pt}%
\pgfsys@defobject{currentmarker}{\pgfqpoint{-0.048611in}{0.000000in}}{\pgfqpoint{0.000000in}{0.000000in}}{%
\pgfpathmoveto{\pgfqpoint{0.000000in}{0.000000in}}%
\pgfpathlineto{\pgfqpoint{-0.048611in}{0.000000in}}%
\pgfusepath{stroke,fill}%
}%
\begin{pgfscope}%
\pgfsys@transformshift{0.787074in}{3.301204in}%
\pgfsys@useobject{currentmarker}{}%
\end{pgfscope}%
\end{pgfscope}%
\begin{pgfscope}%
\definecolor{textcolor}{rgb}{0.000000,0.000000,0.000000}%
\pgfsetstrokecolor{textcolor}%
\pgfsetfillcolor{textcolor}%
\pgftext[x=0.342628in, y=3.253010in, left, base]{\color{textcolor}\sffamily\fontsize{10.000000}{12.000000}\selectfont \(\displaystyle {17500}\)}%
\end{pgfscope}%
\begin{pgfscope}%
\definecolor{textcolor}{rgb}{0.000000,0.000000,0.000000}%
\pgfsetstrokecolor{textcolor}%
\pgfsetfillcolor{textcolor}%
\pgftext[x=0.287073in,y=2.100064in,,bottom,rotate=90.000000]{\color{textcolor}\sffamily\fontsize{10.000000}{12.000000}\selectfont Maximum Memory Usage (MB)}%
\end{pgfscope}%
\begin{pgfscope}%
\pgfsetrectcap%
\pgfsetmiterjoin%
\pgfsetlinewidth{0.803000pt}%
\definecolor{currentstroke}{rgb}{0.000000,0.000000,0.000000}%
\pgfsetstrokecolor{currentstroke}%
\pgfsetdash{}{0pt}%
\pgfpathmoveto{\pgfqpoint{0.787074in}{0.548769in}}%
\pgfpathlineto{\pgfqpoint{0.787074in}{3.651359in}}%
\pgfusepath{stroke}%
\end{pgfscope}%
\begin{pgfscope}%
\pgfsetrectcap%
\pgfsetmiterjoin%
\pgfsetlinewidth{0.803000pt}%
\definecolor{currentstroke}{rgb}{0.000000,0.000000,0.000000}%
\pgfsetstrokecolor{currentstroke}%
\pgfsetdash{}{0pt}%
\pgfpathmoveto{\pgfqpoint{5.850000in}{0.548769in}}%
\pgfpathlineto{\pgfqpoint{5.850000in}{3.651359in}}%
\pgfusepath{stroke}%
\end{pgfscope}%
\begin{pgfscope}%
\pgfsetrectcap%
\pgfsetmiterjoin%
\pgfsetlinewidth{0.803000pt}%
\definecolor{currentstroke}{rgb}{0.000000,0.000000,0.000000}%
\pgfsetstrokecolor{currentstroke}%
\pgfsetdash{}{0pt}%
\pgfpathmoveto{\pgfqpoint{0.787074in}{0.548769in}}%
\pgfpathlineto{\pgfqpoint{5.850000in}{0.548769in}}%
\pgfusepath{stroke}%
\end{pgfscope}%
\begin{pgfscope}%
\pgfsetrectcap%
\pgfsetmiterjoin%
\pgfsetlinewidth{0.803000pt}%
\definecolor{currentstroke}{rgb}{0.000000,0.000000,0.000000}%
\pgfsetstrokecolor{currentstroke}%
\pgfsetdash{}{0pt}%
\pgfpathmoveto{\pgfqpoint{0.787074in}{3.651359in}}%
\pgfpathlineto{\pgfqpoint{5.850000in}{3.651359in}}%
\pgfusepath{stroke}%
\end{pgfscope}%
\begin{pgfscope}%
\definecolor{textcolor}{rgb}{0.000000,0.000000,0.000000}%
\pgfsetstrokecolor{textcolor}%
\pgfsetfillcolor{textcolor}%
\pgftext[x=3.318537in,y=3.734692in,,base]{\color{textcolor}\sffamily\fontsize{12.000000}{14.400000}\selectfont Backward}%
\end{pgfscope}%
\begin{pgfscope}%
\pgfsetbuttcap%
\pgfsetmiterjoin%
\definecolor{currentfill}{rgb}{1.000000,1.000000,1.000000}%
\pgfsetfillcolor{currentfill}%
\pgfsetfillopacity{0.800000}%
\pgfsetlinewidth{1.003750pt}%
\definecolor{currentstroke}{rgb}{0.800000,0.800000,0.800000}%
\pgfsetstrokecolor{currentstroke}%
\pgfsetstrokeopacity{0.800000}%
\pgfsetdash{}{0pt}%
\pgfpathmoveto{\pgfqpoint{4.300417in}{2.957886in}}%
\pgfpathlineto{\pgfqpoint{5.752778in}{2.957886in}}%
\pgfpathquadraticcurveto{\pgfqpoint{5.780556in}{2.957886in}}{\pgfqpoint{5.780556in}{2.985664in}}%
\pgfpathlineto{\pgfqpoint{5.780556in}{3.554136in}}%
\pgfpathquadraticcurveto{\pgfqpoint{5.780556in}{3.581914in}}{\pgfqpoint{5.752778in}{3.581914in}}%
\pgfpathlineto{\pgfqpoint{4.300417in}{3.581914in}}%
\pgfpathquadraticcurveto{\pgfqpoint{4.272639in}{3.581914in}}{\pgfqpoint{4.272639in}{3.554136in}}%
\pgfpathlineto{\pgfqpoint{4.272639in}{2.985664in}}%
\pgfpathquadraticcurveto{\pgfqpoint{4.272639in}{2.957886in}}{\pgfqpoint{4.300417in}{2.957886in}}%
\pgfpathclose%
\pgfusepath{stroke,fill}%
\end{pgfscope}%
\begin{pgfscope}%
\pgfsetbuttcap%
\pgfsetroundjoin%
\definecolor{currentfill}{rgb}{0.121569,0.466667,0.705882}%
\pgfsetfillcolor{currentfill}%
\pgfsetlinewidth{1.003750pt}%
\definecolor{currentstroke}{rgb}{0.121569,0.466667,0.705882}%
\pgfsetstrokecolor{currentstroke}%
\pgfsetdash{}{0pt}%
\pgfsys@defobject{currentmarker}{\pgfqpoint{-0.034722in}{-0.034722in}}{\pgfqpoint{0.034722in}{0.034722in}}{%
\pgfpathmoveto{\pgfqpoint{0.000000in}{-0.034722in}}%
\pgfpathcurveto{\pgfqpoint{0.009208in}{-0.034722in}}{\pgfqpoint{0.018041in}{-0.031064in}}{\pgfqpoint{0.024552in}{-0.024552in}}%
\pgfpathcurveto{\pgfqpoint{0.031064in}{-0.018041in}}{\pgfqpoint{0.034722in}{-0.009208in}}{\pgfqpoint{0.034722in}{0.000000in}}%
\pgfpathcurveto{\pgfqpoint{0.034722in}{0.009208in}}{\pgfqpoint{0.031064in}{0.018041in}}{\pgfqpoint{0.024552in}{0.024552in}}%
\pgfpathcurveto{\pgfqpoint{0.018041in}{0.031064in}}{\pgfqpoint{0.009208in}{0.034722in}}{\pgfqpoint{0.000000in}{0.034722in}}%
\pgfpathcurveto{\pgfqpoint{-0.009208in}{0.034722in}}{\pgfqpoint{-0.018041in}{0.031064in}}{\pgfqpoint{-0.024552in}{0.024552in}}%
\pgfpathcurveto{\pgfqpoint{-0.031064in}{0.018041in}}{\pgfqpoint{-0.034722in}{0.009208in}}{\pgfqpoint{-0.034722in}{0.000000in}}%
\pgfpathcurveto{\pgfqpoint{-0.034722in}{-0.009208in}}{\pgfqpoint{-0.031064in}{-0.018041in}}{\pgfqpoint{-0.024552in}{-0.024552in}}%
\pgfpathcurveto{\pgfqpoint{-0.018041in}{-0.031064in}}{\pgfqpoint{-0.009208in}{-0.034722in}}{\pgfqpoint{0.000000in}{-0.034722in}}%
\pgfpathclose%
\pgfusepath{stroke,fill}%
}%
\begin{pgfscope}%
\pgfsys@transformshift{4.467083in}{3.477748in}%
\pgfsys@useobject{currentmarker}{}%
\end{pgfscope}%
\end{pgfscope}%
\begin{pgfscope}%
\definecolor{textcolor}{rgb}{0.000000,0.000000,0.000000}%
\pgfsetstrokecolor{textcolor}%
\pgfsetfillcolor{textcolor}%
\pgftext[x=4.717083in,y=3.429136in,left,base]{\color{textcolor}\sffamily\fontsize{10.000000}{12.000000}\selectfont No Timeout}%
\end{pgfscope}%
\begin{pgfscope}%
\pgfsetbuttcap%
\pgfsetroundjoin%
\definecolor{currentfill}{rgb}{1.000000,0.498039,0.054902}%
\pgfsetfillcolor{currentfill}%
\pgfsetlinewidth{1.003750pt}%
\definecolor{currentstroke}{rgb}{1.000000,0.498039,0.054902}%
\pgfsetstrokecolor{currentstroke}%
\pgfsetdash{}{0pt}%
\pgfsys@defobject{currentmarker}{\pgfqpoint{-0.034722in}{-0.034722in}}{\pgfqpoint{0.034722in}{0.034722in}}{%
\pgfpathmoveto{\pgfqpoint{0.000000in}{-0.034722in}}%
\pgfpathcurveto{\pgfqpoint{0.009208in}{-0.034722in}}{\pgfqpoint{0.018041in}{-0.031064in}}{\pgfqpoint{0.024552in}{-0.024552in}}%
\pgfpathcurveto{\pgfqpoint{0.031064in}{-0.018041in}}{\pgfqpoint{0.034722in}{-0.009208in}}{\pgfqpoint{0.034722in}{0.000000in}}%
\pgfpathcurveto{\pgfqpoint{0.034722in}{0.009208in}}{\pgfqpoint{0.031064in}{0.018041in}}{\pgfqpoint{0.024552in}{0.024552in}}%
\pgfpathcurveto{\pgfqpoint{0.018041in}{0.031064in}}{\pgfqpoint{0.009208in}{0.034722in}}{\pgfqpoint{0.000000in}{0.034722in}}%
\pgfpathcurveto{\pgfqpoint{-0.009208in}{0.034722in}}{\pgfqpoint{-0.018041in}{0.031064in}}{\pgfqpoint{-0.024552in}{0.024552in}}%
\pgfpathcurveto{\pgfqpoint{-0.031064in}{0.018041in}}{\pgfqpoint{-0.034722in}{0.009208in}}{\pgfqpoint{-0.034722in}{0.000000in}}%
\pgfpathcurveto{\pgfqpoint{-0.034722in}{-0.009208in}}{\pgfqpoint{-0.031064in}{-0.018041in}}{\pgfqpoint{-0.024552in}{-0.024552in}}%
\pgfpathcurveto{\pgfqpoint{-0.018041in}{-0.031064in}}{\pgfqpoint{-0.009208in}{-0.034722in}}{\pgfqpoint{0.000000in}{-0.034722in}}%
\pgfpathclose%
\pgfusepath{stroke,fill}%
}%
\begin{pgfscope}%
\pgfsys@transformshift{4.467083in}{3.284136in}%
\pgfsys@useobject{currentmarker}{}%
\end{pgfscope}%
\end{pgfscope}%
\begin{pgfscope}%
\definecolor{textcolor}{rgb}{0.000000,0.000000,0.000000}%
\pgfsetstrokecolor{textcolor}%
\pgfsetfillcolor{textcolor}%
\pgftext[x=4.717083in,y=3.235525in,left,base]{\color{textcolor}\sffamily\fontsize{10.000000}{12.000000}\selectfont Time Timeout}%
\end{pgfscope}%
\begin{pgfscope}%
\pgfsetbuttcap%
\pgfsetroundjoin%
\definecolor{currentfill}{rgb}{0.839216,0.152941,0.156863}%
\pgfsetfillcolor{currentfill}%
\pgfsetlinewidth{1.003750pt}%
\definecolor{currentstroke}{rgb}{0.839216,0.152941,0.156863}%
\pgfsetstrokecolor{currentstroke}%
\pgfsetdash{}{0pt}%
\pgfsys@defobject{currentmarker}{\pgfqpoint{-0.034722in}{-0.034722in}}{\pgfqpoint{0.034722in}{0.034722in}}{%
\pgfpathmoveto{\pgfqpoint{0.000000in}{-0.034722in}}%
\pgfpathcurveto{\pgfqpoint{0.009208in}{-0.034722in}}{\pgfqpoint{0.018041in}{-0.031064in}}{\pgfqpoint{0.024552in}{-0.024552in}}%
\pgfpathcurveto{\pgfqpoint{0.031064in}{-0.018041in}}{\pgfqpoint{0.034722in}{-0.009208in}}{\pgfqpoint{0.034722in}{0.000000in}}%
\pgfpathcurveto{\pgfqpoint{0.034722in}{0.009208in}}{\pgfqpoint{0.031064in}{0.018041in}}{\pgfqpoint{0.024552in}{0.024552in}}%
\pgfpathcurveto{\pgfqpoint{0.018041in}{0.031064in}}{\pgfqpoint{0.009208in}{0.034722in}}{\pgfqpoint{0.000000in}{0.034722in}}%
\pgfpathcurveto{\pgfqpoint{-0.009208in}{0.034722in}}{\pgfqpoint{-0.018041in}{0.031064in}}{\pgfqpoint{-0.024552in}{0.024552in}}%
\pgfpathcurveto{\pgfqpoint{-0.031064in}{0.018041in}}{\pgfqpoint{-0.034722in}{0.009208in}}{\pgfqpoint{-0.034722in}{0.000000in}}%
\pgfpathcurveto{\pgfqpoint{-0.034722in}{-0.009208in}}{\pgfqpoint{-0.031064in}{-0.018041in}}{\pgfqpoint{-0.024552in}{-0.024552in}}%
\pgfpathcurveto{\pgfqpoint{-0.018041in}{-0.031064in}}{\pgfqpoint{-0.009208in}{-0.034722in}}{\pgfqpoint{0.000000in}{-0.034722in}}%
\pgfpathclose%
\pgfusepath{stroke,fill}%
}%
\begin{pgfscope}%
\pgfsys@transformshift{4.467083in}{3.090525in}%
\pgfsys@useobject{currentmarker}{}%
\end{pgfscope}%
\end{pgfscope}%
\begin{pgfscope}%
\definecolor{textcolor}{rgb}{0.000000,0.000000,0.000000}%
\pgfsetstrokecolor{textcolor}%
\pgfsetfillcolor{textcolor}%
\pgftext[x=4.717083in,y=3.041914in,left,base]{\color{textcolor}\sffamily\fontsize{10.000000}{12.000000}\selectfont Memory Timeout}%
\end{pgfscope}%
\end{pgfpicture}%
\makeatother%
\endgroup%

                }
            \end{subfigure}
            \caption{Classes}
        \end{subfigure}
        \caption{Maximum Memory Consumption in Comparison to Code Size}
        \label{f:maxmemtocodesize}
    \end{figure}

    To conclude, our backward analysis performed a bit better in the time evaluation, which is also reflected in the memory consumption. Again, the results show that the observed edges are way more important for memory consumption than the code size or the sources and sinks. It is not possible to estimate the memory consumption prior nor which analysis direction will use less memory.

    % Basically the answer to RQ1: Is the backwards search efficient enough to perform analysis on real world apps?

    % Basically the answer to RQ2: Can we find a pre-analysis known parameter to decide which analysis is more efficient?
    \FloatBarrier
\end{document}