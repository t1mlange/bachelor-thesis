\documentclass[../draft.tex]{subfiles}

\begin{document}
    \chapter{Performance Evaluation}
    In the last chapter, we have shown that our implementation has the necessary soundness to be viable and yields the expected results. 
    We now evaluate our implementation against the existing implementation in \textsc{FlowDroid}.
    
    \section{DroidBench}
    We already introduced \textsc{Droidbench} in \autoref{s:droidbenchvalidation} to validate the soundness of our backward-directed implementation. In this section, we focus on the performance in comparison to the existing forward-directed implementation in \textsc{FlowDroid}. 

    \textsc{DroidBench} has the advantage that all apps are crafted explicitly for benchmarking taint analysis. So, most tests only contain a single-figure number of sources and sinks. Also, the number of sources and sinks are often equal or differ by one to test whether the tool can differentiate something. These simplify the comparison between both analysis directions as neither one has an initial disadvantage.

    Most test cases are small enough to be analyzed in sub-two seconds on an average four-core desktop CPU from 2012. Our test environment is not isolated, so background tasks and the process scheduler can affect the runtime. The short runtime, together with the variance of the unisolated testing environment, render the runtime unusable as a comparison point. In contrast, edge propagations are deterministic\footnote{This is only true if there are enough resources. \textsc{FlowDroid} tries to gracefully terminate when running low on memory. Also, timeouts result in a non-reproducible number of edge propagations.} and correlate with the runtime. Thus, we only use the number of propagations to compare both implementations.

    The configuration is the same as described in \autoref{s:droidbenchconfig}.

    \subsection{Results}
    We compare all test cases where both implementations yield the same result. When rows only contain hyphens, either the result of the test case differed between the two analyses or the IFDS analysis did not start, e.g., because no sink is in the reachable code. \#I denotes the number of edge propagations inside the infoflow analysis and \#A the number of edge propagations inside the alias analysis. We calculated the absolute difference as $\mathit{Result}_{\mathit{B}} - \mathit{Result}_{\mathit{F}}$. The relative difference calculates as follows: $\frac{\mathit{TotalDifference}}{|\mathit{\#I_\mathit{F} + \#A_\mathit{F}}|}$. Hence, negative values signify the backward analysis performed better. The full results are in \autoref{t:droidbenchevaluation}.

    In general, both implementations have similar average edge propagation counts. There are not many test cases where both perform identically; instead, dependent on the specific test case, the relative difference is between $-1$ and $1$. So, the expected behavior from \autoref{s:complexity} occurred: it highly depends on the analyzed app.
    However, we did not expect cases that let the backward edge propagations explode up to a factor of $10 000\%$, as seen in \code{LifecycleTest#BroadcastReceiverLifecycle3} and others. In contrast, the existing forward implementation only at most a relative difference of $100\%$.

    \footnotesize
    \newcommand{\tsubEight}[1]{\multicolumn{9}{c}{#1}\\\hline}
    \begin{longtable}{l | r | r | r | r | r | r | r | r}
        \centering
        & \multicolumn{2}{c|}{\textbf{Forwards}} & \multicolumn{2}{c|}{\textbf{Backwards}} & \multicolumn{4}{c}{\textbf{Difference}}\\
        \textbf{App Name} & \textbf{\#I} & \textbf{\#A} & \textbf{\#I} & \textbf{\#A} & \textbf{\#I} & \textbf{\#A}& \textbf{Total} & \textbf{Relative}\\
        \hline\hline
        \endhead
        \hline
        \tsubEight{AliasingTest}
        FlowSensitivity1 & $175$ & $72$ & $39$ & $4$ & $-136$ & $-68$ & $-204$ & $-0.83$\\
        Merge1 & $94$ & $44$ & $61$ & $9$ & $-33$ & $-35$ & $-68$ & $-0.49$\\
        SimpleAliasing1 & $35$ & $13$ & $20$ & $3$ & $-15$ & $-10$ & $-25$ & $-0.52$\\
        StrongUpdate1 & $30$ & $13$ & $11$ & $3$ & $-19$ & $-10$ & $-29$ & $-0.67$\\
        \hline
        \tsubEight{AndroidSpecificTest}
        ApplicationModeling1 & $235$ & $103$ & $851$ & $1208$ & $616$ & $1105$ & $1721$ & $5.09$\\
        DirectLeak1 & $3$ & $0$ & $4$ & $0$ & $1$ & $0$ & $1$ & $0.33$\\
        InactiveActivity & $-$ & $-$ & $-$ & $-$ & $-$ & $-$ & $-$ & $-$\\
        Library2 & $5$ & $0$ & $6$ & $0$ & $1$ & $0$ & $1$ & $0.2$\\
        LogNoLeak & $-$ & $-$ & $-$ & $-$ & $-$ & $-$ & $-$ & $-$\\
        Obfuscation1 & $4$ & $0$ & $4$ & $0$ & $0$ & $0$ & $0$ & $0.0$\\
        Parcel1 & $144$ & $15$ & $66$ & $68$ & $-78$ & $53$ & $-25$ & $-0.16$\\
        PrivateDataLeak1 & $410$ & $110$ & $599$ & $835$ & $189$ & $725$ & $914$ & $1.76$\\
        PrivateDataLeak2 & $15$ & $0$ & $5$ & $6$ & $-10$ & $6$ & $-4$ & $-0.27$\\
        PrivateDataLeak3 & $17$ & $2$ & $212$ & $143$ & $195$ & $141$ & $336$ & $17.68$\\
        runPublicAPIField1 & $89$ & $1$ & $62$ & $31$ & $-27$ & $30$ & $3$ & $0.03$\\
        runPublicAPIField2 & $5$ & $0$ & $11$ & $1$ & $6$ & $1$ & $7$ & $1.4$\\
        runView1 & $71$ & $50$ & $69$ & $0$ & $-2$ & $-50$ & $-52$ & $-0.43$\\
        \hline
        \tsubEight{ArrayAndListTest}
        ArrayAccess1 & $77$ & $34$ & $51$ & $100$ & $-26$ & $66$ & $40$ & $0.36$\\
        ArrayAccess2 & $16$ & $4$ & $12$ & $0$ & $-4$ & $-4$ & $-8$ & $-0.4$\\
        ArrayAccess3 & $77$ & $34$ & $51$ & $100$ & $-26$ & $66$ & $40$ & $0.36$\\
        ArrayAccess4 & $164$ & $84$ & $42$ & $21$ & $-122$ & $-63$ & $-185$ & $-0.75$\\
        ArrayAccess5 & $75$ & $5$ & $67$ & $63$ & $-8$ & $58$ & $50$ & $0.62$\\
        ArrayCopy1 & $18$ & $2$ & $9$ & $2$ & $-9$ & $0$ & $-9$ & $-0.45$\\
        ArrayToString1 & $10$ & $1$ & $6$ & $1$ & $-4$ & $0$ & $-4$ & $-0.36$\\
        HashMapAccess1 & $22$ & $5$ & $15$ & $1$ & $-7$ & $-4$ & $-11$ & $-0.41$\\
        ListAccess1 & $85$ & $9$ & $60$ & $97$ & $-25$ & $88$ & $63$ & $0.67$\\
        MultidimensionalArray1 & $29$ & $3$ & $16$ & $23$ & $-13$ & $20$ & $7$ & $0.22$\\
        \hline
        \tsubEight{CallbackTest}
        AnonymousClass1 & $152$ & $0$ & $208$ & $1$ & $56$ & $1$ & $57$ & $0.38$\\
        Button1 & $58$ & $39$ & $43$ & $0$ & $-15$ & $-39$ & $-54$ & $-0.56$\\
        Button2 & $454$ & $66$ & $155$ & $257$ & $-299$ & $191$ & $-108$ & $-0.21$\\
        Button3 & $355$ & $89$ & $109$ & $408$ & $-246$ & $319$ & $73$ & $0.16$\\
        Button4 & $58$ & $39$ & $43$ & $0$ & $-15$ & $-39$ & $-54$ & $-0.56$\\
        Button5 & $80$ & $40$ & $6$ & $6$ & $-74$ & $-34$ & $-108$ & $-0.9$\\
        LocationLeak1 & $617$ & $222$ & $260$ & $300$ & $-357$ & $78$ & $-279$ & $-0.33$\\
        LocationLeak2 & $212$ & $121$ & $152$ & $2$ & $-60$ & $-119$ & $-179$ & $-0.54$\\
        LocationLeak3 & $259$ & $73$ & $104$ & $117$ & $-155$ & $44$ & $-111$ & $-0.33$\\
        MethodOverride1 & $3$ & $0$ & $2$ & $0$ & $-1$ & $0$ & $-1$ & $-0.33$\\
        MultiHandlers1 & $17$ & $0$ & $145$ & $151$ & $128$ & $151$ & $279$ & $16.41$\\
        Ordering1 & $456$ & $151$ & $44$ & $2$ & $-412$ & $-149$ & $-561$ & $-0.92$\\
        RegisterGlobal1 & $207$ & $103$ & $49$ & $0$ & $-158$ & $-103$ & $-261$ & $-0.84$\\
        RegisterGlobal2 & $52$ & $37$ & $43$ & $0$ & $-9$ & $-37$ & $-46$ & $-0.52$\\
        Unregister1 & $11$ & $0$ & $9$ & $1$ & $-2$ & $1$ & $-1$ & $-0.09$\\
        \hline
        \tsubEight{EmulatorDetectionTest}
        Battery1 & $7$ & $0$ & $43$ & $15$ & $36$ & $15$ & $51$ & $7.29$\\
        Bluetooth1 & $4$ & $0$ & $4$ & $0$ & $0$ & $0$ & $0$ & $0.0$\\
        Build1 & $4$ & $0$ & $4$ & $0$ & $0$ & $0$ & $0$ & $0.0$\\
        Contacts1 & $52$ & $0$ & $210$ & $19$ & $158$ & $19$ & $177$ & $3.4$\\
        ContentProvider1 & $13$ & $0$ & $8$ & $0$ & $-5$ & $0$ & $-5$ & $-0.38$\\
        DeviceId1 & $15$ & $0$ & $6$ & $0$ & $-9$ & $0$ & $-9$ & $-0.6$\\
        File1 & $4$ & $0$ & $4$ & $0$ & $0$ & $0$ & $0$ & $0.0$\\
        IMEI1 & $129$ & $0$ & $140$ & $34$ & $11$ & $34$ & $45$ & $0.35$\\
        IP1 & $4$ & $0$ & $29$ & $1$ & $25$ & $1$ & $26$ & $6.5$\\
        PI1 & $6$ & $0$ & $4$ & $0$ & $-2$ & $0$ & $-2$ & $-0.33$\\
        PlayStore1 & $158$ & $0$ & $8$ & $0$ & $-150$ & $0$ & $-150$ & $-0.95$\\
        PlayStore2 & $4$ & $0$ & $4$ & $0$ & $0$ & $0$ & $0$ & $0.0$\\
        Sensors1 & $5$ & $0$ & $4$ & $0$ & $-1$ & $0$ & $-1$ & $-0.2$\\
        SubscriberId1 & $29$ & $0$ & $4$ & $0$ & $-25$ & $0$ & $-25$ & $-0.86$\\
        VoiceMail1 & $4$ & $0$ & $4$ & $0$ & $0$ & $0$ & $0$ & $0.0$\\
        \hline
        \tsubEight{FieldAndObjectSensitivityTest}
        FieldSensitivity1 & $98$ & $50$ & $25$ & $3$ & $-73$ & $-47$ & $-120$ & $-0.81$\\
        FieldSensitivity2 & $35$ & $15$ & $19$ & $0$ & $-16$ & $-15$ & $-31$ & $-0.62$\\
        FieldSensitivity3 & $38$ & $15$ & $16$ & $0$ & $-22$ & $-15$ & $-37$ & $-0.7$\\
        FieldSensitivity4 & $14$ & $6$ & $8$ & $0$ & $-6$ & $-6$ & $-12$ & $-0.6$\\
        InheritedObjects1 & $4$ & $0$ & $6$ & $0$ & $2$ & $0$ & $2$ & $0.5$\\
        ObjectSensitivity1 & $19$ & $7$ & $14$ & $1$ & $-5$ & $-6$ & $-11$ & $-0.42$\\
        ObjectSensitivity2 & $15$ & $8$ & $10$ & $0$ & $-5$ & $-8$ & $-13$ & $-0.57$\\
        \hline
        \tsubEight{GeneralJavaTest}
        Clone1 & $23$ & $2$ & $12$ & $4$ & $-11$ & $2$ & $-9$ & $-0.36$\\
        Exceptions1 & $16$ & $0$ & $13$ & $0$ & $-3$ & $0$ & $-3$ & $-0.19$\\
        Exceptions2 & $22$ & $0$ & $13$ & $0$ & $-9$ & $0$ & $-9$ & $-0.41$\\
        Exceptions3 & $18$ & $0$ & $11$ & $0$ & $-7$ & $0$ & $-7$ & $-0.39$\\
        Exceptions4 & $20$ & $1$ & $22$ & $1$ & $2$ & $0$ & $2$ & $0.1$\\
        Exceptions5 & $13$ & $1$ & $16$ & $1$ & $3$ & $0$ & $3$ & $0.21$\\
        Exceptions6 & $77$ & $12$ & $23$ & $0$ & $-54$ & $-12$ & $-66$ & $-0.74$\\
        Exceptions7 & $71$ & $12$ & $6$ & $0$ & $-65$ & $-12$ & $-77$ & $-0.93$\\
        FactoryMethods1 & $40$ & $0$ & $14$ & $2$ & $-26$ & $2$ & $-24$ & $-0.6$\\
        Loop1 & $93$ & $2$ & $46$ & $7$ & $-47$ & $5$ & $-42$ & $-0.44$\\
        Loop2 & $123$ & $2$ & $74$ & $14$ & $-49$ & $12$ & $-37$ & $-0.3$\\
        Serialization1 & $50$ & $4$ & $22$ & $29$ & $-28$ & $25$ & $-3$ & $-0.06$\\
        SourceCodeSpecific1 & $16$ & $0$ & $45$ & $7$ & $29$ & $7$ & $36$ & $2.25$\\
        StartProcessWithSecret1 & $29$ & $8$ & $17$ & $3$ & $-12$ & $-5$ & $-17$ & $-0.46$\\
        StaticInitialization1 & $-$ & $-$ & $9$ & $0$ & $-$ & $-$ & $-$ & $-$\\
        StaticInitialization2 & $57$ & $29$ & $86$ & $0$ & $29$ & $-29$ & $0$ & $0.0$\\
        StaticInitialization3 & $35$ & $9$ & $5$ & $0$ & $-30$ & $-9$ & $-39$ & $-0.89$\\
        StringFormatter1 & $16$ & $1$ & $10$ & $1$ & $-6$ & $0$ & $-6$ & $-0.35$\\
        StringPatternMatching1 & $23$ & $1$ & $8$ & $6$ & $-15$ & $5$ & $-10$ & $-0.42$\\
        StringToCharArray1 & $91$ & $4$ & $42$ & $6$ & $-49$ & $2$ & $-47$ & $-0.49$\\
        StringToOutputStream1 & $26$ & $3$ & $30$ & $3$ & $4$ & $0$ & $4$ & $0.14$\\
        UnreachableCode & $-$ & $-$ & $-$ & $-$ & $-$ & $-$ & $-$ & $-$\\
        VirtualDispatch1 & $128$ & $31$ & $88$ & $28$ & $-40$ & $-3$ & $-43$ & $-0.27$\\
        VirtualDispatch2 & $7$ & $0$ & $12$ & $0$ & $5$ & $0$ & $5$ & $0.71$\\
        VirtualDispatch3 & $8$ & $0$ & $6$ & $0$ & $-2$ & $0$ & $-2$ & $-0.25$\\
        VirtualDispatch4 & $-$ & $-$ & $-$ & $-$ & $-$ & $-$ & $-$ & $-$\\
        \hline
        \tsubEight{LifecycleTest}
        ActivityEventSequence1 & $58$ & $35$ & $73$ & $0$ & $15$ & $-35$ & $-20$ & $-0.22$\\
        ActivityEventSequence2 & $32$ & $24$ & $77$ & $0$ & $45$ & $-24$ & $21$ & $0.38$\\
        ActivityEventSequence3 & $233$ & $116$ & $156$ & $1$ & $-77$ & $-115$ & $-192$ & $-0.55$\\
        ActivityLifecycle1 & $99$ & $72$ & $156$ & $7$ & $57$ & $-65$ & $-8$ & $-0.05$\\
        ActivityLifecycle2 & $47$ & $34$ & $33$ & $0$ & $-14$ & $-34$ & $-48$ & $-0.59$\\
        ActivityLifecycle3 & $65$ & $31$ & $28$ & $0$ & $-37$ & $-31$ & $-68$ & $-0.71$\\
        ActivityLifecycle4 & $49$ & $33$ & $14$ & $0$ & $-35$ & $-33$ & $-68$ & $-0.83$\\
        ActivitySavedState1 & $20$ & $0$ & $7$ & $1$ & $-13$ & $1$ & $-12$ & $-0.6$\\
        ApplicationLifecycle1 & $37$ & $10$ & $82$ & $0$ & $45$ & $-10$ & $35$ & $0.74$\\
        ApplicationLifecycle2 & $86$ & $17$ & $94$ & $155$ & $8$ & $138$ & $146$ & $1.42$\\
        ApplicationLifecycle3 & $32$ & $12$ & $21$ & $0$ & $-11$ & $-12$ & $-23$ & $-0.52$\\
        AsynchronousEventOrdering1 & $51$ & $31$ & $16$ & $0$ & $-35$ & $-31$ & $-66$ & $-0.8$\\
        BroadcastReceiverLifecycle1 & $4$ & $0$ & $4$ & $0$ & $0$ & $0$ & $0$ & $0.0$\\
        BroadcastReceiverLifecycle2 & $109$ & $44$ & $248$ & $114$ & $139$ & $70$ & $209$ & $1.37$\\
        BroadcastReceiverLifecycle3 & $3$ & $0$ & $195$ & $110$ & $192$ & $110$ & $302$ & $100.67$\\
        EventOrdering1 & $61$ & $29$ & $30$ & $0$ & $-31$ & $-29$ & $-60$ & $-0.67$\\
        FragmentLifecycle1 & $187$ & $127$ & $90$ & $0$ & $-97$ & $-127$ & $-224$ & $-0.71$\\
        FragmentLifecycle2 & $-$ & $-$ & $-$ & $-$ & $-$ & $-$ & $-$ & $-$\\
        ServiceEventSequence1 & $53$ & $20$ & $152$ & $34$ & $99$ & $14$ & $113$ & $1.55$\\
        ServiceEventSequence2 & $64$ & $21$ & $389$ & $220$ & $325$ & $199$ & $524$ & $6.16$\\
        ServiceEventSequence3 & $46$ & $12$ & $275$ & $151$ & $229$ & $139$ & $368$ & $6.34$\\
        ServiceLifecycle1 & $119$ & $44$ & $42$ & $0$ & $-77$ & $-44$ & $-121$ & $-0.74$\\
        ServiceLifecycle2 & $68$ & $20$ & $89$ & $21$ & $21$ & $1$ & $22$ & $0.25$\\
        SharedPreferenceChanged1 & $13$ & $0$ & $20$ & $1$ & $7$ & $1$ & $8$ & $0.62$\\
        \hline
        \tsubEight{ReflectionTest}
        Reflection1 & $15$ & $5$ & $8$ & $0$ & $-7$ & $-5$ & $-12$ & $-0.6$\\
        Reflection2 & $21$ & $5$ & $11$ & $0$ & $-10$ & $-5$ & $-15$ & $-0.58$\\
        Reflection3 & $42$ & $9$ & $62$ & $25$ & $20$ & $16$ & $36$ & $0.71$\\
        Reflection4 & $9$ & $0$ & $8$ & $0$ & $-1$ & $0$ & $-1$ & $-0.11$\\
        Reflection5 & $16$ & $1$ & $11$ & $0$ & $-5$ & $-1$ & $-6$ & $-0.35$\\
        Reflection6 & $7$ & $0$ & $134$ & $51$ & $127$ & $51$ & $178$ & $25.43$\\
        Reflection7 & $15$ & $5$ & $15$ & $11$ & $0$ & $6$ & $6$ & $0.3$\\
        Reflection8 & $35$ & $7$ & $14$ & $0$ & $-21$ & $-7$ & $-28$ & $-0.67$\\
        Reflection9 & $42$ & $7$ & $21$ & $0$ & $-21$ & $-7$ & $-28$ & $-0.57$\\
        \hline
        \tsubEight{ThreadingTest}
        AsyncTask1 & $22$ & $2$ & $11$ & $1$ & $-11$ & $-1$ & $-12$ & $-0.5$\\
        Executor1 & $34$ & $7$ & $17$ & $0$ & $-17$ & $-7$ & $-24$ & $-0.59$\\
        JavaThread1 & $34$ & $7$ & $17$ & $0$ & $-17$ & $-7$ & $-24$ & $-0.59$\\
        JavaThread2 & $62$ & $10$ & $31$ & $8$ & $-31$ & $-2$ & $-33$ & $-0.46$\\
        Looper1 & $49$ & $3$ & $20$ & $16$ & $-29$ & $13$ & $-16$ & $-0.31$\\
        TimerTask1 & $203$ & $28$ & $32$ & $33$ & $-171$ & $5$ & $-166$ & $-0.72$\\
        \hline\hline
        $\varnothing$ Propagations & $70.74$ & $21.42$ & $61.94$ & $41.54$ & $-8.8$ & $20.11$ & $11.32$ & $1.41$\\        
        \caption{DroidBench Performance Evaluation Results}
        \label{t:droidbenchevaluation}
    \end{longtable}
    \normalsize

    \subsection{Result Explanation}
    We define tests with a relative difference greater than $10$ as worth investigating. In the following, we explain why our implementation performed worse than expected.

    \paragraph{PrivateDataLeak3} 
    This test contains two sinks and one source. The tainted data is written to a file, later read from the file and then leaked. \textsc{FlowDroid} does not support tracking taints over files, so it only finds a leak from source to file write but misses the leak from file read to send SMS. 
    Due to EasyTaintWrapper's simplicity, overtainting happens in the backward direction. When \code{FileInputStream fis = openFileInput("out.txt");}
    is called with \code{fis} tainted, EasyTaintWrapper also taints the base object - the \code{MainActivity} in this case. As the \code{MainActivity} has a enourmous scope, the taint has a long lifetime and many other taints could derive from this taint. This taint explains the relative difference of $17.68$.
    Using the more precise SummaryTaintWrapper, the edges reduce to $(51, 16)$ and a relative difference of $2.53$, which is more reasonable. It is still higher because of the second sink. 

    \paragraph{MultiHandlers1}
    Two \code{LocationListener}s are registered in different activities. In both activities, an instance field is a parameter of a sink.
    So there are two possible paths where something could be leaked. The LocationListener does not call any source on the first path, while the second path has an empty setter method killing the taint.
    For the first path, the backward analysis has to propagate the taint into the \code{LocationListener} to notice that this is a dead-end while the forward's search does not even start there.
    For the second path, the backward analysis seems to suffer because it starts at an instance field taint with a larger scope than a local variable.

    \paragraph{BroadcastReceiverLifecycle3}
    The test contains five sinks but only one source. If we only consider the leak path, both implementations perform equally. 
    The four other sinks are responsible for the overhead on edge propagations.
    
    \paragraph{Reflection6}
    The reflective call site has multiple callees in the interprocedural control-flow graph. Backward all of these callees are visited, of which only one contains a source statement. Forwards, the taint is introduced in the callee at the source and just one return site needs to be processed.

    \section{Real World Apps}

    \subsection{Configuration}
    Our test machine is equipped with four Intel Xeon E5-4650 and 1 TB of RAM. We limited the JVM to 50 GB RAM and \textsc{FlowDroid} on 16 threads per instance. We ran at most four instances in parallel to ensure a one-to-one mapping between CPU threads and \textsc{FlowDroid} threads. Note that the test machine is a shared system, but we made sure there are always enough resources for our evaluation available. Still, background services might influence the performance of a single run. To stamp out this factor, we ran each app three times. If there were outliers, we repeated the runs.
    
    For this evaluation, we chose to use a non-default configuration of \textsc{FlowDroid}. First, we disabled static field tracking due to the global scope as described in \autoref{s:complexity}. Next, instead of the \code{EasyTaintWrapper}, we use the \code{SummaryTaintWrapper}, which utilizes \textsc{StubDroid}. \textsc{StubDroid}'s precomputed dataflow summaries are more precise than \code{EasyTaintWrapper}'s simple rules. We set the timeout for the dataflow analysis to 10 minutes. 
    The configuration summary is in \autoref{t:realworldconfig}.

    \begin{table}[ht]
        \centering
        \begin{tabular}{l | l}
            \textbf{Option} & \textbf{Value}\\
            \hline\hline
            Array Size Tainting & disabled\\
            Inspect Sources \& Sinks & disabled\\
            Static Field Tracking & disabled\\
            Ignore Flows in System Packages & enabled\\
            Exclude Soot Library Classes & enabled\\
            Timeout & 10 minutes\\
            Taint Wrapper & \code{SummaryTaintWrapper}\\
        \end{tabular}
        \caption{Real World Apps Configuration}
        \label{t:realworldconfig}
    \end{table}
    
    We did not use the full sources and sinks list included in \textsc{FlowDroid} because such would result in hundreds of sources and sinks per app and probably a long runtime. Instead, we chose to analyze which sensitive and possibly user-identifying data is sent out to the internet. As we want to compare the forwards and backward implementation, it is also essential to not put one at a disadvantage. We opted for a 2:1 ratio of sources to sinks. This decision is based on the results of \textsc{SuSi}, a tool to automatically find sources and sinks in the Android framework \cite{Rasthofer2014}. Their extracted list of sources and sinks contains roughly $2.17$ times more sources than sinks.
    The list of sources and sinks used in this evaluation is in \autoref{t:realworldsources} and \autoref{t:realworldsinks}.

    \begin{table}[ht]
        \centering
        \begin{tabular}{l | l}
            \textbf{Class} & \textbf{Method}\\
            \hline\hline
            android.location.Location & getLatitude()\\
            & getLongitude()\\
            \hline
            android.location.LocationManager & getLastKnownLocation()\\
            \hline
            android.telephony.TelephonyManager & getDeviceId()\\
            & getSubscriberId()\\
            & getSimSerialNumber()\\
            & getLine1Number()\\
            & getImei()\\
            & getMeid()\\
            \hline
            android.bluetooth.BluetoothAdapter & getAddress()\\
            android.net.wifi.WifiInfo & getMacAddress()\\
            & getSSID()\\
            & getIpAddress()\\
            \hline
            java.net.InetAddress & getHostAddress()\\
            \hline
            android.telephony.gsm.GsmCellLocation & getCid()\\
            & getLac()\\
            \hline
            android.content.pm.PackageManager & getInstalledApplications()\\
            & getInstalledPackages()\\
            & queryIntentActivities()\\
            & queryIntentServices()\\
            & queryBroadcastReceivers()\\
            \hline
            android.content.SharedPreferences & getDefaultSharedPreferences()\\
            android.provider.Browser & getAllBookmarks()\\
            & getAllVisitedUrls\\
        \end{tabular}
        \caption{Sources for Real World Apps Evaluation}
        \label{t:realworldsources}
    \end{table}

    \begin{table}[ht]
        \centering
        \begin{tabular}{l | l}
            \textbf{Class} & \textbf{Method}\\
            \hline\hline
            java.net.URL & set()\\
            & openConnection()\\
            \hline
            java.net.URLConnection & connect()\\
            & setRequestProperty()\\
            \hline
            android.net.http.HttpsConnection & openConnection()\\
            \hline
            android.net.http.Headers & setEtag()\\
            & setContentType()\\
            & setLastModified()\\
            & setLocation()\\
            \hline
            android.net.http.AndroidHttpClientConnection & sendRequestHeader()\\
            \hline
            android.net.http.RequestQueue & queueRequest()\\
        \end{tabular}
        \caption{Sinks for Real World Apps Evaluation}
        \label{t:realworldsinks}
    \end{table}

    We used FlowDroid's forwards implementation on the to that date latest upstream commit\footnote{The latest upstream commit was at that time b436733fc4a5130dfe4ce8ddb3f76fd374e9a487.} from the develop branch for the point of comparison. The backward implementation ran on our latest commit with all changes merged from the upstream.

    We chose 200 apps randomly out of a Google Playstore dump from 2021 containing over 6000 apps for our evaluation set. The full list is appended to this work in \autoref{?}.

    \subsection{Results}
    Out of 200 apps, 60 apps do not have any sources or sinks and thus, the analysis did not start. For 15 apps, the analysis aborted with errors. We are left with 125 apps for which both implementations completed the analysis.
    \todo{Preliminary! Only one run without memwatcher}


    
    \begin{table}[ht]
        \centering
        \begin{tabular}{l | r | r}
            \textbf{Metric} & \textbf{Forwards} & \textbf{Backwards}\\
            \hline\hline
            Average Runtime & $498.192s$ & $397.152s$\\
            Median Runtime & $600.0s$ & $600.0s$\\
            \hline
            Abstractions Infoflow & $34223577$ & $12318997$\\
            Abstractions Alias & $12347994$ & $30013930$\\
            Total Abstractions & $46571571$ & $42332927$\\
            \hline
            % Average Max Memory Consumption & $14559.38 GB$ & $25465.59 GB$\\
            \hline
            Memory Timeouts & $3.2\%$ & $6.4\%$\\
            Time Timeouts & $80.8\%$ & $57.6\%$\\                
        \end{tabular}
        \caption{Results}
        \label{t:realworldresults}
    \end{table}

    \begin{table}[ht]
        \centering
        \begin{tabular}{l | r | r}
            & \textbf{Forwards} & \textbf{Backwards}\\
            \hline\hline
            Average Runtime & $19.15s$ & $38.07s$\\
            Median Runtime & $0.0s$ & $0.0s$\\
            \hline
            Abstractions Infoflow & $459868$ & $629202$\\
            Abstractions Alias & $155876$ & $1403538$\\
            Total Abstractions & $615744$ & $2032740$\\
            \hline
            % Average Max Memory Consumption & $4526.75 GB$ & $16005.16 GB$\\            
        \end{tabular}
        \caption{Results without Timeouts}
        \label{t:realworldresultswithouttimeout}
    \end{table}

    \todo{Avg w/o timeouts is only per direction, so bw contains way more entries}


    \begin{figure}
        \centering
        \begin{subfigure}[]{0.45\textwidth}
            \centering
            \resizebox{\columnwidth}{!}{
                %% Creator: Matplotlib, PGF backend
%%
%% To include the figure in your LaTeX document, write
%%   \input{<filename>.pgf}
%%
%% Make sure the required packages are loaded in your preamble
%%   \usepackage{pgf}
%%
%% and, on pdftex
%%   \usepackage[utf8]{inputenc}\DeclareUnicodeCharacter{2212}{-}
%%
%% or, on luatex and xetex
%%   \usepackage{unicode-math}
%%
%% Figures using additional raster images can only be included by \input if
%% they are in the same directory as the main LaTeX file. For loading figures
%% from other directories you can use the `import` package
%%   \usepackage{import}
%%
%% and then include the figures with
%%   \import{<path to file>}{<filename>.pgf}
%%
%% Matplotlib used the following preamble
%%   \usepackage{fontspec}
%%
\begingroup%
\makeatletter%
\begin{pgfpicture}%
\pgfpathrectangle{\pgfpointorigin}{\pgfqpoint{6.000000in}{4.000000in}}%
\pgfusepath{use as bounding box, clip}%
\begin{pgfscope}%
\pgfsetbuttcap%
\pgfsetmiterjoin%
\definecolor{currentfill}{rgb}{1.000000,1.000000,1.000000}%
\pgfsetfillcolor{currentfill}%
\pgfsetlinewidth{0.000000pt}%
\definecolor{currentstroke}{rgb}{1.000000,1.000000,1.000000}%
\pgfsetstrokecolor{currentstroke}%
\pgfsetdash{}{0pt}%
\pgfpathmoveto{\pgfqpoint{0.000000in}{0.000000in}}%
\pgfpathlineto{\pgfqpoint{6.000000in}{0.000000in}}%
\pgfpathlineto{\pgfqpoint{6.000000in}{4.000000in}}%
\pgfpathlineto{\pgfqpoint{0.000000in}{4.000000in}}%
\pgfpathclose%
\pgfusepath{fill}%
\end{pgfscope}%
\begin{pgfscope}%
\pgfsetbuttcap%
\pgfsetmiterjoin%
\definecolor{currentfill}{rgb}{1.000000,1.000000,1.000000}%
\pgfsetfillcolor{currentfill}%
\pgfsetlinewidth{0.000000pt}%
\definecolor{currentstroke}{rgb}{0.000000,0.000000,0.000000}%
\pgfsetstrokecolor{currentstroke}%
\pgfsetstrokeopacity{0.000000}%
\pgfsetdash{}{0pt}%
\pgfpathmoveto{\pgfqpoint{0.750000in}{0.500000in}}%
\pgfpathlineto{\pgfqpoint{5.400000in}{0.500000in}}%
\pgfpathlineto{\pgfqpoint{5.400000in}{3.520000in}}%
\pgfpathlineto{\pgfqpoint{0.750000in}{3.520000in}}%
\pgfpathclose%
\pgfusepath{fill}%
\end{pgfscope}%
\begin{pgfscope}%
\pgfpathrectangle{\pgfqpoint{0.750000in}{0.500000in}}{\pgfqpoint{4.650000in}{3.020000in}}%
\pgfusepath{clip}%
\pgfsetbuttcap%
\pgfsetroundjoin%
\definecolor{currentfill}{rgb}{1.000000,0.498039,0.054902}%
\pgfsetfillcolor{currentfill}%
\pgfsetlinewidth{1.003750pt}%
\definecolor{currentstroke}{rgb}{1.000000,0.498039,0.054902}%
\pgfsetstrokecolor{currentstroke}%
\pgfsetdash{}{0pt}%
\pgfpathmoveto{\pgfqpoint{1.625649in}{2.939952in}}%
\pgfpathcurveto{\pgfqpoint{1.636699in}{2.939952in}}{\pgfqpoint{1.647299in}{2.944342in}}{\pgfqpoint{1.655112in}{2.952156in}}%
\pgfpathcurveto{\pgfqpoint{1.662926in}{2.959969in}}{\pgfqpoint{1.667316in}{2.970568in}}{\pgfqpoint{1.667316in}{2.981618in}}%
\pgfpathcurveto{\pgfqpoint{1.667316in}{2.992668in}}{\pgfqpoint{1.662926in}{3.003267in}}{\pgfqpoint{1.655112in}{3.011081in}}%
\pgfpathcurveto{\pgfqpoint{1.647299in}{3.018895in}}{\pgfqpoint{1.636699in}{3.023285in}}{\pgfqpoint{1.625649in}{3.023285in}}%
\pgfpathcurveto{\pgfqpoint{1.614599in}{3.023285in}}{\pgfqpoint{1.604000in}{3.018895in}}{\pgfqpoint{1.596187in}{3.011081in}}%
\pgfpathcurveto{\pgfqpoint{1.588373in}{3.003267in}}{\pgfqpoint{1.583983in}{2.992668in}}{\pgfqpoint{1.583983in}{2.981618in}}%
\pgfpathcurveto{\pgfqpoint{1.583983in}{2.970568in}}{\pgfqpoint{1.588373in}{2.959969in}}{\pgfqpoint{1.596187in}{2.952156in}}%
\pgfpathcurveto{\pgfqpoint{1.604000in}{2.944342in}}{\pgfqpoint{1.614599in}{2.939952in}}{\pgfqpoint{1.625649in}{2.939952in}}%
\pgfpathclose%
\pgfusepath{stroke,fill}%
\end{pgfscope}%
\begin{pgfscope}%
\pgfpathrectangle{\pgfqpoint{0.750000in}{0.500000in}}{\pgfqpoint{4.650000in}{3.020000in}}%
\pgfusepath{clip}%
\pgfsetbuttcap%
\pgfsetroundjoin%
\definecolor{currentfill}{rgb}{0.121569,0.466667,0.705882}%
\pgfsetfillcolor{currentfill}%
\pgfsetlinewidth{1.003750pt}%
\definecolor{currentstroke}{rgb}{0.121569,0.466667,0.705882}%
\pgfsetstrokecolor{currentstroke}%
\pgfsetdash{}{0pt}%
\pgfpathmoveto{\pgfqpoint{1.323701in}{0.595606in}}%
\pgfpathcurveto{\pgfqpoint{1.334751in}{0.595606in}}{\pgfqpoint{1.345350in}{0.599996in}}{\pgfqpoint{1.353164in}{0.607810in}}%
\pgfpathcurveto{\pgfqpoint{1.360978in}{0.615624in}}{\pgfqpoint{1.365368in}{0.626223in}}{\pgfqpoint{1.365368in}{0.637273in}}%
\pgfpathcurveto{\pgfqpoint{1.365368in}{0.648323in}}{\pgfqpoint{1.360978in}{0.658922in}}{\pgfqpoint{1.353164in}{0.666736in}}%
\pgfpathcurveto{\pgfqpoint{1.345350in}{0.674549in}}{\pgfqpoint{1.334751in}{0.678939in}}{\pgfqpoint{1.323701in}{0.678939in}}%
\pgfpathcurveto{\pgfqpoint{1.312651in}{0.678939in}}{\pgfqpoint{1.302052in}{0.674549in}}{\pgfqpoint{1.294239in}{0.666736in}}%
\pgfpathcurveto{\pgfqpoint{1.286425in}{0.658922in}}{\pgfqpoint{1.282035in}{0.648323in}}{\pgfqpoint{1.282035in}{0.637273in}}%
\pgfpathcurveto{\pgfqpoint{1.282035in}{0.626223in}}{\pgfqpoint{1.286425in}{0.615624in}}{\pgfqpoint{1.294239in}{0.607810in}}%
\pgfpathcurveto{\pgfqpoint{1.302052in}{0.599996in}}{\pgfqpoint{1.312651in}{0.595606in}}{\pgfqpoint{1.323701in}{0.595606in}}%
\pgfpathclose%
\pgfusepath{stroke,fill}%
\end{pgfscope}%
\begin{pgfscope}%
\pgfpathrectangle{\pgfqpoint{0.750000in}{0.500000in}}{\pgfqpoint{4.650000in}{3.020000in}}%
\pgfusepath{clip}%
\pgfsetbuttcap%
\pgfsetroundjoin%
\definecolor{currentfill}{rgb}{1.000000,0.498039,0.054902}%
\pgfsetfillcolor{currentfill}%
\pgfsetlinewidth{1.003750pt}%
\definecolor{currentstroke}{rgb}{1.000000,0.498039,0.054902}%
\pgfsetstrokecolor{currentstroke}%
\pgfsetdash{}{0pt}%
\pgfpathmoveto{\pgfqpoint{2.652273in}{3.099616in}}%
\pgfpathcurveto{\pgfqpoint{2.663323in}{3.099616in}}{\pgfqpoint{2.673922in}{3.104007in}}{\pgfqpoint{2.681736in}{3.111820in}}%
\pgfpathcurveto{\pgfqpoint{2.689549in}{3.119634in}}{\pgfqpoint{2.693939in}{3.130233in}}{\pgfqpoint{2.693939in}{3.141283in}}%
\pgfpathcurveto{\pgfqpoint{2.693939in}{3.152333in}}{\pgfqpoint{2.689549in}{3.162932in}}{\pgfqpoint{2.681736in}{3.170746in}}%
\pgfpathcurveto{\pgfqpoint{2.673922in}{3.178559in}}{\pgfqpoint{2.663323in}{3.182950in}}{\pgfqpoint{2.652273in}{3.182950in}}%
\pgfpathcurveto{\pgfqpoint{2.641223in}{3.182950in}}{\pgfqpoint{2.630624in}{3.178559in}}{\pgfqpoint{2.622810in}{3.170746in}}%
\pgfpathcurveto{\pgfqpoint{2.614996in}{3.162932in}}{\pgfqpoint{2.610606in}{3.152333in}}{\pgfqpoint{2.610606in}{3.141283in}}%
\pgfpathcurveto{\pgfqpoint{2.610606in}{3.130233in}}{\pgfqpoint{2.614996in}{3.119634in}}{\pgfqpoint{2.622810in}{3.111820in}}%
\pgfpathcurveto{\pgfqpoint{2.630624in}{3.104007in}}{\pgfqpoint{2.641223in}{3.099616in}}{\pgfqpoint{2.652273in}{3.099616in}}%
\pgfpathclose%
\pgfusepath{stroke,fill}%
\end{pgfscope}%
\begin{pgfscope}%
\pgfpathrectangle{\pgfqpoint{0.750000in}{0.500000in}}{\pgfqpoint{4.650000in}{3.020000in}}%
\pgfusepath{clip}%
\pgfsetbuttcap%
\pgfsetroundjoin%
\definecolor{currentfill}{rgb}{1.000000,0.498039,0.054902}%
\pgfsetfillcolor{currentfill}%
\pgfsetlinewidth{1.003750pt}%
\definecolor{currentstroke}{rgb}{1.000000,0.498039,0.054902}%
\pgfsetstrokecolor{currentstroke}%
\pgfsetdash{}{0pt}%
\pgfpathmoveto{\pgfqpoint{2.531494in}{2.932163in}}%
\pgfpathcurveto{\pgfqpoint{2.542544in}{2.932163in}}{\pgfqpoint{2.553143in}{2.936553in}}{\pgfqpoint{2.560956in}{2.944367in}}%
\pgfpathcurveto{\pgfqpoint{2.568770in}{2.952181in}}{\pgfqpoint{2.573160in}{2.962780in}}{\pgfqpoint{2.573160in}{2.973830in}}%
\pgfpathcurveto{\pgfqpoint{2.573160in}{2.984880in}}{\pgfqpoint{2.568770in}{2.995479in}}{\pgfqpoint{2.560956in}{3.003293in}}%
\pgfpathcurveto{\pgfqpoint{2.553143in}{3.011106in}}{\pgfqpoint{2.542544in}{3.015496in}}{\pgfqpoint{2.531494in}{3.015496in}}%
\pgfpathcurveto{\pgfqpoint{2.520443in}{3.015496in}}{\pgfqpoint{2.509844in}{3.011106in}}{\pgfqpoint{2.502031in}{3.003293in}}%
\pgfpathcurveto{\pgfqpoint{2.494217in}{2.995479in}}{\pgfqpoint{2.489827in}{2.984880in}}{\pgfqpoint{2.489827in}{2.973830in}}%
\pgfpathcurveto{\pgfqpoint{2.489827in}{2.962780in}}{\pgfqpoint{2.494217in}{2.952181in}}{\pgfqpoint{2.502031in}{2.944367in}}%
\pgfpathcurveto{\pgfqpoint{2.509844in}{2.936553in}}{\pgfqpoint{2.520443in}{2.932163in}}{\pgfqpoint{2.531494in}{2.932163in}}%
\pgfpathclose%
\pgfusepath{stroke,fill}%
\end{pgfscope}%
\begin{pgfscope}%
\pgfpathrectangle{\pgfqpoint{0.750000in}{0.500000in}}{\pgfqpoint{4.650000in}{3.020000in}}%
\pgfusepath{clip}%
\pgfsetbuttcap%
\pgfsetroundjoin%
\definecolor{currentfill}{rgb}{1.000000,0.498039,0.054902}%
\pgfsetfillcolor{currentfill}%
\pgfsetlinewidth{1.003750pt}%
\definecolor{currentstroke}{rgb}{1.000000,0.498039,0.054902}%
\pgfsetstrokecolor{currentstroke}%
\pgfsetdash{}{0pt}%
\pgfpathmoveto{\pgfqpoint{2.108766in}{3.084039in}}%
\pgfpathcurveto{\pgfqpoint{2.119816in}{3.084039in}}{\pgfqpoint{2.130415in}{3.088430in}}{\pgfqpoint{2.138229in}{3.096243in}}%
\pgfpathcurveto{\pgfqpoint{2.146043in}{3.104057in}}{\pgfqpoint{2.150433in}{3.114656in}}{\pgfqpoint{2.150433in}{3.125706in}}%
\pgfpathcurveto{\pgfqpoint{2.150433in}{3.136756in}}{\pgfqpoint{2.146043in}{3.147355in}}{\pgfqpoint{2.138229in}{3.155169in}}%
\pgfpathcurveto{\pgfqpoint{2.130415in}{3.162982in}}{\pgfqpoint{2.119816in}{3.167373in}}{\pgfqpoint{2.108766in}{3.167373in}}%
\pgfpathcurveto{\pgfqpoint{2.097716in}{3.167373in}}{\pgfqpoint{2.087117in}{3.162982in}}{\pgfqpoint{2.079303in}{3.155169in}}%
\pgfpathcurveto{\pgfqpoint{2.071490in}{3.147355in}}{\pgfqpoint{2.067100in}{3.136756in}}{\pgfqpoint{2.067100in}{3.125706in}}%
\pgfpathcurveto{\pgfqpoint{2.067100in}{3.114656in}}{\pgfqpoint{2.071490in}{3.104057in}}{\pgfqpoint{2.079303in}{3.096243in}}%
\pgfpathcurveto{\pgfqpoint{2.087117in}{3.088430in}}{\pgfqpoint{2.097716in}{3.084039in}}{\pgfqpoint{2.108766in}{3.084039in}}%
\pgfpathclose%
\pgfusepath{stroke,fill}%
\end{pgfscope}%
\begin{pgfscope}%
\pgfpathrectangle{\pgfqpoint{0.750000in}{0.500000in}}{\pgfqpoint{4.650000in}{3.020000in}}%
\pgfusepath{clip}%
\pgfsetbuttcap%
\pgfsetroundjoin%
\definecolor{currentfill}{rgb}{1.000000,0.498039,0.054902}%
\pgfsetfillcolor{currentfill}%
\pgfsetlinewidth{1.003750pt}%
\definecolor{currentstroke}{rgb}{1.000000,0.498039,0.054902}%
\pgfsetstrokecolor{currentstroke}%
\pgfsetdash{}{0pt}%
\pgfpathmoveto{\pgfqpoint{2.350325in}{2.936057in}}%
\pgfpathcurveto{\pgfqpoint{2.361375in}{2.936057in}}{\pgfqpoint{2.371974in}{2.940448in}}{\pgfqpoint{2.379787in}{2.948261in}}%
\pgfpathcurveto{\pgfqpoint{2.387601in}{2.956075in}}{\pgfqpoint{2.391991in}{2.966674in}}{\pgfqpoint{2.391991in}{2.977724in}}%
\pgfpathcurveto{\pgfqpoint{2.391991in}{2.988774in}}{\pgfqpoint{2.387601in}{2.999373in}}{\pgfqpoint{2.379787in}{3.007187in}}%
\pgfpathcurveto{\pgfqpoint{2.371974in}{3.015000in}}{\pgfqpoint{2.361375in}{3.019391in}}{\pgfqpoint{2.350325in}{3.019391in}}%
\pgfpathcurveto{\pgfqpoint{2.339275in}{3.019391in}}{\pgfqpoint{2.328676in}{3.015000in}}{\pgfqpoint{2.320862in}{3.007187in}}%
\pgfpathcurveto{\pgfqpoint{2.313048in}{2.999373in}}{\pgfqpoint{2.308658in}{2.988774in}}{\pgfqpoint{2.308658in}{2.977724in}}%
\pgfpathcurveto{\pgfqpoint{2.308658in}{2.966674in}}{\pgfqpoint{2.313048in}{2.956075in}}{\pgfqpoint{2.320862in}{2.948261in}}%
\pgfpathcurveto{\pgfqpoint{2.328676in}{2.940448in}}{\pgfqpoint{2.339275in}{2.936057in}}{\pgfqpoint{2.350325in}{2.936057in}}%
\pgfpathclose%
\pgfusepath{stroke,fill}%
\end{pgfscope}%
\begin{pgfscope}%
\pgfpathrectangle{\pgfqpoint{0.750000in}{0.500000in}}{\pgfqpoint{4.650000in}{3.020000in}}%
\pgfusepath{clip}%
\pgfsetbuttcap%
\pgfsetroundjoin%
\definecolor{currentfill}{rgb}{1.000000,0.498039,0.054902}%
\pgfsetfillcolor{currentfill}%
\pgfsetlinewidth{1.003750pt}%
\definecolor{currentstroke}{rgb}{1.000000,0.498039,0.054902}%
\pgfsetstrokecolor{currentstroke}%
\pgfsetdash{}{0pt}%
\pgfpathmoveto{\pgfqpoint{1.686039in}{3.341061in}}%
\pgfpathcurveto{\pgfqpoint{1.697089in}{3.341061in}}{\pgfqpoint{1.707688in}{3.345451in}}{\pgfqpoint{1.715502in}{3.353264in}}%
\pgfpathcurveto{\pgfqpoint{1.723315in}{3.361078in}}{\pgfqpoint{1.727706in}{3.371677in}}{\pgfqpoint{1.727706in}{3.382727in}}%
\pgfpathcurveto{\pgfqpoint{1.727706in}{3.393777in}}{\pgfqpoint{1.723315in}{3.404376in}}{\pgfqpoint{1.715502in}{3.412190in}}%
\pgfpathcurveto{\pgfqpoint{1.707688in}{3.420004in}}{\pgfqpoint{1.697089in}{3.424394in}}{\pgfqpoint{1.686039in}{3.424394in}}%
\pgfpathcurveto{\pgfqpoint{1.674989in}{3.424394in}}{\pgfqpoint{1.664390in}{3.420004in}}{\pgfqpoint{1.656576in}{3.412190in}}%
\pgfpathcurveto{\pgfqpoint{1.648763in}{3.404376in}}{\pgfqpoint{1.644372in}{3.393777in}}{\pgfqpoint{1.644372in}{3.382727in}}%
\pgfpathcurveto{\pgfqpoint{1.644372in}{3.371677in}}{\pgfqpoint{1.648763in}{3.361078in}}{\pgfqpoint{1.656576in}{3.353264in}}%
\pgfpathcurveto{\pgfqpoint{1.664390in}{3.345451in}}{\pgfqpoint{1.674989in}{3.341061in}}{\pgfqpoint{1.686039in}{3.341061in}}%
\pgfpathclose%
\pgfusepath{stroke,fill}%
\end{pgfscope}%
\begin{pgfscope}%
\pgfpathrectangle{\pgfqpoint{0.750000in}{0.500000in}}{\pgfqpoint{4.650000in}{3.020000in}}%
\pgfusepath{clip}%
\pgfsetbuttcap%
\pgfsetroundjoin%
\definecolor{currentfill}{rgb}{1.000000,0.498039,0.054902}%
\pgfsetfillcolor{currentfill}%
\pgfsetlinewidth{1.003750pt}%
\definecolor{currentstroke}{rgb}{1.000000,0.498039,0.054902}%
\pgfsetstrokecolor{currentstroke}%
\pgfsetdash{}{0pt}%
\pgfpathmoveto{\pgfqpoint{1.504870in}{3.169713in}}%
\pgfpathcurveto{\pgfqpoint{1.515920in}{3.169713in}}{\pgfqpoint{1.526519in}{3.174103in}}{\pgfqpoint{1.534333in}{3.181917in}}%
\pgfpathcurveto{\pgfqpoint{1.542147in}{3.189731in}}{\pgfqpoint{1.546537in}{3.200330in}}{\pgfqpoint{1.546537in}{3.211380in}}%
\pgfpathcurveto{\pgfqpoint{1.546537in}{3.222430in}}{\pgfqpoint{1.542147in}{3.233029in}}{\pgfqpoint{1.534333in}{3.240843in}}%
\pgfpathcurveto{\pgfqpoint{1.526519in}{3.248656in}}{\pgfqpoint{1.515920in}{3.253046in}}{\pgfqpoint{1.504870in}{3.253046in}}%
\pgfpathcurveto{\pgfqpoint{1.493820in}{3.253046in}}{\pgfqpoint{1.483221in}{3.248656in}}{\pgfqpoint{1.475407in}{3.240843in}}%
\pgfpathcurveto{\pgfqpoint{1.467594in}{3.233029in}}{\pgfqpoint{1.463203in}{3.222430in}}{\pgfqpoint{1.463203in}{3.211380in}}%
\pgfpathcurveto{\pgfqpoint{1.463203in}{3.200330in}}{\pgfqpoint{1.467594in}{3.189731in}}{\pgfqpoint{1.475407in}{3.181917in}}%
\pgfpathcurveto{\pgfqpoint{1.483221in}{3.174103in}}{\pgfqpoint{1.493820in}{3.169713in}}{\pgfqpoint{1.504870in}{3.169713in}}%
\pgfpathclose%
\pgfusepath{stroke,fill}%
\end{pgfscope}%
\begin{pgfscope}%
\pgfpathrectangle{\pgfqpoint{0.750000in}{0.500000in}}{\pgfqpoint{4.650000in}{3.020000in}}%
\pgfusepath{clip}%
\pgfsetbuttcap%
\pgfsetroundjoin%
\definecolor{currentfill}{rgb}{1.000000,0.498039,0.054902}%
\pgfsetfillcolor{currentfill}%
\pgfsetlinewidth{1.003750pt}%
\definecolor{currentstroke}{rgb}{1.000000,0.498039,0.054902}%
\pgfsetstrokecolor{currentstroke}%
\pgfsetdash{}{0pt}%
\pgfpathmoveto{\pgfqpoint{1.927597in}{2.932163in}}%
\pgfpathcurveto{\pgfqpoint{1.938648in}{2.932163in}}{\pgfqpoint{1.949247in}{2.936553in}}{\pgfqpoint{1.957060in}{2.944367in}}%
\pgfpathcurveto{\pgfqpoint{1.964874in}{2.952181in}}{\pgfqpoint{1.969264in}{2.962780in}}{\pgfqpoint{1.969264in}{2.973830in}}%
\pgfpathcurveto{\pgfqpoint{1.969264in}{2.984880in}}{\pgfqpoint{1.964874in}{2.995479in}}{\pgfqpoint{1.957060in}{3.003293in}}%
\pgfpathcurveto{\pgfqpoint{1.949247in}{3.011106in}}{\pgfqpoint{1.938648in}{3.015496in}}{\pgfqpoint{1.927597in}{3.015496in}}%
\pgfpathcurveto{\pgfqpoint{1.916547in}{3.015496in}}{\pgfqpoint{1.905948in}{3.011106in}}{\pgfqpoint{1.898135in}{3.003293in}}%
\pgfpathcurveto{\pgfqpoint{1.890321in}{2.995479in}}{\pgfqpoint{1.885931in}{2.984880in}}{\pgfqpoint{1.885931in}{2.973830in}}%
\pgfpathcurveto{\pgfqpoint{1.885931in}{2.962780in}}{\pgfqpoint{1.890321in}{2.952181in}}{\pgfqpoint{1.898135in}{2.944367in}}%
\pgfpathcurveto{\pgfqpoint{1.905948in}{2.936553in}}{\pgfqpoint{1.916547in}{2.932163in}}{\pgfqpoint{1.927597in}{2.932163in}}%
\pgfpathclose%
\pgfusepath{stroke,fill}%
\end{pgfscope}%
\begin{pgfscope}%
\pgfpathrectangle{\pgfqpoint{0.750000in}{0.500000in}}{\pgfqpoint{4.650000in}{3.020000in}}%
\pgfusepath{clip}%
\pgfsetbuttcap%
\pgfsetroundjoin%
\definecolor{currentfill}{rgb}{1.000000,0.498039,0.054902}%
\pgfsetfillcolor{currentfill}%
\pgfsetlinewidth{1.003750pt}%
\definecolor{currentstroke}{rgb}{1.000000,0.498039,0.054902}%
\pgfsetstrokecolor{currentstroke}%
\pgfsetdash{}{0pt}%
\pgfpathmoveto{\pgfqpoint{1.686039in}{2.932163in}}%
\pgfpathcurveto{\pgfqpoint{1.697089in}{2.932163in}}{\pgfqpoint{1.707688in}{2.936553in}}{\pgfqpoint{1.715502in}{2.944367in}}%
\pgfpathcurveto{\pgfqpoint{1.723315in}{2.952181in}}{\pgfqpoint{1.727706in}{2.962780in}}{\pgfqpoint{1.727706in}{2.973830in}}%
\pgfpathcurveto{\pgfqpoint{1.727706in}{2.984880in}}{\pgfqpoint{1.723315in}{2.995479in}}{\pgfqpoint{1.715502in}{3.003293in}}%
\pgfpathcurveto{\pgfqpoint{1.707688in}{3.011106in}}{\pgfqpoint{1.697089in}{3.015496in}}{\pgfqpoint{1.686039in}{3.015496in}}%
\pgfpathcurveto{\pgfqpoint{1.674989in}{3.015496in}}{\pgfqpoint{1.664390in}{3.011106in}}{\pgfqpoint{1.656576in}{3.003293in}}%
\pgfpathcurveto{\pgfqpoint{1.648763in}{2.995479in}}{\pgfqpoint{1.644372in}{2.984880in}}{\pgfqpoint{1.644372in}{2.973830in}}%
\pgfpathcurveto{\pgfqpoint{1.644372in}{2.962780in}}{\pgfqpoint{1.648763in}{2.952181in}}{\pgfqpoint{1.656576in}{2.944367in}}%
\pgfpathcurveto{\pgfqpoint{1.664390in}{2.936553in}}{\pgfqpoint{1.674989in}{2.932163in}}{\pgfqpoint{1.686039in}{2.932163in}}%
\pgfpathclose%
\pgfusepath{stroke,fill}%
\end{pgfscope}%
\begin{pgfscope}%
\pgfpathrectangle{\pgfqpoint{0.750000in}{0.500000in}}{\pgfqpoint{4.650000in}{3.020000in}}%
\pgfusepath{clip}%
\pgfsetbuttcap%
\pgfsetroundjoin%
\definecolor{currentfill}{rgb}{1.000000,0.498039,0.054902}%
\pgfsetfillcolor{currentfill}%
\pgfsetlinewidth{1.003750pt}%
\definecolor{currentstroke}{rgb}{1.000000,0.498039,0.054902}%
\pgfsetstrokecolor{currentstroke}%
\pgfsetdash{}{0pt}%
\pgfpathmoveto{\pgfqpoint{3.437338in}{2.928269in}}%
\pgfpathcurveto{\pgfqpoint{3.448388in}{2.928269in}}{\pgfqpoint{3.458987in}{2.932659in}}{\pgfqpoint{3.466800in}{2.940473in}}%
\pgfpathcurveto{\pgfqpoint{3.474614in}{2.948286in}}{\pgfqpoint{3.479004in}{2.958885in}}{\pgfqpoint{3.479004in}{2.969936in}}%
\pgfpathcurveto{\pgfqpoint{3.479004in}{2.980986in}}{\pgfqpoint{3.474614in}{2.991585in}}{\pgfqpoint{3.466800in}{2.999398in}}%
\pgfpathcurveto{\pgfqpoint{3.458987in}{3.007212in}}{\pgfqpoint{3.448388in}{3.011602in}}{\pgfqpoint{3.437338in}{3.011602in}}%
\pgfpathcurveto{\pgfqpoint{3.426288in}{3.011602in}}{\pgfqpoint{3.415689in}{3.007212in}}{\pgfqpoint{3.407875in}{2.999398in}}%
\pgfpathcurveto{\pgfqpoint{3.400061in}{2.991585in}}{\pgfqpoint{3.395671in}{2.980986in}}{\pgfqpoint{3.395671in}{2.969936in}}%
\pgfpathcurveto{\pgfqpoint{3.395671in}{2.958885in}}{\pgfqpoint{3.400061in}{2.948286in}}{\pgfqpoint{3.407875in}{2.940473in}}%
\pgfpathcurveto{\pgfqpoint{3.415689in}{2.932659in}}{\pgfqpoint{3.426288in}{2.928269in}}{\pgfqpoint{3.437338in}{2.928269in}}%
\pgfpathclose%
\pgfusepath{stroke,fill}%
\end{pgfscope}%
\begin{pgfscope}%
\pgfpathrectangle{\pgfqpoint{0.750000in}{0.500000in}}{\pgfqpoint{4.650000in}{3.020000in}}%
\pgfusepath{clip}%
\pgfsetbuttcap%
\pgfsetroundjoin%
\definecolor{currentfill}{rgb}{1.000000,0.498039,0.054902}%
\pgfsetfillcolor{currentfill}%
\pgfsetlinewidth{1.003750pt}%
\definecolor{currentstroke}{rgb}{1.000000,0.498039,0.054902}%
\pgfsetstrokecolor{currentstroke}%
\pgfsetdash{}{0pt}%
\pgfpathmoveto{\pgfqpoint{1.323701in}{1.876818in}}%
\pgfpathcurveto{\pgfqpoint{1.334751in}{1.876818in}}{\pgfqpoint{1.345350in}{1.881208in}}{\pgfqpoint{1.353164in}{1.889022in}}%
\pgfpathcurveto{\pgfqpoint{1.360978in}{1.896836in}}{\pgfqpoint{1.365368in}{1.907435in}}{\pgfqpoint{1.365368in}{1.918485in}}%
\pgfpathcurveto{\pgfqpoint{1.365368in}{1.929535in}}{\pgfqpoint{1.360978in}{1.940134in}}{\pgfqpoint{1.353164in}{1.947948in}}%
\pgfpathcurveto{\pgfqpoint{1.345350in}{1.955761in}}{\pgfqpoint{1.334751in}{1.960152in}}{\pgfqpoint{1.323701in}{1.960152in}}%
\pgfpathcurveto{\pgfqpoint{1.312651in}{1.960152in}}{\pgfqpoint{1.302052in}{1.955761in}}{\pgfqpoint{1.294239in}{1.947948in}}%
\pgfpathcurveto{\pgfqpoint{1.286425in}{1.940134in}}{\pgfqpoint{1.282035in}{1.929535in}}{\pgfqpoint{1.282035in}{1.918485in}}%
\pgfpathcurveto{\pgfqpoint{1.282035in}{1.907435in}}{\pgfqpoint{1.286425in}{1.896836in}}{\pgfqpoint{1.294239in}{1.889022in}}%
\pgfpathcurveto{\pgfqpoint{1.302052in}{1.881208in}}{\pgfqpoint{1.312651in}{1.876818in}}{\pgfqpoint{1.323701in}{1.876818in}}%
\pgfpathclose%
\pgfusepath{stroke,fill}%
\end{pgfscope}%
\begin{pgfscope}%
\pgfpathrectangle{\pgfqpoint{0.750000in}{0.500000in}}{\pgfqpoint{4.650000in}{3.020000in}}%
\pgfusepath{clip}%
\pgfsetbuttcap%
\pgfsetroundjoin%
\definecolor{currentfill}{rgb}{1.000000,0.498039,0.054902}%
\pgfsetfillcolor{currentfill}%
\pgfsetlinewidth{1.003750pt}%
\definecolor{currentstroke}{rgb}{1.000000,0.498039,0.054902}%
\pgfsetstrokecolor{currentstroke}%
\pgfsetdash{}{0pt}%
\pgfpathmoveto{\pgfqpoint{2.531494in}{2.928269in}}%
\pgfpathcurveto{\pgfqpoint{2.542544in}{2.928269in}}{\pgfqpoint{2.553143in}{2.932659in}}{\pgfqpoint{2.560956in}{2.940473in}}%
\pgfpathcurveto{\pgfqpoint{2.568770in}{2.948286in}}{\pgfqpoint{2.573160in}{2.958885in}}{\pgfqpoint{2.573160in}{2.969936in}}%
\pgfpathcurveto{\pgfqpoint{2.573160in}{2.980986in}}{\pgfqpoint{2.568770in}{2.991585in}}{\pgfqpoint{2.560956in}{2.999398in}}%
\pgfpathcurveto{\pgfqpoint{2.553143in}{3.007212in}}{\pgfqpoint{2.542544in}{3.011602in}}{\pgfqpoint{2.531494in}{3.011602in}}%
\pgfpathcurveto{\pgfqpoint{2.520443in}{3.011602in}}{\pgfqpoint{2.509844in}{3.007212in}}{\pgfqpoint{2.502031in}{2.999398in}}%
\pgfpathcurveto{\pgfqpoint{2.494217in}{2.991585in}}{\pgfqpoint{2.489827in}{2.980986in}}{\pgfqpoint{2.489827in}{2.969936in}}%
\pgfpathcurveto{\pgfqpoint{2.489827in}{2.958885in}}{\pgfqpoint{2.494217in}{2.948286in}}{\pgfqpoint{2.502031in}{2.940473in}}%
\pgfpathcurveto{\pgfqpoint{2.509844in}{2.932659in}}{\pgfqpoint{2.520443in}{2.928269in}}{\pgfqpoint{2.531494in}{2.928269in}}%
\pgfpathclose%
\pgfusepath{stroke,fill}%
\end{pgfscope}%
\begin{pgfscope}%
\pgfpathrectangle{\pgfqpoint{0.750000in}{0.500000in}}{\pgfqpoint{4.650000in}{3.020000in}}%
\pgfusepath{clip}%
\pgfsetbuttcap%
\pgfsetroundjoin%
\definecolor{currentfill}{rgb}{1.000000,0.498039,0.054902}%
\pgfsetfillcolor{currentfill}%
\pgfsetlinewidth{1.003750pt}%
\definecolor{currentstroke}{rgb}{1.000000,0.498039,0.054902}%
\pgfsetstrokecolor{currentstroke}%
\pgfsetdash{}{0pt}%
\pgfpathmoveto{\pgfqpoint{1.867208in}{2.947740in}}%
\pgfpathcurveto{\pgfqpoint{1.878258in}{2.947740in}}{\pgfqpoint{1.888857in}{2.952130in}}{\pgfqpoint{1.896671in}{2.959944in}}%
\pgfpathcurveto{\pgfqpoint{1.904484in}{2.967758in}}{\pgfqpoint{1.908874in}{2.978357in}}{\pgfqpoint{1.908874in}{2.989407in}}%
\pgfpathcurveto{\pgfqpoint{1.908874in}{3.000457in}}{\pgfqpoint{1.904484in}{3.011056in}}{\pgfqpoint{1.896671in}{3.018870in}}%
\pgfpathcurveto{\pgfqpoint{1.888857in}{3.026683in}}{\pgfqpoint{1.878258in}{3.031074in}}{\pgfqpoint{1.867208in}{3.031074in}}%
\pgfpathcurveto{\pgfqpoint{1.856158in}{3.031074in}}{\pgfqpoint{1.845559in}{3.026683in}}{\pgfqpoint{1.837745in}{3.018870in}}%
\pgfpathcurveto{\pgfqpoint{1.829931in}{3.011056in}}{\pgfqpoint{1.825541in}{3.000457in}}{\pgfqpoint{1.825541in}{2.989407in}}%
\pgfpathcurveto{\pgfqpoint{1.825541in}{2.978357in}}{\pgfqpoint{1.829931in}{2.967758in}}{\pgfqpoint{1.837745in}{2.959944in}}%
\pgfpathcurveto{\pgfqpoint{1.845559in}{2.952130in}}{\pgfqpoint{1.856158in}{2.947740in}}{\pgfqpoint{1.867208in}{2.947740in}}%
\pgfpathclose%
\pgfusepath{stroke,fill}%
\end{pgfscope}%
\begin{pgfscope}%
\pgfpathrectangle{\pgfqpoint{0.750000in}{0.500000in}}{\pgfqpoint{4.650000in}{3.020000in}}%
\pgfusepath{clip}%
\pgfsetbuttcap%
\pgfsetroundjoin%
\definecolor{currentfill}{rgb}{1.000000,0.498039,0.054902}%
\pgfsetfillcolor{currentfill}%
\pgfsetlinewidth{1.003750pt}%
\definecolor{currentstroke}{rgb}{1.000000,0.498039,0.054902}%
\pgfsetstrokecolor{currentstroke}%
\pgfsetdash{}{0pt}%
\pgfpathmoveto{\pgfqpoint{1.444481in}{2.939952in}}%
\pgfpathcurveto{\pgfqpoint{1.455531in}{2.939952in}}{\pgfqpoint{1.466130in}{2.944342in}}{\pgfqpoint{1.473943in}{2.952156in}}%
\pgfpathcurveto{\pgfqpoint{1.481757in}{2.959969in}}{\pgfqpoint{1.486147in}{2.970568in}}{\pgfqpoint{1.486147in}{2.981618in}}%
\pgfpathcurveto{\pgfqpoint{1.486147in}{2.992668in}}{\pgfqpoint{1.481757in}{3.003267in}}{\pgfqpoint{1.473943in}{3.011081in}}%
\pgfpathcurveto{\pgfqpoint{1.466130in}{3.018895in}}{\pgfqpoint{1.455531in}{3.023285in}}{\pgfqpoint{1.444481in}{3.023285in}}%
\pgfpathcurveto{\pgfqpoint{1.433430in}{3.023285in}}{\pgfqpoint{1.422831in}{3.018895in}}{\pgfqpoint{1.415018in}{3.011081in}}%
\pgfpathcurveto{\pgfqpoint{1.407204in}{3.003267in}}{\pgfqpoint{1.402814in}{2.992668in}}{\pgfqpoint{1.402814in}{2.981618in}}%
\pgfpathcurveto{\pgfqpoint{1.402814in}{2.970568in}}{\pgfqpoint{1.407204in}{2.959969in}}{\pgfqpoint{1.415018in}{2.952156in}}%
\pgfpathcurveto{\pgfqpoint{1.422831in}{2.944342in}}{\pgfqpoint{1.433430in}{2.939952in}}{\pgfqpoint{1.444481in}{2.939952in}}%
\pgfpathclose%
\pgfusepath{stroke,fill}%
\end{pgfscope}%
\begin{pgfscope}%
\pgfpathrectangle{\pgfqpoint{0.750000in}{0.500000in}}{\pgfqpoint{4.650000in}{3.020000in}}%
\pgfusepath{clip}%
\pgfsetbuttcap%
\pgfsetroundjoin%
\definecolor{currentfill}{rgb}{1.000000,0.498039,0.054902}%
\pgfsetfillcolor{currentfill}%
\pgfsetlinewidth{1.003750pt}%
\definecolor{currentstroke}{rgb}{1.000000,0.498039,0.054902}%
\pgfsetstrokecolor{currentstroke}%
\pgfsetdash{}{0pt}%
\pgfpathmoveto{\pgfqpoint{1.987987in}{2.939952in}}%
\pgfpathcurveto{\pgfqpoint{1.999037in}{2.939952in}}{\pgfqpoint{2.009636in}{2.944342in}}{\pgfqpoint{2.017450in}{2.952156in}}%
\pgfpathcurveto{\pgfqpoint{2.025263in}{2.959969in}}{\pgfqpoint{2.029654in}{2.970568in}}{\pgfqpoint{2.029654in}{2.981618in}}%
\pgfpathcurveto{\pgfqpoint{2.029654in}{2.992668in}}{\pgfqpoint{2.025263in}{3.003267in}}{\pgfqpoint{2.017450in}{3.011081in}}%
\pgfpathcurveto{\pgfqpoint{2.009636in}{3.018895in}}{\pgfqpoint{1.999037in}{3.023285in}}{\pgfqpoint{1.987987in}{3.023285in}}%
\pgfpathcurveto{\pgfqpoint{1.976937in}{3.023285in}}{\pgfqpoint{1.966338in}{3.018895in}}{\pgfqpoint{1.958524in}{3.011081in}}%
\pgfpathcurveto{\pgfqpoint{1.950711in}{3.003267in}}{\pgfqpoint{1.946320in}{2.992668in}}{\pgfqpoint{1.946320in}{2.981618in}}%
\pgfpathcurveto{\pgfqpoint{1.946320in}{2.970568in}}{\pgfqpoint{1.950711in}{2.959969in}}{\pgfqpoint{1.958524in}{2.952156in}}%
\pgfpathcurveto{\pgfqpoint{1.966338in}{2.944342in}}{\pgfqpoint{1.976937in}{2.939952in}}{\pgfqpoint{1.987987in}{2.939952in}}%
\pgfpathclose%
\pgfusepath{stroke,fill}%
\end{pgfscope}%
\begin{pgfscope}%
\pgfpathrectangle{\pgfqpoint{0.750000in}{0.500000in}}{\pgfqpoint{4.650000in}{3.020000in}}%
\pgfusepath{clip}%
\pgfsetbuttcap%
\pgfsetroundjoin%
\definecolor{currentfill}{rgb}{0.121569,0.466667,0.705882}%
\pgfsetfillcolor{currentfill}%
\pgfsetlinewidth{1.003750pt}%
\definecolor{currentstroke}{rgb}{0.121569,0.466667,0.705882}%
\pgfsetstrokecolor{currentstroke}%
\pgfsetdash{}{0pt}%
\pgfpathmoveto{\pgfqpoint{1.202922in}{0.595606in}}%
\pgfpathcurveto{\pgfqpoint{1.213972in}{0.595606in}}{\pgfqpoint{1.224571in}{0.599996in}}{\pgfqpoint{1.232385in}{0.607810in}}%
\pgfpathcurveto{\pgfqpoint{1.240198in}{0.615624in}}{\pgfqpoint{1.244589in}{0.626223in}}{\pgfqpoint{1.244589in}{0.637273in}}%
\pgfpathcurveto{\pgfqpoint{1.244589in}{0.648323in}}{\pgfqpoint{1.240198in}{0.658922in}}{\pgfqpoint{1.232385in}{0.666736in}}%
\pgfpathcurveto{\pgfqpoint{1.224571in}{0.674549in}}{\pgfqpoint{1.213972in}{0.678939in}}{\pgfqpoint{1.202922in}{0.678939in}}%
\pgfpathcurveto{\pgfqpoint{1.191872in}{0.678939in}}{\pgfqpoint{1.181273in}{0.674549in}}{\pgfqpoint{1.173459in}{0.666736in}}%
\pgfpathcurveto{\pgfqpoint{1.165646in}{0.658922in}}{\pgfqpoint{1.161255in}{0.648323in}}{\pgfqpoint{1.161255in}{0.637273in}}%
\pgfpathcurveto{\pgfqpoint{1.161255in}{0.626223in}}{\pgfqpoint{1.165646in}{0.615624in}}{\pgfqpoint{1.173459in}{0.607810in}}%
\pgfpathcurveto{\pgfqpoint{1.181273in}{0.599996in}}{\pgfqpoint{1.191872in}{0.595606in}}{\pgfqpoint{1.202922in}{0.595606in}}%
\pgfpathclose%
\pgfusepath{stroke,fill}%
\end{pgfscope}%
\begin{pgfscope}%
\pgfpathrectangle{\pgfqpoint{0.750000in}{0.500000in}}{\pgfqpoint{4.650000in}{3.020000in}}%
\pgfusepath{clip}%
\pgfsetbuttcap%
\pgfsetroundjoin%
\definecolor{currentfill}{rgb}{1.000000,0.498039,0.054902}%
\pgfsetfillcolor{currentfill}%
\pgfsetlinewidth{1.003750pt}%
\definecolor{currentstroke}{rgb}{1.000000,0.498039,0.054902}%
\pgfsetstrokecolor{currentstroke}%
\pgfsetdash{}{0pt}%
\pgfpathmoveto{\pgfqpoint{1.444481in}{2.932163in}}%
\pgfpathcurveto{\pgfqpoint{1.455531in}{2.932163in}}{\pgfqpoint{1.466130in}{2.936553in}}{\pgfqpoint{1.473943in}{2.944367in}}%
\pgfpathcurveto{\pgfqpoint{1.481757in}{2.952181in}}{\pgfqpoint{1.486147in}{2.962780in}}{\pgfqpoint{1.486147in}{2.973830in}}%
\pgfpathcurveto{\pgfqpoint{1.486147in}{2.984880in}}{\pgfqpoint{1.481757in}{2.995479in}}{\pgfqpoint{1.473943in}{3.003293in}}%
\pgfpathcurveto{\pgfqpoint{1.466130in}{3.011106in}}{\pgfqpoint{1.455531in}{3.015496in}}{\pgfqpoint{1.444481in}{3.015496in}}%
\pgfpathcurveto{\pgfqpoint{1.433430in}{3.015496in}}{\pgfqpoint{1.422831in}{3.011106in}}{\pgfqpoint{1.415018in}{3.003293in}}%
\pgfpathcurveto{\pgfqpoint{1.407204in}{2.995479in}}{\pgfqpoint{1.402814in}{2.984880in}}{\pgfqpoint{1.402814in}{2.973830in}}%
\pgfpathcurveto{\pgfqpoint{1.402814in}{2.962780in}}{\pgfqpoint{1.407204in}{2.952181in}}{\pgfqpoint{1.415018in}{2.944367in}}%
\pgfpathcurveto{\pgfqpoint{1.422831in}{2.936553in}}{\pgfqpoint{1.433430in}{2.932163in}}{\pgfqpoint{1.444481in}{2.932163in}}%
\pgfpathclose%
\pgfusepath{stroke,fill}%
\end{pgfscope}%
\begin{pgfscope}%
\pgfpathrectangle{\pgfqpoint{0.750000in}{0.500000in}}{\pgfqpoint{4.650000in}{3.020000in}}%
\pgfusepath{clip}%
\pgfsetbuttcap%
\pgfsetroundjoin%
\definecolor{currentfill}{rgb}{1.000000,0.498039,0.054902}%
\pgfsetfillcolor{currentfill}%
\pgfsetlinewidth{1.003750pt}%
\definecolor{currentstroke}{rgb}{1.000000,0.498039,0.054902}%
\pgfsetstrokecolor{currentstroke}%
\pgfsetdash{}{0pt}%
\pgfpathmoveto{\pgfqpoint{1.686039in}{2.920480in}}%
\pgfpathcurveto{\pgfqpoint{1.697089in}{2.920480in}}{\pgfqpoint{1.707688in}{2.924871in}}{\pgfqpoint{1.715502in}{2.932684in}}%
\pgfpathcurveto{\pgfqpoint{1.723315in}{2.940498in}}{\pgfqpoint{1.727706in}{2.951097in}}{\pgfqpoint{1.727706in}{2.962147in}}%
\pgfpathcurveto{\pgfqpoint{1.727706in}{2.973197in}}{\pgfqpoint{1.723315in}{2.983796in}}{\pgfqpoint{1.715502in}{2.991610in}}%
\pgfpathcurveto{\pgfqpoint{1.707688in}{2.999423in}}{\pgfqpoint{1.697089in}{3.003814in}}{\pgfqpoint{1.686039in}{3.003814in}}%
\pgfpathcurveto{\pgfqpoint{1.674989in}{3.003814in}}{\pgfqpoint{1.664390in}{2.999423in}}{\pgfqpoint{1.656576in}{2.991610in}}%
\pgfpathcurveto{\pgfqpoint{1.648763in}{2.983796in}}{\pgfqpoint{1.644372in}{2.973197in}}{\pgfqpoint{1.644372in}{2.962147in}}%
\pgfpathcurveto{\pgfqpoint{1.644372in}{2.951097in}}{\pgfqpoint{1.648763in}{2.940498in}}{\pgfqpoint{1.656576in}{2.932684in}}%
\pgfpathcurveto{\pgfqpoint{1.664390in}{2.924871in}}{\pgfqpoint{1.674989in}{2.920480in}}{\pgfqpoint{1.686039in}{2.920480in}}%
\pgfpathclose%
\pgfusepath{stroke,fill}%
\end{pgfscope}%
\begin{pgfscope}%
\pgfpathrectangle{\pgfqpoint{0.750000in}{0.500000in}}{\pgfqpoint{4.650000in}{3.020000in}}%
\pgfusepath{clip}%
\pgfsetbuttcap%
\pgfsetroundjoin%
\definecolor{currentfill}{rgb}{1.000000,0.498039,0.054902}%
\pgfsetfillcolor{currentfill}%
\pgfsetlinewidth{1.003750pt}%
\definecolor{currentstroke}{rgb}{1.000000,0.498039,0.054902}%
\pgfsetstrokecolor{currentstroke}%
\pgfsetdash{}{0pt}%
\pgfpathmoveto{\pgfqpoint{2.289935in}{3.247598in}}%
\pgfpathcurveto{\pgfqpoint{2.300985in}{3.247598in}}{\pgfqpoint{2.311584in}{3.251989in}}{\pgfqpoint{2.319398in}{3.259802in}}%
\pgfpathcurveto{\pgfqpoint{2.327211in}{3.267616in}}{\pgfqpoint{2.331602in}{3.278215in}}{\pgfqpoint{2.331602in}{3.289265in}}%
\pgfpathcurveto{\pgfqpoint{2.331602in}{3.300315in}}{\pgfqpoint{2.327211in}{3.310914in}}{\pgfqpoint{2.319398in}{3.318728in}}%
\pgfpathcurveto{\pgfqpoint{2.311584in}{3.326541in}}{\pgfqpoint{2.300985in}{3.330932in}}{\pgfqpoint{2.289935in}{3.330932in}}%
\pgfpathcurveto{\pgfqpoint{2.278885in}{3.330932in}}{\pgfqpoint{2.268286in}{3.326541in}}{\pgfqpoint{2.260472in}{3.318728in}}%
\pgfpathcurveto{\pgfqpoint{2.252659in}{3.310914in}}{\pgfqpoint{2.248268in}{3.300315in}}{\pgfqpoint{2.248268in}{3.289265in}}%
\pgfpathcurveto{\pgfqpoint{2.248268in}{3.278215in}}{\pgfqpoint{2.252659in}{3.267616in}}{\pgfqpoint{2.260472in}{3.259802in}}%
\pgfpathcurveto{\pgfqpoint{2.268286in}{3.251989in}}{\pgfqpoint{2.278885in}{3.247598in}}{\pgfqpoint{2.289935in}{3.247598in}}%
\pgfpathclose%
\pgfusepath{stroke,fill}%
\end{pgfscope}%
\begin{pgfscope}%
\pgfpathrectangle{\pgfqpoint{0.750000in}{0.500000in}}{\pgfqpoint{4.650000in}{3.020000in}}%
\pgfusepath{clip}%
\pgfsetbuttcap%
\pgfsetroundjoin%
\definecolor{currentfill}{rgb}{1.000000,0.498039,0.054902}%
\pgfsetfillcolor{currentfill}%
\pgfsetlinewidth{1.003750pt}%
\definecolor{currentstroke}{rgb}{1.000000,0.498039,0.054902}%
\pgfsetstrokecolor{currentstroke}%
\pgfsetdash{}{0pt}%
\pgfpathmoveto{\pgfqpoint{2.833442in}{2.936057in}}%
\pgfpathcurveto{\pgfqpoint{2.844492in}{2.936057in}}{\pgfqpoint{2.855091in}{2.940448in}}{\pgfqpoint{2.862904in}{2.948261in}}%
\pgfpathcurveto{\pgfqpoint{2.870718in}{2.956075in}}{\pgfqpoint{2.875108in}{2.966674in}}{\pgfqpoint{2.875108in}{2.977724in}}%
\pgfpathcurveto{\pgfqpoint{2.875108in}{2.988774in}}{\pgfqpoint{2.870718in}{2.999373in}}{\pgfqpoint{2.862904in}{3.007187in}}%
\pgfpathcurveto{\pgfqpoint{2.855091in}{3.015000in}}{\pgfqpoint{2.844492in}{3.019391in}}{\pgfqpoint{2.833442in}{3.019391in}}%
\pgfpathcurveto{\pgfqpoint{2.822391in}{3.019391in}}{\pgfqpoint{2.811792in}{3.015000in}}{\pgfqpoint{2.803979in}{3.007187in}}%
\pgfpathcurveto{\pgfqpoint{2.796165in}{2.999373in}}{\pgfqpoint{2.791775in}{2.988774in}}{\pgfqpoint{2.791775in}{2.977724in}}%
\pgfpathcurveto{\pgfqpoint{2.791775in}{2.966674in}}{\pgfqpoint{2.796165in}{2.956075in}}{\pgfqpoint{2.803979in}{2.948261in}}%
\pgfpathcurveto{\pgfqpoint{2.811792in}{2.940448in}}{\pgfqpoint{2.822391in}{2.936057in}}{\pgfqpoint{2.833442in}{2.936057in}}%
\pgfpathclose%
\pgfusepath{stroke,fill}%
\end{pgfscope}%
\begin{pgfscope}%
\pgfpathrectangle{\pgfqpoint{0.750000in}{0.500000in}}{\pgfqpoint{4.650000in}{3.020000in}}%
\pgfusepath{clip}%
\pgfsetbuttcap%
\pgfsetroundjoin%
\definecolor{currentfill}{rgb}{1.000000,0.498039,0.054902}%
\pgfsetfillcolor{currentfill}%
\pgfsetlinewidth{1.003750pt}%
\definecolor{currentstroke}{rgb}{1.000000,0.498039,0.054902}%
\pgfsetstrokecolor{currentstroke}%
\pgfsetdash{}{0pt}%
\pgfpathmoveto{\pgfqpoint{1.444481in}{2.328553in}}%
\pgfpathcurveto{\pgfqpoint{1.455531in}{2.328553in}}{\pgfqpoint{1.466130in}{2.332943in}}{\pgfqpoint{1.473943in}{2.340756in}}%
\pgfpathcurveto{\pgfqpoint{1.481757in}{2.348570in}}{\pgfqpoint{1.486147in}{2.359169in}}{\pgfqpoint{1.486147in}{2.370219in}}%
\pgfpathcurveto{\pgfqpoint{1.486147in}{2.381269in}}{\pgfqpoint{1.481757in}{2.391868in}}{\pgfqpoint{1.473943in}{2.399682in}}%
\pgfpathcurveto{\pgfqpoint{1.466130in}{2.407496in}}{\pgfqpoint{1.455531in}{2.411886in}}{\pgfqpoint{1.444481in}{2.411886in}}%
\pgfpathcurveto{\pgfqpoint{1.433430in}{2.411886in}}{\pgfqpoint{1.422831in}{2.407496in}}{\pgfqpoint{1.415018in}{2.399682in}}%
\pgfpathcurveto{\pgfqpoint{1.407204in}{2.391868in}}{\pgfqpoint{1.402814in}{2.381269in}}{\pgfqpoint{1.402814in}{2.370219in}}%
\pgfpathcurveto{\pgfqpoint{1.402814in}{2.359169in}}{\pgfqpoint{1.407204in}{2.348570in}}{\pgfqpoint{1.415018in}{2.340756in}}%
\pgfpathcurveto{\pgfqpoint{1.422831in}{2.332943in}}{\pgfqpoint{1.433430in}{2.328553in}}{\pgfqpoint{1.444481in}{2.328553in}}%
\pgfpathclose%
\pgfusepath{stroke,fill}%
\end{pgfscope}%
\begin{pgfscope}%
\pgfpathrectangle{\pgfqpoint{0.750000in}{0.500000in}}{\pgfqpoint{4.650000in}{3.020000in}}%
\pgfusepath{clip}%
\pgfsetbuttcap%
\pgfsetroundjoin%
\definecolor{currentfill}{rgb}{1.000000,0.498039,0.054902}%
\pgfsetfillcolor{currentfill}%
\pgfsetlinewidth{1.003750pt}%
\definecolor{currentstroke}{rgb}{1.000000,0.498039,0.054902}%
\pgfsetstrokecolor{currentstroke}%
\pgfsetdash{}{0pt}%
\pgfpathmoveto{\pgfqpoint{1.746429in}{3.181396in}}%
\pgfpathcurveto{\pgfqpoint{1.757479in}{3.181396in}}{\pgfqpoint{1.768078in}{3.185786in}}{\pgfqpoint{1.775891in}{3.193600in}}%
\pgfpathcurveto{\pgfqpoint{1.783705in}{3.201413in}}{\pgfqpoint{1.788095in}{3.212012in}}{\pgfqpoint{1.788095in}{3.223063in}}%
\pgfpathcurveto{\pgfqpoint{1.788095in}{3.234113in}}{\pgfqpoint{1.783705in}{3.244712in}}{\pgfqpoint{1.775891in}{3.252525in}}%
\pgfpathcurveto{\pgfqpoint{1.768078in}{3.260339in}}{\pgfqpoint{1.757479in}{3.264729in}}{\pgfqpoint{1.746429in}{3.264729in}}%
\pgfpathcurveto{\pgfqpoint{1.735378in}{3.264729in}}{\pgfqpoint{1.724779in}{3.260339in}}{\pgfqpoint{1.716966in}{3.252525in}}%
\pgfpathcurveto{\pgfqpoint{1.709152in}{3.244712in}}{\pgfqpoint{1.704762in}{3.234113in}}{\pgfqpoint{1.704762in}{3.223063in}}%
\pgfpathcurveto{\pgfqpoint{1.704762in}{3.212012in}}{\pgfqpoint{1.709152in}{3.201413in}}{\pgfqpoint{1.716966in}{3.193600in}}%
\pgfpathcurveto{\pgfqpoint{1.724779in}{3.185786in}}{\pgfqpoint{1.735378in}{3.181396in}}{\pgfqpoint{1.746429in}{3.181396in}}%
\pgfpathclose%
\pgfusepath{stroke,fill}%
\end{pgfscope}%
\begin{pgfscope}%
\pgfpathrectangle{\pgfqpoint{0.750000in}{0.500000in}}{\pgfqpoint{4.650000in}{3.020000in}}%
\pgfusepath{clip}%
\pgfsetbuttcap%
\pgfsetroundjoin%
\definecolor{currentfill}{rgb}{0.839216,0.152941,0.156863}%
\pgfsetfillcolor{currentfill}%
\pgfsetlinewidth{1.003750pt}%
\definecolor{currentstroke}{rgb}{0.839216,0.152941,0.156863}%
\pgfsetstrokecolor{currentstroke}%
\pgfsetdash{}{0pt}%
\pgfpathmoveto{\pgfqpoint{1.806818in}{2.936057in}}%
\pgfpathcurveto{\pgfqpoint{1.817868in}{2.936057in}}{\pgfqpoint{1.828467in}{2.940448in}}{\pgfqpoint{1.836281in}{2.948261in}}%
\pgfpathcurveto{\pgfqpoint{1.844095in}{2.956075in}}{\pgfqpoint{1.848485in}{2.966674in}}{\pgfqpoint{1.848485in}{2.977724in}}%
\pgfpathcurveto{\pgfqpoint{1.848485in}{2.988774in}}{\pgfqpoint{1.844095in}{2.999373in}}{\pgfqpoint{1.836281in}{3.007187in}}%
\pgfpathcurveto{\pgfqpoint{1.828467in}{3.015000in}}{\pgfqpoint{1.817868in}{3.019391in}}{\pgfqpoint{1.806818in}{3.019391in}}%
\pgfpathcurveto{\pgfqpoint{1.795768in}{3.019391in}}{\pgfqpoint{1.785169in}{3.015000in}}{\pgfqpoint{1.777355in}{3.007187in}}%
\pgfpathcurveto{\pgfqpoint{1.769542in}{2.999373in}}{\pgfqpoint{1.765152in}{2.988774in}}{\pgfqpoint{1.765152in}{2.977724in}}%
\pgfpathcurveto{\pgfqpoint{1.765152in}{2.966674in}}{\pgfqpoint{1.769542in}{2.956075in}}{\pgfqpoint{1.777355in}{2.948261in}}%
\pgfpathcurveto{\pgfqpoint{1.785169in}{2.940448in}}{\pgfqpoint{1.795768in}{2.936057in}}{\pgfqpoint{1.806818in}{2.936057in}}%
\pgfpathclose%
\pgfusepath{stroke,fill}%
\end{pgfscope}%
\begin{pgfscope}%
\pgfpathrectangle{\pgfqpoint{0.750000in}{0.500000in}}{\pgfqpoint{4.650000in}{3.020000in}}%
\pgfusepath{clip}%
\pgfsetbuttcap%
\pgfsetroundjoin%
\definecolor{currentfill}{rgb}{1.000000,0.498039,0.054902}%
\pgfsetfillcolor{currentfill}%
\pgfsetlinewidth{1.003750pt}%
\definecolor{currentstroke}{rgb}{1.000000,0.498039,0.054902}%
\pgfsetstrokecolor{currentstroke}%
\pgfsetdash{}{0pt}%
\pgfpathmoveto{\pgfqpoint{1.746429in}{2.936057in}}%
\pgfpathcurveto{\pgfqpoint{1.757479in}{2.936057in}}{\pgfqpoint{1.768078in}{2.940448in}}{\pgfqpoint{1.775891in}{2.948261in}}%
\pgfpathcurveto{\pgfqpoint{1.783705in}{2.956075in}}{\pgfqpoint{1.788095in}{2.966674in}}{\pgfqpoint{1.788095in}{2.977724in}}%
\pgfpathcurveto{\pgfqpoint{1.788095in}{2.988774in}}{\pgfqpoint{1.783705in}{2.999373in}}{\pgfqpoint{1.775891in}{3.007187in}}%
\pgfpathcurveto{\pgfqpoint{1.768078in}{3.015000in}}{\pgfqpoint{1.757479in}{3.019391in}}{\pgfqpoint{1.746429in}{3.019391in}}%
\pgfpathcurveto{\pgfqpoint{1.735378in}{3.019391in}}{\pgfqpoint{1.724779in}{3.015000in}}{\pgfqpoint{1.716966in}{3.007187in}}%
\pgfpathcurveto{\pgfqpoint{1.709152in}{2.999373in}}{\pgfqpoint{1.704762in}{2.988774in}}{\pgfqpoint{1.704762in}{2.977724in}}%
\pgfpathcurveto{\pgfqpoint{1.704762in}{2.966674in}}{\pgfqpoint{1.709152in}{2.956075in}}{\pgfqpoint{1.716966in}{2.948261in}}%
\pgfpathcurveto{\pgfqpoint{1.724779in}{2.940448in}}{\pgfqpoint{1.735378in}{2.936057in}}{\pgfqpoint{1.746429in}{2.936057in}}%
\pgfpathclose%
\pgfusepath{stroke,fill}%
\end{pgfscope}%
\begin{pgfscope}%
\pgfpathrectangle{\pgfqpoint{0.750000in}{0.500000in}}{\pgfqpoint{4.650000in}{3.020000in}}%
\pgfusepath{clip}%
\pgfsetbuttcap%
\pgfsetroundjoin%
\definecolor{currentfill}{rgb}{0.121569,0.466667,0.705882}%
\pgfsetfillcolor{currentfill}%
\pgfsetlinewidth{1.003750pt}%
\definecolor{currentstroke}{rgb}{0.121569,0.466667,0.705882}%
\pgfsetstrokecolor{currentstroke}%
\pgfsetdash{}{0pt}%
\pgfpathmoveto{\pgfqpoint{1.021753in}{0.595606in}}%
\pgfpathcurveto{\pgfqpoint{1.032803in}{0.595606in}}{\pgfqpoint{1.043402in}{0.599996in}}{\pgfqpoint{1.051216in}{0.607810in}}%
\pgfpathcurveto{\pgfqpoint{1.059030in}{0.615624in}}{\pgfqpoint{1.063420in}{0.626223in}}{\pgfqpoint{1.063420in}{0.637273in}}%
\pgfpathcurveto{\pgfqpoint{1.063420in}{0.648323in}}{\pgfqpoint{1.059030in}{0.658922in}}{\pgfqpoint{1.051216in}{0.666736in}}%
\pgfpathcurveto{\pgfqpoint{1.043402in}{0.674549in}}{\pgfqpoint{1.032803in}{0.678939in}}{\pgfqpoint{1.021753in}{0.678939in}}%
\pgfpathcurveto{\pgfqpoint{1.010703in}{0.678939in}}{\pgfqpoint{1.000104in}{0.674549in}}{\pgfqpoint{0.992290in}{0.666736in}}%
\pgfpathcurveto{\pgfqpoint{0.984477in}{0.658922in}}{\pgfqpoint{0.980087in}{0.648323in}}{\pgfqpoint{0.980087in}{0.637273in}}%
\pgfpathcurveto{\pgfqpoint{0.980087in}{0.626223in}}{\pgfqpoint{0.984477in}{0.615624in}}{\pgfqpoint{0.992290in}{0.607810in}}%
\pgfpathcurveto{\pgfqpoint{1.000104in}{0.599996in}}{\pgfqpoint{1.010703in}{0.595606in}}{\pgfqpoint{1.021753in}{0.595606in}}%
\pgfpathclose%
\pgfusepath{stroke,fill}%
\end{pgfscope}%
\begin{pgfscope}%
\pgfpathrectangle{\pgfqpoint{0.750000in}{0.500000in}}{\pgfqpoint{4.650000in}{3.020000in}}%
\pgfusepath{clip}%
\pgfsetbuttcap%
\pgfsetroundjoin%
\definecolor{currentfill}{rgb}{1.000000,0.498039,0.054902}%
\pgfsetfillcolor{currentfill}%
\pgfsetlinewidth{1.003750pt}%
\definecolor{currentstroke}{rgb}{1.000000,0.498039,0.054902}%
\pgfsetstrokecolor{currentstroke}%
\pgfsetdash{}{0pt}%
\pgfpathmoveto{\pgfqpoint{1.565260in}{2.932163in}}%
\pgfpathcurveto{\pgfqpoint{1.576310in}{2.932163in}}{\pgfqpoint{1.586909in}{2.936553in}}{\pgfqpoint{1.594723in}{2.944367in}}%
\pgfpathcurveto{\pgfqpoint{1.602536in}{2.952181in}}{\pgfqpoint{1.606926in}{2.962780in}}{\pgfqpoint{1.606926in}{2.973830in}}%
\pgfpathcurveto{\pgfqpoint{1.606926in}{2.984880in}}{\pgfqpoint{1.602536in}{2.995479in}}{\pgfqpoint{1.594723in}{3.003293in}}%
\pgfpathcurveto{\pgfqpoint{1.586909in}{3.011106in}}{\pgfqpoint{1.576310in}{3.015496in}}{\pgfqpoint{1.565260in}{3.015496in}}%
\pgfpathcurveto{\pgfqpoint{1.554210in}{3.015496in}}{\pgfqpoint{1.543611in}{3.011106in}}{\pgfqpoint{1.535797in}{3.003293in}}%
\pgfpathcurveto{\pgfqpoint{1.527983in}{2.995479in}}{\pgfqpoint{1.523593in}{2.984880in}}{\pgfqpoint{1.523593in}{2.973830in}}%
\pgfpathcurveto{\pgfqpoint{1.523593in}{2.962780in}}{\pgfqpoint{1.527983in}{2.952181in}}{\pgfqpoint{1.535797in}{2.944367in}}%
\pgfpathcurveto{\pgfqpoint{1.543611in}{2.936553in}}{\pgfqpoint{1.554210in}{2.932163in}}{\pgfqpoint{1.565260in}{2.932163in}}%
\pgfpathclose%
\pgfusepath{stroke,fill}%
\end{pgfscope}%
\begin{pgfscope}%
\pgfpathrectangle{\pgfqpoint{0.750000in}{0.500000in}}{\pgfqpoint{4.650000in}{3.020000in}}%
\pgfusepath{clip}%
\pgfsetbuttcap%
\pgfsetroundjoin%
\definecolor{currentfill}{rgb}{1.000000,0.498039,0.054902}%
\pgfsetfillcolor{currentfill}%
\pgfsetlinewidth{1.003750pt}%
\definecolor{currentstroke}{rgb}{1.000000,0.498039,0.054902}%
\pgfsetstrokecolor{currentstroke}%
\pgfsetdash{}{0pt}%
\pgfpathmoveto{\pgfqpoint{2.048377in}{2.044271in}}%
\pgfpathcurveto{\pgfqpoint{2.059427in}{2.044271in}}{\pgfqpoint{2.070026in}{2.048662in}}{\pgfqpoint{2.077839in}{2.056475in}}%
\pgfpathcurveto{\pgfqpoint{2.085653in}{2.064289in}}{\pgfqpoint{2.090043in}{2.074888in}}{\pgfqpoint{2.090043in}{2.085938in}}%
\pgfpathcurveto{\pgfqpoint{2.090043in}{2.096988in}}{\pgfqpoint{2.085653in}{2.107587in}}{\pgfqpoint{2.077839in}{2.115401in}}%
\pgfpathcurveto{\pgfqpoint{2.070026in}{2.123215in}}{\pgfqpoint{2.059427in}{2.127605in}}{\pgfqpoint{2.048377in}{2.127605in}}%
\pgfpathcurveto{\pgfqpoint{2.037326in}{2.127605in}}{\pgfqpoint{2.026727in}{2.123215in}}{\pgfqpoint{2.018914in}{2.115401in}}%
\pgfpathcurveto{\pgfqpoint{2.011100in}{2.107587in}}{\pgfqpoint{2.006710in}{2.096988in}}{\pgfqpoint{2.006710in}{2.085938in}}%
\pgfpathcurveto{\pgfqpoint{2.006710in}{2.074888in}}{\pgfqpoint{2.011100in}{2.064289in}}{\pgfqpoint{2.018914in}{2.056475in}}%
\pgfpathcurveto{\pgfqpoint{2.026727in}{2.048662in}}{\pgfqpoint{2.037326in}{2.044271in}}{\pgfqpoint{2.048377in}{2.044271in}}%
\pgfpathclose%
\pgfusepath{stroke,fill}%
\end{pgfscope}%
\begin{pgfscope}%
\pgfpathrectangle{\pgfqpoint{0.750000in}{0.500000in}}{\pgfqpoint{4.650000in}{3.020000in}}%
\pgfusepath{clip}%
\pgfsetbuttcap%
\pgfsetroundjoin%
\definecolor{currentfill}{rgb}{0.839216,0.152941,0.156863}%
\pgfsetfillcolor{currentfill}%
\pgfsetlinewidth{1.003750pt}%
\definecolor{currentstroke}{rgb}{0.839216,0.152941,0.156863}%
\pgfsetstrokecolor{currentstroke}%
\pgfsetdash{}{0pt}%
\pgfpathmoveto{\pgfqpoint{1.746429in}{2.936057in}}%
\pgfpathcurveto{\pgfqpoint{1.757479in}{2.936057in}}{\pgfqpoint{1.768078in}{2.940448in}}{\pgfqpoint{1.775891in}{2.948261in}}%
\pgfpathcurveto{\pgfqpoint{1.783705in}{2.956075in}}{\pgfqpoint{1.788095in}{2.966674in}}{\pgfqpoint{1.788095in}{2.977724in}}%
\pgfpathcurveto{\pgfqpoint{1.788095in}{2.988774in}}{\pgfqpoint{1.783705in}{2.999373in}}{\pgfqpoint{1.775891in}{3.007187in}}%
\pgfpathcurveto{\pgfqpoint{1.768078in}{3.015000in}}{\pgfqpoint{1.757479in}{3.019391in}}{\pgfqpoint{1.746429in}{3.019391in}}%
\pgfpathcurveto{\pgfqpoint{1.735378in}{3.019391in}}{\pgfqpoint{1.724779in}{3.015000in}}{\pgfqpoint{1.716966in}{3.007187in}}%
\pgfpathcurveto{\pgfqpoint{1.709152in}{2.999373in}}{\pgfqpoint{1.704762in}{2.988774in}}{\pgfqpoint{1.704762in}{2.977724in}}%
\pgfpathcurveto{\pgfqpoint{1.704762in}{2.966674in}}{\pgfqpoint{1.709152in}{2.956075in}}{\pgfqpoint{1.716966in}{2.948261in}}%
\pgfpathcurveto{\pgfqpoint{1.724779in}{2.940448in}}{\pgfqpoint{1.735378in}{2.936057in}}{\pgfqpoint{1.746429in}{2.936057in}}%
\pgfpathclose%
\pgfusepath{stroke,fill}%
\end{pgfscope}%
\begin{pgfscope}%
\pgfpathrectangle{\pgfqpoint{0.750000in}{0.500000in}}{\pgfqpoint{4.650000in}{3.020000in}}%
\pgfusepath{clip}%
\pgfsetbuttcap%
\pgfsetroundjoin%
\definecolor{currentfill}{rgb}{1.000000,0.498039,0.054902}%
\pgfsetfillcolor{currentfill}%
\pgfsetlinewidth{1.003750pt}%
\definecolor{currentstroke}{rgb}{1.000000,0.498039,0.054902}%
\pgfsetstrokecolor{currentstroke}%
\pgfsetdash{}{0pt}%
\pgfpathmoveto{\pgfqpoint{5.188636in}{2.877643in}}%
\pgfpathcurveto{\pgfqpoint{5.199686in}{2.877643in}}{\pgfqpoint{5.210286in}{2.882034in}}{\pgfqpoint{5.218099in}{2.889847in}}%
\pgfpathcurveto{\pgfqpoint{5.225913in}{2.897661in}}{\pgfqpoint{5.230303in}{2.908260in}}{\pgfqpoint{5.230303in}{2.919310in}}%
\pgfpathcurveto{\pgfqpoint{5.230303in}{2.930360in}}{\pgfqpoint{5.225913in}{2.940959in}}{\pgfqpoint{5.218099in}{2.948773in}}%
\pgfpathcurveto{\pgfqpoint{5.210286in}{2.956587in}}{\pgfqpoint{5.199686in}{2.960977in}}{\pgfqpoint{5.188636in}{2.960977in}}%
\pgfpathcurveto{\pgfqpoint{5.177586in}{2.960977in}}{\pgfqpoint{5.166987in}{2.956587in}}{\pgfqpoint{5.159174in}{2.948773in}}%
\pgfpathcurveto{\pgfqpoint{5.151360in}{2.940959in}}{\pgfqpoint{5.146970in}{2.930360in}}{\pgfqpoint{5.146970in}{2.919310in}}%
\pgfpathcurveto{\pgfqpoint{5.146970in}{2.908260in}}{\pgfqpoint{5.151360in}{2.897661in}}{\pgfqpoint{5.159174in}{2.889847in}}%
\pgfpathcurveto{\pgfqpoint{5.166987in}{2.882034in}}{\pgfqpoint{5.177586in}{2.877643in}}{\pgfqpoint{5.188636in}{2.877643in}}%
\pgfpathclose%
\pgfusepath{stroke,fill}%
\end{pgfscope}%
\begin{pgfscope}%
\pgfpathrectangle{\pgfqpoint{0.750000in}{0.500000in}}{\pgfqpoint{4.650000in}{3.020000in}}%
\pgfusepath{clip}%
\pgfsetbuttcap%
\pgfsetroundjoin%
\definecolor{currentfill}{rgb}{1.000000,0.498039,0.054902}%
\pgfsetfillcolor{currentfill}%
\pgfsetlinewidth{1.003750pt}%
\definecolor{currentstroke}{rgb}{1.000000,0.498039,0.054902}%
\pgfsetstrokecolor{currentstroke}%
\pgfsetdash{}{0pt}%
\pgfpathmoveto{\pgfqpoint{2.229545in}{2.928269in}}%
\pgfpathcurveto{\pgfqpoint{2.240596in}{2.928269in}}{\pgfqpoint{2.251195in}{2.932659in}}{\pgfqpoint{2.259008in}{2.940473in}}%
\pgfpathcurveto{\pgfqpoint{2.266822in}{2.948286in}}{\pgfqpoint{2.271212in}{2.958885in}}{\pgfqpoint{2.271212in}{2.969936in}}%
\pgfpathcurveto{\pgfqpoint{2.271212in}{2.980986in}}{\pgfqpoint{2.266822in}{2.991585in}}{\pgfqpoint{2.259008in}{2.999398in}}%
\pgfpathcurveto{\pgfqpoint{2.251195in}{3.007212in}}{\pgfqpoint{2.240596in}{3.011602in}}{\pgfqpoint{2.229545in}{3.011602in}}%
\pgfpathcurveto{\pgfqpoint{2.218495in}{3.011602in}}{\pgfqpoint{2.207896in}{3.007212in}}{\pgfqpoint{2.200083in}{2.999398in}}%
\pgfpathcurveto{\pgfqpoint{2.192269in}{2.991585in}}{\pgfqpoint{2.187879in}{2.980986in}}{\pgfqpoint{2.187879in}{2.969936in}}%
\pgfpathcurveto{\pgfqpoint{2.187879in}{2.958885in}}{\pgfqpoint{2.192269in}{2.948286in}}{\pgfqpoint{2.200083in}{2.940473in}}%
\pgfpathcurveto{\pgfqpoint{2.207896in}{2.932659in}}{\pgfqpoint{2.218495in}{2.928269in}}{\pgfqpoint{2.229545in}{2.928269in}}%
\pgfpathclose%
\pgfusepath{stroke,fill}%
\end{pgfscope}%
\begin{pgfscope}%
\pgfpathrectangle{\pgfqpoint{0.750000in}{0.500000in}}{\pgfqpoint{4.650000in}{3.020000in}}%
\pgfusepath{clip}%
\pgfsetbuttcap%
\pgfsetroundjoin%
\definecolor{currentfill}{rgb}{1.000000,0.498039,0.054902}%
\pgfsetfillcolor{currentfill}%
\pgfsetlinewidth{1.003750pt}%
\definecolor{currentstroke}{rgb}{1.000000,0.498039,0.054902}%
\pgfsetstrokecolor{currentstroke}%
\pgfsetdash{}{0pt}%
\pgfpathmoveto{\pgfqpoint{1.746429in}{2.936057in}}%
\pgfpathcurveto{\pgfqpoint{1.757479in}{2.936057in}}{\pgfqpoint{1.768078in}{2.940448in}}{\pgfqpoint{1.775891in}{2.948261in}}%
\pgfpathcurveto{\pgfqpoint{1.783705in}{2.956075in}}{\pgfqpoint{1.788095in}{2.966674in}}{\pgfqpoint{1.788095in}{2.977724in}}%
\pgfpathcurveto{\pgfqpoint{1.788095in}{2.988774in}}{\pgfqpoint{1.783705in}{2.999373in}}{\pgfqpoint{1.775891in}{3.007187in}}%
\pgfpathcurveto{\pgfqpoint{1.768078in}{3.015000in}}{\pgfqpoint{1.757479in}{3.019391in}}{\pgfqpoint{1.746429in}{3.019391in}}%
\pgfpathcurveto{\pgfqpoint{1.735378in}{3.019391in}}{\pgfqpoint{1.724779in}{3.015000in}}{\pgfqpoint{1.716966in}{3.007187in}}%
\pgfpathcurveto{\pgfqpoint{1.709152in}{2.999373in}}{\pgfqpoint{1.704762in}{2.988774in}}{\pgfqpoint{1.704762in}{2.977724in}}%
\pgfpathcurveto{\pgfqpoint{1.704762in}{2.966674in}}{\pgfqpoint{1.709152in}{2.956075in}}{\pgfqpoint{1.716966in}{2.948261in}}%
\pgfpathcurveto{\pgfqpoint{1.724779in}{2.940448in}}{\pgfqpoint{1.735378in}{2.936057in}}{\pgfqpoint{1.746429in}{2.936057in}}%
\pgfpathclose%
\pgfusepath{stroke,fill}%
\end{pgfscope}%
\begin{pgfscope}%
\pgfpathrectangle{\pgfqpoint{0.750000in}{0.500000in}}{\pgfqpoint{4.650000in}{3.020000in}}%
\pgfusepath{clip}%
\pgfsetbuttcap%
\pgfsetroundjoin%
\definecolor{currentfill}{rgb}{0.121569,0.466667,0.705882}%
\pgfsetfillcolor{currentfill}%
\pgfsetlinewidth{1.003750pt}%
\definecolor{currentstroke}{rgb}{0.121569,0.466667,0.705882}%
\pgfsetstrokecolor{currentstroke}%
\pgfsetdash{}{0pt}%
\pgfpathmoveto{\pgfqpoint{1.021753in}{0.595606in}}%
\pgfpathcurveto{\pgfqpoint{1.032803in}{0.595606in}}{\pgfqpoint{1.043402in}{0.599996in}}{\pgfqpoint{1.051216in}{0.607810in}}%
\pgfpathcurveto{\pgfqpoint{1.059030in}{0.615624in}}{\pgfqpoint{1.063420in}{0.626223in}}{\pgfqpoint{1.063420in}{0.637273in}}%
\pgfpathcurveto{\pgfqpoint{1.063420in}{0.648323in}}{\pgfqpoint{1.059030in}{0.658922in}}{\pgfqpoint{1.051216in}{0.666736in}}%
\pgfpathcurveto{\pgfqpoint{1.043402in}{0.674549in}}{\pgfqpoint{1.032803in}{0.678939in}}{\pgfqpoint{1.021753in}{0.678939in}}%
\pgfpathcurveto{\pgfqpoint{1.010703in}{0.678939in}}{\pgfqpoint{1.000104in}{0.674549in}}{\pgfqpoint{0.992290in}{0.666736in}}%
\pgfpathcurveto{\pgfqpoint{0.984477in}{0.658922in}}{\pgfqpoint{0.980087in}{0.648323in}}{\pgfqpoint{0.980087in}{0.637273in}}%
\pgfpathcurveto{\pgfqpoint{0.980087in}{0.626223in}}{\pgfqpoint{0.984477in}{0.615624in}}{\pgfqpoint{0.992290in}{0.607810in}}%
\pgfpathcurveto{\pgfqpoint{1.000104in}{0.599996in}}{\pgfqpoint{1.010703in}{0.595606in}}{\pgfqpoint{1.021753in}{0.595606in}}%
\pgfpathclose%
\pgfusepath{stroke,fill}%
\end{pgfscope}%
\begin{pgfscope}%
\pgfpathrectangle{\pgfqpoint{0.750000in}{0.500000in}}{\pgfqpoint{4.650000in}{3.020000in}}%
\pgfusepath{clip}%
\pgfsetbuttcap%
\pgfsetroundjoin%
\definecolor{currentfill}{rgb}{1.000000,0.498039,0.054902}%
\pgfsetfillcolor{currentfill}%
\pgfsetlinewidth{1.003750pt}%
\definecolor{currentstroke}{rgb}{1.000000,0.498039,0.054902}%
\pgfsetstrokecolor{currentstroke}%
\pgfsetdash{}{0pt}%
\pgfpathmoveto{\pgfqpoint{1.444481in}{2.932163in}}%
\pgfpathcurveto{\pgfqpoint{1.455531in}{2.932163in}}{\pgfqpoint{1.466130in}{2.936553in}}{\pgfqpoint{1.473943in}{2.944367in}}%
\pgfpathcurveto{\pgfqpoint{1.481757in}{2.952181in}}{\pgfqpoint{1.486147in}{2.962780in}}{\pgfqpoint{1.486147in}{2.973830in}}%
\pgfpathcurveto{\pgfqpoint{1.486147in}{2.984880in}}{\pgfqpoint{1.481757in}{2.995479in}}{\pgfqpoint{1.473943in}{3.003293in}}%
\pgfpathcurveto{\pgfqpoint{1.466130in}{3.011106in}}{\pgfqpoint{1.455531in}{3.015496in}}{\pgfqpoint{1.444481in}{3.015496in}}%
\pgfpathcurveto{\pgfqpoint{1.433430in}{3.015496in}}{\pgfqpoint{1.422831in}{3.011106in}}{\pgfqpoint{1.415018in}{3.003293in}}%
\pgfpathcurveto{\pgfqpoint{1.407204in}{2.995479in}}{\pgfqpoint{1.402814in}{2.984880in}}{\pgfqpoint{1.402814in}{2.973830in}}%
\pgfpathcurveto{\pgfqpoint{1.402814in}{2.962780in}}{\pgfqpoint{1.407204in}{2.952181in}}{\pgfqpoint{1.415018in}{2.944367in}}%
\pgfpathcurveto{\pgfqpoint{1.422831in}{2.936553in}}{\pgfqpoint{1.433430in}{2.932163in}}{\pgfqpoint{1.444481in}{2.932163in}}%
\pgfpathclose%
\pgfusepath{stroke,fill}%
\end{pgfscope}%
\begin{pgfscope}%
\pgfpathrectangle{\pgfqpoint{0.750000in}{0.500000in}}{\pgfqpoint{4.650000in}{3.020000in}}%
\pgfusepath{clip}%
\pgfsetbuttcap%
\pgfsetroundjoin%
\definecolor{currentfill}{rgb}{0.121569,0.466667,0.705882}%
\pgfsetfillcolor{currentfill}%
\pgfsetlinewidth{1.003750pt}%
\definecolor{currentstroke}{rgb}{0.121569,0.466667,0.705882}%
\pgfsetstrokecolor{currentstroke}%
\pgfsetdash{}{0pt}%
\pgfpathmoveto{\pgfqpoint{1.625649in}{1.129120in}}%
\pgfpathcurveto{\pgfqpoint{1.636699in}{1.129120in}}{\pgfqpoint{1.647299in}{1.133510in}}{\pgfqpoint{1.655112in}{1.141324in}}%
\pgfpathcurveto{\pgfqpoint{1.662926in}{1.149137in}}{\pgfqpoint{1.667316in}{1.159736in}}{\pgfqpoint{1.667316in}{1.170787in}}%
\pgfpathcurveto{\pgfqpoint{1.667316in}{1.181837in}}{\pgfqpoint{1.662926in}{1.192436in}}{\pgfqpoint{1.655112in}{1.200249in}}%
\pgfpathcurveto{\pgfqpoint{1.647299in}{1.208063in}}{\pgfqpoint{1.636699in}{1.212453in}}{\pgfqpoint{1.625649in}{1.212453in}}%
\pgfpathcurveto{\pgfqpoint{1.614599in}{1.212453in}}{\pgfqpoint{1.604000in}{1.208063in}}{\pgfqpoint{1.596187in}{1.200249in}}%
\pgfpathcurveto{\pgfqpoint{1.588373in}{1.192436in}}{\pgfqpoint{1.583983in}{1.181837in}}{\pgfqpoint{1.583983in}{1.170787in}}%
\pgfpathcurveto{\pgfqpoint{1.583983in}{1.159736in}}{\pgfqpoint{1.588373in}{1.149137in}}{\pgfqpoint{1.596187in}{1.141324in}}%
\pgfpathcurveto{\pgfqpoint{1.604000in}{1.133510in}}{\pgfqpoint{1.614599in}{1.129120in}}{\pgfqpoint{1.625649in}{1.129120in}}%
\pgfpathclose%
\pgfusepath{stroke,fill}%
\end{pgfscope}%
\begin{pgfscope}%
\pgfpathrectangle{\pgfqpoint{0.750000in}{0.500000in}}{\pgfqpoint{4.650000in}{3.020000in}}%
\pgfusepath{clip}%
\pgfsetbuttcap%
\pgfsetroundjoin%
\definecolor{currentfill}{rgb}{1.000000,0.498039,0.054902}%
\pgfsetfillcolor{currentfill}%
\pgfsetlinewidth{1.003750pt}%
\definecolor{currentstroke}{rgb}{1.000000,0.498039,0.054902}%
\pgfsetstrokecolor{currentstroke}%
\pgfsetdash{}{0pt}%
\pgfpathmoveto{\pgfqpoint{1.444481in}{2.932163in}}%
\pgfpathcurveto{\pgfqpoint{1.455531in}{2.932163in}}{\pgfqpoint{1.466130in}{2.936553in}}{\pgfqpoint{1.473943in}{2.944367in}}%
\pgfpathcurveto{\pgfqpoint{1.481757in}{2.952181in}}{\pgfqpoint{1.486147in}{2.962780in}}{\pgfqpoint{1.486147in}{2.973830in}}%
\pgfpathcurveto{\pgfqpoint{1.486147in}{2.984880in}}{\pgfqpoint{1.481757in}{2.995479in}}{\pgfqpoint{1.473943in}{3.003293in}}%
\pgfpathcurveto{\pgfqpoint{1.466130in}{3.011106in}}{\pgfqpoint{1.455531in}{3.015496in}}{\pgfqpoint{1.444481in}{3.015496in}}%
\pgfpathcurveto{\pgfqpoint{1.433430in}{3.015496in}}{\pgfqpoint{1.422831in}{3.011106in}}{\pgfqpoint{1.415018in}{3.003293in}}%
\pgfpathcurveto{\pgfqpoint{1.407204in}{2.995479in}}{\pgfqpoint{1.402814in}{2.984880in}}{\pgfqpoint{1.402814in}{2.973830in}}%
\pgfpathcurveto{\pgfqpoint{1.402814in}{2.962780in}}{\pgfqpoint{1.407204in}{2.952181in}}{\pgfqpoint{1.415018in}{2.944367in}}%
\pgfpathcurveto{\pgfqpoint{1.422831in}{2.936553in}}{\pgfqpoint{1.433430in}{2.932163in}}{\pgfqpoint{1.444481in}{2.932163in}}%
\pgfpathclose%
\pgfusepath{stroke,fill}%
\end{pgfscope}%
\begin{pgfscope}%
\pgfpathrectangle{\pgfqpoint{0.750000in}{0.500000in}}{\pgfqpoint{4.650000in}{3.020000in}}%
\pgfusepath{clip}%
\pgfsetbuttcap%
\pgfsetroundjoin%
\definecolor{currentfill}{rgb}{1.000000,0.498039,0.054902}%
\pgfsetfillcolor{currentfill}%
\pgfsetlinewidth{1.003750pt}%
\definecolor{currentstroke}{rgb}{1.000000,0.498039,0.054902}%
\pgfsetstrokecolor{currentstroke}%
\pgfsetdash{}{0pt}%
\pgfpathmoveto{\pgfqpoint{2.229545in}{2.936057in}}%
\pgfpathcurveto{\pgfqpoint{2.240596in}{2.936057in}}{\pgfqpoint{2.251195in}{2.940448in}}{\pgfqpoint{2.259008in}{2.948261in}}%
\pgfpathcurveto{\pgfqpoint{2.266822in}{2.956075in}}{\pgfqpoint{2.271212in}{2.966674in}}{\pgfqpoint{2.271212in}{2.977724in}}%
\pgfpathcurveto{\pgfqpoint{2.271212in}{2.988774in}}{\pgfqpoint{2.266822in}{2.999373in}}{\pgfqpoint{2.259008in}{3.007187in}}%
\pgfpathcurveto{\pgfqpoint{2.251195in}{3.015000in}}{\pgfqpoint{2.240596in}{3.019391in}}{\pgfqpoint{2.229545in}{3.019391in}}%
\pgfpathcurveto{\pgfqpoint{2.218495in}{3.019391in}}{\pgfqpoint{2.207896in}{3.015000in}}{\pgfqpoint{2.200083in}{3.007187in}}%
\pgfpathcurveto{\pgfqpoint{2.192269in}{2.999373in}}{\pgfqpoint{2.187879in}{2.988774in}}{\pgfqpoint{2.187879in}{2.977724in}}%
\pgfpathcurveto{\pgfqpoint{2.187879in}{2.966674in}}{\pgfqpoint{2.192269in}{2.956075in}}{\pgfqpoint{2.200083in}{2.948261in}}%
\pgfpathcurveto{\pgfqpoint{2.207896in}{2.940448in}}{\pgfqpoint{2.218495in}{2.936057in}}{\pgfqpoint{2.229545in}{2.936057in}}%
\pgfpathclose%
\pgfusepath{stroke,fill}%
\end{pgfscope}%
\begin{pgfscope}%
\pgfpathrectangle{\pgfqpoint{0.750000in}{0.500000in}}{\pgfqpoint{4.650000in}{3.020000in}}%
\pgfusepath{clip}%
\pgfsetbuttcap%
\pgfsetroundjoin%
\definecolor{currentfill}{rgb}{1.000000,0.498039,0.054902}%
\pgfsetfillcolor{currentfill}%
\pgfsetlinewidth{1.003750pt}%
\definecolor{currentstroke}{rgb}{1.000000,0.498039,0.054902}%
\pgfsetstrokecolor{currentstroke}%
\pgfsetdash{}{0pt}%
\pgfpathmoveto{\pgfqpoint{2.893831in}{2.924375in}}%
\pgfpathcurveto{\pgfqpoint{2.904881in}{2.924375in}}{\pgfqpoint{2.915480in}{2.928765in}}{\pgfqpoint{2.923294in}{2.936578in}}%
\pgfpathcurveto{\pgfqpoint{2.931108in}{2.944392in}}{\pgfqpoint{2.935498in}{2.954991in}}{\pgfqpoint{2.935498in}{2.966041in}}%
\pgfpathcurveto{\pgfqpoint{2.935498in}{2.977091in}}{\pgfqpoint{2.931108in}{2.987690in}}{\pgfqpoint{2.923294in}{2.995504in}}%
\pgfpathcurveto{\pgfqpoint{2.915480in}{3.003318in}}{\pgfqpoint{2.904881in}{3.007708in}}{\pgfqpoint{2.893831in}{3.007708in}}%
\pgfpathcurveto{\pgfqpoint{2.882781in}{3.007708in}}{\pgfqpoint{2.872182in}{3.003318in}}{\pgfqpoint{2.864368in}{2.995504in}}%
\pgfpathcurveto{\pgfqpoint{2.856555in}{2.987690in}}{\pgfqpoint{2.852165in}{2.977091in}}{\pgfqpoint{2.852165in}{2.966041in}}%
\pgfpathcurveto{\pgfqpoint{2.852165in}{2.954991in}}{\pgfqpoint{2.856555in}{2.944392in}}{\pgfqpoint{2.864368in}{2.936578in}}%
\pgfpathcurveto{\pgfqpoint{2.872182in}{2.928765in}}{\pgfqpoint{2.882781in}{2.924375in}}{\pgfqpoint{2.893831in}{2.924375in}}%
\pgfpathclose%
\pgfusepath{stroke,fill}%
\end{pgfscope}%
\begin{pgfscope}%
\pgfpathrectangle{\pgfqpoint{0.750000in}{0.500000in}}{\pgfqpoint{4.650000in}{3.020000in}}%
\pgfusepath{clip}%
\pgfsetbuttcap%
\pgfsetroundjoin%
\definecolor{currentfill}{rgb}{1.000000,0.498039,0.054902}%
\pgfsetfillcolor{currentfill}%
\pgfsetlinewidth{1.003750pt}%
\definecolor{currentstroke}{rgb}{1.000000,0.498039,0.054902}%
\pgfsetstrokecolor{currentstroke}%
\pgfsetdash{}{0pt}%
\pgfpathmoveto{\pgfqpoint{3.618506in}{2.928269in}}%
\pgfpathcurveto{\pgfqpoint{3.629557in}{2.928269in}}{\pgfqpoint{3.640156in}{2.932659in}}{\pgfqpoint{3.647969in}{2.940473in}}%
\pgfpathcurveto{\pgfqpoint{3.655783in}{2.948286in}}{\pgfqpoint{3.660173in}{2.958885in}}{\pgfqpoint{3.660173in}{2.969936in}}%
\pgfpathcurveto{\pgfqpoint{3.660173in}{2.980986in}}{\pgfqpoint{3.655783in}{2.991585in}}{\pgfqpoint{3.647969in}{2.999398in}}%
\pgfpathcurveto{\pgfqpoint{3.640156in}{3.007212in}}{\pgfqpoint{3.629557in}{3.011602in}}{\pgfqpoint{3.618506in}{3.011602in}}%
\pgfpathcurveto{\pgfqpoint{3.607456in}{3.011602in}}{\pgfqpoint{3.596857in}{3.007212in}}{\pgfqpoint{3.589044in}{2.999398in}}%
\pgfpathcurveto{\pgfqpoint{3.581230in}{2.991585in}}{\pgfqpoint{3.576840in}{2.980986in}}{\pgfqpoint{3.576840in}{2.969936in}}%
\pgfpathcurveto{\pgfqpoint{3.576840in}{2.958885in}}{\pgfqpoint{3.581230in}{2.948286in}}{\pgfqpoint{3.589044in}{2.940473in}}%
\pgfpathcurveto{\pgfqpoint{3.596857in}{2.932659in}}{\pgfqpoint{3.607456in}{2.928269in}}{\pgfqpoint{3.618506in}{2.928269in}}%
\pgfpathclose%
\pgfusepath{stroke,fill}%
\end{pgfscope}%
\begin{pgfscope}%
\pgfpathrectangle{\pgfqpoint{0.750000in}{0.500000in}}{\pgfqpoint{4.650000in}{3.020000in}}%
\pgfusepath{clip}%
\pgfsetbuttcap%
\pgfsetroundjoin%
\definecolor{currentfill}{rgb}{1.000000,0.498039,0.054902}%
\pgfsetfillcolor{currentfill}%
\pgfsetlinewidth{1.003750pt}%
\definecolor{currentstroke}{rgb}{1.000000,0.498039,0.054902}%
\pgfsetstrokecolor{currentstroke}%
\pgfsetdash{}{0pt}%
\pgfpathmoveto{\pgfqpoint{1.987987in}{2.939952in}}%
\pgfpathcurveto{\pgfqpoint{1.999037in}{2.939952in}}{\pgfqpoint{2.009636in}{2.944342in}}{\pgfqpoint{2.017450in}{2.952156in}}%
\pgfpathcurveto{\pgfqpoint{2.025263in}{2.959969in}}{\pgfqpoint{2.029654in}{2.970568in}}{\pgfqpoint{2.029654in}{2.981618in}}%
\pgfpathcurveto{\pgfqpoint{2.029654in}{2.992668in}}{\pgfqpoint{2.025263in}{3.003267in}}{\pgfqpoint{2.017450in}{3.011081in}}%
\pgfpathcurveto{\pgfqpoint{2.009636in}{3.018895in}}{\pgfqpoint{1.999037in}{3.023285in}}{\pgfqpoint{1.987987in}{3.023285in}}%
\pgfpathcurveto{\pgfqpoint{1.976937in}{3.023285in}}{\pgfqpoint{1.966338in}{3.018895in}}{\pgfqpoint{1.958524in}{3.011081in}}%
\pgfpathcurveto{\pgfqpoint{1.950711in}{3.003267in}}{\pgfqpoint{1.946320in}{2.992668in}}{\pgfqpoint{1.946320in}{2.981618in}}%
\pgfpathcurveto{\pgfqpoint{1.946320in}{2.970568in}}{\pgfqpoint{1.950711in}{2.959969in}}{\pgfqpoint{1.958524in}{2.952156in}}%
\pgfpathcurveto{\pgfqpoint{1.966338in}{2.944342in}}{\pgfqpoint{1.976937in}{2.939952in}}{\pgfqpoint{1.987987in}{2.939952in}}%
\pgfpathclose%
\pgfusepath{stroke,fill}%
\end{pgfscope}%
\begin{pgfscope}%
\pgfpathrectangle{\pgfqpoint{0.750000in}{0.500000in}}{\pgfqpoint{4.650000in}{3.020000in}}%
\pgfusepath{clip}%
\pgfsetbuttcap%
\pgfsetroundjoin%
\definecolor{currentfill}{rgb}{1.000000,0.498039,0.054902}%
\pgfsetfillcolor{currentfill}%
\pgfsetlinewidth{1.003750pt}%
\definecolor{currentstroke}{rgb}{1.000000,0.498039,0.054902}%
\pgfsetstrokecolor{currentstroke}%
\pgfsetdash{}{0pt}%
\pgfpathmoveto{\pgfqpoint{1.686039in}{2.928269in}}%
\pgfpathcurveto{\pgfqpoint{1.697089in}{2.928269in}}{\pgfqpoint{1.707688in}{2.932659in}}{\pgfqpoint{1.715502in}{2.940473in}}%
\pgfpathcurveto{\pgfqpoint{1.723315in}{2.948286in}}{\pgfqpoint{1.727706in}{2.958885in}}{\pgfqpoint{1.727706in}{2.969936in}}%
\pgfpathcurveto{\pgfqpoint{1.727706in}{2.980986in}}{\pgfqpoint{1.723315in}{2.991585in}}{\pgfqpoint{1.715502in}{2.999398in}}%
\pgfpathcurveto{\pgfqpoint{1.707688in}{3.007212in}}{\pgfqpoint{1.697089in}{3.011602in}}{\pgfqpoint{1.686039in}{3.011602in}}%
\pgfpathcurveto{\pgfqpoint{1.674989in}{3.011602in}}{\pgfqpoint{1.664390in}{3.007212in}}{\pgfqpoint{1.656576in}{2.999398in}}%
\pgfpathcurveto{\pgfqpoint{1.648763in}{2.991585in}}{\pgfqpoint{1.644372in}{2.980986in}}{\pgfqpoint{1.644372in}{2.969936in}}%
\pgfpathcurveto{\pgfqpoint{1.644372in}{2.958885in}}{\pgfqpoint{1.648763in}{2.948286in}}{\pgfqpoint{1.656576in}{2.940473in}}%
\pgfpathcurveto{\pgfqpoint{1.664390in}{2.932659in}}{\pgfqpoint{1.674989in}{2.928269in}}{\pgfqpoint{1.686039in}{2.928269in}}%
\pgfpathclose%
\pgfusepath{stroke,fill}%
\end{pgfscope}%
\begin{pgfscope}%
\pgfpathrectangle{\pgfqpoint{0.750000in}{0.500000in}}{\pgfqpoint{4.650000in}{3.020000in}}%
\pgfusepath{clip}%
\pgfsetbuttcap%
\pgfsetroundjoin%
\definecolor{currentfill}{rgb}{1.000000,0.498039,0.054902}%
\pgfsetfillcolor{currentfill}%
\pgfsetlinewidth{1.003750pt}%
\definecolor{currentstroke}{rgb}{1.000000,0.498039,0.054902}%
\pgfsetstrokecolor{currentstroke}%
\pgfsetdash{}{0pt}%
\pgfpathmoveto{\pgfqpoint{2.169156in}{2.959423in}}%
\pgfpathcurveto{\pgfqpoint{2.180206in}{2.959423in}}{\pgfqpoint{2.190805in}{2.963813in}}{\pgfqpoint{2.198619in}{2.971627in}}%
\pgfpathcurveto{\pgfqpoint{2.206432in}{2.979440in}}{\pgfqpoint{2.210823in}{2.990039in}}{\pgfqpoint{2.210823in}{3.001090in}}%
\pgfpathcurveto{\pgfqpoint{2.210823in}{3.012140in}}{\pgfqpoint{2.206432in}{3.022739in}}{\pgfqpoint{2.198619in}{3.030552in}}%
\pgfpathcurveto{\pgfqpoint{2.190805in}{3.038366in}}{\pgfqpoint{2.180206in}{3.042756in}}{\pgfqpoint{2.169156in}{3.042756in}}%
\pgfpathcurveto{\pgfqpoint{2.158106in}{3.042756in}}{\pgfqpoint{2.147507in}{3.038366in}}{\pgfqpoint{2.139693in}{3.030552in}}%
\pgfpathcurveto{\pgfqpoint{2.131879in}{3.022739in}}{\pgfqpoint{2.127489in}{3.012140in}}{\pgfqpoint{2.127489in}{3.001090in}}%
\pgfpathcurveto{\pgfqpoint{2.127489in}{2.990039in}}{\pgfqpoint{2.131879in}{2.979440in}}{\pgfqpoint{2.139693in}{2.971627in}}%
\pgfpathcurveto{\pgfqpoint{2.147507in}{2.963813in}}{\pgfqpoint{2.158106in}{2.959423in}}{\pgfqpoint{2.169156in}{2.959423in}}%
\pgfpathclose%
\pgfusepath{stroke,fill}%
\end{pgfscope}%
\begin{pgfscope}%
\pgfpathrectangle{\pgfqpoint{0.750000in}{0.500000in}}{\pgfqpoint{4.650000in}{3.020000in}}%
\pgfusepath{clip}%
\pgfsetbuttcap%
\pgfsetroundjoin%
\definecolor{currentfill}{rgb}{1.000000,0.498039,0.054902}%
\pgfsetfillcolor{currentfill}%
\pgfsetlinewidth{1.003750pt}%
\definecolor{currentstroke}{rgb}{1.000000,0.498039,0.054902}%
\pgfsetstrokecolor{currentstroke}%
\pgfsetdash{}{0pt}%
\pgfpathmoveto{\pgfqpoint{2.471104in}{2.939952in}}%
\pgfpathcurveto{\pgfqpoint{2.482154in}{2.939952in}}{\pgfqpoint{2.492753in}{2.944342in}}{\pgfqpoint{2.500567in}{2.952156in}}%
\pgfpathcurveto{\pgfqpoint{2.508380in}{2.959969in}}{\pgfqpoint{2.512771in}{2.970568in}}{\pgfqpoint{2.512771in}{2.981618in}}%
\pgfpathcurveto{\pgfqpoint{2.512771in}{2.992668in}}{\pgfqpoint{2.508380in}{3.003267in}}{\pgfqpoint{2.500567in}{3.011081in}}%
\pgfpathcurveto{\pgfqpoint{2.492753in}{3.018895in}}{\pgfqpoint{2.482154in}{3.023285in}}{\pgfqpoint{2.471104in}{3.023285in}}%
\pgfpathcurveto{\pgfqpoint{2.460054in}{3.023285in}}{\pgfqpoint{2.449455in}{3.018895in}}{\pgfqpoint{2.441641in}{3.011081in}}%
\pgfpathcurveto{\pgfqpoint{2.433827in}{3.003267in}}{\pgfqpoint{2.429437in}{2.992668in}}{\pgfqpoint{2.429437in}{2.981618in}}%
\pgfpathcurveto{\pgfqpoint{2.429437in}{2.970568in}}{\pgfqpoint{2.433827in}{2.959969in}}{\pgfqpoint{2.441641in}{2.952156in}}%
\pgfpathcurveto{\pgfqpoint{2.449455in}{2.944342in}}{\pgfqpoint{2.460054in}{2.939952in}}{\pgfqpoint{2.471104in}{2.939952in}}%
\pgfpathclose%
\pgfusepath{stroke,fill}%
\end{pgfscope}%
\begin{pgfscope}%
\pgfpathrectangle{\pgfqpoint{0.750000in}{0.500000in}}{\pgfqpoint{4.650000in}{3.020000in}}%
\pgfusepath{clip}%
\pgfsetbuttcap%
\pgfsetroundjoin%
\definecolor{currentfill}{rgb}{1.000000,0.498039,0.054902}%
\pgfsetfillcolor{currentfill}%
\pgfsetlinewidth{1.003750pt}%
\definecolor{currentstroke}{rgb}{1.000000,0.498039,0.054902}%
\pgfsetstrokecolor{currentstroke}%
\pgfsetdash{}{0pt}%
\pgfpathmoveto{\pgfqpoint{1.323701in}{2.932163in}}%
\pgfpathcurveto{\pgfqpoint{1.334751in}{2.932163in}}{\pgfqpoint{1.345350in}{2.936553in}}{\pgfqpoint{1.353164in}{2.944367in}}%
\pgfpathcurveto{\pgfqpoint{1.360978in}{2.952181in}}{\pgfqpoint{1.365368in}{2.962780in}}{\pgfqpoint{1.365368in}{2.973830in}}%
\pgfpathcurveto{\pgfqpoint{1.365368in}{2.984880in}}{\pgfqpoint{1.360978in}{2.995479in}}{\pgfqpoint{1.353164in}{3.003293in}}%
\pgfpathcurveto{\pgfqpoint{1.345350in}{3.011106in}}{\pgfqpoint{1.334751in}{3.015496in}}{\pgfqpoint{1.323701in}{3.015496in}}%
\pgfpathcurveto{\pgfqpoint{1.312651in}{3.015496in}}{\pgfqpoint{1.302052in}{3.011106in}}{\pgfqpoint{1.294239in}{3.003293in}}%
\pgfpathcurveto{\pgfqpoint{1.286425in}{2.995479in}}{\pgfqpoint{1.282035in}{2.984880in}}{\pgfqpoint{1.282035in}{2.973830in}}%
\pgfpathcurveto{\pgfqpoint{1.282035in}{2.962780in}}{\pgfqpoint{1.286425in}{2.952181in}}{\pgfqpoint{1.294239in}{2.944367in}}%
\pgfpathcurveto{\pgfqpoint{1.302052in}{2.936553in}}{\pgfqpoint{1.312651in}{2.932163in}}{\pgfqpoint{1.323701in}{2.932163in}}%
\pgfpathclose%
\pgfusepath{stroke,fill}%
\end{pgfscope}%
\begin{pgfscope}%
\pgfpathrectangle{\pgfqpoint{0.750000in}{0.500000in}}{\pgfqpoint{4.650000in}{3.020000in}}%
\pgfusepath{clip}%
\pgfsetbuttcap%
\pgfsetroundjoin%
\definecolor{currentfill}{rgb}{1.000000,0.498039,0.054902}%
\pgfsetfillcolor{currentfill}%
\pgfsetlinewidth{1.003750pt}%
\definecolor{currentstroke}{rgb}{1.000000,0.498039,0.054902}%
\pgfsetstrokecolor{currentstroke}%
\pgfsetdash{}{0pt}%
\pgfpathmoveto{\pgfqpoint{1.504870in}{2.936057in}}%
\pgfpathcurveto{\pgfqpoint{1.515920in}{2.936057in}}{\pgfqpoint{1.526519in}{2.940448in}}{\pgfqpoint{1.534333in}{2.948261in}}%
\pgfpathcurveto{\pgfqpoint{1.542147in}{2.956075in}}{\pgfqpoint{1.546537in}{2.966674in}}{\pgfqpoint{1.546537in}{2.977724in}}%
\pgfpathcurveto{\pgfqpoint{1.546537in}{2.988774in}}{\pgfqpoint{1.542147in}{2.999373in}}{\pgfqpoint{1.534333in}{3.007187in}}%
\pgfpathcurveto{\pgfqpoint{1.526519in}{3.015000in}}{\pgfqpoint{1.515920in}{3.019391in}}{\pgfqpoint{1.504870in}{3.019391in}}%
\pgfpathcurveto{\pgfqpoint{1.493820in}{3.019391in}}{\pgfqpoint{1.483221in}{3.015000in}}{\pgfqpoint{1.475407in}{3.007187in}}%
\pgfpathcurveto{\pgfqpoint{1.467594in}{2.999373in}}{\pgfqpoint{1.463203in}{2.988774in}}{\pgfqpoint{1.463203in}{2.977724in}}%
\pgfpathcurveto{\pgfqpoint{1.463203in}{2.966674in}}{\pgfqpoint{1.467594in}{2.956075in}}{\pgfqpoint{1.475407in}{2.948261in}}%
\pgfpathcurveto{\pgfqpoint{1.483221in}{2.940448in}}{\pgfqpoint{1.493820in}{2.936057in}}{\pgfqpoint{1.504870in}{2.936057in}}%
\pgfpathclose%
\pgfusepath{stroke,fill}%
\end{pgfscope}%
\begin{pgfscope}%
\pgfpathrectangle{\pgfqpoint{0.750000in}{0.500000in}}{\pgfqpoint{4.650000in}{3.020000in}}%
\pgfusepath{clip}%
\pgfsetbuttcap%
\pgfsetroundjoin%
\definecolor{currentfill}{rgb}{0.121569,0.466667,0.705882}%
\pgfsetfillcolor{currentfill}%
\pgfsetlinewidth{1.003750pt}%
\definecolor{currentstroke}{rgb}{0.121569,0.466667,0.705882}%
\pgfsetstrokecolor{currentstroke}%
\pgfsetdash{}{0pt}%
\pgfpathmoveto{\pgfqpoint{1.202922in}{0.603395in}}%
\pgfpathcurveto{\pgfqpoint{1.213972in}{0.603395in}}{\pgfqpoint{1.224571in}{0.607785in}}{\pgfqpoint{1.232385in}{0.615598in}}%
\pgfpathcurveto{\pgfqpoint{1.240198in}{0.623412in}}{\pgfqpoint{1.244589in}{0.634011in}}{\pgfqpoint{1.244589in}{0.645061in}}%
\pgfpathcurveto{\pgfqpoint{1.244589in}{0.656111in}}{\pgfqpoint{1.240198in}{0.666710in}}{\pgfqpoint{1.232385in}{0.674524in}}%
\pgfpathcurveto{\pgfqpoint{1.224571in}{0.682338in}}{\pgfqpoint{1.213972in}{0.686728in}}{\pgfqpoint{1.202922in}{0.686728in}}%
\pgfpathcurveto{\pgfqpoint{1.191872in}{0.686728in}}{\pgfqpoint{1.181273in}{0.682338in}}{\pgfqpoint{1.173459in}{0.674524in}}%
\pgfpathcurveto{\pgfqpoint{1.165646in}{0.666710in}}{\pgfqpoint{1.161255in}{0.656111in}}{\pgfqpoint{1.161255in}{0.645061in}}%
\pgfpathcurveto{\pgfqpoint{1.161255in}{0.634011in}}{\pgfqpoint{1.165646in}{0.623412in}}{\pgfqpoint{1.173459in}{0.615598in}}%
\pgfpathcurveto{\pgfqpoint{1.181273in}{0.607785in}}{\pgfqpoint{1.191872in}{0.603395in}}{\pgfqpoint{1.202922in}{0.603395in}}%
\pgfpathclose%
\pgfusepath{stroke,fill}%
\end{pgfscope}%
\begin{pgfscope}%
\pgfpathrectangle{\pgfqpoint{0.750000in}{0.500000in}}{\pgfqpoint{4.650000in}{3.020000in}}%
\pgfusepath{clip}%
\pgfsetbuttcap%
\pgfsetroundjoin%
\definecolor{currentfill}{rgb}{1.000000,0.498039,0.054902}%
\pgfsetfillcolor{currentfill}%
\pgfsetlinewidth{1.003750pt}%
\definecolor{currentstroke}{rgb}{1.000000,0.498039,0.054902}%
\pgfsetstrokecolor{currentstroke}%
\pgfsetdash{}{0pt}%
\pgfpathmoveto{\pgfqpoint{1.625649in}{2.924375in}}%
\pgfpathcurveto{\pgfqpoint{1.636699in}{2.924375in}}{\pgfqpoint{1.647299in}{2.928765in}}{\pgfqpoint{1.655112in}{2.936578in}}%
\pgfpathcurveto{\pgfqpoint{1.662926in}{2.944392in}}{\pgfqpoint{1.667316in}{2.954991in}}{\pgfqpoint{1.667316in}{2.966041in}}%
\pgfpathcurveto{\pgfqpoint{1.667316in}{2.977091in}}{\pgfqpoint{1.662926in}{2.987690in}}{\pgfqpoint{1.655112in}{2.995504in}}%
\pgfpathcurveto{\pgfqpoint{1.647299in}{3.003318in}}{\pgfqpoint{1.636699in}{3.007708in}}{\pgfqpoint{1.625649in}{3.007708in}}%
\pgfpathcurveto{\pgfqpoint{1.614599in}{3.007708in}}{\pgfqpoint{1.604000in}{3.003318in}}{\pgfqpoint{1.596187in}{2.995504in}}%
\pgfpathcurveto{\pgfqpoint{1.588373in}{2.987690in}}{\pgfqpoint{1.583983in}{2.977091in}}{\pgfqpoint{1.583983in}{2.966041in}}%
\pgfpathcurveto{\pgfqpoint{1.583983in}{2.954991in}}{\pgfqpoint{1.588373in}{2.944392in}}{\pgfqpoint{1.596187in}{2.936578in}}%
\pgfpathcurveto{\pgfqpoint{1.604000in}{2.928765in}}{\pgfqpoint{1.614599in}{2.924375in}}{\pgfqpoint{1.625649in}{2.924375in}}%
\pgfpathclose%
\pgfusepath{stroke,fill}%
\end{pgfscope}%
\begin{pgfscope}%
\pgfpathrectangle{\pgfqpoint{0.750000in}{0.500000in}}{\pgfqpoint{4.650000in}{3.020000in}}%
\pgfusepath{clip}%
\pgfsetbuttcap%
\pgfsetroundjoin%
\definecolor{currentfill}{rgb}{0.121569,0.466667,0.705882}%
\pgfsetfillcolor{currentfill}%
\pgfsetlinewidth{1.003750pt}%
\definecolor{currentstroke}{rgb}{0.121569,0.466667,0.705882}%
\pgfsetstrokecolor{currentstroke}%
\pgfsetdash{}{0pt}%
\pgfpathmoveto{\pgfqpoint{1.202922in}{0.595606in}}%
\pgfpathcurveto{\pgfqpoint{1.213972in}{0.595606in}}{\pgfqpoint{1.224571in}{0.599996in}}{\pgfqpoint{1.232385in}{0.607810in}}%
\pgfpathcurveto{\pgfqpoint{1.240198in}{0.615624in}}{\pgfqpoint{1.244589in}{0.626223in}}{\pgfqpoint{1.244589in}{0.637273in}}%
\pgfpathcurveto{\pgfqpoint{1.244589in}{0.648323in}}{\pgfqpoint{1.240198in}{0.658922in}}{\pgfqpoint{1.232385in}{0.666736in}}%
\pgfpathcurveto{\pgfqpoint{1.224571in}{0.674549in}}{\pgfqpoint{1.213972in}{0.678939in}}{\pgfqpoint{1.202922in}{0.678939in}}%
\pgfpathcurveto{\pgfqpoint{1.191872in}{0.678939in}}{\pgfqpoint{1.181273in}{0.674549in}}{\pgfqpoint{1.173459in}{0.666736in}}%
\pgfpathcurveto{\pgfqpoint{1.165646in}{0.658922in}}{\pgfqpoint{1.161255in}{0.648323in}}{\pgfqpoint{1.161255in}{0.637273in}}%
\pgfpathcurveto{\pgfqpoint{1.161255in}{0.626223in}}{\pgfqpoint{1.165646in}{0.615624in}}{\pgfqpoint{1.173459in}{0.607810in}}%
\pgfpathcurveto{\pgfqpoint{1.181273in}{0.599996in}}{\pgfqpoint{1.191872in}{0.595606in}}{\pgfqpoint{1.202922in}{0.595606in}}%
\pgfpathclose%
\pgfusepath{stroke,fill}%
\end{pgfscope}%
\begin{pgfscope}%
\pgfpathrectangle{\pgfqpoint{0.750000in}{0.500000in}}{\pgfqpoint{4.650000in}{3.020000in}}%
\pgfusepath{clip}%
\pgfsetbuttcap%
\pgfsetroundjoin%
\definecolor{currentfill}{rgb}{1.000000,0.498039,0.054902}%
\pgfsetfillcolor{currentfill}%
\pgfsetlinewidth{1.003750pt}%
\definecolor{currentstroke}{rgb}{1.000000,0.498039,0.054902}%
\pgfsetstrokecolor{currentstroke}%
\pgfsetdash{}{0pt}%
\pgfpathmoveto{\pgfqpoint{2.531494in}{2.410332in}}%
\pgfpathcurveto{\pgfqpoint{2.542544in}{2.410332in}}{\pgfqpoint{2.553143in}{2.414722in}}{\pgfqpoint{2.560956in}{2.422536in}}%
\pgfpathcurveto{\pgfqpoint{2.568770in}{2.430350in}}{\pgfqpoint{2.573160in}{2.440949in}}{\pgfqpoint{2.573160in}{2.451999in}}%
\pgfpathcurveto{\pgfqpoint{2.573160in}{2.463049in}}{\pgfqpoint{2.568770in}{2.473648in}}{\pgfqpoint{2.560956in}{2.481461in}}%
\pgfpathcurveto{\pgfqpoint{2.553143in}{2.489275in}}{\pgfqpoint{2.542544in}{2.493665in}}{\pgfqpoint{2.531494in}{2.493665in}}%
\pgfpathcurveto{\pgfqpoint{2.520443in}{2.493665in}}{\pgfqpoint{2.509844in}{2.489275in}}{\pgfqpoint{2.502031in}{2.481461in}}%
\pgfpathcurveto{\pgfqpoint{2.494217in}{2.473648in}}{\pgfqpoint{2.489827in}{2.463049in}}{\pgfqpoint{2.489827in}{2.451999in}}%
\pgfpathcurveto{\pgfqpoint{2.489827in}{2.440949in}}{\pgfqpoint{2.494217in}{2.430350in}}{\pgfqpoint{2.502031in}{2.422536in}}%
\pgfpathcurveto{\pgfqpoint{2.509844in}{2.414722in}}{\pgfqpoint{2.520443in}{2.410332in}}{\pgfqpoint{2.531494in}{2.410332in}}%
\pgfpathclose%
\pgfusepath{stroke,fill}%
\end{pgfscope}%
\begin{pgfscope}%
\pgfpathrectangle{\pgfqpoint{0.750000in}{0.500000in}}{\pgfqpoint{4.650000in}{3.020000in}}%
\pgfusepath{clip}%
\pgfsetbuttcap%
\pgfsetroundjoin%
\definecolor{currentfill}{rgb}{1.000000,0.498039,0.054902}%
\pgfsetfillcolor{currentfill}%
\pgfsetlinewidth{1.003750pt}%
\definecolor{currentstroke}{rgb}{1.000000,0.498039,0.054902}%
\pgfsetstrokecolor{currentstroke}%
\pgfsetdash{}{0pt}%
\pgfpathmoveto{\pgfqpoint{2.652273in}{2.390861in}}%
\pgfpathcurveto{\pgfqpoint{2.663323in}{2.390861in}}{\pgfqpoint{2.673922in}{2.395251in}}{\pgfqpoint{2.681736in}{2.403065in}}%
\pgfpathcurveto{\pgfqpoint{2.689549in}{2.410878in}}{\pgfqpoint{2.693939in}{2.421477in}}{\pgfqpoint{2.693939in}{2.432527in}}%
\pgfpathcurveto{\pgfqpoint{2.693939in}{2.443578in}}{\pgfqpoint{2.689549in}{2.454177in}}{\pgfqpoint{2.681736in}{2.461990in}}%
\pgfpathcurveto{\pgfqpoint{2.673922in}{2.469804in}}{\pgfqpoint{2.663323in}{2.474194in}}{\pgfqpoint{2.652273in}{2.474194in}}%
\pgfpathcurveto{\pgfqpoint{2.641223in}{2.474194in}}{\pgfqpoint{2.630624in}{2.469804in}}{\pgfqpoint{2.622810in}{2.461990in}}%
\pgfpathcurveto{\pgfqpoint{2.614996in}{2.454177in}}{\pgfqpoint{2.610606in}{2.443578in}}{\pgfqpoint{2.610606in}{2.432527in}}%
\pgfpathcurveto{\pgfqpoint{2.610606in}{2.421477in}}{\pgfqpoint{2.614996in}{2.410878in}}{\pgfqpoint{2.622810in}{2.403065in}}%
\pgfpathcurveto{\pgfqpoint{2.630624in}{2.395251in}}{\pgfqpoint{2.641223in}{2.390861in}}{\pgfqpoint{2.652273in}{2.390861in}}%
\pgfpathclose%
\pgfusepath{stroke,fill}%
\end{pgfscope}%
\begin{pgfscope}%
\pgfpathrectangle{\pgfqpoint{0.750000in}{0.500000in}}{\pgfqpoint{4.650000in}{3.020000in}}%
\pgfusepath{clip}%
\pgfsetbuttcap%
\pgfsetroundjoin%
\definecolor{currentfill}{rgb}{1.000000,0.498039,0.054902}%
\pgfsetfillcolor{currentfill}%
\pgfsetlinewidth{1.003750pt}%
\definecolor{currentstroke}{rgb}{1.000000,0.498039,0.054902}%
\pgfsetstrokecolor{currentstroke}%
\pgfsetdash{}{0pt}%
\pgfpathmoveto{\pgfqpoint{2.048377in}{2.932163in}}%
\pgfpathcurveto{\pgfqpoint{2.059427in}{2.932163in}}{\pgfqpoint{2.070026in}{2.936553in}}{\pgfqpoint{2.077839in}{2.944367in}}%
\pgfpathcurveto{\pgfqpoint{2.085653in}{2.952181in}}{\pgfqpoint{2.090043in}{2.962780in}}{\pgfqpoint{2.090043in}{2.973830in}}%
\pgfpathcurveto{\pgfqpoint{2.090043in}{2.984880in}}{\pgfqpoint{2.085653in}{2.995479in}}{\pgfqpoint{2.077839in}{3.003293in}}%
\pgfpathcurveto{\pgfqpoint{2.070026in}{3.011106in}}{\pgfqpoint{2.059427in}{3.015496in}}{\pgfqpoint{2.048377in}{3.015496in}}%
\pgfpathcurveto{\pgfqpoint{2.037326in}{3.015496in}}{\pgfqpoint{2.026727in}{3.011106in}}{\pgfqpoint{2.018914in}{3.003293in}}%
\pgfpathcurveto{\pgfqpoint{2.011100in}{2.995479in}}{\pgfqpoint{2.006710in}{2.984880in}}{\pgfqpoint{2.006710in}{2.973830in}}%
\pgfpathcurveto{\pgfqpoint{2.006710in}{2.962780in}}{\pgfqpoint{2.011100in}{2.952181in}}{\pgfqpoint{2.018914in}{2.944367in}}%
\pgfpathcurveto{\pgfqpoint{2.026727in}{2.936553in}}{\pgfqpoint{2.037326in}{2.932163in}}{\pgfqpoint{2.048377in}{2.932163in}}%
\pgfpathclose%
\pgfusepath{stroke,fill}%
\end{pgfscope}%
\begin{pgfscope}%
\pgfpathrectangle{\pgfqpoint{0.750000in}{0.500000in}}{\pgfqpoint{4.650000in}{3.020000in}}%
\pgfusepath{clip}%
\pgfsetbuttcap%
\pgfsetroundjoin%
\definecolor{currentfill}{rgb}{1.000000,0.498039,0.054902}%
\pgfsetfillcolor{currentfill}%
\pgfsetlinewidth{1.003750pt}%
\definecolor{currentstroke}{rgb}{1.000000,0.498039,0.054902}%
\pgfsetstrokecolor{currentstroke}%
\pgfsetdash{}{0pt}%
\pgfpathmoveto{\pgfqpoint{2.169156in}{2.951634in}}%
\pgfpathcurveto{\pgfqpoint{2.180206in}{2.951634in}}{\pgfqpoint{2.190805in}{2.956025in}}{\pgfqpoint{2.198619in}{2.963838in}}%
\pgfpathcurveto{\pgfqpoint{2.206432in}{2.971652in}}{\pgfqpoint{2.210823in}{2.982251in}}{\pgfqpoint{2.210823in}{2.993301in}}%
\pgfpathcurveto{\pgfqpoint{2.210823in}{3.004351in}}{\pgfqpoint{2.206432in}{3.014950in}}{\pgfqpoint{2.198619in}{3.022764in}}%
\pgfpathcurveto{\pgfqpoint{2.190805in}{3.030577in}}{\pgfqpoint{2.180206in}{3.034968in}}{\pgfqpoint{2.169156in}{3.034968in}}%
\pgfpathcurveto{\pgfqpoint{2.158106in}{3.034968in}}{\pgfqpoint{2.147507in}{3.030577in}}{\pgfqpoint{2.139693in}{3.022764in}}%
\pgfpathcurveto{\pgfqpoint{2.131879in}{3.014950in}}{\pgfqpoint{2.127489in}{3.004351in}}{\pgfqpoint{2.127489in}{2.993301in}}%
\pgfpathcurveto{\pgfqpoint{2.127489in}{2.982251in}}{\pgfqpoint{2.131879in}{2.971652in}}{\pgfqpoint{2.139693in}{2.963838in}}%
\pgfpathcurveto{\pgfqpoint{2.147507in}{2.956025in}}{\pgfqpoint{2.158106in}{2.951634in}}{\pgfqpoint{2.169156in}{2.951634in}}%
\pgfpathclose%
\pgfusepath{stroke,fill}%
\end{pgfscope}%
\begin{pgfscope}%
\pgfpathrectangle{\pgfqpoint{0.750000in}{0.500000in}}{\pgfqpoint{4.650000in}{3.020000in}}%
\pgfusepath{clip}%
\pgfsetbuttcap%
\pgfsetroundjoin%
\definecolor{currentfill}{rgb}{1.000000,0.498039,0.054902}%
\pgfsetfillcolor{currentfill}%
\pgfsetlinewidth{1.003750pt}%
\definecolor{currentstroke}{rgb}{1.000000,0.498039,0.054902}%
\pgfsetstrokecolor{currentstroke}%
\pgfsetdash{}{0pt}%
\pgfpathmoveto{\pgfqpoint{1.927597in}{2.102685in}}%
\pgfpathcurveto{\pgfqpoint{1.938648in}{2.102685in}}{\pgfqpoint{1.949247in}{2.107076in}}{\pgfqpoint{1.957060in}{2.114889in}}%
\pgfpathcurveto{\pgfqpoint{1.964874in}{2.122703in}}{\pgfqpoint{1.969264in}{2.133302in}}{\pgfqpoint{1.969264in}{2.144352in}}%
\pgfpathcurveto{\pgfqpoint{1.969264in}{2.155402in}}{\pgfqpoint{1.964874in}{2.166001in}}{\pgfqpoint{1.957060in}{2.173815in}}%
\pgfpathcurveto{\pgfqpoint{1.949247in}{2.181628in}}{\pgfqpoint{1.938648in}{2.186019in}}{\pgfqpoint{1.927597in}{2.186019in}}%
\pgfpathcurveto{\pgfqpoint{1.916547in}{2.186019in}}{\pgfqpoint{1.905948in}{2.181628in}}{\pgfqpoint{1.898135in}{2.173815in}}%
\pgfpathcurveto{\pgfqpoint{1.890321in}{2.166001in}}{\pgfqpoint{1.885931in}{2.155402in}}{\pgfqpoint{1.885931in}{2.144352in}}%
\pgfpathcurveto{\pgfqpoint{1.885931in}{2.133302in}}{\pgfqpoint{1.890321in}{2.122703in}}{\pgfqpoint{1.898135in}{2.114889in}}%
\pgfpathcurveto{\pgfqpoint{1.905948in}{2.107076in}}{\pgfqpoint{1.916547in}{2.102685in}}{\pgfqpoint{1.927597in}{2.102685in}}%
\pgfpathclose%
\pgfusepath{stroke,fill}%
\end{pgfscope}%
\begin{pgfscope}%
\pgfpathrectangle{\pgfqpoint{0.750000in}{0.500000in}}{\pgfqpoint{4.650000in}{3.020000in}}%
\pgfusepath{clip}%
\pgfsetbuttcap%
\pgfsetroundjoin%
\definecolor{currentfill}{rgb}{1.000000,0.498039,0.054902}%
\pgfsetfillcolor{currentfill}%
\pgfsetlinewidth{1.003750pt}%
\definecolor{currentstroke}{rgb}{1.000000,0.498039,0.054902}%
\pgfsetstrokecolor{currentstroke}%
\pgfsetdash{}{0pt}%
\pgfpathmoveto{\pgfqpoint{1.867208in}{2.932163in}}%
\pgfpathcurveto{\pgfqpoint{1.878258in}{2.932163in}}{\pgfqpoint{1.888857in}{2.936553in}}{\pgfqpoint{1.896671in}{2.944367in}}%
\pgfpathcurveto{\pgfqpoint{1.904484in}{2.952181in}}{\pgfqpoint{1.908874in}{2.962780in}}{\pgfqpoint{1.908874in}{2.973830in}}%
\pgfpathcurveto{\pgfqpoint{1.908874in}{2.984880in}}{\pgfqpoint{1.904484in}{2.995479in}}{\pgfqpoint{1.896671in}{3.003293in}}%
\pgfpathcurveto{\pgfqpoint{1.888857in}{3.011106in}}{\pgfqpoint{1.878258in}{3.015496in}}{\pgfqpoint{1.867208in}{3.015496in}}%
\pgfpathcurveto{\pgfqpoint{1.856158in}{3.015496in}}{\pgfqpoint{1.845559in}{3.011106in}}{\pgfqpoint{1.837745in}{3.003293in}}%
\pgfpathcurveto{\pgfqpoint{1.829931in}{2.995479in}}{\pgfqpoint{1.825541in}{2.984880in}}{\pgfqpoint{1.825541in}{2.973830in}}%
\pgfpathcurveto{\pgfqpoint{1.825541in}{2.962780in}}{\pgfqpoint{1.829931in}{2.952181in}}{\pgfqpoint{1.837745in}{2.944367in}}%
\pgfpathcurveto{\pgfqpoint{1.845559in}{2.936553in}}{\pgfqpoint{1.856158in}{2.932163in}}{\pgfqpoint{1.867208in}{2.932163in}}%
\pgfpathclose%
\pgfusepath{stroke,fill}%
\end{pgfscope}%
\begin{pgfscope}%
\pgfpathrectangle{\pgfqpoint{0.750000in}{0.500000in}}{\pgfqpoint{4.650000in}{3.020000in}}%
\pgfusepath{clip}%
\pgfsetbuttcap%
\pgfsetroundjoin%
\definecolor{currentfill}{rgb}{1.000000,0.498039,0.054902}%
\pgfsetfillcolor{currentfill}%
\pgfsetlinewidth{1.003750pt}%
\definecolor{currentstroke}{rgb}{1.000000,0.498039,0.054902}%
\pgfsetstrokecolor{currentstroke}%
\pgfsetdash{}{0pt}%
\pgfpathmoveto{\pgfqpoint{4.403571in}{2.932163in}}%
\pgfpathcurveto{\pgfqpoint{4.414622in}{2.932163in}}{\pgfqpoint{4.425221in}{2.936553in}}{\pgfqpoint{4.433034in}{2.944367in}}%
\pgfpathcurveto{\pgfqpoint{4.440848in}{2.952181in}}{\pgfqpoint{4.445238in}{2.962780in}}{\pgfqpoint{4.445238in}{2.973830in}}%
\pgfpathcurveto{\pgfqpoint{4.445238in}{2.984880in}}{\pgfqpoint{4.440848in}{2.995479in}}{\pgfqpoint{4.433034in}{3.003293in}}%
\pgfpathcurveto{\pgfqpoint{4.425221in}{3.011106in}}{\pgfqpoint{4.414622in}{3.015496in}}{\pgfqpoint{4.403571in}{3.015496in}}%
\pgfpathcurveto{\pgfqpoint{4.392521in}{3.015496in}}{\pgfqpoint{4.381922in}{3.011106in}}{\pgfqpoint{4.374109in}{3.003293in}}%
\pgfpathcurveto{\pgfqpoint{4.366295in}{2.995479in}}{\pgfqpoint{4.361905in}{2.984880in}}{\pgfqpoint{4.361905in}{2.973830in}}%
\pgfpathcurveto{\pgfqpoint{4.361905in}{2.962780in}}{\pgfqpoint{4.366295in}{2.952181in}}{\pgfqpoint{4.374109in}{2.944367in}}%
\pgfpathcurveto{\pgfqpoint{4.381922in}{2.936553in}}{\pgfqpoint{4.392521in}{2.932163in}}{\pgfqpoint{4.403571in}{2.932163in}}%
\pgfpathclose%
\pgfusepath{stroke,fill}%
\end{pgfscope}%
\begin{pgfscope}%
\pgfpathrectangle{\pgfqpoint{0.750000in}{0.500000in}}{\pgfqpoint{4.650000in}{3.020000in}}%
\pgfusepath{clip}%
\pgfsetbuttcap%
\pgfsetroundjoin%
\definecolor{currentfill}{rgb}{1.000000,0.498039,0.054902}%
\pgfsetfillcolor{currentfill}%
\pgfsetlinewidth{1.003750pt}%
\definecolor{currentstroke}{rgb}{1.000000,0.498039,0.054902}%
\pgfsetstrokecolor{currentstroke}%
\pgfsetdash{}{0pt}%
\pgfpathmoveto{\pgfqpoint{1.504870in}{2.932163in}}%
\pgfpathcurveto{\pgfqpoint{1.515920in}{2.932163in}}{\pgfqpoint{1.526519in}{2.936553in}}{\pgfqpoint{1.534333in}{2.944367in}}%
\pgfpathcurveto{\pgfqpoint{1.542147in}{2.952181in}}{\pgfqpoint{1.546537in}{2.962780in}}{\pgfqpoint{1.546537in}{2.973830in}}%
\pgfpathcurveto{\pgfqpoint{1.546537in}{2.984880in}}{\pgfqpoint{1.542147in}{2.995479in}}{\pgfqpoint{1.534333in}{3.003293in}}%
\pgfpathcurveto{\pgfqpoint{1.526519in}{3.011106in}}{\pgfqpoint{1.515920in}{3.015496in}}{\pgfqpoint{1.504870in}{3.015496in}}%
\pgfpathcurveto{\pgfqpoint{1.493820in}{3.015496in}}{\pgfqpoint{1.483221in}{3.011106in}}{\pgfqpoint{1.475407in}{3.003293in}}%
\pgfpathcurveto{\pgfqpoint{1.467594in}{2.995479in}}{\pgfqpoint{1.463203in}{2.984880in}}{\pgfqpoint{1.463203in}{2.973830in}}%
\pgfpathcurveto{\pgfqpoint{1.463203in}{2.962780in}}{\pgfqpoint{1.467594in}{2.952181in}}{\pgfqpoint{1.475407in}{2.944367in}}%
\pgfpathcurveto{\pgfqpoint{1.483221in}{2.936553in}}{\pgfqpoint{1.493820in}{2.932163in}}{\pgfqpoint{1.504870in}{2.932163in}}%
\pgfpathclose%
\pgfusepath{stroke,fill}%
\end{pgfscope}%
\begin{pgfscope}%
\pgfpathrectangle{\pgfqpoint{0.750000in}{0.500000in}}{\pgfqpoint{4.650000in}{3.020000in}}%
\pgfusepath{clip}%
\pgfsetbuttcap%
\pgfsetroundjoin%
\definecolor{currentfill}{rgb}{0.839216,0.152941,0.156863}%
\pgfsetfillcolor{currentfill}%
\pgfsetlinewidth{1.003750pt}%
\definecolor{currentstroke}{rgb}{0.839216,0.152941,0.156863}%
\pgfsetstrokecolor{currentstroke}%
\pgfsetdash{}{0pt}%
\pgfpathmoveto{\pgfqpoint{1.384091in}{2.932163in}}%
\pgfpathcurveto{\pgfqpoint{1.395141in}{2.932163in}}{\pgfqpoint{1.405740in}{2.936553in}}{\pgfqpoint{1.413554in}{2.944367in}}%
\pgfpathcurveto{\pgfqpoint{1.421367in}{2.952181in}}{\pgfqpoint{1.425758in}{2.962780in}}{\pgfqpoint{1.425758in}{2.973830in}}%
\pgfpathcurveto{\pgfqpoint{1.425758in}{2.984880in}}{\pgfqpoint{1.421367in}{2.995479in}}{\pgfqpoint{1.413554in}{3.003293in}}%
\pgfpathcurveto{\pgfqpoint{1.405740in}{3.011106in}}{\pgfqpoint{1.395141in}{3.015496in}}{\pgfqpoint{1.384091in}{3.015496in}}%
\pgfpathcurveto{\pgfqpoint{1.373041in}{3.015496in}}{\pgfqpoint{1.362442in}{3.011106in}}{\pgfqpoint{1.354628in}{3.003293in}}%
\pgfpathcurveto{\pgfqpoint{1.346815in}{2.995479in}}{\pgfqpoint{1.342424in}{2.984880in}}{\pgfqpoint{1.342424in}{2.973830in}}%
\pgfpathcurveto{\pgfqpoint{1.342424in}{2.962780in}}{\pgfqpoint{1.346815in}{2.952181in}}{\pgfqpoint{1.354628in}{2.944367in}}%
\pgfpathcurveto{\pgfqpoint{1.362442in}{2.936553in}}{\pgfqpoint{1.373041in}{2.932163in}}{\pgfqpoint{1.384091in}{2.932163in}}%
\pgfpathclose%
\pgfusepath{stroke,fill}%
\end{pgfscope}%
\begin{pgfscope}%
\pgfpathrectangle{\pgfqpoint{0.750000in}{0.500000in}}{\pgfqpoint{4.650000in}{3.020000in}}%
\pgfusepath{clip}%
\pgfsetbuttcap%
\pgfsetroundjoin%
\definecolor{currentfill}{rgb}{1.000000,0.498039,0.054902}%
\pgfsetfillcolor{currentfill}%
\pgfsetlinewidth{1.003750pt}%
\definecolor{currentstroke}{rgb}{1.000000,0.498039,0.054902}%
\pgfsetstrokecolor{currentstroke}%
\pgfsetdash{}{0pt}%
\pgfpathmoveto{\pgfqpoint{1.323701in}{2.932163in}}%
\pgfpathcurveto{\pgfqpoint{1.334751in}{2.932163in}}{\pgfqpoint{1.345350in}{2.936553in}}{\pgfqpoint{1.353164in}{2.944367in}}%
\pgfpathcurveto{\pgfqpoint{1.360978in}{2.952181in}}{\pgfqpoint{1.365368in}{2.962780in}}{\pgfqpoint{1.365368in}{2.973830in}}%
\pgfpathcurveto{\pgfqpoint{1.365368in}{2.984880in}}{\pgfqpoint{1.360978in}{2.995479in}}{\pgfqpoint{1.353164in}{3.003293in}}%
\pgfpathcurveto{\pgfqpoint{1.345350in}{3.011106in}}{\pgfqpoint{1.334751in}{3.015496in}}{\pgfqpoint{1.323701in}{3.015496in}}%
\pgfpathcurveto{\pgfqpoint{1.312651in}{3.015496in}}{\pgfqpoint{1.302052in}{3.011106in}}{\pgfqpoint{1.294239in}{3.003293in}}%
\pgfpathcurveto{\pgfqpoint{1.286425in}{2.995479in}}{\pgfqpoint{1.282035in}{2.984880in}}{\pgfqpoint{1.282035in}{2.973830in}}%
\pgfpathcurveto{\pgfqpoint{1.282035in}{2.962780in}}{\pgfqpoint{1.286425in}{2.952181in}}{\pgfqpoint{1.294239in}{2.944367in}}%
\pgfpathcurveto{\pgfqpoint{1.302052in}{2.936553in}}{\pgfqpoint{1.312651in}{2.932163in}}{\pgfqpoint{1.323701in}{2.932163in}}%
\pgfpathclose%
\pgfusepath{stroke,fill}%
\end{pgfscope}%
\begin{pgfscope}%
\pgfpathrectangle{\pgfqpoint{0.750000in}{0.500000in}}{\pgfqpoint{4.650000in}{3.020000in}}%
\pgfusepath{clip}%
\pgfsetbuttcap%
\pgfsetroundjoin%
\definecolor{currentfill}{rgb}{0.121569,0.466667,0.705882}%
\pgfsetfillcolor{currentfill}%
\pgfsetlinewidth{1.003750pt}%
\definecolor{currentstroke}{rgb}{0.121569,0.466667,0.705882}%
\pgfsetstrokecolor{currentstroke}%
\pgfsetdash{}{0pt}%
\pgfpathmoveto{\pgfqpoint{1.323701in}{1.452344in}}%
\pgfpathcurveto{\pgfqpoint{1.334751in}{1.452344in}}{\pgfqpoint{1.345350in}{1.456734in}}{\pgfqpoint{1.353164in}{1.464548in}}%
\pgfpathcurveto{\pgfqpoint{1.360978in}{1.472361in}}{\pgfqpoint{1.365368in}{1.482960in}}{\pgfqpoint{1.365368in}{1.494010in}}%
\pgfpathcurveto{\pgfqpoint{1.365368in}{1.505060in}}{\pgfqpoint{1.360978in}{1.515659in}}{\pgfqpoint{1.353164in}{1.523473in}}%
\pgfpathcurveto{\pgfqpoint{1.345350in}{1.531287in}}{\pgfqpoint{1.334751in}{1.535677in}}{\pgfqpoint{1.323701in}{1.535677in}}%
\pgfpathcurveto{\pgfqpoint{1.312651in}{1.535677in}}{\pgfqpoint{1.302052in}{1.531287in}}{\pgfqpoint{1.294239in}{1.523473in}}%
\pgfpathcurveto{\pgfqpoint{1.286425in}{1.515659in}}{\pgfqpoint{1.282035in}{1.505060in}}{\pgfqpoint{1.282035in}{1.494010in}}%
\pgfpathcurveto{\pgfqpoint{1.282035in}{1.482960in}}{\pgfqpoint{1.286425in}{1.472361in}}{\pgfqpoint{1.294239in}{1.464548in}}%
\pgfpathcurveto{\pgfqpoint{1.302052in}{1.456734in}}{\pgfqpoint{1.312651in}{1.452344in}}{\pgfqpoint{1.323701in}{1.452344in}}%
\pgfpathclose%
\pgfusepath{stroke,fill}%
\end{pgfscope}%
\begin{pgfscope}%
\pgfpathrectangle{\pgfqpoint{0.750000in}{0.500000in}}{\pgfqpoint{4.650000in}{3.020000in}}%
\pgfusepath{clip}%
\pgfsetbuttcap%
\pgfsetroundjoin%
\definecolor{currentfill}{rgb}{1.000000,0.498039,0.054902}%
\pgfsetfillcolor{currentfill}%
\pgfsetlinewidth{1.003750pt}%
\definecolor{currentstroke}{rgb}{1.000000,0.498039,0.054902}%
\pgfsetstrokecolor{currentstroke}%
\pgfsetdash{}{0pt}%
\pgfpathmoveto{\pgfqpoint{1.444481in}{2.936057in}}%
\pgfpathcurveto{\pgfqpoint{1.455531in}{2.936057in}}{\pgfqpoint{1.466130in}{2.940448in}}{\pgfqpoint{1.473943in}{2.948261in}}%
\pgfpathcurveto{\pgfqpoint{1.481757in}{2.956075in}}{\pgfqpoint{1.486147in}{2.966674in}}{\pgfqpoint{1.486147in}{2.977724in}}%
\pgfpathcurveto{\pgfqpoint{1.486147in}{2.988774in}}{\pgfqpoint{1.481757in}{2.999373in}}{\pgfqpoint{1.473943in}{3.007187in}}%
\pgfpathcurveto{\pgfqpoint{1.466130in}{3.015000in}}{\pgfqpoint{1.455531in}{3.019391in}}{\pgfqpoint{1.444481in}{3.019391in}}%
\pgfpathcurveto{\pgfqpoint{1.433430in}{3.019391in}}{\pgfqpoint{1.422831in}{3.015000in}}{\pgfqpoint{1.415018in}{3.007187in}}%
\pgfpathcurveto{\pgfqpoint{1.407204in}{2.999373in}}{\pgfqpoint{1.402814in}{2.988774in}}{\pgfqpoint{1.402814in}{2.977724in}}%
\pgfpathcurveto{\pgfqpoint{1.402814in}{2.966674in}}{\pgfqpoint{1.407204in}{2.956075in}}{\pgfqpoint{1.415018in}{2.948261in}}%
\pgfpathcurveto{\pgfqpoint{1.422831in}{2.940448in}}{\pgfqpoint{1.433430in}{2.936057in}}{\pgfqpoint{1.444481in}{2.936057in}}%
\pgfpathclose%
\pgfusepath{stroke,fill}%
\end{pgfscope}%
\begin{pgfscope}%
\pgfpathrectangle{\pgfqpoint{0.750000in}{0.500000in}}{\pgfqpoint{4.650000in}{3.020000in}}%
\pgfusepath{clip}%
\pgfsetbuttcap%
\pgfsetroundjoin%
\definecolor{currentfill}{rgb}{1.000000,0.498039,0.054902}%
\pgfsetfillcolor{currentfill}%
\pgfsetlinewidth{1.003750pt}%
\definecolor{currentstroke}{rgb}{1.000000,0.498039,0.054902}%
\pgfsetstrokecolor{currentstroke}%
\pgfsetdash{}{0pt}%
\pgfpathmoveto{\pgfqpoint{1.444481in}{2.928269in}}%
\pgfpathcurveto{\pgfqpoint{1.455531in}{2.928269in}}{\pgfqpoint{1.466130in}{2.932659in}}{\pgfqpoint{1.473943in}{2.940473in}}%
\pgfpathcurveto{\pgfqpoint{1.481757in}{2.948286in}}{\pgfqpoint{1.486147in}{2.958885in}}{\pgfqpoint{1.486147in}{2.969936in}}%
\pgfpathcurveto{\pgfqpoint{1.486147in}{2.980986in}}{\pgfqpoint{1.481757in}{2.991585in}}{\pgfqpoint{1.473943in}{2.999398in}}%
\pgfpathcurveto{\pgfqpoint{1.466130in}{3.007212in}}{\pgfqpoint{1.455531in}{3.011602in}}{\pgfqpoint{1.444481in}{3.011602in}}%
\pgfpathcurveto{\pgfqpoint{1.433430in}{3.011602in}}{\pgfqpoint{1.422831in}{3.007212in}}{\pgfqpoint{1.415018in}{2.999398in}}%
\pgfpathcurveto{\pgfqpoint{1.407204in}{2.991585in}}{\pgfqpoint{1.402814in}{2.980986in}}{\pgfqpoint{1.402814in}{2.969936in}}%
\pgfpathcurveto{\pgfqpoint{1.402814in}{2.958885in}}{\pgfqpoint{1.407204in}{2.948286in}}{\pgfqpoint{1.415018in}{2.940473in}}%
\pgfpathcurveto{\pgfqpoint{1.422831in}{2.932659in}}{\pgfqpoint{1.433430in}{2.928269in}}{\pgfqpoint{1.444481in}{2.928269in}}%
\pgfpathclose%
\pgfusepath{stroke,fill}%
\end{pgfscope}%
\begin{pgfscope}%
\pgfpathrectangle{\pgfqpoint{0.750000in}{0.500000in}}{\pgfqpoint{4.650000in}{3.020000in}}%
\pgfusepath{clip}%
\pgfsetbuttcap%
\pgfsetroundjoin%
\definecolor{currentfill}{rgb}{1.000000,0.498039,0.054902}%
\pgfsetfillcolor{currentfill}%
\pgfsetlinewidth{1.003750pt}%
\definecolor{currentstroke}{rgb}{1.000000,0.498039,0.054902}%
\pgfsetstrokecolor{currentstroke}%
\pgfsetdash{}{0pt}%
\pgfpathmoveto{\pgfqpoint{2.289935in}{2.932163in}}%
\pgfpathcurveto{\pgfqpoint{2.300985in}{2.932163in}}{\pgfqpoint{2.311584in}{2.936553in}}{\pgfqpoint{2.319398in}{2.944367in}}%
\pgfpathcurveto{\pgfqpoint{2.327211in}{2.952181in}}{\pgfqpoint{2.331602in}{2.962780in}}{\pgfqpoint{2.331602in}{2.973830in}}%
\pgfpathcurveto{\pgfqpoint{2.331602in}{2.984880in}}{\pgfqpoint{2.327211in}{2.995479in}}{\pgfqpoint{2.319398in}{3.003293in}}%
\pgfpathcurveto{\pgfqpoint{2.311584in}{3.011106in}}{\pgfqpoint{2.300985in}{3.015496in}}{\pgfqpoint{2.289935in}{3.015496in}}%
\pgfpathcurveto{\pgfqpoint{2.278885in}{3.015496in}}{\pgfqpoint{2.268286in}{3.011106in}}{\pgfqpoint{2.260472in}{3.003293in}}%
\pgfpathcurveto{\pgfqpoint{2.252659in}{2.995479in}}{\pgfqpoint{2.248268in}{2.984880in}}{\pgfqpoint{2.248268in}{2.973830in}}%
\pgfpathcurveto{\pgfqpoint{2.248268in}{2.962780in}}{\pgfqpoint{2.252659in}{2.952181in}}{\pgfqpoint{2.260472in}{2.944367in}}%
\pgfpathcurveto{\pgfqpoint{2.268286in}{2.936553in}}{\pgfqpoint{2.278885in}{2.932163in}}{\pgfqpoint{2.289935in}{2.932163in}}%
\pgfpathclose%
\pgfusepath{stroke,fill}%
\end{pgfscope}%
\begin{pgfscope}%
\pgfpathrectangle{\pgfqpoint{0.750000in}{0.500000in}}{\pgfqpoint{4.650000in}{3.020000in}}%
\pgfusepath{clip}%
\pgfsetbuttcap%
\pgfsetroundjoin%
\definecolor{currentfill}{rgb}{1.000000,0.498039,0.054902}%
\pgfsetfillcolor{currentfill}%
\pgfsetlinewidth{1.003750pt}%
\definecolor{currentstroke}{rgb}{1.000000,0.498039,0.054902}%
\pgfsetstrokecolor{currentstroke}%
\pgfsetdash{}{0pt}%
\pgfpathmoveto{\pgfqpoint{2.229545in}{2.939952in}}%
\pgfpathcurveto{\pgfqpoint{2.240596in}{2.939952in}}{\pgfqpoint{2.251195in}{2.944342in}}{\pgfqpoint{2.259008in}{2.952156in}}%
\pgfpathcurveto{\pgfqpoint{2.266822in}{2.959969in}}{\pgfqpoint{2.271212in}{2.970568in}}{\pgfqpoint{2.271212in}{2.981618in}}%
\pgfpathcurveto{\pgfqpoint{2.271212in}{2.992668in}}{\pgfqpoint{2.266822in}{3.003267in}}{\pgfqpoint{2.259008in}{3.011081in}}%
\pgfpathcurveto{\pgfqpoint{2.251195in}{3.018895in}}{\pgfqpoint{2.240596in}{3.023285in}}{\pgfqpoint{2.229545in}{3.023285in}}%
\pgfpathcurveto{\pgfqpoint{2.218495in}{3.023285in}}{\pgfqpoint{2.207896in}{3.018895in}}{\pgfqpoint{2.200083in}{3.011081in}}%
\pgfpathcurveto{\pgfqpoint{2.192269in}{3.003267in}}{\pgfqpoint{2.187879in}{2.992668in}}{\pgfqpoint{2.187879in}{2.981618in}}%
\pgfpathcurveto{\pgfqpoint{2.187879in}{2.970568in}}{\pgfqpoint{2.192269in}{2.959969in}}{\pgfqpoint{2.200083in}{2.952156in}}%
\pgfpathcurveto{\pgfqpoint{2.207896in}{2.944342in}}{\pgfqpoint{2.218495in}{2.939952in}}{\pgfqpoint{2.229545in}{2.939952in}}%
\pgfpathclose%
\pgfusepath{stroke,fill}%
\end{pgfscope}%
\begin{pgfscope}%
\pgfpathrectangle{\pgfqpoint{0.750000in}{0.500000in}}{\pgfqpoint{4.650000in}{3.020000in}}%
\pgfusepath{clip}%
\pgfsetbuttcap%
\pgfsetroundjoin%
\definecolor{currentfill}{rgb}{1.000000,0.498039,0.054902}%
\pgfsetfillcolor{currentfill}%
\pgfsetlinewidth{1.003750pt}%
\definecolor{currentstroke}{rgb}{1.000000,0.498039,0.054902}%
\pgfsetstrokecolor{currentstroke}%
\pgfsetdash{}{0pt}%
\pgfpathmoveto{\pgfqpoint{2.108766in}{2.928269in}}%
\pgfpathcurveto{\pgfqpoint{2.119816in}{2.928269in}}{\pgfqpoint{2.130415in}{2.932659in}}{\pgfqpoint{2.138229in}{2.940473in}}%
\pgfpathcurveto{\pgfqpoint{2.146043in}{2.948286in}}{\pgfqpoint{2.150433in}{2.958885in}}{\pgfqpoint{2.150433in}{2.969936in}}%
\pgfpathcurveto{\pgfqpoint{2.150433in}{2.980986in}}{\pgfqpoint{2.146043in}{2.991585in}}{\pgfqpoint{2.138229in}{2.999398in}}%
\pgfpathcurveto{\pgfqpoint{2.130415in}{3.007212in}}{\pgfqpoint{2.119816in}{3.011602in}}{\pgfqpoint{2.108766in}{3.011602in}}%
\pgfpathcurveto{\pgfqpoint{2.097716in}{3.011602in}}{\pgfqpoint{2.087117in}{3.007212in}}{\pgfqpoint{2.079303in}{2.999398in}}%
\pgfpathcurveto{\pgfqpoint{2.071490in}{2.991585in}}{\pgfqpoint{2.067100in}{2.980986in}}{\pgfqpoint{2.067100in}{2.969936in}}%
\pgfpathcurveto{\pgfqpoint{2.067100in}{2.958885in}}{\pgfqpoint{2.071490in}{2.948286in}}{\pgfqpoint{2.079303in}{2.940473in}}%
\pgfpathcurveto{\pgfqpoint{2.087117in}{2.932659in}}{\pgfqpoint{2.097716in}{2.928269in}}{\pgfqpoint{2.108766in}{2.928269in}}%
\pgfpathclose%
\pgfusepath{stroke,fill}%
\end{pgfscope}%
\begin{pgfscope}%
\pgfpathrectangle{\pgfqpoint{0.750000in}{0.500000in}}{\pgfqpoint{4.650000in}{3.020000in}}%
\pgfusepath{clip}%
\pgfsetbuttcap%
\pgfsetroundjoin%
\definecolor{currentfill}{rgb}{1.000000,0.498039,0.054902}%
\pgfsetfillcolor{currentfill}%
\pgfsetlinewidth{1.003750pt}%
\definecolor{currentstroke}{rgb}{1.000000,0.498039,0.054902}%
\pgfsetstrokecolor{currentstroke}%
\pgfsetdash{}{0pt}%
\pgfpathmoveto{\pgfqpoint{1.384091in}{2.932163in}}%
\pgfpathcurveto{\pgfqpoint{1.395141in}{2.932163in}}{\pgfqpoint{1.405740in}{2.936553in}}{\pgfqpoint{1.413554in}{2.944367in}}%
\pgfpathcurveto{\pgfqpoint{1.421367in}{2.952181in}}{\pgfqpoint{1.425758in}{2.962780in}}{\pgfqpoint{1.425758in}{2.973830in}}%
\pgfpathcurveto{\pgfqpoint{1.425758in}{2.984880in}}{\pgfqpoint{1.421367in}{2.995479in}}{\pgfqpoint{1.413554in}{3.003293in}}%
\pgfpathcurveto{\pgfqpoint{1.405740in}{3.011106in}}{\pgfqpoint{1.395141in}{3.015496in}}{\pgfqpoint{1.384091in}{3.015496in}}%
\pgfpathcurveto{\pgfqpoint{1.373041in}{3.015496in}}{\pgfqpoint{1.362442in}{3.011106in}}{\pgfqpoint{1.354628in}{3.003293in}}%
\pgfpathcurveto{\pgfqpoint{1.346815in}{2.995479in}}{\pgfqpoint{1.342424in}{2.984880in}}{\pgfqpoint{1.342424in}{2.973830in}}%
\pgfpathcurveto{\pgfqpoint{1.342424in}{2.962780in}}{\pgfqpoint{1.346815in}{2.952181in}}{\pgfqpoint{1.354628in}{2.944367in}}%
\pgfpathcurveto{\pgfqpoint{1.362442in}{2.936553in}}{\pgfqpoint{1.373041in}{2.932163in}}{\pgfqpoint{1.384091in}{2.932163in}}%
\pgfpathclose%
\pgfusepath{stroke,fill}%
\end{pgfscope}%
\begin{pgfscope}%
\pgfpathrectangle{\pgfqpoint{0.750000in}{0.500000in}}{\pgfqpoint{4.650000in}{3.020000in}}%
\pgfusepath{clip}%
\pgfsetbuttcap%
\pgfsetroundjoin%
\definecolor{currentfill}{rgb}{1.000000,0.498039,0.054902}%
\pgfsetfillcolor{currentfill}%
\pgfsetlinewidth{1.003750pt}%
\definecolor{currentstroke}{rgb}{1.000000,0.498039,0.054902}%
\pgfsetstrokecolor{currentstroke}%
\pgfsetdash{}{0pt}%
\pgfpathmoveto{\pgfqpoint{1.625649in}{2.608939in}}%
\pgfpathcurveto{\pgfqpoint{1.636699in}{2.608939in}}{\pgfqpoint{1.647299in}{2.613330in}}{\pgfqpoint{1.655112in}{2.621143in}}%
\pgfpathcurveto{\pgfqpoint{1.662926in}{2.628957in}}{\pgfqpoint{1.667316in}{2.639556in}}{\pgfqpoint{1.667316in}{2.650606in}}%
\pgfpathcurveto{\pgfqpoint{1.667316in}{2.661656in}}{\pgfqpoint{1.662926in}{2.672255in}}{\pgfqpoint{1.655112in}{2.680069in}}%
\pgfpathcurveto{\pgfqpoint{1.647299in}{2.687882in}}{\pgfqpoint{1.636699in}{2.692273in}}{\pgfqpoint{1.625649in}{2.692273in}}%
\pgfpathcurveto{\pgfqpoint{1.614599in}{2.692273in}}{\pgfqpoint{1.604000in}{2.687882in}}{\pgfqpoint{1.596187in}{2.680069in}}%
\pgfpathcurveto{\pgfqpoint{1.588373in}{2.672255in}}{\pgfqpoint{1.583983in}{2.661656in}}{\pgfqpoint{1.583983in}{2.650606in}}%
\pgfpathcurveto{\pgfqpoint{1.583983in}{2.639556in}}{\pgfqpoint{1.588373in}{2.628957in}}{\pgfqpoint{1.596187in}{2.621143in}}%
\pgfpathcurveto{\pgfqpoint{1.604000in}{2.613330in}}{\pgfqpoint{1.614599in}{2.608939in}}{\pgfqpoint{1.625649in}{2.608939in}}%
\pgfpathclose%
\pgfusepath{stroke,fill}%
\end{pgfscope}%
\begin{pgfscope}%
\pgfpathrectangle{\pgfqpoint{0.750000in}{0.500000in}}{\pgfqpoint{4.650000in}{3.020000in}}%
\pgfusepath{clip}%
\pgfsetbuttcap%
\pgfsetroundjoin%
\definecolor{currentfill}{rgb}{1.000000,0.498039,0.054902}%
\pgfsetfillcolor{currentfill}%
\pgfsetlinewidth{1.003750pt}%
\definecolor{currentstroke}{rgb}{1.000000,0.498039,0.054902}%
\pgfsetstrokecolor{currentstroke}%
\pgfsetdash{}{0pt}%
\pgfpathmoveto{\pgfqpoint{3.376948in}{2.912692in}}%
\pgfpathcurveto{\pgfqpoint{3.387998in}{2.912692in}}{\pgfqpoint{3.398597in}{2.917082in}}{\pgfqpoint{3.406411in}{2.924896in}}%
\pgfpathcurveto{\pgfqpoint{3.414224in}{2.932709in}}{\pgfqpoint{3.418615in}{2.943308in}}{\pgfqpoint{3.418615in}{2.954358in}}%
\pgfpathcurveto{\pgfqpoint{3.418615in}{2.965409in}}{\pgfqpoint{3.414224in}{2.976008in}}{\pgfqpoint{3.406411in}{2.983821in}}%
\pgfpathcurveto{\pgfqpoint{3.398597in}{2.991635in}}{\pgfqpoint{3.387998in}{2.996025in}}{\pgfqpoint{3.376948in}{2.996025in}}%
\pgfpathcurveto{\pgfqpoint{3.365898in}{2.996025in}}{\pgfqpoint{3.355299in}{2.991635in}}{\pgfqpoint{3.347485in}{2.983821in}}%
\pgfpathcurveto{\pgfqpoint{3.339672in}{2.976008in}}{\pgfqpoint{3.335281in}{2.965409in}}{\pgfqpoint{3.335281in}{2.954358in}}%
\pgfpathcurveto{\pgfqpoint{3.335281in}{2.943308in}}{\pgfqpoint{3.339672in}{2.932709in}}{\pgfqpoint{3.347485in}{2.924896in}}%
\pgfpathcurveto{\pgfqpoint{3.355299in}{2.917082in}}{\pgfqpoint{3.365898in}{2.912692in}}{\pgfqpoint{3.376948in}{2.912692in}}%
\pgfpathclose%
\pgfusepath{stroke,fill}%
\end{pgfscope}%
\begin{pgfscope}%
\pgfpathrectangle{\pgfqpoint{0.750000in}{0.500000in}}{\pgfqpoint{4.650000in}{3.020000in}}%
\pgfusepath{clip}%
\pgfsetbuttcap%
\pgfsetroundjoin%
\definecolor{currentfill}{rgb}{1.000000,0.498039,0.054902}%
\pgfsetfillcolor{currentfill}%
\pgfsetlinewidth{1.003750pt}%
\definecolor{currentstroke}{rgb}{1.000000,0.498039,0.054902}%
\pgfsetstrokecolor{currentstroke}%
\pgfsetdash{}{0pt}%
\pgfpathmoveto{\pgfqpoint{1.504870in}{2.815335in}}%
\pgfpathcurveto{\pgfqpoint{1.515920in}{2.815335in}}{\pgfqpoint{1.526519in}{2.819726in}}{\pgfqpoint{1.534333in}{2.827539in}}%
\pgfpathcurveto{\pgfqpoint{1.542147in}{2.835353in}}{\pgfqpoint{1.546537in}{2.845952in}}{\pgfqpoint{1.546537in}{2.857002in}}%
\pgfpathcurveto{\pgfqpoint{1.546537in}{2.868052in}}{\pgfqpoint{1.542147in}{2.878651in}}{\pgfqpoint{1.534333in}{2.886465in}}%
\pgfpathcurveto{\pgfqpoint{1.526519in}{2.894278in}}{\pgfqpoint{1.515920in}{2.898669in}}{\pgfqpoint{1.504870in}{2.898669in}}%
\pgfpathcurveto{\pgfqpoint{1.493820in}{2.898669in}}{\pgfqpoint{1.483221in}{2.894278in}}{\pgfqpoint{1.475407in}{2.886465in}}%
\pgfpathcurveto{\pgfqpoint{1.467594in}{2.878651in}}{\pgfqpoint{1.463203in}{2.868052in}}{\pgfqpoint{1.463203in}{2.857002in}}%
\pgfpathcurveto{\pgfqpoint{1.463203in}{2.845952in}}{\pgfqpoint{1.467594in}{2.835353in}}{\pgfqpoint{1.475407in}{2.827539in}}%
\pgfpathcurveto{\pgfqpoint{1.483221in}{2.819726in}}{\pgfqpoint{1.493820in}{2.815335in}}{\pgfqpoint{1.504870in}{2.815335in}}%
\pgfpathclose%
\pgfusepath{stroke,fill}%
\end{pgfscope}%
\begin{pgfscope}%
\pgfpathrectangle{\pgfqpoint{0.750000in}{0.500000in}}{\pgfqpoint{4.650000in}{3.020000in}}%
\pgfusepath{clip}%
\pgfsetbuttcap%
\pgfsetroundjoin%
\definecolor{currentfill}{rgb}{0.121569,0.466667,0.705882}%
\pgfsetfillcolor{currentfill}%
\pgfsetlinewidth{1.003750pt}%
\definecolor{currentstroke}{rgb}{0.121569,0.466667,0.705882}%
\pgfsetstrokecolor{currentstroke}%
\pgfsetdash{}{0pt}%
\pgfpathmoveto{\pgfqpoint{1.202922in}{0.599500in}}%
\pgfpathcurveto{\pgfqpoint{1.213972in}{0.599500in}}{\pgfqpoint{1.224571in}{0.603891in}}{\pgfqpoint{1.232385in}{0.611704in}}%
\pgfpathcurveto{\pgfqpoint{1.240198in}{0.619518in}}{\pgfqpoint{1.244589in}{0.630117in}}{\pgfqpoint{1.244589in}{0.641167in}}%
\pgfpathcurveto{\pgfqpoint{1.244589in}{0.652217in}}{\pgfqpoint{1.240198in}{0.662816in}}{\pgfqpoint{1.232385in}{0.670630in}}%
\pgfpathcurveto{\pgfqpoint{1.224571in}{0.678443in}}{\pgfqpoint{1.213972in}{0.682834in}}{\pgfqpoint{1.202922in}{0.682834in}}%
\pgfpathcurveto{\pgfqpoint{1.191872in}{0.682834in}}{\pgfqpoint{1.181273in}{0.678443in}}{\pgfqpoint{1.173459in}{0.670630in}}%
\pgfpathcurveto{\pgfqpoint{1.165646in}{0.662816in}}{\pgfqpoint{1.161255in}{0.652217in}}{\pgfqpoint{1.161255in}{0.641167in}}%
\pgfpathcurveto{\pgfqpoint{1.161255in}{0.630117in}}{\pgfqpoint{1.165646in}{0.619518in}}{\pgfqpoint{1.173459in}{0.611704in}}%
\pgfpathcurveto{\pgfqpoint{1.181273in}{0.603891in}}{\pgfqpoint{1.191872in}{0.599500in}}{\pgfqpoint{1.202922in}{0.599500in}}%
\pgfpathclose%
\pgfusepath{stroke,fill}%
\end{pgfscope}%
\begin{pgfscope}%
\pgfpathrectangle{\pgfqpoint{0.750000in}{0.500000in}}{\pgfqpoint{4.650000in}{3.020000in}}%
\pgfusepath{clip}%
\pgfsetbuttcap%
\pgfsetroundjoin%
\definecolor{currentfill}{rgb}{1.000000,0.498039,0.054902}%
\pgfsetfillcolor{currentfill}%
\pgfsetlinewidth{1.003750pt}%
\definecolor{currentstroke}{rgb}{1.000000,0.498039,0.054902}%
\pgfsetstrokecolor{currentstroke}%
\pgfsetdash{}{0pt}%
\pgfpathmoveto{\pgfqpoint{1.867208in}{2.936057in}}%
\pgfpathcurveto{\pgfqpoint{1.878258in}{2.936057in}}{\pgfqpoint{1.888857in}{2.940448in}}{\pgfqpoint{1.896671in}{2.948261in}}%
\pgfpathcurveto{\pgfqpoint{1.904484in}{2.956075in}}{\pgfqpoint{1.908874in}{2.966674in}}{\pgfqpoint{1.908874in}{2.977724in}}%
\pgfpathcurveto{\pgfqpoint{1.908874in}{2.988774in}}{\pgfqpoint{1.904484in}{2.999373in}}{\pgfqpoint{1.896671in}{3.007187in}}%
\pgfpathcurveto{\pgfqpoint{1.888857in}{3.015000in}}{\pgfqpoint{1.878258in}{3.019391in}}{\pgfqpoint{1.867208in}{3.019391in}}%
\pgfpathcurveto{\pgfqpoint{1.856158in}{3.019391in}}{\pgfqpoint{1.845559in}{3.015000in}}{\pgfqpoint{1.837745in}{3.007187in}}%
\pgfpathcurveto{\pgfqpoint{1.829931in}{2.999373in}}{\pgfqpoint{1.825541in}{2.988774in}}{\pgfqpoint{1.825541in}{2.977724in}}%
\pgfpathcurveto{\pgfqpoint{1.825541in}{2.966674in}}{\pgfqpoint{1.829931in}{2.956075in}}{\pgfqpoint{1.837745in}{2.948261in}}%
\pgfpathcurveto{\pgfqpoint{1.845559in}{2.940448in}}{\pgfqpoint{1.856158in}{2.936057in}}{\pgfqpoint{1.867208in}{2.936057in}}%
\pgfpathclose%
\pgfusepath{stroke,fill}%
\end{pgfscope}%
\begin{pgfscope}%
\pgfpathrectangle{\pgfqpoint{0.750000in}{0.500000in}}{\pgfqpoint{4.650000in}{3.020000in}}%
\pgfusepath{clip}%
\pgfsetbuttcap%
\pgfsetroundjoin%
\definecolor{currentfill}{rgb}{0.121569,0.466667,0.705882}%
\pgfsetfillcolor{currentfill}%
\pgfsetlinewidth{1.003750pt}%
\definecolor{currentstroke}{rgb}{0.121569,0.466667,0.705882}%
\pgfsetstrokecolor{currentstroke}%
\pgfsetdash{}{0pt}%
\pgfpathmoveto{\pgfqpoint{1.263312in}{0.595606in}}%
\pgfpathcurveto{\pgfqpoint{1.274362in}{0.595606in}}{\pgfqpoint{1.284961in}{0.599996in}}{\pgfqpoint{1.292774in}{0.607810in}}%
\pgfpathcurveto{\pgfqpoint{1.300588in}{0.615624in}}{\pgfqpoint{1.304978in}{0.626223in}}{\pgfqpoint{1.304978in}{0.637273in}}%
\pgfpathcurveto{\pgfqpoint{1.304978in}{0.648323in}}{\pgfqpoint{1.300588in}{0.658922in}}{\pgfqpoint{1.292774in}{0.666736in}}%
\pgfpathcurveto{\pgfqpoint{1.284961in}{0.674549in}}{\pgfqpoint{1.274362in}{0.678939in}}{\pgfqpoint{1.263312in}{0.678939in}}%
\pgfpathcurveto{\pgfqpoint{1.252262in}{0.678939in}}{\pgfqpoint{1.241663in}{0.674549in}}{\pgfqpoint{1.233849in}{0.666736in}}%
\pgfpathcurveto{\pgfqpoint{1.226035in}{0.658922in}}{\pgfqpoint{1.221645in}{0.648323in}}{\pgfqpoint{1.221645in}{0.637273in}}%
\pgfpathcurveto{\pgfqpoint{1.221645in}{0.626223in}}{\pgfqpoint{1.226035in}{0.615624in}}{\pgfqpoint{1.233849in}{0.607810in}}%
\pgfpathcurveto{\pgfqpoint{1.241663in}{0.599996in}}{\pgfqpoint{1.252262in}{0.595606in}}{\pgfqpoint{1.263312in}{0.595606in}}%
\pgfpathclose%
\pgfusepath{stroke,fill}%
\end{pgfscope}%
\begin{pgfscope}%
\pgfpathrectangle{\pgfqpoint{0.750000in}{0.500000in}}{\pgfqpoint{4.650000in}{3.020000in}}%
\pgfusepath{clip}%
\pgfsetbuttcap%
\pgfsetroundjoin%
\definecolor{currentfill}{rgb}{1.000000,0.498039,0.054902}%
\pgfsetfillcolor{currentfill}%
\pgfsetlinewidth{1.003750pt}%
\definecolor{currentstroke}{rgb}{1.000000,0.498039,0.054902}%
\pgfsetstrokecolor{currentstroke}%
\pgfsetdash{}{0pt}%
\pgfpathmoveto{\pgfqpoint{1.625649in}{2.928269in}}%
\pgfpathcurveto{\pgfqpoint{1.636699in}{2.928269in}}{\pgfqpoint{1.647299in}{2.932659in}}{\pgfqpoint{1.655112in}{2.940473in}}%
\pgfpathcurveto{\pgfqpoint{1.662926in}{2.948286in}}{\pgfqpoint{1.667316in}{2.958885in}}{\pgfqpoint{1.667316in}{2.969936in}}%
\pgfpathcurveto{\pgfqpoint{1.667316in}{2.980986in}}{\pgfqpoint{1.662926in}{2.991585in}}{\pgfqpoint{1.655112in}{2.999398in}}%
\pgfpathcurveto{\pgfqpoint{1.647299in}{3.007212in}}{\pgfqpoint{1.636699in}{3.011602in}}{\pgfqpoint{1.625649in}{3.011602in}}%
\pgfpathcurveto{\pgfqpoint{1.614599in}{3.011602in}}{\pgfqpoint{1.604000in}{3.007212in}}{\pgfqpoint{1.596187in}{2.999398in}}%
\pgfpathcurveto{\pgfqpoint{1.588373in}{2.991585in}}{\pgfqpoint{1.583983in}{2.980986in}}{\pgfqpoint{1.583983in}{2.969936in}}%
\pgfpathcurveto{\pgfqpoint{1.583983in}{2.958885in}}{\pgfqpoint{1.588373in}{2.948286in}}{\pgfqpoint{1.596187in}{2.940473in}}%
\pgfpathcurveto{\pgfqpoint{1.604000in}{2.932659in}}{\pgfqpoint{1.614599in}{2.928269in}}{\pgfqpoint{1.625649in}{2.928269in}}%
\pgfpathclose%
\pgfusepath{stroke,fill}%
\end{pgfscope}%
\begin{pgfscope}%
\pgfpathrectangle{\pgfqpoint{0.750000in}{0.500000in}}{\pgfqpoint{4.650000in}{3.020000in}}%
\pgfusepath{clip}%
\pgfsetbuttcap%
\pgfsetroundjoin%
\definecolor{currentfill}{rgb}{1.000000,0.498039,0.054902}%
\pgfsetfillcolor{currentfill}%
\pgfsetlinewidth{1.003750pt}%
\definecolor{currentstroke}{rgb}{1.000000,0.498039,0.054902}%
\pgfsetstrokecolor{currentstroke}%
\pgfsetdash{}{0pt}%
\pgfpathmoveto{\pgfqpoint{2.229545in}{2.939952in}}%
\pgfpathcurveto{\pgfqpoint{2.240596in}{2.939952in}}{\pgfqpoint{2.251195in}{2.944342in}}{\pgfqpoint{2.259008in}{2.952156in}}%
\pgfpathcurveto{\pgfqpoint{2.266822in}{2.959969in}}{\pgfqpoint{2.271212in}{2.970568in}}{\pgfqpoint{2.271212in}{2.981618in}}%
\pgfpathcurveto{\pgfqpoint{2.271212in}{2.992668in}}{\pgfqpoint{2.266822in}{3.003267in}}{\pgfqpoint{2.259008in}{3.011081in}}%
\pgfpathcurveto{\pgfqpoint{2.251195in}{3.018895in}}{\pgfqpoint{2.240596in}{3.023285in}}{\pgfqpoint{2.229545in}{3.023285in}}%
\pgfpathcurveto{\pgfqpoint{2.218495in}{3.023285in}}{\pgfqpoint{2.207896in}{3.018895in}}{\pgfqpoint{2.200083in}{3.011081in}}%
\pgfpathcurveto{\pgfqpoint{2.192269in}{3.003267in}}{\pgfqpoint{2.187879in}{2.992668in}}{\pgfqpoint{2.187879in}{2.981618in}}%
\pgfpathcurveto{\pgfqpoint{2.187879in}{2.970568in}}{\pgfqpoint{2.192269in}{2.959969in}}{\pgfqpoint{2.200083in}{2.952156in}}%
\pgfpathcurveto{\pgfqpoint{2.207896in}{2.944342in}}{\pgfqpoint{2.218495in}{2.939952in}}{\pgfqpoint{2.229545in}{2.939952in}}%
\pgfpathclose%
\pgfusepath{stroke,fill}%
\end{pgfscope}%
\begin{pgfscope}%
\pgfpathrectangle{\pgfqpoint{0.750000in}{0.500000in}}{\pgfqpoint{4.650000in}{3.020000in}}%
\pgfusepath{clip}%
\pgfsetbuttcap%
\pgfsetroundjoin%
\definecolor{currentfill}{rgb}{0.121569,0.466667,0.705882}%
\pgfsetfillcolor{currentfill}%
\pgfsetlinewidth{1.003750pt}%
\definecolor{currentstroke}{rgb}{0.121569,0.466667,0.705882}%
\pgfsetstrokecolor{currentstroke}%
\pgfsetdash{}{0pt}%
\pgfpathmoveto{\pgfqpoint{1.021753in}{0.595606in}}%
\pgfpathcurveto{\pgfqpoint{1.032803in}{0.595606in}}{\pgfqpoint{1.043402in}{0.599996in}}{\pgfqpoint{1.051216in}{0.607810in}}%
\pgfpathcurveto{\pgfqpoint{1.059030in}{0.615624in}}{\pgfqpoint{1.063420in}{0.626223in}}{\pgfqpoint{1.063420in}{0.637273in}}%
\pgfpathcurveto{\pgfqpoint{1.063420in}{0.648323in}}{\pgfqpoint{1.059030in}{0.658922in}}{\pgfqpoint{1.051216in}{0.666736in}}%
\pgfpathcurveto{\pgfqpoint{1.043402in}{0.674549in}}{\pgfqpoint{1.032803in}{0.678939in}}{\pgfqpoint{1.021753in}{0.678939in}}%
\pgfpathcurveto{\pgfqpoint{1.010703in}{0.678939in}}{\pgfqpoint{1.000104in}{0.674549in}}{\pgfqpoint{0.992290in}{0.666736in}}%
\pgfpathcurveto{\pgfqpoint{0.984477in}{0.658922in}}{\pgfqpoint{0.980087in}{0.648323in}}{\pgfqpoint{0.980087in}{0.637273in}}%
\pgfpathcurveto{\pgfqpoint{0.980087in}{0.626223in}}{\pgfqpoint{0.984477in}{0.615624in}}{\pgfqpoint{0.992290in}{0.607810in}}%
\pgfpathcurveto{\pgfqpoint{1.000104in}{0.599996in}}{\pgfqpoint{1.010703in}{0.595606in}}{\pgfqpoint{1.021753in}{0.595606in}}%
\pgfpathclose%
\pgfusepath{stroke,fill}%
\end{pgfscope}%
\begin{pgfscope}%
\pgfpathrectangle{\pgfqpoint{0.750000in}{0.500000in}}{\pgfqpoint{4.650000in}{3.020000in}}%
\pgfusepath{clip}%
\pgfsetbuttcap%
\pgfsetroundjoin%
\definecolor{currentfill}{rgb}{1.000000,0.498039,0.054902}%
\pgfsetfillcolor{currentfill}%
\pgfsetlinewidth{1.003750pt}%
\definecolor{currentstroke}{rgb}{1.000000,0.498039,0.054902}%
\pgfsetstrokecolor{currentstroke}%
\pgfsetdash{}{0pt}%
\pgfpathmoveto{\pgfqpoint{1.625649in}{2.990577in}}%
\pgfpathcurveto{\pgfqpoint{1.636699in}{2.990577in}}{\pgfqpoint{1.647299in}{2.994967in}}{\pgfqpoint{1.655112in}{3.002781in}}%
\pgfpathcurveto{\pgfqpoint{1.662926in}{3.010595in}}{\pgfqpoint{1.667316in}{3.021194in}}{\pgfqpoint{1.667316in}{3.032244in}}%
\pgfpathcurveto{\pgfqpoint{1.667316in}{3.043294in}}{\pgfqpoint{1.662926in}{3.053893in}}{\pgfqpoint{1.655112in}{3.061706in}}%
\pgfpathcurveto{\pgfqpoint{1.647299in}{3.069520in}}{\pgfqpoint{1.636699in}{3.073910in}}{\pgfqpoint{1.625649in}{3.073910in}}%
\pgfpathcurveto{\pgfqpoint{1.614599in}{3.073910in}}{\pgfqpoint{1.604000in}{3.069520in}}{\pgfqpoint{1.596187in}{3.061706in}}%
\pgfpathcurveto{\pgfqpoint{1.588373in}{3.053893in}}{\pgfqpoint{1.583983in}{3.043294in}}{\pgfqpoint{1.583983in}{3.032244in}}%
\pgfpathcurveto{\pgfqpoint{1.583983in}{3.021194in}}{\pgfqpoint{1.588373in}{3.010595in}}{\pgfqpoint{1.596187in}{3.002781in}}%
\pgfpathcurveto{\pgfqpoint{1.604000in}{2.994967in}}{\pgfqpoint{1.614599in}{2.990577in}}{\pgfqpoint{1.625649in}{2.990577in}}%
\pgfpathclose%
\pgfusepath{stroke,fill}%
\end{pgfscope}%
\begin{pgfscope}%
\pgfpathrectangle{\pgfqpoint{0.750000in}{0.500000in}}{\pgfqpoint{4.650000in}{3.020000in}}%
\pgfusepath{clip}%
\pgfsetbuttcap%
\pgfsetroundjoin%
\definecolor{currentfill}{rgb}{1.000000,0.498039,0.054902}%
\pgfsetfillcolor{currentfill}%
\pgfsetlinewidth{1.003750pt}%
\definecolor{currentstroke}{rgb}{1.000000,0.498039,0.054902}%
\pgfsetstrokecolor{currentstroke}%
\pgfsetdash{}{0pt}%
\pgfpathmoveto{\pgfqpoint{1.625649in}{2.998366in}}%
\pgfpathcurveto{\pgfqpoint{1.636699in}{2.998366in}}{\pgfqpoint{1.647299in}{3.002756in}}{\pgfqpoint{1.655112in}{3.010569in}}%
\pgfpathcurveto{\pgfqpoint{1.662926in}{3.018383in}}{\pgfqpoint{1.667316in}{3.028982in}}{\pgfqpoint{1.667316in}{3.040032in}}%
\pgfpathcurveto{\pgfqpoint{1.667316in}{3.051082in}}{\pgfqpoint{1.662926in}{3.061681in}}{\pgfqpoint{1.655112in}{3.069495in}}%
\pgfpathcurveto{\pgfqpoint{1.647299in}{3.077309in}}{\pgfqpoint{1.636699in}{3.081699in}}{\pgfqpoint{1.625649in}{3.081699in}}%
\pgfpathcurveto{\pgfqpoint{1.614599in}{3.081699in}}{\pgfqpoint{1.604000in}{3.077309in}}{\pgfqpoint{1.596187in}{3.069495in}}%
\pgfpathcurveto{\pgfqpoint{1.588373in}{3.061681in}}{\pgfqpoint{1.583983in}{3.051082in}}{\pgfqpoint{1.583983in}{3.040032in}}%
\pgfpathcurveto{\pgfqpoint{1.583983in}{3.028982in}}{\pgfqpoint{1.588373in}{3.018383in}}{\pgfqpoint{1.596187in}{3.010569in}}%
\pgfpathcurveto{\pgfqpoint{1.604000in}{3.002756in}}{\pgfqpoint{1.614599in}{2.998366in}}{\pgfqpoint{1.625649in}{2.998366in}}%
\pgfpathclose%
\pgfusepath{stroke,fill}%
\end{pgfscope}%
\begin{pgfscope}%
\pgfpathrectangle{\pgfqpoint{0.750000in}{0.500000in}}{\pgfqpoint{4.650000in}{3.020000in}}%
\pgfusepath{clip}%
\pgfsetbuttcap%
\pgfsetroundjoin%
\definecolor{currentfill}{rgb}{1.000000,0.498039,0.054902}%
\pgfsetfillcolor{currentfill}%
\pgfsetlinewidth{1.003750pt}%
\definecolor{currentstroke}{rgb}{1.000000,0.498039,0.054902}%
\pgfsetstrokecolor{currentstroke}%
\pgfsetdash{}{0pt}%
\pgfpathmoveto{\pgfqpoint{1.565260in}{2.932163in}}%
\pgfpathcurveto{\pgfqpoint{1.576310in}{2.932163in}}{\pgfqpoint{1.586909in}{2.936553in}}{\pgfqpoint{1.594723in}{2.944367in}}%
\pgfpathcurveto{\pgfqpoint{1.602536in}{2.952181in}}{\pgfqpoint{1.606926in}{2.962780in}}{\pgfqpoint{1.606926in}{2.973830in}}%
\pgfpathcurveto{\pgfqpoint{1.606926in}{2.984880in}}{\pgfqpoint{1.602536in}{2.995479in}}{\pgfqpoint{1.594723in}{3.003293in}}%
\pgfpathcurveto{\pgfqpoint{1.586909in}{3.011106in}}{\pgfqpoint{1.576310in}{3.015496in}}{\pgfqpoint{1.565260in}{3.015496in}}%
\pgfpathcurveto{\pgfqpoint{1.554210in}{3.015496in}}{\pgfqpoint{1.543611in}{3.011106in}}{\pgfqpoint{1.535797in}{3.003293in}}%
\pgfpathcurveto{\pgfqpoint{1.527983in}{2.995479in}}{\pgfqpoint{1.523593in}{2.984880in}}{\pgfqpoint{1.523593in}{2.973830in}}%
\pgfpathcurveto{\pgfqpoint{1.523593in}{2.962780in}}{\pgfqpoint{1.527983in}{2.952181in}}{\pgfqpoint{1.535797in}{2.944367in}}%
\pgfpathcurveto{\pgfqpoint{1.543611in}{2.936553in}}{\pgfqpoint{1.554210in}{2.932163in}}{\pgfqpoint{1.565260in}{2.932163in}}%
\pgfpathclose%
\pgfusepath{stroke,fill}%
\end{pgfscope}%
\begin{pgfscope}%
\pgfpathrectangle{\pgfqpoint{0.750000in}{0.500000in}}{\pgfqpoint{4.650000in}{3.020000in}}%
\pgfusepath{clip}%
\pgfsetbuttcap%
\pgfsetroundjoin%
\definecolor{currentfill}{rgb}{1.000000,0.498039,0.054902}%
\pgfsetfillcolor{currentfill}%
\pgfsetlinewidth{1.003750pt}%
\definecolor{currentstroke}{rgb}{1.000000,0.498039,0.054902}%
\pgfsetstrokecolor{currentstroke}%
\pgfsetdash{}{0pt}%
\pgfpathmoveto{\pgfqpoint{1.806818in}{2.936057in}}%
\pgfpathcurveto{\pgfqpoint{1.817868in}{2.936057in}}{\pgfqpoint{1.828467in}{2.940448in}}{\pgfqpoint{1.836281in}{2.948261in}}%
\pgfpathcurveto{\pgfqpoint{1.844095in}{2.956075in}}{\pgfqpoint{1.848485in}{2.966674in}}{\pgfqpoint{1.848485in}{2.977724in}}%
\pgfpathcurveto{\pgfqpoint{1.848485in}{2.988774in}}{\pgfqpoint{1.844095in}{2.999373in}}{\pgfqpoint{1.836281in}{3.007187in}}%
\pgfpathcurveto{\pgfqpoint{1.828467in}{3.015000in}}{\pgfqpoint{1.817868in}{3.019391in}}{\pgfqpoint{1.806818in}{3.019391in}}%
\pgfpathcurveto{\pgfqpoint{1.795768in}{3.019391in}}{\pgfqpoint{1.785169in}{3.015000in}}{\pgfqpoint{1.777355in}{3.007187in}}%
\pgfpathcurveto{\pgfqpoint{1.769542in}{2.999373in}}{\pgfqpoint{1.765152in}{2.988774in}}{\pgfqpoint{1.765152in}{2.977724in}}%
\pgfpathcurveto{\pgfqpoint{1.765152in}{2.966674in}}{\pgfqpoint{1.769542in}{2.956075in}}{\pgfqpoint{1.777355in}{2.948261in}}%
\pgfpathcurveto{\pgfqpoint{1.785169in}{2.940448in}}{\pgfqpoint{1.795768in}{2.936057in}}{\pgfqpoint{1.806818in}{2.936057in}}%
\pgfpathclose%
\pgfusepath{stroke,fill}%
\end{pgfscope}%
\begin{pgfscope}%
\pgfpathrectangle{\pgfqpoint{0.750000in}{0.500000in}}{\pgfqpoint{4.650000in}{3.020000in}}%
\pgfusepath{clip}%
\pgfsetbuttcap%
\pgfsetroundjoin%
\definecolor{currentfill}{rgb}{1.000000,0.498039,0.054902}%
\pgfsetfillcolor{currentfill}%
\pgfsetlinewidth{1.003750pt}%
\definecolor{currentstroke}{rgb}{1.000000,0.498039,0.054902}%
\pgfsetstrokecolor{currentstroke}%
\pgfsetdash{}{0pt}%
\pgfpathmoveto{\pgfqpoint{1.565260in}{2.943846in}}%
\pgfpathcurveto{\pgfqpoint{1.576310in}{2.943846in}}{\pgfqpoint{1.586909in}{2.948236in}}{\pgfqpoint{1.594723in}{2.956050in}}%
\pgfpathcurveto{\pgfqpoint{1.602536in}{2.963863in}}{\pgfqpoint{1.606926in}{2.974462in}}{\pgfqpoint{1.606926in}{2.985513in}}%
\pgfpathcurveto{\pgfqpoint{1.606926in}{2.996563in}}{\pgfqpoint{1.602536in}{3.007162in}}{\pgfqpoint{1.594723in}{3.014975in}}%
\pgfpathcurveto{\pgfqpoint{1.586909in}{3.022789in}}{\pgfqpoint{1.576310in}{3.027179in}}{\pgfqpoint{1.565260in}{3.027179in}}%
\pgfpathcurveto{\pgfqpoint{1.554210in}{3.027179in}}{\pgfqpoint{1.543611in}{3.022789in}}{\pgfqpoint{1.535797in}{3.014975in}}%
\pgfpathcurveto{\pgfqpoint{1.527983in}{3.007162in}}{\pgfqpoint{1.523593in}{2.996563in}}{\pgfqpoint{1.523593in}{2.985513in}}%
\pgfpathcurveto{\pgfqpoint{1.523593in}{2.974462in}}{\pgfqpoint{1.527983in}{2.963863in}}{\pgfqpoint{1.535797in}{2.956050in}}%
\pgfpathcurveto{\pgfqpoint{1.543611in}{2.948236in}}{\pgfqpoint{1.554210in}{2.943846in}}{\pgfqpoint{1.565260in}{2.943846in}}%
\pgfpathclose%
\pgfusepath{stroke,fill}%
\end{pgfscope}%
\begin{pgfscope}%
\pgfpathrectangle{\pgfqpoint{0.750000in}{0.500000in}}{\pgfqpoint{4.650000in}{3.020000in}}%
\pgfusepath{clip}%
\pgfsetbuttcap%
\pgfsetroundjoin%
\definecolor{currentfill}{rgb}{1.000000,0.498039,0.054902}%
\pgfsetfillcolor{currentfill}%
\pgfsetlinewidth{1.003750pt}%
\definecolor{currentstroke}{rgb}{1.000000,0.498039,0.054902}%
\pgfsetstrokecolor{currentstroke}%
\pgfsetdash{}{0pt}%
\pgfpathmoveto{\pgfqpoint{1.504870in}{2.932163in}}%
\pgfpathcurveto{\pgfqpoint{1.515920in}{2.932163in}}{\pgfqpoint{1.526519in}{2.936553in}}{\pgfqpoint{1.534333in}{2.944367in}}%
\pgfpathcurveto{\pgfqpoint{1.542147in}{2.952181in}}{\pgfqpoint{1.546537in}{2.962780in}}{\pgfqpoint{1.546537in}{2.973830in}}%
\pgfpathcurveto{\pgfqpoint{1.546537in}{2.984880in}}{\pgfqpoint{1.542147in}{2.995479in}}{\pgfqpoint{1.534333in}{3.003293in}}%
\pgfpathcurveto{\pgfqpoint{1.526519in}{3.011106in}}{\pgfqpoint{1.515920in}{3.015496in}}{\pgfqpoint{1.504870in}{3.015496in}}%
\pgfpathcurveto{\pgfqpoint{1.493820in}{3.015496in}}{\pgfqpoint{1.483221in}{3.011106in}}{\pgfqpoint{1.475407in}{3.003293in}}%
\pgfpathcurveto{\pgfqpoint{1.467594in}{2.995479in}}{\pgfqpoint{1.463203in}{2.984880in}}{\pgfqpoint{1.463203in}{2.973830in}}%
\pgfpathcurveto{\pgfqpoint{1.463203in}{2.962780in}}{\pgfqpoint{1.467594in}{2.952181in}}{\pgfqpoint{1.475407in}{2.944367in}}%
\pgfpathcurveto{\pgfqpoint{1.483221in}{2.936553in}}{\pgfqpoint{1.493820in}{2.932163in}}{\pgfqpoint{1.504870in}{2.932163in}}%
\pgfpathclose%
\pgfusepath{stroke,fill}%
\end{pgfscope}%
\begin{pgfscope}%
\pgfpathrectangle{\pgfqpoint{0.750000in}{0.500000in}}{\pgfqpoint{4.650000in}{3.020000in}}%
\pgfusepath{clip}%
\pgfsetbuttcap%
\pgfsetroundjoin%
\definecolor{currentfill}{rgb}{0.121569,0.466667,0.705882}%
\pgfsetfillcolor{currentfill}%
\pgfsetlinewidth{1.003750pt}%
\definecolor{currentstroke}{rgb}{0.121569,0.466667,0.705882}%
\pgfsetstrokecolor{currentstroke}%
\pgfsetdash{}{0pt}%
\pgfpathmoveto{\pgfqpoint{1.384091in}{0.611183in}}%
\pgfpathcurveto{\pgfqpoint{1.395141in}{0.611183in}}{\pgfqpoint{1.405740in}{0.615573in}}{\pgfqpoint{1.413554in}{0.623387in}}%
\pgfpathcurveto{\pgfqpoint{1.421367in}{0.631201in}}{\pgfqpoint{1.425758in}{0.641800in}}{\pgfqpoint{1.425758in}{0.652850in}}%
\pgfpathcurveto{\pgfqpoint{1.425758in}{0.663900in}}{\pgfqpoint{1.421367in}{0.674499in}}{\pgfqpoint{1.413554in}{0.682313in}}%
\pgfpathcurveto{\pgfqpoint{1.405740in}{0.690126in}}{\pgfqpoint{1.395141in}{0.694516in}}{\pgfqpoint{1.384091in}{0.694516in}}%
\pgfpathcurveto{\pgfqpoint{1.373041in}{0.694516in}}{\pgfqpoint{1.362442in}{0.690126in}}{\pgfqpoint{1.354628in}{0.682313in}}%
\pgfpathcurveto{\pgfqpoint{1.346815in}{0.674499in}}{\pgfqpoint{1.342424in}{0.663900in}}{\pgfqpoint{1.342424in}{0.652850in}}%
\pgfpathcurveto{\pgfqpoint{1.342424in}{0.641800in}}{\pgfqpoint{1.346815in}{0.631201in}}{\pgfqpoint{1.354628in}{0.623387in}}%
\pgfpathcurveto{\pgfqpoint{1.362442in}{0.615573in}}{\pgfqpoint{1.373041in}{0.611183in}}{\pgfqpoint{1.384091in}{0.611183in}}%
\pgfpathclose%
\pgfusepath{stroke,fill}%
\end{pgfscope}%
\begin{pgfscope}%
\pgfpathrectangle{\pgfqpoint{0.750000in}{0.500000in}}{\pgfqpoint{4.650000in}{3.020000in}}%
\pgfusepath{clip}%
\pgfsetbuttcap%
\pgfsetroundjoin%
\definecolor{currentfill}{rgb}{1.000000,0.498039,0.054902}%
\pgfsetfillcolor{currentfill}%
\pgfsetlinewidth{1.003750pt}%
\definecolor{currentstroke}{rgb}{1.000000,0.498039,0.054902}%
\pgfsetstrokecolor{currentstroke}%
\pgfsetdash{}{0pt}%
\pgfpathmoveto{\pgfqpoint{1.927597in}{2.955529in}}%
\pgfpathcurveto{\pgfqpoint{1.938648in}{2.955529in}}{\pgfqpoint{1.949247in}{2.959919in}}{\pgfqpoint{1.957060in}{2.967733in}}%
\pgfpathcurveto{\pgfqpoint{1.964874in}{2.975546in}}{\pgfqpoint{1.969264in}{2.986145in}}{\pgfqpoint{1.969264in}{2.997195in}}%
\pgfpathcurveto{\pgfqpoint{1.969264in}{3.008245in}}{\pgfqpoint{1.964874in}{3.018845in}}{\pgfqpoint{1.957060in}{3.026658in}}%
\pgfpathcurveto{\pgfqpoint{1.949247in}{3.034472in}}{\pgfqpoint{1.938648in}{3.038862in}}{\pgfqpoint{1.927597in}{3.038862in}}%
\pgfpathcurveto{\pgfqpoint{1.916547in}{3.038862in}}{\pgfqpoint{1.905948in}{3.034472in}}{\pgfqpoint{1.898135in}{3.026658in}}%
\pgfpathcurveto{\pgfqpoint{1.890321in}{3.018845in}}{\pgfqpoint{1.885931in}{3.008245in}}{\pgfqpoint{1.885931in}{2.997195in}}%
\pgfpathcurveto{\pgfqpoint{1.885931in}{2.986145in}}{\pgfqpoint{1.890321in}{2.975546in}}{\pgfqpoint{1.898135in}{2.967733in}}%
\pgfpathcurveto{\pgfqpoint{1.905948in}{2.959919in}}{\pgfqpoint{1.916547in}{2.955529in}}{\pgfqpoint{1.927597in}{2.955529in}}%
\pgfpathclose%
\pgfusepath{stroke,fill}%
\end{pgfscope}%
\begin{pgfscope}%
\pgfpathrectangle{\pgfqpoint{0.750000in}{0.500000in}}{\pgfqpoint{4.650000in}{3.020000in}}%
\pgfusepath{clip}%
\pgfsetbuttcap%
\pgfsetroundjoin%
\definecolor{currentfill}{rgb}{1.000000,0.498039,0.054902}%
\pgfsetfillcolor{currentfill}%
\pgfsetlinewidth{1.003750pt}%
\definecolor{currentstroke}{rgb}{1.000000,0.498039,0.054902}%
\pgfsetstrokecolor{currentstroke}%
\pgfsetdash{}{0pt}%
\pgfpathmoveto{\pgfqpoint{3.014610in}{2.928269in}}%
\pgfpathcurveto{\pgfqpoint{3.025661in}{2.928269in}}{\pgfqpoint{3.036260in}{2.932659in}}{\pgfqpoint{3.044073in}{2.940473in}}%
\pgfpathcurveto{\pgfqpoint{3.051887in}{2.948286in}}{\pgfqpoint{3.056277in}{2.958885in}}{\pgfqpoint{3.056277in}{2.969936in}}%
\pgfpathcurveto{\pgfqpoint{3.056277in}{2.980986in}}{\pgfqpoint{3.051887in}{2.991585in}}{\pgfqpoint{3.044073in}{2.999398in}}%
\pgfpathcurveto{\pgfqpoint{3.036260in}{3.007212in}}{\pgfqpoint{3.025661in}{3.011602in}}{\pgfqpoint{3.014610in}{3.011602in}}%
\pgfpathcurveto{\pgfqpoint{3.003560in}{3.011602in}}{\pgfqpoint{2.992961in}{3.007212in}}{\pgfqpoint{2.985148in}{2.999398in}}%
\pgfpathcurveto{\pgfqpoint{2.977334in}{2.991585in}}{\pgfqpoint{2.972944in}{2.980986in}}{\pgfqpoint{2.972944in}{2.969936in}}%
\pgfpathcurveto{\pgfqpoint{2.972944in}{2.958885in}}{\pgfqpoint{2.977334in}{2.948286in}}{\pgfqpoint{2.985148in}{2.940473in}}%
\pgfpathcurveto{\pgfqpoint{2.992961in}{2.932659in}}{\pgfqpoint{3.003560in}{2.928269in}}{\pgfqpoint{3.014610in}{2.928269in}}%
\pgfpathclose%
\pgfusepath{stroke,fill}%
\end{pgfscope}%
\begin{pgfscope}%
\pgfpathrectangle{\pgfqpoint{0.750000in}{0.500000in}}{\pgfqpoint{4.650000in}{3.020000in}}%
\pgfusepath{clip}%
\pgfsetbuttcap%
\pgfsetroundjoin%
\definecolor{currentfill}{rgb}{1.000000,0.498039,0.054902}%
\pgfsetfillcolor{currentfill}%
\pgfsetlinewidth{1.003750pt}%
\definecolor{currentstroke}{rgb}{1.000000,0.498039,0.054902}%
\pgfsetstrokecolor{currentstroke}%
\pgfsetdash{}{0pt}%
\pgfpathmoveto{\pgfqpoint{1.384091in}{2.936057in}}%
\pgfpathcurveto{\pgfqpoint{1.395141in}{2.936057in}}{\pgfqpoint{1.405740in}{2.940448in}}{\pgfqpoint{1.413554in}{2.948261in}}%
\pgfpathcurveto{\pgfqpoint{1.421367in}{2.956075in}}{\pgfqpoint{1.425758in}{2.966674in}}{\pgfqpoint{1.425758in}{2.977724in}}%
\pgfpathcurveto{\pgfqpoint{1.425758in}{2.988774in}}{\pgfqpoint{1.421367in}{2.999373in}}{\pgfqpoint{1.413554in}{3.007187in}}%
\pgfpathcurveto{\pgfqpoint{1.405740in}{3.015000in}}{\pgfqpoint{1.395141in}{3.019391in}}{\pgfqpoint{1.384091in}{3.019391in}}%
\pgfpathcurveto{\pgfqpoint{1.373041in}{3.019391in}}{\pgfqpoint{1.362442in}{3.015000in}}{\pgfqpoint{1.354628in}{3.007187in}}%
\pgfpathcurveto{\pgfqpoint{1.346815in}{2.999373in}}{\pgfqpoint{1.342424in}{2.988774in}}{\pgfqpoint{1.342424in}{2.977724in}}%
\pgfpathcurveto{\pgfqpoint{1.342424in}{2.966674in}}{\pgfqpoint{1.346815in}{2.956075in}}{\pgfqpoint{1.354628in}{2.948261in}}%
\pgfpathcurveto{\pgfqpoint{1.362442in}{2.940448in}}{\pgfqpoint{1.373041in}{2.936057in}}{\pgfqpoint{1.384091in}{2.936057in}}%
\pgfpathclose%
\pgfusepath{stroke,fill}%
\end{pgfscope}%
\begin{pgfscope}%
\pgfpathrectangle{\pgfqpoint{0.750000in}{0.500000in}}{\pgfqpoint{4.650000in}{3.020000in}}%
\pgfusepath{clip}%
\pgfsetbuttcap%
\pgfsetroundjoin%
\definecolor{currentfill}{rgb}{1.000000,0.498039,0.054902}%
\pgfsetfillcolor{currentfill}%
\pgfsetlinewidth{1.003750pt}%
\definecolor{currentstroke}{rgb}{1.000000,0.498039,0.054902}%
\pgfsetstrokecolor{currentstroke}%
\pgfsetdash{}{0pt}%
\pgfpathmoveto{\pgfqpoint{1.323701in}{3.239810in}}%
\pgfpathcurveto{\pgfqpoint{1.334751in}{3.239810in}}{\pgfqpoint{1.345350in}{3.244200in}}{\pgfqpoint{1.353164in}{3.252014in}}%
\pgfpathcurveto{\pgfqpoint{1.360978in}{3.259827in}}{\pgfqpoint{1.365368in}{3.270426in}}{\pgfqpoint{1.365368in}{3.281476in}}%
\pgfpathcurveto{\pgfqpoint{1.365368in}{3.292527in}}{\pgfqpoint{1.360978in}{3.303126in}}{\pgfqpoint{1.353164in}{3.310939in}}%
\pgfpathcurveto{\pgfqpoint{1.345350in}{3.318753in}}{\pgfqpoint{1.334751in}{3.323143in}}{\pgfqpoint{1.323701in}{3.323143in}}%
\pgfpathcurveto{\pgfqpoint{1.312651in}{3.323143in}}{\pgfqpoint{1.302052in}{3.318753in}}{\pgfqpoint{1.294239in}{3.310939in}}%
\pgfpathcurveto{\pgfqpoint{1.286425in}{3.303126in}}{\pgfqpoint{1.282035in}{3.292527in}}{\pgfqpoint{1.282035in}{3.281476in}}%
\pgfpathcurveto{\pgfqpoint{1.282035in}{3.270426in}}{\pgfqpoint{1.286425in}{3.259827in}}{\pgfqpoint{1.294239in}{3.252014in}}%
\pgfpathcurveto{\pgfqpoint{1.302052in}{3.244200in}}{\pgfqpoint{1.312651in}{3.239810in}}{\pgfqpoint{1.323701in}{3.239810in}}%
\pgfpathclose%
\pgfusepath{stroke,fill}%
\end{pgfscope}%
\begin{pgfscope}%
\pgfpathrectangle{\pgfqpoint{0.750000in}{0.500000in}}{\pgfqpoint{4.650000in}{3.020000in}}%
\pgfusepath{clip}%
\pgfsetbuttcap%
\pgfsetroundjoin%
\definecolor{currentfill}{rgb}{1.000000,0.498039,0.054902}%
\pgfsetfillcolor{currentfill}%
\pgfsetlinewidth{1.003750pt}%
\definecolor{currentstroke}{rgb}{1.000000,0.498039,0.054902}%
\pgfsetstrokecolor{currentstroke}%
\pgfsetdash{}{0pt}%
\pgfpathmoveto{\pgfqpoint{1.625649in}{2.936057in}}%
\pgfpathcurveto{\pgfqpoint{1.636699in}{2.936057in}}{\pgfqpoint{1.647299in}{2.940448in}}{\pgfqpoint{1.655112in}{2.948261in}}%
\pgfpathcurveto{\pgfqpoint{1.662926in}{2.956075in}}{\pgfqpoint{1.667316in}{2.966674in}}{\pgfqpoint{1.667316in}{2.977724in}}%
\pgfpathcurveto{\pgfqpoint{1.667316in}{2.988774in}}{\pgfqpoint{1.662926in}{2.999373in}}{\pgfqpoint{1.655112in}{3.007187in}}%
\pgfpathcurveto{\pgfqpoint{1.647299in}{3.015000in}}{\pgfqpoint{1.636699in}{3.019391in}}{\pgfqpoint{1.625649in}{3.019391in}}%
\pgfpathcurveto{\pgfqpoint{1.614599in}{3.019391in}}{\pgfqpoint{1.604000in}{3.015000in}}{\pgfqpoint{1.596187in}{3.007187in}}%
\pgfpathcurveto{\pgfqpoint{1.588373in}{2.999373in}}{\pgfqpoint{1.583983in}{2.988774in}}{\pgfqpoint{1.583983in}{2.977724in}}%
\pgfpathcurveto{\pgfqpoint{1.583983in}{2.966674in}}{\pgfqpoint{1.588373in}{2.956075in}}{\pgfqpoint{1.596187in}{2.948261in}}%
\pgfpathcurveto{\pgfqpoint{1.604000in}{2.940448in}}{\pgfqpoint{1.614599in}{2.936057in}}{\pgfqpoint{1.625649in}{2.936057in}}%
\pgfpathclose%
\pgfusepath{stroke,fill}%
\end{pgfscope}%
\begin{pgfscope}%
\pgfpathrectangle{\pgfqpoint{0.750000in}{0.500000in}}{\pgfqpoint{4.650000in}{3.020000in}}%
\pgfusepath{clip}%
\pgfsetbuttcap%
\pgfsetroundjoin%
\definecolor{currentfill}{rgb}{0.121569,0.466667,0.705882}%
\pgfsetfillcolor{currentfill}%
\pgfsetlinewidth{1.003750pt}%
\definecolor{currentstroke}{rgb}{0.121569,0.466667,0.705882}%
\pgfsetstrokecolor{currentstroke}%
\pgfsetdash{}{0pt}%
\pgfpathmoveto{\pgfqpoint{1.263312in}{0.595606in}}%
\pgfpathcurveto{\pgfqpoint{1.274362in}{0.595606in}}{\pgfqpoint{1.284961in}{0.599996in}}{\pgfqpoint{1.292774in}{0.607810in}}%
\pgfpathcurveto{\pgfqpoint{1.300588in}{0.615624in}}{\pgfqpoint{1.304978in}{0.626223in}}{\pgfqpoint{1.304978in}{0.637273in}}%
\pgfpathcurveto{\pgfqpoint{1.304978in}{0.648323in}}{\pgfqpoint{1.300588in}{0.658922in}}{\pgfqpoint{1.292774in}{0.666736in}}%
\pgfpathcurveto{\pgfqpoint{1.284961in}{0.674549in}}{\pgfqpoint{1.274362in}{0.678939in}}{\pgfqpoint{1.263312in}{0.678939in}}%
\pgfpathcurveto{\pgfqpoint{1.252262in}{0.678939in}}{\pgfqpoint{1.241663in}{0.674549in}}{\pgfqpoint{1.233849in}{0.666736in}}%
\pgfpathcurveto{\pgfqpoint{1.226035in}{0.658922in}}{\pgfqpoint{1.221645in}{0.648323in}}{\pgfqpoint{1.221645in}{0.637273in}}%
\pgfpathcurveto{\pgfqpoint{1.221645in}{0.626223in}}{\pgfqpoint{1.226035in}{0.615624in}}{\pgfqpoint{1.233849in}{0.607810in}}%
\pgfpathcurveto{\pgfqpoint{1.241663in}{0.599996in}}{\pgfqpoint{1.252262in}{0.595606in}}{\pgfqpoint{1.263312in}{0.595606in}}%
\pgfpathclose%
\pgfusepath{stroke,fill}%
\end{pgfscope}%
\begin{pgfscope}%
\pgfpathrectangle{\pgfqpoint{0.750000in}{0.500000in}}{\pgfqpoint{4.650000in}{3.020000in}}%
\pgfusepath{clip}%
\pgfsetbuttcap%
\pgfsetroundjoin%
\definecolor{currentfill}{rgb}{0.121569,0.466667,0.705882}%
\pgfsetfillcolor{currentfill}%
\pgfsetlinewidth{1.003750pt}%
\definecolor{currentstroke}{rgb}{0.121569,0.466667,0.705882}%
\pgfsetstrokecolor{currentstroke}%
\pgfsetdash{}{0pt}%
\pgfpathmoveto{\pgfqpoint{1.384091in}{0.595606in}}%
\pgfpathcurveto{\pgfqpoint{1.395141in}{0.595606in}}{\pgfqpoint{1.405740in}{0.599996in}}{\pgfqpoint{1.413554in}{0.607810in}}%
\pgfpathcurveto{\pgfqpoint{1.421367in}{0.615624in}}{\pgfqpoint{1.425758in}{0.626223in}}{\pgfqpoint{1.425758in}{0.637273in}}%
\pgfpathcurveto{\pgfqpoint{1.425758in}{0.648323in}}{\pgfqpoint{1.421367in}{0.658922in}}{\pgfqpoint{1.413554in}{0.666736in}}%
\pgfpathcurveto{\pgfqpoint{1.405740in}{0.674549in}}{\pgfqpoint{1.395141in}{0.678939in}}{\pgfqpoint{1.384091in}{0.678939in}}%
\pgfpathcurveto{\pgfqpoint{1.373041in}{0.678939in}}{\pgfqpoint{1.362442in}{0.674549in}}{\pgfqpoint{1.354628in}{0.666736in}}%
\pgfpathcurveto{\pgfqpoint{1.346815in}{0.658922in}}{\pgfqpoint{1.342424in}{0.648323in}}{\pgfqpoint{1.342424in}{0.637273in}}%
\pgfpathcurveto{\pgfqpoint{1.342424in}{0.626223in}}{\pgfqpoint{1.346815in}{0.615624in}}{\pgfqpoint{1.354628in}{0.607810in}}%
\pgfpathcurveto{\pgfqpoint{1.362442in}{0.599996in}}{\pgfqpoint{1.373041in}{0.595606in}}{\pgfqpoint{1.384091in}{0.595606in}}%
\pgfpathclose%
\pgfusepath{stroke,fill}%
\end{pgfscope}%
\begin{pgfscope}%
\pgfpathrectangle{\pgfqpoint{0.750000in}{0.500000in}}{\pgfqpoint{4.650000in}{3.020000in}}%
\pgfusepath{clip}%
\pgfsetbuttcap%
\pgfsetroundjoin%
\definecolor{currentfill}{rgb}{0.121569,0.466667,0.705882}%
\pgfsetfillcolor{currentfill}%
\pgfsetlinewidth{1.003750pt}%
\definecolor{currentstroke}{rgb}{0.121569,0.466667,0.705882}%
\pgfsetstrokecolor{currentstroke}%
\pgfsetdash{}{0pt}%
\pgfpathmoveto{\pgfqpoint{1.867208in}{0.665703in}}%
\pgfpathcurveto{\pgfqpoint{1.878258in}{0.665703in}}{\pgfqpoint{1.888857in}{0.670093in}}{\pgfqpoint{1.896671in}{0.677907in}}%
\pgfpathcurveto{\pgfqpoint{1.904484in}{0.685720in}}{\pgfqpoint{1.908874in}{0.696319in}}{\pgfqpoint{1.908874in}{0.707369in}}%
\pgfpathcurveto{\pgfqpoint{1.908874in}{0.718420in}}{\pgfqpoint{1.904484in}{0.729019in}}{\pgfqpoint{1.896671in}{0.736832in}}%
\pgfpathcurveto{\pgfqpoint{1.888857in}{0.744646in}}{\pgfqpoint{1.878258in}{0.749036in}}{\pgfqpoint{1.867208in}{0.749036in}}%
\pgfpathcurveto{\pgfqpoint{1.856158in}{0.749036in}}{\pgfqpoint{1.845559in}{0.744646in}}{\pgfqpoint{1.837745in}{0.736832in}}%
\pgfpathcurveto{\pgfqpoint{1.829931in}{0.729019in}}{\pgfqpoint{1.825541in}{0.718420in}}{\pgfqpoint{1.825541in}{0.707369in}}%
\pgfpathcurveto{\pgfqpoint{1.825541in}{0.696319in}}{\pgfqpoint{1.829931in}{0.685720in}}{\pgfqpoint{1.837745in}{0.677907in}}%
\pgfpathcurveto{\pgfqpoint{1.845559in}{0.670093in}}{\pgfqpoint{1.856158in}{0.665703in}}{\pgfqpoint{1.867208in}{0.665703in}}%
\pgfpathclose%
\pgfusepath{stroke,fill}%
\end{pgfscope}%
\begin{pgfscope}%
\pgfpathrectangle{\pgfqpoint{0.750000in}{0.500000in}}{\pgfqpoint{4.650000in}{3.020000in}}%
\pgfusepath{clip}%
\pgfsetbuttcap%
\pgfsetroundjoin%
\definecolor{currentfill}{rgb}{1.000000,0.498039,0.054902}%
\pgfsetfillcolor{currentfill}%
\pgfsetlinewidth{1.003750pt}%
\definecolor{currentstroke}{rgb}{1.000000,0.498039,0.054902}%
\pgfsetstrokecolor{currentstroke}%
\pgfsetdash{}{0pt}%
\pgfpathmoveto{\pgfqpoint{1.444481in}{2.932163in}}%
\pgfpathcurveto{\pgfqpoint{1.455531in}{2.932163in}}{\pgfqpoint{1.466130in}{2.936553in}}{\pgfqpoint{1.473943in}{2.944367in}}%
\pgfpathcurveto{\pgfqpoint{1.481757in}{2.952181in}}{\pgfqpoint{1.486147in}{2.962780in}}{\pgfqpoint{1.486147in}{2.973830in}}%
\pgfpathcurveto{\pgfqpoint{1.486147in}{2.984880in}}{\pgfqpoint{1.481757in}{2.995479in}}{\pgfqpoint{1.473943in}{3.003293in}}%
\pgfpathcurveto{\pgfqpoint{1.466130in}{3.011106in}}{\pgfqpoint{1.455531in}{3.015496in}}{\pgfqpoint{1.444481in}{3.015496in}}%
\pgfpathcurveto{\pgfqpoint{1.433430in}{3.015496in}}{\pgfqpoint{1.422831in}{3.011106in}}{\pgfqpoint{1.415018in}{3.003293in}}%
\pgfpathcurveto{\pgfqpoint{1.407204in}{2.995479in}}{\pgfqpoint{1.402814in}{2.984880in}}{\pgfqpoint{1.402814in}{2.973830in}}%
\pgfpathcurveto{\pgfqpoint{1.402814in}{2.962780in}}{\pgfqpoint{1.407204in}{2.952181in}}{\pgfqpoint{1.415018in}{2.944367in}}%
\pgfpathcurveto{\pgfqpoint{1.422831in}{2.936553in}}{\pgfqpoint{1.433430in}{2.932163in}}{\pgfqpoint{1.444481in}{2.932163in}}%
\pgfpathclose%
\pgfusepath{stroke,fill}%
\end{pgfscope}%
\begin{pgfscope}%
\pgfpathrectangle{\pgfqpoint{0.750000in}{0.500000in}}{\pgfqpoint{4.650000in}{3.020000in}}%
\pgfusepath{clip}%
\pgfsetbuttcap%
\pgfsetroundjoin%
\definecolor{currentfill}{rgb}{0.839216,0.152941,0.156863}%
\pgfsetfillcolor{currentfill}%
\pgfsetlinewidth{1.003750pt}%
\definecolor{currentstroke}{rgb}{0.839216,0.152941,0.156863}%
\pgfsetstrokecolor{currentstroke}%
\pgfsetdash{}{0pt}%
\pgfpathmoveto{\pgfqpoint{2.169156in}{1.086283in}}%
\pgfpathcurveto{\pgfqpoint{2.180206in}{1.086283in}}{\pgfqpoint{2.190805in}{1.090673in}}{\pgfqpoint{2.198619in}{1.098487in}}%
\pgfpathcurveto{\pgfqpoint{2.206432in}{1.106301in}}{\pgfqpoint{2.210823in}{1.116900in}}{\pgfqpoint{2.210823in}{1.127950in}}%
\pgfpathcurveto{\pgfqpoint{2.210823in}{1.139000in}}{\pgfqpoint{2.206432in}{1.149599in}}{\pgfqpoint{2.198619in}{1.157412in}}%
\pgfpathcurveto{\pgfqpoint{2.190805in}{1.165226in}}{\pgfqpoint{2.180206in}{1.169616in}}{\pgfqpoint{2.169156in}{1.169616in}}%
\pgfpathcurveto{\pgfqpoint{2.158106in}{1.169616in}}{\pgfqpoint{2.147507in}{1.165226in}}{\pgfqpoint{2.139693in}{1.157412in}}%
\pgfpathcurveto{\pgfqpoint{2.131879in}{1.149599in}}{\pgfqpoint{2.127489in}{1.139000in}}{\pgfqpoint{2.127489in}{1.127950in}}%
\pgfpathcurveto{\pgfqpoint{2.127489in}{1.116900in}}{\pgfqpoint{2.131879in}{1.106301in}}{\pgfqpoint{2.139693in}{1.098487in}}%
\pgfpathcurveto{\pgfqpoint{2.147507in}{1.090673in}}{\pgfqpoint{2.158106in}{1.086283in}}{\pgfqpoint{2.169156in}{1.086283in}}%
\pgfpathclose%
\pgfusepath{stroke,fill}%
\end{pgfscope}%
\begin{pgfscope}%
\pgfpathrectangle{\pgfqpoint{0.750000in}{0.500000in}}{\pgfqpoint{4.650000in}{3.020000in}}%
\pgfusepath{clip}%
\pgfsetbuttcap%
\pgfsetroundjoin%
\definecolor{currentfill}{rgb}{1.000000,0.498039,0.054902}%
\pgfsetfillcolor{currentfill}%
\pgfsetlinewidth{1.003750pt}%
\definecolor{currentstroke}{rgb}{1.000000,0.498039,0.054902}%
\pgfsetstrokecolor{currentstroke}%
\pgfsetdash{}{0pt}%
\pgfpathmoveto{\pgfqpoint{2.048377in}{2.932163in}}%
\pgfpathcurveto{\pgfqpoint{2.059427in}{2.932163in}}{\pgfqpoint{2.070026in}{2.936553in}}{\pgfqpoint{2.077839in}{2.944367in}}%
\pgfpathcurveto{\pgfqpoint{2.085653in}{2.952181in}}{\pgfqpoint{2.090043in}{2.962780in}}{\pgfqpoint{2.090043in}{2.973830in}}%
\pgfpathcurveto{\pgfqpoint{2.090043in}{2.984880in}}{\pgfqpoint{2.085653in}{2.995479in}}{\pgfqpoint{2.077839in}{3.003293in}}%
\pgfpathcurveto{\pgfqpoint{2.070026in}{3.011106in}}{\pgfqpoint{2.059427in}{3.015496in}}{\pgfqpoint{2.048377in}{3.015496in}}%
\pgfpathcurveto{\pgfqpoint{2.037326in}{3.015496in}}{\pgfqpoint{2.026727in}{3.011106in}}{\pgfqpoint{2.018914in}{3.003293in}}%
\pgfpathcurveto{\pgfqpoint{2.011100in}{2.995479in}}{\pgfqpoint{2.006710in}{2.984880in}}{\pgfqpoint{2.006710in}{2.973830in}}%
\pgfpathcurveto{\pgfqpoint{2.006710in}{2.962780in}}{\pgfqpoint{2.011100in}{2.952181in}}{\pgfqpoint{2.018914in}{2.944367in}}%
\pgfpathcurveto{\pgfqpoint{2.026727in}{2.936553in}}{\pgfqpoint{2.037326in}{2.932163in}}{\pgfqpoint{2.048377in}{2.932163in}}%
\pgfpathclose%
\pgfusepath{stroke,fill}%
\end{pgfscope}%
\begin{pgfscope}%
\pgfpathrectangle{\pgfqpoint{0.750000in}{0.500000in}}{\pgfqpoint{4.650000in}{3.020000in}}%
\pgfusepath{clip}%
\pgfsetbuttcap%
\pgfsetroundjoin%
\definecolor{currentfill}{rgb}{1.000000,0.498039,0.054902}%
\pgfsetfillcolor{currentfill}%
\pgfsetlinewidth{1.003750pt}%
\definecolor{currentstroke}{rgb}{1.000000,0.498039,0.054902}%
\pgfsetstrokecolor{currentstroke}%
\pgfsetdash{}{0pt}%
\pgfpathmoveto{\pgfqpoint{1.987987in}{2.932163in}}%
\pgfpathcurveto{\pgfqpoint{1.999037in}{2.932163in}}{\pgfqpoint{2.009636in}{2.936553in}}{\pgfqpoint{2.017450in}{2.944367in}}%
\pgfpathcurveto{\pgfqpoint{2.025263in}{2.952181in}}{\pgfqpoint{2.029654in}{2.962780in}}{\pgfqpoint{2.029654in}{2.973830in}}%
\pgfpathcurveto{\pgfqpoint{2.029654in}{2.984880in}}{\pgfqpoint{2.025263in}{2.995479in}}{\pgfqpoint{2.017450in}{3.003293in}}%
\pgfpathcurveto{\pgfqpoint{2.009636in}{3.011106in}}{\pgfqpoint{1.999037in}{3.015496in}}{\pgfqpoint{1.987987in}{3.015496in}}%
\pgfpathcurveto{\pgfqpoint{1.976937in}{3.015496in}}{\pgfqpoint{1.966338in}{3.011106in}}{\pgfqpoint{1.958524in}{3.003293in}}%
\pgfpathcurveto{\pgfqpoint{1.950711in}{2.995479in}}{\pgfqpoint{1.946320in}{2.984880in}}{\pgfqpoint{1.946320in}{2.973830in}}%
\pgfpathcurveto{\pgfqpoint{1.946320in}{2.962780in}}{\pgfqpoint{1.950711in}{2.952181in}}{\pgfqpoint{1.958524in}{2.944367in}}%
\pgfpathcurveto{\pgfqpoint{1.966338in}{2.936553in}}{\pgfqpoint{1.976937in}{2.932163in}}{\pgfqpoint{1.987987in}{2.932163in}}%
\pgfpathclose%
\pgfusepath{stroke,fill}%
\end{pgfscope}%
\begin{pgfscope}%
\pgfpathrectangle{\pgfqpoint{0.750000in}{0.500000in}}{\pgfqpoint{4.650000in}{3.020000in}}%
\pgfusepath{clip}%
\pgfsetbuttcap%
\pgfsetroundjoin%
\definecolor{currentfill}{rgb}{1.000000,0.498039,0.054902}%
\pgfsetfillcolor{currentfill}%
\pgfsetlinewidth{1.003750pt}%
\definecolor{currentstroke}{rgb}{1.000000,0.498039,0.054902}%
\pgfsetstrokecolor{currentstroke}%
\pgfsetdash{}{0pt}%
\pgfpathmoveto{\pgfqpoint{1.806818in}{2.928269in}}%
\pgfpathcurveto{\pgfqpoint{1.817868in}{2.928269in}}{\pgfqpoint{1.828467in}{2.932659in}}{\pgfqpoint{1.836281in}{2.940473in}}%
\pgfpathcurveto{\pgfqpoint{1.844095in}{2.948286in}}{\pgfqpoint{1.848485in}{2.958885in}}{\pgfqpoint{1.848485in}{2.969936in}}%
\pgfpathcurveto{\pgfqpoint{1.848485in}{2.980986in}}{\pgfqpoint{1.844095in}{2.991585in}}{\pgfqpoint{1.836281in}{2.999398in}}%
\pgfpathcurveto{\pgfqpoint{1.828467in}{3.007212in}}{\pgfqpoint{1.817868in}{3.011602in}}{\pgfqpoint{1.806818in}{3.011602in}}%
\pgfpathcurveto{\pgfqpoint{1.795768in}{3.011602in}}{\pgfqpoint{1.785169in}{3.007212in}}{\pgfqpoint{1.777355in}{2.999398in}}%
\pgfpathcurveto{\pgfqpoint{1.769542in}{2.991585in}}{\pgfqpoint{1.765152in}{2.980986in}}{\pgfqpoint{1.765152in}{2.969936in}}%
\pgfpathcurveto{\pgfqpoint{1.765152in}{2.958885in}}{\pgfqpoint{1.769542in}{2.948286in}}{\pgfqpoint{1.777355in}{2.940473in}}%
\pgfpathcurveto{\pgfqpoint{1.785169in}{2.932659in}}{\pgfqpoint{1.795768in}{2.928269in}}{\pgfqpoint{1.806818in}{2.928269in}}%
\pgfpathclose%
\pgfusepath{stroke,fill}%
\end{pgfscope}%
\begin{pgfscope}%
\pgfpathrectangle{\pgfqpoint{0.750000in}{0.500000in}}{\pgfqpoint{4.650000in}{3.020000in}}%
\pgfusepath{clip}%
\pgfsetbuttcap%
\pgfsetroundjoin%
\definecolor{currentfill}{rgb}{1.000000,0.498039,0.054902}%
\pgfsetfillcolor{currentfill}%
\pgfsetlinewidth{1.003750pt}%
\definecolor{currentstroke}{rgb}{1.000000,0.498039,0.054902}%
\pgfsetstrokecolor{currentstroke}%
\pgfsetdash{}{0pt}%
\pgfpathmoveto{\pgfqpoint{1.867208in}{2.936057in}}%
\pgfpathcurveto{\pgfqpoint{1.878258in}{2.936057in}}{\pgfqpoint{1.888857in}{2.940448in}}{\pgfqpoint{1.896671in}{2.948261in}}%
\pgfpathcurveto{\pgfqpoint{1.904484in}{2.956075in}}{\pgfqpoint{1.908874in}{2.966674in}}{\pgfqpoint{1.908874in}{2.977724in}}%
\pgfpathcurveto{\pgfqpoint{1.908874in}{2.988774in}}{\pgfqpoint{1.904484in}{2.999373in}}{\pgfqpoint{1.896671in}{3.007187in}}%
\pgfpathcurveto{\pgfqpoint{1.888857in}{3.015000in}}{\pgfqpoint{1.878258in}{3.019391in}}{\pgfqpoint{1.867208in}{3.019391in}}%
\pgfpathcurveto{\pgfqpoint{1.856158in}{3.019391in}}{\pgfqpoint{1.845559in}{3.015000in}}{\pgfqpoint{1.837745in}{3.007187in}}%
\pgfpathcurveto{\pgfqpoint{1.829931in}{2.999373in}}{\pgfqpoint{1.825541in}{2.988774in}}{\pgfqpoint{1.825541in}{2.977724in}}%
\pgfpathcurveto{\pgfqpoint{1.825541in}{2.966674in}}{\pgfqpoint{1.829931in}{2.956075in}}{\pgfqpoint{1.837745in}{2.948261in}}%
\pgfpathcurveto{\pgfqpoint{1.845559in}{2.940448in}}{\pgfqpoint{1.856158in}{2.936057in}}{\pgfqpoint{1.867208in}{2.936057in}}%
\pgfpathclose%
\pgfusepath{stroke,fill}%
\end{pgfscope}%
\begin{pgfscope}%
\pgfpathrectangle{\pgfqpoint{0.750000in}{0.500000in}}{\pgfqpoint{4.650000in}{3.020000in}}%
\pgfusepath{clip}%
\pgfsetbuttcap%
\pgfsetroundjoin%
\definecolor{currentfill}{rgb}{1.000000,0.498039,0.054902}%
\pgfsetfillcolor{currentfill}%
\pgfsetlinewidth{1.003750pt}%
\definecolor{currentstroke}{rgb}{1.000000,0.498039,0.054902}%
\pgfsetstrokecolor{currentstroke}%
\pgfsetdash{}{0pt}%
\pgfpathmoveto{\pgfqpoint{1.867208in}{2.939952in}}%
\pgfpathcurveto{\pgfqpoint{1.878258in}{2.939952in}}{\pgfqpoint{1.888857in}{2.944342in}}{\pgfqpoint{1.896671in}{2.952156in}}%
\pgfpathcurveto{\pgfqpoint{1.904484in}{2.959969in}}{\pgfqpoint{1.908874in}{2.970568in}}{\pgfqpoint{1.908874in}{2.981618in}}%
\pgfpathcurveto{\pgfqpoint{1.908874in}{2.992668in}}{\pgfqpoint{1.904484in}{3.003267in}}{\pgfqpoint{1.896671in}{3.011081in}}%
\pgfpathcurveto{\pgfqpoint{1.888857in}{3.018895in}}{\pgfqpoint{1.878258in}{3.023285in}}{\pgfqpoint{1.867208in}{3.023285in}}%
\pgfpathcurveto{\pgfqpoint{1.856158in}{3.023285in}}{\pgfqpoint{1.845559in}{3.018895in}}{\pgfqpoint{1.837745in}{3.011081in}}%
\pgfpathcurveto{\pgfqpoint{1.829931in}{3.003267in}}{\pgfqpoint{1.825541in}{2.992668in}}{\pgfqpoint{1.825541in}{2.981618in}}%
\pgfpathcurveto{\pgfqpoint{1.825541in}{2.970568in}}{\pgfqpoint{1.829931in}{2.959969in}}{\pgfqpoint{1.837745in}{2.952156in}}%
\pgfpathcurveto{\pgfqpoint{1.845559in}{2.944342in}}{\pgfqpoint{1.856158in}{2.939952in}}{\pgfqpoint{1.867208in}{2.939952in}}%
\pgfpathclose%
\pgfusepath{stroke,fill}%
\end{pgfscope}%
\begin{pgfscope}%
\pgfpathrectangle{\pgfqpoint{0.750000in}{0.500000in}}{\pgfqpoint{4.650000in}{3.020000in}}%
\pgfusepath{clip}%
\pgfsetbuttcap%
\pgfsetroundjoin%
\definecolor{currentfill}{rgb}{1.000000,0.498039,0.054902}%
\pgfsetfillcolor{currentfill}%
\pgfsetlinewidth{1.003750pt}%
\definecolor{currentstroke}{rgb}{1.000000,0.498039,0.054902}%
\pgfsetstrokecolor{currentstroke}%
\pgfsetdash{}{0pt}%
\pgfpathmoveto{\pgfqpoint{1.867208in}{2.936057in}}%
\pgfpathcurveto{\pgfqpoint{1.878258in}{2.936057in}}{\pgfqpoint{1.888857in}{2.940448in}}{\pgfqpoint{1.896671in}{2.948261in}}%
\pgfpathcurveto{\pgfqpoint{1.904484in}{2.956075in}}{\pgfqpoint{1.908874in}{2.966674in}}{\pgfqpoint{1.908874in}{2.977724in}}%
\pgfpathcurveto{\pgfqpoint{1.908874in}{2.988774in}}{\pgfqpoint{1.904484in}{2.999373in}}{\pgfqpoint{1.896671in}{3.007187in}}%
\pgfpathcurveto{\pgfqpoint{1.888857in}{3.015000in}}{\pgfqpoint{1.878258in}{3.019391in}}{\pgfqpoint{1.867208in}{3.019391in}}%
\pgfpathcurveto{\pgfqpoint{1.856158in}{3.019391in}}{\pgfqpoint{1.845559in}{3.015000in}}{\pgfqpoint{1.837745in}{3.007187in}}%
\pgfpathcurveto{\pgfqpoint{1.829931in}{2.999373in}}{\pgfqpoint{1.825541in}{2.988774in}}{\pgfqpoint{1.825541in}{2.977724in}}%
\pgfpathcurveto{\pgfqpoint{1.825541in}{2.966674in}}{\pgfqpoint{1.829931in}{2.956075in}}{\pgfqpoint{1.837745in}{2.948261in}}%
\pgfpathcurveto{\pgfqpoint{1.845559in}{2.940448in}}{\pgfqpoint{1.856158in}{2.936057in}}{\pgfqpoint{1.867208in}{2.936057in}}%
\pgfpathclose%
\pgfusepath{stroke,fill}%
\end{pgfscope}%
\begin{pgfscope}%
\pgfpathrectangle{\pgfqpoint{0.750000in}{0.500000in}}{\pgfqpoint{4.650000in}{3.020000in}}%
\pgfusepath{clip}%
\pgfsetbuttcap%
\pgfsetroundjoin%
\definecolor{currentfill}{rgb}{1.000000,0.498039,0.054902}%
\pgfsetfillcolor{currentfill}%
\pgfsetlinewidth{1.003750pt}%
\definecolor{currentstroke}{rgb}{1.000000,0.498039,0.054902}%
\pgfsetstrokecolor{currentstroke}%
\pgfsetdash{}{0pt}%
\pgfpathmoveto{\pgfqpoint{2.773052in}{2.936057in}}%
\pgfpathcurveto{\pgfqpoint{2.784102in}{2.936057in}}{\pgfqpoint{2.794701in}{2.940448in}}{\pgfqpoint{2.802515in}{2.948261in}}%
\pgfpathcurveto{\pgfqpoint{2.810328in}{2.956075in}}{\pgfqpoint{2.814719in}{2.966674in}}{\pgfqpoint{2.814719in}{2.977724in}}%
\pgfpathcurveto{\pgfqpoint{2.814719in}{2.988774in}}{\pgfqpoint{2.810328in}{2.999373in}}{\pgfqpoint{2.802515in}{3.007187in}}%
\pgfpathcurveto{\pgfqpoint{2.794701in}{3.015000in}}{\pgfqpoint{2.784102in}{3.019391in}}{\pgfqpoint{2.773052in}{3.019391in}}%
\pgfpathcurveto{\pgfqpoint{2.762002in}{3.019391in}}{\pgfqpoint{2.751403in}{3.015000in}}{\pgfqpoint{2.743589in}{3.007187in}}%
\pgfpathcurveto{\pgfqpoint{2.735776in}{2.999373in}}{\pgfqpoint{2.731385in}{2.988774in}}{\pgfqpoint{2.731385in}{2.977724in}}%
\pgfpathcurveto{\pgfqpoint{2.731385in}{2.966674in}}{\pgfqpoint{2.735776in}{2.956075in}}{\pgfqpoint{2.743589in}{2.948261in}}%
\pgfpathcurveto{\pgfqpoint{2.751403in}{2.940448in}}{\pgfqpoint{2.762002in}{2.936057in}}{\pgfqpoint{2.773052in}{2.936057in}}%
\pgfpathclose%
\pgfusepath{stroke,fill}%
\end{pgfscope}%
\begin{pgfscope}%
\pgfpathrectangle{\pgfqpoint{0.750000in}{0.500000in}}{\pgfqpoint{4.650000in}{3.020000in}}%
\pgfusepath{clip}%
\pgfsetbuttcap%
\pgfsetroundjoin%
\definecolor{currentfill}{rgb}{1.000000,0.498039,0.054902}%
\pgfsetfillcolor{currentfill}%
\pgfsetlinewidth{1.003750pt}%
\definecolor{currentstroke}{rgb}{1.000000,0.498039,0.054902}%
\pgfsetstrokecolor{currentstroke}%
\pgfsetdash{}{0pt}%
\pgfpathmoveto{\pgfqpoint{3.980844in}{2.924375in}}%
\pgfpathcurveto{\pgfqpoint{3.991894in}{2.924375in}}{\pgfqpoint{4.002493in}{2.928765in}}{\pgfqpoint{4.010307in}{2.936578in}}%
\pgfpathcurveto{\pgfqpoint{4.018121in}{2.944392in}}{\pgfqpoint{4.022511in}{2.954991in}}{\pgfqpoint{4.022511in}{2.966041in}}%
\pgfpathcurveto{\pgfqpoint{4.022511in}{2.977091in}}{\pgfqpoint{4.018121in}{2.987690in}}{\pgfqpoint{4.010307in}{2.995504in}}%
\pgfpathcurveto{\pgfqpoint{4.002493in}{3.003318in}}{\pgfqpoint{3.991894in}{3.007708in}}{\pgfqpoint{3.980844in}{3.007708in}}%
\pgfpathcurveto{\pgfqpoint{3.969794in}{3.007708in}}{\pgfqpoint{3.959195in}{3.003318in}}{\pgfqpoint{3.951381in}{2.995504in}}%
\pgfpathcurveto{\pgfqpoint{3.943568in}{2.987690in}}{\pgfqpoint{3.939177in}{2.977091in}}{\pgfqpoint{3.939177in}{2.966041in}}%
\pgfpathcurveto{\pgfqpoint{3.939177in}{2.954991in}}{\pgfqpoint{3.943568in}{2.944392in}}{\pgfqpoint{3.951381in}{2.936578in}}%
\pgfpathcurveto{\pgfqpoint{3.959195in}{2.928765in}}{\pgfqpoint{3.969794in}{2.924375in}}{\pgfqpoint{3.980844in}{2.924375in}}%
\pgfpathclose%
\pgfusepath{stroke,fill}%
\end{pgfscope}%
\begin{pgfscope}%
\pgfpathrectangle{\pgfqpoint{0.750000in}{0.500000in}}{\pgfqpoint{4.650000in}{3.020000in}}%
\pgfusepath{clip}%
\pgfsetbuttcap%
\pgfsetroundjoin%
\definecolor{currentfill}{rgb}{1.000000,0.498039,0.054902}%
\pgfsetfillcolor{currentfill}%
\pgfsetlinewidth{1.003750pt}%
\definecolor{currentstroke}{rgb}{1.000000,0.498039,0.054902}%
\pgfsetstrokecolor{currentstroke}%
\pgfsetdash{}{0pt}%
\pgfpathmoveto{\pgfqpoint{1.867208in}{2.932163in}}%
\pgfpathcurveto{\pgfqpoint{1.878258in}{2.932163in}}{\pgfqpoint{1.888857in}{2.936553in}}{\pgfqpoint{1.896671in}{2.944367in}}%
\pgfpathcurveto{\pgfqpoint{1.904484in}{2.952181in}}{\pgfqpoint{1.908874in}{2.962780in}}{\pgfqpoint{1.908874in}{2.973830in}}%
\pgfpathcurveto{\pgfqpoint{1.908874in}{2.984880in}}{\pgfqpoint{1.904484in}{2.995479in}}{\pgfqpoint{1.896671in}{3.003293in}}%
\pgfpathcurveto{\pgfqpoint{1.888857in}{3.011106in}}{\pgfqpoint{1.878258in}{3.015496in}}{\pgfqpoint{1.867208in}{3.015496in}}%
\pgfpathcurveto{\pgfqpoint{1.856158in}{3.015496in}}{\pgfqpoint{1.845559in}{3.011106in}}{\pgfqpoint{1.837745in}{3.003293in}}%
\pgfpathcurveto{\pgfqpoint{1.829931in}{2.995479in}}{\pgfqpoint{1.825541in}{2.984880in}}{\pgfqpoint{1.825541in}{2.973830in}}%
\pgfpathcurveto{\pgfqpoint{1.825541in}{2.962780in}}{\pgfqpoint{1.829931in}{2.952181in}}{\pgfqpoint{1.837745in}{2.944367in}}%
\pgfpathcurveto{\pgfqpoint{1.845559in}{2.936553in}}{\pgfqpoint{1.856158in}{2.932163in}}{\pgfqpoint{1.867208in}{2.932163in}}%
\pgfpathclose%
\pgfusepath{stroke,fill}%
\end{pgfscope}%
\begin{pgfscope}%
\pgfpathrectangle{\pgfqpoint{0.750000in}{0.500000in}}{\pgfqpoint{4.650000in}{3.020000in}}%
\pgfusepath{clip}%
\pgfsetbuttcap%
\pgfsetroundjoin%
\definecolor{currentfill}{rgb}{1.000000,0.498039,0.054902}%
\pgfsetfillcolor{currentfill}%
\pgfsetlinewidth{1.003750pt}%
\definecolor{currentstroke}{rgb}{1.000000,0.498039,0.054902}%
\pgfsetstrokecolor{currentstroke}%
\pgfsetdash{}{0pt}%
\pgfpathmoveto{\pgfqpoint{1.746429in}{2.936057in}}%
\pgfpathcurveto{\pgfqpoint{1.757479in}{2.936057in}}{\pgfqpoint{1.768078in}{2.940448in}}{\pgfqpoint{1.775891in}{2.948261in}}%
\pgfpathcurveto{\pgfqpoint{1.783705in}{2.956075in}}{\pgfqpoint{1.788095in}{2.966674in}}{\pgfqpoint{1.788095in}{2.977724in}}%
\pgfpathcurveto{\pgfqpoint{1.788095in}{2.988774in}}{\pgfqpoint{1.783705in}{2.999373in}}{\pgfqpoint{1.775891in}{3.007187in}}%
\pgfpathcurveto{\pgfqpoint{1.768078in}{3.015000in}}{\pgfqpoint{1.757479in}{3.019391in}}{\pgfqpoint{1.746429in}{3.019391in}}%
\pgfpathcurveto{\pgfqpoint{1.735378in}{3.019391in}}{\pgfqpoint{1.724779in}{3.015000in}}{\pgfqpoint{1.716966in}{3.007187in}}%
\pgfpathcurveto{\pgfqpoint{1.709152in}{2.999373in}}{\pgfqpoint{1.704762in}{2.988774in}}{\pgfqpoint{1.704762in}{2.977724in}}%
\pgfpathcurveto{\pgfqpoint{1.704762in}{2.966674in}}{\pgfqpoint{1.709152in}{2.956075in}}{\pgfqpoint{1.716966in}{2.948261in}}%
\pgfpathcurveto{\pgfqpoint{1.724779in}{2.940448in}}{\pgfqpoint{1.735378in}{2.936057in}}{\pgfqpoint{1.746429in}{2.936057in}}%
\pgfpathclose%
\pgfusepath{stroke,fill}%
\end{pgfscope}%
\begin{pgfscope}%
\pgfpathrectangle{\pgfqpoint{0.750000in}{0.500000in}}{\pgfqpoint{4.650000in}{3.020000in}}%
\pgfusepath{clip}%
\pgfsetbuttcap%
\pgfsetroundjoin%
\definecolor{currentfill}{rgb}{1.000000,0.498039,0.054902}%
\pgfsetfillcolor{currentfill}%
\pgfsetlinewidth{1.003750pt}%
\definecolor{currentstroke}{rgb}{1.000000,0.498039,0.054902}%
\pgfsetstrokecolor{currentstroke}%
\pgfsetdash{}{0pt}%
\pgfpathmoveto{\pgfqpoint{1.686039in}{2.928269in}}%
\pgfpathcurveto{\pgfqpoint{1.697089in}{2.928269in}}{\pgfqpoint{1.707688in}{2.932659in}}{\pgfqpoint{1.715502in}{2.940473in}}%
\pgfpathcurveto{\pgfqpoint{1.723315in}{2.948286in}}{\pgfqpoint{1.727706in}{2.958885in}}{\pgfqpoint{1.727706in}{2.969936in}}%
\pgfpathcurveto{\pgfqpoint{1.727706in}{2.980986in}}{\pgfqpoint{1.723315in}{2.991585in}}{\pgfqpoint{1.715502in}{2.999398in}}%
\pgfpathcurveto{\pgfqpoint{1.707688in}{3.007212in}}{\pgfqpoint{1.697089in}{3.011602in}}{\pgfqpoint{1.686039in}{3.011602in}}%
\pgfpathcurveto{\pgfqpoint{1.674989in}{3.011602in}}{\pgfqpoint{1.664390in}{3.007212in}}{\pgfqpoint{1.656576in}{2.999398in}}%
\pgfpathcurveto{\pgfqpoint{1.648763in}{2.991585in}}{\pgfqpoint{1.644372in}{2.980986in}}{\pgfqpoint{1.644372in}{2.969936in}}%
\pgfpathcurveto{\pgfqpoint{1.644372in}{2.958885in}}{\pgfqpoint{1.648763in}{2.948286in}}{\pgfqpoint{1.656576in}{2.940473in}}%
\pgfpathcurveto{\pgfqpoint{1.664390in}{2.932659in}}{\pgfqpoint{1.674989in}{2.928269in}}{\pgfqpoint{1.686039in}{2.928269in}}%
\pgfpathclose%
\pgfusepath{stroke,fill}%
\end{pgfscope}%
\begin{pgfscope}%
\pgfpathrectangle{\pgfqpoint{0.750000in}{0.500000in}}{\pgfqpoint{4.650000in}{3.020000in}}%
\pgfusepath{clip}%
\pgfsetbuttcap%
\pgfsetroundjoin%
\definecolor{currentfill}{rgb}{1.000000,0.498039,0.054902}%
\pgfsetfillcolor{currentfill}%
\pgfsetlinewidth{1.003750pt}%
\definecolor{currentstroke}{rgb}{1.000000,0.498039,0.054902}%
\pgfsetstrokecolor{currentstroke}%
\pgfsetdash{}{0pt}%
\pgfpathmoveto{\pgfqpoint{1.323701in}{2.943846in}}%
\pgfpathcurveto{\pgfqpoint{1.334751in}{2.943846in}}{\pgfqpoint{1.345350in}{2.948236in}}{\pgfqpoint{1.353164in}{2.956050in}}%
\pgfpathcurveto{\pgfqpoint{1.360978in}{2.963863in}}{\pgfqpoint{1.365368in}{2.974462in}}{\pgfqpoint{1.365368in}{2.985513in}}%
\pgfpathcurveto{\pgfqpoint{1.365368in}{2.996563in}}{\pgfqpoint{1.360978in}{3.007162in}}{\pgfqpoint{1.353164in}{3.014975in}}%
\pgfpathcurveto{\pgfqpoint{1.345350in}{3.022789in}}{\pgfqpoint{1.334751in}{3.027179in}}{\pgfqpoint{1.323701in}{3.027179in}}%
\pgfpathcurveto{\pgfqpoint{1.312651in}{3.027179in}}{\pgfqpoint{1.302052in}{3.022789in}}{\pgfqpoint{1.294239in}{3.014975in}}%
\pgfpathcurveto{\pgfqpoint{1.286425in}{3.007162in}}{\pgfqpoint{1.282035in}{2.996563in}}{\pgfqpoint{1.282035in}{2.985513in}}%
\pgfpathcurveto{\pgfqpoint{1.282035in}{2.974462in}}{\pgfqpoint{1.286425in}{2.963863in}}{\pgfqpoint{1.294239in}{2.956050in}}%
\pgfpathcurveto{\pgfqpoint{1.302052in}{2.948236in}}{\pgfqpoint{1.312651in}{2.943846in}}{\pgfqpoint{1.323701in}{2.943846in}}%
\pgfpathclose%
\pgfusepath{stroke,fill}%
\end{pgfscope}%
\begin{pgfscope}%
\pgfpathrectangle{\pgfqpoint{0.750000in}{0.500000in}}{\pgfqpoint{4.650000in}{3.020000in}}%
\pgfusepath{clip}%
\pgfsetbuttcap%
\pgfsetroundjoin%
\definecolor{currentfill}{rgb}{1.000000,0.498039,0.054902}%
\pgfsetfillcolor{currentfill}%
\pgfsetlinewidth{1.003750pt}%
\definecolor{currentstroke}{rgb}{1.000000,0.498039,0.054902}%
\pgfsetstrokecolor{currentstroke}%
\pgfsetdash{}{0pt}%
\pgfpathmoveto{\pgfqpoint{1.806818in}{2.928269in}}%
\pgfpathcurveto{\pgfqpoint{1.817868in}{2.928269in}}{\pgfqpoint{1.828467in}{2.932659in}}{\pgfqpoint{1.836281in}{2.940473in}}%
\pgfpathcurveto{\pgfqpoint{1.844095in}{2.948286in}}{\pgfqpoint{1.848485in}{2.958885in}}{\pgfqpoint{1.848485in}{2.969936in}}%
\pgfpathcurveto{\pgfqpoint{1.848485in}{2.980986in}}{\pgfqpoint{1.844095in}{2.991585in}}{\pgfqpoint{1.836281in}{2.999398in}}%
\pgfpathcurveto{\pgfqpoint{1.828467in}{3.007212in}}{\pgfqpoint{1.817868in}{3.011602in}}{\pgfqpoint{1.806818in}{3.011602in}}%
\pgfpathcurveto{\pgfqpoint{1.795768in}{3.011602in}}{\pgfqpoint{1.785169in}{3.007212in}}{\pgfqpoint{1.777355in}{2.999398in}}%
\pgfpathcurveto{\pgfqpoint{1.769542in}{2.991585in}}{\pgfqpoint{1.765152in}{2.980986in}}{\pgfqpoint{1.765152in}{2.969936in}}%
\pgfpathcurveto{\pgfqpoint{1.765152in}{2.958885in}}{\pgfqpoint{1.769542in}{2.948286in}}{\pgfqpoint{1.777355in}{2.940473in}}%
\pgfpathcurveto{\pgfqpoint{1.785169in}{2.932659in}}{\pgfqpoint{1.795768in}{2.928269in}}{\pgfqpoint{1.806818in}{2.928269in}}%
\pgfpathclose%
\pgfusepath{stroke,fill}%
\end{pgfscope}%
\begin{pgfscope}%
\pgfpathrectangle{\pgfqpoint{0.750000in}{0.500000in}}{\pgfqpoint{4.650000in}{3.020000in}}%
\pgfusepath{clip}%
\pgfsetbuttcap%
\pgfsetroundjoin%
\definecolor{currentfill}{rgb}{0.121569,0.466667,0.705882}%
\pgfsetfillcolor{currentfill}%
\pgfsetlinewidth{1.003750pt}%
\definecolor{currentstroke}{rgb}{0.121569,0.466667,0.705882}%
\pgfsetstrokecolor{currentstroke}%
\pgfsetdash{}{0pt}%
\pgfpathmoveto{\pgfqpoint{1.021753in}{0.595606in}}%
\pgfpathcurveto{\pgfqpoint{1.032803in}{0.595606in}}{\pgfqpoint{1.043402in}{0.599996in}}{\pgfqpoint{1.051216in}{0.607810in}}%
\pgfpathcurveto{\pgfqpoint{1.059030in}{0.615624in}}{\pgfqpoint{1.063420in}{0.626223in}}{\pgfqpoint{1.063420in}{0.637273in}}%
\pgfpathcurveto{\pgfqpoint{1.063420in}{0.648323in}}{\pgfqpoint{1.059030in}{0.658922in}}{\pgfqpoint{1.051216in}{0.666736in}}%
\pgfpathcurveto{\pgfqpoint{1.043402in}{0.674549in}}{\pgfqpoint{1.032803in}{0.678939in}}{\pgfqpoint{1.021753in}{0.678939in}}%
\pgfpathcurveto{\pgfqpoint{1.010703in}{0.678939in}}{\pgfqpoint{1.000104in}{0.674549in}}{\pgfqpoint{0.992290in}{0.666736in}}%
\pgfpathcurveto{\pgfqpoint{0.984477in}{0.658922in}}{\pgfqpoint{0.980087in}{0.648323in}}{\pgfqpoint{0.980087in}{0.637273in}}%
\pgfpathcurveto{\pgfqpoint{0.980087in}{0.626223in}}{\pgfqpoint{0.984477in}{0.615624in}}{\pgfqpoint{0.992290in}{0.607810in}}%
\pgfpathcurveto{\pgfqpoint{1.000104in}{0.599996in}}{\pgfqpoint{1.010703in}{0.595606in}}{\pgfqpoint{1.021753in}{0.595606in}}%
\pgfpathclose%
\pgfusepath{stroke,fill}%
\end{pgfscope}%
\begin{pgfscope}%
\pgfpathrectangle{\pgfqpoint{0.750000in}{0.500000in}}{\pgfqpoint{4.650000in}{3.020000in}}%
\pgfusepath{clip}%
\pgfsetbuttcap%
\pgfsetroundjoin%
\definecolor{currentfill}{rgb}{1.000000,0.498039,0.054902}%
\pgfsetfillcolor{currentfill}%
\pgfsetlinewidth{1.003750pt}%
\definecolor{currentstroke}{rgb}{1.000000,0.498039,0.054902}%
\pgfsetstrokecolor{currentstroke}%
\pgfsetdash{}{0pt}%
\pgfpathmoveto{\pgfqpoint{1.686039in}{2.928269in}}%
\pgfpathcurveto{\pgfqpoint{1.697089in}{2.928269in}}{\pgfqpoint{1.707688in}{2.932659in}}{\pgfqpoint{1.715502in}{2.940473in}}%
\pgfpathcurveto{\pgfqpoint{1.723315in}{2.948286in}}{\pgfqpoint{1.727706in}{2.958885in}}{\pgfqpoint{1.727706in}{2.969936in}}%
\pgfpathcurveto{\pgfqpoint{1.727706in}{2.980986in}}{\pgfqpoint{1.723315in}{2.991585in}}{\pgfqpoint{1.715502in}{2.999398in}}%
\pgfpathcurveto{\pgfqpoint{1.707688in}{3.007212in}}{\pgfqpoint{1.697089in}{3.011602in}}{\pgfqpoint{1.686039in}{3.011602in}}%
\pgfpathcurveto{\pgfqpoint{1.674989in}{3.011602in}}{\pgfqpoint{1.664390in}{3.007212in}}{\pgfqpoint{1.656576in}{2.999398in}}%
\pgfpathcurveto{\pgfqpoint{1.648763in}{2.991585in}}{\pgfqpoint{1.644372in}{2.980986in}}{\pgfqpoint{1.644372in}{2.969936in}}%
\pgfpathcurveto{\pgfqpoint{1.644372in}{2.958885in}}{\pgfqpoint{1.648763in}{2.948286in}}{\pgfqpoint{1.656576in}{2.940473in}}%
\pgfpathcurveto{\pgfqpoint{1.664390in}{2.932659in}}{\pgfqpoint{1.674989in}{2.928269in}}{\pgfqpoint{1.686039in}{2.928269in}}%
\pgfpathclose%
\pgfusepath{stroke,fill}%
\end{pgfscope}%
\begin{pgfscope}%
\pgfpathrectangle{\pgfqpoint{0.750000in}{0.500000in}}{\pgfqpoint{4.650000in}{3.020000in}}%
\pgfusepath{clip}%
\pgfsetbuttcap%
\pgfsetroundjoin%
\definecolor{currentfill}{rgb}{1.000000,0.498039,0.054902}%
\pgfsetfillcolor{currentfill}%
\pgfsetlinewidth{1.003750pt}%
\definecolor{currentstroke}{rgb}{1.000000,0.498039,0.054902}%
\pgfsetstrokecolor{currentstroke}%
\pgfsetdash{}{0pt}%
\pgfpathmoveto{\pgfqpoint{1.504870in}{2.928269in}}%
\pgfpathcurveto{\pgfqpoint{1.515920in}{2.928269in}}{\pgfqpoint{1.526519in}{2.932659in}}{\pgfqpoint{1.534333in}{2.940473in}}%
\pgfpathcurveto{\pgfqpoint{1.542147in}{2.948286in}}{\pgfqpoint{1.546537in}{2.958885in}}{\pgfqpoint{1.546537in}{2.969936in}}%
\pgfpathcurveto{\pgfqpoint{1.546537in}{2.980986in}}{\pgfqpoint{1.542147in}{2.991585in}}{\pgfqpoint{1.534333in}{2.999398in}}%
\pgfpathcurveto{\pgfqpoint{1.526519in}{3.007212in}}{\pgfqpoint{1.515920in}{3.011602in}}{\pgfqpoint{1.504870in}{3.011602in}}%
\pgfpathcurveto{\pgfqpoint{1.493820in}{3.011602in}}{\pgfqpoint{1.483221in}{3.007212in}}{\pgfqpoint{1.475407in}{2.999398in}}%
\pgfpathcurveto{\pgfqpoint{1.467594in}{2.991585in}}{\pgfqpoint{1.463203in}{2.980986in}}{\pgfqpoint{1.463203in}{2.969936in}}%
\pgfpathcurveto{\pgfqpoint{1.463203in}{2.958885in}}{\pgfqpoint{1.467594in}{2.948286in}}{\pgfqpoint{1.475407in}{2.940473in}}%
\pgfpathcurveto{\pgfqpoint{1.483221in}{2.932659in}}{\pgfqpoint{1.493820in}{2.928269in}}{\pgfqpoint{1.504870in}{2.928269in}}%
\pgfpathclose%
\pgfusepath{stroke,fill}%
\end{pgfscope}%
\begin{pgfscope}%
\pgfpathrectangle{\pgfqpoint{0.750000in}{0.500000in}}{\pgfqpoint{4.650000in}{3.020000in}}%
\pgfusepath{clip}%
\pgfsetbuttcap%
\pgfsetroundjoin%
\definecolor{currentfill}{rgb}{1.000000,0.498039,0.054902}%
\pgfsetfillcolor{currentfill}%
\pgfsetlinewidth{1.003750pt}%
\definecolor{currentstroke}{rgb}{1.000000,0.498039,0.054902}%
\pgfsetstrokecolor{currentstroke}%
\pgfsetdash{}{0pt}%
\pgfpathmoveto{\pgfqpoint{1.444481in}{2.936057in}}%
\pgfpathcurveto{\pgfqpoint{1.455531in}{2.936057in}}{\pgfqpoint{1.466130in}{2.940448in}}{\pgfqpoint{1.473943in}{2.948261in}}%
\pgfpathcurveto{\pgfqpoint{1.481757in}{2.956075in}}{\pgfqpoint{1.486147in}{2.966674in}}{\pgfqpoint{1.486147in}{2.977724in}}%
\pgfpathcurveto{\pgfqpoint{1.486147in}{2.988774in}}{\pgfqpoint{1.481757in}{2.999373in}}{\pgfqpoint{1.473943in}{3.007187in}}%
\pgfpathcurveto{\pgfqpoint{1.466130in}{3.015000in}}{\pgfqpoint{1.455531in}{3.019391in}}{\pgfqpoint{1.444481in}{3.019391in}}%
\pgfpathcurveto{\pgfqpoint{1.433430in}{3.019391in}}{\pgfqpoint{1.422831in}{3.015000in}}{\pgfqpoint{1.415018in}{3.007187in}}%
\pgfpathcurveto{\pgfqpoint{1.407204in}{2.999373in}}{\pgfqpoint{1.402814in}{2.988774in}}{\pgfqpoint{1.402814in}{2.977724in}}%
\pgfpathcurveto{\pgfqpoint{1.402814in}{2.966674in}}{\pgfqpoint{1.407204in}{2.956075in}}{\pgfqpoint{1.415018in}{2.948261in}}%
\pgfpathcurveto{\pgfqpoint{1.422831in}{2.940448in}}{\pgfqpoint{1.433430in}{2.936057in}}{\pgfqpoint{1.444481in}{2.936057in}}%
\pgfpathclose%
\pgfusepath{stroke,fill}%
\end{pgfscope}%
\begin{pgfscope}%
\pgfpathrectangle{\pgfqpoint{0.750000in}{0.500000in}}{\pgfqpoint{4.650000in}{3.020000in}}%
\pgfusepath{clip}%
\pgfsetbuttcap%
\pgfsetroundjoin%
\definecolor{currentfill}{rgb}{1.000000,0.498039,0.054902}%
\pgfsetfillcolor{currentfill}%
\pgfsetlinewidth{1.003750pt}%
\definecolor{currentstroke}{rgb}{1.000000,0.498039,0.054902}%
\pgfsetstrokecolor{currentstroke}%
\pgfsetdash{}{0pt}%
\pgfpathmoveto{\pgfqpoint{1.444481in}{3.045097in}}%
\pgfpathcurveto{\pgfqpoint{1.455531in}{3.045097in}}{\pgfqpoint{1.466130in}{3.049487in}}{\pgfqpoint{1.473943in}{3.057301in}}%
\pgfpathcurveto{\pgfqpoint{1.481757in}{3.065114in}}{\pgfqpoint{1.486147in}{3.075713in}}{\pgfqpoint{1.486147in}{3.086763in}}%
\pgfpathcurveto{\pgfqpoint{1.486147in}{3.097814in}}{\pgfqpoint{1.481757in}{3.108413in}}{\pgfqpoint{1.473943in}{3.116226in}}%
\pgfpathcurveto{\pgfqpoint{1.466130in}{3.124040in}}{\pgfqpoint{1.455531in}{3.128430in}}{\pgfqpoint{1.444481in}{3.128430in}}%
\pgfpathcurveto{\pgfqpoint{1.433430in}{3.128430in}}{\pgfqpoint{1.422831in}{3.124040in}}{\pgfqpoint{1.415018in}{3.116226in}}%
\pgfpathcurveto{\pgfqpoint{1.407204in}{3.108413in}}{\pgfqpoint{1.402814in}{3.097814in}}{\pgfqpoint{1.402814in}{3.086763in}}%
\pgfpathcurveto{\pgfqpoint{1.402814in}{3.075713in}}{\pgfqpoint{1.407204in}{3.065114in}}{\pgfqpoint{1.415018in}{3.057301in}}%
\pgfpathcurveto{\pgfqpoint{1.422831in}{3.049487in}}{\pgfqpoint{1.433430in}{3.045097in}}{\pgfqpoint{1.444481in}{3.045097in}}%
\pgfpathclose%
\pgfusepath{stroke,fill}%
\end{pgfscope}%
\begin{pgfscope}%
\pgfpathrectangle{\pgfqpoint{0.750000in}{0.500000in}}{\pgfqpoint{4.650000in}{3.020000in}}%
\pgfusepath{clip}%
\pgfsetbuttcap%
\pgfsetroundjoin%
\definecolor{currentfill}{rgb}{1.000000,0.498039,0.054902}%
\pgfsetfillcolor{currentfill}%
\pgfsetlinewidth{1.003750pt}%
\definecolor{currentstroke}{rgb}{1.000000,0.498039,0.054902}%
\pgfsetstrokecolor{currentstroke}%
\pgfsetdash{}{0pt}%
\pgfpathmoveto{\pgfqpoint{1.323701in}{2.936057in}}%
\pgfpathcurveto{\pgfqpoint{1.334751in}{2.936057in}}{\pgfqpoint{1.345350in}{2.940448in}}{\pgfqpoint{1.353164in}{2.948261in}}%
\pgfpathcurveto{\pgfqpoint{1.360978in}{2.956075in}}{\pgfqpoint{1.365368in}{2.966674in}}{\pgfqpoint{1.365368in}{2.977724in}}%
\pgfpathcurveto{\pgfqpoint{1.365368in}{2.988774in}}{\pgfqpoint{1.360978in}{2.999373in}}{\pgfqpoint{1.353164in}{3.007187in}}%
\pgfpathcurveto{\pgfqpoint{1.345350in}{3.015000in}}{\pgfqpoint{1.334751in}{3.019391in}}{\pgfqpoint{1.323701in}{3.019391in}}%
\pgfpathcurveto{\pgfqpoint{1.312651in}{3.019391in}}{\pgfqpoint{1.302052in}{3.015000in}}{\pgfqpoint{1.294239in}{3.007187in}}%
\pgfpathcurveto{\pgfqpoint{1.286425in}{2.999373in}}{\pgfqpoint{1.282035in}{2.988774in}}{\pgfqpoint{1.282035in}{2.977724in}}%
\pgfpathcurveto{\pgfqpoint{1.282035in}{2.966674in}}{\pgfqpoint{1.286425in}{2.956075in}}{\pgfqpoint{1.294239in}{2.948261in}}%
\pgfpathcurveto{\pgfqpoint{1.302052in}{2.940448in}}{\pgfqpoint{1.312651in}{2.936057in}}{\pgfqpoint{1.323701in}{2.936057in}}%
\pgfpathclose%
\pgfusepath{stroke,fill}%
\end{pgfscope}%
\begin{pgfscope}%
\pgfpathrectangle{\pgfqpoint{0.750000in}{0.500000in}}{\pgfqpoint{4.650000in}{3.020000in}}%
\pgfusepath{clip}%
\pgfsetbuttcap%
\pgfsetroundjoin%
\definecolor{currentfill}{rgb}{1.000000,0.498039,0.054902}%
\pgfsetfillcolor{currentfill}%
\pgfsetlinewidth{1.003750pt}%
\definecolor{currentstroke}{rgb}{1.000000,0.498039,0.054902}%
\pgfsetstrokecolor{currentstroke}%
\pgfsetdash{}{0pt}%
\pgfpathmoveto{\pgfqpoint{2.229545in}{2.936057in}}%
\pgfpathcurveto{\pgfqpoint{2.240596in}{2.936057in}}{\pgfqpoint{2.251195in}{2.940448in}}{\pgfqpoint{2.259008in}{2.948261in}}%
\pgfpathcurveto{\pgfqpoint{2.266822in}{2.956075in}}{\pgfqpoint{2.271212in}{2.966674in}}{\pgfqpoint{2.271212in}{2.977724in}}%
\pgfpathcurveto{\pgfqpoint{2.271212in}{2.988774in}}{\pgfqpoint{2.266822in}{2.999373in}}{\pgfqpoint{2.259008in}{3.007187in}}%
\pgfpathcurveto{\pgfqpoint{2.251195in}{3.015000in}}{\pgfqpoint{2.240596in}{3.019391in}}{\pgfqpoint{2.229545in}{3.019391in}}%
\pgfpathcurveto{\pgfqpoint{2.218495in}{3.019391in}}{\pgfqpoint{2.207896in}{3.015000in}}{\pgfqpoint{2.200083in}{3.007187in}}%
\pgfpathcurveto{\pgfqpoint{2.192269in}{2.999373in}}{\pgfqpoint{2.187879in}{2.988774in}}{\pgfqpoint{2.187879in}{2.977724in}}%
\pgfpathcurveto{\pgfqpoint{2.187879in}{2.966674in}}{\pgfqpoint{2.192269in}{2.956075in}}{\pgfqpoint{2.200083in}{2.948261in}}%
\pgfpathcurveto{\pgfqpoint{2.207896in}{2.940448in}}{\pgfqpoint{2.218495in}{2.936057in}}{\pgfqpoint{2.229545in}{2.936057in}}%
\pgfpathclose%
\pgfusepath{stroke,fill}%
\end{pgfscope}%
\begin{pgfscope}%
\pgfpathrectangle{\pgfqpoint{0.750000in}{0.500000in}}{\pgfqpoint{4.650000in}{3.020000in}}%
\pgfusepath{clip}%
\pgfsetbuttcap%
\pgfsetroundjoin%
\definecolor{currentfill}{rgb}{1.000000,0.498039,0.054902}%
\pgfsetfillcolor{currentfill}%
\pgfsetlinewidth{1.003750pt}%
\definecolor{currentstroke}{rgb}{1.000000,0.498039,0.054902}%
\pgfsetstrokecolor{currentstroke}%
\pgfsetdash{}{0pt}%
\pgfpathmoveto{\pgfqpoint{1.686039in}{3.185290in}}%
\pgfpathcurveto{\pgfqpoint{1.697089in}{3.185290in}}{\pgfqpoint{1.707688in}{3.189680in}}{\pgfqpoint{1.715502in}{3.197494in}}%
\pgfpathcurveto{\pgfqpoint{1.723315in}{3.205308in}}{\pgfqpoint{1.727706in}{3.215907in}}{\pgfqpoint{1.727706in}{3.226957in}}%
\pgfpathcurveto{\pgfqpoint{1.727706in}{3.238007in}}{\pgfqpoint{1.723315in}{3.248606in}}{\pgfqpoint{1.715502in}{3.256420in}}%
\pgfpathcurveto{\pgfqpoint{1.707688in}{3.264233in}}{\pgfqpoint{1.697089in}{3.268623in}}{\pgfqpoint{1.686039in}{3.268623in}}%
\pgfpathcurveto{\pgfqpoint{1.674989in}{3.268623in}}{\pgfqpoint{1.664390in}{3.264233in}}{\pgfqpoint{1.656576in}{3.256420in}}%
\pgfpathcurveto{\pgfqpoint{1.648763in}{3.248606in}}{\pgfqpoint{1.644372in}{3.238007in}}{\pgfqpoint{1.644372in}{3.226957in}}%
\pgfpathcurveto{\pgfqpoint{1.644372in}{3.215907in}}{\pgfqpoint{1.648763in}{3.205308in}}{\pgfqpoint{1.656576in}{3.197494in}}%
\pgfpathcurveto{\pgfqpoint{1.664390in}{3.189680in}}{\pgfqpoint{1.674989in}{3.185290in}}{\pgfqpoint{1.686039in}{3.185290in}}%
\pgfpathclose%
\pgfusepath{stroke,fill}%
\end{pgfscope}%
\begin{pgfscope}%
\pgfpathrectangle{\pgfqpoint{0.750000in}{0.500000in}}{\pgfqpoint{4.650000in}{3.020000in}}%
\pgfusepath{clip}%
\pgfsetbuttcap%
\pgfsetroundjoin%
\definecolor{currentfill}{rgb}{1.000000,0.498039,0.054902}%
\pgfsetfillcolor{currentfill}%
\pgfsetlinewidth{1.003750pt}%
\definecolor{currentstroke}{rgb}{1.000000,0.498039,0.054902}%
\pgfsetstrokecolor{currentstroke}%
\pgfsetdash{}{0pt}%
\pgfpathmoveto{\pgfqpoint{1.927597in}{2.838701in}}%
\pgfpathcurveto{\pgfqpoint{1.938648in}{2.838701in}}{\pgfqpoint{1.949247in}{2.843091in}}{\pgfqpoint{1.957060in}{2.850905in}}%
\pgfpathcurveto{\pgfqpoint{1.964874in}{2.858718in}}{\pgfqpoint{1.969264in}{2.869317in}}{\pgfqpoint{1.969264in}{2.880368in}}%
\pgfpathcurveto{\pgfqpoint{1.969264in}{2.891418in}}{\pgfqpoint{1.964874in}{2.902017in}}{\pgfqpoint{1.957060in}{2.909830in}}%
\pgfpathcurveto{\pgfqpoint{1.949247in}{2.917644in}}{\pgfqpoint{1.938648in}{2.922034in}}{\pgfqpoint{1.927597in}{2.922034in}}%
\pgfpathcurveto{\pgfqpoint{1.916547in}{2.922034in}}{\pgfqpoint{1.905948in}{2.917644in}}{\pgfqpoint{1.898135in}{2.909830in}}%
\pgfpathcurveto{\pgfqpoint{1.890321in}{2.902017in}}{\pgfqpoint{1.885931in}{2.891418in}}{\pgfqpoint{1.885931in}{2.880368in}}%
\pgfpathcurveto{\pgfqpoint{1.885931in}{2.869317in}}{\pgfqpoint{1.890321in}{2.858718in}}{\pgfqpoint{1.898135in}{2.850905in}}%
\pgfpathcurveto{\pgfqpoint{1.905948in}{2.843091in}}{\pgfqpoint{1.916547in}{2.838701in}}{\pgfqpoint{1.927597in}{2.838701in}}%
\pgfpathclose%
\pgfusepath{stroke,fill}%
\end{pgfscope}%
\begin{pgfscope}%
\pgfpathrectangle{\pgfqpoint{0.750000in}{0.500000in}}{\pgfqpoint{4.650000in}{3.020000in}}%
\pgfusepath{clip}%
\pgfsetbuttcap%
\pgfsetroundjoin%
\definecolor{currentfill}{rgb}{1.000000,0.498039,0.054902}%
\pgfsetfillcolor{currentfill}%
\pgfsetlinewidth{1.003750pt}%
\definecolor{currentstroke}{rgb}{1.000000,0.498039,0.054902}%
\pgfsetstrokecolor{currentstroke}%
\pgfsetdash{}{0pt}%
\pgfpathmoveto{\pgfqpoint{1.987987in}{2.978894in}}%
\pgfpathcurveto{\pgfqpoint{1.999037in}{2.978894in}}{\pgfqpoint{2.009636in}{2.983285in}}{\pgfqpoint{2.017450in}{2.991098in}}%
\pgfpathcurveto{\pgfqpoint{2.025263in}{2.998912in}}{\pgfqpoint{2.029654in}{3.009511in}}{\pgfqpoint{2.029654in}{3.020561in}}%
\pgfpathcurveto{\pgfqpoint{2.029654in}{3.031611in}}{\pgfqpoint{2.025263in}{3.042210in}}{\pgfqpoint{2.017450in}{3.050024in}}%
\pgfpathcurveto{\pgfqpoint{2.009636in}{3.057837in}}{\pgfqpoint{1.999037in}{3.062228in}}{\pgfqpoint{1.987987in}{3.062228in}}%
\pgfpathcurveto{\pgfqpoint{1.976937in}{3.062228in}}{\pgfqpoint{1.966338in}{3.057837in}}{\pgfqpoint{1.958524in}{3.050024in}}%
\pgfpathcurveto{\pgfqpoint{1.950711in}{3.042210in}}{\pgfqpoint{1.946320in}{3.031611in}}{\pgfqpoint{1.946320in}{3.020561in}}%
\pgfpathcurveto{\pgfqpoint{1.946320in}{3.009511in}}{\pgfqpoint{1.950711in}{2.998912in}}{\pgfqpoint{1.958524in}{2.991098in}}%
\pgfpathcurveto{\pgfqpoint{1.966338in}{2.983285in}}{\pgfqpoint{1.976937in}{2.978894in}}{\pgfqpoint{1.987987in}{2.978894in}}%
\pgfpathclose%
\pgfusepath{stroke,fill}%
\end{pgfscope}%
\begin{pgfscope}%
\pgfpathrectangle{\pgfqpoint{0.750000in}{0.500000in}}{\pgfqpoint{4.650000in}{3.020000in}}%
\pgfusepath{clip}%
\pgfsetbuttcap%
\pgfsetroundjoin%
\definecolor{currentfill}{rgb}{1.000000,0.498039,0.054902}%
\pgfsetfillcolor{currentfill}%
\pgfsetlinewidth{1.003750pt}%
\definecolor{currentstroke}{rgb}{1.000000,0.498039,0.054902}%
\pgfsetstrokecolor{currentstroke}%
\pgfsetdash{}{0pt}%
\pgfpathmoveto{\pgfqpoint{1.686039in}{2.932163in}}%
\pgfpathcurveto{\pgfqpoint{1.697089in}{2.932163in}}{\pgfqpoint{1.707688in}{2.936553in}}{\pgfqpoint{1.715502in}{2.944367in}}%
\pgfpathcurveto{\pgfqpoint{1.723315in}{2.952181in}}{\pgfqpoint{1.727706in}{2.962780in}}{\pgfqpoint{1.727706in}{2.973830in}}%
\pgfpathcurveto{\pgfqpoint{1.727706in}{2.984880in}}{\pgfqpoint{1.723315in}{2.995479in}}{\pgfqpoint{1.715502in}{3.003293in}}%
\pgfpathcurveto{\pgfqpoint{1.707688in}{3.011106in}}{\pgfqpoint{1.697089in}{3.015496in}}{\pgfqpoint{1.686039in}{3.015496in}}%
\pgfpathcurveto{\pgfqpoint{1.674989in}{3.015496in}}{\pgfqpoint{1.664390in}{3.011106in}}{\pgfqpoint{1.656576in}{3.003293in}}%
\pgfpathcurveto{\pgfqpoint{1.648763in}{2.995479in}}{\pgfqpoint{1.644372in}{2.984880in}}{\pgfqpoint{1.644372in}{2.973830in}}%
\pgfpathcurveto{\pgfqpoint{1.644372in}{2.962780in}}{\pgfqpoint{1.648763in}{2.952181in}}{\pgfqpoint{1.656576in}{2.944367in}}%
\pgfpathcurveto{\pgfqpoint{1.664390in}{2.936553in}}{\pgfqpoint{1.674989in}{2.932163in}}{\pgfqpoint{1.686039in}{2.932163in}}%
\pgfpathclose%
\pgfusepath{stroke,fill}%
\end{pgfscope}%
\begin{pgfscope}%
\pgfpathrectangle{\pgfqpoint{0.750000in}{0.500000in}}{\pgfqpoint{4.650000in}{3.020000in}}%
\pgfusepath{clip}%
\pgfsetbuttcap%
\pgfsetroundjoin%
\definecolor{currentfill}{rgb}{1.000000,0.498039,0.054902}%
\pgfsetfillcolor{currentfill}%
\pgfsetlinewidth{1.003750pt}%
\definecolor{currentstroke}{rgb}{1.000000,0.498039,0.054902}%
\pgfsetstrokecolor{currentstroke}%
\pgfsetdash{}{0pt}%
\pgfpathmoveto{\pgfqpoint{1.565260in}{2.932163in}}%
\pgfpathcurveto{\pgfqpoint{1.576310in}{2.932163in}}{\pgfqpoint{1.586909in}{2.936553in}}{\pgfqpoint{1.594723in}{2.944367in}}%
\pgfpathcurveto{\pgfqpoint{1.602536in}{2.952181in}}{\pgfqpoint{1.606926in}{2.962780in}}{\pgfqpoint{1.606926in}{2.973830in}}%
\pgfpathcurveto{\pgfqpoint{1.606926in}{2.984880in}}{\pgfqpoint{1.602536in}{2.995479in}}{\pgfqpoint{1.594723in}{3.003293in}}%
\pgfpathcurveto{\pgfqpoint{1.586909in}{3.011106in}}{\pgfqpoint{1.576310in}{3.015496in}}{\pgfqpoint{1.565260in}{3.015496in}}%
\pgfpathcurveto{\pgfqpoint{1.554210in}{3.015496in}}{\pgfqpoint{1.543611in}{3.011106in}}{\pgfqpoint{1.535797in}{3.003293in}}%
\pgfpathcurveto{\pgfqpoint{1.527983in}{2.995479in}}{\pgfqpoint{1.523593in}{2.984880in}}{\pgfqpoint{1.523593in}{2.973830in}}%
\pgfpathcurveto{\pgfqpoint{1.523593in}{2.962780in}}{\pgfqpoint{1.527983in}{2.952181in}}{\pgfqpoint{1.535797in}{2.944367in}}%
\pgfpathcurveto{\pgfqpoint{1.543611in}{2.936553in}}{\pgfqpoint{1.554210in}{2.932163in}}{\pgfqpoint{1.565260in}{2.932163in}}%
\pgfpathclose%
\pgfusepath{stroke,fill}%
\end{pgfscope}%
\begin{pgfscope}%
\pgfpathrectangle{\pgfqpoint{0.750000in}{0.500000in}}{\pgfqpoint{4.650000in}{3.020000in}}%
\pgfusepath{clip}%
\pgfsetbuttcap%
\pgfsetroundjoin%
\definecolor{currentfill}{rgb}{1.000000,0.498039,0.054902}%
\pgfsetfillcolor{currentfill}%
\pgfsetlinewidth{1.003750pt}%
\definecolor{currentstroke}{rgb}{1.000000,0.498039,0.054902}%
\pgfsetstrokecolor{currentstroke}%
\pgfsetdash{}{0pt}%
\pgfpathmoveto{\pgfqpoint{2.591883in}{2.928269in}}%
\pgfpathcurveto{\pgfqpoint{2.602933in}{2.928269in}}{\pgfqpoint{2.613532in}{2.932659in}}{\pgfqpoint{2.621346in}{2.940473in}}%
\pgfpathcurveto{\pgfqpoint{2.629160in}{2.948286in}}{\pgfqpoint{2.633550in}{2.958885in}}{\pgfqpoint{2.633550in}{2.969936in}}%
\pgfpathcurveto{\pgfqpoint{2.633550in}{2.980986in}}{\pgfqpoint{2.629160in}{2.991585in}}{\pgfqpoint{2.621346in}{2.999398in}}%
\pgfpathcurveto{\pgfqpoint{2.613532in}{3.007212in}}{\pgfqpoint{2.602933in}{3.011602in}}{\pgfqpoint{2.591883in}{3.011602in}}%
\pgfpathcurveto{\pgfqpoint{2.580833in}{3.011602in}}{\pgfqpoint{2.570234in}{3.007212in}}{\pgfqpoint{2.562420in}{2.999398in}}%
\pgfpathcurveto{\pgfqpoint{2.554607in}{2.991585in}}{\pgfqpoint{2.550216in}{2.980986in}}{\pgfqpoint{2.550216in}{2.969936in}}%
\pgfpathcurveto{\pgfqpoint{2.550216in}{2.958885in}}{\pgfqpoint{2.554607in}{2.948286in}}{\pgfqpoint{2.562420in}{2.940473in}}%
\pgfpathcurveto{\pgfqpoint{2.570234in}{2.932659in}}{\pgfqpoint{2.580833in}{2.928269in}}{\pgfqpoint{2.591883in}{2.928269in}}%
\pgfpathclose%
\pgfusepath{stroke,fill}%
\end{pgfscope}%
\begin{pgfscope}%
\pgfpathrectangle{\pgfqpoint{0.750000in}{0.500000in}}{\pgfqpoint{4.650000in}{3.020000in}}%
\pgfusepath{clip}%
\pgfsetbuttcap%
\pgfsetroundjoin%
\definecolor{currentfill}{rgb}{0.121569,0.466667,0.705882}%
\pgfsetfillcolor{currentfill}%
\pgfsetlinewidth{1.003750pt}%
\definecolor{currentstroke}{rgb}{0.121569,0.466667,0.705882}%
\pgfsetstrokecolor{currentstroke}%
\pgfsetdash{}{0pt}%
\pgfpathmoveto{\pgfqpoint{0.961364in}{0.595606in}}%
\pgfpathcurveto{\pgfqpoint{0.972414in}{0.595606in}}{\pgfqpoint{0.983013in}{0.599996in}}{\pgfqpoint{0.990826in}{0.607810in}}%
\pgfpathcurveto{\pgfqpoint{0.998640in}{0.615624in}}{\pgfqpoint{1.003030in}{0.626223in}}{\pgfqpoint{1.003030in}{0.637273in}}%
\pgfpathcurveto{\pgfqpoint{1.003030in}{0.648323in}}{\pgfqpoint{0.998640in}{0.658922in}}{\pgfqpoint{0.990826in}{0.666736in}}%
\pgfpathcurveto{\pgfqpoint{0.983013in}{0.674549in}}{\pgfqpoint{0.972414in}{0.678939in}}{\pgfqpoint{0.961364in}{0.678939in}}%
\pgfpathcurveto{\pgfqpoint{0.950314in}{0.678939in}}{\pgfqpoint{0.939714in}{0.674549in}}{\pgfqpoint{0.931901in}{0.666736in}}%
\pgfpathcurveto{\pgfqpoint{0.924087in}{0.658922in}}{\pgfqpoint{0.919697in}{0.648323in}}{\pgfqpoint{0.919697in}{0.637273in}}%
\pgfpathcurveto{\pgfqpoint{0.919697in}{0.626223in}}{\pgfqpoint{0.924087in}{0.615624in}}{\pgfqpoint{0.931901in}{0.607810in}}%
\pgfpathcurveto{\pgfqpoint{0.939714in}{0.599996in}}{\pgfqpoint{0.950314in}{0.595606in}}{\pgfqpoint{0.961364in}{0.595606in}}%
\pgfpathclose%
\pgfusepath{stroke,fill}%
\end{pgfscope}%
\begin{pgfscope}%
\pgfpathrectangle{\pgfqpoint{0.750000in}{0.500000in}}{\pgfqpoint{4.650000in}{3.020000in}}%
\pgfusepath{clip}%
\pgfsetbuttcap%
\pgfsetroundjoin%
\definecolor{currentfill}{rgb}{0.121569,0.466667,0.705882}%
\pgfsetfillcolor{currentfill}%
\pgfsetlinewidth{1.003750pt}%
\definecolor{currentstroke}{rgb}{0.121569,0.466667,0.705882}%
\pgfsetstrokecolor{currentstroke}%
\pgfsetdash{}{0pt}%
\pgfpathmoveto{\pgfqpoint{1.202922in}{0.599500in}}%
\pgfpathcurveto{\pgfqpoint{1.213972in}{0.599500in}}{\pgfqpoint{1.224571in}{0.603891in}}{\pgfqpoint{1.232385in}{0.611704in}}%
\pgfpathcurveto{\pgfqpoint{1.240198in}{0.619518in}}{\pgfqpoint{1.244589in}{0.630117in}}{\pgfqpoint{1.244589in}{0.641167in}}%
\pgfpathcurveto{\pgfqpoint{1.244589in}{0.652217in}}{\pgfqpoint{1.240198in}{0.662816in}}{\pgfqpoint{1.232385in}{0.670630in}}%
\pgfpathcurveto{\pgfqpoint{1.224571in}{0.678443in}}{\pgfqpoint{1.213972in}{0.682834in}}{\pgfqpoint{1.202922in}{0.682834in}}%
\pgfpathcurveto{\pgfqpoint{1.191872in}{0.682834in}}{\pgfqpoint{1.181273in}{0.678443in}}{\pgfqpoint{1.173459in}{0.670630in}}%
\pgfpathcurveto{\pgfqpoint{1.165646in}{0.662816in}}{\pgfqpoint{1.161255in}{0.652217in}}{\pgfqpoint{1.161255in}{0.641167in}}%
\pgfpathcurveto{\pgfqpoint{1.161255in}{0.630117in}}{\pgfqpoint{1.165646in}{0.619518in}}{\pgfqpoint{1.173459in}{0.611704in}}%
\pgfpathcurveto{\pgfqpoint{1.181273in}{0.603891in}}{\pgfqpoint{1.191872in}{0.599500in}}{\pgfqpoint{1.202922in}{0.599500in}}%
\pgfpathclose%
\pgfusepath{stroke,fill}%
\end{pgfscope}%
\begin{pgfscope}%
\pgfpathrectangle{\pgfqpoint{0.750000in}{0.500000in}}{\pgfqpoint{4.650000in}{3.020000in}}%
\pgfusepath{clip}%
\pgfsetbuttcap%
\pgfsetroundjoin%
\definecolor{currentfill}{rgb}{1.000000,0.498039,0.054902}%
\pgfsetfillcolor{currentfill}%
\pgfsetlinewidth{1.003750pt}%
\definecolor{currentstroke}{rgb}{1.000000,0.498039,0.054902}%
\pgfsetstrokecolor{currentstroke}%
\pgfsetdash{}{0pt}%
\pgfpathmoveto{\pgfqpoint{1.504870in}{2.904903in}}%
\pgfpathcurveto{\pgfqpoint{1.515920in}{2.904903in}}{\pgfqpoint{1.526519in}{2.909294in}}{\pgfqpoint{1.534333in}{2.917107in}}%
\pgfpathcurveto{\pgfqpoint{1.542147in}{2.924921in}}{\pgfqpoint{1.546537in}{2.935520in}}{\pgfqpoint{1.546537in}{2.946570in}}%
\pgfpathcurveto{\pgfqpoint{1.546537in}{2.957620in}}{\pgfqpoint{1.542147in}{2.968219in}}{\pgfqpoint{1.534333in}{2.976033in}}%
\pgfpathcurveto{\pgfqpoint{1.526519in}{2.983846in}}{\pgfqpoint{1.515920in}{2.988237in}}{\pgfqpoint{1.504870in}{2.988237in}}%
\pgfpathcurveto{\pgfqpoint{1.493820in}{2.988237in}}{\pgfqpoint{1.483221in}{2.983846in}}{\pgfqpoint{1.475407in}{2.976033in}}%
\pgfpathcurveto{\pgfqpoint{1.467594in}{2.968219in}}{\pgfqpoint{1.463203in}{2.957620in}}{\pgfqpoint{1.463203in}{2.946570in}}%
\pgfpathcurveto{\pgfqpoint{1.463203in}{2.935520in}}{\pgfqpoint{1.467594in}{2.924921in}}{\pgfqpoint{1.475407in}{2.917107in}}%
\pgfpathcurveto{\pgfqpoint{1.483221in}{2.909294in}}{\pgfqpoint{1.493820in}{2.904903in}}{\pgfqpoint{1.504870in}{2.904903in}}%
\pgfpathclose%
\pgfusepath{stroke,fill}%
\end{pgfscope}%
\begin{pgfscope}%
\pgfpathrectangle{\pgfqpoint{0.750000in}{0.500000in}}{\pgfqpoint{4.650000in}{3.020000in}}%
\pgfusepath{clip}%
\pgfsetbuttcap%
\pgfsetroundjoin%
\definecolor{currentfill}{rgb}{1.000000,0.498039,0.054902}%
\pgfsetfillcolor{currentfill}%
\pgfsetlinewidth{1.003750pt}%
\definecolor{currentstroke}{rgb}{1.000000,0.498039,0.054902}%
\pgfsetstrokecolor{currentstroke}%
\pgfsetdash{}{0pt}%
\pgfpathmoveto{\pgfqpoint{1.323701in}{2.994471in}}%
\pgfpathcurveto{\pgfqpoint{1.334751in}{2.994471in}}{\pgfqpoint{1.345350in}{2.998862in}}{\pgfqpoint{1.353164in}{3.006675in}}%
\pgfpathcurveto{\pgfqpoint{1.360978in}{3.014489in}}{\pgfqpoint{1.365368in}{3.025088in}}{\pgfqpoint{1.365368in}{3.036138in}}%
\pgfpathcurveto{\pgfqpoint{1.365368in}{3.047188in}}{\pgfqpoint{1.360978in}{3.057787in}}{\pgfqpoint{1.353164in}{3.065601in}}%
\pgfpathcurveto{\pgfqpoint{1.345350in}{3.073414in}}{\pgfqpoint{1.334751in}{3.077805in}}{\pgfqpoint{1.323701in}{3.077805in}}%
\pgfpathcurveto{\pgfqpoint{1.312651in}{3.077805in}}{\pgfqpoint{1.302052in}{3.073414in}}{\pgfqpoint{1.294239in}{3.065601in}}%
\pgfpathcurveto{\pgfqpoint{1.286425in}{3.057787in}}{\pgfqpoint{1.282035in}{3.047188in}}{\pgfqpoint{1.282035in}{3.036138in}}%
\pgfpathcurveto{\pgfqpoint{1.282035in}{3.025088in}}{\pgfqpoint{1.286425in}{3.014489in}}{\pgfqpoint{1.294239in}{3.006675in}}%
\pgfpathcurveto{\pgfqpoint{1.302052in}{2.998862in}}{\pgfqpoint{1.312651in}{2.994471in}}{\pgfqpoint{1.323701in}{2.994471in}}%
\pgfpathclose%
\pgfusepath{stroke,fill}%
\end{pgfscope}%
\begin{pgfscope}%
\pgfpathrectangle{\pgfqpoint{0.750000in}{0.500000in}}{\pgfqpoint{4.650000in}{3.020000in}}%
\pgfusepath{clip}%
\pgfsetbuttcap%
\pgfsetroundjoin%
\definecolor{currentfill}{rgb}{0.121569,0.466667,0.705882}%
\pgfsetfillcolor{currentfill}%
\pgfsetlinewidth{1.003750pt}%
\definecolor{currentstroke}{rgb}{0.121569,0.466667,0.705882}%
\pgfsetstrokecolor{currentstroke}%
\pgfsetdash{}{0pt}%
\pgfpathmoveto{\pgfqpoint{0.961364in}{0.595606in}}%
\pgfpathcurveto{\pgfqpoint{0.972414in}{0.595606in}}{\pgfqpoint{0.983013in}{0.599996in}}{\pgfqpoint{0.990826in}{0.607810in}}%
\pgfpathcurveto{\pgfqpoint{0.998640in}{0.615624in}}{\pgfqpoint{1.003030in}{0.626223in}}{\pgfqpoint{1.003030in}{0.637273in}}%
\pgfpathcurveto{\pgfqpoint{1.003030in}{0.648323in}}{\pgfqpoint{0.998640in}{0.658922in}}{\pgfqpoint{0.990826in}{0.666736in}}%
\pgfpathcurveto{\pgfqpoint{0.983013in}{0.674549in}}{\pgfqpoint{0.972414in}{0.678939in}}{\pgfqpoint{0.961364in}{0.678939in}}%
\pgfpathcurveto{\pgfqpoint{0.950314in}{0.678939in}}{\pgfqpoint{0.939714in}{0.674549in}}{\pgfqpoint{0.931901in}{0.666736in}}%
\pgfpathcurveto{\pgfqpoint{0.924087in}{0.658922in}}{\pgfqpoint{0.919697in}{0.648323in}}{\pgfqpoint{0.919697in}{0.637273in}}%
\pgfpathcurveto{\pgfqpoint{0.919697in}{0.626223in}}{\pgfqpoint{0.924087in}{0.615624in}}{\pgfqpoint{0.931901in}{0.607810in}}%
\pgfpathcurveto{\pgfqpoint{0.939714in}{0.599996in}}{\pgfqpoint{0.950314in}{0.595606in}}{\pgfqpoint{0.961364in}{0.595606in}}%
\pgfpathclose%
\pgfusepath{stroke,fill}%
\end{pgfscope}%
\begin{pgfscope}%
\pgfpathrectangle{\pgfqpoint{0.750000in}{0.500000in}}{\pgfqpoint{4.650000in}{3.020000in}}%
\pgfusepath{clip}%
\pgfsetbuttcap%
\pgfsetroundjoin%
\definecolor{currentfill}{rgb}{0.121569,0.466667,0.705882}%
\pgfsetfillcolor{currentfill}%
\pgfsetlinewidth{1.003750pt}%
\definecolor{currentstroke}{rgb}{0.121569,0.466667,0.705882}%
\pgfsetstrokecolor{currentstroke}%
\pgfsetdash{}{0pt}%
\pgfpathmoveto{\pgfqpoint{1.021753in}{0.595606in}}%
\pgfpathcurveto{\pgfqpoint{1.032803in}{0.595606in}}{\pgfqpoint{1.043402in}{0.599996in}}{\pgfqpoint{1.051216in}{0.607810in}}%
\pgfpathcurveto{\pgfqpoint{1.059030in}{0.615624in}}{\pgfqpoint{1.063420in}{0.626223in}}{\pgfqpoint{1.063420in}{0.637273in}}%
\pgfpathcurveto{\pgfqpoint{1.063420in}{0.648323in}}{\pgfqpoint{1.059030in}{0.658922in}}{\pgfqpoint{1.051216in}{0.666736in}}%
\pgfpathcurveto{\pgfqpoint{1.043402in}{0.674549in}}{\pgfqpoint{1.032803in}{0.678939in}}{\pgfqpoint{1.021753in}{0.678939in}}%
\pgfpathcurveto{\pgfqpoint{1.010703in}{0.678939in}}{\pgfqpoint{1.000104in}{0.674549in}}{\pgfqpoint{0.992290in}{0.666736in}}%
\pgfpathcurveto{\pgfqpoint{0.984477in}{0.658922in}}{\pgfqpoint{0.980087in}{0.648323in}}{\pgfqpoint{0.980087in}{0.637273in}}%
\pgfpathcurveto{\pgfqpoint{0.980087in}{0.626223in}}{\pgfqpoint{0.984477in}{0.615624in}}{\pgfqpoint{0.992290in}{0.607810in}}%
\pgfpathcurveto{\pgfqpoint{1.000104in}{0.599996in}}{\pgfqpoint{1.010703in}{0.595606in}}{\pgfqpoint{1.021753in}{0.595606in}}%
\pgfpathclose%
\pgfusepath{stroke,fill}%
\end{pgfscope}%
\begin{pgfscope}%
\pgfpathrectangle{\pgfqpoint{0.750000in}{0.500000in}}{\pgfqpoint{4.650000in}{3.020000in}}%
\pgfusepath{clip}%
\pgfsetbuttcap%
\pgfsetroundjoin%
\definecolor{currentfill}{rgb}{1.000000,0.498039,0.054902}%
\pgfsetfillcolor{currentfill}%
\pgfsetlinewidth{1.003750pt}%
\definecolor{currentstroke}{rgb}{1.000000,0.498039,0.054902}%
\pgfsetstrokecolor{currentstroke}%
\pgfsetdash{}{0pt}%
\pgfpathmoveto{\pgfqpoint{1.927597in}{2.936057in}}%
\pgfpathcurveto{\pgfqpoint{1.938648in}{2.936057in}}{\pgfqpoint{1.949247in}{2.940448in}}{\pgfqpoint{1.957060in}{2.948261in}}%
\pgfpathcurveto{\pgfqpoint{1.964874in}{2.956075in}}{\pgfqpoint{1.969264in}{2.966674in}}{\pgfqpoint{1.969264in}{2.977724in}}%
\pgfpathcurveto{\pgfqpoint{1.969264in}{2.988774in}}{\pgfqpoint{1.964874in}{2.999373in}}{\pgfqpoint{1.957060in}{3.007187in}}%
\pgfpathcurveto{\pgfqpoint{1.949247in}{3.015000in}}{\pgfqpoint{1.938648in}{3.019391in}}{\pgfqpoint{1.927597in}{3.019391in}}%
\pgfpathcurveto{\pgfqpoint{1.916547in}{3.019391in}}{\pgfqpoint{1.905948in}{3.015000in}}{\pgfqpoint{1.898135in}{3.007187in}}%
\pgfpathcurveto{\pgfqpoint{1.890321in}{2.999373in}}{\pgfqpoint{1.885931in}{2.988774in}}{\pgfqpoint{1.885931in}{2.977724in}}%
\pgfpathcurveto{\pgfqpoint{1.885931in}{2.966674in}}{\pgfqpoint{1.890321in}{2.956075in}}{\pgfqpoint{1.898135in}{2.948261in}}%
\pgfpathcurveto{\pgfqpoint{1.905948in}{2.940448in}}{\pgfqpoint{1.916547in}{2.936057in}}{\pgfqpoint{1.927597in}{2.936057in}}%
\pgfpathclose%
\pgfusepath{stroke,fill}%
\end{pgfscope}%
\begin{pgfscope}%
\pgfpathrectangle{\pgfqpoint{0.750000in}{0.500000in}}{\pgfqpoint{4.650000in}{3.020000in}}%
\pgfusepath{clip}%
\pgfsetbuttcap%
\pgfsetroundjoin%
\definecolor{currentfill}{rgb}{1.000000,0.498039,0.054902}%
\pgfsetfillcolor{currentfill}%
\pgfsetlinewidth{1.003750pt}%
\definecolor{currentstroke}{rgb}{1.000000,0.498039,0.054902}%
\pgfsetstrokecolor{currentstroke}%
\pgfsetdash{}{0pt}%
\pgfpathmoveto{\pgfqpoint{1.806818in}{2.936057in}}%
\pgfpathcurveto{\pgfqpoint{1.817868in}{2.936057in}}{\pgfqpoint{1.828467in}{2.940448in}}{\pgfqpoint{1.836281in}{2.948261in}}%
\pgfpathcurveto{\pgfqpoint{1.844095in}{2.956075in}}{\pgfqpoint{1.848485in}{2.966674in}}{\pgfqpoint{1.848485in}{2.977724in}}%
\pgfpathcurveto{\pgfqpoint{1.848485in}{2.988774in}}{\pgfqpoint{1.844095in}{2.999373in}}{\pgfqpoint{1.836281in}{3.007187in}}%
\pgfpathcurveto{\pgfqpoint{1.828467in}{3.015000in}}{\pgfqpoint{1.817868in}{3.019391in}}{\pgfqpoint{1.806818in}{3.019391in}}%
\pgfpathcurveto{\pgfqpoint{1.795768in}{3.019391in}}{\pgfqpoint{1.785169in}{3.015000in}}{\pgfqpoint{1.777355in}{3.007187in}}%
\pgfpathcurveto{\pgfqpoint{1.769542in}{2.999373in}}{\pgfqpoint{1.765152in}{2.988774in}}{\pgfqpoint{1.765152in}{2.977724in}}%
\pgfpathcurveto{\pgfqpoint{1.765152in}{2.966674in}}{\pgfqpoint{1.769542in}{2.956075in}}{\pgfqpoint{1.777355in}{2.948261in}}%
\pgfpathcurveto{\pgfqpoint{1.785169in}{2.940448in}}{\pgfqpoint{1.795768in}{2.936057in}}{\pgfqpoint{1.806818in}{2.936057in}}%
\pgfpathclose%
\pgfusepath{stroke,fill}%
\end{pgfscope}%
\begin{pgfscope}%
\pgfsetbuttcap%
\pgfsetroundjoin%
\definecolor{currentfill}{rgb}{0.000000,0.000000,0.000000}%
\pgfsetfillcolor{currentfill}%
\pgfsetlinewidth{0.803000pt}%
\definecolor{currentstroke}{rgb}{0.000000,0.000000,0.000000}%
\pgfsetstrokecolor{currentstroke}%
\pgfsetdash{}{0pt}%
\pgfsys@defobject{currentmarker}{\pgfqpoint{0.000000in}{-0.048611in}}{\pgfqpoint{0.000000in}{0.000000in}}{%
\pgfpathmoveto{\pgfqpoint{0.000000in}{0.000000in}}%
\pgfpathlineto{\pgfqpoint{0.000000in}{-0.048611in}}%
\pgfusepath{stroke,fill}%
}%
\begin{pgfscope}%
\pgfsys@transformshift{0.900974in}{0.500000in}%
\pgfsys@useobject{currentmarker}{}%
\end{pgfscope}%
\end{pgfscope}%
\begin{pgfscope}%
\definecolor{textcolor}{rgb}{0.000000,0.000000,0.000000}%
\pgfsetstrokecolor{textcolor}%
\pgfsetfillcolor{textcolor}%
\pgftext[x=0.900974in,y=0.402778in,,top]{\color{textcolor}\rmfamily\fontsize{10.000000}{12.000000}\selectfont \(\displaystyle {0}\)}%
\end{pgfscope}%
\begin{pgfscope}%
\pgfsetbuttcap%
\pgfsetroundjoin%
\definecolor{currentfill}{rgb}{0.000000,0.000000,0.000000}%
\pgfsetfillcolor{currentfill}%
\pgfsetlinewidth{0.803000pt}%
\definecolor{currentstroke}{rgb}{0.000000,0.000000,0.000000}%
\pgfsetstrokecolor{currentstroke}%
\pgfsetdash{}{0pt}%
\pgfsys@defobject{currentmarker}{\pgfqpoint{0.000000in}{-0.048611in}}{\pgfqpoint{0.000000in}{0.000000in}}{%
\pgfpathmoveto{\pgfqpoint{0.000000in}{0.000000in}}%
\pgfpathlineto{\pgfqpoint{0.000000in}{-0.048611in}}%
\pgfusepath{stroke,fill}%
}%
\begin{pgfscope}%
\pgfsys@transformshift{1.504870in}{0.500000in}%
\pgfsys@useobject{currentmarker}{}%
\end{pgfscope}%
\end{pgfscope}%
\begin{pgfscope}%
\definecolor{textcolor}{rgb}{0.000000,0.000000,0.000000}%
\pgfsetstrokecolor{textcolor}%
\pgfsetfillcolor{textcolor}%
\pgftext[x=1.504870in,y=0.402778in,,top]{\color{textcolor}\rmfamily\fontsize{10.000000}{12.000000}\selectfont \(\displaystyle {10}\)}%
\end{pgfscope}%
\begin{pgfscope}%
\pgfsetbuttcap%
\pgfsetroundjoin%
\definecolor{currentfill}{rgb}{0.000000,0.000000,0.000000}%
\pgfsetfillcolor{currentfill}%
\pgfsetlinewidth{0.803000pt}%
\definecolor{currentstroke}{rgb}{0.000000,0.000000,0.000000}%
\pgfsetstrokecolor{currentstroke}%
\pgfsetdash{}{0pt}%
\pgfsys@defobject{currentmarker}{\pgfqpoint{0.000000in}{-0.048611in}}{\pgfqpoint{0.000000in}{0.000000in}}{%
\pgfpathmoveto{\pgfqpoint{0.000000in}{0.000000in}}%
\pgfpathlineto{\pgfqpoint{0.000000in}{-0.048611in}}%
\pgfusepath{stroke,fill}%
}%
\begin{pgfscope}%
\pgfsys@transformshift{2.108766in}{0.500000in}%
\pgfsys@useobject{currentmarker}{}%
\end{pgfscope}%
\end{pgfscope}%
\begin{pgfscope}%
\definecolor{textcolor}{rgb}{0.000000,0.000000,0.000000}%
\pgfsetstrokecolor{textcolor}%
\pgfsetfillcolor{textcolor}%
\pgftext[x=2.108766in,y=0.402778in,,top]{\color{textcolor}\rmfamily\fontsize{10.000000}{12.000000}\selectfont \(\displaystyle {20}\)}%
\end{pgfscope}%
\begin{pgfscope}%
\pgfsetbuttcap%
\pgfsetroundjoin%
\definecolor{currentfill}{rgb}{0.000000,0.000000,0.000000}%
\pgfsetfillcolor{currentfill}%
\pgfsetlinewidth{0.803000pt}%
\definecolor{currentstroke}{rgb}{0.000000,0.000000,0.000000}%
\pgfsetstrokecolor{currentstroke}%
\pgfsetdash{}{0pt}%
\pgfsys@defobject{currentmarker}{\pgfqpoint{0.000000in}{-0.048611in}}{\pgfqpoint{0.000000in}{0.000000in}}{%
\pgfpathmoveto{\pgfqpoint{0.000000in}{0.000000in}}%
\pgfpathlineto{\pgfqpoint{0.000000in}{-0.048611in}}%
\pgfusepath{stroke,fill}%
}%
\begin{pgfscope}%
\pgfsys@transformshift{2.712662in}{0.500000in}%
\pgfsys@useobject{currentmarker}{}%
\end{pgfscope}%
\end{pgfscope}%
\begin{pgfscope}%
\definecolor{textcolor}{rgb}{0.000000,0.000000,0.000000}%
\pgfsetstrokecolor{textcolor}%
\pgfsetfillcolor{textcolor}%
\pgftext[x=2.712662in,y=0.402778in,,top]{\color{textcolor}\rmfamily\fontsize{10.000000}{12.000000}\selectfont \(\displaystyle {30}\)}%
\end{pgfscope}%
\begin{pgfscope}%
\pgfsetbuttcap%
\pgfsetroundjoin%
\definecolor{currentfill}{rgb}{0.000000,0.000000,0.000000}%
\pgfsetfillcolor{currentfill}%
\pgfsetlinewidth{0.803000pt}%
\definecolor{currentstroke}{rgb}{0.000000,0.000000,0.000000}%
\pgfsetstrokecolor{currentstroke}%
\pgfsetdash{}{0pt}%
\pgfsys@defobject{currentmarker}{\pgfqpoint{0.000000in}{-0.048611in}}{\pgfqpoint{0.000000in}{0.000000in}}{%
\pgfpathmoveto{\pgfqpoint{0.000000in}{0.000000in}}%
\pgfpathlineto{\pgfqpoint{0.000000in}{-0.048611in}}%
\pgfusepath{stroke,fill}%
}%
\begin{pgfscope}%
\pgfsys@transformshift{3.316558in}{0.500000in}%
\pgfsys@useobject{currentmarker}{}%
\end{pgfscope}%
\end{pgfscope}%
\begin{pgfscope}%
\definecolor{textcolor}{rgb}{0.000000,0.000000,0.000000}%
\pgfsetstrokecolor{textcolor}%
\pgfsetfillcolor{textcolor}%
\pgftext[x=3.316558in,y=0.402778in,,top]{\color{textcolor}\rmfamily\fontsize{10.000000}{12.000000}\selectfont \(\displaystyle {40}\)}%
\end{pgfscope}%
\begin{pgfscope}%
\pgfsetbuttcap%
\pgfsetroundjoin%
\definecolor{currentfill}{rgb}{0.000000,0.000000,0.000000}%
\pgfsetfillcolor{currentfill}%
\pgfsetlinewidth{0.803000pt}%
\definecolor{currentstroke}{rgb}{0.000000,0.000000,0.000000}%
\pgfsetstrokecolor{currentstroke}%
\pgfsetdash{}{0pt}%
\pgfsys@defobject{currentmarker}{\pgfqpoint{0.000000in}{-0.048611in}}{\pgfqpoint{0.000000in}{0.000000in}}{%
\pgfpathmoveto{\pgfqpoint{0.000000in}{0.000000in}}%
\pgfpathlineto{\pgfqpoint{0.000000in}{-0.048611in}}%
\pgfusepath{stroke,fill}%
}%
\begin{pgfscope}%
\pgfsys@transformshift{3.920455in}{0.500000in}%
\pgfsys@useobject{currentmarker}{}%
\end{pgfscope}%
\end{pgfscope}%
\begin{pgfscope}%
\definecolor{textcolor}{rgb}{0.000000,0.000000,0.000000}%
\pgfsetstrokecolor{textcolor}%
\pgfsetfillcolor{textcolor}%
\pgftext[x=3.920455in,y=0.402778in,,top]{\color{textcolor}\rmfamily\fontsize{10.000000}{12.000000}\selectfont \(\displaystyle {50}\)}%
\end{pgfscope}%
\begin{pgfscope}%
\pgfsetbuttcap%
\pgfsetroundjoin%
\definecolor{currentfill}{rgb}{0.000000,0.000000,0.000000}%
\pgfsetfillcolor{currentfill}%
\pgfsetlinewidth{0.803000pt}%
\definecolor{currentstroke}{rgb}{0.000000,0.000000,0.000000}%
\pgfsetstrokecolor{currentstroke}%
\pgfsetdash{}{0pt}%
\pgfsys@defobject{currentmarker}{\pgfqpoint{0.000000in}{-0.048611in}}{\pgfqpoint{0.000000in}{0.000000in}}{%
\pgfpathmoveto{\pgfqpoint{0.000000in}{0.000000in}}%
\pgfpathlineto{\pgfqpoint{0.000000in}{-0.048611in}}%
\pgfusepath{stroke,fill}%
}%
\begin{pgfscope}%
\pgfsys@transformshift{4.524351in}{0.500000in}%
\pgfsys@useobject{currentmarker}{}%
\end{pgfscope}%
\end{pgfscope}%
\begin{pgfscope}%
\definecolor{textcolor}{rgb}{0.000000,0.000000,0.000000}%
\pgfsetstrokecolor{textcolor}%
\pgfsetfillcolor{textcolor}%
\pgftext[x=4.524351in,y=0.402778in,,top]{\color{textcolor}\rmfamily\fontsize{10.000000}{12.000000}\selectfont \(\displaystyle {60}\)}%
\end{pgfscope}%
\begin{pgfscope}%
\pgfsetbuttcap%
\pgfsetroundjoin%
\definecolor{currentfill}{rgb}{0.000000,0.000000,0.000000}%
\pgfsetfillcolor{currentfill}%
\pgfsetlinewidth{0.803000pt}%
\definecolor{currentstroke}{rgb}{0.000000,0.000000,0.000000}%
\pgfsetstrokecolor{currentstroke}%
\pgfsetdash{}{0pt}%
\pgfsys@defobject{currentmarker}{\pgfqpoint{0.000000in}{-0.048611in}}{\pgfqpoint{0.000000in}{0.000000in}}{%
\pgfpathmoveto{\pgfqpoint{0.000000in}{0.000000in}}%
\pgfpathlineto{\pgfqpoint{0.000000in}{-0.048611in}}%
\pgfusepath{stroke,fill}%
}%
\begin{pgfscope}%
\pgfsys@transformshift{5.128247in}{0.500000in}%
\pgfsys@useobject{currentmarker}{}%
\end{pgfscope}%
\end{pgfscope}%
\begin{pgfscope}%
\definecolor{textcolor}{rgb}{0.000000,0.000000,0.000000}%
\pgfsetstrokecolor{textcolor}%
\pgfsetfillcolor{textcolor}%
\pgftext[x=5.128247in,y=0.402778in,,top]{\color{textcolor}\rmfamily\fontsize{10.000000}{12.000000}\selectfont \(\displaystyle {70}\)}%
\end{pgfscope}%
\begin{pgfscope}%
\definecolor{textcolor}{rgb}{0.000000,0.000000,0.000000}%
\pgfsetstrokecolor{textcolor}%
\pgfsetfillcolor{textcolor}%
\pgftext[x=3.075000in,y=0.223889in,,top]{\color{textcolor}\rmfamily\fontsize{10.000000}{12.000000}\selectfont Number of Sources}%
\end{pgfscope}%
\begin{pgfscope}%
\pgfsetbuttcap%
\pgfsetroundjoin%
\definecolor{currentfill}{rgb}{0.000000,0.000000,0.000000}%
\pgfsetfillcolor{currentfill}%
\pgfsetlinewidth{0.803000pt}%
\definecolor{currentstroke}{rgb}{0.000000,0.000000,0.000000}%
\pgfsetstrokecolor{currentstroke}%
\pgfsetdash{}{0pt}%
\pgfsys@defobject{currentmarker}{\pgfqpoint{-0.048611in}{0.000000in}}{\pgfqpoint{0.000000in}{0.000000in}}{%
\pgfpathmoveto{\pgfqpoint{0.000000in}{0.000000in}}%
\pgfpathlineto{\pgfqpoint{-0.048611in}{0.000000in}}%
\pgfusepath{stroke,fill}%
}%
\begin{pgfscope}%
\pgfsys@transformshift{0.750000in}{0.637273in}%
\pgfsys@useobject{currentmarker}{}%
\end{pgfscope}%
\end{pgfscope}%
\begin{pgfscope}%
\definecolor{textcolor}{rgb}{0.000000,0.000000,0.000000}%
\pgfsetstrokecolor{textcolor}%
\pgfsetfillcolor{textcolor}%
\pgftext[x=0.583333in, y=0.589078in, left, base]{\color{textcolor}\rmfamily\fontsize{10.000000}{12.000000}\selectfont \(\displaystyle {0}\)}%
\end{pgfscope}%
\begin{pgfscope}%
\pgfsetbuttcap%
\pgfsetroundjoin%
\definecolor{currentfill}{rgb}{0.000000,0.000000,0.000000}%
\pgfsetfillcolor{currentfill}%
\pgfsetlinewidth{0.803000pt}%
\definecolor{currentstroke}{rgb}{0.000000,0.000000,0.000000}%
\pgfsetstrokecolor{currentstroke}%
\pgfsetdash{}{0pt}%
\pgfsys@defobject{currentmarker}{\pgfqpoint{-0.048611in}{0.000000in}}{\pgfqpoint{0.000000in}{0.000000in}}{%
\pgfpathmoveto{\pgfqpoint{0.000000in}{0.000000in}}%
\pgfpathlineto{\pgfqpoint{-0.048611in}{0.000000in}}%
\pgfusepath{stroke,fill}%
}%
\begin{pgfscope}%
\pgfsys@transformshift{0.750000in}{1.026699in}%
\pgfsys@useobject{currentmarker}{}%
\end{pgfscope}%
\end{pgfscope}%
\begin{pgfscope}%
\definecolor{textcolor}{rgb}{0.000000,0.000000,0.000000}%
\pgfsetstrokecolor{textcolor}%
\pgfsetfillcolor{textcolor}%
\pgftext[x=0.444444in, y=0.978504in, left, base]{\color{textcolor}\rmfamily\fontsize{10.000000}{12.000000}\selectfont \(\displaystyle {100}\)}%
\end{pgfscope}%
\begin{pgfscope}%
\pgfsetbuttcap%
\pgfsetroundjoin%
\definecolor{currentfill}{rgb}{0.000000,0.000000,0.000000}%
\pgfsetfillcolor{currentfill}%
\pgfsetlinewidth{0.803000pt}%
\definecolor{currentstroke}{rgb}{0.000000,0.000000,0.000000}%
\pgfsetstrokecolor{currentstroke}%
\pgfsetdash{}{0pt}%
\pgfsys@defobject{currentmarker}{\pgfqpoint{-0.048611in}{0.000000in}}{\pgfqpoint{0.000000in}{0.000000in}}{%
\pgfpathmoveto{\pgfqpoint{0.000000in}{0.000000in}}%
\pgfpathlineto{\pgfqpoint{-0.048611in}{0.000000in}}%
\pgfusepath{stroke,fill}%
}%
\begin{pgfscope}%
\pgfsys@transformshift{0.750000in}{1.416125in}%
\pgfsys@useobject{currentmarker}{}%
\end{pgfscope}%
\end{pgfscope}%
\begin{pgfscope}%
\definecolor{textcolor}{rgb}{0.000000,0.000000,0.000000}%
\pgfsetstrokecolor{textcolor}%
\pgfsetfillcolor{textcolor}%
\pgftext[x=0.444444in, y=1.367931in, left, base]{\color{textcolor}\rmfamily\fontsize{10.000000}{12.000000}\selectfont \(\displaystyle {200}\)}%
\end{pgfscope}%
\begin{pgfscope}%
\pgfsetbuttcap%
\pgfsetroundjoin%
\definecolor{currentfill}{rgb}{0.000000,0.000000,0.000000}%
\pgfsetfillcolor{currentfill}%
\pgfsetlinewidth{0.803000pt}%
\definecolor{currentstroke}{rgb}{0.000000,0.000000,0.000000}%
\pgfsetstrokecolor{currentstroke}%
\pgfsetdash{}{0pt}%
\pgfsys@defobject{currentmarker}{\pgfqpoint{-0.048611in}{0.000000in}}{\pgfqpoint{0.000000in}{0.000000in}}{%
\pgfpathmoveto{\pgfqpoint{0.000000in}{0.000000in}}%
\pgfpathlineto{\pgfqpoint{-0.048611in}{0.000000in}}%
\pgfusepath{stroke,fill}%
}%
\begin{pgfscope}%
\pgfsys@transformshift{0.750000in}{1.805551in}%
\pgfsys@useobject{currentmarker}{}%
\end{pgfscope}%
\end{pgfscope}%
\begin{pgfscope}%
\definecolor{textcolor}{rgb}{0.000000,0.000000,0.000000}%
\pgfsetstrokecolor{textcolor}%
\pgfsetfillcolor{textcolor}%
\pgftext[x=0.444444in, y=1.757357in, left, base]{\color{textcolor}\rmfamily\fontsize{10.000000}{12.000000}\selectfont \(\displaystyle {300}\)}%
\end{pgfscope}%
\begin{pgfscope}%
\pgfsetbuttcap%
\pgfsetroundjoin%
\definecolor{currentfill}{rgb}{0.000000,0.000000,0.000000}%
\pgfsetfillcolor{currentfill}%
\pgfsetlinewidth{0.803000pt}%
\definecolor{currentstroke}{rgb}{0.000000,0.000000,0.000000}%
\pgfsetstrokecolor{currentstroke}%
\pgfsetdash{}{0pt}%
\pgfsys@defobject{currentmarker}{\pgfqpoint{-0.048611in}{0.000000in}}{\pgfqpoint{0.000000in}{0.000000in}}{%
\pgfpathmoveto{\pgfqpoint{0.000000in}{0.000000in}}%
\pgfpathlineto{\pgfqpoint{-0.048611in}{0.000000in}}%
\pgfusepath{stroke,fill}%
}%
\begin{pgfscope}%
\pgfsys@transformshift{0.750000in}{2.194977in}%
\pgfsys@useobject{currentmarker}{}%
\end{pgfscope}%
\end{pgfscope}%
\begin{pgfscope}%
\definecolor{textcolor}{rgb}{0.000000,0.000000,0.000000}%
\pgfsetstrokecolor{textcolor}%
\pgfsetfillcolor{textcolor}%
\pgftext[x=0.444444in, y=2.146783in, left, base]{\color{textcolor}\rmfamily\fontsize{10.000000}{12.000000}\selectfont \(\displaystyle {400}\)}%
\end{pgfscope}%
\begin{pgfscope}%
\pgfsetbuttcap%
\pgfsetroundjoin%
\definecolor{currentfill}{rgb}{0.000000,0.000000,0.000000}%
\pgfsetfillcolor{currentfill}%
\pgfsetlinewidth{0.803000pt}%
\definecolor{currentstroke}{rgb}{0.000000,0.000000,0.000000}%
\pgfsetstrokecolor{currentstroke}%
\pgfsetdash{}{0pt}%
\pgfsys@defobject{currentmarker}{\pgfqpoint{-0.048611in}{0.000000in}}{\pgfqpoint{0.000000in}{0.000000in}}{%
\pgfpathmoveto{\pgfqpoint{0.000000in}{0.000000in}}%
\pgfpathlineto{\pgfqpoint{-0.048611in}{0.000000in}}%
\pgfusepath{stroke,fill}%
}%
\begin{pgfscope}%
\pgfsys@transformshift{0.750000in}{2.584404in}%
\pgfsys@useobject{currentmarker}{}%
\end{pgfscope}%
\end{pgfscope}%
\begin{pgfscope}%
\definecolor{textcolor}{rgb}{0.000000,0.000000,0.000000}%
\pgfsetstrokecolor{textcolor}%
\pgfsetfillcolor{textcolor}%
\pgftext[x=0.444444in, y=2.536209in, left, base]{\color{textcolor}\rmfamily\fontsize{10.000000}{12.000000}\selectfont \(\displaystyle {500}\)}%
\end{pgfscope}%
\begin{pgfscope}%
\pgfsetbuttcap%
\pgfsetroundjoin%
\definecolor{currentfill}{rgb}{0.000000,0.000000,0.000000}%
\pgfsetfillcolor{currentfill}%
\pgfsetlinewidth{0.803000pt}%
\definecolor{currentstroke}{rgb}{0.000000,0.000000,0.000000}%
\pgfsetstrokecolor{currentstroke}%
\pgfsetdash{}{0pt}%
\pgfsys@defobject{currentmarker}{\pgfqpoint{-0.048611in}{0.000000in}}{\pgfqpoint{0.000000in}{0.000000in}}{%
\pgfpathmoveto{\pgfqpoint{0.000000in}{0.000000in}}%
\pgfpathlineto{\pgfqpoint{-0.048611in}{0.000000in}}%
\pgfusepath{stroke,fill}%
}%
\begin{pgfscope}%
\pgfsys@transformshift{0.750000in}{2.973830in}%
\pgfsys@useobject{currentmarker}{}%
\end{pgfscope}%
\end{pgfscope}%
\begin{pgfscope}%
\definecolor{textcolor}{rgb}{0.000000,0.000000,0.000000}%
\pgfsetstrokecolor{textcolor}%
\pgfsetfillcolor{textcolor}%
\pgftext[x=0.444444in, y=2.925635in, left, base]{\color{textcolor}\rmfamily\fontsize{10.000000}{12.000000}\selectfont \(\displaystyle {600}\)}%
\end{pgfscope}%
\begin{pgfscope}%
\pgfsetbuttcap%
\pgfsetroundjoin%
\definecolor{currentfill}{rgb}{0.000000,0.000000,0.000000}%
\pgfsetfillcolor{currentfill}%
\pgfsetlinewidth{0.803000pt}%
\definecolor{currentstroke}{rgb}{0.000000,0.000000,0.000000}%
\pgfsetstrokecolor{currentstroke}%
\pgfsetdash{}{0pt}%
\pgfsys@defobject{currentmarker}{\pgfqpoint{-0.048611in}{0.000000in}}{\pgfqpoint{0.000000in}{0.000000in}}{%
\pgfpathmoveto{\pgfqpoint{0.000000in}{0.000000in}}%
\pgfpathlineto{\pgfqpoint{-0.048611in}{0.000000in}}%
\pgfusepath{stroke,fill}%
}%
\begin{pgfscope}%
\pgfsys@transformshift{0.750000in}{3.363256in}%
\pgfsys@useobject{currentmarker}{}%
\end{pgfscope}%
\end{pgfscope}%
\begin{pgfscope}%
\definecolor{textcolor}{rgb}{0.000000,0.000000,0.000000}%
\pgfsetstrokecolor{textcolor}%
\pgfsetfillcolor{textcolor}%
\pgftext[x=0.444444in, y=3.315062in, left, base]{\color{textcolor}\rmfamily\fontsize{10.000000}{12.000000}\selectfont \(\displaystyle {700}\)}%
\end{pgfscope}%
\begin{pgfscope}%
\definecolor{textcolor}{rgb}{0.000000,0.000000,0.000000}%
\pgfsetstrokecolor{textcolor}%
\pgfsetfillcolor{textcolor}%
\pgftext[x=0.388888in,y=2.010000in,,bottom,rotate=90.000000]{\color{textcolor}\rmfamily\fontsize{10.000000}{12.000000}\selectfont Dataflow Time}%
\end{pgfscope}%
\begin{pgfscope}%
\pgfsetrectcap%
\pgfsetmiterjoin%
\pgfsetlinewidth{0.803000pt}%
\definecolor{currentstroke}{rgb}{0.000000,0.000000,0.000000}%
\pgfsetstrokecolor{currentstroke}%
\pgfsetdash{}{0pt}%
\pgfpathmoveto{\pgfqpoint{0.750000in}{0.500000in}}%
\pgfpathlineto{\pgfqpoint{0.750000in}{3.520000in}}%
\pgfusepath{stroke}%
\end{pgfscope}%
\begin{pgfscope}%
\pgfsetrectcap%
\pgfsetmiterjoin%
\pgfsetlinewidth{0.803000pt}%
\definecolor{currentstroke}{rgb}{0.000000,0.000000,0.000000}%
\pgfsetstrokecolor{currentstroke}%
\pgfsetdash{}{0pt}%
\pgfpathmoveto{\pgfqpoint{5.400000in}{0.500000in}}%
\pgfpathlineto{\pgfqpoint{5.400000in}{3.520000in}}%
\pgfusepath{stroke}%
\end{pgfscope}%
\begin{pgfscope}%
\pgfsetrectcap%
\pgfsetmiterjoin%
\pgfsetlinewidth{0.803000pt}%
\definecolor{currentstroke}{rgb}{0.000000,0.000000,0.000000}%
\pgfsetstrokecolor{currentstroke}%
\pgfsetdash{}{0pt}%
\pgfpathmoveto{\pgfqpoint{0.750000in}{0.500000in}}%
\pgfpathlineto{\pgfqpoint{5.400000in}{0.500000in}}%
\pgfusepath{stroke}%
\end{pgfscope}%
\begin{pgfscope}%
\pgfsetrectcap%
\pgfsetmiterjoin%
\pgfsetlinewidth{0.803000pt}%
\definecolor{currentstroke}{rgb}{0.000000,0.000000,0.000000}%
\pgfsetstrokecolor{currentstroke}%
\pgfsetdash{}{0pt}%
\pgfpathmoveto{\pgfqpoint{0.750000in}{3.520000in}}%
\pgfpathlineto{\pgfqpoint{5.400000in}{3.520000in}}%
\pgfusepath{stroke}%
\end{pgfscope}%
\begin{pgfscope}%
\definecolor{textcolor}{rgb}{0.000000,0.000000,0.000000}%
\pgfsetstrokecolor{textcolor}%
\pgfsetfillcolor{textcolor}%
\pgftext[x=3.075000in,y=3.603333in,,base]{\color{textcolor}\rmfamily\fontsize{12.000000}{14.400000}\selectfont Forwards}%
\end{pgfscope}%
\begin{pgfscope}%
\pgfsetbuttcap%
\pgfsetmiterjoin%
\definecolor{currentfill}{rgb}{1.000000,1.000000,1.000000}%
\pgfsetfillcolor{currentfill}%
\pgfsetfillopacity{0.800000}%
\pgfsetlinewidth{1.003750pt}%
\definecolor{currentstroke}{rgb}{0.800000,0.800000,0.800000}%
\pgfsetstrokecolor{currentstroke}%
\pgfsetstrokeopacity{0.800000}%
\pgfsetdash{}{0pt}%
\pgfpathmoveto{\pgfqpoint{3.793194in}{0.569444in}}%
\pgfpathlineto{\pgfqpoint{5.302778in}{0.569444in}}%
\pgfpathquadraticcurveto{\pgfqpoint{5.330556in}{0.569444in}}{\pgfqpoint{5.330556in}{0.597222in}}%
\pgfpathlineto{\pgfqpoint{5.330556in}{1.165694in}}%
\pgfpathquadraticcurveto{\pgfqpoint{5.330556in}{1.193472in}}{\pgfqpoint{5.302778in}{1.193472in}}%
\pgfpathlineto{\pgfqpoint{3.793194in}{1.193472in}}%
\pgfpathquadraticcurveto{\pgfqpoint{3.765417in}{1.193472in}}{\pgfqpoint{3.765417in}{1.165694in}}%
\pgfpathlineto{\pgfqpoint{3.765417in}{0.597222in}}%
\pgfpathquadraticcurveto{\pgfqpoint{3.765417in}{0.569444in}}{\pgfqpoint{3.793194in}{0.569444in}}%
\pgfpathclose%
\pgfusepath{stroke,fill}%
\end{pgfscope}%
\begin{pgfscope}%
\pgfsetbuttcap%
\pgfsetroundjoin%
\definecolor{currentfill}{rgb}{0.121569,0.466667,0.705882}%
\pgfsetfillcolor{currentfill}%
\pgfsetlinewidth{1.003750pt}%
\definecolor{currentstroke}{rgb}{0.121569,0.466667,0.705882}%
\pgfsetstrokecolor{currentstroke}%
\pgfsetdash{}{0pt}%
\pgfsys@defobject{currentmarker}{\pgfqpoint{-0.034722in}{-0.034722in}}{\pgfqpoint{0.034722in}{0.034722in}}{%
\pgfpathmoveto{\pgfqpoint{0.000000in}{-0.034722in}}%
\pgfpathcurveto{\pgfqpoint{0.009208in}{-0.034722in}}{\pgfqpoint{0.018041in}{-0.031064in}}{\pgfqpoint{0.024552in}{-0.024552in}}%
\pgfpathcurveto{\pgfqpoint{0.031064in}{-0.018041in}}{\pgfqpoint{0.034722in}{-0.009208in}}{\pgfqpoint{0.034722in}{0.000000in}}%
\pgfpathcurveto{\pgfqpoint{0.034722in}{0.009208in}}{\pgfqpoint{0.031064in}{0.018041in}}{\pgfqpoint{0.024552in}{0.024552in}}%
\pgfpathcurveto{\pgfqpoint{0.018041in}{0.031064in}}{\pgfqpoint{0.009208in}{0.034722in}}{\pgfqpoint{0.000000in}{0.034722in}}%
\pgfpathcurveto{\pgfqpoint{-0.009208in}{0.034722in}}{\pgfqpoint{-0.018041in}{0.031064in}}{\pgfqpoint{-0.024552in}{0.024552in}}%
\pgfpathcurveto{\pgfqpoint{-0.031064in}{0.018041in}}{\pgfqpoint{-0.034722in}{0.009208in}}{\pgfqpoint{-0.034722in}{0.000000in}}%
\pgfpathcurveto{\pgfqpoint{-0.034722in}{-0.009208in}}{\pgfqpoint{-0.031064in}{-0.018041in}}{\pgfqpoint{-0.024552in}{-0.024552in}}%
\pgfpathcurveto{\pgfqpoint{-0.018041in}{-0.031064in}}{\pgfqpoint{-0.009208in}{-0.034722in}}{\pgfqpoint{0.000000in}{-0.034722in}}%
\pgfpathclose%
\pgfusepath{stroke,fill}%
}%
\begin{pgfscope}%
\pgfsys@transformshift{3.959861in}{1.089306in}%
\pgfsys@useobject{currentmarker}{}%
\end{pgfscope}%
\end{pgfscope}%
\begin{pgfscope}%
\definecolor{textcolor}{rgb}{0.000000,0.000000,0.000000}%
\pgfsetstrokecolor{textcolor}%
\pgfsetfillcolor{textcolor}%
\pgftext[x=4.209861in,y=1.040694in,left,base]{\color{textcolor}\rmfamily\fontsize{10.000000}{12.000000}\selectfont No Timeout}%
\end{pgfscope}%
\begin{pgfscope}%
\pgfsetbuttcap%
\pgfsetroundjoin%
\definecolor{currentfill}{rgb}{1.000000,0.498039,0.054902}%
\pgfsetfillcolor{currentfill}%
\pgfsetlinewidth{1.003750pt}%
\definecolor{currentstroke}{rgb}{1.000000,0.498039,0.054902}%
\pgfsetstrokecolor{currentstroke}%
\pgfsetdash{}{0pt}%
\pgfsys@defobject{currentmarker}{\pgfqpoint{-0.034722in}{-0.034722in}}{\pgfqpoint{0.034722in}{0.034722in}}{%
\pgfpathmoveto{\pgfqpoint{0.000000in}{-0.034722in}}%
\pgfpathcurveto{\pgfqpoint{0.009208in}{-0.034722in}}{\pgfqpoint{0.018041in}{-0.031064in}}{\pgfqpoint{0.024552in}{-0.024552in}}%
\pgfpathcurveto{\pgfqpoint{0.031064in}{-0.018041in}}{\pgfqpoint{0.034722in}{-0.009208in}}{\pgfqpoint{0.034722in}{0.000000in}}%
\pgfpathcurveto{\pgfqpoint{0.034722in}{0.009208in}}{\pgfqpoint{0.031064in}{0.018041in}}{\pgfqpoint{0.024552in}{0.024552in}}%
\pgfpathcurveto{\pgfqpoint{0.018041in}{0.031064in}}{\pgfqpoint{0.009208in}{0.034722in}}{\pgfqpoint{0.000000in}{0.034722in}}%
\pgfpathcurveto{\pgfqpoint{-0.009208in}{0.034722in}}{\pgfqpoint{-0.018041in}{0.031064in}}{\pgfqpoint{-0.024552in}{0.024552in}}%
\pgfpathcurveto{\pgfqpoint{-0.031064in}{0.018041in}}{\pgfqpoint{-0.034722in}{0.009208in}}{\pgfqpoint{-0.034722in}{0.000000in}}%
\pgfpathcurveto{\pgfqpoint{-0.034722in}{-0.009208in}}{\pgfqpoint{-0.031064in}{-0.018041in}}{\pgfqpoint{-0.024552in}{-0.024552in}}%
\pgfpathcurveto{\pgfqpoint{-0.018041in}{-0.031064in}}{\pgfqpoint{-0.009208in}{-0.034722in}}{\pgfqpoint{0.000000in}{-0.034722in}}%
\pgfpathclose%
\pgfusepath{stroke,fill}%
}%
\begin{pgfscope}%
\pgfsys@transformshift{3.959861in}{0.895694in}%
\pgfsys@useobject{currentmarker}{}%
\end{pgfscope}%
\end{pgfscope}%
\begin{pgfscope}%
\definecolor{textcolor}{rgb}{0.000000,0.000000,0.000000}%
\pgfsetstrokecolor{textcolor}%
\pgfsetfillcolor{textcolor}%
\pgftext[x=4.209861in,y=0.847083in,left,base]{\color{textcolor}\rmfamily\fontsize{10.000000}{12.000000}\selectfont Time Timeout}%
\end{pgfscope}%
\begin{pgfscope}%
\pgfsetbuttcap%
\pgfsetroundjoin%
\definecolor{currentfill}{rgb}{0.839216,0.152941,0.156863}%
\pgfsetfillcolor{currentfill}%
\pgfsetlinewidth{1.003750pt}%
\definecolor{currentstroke}{rgb}{0.839216,0.152941,0.156863}%
\pgfsetstrokecolor{currentstroke}%
\pgfsetdash{}{0pt}%
\pgfsys@defobject{currentmarker}{\pgfqpoint{-0.034722in}{-0.034722in}}{\pgfqpoint{0.034722in}{0.034722in}}{%
\pgfpathmoveto{\pgfqpoint{0.000000in}{-0.034722in}}%
\pgfpathcurveto{\pgfqpoint{0.009208in}{-0.034722in}}{\pgfqpoint{0.018041in}{-0.031064in}}{\pgfqpoint{0.024552in}{-0.024552in}}%
\pgfpathcurveto{\pgfqpoint{0.031064in}{-0.018041in}}{\pgfqpoint{0.034722in}{-0.009208in}}{\pgfqpoint{0.034722in}{0.000000in}}%
\pgfpathcurveto{\pgfqpoint{0.034722in}{0.009208in}}{\pgfqpoint{0.031064in}{0.018041in}}{\pgfqpoint{0.024552in}{0.024552in}}%
\pgfpathcurveto{\pgfqpoint{0.018041in}{0.031064in}}{\pgfqpoint{0.009208in}{0.034722in}}{\pgfqpoint{0.000000in}{0.034722in}}%
\pgfpathcurveto{\pgfqpoint{-0.009208in}{0.034722in}}{\pgfqpoint{-0.018041in}{0.031064in}}{\pgfqpoint{-0.024552in}{0.024552in}}%
\pgfpathcurveto{\pgfqpoint{-0.031064in}{0.018041in}}{\pgfqpoint{-0.034722in}{0.009208in}}{\pgfqpoint{-0.034722in}{0.000000in}}%
\pgfpathcurveto{\pgfqpoint{-0.034722in}{-0.009208in}}{\pgfqpoint{-0.031064in}{-0.018041in}}{\pgfqpoint{-0.024552in}{-0.024552in}}%
\pgfpathcurveto{\pgfqpoint{-0.018041in}{-0.031064in}}{\pgfqpoint{-0.009208in}{-0.034722in}}{\pgfqpoint{0.000000in}{-0.034722in}}%
\pgfpathclose%
\pgfusepath{stroke,fill}%
}%
\begin{pgfscope}%
\pgfsys@transformshift{3.959861in}{0.702083in}%
\pgfsys@useobject{currentmarker}{}%
\end{pgfscope}%
\end{pgfscope}%
\begin{pgfscope}%
\definecolor{textcolor}{rgb}{0.000000,0.000000,0.000000}%
\pgfsetstrokecolor{textcolor}%
\pgfsetfillcolor{textcolor}%
\pgftext[x=4.209861in,y=0.653472in,left,base]{\color{textcolor}\rmfamily\fontsize{10.000000}{12.000000}\selectfont Memory Timeout}%
\end{pgfscope}%
\end{pgfpicture}%
\makeatother%
\endgroup%

            }
        \end{subfigure}
        \qquad
        \begin{subfigure}[]{0.45\textwidth}
            \centering
            \resizebox{\columnwidth}{!}{
                %% Creator: Matplotlib, PGF backend
%%
%% To include the figure in your LaTeX document, write
%%   \input{<filename>.pgf}
%%
%% Make sure the required packages are loaded in your preamble
%%   \usepackage{pgf}
%%
%% and, on pdftex
%%   \usepackage[utf8]{inputenc}\DeclareUnicodeCharacter{2212}{-}
%%
%% or, on luatex and xetex
%%   \usepackage{unicode-math}
%%
%% Figures using additional raster images can only be included by \input if
%% they are in the same directory as the main LaTeX file. For loading figures
%% from other directories you can use the `import` package
%%   \usepackage{import}
%%
%% and then include the figures with
%%   \import{<path to file>}{<filename>.pgf}
%%
%% Matplotlib used the following preamble
%%   \usepackage{amsmath}
%%   \usepackage{fontspec}
%%
\begingroup%
\makeatletter%
\begin{pgfpicture}%
\pgfpathrectangle{\pgfpointorigin}{\pgfqpoint{6.000000in}{4.000000in}}%
\pgfusepath{use as bounding box, clip}%
\begin{pgfscope}%
\pgfsetbuttcap%
\pgfsetmiterjoin%
\definecolor{currentfill}{rgb}{1.000000,1.000000,1.000000}%
\pgfsetfillcolor{currentfill}%
\pgfsetlinewidth{0.000000pt}%
\definecolor{currentstroke}{rgb}{1.000000,1.000000,1.000000}%
\pgfsetstrokecolor{currentstroke}%
\pgfsetdash{}{0pt}%
\pgfpathmoveto{\pgfqpoint{0.000000in}{0.000000in}}%
\pgfpathlineto{\pgfqpoint{6.000000in}{0.000000in}}%
\pgfpathlineto{\pgfqpoint{6.000000in}{4.000000in}}%
\pgfpathlineto{\pgfqpoint{0.000000in}{4.000000in}}%
\pgfpathclose%
\pgfusepath{fill}%
\end{pgfscope}%
\begin{pgfscope}%
\pgfsetbuttcap%
\pgfsetmiterjoin%
\definecolor{currentfill}{rgb}{1.000000,1.000000,1.000000}%
\pgfsetfillcolor{currentfill}%
\pgfsetlinewidth{0.000000pt}%
\definecolor{currentstroke}{rgb}{0.000000,0.000000,0.000000}%
\pgfsetstrokecolor{currentstroke}%
\pgfsetstrokeopacity{0.000000}%
\pgfsetdash{}{0pt}%
\pgfpathmoveto{\pgfqpoint{0.648703in}{0.548769in}}%
\pgfpathlineto{\pgfqpoint{5.850000in}{0.548769in}}%
\pgfpathlineto{\pgfqpoint{5.850000in}{3.651359in}}%
\pgfpathlineto{\pgfqpoint{0.648703in}{3.651359in}}%
\pgfpathclose%
\pgfusepath{fill}%
\end{pgfscope}%
\begin{pgfscope}%
\pgfpathrectangle{\pgfqpoint{0.648703in}{0.548769in}}{\pgfqpoint{5.201297in}{3.102590in}}%
\pgfusepath{clip}%
\pgfsetbuttcap%
\pgfsetroundjoin%
\definecolor{currentfill}{rgb}{0.121569,0.466667,0.705882}%
\pgfsetfillcolor{currentfill}%
\pgfsetlinewidth{1.003750pt}%
\definecolor{currentstroke}{rgb}{0.121569,0.466667,0.705882}%
\pgfsetstrokecolor{currentstroke}%
\pgfsetdash{}{0pt}%
\pgfpathmoveto{\pgfqpoint{0.949899in}{0.648129in}}%
\pgfpathcurveto{\pgfqpoint{0.960949in}{0.648129in}}{\pgfqpoint{0.971548in}{0.652519in}}{\pgfqpoint{0.979362in}{0.660333in}}%
\pgfpathcurveto{\pgfqpoint{0.987176in}{0.668146in}}{\pgfqpoint{0.991566in}{0.678745in}}{\pgfqpoint{0.991566in}{0.689796in}}%
\pgfpathcurveto{\pgfqpoint{0.991566in}{0.700846in}}{\pgfqpoint{0.987176in}{0.711445in}}{\pgfqpoint{0.979362in}{0.719258in}}%
\pgfpathcurveto{\pgfqpoint{0.971548in}{0.727072in}}{\pgfqpoint{0.960949in}{0.731462in}}{\pgfqpoint{0.949899in}{0.731462in}}%
\pgfpathcurveto{\pgfqpoint{0.938849in}{0.731462in}}{\pgfqpoint{0.928250in}{0.727072in}}{\pgfqpoint{0.920437in}{0.719258in}}%
\pgfpathcurveto{\pgfqpoint{0.912623in}{0.711445in}}{\pgfqpoint{0.908233in}{0.700846in}}{\pgfqpoint{0.908233in}{0.689796in}}%
\pgfpathcurveto{\pgfqpoint{0.908233in}{0.678745in}}{\pgfqpoint{0.912623in}{0.668146in}}{\pgfqpoint{0.920437in}{0.660333in}}%
\pgfpathcurveto{\pgfqpoint{0.928250in}{0.652519in}}{\pgfqpoint{0.938849in}{0.648129in}}{\pgfqpoint{0.949899in}{0.648129in}}%
\pgfpathclose%
\pgfusepath{stroke,fill}%
\end{pgfscope}%
\begin{pgfscope}%
\pgfpathrectangle{\pgfqpoint{0.648703in}{0.548769in}}{\pgfqpoint{5.201297in}{3.102590in}}%
\pgfusepath{clip}%
\pgfsetbuttcap%
\pgfsetroundjoin%
\definecolor{currentfill}{rgb}{0.121569,0.466667,0.705882}%
\pgfsetfillcolor{currentfill}%
\pgfsetlinewidth{1.003750pt}%
\definecolor{currentstroke}{rgb}{0.121569,0.466667,0.705882}%
\pgfsetstrokecolor{currentstroke}%
\pgfsetdash{}{0pt}%
\pgfpathmoveto{\pgfqpoint{2.051046in}{3.124394in}}%
\pgfpathcurveto{\pgfqpoint{2.062096in}{3.124394in}}{\pgfqpoint{2.072695in}{3.128784in}}{\pgfqpoint{2.080508in}{3.136598in}}%
\pgfpathcurveto{\pgfqpoint{2.088322in}{3.144411in}}{\pgfqpoint{2.092712in}{3.155010in}}{\pgfqpoint{2.092712in}{3.166060in}}%
\pgfpathcurveto{\pgfqpoint{2.092712in}{3.177111in}}{\pgfqpoint{2.088322in}{3.187710in}}{\pgfqpoint{2.080508in}{3.195523in}}%
\pgfpathcurveto{\pgfqpoint{2.072695in}{3.203337in}}{\pgfqpoint{2.062096in}{3.207727in}}{\pgfqpoint{2.051046in}{3.207727in}}%
\pgfpathcurveto{\pgfqpoint{2.039995in}{3.207727in}}{\pgfqpoint{2.029396in}{3.203337in}}{\pgfqpoint{2.021583in}{3.195523in}}%
\pgfpathcurveto{\pgfqpoint{2.013769in}{3.187710in}}{\pgfqpoint{2.009379in}{3.177111in}}{\pgfqpoint{2.009379in}{3.166060in}}%
\pgfpathcurveto{\pgfqpoint{2.009379in}{3.155010in}}{\pgfqpoint{2.013769in}{3.144411in}}{\pgfqpoint{2.021583in}{3.136598in}}%
\pgfpathcurveto{\pgfqpoint{2.029396in}{3.128784in}}{\pgfqpoint{2.039995in}{3.124394in}}{\pgfqpoint{2.051046in}{3.124394in}}%
\pgfpathclose%
\pgfusepath{stroke,fill}%
\end{pgfscope}%
\begin{pgfscope}%
\pgfpathrectangle{\pgfqpoint{0.648703in}{0.548769in}}{\pgfqpoint{5.201297in}{3.102590in}}%
\pgfusepath{clip}%
\pgfsetbuttcap%
\pgfsetroundjoin%
\definecolor{currentfill}{rgb}{1.000000,0.498039,0.054902}%
\pgfsetfillcolor{currentfill}%
\pgfsetlinewidth{1.003750pt}%
\definecolor{currentstroke}{rgb}{1.000000,0.498039,0.054902}%
\pgfsetstrokecolor{currentstroke}%
\pgfsetdash{}{0pt}%
\pgfpathmoveto{\pgfqpoint{0.885126in}{3.140985in}}%
\pgfpathcurveto{\pgfqpoint{0.896176in}{3.140985in}}{\pgfqpoint{0.906775in}{3.145375in}}{\pgfqpoint{0.914589in}{3.153189in}}%
\pgfpathcurveto{\pgfqpoint{0.922402in}{3.161003in}}{\pgfqpoint{0.926793in}{3.171602in}}{\pgfqpoint{0.926793in}{3.182652in}}%
\pgfpathcurveto{\pgfqpoint{0.926793in}{3.193702in}}{\pgfqpoint{0.922402in}{3.204301in}}{\pgfqpoint{0.914589in}{3.212115in}}%
\pgfpathcurveto{\pgfqpoint{0.906775in}{3.219928in}}{\pgfqpoint{0.896176in}{3.224319in}}{\pgfqpoint{0.885126in}{3.224319in}}%
\pgfpathcurveto{\pgfqpoint{0.874076in}{3.224319in}}{\pgfqpoint{0.863477in}{3.219928in}}{\pgfqpoint{0.855663in}{3.212115in}}%
\pgfpathcurveto{\pgfqpoint{0.847850in}{3.204301in}}{\pgfqpoint{0.843459in}{3.193702in}}{\pgfqpoint{0.843459in}{3.182652in}}%
\pgfpathcurveto{\pgfqpoint{0.843459in}{3.171602in}}{\pgfqpoint{0.847850in}{3.161003in}}{\pgfqpoint{0.855663in}{3.153189in}}%
\pgfpathcurveto{\pgfqpoint{0.863477in}{3.145375in}}{\pgfqpoint{0.874076in}{3.140985in}}{\pgfqpoint{0.885126in}{3.140985in}}%
\pgfpathclose%
\pgfusepath{stroke,fill}%
\end{pgfscope}%
\begin{pgfscope}%
\pgfpathrectangle{\pgfqpoint{0.648703in}{0.548769in}}{\pgfqpoint{5.201297in}{3.102590in}}%
\pgfusepath{clip}%
\pgfsetbuttcap%
\pgfsetroundjoin%
\definecolor{currentfill}{rgb}{0.121569,0.466667,0.705882}%
\pgfsetfillcolor{currentfill}%
\pgfsetlinewidth{1.003750pt}%
\definecolor{currentstroke}{rgb}{0.121569,0.466667,0.705882}%
\pgfsetstrokecolor{currentstroke}%
\pgfsetdash{}{0pt}%
\pgfpathmoveto{\pgfqpoint{2.439685in}{3.132690in}}%
\pgfpathcurveto{\pgfqpoint{2.450735in}{3.132690in}}{\pgfqpoint{2.461335in}{3.137080in}}{\pgfqpoint{2.469148in}{3.144893in}}%
\pgfpathcurveto{\pgfqpoint{2.476962in}{3.152707in}}{\pgfqpoint{2.481352in}{3.163306in}}{\pgfqpoint{2.481352in}{3.174356in}}%
\pgfpathcurveto{\pgfqpoint{2.481352in}{3.185406in}}{\pgfqpoint{2.476962in}{3.196005in}}{\pgfqpoint{2.469148in}{3.203819in}}%
\pgfpathcurveto{\pgfqpoint{2.461335in}{3.211633in}}{\pgfqpoint{2.450735in}{3.216023in}}{\pgfqpoint{2.439685in}{3.216023in}}%
\pgfpathcurveto{\pgfqpoint{2.428635in}{3.216023in}}{\pgfqpoint{2.418036in}{3.211633in}}{\pgfqpoint{2.410223in}{3.203819in}}%
\pgfpathcurveto{\pgfqpoint{2.402409in}{3.196005in}}{\pgfqpoint{2.398019in}{3.185406in}}{\pgfqpoint{2.398019in}{3.174356in}}%
\pgfpathcurveto{\pgfqpoint{2.398019in}{3.163306in}}{\pgfqpoint{2.402409in}{3.152707in}}{\pgfqpoint{2.410223in}{3.144893in}}%
\pgfpathcurveto{\pgfqpoint{2.418036in}{3.137080in}}{\pgfqpoint{2.428635in}{3.132690in}}{\pgfqpoint{2.439685in}{3.132690in}}%
\pgfpathclose%
\pgfusepath{stroke,fill}%
\end{pgfscope}%
\begin{pgfscope}%
\pgfpathrectangle{\pgfqpoint{0.648703in}{0.548769in}}{\pgfqpoint{5.201297in}{3.102590in}}%
\pgfusepath{clip}%
\pgfsetbuttcap%
\pgfsetroundjoin%
\definecolor{currentfill}{rgb}{1.000000,0.498039,0.054902}%
\pgfsetfillcolor{currentfill}%
\pgfsetlinewidth{1.003750pt}%
\definecolor{currentstroke}{rgb}{1.000000,0.498039,0.054902}%
\pgfsetstrokecolor{currentstroke}%
\pgfsetdash{}{0pt}%
\pgfpathmoveto{\pgfqpoint{1.338539in}{3.136837in}}%
\pgfpathcurveto{\pgfqpoint{1.349589in}{3.136837in}}{\pgfqpoint{1.360188in}{3.141228in}}{\pgfqpoint{1.368002in}{3.149041in}}%
\pgfpathcurveto{\pgfqpoint{1.375816in}{3.156855in}}{\pgfqpoint{1.380206in}{3.167454in}}{\pgfqpoint{1.380206in}{3.178504in}}%
\pgfpathcurveto{\pgfqpoint{1.380206in}{3.189554in}}{\pgfqpoint{1.375816in}{3.200153in}}{\pgfqpoint{1.368002in}{3.207967in}}%
\pgfpathcurveto{\pgfqpoint{1.360188in}{3.215780in}}{\pgfqpoint{1.349589in}{3.220171in}}{\pgfqpoint{1.338539in}{3.220171in}}%
\pgfpathcurveto{\pgfqpoint{1.327489in}{3.220171in}}{\pgfqpoint{1.316890in}{3.215780in}}{\pgfqpoint{1.309076in}{3.207967in}}%
\pgfpathcurveto{\pgfqpoint{1.301263in}{3.200153in}}{\pgfqpoint{1.296872in}{3.189554in}}{\pgfqpoint{1.296872in}{3.178504in}}%
\pgfpathcurveto{\pgfqpoint{1.296872in}{3.167454in}}{\pgfqpoint{1.301263in}{3.156855in}}{\pgfqpoint{1.309076in}{3.149041in}}%
\pgfpathcurveto{\pgfqpoint{1.316890in}{3.141228in}}{\pgfqpoint{1.327489in}{3.136837in}}{\pgfqpoint{1.338539in}{3.136837in}}%
\pgfpathclose%
\pgfusepath{stroke,fill}%
\end{pgfscope}%
\begin{pgfscope}%
\pgfpathrectangle{\pgfqpoint{0.648703in}{0.548769in}}{\pgfqpoint{5.201297in}{3.102590in}}%
\pgfusepath{clip}%
\pgfsetbuttcap%
\pgfsetroundjoin%
\definecolor{currentfill}{rgb}{0.121569,0.466667,0.705882}%
\pgfsetfillcolor{currentfill}%
\pgfsetlinewidth{1.003750pt}%
\definecolor{currentstroke}{rgb}{0.121569,0.466667,0.705882}%
\pgfsetstrokecolor{currentstroke}%
\pgfsetdash{}{0pt}%
\pgfpathmoveto{\pgfqpoint{4.123791in}{3.132690in}}%
\pgfpathcurveto{\pgfqpoint{4.134841in}{3.132690in}}{\pgfqpoint{4.145441in}{3.137080in}}{\pgfqpoint{4.153254in}{3.144893in}}%
\pgfpathcurveto{\pgfqpoint{4.161068in}{3.152707in}}{\pgfqpoint{4.165458in}{3.163306in}}{\pgfqpoint{4.165458in}{3.174356in}}%
\pgfpathcurveto{\pgfqpoint{4.165458in}{3.185406in}}{\pgfqpoint{4.161068in}{3.196005in}}{\pgfqpoint{4.153254in}{3.203819in}}%
\pgfpathcurveto{\pgfqpoint{4.145441in}{3.211633in}}{\pgfqpoint{4.134841in}{3.216023in}}{\pgfqpoint{4.123791in}{3.216023in}}%
\pgfpathcurveto{\pgfqpoint{4.112741in}{3.216023in}}{\pgfqpoint{4.102142in}{3.211633in}}{\pgfqpoint{4.094329in}{3.203819in}}%
\pgfpathcurveto{\pgfqpoint{4.086515in}{3.196005in}}{\pgfqpoint{4.082125in}{3.185406in}}{\pgfqpoint{4.082125in}{3.174356in}}%
\pgfpathcurveto{\pgfqpoint{4.082125in}{3.163306in}}{\pgfqpoint{4.086515in}{3.152707in}}{\pgfqpoint{4.094329in}{3.144893in}}%
\pgfpathcurveto{\pgfqpoint{4.102142in}{3.137080in}}{\pgfqpoint{4.112741in}{3.132690in}}{\pgfqpoint{4.123791in}{3.132690in}}%
\pgfpathclose%
\pgfusepath{stroke,fill}%
\end{pgfscope}%
\begin{pgfscope}%
\pgfpathrectangle{\pgfqpoint{0.648703in}{0.548769in}}{\pgfqpoint{5.201297in}{3.102590in}}%
\pgfusepath{clip}%
\pgfsetbuttcap%
\pgfsetroundjoin%
\definecolor{currentfill}{rgb}{0.121569,0.466667,0.705882}%
\pgfsetfillcolor{currentfill}%
\pgfsetlinewidth{1.003750pt}%
\definecolor{currentstroke}{rgb}{0.121569,0.466667,0.705882}%
\pgfsetstrokecolor{currentstroke}%
\pgfsetdash{}{0pt}%
\pgfpathmoveto{\pgfqpoint{1.856726in}{3.128542in}}%
\pgfpathcurveto{\pgfqpoint{1.867776in}{3.128542in}}{\pgfqpoint{1.878375in}{3.132932in}}{\pgfqpoint{1.886188in}{3.140746in}}%
\pgfpathcurveto{\pgfqpoint{1.894002in}{3.148559in}}{\pgfqpoint{1.898392in}{3.159158in}}{\pgfqpoint{1.898392in}{3.170208in}}%
\pgfpathcurveto{\pgfqpoint{1.898392in}{3.181258in}}{\pgfqpoint{1.894002in}{3.191857in}}{\pgfqpoint{1.886188in}{3.199671in}}%
\pgfpathcurveto{\pgfqpoint{1.878375in}{3.207485in}}{\pgfqpoint{1.867776in}{3.211875in}}{\pgfqpoint{1.856726in}{3.211875in}}%
\pgfpathcurveto{\pgfqpoint{1.845675in}{3.211875in}}{\pgfqpoint{1.835076in}{3.207485in}}{\pgfqpoint{1.827263in}{3.199671in}}%
\pgfpathcurveto{\pgfqpoint{1.819449in}{3.191857in}}{\pgfqpoint{1.815059in}{3.181258in}}{\pgfqpoint{1.815059in}{3.170208in}}%
\pgfpathcurveto{\pgfqpoint{1.815059in}{3.159158in}}{\pgfqpoint{1.819449in}{3.148559in}}{\pgfqpoint{1.827263in}{3.140746in}}%
\pgfpathcurveto{\pgfqpoint{1.835076in}{3.132932in}}{\pgfqpoint{1.845675in}{3.128542in}}{\pgfqpoint{1.856726in}{3.128542in}}%
\pgfpathclose%
\pgfusepath{stroke,fill}%
\end{pgfscope}%
\begin{pgfscope}%
\pgfpathrectangle{\pgfqpoint{0.648703in}{0.548769in}}{\pgfqpoint{5.201297in}{3.102590in}}%
\pgfusepath{clip}%
\pgfsetbuttcap%
\pgfsetroundjoin%
\definecolor{currentfill}{rgb}{1.000000,0.498039,0.054902}%
\pgfsetfillcolor{currentfill}%
\pgfsetlinewidth{1.003750pt}%
\definecolor{currentstroke}{rgb}{1.000000,0.498039,0.054902}%
\pgfsetstrokecolor{currentstroke}%
\pgfsetdash{}{0pt}%
\pgfpathmoveto{\pgfqpoint{1.597632in}{3.149281in}}%
\pgfpathcurveto{\pgfqpoint{1.608682in}{3.149281in}}{\pgfqpoint{1.619282in}{3.153671in}}{\pgfqpoint{1.627095in}{3.161485in}}%
\pgfpathcurveto{\pgfqpoint{1.634909in}{3.169298in}}{\pgfqpoint{1.639299in}{3.179897in}}{\pgfqpoint{1.639299in}{3.190948in}}%
\pgfpathcurveto{\pgfqpoint{1.639299in}{3.201998in}}{\pgfqpoint{1.634909in}{3.212597in}}{\pgfqpoint{1.627095in}{3.220410in}}%
\pgfpathcurveto{\pgfqpoint{1.619282in}{3.228224in}}{\pgfqpoint{1.608682in}{3.232614in}}{\pgfqpoint{1.597632in}{3.232614in}}%
\pgfpathcurveto{\pgfqpoint{1.586582in}{3.232614in}}{\pgfqpoint{1.575983in}{3.228224in}}{\pgfqpoint{1.568170in}{3.220410in}}%
\pgfpathcurveto{\pgfqpoint{1.560356in}{3.212597in}}{\pgfqpoint{1.555966in}{3.201998in}}{\pgfqpoint{1.555966in}{3.190948in}}%
\pgfpathcurveto{\pgfqpoint{1.555966in}{3.179897in}}{\pgfqpoint{1.560356in}{3.169298in}}{\pgfqpoint{1.568170in}{3.161485in}}%
\pgfpathcurveto{\pgfqpoint{1.575983in}{3.153671in}}{\pgfqpoint{1.586582in}{3.149281in}}{\pgfqpoint{1.597632in}{3.149281in}}%
\pgfpathclose%
\pgfusepath{stroke,fill}%
\end{pgfscope}%
\begin{pgfscope}%
\pgfpathrectangle{\pgfqpoint{0.648703in}{0.548769in}}{\pgfqpoint{5.201297in}{3.102590in}}%
\pgfusepath{clip}%
\pgfsetbuttcap%
\pgfsetroundjoin%
\definecolor{currentfill}{rgb}{1.000000,0.498039,0.054902}%
\pgfsetfillcolor{currentfill}%
\pgfsetlinewidth{1.003750pt}%
\definecolor{currentstroke}{rgb}{1.000000,0.498039,0.054902}%
\pgfsetstrokecolor{currentstroke}%
\pgfsetdash{}{0pt}%
\pgfpathmoveto{\pgfqpoint{1.273766in}{3.240534in}}%
\pgfpathcurveto{\pgfqpoint{1.284816in}{3.240534in}}{\pgfqpoint{1.295415in}{3.244924in}}{\pgfqpoint{1.303229in}{3.252737in}}%
\pgfpathcurveto{\pgfqpoint{1.311042in}{3.260551in}}{\pgfqpoint{1.315432in}{3.271150in}}{\pgfqpoint{1.315432in}{3.282200in}}%
\pgfpathcurveto{\pgfqpoint{1.315432in}{3.293250in}}{\pgfqpoint{1.311042in}{3.303849in}}{\pgfqpoint{1.303229in}{3.311663in}}%
\pgfpathcurveto{\pgfqpoint{1.295415in}{3.319477in}}{\pgfqpoint{1.284816in}{3.323867in}}{\pgfqpoint{1.273766in}{3.323867in}}%
\pgfpathcurveto{\pgfqpoint{1.262716in}{3.323867in}}{\pgfqpoint{1.252117in}{3.319477in}}{\pgfqpoint{1.244303in}{3.311663in}}%
\pgfpathcurveto{\pgfqpoint{1.236489in}{3.303849in}}{\pgfqpoint{1.232099in}{3.293250in}}{\pgfqpoint{1.232099in}{3.282200in}}%
\pgfpathcurveto{\pgfqpoint{1.232099in}{3.271150in}}{\pgfqpoint{1.236489in}{3.260551in}}{\pgfqpoint{1.244303in}{3.252737in}}%
\pgfpathcurveto{\pgfqpoint{1.252117in}{3.244924in}}{\pgfqpoint{1.262716in}{3.240534in}}{\pgfqpoint{1.273766in}{3.240534in}}%
\pgfpathclose%
\pgfusepath{stroke,fill}%
\end{pgfscope}%
\begin{pgfscope}%
\pgfpathrectangle{\pgfqpoint{0.648703in}{0.548769in}}{\pgfqpoint{5.201297in}{3.102590in}}%
\pgfusepath{clip}%
\pgfsetbuttcap%
\pgfsetroundjoin%
\definecolor{currentfill}{rgb}{0.121569,0.466667,0.705882}%
\pgfsetfillcolor{currentfill}%
\pgfsetlinewidth{1.003750pt}%
\definecolor{currentstroke}{rgb}{0.121569,0.466667,0.705882}%
\pgfsetstrokecolor{currentstroke}%
\pgfsetdash{}{0pt}%
\pgfpathmoveto{\pgfqpoint{1.273766in}{0.664720in}}%
\pgfpathcurveto{\pgfqpoint{1.284816in}{0.664720in}}{\pgfqpoint{1.295415in}{0.669111in}}{\pgfqpoint{1.303229in}{0.676924in}}%
\pgfpathcurveto{\pgfqpoint{1.311042in}{0.684738in}}{\pgfqpoint{1.315432in}{0.695337in}}{\pgfqpoint{1.315432in}{0.706387in}}%
\pgfpathcurveto{\pgfqpoint{1.315432in}{0.717437in}}{\pgfqpoint{1.311042in}{0.728036in}}{\pgfqpoint{1.303229in}{0.735850in}}%
\pgfpathcurveto{\pgfqpoint{1.295415in}{0.743663in}}{\pgfqpoint{1.284816in}{0.748054in}}{\pgfqpoint{1.273766in}{0.748054in}}%
\pgfpathcurveto{\pgfqpoint{1.262716in}{0.748054in}}{\pgfqpoint{1.252117in}{0.743663in}}{\pgfqpoint{1.244303in}{0.735850in}}%
\pgfpathcurveto{\pgfqpoint{1.236489in}{0.728036in}}{\pgfqpoint{1.232099in}{0.717437in}}{\pgfqpoint{1.232099in}{0.706387in}}%
\pgfpathcurveto{\pgfqpoint{1.232099in}{0.695337in}}{\pgfqpoint{1.236489in}{0.684738in}}{\pgfqpoint{1.244303in}{0.676924in}}%
\pgfpathcurveto{\pgfqpoint{1.252117in}{0.669111in}}{\pgfqpoint{1.262716in}{0.664720in}}{\pgfqpoint{1.273766in}{0.664720in}}%
\pgfpathclose%
\pgfusepath{stroke,fill}%
\end{pgfscope}%
\begin{pgfscope}%
\pgfpathrectangle{\pgfqpoint{0.648703in}{0.548769in}}{\pgfqpoint{5.201297in}{3.102590in}}%
\pgfusepath{clip}%
\pgfsetbuttcap%
\pgfsetroundjoin%
\definecolor{currentfill}{rgb}{1.000000,0.498039,0.054902}%
\pgfsetfillcolor{currentfill}%
\pgfsetlinewidth{1.003750pt}%
\definecolor{currentstroke}{rgb}{1.000000,0.498039,0.054902}%
\pgfsetstrokecolor{currentstroke}%
\pgfsetdash{}{0pt}%
\pgfpathmoveto{\pgfqpoint{0.949899in}{3.157577in}}%
\pgfpathcurveto{\pgfqpoint{0.960949in}{3.157577in}}{\pgfqpoint{0.971548in}{3.161967in}}{\pgfqpoint{0.979362in}{3.169780in}}%
\pgfpathcurveto{\pgfqpoint{0.987176in}{3.177594in}}{\pgfqpoint{0.991566in}{3.188193in}}{\pgfqpoint{0.991566in}{3.199243in}}%
\pgfpathcurveto{\pgfqpoint{0.991566in}{3.210293in}}{\pgfqpoint{0.987176in}{3.220892in}}{\pgfqpoint{0.979362in}{3.228706in}}%
\pgfpathcurveto{\pgfqpoint{0.971548in}{3.236520in}}{\pgfqpoint{0.960949in}{3.240910in}}{\pgfqpoint{0.949899in}{3.240910in}}%
\pgfpathcurveto{\pgfqpoint{0.938849in}{3.240910in}}{\pgfqpoint{0.928250in}{3.236520in}}{\pgfqpoint{0.920437in}{3.228706in}}%
\pgfpathcurveto{\pgfqpoint{0.912623in}{3.220892in}}{\pgfqpoint{0.908233in}{3.210293in}}{\pgfqpoint{0.908233in}{3.199243in}}%
\pgfpathcurveto{\pgfqpoint{0.908233in}{3.188193in}}{\pgfqpoint{0.912623in}{3.177594in}}{\pgfqpoint{0.920437in}{3.169780in}}%
\pgfpathcurveto{\pgfqpoint{0.928250in}{3.161967in}}{\pgfqpoint{0.938849in}{3.157577in}}{\pgfqpoint{0.949899in}{3.157577in}}%
\pgfpathclose%
\pgfusepath{stroke,fill}%
\end{pgfscope}%
\begin{pgfscope}%
\pgfpathrectangle{\pgfqpoint{0.648703in}{0.548769in}}{\pgfqpoint{5.201297in}{3.102590in}}%
\pgfusepath{clip}%
\pgfsetbuttcap%
\pgfsetroundjoin%
\definecolor{currentfill}{rgb}{1.000000,0.498039,0.054902}%
\pgfsetfillcolor{currentfill}%
\pgfsetlinewidth{1.003750pt}%
\definecolor{currentstroke}{rgb}{1.000000,0.498039,0.054902}%
\pgfsetstrokecolor{currentstroke}%
\pgfsetdash{}{0pt}%
\pgfpathmoveto{\pgfqpoint{1.403312in}{3.140985in}}%
\pgfpathcurveto{\pgfqpoint{1.414363in}{3.140985in}}{\pgfqpoint{1.424962in}{3.145375in}}{\pgfqpoint{1.432775in}{3.153189in}}%
\pgfpathcurveto{\pgfqpoint{1.440589in}{3.161003in}}{\pgfqpoint{1.444979in}{3.171602in}}{\pgfqpoint{1.444979in}{3.182652in}}%
\pgfpathcurveto{\pgfqpoint{1.444979in}{3.193702in}}{\pgfqpoint{1.440589in}{3.204301in}}{\pgfqpoint{1.432775in}{3.212115in}}%
\pgfpathcurveto{\pgfqpoint{1.424962in}{3.219928in}}{\pgfqpoint{1.414363in}{3.224319in}}{\pgfqpoint{1.403312in}{3.224319in}}%
\pgfpathcurveto{\pgfqpoint{1.392262in}{3.224319in}}{\pgfqpoint{1.381663in}{3.219928in}}{\pgfqpoint{1.373850in}{3.212115in}}%
\pgfpathcurveto{\pgfqpoint{1.366036in}{3.204301in}}{\pgfqpoint{1.361646in}{3.193702in}}{\pgfqpoint{1.361646in}{3.182652in}}%
\pgfpathcurveto{\pgfqpoint{1.361646in}{3.171602in}}{\pgfqpoint{1.366036in}{3.161003in}}{\pgfqpoint{1.373850in}{3.153189in}}%
\pgfpathcurveto{\pgfqpoint{1.381663in}{3.145375in}}{\pgfqpoint{1.392262in}{3.140985in}}{\pgfqpoint{1.403312in}{3.140985in}}%
\pgfpathclose%
\pgfusepath{stroke,fill}%
\end{pgfscope}%
\begin{pgfscope}%
\pgfpathrectangle{\pgfqpoint{0.648703in}{0.548769in}}{\pgfqpoint{5.201297in}{3.102590in}}%
\pgfusepath{clip}%
\pgfsetbuttcap%
\pgfsetroundjoin%
\definecolor{currentfill}{rgb}{1.000000,0.498039,0.054902}%
\pgfsetfillcolor{currentfill}%
\pgfsetlinewidth{1.003750pt}%
\definecolor{currentstroke}{rgb}{1.000000,0.498039,0.054902}%
\pgfsetstrokecolor{currentstroke}%
\pgfsetdash{}{0pt}%
\pgfpathmoveto{\pgfqpoint{1.208993in}{3.136837in}}%
\pgfpathcurveto{\pgfqpoint{1.220043in}{3.136837in}}{\pgfqpoint{1.230642in}{3.141228in}}{\pgfqpoint{1.238455in}{3.149041in}}%
\pgfpathcurveto{\pgfqpoint{1.246269in}{3.156855in}}{\pgfqpoint{1.250659in}{3.167454in}}{\pgfqpoint{1.250659in}{3.178504in}}%
\pgfpathcurveto{\pgfqpoint{1.250659in}{3.189554in}}{\pgfqpoint{1.246269in}{3.200153in}}{\pgfqpoint{1.238455in}{3.207967in}}%
\pgfpathcurveto{\pgfqpoint{1.230642in}{3.215780in}}{\pgfqpoint{1.220043in}{3.220171in}}{\pgfqpoint{1.208993in}{3.220171in}}%
\pgfpathcurveto{\pgfqpoint{1.197942in}{3.220171in}}{\pgfqpoint{1.187343in}{3.215780in}}{\pgfqpoint{1.179530in}{3.207967in}}%
\pgfpathcurveto{\pgfqpoint{1.171716in}{3.200153in}}{\pgfqpoint{1.167326in}{3.189554in}}{\pgfqpoint{1.167326in}{3.178504in}}%
\pgfpathcurveto{\pgfqpoint{1.167326in}{3.167454in}}{\pgfqpoint{1.171716in}{3.156855in}}{\pgfqpoint{1.179530in}{3.149041in}}%
\pgfpathcurveto{\pgfqpoint{1.187343in}{3.141228in}}{\pgfqpoint{1.197942in}{3.136837in}}{\pgfqpoint{1.208993in}{3.136837in}}%
\pgfpathclose%
\pgfusepath{stroke,fill}%
\end{pgfscope}%
\begin{pgfscope}%
\pgfpathrectangle{\pgfqpoint{0.648703in}{0.548769in}}{\pgfqpoint{5.201297in}{3.102590in}}%
\pgfusepath{clip}%
\pgfsetbuttcap%
\pgfsetroundjoin%
\definecolor{currentfill}{rgb}{1.000000,0.498039,0.054902}%
\pgfsetfillcolor{currentfill}%
\pgfsetlinewidth{1.003750pt}%
\definecolor{currentstroke}{rgb}{1.000000,0.498039,0.054902}%
\pgfsetstrokecolor{currentstroke}%
\pgfsetdash{}{0pt}%
\pgfpathmoveto{\pgfqpoint{1.403312in}{3.136837in}}%
\pgfpathcurveto{\pgfqpoint{1.414363in}{3.136837in}}{\pgfqpoint{1.424962in}{3.141228in}}{\pgfqpoint{1.432775in}{3.149041in}}%
\pgfpathcurveto{\pgfqpoint{1.440589in}{3.156855in}}{\pgfqpoint{1.444979in}{3.167454in}}{\pgfqpoint{1.444979in}{3.178504in}}%
\pgfpathcurveto{\pgfqpoint{1.444979in}{3.189554in}}{\pgfqpoint{1.440589in}{3.200153in}}{\pgfqpoint{1.432775in}{3.207967in}}%
\pgfpathcurveto{\pgfqpoint{1.424962in}{3.215780in}}{\pgfqpoint{1.414363in}{3.220171in}}{\pgfqpoint{1.403312in}{3.220171in}}%
\pgfpathcurveto{\pgfqpoint{1.392262in}{3.220171in}}{\pgfqpoint{1.381663in}{3.215780in}}{\pgfqpoint{1.373850in}{3.207967in}}%
\pgfpathcurveto{\pgfqpoint{1.366036in}{3.200153in}}{\pgfqpoint{1.361646in}{3.189554in}}{\pgfqpoint{1.361646in}{3.178504in}}%
\pgfpathcurveto{\pgfqpoint{1.361646in}{3.167454in}}{\pgfqpoint{1.366036in}{3.156855in}}{\pgfqpoint{1.373850in}{3.149041in}}%
\pgfpathcurveto{\pgfqpoint{1.381663in}{3.141228in}}{\pgfqpoint{1.392262in}{3.136837in}}{\pgfqpoint{1.403312in}{3.136837in}}%
\pgfpathclose%
\pgfusepath{stroke,fill}%
\end{pgfscope}%
\begin{pgfscope}%
\pgfpathrectangle{\pgfqpoint{0.648703in}{0.548769in}}{\pgfqpoint{5.201297in}{3.102590in}}%
\pgfusepath{clip}%
\pgfsetbuttcap%
\pgfsetroundjoin%
\definecolor{currentfill}{rgb}{1.000000,0.498039,0.054902}%
\pgfsetfillcolor{currentfill}%
\pgfsetlinewidth{1.003750pt}%
\definecolor{currentstroke}{rgb}{1.000000,0.498039,0.054902}%
\pgfsetstrokecolor{currentstroke}%
\pgfsetdash{}{0pt}%
\pgfpathmoveto{\pgfqpoint{2.374912in}{3.136837in}}%
\pgfpathcurveto{\pgfqpoint{2.385962in}{3.136837in}}{\pgfqpoint{2.396561in}{3.141228in}}{\pgfqpoint{2.404375in}{3.149041in}}%
\pgfpathcurveto{\pgfqpoint{2.412188in}{3.156855in}}{\pgfqpoint{2.416579in}{3.167454in}}{\pgfqpoint{2.416579in}{3.178504in}}%
\pgfpathcurveto{\pgfqpoint{2.416579in}{3.189554in}}{\pgfqpoint{2.412188in}{3.200153in}}{\pgfqpoint{2.404375in}{3.207967in}}%
\pgfpathcurveto{\pgfqpoint{2.396561in}{3.215780in}}{\pgfqpoint{2.385962in}{3.220171in}}{\pgfqpoint{2.374912in}{3.220171in}}%
\pgfpathcurveto{\pgfqpoint{2.363862in}{3.220171in}}{\pgfqpoint{2.353263in}{3.215780in}}{\pgfqpoint{2.345449in}{3.207967in}}%
\pgfpathcurveto{\pgfqpoint{2.337636in}{3.200153in}}{\pgfqpoint{2.333245in}{3.189554in}}{\pgfqpoint{2.333245in}{3.178504in}}%
\pgfpathcurveto{\pgfqpoint{2.333245in}{3.167454in}}{\pgfqpoint{2.337636in}{3.156855in}}{\pgfqpoint{2.345449in}{3.149041in}}%
\pgfpathcurveto{\pgfqpoint{2.353263in}{3.141228in}}{\pgfqpoint{2.363862in}{3.136837in}}{\pgfqpoint{2.374912in}{3.136837in}}%
\pgfpathclose%
\pgfusepath{stroke,fill}%
\end{pgfscope}%
\begin{pgfscope}%
\pgfpathrectangle{\pgfqpoint{0.648703in}{0.548769in}}{\pgfqpoint{5.201297in}{3.102590in}}%
\pgfusepath{clip}%
\pgfsetbuttcap%
\pgfsetroundjoin%
\definecolor{currentfill}{rgb}{1.000000,0.498039,0.054902}%
\pgfsetfillcolor{currentfill}%
\pgfsetlinewidth{1.003750pt}%
\definecolor{currentstroke}{rgb}{1.000000,0.498039,0.054902}%
\pgfsetstrokecolor{currentstroke}%
\pgfsetdash{}{0pt}%
\pgfpathmoveto{\pgfqpoint{1.468086in}{3.136837in}}%
\pgfpathcurveto{\pgfqpoint{1.479136in}{3.136837in}}{\pgfqpoint{1.489735in}{3.141228in}}{\pgfqpoint{1.497549in}{3.149041in}}%
\pgfpathcurveto{\pgfqpoint{1.505362in}{3.156855in}}{\pgfqpoint{1.509752in}{3.167454in}}{\pgfqpoint{1.509752in}{3.178504in}}%
\pgfpathcurveto{\pgfqpoint{1.509752in}{3.189554in}}{\pgfqpoint{1.505362in}{3.200153in}}{\pgfqpoint{1.497549in}{3.207967in}}%
\pgfpathcurveto{\pgfqpoint{1.489735in}{3.215780in}}{\pgfqpoint{1.479136in}{3.220171in}}{\pgfqpoint{1.468086in}{3.220171in}}%
\pgfpathcurveto{\pgfqpoint{1.457036in}{3.220171in}}{\pgfqpoint{1.446437in}{3.215780in}}{\pgfqpoint{1.438623in}{3.207967in}}%
\pgfpathcurveto{\pgfqpoint{1.430809in}{3.200153in}}{\pgfqpoint{1.426419in}{3.189554in}}{\pgfqpoint{1.426419in}{3.178504in}}%
\pgfpathcurveto{\pgfqpoint{1.426419in}{3.167454in}}{\pgfqpoint{1.430809in}{3.156855in}}{\pgfqpoint{1.438623in}{3.149041in}}%
\pgfpathcurveto{\pgfqpoint{1.446437in}{3.141228in}}{\pgfqpoint{1.457036in}{3.136837in}}{\pgfqpoint{1.468086in}{3.136837in}}%
\pgfpathclose%
\pgfusepath{stroke,fill}%
\end{pgfscope}%
\begin{pgfscope}%
\pgfpathrectangle{\pgfqpoint{0.648703in}{0.548769in}}{\pgfqpoint{5.201297in}{3.102590in}}%
\pgfusepath{clip}%
\pgfsetbuttcap%
\pgfsetroundjoin%
\definecolor{currentfill}{rgb}{1.000000,0.498039,0.054902}%
\pgfsetfillcolor{currentfill}%
\pgfsetlinewidth{1.003750pt}%
\definecolor{currentstroke}{rgb}{1.000000,0.498039,0.054902}%
\pgfsetstrokecolor{currentstroke}%
\pgfsetdash{}{0pt}%
\pgfpathmoveto{\pgfqpoint{1.468086in}{3.136837in}}%
\pgfpathcurveto{\pgfqpoint{1.479136in}{3.136837in}}{\pgfqpoint{1.489735in}{3.141228in}}{\pgfqpoint{1.497549in}{3.149041in}}%
\pgfpathcurveto{\pgfqpoint{1.505362in}{3.156855in}}{\pgfqpoint{1.509752in}{3.167454in}}{\pgfqpoint{1.509752in}{3.178504in}}%
\pgfpathcurveto{\pgfqpoint{1.509752in}{3.189554in}}{\pgfqpoint{1.505362in}{3.200153in}}{\pgfqpoint{1.497549in}{3.207967in}}%
\pgfpathcurveto{\pgfqpoint{1.489735in}{3.215780in}}{\pgfqpoint{1.479136in}{3.220171in}}{\pgfqpoint{1.468086in}{3.220171in}}%
\pgfpathcurveto{\pgfqpoint{1.457036in}{3.220171in}}{\pgfqpoint{1.446437in}{3.215780in}}{\pgfqpoint{1.438623in}{3.207967in}}%
\pgfpathcurveto{\pgfqpoint{1.430809in}{3.200153in}}{\pgfqpoint{1.426419in}{3.189554in}}{\pgfqpoint{1.426419in}{3.178504in}}%
\pgfpathcurveto{\pgfqpoint{1.426419in}{3.167454in}}{\pgfqpoint{1.430809in}{3.156855in}}{\pgfqpoint{1.438623in}{3.149041in}}%
\pgfpathcurveto{\pgfqpoint{1.446437in}{3.141228in}}{\pgfqpoint{1.457036in}{3.136837in}}{\pgfqpoint{1.468086in}{3.136837in}}%
\pgfpathclose%
\pgfusepath{stroke,fill}%
\end{pgfscope}%
\begin{pgfscope}%
\pgfpathrectangle{\pgfqpoint{0.648703in}{0.548769in}}{\pgfqpoint{5.201297in}{3.102590in}}%
\pgfusepath{clip}%
\pgfsetbuttcap%
\pgfsetroundjoin%
\definecolor{currentfill}{rgb}{0.121569,0.466667,0.705882}%
\pgfsetfillcolor{currentfill}%
\pgfsetlinewidth{1.003750pt}%
\definecolor{currentstroke}{rgb}{0.121569,0.466667,0.705882}%
\pgfsetstrokecolor{currentstroke}%
\pgfsetdash{}{0pt}%
\pgfpathmoveto{\pgfqpoint{1.532859in}{2.846488in}}%
\pgfpathcurveto{\pgfqpoint{1.543909in}{2.846488in}}{\pgfqpoint{1.554508in}{2.850878in}}{\pgfqpoint{1.562322in}{2.858692in}}%
\pgfpathcurveto{\pgfqpoint{1.570135in}{2.866506in}}{\pgfqpoint{1.574526in}{2.877105in}}{\pgfqpoint{1.574526in}{2.888155in}}%
\pgfpathcurveto{\pgfqpoint{1.574526in}{2.899205in}}{\pgfqpoint{1.570135in}{2.909804in}}{\pgfqpoint{1.562322in}{2.917617in}}%
\pgfpathcurveto{\pgfqpoint{1.554508in}{2.925431in}}{\pgfqpoint{1.543909in}{2.929821in}}{\pgfqpoint{1.532859in}{2.929821in}}%
\pgfpathcurveto{\pgfqpoint{1.521809in}{2.929821in}}{\pgfqpoint{1.511210in}{2.925431in}}{\pgfqpoint{1.503396in}{2.917617in}}%
\pgfpathcurveto{\pgfqpoint{1.495583in}{2.909804in}}{\pgfqpoint{1.491192in}{2.899205in}}{\pgfqpoint{1.491192in}{2.888155in}}%
\pgfpathcurveto{\pgfqpoint{1.491192in}{2.877105in}}{\pgfqpoint{1.495583in}{2.866506in}}{\pgfqpoint{1.503396in}{2.858692in}}%
\pgfpathcurveto{\pgfqpoint{1.511210in}{2.850878in}}{\pgfqpoint{1.521809in}{2.846488in}}{\pgfqpoint{1.532859in}{2.846488in}}%
\pgfpathclose%
\pgfusepath{stroke,fill}%
\end{pgfscope}%
\begin{pgfscope}%
\pgfpathrectangle{\pgfqpoint{0.648703in}{0.548769in}}{\pgfqpoint{5.201297in}{3.102590in}}%
\pgfusepath{clip}%
\pgfsetbuttcap%
\pgfsetroundjoin%
\definecolor{currentfill}{rgb}{0.121569,0.466667,0.705882}%
\pgfsetfillcolor{currentfill}%
\pgfsetlinewidth{1.003750pt}%
\definecolor{currentstroke}{rgb}{0.121569,0.466667,0.705882}%
\pgfsetstrokecolor{currentstroke}%
\pgfsetdash{}{0pt}%
\pgfpathmoveto{\pgfqpoint{1.208993in}{3.074620in}}%
\pgfpathcurveto{\pgfqpoint{1.220043in}{3.074620in}}{\pgfqpoint{1.230642in}{3.079010in}}{\pgfqpoint{1.238455in}{3.086824in}}%
\pgfpathcurveto{\pgfqpoint{1.246269in}{3.094637in}}{\pgfqpoint{1.250659in}{3.105236in}}{\pgfqpoint{1.250659in}{3.116286in}}%
\pgfpathcurveto{\pgfqpoint{1.250659in}{3.127336in}}{\pgfqpoint{1.246269in}{3.137935in}}{\pgfqpoint{1.238455in}{3.145749in}}%
\pgfpathcurveto{\pgfqpoint{1.230642in}{3.153563in}}{\pgfqpoint{1.220043in}{3.157953in}}{\pgfqpoint{1.208993in}{3.157953in}}%
\pgfpathcurveto{\pgfqpoint{1.197942in}{3.157953in}}{\pgfqpoint{1.187343in}{3.153563in}}{\pgfqpoint{1.179530in}{3.145749in}}%
\pgfpathcurveto{\pgfqpoint{1.171716in}{3.137935in}}{\pgfqpoint{1.167326in}{3.127336in}}{\pgfqpoint{1.167326in}{3.116286in}}%
\pgfpathcurveto{\pgfqpoint{1.167326in}{3.105236in}}{\pgfqpoint{1.171716in}{3.094637in}}{\pgfqpoint{1.179530in}{3.086824in}}%
\pgfpathcurveto{\pgfqpoint{1.187343in}{3.079010in}}{\pgfqpoint{1.197942in}{3.074620in}}{\pgfqpoint{1.208993in}{3.074620in}}%
\pgfpathclose%
\pgfusepath{stroke,fill}%
\end{pgfscope}%
\begin{pgfscope}%
\pgfpathrectangle{\pgfqpoint{0.648703in}{0.548769in}}{\pgfqpoint{5.201297in}{3.102590in}}%
\pgfusepath{clip}%
\pgfsetbuttcap%
\pgfsetroundjoin%
\definecolor{currentfill}{rgb}{1.000000,0.498039,0.054902}%
\pgfsetfillcolor{currentfill}%
\pgfsetlinewidth{1.003750pt}%
\definecolor{currentstroke}{rgb}{1.000000,0.498039,0.054902}%
\pgfsetstrokecolor{currentstroke}%
\pgfsetdash{}{0pt}%
\pgfpathmoveto{\pgfqpoint{2.374912in}{3.315195in}}%
\pgfpathcurveto{\pgfqpoint{2.385962in}{3.315195in}}{\pgfqpoint{2.396561in}{3.319585in}}{\pgfqpoint{2.404375in}{3.327399in}}%
\pgfpathcurveto{\pgfqpoint{2.412188in}{3.335212in}}{\pgfqpoint{2.416579in}{3.345811in}}{\pgfqpoint{2.416579in}{3.356861in}}%
\pgfpathcurveto{\pgfqpoint{2.416579in}{3.367912in}}{\pgfqpoint{2.412188in}{3.378511in}}{\pgfqpoint{2.404375in}{3.386324in}}%
\pgfpathcurveto{\pgfqpoint{2.396561in}{3.394138in}}{\pgfqpoint{2.385962in}{3.398528in}}{\pgfqpoint{2.374912in}{3.398528in}}%
\pgfpathcurveto{\pgfqpoint{2.363862in}{3.398528in}}{\pgfqpoint{2.353263in}{3.394138in}}{\pgfqpoint{2.345449in}{3.386324in}}%
\pgfpathcurveto{\pgfqpoint{2.337636in}{3.378511in}}{\pgfqpoint{2.333245in}{3.367912in}}{\pgfqpoint{2.333245in}{3.356861in}}%
\pgfpathcurveto{\pgfqpoint{2.333245in}{3.345811in}}{\pgfqpoint{2.337636in}{3.335212in}}{\pgfqpoint{2.345449in}{3.327399in}}%
\pgfpathcurveto{\pgfqpoint{2.353263in}{3.319585in}}{\pgfqpoint{2.363862in}{3.315195in}}{\pgfqpoint{2.374912in}{3.315195in}}%
\pgfpathclose%
\pgfusepath{stroke,fill}%
\end{pgfscope}%
\begin{pgfscope}%
\pgfpathrectangle{\pgfqpoint{0.648703in}{0.548769in}}{\pgfqpoint{5.201297in}{3.102590in}}%
\pgfusepath{clip}%
\pgfsetbuttcap%
\pgfsetroundjoin%
\definecolor{currentfill}{rgb}{0.121569,0.466667,0.705882}%
\pgfsetfillcolor{currentfill}%
\pgfsetlinewidth{1.003750pt}%
\definecolor{currentstroke}{rgb}{0.121569,0.466667,0.705882}%
\pgfsetstrokecolor{currentstroke}%
\pgfsetdash{}{0pt}%
\pgfpathmoveto{\pgfqpoint{1.208993in}{0.648129in}}%
\pgfpathcurveto{\pgfqpoint{1.220043in}{0.648129in}}{\pgfqpoint{1.230642in}{0.652519in}}{\pgfqpoint{1.238455in}{0.660333in}}%
\pgfpathcurveto{\pgfqpoint{1.246269in}{0.668146in}}{\pgfqpoint{1.250659in}{0.678745in}}{\pgfqpoint{1.250659in}{0.689796in}}%
\pgfpathcurveto{\pgfqpoint{1.250659in}{0.700846in}}{\pgfqpoint{1.246269in}{0.711445in}}{\pgfqpoint{1.238455in}{0.719258in}}%
\pgfpathcurveto{\pgfqpoint{1.230642in}{0.727072in}}{\pgfqpoint{1.220043in}{0.731462in}}{\pgfqpoint{1.208993in}{0.731462in}}%
\pgfpathcurveto{\pgfqpoint{1.197942in}{0.731462in}}{\pgfqpoint{1.187343in}{0.727072in}}{\pgfqpoint{1.179530in}{0.719258in}}%
\pgfpathcurveto{\pgfqpoint{1.171716in}{0.711445in}}{\pgfqpoint{1.167326in}{0.700846in}}{\pgfqpoint{1.167326in}{0.689796in}}%
\pgfpathcurveto{\pgfqpoint{1.167326in}{0.678745in}}{\pgfqpoint{1.171716in}{0.668146in}}{\pgfqpoint{1.179530in}{0.660333in}}%
\pgfpathcurveto{\pgfqpoint{1.187343in}{0.652519in}}{\pgfqpoint{1.197942in}{0.648129in}}{\pgfqpoint{1.208993in}{0.648129in}}%
\pgfpathclose%
\pgfusepath{stroke,fill}%
\end{pgfscope}%
\begin{pgfscope}%
\pgfpathrectangle{\pgfqpoint{0.648703in}{0.548769in}}{\pgfqpoint{5.201297in}{3.102590in}}%
\pgfusepath{clip}%
\pgfsetbuttcap%
\pgfsetroundjoin%
\definecolor{currentfill}{rgb}{0.121569,0.466667,0.705882}%
\pgfsetfillcolor{currentfill}%
\pgfsetlinewidth{1.003750pt}%
\definecolor{currentstroke}{rgb}{0.121569,0.466667,0.705882}%
\pgfsetstrokecolor{currentstroke}%
\pgfsetdash{}{0pt}%
\pgfpathmoveto{\pgfqpoint{0.885126in}{1.166610in}}%
\pgfpathcurveto{\pgfqpoint{0.896176in}{1.166610in}}{\pgfqpoint{0.906775in}{1.171000in}}{\pgfqpoint{0.914589in}{1.178814in}}%
\pgfpathcurveto{\pgfqpoint{0.922402in}{1.186627in}}{\pgfqpoint{0.926793in}{1.197226in}}{\pgfqpoint{0.926793in}{1.208277in}}%
\pgfpathcurveto{\pgfqpoint{0.926793in}{1.219327in}}{\pgfqpoint{0.922402in}{1.229926in}}{\pgfqpoint{0.914589in}{1.237739in}}%
\pgfpathcurveto{\pgfqpoint{0.906775in}{1.245553in}}{\pgfqpoint{0.896176in}{1.249943in}}{\pgfqpoint{0.885126in}{1.249943in}}%
\pgfpathcurveto{\pgfqpoint{0.874076in}{1.249943in}}{\pgfqpoint{0.863477in}{1.245553in}}{\pgfqpoint{0.855663in}{1.237739in}}%
\pgfpathcurveto{\pgfqpoint{0.847850in}{1.229926in}}{\pgfqpoint{0.843459in}{1.219327in}}{\pgfqpoint{0.843459in}{1.208277in}}%
\pgfpathcurveto{\pgfqpoint{0.843459in}{1.197226in}}{\pgfqpoint{0.847850in}{1.186627in}}{\pgfqpoint{0.855663in}{1.178814in}}%
\pgfpathcurveto{\pgfqpoint{0.863477in}{1.171000in}}{\pgfqpoint{0.874076in}{1.166610in}}{\pgfqpoint{0.885126in}{1.166610in}}%
\pgfpathclose%
\pgfusepath{stroke,fill}%
\end{pgfscope}%
\begin{pgfscope}%
\pgfpathrectangle{\pgfqpoint{0.648703in}{0.548769in}}{\pgfqpoint{5.201297in}{3.102590in}}%
\pgfusepath{clip}%
\pgfsetbuttcap%
\pgfsetroundjoin%
\definecolor{currentfill}{rgb}{1.000000,0.498039,0.054902}%
\pgfsetfillcolor{currentfill}%
\pgfsetlinewidth{1.003750pt}%
\definecolor{currentstroke}{rgb}{1.000000,0.498039,0.054902}%
\pgfsetstrokecolor{currentstroke}%
\pgfsetdash{}{0pt}%
\pgfpathmoveto{\pgfqpoint{1.532859in}{3.136837in}}%
\pgfpathcurveto{\pgfqpoint{1.543909in}{3.136837in}}{\pgfqpoint{1.554508in}{3.141228in}}{\pgfqpoint{1.562322in}{3.149041in}}%
\pgfpathcurveto{\pgfqpoint{1.570135in}{3.156855in}}{\pgfqpoint{1.574526in}{3.167454in}}{\pgfqpoint{1.574526in}{3.178504in}}%
\pgfpathcurveto{\pgfqpoint{1.574526in}{3.189554in}}{\pgfqpoint{1.570135in}{3.200153in}}{\pgfqpoint{1.562322in}{3.207967in}}%
\pgfpathcurveto{\pgfqpoint{1.554508in}{3.215780in}}{\pgfqpoint{1.543909in}{3.220171in}}{\pgfqpoint{1.532859in}{3.220171in}}%
\pgfpathcurveto{\pgfqpoint{1.521809in}{3.220171in}}{\pgfqpoint{1.511210in}{3.215780in}}{\pgfqpoint{1.503396in}{3.207967in}}%
\pgfpathcurveto{\pgfqpoint{1.495583in}{3.200153in}}{\pgfqpoint{1.491192in}{3.189554in}}{\pgfqpoint{1.491192in}{3.178504in}}%
\pgfpathcurveto{\pgfqpoint{1.491192in}{3.167454in}}{\pgfqpoint{1.495583in}{3.156855in}}{\pgfqpoint{1.503396in}{3.149041in}}%
\pgfpathcurveto{\pgfqpoint{1.511210in}{3.141228in}}{\pgfqpoint{1.521809in}{3.136837in}}{\pgfqpoint{1.532859in}{3.136837in}}%
\pgfpathclose%
\pgfusepath{stroke,fill}%
\end{pgfscope}%
\begin{pgfscope}%
\pgfpathrectangle{\pgfqpoint{0.648703in}{0.548769in}}{\pgfqpoint{5.201297in}{3.102590in}}%
\pgfusepath{clip}%
\pgfsetbuttcap%
\pgfsetroundjoin%
\definecolor{currentfill}{rgb}{0.121569,0.466667,0.705882}%
\pgfsetfillcolor{currentfill}%
\pgfsetlinewidth{1.003750pt}%
\definecolor{currentstroke}{rgb}{0.121569,0.466667,0.705882}%
\pgfsetstrokecolor{currentstroke}%
\pgfsetdash{}{0pt}%
\pgfpathmoveto{\pgfqpoint{3.994245in}{3.132690in}}%
\pgfpathcurveto{\pgfqpoint{4.005295in}{3.132690in}}{\pgfqpoint{4.015894in}{3.137080in}}{\pgfqpoint{4.023708in}{3.144893in}}%
\pgfpathcurveto{\pgfqpoint{4.031521in}{3.152707in}}{\pgfqpoint{4.035911in}{3.163306in}}{\pgfqpoint{4.035911in}{3.174356in}}%
\pgfpathcurveto{\pgfqpoint{4.035911in}{3.185406in}}{\pgfqpoint{4.031521in}{3.196005in}}{\pgfqpoint{4.023708in}{3.203819in}}%
\pgfpathcurveto{\pgfqpoint{4.015894in}{3.211633in}}{\pgfqpoint{4.005295in}{3.216023in}}{\pgfqpoint{3.994245in}{3.216023in}}%
\pgfpathcurveto{\pgfqpoint{3.983195in}{3.216023in}}{\pgfqpoint{3.972596in}{3.211633in}}{\pgfqpoint{3.964782in}{3.203819in}}%
\pgfpathcurveto{\pgfqpoint{3.956968in}{3.196005in}}{\pgfqpoint{3.952578in}{3.185406in}}{\pgfqpoint{3.952578in}{3.174356in}}%
\pgfpathcurveto{\pgfqpoint{3.952578in}{3.163306in}}{\pgfqpoint{3.956968in}{3.152707in}}{\pgfqpoint{3.964782in}{3.144893in}}%
\pgfpathcurveto{\pgfqpoint{3.972596in}{3.137080in}}{\pgfqpoint{3.983195in}{3.132690in}}{\pgfqpoint{3.994245in}{3.132690in}}%
\pgfpathclose%
\pgfusepath{stroke,fill}%
\end{pgfscope}%
\begin{pgfscope}%
\pgfpathrectangle{\pgfqpoint{0.648703in}{0.548769in}}{\pgfqpoint{5.201297in}{3.102590in}}%
\pgfusepath{clip}%
\pgfsetbuttcap%
\pgfsetroundjoin%
\definecolor{currentfill}{rgb}{0.839216,0.152941,0.156863}%
\pgfsetfillcolor{currentfill}%
\pgfsetlinewidth{1.003750pt}%
\definecolor{currentstroke}{rgb}{0.839216,0.152941,0.156863}%
\pgfsetstrokecolor{currentstroke}%
\pgfsetdash{}{0pt}%
\pgfpathmoveto{\pgfqpoint{1.144219in}{3.149281in}}%
\pgfpathcurveto{\pgfqpoint{1.155269in}{3.149281in}}{\pgfqpoint{1.165868in}{3.153671in}}{\pgfqpoint{1.173682in}{3.161485in}}%
\pgfpathcurveto{\pgfqpoint{1.181496in}{3.169298in}}{\pgfqpoint{1.185886in}{3.179897in}}{\pgfqpoint{1.185886in}{3.190948in}}%
\pgfpathcurveto{\pgfqpoint{1.185886in}{3.201998in}}{\pgfqpoint{1.181496in}{3.212597in}}{\pgfqpoint{1.173682in}{3.220410in}}%
\pgfpathcurveto{\pgfqpoint{1.165868in}{3.228224in}}{\pgfqpoint{1.155269in}{3.232614in}}{\pgfqpoint{1.144219in}{3.232614in}}%
\pgfpathcurveto{\pgfqpoint{1.133169in}{3.232614in}}{\pgfqpoint{1.122570in}{3.228224in}}{\pgfqpoint{1.114756in}{3.220410in}}%
\pgfpathcurveto{\pgfqpoint{1.106943in}{3.212597in}}{\pgfqpoint{1.102553in}{3.201998in}}{\pgfqpoint{1.102553in}{3.190948in}}%
\pgfpathcurveto{\pgfqpoint{1.102553in}{3.179897in}}{\pgfqpoint{1.106943in}{3.169298in}}{\pgfqpoint{1.114756in}{3.161485in}}%
\pgfpathcurveto{\pgfqpoint{1.122570in}{3.153671in}}{\pgfqpoint{1.133169in}{3.149281in}}{\pgfqpoint{1.144219in}{3.149281in}}%
\pgfpathclose%
\pgfusepath{stroke,fill}%
\end{pgfscope}%
\begin{pgfscope}%
\pgfpathrectangle{\pgfqpoint{0.648703in}{0.548769in}}{\pgfqpoint{5.201297in}{3.102590in}}%
\pgfusepath{clip}%
\pgfsetbuttcap%
\pgfsetroundjoin%
\definecolor{currentfill}{rgb}{1.000000,0.498039,0.054902}%
\pgfsetfillcolor{currentfill}%
\pgfsetlinewidth{1.003750pt}%
\definecolor{currentstroke}{rgb}{1.000000,0.498039,0.054902}%
\pgfsetstrokecolor{currentstroke}%
\pgfsetdash{}{0pt}%
\pgfpathmoveto{\pgfqpoint{0.949899in}{3.174168in}}%
\pgfpathcurveto{\pgfqpoint{0.960949in}{3.174168in}}{\pgfqpoint{0.971548in}{3.178558in}}{\pgfqpoint{0.979362in}{3.186372in}}%
\pgfpathcurveto{\pgfqpoint{0.987176in}{3.194185in}}{\pgfqpoint{0.991566in}{3.204785in}}{\pgfqpoint{0.991566in}{3.215835in}}%
\pgfpathcurveto{\pgfqpoint{0.991566in}{3.226885in}}{\pgfqpoint{0.987176in}{3.237484in}}{\pgfqpoint{0.979362in}{3.245297in}}%
\pgfpathcurveto{\pgfqpoint{0.971548in}{3.253111in}}{\pgfqpoint{0.960949in}{3.257501in}}{\pgfqpoint{0.949899in}{3.257501in}}%
\pgfpathcurveto{\pgfqpoint{0.938849in}{3.257501in}}{\pgfqpoint{0.928250in}{3.253111in}}{\pgfqpoint{0.920437in}{3.245297in}}%
\pgfpathcurveto{\pgfqpoint{0.912623in}{3.237484in}}{\pgfqpoint{0.908233in}{3.226885in}}{\pgfqpoint{0.908233in}{3.215835in}}%
\pgfpathcurveto{\pgfqpoint{0.908233in}{3.204785in}}{\pgfqpoint{0.912623in}{3.194185in}}{\pgfqpoint{0.920437in}{3.186372in}}%
\pgfpathcurveto{\pgfqpoint{0.928250in}{3.178558in}}{\pgfqpoint{0.938849in}{3.174168in}}{\pgfqpoint{0.949899in}{3.174168in}}%
\pgfpathclose%
\pgfusepath{stroke,fill}%
\end{pgfscope}%
\begin{pgfscope}%
\pgfpathrectangle{\pgfqpoint{0.648703in}{0.548769in}}{\pgfqpoint{5.201297in}{3.102590in}}%
\pgfusepath{clip}%
\pgfsetbuttcap%
\pgfsetroundjoin%
\definecolor{currentfill}{rgb}{0.121569,0.466667,0.705882}%
\pgfsetfillcolor{currentfill}%
\pgfsetlinewidth{1.003750pt}%
\definecolor{currentstroke}{rgb}{0.121569,0.466667,0.705882}%
\pgfsetstrokecolor{currentstroke}%
\pgfsetdash{}{0pt}%
\pgfpathmoveto{\pgfqpoint{1.273766in}{0.656425in}}%
\pgfpathcurveto{\pgfqpoint{1.284816in}{0.656425in}}{\pgfqpoint{1.295415in}{0.660815in}}{\pgfqpoint{1.303229in}{0.668629in}}%
\pgfpathcurveto{\pgfqpoint{1.311042in}{0.676442in}}{\pgfqpoint{1.315432in}{0.687041in}}{\pgfqpoint{1.315432in}{0.698091in}}%
\pgfpathcurveto{\pgfqpoint{1.315432in}{0.709141in}}{\pgfqpoint{1.311042in}{0.719740in}}{\pgfqpoint{1.303229in}{0.727554in}}%
\pgfpathcurveto{\pgfqpoint{1.295415in}{0.735368in}}{\pgfqpoint{1.284816in}{0.739758in}}{\pgfqpoint{1.273766in}{0.739758in}}%
\pgfpathcurveto{\pgfqpoint{1.262716in}{0.739758in}}{\pgfqpoint{1.252117in}{0.735368in}}{\pgfqpoint{1.244303in}{0.727554in}}%
\pgfpathcurveto{\pgfqpoint{1.236489in}{0.719740in}}{\pgfqpoint{1.232099in}{0.709141in}}{\pgfqpoint{1.232099in}{0.698091in}}%
\pgfpathcurveto{\pgfqpoint{1.232099in}{0.687041in}}{\pgfqpoint{1.236489in}{0.676442in}}{\pgfqpoint{1.244303in}{0.668629in}}%
\pgfpathcurveto{\pgfqpoint{1.252117in}{0.660815in}}{\pgfqpoint{1.262716in}{0.656425in}}{\pgfqpoint{1.273766in}{0.656425in}}%
\pgfpathclose%
\pgfusepath{stroke,fill}%
\end{pgfscope}%
\begin{pgfscope}%
\pgfpathrectangle{\pgfqpoint{0.648703in}{0.548769in}}{\pgfqpoint{5.201297in}{3.102590in}}%
\pgfusepath{clip}%
\pgfsetbuttcap%
\pgfsetroundjoin%
\definecolor{currentfill}{rgb}{0.121569,0.466667,0.705882}%
\pgfsetfillcolor{currentfill}%
\pgfsetlinewidth{1.003750pt}%
\definecolor{currentstroke}{rgb}{0.121569,0.466667,0.705882}%
\pgfsetstrokecolor{currentstroke}%
\pgfsetdash{}{0pt}%
\pgfpathmoveto{\pgfqpoint{1.856726in}{3.124394in}}%
\pgfpathcurveto{\pgfqpoint{1.867776in}{3.124394in}}{\pgfqpoint{1.878375in}{3.128784in}}{\pgfqpoint{1.886188in}{3.136598in}}%
\pgfpathcurveto{\pgfqpoint{1.894002in}{3.144411in}}{\pgfqpoint{1.898392in}{3.155010in}}{\pgfqpoint{1.898392in}{3.166060in}}%
\pgfpathcurveto{\pgfqpoint{1.898392in}{3.177111in}}{\pgfqpoint{1.894002in}{3.187710in}}{\pgfqpoint{1.886188in}{3.195523in}}%
\pgfpathcurveto{\pgfqpoint{1.878375in}{3.203337in}}{\pgfqpoint{1.867776in}{3.207727in}}{\pgfqpoint{1.856726in}{3.207727in}}%
\pgfpathcurveto{\pgfqpoint{1.845675in}{3.207727in}}{\pgfqpoint{1.835076in}{3.203337in}}{\pgfqpoint{1.827263in}{3.195523in}}%
\pgfpathcurveto{\pgfqpoint{1.819449in}{3.187710in}}{\pgfqpoint{1.815059in}{3.177111in}}{\pgfqpoint{1.815059in}{3.166060in}}%
\pgfpathcurveto{\pgfqpoint{1.815059in}{3.155010in}}{\pgfqpoint{1.819449in}{3.144411in}}{\pgfqpoint{1.827263in}{3.136598in}}%
\pgfpathcurveto{\pgfqpoint{1.835076in}{3.128784in}}{\pgfqpoint{1.845675in}{3.124394in}}{\pgfqpoint{1.856726in}{3.124394in}}%
\pgfpathclose%
\pgfusepath{stroke,fill}%
\end{pgfscope}%
\begin{pgfscope}%
\pgfpathrectangle{\pgfqpoint{0.648703in}{0.548769in}}{\pgfqpoint{5.201297in}{3.102590in}}%
\pgfusepath{clip}%
\pgfsetbuttcap%
\pgfsetroundjoin%
\definecolor{currentfill}{rgb}{0.121569,0.466667,0.705882}%
\pgfsetfillcolor{currentfill}%
\pgfsetlinewidth{1.003750pt}%
\definecolor{currentstroke}{rgb}{0.121569,0.466667,0.705882}%
\pgfsetstrokecolor{currentstroke}%
\pgfsetdash{}{0pt}%
\pgfpathmoveto{\pgfqpoint{0.885126in}{0.648129in}}%
\pgfpathcurveto{\pgfqpoint{0.896176in}{0.648129in}}{\pgfqpoint{0.906775in}{0.652519in}}{\pgfqpoint{0.914589in}{0.660333in}}%
\pgfpathcurveto{\pgfqpoint{0.922402in}{0.668146in}}{\pgfqpoint{0.926793in}{0.678745in}}{\pgfqpoint{0.926793in}{0.689796in}}%
\pgfpathcurveto{\pgfqpoint{0.926793in}{0.700846in}}{\pgfqpoint{0.922402in}{0.711445in}}{\pgfqpoint{0.914589in}{0.719258in}}%
\pgfpathcurveto{\pgfqpoint{0.906775in}{0.727072in}}{\pgfqpoint{0.896176in}{0.731462in}}{\pgfqpoint{0.885126in}{0.731462in}}%
\pgfpathcurveto{\pgfqpoint{0.874076in}{0.731462in}}{\pgfqpoint{0.863477in}{0.727072in}}{\pgfqpoint{0.855663in}{0.719258in}}%
\pgfpathcurveto{\pgfqpoint{0.847850in}{0.711445in}}{\pgfqpoint{0.843459in}{0.700846in}}{\pgfqpoint{0.843459in}{0.689796in}}%
\pgfpathcurveto{\pgfqpoint{0.843459in}{0.678745in}}{\pgfqpoint{0.847850in}{0.668146in}}{\pgfqpoint{0.855663in}{0.660333in}}%
\pgfpathcurveto{\pgfqpoint{0.863477in}{0.652519in}}{\pgfqpoint{0.874076in}{0.648129in}}{\pgfqpoint{0.885126in}{0.648129in}}%
\pgfpathclose%
\pgfusepath{stroke,fill}%
\end{pgfscope}%
\begin{pgfscope}%
\pgfpathrectangle{\pgfqpoint{0.648703in}{0.548769in}}{\pgfqpoint{5.201297in}{3.102590in}}%
\pgfusepath{clip}%
\pgfsetbuttcap%
\pgfsetroundjoin%
\definecolor{currentfill}{rgb}{1.000000,0.498039,0.054902}%
\pgfsetfillcolor{currentfill}%
\pgfsetlinewidth{1.003750pt}%
\definecolor{currentstroke}{rgb}{1.000000,0.498039,0.054902}%
\pgfsetstrokecolor{currentstroke}%
\pgfsetdash{}{0pt}%
\pgfpathmoveto{\pgfqpoint{1.856726in}{3.136837in}}%
\pgfpathcurveto{\pgfqpoint{1.867776in}{3.136837in}}{\pgfqpoint{1.878375in}{3.141228in}}{\pgfqpoint{1.886188in}{3.149041in}}%
\pgfpathcurveto{\pgfqpoint{1.894002in}{3.156855in}}{\pgfqpoint{1.898392in}{3.167454in}}{\pgfqpoint{1.898392in}{3.178504in}}%
\pgfpathcurveto{\pgfqpoint{1.898392in}{3.189554in}}{\pgfqpoint{1.894002in}{3.200153in}}{\pgfqpoint{1.886188in}{3.207967in}}%
\pgfpathcurveto{\pgfqpoint{1.878375in}{3.215780in}}{\pgfqpoint{1.867776in}{3.220171in}}{\pgfqpoint{1.856726in}{3.220171in}}%
\pgfpathcurveto{\pgfqpoint{1.845675in}{3.220171in}}{\pgfqpoint{1.835076in}{3.215780in}}{\pgfqpoint{1.827263in}{3.207967in}}%
\pgfpathcurveto{\pgfqpoint{1.819449in}{3.200153in}}{\pgfqpoint{1.815059in}{3.189554in}}{\pgfqpoint{1.815059in}{3.178504in}}%
\pgfpathcurveto{\pgfqpoint{1.815059in}{3.167454in}}{\pgfqpoint{1.819449in}{3.156855in}}{\pgfqpoint{1.827263in}{3.149041in}}%
\pgfpathcurveto{\pgfqpoint{1.835076in}{3.141228in}}{\pgfqpoint{1.845675in}{3.136837in}}{\pgfqpoint{1.856726in}{3.136837in}}%
\pgfpathclose%
\pgfusepath{stroke,fill}%
\end{pgfscope}%
\begin{pgfscope}%
\pgfpathrectangle{\pgfqpoint{0.648703in}{0.548769in}}{\pgfqpoint{5.201297in}{3.102590in}}%
\pgfusepath{clip}%
\pgfsetbuttcap%
\pgfsetroundjoin%
\definecolor{currentfill}{rgb}{1.000000,0.498039,0.054902}%
\pgfsetfillcolor{currentfill}%
\pgfsetlinewidth{1.003750pt}%
\definecolor{currentstroke}{rgb}{1.000000,0.498039,0.054902}%
\pgfsetstrokecolor{currentstroke}%
\pgfsetdash{}{0pt}%
\pgfpathmoveto{\pgfqpoint{1.727179in}{3.140985in}}%
\pgfpathcurveto{\pgfqpoint{1.738229in}{3.140985in}}{\pgfqpoint{1.748828in}{3.145375in}}{\pgfqpoint{1.756642in}{3.153189in}}%
\pgfpathcurveto{\pgfqpoint{1.764455in}{3.161003in}}{\pgfqpoint{1.768846in}{3.171602in}}{\pgfqpoint{1.768846in}{3.182652in}}%
\pgfpathcurveto{\pgfqpoint{1.768846in}{3.193702in}}{\pgfqpoint{1.764455in}{3.204301in}}{\pgfqpoint{1.756642in}{3.212115in}}%
\pgfpathcurveto{\pgfqpoint{1.748828in}{3.219928in}}{\pgfqpoint{1.738229in}{3.224319in}}{\pgfqpoint{1.727179in}{3.224319in}}%
\pgfpathcurveto{\pgfqpoint{1.716129in}{3.224319in}}{\pgfqpoint{1.705530in}{3.219928in}}{\pgfqpoint{1.697716in}{3.212115in}}%
\pgfpathcurveto{\pgfqpoint{1.689903in}{3.204301in}}{\pgfqpoint{1.685512in}{3.193702in}}{\pgfqpoint{1.685512in}{3.182652in}}%
\pgfpathcurveto{\pgfqpoint{1.685512in}{3.171602in}}{\pgfqpoint{1.689903in}{3.161003in}}{\pgfqpoint{1.697716in}{3.153189in}}%
\pgfpathcurveto{\pgfqpoint{1.705530in}{3.145375in}}{\pgfqpoint{1.716129in}{3.140985in}}{\pgfqpoint{1.727179in}{3.140985in}}%
\pgfpathclose%
\pgfusepath{stroke,fill}%
\end{pgfscope}%
\begin{pgfscope}%
\pgfpathrectangle{\pgfqpoint{0.648703in}{0.548769in}}{\pgfqpoint{5.201297in}{3.102590in}}%
\pgfusepath{clip}%
\pgfsetbuttcap%
\pgfsetroundjoin%
\definecolor{currentfill}{rgb}{1.000000,0.498039,0.054902}%
\pgfsetfillcolor{currentfill}%
\pgfsetlinewidth{1.003750pt}%
\definecolor{currentstroke}{rgb}{1.000000,0.498039,0.054902}%
\pgfsetstrokecolor{currentstroke}%
\pgfsetdash{}{0pt}%
\pgfpathmoveto{\pgfqpoint{1.986272in}{3.165872in}}%
\pgfpathcurveto{\pgfqpoint{1.997322in}{3.165872in}}{\pgfqpoint{2.007921in}{3.170263in}}{\pgfqpoint{2.015735in}{3.178076in}}%
\pgfpathcurveto{\pgfqpoint{2.023549in}{3.185890in}}{\pgfqpoint{2.027939in}{3.196489in}}{\pgfqpoint{2.027939in}{3.207539in}}%
\pgfpathcurveto{\pgfqpoint{2.027939in}{3.218589in}}{\pgfqpoint{2.023549in}{3.229188in}}{\pgfqpoint{2.015735in}{3.237002in}}%
\pgfpathcurveto{\pgfqpoint{2.007921in}{3.244815in}}{\pgfqpoint{1.997322in}{3.249206in}}{\pgfqpoint{1.986272in}{3.249206in}}%
\pgfpathcurveto{\pgfqpoint{1.975222in}{3.249206in}}{\pgfqpoint{1.964623in}{3.244815in}}{\pgfqpoint{1.956809in}{3.237002in}}%
\pgfpathcurveto{\pgfqpoint{1.948996in}{3.229188in}}{\pgfqpoint{1.944606in}{3.218589in}}{\pgfqpoint{1.944606in}{3.207539in}}%
\pgfpathcurveto{\pgfqpoint{1.944606in}{3.196489in}}{\pgfqpoint{1.948996in}{3.185890in}}{\pgfqpoint{1.956809in}{3.178076in}}%
\pgfpathcurveto{\pgfqpoint{1.964623in}{3.170263in}}{\pgfqpoint{1.975222in}{3.165872in}}{\pgfqpoint{1.986272in}{3.165872in}}%
\pgfpathclose%
\pgfusepath{stroke,fill}%
\end{pgfscope}%
\begin{pgfscope}%
\pgfpathrectangle{\pgfqpoint{0.648703in}{0.548769in}}{\pgfqpoint{5.201297in}{3.102590in}}%
\pgfusepath{clip}%
\pgfsetbuttcap%
\pgfsetroundjoin%
\definecolor{currentfill}{rgb}{1.000000,0.498039,0.054902}%
\pgfsetfillcolor{currentfill}%
\pgfsetlinewidth{1.003750pt}%
\definecolor{currentstroke}{rgb}{1.000000,0.498039,0.054902}%
\pgfsetstrokecolor{currentstroke}%
\pgfsetdash{}{0pt}%
\pgfpathmoveto{\pgfqpoint{1.079446in}{3.145133in}}%
\pgfpathcurveto{\pgfqpoint{1.090496in}{3.145133in}}{\pgfqpoint{1.101095in}{3.149523in}}{\pgfqpoint{1.108909in}{3.157337in}}%
\pgfpathcurveto{\pgfqpoint{1.116722in}{3.165151in}}{\pgfqpoint{1.121113in}{3.175750in}}{\pgfqpoint{1.121113in}{3.186800in}}%
\pgfpathcurveto{\pgfqpoint{1.121113in}{3.197850in}}{\pgfqpoint{1.116722in}{3.208449in}}{\pgfqpoint{1.108909in}{3.216262in}}%
\pgfpathcurveto{\pgfqpoint{1.101095in}{3.224076in}}{\pgfqpoint{1.090496in}{3.228466in}}{\pgfqpoint{1.079446in}{3.228466in}}%
\pgfpathcurveto{\pgfqpoint{1.068396in}{3.228466in}}{\pgfqpoint{1.057797in}{3.224076in}}{\pgfqpoint{1.049983in}{3.216262in}}%
\pgfpathcurveto{\pgfqpoint{1.042170in}{3.208449in}}{\pgfqpoint{1.037779in}{3.197850in}}{\pgfqpoint{1.037779in}{3.186800in}}%
\pgfpathcurveto{\pgfqpoint{1.037779in}{3.175750in}}{\pgfqpoint{1.042170in}{3.165151in}}{\pgfqpoint{1.049983in}{3.157337in}}%
\pgfpathcurveto{\pgfqpoint{1.057797in}{3.149523in}}{\pgfqpoint{1.068396in}{3.145133in}}{\pgfqpoint{1.079446in}{3.145133in}}%
\pgfpathclose%
\pgfusepath{stroke,fill}%
\end{pgfscope}%
\begin{pgfscope}%
\pgfpathrectangle{\pgfqpoint{0.648703in}{0.548769in}}{\pgfqpoint{5.201297in}{3.102590in}}%
\pgfusepath{clip}%
\pgfsetbuttcap%
\pgfsetroundjoin%
\definecolor{currentfill}{rgb}{1.000000,0.498039,0.054902}%
\pgfsetfillcolor{currentfill}%
\pgfsetlinewidth{1.003750pt}%
\definecolor{currentstroke}{rgb}{1.000000,0.498039,0.054902}%
\pgfsetstrokecolor{currentstroke}%
\pgfsetdash{}{0pt}%
\pgfpathmoveto{\pgfqpoint{1.079446in}{3.157577in}}%
\pgfpathcurveto{\pgfqpoint{1.090496in}{3.157577in}}{\pgfqpoint{1.101095in}{3.161967in}}{\pgfqpoint{1.108909in}{3.169780in}}%
\pgfpathcurveto{\pgfqpoint{1.116722in}{3.177594in}}{\pgfqpoint{1.121113in}{3.188193in}}{\pgfqpoint{1.121113in}{3.199243in}}%
\pgfpathcurveto{\pgfqpoint{1.121113in}{3.210293in}}{\pgfqpoint{1.116722in}{3.220892in}}{\pgfqpoint{1.108909in}{3.228706in}}%
\pgfpathcurveto{\pgfqpoint{1.101095in}{3.236520in}}{\pgfqpoint{1.090496in}{3.240910in}}{\pgfqpoint{1.079446in}{3.240910in}}%
\pgfpathcurveto{\pgfqpoint{1.068396in}{3.240910in}}{\pgfqpoint{1.057797in}{3.236520in}}{\pgfqpoint{1.049983in}{3.228706in}}%
\pgfpathcurveto{\pgfqpoint{1.042170in}{3.220892in}}{\pgfqpoint{1.037779in}{3.210293in}}{\pgfqpoint{1.037779in}{3.199243in}}%
\pgfpathcurveto{\pgfqpoint{1.037779in}{3.188193in}}{\pgfqpoint{1.042170in}{3.177594in}}{\pgfqpoint{1.049983in}{3.169780in}}%
\pgfpathcurveto{\pgfqpoint{1.057797in}{3.161967in}}{\pgfqpoint{1.068396in}{3.157577in}}{\pgfqpoint{1.079446in}{3.157577in}}%
\pgfpathclose%
\pgfusepath{stroke,fill}%
\end{pgfscope}%
\begin{pgfscope}%
\pgfpathrectangle{\pgfqpoint{0.648703in}{0.548769in}}{\pgfqpoint{5.201297in}{3.102590in}}%
\pgfusepath{clip}%
\pgfsetbuttcap%
\pgfsetroundjoin%
\definecolor{currentfill}{rgb}{1.000000,0.498039,0.054902}%
\pgfsetfillcolor{currentfill}%
\pgfsetlinewidth{1.003750pt}%
\definecolor{currentstroke}{rgb}{1.000000,0.498039,0.054902}%
\pgfsetstrokecolor{currentstroke}%
\pgfsetdash{}{0pt}%
\pgfpathmoveto{\pgfqpoint{1.532859in}{3.323490in}}%
\pgfpathcurveto{\pgfqpoint{1.543909in}{3.323490in}}{\pgfqpoint{1.554508in}{3.327881in}}{\pgfqpoint{1.562322in}{3.335694in}}%
\pgfpathcurveto{\pgfqpoint{1.570135in}{3.343508in}}{\pgfqpoint{1.574526in}{3.354107in}}{\pgfqpoint{1.574526in}{3.365157in}}%
\pgfpathcurveto{\pgfqpoint{1.574526in}{3.376207in}}{\pgfqpoint{1.570135in}{3.386806in}}{\pgfqpoint{1.562322in}{3.394620in}}%
\pgfpathcurveto{\pgfqpoint{1.554508in}{3.402434in}}{\pgfqpoint{1.543909in}{3.406824in}}{\pgfqpoint{1.532859in}{3.406824in}}%
\pgfpathcurveto{\pgfqpoint{1.521809in}{3.406824in}}{\pgfqpoint{1.511210in}{3.402434in}}{\pgfqpoint{1.503396in}{3.394620in}}%
\pgfpathcurveto{\pgfqpoint{1.495583in}{3.386806in}}{\pgfqpoint{1.491192in}{3.376207in}}{\pgfqpoint{1.491192in}{3.365157in}}%
\pgfpathcurveto{\pgfqpoint{1.491192in}{3.354107in}}{\pgfqpoint{1.495583in}{3.343508in}}{\pgfqpoint{1.503396in}{3.335694in}}%
\pgfpathcurveto{\pgfqpoint{1.511210in}{3.327881in}}{\pgfqpoint{1.521809in}{3.323490in}}{\pgfqpoint{1.532859in}{3.323490in}}%
\pgfpathclose%
\pgfusepath{stroke,fill}%
\end{pgfscope}%
\begin{pgfscope}%
\pgfpathrectangle{\pgfqpoint{0.648703in}{0.548769in}}{\pgfqpoint{5.201297in}{3.102590in}}%
\pgfusepath{clip}%
\pgfsetbuttcap%
\pgfsetroundjoin%
\definecolor{currentfill}{rgb}{0.121569,0.466667,0.705882}%
\pgfsetfillcolor{currentfill}%
\pgfsetlinewidth{1.003750pt}%
\definecolor{currentstroke}{rgb}{0.121569,0.466667,0.705882}%
\pgfsetstrokecolor{currentstroke}%
\pgfsetdash{}{0pt}%
\pgfpathmoveto{\pgfqpoint{0.949899in}{2.410964in}}%
\pgfpathcurveto{\pgfqpoint{0.960949in}{2.410964in}}{\pgfqpoint{0.971548in}{2.415354in}}{\pgfqpoint{0.979362in}{2.423168in}}%
\pgfpathcurveto{\pgfqpoint{0.987176in}{2.430982in}}{\pgfqpoint{0.991566in}{2.441581in}}{\pgfqpoint{0.991566in}{2.452631in}}%
\pgfpathcurveto{\pgfqpoint{0.991566in}{2.463681in}}{\pgfqpoint{0.987176in}{2.474280in}}{\pgfqpoint{0.979362in}{2.482094in}}%
\pgfpathcurveto{\pgfqpoint{0.971548in}{2.489907in}}{\pgfqpoint{0.960949in}{2.494297in}}{\pgfqpoint{0.949899in}{2.494297in}}%
\pgfpathcurveto{\pgfqpoint{0.938849in}{2.494297in}}{\pgfqpoint{0.928250in}{2.489907in}}{\pgfqpoint{0.920437in}{2.482094in}}%
\pgfpathcurveto{\pgfqpoint{0.912623in}{2.474280in}}{\pgfqpoint{0.908233in}{2.463681in}}{\pgfqpoint{0.908233in}{2.452631in}}%
\pgfpathcurveto{\pgfqpoint{0.908233in}{2.441581in}}{\pgfqpoint{0.912623in}{2.430982in}}{\pgfqpoint{0.920437in}{2.423168in}}%
\pgfpathcurveto{\pgfqpoint{0.928250in}{2.415354in}}{\pgfqpoint{0.938849in}{2.410964in}}{\pgfqpoint{0.949899in}{2.410964in}}%
\pgfpathclose%
\pgfusepath{stroke,fill}%
\end{pgfscope}%
\begin{pgfscope}%
\pgfpathrectangle{\pgfqpoint{0.648703in}{0.548769in}}{\pgfqpoint{5.201297in}{3.102590in}}%
\pgfusepath{clip}%
\pgfsetbuttcap%
\pgfsetroundjoin%
\definecolor{currentfill}{rgb}{1.000000,0.498039,0.054902}%
\pgfsetfillcolor{currentfill}%
\pgfsetlinewidth{1.003750pt}%
\definecolor{currentstroke}{rgb}{1.000000,0.498039,0.054902}%
\pgfsetstrokecolor{currentstroke}%
\pgfsetdash{}{0pt}%
\pgfpathmoveto{\pgfqpoint{1.208993in}{3.199055in}}%
\pgfpathcurveto{\pgfqpoint{1.220043in}{3.199055in}}{\pgfqpoint{1.230642in}{3.203445in}}{\pgfqpoint{1.238455in}{3.211259in}}%
\pgfpathcurveto{\pgfqpoint{1.246269in}{3.219073in}}{\pgfqpoint{1.250659in}{3.229672in}}{\pgfqpoint{1.250659in}{3.240722in}}%
\pgfpathcurveto{\pgfqpoint{1.250659in}{3.251772in}}{\pgfqpoint{1.246269in}{3.262371in}}{\pgfqpoint{1.238455in}{3.270185in}}%
\pgfpathcurveto{\pgfqpoint{1.230642in}{3.277998in}}{\pgfqpoint{1.220043in}{3.282388in}}{\pgfqpoint{1.208993in}{3.282388in}}%
\pgfpathcurveto{\pgfqpoint{1.197942in}{3.282388in}}{\pgfqpoint{1.187343in}{3.277998in}}{\pgfqpoint{1.179530in}{3.270185in}}%
\pgfpathcurveto{\pgfqpoint{1.171716in}{3.262371in}}{\pgfqpoint{1.167326in}{3.251772in}}{\pgfqpoint{1.167326in}{3.240722in}}%
\pgfpathcurveto{\pgfqpoint{1.167326in}{3.229672in}}{\pgfqpoint{1.171716in}{3.219073in}}{\pgfqpoint{1.179530in}{3.211259in}}%
\pgfpathcurveto{\pgfqpoint{1.187343in}{3.203445in}}{\pgfqpoint{1.197942in}{3.199055in}}{\pgfqpoint{1.208993in}{3.199055in}}%
\pgfpathclose%
\pgfusepath{stroke,fill}%
\end{pgfscope}%
\begin{pgfscope}%
\pgfpathrectangle{\pgfqpoint{0.648703in}{0.548769in}}{\pgfqpoint{5.201297in}{3.102590in}}%
\pgfusepath{clip}%
\pgfsetbuttcap%
\pgfsetroundjoin%
\definecolor{currentfill}{rgb}{0.121569,0.466667,0.705882}%
\pgfsetfillcolor{currentfill}%
\pgfsetlinewidth{1.003750pt}%
\definecolor{currentstroke}{rgb}{0.121569,0.466667,0.705882}%
\pgfsetstrokecolor{currentstroke}%
\pgfsetdash{}{0pt}%
\pgfpathmoveto{\pgfqpoint{2.439685in}{3.124394in}}%
\pgfpathcurveto{\pgfqpoint{2.450735in}{3.124394in}}{\pgfqpoint{2.461335in}{3.128784in}}{\pgfqpoint{2.469148in}{3.136598in}}%
\pgfpathcurveto{\pgfqpoint{2.476962in}{3.144411in}}{\pgfqpoint{2.481352in}{3.155010in}}{\pgfqpoint{2.481352in}{3.166060in}}%
\pgfpathcurveto{\pgfqpoint{2.481352in}{3.177111in}}{\pgfqpoint{2.476962in}{3.187710in}}{\pgfqpoint{2.469148in}{3.195523in}}%
\pgfpathcurveto{\pgfqpoint{2.461335in}{3.203337in}}{\pgfqpoint{2.450735in}{3.207727in}}{\pgfqpoint{2.439685in}{3.207727in}}%
\pgfpathcurveto{\pgfqpoint{2.428635in}{3.207727in}}{\pgfqpoint{2.418036in}{3.203337in}}{\pgfqpoint{2.410223in}{3.195523in}}%
\pgfpathcurveto{\pgfqpoint{2.402409in}{3.187710in}}{\pgfqpoint{2.398019in}{3.177111in}}{\pgfqpoint{2.398019in}{3.166060in}}%
\pgfpathcurveto{\pgfqpoint{2.398019in}{3.155010in}}{\pgfqpoint{2.402409in}{3.144411in}}{\pgfqpoint{2.410223in}{3.136598in}}%
\pgfpathcurveto{\pgfqpoint{2.418036in}{3.128784in}}{\pgfqpoint{2.428635in}{3.124394in}}{\pgfqpoint{2.439685in}{3.124394in}}%
\pgfpathclose%
\pgfusepath{stroke,fill}%
\end{pgfscope}%
\begin{pgfscope}%
\pgfpathrectangle{\pgfqpoint{0.648703in}{0.548769in}}{\pgfqpoint{5.201297in}{3.102590in}}%
\pgfusepath{clip}%
\pgfsetbuttcap%
\pgfsetroundjoin%
\definecolor{currentfill}{rgb}{0.121569,0.466667,0.705882}%
\pgfsetfillcolor{currentfill}%
\pgfsetlinewidth{1.003750pt}%
\definecolor{currentstroke}{rgb}{0.121569,0.466667,0.705882}%
\pgfsetstrokecolor{currentstroke}%
\pgfsetdash{}{0pt}%
\pgfpathmoveto{\pgfqpoint{2.310139in}{3.132690in}}%
\pgfpathcurveto{\pgfqpoint{2.321189in}{3.132690in}}{\pgfqpoint{2.331788in}{3.137080in}}{\pgfqpoint{2.339602in}{3.144893in}}%
\pgfpathcurveto{\pgfqpoint{2.347415in}{3.152707in}}{\pgfqpoint{2.351805in}{3.163306in}}{\pgfqpoint{2.351805in}{3.174356in}}%
\pgfpathcurveto{\pgfqpoint{2.351805in}{3.185406in}}{\pgfqpoint{2.347415in}{3.196005in}}{\pgfqpoint{2.339602in}{3.203819in}}%
\pgfpathcurveto{\pgfqpoint{2.331788in}{3.211633in}}{\pgfqpoint{2.321189in}{3.216023in}}{\pgfqpoint{2.310139in}{3.216023in}}%
\pgfpathcurveto{\pgfqpoint{2.299089in}{3.216023in}}{\pgfqpoint{2.288490in}{3.211633in}}{\pgfqpoint{2.280676in}{3.203819in}}%
\pgfpathcurveto{\pgfqpoint{2.272862in}{3.196005in}}{\pgfqpoint{2.268472in}{3.185406in}}{\pgfqpoint{2.268472in}{3.174356in}}%
\pgfpathcurveto{\pgfqpoint{2.268472in}{3.163306in}}{\pgfqpoint{2.272862in}{3.152707in}}{\pgfqpoint{2.280676in}{3.144893in}}%
\pgfpathcurveto{\pgfqpoint{2.288490in}{3.137080in}}{\pgfqpoint{2.299089in}{3.132690in}}{\pgfqpoint{2.310139in}{3.132690in}}%
\pgfpathclose%
\pgfusepath{stroke,fill}%
\end{pgfscope}%
\begin{pgfscope}%
\pgfpathrectangle{\pgfqpoint{0.648703in}{0.548769in}}{\pgfqpoint{5.201297in}{3.102590in}}%
\pgfusepath{clip}%
\pgfsetbuttcap%
\pgfsetroundjoin%
\definecolor{currentfill}{rgb}{0.121569,0.466667,0.705882}%
\pgfsetfillcolor{currentfill}%
\pgfsetlinewidth{1.003750pt}%
\definecolor{currentstroke}{rgb}{0.121569,0.466667,0.705882}%
\pgfsetstrokecolor{currentstroke}%
\pgfsetdash{}{0pt}%
\pgfpathmoveto{\pgfqpoint{2.569232in}{3.132690in}}%
\pgfpathcurveto{\pgfqpoint{2.580282in}{3.132690in}}{\pgfqpoint{2.590881in}{3.137080in}}{\pgfqpoint{2.598695in}{3.144893in}}%
\pgfpathcurveto{\pgfqpoint{2.606508in}{3.152707in}}{\pgfqpoint{2.610899in}{3.163306in}}{\pgfqpoint{2.610899in}{3.174356in}}%
\pgfpathcurveto{\pgfqpoint{2.610899in}{3.185406in}}{\pgfqpoint{2.606508in}{3.196005in}}{\pgfqpoint{2.598695in}{3.203819in}}%
\pgfpathcurveto{\pgfqpoint{2.590881in}{3.211633in}}{\pgfqpoint{2.580282in}{3.216023in}}{\pgfqpoint{2.569232in}{3.216023in}}%
\pgfpathcurveto{\pgfqpoint{2.558182in}{3.216023in}}{\pgfqpoint{2.547583in}{3.211633in}}{\pgfqpoint{2.539769in}{3.203819in}}%
\pgfpathcurveto{\pgfqpoint{2.531956in}{3.196005in}}{\pgfqpoint{2.527565in}{3.185406in}}{\pgfqpoint{2.527565in}{3.174356in}}%
\pgfpathcurveto{\pgfqpoint{2.527565in}{3.163306in}}{\pgfqpoint{2.531956in}{3.152707in}}{\pgfqpoint{2.539769in}{3.144893in}}%
\pgfpathcurveto{\pgfqpoint{2.547583in}{3.137080in}}{\pgfqpoint{2.558182in}{3.132690in}}{\pgfqpoint{2.569232in}{3.132690in}}%
\pgfpathclose%
\pgfusepath{stroke,fill}%
\end{pgfscope}%
\begin{pgfscope}%
\pgfpathrectangle{\pgfqpoint{0.648703in}{0.548769in}}{\pgfqpoint{5.201297in}{3.102590in}}%
\pgfusepath{clip}%
\pgfsetbuttcap%
\pgfsetroundjoin%
\definecolor{currentfill}{rgb}{1.000000,0.498039,0.054902}%
\pgfsetfillcolor{currentfill}%
\pgfsetlinewidth{1.003750pt}%
\definecolor{currentstroke}{rgb}{1.000000,0.498039,0.054902}%
\pgfsetstrokecolor{currentstroke}%
\pgfsetdash{}{0pt}%
\pgfpathmoveto{\pgfqpoint{0.949899in}{3.136837in}}%
\pgfpathcurveto{\pgfqpoint{0.960949in}{3.136837in}}{\pgfqpoint{0.971548in}{3.141228in}}{\pgfqpoint{0.979362in}{3.149041in}}%
\pgfpathcurveto{\pgfqpoint{0.987176in}{3.156855in}}{\pgfqpoint{0.991566in}{3.167454in}}{\pgfqpoint{0.991566in}{3.178504in}}%
\pgfpathcurveto{\pgfqpoint{0.991566in}{3.189554in}}{\pgfqpoint{0.987176in}{3.200153in}}{\pgfqpoint{0.979362in}{3.207967in}}%
\pgfpathcurveto{\pgfqpoint{0.971548in}{3.215780in}}{\pgfqpoint{0.960949in}{3.220171in}}{\pgfqpoint{0.949899in}{3.220171in}}%
\pgfpathcurveto{\pgfqpoint{0.938849in}{3.220171in}}{\pgfqpoint{0.928250in}{3.215780in}}{\pgfqpoint{0.920437in}{3.207967in}}%
\pgfpathcurveto{\pgfqpoint{0.912623in}{3.200153in}}{\pgfqpoint{0.908233in}{3.189554in}}{\pgfqpoint{0.908233in}{3.178504in}}%
\pgfpathcurveto{\pgfqpoint{0.908233in}{3.167454in}}{\pgfqpoint{0.912623in}{3.156855in}}{\pgfqpoint{0.920437in}{3.149041in}}%
\pgfpathcurveto{\pgfqpoint{0.928250in}{3.141228in}}{\pgfqpoint{0.938849in}{3.136837in}}{\pgfqpoint{0.949899in}{3.136837in}}%
\pgfpathclose%
\pgfusepath{stroke,fill}%
\end{pgfscope}%
\begin{pgfscope}%
\pgfpathrectangle{\pgfqpoint{0.648703in}{0.548769in}}{\pgfqpoint{5.201297in}{3.102590in}}%
\pgfusepath{clip}%
\pgfsetbuttcap%
\pgfsetroundjoin%
\definecolor{currentfill}{rgb}{1.000000,0.498039,0.054902}%
\pgfsetfillcolor{currentfill}%
\pgfsetlinewidth{1.003750pt}%
\definecolor{currentstroke}{rgb}{1.000000,0.498039,0.054902}%
\pgfsetstrokecolor{currentstroke}%
\pgfsetdash{}{0pt}%
\pgfpathmoveto{\pgfqpoint{1.014673in}{3.244681in}}%
\pgfpathcurveto{\pgfqpoint{1.025723in}{3.244681in}}{\pgfqpoint{1.036322in}{3.249072in}}{\pgfqpoint{1.044135in}{3.256885in}}%
\pgfpathcurveto{\pgfqpoint{1.051949in}{3.264699in}}{\pgfqpoint{1.056339in}{3.275298in}}{\pgfqpoint{1.056339in}{3.286348in}}%
\pgfpathcurveto{\pgfqpoint{1.056339in}{3.297398in}}{\pgfqpoint{1.051949in}{3.307997in}}{\pgfqpoint{1.044135in}{3.315811in}}%
\pgfpathcurveto{\pgfqpoint{1.036322in}{3.323624in}}{\pgfqpoint{1.025723in}{3.328015in}}{\pgfqpoint{1.014673in}{3.328015in}}%
\pgfpathcurveto{\pgfqpoint{1.003622in}{3.328015in}}{\pgfqpoint{0.993023in}{3.323624in}}{\pgfqpoint{0.985210in}{3.315811in}}%
\pgfpathcurveto{\pgfqpoint{0.977396in}{3.307997in}}{\pgfqpoint{0.973006in}{3.297398in}}{\pgfqpoint{0.973006in}{3.286348in}}%
\pgfpathcurveto{\pgfqpoint{0.973006in}{3.275298in}}{\pgfqpoint{0.977396in}{3.264699in}}{\pgfqpoint{0.985210in}{3.256885in}}%
\pgfpathcurveto{\pgfqpoint{0.993023in}{3.249072in}}{\pgfqpoint{1.003622in}{3.244681in}}{\pgfqpoint{1.014673in}{3.244681in}}%
\pgfpathclose%
\pgfusepath{stroke,fill}%
\end{pgfscope}%
\begin{pgfscope}%
\pgfpathrectangle{\pgfqpoint{0.648703in}{0.548769in}}{\pgfqpoint{5.201297in}{3.102590in}}%
\pgfusepath{clip}%
\pgfsetbuttcap%
\pgfsetroundjoin%
\definecolor{currentfill}{rgb}{1.000000,0.498039,0.054902}%
\pgfsetfillcolor{currentfill}%
\pgfsetlinewidth{1.003750pt}%
\definecolor{currentstroke}{rgb}{1.000000,0.498039,0.054902}%
\pgfsetstrokecolor{currentstroke}%
\pgfsetdash{}{0pt}%
\pgfpathmoveto{\pgfqpoint{2.374912in}{3.190759in}}%
\pgfpathcurveto{\pgfqpoint{2.385962in}{3.190759in}}{\pgfqpoint{2.396561in}{3.195150in}}{\pgfqpoint{2.404375in}{3.202963in}}%
\pgfpathcurveto{\pgfqpoint{2.412188in}{3.210777in}}{\pgfqpoint{2.416579in}{3.221376in}}{\pgfqpoint{2.416579in}{3.232426in}}%
\pgfpathcurveto{\pgfqpoint{2.416579in}{3.243476in}}{\pgfqpoint{2.412188in}{3.254075in}}{\pgfqpoint{2.404375in}{3.261889in}}%
\pgfpathcurveto{\pgfqpoint{2.396561in}{3.269702in}}{\pgfqpoint{2.385962in}{3.274093in}}{\pgfqpoint{2.374912in}{3.274093in}}%
\pgfpathcurveto{\pgfqpoint{2.363862in}{3.274093in}}{\pgfqpoint{2.353263in}{3.269702in}}{\pgfqpoint{2.345449in}{3.261889in}}%
\pgfpathcurveto{\pgfqpoint{2.337636in}{3.254075in}}{\pgfqpoint{2.333245in}{3.243476in}}{\pgfqpoint{2.333245in}{3.232426in}}%
\pgfpathcurveto{\pgfqpoint{2.333245in}{3.221376in}}{\pgfqpoint{2.337636in}{3.210777in}}{\pgfqpoint{2.345449in}{3.202963in}}%
\pgfpathcurveto{\pgfqpoint{2.353263in}{3.195150in}}{\pgfqpoint{2.363862in}{3.190759in}}{\pgfqpoint{2.374912in}{3.190759in}}%
\pgfpathclose%
\pgfusepath{stroke,fill}%
\end{pgfscope}%
\begin{pgfscope}%
\pgfpathrectangle{\pgfqpoint{0.648703in}{0.548769in}}{\pgfqpoint{5.201297in}{3.102590in}}%
\pgfusepath{clip}%
\pgfsetbuttcap%
\pgfsetroundjoin%
\definecolor{currentfill}{rgb}{0.121569,0.466667,0.705882}%
\pgfsetfillcolor{currentfill}%
\pgfsetlinewidth{1.003750pt}%
\definecolor{currentstroke}{rgb}{0.121569,0.466667,0.705882}%
\pgfsetstrokecolor{currentstroke}%
\pgfsetdash{}{0pt}%
\pgfpathmoveto{\pgfqpoint{1.273766in}{0.664720in}}%
\pgfpathcurveto{\pgfqpoint{1.284816in}{0.664720in}}{\pgfqpoint{1.295415in}{0.669111in}}{\pgfqpoint{1.303229in}{0.676924in}}%
\pgfpathcurveto{\pgfqpoint{1.311042in}{0.684738in}}{\pgfqpoint{1.315432in}{0.695337in}}{\pgfqpoint{1.315432in}{0.706387in}}%
\pgfpathcurveto{\pgfqpoint{1.315432in}{0.717437in}}{\pgfqpoint{1.311042in}{0.728036in}}{\pgfqpoint{1.303229in}{0.735850in}}%
\pgfpathcurveto{\pgfqpoint{1.295415in}{0.743663in}}{\pgfqpoint{1.284816in}{0.748054in}}{\pgfqpoint{1.273766in}{0.748054in}}%
\pgfpathcurveto{\pgfqpoint{1.262716in}{0.748054in}}{\pgfqpoint{1.252117in}{0.743663in}}{\pgfqpoint{1.244303in}{0.735850in}}%
\pgfpathcurveto{\pgfqpoint{1.236489in}{0.728036in}}{\pgfqpoint{1.232099in}{0.717437in}}{\pgfqpoint{1.232099in}{0.706387in}}%
\pgfpathcurveto{\pgfqpoint{1.232099in}{0.695337in}}{\pgfqpoint{1.236489in}{0.684738in}}{\pgfqpoint{1.244303in}{0.676924in}}%
\pgfpathcurveto{\pgfqpoint{1.252117in}{0.669111in}}{\pgfqpoint{1.262716in}{0.664720in}}{\pgfqpoint{1.273766in}{0.664720in}}%
\pgfpathclose%
\pgfusepath{stroke,fill}%
\end{pgfscope}%
\begin{pgfscope}%
\pgfpathrectangle{\pgfqpoint{0.648703in}{0.548769in}}{\pgfqpoint{5.201297in}{3.102590in}}%
\pgfusepath{clip}%
\pgfsetbuttcap%
\pgfsetroundjoin%
\definecolor{currentfill}{rgb}{1.000000,0.498039,0.054902}%
\pgfsetfillcolor{currentfill}%
\pgfsetlinewidth{1.003750pt}%
\definecolor{currentstroke}{rgb}{1.000000,0.498039,0.054902}%
\pgfsetstrokecolor{currentstroke}%
\pgfsetdash{}{0pt}%
\pgfpathmoveto{\pgfqpoint{1.727179in}{3.145133in}}%
\pgfpathcurveto{\pgfqpoint{1.738229in}{3.145133in}}{\pgfqpoint{1.748828in}{3.149523in}}{\pgfqpoint{1.756642in}{3.157337in}}%
\pgfpathcurveto{\pgfqpoint{1.764455in}{3.165151in}}{\pgfqpoint{1.768846in}{3.175750in}}{\pgfqpoint{1.768846in}{3.186800in}}%
\pgfpathcurveto{\pgfqpoint{1.768846in}{3.197850in}}{\pgfqpoint{1.764455in}{3.208449in}}{\pgfqpoint{1.756642in}{3.216262in}}%
\pgfpathcurveto{\pgfqpoint{1.748828in}{3.224076in}}{\pgfqpoint{1.738229in}{3.228466in}}{\pgfqpoint{1.727179in}{3.228466in}}%
\pgfpathcurveto{\pgfqpoint{1.716129in}{3.228466in}}{\pgfqpoint{1.705530in}{3.224076in}}{\pgfqpoint{1.697716in}{3.216262in}}%
\pgfpathcurveto{\pgfqpoint{1.689903in}{3.208449in}}{\pgfqpoint{1.685512in}{3.197850in}}{\pgfqpoint{1.685512in}{3.186800in}}%
\pgfpathcurveto{\pgfqpoint{1.685512in}{3.175750in}}{\pgfqpoint{1.689903in}{3.165151in}}{\pgfqpoint{1.697716in}{3.157337in}}%
\pgfpathcurveto{\pgfqpoint{1.705530in}{3.149523in}}{\pgfqpoint{1.716129in}{3.145133in}}{\pgfqpoint{1.727179in}{3.145133in}}%
\pgfpathclose%
\pgfusepath{stroke,fill}%
\end{pgfscope}%
\begin{pgfscope}%
\pgfpathrectangle{\pgfqpoint{0.648703in}{0.548769in}}{\pgfqpoint{5.201297in}{3.102590in}}%
\pgfusepath{clip}%
\pgfsetbuttcap%
\pgfsetroundjoin%
\definecolor{currentfill}{rgb}{0.121569,0.466667,0.705882}%
\pgfsetfillcolor{currentfill}%
\pgfsetlinewidth{1.003750pt}%
\definecolor{currentstroke}{rgb}{0.121569,0.466667,0.705882}%
\pgfsetstrokecolor{currentstroke}%
\pgfsetdash{}{0pt}%
\pgfpathmoveto{\pgfqpoint{1.144219in}{0.648129in}}%
\pgfpathcurveto{\pgfqpoint{1.155269in}{0.648129in}}{\pgfqpoint{1.165868in}{0.652519in}}{\pgfqpoint{1.173682in}{0.660333in}}%
\pgfpathcurveto{\pgfqpoint{1.181496in}{0.668146in}}{\pgfqpoint{1.185886in}{0.678745in}}{\pgfqpoint{1.185886in}{0.689796in}}%
\pgfpathcurveto{\pgfqpoint{1.185886in}{0.700846in}}{\pgfqpoint{1.181496in}{0.711445in}}{\pgfqpoint{1.173682in}{0.719258in}}%
\pgfpathcurveto{\pgfqpoint{1.165868in}{0.727072in}}{\pgfqpoint{1.155269in}{0.731462in}}{\pgfqpoint{1.144219in}{0.731462in}}%
\pgfpathcurveto{\pgfqpoint{1.133169in}{0.731462in}}{\pgfqpoint{1.122570in}{0.727072in}}{\pgfqpoint{1.114756in}{0.719258in}}%
\pgfpathcurveto{\pgfqpoint{1.106943in}{0.711445in}}{\pgfqpoint{1.102553in}{0.700846in}}{\pgfqpoint{1.102553in}{0.689796in}}%
\pgfpathcurveto{\pgfqpoint{1.102553in}{0.678745in}}{\pgfqpoint{1.106943in}{0.668146in}}{\pgfqpoint{1.114756in}{0.660333in}}%
\pgfpathcurveto{\pgfqpoint{1.122570in}{0.652519in}}{\pgfqpoint{1.133169in}{0.648129in}}{\pgfqpoint{1.144219in}{0.648129in}}%
\pgfpathclose%
\pgfusepath{stroke,fill}%
\end{pgfscope}%
\begin{pgfscope}%
\pgfpathrectangle{\pgfqpoint{0.648703in}{0.548769in}}{\pgfqpoint{5.201297in}{3.102590in}}%
\pgfusepath{clip}%
\pgfsetbuttcap%
\pgfsetroundjoin%
\definecolor{currentfill}{rgb}{1.000000,0.498039,0.054902}%
\pgfsetfillcolor{currentfill}%
\pgfsetlinewidth{1.003750pt}%
\definecolor{currentstroke}{rgb}{1.000000,0.498039,0.054902}%
\pgfsetstrokecolor{currentstroke}%
\pgfsetdash{}{0pt}%
\pgfpathmoveto{\pgfqpoint{2.893099in}{3.149281in}}%
\pgfpathcurveto{\pgfqpoint{2.904149in}{3.149281in}}{\pgfqpoint{2.914748in}{3.153671in}}{\pgfqpoint{2.922561in}{3.161485in}}%
\pgfpathcurveto{\pgfqpoint{2.930375in}{3.169298in}}{\pgfqpoint{2.934765in}{3.179897in}}{\pgfqpoint{2.934765in}{3.190948in}}%
\pgfpathcurveto{\pgfqpoint{2.934765in}{3.201998in}}{\pgfqpoint{2.930375in}{3.212597in}}{\pgfqpoint{2.922561in}{3.220410in}}%
\pgfpathcurveto{\pgfqpoint{2.914748in}{3.228224in}}{\pgfqpoint{2.904149in}{3.232614in}}{\pgfqpoint{2.893099in}{3.232614in}}%
\pgfpathcurveto{\pgfqpoint{2.882048in}{3.232614in}}{\pgfqpoint{2.871449in}{3.228224in}}{\pgfqpoint{2.863636in}{3.220410in}}%
\pgfpathcurveto{\pgfqpoint{2.855822in}{3.212597in}}{\pgfqpoint{2.851432in}{3.201998in}}{\pgfqpoint{2.851432in}{3.190948in}}%
\pgfpathcurveto{\pgfqpoint{2.851432in}{3.179897in}}{\pgfqpoint{2.855822in}{3.169298in}}{\pgfqpoint{2.863636in}{3.161485in}}%
\pgfpathcurveto{\pgfqpoint{2.871449in}{3.153671in}}{\pgfqpoint{2.882048in}{3.149281in}}{\pgfqpoint{2.893099in}{3.149281in}}%
\pgfpathclose%
\pgfusepath{stroke,fill}%
\end{pgfscope}%
\begin{pgfscope}%
\pgfpathrectangle{\pgfqpoint{0.648703in}{0.548769in}}{\pgfqpoint{5.201297in}{3.102590in}}%
\pgfusepath{clip}%
\pgfsetbuttcap%
\pgfsetroundjoin%
\definecolor{currentfill}{rgb}{0.121569,0.466667,0.705882}%
\pgfsetfillcolor{currentfill}%
\pgfsetlinewidth{1.003750pt}%
\definecolor{currentstroke}{rgb}{0.121569,0.466667,0.705882}%
\pgfsetstrokecolor{currentstroke}%
\pgfsetdash{}{0pt}%
\pgfpathmoveto{\pgfqpoint{0.949899in}{3.124394in}}%
\pgfpathcurveto{\pgfqpoint{0.960949in}{3.124394in}}{\pgfqpoint{0.971548in}{3.128784in}}{\pgfqpoint{0.979362in}{3.136598in}}%
\pgfpathcurveto{\pgfqpoint{0.987176in}{3.144411in}}{\pgfqpoint{0.991566in}{3.155010in}}{\pgfqpoint{0.991566in}{3.166060in}}%
\pgfpathcurveto{\pgfqpoint{0.991566in}{3.177111in}}{\pgfqpoint{0.987176in}{3.187710in}}{\pgfqpoint{0.979362in}{3.195523in}}%
\pgfpathcurveto{\pgfqpoint{0.971548in}{3.203337in}}{\pgfqpoint{0.960949in}{3.207727in}}{\pgfqpoint{0.949899in}{3.207727in}}%
\pgfpathcurveto{\pgfqpoint{0.938849in}{3.207727in}}{\pgfqpoint{0.928250in}{3.203337in}}{\pgfqpoint{0.920437in}{3.195523in}}%
\pgfpathcurveto{\pgfqpoint{0.912623in}{3.187710in}}{\pgfqpoint{0.908233in}{3.177111in}}{\pgfqpoint{0.908233in}{3.166060in}}%
\pgfpathcurveto{\pgfqpoint{0.908233in}{3.155010in}}{\pgfqpoint{0.912623in}{3.144411in}}{\pgfqpoint{0.920437in}{3.136598in}}%
\pgfpathcurveto{\pgfqpoint{0.928250in}{3.128784in}}{\pgfqpoint{0.938849in}{3.124394in}}{\pgfqpoint{0.949899in}{3.124394in}}%
\pgfpathclose%
\pgfusepath{stroke,fill}%
\end{pgfscope}%
\begin{pgfscope}%
\pgfpathrectangle{\pgfqpoint{0.648703in}{0.548769in}}{\pgfqpoint{5.201297in}{3.102590in}}%
\pgfusepath{clip}%
\pgfsetbuttcap%
\pgfsetroundjoin%
\definecolor{currentfill}{rgb}{1.000000,0.498039,0.054902}%
\pgfsetfillcolor{currentfill}%
\pgfsetlinewidth{1.003750pt}%
\definecolor{currentstroke}{rgb}{1.000000,0.498039,0.054902}%
\pgfsetstrokecolor{currentstroke}%
\pgfsetdash{}{0pt}%
\pgfpathmoveto{\pgfqpoint{1.144219in}{3.136837in}}%
\pgfpathcurveto{\pgfqpoint{1.155269in}{3.136837in}}{\pgfqpoint{1.165868in}{3.141228in}}{\pgfqpoint{1.173682in}{3.149041in}}%
\pgfpathcurveto{\pgfqpoint{1.181496in}{3.156855in}}{\pgfqpoint{1.185886in}{3.167454in}}{\pgfqpoint{1.185886in}{3.178504in}}%
\pgfpathcurveto{\pgfqpoint{1.185886in}{3.189554in}}{\pgfqpoint{1.181496in}{3.200153in}}{\pgfqpoint{1.173682in}{3.207967in}}%
\pgfpathcurveto{\pgfqpoint{1.165868in}{3.215780in}}{\pgfqpoint{1.155269in}{3.220171in}}{\pgfqpoint{1.144219in}{3.220171in}}%
\pgfpathcurveto{\pgfqpoint{1.133169in}{3.220171in}}{\pgfqpoint{1.122570in}{3.215780in}}{\pgfqpoint{1.114756in}{3.207967in}}%
\pgfpathcurveto{\pgfqpoint{1.106943in}{3.200153in}}{\pgfqpoint{1.102553in}{3.189554in}}{\pgfqpoint{1.102553in}{3.178504in}}%
\pgfpathcurveto{\pgfqpoint{1.102553in}{3.167454in}}{\pgfqpoint{1.106943in}{3.156855in}}{\pgfqpoint{1.114756in}{3.149041in}}%
\pgfpathcurveto{\pgfqpoint{1.122570in}{3.141228in}}{\pgfqpoint{1.133169in}{3.136837in}}{\pgfqpoint{1.144219in}{3.136837in}}%
\pgfpathclose%
\pgfusepath{stroke,fill}%
\end{pgfscope}%
\begin{pgfscope}%
\pgfpathrectangle{\pgfqpoint{0.648703in}{0.548769in}}{\pgfqpoint{5.201297in}{3.102590in}}%
\pgfusepath{clip}%
\pgfsetbuttcap%
\pgfsetroundjoin%
\definecolor{currentfill}{rgb}{1.000000,0.498039,0.054902}%
\pgfsetfillcolor{currentfill}%
\pgfsetlinewidth{1.003750pt}%
\definecolor{currentstroke}{rgb}{1.000000,0.498039,0.054902}%
\pgfsetstrokecolor{currentstroke}%
\pgfsetdash{}{0pt}%
\pgfpathmoveto{\pgfqpoint{0.949899in}{3.140985in}}%
\pgfpathcurveto{\pgfqpoint{0.960949in}{3.140985in}}{\pgfqpoint{0.971548in}{3.145375in}}{\pgfqpoint{0.979362in}{3.153189in}}%
\pgfpathcurveto{\pgfqpoint{0.987176in}{3.161003in}}{\pgfqpoint{0.991566in}{3.171602in}}{\pgfqpoint{0.991566in}{3.182652in}}%
\pgfpathcurveto{\pgfqpoint{0.991566in}{3.193702in}}{\pgfqpoint{0.987176in}{3.204301in}}{\pgfqpoint{0.979362in}{3.212115in}}%
\pgfpathcurveto{\pgfqpoint{0.971548in}{3.219928in}}{\pgfqpoint{0.960949in}{3.224319in}}{\pgfqpoint{0.949899in}{3.224319in}}%
\pgfpathcurveto{\pgfqpoint{0.938849in}{3.224319in}}{\pgfqpoint{0.928250in}{3.219928in}}{\pgfqpoint{0.920437in}{3.212115in}}%
\pgfpathcurveto{\pgfqpoint{0.912623in}{3.204301in}}{\pgfqpoint{0.908233in}{3.193702in}}{\pgfqpoint{0.908233in}{3.182652in}}%
\pgfpathcurveto{\pgfqpoint{0.908233in}{3.171602in}}{\pgfqpoint{0.912623in}{3.161003in}}{\pgfqpoint{0.920437in}{3.153189in}}%
\pgfpathcurveto{\pgfqpoint{0.928250in}{3.145375in}}{\pgfqpoint{0.938849in}{3.140985in}}{\pgfqpoint{0.949899in}{3.140985in}}%
\pgfpathclose%
\pgfusepath{stroke,fill}%
\end{pgfscope}%
\begin{pgfscope}%
\pgfpathrectangle{\pgfqpoint{0.648703in}{0.548769in}}{\pgfqpoint{5.201297in}{3.102590in}}%
\pgfusepath{clip}%
\pgfsetbuttcap%
\pgfsetroundjoin%
\definecolor{currentfill}{rgb}{1.000000,0.498039,0.054902}%
\pgfsetfillcolor{currentfill}%
\pgfsetlinewidth{1.003750pt}%
\definecolor{currentstroke}{rgb}{1.000000,0.498039,0.054902}%
\pgfsetstrokecolor{currentstroke}%
\pgfsetdash{}{0pt}%
\pgfpathmoveto{\pgfqpoint{1.208993in}{3.145133in}}%
\pgfpathcurveto{\pgfqpoint{1.220043in}{3.145133in}}{\pgfqpoint{1.230642in}{3.149523in}}{\pgfqpoint{1.238455in}{3.157337in}}%
\pgfpathcurveto{\pgfqpoint{1.246269in}{3.165151in}}{\pgfqpoint{1.250659in}{3.175750in}}{\pgfqpoint{1.250659in}{3.186800in}}%
\pgfpathcurveto{\pgfqpoint{1.250659in}{3.197850in}}{\pgfqpoint{1.246269in}{3.208449in}}{\pgfqpoint{1.238455in}{3.216262in}}%
\pgfpathcurveto{\pgfqpoint{1.230642in}{3.224076in}}{\pgfqpoint{1.220043in}{3.228466in}}{\pgfqpoint{1.208993in}{3.228466in}}%
\pgfpathcurveto{\pgfqpoint{1.197942in}{3.228466in}}{\pgfqpoint{1.187343in}{3.224076in}}{\pgfqpoint{1.179530in}{3.216262in}}%
\pgfpathcurveto{\pgfqpoint{1.171716in}{3.208449in}}{\pgfqpoint{1.167326in}{3.197850in}}{\pgfqpoint{1.167326in}{3.186800in}}%
\pgfpathcurveto{\pgfqpoint{1.167326in}{3.175750in}}{\pgfqpoint{1.171716in}{3.165151in}}{\pgfqpoint{1.179530in}{3.157337in}}%
\pgfpathcurveto{\pgfqpoint{1.187343in}{3.149523in}}{\pgfqpoint{1.197942in}{3.145133in}}{\pgfqpoint{1.208993in}{3.145133in}}%
\pgfpathclose%
\pgfusepath{stroke,fill}%
\end{pgfscope}%
\begin{pgfscope}%
\pgfpathrectangle{\pgfqpoint{0.648703in}{0.548769in}}{\pgfqpoint{5.201297in}{3.102590in}}%
\pgfusepath{clip}%
\pgfsetbuttcap%
\pgfsetroundjoin%
\definecolor{currentfill}{rgb}{0.121569,0.466667,0.705882}%
\pgfsetfillcolor{currentfill}%
\pgfsetlinewidth{1.003750pt}%
\definecolor{currentstroke}{rgb}{0.121569,0.466667,0.705882}%
\pgfsetstrokecolor{currentstroke}%
\pgfsetdash{}{0pt}%
\pgfpathmoveto{\pgfqpoint{0.949899in}{3.132690in}}%
\pgfpathcurveto{\pgfqpoint{0.960949in}{3.132690in}}{\pgfqpoint{0.971548in}{3.137080in}}{\pgfqpoint{0.979362in}{3.144893in}}%
\pgfpathcurveto{\pgfqpoint{0.987176in}{3.152707in}}{\pgfqpoint{0.991566in}{3.163306in}}{\pgfqpoint{0.991566in}{3.174356in}}%
\pgfpathcurveto{\pgfqpoint{0.991566in}{3.185406in}}{\pgfqpoint{0.987176in}{3.196005in}}{\pgfqpoint{0.979362in}{3.203819in}}%
\pgfpathcurveto{\pgfqpoint{0.971548in}{3.211633in}}{\pgfqpoint{0.960949in}{3.216023in}}{\pgfqpoint{0.949899in}{3.216023in}}%
\pgfpathcurveto{\pgfqpoint{0.938849in}{3.216023in}}{\pgfqpoint{0.928250in}{3.211633in}}{\pgfqpoint{0.920437in}{3.203819in}}%
\pgfpathcurveto{\pgfqpoint{0.912623in}{3.196005in}}{\pgfqpoint{0.908233in}{3.185406in}}{\pgfqpoint{0.908233in}{3.174356in}}%
\pgfpathcurveto{\pgfqpoint{0.908233in}{3.163306in}}{\pgfqpoint{0.912623in}{3.152707in}}{\pgfqpoint{0.920437in}{3.144893in}}%
\pgfpathcurveto{\pgfqpoint{0.928250in}{3.137080in}}{\pgfqpoint{0.938849in}{3.132690in}}{\pgfqpoint{0.949899in}{3.132690in}}%
\pgfpathclose%
\pgfusepath{stroke,fill}%
\end{pgfscope}%
\begin{pgfscope}%
\pgfpathrectangle{\pgfqpoint{0.648703in}{0.548769in}}{\pgfqpoint{5.201297in}{3.102590in}}%
\pgfusepath{clip}%
\pgfsetbuttcap%
\pgfsetroundjoin%
\definecolor{currentfill}{rgb}{0.121569,0.466667,0.705882}%
\pgfsetfillcolor{currentfill}%
\pgfsetlinewidth{1.003750pt}%
\definecolor{currentstroke}{rgb}{0.121569,0.466667,0.705882}%
\pgfsetstrokecolor{currentstroke}%
\pgfsetdash{}{0pt}%
\pgfpathmoveto{\pgfqpoint{1.403312in}{0.660572in}}%
\pgfpathcurveto{\pgfqpoint{1.414363in}{0.660572in}}{\pgfqpoint{1.424962in}{0.664963in}}{\pgfqpoint{1.432775in}{0.672776in}}%
\pgfpathcurveto{\pgfqpoint{1.440589in}{0.680590in}}{\pgfqpoint{1.444979in}{0.691189in}}{\pgfqpoint{1.444979in}{0.702239in}}%
\pgfpathcurveto{\pgfqpoint{1.444979in}{0.713289in}}{\pgfqpoint{1.440589in}{0.723888in}}{\pgfqpoint{1.432775in}{0.731702in}}%
\pgfpathcurveto{\pgfqpoint{1.424962in}{0.739516in}}{\pgfqpoint{1.414363in}{0.743906in}}{\pgfqpoint{1.403312in}{0.743906in}}%
\pgfpathcurveto{\pgfqpoint{1.392262in}{0.743906in}}{\pgfqpoint{1.381663in}{0.739516in}}{\pgfqpoint{1.373850in}{0.731702in}}%
\pgfpathcurveto{\pgfqpoint{1.366036in}{0.723888in}}{\pgfqpoint{1.361646in}{0.713289in}}{\pgfqpoint{1.361646in}{0.702239in}}%
\pgfpathcurveto{\pgfqpoint{1.361646in}{0.691189in}}{\pgfqpoint{1.366036in}{0.680590in}}{\pgfqpoint{1.373850in}{0.672776in}}%
\pgfpathcurveto{\pgfqpoint{1.381663in}{0.664963in}}{\pgfqpoint{1.392262in}{0.660572in}}{\pgfqpoint{1.403312in}{0.660572in}}%
\pgfpathclose%
\pgfusepath{stroke,fill}%
\end{pgfscope}%
\begin{pgfscope}%
\pgfpathrectangle{\pgfqpoint{0.648703in}{0.548769in}}{\pgfqpoint{5.201297in}{3.102590in}}%
\pgfusepath{clip}%
\pgfsetbuttcap%
\pgfsetroundjoin%
\definecolor{currentfill}{rgb}{1.000000,0.498039,0.054902}%
\pgfsetfillcolor{currentfill}%
\pgfsetlinewidth{1.003750pt}%
\definecolor{currentstroke}{rgb}{1.000000,0.498039,0.054902}%
\pgfsetstrokecolor{currentstroke}%
\pgfsetdash{}{0pt}%
\pgfpathmoveto{\pgfqpoint{1.014673in}{3.136837in}}%
\pgfpathcurveto{\pgfqpoint{1.025723in}{3.136837in}}{\pgfqpoint{1.036322in}{3.141228in}}{\pgfqpoint{1.044135in}{3.149041in}}%
\pgfpathcurveto{\pgfqpoint{1.051949in}{3.156855in}}{\pgfqpoint{1.056339in}{3.167454in}}{\pgfqpoint{1.056339in}{3.178504in}}%
\pgfpathcurveto{\pgfqpoint{1.056339in}{3.189554in}}{\pgfqpoint{1.051949in}{3.200153in}}{\pgfqpoint{1.044135in}{3.207967in}}%
\pgfpathcurveto{\pgfqpoint{1.036322in}{3.215780in}}{\pgfqpoint{1.025723in}{3.220171in}}{\pgfqpoint{1.014673in}{3.220171in}}%
\pgfpathcurveto{\pgfqpoint{1.003622in}{3.220171in}}{\pgfqpoint{0.993023in}{3.215780in}}{\pgfqpoint{0.985210in}{3.207967in}}%
\pgfpathcurveto{\pgfqpoint{0.977396in}{3.200153in}}{\pgfqpoint{0.973006in}{3.189554in}}{\pgfqpoint{0.973006in}{3.178504in}}%
\pgfpathcurveto{\pgfqpoint{0.973006in}{3.167454in}}{\pgfqpoint{0.977396in}{3.156855in}}{\pgfqpoint{0.985210in}{3.149041in}}%
\pgfpathcurveto{\pgfqpoint{0.993023in}{3.141228in}}{\pgfqpoint{1.003622in}{3.136837in}}{\pgfqpoint{1.014673in}{3.136837in}}%
\pgfpathclose%
\pgfusepath{stroke,fill}%
\end{pgfscope}%
\begin{pgfscope}%
\pgfpathrectangle{\pgfqpoint{0.648703in}{0.548769in}}{\pgfqpoint{5.201297in}{3.102590in}}%
\pgfusepath{clip}%
\pgfsetbuttcap%
\pgfsetroundjoin%
\definecolor{currentfill}{rgb}{1.000000,0.498039,0.054902}%
\pgfsetfillcolor{currentfill}%
\pgfsetlinewidth{1.003750pt}%
\definecolor{currentstroke}{rgb}{1.000000,0.498039,0.054902}%
\pgfsetstrokecolor{currentstroke}%
\pgfsetdash{}{0pt}%
\pgfpathmoveto{\pgfqpoint{1.208993in}{3.157577in}}%
\pgfpathcurveto{\pgfqpoint{1.220043in}{3.157577in}}{\pgfqpoint{1.230642in}{3.161967in}}{\pgfqpoint{1.238455in}{3.169780in}}%
\pgfpathcurveto{\pgfqpoint{1.246269in}{3.177594in}}{\pgfqpoint{1.250659in}{3.188193in}}{\pgfqpoint{1.250659in}{3.199243in}}%
\pgfpathcurveto{\pgfqpoint{1.250659in}{3.210293in}}{\pgfqpoint{1.246269in}{3.220892in}}{\pgfqpoint{1.238455in}{3.228706in}}%
\pgfpathcurveto{\pgfqpoint{1.230642in}{3.236520in}}{\pgfqpoint{1.220043in}{3.240910in}}{\pgfqpoint{1.208993in}{3.240910in}}%
\pgfpathcurveto{\pgfqpoint{1.197942in}{3.240910in}}{\pgfqpoint{1.187343in}{3.236520in}}{\pgfqpoint{1.179530in}{3.228706in}}%
\pgfpathcurveto{\pgfqpoint{1.171716in}{3.220892in}}{\pgfqpoint{1.167326in}{3.210293in}}{\pgfqpoint{1.167326in}{3.199243in}}%
\pgfpathcurveto{\pgfqpoint{1.167326in}{3.188193in}}{\pgfqpoint{1.171716in}{3.177594in}}{\pgfqpoint{1.179530in}{3.169780in}}%
\pgfpathcurveto{\pgfqpoint{1.187343in}{3.161967in}}{\pgfqpoint{1.197942in}{3.157577in}}{\pgfqpoint{1.208993in}{3.157577in}}%
\pgfpathclose%
\pgfusepath{stroke,fill}%
\end{pgfscope}%
\begin{pgfscope}%
\pgfpathrectangle{\pgfqpoint{0.648703in}{0.548769in}}{\pgfqpoint{5.201297in}{3.102590in}}%
\pgfusepath{clip}%
\pgfsetbuttcap%
\pgfsetroundjoin%
\definecolor{currentfill}{rgb}{0.121569,0.466667,0.705882}%
\pgfsetfillcolor{currentfill}%
\pgfsetlinewidth{1.003750pt}%
\definecolor{currentstroke}{rgb}{0.121569,0.466667,0.705882}%
\pgfsetstrokecolor{currentstroke}%
\pgfsetdash{}{0pt}%
\pgfpathmoveto{\pgfqpoint{2.051046in}{0.648129in}}%
\pgfpathcurveto{\pgfqpoint{2.062096in}{0.648129in}}{\pgfqpoint{2.072695in}{0.652519in}}{\pgfqpoint{2.080508in}{0.660333in}}%
\pgfpathcurveto{\pgfqpoint{2.088322in}{0.668146in}}{\pgfqpoint{2.092712in}{0.678745in}}{\pgfqpoint{2.092712in}{0.689796in}}%
\pgfpathcurveto{\pgfqpoint{2.092712in}{0.700846in}}{\pgfqpoint{2.088322in}{0.711445in}}{\pgfqpoint{2.080508in}{0.719258in}}%
\pgfpathcurveto{\pgfqpoint{2.072695in}{0.727072in}}{\pgfqpoint{2.062096in}{0.731462in}}{\pgfqpoint{2.051046in}{0.731462in}}%
\pgfpathcurveto{\pgfqpoint{2.039995in}{0.731462in}}{\pgfqpoint{2.029396in}{0.727072in}}{\pgfqpoint{2.021583in}{0.719258in}}%
\pgfpathcurveto{\pgfqpoint{2.013769in}{0.711445in}}{\pgfqpoint{2.009379in}{0.700846in}}{\pgfqpoint{2.009379in}{0.689796in}}%
\pgfpathcurveto{\pgfqpoint{2.009379in}{0.678745in}}{\pgfqpoint{2.013769in}{0.668146in}}{\pgfqpoint{2.021583in}{0.660333in}}%
\pgfpathcurveto{\pgfqpoint{2.029396in}{0.652519in}}{\pgfqpoint{2.039995in}{0.648129in}}{\pgfqpoint{2.051046in}{0.648129in}}%
\pgfpathclose%
\pgfusepath{stroke,fill}%
\end{pgfscope}%
\begin{pgfscope}%
\pgfpathrectangle{\pgfqpoint{0.648703in}{0.548769in}}{\pgfqpoint{5.201297in}{3.102590in}}%
\pgfusepath{clip}%
\pgfsetbuttcap%
\pgfsetroundjoin%
\definecolor{currentfill}{rgb}{0.121569,0.466667,0.705882}%
\pgfsetfillcolor{currentfill}%
\pgfsetlinewidth{1.003750pt}%
\definecolor{currentstroke}{rgb}{0.121569,0.466667,0.705882}%
\pgfsetstrokecolor{currentstroke}%
\pgfsetdash{}{0pt}%
\pgfpathmoveto{\pgfqpoint{1.921499in}{0.648129in}}%
\pgfpathcurveto{\pgfqpoint{1.932549in}{0.648129in}}{\pgfqpoint{1.943148in}{0.652519in}}{\pgfqpoint{1.950962in}{0.660333in}}%
\pgfpathcurveto{\pgfqpoint{1.958775in}{0.668146in}}{\pgfqpoint{1.963166in}{0.678745in}}{\pgfqpoint{1.963166in}{0.689796in}}%
\pgfpathcurveto{\pgfqpoint{1.963166in}{0.700846in}}{\pgfqpoint{1.958775in}{0.711445in}}{\pgfqpoint{1.950962in}{0.719258in}}%
\pgfpathcurveto{\pgfqpoint{1.943148in}{0.727072in}}{\pgfqpoint{1.932549in}{0.731462in}}{\pgfqpoint{1.921499in}{0.731462in}}%
\pgfpathcurveto{\pgfqpoint{1.910449in}{0.731462in}}{\pgfqpoint{1.899850in}{0.727072in}}{\pgfqpoint{1.892036in}{0.719258in}}%
\pgfpathcurveto{\pgfqpoint{1.884223in}{0.711445in}}{\pgfqpoint{1.879832in}{0.700846in}}{\pgfqpoint{1.879832in}{0.689796in}}%
\pgfpathcurveto{\pgfqpoint{1.879832in}{0.678745in}}{\pgfqpoint{1.884223in}{0.668146in}}{\pgfqpoint{1.892036in}{0.660333in}}%
\pgfpathcurveto{\pgfqpoint{1.899850in}{0.652519in}}{\pgfqpoint{1.910449in}{0.648129in}}{\pgfqpoint{1.921499in}{0.648129in}}%
\pgfpathclose%
\pgfusepath{stroke,fill}%
\end{pgfscope}%
\begin{pgfscope}%
\pgfpathrectangle{\pgfqpoint{0.648703in}{0.548769in}}{\pgfqpoint{5.201297in}{3.102590in}}%
\pgfusepath{clip}%
\pgfsetbuttcap%
\pgfsetroundjoin%
\definecolor{currentfill}{rgb}{0.121569,0.466667,0.705882}%
\pgfsetfillcolor{currentfill}%
\pgfsetlinewidth{1.003750pt}%
\definecolor{currentstroke}{rgb}{0.121569,0.466667,0.705882}%
\pgfsetstrokecolor{currentstroke}%
\pgfsetdash{}{0pt}%
\pgfpathmoveto{\pgfqpoint{1.856726in}{3.132690in}}%
\pgfpathcurveto{\pgfqpoint{1.867776in}{3.132690in}}{\pgfqpoint{1.878375in}{3.137080in}}{\pgfqpoint{1.886188in}{3.144893in}}%
\pgfpathcurveto{\pgfqpoint{1.894002in}{3.152707in}}{\pgfqpoint{1.898392in}{3.163306in}}{\pgfqpoint{1.898392in}{3.174356in}}%
\pgfpathcurveto{\pgfqpoint{1.898392in}{3.185406in}}{\pgfqpoint{1.894002in}{3.196005in}}{\pgfqpoint{1.886188in}{3.203819in}}%
\pgfpathcurveto{\pgfqpoint{1.878375in}{3.211633in}}{\pgfqpoint{1.867776in}{3.216023in}}{\pgfqpoint{1.856726in}{3.216023in}}%
\pgfpathcurveto{\pgfqpoint{1.845675in}{3.216023in}}{\pgfqpoint{1.835076in}{3.211633in}}{\pgfqpoint{1.827263in}{3.203819in}}%
\pgfpathcurveto{\pgfqpoint{1.819449in}{3.196005in}}{\pgfqpoint{1.815059in}{3.185406in}}{\pgfqpoint{1.815059in}{3.174356in}}%
\pgfpathcurveto{\pgfqpoint{1.815059in}{3.163306in}}{\pgfqpoint{1.819449in}{3.152707in}}{\pgfqpoint{1.827263in}{3.144893in}}%
\pgfpathcurveto{\pgfqpoint{1.835076in}{3.137080in}}{\pgfqpoint{1.845675in}{3.132690in}}{\pgfqpoint{1.856726in}{3.132690in}}%
\pgfpathclose%
\pgfusepath{stroke,fill}%
\end{pgfscope}%
\begin{pgfscope}%
\pgfpathrectangle{\pgfqpoint{0.648703in}{0.548769in}}{\pgfqpoint{5.201297in}{3.102590in}}%
\pgfusepath{clip}%
\pgfsetbuttcap%
\pgfsetroundjoin%
\definecolor{currentfill}{rgb}{1.000000,0.498039,0.054902}%
\pgfsetfillcolor{currentfill}%
\pgfsetlinewidth{1.003750pt}%
\definecolor{currentstroke}{rgb}{1.000000,0.498039,0.054902}%
\pgfsetstrokecolor{currentstroke}%
\pgfsetdash{}{0pt}%
\pgfpathmoveto{\pgfqpoint{2.310139in}{3.219794in}}%
\pgfpathcurveto{\pgfqpoint{2.321189in}{3.219794in}}{\pgfqpoint{2.331788in}{3.224185in}}{\pgfqpoint{2.339602in}{3.231998in}}%
\pgfpathcurveto{\pgfqpoint{2.347415in}{3.239812in}}{\pgfqpoint{2.351805in}{3.250411in}}{\pgfqpoint{2.351805in}{3.261461in}}%
\pgfpathcurveto{\pgfqpoint{2.351805in}{3.272511in}}{\pgfqpoint{2.347415in}{3.283110in}}{\pgfqpoint{2.339602in}{3.290924in}}%
\pgfpathcurveto{\pgfqpoint{2.331788in}{3.298737in}}{\pgfqpoint{2.321189in}{3.303128in}}{\pgfqpoint{2.310139in}{3.303128in}}%
\pgfpathcurveto{\pgfqpoint{2.299089in}{3.303128in}}{\pgfqpoint{2.288490in}{3.298737in}}{\pgfqpoint{2.280676in}{3.290924in}}%
\pgfpathcurveto{\pgfqpoint{2.272862in}{3.283110in}}{\pgfqpoint{2.268472in}{3.272511in}}{\pgfqpoint{2.268472in}{3.261461in}}%
\pgfpathcurveto{\pgfqpoint{2.268472in}{3.250411in}}{\pgfqpoint{2.272862in}{3.239812in}}{\pgfqpoint{2.280676in}{3.231998in}}%
\pgfpathcurveto{\pgfqpoint{2.288490in}{3.224185in}}{\pgfqpoint{2.299089in}{3.219794in}}{\pgfqpoint{2.310139in}{3.219794in}}%
\pgfpathclose%
\pgfusepath{stroke,fill}%
\end{pgfscope}%
\begin{pgfscope}%
\pgfpathrectangle{\pgfqpoint{0.648703in}{0.548769in}}{\pgfqpoint{5.201297in}{3.102590in}}%
\pgfusepath{clip}%
\pgfsetbuttcap%
\pgfsetroundjoin%
\definecolor{currentfill}{rgb}{1.000000,0.498039,0.054902}%
\pgfsetfillcolor{currentfill}%
\pgfsetlinewidth{1.003750pt}%
\definecolor{currentstroke}{rgb}{1.000000,0.498039,0.054902}%
\pgfsetstrokecolor{currentstroke}%
\pgfsetdash{}{0pt}%
\pgfpathmoveto{\pgfqpoint{1.791952in}{3.140985in}}%
\pgfpathcurveto{\pgfqpoint{1.803002in}{3.140985in}}{\pgfqpoint{1.813601in}{3.145375in}}{\pgfqpoint{1.821415in}{3.153189in}}%
\pgfpathcurveto{\pgfqpoint{1.829229in}{3.161003in}}{\pgfqpoint{1.833619in}{3.171602in}}{\pgfqpoint{1.833619in}{3.182652in}}%
\pgfpathcurveto{\pgfqpoint{1.833619in}{3.193702in}}{\pgfqpoint{1.829229in}{3.204301in}}{\pgfqpoint{1.821415in}{3.212115in}}%
\pgfpathcurveto{\pgfqpoint{1.813601in}{3.219928in}}{\pgfqpoint{1.803002in}{3.224319in}}{\pgfqpoint{1.791952in}{3.224319in}}%
\pgfpathcurveto{\pgfqpoint{1.780902in}{3.224319in}}{\pgfqpoint{1.770303in}{3.219928in}}{\pgfqpoint{1.762490in}{3.212115in}}%
\pgfpathcurveto{\pgfqpoint{1.754676in}{3.204301in}}{\pgfqpoint{1.750286in}{3.193702in}}{\pgfqpoint{1.750286in}{3.182652in}}%
\pgfpathcurveto{\pgfqpoint{1.750286in}{3.171602in}}{\pgfqpoint{1.754676in}{3.161003in}}{\pgfqpoint{1.762490in}{3.153189in}}%
\pgfpathcurveto{\pgfqpoint{1.770303in}{3.145375in}}{\pgfqpoint{1.780902in}{3.140985in}}{\pgfqpoint{1.791952in}{3.140985in}}%
\pgfpathclose%
\pgfusepath{stroke,fill}%
\end{pgfscope}%
\begin{pgfscope}%
\pgfpathrectangle{\pgfqpoint{0.648703in}{0.548769in}}{\pgfqpoint{5.201297in}{3.102590in}}%
\pgfusepath{clip}%
\pgfsetbuttcap%
\pgfsetroundjoin%
\definecolor{currentfill}{rgb}{1.000000,0.498039,0.054902}%
\pgfsetfillcolor{currentfill}%
\pgfsetlinewidth{1.003750pt}%
\definecolor{currentstroke}{rgb}{1.000000,0.498039,0.054902}%
\pgfsetstrokecolor{currentstroke}%
\pgfsetdash{}{0pt}%
\pgfpathmoveto{\pgfqpoint{1.662406in}{3.145133in}}%
\pgfpathcurveto{\pgfqpoint{1.673456in}{3.145133in}}{\pgfqpoint{1.684055in}{3.149523in}}{\pgfqpoint{1.691868in}{3.157337in}}%
\pgfpathcurveto{\pgfqpoint{1.699682in}{3.165151in}}{\pgfqpoint{1.704072in}{3.175750in}}{\pgfqpoint{1.704072in}{3.186800in}}%
\pgfpathcurveto{\pgfqpoint{1.704072in}{3.197850in}}{\pgfqpoint{1.699682in}{3.208449in}}{\pgfqpoint{1.691868in}{3.216262in}}%
\pgfpathcurveto{\pgfqpoint{1.684055in}{3.224076in}}{\pgfqpoint{1.673456in}{3.228466in}}{\pgfqpoint{1.662406in}{3.228466in}}%
\pgfpathcurveto{\pgfqpoint{1.651356in}{3.228466in}}{\pgfqpoint{1.640757in}{3.224076in}}{\pgfqpoint{1.632943in}{3.216262in}}%
\pgfpathcurveto{\pgfqpoint{1.625129in}{3.208449in}}{\pgfqpoint{1.620739in}{3.197850in}}{\pgfqpoint{1.620739in}{3.186800in}}%
\pgfpathcurveto{\pgfqpoint{1.620739in}{3.175750in}}{\pgfqpoint{1.625129in}{3.165151in}}{\pgfqpoint{1.632943in}{3.157337in}}%
\pgfpathcurveto{\pgfqpoint{1.640757in}{3.149523in}}{\pgfqpoint{1.651356in}{3.145133in}}{\pgfqpoint{1.662406in}{3.145133in}}%
\pgfpathclose%
\pgfusepath{stroke,fill}%
\end{pgfscope}%
\begin{pgfscope}%
\pgfpathrectangle{\pgfqpoint{0.648703in}{0.548769in}}{\pgfqpoint{5.201297in}{3.102590in}}%
\pgfusepath{clip}%
\pgfsetbuttcap%
\pgfsetroundjoin%
\definecolor{currentfill}{rgb}{1.000000,0.498039,0.054902}%
\pgfsetfillcolor{currentfill}%
\pgfsetlinewidth{1.003750pt}%
\definecolor{currentstroke}{rgb}{1.000000,0.498039,0.054902}%
\pgfsetstrokecolor{currentstroke}%
\pgfsetdash{}{0pt}%
\pgfpathmoveto{\pgfqpoint{1.597632in}{3.136837in}}%
\pgfpathcurveto{\pgfqpoint{1.608682in}{3.136837in}}{\pgfqpoint{1.619282in}{3.141228in}}{\pgfqpoint{1.627095in}{3.149041in}}%
\pgfpathcurveto{\pgfqpoint{1.634909in}{3.156855in}}{\pgfqpoint{1.639299in}{3.167454in}}{\pgfqpoint{1.639299in}{3.178504in}}%
\pgfpathcurveto{\pgfqpoint{1.639299in}{3.189554in}}{\pgfqpoint{1.634909in}{3.200153in}}{\pgfqpoint{1.627095in}{3.207967in}}%
\pgfpathcurveto{\pgfqpoint{1.619282in}{3.215780in}}{\pgfqpoint{1.608682in}{3.220171in}}{\pgfqpoint{1.597632in}{3.220171in}}%
\pgfpathcurveto{\pgfqpoint{1.586582in}{3.220171in}}{\pgfqpoint{1.575983in}{3.215780in}}{\pgfqpoint{1.568170in}{3.207967in}}%
\pgfpathcurveto{\pgfqpoint{1.560356in}{3.200153in}}{\pgfqpoint{1.555966in}{3.189554in}}{\pgfqpoint{1.555966in}{3.178504in}}%
\pgfpathcurveto{\pgfqpoint{1.555966in}{3.167454in}}{\pgfqpoint{1.560356in}{3.156855in}}{\pgfqpoint{1.568170in}{3.149041in}}%
\pgfpathcurveto{\pgfqpoint{1.575983in}{3.141228in}}{\pgfqpoint{1.586582in}{3.136837in}}{\pgfqpoint{1.597632in}{3.136837in}}%
\pgfpathclose%
\pgfusepath{stroke,fill}%
\end{pgfscope}%
\begin{pgfscope}%
\pgfpathrectangle{\pgfqpoint{0.648703in}{0.548769in}}{\pgfqpoint{5.201297in}{3.102590in}}%
\pgfusepath{clip}%
\pgfsetbuttcap%
\pgfsetroundjoin%
\definecolor{currentfill}{rgb}{0.121569,0.466667,0.705882}%
\pgfsetfillcolor{currentfill}%
\pgfsetlinewidth{1.003750pt}%
\definecolor{currentstroke}{rgb}{0.121569,0.466667,0.705882}%
\pgfsetstrokecolor{currentstroke}%
\pgfsetdash{}{0pt}%
\pgfpathmoveto{\pgfqpoint{1.532859in}{0.648129in}}%
\pgfpathcurveto{\pgfqpoint{1.543909in}{0.648129in}}{\pgfqpoint{1.554508in}{0.652519in}}{\pgfqpoint{1.562322in}{0.660333in}}%
\pgfpathcurveto{\pgfqpoint{1.570135in}{0.668146in}}{\pgfqpoint{1.574526in}{0.678745in}}{\pgfqpoint{1.574526in}{0.689796in}}%
\pgfpathcurveto{\pgfqpoint{1.574526in}{0.700846in}}{\pgfqpoint{1.570135in}{0.711445in}}{\pgfqpoint{1.562322in}{0.719258in}}%
\pgfpathcurveto{\pgfqpoint{1.554508in}{0.727072in}}{\pgfqpoint{1.543909in}{0.731462in}}{\pgfqpoint{1.532859in}{0.731462in}}%
\pgfpathcurveto{\pgfqpoint{1.521809in}{0.731462in}}{\pgfqpoint{1.511210in}{0.727072in}}{\pgfqpoint{1.503396in}{0.719258in}}%
\pgfpathcurveto{\pgfqpoint{1.495583in}{0.711445in}}{\pgfqpoint{1.491192in}{0.700846in}}{\pgfqpoint{1.491192in}{0.689796in}}%
\pgfpathcurveto{\pgfqpoint{1.491192in}{0.678745in}}{\pgfqpoint{1.495583in}{0.668146in}}{\pgfqpoint{1.503396in}{0.660333in}}%
\pgfpathcurveto{\pgfqpoint{1.511210in}{0.652519in}}{\pgfqpoint{1.521809in}{0.648129in}}{\pgfqpoint{1.532859in}{0.648129in}}%
\pgfpathclose%
\pgfusepath{stroke,fill}%
\end{pgfscope}%
\begin{pgfscope}%
\pgfpathrectangle{\pgfqpoint{0.648703in}{0.548769in}}{\pgfqpoint{5.201297in}{3.102590in}}%
\pgfusepath{clip}%
\pgfsetbuttcap%
\pgfsetroundjoin%
\definecolor{currentfill}{rgb}{0.121569,0.466667,0.705882}%
\pgfsetfillcolor{currentfill}%
\pgfsetlinewidth{1.003750pt}%
\definecolor{currentstroke}{rgb}{0.121569,0.466667,0.705882}%
\pgfsetstrokecolor{currentstroke}%
\pgfsetdash{}{0pt}%
\pgfpathmoveto{\pgfqpoint{1.597632in}{3.132690in}}%
\pgfpathcurveto{\pgfqpoint{1.608682in}{3.132690in}}{\pgfqpoint{1.619282in}{3.137080in}}{\pgfqpoint{1.627095in}{3.144893in}}%
\pgfpathcurveto{\pgfqpoint{1.634909in}{3.152707in}}{\pgfqpoint{1.639299in}{3.163306in}}{\pgfqpoint{1.639299in}{3.174356in}}%
\pgfpathcurveto{\pgfqpoint{1.639299in}{3.185406in}}{\pgfqpoint{1.634909in}{3.196005in}}{\pgfqpoint{1.627095in}{3.203819in}}%
\pgfpathcurveto{\pgfqpoint{1.619282in}{3.211633in}}{\pgfqpoint{1.608682in}{3.216023in}}{\pgfqpoint{1.597632in}{3.216023in}}%
\pgfpathcurveto{\pgfqpoint{1.586582in}{3.216023in}}{\pgfqpoint{1.575983in}{3.211633in}}{\pgfqpoint{1.568170in}{3.203819in}}%
\pgfpathcurveto{\pgfqpoint{1.560356in}{3.196005in}}{\pgfqpoint{1.555966in}{3.185406in}}{\pgfqpoint{1.555966in}{3.174356in}}%
\pgfpathcurveto{\pgfqpoint{1.555966in}{3.163306in}}{\pgfqpoint{1.560356in}{3.152707in}}{\pgfqpoint{1.568170in}{3.144893in}}%
\pgfpathcurveto{\pgfqpoint{1.575983in}{3.137080in}}{\pgfqpoint{1.586582in}{3.132690in}}{\pgfqpoint{1.597632in}{3.132690in}}%
\pgfpathclose%
\pgfusepath{stroke,fill}%
\end{pgfscope}%
\begin{pgfscope}%
\pgfpathrectangle{\pgfqpoint{0.648703in}{0.548769in}}{\pgfqpoint{5.201297in}{3.102590in}}%
\pgfusepath{clip}%
\pgfsetbuttcap%
\pgfsetroundjoin%
\definecolor{currentfill}{rgb}{1.000000,0.498039,0.054902}%
\pgfsetfillcolor{currentfill}%
\pgfsetlinewidth{1.003750pt}%
\definecolor{currentstroke}{rgb}{1.000000,0.498039,0.054902}%
\pgfsetstrokecolor{currentstroke}%
\pgfsetdash{}{0pt}%
\pgfpathmoveto{\pgfqpoint{1.208993in}{3.145133in}}%
\pgfpathcurveto{\pgfqpoint{1.220043in}{3.145133in}}{\pgfqpoint{1.230642in}{3.149523in}}{\pgfqpoint{1.238455in}{3.157337in}}%
\pgfpathcurveto{\pgfqpoint{1.246269in}{3.165151in}}{\pgfqpoint{1.250659in}{3.175750in}}{\pgfqpoint{1.250659in}{3.186800in}}%
\pgfpathcurveto{\pgfqpoint{1.250659in}{3.197850in}}{\pgfqpoint{1.246269in}{3.208449in}}{\pgfqpoint{1.238455in}{3.216262in}}%
\pgfpathcurveto{\pgfqpoint{1.230642in}{3.224076in}}{\pgfqpoint{1.220043in}{3.228466in}}{\pgfqpoint{1.208993in}{3.228466in}}%
\pgfpathcurveto{\pgfqpoint{1.197942in}{3.228466in}}{\pgfqpoint{1.187343in}{3.224076in}}{\pgfqpoint{1.179530in}{3.216262in}}%
\pgfpathcurveto{\pgfqpoint{1.171716in}{3.208449in}}{\pgfqpoint{1.167326in}{3.197850in}}{\pgfqpoint{1.167326in}{3.186800in}}%
\pgfpathcurveto{\pgfqpoint{1.167326in}{3.175750in}}{\pgfqpoint{1.171716in}{3.165151in}}{\pgfqpoint{1.179530in}{3.157337in}}%
\pgfpathcurveto{\pgfqpoint{1.187343in}{3.149523in}}{\pgfqpoint{1.197942in}{3.145133in}}{\pgfqpoint{1.208993in}{3.145133in}}%
\pgfpathclose%
\pgfusepath{stroke,fill}%
\end{pgfscope}%
\begin{pgfscope}%
\pgfpathrectangle{\pgfqpoint{0.648703in}{0.548769in}}{\pgfqpoint{5.201297in}{3.102590in}}%
\pgfusepath{clip}%
\pgfsetbuttcap%
\pgfsetroundjoin%
\definecolor{currentfill}{rgb}{0.121569,0.466667,0.705882}%
\pgfsetfillcolor{currentfill}%
\pgfsetlinewidth{1.003750pt}%
\definecolor{currentstroke}{rgb}{0.121569,0.466667,0.705882}%
\pgfsetstrokecolor{currentstroke}%
\pgfsetdash{}{0pt}%
\pgfpathmoveto{\pgfqpoint{0.949899in}{3.132690in}}%
\pgfpathcurveto{\pgfqpoint{0.960949in}{3.132690in}}{\pgfqpoint{0.971548in}{3.137080in}}{\pgfqpoint{0.979362in}{3.144893in}}%
\pgfpathcurveto{\pgfqpoint{0.987176in}{3.152707in}}{\pgfqpoint{0.991566in}{3.163306in}}{\pgfqpoint{0.991566in}{3.174356in}}%
\pgfpathcurveto{\pgfqpoint{0.991566in}{3.185406in}}{\pgfqpoint{0.987176in}{3.196005in}}{\pgfqpoint{0.979362in}{3.203819in}}%
\pgfpathcurveto{\pgfqpoint{0.971548in}{3.211633in}}{\pgfqpoint{0.960949in}{3.216023in}}{\pgfqpoint{0.949899in}{3.216023in}}%
\pgfpathcurveto{\pgfqpoint{0.938849in}{3.216023in}}{\pgfqpoint{0.928250in}{3.211633in}}{\pgfqpoint{0.920437in}{3.203819in}}%
\pgfpathcurveto{\pgfqpoint{0.912623in}{3.196005in}}{\pgfqpoint{0.908233in}{3.185406in}}{\pgfqpoint{0.908233in}{3.174356in}}%
\pgfpathcurveto{\pgfqpoint{0.908233in}{3.163306in}}{\pgfqpoint{0.912623in}{3.152707in}}{\pgfqpoint{0.920437in}{3.144893in}}%
\pgfpathcurveto{\pgfqpoint{0.928250in}{3.137080in}}{\pgfqpoint{0.938849in}{3.132690in}}{\pgfqpoint{0.949899in}{3.132690in}}%
\pgfpathclose%
\pgfusepath{stroke,fill}%
\end{pgfscope}%
\begin{pgfscope}%
\pgfpathrectangle{\pgfqpoint{0.648703in}{0.548769in}}{\pgfqpoint{5.201297in}{3.102590in}}%
\pgfusepath{clip}%
\pgfsetbuttcap%
\pgfsetroundjoin%
\definecolor{currentfill}{rgb}{1.000000,0.498039,0.054902}%
\pgfsetfillcolor{currentfill}%
\pgfsetlinewidth{1.003750pt}%
\definecolor{currentstroke}{rgb}{1.000000,0.498039,0.054902}%
\pgfsetstrokecolor{currentstroke}%
\pgfsetdash{}{0pt}%
\pgfpathmoveto{\pgfqpoint{1.338539in}{3.157577in}}%
\pgfpathcurveto{\pgfqpoint{1.349589in}{3.157577in}}{\pgfqpoint{1.360188in}{3.161967in}}{\pgfqpoint{1.368002in}{3.169780in}}%
\pgfpathcurveto{\pgfqpoint{1.375816in}{3.177594in}}{\pgfqpoint{1.380206in}{3.188193in}}{\pgfqpoint{1.380206in}{3.199243in}}%
\pgfpathcurveto{\pgfqpoint{1.380206in}{3.210293in}}{\pgfqpoint{1.375816in}{3.220892in}}{\pgfqpoint{1.368002in}{3.228706in}}%
\pgfpathcurveto{\pgfqpoint{1.360188in}{3.236520in}}{\pgfqpoint{1.349589in}{3.240910in}}{\pgfqpoint{1.338539in}{3.240910in}}%
\pgfpathcurveto{\pgfqpoint{1.327489in}{3.240910in}}{\pgfqpoint{1.316890in}{3.236520in}}{\pgfqpoint{1.309076in}{3.228706in}}%
\pgfpathcurveto{\pgfqpoint{1.301263in}{3.220892in}}{\pgfqpoint{1.296872in}{3.210293in}}{\pgfqpoint{1.296872in}{3.199243in}}%
\pgfpathcurveto{\pgfqpoint{1.296872in}{3.188193in}}{\pgfqpoint{1.301263in}{3.177594in}}{\pgfqpoint{1.309076in}{3.169780in}}%
\pgfpathcurveto{\pgfqpoint{1.316890in}{3.161967in}}{\pgfqpoint{1.327489in}{3.157577in}}{\pgfqpoint{1.338539in}{3.157577in}}%
\pgfpathclose%
\pgfusepath{stroke,fill}%
\end{pgfscope}%
\begin{pgfscope}%
\pgfpathrectangle{\pgfqpoint{0.648703in}{0.548769in}}{\pgfqpoint{5.201297in}{3.102590in}}%
\pgfusepath{clip}%
\pgfsetbuttcap%
\pgfsetroundjoin%
\definecolor{currentfill}{rgb}{1.000000,0.498039,0.054902}%
\pgfsetfillcolor{currentfill}%
\pgfsetlinewidth{1.003750pt}%
\definecolor{currentstroke}{rgb}{1.000000,0.498039,0.054902}%
\pgfsetstrokecolor{currentstroke}%
\pgfsetdash{}{0pt}%
\pgfpathmoveto{\pgfqpoint{1.144219in}{3.145133in}}%
\pgfpathcurveto{\pgfqpoint{1.155269in}{3.145133in}}{\pgfqpoint{1.165868in}{3.149523in}}{\pgfqpoint{1.173682in}{3.157337in}}%
\pgfpathcurveto{\pgfqpoint{1.181496in}{3.165151in}}{\pgfqpoint{1.185886in}{3.175750in}}{\pgfqpoint{1.185886in}{3.186800in}}%
\pgfpathcurveto{\pgfqpoint{1.185886in}{3.197850in}}{\pgfqpoint{1.181496in}{3.208449in}}{\pgfqpoint{1.173682in}{3.216262in}}%
\pgfpathcurveto{\pgfqpoint{1.165868in}{3.224076in}}{\pgfqpoint{1.155269in}{3.228466in}}{\pgfqpoint{1.144219in}{3.228466in}}%
\pgfpathcurveto{\pgfqpoint{1.133169in}{3.228466in}}{\pgfqpoint{1.122570in}{3.224076in}}{\pgfqpoint{1.114756in}{3.216262in}}%
\pgfpathcurveto{\pgfqpoint{1.106943in}{3.208449in}}{\pgfqpoint{1.102553in}{3.197850in}}{\pgfqpoint{1.102553in}{3.186800in}}%
\pgfpathcurveto{\pgfqpoint{1.102553in}{3.175750in}}{\pgfqpoint{1.106943in}{3.165151in}}{\pgfqpoint{1.114756in}{3.157337in}}%
\pgfpathcurveto{\pgfqpoint{1.122570in}{3.149523in}}{\pgfqpoint{1.133169in}{3.145133in}}{\pgfqpoint{1.144219in}{3.145133in}}%
\pgfpathclose%
\pgfusepath{stroke,fill}%
\end{pgfscope}%
\begin{pgfscope}%
\pgfpathrectangle{\pgfqpoint{0.648703in}{0.548769in}}{\pgfqpoint{5.201297in}{3.102590in}}%
\pgfusepath{clip}%
\pgfsetbuttcap%
\pgfsetroundjoin%
\definecolor{currentfill}{rgb}{1.000000,0.498039,0.054902}%
\pgfsetfillcolor{currentfill}%
\pgfsetlinewidth{1.003750pt}%
\definecolor{currentstroke}{rgb}{1.000000,0.498039,0.054902}%
\pgfsetstrokecolor{currentstroke}%
\pgfsetdash{}{0pt}%
\pgfpathmoveto{\pgfqpoint{1.856726in}{3.190759in}}%
\pgfpathcurveto{\pgfqpoint{1.867776in}{3.190759in}}{\pgfqpoint{1.878375in}{3.195150in}}{\pgfqpoint{1.886188in}{3.202963in}}%
\pgfpathcurveto{\pgfqpoint{1.894002in}{3.210777in}}{\pgfqpoint{1.898392in}{3.221376in}}{\pgfqpoint{1.898392in}{3.232426in}}%
\pgfpathcurveto{\pgfqpoint{1.898392in}{3.243476in}}{\pgfqpoint{1.894002in}{3.254075in}}{\pgfqpoint{1.886188in}{3.261889in}}%
\pgfpathcurveto{\pgfqpoint{1.878375in}{3.269702in}}{\pgfqpoint{1.867776in}{3.274093in}}{\pgfqpoint{1.856726in}{3.274093in}}%
\pgfpathcurveto{\pgfqpoint{1.845675in}{3.274093in}}{\pgfqpoint{1.835076in}{3.269702in}}{\pgfqpoint{1.827263in}{3.261889in}}%
\pgfpathcurveto{\pgfqpoint{1.819449in}{3.254075in}}{\pgfqpoint{1.815059in}{3.243476in}}{\pgfqpoint{1.815059in}{3.232426in}}%
\pgfpathcurveto{\pgfqpoint{1.815059in}{3.221376in}}{\pgfqpoint{1.819449in}{3.210777in}}{\pgfqpoint{1.827263in}{3.202963in}}%
\pgfpathcurveto{\pgfqpoint{1.835076in}{3.195150in}}{\pgfqpoint{1.845675in}{3.190759in}}{\pgfqpoint{1.856726in}{3.190759in}}%
\pgfpathclose%
\pgfusepath{stroke,fill}%
\end{pgfscope}%
\begin{pgfscope}%
\pgfpathrectangle{\pgfqpoint{0.648703in}{0.548769in}}{\pgfqpoint{5.201297in}{3.102590in}}%
\pgfusepath{clip}%
\pgfsetbuttcap%
\pgfsetroundjoin%
\definecolor{currentfill}{rgb}{1.000000,0.498039,0.054902}%
\pgfsetfillcolor{currentfill}%
\pgfsetlinewidth{1.003750pt}%
\definecolor{currentstroke}{rgb}{1.000000,0.498039,0.054902}%
\pgfsetstrokecolor{currentstroke}%
\pgfsetdash{}{0pt}%
\pgfpathmoveto{\pgfqpoint{1.273766in}{3.207351in}}%
\pgfpathcurveto{\pgfqpoint{1.284816in}{3.207351in}}{\pgfqpoint{1.295415in}{3.211741in}}{\pgfqpoint{1.303229in}{3.219555in}}%
\pgfpathcurveto{\pgfqpoint{1.311042in}{3.227368in}}{\pgfqpoint{1.315432in}{3.237967in}}{\pgfqpoint{1.315432in}{3.249017in}}%
\pgfpathcurveto{\pgfqpoint{1.315432in}{3.260068in}}{\pgfqpoint{1.311042in}{3.270667in}}{\pgfqpoint{1.303229in}{3.278480in}}%
\pgfpathcurveto{\pgfqpoint{1.295415in}{3.286294in}}{\pgfqpoint{1.284816in}{3.290684in}}{\pgfqpoint{1.273766in}{3.290684in}}%
\pgfpathcurveto{\pgfqpoint{1.262716in}{3.290684in}}{\pgfqpoint{1.252117in}{3.286294in}}{\pgfqpoint{1.244303in}{3.278480in}}%
\pgfpathcurveto{\pgfqpoint{1.236489in}{3.270667in}}{\pgfqpoint{1.232099in}{3.260068in}}{\pgfqpoint{1.232099in}{3.249017in}}%
\pgfpathcurveto{\pgfqpoint{1.232099in}{3.237967in}}{\pgfqpoint{1.236489in}{3.227368in}}{\pgfqpoint{1.244303in}{3.219555in}}%
\pgfpathcurveto{\pgfqpoint{1.252117in}{3.211741in}}{\pgfqpoint{1.262716in}{3.207351in}}{\pgfqpoint{1.273766in}{3.207351in}}%
\pgfpathclose%
\pgfusepath{stroke,fill}%
\end{pgfscope}%
\begin{pgfscope}%
\pgfpathrectangle{\pgfqpoint{0.648703in}{0.548769in}}{\pgfqpoint{5.201297in}{3.102590in}}%
\pgfusepath{clip}%
\pgfsetbuttcap%
\pgfsetroundjoin%
\definecolor{currentfill}{rgb}{0.121569,0.466667,0.705882}%
\pgfsetfillcolor{currentfill}%
\pgfsetlinewidth{1.003750pt}%
\definecolor{currentstroke}{rgb}{0.121569,0.466667,0.705882}%
\pgfsetstrokecolor{currentstroke}%
\pgfsetdash{}{0pt}%
\pgfpathmoveto{\pgfqpoint{0.949899in}{0.648129in}}%
\pgfpathcurveto{\pgfqpoint{0.960949in}{0.648129in}}{\pgfqpoint{0.971548in}{0.652519in}}{\pgfqpoint{0.979362in}{0.660333in}}%
\pgfpathcurveto{\pgfqpoint{0.987176in}{0.668146in}}{\pgfqpoint{0.991566in}{0.678745in}}{\pgfqpoint{0.991566in}{0.689796in}}%
\pgfpathcurveto{\pgfqpoint{0.991566in}{0.700846in}}{\pgfqpoint{0.987176in}{0.711445in}}{\pgfqpoint{0.979362in}{0.719258in}}%
\pgfpathcurveto{\pgfqpoint{0.971548in}{0.727072in}}{\pgfqpoint{0.960949in}{0.731462in}}{\pgfqpoint{0.949899in}{0.731462in}}%
\pgfpathcurveto{\pgfqpoint{0.938849in}{0.731462in}}{\pgfqpoint{0.928250in}{0.727072in}}{\pgfqpoint{0.920437in}{0.719258in}}%
\pgfpathcurveto{\pgfqpoint{0.912623in}{0.711445in}}{\pgfqpoint{0.908233in}{0.700846in}}{\pgfqpoint{0.908233in}{0.689796in}}%
\pgfpathcurveto{\pgfqpoint{0.908233in}{0.678745in}}{\pgfqpoint{0.912623in}{0.668146in}}{\pgfqpoint{0.920437in}{0.660333in}}%
\pgfpathcurveto{\pgfqpoint{0.928250in}{0.652519in}}{\pgfqpoint{0.938849in}{0.648129in}}{\pgfqpoint{0.949899in}{0.648129in}}%
\pgfpathclose%
\pgfusepath{stroke,fill}%
\end{pgfscope}%
\begin{pgfscope}%
\pgfpathrectangle{\pgfqpoint{0.648703in}{0.548769in}}{\pgfqpoint{5.201297in}{3.102590in}}%
\pgfusepath{clip}%
\pgfsetbuttcap%
\pgfsetroundjoin%
\definecolor{currentfill}{rgb}{0.121569,0.466667,0.705882}%
\pgfsetfillcolor{currentfill}%
\pgfsetlinewidth{1.003750pt}%
\definecolor{currentstroke}{rgb}{0.121569,0.466667,0.705882}%
\pgfsetstrokecolor{currentstroke}%
\pgfsetdash{}{0pt}%
\pgfpathmoveto{\pgfqpoint{1.273766in}{0.656425in}}%
\pgfpathcurveto{\pgfqpoint{1.284816in}{0.656425in}}{\pgfqpoint{1.295415in}{0.660815in}}{\pgfqpoint{1.303229in}{0.668629in}}%
\pgfpathcurveto{\pgfqpoint{1.311042in}{0.676442in}}{\pgfqpoint{1.315432in}{0.687041in}}{\pgfqpoint{1.315432in}{0.698091in}}%
\pgfpathcurveto{\pgfqpoint{1.315432in}{0.709141in}}{\pgfqpoint{1.311042in}{0.719740in}}{\pgfqpoint{1.303229in}{0.727554in}}%
\pgfpathcurveto{\pgfqpoint{1.295415in}{0.735368in}}{\pgfqpoint{1.284816in}{0.739758in}}{\pgfqpoint{1.273766in}{0.739758in}}%
\pgfpathcurveto{\pgfqpoint{1.262716in}{0.739758in}}{\pgfqpoint{1.252117in}{0.735368in}}{\pgfqpoint{1.244303in}{0.727554in}}%
\pgfpathcurveto{\pgfqpoint{1.236489in}{0.719740in}}{\pgfqpoint{1.232099in}{0.709141in}}{\pgfqpoint{1.232099in}{0.698091in}}%
\pgfpathcurveto{\pgfqpoint{1.232099in}{0.687041in}}{\pgfqpoint{1.236489in}{0.676442in}}{\pgfqpoint{1.244303in}{0.668629in}}%
\pgfpathcurveto{\pgfqpoint{1.252117in}{0.660815in}}{\pgfqpoint{1.262716in}{0.656425in}}{\pgfqpoint{1.273766in}{0.656425in}}%
\pgfpathclose%
\pgfusepath{stroke,fill}%
\end{pgfscope}%
\begin{pgfscope}%
\pgfpathrectangle{\pgfqpoint{0.648703in}{0.548769in}}{\pgfqpoint{5.201297in}{3.102590in}}%
\pgfusepath{clip}%
\pgfsetbuttcap%
\pgfsetroundjoin%
\definecolor{currentfill}{rgb}{0.121569,0.466667,0.705882}%
\pgfsetfillcolor{currentfill}%
\pgfsetlinewidth{1.003750pt}%
\definecolor{currentstroke}{rgb}{0.121569,0.466667,0.705882}%
\pgfsetstrokecolor{currentstroke}%
\pgfsetdash{}{0pt}%
\pgfpathmoveto{\pgfqpoint{1.144219in}{3.099507in}}%
\pgfpathcurveto{\pgfqpoint{1.155269in}{3.099507in}}{\pgfqpoint{1.165868in}{3.103897in}}{\pgfqpoint{1.173682in}{3.111711in}}%
\pgfpathcurveto{\pgfqpoint{1.181496in}{3.119524in}}{\pgfqpoint{1.185886in}{3.130123in}}{\pgfqpoint{1.185886in}{3.141173in}}%
\pgfpathcurveto{\pgfqpoint{1.185886in}{3.152224in}}{\pgfqpoint{1.181496in}{3.162823in}}{\pgfqpoint{1.173682in}{3.170636in}}%
\pgfpathcurveto{\pgfqpoint{1.165868in}{3.178450in}}{\pgfqpoint{1.155269in}{3.182840in}}{\pgfqpoint{1.144219in}{3.182840in}}%
\pgfpathcurveto{\pgfqpoint{1.133169in}{3.182840in}}{\pgfqpoint{1.122570in}{3.178450in}}{\pgfqpoint{1.114756in}{3.170636in}}%
\pgfpathcurveto{\pgfqpoint{1.106943in}{3.162823in}}{\pgfqpoint{1.102553in}{3.152224in}}{\pgfqpoint{1.102553in}{3.141173in}}%
\pgfpathcurveto{\pgfqpoint{1.102553in}{3.130123in}}{\pgfqpoint{1.106943in}{3.119524in}}{\pgfqpoint{1.114756in}{3.111711in}}%
\pgfpathcurveto{\pgfqpoint{1.122570in}{3.103897in}}{\pgfqpoint{1.133169in}{3.099507in}}{\pgfqpoint{1.144219in}{3.099507in}}%
\pgfpathclose%
\pgfusepath{stroke,fill}%
\end{pgfscope}%
\begin{pgfscope}%
\pgfpathrectangle{\pgfqpoint{0.648703in}{0.548769in}}{\pgfqpoint{5.201297in}{3.102590in}}%
\pgfusepath{clip}%
\pgfsetbuttcap%
\pgfsetroundjoin%
\definecolor{currentfill}{rgb}{1.000000,0.498039,0.054902}%
\pgfsetfillcolor{currentfill}%
\pgfsetlinewidth{1.003750pt}%
\definecolor{currentstroke}{rgb}{1.000000,0.498039,0.054902}%
\pgfsetstrokecolor{currentstroke}%
\pgfsetdash{}{0pt}%
\pgfpathmoveto{\pgfqpoint{1.208993in}{3.136837in}}%
\pgfpathcurveto{\pgfqpoint{1.220043in}{3.136837in}}{\pgfqpoint{1.230642in}{3.141228in}}{\pgfqpoint{1.238455in}{3.149041in}}%
\pgfpathcurveto{\pgfqpoint{1.246269in}{3.156855in}}{\pgfqpoint{1.250659in}{3.167454in}}{\pgfqpoint{1.250659in}{3.178504in}}%
\pgfpathcurveto{\pgfqpoint{1.250659in}{3.189554in}}{\pgfqpoint{1.246269in}{3.200153in}}{\pgfqpoint{1.238455in}{3.207967in}}%
\pgfpathcurveto{\pgfqpoint{1.230642in}{3.215780in}}{\pgfqpoint{1.220043in}{3.220171in}}{\pgfqpoint{1.208993in}{3.220171in}}%
\pgfpathcurveto{\pgfqpoint{1.197942in}{3.220171in}}{\pgfqpoint{1.187343in}{3.215780in}}{\pgfqpoint{1.179530in}{3.207967in}}%
\pgfpathcurveto{\pgfqpoint{1.171716in}{3.200153in}}{\pgfqpoint{1.167326in}{3.189554in}}{\pgfqpoint{1.167326in}{3.178504in}}%
\pgfpathcurveto{\pgfqpoint{1.167326in}{3.167454in}}{\pgfqpoint{1.171716in}{3.156855in}}{\pgfqpoint{1.179530in}{3.149041in}}%
\pgfpathcurveto{\pgfqpoint{1.187343in}{3.141228in}}{\pgfqpoint{1.197942in}{3.136837in}}{\pgfqpoint{1.208993in}{3.136837in}}%
\pgfpathclose%
\pgfusepath{stroke,fill}%
\end{pgfscope}%
\begin{pgfscope}%
\pgfpathrectangle{\pgfqpoint{0.648703in}{0.548769in}}{\pgfqpoint{5.201297in}{3.102590in}}%
\pgfusepath{clip}%
\pgfsetbuttcap%
\pgfsetroundjoin%
\definecolor{currentfill}{rgb}{0.121569,0.466667,0.705882}%
\pgfsetfillcolor{currentfill}%
\pgfsetlinewidth{1.003750pt}%
\definecolor{currentstroke}{rgb}{0.121569,0.466667,0.705882}%
\pgfsetstrokecolor{currentstroke}%
\pgfsetdash{}{0pt}%
\pgfpathmoveto{\pgfqpoint{1.208993in}{0.648129in}}%
\pgfpathcurveto{\pgfqpoint{1.220043in}{0.648129in}}{\pgfqpoint{1.230642in}{0.652519in}}{\pgfqpoint{1.238455in}{0.660333in}}%
\pgfpathcurveto{\pgfqpoint{1.246269in}{0.668146in}}{\pgfqpoint{1.250659in}{0.678745in}}{\pgfqpoint{1.250659in}{0.689796in}}%
\pgfpathcurveto{\pgfqpoint{1.250659in}{0.700846in}}{\pgfqpoint{1.246269in}{0.711445in}}{\pgfqpoint{1.238455in}{0.719258in}}%
\pgfpathcurveto{\pgfqpoint{1.230642in}{0.727072in}}{\pgfqpoint{1.220043in}{0.731462in}}{\pgfqpoint{1.208993in}{0.731462in}}%
\pgfpathcurveto{\pgfqpoint{1.197942in}{0.731462in}}{\pgfqpoint{1.187343in}{0.727072in}}{\pgfqpoint{1.179530in}{0.719258in}}%
\pgfpathcurveto{\pgfqpoint{1.171716in}{0.711445in}}{\pgfqpoint{1.167326in}{0.700846in}}{\pgfqpoint{1.167326in}{0.689796in}}%
\pgfpathcurveto{\pgfqpoint{1.167326in}{0.678745in}}{\pgfqpoint{1.171716in}{0.668146in}}{\pgfqpoint{1.179530in}{0.660333in}}%
\pgfpathcurveto{\pgfqpoint{1.187343in}{0.652519in}}{\pgfqpoint{1.197942in}{0.648129in}}{\pgfqpoint{1.208993in}{0.648129in}}%
\pgfpathclose%
\pgfusepath{stroke,fill}%
\end{pgfscope}%
\begin{pgfscope}%
\pgfpathrectangle{\pgfqpoint{0.648703in}{0.548769in}}{\pgfqpoint{5.201297in}{3.102590in}}%
\pgfusepath{clip}%
\pgfsetbuttcap%
\pgfsetroundjoin%
\definecolor{currentfill}{rgb}{0.121569,0.466667,0.705882}%
\pgfsetfillcolor{currentfill}%
\pgfsetlinewidth{1.003750pt}%
\definecolor{currentstroke}{rgb}{0.121569,0.466667,0.705882}%
\pgfsetstrokecolor{currentstroke}%
\pgfsetdash{}{0pt}%
\pgfpathmoveto{\pgfqpoint{0.949899in}{0.648129in}}%
\pgfpathcurveto{\pgfqpoint{0.960949in}{0.648129in}}{\pgfqpoint{0.971548in}{0.652519in}}{\pgfqpoint{0.979362in}{0.660333in}}%
\pgfpathcurveto{\pgfqpoint{0.987176in}{0.668146in}}{\pgfqpoint{0.991566in}{0.678745in}}{\pgfqpoint{0.991566in}{0.689796in}}%
\pgfpathcurveto{\pgfqpoint{0.991566in}{0.700846in}}{\pgfqpoint{0.987176in}{0.711445in}}{\pgfqpoint{0.979362in}{0.719258in}}%
\pgfpathcurveto{\pgfqpoint{0.971548in}{0.727072in}}{\pgfqpoint{0.960949in}{0.731462in}}{\pgfqpoint{0.949899in}{0.731462in}}%
\pgfpathcurveto{\pgfqpoint{0.938849in}{0.731462in}}{\pgfqpoint{0.928250in}{0.727072in}}{\pgfqpoint{0.920437in}{0.719258in}}%
\pgfpathcurveto{\pgfqpoint{0.912623in}{0.711445in}}{\pgfqpoint{0.908233in}{0.700846in}}{\pgfqpoint{0.908233in}{0.689796in}}%
\pgfpathcurveto{\pgfqpoint{0.908233in}{0.678745in}}{\pgfqpoint{0.912623in}{0.668146in}}{\pgfqpoint{0.920437in}{0.660333in}}%
\pgfpathcurveto{\pgfqpoint{0.928250in}{0.652519in}}{\pgfqpoint{0.938849in}{0.648129in}}{\pgfqpoint{0.949899in}{0.648129in}}%
\pgfpathclose%
\pgfusepath{stroke,fill}%
\end{pgfscope}%
\begin{pgfscope}%
\pgfpathrectangle{\pgfqpoint{0.648703in}{0.548769in}}{\pgfqpoint{5.201297in}{3.102590in}}%
\pgfusepath{clip}%
\pgfsetbuttcap%
\pgfsetroundjoin%
\definecolor{currentfill}{rgb}{0.839216,0.152941,0.156863}%
\pgfsetfillcolor{currentfill}%
\pgfsetlinewidth{1.003750pt}%
\definecolor{currentstroke}{rgb}{0.839216,0.152941,0.156863}%
\pgfsetstrokecolor{currentstroke}%
\pgfsetdash{}{0pt}%
\pgfpathmoveto{\pgfqpoint{0.885126in}{3.410595in}}%
\pgfpathcurveto{\pgfqpoint{0.896176in}{3.410595in}}{\pgfqpoint{0.906775in}{3.414986in}}{\pgfqpoint{0.914589in}{3.422799in}}%
\pgfpathcurveto{\pgfqpoint{0.922402in}{3.430613in}}{\pgfqpoint{0.926793in}{3.441212in}}{\pgfqpoint{0.926793in}{3.452262in}}%
\pgfpathcurveto{\pgfqpoint{0.926793in}{3.463312in}}{\pgfqpoint{0.922402in}{3.473911in}}{\pgfqpoint{0.914589in}{3.481725in}}%
\pgfpathcurveto{\pgfqpoint{0.906775in}{3.489538in}}{\pgfqpoint{0.896176in}{3.493929in}}{\pgfqpoint{0.885126in}{3.493929in}}%
\pgfpathcurveto{\pgfqpoint{0.874076in}{3.493929in}}{\pgfqpoint{0.863477in}{3.489538in}}{\pgfqpoint{0.855663in}{3.481725in}}%
\pgfpathcurveto{\pgfqpoint{0.847850in}{3.473911in}}{\pgfqpoint{0.843459in}{3.463312in}}{\pgfqpoint{0.843459in}{3.452262in}}%
\pgfpathcurveto{\pgfqpoint{0.843459in}{3.441212in}}{\pgfqpoint{0.847850in}{3.430613in}}{\pgfqpoint{0.855663in}{3.422799in}}%
\pgfpathcurveto{\pgfqpoint{0.863477in}{3.414986in}}{\pgfqpoint{0.874076in}{3.410595in}}{\pgfqpoint{0.885126in}{3.410595in}}%
\pgfpathclose%
\pgfusepath{stroke,fill}%
\end{pgfscope}%
\begin{pgfscope}%
\pgfpathrectangle{\pgfqpoint{0.648703in}{0.548769in}}{\pgfqpoint{5.201297in}{3.102590in}}%
\pgfusepath{clip}%
\pgfsetbuttcap%
\pgfsetroundjoin%
\definecolor{currentfill}{rgb}{0.121569,0.466667,0.705882}%
\pgfsetfillcolor{currentfill}%
\pgfsetlinewidth{1.003750pt}%
\definecolor{currentstroke}{rgb}{0.121569,0.466667,0.705882}%
\pgfsetstrokecolor{currentstroke}%
\pgfsetdash{}{0pt}%
\pgfpathmoveto{\pgfqpoint{2.957872in}{3.128542in}}%
\pgfpathcurveto{\pgfqpoint{2.968922in}{3.128542in}}{\pgfqpoint{2.979521in}{3.132932in}}{\pgfqpoint{2.987335in}{3.140746in}}%
\pgfpathcurveto{\pgfqpoint{2.995148in}{3.148559in}}{\pgfqpoint{2.999538in}{3.159158in}}{\pgfqpoint{2.999538in}{3.170208in}}%
\pgfpathcurveto{\pgfqpoint{2.999538in}{3.181258in}}{\pgfqpoint{2.995148in}{3.191857in}}{\pgfqpoint{2.987335in}{3.199671in}}%
\pgfpathcurveto{\pgfqpoint{2.979521in}{3.207485in}}{\pgfqpoint{2.968922in}{3.211875in}}{\pgfqpoint{2.957872in}{3.211875in}}%
\pgfpathcurveto{\pgfqpoint{2.946822in}{3.211875in}}{\pgfqpoint{2.936223in}{3.207485in}}{\pgfqpoint{2.928409in}{3.199671in}}%
\pgfpathcurveto{\pgfqpoint{2.920595in}{3.191857in}}{\pgfqpoint{2.916205in}{3.181258in}}{\pgfqpoint{2.916205in}{3.170208in}}%
\pgfpathcurveto{\pgfqpoint{2.916205in}{3.159158in}}{\pgfqpoint{2.920595in}{3.148559in}}{\pgfqpoint{2.928409in}{3.140746in}}%
\pgfpathcurveto{\pgfqpoint{2.936223in}{3.132932in}}{\pgfqpoint{2.946822in}{3.128542in}}{\pgfqpoint{2.957872in}{3.128542in}}%
\pgfpathclose%
\pgfusepath{stroke,fill}%
\end{pgfscope}%
\begin{pgfscope}%
\pgfpathrectangle{\pgfqpoint{0.648703in}{0.548769in}}{\pgfqpoint{5.201297in}{3.102590in}}%
\pgfusepath{clip}%
\pgfsetbuttcap%
\pgfsetroundjoin%
\definecolor{currentfill}{rgb}{1.000000,0.498039,0.054902}%
\pgfsetfillcolor{currentfill}%
\pgfsetlinewidth{1.003750pt}%
\definecolor{currentstroke}{rgb}{1.000000,0.498039,0.054902}%
\pgfsetstrokecolor{currentstroke}%
\pgfsetdash{}{0pt}%
\pgfpathmoveto{\pgfqpoint{1.727179in}{3.315195in}}%
\pgfpathcurveto{\pgfqpoint{1.738229in}{3.315195in}}{\pgfqpoint{1.748828in}{3.319585in}}{\pgfqpoint{1.756642in}{3.327399in}}%
\pgfpathcurveto{\pgfqpoint{1.764455in}{3.335212in}}{\pgfqpoint{1.768846in}{3.345811in}}{\pgfqpoint{1.768846in}{3.356861in}}%
\pgfpathcurveto{\pgfqpoint{1.768846in}{3.367912in}}{\pgfqpoint{1.764455in}{3.378511in}}{\pgfqpoint{1.756642in}{3.386324in}}%
\pgfpathcurveto{\pgfqpoint{1.748828in}{3.394138in}}{\pgfqpoint{1.738229in}{3.398528in}}{\pgfqpoint{1.727179in}{3.398528in}}%
\pgfpathcurveto{\pgfqpoint{1.716129in}{3.398528in}}{\pgfqpoint{1.705530in}{3.394138in}}{\pgfqpoint{1.697716in}{3.386324in}}%
\pgfpathcurveto{\pgfqpoint{1.689903in}{3.378511in}}{\pgfqpoint{1.685512in}{3.367912in}}{\pgfqpoint{1.685512in}{3.356861in}}%
\pgfpathcurveto{\pgfqpoint{1.685512in}{3.345811in}}{\pgfqpoint{1.689903in}{3.335212in}}{\pgfqpoint{1.697716in}{3.327399in}}%
\pgfpathcurveto{\pgfqpoint{1.705530in}{3.319585in}}{\pgfqpoint{1.716129in}{3.315195in}}{\pgfqpoint{1.727179in}{3.315195in}}%
\pgfpathclose%
\pgfusepath{stroke,fill}%
\end{pgfscope}%
\begin{pgfscope}%
\pgfpathrectangle{\pgfqpoint{0.648703in}{0.548769in}}{\pgfqpoint{5.201297in}{3.102590in}}%
\pgfusepath{clip}%
\pgfsetbuttcap%
\pgfsetroundjoin%
\definecolor{currentfill}{rgb}{1.000000,0.498039,0.054902}%
\pgfsetfillcolor{currentfill}%
\pgfsetlinewidth{1.003750pt}%
\definecolor{currentstroke}{rgb}{1.000000,0.498039,0.054902}%
\pgfsetstrokecolor{currentstroke}%
\pgfsetdash{}{0pt}%
\pgfpathmoveto{\pgfqpoint{1.208993in}{3.136837in}}%
\pgfpathcurveto{\pgfqpoint{1.220043in}{3.136837in}}{\pgfqpoint{1.230642in}{3.141228in}}{\pgfqpoint{1.238455in}{3.149041in}}%
\pgfpathcurveto{\pgfqpoint{1.246269in}{3.156855in}}{\pgfqpoint{1.250659in}{3.167454in}}{\pgfqpoint{1.250659in}{3.178504in}}%
\pgfpathcurveto{\pgfqpoint{1.250659in}{3.189554in}}{\pgfqpoint{1.246269in}{3.200153in}}{\pgfqpoint{1.238455in}{3.207967in}}%
\pgfpathcurveto{\pgfqpoint{1.230642in}{3.215780in}}{\pgfqpoint{1.220043in}{3.220171in}}{\pgfqpoint{1.208993in}{3.220171in}}%
\pgfpathcurveto{\pgfqpoint{1.197942in}{3.220171in}}{\pgfqpoint{1.187343in}{3.215780in}}{\pgfqpoint{1.179530in}{3.207967in}}%
\pgfpathcurveto{\pgfqpoint{1.171716in}{3.200153in}}{\pgfqpoint{1.167326in}{3.189554in}}{\pgfqpoint{1.167326in}{3.178504in}}%
\pgfpathcurveto{\pgfqpoint{1.167326in}{3.167454in}}{\pgfqpoint{1.171716in}{3.156855in}}{\pgfqpoint{1.179530in}{3.149041in}}%
\pgfpathcurveto{\pgfqpoint{1.187343in}{3.141228in}}{\pgfqpoint{1.197942in}{3.136837in}}{\pgfqpoint{1.208993in}{3.136837in}}%
\pgfpathclose%
\pgfusepath{stroke,fill}%
\end{pgfscope}%
\begin{pgfscope}%
\pgfpathrectangle{\pgfqpoint{0.648703in}{0.548769in}}{\pgfqpoint{5.201297in}{3.102590in}}%
\pgfusepath{clip}%
\pgfsetbuttcap%
\pgfsetroundjoin%
\definecolor{currentfill}{rgb}{1.000000,0.498039,0.054902}%
\pgfsetfillcolor{currentfill}%
\pgfsetlinewidth{1.003750pt}%
\definecolor{currentstroke}{rgb}{1.000000,0.498039,0.054902}%
\pgfsetstrokecolor{currentstroke}%
\pgfsetdash{}{0pt}%
\pgfpathmoveto{\pgfqpoint{3.864698in}{3.136837in}}%
\pgfpathcurveto{\pgfqpoint{3.875748in}{3.136837in}}{\pgfqpoint{3.886347in}{3.141228in}}{\pgfqpoint{3.894161in}{3.149041in}}%
\pgfpathcurveto{\pgfqpoint{3.901975in}{3.156855in}}{\pgfqpoint{3.906365in}{3.167454in}}{\pgfqpoint{3.906365in}{3.178504in}}%
\pgfpathcurveto{\pgfqpoint{3.906365in}{3.189554in}}{\pgfqpoint{3.901975in}{3.200153in}}{\pgfqpoint{3.894161in}{3.207967in}}%
\pgfpathcurveto{\pgfqpoint{3.886347in}{3.215780in}}{\pgfqpoint{3.875748in}{3.220171in}}{\pgfqpoint{3.864698in}{3.220171in}}%
\pgfpathcurveto{\pgfqpoint{3.853648in}{3.220171in}}{\pgfqpoint{3.843049in}{3.215780in}}{\pgfqpoint{3.835235in}{3.207967in}}%
\pgfpathcurveto{\pgfqpoint{3.827422in}{3.200153in}}{\pgfqpoint{3.823031in}{3.189554in}}{\pgfqpoint{3.823031in}{3.178504in}}%
\pgfpathcurveto{\pgfqpoint{3.823031in}{3.167454in}}{\pgfqpoint{3.827422in}{3.156855in}}{\pgfqpoint{3.835235in}{3.149041in}}%
\pgfpathcurveto{\pgfqpoint{3.843049in}{3.141228in}}{\pgfqpoint{3.853648in}{3.136837in}}{\pgfqpoint{3.864698in}{3.136837in}}%
\pgfpathclose%
\pgfusepath{stroke,fill}%
\end{pgfscope}%
\begin{pgfscope}%
\pgfpathrectangle{\pgfqpoint{0.648703in}{0.548769in}}{\pgfqpoint{5.201297in}{3.102590in}}%
\pgfusepath{clip}%
\pgfsetbuttcap%
\pgfsetroundjoin%
\definecolor{currentfill}{rgb}{1.000000,0.498039,0.054902}%
\pgfsetfillcolor{currentfill}%
\pgfsetlinewidth{1.003750pt}%
\definecolor{currentstroke}{rgb}{1.000000,0.498039,0.054902}%
\pgfsetstrokecolor{currentstroke}%
\pgfsetdash{}{0pt}%
\pgfpathmoveto{\pgfqpoint{3.799925in}{3.140985in}}%
\pgfpathcurveto{\pgfqpoint{3.810975in}{3.140985in}}{\pgfqpoint{3.821574in}{3.145375in}}{\pgfqpoint{3.829388in}{3.153189in}}%
\pgfpathcurveto{\pgfqpoint{3.837201in}{3.161003in}}{\pgfqpoint{3.841591in}{3.171602in}}{\pgfqpoint{3.841591in}{3.182652in}}%
\pgfpathcurveto{\pgfqpoint{3.841591in}{3.193702in}}{\pgfqpoint{3.837201in}{3.204301in}}{\pgfqpoint{3.829388in}{3.212115in}}%
\pgfpathcurveto{\pgfqpoint{3.821574in}{3.219928in}}{\pgfqpoint{3.810975in}{3.224319in}}{\pgfqpoint{3.799925in}{3.224319in}}%
\pgfpathcurveto{\pgfqpoint{3.788875in}{3.224319in}}{\pgfqpoint{3.778276in}{3.219928in}}{\pgfqpoint{3.770462in}{3.212115in}}%
\pgfpathcurveto{\pgfqpoint{3.762648in}{3.204301in}}{\pgfqpoint{3.758258in}{3.193702in}}{\pgfqpoint{3.758258in}{3.182652in}}%
\pgfpathcurveto{\pgfqpoint{3.758258in}{3.171602in}}{\pgfqpoint{3.762648in}{3.161003in}}{\pgfqpoint{3.770462in}{3.153189in}}%
\pgfpathcurveto{\pgfqpoint{3.778276in}{3.145375in}}{\pgfqpoint{3.788875in}{3.140985in}}{\pgfqpoint{3.799925in}{3.140985in}}%
\pgfpathclose%
\pgfusepath{stroke,fill}%
\end{pgfscope}%
\begin{pgfscope}%
\pgfpathrectangle{\pgfqpoint{0.648703in}{0.548769in}}{\pgfqpoint{5.201297in}{3.102590in}}%
\pgfusepath{clip}%
\pgfsetbuttcap%
\pgfsetroundjoin%
\definecolor{currentfill}{rgb}{1.000000,0.498039,0.054902}%
\pgfsetfillcolor{currentfill}%
\pgfsetlinewidth{1.003750pt}%
\definecolor{currentstroke}{rgb}{1.000000,0.498039,0.054902}%
\pgfsetstrokecolor{currentstroke}%
\pgfsetdash{}{0pt}%
\pgfpathmoveto{\pgfqpoint{1.403312in}{3.136837in}}%
\pgfpathcurveto{\pgfqpoint{1.414363in}{3.136837in}}{\pgfqpoint{1.424962in}{3.141228in}}{\pgfqpoint{1.432775in}{3.149041in}}%
\pgfpathcurveto{\pgfqpoint{1.440589in}{3.156855in}}{\pgfqpoint{1.444979in}{3.167454in}}{\pgfqpoint{1.444979in}{3.178504in}}%
\pgfpathcurveto{\pgfqpoint{1.444979in}{3.189554in}}{\pgfqpoint{1.440589in}{3.200153in}}{\pgfqpoint{1.432775in}{3.207967in}}%
\pgfpathcurveto{\pgfqpoint{1.424962in}{3.215780in}}{\pgfqpoint{1.414363in}{3.220171in}}{\pgfqpoint{1.403312in}{3.220171in}}%
\pgfpathcurveto{\pgfqpoint{1.392262in}{3.220171in}}{\pgfqpoint{1.381663in}{3.215780in}}{\pgfqpoint{1.373850in}{3.207967in}}%
\pgfpathcurveto{\pgfqpoint{1.366036in}{3.200153in}}{\pgfqpoint{1.361646in}{3.189554in}}{\pgfqpoint{1.361646in}{3.178504in}}%
\pgfpathcurveto{\pgfqpoint{1.361646in}{3.167454in}}{\pgfqpoint{1.366036in}{3.156855in}}{\pgfqpoint{1.373850in}{3.149041in}}%
\pgfpathcurveto{\pgfqpoint{1.381663in}{3.141228in}}{\pgfqpoint{1.392262in}{3.136837in}}{\pgfqpoint{1.403312in}{3.136837in}}%
\pgfpathclose%
\pgfusepath{stroke,fill}%
\end{pgfscope}%
\begin{pgfscope}%
\pgfpathrectangle{\pgfqpoint{0.648703in}{0.548769in}}{\pgfqpoint{5.201297in}{3.102590in}}%
\pgfusepath{clip}%
\pgfsetbuttcap%
\pgfsetroundjoin%
\definecolor{currentfill}{rgb}{1.000000,0.498039,0.054902}%
\pgfsetfillcolor{currentfill}%
\pgfsetlinewidth{1.003750pt}%
\definecolor{currentstroke}{rgb}{1.000000,0.498039,0.054902}%
\pgfsetstrokecolor{currentstroke}%
\pgfsetdash{}{0pt}%
\pgfpathmoveto{\pgfqpoint{3.022645in}{3.356673in}}%
\pgfpathcurveto{\pgfqpoint{3.033695in}{3.356673in}}{\pgfqpoint{3.044294in}{3.361064in}}{\pgfqpoint{3.052108in}{3.368877in}}%
\pgfpathcurveto{\pgfqpoint{3.059922in}{3.376691in}}{\pgfqpoint{3.064312in}{3.387290in}}{\pgfqpoint{3.064312in}{3.398340in}}%
\pgfpathcurveto{\pgfqpoint{3.064312in}{3.409390in}}{\pgfqpoint{3.059922in}{3.419989in}}{\pgfqpoint{3.052108in}{3.427803in}}%
\pgfpathcurveto{\pgfqpoint{3.044294in}{3.435616in}}{\pgfqpoint{3.033695in}{3.440007in}}{\pgfqpoint{3.022645in}{3.440007in}}%
\pgfpathcurveto{\pgfqpoint{3.011595in}{3.440007in}}{\pgfqpoint{3.000996in}{3.435616in}}{\pgfqpoint{2.993182in}{3.427803in}}%
\pgfpathcurveto{\pgfqpoint{2.985369in}{3.419989in}}{\pgfqpoint{2.980978in}{3.409390in}}{\pgfqpoint{2.980978in}{3.398340in}}%
\pgfpathcurveto{\pgfqpoint{2.980978in}{3.387290in}}{\pgfqpoint{2.985369in}{3.376691in}}{\pgfqpoint{2.993182in}{3.368877in}}%
\pgfpathcurveto{\pgfqpoint{3.000996in}{3.361064in}}{\pgfqpoint{3.011595in}{3.356673in}}{\pgfqpoint{3.022645in}{3.356673in}}%
\pgfpathclose%
\pgfusepath{stroke,fill}%
\end{pgfscope}%
\begin{pgfscope}%
\pgfpathrectangle{\pgfqpoint{0.648703in}{0.548769in}}{\pgfqpoint{5.201297in}{3.102590in}}%
\pgfusepath{clip}%
\pgfsetbuttcap%
\pgfsetroundjoin%
\definecolor{currentfill}{rgb}{1.000000,0.498039,0.054902}%
\pgfsetfillcolor{currentfill}%
\pgfsetlinewidth{1.003750pt}%
\definecolor{currentstroke}{rgb}{1.000000,0.498039,0.054902}%
\pgfsetstrokecolor{currentstroke}%
\pgfsetdash{}{0pt}%
\pgfpathmoveto{\pgfqpoint{2.634005in}{3.219794in}}%
\pgfpathcurveto{\pgfqpoint{2.645055in}{3.219794in}}{\pgfqpoint{2.655654in}{3.224185in}}{\pgfqpoint{2.663468in}{3.231998in}}%
\pgfpathcurveto{\pgfqpoint{2.671282in}{3.239812in}}{\pgfqpoint{2.675672in}{3.250411in}}{\pgfqpoint{2.675672in}{3.261461in}}%
\pgfpathcurveto{\pgfqpoint{2.675672in}{3.272511in}}{\pgfqpoint{2.671282in}{3.283110in}}{\pgfqpoint{2.663468in}{3.290924in}}%
\pgfpathcurveto{\pgfqpoint{2.655654in}{3.298737in}}{\pgfqpoint{2.645055in}{3.303128in}}{\pgfqpoint{2.634005in}{3.303128in}}%
\pgfpathcurveto{\pgfqpoint{2.622955in}{3.303128in}}{\pgfqpoint{2.612356in}{3.298737in}}{\pgfqpoint{2.604543in}{3.290924in}}%
\pgfpathcurveto{\pgfqpoint{2.596729in}{3.283110in}}{\pgfqpoint{2.592339in}{3.272511in}}{\pgfqpoint{2.592339in}{3.261461in}}%
\pgfpathcurveto{\pgfqpoint{2.592339in}{3.250411in}}{\pgfqpoint{2.596729in}{3.239812in}}{\pgfqpoint{2.604543in}{3.231998in}}%
\pgfpathcurveto{\pgfqpoint{2.612356in}{3.224185in}}{\pgfqpoint{2.622955in}{3.219794in}}{\pgfqpoint{2.634005in}{3.219794in}}%
\pgfpathclose%
\pgfusepath{stroke,fill}%
\end{pgfscope}%
\begin{pgfscope}%
\pgfpathrectangle{\pgfqpoint{0.648703in}{0.548769in}}{\pgfqpoint{5.201297in}{3.102590in}}%
\pgfusepath{clip}%
\pgfsetbuttcap%
\pgfsetroundjoin%
\definecolor{currentfill}{rgb}{0.121569,0.466667,0.705882}%
\pgfsetfillcolor{currentfill}%
\pgfsetlinewidth{1.003750pt}%
\definecolor{currentstroke}{rgb}{0.121569,0.466667,0.705882}%
\pgfsetstrokecolor{currentstroke}%
\pgfsetdash{}{0pt}%
\pgfpathmoveto{\pgfqpoint{1.014673in}{3.120246in}}%
\pgfpathcurveto{\pgfqpoint{1.025723in}{3.120246in}}{\pgfqpoint{1.036322in}{3.124636in}}{\pgfqpoint{1.044135in}{3.132450in}}%
\pgfpathcurveto{\pgfqpoint{1.051949in}{3.140263in}}{\pgfqpoint{1.056339in}{3.150862in}}{\pgfqpoint{1.056339in}{3.161913in}}%
\pgfpathcurveto{\pgfqpoint{1.056339in}{3.172963in}}{\pgfqpoint{1.051949in}{3.183562in}}{\pgfqpoint{1.044135in}{3.191375in}}%
\pgfpathcurveto{\pgfqpoint{1.036322in}{3.199189in}}{\pgfqpoint{1.025723in}{3.203579in}}{\pgfqpoint{1.014673in}{3.203579in}}%
\pgfpathcurveto{\pgfqpoint{1.003622in}{3.203579in}}{\pgfqpoint{0.993023in}{3.199189in}}{\pgfqpoint{0.985210in}{3.191375in}}%
\pgfpathcurveto{\pgfqpoint{0.977396in}{3.183562in}}{\pgfqpoint{0.973006in}{3.172963in}}{\pgfqpoint{0.973006in}{3.161913in}}%
\pgfpathcurveto{\pgfqpoint{0.973006in}{3.150862in}}{\pgfqpoint{0.977396in}{3.140263in}}{\pgfqpoint{0.985210in}{3.132450in}}%
\pgfpathcurveto{\pgfqpoint{0.993023in}{3.124636in}}{\pgfqpoint{1.003622in}{3.120246in}}{\pgfqpoint{1.014673in}{3.120246in}}%
\pgfpathclose%
\pgfusepath{stroke,fill}%
\end{pgfscope}%
\begin{pgfscope}%
\pgfpathrectangle{\pgfqpoint{0.648703in}{0.548769in}}{\pgfqpoint{5.201297in}{3.102590in}}%
\pgfusepath{clip}%
\pgfsetbuttcap%
\pgfsetroundjoin%
\definecolor{currentfill}{rgb}{1.000000,0.498039,0.054902}%
\pgfsetfillcolor{currentfill}%
\pgfsetlinewidth{1.003750pt}%
\definecolor{currentstroke}{rgb}{1.000000,0.498039,0.054902}%
\pgfsetstrokecolor{currentstroke}%
\pgfsetdash{}{0pt}%
\pgfpathmoveto{\pgfqpoint{2.439685in}{3.140985in}}%
\pgfpathcurveto{\pgfqpoint{2.450735in}{3.140985in}}{\pgfqpoint{2.461335in}{3.145375in}}{\pgfqpoint{2.469148in}{3.153189in}}%
\pgfpathcurveto{\pgfqpoint{2.476962in}{3.161003in}}{\pgfqpoint{2.481352in}{3.171602in}}{\pgfqpoint{2.481352in}{3.182652in}}%
\pgfpathcurveto{\pgfqpoint{2.481352in}{3.193702in}}{\pgfqpoint{2.476962in}{3.204301in}}{\pgfqpoint{2.469148in}{3.212115in}}%
\pgfpathcurveto{\pgfqpoint{2.461335in}{3.219928in}}{\pgfqpoint{2.450735in}{3.224319in}}{\pgfqpoint{2.439685in}{3.224319in}}%
\pgfpathcurveto{\pgfqpoint{2.428635in}{3.224319in}}{\pgfqpoint{2.418036in}{3.219928in}}{\pgfqpoint{2.410223in}{3.212115in}}%
\pgfpathcurveto{\pgfqpoint{2.402409in}{3.204301in}}{\pgfqpoint{2.398019in}{3.193702in}}{\pgfqpoint{2.398019in}{3.182652in}}%
\pgfpathcurveto{\pgfqpoint{2.398019in}{3.171602in}}{\pgfqpoint{2.402409in}{3.161003in}}{\pgfqpoint{2.410223in}{3.153189in}}%
\pgfpathcurveto{\pgfqpoint{2.418036in}{3.145375in}}{\pgfqpoint{2.428635in}{3.140985in}}{\pgfqpoint{2.439685in}{3.140985in}}%
\pgfpathclose%
\pgfusepath{stroke,fill}%
\end{pgfscope}%
\begin{pgfscope}%
\pgfpathrectangle{\pgfqpoint{0.648703in}{0.548769in}}{\pgfqpoint{5.201297in}{3.102590in}}%
\pgfusepath{clip}%
\pgfsetbuttcap%
\pgfsetroundjoin%
\definecolor{currentfill}{rgb}{1.000000,0.498039,0.054902}%
\pgfsetfillcolor{currentfill}%
\pgfsetlinewidth{1.003750pt}%
\definecolor{currentstroke}{rgb}{1.000000,0.498039,0.054902}%
\pgfsetstrokecolor{currentstroke}%
\pgfsetdash{}{0pt}%
\pgfpathmoveto{\pgfqpoint{1.986272in}{3.145133in}}%
\pgfpathcurveto{\pgfqpoint{1.997322in}{3.145133in}}{\pgfqpoint{2.007921in}{3.149523in}}{\pgfqpoint{2.015735in}{3.157337in}}%
\pgfpathcurveto{\pgfqpoint{2.023549in}{3.165151in}}{\pgfqpoint{2.027939in}{3.175750in}}{\pgfqpoint{2.027939in}{3.186800in}}%
\pgfpathcurveto{\pgfqpoint{2.027939in}{3.197850in}}{\pgfqpoint{2.023549in}{3.208449in}}{\pgfqpoint{2.015735in}{3.216262in}}%
\pgfpathcurveto{\pgfqpoint{2.007921in}{3.224076in}}{\pgfqpoint{1.997322in}{3.228466in}}{\pgfqpoint{1.986272in}{3.228466in}}%
\pgfpathcurveto{\pgfqpoint{1.975222in}{3.228466in}}{\pgfqpoint{1.964623in}{3.224076in}}{\pgfqpoint{1.956809in}{3.216262in}}%
\pgfpathcurveto{\pgfqpoint{1.948996in}{3.208449in}}{\pgfqpoint{1.944606in}{3.197850in}}{\pgfqpoint{1.944606in}{3.186800in}}%
\pgfpathcurveto{\pgfqpoint{1.944606in}{3.175750in}}{\pgfqpoint{1.948996in}{3.165151in}}{\pgfqpoint{1.956809in}{3.157337in}}%
\pgfpathcurveto{\pgfqpoint{1.964623in}{3.149523in}}{\pgfqpoint{1.975222in}{3.145133in}}{\pgfqpoint{1.986272in}{3.145133in}}%
\pgfpathclose%
\pgfusepath{stroke,fill}%
\end{pgfscope}%
\begin{pgfscope}%
\pgfpathrectangle{\pgfqpoint{0.648703in}{0.548769in}}{\pgfqpoint{5.201297in}{3.102590in}}%
\pgfusepath{clip}%
\pgfsetbuttcap%
\pgfsetroundjoin%
\definecolor{currentfill}{rgb}{1.000000,0.498039,0.054902}%
\pgfsetfillcolor{currentfill}%
\pgfsetlinewidth{1.003750pt}%
\definecolor{currentstroke}{rgb}{1.000000,0.498039,0.054902}%
\pgfsetstrokecolor{currentstroke}%
\pgfsetdash{}{0pt}%
\pgfpathmoveto{\pgfqpoint{1.921499in}{3.136837in}}%
\pgfpathcurveto{\pgfqpoint{1.932549in}{3.136837in}}{\pgfqpoint{1.943148in}{3.141228in}}{\pgfqpoint{1.950962in}{3.149041in}}%
\pgfpathcurveto{\pgfqpoint{1.958775in}{3.156855in}}{\pgfqpoint{1.963166in}{3.167454in}}{\pgfqpoint{1.963166in}{3.178504in}}%
\pgfpathcurveto{\pgfqpoint{1.963166in}{3.189554in}}{\pgfqpoint{1.958775in}{3.200153in}}{\pgfqpoint{1.950962in}{3.207967in}}%
\pgfpathcurveto{\pgfqpoint{1.943148in}{3.215780in}}{\pgfqpoint{1.932549in}{3.220171in}}{\pgfqpoint{1.921499in}{3.220171in}}%
\pgfpathcurveto{\pgfqpoint{1.910449in}{3.220171in}}{\pgfqpoint{1.899850in}{3.215780in}}{\pgfqpoint{1.892036in}{3.207967in}}%
\pgfpathcurveto{\pgfqpoint{1.884223in}{3.200153in}}{\pgfqpoint{1.879832in}{3.189554in}}{\pgfqpoint{1.879832in}{3.178504in}}%
\pgfpathcurveto{\pgfqpoint{1.879832in}{3.167454in}}{\pgfqpoint{1.884223in}{3.156855in}}{\pgfqpoint{1.892036in}{3.149041in}}%
\pgfpathcurveto{\pgfqpoint{1.899850in}{3.141228in}}{\pgfqpoint{1.910449in}{3.136837in}}{\pgfqpoint{1.921499in}{3.136837in}}%
\pgfpathclose%
\pgfusepath{stroke,fill}%
\end{pgfscope}%
\begin{pgfscope}%
\pgfpathrectangle{\pgfqpoint{0.648703in}{0.548769in}}{\pgfqpoint{5.201297in}{3.102590in}}%
\pgfusepath{clip}%
\pgfsetbuttcap%
\pgfsetroundjoin%
\definecolor{currentfill}{rgb}{0.121569,0.466667,0.705882}%
\pgfsetfillcolor{currentfill}%
\pgfsetlinewidth{1.003750pt}%
\definecolor{currentstroke}{rgb}{0.121569,0.466667,0.705882}%
\pgfsetstrokecolor{currentstroke}%
\pgfsetdash{}{0pt}%
\pgfpathmoveto{\pgfqpoint{1.921499in}{3.132690in}}%
\pgfpathcurveto{\pgfqpoint{1.932549in}{3.132690in}}{\pgfqpoint{1.943148in}{3.137080in}}{\pgfqpoint{1.950962in}{3.144893in}}%
\pgfpathcurveto{\pgfqpoint{1.958775in}{3.152707in}}{\pgfqpoint{1.963166in}{3.163306in}}{\pgfqpoint{1.963166in}{3.174356in}}%
\pgfpathcurveto{\pgfqpoint{1.963166in}{3.185406in}}{\pgfqpoint{1.958775in}{3.196005in}}{\pgfqpoint{1.950962in}{3.203819in}}%
\pgfpathcurveto{\pgfqpoint{1.943148in}{3.211633in}}{\pgfqpoint{1.932549in}{3.216023in}}{\pgfqpoint{1.921499in}{3.216023in}}%
\pgfpathcurveto{\pgfqpoint{1.910449in}{3.216023in}}{\pgfqpoint{1.899850in}{3.211633in}}{\pgfqpoint{1.892036in}{3.203819in}}%
\pgfpathcurveto{\pgfqpoint{1.884223in}{3.196005in}}{\pgfqpoint{1.879832in}{3.185406in}}{\pgfqpoint{1.879832in}{3.174356in}}%
\pgfpathcurveto{\pgfqpoint{1.879832in}{3.163306in}}{\pgfqpoint{1.884223in}{3.152707in}}{\pgfqpoint{1.892036in}{3.144893in}}%
\pgfpathcurveto{\pgfqpoint{1.899850in}{3.137080in}}{\pgfqpoint{1.910449in}{3.132690in}}{\pgfqpoint{1.921499in}{3.132690in}}%
\pgfpathclose%
\pgfusepath{stroke,fill}%
\end{pgfscope}%
\begin{pgfscope}%
\pgfpathrectangle{\pgfqpoint{0.648703in}{0.548769in}}{\pgfqpoint{5.201297in}{3.102590in}}%
\pgfusepath{clip}%
\pgfsetbuttcap%
\pgfsetroundjoin%
\definecolor{currentfill}{rgb}{0.121569,0.466667,0.705882}%
\pgfsetfillcolor{currentfill}%
\pgfsetlinewidth{1.003750pt}%
\definecolor{currentstroke}{rgb}{0.121569,0.466667,0.705882}%
\pgfsetstrokecolor{currentstroke}%
\pgfsetdash{}{0pt}%
\pgfpathmoveto{\pgfqpoint{1.532859in}{3.132690in}}%
\pgfpathcurveto{\pgfqpoint{1.543909in}{3.132690in}}{\pgfqpoint{1.554508in}{3.137080in}}{\pgfqpoint{1.562322in}{3.144893in}}%
\pgfpathcurveto{\pgfqpoint{1.570135in}{3.152707in}}{\pgfqpoint{1.574526in}{3.163306in}}{\pgfqpoint{1.574526in}{3.174356in}}%
\pgfpathcurveto{\pgfqpoint{1.574526in}{3.185406in}}{\pgfqpoint{1.570135in}{3.196005in}}{\pgfqpoint{1.562322in}{3.203819in}}%
\pgfpathcurveto{\pgfqpoint{1.554508in}{3.211633in}}{\pgfqpoint{1.543909in}{3.216023in}}{\pgfqpoint{1.532859in}{3.216023in}}%
\pgfpathcurveto{\pgfqpoint{1.521809in}{3.216023in}}{\pgfqpoint{1.511210in}{3.211633in}}{\pgfqpoint{1.503396in}{3.203819in}}%
\pgfpathcurveto{\pgfqpoint{1.495583in}{3.196005in}}{\pgfqpoint{1.491192in}{3.185406in}}{\pgfqpoint{1.491192in}{3.174356in}}%
\pgfpathcurveto{\pgfqpoint{1.491192in}{3.163306in}}{\pgfqpoint{1.495583in}{3.152707in}}{\pgfqpoint{1.503396in}{3.144893in}}%
\pgfpathcurveto{\pgfqpoint{1.511210in}{3.137080in}}{\pgfqpoint{1.521809in}{3.132690in}}{\pgfqpoint{1.532859in}{3.132690in}}%
\pgfpathclose%
\pgfusepath{stroke,fill}%
\end{pgfscope}%
\begin{pgfscope}%
\pgfpathrectangle{\pgfqpoint{0.648703in}{0.548769in}}{\pgfqpoint{5.201297in}{3.102590in}}%
\pgfusepath{clip}%
\pgfsetbuttcap%
\pgfsetroundjoin%
\definecolor{currentfill}{rgb}{1.000000,0.498039,0.054902}%
\pgfsetfillcolor{currentfill}%
\pgfsetlinewidth{1.003750pt}%
\definecolor{currentstroke}{rgb}{1.000000,0.498039,0.054902}%
\pgfsetstrokecolor{currentstroke}%
\pgfsetdash{}{0pt}%
\pgfpathmoveto{\pgfqpoint{5.613577in}{3.136837in}}%
\pgfpathcurveto{\pgfqpoint{5.624628in}{3.136837in}}{\pgfqpoint{5.635227in}{3.141228in}}{\pgfqpoint{5.643040in}{3.149041in}}%
\pgfpathcurveto{\pgfqpoint{5.650854in}{3.156855in}}{\pgfqpoint{5.655244in}{3.167454in}}{\pgfqpoint{5.655244in}{3.178504in}}%
\pgfpathcurveto{\pgfqpoint{5.655244in}{3.189554in}}{\pgfqpoint{5.650854in}{3.200153in}}{\pgfqpoint{5.643040in}{3.207967in}}%
\pgfpathcurveto{\pgfqpoint{5.635227in}{3.215780in}}{\pgfqpoint{5.624628in}{3.220171in}}{\pgfqpoint{5.613577in}{3.220171in}}%
\pgfpathcurveto{\pgfqpoint{5.602527in}{3.220171in}}{\pgfqpoint{5.591928in}{3.215780in}}{\pgfqpoint{5.584115in}{3.207967in}}%
\pgfpathcurveto{\pgfqpoint{5.576301in}{3.200153in}}{\pgfqpoint{5.571911in}{3.189554in}}{\pgfqpoint{5.571911in}{3.178504in}}%
\pgfpathcurveto{\pgfqpoint{5.571911in}{3.167454in}}{\pgfqpoint{5.576301in}{3.156855in}}{\pgfqpoint{5.584115in}{3.149041in}}%
\pgfpathcurveto{\pgfqpoint{5.591928in}{3.141228in}}{\pgfqpoint{5.602527in}{3.136837in}}{\pgfqpoint{5.613577in}{3.136837in}}%
\pgfpathclose%
\pgfusepath{stroke,fill}%
\end{pgfscope}%
\begin{pgfscope}%
\pgfpathrectangle{\pgfqpoint{0.648703in}{0.548769in}}{\pgfqpoint{5.201297in}{3.102590in}}%
\pgfusepath{clip}%
\pgfsetbuttcap%
\pgfsetroundjoin%
\definecolor{currentfill}{rgb}{0.839216,0.152941,0.156863}%
\pgfsetfillcolor{currentfill}%
\pgfsetlinewidth{1.003750pt}%
\definecolor{currentstroke}{rgb}{0.839216,0.152941,0.156863}%
\pgfsetstrokecolor{currentstroke}%
\pgfsetdash{}{0pt}%
\pgfpathmoveto{\pgfqpoint{2.569232in}{3.120246in}}%
\pgfpathcurveto{\pgfqpoint{2.580282in}{3.120246in}}{\pgfqpoint{2.590881in}{3.124636in}}{\pgfqpoint{2.598695in}{3.132450in}}%
\pgfpathcurveto{\pgfqpoint{2.606508in}{3.140263in}}{\pgfqpoint{2.610899in}{3.150862in}}{\pgfqpoint{2.610899in}{3.161913in}}%
\pgfpathcurveto{\pgfqpoint{2.610899in}{3.172963in}}{\pgfqpoint{2.606508in}{3.183562in}}{\pgfqpoint{2.598695in}{3.191375in}}%
\pgfpathcurveto{\pgfqpoint{2.590881in}{3.199189in}}{\pgfqpoint{2.580282in}{3.203579in}}{\pgfqpoint{2.569232in}{3.203579in}}%
\pgfpathcurveto{\pgfqpoint{2.558182in}{3.203579in}}{\pgfqpoint{2.547583in}{3.199189in}}{\pgfqpoint{2.539769in}{3.191375in}}%
\pgfpathcurveto{\pgfqpoint{2.531956in}{3.183562in}}{\pgfqpoint{2.527565in}{3.172963in}}{\pgfqpoint{2.527565in}{3.161913in}}%
\pgfpathcurveto{\pgfqpoint{2.527565in}{3.150862in}}{\pgfqpoint{2.531956in}{3.140263in}}{\pgfqpoint{2.539769in}{3.132450in}}%
\pgfpathcurveto{\pgfqpoint{2.547583in}{3.124636in}}{\pgfqpoint{2.558182in}{3.120246in}}{\pgfqpoint{2.569232in}{3.120246in}}%
\pgfpathclose%
\pgfusepath{stroke,fill}%
\end{pgfscope}%
\begin{pgfscope}%
\pgfpathrectangle{\pgfqpoint{0.648703in}{0.548769in}}{\pgfqpoint{5.201297in}{3.102590in}}%
\pgfusepath{clip}%
\pgfsetbuttcap%
\pgfsetroundjoin%
\definecolor{currentfill}{rgb}{1.000000,0.498039,0.054902}%
\pgfsetfillcolor{currentfill}%
\pgfsetlinewidth{1.003750pt}%
\definecolor{currentstroke}{rgb}{1.000000,0.498039,0.054902}%
\pgfsetstrokecolor{currentstroke}%
\pgfsetdash{}{0pt}%
\pgfpathmoveto{\pgfqpoint{1.791952in}{3.244681in}}%
\pgfpathcurveto{\pgfqpoint{1.803002in}{3.244681in}}{\pgfqpoint{1.813601in}{3.249072in}}{\pgfqpoint{1.821415in}{3.256885in}}%
\pgfpathcurveto{\pgfqpoint{1.829229in}{3.264699in}}{\pgfqpoint{1.833619in}{3.275298in}}{\pgfqpoint{1.833619in}{3.286348in}}%
\pgfpathcurveto{\pgfqpoint{1.833619in}{3.297398in}}{\pgfqpoint{1.829229in}{3.307997in}}{\pgfqpoint{1.821415in}{3.315811in}}%
\pgfpathcurveto{\pgfqpoint{1.813601in}{3.323624in}}{\pgfqpoint{1.803002in}{3.328015in}}{\pgfqpoint{1.791952in}{3.328015in}}%
\pgfpathcurveto{\pgfqpoint{1.780902in}{3.328015in}}{\pgfqpoint{1.770303in}{3.323624in}}{\pgfqpoint{1.762490in}{3.315811in}}%
\pgfpathcurveto{\pgfqpoint{1.754676in}{3.307997in}}{\pgfqpoint{1.750286in}{3.297398in}}{\pgfqpoint{1.750286in}{3.286348in}}%
\pgfpathcurveto{\pgfqpoint{1.750286in}{3.275298in}}{\pgfqpoint{1.754676in}{3.264699in}}{\pgfqpoint{1.762490in}{3.256885in}}%
\pgfpathcurveto{\pgfqpoint{1.770303in}{3.249072in}}{\pgfqpoint{1.780902in}{3.244681in}}{\pgfqpoint{1.791952in}{3.244681in}}%
\pgfpathclose%
\pgfusepath{stroke,fill}%
\end{pgfscope}%
\begin{pgfscope}%
\pgfpathrectangle{\pgfqpoint{0.648703in}{0.548769in}}{\pgfqpoint{5.201297in}{3.102590in}}%
\pgfusepath{clip}%
\pgfsetbuttcap%
\pgfsetroundjoin%
\definecolor{currentfill}{rgb}{0.121569,0.466667,0.705882}%
\pgfsetfillcolor{currentfill}%
\pgfsetlinewidth{1.003750pt}%
\definecolor{currentstroke}{rgb}{0.121569,0.466667,0.705882}%
\pgfsetstrokecolor{currentstroke}%
\pgfsetdash{}{0pt}%
\pgfpathmoveto{\pgfqpoint{2.374912in}{3.128542in}}%
\pgfpathcurveto{\pgfqpoint{2.385962in}{3.128542in}}{\pgfqpoint{2.396561in}{3.132932in}}{\pgfqpoint{2.404375in}{3.140746in}}%
\pgfpathcurveto{\pgfqpoint{2.412188in}{3.148559in}}{\pgfqpoint{2.416579in}{3.159158in}}{\pgfqpoint{2.416579in}{3.170208in}}%
\pgfpathcurveto{\pgfqpoint{2.416579in}{3.181258in}}{\pgfqpoint{2.412188in}{3.191857in}}{\pgfqpoint{2.404375in}{3.199671in}}%
\pgfpathcurveto{\pgfqpoint{2.396561in}{3.207485in}}{\pgfqpoint{2.385962in}{3.211875in}}{\pgfqpoint{2.374912in}{3.211875in}}%
\pgfpathcurveto{\pgfqpoint{2.363862in}{3.211875in}}{\pgfqpoint{2.353263in}{3.207485in}}{\pgfqpoint{2.345449in}{3.199671in}}%
\pgfpathcurveto{\pgfqpoint{2.337636in}{3.191857in}}{\pgfqpoint{2.333245in}{3.181258in}}{\pgfqpoint{2.333245in}{3.170208in}}%
\pgfpathcurveto{\pgfqpoint{2.333245in}{3.159158in}}{\pgfqpoint{2.337636in}{3.148559in}}{\pgfqpoint{2.345449in}{3.140746in}}%
\pgfpathcurveto{\pgfqpoint{2.353263in}{3.132932in}}{\pgfqpoint{2.363862in}{3.128542in}}{\pgfqpoint{2.374912in}{3.128542in}}%
\pgfpathclose%
\pgfusepath{stroke,fill}%
\end{pgfscope}%
\begin{pgfscope}%
\pgfpathrectangle{\pgfqpoint{0.648703in}{0.548769in}}{\pgfqpoint{5.201297in}{3.102590in}}%
\pgfusepath{clip}%
\pgfsetbuttcap%
\pgfsetroundjoin%
\definecolor{currentfill}{rgb}{1.000000,0.498039,0.054902}%
\pgfsetfillcolor{currentfill}%
\pgfsetlinewidth{1.003750pt}%
\definecolor{currentstroke}{rgb}{1.000000,0.498039,0.054902}%
\pgfsetstrokecolor{currentstroke}%
\pgfsetdash{}{0pt}%
\pgfpathmoveto{\pgfqpoint{2.180592in}{3.136837in}}%
\pgfpathcurveto{\pgfqpoint{2.191642in}{3.136837in}}{\pgfqpoint{2.202241in}{3.141228in}}{\pgfqpoint{2.210055in}{3.149041in}}%
\pgfpathcurveto{\pgfqpoint{2.217869in}{3.156855in}}{\pgfqpoint{2.222259in}{3.167454in}}{\pgfqpoint{2.222259in}{3.178504in}}%
\pgfpathcurveto{\pgfqpoint{2.222259in}{3.189554in}}{\pgfqpoint{2.217869in}{3.200153in}}{\pgfqpoint{2.210055in}{3.207967in}}%
\pgfpathcurveto{\pgfqpoint{2.202241in}{3.215780in}}{\pgfqpoint{2.191642in}{3.220171in}}{\pgfqpoint{2.180592in}{3.220171in}}%
\pgfpathcurveto{\pgfqpoint{2.169542in}{3.220171in}}{\pgfqpoint{2.158943in}{3.215780in}}{\pgfqpoint{2.151129in}{3.207967in}}%
\pgfpathcurveto{\pgfqpoint{2.143316in}{3.200153in}}{\pgfqpoint{2.138925in}{3.189554in}}{\pgfqpoint{2.138925in}{3.178504in}}%
\pgfpathcurveto{\pgfqpoint{2.138925in}{3.167454in}}{\pgfqpoint{2.143316in}{3.156855in}}{\pgfqpoint{2.151129in}{3.149041in}}%
\pgfpathcurveto{\pgfqpoint{2.158943in}{3.141228in}}{\pgfqpoint{2.169542in}{3.136837in}}{\pgfqpoint{2.180592in}{3.136837in}}%
\pgfpathclose%
\pgfusepath{stroke,fill}%
\end{pgfscope}%
\begin{pgfscope}%
\pgfpathrectangle{\pgfqpoint{0.648703in}{0.548769in}}{\pgfqpoint{5.201297in}{3.102590in}}%
\pgfusepath{clip}%
\pgfsetbuttcap%
\pgfsetroundjoin%
\definecolor{currentfill}{rgb}{1.000000,0.498039,0.054902}%
\pgfsetfillcolor{currentfill}%
\pgfsetlinewidth{1.003750pt}%
\definecolor{currentstroke}{rgb}{1.000000,0.498039,0.054902}%
\pgfsetstrokecolor{currentstroke}%
\pgfsetdash{}{0pt}%
\pgfpathmoveto{\pgfqpoint{2.245365in}{3.136837in}}%
\pgfpathcurveto{\pgfqpoint{2.256416in}{3.136837in}}{\pgfqpoint{2.267015in}{3.141228in}}{\pgfqpoint{2.274828in}{3.149041in}}%
\pgfpathcurveto{\pgfqpoint{2.282642in}{3.156855in}}{\pgfqpoint{2.287032in}{3.167454in}}{\pgfqpoint{2.287032in}{3.178504in}}%
\pgfpathcurveto{\pgfqpoint{2.287032in}{3.189554in}}{\pgfqpoint{2.282642in}{3.200153in}}{\pgfqpoint{2.274828in}{3.207967in}}%
\pgfpathcurveto{\pgfqpoint{2.267015in}{3.215780in}}{\pgfqpoint{2.256416in}{3.220171in}}{\pgfqpoint{2.245365in}{3.220171in}}%
\pgfpathcurveto{\pgfqpoint{2.234315in}{3.220171in}}{\pgfqpoint{2.223716in}{3.215780in}}{\pgfqpoint{2.215903in}{3.207967in}}%
\pgfpathcurveto{\pgfqpoint{2.208089in}{3.200153in}}{\pgfqpoint{2.203699in}{3.189554in}}{\pgfqpoint{2.203699in}{3.178504in}}%
\pgfpathcurveto{\pgfqpoint{2.203699in}{3.167454in}}{\pgfqpoint{2.208089in}{3.156855in}}{\pgfqpoint{2.215903in}{3.149041in}}%
\pgfpathcurveto{\pgfqpoint{2.223716in}{3.141228in}}{\pgfqpoint{2.234315in}{3.136837in}}{\pgfqpoint{2.245365in}{3.136837in}}%
\pgfpathclose%
\pgfusepath{stroke,fill}%
\end{pgfscope}%
\begin{pgfscope}%
\pgfpathrectangle{\pgfqpoint{0.648703in}{0.548769in}}{\pgfqpoint{5.201297in}{3.102590in}}%
\pgfusepath{clip}%
\pgfsetbuttcap%
\pgfsetroundjoin%
\definecolor{currentfill}{rgb}{1.000000,0.498039,0.054902}%
\pgfsetfillcolor{currentfill}%
\pgfsetlinewidth{1.003750pt}%
\definecolor{currentstroke}{rgb}{1.000000,0.498039,0.054902}%
\pgfsetstrokecolor{currentstroke}%
\pgfsetdash{}{0pt}%
\pgfpathmoveto{\pgfqpoint{1.921499in}{3.136837in}}%
\pgfpathcurveto{\pgfqpoint{1.932549in}{3.136837in}}{\pgfqpoint{1.943148in}{3.141228in}}{\pgfqpoint{1.950962in}{3.149041in}}%
\pgfpathcurveto{\pgfqpoint{1.958775in}{3.156855in}}{\pgfqpoint{1.963166in}{3.167454in}}{\pgfqpoint{1.963166in}{3.178504in}}%
\pgfpathcurveto{\pgfqpoint{1.963166in}{3.189554in}}{\pgfqpoint{1.958775in}{3.200153in}}{\pgfqpoint{1.950962in}{3.207967in}}%
\pgfpathcurveto{\pgfqpoint{1.943148in}{3.215780in}}{\pgfqpoint{1.932549in}{3.220171in}}{\pgfqpoint{1.921499in}{3.220171in}}%
\pgfpathcurveto{\pgfqpoint{1.910449in}{3.220171in}}{\pgfqpoint{1.899850in}{3.215780in}}{\pgfqpoint{1.892036in}{3.207967in}}%
\pgfpathcurveto{\pgfqpoint{1.884223in}{3.200153in}}{\pgfqpoint{1.879832in}{3.189554in}}{\pgfqpoint{1.879832in}{3.178504in}}%
\pgfpathcurveto{\pgfqpoint{1.879832in}{3.167454in}}{\pgfqpoint{1.884223in}{3.156855in}}{\pgfqpoint{1.892036in}{3.149041in}}%
\pgfpathcurveto{\pgfqpoint{1.899850in}{3.141228in}}{\pgfqpoint{1.910449in}{3.136837in}}{\pgfqpoint{1.921499in}{3.136837in}}%
\pgfpathclose%
\pgfusepath{stroke,fill}%
\end{pgfscope}%
\begin{pgfscope}%
\pgfpathrectangle{\pgfqpoint{0.648703in}{0.548769in}}{\pgfqpoint{5.201297in}{3.102590in}}%
\pgfusepath{clip}%
\pgfsetbuttcap%
\pgfsetroundjoin%
\definecolor{currentfill}{rgb}{0.121569,0.466667,0.705882}%
\pgfsetfillcolor{currentfill}%
\pgfsetlinewidth{1.003750pt}%
\definecolor{currentstroke}{rgb}{0.121569,0.466667,0.705882}%
\pgfsetstrokecolor{currentstroke}%
\pgfsetdash{}{0pt}%
\pgfpathmoveto{\pgfqpoint{2.245365in}{3.132690in}}%
\pgfpathcurveto{\pgfqpoint{2.256416in}{3.132690in}}{\pgfqpoint{2.267015in}{3.137080in}}{\pgfqpoint{2.274828in}{3.144893in}}%
\pgfpathcurveto{\pgfqpoint{2.282642in}{3.152707in}}{\pgfqpoint{2.287032in}{3.163306in}}{\pgfqpoint{2.287032in}{3.174356in}}%
\pgfpathcurveto{\pgfqpoint{2.287032in}{3.185406in}}{\pgfqpoint{2.282642in}{3.196005in}}{\pgfqpoint{2.274828in}{3.203819in}}%
\pgfpathcurveto{\pgfqpoint{2.267015in}{3.211633in}}{\pgfqpoint{2.256416in}{3.216023in}}{\pgfqpoint{2.245365in}{3.216023in}}%
\pgfpathcurveto{\pgfqpoint{2.234315in}{3.216023in}}{\pgfqpoint{2.223716in}{3.211633in}}{\pgfqpoint{2.215903in}{3.203819in}}%
\pgfpathcurveto{\pgfqpoint{2.208089in}{3.196005in}}{\pgfqpoint{2.203699in}{3.185406in}}{\pgfqpoint{2.203699in}{3.174356in}}%
\pgfpathcurveto{\pgfqpoint{2.203699in}{3.163306in}}{\pgfqpoint{2.208089in}{3.152707in}}{\pgfqpoint{2.215903in}{3.144893in}}%
\pgfpathcurveto{\pgfqpoint{2.223716in}{3.137080in}}{\pgfqpoint{2.234315in}{3.132690in}}{\pgfqpoint{2.245365in}{3.132690in}}%
\pgfpathclose%
\pgfusepath{stroke,fill}%
\end{pgfscope}%
\begin{pgfscope}%
\pgfpathrectangle{\pgfqpoint{0.648703in}{0.548769in}}{\pgfqpoint{5.201297in}{3.102590in}}%
\pgfusepath{clip}%
\pgfsetbuttcap%
\pgfsetroundjoin%
\definecolor{currentfill}{rgb}{0.121569,0.466667,0.705882}%
\pgfsetfillcolor{currentfill}%
\pgfsetlinewidth{1.003750pt}%
\definecolor{currentstroke}{rgb}{0.121569,0.466667,0.705882}%
\pgfsetstrokecolor{currentstroke}%
\pgfsetdash{}{0pt}%
\pgfpathmoveto{\pgfqpoint{4.836298in}{3.107802in}}%
\pgfpathcurveto{\pgfqpoint{4.847348in}{3.107802in}}{\pgfqpoint{4.857947in}{3.112193in}}{\pgfqpoint{4.865761in}{3.120006in}}%
\pgfpathcurveto{\pgfqpoint{4.873574in}{3.127820in}}{\pgfqpoint{4.877964in}{3.138419in}}{\pgfqpoint{4.877964in}{3.149469in}}%
\pgfpathcurveto{\pgfqpoint{4.877964in}{3.160519in}}{\pgfqpoint{4.873574in}{3.171118in}}{\pgfqpoint{4.865761in}{3.178932in}}%
\pgfpathcurveto{\pgfqpoint{4.857947in}{3.186745in}}{\pgfqpoint{4.847348in}{3.191136in}}{\pgfqpoint{4.836298in}{3.191136in}}%
\pgfpathcurveto{\pgfqpoint{4.825248in}{3.191136in}}{\pgfqpoint{4.814649in}{3.186745in}}{\pgfqpoint{4.806835in}{3.178932in}}%
\pgfpathcurveto{\pgfqpoint{4.799021in}{3.171118in}}{\pgfqpoint{4.794631in}{3.160519in}}{\pgfqpoint{4.794631in}{3.149469in}}%
\pgfpathcurveto{\pgfqpoint{4.794631in}{3.138419in}}{\pgfqpoint{4.799021in}{3.127820in}}{\pgfqpoint{4.806835in}{3.120006in}}%
\pgfpathcurveto{\pgfqpoint{4.814649in}{3.112193in}}{\pgfqpoint{4.825248in}{3.107802in}}{\pgfqpoint{4.836298in}{3.107802in}}%
\pgfpathclose%
\pgfusepath{stroke,fill}%
\end{pgfscope}%
\begin{pgfscope}%
\pgfpathrectangle{\pgfqpoint{0.648703in}{0.548769in}}{\pgfqpoint{5.201297in}{3.102590in}}%
\pgfusepath{clip}%
\pgfsetbuttcap%
\pgfsetroundjoin%
\definecolor{currentfill}{rgb}{0.121569,0.466667,0.705882}%
\pgfsetfillcolor{currentfill}%
\pgfsetlinewidth{1.003750pt}%
\definecolor{currentstroke}{rgb}{0.121569,0.466667,0.705882}%
\pgfsetstrokecolor{currentstroke}%
\pgfsetdash{}{0pt}%
\pgfpathmoveto{\pgfqpoint{2.634005in}{3.132690in}}%
\pgfpathcurveto{\pgfqpoint{2.645055in}{3.132690in}}{\pgfqpoint{2.655654in}{3.137080in}}{\pgfqpoint{2.663468in}{3.144893in}}%
\pgfpathcurveto{\pgfqpoint{2.671282in}{3.152707in}}{\pgfqpoint{2.675672in}{3.163306in}}{\pgfqpoint{2.675672in}{3.174356in}}%
\pgfpathcurveto{\pgfqpoint{2.675672in}{3.185406in}}{\pgfqpoint{2.671282in}{3.196005in}}{\pgfqpoint{2.663468in}{3.203819in}}%
\pgfpathcurveto{\pgfqpoint{2.655654in}{3.211633in}}{\pgfqpoint{2.645055in}{3.216023in}}{\pgfqpoint{2.634005in}{3.216023in}}%
\pgfpathcurveto{\pgfqpoint{2.622955in}{3.216023in}}{\pgfqpoint{2.612356in}{3.211633in}}{\pgfqpoint{2.604543in}{3.203819in}}%
\pgfpathcurveto{\pgfqpoint{2.596729in}{3.196005in}}{\pgfqpoint{2.592339in}{3.185406in}}{\pgfqpoint{2.592339in}{3.174356in}}%
\pgfpathcurveto{\pgfqpoint{2.592339in}{3.163306in}}{\pgfqpoint{2.596729in}{3.152707in}}{\pgfqpoint{2.604543in}{3.144893in}}%
\pgfpathcurveto{\pgfqpoint{2.612356in}{3.137080in}}{\pgfqpoint{2.622955in}{3.132690in}}{\pgfqpoint{2.634005in}{3.132690in}}%
\pgfpathclose%
\pgfusepath{stroke,fill}%
\end{pgfscope}%
\begin{pgfscope}%
\pgfpathrectangle{\pgfqpoint{0.648703in}{0.548769in}}{\pgfqpoint{5.201297in}{3.102590in}}%
\pgfusepath{clip}%
\pgfsetbuttcap%
\pgfsetroundjoin%
\definecolor{currentfill}{rgb}{0.121569,0.466667,0.705882}%
\pgfsetfillcolor{currentfill}%
\pgfsetlinewidth{1.003750pt}%
\definecolor{currentstroke}{rgb}{0.121569,0.466667,0.705882}%
\pgfsetstrokecolor{currentstroke}%
\pgfsetdash{}{0pt}%
\pgfpathmoveto{\pgfqpoint{1.144219in}{0.648129in}}%
\pgfpathcurveto{\pgfqpoint{1.155269in}{0.648129in}}{\pgfqpoint{1.165868in}{0.652519in}}{\pgfqpoint{1.173682in}{0.660333in}}%
\pgfpathcurveto{\pgfqpoint{1.181496in}{0.668146in}}{\pgfqpoint{1.185886in}{0.678745in}}{\pgfqpoint{1.185886in}{0.689796in}}%
\pgfpathcurveto{\pgfqpoint{1.185886in}{0.700846in}}{\pgfqpoint{1.181496in}{0.711445in}}{\pgfqpoint{1.173682in}{0.719258in}}%
\pgfpathcurveto{\pgfqpoint{1.165868in}{0.727072in}}{\pgfqpoint{1.155269in}{0.731462in}}{\pgfqpoint{1.144219in}{0.731462in}}%
\pgfpathcurveto{\pgfqpoint{1.133169in}{0.731462in}}{\pgfqpoint{1.122570in}{0.727072in}}{\pgfqpoint{1.114756in}{0.719258in}}%
\pgfpathcurveto{\pgfqpoint{1.106943in}{0.711445in}}{\pgfqpoint{1.102553in}{0.700846in}}{\pgfqpoint{1.102553in}{0.689796in}}%
\pgfpathcurveto{\pgfqpoint{1.102553in}{0.678745in}}{\pgfqpoint{1.106943in}{0.668146in}}{\pgfqpoint{1.114756in}{0.660333in}}%
\pgfpathcurveto{\pgfqpoint{1.122570in}{0.652519in}}{\pgfqpoint{1.133169in}{0.648129in}}{\pgfqpoint{1.144219in}{0.648129in}}%
\pgfpathclose%
\pgfusepath{stroke,fill}%
\end{pgfscope}%
\begin{pgfscope}%
\pgfpathrectangle{\pgfqpoint{0.648703in}{0.548769in}}{\pgfqpoint{5.201297in}{3.102590in}}%
\pgfusepath{clip}%
\pgfsetbuttcap%
\pgfsetroundjoin%
\definecolor{currentfill}{rgb}{1.000000,0.498039,0.054902}%
\pgfsetfillcolor{currentfill}%
\pgfsetlinewidth{1.003750pt}%
\definecolor{currentstroke}{rgb}{1.000000,0.498039,0.054902}%
\pgfsetstrokecolor{currentstroke}%
\pgfsetdash{}{0pt}%
\pgfpathmoveto{\pgfqpoint{2.698779in}{3.140985in}}%
\pgfpathcurveto{\pgfqpoint{2.709829in}{3.140985in}}{\pgfqpoint{2.720428in}{3.145375in}}{\pgfqpoint{2.728241in}{3.153189in}}%
\pgfpathcurveto{\pgfqpoint{2.736055in}{3.161003in}}{\pgfqpoint{2.740445in}{3.171602in}}{\pgfqpoint{2.740445in}{3.182652in}}%
\pgfpathcurveto{\pgfqpoint{2.740445in}{3.193702in}}{\pgfqpoint{2.736055in}{3.204301in}}{\pgfqpoint{2.728241in}{3.212115in}}%
\pgfpathcurveto{\pgfqpoint{2.720428in}{3.219928in}}{\pgfqpoint{2.709829in}{3.224319in}}{\pgfqpoint{2.698779in}{3.224319in}}%
\pgfpathcurveto{\pgfqpoint{2.687728in}{3.224319in}}{\pgfqpoint{2.677129in}{3.219928in}}{\pgfqpoint{2.669316in}{3.212115in}}%
\pgfpathcurveto{\pgfqpoint{2.661502in}{3.204301in}}{\pgfqpoint{2.657112in}{3.193702in}}{\pgfqpoint{2.657112in}{3.182652in}}%
\pgfpathcurveto{\pgfqpoint{2.657112in}{3.171602in}}{\pgfqpoint{2.661502in}{3.161003in}}{\pgfqpoint{2.669316in}{3.153189in}}%
\pgfpathcurveto{\pgfqpoint{2.677129in}{3.145375in}}{\pgfqpoint{2.687728in}{3.140985in}}{\pgfqpoint{2.698779in}{3.140985in}}%
\pgfpathclose%
\pgfusepath{stroke,fill}%
\end{pgfscope}%
\begin{pgfscope}%
\pgfpathrectangle{\pgfqpoint{0.648703in}{0.548769in}}{\pgfqpoint{5.201297in}{3.102590in}}%
\pgfusepath{clip}%
\pgfsetbuttcap%
\pgfsetroundjoin%
\definecolor{currentfill}{rgb}{1.000000,0.498039,0.054902}%
\pgfsetfillcolor{currentfill}%
\pgfsetlinewidth{1.003750pt}%
\definecolor{currentstroke}{rgb}{1.000000,0.498039,0.054902}%
\pgfsetstrokecolor{currentstroke}%
\pgfsetdash{}{0pt}%
\pgfpathmoveto{\pgfqpoint{1.727179in}{3.145133in}}%
\pgfpathcurveto{\pgfqpoint{1.738229in}{3.145133in}}{\pgfqpoint{1.748828in}{3.149523in}}{\pgfqpoint{1.756642in}{3.157337in}}%
\pgfpathcurveto{\pgfqpoint{1.764455in}{3.165151in}}{\pgfqpoint{1.768846in}{3.175750in}}{\pgfqpoint{1.768846in}{3.186800in}}%
\pgfpathcurveto{\pgfqpoint{1.768846in}{3.197850in}}{\pgfqpoint{1.764455in}{3.208449in}}{\pgfqpoint{1.756642in}{3.216262in}}%
\pgfpathcurveto{\pgfqpoint{1.748828in}{3.224076in}}{\pgfqpoint{1.738229in}{3.228466in}}{\pgfqpoint{1.727179in}{3.228466in}}%
\pgfpathcurveto{\pgfqpoint{1.716129in}{3.228466in}}{\pgfqpoint{1.705530in}{3.224076in}}{\pgfqpoint{1.697716in}{3.216262in}}%
\pgfpathcurveto{\pgfqpoint{1.689903in}{3.208449in}}{\pgfqpoint{1.685512in}{3.197850in}}{\pgfqpoint{1.685512in}{3.186800in}}%
\pgfpathcurveto{\pgfqpoint{1.685512in}{3.175750in}}{\pgfqpoint{1.689903in}{3.165151in}}{\pgfqpoint{1.697716in}{3.157337in}}%
\pgfpathcurveto{\pgfqpoint{1.705530in}{3.149523in}}{\pgfqpoint{1.716129in}{3.145133in}}{\pgfqpoint{1.727179in}{3.145133in}}%
\pgfpathclose%
\pgfusepath{stroke,fill}%
\end{pgfscope}%
\begin{pgfscope}%
\pgfpathrectangle{\pgfqpoint{0.648703in}{0.548769in}}{\pgfqpoint{5.201297in}{3.102590in}}%
\pgfusepath{clip}%
\pgfsetbuttcap%
\pgfsetroundjoin%
\definecolor{currentfill}{rgb}{1.000000,0.498039,0.054902}%
\pgfsetfillcolor{currentfill}%
\pgfsetlinewidth{1.003750pt}%
\definecolor{currentstroke}{rgb}{1.000000,0.498039,0.054902}%
\pgfsetstrokecolor{currentstroke}%
\pgfsetdash{}{0pt}%
\pgfpathmoveto{\pgfqpoint{1.856726in}{3.140985in}}%
\pgfpathcurveto{\pgfqpoint{1.867776in}{3.140985in}}{\pgfqpoint{1.878375in}{3.145375in}}{\pgfqpoint{1.886188in}{3.153189in}}%
\pgfpathcurveto{\pgfqpoint{1.894002in}{3.161003in}}{\pgfqpoint{1.898392in}{3.171602in}}{\pgfqpoint{1.898392in}{3.182652in}}%
\pgfpathcurveto{\pgfqpoint{1.898392in}{3.193702in}}{\pgfqpoint{1.894002in}{3.204301in}}{\pgfqpoint{1.886188in}{3.212115in}}%
\pgfpathcurveto{\pgfqpoint{1.878375in}{3.219928in}}{\pgfqpoint{1.867776in}{3.224319in}}{\pgfqpoint{1.856726in}{3.224319in}}%
\pgfpathcurveto{\pgfqpoint{1.845675in}{3.224319in}}{\pgfqpoint{1.835076in}{3.219928in}}{\pgfqpoint{1.827263in}{3.212115in}}%
\pgfpathcurveto{\pgfqpoint{1.819449in}{3.204301in}}{\pgfqpoint{1.815059in}{3.193702in}}{\pgfqpoint{1.815059in}{3.182652in}}%
\pgfpathcurveto{\pgfqpoint{1.815059in}{3.171602in}}{\pgfqpoint{1.819449in}{3.161003in}}{\pgfqpoint{1.827263in}{3.153189in}}%
\pgfpathcurveto{\pgfqpoint{1.835076in}{3.145375in}}{\pgfqpoint{1.845675in}{3.140985in}}{\pgfqpoint{1.856726in}{3.140985in}}%
\pgfpathclose%
\pgfusepath{stroke,fill}%
\end{pgfscope}%
\begin{pgfscope}%
\pgfpathrectangle{\pgfqpoint{0.648703in}{0.548769in}}{\pgfqpoint{5.201297in}{3.102590in}}%
\pgfusepath{clip}%
\pgfsetbuttcap%
\pgfsetroundjoin%
\definecolor{currentfill}{rgb}{0.121569,0.466667,0.705882}%
\pgfsetfillcolor{currentfill}%
\pgfsetlinewidth{1.003750pt}%
\definecolor{currentstroke}{rgb}{0.121569,0.466667,0.705882}%
\pgfsetstrokecolor{currentstroke}%
\pgfsetdash{}{0pt}%
\pgfpathmoveto{\pgfqpoint{1.338539in}{0.697903in}}%
\pgfpathcurveto{\pgfqpoint{1.349589in}{0.697903in}}{\pgfqpoint{1.360188in}{0.702293in}}{\pgfqpoint{1.368002in}{0.710107in}}%
\pgfpathcurveto{\pgfqpoint{1.375816in}{0.717921in}}{\pgfqpoint{1.380206in}{0.728520in}}{\pgfqpoint{1.380206in}{0.739570in}}%
\pgfpathcurveto{\pgfqpoint{1.380206in}{0.750620in}}{\pgfqpoint{1.375816in}{0.761219in}}{\pgfqpoint{1.368002in}{0.769033in}}%
\pgfpathcurveto{\pgfqpoint{1.360188in}{0.776846in}}{\pgfqpoint{1.349589in}{0.781236in}}{\pgfqpoint{1.338539in}{0.781236in}}%
\pgfpathcurveto{\pgfqpoint{1.327489in}{0.781236in}}{\pgfqpoint{1.316890in}{0.776846in}}{\pgfqpoint{1.309076in}{0.769033in}}%
\pgfpathcurveto{\pgfqpoint{1.301263in}{0.761219in}}{\pgfqpoint{1.296872in}{0.750620in}}{\pgfqpoint{1.296872in}{0.739570in}}%
\pgfpathcurveto{\pgfqpoint{1.296872in}{0.728520in}}{\pgfqpoint{1.301263in}{0.717921in}}{\pgfqpoint{1.309076in}{0.710107in}}%
\pgfpathcurveto{\pgfqpoint{1.316890in}{0.702293in}}{\pgfqpoint{1.327489in}{0.697903in}}{\pgfqpoint{1.338539in}{0.697903in}}%
\pgfpathclose%
\pgfusepath{stroke,fill}%
\end{pgfscope}%
\begin{pgfscope}%
\pgfpathrectangle{\pgfqpoint{0.648703in}{0.548769in}}{\pgfqpoint{5.201297in}{3.102590in}}%
\pgfusepath{clip}%
\pgfsetbuttcap%
\pgfsetroundjoin%
\definecolor{currentfill}{rgb}{1.000000,0.498039,0.054902}%
\pgfsetfillcolor{currentfill}%
\pgfsetlinewidth{1.003750pt}%
\definecolor{currentstroke}{rgb}{1.000000,0.498039,0.054902}%
\pgfsetstrokecolor{currentstroke}%
\pgfsetdash{}{0pt}%
\pgfpathmoveto{\pgfqpoint{2.634005in}{3.140985in}}%
\pgfpathcurveto{\pgfqpoint{2.645055in}{3.140985in}}{\pgfqpoint{2.655654in}{3.145375in}}{\pgfqpoint{2.663468in}{3.153189in}}%
\pgfpathcurveto{\pgfqpoint{2.671282in}{3.161003in}}{\pgfqpoint{2.675672in}{3.171602in}}{\pgfqpoint{2.675672in}{3.182652in}}%
\pgfpathcurveto{\pgfqpoint{2.675672in}{3.193702in}}{\pgfqpoint{2.671282in}{3.204301in}}{\pgfqpoint{2.663468in}{3.212115in}}%
\pgfpathcurveto{\pgfqpoint{2.655654in}{3.219928in}}{\pgfqpoint{2.645055in}{3.224319in}}{\pgfqpoint{2.634005in}{3.224319in}}%
\pgfpathcurveto{\pgfqpoint{2.622955in}{3.224319in}}{\pgfqpoint{2.612356in}{3.219928in}}{\pgfqpoint{2.604543in}{3.212115in}}%
\pgfpathcurveto{\pgfqpoint{2.596729in}{3.204301in}}{\pgfqpoint{2.592339in}{3.193702in}}{\pgfqpoint{2.592339in}{3.182652in}}%
\pgfpathcurveto{\pgfqpoint{2.592339in}{3.171602in}}{\pgfqpoint{2.596729in}{3.161003in}}{\pgfqpoint{2.604543in}{3.153189in}}%
\pgfpathcurveto{\pgfqpoint{2.612356in}{3.145375in}}{\pgfqpoint{2.622955in}{3.140985in}}{\pgfqpoint{2.634005in}{3.140985in}}%
\pgfpathclose%
\pgfusepath{stroke,fill}%
\end{pgfscope}%
\begin{pgfscope}%
\pgfpathrectangle{\pgfqpoint{0.648703in}{0.548769in}}{\pgfqpoint{5.201297in}{3.102590in}}%
\pgfusepath{clip}%
\pgfsetbuttcap%
\pgfsetroundjoin%
\definecolor{currentfill}{rgb}{0.121569,0.466667,0.705882}%
\pgfsetfillcolor{currentfill}%
\pgfsetlinewidth{1.003750pt}%
\definecolor{currentstroke}{rgb}{0.121569,0.466667,0.705882}%
\pgfsetstrokecolor{currentstroke}%
\pgfsetdash{}{0pt}%
\pgfpathmoveto{\pgfqpoint{2.439685in}{3.132690in}}%
\pgfpathcurveto{\pgfqpoint{2.450735in}{3.132690in}}{\pgfqpoint{2.461335in}{3.137080in}}{\pgfqpoint{2.469148in}{3.144893in}}%
\pgfpathcurveto{\pgfqpoint{2.476962in}{3.152707in}}{\pgfqpoint{2.481352in}{3.163306in}}{\pgfqpoint{2.481352in}{3.174356in}}%
\pgfpathcurveto{\pgfqpoint{2.481352in}{3.185406in}}{\pgfqpoint{2.476962in}{3.196005in}}{\pgfqpoint{2.469148in}{3.203819in}}%
\pgfpathcurveto{\pgfqpoint{2.461335in}{3.211633in}}{\pgfqpoint{2.450735in}{3.216023in}}{\pgfqpoint{2.439685in}{3.216023in}}%
\pgfpathcurveto{\pgfqpoint{2.428635in}{3.216023in}}{\pgfqpoint{2.418036in}{3.211633in}}{\pgfqpoint{2.410223in}{3.203819in}}%
\pgfpathcurveto{\pgfqpoint{2.402409in}{3.196005in}}{\pgfqpoint{2.398019in}{3.185406in}}{\pgfqpoint{2.398019in}{3.174356in}}%
\pgfpathcurveto{\pgfqpoint{2.398019in}{3.163306in}}{\pgfqpoint{2.402409in}{3.152707in}}{\pgfqpoint{2.410223in}{3.144893in}}%
\pgfpathcurveto{\pgfqpoint{2.418036in}{3.137080in}}{\pgfqpoint{2.428635in}{3.132690in}}{\pgfqpoint{2.439685in}{3.132690in}}%
\pgfpathclose%
\pgfusepath{stroke,fill}%
\end{pgfscope}%
\begin{pgfscope}%
\pgfpathrectangle{\pgfqpoint{0.648703in}{0.548769in}}{\pgfqpoint{5.201297in}{3.102590in}}%
\pgfusepath{clip}%
\pgfsetbuttcap%
\pgfsetroundjoin%
\definecolor{currentfill}{rgb}{1.000000,0.498039,0.054902}%
\pgfsetfillcolor{currentfill}%
\pgfsetlinewidth{1.003750pt}%
\definecolor{currentstroke}{rgb}{1.000000,0.498039,0.054902}%
\pgfsetstrokecolor{currentstroke}%
\pgfsetdash{}{0pt}%
\pgfpathmoveto{\pgfqpoint{3.087418in}{3.140985in}}%
\pgfpathcurveto{\pgfqpoint{3.098469in}{3.140985in}}{\pgfqpoint{3.109068in}{3.145375in}}{\pgfqpoint{3.116881in}{3.153189in}}%
\pgfpathcurveto{\pgfqpoint{3.124695in}{3.161003in}}{\pgfqpoint{3.129085in}{3.171602in}}{\pgfqpoint{3.129085in}{3.182652in}}%
\pgfpathcurveto{\pgfqpoint{3.129085in}{3.193702in}}{\pgfqpoint{3.124695in}{3.204301in}}{\pgfqpoint{3.116881in}{3.212115in}}%
\pgfpathcurveto{\pgfqpoint{3.109068in}{3.219928in}}{\pgfqpoint{3.098469in}{3.224319in}}{\pgfqpoint{3.087418in}{3.224319in}}%
\pgfpathcurveto{\pgfqpoint{3.076368in}{3.224319in}}{\pgfqpoint{3.065769in}{3.219928in}}{\pgfqpoint{3.057956in}{3.212115in}}%
\pgfpathcurveto{\pgfqpoint{3.050142in}{3.204301in}}{\pgfqpoint{3.045752in}{3.193702in}}{\pgfqpoint{3.045752in}{3.182652in}}%
\pgfpathcurveto{\pgfqpoint{3.045752in}{3.171602in}}{\pgfqpoint{3.050142in}{3.161003in}}{\pgfqpoint{3.057956in}{3.153189in}}%
\pgfpathcurveto{\pgfqpoint{3.065769in}{3.145375in}}{\pgfqpoint{3.076368in}{3.140985in}}{\pgfqpoint{3.087418in}{3.140985in}}%
\pgfpathclose%
\pgfusepath{stroke,fill}%
\end{pgfscope}%
\begin{pgfscope}%
\pgfpathrectangle{\pgfqpoint{0.648703in}{0.548769in}}{\pgfqpoint{5.201297in}{3.102590in}}%
\pgfusepath{clip}%
\pgfsetbuttcap%
\pgfsetroundjoin%
\definecolor{currentfill}{rgb}{0.121569,0.466667,0.705882}%
\pgfsetfillcolor{currentfill}%
\pgfsetlinewidth{1.003750pt}%
\definecolor{currentstroke}{rgb}{0.121569,0.466667,0.705882}%
\pgfsetstrokecolor{currentstroke}%
\pgfsetdash{}{0pt}%
\pgfpathmoveto{\pgfqpoint{1.727179in}{3.132690in}}%
\pgfpathcurveto{\pgfqpoint{1.738229in}{3.132690in}}{\pgfqpoint{1.748828in}{3.137080in}}{\pgfqpoint{1.756642in}{3.144893in}}%
\pgfpathcurveto{\pgfqpoint{1.764455in}{3.152707in}}{\pgfqpoint{1.768846in}{3.163306in}}{\pgfqpoint{1.768846in}{3.174356in}}%
\pgfpathcurveto{\pgfqpoint{1.768846in}{3.185406in}}{\pgfqpoint{1.764455in}{3.196005in}}{\pgfqpoint{1.756642in}{3.203819in}}%
\pgfpathcurveto{\pgfqpoint{1.748828in}{3.211633in}}{\pgfqpoint{1.738229in}{3.216023in}}{\pgfqpoint{1.727179in}{3.216023in}}%
\pgfpathcurveto{\pgfqpoint{1.716129in}{3.216023in}}{\pgfqpoint{1.705530in}{3.211633in}}{\pgfqpoint{1.697716in}{3.203819in}}%
\pgfpathcurveto{\pgfqpoint{1.689903in}{3.196005in}}{\pgfqpoint{1.685512in}{3.185406in}}{\pgfqpoint{1.685512in}{3.174356in}}%
\pgfpathcurveto{\pgfqpoint{1.685512in}{3.163306in}}{\pgfqpoint{1.689903in}{3.152707in}}{\pgfqpoint{1.697716in}{3.144893in}}%
\pgfpathcurveto{\pgfqpoint{1.705530in}{3.137080in}}{\pgfqpoint{1.716129in}{3.132690in}}{\pgfqpoint{1.727179in}{3.132690in}}%
\pgfpathclose%
\pgfusepath{stroke,fill}%
\end{pgfscope}%
\begin{pgfscope}%
\pgfpathrectangle{\pgfqpoint{0.648703in}{0.548769in}}{\pgfqpoint{5.201297in}{3.102590in}}%
\pgfusepath{clip}%
\pgfsetbuttcap%
\pgfsetroundjoin%
\definecolor{currentfill}{rgb}{1.000000,0.498039,0.054902}%
\pgfsetfillcolor{currentfill}%
\pgfsetlinewidth{1.003750pt}%
\definecolor{currentstroke}{rgb}{1.000000,0.498039,0.054902}%
\pgfsetstrokecolor{currentstroke}%
\pgfsetdash{}{0pt}%
\pgfpathmoveto{\pgfqpoint{2.180592in}{3.140985in}}%
\pgfpathcurveto{\pgfqpoint{2.191642in}{3.140985in}}{\pgfqpoint{2.202241in}{3.145375in}}{\pgfqpoint{2.210055in}{3.153189in}}%
\pgfpathcurveto{\pgfqpoint{2.217869in}{3.161003in}}{\pgfqpoint{2.222259in}{3.171602in}}{\pgfqpoint{2.222259in}{3.182652in}}%
\pgfpathcurveto{\pgfqpoint{2.222259in}{3.193702in}}{\pgfqpoint{2.217869in}{3.204301in}}{\pgfqpoint{2.210055in}{3.212115in}}%
\pgfpathcurveto{\pgfqpoint{2.202241in}{3.219928in}}{\pgfqpoint{2.191642in}{3.224319in}}{\pgfqpoint{2.180592in}{3.224319in}}%
\pgfpathcurveto{\pgfqpoint{2.169542in}{3.224319in}}{\pgfqpoint{2.158943in}{3.219928in}}{\pgfqpoint{2.151129in}{3.212115in}}%
\pgfpathcurveto{\pgfqpoint{2.143316in}{3.204301in}}{\pgfqpoint{2.138925in}{3.193702in}}{\pgfqpoint{2.138925in}{3.182652in}}%
\pgfpathcurveto{\pgfqpoint{2.138925in}{3.171602in}}{\pgfqpoint{2.143316in}{3.161003in}}{\pgfqpoint{2.151129in}{3.153189in}}%
\pgfpathcurveto{\pgfqpoint{2.158943in}{3.145375in}}{\pgfqpoint{2.169542in}{3.140985in}}{\pgfqpoint{2.180592in}{3.140985in}}%
\pgfpathclose%
\pgfusepath{stroke,fill}%
\end{pgfscope}%
\begin{pgfscope}%
\pgfpathrectangle{\pgfqpoint{0.648703in}{0.548769in}}{\pgfqpoint{5.201297in}{3.102590in}}%
\pgfusepath{clip}%
\pgfsetbuttcap%
\pgfsetroundjoin%
\definecolor{currentfill}{rgb}{1.000000,0.498039,0.054902}%
\pgfsetfillcolor{currentfill}%
\pgfsetlinewidth{1.003750pt}%
\definecolor{currentstroke}{rgb}{1.000000,0.498039,0.054902}%
\pgfsetstrokecolor{currentstroke}%
\pgfsetdash{}{0pt}%
\pgfpathmoveto{\pgfqpoint{1.338539in}{3.145133in}}%
\pgfpathcurveto{\pgfqpoint{1.349589in}{3.145133in}}{\pgfqpoint{1.360188in}{3.149523in}}{\pgfqpoint{1.368002in}{3.157337in}}%
\pgfpathcurveto{\pgfqpoint{1.375816in}{3.165151in}}{\pgfqpoint{1.380206in}{3.175750in}}{\pgfqpoint{1.380206in}{3.186800in}}%
\pgfpathcurveto{\pgfqpoint{1.380206in}{3.197850in}}{\pgfqpoint{1.375816in}{3.208449in}}{\pgfqpoint{1.368002in}{3.216262in}}%
\pgfpathcurveto{\pgfqpoint{1.360188in}{3.224076in}}{\pgfqpoint{1.349589in}{3.228466in}}{\pgfqpoint{1.338539in}{3.228466in}}%
\pgfpathcurveto{\pgfqpoint{1.327489in}{3.228466in}}{\pgfqpoint{1.316890in}{3.224076in}}{\pgfqpoint{1.309076in}{3.216262in}}%
\pgfpathcurveto{\pgfqpoint{1.301263in}{3.208449in}}{\pgfqpoint{1.296872in}{3.197850in}}{\pgfqpoint{1.296872in}{3.186800in}}%
\pgfpathcurveto{\pgfqpoint{1.296872in}{3.175750in}}{\pgfqpoint{1.301263in}{3.165151in}}{\pgfqpoint{1.309076in}{3.157337in}}%
\pgfpathcurveto{\pgfqpoint{1.316890in}{3.149523in}}{\pgfqpoint{1.327489in}{3.145133in}}{\pgfqpoint{1.338539in}{3.145133in}}%
\pgfpathclose%
\pgfusepath{stroke,fill}%
\end{pgfscope}%
\begin{pgfscope}%
\pgfpathrectangle{\pgfqpoint{0.648703in}{0.548769in}}{\pgfqpoint{5.201297in}{3.102590in}}%
\pgfusepath{clip}%
\pgfsetbuttcap%
\pgfsetroundjoin%
\definecolor{currentfill}{rgb}{1.000000,0.498039,0.054902}%
\pgfsetfillcolor{currentfill}%
\pgfsetlinewidth{1.003750pt}%
\definecolor{currentstroke}{rgb}{1.000000,0.498039,0.054902}%
\pgfsetstrokecolor{currentstroke}%
\pgfsetdash{}{0pt}%
\pgfpathmoveto{\pgfqpoint{2.310139in}{3.145133in}}%
\pgfpathcurveto{\pgfqpoint{2.321189in}{3.145133in}}{\pgfqpoint{2.331788in}{3.149523in}}{\pgfqpoint{2.339602in}{3.157337in}}%
\pgfpathcurveto{\pgfqpoint{2.347415in}{3.165151in}}{\pgfqpoint{2.351805in}{3.175750in}}{\pgfqpoint{2.351805in}{3.186800in}}%
\pgfpathcurveto{\pgfqpoint{2.351805in}{3.197850in}}{\pgfqpoint{2.347415in}{3.208449in}}{\pgfqpoint{2.339602in}{3.216262in}}%
\pgfpathcurveto{\pgfqpoint{2.331788in}{3.224076in}}{\pgfqpoint{2.321189in}{3.228466in}}{\pgfqpoint{2.310139in}{3.228466in}}%
\pgfpathcurveto{\pgfqpoint{2.299089in}{3.228466in}}{\pgfqpoint{2.288490in}{3.224076in}}{\pgfqpoint{2.280676in}{3.216262in}}%
\pgfpathcurveto{\pgfqpoint{2.272862in}{3.208449in}}{\pgfqpoint{2.268472in}{3.197850in}}{\pgfqpoint{2.268472in}{3.186800in}}%
\pgfpathcurveto{\pgfqpoint{2.268472in}{3.175750in}}{\pgfqpoint{2.272862in}{3.165151in}}{\pgfqpoint{2.280676in}{3.157337in}}%
\pgfpathcurveto{\pgfqpoint{2.288490in}{3.149523in}}{\pgfqpoint{2.299089in}{3.145133in}}{\pgfqpoint{2.310139in}{3.145133in}}%
\pgfpathclose%
\pgfusepath{stroke,fill}%
\end{pgfscope}%
\begin{pgfscope}%
\pgfpathrectangle{\pgfqpoint{0.648703in}{0.548769in}}{\pgfqpoint{5.201297in}{3.102590in}}%
\pgfusepath{clip}%
\pgfsetbuttcap%
\pgfsetroundjoin%
\definecolor{currentfill}{rgb}{1.000000,0.498039,0.054902}%
\pgfsetfillcolor{currentfill}%
\pgfsetlinewidth{1.003750pt}%
\definecolor{currentstroke}{rgb}{1.000000,0.498039,0.054902}%
\pgfsetstrokecolor{currentstroke}%
\pgfsetdash{}{0pt}%
\pgfpathmoveto{\pgfqpoint{2.634005in}{3.136837in}}%
\pgfpathcurveto{\pgfqpoint{2.645055in}{3.136837in}}{\pgfqpoint{2.655654in}{3.141228in}}{\pgfqpoint{2.663468in}{3.149041in}}%
\pgfpathcurveto{\pgfqpoint{2.671282in}{3.156855in}}{\pgfqpoint{2.675672in}{3.167454in}}{\pgfqpoint{2.675672in}{3.178504in}}%
\pgfpathcurveto{\pgfqpoint{2.675672in}{3.189554in}}{\pgfqpoint{2.671282in}{3.200153in}}{\pgfqpoint{2.663468in}{3.207967in}}%
\pgfpathcurveto{\pgfqpoint{2.655654in}{3.215780in}}{\pgfqpoint{2.645055in}{3.220171in}}{\pgfqpoint{2.634005in}{3.220171in}}%
\pgfpathcurveto{\pgfqpoint{2.622955in}{3.220171in}}{\pgfqpoint{2.612356in}{3.215780in}}{\pgfqpoint{2.604543in}{3.207967in}}%
\pgfpathcurveto{\pgfqpoint{2.596729in}{3.200153in}}{\pgfqpoint{2.592339in}{3.189554in}}{\pgfqpoint{2.592339in}{3.178504in}}%
\pgfpathcurveto{\pgfqpoint{2.592339in}{3.167454in}}{\pgfqpoint{2.596729in}{3.156855in}}{\pgfqpoint{2.604543in}{3.149041in}}%
\pgfpathcurveto{\pgfqpoint{2.612356in}{3.141228in}}{\pgfqpoint{2.622955in}{3.136837in}}{\pgfqpoint{2.634005in}{3.136837in}}%
\pgfpathclose%
\pgfusepath{stroke,fill}%
\end{pgfscope}%
\begin{pgfscope}%
\pgfpathrectangle{\pgfqpoint{0.648703in}{0.548769in}}{\pgfqpoint{5.201297in}{3.102590in}}%
\pgfusepath{clip}%
\pgfsetbuttcap%
\pgfsetroundjoin%
\definecolor{currentfill}{rgb}{1.000000,0.498039,0.054902}%
\pgfsetfillcolor{currentfill}%
\pgfsetlinewidth{1.003750pt}%
\definecolor{currentstroke}{rgb}{1.000000,0.498039,0.054902}%
\pgfsetstrokecolor{currentstroke}%
\pgfsetdash{}{0pt}%
\pgfpathmoveto{\pgfqpoint{2.180592in}{3.468665in}}%
\pgfpathcurveto{\pgfqpoint{2.191642in}{3.468665in}}{\pgfqpoint{2.202241in}{3.473055in}}{\pgfqpoint{2.210055in}{3.480869in}}%
\pgfpathcurveto{\pgfqpoint{2.217869in}{3.488683in}}{\pgfqpoint{2.222259in}{3.499282in}}{\pgfqpoint{2.222259in}{3.510332in}}%
\pgfpathcurveto{\pgfqpoint{2.222259in}{3.521382in}}{\pgfqpoint{2.217869in}{3.531981in}}{\pgfqpoint{2.210055in}{3.539795in}}%
\pgfpathcurveto{\pgfqpoint{2.202241in}{3.547608in}}{\pgfqpoint{2.191642in}{3.551998in}}{\pgfqpoint{2.180592in}{3.551998in}}%
\pgfpathcurveto{\pgfqpoint{2.169542in}{3.551998in}}{\pgfqpoint{2.158943in}{3.547608in}}{\pgfqpoint{2.151129in}{3.539795in}}%
\pgfpathcurveto{\pgfqpoint{2.143316in}{3.531981in}}{\pgfqpoint{2.138925in}{3.521382in}}{\pgfqpoint{2.138925in}{3.510332in}}%
\pgfpathcurveto{\pgfqpoint{2.138925in}{3.499282in}}{\pgfqpoint{2.143316in}{3.488683in}}{\pgfqpoint{2.151129in}{3.480869in}}%
\pgfpathcurveto{\pgfqpoint{2.158943in}{3.473055in}}{\pgfqpoint{2.169542in}{3.468665in}}{\pgfqpoint{2.180592in}{3.468665in}}%
\pgfpathclose%
\pgfusepath{stroke,fill}%
\end{pgfscope}%
\begin{pgfscope}%
\pgfpathrectangle{\pgfqpoint{0.648703in}{0.548769in}}{\pgfqpoint{5.201297in}{3.102590in}}%
\pgfusepath{clip}%
\pgfsetbuttcap%
\pgfsetroundjoin%
\definecolor{currentfill}{rgb}{1.000000,0.498039,0.054902}%
\pgfsetfillcolor{currentfill}%
\pgfsetlinewidth{1.003750pt}%
\definecolor{currentstroke}{rgb}{1.000000,0.498039,0.054902}%
\pgfsetstrokecolor{currentstroke}%
\pgfsetdash{}{0pt}%
\pgfpathmoveto{\pgfqpoint{1.532859in}{3.136837in}}%
\pgfpathcurveto{\pgfqpoint{1.543909in}{3.136837in}}{\pgfqpoint{1.554508in}{3.141228in}}{\pgfqpoint{1.562322in}{3.149041in}}%
\pgfpathcurveto{\pgfqpoint{1.570135in}{3.156855in}}{\pgfqpoint{1.574526in}{3.167454in}}{\pgfqpoint{1.574526in}{3.178504in}}%
\pgfpathcurveto{\pgfqpoint{1.574526in}{3.189554in}}{\pgfqpoint{1.570135in}{3.200153in}}{\pgfqpoint{1.562322in}{3.207967in}}%
\pgfpathcurveto{\pgfqpoint{1.554508in}{3.215780in}}{\pgfqpoint{1.543909in}{3.220171in}}{\pgfqpoint{1.532859in}{3.220171in}}%
\pgfpathcurveto{\pgfqpoint{1.521809in}{3.220171in}}{\pgfqpoint{1.511210in}{3.215780in}}{\pgfqpoint{1.503396in}{3.207967in}}%
\pgfpathcurveto{\pgfqpoint{1.495583in}{3.200153in}}{\pgfqpoint{1.491192in}{3.189554in}}{\pgfqpoint{1.491192in}{3.178504in}}%
\pgfpathcurveto{\pgfqpoint{1.491192in}{3.167454in}}{\pgfqpoint{1.495583in}{3.156855in}}{\pgfqpoint{1.503396in}{3.149041in}}%
\pgfpathcurveto{\pgfqpoint{1.511210in}{3.141228in}}{\pgfqpoint{1.521809in}{3.136837in}}{\pgfqpoint{1.532859in}{3.136837in}}%
\pgfpathclose%
\pgfusepath{stroke,fill}%
\end{pgfscope}%
\begin{pgfscope}%
\pgfpathrectangle{\pgfqpoint{0.648703in}{0.548769in}}{\pgfqpoint{5.201297in}{3.102590in}}%
\pgfusepath{clip}%
\pgfsetbuttcap%
\pgfsetroundjoin%
\definecolor{currentfill}{rgb}{1.000000,0.498039,0.054902}%
\pgfsetfillcolor{currentfill}%
\pgfsetlinewidth{1.003750pt}%
\definecolor{currentstroke}{rgb}{1.000000,0.498039,0.054902}%
\pgfsetstrokecolor{currentstroke}%
\pgfsetdash{}{0pt}%
\pgfpathmoveto{\pgfqpoint{2.957872in}{3.136837in}}%
\pgfpathcurveto{\pgfqpoint{2.968922in}{3.136837in}}{\pgfqpoint{2.979521in}{3.141228in}}{\pgfqpoint{2.987335in}{3.149041in}}%
\pgfpathcurveto{\pgfqpoint{2.995148in}{3.156855in}}{\pgfqpoint{2.999538in}{3.167454in}}{\pgfqpoint{2.999538in}{3.178504in}}%
\pgfpathcurveto{\pgfqpoint{2.999538in}{3.189554in}}{\pgfqpoint{2.995148in}{3.200153in}}{\pgfqpoint{2.987335in}{3.207967in}}%
\pgfpathcurveto{\pgfqpoint{2.979521in}{3.215780in}}{\pgfqpoint{2.968922in}{3.220171in}}{\pgfqpoint{2.957872in}{3.220171in}}%
\pgfpathcurveto{\pgfqpoint{2.946822in}{3.220171in}}{\pgfqpoint{2.936223in}{3.215780in}}{\pgfqpoint{2.928409in}{3.207967in}}%
\pgfpathcurveto{\pgfqpoint{2.920595in}{3.200153in}}{\pgfqpoint{2.916205in}{3.189554in}}{\pgfqpoint{2.916205in}{3.178504in}}%
\pgfpathcurveto{\pgfqpoint{2.916205in}{3.167454in}}{\pgfqpoint{2.920595in}{3.156855in}}{\pgfqpoint{2.928409in}{3.149041in}}%
\pgfpathcurveto{\pgfqpoint{2.936223in}{3.141228in}}{\pgfqpoint{2.946822in}{3.136837in}}{\pgfqpoint{2.957872in}{3.136837in}}%
\pgfpathclose%
\pgfusepath{stroke,fill}%
\end{pgfscope}%
\begin{pgfscope}%
\pgfpathrectangle{\pgfqpoint{0.648703in}{0.548769in}}{\pgfqpoint{5.201297in}{3.102590in}}%
\pgfusepath{clip}%
\pgfsetbuttcap%
\pgfsetroundjoin%
\definecolor{currentfill}{rgb}{1.000000,0.498039,0.054902}%
\pgfsetfillcolor{currentfill}%
\pgfsetlinewidth{1.003750pt}%
\definecolor{currentstroke}{rgb}{1.000000,0.498039,0.054902}%
\pgfsetstrokecolor{currentstroke}%
\pgfsetdash{}{0pt}%
\pgfpathmoveto{\pgfqpoint{1.403312in}{3.136837in}}%
\pgfpathcurveto{\pgfqpoint{1.414363in}{3.136837in}}{\pgfqpoint{1.424962in}{3.141228in}}{\pgfqpoint{1.432775in}{3.149041in}}%
\pgfpathcurveto{\pgfqpoint{1.440589in}{3.156855in}}{\pgfqpoint{1.444979in}{3.167454in}}{\pgfqpoint{1.444979in}{3.178504in}}%
\pgfpathcurveto{\pgfqpoint{1.444979in}{3.189554in}}{\pgfqpoint{1.440589in}{3.200153in}}{\pgfqpoint{1.432775in}{3.207967in}}%
\pgfpathcurveto{\pgfqpoint{1.424962in}{3.215780in}}{\pgfqpoint{1.414363in}{3.220171in}}{\pgfqpoint{1.403312in}{3.220171in}}%
\pgfpathcurveto{\pgfqpoint{1.392262in}{3.220171in}}{\pgfqpoint{1.381663in}{3.215780in}}{\pgfqpoint{1.373850in}{3.207967in}}%
\pgfpathcurveto{\pgfqpoint{1.366036in}{3.200153in}}{\pgfqpoint{1.361646in}{3.189554in}}{\pgfqpoint{1.361646in}{3.178504in}}%
\pgfpathcurveto{\pgfqpoint{1.361646in}{3.167454in}}{\pgfqpoint{1.366036in}{3.156855in}}{\pgfqpoint{1.373850in}{3.149041in}}%
\pgfpathcurveto{\pgfqpoint{1.381663in}{3.141228in}}{\pgfqpoint{1.392262in}{3.136837in}}{\pgfqpoint{1.403312in}{3.136837in}}%
\pgfpathclose%
\pgfusepath{stroke,fill}%
\end{pgfscope}%
\begin{pgfscope}%
\pgfpathrectangle{\pgfqpoint{0.648703in}{0.548769in}}{\pgfqpoint{5.201297in}{3.102590in}}%
\pgfusepath{clip}%
\pgfsetbuttcap%
\pgfsetroundjoin%
\definecolor{currentfill}{rgb}{0.839216,0.152941,0.156863}%
\pgfsetfillcolor{currentfill}%
\pgfsetlinewidth{1.003750pt}%
\definecolor{currentstroke}{rgb}{0.839216,0.152941,0.156863}%
\pgfsetstrokecolor{currentstroke}%
\pgfsetdash{}{0pt}%
\pgfpathmoveto{\pgfqpoint{2.439685in}{3.136837in}}%
\pgfpathcurveto{\pgfqpoint{2.450735in}{3.136837in}}{\pgfqpoint{2.461335in}{3.141228in}}{\pgfqpoint{2.469148in}{3.149041in}}%
\pgfpathcurveto{\pgfqpoint{2.476962in}{3.156855in}}{\pgfqpoint{2.481352in}{3.167454in}}{\pgfqpoint{2.481352in}{3.178504in}}%
\pgfpathcurveto{\pgfqpoint{2.481352in}{3.189554in}}{\pgfqpoint{2.476962in}{3.200153in}}{\pgfqpoint{2.469148in}{3.207967in}}%
\pgfpathcurveto{\pgfqpoint{2.461335in}{3.215780in}}{\pgfqpoint{2.450735in}{3.220171in}}{\pgfqpoint{2.439685in}{3.220171in}}%
\pgfpathcurveto{\pgfqpoint{2.428635in}{3.220171in}}{\pgfqpoint{2.418036in}{3.215780in}}{\pgfqpoint{2.410223in}{3.207967in}}%
\pgfpathcurveto{\pgfqpoint{2.402409in}{3.200153in}}{\pgfqpoint{2.398019in}{3.189554in}}{\pgfqpoint{2.398019in}{3.178504in}}%
\pgfpathcurveto{\pgfqpoint{2.398019in}{3.167454in}}{\pgfqpoint{2.402409in}{3.156855in}}{\pgfqpoint{2.410223in}{3.149041in}}%
\pgfpathcurveto{\pgfqpoint{2.418036in}{3.141228in}}{\pgfqpoint{2.428635in}{3.136837in}}{\pgfqpoint{2.439685in}{3.136837in}}%
\pgfpathclose%
\pgfusepath{stroke,fill}%
\end{pgfscope}%
\begin{pgfscope}%
\pgfpathrectangle{\pgfqpoint{0.648703in}{0.548769in}}{\pgfqpoint{5.201297in}{3.102590in}}%
\pgfusepath{clip}%
\pgfsetbuttcap%
\pgfsetroundjoin%
\definecolor{currentfill}{rgb}{1.000000,0.498039,0.054902}%
\pgfsetfillcolor{currentfill}%
\pgfsetlinewidth{1.003750pt}%
\definecolor{currentstroke}{rgb}{1.000000,0.498039,0.054902}%
\pgfsetstrokecolor{currentstroke}%
\pgfsetdash{}{0pt}%
\pgfpathmoveto{\pgfqpoint{1.014673in}{3.149281in}}%
\pgfpathcurveto{\pgfqpoint{1.025723in}{3.149281in}}{\pgfqpoint{1.036322in}{3.153671in}}{\pgfqpoint{1.044135in}{3.161485in}}%
\pgfpathcurveto{\pgfqpoint{1.051949in}{3.169298in}}{\pgfqpoint{1.056339in}{3.179897in}}{\pgfqpoint{1.056339in}{3.190948in}}%
\pgfpathcurveto{\pgfqpoint{1.056339in}{3.201998in}}{\pgfqpoint{1.051949in}{3.212597in}}{\pgfqpoint{1.044135in}{3.220410in}}%
\pgfpathcurveto{\pgfqpoint{1.036322in}{3.228224in}}{\pgfqpoint{1.025723in}{3.232614in}}{\pgfqpoint{1.014673in}{3.232614in}}%
\pgfpathcurveto{\pgfqpoint{1.003622in}{3.232614in}}{\pgfqpoint{0.993023in}{3.228224in}}{\pgfqpoint{0.985210in}{3.220410in}}%
\pgfpathcurveto{\pgfqpoint{0.977396in}{3.212597in}}{\pgfqpoint{0.973006in}{3.201998in}}{\pgfqpoint{0.973006in}{3.190948in}}%
\pgfpathcurveto{\pgfqpoint{0.973006in}{3.179897in}}{\pgfqpoint{0.977396in}{3.169298in}}{\pgfqpoint{0.985210in}{3.161485in}}%
\pgfpathcurveto{\pgfqpoint{0.993023in}{3.153671in}}{\pgfqpoint{1.003622in}{3.149281in}}{\pgfqpoint{1.014673in}{3.149281in}}%
\pgfpathclose%
\pgfusepath{stroke,fill}%
\end{pgfscope}%
\begin{pgfscope}%
\pgfpathrectangle{\pgfqpoint{0.648703in}{0.548769in}}{\pgfqpoint{5.201297in}{3.102590in}}%
\pgfusepath{clip}%
\pgfsetbuttcap%
\pgfsetroundjoin%
\definecolor{currentfill}{rgb}{1.000000,0.498039,0.054902}%
\pgfsetfillcolor{currentfill}%
\pgfsetlinewidth{1.003750pt}%
\definecolor{currentstroke}{rgb}{1.000000,0.498039,0.054902}%
\pgfsetstrokecolor{currentstroke}%
\pgfsetdash{}{0pt}%
\pgfpathmoveto{\pgfqpoint{2.634005in}{3.286160in}}%
\pgfpathcurveto{\pgfqpoint{2.645055in}{3.286160in}}{\pgfqpoint{2.655654in}{3.290550in}}{\pgfqpoint{2.663468in}{3.298364in}}%
\pgfpathcurveto{\pgfqpoint{2.671282in}{3.306177in}}{\pgfqpoint{2.675672in}{3.316776in}}{\pgfqpoint{2.675672in}{3.327827in}}%
\pgfpathcurveto{\pgfqpoint{2.675672in}{3.338877in}}{\pgfqpoint{2.671282in}{3.349476in}}{\pgfqpoint{2.663468in}{3.357289in}}%
\pgfpathcurveto{\pgfqpoint{2.655654in}{3.365103in}}{\pgfqpoint{2.645055in}{3.369493in}}{\pgfqpoint{2.634005in}{3.369493in}}%
\pgfpathcurveto{\pgfqpoint{2.622955in}{3.369493in}}{\pgfqpoint{2.612356in}{3.365103in}}{\pgfqpoint{2.604543in}{3.357289in}}%
\pgfpathcurveto{\pgfqpoint{2.596729in}{3.349476in}}{\pgfqpoint{2.592339in}{3.338877in}}{\pgfqpoint{2.592339in}{3.327827in}}%
\pgfpathcurveto{\pgfqpoint{2.592339in}{3.316776in}}{\pgfqpoint{2.596729in}{3.306177in}}{\pgfqpoint{2.604543in}{3.298364in}}%
\pgfpathcurveto{\pgfqpoint{2.612356in}{3.290550in}}{\pgfqpoint{2.622955in}{3.286160in}}{\pgfqpoint{2.634005in}{3.286160in}}%
\pgfpathclose%
\pgfusepath{stroke,fill}%
\end{pgfscope}%
\begin{pgfscope}%
\pgfpathrectangle{\pgfqpoint{0.648703in}{0.548769in}}{\pgfqpoint{5.201297in}{3.102590in}}%
\pgfusepath{clip}%
\pgfsetbuttcap%
\pgfsetroundjoin%
\definecolor{currentfill}{rgb}{1.000000,0.498039,0.054902}%
\pgfsetfillcolor{currentfill}%
\pgfsetlinewidth{1.003750pt}%
\definecolor{currentstroke}{rgb}{1.000000,0.498039,0.054902}%
\pgfsetstrokecolor{currentstroke}%
\pgfsetdash{}{0pt}%
\pgfpathmoveto{\pgfqpoint{1.921499in}{3.136837in}}%
\pgfpathcurveto{\pgfqpoint{1.932549in}{3.136837in}}{\pgfqpoint{1.943148in}{3.141228in}}{\pgfqpoint{1.950962in}{3.149041in}}%
\pgfpathcurveto{\pgfqpoint{1.958775in}{3.156855in}}{\pgfqpoint{1.963166in}{3.167454in}}{\pgfqpoint{1.963166in}{3.178504in}}%
\pgfpathcurveto{\pgfqpoint{1.963166in}{3.189554in}}{\pgfqpoint{1.958775in}{3.200153in}}{\pgfqpoint{1.950962in}{3.207967in}}%
\pgfpathcurveto{\pgfqpoint{1.943148in}{3.215780in}}{\pgfqpoint{1.932549in}{3.220171in}}{\pgfqpoint{1.921499in}{3.220171in}}%
\pgfpathcurveto{\pgfqpoint{1.910449in}{3.220171in}}{\pgfqpoint{1.899850in}{3.215780in}}{\pgfqpoint{1.892036in}{3.207967in}}%
\pgfpathcurveto{\pgfqpoint{1.884223in}{3.200153in}}{\pgfqpoint{1.879832in}{3.189554in}}{\pgfqpoint{1.879832in}{3.178504in}}%
\pgfpathcurveto{\pgfqpoint{1.879832in}{3.167454in}}{\pgfqpoint{1.884223in}{3.156855in}}{\pgfqpoint{1.892036in}{3.149041in}}%
\pgfpathcurveto{\pgfqpoint{1.899850in}{3.141228in}}{\pgfqpoint{1.910449in}{3.136837in}}{\pgfqpoint{1.921499in}{3.136837in}}%
\pgfpathclose%
\pgfusepath{stroke,fill}%
\end{pgfscope}%
\begin{pgfscope}%
\pgfpathrectangle{\pgfqpoint{0.648703in}{0.548769in}}{\pgfqpoint{5.201297in}{3.102590in}}%
\pgfusepath{clip}%
\pgfsetbuttcap%
\pgfsetroundjoin%
\definecolor{currentfill}{rgb}{1.000000,0.498039,0.054902}%
\pgfsetfillcolor{currentfill}%
\pgfsetlinewidth{1.003750pt}%
\definecolor{currentstroke}{rgb}{1.000000,0.498039,0.054902}%
\pgfsetstrokecolor{currentstroke}%
\pgfsetdash{}{0pt}%
\pgfpathmoveto{\pgfqpoint{1.273766in}{3.149281in}}%
\pgfpathcurveto{\pgfqpoint{1.284816in}{3.149281in}}{\pgfqpoint{1.295415in}{3.153671in}}{\pgfqpoint{1.303229in}{3.161485in}}%
\pgfpathcurveto{\pgfqpoint{1.311042in}{3.169298in}}{\pgfqpoint{1.315432in}{3.179897in}}{\pgfqpoint{1.315432in}{3.190948in}}%
\pgfpathcurveto{\pgfqpoint{1.315432in}{3.201998in}}{\pgfqpoint{1.311042in}{3.212597in}}{\pgfqpoint{1.303229in}{3.220410in}}%
\pgfpathcurveto{\pgfqpoint{1.295415in}{3.228224in}}{\pgfqpoint{1.284816in}{3.232614in}}{\pgfqpoint{1.273766in}{3.232614in}}%
\pgfpathcurveto{\pgfqpoint{1.262716in}{3.232614in}}{\pgfqpoint{1.252117in}{3.228224in}}{\pgfqpoint{1.244303in}{3.220410in}}%
\pgfpathcurveto{\pgfqpoint{1.236489in}{3.212597in}}{\pgfqpoint{1.232099in}{3.201998in}}{\pgfqpoint{1.232099in}{3.190948in}}%
\pgfpathcurveto{\pgfqpoint{1.232099in}{3.179897in}}{\pgfqpoint{1.236489in}{3.169298in}}{\pgfqpoint{1.244303in}{3.161485in}}%
\pgfpathcurveto{\pgfqpoint{1.252117in}{3.153671in}}{\pgfqpoint{1.262716in}{3.149281in}}{\pgfqpoint{1.273766in}{3.149281in}}%
\pgfpathclose%
\pgfusepath{stroke,fill}%
\end{pgfscope}%
\begin{pgfscope}%
\pgfpathrectangle{\pgfqpoint{0.648703in}{0.548769in}}{\pgfqpoint{5.201297in}{3.102590in}}%
\pgfusepath{clip}%
\pgfsetbuttcap%
\pgfsetroundjoin%
\definecolor{currentfill}{rgb}{0.121569,0.466667,0.705882}%
\pgfsetfillcolor{currentfill}%
\pgfsetlinewidth{1.003750pt}%
\definecolor{currentstroke}{rgb}{0.121569,0.466667,0.705882}%
\pgfsetstrokecolor{currentstroke}%
\pgfsetdash{}{0pt}%
\pgfpathmoveto{\pgfqpoint{2.763552in}{3.128542in}}%
\pgfpathcurveto{\pgfqpoint{2.774602in}{3.128542in}}{\pgfqpoint{2.785201in}{3.132932in}}{\pgfqpoint{2.793015in}{3.140746in}}%
\pgfpathcurveto{\pgfqpoint{2.800828in}{3.148559in}}{\pgfqpoint{2.805219in}{3.159158in}}{\pgfqpoint{2.805219in}{3.170208in}}%
\pgfpathcurveto{\pgfqpoint{2.805219in}{3.181258in}}{\pgfqpoint{2.800828in}{3.191857in}}{\pgfqpoint{2.793015in}{3.199671in}}%
\pgfpathcurveto{\pgfqpoint{2.785201in}{3.207485in}}{\pgfqpoint{2.774602in}{3.211875in}}{\pgfqpoint{2.763552in}{3.211875in}}%
\pgfpathcurveto{\pgfqpoint{2.752502in}{3.211875in}}{\pgfqpoint{2.741903in}{3.207485in}}{\pgfqpoint{2.734089in}{3.199671in}}%
\pgfpathcurveto{\pgfqpoint{2.726275in}{3.191857in}}{\pgfqpoint{2.721885in}{3.181258in}}{\pgfqpoint{2.721885in}{3.170208in}}%
\pgfpathcurveto{\pgfqpoint{2.721885in}{3.159158in}}{\pgfqpoint{2.726275in}{3.148559in}}{\pgfqpoint{2.734089in}{3.140746in}}%
\pgfpathcurveto{\pgfqpoint{2.741903in}{3.132932in}}{\pgfqpoint{2.752502in}{3.128542in}}{\pgfqpoint{2.763552in}{3.128542in}}%
\pgfpathclose%
\pgfusepath{stroke,fill}%
\end{pgfscope}%
\begin{pgfscope}%
\pgfpathrectangle{\pgfqpoint{0.648703in}{0.548769in}}{\pgfqpoint{5.201297in}{3.102590in}}%
\pgfusepath{clip}%
\pgfsetbuttcap%
\pgfsetroundjoin%
\definecolor{currentfill}{rgb}{1.000000,0.498039,0.054902}%
\pgfsetfillcolor{currentfill}%
\pgfsetlinewidth{1.003750pt}%
\definecolor{currentstroke}{rgb}{1.000000,0.498039,0.054902}%
\pgfsetstrokecolor{currentstroke}%
\pgfsetdash{}{0pt}%
\pgfpathmoveto{\pgfqpoint{2.828325in}{3.232238in}}%
\pgfpathcurveto{\pgfqpoint{2.839375in}{3.232238in}}{\pgfqpoint{2.849974in}{3.236628in}}{\pgfqpoint{2.857788in}{3.244442in}}%
\pgfpathcurveto{\pgfqpoint{2.865602in}{3.252255in}}{\pgfqpoint{2.869992in}{3.262854in}}{\pgfqpoint{2.869992in}{3.273905in}}%
\pgfpathcurveto{\pgfqpoint{2.869992in}{3.284955in}}{\pgfqpoint{2.865602in}{3.295554in}}{\pgfqpoint{2.857788in}{3.303367in}}%
\pgfpathcurveto{\pgfqpoint{2.849974in}{3.311181in}}{\pgfqpoint{2.839375in}{3.315571in}}{\pgfqpoint{2.828325in}{3.315571in}}%
\pgfpathcurveto{\pgfqpoint{2.817275in}{3.315571in}}{\pgfqpoint{2.806676in}{3.311181in}}{\pgfqpoint{2.798862in}{3.303367in}}%
\pgfpathcurveto{\pgfqpoint{2.791049in}{3.295554in}}{\pgfqpoint{2.786659in}{3.284955in}}{\pgfqpoint{2.786659in}{3.273905in}}%
\pgfpathcurveto{\pgfqpoint{2.786659in}{3.262854in}}{\pgfqpoint{2.791049in}{3.252255in}}{\pgfqpoint{2.798862in}{3.244442in}}%
\pgfpathcurveto{\pgfqpoint{2.806676in}{3.236628in}}{\pgfqpoint{2.817275in}{3.232238in}}{\pgfqpoint{2.828325in}{3.232238in}}%
\pgfpathclose%
\pgfusepath{stroke,fill}%
\end{pgfscope}%
\begin{pgfscope}%
\pgfpathrectangle{\pgfqpoint{0.648703in}{0.548769in}}{\pgfqpoint{5.201297in}{3.102590in}}%
\pgfusepath{clip}%
\pgfsetbuttcap%
\pgfsetroundjoin%
\definecolor{currentfill}{rgb}{0.121569,0.466667,0.705882}%
\pgfsetfillcolor{currentfill}%
\pgfsetlinewidth{1.003750pt}%
\definecolor{currentstroke}{rgb}{0.121569,0.466667,0.705882}%
\pgfsetstrokecolor{currentstroke}%
\pgfsetdash{}{0pt}%
\pgfpathmoveto{\pgfqpoint{1.662406in}{0.648129in}}%
\pgfpathcurveto{\pgfqpoint{1.673456in}{0.648129in}}{\pgfqpoint{1.684055in}{0.652519in}}{\pgfqpoint{1.691868in}{0.660333in}}%
\pgfpathcurveto{\pgfqpoint{1.699682in}{0.668146in}}{\pgfqpoint{1.704072in}{0.678745in}}{\pgfqpoint{1.704072in}{0.689796in}}%
\pgfpathcurveto{\pgfqpoint{1.704072in}{0.700846in}}{\pgfqpoint{1.699682in}{0.711445in}}{\pgfqpoint{1.691868in}{0.719258in}}%
\pgfpathcurveto{\pgfqpoint{1.684055in}{0.727072in}}{\pgfqpoint{1.673456in}{0.731462in}}{\pgfqpoint{1.662406in}{0.731462in}}%
\pgfpathcurveto{\pgfqpoint{1.651356in}{0.731462in}}{\pgfqpoint{1.640757in}{0.727072in}}{\pgfqpoint{1.632943in}{0.719258in}}%
\pgfpathcurveto{\pgfqpoint{1.625129in}{0.711445in}}{\pgfqpoint{1.620739in}{0.700846in}}{\pgfqpoint{1.620739in}{0.689796in}}%
\pgfpathcurveto{\pgfqpoint{1.620739in}{0.678745in}}{\pgfqpoint{1.625129in}{0.668146in}}{\pgfqpoint{1.632943in}{0.660333in}}%
\pgfpathcurveto{\pgfqpoint{1.640757in}{0.652519in}}{\pgfqpoint{1.651356in}{0.648129in}}{\pgfqpoint{1.662406in}{0.648129in}}%
\pgfpathclose%
\pgfusepath{stroke,fill}%
\end{pgfscope}%
\begin{pgfscope}%
\pgfpathrectangle{\pgfqpoint{0.648703in}{0.548769in}}{\pgfqpoint{5.201297in}{3.102590in}}%
\pgfusepath{clip}%
\pgfsetbuttcap%
\pgfsetroundjoin%
\definecolor{currentfill}{rgb}{1.000000,0.498039,0.054902}%
\pgfsetfillcolor{currentfill}%
\pgfsetlinewidth{1.003750pt}%
\definecolor{currentstroke}{rgb}{1.000000,0.498039,0.054902}%
\pgfsetstrokecolor{currentstroke}%
\pgfsetdash{}{0pt}%
\pgfpathmoveto{\pgfqpoint{1.727179in}{3.136837in}}%
\pgfpathcurveto{\pgfqpoint{1.738229in}{3.136837in}}{\pgfqpoint{1.748828in}{3.141228in}}{\pgfqpoint{1.756642in}{3.149041in}}%
\pgfpathcurveto{\pgfqpoint{1.764455in}{3.156855in}}{\pgfqpoint{1.768846in}{3.167454in}}{\pgfqpoint{1.768846in}{3.178504in}}%
\pgfpathcurveto{\pgfqpoint{1.768846in}{3.189554in}}{\pgfqpoint{1.764455in}{3.200153in}}{\pgfqpoint{1.756642in}{3.207967in}}%
\pgfpathcurveto{\pgfqpoint{1.748828in}{3.215780in}}{\pgfqpoint{1.738229in}{3.220171in}}{\pgfqpoint{1.727179in}{3.220171in}}%
\pgfpathcurveto{\pgfqpoint{1.716129in}{3.220171in}}{\pgfqpoint{1.705530in}{3.215780in}}{\pgfqpoint{1.697716in}{3.207967in}}%
\pgfpathcurveto{\pgfqpoint{1.689903in}{3.200153in}}{\pgfqpoint{1.685512in}{3.189554in}}{\pgfqpoint{1.685512in}{3.178504in}}%
\pgfpathcurveto{\pgfqpoint{1.685512in}{3.167454in}}{\pgfqpoint{1.689903in}{3.156855in}}{\pgfqpoint{1.697716in}{3.149041in}}%
\pgfpathcurveto{\pgfqpoint{1.705530in}{3.141228in}}{\pgfqpoint{1.716129in}{3.136837in}}{\pgfqpoint{1.727179in}{3.136837in}}%
\pgfpathclose%
\pgfusepath{stroke,fill}%
\end{pgfscope}%
\begin{pgfscope}%
\pgfpathrectangle{\pgfqpoint{0.648703in}{0.548769in}}{\pgfqpoint{5.201297in}{3.102590in}}%
\pgfusepath{clip}%
\pgfsetbuttcap%
\pgfsetroundjoin%
\definecolor{currentfill}{rgb}{1.000000,0.498039,0.054902}%
\pgfsetfillcolor{currentfill}%
\pgfsetlinewidth{1.003750pt}%
\definecolor{currentstroke}{rgb}{1.000000,0.498039,0.054902}%
\pgfsetstrokecolor{currentstroke}%
\pgfsetdash{}{0pt}%
\pgfpathmoveto{\pgfqpoint{0.885126in}{3.140985in}}%
\pgfpathcurveto{\pgfqpoint{0.896176in}{3.140985in}}{\pgfqpoint{0.906775in}{3.145375in}}{\pgfqpoint{0.914589in}{3.153189in}}%
\pgfpathcurveto{\pgfqpoint{0.922402in}{3.161003in}}{\pgfqpoint{0.926793in}{3.171602in}}{\pgfqpoint{0.926793in}{3.182652in}}%
\pgfpathcurveto{\pgfqpoint{0.926793in}{3.193702in}}{\pgfqpoint{0.922402in}{3.204301in}}{\pgfqpoint{0.914589in}{3.212115in}}%
\pgfpathcurveto{\pgfqpoint{0.906775in}{3.219928in}}{\pgfqpoint{0.896176in}{3.224319in}}{\pgfqpoint{0.885126in}{3.224319in}}%
\pgfpathcurveto{\pgfqpoint{0.874076in}{3.224319in}}{\pgfqpoint{0.863477in}{3.219928in}}{\pgfqpoint{0.855663in}{3.212115in}}%
\pgfpathcurveto{\pgfqpoint{0.847850in}{3.204301in}}{\pgfqpoint{0.843459in}{3.193702in}}{\pgfqpoint{0.843459in}{3.182652in}}%
\pgfpathcurveto{\pgfqpoint{0.843459in}{3.171602in}}{\pgfqpoint{0.847850in}{3.161003in}}{\pgfqpoint{0.855663in}{3.153189in}}%
\pgfpathcurveto{\pgfqpoint{0.863477in}{3.145375in}}{\pgfqpoint{0.874076in}{3.140985in}}{\pgfqpoint{0.885126in}{3.140985in}}%
\pgfpathclose%
\pgfusepath{stroke,fill}%
\end{pgfscope}%
\begin{pgfscope}%
\pgfpathrectangle{\pgfqpoint{0.648703in}{0.548769in}}{\pgfqpoint{5.201297in}{3.102590in}}%
\pgfusepath{clip}%
\pgfsetbuttcap%
\pgfsetroundjoin%
\definecolor{currentfill}{rgb}{1.000000,0.498039,0.054902}%
\pgfsetfillcolor{currentfill}%
\pgfsetlinewidth{1.003750pt}%
\definecolor{currentstroke}{rgb}{1.000000,0.498039,0.054902}%
\pgfsetstrokecolor{currentstroke}%
\pgfsetdash{}{0pt}%
\pgfpathmoveto{\pgfqpoint{1.144219in}{3.149281in}}%
\pgfpathcurveto{\pgfqpoint{1.155269in}{3.149281in}}{\pgfqpoint{1.165868in}{3.153671in}}{\pgfqpoint{1.173682in}{3.161485in}}%
\pgfpathcurveto{\pgfqpoint{1.181496in}{3.169298in}}{\pgfqpoint{1.185886in}{3.179897in}}{\pgfqpoint{1.185886in}{3.190948in}}%
\pgfpathcurveto{\pgfqpoint{1.185886in}{3.201998in}}{\pgfqpoint{1.181496in}{3.212597in}}{\pgfqpoint{1.173682in}{3.220410in}}%
\pgfpathcurveto{\pgfqpoint{1.165868in}{3.228224in}}{\pgfqpoint{1.155269in}{3.232614in}}{\pgfqpoint{1.144219in}{3.232614in}}%
\pgfpathcurveto{\pgfqpoint{1.133169in}{3.232614in}}{\pgfqpoint{1.122570in}{3.228224in}}{\pgfqpoint{1.114756in}{3.220410in}}%
\pgfpathcurveto{\pgfqpoint{1.106943in}{3.212597in}}{\pgfqpoint{1.102553in}{3.201998in}}{\pgfqpoint{1.102553in}{3.190948in}}%
\pgfpathcurveto{\pgfqpoint{1.102553in}{3.179897in}}{\pgfqpoint{1.106943in}{3.169298in}}{\pgfqpoint{1.114756in}{3.161485in}}%
\pgfpathcurveto{\pgfqpoint{1.122570in}{3.153671in}}{\pgfqpoint{1.133169in}{3.149281in}}{\pgfqpoint{1.144219in}{3.149281in}}%
\pgfpathclose%
\pgfusepath{stroke,fill}%
\end{pgfscope}%
\begin{pgfscope}%
\pgfpathrectangle{\pgfqpoint{0.648703in}{0.548769in}}{\pgfqpoint{5.201297in}{3.102590in}}%
\pgfusepath{clip}%
\pgfsetbuttcap%
\pgfsetroundjoin%
\definecolor{currentfill}{rgb}{1.000000,0.498039,0.054902}%
\pgfsetfillcolor{currentfill}%
\pgfsetlinewidth{1.003750pt}%
\definecolor{currentstroke}{rgb}{1.000000,0.498039,0.054902}%
\pgfsetstrokecolor{currentstroke}%
\pgfsetdash{}{0pt}%
\pgfpathmoveto{\pgfqpoint{2.115819in}{3.140985in}}%
\pgfpathcurveto{\pgfqpoint{2.126869in}{3.140985in}}{\pgfqpoint{2.137468in}{3.145375in}}{\pgfqpoint{2.145282in}{3.153189in}}%
\pgfpathcurveto{\pgfqpoint{2.153095in}{3.161003in}}{\pgfqpoint{2.157485in}{3.171602in}}{\pgfqpoint{2.157485in}{3.182652in}}%
\pgfpathcurveto{\pgfqpoint{2.157485in}{3.193702in}}{\pgfqpoint{2.153095in}{3.204301in}}{\pgfqpoint{2.145282in}{3.212115in}}%
\pgfpathcurveto{\pgfqpoint{2.137468in}{3.219928in}}{\pgfqpoint{2.126869in}{3.224319in}}{\pgfqpoint{2.115819in}{3.224319in}}%
\pgfpathcurveto{\pgfqpoint{2.104769in}{3.224319in}}{\pgfqpoint{2.094170in}{3.219928in}}{\pgfqpoint{2.086356in}{3.212115in}}%
\pgfpathcurveto{\pgfqpoint{2.078542in}{3.204301in}}{\pgfqpoint{2.074152in}{3.193702in}}{\pgfqpoint{2.074152in}{3.182652in}}%
\pgfpathcurveto{\pgfqpoint{2.074152in}{3.171602in}}{\pgfqpoint{2.078542in}{3.161003in}}{\pgfqpoint{2.086356in}{3.153189in}}%
\pgfpathcurveto{\pgfqpoint{2.094170in}{3.145375in}}{\pgfqpoint{2.104769in}{3.140985in}}{\pgfqpoint{2.115819in}{3.140985in}}%
\pgfpathclose%
\pgfusepath{stroke,fill}%
\end{pgfscope}%
\begin{pgfscope}%
\pgfpathrectangle{\pgfqpoint{0.648703in}{0.548769in}}{\pgfqpoint{5.201297in}{3.102590in}}%
\pgfusepath{clip}%
\pgfsetbuttcap%
\pgfsetroundjoin%
\definecolor{currentfill}{rgb}{1.000000,0.498039,0.054902}%
\pgfsetfillcolor{currentfill}%
\pgfsetlinewidth{1.003750pt}%
\definecolor{currentstroke}{rgb}{1.000000,0.498039,0.054902}%
\pgfsetstrokecolor{currentstroke}%
\pgfsetdash{}{0pt}%
\pgfpathmoveto{\pgfqpoint{2.051046in}{3.136837in}}%
\pgfpathcurveto{\pgfqpoint{2.062096in}{3.136837in}}{\pgfqpoint{2.072695in}{3.141228in}}{\pgfqpoint{2.080508in}{3.149041in}}%
\pgfpathcurveto{\pgfqpoint{2.088322in}{3.156855in}}{\pgfqpoint{2.092712in}{3.167454in}}{\pgfqpoint{2.092712in}{3.178504in}}%
\pgfpathcurveto{\pgfqpoint{2.092712in}{3.189554in}}{\pgfqpoint{2.088322in}{3.200153in}}{\pgfqpoint{2.080508in}{3.207967in}}%
\pgfpathcurveto{\pgfqpoint{2.072695in}{3.215780in}}{\pgfqpoint{2.062096in}{3.220171in}}{\pgfqpoint{2.051046in}{3.220171in}}%
\pgfpathcurveto{\pgfqpoint{2.039995in}{3.220171in}}{\pgfqpoint{2.029396in}{3.215780in}}{\pgfqpoint{2.021583in}{3.207967in}}%
\pgfpathcurveto{\pgfqpoint{2.013769in}{3.200153in}}{\pgfqpoint{2.009379in}{3.189554in}}{\pgfqpoint{2.009379in}{3.178504in}}%
\pgfpathcurveto{\pgfqpoint{2.009379in}{3.167454in}}{\pgfqpoint{2.013769in}{3.156855in}}{\pgfqpoint{2.021583in}{3.149041in}}%
\pgfpathcurveto{\pgfqpoint{2.029396in}{3.141228in}}{\pgfqpoint{2.039995in}{3.136837in}}{\pgfqpoint{2.051046in}{3.136837in}}%
\pgfpathclose%
\pgfusepath{stroke,fill}%
\end{pgfscope}%
\begin{pgfscope}%
\pgfpathrectangle{\pgfqpoint{0.648703in}{0.548769in}}{\pgfqpoint{5.201297in}{3.102590in}}%
\pgfusepath{clip}%
\pgfsetbuttcap%
\pgfsetroundjoin%
\definecolor{currentfill}{rgb}{1.000000,0.498039,0.054902}%
\pgfsetfillcolor{currentfill}%
\pgfsetlinewidth{1.003750pt}%
\definecolor{currentstroke}{rgb}{1.000000,0.498039,0.054902}%
\pgfsetstrokecolor{currentstroke}%
\pgfsetdash{}{0pt}%
\pgfpathmoveto{\pgfqpoint{2.051046in}{3.145133in}}%
\pgfpathcurveto{\pgfqpoint{2.062096in}{3.145133in}}{\pgfqpoint{2.072695in}{3.149523in}}{\pgfqpoint{2.080508in}{3.157337in}}%
\pgfpathcurveto{\pgfqpoint{2.088322in}{3.165151in}}{\pgfqpoint{2.092712in}{3.175750in}}{\pgfqpoint{2.092712in}{3.186800in}}%
\pgfpathcurveto{\pgfqpoint{2.092712in}{3.197850in}}{\pgfqpoint{2.088322in}{3.208449in}}{\pgfqpoint{2.080508in}{3.216262in}}%
\pgfpathcurveto{\pgfqpoint{2.072695in}{3.224076in}}{\pgfqpoint{2.062096in}{3.228466in}}{\pgfqpoint{2.051046in}{3.228466in}}%
\pgfpathcurveto{\pgfqpoint{2.039995in}{3.228466in}}{\pgfqpoint{2.029396in}{3.224076in}}{\pgfqpoint{2.021583in}{3.216262in}}%
\pgfpathcurveto{\pgfqpoint{2.013769in}{3.208449in}}{\pgfqpoint{2.009379in}{3.197850in}}{\pgfqpoint{2.009379in}{3.186800in}}%
\pgfpathcurveto{\pgfqpoint{2.009379in}{3.175750in}}{\pgfqpoint{2.013769in}{3.165151in}}{\pgfqpoint{2.021583in}{3.157337in}}%
\pgfpathcurveto{\pgfqpoint{2.029396in}{3.149523in}}{\pgfqpoint{2.039995in}{3.145133in}}{\pgfqpoint{2.051046in}{3.145133in}}%
\pgfpathclose%
\pgfusepath{stroke,fill}%
\end{pgfscope}%
\begin{pgfscope}%
\pgfpathrectangle{\pgfqpoint{0.648703in}{0.548769in}}{\pgfqpoint{5.201297in}{3.102590in}}%
\pgfusepath{clip}%
\pgfsetbuttcap%
\pgfsetroundjoin%
\definecolor{currentfill}{rgb}{0.121569,0.466667,0.705882}%
\pgfsetfillcolor{currentfill}%
\pgfsetlinewidth{1.003750pt}%
\definecolor{currentstroke}{rgb}{0.121569,0.466667,0.705882}%
\pgfsetstrokecolor{currentstroke}%
\pgfsetdash{}{0pt}%
\pgfpathmoveto{\pgfqpoint{3.022645in}{3.132690in}}%
\pgfpathcurveto{\pgfqpoint{3.033695in}{3.132690in}}{\pgfqpoint{3.044294in}{3.137080in}}{\pgfqpoint{3.052108in}{3.144893in}}%
\pgfpathcurveto{\pgfqpoint{3.059922in}{3.152707in}}{\pgfqpoint{3.064312in}{3.163306in}}{\pgfqpoint{3.064312in}{3.174356in}}%
\pgfpathcurveto{\pgfqpoint{3.064312in}{3.185406in}}{\pgfqpoint{3.059922in}{3.196005in}}{\pgfqpoint{3.052108in}{3.203819in}}%
\pgfpathcurveto{\pgfqpoint{3.044294in}{3.211633in}}{\pgfqpoint{3.033695in}{3.216023in}}{\pgfqpoint{3.022645in}{3.216023in}}%
\pgfpathcurveto{\pgfqpoint{3.011595in}{3.216023in}}{\pgfqpoint{3.000996in}{3.211633in}}{\pgfqpoint{2.993182in}{3.203819in}}%
\pgfpathcurveto{\pgfqpoint{2.985369in}{3.196005in}}{\pgfqpoint{2.980978in}{3.185406in}}{\pgfqpoint{2.980978in}{3.174356in}}%
\pgfpathcurveto{\pgfqpoint{2.980978in}{3.163306in}}{\pgfqpoint{2.985369in}{3.152707in}}{\pgfqpoint{2.993182in}{3.144893in}}%
\pgfpathcurveto{\pgfqpoint{3.000996in}{3.137080in}}{\pgfqpoint{3.011595in}{3.132690in}}{\pgfqpoint{3.022645in}{3.132690in}}%
\pgfpathclose%
\pgfusepath{stroke,fill}%
\end{pgfscope}%
\begin{pgfscope}%
\pgfpathrectangle{\pgfqpoint{0.648703in}{0.548769in}}{\pgfqpoint{5.201297in}{3.102590in}}%
\pgfusepath{clip}%
\pgfsetbuttcap%
\pgfsetroundjoin%
\definecolor{currentfill}{rgb}{1.000000,0.498039,0.054902}%
\pgfsetfillcolor{currentfill}%
\pgfsetlinewidth{1.003750pt}%
\definecolor{currentstroke}{rgb}{1.000000,0.498039,0.054902}%
\pgfsetstrokecolor{currentstroke}%
\pgfsetdash{}{0pt}%
\pgfpathmoveto{\pgfqpoint{1.273766in}{3.348378in}}%
\pgfpathcurveto{\pgfqpoint{1.284816in}{3.348378in}}{\pgfqpoint{1.295415in}{3.352768in}}{\pgfqpoint{1.303229in}{3.360581in}}%
\pgfpathcurveto{\pgfqpoint{1.311042in}{3.368395in}}{\pgfqpoint{1.315432in}{3.378994in}}{\pgfqpoint{1.315432in}{3.390044in}}%
\pgfpathcurveto{\pgfqpoint{1.315432in}{3.401094in}}{\pgfqpoint{1.311042in}{3.411693in}}{\pgfqpoint{1.303229in}{3.419507in}}%
\pgfpathcurveto{\pgfqpoint{1.295415in}{3.427321in}}{\pgfqpoint{1.284816in}{3.431711in}}{\pgfqpoint{1.273766in}{3.431711in}}%
\pgfpathcurveto{\pgfqpoint{1.262716in}{3.431711in}}{\pgfqpoint{1.252117in}{3.427321in}}{\pgfqpoint{1.244303in}{3.419507in}}%
\pgfpathcurveto{\pgfqpoint{1.236489in}{3.411693in}}{\pgfqpoint{1.232099in}{3.401094in}}{\pgfqpoint{1.232099in}{3.390044in}}%
\pgfpathcurveto{\pgfqpoint{1.232099in}{3.378994in}}{\pgfqpoint{1.236489in}{3.368395in}}{\pgfqpoint{1.244303in}{3.360581in}}%
\pgfpathcurveto{\pgfqpoint{1.252117in}{3.352768in}}{\pgfqpoint{1.262716in}{3.348378in}}{\pgfqpoint{1.273766in}{3.348378in}}%
\pgfpathclose%
\pgfusepath{stroke,fill}%
\end{pgfscope}%
\begin{pgfscope}%
\pgfsetbuttcap%
\pgfsetroundjoin%
\definecolor{currentfill}{rgb}{0.000000,0.000000,0.000000}%
\pgfsetfillcolor{currentfill}%
\pgfsetlinewidth{0.803000pt}%
\definecolor{currentstroke}{rgb}{0.000000,0.000000,0.000000}%
\pgfsetstrokecolor{currentstroke}%
\pgfsetdash{}{0pt}%
\pgfsys@defobject{currentmarker}{\pgfqpoint{0.000000in}{-0.048611in}}{\pgfqpoint{0.000000in}{0.000000in}}{%
\pgfpathmoveto{\pgfqpoint{0.000000in}{0.000000in}}%
\pgfpathlineto{\pgfqpoint{0.000000in}{-0.048611in}}%
\pgfusepath{stroke,fill}%
}%
\begin{pgfscope}%
\pgfsys@transformshift{0.820353in}{0.548769in}%
\pgfsys@useobject{currentmarker}{}%
\end{pgfscope}%
\end{pgfscope}%
\begin{pgfscope}%
\definecolor{textcolor}{rgb}{0.000000,0.000000,0.000000}%
\pgfsetstrokecolor{textcolor}%
\pgfsetfillcolor{textcolor}%
\pgftext[x=0.820353in,y=0.451547in,,top]{\color{textcolor}\sffamily\fontsize{10.000000}{12.000000}\selectfont \(\displaystyle {0}\)}%
\end{pgfscope}%
\begin{pgfscope}%
\pgfsetbuttcap%
\pgfsetroundjoin%
\definecolor{currentfill}{rgb}{0.000000,0.000000,0.000000}%
\pgfsetfillcolor{currentfill}%
\pgfsetlinewidth{0.803000pt}%
\definecolor{currentstroke}{rgb}{0.000000,0.000000,0.000000}%
\pgfsetstrokecolor{currentstroke}%
\pgfsetdash{}{0pt}%
\pgfsys@defobject{currentmarker}{\pgfqpoint{0.000000in}{-0.048611in}}{\pgfqpoint{0.000000in}{0.000000in}}{%
\pgfpathmoveto{\pgfqpoint{0.000000in}{0.000000in}}%
\pgfpathlineto{\pgfqpoint{0.000000in}{-0.048611in}}%
\pgfusepath{stroke,fill}%
}%
\begin{pgfscope}%
\pgfsys@transformshift{1.468086in}{0.548769in}%
\pgfsys@useobject{currentmarker}{}%
\end{pgfscope}%
\end{pgfscope}%
\begin{pgfscope}%
\definecolor{textcolor}{rgb}{0.000000,0.000000,0.000000}%
\pgfsetstrokecolor{textcolor}%
\pgfsetfillcolor{textcolor}%
\pgftext[x=1.468086in,y=0.451547in,,top]{\color{textcolor}\sffamily\fontsize{10.000000}{12.000000}\selectfont \(\displaystyle {10}\)}%
\end{pgfscope}%
\begin{pgfscope}%
\pgfsetbuttcap%
\pgfsetroundjoin%
\definecolor{currentfill}{rgb}{0.000000,0.000000,0.000000}%
\pgfsetfillcolor{currentfill}%
\pgfsetlinewidth{0.803000pt}%
\definecolor{currentstroke}{rgb}{0.000000,0.000000,0.000000}%
\pgfsetstrokecolor{currentstroke}%
\pgfsetdash{}{0pt}%
\pgfsys@defobject{currentmarker}{\pgfqpoint{0.000000in}{-0.048611in}}{\pgfqpoint{0.000000in}{0.000000in}}{%
\pgfpathmoveto{\pgfqpoint{0.000000in}{0.000000in}}%
\pgfpathlineto{\pgfqpoint{0.000000in}{-0.048611in}}%
\pgfusepath{stroke,fill}%
}%
\begin{pgfscope}%
\pgfsys@transformshift{2.115819in}{0.548769in}%
\pgfsys@useobject{currentmarker}{}%
\end{pgfscope}%
\end{pgfscope}%
\begin{pgfscope}%
\definecolor{textcolor}{rgb}{0.000000,0.000000,0.000000}%
\pgfsetstrokecolor{textcolor}%
\pgfsetfillcolor{textcolor}%
\pgftext[x=2.115819in,y=0.451547in,,top]{\color{textcolor}\sffamily\fontsize{10.000000}{12.000000}\selectfont \(\displaystyle {20}\)}%
\end{pgfscope}%
\begin{pgfscope}%
\pgfsetbuttcap%
\pgfsetroundjoin%
\definecolor{currentfill}{rgb}{0.000000,0.000000,0.000000}%
\pgfsetfillcolor{currentfill}%
\pgfsetlinewidth{0.803000pt}%
\definecolor{currentstroke}{rgb}{0.000000,0.000000,0.000000}%
\pgfsetstrokecolor{currentstroke}%
\pgfsetdash{}{0pt}%
\pgfsys@defobject{currentmarker}{\pgfqpoint{0.000000in}{-0.048611in}}{\pgfqpoint{0.000000in}{0.000000in}}{%
\pgfpathmoveto{\pgfqpoint{0.000000in}{0.000000in}}%
\pgfpathlineto{\pgfqpoint{0.000000in}{-0.048611in}}%
\pgfusepath{stroke,fill}%
}%
\begin{pgfscope}%
\pgfsys@transformshift{2.763552in}{0.548769in}%
\pgfsys@useobject{currentmarker}{}%
\end{pgfscope}%
\end{pgfscope}%
\begin{pgfscope}%
\definecolor{textcolor}{rgb}{0.000000,0.000000,0.000000}%
\pgfsetstrokecolor{textcolor}%
\pgfsetfillcolor{textcolor}%
\pgftext[x=2.763552in,y=0.451547in,,top]{\color{textcolor}\sffamily\fontsize{10.000000}{12.000000}\selectfont \(\displaystyle {30}\)}%
\end{pgfscope}%
\begin{pgfscope}%
\pgfsetbuttcap%
\pgfsetroundjoin%
\definecolor{currentfill}{rgb}{0.000000,0.000000,0.000000}%
\pgfsetfillcolor{currentfill}%
\pgfsetlinewidth{0.803000pt}%
\definecolor{currentstroke}{rgb}{0.000000,0.000000,0.000000}%
\pgfsetstrokecolor{currentstroke}%
\pgfsetdash{}{0pt}%
\pgfsys@defobject{currentmarker}{\pgfqpoint{0.000000in}{-0.048611in}}{\pgfqpoint{0.000000in}{0.000000in}}{%
\pgfpathmoveto{\pgfqpoint{0.000000in}{0.000000in}}%
\pgfpathlineto{\pgfqpoint{0.000000in}{-0.048611in}}%
\pgfusepath{stroke,fill}%
}%
\begin{pgfscope}%
\pgfsys@transformshift{3.411285in}{0.548769in}%
\pgfsys@useobject{currentmarker}{}%
\end{pgfscope}%
\end{pgfscope}%
\begin{pgfscope}%
\definecolor{textcolor}{rgb}{0.000000,0.000000,0.000000}%
\pgfsetstrokecolor{textcolor}%
\pgfsetfillcolor{textcolor}%
\pgftext[x=3.411285in,y=0.451547in,,top]{\color{textcolor}\sffamily\fontsize{10.000000}{12.000000}\selectfont \(\displaystyle {40}\)}%
\end{pgfscope}%
\begin{pgfscope}%
\pgfsetbuttcap%
\pgfsetroundjoin%
\definecolor{currentfill}{rgb}{0.000000,0.000000,0.000000}%
\pgfsetfillcolor{currentfill}%
\pgfsetlinewidth{0.803000pt}%
\definecolor{currentstroke}{rgb}{0.000000,0.000000,0.000000}%
\pgfsetstrokecolor{currentstroke}%
\pgfsetdash{}{0pt}%
\pgfsys@defobject{currentmarker}{\pgfqpoint{0.000000in}{-0.048611in}}{\pgfqpoint{0.000000in}{0.000000in}}{%
\pgfpathmoveto{\pgfqpoint{0.000000in}{0.000000in}}%
\pgfpathlineto{\pgfqpoint{0.000000in}{-0.048611in}}%
\pgfusepath{stroke,fill}%
}%
\begin{pgfscope}%
\pgfsys@transformshift{4.059018in}{0.548769in}%
\pgfsys@useobject{currentmarker}{}%
\end{pgfscope}%
\end{pgfscope}%
\begin{pgfscope}%
\definecolor{textcolor}{rgb}{0.000000,0.000000,0.000000}%
\pgfsetstrokecolor{textcolor}%
\pgfsetfillcolor{textcolor}%
\pgftext[x=4.059018in,y=0.451547in,,top]{\color{textcolor}\sffamily\fontsize{10.000000}{12.000000}\selectfont \(\displaystyle {50}\)}%
\end{pgfscope}%
\begin{pgfscope}%
\pgfsetbuttcap%
\pgfsetroundjoin%
\definecolor{currentfill}{rgb}{0.000000,0.000000,0.000000}%
\pgfsetfillcolor{currentfill}%
\pgfsetlinewidth{0.803000pt}%
\definecolor{currentstroke}{rgb}{0.000000,0.000000,0.000000}%
\pgfsetstrokecolor{currentstroke}%
\pgfsetdash{}{0pt}%
\pgfsys@defobject{currentmarker}{\pgfqpoint{0.000000in}{-0.048611in}}{\pgfqpoint{0.000000in}{0.000000in}}{%
\pgfpathmoveto{\pgfqpoint{0.000000in}{0.000000in}}%
\pgfpathlineto{\pgfqpoint{0.000000in}{-0.048611in}}%
\pgfusepath{stroke,fill}%
}%
\begin{pgfscope}%
\pgfsys@transformshift{4.706751in}{0.548769in}%
\pgfsys@useobject{currentmarker}{}%
\end{pgfscope}%
\end{pgfscope}%
\begin{pgfscope}%
\definecolor{textcolor}{rgb}{0.000000,0.000000,0.000000}%
\pgfsetstrokecolor{textcolor}%
\pgfsetfillcolor{textcolor}%
\pgftext[x=4.706751in,y=0.451547in,,top]{\color{textcolor}\sffamily\fontsize{10.000000}{12.000000}\selectfont \(\displaystyle {60}\)}%
\end{pgfscope}%
\begin{pgfscope}%
\pgfsetbuttcap%
\pgfsetroundjoin%
\definecolor{currentfill}{rgb}{0.000000,0.000000,0.000000}%
\pgfsetfillcolor{currentfill}%
\pgfsetlinewidth{0.803000pt}%
\definecolor{currentstroke}{rgb}{0.000000,0.000000,0.000000}%
\pgfsetstrokecolor{currentstroke}%
\pgfsetdash{}{0pt}%
\pgfsys@defobject{currentmarker}{\pgfqpoint{0.000000in}{-0.048611in}}{\pgfqpoint{0.000000in}{0.000000in}}{%
\pgfpathmoveto{\pgfqpoint{0.000000in}{0.000000in}}%
\pgfpathlineto{\pgfqpoint{0.000000in}{-0.048611in}}%
\pgfusepath{stroke,fill}%
}%
\begin{pgfscope}%
\pgfsys@transformshift{5.354484in}{0.548769in}%
\pgfsys@useobject{currentmarker}{}%
\end{pgfscope}%
\end{pgfscope}%
\begin{pgfscope}%
\definecolor{textcolor}{rgb}{0.000000,0.000000,0.000000}%
\pgfsetstrokecolor{textcolor}%
\pgfsetfillcolor{textcolor}%
\pgftext[x=5.354484in,y=0.451547in,,top]{\color{textcolor}\sffamily\fontsize{10.000000}{12.000000}\selectfont \(\displaystyle {70}\)}%
\end{pgfscope}%
\begin{pgfscope}%
\definecolor{textcolor}{rgb}{0.000000,0.000000,0.000000}%
\pgfsetstrokecolor{textcolor}%
\pgfsetfillcolor{textcolor}%
\pgftext[x=3.249352in,y=0.272658in,,top]{\color{textcolor}\sffamily\fontsize{10.000000}{12.000000}\selectfont Number of Sinks}%
\end{pgfscope}%
\begin{pgfscope}%
\pgfsetbuttcap%
\pgfsetroundjoin%
\definecolor{currentfill}{rgb}{0.000000,0.000000,0.000000}%
\pgfsetfillcolor{currentfill}%
\pgfsetlinewidth{0.803000pt}%
\definecolor{currentstroke}{rgb}{0.000000,0.000000,0.000000}%
\pgfsetstrokecolor{currentstroke}%
\pgfsetdash{}{0pt}%
\pgfsys@defobject{currentmarker}{\pgfqpoint{-0.048611in}{0.000000in}}{\pgfqpoint{0.000000in}{0.000000in}}{%
\pgfpathmoveto{\pgfqpoint{0.000000in}{0.000000in}}%
\pgfpathlineto{\pgfqpoint{-0.048611in}{0.000000in}}%
\pgfusepath{stroke,fill}%
}%
\begin{pgfscope}%
\pgfsys@transformshift{0.648703in}{0.689796in}%
\pgfsys@useobject{currentmarker}{}%
\end{pgfscope}%
\end{pgfscope}%
\begin{pgfscope}%
\definecolor{textcolor}{rgb}{0.000000,0.000000,0.000000}%
\pgfsetstrokecolor{textcolor}%
\pgfsetfillcolor{textcolor}%
\pgftext[x=0.482036in, y=0.641601in, left, base]{\color{textcolor}\sffamily\fontsize{10.000000}{12.000000}\selectfont \(\displaystyle {0}\)}%
\end{pgfscope}%
\begin{pgfscope}%
\pgfsetbuttcap%
\pgfsetroundjoin%
\definecolor{currentfill}{rgb}{0.000000,0.000000,0.000000}%
\pgfsetfillcolor{currentfill}%
\pgfsetlinewidth{0.803000pt}%
\definecolor{currentstroke}{rgb}{0.000000,0.000000,0.000000}%
\pgfsetstrokecolor{currentstroke}%
\pgfsetdash{}{0pt}%
\pgfsys@defobject{currentmarker}{\pgfqpoint{-0.048611in}{0.000000in}}{\pgfqpoint{0.000000in}{0.000000in}}{%
\pgfpathmoveto{\pgfqpoint{0.000000in}{0.000000in}}%
\pgfpathlineto{\pgfqpoint{-0.048611in}{0.000000in}}%
\pgfusepath{stroke,fill}%
}%
\begin{pgfscope}%
\pgfsys@transformshift{0.648703in}{1.104580in}%
\pgfsys@useobject{currentmarker}{}%
\end{pgfscope}%
\end{pgfscope}%
\begin{pgfscope}%
\definecolor{textcolor}{rgb}{0.000000,0.000000,0.000000}%
\pgfsetstrokecolor{textcolor}%
\pgfsetfillcolor{textcolor}%
\pgftext[x=0.343147in, y=1.056386in, left, base]{\color{textcolor}\sffamily\fontsize{10.000000}{12.000000}\selectfont \(\displaystyle {100}\)}%
\end{pgfscope}%
\begin{pgfscope}%
\pgfsetbuttcap%
\pgfsetroundjoin%
\definecolor{currentfill}{rgb}{0.000000,0.000000,0.000000}%
\pgfsetfillcolor{currentfill}%
\pgfsetlinewidth{0.803000pt}%
\definecolor{currentstroke}{rgb}{0.000000,0.000000,0.000000}%
\pgfsetstrokecolor{currentstroke}%
\pgfsetdash{}{0pt}%
\pgfsys@defobject{currentmarker}{\pgfqpoint{-0.048611in}{0.000000in}}{\pgfqpoint{0.000000in}{0.000000in}}{%
\pgfpathmoveto{\pgfqpoint{0.000000in}{0.000000in}}%
\pgfpathlineto{\pgfqpoint{-0.048611in}{0.000000in}}%
\pgfusepath{stroke,fill}%
}%
\begin{pgfscope}%
\pgfsys@transformshift{0.648703in}{1.519365in}%
\pgfsys@useobject{currentmarker}{}%
\end{pgfscope}%
\end{pgfscope}%
\begin{pgfscope}%
\definecolor{textcolor}{rgb}{0.000000,0.000000,0.000000}%
\pgfsetstrokecolor{textcolor}%
\pgfsetfillcolor{textcolor}%
\pgftext[x=0.343147in, y=1.471171in, left, base]{\color{textcolor}\sffamily\fontsize{10.000000}{12.000000}\selectfont \(\displaystyle {200}\)}%
\end{pgfscope}%
\begin{pgfscope}%
\pgfsetbuttcap%
\pgfsetroundjoin%
\definecolor{currentfill}{rgb}{0.000000,0.000000,0.000000}%
\pgfsetfillcolor{currentfill}%
\pgfsetlinewidth{0.803000pt}%
\definecolor{currentstroke}{rgb}{0.000000,0.000000,0.000000}%
\pgfsetstrokecolor{currentstroke}%
\pgfsetdash{}{0pt}%
\pgfsys@defobject{currentmarker}{\pgfqpoint{-0.048611in}{0.000000in}}{\pgfqpoint{0.000000in}{0.000000in}}{%
\pgfpathmoveto{\pgfqpoint{0.000000in}{0.000000in}}%
\pgfpathlineto{\pgfqpoint{-0.048611in}{0.000000in}}%
\pgfusepath{stroke,fill}%
}%
\begin{pgfscope}%
\pgfsys@transformshift{0.648703in}{1.934150in}%
\pgfsys@useobject{currentmarker}{}%
\end{pgfscope}%
\end{pgfscope}%
\begin{pgfscope}%
\definecolor{textcolor}{rgb}{0.000000,0.000000,0.000000}%
\pgfsetstrokecolor{textcolor}%
\pgfsetfillcolor{textcolor}%
\pgftext[x=0.343147in, y=1.885955in, left, base]{\color{textcolor}\sffamily\fontsize{10.000000}{12.000000}\selectfont \(\displaystyle {300}\)}%
\end{pgfscope}%
\begin{pgfscope}%
\pgfsetbuttcap%
\pgfsetroundjoin%
\definecolor{currentfill}{rgb}{0.000000,0.000000,0.000000}%
\pgfsetfillcolor{currentfill}%
\pgfsetlinewidth{0.803000pt}%
\definecolor{currentstroke}{rgb}{0.000000,0.000000,0.000000}%
\pgfsetstrokecolor{currentstroke}%
\pgfsetdash{}{0pt}%
\pgfsys@defobject{currentmarker}{\pgfqpoint{-0.048611in}{0.000000in}}{\pgfqpoint{0.000000in}{0.000000in}}{%
\pgfpathmoveto{\pgfqpoint{0.000000in}{0.000000in}}%
\pgfpathlineto{\pgfqpoint{-0.048611in}{0.000000in}}%
\pgfusepath{stroke,fill}%
}%
\begin{pgfscope}%
\pgfsys@transformshift{0.648703in}{2.348935in}%
\pgfsys@useobject{currentmarker}{}%
\end{pgfscope}%
\end{pgfscope}%
\begin{pgfscope}%
\definecolor{textcolor}{rgb}{0.000000,0.000000,0.000000}%
\pgfsetstrokecolor{textcolor}%
\pgfsetfillcolor{textcolor}%
\pgftext[x=0.343147in, y=2.300740in, left, base]{\color{textcolor}\sffamily\fontsize{10.000000}{12.000000}\selectfont \(\displaystyle {400}\)}%
\end{pgfscope}%
\begin{pgfscope}%
\pgfsetbuttcap%
\pgfsetroundjoin%
\definecolor{currentfill}{rgb}{0.000000,0.000000,0.000000}%
\pgfsetfillcolor{currentfill}%
\pgfsetlinewidth{0.803000pt}%
\definecolor{currentstroke}{rgb}{0.000000,0.000000,0.000000}%
\pgfsetstrokecolor{currentstroke}%
\pgfsetdash{}{0pt}%
\pgfsys@defobject{currentmarker}{\pgfqpoint{-0.048611in}{0.000000in}}{\pgfqpoint{0.000000in}{0.000000in}}{%
\pgfpathmoveto{\pgfqpoint{0.000000in}{0.000000in}}%
\pgfpathlineto{\pgfqpoint{-0.048611in}{0.000000in}}%
\pgfusepath{stroke,fill}%
}%
\begin{pgfscope}%
\pgfsys@transformshift{0.648703in}{2.763719in}%
\pgfsys@useobject{currentmarker}{}%
\end{pgfscope}%
\end{pgfscope}%
\begin{pgfscope}%
\definecolor{textcolor}{rgb}{0.000000,0.000000,0.000000}%
\pgfsetstrokecolor{textcolor}%
\pgfsetfillcolor{textcolor}%
\pgftext[x=0.343147in, y=2.715525in, left, base]{\color{textcolor}\sffamily\fontsize{10.000000}{12.000000}\selectfont \(\displaystyle {500}\)}%
\end{pgfscope}%
\begin{pgfscope}%
\pgfsetbuttcap%
\pgfsetroundjoin%
\definecolor{currentfill}{rgb}{0.000000,0.000000,0.000000}%
\pgfsetfillcolor{currentfill}%
\pgfsetlinewidth{0.803000pt}%
\definecolor{currentstroke}{rgb}{0.000000,0.000000,0.000000}%
\pgfsetstrokecolor{currentstroke}%
\pgfsetdash{}{0pt}%
\pgfsys@defobject{currentmarker}{\pgfqpoint{-0.048611in}{0.000000in}}{\pgfqpoint{0.000000in}{0.000000in}}{%
\pgfpathmoveto{\pgfqpoint{0.000000in}{0.000000in}}%
\pgfpathlineto{\pgfqpoint{-0.048611in}{0.000000in}}%
\pgfusepath{stroke,fill}%
}%
\begin{pgfscope}%
\pgfsys@transformshift{0.648703in}{3.178504in}%
\pgfsys@useobject{currentmarker}{}%
\end{pgfscope}%
\end{pgfscope}%
\begin{pgfscope}%
\definecolor{textcolor}{rgb}{0.000000,0.000000,0.000000}%
\pgfsetstrokecolor{textcolor}%
\pgfsetfillcolor{textcolor}%
\pgftext[x=0.343147in, y=3.130310in, left, base]{\color{textcolor}\sffamily\fontsize{10.000000}{12.000000}\selectfont \(\displaystyle {600}\)}%
\end{pgfscope}%
\begin{pgfscope}%
\pgfsetbuttcap%
\pgfsetroundjoin%
\definecolor{currentfill}{rgb}{0.000000,0.000000,0.000000}%
\pgfsetfillcolor{currentfill}%
\pgfsetlinewidth{0.803000pt}%
\definecolor{currentstroke}{rgb}{0.000000,0.000000,0.000000}%
\pgfsetstrokecolor{currentstroke}%
\pgfsetdash{}{0pt}%
\pgfsys@defobject{currentmarker}{\pgfqpoint{-0.048611in}{0.000000in}}{\pgfqpoint{0.000000in}{0.000000in}}{%
\pgfpathmoveto{\pgfqpoint{0.000000in}{0.000000in}}%
\pgfpathlineto{\pgfqpoint{-0.048611in}{0.000000in}}%
\pgfusepath{stroke,fill}%
}%
\begin{pgfscope}%
\pgfsys@transformshift{0.648703in}{3.593289in}%
\pgfsys@useobject{currentmarker}{}%
\end{pgfscope}%
\end{pgfscope}%
\begin{pgfscope}%
\definecolor{textcolor}{rgb}{0.000000,0.000000,0.000000}%
\pgfsetstrokecolor{textcolor}%
\pgfsetfillcolor{textcolor}%
\pgftext[x=0.343147in, y=3.545094in, left, base]{\color{textcolor}\sffamily\fontsize{10.000000}{12.000000}\selectfont \(\displaystyle {700}\)}%
\end{pgfscope}%
\begin{pgfscope}%
\definecolor{textcolor}{rgb}{0.000000,0.000000,0.000000}%
\pgfsetstrokecolor{textcolor}%
\pgfsetfillcolor{textcolor}%
\pgftext[x=0.287592in,y=2.100064in,,bottom,rotate=90.000000]{\color{textcolor}\sffamily\fontsize{10.000000}{12.000000}\selectfont Data Flow Time (s)}%
\end{pgfscope}%
\begin{pgfscope}%
\pgfsetrectcap%
\pgfsetmiterjoin%
\pgfsetlinewidth{0.803000pt}%
\definecolor{currentstroke}{rgb}{0.000000,0.000000,0.000000}%
\pgfsetstrokecolor{currentstroke}%
\pgfsetdash{}{0pt}%
\pgfpathmoveto{\pgfqpoint{0.648703in}{0.548769in}}%
\pgfpathlineto{\pgfqpoint{0.648703in}{3.651359in}}%
\pgfusepath{stroke}%
\end{pgfscope}%
\begin{pgfscope}%
\pgfsetrectcap%
\pgfsetmiterjoin%
\pgfsetlinewidth{0.803000pt}%
\definecolor{currentstroke}{rgb}{0.000000,0.000000,0.000000}%
\pgfsetstrokecolor{currentstroke}%
\pgfsetdash{}{0pt}%
\pgfpathmoveto{\pgfqpoint{5.850000in}{0.548769in}}%
\pgfpathlineto{\pgfqpoint{5.850000in}{3.651359in}}%
\pgfusepath{stroke}%
\end{pgfscope}%
\begin{pgfscope}%
\pgfsetrectcap%
\pgfsetmiterjoin%
\pgfsetlinewidth{0.803000pt}%
\definecolor{currentstroke}{rgb}{0.000000,0.000000,0.000000}%
\pgfsetstrokecolor{currentstroke}%
\pgfsetdash{}{0pt}%
\pgfpathmoveto{\pgfqpoint{0.648703in}{0.548769in}}%
\pgfpathlineto{\pgfqpoint{5.850000in}{0.548769in}}%
\pgfusepath{stroke}%
\end{pgfscope}%
\begin{pgfscope}%
\pgfsetrectcap%
\pgfsetmiterjoin%
\pgfsetlinewidth{0.803000pt}%
\definecolor{currentstroke}{rgb}{0.000000,0.000000,0.000000}%
\pgfsetstrokecolor{currentstroke}%
\pgfsetdash{}{0pt}%
\pgfpathmoveto{\pgfqpoint{0.648703in}{3.651359in}}%
\pgfpathlineto{\pgfqpoint{5.850000in}{3.651359in}}%
\pgfusepath{stroke}%
\end{pgfscope}%
\begin{pgfscope}%
\definecolor{textcolor}{rgb}{0.000000,0.000000,0.000000}%
\pgfsetstrokecolor{textcolor}%
\pgfsetfillcolor{textcolor}%
\pgftext[x=3.249352in,y=3.734692in,,base]{\color{textcolor}\sffamily\fontsize{12.000000}{14.400000}\selectfont Forward}%
\end{pgfscope}%
\begin{pgfscope}%
\pgfsetbuttcap%
\pgfsetmiterjoin%
\definecolor{currentfill}{rgb}{1.000000,1.000000,1.000000}%
\pgfsetfillcolor{currentfill}%
\pgfsetfillopacity{0.800000}%
\pgfsetlinewidth{1.003750pt}%
\definecolor{currentstroke}{rgb}{0.800000,0.800000,0.800000}%
\pgfsetstrokecolor{currentstroke}%
\pgfsetstrokeopacity{0.800000}%
\pgfsetdash{}{0pt}%
\pgfpathmoveto{\pgfqpoint{4.300417in}{0.618213in}}%
\pgfpathlineto{\pgfqpoint{5.752778in}{0.618213in}}%
\pgfpathquadraticcurveto{\pgfqpoint{5.780556in}{0.618213in}}{\pgfqpoint{5.780556in}{0.645991in}}%
\pgfpathlineto{\pgfqpoint{5.780556in}{1.214463in}}%
\pgfpathquadraticcurveto{\pgfqpoint{5.780556in}{1.242241in}}{\pgfqpoint{5.752778in}{1.242241in}}%
\pgfpathlineto{\pgfqpoint{4.300417in}{1.242241in}}%
\pgfpathquadraticcurveto{\pgfqpoint{4.272639in}{1.242241in}}{\pgfqpoint{4.272639in}{1.214463in}}%
\pgfpathlineto{\pgfqpoint{4.272639in}{0.645991in}}%
\pgfpathquadraticcurveto{\pgfqpoint{4.272639in}{0.618213in}}{\pgfqpoint{4.300417in}{0.618213in}}%
\pgfpathclose%
\pgfusepath{stroke,fill}%
\end{pgfscope}%
\begin{pgfscope}%
\pgfsetbuttcap%
\pgfsetroundjoin%
\definecolor{currentfill}{rgb}{0.121569,0.466667,0.705882}%
\pgfsetfillcolor{currentfill}%
\pgfsetlinewidth{1.003750pt}%
\definecolor{currentstroke}{rgb}{0.121569,0.466667,0.705882}%
\pgfsetstrokecolor{currentstroke}%
\pgfsetdash{}{0pt}%
\pgfsys@defobject{currentmarker}{\pgfqpoint{-0.034722in}{-0.034722in}}{\pgfqpoint{0.034722in}{0.034722in}}{%
\pgfpathmoveto{\pgfqpoint{0.000000in}{-0.034722in}}%
\pgfpathcurveto{\pgfqpoint{0.009208in}{-0.034722in}}{\pgfqpoint{0.018041in}{-0.031064in}}{\pgfqpoint{0.024552in}{-0.024552in}}%
\pgfpathcurveto{\pgfqpoint{0.031064in}{-0.018041in}}{\pgfqpoint{0.034722in}{-0.009208in}}{\pgfqpoint{0.034722in}{0.000000in}}%
\pgfpathcurveto{\pgfqpoint{0.034722in}{0.009208in}}{\pgfqpoint{0.031064in}{0.018041in}}{\pgfqpoint{0.024552in}{0.024552in}}%
\pgfpathcurveto{\pgfqpoint{0.018041in}{0.031064in}}{\pgfqpoint{0.009208in}{0.034722in}}{\pgfqpoint{0.000000in}{0.034722in}}%
\pgfpathcurveto{\pgfqpoint{-0.009208in}{0.034722in}}{\pgfqpoint{-0.018041in}{0.031064in}}{\pgfqpoint{-0.024552in}{0.024552in}}%
\pgfpathcurveto{\pgfqpoint{-0.031064in}{0.018041in}}{\pgfqpoint{-0.034722in}{0.009208in}}{\pgfqpoint{-0.034722in}{0.000000in}}%
\pgfpathcurveto{\pgfqpoint{-0.034722in}{-0.009208in}}{\pgfqpoint{-0.031064in}{-0.018041in}}{\pgfqpoint{-0.024552in}{-0.024552in}}%
\pgfpathcurveto{\pgfqpoint{-0.018041in}{-0.031064in}}{\pgfqpoint{-0.009208in}{-0.034722in}}{\pgfqpoint{0.000000in}{-0.034722in}}%
\pgfpathclose%
\pgfusepath{stroke,fill}%
}%
\begin{pgfscope}%
\pgfsys@transformshift{4.467083in}{1.138074in}%
\pgfsys@useobject{currentmarker}{}%
\end{pgfscope}%
\end{pgfscope}%
\begin{pgfscope}%
\definecolor{textcolor}{rgb}{0.000000,0.000000,0.000000}%
\pgfsetstrokecolor{textcolor}%
\pgfsetfillcolor{textcolor}%
\pgftext[x=4.717083in,y=1.089463in,left,base]{\color{textcolor}\sffamily\fontsize{10.000000}{12.000000}\selectfont No Timeout}%
\end{pgfscope}%
\begin{pgfscope}%
\pgfsetbuttcap%
\pgfsetroundjoin%
\definecolor{currentfill}{rgb}{1.000000,0.498039,0.054902}%
\pgfsetfillcolor{currentfill}%
\pgfsetlinewidth{1.003750pt}%
\definecolor{currentstroke}{rgb}{1.000000,0.498039,0.054902}%
\pgfsetstrokecolor{currentstroke}%
\pgfsetdash{}{0pt}%
\pgfsys@defobject{currentmarker}{\pgfqpoint{-0.034722in}{-0.034722in}}{\pgfqpoint{0.034722in}{0.034722in}}{%
\pgfpathmoveto{\pgfqpoint{0.000000in}{-0.034722in}}%
\pgfpathcurveto{\pgfqpoint{0.009208in}{-0.034722in}}{\pgfqpoint{0.018041in}{-0.031064in}}{\pgfqpoint{0.024552in}{-0.024552in}}%
\pgfpathcurveto{\pgfqpoint{0.031064in}{-0.018041in}}{\pgfqpoint{0.034722in}{-0.009208in}}{\pgfqpoint{0.034722in}{0.000000in}}%
\pgfpathcurveto{\pgfqpoint{0.034722in}{0.009208in}}{\pgfqpoint{0.031064in}{0.018041in}}{\pgfqpoint{0.024552in}{0.024552in}}%
\pgfpathcurveto{\pgfqpoint{0.018041in}{0.031064in}}{\pgfqpoint{0.009208in}{0.034722in}}{\pgfqpoint{0.000000in}{0.034722in}}%
\pgfpathcurveto{\pgfqpoint{-0.009208in}{0.034722in}}{\pgfqpoint{-0.018041in}{0.031064in}}{\pgfqpoint{-0.024552in}{0.024552in}}%
\pgfpathcurveto{\pgfqpoint{-0.031064in}{0.018041in}}{\pgfqpoint{-0.034722in}{0.009208in}}{\pgfqpoint{-0.034722in}{0.000000in}}%
\pgfpathcurveto{\pgfqpoint{-0.034722in}{-0.009208in}}{\pgfqpoint{-0.031064in}{-0.018041in}}{\pgfqpoint{-0.024552in}{-0.024552in}}%
\pgfpathcurveto{\pgfqpoint{-0.018041in}{-0.031064in}}{\pgfqpoint{-0.009208in}{-0.034722in}}{\pgfqpoint{0.000000in}{-0.034722in}}%
\pgfpathclose%
\pgfusepath{stroke,fill}%
}%
\begin{pgfscope}%
\pgfsys@transformshift{4.467083in}{0.944463in}%
\pgfsys@useobject{currentmarker}{}%
\end{pgfscope}%
\end{pgfscope}%
\begin{pgfscope}%
\definecolor{textcolor}{rgb}{0.000000,0.000000,0.000000}%
\pgfsetstrokecolor{textcolor}%
\pgfsetfillcolor{textcolor}%
\pgftext[x=4.717083in,y=0.895852in,left,base]{\color{textcolor}\sffamily\fontsize{10.000000}{12.000000}\selectfont Time Timeout}%
\end{pgfscope}%
\begin{pgfscope}%
\pgfsetbuttcap%
\pgfsetroundjoin%
\definecolor{currentfill}{rgb}{0.839216,0.152941,0.156863}%
\pgfsetfillcolor{currentfill}%
\pgfsetlinewidth{1.003750pt}%
\definecolor{currentstroke}{rgb}{0.839216,0.152941,0.156863}%
\pgfsetstrokecolor{currentstroke}%
\pgfsetdash{}{0pt}%
\pgfsys@defobject{currentmarker}{\pgfqpoint{-0.034722in}{-0.034722in}}{\pgfqpoint{0.034722in}{0.034722in}}{%
\pgfpathmoveto{\pgfqpoint{0.000000in}{-0.034722in}}%
\pgfpathcurveto{\pgfqpoint{0.009208in}{-0.034722in}}{\pgfqpoint{0.018041in}{-0.031064in}}{\pgfqpoint{0.024552in}{-0.024552in}}%
\pgfpathcurveto{\pgfqpoint{0.031064in}{-0.018041in}}{\pgfqpoint{0.034722in}{-0.009208in}}{\pgfqpoint{0.034722in}{0.000000in}}%
\pgfpathcurveto{\pgfqpoint{0.034722in}{0.009208in}}{\pgfqpoint{0.031064in}{0.018041in}}{\pgfqpoint{0.024552in}{0.024552in}}%
\pgfpathcurveto{\pgfqpoint{0.018041in}{0.031064in}}{\pgfqpoint{0.009208in}{0.034722in}}{\pgfqpoint{0.000000in}{0.034722in}}%
\pgfpathcurveto{\pgfqpoint{-0.009208in}{0.034722in}}{\pgfqpoint{-0.018041in}{0.031064in}}{\pgfqpoint{-0.024552in}{0.024552in}}%
\pgfpathcurveto{\pgfqpoint{-0.031064in}{0.018041in}}{\pgfqpoint{-0.034722in}{0.009208in}}{\pgfqpoint{-0.034722in}{0.000000in}}%
\pgfpathcurveto{\pgfqpoint{-0.034722in}{-0.009208in}}{\pgfqpoint{-0.031064in}{-0.018041in}}{\pgfqpoint{-0.024552in}{-0.024552in}}%
\pgfpathcurveto{\pgfqpoint{-0.018041in}{-0.031064in}}{\pgfqpoint{-0.009208in}{-0.034722in}}{\pgfqpoint{0.000000in}{-0.034722in}}%
\pgfpathclose%
\pgfusepath{stroke,fill}%
}%
\begin{pgfscope}%
\pgfsys@transformshift{4.467083in}{0.750852in}%
\pgfsys@useobject{currentmarker}{}%
\end{pgfscope}%
\end{pgfscope}%
\begin{pgfscope}%
\definecolor{textcolor}{rgb}{0.000000,0.000000,0.000000}%
\pgfsetstrokecolor{textcolor}%
\pgfsetfillcolor{textcolor}%
\pgftext[x=4.717083in,y=0.702241in,left,base]{\color{textcolor}\sffamily\fontsize{10.000000}{12.000000}\selectfont Memory Timeout}%
\end{pgfscope}%
\end{pgfpicture}%
\makeatother%
\endgroup%

            }
        \end{subfigure}
        \qquad
        \begin{subfigure}[]{0.45\textwidth}
            \centering
            \resizebox{\columnwidth}{!}{
                %% Creator: Matplotlib, PGF backend
%%
%% To include the figure in your LaTeX document, write
%%   \input{<filename>.pgf}
%%
%% Make sure the required packages are loaded in your preamble
%%   \usepackage{pgf}
%%
%% and, on pdftex
%%   \usepackage[utf8]{inputenc}\DeclareUnicodeCharacter{2212}{-}
%%
%% or, on luatex and xetex
%%   \usepackage{unicode-math}
%%
%% Figures using additional raster images can only be included by \input if
%% they are in the same directory as the main LaTeX file. For loading figures
%% from other directories you can use the `import` package
%%   \usepackage{import}
%%
%% and then include the figures with
%%   \import{<path to file>}{<filename>.pgf}
%%
%% Matplotlib used the following preamble
%%   \usepackage{amsmath}
%%   \usepackage{fontspec}
%%
\begingroup%
\makeatletter%
\begin{pgfpicture}%
\pgfpathrectangle{\pgfpointorigin}{\pgfqpoint{6.000000in}{4.000000in}}%
\pgfusepath{use as bounding box, clip}%
\begin{pgfscope}%
\pgfsetbuttcap%
\pgfsetmiterjoin%
\definecolor{currentfill}{rgb}{1.000000,1.000000,1.000000}%
\pgfsetfillcolor{currentfill}%
\pgfsetlinewidth{0.000000pt}%
\definecolor{currentstroke}{rgb}{1.000000,1.000000,1.000000}%
\pgfsetstrokecolor{currentstroke}%
\pgfsetdash{}{0pt}%
\pgfpathmoveto{\pgfqpoint{0.000000in}{0.000000in}}%
\pgfpathlineto{\pgfqpoint{6.000000in}{0.000000in}}%
\pgfpathlineto{\pgfqpoint{6.000000in}{4.000000in}}%
\pgfpathlineto{\pgfqpoint{0.000000in}{4.000000in}}%
\pgfpathclose%
\pgfusepath{fill}%
\end{pgfscope}%
\begin{pgfscope}%
\pgfsetbuttcap%
\pgfsetmiterjoin%
\definecolor{currentfill}{rgb}{1.000000,1.000000,1.000000}%
\pgfsetfillcolor{currentfill}%
\pgfsetlinewidth{0.000000pt}%
\definecolor{currentstroke}{rgb}{0.000000,0.000000,0.000000}%
\pgfsetstrokecolor{currentstroke}%
\pgfsetstrokeopacity{0.000000}%
\pgfsetdash{}{0pt}%
\pgfpathmoveto{\pgfqpoint{0.648703in}{0.548769in}}%
\pgfpathlineto{\pgfqpoint{5.843853in}{0.548769in}}%
\pgfpathlineto{\pgfqpoint{5.843853in}{3.651359in}}%
\pgfpathlineto{\pgfqpoint{0.648703in}{3.651359in}}%
\pgfpathclose%
\pgfusepath{fill}%
\end{pgfscope}%
\begin{pgfscope}%
\pgfpathrectangle{\pgfqpoint{0.648703in}{0.548769in}}{\pgfqpoint{5.195150in}{3.102590in}}%
\pgfusepath{clip}%
\pgfsetbuttcap%
\pgfsetroundjoin%
\definecolor{currentfill}{rgb}{0.121569,0.466667,0.705882}%
\pgfsetfillcolor{currentfill}%
\pgfsetlinewidth{1.003750pt}%
\definecolor{currentstroke}{rgb}{0.121569,0.466667,0.705882}%
\pgfsetstrokecolor{currentstroke}%
\pgfsetdash{}{0pt}%
\pgfpathmoveto{\pgfqpoint{1.381990in}{0.673501in}}%
\pgfpathcurveto{\pgfqpoint{1.393040in}{0.673501in}}{\pgfqpoint{1.403639in}{0.677891in}}{\pgfqpoint{1.411453in}{0.685705in}}%
\pgfpathcurveto{\pgfqpoint{1.419266in}{0.693519in}}{\pgfqpoint{1.423657in}{0.704118in}}{\pgfqpoint{1.423657in}{0.715168in}}%
\pgfpathcurveto{\pgfqpoint{1.423657in}{0.726218in}}{\pgfqpoint{1.419266in}{0.736817in}}{\pgfqpoint{1.411453in}{0.744631in}}%
\pgfpathcurveto{\pgfqpoint{1.403639in}{0.752444in}}{\pgfqpoint{1.393040in}{0.756834in}}{\pgfqpoint{1.381990in}{0.756834in}}%
\pgfpathcurveto{\pgfqpoint{1.370940in}{0.756834in}}{\pgfqpoint{1.360341in}{0.752444in}}{\pgfqpoint{1.352527in}{0.744631in}}%
\pgfpathcurveto{\pgfqpoint{1.344714in}{0.736817in}}{\pgfqpoint{1.340323in}{0.726218in}}{\pgfqpoint{1.340323in}{0.715168in}}%
\pgfpathcurveto{\pgfqpoint{1.340323in}{0.704118in}}{\pgfqpoint{1.344714in}{0.693519in}}{\pgfqpoint{1.352527in}{0.685705in}}%
\pgfpathcurveto{\pgfqpoint{1.360341in}{0.677891in}}{\pgfqpoint{1.370940in}{0.673501in}}{\pgfqpoint{1.381990in}{0.673501in}}%
\pgfpathclose%
\pgfusepath{stroke,fill}%
\end{pgfscope}%
\begin{pgfscope}%
\pgfpathrectangle{\pgfqpoint{0.648703in}{0.548769in}}{\pgfqpoint{5.195150in}{3.102590in}}%
\pgfusepath{clip}%
\pgfsetbuttcap%
\pgfsetroundjoin%
\definecolor{currentfill}{rgb}{1.000000,0.498039,0.054902}%
\pgfsetfillcolor{currentfill}%
\pgfsetlinewidth{1.003750pt}%
\definecolor{currentstroke}{rgb}{1.000000,0.498039,0.054902}%
\pgfsetstrokecolor{currentstroke}%
\pgfsetdash{}{0pt}%
\pgfpathmoveto{\pgfqpoint{3.039135in}{3.185343in}}%
\pgfpathcurveto{\pgfqpoint{3.050185in}{3.185343in}}{\pgfqpoint{3.060784in}{3.189733in}}{\pgfqpoint{3.068598in}{3.197547in}}%
\pgfpathcurveto{\pgfqpoint{3.076412in}{3.205360in}}{\pgfqpoint{3.080802in}{3.215959in}}{\pgfqpoint{3.080802in}{3.227010in}}%
\pgfpathcurveto{\pgfqpoint{3.080802in}{3.238060in}}{\pgfqpoint{3.076412in}{3.248659in}}{\pgfqpoint{3.068598in}{3.256472in}}%
\pgfpathcurveto{\pgfqpoint{3.060784in}{3.264286in}}{\pgfqpoint{3.050185in}{3.268676in}}{\pgfqpoint{3.039135in}{3.268676in}}%
\pgfpathcurveto{\pgfqpoint{3.028085in}{3.268676in}}{\pgfqpoint{3.017486in}{3.264286in}}{\pgfqpoint{3.009672in}{3.256472in}}%
\pgfpathcurveto{\pgfqpoint{3.001859in}{3.248659in}}{\pgfqpoint{2.997468in}{3.238060in}}{\pgfqpoint{2.997468in}{3.227010in}}%
\pgfpathcurveto{\pgfqpoint{2.997468in}{3.215959in}}{\pgfqpoint{3.001859in}{3.205360in}}{\pgfqpoint{3.009672in}{3.197547in}}%
\pgfpathcurveto{\pgfqpoint{3.017486in}{3.189733in}}{\pgfqpoint{3.028085in}{3.185343in}}{\pgfqpoint{3.039135in}{3.185343in}}%
\pgfpathclose%
\pgfusepath{stroke,fill}%
\end{pgfscope}%
\begin{pgfscope}%
\pgfpathrectangle{\pgfqpoint{0.648703in}{0.548769in}}{\pgfqpoint{5.195150in}{3.102590in}}%
\pgfusepath{clip}%
\pgfsetbuttcap%
\pgfsetroundjoin%
\definecolor{currentfill}{rgb}{0.121569,0.466667,0.705882}%
\pgfsetfillcolor{currentfill}%
\pgfsetlinewidth{1.003750pt}%
\definecolor{currentstroke}{rgb}{0.121569,0.466667,0.705882}%
\pgfsetstrokecolor{currentstroke}%
\pgfsetdash{}{0pt}%
\pgfpathmoveto{\pgfqpoint{2.459134in}{0.652358in}}%
\pgfpathcurveto{\pgfqpoint{2.470184in}{0.652358in}}{\pgfqpoint{2.480784in}{0.656748in}}{\pgfqpoint{2.488597in}{0.664562in}}%
\pgfpathcurveto{\pgfqpoint{2.496411in}{0.672375in}}{\pgfqpoint{2.500801in}{0.682974in}}{\pgfqpoint{2.500801in}{0.694024in}}%
\pgfpathcurveto{\pgfqpoint{2.500801in}{0.705074in}}{\pgfqpoint{2.496411in}{0.715673in}}{\pgfqpoint{2.488597in}{0.723487in}}%
\pgfpathcurveto{\pgfqpoint{2.480784in}{0.731301in}}{\pgfqpoint{2.470184in}{0.735691in}}{\pgfqpoint{2.459134in}{0.735691in}}%
\pgfpathcurveto{\pgfqpoint{2.448084in}{0.735691in}}{\pgfqpoint{2.437485in}{0.731301in}}{\pgfqpoint{2.429672in}{0.723487in}}%
\pgfpathcurveto{\pgfqpoint{2.421858in}{0.715673in}}{\pgfqpoint{2.417468in}{0.705074in}}{\pgfqpoint{2.417468in}{0.694024in}}%
\pgfpathcurveto{\pgfqpoint{2.417468in}{0.682974in}}{\pgfqpoint{2.421858in}{0.672375in}}{\pgfqpoint{2.429672in}{0.664562in}}%
\pgfpathcurveto{\pgfqpoint{2.437485in}{0.656748in}}{\pgfqpoint{2.448084in}{0.652358in}}{\pgfqpoint{2.459134in}{0.652358in}}%
\pgfpathclose%
\pgfusepath{stroke,fill}%
\end{pgfscope}%
\begin{pgfscope}%
\pgfpathrectangle{\pgfqpoint{0.648703in}{0.548769in}}{\pgfqpoint{5.195150in}{3.102590in}}%
\pgfusepath{clip}%
\pgfsetbuttcap%
\pgfsetroundjoin%
\definecolor{currentfill}{rgb}{0.121569,0.466667,0.705882}%
\pgfsetfillcolor{currentfill}%
\pgfsetlinewidth{1.003750pt}%
\definecolor{currentstroke}{rgb}{0.121569,0.466667,0.705882}%
\pgfsetstrokecolor{currentstroke}%
\pgfsetdash{}{0pt}%
\pgfpathmoveto{\pgfqpoint{2.790563in}{3.181114in}}%
\pgfpathcurveto{\pgfqpoint{2.801613in}{3.181114in}}{\pgfqpoint{2.812213in}{3.185504in}}{\pgfqpoint{2.820026in}{3.193318in}}%
\pgfpathcurveto{\pgfqpoint{2.827840in}{3.201132in}}{\pgfqpoint{2.832230in}{3.211731in}}{\pgfqpoint{2.832230in}{3.222781in}}%
\pgfpathcurveto{\pgfqpoint{2.832230in}{3.233831in}}{\pgfqpoint{2.827840in}{3.244430in}}{\pgfqpoint{2.820026in}{3.252244in}}%
\pgfpathcurveto{\pgfqpoint{2.812213in}{3.260057in}}{\pgfqpoint{2.801613in}{3.264448in}}{\pgfqpoint{2.790563in}{3.264448in}}%
\pgfpathcurveto{\pgfqpoint{2.779513in}{3.264448in}}{\pgfqpoint{2.768914in}{3.260057in}}{\pgfqpoint{2.761101in}{3.252244in}}%
\pgfpathcurveto{\pgfqpoint{2.753287in}{3.244430in}}{\pgfqpoint{2.748897in}{3.233831in}}{\pgfqpoint{2.748897in}{3.222781in}}%
\pgfpathcurveto{\pgfqpoint{2.748897in}{3.211731in}}{\pgfqpoint{2.753287in}{3.201132in}}{\pgfqpoint{2.761101in}{3.193318in}}%
\pgfpathcurveto{\pgfqpoint{2.768914in}{3.185504in}}{\pgfqpoint{2.779513in}{3.181114in}}{\pgfqpoint{2.790563in}{3.181114in}}%
\pgfpathclose%
\pgfusepath{stroke,fill}%
\end{pgfscope}%
\begin{pgfscope}%
\pgfpathrectangle{\pgfqpoint{0.648703in}{0.548769in}}{\pgfqpoint{5.195150in}{3.102590in}}%
\pgfusepath{clip}%
\pgfsetbuttcap%
\pgfsetroundjoin%
\definecolor{currentfill}{rgb}{1.000000,0.498039,0.054902}%
\pgfsetfillcolor{currentfill}%
\pgfsetlinewidth{1.003750pt}%
\definecolor{currentstroke}{rgb}{1.000000,0.498039,0.054902}%
\pgfsetstrokecolor{currentstroke}%
\pgfsetdash{}{0pt}%
\pgfpathmoveto{\pgfqpoint{1.879134in}{3.198029in}}%
\pgfpathcurveto{\pgfqpoint{1.890184in}{3.198029in}}{\pgfqpoint{1.900783in}{3.202419in}}{\pgfqpoint{1.908596in}{3.210233in}}%
\pgfpathcurveto{\pgfqpoint{1.916410in}{3.218046in}}{\pgfqpoint{1.920800in}{3.228646in}}{\pgfqpoint{1.920800in}{3.239696in}}%
\pgfpathcurveto{\pgfqpoint{1.920800in}{3.250746in}}{\pgfqpoint{1.916410in}{3.261345in}}{\pgfqpoint{1.908596in}{3.269158in}}%
\pgfpathcurveto{\pgfqpoint{1.900783in}{3.276972in}}{\pgfqpoint{1.890184in}{3.281362in}}{\pgfqpoint{1.879134in}{3.281362in}}%
\pgfpathcurveto{\pgfqpoint{1.868083in}{3.281362in}}{\pgfqpoint{1.857484in}{3.276972in}}{\pgfqpoint{1.849671in}{3.269158in}}%
\pgfpathcurveto{\pgfqpoint{1.841857in}{3.261345in}}{\pgfqpoint{1.837467in}{3.250746in}}{\pgfqpoint{1.837467in}{3.239696in}}%
\pgfpathcurveto{\pgfqpoint{1.837467in}{3.228646in}}{\pgfqpoint{1.841857in}{3.218046in}}{\pgfqpoint{1.849671in}{3.210233in}}%
\pgfpathcurveto{\pgfqpoint{1.857484in}{3.202419in}}{\pgfqpoint{1.868083in}{3.198029in}}{\pgfqpoint{1.879134in}{3.198029in}}%
\pgfpathclose%
\pgfusepath{stroke,fill}%
\end{pgfscope}%
\begin{pgfscope}%
\pgfpathrectangle{\pgfqpoint{0.648703in}{0.548769in}}{\pgfqpoint{5.195150in}{3.102590in}}%
\pgfusepath{clip}%
\pgfsetbuttcap%
\pgfsetroundjoin%
\definecolor{currentfill}{rgb}{1.000000,0.498039,0.054902}%
\pgfsetfillcolor{currentfill}%
\pgfsetlinewidth{1.003750pt}%
\definecolor{currentstroke}{rgb}{1.000000,0.498039,0.054902}%
\pgfsetstrokecolor{currentstroke}%
\pgfsetdash{}{0pt}%
\pgfpathmoveto{\pgfqpoint{1.630562in}{3.248773in}}%
\pgfpathcurveto{\pgfqpoint{1.641612in}{3.248773in}}{\pgfqpoint{1.652211in}{3.253164in}}{\pgfqpoint{1.660025in}{3.260977in}}%
\pgfpathcurveto{\pgfqpoint{1.667838in}{3.268791in}}{\pgfqpoint{1.672229in}{3.279390in}}{\pgfqpoint{1.672229in}{3.290440in}}%
\pgfpathcurveto{\pgfqpoint{1.672229in}{3.301490in}}{\pgfqpoint{1.667838in}{3.312089in}}{\pgfqpoint{1.660025in}{3.319903in}}%
\pgfpathcurveto{\pgfqpoint{1.652211in}{3.327716in}}{\pgfqpoint{1.641612in}{3.332107in}}{\pgfqpoint{1.630562in}{3.332107in}}%
\pgfpathcurveto{\pgfqpoint{1.619512in}{3.332107in}}{\pgfqpoint{1.608913in}{3.327716in}}{\pgfqpoint{1.601099in}{3.319903in}}%
\pgfpathcurveto{\pgfqpoint{1.593285in}{3.312089in}}{\pgfqpoint{1.588895in}{3.301490in}}{\pgfqpoint{1.588895in}{3.290440in}}%
\pgfpathcurveto{\pgfqpoint{1.588895in}{3.279390in}}{\pgfqpoint{1.593285in}{3.268791in}}{\pgfqpoint{1.601099in}{3.260977in}}%
\pgfpathcurveto{\pgfqpoint{1.608913in}{3.253164in}}{\pgfqpoint{1.619512in}{3.248773in}}{\pgfqpoint{1.630562in}{3.248773in}}%
\pgfpathclose%
\pgfusepath{stroke,fill}%
\end{pgfscope}%
\begin{pgfscope}%
\pgfpathrectangle{\pgfqpoint{0.648703in}{0.548769in}}{\pgfqpoint{5.195150in}{3.102590in}}%
\pgfusepath{clip}%
\pgfsetbuttcap%
\pgfsetroundjoin%
\definecolor{currentfill}{rgb}{1.000000,0.498039,0.054902}%
\pgfsetfillcolor{currentfill}%
\pgfsetlinewidth{1.003750pt}%
\definecolor{currentstroke}{rgb}{1.000000,0.498039,0.054902}%
\pgfsetstrokecolor{currentstroke}%
\pgfsetdash{}{0pt}%
\pgfpathmoveto{\pgfqpoint{2.210563in}{3.189572in}}%
\pgfpathcurveto{\pgfqpoint{2.221613in}{3.189572in}}{\pgfqpoint{2.232212in}{3.193962in}}{\pgfqpoint{2.240025in}{3.201775in}}%
\pgfpathcurveto{\pgfqpoint{2.247839in}{3.209589in}}{\pgfqpoint{2.252229in}{3.220188in}}{\pgfqpoint{2.252229in}{3.231238in}}%
\pgfpathcurveto{\pgfqpoint{2.252229in}{3.242288in}}{\pgfqpoint{2.247839in}{3.252887in}}{\pgfqpoint{2.240025in}{3.260701in}}%
\pgfpathcurveto{\pgfqpoint{2.232212in}{3.268515in}}{\pgfqpoint{2.221613in}{3.272905in}}{\pgfqpoint{2.210563in}{3.272905in}}%
\pgfpathcurveto{\pgfqpoint{2.199512in}{3.272905in}}{\pgfqpoint{2.188913in}{3.268515in}}{\pgfqpoint{2.181100in}{3.260701in}}%
\pgfpathcurveto{\pgfqpoint{2.173286in}{3.252887in}}{\pgfqpoint{2.168896in}{3.242288in}}{\pgfqpoint{2.168896in}{3.231238in}}%
\pgfpathcurveto{\pgfqpoint{2.168896in}{3.220188in}}{\pgfqpoint{2.173286in}{3.209589in}}{\pgfqpoint{2.181100in}{3.201775in}}%
\pgfpathcurveto{\pgfqpoint{2.188913in}{3.193962in}}{\pgfqpoint{2.199512in}{3.189572in}}{\pgfqpoint{2.210563in}{3.189572in}}%
\pgfpathclose%
\pgfusepath{stroke,fill}%
\end{pgfscope}%
\begin{pgfscope}%
\pgfpathrectangle{\pgfqpoint{0.648703in}{0.548769in}}{\pgfqpoint{5.195150in}{3.102590in}}%
\pgfusepath{clip}%
\pgfsetbuttcap%
\pgfsetroundjoin%
\definecolor{currentfill}{rgb}{1.000000,0.498039,0.054902}%
\pgfsetfillcolor{currentfill}%
\pgfsetlinewidth{1.003750pt}%
\definecolor{currentstroke}{rgb}{1.000000,0.498039,0.054902}%
\pgfsetstrokecolor{currentstroke}%
\pgfsetdash{}{0pt}%
\pgfpathmoveto{\pgfqpoint{1.879134in}{3.189572in}}%
\pgfpathcurveto{\pgfqpoint{1.890184in}{3.189572in}}{\pgfqpoint{1.900783in}{3.193962in}}{\pgfqpoint{1.908596in}{3.201775in}}%
\pgfpathcurveto{\pgfqpoint{1.916410in}{3.209589in}}{\pgfqpoint{1.920800in}{3.220188in}}{\pgfqpoint{1.920800in}{3.231238in}}%
\pgfpathcurveto{\pgfqpoint{1.920800in}{3.242288in}}{\pgfqpoint{1.916410in}{3.252887in}}{\pgfqpoint{1.908596in}{3.260701in}}%
\pgfpathcurveto{\pgfqpoint{1.900783in}{3.268515in}}{\pgfqpoint{1.890184in}{3.272905in}}{\pgfqpoint{1.879134in}{3.272905in}}%
\pgfpathcurveto{\pgfqpoint{1.868083in}{3.272905in}}{\pgfqpoint{1.857484in}{3.268515in}}{\pgfqpoint{1.849671in}{3.260701in}}%
\pgfpathcurveto{\pgfqpoint{1.841857in}{3.252887in}}{\pgfqpoint{1.837467in}{3.242288in}}{\pgfqpoint{1.837467in}{3.231238in}}%
\pgfpathcurveto{\pgfqpoint{1.837467in}{3.220188in}}{\pgfqpoint{1.841857in}{3.209589in}}{\pgfqpoint{1.849671in}{3.201775in}}%
\pgfpathcurveto{\pgfqpoint{1.857484in}{3.193962in}}{\pgfqpoint{1.868083in}{3.189572in}}{\pgfqpoint{1.879134in}{3.189572in}}%
\pgfpathclose%
\pgfusepath{stroke,fill}%
\end{pgfscope}%
\begin{pgfscope}%
\pgfpathrectangle{\pgfqpoint{0.648703in}{0.548769in}}{\pgfqpoint{5.195150in}{3.102590in}}%
\pgfusepath{clip}%
\pgfsetbuttcap%
\pgfsetroundjoin%
\definecolor{currentfill}{rgb}{0.121569,0.466667,0.705882}%
\pgfsetfillcolor{currentfill}%
\pgfsetlinewidth{1.003750pt}%
\definecolor{currentstroke}{rgb}{0.121569,0.466667,0.705882}%
\pgfsetstrokecolor{currentstroke}%
\pgfsetdash{}{0pt}%
\pgfpathmoveto{\pgfqpoint{4.281994in}{3.181114in}}%
\pgfpathcurveto{\pgfqpoint{4.293044in}{3.181114in}}{\pgfqpoint{4.303643in}{3.185504in}}{\pgfqpoint{4.311457in}{3.193318in}}%
\pgfpathcurveto{\pgfqpoint{4.319270in}{3.201132in}}{\pgfqpoint{4.323661in}{3.211731in}}{\pgfqpoint{4.323661in}{3.222781in}}%
\pgfpathcurveto{\pgfqpoint{4.323661in}{3.233831in}}{\pgfqpoint{4.319270in}{3.244430in}}{\pgfqpoint{4.311457in}{3.252244in}}%
\pgfpathcurveto{\pgfqpoint{4.303643in}{3.260057in}}{\pgfqpoint{4.293044in}{3.264448in}}{\pgfqpoint{4.281994in}{3.264448in}}%
\pgfpathcurveto{\pgfqpoint{4.270944in}{3.264448in}}{\pgfqpoint{4.260345in}{3.260057in}}{\pgfqpoint{4.252531in}{3.252244in}}%
\pgfpathcurveto{\pgfqpoint{4.244717in}{3.244430in}}{\pgfqpoint{4.240327in}{3.233831in}}{\pgfqpoint{4.240327in}{3.222781in}}%
\pgfpathcurveto{\pgfqpoint{4.240327in}{3.211731in}}{\pgfqpoint{4.244717in}{3.201132in}}{\pgfqpoint{4.252531in}{3.193318in}}%
\pgfpathcurveto{\pgfqpoint{4.260345in}{3.185504in}}{\pgfqpoint{4.270944in}{3.181114in}}{\pgfqpoint{4.281994in}{3.181114in}}%
\pgfpathclose%
\pgfusepath{stroke,fill}%
\end{pgfscope}%
\begin{pgfscope}%
\pgfpathrectangle{\pgfqpoint{0.648703in}{0.548769in}}{\pgfqpoint{5.195150in}{3.102590in}}%
\pgfusepath{clip}%
\pgfsetbuttcap%
\pgfsetroundjoin%
\definecolor{currentfill}{rgb}{0.121569,0.466667,0.705882}%
\pgfsetfillcolor{currentfill}%
\pgfsetlinewidth{1.003750pt}%
\definecolor{currentstroke}{rgb}{0.121569,0.466667,0.705882}%
\pgfsetstrokecolor{currentstroke}%
\pgfsetdash{}{0pt}%
\pgfpathmoveto{\pgfqpoint{1.381990in}{0.648129in}}%
\pgfpathcurveto{\pgfqpoint{1.393040in}{0.648129in}}{\pgfqpoint{1.403639in}{0.652519in}}{\pgfqpoint{1.411453in}{0.660333in}}%
\pgfpathcurveto{\pgfqpoint{1.419266in}{0.668146in}}{\pgfqpoint{1.423657in}{0.678745in}}{\pgfqpoint{1.423657in}{0.689796in}}%
\pgfpathcurveto{\pgfqpoint{1.423657in}{0.700846in}}{\pgfqpoint{1.419266in}{0.711445in}}{\pgfqpoint{1.411453in}{0.719258in}}%
\pgfpathcurveto{\pgfqpoint{1.403639in}{0.727072in}}{\pgfqpoint{1.393040in}{0.731462in}}{\pgfqpoint{1.381990in}{0.731462in}}%
\pgfpathcurveto{\pgfqpoint{1.370940in}{0.731462in}}{\pgfqpoint{1.360341in}{0.727072in}}{\pgfqpoint{1.352527in}{0.719258in}}%
\pgfpathcurveto{\pgfqpoint{1.344714in}{0.711445in}}{\pgfqpoint{1.340323in}{0.700846in}}{\pgfqpoint{1.340323in}{0.689796in}}%
\pgfpathcurveto{\pgfqpoint{1.340323in}{0.678745in}}{\pgfqpoint{1.344714in}{0.668146in}}{\pgfqpoint{1.352527in}{0.660333in}}%
\pgfpathcurveto{\pgfqpoint{1.360341in}{0.652519in}}{\pgfqpoint{1.370940in}{0.648129in}}{\pgfqpoint{1.381990in}{0.648129in}}%
\pgfpathclose%
\pgfusepath{stroke,fill}%
\end{pgfscope}%
\begin{pgfscope}%
\pgfpathrectangle{\pgfqpoint{0.648703in}{0.548769in}}{\pgfqpoint{5.195150in}{3.102590in}}%
\pgfusepath{clip}%
\pgfsetbuttcap%
\pgfsetroundjoin%
\definecolor{currentfill}{rgb}{0.839216,0.152941,0.156863}%
\pgfsetfillcolor{currentfill}%
\pgfsetlinewidth{1.003750pt}%
\definecolor{currentstroke}{rgb}{0.839216,0.152941,0.156863}%
\pgfsetstrokecolor{currentstroke}%
\pgfsetdash{}{0pt}%
\pgfpathmoveto{\pgfqpoint{3.039135in}{3.202258in}}%
\pgfpathcurveto{\pgfqpoint{3.050185in}{3.202258in}}{\pgfqpoint{3.060784in}{3.206648in}}{\pgfqpoint{3.068598in}{3.214462in}}%
\pgfpathcurveto{\pgfqpoint{3.076412in}{3.222275in}}{\pgfqpoint{3.080802in}{3.232874in}}{\pgfqpoint{3.080802in}{3.243924in}}%
\pgfpathcurveto{\pgfqpoint{3.080802in}{3.254974in}}{\pgfqpoint{3.076412in}{3.265573in}}{\pgfqpoint{3.068598in}{3.273387in}}%
\pgfpathcurveto{\pgfqpoint{3.060784in}{3.281201in}}{\pgfqpoint{3.050185in}{3.285591in}}{\pgfqpoint{3.039135in}{3.285591in}}%
\pgfpathcurveto{\pgfqpoint{3.028085in}{3.285591in}}{\pgfqpoint{3.017486in}{3.281201in}}{\pgfqpoint{3.009672in}{3.273387in}}%
\pgfpathcurveto{\pgfqpoint{3.001859in}{3.265573in}}{\pgfqpoint{2.997468in}{3.254974in}}{\pgfqpoint{2.997468in}{3.243924in}}%
\pgfpathcurveto{\pgfqpoint{2.997468in}{3.232874in}}{\pgfqpoint{3.001859in}{3.222275in}}{\pgfqpoint{3.009672in}{3.214462in}}%
\pgfpathcurveto{\pgfqpoint{3.017486in}{3.206648in}}{\pgfqpoint{3.028085in}{3.202258in}}{\pgfqpoint{3.039135in}{3.202258in}}%
\pgfpathclose%
\pgfusepath{stroke,fill}%
\end{pgfscope}%
\begin{pgfscope}%
\pgfpathrectangle{\pgfqpoint{0.648703in}{0.548769in}}{\pgfqpoint{5.195150in}{3.102590in}}%
\pgfusepath{clip}%
\pgfsetbuttcap%
\pgfsetroundjoin%
\definecolor{currentfill}{rgb}{1.000000,0.498039,0.054902}%
\pgfsetfillcolor{currentfill}%
\pgfsetlinewidth{1.003750pt}%
\definecolor{currentstroke}{rgb}{1.000000,0.498039,0.054902}%
\pgfsetstrokecolor{currentstroke}%
\pgfsetdash{}{0pt}%
\pgfpathmoveto{\pgfqpoint{2.127705in}{3.185343in}}%
\pgfpathcurveto{\pgfqpoint{2.138755in}{3.185343in}}{\pgfqpoint{2.149355in}{3.189733in}}{\pgfqpoint{2.157168in}{3.197547in}}%
\pgfpathcurveto{\pgfqpoint{2.164982in}{3.205360in}}{\pgfqpoint{2.169372in}{3.215959in}}{\pgfqpoint{2.169372in}{3.227010in}}%
\pgfpathcurveto{\pgfqpoint{2.169372in}{3.238060in}}{\pgfqpoint{2.164982in}{3.248659in}}{\pgfqpoint{2.157168in}{3.256472in}}%
\pgfpathcurveto{\pgfqpoint{2.149355in}{3.264286in}}{\pgfqpoint{2.138755in}{3.268676in}}{\pgfqpoint{2.127705in}{3.268676in}}%
\pgfpathcurveto{\pgfqpoint{2.116655in}{3.268676in}}{\pgfqpoint{2.106056in}{3.264286in}}{\pgfqpoint{2.098243in}{3.256472in}}%
\pgfpathcurveto{\pgfqpoint{2.090429in}{3.248659in}}{\pgfqpoint{2.086039in}{3.238060in}}{\pgfqpoint{2.086039in}{3.227010in}}%
\pgfpathcurveto{\pgfqpoint{2.086039in}{3.215959in}}{\pgfqpoint{2.090429in}{3.205360in}}{\pgfqpoint{2.098243in}{3.197547in}}%
\pgfpathcurveto{\pgfqpoint{2.106056in}{3.189733in}}{\pgfqpoint{2.116655in}{3.185343in}}{\pgfqpoint{2.127705in}{3.185343in}}%
\pgfpathclose%
\pgfusepath{stroke,fill}%
\end{pgfscope}%
\begin{pgfscope}%
\pgfpathrectangle{\pgfqpoint{0.648703in}{0.548769in}}{\pgfqpoint{5.195150in}{3.102590in}}%
\pgfusepath{clip}%
\pgfsetbuttcap%
\pgfsetroundjoin%
\definecolor{currentfill}{rgb}{0.121569,0.466667,0.705882}%
\pgfsetfillcolor{currentfill}%
\pgfsetlinewidth{1.003750pt}%
\definecolor{currentstroke}{rgb}{0.121569,0.466667,0.705882}%
\pgfsetstrokecolor{currentstroke}%
\pgfsetdash{}{0pt}%
\pgfpathmoveto{\pgfqpoint{1.547705in}{0.774990in}}%
\pgfpathcurveto{\pgfqpoint{1.558755in}{0.774990in}}{\pgfqpoint{1.569354in}{0.779380in}}{\pgfqpoint{1.577167in}{0.787194in}}%
\pgfpathcurveto{\pgfqpoint{1.584981in}{0.795007in}}{\pgfqpoint{1.589371in}{0.805606in}}{\pgfqpoint{1.589371in}{0.816656in}}%
\pgfpathcurveto{\pgfqpoint{1.589371in}{0.827706in}}{\pgfqpoint{1.584981in}{0.838305in}}{\pgfqpoint{1.577167in}{0.846119in}}%
\pgfpathcurveto{\pgfqpoint{1.569354in}{0.853933in}}{\pgfqpoint{1.558755in}{0.858323in}}{\pgfqpoint{1.547705in}{0.858323in}}%
\pgfpathcurveto{\pgfqpoint{1.536654in}{0.858323in}}{\pgfqpoint{1.526055in}{0.853933in}}{\pgfqpoint{1.518242in}{0.846119in}}%
\pgfpathcurveto{\pgfqpoint{1.510428in}{0.838305in}}{\pgfqpoint{1.506038in}{0.827706in}}{\pgfqpoint{1.506038in}{0.816656in}}%
\pgfpathcurveto{\pgfqpoint{1.506038in}{0.805606in}}{\pgfqpoint{1.510428in}{0.795007in}}{\pgfqpoint{1.518242in}{0.787194in}}%
\pgfpathcurveto{\pgfqpoint{1.526055in}{0.779380in}}{\pgfqpoint{1.536654in}{0.774990in}}{\pgfqpoint{1.547705in}{0.774990in}}%
\pgfpathclose%
\pgfusepath{stroke,fill}%
\end{pgfscope}%
\begin{pgfscope}%
\pgfpathrectangle{\pgfqpoint{0.648703in}{0.548769in}}{\pgfqpoint{5.195150in}{3.102590in}}%
\pgfusepath{clip}%
\pgfsetbuttcap%
\pgfsetroundjoin%
\definecolor{currentfill}{rgb}{0.121569,0.466667,0.705882}%
\pgfsetfillcolor{currentfill}%
\pgfsetlinewidth{1.003750pt}%
\definecolor{currentstroke}{rgb}{0.121569,0.466667,0.705882}%
\pgfsetstrokecolor{currentstroke}%
\pgfsetdash{}{0pt}%
\pgfpathmoveto{\pgfqpoint{2.293420in}{3.181114in}}%
\pgfpathcurveto{\pgfqpoint{2.304470in}{3.181114in}}{\pgfqpoint{2.315069in}{3.185504in}}{\pgfqpoint{2.322883in}{3.193318in}}%
\pgfpathcurveto{\pgfqpoint{2.330696in}{3.201132in}}{\pgfqpoint{2.335087in}{3.211731in}}{\pgfqpoint{2.335087in}{3.222781in}}%
\pgfpathcurveto{\pgfqpoint{2.335087in}{3.233831in}}{\pgfqpoint{2.330696in}{3.244430in}}{\pgfqpoint{2.322883in}{3.252244in}}%
\pgfpathcurveto{\pgfqpoint{2.315069in}{3.260057in}}{\pgfqpoint{2.304470in}{3.264448in}}{\pgfqpoint{2.293420in}{3.264448in}}%
\pgfpathcurveto{\pgfqpoint{2.282370in}{3.264448in}}{\pgfqpoint{2.271771in}{3.260057in}}{\pgfqpoint{2.263957in}{3.252244in}}%
\pgfpathcurveto{\pgfqpoint{2.256143in}{3.244430in}}{\pgfqpoint{2.251753in}{3.233831in}}{\pgfqpoint{2.251753in}{3.222781in}}%
\pgfpathcurveto{\pgfqpoint{2.251753in}{3.211731in}}{\pgfqpoint{2.256143in}{3.201132in}}{\pgfqpoint{2.263957in}{3.193318in}}%
\pgfpathcurveto{\pgfqpoint{2.271771in}{3.185504in}}{\pgfqpoint{2.282370in}{3.181114in}}{\pgfqpoint{2.293420in}{3.181114in}}%
\pgfpathclose%
\pgfusepath{stroke,fill}%
\end{pgfscope}%
\begin{pgfscope}%
\pgfpathrectangle{\pgfqpoint{0.648703in}{0.548769in}}{\pgfqpoint{5.195150in}{3.102590in}}%
\pgfusepath{clip}%
\pgfsetbuttcap%
\pgfsetroundjoin%
\definecolor{currentfill}{rgb}{0.121569,0.466667,0.705882}%
\pgfsetfillcolor{currentfill}%
\pgfsetlinewidth{1.003750pt}%
\definecolor{currentstroke}{rgb}{0.121569,0.466667,0.705882}%
\pgfsetstrokecolor{currentstroke}%
\pgfsetdash{}{0pt}%
\pgfpathmoveto{\pgfqpoint{1.216276in}{0.783447in}}%
\pgfpathcurveto{\pgfqpoint{1.227326in}{0.783447in}}{\pgfqpoint{1.237925in}{0.787837in}}{\pgfqpoint{1.245738in}{0.795651in}}%
\pgfpathcurveto{\pgfqpoint{1.253552in}{0.803465in}}{\pgfqpoint{1.257942in}{0.814064in}}{\pgfqpoint{1.257942in}{0.825114in}}%
\pgfpathcurveto{\pgfqpoint{1.257942in}{0.836164in}}{\pgfqpoint{1.253552in}{0.846763in}}{\pgfqpoint{1.245738in}{0.854576in}}%
\pgfpathcurveto{\pgfqpoint{1.237925in}{0.862390in}}{\pgfqpoint{1.227326in}{0.866780in}}{\pgfqpoint{1.216276in}{0.866780in}}%
\pgfpathcurveto{\pgfqpoint{1.205225in}{0.866780in}}{\pgfqpoint{1.194626in}{0.862390in}}{\pgfqpoint{1.186813in}{0.854576in}}%
\pgfpathcurveto{\pgfqpoint{1.178999in}{0.846763in}}{\pgfqpoint{1.174609in}{0.836164in}}{\pgfqpoint{1.174609in}{0.825114in}}%
\pgfpathcurveto{\pgfqpoint{1.174609in}{0.814064in}}{\pgfqpoint{1.178999in}{0.803465in}}{\pgfqpoint{1.186813in}{0.795651in}}%
\pgfpathcurveto{\pgfqpoint{1.194626in}{0.787837in}}{\pgfqpoint{1.205225in}{0.783447in}}{\pgfqpoint{1.216276in}{0.783447in}}%
\pgfpathclose%
\pgfusepath{stroke,fill}%
\end{pgfscope}%
\begin{pgfscope}%
\pgfpathrectangle{\pgfqpoint{0.648703in}{0.548769in}}{\pgfqpoint{5.195150in}{3.102590in}}%
\pgfusepath{clip}%
\pgfsetbuttcap%
\pgfsetroundjoin%
\definecolor{currentfill}{rgb}{0.121569,0.466667,0.705882}%
\pgfsetfillcolor{currentfill}%
\pgfsetlinewidth{1.003750pt}%
\definecolor{currentstroke}{rgb}{0.121569,0.466667,0.705882}%
\pgfsetstrokecolor{currentstroke}%
\pgfsetdash{}{0pt}%
\pgfpathmoveto{\pgfqpoint{1.547705in}{0.652358in}}%
\pgfpathcurveto{\pgfqpoint{1.558755in}{0.652358in}}{\pgfqpoint{1.569354in}{0.656748in}}{\pgfqpoint{1.577167in}{0.664562in}}%
\pgfpathcurveto{\pgfqpoint{1.584981in}{0.672375in}}{\pgfqpoint{1.589371in}{0.682974in}}{\pgfqpoint{1.589371in}{0.694024in}}%
\pgfpathcurveto{\pgfqpoint{1.589371in}{0.705074in}}{\pgfqpoint{1.584981in}{0.715673in}}{\pgfqpoint{1.577167in}{0.723487in}}%
\pgfpathcurveto{\pgfqpoint{1.569354in}{0.731301in}}{\pgfqpoint{1.558755in}{0.735691in}}{\pgfqpoint{1.547705in}{0.735691in}}%
\pgfpathcurveto{\pgfqpoint{1.536654in}{0.735691in}}{\pgfqpoint{1.526055in}{0.731301in}}{\pgfqpoint{1.518242in}{0.723487in}}%
\pgfpathcurveto{\pgfqpoint{1.510428in}{0.715673in}}{\pgfqpoint{1.506038in}{0.705074in}}{\pgfqpoint{1.506038in}{0.694024in}}%
\pgfpathcurveto{\pgfqpoint{1.506038in}{0.682974in}}{\pgfqpoint{1.510428in}{0.672375in}}{\pgfqpoint{1.518242in}{0.664562in}}%
\pgfpathcurveto{\pgfqpoint{1.526055in}{0.656748in}}{\pgfqpoint{1.536654in}{0.652358in}}{\pgfqpoint{1.547705in}{0.652358in}}%
\pgfpathclose%
\pgfusepath{stroke,fill}%
\end{pgfscope}%
\begin{pgfscope}%
\pgfpathrectangle{\pgfqpoint{0.648703in}{0.548769in}}{\pgfqpoint{5.195150in}{3.102590in}}%
\pgfusepath{clip}%
\pgfsetbuttcap%
\pgfsetroundjoin%
\definecolor{currentfill}{rgb}{1.000000,0.498039,0.054902}%
\pgfsetfillcolor{currentfill}%
\pgfsetlinewidth{1.003750pt}%
\definecolor{currentstroke}{rgb}{1.000000,0.498039,0.054902}%
\pgfsetstrokecolor{currentstroke}%
\pgfsetdash{}{0pt}%
\pgfpathmoveto{\pgfqpoint{1.879134in}{3.219172in}}%
\pgfpathcurveto{\pgfqpoint{1.890184in}{3.219172in}}{\pgfqpoint{1.900783in}{3.223563in}}{\pgfqpoint{1.908596in}{3.231376in}}%
\pgfpathcurveto{\pgfqpoint{1.916410in}{3.239190in}}{\pgfqpoint{1.920800in}{3.249789in}}{\pgfqpoint{1.920800in}{3.260839in}}%
\pgfpathcurveto{\pgfqpoint{1.920800in}{3.271889in}}{\pgfqpoint{1.916410in}{3.282488in}}{\pgfqpoint{1.908596in}{3.290302in}}%
\pgfpathcurveto{\pgfqpoint{1.900783in}{3.298116in}}{\pgfqpoint{1.890184in}{3.302506in}}{\pgfqpoint{1.879134in}{3.302506in}}%
\pgfpathcurveto{\pgfqpoint{1.868083in}{3.302506in}}{\pgfqpoint{1.857484in}{3.298116in}}{\pgfqpoint{1.849671in}{3.290302in}}%
\pgfpathcurveto{\pgfqpoint{1.841857in}{3.282488in}}{\pgfqpoint{1.837467in}{3.271889in}}{\pgfqpoint{1.837467in}{3.260839in}}%
\pgfpathcurveto{\pgfqpoint{1.837467in}{3.249789in}}{\pgfqpoint{1.841857in}{3.239190in}}{\pgfqpoint{1.849671in}{3.231376in}}%
\pgfpathcurveto{\pgfqpoint{1.857484in}{3.223563in}}{\pgfqpoint{1.868083in}{3.219172in}}{\pgfqpoint{1.879134in}{3.219172in}}%
\pgfpathclose%
\pgfusepath{stroke,fill}%
\end{pgfscope}%
\begin{pgfscope}%
\pgfpathrectangle{\pgfqpoint{0.648703in}{0.548769in}}{\pgfqpoint{5.195150in}{3.102590in}}%
\pgfusepath{clip}%
\pgfsetbuttcap%
\pgfsetroundjoin%
\definecolor{currentfill}{rgb}{0.121569,0.466667,0.705882}%
\pgfsetfillcolor{currentfill}%
\pgfsetlinewidth{1.003750pt}%
\definecolor{currentstroke}{rgb}{0.121569,0.466667,0.705882}%
\pgfsetstrokecolor{currentstroke}%
\pgfsetdash{}{0pt}%
\pgfpathmoveto{\pgfqpoint{3.453421in}{0.648129in}}%
\pgfpathcurveto{\pgfqpoint{3.464471in}{0.648129in}}{\pgfqpoint{3.475071in}{0.652519in}}{\pgfqpoint{3.482884in}{0.660333in}}%
\pgfpathcurveto{\pgfqpoint{3.490698in}{0.668146in}}{\pgfqpoint{3.495088in}{0.678745in}}{\pgfqpoint{3.495088in}{0.689796in}}%
\pgfpathcurveto{\pgfqpoint{3.495088in}{0.700846in}}{\pgfqpoint{3.490698in}{0.711445in}}{\pgfqpoint{3.482884in}{0.719258in}}%
\pgfpathcurveto{\pgfqpoint{3.475071in}{0.727072in}}{\pgfqpoint{3.464471in}{0.731462in}}{\pgfqpoint{3.453421in}{0.731462in}}%
\pgfpathcurveto{\pgfqpoint{3.442371in}{0.731462in}}{\pgfqpoint{3.431772in}{0.727072in}}{\pgfqpoint{3.423959in}{0.719258in}}%
\pgfpathcurveto{\pgfqpoint{3.416145in}{0.711445in}}{\pgfqpoint{3.411755in}{0.700846in}}{\pgfqpoint{3.411755in}{0.689796in}}%
\pgfpathcurveto{\pgfqpoint{3.411755in}{0.678745in}}{\pgfqpoint{3.416145in}{0.668146in}}{\pgfqpoint{3.423959in}{0.660333in}}%
\pgfpathcurveto{\pgfqpoint{3.431772in}{0.652519in}}{\pgfqpoint{3.442371in}{0.648129in}}{\pgfqpoint{3.453421in}{0.648129in}}%
\pgfpathclose%
\pgfusepath{stroke,fill}%
\end{pgfscope}%
\begin{pgfscope}%
\pgfpathrectangle{\pgfqpoint{0.648703in}{0.548769in}}{\pgfqpoint{5.195150in}{3.102590in}}%
\pgfusepath{clip}%
\pgfsetbuttcap%
\pgfsetroundjoin%
\definecolor{currentfill}{rgb}{0.121569,0.466667,0.705882}%
\pgfsetfillcolor{currentfill}%
\pgfsetlinewidth{1.003750pt}%
\definecolor{currentstroke}{rgb}{0.121569,0.466667,0.705882}%
\pgfsetstrokecolor{currentstroke}%
\pgfsetdash{}{0pt}%
\pgfpathmoveto{\pgfqpoint{1.547705in}{0.648129in}}%
\pgfpathcurveto{\pgfqpoint{1.558755in}{0.648129in}}{\pgfqpoint{1.569354in}{0.652519in}}{\pgfqpoint{1.577167in}{0.660333in}}%
\pgfpathcurveto{\pgfqpoint{1.584981in}{0.668146in}}{\pgfqpoint{1.589371in}{0.678745in}}{\pgfqpoint{1.589371in}{0.689796in}}%
\pgfpathcurveto{\pgfqpoint{1.589371in}{0.700846in}}{\pgfqpoint{1.584981in}{0.711445in}}{\pgfqpoint{1.577167in}{0.719258in}}%
\pgfpathcurveto{\pgfqpoint{1.569354in}{0.727072in}}{\pgfqpoint{1.558755in}{0.731462in}}{\pgfqpoint{1.547705in}{0.731462in}}%
\pgfpathcurveto{\pgfqpoint{1.536654in}{0.731462in}}{\pgfqpoint{1.526055in}{0.727072in}}{\pgfqpoint{1.518242in}{0.719258in}}%
\pgfpathcurveto{\pgfqpoint{1.510428in}{0.711445in}}{\pgfqpoint{1.506038in}{0.700846in}}{\pgfqpoint{1.506038in}{0.689796in}}%
\pgfpathcurveto{\pgfqpoint{1.506038in}{0.678745in}}{\pgfqpoint{1.510428in}{0.668146in}}{\pgfqpoint{1.518242in}{0.660333in}}%
\pgfpathcurveto{\pgfqpoint{1.526055in}{0.652519in}}{\pgfqpoint{1.536654in}{0.648129in}}{\pgfqpoint{1.547705in}{0.648129in}}%
\pgfpathclose%
\pgfusepath{stroke,fill}%
\end{pgfscope}%
\begin{pgfscope}%
\pgfpathrectangle{\pgfqpoint{0.648703in}{0.548769in}}{\pgfqpoint{5.195150in}{3.102590in}}%
\pgfusepath{clip}%
\pgfsetbuttcap%
\pgfsetroundjoin%
\definecolor{currentfill}{rgb}{1.000000,0.498039,0.054902}%
\pgfsetfillcolor{currentfill}%
\pgfsetlinewidth{1.003750pt}%
\definecolor{currentstroke}{rgb}{1.000000,0.498039,0.054902}%
\pgfsetstrokecolor{currentstroke}%
\pgfsetdash{}{0pt}%
\pgfpathmoveto{\pgfqpoint{1.961991in}{3.468665in}}%
\pgfpathcurveto{\pgfqpoint{1.973041in}{3.468665in}}{\pgfqpoint{1.983640in}{3.473055in}}{\pgfqpoint{1.991454in}{3.480869in}}%
\pgfpathcurveto{\pgfqpoint{1.999267in}{3.488683in}}{\pgfqpoint{2.003658in}{3.499282in}}{\pgfqpoint{2.003658in}{3.510332in}}%
\pgfpathcurveto{\pgfqpoint{2.003658in}{3.521382in}}{\pgfqpoint{1.999267in}{3.531981in}}{\pgfqpoint{1.991454in}{3.539795in}}%
\pgfpathcurveto{\pgfqpoint{1.983640in}{3.547608in}}{\pgfqpoint{1.973041in}{3.551998in}}{\pgfqpoint{1.961991in}{3.551998in}}%
\pgfpathcurveto{\pgfqpoint{1.950941in}{3.551998in}}{\pgfqpoint{1.940342in}{3.547608in}}{\pgfqpoint{1.932528in}{3.539795in}}%
\pgfpathcurveto{\pgfqpoint{1.924714in}{3.531981in}}{\pgfqpoint{1.920324in}{3.521382in}}{\pgfqpoint{1.920324in}{3.510332in}}%
\pgfpathcurveto{\pgfqpoint{1.920324in}{3.499282in}}{\pgfqpoint{1.924714in}{3.488683in}}{\pgfqpoint{1.932528in}{3.480869in}}%
\pgfpathcurveto{\pgfqpoint{1.940342in}{3.473055in}}{\pgfqpoint{1.950941in}{3.468665in}}{\pgfqpoint{1.961991in}{3.468665in}}%
\pgfpathclose%
\pgfusepath{stroke,fill}%
\end{pgfscope}%
\begin{pgfscope}%
\pgfpathrectangle{\pgfqpoint{0.648703in}{0.548769in}}{\pgfqpoint{5.195150in}{3.102590in}}%
\pgfusepath{clip}%
\pgfsetbuttcap%
\pgfsetroundjoin%
\definecolor{currentfill}{rgb}{1.000000,0.498039,0.054902}%
\pgfsetfillcolor{currentfill}%
\pgfsetlinewidth{1.003750pt}%
\definecolor{currentstroke}{rgb}{1.000000,0.498039,0.054902}%
\pgfsetstrokecolor{currentstroke}%
\pgfsetdash{}{0pt}%
\pgfpathmoveto{\pgfqpoint{2.044848in}{3.193800in}}%
\pgfpathcurveto{\pgfqpoint{2.055898in}{3.193800in}}{\pgfqpoint{2.066497in}{3.198191in}}{\pgfqpoint{2.074311in}{3.206004in}}%
\pgfpathcurveto{\pgfqpoint{2.082125in}{3.213818in}}{\pgfqpoint{2.086515in}{3.224417in}}{\pgfqpoint{2.086515in}{3.235467in}}%
\pgfpathcurveto{\pgfqpoint{2.086515in}{3.246517in}}{\pgfqpoint{2.082125in}{3.257116in}}{\pgfqpoint{2.074311in}{3.264930in}}%
\pgfpathcurveto{\pgfqpoint{2.066497in}{3.272743in}}{\pgfqpoint{2.055898in}{3.277134in}}{\pgfqpoint{2.044848in}{3.277134in}}%
\pgfpathcurveto{\pgfqpoint{2.033798in}{3.277134in}}{\pgfqpoint{2.023199in}{3.272743in}}{\pgfqpoint{2.015385in}{3.264930in}}%
\pgfpathcurveto{\pgfqpoint{2.007572in}{3.257116in}}{\pgfqpoint{2.003181in}{3.246517in}}{\pgfqpoint{2.003181in}{3.235467in}}%
\pgfpathcurveto{\pgfqpoint{2.003181in}{3.224417in}}{\pgfqpoint{2.007572in}{3.213818in}}{\pgfqpoint{2.015385in}{3.206004in}}%
\pgfpathcurveto{\pgfqpoint{2.023199in}{3.198191in}}{\pgfqpoint{2.033798in}{3.193800in}}{\pgfqpoint{2.044848in}{3.193800in}}%
\pgfpathclose%
\pgfusepath{stroke,fill}%
\end{pgfscope}%
\begin{pgfscope}%
\pgfpathrectangle{\pgfqpoint{0.648703in}{0.548769in}}{\pgfqpoint{5.195150in}{3.102590in}}%
\pgfusepath{clip}%
\pgfsetbuttcap%
\pgfsetroundjoin%
\definecolor{currentfill}{rgb}{1.000000,0.498039,0.054902}%
\pgfsetfillcolor{currentfill}%
\pgfsetlinewidth{1.003750pt}%
\definecolor{currentstroke}{rgb}{1.000000,0.498039,0.054902}%
\pgfsetstrokecolor{currentstroke}%
\pgfsetdash{}{0pt}%
\pgfpathmoveto{\pgfqpoint{1.961991in}{3.231859in}}%
\pgfpathcurveto{\pgfqpoint{1.973041in}{3.231859in}}{\pgfqpoint{1.983640in}{3.236249in}}{\pgfqpoint{1.991454in}{3.244062in}}%
\pgfpathcurveto{\pgfqpoint{1.999267in}{3.251876in}}{\pgfqpoint{2.003658in}{3.262475in}}{\pgfqpoint{2.003658in}{3.273525in}}%
\pgfpathcurveto{\pgfqpoint{2.003658in}{3.284575in}}{\pgfqpoint{1.999267in}{3.295174in}}{\pgfqpoint{1.991454in}{3.302988in}}%
\pgfpathcurveto{\pgfqpoint{1.983640in}{3.310802in}}{\pgfqpoint{1.973041in}{3.315192in}}{\pgfqpoint{1.961991in}{3.315192in}}%
\pgfpathcurveto{\pgfqpoint{1.950941in}{3.315192in}}{\pgfqpoint{1.940342in}{3.310802in}}{\pgfqpoint{1.932528in}{3.302988in}}%
\pgfpathcurveto{\pgfqpoint{1.924714in}{3.295174in}}{\pgfqpoint{1.920324in}{3.284575in}}{\pgfqpoint{1.920324in}{3.273525in}}%
\pgfpathcurveto{\pgfqpoint{1.920324in}{3.262475in}}{\pgfqpoint{1.924714in}{3.251876in}}{\pgfqpoint{1.932528in}{3.244062in}}%
\pgfpathcurveto{\pgfqpoint{1.940342in}{3.236249in}}{\pgfqpoint{1.950941in}{3.231859in}}{\pgfqpoint{1.961991in}{3.231859in}}%
\pgfpathclose%
\pgfusepath{stroke,fill}%
\end{pgfscope}%
\begin{pgfscope}%
\pgfpathrectangle{\pgfqpoint{0.648703in}{0.548769in}}{\pgfqpoint{5.195150in}{3.102590in}}%
\pgfusepath{clip}%
\pgfsetbuttcap%
\pgfsetroundjoin%
\definecolor{currentfill}{rgb}{0.121569,0.466667,0.705882}%
\pgfsetfillcolor{currentfill}%
\pgfsetlinewidth{1.003750pt}%
\definecolor{currentstroke}{rgb}{0.121569,0.466667,0.705882}%
\pgfsetstrokecolor{currentstroke}%
\pgfsetdash{}{0pt}%
\pgfpathmoveto{\pgfqpoint{0.967704in}{0.681958in}}%
\pgfpathcurveto{\pgfqpoint{0.978754in}{0.681958in}}{\pgfqpoint{0.989353in}{0.686349in}}{\pgfqpoint{0.997167in}{0.694162in}}%
\pgfpathcurveto{\pgfqpoint{1.004980in}{0.701976in}}{\pgfqpoint{1.009370in}{0.712575in}}{\pgfqpoint{1.009370in}{0.723625in}}%
\pgfpathcurveto{\pgfqpoint{1.009370in}{0.734675in}}{\pgfqpoint{1.004980in}{0.745274in}}{\pgfqpoint{0.997167in}{0.753088in}}%
\pgfpathcurveto{\pgfqpoint{0.989353in}{0.760902in}}{\pgfqpoint{0.978754in}{0.765292in}}{\pgfqpoint{0.967704in}{0.765292in}}%
\pgfpathcurveto{\pgfqpoint{0.956654in}{0.765292in}}{\pgfqpoint{0.946055in}{0.760902in}}{\pgfqpoint{0.938241in}{0.753088in}}%
\pgfpathcurveto{\pgfqpoint{0.930427in}{0.745274in}}{\pgfqpoint{0.926037in}{0.734675in}}{\pgfqpoint{0.926037in}{0.723625in}}%
\pgfpathcurveto{\pgfqpoint{0.926037in}{0.712575in}}{\pgfqpoint{0.930427in}{0.701976in}}{\pgfqpoint{0.938241in}{0.694162in}}%
\pgfpathcurveto{\pgfqpoint{0.946055in}{0.686349in}}{\pgfqpoint{0.956654in}{0.681958in}}{\pgfqpoint{0.967704in}{0.681958in}}%
\pgfpathclose%
\pgfusepath{stroke,fill}%
\end{pgfscope}%
\begin{pgfscope}%
\pgfpathrectangle{\pgfqpoint{0.648703in}{0.548769in}}{\pgfqpoint{5.195150in}{3.102590in}}%
\pgfusepath{clip}%
\pgfsetbuttcap%
\pgfsetroundjoin%
\definecolor{currentfill}{rgb}{0.121569,0.466667,0.705882}%
\pgfsetfillcolor{currentfill}%
\pgfsetlinewidth{1.003750pt}%
\definecolor{currentstroke}{rgb}{0.121569,0.466667,0.705882}%
\pgfsetstrokecolor{currentstroke}%
\pgfsetdash{}{0pt}%
\pgfpathmoveto{\pgfqpoint{1.713419in}{0.758075in}}%
\pgfpathcurveto{\pgfqpoint{1.724469in}{0.758075in}}{\pgfqpoint{1.735068in}{0.762465in}}{\pgfqpoint{1.742882in}{0.770279in}}%
\pgfpathcurveto{\pgfqpoint{1.750695in}{0.778092in}}{\pgfqpoint{1.755086in}{0.788691in}}{\pgfqpoint{1.755086in}{0.799742in}}%
\pgfpathcurveto{\pgfqpoint{1.755086in}{0.810792in}}{\pgfqpoint{1.750695in}{0.821391in}}{\pgfqpoint{1.742882in}{0.829204in}}%
\pgfpathcurveto{\pgfqpoint{1.735068in}{0.837018in}}{\pgfqpoint{1.724469in}{0.841408in}}{\pgfqpoint{1.713419in}{0.841408in}}%
\pgfpathcurveto{\pgfqpoint{1.702369in}{0.841408in}}{\pgfqpoint{1.691770in}{0.837018in}}{\pgfqpoint{1.683956in}{0.829204in}}%
\pgfpathcurveto{\pgfqpoint{1.676143in}{0.821391in}}{\pgfqpoint{1.671752in}{0.810792in}}{\pgfqpoint{1.671752in}{0.799742in}}%
\pgfpathcurveto{\pgfqpoint{1.671752in}{0.788691in}}{\pgfqpoint{1.676143in}{0.778092in}}{\pgfqpoint{1.683956in}{0.770279in}}%
\pgfpathcurveto{\pgfqpoint{1.691770in}{0.762465in}}{\pgfqpoint{1.702369in}{0.758075in}}{\pgfqpoint{1.713419in}{0.758075in}}%
\pgfpathclose%
\pgfusepath{stroke,fill}%
\end{pgfscope}%
\begin{pgfscope}%
\pgfpathrectangle{\pgfqpoint{0.648703in}{0.548769in}}{\pgfqpoint{5.195150in}{3.102590in}}%
\pgfusepath{clip}%
\pgfsetbuttcap%
\pgfsetroundjoin%
\definecolor{currentfill}{rgb}{1.000000,0.498039,0.054902}%
\pgfsetfillcolor{currentfill}%
\pgfsetlinewidth{1.003750pt}%
\definecolor{currentstroke}{rgb}{1.000000,0.498039,0.054902}%
\pgfsetstrokecolor{currentstroke}%
\pgfsetdash{}{0pt}%
\pgfpathmoveto{\pgfqpoint{2.376277in}{3.185343in}}%
\pgfpathcurveto{\pgfqpoint{2.387327in}{3.185343in}}{\pgfqpoint{2.397926in}{3.189733in}}{\pgfqpoint{2.405740in}{3.197547in}}%
\pgfpathcurveto{\pgfqpoint{2.413554in}{3.205360in}}{\pgfqpoint{2.417944in}{3.215959in}}{\pgfqpoint{2.417944in}{3.227010in}}%
\pgfpathcurveto{\pgfqpoint{2.417944in}{3.238060in}}{\pgfqpoint{2.413554in}{3.248659in}}{\pgfqpoint{2.405740in}{3.256472in}}%
\pgfpathcurveto{\pgfqpoint{2.397926in}{3.264286in}}{\pgfqpoint{2.387327in}{3.268676in}}{\pgfqpoint{2.376277in}{3.268676in}}%
\pgfpathcurveto{\pgfqpoint{2.365227in}{3.268676in}}{\pgfqpoint{2.354628in}{3.264286in}}{\pgfqpoint{2.346814in}{3.256472in}}%
\pgfpathcurveto{\pgfqpoint{2.339001in}{3.248659in}}{\pgfqpoint{2.334610in}{3.238060in}}{\pgfqpoint{2.334610in}{3.227010in}}%
\pgfpathcurveto{\pgfqpoint{2.334610in}{3.215959in}}{\pgfqpoint{2.339001in}{3.205360in}}{\pgfqpoint{2.346814in}{3.197547in}}%
\pgfpathcurveto{\pgfqpoint{2.354628in}{3.189733in}}{\pgfqpoint{2.365227in}{3.185343in}}{\pgfqpoint{2.376277in}{3.185343in}}%
\pgfpathclose%
\pgfusepath{stroke,fill}%
\end{pgfscope}%
\begin{pgfscope}%
\pgfpathrectangle{\pgfqpoint{0.648703in}{0.548769in}}{\pgfqpoint{5.195150in}{3.102590in}}%
\pgfusepath{clip}%
\pgfsetbuttcap%
\pgfsetroundjoin%
\definecolor{currentfill}{rgb}{1.000000,0.498039,0.054902}%
\pgfsetfillcolor{currentfill}%
\pgfsetlinewidth{1.003750pt}%
\definecolor{currentstroke}{rgb}{1.000000,0.498039,0.054902}%
\pgfsetstrokecolor{currentstroke}%
\pgfsetdash{}{0pt}%
\pgfpathmoveto{\pgfqpoint{1.961991in}{3.202258in}}%
\pgfpathcurveto{\pgfqpoint{1.973041in}{3.202258in}}{\pgfqpoint{1.983640in}{3.206648in}}{\pgfqpoint{1.991454in}{3.214462in}}%
\pgfpathcurveto{\pgfqpoint{1.999267in}{3.222275in}}{\pgfqpoint{2.003658in}{3.232874in}}{\pgfqpoint{2.003658in}{3.243924in}}%
\pgfpathcurveto{\pgfqpoint{2.003658in}{3.254974in}}{\pgfqpoint{1.999267in}{3.265573in}}{\pgfqpoint{1.991454in}{3.273387in}}%
\pgfpathcurveto{\pgfqpoint{1.983640in}{3.281201in}}{\pgfqpoint{1.973041in}{3.285591in}}{\pgfqpoint{1.961991in}{3.285591in}}%
\pgfpathcurveto{\pgfqpoint{1.950941in}{3.285591in}}{\pgfqpoint{1.940342in}{3.281201in}}{\pgfqpoint{1.932528in}{3.273387in}}%
\pgfpathcurveto{\pgfqpoint{1.924714in}{3.265573in}}{\pgfqpoint{1.920324in}{3.254974in}}{\pgfqpoint{1.920324in}{3.243924in}}%
\pgfpathcurveto{\pgfqpoint{1.920324in}{3.232874in}}{\pgfqpoint{1.924714in}{3.222275in}}{\pgfqpoint{1.932528in}{3.214462in}}%
\pgfpathcurveto{\pgfqpoint{1.940342in}{3.206648in}}{\pgfqpoint{1.950941in}{3.202258in}}{\pgfqpoint{1.961991in}{3.202258in}}%
\pgfpathclose%
\pgfusepath{stroke,fill}%
\end{pgfscope}%
\begin{pgfscope}%
\pgfpathrectangle{\pgfqpoint{0.648703in}{0.548769in}}{\pgfqpoint{5.195150in}{3.102590in}}%
\pgfusepath{clip}%
\pgfsetbuttcap%
\pgfsetroundjoin%
\definecolor{currentfill}{rgb}{1.000000,0.498039,0.054902}%
\pgfsetfillcolor{currentfill}%
\pgfsetlinewidth{1.003750pt}%
\definecolor{currentstroke}{rgb}{1.000000,0.498039,0.054902}%
\pgfsetstrokecolor{currentstroke}%
\pgfsetdash{}{0pt}%
\pgfpathmoveto{\pgfqpoint{2.624849in}{3.189572in}}%
\pgfpathcurveto{\pgfqpoint{2.635899in}{3.189572in}}{\pgfqpoint{2.646498in}{3.193962in}}{\pgfqpoint{2.654312in}{3.201775in}}%
\pgfpathcurveto{\pgfqpoint{2.662125in}{3.209589in}}{\pgfqpoint{2.666516in}{3.220188in}}{\pgfqpoint{2.666516in}{3.231238in}}%
\pgfpathcurveto{\pgfqpoint{2.666516in}{3.242288in}}{\pgfqpoint{2.662125in}{3.252887in}}{\pgfqpoint{2.654312in}{3.260701in}}%
\pgfpathcurveto{\pgfqpoint{2.646498in}{3.268515in}}{\pgfqpoint{2.635899in}{3.272905in}}{\pgfqpoint{2.624849in}{3.272905in}}%
\pgfpathcurveto{\pgfqpoint{2.613799in}{3.272905in}}{\pgfqpoint{2.603200in}{3.268515in}}{\pgfqpoint{2.595386in}{3.260701in}}%
\pgfpathcurveto{\pgfqpoint{2.587572in}{3.252887in}}{\pgfqpoint{2.583182in}{3.242288in}}{\pgfqpoint{2.583182in}{3.231238in}}%
\pgfpathcurveto{\pgfqpoint{2.583182in}{3.220188in}}{\pgfqpoint{2.587572in}{3.209589in}}{\pgfqpoint{2.595386in}{3.201775in}}%
\pgfpathcurveto{\pgfqpoint{2.603200in}{3.193962in}}{\pgfqpoint{2.613799in}{3.189572in}}{\pgfqpoint{2.624849in}{3.189572in}}%
\pgfpathclose%
\pgfusepath{stroke,fill}%
\end{pgfscope}%
\begin{pgfscope}%
\pgfpathrectangle{\pgfqpoint{0.648703in}{0.548769in}}{\pgfqpoint{5.195150in}{3.102590in}}%
\pgfusepath{clip}%
\pgfsetbuttcap%
\pgfsetroundjoin%
\definecolor{currentfill}{rgb}{1.000000,0.498039,0.054902}%
\pgfsetfillcolor{currentfill}%
\pgfsetlinewidth{1.003750pt}%
\definecolor{currentstroke}{rgb}{1.000000,0.498039,0.054902}%
\pgfsetstrokecolor{currentstroke}%
\pgfsetdash{}{0pt}%
\pgfpathmoveto{\pgfqpoint{1.961991in}{3.206486in}}%
\pgfpathcurveto{\pgfqpoint{1.973041in}{3.206486in}}{\pgfqpoint{1.983640in}{3.210877in}}{\pgfqpoint{1.991454in}{3.218690in}}%
\pgfpathcurveto{\pgfqpoint{1.999267in}{3.226504in}}{\pgfqpoint{2.003658in}{3.237103in}}{\pgfqpoint{2.003658in}{3.248153in}}%
\pgfpathcurveto{\pgfqpoint{2.003658in}{3.259203in}}{\pgfqpoint{1.999267in}{3.269802in}}{\pgfqpoint{1.991454in}{3.277616in}}%
\pgfpathcurveto{\pgfqpoint{1.983640in}{3.285429in}}{\pgfqpoint{1.973041in}{3.289820in}}{\pgfqpoint{1.961991in}{3.289820in}}%
\pgfpathcurveto{\pgfqpoint{1.950941in}{3.289820in}}{\pgfqpoint{1.940342in}{3.285429in}}{\pgfqpoint{1.932528in}{3.277616in}}%
\pgfpathcurveto{\pgfqpoint{1.924714in}{3.269802in}}{\pgfqpoint{1.920324in}{3.259203in}}{\pgfqpoint{1.920324in}{3.248153in}}%
\pgfpathcurveto{\pgfqpoint{1.920324in}{3.237103in}}{\pgfqpoint{1.924714in}{3.226504in}}{\pgfqpoint{1.932528in}{3.218690in}}%
\pgfpathcurveto{\pgfqpoint{1.940342in}{3.210877in}}{\pgfqpoint{1.950941in}{3.206486in}}{\pgfqpoint{1.961991in}{3.206486in}}%
\pgfpathclose%
\pgfusepath{stroke,fill}%
\end{pgfscope}%
\begin{pgfscope}%
\pgfpathrectangle{\pgfqpoint{0.648703in}{0.548769in}}{\pgfqpoint{5.195150in}{3.102590in}}%
\pgfusepath{clip}%
\pgfsetbuttcap%
\pgfsetroundjoin%
\definecolor{currentfill}{rgb}{0.121569,0.466667,0.705882}%
\pgfsetfillcolor{currentfill}%
\pgfsetlinewidth{1.003750pt}%
\definecolor{currentstroke}{rgb}{0.121569,0.466667,0.705882}%
\pgfsetstrokecolor{currentstroke}%
\pgfsetdash{}{0pt}%
\pgfpathmoveto{\pgfqpoint{0.967704in}{0.648129in}}%
\pgfpathcurveto{\pgfqpoint{0.978754in}{0.648129in}}{\pgfqpoint{0.989353in}{0.652519in}}{\pgfqpoint{0.997167in}{0.660333in}}%
\pgfpathcurveto{\pgfqpoint{1.004980in}{0.668146in}}{\pgfqpoint{1.009370in}{0.678745in}}{\pgfqpoint{1.009370in}{0.689796in}}%
\pgfpathcurveto{\pgfqpoint{1.009370in}{0.700846in}}{\pgfqpoint{1.004980in}{0.711445in}}{\pgfqpoint{0.997167in}{0.719258in}}%
\pgfpathcurveto{\pgfqpoint{0.989353in}{0.727072in}}{\pgfqpoint{0.978754in}{0.731462in}}{\pgfqpoint{0.967704in}{0.731462in}}%
\pgfpathcurveto{\pgfqpoint{0.956654in}{0.731462in}}{\pgfqpoint{0.946055in}{0.727072in}}{\pgfqpoint{0.938241in}{0.719258in}}%
\pgfpathcurveto{\pgfqpoint{0.930427in}{0.711445in}}{\pgfqpoint{0.926037in}{0.700846in}}{\pgfqpoint{0.926037in}{0.689796in}}%
\pgfpathcurveto{\pgfqpoint{0.926037in}{0.678745in}}{\pgfqpoint{0.930427in}{0.668146in}}{\pgfqpoint{0.938241in}{0.660333in}}%
\pgfpathcurveto{\pgfqpoint{0.946055in}{0.652519in}}{\pgfqpoint{0.956654in}{0.648129in}}{\pgfqpoint{0.967704in}{0.648129in}}%
\pgfpathclose%
\pgfusepath{stroke,fill}%
\end{pgfscope}%
\begin{pgfscope}%
\pgfpathrectangle{\pgfqpoint{0.648703in}{0.548769in}}{\pgfqpoint{5.195150in}{3.102590in}}%
\pgfusepath{clip}%
\pgfsetbuttcap%
\pgfsetroundjoin%
\definecolor{currentfill}{rgb}{0.121569,0.466667,0.705882}%
\pgfsetfillcolor{currentfill}%
\pgfsetlinewidth{1.003750pt}%
\definecolor{currentstroke}{rgb}{0.121569,0.466667,0.705882}%
\pgfsetstrokecolor{currentstroke}%
\pgfsetdash{}{0pt}%
\pgfpathmoveto{\pgfqpoint{1.547705in}{1.692615in}}%
\pgfpathcurveto{\pgfqpoint{1.558755in}{1.692615in}}{\pgfqpoint{1.569354in}{1.697006in}}{\pgfqpoint{1.577167in}{1.704819in}}%
\pgfpathcurveto{\pgfqpoint{1.584981in}{1.712633in}}{\pgfqpoint{1.589371in}{1.723232in}}{\pgfqpoint{1.589371in}{1.734282in}}%
\pgfpathcurveto{\pgfqpoint{1.589371in}{1.745332in}}{\pgfqpoint{1.584981in}{1.755931in}}{\pgfqpoint{1.577167in}{1.763745in}}%
\pgfpathcurveto{\pgfqpoint{1.569354in}{1.771558in}}{\pgfqpoint{1.558755in}{1.775949in}}{\pgfqpoint{1.547705in}{1.775949in}}%
\pgfpathcurveto{\pgfqpoint{1.536654in}{1.775949in}}{\pgfqpoint{1.526055in}{1.771558in}}{\pgfqpoint{1.518242in}{1.763745in}}%
\pgfpathcurveto{\pgfqpoint{1.510428in}{1.755931in}}{\pgfqpoint{1.506038in}{1.745332in}}{\pgfqpoint{1.506038in}{1.734282in}}%
\pgfpathcurveto{\pgfqpoint{1.506038in}{1.723232in}}{\pgfqpoint{1.510428in}{1.712633in}}{\pgfqpoint{1.518242in}{1.704819in}}%
\pgfpathcurveto{\pgfqpoint{1.526055in}{1.697006in}}{\pgfqpoint{1.536654in}{1.692615in}}{\pgfqpoint{1.547705in}{1.692615in}}%
\pgfpathclose%
\pgfusepath{stroke,fill}%
\end{pgfscope}%
\begin{pgfscope}%
\pgfpathrectangle{\pgfqpoint{0.648703in}{0.548769in}}{\pgfqpoint{5.195150in}{3.102590in}}%
\pgfusepath{clip}%
\pgfsetbuttcap%
\pgfsetroundjoin%
\definecolor{currentfill}{rgb}{0.121569,0.466667,0.705882}%
\pgfsetfillcolor{currentfill}%
\pgfsetlinewidth{1.003750pt}%
\definecolor{currentstroke}{rgb}{0.121569,0.466667,0.705882}%
\pgfsetstrokecolor{currentstroke}%
\pgfsetdash{}{0pt}%
\pgfpathmoveto{\pgfqpoint{1.796276in}{0.648129in}}%
\pgfpathcurveto{\pgfqpoint{1.807326in}{0.648129in}}{\pgfqpoint{1.817926in}{0.652519in}}{\pgfqpoint{1.825739in}{0.660333in}}%
\pgfpathcurveto{\pgfqpoint{1.833553in}{0.668146in}}{\pgfqpoint{1.837943in}{0.678745in}}{\pgfqpoint{1.837943in}{0.689796in}}%
\pgfpathcurveto{\pgfqpoint{1.837943in}{0.700846in}}{\pgfqpoint{1.833553in}{0.711445in}}{\pgfqpoint{1.825739in}{0.719258in}}%
\pgfpathcurveto{\pgfqpoint{1.817926in}{0.727072in}}{\pgfqpoint{1.807326in}{0.731462in}}{\pgfqpoint{1.796276in}{0.731462in}}%
\pgfpathcurveto{\pgfqpoint{1.785226in}{0.731462in}}{\pgfqpoint{1.774627in}{0.727072in}}{\pgfqpoint{1.766814in}{0.719258in}}%
\pgfpathcurveto{\pgfqpoint{1.759000in}{0.711445in}}{\pgfqpoint{1.754610in}{0.700846in}}{\pgfqpoint{1.754610in}{0.689796in}}%
\pgfpathcurveto{\pgfqpoint{1.754610in}{0.678745in}}{\pgfqpoint{1.759000in}{0.668146in}}{\pgfqpoint{1.766814in}{0.660333in}}%
\pgfpathcurveto{\pgfqpoint{1.774627in}{0.652519in}}{\pgfqpoint{1.785226in}{0.648129in}}{\pgfqpoint{1.796276in}{0.648129in}}%
\pgfpathclose%
\pgfusepath{stroke,fill}%
\end{pgfscope}%
\begin{pgfscope}%
\pgfpathrectangle{\pgfqpoint{0.648703in}{0.548769in}}{\pgfqpoint{5.195150in}{3.102590in}}%
\pgfusepath{clip}%
\pgfsetbuttcap%
\pgfsetroundjoin%
\definecolor{currentfill}{rgb}{1.000000,0.498039,0.054902}%
\pgfsetfillcolor{currentfill}%
\pgfsetlinewidth{1.003750pt}%
\definecolor{currentstroke}{rgb}{1.000000,0.498039,0.054902}%
\pgfsetstrokecolor{currentstroke}%
\pgfsetdash{}{0pt}%
\pgfpathmoveto{\pgfqpoint{2.624849in}{3.189572in}}%
\pgfpathcurveto{\pgfqpoint{2.635899in}{3.189572in}}{\pgfqpoint{2.646498in}{3.193962in}}{\pgfqpoint{2.654312in}{3.201775in}}%
\pgfpathcurveto{\pgfqpoint{2.662125in}{3.209589in}}{\pgfqpoint{2.666516in}{3.220188in}}{\pgfqpoint{2.666516in}{3.231238in}}%
\pgfpathcurveto{\pgfqpoint{2.666516in}{3.242288in}}{\pgfqpoint{2.662125in}{3.252887in}}{\pgfqpoint{2.654312in}{3.260701in}}%
\pgfpathcurveto{\pgfqpoint{2.646498in}{3.268515in}}{\pgfqpoint{2.635899in}{3.272905in}}{\pgfqpoint{2.624849in}{3.272905in}}%
\pgfpathcurveto{\pgfqpoint{2.613799in}{3.272905in}}{\pgfqpoint{2.603200in}{3.268515in}}{\pgfqpoint{2.595386in}{3.260701in}}%
\pgfpathcurveto{\pgfqpoint{2.587572in}{3.252887in}}{\pgfqpoint{2.583182in}{3.242288in}}{\pgfqpoint{2.583182in}{3.231238in}}%
\pgfpathcurveto{\pgfqpoint{2.583182in}{3.220188in}}{\pgfqpoint{2.587572in}{3.209589in}}{\pgfqpoint{2.595386in}{3.201775in}}%
\pgfpathcurveto{\pgfqpoint{2.603200in}{3.193962in}}{\pgfqpoint{2.613799in}{3.189572in}}{\pgfqpoint{2.624849in}{3.189572in}}%
\pgfpathclose%
\pgfusepath{stroke,fill}%
\end{pgfscope}%
\begin{pgfscope}%
\pgfpathrectangle{\pgfqpoint{0.648703in}{0.548769in}}{\pgfqpoint{5.195150in}{3.102590in}}%
\pgfusepath{clip}%
\pgfsetbuttcap%
\pgfsetroundjoin%
\definecolor{currentfill}{rgb}{0.121569,0.466667,0.705882}%
\pgfsetfillcolor{currentfill}%
\pgfsetlinewidth{1.003750pt}%
\definecolor{currentstroke}{rgb}{0.121569,0.466667,0.705882}%
\pgfsetstrokecolor{currentstroke}%
\pgfsetdash{}{0pt}%
\pgfpathmoveto{\pgfqpoint{3.536279in}{3.181114in}}%
\pgfpathcurveto{\pgfqpoint{3.547329in}{3.181114in}}{\pgfqpoint{3.557928in}{3.185504in}}{\pgfqpoint{3.565741in}{3.193318in}}%
\pgfpathcurveto{\pgfqpoint{3.573555in}{3.201132in}}{\pgfqpoint{3.577945in}{3.211731in}}{\pgfqpoint{3.577945in}{3.222781in}}%
\pgfpathcurveto{\pgfqpoint{3.577945in}{3.233831in}}{\pgfqpoint{3.573555in}{3.244430in}}{\pgfqpoint{3.565741in}{3.252244in}}%
\pgfpathcurveto{\pgfqpoint{3.557928in}{3.260057in}}{\pgfqpoint{3.547329in}{3.264448in}}{\pgfqpoint{3.536279in}{3.264448in}}%
\pgfpathcurveto{\pgfqpoint{3.525228in}{3.264448in}}{\pgfqpoint{3.514629in}{3.260057in}}{\pgfqpoint{3.506816in}{3.252244in}}%
\pgfpathcurveto{\pgfqpoint{3.499002in}{3.244430in}}{\pgfqpoint{3.494612in}{3.233831in}}{\pgfqpoint{3.494612in}{3.222781in}}%
\pgfpathcurveto{\pgfqpoint{3.494612in}{3.211731in}}{\pgfqpoint{3.499002in}{3.201132in}}{\pgfqpoint{3.506816in}{3.193318in}}%
\pgfpathcurveto{\pgfqpoint{3.514629in}{3.185504in}}{\pgfqpoint{3.525228in}{3.181114in}}{\pgfqpoint{3.536279in}{3.181114in}}%
\pgfpathclose%
\pgfusepath{stroke,fill}%
\end{pgfscope}%
\begin{pgfscope}%
\pgfpathrectangle{\pgfqpoint{0.648703in}{0.548769in}}{\pgfqpoint{5.195150in}{3.102590in}}%
\pgfusepath{clip}%
\pgfsetbuttcap%
\pgfsetroundjoin%
\definecolor{currentfill}{rgb}{1.000000,0.498039,0.054902}%
\pgfsetfillcolor{currentfill}%
\pgfsetlinewidth{1.003750pt}%
\definecolor{currentstroke}{rgb}{1.000000,0.498039,0.054902}%
\pgfsetstrokecolor{currentstroke}%
\pgfsetdash{}{0pt}%
\pgfpathmoveto{\pgfqpoint{4.530566in}{3.189572in}}%
\pgfpathcurveto{\pgfqpoint{4.541616in}{3.189572in}}{\pgfqpoint{4.552215in}{3.193962in}}{\pgfqpoint{4.560028in}{3.201775in}}%
\pgfpathcurveto{\pgfqpoint{4.567842in}{3.209589in}}{\pgfqpoint{4.572232in}{3.220188in}}{\pgfqpoint{4.572232in}{3.231238in}}%
\pgfpathcurveto{\pgfqpoint{4.572232in}{3.242288in}}{\pgfqpoint{4.567842in}{3.252887in}}{\pgfqpoint{4.560028in}{3.260701in}}%
\pgfpathcurveto{\pgfqpoint{4.552215in}{3.268515in}}{\pgfqpoint{4.541616in}{3.272905in}}{\pgfqpoint{4.530566in}{3.272905in}}%
\pgfpathcurveto{\pgfqpoint{4.519516in}{3.272905in}}{\pgfqpoint{4.508916in}{3.268515in}}{\pgfqpoint{4.501103in}{3.260701in}}%
\pgfpathcurveto{\pgfqpoint{4.493289in}{3.252887in}}{\pgfqpoint{4.488899in}{3.242288in}}{\pgfqpoint{4.488899in}{3.231238in}}%
\pgfpathcurveto{\pgfqpoint{4.488899in}{3.220188in}}{\pgfqpoint{4.493289in}{3.209589in}}{\pgfqpoint{4.501103in}{3.201775in}}%
\pgfpathcurveto{\pgfqpoint{4.508916in}{3.193962in}}{\pgfqpoint{4.519516in}{3.189572in}}{\pgfqpoint{4.530566in}{3.189572in}}%
\pgfpathclose%
\pgfusepath{stroke,fill}%
\end{pgfscope}%
\begin{pgfscope}%
\pgfpathrectangle{\pgfqpoint{0.648703in}{0.548769in}}{\pgfqpoint{5.195150in}{3.102590in}}%
\pgfusepath{clip}%
\pgfsetbuttcap%
\pgfsetroundjoin%
\definecolor{currentfill}{rgb}{1.000000,0.498039,0.054902}%
\pgfsetfillcolor{currentfill}%
\pgfsetlinewidth{1.003750pt}%
\definecolor{currentstroke}{rgb}{1.000000,0.498039,0.054902}%
\pgfsetstrokecolor{currentstroke}%
\pgfsetdash{}{0pt}%
\pgfpathmoveto{\pgfqpoint{2.293420in}{3.214944in}}%
\pgfpathcurveto{\pgfqpoint{2.304470in}{3.214944in}}{\pgfqpoint{2.315069in}{3.219334in}}{\pgfqpoint{2.322883in}{3.227148in}}%
\pgfpathcurveto{\pgfqpoint{2.330696in}{3.234961in}}{\pgfqpoint{2.335087in}{3.245560in}}{\pgfqpoint{2.335087in}{3.256610in}}%
\pgfpathcurveto{\pgfqpoint{2.335087in}{3.267661in}}{\pgfqpoint{2.330696in}{3.278260in}}{\pgfqpoint{2.322883in}{3.286073in}}%
\pgfpathcurveto{\pgfqpoint{2.315069in}{3.293887in}}{\pgfqpoint{2.304470in}{3.298277in}}{\pgfqpoint{2.293420in}{3.298277in}}%
\pgfpathcurveto{\pgfqpoint{2.282370in}{3.298277in}}{\pgfqpoint{2.271771in}{3.293887in}}{\pgfqpoint{2.263957in}{3.286073in}}%
\pgfpathcurveto{\pgfqpoint{2.256143in}{3.278260in}}{\pgfqpoint{2.251753in}{3.267661in}}{\pgfqpoint{2.251753in}{3.256610in}}%
\pgfpathcurveto{\pgfqpoint{2.251753in}{3.245560in}}{\pgfqpoint{2.256143in}{3.234961in}}{\pgfqpoint{2.263957in}{3.227148in}}%
\pgfpathcurveto{\pgfqpoint{2.271771in}{3.219334in}}{\pgfqpoint{2.282370in}{3.214944in}}{\pgfqpoint{2.293420in}{3.214944in}}%
\pgfpathclose%
\pgfusepath{stroke,fill}%
\end{pgfscope}%
\begin{pgfscope}%
\pgfpathrectangle{\pgfqpoint{0.648703in}{0.548769in}}{\pgfqpoint{5.195150in}{3.102590in}}%
\pgfusepath{clip}%
\pgfsetbuttcap%
\pgfsetroundjoin%
\definecolor{currentfill}{rgb}{1.000000,0.498039,0.054902}%
\pgfsetfillcolor{currentfill}%
\pgfsetlinewidth{1.003750pt}%
\definecolor{currentstroke}{rgb}{1.000000,0.498039,0.054902}%
\pgfsetstrokecolor{currentstroke}%
\pgfsetdash{}{0pt}%
\pgfpathmoveto{\pgfqpoint{1.879134in}{3.185343in}}%
\pgfpathcurveto{\pgfqpoint{1.890184in}{3.185343in}}{\pgfqpoint{1.900783in}{3.189733in}}{\pgfqpoint{1.908596in}{3.197547in}}%
\pgfpathcurveto{\pgfqpoint{1.916410in}{3.205360in}}{\pgfqpoint{1.920800in}{3.215959in}}{\pgfqpoint{1.920800in}{3.227010in}}%
\pgfpathcurveto{\pgfqpoint{1.920800in}{3.238060in}}{\pgfqpoint{1.916410in}{3.248659in}}{\pgfqpoint{1.908596in}{3.256472in}}%
\pgfpathcurveto{\pgfqpoint{1.900783in}{3.264286in}}{\pgfqpoint{1.890184in}{3.268676in}}{\pgfqpoint{1.879134in}{3.268676in}}%
\pgfpathcurveto{\pgfqpoint{1.868083in}{3.268676in}}{\pgfqpoint{1.857484in}{3.264286in}}{\pgfqpoint{1.849671in}{3.256472in}}%
\pgfpathcurveto{\pgfqpoint{1.841857in}{3.248659in}}{\pgfqpoint{1.837467in}{3.238060in}}{\pgfqpoint{1.837467in}{3.227010in}}%
\pgfpathcurveto{\pgfqpoint{1.837467in}{3.215959in}}{\pgfqpoint{1.841857in}{3.205360in}}{\pgfqpoint{1.849671in}{3.197547in}}%
\pgfpathcurveto{\pgfqpoint{1.857484in}{3.189733in}}{\pgfqpoint{1.868083in}{3.185343in}}{\pgfqpoint{1.879134in}{3.185343in}}%
\pgfpathclose%
\pgfusepath{stroke,fill}%
\end{pgfscope}%
\begin{pgfscope}%
\pgfpathrectangle{\pgfqpoint{0.648703in}{0.548769in}}{\pgfqpoint{5.195150in}{3.102590in}}%
\pgfusepath{clip}%
\pgfsetbuttcap%
\pgfsetroundjoin%
\definecolor{currentfill}{rgb}{1.000000,0.498039,0.054902}%
\pgfsetfillcolor{currentfill}%
\pgfsetlinewidth{1.003750pt}%
\definecolor{currentstroke}{rgb}{1.000000,0.498039,0.054902}%
\pgfsetstrokecolor{currentstroke}%
\pgfsetdash{}{0pt}%
\pgfpathmoveto{\pgfqpoint{2.956278in}{3.198029in}}%
\pgfpathcurveto{\pgfqpoint{2.967328in}{3.198029in}}{\pgfqpoint{2.977927in}{3.202419in}}{\pgfqpoint{2.985741in}{3.210233in}}%
\pgfpathcurveto{\pgfqpoint{2.993554in}{3.218046in}}{\pgfqpoint{2.997945in}{3.228646in}}{\pgfqpoint{2.997945in}{3.239696in}}%
\pgfpathcurveto{\pgfqpoint{2.997945in}{3.250746in}}{\pgfqpoint{2.993554in}{3.261345in}}{\pgfqpoint{2.985741in}{3.269158in}}%
\pgfpathcurveto{\pgfqpoint{2.977927in}{3.276972in}}{\pgfqpoint{2.967328in}{3.281362in}}{\pgfqpoint{2.956278in}{3.281362in}}%
\pgfpathcurveto{\pgfqpoint{2.945228in}{3.281362in}}{\pgfqpoint{2.934629in}{3.276972in}}{\pgfqpoint{2.926815in}{3.269158in}}%
\pgfpathcurveto{\pgfqpoint{2.919001in}{3.261345in}}{\pgfqpoint{2.914611in}{3.250746in}}{\pgfqpoint{2.914611in}{3.239696in}}%
\pgfpathcurveto{\pgfqpoint{2.914611in}{3.228646in}}{\pgfqpoint{2.919001in}{3.218046in}}{\pgfqpoint{2.926815in}{3.210233in}}%
\pgfpathcurveto{\pgfqpoint{2.934629in}{3.202419in}}{\pgfqpoint{2.945228in}{3.198029in}}{\pgfqpoint{2.956278in}{3.198029in}}%
\pgfpathclose%
\pgfusepath{stroke,fill}%
\end{pgfscope}%
\begin{pgfscope}%
\pgfpathrectangle{\pgfqpoint{0.648703in}{0.548769in}}{\pgfqpoint{5.195150in}{3.102590in}}%
\pgfusepath{clip}%
\pgfsetbuttcap%
\pgfsetroundjoin%
\definecolor{currentfill}{rgb}{0.839216,0.152941,0.156863}%
\pgfsetfillcolor{currentfill}%
\pgfsetlinewidth{1.003750pt}%
\definecolor{currentstroke}{rgb}{0.839216,0.152941,0.156863}%
\pgfsetstrokecolor{currentstroke}%
\pgfsetdash{}{0pt}%
\pgfpathmoveto{\pgfqpoint{1.630562in}{3.193800in}}%
\pgfpathcurveto{\pgfqpoint{1.641612in}{3.193800in}}{\pgfqpoint{1.652211in}{3.198191in}}{\pgfqpoint{1.660025in}{3.206004in}}%
\pgfpathcurveto{\pgfqpoint{1.667838in}{3.213818in}}{\pgfqpoint{1.672229in}{3.224417in}}{\pgfqpoint{1.672229in}{3.235467in}}%
\pgfpathcurveto{\pgfqpoint{1.672229in}{3.246517in}}{\pgfqpoint{1.667838in}{3.257116in}}{\pgfqpoint{1.660025in}{3.264930in}}%
\pgfpathcurveto{\pgfqpoint{1.652211in}{3.272743in}}{\pgfqpoint{1.641612in}{3.277134in}}{\pgfqpoint{1.630562in}{3.277134in}}%
\pgfpathcurveto{\pgfqpoint{1.619512in}{3.277134in}}{\pgfqpoint{1.608913in}{3.272743in}}{\pgfqpoint{1.601099in}{3.264930in}}%
\pgfpathcurveto{\pgfqpoint{1.593285in}{3.257116in}}{\pgfqpoint{1.588895in}{3.246517in}}{\pgfqpoint{1.588895in}{3.235467in}}%
\pgfpathcurveto{\pgfqpoint{1.588895in}{3.224417in}}{\pgfqpoint{1.593285in}{3.213818in}}{\pgfqpoint{1.601099in}{3.206004in}}%
\pgfpathcurveto{\pgfqpoint{1.608913in}{3.198191in}}{\pgfqpoint{1.619512in}{3.193800in}}{\pgfqpoint{1.630562in}{3.193800in}}%
\pgfpathclose%
\pgfusepath{stroke,fill}%
\end{pgfscope}%
\begin{pgfscope}%
\pgfpathrectangle{\pgfqpoint{0.648703in}{0.548769in}}{\pgfqpoint{5.195150in}{3.102590in}}%
\pgfusepath{clip}%
\pgfsetbuttcap%
\pgfsetroundjoin%
\definecolor{currentfill}{rgb}{0.121569,0.466667,0.705882}%
\pgfsetfillcolor{currentfill}%
\pgfsetlinewidth{1.003750pt}%
\definecolor{currentstroke}{rgb}{0.121569,0.466667,0.705882}%
\pgfsetstrokecolor{currentstroke}%
\pgfsetdash{}{0pt}%
\pgfpathmoveto{\pgfqpoint{1.216276in}{0.758075in}}%
\pgfpathcurveto{\pgfqpoint{1.227326in}{0.758075in}}{\pgfqpoint{1.237925in}{0.762465in}}{\pgfqpoint{1.245738in}{0.770279in}}%
\pgfpathcurveto{\pgfqpoint{1.253552in}{0.778092in}}{\pgfqpoint{1.257942in}{0.788691in}}{\pgfqpoint{1.257942in}{0.799742in}}%
\pgfpathcurveto{\pgfqpoint{1.257942in}{0.810792in}}{\pgfqpoint{1.253552in}{0.821391in}}{\pgfqpoint{1.245738in}{0.829204in}}%
\pgfpathcurveto{\pgfqpoint{1.237925in}{0.837018in}}{\pgfqpoint{1.227326in}{0.841408in}}{\pgfqpoint{1.216276in}{0.841408in}}%
\pgfpathcurveto{\pgfqpoint{1.205225in}{0.841408in}}{\pgfqpoint{1.194626in}{0.837018in}}{\pgfqpoint{1.186813in}{0.829204in}}%
\pgfpathcurveto{\pgfqpoint{1.178999in}{0.821391in}}{\pgfqpoint{1.174609in}{0.810792in}}{\pgfqpoint{1.174609in}{0.799742in}}%
\pgfpathcurveto{\pgfqpoint{1.174609in}{0.788691in}}{\pgfqpoint{1.178999in}{0.778092in}}{\pgfqpoint{1.186813in}{0.770279in}}%
\pgfpathcurveto{\pgfqpoint{1.194626in}{0.762465in}}{\pgfqpoint{1.205225in}{0.758075in}}{\pgfqpoint{1.216276in}{0.758075in}}%
\pgfpathclose%
\pgfusepath{stroke,fill}%
\end{pgfscope}%
\begin{pgfscope}%
\pgfpathrectangle{\pgfqpoint{0.648703in}{0.548769in}}{\pgfqpoint{5.195150in}{3.102590in}}%
\pgfusepath{clip}%
\pgfsetbuttcap%
\pgfsetroundjoin%
\definecolor{currentfill}{rgb}{1.000000,0.498039,0.054902}%
\pgfsetfillcolor{currentfill}%
\pgfsetlinewidth{1.003750pt}%
\definecolor{currentstroke}{rgb}{1.000000,0.498039,0.054902}%
\pgfsetstrokecolor{currentstroke}%
\pgfsetdash{}{0pt}%
\pgfpathmoveto{\pgfqpoint{1.796276in}{3.193800in}}%
\pgfpathcurveto{\pgfqpoint{1.807326in}{3.193800in}}{\pgfqpoint{1.817926in}{3.198191in}}{\pgfqpoint{1.825739in}{3.206004in}}%
\pgfpathcurveto{\pgfqpoint{1.833553in}{3.213818in}}{\pgfqpoint{1.837943in}{3.224417in}}{\pgfqpoint{1.837943in}{3.235467in}}%
\pgfpathcurveto{\pgfqpoint{1.837943in}{3.246517in}}{\pgfqpoint{1.833553in}{3.257116in}}{\pgfqpoint{1.825739in}{3.264930in}}%
\pgfpathcurveto{\pgfqpoint{1.817926in}{3.272743in}}{\pgfqpoint{1.807326in}{3.277134in}}{\pgfqpoint{1.796276in}{3.277134in}}%
\pgfpathcurveto{\pgfqpoint{1.785226in}{3.277134in}}{\pgfqpoint{1.774627in}{3.272743in}}{\pgfqpoint{1.766814in}{3.264930in}}%
\pgfpathcurveto{\pgfqpoint{1.759000in}{3.257116in}}{\pgfqpoint{1.754610in}{3.246517in}}{\pgfqpoint{1.754610in}{3.235467in}}%
\pgfpathcurveto{\pgfqpoint{1.754610in}{3.224417in}}{\pgfqpoint{1.759000in}{3.213818in}}{\pgfqpoint{1.766814in}{3.206004in}}%
\pgfpathcurveto{\pgfqpoint{1.774627in}{3.198191in}}{\pgfqpoint{1.785226in}{3.193800in}}{\pgfqpoint{1.796276in}{3.193800in}}%
\pgfpathclose%
\pgfusepath{stroke,fill}%
\end{pgfscope}%
\begin{pgfscope}%
\pgfpathrectangle{\pgfqpoint{0.648703in}{0.548769in}}{\pgfqpoint{5.195150in}{3.102590in}}%
\pgfusepath{clip}%
\pgfsetbuttcap%
\pgfsetroundjoin%
\definecolor{currentfill}{rgb}{0.121569,0.466667,0.705882}%
\pgfsetfillcolor{currentfill}%
\pgfsetlinewidth{1.003750pt}%
\definecolor{currentstroke}{rgb}{0.121569,0.466667,0.705882}%
\pgfsetstrokecolor{currentstroke}%
\pgfsetdash{}{0pt}%
\pgfpathmoveto{\pgfqpoint{1.216276in}{0.648129in}}%
\pgfpathcurveto{\pgfqpoint{1.227326in}{0.648129in}}{\pgfqpoint{1.237925in}{0.652519in}}{\pgfqpoint{1.245738in}{0.660333in}}%
\pgfpathcurveto{\pgfqpoint{1.253552in}{0.668146in}}{\pgfqpoint{1.257942in}{0.678745in}}{\pgfqpoint{1.257942in}{0.689796in}}%
\pgfpathcurveto{\pgfqpoint{1.257942in}{0.700846in}}{\pgfqpoint{1.253552in}{0.711445in}}{\pgfqpoint{1.245738in}{0.719258in}}%
\pgfpathcurveto{\pgfqpoint{1.237925in}{0.727072in}}{\pgfqpoint{1.227326in}{0.731462in}}{\pgfqpoint{1.216276in}{0.731462in}}%
\pgfpathcurveto{\pgfqpoint{1.205225in}{0.731462in}}{\pgfqpoint{1.194626in}{0.727072in}}{\pgfqpoint{1.186813in}{0.719258in}}%
\pgfpathcurveto{\pgfqpoint{1.178999in}{0.711445in}}{\pgfqpoint{1.174609in}{0.700846in}}{\pgfqpoint{1.174609in}{0.689796in}}%
\pgfpathcurveto{\pgfqpoint{1.174609in}{0.678745in}}{\pgfqpoint{1.178999in}{0.668146in}}{\pgfqpoint{1.186813in}{0.660333in}}%
\pgfpathcurveto{\pgfqpoint{1.194626in}{0.652519in}}{\pgfqpoint{1.205225in}{0.648129in}}{\pgfqpoint{1.216276in}{0.648129in}}%
\pgfpathclose%
\pgfusepath{stroke,fill}%
\end{pgfscope}%
\begin{pgfscope}%
\pgfpathrectangle{\pgfqpoint{0.648703in}{0.548769in}}{\pgfqpoint{5.195150in}{3.102590in}}%
\pgfusepath{clip}%
\pgfsetbuttcap%
\pgfsetroundjoin%
\definecolor{currentfill}{rgb}{1.000000,0.498039,0.054902}%
\pgfsetfillcolor{currentfill}%
\pgfsetlinewidth{1.003750pt}%
\definecolor{currentstroke}{rgb}{1.000000,0.498039,0.054902}%
\pgfsetstrokecolor{currentstroke}%
\pgfsetdash{}{0pt}%
\pgfpathmoveto{\pgfqpoint{3.039135in}{3.193800in}}%
\pgfpathcurveto{\pgfqpoint{3.050185in}{3.193800in}}{\pgfqpoint{3.060784in}{3.198191in}}{\pgfqpoint{3.068598in}{3.206004in}}%
\pgfpathcurveto{\pgfqpoint{3.076412in}{3.213818in}}{\pgfqpoint{3.080802in}{3.224417in}}{\pgfqpoint{3.080802in}{3.235467in}}%
\pgfpathcurveto{\pgfqpoint{3.080802in}{3.246517in}}{\pgfqpoint{3.076412in}{3.257116in}}{\pgfqpoint{3.068598in}{3.264930in}}%
\pgfpathcurveto{\pgfqpoint{3.060784in}{3.272743in}}{\pgfqpoint{3.050185in}{3.277134in}}{\pgfqpoint{3.039135in}{3.277134in}}%
\pgfpathcurveto{\pgfqpoint{3.028085in}{3.277134in}}{\pgfqpoint{3.017486in}{3.272743in}}{\pgfqpoint{3.009672in}{3.264930in}}%
\pgfpathcurveto{\pgfqpoint{3.001859in}{3.257116in}}{\pgfqpoint{2.997468in}{3.246517in}}{\pgfqpoint{2.997468in}{3.235467in}}%
\pgfpathcurveto{\pgfqpoint{2.997468in}{3.224417in}}{\pgfqpoint{3.001859in}{3.213818in}}{\pgfqpoint{3.009672in}{3.206004in}}%
\pgfpathcurveto{\pgfqpoint{3.017486in}{3.198191in}}{\pgfqpoint{3.028085in}{3.193800in}}{\pgfqpoint{3.039135in}{3.193800in}}%
\pgfpathclose%
\pgfusepath{stroke,fill}%
\end{pgfscope}%
\begin{pgfscope}%
\pgfpathrectangle{\pgfqpoint{0.648703in}{0.548769in}}{\pgfqpoint{5.195150in}{3.102590in}}%
\pgfusepath{clip}%
\pgfsetbuttcap%
\pgfsetroundjoin%
\definecolor{currentfill}{rgb}{1.000000,0.498039,0.054902}%
\pgfsetfillcolor{currentfill}%
\pgfsetlinewidth{1.003750pt}%
\definecolor{currentstroke}{rgb}{1.000000,0.498039,0.054902}%
\pgfsetstrokecolor{currentstroke}%
\pgfsetdash{}{0pt}%
\pgfpathmoveto{\pgfqpoint{3.204850in}{3.185343in}}%
\pgfpathcurveto{\pgfqpoint{3.215900in}{3.185343in}}{\pgfqpoint{3.226499in}{3.189733in}}{\pgfqpoint{3.234312in}{3.197547in}}%
\pgfpathcurveto{\pgfqpoint{3.242126in}{3.205360in}}{\pgfqpoint{3.246516in}{3.215959in}}{\pgfqpoint{3.246516in}{3.227010in}}%
\pgfpathcurveto{\pgfqpoint{3.246516in}{3.238060in}}{\pgfqpoint{3.242126in}{3.248659in}}{\pgfqpoint{3.234312in}{3.256472in}}%
\pgfpathcurveto{\pgfqpoint{3.226499in}{3.264286in}}{\pgfqpoint{3.215900in}{3.268676in}}{\pgfqpoint{3.204850in}{3.268676in}}%
\pgfpathcurveto{\pgfqpoint{3.193799in}{3.268676in}}{\pgfqpoint{3.183200in}{3.264286in}}{\pgfqpoint{3.175387in}{3.256472in}}%
\pgfpathcurveto{\pgfqpoint{3.167573in}{3.248659in}}{\pgfqpoint{3.163183in}{3.238060in}}{\pgfqpoint{3.163183in}{3.227010in}}%
\pgfpathcurveto{\pgfqpoint{3.163183in}{3.215959in}}{\pgfqpoint{3.167573in}{3.205360in}}{\pgfqpoint{3.175387in}{3.197547in}}%
\pgfpathcurveto{\pgfqpoint{3.183200in}{3.189733in}}{\pgfqpoint{3.193799in}{3.185343in}}{\pgfqpoint{3.204850in}{3.185343in}}%
\pgfpathclose%
\pgfusepath{stroke,fill}%
\end{pgfscope}%
\begin{pgfscope}%
\pgfpathrectangle{\pgfqpoint{0.648703in}{0.548769in}}{\pgfqpoint{5.195150in}{3.102590in}}%
\pgfusepath{clip}%
\pgfsetbuttcap%
\pgfsetroundjoin%
\definecolor{currentfill}{rgb}{1.000000,0.498039,0.054902}%
\pgfsetfillcolor{currentfill}%
\pgfsetlinewidth{1.003750pt}%
\definecolor{currentstroke}{rgb}{1.000000,0.498039,0.054902}%
\pgfsetstrokecolor{currentstroke}%
\pgfsetdash{}{0pt}%
\pgfpathmoveto{\pgfqpoint{2.376277in}{3.198029in}}%
\pgfpathcurveto{\pgfqpoint{2.387327in}{3.198029in}}{\pgfqpoint{2.397926in}{3.202419in}}{\pgfqpoint{2.405740in}{3.210233in}}%
\pgfpathcurveto{\pgfqpoint{2.413554in}{3.218046in}}{\pgfqpoint{2.417944in}{3.228646in}}{\pgfqpoint{2.417944in}{3.239696in}}%
\pgfpathcurveto{\pgfqpoint{2.417944in}{3.250746in}}{\pgfqpoint{2.413554in}{3.261345in}}{\pgfqpoint{2.405740in}{3.269158in}}%
\pgfpathcurveto{\pgfqpoint{2.397926in}{3.276972in}}{\pgfqpoint{2.387327in}{3.281362in}}{\pgfqpoint{2.376277in}{3.281362in}}%
\pgfpathcurveto{\pgfqpoint{2.365227in}{3.281362in}}{\pgfqpoint{2.354628in}{3.276972in}}{\pgfqpoint{2.346814in}{3.269158in}}%
\pgfpathcurveto{\pgfqpoint{2.339001in}{3.261345in}}{\pgfqpoint{2.334610in}{3.250746in}}{\pgfqpoint{2.334610in}{3.239696in}}%
\pgfpathcurveto{\pgfqpoint{2.334610in}{3.228646in}}{\pgfqpoint{2.339001in}{3.218046in}}{\pgfqpoint{2.346814in}{3.210233in}}%
\pgfpathcurveto{\pgfqpoint{2.354628in}{3.202419in}}{\pgfqpoint{2.365227in}{3.198029in}}{\pgfqpoint{2.376277in}{3.198029in}}%
\pgfpathclose%
\pgfusepath{stroke,fill}%
\end{pgfscope}%
\begin{pgfscope}%
\pgfpathrectangle{\pgfqpoint{0.648703in}{0.548769in}}{\pgfqpoint{5.195150in}{3.102590in}}%
\pgfusepath{clip}%
\pgfsetbuttcap%
\pgfsetroundjoin%
\definecolor{currentfill}{rgb}{1.000000,0.498039,0.054902}%
\pgfsetfillcolor{currentfill}%
\pgfsetlinewidth{1.003750pt}%
\definecolor{currentstroke}{rgb}{1.000000,0.498039,0.054902}%
\pgfsetstrokecolor{currentstroke}%
\pgfsetdash{}{0pt}%
\pgfpathmoveto{\pgfqpoint{2.541992in}{3.185343in}}%
\pgfpathcurveto{\pgfqpoint{2.553042in}{3.185343in}}{\pgfqpoint{2.563641in}{3.189733in}}{\pgfqpoint{2.571454in}{3.197547in}}%
\pgfpathcurveto{\pgfqpoint{2.579268in}{3.205360in}}{\pgfqpoint{2.583658in}{3.215959in}}{\pgfqpoint{2.583658in}{3.227010in}}%
\pgfpathcurveto{\pgfqpoint{2.583658in}{3.238060in}}{\pgfqpoint{2.579268in}{3.248659in}}{\pgfqpoint{2.571454in}{3.256472in}}%
\pgfpathcurveto{\pgfqpoint{2.563641in}{3.264286in}}{\pgfqpoint{2.553042in}{3.268676in}}{\pgfqpoint{2.541992in}{3.268676in}}%
\pgfpathcurveto{\pgfqpoint{2.530941in}{3.268676in}}{\pgfqpoint{2.520342in}{3.264286in}}{\pgfqpoint{2.512529in}{3.256472in}}%
\pgfpathcurveto{\pgfqpoint{2.504715in}{3.248659in}}{\pgfqpoint{2.500325in}{3.238060in}}{\pgfqpoint{2.500325in}{3.227010in}}%
\pgfpathcurveto{\pgfqpoint{2.500325in}{3.215959in}}{\pgfqpoint{2.504715in}{3.205360in}}{\pgfqpoint{2.512529in}{3.197547in}}%
\pgfpathcurveto{\pgfqpoint{2.520342in}{3.189733in}}{\pgfqpoint{2.530941in}{3.185343in}}{\pgfqpoint{2.541992in}{3.185343in}}%
\pgfpathclose%
\pgfusepath{stroke,fill}%
\end{pgfscope}%
\begin{pgfscope}%
\pgfpathrectangle{\pgfqpoint{0.648703in}{0.548769in}}{\pgfqpoint{5.195150in}{3.102590in}}%
\pgfusepath{clip}%
\pgfsetbuttcap%
\pgfsetroundjoin%
\definecolor{currentfill}{rgb}{1.000000,0.498039,0.054902}%
\pgfsetfillcolor{currentfill}%
\pgfsetlinewidth{1.003750pt}%
\definecolor{currentstroke}{rgb}{1.000000,0.498039,0.054902}%
\pgfsetstrokecolor{currentstroke}%
\pgfsetdash{}{0pt}%
\pgfpathmoveto{\pgfqpoint{2.210563in}{3.193800in}}%
\pgfpathcurveto{\pgfqpoint{2.221613in}{3.193800in}}{\pgfqpoint{2.232212in}{3.198191in}}{\pgfqpoint{2.240025in}{3.206004in}}%
\pgfpathcurveto{\pgfqpoint{2.247839in}{3.213818in}}{\pgfqpoint{2.252229in}{3.224417in}}{\pgfqpoint{2.252229in}{3.235467in}}%
\pgfpathcurveto{\pgfqpoint{2.252229in}{3.246517in}}{\pgfqpoint{2.247839in}{3.257116in}}{\pgfqpoint{2.240025in}{3.264930in}}%
\pgfpathcurveto{\pgfqpoint{2.232212in}{3.272743in}}{\pgfqpoint{2.221613in}{3.277134in}}{\pgfqpoint{2.210563in}{3.277134in}}%
\pgfpathcurveto{\pgfqpoint{2.199512in}{3.277134in}}{\pgfqpoint{2.188913in}{3.272743in}}{\pgfqpoint{2.181100in}{3.264930in}}%
\pgfpathcurveto{\pgfqpoint{2.173286in}{3.257116in}}{\pgfqpoint{2.168896in}{3.246517in}}{\pgfqpoint{2.168896in}{3.235467in}}%
\pgfpathcurveto{\pgfqpoint{2.168896in}{3.224417in}}{\pgfqpoint{2.173286in}{3.213818in}}{\pgfqpoint{2.181100in}{3.206004in}}%
\pgfpathcurveto{\pgfqpoint{2.188913in}{3.198191in}}{\pgfqpoint{2.199512in}{3.193800in}}{\pgfqpoint{2.210563in}{3.193800in}}%
\pgfpathclose%
\pgfusepath{stroke,fill}%
\end{pgfscope}%
\begin{pgfscope}%
\pgfpathrectangle{\pgfqpoint{0.648703in}{0.548769in}}{\pgfqpoint{5.195150in}{3.102590in}}%
\pgfusepath{clip}%
\pgfsetbuttcap%
\pgfsetroundjoin%
\definecolor{currentfill}{rgb}{1.000000,0.498039,0.054902}%
\pgfsetfillcolor{currentfill}%
\pgfsetlinewidth{1.003750pt}%
\definecolor{currentstroke}{rgb}{1.000000,0.498039,0.054902}%
\pgfsetstrokecolor{currentstroke}%
\pgfsetdash{}{0pt}%
\pgfpathmoveto{\pgfqpoint{2.127705in}{3.193800in}}%
\pgfpathcurveto{\pgfqpoint{2.138755in}{3.193800in}}{\pgfqpoint{2.149355in}{3.198191in}}{\pgfqpoint{2.157168in}{3.206004in}}%
\pgfpathcurveto{\pgfqpoint{2.164982in}{3.213818in}}{\pgfqpoint{2.169372in}{3.224417in}}{\pgfqpoint{2.169372in}{3.235467in}}%
\pgfpathcurveto{\pgfqpoint{2.169372in}{3.246517in}}{\pgfqpoint{2.164982in}{3.257116in}}{\pgfqpoint{2.157168in}{3.264930in}}%
\pgfpathcurveto{\pgfqpoint{2.149355in}{3.272743in}}{\pgfqpoint{2.138755in}{3.277134in}}{\pgfqpoint{2.127705in}{3.277134in}}%
\pgfpathcurveto{\pgfqpoint{2.116655in}{3.277134in}}{\pgfqpoint{2.106056in}{3.272743in}}{\pgfqpoint{2.098243in}{3.264930in}}%
\pgfpathcurveto{\pgfqpoint{2.090429in}{3.257116in}}{\pgfqpoint{2.086039in}{3.246517in}}{\pgfqpoint{2.086039in}{3.235467in}}%
\pgfpathcurveto{\pgfqpoint{2.086039in}{3.224417in}}{\pgfqpoint{2.090429in}{3.213818in}}{\pgfqpoint{2.098243in}{3.206004in}}%
\pgfpathcurveto{\pgfqpoint{2.106056in}{3.198191in}}{\pgfqpoint{2.116655in}{3.193800in}}{\pgfqpoint{2.127705in}{3.193800in}}%
\pgfpathclose%
\pgfusepath{stroke,fill}%
\end{pgfscope}%
\begin{pgfscope}%
\pgfpathrectangle{\pgfqpoint{0.648703in}{0.548769in}}{\pgfqpoint{5.195150in}{3.102590in}}%
\pgfusepath{clip}%
\pgfsetbuttcap%
\pgfsetroundjoin%
\definecolor{currentfill}{rgb}{0.121569,0.466667,0.705882}%
\pgfsetfillcolor{currentfill}%
\pgfsetlinewidth{1.003750pt}%
\definecolor{currentstroke}{rgb}{0.121569,0.466667,0.705882}%
\pgfsetstrokecolor{currentstroke}%
\pgfsetdash{}{0pt}%
\pgfpathmoveto{\pgfqpoint{5.607710in}{0.652358in}}%
\pgfpathcurveto{\pgfqpoint{5.618760in}{0.652358in}}{\pgfqpoint{5.629359in}{0.656748in}}{\pgfqpoint{5.637173in}{0.664562in}}%
\pgfpathcurveto{\pgfqpoint{5.644986in}{0.672375in}}{\pgfqpoint{5.649377in}{0.682974in}}{\pgfqpoint{5.649377in}{0.694024in}}%
\pgfpathcurveto{\pgfqpoint{5.649377in}{0.705074in}}{\pgfqpoint{5.644986in}{0.715673in}}{\pgfqpoint{5.637173in}{0.723487in}}%
\pgfpathcurveto{\pgfqpoint{5.629359in}{0.731301in}}{\pgfqpoint{5.618760in}{0.735691in}}{\pgfqpoint{5.607710in}{0.735691in}}%
\pgfpathcurveto{\pgfqpoint{5.596660in}{0.735691in}}{\pgfqpoint{5.586061in}{0.731301in}}{\pgfqpoint{5.578247in}{0.723487in}}%
\pgfpathcurveto{\pgfqpoint{5.570433in}{0.715673in}}{\pgfqpoint{5.566043in}{0.705074in}}{\pgfqpoint{5.566043in}{0.694024in}}%
\pgfpathcurveto{\pgfqpoint{5.566043in}{0.682974in}}{\pgfqpoint{5.570433in}{0.672375in}}{\pgfqpoint{5.578247in}{0.664562in}}%
\pgfpathcurveto{\pgfqpoint{5.586061in}{0.656748in}}{\pgfqpoint{5.596660in}{0.652358in}}{\pgfqpoint{5.607710in}{0.652358in}}%
\pgfpathclose%
\pgfusepath{stroke,fill}%
\end{pgfscope}%
\begin{pgfscope}%
\pgfpathrectangle{\pgfqpoint{0.648703in}{0.548769in}}{\pgfqpoint{5.195150in}{3.102590in}}%
\pgfusepath{clip}%
\pgfsetbuttcap%
\pgfsetroundjoin%
\definecolor{currentfill}{rgb}{0.121569,0.466667,0.705882}%
\pgfsetfillcolor{currentfill}%
\pgfsetlinewidth{1.003750pt}%
\definecolor{currentstroke}{rgb}{0.121569,0.466667,0.705882}%
\pgfsetstrokecolor{currentstroke}%
\pgfsetdash{}{0pt}%
\pgfpathmoveto{\pgfqpoint{1.630562in}{0.648129in}}%
\pgfpathcurveto{\pgfqpoint{1.641612in}{0.648129in}}{\pgfqpoint{1.652211in}{0.652519in}}{\pgfqpoint{1.660025in}{0.660333in}}%
\pgfpathcurveto{\pgfqpoint{1.667838in}{0.668146in}}{\pgfqpoint{1.672229in}{0.678745in}}{\pgfqpoint{1.672229in}{0.689796in}}%
\pgfpathcurveto{\pgfqpoint{1.672229in}{0.700846in}}{\pgfqpoint{1.667838in}{0.711445in}}{\pgfqpoint{1.660025in}{0.719258in}}%
\pgfpathcurveto{\pgfqpoint{1.652211in}{0.727072in}}{\pgfqpoint{1.641612in}{0.731462in}}{\pgfqpoint{1.630562in}{0.731462in}}%
\pgfpathcurveto{\pgfqpoint{1.619512in}{0.731462in}}{\pgfqpoint{1.608913in}{0.727072in}}{\pgfqpoint{1.601099in}{0.719258in}}%
\pgfpathcurveto{\pgfqpoint{1.593285in}{0.711445in}}{\pgfqpoint{1.588895in}{0.700846in}}{\pgfqpoint{1.588895in}{0.689796in}}%
\pgfpathcurveto{\pgfqpoint{1.588895in}{0.678745in}}{\pgfqpoint{1.593285in}{0.668146in}}{\pgfqpoint{1.601099in}{0.660333in}}%
\pgfpathcurveto{\pgfqpoint{1.608913in}{0.652519in}}{\pgfqpoint{1.619512in}{0.648129in}}{\pgfqpoint{1.630562in}{0.648129in}}%
\pgfpathclose%
\pgfusepath{stroke,fill}%
\end{pgfscope}%
\begin{pgfscope}%
\pgfpathrectangle{\pgfqpoint{0.648703in}{0.548769in}}{\pgfqpoint{5.195150in}{3.102590in}}%
\pgfusepath{clip}%
\pgfsetbuttcap%
\pgfsetroundjoin%
\definecolor{currentfill}{rgb}{0.121569,0.466667,0.705882}%
\pgfsetfillcolor{currentfill}%
\pgfsetlinewidth{1.003750pt}%
\definecolor{currentstroke}{rgb}{0.121569,0.466667,0.705882}%
\pgfsetstrokecolor{currentstroke}%
\pgfsetdash{}{0pt}%
\pgfpathmoveto{\pgfqpoint{1.464847in}{0.648129in}}%
\pgfpathcurveto{\pgfqpoint{1.475897in}{0.648129in}}{\pgfqpoint{1.486497in}{0.652519in}}{\pgfqpoint{1.494310in}{0.660333in}}%
\pgfpathcurveto{\pgfqpoint{1.502124in}{0.668146in}}{\pgfqpoint{1.506514in}{0.678745in}}{\pgfqpoint{1.506514in}{0.689796in}}%
\pgfpathcurveto{\pgfqpoint{1.506514in}{0.700846in}}{\pgfqpoint{1.502124in}{0.711445in}}{\pgfqpoint{1.494310in}{0.719258in}}%
\pgfpathcurveto{\pgfqpoint{1.486497in}{0.727072in}}{\pgfqpoint{1.475897in}{0.731462in}}{\pgfqpoint{1.464847in}{0.731462in}}%
\pgfpathcurveto{\pgfqpoint{1.453797in}{0.731462in}}{\pgfqpoint{1.443198in}{0.727072in}}{\pgfqpoint{1.435385in}{0.719258in}}%
\pgfpathcurveto{\pgfqpoint{1.427571in}{0.711445in}}{\pgfqpoint{1.423181in}{0.700846in}}{\pgfqpoint{1.423181in}{0.689796in}}%
\pgfpathcurveto{\pgfqpoint{1.423181in}{0.678745in}}{\pgfqpoint{1.427571in}{0.668146in}}{\pgfqpoint{1.435385in}{0.660333in}}%
\pgfpathcurveto{\pgfqpoint{1.443198in}{0.652519in}}{\pgfqpoint{1.453797in}{0.648129in}}{\pgfqpoint{1.464847in}{0.648129in}}%
\pgfpathclose%
\pgfusepath{stroke,fill}%
\end{pgfscope}%
\begin{pgfscope}%
\pgfpathrectangle{\pgfqpoint{0.648703in}{0.548769in}}{\pgfqpoint{5.195150in}{3.102590in}}%
\pgfusepath{clip}%
\pgfsetbuttcap%
\pgfsetroundjoin%
\definecolor{currentfill}{rgb}{0.121569,0.466667,0.705882}%
\pgfsetfillcolor{currentfill}%
\pgfsetlinewidth{1.003750pt}%
\definecolor{currentstroke}{rgb}{0.121569,0.466667,0.705882}%
\pgfsetstrokecolor{currentstroke}%
\pgfsetdash{}{0pt}%
\pgfpathmoveto{\pgfqpoint{1.381990in}{0.648129in}}%
\pgfpathcurveto{\pgfqpoint{1.393040in}{0.648129in}}{\pgfqpoint{1.403639in}{0.652519in}}{\pgfqpoint{1.411453in}{0.660333in}}%
\pgfpathcurveto{\pgfqpoint{1.419266in}{0.668146in}}{\pgfqpoint{1.423657in}{0.678745in}}{\pgfqpoint{1.423657in}{0.689796in}}%
\pgfpathcurveto{\pgfqpoint{1.423657in}{0.700846in}}{\pgfqpoint{1.419266in}{0.711445in}}{\pgfqpoint{1.411453in}{0.719258in}}%
\pgfpathcurveto{\pgfqpoint{1.403639in}{0.727072in}}{\pgfqpoint{1.393040in}{0.731462in}}{\pgfqpoint{1.381990in}{0.731462in}}%
\pgfpathcurveto{\pgfqpoint{1.370940in}{0.731462in}}{\pgfqpoint{1.360341in}{0.727072in}}{\pgfqpoint{1.352527in}{0.719258in}}%
\pgfpathcurveto{\pgfqpoint{1.344714in}{0.711445in}}{\pgfqpoint{1.340323in}{0.700846in}}{\pgfqpoint{1.340323in}{0.689796in}}%
\pgfpathcurveto{\pgfqpoint{1.340323in}{0.678745in}}{\pgfqpoint{1.344714in}{0.668146in}}{\pgfqpoint{1.352527in}{0.660333in}}%
\pgfpathcurveto{\pgfqpoint{1.360341in}{0.652519in}}{\pgfqpoint{1.370940in}{0.648129in}}{\pgfqpoint{1.381990in}{0.648129in}}%
\pgfpathclose%
\pgfusepath{stroke,fill}%
\end{pgfscope}%
\begin{pgfscope}%
\pgfpathrectangle{\pgfqpoint{0.648703in}{0.548769in}}{\pgfqpoint{5.195150in}{3.102590in}}%
\pgfusepath{clip}%
\pgfsetbuttcap%
\pgfsetroundjoin%
\definecolor{currentfill}{rgb}{0.121569,0.466667,0.705882}%
\pgfsetfillcolor{currentfill}%
\pgfsetlinewidth{1.003750pt}%
\definecolor{currentstroke}{rgb}{0.121569,0.466667,0.705882}%
\pgfsetstrokecolor{currentstroke}%
\pgfsetdash{}{0pt}%
\pgfpathmoveto{\pgfqpoint{1.381990in}{0.648129in}}%
\pgfpathcurveto{\pgfqpoint{1.393040in}{0.648129in}}{\pgfqpoint{1.403639in}{0.652519in}}{\pgfqpoint{1.411453in}{0.660333in}}%
\pgfpathcurveto{\pgfqpoint{1.419266in}{0.668146in}}{\pgfqpoint{1.423657in}{0.678745in}}{\pgfqpoint{1.423657in}{0.689796in}}%
\pgfpathcurveto{\pgfqpoint{1.423657in}{0.700846in}}{\pgfqpoint{1.419266in}{0.711445in}}{\pgfqpoint{1.411453in}{0.719258in}}%
\pgfpathcurveto{\pgfqpoint{1.403639in}{0.727072in}}{\pgfqpoint{1.393040in}{0.731462in}}{\pgfqpoint{1.381990in}{0.731462in}}%
\pgfpathcurveto{\pgfqpoint{1.370940in}{0.731462in}}{\pgfqpoint{1.360341in}{0.727072in}}{\pgfqpoint{1.352527in}{0.719258in}}%
\pgfpathcurveto{\pgfqpoint{1.344714in}{0.711445in}}{\pgfqpoint{1.340323in}{0.700846in}}{\pgfqpoint{1.340323in}{0.689796in}}%
\pgfpathcurveto{\pgfqpoint{1.340323in}{0.678745in}}{\pgfqpoint{1.344714in}{0.668146in}}{\pgfqpoint{1.352527in}{0.660333in}}%
\pgfpathcurveto{\pgfqpoint{1.360341in}{0.652519in}}{\pgfqpoint{1.370940in}{0.648129in}}{\pgfqpoint{1.381990in}{0.648129in}}%
\pgfpathclose%
\pgfusepath{stroke,fill}%
\end{pgfscope}%
\begin{pgfscope}%
\pgfpathrectangle{\pgfqpoint{0.648703in}{0.548769in}}{\pgfqpoint{5.195150in}{3.102590in}}%
\pgfusepath{clip}%
\pgfsetbuttcap%
\pgfsetroundjoin%
\definecolor{currentfill}{rgb}{0.121569,0.466667,0.705882}%
\pgfsetfillcolor{currentfill}%
\pgfsetlinewidth{1.003750pt}%
\definecolor{currentstroke}{rgb}{0.121569,0.466667,0.705882}%
\pgfsetstrokecolor{currentstroke}%
\pgfsetdash{}{0pt}%
\pgfpathmoveto{\pgfqpoint{1.547705in}{0.648129in}}%
\pgfpathcurveto{\pgfqpoint{1.558755in}{0.648129in}}{\pgfqpoint{1.569354in}{0.652519in}}{\pgfqpoint{1.577167in}{0.660333in}}%
\pgfpathcurveto{\pgfqpoint{1.584981in}{0.668146in}}{\pgfqpoint{1.589371in}{0.678745in}}{\pgfqpoint{1.589371in}{0.689796in}}%
\pgfpathcurveto{\pgfqpoint{1.589371in}{0.700846in}}{\pgfqpoint{1.584981in}{0.711445in}}{\pgfqpoint{1.577167in}{0.719258in}}%
\pgfpathcurveto{\pgfqpoint{1.569354in}{0.727072in}}{\pgfqpoint{1.558755in}{0.731462in}}{\pgfqpoint{1.547705in}{0.731462in}}%
\pgfpathcurveto{\pgfqpoint{1.536654in}{0.731462in}}{\pgfqpoint{1.526055in}{0.727072in}}{\pgfqpoint{1.518242in}{0.719258in}}%
\pgfpathcurveto{\pgfqpoint{1.510428in}{0.711445in}}{\pgfqpoint{1.506038in}{0.700846in}}{\pgfqpoint{1.506038in}{0.689796in}}%
\pgfpathcurveto{\pgfqpoint{1.506038in}{0.678745in}}{\pgfqpoint{1.510428in}{0.668146in}}{\pgfqpoint{1.518242in}{0.660333in}}%
\pgfpathcurveto{\pgfqpoint{1.526055in}{0.652519in}}{\pgfqpoint{1.536654in}{0.648129in}}{\pgfqpoint{1.547705in}{0.648129in}}%
\pgfpathclose%
\pgfusepath{stroke,fill}%
\end{pgfscope}%
\begin{pgfscope}%
\pgfpathrectangle{\pgfqpoint{0.648703in}{0.548769in}}{\pgfqpoint{5.195150in}{3.102590in}}%
\pgfusepath{clip}%
\pgfsetbuttcap%
\pgfsetroundjoin%
\definecolor{currentfill}{rgb}{0.839216,0.152941,0.156863}%
\pgfsetfillcolor{currentfill}%
\pgfsetlinewidth{1.003750pt}%
\definecolor{currentstroke}{rgb}{0.839216,0.152941,0.156863}%
\pgfsetstrokecolor{currentstroke}%
\pgfsetdash{}{0pt}%
\pgfpathmoveto{\pgfqpoint{1.547705in}{3.189572in}}%
\pgfpathcurveto{\pgfqpoint{1.558755in}{3.189572in}}{\pgfqpoint{1.569354in}{3.193962in}}{\pgfqpoint{1.577167in}{3.201775in}}%
\pgfpathcurveto{\pgfqpoint{1.584981in}{3.209589in}}{\pgfqpoint{1.589371in}{3.220188in}}{\pgfqpoint{1.589371in}{3.231238in}}%
\pgfpathcurveto{\pgfqpoint{1.589371in}{3.242288in}}{\pgfqpoint{1.584981in}{3.252887in}}{\pgfqpoint{1.577167in}{3.260701in}}%
\pgfpathcurveto{\pgfqpoint{1.569354in}{3.268515in}}{\pgfqpoint{1.558755in}{3.272905in}}{\pgfqpoint{1.547705in}{3.272905in}}%
\pgfpathcurveto{\pgfqpoint{1.536654in}{3.272905in}}{\pgfqpoint{1.526055in}{3.268515in}}{\pgfqpoint{1.518242in}{3.260701in}}%
\pgfpathcurveto{\pgfqpoint{1.510428in}{3.252887in}}{\pgfqpoint{1.506038in}{3.242288in}}{\pgfqpoint{1.506038in}{3.231238in}}%
\pgfpathcurveto{\pgfqpoint{1.506038in}{3.220188in}}{\pgfqpoint{1.510428in}{3.209589in}}{\pgfqpoint{1.518242in}{3.201775in}}%
\pgfpathcurveto{\pgfqpoint{1.526055in}{3.193962in}}{\pgfqpoint{1.536654in}{3.189572in}}{\pgfqpoint{1.547705in}{3.189572in}}%
\pgfpathclose%
\pgfusepath{stroke,fill}%
\end{pgfscope}%
\begin{pgfscope}%
\pgfpathrectangle{\pgfqpoint{0.648703in}{0.548769in}}{\pgfqpoint{5.195150in}{3.102590in}}%
\pgfusepath{clip}%
\pgfsetbuttcap%
\pgfsetroundjoin%
\definecolor{currentfill}{rgb}{0.121569,0.466667,0.705882}%
\pgfsetfillcolor{currentfill}%
\pgfsetlinewidth{1.003750pt}%
\definecolor{currentstroke}{rgb}{0.121569,0.466667,0.705882}%
\pgfsetstrokecolor{currentstroke}%
\pgfsetdash{}{0pt}%
\pgfpathmoveto{\pgfqpoint{2.707706in}{3.181114in}}%
\pgfpathcurveto{\pgfqpoint{2.718756in}{3.181114in}}{\pgfqpoint{2.729355in}{3.185504in}}{\pgfqpoint{2.737169in}{3.193318in}}%
\pgfpathcurveto{\pgfqpoint{2.744983in}{3.201132in}}{\pgfqpoint{2.749373in}{3.211731in}}{\pgfqpoint{2.749373in}{3.222781in}}%
\pgfpathcurveto{\pgfqpoint{2.749373in}{3.233831in}}{\pgfqpoint{2.744983in}{3.244430in}}{\pgfqpoint{2.737169in}{3.252244in}}%
\pgfpathcurveto{\pgfqpoint{2.729355in}{3.260057in}}{\pgfqpoint{2.718756in}{3.264448in}}{\pgfqpoint{2.707706in}{3.264448in}}%
\pgfpathcurveto{\pgfqpoint{2.696656in}{3.264448in}}{\pgfqpoint{2.686057in}{3.260057in}}{\pgfqpoint{2.678243in}{3.252244in}}%
\pgfpathcurveto{\pgfqpoint{2.670430in}{3.244430in}}{\pgfqpoint{2.666039in}{3.233831in}}{\pgfqpoint{2.666039in}{3.222781in}}%
\pgfpathcurveto{\pgfqpoint{2.666039in}{3.211731in}}{\pgfqpoint{2.670430in}{3.201132in}}{\pgfqpoint{2.678243in}{3.193318in}}%
\pgfpathcurveto{\pgfqpoint{2.686057in}{3.185504in}}{\pgfqpoint{2.696656in}{3.181114in}}{\pgfqpoint{2.707706in}{3.181114in}}%
\pgfpathclose%
\pgfusepath{stroke,fill}%
\end{pgfscope}%
\begin{pgfscope}%
\pgfpathrectangle{\pgfqpoint{0.648703in}{0.548769in}}{\pgfqpoint{5.195150in}{3.102590in}}%
\pgfusepath{clip}%
\pgfsetbuttcap%
\pgfsetroundjoin%
\definecolor{currentfill}{rgb}{1.000000,0.498039,0.054902}%
\pgfsetfillcolor{currentfill}%
\pgfsetlinewidth{1.003750pt}%
\definecolor{currentstroke}{rgb}{1.000000,0.498039,0.054902}%
\pgfsetstrokecolor{currentstroke}%
\pgfsetdash{}{0pt}%
\pgfpathmoveto{\pgfqpoint{2.624849in}{3.193800in}}%
\pgfpathcurveto{\pgfqpoint{2.635899in}{3.193800in}}{\pgfqpoint{2.646498in}{3.198191in}}{\pgfqpoint{2.654312in}{3.206004in}}%
\pgfpathcurveto{\pgfqpoint{2.662125in}{3.213818in}}{\pgfqpoint{2.666516in}{3.224417in}}{\pgfqpoint{2.666516in}{3.235467in}}%
\pgfpathcurveto{\pgfqpoint{2.666516in}{3.246517in}}{\pgfqpoint{2.662125in}{3.257116in}}{\pgfqpoint{2.654312in}{3.264930in}}%
\pgfpathcurveto{\pgfqpoint{2.646498in}{3.272743in}}{\pgfqpoint{2.635899in}{3.277134in}}{\pgfqpoint{2.624849in}{3.277134in}}%
\pgfpathcurveto{\pgfqpoint{2.613799in}{3.277134in}}{\pgfqpoint{2.603200in}{3.272743in}}{\pgfqpoint{2.595386in}{3.264930in}}%
\pgfpathcurveto{\pgfqpoint{2.587572in}{3.257116in}}{\pgfqpoint{2.583182in}{3.246517in}}{\pgfqpoint{2.583182in}{3.235467in}}%
\pgfpathcurveto{\pgfqpoint{2.583182in}{3.224417in}}{\pgfqpoint{2.587572in}{3.213818in}}{\pgfqpoint{2.595386in}{3.206004in}}%
\pgfpathcurveto{\pgfqpoint{2.603200in}{3.198191in}}{\pgfqpoint{2.613799in}{3.193800in}}{\pgfqpoint{2.624849in}{3.193800in}}%
\pgfpathclose%
\pgfusepath{stroke,fill}%
\end{pgfscope}%
\begin{pgfscope}%
\pgfpathrectangle{\pgfqpoint{0.648703in}{0.548769in}}{\pgfqpoint{5.195150in}{3.102590in}}%
\pgfusepath{clip}%
\pgfsetbuttcap%
\pgfsetroundjoin%
\definecolor{currentfill}{rgb}{1.000000,0.498039,0.054902}%
\pgfsetfillcolor{currentfill}%
\pgfsetlinewidth{1.003750pt}%
\definecolor{currentstroke}{rgb}{1.000000,0.498039,0.054902}%
\pgfsetstrokecolor{currentstroke}%
\pgfsetdash{}{0pt}%
\pgfpathmoveto{\pgfqpoint{2.459134in}{3.193800in}}%
\pgfpathcurveto{\pgfqpoint{2.470184in}{3.193800in}}{\pgfqpoint{2.480784in}{3.198191in}}{\pgfqpoint{2.488597in}{3.206004in}}%
\pgfpathcurveto{\pgfqpoint{2.496411in}{3.213818in}}{\pgfqpoint{2.500801in}{3.224417in}}{\pgfqpoint{2.500801in}{3.235467in}}%
\pgfpathcurveto{\pgfqpoint{2.500801in}{3.246517in}}{\pgfqpoint{2.496411in}{3.257116in}}{\pgfqpoint{2.488597in}{3.264930in}}%
\pgfpathcurveto{\pgfqpoint{2.480784in}{3.272743in}}{\pgfqpoint{2.470184in}{3.277134in}}{\pgfqpoint{2.459134in}{3.277134in}}%
\pgfpathcurveto{\pgfqpoint{2.448084in}{3.277134in}}{\pgfqpoint{2.437485in}{3.272743in}}{\pgfqpoint{2.429672in}{3.264930in}}%
\pgfpathcurveto{\pgfqpoint{2.421858in}{3.257116in}}{\pgfqpoint{2.417468in}{3.246517in}}{\pgfqpoint{2.417468in}{3.235467in}}%
\pgfpathcurveto{\pgfqpoint{2.417468in}{3.224417in}}{\pgfqpoint{2.421858in}{3.213818in}}{\pgfqpoint{2.429672in}{3.206004in}}%
\pgfpathcurveto{\pgfqpoint{2.437485in}{3.198191in}}{\pgfqpoint{2.448084in}{3.193800in}}{\pgfqpoint{2.459134in}{3.193800in}}%
\pgfpathclose%
\pgfusepath{stroke,fill}%
\end{pgfscope}%
\begin{pgfscope}%
\pgfpathrectangle{\pgfqpoint{0.648703in}{0.548769in}}{\pgfqpoint{5.195150in}{3.102590in}}%
\pgfusepath{clip}%
\pgfsetbuttcap%
\pgfsetroundjoin%
\definecolor{currentfill}{rgb}{0.121569,0.466667,0.705882}%
\pgfsetfillcolor{currentfill}%
\pgfsetlinewidth{1.003750pt}%
\definecolor{currentstroke}{rgb}{0.121569,0.466667,0.705882}%
\pgfsetstrokecolor{currentstroke}%
\pgfsetdash{}{0pt}%
\pgfpathmoveto{\pgfqpoint{1.464847in}{2.551039in}}%
\pgfpathcurveto{\pgfqpoint{1.475897in}{2.551039in}}{\pgfqpoint{1.486497in}{2.555430in}}{\pgfqpoint{1.494310in}{2.563243in}}%
\pgfpathcurveto{\pgfqpoint{1.502124in}{2.571057in}}{\pgfqpoint{1.506514in}{2.581656in}}{\pgfqpoint{1.506514in}{2.592706in}}%
\pgfpathcurveto{\pgfqpoint{1.506514in}{2.603756in}}{\pgfqpoint{1.502124in}{2.614355in}}{\pgfqpoint{1.494310in}{2.622169in}}%
\pgfpathcurveto{\pgfqpoint{1.486497in}{2.629982in}}{\pgfqpoint{1.475897in}{2.634373in}}{\pgfqpoint{1.464847in}{2.634373in}}%
\pgfpathcurveto{\pgfqpoint{1.453797in}{2.634373in}}{\pgfqpoint{1.443198in}{2.629982in}}{\pgfqpoint{1.435385in}{2.622169in}}%
\pgfpathcurveto{\pgfqpoint{1.427571in}{2.614355in}}{\pgfqpoint{1.423181in}{2.603756in}}{\pgfqpoint{1.423181in}{2.592706in}}%
\pgfpathcurveto{\pgfqpoint{1.423181in}{2.581656in}}{\pgfqpoint{1.427571in}{2.571057in}}{\pgfqpoint{1.435385in}{2.563243in}}%
\pgfpathcurveto{\pgfqpoint{1.443198in}{2.555430in}}{\pgfqpoint{1.453797in}{2.551039in}}{\pgfqpoint{1.464847in}{2.551039in}}%
\pgfpathclose%
\pgfusepath{stroke,fill}%
\end{pgfscope}%
\begin{pgfscope}%
\pgfpathrectangle{\pgfqpoint{0.648703in}{0.548769in}}{\pgfqpoint{5.195150in}{3.102590in}}%
\pgfusepath{clip}%
\pgfsetbuttcap%
\pgfsetroundjoin%
\definecolor{currentfill}{rgb}{0.121569,0.466667,0.705882}%
\pgfsetfillcolor{currentfill}%
\pgfsetlinewidth{1.003750pt}%
\definecolor{currentstroke}{rgb}{0.121569,0.466667,0.705882}%
\pgfsetstrokecolor{currentstroke}%
\pgfsetdash{}{0pt}%
\pgfpathmoveto{\pgfqpoint{1.796276in}{0.648129in}}%
\pgfpathcurveto{\pgfqpoint{1.807326in}{0.648129in}}{\pgfqpoint{1.817926in}{0.652519in}}{\pgfqpoint{1.825739in}{0.660333in}}%
\pgfpathcurveto{\pgfqpoint{1.833553in}{0.668146in}}{\pgfqpoint{1.837943in}{0.678745in}}{\pgfqpoint{1.837943in}{0.689796in}}%
\pgfpathcurveto{\pgfqpoint{1.837943in}{0.700846in}}{\pgfqpoint{1.833553in}{0.711445in}}{\pgfqpoint{1.825739in}{0.719258in}}%
\pgfpathcurveto{\pgfqpoint{1.817926in}{0.727072in}}{\pgfqpoint{1.807326in}{0.731462in}}{\pgfqpoint{1.796276in}{0.731462in}}%
\pgfpathcurveto{\pgfqpoint{1.785226in}{0.731462in}}{\pgfqpoint{1.774627in}{0.727072in}}{\pgfqpoint{1.766814in}{0.719258in}}%
\pgfpathcurveto{\pgfqpoint{1.759000in}{0.711445in}}{\pgfqpoint{1.754610in}{0.700846in}}{\pgfqpoint{1.754610in}{0.689796in}}%
\pgfpathcurveto{\pgfqpoint{1.754610in}{0.678745in}}{\pgfqpoint{1.759000in}{0.668146in}}{\pgfqpoint{1.766814in}{0.660333in}}%
\pgfpathcurveto{\pgfqpoint{1.774627in}{0.652519in}}{\pgfqpoint{1.785226in}{0.648129in}}{\pgfqpoint{1.796276in}{0.648129in}}%
\pgfpathclose%
\pgfusepath{stroke,fill}%
\end{pgfscope}%
\begin{pgfscope}%
\pgfpathrectangle{\pgfqpoint{0.648703in}{0.548769in}}{\pgfqpoint{5.195150in}{3.102590in}}%
\pgfusepath{clip}%
\pgfsetbuttcap%
\pgfsetroundjoin%
\definecolor{currentfill}{rgb}{1.000000,0.498039,0.054902}%
\pgfsetfillcolor{currentfill}%
\pgfsetlinewidth{1.003750pt}%
\definecolor{currentstroke}{rgb}{1.000000,0.498039,0.054902}%
\pgfsetstrokecolor{currentstroke}%
\pgfsetdash{}{0pt}%
\pgfpathmoveto{\pgfqpoint{4.199137in}{3.198029in}}%
\pgfpathcurveto{\pgfqpoint{4.210187in}{3.198029in}}{\pgfqpoint{4.220786in}{3.202419in}}{\pgfqpoint{4.228599in}{3.210233in}}%
\pgfpathcurveto{\pgfqpoint{4.236413in}{3.218046in}}{\pgfqpoint{4.240803in}{3.228646in}}{\pgfqpoint{4.240803in}{3.239696in}}%
\pgfpathcurveto{\pgfqpoint{4.240803in}{3.250746in}}{\pgfqpoint{4.236413in}{3.261345in}}{\pgfqpoint{4.228599in}{3.269158in}}%
\pgfpathcurveto{\pgfqpoint{4.220786in}{3.276972in}}{\pgfqpoint{4.210187in}{3.281362in}}{\pgfqpoint{4.199137in}{3.281362in}}%
\pgfpathcurveto{\pgfqpoint{4.188086in}{3.281362in}}{\pgfqpoint{4.177487in}{3.276972in}}{\pgfqpoint{4.169674in}{3.269158in}}%
\pgfpathcurveto{\pgfqpoint{4.161860in}{3.261345in}}{\pgfqpoint{4.157470in}{3.250746in}}{\pgfqpoint{4.157470in}{3.239696in}}%
\pgfpathcurveto{\pgfqpoint{4.157470in}{3.228646in}}{\pgfqpoint{4.161860in}{3.218046in}}{\pgfqpoint{4.169674in}{3.210233in}}%
\pgfpathcurveto{\pgfqpoint{4.177487in}{3.202419in}}{\pgfqpoint{4.188086in}{3.198029in}}{\pgfqpoint{4.199137in}{3.198029in}}%
\pgfpathclose%
\pgfusepath{stroke,fill}%
\end{pgfscope}%
\begin{pgfscope}%
\pgfpathrectangle{\pgfqpoint{0.648703in}{0.548769in}}{\pgfqpoint{5.195150in}{3.102590in}}%
\pgfusepath{clip}%
\pgfsetbuttcap%
\pgfsetroundjoin%
\definecolor{currentfill}{rgb}{1.000000,0.498039,0.054902}%
\pgfsetfillcolor{currentfill}%
\pgfsetlinewidth{1.003750pt}%
\definecolor{currentstroke}{rgb}{1.000000,0.498039,0.054902}%
\pgfsetstrokecolor{currentstroke}%
\pgfsetdash{}{0pt}%
\pgfpathmoveto{\pgfqpoint{1.630562in}{3.189572in}}%
\pgfpathcurveto{\pgfqpoint{1.641612in}{3.189572in}}{\pgfqpoint{1.652211in}{3.193962in}}{\pgfqpoint{1.660025in}{3.201775in}}%
\pgfpathcurveto{\pgfqpoint{1.667838in}{3.209589in}}{\pgfqpoint{1.672229in}{3.220188in}}{\pgfqpoint{1.672229in}{3.231238in}}%
\pgfpathcurveto{\pgfqpoint{1.672229in}{3.242288in}}{\pgfqpoint{1.667838in}{3.252887in}}{\pgfqpoint{1.660025in}{3.260701in}}%
\pgfpathcurveto{\pgfqpoint{1.652211in}{3.268515in}}{\pgfqpoint{1.641612in}{3.272905in}}{\pgfqpoint{1.630562in}{3.272905in}}%
\pgfpathcurveto{\pgfqpoint{1.619512in}{3.272905in}}{\pgfqpoint{1.608913in}{3.268515in}}{\pgfqpoint{1.601099in}{3.260701in}}%
\pgfpathcurveto{\pgfqpoint{1.593285in}{3.252887in}}{\pgfqpoint{1.588895in}{3.242288in}}{\pgfqpoint{1.588895in}{3.231238in}}%
\pgfpathcurveto{\pgfqpoint{1.588895in}{3.220188in}}{\pgfqpoint{1.593285in}{3.209589in}}{\pgfqpoint{1.601099in}{3.201775in}}%
\pgfpathcurveto{\pgfqpoint{1.608913in}{3.193962in}}{\pgfqpoint{1.619512in}{3.189572in}}{\pgfqpoint{1.630562in}{3.189572in}}%
\pgfpathclose%
\pgfusepath{stroke,fill}%
\end{pgfscope}%
\begin{pgfscope}%
\pgfpathrectangle{\pgfqpoint{0.648703in}{0.548769in}}{\pgfqpoint{5.195150in}{3.102590in}}%
\pgfusepath{clip}%
\pgfsetbuttcap%
\pgfsetroundjoin%
\definecolor{currentfill}{rgb}{0.121569,0.466667,0.705882}%
\pgfsetfillcolor{currentfill}%
\pgfsetlinewidth{1.003750pt}%
\definecolor{currentstroke}{rgb}{0.121569,0.466667,0.705882}%
\pgfsetstrokecolor{currentstroke}%
\pgfsetdash{}{0pt}%
\pgfpathmoveto{\pgfqpoint{1.216276in}{0.808819in}}%
\pgfpathcurveto{\pgfqpoint{1.227326in}{0.808819in}}{\pgfqpoint{1.237925in}{0.813209in}}{\pgfqpoint{1.245738in}{0.821023in}}%
\pgfpathcurveto{\pgfqpoint{1.253552in}{0.828837in}}{\pgfqpoint{1.257942in}{0.839436in}}{\pgfqpoint{1.257942in}{0.850486in}}%
\pgfpathcurveto{\pgfqpoint{1.257942in}{0.861536in}}{\pgfqpoint{1.253552in}{0.872135in}}{\pgfqpoint{1.245738in}{0.879949in}}%
\pgfpathcurveto{\pgfqpoint{1.237925in}{0.887762in}}{\pgfqpoint{1.227326in}{0.892152in}}{\pgfqpoint{1.216276in}{0.892152in}}%
\pgfpathcurveto{\pgfqpoint{1.205225in}{0.892152in}}{\pgfqpoint{1.194626in}{0.887762in}}{\pgfqpoint{1.186813in}{0.879949in}}%
\pgfpathcurveto{\pgfqpoint{1.178999in}{0.872135in}}{\pgfqpoint{1.174609in}{0.861536in}}{\pgfqpoint{1.174609in}{0.850486in}}%
\pgfpathcurveto{\pgfqpoint{1.174609in}{0.839436in}}{\pgfqpoint{1.178999in}{0.828837in}}{\pgfqpoint{1.186813in}{0.821023in}}%
\pgfpathcurveto{\pgfqpoint{1.194626in}{0.813209in}}{\pgfqpoint{1.205225in}{0.808819in}}{\pgfqpoint{1.216276in}{0.808819in}}%
\pgfpathclose%
\pgfusepath{stroke,fill}%
\end{pgfscope}%
\begin{pgfscope}%
\pgfpathrectangle{\pgfqpoint{0.648703in}{0.548769in}}{\pgfqpoint{5.195150in}{3.102590in}}%
\pgfusepath{clip}%
\pgfsetbuttcap%
\pgfsetroundjoin%
\definecolor{currentfill}{rgb}{0.839216,0.152941,0.156863}%
\pgfsetfillcolor{currentfill}%
\pgfsetlinewidth{1.003750pt}%
\definecolor{currentstroke}{rgb}{0.839216,0.152941,0.156863}%
\pgfsetstrokecolor{currentstroke}%
\pgfsetdash{}{0pt}%
\pgfpathmoveto{\pgfqpoint{2.127705in}{3.198029in}}%
\pgfpathcurveto{\pgfqpoint{2.138755in}{3.198029in}}{\pgfqpoint{2.149355in}{3.202419in}}{\pgfqpoint{2.157168in}{3.210233in}}%
\pgfpathcurveto{\pgfqpoint{2.164982in}{3.218046in}}{\pgfqpoint{2.169372in}{3.228646in}}{\pgfqpoint{2.169372in}{3.239696in}}%
\pgfpathcurveto{\pgfqpoint{2.169372in}{3.250746in}}{\pgfqpoint{2.164982in}{3.261345in}}{\pgfqpoint{2.157168in}{3.269158in}}%
\pgfpathcurveto{\pgfqpoint{2.149355in}{3.276972in}}{\pgfqpoint{2.138755in}{3.281362in}}{\pgfqpoint{2.127705in}{3.281362in}}%
\pgfpathcurveto{\pgfqpoint{2.116655in}{3.281362in}}{\pgfqpoint{2.106056in}{3.276972in}}{\pgfqpoint{2.098243in}{3.269158in}}%
\pgfpathcurveto{\pgfqpoint{2.090429in}{3.261345in}}{\pgfqpoint{2.086039in}{3.250746in}}{\pgfqpoint{2.086039in}{3.239696in}}%
\pgfpathcurveto{\pgfqpoint{2.086039in}{3.228646in}}{\pgfqpoint{2.090429in}{3.218046in}}{\pgfqpoint{2.098243in}{3.210233in}}%
\pgfpathcurveto{\pgfqpoint{2.106056in}{3.202419in}}{\pgfqpoint{2.116655in}{3.198029in}}{\pgfqpoint{2.127705in}{3.198029in}}%
\pgfpathclose%
\pgfusepath{stroke,fill}%
\end{pgfscope}%
\begin{pgfscope}%
\pgfpathrectangle{\pgfqpoint{0.648703in}{0.548769in}}{\pgfqpoint{5.195150in}{3.102590in}}%
\pgfusepath{clip}%
\pgfsetbuttcap%
\pgfsetroundjoin%
\definecolor{currentfill}{rgb}{0.121569,0.466667,0.705882}%
\pgfsetfillcolor{currentfill}%
\pgfsetlinewidth{1.003750pt}%
\definecolor{currentstroke}{rgb}{0.121569,0.466667,0.705882}%
\pgfsetstrokecolor{currentstroke}%
\pgfsetdash{}{0pt}%
\pgfpathmoveto{\pgfqpoint{1.299133in}{0.648129in}}%
\pgfpathcurveto{\pgfqpoint{1.310183in}{0.648129in}}{\pgfqpoint{1.320782in}{0.652519in}}{\pgfqpoint{1.328596in}{0.660333in}}%
\pgfpathcurveto{\pgfqpoint{1.336409in}{0.668146in}}{\pgfqpoint{1.340800in}{0.678745in}}{\pgfqpoint{1.340800in}{0.689796in}}%
\pgfpathcurveto{\pgfqpoint{1.340800in}{0.700846in}}{\pgfqpoint{1.336409in}{0.711445in}}{\pgfqpoint{1.328596in}{0.719258in}}%
\pgfpathcurveto{\pgfqpoint{1.320782in}{0.727072in}}{\pgfqpoint{1.310183in}{0.731462in}}{\pgfqpoint{1.299133in}{0.731462in}}%
\pgfpathcurveto{\pgfqpoint{1.288083in}{0.731462in}}{\pgfqpoint{1.277484in}{0.727072in}}{\pgfqpoint{1.269670in}{0.719258in}}%
\pgfpathcurveto{\pgfqpoint{1.261856in}{0.711445in}}{\pgfqpoint{1.257466in}{0.700846in}}{\pgfqpoint{1.257466in}{0.689796in}}%
\pgfpathcurveto{\pgfqpoint{1.257466in}{0.678745in}}{\pgfqpoint{1.261856in}{0.668146in}}{\pgfqpoint{1.269670in}{0.660333in}}%
\pgfpathcurveto{\pgfqpoint{1.277484in}{0.652519in}}{\pgfqpoint{1.288083in}{0.648129in}}{\pgfqpoint{1.299133in}{0.648129in}}%
\pgfpathclose%
\pgfusepath{stroke,fill}%
\end{pgfscope}%
\begin{pgfscope}%
\pgfpathrectangle{\pgfqpoint{0.648703in}{0.548769in}}{\pgfqpoint{5.195150in}{3.102590in}}%
\pgfusepath{clip}%
\pgfsetbuttcap%
\pgfsetroundjoin%
\definecolor{currentfill}{rgb}{1.000000,0.498039,0.054902}%
\pgfsetfillcolor{currentfill}%
\pgfsetlinewidth{1.003750pt}%
\definecolor{currentstroke}{rgb}{1.000000,0.498039,0.054902}%
\pgfsetstrokecolor{currentstroke}%
\pgfsetdash{}{0pt}%
\pgfpathmoveto{\pgfqpoint{1.796276in}{3.185343in}}%
\pgfpathcurveto{\pgfqpoint{1.807326in}{3.185343in}}{\pgfqpoint{1.817926in}{3.189733in}}{\pgfqpoint{1.825739in}{3.197547in}}%
\pgfpathcurveto{\pgfqpoint{1.833553in}{3.205360in}}{\pgfqpoint{1.837943in}{3.215959in}}{\pgfqpoint{1.837943in}{3.227010in}}%
\pgfpathcurveto{\pgfqpoint{1.837943in}{3.238060in}}{\pgfqpoint{1.833553in}{3.248659in}}{\pgfqpoint{1.825739in}{3.256472in}}%
\pgfpathcurveto{\pgfqpoint{1.817926in}{3.264286in}}{\pgfqpoint{1.807326in}{3.268676in}}{\pgfqpoint{1.796276in}{3.268676in}}%
\pgfpathcurveto{\pgfqpoint{1.785226in}{3.268676in}}{\pgfqpoint{1.774627in}{3.264286in}}{\pgfqpoint{1.766814in}{3.256472in}}%
\pgfpathcurveto{\pgfqpoint{1.759000in}{3.248659in}}{\pgfqpoint{1.754610in}{3.238060in}}{\pgfqpoint{1.754610in}{3.227010in}}%
\pgfpathcurveto{\pgfqpoint{1.754610in}{3.215959in}}{\pgfqpoint{1.759000in}{3.205360in}}{\pgfqpoint{1.766814in}{3.197547in}}%
\pgfpathcurveto{\pgfqpoint{1.774627in}{3.189733in}}{\pgfqpoint{1.785226in}{3.185343in}}{\pgfqpoint{1.796276in}{3.185343in}}%
\pgfpathclose%
\pgfusepath{stroke,fill}%
\end{pgfscope}%
\begin{pgfscope}%
\pgfpathrectangle{\pgfqpoint{0.648703in}{0.548769in}}{\pgfqpoint{5.195150in}{3.102590in}}%
\pgfusepath{clip}%
\pgfsetbuttcap%
\pgfsetroundjoin%
\definecolor{currentfill}{rgb}{0.121569,0.466667,0.705882}%
\pgfsetfillcolor{currentfill}%
\pgfsetlinewidth{1.003750pt}%
\definecolor{currentstroke}{rgb}{0.121569,0.466667,0.705882}%
\pgfsetstrokecolor{currentstroke}%
\pgfsetdash{}{0pt}%
\pgfpathmoveto{\pgfqpoint{0.967704in}{0.939909in}}%
\pgfpathcurveto{\pgfqpoint{0.978754in}{0.939909in}}{\pgfqpoint{0.989353in}{0.944299in}}{\pgfqpoint{0.997167in}{0.952112in}}%
\pgfpathcurveto{\pgfqpoint{1.004980in}{0.959926in}}{\pgfqpoint{1.009370in}{0.970525in}}{\pgfqpoint{1.009370in}{0.981575in}}%
\pgfpathcurveto{\pgfqpoint{1.009370in}{0.992625in}}{\pgfqpoint{1.004980in}{1.003224in}}{\pgfqpoint{0.997167in}{1.011038in}}%
\pgfpathcurveto{\pgfqpoint{0.989353in}{1.018852in}}{\pgfqpoint{0.978754in}{1.023242in}}{\pgfqpoint{0.967704in}{1.023242in}}%
\pgfpathcurveto{\pgfqpoint{0.956654in}{1.023242in}}{\pgfqpoint{0.946055in}{1.018852in}}{\pgfqpoint{0.938241in}{1.011038in}}%
\pgfpathcurveto{\pgfqpoint{0.930427in}{1.003224in}}{\pgfqpoint{0.926037in}{0.992625in}}{\pgfqpoint{0.926037in}{0.981575in}}%
\pgfpathcurveto{\pgfqpoint{0.926037in}{0.970525in}}{\pgfqpoint{0.930427in}{0.959926in}}{\pgfqpoint{0.938241in}{0.952112in}}%
\pgfpathcurveto{\pgfqpoint{0.946055in}{0.944299in}}{\pgfqpoint{0.956654in}{0.939909in}}{\pgfqpoint{0.967704in}{0.939909in}}%
\pgfpathclose%
\pgfusepath{stroke,fill}%
\end{pgfscope}%
\begin{pgfscope}%
\pgfpathrectangle{\pgfqpoint{0.648703in}{0.548769in}}{\pgfqpoint{5.195150in}{3.102590in}}%
\pgfusepath{clip}%
\pgfsetbuttcap%
\pgfsetroundjoin%
\definecolor{currentfill}{rgb}{0.121569,0.466667,0.705882}%
\pgfsetfillcolor{currentfill}%
\pgfsetlinewidth{1.003750pt}%
\definecolor{currentstroke}{rgb}{0.121569,0.466667,0.705882}%
\pgfsetstrokecolor{currentstroke}%
\pgfsetdash{}{0pt}%
\pgfpathmoveto{\pgfqpoint{1.796276in}{0.648129in}}%
\pgfpathcurveto{\pgfqpoint{1.807326in}{0.648129in}}{\pgfqpoint{1.817926in}{0.652519in}}{\pgfqpoint{1.825739in}{0.660333in}}%
\pgfpathcurveto{\pgfqpoint{1.833553in}{0.668146in}}{\pgfqpoint{1.837943in}{0.678745in}}{\pgfqpoint{1.837943in}{0.689796in}}%
\pgfpathcurveto{\pgfqpoint{1.837943in}{0.700846in}}{\pgfqpoint{1.833553in}{0.711445in}}{\pgfqpoint{1.825739in}{0.719258in}}%
\pgfpathcurveto{\pgfqpoint{1.817926in}{0.727072in}}{\pgfqpoint{1.807326in}{0.731462in}}{\pgfqpoint{1.796276in}{0.731462in}}%
\pgfpathcurveto{\pgfqpoint{1.785226in}{0.731462in}}{\pgfqpoint{1.774627in}{0.727072in}}{\pgfqpoint{1.766814in}{0.719258in}}%
\pgfpathcurveto{\pgfqpoint{1.759000in}{0.711445in}}{\pgfqpoint{1.754610in}{0.700846in}}{\pgfqpoint{1.754610in}{0.689796in}}%
\pgfpathcurveto{\pgfqpoint{1.754610in}{0.678745in}}{\pgfqpoint{1.759000in}{0.668146in}}{\pgfqpoint{1.766814in}{0.660333in}}%
\pgfpathcurveto{\pgfqpoint{1.774627in}{0.652519in}}{\pgfqpoint{1.785226in}{0.648129in}}{\pgfqpoint{1.796276in}{0.648129in}}%
\pgfpathclose%
\pgfusepath{stroke,fill}%
\end{pgfscope}%
\begin{pgfscope}%
\pgfpathrectangle{\pgfqpoint{0.648703in}{0.548769in}}{\pgfqpoint{5.195150in}{3.102590in}}%
\pgfusepath{clip}%
\pgfsetbuttcap%
\pgfsetroundjoin%
\definecolor{currentfill}{rgb}{1.000000,0.498039,0.054902}%
\pgfsetfillcolor{currentfill}%
\pgfsetlinewidth{1.003750pt}%
\definecolor{currentstroke}{rgb}{1.000000,0.498039,0.054902}%
\pgfsetstrokecolor{currentstroke}%
\pgfsetdash{}{0pt}%
\pgfpathmoveto{\pgfqpoint{1.796276in}{3.185343in}}%
\pgfpathcurveto{\pgfqpoint{1.807326in}{3.185343in}}{\pgfqpoint{1.817926in}{3.189733in}}{\pgfqpoint{1.825739in}{3.197547in}}%
\pgfpathcurveto{\pgfqpoint{1.833553in}{3.205360in}}{\pgfqpoint{1.837943in}{3.215959in}}{\pgfqpoint{1.837943in}{3.227010in}}%
\pgfpathcurveto{\pgfqpoint{1.837943in}{3.238060in}}{\pgfqpoint{1.833553in}{3.248659in}}{\pgfqpoint{1.825739in}{3.256472in}}%
\pgfpathcurveto{\pgfqpoint{1.817926in}{3.264286in}}{\pgfqpoint{1.807326in}{3.268676in}}{\pgfqpoint{1.796276in}{3.268676in}}%
\pgfpathcurveto{\pgfqpoint{1.785226in}{3.268676in}}{\pgfqpoint{1.774627in}{3.264286in}}{\pgfqpoint{1.766814in}{3.256472in}}%
\pgfpathcurveto{\pgfqpoint{1.759000in}{3.248659in}}{\pgfqpoint{1.754610in}{3.238060in}}{\pgfqpoint{1.754610in}{3.227010in}}%
\pgfpathcurveto{\pgfqpoint{1.754610in}{3.215959in}}{\pgfqpoint{1.759000in}{3.205360in}}{\pgfqpoint{1.766814in}{3.197547in}}%
\pgfpathcurveto{\pgfqpoint{1.774627in}{3.189733in}}{\pgfqpoint{1.785226in}{3.185343in}}{\pgfqpoint{1.796276in}{3.185343in}}%
\pgfpathclose%
\pgfusepath{stroke,fill}%
\end{pgfscope}%
\begin{pgfscope}%
\pgfpathrectangle{\pgfqpoint{0.648703in}{0.548769in}}{\pgfqpoint{5.195150in}{3.102590in}}%
\pgfusepath{clip}%
\pgfsetbuttcap%
\pgfsetroundjoin%
\definecolor{currentfill}{rgb}{0.121569,0.466667,0.705882}%
\pgfsetfillcolor{currentfill}%
\pgfsetlinewidth{1.003750pt}%
\definecolor{currentstroke}{rgb}{0.121569,0.466667,0.705882}%
\pgfsetstrokecolor{currentstroke}%
\pgfsetdash{}{0pt}%
\pgfpathmoveto{\pgfqpoint{1.713419in}{0.648129in}}%
\pgfpathcurveto{\pgfqpoint{1.724469in}{0.648129in}}{\pgfqpoint{1.735068in}{0.652519in}}{\pgfqpoint{1.742882in}{0.660333in}}%
\pgfpathcurveto{\pgfqpoint{1.750695in}{0.668146in}}{\pgfqpoint{1.755086in}{0.678745in}}{\pgfqpoint{1.755086in}{0.689796in}}%
\pgfpathcurveto{\pgfqpoint{1.755086in}{0.700846in}}{\pgfqpoint{1.750695in}{0.711445in}}{\pgfqpoint{1.742882in}{0.719258in}}%
\pgfpathcurveto{\pgfqpoint{1.735068in}{0.727072in}}{\pgfqpoint{1.724469in}{0.731462in}}{\pgfqpoint{1.713419in}{0.731462in}}%
\pgfpathcurveto{\pgfqpoint{1.702369in}{0.731462in}}{\pgfqpoint{1.691770in}{0.727072in}}{\pgfqpoint{1.683956in}{0.719258in}}%
\pgfpathcurveto{\pgfqpoint{1.676143in}{0.711445in}}{\pgfqpoint{1.671752in}{0.700846in}}{\pgfqpoint{1.671752in}{0.689796in}}%
\pgfpathcurveto{\pgfqpoint{1.671752in}{0.678745in}}{\pgfqpoint{1.676143in}{0.668146in}}{\pgfqpoint{1.683956in}{0.660333in}}%
\pgfpathcurveto{\pgfqpoint{1.691770in}{0.652519in}}{\pgfqpoint{1.702369in}{0.648129in}}{\pgfqpoint{1.713419in}{0.648129in}}%
\pgfpathclose%
\pgfusepath{stroke,fill}%
\end{pgfscope}%
\begin{pgfscope}%
\pgfpathrectangle{\pgfqpoint{0.648703in}{0.548769in}}{\pgfqpoint{5.195150in}{3.102590in}}%
\pgfusepath{clip}%
\pgfsetbuttcap%
\pgfsetroundjoin%
\definecolor{currentfill}{rgb}{1.000000,0.498039,0.054902}%
\pgfsetfillcolor{currentfill}%
\pgfsetlinewidth{1.003750pt}%
\definecolor{currentstroke}{rgb}{1.000000,0.498039,0.054902}%
\pgfsetstrokecolor{currentstroke}%
\pgfsetdash{}{0pt}%
\pgfpathmoveto{\pgfqpoint{2.044848in}{3.278374in}}%
\pgfpathcurveto{\pgfqpoint{2.055898in}{3.278374in}}{\pgfqpoint{2.066497in}{3.282764in}}{\pgfqpoint{2.074311in}{3.290578in}}%
\pgfpathcurveto{\pgfqpoint{2.082125in}{3.298392in}}{\pgfqpoint{2.086515in}{3.308991in}}{\pgfqpoint{2.086515in}{3.320041in}}%
\pgfpathcurveto{\pgfqpoint{2.086515in}{3.331091in}}{\pgfqpoint{2.082125in}{3.341690in}}{\pgfqpoint{2.074311in}{3.349504in}}%
\pgfpathcurveto{\pgfqpoint{2.066497in}{3.357317in}}{\pgfqpoint{2.055898in}{3.361707in}}{\pgfqpoint{2.044848in}{3.361707in}}%
\pgfpathcurveto{\pgfqpoint{2.033798in}{3.361707in}}{\pgfqpoint{2.023199in}{3.357317in}}{\pgfqpoint{2.015385in}{3.349504in}}%
\pgfpathcurveto{\pgfqpoint{2.007572in}{3.341690in}}{\pgfqpoint{2.003181in}{3.331091in}}{\pgfqpoint{2.003181in}{3.320041in}}%
\pgfpathcurveto{\pgfqpoint{2.003181in}{3.308991in}}{\pgfqpoint{2.007572in}{3.298392in}}{\pgfqpoint{2.015385in}{3.290578in}}%
\pgfpathcurveto{\pgfqpoint{2.023199in}{3.282764in}}{\pgfqpoint{2.033798in}{3.278374in}}{\pgfqpoint{2.044848in}{3.278374in}}%
\pgfpathclose%
\pgfusepath{stroke,fill}%
\end{pgfscope}%
\begin{pgfscope}%
\pgfpathrectangle{\pgfqpoint{0.648703in}{0.548769in}}{\pgfqpoint{5.195150in}{3.102590in}}%
\pgfusepath{clip}%
\pgfsetbuttcap%
\pgfsetroundjoin%
\definecolor{currentfill}{rgb}{1.000000,0.498039,0.054902}%
\pgfsetfillcolor{currentfill}%
\pgfsetlinewidth{1.003750pt}%
\definecolor{currentstroke}{rgb}{1.000000,0.498039,0.054902}%
\pgfsetstrokecolor{currentstroke}%
\pgfsetdash{}{0pt}%
\pgfpathmoveto{\pgfqpoint{1.713419in}{3.202258in}}%
\pgfpathcurveto{\pgfqpoint{1.724469in}{3.202258in}}{\pgfqpoint{1.735068in}{3.206648in}}{\pgfqpoint{1.742882in}{3.214462in}}%
\pgfpathcurveto{\pgfqpoint{1.750695in}{3.222275in}}{\pgfqpoint{1.755086in}{3.232874in}}{\pgfqpoint{1.755086in}{3.243924in}}%
\pgfpathcurveto{\pgfqpoint{1.755086in}{3.254974in}}{\pgfqpoint{1.750695in}{3.265573in}}{\pgfqpoint{1.742882in}{3.273387in}}%
\pgfpathcurveto{\pgfqpoint{1.735068in}{3.281201in}}{\pgfqpoint{1.724469in}{3.285591in}}{\pgfqpoint{1.713419in}{3.285591in}}%
\pgfpathcurveto{\pgfqpoint{1.702369in}{3.285591in}}{\pgfqpoint{1.691770in}{3.281201in}}{\pgfqpoint{1.683956in}{3.273387in}}%
\pgfpathcurveto{\pgfqpoint{1.676143in}{3.265573in}}{\pgfqpoint{1.671752in}{3.254974in}}{\pgfqpoint{1.671752in}{3.243924in}}%
\pgfpathcurveto{\pgfqpoint{1.671752in}{3.232874in}}{\pgfqpoint{1.676143in}{3.222275in}}{\pgfqpoint{1.683956in}{3.214462in}}%
\pgfpathcurveto{\pgfqpoint{1.691770in}{3.206648in}}{\pgfqpoint{1.702369in}{3.202258in}}{\pgfqpoint{1.713419in}{3.202258in}}%
\pgfpathclose%
\pgfusepath{stroke,fill}%
\end{pgfscope}%
\begin{pgfscope}%
\pgfpathrectangle{\pgfqpoint{0.648703in}{0.548769in}}{\pgfqpoint{5.195150in}{3.102590in}}%
\pgfusepath{clip}%
\pgfsetbuttcap%
\pgfsetroundjoin%
\definecolor{currentfill}{rgb}{0.121569,0.466667,0.705882}%
\pgfsetfillcolor{currentfill}%
\pgfsetlinewidth{1.003750pt}%
\definecolor{currentstroke}{rgb}{0.121569,0.466667,0.705882}%
\pgfsetstrokecolor{currentstroke}%
\pgfsetdash{}{0pt}%
\pgfpathmoveto{\pgfqpoint{1.630562in}{0.648129in}}%
\pgfpathcurveto{\pgfqpoint{1.641612in}{0.648129in}}{\pgfqpoint{1.652211in}{0.652519in}}{\pgfqpoint{1.660025in}{0.660333in}}%
\pgfpathcurveto{\pgfqpoint{1.667838in}{0.668146in}}{\pgfqpoint{1.672229in}{0.678745in}}{\pgfqpoint{1.672229in}{0.689796in}}%
\pgfpathcurveto{\pgfqpoint{1.672229in}{0.700846in}}{\pgfqpoint{1.667838in}{0.711445in}}{\pgfqpoint{1.660025in}{0.719258in}}%
\pgfpathcurveto{\pgfqpoint{1.652211in}{0.727072in}}{\pgfqpoint{1.641612in}{0.731462in}}{\pgfqpoint{1.630562in}{0.731462in}}%
\pgfpathcurveto{\pgfqpoint{1.619512in}{0.731462in}}{\pgfqpoint{1.608913in}{0.727072in}}{\pgfqpoint{1.601099in}{0.719258in}}%
\pgfpathcurveto{\pgfqpoint{1.593285in}{0.711445in}}{\pgfqpoint{1.588895in}{0.700846in}}{\pgfqpoint{1.588895in}{0.689796in}}%
\pgfpathcurveto{\pgfqpoint{1.588895in}{0.678745in}}{\pgfqpoint{1.593285in}{0.668146in}}{\pgfqpoint{1.601099in}{0.660333in}}%
\pgfpathcurveto{\pgfqpoint{1.608913in}{0.652519in}}{\pgfqpoint{1.619512in}{0.648129in}}{\pgfqpoint{1.630562in}{0.648129in}}%
\pgfpathclose%
\pgfusepath{stroke,fill}%
\end{pgfscope}%
\begin{pgfscope}%
\pgfpathrectangle{\pgfqpoint{0.648703in}{0.548769in}}{\pgfqpoint{5.195150in}{3.102590in}}%
\pgfusepath{clip}%
\pgfsetbuttcap%
\pgfsetroundjoin%
\definecolor{currentfill}{rgb}{1.000000,0.498039,0.054902}%
\pgfsetfillcolor{currentfill}%
\pgfsetlinewidth{1.003750pt}%
\definecolor{currentstroke}{rgb}{1.000000,0.498039,0.054902}%
\pgfsetstrokecolor{currentstroke}%
\pgfsetdash{}{0pt}%
\pgfpathmoveto{\pgfqpoint{1.464847in}{3.202258in}}%
\pgfpathcurveto{\pgfqpoint{1.475897in}{3.202258in}}{\pgfqpoint{1.486497in}{3.206648in}}{\pgfqpoint{1.494310in}{3.214462in}}%
\pgfpathcurveto{\pgfqpoint{1.502124in}{3.222275in}}{\pgfqpoint{1.506514in}{3.232874in}}{\pgfqpoint{1.506514in}{3.243924in}}%
\pgfpathcurveto{\pgfqpoint{1.506514in}{3.254974in}}{\pgfqpoint{1.502124in}{3.265573in}}{\pgfqpoint{1.494310in}{3.273387in}}%
\pgfpathcurveto{\pgfqpoint{1.486497in}{3.281201in}}{\pgfqpoint{1.475897in}{3.285591in}}{\pgfqpoint{1.464847in}{3.285591in}}%
\pgfpathcurveto{\pgfqpoint{1.453797in}{3.285591in}}{\pgfqpoint{1.443198in}{3.281201in}}{\pgfqpoint{1.435385in}{3.273387in}}%
\pgfpathcurveto{\pgfqpoint{1.427571in}{3.265573in}}{\pgfqpoint{1.423181in}{3.254974in}}{\pgfqpoint{1.423181in}{3.243924in}}%
\pgfpathcurveto{\pgfqpoint{1.423181in}{3.232874in}}{\pgfqpoint{1.427571in}{3.222275in}}{\pgfqpoint{1.435385in}{3.214462in}}%
\pgfpathcurveto{\pgfqpoint{1.443198in}{3.206648in}}{\pgfqpoint{1.453797in}{3.202258in}}{\pgfqpoint{1.464847in}{3.202258in}}%
\pgfpathclose%
\pgfusepath{stroke,fill}%
\end{pgfscope}%
\begin{pgfscope}%
\pgfpathrectangle{\pgfqpoint{0.648703in}{0.548769in}}{\pgfqpoint{5.195150in}{3.102590in}}%
\pgfusepath{clip}%
\pgfsetbuttcap%
\pgfsetroundjoin%
\definecolor{currentfill}{rgb}{0.121569,0.466667,0.705882}%
\pgfsetfillcolor{currentfill}%
\pgfsetlinewidth{1.003750pt}%
\definecolor{currentstroke}{rgb}{0.121569,0.466667,0.705882}%
\pgfsetstrokecolor{currentstroke}%
\pgfsetdash{}{0pt}%
\pgfpathmoveto{\pgfqpoint{2.210563in}{1.726445in}}%
\pgfpathcurveto{\pgfqpoint{2.221613in}{1.726445in}}{\pgfqpoint{2.232212in}{1.730835in}}{\pgfqpoint{2.240025in}{1.738649in}}%
\pgfpathcurveto{\pgfqpoint{2.247839in}{1.746462in}}{\pgfqpoint{2.252229in}{1.757061in}}{\pgfqpoint{2.252229in}{1.768112in}}%
\pgfpathcurveto{\pgfqpoint{2.252229in}{1.779162in}}{\pgfqpoint{2.247839in}{1.789761in}}{\pgfqpoint{2.240025in}{1.797574in}}%
\pgfpathcurveto{\pgfqpoint{2.232212in}{1.805388in}}{\pgfqpoint{2.221613in}{1.809778in}}{\pgfqpoint{2.210563in}{1.809778in}}%
\pgfpathcurveto{\pgfqpoint{2.199512in}{1.809778in}}{\pgfqpoint{2.188913in}{1.805388in}}{\pgfqpoint{2.181100in}{1.797574in}}%
\pgfpathcurveto{\pgfqpoint{2.173286in}{1.789761in}}{\pgfqpoint{2.168896in}{1.779162in}}{\pgfqpoint{2.168896in}{1.768112in}}%
\pgfpathcurveto{\pgfqpoint{2.168896in}{1.757061in}}{\pgfqpoint{2.173286in}{1.746462in}}{\pgfqpoint{2.181100in}{1.738649in}}%
\pgfpathcurveto{\pgfqpoint{2.188913in}{1.730835in}}{\pgfqpoint{2.199512in}{1.726445in}}{\pgfqpoint{2.210563in}{1.726445in}}%
\pgfpathclose%
\pgfusepath{stroke,fill}%
\end{pgfscope}%
\begin{pgfscope}%
\pgfpathrectangle{\pgfqpoint{0.648703in}{0.548769in}}{\pgfqpoint{5.195150in}{3.102590in}}%
\pgfusepath{clip}%
\pgfsetbuttcap%
\pgfsetroundjoin%
\definecolor{currentfill}{rgb}{0.121569,0.466667,0.705882}%
\pgfsetfillcolor{currentfill}%
\pgfsetlinewidth{1.003750pt}%
\definecolor{currentstroke}{rgb}{0.121569,0.466667,0.705882}%
\pgfsetstrokecolor{currentstroke}%
\pgfsetdash{}{0pt}%
\pgfpathmoveto{\pgfqpoint{3.701993in}{0.918765in}}%
\pgfpathcurveto{\pgfqpoint{3.713043in}{0.918765in}}{\pgfqpoint{3.723642in}{0.923155in}}{\pgfqpoint{3.731456in}{0.930969in}}%
\pgfpathcurveto{\pgfqpoint{3.739270in}{0.938783in}}{\pgfqpoint{3.743660in}{0.949382in}}{\pgfqpoint{3.743660in}{0.960432in}}%
\pgfpathcurveto{\pgfqpoint{3.743660in}{0.971482in}}{\pgfqpoint{3.739270in}{0.982081in}}{\pgfqpoint{3.731456in}{0.989895in}}%
\pgfpathcurveto{\pgfqpoint{3.723642in}{0.997708in}}{\pgfqpoint{3.713043in}{1.002098in}}{\pgfqpoint{3.701993in}{1.002098in}}%
\pgfpathcurveto{\pgfqpoint{3.690943in}{1.002098in}}{\pgfqpoint{3.680344in}{0.997708in}}{\pgfqpoint{3.672530in}{0.989895in}}%
\pgfpathcurveto{\pgfqpoint{3.664717in}{0.982081in}}{\pgfqpoint{3.660326in}{0.971482in}}{\pgfqpoint{3.660326in}{0.960432in}}%
\pgfpathcurveto{\pgfqpoint{3.660326in}{0.949382in}}{\pgfqpoint{3.664717in}{0.938783in}}{\pgfqpoint{3.672530in}{0.930969in}}%
\pgfpathcurveto{\pgfqpoint{3.680344in}{0.923155in}}{\pgfqpoint{3.690943in}{0.918765in}}{\pgfqpoint{3.701993in}{0.918765in}}%
\pgfpathclose%
\pgfusepath{stroke,fill}%
\end{pgfscope}%
\begin{pgfscope}%
\pgfpathrectangle{\pgfqpoint{0.648703in}{0.548769in}}{\pgfqpoint{5.195150in}{3.102590in}}%
\pgfusepath{clip}%
\pgfsetbuttcap%
\pgfsetroundjoin%
\definecolor{currentfill}{rgb}{1.000000,0.498039,0.054902}%
\pgfsetfillcolor{currentfill}%
\pgfsetlinewidth{1.003750pt}%
\definecolor{currentstroke}{rgb}{1.000000,0.498039,0.054902}%
\pgfsetstrokecolor{currentstroke}%
\pgfsetdash{}{0pt}%
\pgfpathmoveto{\pgfqpoint{1.464847in}{3.185343in}}%
\pgfpathcurveto{\pgfqpoint{1.475897in}{3.185343in}}{\pgfqpoint{1.486497in}{3.189733in}}{\pgfqpoint{1.494310in}{3.197547in}}%
\pgfpathcurveto{\pgfqpoint{1.502124in}{3.205360in}}{\pgfqpoint{1.506514in}{3.215959in}}{\pgfqpoint{1.506514in}{3.227010in}}%
\pgfpathcurveto{\pgfqpoint{1.506514in}{3.238060in}}{\pgfqpoint{1.502124in}{3.248659in}}{\pgfqpoint{1.494310in}{3.256472in}}%
\pgfpathcurveto{\pgfqpoint{1.486497in}{3.264286in}}{\pgfqpoint{1.475897in}{3.268676in}}{\pgfqpoint{1.464847in}{3.268676in}}%
\pgfpathcurveto{\pgfqpoint{1.453797in}{3.268676in}}{\pgfqpoint{1.443198in}{3.264286in}}{\pgfqpoint{1.435385in}{3.256472in}}%
\pgfpathcurveto{\pgfqpoint{1.427571in}{3.248659in}}{\pgfqpoint{1.423181in}{3.238060in}}{\pgfqpoint{1.423181in}{3.227010in}}%
\pgfpathcurveto{\pgfqpoint{1.423181in}{3.215959in}}{\pgfqpoint{1.427571in}{3.205360in}}{\pgfqpoint{1.435385in}{3.197547in}}%
\pgfpathcurveto{\pgfqpoint{1.443198in}{3.189733in}}{\pgfqpoint{1.453797in}{3.185343in}}{\pgfqpoint{1.464847in}{3.185343in}}%
\pgfpathclose%
\pgfusepath{stroke,fill}%
\end{pgfscope}%
\begin{pgfscope}%
\pgfpathrectangle{\pgfqpoint{0.648703in}{0.548769in}}{\pgfqpoint{5.195150in}{3.102590in}}%
\pgfusepath{clip}%
\pgfsetbuttcap%
\pgfsetroundjoin%
\definecolor{currentfill}{rgb}{0.121569,0.466667,0.705882}%
\pgfsetfillcolor{currentfill}%
\pgfsetlinewidth{1.003750pt}%
\definecolor{currentstroke}{rgb}{0.121569,0.466667,0.705882}%
\pgfsetstrokecolor{currentstroke}%
\pgfsetdash{}{0pt}%
\pgfpathmoveto{\pgfqpoint{1.381990in}{0.648129in}}%
\pgfpathcurveto{\pgfqpoint{1.393040in}{0.648129in}}{\pgfqpoint{1.403639in}{0.652519in}}{\pgfqpoint{1.411453in}{0.660333in}}%
\pgfpathcurveto{\pgfqpoint{1.419266in}{0.668146in}}{\pgfqpoint{1.423657in}{0.678745in}}{\pgfqpoint{1.423657in}{0.689796in}}%
\pgfpathcurveto{\pgfqpoint{1.423657in}{0.700846in}}{\pgfqpoint{1.419266in}{0.711445in}}{\pgfqpoint{1.411453in}{0.719258in}}%
\pgfpathcurveto{\pgfqpoint{1.403639in}{0.727072in}}{\pgfqpoint{1.393040in}{0.731462in}}{\pgfqpoint{1.381990in}{0.731462in}}%
\pgfpathcurveto{\pgfqpoint{1.370940in}{0.731462in}}{\pgfqpoint{1.360341in}{0.727072in}}{\pgfqpoint{1.352527in}{0.719258in}}%
\pgfpathcurveto{\pgfqpoint{1.344714in}{0.711445in}}{\pgfqpoint{1.340323in}{0.700846in}}{\pgfqpoint{1.340323in}{0.689796in}}%
\pgfpathcurveto{\pgfqpoint{1.340323in}{0.678745in}}{\pgfqpoint{1.344714in}{0.668146in}}{\pgfqpoint{1.352527in}{0.660333in}}%
\pgfpathcurveto{\pgfqpoint{1.360341in}{0.652519in}}{\pgfqpoint{1.370940in}{0.648129in}}{\pgfqpoint{1.381990in}{0.648129in}}%
\pgfpathclose%
\pgfusepath{stroke,fill}%
\end{pgfscope}%
\begin{pgfscope}%
\pgfpathrectangle{\pgfqpoint{0.648703in}{0.548769in}}{\pgfqpoint{5.195150in}{3.102590in}}%
\pgfusepath{clip}%
\pgfsetbuttcap%
\pgfsetroundjoin%
\definecolor{currentfill}{rgb}{1.000000,0.498039,0.054902}%
\pgfsetfillcolor{currentfill}%
\pgfsetlinewidth{1.003750pt}%
\definecolor{currentstroke}{rgb}{1.000000,0.498039,0.054902}%
\pgfsetstrokecolor{currentstroke}%
\pgfsetdash{}{0pt}%
\pgfpathmoveto{\pgfqpoint{1.796276in}{3.206486in}}%
\pgfpathcurveto{\pgfqpoint{1.807326in}{3.206486in}}{\pgfqpoint{1.817926in}{3.210877in}}{\pgfqpoint{1.825739in}{3.218690in}}%
\pgfpathcurveto{\pgfqpoint{1.833553in}{3.226504in}}{\pgfqpoint{1.837943in}{3.237103in}}{\pgfqpoint{1.837943in}{3.248153in}}%
\pgfpathcurveto{\pgfqpoint{1.837943in}{3.259203in}}{\pgfqpoint{1.833553in}{3.269802in}}{\pgfqpoint{1.825739in}{3.277616in}}%
\pgfpathcurveto{\pgfqpoint{1.817926in}{3.285429in}}{\pgfqpoint{1.807326in}{3.289820in}}{\pgfqpoint{1.796276in}{3.289820in}}%
\pgfpathcurveto{\pgfqpoint{1.785226in}{3.289820in}}{\pgfqpoint{1.774627in}{3.285429in}}{\pgfqpoint{1.766814in}{3.277616in}}%
\pgfpathcurveto{\pgfqpoint{1.759000in}{3.269802in}}{\pgfqpoint{1.754610in}{3.259203in}}{\pgfqpoint{1.754610in}{3.248153in}}%
\pgfpathcurveto{\pgfqpoint{1.754610in}{3.237103in}}{\pgfqpoint{1.759000in}{3.226504in}}{\pgfqpoint{1.766814in}{3.218690in}}%
\pgfpathcurveto{\pgfqpoint{1.774627in}{3.210877in}}{\pgfqpoint{1.785226in}{3.206486in}}{\pgfqpoint{1.796276in}{3.206486in}}%
\pgfpathclose%
\pgfusepath{stroke,fill}%
\end{pgfscope}%
\begin{pgfscope}%
\pgfpathrectangle{\pgfqpoint{0.648703in}{0.548769in}}{\pgfqpoint{5.195150in}{3.102590in}}%
\pgfusepath{clip}%
\pgfsetbuttcap%
\pgfsetroundjoin%
\definecolor{currentfill}{rgb}{1.000000,0.498039,0.054902}%
\pgfsetfillcolor{currentfill}%
\pgfsetlinewidth{1.003750pt}%
\definecolor{currentstroke}{rgb}{1.000000,0.498039,0.054902}%
\pgfsetstrokecolor{currentstroke}%
\pgfsetdash{}{0pt}%
\pgfpathmoveto{\pgfqpoint{1.299133in}{3.185343in}}%
\pgfpathcurveto{\pgfqpoint{1.310183in}{3.185343in}}{\pgfqpoint{1.320782in}{3.189733in}}{\pgfqpoint{1.328596in}{3.197547in}}%
\pgfpathcurveto{\pgfqpoint{1.336409in}{3.205360in}}{\pgfqpoint{1.340800in}{3.215959in}}{\pgfqpoint{1.340800in}{3.227010in}}%
\pgfpathcurveto{\pgfqpoint{1.340800in}{3.238060in}}{\pgfqpoint{1.336409in}{3.248659in}}{\pgfqpoint{1.328596in}{3.256472in}}%
\pgfpathcurveto{\pgfqpoint{1.320782in}{3.264286in}}{\pgfqpoint{1.310183in}{3.268676in}}{\pgfqpoint{1.299133in}{3.268676in}}%
\pgfpathcurveto{\pgfqpoint{1.288083in}{3.268676in}}{\pgfqpoint{1.277484in}{3.264286in}}{\pgfqpoint{1.269670in}{3.256472in}}%
\pgfpathcurveto{\pgfqpoint{1.261856in}{3.248659in}}{\pgfqpoint{1.257466in}{3.238060in}}{\pgfqpoint{1.257466in}{3.227010in}}%
\pgfpathcurveto{\pgfqpoint{1.257466in}{3.215959in}}{\pgfqpoint{1.261856in}{3.205360in}}{\pgfqpoint{1.269670in}{3.197547in}}%
\pgfpathcurveto{\pgfqpoint{1.277484in}{3.189733in}}{\pgfqpoint{1.288083in}{3.185343in}}{\pgfqpoint{1.299133in}{3.185343in}}%
\pgfpathclose%
\pgfusepath{stroke,fill}%
\end{pgfscope}%
\begin{pgfscope}%
\pgfpathrectangle{\pgfqpoint{0.648703in}{0.548769in}}{\pgfqpoint{5.195150in}{3.102590in}}%
\pgfusepath{clip}%
\pgfsetbuttcap%
\pgfsetroundjoin%
\definecolor{currentfill}{rgb}{1.000000,0.498039,0.054902}%
\pgfsetfillcolor{currentfill}%
\pgfsetlinewidth{1.003750pt}%
\definecolor{currentstroke}{rgb}{1.000000,0.498039,0.054902}%
\pgfsetstrokecolor{currentstroke}%
\pgfsetdash{}{0pt}%
\pgfpathmoveto{\pgfqpoint{1.464847in}{3.193800in}}%
\pgfpathcurveto{\pgfqpoint{1.475897in}{3.193800in}}{\pgfqpoint{1.486497in}{3.198191in}}{\pgfqpoint{1.494310in}{3.206004in}}%
\pgfpathcurveto{\pgfqpoint{1.502124in}{3.213818in}}{\pgfqpoint{1.506514in}{3.224417in}}{\pgfqpoint{1.506514in}{3.235467in}}%
\pgfpathcurveto{\pgfqpoint{1.506514in}{3.246517in}}{\pgfqpoint{1.502124in}{3.257116in}}{\pgfqpoint{1.494310in}{3.264930in}}%
\pgfpathcurveto{\pgfqpoint{1.486497in}{3.272743in}}{\pgfqpoint{1.475897in}{3.277134in}}{\pgfqpoint{1.464847in}{3.277134in}}%
\pgfpathcurveto{\pgfqpoint{1.453797in}{3.277134in}}{\pgfqpoint{1.443198in}{3.272743in}}{\pgfqpoint{1.435385in}{3.264930in}}%
\pgfpathcurveto{\pgfqpoint{1.427571in}{3.257116in}}{\pgfqpoint{1.423181in}{3.246517in}}{\pgfqpoint{1.423181in}{3.235467in}}%
\pgfpathcurveto{\pgfqpoint{1.423181in}{3.224417in}}{\pgfqpoint{1.427571in}{3.213818in}}{\pgfqpoint{1.435385in}{3.206004in}}%
\pgfpathcurveto{\pgfqpoint{1.443198in}{3.198191in}}{\pgfqpoint{1.453797in}{3.193800in}}{\pgfqpoint{1.464847in}{3.193800in}}%
\pgfpathclose%
\pgfusepath{stroke,fill}%
\end{pgfscope}%
\begin{pgfscope}%
\pgfpathrectangle{\pgfqpoint{0.648703in}{0.548769in}}{\pgfqpoint{5.195150in}{3.102590in}}%
\pgfusepath{clip}%
\pgfsetbuttcap%
\pgfsetroundjoin%
\definecolor{currentfill}{rgb}{1.000000,0.498039,0.054902}%
\pgfsetfillcolor{currentfill}%
\pgfsetlinewidth{1.003750pt}%
\definecolor{currentstroke}{rgb}{1.000000,0.498039,0.054902}%
\pgfsetstrokecolor{currentstroke}%
\pgfsetdash{}{0pt}%
\pgfpathmoveto{\pgfqpoint{2.127705in}{3.198029in}}%
\pgfpathcurveto{\pgfqpoint{2.138755in}{3.198029in}}{\pgfqpoint{2.149355in}{3.202419in}}{\pgfqpoint{2.157168in}{3.210233in}}%
\pgfpathcurveto{\pgfqpoint{2.164982in}{3.218046in}}{\pgfqpoint{2.169372in}{3.228646in}}{\pgfqpoint{2.169372in}{3.239696in}}%
\pgfpathcurveto{\pgfqpoint{2.169372in}{3.250746in}}{\pgfqpoint{2.164982in}{3.261345in}}{\pgfqpoint{2.157168in}{3.269158in}}%
\pgfpathcurveto{\pgfqpoint{2.149355in}{3.276972in}}{\pgfqpoint{2.138755in}{3.281362in}}{\pgfqpoint{2.127705in}{3.281362in}}%
\pgfpathcurveto{\pgfqpoint{2.116655in}{3.281362in}}{\pgfqpoint{2.106056in}{3.276972in}}{\pgfqpoint{2.098243in}{3.269158in}}%
\pgfpathcurveto{\pgfqpoint{2.090429in}{3.261345in}}{\pgfqpoint{2.086039in}{3.250746in}}{\pgfqpoint{2.086039in}{3.239696in}}%
\pgfpathcurveto{\pgfqpoint{2.086039in}{3.228646in}}{\pgfqpoint{2.090429in}{3.218046in}}{\pgfqpoint{2.098243in}{3.210233in}}%
\pgfpathcurveto{\pgfqpoint{2.106056in}{3.202419in}}{\pgfqpoint{2.116655in}{3.198029in}}{\pgfqpoint{2.127705in}{3.198029in}}%
\pgfpathclose%
\pgfusepath{stroke,fill}%
\end{pgfscope}%
\begin{pgfscope}%
\pgfpathrectangle{\pgfqpoint{0.648703in}{0.548769in}}{\pgfqpoint{5.195150in}{3.102590in}}%
\pgfusepath{clip}%
\pgfsetbuttcap%
\pgfsetroundjoin%
\definecolor{currentfill}{rgb}{1.000000,0.498039,0.054902}%
\pgfsetfillcolor{currentfill}%
\pgfsetlinewidth{1.003750pt}%
\definecolor{currentstroke}{rgb}{1.000000,0.498039,0.054902}%
\pgfsetstrokecolor{currentstroke}%
\pgfsetdash{}{0pt}%
\pgfpathmoveto{\pgfqpoint{1.547705in}{3.202258in}}%
\pgfpathcurveto{\pgfqpoint{1.558755in}{3.202258in}}{\pgfqpoint{1.569354in}{3.206648in}}{\pgfqpoint{1.577167in}{3.214462in}}%
\pgfpathcurveto{\pgfqpoint{1.584981in}{3.222275in}}{\pgfqpoint{1.589371in}{3.232874in}}{\pgfqpoint{1.589371in}{3.243924in}}%
\pgfpathcurveto{\pgfqpoint{1.589371in}{3.254974in}}{\pgfqpoint{1.584981in}{3.265573in}}{\pgfqpoint{1.577167in}{3.273387in}}%
\pgfpathcurveto{\pgfqpoint{1.569354in}{3.281201in}}{\pgfqpoint{1.558755in}{3.285591in}}{\pgfqpoint{1.547705in}{3.285591in}}%
\pgfpathcurveto{\pgfqpoint{1.536654in}{3.285591in}}{\pgfqpoint{1.526055in}{3.281201in}}{\pgfqpoint{1.518242in}{3.273387in}}%
\pgfpathcurveto{\pgfqpoint{1.510428in}{3.265573in}}{\pgfqpoint{1.506038in}{3.254974in}}{\pgfqpoint{1.506038in}{3.243924in}}%
\pgfpathcurveto{\pgfqpoint{1.506038in}{3.232874in}}{\pgfqpoint{1.510428in}{3.222275in}}{\pgfqpoint{1.518242in}{3.214462in}}%
\pgfpathcurveto{\pgfqpoint{1.526055in}{3.206648in}}{\pgfqpoint{1.536654in}{3.202258in}}{\pgfqpoint{1.547705in}{3.202258in}}%
\pgfpathclose%
\pgfusepath{stroke,fill}%
\end{pgfscope}%
\begin{pgfscope}%
\pgfpathrectangle{\pgfqpoint{0.648703in}{0.548769in}}{\pgfqpoint{5.195150in}{3.102590in}}%
\pgfusepath{clip}%
\pgfsetbuttcap%
\pgfsetroundjoin%
\definecolor{currentfill}{rgb}{0.121569,0.466667,0.705882}%
\pgfsetfillcolor{currentfill}%
\pgfsetlinewidth{1.003750pt}%
\definecolor{currentstroke}{rgb}{0.121569,0.466667,0.705882}%
\pgfsetstrokecolor{currentstroke}%
\pgfsetdash{}{0pt}%
\pgfpathmoveto{\pgfqpoint{2.541992in}{3.181114in}}%
\pgfpathcurveto{\pgfqpoint{2.553042in}{3.181114in}}{\pgfqpoint{2.563641in}{3.185504in}}{\pgfqpoint{2.571454in}{3.193318in}}%
\pgfpathcurveto{\pgfqpoint{2.579268in}{3.201132in}}{\pgfqpoint{2.583658in}{3.211731in}}{\pgfqpoint{2.583658in}{3.222781in}}%
\pgfpathcurveto{\pgfqpoint{2.583658in}{3.233831in}}{\pgfqpoint{2.579268in}{3.244430in}}{\pgfqpoint{2.571454in}{3.252244in}}%
\pgfpathcurveto{\pgfqpoint{2.563641in}{3.260057in}}{\pgfqpoint{2.553042in}{3.264448in}}{\pgfqpoint{2.541992in}{3.264448in}}%
\pgfpathcurveto{\pgfqpoint{2.530941in}{3.264448in}}{\pgfqpoint{2.520342in}{3.260057in}}{\pgfqpoint{2.512529in}{3.252244in}}%
\pgfpathcurveto{\pgfqpoint{2.504715in}{3.244430in}}{\pgfqpoint{2.500325in}{3.233831in}}{\pgfqpoint{2.500325in}{3.222781in}}%
\pgfpathcurveto{\pgfqpoint{2.500325in}{3.211731in}}{\pgfqpoint{2.504715in}{3.201132in}}{\pgfqpoint{2.512529in}{3.193318in}}%
\pgfpathcurveto{\pgfqpoint{2.520342in}{3.185504in}}{\pgfqpoint{2.530941in}{3.181114in}}{\pgfqpoint{2.541992in}{3.181114in}}%
\pgfpathclose%
\pgfusepath{stroke,fill}%
\end{pgfscope}%
\begin{pgfscope}%
\pgfpathrectangle{\pgfqpoint{0.648703in}{0.548769in}}{\pgfqpoint{5.195150in}{3.102590in}}%
\pgfusepath{clip}%
\pgfsetbuttcap%
\pgfsetroundjoin%
\definecolor{currentfill}{rgb}{1.000000,0.498039,0.054902}%
\pgfsetfillcolor{currentfill}%
\pgfsetlinewidth{1.003750pt}%
\definecolor{currentstroke}{rgb}{1.000000,0.498039,0.054902}%
\pgfsetstrokecolor{currentstroke}%
\pgfsetdash{}{0pt}%
\pgfpathmoveto{\pgfqpoint{2.376277in}{3.244545in}}%
\pgfpathcurveto{\pgfqpoint{2.387327in}{3.244545in}}{\pgfqpoint{2.397926in}{3.248935in}}{\pgfqpoint{2.405740in}{3.256748in}}%
\pgfpathcurveto{\pgfqpoint{2.413554in}{3.264562in}}{\pgfqpoint{2.417944in}{3.275161in}}{\pgfqpoint{2.417944in}{3.286211in}}%
\pgfpathcurveto{\pgfqpoint{2.417944in}{3.297261in}}{\pgfqpoint{2.413554in}{3.307860in}}{\pgfqpoint{2.405740in}{3.315674in}}%
\pgfpathcurveto{\pgfqpoint{2.397926in}{3.323488in}}{\pgfqpoint{2.387327in}{3.327878in}}{\pgfqpoint{2.376277in}{3.327878in}}%
\pgfpathcurveto{\pgfqpoint{2.365227in}{3.327878in}}{\pgfqpoint{2.354628in}{3.323488in}}{\pgfqpoint{2.346814in}{3.315674in}}%
\pgfpathcurveto{\pgfqpoint{2.339001in}{3.307860in}}{\pgfqpoint{2.334610in}{3.297261in}}{\pgfqpoint{2.334610in}{3.286211in}}%
\pgfpathcurveto{\pgfqpoint{2.334610in}{3.275161in}}{\pgfqpoint{2.339001in}{3.264562in}}{\pgfqpoint{2.346814in}{3.256748in}}%
\pgfpathcurveto{\pgfqpoint{2.354628in}{3.248935in}}{\pgfqpoint{2.365227in}{3.244545in}}{\pgfqpoint{2.376277in}{3.244545in}}%
\pgfpathclose%
\pgfusepath{stroke,fill}%
\end{pgfscope}%
\begin{pgfscope}%
\pgfpathrectangle{\pgfqpoint{0.648703in}{0.548769in}}{\pgfqpoint{5.195150in}{3.102590in}}%
\pgfusepath{clip}%
\pgfsetbuttcap%
\pgfsetroundjoin%
\definecolor{currentfill}{rgb}{1.000000,0.498039,0.054902}%
\pgfsetfillcolor{currentfill}%
\pgfsetlinewidth{1.003750pt}%
\definecolor{currentstroke}{rgb}{1.000000,0.498039,0.054902}%
\pgfsetstrokecolor{currentstroke}%
\pgfsetdash{}{0pt}%
\pgfpathmoveto{\pgfqpoint{2.293420in}{3.189572in}}%
\pgfpathcurveto{\pgfqpoint{2.304470in}{3.189572in}}{\pgfqpoint{2.315069in}{3.193962in}}{\pgfqpoint{2.322883in}{3.201775in}}%
\pgfpathcurveto{\pgfqpoint{2.330696in}{3.209589in}}{\pgfqpoint{2.335087in}{3.220188in}}{\pgfqpoint{2.335087in}{3.231238in}}%
\pgfpathcurveto{\pgfqpoint{2.335087in}{3.242288in}}{\pgfqpoint{2.330696in}{3.252887in}}{\pgfqpoint{2.322883in}{3.260701in}}%
\pgfpathcurveto{\pgfqpoint{2.315069in}{3.268515in}}{\pgfqpoint{2.304470in}{3.272905in}}{\pgfqpoint{2.293420in}{3.272905in}}%
\pgfpathcurveto{\pgfqpoint{2.282370in}{3.272905in}}{\pgfqpoint{2.271771in}{3.268515in}}{\pgfqpoint{2.263957in}{3.260701in}}%
\pgfpathcurveto{\pgfqpoint{2.256143in}{3.252887in}}{\pgfqpoint{2.251753in}{3.242288in}}{\pgfqpoint{2.251753in}{3.231238in}}%
\pgfpathcurveto{\pgfqpoint{2.251753in}{3.220188in}}{\pgfqpoint{2.256143in}{3.209589in}}{\pgfqpoint{2.263957in}{3.201775in}}%
\pgfpathcurveto{\pgfqpoint{2.271771in}{3.193962in}}{\pgfqpoint{2.282370in}{3.189572in}}{\pgfqpoint{2.293420in}{3.189572in}}%
\pgfpathclose%
\pgfusepath{stroke,fill}%
\end{pgfscope}%
\begin{pgfscope}%
\pgfpathrectangle{\pgfqpoint{0.648703in}{0.548769in}}{\pgfqpoint{5.195150in}{3.102590in}}%
\pgfusepath{clip}%
\pgfsetbuttcap%
\pgfsetroundjoin%
\definecolor{currentfill}{rgb}{0.121569,0.466667,0.705882}%
\pgfsetfillcolor{currentfill}%
\pgfsetlinewidth{1.003750pt}%
\definecolor{currentstroke}{rgb}{0.121569,0.466667,0.705882}%
\pgfsetstrokecolor{currentstroke}%
\pgfsetdash{}{0pt}%
\pgfpathmoveto{\pgfqpoint{2.044848in}{0.656586in}}%
\pgfpathcurveto{\pgfqpoint{2.055898in}{0.656586in}}{\pgfqpoint{2.066497in}{0.660977in}}{\pgfqpoint{2.074311in}{0.668790in}}%
\pgfpathcurveto{\pgfqpoint{2.082125in}{0.676604in}}{\pgfqpoint{2.086515in}{0.687203in}}{\pgfqpoint{2.086515in}{0.698253in}}%
\pgfpathcurveto{\pgfqpoint{2.086515in}{0.709303in}}{\pgfqpoint{2.082125in}{0.719902in}}{\pgfqpoint{2.074311in}{0.727716in}}%
\pgfpathcurveto{\pgfqpoint{2.066497in}{0.735529in}}{\pgfqpoint{2.055898in}{0.739920in}}{\pgfqpoint{2.044848in}{0.739920in}}%
\pgfpathcurveto{\pgfqpoint{2.033798in}{0.739920in}}{\pgfqpoint{2.023199in}{0.735529in}}{\pgfqpoint{2.015385in}{0.727716in}}%
\pgfpathcurveto{\pgfqpoint{2.007572in}{0.719902in}}{\pgfqpoint{2.003181in}{0.709303in}}{\pgfqpoint{2.003181in}{0.698253in}}%
\pgfpathcurveto{\pgfqpoint{2.003181in}{0.687203in}}{\pgfqpoint{2.007572in}{0.676604in}}{\pgfqpoint{2.015385in}{0.668790in}}%
\pgfpathcurveto{\pgfqpoint{2.023199in}{0.660977in}}{\pgfqpoint{2.033798in}{0.656586in}}{\pgfqpoint{2.044848in}{0.656586in}}%
\pgfpathclose%
\pgfusepath{stroke,fill}%
\end{pgfscope}%
\begin{pgfscope}%
\pgfpathrectangle{\pgfqpoint{0.648703in}{0.548769in}}{\pgfqpoint{5.195150in}{3.102590in}}%
\pgfusepath{clip}%
\pgfsetbuttcap%
\pgfsetroundjoin%
\definecolor{currentfill}{rgb}{1.000000,0.498039,0.054902}%
\pgfsetfillcolor{currentfill}%
\pgfsetlinewidth{1.003750pt}%
\definecolor{currentstroke}{rgb}{1.000000,0.498039,0.054902}%
\pgfsetstrokecolor{currentstroke}%
\pgfsetdash{}{0pt}%
\pgfpathmoveto{\pgfqpoint{2.127705in}{3.210715in}}%
\pgfpathcurveto{\pgfqpoint{2.138755in}{3.210715in}}{\pgfqpoint{2.149355in}{3.215105in}}{\pgfqpoint{2.157168in}{3.222919in}}%
\pgfpathcurveto{\pgfqpoint{2.164982in}{3.230733in}}{\pgfqpoint{2.169372in}{3.241332in}}{\pgfqpoint{2.169372in}{3.252382in}}%
\pgfpathcurveto{\pgfqpoint{2.169372in}{3.263432in}}{\pgfqpoint{2.164982in}{3.274031in}}{\pgfqpoint{2.157168in}{3.281844in}}%
\pgfpathcurveto{\pgfqpoint{2.149355in}{3.289658in}}{\pgfqpoint{2.138755in}{3.294048in}}{\pgfqpoint{2.127705in}{3.294048in}}%
\pgfpathcurveto{\pgfqpoint{2.116655in}{3.294048in}}{\pgfqpoint{2.106056in}{3.289658in}}{\pgfqpoint{2.098243in}{3.281844in}}%
\pgfpathcurveto{\pgfqpoint{2.090429in}{3.274031in}}{\pgfqpoint{2.086039in}{3.263432in}}{\pgfqpoint{2.086039in}{3.252382in}}%
\pgfpathcurveto{\pgfqpoint{2.086039in}{3.241332in}}{\pgfqpoint{2.090429in}{3.230733in}}{\pgfqpoint{2.098243in}{3.222919in}}%
\pgfpathcurveto{\pgfqpoint{2.106056in}{3.215105in}}{\pgfqpoint{2.116655in}{3.210715in}}{\pgfqpoint{2.127705in}{3.210715in}}%
\pgfpathclose%
\pgfusepath{stroke,fill}%
\end{pgfscope}%
\begin{pgfscope}%
\pgfpathrectangle{\pgfqpoint{0.648703in}{0.548769in}}{\pgfqpoint{5.195150in}{3.102590in}}%
\pgfusepath{clip}%
\pgfsetbuttcap%
\pgfsetroundjoin%
\definecolor{currentfill}{rgb}{1.000000,0.498039,0.054902}%
\pgfsetfillcolor{currentfill}%
\pgfsetlinewidth{1.003750pt}%
\definecolor{currentstroke}{rgb}{1.000000,0.498039,0.054902}%
\pgfsetstrokecolor{currentstroke}%
\pgfsetdash{}{0pt}%
\pgfpathmoveto{\pgfqpoint{2.127705in}{3.214944in}}%
\pgfpathcurveto{\pgfqpoint{2.138755in}{3.214944in}}{\pgfqpoint{2.149355in}{3.219334in}}{\pgfqpoint{2.157168in}{3.227148in}}%
\pgfpathcurveto{\pgfqpoint{2.164982in}{3.234961in}}{\pgfqpoint{2.169372in}{3.245560in}}{\pgfqpoint{2.169372in}{3.256610in}}%
\pgfpathcurveto{\pgfqpoint{2.169372in}{3.267661in}}{\pgfqpoint{2.164982in}{3.278260in}}{\pgfqpoint{2.157168in}{3.286073in}}%
\pgfpathcurveto{\pgfqpoint{2.149355in}{3.293887in}}{\pgfqpoint{2.138755in}{3.298277in}}{\pgfqpoint{2.127705in}{3.298277in}}%
\pgfpathcurveto{\pgfqpoint{2.116655in}{3.298277in}}{\pgfqpoint{2.106056in}{3.293887in}}{\pgfqpoint{2.098243in}{3.286073in}}%
\pgfpathcurveto{\pgfqpoint{2.090429in}{3.278260in}}{\pgfqpoint{2.086039in}{3.267661in}}{\pgfqpoint{2.086039in}{3.256610in}}%
\pgfpathcurveto{\pgfqpoint{2.086039in}{3.245560in}}{\pgfqpoint{2.090429in}{3.234961in}}{\pgfqpoint{2.098243in}{3.227148in}}%
\pgfpathcurveto{\pgfqpoint{2.106056in}{3.219334in}}{\pgfqpoint{2.116655in}{3.214944in}}{\pgfqpoint{2.127705in}{3.214944in}}%
\pgfpathclose%
\pgfusepath{stroke,fill}%
\end{pgfscope}%
\begin{pgfscope}%
\pgfpathrectangle{\pgfqpoint{0.648703in}{0.548769in}}{\pgfqpoint{5.195150in}{3.102590in}}%
\pgfusepath{clip}%
\pgfsetbuttcap%
\pgfsetroundjoin%
\definecolor{currentfill}{rgb}{1.000000,0.498039,0.054902}%
\pgfsetfillcolor{currentfill}%
\pgfsetlinewidth{1.003750pt}%
\definecolor{currentstroke}{rgb}{1.000000,0.498039,0.054902}%
\pgfsetstrokecolor{currentstroke}%
\pgfsetdash{}{0pt}%
\pgfpathmoveto{\pgfqpoint{2.127705in}{3.202258in}}%
\pgfpathcurveto{\pgfqpoint{2.138755in}{3.202258in}}{\pgfqpoint{2.149355in}{3.206648in}}{\pgfqpoint{2.157168in}{3.214462in}}%
\pgfpathcurveto{\pgfqpoint{2.164982in}{3.222275in}}{\pgfqpoint{2.169372in}{3.232874in}}{\pgfqpoint{2.169372in}{3.243924in}}%
\pgfpathcurveto{\pgfqpoint{2.169372in}{3.254974in}}{\pgfqpoint{2.164982in}{3.265573in}}{\pgfqpoint{2.157168in}{3.273387in}}%
\pgfpathcurveto{\pgfqpoint{2.149355in}{3.281201in}}{\pgfqpoint{2.138755in}{3.285591in}}{\pgfqpoint{2.127705in}{3.285591in}}%
\pgfpathcurveto{\pgfqpoint{2.116655in}{3.285591in}}{\pgfqpoint{2.106056in}{3.281201in}}{\pgfqpoint{2.098243in}{3.273387in}}%
\pgfpathcurveto{\pgfqpoint{2.090429in}{3.265573in}}{\pgfqpoint{2.086039in}{3.254974in}}{\pgfqpoint{2.086039in}{3.243924in}}%
\pgfpathcurveto{\pgfqpoint{2.086039in}{3.232874in}}{\pgfqpoint{2.090429in}{3.222275in}}{\pgfqpoint{2.098243in}{3.214462in}}%
\pgfpathcurveto{\pgfqpoint{2.106056in}{3.206648in}}{\pgfqpoint{2.116655in}{3.202258in}}{\pgfqpoint{2.127705in}{3.202258in}}%
\pgfpathclose%
\pgfusepath{stroke,fill}%
\end{pgfscope}%
\begin{pgfscope}%
\pgfpathrectangle{\pgfqpoint{0.648703in}{0.548769in}}{\pgfqpoint{5.195150in}{3.102590in}}%
\pgfusepath{clip}%
\pgfsetbuttcap%
\pgfsetroundjoin%
\definecolor{currentfill}{rgb}{1.000000,0.498039,0.054902}%
\pgfsetfillcolor{currentfill}%
\pgfsetlinewidth{1.003750pt}%
\definecolor{currentstroke}{rgb}{1.000000,0.498039,0.054902}%
\pgfsetstrokecolor{currentstroke}%
\pgfsetdash{}{0pt}%
\pgfpathmoveto{\pgfqpoint{3.370564in}{3.189572in}}%
\pgfpathcurveto{\pgfqpoint{3.381614in}{3.189572in}}{\pgfqpoint{3.392213in}{3.193962in}}{\pgfqpoint{3.400027in}{3.201775in}}%
\pgfpathcurveto{\pgfqpoint{3.407841in}{3.209589in}}{\pgfqpoint{3.412231in}{3.220188in}}{\pgfqpoint{3.412231in}{3.231238in}}%
\pgfpathcurveto{\pgfqpoint{3.412231in}{3.242288in}}{\pgfqpoint{3.407841in}{3.252887in}}{\pgfqpoint{3.400027in}{3.260701in}}%
\pgfpathcurveto{\pgfqpoint{3.392213in}{3.268515in}}{\pgfqpoint{3.381614in}{3.272905in}}{\pgfqpoint{3.370564in}{3.272905in}}%
\pgfpathcurveto{\pgfqpoint{3.359514in}{3.272905in}}{\pgfqpoint{3.348915in}{3.268515in}}{\pgfqpoint{3.341101in}{3.260701in}}%
\pgfpathcurveto{\pgfqpoint{3.333288in}{3.252887in}}{\pgfqpoint{3.328897in}{3.242288in}}{\pgfqpoint{3.328897in}{3.231238in}}%
\pgfpathcurveto{\pgfqpoint{3.328897in}{3.220188in}}{\pgfqpoint{3.333288in}{3.209589in}}{\pgfqpoint{3.341101in}{3.201775in}}%
\pgfpathcurveto{\pgfqpoint{3.348915in}{3.193962in}}{\pgfqpoint{3.359514in}{3.189572in}}{\pgfqpoint{3.370564in}{3.189572in}}%
\pgfpathclose%
\pgfusepath{stroke,fill}%
\end{pgfscope}%
\begin{pgfscope}%
\pgfpathrectangle{\pgfqpoint{0.648703in}{0.548769in}}{\pgfqpoint{5.195150in}{3.102590in}}%
\pgfusepath{clip}%
\pgfsetbuttcap%
\pgfsetroundjoin%
\definecolor{currentfill}{rgb}{0.839216,0.152941,0.156863}%
\pgfsetfillcolor{currentfill}%
\pgfsetlinewidth{1.003750pt}%
\definecolor{currentstroke}{rgb}{0.839216,0.152941,0.156863}%
\pgfsetstrokecolor{currentstroke}%
\pgfsetdash{}{0pt}%
\pgfpathmoveto{\pgfqpoint{5.027709in}{3.181114in}}%
\pgfpathcurveto{\pgfqpoint{5.038759in}{3.181114in}}{\pgfqpoint{5.049358in}{3.185504in}}{\pgfqpoint{5.057172in}{3.193318in}}%
\pgfpathcurveto{\pgfqpoint{5.064986in}{3.201132in}}{\pgfqpoint{5.069376in}{3.211731in}}{\pgfqpoint{5.069376in}{3.222781in}}%
\pgfpathcurveto{\pgfqpoint{5.069376in}{3.233831in}}{\pgfqpoint{5.064986in}{3.244430in}}{\pgfqpoint{5.057172in}{3.252244in}}%
\pgfpathcurveto{\pgfqpoint{5.049358in}{3.260057in}}{\pgfqpoint{5.038759in}{3.264448in}}{\pgfqpoint{5.027709in}{3.264448in}}%
\pgfpathcurveto{\pgfqpoint{5.016659in}{3.264448in}}{\pgfqpoint{5.006060in}{3.260057in}}{\pgfqpoint{4.998246in}{3.252244in}}%
\pgfpathcurveto{\pgfqpoint{4.990433in}{3.244430in}}{\pgfqpoint{4.986042in}{3.233831in}}{\pgfqpoint{4.986042in}{3.222781in}}%
\pgfpathcurveto{\pgfqpoint{4.986042in}{3.211731in}}{\pgfqpoint{4.990433in}{3.201132in}}{\pgfqpoint{4.998246in}{3.193318in}}%
\pgfpathcurveto{\pgfqpoint{5.006060in}{3.185504in}}{\pgfqpoint{5.016659in}{3.181114in}}{\pgfqpoint{5.027709in}{3.181114in}}%
\pgfpathclose%
\pgfusepath{stroke,fill}%
\end{pgfscope}%
\begin{pgfscope}%
\pgfpathrectangle{\pgfqpoint{0.648703in}{0.548769in}}{\pgfqpoint{5.195150in}{3.102590in}}%
\pgfusepath{clip}%
\pgfsetbuttcap%
\pgfsetroundjoin%
\definecolor{currentfill}{rgb}{1.000000,0.498039,0.054902}%
\pgfsetfillcolor{currentfill}%
\pgfsetlinewidth{1.003750pt}%
\definecolor{currentstroke}{rgb}{1.000000,0.498039,0.054902}%
\pgfsetstrokecolor{currentstroke}%
\pgfsetdash{}{0pt}%
\pgfpathmoveto{\pgfqpoint{2.127705in}{3.244545in}}%
\pgfpathcurveto{\pgfqpoint{2.138755in}{3.244545in}}{\pgfqpoint{2.149355in}{3.248935in}}{\pgfqpoint{2.157168in}{3.256748in}}%
\pgfpathcurveto{\pgfqpoint{2.164982in}{3.264562in}}{\pgfqpoint{2.169372in}{3.275161in}}{\pgfqpoint{2.169372in}{3.286211in}}%
\pgfpathcurveto{\pgfqpoint{2.169372in}{3.297261in}}{\pgfqpoint{2.164982in}{3.307860in}}{\pgfqpoint{2.157168in}{3.315674in}}%
\pgfpathcurveto{\pgfqpoint{2.149355in}{3.323488in}}{\pgfqpoint{2.138755in}{3.327878in}}{\pgfqpoint{2.127705in}{3.327878in}}%
\pgfpathcurveto{\pgfqpoint{2.116655in}{3.327878in}}{\pgfqpoint{2.106056in}{3.323488in}}{\pgfqpoint{2.098243in}{3.315674in}}%
\pgfpathcurveto{\pgfqpoint{2.090429in}{3.307860in}}{\pgfqpoint{2.086039in}{3.297261in}}{\pgfqpoint{2.086039in}{3.286211in}}%
\pgfpathcurveto{\pgfqpoint{2.086039in}{3.275161in}}{\pgfqpoint{2.090429in}{3.264562in}}{\pgfqpoint{2.098243in}{3.256748in}}%
\pgfpathcurveto{\pgfqpoint{2.106056in}{3.248935in}}{\pgfqpoint{2.116655in}{3.244545in}}{\pgfqpoint{2.127705in}{3.244545in}}%
\pgfpathclose%
\pgfusepath{stroke,fill}%
\end{pgfscope}%
\begin{pgfscope}%
\pgfpathrectangle{\pgfqpoint{0.648703in}{0.548769in}}{\pgfqpoint{5.195150in}{3.102590in}}%
\pgfusepath{clip}%
\pgfsetbuttcap%
\pgfsetroundjoin%
\definecolor{currentfill}{rgb}{1.000000,0.498039,0.054902}%
\pgfsetfillcolor{currentfill}%
\pgfsetlinewidth{1.003750pt}%
\definecolor{currentstroke}{rgb}{1.000000,0.498039,0.054902}%
\pgfsetstrokecolor{currentstroke}%
\pgfsetdash{}{0pt}%
\pgfpathmoveto{\pgfqpoint{1.961991in}{3.202258in}}%
\pgfpathcurveto{\pgfqpoint{1.973041in}{3.202258in}}{\pgfqpoint{1.983640in}{3.206648in}}{\pgfqpoint{1.991454in}{3.214462in}}%
\pgfpathcurveto{\pgfqpoint{1.999267in}{3.222275in}}{\pgfqpoint{2.003658in}{3.232874in}}{\pgfqpoint{2.003658in}{3.243924in}}%
\pgfpathcurveto{\pgfqpoint{2.003658in}{3.254974in}}{\pgfqpoint{1.999267in}{3.265573in}}{\pgfqpoint{1.991454in}{3.273387in}}%
\pgfpathcurveto{\pgfqpoint{1.983640in}{3.281201in}}{\pgfqpoint{1.973041in}{3.285591in}}{\pgfqpoint{1.961991in}{3.285591in}}%
\pgfpathcurveto{\pgfqpoint{1.950941in}{3.285591in}}{\pgfqpoint{1.940342in}{3.281201in}}{\pgfqpoint{1.932528in}{3.273387in}}%
\pgfpathcurveto{\pgfqpoint{1.924714in}{3.265573in}}{\pgfqpoint{1.920324in}{3.254974in}}{\pgfqpoint{1.920324in}{3.243924in}}%
\pgfpathcurveto{\pgfqpoint{1.920324in}{3.232874in}}{\pgfqpoint{1.924714in}{3.222275in}}{\pgfqpoint{1.932528in}{3.214462in}}%
\pgfpathcurveto{\pgfqpoint{1.940342in}{3.206648in}}{\pgfqpoint{1.950941in}{3.202258in}}{\pgfqpoint{1.961991in}{3.202258in}}%
\pgfpathclose%
\pgfusepath{stroke,fill}%
\end{pgfscope}%
\begin{pgfscope}%
\pgfpathrectangle{\pgfqpoint{0.648703in}{0.548769in}}{\pgfqpoint{5.195150in}{3.102590in}}%
\pgfusepath{clip}%
\pgfsetbuttcap%
\pgfsetroundjoin%
\definecolor{currentfill}{rgb}{1.000000,0.498039,0.054902}%
\pgfsetfillcolor{currentfill}%
\pgfsetlinewidth{1.003750pt}%
\definecolor{currentstroke}{rgb}{1.000000,0.498039,0.054902}%
\pgfsetstrokecolor{currentstroke}%
\pgfsetdash{}{0pt}%
\pgfpathmoveto{\pgfqpoint{1.879134in}{3.185343in}}%
\pgfpathcurveto{\pgfqpoint{1.890184in}{3.185343in}}{\pgfqpoint{1.900783in}{3.189733in}}{\pgfqpoint{1.908596in}{3.197547in}}%
\pgfpathcurveto{\pgfqpoint{1.916410in}{3.205360in}}{\pgfqpoint{1.920800in}{3.215959in}}{\pgfqpoint{1.920800in}{3.227010in}}%
\pgfpathcurveto{\pgfqpoint{1.920800in}{3.238060in}}{\pgfqpoint{1.916410in}{3.248659in}}{\pgfqpoint{1.908596in}{3.256472in}}%
\pgfpathcurveto{\pgfqpoint{1.900783in}{3.264286in}}{\pgfqpoint{1.890184in}{3.268676in}}{\pgfqpoint{1.879134in}{3.268676in}}%
\pgfpathcurveto{\pgfqpoint{1.868083in}{3.268676in}}{\pgfqpoint{1.857484in}{3.264286in}}{\pgfqpoint{1.849671in}{3.256472in}}%
\pgfpathcurveto{\pgfqpoint{1.841857in}{3.248659in}}{\pgfqpoint{1.837467in}{3.238060in}}{\pgfqpoint{1.837467in}{3.227010in}}%
\pgfpathcurveto{\pgfqpoint{1.837467in}{3.215959in}}{\pgfqpoint{1.841857in}{3.205360in}}{\pgfqpoint{1.849671in}{3.197547in}}%
\pgfpathcurveto{\pgfqpoint{1.857484in}{3.189733in}}{\pgfqpoint{1.868083in}{3.185343in}}{\pgfqpoint{1.879134in}{3.185343in}}%
\pgfpathclose%
\pgfusepath{stroke,fill}%
\end{pgfscope}%
\begin{pgfscope}%
\pgfpathrectangle{\pgfqpoint{0.648703in}{0.548769in}}{\pgfqpoint{5.195150in}{3.102590in}}%
\pgfusepath{clip}%
\pgfsetbuttcap%
\pgfsetroundjoin%
\definecolor{currentfill}{rgb}{1.000000,0.498039,0.054902}%
\pgfsetfillcolor{currentfill}%
\pgfsetlinewidth{1.003750pt}%
\definecolor{currentstroke}{rgb}{1.000000,0.498039,0.054902}%
\pgfsetstrokecolor{currentstroke}%
\pgfsetdash{}{0pt}%
\pgfpathmoveto{\pgfqpoint{1.381990in}{3.312204in}}%
\pgfpathcurveto{\pgfqpoint{1.393040in}{3.312204in}}{\pgfqpoint{1.403639in}{3.316594in}}{\pgfqpoint{1.411453in}{3.324407in}}%
\pgfpathcurveto{\pgfqpoint{1.419266in}{3.332221in}}{\pgfqpoint{1.423657in}{3.342820in}}{\pgfqpoint{1.423657in}{3.353870in}}%
\pgfpathcurveto{\pgfqpoint{1.423657in}{3.364920in}}{\pgfqpoint{1.419266in}{3.375519in}}{\pgfqpoint{1.411453in}{3.383333in}}%
\pgfpathcurveto{\pgfqpoint{1.403639in}{3.391147in}}{\pgfqpoint{1.393040in}{3.395537in}}{\pgfqpoint{1.381990in}{3.395537in}}%
\pgfpathcurveto{\pgfqpoint{1.370940in}{3.395537in}}{\pgfqpoint{1.360341in}{3.391147in}}{\pgfqpoint{1.352527in}{3.383333in}}%
\pgfpathcurveto{\pgfqpoint{1.344714in}{3.375519in}}{\pgfqpoint{1.340323in}{3.364920in}}{\pgfqpoint{1.340323in}{3.353870in}}%
\pgfpathcurveto{\pgfqpoint{1.340323in}{3.342820in}}{\pgfqpoint{1.344714in}{3.332221in}}{\pgfqpoint{1.352527in}{3.324407in}}%
\pgfpathcurveto{\pgfqpoint{1.360341in}{3.316594in}}{\pgfqpoint{1.370940in}{3.312204in}}{\pgfqpoint{1.381990in}{3.312204in}}%
\pgfpathclose%
\pgfusepath{stroke,fill}%
\end{pgfscope}%
\begin{pgfscope}%
\pgfpathrectangle{\pgfqpoint{0.648703in}{0.548769in}}{\pgfqpoint{5.195150in}{3.102590in}}%
\pgfusepath{clip}%
\pgfsetbuttcap%
\pgfsetroundjoin%
\definecolor{currentfill}{rgb}{1.000000,0.498039,0.054902}%
\pgfsetfillcolor{currentfill}%
\pgfsetlinewidth{1.003750pt}%
\definecolor{currentstroke}{rgb}{1.000000,0.498039,0.054902}%
\pgfsetstrokecolor{currentstroke}%
\pgfsetdash{}{0pt}%
\pgfpathmoveto{\pgfqpoint{0.967704in}{3.210715in}}%
\pgfpathcurveto{\pgfqpoint{0.978754in}{3.210715in}}{\pgfqpoint{0.989353in}{3.215105in}}{\pgfqpoint{0.997167in}{3.222919in}}%
\pgfpathcurveto{\pgfqpoint{1.004980in}{3.230733in}}{\pgfqpoint{1.009370in}{3.241332in}}{\pgfqpoint{1.009370in}{3.252382in}}%
\pgfpathcurveto{\pgfqpoint{1.009370in}{3.263432in}}{\pgfqpoint{1.004980in}{3.274031in}}{\pgfqpoint{0.997167in}{3.281844in}}%
\pgfpathcurveto{\pgfqpoint{0.989353in}{3.289658in}}{\pgfqpoint{0.978754in}{3.294048in}}{\pgfqpoint{0.967704in}{3.294048in}}%
\pgfpathcurveto{\pgfqpoint{0.956654in}{3.294048in}}{\pgfqpoint{0.946055in}{3.289658in}}{\pgfqpoint{0.938241in}{3.281844in}}%
\pgfpathcurveto{\pgfqpoint{0.930427in}{3.274031in}}{\pgfqpoint{0.926037in}{3.263432in}}{\pgfqpoint{0.926037in}{3.252382in}}%
\pgfpathcurveto{\pgfqpoint{0.926037in}{3.241332in}}{\pgfqpoint{0.930427in}{3.230733in}}{\pgfqpoint{0.938241in}{3.222919in}}%
\pgfpathcurveto{\pgfqpoint{0.946055in}{3.215105in}}{\pgfqpoint{0.956654in}{3.210715in}}{\pgfqpoint{0.967704in}{3.210715in}}%
\pgfpathclose%
\pgfusepath{stroke,fill}%
\end{pgfscope}%
\begin{pgfscope}%
\pgfpathrectangle{\pgfqpoint{0.648703in}{0.548769in}}{\pgfqpoint{5.195150in}{3.102590in}}%
\pgfusepath{clip}%
\pgfsetbuttcap%
\pgfsetroundjoin%
\definecolor{currentfill}{rgb}{1.000000,0.498039,0.054902}%
\pgfsetfillcolor{currentfill}%
\pgfsetlinewidth{1.003750pt}%
\definecolor{currentstroke}{rgb}{1.000000,0.498039,0.054902}%
\pgfsetstrokecolor{currentstroke}%
\pgfsetdash{}{0pt}%
\pgfpathmoveto{\pgfqpoint{1.879134in}{3.236087in}}%
\pgfpathcurveto{\pgfqpoint{1.890184in}{3.236087in}}{\pgfqpoint{1.900783in}{3.240477in}}{\pgfqpoint{1.908596in}{3.248291in}}%
\pgfpathcurveto{\pgfqpoint{1.916410in}{3.256105in}}{\pgfqpoint{1.920800in}{3.266704in}}{\pgfqpoint{1.920800in}{3.277754in}}%
\pgfpathcurveto{\pgfqpoint{1.920800in}{3.288804in}}{\pgfqpoint{1.916410in}{3.299403in}}{\pgfqpoint{1.908596in}{3.307217in}}%
\pgfpathcurveto{\pgfqpoint{1.900783in}{3.315030in}}{\pgfqpoint{1.890184in}{3.319421in}}{\pgfqpoint{1.879134in}{3.319421in}}%
\pgfpathcurveto{\pgfqpoint{1.868083in}{3.319421in}}{\pgfqpoint{1.857484in}{3.315030in}}{\pgfqpoint{1.849671in}{3.307217in}}%
\pgfpathcurveto{\pgfqpoint{1.841857in}{3.299403in}}{\pgfqpoint{1.837467in}{3.288804in}}{\pgfqpoint{1.837467in}{3.277754in}}%
\pgfpathcurveto{\pgfqpoint{1.837467in}{3.266704in}}{\pgfqpoint{1.841857in}{3.256105in}}{\pgfqpoint{1.849671in}{3.248291in}}%
\pgfpathcurveto{\pgfqpoint{1.857484in}{3.240477in}}{\pgfqpoint{1.868083in}{3.236087in}}{\pgfqpoint{1.879134in}{3.236087in}}%
\pgfpathclose%
\pgfusepath{stroke,fill}%
\end{pgfscope}%
\begin{pgfscope}%
\pgfpathrectangle{\pgfqpoint{0.648703in}{0.548769in}}{\pgfqpoint{5.195150in}{3.102590in}}%
\pgfusepath{clip}%
\pgfsetbuttcap%
\pgfsetroundjoin%
\definecolor{currentfill}{rgb}{0.121569,0.466667,0.705882}%
\pgfsetfillcolor{currentfill}%
\pgfsetlinewidth{1.003750pt}%
\definecolor{currentstroke}{rgb}{0.121569,0.466667,0.705882}%
\pgfsetstrokecolor{currentstroke}%
\pgfsetdash{}{0pt}%
\pgfpathmoveto{\pgfqpoint{1.630562in}{2.808990in}}%
\pgfpathcurveto{\pgfqpoint{1.641612in}{2.808990in}}{\pgfqpoint{1.652211in}{2.813380in}}{\pgfqpoint{1.660025in}{2.821193in}}%
\pgfpathcurveto{\pgfqpoint{1.667838in}{2.829007in}}{\pgfqpoint{1.672229in}{2.839606in}}{\pgfqpoint{1.672229in}{2.850656in}}%
\pgfpathcurveto{\pgfqpoint{1.672229in}{2.861706in}}{\pgfqpoint{1.667838in}{2.872305in}}{\pgfqpoint{1.660025in}{2.880119in}}%
\pgfpathcurveto{\pgfqpoint{1.652211in}{2.887933in}}{\pgfqpoint{1.641612in}{2.892323in}}{\pgfqpoint{1.630562in}{2.892323in}}%
\pgfpathcurveto{\pgfqpoint{1.619512in}{2.892323in}}{\pgfqpoint{1.608913in}{2.887933in}}{\pgfqpoint{1.601099in}{2.880119in}}%
\pgfpathcurveto{\pgfqpoint{1.593285in}{2.872305in}}{\pgfqpoint{1.588895in}{2.861706in}}{\pgfqpoint{1.588895in}{2.850656in}}%
\pgfpathcurveto{\pgfqpoint{1.588895in}{2.839606in}}{\pgfqpoint{1.593285in}{2.829007in}}{\pgfqpoint{1.601099in}{2.821193in}}%
\pgfpathcurveto{\pgfqpoint{1.608913in}{2.813380in}}{\pgfqpoint{1.619512in}{2.808990in}}{\pgfqpoint{1.630562in}{2.808990in}}%
\pgfpathclose%
\pgfusepath{stroke,fill}%
\end{pgfscope}%
\begin{pgfscope}%
\pgfpathrectangle{\pgfqpoint{0.648703in}{0.548769in}}{\pgfqpoint{5.195150in}{3.102590in}}%
\pgfusepath{clip}%
\pgfsetbuttcap%
\pgfsetroundjoin%
\definecolor{currentfill}{rgb}{0.121569,0.466667,0.705882}%
\pgfsetfillcolor{currentfill}%
\pgfsetlinewidth{1.003750pt}%
\definecolor{currentstroke}{rgb}{0.121569,0.466667,0.705882}%
\pgfsetstrokecolor{currentstroke}%
\pgfsetdash{}{0pt}%
\pgfpathmoveto{\pgfqpoint{1.547705in}{0.648129in}}%
\pgfpathcurveto{\pgfqpoint{1.558755in}{0.648129in}}{\pgfqpoint{1.569354in}{0.652519in}}{\pgfqpoint{1.577167in}{0.660333in}}%
\pgfpathcurveto{\pgfqpoint{1.584981in}{0.668146in}}{\pgfqpoint{1.589371in}{0.678745in}}{\pgfqpoint{1.589371in}{0.689796in}}%
\pgfpathcurveto{\pgfqpoint{1.589371in}{0.700846in}}{\pgfqpoint{1.584981in}{0.711445in}}{\pgfqpoint{1.577167in}{0.719258in}}%
\pgfpathcurveto{\pgfqpoint{1.569354in}{0.727072in}}{\pgfqpoint{1.558755in}{0.731462in}}{\pgfqpoint{1.547705in}{0.731462in}}%
\pgfpathcurveto{\pgfqpoint{1.536654in}{0.731462in}}{\pgfqpoint{1.526055in}{0.727072in}}{\pgfqpoint{1.518242in}{0.719258in}}%
\pgfpathcurveto{\pgfqpoint{1.510428in}{0.711445in}}{\pgfqpoint{1.506038in}{0.700846in}}{\pgfqpoint{1.506038in}{0.689796in}}%
\pgfpathcurveto{\pgfqpoint{1.506038in}{0.678745in}}{\pgfqpoint{1.510428in}{0.668146in}}{\pgfqpoint{1.518242in}{0.660333in}}%
\pgfpathcurveto{\pgfqpoint{1.526055in}{0.652519in}}{\pgfqpoint{1.536654in}{0.648129in}}{\pgfqpoint{1.547705in}{0.648129in}}%
\pgfpathclose%
\pgfusepath{stroke,fill}%
\end{pgfscope}%
\begin{pgfscope}%
\pgfpathrectangle{\pgfqpoint{0.648703in}{0.548769in}}{\pgfqpoint{5.195150in}{3.102590in}}%
\pgfusepath{clip}%
\pgfsetbuttcap%
\pgfsetroundjoin%
\definecolor{currentfill}{rgb}{0.121569,0.466667,0.705882}%
\pgfsetfillcolor{currentfill}%
\pgfsetlinewidth{1.003750pt}%
\definecolor{currentstroke}{rgb}{0.121569,0.466667,0.705882}%
\pgfsetstrokecolor{currentstroke}%
\pgfsetdash{}{0pt}%
\pgfpathmoveto{\pgfqpoint{1.381990in}{0.648129in}}%
\pgfpathcurveto{\pgfqpoint{1.393040in}{0.648129in}}{\pgfqpoint{1.403639in}{0.652519in}}{\pgfqpoint{1.411453in}{0.660333in}}%
\pgfpathcurveto{\pgfqpoint{1.419266in}{0.668146in}}{\pgfqpoint{1.423657in}{0.678745in}}{\pgfqpoint{1.423657in}{0.689796in}}%
\pgfpathcurveto{\pgfqpoint{1.423657in}{0.700846in}}{\pgfqpoint{1.419266in}{0.711445in}}{\pgfqpoint{1.411453in}{0.719258in}}%
\pgfpathcurveto{\pgfqpoint{1.403639in}{0.727072in}}{\pgfqpoint{1.393040in}{0.731462in}}{\pgfqpoint{1.381990in}{0.731462in}}%
\pgfpathcurveto{\pgfqpoint{1.370940in}{0.731462in}}{\pgfqpoint{1.360341in}{0.727072in}}{\pgfqpoint{1.352527in}{0.719258in}}%
\pgfpathcurveto{\pgfqpoint{1.344714in}{0.711445in}}{\pgfqpoint{1.340323in}{0.700846in}}{\pgfqpoint{1.340323in}{0.689796in}}%
\pgfpathcurveto{\pgfqpoint{1.340323in}{0.678745in}}{\pgfqpoint{1.344714in}{0.668146in}}{\pgfqpoint{1.352527in}{0.660333in}}%
\pgfpathcurveto{\pgfqpoint{1.360341in}{0.652519in}}{\pgfqpoint{1.370940in}{0.648129in}}{\pgfqpoint{1.381990in}{0.648129in}}%
\pgfpathclose%
\pgfusepath{stroke,fill}%
\end{pgfscope}%
\begin{pgfscope}%
\pgfpathrectangle{\pgfqpoint{0.648703in}{0.548769in}}{\pgfqpoint{5.195150in}{3.102590in}}%
\pgfusepath{clip}%
\pgfsetbuttcap%
\pgfsetroundjoin%
\definecolor{currentfill}{rgb}{1.000000,0.498039,0.054902}%
\pgfsetfillcolor{currentfill}%
\pgfsetlinewidth{1.003750pt}%
\definecolor{currentstroke}{rgb}{1.000000,0.498039,0.054902}%
\pgfsetstrokecolor{currentstroke}%
\pgfsetdash{}{0pt}%
\pgfpathmoveto{\pgfqpoint{2.624849in}{3.185343in}}%
\pgfpathcurveto{\pgfqpoint{2.635899in}{3.185343in}}{\pgfqpoint{2.646498in}{3.189733in}}{\pgfqpoint{2.654312in}{3.197547in}}%
\pgfpathcurveto{\pgfqpoint{2.662125in}{3.205360in}}{\pgfqpoint{2.666516in}{3.215959in}}{\pgfqpoint{2.666516in}{3.227010in}}%
\pgfpathcurveto{\pgfqpoint{2.666516in}{3.238060in}}{\pgfqpoint{2.662125in}{3.248659in}}{\pgfqpoint{2.654312in}{3.256472in}}%
\pgfpathcurveto{\pgfqpoint{2.646498in}{3.264286in}}{\pgfqpoint{2.635899in}{3.268676in}}{\pgfqpoint{2.624849in}{3.268676in}}%
\pgfpathcurveto{\pgfqpoint{2.613799in}{3.268676in}}{\pgfqpoint{2.603200in}{3.264286in}}{\pgfqpoint{2.595386in}{3.256472in}}%
\pgfpathcurveto{\pgfqpoint{2.587572in}{3.248659in}}{\pgfqpoint{2.583182in}{3.238060in}}{\pgfqpoint{2.583182in}{3.227010in}}%
\pgfpathcurveto{\pgfqpoint{2.583182in}{3.215959in}}{\pgfqpoint{2.587572in}{3.205360in}}{\pgfqpoint{2.595386in}{3.197547in}}%
\pgfpathcurveto{\pgfqpoint{2.603200in}{3.189733in}}{\pgfqpoint{2.613799in}{3.185343in}}{\pgfqpoint{2.624849in}{3.185343in}}%
\pgfpathclose%
\pgfusepath{stroke,fill}%
\end{pgfscope}%
\begin{pgfscope}%
\pgfpathrectangle{\pgfqpoint{0.648703in}{0.548769in}}{\pgfqpoint{5.195150in}{3.102590in}}%
\pgfusepath{clip}%
\pgfsetbuttcap%
\pgfsetroundjoin%
\definecolor{currentfill}{rgb}{1.000000,0.498039,0.054902}%
\pgfsetfillcolor{currentfill}%
\pgfsetlinewidth{1.003750pt}%
\definecolor{currentstroke}{rgb}{1.000000,0.498039,0.054902}%
\pgfsetstrokecolor{currentstroke}%
\pgfsetdash{}{0pt}%
\pgfpathmoveto{\pgfqpoint{1.879134in}{3.405235in}}%
\pgfpathcurveto{\pgfqpoint{1.890184in}{3.405235in}}{\pgfqpoint{1.900783in}{3.409625in}}{\pgfqpoint{1.908596in}{3.417439in}}%
\pgfpathcurveto{\pgfqpoint{1.916410in}{3.425252in}}{\pgfqpoint{1.920800in}{3.435851in}}{\pgfqpoint{1.920800in}{3.446901in}}%
\pgfpathcurveto{\pgfqpoint{1.920800in}{3.457952in}}{\pgfqpoint{1.916410in}{3.468551in}}{\pgfqpoint{1.908596in}{3.476364in}}%
\pgfpathcurveto{\pgfqpoint{1.900783in}{3.484178in}}{\pgfqpoint{1.890184in}{3.488568in}}{\pgfqpoint{1.879134in}{3.488568in}}%
\pgfpathcurveto{\pgfqpoint{1.868083in}{3.488568in}}{\pgfqpoint{1.857484in}{3.484178in}}{\pgfqpoint{1.849671in}{3.476364in}}%
\pgfpathcurveto{\pgfqpoint{1.841857in}{3.468551in}}{\pgfqpoint{1.837467in}{3.457952in}}{\pgfqpoint{1.837467in}{3.446901in}}%
\pgfpathcurveto{\pgfqpoint{1.837467in}{3.435851in}}{\pgfqpoint{1.841857in}{3.425252in}}{\pgfqpoint{1.849671in}{3.417439in}}%
\pgfpathcurveto{\pgfqpoint{1.857484in}{3.409625in}}{\pgfqpoint{1.868083in}{3.405235in}}{\pgfqpoint{1.879134in}{3.405235in}}%
\pgfpathclose%
\pgfusepath{stroke,fill}%
\end{pgfscope}%
\begin{pgfscope}%
\pgfpathrectangle{\pgfqpoint{0.648703in}{0.548769in}}{\pgfqpoint{5.195150in}{3.102590in}}%
\pgfusepath{clip}%
\pgfsetbuttcap%
\pgfsetroundjoin%
\definecolor{currentfill}{rgb}{1.000000,0.498039,0.054902}%
\pgfsetfillcolor{currentfill}%
\pgfsetlinewidth{1.003750pt}%
\definecolor{currentstroke}{rgb}{1.000000,0.498039,0.054902}%
\pgfsetstrokecolor{currentstroke}%
\pgfsetdash{}{0pt}%
\pgfpathmoveto{\pgfqpoint{2.210563in}{3.193800in}}%
\pgfpathcurveto{\pgfqpoint{2.221613in}{3.193800in}}{\pgfqpoint{2.232212in}{3.198191in}}{\pgfqpoint{2.240025in}{3.206004in}}%
\pgfpathcurveto{\pgfqpoint{2.247839in}{3.213818in}}{\pgfqpoint{2.252229in}{3.224417in}}{\pgfqpoint{2.252229in}{3.235467in}}%
\pgfpathcurveto{\pgfqpoint{2.252229in}{3.246517in}}{\pgfqpoint{2.247839in}{3.257116in}}{\pgfqpoint{2.240025in}{3.264930in}}%
\pgfpathcurveto{\pgfqpoint{2.232212in}{3.272743in}}{\pgfqpoint{2.221613in}{3.277134in}}{\pgfqpoint{2.210563in}{3.277134in}}%
\pgfpathcurveto{\pgfqpoint{2.199512in}{3.277134in}}{\pgfqpoint{2.188913in}{3.272743in}}{\pgfqpoint{2.181100in}{3.264930in}}%
\pgfpathcurveto{\pgfqpoint{2.173286in}{3.257116in}}{\pgfqpoint{2.168896in}{3.246517in}}{\pgfqpoint{2.168896in}{3.235467in}}%
\pgfpathcurveto{\pgfqpoint{2.168896in}{3.224417in}}{\pgfqpoint{2.173286in}{3.213818in}}{\pgfqpoint{2.181100in}{3.206004in}}%
\pgfpathcurveto{\pgfqpoint{2.188913in}{3.198191in}}{\pgfqpoint{2.199512in}{3.193800in}}{\pgfqpoint{2.210563in}{3.193800in}}%
\pgfpathclose%
\pgfusepath{stroke,fill}%
\end{pgfscope}%
\begin{pgfscope}%
\pgfpathrectangle{\pgfqpoint{0.648703in}{0.548769in}}{\pgfqpoint{5.195150in}{3.102590in}}%
\pgfusepath{clip}%
\pgfsetbuttcap%
\pgfsetroundjoin%
\definecolor{currentfill}{rgb}{1.000000,0.498039,0.054902}%
\pgfsetfillcolor{currentfill}%
\pgfsetlinewidth{1.003750pt}%
\definecolor{currentstroke}{rgb}{1.000000,0.498039,0.054902}%
\pgfsetstrokecolor{currentstroke}%
\pgfsetdash{}{0pt}%
\pgfpathmoveto{\pgfqpoint{2.293420in}{3.193800in}}%
\pgfpathcurveto{\pgfqpoint{2.304470in}{3.193800in}}{\pgfqpoint{2.315069in}{3.198191in}}{\pgfqpoint{2.322883in}{3.206004in}}%
\pgfpathcurveto{\pgfqpoint{2.330696in}{3.213818in}}{\pgfqpoint{2.335087in}{3.224417in}}{\pgfqpoint{2.335087in}{3.235467in}}%
\pgfpathcurveto{\pgfqpoint{2.335087in}{3.246517in}}{\pgfqpoint{2.330696in}{3.257116in}}{\pgfqpoint{2.322883in}{3.264930in}}%
\pgfpathcurveto{\pgfqpoint{2.315069in}{3.272743in}}{\pgfqpoint{2.304470in}{3.277134in}}{\pgfqpoint{2.293420in}{3.277134in}}%
\pgfpathcurveto{\pgfqpoint{2.282370in}{3.277134in}}{\pgfqpoint{2.271771in}{3.272743in}}{\pgfqpoint{2.263957in}{3.264930in}}%
\pgfpathcurveto{\pgfqpoint{2.256143in}{3.257116in}}{\pgfqpoint{2.251753in}{3.246517in}}{\pgfqpoint{2.251753in}{3.235467in}}%
\pgfpathcurveto{\pgfqpoint{2.251753in}{3.224417in}}{\pgfqpoint{2.256143in}{3.213818in}}{\pgfqpoint{2.263957in}{3.206004in}}%
\pgfpathcurveto{\pgfqpoint{2.271771in}{3.198191in}}{\pgfqpoint{2.282370in}{3.193800in}}{\pgfqpoint{2.293420in}{3.193800in}}%
\pgfpathclose%
\pgfusepath{stroke,fill}%
\end{pgfscope}%
\begin{pgfscope}%
\pgfpathrectangle{\pgfqpoint{0.648703in}{0.548769in}}{\pgfqpoint{5.195150in}{3.102590in}}%
\pgfusepath{clip}%
\pgfsetbuttcap%
\pgfsetroundjoin%
\definecolor{currentfill}{rgb}{1.000000,0.498039,0.054902}%
\pgfsetfillcolor{currentfill}%
\pgfsetlinewidth{1.003750pt}%
\definecolor{currentstroke}{rgb}{1.000000,0.498039,0.054902}%
\pgfsetstrokecolor{currentstroke}%
\pgfsetdash{}{0pt}%
\pgfpathmoveto{\pgfqpoint{1.879134in}{3.358719in}}%
\pgfpathcurveto{\pgfqpoint{1.890184in}{3.358719in}}{\pgfqpoint{1.900783in}{3.363109in}}{\pgfqpoint{1.908596in}{3.370923in}}%
\pgfpathcurveto{\pgfqpoint{1.916410in}{3.378737in}}{\pgfqpoint{1.920800in}{3.389336in}}{\pgfqpoint{1.920800in}{3.400386in}}%
\pgfpathcurveto{\pgfqpoint{1.920800in}{3.411436in}}{\pgfqpoint{1.916410in}{3.422035in}}{\pgfqpoint{1.908596in}{3.429849in}}%
\pgfpathcurveto{\pgfqpoint{1.900783in}{3.437662in}}{\pgfqpoint{1.890184in}{3.442053in}}{\pgfqpoint{1.879134in}{3.442053in}}%
\pgfpathcurveto{\pgfqpoint{1.868083in}{3.442053in}}{\pgfqpoint{1.857484in}{3.437662in}}{\pgfqpoint{1.849671in}{3.429849in}}%
\pgfpathcurveto{\pgfqpoint{1.841857in}{3.422035in}}{\pgfqpoint{1.837467in}{3.411436in}}{\pgfqpoint{1.837467in}{3.400386in}}%
\pgfpathcurveto{\pgfqpoint{1.837467in}{3.389336in}}{\pgfqpoint{1.841857in}{3.378737in}}{\pgfqpoint{1.849671in}{3.370923in}}%
\pgfpathcurveto{\pgfqpoint{1.857484in}{3.363109in}}{\pgfqpoint{1.868083in}{3.358719in}}{\pgfqpoint{1.879134in}{3.358719in}}%
\pgfpathclose%
\pgfusepath{stroke,fill}%
\end{pgfscope}%
\begin{pgfscope}%
\pgfpathrectangle{\pgfqpoint{0.648703in}{0.548769in}}{\pgfqpoint{5.195150in}{3.102590in}}%
\pgfusepath{clip}%
\pgfsetbuttcap%
\pgfsetroundjoin%
\definecolor{currentfill}{rgb}{1.000000,0.498039,0.054902}%
\pgfsetfillcolor{currentfill}%
\pgfsetlinewidth{1.003750pt}%
\definecolor{currentstroke}{rgb}{1.000000,0.498039,0.054902}%
\pgfsetstrokecolor{currentstroke}%
\pgfsetdash{}{0pt}%
\pgfpathmoveto{\pgfqpoint{1.713419in}{3.362948in}}%
\pgfpathcurveto{\pgfqpoint{1.724469in}{3.362948in}}{\pgfqpoint{1.735068in}{3.367338in}}{\pgfqpoint{1.742882in}{3.375152in}}%
\pgfpathcurveto{\pgfqpoint{1.750695in}{3.382965in}}{\pgfqpoint{1.755086in}{3.393564in}}{\pgfqpoint{1.755086in}{3.404615in}}%
\pgfpathcurveto{\pgfqpoint{1.755086in}{3.415665in}}{\pgfqpoint{1.750695in}{3.426264in}}{\pgfqpoint{1.742882in}{3.434077in}}%
\pgfpathcurveto{\pgfqpoint{1.735068in}{3.441891in}}{\pgfqpoint{1.724469in}{3.446281in}}{\pgfqpoint{1.713419in}{3.446281in}}%
\pgfpathcurveto{\pgfqpoint{1.702369in}{3.446281in}}{\pgfqpoint{1.691770in}{3.441891in}}{\pgfqpoint{1.683956in}{3.434077in}}%
\pgfpathcurveto{\pgfqpoint{1.676143in}{3.426264in}}{\pgfqpoint{1.671752in}{3.415665in}}{\pgfqpoint{1.671752in}{3.404615in}}%
\pgfpathcurveto{\pgfqpoint{1.671752in}{3.393564in}}{\pgfqpoint{1.676143in}{3.382965in}}{\pgfqpoint{1.683956in}{3.375152in}}%
\pgfpathcurveto{\pgfqpoint{1.691770in}{3.367338in}}{\pgfqpoint{1.702369in}{3.362948in}}{\pgfqpoint{1.713419in}{3.362948in}}%
\pgfpathclose%
\pgfusepath{stroke,fill}%
\end{pgfscope}%
\begin{pgfscope}%
\pgfpathrectangle{\pgfqpoint{0.648703in}{0.548769in}}{\pgfqpoint{5.195150in}{3.102590in}}%
\pgfusepath{clip}%
\pgfsetbuttcap%
\pgfsetroundjoin%
\definecolor{currentfill}{rgb}{1.000000,0.498039,0.054902}%
\pgfsetfillcolor{currentfill}%
\pgfsetlinewidth{1.003750pt}%
\definecolor{currentstroke}{rgb}{1.000000,0.498039,0.054902}%
\pgfsetstrokecolor{currentstroke}%
\pgfsetdash{}{0pt}%
\pgfpathmoveto{\pgfqpoint{3.121992in}{3.257231in}}%
\pgfpathcurveto{\pgfqpoint{3.133042in}{3.257231in}}{\pgfqpoint{3.143642in}{3.261621in}}{\pgfqpoint{3.151455in}{3.269435in}}%
\pgfpathcurveto{\pgfqpoint{3.159269in}{3.277248in}}{\pgfqpoint{3.163659in}{3.287847in}}{\pgfqpoint{3.163659in}{3.298897in}}%
\pgfpathcurveto{\pgfqpoint{3.163659in}{3.309947in}}{\pgfqpoint{3.159269in}{3.320546in}}{\pgfqpoint{3.151455in}{3.328360in}}%
\pgfpathcurveto{\pgfqpoint{3.143642in}{3.336174in}}{\pgfqpoint{3.133042in}{3.340564in}}{\pgfqpoint{3.121992in}{3.340564in}}%
\pgfpathcurveto{\pgfqpoint{3.110942in}{3.340564in}}{\pgfqpoint{3.100343in}{3.336174in}}{\pgfqpoint{3.092530in}{3.328360in}}%
\pgfpathcurveto{\pgfqpoint{3.084716in}{3.320546in}}{\pgfqpoint{3.080326in}{3.309947in}}{\pgfqpoint{3.080326in}{3.298897in}}%
\pgfpathcurveto{\pgfqpoint{3.080326in}{3.287847in}}{\pgfqpoint{3.084716in}{3.277248in}}{\pgfqpoint{3.092530in}{3.269435in}}%
\pgfpathcurveto{\pgfqpoint{3.100343in}{3.261621in}}{\pgfqpoint{3.110942in}{3.257231in}}{\pgfqpoint{3.121992in}{3.257231in}}%
\pgfpathclose%
\pgfusepath{stroke,fill}%
\end{pgfscope}%
\begin{pgfscope}%
\pgfpathrectangle{\pgfqpoint{0.648703in}{0.548769in}}{\pgfqpoint{5.195150in}{3.102590in}}%
\pgfusepath{clip}%
\pgfsetbuttcap%
\pgfsetroundjoin%
\definecolor{currentfill}{rgb}{0.121569,0.466667,0.705882}%
\pgfsetfillcolor{currentfill}%
\pgfsetlinewidth{1.003750pt}%
\definecolor{currentstroke}{rgb}{0.121569,0.466667,0.705882}%
\pgfsetstrokecolor{currentstroke}%
\pgfsetdash{}{0pt}%
\pgfpathmoveto{\pgfqpoint{0.884847in}{0.648129in}}%
\pgfpathcurveto{\pgfqpoint{0.895897in}{0.648129in}}{\pgfqpoint{0.906496in}{0.652519in}}{\pgfqpoint{0.914309in}{0.660333in}}%
\pgfpathcurveto{\pgfqpoint{0.922123in}{0.668146in}}{\pgfqpoint{0.926513in}{0.678745in}}{\pgfqpoint{0.926513in}{0.689796in}}%
\pgfpathcurveto{\pgfqpoint{0.926513in}{0.700846in}}{\pgfqpoint{0.922123in}{0.711445in}}{\pgfqpoint{0.914309in}{0.719258in}}%
\pgfpathcurveto{\pgfqpoint{0.906496in}{0.727072in}}{\pgfqpoint{0.895897in}{0.731462in}}{\pgfqpoint{0.884847in}{0.731462in}}%
\pgfpathcurveto{\pgfqpoint{0.873796in}{0.731462in}}{\pgfqpoint{0.863197in}{0.727072in}}{\pgfqpoint{0.855384in}{0.719258in}}%
\pgfpathcurveto{\pgfqpoint{0.847570in}{0.711445in}}{\pgfqpoint{0.843180in}{0.700846in}}{\pgfqpoint{0.843180in}{0.689796in}}%
\pgfpathcurveto{\pgfqpoint{0.843180in}{0.678745in}}{\pgfqpoint{0.847570in}{0.668146in}}{\pgfqpoint{0.855384in}{0.660333in}}%
\pgfpathcurveto{\pgfqpoint{0.863197in}{0.652519in}}{\pgfqpoint{0.873796in}{0.648129in}}{\pgfqpoint{0.884847in}{0.648129in}}%
\pgfpathclose%
\pgfusepath{stroke,fill}%
\end{pgfscope}%
\begin{pgfscope}%
\pgfpathrectangle{\pgfqpoint{0.648703in}{0.548769in}}{\pgfqpoint{5.195150in}{3.102590in}}%
\pgfusepath{clip}%
\pgfsetbuttcap%
\pgfsetroundjoin%
\definecolor{currentfill}{rgb}{0.121569,0.466667,0.705882}%
\pgfsetfillcolor{currentfill}%
\pgfsetlinewidth{1.003750pt}%
\definecolor{currentstroke}{rgb}{0.121569,0.466667,0.705882}%
\pgfsetstrokecolor{currentstroke}%
\pgfsetdash{}{0pt}%
\pgfpathmoveto{\pgfqpoint{1.216276in}{0.796133in}}%
\pgfpathcurveto{\pgfqpoint{1.227326in}{0.796133in}}{\pgfqpoint{1.237925in}{0.800523in}}{\pgfqpoint{1.245738in}{0.808337in}}%
\pgfpathcurveto{\pgfqpoint{1.253552in}{0.816151in}}{\pgfqpoint{1.257942in}{0.826750in}}{\pgfqpoint{1.257942in}{0.837800in}}%
\pgfpathcurveto{\pgfqpoint{1.257942in}{0.848850in}}{\pgfqpoint{1.253552in}{0.859449in}}{\pgfqpoint{1.245738in}{0.867263in}}%
\pgfpathcurveto{\pgfqpoint{1.237925in}{0.875076in}}{\pgfqpoint{1.227326in}{0.879466in}}{\pgfqpoint{1.216276in}{0.879466in}}%
\pgfpathcurveto{\pgfqpoint{1.205225in}{0.879466in}}{\pgfqpoint{1.194626in}{0.875076in}}{\pgfqpoint{1.186813in}{0.867263in}}%
\pgfpathcurveto{\pgfqpoint{1.178999in}{0.859449in}}{\pgfqpoint{1.174609in}{0.848850in}}{\pgfqpoint{1.174609in}{0.837800in}}%
\pgfpathcurveto{\pgfqpoint{1.174609in}{0.826750in}}{\pgfqpoint{1.178999in}{0.816151in}}{\pgfqpoint{1.186813in}{0.808337in}}%
\pgfpathcurveto{\pgfqpoint{1.194626in}{0.800523in}}{\pgfqpoint{1.205225in}{0.796133in}}{\pgfqpoint{1.216276in}{0.796133in}}%
\pgfpathclose%
\pgfusepath{stroke,fill}%
\end{pgfscope}%
\begin{pgfscope}%
\pgfpathrectangle{\pgfqpoint{0.648703in}{0.548769in}}{\pgfqpoint{5.195150in}{3.102590in}}%
\pgfusepath{clip}%
\pgfsetbuttcap%
\pgfsetroundjoin%
\definecolor{currentfill}{rgb}{0.121569,0.466667,0.705882}%
\pgfsetfillcolor{currentfill}%
\pgfsetlinewidth{1.003750pt}%
\definecolor{currentstroke}{rgb}{0.121569,0.466667,0.705882}%
\pgfsetstrokecolor{currentstroke}%
\pgfsetdash{}{0pt}%
\pgfpathmoveto{\pgfqpoint{1.630562in}{3.155742in}}%
\pgfpathcurveto{\pgfqpoint{1.641612in}{3.155742in}}{\pgfqpoint{1.652211in}{3.160132in}}{\pgfqpoint{1.660025in}{3.167946in}}%
\pgfpathcurveto{\pgfqpoint{1.667838in}{3.175760in}}{\pgfqpoint{1.672229in}{3.186359in}}{\pgfqpoint{1.672229in}{3.197409in}}%
\pgfpathcurveto{\pgfqpoint{1.672229in}{3.208459in}}{\pgfqpoint{1.667838in}{3.219058in}}{\pgfqpoint{1.660025in}{3.226872in}}%
\pgfpathcurveto{\pgfqpoint{1.652211in}{3.234685in}}{\pgfqpoint{1.641612in}{3.239075in}}{\pgfqpoint{1.630562in}{3.239075in}}%
\pgfpathcurveto{\pgfqpoint{1.619512in}{3.239075in}}{\pgfqpoint{1.608913in}{3.234685in}}{\pgfqpoint{1.601099in}{3.226872in}}%
\pgfpathcurveto{\pgfqpoint{1.593285in}{3.219058in}}{\pgfqpoint{1.588895in}{3.208459in}}{\pgfqpoint{1.588895in}{3.197409in}}%
\pgfpathcurveto{\pgfqpoint{1.588895in}{3.186359in}}{\pgfqpoint{1.593285in}{3.175760in}}{\pgfqpoint{1.601099in}{3.167946in}}%
\pgfpathcurveto{\pgfqpoint{1.608913in}{3.160132in}}{\pgfqpoint{1.619512in}{3.155742in}}{\pgfqpoint{1.630562in}{3.155742in}}%
\pgfpathclose%
\pgfusepath{stroke,fill}%
\end{pgfscope}%
\begin{pgfscope}%
\pgfpathrectangle{\pgfqpoint{0.648703in}{0.548769in}}{\pgfqpoint{5.195150in}{3.102590in}}%
\pgfusepath{clip}%
\pgfsetbuttcap%
\pgfsetroundjoin%
\definecolor{currentfill}{rgb}{0.121569,0.466667,0.705882}%
\pgfsetfillcolor{currentfill}%
\pgfsetlinewidth{1.003750pt}%
\definecolor{currentstroke}{rgb}{0.121569,0.466667,0.705882}%
\pgfsetstrokecolor{currentstroke}%
\pgfsetdash{}{0pt}%
\pgfpathmoveto{\pgfqpoint{1.381990in}{0.648129in}}%
\pgfpathcurveto{\pgfqpoint{1.393040in}{0.648129in}}{\pgfqpoint{1.403639in}{0.652519in}}{\pgfqpoint{1.411453in}{0.660333in}}%
\pgfpathcurveto{\pgfqpoint{1.419266in}{0.668146in}}{\pgfqpoint{1.423657in}{0.678745in}}{\pgfqpoint{1.423657in}{0.689796in}}%
\pgfpathcurveto{\pgfqpoint{1.423657in}{0.700846in}}{\pgfqpoint{1.419266in}{0.711445in}}{\pgfqpoint{1.411453in}{0.719258in}}%
\pgfpathcurveto{\pgfqpoint{1.403639in}{0.727072in}}{\pgfqpoint{1.393040in}{0.731462in}}{\pgfqpoint{1.381990in}{0.731462in}}%
\pgfpathcurveto{\pgfqpoint{1.370940in}{0.731462in}}{\pgfqpoint{1.360341in}{0.727072in}}{\pgfqpoint{1.352527in}{0.719258in}}%
\pgfpathcurveto{\pgfqpoint{1.344714in}{0.711445in}}{\pgfqpoint{1.340323in}{0.700846in}}{\pgfqpoint{1.340323in}{0.689796in}}%
\pgfpathcurveto{\pgfqpoint{1.340323in}{0.678745in}}{\pgfqpoint{1.344714in}{0.668146in}}{\pgfqpoint{1.352527in}{0.660333in}}%
\pgfpathcurveto{\pgfqpoint{1.360341in}{0.652519in}}{\pgfqpoint{1.370940in}{0.648129in}}{\pgfqpoint{1.381990in}{0.648129in}}%
\pgfpathclose%
\pgfusepath{stroke,fill}%
\end{pgfscope}%
\begin{pgfscope}%
\pgfpathrectangle{\pgfqpoint{0.648703in}{0.548769in}}{\pgfqpoint{5.195150in}{3.102590in}}%
\pgfusepath{clip}%
\pgfsetbuttcap%
\pgfsetroundjoin%
\definecolor{currentfill}{rgb}{0.121569,0.466667,0.705882}%
\pgfsetfillcolor{currentfill}%
\pgfsetlinewidth{1.003750pt}%
\definecolor{currentstroke}{rgb}{0.121569,0.466667,0.705882}%
\pgfsetstrokecolor{currentstroke}%
\pgfsetdash{}{0pt}%
\pgfpathmoveto{\pgfqpoint{0.884847in}{0.648129in}}%
\pgfpathcurveto{\pgfqpoint{0.895897in}{0.648129in}}{\pgfqpoint{0.906496in}{0.652519in}}{\pgfqpoint{0.914309in}{0.660333in}}%
\pgfpathcurveto{\pgfqpoint{0.922123in}{0.668146in}}{\pgfqpoint{0.926513in}{0.678745in}}{\pgfqpoint{0.926513in}{0.689796in}}%
\pgfpathcurveto{\pgfqpoint{0.926513in}{0.700846in}}{\pgfqpoint{0.922123in}{0.711445in}}{\pgfqpoint{0.914309in}{0.719258in}}%
\pgfpathcurveto{\pgfqpoint{0.906496in}{0.727072in}}{\pgfqpoint{0.895897in}{0.731462in}}{\pgfqpoint{0.884847in}{0.731462in}}%
\pgfpathcurveto{\pgfqpoint{0.873796in}{0.731462in}}{\pgfqpoint{0.863197in}{0.727072in}}{\pgfqpoint{0.855384in}{0.719258in}}%
\pgfpathcurveto{\pgfqpoint{0.847570in}{0.711445in}}{\pgfqpoint{0.843180in}{0.700846in}}{\pgfqpoint{0.843180in}{0.689796in}}%
\pgfpathcurveto{\pgfqpoint{0.843180in}{0.678745in}}{\pgfqpoint{0.847570in}{0.668146in}}{\pgfqpoint{0.855384in}{0.660333in}}%
\pgfpathcurveto{\pgfqpoint{0.863197in}{0.652519in}}{\pgfqpoint{0.873796in}{0.648129in}}{\pgfqpoint{0.884847in}{0.648129in}}%
\pgfpathclose%
\pgfusepath{stroke,fill}%
\end{pgfscope}%
\begin{pgfscope}%
\pgfpathrectangle{\pgfqpoint{0.648703in}{0.548769in}}{\pgfqpoint{5.195150in}{3.102590in}}%
\pgfusepath{clip}%
\pgfsetbuttcap%
\pgfsetroundjoin%
\definecolor{currentfill}{rgb}{0.121569,0.466667,0.705882}%
\pgfsetfillcolor{currentfill}%
\pgfsetlinewidth{1.003750pt}%
\definecolor{currentstroke}{rgb}{0.121569,0.466667,0.705882}%
\pgfsetstrokecolor{currentstroke}%
\pgfsetdash{}{0pt}%
\pgfpathmoveto{\pgfqpoint{0.967704in}{2.512981in}}%
\pgfpathcurveto{\pgfqpoint{0.978754in}{2.512981in}}{\pgfqpoint{0.989353in}{2.517371in}}{\pgfqpoint{0.997167in}{2.525185in}}%
\pgfpathcurveto{\pgfqpoint{1.004980in}{2.532999in}}{\pgfqpoint{1.009370in}{2.543598in}}{\pgfqpoint{1.009370in}{2.554648in}}%
\pgfpathcurveto{\pgfqpoint{1.009370in}{2.565698in}}{\pgfqpoint{1.004980in}{2.576297in}}{\pgfqpoint{0.997167in}{2.584111in}}%
\pgfpathcurveto{\pgfqpoint{0.989353in}{2.591924in}}{\pgfqpoint{0.978754in}{2.596315in}}{\pgfqpoint{0.967704in}{2.596315in}}%
\pgfpathcurveto{\pgfqpoint{0.956654in}{2.596315in}}{\pgfqpoint{0.946055in}{2.591924in}}{\pgfqpoint{0.938241in}{2.584111in}}%
\pgfpathcurveto{\pgfqpoint{0.930427in}{2.576297in}}{\pgfqpoint{0.926037in}{2.565698in}}{\pgfqpoint{0.926037in}{2.554648in}}%
\pgfpathcurveto{\pgfqpoint{0.926037in}{2.543598in}}{\pgfqpoint{0.930427in}{2.532999in}}{\pgfqpoint{0.938241in}{2.525185in}}%
\pgfpathcurveto{\pgfqpoint{0.946055in}{2.517371in}}{\pgfqpoint{0.956654in}{2.512981in}}{\pgfqpoint{0.967704in}{2.512981in}}%
\pgfpathclose%
\pgfusepath{stroke,fill}%
\end{pgfscope}%
\begin{pgfscope}%
\pgfpathrectangle{\pgfqpoint{0.648703in}{0.548769in}}{\pgfqpoint{5.195150in}{3.102590in}}%
\pgfusepath{clip}%
\pgfsetbuttcap%
\pgfsetroundjoin%
\definecolor{currentfill}{rgb}{1.000000,0.498039,0.054902}%
\pgfsetfillcolor{currentfill}%
\pgfsetlinewidth{1.003750pt}%
\definecolor{currentstroke}{rgb}{1.000000,0.498039,0.054902}%
\pgfsetstrokecolor{currentstroke}%
\pgfsetdash{}{0pt}%
\pgfpathmoveto{\pgfqpoint{2.210563in}{3.189572in}}%
\pgfpathcurveto{\pgfqpoint{2.221613in}{3.189572in}}{\pgfqpoint{2.232212in}{3.193962in}}{\pgfqpoint{2.240025in}{3.201775in}}%
\pgfpathcurveto{\pgfqpoint{2.247839in}{3.209589in}}{\pgfqpoint{2.252229in}{3.220188in}}{\pgfqpoint{2.252229in}{3.231238in}}%
\pgfpathcurveto{\pgfqpoint{2.252229in}{3.242288in}}{\pgfqpoint{2.247839in}{3.252887in}}{\pgfqpoint{2.240025in}{3.260701in}}%
\pgfpathcurveto{\pgfqpoint{2.232212in}{3.268515in}}{\pgfqpoint{2.221613in}{3.272905in}}{\pgfqpoint{2.210563in}{3.272905in}}%
\pgfpathcurveto{\pgfqpoint{2.199512in}{3.272905in}}{\pgfqpoint{2.188913in}{3.268515in}}{\pgfqpoint{2.181100in}{3.260701in}}%
\pgfpathcurveto{\pgfqpoint{2.173286in}{3.252887in}}{\pgfqpoint{2.168896in}{3.242288in}}{\pgfqpoint{2.168896in}{3.231238in}}%
\pgfpathcurveto{\pgfqpoint{2.168896in}{3.220188in}}{\pgfqpoint{2.173286in}{3.209589in}}{\pgfqpoint{2.181100in}{3.201775in}}%
\pgfpathcurveto{\pgfqpoint{2.188913in}{3.193962in}}{\pgfqpoint{2.199512in}{3.189572in}}{\pgfqpoint{2.210563in}{3.189572in}}%
\pgfpathclose%
\pgfusepath{stroke,fill}%
\end{pgfscope}%
\begin{pgfscope}%
\pgfpathrectangle{\pgfqpoint{0.648703in}{0.548769in}}{\pgfqpoint{5.195150in}{3.102590in}}%
\pgfusepath{clip}%
\pgfsetbuttcap%
\pgfsetroundjoin%
\definecolor{currentfill}{rgb}{1.000000,0.498039,0.054902}%
\pgfsetfillcolor{currentfill}%
\pgfsetlinewidth{1.003750pt}%
\definecolor{currentstroke}{rgb}{1.000000,0.498039,0.054902}%
\pgfsetstrokecolor{currentstroke}%
\pgfsetdash{}{0pt}%
\pgfpathmoveto{\pgfqpoint{2.044848in}{3.206486in}}%
\pgfpathcurveto{\pgfqpoint{2.055898in}{3.206486in}}{\pgfqpoint{2.066497in}{3.210877in}}{\pgfqpoint{2.074311in}{3.218690in}}%
\pgfpathcurveto{\pgfqpoint{2.082125in}{3.226504in}}{\pgfqpoint{2.086515in}{3.237103in}}{\pgfqpoint{2.086515in}{3.248153in}}%
\pgfpathcurveto{\pgfqpoint{2.086515in}{3.259203in}}{\pgfqpoint{2.082125in}{3.269802in}}{\pgfqpoint{2.074311in}{3.277616in}}%
\pgfpathcurveto{\pgfqpoint{2.066497in}{3.285429in}}{\pgfqpoint{2.055898in}{3.289820in}}{\pgfqpoint{2.044848in}{3.289820in}}%
\pgfpathcurveto{\pgfqpoint{2.033798in}{3.289820in}}{\pgfqpoint{2.023199in}{3.285429in}}{\pgfqpoint{2.015385in}{3.277616in}}%
\pgfpathcurveto{\pgfqpoint{2.007572in}{3.269802in}}{\pgfqpoint{2.003181in}{3.259203in}}{\pgfqpoint{2.003181in}{3.248153in}}%
\pgfpathcurveto{\pgfqpoint{2.003181in}{3.237103in}}{\pgfqpoint{2.007572in}{3.226504in}}{\pgfqpoint{2.015385in}{3.218690in}}%
\pgfpathcurveto{\pgfqpoint{2.023199in}{3.210877in}}{\pgfqpoint{2.033798in}{3.206486in}}{\pgfqpoint{2.044848in}{3.206486in}}%
\pgfpathclose%
\pgfusepath{stroke,fill}%
\end{pgfscope}%
\begin{pgfscope}%
\pgfsetbuttcap%
\pgfsetroundjoin%
\definecolor{currentfill}{rgb}{0.000000,0.000000,0.000000}%
\pgfsetfillcolor{currentfill}%
\pgfsetlinewidth{0.803000pt}%
\definecolor{currentstroke}{rgb}{0.000000,0.000000,0.000000}%
\pgfsetstrokecolor{currentstroke}%
\pgfsetdash{}{0pt}%
\pgfsys@defobject{currentmarker}{\pgfqpoint{0.000000in}{-0.048611in}}{\pgfqpoint{0.000000in}{0.000000in}}{%
\pgfpathmoveto{\pgfqpoint{0.000000in}{0.000000in}}%
\pgfpathlineto{\pgfqpoint{0.000000in}{-0.048611in}}%
\pgfusepath{stroke,fill}%
}%
\begin{pgfscope}%
\pgfsys@transformshift{0.801989in}{0.548769in}%
\pgfsys@useobject{currentmarker}{}%
\end{pgfscope}%
\end{pgfscope}%
\begin{pgfscope}%
\definecolor{textcolor}{rgb}{0.000000,0.000000,0.000000}%
\pgfsetstrokecolor{textcolor}%
\pgfsetfillcolor{textcolor}%
\pgftext[x=0.801989in,y=0.451547in,,top]{\color{textcolor}\sffamily\fontsize{10.000000}{12.000000}\selectfont \(\displaystyle {0}\)}%
\end{pgfscope}%
\begin{pgfscope}%
\pgfsetbuttcap%
\pgfsetroundjoin%
\definecolor{currentfill}{rgb}{0.000000,0.000000,0.000000}%
\pgfsetfillcolor{currentfill}%
\pgfsetlinewidth{0.803000pt}%
\definecolor{currentstroke}{rgb}{0.000000,0.000000,0.000000}%
\pgfsetstrokecolor{currentstroke}%
\pgfsetdash{}{0pt}%
\pgfsys@defobject{currentmarker}{\pgfqpoint{0.000000in}{-0.048611in}}{\pgfqpoint{0.000000in}{0.000000in}}{%
\pgfpathmoveto{\pgfqpoint{0.000000in}{0.000000in}}%
\pgfpathlineto{\pgfqpoint{0.000000in}{-0.048611in}}%
\pgfusepath{stroke,fill}%
}%
\begin{pgfscope}%
\pgfsys@transformshift{1.630562in}{0.548769in}%
\pgfsys@useobject{currentmarker}{}%
\end{pgfscope}%
\end{pgfscope}%
\begin{pgfscope}%
\definecolor{textcolor}{rgb}{0.000000,0.000000,0.000000}%
\pgfsetstrokecolor{textcolor}%
\pgfsetfillcolor{textcolor}%
\pgftext[x=1.630562in,y=0.451547in,,top]{\color{textcolor}\sffamily\fontsize{10.000000}{12.000000}\selectfont \(\displaystyle {10}\)}%
\end{pgfscope}%
\begin{pgfscope}%
\pgfsetbuttcap%
\pgfsetroundjoin%
\definecolor{currentfill}{rgb}{0.000000,0.000000,0.000000}%
\pgfsetfillcolor{currentfill}%
\pgfsetlinewidth{0.803000pt}%
\definecolor{currentstroke}{rgb}{0.000000,0.000000,0.000000}%
\pgfsetstrokecolor{currentstroke}%
\pgfsetdash{}{0pt}%
\pgfsys@defobject{currentmarker}{\pgfqpoint{0.000000in}{-0.048611in}}{\pgfqpoint{0.000000in}{0.000000in}}{%
\pgfpathmoveto{\pgfqpoint{0.000000in}{0.000000in}}%
\pgfpathlineto{\pgfqpoint{0.000000in}{-0.048611in}}%
\pgfusepath{stroke,fill}%
}%
\begin{pgfscope}%
\pgfsys@transformshift{2.459134in}{0.548769in}%
\pgfsys@useobject{currentmarker}{}%
\end{pgfscope}%
\end{pgfscope}%
\begin{pgfscope}%
\definecolor{textcolor}{rgb}{0.000000,0.000000,0.000000}%
\pgfsetstrokecolor{textcolor}%
\pgfsetfillcolor{textcolor}%
\pgftext[x=2.459134in,y=0.451547in,,top]{\color{textcolor}\sffamily\fontsize{10.000000}{12.000000}\selectfont \(\displaystyle {20}\)}%
\end{pgfscope}%
\begin{pgfscope}%
\pgfsetbuttcap%
\pgfsetroundjoin%
\definecolor{currentfill}{rgb}{0.000000,0.000000,0.000000}%
\pgfsetfillcolor{currentfill}%
\pgfsetlinewidth{0.803000pt}%
\definecolor{currentstroke}{rgb}{0.000000,0.000000,0.000000}%
\pgfsetstrokecolor{currentstroke}%
\pgfsetdash{}{0pt}%
\pgfsys@defobject{currentmarker}{\pgfqpoint{0.000000in}{-0.048611in}}{\pgfqpoint{0.000000in}{0.000000in}}{%
\pgfpathmoveto{\pgfqpoint{0.000000in}{0.000000in}}%
\pgfpathlineto{\pgfqpoint{0.000000in}{-0.048611in}}%
\pgfusepath{stroke,fill}%
}%
\begin{pgfscope}%
\pgfsys@transformshift{3.287707in}{0.548769in}%
\pgfsys@useobject{currentmarker}{}%
\end{pgfscope}%
\end{pgfscope}%
\begin{pgfscope}%
\definecolor{textcolor}{rgb}{0.000000,0.000000,0.000000}%
\pgfsetstrokecolor{textcolor}%
\pgfsetfillcolor{textcolor}%
\pgftext[x=3.287707in,y=0.451547in,,top]{\color{textcolor}\sffamily\fontsize{10.000000}{12.000000}\selectfont \(\displaystyle {30}\)}%
\end{pgfscope}%
\begin{pgfscope}%
\pgfsetbuttcap%
\pgfsetroundjoin%
\definecolor{currentfill}{rgb}{0.000000,0.000000,0.000000}%
\pgfsetfillcolor{currentfill}%
\pgfsetlinewidth{0.803000pt}%
\definecolor{currentstroke}{rgb}{0.000000,0.000000,0.000000}%
\pgfsetstrokecolor{currentstroke}%
\pgfsetdash{}{0pt}%
\pgfsys@defobject{currentmarker}{\pgfqpoint{0.000000in}{-0.048611in}}{\pgfqpoint{0.000000in}{0.000000in}}{%
\pgfpathmoveto{\pgfqpoint{0.000000in}{0.000000in}}%
\pgfpathlineto{\pgfqpoint{0.000000in}{-0.048611in}}%
\pgfusepath{stroke,fill}%
}%
\begin{pgfscope}%
\pgfsys@transformshift{4.116279in}{0.548769in}%
\pgfsys@useobject{currentmarker}{}%
\end{pgfscope}%
\end{pgfscope}%
\begin{pgfscope}%
\definecolor{textcolor}{rgb}{0.000000,0.000000,0.000000}%
\pgfsetstrokecolor{textcolor}%
\pgfsetfillcolor{textcolor}%
\pgftext[x=4.116279in,y=0.451547in,,top]{\color{textcolor}\sffamily\fontsize{10.000000}{12.000000}\selectfont \(\displaystyle {40}\)}%
\end{pgfscope}%
\begin{pgfscope}%
\pgfsetbuttcap%
\pgfsetroundjoin%
\definecolor{currentfill}{rgb}{0.000000,0.000000,0.000000}%
\pgfsetfillcolor{currentfill}%
\pgfsetlinewidth{0.803000pt}%
\definecolor{currentstroke}{rgb}{0.000000,0.000000,0.000000}%
\pgfsetstrokecolor{currentstroke}%
\pgfsetdash{}{0pt}%
\pgfsys@defobject{currentmarker}{\pgfqpoint{0.000000in}{-0.048611in}}{\pgfqpoint{0.000000in}{0.000000in}}{%
\pgfpathmoveto{\pgfqpoint{0.000000in}{0.000000in}}%
\pgfpathlineto{\pgfqpoint{0.000000in}{-0.048611in}}%
\pgfusepath{stroke,fill}%
}%
\begin{pgfscope}%
\pgfsys@transformshift{4.944852in}{0.548769in}%
\pgfsys@useobject{currentmarker}{}%
\end{pgfscope}%
\end{pgfscope}%
\begin{pgfscope}%
\definecolor{textcolor}{rgb}{0.000000,0.000000,0.000000}%
\pgfsetstrokecolor{textcolor}%
\pgfsetfillcolor{textcolor}%
\pgftext[x=4.944852in,y=0.451547in,,top]{\color{textcolor}\sffamily\fontsize{10.000000}{12.000000}\selectfont \(\displaystyle {50}\)}%
\end{pgfscope}%
\begin{pgfscope}%
\pgfsetbuttcap%
\pgfsetroundjoin%
\definecolor{currentfill}{rgb}{0.000000,0.000000,0.000000}%
\pgfsetfillcolor{currentfill}%
\pgfsetlinewidth{0.803000pt}%
\definecolor{currentstroke}{rgb}{0.000000,0.000000,0.000000}%
\pgfsetstrokecolor{currentstroke}%
\pgfsetdash{}{0pt}%
\pgfsys@defobject{currentmarker}{\pgfqpoint{0.000000in}{-0.048611in}}{\pgfqpoint{0.000000in}{0.000000in}}{%
\pgfpathmoveto{\pgfqpoint{0.000000in}{0.000000in}}%
\pgfpathlineto{\pgfqpoint{0.000000in}{-0.048611in}}%
\pgfusepath{stroke,fill}%
}%
\begin{pgfscope}%
\pgfsys@transformshift{5.773424in}{0.548769in}%
\pgfsys@useobject{currentmarker}{}%
\end{pgfscope}%
\end{pgfscope}%
\begin{pgfscope}%
\definecolor{textcolor}{rgb}{0.000000,0.000000,0.000000}%
\pgfsetstrokecolor{textcolor}%
\pgfsetfillcolor{textcolor}%
\pgftext[x=5.773424in,y=0.451547in,,top]{\color{textcolor}\sffamily\fontsize{10.000000}{12.000000}\selectfont \(\displaystyle {60}\)}%
\end{pgfscope}%
\begin{pgfscope}%
\definecolor{textcolor}{rgb}{0.000000,0.000000,0.000000}%
\pgfsetstrokecolor{textcolor}%
\pgfsetfillcolor{textcolor}%
\pgftext[x=3.246278in,y=0.272658in,,top]{\color{textcolor}\sffamily\fontsize{10.000000}{12.000000}\selectfont Number of Sources}%
\end{pgfscope}%
\begin{pgfscope}%
\pgfsetbuttcap%
\pgfsetroundjoin%
\definecolor{currentfill}{rgb}{0.000000,0.000000,0.000000}%
\pgfsetfillcolor{currentfill}%
\pgfsetlinewidth{0.803000pt}%
\definecolor{currentstroke}{rgb}{0.000000,0.000000,0.000000}%
\pgfsetstrokecolor{currentstroke}%
\pgfsetdash{}{0pt}%
\pgfsys@defobject{currentmarker}{\pgfqpoint{-0.048611in}{0.000000in}}{\pgfqpoint{0.000000in}{0.000000in}}{%
\pgfpathmoveto{\pgfqpoint{0.000000in}{0.000000in}}%
\pgfpathlineto{\pgfqpoint{-0.048611in}{0.000000in}}%
\pgfusepath{stroke,fill}%
}%
\begin{pgfscope}%
\pgfsys@transformshift{0.648703in}{0.689796in}%
\pgfsys@useobject{currentmarker}{}%
\end{pgfscope}%
\end{pgfscope}%
\begin{pgfscope}%
\definecolor{textcolor}{rgb}{0.000000,0.000000,0.000000}%
\pgfsetstrokecolor{textcolor}%
\pgfsetfillcolor{textcolor}%
\pgftext[x=0.482036in, y=0.641601in, left, base]{\color{textcolor}\sffamily\fontsize{10.000000}{12.000000}\selectfont \(\displaystyle {0}\)}%
\end{pgfscope}%
\begin{pgfscope}%
\pgfsetbuttcap%
\pgfsetroundjoin%
\definecolor{currentfill}{rgb}{0.000000,0.000000,0.000000}%
\pgfsetfillcolor{currentfill}%
\pgfsetlinewidth{0.803000pt}%
\definecolor{currentstroke}{rgb}{0.000000,0.000000,0.000000}%
\pgfsetstrokecolor{currentstroke}%
\pgfsetdash{}{0pt}%
\pgfsys@defobject{currentmarker}{\pgfqpoint{-0.048611in}{0.000000in}}{\pgfqpoint{0.000000in}{0.000000in}}{%
\pgfpathmoveto{\pgfqpoint{0.000000in}{0.000000in}}%
\pgfpathlineto{\pgfqpoint{-0.048611in}{0.000000in}}%
\pgfusepath{stroke,fill}%
}%
\begin{pgfscope}%
\pgfsys@transformshift{0.648703in}{1.112665in}%
\pgfsys@useobject{currentmarker}{}%
\end{pgfscope}%
\end{pgfscope}%
\begin{pgfscope}%
\definecolor{textcolor}{rgb}{0.000000,0.000000,0.000000}%
\pgfsetstrokecolor{textcolor}%
\pgfsetfillcolor{textcolor}%
\pgftext[x=0.343147in, y=1.064470in, left, base]{\color{textcolor}\sffamily\fontsize{10.000000}{12.000000}\selectfont \(\displaystyle {100}\)}%
\end{pgfscope}%
\begin{pgfscope}%
\pgfsetbuttcap%
\pgfsetroundjoin%
\definecolor{currentfill}{rgb}{0.000000,0.000000,0.000000}%
\pgfsetfillcolor{currentfill}%
\pgfsetlinewidth{0.803000pt}%
\definecolor{currentstroke}{rgb}{0.000000,0.000000,0.000000}%
\pgfsetstrokecolor{currentstroke}%
\pgfsetdash{}{0pt}%
\pgfsys@defobject{currentmarker}{\pgfqpoint{-0.048611in}{0.000000in}}{\pgfqpoint{0.000000in}{0.000000in}}{%
\pgfpathmoveto{\pgfqpoint{0.000000in}{0.000000in}}%
\pgfpathlineto{\pgfqpoint{-0.048611in}{0.000000in}}%
\pgfusepath{stroke,fill}%
}%
\begin{pgfscope}%
\pgfsys@transformshift{0.648703in}{1.535534in}%
\pgfsys@useobject{currentmarker}{}%
\end{pgfscope}%
\end{pgfscope}%
\begin{pgfscope}%
\definecolor{textcolor}{rgb}{0.000000,0.000000,0.000000}%
\pgfsetstrokecolor{textcolor}%
\pgfsetfillcolor{textcolor}%
\pgftext[x=0.343147in, y=1.487339in, left, base]{\color{textcolor}\sffamily\fontsize{10.000000}{12.000000}\selectfont \(\displaystyle {200}\)}%
\end{pgfscope}%
\begin{pgfscope}%
\pgfsetbuttcap%
\pgfsetroundjoin%
\definecolor{currentfill}{rgb}{0.000000,0.000000,0.000000}%
\pgfsetfillcolor{currentfill}%
\pgfsetlinewidth{0.803000pt}%
\definecolor{currentstroke}{rgb}{0.000000,0.000000,0.000000}%
\pgfsetstrokecolor{currentstroke}%
\pgfsetdash{}{0pt}%
\pgfsys@defobject{currentmarker}{\pgfqpoint{-0.048611in}{0.000000in}}{\pgfqpoint{0.000000in}{0.000000in}}{%
\pgfpathmoveto{\pgfqpoint{0.000000in}{0.000000in}}%
\pgfpathlineto{\pgfqpoint{-0.048611in}{0.000000in}}%
\pgfusepath{stroke,fill}%
}%
\begin{pgfscope}%
\pgfsys@transformshift{0.648703in}{1.958403in}%
\pgfsys@useobject{currentmarker}{}%
\end{pgfscope}%
\end{pgfscope}%
\begin{pgfscope}%
\definecolor{textcolor}{rgb}{0.000000,0.000000,0.000000}%
\pgfsetstrokecolor{textcolor}%
\pgfsetfillcolor{textcolor}%
\pgftext[x=0.343147in, y=1.910208in, left, base]{\color{textcolor}\sffamily\fontsize{10.000000}{12.000000}\selectfont \(\displaystyle {300}\)}%
\end{pgfscope}%
\begin{pgfscope}%
\pgfsetbuttcap%
\pgfsetroundjoin%
\definecolor{currentfill}{rgb}{0.000000,0.000000,0.000000}%
\pgfsetfillcolor{currentfill}%
\pgfsetlinewidth{0.803000pt}%
\definecolor{currentstroke}{rgb}{0.000000,0.000000,0.000000}%
\pgfsetstrokecolor{currentstroke}%
\pgfsetdash{}{0pt}%
\pgfsys@defobject{currentmarker}{\pgfqpoint{-0.048611in}{0.000000in}}{\pgfqpoint{0.000000in}{0.000000in}}{%
\pgfpathmoveto{\pgfqpoint{0.000000in}{0.000000in}}%
\pgfpathlineto{\pgfqpoint{-0.048611in}{0.000000in}}%
\pgfusepath{stroke,fill}%
}%
\begin{pgfscope}%
\pgfsys@transformshift{0.648703in}{2.381272in}%
\pgfsys@useobject{currentmarker}{}%
\end{pgfscope}%
\end{pgfscope}%
\begin{pgfscope}%
\definecolor{textcolor}{rgb}{0.000000,0.000000,0.000000}%
\pgfsetstrokecolor{textcolor}%
\pgfsetfillcolor{textcolor}%
\pgftext[x=0.343147in, y=2.333077in, left, base]{\color{textcolor}\sffamily\fontsize{10.000000}{12.000000}\selectfont \(\displaystyle {400}\)}%
\end{pgfscope}%
\begin{pgfscope}%
\pgfsetbuttcap%
\pgfsetroundjoin%
\definecolor{currentfill}{rgb}{0.000000,0.000000,0.000000}%
\pgfsetfillcolor{currentfill}%
\pgfsetlinewidth{0.803000pt}%
\definecolor{currentstroke}{rgb}{0.000000,0.000000,0.000000}%
\pgfsetstrokecolor{currentstroke}%
\pgfsetdash{}{0pt}%
\pgfsys@defobject{currentmarker}{\pgfqpoint{-0.048611in}{0.000000in}}{\pgfqpoint{0.000000in}{0.000000in}}{%
\pgfpathmoveto{\pgfqpoint{0.000000in}{0.000000in}}%
\pgfpathlineto{\pgfqpoint{-0.048611in}{0.000000in}}%
\pgfusepath{stroke,fill}%
}%
\begin{pgfscope}%
\pgfsys@transformshift{0.648703in}{2.804141in}%
\pgfsys@useobject{currentmarker}{}%
\end{pgfscope}%
\end{pgfscope}%
\begin{pgfscope}%
\definecolor{textcolor}{rgb}{0.000000,0.000000,0.000000}%
\pgfsetstrokecolor{textcolor}%
\pgfsetfillcolor{textcolor}%
\pgftext[x=0.343147in, y=2.755946in, left, base]{\color{textcolor}\sffamily\fontsize{10.000000}{12.000000}\selectfont \(\displaystyle {500}\)}%
\end{pgfscope}%
\begin{pgfscope}%
\pgfsetbuttcap%
\pgfsetroundjoin%
\definecolor{currentfill}{rgb}{0.000000,0.000000,0.000000}%
\pgfsetfillcolor{currentfill}%
\pgfsetlinewidth{0.803000pt}%
\definecolor{currentstroke}{rgb}{0.000000,0.000000,0.000000}%
\pgfsetstrokecolor{currentstroke}%
\pgfsetdash{}{0pt}%
\pgfsys@defobject{currentmarker}{\pgfqpoint{-0.048611in}{0.000000in}}{\pgfqpoint{0.000000in}{0.000000in}}{%
\pgfpathmoveto{\pgfqpoint{0.000000in}{0.000000in}}%
\pgfpathlineto{\pgfqpoint{-0.048611in}{0.000000in}}%
\pgfusepath{stroke,fill}%
}%
\begin{pgfscope}%
\pgfsys@transformshift{0.648703in}{3.227010in}%
\pgfsys@useobject{currentmarker}{}%
\end{pgfscope}%
\end{pgfscope}%
\begin{pgfscope}%
\definecolor{textcolor}{rgb}{0.000000,0.000000,0.000000}%
\pgfsetstrokecolor{textcolor}%
\pgfsetfillcolor{textcolor}%
\pgftext[x=0.343147in, y=3.178815in, left, base]{\color{textcolor}\sffamily\fontsize{10.000000}{12.000000}\selectfont \(\displaystyle {600}\)}%
\end{pgfscope}%
\begin{pgfscope}%
\pgfsetbuttcap%
\pgfsetroundjoin%
\definecolor{currentfill}{rgb}{0.000000,0.000000,0.000000}%
\pgfsetfillcolor{currentfill}%
\pgfsetlinewidth{0.803000pt}%
\definecolor{currentstroke}{rgb}{0.000000,0.000000,0.000000}%
\pgfsetstrokecolor{currentstroke}%
\pgfsetdash{}{0pt}%
\pgfsys@defobject{currentmarker}{\pgfqpoint{-0.048611in}{0.000000in}}{\pgfqpoint{0.000000in}{0.000000in}}{%
\pgfpathmoveto{\pgfqpoint{0.000000in}{0.000000in}}%
\pgfpathlineto{\pgfqpoint{-0.048611in}{0.000000in}}%
\pgfusepath{stroke,fill}%
}%
\begin{pgfscope}%
\pgfsys@transformshift{0.648703in}{3.649879in}%
\pgfsys@useobject{currentmarker}{}%
\end{pgfscope}%
\end{pgfscope}%
\begin{pgfscope}%
\definecolor{textcolor}{rgb}{0.000000,0.000000,0.000000}%
\pgfsetstrokecolor{textcolor}%
\pgfsetfillcolor{textcolor}%
\pgftext[x=0.343147in, y=3.601684in, left, base]{\color{textcolor}\sffamily\fontsize{10.000000}{12.000000}\selectfont \(\displaystyle {700}\)}%
\end{pgfscope}%
\begin{pgfscope}%
\definecolor{textcolor}{rgb}{0.000000,0.000000,0.000000}%
\pgfsetstrokecolor{textcolor}%
\pgfsetfillcolor{textcolor}%
\pgftext[x=0.287592in,y=2.100064in,,bottom,rotate=90.000000]{\color{textcolor}\sffamily\fontsize{10.000000}{12.000000}\selectfont Data Flow Time (s)}%
\end{pgfscope}%
\begin{pgfscope}%
\pgfsetrectcap%
\pgfsetmiterjoin%
\pgfsetlinewidth{0.803000pt}%
\definecolor{currentstroke}{rgb}{0.000000,0.000000,0.000000}%
\pgfsetstrokecolor{currentstroke}%
\pgfsetdash{}{0pt}%
\pgfpathmoveto{\pgfqpoint{0.648703in}{0.548769in}}%
\pgfpathlineto{\pgfqpoint{0.648703in}{3.651359in}}%
\pgfusepath{stroke}%
\end{pgfscope}%
\begin{pgfscope}%
\pgfsetrectcap%
\pgfsetmiterjoin%
\pgfsetlinewidth{0.803000pt}%
\definecolor{currentstroke}{rgb}{0.000000,0.000000,0.000000}%
\pgfsetstrokecolor{currentstroke}%
\pgfsetdash{}{0pt}%
\pgfpathmoveto{\pgfqpoint{5.843853in}{0.548769in}}%
\pgfpathlineto{\pgfqpoint{5.843853in}{3.651359in}}%
\pgfusepath{stroke}%
\end{pgfscope}%
\begin{pgfscope}%
\pgfsetrectcap%
\pgfsetmiterjoin%
\pgfsetlinewidth{0.803000pt}%
\definecolor{currentstroke}{rgb}{0.000000,0.000000,0.000000}%
\pgfsetstrokecolor{currentstroke}%
\pgfsetdash{}{0pt}%
\pgfpathmoveto{\pgfqpoint{0.648703in}{0.548769in}}%
\pgfpathlineto{\pgfqpoint{5.843853in}{0.548769in}}%
\pgfusepath{stroke}%
\end{pgfscope}%
\begin{pgfscope}%
\pgfsetrectcap%
\pgfsetmiterjoin%
\pgfsetlinewidth{0.803000pt}%
\definecolor{currentstroke}{rgb}{0.000000,0.000000,0.000000}%
\pgfsetstrokecolor{currentstroke}%
\pgfsetdash{}{0pt}%
\pgfpathmoveto{\pgfqpoint{0.648703in}{3.651359in}}%
\pgfpathlineto{\pgfqpoint{5.843853in}{3.651359in}}%
\pgfusepath{stroke}%
\end{pgfscope}%
\begin{pgfscope}%
\definecolor{textcolor}{rgb}{0.000000,0.000000,0.000000}%
\pgfsetstrokecolor{textcolor}%
\pgfsetfillcolor{textcolor}%
\pgftext[x=3.246278in,y=3.734692in,,base]{\color{textcolor}\sffamily\fontsize{12.000000}{14.400000}\selectfont Backwards}%
\end{pgfscope}%
\begin{pgfscope}%
\pgfsetbuttcap%
\pgfsetmiterjoin%
\definecolor{currentfill}{rgb}{1.000000,1.000000,1.000000}%
\pgfsetfillcolor{currentfill}%
\pgfsetfillopacity{0.800000}%
\pgfsetlinewidth{1.003750pt}%
\definecolor{currentstroke}{rgb}{0.800000,0.800000,0.800000}%
\pgfsetstrokecolor{currentstroke}%
\pgfsetstrokeopacity{0.800000}%
\pgfsetdash{}{0pt}%
\pgfpathmoveto{\pgfqpoint{4.294270in}{1.788050in}}%
\pgfpathlineto{\pgfqpoint{5.746631in}{1.788050in}}%
\pgfpathquadraticcurveto{\pgfqpoint{5.774409in}{1.788050in}}{\pgfqpoint{5.774409in}{1.815828in}}%
\pgfpathlineto{\pgfqpoint{5.774409in}{2.384300in}}%
\pgfpathquadraticcurveto{\pgfqpoint{5.774409in}{2.412078in}}{\pgfqpoint{5.746631in}{2.412078in}}%
\pgfpathlineto{\pgfqpoint{4.294270in}{2.412078in}}%
\pgfpathquadraticcurveto{\pgfqpoint{4.266492in}{2.412078in}}{\pgfqpoint{4.266492in}{2.384300in}}%
\pgfpathlineto{\pgfqpoint{4.266492in}{1.815828in}}%
\pgfpathquadraticcurveto{\pgfqpoint{4.266492in}{1.788050in}}{\pgfqpoint{4.294270in}{1.788050in}}%
\pgfpathclose%
\pgfusepath{stroke,fill}%
\end{pgfscope}%
\begin{pgfscope}%
\pgfsetbuttcap%
\pgfsetroundjoin%
\definecolor{currentfill}{rgb}{0.121569,0.466667,0.705882}%
\pgfsetfillcolor{currentfill}%
\pgfsetlinewidth{1.003750pt}%
\definecolor{currentstroke}{rgb}{0.121569,0.466667,0.705882}%
\pgfsetstrokecolor{currentstroke}%
\pgfsetdash{}{0pt}%
\pgfsys@defobject{currentmarker}{\pgfqpoint{-0.034722in}{-0.034722in}}{\pgfqpoint{0.034722in}{0.034722in}}{%
\pgfpathmoveto{\pgfqpoint{0.000000in}{-0.034722in}}%
\pgfpathcurveto{\pgfqpoint{0.009208in}{-0.034722in}}{\pgfqpoint{0.018041in}{-0.031064in}}{\pgfqpoint{0.024552in}{-0.024552in}}%
\pgfpathcurveto{\pgfqpoint{0.031064in}{-0.018041in}}{\pgfqpoint{0.034722in}{-0.009208in}}{\pgfqpoint{0.034722in}{0.000000in}}%
\pgfpathcurveto{\pgfqpoint{0.034722in}{0.009208in}}{\pgfqpoint{0.031064in}{0.018041in}}{\pgfqpoint{0.024552in}{0.024552in}}%
\pgfpathcurveto{\pgfqpoint{0.018041in}{0.031064in}}{\pgfqpoint{0.009208in}{0.034722in}}{\pgfqpoint{0.000000in}{0.034722in}}%
\pgfpathcurveto{\pgfqpoint{-0.009208in}{0.034722in}}{\pgfqpoint{-0.018041in}{0.031064in}}{\pgfqpoint{-0.024552in}{0.024552in}}%
\pgfpathcurveto{\pgfqpoint{-0.031064in}{0.018041in}}{\pgfqpoint{-0.034722in}{0.009208in}}{\pgfqpoint{-0.034722in}{0.000000in}}%
\pgfpathcurveto{\pgfqpoint{-0.034722in}{-0.009208in}}{\pgfqpoint{-0.031064in}{-0.018041in}}{\pgfqpoint{-0.024552in}{-0.024552in}}%
\pgfpathcurveto{\pgfqpoint{-0.018041in}{-0.031064in}}{\pgfqpoint{-0.009208in}{-0.034722in}}{\pgfqpoint{0.000000in}{-0.034722in}}%
\pgfpathclose%
\pgfusepath{stroke,fill}%
}%
\begin{pgfscope}%
\pgfsys@transformshift{4.460936in}{2.307911in}%
\pgfsys@useobject{currentmarker}{}%
\end{pgfscope}%
\end{pgfscope}%
\begin{pgfscope}%
\definecolor{textcolor}{rgb}{0.000000,0.000000,0.000000}%
\pgfsetstrokecolor{textcolor}%
\pgfsetfillcolor{textcolor}%
\pgftext[x=4.710936in,y=2.259300in,left,base]{\color{textcolor}\sffamily\fontsize{10.000000}{12.000000}\selectfont No Timeout}%
\end{pgfscope}%
\begin{pgfscope}%
\pgfsetbuttcap%
\pgfsetroundjoin%
\definecolor{currentfill}{rgb}{1.000000,0.498039,0.054902}%
\pgfsetfillcolor{currentfill}%
\pgfsetlinewidth{1.003750pt}%
\definecolor{currentstroke}{rgb}{1.000000,0.498039,0.054902}%
\pgfsetstrokecolor{currentstroke}%
\pgfsetdash{}{0pt}%
\pgfsys@defobject{currentmarker}{\pgfqpoint{-0.034722in}{-0.034722in}}{\pgfqpoint{0.034722in}{0.034722in}}{%
\pgfpathmoveto{\pgfqpoint{0.000000in}{-0.034722in}}%
\pgfpathcurveto{\pgfqpoint{0.009208in}{-0.034722in}}{\pgfqpoint{0.018041in}{-0.031064in}}{\pgfqpoint{0.024552in}{-0.024552in}}%
\pgfpathcurveto{\pgfqpoint{0.031064in}{-0.018041in}}{\pgfqpoint{0.034722in}{-0.009208in}}{\pgfqpoint{0.034722in}{0.000000in}}%
\pgfpathcurveto{\pgfqpoint{0.034722in}{0.009208in}}{\pgfqpoint{0.031064in}{0.018041in}}{\pgfqpoint{0.024552in}{0.024552in}}%
\pgfpathcurveto{\pgfqpoint{0.018041in}{0.031064in}}{\pgfqpoint{0.009208in}{0.034722in}}{\pgfqpoint{0.000000in}{0.034722in}}%
\pgfpathcurveto{\pgfqpoint{-0.009208in}{0.034722in}}{\pgfqpoint{-0.018041in}{0.031064in}}{\pgfqpoint{-0.024552in}{0.024552in}}%
\pgfpathcurveto{\pgfqpoint{-0.031064in}{0.018041in}}{\pgfqpoint{-0.034722in}{0.009208in}}{\pgfqpoint{-0.034722in}{0.000000in}}%
\pgfpathcurveto{\pgfqpoint{-0.034722in}{-0.009208in}}{\pgfqpoint{-0.031064in}{-0.018041in}}{\pgfqpoint{-0.024552in}{-0.024552in}}%
\pgfpathcurveto{\pgfqpoint{-0.018041in}{-0.031064in}}{\pgfqpoint{-0.009208in}{-0.034722in}}{\pgfqpoint{0.000000in}{-0.034722in}}%
\pgfpathclose%
\pgfusepath{stroke,fill}%
}%
\begin{pgfscope}%
\pgfsys@transformshift{4.460936in}{2.114300in}%
\pgfsys@useobject{currentmarker}{}%
\end{pgfscope}%
\end{pgfscope}%
\begin{pgfscope}%
\definecolor{textcolor}{rgb}{0.000000,0.000000,0.000000}%
\pgfsetstrokecolor{textcolor}%
\pgfsetfillcolor{textcolor}%
\pgftext[x=4.710936in,y=2.065689in,left,base]{\color{textcolor}\sffamily\fontsize{10.000000}{12.000000}\selectfont Time Timeout}%
\end{pgfscope}%
\begin{pgfscope}%
\pgfsetbuttcap%
\pgfsetroundjoin%
\definecolor{currentfill}{rgb}{0.839216,0.152941,0.156863}%
\pgfsetfillcolor{currentfill}%
\pgfsetlinewidth{1.003750pt}%
\definecolor{currentstroke}{rgb}{0.839216,0.152941,0.156863}%
\pgfsetstrokecolor{currentstroke}%
\pgfsetdash{}{0pt}%
\pgfsys@defobject{currentmarker}{\pgfqpoint{-0.034722in}{-0.034722in}}{\pgfqpoint{0.034722in}{0.034722in}}{%
\pgfpathmoveto{\pgfqpoint{0.000000in}{-0.034722in}}%
\pgfpathcurveto{\pgfqpoint{0.009208in}{-0.034722in}}{\pgfqpoint{0.018041in}{-0.031064in}}{\pgfqpoint{0.024552in}{-0.024552in}}%
\pgfpathcurveto{\pgfqpoint{0.031064in}{-0.018041in}}{\pgfqpoint{0.034722in}{-0.009208in}}{\pgfqpoint{0.034722in}{0.000000in}}%
\pgfpathcurveto{\pgfqpoint{0.034722in}{0.009208in}}{\pgfqpoint{0.031064in}{0.018041in}}{\pgfqpoint{0.024552in}{0.024552in}}%
\pgfpathcurveto{\pgfqpoint{0.018041in}{0.031064in}}{\pgfqpoint{0.009208in}{0.034722in}}{\pgfqpoint{0.000000in}{0.034722in}}%
\pgfpathcurveto{\pgfqpoint{-0.009208in}{0.034722in}}{\pgfqpoint{-0.018041in}{0.031064in}}{\pgfqpoint{-0.024552in}{0.024552in}}%
\pgfpathcurveto{\pgfqpoint{-0.031064in}{0.018041in}}{\pgfqpoint{-0.034722in}{0.009208in}}{\pgfqpoint{-0.034722in}{0.000000in}}%
\pgfpathcurveto{\pgfqpoint{-0.034722in}{-0.009208in}}{\pgfqpoint{-0.031064in}{-0.018041in}}{\pgfqpoint{-0.024552in}{-0.024552in}}%
\pgfpathcurveto{\pgfqpoint{-0.018041in}{-0.031064in}}{\pgfqpoint{-0.009208in}{-0.034722in}}{\pgfqpoint{0.000000in}{-0.034722in}}%
\pgfpathclose%
\pgfusepath{stroke,fill}%
}%
\begin{pgfscope}%
\pgfsys@transformshift{4.460936in}{1.920689in}%
\pgfsys@useobject{currentmarker}{}%
\end{pgfscope}%
\end{pgfscope}%
\begin{pgfscope}%
\definecolor{textcolor}{rgb}{0.000000,0.000000,0.000000}%
\pgfsetstrokecolor{textcolor}%
\pgfsetfillcolor{textcolor}%
\pgftext[x=4.710936in,y=1.872078in,left,base]{\color{textcolor}\sffamily\fontsize{10.000000}{12.000000}\selectfont Memory Timeout}%
\end{pgfscope}%
\end{pgfpicture}%
\makeatother%
\endgroup%

            }
        \end{subfigure}
        \qquad
        \begin{subfigure}[]{0.45\textwidth}
            \centering
            \resizebox{\columnwidth}{!}{
                %% Creator: Matplotlib, PGF backend
%%
%% To include the figure in your LaTeX document, write
%%   \input{<filename>.pgf}
%%
%% Make sure the required packages are loaded in your preamble
%%   \usepackage{pgf}
%%
%% and, on pdftex
%%   \usepackage[utf8]{inputenc}\DeclareUnicodeCharacter{2212}{-}
%%
%% or, on luatex and xetex
%%   \usepackage{unicode-math}
%%
%% Figures using additional raster images can only be included by \input if
%% they are in the same directory as the main LaTeX file. For loading figures
%% from other directories you can use the `import` package
%%   \usepackage{import}
%%
%% and then include the figures with
%%   \import{<path to file>}{<filename>.pgf}
%%
%% Matplotlib used the following preamble
%%   \usepackage{amsmath}
%%   \usepackage{fontspec}
%%
\begingroup%
\makeatletter%
\begin{pgfpicture}%
\pgfpathrectangle{\pgfpointorigin}{\pgfqpoint{6.000000in}{4.000000in}}%
\pgfusepath{use as bounding box, clip}%
\begin{pgfscope}%
\pgfsetbuttcap%
\pgfsetmiterjoin%
\definecolor{currentfill}{rgb}{1.000000,1.000000,1.000000}%
\pgfsetfillcolor{currentfill}%
\pgfsetlinewidth{0.000000pt}%
\definecolor{currentstroke}{rgb}{1.000000,1.000000,1.000000}%
\pgfsetstrokecolor{currentstroke}%
\pgfsetdash{}{0pt}%
\pgfpathmoveto{\pgfqpoint{0.000000in}{0.000000in}}%
\pgfpathlineto{\pgfqpoint{6.000000in}{0.000000in}}%
\pgfpathlineto{\pgfqpoint{6.000000in}{4.000000in}}%
\pgfpathlineto{\pgfqpoint{0.000000in}{4.000000in}}%
\pgfpathclose%
\pgfusepath{fill}%
\end{pgfscope}%
\begin{pgfscope}%
\pgfsetbuttcap%
\pgfsetmiterjoin%
\definecolor{currentfill}{rgb}{1.000000,1.000000,1.000000}%
\pgfsetfillcolor{currentfill}%
\pgfsetlinewidth{0.000000pt}%
\definecolor{currentstroke}{rgb}{0.000000,0.000000,0.000000}%
\pgfsetstrokecolor{currentstroke}%
\pgfsetstrokeopacity{0.000000}%
\pgfsetdash{}{0pt}%
\pgfpathmoveto{\pgfqpoint{0.648703in}{0.548769in}}%
\pgfpathlineto{\pgfqpoint{5.850000in}{0.548769in}}%
\pgfpathlineto{\pgfqpoint{5.850000in}{3.651359in}}%
\pgfpathlineto{\pgfqpoint{0.648703in}{3.651359in}}%
\pgfpathclose%
\pgfusepath{fill}%
\end{pgfscope}%
\begin{pgfscope}%
\pgfpathrectangle{\pgfqpoint{0.648703in}{0.548769in}}{\pgfqpoint{5.201297in}{3.102590in}}%
\pgfusepath{clip}%
\pgfsetbuttcap%
\pgfsetroundjoin%
\definecolor{currentfill}{rgb}{0.121569,0.466667,0.705882}%
\pgfsetfillcolor{currentfill}%
\pgfsetlinewidth{1.003750pt}%
\definecolor{currentstroke}{rgb}{0.121569,0.466667,0.705882}%
\pgfsetstrokecolor{currentstroke}%
\pgfsetdash{}{0pt}%
\pgfpathmoveto{\pgfqpoint{0.962642in}{0.673501in}}%
\pgfpathcurveto{\pgfqpoint{0.973692in}{0.673501in}}{\pgfqpoint{0.984291in}{0.677891in}}{\pgfqpoint{0.992104in}{0.685705in}}%
\pgfpathcurveto{\pgfqpoint{0.999918in}{0.693519in}}{\pgfqpoint{1.004308in}{0.704118in}}{\pgfqpoint{1.004308in}{0.715168in}}%
\pgfpathcurveto{\pgfqpoint{1.004308in}{0.726218in}}{\pgfqpoint{0.999918in}{0.736817in}}{\pgfqpoint{0.992104in}{0.744631in}}%
\pgfpathcurveto{\pgfqpoint{0.984291in}{0.752444in}}{\pgfqpoint{0.973692in}{0.756834in}}{\pgfqpoint{0.962642in}{0.756834in}}%
\pgfpathcurveto{\pgfqpoint{0.951591in}{0.756834in}}{\pgfqpoint{0.940992in}{0.752444in}}{\pgfqpoint{0.933179in}{0.744631in}}%
\pgfpathcurveto{\pgfqpoint{0.925365in}{0.736817in}}{\pgfqpoint{0.920975in}{0.726218in}}{\pgfqpoint{0.920975in}{0.715168in}}%
\pgfpathcurveto{\pgfqpoint{0.920975in}{0.704118in}}{\pgfqpoint{0.925365in}{0.693519in}}{\pgfqpoint{0.933179in}{0.685705in}}%
\pgfpathcurveto{\pgfqpoint{0.940992in}{0.677891in}}{\pgfqpoint{0.951591in}{0.673501in}}{\pgfqpoint{0.962642in}{0.673501in}}%
\pgfpathclose%
\pgfusepath{stroke,fill}%
\end{pgfscope}%
\begin{pgfscope}%
\pgfpathrectangle{\pgfqpoint{0.648703in}{0.548769in}}{\pgfqpoint{5.201297in}{3.102590in}}%
\pgfusepath{clip}%
\pgfsetbuttcap%
\pgfsetroundjoin%
\definecolor{currentfill}{rgb}{1.000000,0.498039,0.054902}%
\pgfsetfillcolor{currentfill}%
\pgfsetlinewidth{1.003750pt}%
\definecolor{currentstroke}{rgb}{1.000000,0.498039,0.054902}%
\pgfsetstrokecolor{currentstroke}%
\pgfsetdash{}{0pt}%
\pgfpathmoveto{\pgfqpoint{2.745500in}{3.185343in}}%
\pgfpathcurveto{\pgfqpoint{2.756550in}{3.185343in}}{\pgfqpoint{2.767149in}{3.189733in}}{\pgfqpoint{2.774963in}{3.197547in}}%
\pgfpathcurveto{\pgfqpoint{2.782777in}{3.205360in}}{\pgfqpoint{2.787167in}{3.215959in}}{\pgfqpoint{2.787167in}{3.227010in}}%
\pgfpathcurveto{\pgfqpoint{2.787167in}{3.238060in}}{\pgfqpoint{2.782777in}{3.248659in}}{\pgfqpoint{2.774963in}{3.256472in}}%
\pgfpathcurveto{\pgfqpoint{2.767149in}{3.264286in}}{\pgfqpoint{2.756550in}{3.268676in}}{\pgfqpoint{2.745500in}{3.268676in}}%
\pgfpathcurveto{\pgfqpoint{2.734450in}{3.268676in}}{\pgfqpoint{2.723851in}{3.264286in}}{\pgfqpoint{2.716038in}{3.256472in}}%
\pgfpathcurveto{\pgfqpoint{2.708224in}{3.248659in}}{\pgfqpoint{2.703834in}{3.238060in}}{\pgfqpoint{2.703834in}{3.227010in}}%
\pgfpathcurveto{\pgfqpoint{2.703834in}{3.215959in}}{\pgfqpoint{2.708224in}{3.205360in}}{\pgfqpoint{2.716038in}{3.197547in}}%
\pgfpathcurveto{\pgfqpoint{2.723851in}{3.189733in}}{\pgfqpoint{2.734450in}{3.185343in}}{\pgfqpoint{2.745500in}{3.185343in}}%
\pgfpathclose%
\pgfusepath{stroke,fill}%
\end{pgfscope}%
\begin{pgfscope}%
\pgfpathrectangle{\pgfqpoint{0.648703in}{0.548769in}}{\pgfqpoint{5.201297in}{3.102590in}}%
\pgfusepath{clip}%
\pgfsetbuttcap%
\pgfsetroundjoin%
\definecolor{currentfill}{rgb}{0.121569,0.466667,0.705882}%
\pgfsetfillcolor{currentfill}%
\pgfsetlinewidth{1.003750pt}%
\definecolor{currentstroke}{rgb}{0.121569,0.466667,0.705882}%
\pgfsetstrokecolor{currentstroke}%
\pgfsetdash{}{0pt}%
\pgfpathmoveto{\pgfqpoint{1.427735in}{0.652358in}}%
\pgfpathcurveto{\pgfqpoint{1.438785in}{0.652358in}}{\pgfqpoint{1.449384in}{0.656748in}}{\pgfqpoint{1.457198in}{0.664562in}}%
\pgfpathcurveto{\pgfqpoint{1.465012in}{0.672375in}}{\pgfqpoint{1.469402in}{0.682974in}}{\pgfqpoint{1.469402in}{0.694024in}}%
\pgfpathcurveto{\pgfqpoint{1.469402in}{0.705074in}}{\pgfqpoint{1.465012in}{0.715673in}}{\pgfqpoint{1.457198in}{0.723487in}}%
\pgfpathcurveto{\pgfqpoint{1.449384in}{0.731301in}}{\pgfqpoint{1.438785in}{0.735691in}}{\pgfqpoint{1.427735in}{0.735691in}}%
\pgfpathcurveto{\pgfqpoint{1.416685in}{0.735691in}}{\pgfqpoint{1.406086in}{0.731301in}}{\pgfqpoint{1.398272in}{0.723487in}}%
\pgfpathcurveto{\pgfqpoint{1.390459in}{0.715673in}}{\pgfqpoint{1.386069in}{0.705074in}}{\pgfqpoint{1.386069in}{0.694024in}}%
\pgfpathcurveto{\pgfqpoint{1.386069in}{0.682974in}}{\pgfqpoint{1.390459in}{0.672375in}}{\pgfqpoint{1.398272in}{0.664562in}}%
\pgfpathcurveto{\pgfqpoint{1.406086in}{0.656748in}}{\pgfqpoint{1.416685in}{0.652358in}}{\pgfqpoint{1.427735in}{0.652358in}}%
\pgfpathclose%
\pgfusepath{stroke,fill}%
\end{pgfscope}%
\begin{pgfscope}%
\pgfpathrectangle{\pgfqpoint{0.648703in}{0.548769in}}{\pgfqpoint{5.201297in}{3.102590in}}%
\pgfusepath{clip}%
\pgfsetbuttcap%
\pgfsetroundjoin%
\definecolor{currentfill}{rgb}{0.121569,0.466667,0.705882}%
\pgfsetfillcolor{currentfill}%
\pgfsetlinewidth{1.003750pt}%
\definecolor{currentstroke}{rgb}{0.121569,0.466667,0.705882}%
\pgfsetstrokecolor{currentstroke}%
\pgfsetdash{}{0pt}%
\pgfpathmoveto{\pgfqpoint{4.760906in}{3.181114in}}%
\pgfpathcurveto{\pgfqpoint{4.771956in}{3.181114in}}{\pgfqpoint{4.782555in}{3.185504in}}{\pgfqpoint{4.790369in}{3.193318in}}%
\pgfpathcurveto{\pgfqpoint{4.798182in}{3.201132in}}{\pgfqpoint{4.802573in}{3.211731in}}{\pgfqpoint{4.802573in}{3.222781in}}%
\pgfpathcurveto{\pgfqpoint{4.802573in}{3.233831in}}{\pgfqpoint{4.798182in}{3.244430in}}{\pgfqpoint{4.790369in}{3.252244in}}%
\pgfpathcurveto{\pgfqpoint{4.782555in}{3.260057in}}{\pgfqpoint{4.771956in}{3.264448in}}{\pgfqpoint{4.760906in}{3.264448in}}%
\pgfpathcurveto{\pgfqpoint{4.749856in}{3.264448in}}{\pgfqpoint{4.739257in}{3.260057in}}{\pgfqpoint{4.731443in}{3.252244in}}%
\pgfpathcurveto{\pgfqpoint{4.723629in}{3.244430in}}{\pgfqpoint{4.719239in}{3.233831in}}{\pgfqpoint{4.719239in}{3.222781in}}%
\pgfpathcurveto{\pgfqpoint{4.719239in}{3.211731in}}{\pgfqpoint{4.723629in}{3.201132in}}{\pgfqpoint{4.731443in}{3.193318in}}%
\pgfpathcurveto{\pgfqpoint{4.739257in}{3.185504in}}{\pgfqpoint{4.749856in}{3.181114in}}{\pgfqpoint{4.760906in}{3.181114in}}%
\pgfpathclose%
\pgfusepath{stroke,fill}%
\end{pgfscope}%
\begin{pgfscope}%
\pgfpathrectangle{\pgfqpoint{0.648703in}{0.548769in}}{\pgfqpoint{5.201297in}{3.102590in}}%
\pgfusepath{clip}%
\pgfsetbuttcap%
\pgfsetroundjoin%
\definecolor{currentfill}{rgb}{1.000000,0.498039,0.054902}%
\pgfsetfillcolor{currentfill}%
\pgfsetlinewidth{1.003750pt}%
\definecolor{currentstroke}{rgb}{1.000000,0.498039,0.054902}%
\pgfsetstrokecolor{currentstroke}%
\pgfsetdash{}{0pt}%
\pgfpathmoveto{\pgfqpoint{3.365625in}{3.198029in}}%
\pgfpathcurveto{\pgfqpoint{3.376675in}{3.198029in}}{\pgfqpoint{3.387274in}{3.202419in}}{\pgfqpoint{3.395088in}{3.210233in}}%
\pgfpathcurveto{\pgfqpoint{3.402902in}{3.218046in}}{\pgfqpoint{3.407292in}{3.228646in}}{\pgfqpoint{3.407292in}{3.239696in}}%
\pgfpathcurveto{\pgfqpoint{3.407292in}{3.250746in}}{\pgfqpoint{3.402902in}{3.261345in}}{\pgfqpoint{3.395088in}{3.269158in}}%
\pgfpathcurveto{\pgfqpoint{3.387274in}{3.276972in}}{\pgfqpoint{3.376675in}{3.281362in}}{\pgfqpoint{3.365625in}{3.281362in}}%
\pgfpathcurveto{\pgfqpoint{3.354575in}{3.281362in}}{\pgfqpoint{3.343976in}{3.276972in}}{\pgfqpoint{3.336162in}{3.269158in}}%
\pgfpathcurveto{\pgfqpoint{3.328349in}{3.261345in}}{\pgfqpoint{3.323958in}{3.250746in}}{\pgfqpoint{3.323958in}{3.239696in}}%
\pgfpathcurveto{\pgfqpoint{3.323958in}{3.228646in}}{\pgfqpoint{3.328349in}{3.218046in}}{\pgfqpoint{3.336162in}{3.210233in}}%
\pgfpathcurveto{\pgfqpoint{3.343976in}{3.202419in}}{\pgfqpoint{3.354575in}{3.198029in}}{\pgfqpoint{3.365625in}{3.198029in}}%
\pgfpathclose%
\pgfusepath{stroke,fill}%
\end{pgfscope}%
\begin{pgfscope}%
\pgfpathrectangle{\pgfqpoint{0.648703in}{0.548769in}}{\pgfqpoint{5.201297in}{3.102590in}}%
\pgfusepath{clip}%
\pgfsetbuttcap%
\pgfsetroundjoin%
\definecolor{currentfill}{rgb}{1.000000,0.498039,0.054902}%
\pgfsetfillcolor{currentfill}%
\pgfsetlinewidth{1.003750pt}%
\definecolor{currentstroke}{rgb}{1.000000,0.498039,0.054902}%
\pgfsetstrokecolor{currentstroke}%
\pgfsetdash{}{0pt}%
\pgfpathmoveto{\pgfqpoint{1.892829in}{3.248773in}}%
\pgfpathcurveto{\pgfqpoint{1.903879in}{3.248773in}}{\pgfqpoint{1.914478in}{3.253164in}}{\pgfqpoint{1.922292in}{3.260977in}}%
\pgfpathcurveto{\pgfqpoint{1.930105in}{3.268791in}}{\pgfqpoint{1.934495in}{3.279390in}}{\pgfqpoint{1.934495in}{3.290440in}}%
\pgfpathcurveto{\pgfqpoint{1.934495in}{3.301490in}}{\pgfqpoint{1.930105in}{3.312089in}}{\pgfqpoint{1.922292in}{3.319903in}}%
\pgfpathcurveto{\pgfqpoint{1.914478in}{3.327716in}}{\pgfqpoint{1.903879in}{3.332107in}}{\pgfqpoint{1.892829in}{3.332107in}}%
\pgfpathcurveto{\pgfqpoint{1.881779in}{3.332107in}}{\pgfqpoint{1.871180in}{3.327716in}}{\pgfqpoint{1.863366in}{3.319903in}}%
\pgfpathcurveto{\pgfqpoint{1.855552in}{3.312089in}}{\pgfqpoint{1.851162in}{3.301490in}}{\pgfqpoint{1.851162in}{3.290440in}}%
\pgfpathcurveto{\pgfqpoint{1.851162in}{3.279390in}}{\pgfqpoint{1.855552in}{3.268791in}}{\pgfqpoint{1.863366in}{3.260977in}}%
\pgfpathcurveto{\pgfqpoint{1.871180in}{3.253164in}}{\pgfqpoint{1.881779in}{3.248773in}}{\pgfqpoint{1.892829in}{3.248773in}}%
\pgfpathclose%
\pgfusepath{stroke,fill}%
\end{pgfscope}%
\begin{pgfscope}%
\pgfpathrectangle{\pgfqpoint{0.648703in}{0.548769in}}{\pgfqpoint{5.201297in}{3.102590in}}%
\pgfusepath{clip}%
\pgfsetbuttcap%
\pgfsetroundjoin%
\definecolor{currentfill}{rgb}{1.000000,0.498039,0.054902}%
\pgfsetfillcolor{currentfill}%
\pgfsetlinewidth{1.003750pt}%
\definecolor{currentstroke}{rgb}{1.000000,0.498039,0.054902}%
\pgfsetstrokecolor{currentstroke}%
\pgfsetdash{}{0pt}%
\pgfpathmoveto{\pgfqpoint{2.047860in}{3.189572in}}%
\pgfpathcurveto{\pgfqpoint{2.058910in}{3.189572in}}{\pgfqpoint{2.069509in}{3.193962in}}{\pgfqpoint{2.077323in}{3.201775in}}%
\pgfpathcurveto{\pgfqpoint{2.085136in}{3.209589in}}{\pgfqpoint{2.089527in}{3.220188in}}{\pgfqpoint{2.089527in}{3.231238in}}%
\pgfpathcurveto{\pgfqpoint{2.089527in}{3.242288in}}{\pgfqpoint{2.085136in}{3.252887in}}{\pgfqpoint{2.077323in}{3.260701in}}%
\pgfpathcurveto{\pgfqpoint{2.069509in}{3.268515in}}{\pgfqpoint{2.058910in}{3.272905in}}{\pgfqpoint{2.047860in}{3.272905in}}%
\pgfpathcurveto{\pgfqpoint{2.036810in}{3.272905in}}{\pgfqpoint{2.026211in}{3.268515in}}{\pgfqpoint{2.018397in}{3.260701in}}%
\pgfpathcurveto{\pgfqpoint{2.010584in}{3.252887in}}{\pgfqpoint{2.006193in}{3.242288in}}{\pgfqpoint{2.006193in}{3.231238in}}%
\pgfpathcurveto{\pgfqpoint{2.006193in}{3.220188in}}{\pgfqpoint{2.010584in}{3.209589in}}{\pgfqpoint{2.018397in}{3.201775in}}%
\pgfpathcurveto{\pgfqpoint{2.026211in}{3.193962in}}{\pgfqpoint{2.036810in}{3.189572in}}{\pgfqpoint{2.047860in}{3.189572in}}%
\pgfpathclose%
\pgfusepath{stroke,fill}%
\end{pgfscope}%
\begin{pgfscope}%
\pgfpathrectangle{\pgfqpoint{0.648703in}{0.548769in}}{\pgfqpoint{5.201297in}{3.102590in}}%
\pgfusepath{clip}%
\pgfsetbuttcap%
\pgfsetroundjoin%
\definecolor{currentfill}{rgb}{1.000000,0.498039,0.054902}%
\pgfsetfillcolor{currentfill}%
\pgfsetlinewidth{1.003750pt}%
\definecolor{currentstroke}{rgb}{1.000000,0.498039,0.054902}%
\pgfsetstrokecolor{currentstroke}%
\pgfsetdash{}{0pt}%
\pgfpathmoveto{\pgfqpoint{1.272704in}{3.189572in}}%
\pgfpathcurveto{\pgfqpoint{1.283754in}{3.189572in}}{\pgfqpoint{1.294353in}{3.193962in}}{\pgfqpoint{1.302167in}{3.201775in}}%
\pgfpathcurveto{\pgfqpoint{1.309980in}{3.209589in}}{\pgfqpoint{1.314371in}{3.220188in}}{\pgfqpoint{1.314371in}{3.231238in}}%
\pgfpathcurveto{\pgfqpoint{1.314371in}{3.242288in}}{\pgfqpoint{1.309980in}{3.252887in}}{\pgfqpoint{1.302167in}{3.260701in}}%
\pgfpathcurveto{\pgfqpoint{1.294353in}{3.268515in}}{\pgfqpoint{1.283754in}{3.272905in}}{\pgfqpoint{1.272704in}{3.272905in}}%
\pgfpathcurveto{\pgfqpoint{1.261654in}{3.272905in}}{\pgfqpoint{1.251055in}{3.268515in}}{\pgfqpoint{1.243241in}{3.260701in}}%
\pgfpathcurveto{\pgfqpoint{1.235428in}{3.252887in}}{\pgfqpoint{1.231037in}{3.242288in}}{\pgfqpoint{1.231037in}{3.231238in}}%
\pgfpathcurveto{\pgfqpoint{1.231037in}{3.220188in}}{\pgfqpoint{1.235428in}{3.209589in}}{\pgfqpoint{1.243241in}{3.201775in}}%
\pgfpathcurveto{\pgfqpoint{1.251055in}{3.193962in}}{\pgfqpoint{1.261654in}{3.189572in}}{\pgfqpoint{1.272704in}{3.189572in}}%
\pgfpathclose%
\pgfusepath{stroke,fill}%
\end{pgfscope}%
\begin{pgfscope}%
\pgfpathrectangle{\pgfqpoint{0.648703in}{0.548769in}}{\pgfqpoint{5.201297in}{3.102590in}}%
\pgfusepath{clip}%
\pgfsetbuttcap%
\pgfsetroundjoin%
\definecolor{currentfill}{rgb}{0.121569,0.466667,0.705882}%
\pgfsetfillcolor{currentfill}%
\pgfsetlinewidth{1.003750pt}%
\definecolor{currentstroke}{rgb}{0.121569,0.466667,0.705882}%
\pgfsetstrokecolor{currentstroke}%
\pgfsetdash{}{0pt}%
\pgfpathmoveto{\pgfqpoint{4.450843in}{3.181114in}}%
\pgfpathcurveto{\pgfqpoint{4.461894in}{3.181114in}}{\pgfqpoint{4.472493in}{3.185504in}}{\pgfqpoint{4.480306in}{3.193318in}}%
\pgfpathcurveto{\pgfqpoint{4.488120in}{3.201132in}}{\pgfqpoint{4.492510in}{3.211731in}}{\pgfqpoint{4.492510in}{3.222781in}}%
\pgfpathcurveto{\pgfqpoint{4.492510in}{3.233831in}}{\pgfqpoint{4.488120in}{3.244430in}}{\pgfqpoint{4.480306in}{3.252244in}}%
\pgfpathcurveto{\pgfqpoint{4.472493in}{3.260057in}}{\pgfqpoint{4.461894in}{3.264448in}}{\pgfqpoint{4.450843in}{3.264448in}}%
\pgfpathcurveto{\pgfqpoint{4.439793in}{3.264448in}}{\pgfqpoint{4.429194in}{3.260057in}}{\pgfqpoint{4.421381in}{3.252244in}}%
\pgfpathcurveto{\pgfqpoint{4.413567in}{3.244430in}}{\pgfqpoint{4.409177in}{3.233831in}}{\pgfqpoint{4.409177in}{3.222781in}}%
\pgfpathcurveto{\pgfqpoint{4.409177in}{3.211731in}}{\pgfqpoint{4.413567in}{3.201132in}}{\pgfqpoint{4.421381in}{3.193318in}}%
\pgfpathcurveto{\pgfqpoint{4.429194in}{3.185504in}}{\pgfqpoint{4.439793in}{3.181114in}}{\pgfqpoint{4.450843in}{3.181114in}}%
\pgfpathclose%
\pgfusepath{stroke,fill}%
\end{pgfscope}%
\begin{pgfscope}%
\pgfpathrectangle{\pgfqpoint{0.648703in}{0.548769in}}{\pgfqpoint{5.201297in}{3.102590in}}%
\pgfusepath{clip}%
\pgfsetbuttcap%
\pgfsetroundjoin%
\definecolor{currentfill}{rgb}{0.121569,0.466667,0.705882}%
\pgfsetfillcolor{currentfill}%
\pgfsetlinewidth{1.003750pt}%
\definecolor{currentstroke}{rgb}{0.121569,0.466667,0.705882}%
\pgfsetstrokecolor{currentstroke}%
\pgfsetdash{}{0pt}%
\pgfpathmoveto{\pgfqpoint{1.737798in}{0.648129in}}%
\pgfpathcurveto{\pgfqpoint{1.748848in}{0.648129in}}{\pgfqpoint{1.759447in}{0.652519in}}{\pgfqpoint{1.767260in}{0.660333in}}%
\pgfpathcurveto{\pgfqpoint{1.775074in}{0.668146in}}{\pgfqpoint{1.779464in}{0.678745in}}{\pgfqpoint{1.779464in}{0.689796in}}%
\pgfpathcurveto{\pgfqpoint{1.779464in}{0.700846in}}{\pgfqpoint{1.775074in}{0.711445in}}{\pgfqpoint{1.767260in}{0.719258in}}%
\pgfpathcurveto{\pgfqpoint{1.759447in}{0.727072in}}{\pgfqpoint{1.748848in}{0.731462in}}{\pgfqpoint{1.737798in}{0.731462in}}%
\pgfpathcurveto{\pgfqpoint{1.726747in}{0.731462in}}{\pgfqpoint{1.716148in}{0.727072in}}{\pgfqpoint{1.708335in}{0.719258in}}%
\pgfpathcurveto{\pgfqpoint{1.700521in}{0.711445in}}{\pgfqpoint{1.696131in}{0.700846in}}{\pgfqpoint{1.696131in}{0.689796in}}%
\pgfpathcurveto{\pgfqpoint{1.696131in}{0.678745in}}{\pgfqpoint{1.700521in}{0.668146in}}{\pgfqpoint{1.708335in}{0.660333in}}%
\pgfpathcurveto{\pgfqpoint{1.716148in}{0.652519in}}{\pgfqpoint{1.726747in}{0.648129in}}{\pgfqpoint{1.737798in}{0.648129in}}%
\pgfpathclose%
\pgfusepath{stroke,fill}%
\end{pgfscope}%
\begin{pgfscope}%
\pgfpathrectangle{\pgfqpoint{0.648703in}{0.548769in}}{\pgfqpoint{5.201297in}{3.102590in}}%
\pgfusepath{clip}%
\pgfsetbuttcap%
\pgfsetroundjoin%
\definecolor{currentfill}{rgb}{0.839216,0.152941,0.156863}%
\pgfsetfillcolor{currentfill}%
\pgfsetlinewidth{1.003750pt}%
\definecolor{currentstroke}{rgb}{0.839216,0.152941,0.156863}%
\pgfsetstrokecolor{currentstroke}%
\pgfsetdash{}{0pt}%
\pgfpathmoveto{\pgfqpoint{4.373328in}{3.202258in}}%
\pgfpathcurveto{\pgfqpoint{4.384378in}{3.202258in}}{\pgfqpoint{4.394977in}{3.206648in}}{\pgfqpoint{4.402791in}{3.214462in}}%
\pgfpathcurveto{\pgfqpoint{4.410604in}{3.222275in}}{\pgfqpoint{4.414995in}{3.232874in}}{\pgfqpoint{4.414995in}{3.243924in}}%
\pgfpathcurveto{\pgfqpoint{4.414995in}{3.254974in}}{\pgfqpoint{4.410604in}{3.265573in}}{\pgfqpoint{4.402791in}{3.273387in}}%
\pgfpathcurveto{\pgfqpoint{4.394977in}{3.281201in}}{\pgfqpoint{4.384378in}{3.285591in}}{\pgfqpoint{4.373328in}{3.285591in}}%
\pgfpathcurveto{\pgfqpoint{4.362278in}{3.285591in}}{\pgfqpoint{4.351679in}{3.281201in}}{\pgfqpoint{4.343865in}{3.273387in}}%
\pgfpathcurveto{\pgfqpoint{4.336051in}{3.265573in}}{\pgfqpoint{4.331661in}{3.254974in}}{\pgfqpoint{4.331661in}{3.243924in}}%
\pgfpathcurveto{\pgfqpoint{4.331661in}{3.232874in}}{\pgfqpoint{4.336051in}{3.222275in}}{\pgfqpoint{4.343865in}{3.214462in}}%
\pgfpathcurveto{\pgfqpoint{4.351679in}{3.206648in}}{\pgfqpoint{4.362278in}{3.202258in}}{\pgfqpoint{4.373328in}{3.202258in}}%
\pgfpathclose%
\pgfusepath{stroke,fill}%
\end{pgfscope}%
\begin{pgfscope}%
\pgfpathrectangle{\pgfqpoint{0.648703in}{0.548769in}}{\pgfqpoint{5.201297in}{3.102590in}}%
\pgfusepath{clip}%
\pgfsetbuttcap%
\pgfsetroundjoin%
\definecolor{currentfill}{rgb}{1.000000,0.498039,0.054902}%
\pgfsetfillcolor{currentfill}%
\pgfsetlinewidth{1.003750pt}%
\definecolor{currentstroke}{rgb}{1.000000,0.498039,0.054902}%
\pgfsetstrokecolor{currentstroke}%
\pgfsetdash{}{0pt}%
\pgfpathmoveto{\pgfqpoint{1.505251in}{3.185343in}}%
\pgfpathcurveto{\pgfqpoint{1.516301in}{3.185343in}}{\pgfqpoint{1.526900in}{3.189733in}}{\pgfqpoint{1.534714in}{3.197547in}}%
\pgfpathcurveto{\pgfqpoint{1.542527in}{3.205360in}}{\pgfqpoint{1.546917in}{3.215959in}}{\pgfqpoint{1.546917in}{3.227010in}}%
\pgfpathcurveto{\pgfqpoint{1.546917in}{3.238060in}}{\pgfqpoint{1.542527in}{3.248659in}}{\pgfqpoint{1.534714in}{3.256472in}}%
\pgfpathcurveto{\pgfqpoint{1.526900in}{3.264286in}}{\pgfqpoint{1.516301in}{3.268676in}}{\pgfqpoint{1.505251in}{3.268676in}}%
\pgfpathcurveto{\pgfqpoint{1.494201in}{3.268676in}}{\pgfqpoint{1.483602in}{3.264286in}}{\pgfqpoint{1.475788in}{3.256472in}}%
\pgfpathcurveto{\pgfqpoint{1.467974in}{3.248659in}}{\pgfqpoint{1.463584in}{3.238060in}}{\pgfqpoint{1.463584in}{3.227010in}}%
\pgfpathcurveto{\pgfqpoint{1.463584in}{3.215959in}}{\pgfqpoint{1.467974in}{3.205360in}}{\pgfqpoint{1.475788in}{3.197547in}}%
\pgfpathcurveto{\pgfqpoint{1.483602in}{3.189733in}}{\pgfqpoint{1.494201in}{3.185343in}}{\pgfqpoint{1.505251in}{3.185343in}}%
\pgfpathclose%
\pgfusepath{stroke,fill}%
\end{pgfscope}%
\begin{pgfscope}%
\pgfpathrectangle{\pgfqpoint{0.648703in}{0.548769in}}{\pgfqpoint{5.201297in}{3.102590in}}%
\pgfusepath{clip}%
\pgfsetbuttcap%
\pgfsetroundjoin%
\definecolor{currentfill}{rgb}{0.121569,0.466667,0.705882}%
\pgfsetfillcolor{currentfill}%
\pgfsetlinewidth{1.003750pt}%
\definecolor{currentstroke}{rgb}{0.121569,0.466667,0.705882}%
\pgfsetstrokecolor{currentstroke}%
\pgfsetdash{}{0pt}%
\pgfpathmoveto{\pgfqpoint{1.350220in}{0.774990in}}%
\pgfpathcurveto{\pgfqpoint{1.361270in}{0.774990in}}{\pgfqpoint{1.371869in}{0.779380in}}{\pgfqpoint{1.379682in}{0.787194in}}%
\pgfpathcurveto{\pgfqpoint{1.387496in}{0.795007in}}{\pgfqpoint{1.391886in}{0.805606in}}{\pgfqpoint{1.391886in}{0.816656in}}%
\pgfpathcurveto{\pgfqpoint{1.391886in}{0.827706in}}{\pgfqpoint{1.387496in}{0.838305in}}{\pgfqpoint{1.379682in}{0.846119in}}%
\pgfpathcurveto{\pgfqpoint{1.371869in}{0.853933in}}{\pgfqpoint{1.361270in}{0.858323in}}{\pgfqpoint{1.350220in}{0.858323in}}%
\pgfpathcurveto{\pgfqpoint{1.339169in}{0.858323in}}{\pgfqpoint{1.328570in}{0.853933in}}{\pgfqpoint{1.320757in}{0.846119in}}%
\pgfpathcurveto{\pgfqpoint{1.312943in}{0.838305in}}{\pgfqpoint{1.308553in}{0.827706in}}{\pgfqpoint{1.308553in}{0.816656in}}%
\pgfpathcurveto{\pgfqpoint{1.308553in}{0.805606in}}{\pgfqpoint{1.312943in}{0.795007in}}{\pgfqpoint{1.320757in}{0.787194in}}%
\pgfpathcurveto{\pgfqpoint{1.328570in}{0.779380in}}{\pgfqpoint{1.339169in}{0.774990in}}{\pgfqpoint{1.350220in}{0.774990in}}%
\pgfpathclose%
\pgfusepath{stroke,fill}%
\end{pgfscope}%
\begin{pgfscope}%
\pgfpathrectangle{\pgfqpoint{0.648703in}{0.548769in}}{\pgfqpoint{5.201297in}{3.102590in}}%
\pgfusepath{clip}%
\pgfsetbuttcap%
\pgfsetroundjoin%
\definecolor{currentfill}{rgb}{0.121569,0.466667,0.705882}%
\pgfsetfillcolor{currentfill}%
\pgfsetlinewidth{1.003750pt}%
\definecolor{currentstroke}{rgb}{0.121569,0.466667,0.705882}%
\pgfsetstrokecolor{currentstroke}%
\pgfsetdash{}{0pt}%
\pgfpathmoveto{\pgfqpoint{2.978047in}{3.181114in}}%
\pgfpathcurveto{\pgfqpoint{2.989097in}{3.181114in}}{\pgfqpoint{2.999696in}{3.185504in}}{\pgfqpoint{3.007510in}{3.193318in}}%
\pgfpathcurveto{\pgfqpoint{3.015324in}{3.201132in}}{\pgfqpoint{3.019714in}{3.211731in}}{\pgfqpoint{3.019714in}{3.222781in}}%
\pgfpathcurveto{\pgfqpoint{3.019714in}{3.233831in}}{\pgfqpoint{3.015324in}{3.244430in}}{\pgfqpoint{3.007510in}{3.252244in}}%
\pgfpathcurveto{\pgfqpoint{2.999696in}{3.260057in}}{\pgfqpoint{2.989097in}{3.264448in}}{\pgfqpoint{2.978047in}{3.264448in}}%
\pgfpathcurveto{\pgfqpoint{2.966997in}{3.264448in}}{\pgfqpoint{2.956398in}{3.260057in}}{\pgfqpoint{2.948584in}{3.252244in}}%
\pgfpathcurveto{\pgfqpoint{2.940771in}{3.244430in}}{\pgfqpoint{2.936380in}{3.233831in}}{\pgfqpoint{2.936380in}{3.222781in}}%
\pgfpathcurveto{\pgfqpoint{2.936380in}{3.211731in}}{\pgfqpoint{2.940771in}{3.201132in}}{\pgfqpoint{2.948584in}{3.193318in}}%
\pgfpathcurveto{\pgfqpoint{2.956398in}{3.185504in}}{\pgfqpoint{2.966997in}{3.181114in}}{\pgfqpoint{2.978047in}{3.181114in}}%
\pgfpathclose%
\pgfusepath{stroke,fill}%
\end{pgfscope}%
\begin{pgfscope}%
\pgfpathrectangle{\pgfqpoint{0.648703in}{0.548769in}}{\pgfqpoint{5.201297in}{3.102590in}}%
\pgfusepath{clip}%
\pgfsetbuttcap%
\pgfsetroundjoin%
\definecolor{currentfill}{rgb}{0.121569,0.466667,0.705882}%
\pgfsetfillcolor{currentfill}%
\pgfsetlinewidth{1.003750pt}%
\definecolor{currentstroke}{rgb}{0.121569,0.466667,0.705882}%
\pgfsetstrokecolor{currentstroke}%
\pgfsetdash{}{0pt}%
\pgfpathmoveto{\pgfqpoint{1.350220in}{0.783447in}}%
\pgfpathcurveto{\pgfqpoint{1.361270in}{0.783447in}}{\pgfqpoint{1.371869in}{0.787837in}}{\pgfqpoint{1.379682in}{0.795651in}}%
\pgfpathcurveto{\pgfqpoint{1.387496in}{0.803465in}}{\pgfqpoint{1.391886in}{0.814064in}}{\pgfqpoint{1.391886in}{0.825114in}}%
\pgfpathcurveto{\pgfqpoint{1.391886in}{0.836164in}}{\pgfqpoint{1.387496in}{0.846763in}}{\pgfqpoint{1.379682in}{0.854576in}}%
\pgfpathcurveto{\pgfqpoint{1.371869in}{0.862390in}}{\pgfqpoint{1.361270in}{0.866780in}}{\pgfqpoint{1.350220in}{0.866780in}}%
\pgfpathcurveto{\pgfqpoint{1.339169in}{0.866780in}}{\pgfqpoint{1.328570in}{0.862390in}}{\pgfqpoint{1.320757in}{0.854576in}}%
\pgfpathcurveto{\pgfqpoint{1.312943in}{0.846763in}}{\pgfqpoint{1.308553in}{0.836164in}}{\pgfqpoint{1.308553in}{0.825114in}}%
\pgfpathcurveto{\pgfqpoint{1.308553in}{0.814064in}}{\pgfqpoint{1.312943in}{0.803465in}}{\pgfqpoint{1.320757in}{0.795651in}}%
\pgfpathcurveto{\pgfqpoint{1.328570in}{0.787837in}}{\pgfqpoint{1.339169in}{0.783447in}}{\pgfqpoint{1.350220in}{0.783447in}}%
\pgfpathclose%
\pgfusepath{stroke,fill}%
\end{pgfscope}%
\begin{pgfscope}%
\pgfpathrectangle{\pgfqpoint{0.648703in}{0.548769in}}{\pgfqpoint{5.201297in}{3.102590in}}%
\pgfusepath{clip}%
\pgfsetbuttcap%
\pgfsetroundjoin%
\definecolor{currentfill}{rgb}{0.121569,0.466667,0.705882}%
\pgfsetfillcolor{currentfill}%
\pgfsetlinewidth{1.003750pt}%
\definecolor{currentstroke}{rgb}{0.121569,0.466667,0.705882}%
\pgfsetstrokecolor{currentstroke}%
\pgfsetdash{}{0pt}%
\pgfpathmoveto{\pgfqpoint{0.962642in}{0.652358in}}%
\pgfpathcurveto{\pgfqpoint{0.973692in}{0.652358in}}{\pgfqpoint{0.984291in}{0.656748in}}{\pgfqpoint{0.992104in}{0.664562in}}%
\pgfpathcurveto{\pgfqpoint{0.999918in}{0.672375in}}{\pgfqpoint{1.004308in}{0.682974in}}{\pgfqpoint{1.004308in}{0.694024in}}%
\pgfpathcurveto{\pgfqpoint{1.004308in}{0.705074in}}{\pgfqpoint{0.999918in}{0.715673in}}{\pgfqpoint{0.992104in}{0.723487in}}%
\pgfpathcurveto{\pgfqpoint{0.984291in}{0.731301in}}{\pgfqpoint{0.973692in}{0.735691in}}{\pgfqpoint{0.962642in}{0.735691in}}%
\pgfpathcurveto{\pgfqpoint{0.951591in}{0.735691in}}{\pgfqpoint{0.940992in}{0.731301in}}{\pgfqpoint{0.933179in}{0.723487in}}%
\pgfpathcurveto{\pgfqpoint{0.925365in}{0.715673in}}{\pgfqpoint{0.920975in}{0.705074in}}{\pgfqpoint{0.920975in}{0.694024in}}%
\pgfpathcurveto{\pgfqpoint{0.920975in}{0.682974in}}{\pgfqpoint{0.925365in}{0.672375in}}{\pgfqpoint{0.933179in}{0.664562in}}%
\pgfpathcurveto{\pgfqpoint{0.940992in}{0.656748in}}{\pgfqpoint{0.951591in}{0.652358in}}{\pgfqpoint{0.962642in}{0.652358in}}%
\pgfpathclose%
\pgfusepath{stroke,fill}%
\end{pgfscope}%
\begin{pgfscope}%
\pgfpathrectangle{\pgfqpoint{0.648703in}{0.548769in}}{\pgfqpoint{5.201297in}{3.102590in}}%
\pgfusepath{clip}%
\pgfsetbuttcap%
\pgfsetroundjoin%
\definecolor{currentfill}{rgb}{1.000000,0.498039,0.054902}%
\pgfsetfillcolor{currentfill}%
\pgfsetlinewidth{1.003750pt}%
\definecolor{currentstroke}{rgb}{1.000000,0.498039,0.054902}%
\pgfsetstrokecolor{currentstroke}%
\pgfsetdash{}{0pt}%
\pgfpathmoveto{\pgfqpoint{1.040157in}{3.219172in}}%
\pgfpathcurveto{\pgfqpoint{1.051207in}{3.219172in}}{\pgfqpoint{1.061806in}{3.223563in}}{\pgfqpoint{1.069620in}{3.231376in}}%
\pgfpathcurveto{\pgfqpoint{1.077434in}{3.239190in}}{\pgfqpoint{1.081824in}{3.249789in}}{\pgfqpoint{1.081824in}{3.260839in}}%
\pgfpathcurveto{\pgfqpoint{1.081824in}{3.271889in}}{\pgfqpoint{1.077434in}{3.282488in}}{\pgfqpoint{1.069620in}{3.290302in}}%
\pgfpathcurveto{\pgfqpoint{1.061806in}{3.298116in}}{\pgfqpoint{1.051207in}{3.302506in}}{\pgfqpoint{1.040157in}{3.302506in}}%
\pgfpathcurveto{\pgfqpoint{1.029107in}{3.302506in}}{\pgfqpoint{1.018508in}{3.298116in}}{\pgfqpoint{1.010694in}{3.290302in}}%
\pgfpathcurveto{\pgfqpoint{1.002881in}{3.282488in}}{\pgfqpoint{0.998491in}{3.271889in}}{\pgfqpoint{0.998491in}{3.260839in}}%
\pgfpathcurveto{\pgfqpoint{0.998491in}{3.249789in}}{\pgfqpoint{1.002881in}{3.239190in}}{\pgfqpoint{1.010694in}{3.231376in}}%
\pgfpathcurveto{\pgfqpoint{1.018508in}{3.223563in}}{\pgfqpoint{1.029107in}{3.219172in}}{\pgfqpoint{1.040157in}{3.219172in}}%
\pgfpathclose%
\pgfusepath{stroke,fill}%
\end{pgfscope}%
\begin{pgfscope}%
\pgfpathrectangle{\pgfqpoint{0.648703in}{0.548769in}}{\pgfqpoint{5.201297in}{3.102590in}}%
\pgfusepath{clip}%
\pgfsetbuttcap%
\pgfsetroundjoin%
\definecolor{currentfill}{rgb}{0.121569,0.466667,0.705882}%
\pgfsetfillcolor{currentfill}%
\pgfsetlinewidth{1.003750pt}%
\definecolor{currentstroke}{rgb}{0.121569,0.466667,0.705882}%
\pgfsetstrokecolor{currentstroke}%
\pgfsetdash{}{0pt}%
\pgfpathmoveto{\pgfqpoint{1.505251in}{0.648129in}}%
\pgfpathcurveto{\pgfqpoint{1.516301in}{0.648129in}}{\pgfqpoint{1.526900in}{0.652519in}}{\pgfqpoint{1.534714in}{0.660333in}}%
\pgfpathcurveto{\pgfqpoint{1.542527in}{0.668146in}}{\pgfqpoint{1.546917in}{0.678745in}}{\pgfqpoint{1.546917in}{0.689796in}}%
\pgfpathcurveto{\pgfqpoint{1.546917in}{0.700846in}}{\pgfqpoint{1.542527in}{0.711445in}}{\pgfqpoint{1.534714in}{0.719258in}}%
\pgfpathcurveto{\pgfqpoint{1.526900in}{0.727072in}}{\pgfqpoint{1.516301in}{0.731462in}}{\pgfqpoint{1.505251in}{0.731462in}}%
\pgfpathcurveto{\pgfqpoint{1.494201in}{0.731462in}}{\pgfqpoint{1.483602in}{0.727072in}}{\pgfqpoint{1.475788in}{0.719258in}}%
\pgfpathcurveto{\pgfqpoint{1.467974in}{0.711445in}}{\pgfqpoint{1.463584in}{0.700846in}}{\pgfqpoint{1.463584in}{0.689796in}}%
\pgfpathcurveto{\pgfqpoint{1.463584in}{0.678745in}}{\pgfqpoint{1.467974in}{0.668146in}}{\pgfqpoint{1.475788in}{0.660333in}}%
\pgfpathcurveto{\pgfqpoint{1.483602in}{0.652519in}}{\pgfqpoint{1.494201in}{0.648129in}}{\pgfqpoint{1.505251in}{0.648129in}}%
\pgfpathclose%
\pgfusepath{stroke,fill}%
\end{pgfscope}%
\begin{pgfscope}%
\pgfpathrectangle{\pgfqpoint{0.648703in}{0.548769in}}{\pgfqpoint{5.201297in}{3.102590in}}%
\pgfusepath{clip}%
\pgfsetbuttcap%
\pgfsetroundjoin%
\definecolor{currentfill}{rgb}{0.121569,0.466667,0.705882}%
\pgfsetfillcolor{currentfill}%
\pgfsetlinewidth{1.003750pt}%
\definecolor{currentstroke}{rgb}{0.121569,0.466667,0.705882}%
\pgfsetstrokecolor{currentstroke}%
\pgfsetdash{}{0pt}%
\pgfpathmoveto{\pgfqpoint{1.272704in}{0.648129in}}%
\pgfpathcurveto{\pgfqpoint{1.283754in}{0.648129in}}{\pgfqpoint{1.294353in}{0.652519in}}{\pgfqpoint{1.302167in}{0.660333in}}%
\pgfpathcurveto{\pgfqpoint{1.309980in}{0.668146in}}{\pgfqpoint{1.314371in}{0.678745in}}{\pgfqpoint{1.314371in}{0.689796in}}%
\pgfpathcurveto{\pgfqpoint{1.314371in}{0.700846in}}{\pgfqpoint{1.309980in}{0.711445in}}{\pgfqpoint{1.302167in}{0.719258in}}%
\pgfpathcurveto{\pgfqpoint{1.294353in}{0.727072in}}{\pgfqpoint{1.283754in}{0.731462in}}{\pgfqpoint{1.272704in}{0.731462in}}%
\pgfpathcurveto{\pgfqpoint{1.261654in}{0.731462in}}{\pgfqpoint{1.251055in}{0.727072in}}{\pgfqpoint{1.243241in}{0.719258in}}%
\pgfpathcurveto{\pgfqpoint{1.235428in}{0.711445in}}{\pgfqpoint{1.231037in}{0.700846in}}{\pgfqpoint{1.231037in}{0.689796in}}%
\pgfpathcurveto{\pgfqpoint{1.231037in}{0.678745in}}{\pgfqpoint{1.235428in}{0.668146in}}{\pgfqpoint{1.243241in}{0.660333in}}%
\pgfpathcurveto{\pgfqpoint{1.251055in}{0.652519in}}{\pgfqpoint{1.261654in}{0.648129in}}{\pgfqpoint{1.272704in}{0.648129in}}%
\pgfpathclose%
\pgfusepath{stroke,fill}%
\end{pgfscope}%
\begin{pgfscope}%
\pgfpathrectangle{\pgfqpoint{0.648703in}{0.548769in}}{\pgfqpoint{5.201297in}{3.102590in}}%
\pgfusepath{clip}%
\pgfsetbuttcap%
\pgfsetroundjoin%
\definecolor{currentfill}{rgb}{1.000000,0.498039,0.054902}%
\pgfsetfillcolor{currentfill}%
\pgfsetlinewidth{1.003750pt}%
\definecolor{currentstroke}{rgb}{1.000000,0.498039,0.054902}%
\pgfsetstrokecolor{currentstroke}%
\pgfsetdash{}{0pt}%
\pgfpathmoveto{\pgfqpoint{2.745500in}{3.468665in}}%
\pgfpathcurveto{\pgfqpoint{2.756550in}{3.468665in}}{\pgfqpoint{2.767149in}{3.473055in}}{\pgfqpoint{2.774963in}{3.480869in}}%
\pgfpathcurveto{\pgfqpoint{2.782777in}{3.488683in}}{\pgfqpoint{2.787167in}{3.499282in}}{\pgfqpoint{2.787167in}{3.510332in}}%
\pgfpathcurveto{\pgfqpoint{2.787167in}{3.521382in}}{\pgfqpoint{2.782777in}{3.531981in}}{\pgfqpoint{2.774963in}{3.539795in}}%
\pgfpathcurveto{\pgfqpoint{2.767149in}{3.547608in}}{\pgfqpoint{2.756550in}{3.551998in}}{\pgfqpoint{2.745500in}{3.551998in}}%
\pgfpathcurveto{\pgfqpoint{2.734450in}{3.551998in}}{\pgfqpoint{2.723851in}{3.547608in}}{\pgfqpoint{2.716038in}{3.539795in}}%
\pgfpathcurveto{\pgfqpoint{2.708224in}{3.531981in}}{\pgfqpoint{2.703834in}{3.521382in}}{\pgfqpoint{2.703834in}{3.510332in}}%
\pgfpathcurveto{\pgfqpoint{2.703834in}{3.499282in}}{\pgfqpoint{2.708224in}{3.488683in}}{\pgfqpoint{2.716038in}{3.480869in}}%
\pgfpathcurveto{\pgfqpoint{2.723851in}{3.473055in}}{\pgfqpoint{2.734450in}{3.468665in}}{\pgfqpoint{2.745500in}{3.468665in}}%
\pgfpathclose%
\pgfusepath{stroke,fill}%
\end{pgfscope}%
\begin{pgfscope}%
\pgfpathrectangle{\pgfqpoint{0.648703in}{0.548769in}}{\pgfqpoint{5.201297in}{3.102590in}}%
\pgfusepath{clip}%
\pgfsetbuttcap%
\pgfsetroundjoin%
\definecolor{currentfill}{rgb}{1.000000,0.498039,0.054902}%
\pgfsetfillcolor{currentfill}%
\pgfsetlinewidth{1.003750pt}%
\definecolor{currentstroke}{rgb}{1.000000,0.498039,0.054902}%
\pgfsetstrokecolor{currentstroke}%
\pgfsetdash{}{0pt}%
\pgfpathmoveto{\pgfqpoint{1.505251in}{3.193800in}}%
\pgfpathcurveto{\pgfqpoint{1.516301in}{3.193800in}}{\pgfqpoint{1.526900in}{3.198191in}}{\pgfqpoint{1.534714in}{3.206004in}}%
\pgfpathcurveto{\pgfqpoint{1.542527in}{3.213818in}}{\pgfqpoint{1.546917in}{3.224417in}}{\pgfqpoint{1.546917in}{3.235467in}}%
\pgfpathcurveto{\pgfqpoint{1.546917in}{3.246517in}}{\pgfqpoint{1.542527in}{3.257116in}}{\pgfqpoint{1.534714in}{3.264930in}}%
\pgfpathcurveto{\pgfqpoint{1.526900in}{3.272743in}}{\pgfqpoint{1.516301in}{3.277134in}}{\pgfqpoint{1.505251in}{3.277134in}}%
\pgfpathcurveto{\pgfqpoint{1.494201in}{3.277134in}}{\pgfqpoint{1.483602in}{3.272743in}}{\pgfqpoint{1.475788in}{3.264930in}}%
\pgfpathcurveto{\pgfqpoint{1.467974in}{3.257116in}}{\pgfqpoint{1.463584in}{3.246517in}}{\pgfqpoint{1.463584in}{3.235467in}}%
\pgfpathcurveto{\pgfqpoint{1.463584in}{3.224417in}}{\pgfqpoint{1.467974in}{3.213818in}}{\pgfqpoint{1.475788in}{3.206004in}}%
\pgfpathcurveto{\pgfqpoint{1.483602in}{3.198191in}}{\pgfqpoint{1.494201in}{3.193800in}}{\pgfqpoint{1.505251in}{3.193800in}}%
\pgfpathclose%
\pgfusepath{stroke,fill}%
\end{pgfscope}%
\begin{pgfscope}%
\pgfpathrectangle{\pgfqpoint{0.648703in}{0.548769in}}{\pgfqpoint{5.201297in}{3.102590in}}%
\pgfusepath{clip}%
\pgfsetbuttcap%
\pgfsetroundjoin%
\definecolor{currentfill}{rgb}{1.000000,0.498039,0.054902}%
\pgfsetfillcolor{currentfill}%
\pgfsetlinewidth{1.003750pt}%
\definecolor{currentstroke}{rgb}{1.000000,0.498039,0.054902}%
\pgfsetstrokecolor{currentstroke}%
\pgfsetdash{}{0pt}%
\pgfpathmoveto{\pgfqpoint{2.667985in}{3.231859in}}%
\pgfpathcurveto{\pgfqpoint{2.679035in}{3.231859in}}{\pgfqpoint{2.689634in}{3.236249in}}{\pgfqpoint{2.697448in}{3.244062in}}%
\pgfpathcurveto{\pgfqpoint{2.705261in}{3.251876in}}{\pgfqpoint{2.709651in}{3.262475in}}{\pgfqpoint{2.709651in}{3.273525in}}%
\pgfpathcurveto{\pgfqpoint{2.709651in}{3.284575in}}{\pgfqpoint{2.705261in}{3.295174in}}{\pgfqpoint{2.697448in}{3.302988in}}%
\pgfpathcurveto{\pgfqpoint{2.689634in}{3.310802in}}{\pgfqpoint{2.679035in}{3.315192in}}{\pgfqpoint{2.667985in}{3.315192in}}%
\pgfpathcurveto{\pgfqpoint{2.656935in}{3.315192in}}{\pgfqpoint{2.646336in}{3.310802in}}{\pgfqpoint{2.638522in}{3.302988in}}%
\pgfpathcurveto{\pgfqpoint{2.630708in}{3.295174in}}{\pgfqpoint{2.626318in}{3.284575in}}{\pgfqpoint{2.626318in}{3.273525in}}%
\pgfpathcurveto{\pgfqpoint{2.626318in}{3.262475in}}{\pgfqpoint{2.630708in}{3.251876in}}{\pgfqpoint{2.638522in}{3.244062in}}%
\pgfpathcurveto{\pgfqpoint{2.646336in}{3.236249in}}{\pgfqpoint{2.656935in}{3.231859in}}{\pgfqpoint{2.667985in}{3.231859in}}%
\pgfpathclose%
\pgfusepath{stroke,fill}%
\end{pgfscope}%
\begin{pgfscope}%
\pgfpathrectangle{\pgfqpoint{0.648703in}{0.548769in}}{\pgfqpoint{5.201297in}{3.102590in}}%
\pgfusepath{clip}%
\pgfsetbuttcap%
\pgfsetroundjoin%
\definecolor{currentfill}{rgb}{0.121569,0.466667,0.705882}%
\pgfsetfillcolor{currentfill}%
\pgfsetlinewidth{1.003750pt}%
\definecolor{currentstroke}{rgb}{0.121569,0.466667,0.705882}%
\pgfsetstrokecolor{currentstroke}%
\pgfsetdash{}{0pt}%
\pgfpathmoveto{\pgfqpoint{1.815313in}{0.681958in}}%
\pgfpathcurveto{\pgfqpoint{1.826363in}{0.681958in}}{\pgfqpoint{1.836962in}{0.686349in}}{\pgfqpoint{1.844776in}{0.694162in}}%
\pgfpathcurveto{\pgfqpoint{1.852590in}{0.701976in}}{\pgfqpoint{1.856980in}{0.712575in}}{\pgfqpoint{1.856980in}{0.723625in}}%
\pgfpathcurveto{\pgfqpoint{1.856980in}{0.734675in}}{\pgfqpoint{1.852590in}{0.745274in}}{\pgfqpoint{1.844776in}{0.753088in}}%
\pgfpathcurveto{\pgfqpoint{1.836962in}{0.760902in}}{\pgfqpoint{1.826363in}{0.765292in}}{\pgfqpoint{1.815313in}{0.765292in}}%
\pgfpathcurveto{\pgfqpoint{1.804263in}{0.765292in}}{\pgfqpoint{1.793664in}{0.760902in}}{\pgfqpoint{1.785850in}{0.753088in}}%
\pgfpathcurveto{\pgfqpoint{1.778037in}{0.745274in}}{\pgfqpoint{1.773646in}{0.734675in}}{\pgfqpoint{1.773646in}{0.723625in}}%
\pgfpathcurveto{\pgfqpoint{1.773646in}{0.712575in}}{\pgfqpoint{1.778037in}{0.701976in}}{\pgfqpoint{1.785850in}{0.694162in}}%
\pgfpathcurveto{\pgfqpoint{1.793664in}{0.686349in}}{\pgfqpoint{1.804263in}{0.681958in}}{\pgfqpoint{1.815313in}{0.681958in}}%
\pgfpathclose%
\pgfusepath{stroke,fill}%
\end{pgfscope}%
\begin{pgfscope}%
\pgfpathrectangle{\pgfqpoint{0.648703in}{0.548769in}}{\pgfqpoint{5.201297in}{3.102590in}}%
\pgfusepath{clip}%
\pgfsetbuttcap%
\pgfsetroundjoin%
\definecolor{currentfill}{rgb}{0.121569,0.466667,0.705882}%
\pgfsetfillcolor{currentfill}%
\pgfsetlinewidth{1.003750pt}%
\definecolor{currentstroke}{rgb}{0.121569,0.466667,0.705882}%
\pgfsetstrokecolor{currentstroke}%
\pgfsetdash{}{0pt}%
\pgfpathmoveto{\pgfqpoint{1.582766in}{0.758075in}}%
\pgfpathcurveto{\pgfqpoint{1.593816in}{0.758075in}}{\pgfqpoint{1.604416in}{0.762465in}}{\pgfqpoint{1.612229in}{0.770279in}}%
\pgfpathcurveto{\pgfqpoint{1.620043in}{0.778092in}}{\pgfqpoint{1.624433in}{0.788691in}}{\pgfqpoint{1.624433in}{0.799742in}}%
\pgfpathcurveto{\pgfqpoint{1.624433in}{0.810792in}}{\pgfqpoint{1.620043in}{0.821391in}}{\pgfqpoint{1.612229in}{0.829204in}}%
\pgfpathcurveto{\pgfqpoint{1.604416in}{0.837018in}}{\pgfqpoint{1.593816in}{0.841408in}}{\pgfqpoint{1.582766in}{0.841408in}}%
\pgfpathcurveto{\pgfqpoint{1.571716in}{0.841408in}}{\pgfqpoint{1.561117in}{0.837018in}}{\pgfqpoint{1.553304in}{0.829204in}}%
\pgfpathcurveto{\pgfqpoint{1.545490in}{0.821391in}}{\pgfqpoint{1.541100in}{0.810792in}}{\pgfqpoint{1.541100in}{0.799742in}}%
\pgfpathcurveto{\pgfqpoint{1.541100in}{0.788691in}}{\pgfqpoint{1.545490in}{0.778092in}}{\pgfqpoint{1.553304in}{0.770279in}}%
\pgfpathcurveto{\pgfqpoint{1.561117in}{0.762465in}}{\pgfqpoint{1.571716in}{0.758075in}}{\pgfqpoint{1.582766in}{0.758075in}}%
\pgfpathclose%
\pgfusepath{stroke,fill}%
\end{pgfscope}%
\begin{pgfscope}%
\pgfpathrectangle{\pgfqpoint{0.648703in}{0.548769in}}{\pgfqpoint{5.201297in}{3.102590in}}%
\pgfusepath{clip}%
\pgfsetbuttcap%
\pgfsetroundjoin%
\definecolor{currentfill}{rgb}{1.000000,0.498039,0.054902}%
\pgfsetfillcolor{currentfill}%
\pgfsetlinewidth{1.003750pt}%
\definecolor{currentstroke}{rgb}{1.000000,0.498039,0.054902}%
\pgfsetstrokecolor{currentstroke}%
\pgfsetdash{}{0pt}%
\pgfpathmoveto{\pgfqpoint{1.582766in}{3.185343in}}%
\pgfpathcurveto{\pgfqpoint{1.593816in}{3.185343in}}{\pgfqpoint{1.604416in}{3.189733in}}{\pgfqpoint{1.612229in}{3.197547in}}%
\pgfpathcurveto{\pgfqpoint{1.620043in}{3.205360in}}{\pgfqpoint{1.624433in}{3.215959in}}{\pgfqpoint{1.624433in}{3.227010in}}%
\pgfpathcurveto{\pgfqpoint{1.624433in}{3.238060in}}{\pgfqpoint{1.620043in}{3.248659in}}{\pgfqpoint{1.612229in}{3.256472in}}%
\pgfpathcurveto{\pgfqpoint{1.604416in}{3.264286in}}{\pgfqpoint{1.593816in}{3.268676in}}{\pgfqpoint{1.582766in}{3.268676in}}%
\pgfpathcurveto{\pgfqpoint{1.571716in}{3.268676in}}{\pgfqpoint{1.561117in}{3.264286in}}{\pgfqpoint{1.553304in}{3.256472in}}%
\pgfpathcurveto{\pgfqpoint{1.545490in}{3.248659in}}{\pgfqpoint{1.541100in}{3.238060in}}{\pgfqpoint{1.541100in}{3.227010in}}%
\pgfpathcurveto{\pgfqpoint{1.541100in}{3.215959in}}{\pgfqpoint{1.545490in}{3.205360in}}{\pgfqpoint{1.553304in}{3.197547in}}%
\pgfpathcurveto{\pgfqpoint{1.561117in}{3.189733in}}{\pgfqpoint{1.571716in}{3.185343in}}{\pgfqpoint{1.582766in}{3.185343in}}%
\pgfpathclose%
\pgfusepath{stroke,fill}%
\end{pgfscope}%
\begin{pgfscope}%
\pgfpathrectangle{\pgfqpoint{0.648703in}{0.548769in}}{\pgfqpoint{5.201297in}{3.102590in}}%
\pgfusepath{clip}%
\pgfsetbuttcap%
\pgfsetroundjoin%
\definecolor{currentfill}{rgb}{1.000000,0.498039,0.054902}%
\pgfsetfillcolor{currentfill}%
\pgfsetlinewidth{1.003750pt}%
\definecolor{currentstroke}{rgb}{1.000000,0.498039,0.054902}%
\pgfsetstrokecolor{currentstroke}%
\pgfsetdash{}{0pt}%
\pgfpathmoveto{\pgfqpoint{1.660282in}{3.202258in}}%
\pgfpathcurveto{\pgfqpoint{1.671332in}{3.202258in}}{\pgfqpoint{1.681931in}{3.206648in}}{\pgfqpoint{1.689745in}{3.214462in}}%
\pgfpathcurveto{\pgfqpoint{1.697558in}{3.222275in}}{\pgfqpoint{1.701949in}{3.232874in}}{\pgfqpoint{1.701949in}{3.243924in}}%
\pgfpathcurveto{\pgfqpoint{1.701949in}{3.254974in}}{\pgfqpoint{1.697558in}{3.265573in}}{\pgfqpoint{1.689745in}{3.273387in}}%
\pgfpathcurveto{\pgfqpoint{1.681931in}{3.281201in}}{\pgfqpoint{1.671332in}{3.285591in}}{\pgfqpoint{1.660282in}{3.285591in}}%
\pgfpathcurveto{\pgfqpoint{1.649232in}{3.285591in}}{\pgfqpoint{1.638633in}{3.281201in}}{\pgfqpoint{1.630819in}{3.273387in}}%
\pgfpathcurveto{\pgfqpoint{1.623006in}{3.265573in}}{\pgfqpoint{1.618615in}{3.254974in}}{\pgfqpoint{1.618615in}{3.243924in}}%
\pgfpathcurveto{\pgfqpoint{1.618615in}{3.232874in}}{\pgfqpoint{1.623006in}{3.222275in}}{\pgfqpoint{1.630819in}{3.214462in}}%
\pgfpathcurveto{\pgfqpoint{1.638633in}{3.206648in}}{\pgfqpoint{1.649232in}{3.202258in}}{\pgfqpoint{1.660282in}{3.202258in}}%
\pgfpathclose%
\pgfusepath{stroke,fill}%
\end{pgfscope}%
\begin{pgfscope}%
\pgfpathrectangle{\pgfqpoint{0.648703in}{0.548769in}}{\pgfqpoint{5.201297in}{3.102590in}}%
\pgfusepath{clip}%
\pgfsetbuttcap%
\pgfsetroundjoin%
\definecolor{currentfill}{rgb}{1.000000,0.498039,0.054902}%
\pgfsetfillcolor{currentfill}%
\pgfsetlinewidth{1.003750pt}%
\definecolor{currentstroke}{rgb}{1.000000,0.498039,0.054902}%
\pgfsetstrokecolor{currentstroke}%
\pgfsetdash{}{0pt}%
\pgfpathmoveto{\pgfqpoint{2.202891in}{3.189572in}}%
\pgfpathcurveto{\pgfqpoint{2.213941in}{3.189572in}}{\pgfqpoint{2.224540in}{3.193962in}}{\pgfqpoint{2.232354in}{3.201775in}}%
\pgfpathcurveto{\pgfqpoint{2.240168in}{3.209589in}}{\pgfqpoint{2.244558in}{3.220188in}}{\pgfqpoint{2.244558in}{3.231238in}}%
\pgfpathcurveto{\pgfqpoint{2.244558in}{3.242288in}}{\pgfqpoint{2.240168in}{3.252887in}}{\pgfqpoint{2.232354in}{3.260701in}}%
\pgfpathcurveto{\pgfqpoint{2.224540in}{3.268515in}}{\pgfqpoint{2.213941in}{3.272905in}}{\pgfqpoint{2.202891in}{3.272905in}}%
\pgfpathcurveto{\pgfqpoint{2.191841in}{3.272905in}}{\pgfqpoint{2.181242in}{3.268515in}}{\pgfqpoint{2.173428in}{3.260701in}}%
\pgfpathcurveto{\pgfqpoint{2.165615in}{3.252887in}}{\pgfqpoint{2.161224in}{3.242288in}}{\pgfqpoint{2.161224in}{3.231238in}}%
\pgfpathcurveto{\pgfqpoint{2.161224in}{3.220188in}}{\pgfqpoint{2.165615in}{3.209589in}}{\pgfqpoint{2.173428in}{3.201775in}}%
\pgfpathcurveto{\pgfqpoint{2.181242in}{3.193962in}}{\pgfqpoint{2.191841in}{3.189572in}}{\pgfqpoint{2.202891in}{3.189572in}}%
\pgfpathclose%
\pgfusepath{stroke,fill}%
\end{pgfscope}%
\begin{pgfscope}%
\pgfpathrectangle{\pgfqpoint{0.648703in}{0.548769in}}{\pgfqpoint{5.201297in}{3.102590in}}%
\pgfusepath{clip}%
\pgfsetbuttcap%
\pgfsetroundjoin%
\definecolor{currentfill}{rgb}{1.000000,0.498039,0.054902}%
\pgfsetfillcolor{currentfill}%
\pgfsetlinewidth{1.003750pt}%
\definecolor{currentstroke}{rgb}{1.000000,0.498039,0.054902}%
\pgfsetstrokecolor{currentstroke}%
\pgfsetdash{}{0pt}%
\pgfpathmoveto{\pgfqpoint{2.667985in}{3.206486in}}%
\pgfpathcurveto{\pgfqpoint{2.679035in}{3.206486in}}{\pgfqpoint{2.689634in}{3.210877in}}{\pgfqpoint{2.697448in}{3.218690in}}%
\pgfpathcurveto{\pgfqpoint{2.705261in}{3.226504in}}{\pgfqpoint{2.709651in}{3.237103in}}{\pgfqpoint{2.709651in}{3.248153in}}%
\pgfpathcurveto{\pgfqpoint{2.709651in}{3.259203in}}{\pgfqpoint{2.705261in}{3.269802in}}{\pgfqpoint{2.697448in}{3.277616in}}%
\pgfpathcurveto{\pgfqpoint{2.689634in}{3.285429in}}{\pgfqpoint{2.679035in}{3.289820in}}{\pgfqpoint{2.667985in}{3.289820in}}%
\pgfpathcurveto{\pgfqpoint{2.656935in}{3.289820in}}{\pgfqpoint{2.646336in}{3.285429in}}{\pgfqpoint{2.638522in}{3.277616in}}%
\pgfpathcurveto{\pgfqpoint{2.630708in}{3.269802in}}{\pgfqpoint{2.626318in}{3.259203in}}{\pgfqpoint{2.626318in}{3.248153in}}%
\pgfpathcurveto{\pgfqpoint{2.626318in}{3.237103in}}{\pgfqpoint{2.630708in}{3.226504in}}{\pgfqpoint{2.638522in}{3.218690in}}%
\pgfpathcurveto{\pgfqpoint{2.646336in}{3.210877in}}{\pgfqpoint{2.656935in}{3.206486in}}{\pgfqpoint{2.667985in}{3.206486in}}%
\pgfpathclose%
\pgfusepath{stroke,fill}%
\end{pgfscope}%
\begin{pgfscope}%
\pgfpathrectangle{\pgfqpoint{0.648703in}{0.548769in}}{\pgfqpoint{5.201297in}{3.102590in}}%
\pgfusepath{clip}%
\pgfsetbuttcap%
\pgfsetroundjoin%
\definecolor{currentfill}{rgb}{0.121569,0.466667,0.705882}%
\pgfsetfillcolor{currentfill}%
\pgfsetlinewidth{1.003750pt}%
\definecolor{currentstroke}{rgb}{0.121569,0.466667,0.705882}%
\pgfsetstrokecolor{currentstroke}%
\pgfsetdash{}{0pt}%
\pgfpathmoveto{\pgfqpoint{1.272704in}{0.648129in}}%
\pgfpathcurveto{\pgfqpoint{1.283754in}{0.648129in}}{\pgfqpoint{1.294353in}{0.652519in}}{\pgfqpoint{1.302167in}{0.660333in}}%
\pgfpathcurveto{\pgfqpoint{1.309980in}{0.668146in}}{\pgfqpoint{1.314371in}{0.678745in}}{\pgfqpoint{1.314371in}{0.689796in}}%
\pgfpathcurveto{\pgfqpoint{1.314371in}{0.700846in}}{\pgfqpoint{1.309980in}{0.711445in}}{\pgfqpoint{1.302167in}{0.719258in}}%
\pgfpathcurveto{\pgfqpoint{1.294353in}{0.727072in}}{\pgfqpoint{1.283754in}{0.731462in}}{\pgfqpoint{1.272704in}{0.731462in}}%
\pgfpathcurveto{\pgfqpoint{1.261654in}{0.731462in}}{\pgfqpoint{1.251055in}{0.727072in}}{\pgfqpoint{1.243241in}{0.719258in}}%
\pgfpathcurveto{\pgfqpoint{1.235428in}{0.711445in}}{\pgfqpoint{1.231037in}{0.700846in}}{\pgfqpoint{1.231037in}{0.689796in}}%
\pgfpathcurveto{\pgfqpoint{1.231037in}{0.678745in}}{\pgfqpoint{1.235428in}{0.668146in}}{\pgfqpoint{1.243241in}{0.660333in}}%
\pgfpathcurveto{\pgfqpoint{1.251055in}{0.652519in}}{\pgfqpoint{1.261654in}{0.648129in}}{\pgfqpoint{1.272704in}{0.648129in}}%
\pgfpathclose%
\pgfusepath{stroke,fill}%
\end{pgfscope}%
\begin{pgfscope}%
\pgfpathrectangle{\pgfqpoint{0.648703in}{0.548769in}}{\pgfqpoint{5.201297in}{3.102590in}}%
\pgfusepath{clip}%
\pgfsetbuttcap%
\pgfsetroundjoin%
\definecolor{currentfill}{rgb}{0.121569,0.466667,0.705882}%
\pgfsetfillcolor{currentfill}%
\pgfsetlinewidth{1.003750pt}%
\definecolor{currentstroke}{rgb}{0.121569,0.466667,0.705882}%
\pgfsetstrokecolor{currentstroke}%
\pgfsetdash{}{0pt}%
\pgfpathmoveto{\pgfqpoint{0.885126in}{1.692615in}}%
\pgfpathcurveto{\pgfqpoint{0.896176in}{1.692615in}}{\pgfqpoint{0.906775in}{1.697006in}}{\pgfqpoint{0.914589in}{1.704819in}}%
\pgfpathcurveto{\pgfqpoint{0.922402in}{1.712633in}}{\pgfqpoint{0.926793in}{1.723232in}}{\pgfqpoint{0.926793in}{1.734282in}}%
\pgfpathcurveto{\pgfqpoint{0.926793in}{1.745332in}}{\pgfqpoint{0.922402in}{1.755931in}}{\pgfqpoint{0.914589in}{1.763745in}}%
\pgfpathcurveto{\pgfqpoint{0.906775in}{1.771558in}}{\pgfqpoint{0.896176in}{1.775949in}}{\pgfqpoint{0.885126in}{1.775949in}}%
\pgfpathcurveto{\pgfqpoint{0.874076in}{1.775949in}}{\pgfqpoint{0.863477in}{1.771558in}}{\pgfqpoint{0.855663in}{1.763745in}}%
\pgfpathcurveto{\pgfqpoint{0.847850in}{1.755931in}}{\pgfqpoint{0.843459in}{1.745332in}}{\pgfqpoint{0.843459in}{1.734282in}}%
\pgfpathcurveto{\pgfqpoint{0.843459in}{1.723232in}}{\pgfqpoint{0.847850in}{1.712633in}}{\pgfqpoint{0.855663in}{1.704819in}}%
\pgfpathcurveto{\pgfqpoint{0.863477in}{1.697006in}}{\pgfqpoint{0.874076in}{1.692615in}}{\pgfqpoint{0.885126in}{1.692615in}}%
\pgfpathclose%
\pgfusepath{stroke,fill}%
\end{pgfscope}%
\begin{pgfscope}%
\pgfpathrectangle{\pgfqpoint{0.648703in}{0.548769in}}{\pgfqpoint{5.201297in}{3.102590in}}%
\pgfusepath{clip}%
\pgfsetbuttcap%
\pgfsetroundjoin%
\definecolor{currentfill}{rgb}{0.121569,0.466667,0.705882}%
\pgfsetfillcolor{currentfill}%
\pgfsetlinewidth{1.003750pt}%
\definecolor{currentstroke}{rgb}{0.121569,0.466667,0.705882}%
\pgfsetstrokecolor{currentstroke}%
\pgfsetdash{}{0pt}%
\pgfpathmoveto{\pgfqpoint{0.885126in}{0.648129in}}%
\pgfpathcurveto{\pgfqpoint{0.896176in}{0.648129in}}{\pgfqpoint{0.906775in}{0.652519in}}{\pgfqpoint{0.914589in}{0.660333in}}%
\pgfpathcurveto{\pgfqpoint{0.922402in}{0.668146in}}{\pgfqpoint{0.926793in}{0.678745in}}{\pgfqpoint{0.926793in}{0.689796in}}%
\pgfpathcurveto{\pgfqpoint{0.926793in}{0.700846in}}{\pgfqpoint{0.922402in}{0.711445in}}{\pgfqpoint{0.914589in}{0.719258in}}%
\pgfpathcurveto{\pgfqpoint{0.906775in}{0.727072in}}{\pgfqpoint{0.896176in}{0.731462in}}{\pgfqpoint{0.885126in}{0.731462in}}%
\pgfpathcurveto{\pgfqpoint{0.874076in}{0.731462in}}{\pgfqpoint{0.863477in}{0.727072in}}{\pgfqpoint{0.855663in}{0.719258in}}%
\pgfpathcurveto{\pgfqpoint{0.847850in}{0.711445in}}{\pgfqpoint{0.843459in}{0.700846in}}{\pgfqpoint{0.843459in}{0.689796in}}%
\pgfpathcurveto{\pgfqpoint{0.843459in}{0.678745in}}{\pgfqpoint{0.847850in}{0.668146in}}{\pgfqpoint{0.855663in}{0.660333in}}%
\pgfpathcurveto{\pgfqpoint{0.863477in}{0.652519in}}{\pgfqpoint{0.874076in}{0.648129in}}{\pgfqpoint{0.885126in}{0.648129in}}%
\pgfpathclose%
\pgfusepath{stroke,fill}%
\end{pgfscope}%
\begin{pgfscope}%
\pgfpathrectangle{\pgfqpoint{0.648703in}{0.548769in}}{\pgfqpoint{5.201297in}{3.102590in}}%
\pgfusepath{clip}%
\pgfsetbuttcap%
\pgfsetroundjoin%
\definecolor{currentfill}{rgb}{1.000000,0.498039,0.054902}%
\pgfsetfillcolor{currentfill}%
\pgfsetlinewidth{1.003750pt}%
\definecolor{currentstroke}{rgb}{1.000000,0.498039,0.054902}%
\pgfsetstrokecolor{currentstroke}%
\pgfsetdash{}{0pt}%
\pgfpathmoveto{\pgfqpoint{2.125376in}{3.189572in}}%
\pgfpathcurveto{\pgfqpoint{2.136426in}{3.189572in}}{\pgfqpoint{2.147025in}{3.193962in}}{\pgfqpoint{2.154838in}{3.201775in}}%
\pgfpathcurveto{\pgfqpoint{2.162652in}{3.209589in}}{\pgfqpoint{2.167042in}{3.220188in}}{\pgfqpoint{2.167042in}{3.231238in}}%
\pgfpathcurveto{\pgfqpoint{2.167042in}{3.242288in}}{\pgfqpoint{2.162652in}{3.252887in}}{\pgfqpoint{2.154838in}{3.260701in}}%
\pgfpathcurveto{\pgfqpoint{2.147025in}{3.268515in}}{\pgfqpoint{2.136426in}{3.272905in}}{\pgfqpoint{2.125376in}{3.272905in}}%
\pgfpathcurveto{\pgfqpoint{2.114325in}{3.272905in}}{\pgfqpoint{2.103726in}{3.268515in}}{\pgfqpoint{2.095913in}{3.260701in}}%
\pgfpathcurveto{\pgfqpoint{2.088099in}{3.252887in}}{\pgfqpoint{2.083709in}{3.242288in}}{\pgfqpoint{2.083709in}{3.231238in}}%
\pgfpathcurveto{\pgfqpoint{2.083709in}{3.220188in}}{\pgfqpoint{2.088099in}{3.209589in}}{\pgfqpoint{2.095913in}{3.201775in}}%
\pgfpathcurveto{\pgfqpoint{2.103726in}{3.193962in}}{\pgfqpoint{2.114325in}{3.189572in}}{\pgfqpoint{2.125376in}{3.189572in}}%
\pgfpathclose%
\pgfusepath{stroke,fill}%
\end{pgfscope}%
\begin{pgfscope}%
\pgfpathrectangle{\pgfqpoint{0.648703in}{0.548769in}}{\pgfqpoint{5.201297in}{3.102590in}}%
\pgfusepath{clip}%
\pgfsetbuttcap%
\pgfsetroundjoin%
\definecolor{currentfill}{rgb}{0.121569,0.466667,0.705882}%
\pgfsetfillcolor{currentfill}%
\pgfsetlinewidth{1.003750pt}%
\definecolor{currentstroke}{rgb}{0.121569,0.466667,0.705882}%
\pgfsetstrokecolor{currentstroke}%
\pgfsetdash{}{0pt}%
\pgfpathmoveto{\pgfqpoint{2.125376in}{3.181114in}}%
\pgfpathcurveto{\pgfqpoint{2.136426in}{3.181114in}}{\pgfqpoint{2.147025in}{3.185504in}}{\pgfqpoint{2.154838in}{3.193318in}}%
\pgfpathcurveto{\pgfqpoint{2.162652in}{3.201132in}}{\pgfqpoint{2.167042in}{3.211731in}}{\pgfqpoint{2.167042in}{3.222781in}}%
\pgfpathcurveto{\pgfqpoint{2.167042in}{3.233831in}}{\pgfqpoint{2.162652in}{3.244430in}}{\pgfqpoint{2.154838in}{3.252244in}}%
\pgfpathcurveto{\pgfqpoint{2.147025in}{3.260057in}}{\pgfqpoint{2.136426in}{3.264448in}}{\pgfqpoint{2.125376in}{3.264448in}}%
\pgfpathcurveto{\pgfqpoint{2.114325in}{3.264448in}}{\pgfqpoint{2.103726in}{3.260057in}}{\pgfqpoint{2.095913in}{3.252244in}}%
\pgfpathcurveto{\pgfqpoint{2.088099in}{3.244430in}}{\pgfqpoint{2.083709in}{3.233831in}}{\pgfqpoint{2.083709in}{3.222781in}}%
\pgfpathcurveto{\pgfqpoint{2.083709in}{3.211731in}}{\pgfqpoint{2.088099in}{3.201132in}}{\pgfqpoint{2.095913in}{3.193318in}}%
\pgfpathcurveto{\pgfqpoint{2.103726in}{3.185504in}}{\pgfqpoint{2.114325in}{3.181114in}}{\pgfqpoint{2.125376in}{3.181114in}}%
\pgfpathclose%
\pgfusepath{stroke,fill}%
\end{pgfscope}%
\begin{pgfscope}%
\pgfpathrectangle{\pgfqpoint{0.648703in}{0.548769in}}{\pgfqpoint{5.201297in}{3.102590in}}%
\pgfusepath{clip}%
\pgfsetbuttcap%
\pgfsetroundjoin%
\definecolor{currentfill}{rgb}{1.000000,0.498039,0.054902}%
\pgfsetfillcolor{currentfill}%
\pgfsetlinewidth{1.003750pt}%
\definecolor{currentstroke}{rgb}{1.000000,0.498039,0.054902}%
\pgfsetstrokecolor{currentstroke}%
\pgfsetdash{}{0pt}%
\pgfpathmoveto{\pgfqpoint{4.605875in}{3.189572in}}%
\pgfpathcurveto{\pgfqpoint{4.616925in}{3.189572in}}{\pgfqpoint{4.627524in}{3.193962in}}{\pgfqpoint{4.635337in}{3.201775in}}%
\pgfpathcurveto{\pgfqpoint{4.643151in}{3.209589in}}{\pgfqpoint{4.647541in}{3.220188in}}{\pgfqpoint{4.647541in}{3.231238in}}%
\pgfpathcurveto{\pgfqpoint{4.647541in}{3.242288in}}{\pgfqpoint{4.643151in}{3.252887in}}{\pgfqpoint{4.635337in}{3.260701in}}%
\pgfpathcurveto{\pgfqpoint{4.627524in}{3.268515in}}{\pgfqpoint{4.616925in}{3.272905in}}{\pgfqpoint{4.605875in}{3.272905in}}%
\pgfpathcurveto{\pgfqpoint{4.594825in}{3.272905in}}{\pgfqpoint{4.584226in}{3.268515in}}{\pgfqpoint{4.576412in}{3.260701in}}%
\pgfpathcurveto{\pgfqpoint{4.568598in}{3.252887in}}{\pgfqpoint{4.564208in}{3.242288in}}{\pgfqpoint{4.564208in}{3.231238in}}%
\pgfpathcurveto{\pgfqpoint{4.564208in}{3.220188in}}{\pgfqpoint{4.568598in}{3.209589in}}{\pgfqpoint{4.576412in}{3.201775in}}%
\pgfpathcurveto{\pgfqpoint{4.584226in}{3.193962in}}{\pgfqpoint{4.594825in}{3.189572in}}{\pgfqpoint{4.605875in}{3.189572in}}%
\pgfpathclose%
\pgfusepath{stroke,fill}%
\end{pgfscope}%
\begin{pgfscope}%
\pgfpathrectangle{\pgfqpoint{0.648703in}{0.548769in}}{\pgfqpoint{5.201297in}{3.102590in}}%
\pgfusepath{clip}%
\pgfsetbuttcap%
\pgfsetroundjoin%
\definecolor{currentfill}{rgb}{1.000000,0.498039,0.054902}%
\pgfsetfillcolor{currentfill}%
\pgfsetlinewidth{1.003750pt}%
\definecolor{currentstroke}{rgb}{1.000000,0.498039,0.054902}%
\pgfsetstrokecolor{currentstroke}%
\pgfsetdash{}{0pt}%
\pgfpathmoveto{\pgfqpoint{1.195188in}{3.214944in}}%
\pgfpathcurveto{\pgfqpoint{1.206239in}{3.214944in}}{\pgfqpoint{1.216838in}{3.219334in}}{\pgfqpoint{1.224651in}{3.227148in}}%
\pgfpathcurveto{\pgfqpoint{1.232465in}{3.234961in}}{\pgfqpoint{1.236855in}{3.245560in}}{\pgfqpoint{1.236855in}{3.256610in}}%
\pgfpathcurveto{\pgfqpoint{1.236855in}{3.267661in}}{\pgfqpoint{1.232465in}{3.278260in}}{\pgfqpoint{1.224651in}{3.286073in}}%
\pgfpathcurveto{\pgfqpoint{1.216838in}{3.293887in}}{\pgfqpoint{1.206239in}{3.298277in}}{\pgfqpoint{1.195188in}{3.298277in}}%
\pgfpathcurveto{\pgfqpoint{1.184138in}{3.298277in}}{\pgfqpoint{1.173539in}{3.293887in}}{\pgfqpoint{1.165726in}{3.286073in}}%
\pgfpathcurveto{\pgfqpoint{1.157912in}{3.278260in}}{\pgfqpoint{1.153522in}{3.267661in}}{\pgfqpoint{1.153522in}{3.256610in}}%
\pgfpathcurveto{\pgfqpoint{1.153522in}{3.245560in}}{\pgfqpoint{1.157912in}{3.234961in}}{\pgfqpoint{1.165726in}{3.227148in}}%
\pgfpathcurveto{\pgfqpoint{1.173539in}{3.219334in}}{\pgfqpoint{1.184138in}{3.214944in}}{\pgfqpoint{1.195188in}{3.214944in}}%
\pgfpathclose%
\pgfusepath{stroke,fill}%
\end{pgfscope}%
\begin{pgfscope}%
\pgfpathrectangle{\pgfqpoint{0.648703in}{0.548769in}}{\pgfqpoint{5.201297in}{3.102590in}}%
\pgfusepath{clip}%
\pgfsetbuttcap%
\pgfsetroundjoin%
\definecolor{currentfill}{rgb}{1.000000,0.498039,0.054902}%
\pgfsetfillcolor{currentfill}%
\pgfsetlinewidth{1.003750pt}%
\definecolor{currentstroke}{rgb}{1.000000,0.498039,0.054902}%
\pgfsetstrokecolor{currentstroke}%
\pgfsetdash{}{0pt}%
\pgfpathmoveto{\pgfqpoint{1.660282in}{3.185343in}}%
\pgfpathcurveto{\pgfqpoint{1.671332in}{3.185343in}}{\pgfqpoint{1.681931in}{3.189733in}}{\pgfqpoint{1.689745in}{3.197547in}}%
\pgfpathcurveto{\pgfqpoint{1.697558in}{3.205360in}}{\pgfqpoint{1.701949in}{3.215959in}}{\pgfqpoint{1.701949in}{3.227010in}}%
\pgfpathcurveto{\pgfqpoint{1.701949in}{3.238060in}}{\pgfqpoint{1.697558in}{3.248659in}}{\pgfqpoint{1.689745in}{3.256472in}}%
\pgfpathcurveto{\pgfqpoint{1.681931in}{3.264286in}}{\pgfqpoint{1.671332in}{3.268676in}}{\pgfqpoint{1.660282in}{3.268676in}}%
\pgfpathcurveto{\pgfqpoint{1.649232in}{3.268676in}}{\pgfqpoint{1.638633in}{3.264286in}}{\pgfqpoint{1.630819in}{3.256472in}}%
\pgfpathcurveto{\pgfqpoint{1.623006in}{3.248659in}}{\pgfqpoint{1.618615in}{3.238060in}}{\pgfqpoint{1.618615in}{3.227010in}}%
\pgfpathcurveto{\pgfqpoint{1.618615in}{3.215959in}}{\pgfqpoint{1.623006in}{3.205360in}}{\pgfqpoint{1.630819in}{3.197547in}}%
\pgfpathcurveto{\pgfqpoint{1.638633in}{3.189733in}}{\pgfqpoint{1.649232in}{3.185343in}}{\pgfqpoint{1.660282in}{3.185343in}}%
\pgfpathclose%
\pgfusepath{stroke,fill}%
\end{pgfscope}%
\begin{pgfscope}%
\pgfpathrectangle{\pgfqpoint{0.648703in}{0.548769in}}{\pgfqpoint{5.201297in}{3.102590in}}%
\pgfusepath{clip}%
\pgfsetbuttcap%
\pgfsetroundjoin%
\definecolor{currentfill}{rgb}{1.000000,0.498039,0.054902}%
\pgfsetfillcolor{currentfill}%
\pgfsetlinewidth{1.003750pt}%
\definecolor{currentstroke}{rgb}{1.000000,0.498039,0.054902}%
\pgfsetstrokecolor{currentstroke}%
\pgfsetdash{}{0pt}%
\pgfpathmoveto{\pgfqpoint{2.900532in}{3.198029in}}%
\pgfpathcurveto{\pgfqpoint{2.911582in}{3.198029in}}{\pgfqpoint{2.922181in}{3.202419in}}{\pgfqpoint{2.929994in}{3.210233in}}%
\pgfpathcurveto{\pgfqpoint{2.937808in}{3.218046in}}{\pgfqpoint{2.942198in}{3.228646in}}{\pgfqpoint{2.942198in}{3.239696in}}%
\pgfpathcurveto{\pgfqpoint{2.942198in}{3.250746in}}{\pgfqpoint{2.937808in}{3.261345in}}{\pgfqpoint{2.929994in}{3.269158in}}%
\pgfpathcurveto{\pgfqpoint{2.922181in}{3.276972in}}{\pgfqpoint{2.911582in}{3.281362in}}{\pgfqpoint{2.900532in}{3.281362in}}%
\pgfpathcurveto{\pgfqpoint{2.889481in}{3.281362in}}{\pgfqpoint{2.878882in}{3.276972in}}{\pgfqpoint{2.871069in}{3.269158in}}%
\pgfpathcurveto{\pgfqpoint{2.863255in}{3.261345in}}{\pgfqpoint{2.858865in}{3.250746in}}{\pgfqpoint{2.858865in}{3.239696in}}%
\pgfpathcurveto{\pgfqpoint{2.858865in}{3.228646in}}{\pgfqpoint{2.863255in}{3.218046in}}{\pgfqpoint{2.871069in}{3.210233in}}%
\pgfpathcurveto{\pgfqpoint{2.878882in}{3.202419in}}{\pgfqpoint{2.889481in}{3.198029in}}{\pgfqpoint{2.900532in}{3.198029in}}%
\pgfpathclose%
\pgfusepath{stroke,fill}%
\end{pgfscope}%
\begin{pgfscope}%
\pgfpathrectangle{\pgfqpoint{0.648703in}{0.548769in}}{\pgfqpoint{5.201297in}{3.102590in}}%
\pgfusepath{clip}%
\pgfsetbuttcap%
\pgfsetroundjoin%
\definecolor{currentfill}{rgb}{0.839216,0.152941,0.156863}%
\pgfsetfillcolor{currentfill}%
\pgfsetlinewidth{1.003750pt}%
\definecolor{currentstroke}{rgb}{0.839216,0.152941,0.156863}%
\pgfsetstrokecolor{currentstroke}%
\pgfsetdash{}{0pt}%
\pgfpathmoveto{\pgfqpoint{1.970344in}{3.193800in}}%
\pgfpathcurveto{\pgfqpoint{1.981394in}{3.193800in}}{\pgfqpoint{1.991994in}{3.198191in}}{\pgfqpoint{1.999807in}{3.206004in}}%
\pgfpathcurveto{\pgfqpoint{2.007621in}{3.213818in}}{\pgfqpoint{2.012011in}{3.224417in}}{\pgfqpoint{2.012011in}{3.235467in}}%
\pgfpathcurveto{\pgfqpoint{2.012011in}{3.246517in}}{\pgfqpoint{2.007621in}{3.257116in}}{\pgfqpoint{1.999807in}{3.264930in}}%
\pgfpathcurveto{\pgfqpoint{1.991994in}{3.272743in}}{\pgfqpoint{1.981394in}{3.277134in}}{\pgfqpoint{1.970344in}{3.277134in}}%
\pgfpathcurveto{\pgfqpoint{1.959294in}{3.277134in}}{\pgfqpoint{1.948695in}{3.272743in}}{\pgfqpoint{1.940882in}{3.264930in}}%
\pgfpathcurveto{\pgfqpoint{1.933068in}{3.257116in}}{\pgfqpoint{1.928678in}{3.246517in}}{\pgfqpoint{1.928678in}{3.235467in}}%
\pgfpathcurveto{\pgfqpoint{1.928678in}{3.224417in}}{\pgfqpoint{1.933068in}{3.213818in}}{\pgfqpoint{1.940882in}{3.206004in}}%
\pgfpathcurveto{\pgfqpoint{1.948695in}{3.198191in}}{\pgfqpoint{1.959294in}{3.193800in}}{\pgfqpoint{1.970344in}{3.193800in}}%
\pgfpathclose%
\pgfusepath{stroke,fill}%
\end{pgfscope}%
\begin{pgfscope}%
\pgfpathrectangle{\pgfqpoint{0.648703in}{0.548769in}}{\pgfqpoint{5.201297in}{3.102590in}}%
\pgfusepath{clip}%
\pgfsetbuttcap%
\pgfsetroundjoin%
\definecolor{currentfill}{rgb}{0.121569,0.466667,0.705882}%
\pgfsetfillcolor{currentfill}%
\pgfsetlinewidth{1.003750pt}%
\definecolor{currentstroke}{rgb}{0.121569,0.466667,0.705882}%
\pgfsetstrokecolor{currentstroke}%
\pgfsetdash{}{0pt}%
\pgfpathmoveto{\pgfqpoint{1.350220in}{0.758075in}}%
\pgfpathcurveto{\pgfqpoint{1.361270in}{0.758075in}}{\pgfqpoint{1.371869in}{0.762465in}}{\pgfqpoint{1.379682in}{0.770279in}}%
\pgfpathcurveto{\pgfqpoint{1.387496in}{0.778092in}}{\pgfqpoint{1.391886in}{0.788691in}}{\pgfqpoint{1.391886in}{0.799742in}}%
\pgfpathcurveto{\pgfqpoint{1.391886in}{0.810792in}}{\pgfqpoint{1.387496in}{0.821391in}}{\pgfqpoint{1.379682in}{0.829204in}}%
\pgfpathcurveto{\pgfqpoint{1.371869in}{0.837018in}}{\pgfqpoint{1.361270in}{0.841408in}}{\pgfqpoint{1.350220in}{0.841408in}}%
\pgfpathcurveto{\pgfqpoint{1.339169in}{0.841408in}}{\pgfqpoint{1.328570in}{0.837018in}}{\pgfqpoint{1.320757in}{0.829204in}}%
\pgfpathcurveto{\pgfqpoint{1.312943in}{0.821391in}}{\pgfqpoint{1.308553in}{0.810792in}}{\pgfqpoint{1.308553in}{0.799742in}}%
\pgfpathcurveto{\pgfqpoint{1.308553in}{0.788691in}}{\pgfqpoint{1.312943in}{0.778092in}}{\pgfqpoint{1.320757in}{0.770279in}}%
\pgfpathcurveto{\pgfqpoint{1.328570in}{0.762465in}}{\pgfqpoint{1.339169in}{0.758075in}}{\pgfqpoint{1.350220in}{0.758075in}}%
\pgfpathclose%
\pgfusepath{stroke,fill}%
\end{pgfscope}%
\begin{pgfscope}%
\pgfpathrectangle{\pgfqpoint{0.648703in}{0.548769in}}{\pgfqpoint{5.201297in}{3.102590in}}%
\pgfusepath{clip}%
\pgfsetbuttcap%
\pgfsetroundjoin%
\definecolor{currentfill}{rgb}{1.000000,0.498039,0.054902}%
\pgfsetfillcolor{currentfill}%
\pgfsetlinewidth{1.003750pt}%
\definecolor{currentstroke}{rgb}{1.000000,0.498039,0.054902}%
\pgfsetstrokecolor{currentstroke}%
\pgfsetdash{}{0pt}%
\pgfpathmoveto{\pgfqpoint{2.047860in}{3.193800in}}%
\pgfpathcurveto{\pgfqpoint{2.058910in}{3.193800in}}{\pgfqpoint{2.069509in}{3.198191in}}{\pgfqpoint{2.077323in}{3.206004in}}%
\pgfpathcurveto{\pgfqpoint{2.085136in}{3.213818in}}{\pgfqpoint{2.089527in}{3.224417in}}{\pgfqpoint{2.089527in}{3.235467in}}%
\pgfpathcurveto{\pgfqpoint{2.089527in}{3.246517in}}{\pgfqpoint{2.085136in}{3.257116in}}{\pgfqpoint{2.077323in}{3.264930in}}%
\pgfpathcurveto{\pgfqpoint{2.069509in}{3.272743in}}{\pgfqpoint{2.058910in}{3.277134in}}{\pgfqpoint{2.047860in}{3.277134in}}%
\pgfpathcurveto{\pgfqpoint{2.036810in}{3.277134in}}{\pgfqpoint{2.026211in}{3.272743in}}{\pgfqpoint{2.018397in}{3.264930in}}%
\pgfpathcurveto{\pgfqpoint{2.010584in}{3.257116in}}{\pgfqpoint{2.006193in}{3.246517in}}{\pgfqpoint{2.006193in}{3.235467in}}%
\pgfpathcurveto{\pgfqpoint{2.006193in}{3.224417in}}{\pgfqpoint{2.010584in}{3.213818in}}{\pgfqpoint{2.018397in}{3.206004in}}%
\pgfpathcurveto{\pgfqpoint{2.026211in}{3.198191in}}{\pgfqpoint{2.036810in}{3.193800in}}{\pgfqpoint{2.047860in}{3.193800in}}%
\pgfpathclose%
\pgfusepath{stroke,fill}%
\end{pgfscope}%
\begin{pgfscope}%
\pgfpathrectangle{\pgfqpoint{0.648703in}{0.548769in}}{\pgfqpoint{5.201297in}{3.102590in}}%
\pgfusepath{clip}%
\pgfsetbuttcap%
\pgfsetroundjoin%
\definecolor{currentfill}{rgb}{0.121569,0.466667,0.705882}%
\pgfsetfillcolor{currentfill}%
\pgfsetlinewidth{1.003750pt}%
\definecolor{currentstroke}{rgb}{0.121569,0.466667,0.705882}%
\pgfsetstrokecolor{currentstroke}%
\pgfsetdash{}{0pt}%
\pgfpathmoveto{\pgfqpoint{0.885126in}{0.648129in}}%
\pgfpathcurveto{\pgfqpoint{0.896176in}{0.648129in}}{\pgfqpoint{0.906775in}{0.652519in}}{\pgfqpoint{0.914589in}{0.660333in}}%
\pgfpathcurveto{\pgfqpoint{0.922402in}{0.668146in}}{\pgfqpoint{0.926793in}{0.678745in}}{\pgfqpoint{0.926793in}{0.689796in}}%
\pgfpathcurveto{\pgfqpoint{0.926793in}{0.700846in}}{\pgfqpoint{0.922402in}{0.711445in}}{\pgfqpoint{0.914589in}{0.719258in}}%
\pgfpathcurveto{\pgfqpoint{0.906775in}{0.727072in}}{\pgfqpoint{0.896176in}{0.731462in}}{\pgfqpoint{0.885126in}{0.731462in}}%
\pgfpathcurveto{\pgfqpoint{0.874076in}{0.731462in}}{\pgfqpoint{0.863477in}{0.727072in}}{\pgfqpoint{0.855663in}{0.719258in}}%
\pgfpathcurveto{\pgfqpoint{0.847850in}{0.711445in}}{\pgfqpoint{0.843459in}{0.700846in}}{\pgfqpoint{0.843459in}{0.689796in}}%
\pgfpathcurveto{\pgfqpoint{0.843459in}{0.678745in}}{\pgfqpoint{0.847850in}{0.668146in}}{\pgfqpoint{0.855663in}{0.660333in}}%
\pgfpathcurveto{\pgfqpoint{0.863477in}{0.652519in}}{\pgfqpoint{0.874076in}{0.648129in}}{\pgfqpoint{0.885126in}{0.648129in}}%
\pgfpathclose%
\pgfusepath{stroke,fill}%
\end{pgfscope}%
\begin{pgfscope}%
\pgfpathrectangle{\pgfqpoint{0.648703in}{0.548769in}}{\pgfqpoint{5.201297in}{3.102590in}}%
\pgfusepath{clip}%
\pgfsetbuttcap%
\pgfsetroundjoin%
\definecolor{currentfill}{rgb}{1.000000,0.498039,0.054902}%
\pgfsetfillcolor{currentfill}%
\pgfsetlinewidth{1.003750pt}%
\definecolor{currentstroke}{rgb}{1.000000,0.498039,0.054902}%
\pgfsetstrokecolor{currentstroke}%
\pgfsetdash{}{0pt}%
\pgfpathmoveto{\pgfqpoint{2.667985in}{3.193800in}}%
\pgfpathcurveto{\pgfqpoint{2.679035in}{3.193800in}}{\pgfqpoint{2.689634in}{3.198191in}}{\pgfqpoint{2.697448in}{3.206004in}}%
\pgfpathcurveto{\pgfqpoint{2.705261in}{3.213818in}}{\pgfqpoint{2.709651in}{3.224417in}}{\pgfqpoint{2.709651in}{3.235467in}}%
\pgfpathcurveto{\pgfqpoint{2.709651in}{3.246517in}}{\pgfqpoint{2.705261in}{3.257116in}}{\pgfqpoint{2.697448in}{3.264930in}}%
\pgfpathcurveto{\pgfqpoint{2.689634in}{3.272743in}}{\pgfqpoint{2.679035in}{3.277134in}}{\pgfqpoint{2.667985in}{3.277134in}}%
\pgfpathcurveto{\pgfqpoint{2.656935in}{3.277134in}}{\pgfqpoint{2.646336in}{3.272743in}}{\pgfqpoint{2.638522in}{3.264930in}}%
\pgfpathcurveto{\pgfqpoint{2.630708in}{3.257116in}}{\pgfqpoint{2.626318in}{3.246517in}}{\pgfqpoint{2.626318in}{3.235467in}}%
\pgfpathcurveto{\pgfqpoint{2.626318in}{3.224417in}}{\pgfqpoint{2.630708in}{3.213818in}}{\pgfqpoint{2.638522in}{3.206004in}}%
\pgfpathcurveto{\pgfqpoint{2.646336in}{3.198191in}}{\pgfqpoint{2.656935in}{3.193800in}}{\pgfqpoint{2.667985in}{3.193800in}}%
\pgfpathclose%
\pgfusepath{stroke,fill}%
\end{pgfscope}%
\begin{pgfscope}%
\pgfpathrectangle{\pgfqpoint{0.648703in}{0.548769in}}{\pgfqpoint{5.201297in}{3.102590in}}%
\pgfusepath{clip}%
\pgfsetbuttcap%
\pgfsetroundjoin%
\definecolor{currentfill}{rgb}{1.000000,0.498039,0.054902}%
\pgfsetfillcolor{currentfill}%
\pgfsetlinewidth{1.003750pt}%
\definecolor{currentstroke}{rgb}{1.000000,0.498039,0.054902}%
\pgfsetstrokecolor{currentstroke}%
\pgfsetdash{}{0pt}%
\pgfpathmoveto{\pgfqpoint{2.435438in}{3.185343in}}%
\pgfpathcurveto{\pgfqpoint{2.446488in}{3.185343in}}{\pgfqpoint{2.457087in}{3.189733in}}{\pgfqpoint{2.464901in}{3.197547in}}%
\pgfpathcurveto{\pgfqpoint{2.472714in}{3.205360in}}{\pgfqpoint{2.477105in}{3.215959in}}{\pgfqpoint{2.477105in}{3.227010in}}%
\pgfpathcurveto{\pgfqpoint{2.477105in}{3.238060in}}{\pgfqpoint{2.472714in}{3.248659in}}{\pgfqpoint{2.464901in}{3.256472in}}%
\pgfpathcurveto{\pgfqpoint{2.457087in}{3.264286in}}{\pgfqpoint{2.446488in}{3.268676in}}{\pgfqpoint{2.435438in}{3.268676in}}%
\pgfpathcurveto{\pgfqpoint{2.424388in}{3.268676in}}{\pgfqpoint{2.413789in}{3.264286in}}{\pgfqpoint{2.405975in}{3.256472in}}%
\pgfpathcurveto{\pgfqpoint{2.398162in}{3.248659in}}{\pgfqpoint{2.393771in}{3.238060in}}{\pgfqpoint{2.393771in}{3.227010in}}%
\pgfpathcurveto{\pgfqpoint{2.393771in}{3.215959in}}{\pgfqpoint{2.398162in}{3.205360in}}{\pgfqpoint{2.405975in}{3.197547in}}%
\pgfpathcurveto{\pgfqpoint{2.413789in}{3.189733in}}{\pgfqpoint{2.424388in}{3.185343in}}{\pgfqpoint{2.435438in}{3.185343in}}%
\pgfpathclose%
\pgfusepath{stroke,fill}%
\end{pgfscope}%
\begin{pgfscope}%
\pgfpathrectangle{\pgfqpoint{0.648703in}{0.548769in}}{\pgfqpoint{5.201297in}{3.102590in}}%
\pgfusepath{clip}%
\pgfsetbuttcap%
\pgfsetroundjoin%
\definecolor{currentfill}{rgb}{1.000000,0.498039,0.054902}%
\pgfsetfillcolor{currentfill}%
\pgfsetlinewidth{1.003750pt}%
\definecolor{currentstroke}{rgb}{1.000000,0.498039,0.054902}%
\pgfsetstrokecolor{currentstroke}%
\pgfsetdash{}{0pt}%
\pgfpathmoveto{\pgfqpoint{2.047860in}{3.198029in}}%
\pgfpathcurveto{\pgfqpoint{2.058910in}{3.198029in}}{\pgfqpoint{2.069509in}{3.202419in}}{\pgfqpoint{2.077323in}{3.210233in}}%
\pgfpathcurveto{\pgfqpoint{2.085136in}{3.218046in}}{\pgfqpoint{2.089527in}{3.228646in}}{\pgfqpoint{2.089527in}{3.239696in}}%
\pgfpathcurveto{\pgfqpoint{2.089527in}{3.250746in}}{\pgfqpoint{2.085136in}{3.261345in}}{\pgfqpoint{2.077323in}{3.269158in}}%
\pgfpathcurveto{\pgfqpoint{2.069509in}{3.276972in}}{\pgfqpoint{2.058910in}{3.281362in}}{\pgfqpoint{2.047860in}{3.281362in}}%
\pgfpathcurveto{\pgfqpoint{2.036810in}{3.281362in}}{\pgfqpoint{2.026211in}{3.276972in}}{\pgfqpoint{2.018397in}{3.269158in}}%
\pgfpathcurveto{\pgfqpoint{2.010584in}{3.261345in}}{\pgfqpoint{2.006193in}{3.250746in}}{\pgfqpoint{2.006193in}{3.239696in}}%
\pgfpathcurveto{\pgfqpoint{2.006193in}{3.228646in}}{\pgfqpoint{2.010584in}{3.218046in}}{\pgfqpoint{2.018397in}{3.210233in}}%
\pgfpathcurveto{\pgfqpoint{2.026211in}{3.202419in}}{\pgfqpoint{2.036810in}{3.198029in}}{\pgfqpoint{2.047860in}{3.198029in}}%
\pgfpathclose%
\pgfusepath{stroke,fill}%
\end{pgfscope}%
\begin{pgfscope}%
\pgfpathrectangle{\pgfqpoint{0.648703in}{0.548769in}}{\pgfqpoint{5.201297in}{3.102590in}}%
\pgfusepath{clip}%
\pgfsetbuttcap%
\pgfsetroundjoin%
\definecolor{currentfill}{rgb}{1.000000,0.498039,0.054902}%
\pgfsetfillcolor{currentfill}%
\pgfsetlinewidth{1.003750pt}%
\definecolor{currentstroke}{rgb}{1.000000,0.498039,0.054902}%
\pgfsetstrokecolor{currentstroke}%
\pgfsetdash{}{0pt}%
\pgfpathmoveto{\pgfqpoint{2.512954in}{3.185343in}}%
\pgfpathcurveto{\pgfqpoint{2.524004in}{3.185343in}}{\pgfqpoint{2.534603in}{3.189733in}}{\pgfqpoint{2.542416in}{3.197547in}}%
\pgfpathcurveto{\pgfqpoint{2.550230in}{3.205360in}}{\pgfqpoint{2.554620in}{3.215959in}}{\pgfqpoint{2.554620in}{3.227010in}}%
\pgfpathcurveto{\pgfqpoint{2.554620in}{3.238060in}}{\pgfqpoint{2.550230in}{3.248659in}}{\pgfqpoint{2.542416in}{3.256472in}}%
\pgfpathcurveto{\pgfqpoint{2.534603in}{3.264286in}}{\pgfqpoint{2.524004in}{3.268676in}}{\pgfqpoint{2.512954in}{3.268676in}}%
\pgfpathcurveto{\pgfqpoint{2.501903in}{3.268676in}}{\pgfqpoint{2.491304in}{3.264286in}}{\pgfqpoint{2.483491in}{3.256472in}}%
\pgfpathcurveto{\pgfqpoint{2.475677in}{3.248659in}}{\pgfqpoint{2.471287in}{3.238060in}}{\pgfqpoint{2.471287in}{3.227010in}}%
\pgfpathcurveto{\pgfqpoint{2.471287in}{3.215959in}}{\pgfqpoint{2.475677in}{3.205360in}}{\pgfqpoint{2.483491in}{3.197547in}}%
\pgfpathcurveto{\pgfqpoint{2.491304in}{3.189733in}}{\pgfqpoint{2.501903in}{3.185343in}}{\pgfqpoint{2.512954in}{3.185343in}}%
\pgfpathclose%
\pgfusepath{stroke,fill}%
\end{pgfscope}%
\begin{pgfscope}%
\pgfpathrectangle{\pgfqpoint{0.648703in}{0.548769in}}{\pgfqpoint{5.201297in}{3.102590in}}%
\pgfusepath{clip}%
\pgfsetbuttcap%
\pgfsetroundjoin%
\definecolor{currentfill}{rgb}{1.000000,0.498039,0.054902}%
\pgfsetfillcolor{currentfill}%
\pgfsetlinewidth{1.003750pt}%
\definecolor{currentstroke}{rgb}{1.000000,0.498039,0.054902}%
\pgfsetstrokecolor{currentstroke}%
\pgfsetdash{}{0pt}%
\pgfpathmoveto{\pgfqpoint{1.892829in}{3.193800in}}%
\pgfpathcurveto{\pgfqpoint{1.903879in}{3.193800in}}{\pgfqpoint{1.914478in}{3.198191in}}{\pgfqpoint{1.922292in}{3.206004in}}%
\pgfpathcurveto{\pgfqpoint{1.930105in}{3.213818in}}{\pgfqpoint{1.934495in}{3.224417in}}{\pgfqpoint{1.934495in}{3.235467in}}%
\pgfpathcurveto{\pgfqpoint{1.934495in}{3.246517in}}{\pgfqpoint{1.930105in}{3.257116in}}{\pgfqpoint{1.922292in}{3.264930in}}%
\pgfpathcurveto{\pgfqpoint{1.914478in}{3.272743in}}{\pgfqpoint{1.903879in}{3.277134in}}{\pgfqpoint{1.892829in}{3.277134in}}%
\pgfpathcurveto{\pgfqpoint{1.881779in}{3.277134in}}{\pgfqpoint{1.871180in}{3.272743in}}{\pgfqpoint{1.863366in}{3.264930in}}%
\pgfpathcurveto{\pgfqpoint{1.855552in}{3.257116in}}{\pgfqpoint{1.851162in}{3.246517in}}{\pgfqpoint{1.851162in}{3.235467in}}%
\pgfpathcurveto{\pgfqpoint{1.851162in}{3.224417in}}{\pgfqpoint{1.855552in}{3.213818in}}{\pgfqpoint{1.863366in}{3.206004in}}%
\pgfpathcurveto{\pgfqpoint{1.871180in}{3.198191in}}{\pgfqpoint{1.881779in}{3.193800in}}{\pgfqpoint{1.892829in}{3.193800in}}%
\pgfpathclose%
\pgfusepath{stroke,fill}%
\end{pgfscope}%
\begin{pgfscope}%
\pgfpathrectangle{\pgfqpoint{0.648703in}{0.548769in}}{\pgfqpoint{5.201297in}{3.102590in}}%
\pgfusepath{clip}%
\pgfsetbuttcap%
\pgfsetroundjoin%
\definecolor{currentfill}{rgb}{1.000000,0.498039,0.054902}%
\pgfsetfillcolor{currentfill}%
\pgfsetlinewidth{1.003750pt}%
\definecolor{currentstroke}{rgb}{1.000000,0.498039,0.054902}%
\pgfsetstrokecolor{currentstroke}%
\pgfsetdash{}{0pt}%
\pgfpathmoveto{\pgfqpoint{2.125376in}{3.193800in}}%
\pgfpathcurveto{\pgfqpoint{2.136426in}{3.193800in}}{\pgfqpoint{2.147025in}{3.198191in}}{\pgfqpoint{2.154838in}{3.206004in}}%
\pgfpathcurveto{\pgfqpoint{2.162652in}{3.213818in}}{\pgfqpoint{2.167042in}{3.224417in}}{\pgfqpoint{2.167042in}{3.235467in}}%
\pgfpathcurveto{\pgfqpoint{2.167042in}{3.246517in}}{\pgfqpoint{2.162652in}{3.257116in}}{\pgfqpoint{2.154838in}{3.264930in}}%
\pgfpathcurveto{\pgfqpoint{2.147025in}{3.272743in}}{\pgfqpoint{2.136426in}{3.277134in}}{\pgfqpoint{2.125376in}{3.277134in}}%
\pgfpathcurveto{\pgfqpoint{2.114325in}{3.277134in}}{\pgfqpoint{2.103726in}{3.272743in}}{\pgfqpoint{2.095913in}{3.264930in}}%
\pgfpathcurveto{\pgfqpoint{2.088099in}{3.257116in}}{\pgfqpoint{2.083709in}{3.246517in}}{\pgfqpoint{2.083709in}{3.235467in}}%
\pgfpathcurveto{\pgfqpoint{2.083709in}{3.224417in}}{\pgfqpoint{2.088099in}{3.213818in}}{\pgfqpoint{2.095913in}{3.206004in}}%
\pgfpathcurveto{\pgfqpoint{2.103726in}{3.198191in}}{\pgfqpoint{2.114325in}{3.193800in}}{\pgfqpoint{2.125376in}{3.193800in}}%
\pgfpathclose%
\pgfusepath{stroke,fill}%
\end{pgfscope}%
\begin{pgfscope}%
\pgfpathrectangle{\pgfqpoint{0.648703in}{0.548769in}}{\pgfqpoint{5.201297in}{3.102590in}}%
\pgfusepath{clip}%
\pgfsetbuttcap%
\pgfsetroundjoin%
\definecolor{currentfill}{rgb}{0.121569,0.466667,0.705882}%
\pgfsetfillcolor{currentfill}%
\pgfsetlinewidth{1.003750pt}%
\definecolor{currentstroke}{rgb}{0.121569,0.466667,0.705882}%
\pgfsetstrokecolor{currentstroke}%
\pgfsetdash{}{0pt}%
\pgfpathmoveto{\pgfqpoint{2.202891in}{0.652358in}}%
\pgfpathcurveto{\pgfqpoint{2.213941in}{0.652358in}}{\pgfqpoint{2.224540in}{0.656748in}}{\pgfqpoint{2.232354in}{0.664562in}}%
\pgfpathcurveto{\pgfqpoint{2.240168in}{0.672375in}}{\pgfqpoint{2.244558in}{0.682974in}}{\pgfqpoint{2.244558in}{0.694024in}}%
\pgfpathcurveto{\pgfqpoint{2.244558in}{0.705074in}}{\pgfqpoint{2.240168in}{0.715673in}}{\pgfqpoint{2.232354in}{0.723487in}}%
\pgfpathcurveto{\pgfqpoint{2.224540in}{0.731301in}}{\pgfqpoint{2.213941in}{0.735691in}}{\pgfqpoint{2.202891in}{0.735691in}}%
\pgfpathcurveto{\pgfqpoint{2.191841in}{0.735691in}}{\pgfqpoint{2.181242in}{0.731301in}}{\pgfqpoint{2.173428in}{0.723487in}}%
\pgfpathcurveto{\pgfqpoint{2.165615in}{0.715673in}}{\pgfqpoint{2.161224in}{0.705074in}}{\pgfqpoint{2.161224in}{0.694024in}}%
\pgfpathcurveto{\pgfqpoint{2.161224in}{0.682974in}}{\pgfqpoint{2.165615in}{0.672375in}}{\pgfqpoint{2.173428in}{0.664562in}}%
\pgfpathcurveto{\pgfqpoint{2.181242in}{0.656748in}}{\pgfqpoint{2.191841in}{0.652358in}}{\pgfqpoint{2.202891in}{0.652358in}}%
\pgfpathclose%
\pgfusepath{stroke,fill}%
\end{pgfscope}%
\begin{pgfscope}%
\pgfpathrectangle{\pgfqpoint{0.648703in}{0.548769in}}{\pgfqpoint{5.201297in}{3.102590in}}%
\pgfusepath{clip}%
\pgfsetbuttcap%
\pgfsetroundjoin%
\definecolor{currentfill}{rgb}{0.121569,0.466667,0.705882}%
\pgfsetfillcolor{currentfill}%
\pgfsetlinewidth{1.003750pt}%
\definecolor{currentstroke}{rgb}{0.121569,0.466667,0.705882}%
\pgfsetstrokecolor{currentstroke}%
\pgfsetdash{}{0pt}%
\pgfpathmoveto{\pgfqpoint{1.117673in}{0.648129in}}%
\pgfpathcurveto{\pgfqpoint{1.128723in}{0.648129in}}{\pgfqpoint{1.139322in}{0.652519in}}{\pgfqpoint{1.147136in}{0.660333in}}%
\pgfpathcurveto{\pgfqpoint{1.154949in}{0.668146in}}{\pgfqpoint{1.159339in}{0.678745in}}{\pgfqpoint{1.159339in}{0.689796in}}%
\pgfpathcurveto{\pgfqpoint{1.159339in}{0.700846in}}{\pgfqpoint{1.154949in}{0.711445in}}{\pgfqpoint{1.147136in}{0.719258in}}%
\pgfpathcurveto{\pgfqpoint{1.139322in}{0.727072in}}{\pgfqpoint{1.128723in}{0.731462in}}{\pgfqpoint{1.117673in}{0.731462in}}%
\pgfpathcurveto{\pgfqpoint{1.106623in}{0.731462in}}{\pgfqpoint{1.096024in}{0.727072in}}{\pgfqpoint{1.088210in}{0.719258in}}%
\pgfpathcurveto{\pgfqpoint{1.080396in}{0.711445in}}{\pgfqpoint{1.076006in}{0.700846in}}{\pgfqpoint{1.076006in}{0.689796in}}%
\pgfpathcurveto{\pgfqpoint{1.076006in}{0.678745in}}{\pgfqpoint{1.080396in}{0.668146in}}{\pgfqpoint{1.088210in}{0.660333in}}%
\pgfpathcurveto{\pgfqpoint{1.096024in}{0.652519in}}{\pgfqpoint{1.106623in}{0.648129in}}{\pgfqpoint{1.117673in}{0.648129in}}%
\pgfpathclose%
\pgfusepath{stroke,fill}%
\end{pgfscope}%
\begin{pgfscope}%
\pgfpathrectangle{\pgfqpoint{0.648703in}{0.548769in}}{\pgfqpoint{5.201297in}{3.102590in}}%
\pgfusepath{clip}%
\pgfsetbuttcap%
\pgfsetroundjoin%
\definecolor{currentfill}{rgb}{0.121569,0.466667,0.705882}%
\pgfsetfillcolor{currentfill}%
\pgfsetlinewidth{1.003750pt}%
\definecolor{currentstroke}{rgb}{0.121569,0.466667,0.705882}%
\pgfsetstrokecolor{currentstroke}%
\pgfsetdash{}{0pt}%
\pgfpathmoveto{\pgfqpoint{1.117673in}{0.648129in}}%
\pgfpathcurveto{\pgfqpoint{1.128723in}{0.648129in}}{\pgfqpoint{1.139322in}{0.652519in}}{\pgfqpoint{1.147136in}{0.660333in}}%
\pgfpathcurveto{\pgfqpoint{1.154949in}{0.668146in}}{\pgfqpoint{1.159339in}{0.678745in}}{\pgfqpoint{1.159339in}{0.689796in}}%
\pgfpathcurveto{\pgfqpoint{1.159339in}{0.700846in}}{\pgfqpoint{1.154949in}{0.711445in}}{\pgfqpoint{1.147136in}{0.719258in}}%
\pgfpathcurveto{\pgfqpoint{1.139322in}{0.727072in}}{\pgfqpoint{1.128723in}{0.731462in}}{\pgfqpoint{1.117673in}{0.731462in}}%
\pgfpathcurveto{\pgfqpoint{1.106623in}{0.731462in}}{\pgfqpoint{1.096024in}{0.727072in}}{\pgfqpoint{1.088210in}{0.719258in}}%
\pgfpathcurveto{\pgfqpoint{1.080396in}{0.711445in}}{\pgfqpoint{1.076006in}{0.700846in}}{\pgfqpoint{1.076006in}{0.689796in}}%
\pgfpathcurveto{\pgfqpoint{1.076006in}{0.678745in}}{\pgfqpoint{1.080396in}{0.668146in}}{\pgfqpoint{1.088210in}{0.660333in}}%
\pgfpathcurveto{\pgfqpoint{1.096024in}{0.652519in}}{\pgfqpoint{1.106623in}{0.648129in}}{\pgfqpoint{1.117673in}{0.648129in}}%
\pgfpathclose%
\pgfusepath{stroke,fill}%
\end{pgfscope}%
\begin{pgfscope}%
\pgfpathrectangle{\pgfqpoint{0.648703in}{0.548769in}}{\pgfqpoint{5.201297in}{3.102590in}}%
\pgfusepath{clip}%
\pgfsetbuttcap%
\pgfsetroundjoin%
\definecolor{currentfill}{rgb}{0.121569,0.466667,0.705882}%
\pgfsetfillcolor{currentfill}%
\pgfsetlinewidth{1.003750pt}%
\definecolor{currentstroke}{rgb}{0.121569,0.466667,0.705882}%
\pgfsetstrokecolor{currentstroke}%
\pgfsetdash{}{0pt}%
\pgfpathmoveto{\pgfqpoint{1.660282in}{0.648129in}}%
\pgfpathcurveto{\pgfqpoint{1.671332in}{0.648129in}}{\pgfqpoint{1.681931in}{0.652519in}}{\pgfqpoint{1.689745in}{0.660333in}}%
\pgfpathcurveto{\pgfqpoint{1.697558in}{0.668146in}}{\pgfqpoint{1.701949in}{0.678745in}}{\pgfqpoint{1.701949in}{0.689796in}}%
\pgfpathcurveto{\pgfqpoint{1.701949in}{0.700846in}}{\pgfqpoint{1.697558in}{0.711445in}}{\pgfqpoint{1.689745in}{0.719258in}}%
\pgfpathcurveto{\pgfqpoint{1.681931in}{0.727072in}}{\pgfqpoint{1.671332in}{0.731462in}}{\pgfqpoint{1.660282in}{0.731462in}}%
\pgfpathcurveto{\pgfqpoint{1.649232in}{0.731462in}}{\pgfqpoint{1.638633in}{0.727072in}}{\pgfqpoint{1.630819in}{0.719258in}}%
\pgfpathcurveto{\pgfqpoint{1.623006in}{0.711445in}}{\pgfqpoint{1.618615in}{0.700846in}}{\pgfqpoint{1.618615in}{0.689796in}}%
\pgfpathcurveto{\pgfqpoint{1.618615in}{0.678745in}}{\pgfqpoint{1.623006in}{0.668146in}}{\pgfqpoint{1.630819in}{0.660333in}}%
\pgfpathcurveto{\pgfqpoint{1.638633in}{0.652519in}}{\pgfqpoint{1.649232in}{0.648129in}}{\pgfqpoint{1.660282in}{0.648129in}}%
\pgfpathclose%
\pgfusepath{stroke,fill}%
\end{pgfscope}%
\begin{pgfscope}%
\pgfpathrectangle{\pgfqpoint{0.648703in}{0.548769in}}{\pgfqpoint{5.201297in}{3.102590in}}%
\pgfusepath{clip}%
\pgfsetbuttcap%
\pgfsetroundjoin%
\definecolor{currentfill}{rgb}{0.121569,0.466667,0.705882}%
\pgfsetfillcolor{currentfill}%
\pgfsetlinewidth{1.003750pt}%
\definecolor{currentstroke}{rgb}{0.121569,0.466667,0.705882}%
\pgfsetstrokecolor{currentstroke}%
\pgfsetdash{}{0pt}%
\pgfpathmoveto{\pgfqpoint{0.962642in}{0.648129in}}%
\pgfpathcurveto{\pgfqpoint{0.973692in}{0.648129in}}{\pgfqpoint{0.984291in}{0.652519in}}{\pgfqpoint{0.992104in}{0.660333in}}%
\pgfpathcurveto{\pgfqpoint{0.999918in}{0.668146in}}{\pgfqpoint{1.004308in}{0.678745in}}{\pgfqpoint{1.004308in}{0.689796in}}%
\pgfpathcurveto{\pgfqpoint{1.004308in}{0.700846in}}{\pgfqpoint{0.999918in}{0.711445in}}{\pgfqpoint{0.992104in}{0.719258in}}%
\pgfpathcurveto{\pgfqpoint{0.984291in}{0.727072in}}{\pgfqpoint{0.973692in}{0.731462in}}{\pgfqpoint{0.962642in}{0.731462in}}%
\pgfpathcurveto{\pgfqpoint{0.951591in}{0.731462in}}{\pgfqpoint{0.940992in}{0.727072in}}{\pgfqpoint{0.933179in}{0.719258in}}%
\pgfpathcurveto{\pgfqpoint{0.925365in}{0.711445in}}{\pgfqpoint{0.920975in}{0.700846in}}{\pgfqpoint{0.920975in}{0.689796in}}%
\pgfpathcurveto{\pgfqpoint{0.920975in}{0.678745in}}{\pgfqpoint{0.925365in}{0.668146in}}{\pgfqpoint{0.933179in}{0.660333in}}%
\pgfpathcurveto{\pgfqpoint{0.940992in}{0.652519in}}{\pgfqpoint{0.951591in}{0.648129in}}{\pgfqpoint{0.962642in}{0.648129in}}%
\pgfpathclose%
\pgfusepath{stroke,fill}%
\end{pgfscope}%
\begin{pgfscope}%
\pgfpathrectangle{\pgfqpoint{0.648703in}{0.548769in}}{\pgfqpoint{5.201297in}{3.102590in}}%
\pgfusepath{clip}%
\pgfsetbuttcap%
\pgfsetroundjoin%
\definecolor{currentfill}{rgb}{0.121569,0.466667,0.705882}%
\pgfsetfillcolor{currentfill}%
\pgfsetlinewidth{1.003750pt}%
\definecolor{currentstroke}{rgb}{0.121569,0.466667,0.705882}%
\pgfsetstrokecolor{currentstroke}%
\pgfsetdash{}{0pt}%
\pgfpathmoveto{\pgfqpoint{1.272704in}{0.648129in}}%
\pgfpathcurveto{\pgfqpoint{1.283754in}{0.648129in}}{\pgfqpoint{1.294353in}{0.652519in}}{\pgfqpoint{1.302167in}{0.660333in}}%
\pgfpathcurveto{\pgfqpoint{1.309980in}{0.668146in}}{\pgfqpoint{1.314371in}{0.678745in}}{\pgfqpoint{1.314371in}{0.689796in}}%
\pgfpathcurveto{\pgfqpoint{1.314371in}{0.700846in}}{\pgfqpoint{1.309980in}{0.711445in}}{\pgfqpoint{1.302167in}{0.719258in}}%
\pgfpathcurveto{\pgfqpoint{1.294353in}{0.727072in}}{\pgfqpoint{1.283754in}{0.731462in}}{\pgfqpoint{1.272704in}{0.731462in}}%
\pgfpathcurveto{\pgfqpoint{1.261654in}{0.731462in}}{\pgfqpoint{1.251055in}{0.727072in}}{\pgfqpoint{1.243241in}{0.719258in}}%
\pgfpathcurveto{\pgfqpoint{1.235428in}{0.711445in}}{\pgfqpoint{1.231037in}{0.700846in}}{\pgfqpoint{1.231037in}{0.689796in}}%
\pgfpathcurveto{\pgfqpoint{1.231037in}{0.678745in}}{\pgfqpoint{1.235428in}{0.668146in}}{\pgfqpoint{1.243241in}{0.660333in}}%
\pgfpathcurveto{\pgfqpoint{1.251055in}{0.652519in}}{\pgfqpoint{1.261654in}{0.648129in}}{\pgfqpoint{1.272704in}{0.648129in}}%
\pgfpathclose%
\pgfusepath{stroke,fill}%
\end{pgfscope}%
\begin{pgfscope}%
\pgfpathrectangle{\pgfqpoint{0.648703in}{0.548769in}}{\pgfqpoint{5.201297in}{3.102590in}}%
\pgfusepath{clip}%
\pgfsetbuttcap%
\pgfsetroundjoin%
\definecolor{currentfill}{rgb}{0.839216,0.152941,0.156863}%
\pgfsetfillcolor{currentfill}%
\pgfsetlinewidth{1.003750pt}%
\definecolor{currentstroke}{rgb}{0.839216,0.152941,0.156863}%
\pgfsetstrokecolor{currentstroke}%
\pgfsetdash{}{0pt}%
\pgfpathmoveto{\pgfqpoint{2.512954in}{3.189572in}}%
\pgfpathcurveto{\pgfqpoint{2.524004in}{3.189572in}}{\pgfqpoint{2.534603in}{3.193962in}}{\pgfqpoint{2.542416in}{3.201775in}}%
\pgfpathcurveto{\pgfqpoint{2.550230in}{3.209589in}}{\pgfqpoint{2.554620in}{3.220188in}}{\pgfqpoint{2.554620in}{3.231238in}}%
\pgfpathcurveto{\pgfqpoint{2.554620in}{3.242288in}}{\pgfqpoint{2.550230in}{3.252887in}}{\pgfqpoint{2.542416in}{3.260701in}}%
\pgfpathcurveto{\pgfqpoint{2.534603in}{3.268515in}}{\pgfqpoint{2.524004in}{3.272905in}}{\pgfqpoint{2.512954in}{3.272905in}}%
\pgfpathcurveto{\pgfqpoint{2.501903in}{3.272905in}}{\pgfqpoint{2.491304in}{3.268515in}}{\pgfqpoint{2.483491in}{3.260701in}}%
\pgfpathcurveto{\pgfqpoint{2.475677in}{3.252887in}}{\pgfqpoint{2.471287in}{3.242288in}}{\pgfqpoint{2.471287in}{3.231238in}}%
\pgfpathcurveto{\pgfqpoint{2.471287in}{3.220188in}}{\pgfqpoint{2.475677in}{3.209589in}}{\pgfqpoint{2.483491in}{3.201775in}}%
\pgfpathcurveto{\pgfqpoint{2.491304in}{3.193962in}}{\pgfqpoint{2.501903in}{3.189572in}}{\pgfqpoint{2.512954in}{3.189572in}}%
\pgfpathclose%
\pgfusepath{stroke,fill}%
\end{pgfscope}%
\begin{pgfscope}%
\pgfpathrectangle{\pgfqpoint{0.648703in}{0.548769in}}{\pgfqpoint{5.201297in}{3.102590in}}%
\pgfusepath{clip}%
\pgfsetbuttcap%
\pgfsetroundjoin%
\definecolor{currentfill}{rgb}{0.121569,0.466667,0.705882}%
\pgfsetfillcolor{currentfill}%
\pgfsetlinewidth{1.003750pt}%
\definecolor{currentstroke}{rgb}{0.121569,0.466667,0.705882}%
\pgfsetstrokecolor{currentstroke}%
\pgfsetdash{}{0pt}%
\pgfpathmoveto{\pgfqpoint{2.745500in}{3.181114in}}%
\pgfpathcurveto{\pgfqpoint{2.756550in}{3.181114in}}{\pgfqpoint{2.767149in}{3.185504in}}{\pgfqpoint{2.774963in}{3.193318in}}%
\pgfpathcurveto{\pgfqpoint{2.782777in}{3.201132in}}{\pgfqpoint{2.787167in}{3.211731in}}{\pgfqpoint{2.787167in}{3.222781in}}%
\pgfpathcurveto{\pgfqpoint{2.787167in}{3.233831in}}{\pgfqpoint{2.782777in}{3.244430in}}{\pgfqpoint{2.774963in}{3.252244in}}%
\pgfpathcurveto{\pgfqpoint{2.767149in}{3.260057in}}{\pgfqpoint{2.756550in}{3.264448in}}{\pgfqpoint{2.745500in}{3.264448in}}%
\pgfpathcurveto{\pgfqpoint{2.734450in}{3.264448in}}{\pgfqpoint{2.723851in}{3.260057in}}{\pgfqpoint{2.716038in}{3.252244in}}%
\pgfpathcurveto{\pgfqpoint{2.708224in}{3.244430in}}{\pgfqpoint{2.703834in}{3.233831in}}{\pgfqpoint{2.703834in}{3.222781in}}%
\pgfpathcurveto{\pgfqpoint{2.703834in}{3.211731in}}{\pgfqpoint{2.708224in}{3.201132in}}{\pgfqpoint{2.716038in}{3.193318in}}%
\pgfpathcurveto{\pgfqpoint{2.723851in}{3.185504in}}{\pgfqpoint{2.734450in}{3.181114in}}{\pgfqpoint{2.745500in}{3.181114in}}%
\pgfpathclose%
\pgfusepath{stroke,fill}%
\end{pgfscope}%
\begin{pgfscope}%
\pgfpathrectangle{\pgfqpoint{0.648703in}{0.548769in}}{\pgfqpoint{5.201297in}{3.102590in}}%
\pgfusepath{clip}%
\pgfsetbuttcap%
\pgfsetroundjoin%
\definecolor{currentfill}{rgb}{1.000000,0.498039,0.054902}%
\pgfsetfillcolor{currentfill}%
\pgfsetlinewidth{1.003750pt}%
\definecolor{currentstroke}{rgb}{1.000000,0.498039,0.054902}%
\pgfsetstrokecolor{currentstroke}%
\pgfsetdash{}{0pt}%
\pgfpathmoveto{\pgfqpoint{2.590469in}{3.193800in}}%
\pgfpathcurveto{\pgfqpoint{2.601519in}{3.193800in}}{\pgfqpoint{2.612118in}{3.198191in}}{\pgfqpoint{2.619932in}{3.206004in}}%
\pgfpathcurveto{\pgfqpoint{2.627746in}{3.213818in}}{\pgfqpoint{2.632136in}{3.224417in}}{\pgfqpoint{2.632136in}{3.235467in}}%
\pgfpathcurveto{\pgfqpoint{2.632136in}{3.246517in}}{\pgfqpoint{2.627746in}{3.257116in}}{\pgfqpoint{2.619932in}{3.264930in}}%
\pgfpathcurveto{\pgfqpoint{2.612118in}{3.272743in}}{\pgfqpoint{2.601519in}{3.277134in}}{\pgfqpoint{2.590469in}{3.277134in}}%
\pgfpathcurveto{\pgfqpoint{2.579419in}{3.277134in}}{\pgfqpoint{2.568820in}{3.272743in}}{\pgfqpoint{2.561006in}{3.264930in}}%
\pgfpathcurveto{\pgfqpoint{2.553193in}{3.257116in}}{\pgfqpoint{2.548802in}{3.246517in}}{\pgfqpoint{2.548802in}{3.235467in}}%
\pgfpathcurveto{\pgfqpoint{2.548802in}{3.224417in}}{\pgfqpoint{2.553193in}{3.213818in}}{\pgfqpoint{2.561006in}{3.206004in}}%
\pgfpathcurveto{\pgfqpoint{2.568820in}{3.198191in}}{\pgfqpoint{2.579419in}{3.193800in}}{\pgfqpoint{2.590469in}{3.193800in}}%
\pgfpathclose%
\pgfusepath{stroke,fill}%
\end{pgfscope}%
\begin{pgfscope}%
\pgfpathrectangle{\pgfqpoint{0.648703in}{0.548769in}}{\pgfqpoint{5.201297in}{3.102590in}}%
\pgfusepath{clip}%
\pgfsetbuttcap%
\pgfsetroundjoin%
\definecolor{currentfill}{rgb}{1.000000,0.498039,0.054902}%
\pgfsetfillcolor{currentfill}%
\pgfsetlinewidth{1.003750pt}%
\definecolor{currentstroke}{rgb}{1.000000,0.498039,0.054902}%
\pgfsetstrokecolor{currentstroke}%
\pgfsetdash{}{0pt}%
\pgfpathmoveto{\pgfqpoint{2.900532in}{3.193800in}}%
\pgfpathcurveto{\pgfqpoint{2.911582in}{3.193800in}}{\pgfqpoint{2.922181in}{3.198191in}}{\pgfqpoint{2.929994in}{3.206004in}}%
\pgfpathcurveto{\pgfqpoint{2.937808in}{3.213818in}}{\pgfqpoint{2.942198in}{3.224417in}}{\pgfqpoint{2.942198in}{3.235467in}}%
\pgfpathcurveto{\pgfqpoint{2.942198in}{3.246517in}}{\pgfqpoint{2.937808in}{3.257116in}}{\pgfqpoint{2.929994in}{3.264930in}}%
\pgfpathcurveto{\pgfqpoint{2.922181in}{3.272743in}}{\pgfqpoint{2.911582in}{3.277134in}}{\pgfqpoint{2.900532in}{3.277134in}}%
\pgfpathcurveto{\pgfqpoint{2.889481in}{3.277134in}}{\pgfqpoint{2.878882in}{3.272743in}}{\pgfqpoint{2.871069in}{3.264930in}}%
\pgfpathcurveto{\pgfqpoint{2.863255in}{3.257116in}}{\pgfqpoint{2.858865in}{3.246517in}}{\pgfqpoint{2.858865in}{3.235467in}}%
\pgfpathcurveto{\pgfqpoint{2.858865in}{3.224417in}}{\pgfqpoint{2.863255in}{3.213818in}}{\pgfqpoint{2.871069in}{3.206004in}}%
\pgfpathcurveto{\pgfqpoint{2.878882in}{3.198191in}}{\pgfqpoint{2.889481in}{3.193800in}}{\pgfqpoint{2.900532in}{3.193800in}}%
\pgfpathclose%
\pgfusepath{stroke,fill}%
\end{pgfscope}%
\begin{pgfscope}%
\pgfpathrectangle{\pgfqpoint{0.648703in}{0.548769in}}{\pgfqpoint{5.201297in}{3.102590in}}%
\pgfusepath{clip}%
\pgfsetbuttcap%
\pgfsetroundjoin%
\definecolor{currentfill}{rgb}{0.121569,0.466667,0.705882}%
\pgfsetfillcolor{currentfill}%
\pgfsetlinewidth{1.003750pt}%
\definecolor{currentstroke}{rgb}{0.121569,0.466667,0.705882}%
\pgfsetstrokecolor{currentstroke}%
\pgfsetdash{}{0pt}%
\pgfpathmoveto{\pgfqpoint{1.195188in}{2.551039in}}%
\pgfpathcurveto{\pgfqpoint{1.206239in}{2.551039in}}{\pgfqpoint{1.216838in}{2.555430in}}{\pgfqpoint{1.224651in}{2.563243in}}%
\pgfpathcurveto{\pgfqpoint{1.232465in}{2.571057in}}{\pgfqpoint{1.236855in}{2.581656in}}{\pgfqpoint{1.236855in}{2.592706in}}%
\pgfpathcurveto{\pgfqpoint{1.236855in}{2.603756in}}{\pgfqpoint{1.232465in}{2.614355in}}{\pgfqpoint{1.224651in}{2.622169in}}%
\pgfpathcurveto{\pgfqpoint{1.216838in}{2.629982in}}{\pgfqpoint{1.206239in}{2.634373in}}{\pgfqpoint{1.195188in}{2.634373in}}%
\pgfpathcurveto{\pgfqpoint{1.184138in}{2.634373in}}{\pgfqpoint{1.173539in}{2.629982in}}{\pgfqpoint{1.165726in}{2.622169in}}%
\pgfpathcurveto{\pgfqpoint{1.157912in}{2.614355in}}{\pgfqpoint{1.153522in}{2.603756in}}{\pgfqpoint{1.153522in}{2.592706in}}%
\pgfpathcurveto{\pgfqpoint{1.153522in}{2.581656in}}{\pgfqpoint{1.157912in}{2.571057in}}{\pgfqpoint{1.165726in}{2.563243in}}%
\pgfpathcurveto{\pgfqpoint{1.173539in}{2.555430in}}{\pgfqpoint{1.184138in}{2.551039in}}{\pgfqpoint{1.195188in}{2.551039in}}%
\pgfpathclose%
\pgfusepath{stroke,fill}%
\end{pgfscope}%
\begin{pgfscope}%
\pgfpathrectangle{\pgfqpoint{0.648703in}{0.548769in}}{\pgfqpoint{5.201297in}{3.102590in}}%
\pgfusepath{clip}%
\pgfsetbuttcap%
\pgfsetroundjoin%
\definecolor{currentfill}{rgb}{0.121569,0.466667,0.705882}%
\pgfsetfillcolor{currentfill}%
\pgfsetlinewidth{1.003750pt}%
\definecolor{currentstroke}{rgb}{0.121569,0.466667,0.705882}%
\pgfsetstrokecolor{currentstroke}%
\pgfsetdash{}{0pt}%
\pgfpathmoveto{\pgfqpoint{0.962642in}{0.648129in}}%
\pgfpathcurveto{\pgfqpoint{0.973692in}{0.648129in}}{\pgfqpoint{0.984291in}{0.652519in}}{\pgfqpoint{0.992104in}{0.660333in}}%
\pgfpathcurveto{\pgfqpoint{0.999918in}{0.668146in}}{\pgfqpoint{1.004308in}{0.678745in}}{\pgfqpoint{1.004308in}{0.689796in}}%
\pgfpathcurveto{\pgfqpoint{1.004308in}{0.700846in}}{\pgfqpoint{0.999918in}{0.711445in}}{\pgfqpoint{0.992104in}{0.719258in}}%
\pgfpathcurveto{\pgfqpoint{0.984291in}{0.727072in}}{\pgfqpoint{0.973692in}{0.731462in}}{\pgfqpoint{0.962642in}{0.731462in}}%
\pgfpathcurveto{\pgfqpoint{0.951591in}{0.731462in}}{\pgfqpoint{0.940992in}{0.727072in}}{\pgfqpoint{0.933179in}{0.719258in}}%
\pgfpathcurveto{\pgfqpoint{0.925365in}{0.711445in}}{\pgfqpoint{0.920975in}{0.700846in}}{\pgfqpoint{0.920975in}{0.689796in}}%
\pgfpathcurveto{\pgfqpoint{0.920975in}{0.678745in}}{\pgfqpoint{0.925365in}{0.668146in}}{\pgfqpoint{0.933179in}{0.660333in}}%
\pgfpathcurveto{\pgfqpoint{0.940992in}{0.652519in}}{\pgfqpoint{0.951591in}{0.648129in}}{\pgfqpoint{0.962642in}{0.648129in}}%
\pgfpathclose%
\pgfusepath{stroke,fill}%
\end{pgfscope}%
\begin{pgfscope}%
\pgfpathrectangle{\pgfqpoint{0.648703in}{0.548769in}}{\pgfqpoint{5.201297in}{3.102590in}}%
\pgfusepath{clip}%
\pgfsetbuttcap%
\pgfsetroundjoin%
\definecolor{currentfill}{rgb}{1.000000,0.498039,0.054902}%
\pgfsetfillcolor{currentfill}%
\pgfsetlinewidth{1.003750pt}%
\definecolor{currentstroke}{rgb}{1.000000,0.498039,0.054902}%
\pgfsetstrokecolor{currentstroke}%
\pgfsetdash{}{0pt}%
\pgfpathmoveto{\pgfqpoint{5.613577in}{3.198029in}}%
\pgfpathcurveto{\pgfqpoint{5.624628in}{3.198029in}}{\pgfqpoint{5.635227in}{3.202419in}}{\pgfqpoint{5.643040in}{3.210233in}}%
\pgfpathcurveto{\pgfqpoint{5.650854in}{3.218046in}}{\pgfqpoint{5.655244in}{3.228646in}}{\pgfqpoint{5.655244in}{3.239696in}}%
\pgfpathcurveto{\pgfqpoint{5.655244in}{3.250746in}}{\pgfqpoint{5.650854in}{3.261345in}}{\pgfqpoint{5.643040in}{3.269158in}}%
\pgfpathcurveto{\pgfqpoint{5.635227in}{3.276972in}}{\pgfqpoint{5.624628in}{3.281362in}}{\pgfqpoint{5.613577in}{3.281362in}}%
\pgfpathcurveto{\pgfqpoint{5.602527in}{3.281362in}}{\pgfqpoint{5.591928in}{3.276972in}}{\pgfqpoint{5.584115in}{3.269158in}}%
\pgfpathcurveto{\pgfqpoint{5.576301in}{3.261345in}}{\pgfqpoint{5.571911in}{3.250746in}}{\pgfqpoint{5.571911in}{3.239696in}}%
\pgfpathcurveto{\pgfqpoint{5.571911in}{3.228646in}}{\pgfqpoint{5.576301in}{3.218046in}}{\pgfqpoint{5.584115in}{3.210233in}}%
\pgfpathcurveto{\pgfqpoint{5.591928in}{3.202419in}}{\pgfqpoint{5.602527in}{3.198029in}}{\pgfqpoint{5.613577in}{3.198029in}}%
\pgfpathclose%
\pgfusepath{stroke,fill}%
\end{pgfscope}%
\begin{pgfscope}%
\pgfpathrectangle{\pgfqpoint{0.648703in}{0.548769in}}{\pgfqpoint{5.201297in}{3.102590in}}%
\pgfusepath{clip}%
\pgfsetbuttcap%
\pgfsetroundjoin%
\definecolor{currentfill}{rgb}{1.000000,0.498039,0.054902}%
\pgfsetfillcolor{currentfill}%
\pgfsetlinewidth{1.003750pt}%
\definecolor{currentstroke}{rgb}{1.000000,0.498039,0.054902}%
\pgfsetstrokecolor{currentstroke}%
\pgfsetdash{}{0pt}%
\pgfpathmoveto{\pgfqpoint{2.667985in}{3.189572in}}%
\pgfpathcurveto{\pgfqpoint{2.679035in}{3.189572in}}{\pgfqpoint{2.689634in}{3.193962in}}{\pgfqpoint{2.697448in}{3.201775in}}%
\pgfpathcurveto{\pgfqpoint{2.705261in}{3.209589in}}{\pgfqpoint{2.709651in}{3.220188in}}{\pgfqpoint{2.709651in}{3.231238in}}%
\pgfpathcurveto{\pgfqpoint{2.709651in}{3.242288in}}{\pgfqpoint{2.705261in}{3.252887in}}{\pgfqpoint{2.697448in}{3.260701in}}%
\pgfpathcurveto{\pgfqpoint{2.689634in}{3.268515in}}{\pgfqpoint{2.679035in}{3.272905in}}{\pgfqpoint{2.667985in}{3.272905in}}%
\pgfpathcurveto{\pgfqpoint{2.656935in}{3.272905in}}{\pgfqpoint{2.646336in}{3.268515in}}{\pgfqpoint{2.638522in}{3.260701in}}%
\pgfpathcurveto{\pgfqpoint{2.630708in}{3.252887in}}{\pgfqpoint{2.626318in}{3.242288in}}{\pgfqpoint{2.626318in}{3.231238in}}%
\pgfpathcurveto{\pgfqpoint{2.626318in}{3.220188in}}{\pgfqpoint{2.630708in}{3.209589in}}{\pgfqpoint{2.638522in}{3.201775in}}%
\pgfpathcurveto{\pgfqpoint{2.646336in}{3.193962in}}{\pgfqpoint{2.656935in}{3.189572in}}{\pgfqpoint{2.667985in}{3.189572in}}%
\pgfpathclose%
\pgfusepath{stroke,fill}%
\end{pgfscope}%
\begin{pgfscope}%
\pgfpathrectangle{\pgfqpoint{0.648703in}{0.548769in}}{\pgfqpoint{5.201297in}{3.102590in}}%
\pgfusepath{clip}%
\pgfsetbuttcap%
\pgfsetroundjoin%
\definecolor{currentfill}{rgb}{0.121569,0.466667,0.705882}%
\pgfsetfillcolor{currentfill}%
\pgfsetlinewidth{1.003750pt}%
\definecolor{currentstroke}{rgb}{0.121569,0.466667,0.705882}%
\pgfsetstrokecolor{currentstroke}%
\pgfsetdash{}{0pt}%
\pgfpathmoveto{\pgfqpoint{1.350220in}{0.808819in}}%
\pgfpathcurveto{\pgfqpoint{1.361270in}{0.808819in}}{\pgfqpoint{1.371869in}{0.813209in}}{\pgfqpoint{1.379682in}{0.821023in}}%
\pgfpathcurveto{\pgfqpoint{1.387496in}{0.828837in}}{\pgfqpoint{1.391886in}{0.839436in}}{\pgfqpoint{1.391886in}{0.850486in}}%
\pgfpathcurveto{\pgfqpoint{1.391886in}{0.861536in}}{\pgfqpoint{1.387496in}{0.872135in}}{\pgfqpoint{1.379682in}{0.879949in}}%
\pgfpathcurveto{\pgfqpoint{1.371869in}{0.887762in}}{\pgfqpoint{1.361270in}{0.892152in}}{\pgfqpoint{1.350220in}{0.892152in}}%
\pgfpathcurveto{\pgfqpoint{1.339169in}{0.892152in}}{\pgfqpoint{1.328570in}{0.887762in}}{\pgfqpoint{1.320757in}{0.879949in}}%
\pgfpathcurveto{\pgfqpoint{1.312943in}{0.872135in}}{\pgfqpoint{1.308553in}{0.861536in}}{\pgfqpoint{1.308553in}{0.850486in}}%
\pgfpathcurveto{\pgfqpoint{1.308553in}{0.839436in}}{\pgfqpoint{1.312943in}{0.828837in}}{\pgfqpoint{1.320757in}{0.821023in}}%
\pgfpathcurveto{\pgfqpoint{1.328570in}{0.813209in}}{\pgfqpoint{1.339169in}{0.808819in}}{\pgfqpoint{1.350220in}{0.808819in}}%
\pgfpathclose%
\pgfusepath{stroke,fill}%
\end{pgfscope}%
\begin{pgfscope}%
\pgfpathrectangle{\pgfqpoint{0.648703in}{0.548769in}}{\pgfqpoint{5.201297in}{3.102590in}}%
\pgfusepath{clip}%
\pgfsetbuttcap%
\pgfsetroundjoin%
\definecolor{currentfill}{rgb}{0.839216,0.152941,0.156863}%
\pgfsetfillcolor{currentfill}%
\pgfsetlinewidth{1.003750pt}%
\definecolor{currentstroke}{rgb}{0.839216,0.152941,0.156863}%
\pgfsetstrokecolor{currentstroke}%
\pgfsetdash{}{0pt}%
\pgfpathmoveto{\pgfqpoint{1.892829in}{3.198029in}}%
\pgfpathcurveto{\pgfqpoint{1.903879in}{3.198029in}}{\pgfqpoint{1.914478in}{3.202419in}}{\pgfqpoint{1.922292in}{3.210233in}}%
\pgfpathcurveto{\pgfqpoint{1.930105in}{3.218046in}}{\pgfqpoint{1.934495in}{3.228646in}}{\pgfqpoint{1.934495in}{3.239696in}}%
\pgfpathcurveto{\pgfqpoint{1.934495in}{3.250746in}}{\pgfqpoint{1.930105in}{3.261345in}}{\pgfqpoint{1.922292in}{3.269158in}}%
\pgfpathcurveto{\pgfqpoint{1.914478in}{3.276972in}}{\pgfqpoint{1.903879in}{3.281362in}}{\pgfqpoint{1.892829in}{3.281362in}}%
\pgfpathcurveto{\pgfqpoint{1.881779in}{3.281362in}}{\pgfqpoint{1.871180in}{3.276972in}}{\pgfqpoint{1.863366in}{3.269158in}}%
\pgfpathcurveto{\pgfqpoint{1.855552in}{3.261345in}}{\pgfqpoint{1.851162in}{3.250746in}}{\pgfqpoint{1.851162in}{3.239696in}}%
\pgfpathcurveto{\pgfqpoint{1.851162in}{3.228646in}}{\pgfqpoint{1.855552in}{3.218046in}}{\pgfqpoint{1.863366in}{3.210233in}}%
\pgfpathcurveto{\pgfqpoint{1.871180in}{3.202419in}}{\pgfqpoint{1.881779in}{3.198029in}}{\pgfqpoint{1.892829in}{3.198029in}}%
\pgfpathclose%
\pgfusepath{stroke,fill}%
\end{pgfscope}%
\begin{pgfscope}%
\pgfpathrectangle{\pgfqpoint{0.648703in}{0.548769in}}{\pgfqpoint{5.201297in}{3.102590in}}%
\pgfusepath{clip}%
\pgfsetbuttcap%
\pgfsetroundjoin%
\definecolor{currentfill}{rgb}{0.121569,0.466667,0.705882}%
\pgfsetfillcolor{currentfill}%
\pgfsetlinewidth{1.003750pt}%
\definecolor{currentstroke}{rgb}{0.121569,0.466667,0.705882}%
\pgfsetstrokecolor{currentstroke}%
\pgfsetdash{}{0pt}%
\pgfpathmoveto{\pgfqpoint{1.195188in}{0.648129in}}%
\pgfpathcurveto{\pgfqpoint{1.206239in}{0.648129in}}{\pgfqpoint{1.216838in}{0.652519in}}{\pgfqpoint{1.224651in}{0.660333in}}%
\pgfpathcurveto{\pgfqpoint{1.232465in}{0.668146in}}{\pgfqpoint{1.236855in}{0.678745in}}{\pgfqpoint{1.236855in}{0.689796in}}%
\pgfpathcurveto{\pgfqpoint{1.236855in}{0.700846in}}{\pgfqpoint{1.232465in}{0.711445in}}{\pgfqpoint{1.224651in}{0.719258in}}%
\pgfpathcurveto{\pgfqpoint{1.216838in}{0.727072in}}{\pgfqpoint{1.206239in}{0.731462in}}{\pgfqpoint{1.195188in}{0.731462in}}%
\pgfpathcurveto{\pgfqpoint{1.184138in}{0.731462in}}{\pgfqpoint{1.173539in}{0.727072in}}{\pgfqpoint{1.165726in}{0.719258in}}%
\pgfpathcurveto{\pgfqpoint{1.157912in}{0.711445in}}{\pgfqpoint{1.153522in}{0.700846in}}{\pgfqpoint{1.153522in}{0.689796in}}%
\pgfpathcurveto{\pgfqpoint{1.153522in}{0.678745in}}{\pgfqpoint{1.157912in}{0.668146in}}{\pgfqpoint{1.165726in}{0.660333in}}%
\pgfpathcurveto{\pgfqpoint{1.173539in}{0.652519in}}{\pgfqpoint{1.184138in}{0.648129in}}{\pgfqpoint{1.195188in}{0.648129in}}%
\pgfpathclose%
\pgfusepath{stroke,fill}%
\end{pgfscope}%
\begin{pgfscope}%
\pgfpathrectangle{\pgfqpoint{0.648703in}{0.548769in}}{\pgfqpoint{5.201297in}{3.102590in}}%
\pgfusepath{clip}%
\pgfsetbuttcap%
\pgfsetroundjoin%
\definecolor{currentfill}{rgb}{1.000000,0.498039,0.054902}%
\pgfsetfillcolor{currentfill}%
\pgfsetlinewidth{1.003750pt}%
\definecolor{currentstroke}{rgb}{1.000000,0.498039,0.054902}%
\pgfsetstrokecolor{currentstroke}%
\pgfsetdash{}{0pt}%
\pgfpathmoveto{\pgfqpoint{3.288110in}{3.185343in}}%
\pgfpathcurveto{\pgfqpoint{3.299160in}{3.185343in}}{\pgfqpoint{3.309759in}{3.189733in}}{\pgfqpoint{3.317572in}{3.197547in}}%
\pgfpathcurveto{\pgfqpoint{3.325386in}{3.205360in}}{\pgfqpoint{3.329776in}{3.215959in}}{\pgfqpoint{3.329776in}{3.227010in}}%
\pgfpathcurveto{\pgfqpoint{3.329776in}{3.238060in}}{\pgfqpoint{3.325386in}{3.248659in}}{\pgfqpoint{3.317572in}{3.256472in}}%
\pgfpathcurveto{\pgfqpoint{3.309759in}{3.264286in}}{\pgfqpoint{3.299160in}{3.268676in}}{\pgfqpoint{3.288110in}{3.268676in}}%
\pgfpathcurveto{\pgfqpoint{3.277059in}{3.268676in}}{\pgfqpoint{3.266460in}{3.264286in}}{\pgfqpoint{3.258647in}{3.256472in}}%
\pgfpathcurveto{\pgfqpoint{3.250833in}{3.248659in}}{\pgfqpoint{3.246443in}{3.238060in}}{\pgfqpoint{3.246443in}{3.227010in}}%
\pgfpathcurveto{\pgfqpoint{3.246443in}{3.215959in}}{\pgfqpoint{3.250833in}{3.205360in}}{\pgfqpoint{3.258647in}{3.197547in}}%
\pgfpathcurveto{\pgfqpoint{3.266460in}{3.189733in}}{\pgfqpoint{3.277059in}{3.185343in}}{\pgfqpoint{3.288110in}{3.185343in}}%
\pgfpathclose%
\pgfusepath{stroke,fill}%
\end{pgfscope}%
\begin{pgfscope}%
\pgfpathrectangle{\pgfqpoint{0.648703in}{0.548769in}}{\pgfqpoint{5.201297in}{3.102590in}}%
\pgfusepath{clip}%
\pgfsetbuttcap%
\pgfsetroundjoin%
\definecolor{currentfill}{rgb}{0.121569,0.466667,0.705882}%
\pgfsetfillcolor{currentfill}%
\pgfsetlinewidth{1.003750pt}%
\definecolor{currentstroke}{rgb}{0.121569,0.466667,0.705882}%
\pgfsetstrokecolor{currentstroke}%
\pgfsetdash{}{0pt}%
\pgfpathmoveto{\pgfqpoint{1.195188in}{0.939909in}}%
\pgfpathcurveto{\pgfqpoint{1.206239in}{0.939909in}}{\pgfqpoint{1.216838in}{0.944299in}}{\pgfqpoint{1.224651in}{0.952112in}}%
\pgfpathcurveto{\pgfqpoint{1.232465in}{0.959926in}}{\pgfqpoint{1.236855in}{0.970525in}}{\pgfqpoint{1.236855in}{0.981575in}}%
\pgfpathcurveto{\pgfqpoint{1.236855in}{0.992625in}}{\pgfqpoint{1.232465in}{1.003224in}}{\pgfqpoint{1.224651in}{1.011038in}}%
\pgfpathcurveto{\pgfqpoint{1.216838in}{1.018852in}}{\pgfqpoint{1.206239in}{1.023242in}}{\pgfqpoint{1.195188in}{1.023242in}}%
\pgfpathcurveto{\pgfqpoint{1.184138in}{1.023242in}}{\pgfqpoint{1.173539in}{1.018852in}}{\pgfqpoint{1.165726in}{1.011038in}}%
\pgfpathcurveto{\pgfqpoint{1.157912in}{1.003224in}}{\pgfqpoint{1.153522in}{0.992625in}}{\pgfqpoint{1.153522in}{0.981575in}}%
\pgfpathcurveto{\pgfqpoint{1.153522in}{0.970525in}}{\pgfqpoint{1.157912in}{0.959926in}}{\pgfqpoint{1.165726in}{0.952112in}}%
\pgfpathcurveto{\pgfqpoint{1.173539in}{0.944299in}}{\pgfqpoint{1.184138in}{0.939909in}}{\pgfqpoint{1.195188in}{0.939909in}}%
\pgfpathclose%
\pgfusepath{stroke,fill}%
\end{pgfscope}%
\begin{pgfscope}%
\pgfpathrectangle{\pgfqpoint{0.648703in}{0.548769in}}{\pgfqpoint{5.201297in}{3.102590in}}%
\pgfusepath{clip}%
\pgfsetbuttcap%
\pgfsetroundjoin%
\definecolor{currentfill}{rgb}{0.121569,0.466667,0.705882}%
\pgfsetfillcolor{currentfill}%
\pgfsetlinewidth{1.003750pt}%
\definecolor{currentstroke}{rgb}{0.121569,0.466667,0.705882}%
\pgfsetstrokecolor{currentstroke}%
\pgfsetdash{}{0pt}%
\pgfpathmoveto{\pgfqpoint{0.962642in}{0.648129in}}%
\pgfpathcurveto{\pgfqpoint{0.973692in}{0.648129in}}{\pgfqpoint{0.984291in}{0.652519in}}{\pgfqpoint{0.992104in}{0.660333in}}%
\pgfpathcurveto{\pgfqpoint{0.999918in}{0.668146in}}{\pgfqpoint{1.004308in}{0.678745in}}{\pgfqpoint{1.004308in}{0.689796in}}%
\pgfpathcurveto{\pgfqpoint{1.004308in}{0.700846in}}{\pgfqpoint{0.999918in}{0.711445in}}{\pgfqpoint{0.992104in}{0.719258in}}%
\pgfpathcurveto{\pgfqpoint{0.984291in}{0.727072in}}{\pgfqpoint{0.973692in}{0.731462in}}{\pgfqpoint{0.962642in}{0.731462in}}%
\pgfpathcurveto{\pgfqpoint{0.951591in}{0.731462in}}{\pgfqpoint{0.940992in}{0.727072in}}{\pgfqpoint{0.933179in}{0.719258in}}%
\pgfpathcurveto{\pgfqpoint{0.925365in}{0.711445in}}{\pgfqpoint{0.920975in}{0.700846in}}{\pgfqpoint{0.920975in}{0.689796in}}%
\pgfpathcurveto{\pgfqpoint{0.920975in}{0.678745in}}{\pgfqpoint{0.925365in}{0.668146in}}{\pgfqpoint{0.933179in}{0.660333in}}%
\pgfpathcurveto{\pgfqpoint{0.940992in}{0.652519in}}{\pgfqpoint{0.951591in}{0.648129in}}{\pgfqpoint{0.962642in}{0.648129in}}%
\pgfpathclose%
\pgfusepath{stroke,fill}%
\end{pgfscope}%
\begin{pgfscope}%
\pgfpathrectangle{\pgfqpoint{0.648703in}{0.548769in}}{\pgfqpoint{5.201297in}{3.102590in}}%
\pgfusepath{clip}%
\pgfsetbuttcap%
\pgfsetroundjoin%
\definecolor{currentfill}{rgb}{1.000000,0.498039,0.054902}%
\pgfsetfillcolor{currentfill}%
\pgfsetlinewidth{1.003750pt}%
\definecolor{currentstroke}{rgb}{1.000000,0.498039,0.054902}%
\pgfsetstrokecolor{currentstroke}%
\pgfsetdash{}{0pt}%
\pgfpathmoveto{\pgfqpoint{1.195188in}{3.185343in}}%
\pgfpathcurveto{\pgfqpoint{1.206239in}{3.185343in}}{\pgfqpoint{1.216838in}{3.189733in}}{\pgfqpoint{1.224651in}{3.197547in}}%
\pgfpathcurveto{\pgfqpoint{1.232465in}{3.205360in}}{\pgfqpoint{1.236855in}{3.215959in}}{\pgfqpoint{1.236855in}{3.227010in}}%
\pgfpathcurveto{\pgfqpoint{1.236855in}{3.238060in}}{\pgfqpoint{1.232465in}{3.248659in}}{\pgfqpoint{1.224651in}{3.256472in}}%
\pgfpathcurveto{\pgfqpoint{1.216838in}{3.264286in}}{\pgfqpoint{1.206239in}{3.268676in}}{\pgfqpoint{1.195188in}{3.268676in}}%
\pgfpathcurveto{\pgfqpoint{1.184138in}{3.268676in}}{\pgfqpoint{1.173539in}{3.264286in}}{\pgfqpoint{1.165726in}{3.256472in}}%
\pgfpathcurveto{\pgfqpoint{1.157912in}{3.248659in}}{\pgfqpoint{1.153522in}{3.238060in}}{\pgfqpoint{1.153522in}{3.227010in}}%
\pgfpathcurveto{\pgfqpoint{1.153522in}{3.215959in}}{\pgfqpoint{1.157912in}{3.205360in}}{\pgfqpoint{1.165726in}{3.197547in}}%
\pgfpathcurveto{\pgfqpoint{1.173539in}{3.189733in}}{\pgfqpoint{1.184138in}{3.185343in}}{\pgfqpoint{1.195188in}{3.185343in}}%
\pgfpathclose%
\pgfusepath{stroke,fill}%
\end{pgfscope}%
\begin{pgfscope}%
\pgfpathrectangle{\pgfqpoint{0.648703in}{0.548769in}}{\pgfqpoint{5.201297in}{3.102590in}}%
\pgfusepath{clip}%
\pgfsetbuttcap%
\pgfsetroundjoin%
\definecolor{currentfill}{rgb}{0.121569,0.466667,0.705882}%
\pgfsetfillcolor{currentfill}%
\pgfsetlinewidth{1.003750pt}%
\definecolor{currentstroke}{rgb}{0.121569,0.466667,0.705882}%
\pgfsetstrokecolor{currentstroke}%
\pgfsetdash{}{0pt}%
\pgfpathmoveto{\pgfqpoint{0.962642in}{0.648129in}}%
\pgfpathcurveto{\pgfqpoint{0.973692in}{0.648129in}}{\pgfqpoint{0.984291in}{0.652519in}}{\pgfqpoint{0.992104in}{0.660333in}}%
\pgfpathcurveto{\pgfqpoint{0.999918in}{0.668146in}}{\pgfqpoint{1.004308in}{0.678745in}}{\pgfqpoint{1.004308in}{0.689796in}}%
\pgfpathcurveto{\pgfqpoint{1.004308in}{0.700846in}}{\pgfqpoint{0.999918in}{0.711445in}}{\pgfqpoint{0.992104in}{0.719258in}}%
\pgfpathcurveto{\pgfqpoint{0.984291in}{0.727072in}}{\pgfqpoint{0.973692in}{0.731462in}}{\pgfqpoint{0.962642in}{0.731462in}}%
\pgfpathcurveto{\pgfqpoint{0.951591in}{0.731462in}}{\pgfqpoint{0.940992in}{0.727072in}}{\pgfqpoint{0.933179in}{0.719258in}}%
\pgfpathcurveto{\pgfqpoint{0.925365in}{0.711445in}}{\pgfqpoint{0.920975in}{0.700846in}}{\pgfqpoint{0.920975in}{0.689796in}}%
\pgfpathcurveto{\pgfqpoint{0.920975in}{0.678745in}}{\pgfqpoint{0.925365in}{0.668146in}}{\pgfqpoint{0.933179in}{0.660333in}}%
\pgfpathcurveto{\pgfqpoint{0.940992in}{0.652519in}}{\pgfqpoint{0.951591in}{0.648129in}}{\pgfqpoint{0.962642in}{0.648129in}}%
\pgfpathclose%
\pgfusepath{stroke,fill}%
\end{pgfscope}%
\begin{pgfscope}%
\pgfpathrectangle{\pgfqpoint{0.648703in}{0.548769in}}{\pgfqpoint{5.201297in}{3.102590in}}%
\pgfusepath{clip}%
\pgfsetbuttcap%
\pgfsetroundjoin%
\definecolor{currentfill}{rgb}{1.000000,0.498039,0.054902}%
\pgfsetfillcolor{currentfill}%
\pgfsetlinewidth{1.003750pt}%
\definecolor{currentstroke}{rgb}{1.000000,0.498039,0.054902}%
\pgfsetstrokecolor{currentstroke}%
\pgfsetdash{}{0pt}%
\pgfpathmoveto{\pgfqpoint{1.272704in}{3.278374in}}%
\pgfpathcurveto{\pgfqpoint{1.283754in}{3.278374in}}{\pgfqpoint{1.294353in}{3.282764in}}{\pgfqpoint{1.302167in}{3.290578in}}%
\pgfpathcurveto{\pgfqpoint{1.309980in}{3.298392in}}{\pgfqpoint{1.314371in}{3.308991in}}{\pgfqpoint{1.314371in}{3.320041in}}%
\pgfpathcurveto{\pgfqpoint{1.314371in}{3.331091in}}{\pgfqpoint{1.309980in}{3.341690in}}{\pgfqpoint{1.302167in}{3.349504in}}%
\pgfpathcurveto{\pgfqpoint{1.294353in}{3.357317in}}{\pgfqpoint{1.283754in}{3.361707in}}{\pgfqpoint{1.272704in}{3.361707in}}%
\pgfpathcurveto{\pgfqpoint{1.261654in}{3.361707in}}{\pgfqpoint{1.251055in}{3.357317in}}{\pgfqpoint{1.243241in}{3.349504in}}%
\pgfpathcurveto{\pgfqpoint{1.235428in}{3.341690in}}{\pgfqpoint{1.231037in}{3.331091in}}{\pgfqpoint{1.231037in}{3.320041in}}%
\pgfpathcurveto{\pgfqpoint{1.231037in}{3.308991in}}{\pgfqpoint{1.235428in}{3.298392in}}{\pgfqpoint{1.243241in}{3.290578in}}%
\pgfpathcurveto{\pgfqpoint{1.251055in}{3.282764in}}{\pgfqpoint{1.261654in}{3.278374in}}{\pgfqpoint{1.272704in}{3.278374in}}%
\pgfpathclose%
\pgfusepath{stroke,fill}%
\end{pgfscope}%
\begin{pgfscope}%
\pgfpathrectangle{\pgfqpoint{0.648703in}{0.548769in}}{\pgfqpoint{5.201297in}{3.102590in}}%
\pgfusepath{clip}%
\pgfsetbuttcap%
\pgfsetroundjoin%
\definecolor{currentfill}{rgb}{1.000000,0.498039,0.054902}%
\pgfsetfillcolor{currentfill}%
\pgfsetlinewidth{1.003750pt}%
\definecolor{currentstroke}{rgb}{1.000000,0.498039,0.054902}%
\pgfsetstrokecolor{currentstroke}%
\pgfsetdash{}{0pt}%
\pgfpathmoveto{\pgfqpoint{1.892829in}{3.202258in}}%
\pgfpathcurveto{\pgfqpoint{1.903879in}{3.202258in}}{\pgfqpoint{1.914478in}{3.206648in}}{\pgfqpoint{1.922292in}{3.214462in}}%
\pgfpathcurveto{\pgfqpoint{1.930105in}{3.222275in}}{\pgfqpoint{1.934495in}{3.232874in}}{\pgfqpoint{1.934495in}{3.243924in}}%
\pgfpathcurveto{\pgfqpoint{1.934495in}{3.254974in}}{\pgfqpoint{1.930105in}{3.265573in}}{\pgfqpoint{1.922292in}{3.273387in}}%
\pgfpathcurveto{\pgfqpoint{1.914478in}{3.281201in}}{\pgfqpoint{1.903879in}{3.285591in}}{\pgfqpoint{1.892829in}{3.285591in}}%
\pgfpathcurveto{\pgfqpoint{1.881779in}{3.285591in}}{\pgfqpoint{1.871180in}{3.281201in}}{\pgfqpoint{1.863366in}{3.273387in}}%
\pgfpathcurveto{\pgfqpoint{1.855552in}{3.265573in}}{\pgfqpoint{1.851162in}{3.254974in}}{\pgfqpoint{1.851162in}{3.243924in}}%
\pgfpathcurveto{\pgfqpoint{1.851162in}{3.232874in}}{\pgfqpoint{1.855552in}{3.222275in}}{\pgfqpoint{1.863366in}{3.214462in}}%
\pgfpathcurveto{\pgfqpoint{1.871180in}{3.206648in}}{\pgfqpoint{1.881779in}{3.202258in}}{\pgfqpoint{1.892829in}{3.202258in}}%
\pgfpathclose%
\pgfusepath{stroke,fill}%
\end{pgfscope}%
\begin{pgfscope}%
\pgfpathrectangle{\pgfqpoint{0.648703in}{0.548769in}}{\pgfqpoint{5.201297in}{3.102590in}}%
\pgfusepath{clip}%
\pgfsetbuttcap%
\pgfsetroundjoin%
\definecolor{currentfill}{rgb}{0.121569,0.466667,0.705882}%
\pgfsetfillcolor{currentfill}%
\pgfsetlinewidth{1.003750pt}%
\definecolor{currentstroke}{rgb}{0.121569,0.466667,0.705882}%
\pgfsetstrokecolor{currentstroke}%
\pgfsetdash{}{0pt}%
\pgfpathmoveto{\pgfqpoint{0.962642in}{0.648129in}}%
\pgfpathcurveto{\pgfqpoint{0.973692in}{0.648129in}}{\pgfqpoint{0.984291in}{0.652519in}}{\pgfqpoint{0.992104in}{0.660333in}}%
\pgfpathcurveto{\pgfqpoint{0.999918in}{0.668146in}}{\pgfqpoint{1.004308in}{0.678745in}}{\pgfqpoint{1.004308in}{0.689796in}}%
\pgfpathcurveto{\pgfqpoint{1.004308in}{0.700846in}}{\pgfqpoint{0.999918in}{0.711445in}}{\pgfqpoint{0.992104in}{0.719258in}}%
\pgfpathcurveto{\pgfqpoint{0.984291in}{0.727072in}}{\pgfqpoint{0.973692in}{0.731462in}}{\pgfqpoint{0.962642in}{0.731462in}}%
\pgfpathcurveto{\pgfqpoint{0.951591in}{0.731462in}}{\pgfqpoint{0.940992in}{0.727072in}}{\pgfqpoint{0.933179in}{0.719258in}}%
\pgfpathcurveto{\pgfqpoint{0.925365in}{0.711445in}}{\pgfqpoint{0.920975in}{0.700846in}}{\pgfqpoint{0.920975in}{0.689796in}}%
\pgfpathcurveto{\pgfqpoint{0.920975in}{0.678745in}}{\pgfqpoint{0.925365in}{0.668146in}}{\pgfqpoint{0.933179in}{0.660333in}}%
\pgfpathcurveto{\pgfqpoint{0.940992in}{0.652519in}}{\pgfqpoint{0.951591in}{0.648129in}}{\pgfqpoint{0.962642in}{0.648129in}}%
\pgfpathclose%
\pgfusepath{stroke,fill}%
\end{pgfscope}%
\begin{pgfscope}%
\pgfpathrectangle{\pgfqpoint{0.648703in}{0.548769in}}{\pgfqpoint{5.201297in}{3.102590in}}%
\pgfusepath{clip}%
\pgfsetbuttcap%
\pgfsetroundjoin%
\definecolor{currentfill}{rgb}{1.000000,0.498039,0.054902}%
\pgfsetfillcolor{currentfill}%
\pgfsetlinewidth{1.003750pt}%
\definecolor{currentstroke}{rgb}{1.000000,0.498039,0.054902}%
\pgfsetstrokecolor{currentstroke}%
\pgfsetdash{}{0pt}%
\pgfpathmoveto{\pgfqpoint{1.505251in}{3.202258in}}%
\pgfpathcurveto{\pgfqpoint{1.516301in}{3.202258in}}{\pgfqpoint{1.526900in}{3.206648in}}{\pgfqpoint{1.534714in}{3.214462in}}%
\pgfpathcurveto{\pgfqpoint{1.542527in}{3.222275in}}{\pgfqpoint{1.546917in}{3.232874in}}{\pgfqpoint{1.546917in}{3.243924in}}%
\pgfpathcurveto{\pgfqpoint{1.546917in}{3.254974in}}{\pgfqpoint{1.542527in}{3.265573in}}{\pgfqpoint{1.534714in}{3.273387in}}%
\pgfpathcurveto{\pgfqpoint{1.526900in}{3.281201in}}{\pgfqpoint{1.516301in}{3.285591in}}{\pgfqpoint{1.505251in}{3.285591in}}%
\pgfpathcurveto{\pgfqpoint{1.494201in}{3.285591in}}{\pgfqpoint{1.483602in}{3.281201in}}{\pgfqpoint{1.475788in}{3.273387in}}%
\pgfpathcurveto{\pgfqpoint{1.467974in}{3.265573in}}{\pgfqpoint{1.463584in}{3.254974in}}{\pgfqpoint{1.463584in}{3.243924in}}%
\pgfpathcurveto{\pgfqpoint{1.463584in}{3.232874in}}{\pgfqpoint{1.467974in}{3.222275in}}{\pgfqpoint{1.475788in}{3.214462in}}%
\pgfpathcurveto{\pgfqpoint{1.483602in}{3.206648in}}{\pgfqpoint{1.494201in}{3.202258in}}{\pgfqpoint{1.505251in}{3.202258in}}%
\pgfpathclose%
\pgfusepath{stroke,fill}%
\end{pgfscope}%
\begin{pgfscope}%
\pgfpathrectangle{\pgfqpoint{0.648703in}{0.548769in}}{\pgfqpoint{5.201297in}{3.102590in}}%
\pgfusepath{clip}%
\pgfsetbuttcap%
\pgfsetroundjoin%
\definecolor{currentfill}{rgb}{0.121569,0.466667,0.705882}%
\pgfsetfillcolor{currentfill}%
\pgfsetlinewidth{1.003750pt}%
\definecolor{currentstroke}{rgb}{0.121569,0.466667,0.705882}%
\pgfsetstrokecolor{currentstroke}%
\pgfsetdash{}{0pt}%
\pgfpathmoveto{\pgfqpoint{2.280407in}{1.726445in}}%
\pgfpathcurveto{\pgfqpoint{2.291457in}{1.726445in}}{\pgfqpoint{2.302056in}{1.730835in}}{\pgfqpoint{2.309870in}{1.738649in}}%
\pgfpathcurveto{\pgfqpoint{2.317683in}{1.746462in}}{\pgfqpoint{2.322073in}{1.757061in}}{\pgfqpoint{2.322073in}{1.768112in}}%
\pgfpathcurveto{\pgfqpoint{2.322073in}{1.779162in}}{\pgfqpoint{2.317683in}{1.789761in}}{\pgfqpoint{2.309870in}{1.797574in}}%
\pgfpathcurveto{\pgfqpoint{2.302056in}{1.805388in}}{\pgfqpoint{2.291457in}{1.809778in}}{\pgfqpoint{2.280407in}{1.809778in}}%
\pgfpathcurveto{\pgfqpoint{2.269357in}{1.809778in}}{\pgfqpoint{2.258758in}{1.805388in}}{\pgfqpoint{2.250944in}{1.797574in}}%
\pgfpathcurveto{\pgfqpoint{2.243130in}{1.789761in}}{\pgfqpoint{2.238740in}{1.779162in}}{\pgfqpoint{2.238740in}{1.768112in}}%
\pgfpathcurveto{\pgfqpoint{2.238740in}{1.757061in}}{\pgfqpoint{2.243130in}{1.746462in}}{\pgfqpoint{2.250944in}{1.738649in}}%
\pgfpathcurveto{\pgfqpoint{2.258758in}{1.730835in}}{\pgfqpoint{2.269357in}{1.726445in}}{\pgfqpoint{2.280407in}{1.726445in}}%
\pgfpathclose%
\pgfusepath{stroke,fill}%
\end{pgfscope}%
\begin{pgfscope}%
\pgfpathrectangle{\pgfqpoint{0.648703in}{0.548769in}}{\pgfqpoint{5.201297in}{3.102590in}}%
\pgfusepath{clip}%
\pgfsetbuttcap%
\pgfsetroundjoin%
\definecolor{currentfill}{rgb}{0.121569,0.466667,0.705882}%
\pgfsetfillcolor{currentfill}%
\pgfsetlinewidth{1.003750pt}%
\definecolor{currentstroke}{rgb}{0.121569,0.466667,0.705882}%
\pgfsetstrokecolor{currentstroke}%
\pgfsetdash{}{0pt}%
\pgfpathmoveto{\pgfqpoint{3.443141in}{0.918765in}}%
\pgfpathcurveto{\pgfqpoint{3.454191in}{0.918765in}}{\pgfqpoint{3.464790in}{0.923155in}}{\pgfqpoint{3.472603in}{0.930969in}}%
\pgfpathcurveto{\pgfqpoint{3.480417in}{0.938783in}}{\pgfqpoint{3.484807in}{0.949382in}}{\pgfqpoint{3.484807in}{0.960432in}}%
\pgfpathcurveto{\pgfqpoint{3.484807in}{0.971482in}}{\pgfqpoint{3.480417in}{0.982081in}}{\pgfqpoint{3.472603in}{0.989895in}}%
\pgfpathcurveto{\pgfqpoint{3.464790in}{0.997708in}}{\pgfqpoint{3.454191in}{1.002098in}}{\pgfqpoint{3.443141in}{1.002098in}}%
\pgfpathcurveto{\pgfqpoint{3.432091in}{1.002098in}}{\pgfqpoint{3.421492in}{0.997708in}}{\pgfqpoint{3.413678in}{0.989895in}}%
\pgfpathcurveto{\pgfqpoint{3.405864in}{0.982081in}}{\pgfqpoint{3.401474in}{0.971482in}}{\pgfqpoint{3.401474in}{0.960432in}}%
\pgfpathcurveto{\pgfqpoint{3.401474in}{0.949382in}}{\pgfqpoint{3.405864in}{0.938783in}}{\pgfqpoint{3.413678in}{0.930969in}}%
\pgfpathcurveto{\pgfqpoint{3.421492in}{0.923155in}}{\pgfqpoint{3.432091in}{0.918765in}}{\pgfqpoint{3.443141in}{0.918765in}}%
\pgfpathclose%
\pgfusepath{stroke,fill}%
\end{pgfscope}%
\begin{pgfscope}%
\pgfpathrectangle{\pgfqpoint{0.648703in}{0.548769in}}{\pgfqpoint{5.201297in}{3.102590in}}%
\pgfusepath{clip}%
\pgfsetbuttcap%
\pgfsetroundjoin%
\definecolor{currentfill}{rgb}{1.000000,0.498039,0.054902}%
\pgfsetfillcolor{currentfill}%
\pgfsetlinewidth{1.003750pt}%
\definecolor{currentstroke}{rgb}{1.000000,0.498039,0.054902}%
\pgfsetstrokecolor{currentstroke}%
\pgfsetdash{}{0pt}%
\pgfpathmoveto{\pgfqpoint{1.040157in}{3.185343in}}%
\pgfpathcurveto{\pgfqpoint{1.051207in}{3.185343in}}{\pgfqpoint{1.061806in}{3.189733in}}{\pgfqpoint{1.069620in}{3.197547in}}%
\pgfpathcurveto{\pgfqpoint{1.077434in}{3.205360in}}{\pgfqpoint{1.081824in}{3.215959in}}{\pgfqpoint{1.081824in}{3.227010in}}%
\pgfpathcurveto{\pgfqpoint{1.081824in}{3.238060in}}{\pgfqpoint{1.077434in}{3.248659in}}{\pgfqpoint{1.069620in}{3.256472in}}%
\pgfpathcurveto{\pgfqpoint{1.061806in}{3.264286in}}{\pgfqpoint{1.051207in}{3.268676in}}{\pgfqpoint{1.040157in}{3.268676in}}%
\pgfpathcurveto{\pgfqpoint{1.029107in}{3.268676in}}{\pgfqpoint{1.018508in}{3.264286in}}{\pgfqpoint{1.010694in}{3.256472in}}%
\pgfpathcurveto{\pgfqpoint{1.002881in}{3.248659in}}{\pgfqpoint{0.998491in}{3.238060in}}{\pgfqpoint{0.998491in}{3.227010in}}%
\pgfpathcurveto{\pgfqpoint{0.998491in}{3.215959in}}{\pgfqpoint{1.002881in}{3.205360in}}{\pgfqpoint{1.010694in}{3.197547in}}%
\pgfpathcurveto{\pgfqpoint{1.018508in}{3.189733in}}{\pgfqpoint{1.029107in}{3.185343in}}{\pgfqpoint{1.040157in}{3.185343in}}%
\pgfpathclose%
\pgfusepath{stroke,fill}%
\end{pgfscope}%
\begin{pgfscope}%
\pgfpathrectangle{\pgfqpoint{0.648703in}{0.548769in}}{\pgfqpoint{5.201297in}{3.102590in}}%
\pgfusepath{clip}%
\pgfsetbuttcap%
\pgfsetroundjoin%
\definecolor{currentfill}{rgb}{0.121569,0.466667,0.705882}%
\pgfsetfillcolor{currentfill}%
\pgfsetlinewidth{1.003750pt}%
\definecolor{currentstroke}{rgb}{0.121569,0.466667,0.705882}%
\pgfsetstrokecolor{currentstroke}%
\pgfsetdash{}{0pt}%
\pgfpathmoveto{\pgfqpoint{1.272704in}{0.648129in}}%
\pgfpathcurveto{\pgfqpoint{1.283754in}{0.648129in}}{\pgfqpoint{1.294353in}{0.652519in}}{\pgfqpoint{1.302167in}{0.660333in}}%
\pgfpathcurveto{\pgfqpoint{1.309980in}{0.668146in}}{\pgfqpoint{1.314371in}{0.678745in}}{\pgfqpoint{1.314371in}{0.689796in}}%
\pgfpathcurveto{\pgfqpoint{1.314371in}{0.700846in}}{\pgfqpoint{1.309980in}{0.711445in}}{\pgfqpoint{1.302167in}{0.719258in}}%
\pgfpathcurveto{\pgfqpoint{1.294353in}{0.727072in}}{\pgfqpoint{1.283754in}{0.731462in}}{\pgfqpoint{1.272704in}{0.731462in}}%
\pgfpathcurveto{\pgfqpoint{1.261654in}{0.731462in}}{\pgfqpoint{1.251055in}{0.727072in}}{\pgfqpoint{1.243241in}{0.719258in}}%
\pgfpathcurveto{\pgfqpoint{1.235428in}{0.711445in}}{\pgfqpoint{1.231037in}{0.700846in}}{\pgfqpoint{1.231037in}{0.689796in}}%
\pgfpathcurveto{\pgfqpoint{1.231037in}{0.678745in}}{\pgfqpoint{1.235428in}{0.668146in}}{\pgfqpoint{1.243241in}{0.660333in}}%
\pgfpathcurveto{\pgfqpoint{1.251055in}{0.652519in}}{\pgfqpoint{1.261654in}{0.648129in}}{\pgfqpoint{1.272704in}{0.648129in}}%
\pgfpathclose%
\pgfusepath{stroke,fill}%
\end{pgfscope}%
\begin{pgfscope}%
\pgfpathrectangle{\pgfqpoint{0.648703in}{0.548769in}}{\pgfqpoint{5.201297in}{3.102590in}}%
\pgfusepath{clip}%
\pgfsetbuttcap%
\pgfsetroundjoin%
\definecolor{currentfill}{rgb}{1.000000,0.498039,0.054902}%
\pgfsetfillcolor{currentfill}%
\pgfsetlinewidth{1.003750pt}%
\definecolor{currentstroke}{rgb}{1.000000,0.498039,0.054902}%
\pgfsetstrokecolor{currentstroke}%
\pgfsetdash{}{0pt}%
\pgfpathmoveto{\pgfqpoint{2.047860in}{3.206486in}}%
\pgfpathcurveto{\pgfqpoint{2.058910in}{3.206486in}}{\pgfqpoint{2.069509in}{3.210877in}}{\pgfqpoint{2.077323in}{3.218690in}}%
\pgfpathcurveto{\pgfqpoint{2.085136in}{3.226504in}}{\pgfqpoint{2.089527in}{3.237103in}}{\pgfqpoint{2.089527in}{3.248153in}}%
\pgfpathcurveto{\pgfqpoint{2.089527in}{3.259203in}}{\pgfqpoint{2.085136in}{3.269802in}}{\pgfqpoint{2.077323in}{3.277616in}}%
\pgfpathcurveto{\pgfqpoint{2.069509in}{3.285429in}}{\pgfqpoint{2.058910in}{3.289820in}}{\pgfqpoint{2.047860in}{3.289820in}}%
\pgfpathcurveto{\pgfqpoint{2.036810in}{3.289820in}}{\pgfqpoint{2.026211in}{3.285429in}}{\pgfqpoint{2.018397in}{3.277616in}}%
\pgfpathcurveto{\pgfqpoint{2.010584in}{3.269802in}}{\pgfqpoint{2.006193in}{3.259203in}}{\pgfqpoint{2.006193in}{3.248153in}}%
\pgfpathcurveto{\pgfqpoint{2.006193in}{3.237103in}}{\pgfqpoint{2.010584in}{3.226504in}}{\pgfqpoint{2.018397in}{3.218690in}}%
\pgfpathcurveto{\pgfqpoint{2.026211in}{3.210877in}}{\pgfqpoint{2.036810in}{3.206486in}}{\pgfqpoint{2.047860in}{3.206486in}}%
\pgfpathclose%
\pgfusepath{stroke,fill}%
\end{pgfscope}%
\begin{pgfscope}%
\pgfpathrectangle{\pgfqpoint{0.648703in}{0.548769in}}{\pgfqpoint{5.201297in}{3.102590in}}%
\pgfusepath{clip}%
\pgfsetbuttcap%
\pgfsetroundjoin%
\definecolor{currentfill}{rgb}{1.000000,0.498039,0.054902}%
\pgfsetfillcolor{currentfill}%
\pgfsetlinewidth{1.003750pt}%
\definecolor{currentstroke}{rgb}{1.000000,0.498039,0.054902}%
\pgfsetstrokecolor{currentstroke}%
\pgfsetdash{}{0pt}%
\pgfpathmoveto{\pgfqpoint{2.280407in}{3.185343in}}%
\pgfpathcurveto{\pgfqpoint{2.291457in}{3.185343in}}{\pgfqpoint{2.302056in}{3.189733in}}{\pgfqpoint{2.309870in}{3.197547in}}%
\pgfpathcurveto{\pgfqpoint{2.317683in}{3.205360in}}{\pgfqpoint{2.322073in}{3.215959in}}{\pgfqpoint{2.322073in}{3.227010in}}%
\pgfpathcurveto{\pgfqpoint{2.322073in}{3.238060in}}{\pgfqpoint{2.317683in}{3.248659in}}{\pgfqpoint{2.309870in}{3.256472in}}%
\pgfpathcurveto{\pgfqpoint{2.302056in}{3.264286in}}{\pgfqpoint{2.291457in}{3.268676in}}{\pgfqpoint{2.280407in}{3.268676in}}%
\pgfpathcurveto{\pgfqpoint{2.269357in}{3.268676in}}{\pgfqpoint{2.258758in}{3.264286in}}{\pgfqpoint{2.250944in}{3.256472in}}%
\pgfpathcurveto{\pgfqpoint{2.243130in}{3.248659in}}{\pgfqpoint{2.238740in}{3.238060in}}{\pgfqpoint{2.238740in}{3.227010in}}%
\pgfpathcurveto{\pgfqpoint{2.238740in}{3.215959in}}{\pgfqpoint{2.243130in}{3.205360in}}{\pgfqpoint{2.250944in}{3.197547in}}%
\pgfpathcurveto{\pgfqpoint{2.258758in}{3.189733in}}{\pgfqpoint{2.269357in}{3.185343in}}{\pgfqpoint{2.280407in}{3.185343in}}%
\pgfpathclose%
\pgfusepath{stroke,fill}%
\end{pgfscope}%
\begin{pgfscope}%
\pgfpathrectangle{\pgfqpoint{0.648703in}{0.548769in}}{\pgfqpoint{5.201297in}{3.102590in}}%
\pgfusepath{clip}%
\pgfsetbuttcap%
\pgfsetroundjoin%
\definecolor{currentfill}{rgb}{1.000000,0.498039,0.054902}%
\pgfsetfillcolor{currentfill}%
\pgfsetlinewidth{1.003750pt}%
\definecolor{currentstroke}{rgb}{1.000000,0.498039,0.054902}%
\pgfsetstrokecolor{currentstroke}%
\pgfsetdash{}{0pt}%
\pgfpathmoveto{\pgfqpoint{2.125376in}{3.193800in}}%
\pgfpathcurveto{\pgfqpoint{2.136426in}{3.193800in}}{\pgfqpoint{2.147025in}{3.198191in}}{\pgfqpoint{2.154838in}{3.206004in}}%
\pgfpathcurveto{\pgfqpoint{2.162652in}{3.213818in}}{\pgfqpoint{2.167042in}{3.224417in}}{\pgfqpoint{2.167042in}{3.235467in}}%
\pgfpathcurveto{\pgfqpoint{2.167042in}{3.246517in}}{\pgfqpoint{2.162652in}{3.257116in}}{\pgfqpoint{2.154838in}{3.264930in}}%
\pgfpathcurveto{\pgfqpoint{2.147025in}{3.272743in}}{\pgfqpoint{2.136426in}{3.277134in}}{\pgfqpoint{2.125376in}{3.277134in}}%
\pgfpathcurveto{\pgfqpoint{2.114325in}{3.277134in}}{\pgfqpoint{2.103726in}{3.272743in}}{\pgfqpoint{2.095913in}{3.264930in}}%
\pgfpathcurveto{\pgfqpoint{2.088099in}{3.257116in}}{\pgfqpoint{2.083709in}{3.246517in}}{\pgfqpoint{2.083709in}{3.235467in}}%
\pgfpathcurveto{\pgfqpoint{2.083709in}{3.224417in}}{\pgfqpoint{2.088099in}{3.213818in}}{\pgfqpoint{2.095913in}{3.206004in}}%
\pgfpathcurveto{\pgfqpoint{2.103726in}{3.198191in}}{\pgfqpoint{2.114325in}{3.193800in}}{\pgfqpoint{2.125376in}{3.193800in}}%
\pgfpathclose%
\pgfusepath{stroke,fill}%
\end{pgfscope}%
\begin{pgfscope}%
\pgfpathrectangle{\pgfqpoint{0.648703in}{0.548769in}}{\pgfqpoint{5.201297in}{3.102590in}}%
\pgfusepath{clip}%
\pgfsetbuttcap%
\pgfsetroundjoin%
\definecolor{currentfill}{rgb}{1.000000,0.498039,0.054902}%
\pgfsetfillcolor{currentfill}%
\pgfsetlinewidth{1.003750pt}%
\definecolor{currentstroke}{rgb}{1.000000,0.498039,0.054902}%
\pgfsetstrokecolor{currentstroke}%
\pgfsetdash{}{0pt}%
\pgfpathmoveto{\pgfqpoint{1.427735in}{3.198029in}}%
\pgfpathcurveto{\pgfqpoint{1.438785in}{3.198029in}}{\pgfqpoint{1.449384in}{3.202419in}}{\pgfqpoint{1.457198in}{3.210233in}}%
\pgfpathcurveto{\pgfqpoint{1.465012in}{3.218046in}}{\pgfqpoint{1.469402in}{3.228646in}}{\pgfqpoint{1.469402in}{3.239696in}}%
\pgfpathcurveto{\pgfqpoint{1.469402in}{3.250746in}}{\pgfqpoint{1.465012in}{3.261345in}}{\pgfqpoint{1.457198in}{3.269158in}}%
\pgfpathcurveto{\pgfqpoint{1.449384in}{3.276972in}}{\pgfqpoint{1.438785in}{3.281362in}}{\pgfqpoint{1.427735in}{3.281362in}}%
\pgfpathcurveto{\pgfqpoint{1.416685in}{3.281362in}}{\pgfqpoint{1.406086in}{3.276972in}}{\pgfqpoint{1.398272in}{3.269158in}}%
\pgfpathcurveto{\pgfqpoint{1.390459in}{3.261345in}}{\pgfqpoint{1.386069in}{3.250746in}}{\pgfqpoint{1.386069in}{3.239696in}}%
\pgfpathcurveto{\pgfqpoint{1.386069in}{3.228646in}}{\pgfqpoint{1.390459in}{3.218046in}}{\pgfqpoint{1.398272in}{3.210233in}}%
\pgfpathcurveto{\pgfqpoint{1.406086in}{3.202419in}}{\pgfqpoint{1.416685in}{3.198029in}}{\pgfqpoint{1.427735in}{3.198029in}}%
\pgfpathclose%
\pgfusepath{stroke,fill}%
\end{pgfscope}%
\begin{pgfscope}%
\pgfpathrectangle{\pgfqpoint{0.648703in}{0.548769in}}{\pgfqpoint{5.201297in}{3.102590in}}%
\pgfusepath{clip}%
\pgfsetbuttcap%
\pgfsetroundjoin%
\definecolor{currentfill}{rgb}{1.000000,0.498039,0.054902}%
\pgfsetfillcolor{currentfill}%
\pgfsetlinewidth{1.003750pt}%
\definecolor{currentstroke}{rgb}{1.000000,0.498039,0.054902}%
\pgfsetstrokecolor{currentstroke}%
\pgfsetdash{}{0pt}%
\pgfpathmoveto{\pgfqpoint{2.978047in}{3.202258in}}%
\pgfpathcurveto{\pgfqpoint{2.989097in}{3.202258in}}{\pgfqpoint{2.999696in}{3.206648in}}{\pgfqpoint{3.007510in}{3.214462in}}%
\pgfpathcurveto{\pgfqpoint{3.015324in}{3.222275in}}{\pgfqpoint{3.019714in}{3.232874in}}{\pgfqpoint{3.019714in}{3.243924in}}%
\pgfpathcurveto{\pgfqpoint{3.019714in}{3.254974in}}{\pgfqpoint{3.015324in}{3.265573in}}{\pgfqpoint{3.007510in}{3.273387in}}%
\pgfpathcurveto{\pgfqpoint{2.999696in}{3.281201in}}{\pgfqpoint{2.989097in}{3.285591in}}{\pgfqpoint{2.978047in}{3.285591in}}%
\pgfpathcurveto{\pgfqpoint{2.966997in}{3.285591in}}{\pgfqpoint{2.956398in}{3.281201in}}{\pgfqpoint{2.948584in}{3.273387in}}%
\pgfpathcurveto{\pgfqpoint{2.940771in}{3.265573in}}{\pgfqpoint{2.936380in}{3.254974in}}{\pgfqpoint{2.936380in}{3.243924in}}%
\pgfpathcurveto{\pgfqpoint{2.936380in}{3.232874in}}{\pgfqpoint{2.940771in}{3.222275in}}{\pgfqpoint{2.948584in}{3.214462in}}%
\pgfpathcurveto{\pgfqpoint{2.956398in}{3.206648in}}{\pgfqpoint{2.966997in}{3.202258in}}{\pgfqpoint{2.978047in}{3.202258in}}%
\pgfpathclose%
\pgfusepath{stroke,fill}%
\end{pgfscope}%
\begin{pgfscope}%
\pgfpathrectangle{\pgfqpoint{0.648703in}{0.548769in}}{\pgfqpoint{5.201297in}{3.102590in}}%
\pgfusepath{clip}%
\pgfsetbuttcap%
\pgfsetroundjoin%
\definecolor{currentfill}{rgb}{0.121569,0.466667,0.705882}%
\pgfsetfillcolor{currentfill}%
\pgfsetlinewidth{1.003750pt}%
\definecolor{currentstroke}{rgb}{0.121569,0.466667,0.705882}%
\pgfsetstrokecolor{currentstroke}%
\pgfsetdash{}{0pt}%
\pgfpathmoveto{\pgfqpoint{3.520656in}{3.181114in}}%
\pgfpathcurveto{\pgfqpoint{3.531706in}{3.181114in}}{\pgfqpoint{3.542305in}{3.185504in}}{\pgfqpoint{3.550119in}{3.193318in}}%
\pgfpathcurveto{\pgfqpoint{3.557933in}{3.201132in}}{\pgfqpoint{3.562323in}{3.211731in}}{\pgfqpoint{3.562323in}{3.222781in}}%
\pgfpathcurveto{\pgfqpoint{3.562323in}{3.233831in}}{\pgfqpoint{3.557933in}{3.244430in}}{\pgfqpoint{3.550119in}{3.252244in}}%
\pgfpathcurveto{\pgfqpoint{3.542305in}{3.260057in}}{\pgfqpoint{3.531706in}{3.264448in}}{\pgfqpoint{3.520656in}{3.264448in}}%
\pgfpathcurveto{\pgfqpoint{3.509606in}{3.264448in}}{\pgfqpoint{3.499007in}{3.260057in}}{\pgfqpoint{3.491194in}{3.252244in}}%
\pgfpathcurveto{\pgfqpoint{3.483380in}{3.244430in}}{\pgfqpoint{3.478990in}{3.233831in}}{\pgfqpoint{3.478990in}{3.222781in}}%
\pgfpathcurveto{\pgfqpoint{3.478990in}{3.211731in}}{\pgfqpoint{3.483380in}{3.201132in}}{\pgfqpoint{3.491194in}{3.193318in}}%
\pgfpathcurveto{\pgfqpoint{3.499007in}{3.185504in}}{\pgfqpoint{3.509606in}{3.181114in}}{\pgfqpoint{3.520656in}{3.181114in}}%
\pgfpathclose%
\pgfusepath{stroke,fill}%
\end{pgfscope}%
\begin{pgfscope}%
\pgfpathrectangle{\pgfqpoint{0.648703in}{0.548769in}}{\pgfqpoint{5.201297in}{3.102590in}}%
\pgfusepath{clip}%
\pgfsetbuttcap%
\pgfsetroundjoin%
\definecolor{currentfill}{rgb}{1.000000,0.498039,0.054902}%
\pgfsetfillcolor{currentfill}%
\pgfsetlinewidth{1.003750pt}%
\definecolor{currentstroke}{rgb}{1.000000,0.498039,0.054902}%
\pgfsetstrokecolor{currentstroke}%
\pgfsetdash{}{0pt}%
\pgfpathmoveto{\pgfqpoint{1.892829in}{3.244545in}}%
\pgfpathcurveto{\pgfqpoint{1.903879in}{3.244545in}}{\pgfqpoint{1.914478in}{3.248935in}}{\pgfqpoint{1.922292in}{3.256748in}}%
\pgfpathcurveto{\pgfqpoint{1.930105in}{3.264562in}}{\pgfqpoint{1.934495in}{3.275161in}}{\pgfqpoint{1.934495in}{3.286211in}}%
\pgfpathcurveto{\pgfqpoint{1.934495in}{3.297261in}}{\pgfqpoint{1.930105in}{3.307860in}}{\pgfqpoint{1.922292in}{3.315674in}}%
\pgfpathcurveto{\pgfqpoint{1.914478in}{3.323488in}}{\pgfqpoint{1.903879in}{3.327878in}}{\pgfqpoint{1.892829in}{3.327878in}}%
\pgfpathcurveto{\pgfqpoint{1.881779in}{3.327878in}}{\pgfqpoint{1.871180in}{3.323488in}}{\pgfqpoint{1.863366in}{3.315674in}}%
\pgfpathcurveto{\pgfqpoint{1.855552in}{3.307860in}}{\pgfqpoint{1.851162in}{3.297261in}}{\pgfqpoint{1.851162in}{3.286211in}}%
\pgfpathcurveto{\pgfqpoint{1.851162in}{3.275161in}}{\pgfqpoint{1.855552in}{3.264562in}}{\pgfqpoint{1.863366in}{3.256748in}}%
\pgfpathcurveto{\pgfqpoint{1.871180in}{3.248935in}}{\pgfqpoint{1.881779in}{3.244545in}}{\pgfqpoint{1.892829in}{3.244545in}}%
\pgfpathclose%
\pgfusepath{stroke,fill}%
\end{pgfscope}%
\begin{pgfscope}%
\pgfpathrectangle{\pgfqpoint{0.648703in}{0.548769in}}{\pgfqpoint{5.201297in}{3.102590in}}%
\pgfusepath{clip}%
\pgfsetbuttcap%
\pgfsetroundjoin%
\definecolor{currentfill}{rgb}{1.000000,0.498039,0.054902}%
\pgfsetfillcolor{currentfill}%
\pgfsetlinewidth{1.003750pt}%
\definecolor{currentstroke}{rgb}{1.000000,0.498039,0.054902}%
\pgfsetstrokecolor{currentstroke}%
\pgfsetdash{}{0pt}%
\pgfpathmoveto{\pgfqpoint{2.435438in}{3.189572in}}%
\pgfpathcurveto{\pgfqpoint{2.446488in}{3.189572in}}{\pgfqpoint{2.457087in}{3.193962in}}{\pgfqpoint{2.464901in}{3.201775in}}%
\pgfpathcurveto{\pgfqpoint{2.472714in}{3.209589in}}{\pgfqpoint{2.477105in}{3.220188in}}{\pgfqpoint{2.477105in}{3.231238in}}%
\pgfpathcurveto{\pgfqpoint{2.477105in}{3.242288in}}{\pgfqpoint{2.472714in}{3.252887in}}{\pgfqpoint{2.464901in}{3.260701in}}%
\pgfpathcurveto{\pgfqpoint{2.457087in}{3.268515in}}{\pgfqpoint{2.446488in}{3.272905in}}{\pgfqpoint{2.435438in}{3.272905in}}%
\pgfpathcurveto{\pgfqpoint{2.424388in}{3.272905in}}{\pgfqpoint{2.413789in}{3.268515in}}{\pgfqpoint{2.405975in}{3.260701in}}%
\pgfpathcurveto{\pgfqpoint{2.398162in}{3.252887in}}{\pgfqpoint{2.393771in}{3.242288in}}{\pgfqpoint{2.393771in}{3.231238in}}%
\pgfpathcurveto{\pgfqpoint{2.393771in}{3.220188in}}{\pgfqpoint{2.398162in}{3.209589in}}{\pgfqpoint{2.405975in}{3.201775in}}%
\pgfpathcurveto{\pgfqpoint{2.413789in}{3.193962in}}{\pgfqpoint{2.424388in}{3.189572in}}{\pgfqpoint{2.435438in}{3.189572in}}%
\pgfpathclose%
\pgfusepath{stroke,fill}%
\end{pgfscope}%
\begin{pgfscope}%
\pgfpathrectangle{\pgfqpoint{0.648703in}{0.548769in}}{\pgfqpoint{5.201297in}{3.102590in}}%
\pgfusepath{clip}%
\pgfsetbuttcap%
\pgfsetroundjoin%
\definecolor{currentfill}{rgb}{0.121569,0.466667,0.705882}%
\pgfsetfillcolor{currentfill}%
\pgfsetlinewidth{1.003750pt}%
\definecolor{currentstroke}{rgb}{0.121569,0.466667,0.705882}%
\pgfsetstrokecolor{currentstroke}%
\pgfsetdash{}{0pt}%
\pgfpathmoveto{\pgfqpoint{2.047860in}{0.656586in}}%
\pgfpathcurveto{\pgfqpoint{2.058910in}{0.656586in}}{\pgfqpoint{2.069509in}{0.660977in}}{\pgfqpoint{2.077323in}{0.668790in}}%
\pgfpathcurveto{\pgfqpoint{2.085136in}{0.676604in}}{\pgfqpoint{2.089527in}{0.687203in}}{\pgfqpoint{2.089527in}{0.698253in}}%
\pgfpathcurveto{\pgfqpoint{2.089527in}{0.709303in}}{\pgfqpoint{2.085136in}{0.719902in}}{\pgfqpoint{2.077323in}{0.727716in}}%
\pgfpathcurveto{\pgfqpoint{2.069509in}{0.735529in}}{\pgfqpoint{2.058910in}{0.739920in}}{\pgfqpoint{2.047860in}{0.739920in}}%
\pgfpathcurveto{\pgfqpoint{2.036810in}{0.739920in}}{\pgfqpoint{2.026211in}{0.735529in}}{\pgfqpoint{2.018397in}{0.727716in}}%
\pgfpathcurveto{\pgfqpoint{2.010584in}{0.719902in}}{\pgfqpoint{2.006193in}{0.709303in}}{\pgfqpoint{2.006193in}{0.698253in}}%
\pgfpathcurveto{\pgfqpoint{2.006193in}{0.687203in}}{\pgfqpoint{2.010584in}{0.676604in}}{\pgfqpoint{2.018397in}{0.668790in}}%
\pgfpathcurveto{\pgfqpoint{2.026211in}{0.660977in}}{\pgfqpoint{2.036810in}{0.656586in}}{\pgfqpoint{2.047860in}{0.656586in}}%
\pgfpathclose%
\pgfusepath{stroke,fill}%
\end{pgfscope}%
\begin{pgfscope}%
\pgfpathrectangle{\pgfqpoint{0.648703in}{0.548769in}}{\pgfqpoint{5.201297in}{3.102590in}}%
\pgfusepath{clip}%
\pgfsetbuttcap%
\pgfsetroundjoin%
\definecolor{currentfill}{rgb}{1.000000,0.498039,0.054902}%
\pgfsetfillcolor{currentfill}%
\pgfsetlinewidth{1.003750pt}%
\definecolor{currentstroke}{rgb}{1.000000,0.498039,0.054902}%
\pgfsetstrokecolor{currentstroke}%
\pgfsetdash{}{0pt}%
\pgfpathmoveto{\pgfqpoint{1.427735in}{3.210715in}}%
\pgfpathcurveto{\pgfqpoint{1.438785in}{3.210715in}}{\pgfqpoint{1.449384in}{3.215105in}}{\pgfqpoint{1.457198in}{3.222919in}}%
\pgfpathcurveto{\pgfqpoint{1.465012in}{3.230733in}}{\pgfqpoint{1.469402in}{3.241332in}}{\pgfqpoint{1.469402in}{3.252382in}}%
\pgfpathcurveto{\pgfqpoint{1.469402in}{3.263432in}}{\pgfqpoint{1.465012in}{3.274031in}}{\pgfqpoint{1.457198in}{3.281844in}}%
\pgfpathcurveto{\pgfqpoint{1.449384in}{3.289658in}}{\pgfqpoint{1.438785in}{3.294048in}}{\pgfqpoint{1.427735in}{3.294048in}}%
\pgfpathcurveto{\pgfqpoint{1.416685in}{3.294048in}}{\pgfqpoint{1.406086in}{3.289658in}}{\pgfqpoint{1.398272in}{3.281844in}}%
\pgfpathcurveto{\pgfqpoint{1.390459in}{3.274031in}}{\pgfqpoint{1.386069in}{3.263432in}}{\pgfqpoint{1.386069in}{3.252382in}}%
\pgfpathcurveto{\pgfqpoint{1.386069in}{3.241332in}}{\pgfqpoint{1.390459in}{3.230733in}}{\pgfqpoint{1.398272in}{3.222919in}}%
\pgfpathcurveto{\pgfqpoint{1.406086in}{3.215105in}}{\pgfqpoint{1.416685in}{3.210715in}}{\pgfqpoint{1.427735in}{3.210715in}}%
\pgfpathclose%
\pgfusepath{stroke,fill}%
\end{pgfscope}%
\begin{pgfscope}%
\pgfpathrectangle{\pgfqpoint{0.648703in}{0.548769in}}{\pgfqpoint{5.201297in}{3.102590in}}%
\pgfusepath{clip}%
\pgfsetbuttcap%
\pgfsetroundjoin%
\definecolor{currentfill}{rgb}{1.000000,0.498039,0.054902}%
\pgfsetfillcolor{currentfill}%
\pgfsetlinewidth{1.003750pt}%
\definecolor{currentstroke}{rgb}{1.000000,0.498039,0.054902}%
\pgfsetstrokecolor{currentstroke}%
\pgfsetdash{}{0pt}%
\pgfpathmoveto{\pgfqpoint{2.590469in}{3.214944in}}%
\pgfpathcurveto{\pgfqpoint{2.601519in}{3.214944in}}{\pgfqpoint{2.612118in}{3.219334in}}{\pgfqpoint{2.619932in}{3.227148in}}%
\pgfpathcurveto{\pgfqpoint{2.627746in}{3.234961in}}{\pgfqpoint{2.632136in}{3.245560in}}{\pgfqpoint{2.632136in}{3.256610in}}%
\pgfpathcurveto{\pgfqpoint{2.632136in}{3.267661in}}{\pgfqpoint{2.627746in}{3.278260in}}{\pgfqpoint{2.619932in}{3.286073in}}%
\pgfpathcurveto{\pgfqpoint{2.612118in}{3.293887in}}{\pgfqpoint{2.601519in}{3.298277in}}{\pgfqpoint{2.590469in}{3.298277in}}%
\pgfpathcurveto{\pgfqpoint{2.579419in}{3.298277in}}{\pgfqpoint{2.568820in}{3.293887in}}{\pgfqpoint{2.561006in}{3.286073in}}%
\pgfpathcurveto{\pgfqpoint{2.553193in}{3.278260in}}{\pgfqpoint{2.548802in}{3.267661in}}{\pgfqpoint{2.548802in}{3.256610in}}%
\pgfpathcurveto{\pgfqpoint{2.548802in}{3.245560in}}{\pgfqpoint{2.553193in}{3.234961in}}{\pgfqpoint{2.561006in}{3.227148in}}%
\pgfpathcurveto{\pgfqpoint{2.568820in}{3.219334in}}{\pgfqpoint{2.579419in}{3.214944in}}{\pgfqpoint{2.590469in}{3.214944in}}%
\pgfpathclose%
\pgfusepath{stroke,fill}%
\end{pgfscope}%
\begin{pgfscope}%
\pgfpathrectangle{\pgfqpoint{0.648703in}{0.548769in}}{\pgfqpoint{5.201297in}{3.102590in}}%
\pgfusepath{clip}%
\pgfsetbuttcap%
\pgfsetroundjoin%
\definecolor{currentfill}{rgb}{1.000000,0.498039,0.054902}%
\pgfsetfillcolor{currentfill}%
\pgfsetlinewidth{1.003750pt}%
\definecolor{currentstroke}{rgb}{1.000000,0.498039,0.054902}%
\pgfsetstrokecolor{currentstroke}%
\pgfsetdash{}{0pt}%
\pgfpathmoveto{\pgfqpoint{1.970344in}{3.202258in}}%
\pgfpathcurveto{\pgfqpoint{1.981394in}{3.202258in}}{\pgfqpoint{1.991994in}{3.206648in}}{\pgfqpoint{1.999807in}{3.214462in}}%
\pgfpathcurveto{\pgfqpoint{2.007621in}{3.222275in}}{\pgfqpoint{2.012011in}{3.232874in}}{\pgfqpoint{2.012011in}{3.243924in}}%
\pgfpathcurveto{\pgfqpoint{2.012011in}{3.254974in}}{\pgfqpoint{2.007621in}{3.265573in}}{\pgfqpoint{1.999807in}{3.273387in}}%
\pgfpathcurveto{\pgfqpoint{1.991994in}{3.281201in}}{\pgfqpoint{1.981394in}{3.285591in}}{\pgfqpoint{1.970344in}{3.285591in}}%
\pgfpathcurveto{\pgfqpoint{1.959294in}{3.285591in}}{\pgfqpoint{1.948695in}{3.281201in}}{\pgfqpoint{1.940882in}{3.273387in}}%
\pgfpathcurveto{\pgfqpoint{1.933068in}{3.265573in}}{\pgfqpoint{1.928678in}{3.254974in}}{\pgfqpoint{1.928678in}{3.243924in}}%
\pgfpathcurveto{\pgfqpoint{1.928678in}{3.232874in}}{\pgfqpoint{1.933068in}{3.222275in}}{\pgfqpoint{1.940882in}{3.214462in}}%
\pgfpathcurveto{\pgfqpoint{1.948695in}{3.206648in}}{\pgfqpoint{1.959294in}{3.202258in}}{\pgfqpoint{1.970344in}{3.202258in}}%
\pgfpathclose%
\pgfusepath{stroke,fill}%
\end{pgfscope}%
\begin{pgfscope}%
\pgfpathrectangle{\pgfqpoint{0.648703in}{0.548769in}}{\pgfqpoint{5.201297in}{3.102590in}}%
\pgfusepath{clip}%
\pgfsetbuttcap%
\pgfsetroundjoin%
\definecolor{currentfill}{rgb}{1.000000,0.498039,0.054902}%
\pgfsetfillcolor{currentfill}%
\pgfsetlinewidth{1.003750pt}%
\definecolor{currentstroke}{rgb}{1.000000,0.498039,0.054902}%
\pgfsetstrokecolor{currentstroke}%
\pgfsetdash{}{0pt}%
\pgfpathmoveto{\pgfqpoint{2.590469in}{3.189572in}}%
\pgfpathcurveto{\pgfqpoint{2.601519in}{3.189572in}}{\pgfqpoint{2.612118in}{3.193962in}}{\pgfqpoint{2.619932in}{3.201775in}}%
\pgfpathcurveto{\pgfqpoint{2.627746in}{3.209589in}}{\pgfqpoint{2.632136in}{3.220188in}}{\pgfqpoint{2.632136in}{3.231238in}}%
\pgfpathcurveto{\pgfqpoint{2.632136in}{3.242288in}}{\pgfqpoint{2.627746in}{3.252887in}}{\pgfqpoint{2.619932in}{3.260701in}}%
\pgfpathcurveto{\pgfqpoint{2.612118in}{3.268515in}}{\pgfqpoint{2.601519in}{3.272905in}}{\pgfqpoint{2.590469in}{3.272905in}}%
\pgfpathcurveto{\pgfqpoint{2.579419in}{3.272905in}}{\pgfqpoint{2.568820in}{3.268515in}}{\pgfqpoint{2.561006in}{3.260701in}}%
\pgfpathcurveto{\pgfqpoint{2.553193in}{3.252887in}}{\pgfqpoint{2.548802in}{3.242288in}}{\pgfqpoint{2.548802in}{3.231238in}}%
\pgfpathcurveto{\pgfqpoint{2.548802in}{3.220188in}}{\pgfqpoint{2.553193in}{3.209589in}}{\pgfqpoint{2.561006in}{3.201775in}}%
\pgfpathcurveto{\pgfqpoint{2.568820in}{3.193962in}}{\pgfqpoint{2.579419in}{3.189572in}}{\pgfqpoint{2.590469in}{3.189572in}}%
\pgfpathclose%
\pgfusepath{stroke,fill}%
\end{pgfscope}%
\begin{pgfscope}%
\pgfpathrectangle{\pgfqpoint{0.648703in}{0.548769in}}{\pgfqpoint{5.201297in}{3.102590in}}%
\pgfusepath{clip}%
\pgfsetbuttcap%
\pgfsetroundjoin%
\definecolor{currentfill}{rgb}{0.839216,0.152941,0.156863}%
\pgfsetfillcolor{currentfill}%
\pgfsetlinewidth{1.003750pt}%
\definecolor{currentstroke}{rgb}{0.839216,0.152941,0.156863}%
\pgfsetstrokecolor{currentstroke}%
\pgfsetdash{}{0pt}%
\pgfpathmoveto{\pgfqpoint{2.978047in}{3.181114in}}%
\pgfpathcurveto{\pgfqpoint{2.989097in}{3.181114in}}{\pgfqpoint{2.999696in}{3.185504in}}{\pgfqpoint{3.007510in}{3.193318in}}%
\pgfpathcurveto{\pgfqpoint{3.015324in}{3.201132in}}{\pgfqpoint{3.019714in}{3.211731in}}{\pgfqpoint{3.019714in}{3.222781in}}%
\pgfpathcurveto{\pgfqpoint{3.019714in}{3.233831in}}{\pgfqpoint{3.015324in}{3.244430in}}{\pgfqpoint{3.007510in}{3.252244in}}%
\pgfpathcurveto{\pgfqpoint{2.999696in}{3.260057in}}{\pgfqpoint{2.989097in}{3.264448in}}{\pgfqpoint{2.978047in}{3.264448in}}%
\pgfpathcurveto{\pgfqpoint{2.966997in}{3.264448in}}{\pgfqpoint{2.956398in}{3.260057in}}{\pgfqpoint{2.948584in}{3.252244in}}%
\pgfpathcurveto{\pgfqpoint{2.940771in}{3.244430in}}{\pgfqpoint{2.936380in}{3.233831in}}{\pgfqpoint{2.936380in}{3.222781in}}%
\pgfpathcurveto{\pgfqpoint{2.936380in}{3.211731in}}{\pgfqpoint{2.940771in}{3.201132in}}{\pgfqpoint{2.948584in}{3.193318in}}%
\pgfpathcurveto{\pgfqpoint{2.956398in}{3.185504in}}{\pgfqpoint{2.966997in}{3.181114in}}{\pgfqpoint{2.978047in}{3.181114in}}%
\pgfpathclose%
\pgfusepath{stroke,fill}%
\end{pgfscope}%
\begin{pgfscope}%
\pgfpathrectangle{\pgfqpoint{0.648703in}{0.548769in}}{\pgfqpoint{5.201297in}{3.102590in}}%
\pgfusepath{clip}%
\pgfsetbuttcap%
\pgfsetroundjoin%
\definecolor{currentfill}{rgb}{1.000000,0.498039,0.054902}%
\pgfsetfillcolor{currentfill}%
\pgfsetlinewidth{1.003750pt}%
\definecolor{currentstroke}{rgb}{1.000000,0.498039,0.054902}%
\pgfsetstrokecolor{currentstroke}%
\pgfsetdash{}{0pt}%
\pgfpathmoveto{\pgfqpoint{1.815313in}{3.244545in}}%
\pgfpathcurveto{\pgfqpoint{1.826363in}{3.244545in}}{\pgfqpoint{1.836962in}{3.248935in}}{\pgfqpoint{1.844776in}{3.256748in}}%
\pgfpathcurveto{\pgfqpoint{1.852590in}{3.264562in}}{\pgfqpoint{1.856980in}{3.275161in}}{\pgfqpoint{1.856980in}{3.286211in}}%
\pgfpathcurveto{\pgfqpoint{1.856980in}{3.297261in}}{\pgfqpoint{1.852590in}{3.307860in}}{\pgfqpoint{1.844776in}{3.315674in}}%
\pgfpathcurveto{\pgfqpoint{1.836962in}{3.323488in}}{\pgfqpoint{1.826363in}{3.327878in}}{\pgfqpoint{1.815313in}{3.327878in}}%
\pgfpathcurveto{\pgfqpoint{1.804263in}{3.327878in}}{\pgfqpoint{1.793664in}{3.323488in}}{\pgfqpoint{1.785850in}{3.315674in}}%
\pgfpathcurveto{\pgfqpoint{1.778037in}{3.307860in}}{\pgfqpoint{1.773646in}{3.297261in}}{\pgfqpoint{1.773646in}{3.286211in}}%
\pgfpathcurveto{\pgfqpoint{1.773646in}{3.275161in}}{\pgfqpoint{1.778037in}{3.264562in}}{\pgfqpoint{1.785850in}{3.256748in}}%
\pgfpathcurveto{\pgfqpoint{1.793664in}{3.248935in}}{\pgfqpoint{1.804263in}{3.244545in}}{\pgfqpoint{1.815313in}{3.244545in}}%
\pgfpathclose%
\pgfusepath{stroke,fill}%
\end{pgfscope}%
\begin{pgfscope}%
\pgfpathrectangle{\pgfqpoint{0.648703in}{0.548769in}}{\pgfqpoint{5.201297in}{3.102590in}}%
\pgfusepath{clip}%
\pgfsetbuttcap%
\pgfsetroundjoin%
\definecolor{currentfill}{rgb}{1.000000,0.498039,0.054902}%
\pgfsetfillcolor{currentfill}%
\pgfsetlinewidth{1.003750pt}%
\definecolor{currentstroke}{rgb}{1.000000,0.498039,0.054902}%
\pgfsetstrokecolor{currentstroke}%
\pgfsetdash{}{0pt}%
\pgfpathmoveto{\pgfqpoint{2.435438in}{3.202258in}}%
\pgfpathcurveto{\pgfqpoint{2.446488in}{3.202258in}}{\pgfqpoint{2.457087in}{3.206648in}}{\pgfqpoint{2.464901in}{3.214462in}}%
\pgfpathcurveto{\pgfqpoint{2.472714in}{3.222275in}}{\pgfqpoint{2.477105in}{3.232874in}}{\pgfqpoint{2.477105in}{3.243924in}}%
\pgfpathcurveto{\pgfqpoint{2.477105in}{3.254974in}}{\pgfqpoint{2.472714in}{3.265573in}}{\pgfqpoint{2.464901in}{3.273387in}}%
\pgfpathcurveto{\pgfqpoint{2.457087in}{3.281201in}}{\pgfqpoint{2.446488in}{3.285591in}}{\pgfqpoint{2.435438in}{3.285591in}}%
\pgfpathcurveto{\pgfqpoint{2.424388in}{3.285591in}}{\pgfqpoint{2.413789in}{3.281201in}}{\pgfqpoint{2.405975in}{3.273387in}}%
\pgfpathcurveto{\pgfqpoint{2.398162in}{3.265573in}}{\pgfqpoint{2.393771in}{3.254974in}}{\pgfqpoint{2.393771in}{3.243924in}}%
\pgfpathcurveto{\pgfqpoint{2.393771in}{3.232874in}}{\pgfqpoint{2.398162in}{3.222275in}}{\pgfqpoint{2.405975in}{3.214462in}}%
\pgfpathcurveto{\pgfqpoint{2.413789in}{3.206648in}}{\pgfqpoint{2.424388in}{3.202258in}}{\pgfqpoint{2.435438in}{3.202258in}}%
\pgfpathclose%
\pgfusepath{stroke,fill}%
\end{pgfscope}%
\begin{pgfscope}%
\pgfpathrectangle{\pgfqpoint{0.648703in}{0.548769in}}{\pgfqpoint{5.201297in}{3.102590in}}%
\pgfusepath{clip}%
\pgfsetbuttcap%
\pgfsetroundjoin%
\definecolor{currentfill}{rgb}{1.000000,0.498039,0.054902}%
\pgfsetfillcolor{currentfill}%
\pgfsetlinewidth{1.003750pt}%
\definecolor{currentstroke}{rgb}{1.000000,0.498039,0.054902}%
\pgfsetstrokecolor{currentstroke}%
\pgfsetdash{}{0pt}%
\pgfpathmoveto{\pgfqpoint{1.350220in}{3.185343in}}%
\pgfpathcurveto{\pgfqpoint{1.361270in}{3.185343in}}{\pgfqpoint{1.371869in}{3.189733in}}{\pgfqpoint{1.379682in}{3.197547in}}%
\pgfpathcurveto{\pgfqpoint{1.387496in}{3.205360in}}{\pgfqpoint{1.391886in}{3.215959in}}{\pgfqpoint{1.391886in}{3.227010in}}%
\pgfpathcurveto{\pgfqpoint{1.391886in}{3.238060in}}{\pgfqpoint{1.387496in}{3.248659in}}{\pgfqpoint{1.379682in}{3.256472in}}%
\pgfpathcurveto{\pgfqpoint{1.371869in}{3.264286in}}{\pgfqpoint{1.361270in}{3.268676in}}{\pgfqpoint{1.350220in}{3.268676in}}%
\pgfpathcurveto{\pgfqpoint{1.339169in}{3.268676in}}{\pgfqpoint{1.328570in}{3.264286in}}{\pgfqpoint{1.320757in}{3.256472in}}%
\pgfpathcurveto{\pgfqpoint{1.312943in}{3.248659in}}{\pgfqpoint{1.308553in}{3.238060in}}{\pgfqpoint{1.308553in}{3.227010in}}%
\pgfpathcurveto{\pgfqpoint{1.308553in}{3.215959in}}{\pgfqpoint{1.312943in}{3.205360in}}{\pgfqpoint{1.320757in}{3.197547in}}%
\pgfpathcurveto{\pgfqpoint{1.328570in}{3.189733in}}{\pgfqpoint{1.339169in}{3.185343in}}{\pgfqpoint{1.350220in}{3.185343in}}%
\pgfpathclose%
\pgfusepath{stroke,fill}%
\end{pgfscope}%
\begin{pgfscope}%
\pgfpathrectangle{\pgfqpoint{0.648703in}{0.548769in}}{\pgfqpoint{5.201297in}{3.102590in}}%
\pgfusepath{clip}%
\pgfsetbuttcap%
\pgfsetroundjoin%
\definecolor{currentfill}{rgb}{1.000000,0.498039,0.054902}%
\pgfsetfillcolor{currentfill}%
\pgfsetlinewidth{1.003750pt}%
\definecolor{currentstroke}{rgb}{1.000000,0.498039,0.054902}%
\pgfsetstrokecolor{currentstroke}%
\pgfsetdash{}{0pt}%
\pgfpathmoveto{\pgfqpoint{1.737798in}{3.312204in}}%
\pgfpathcurveto{\pgfqpoint{1.748848in}{3.312204in}}{\pgfqpoint{1.759447in}{3.316594in}}{\pgfqpoint{1.767260in}{3.324407in}}%
\pgfpathcurveto{\pgfqpoint{1.775074in}{3.332221in}}{\pgfqpoint{1.779464in}{3.342820in}}{\pgfqpoint{1.779464in}{3.353870in}}%
\pgfpathcurveto{\pgfqpoint{1.779464in}{3.364920in}}{\pgfqpoint{1.775074in}{3.375519in}}{\pgfqpoint{1.767260in}{3.383333in}}%
\pgfpathcurveto{\pgfqpoint{1.759447in}{3.391147in}}{\pgfqpoint{1.748848in}{3.395537in}}{\pgfqpoint{1.737798in}{3.395537in}}%
\pgfpathcurveto{\pgfqpoint{1.726747in}{3.395537in}}{\pgfqpoint{1.716148in}{3.391147in}}{\pgfqpoint{1.708335in}{3.383333in}}%
\pgfpathcurveto{\pgfqpoint{1.700521in}{3.375519in}}{\pgfqpoint{1.696131in}{3.364920in}}{\pgfqpoint{1.696131in}{3.353870in}}%
\pgfpathcurveto{\pgfqpoint{1.696131in}{3.342820in}}{\pgfqpoint{1.700521in}{3.332221in}}{\pgfqpoint{1.708335in}{3.324407in}}%
\pgfpathcurveto{\pgfqpoint{1.716148in}{3.316594in}}{\pgfqpoint{1.726747in}{3.312204in}}{\pgfqpoint{1.737798in}{3.312204in}}%
\pgfpathclose%
\pgfusepath{stroke,fill}%
\end{pgfscope}%
\begin{pgfscope}%
\pgfpathrectangle{\pgfqpoint{0.648703in}{0.548769in}}{\pgfqpoint{5.201297in}{3.102590in}}%
\pgfusepath{clip}%
\pgfsetbuttcap%
\pgfsetroundjoin%
\definecolor{currentfill}{rgb}{1.000000,0.498039,0.054902}%
\pgfsetfillcolor{currentfill}%
\pgfsetlinewidth{1.003750pt}%
\definecolor{currentstroke}{rgb}{1.000000,0.498039,0.054902}%
\pgfsetstrokecolor{currentstroke}%
\pgfsetdash{}{0pt}%
\pgfpathmoveto{\pgfqpoint{1.660282in}{3.210715in}}%
\pgfpathcurveto{\pgfqpoint{1.671332in}{3.210715in}}{\pgfqpoint{1.681931in}{3.215105in}}{\pgfqpoint{1.689745in}{3.222919in}}%
\pgfpathcurveto{\pgfqpoint{1.697558in}{3.230733in}}{\pgfqpoint{1.701949in}{3.241332in}}{\pgfqpoint{1.701949in}{3.252382in}}%
\pgfpathcurveto{\pgfqpoint{1.701949in}{3.263432in}}{\pgfqpoint{1.697558in}{3.274031in}}{\pgfqpoint{1.689745in}{3.281844in}}%
\pgfpathcurveto{\pgfqpoint{1.681931in}{3.289658in}}{\pgfqpoint{1.671332in}{3.294048in}}{\pgfqpoint{1.660282in}{3.294048in}}%
\pgfpathcurveto{\pgfqpoint{1.649232in}{3.294048in}}{\pgfqpoint{1.638633in}{3.289658in}}{\pgfqpoint{1.630819in}{3.281844in}}%
\pgfpathcurveto{\pgfqpoint{1.623006in}{3.274031in}}{\pgfqpoint{1.618615in}{3.263432in}}{\pgfqpoint{1.618615in}{3.252382in}}%
\pgfpathcurveto{\pgfqpoint{1.618615in}{3.241332in}}{\pgfqpoint{1.623006in}{3.230733in}}{\pgfqpoint{1.630819in}{3.222919in}}%
\pgfpathcurveto{\pgfqpoint{1.638633in}{3.215105in}}{\pgfqpoint{1.649232in}{3.210715in}}{\pgfqpoint{1.660282in}{3.210715in}}%
\pgfpathclose%
\pgfusepath{stroke,fill}%
\end{pgfscope}%
\begin{pgfscope}%
\pgfpathrectangle{\pgfqpoint{0.648703in}{0.548769in}}{\pgfqpoint{5.201297in}{3.102590in}}%
\pgfusepath{clip}%
\pgfsetbuttcap%
\pgfsetroundjoin%
\definecolor{currentfill}{rgb}{1.000000,0.498039,0.054902}%
\pgfsetfillcolor{currentfill}%
\pgfsetlinewidth{1.003750pt}%
\definecolor{currentstroke}{rgb}{1.000000,0.498039,0.054902}%
\pgfsetstrokecolor{currentstroke}%
\pgfsetdash{}{0pt}%
\pgfpathmoveto{\pgfqpoint{1.737798in}{3.236087in}}%
\pgfpathcurveto{\pgfqpoint{1.748848in}{3.236087in}}{\pgfqpoint{1.759447in}{3.240477in}}{\pgfqpoint{1.767260in}{3.248291in}}%
\pgfpathcurveto{\pgfqpoint{1.775074in}{3.256105in}}{\pgfqpoint{1.779464in}{3.266704in}}{\pgfqpoint{1.779464in}{3.277754in}}%
\pgfpathcurveto{\pgfqpoint{1.779464in}{3.288804in}}{\pgfqpoint{1.775074in}{3.299403in}}{\pgfqpoint{1.767260in}{3.307217in}}%
\pgfpathcurveto{\pgfqpoint{1.759447in}{3.315030in}}{\pgfqpoint{1.748848in}{3.319421in}}{\pgfqpoint{1.737798in}{3.319421in}}%
\pgfpathcurveto{\pgfqpoint{1.726747in}{3.319421in}}{\pgfqpoint{1.716148in}{3.315030in}}{\pgfqpoint{1.708335in}{3.307217in}}%
\pgfpathcurveto{\pgfqpoint{1.700521in}{3.299403in}}{\pgfqpoint{1.696131in}{3.288804in}}{\pgfqpoint{1.696131in}{3.277754in}}%
\pgfpathcurveto{\pgfqpoint{1.696131in}{3.266704in}}{\pgfqpoint{1.700521in}{3.256105in}}{\pgfqpoint{1.708335in}{3.248291in}}%
\pgfpathcurveto{\pgfqpoint{1.716148in}{3.240477in}}{\pgfqpoint{1.726747in}{3.236087in}}{\pgfqpoint{1.737798in}{3.236087in}}%
\pgfpathclose%
\pgfusepath{stroke,fill}%
\end{pgfscope}%
\begin{pgfscope}%
\pgfpathrectangle{\pgfqpoint{0.648703in}{0.548769in}}{\pgfqpoint{5.201297in}{3.102590in}}%
\pgfusepath{clip}%
\pgfsetbuttcap%
\pgfsetroundjoin%
\definecolor{currentfill}{rgb}{0.121569,0.466667,0.705882}%
\pgfsetfillcolor{currentfill}%
\pgfsetlinewidth{1.003750pt}%
\definecolor{currentstroke}{rgb}{0.121569,0.466667,0.705882}%
\pgfsetstrokecolor{currentstroke}%
\pgfsetdash{}{0pt}%
\pgfpathmoveto{\pgfqpoint{3.365625in}{2.808990in}}%
\pgfpathcurveto{\pgfqpoint{3.376675in}{2.808990in}}{\pgfqpoint{3.387274in}{2.813380in}}{\pgfqpoint{3.395088in}{2.821193in}}%
\pgfpathcurveto{\pgfqpoint{3.402902in}{2.829007in}}{\pgfqpoint{3.407292in}{2.839606in}}{\pgfqpoint{3.407292in}{2.850656in}}%
\pgfpathcurveto{\pgfqpoint{3.407292in}{2.861706in}}{\pgfqpoint{3.402902in}{2.872305in}}{\pgfqpoint{3.395088in}{2.880119in}}%
\pgfpathcurveto{\pgfqpoint{3.387274in}{2.887933in}}{\pgfqpoint{3.376675in}{2.892323in}}{\pgfqpoint{3.365625in}{2.892323in}}%
\pgfpathcurveto{\pgfqpoint{3.354575in}{2.892323in}}{\pgfqpoint{3.343976in}{2.887933in}}{\pgfqpoint{3.336162in}{2.880119in}}%
\pgfpathcurveto{\pgfqpoint{3.328349in}{2.872305in}}{\pgfqpoint{3.323958in}{2.861706in}}{\pgfqpoint{3.323958in}{2.850656in}}%
\pgfpathcurveto{\pgfqpoint{3.323958in}{2.839606in}}{\pgfqpoint{3.328349in}{2.829007in}}{\pgfqpoint{3.336162in}{2.821193in}}%
\pgfpathcurveto{\pgfqpoint{3.343976in}{2.813380in}}{\pgfqpoint{3.354575in}{2.808990in}}{\pgfqpoint{3.365625in}{2.808990in}}%
\pgfpathclose%
\pgfusepath{stroke,fill}%
\end{pgfscope}%
\begin{pgfscope}%
\pgfpathrectangle{\pgfqpoint{0.648703in}{0.548769in}}{\pgfqpoint{5.201297in}{3.102590in}}%
\pgfusepath{clip}%
\pgfsetbuttcap%
\pgfsetroundjoin%
\definecolor{currentfill}{rgb}{0.121569,0.466667,0.705882}%
\pgfsetfillcolor{currentfill}%
\pgfsetlinewidth{1.003750pt}%
\definecolor{currentstroke}{rgb}{0.121569,0.466667,0.705882}%
\pgfsetstrokecolor{currentstroke}%
\pgfsetdash{}{0pt}%
\pgfpathmoveto{\pgfqpoint{1.272704in}{0.648129in}}%
\pgfpathcurveto{\pgfqpoint{1.283754in}{0.648129in}}{\pgfqpoint{1.294353in}{0.652519in}}{\pgfqpoint{1.302167in}{0.660333in}}%
\pgfpathcurveto{\pgfqpoint{1.309980in}{0.668146in}}{\pgfqpoint{1.314371in}{0.678745in}}{\pgfqpoint{1.314371in}{0.689796in}}%
\pgfpathcurveto{\pgfqpoint{1.314371in}{0.700846in}}{\pgfqpoint{1.309980in}{0.711445in}}{\pgfqpoint{1.302167in}{0.719258in}}%
\pgfpathcurveto{\pgfqpoint{1.294353in}{0.727072in}}{\pgfqpoint{1.283754in}{0.731462in}}{\pgfqpoint{1.272704in}{0.731462in}}%
\pgfpathcurveto{\pgfqpoint{1.261654in}{0.731462in}}{\pgfqpoint{1.251055in}{0.727072in}}{\pgfqpoint{1.243241in}{0.719258in}}%
\pgfpathcurveto{\pgfqpoint{1.235428in}{0.711445in}}{\pgfqpoint{1.231037in}{0.700846in}}{\pgfqpoint{1.231037in}{0.689796in}}%
\pgfpathcurveto{\pgfqpoint{1.231037in}{0.678745in}}{\pgfqpoint{1.235428in}{0.668146in}}{\pgfqpoint{1.243241in}{0.660333in}}%
\pgfpathcurveto{\pgfqpoint{1.251055in}{0.652519in}}{\pgfqpoint{1.261654in}{0.648129in}}{\pgfqpoint{1.272704in}{0.648129in}}%
\pgfpathclose%
\pgfusepath{stroke,fill}%
\end{pgfscope}%
\begin{pgfscope}%
\pgfpathrectangle{\pgfqpoint{0.648703in}{0.548769in}}{\pgfqpoint{5.201297in}{3.102590in}}%
\pgfusepath{clip}%
\pgfsetbuttcap%
\pgfsetroundjoin%
\definecolor{currentfill}{rgb}{0.121569,0.466667,0.705882}%
\pgfsetfillcolor{currentfill}%
\pgfsetlinewidth{1.003750pt}%
\definecolor{currentstroke}{rgb}{0.121569,0.466667,0.705882}%
\pgfsetstrokecolor{currentstroke}%
\pgfsetdash{}{0pt}%
\pgfpathmoveto{\pgfqpoint{1.427735in}{0.648129in}}%
\pgfpathcurveto{\pgfqpoint{1.438785in}{0.648129in}}{\pgfqpoint{1.449384in}{0.652519in}}{\pgfqpoint{1.457198in}{0.660333in}}%
\pgfpathcurveto{\pgfqpoint{1.465012in}{0.668146in}}{\pgfqpoint{1.469402in}{0.678745in}}{\pgfqpoint{1.469402in}{0.689796in}}%
\pgfpathcurveto{\pgfqpoint{1.469402in}{0.700846in}}{\pgfqpoint{1.465012in}{0.711445in}}{\pgfqpoint{1.457198in}{0.719258in}}%
\pgfpathcurveto{\pgfqpoint{1.449384in}{0.727072in}}{\pgfqpoint{1.438785in}{0.731462in}}{\pgfqpoint{1.427735in}{0.731462in}}%
\pgfpathcurveto{\pgfqpoint{1.416685in}{0.731462in}}{\pgfqpoint{1.406086in}{0.727072in}}{\pgfqpoint{1.398272in}{0.719258in}}%
\pgfpathcurveto{\pgfqpoint{1.390459in}{0.711445in}}{\pgfqpoint{1.386069in}{0.700846in}}{\pgfqpoint{1.386069in}{0.689796in}}%
\pgfpathcurveto{\pgfqpoint{1.386069in}{0.678745in}}{\pgfqpoint{1.390459in}{0.668146in}}{\pgfqpoint{1.398272in}{0.660333in}}%
\pgfpathcurveto{\pgfqpoint{1.406086in}{0.652519in}}{\pgfqpoint{1.416685in}{0.648129in}}{\pgfqpoint{1.427735in}{0.648129in}}%
\pgfpathclose%
\pgfusepath{stroke,fill}%
\end{pgfscope}%
\begin{pgfscope}%
\pgfpathrectangle{\pgfqpoint{0.648703in}{0.548769in}}{\pgfqpoint{5.201297in}{3.102590in}}%
\pgfusepath{clip}%
\pgfsetbuttcap%
\pgfsetroundjoin%
\definecolor{currentfill}{rgb}{1.000000,0.498039,0.054902}%
\pgfsetfillcolor{currentfill}%
\pgfsetlinewidth{1.003750pt}%
\definecolor{currentstroke}{rgb}{1.000000,0.498039,0.054902}%
\pgfsetstrokecolor{currentstroke}%
\pgfsetdash{}{0pt}%
\pgfpathmoveto{\pgfqpoint{1.505251in}{3.185343in}}%
\pgfpathcurveto{\pgfqpoint{1.516301in}{3.185343in}}{\pgfqpoint{1.526900in}{3.189733in}}{\pgfqpoint{1.534714in}{3.197547in}}%
\pgfpathcurveto{\pgfqpoint{1.542527in}{3.205360in}}{\pgfqpoint{1.546917in}{3.215959in}}{\pgfqpoint{1.546917in}{3.227010in}}%
\pgfpathcurveto{\pgfqpoint{1.546917in}{3.238060in}}{\pgfqpoint{1.542527in}{3.248659in}}{\pgfqpoint{1.534714in}{3.256472in}}%
\pgfpathcurveto{\pgfqpoint{1.526900in}{3.264286in}}{\pgfqpoint{1.516301in}{3.268676in}}{\pgfqpoint{1.505251in}{3.268676in}}%
\pgfpathcurveto{\pgfqpoint{1.494201in}{3.268676in}}{\pgfqpoint{1.483602in}{3.264286in}}{\pgfqpoint{1.475788in}{3.256472in}}%
\pgfpathcurveto{\pgfqpoint{1.467974in}{3.248659in}}{\pgfqpoint{1.463584in}{3.238060in}}{\pgfqpoint{1.463584in}{3.227010in}}%
\pgfpathcurveto{\pgfqpoint{1.463584in}{3.215959in}}{\pgfqpoint{1.467974in}{3.205360in}}{\pgfqpoint{1.475788in}{3.197547in}}%
\pgfpathcurveto{\pgfqpoint{1.483602in}{3.189733in}}{\pgfqpoint{1.494201in}{3.185343in}}{\pgfqpoint{1.505251in}{3.185343in}}%
\pgfpathclose%
\pgfusepath{stroke,fill}%
\end{pgfscope}%
\begin{pgfscope}%
\pgfpathrectangle{\pgfqpoint{0.648703in}{0.548769in}}{\pgfqpoint{5.201297in}{3.102590in}}%
\pgfusepath{clip}%
\pgfsetbuttcap%
\pgfsetroundjoin%
\definecolor{currentfill}{rgb}{1.000000,0.498039,0.054902}%
\pgfsetfillcolor{currentfill}%
\pgfsetlinewidth{1.003750pt}%
\definecolor{currentstroke}{rgb}{1.000000,0.498039,0.054902}%
\pgfsetstrokecolor{currentstroke}%
\pgfsetdash{}{0pt}%
\pgfpathmoveto{\pgfqpoint{1.195188in}{3.405235in}}%
\pgfpathcurveto{\pgfqpoint{1.206239in}{3.405235in}}{\pgfqpoint{1.216838in}{3.409625in}}{\pgfqpoint{1.224651in}{3.417439in}}%
\pgfpathcurveto{\pgfqpoint{1.232465in}{3.425252in}}{\pgfqpoint{1.236855in}{3.435851in}}{\pgfqpoint{1.236855in}{3.446901in}}%
\pgfpathcurveto{\pgfqpoint{1.236855in}{3.457952in}}{\pgfqpoint{1.232465in}{3.468551in}}{\pgfqpoint{1.224651in}{3.476364in}}%
\pgfpathcurveto{\pgfqpoint{1.216838in}{3.484178in}}{\pgfqpoint{1.206239in}{3.488568in}}{\pgfqpoint{1.195188in}{3.488568in}}%
\pgfpathcurveto{\pgfqpoint{1.184138in}{3.488568in}}{\pgfqpoint{1.173539in}{3.484178in}}{\pgfqpoint{1.165726in}{3.476364in}}%
\pgfpathcurveto{\pgfqpoint{1.157912in}{3.468551in}}{\pgfqpoint{1.153522in}{3.457952in}}{\pgfqpoint{1.153522in}{3.446901in}}%
\pgfpathcurveto{\pgfqpoint{1.153522in}{3.435851in}}{\pgfqpoint{1.157912in}{3.425252in}}{\pgfqpoint{1.165726in}{3.417439in}}%
\pgfpathcurveto{\pgfqpoint{1.173539in}{3.409625in}}{\pgfqpoint{1.184138in}{3.405235in}}{\pgfqpoint{1.195188in}{3.405235in}}%
\pgfpathclose%
\pgfusepath{stroke,fill}%
\end{pgfscope}%
\begin{pgfscope}%
\pgfpathrectangle{\pgfqpoint{0.648703in}{0.548769in}}{\pgfqpoint{5.201297in}{3.102590in}}%
\pgfusepath{clip}%
\pgfsetbuttcap%
\pgfsetroundjoin%
\definecolor{currentfill}{rgb}{1.000000,0.498039,0.054902}%
\pgfsetfillcolor{currentfill}%
\pgfsetlinewidth{1.003750pt}%
\definecolor{currentstroke}{rgb}{1.000000,0.498039,0.054902}%
\pgfsetstrokecolor{currentstroke}%
\pgfsetdash{}{0pt}%
\pgfpathmoveto{\pgfqpoint{2.047860in}{3.193800in}}%
\pgfpathcurveto{\pgfqpoint{2.058910in}{3.193800in}}{\pgfqpoint{2.069509in}{3.198191in}}{\pgfqpoint{2.077323in}{3.206004in}}%
\pgfpathcurveto{\pgfqpoint{2.085136in}{3.213818in}}{\pgfqpoint{2.089527in}{3.224417in}}{\pgfqpoint{2.089527in}{3.235467in}}%
\pgfpathcurveto{\pgfqpoint{2.089527in}{3.246517in}}{\pgfqpoint{2.085136in}{3.257116in}}{\pgfqpoint{2.077323in}{3.264930in}}%
\pgfpathcurveto{\pgfqpoint{2.069509in}{3.272743in}}{\pgfqpoint{2.058910in}{3.277134in}}{\pgfqpoint{2.047860in}{3.277134in}}%
\pgfpathcurveto{\pgfqpoint{2.036810in}{3.277134in}}{\pgfqpoint{2.026211in}{3.272743in}}{\pgfqpoint{2.018397in}{3.264930in}}%
\pgfpathcurveto{\pgfqpoint{2.010584in}{3.257116in}}{\pgfqpoint{2.006193in}{3.246517in}}{\pgfqpoint{2.006193in}{3.235467in}}%
\pgfpathcurveto{\pgfqpoint{2.006193in}{3.224417in}}{\pgfqpoint{2.010584in}{3.213818in}}{\pgfqpoint{2.018397in}{3.206004in}}%
\pgfpathcurveto{\pgfqpoint{2.026211in}{3.198191in}}{\pgfqpoint{2.036810in}{3.193800in}}{\pgfqpoint{2.047860in}{3.193800in}}%
\pgfpathclose%
\pgfusepath{stroke,fill}%
\end{pgfscope}%
\begin{pgfscope}%
\pgfpathrectangle{\pgfqpoint{0.648703in}{0.548769in}}{\pgfqpoint{5.201297in}{3.102590in}}%
\pgfusepath{clip}%
\pgfsetbuttcap%
\pgfsetroundjoin%
\definecolor{currentfill}{rgb}{1.000000,0.498039,0.054902}%
\pgfsetfillcolor{currentfill}%
\pgfsetlinewidth{1.003750pt}%
\definecolor{currentstroke}{rgb}{1.000000,0.498039,0.054902}%
\pgfsetstrokecolor{currentstroke}%
\pgfsetdash{}{0pt}%
\pgfpathmoveto{\pgfqpoint{2.745500in}{3.193800in}}%
\pgfpathcurveto{\pgfqpoint{2.756550in}{3.193800in}}{\pgfqpoint{2.767149in}{3.198191in}}{\pgfqpoint{2.774963in}{3.206004in}}%
\pgfpathcurveto{\pgfqpoint{2.782777in}{3.213818in}}{\pgfqpoint{2.787167in}{3.224417in}}{\pgfqpoint{2.787167in}{3.235467in}}%
\pgfpathcurveto{\pgfqpoint{2.787167in}{3.246517in}}{\pgfqpoint{2.782777in}{3.257116in}}{\pgfqpoint{2.774963in}{3.264930in}}%
\pgfpathcurveto{\pgfqpoint{2.767149in}{3.272743in}}{\pgfqpoint{2.756550in}{3.277134in}}{\pgfqpoint{2.745500in}{3.277134in}}%
\pgfpathcurveto{\pgfqpoint{2.734450in}{3.277134in}}{\pgfqpoint{2.723851in}{3.272743in}}{\pgfqpoint{2.716038in}{3.264930in}}%
\pgfpathcurveto{\pgfqpoint{2.708224in}{3.257116in}}{\pgfqpoint{2.703834in}{3.246517in}}{\pgfqpoint{2.703834in}{3.235467in}}%
\pgfpathcurveto{\pgfqpoint{2.703834in}{3.224417in}}{\pgfqpoint{2.708224in}{3.213818in}}{\pgfqpoint{2.716038in}{3.206004in}}%
\pgfpathcurveto{\pgfqpoint{2.723851in}{3.198191in}}{\pgfqpoint{2.734450in}{3.193800in}}{\pgfqpoint{2.745500in}{3.193800in}}%
\pgfpathclose%
\pgfusepath{stroke,fill}%
\end{pgfscope}%
\begin{pgfscope}%
\pgfpathrectangle{\pgfqpoint{0.648703in}{0.548769in}}{\pgfqpoint{5.201297in}{3.102590in}}%
\pgfusepath{clip}%
\pgfsetbuttcap%
\pgfsetroundjoin%
\definecolor{currentfill}{rgb}{1.000000,0.498039,0.054902}%
\pgfsetfillcolor{currentfill}%
\pgfsetlinewidth{1.003750pt}%
\definecolor{currentstroke}{rgb}{1.000000,0.498039,0.054902}%
\pgfsetstrokecolor{currentstroke}%
\pgfsetdash{}{0pt}%
\pgfpathmoveto{\pgfqpoint{1.350220in}{3.358719in}}%
\pgfpathcurveto{\pgfqpoint{1.361270in}{3.358719in}}{\pgfqpoint{1.371869in}{3.363109in}}{\pgfqpoint{1.379682in}{3.370923in}}%
\pgfpathcurveto{\pgfqpoint{1.387496in}{3.378737in}}{\pgfqpoint{1.391886in}{3.389336in}}{\pgfqpoint{1.391886in}{3.400386in}}%
\pgfpathcurveto{\pgfqpoint{1.391886in}{3.411436in}}{\pgfqpoint{1.387496in}{3.422035in}}{\pgfqpoint{1.379682in}{3.429849in}}%
\pgfpathcurveto{\pgfqpoint{1.371869in}{3.437662in}}{\pgfqpoint{1.361270in}{3.442053in}}{\pgfqpoint{1.350220in}{3.442053in}}%
\pgfpathcurveto{\pgfqpoint{1.339169in}{3.442053in}}{\pgfqpoint{1.328570in}{3.437662in}}{\pgfqpoint{1.320757in}{3.429849in}}%
\pgfpathcurveto{\pgfqpoint{1.312943in}{3.422035in}}{\pgfqpoint{1.308553in}{3.411436in}}{\pgfqpoint{1.308553in}{3.400386in}}%
\pgfpathcurveto{\pgfqpoint{1.308553in}{3.389336in}}{\pgfqpoint{1.312943in}{3.378737in}}{\pgfqpoint{1.320757in}{3.370923in}}%
\pgfpathcurveto{\pgfqpoint{1.328570in}{3.363109in}}{\pgfqpoint{1.339169in}{3.358719in}}{\pgfqpoint{1.350220in}{3.358719in}}%
\pgfpathclose%
\pgfusepath{stroke,fill}%
\end{pgfscope}%
\begin{pgfscope}%
\pgfpathrectangle{\pgfqpoint{0.648703in}{0.548769in}}{\pgfqpoint{5.201297in}{3.102590in}}%
\pgfusepath{clip}%
\pgfsetbuttcap%
\pgfsetroundjoin%
\definecolor{currentfill}{rgb}{1.000000,0.498039,0.054902}%
\pgfsetfillcolor{currentfill}%
\pgfsetlinewidth{1.003750pt}%
\definecolor{currentstroke}{rgb}{1.000000,0.498039,0.054902}%
\pgfsetstrokecolor{currentstroke}%
\pgfsetdash{}{0pt}%
\pgfpathmoveto{\pgfqpoint{1.040157in}{3.362948in}}%
\pgfpathcurveto{\pgfqpoint{1.051207in}{3.362948in}}{\pgfqpoint{1.061806in}{3.367338in}}{\pgfqpoint{1.069620in}{3.375152in}}%
\pgfpathcurveto{\pgfqpoint{1.077434in}{3.382965in}}{\pgfqpoint{1.081824in}{3.393564in}}{\pgfqpoint{1.081824in}{3.404615in}}%
\pgfpathcurveto{\pgfqpoint{1.081824in}{3.415665in}}{\pgfqpoint{1.077434in}{3.426264in}}{\pgfqpoint{1.069620in}{3.434077in}}%
\pgfpathcurveto{\pgfqpoint{1.061806in}{3.441891in}}{\pgfqpoint{1.051207in}{3.446281in}}{\pgfqpoint{1.040157in}{3.446281in}}%
\pgfpathcurveto{\pgfqpoint{1.029107in}{3.446281in}}{\pgfqpoint{1.018508in}{3.441891in}}{\pgfqpoint{1.010694in}{3.434077in}}%
\pgfpathcurveto{\pgfqpoint{1.002881in}{3.426264in}}{\pgfqpoint{0.998491in}{3.415665in}}{\pgfqpoint{0.998491in}{3.404615in}}%
\pgfpathcurveto{\pgfqpoint{0.998491in}{3.393564in}}{\pgfqpoint{1.002881in}{3.382965in}}{\pgfqpoint{1.010694in}{3.375152in}}%
\pgfpathcurveto{\pgfqpoint{1.018508in}{3.367338in}}{\pgfqpoint{1.029107in}{3.362948in}}{\pgfqpoint{1.040157in}{3.362948in}}%
\pgfpathclose%
\pgfusepath{stroke,fill}%
\end{pgfscope}%
\begin{pgfscope}%
\pgfpathrectangle{\pgfqpoint{0.648703in}{0.548769in}}{\pgfqpoint{5.201297in}{3.102590in}}%
\pgfusepath{clip}%
\pgfsetbuttcap%
\pgfsetroundjoin%
\definecolor{currentfill}{rgb}{1.000000,0.498039,0.054902}%
\pgfsetfillcolor{currentfill}%
\pgfsetlinewidth{1.003750pt}%
\definecolor{currentstroke}{rgb}{1.000000,0.498039,0.054902}%
\pgfsetstrokecolor{currentstroke}%
\pgfsetdash{}{0pt}%
\pgfpathmoveto{\pgfqpoint{2.978047in}{3.257231in}}%
\pgfpathcurveto{\pgfqpoint{2.989097in}{3.257231in}}{\pgfqpoint{2.999696in}{3.261621in}}{\pgfqpoint{3.007510in}{3.269435in}}%
\pgfpathcurveto{\pgfqpoint{3.015324in}{3.277248in}}{\pgfqpoint{3.019714in}{3.287847in}}{\pgfqpoint{3.019714in}{3.298897in}}%
\pgfpathcurveto{\pgfqpoint{3.019714in}{3.309947in}}{\pgfqpoint{3.015324in}{3.320546in}}{\pgfqpoint{3.007510in}{3.328360in}}%
\pgfpathcurveto{\pgfqpoint{2.999696in}{3.336174in}}{\pgfqpoint{2.989097in}{3.340564in}}{\pgfqpoint{2.978047in}{3.340564in}}%
\pgfpathcurveto{\pgfqpoint{2.966997in}{3.340564in}}{\pgfqpoint{2.956398in}{3.336174in}}{\pgfqpoint{2.948584in}{3.328360in}}%
\pgfpathcurveto{\pgfqpoint{2.940771in}{3.320546in}}{\pgfqpoint{2.936380in}{3.309947in}}{\pgfqpoint{2.936380in}{3.298897in}}%
\pgfpathcurveto{\pgfqpoint{2.936380in}{3.287847in}}{\pgfqpoint{2.940771in}{3.277248in}}{\pgfqpoint{2.948584in}{3.269435in}}%
\pgfpathcurveto{\pgfqpoint{2.956398in}{3.261621in}}{\pgfqpoint{2.966997in}{3.257231in}}{\pgfqpoint{2.978047in}{3.257231in}}%
\pgfpathclose%
\pgfusepath{stroke,fill}%
\end{pgfscope}%
\begin{pgfscope}%
\pgfpathrectangle{\pgfqpoint{0.648703in}{0.548769in}}{\pgfqpoint{5.201297in}{3.102590in}}%
\pgfusepath{clip}%
\pgfsetbuttcap%
\pgfsetroundjoin%
\definecolor{currentfill}{rgb}{0.121569,0.466667,0.705882}%
\pgfsetfillcolor{currentfill}%
\pgfsetlinewidth{1.003750pt}%
\definecolor{currentstroke}{rgb}{0.121569,0.466667,0.705882}%
\pgfsetstrokecolor{currentstroke}%
\pgfsetdash{}{0pt}%
\pgfpathmoveto{\pgfqpoint{0.962642in}{0.648129in}}%
\pgfpathcurveto{\pgfqpoint{0.973692in}{0.648129in}}{\pgfqpoint{0.984291in}{0.652519in}}{\pgfqpoint{0.992104in}{0.660333in}}%
\pgfpathcurveto{\pgfqpoint{0.999918in}{0.668146in}}{\pgfqpoint{1.004308in}{0.678745in}}{\pgfqpoint{1.004308in}{0.689796in}}%
\pgfpathcurveto{\pgfqpoint{1.004308in}{0.700846in}}{\pgfqpoint{0.999918in}{0.711445in}}{\pgfqpoint{0.992104in}{0.719258in}}%
\pgfpathcurveto{\pgfqpoint{0.984291in}{0.727072in}}{\pgfqpoint{0.973692in}{0.731462in}}{\pgfqpoint{0.962642in}{0.731462in}}%
\pgfpathcurveto{\pgfqpoint{0.951591in}{0.731462in}}{\pgfqpoint{0.940992in}{0.727072in}}{\pgfqpoint{0.933179in}{0.719258in}}%
\pgfpathcurveto{\pgfqpoint{0.925365in}{0.711445in}}{\pgfqpoint{0.920975in}{0.700846in}}{\pgfqpoint{0.920975in}{0.689796in}}%
\pgfpathcurveto{\pgfqpoint{0.920975in}{0.678745in}}{\pgfqpoint{0.925365in}{0.668146in}}{\pgfqpoint{0.933179in}{0.660333in}}%
\pgfpathcurveto{\pgfqpoint{0.940992in}{0.652519in}}{\pgfqpoint{0.951591in}{0.648129in}}{\pgfqpoint{0.962642in}{0.648129in}}%
\pgfpathclose%
\pgfusepath{stroke,fill}%
\end{pgfscope}%
\begin{pgfscope}%
\pgfpathrectangle{\pgfqpoint{0.648703in}{0.548769in}}{\pgfqpoint{5.201297in}{3.102590in}}%
\pgfusepath{clip}%
\pgfsetbuttcap%
\pgfsetroundjoin%
\definecolor{currentfill}{rgb}{0.121569,0.466667,0.705882}%
\pgfsetfillcolor{currentfill}%
\pgfsetlinewidth{1.003750pt}%
\definecolor{currentstroke}{rgb}{0.121569,0.466667,0.705882}%
\pgfsetstrokecolor{currentstroke}%
\pgfsetdash{}{0pt}%
\pgfpathmoveto{\pgfqpoint{1.350220in}{0.796133in}}%
\pgfpathcurveto{\pgfqpoint{1.361270in}{0.796133in}}{\pgfqpoint{1.371869in}{0.800523in}}{\pgfqpoint{1.379682in}{0.808337in}}%
\pgfpathcurveto{\pgfqpoint{1.387496in}{0.816151in}}{\pgfqpoint{1.391886in}{0.826750in}}{\pgfqpoint{1.391886in}{0.837800in}}%
\pgfpathcurveto{\pgfqpoint{1.391886in}{0.848850in}}{\pgfqpoint{1.387496in}{0.859449in}}{\pgfqpoint{1.379682in}{0.867263in}}%
\pgfpathcurveto{\pgfqpoint{1.371869in}{0.875076in}}{\pgfqpoint{1.361270in}{0.879466in}}{\pgfqpoint{1.350220in}{0.879466in}}%
\pgfpathcurveto{\pgfqpoint{1.339169in}{0.879466in}}{\pgfqpoint{1.328570in}{0.875076in}}{\pgfqpoint{1.320757in}{0.867263in}}%
\pgfpathcurveto{\pgfqpoint{1.312943in}{0.859449in}}{\pgfqpoint{1.308553in}{0.848850in}}{\pgfqpoint{1.308553in}{0.837800in}}%
\pgfpathcurveto{\pgfqpoint{1.308553in}{0.826750in}}{\pgfqpoint{1.312943in}{0.816151in}}{\pgfqpoint{1.320757in}{0.808337in}}%
\pgfpathcurveto{\pgfqpoint{1.328570in}{0.800523in}}{\pgfqpoint{1.339169in}{0.796133in}}{\pgfqpoint{1.350220in}{0.796133in}}%
\pgfpathclose%
\pgfusepath{stroke,fill}%
\end{pgfscope}%
\begin{pgfscope}%
\pgfpathrectangle{\pgfqpoint{0.648703in}{0.548769in}}{\pgfqpoint{5.201297in}{3.102590in}}%
\pgfusepath{clip}%
\pgfsetbuttcap%
\pgfsetroundjoin%
\definecolor{currentfill}{rgb}{0.121569,0.466667,0.705882}%
\pgfsetfillcolor{currentfill}%
\pgfsetlinewidth{1.003750pt}%
\definecolor{currentstroke}{rgb}{0.121569,0.466667,0.705882}%
\pgfsetstrokecolor{currentstroke}%
\pgfsetdash{}{0pt}%
\pgfpathmoveto{\pgfqpoint{1.195188in}{3.155742in}}%
\pgfpathcurveto{\pgfqpoint{1.206239in}{3.155742in}}{\pgfqpoint{1.216838in}{3.160132in}}{\pgfqpoint{1.224651in}{3.167946in}}%
\pgfpathcurveto{\pgfqpoint{1.232465in}{3.175760in}}{\pgfqpoint{1.236855in}{3.186359in}}{\pgfqpoint{1.236855in}{3.197409in}}%
\pgfpathcurveto{\pgfqpoint{1.236855in}{3.208459in}}{\pgfqpoint{1.232465in}{3.219058in}}{\pgfqpoint{1.224651in}{3.226872in}}%
\pgfpathcurveto{\pgfqpoint{1.216838in}{3.234685in}}{\pgfqpoint{1.206239in}{3.239075in}}{\pgfqpoint{1.195188in}{3.239075in}}%
\pgfpathcurveto{\pgfqpoint{1.184138in}{3.239075in}}{\pgfqpoint{1.173539in}{3.234685in}}{\pgfqpoint{1.165726in}{3.226872in}}%
\pgfpathcurveto{\pgfqpoint{1.157912in}{3.219058in}}{\pgfqpoint{1.153522in}{3.208459in}}{\pgfqpoint{1.153522in}{3.197409in}}%
\pgfpathcurveto{\pgfqpoint{1.153522in}{3.186359in}}{\pgfqpoint{1.157912in}{3.175760in}}{\pgfqpoint{1.165726in}{3.167946in}}%
\pgfpathcurveto{\pgfqpoint{1.173539in}{3.160132in}}{\pgfqpoint{1.184138in}{3.155742in}}{\pgfqpoint{1.195188in}{3.155742in}}%
\pgfpathclose%
\pgfusepath{stroke,fill}%
\end{pgfscope}%
\begin{pgfscope}%
\pgfpathrectangle{\pgfqpoint{0.648703in}{0.548769in}}{\pgfqpoint{5.201297in}{3.102590in}}%
\pgfusepath{clip}%
\pgfsetbuttcap%
\pgfsetroundjoin%
\definecolor{currentfill}{rgb}{0.121569,0.466667,0.705882}%
\pgfsetfillcolor{currentfill}%
\pgfsetlinewidth{1.003750pt}%
\definecolor{currentstroke}{rgb}{0.121569,0.466667,0.705882}%
\pgfsetstrokecolor{currentstroke}%
\pgfsetdash{}{0pt}%
\pgfpathmoveto{\pgfqpoint{1.272704in}{0.648129in}}%
\pgfpathcurveto{\pgfqpoint{1.283754in}{0.648129in}}{\pgfqpoint{1.294353in}{0.652519in}}{\pgfqpoint{1.302167in}{0.660333in}}%
\pgfpathcurveto{\pgfqpoint{1.309980in}{0.668146in}}{\pgfqpoint{1.314371in}{0.678745in}}{\pgfqpoint{1.314371in}{0.689796in}}%
\pgfpathcurveto{\pgfqpoint{1.314371in}{0.700846in}}{\pgfqpoint{1.309980in}{0.711445in}}{\pgfqpoint{1.302167in}{0.719258in}}%
\pgfpathcurveto{\pgfqpoint{1.294353in}{0.727072in}}{\pgfqpoint{1.283754in}{0.731462in}}{\pgfqpoint{1.272704in}{0.731462in}}%
\pgfpathcurveto{\pgfqpoint{1.261654in}{0.731462in}}{\pgfqpoint{1.251055in}{0.727072in}}{\pgfqpoint{1.243241in}{0.719258in}}%
\pgfpathcurveto{\pgfqpoint{1.235428in}{0.711445in}}{\pgfqpoint{1.231037in}{0.700846in}}{\pgfqpoint{1.231037in}{0.689796in}}%
\pgfpathcurveto{\pgfqpoint{1.231037in}{0.678745in}}{\pgfqpoint{1.235428in}{0.668146in}}{\pgfqpoint{1.243241in}{0.660333in}}%
\pgfpathcurveto{\pgfqpoint{1.251055in}{0.652519in}}{\pgfqpoint{1.261654in}{0.648129in}}{\pgfqpoint{1.272704in}{0.648129in}}%
\pgfpathclose%
\pgfusepath{stroke,fill}%
\end{pgfscope}%
\begin{pgfscope}%
\pgfpathrectangle{\pgfqpoint{0.648703in}{0.548769in}}{\pgfqpoint{5.201297in}{3.102590in}}%
\pgfusepath{clip}%
\pgfsetbuttcap%
\pgfsetroundjoin%
\definecolor{currentfill}{rgb}{0.121569,0.466667,0.705882}%
\pgfsetfillcolor{currentfill}%
\pgfsetlinewidth{1.003750pt}%
\definecolor{currentstroke}{rgb}{0.121569,0.466667,0.705882}%
\pgfsetstrokecolor{currentstroke}%
\pgfsetdash{}{0pt}%
\pgfpathmoveto{\pgfqpoint{1.272704in}{0.648129in}}%
\pgfpathcurveto{\pgfqpoint{1.283754in}{0.648129in}}{\pgfqpoint{1.294353in}{0.652519in}}{\pgfqpoint{1.302167in}{0.660333in}}%
\pgfpathcurveto{\pgfqpoint{1.309980in}{0.668146in}}{\pgfqpoint{1.314371in}{0.678745in}}{\pgfqpoint{1.314371in}{0.689796in}}%
\pgfpathcurveto{\pgfqpoint{1.314371in}{0.700846in}}{\pgfqpoint{1.309980in}{0.711445in}}{\pgfqpoint{1.302167in}{0.719258in}}%
\pgfpathcurveto{\pgfqpoint{1.294353in}{0.727072in}}{\pgfqpoint{1.283754in}{0.731462in}}{\pgfqpoint{1.272704in}{0.731462in}}%
\pgfpathcurveto{\pgfqpoint{1.261654in}{0.731462in}}{\pgfqpoint{1.251055in}{0.727072in}}{\pgfqpoint{1.243241in}{0.719258in}}%
\pgfpathcurveto{\pgfqpoint{1.235428in}{0.711445in}}{\pgfqpoint{1.231037in}{0.700846in}}{\pgfqpoint{1.231037in}{0.689796in}}%
\pgfpathcurveto{\pgfqpoint{1.231037in}{0.678745in}}{\pgfqpoint{1.235428in}{0.668146in}}{\pgfqpoint{1.243241in}{0.660333in}}%
\pgfpathcurveto{\pgfqpoint{1.251055in}{0.652519in}}{\pgfqpoint{1.261654in}{0.648129in}}{\pgfqpoint{1.272704in}{0.648129in}}%
\pgfpathclose%
\pgfusepath{stroke,fill}%
\end{pgfscope}%
\begin{pgfscope}%
\pgfpathrectangle{\pgfqpoint{0.648703in}{0.548769in}}{\pgfqpoint{5.201297in}{3.102590in}}%
\pgfusepath{clip}%
\pgfsetbuttcap%
\pgfsetroundjoin%
\definecolor{currentfill}{rgb}{0.121569,0.466667,0.705882}%
\pgfsetfillcolor{currentfill}%
\pgfsetlinewidth{1.003750pt}%
\definecolor{currentstroke}{rgb}{0.121569,0.466667,0.705882}%
\pgfsetstrokecolor{currentstroke}%
\pgfsetdash{}{0pt}%
\pgfpathmoveto{\pgfqpoint{0.962642in}{2.512981in}}%
\pgfpathcurveto{\pgfqpoint{0.973692in}{2.512981in}}{\pgfqpoint{0.984291in}{2.517371in}}{\pgfqpoint{0.992104in}{2.525185in}}%
\pgfpathcurveto{\pgfqpoint{0.999918in}{2.532999in}}{\pgfqpoint{1.004308in}{2.543598in}}{\pgfqpoint{1.004308in}{2.554648in}}%
\pgfpathcurveto{\pgfqpoint{1.004308in}{2.565698in}}{\pgfqpoint{0.999918in}{2.576297in}}{\pgfqpoint{0.992104in}{2.584111in}}%
\pgfpathcurveto{\pgfqpoint{0.984291in}{2.591924in}}{\pgfqpoint{0.973692in}{2.596315in}}{\pgfqpoint{0.962642in}{2.596315in}}%
\pgfpathcurveto{\pgfqpoint{0.951591in}{2.596315in}}{\pgfqpoint{0.940992in}{2.591924in}}{\pgfqpoint{0.933179in}{2.584111in}}%
\pgfpathcurveto{\pgfqpoint{0.925365in}{2.576297in}}{\pgfqpoint{0.920975in}{2.565698in}}{\pgfqpoint{0.920975in}{2.554648in}}%
\pgfpathcurveto{\pgfqpoint{0.920975in}{2.543598in}}{\pgfqpoint{0.925365in}{2.532999in}}{\pgfqpoint{0.933179in}{2.525185in}}%
\pgfpathcurveto{\pgfqpoint{0.940992in}{2.517371in}}{\pgfqpoint{0.951591in}{2.512981in}}{\pgfqpoint{0.962642in}{2.512981in}}%
\pgfpathclose%
\pgfusepath{stroke,fill}%
\end{pgfscope}%
\begin{pgfscope}%
\pgfpathrectangle{\pgfqpoint{0.648703in}{0.548769in}}{\pgfqpoint{5.201297in}{3.102590in}}%
\pgfusepath{clip}%
\pgfsetbuttcap%
\pgfsetroundjoin%
\definecolor{currentfill}{rgb}{1.000000,0.498039,0.054902}%
\pgfsetfillcolor{currentfill}%
\pgfsetlinewidth{1.003750pt}%
\definecolor{currentstroke}{rgb}{1.000000,0.498039,0.054902}%
\pgfsetstrokecolor{currentstroke}%
\pgfsetdash{}{0pt}%
\pgfpathmoveto{\pgfqpoint{2.125376in}{3.189572in}}%
\pgfpathcurveto{\pgfqpoint{2.136426in}{3.189572in}}{\pgfqpoint{2.147025in}{3.193962in}}{\pgfqpoint{2.154838in}{3.201775in}}%
\pgfpathcurveto{\pgfqpoint{2.162652in}{3.209589in}}{\pgfqpoint{2.167042in}{3.220188in}}{\pgfqpoint{2.167042in}{3.231238in}}%
\pgfpathcurveto{\pgfqpoint{2.167042in}{3.242288in}}{\pgfqpoint{2.162652in}{3.252887in}}{\pgfqpoint{2.154838in}{3.260701in}}%
\pgfpathcurveto{\pgfqpoint{2.147025in}{3.268515in}}{\pgfqpoint{2.136426in}{3.272905in}}{\pgfqpoint{2.125376in}{3.272905in}}%
\pgfpathcurveto{\pgfqpoint{2.114325in}{3.272905in}}{\pgfqpoint{2.103726in}{3.268515in}}{\pgfqpoint{2.095913in}{3.260701in}}%
\pgfpathcurveto{\pgfqpoint{2.088099in}{3.252887in}}{\pgfqpoint{2.083709in}{3.242288in}}{\pgfqpoint{2.083709in}{3.231238in}}%
\pgfpathcurveto{\pgfqpoint{2.083709in}{3.220188in}}{\pgfqpoint{2.088099in}{3.209589in}}{\pgfqpoint{2.095913in}{3.201775in}}%
\pgfpathcurveto{\pgfqpoint{2.103726in}{3.193962in}}{\pgfqpoint{2.114325in}{3.189572in}}{\pgfqpoint{2.125376in}{3.189572in}}%
\pgfpathclose%
\pgfusepath{stroke,fill}%
\end{pgfscope}%
\begin{pgfscope}%
\pgfpathrectangle{\pgfqpoint{0.648703in}{0.548769in}}{\pgfqpoint{5.201297in}{3.102590in}}%
\pgfusepath{clip}%
\pgfsetbuttcap%
\pgfsetroundjoin%
\definecolor{currentfill}{rgb}{1.000000,0.498039,0.054902}%
\pgfsetfillcolor{currentfill}%
\pgfsetlinewidth{1.003750pt}%
\definecolor{currentstroke}{rgb}{1.000000,0.498039,0.054902}%
\pgfsetstrokecolor{currentstroke}%
\pgfsetdash{}{0pt}%
\pgfpathmoveto{\pgfqpoint{0.885126in}{3.206486in}}%
\pgfpathcurveto{\pgfqpoint{0.896176in}{3.206486in}}{\pgfqpoint{0.906775in}{3.210877in}}{\pgfqpoint{0.914589in}{3.218690in}}%
\pgfpathcurveto{\pgfqpoint{0.922402in}{3.226504in}}{\pgfqpoint{0.926793in}{3.237103in}}{\pgfqpoint{0.926793in}{3.248153in}}%
\pgfpathcurveto{\pgfqpoint{0.926793in}{3.259203in}}{\pgfqpoint{0.922402in}{3.269802in}}{\pgfqpoint{0.914589in}{3.277616in}}%
\pgfpathcurveto{\pgfqpoint{0.906775in}{3.285429in}}{\pgfqpoint{0.896176in}{3.289820in}}{\pgfqpoint{0.885126in}{3.289820in}}%
\pgfpathcurveto{\pgfqpoint{0.874076in}{3.289820in}}{\pgfqpoint{0.863477in}{3.285429in}}{\pgfqpoint{0.855663in}{3.277616in}}%
\pgfpathcurveto{\pgfqpoint{0.847850in}{3.269802in}}{\pgfqpoint{0.843459in}{3.259203in}}{\pgfqpoint{0.843459in}{3.248153in}}%
\pgfpathcurveto{\pgfqpoint{0.843459in}{3.237103in}}{\pgfqpoint{0.847850in}{3.226504in}}{\pgfqpoint{0.855663in}{3.218690in}}%
\pgfpathcurveto{\pgfqpoint{0.863477in}{3.210877in}}{\pgfqpoint{0.874076in}{3.206486in}}{\pgfqpoint{0.885126in}{3.206486in}}%
\pgfpathclose%
\pgfusepath{stroke,fill}%
\end{pgfscope}%
\begin{pgfscope}%
\pgfsetbuttcap%
\pgfsetroundjoin%
\definecolor{currentfill}{rgb}{0.000000,0.000000,0.000000}%
\pgfsetfillcolor{currentfill}%
\pgfsetlinewidth{0.803000pt}%
\definecolor{currentstroke}{rgb}{0.000000,0.000000,0.000000}%
\pgfsetstrokecolor{currentstroke}%
\pgfsetdash{}{0pt}%
\pgfsys@defobject{currentmarker}{\pgfqpoint{0.000000in}{-0.048611in}}{\pgfqpoint{0.000000in}{0.000000in}}{%
\pgfpathmoveto{\pgfqpoint{0.000000in}{0.000000in}}%
\pgfpathlineto{\pgfqpoint{0.000000in}{-0.048611in}}%
\pgfusepath{stroke,fill}%
}%
\begin{pgfscope}%
\pgfsys@transformshift{0.807610in}{0.548769in}%
\pgfsys@useobject{currentmarker}{}%
\end{pgfscope}%
\end{pgfscope}%
\begin{pgfscope}%
\definecolor{textcolor}{rgb}{0.000000,0.000000,0.000000}%
\pgfsetstrokecolor{textcolor}%
\pgfsetfillcolor{textcolor}%
\pgftext[x=0.807610in,y=0.451547in,,top]{\color{textcolor}\sffamily\fontsize{10.000000}{12.000000}\selectfont \(\displaystyle {0}\)}%
\end{pgfscope}%
\begin{pgfscope}%
\pgfsetbuttcap%
\pgfsetroundjoin%
\definecolor{currentfill}{rgb}{0.000000,0.000000,0.000000}%
\pgfsetfillcolor{currentfill}%
\pgfsetlinewidth{0.803000pt}%
\definecolor{currentstroke}{rgb}{0.000000,0.000000,0.000000}%
\pgfsetstrokecolor{currentstroke}%
\pgfsetdash{}{0pt}%
\pgfsys@defobject{currentmarker}{\pgfqpoint{0.000000in}{-0.048611in}}{\pgfqpoint{0.000000in}{0.000000in}}{%
\pgfpathmoveto{\pgfqpoint{0.000000in}{0.000000in}}%
\pgfpathlineto{\pgfqpoint{0.000000in}{-0.048611in}}%
\pgfusepath{stroke,fill}%
}%
\begin{pgfscope}%
\pgfsys@transformshift{1.582766in}{0.548769in}%
\pgfsys@useobject{currentmarker}{}%
\end{pgfscope}%
\end{pgfscope}%
\begin{pgfscope}%
\definecolor{textcolor}{rgb}{0.000000,0.000000,0.000000}%
\pgfsetstrokecolor{textcolor}%
\pgfsetfillcolor{textcolor}%
\pgftext[x=1.582766in,y=0.451547in,,top]{\color{textcolor}\sffamily\fontsize{10.000000}{12.000000}\selectfont \(\displaystyle {10}\)}%
\end{pgfscope}%
\begin{pgfscope}%
\pgfsetbuttcap%
\pgfsetroundjoin%
\definecolor{currentfill}{rgb}{0.000000,0.000000,0.000000}%
\pgfsetfillcolor{currentfill}%
\pgfsetlinewidth{0.803000pt}%
\definecolor{currentstroke}{rgb}{0.000000,0.000000,0.000000}%
\pgfsetstrokecolor{currentstroke}%
\pgfsetdash{}{0pt}%
\pgfsys@defobject{currentmarker}{\pgfqpoint{0.000000in}{-0.048611in}}{\pgfqpoint{0.000000in}{0.000000in}}{%
\pgfpathmoveto{\pgfqpoint{0.000000in}{0.000000in}}%
\pgfpathlineto{\pgfqpoint{0.000000in}{-0.048611in}}%
\pgfusepath{stroke,fill}%
}%
\begin{pgfscope}%
\pgfsys@transformshift{2.357922in}{0.548769in}%
\pgfsys@useobject{currentmarker}{}%
\end{pgfscope}%
\end{pgfscope}%
\begin{pgfscope}%
\definecolor{textcolor}{rgb}{0.000000,0.000000,0.000000}%
\pgfsetstrokecolor{textcolor}%
\pgfsetfillcolor{textcolor}%
\pgftext[x=2.357922in,y=0.451547in,,top]{\color{textcolor}\sffamily\fontsize{10.000000}{12.000000}\selectfont \(\displaystyle {20}\)}%
\end{pgfscope}%
\begin{pgfscope}%
\pgfsetbuttcap%
\pgfsetroundjoin%
\definecolor{currentfill}{rgb}{0.000000,0.000000,0.000000}%
\pgfsetfillcolor{currentfill}%
\pgfsetlinewidth{0.803000pt}%
\definecolor{currentstroke}{rgb}{0.000000,0.000000,0.000000}%
\pgfsetstrokecolor{currentstroke}%
\pgfsetdash{}{0pt}%
\pgfsys@defobject{currentmarker}{\pgfqpoint{0.000000in}{-0.048611in}}{\pgfqpoint{0.000000in}{0.000000in}}{%
\pgfpathmoveto{\pgfqpoint{0.000000in}{0.000000in}}%
\pgfpathlineto{\pgfqpoint{0.000000in}{-0.048611in}}%
\pgfusepath{stroke,fill}%
}%
\begin{pgfscope}%
\pgfsys@transformshift{3.133078in}{0.548769in}%
\pgfsys@useobject{currentmarker}{}%
\end{pgfscope}%
\end{pgfscope}%
\begin{pgfscope}%
\definecolor{textcolor}{rgb}{0.000000,0.000000,0.000000}%
\pgfsetstrokecolor{textcolor}%
\pgfsetfillcolor{textcolor}%
\pgftext[x=3.133078in,y=0.451547in,,top]{\color{textcolor}\sffamily\fontsize{10.000000}{12.000000}\selectfont \(\displaystyle {30}\)}%
\end{pgfscope}%
\begin{pgfscope}%
\pgfsetbuttcap%
\pgfsetroundjoin%
\definecolor{currentfill}{rgb}{0.000000,0.000000,0.000000}%
\pgfsetfillcolor{currentfill}%
\pgfsetlinewidth{0.803000pt}%
\definecolor{currentstroke}{rgb}{0.000000,0.000000,0.000000}%
\pgfsetstrokecolor{currentstroke}%
\pgfsetdash{}{0pt}%
\pgfsys@defobject{currentmarker}{\pgfqpoint{0.000000in}{-0.048611in}}{\pgfqpoint{0.000000in}{0.000000in}}{%
\pgfpathmoveto{\pgfqpoint{0.000000in}{0.000000in}}%
\pgfpathlineto{\pgfqpoint{0.000000in}{-0.048611in}}%
\pgfusepath{stroke,fill}%
}%
\begin{pgfscope}%
\pgfsys@transformshift{3.908234in}{0.548769in}%
\pgfsys@useobject{currentmarker}{}%
\end{pgfscope}%
\end{pgfscope}%
\begin{pgfscope}%
\definecolor{textcolor}{rgb}{0.000000,0.000000,0.000000}%
\pgfsetstrokecolor{textcolor}%
\pgfsetfillcolor{textcolor}%
\pgftext[x=3.908234in,y=0.451547in,,top]{\color{textcolor}\sffamily\fontsize{10.000000}{12.000000}\selectfont \(\displaystyle {40}\)}%
\end{pgfscope}%
\begin{pgfscope}%
\pgfsetbuttcap%
\pgfsetroundjoin%
\definecolor{currentfill}{rgb}{0.000000,0.000000,0.000000}%
\pgfsetfillcolor{currentfill}%
\pgfsetlinewidth{0.803000pt}%
\definecolor{currentstroke}{rgb}{0.000000,0.000000,0.000000}%
\pgfsetstrokecolor{currentstroke}%
\pgfsetdash{}{0pt}%
\pgfsys@defobject{currentmarker}{\pgfqpoint{0.000000in}{-0.048611in}}{\pgfqpoint{0.000000in}{0.000000in}}{%
\pgfpathmoveto{\pgfqpoint{0.000000in}{0.000000in}}%
\pgfpathlineto{\pgfqpoint{0.000000in}{-0.048611in}}%
\pgfusepath{stroke,fill}%
}%
\begin{pgfscope}%
\pgfsys@transformshift{4.683390in}{0.548769in}%
\pgfsys@useobject{currentmarker}{}%
\end{pgfscope}%
\end{pgfscope}%
\begin{pgfscope}%
\definecolor{textcolor}{rgb}{0.000000,0.000000,0.000000}%
\pgfsetstrokecolor{textcolor}%
\pgfsetfillcolor{textcolor}%
\pgftext[x=4.683390in,y=0.451547in,,top]{\color{textcolor}\sffamily\fontsize{10.000000}{12.000000}\selectfont \(\displaystyle {50}\)}%
\end{pgfscope}%
\begin{pgfscope}%
\pgfsetbuttcap%
\pgfsetroundjoin%
\definecolor{currentfill}{rgb}{0.000000,0.000000,0.000000}%
\pgfsetfillcolor{currentfill}%
\pgfsetlinewidth{0.803000pt}%
\definecolor{currentstroke}{rgb}{0.000000,0.000000,0.000000}%
\pgfsetstrokecolor{currentstroke}%
\pgfsetdash{}{0pt}%
\pgfsys@defobject{currentmarker}{\pgfqpoint{0.000000in}{-0.048611in}}{\pgfqpoint{0.000000in}{0.000000in}}{%
\pgfpathmoveto{\pgfqpoint{0.000000in}{0.000000in}}%
\pgfpathlineto{\pgfqpoint{0.000000in}{-0.048611in}}%
\pgfusepath{stroke,fill}%
}%
\begin{pgfscope}%
\pgfsys@transformshift{5.458546in}{0.548769in}%
\pgfsys@useobject{currentmarker}{}%
\end{pgfscope}%
\end{pgfscope}%
\begin{pgfscope}%
\definecolor{textcolor}{rgb}{0.000000,0.000000,0.000000}%
\pgfsetstrokecolor{textcolor}%
\pgfsetfillcolor{textcolor}%
\pgftext[x=5.458546in,y=0.451547in,,top]{\color{textcolor}\sffamily\fontsize{10.000000}{12.000000}\selectfont \(\displaystyle {60}\)}%
\end{pgfscope}%
\begin{pgfscope}%
\definecolor{textcolor}{rgb}{0.000000,0.000000,0.000000}%
\pgfsetstrokecolor{textcolor}%
\pgfsetfillcolor{textcolor}%
\pgftext[x=3.249352in,y=0.272658in,,top]{\color{textcolor}\sffamily\fontsize{10.000000}{12.000000}\selectfont Number of Sinks}%
\end{pgfscope}%
\begin{pgfscope}%
\pgfsetbuttcap%
\pgfsetroundjoin%
\definecolor{currentfill}{rgb}{0.000000,0.000000,0.000000}%
\pgfsetfillcolor{currentfill}%
\pgfsetlinewidth{0.803000pt}%
\definecolor{currentstroke}{rgb}{0.000000,0.000000,0.000000}%
\pgfsetstrokecolor{currentstroke}%
\pgfsetdash{}{0pt}%
\pgfsys@defobject{currentmarker}{\pgfqpoint{-0.048611in}{0.000000in}}{\pgfqpoint{0.000000in}{0.000000in}}{%
\pgfpathmoveto{\pgfqpoint{0.000000in}{0.000000in}}%
\pgfpathlineto{\pgfqpoint{-0.048611in}{0.000000in}}%
\pgfusepath{stroke,fill}%
}%
\begin{pgfscope}%
\pgfsys@transformshift{0.648703in}{0.689796in}%
\pgfsys@useobject{currentmarker}{}%
\end{pgfscope}%
\end{pgfscope}%
\begin{pgfscope}%
\definecolor{textcolor}{rgb}{0.000000,0.000000,0.000000}%
\pgfsetstrokecolor{textcolor}%
\pgfsetfillcolor{textcolor}%
\pgftext[x=0.482036in, y=0.641601in, left, base]{\color{textcolor}\sffamily\fontsize{10.000000}{12.000000}\selectfont \(\displaystyle {0}\)}%
\end{pgfscope}%
\begin{pgfscope}%
\pgfsetbuttcap%
\pgfsetroundjoin%
\definecolor{currentfill}{rgb}{0.000000,0.000000,0.000000}%
\pgfsetfillcolor{currentfill}%
\pgfsetlinewidth{0.803000pt}%
\definecolor{currentstroke}{rgb}{0.000000,0.000000,0.000000}%
\pgfsetstrokecolor{currentstroke}%
\pgfsetdash{}{0pt}%
\pgfsys@defobject{currentmarker}{\pgfqpoint{-0.048611in}{0.000000in}}{\pgfqpoint{0.000000in}{0.000000in}}{%
\pgfpathmoveto{\pgfqpoint{0.000000in}{0.000000in}}%
\pgfpathlineto{\pgfqpoint{-0.048611in}{0.000000in}}%
\pgfusepath{stroke,fill}%
}%
\begin{pgfscope}%
\pgfsys@transformshift{0.648703in}{1.112665in}%
\pgfsys@useobject{currentmarker}{}%
\end{pgfscope}%
\end{pgfscope}%
\begin{pgfscope}%
\definecolor{textcolor}{rgb}{0.000000,0.000000,0.000000}%
\pgfsetstrokecolor{textcolor}%
\pgfsetfillcolor{textcolor}%
\pgftext[x=0.343147in, y=1.064470in, left, base]{\color{textcolor}\sffamily\fontsize{10.000000}{12.000000}\selectfont \(\displaystyle {100}\)}%
\end{pgfscope}%
\begin{pgfscope}%
\pgfsetbuttcap%
\pgfsetroundjoin%
\definecolor{currentfill}{rgb}{0.000000,0.000000,0.000000}%
\pgfsetfillcolor{currentfill}%
\pgfsetlinewidth{0.803000pt}%
\definecolor{currentstroke}{rgb}{0.000000,0.000000,0.000000}%
\pgfsetstrokecolor{currentstroke}%
\pgfsetdash{}{0pt}%
\pgfsys@defobject{currentmarker}{\pgfqpoint{-0.048611in}{0.000000in}}{\pgfqpoint{0.000000in}{0.000000in}}{%
\pgfpathmoveto{\pgfqpoint{0.000000in}{0.000000in}}%
\pgfpathlineto{\pgfqpoint{-0.048611in}{0.000000in}}%
\pgfusepath{stroke,fill}%
}%
\begin{pgfscope}%
\pgfsys@transformshift{0.648703in}{1.535534in}%
\pgfsys@useobject{currentmarker}{}%
\end{pgfscope}%
\end{pgfscope}%
\begin{pgfscope}%
\definecolor{textcolor}{rgb}{0.000000,0.000000,0.000000}%
\pgfsetstrokecolor{textcolor}%
\pgfsetfillcolor{textcolor}%
\pgftext[x=0.343147in, y=1.487339in, left, base]{\color{textcolor}\sffamily\fontsize{10.000000}{12.000000}\selectfont \(\displaystyle {200}\)}%
\end{pgfscope}%
\begin{pgfscope}%
\pgfsetbuttcap%
\pgfsetroundjoin%
\definecolor{currentfill}{rgb}{0.000000,0.000000,0.000000}%
\pgfsetfillcolor{currentfill}%
\pgfsetlinewidth{0.803000pt}%
\definecolor{currentstroke}{rgb}{0.000000,0.000000,0.000000}%
\pgfsetstrokecolor{currentstroke}%
\pgfsetdash{}{0pt}%
\pgfsys@defobject{currentmarker}{\pgfqpoint{-0.048611in}{0.000000in}}{\pgfqpoint{0.000000in}{0.000000in}}{%
\pgfpathmoveto{\pgfqpoint{0.000000in}{0.000000in}}%
\pgfpathlineto{\pgfqpoint{-0.048611in}{0.000000in}}%
\pgfusepath{stroke,fill}%
}%
\begin{pgfscope}%
\pgfsys@transformshift{0.648703in}{1.958403in}%
\pgfsys@useobject{currentmarker}{}%
\end{pgfscope}%
\end{pgfscope}%
\begin{pgfscope}%
\definecolor{textcolor}{rgb}{0.000000,0.000000,0.000000}%
\pgfsetstrokecolor{textcolor}%
\pgfsetfillcolor{textcolor}%
\pgftext[x=0.343147in, y=1.910208in, left, base]{\color{textcolor}\sffamily\fontsize{10.000000}{12.000000}\selectfont \(\displaystyle {300}\)}%
\end{pgfscope}%
\begin{pgfscope}%
\pgfsetbuttcap%
\pgfsetroundjoin%
\definecolor{currentfill}{rgb}{0.000000,0.000000,0.000000}%
\pgfsetfillcolor{currentfill}%
\pgfsetlinewidth{0.803000pt}%
\definecolor{currentstroke}{rgb}{0.000000,0.000000,0.000000}%
\pgfsetstrokecolor{currentstroke}%
\pgfsetdash{}{0pt}%
\pgfsys@defobject{currentmarker}{\pgfqpoint{-0.048611in}{0.000000in}}{\pgfqpoint{0.000000in}{0.000000in}}{%
\pgfpathmoveto{\pgfqpoint{0.000000in}{0.000000in}}%
\pgfpathlineto{\pgfqpoint{-0.048611in}{0.000000in}}%
\pgfusepath{stroke,fill}%
}%
\begin{pgfscope}%
\pgfsys@transformshift{0.648703in}{2.381272in}%
\pgfsys@useobject{currentmarker}{}%
\end{pgfscope}%
\end{pgfscope}%
\begin{pgfscope}%
\definecolor{textcolor}{rgb}{0.000000,0.000000,0.000000}%
\pgfsetstrokecolor{textcolor}%
\pgfsetfillcolor{textcolor}%
\pgftext[x=0.343147in, y=2.333077in, left, base]{\color{textcolor}\sffamily\fontsize{10.000000}{12.000000}\selectfont \(\displaystyle {400}\)}%
\end{pgfscope}%
\begin{pgfscope}%
\pgfsetbuttcap%
\pgfsetroundjoin%
\definecolor{currentfill}{rgb}{0.000000,0.000000,0.000000}%
\pgfsetfillcolor{currentfill}%
\pgfsetlinewidth{0.803000pt}%
\definecolor{currentstroke}{rgb}{0.000000,0.000000,0.000000}%
\pgfsetstrokecolor{currentstroke}%
\pgfsetdash{}{0pt}%
\pgfsys@defobject{currentmarker}{\pgfqpoint{-0.048611in}{0.000000in}}{\pgfqpoint{0.000000in}{0.000000in}}{%
\pgfpathmoveto{\pgfqpoint{0.000000in}{0.000000in}}%
\pgfpathlineto{\pgfqpoint{-0.048611in}{0.000000in}}%
\pgfusepath{stroke,fill}%
}%
\begin{pgfscope}%
\pgfsys@transformshift{0.648703in}{2.804141in}%
\pgfsys@useobject{currentmarker}{}%
\end{pgfscope}%
\end{pgfscope}%
\begin{pgfscope}%
\definecolor{textcolor}{rgb}{0.000000,0.000000,0.000000}%
\pgfsetstrokecolor{textcolor}%
\pgfsetfillcolor{textcolor}%
\pgftext[x=0.343147in, y=2.755946in, left, base]{\color{textcolor}\sffamily\fontsize{10.000000}{12.000000}\selectfont \(\displaystyle {500}\)}%
\end{pgfscope}%
\begin{pgfscope}%
\pgfsetbuttcap%
\pgfsetroundjoin%
\definecolor{currentfill}{rgb}{0.000000,0.000000,0.000000}%
\pgfsetfillcolor{currentfill}%
\pgfsetlinewidth{0.803000pt}%
\definecolor{currentstroke}{rgb}{0.000000,0.000000,0.000000}%
\pgfsetstrokecolor{currentstroke}%
\pgfsetdash{}{0pt}%
\pgfsys@defobject{currentmarker}{\pgfqpoint{-0.048611in}{0.000000in}}{\pgfqpoint{0.000000in}{0.000000in}}{%
\pgfpathmoveto{\pgfqpoint{0.000000in}{0.000000in}}%
\pgfpathlineto{\pgfqpoint{-0.048611in}{0.000000in}}%
\pgfusepath{stroke,fill}%
}%
\begin{pgfscope}%
\pgfsys@transformshift{0.648703in}{3.227010in}%
\pgfsys@useobject{currentmarker}{}%
\end{pgfscope}%
\end{pgfscope}%
\begin{pgfscope}%
\definecolor{textcolor}{rgb}{0.000000,0.000000,0.000000}%
\pgfsetstrokecolor{textcolor}%
\pgfsetfillcolor{textcolor}%
\pgftext[x=0.343147in, y=3.178815in, left, base]{\color{textcolor}\sffamily\fontsize{10.000000}{12.000000}\selectfont \(\displaystyle {600}\)}%
\end{pgfscope}%
\begin{pgfscope}%
\pgfsetbuttcap%
\pgfsetroundjoin%
\definecolor{currentfill}{rgb}{0.000000,0.000000,0.000000}%
\pgfsetfillcolor{currentfill}%
\pgfsetlinewidth{0.803000pt}%
\definecolor{currentstroke}{rgb}{0.000000,0.000000,0.000000}%
\pgfsetstrokecolor{currentstroke}%
\pgfsetdash{}{0pt}%
\pgfsys@defobject{currentmarker}{\pgfqpoint{-0.048611in}{0.000000in}}{\pgfqpoint{0.000000in}{0.000000in}}{%
\pgfpathmoveto{\pgfqpoint{0.000000in}{0.000000in}}%
\pgfpathlineto{\pgfqpoint{-0.048611in}{0.000000in}}%
\pgfusepath{stroke,fill}%
}%
\begin{pgfscope}%
\pgfsys@transformshift{0.648703in}{3.649879in}%
\pgfsys@useobject{currentmarker}{}%
\end{pgfscope}%
\end{pgfscope}%
\begin{pgfscope}%
\definecolor{textcolor}{rgb}{0.000000,0.000000,0.000000}%
\pgfsetstrokecolor{textcolor}%
\pgfsetfillcolor{textcolor}%
\pgftext[x=0.343147in, y=3.601684in, left, base]{\color{textcolor}\sffamily\fontsize{10.000000}{12.000000}\selectfont \(\displaystyle {700}\)}%
\end{pgfscope}%
\begin{pgfscope}%
\definecolor{textcolor}{rgb}{0.000000,0.000000,0.000000}%
\pgfsetstrokecolor{textcolor}%
\pgfsetfillcolor{textcolor}%
\pgftext[x=0.287592in,y=2.100064in,,bottom,rotate=90.000000]{\color{textcolor}\sffamily\fontsize{10.000000}{12.000000}\selectfont Data Flow Time (s)}%
\end{pgfscope}%
\begin{pgfscope}%
\pgfsetrectcap%
\pgfsetmiterjoin%
\pgfsetlinewidth{0.803000pt}%
\definecolor{currentstroke}{rgb}{0.000000,0.000000,0.000000}%
\pgfsetstrokecolor{currentstroke}%
\pgfsetdash{}{0pt}%
\pgfpathmoveto{\pgfqpoint{0.648703in}{0.548769in}}%
\pgfpathlineto{\pgfqpoint{0.648703in}{3.651359in}}%
\pgfusepath{stroke}%
\end{pgfscope}%
\begin{pgfscope}%
\pgfsetrectcap%
\pgfsetmiterjoin%
\pgfsetlinewidth{0.803000pt}%
\definecolor{currentstroke}{rgb}{0.000000,0.000000,0.000000}%
\pgfsetstrokecolor{currentstroke}%
\pgfsetdash{}{0pt}%
\pgfpathmoveto{\pgfqpoint{5.850000in}{0.548769in}}%
\pgfpathlineto{\pgfqpoint{5.850000in}{3.651359in}}%
\pgfusepath{stroke}%
\end{pgfscope}%
\begin{pgfscope}%
\pgfsetrectcap%
\pgfsetmiterjoin%
\pgfsetlinewidth{0.803000pt}%
\definecolor{currentstroke}{rgb}{0.000000,0.000000,0.000000}%
\pgfsetstrokecolor{currentstroke}%
\pgfsetdash{}{0pt}%
\pgfpathmoveto{\pgfqpoint{0.648703in}{0.548769in}}%
\pgfpathlineto{\pgfqpoint{5.850000in}{0.548769in}}%
\pgfusepath{stroke}%
\end{pgfscope}%
\begin{pgfscope}%
\pgfsetrectcap%
\pgfsetmiterjoin%
\pgfsetlinewidth{0.803000pt}%
\definecolor{currentstroke}{rgb}{0.000000,0.000000,0.000000}%
\pgfsetstrokecolor{currentstroke}%
\pgfsetdash{}{0pt}%
\pgfpathmoveto{\pgfqpoint{0.648703in}{3.651359in}}%
\pgfpathlineto{\pgfqpoint{5.850000in}{3.651359in}}%
\pgfusepath{stroke}%
\end{pgfscope}%
\begin{pgfscope}%
\definecolor{textcolor}{rgb}{0.000000,0.000000,0.000000}%
\pgfsetstrokecolor{textcolor}%
\pgfsetfillcolor{textcolor}%
\pgftext[x=3.249352in,y=3.734692in,,base]{\color{textcolor}\sffamily\fontsize{12.000000}{14.400000}\selectfont Backwards}%
\end{pgfscope}%
\begin{pgfscope}%
\pgfsetbuttcap%
\pgfsetmiterjoin%
\definecolor{currentfill}{rgb}{1.000000,1.000000,1.000000}%
\pgfsetfillcolor{currentfill}%
\pgfsetfillopacity{0.800000}%
\pgfsetlinewidth{1.003750pt}%
\definecolor{currentstroke}{rgb}{0.800000,0.800000,0.800000}%
\pgfsetstrokecolor{currentstroke}%
\pgfsetstrokeopacity{0.800000}%
\pgfsetdash{}{0pt}%
\pgfpathmoveto{\pgfqpoint{4.300417in}{0.618213in}}%
\pgfpathlineto{\pgfqpoint{5.752778in}{0.618213in}}%
\pgfpathquadraticcurveto{\pgfqpoint{5.780556in}{0.618213in}}{\pgfqpoint{5.780556in}{0.645991in}}%
\pgfpathlineto{\pgfqpoint{5.780556in}{1.214463in}}%
\pgfpathquadraticcurveto{\pgfqpoint{5.780556in}{1.242241in}}{\pgfqpoint{5.752778in}{1.242241in}}%
\pgfpathlineto{\pgfqpoint{4.300417in}{1.242241in}}%
\pgfpathquadraticcurveto{\pgfqpoint{4.272639in}{1.242241in}}{\pgfqpoint{4.272639in}{1.214463in}}%
\pgfpathlineto{\pgfqpoint{4.272639in}{0.645991in}}%
\pgfpathquadraticcurveto{\pgfqpoint{4.272639in}{0.618213in}}{\pgfqpoint{4.300417in}{0.618213in}}%
\pgfpathclose%
\pgfusepath{stroke,fill}%
\end{pgfscope}%
\begin{pgfscope}%
\pgfsetbuttcap%
\pgfsetroundjoin%
\definecolor{currentfill}{rgb}{0.121569,0.466667,0.705882}%
\pgfsetfillcolor{currentfill}%
\pgfsetlinewidth{1.003750pt}%
\definecolor{currentstroke}{rgb}{0.121569,0.466667,0.705882}%
\pgfsetstrokecolor{currentstroke}%
\pgfsetdash{}{0pt}%
\pgfsys@defobject{currentmarker}{\pgfqpoint{-0.034722in}{-0.034722in}}{\pgfqpoint{0.034722in}{0.034722in}}{%
\pgfpathmoveto{\pgfqpoint{0.000000in}{-0.034722in}}%
\pgfpathcurveto{\pgfqpoint{0.009208in}{-0.034722in}}{\pgfqpoint{0.018041in}{-0.031064in}}{\pgfqpoint{0.024552in}{-0.024552in}}%
\pgfpathcurveto{\pgfqpoint{0.031064in}{-0.018041in}}{\pgfqpoint{0.034722in}{-0.009208in}}{\pgfqpoint{0.034722in}{0.000000in}}%
\pgfpathcurveto{\pgfqpoint{0.034722in}{0.009208in}}{\pgfqpoint{0.031064in}{0.018041in}}{\pgfqpoint{0.024552in}{0.024552in}}%
\pgfpathcurveto{\pgfqpoint{0.018041in}{0.031064in}}{\pgfqpoint{0.009208in}{0.034722in}}{\pgfqpoint{0.000000in}{0.034722in}}%
\pgfpathcurveto{\pgfqpoint{-0.009208in}{0.034722in}}{\pgfqpoint{-0.018041in}{0.031064in}}{\pgfqpoint{-0.024552in}{0.024552in}}%
\pgfpathcurveto{\pgfqpoint{-0.031064in}{0.018041in}}{\pgfqpoint{-0.034722in}{0.009208in}}{\pgfqpoint{-0.034722in}{0.000000in}}%
\pgfpathcurveto{\pgfqpoint{-0.034722in}{-0.009208in}}{\pgfqpoint{-0.031064in}{-0.018041in}}{\pgfqpoint{-0.024552in}{-0.024552in}}%
\pgfpathcurveto{\pgfqpoint{-0.018041in}{-0.031064in}}{\pgfqpoint{-0.009208in}{-0.034722in}}{\pgfqpoint{0.000000in}{-0.034722in}}%
\pgfpathclose%
\pgfusepath{stroke,fill}%
}%
\begin{pgfscope}%
\pgfsys@transformshift{4.467083in}{1.138074in}%
\pgfsys@useobject{currentmarker}{}%
\end{pgfscope}%
\end{pgfscope}%
\begin{pgfscope}%
\definecolor{textcolor}{rgb}{0.000000,0.000000,0.000000}%
\pgfsetstrokecolor{textcolor}%
\pgfsetfillcolor{textcolor}%
\pgftext[x=4.717083in,y=1.089463in,left,base]{\color{textcolor}\sffamily\fontsize{10.000000}{12.000000}\selectfont No Timeout}%
\end{pgfscope}%
\begin{pgfscope}%
\pgfsetbuttcap%
\pgfsetroundjoin%
\definecolor{currentfill}{rgb}{1.000000,0.498039,0.054902}%
\pgfsetfillcolor{currentfill}%
\pgfsetlinewidth{1.003750pt}%
\definecolor{currentstroke}{rgb}{1.000000,0.498039,0.054902}%
\pgfsetstrokecolor{currentstroke}%
\pgfsetdash{}{0pt}%
\pgfsys@defobject{currentmarker}{\pgfqpoint{-0.034722in}{-0.034722in}}{\pgfqpoint{0.034722in}{0.034722in}}{%
\pgfpathmoveto{\pgfqpoint{0.000000in}{-0.034722in}}%
\pgfpathcurveto{\pgfqpoint{0.009208in}{-0.034722in}}{\pgfqpoint{0.018041in}{-0.031064in}}{\pgfqpoint{0.024552in}{-0.024552in}}%
\pgfpathcurveto{\pgfqpoint{0.031064in}{-0.018041in}}{\pgfqpoint{0.034722in}{-0.009208in}}{\pgfqpoint{0.034722in}{0.000000in}}%
\pgfpathcurveto{\pgfqpoint{0.034722in}{0.009208in}}{\pgfqpoint{0.031064in}{0.018041in}}{\pgfqpoint{0.024552in}{0.024552in}}%
\pgfpathcurveto{\pgfqpoint{0.018041in}{0.031064in}}{\pgfqpoint{0.009208in}{0.034722in}}{\pgfqpoint{0.000000in}{0.034722in}}%
\pgfpathcurveto{\pgfqpoint{-0.009208in}{0.034722in}}{\pgfqpoint{-0.018041in}{0.031064in}}{\pgfqpoint{-0.024552in}{0.024552in}}%
\pgfpathcurveto{\pgfqpoint{-0.031064in}{0.018041in}}{\pgfqpoint{-0.034722in}{0.009208in}}{\pgfqpoint{-0.034722in}{0.000000in}}%
\pgfpathcurveto{\pgfqpoint{-0.034722in}{-0.009208in}}{\pgfqpoint{-0.031064in}{-0.018041in}}{\pgfqpoint{-0.024552in}{-0.024552in}}%
\pgfpathcurveto{\pgfqpoint{-0.018041in}{-0.031064in}}{\pgfqpoint{-0.009208in}{-0.034722in}}{\pgfqpoint{0.000000in}{-0.034722in}}%
\pgfpathclose%
\pgfusepath{stroke,fill}%
}%
\begin{pgfscope}%
\pgfsys@transformshift{4.467083in}{0.944463in}%
\pgfsys@useobject{currentmarker}{}%
\end{pgfscope}%
\end{pgfscope}%
\begin{pgfscope}%
\definecolor{textcolor}{rgb}{0.000000,0.000000,0.000000}%
\pgfsetstrokecolor{textcolor}%
\pgfsetfillcolor{textcolor}%
\pgftext[x=4.717083in,y=0.895852in,left,base]{\color{textcolor}\sffamily\fontsize{10.000000}{12.000000}\selectfont Time Timeout}%
\end{pgfscope}%
\begin{pgfscope}%
\pgfsetbuttcap%
\pgfsetroundjoin%
\definecolor{currentfill}{rgb}{0.839216,0.152941,0.156863}%
\pgfsetfillcolor{currentfill}%
\pgfsetlinewidth{1.003750pt}%
\definecolor{currentstroke}{rgb}{0.839216,0.152941,0.156863}%
\pgfsetstrokecolor{currentstroke}%
\pgfsetdash{}{0pt}%
\pgfsys@defobject{currentmarker}{\pgfqpoint{-0.034722in}{-0.034722in}}{\pgfqpoint{0.034722in}{0.034722in}}{%
\pgfpathmoveto{\pgfqpoint{0.000000in}{-0.034722in}}%
\pgfpathcurveto{\pgfqpoint{0.009208in}{-0.034722in}}{\pgfqpoint{0.018041in}{-0.031064in}}{\pgfqpoint{0.024552in}{-0.024552in}}%
\pgfpathcurveto{\pgfqpoint{0.031064in}{-0.018041in}}{\pgfqpoint{0.034722in}{-0.009208in}}{\pgfqpoint{0.034722in}{0.000000in}}%
\pgfpathcurveto{\pgfqpoint{0.034722in}{0.009208in}}{\pgfqpoint{0.031064in}{0.018041in}}{\pgfqpoint{0.024552in}{0.024552in}}%
\pgfpathcurveto{\pgfqpoint{0.018041in}{0.031064in}}{\pgfqpoint{0.009208in}{0.034722in}}{\pgfqpoint{0.000000in}{0.034722in}}%
\pgfpathcurveto{\pgfqpoint{-0.009208in}{0.034722in}}{\pgfqpoint{-0.018041in}{0.031064in}}{\pgfqpoint{-0.024552in}{0.024552in}}%
\pgfpathcurveto{\pgfqpoint{-0.031064in}{0.018041in}}{\pgfqpoint{-0.034722in}{0.009208in}}{\pgfqpoint{-0.034722in}{0.000000in}}%
\pgfpathcurveto{\pgfqpoint{-0.034722in}{-0.009208in}}{\pgfqpoint{-0.031064in}{-0.018041in}}{\pgfqpoint{-0.024552in}{-0.024552in}}%
\pgfpathcurveto{\pgfqpoint{-0.018041in}{-0.031064in}}{\pgfqpoint{-0.009208in}{-0.034722in}}{\pgfqpoint{0.000000in}{-0.034722in}}%
\pgfpathclose%
\pgfusepath{stroke,fill}%
}%
\begin{pgfscope}%
\pgfsys@transformshift{4.467083in}{0.750852in}%
\pgfsys@useobject{currentmarker}{}%
\end{pgfscope}%
\end{pgfscope}%
\begin{pgfscope}%
\definecolor{textcolor}{rgb}{0.000000,0.000000,0.000000}%
\pgfsetstrokecolor{textcolor}%
\pgfsetfillcolor{textcolor}%
\pgftext[x=4.717083in,y=0.702241in,left,base]{\color{textcolor}\sffamily\fontsize{10.000000}{12.000000}\selectfont Memory Timeout}%
\end{pgfscope}%
\end{pgfpicture}%
\makeatother%
\endgroup%

            }
        \end{subfigure}
        \caption{Dataflow time in comparison to source and sink count}
    \end{figure}

    \begin{figure}
        \resizebox{\columnwidth}{!}{
            %% Creator: Matplotlib, PGF backend
%%
%% To include the figure in your LaTeX document, write
%%   \input{<filename>.pgf}
%%
%% Make sure the required packages are loaded in your preamble
%%   \usepackage{pgf}
%%
%% and, on pdftex
%%   \usepackage[utf8]{inputenc}\DeclareUnicodeCharacter{2212}{-}
%%
%% or, on luatex and xetex
%%   \usepackage{unicode-math}
%%
%% Figures using additional raster images can only be included by \input if
%% they are in the same directory as the main LaTeX file. For loading figures
%% from other directories you can use the `import` package
%%   \usepackage{import}
%%
%% and then include the figures with
%%   \import{<path to file>}{<filename>.pgf}
%%
%% Matplotlib used the following preamble
%%   \usepackage{fontspec}
%%
\begingroup%
\makeatletter%
\begin{pgfpicture}%
\pgfpathrectangle{\pgfpointorigin}{\pgfqpoint{20.000000in}{5.000000in}}%
\pgfusepath{use as bounding box, clip}%
\begin{pgfscope}%
\pgfsetbuttcap%
\pgfsetmiterjoin%
\definecolor{currentfill}{rgb}{1.000000,1.000000,1.000000}%
\pgfsetfillcolor{currentfill}%
\pgfsetlinewidth{0.000000pt}%
\definecolor{currentstroke}{rgb}{1.000000,1.000000,1.000000}%
\pgfsetstrokecolor{currentstroke}%
\pgfsetdash{}{0pt}%
\pgfpathmoveto{\pgfqpoint{0.000000in}{0.000000in}}%
\pgfpathlineto{\pgfqpoint{20.000000in}{0.000000in}}%
\pgfpathlineto{\pgfqpoint{20.000000in}{5.000000in}}%
\pgfpathlineto{\pgfqpoint{0.000000in}{5.000000in}}%
\pgfpathclose%
\pgfusepath{fill}%
\end{pgfscope}%
\begin{pgfscope}%
\pgfsetbuttcap%
\pgfsetmiterjoin%
\definecolor{currentfill}{rgb}{1.000000,1.000000,1.000000}%
\pgfsetfillcolor{currentfill}%
\pgfsetlinewidth{0.000000pt}%
\definecolor{currentstroke}{rgb}{0.000000,0.000000,0.000000}%
\pgfsetstrokecolor{currentstroke}%
\pgfsetstrokeopacity{0.000000}%
\pgfsetdash{}{0pt}%
\pgfpathmoveto{\pgfqpoint{2.500000in}{0.625000in}}%
\pgfpathlineto{\pgfqpoint{18.000000in}{0.625000in}}%
\pgfpathlineto{\pgfqpoint{18.000000in}{4.400000in}}%
\pgfpathlineto{\pgfqpoint{2.500000in}{4.400000in}}%
\pgfpathclose%
\pgfusepath{fill}%
\end{pgfscope}%
\begin{pgfscope}%
\pgfpathrectangle{\pgfqpoint{2.500000in}{0.625000in}}{\pgfqpoint{15.500000in}{3.775000in}}%
\pgfusepath{clip}%
\pgfsetbuttcap%
\pgfsetmiterjoin%
\definecolor{currentfill}{rgb}{0.121569,0.466667,0.705882}%
\pgfsetfillcolor{currentfill}%
\pgfsetlinewidth{0.000000pt}%
\definecolor{currentstroke}{rgb}{0.000000,0.000000,0.000000}%
\pgfsetstrokecolor{currentstroke}%
\pgfsetstrokeopacity{0.000000}%
\pgfsetdash{}{0pt}%
\pgfpathmoveto{\pgfqpoint{3.204545in}{2.565994in}}%
\pgfpathlineto{\pgfqpoint{3.229223in}{2.565994in}}%
\pgfpathlineto{\pgfqpoint{3.229223in}{2.582453in}}%
\pgfpathlineto{\pgfqpoint{3.204545in}{2.582453in}}%
\pgfpathclose%
\pgfusepath{fill}%
\end{pgfscope}%
\begin{pgfscope}%
\pgfpathrectangle{\pgfqpoint{2.500000in}{0.625000in}}{\pgfqpoint{15.500000in}{3.775000in}}%
\pgfusepath{clip}%
\pgfsetbuttcap%
\pgfsetmiterjoin%
\definecolor{currentfill}{rgb}{0.121569,0.466667,0.705882}%
\pgfsetfillcolor{currentfill}%
\pgfsetlinewidth{0.000000pt}%
\definecolor{currentstroke}{rgb}{0.000000,0.000000,0.000000}%
\pgfsetstrokecolor{currentstroke}%
\pgfsetstrokeopacity{0.000000}%
\pgfsetdash{}{0pt}%
\pgfpathmoveto{\pgfqpoint{3.327933in}{2.565994in}}%
\pgfpathlineto{\pgfqpoint{3.352611in}{2.565994in}}%
\pgfpathlineto{\pgfqpoint{3.352611in}{2.568737in}}%
\pgfpathlineto{\pgfqpoint{3.327933in}{2.568737in}}%
\pgfpathclose%
\pgfusepath{fill}%
\end{pgfscope}%
\begin{pgfscope}%
\pgfpathrectangle{\pgfqpoint{2.500000in}{0.625000in}}{\pgfqpoint{15.500000in}{3.775000in}}%
\pgfusepath{clip}%
\pgfsetbuttcap%
\pgfsetmiterjoin%
\definecolor{currentfill}{rgb}{0.121569,0.466667,0.705882}%
\pgfsetfillcolor{currentfill}%
\pgfsetlinewidth{0.000000pt}%
\definecolor{currentstroke}{rgb}{0.000000,0.000000,0.000000}%
\pgfsetstrokecolor{currentstroke}%
\pgfsetstrokeopacity{0.000000}%
\pgfsetdash{}{0pt}%
\pgfpathmoveto{\pgfqpoint{3.451321in}{2.565994in}}%
\pgfpathlineto{\pgfqpoint{3.475999in}{2.565994in}}%
\pgfpathlineto{\pgfqpoint{3.475999in}{0.922781in}}%
\pgfpathlineto{\pgfqpoint{3.451321in}{0.922781in}}%
\pgfpathclose%
\pgfusepath{fill}%
\end{pgfscope}%
\begin{pgfscope}%
\pgfpathrectangle{\pgfqpoint{2.500000in}{0.625000in}}{\pgfqpoint{15.500000in}{3.775000in}}%
\pgfusepath{clip}%
\pgfsetbuttcap%
\pgfsetmiterjoin%
\definecolor{currentfill}{rgb}{0.121569,0.466667,0.705882}%
\pgfsetfillcolor{currentfill}%
\pgfsetlinewidth{0.000000pt}%
\definecolor{currentstroke}{rgb}{0.000000,0.000000,0.000000}%
\pgfsetstrokecolor{currentstroke}%
\pgfsetstrokeopacity{0.000000}%
\pgfsetdash{}{0pt}%
\pgfpathmoveto{\pgfqpoint{3.574709in}{2.565994in}}%
\pgfpathlineto{\pgfqpoint{3.599387in}{2.565994in}}%
\pgfpathlineto{\pgfqpoint{3.599387in}{2.565994in}}%
\pgfpathlineto{\pgfqpoint{3.574709in}{2.565994in}}%
\pgfpathclose%
\pgfusepath{fill}%
\end{pgfscope}%
\begin{pgfscope}%
\pgfpathrectangle{\pgfqpoint{2.500000in}{0.625000in}}{\pgfqpoint{15.500000in}{3.775000in}}%
\pgfusepath{clip}%
\pgfsetbuttcap%
\pgfsetmiterjoin%
\definecolor{currentfill}{rgb}{0.121569,0.466667,0.705882}%
\pgfsetfillcolor{currentfill}%
\pgfsetlinewidth{0.000000pt}%
\definecolor{currentstroke}{rgb}{0.000000,0.000000,0.000000}%
\pgfsetstrokecolor{currentstroke}%
\pgfsetstrokeopacity{0.000000}%
\pgfsetdash{}{0pt}%
\pgfpathmoveto{\pgfqpoint{3.698097in}{2.565994in}}%
\pgfpathlineto{\pgfqpoint{3.722775in}{2.565994in}}%
\pgfpathlineto{\pgfqpoint{3.722775in}{2.579710in}}%
\pgfpathlineto{\pgfqpoint{3.698097in}{2.579710in}}%
\pgfpathclose%
\pgfusepath{fill}%
\end{pgfscope}%
\begin{pgfscope}%
\pgfpathrectangle{\pgfqpoint{2.500000in}{0.625000in}}{\pgfqpoint{15.500000in}{3.775000in}}%
\pgfusepath{clip}%
\pgfsetbuttcap%
\pgfsetmiterjoin%
\definecolor{currentfill}{rgb}{0.121569,0.466667,0.705882}%
\pgfsetfillcolor{currentfill}%
\pgfsetlinewidth{0.000000pt}%
\definecolor{currentstroke}{rgb}{0.000000,0.000000,0.000000}%
\pgfsetstrokecolor{currentstroke}%
\pgfsetstrokeopacity{0.000000}%
\pgfsetdash{}{0pt}%
\pgfpathmoveto{\pgfqpoint{3.821485in}{2.565994in}}%
\pgfpathlineto{\pgfqpoint{3.846163in}{2.565994in}}%
\pgfpathlineto{\pgfqpoint{3.846163in}{2.489182in}}%
\pgfpathlineto{\pgfqpoint{3.821485in}{2.489182in}}%
\pgfpathclose%
\pgfusepath{fill}%
\end{pgfscope}%
\begin{pgfscope}%
\pgfpathrectangle{\pgfqpoint{2.500000in}{0.625000in}}{\pgfqpoint{15.500000in}{3.775000in}}%
\pgfusepath{clip}%
\pgfsetbuttcap%
\pgfsetmiterjoin%
\definecolor{currentfill}{rgb}{0.121569,0.466667,0.705882}%
\pgfsetfillcolor{currentfill}%
\pgfsetlinewidth{0.000000pt}%
\definecolor{currentstroke}{rgb}{0.000000,0.000000,0.000000}%
\pgfsetstrokecolor{currentstroke}%
\pgfsetstrokeopacity{0.000000}%
\pgfsetdash{}{0pt}%
\pgfpathmoveto{\pgfqpoint{3.944873in}{2.565994in}}%
\pgfpathlineto{\pgfqpoint{3.969551in}{2.565994in}}%
\pgfpathlineto{\pgfqpoint{3.969551in}{2.574223in}}%
\pgfpathlineto{\pgfqpoint{3.944873in}{2.574223in}}%
\pgfpathclose%
\pgfusepath{fill}%
\end{pgfscope}%
\begin{pgfscope}%
\pgfpathrectangle{\pgfqpoint{2.500000in}{0.625000in}}{\pgfqpoint{15.500000in}{3.775000in}}%
\pgfusepath{clip}%
\pgfsetbuttcap%
\pgfsetmiterjoin%
\definecolor{currentfill}{rgb}{0.121569,0.466667,0.705882}%
\pgfsetfillcolor{currentfill}%
\pgfsetlinewidth{0.000000pt}%
\definecolor{currentstroke}{rgb}{0.000000,0.000000,0.000000}%
\pgfsetstrokecolor{currentstroke}%
\pgfsetstrokeopacity{0.000000}%
\pgfsetdash{}{0pt}%
\pgfpathmoveto{\pgfqpoint{4.068261in}{2.565994in}}%
\pgfpathlineto{\pgfqpoint{4.092939in}{2.565994in}}%
\pgfpathlineto{\pgfqpoint{4.092939in}{2.568737in}}%
\pgfpathlineto{\pgfqpoint{4.068261in}{2.568737in}}%
\pgfpathclose%
\pgfusepath{fill}%
\end{pgfscope}%
\begin{pgfscope}%
\pgfpathrectangle{\pgfqpoint{2.500000in}{0.625000in}}{\pgfqpoint{15.500000in}{3.775000in}}%
\pgfusepath{clip}%
\pgfsetbuttcap%
\pgfsetmiterjoin%
\definecolor{currentfill}{rgb}{0.121569,0.466667,0.705882}%
\pgfsetfillcolor{currentfill}%
\pgfsetlinewidth{0.000000pt}%
\definecolor{currentstroke}{rgb}{0.000000,0.000000,0.000000}%
\pgfsetstrokecolor{currentstroke}%
\pgfsetstrokeopacity{0.000000}%
\pgfsetdash{}{0pt}%
\pgfpathmoveto{\pgfqpoint{4.191649in}{2.565994in}}%
\pgfpathlineto{\pgfqpoint{4.216327in}{2.565994in}}%
\pgfpathlineto{\pgfqpoint{4.216327in}{2.563250in}}%
\pgfpathlineto{\pgfqpoint{4.191649in}{2.563250in}}%
\pgfpathclose%
\pgfusepath{fill}%
\end{pgfscope}%
\begin{pgfscope}%
\pgfpathrectangle{\pgfqpoint{2.500000in}{0.625000in}}{\pgfqpoint{15.500000in}{3.775000in}}%
\pgfusepath{clip}%
\pgfsetbuttcap%
\pgfsetmiterjoin%
\definecolor{currentfill}{rgb}{0.121569,0.466667,0.705882}%
\pgfsetfillcolor{currentfill}%
\pgfsetlinewidth{0.000000pt}%
\definecolor{currentstroke}{rgb}{0.000000,0.000000,0.000000}%
\pgfsetstrokecolor{currentstroke}%
\pgfsetstrokeopacity{0.000000}%
\pgfsetdash{}{0pt}%
\pgfpathmoveto{\pgfqpoint{4.315037in}{2.565994in}}%
\pgfpathlineto{\pgfqpoint{4.339715in}{2.565994in}}%
\pgfpathlineto{\pgfqpoint{4.339715in}{0.911808in}}%
\pgfpathlineto{\pgfqpoint{4.315037in}{0.911808in}}%
\pgfpathclose%
\pgfusepath{fill}%
\end{pgfscope}%
\begin{pgfscope}%
\pgfpathrectangle{\pgfqpoint{2.500000in}{0.625000in}}{\pgfqpoint{15.500000in}{3.775000in}}%
\pgfusepath{clip}%
\pgfsetbuttcap%
\pgfsetmiterjoin%
\definecolor{currentfill}{rgb}{0.121569,0.466667,0.705882}%
\pgfsetfillcolor{currentfill}%
\pgfsetlinewidth{0.000000pt}%
\definecolor{currentstroke}{rgb}{0.000000,0.000000,0.000000}%
\pgfsetstrokecolor{currentstroke}%
\pgfsetstrokeopacity{0.000000}%
\pgfsetdash{}{0pt}%
\pgfpathmoveto{\pgfqpoint{4.438425in}{2.565994in}}%
\pgfpathlineto{\pgfqpoint{4.463103in}{2.565994in}}%
\pgfpathlineto{\pgfqpoint{4.463103in}{2.574223in}}%
\pgfpathlineto{\pgfqpoint{4.438425in}{2.574223in}}%
\pgfpathclose%
\pgfusepath{fill}%
\end{pgfscope}%
\begin{pgfscope}%
\pgfpathrectangle{\pgfqpoint{2.500000in}{0.625000in}}{\pgfqpoint{15.500000in}{3.775000in}}%
\pgfusepath{clip}%
\pgfsetbuttcap%
\pgfsetmiterjoin%
\definecolor{currentfill}{rgb}{0.121569,0.466667,0.705882}%
\pgfsetfillcolor{currentfill}%
\pgfsetlinewidth{0.000000pt}%
\definecolor{currentstroke}{rgb}{0.000000,0.000000,0.000000}%
\pgfsetstrokecolor{currentstroke}%
\pgfsetstrokeopacity{0.000000}%
\pgfsetdash{}{0pt}%
\pgfpathmoveto{\pgfqpoint{4.561813in}{2.565994in}}%
\pgfpathlineto{\pgfqpoint{4.586491in}{2.565994in}}%
\pgfpathlineto{\pgfqpoint{4.586491in}{2.565994in}}%
\pgfpathlineto{\pgfqpoint{4.561813in}{2.565994in}}%
\pgfpathclose%
\pgfusepath{fill}%
\end{pgfscope}%
\begin{pgfscope}%
\pgfpathrectangle{\pgfqpoint{2.500000in}{0.625000in}}{\pgfqpoint{15.500000in}{3.775000in}}%
\pgfusepath{clip}%
\pgfsetbuttcap%
\pgfsetmiterjoin%
\definecolor{currentfill}{rgb}{0.121569,0.466667,0.705882}%
\pgfsetfillcolor{currentfill}%
\pgfsetlinewidth{0.000000pt}%
\definecolor{currentstroke}{rgb}{0.000000,0.000000,0.000000}%
\pgfsetstrokecolor{currentstroke}%
\pgfsetstrokeopacity{0.000000}%
\pgfsetdash{}{0pt}%
\pgfpathmoveto{\pgfqpoint{4.685201in}{2.565994in}}%
\pgfpathlineto{\pgfqpoint{4.709879in}{2.565994in}}%
\pgfpathlineto{\pgfqpoint{4.709879in}{0.933754in}}%
\pgfpathlineto{\pgfqpoint{4.685201in}{0.933754in}}%
\pgfpathclose%
\pgfusepath{fill}%
\end{pgfscope}%
\begin{pgfscope}%
\pgfpathrectangle{\pgfqpoint{2.500000in}{0.625000in}}{\pgfqpoint{15.500000in}{3.775000in}}%
\pgfusepath{clip}%
\pgfsetbuttcap%
\pgfsetmiterjoin%
\definecolor{currentfill}{rgb}{0.121569,0.466667,0.705882}%
\pgfsetfillcolor{currentfill}%
\pgfsetlinewidth{0.000000pt}%
\definecolor{currentstroke}{rgb}{0.000000,0.000000,0.000000}%
\pgfsetstrokecolor{currentstroke}%
\pgfsetstrokeopacity{0.000000}%
\pgfsetdash{}{0pt}%
\pgfpathmoveto{\pgfqpoint{4.808589in}{2.565994in}}%
\pgfpathlineto{\pgfqpoint{4.833267in}{2.565994in}}%
\pgfpathlineto{\pgfqpoint{4.833267in}{2.508385in}}%
\pgfpathlineto{\pgfqpoint{4.808589in}{2.508385in}}%
\pgfpathclose%
\pgfusepath{fill}%
\end{pgfscope}%
\begin{pgfscope}%
\pgfpathrectangle{\pgfqpoint{2.500000in}{0.625000in}}{\pgfqpoint{15.500000in}{3.775000in}}%
\pgfusepath{clip}%
\pgfsetbuttcap%
\pgfsetmiterjoin%
\definecolor{currentfill}{rgb}{0.121569,0.466667,0.705882}%
\pgfsetfillcolor{currentfill}%
\pgfsetlinewidth{0.000000pt}%
\definecolor{currentstroke}{rgb}{0.000000,0.000000,0.000000}%
\pgfsetstrokecolor{currentstroke}%
\pgfsetstrokeopacity{0.000000}%
\pgfsetdash{}{0pt}%
\pgfpathmoveto{\pgfqpoint{4.931977in}{2.565994in}}%
\pgfpathlineto{\pgfqpoint{4.956655in}{2.565994in}}%
\pgfpathlineto{\pgfqpoint{4.956655in}{2.642805in}}%
\pgfpathlineto{\pgfqpoint{4.931977in}{2.642805in}}%
\pgfpathclose%
\pgfusepath{fill}%
\end{pgfscope}%
\begin{pgfscope}%
\pgfpathrectangle{\pgfqpoint{2.500000in}{0.625000in}}{\pgfqpoint{15.500000in}{3.775000in}}%
\pgfusepath{clip}%
\pgfsetbuttcap%
\pgfsetmiterjoin%
\definecolor{currentfill}{rgb}{0.121569,0.466667,0.705882}%
\pgfsetfillcolor{currentfill}%
\pgfsetlinewidth{0.000000pt}%
\definecolor{currentstroke}{rgb}{0.000000,0.000000,0.000000}%
\pgfsetstrokecolor{currentstroke}%
\pgfsetstrokeopacity{0.000000}%
\pgfsetdash{}{0pt}%
\pgfpathmoveto{\pgfqpoint{5.055365in}{2.565994in}}%
\pgfpathlineto{\pgfqpoint{5.080043in}{2.565994in}}%
\pgfpathlineto{\pgfqpoint{5.080043in}{0.909065in}}%
\pgfpathlineto{\pgfqpoint{5.055365in}{0.909065in}}%
\pgfpathclose%
\pgfusepath{fill}%
\end{pgfscope}%
\begin{pgfscope}%
\pgfpathrectangle{\pgfqpoint{2.500000in}{0.625000in}}{\pgfqpoint{15.500000in}{3.775000in}}%
\pgfusepath{clip}%
\pgfsetbuttcap%
\pgfsetmiterjoin%
\definecolor{currentfill}{rgb}{0.121569,0.466667,0.705882}%
\pgfsetfillcolor{currentfill}%
\pgfsetlinewidth{0.000000pt}%
\definecolor{currentstroke}{rgb}{0.000000,0.000000,0.000000}%
\pgfsetstrokecolor{currentstroke}%
\pgfsetstrokeopacity{0.000000}%
\pgfsetdash{}{0pt}%
\pgfpathmoveto{\pgfqpoint{5.178753in}{2.565994in}}%
\pgfpathlineto{\pgfqpoint{5.203431in}{2.565994in}}%
\pgfpathlineto{\pgfqpoint{5.203431in}{2.598913in}}%
\pgfpathlineto{\pgfqpoint{5.178753in}{2.598913in}}%
\pgfpathclose%
\pgfusepath{fill}%
\end{pgfscope}%
\begin{pgfscope}%
\pgfpathrectangle{\pgfqpoint{2.500000in}{0.625000in}}{\pgfqpoint{15.500000in}{3.775000in}}%
\pgfusepath{clip}%
\pgfsetbuttcap%
\pgfsetmiterjoin%
\definecolor{currentfill}{rgb}{0.121569,0.466667,0.705882}%
\pgfsetfillcolor{currentfill}%
\pgfsetlinewidth{0.000000pt}%
\definecolor{currentstroke}{rgb}{0.000000,0.000000,0.000000}%
\pgfsetstrokecolor{currentstroke}%
\pgfsetstrokeopacity{0.000000}%
\pgfsetdash{}{0pt}%
\pgfpathmoveto{\pgfqpoint{5.302141in}{2.565994in}}%
\pgfpathlineto{\pgfqpoint{5.326819in}{2.565994in}}%
\pgfpathlineto{\pgfqpoint{5.326819in}{0.917294in}}%
\pgfpathlineto{\pgfqpoint{5.302141in}{0.917294in}}%
\pgfpathclose%
\pgfusepath{fill}%
\end{pgfscope}%
\begin{pgfscope}%
\pgfpathrectangle{\pgfqpoint{2.500000in}{0.625000in}}{\pgfqpoint{15.500000in}{3.775000in}}%
\pgfusepath{clip}%
\pgfsetbuttcap%
\pgfsetmiterjoin%
\definecolor{currentfill}{rgb}{0.121569,0.466667,0.705882}%
\pgfsetfillcolor{currentfill}%
\pgfsetlinewidth{0.000000pt}%
\definecolor{currentstroke}{rgb}{0.000000,0.000000,0.000000}%
\pgfsetstrokecolor{currentstroke}%
\pgfsetstrokeopacity{0.000000}%
\pgfsetdash{}{0pt}%
\pgfpathmoveto{\pgfqpoint{5.425529in}{2.565994in}}%
\pgfpathlineto{\pgfqpoint{5.450207in}{2.565994in}}%
\pgfpathlineto{\pgfqpoint{5.450207in}{0.920038in}}%
\pgfpathlineto{\pgfqpoint{5.425529in}{0.920038in}}%
\pgfpathclose%
\pgfusepath{fill}%
\end{pgfscope}%
\begin{pgfscope}%
\pgfpathrectangle{\pgfqpoint{2.500000in}{0.625000in}}{\pgfqpoint{15.500000in}{3.775000in}}%
\pgfusepath{clip}%
\pgfsetbuttcap%
\pgfsetmiterjoin%
\definecolor{currentfill}{rgb}{0.121569,0.466667,0.705882}%
\pgfsetfillcolor{currentfill}%
\pgfsetlinewidth{0.000000pt}%
\definecolor{currentstroke}{rgb}{0.000000,0.000000,0.000000}%
\pgfsetstrokecolor{currentstroke}%
\pgfsetstrokeopacity{0.000000}%
\pgfsetdash{}{0pt}%
\pgfpathmoveto{\pgfqpoint{5.548917in}{2.565994in}}%
\pgfpathlineto{\pgfqpoint{5.573595in}{2.565994in}}%
\pgfpathlineto{\pgfqpoint{5.573595in}{2.747049in}}%
\pgfpathlineto{\pgfqpoint{5.548917in}{2.747049in}}%
\pgfpathclose%
\pgfusepath{fill}%
\end{pgfscope}%
\begin{pgfscope}%
\pgfpathrectangle{\pgfqpoint{2.500000in}{0.625000in}}{\pgfqpoint{15.500000in}{3.775000in}}%
\pgfusepath{clip}%
\pgfsetbuttcap%
\pgfsetmiterjoin%
\definecolor{currentfill}{rgb}{0.121569,0.466667,0.705882}%
\pgfsetfillcolor{currentfill}%
\pgfsetlinewidth{0.000000pt}%
\definecolor{currentstroke}{rgb}{0.000000,0.000000,0.000000}%
\pgfsetstrokecolor{currentstroke}%
\pgfsetstrokeopacity{0.000000}%
\pgfsetdash{}{0pt}%
\pgfpathmoveto{\pgfqpoint{5.672305in}{2.565994in}}%
\pgfpathlineto{\pgfqpoint{5.696983in}{2.565994in}}%
\pgfpathlineto{\pgfqpoint{5.696983in}{2.571480in}}%
\pgfpathlineto{\pgfqpoint{5.672305in}{2.571480in}}%
\pgfpathclose%
\pgfusepath{fill}%
\end{pgfscope}%
\begin{pgfscope}%
\pgfpathrectangle{\pgfqpoint{2.500000in}{0.625000in}}{\pgfqpoint{15.500000in}{3.775000in}}%
\pgfusepath{clip}%
\pgfsetbuttcap%
\pgfsetmiterjoin%
\definecolor{currentfill}{rgb}{0.121569,0.466667,0.705882}%
\pgfsetfillcolor{currentfill}%
\pgfsetlinewidth{0.000000pt}%
\definecolor{currentstroke}{rgb}{0.000000,0.000000,0.000000}%
\pgfsetstrokecolor{currentstroke}%
\pgfsetstrokeopacity{0.000000}%
\pgfsetdash{}{0pt}%
\pgfpathmoveto{\pgfqpoint{5.795693in}{2.565994in}}%
\pgfpathlineto{\pgfqpoint{5.820371in}{2.565994in}}%
\pgfpathlineto{\pgfqpoint{5.820371in}{2.596169in}}%
\pgfpathlineto{\pgfqpoint{5.795693in}{2.596169in}}%
\pgfpathclose%
\pgfusepath{fill}%
\end{pgfscope}%
\begin{pgfscope}%
\pgfpathrectangle{\pgfqpoint{2.500000in}{0.625000in}}{\pgfqpoint{15.500000in}{3.775000in}}%
\pgfusepath{clip}%
\pgfsetbuttcap%
\pgfsetmiterjoin%
\definecolor{currentfill}{rgb}{0.121569,0.466667,0.705882}%
\pgfsetfillcolor{currentfill}%
\pgfsetlinewidth{0.000000pt}%
\definecolor{currentstroke}{rgb}{0.000000,0.000000,0.000000}%
\pgfsetstrokecolor{currentstroke}%
\pgfsetstrokeopacity{0.000000}%
\pgfsetdash{}{0pt}%
\pgfpathmoveto{\pgfqpoint{5.919081in}{2.565994in}}%
\pgfpathlineto{\pgfqpoint{5.943759in}{2.565994in}}%
\pgfpathlineto{\pgfqpoint{5.943759in}{2.587940in}}%
\pgfpathlineto{\pgfqpoint{5.919081in}{2.587940in}}%
\pgfpathclose%
\pgfusepath{fill}%
\end{pgfscope}%
\begin{pgfscope}%
\pgfpathrectangle{\pgfqpoint{2.500000in}{0.625000in}}{\pgfqpoint{15.500000in}{3.775000in}}%
\pgfusepath{clip}%
\pgfsetbuttcap%
\pgfsetmiterjoin%
\definecolor{currentfill}{rgb}{0.121569,0.466667,0.705882}%
\pgfsetfillcolor{currentfill}%
\pgfsetlinewidth{0.000000pt}%
\definecolor{currentstroke}{rgb}{0.000000,0.000000,0.000000}%
\pgfsetstrokecolor{currentstroke}%
\pgfsetstrokeopacity{0.000000}%
\pgfsetdash{}{0pt}%
\pgfpathmoveto{\pgfqpoint{6.042469in}{2.565994in}}%
\pgfpathlineto{\pgfqpoint{6.067147in}{2.565994in}}%
\pgfpathlineto{\pgfqpoint{6.067147in}{0.991362in}}%
\pgfpathlineto{\pgfqpoint{6.042469in}{0.991362in}}%
\pgfpathclose%
\pgfusepath{fill}%
\end{pgfscope}%
\begin{pgfscope}%
\pgfpathrectangle{\pgfqpoint{2.500000in}{0.625000in}}{\pgfqpoint{15.500000in}{3.775000in}}%
\pgfusepath{clip}%
\pgfsetbuttcap%
\pgfsetmiterjoin%
\definecolor{currentfill}{rgb}{0.121569,0.466667,0.705882}%
\pgfsetfillcolor{currentfill}%
\pgfsetlinewidth{0.000000pt}%
\definecolor{currentstroke}{rgb}{0.000000,0.000000,0.000000}%
\pgfsetstrokecolor{currentstroke}%
\pgfsetstrokeopacity{0.000000}%
\pgfsetdash{}{0pt}%
\pgfpathmoveto{\pgfqpoint{6.165857in}{2.565994in}}%
\pgfpathlineto{\pgfqpoint{6.190535in}{2.565994in}}%
\pgfpathlineto{\pgfqpoint{6.190535in}{2.565994in}}%
\pgfpathlineto{\pgfqpoint{6.165857in}{2.565994in}}%
\pgfpathclose%
\pgfusepath{fill}%
\end{pgfscope}%
\begin{pgfscope}%
\pgfpathrectangle{\pgfqpoint{2.500000in}{0.625000in}}{\pgfqpoint{15.500000in}{3.775000in}}%
\pgfusepath{clip}%
\pgfsetbuttcap%
\pgfsetmiterjoin%
\definecolor{currentfill}{rgb}{0.121569,0.466667,0.705882}%
\pgfsetfillcolor{currentfill}%
\pgfsetlinewidth{0.000000pt}%
\definecolor{currentstroke}{rgb}{0.000000,0.000000,0.000000}%
\pgfsetstrokecolor{currentstroke}%
\pgfsetstrokeopacity{0.000000}%
\pgfsetdash{}{0pt}%
\pgfpathmoveto{\pgfqpoint{6.289245in}{2.565994in}}%
\pgfpathlineto{\pgfqpoint{6.313923in}{2.565994in}}%
\pgfpathlineto{\pgfqpoint{6.313923in}{2.768995in}}%
\pgfpathlineto{\pgfqpoint{6.289245in}{2.768995in}}%
\pgfpathclose%
\pgfusepath{fill}%
\end{pgfscope}%
\begin{pgfscope}%
\pgfpathrectangle{\pgfqpoint{2.500000in}{0.625000in}}{\pgfqpoint{15.500000in}{3.775000in}}%
\pgfusepath{clip}%
\pgfsetbuttcap%
\pgfsetmiterjoin%
\definecolor{currentfill}{rgb}{0.121569,0.466667,0.705882}%
\pgfsetfillcolor{currentfill}%
\pgfsetlinewidth{0.000000pt}%
\definecolor{currentstroke}{rgb}{0.000000,0.000000,0.000000}%
\pgfsetstrokecolor{currentstroke}%
\pgfsetstrokeopacity{0.000000}%
\pgfsetdash{}{0pt}%
\pgfpathmoveto{\pgfqpoint{6.412633in}{2.565994in}}%
\pgfpathlineto{\pgfqpoint{6.437311in}{2.565994in}}%
\pgfpathlineto{\pgfqpoint{6.437311in}{2.563250in}}%
\pgfpathlineto{\pgfqpoint{6.412633in}{2.563250in}}%
\pgfpathclose%
\pgfusepath{fill}%
\end{pgfscope}%
\begin{pgfscope}%
\pgfpathrectangle{\pgfqpoint{2.500000in}{0.625000in}}{\pgfqpoint{15.500000in}{3.775000in}}%
\pgfusepath{clip}%
\pgfsetbuttcap%
\pgfsetmiterjoin%
\definecolor{currentfill}{rgb}{0.121569,0.466667,0.705882}%
\pgfsetfillcolor{currentfill}%
\pgfsetlinewidth{0.000000pt}%
\definecolor{currentstroke}{rgb}{0.000000,0.000000,0.000000}%
\pgfsetstrokecolor{currentstroke}%
\pgfsetstrokeopacity{0.000000}%
\pgfsetdash{}{0pt}%
\pgfpathmoveto{\pgfqpoint{6.536021in}{2.565994in}}%
\pgfpathlineto{\pgfqpoint{6.560699in}{2.565994in}}%
\pgfpathlineto{\pgfqpoint{6.560699in}{2.461750in}}%
\pgfpathlineto{\pgfqpoint{6.536021in}{2.461750in}}%
\pgfpathclose%
\pgfusepath{fill}%
\end{pgfscope}%
\begin{pgfscope}%
\pgfpathrectangle{\pgfqpoint{2.500000in}{0.625000in}}{\pgfqpoint{15.500000in}{3.775000in}}%
\pgfusepath{clip}%
\pgfsetbuttcap%
\pgfsetmiterjoin%
\definecolor{currentfill}{rgb}{0.121569,0.466667,0.705882}%
\pgfsetfillcolor{currentfill}%
\pgfsetlinewidth{0.000000pt}%
\definecolor{currentstroke}{rgb}{0.000000,0.000000,0.000000}%
\pgfsetstrokecolor{currentstroke}%
\pgfsetstrokeopacity{0.000000}%
\pgfsetdash{}{0pt}%
\pgfpathmoveto{\pgfqpoint{6.659409in}{2.565994in}}%
\pgfpathlineto{\pgfqpoint{6.684087in}{2.565994in}}%
\pgfpathlineto{\pgfqpoint{6.684087in}{2.565994in}}%
\pgfpathlineto{\pgfqpoint{6.659409in}{2.565994in}}%
\pgfpathclose%
\pgfusepath{fill}%
\end{pgfscope}%
\begin{pgfscope}%
\pgfpathrectangle{\pgfqpoint{2.500000in}{0.625000in}}{\pgfqpoint{15.500000in}{3.775000in}}%
\pgfusepath{clip}%
\pgfsetbuttcap%
\pgfsetmiterjoin%
\definecolor{currentfill}{rgb}{0.121569,0.466667,0.705882}%
\pgfsetfillcolor{currentfill}%
\pgfsetlinewidth{0.000000pt}%
\definecolor{currentstroke}{rgb}{0.000000,0.000000,0.000000}%
\pgfsetstrokecolor{currentstroke}%
\pgfsetstrokeopacity{0.000000}%
\pgfsetdash{}{0pt}%
\pgfpathmoveto{\pgfqpoint{6.782797in}{2.565994in}}%
\pgfpathlineto{\pgfqpoint{6.807475in}{2.565994in}}%
\pgfpathlineto{\pgfqpoint{6.807475in}{1.594880in}}%
\pgfpathlineto{\pgfqpoint{6.782797in}{1.594880in}}%
\pgfpathclose%
\pgfusepath{fill}%
\end{pgfscope}%
\begin{pgfscope}%
\pgfpathrectangle{\pgfqpoint{2.500000in}{0.625000in}}{\pgfqpoint{15.500000in}{3.775000in}}%
\pgfusepath{clip}%
\pgfsetbuttcap%
\pgfsetmiterjoin%
\definecolor{currentfill}{rgb}{0.121569,0.466667,0.705882}%
\pgfsetfillcolor{currentfill}%
\pgfsetlinewidth{0.000000pt}%
\definecolor{currentstroke}{rgb}{0.000000,0.000000,0.000000}%
\pgfsetstrokecolor{currentstroke}%
\pgfsetstrokeopacity{0.000000}%
\pgfsetdash{}{0pt}%
\pgfpathmoveto{\pgfqpoint{6.906185in}{2.565994in}}%
\pgfpathlineto{\pgfqpoint{6.930863in}{2.565994in}}%
\pgfpathlineto{\pgfqpoint{6.930863in}{2.223086in}}%
\pgfpathlineto{\pgfqpoint{6.906185in}{2.223086in}}%
\pgfpathclose%
\pgfusepath{fill}%
\end{pgfscope}%
\begin{pgfscope}%
\pgfpathrectangle{\pgfqpoint{2.500000in}{0.625000in}}{\pgfqpoint{15.500000in}{3.775000in}}%
\pgfusepath{clip}%
\pgfsetbuttcap%
\pgfsetmiterjoin%
\definecolor{currentfill}{rgb}{0.121569,0.466667,0.705882}%
\pgfsetfillcolor{currentfill}%
\pgfsetlinewidth{0.000000pt}%
\definecolor{currentstroke}{rgb}{0.000000,0.000000,0.000000}%
\pgfsetstrokecolor{currentstroke}%
\pgfsetstrokeopacity{0.000000}%
\pgfsetdash{}{0pt}%
\pgfpathmoveto{\pgfqpoint{7.029573in}{2.565994in}}%
\pgfpathlineto{\pgfqpoint{7.054251in}{2.565994in}}%
\pgfpathlineto{\pgfqpoint{7.054251in}{2.568737in}}%
\pgfpathlineto{\pgfqpoint{7.029573in}{2.568737in}}%
\pgfpathclose%
\pgfusepath{fill}%
\end{pgfscope}%
\begin{pgfscope}%
\pgfpathrectangle{\pgfqpoint{2.500000in}{0.625000in}}{\pgfqpoint{15.500000in}{3.775000in}}%
\pgfusepath{clip}%
\pgfsetbuttcap%
\pgfsetmiterjoin%
\definecolor{currentfill}{rgb}{0.121569,0.466667,0.705882}%
\pgfsetfillcolor{currentfill}%
\pgfsetlinewidth{0.000000pt}%
\definecolor{currentstroke}{rgb}{0.000000,0.000000,0.000000}%
\pgfsetstrokecolor{currentstroke}%
\pgfsetstrokeopacity{0.000000}%
\pgfsetdash{}{0pt}%
\pgfpathmoveto{\pgfqpoint{7.152961in}{2.565994in}}%
\pgfpathlineto{\pgfqpoint{7.177639in}{2.565994in}}%
\pgfpathlineto{\pgfqpoint{7.177639in}{2.565994in}}%
\pgfpathlineto{\pgfqpoint{7.152961in}{2.565994in}}%
\pgfpathclose%
\pgfusepath{fill}%
\end{pgfscope}%
\begin{pgfscope}%
\pgfpathrectangle{\pgfqpoint{2.500000in}{0.625000in}}{\pgfqpoint{15.500000in}{3.775000in}}%
\pgfusepath{clip}%
\pgfsetbuttcap%
\pgfsetmiterjoin%
\definecolor{currentfill}{rgb}{0.121569,0.466667,0.705882}%
\pgfsetfillcolor{currentfill}%
\pgfsetlinewidth{0.000000pt}%
\definecolor{currentstroke}{rgb}{0.000000,0.000000,0.000000}%
\pgfsetstrokecolor{currentstroke}%
\pgfsetstrokeopacity{0.000000}%
\pgfsetdash{}{0pt}%
\pgfpathmoveto{\pgfqpoint{7.276349in}{2.565994in}}%
\pgfpathlineto{\pgfqpoint{7.301027in}{2.565994in}}%
\pgfpathlineto{\pgfqpoint{7.301027in}{2.571480in}}%
\pgfpathlineto{\pgfqpoint{7.276349in}{2.571480in}}%
\pgfpathclose%
\pgfusepath{fill}%
\end{pgfscope}%
\begin{pgfscope}%
\pgfpathrectangle{\pgfqpoint{2.500000in}{0.625000in}}{\pgfqpoint{15.500000in}{3.775000in}}%
\pgfusepath{clip}%
\pgfsetbuttcap%
\pgfsetmiterjoin%
\definecolor{currentfill}{rgb}{0.121569,0.466667,0.705882}%
\pgfsetfillcolor{currentfill}%
\pgfsetlinewidth{0.000000pt}%
\definecolor{currentstroke}{rgb}{0.000000,0.000000,0.000000}%
\pgfsetstrokecolor{currentstroke}%
\pgfsetstrokeopacity{0.000000}%
\pgfsetdash{}{0pt}%
\pgfpathmoveto{\pgfqpoint{7.399737in}{2.565994in}}%
\pgfpathlineto{\pgfqpoint{7.424415in}{2.565994in}}%
\pgfpathlineto{\pgfqpoint{7.424415in}{2.576967in}}%
\pgfpathlineto{\pgfqpoint{7.399737in}{2.576967in}}%
\pgfpathclose%
\pgfusepath{fill}%
\end{pgfscope}%
\begin{pgfscope}%
\pgfpathrectangle{\pgfqpoint{2.500000in}{0.625000in}}{\pgfqpoint{15.500000in}{3.775000in}}%
\pgfusepath{clip}%
\pgfsetbuttcap%
\pgfsetmiterjoin%
\definecolor{currentfill}{rgb}{0.121569,0.466667,0.705882}%
\pgfsetfillcolor{currentfill}%
\pgfsetlinewidth{0.000000pt}%
\definecolor{currentstroke}{rgb}{0.000000,0.000000,0.000000}%
\pgfsetstrokecolor{currentstroke}%
\pgfsetstrokeopacity{0.000000}%
\pgfsetdash{}{0pt}%
\pgfpathmoveto{\pgfqpoint{7.523125in}{2.565994in}}%
\pgfpathlineto{\pgfqpoint{7.547803in}{2.565994in}}%
\pgfpathlineto{\pgfqpoint{7.547803in}{2.568737in}}%
\pgfpathlineto{\pgfqpoint{7.523125in}{2.568737in}}%
\pgfpathclose%
\pgfusepath{fill}%
\end{pgfscope}%
\begin{pgfscope}%
\pgfpathrectangle{\pgfqpoint{2.500000in}{0.625000in}}{\pgfqpoint{15.500000in}{3.775000in}}%
\pgfusepath{clip}%
\pgfsetbuttcap%
\pgfsetmiterjoin%
\definecolor{currentfill}{rgb}{0.121569,0.466667,0.705882}%
\pgfsetfillcolor{currentfill}%
\pgfsetlinewidth{0.000000pt}%
\definecolor{currentstroke}{rgb}{0.000000,0.000000,0.000000}%
\pgfsetstrokecolor{currentstroke}%
\pgfsetstrokeopacity{0.000000}%
\pgfsetdash{}{0pt}%
\pgfpathmoveto{\pgfqpoint{7.646513in}{2.565994in}}%
\pgfpathlineto{\pgfqpoint{7.671191in}{2.565994in}}%
\pgfpathlineto{\pgfqpoint{7.671191in}{2.585196in}}%
\pgfpathlineto{\pgfqpoint{7.646513in}{2.585196in}}%
\pgfpathclose%
\pgfusepath{fill}%
\end{pgfscope}%
\begin{pgfscope}%
\pgfpathrectangle{\pgfqpoint{2.500000in}{0.625000in}}{\pgfqpoint{15.500000in}{3.775000in}}%
\pgfusepath{clip}%
\pgfsetbuttcap%
\pgfsetmiterjoin%
\definecolor{currentfill}{rgb}{0.121569,0.466667,0.705882}%
\pgfsetfillcolor{currentfill}%
\pgfsetlinewidth{0.000000pt}%
\definecolor{currentstroke}{rgb}{0.000000,0.000000,0.000000}%
\pgfsetstrokecolor{currentstroke}%
\pgfsetstrokeopacity{0.000000}%
\pgfsetdash{}{0pt}%
\pgfpathmoveto{\pgfqpoint{7.769901in}{2.565994in}}%
\pgfpathlineto{\pgfqpoint{7.794579in}{2.565994in}}%
\pgfpathlineto{\pgfqpoint{7.794579in}{2.500155in}}%
\pgfpathlineto{\pgfqpoint{7.769901in}{2.500155in}}%
\pgfpathclose%
\pgfusepath{fill}%
\end{pgfscope}%
\begin{pgfscope}%
\pgfpathrectangle{\pgfqpoint{2.500000in}{0.625000in}}{\pgfqpoint{15.500000in}{3.775000in}}%
\pgfusepath{clip}%
\pgfsetbuttcap%
\pgfsetmiterjoin%
\definecolor{currentfill}{rgb}{0.121569,0.466667,0.705882}%
\pgfsetfillcolor{currentfill}%
\pgfsetlinewidth{0.000000pt}%
\definecolor{currentstroke}{rgb}{0.000000,0.000000,0.000000}%
\pgfsetstrokecolor{currentstroke}%
\pgfsetstrokeopacity{0.000000}%
\pgfsetdash{}{0pt}%
\pgfpathmoveto{\pgfqpoint{7.893289in}{2.565994in}}%
\pgfpathlineto{\pgfqpoint{7.917967in}{2.565994in}}%
\pgfpathlineto{\pgfqpoint{7.917967in}{2.631832in}}%
\pgfpathlineto{\pgfqpoint{7.893289in}{2.631832in}}%
\pgfpathclose%
\pgfusepath{fill}%
\end{pgfscope}%
\begin{pgfscope}%
\pgfpathrectangle{\pgfqpoint{2.500000in}{0.625000in}}{\pgfqpoint{15.500000in}{3.775000in}}%
\pgfusepath{clip}%
\pgfsetbuttcap%
\pgfsetmiterjoin%
\definecolor{currentfill}{rgb}{0.121569,0.466667,0.705882}%
\pgfsetfillcolor{currentfill}%
\pgfsetlinewidth{0.000000pt}%
\definecolor{currentstroke}{rgb}{0.000000,0.000000,0.000000}%
\pgfsetstrokecolor{currentstroke}%
\pgfsetstrokeopacity{0.000000}%
\pgfsetdash{}{0pt}%
\pgfpathmoveto{\pgfqpoint{8.016677in}{2.565994in}}%
\pgfpathlineto{\pgfqpoint{8.041355in}{2.565994in}}%
\pgfpathlineto{\pgfqpoint{8.041355in}{2.579710in}}%
\pgfpathlineto{\pgfqpoint{8.016677in}{2.579710in}}%
\pgfpathclose%
\pgfusepath{fill}%
\end{pgfscope}%
\begin{pgfscope}%
\pgfpathrectangle{\pgfqpoint{2.500000in}{0.625000in}}{\pgfqpoint{15.500000in}{3.775000in}}%
\pgfusepath{clip}%
\pgfsetbuttcap%
\pgfsetmiterjoin%
\definecolor{currentfill}{rgb}{0.121569,0.466667,0.705882}%
\pgfsetfillcolor{currentfill}%
\pgfsetlinewidth{0.000000pt}%
\definecolor{currentstroke}{rgb}{0.000000,0.000000,0.000000}%
\pgfsetstrokecolor{currentstroke}%
\pgfsetstrokeopacity{0.000000}%
\pgfsetdash{}{0pt}%
\pgfpathmoveto{\pgfqpoint{8.140065in}{2.565994in}}%
\pgfpathlineto{\pgfqpoint{8.164743in}{2.565994in}}%
\pgfpathlineto{\pgfqpoint{8.164743in}{2.565994in}}%
\pgfpathlineto{\pgfqpoint{8.140065in}{2.565994in}}%
\pgfpathclose%
\pgfusepath{fill}%
\end{pgfscope}%
\begin{pgfscope}%
\pgfpathrectangle{\pgfqpoint{2.500000in}{0.625000in}}{\pgfqpoint{15.500000in}{3.775000in}}%
\pgfusepath{clip}%
\pgfsetbuttcap%
\pgfsetmiterjoin%
\definecolor{currentfill}{rgb}{0.121569,0.466667,0.705882}%
\pgfsetfillcolor{currentfill}%
\pgfsetlinewidth{0.000000pt}%
\definecolor{currentstroke}{rgb}{0.000000,0.000000,0.000000}%
\pgfsetstrokecolor{currentstroke}%
\pgfsetstrokeopacity{0.000000}%
\pgfsetdash{}{0pt}%
\pgfpathmoveto{\pgfqpoint{8.263453in}{2.565994in}}%
\pgfpathlineto{\pgfqpoint{8.288131in}{2.565994in}}%
\pgfpathlineto{\pgfqpoint{8.288131in}{2.576967in}}%
\pgfpathlineto{\pgfqpoint{8.263453in}{2.576967in}}%
\pgfpathclose%
\pgfusepath{fill}%
\end{pgfscope}%
\begin{pgfscope}%
\pgfpathrectangle{\pgfqpoint{2.500000in}{0.625000in}}{\pgfqpoint{15.500000in}{3.775000in}}%
\pgfusepath{clip}%
\pgfsetbuttcap%
\pgfsetmiterjoin%
\definecolor{currentfill}{rgb}{0.121569,0.466667,0.705882}%
\pgfsetfillcolor{currentfill}%
\pgfsetlinewidth{0.000000pt}%
\definecolor{currentstroke}{rgb}{0.000000,0.000000,0.000000}%
\pgfsetstrokecolor{currentstroke}%
\pgfsetstrokeopacity{0.000000}%
\pgfsetdash{}{0pt}%
\pgfpathmoveto{\pgfqpoint{8.386841in}{2.565994in}}%
\pgfpathlineto{\pgfqpoint{8.411519in}{2.565994in}}%
\pgfpathlineto{\pgfqpoint{8.411519in}{2.565994in}}%
\pgfpathlineto{\pgfqpoint{8.386841in}{2.565994in}}%
\pgfpathclose%
\pgfusepath{fill}%
\end{pgfscope}%
\begin{pgfscope}%
\pgfpathrectangle{\pgfqpoint{2.500000in}{0.625000in}}{\pgfqpoint{15.500000in}{3.775000in}}%
\pgfusepath{clip}%
\pgfsetbuttcap%
\pgfsetmiterjoin%
\definecolor{currentfill}{rgb}{0.121569,0.466667,0.705882}%
\pgfsetfillcolor{currentfill}%
\pgfsetlinewidth{0.000000pt}%
\definecolor{currentstroke}{rgb}{0.000000,0.000000,0.000000}%
\pgfsetstrokecolor{currentstroke}%
\pgfsetstrokeopacity{0.000000}%
\pgfsetdash{}{0pt}%
\pgfpathmoveto{\pgfqpoint{8.510229in}{2.565994in}}%
\pgfpathlineto{\pgfqpoint{8.534907in}{2.565994in}}%
\pgfpathlineto{\pgfqpoint{8.534907in}{2.574223in}}%
\pgfpathlineto{\pgfqpoint{8.510229in}{2.574223in}}%
\pgfpathclose%
\pgfusepath{fill}%
\end{pgfscope}%
\begin{pgfscope}%
\pgfpathrectangle{\pgfqpoint{2.500000in}{0.625000in}}{\pgfqpoint{15.500000in}{3.775000in}}%
\pgfusepath{clip}%
\pgfsetbuttcap%
\pgfsetmiterjoin%
\definecolor{currentfill}{rgb}{0.121569,0.466667,0.705882}%
\pgfsetfillcolor{currentfill}%
\pgfsetlinewidth{0.000000pt}%
\definecolor{currentstroke}{rgb}{0.000000,0.000000,0.000000}%
\pgfsetstrokecolor{currentstroke}%
\pgfsetstrokeopacity{0.000000}%
\pgfsetdash{}{0pt}%
\pgfpathmoveto{\pgfqpoint{8.633617in}{2.565994in}}%
\pgfpathlineto{\pgfqpoint{8.658295in}{2.565994in}}%
\pgfpathlineto{\pgfqpoint{8.658295in}{2.565994in}}%
\pgfpathlineto{\pgfqpoint{8.633617in}{2.565994in}}%
\pgfpathclose%
\pgfusepath{fill}%
\end{pgfscope}%
\begin{pgfscope}%
\pgfpathrectangle{\pgfqpoint{2.500000in}{0.625000in}}{\pgfqpoint{15.500000in}{3.775000in}}%
\pgfusepath{clip}%
\pgfsetbuttcap%
\pgfsetmiterjoin%
\definecolor{currentfill}{rgb}{0.121569,0.466667,0.705882}%
\pgfsetfillcolor{currentfill}%
\pgfsetlinewidth{0.000000pt}%
\definecolor{currentstroke}{rgb}{0.000000,0.000000,0.000000}%
\pgfsetstrokecolor{currentstroke}%
\pgfsetstrokeopacity{0.000000}%
\pgfsetdash{}{0pt}%
\pgfpathmoveto{\pgfqpoint{8.757005in}{2.565994in}}%
\pgfpathlineto{\pgfqpoint{8.781683in}{2.565994in}}%
\pgfpathlineto{\pgfqpoint{8.781683in}{2.568737in}}%
\pgfpathlineto{\pgfqpoint{8.757005in}{2.568737in}}%
\pgfpathclose%
\pgfusepath{fill}%
\end{pgfscope}%
\begin{pgfscope}%
\pgfpathrectangle{\pgfqpoint{2.500000in}{0.625000in}}{\pgfqpoint{15.500000in}{3.775000in}}%
\pgfusepath{clip}%
\pgfsetbuttcap%
\pgfsetmiterjoin%
\definecolor{currentfill}{rgb}{0.121569,0.466667,0.705882}%
\pgfsetfillcolor{currentfill}%
\pgfsetlinewidth{0.000000pt}%
\definecolor{currentstroke}{rgb}{0.000000,0.000000,0.000000}%
\pgfsetstrokecolor{currentstroke}%
\pgfsetstrokeopacity{0.000000}%
\pgfsetdash{}{0pt}%
\pgfpathmoveto{\pgfqpoint{8.880393in}{2.565994in}}%
\pgfpathlineto{\pgfqpoint{8.905071in}{2.565994in}}%
\pgfpathlineto{\pgfqpoint{8.905071in}{2.571480in}}%
\pgfpathlineto{\pgfqpoint{8.880393in}{2.571480in}}%
\pgfpathclose%
\pgfusepath{fill}%
\end{pgfscope}%
\begin{pgfscope}%
\pgfpathrectangle{\pgfqpoint{2.500000in}{0.625000in}}{\pgfqpoint{15.500000in}{3.775000in}}%
\pgfusepath{clip}%
\pgfsetbuttcap%
\pgfsetmiterjoin%
\definecolor{currentfill}{rgb}{0.121569,0.466667,0.705882}%
\pgfsetfillcolor{currentfill}%
\pgfsetlinewidth{0.000000pt}%
\definecolor{currentstroke}{rgb}{0.000000,0.000000,0.000000}%
\pgfsetstrokecolor{currentstroke}%
\pgfsetstrokeopacity{0.000000}%
\pgfsetdash{}{0pt}%
\pgfpathmoveto{\pgfqpoint{9.003781in}{2.565994in}}%
\pgfpathlineto{\pgfqpoint{9.028459in}{2.565994in}}%
\pgfpathlineto{\pgfqpoint{9.028459in}{0.903578in}}%
\pgfpathlineto{\pgfqpoint{9.003781in}{0.903578in}}%
\pgfpathclose%
\pgfusepath{fill}%
\end{pgfscope}%
\begin{pgfscope}%
\pgfpathrectangle{\pgfqpoint{2.500000in}{0.625000in}}{\pgfqpoint{15.500000in}{3.775000in}}%
\pgfusepath{clip}%
\pgfsetbuttcap%
\pgfsetmiterjoin%
\definecolor{currentfill}{rgb}{0.121569,0.466667,0.705882}%
\pgfsetfillcolor{currentfill}%
\pgfsetlinewidth{0.000000pt}%
\definecolor{currentstroke}{rgb}{0.000000,0.000000,0.000000}%
\pgfsetstrokecolor{currentstroke}%
\pgfsetstrokeopacity{0.000000}%
\pgfsetdash{}{0pt}%
\pgfpathmoveto{\pgfqpoint{9.127169in}{2.565994in}}%
\pgfpathlineto{\pgfqpoint{9.151847in}{2.565994in}}%
\pgfpathlineto{\pgfqpoint{9.151847in}{0.914551in}}%
\pgfpathlineto{\pgfqpoint{9.127169in}{0.914551in}}%
\pgfpathclose%
\pgfusepath{fill}%
\end{pgfscope}%
\begin{pgfscope}%
\pgfpathrectangle{\pgfqpoint{2.500000in}{0.625000in}}{\pgfqpoint{15.500000in}{3.775000in}}%
\pgfusepath{clip}%
\pgfsetbuttcap%
\pgfsetmiterjoin%
\definecolor{currentfill}{rgb}{0.121569,0.466667,0.705882}%
\pgfsetfillcolor{currentfill}%
\pgfsetlinewidth{0.000000pt}%
\definecolor{currentstroke}{rgb}{0.000000,0.000000,0.000000}%
\pgfsetstrokecolor{currentstroke}%
\pgfsetstrokeopacity{0.000000}%
\pgfsetdash{}{0pt}%
\pgfpathmoveto{\pgfqpoint{9.250557in}{2.565994in}}%
\pgfpathlineto{\pgfqpoint{9.275235in}{2.565994in}}%
\pgfpathlineto{\pgfqpoint{9.275235in}{0.906321in}}%
\pgfpathlineto{\pgfqpoint{9.250557in}{0.906321in}}%
\pgfpathclose%
\pgfusepath{fill}%
\end{pgfscope}%
\begin{pgfscope}%
\pgfpathrectangle{\pgfqpoint{2.500000in}{0.625000in}}{\pgfqpoint{15.500000in}{3.775000in}}%
\pgfusepath{clip}%
\pgfsetbuttcap%
\pgfsetmiterjoin%
\definecolor{currentfill}{rgb}{0.121569,0.466667,0.705882}%
\pgfsetfillcolor{currentfill}%
\pgfsetlinewidth{0.000000pt}%
\definecolor{currentstroke}{rgb}{0.000000,0.000000,0.000000}%
\pgfsetstrokecolor{currentstroke}%
\pgfsetstrokeopacity{0.000000}%
\pgfsetdash{}{0pt}%
\pgfpathmoveto{\pgfqpoint{9.373945in}{2.565994in}}%
\pgfpathlineto{\pgfqpoint{9.398623in}{2.565994in}}%
\pgfpathlineto{\pgfqpoint{9.398623in}{0.796591in}}%
\pgfpathlineto{\pgfqpoint{9.373945in}{0.796591in}}%
\pgfpathclose%
\pgfusepath{fill}%
\end{pgfscope}%
\begin{pgfscope}%
\pgfpathrectangle{\pgfqpoint{2.500000in}{0.625000in}}{\pgfqpoint{15.500000in}{3.775000in}}%
\pgfusepath{clip}%
\pgfsetbuttcap%
\pgfsetmiterjoin%
\definecolor{currentfill}{rgb}{0.121569,0.466667,0.705882}%
\pgfsetfillcolor{currentfill}%
\pgfsetlinewidth{0.000000pt}%
\definecolor{currentstroke}{rgb}{0.000000,0.000000,0.000000}%
\pgfsetstrokecolor{currentstroke}%
\pgfsetstrokeopacity{0.000000}%
\pgfsetdash{}{0pt}%
\pgfpathmoveto{\pgfqpoint{9.497333in}{2.565994in}}%
\pgfpathlineto{\pgfqpoint{9.522011in}{2.565994in}}%
\pgfpathlineto{\pgfqpoint{9.522011in}{1.400108in}}%
\pgfpathlineto{\pgfqpoint{9.497333in}{1.400108in}}%
\pgfpathclose%
\pgfusepath{fill}%
\end{pgfscope}%
\begin{pgfscope}%
\pgfpathrectangle{\pgfqpoint{2.500000in}{0.625000in}}{\pgfqpoint{15.500000in}{3.775000in}}%
\pgfusepath{clip}%
\pgfsetbuttcap%
\pgfsetmiterjoin%
\definecolor{currentfill}{rgb}{0.121569,0.466667,0.705882}%
\pgfsetfillcolor{currentfill}%
\pgfsetlinewidth{0.000000pt}%
\definecolor{currentstroke}{rgb}{0.000000,0.000000,0.000000}%
\pgfsetstrokecolor{currentstroke}%
\pgfsetstrokeopacity{0.000000}%
\pgfsetdash{}{0pt}%
\pgfpathmoveto{\pgfqpoint{9.620721in}{2.565994in}}%
\pgfpathlineto{\pgfqpoint{9.645399in}{2.565994in}}%
\pgfpathlineto{\pgfqpoint{9.645399in}{0.878889in}}%
\pgfpathlineto{\pgfqpoint{9.620721in}{0.878889in}}%
\pgfpathclose%
\pgfusepath{fill}%
\end{pgfscope}%
\begin{pgfscope}%
\pgfpathrectangle{\pgfqpoint{2.500000in}{0.625000in}}{\pgfqpoint{15.500000in}{3.775000in}}%
\pgfusepath{clip}%
\pgfsetbuttcap%
\pgfsetmiterjoin%
\definecolor{currentfill}{rgb}{0.121569,0.466667,0.705882}%
\pgfsetfillcolor{currentfill}%
\pgfsetlinewidth{0.000000pt}%
\definecolor{currentstroke}{rgb}{0.000000,0.000000,0.000000}%
\pgfsetstrokecolor{currentstroke}%
\pgfsetstrokeopacity{0.000000}%
\pgfsetdash{}{0pt}%
\pgfpathmoveto{\pgfqpoint{9.744109in}{2.565994in}}%
\pgfpathlineto{\pgfqpoint{9.768787in}{2.565994in}}%
\pgfpathlineto{\pgfqpoint{9.768787in}{2.571480in}}%
\pgfpathlineto{\pgfqpoint{9.744109in}{2.571480in}}%
\pgfpathclose%
\pgfusepath{fill}%
\end{pgfscope}%
\begin{pgfscope}%
\pgfpathrectangle{\pgfqpoint{2.500000in}{0.625000in}}{\pgfqpoint{15.500000in}{3.775000in}}%
\pgfusepath{clip}%
\pgfsetbuttcap%
\pgfsetmiterjoin%
\definecolor{currentfill}{rgb}{0.121569,0.466667,0.705882}%
\pgfsetfillcolor{currentfill}%
\pgfsetlinewidth{0.000000pt}%
\definecolor{currentstroke}{rgb}{0.000000,0.000000,0.000000}%
\pgfsetstrokecolor{currentstroke}%
\pgfsetstrokeopacity{0.000000}%
\pgfsetdash{}{0pt}%
\pgfpathmoveto{\pgfqpoint{9.867497in}{2.565994in}}%
\pgfpathlineto{\pgfqpoint{9.892175in}{2.565994in}}%
\pgfpathlineto{\pgfqpoint{9.892175in}{2.571480in}}%
\pgfpathlineto{\pgfqpoint{9.867497in}{2.571480in}}%
\pgfpathclose%
\pgfusepath{fill}%
\end{pgfscope}%
\begin{pgfscope}%
\pgfpathrectangle{\pgfqpoint{2.500000in}{0.625000in}}{\pgfqpoint{15.500000in}{3.775000in}}%
\pgfusepath{clip}%
\pgfsetbuttcap%
\pgfsetmiterjoin%
\definecolor{currentfill}{rgb}{0.121569,0.466667,0.705882}%
\pgfsetfillcolor{currentfill}%
\pgfsetlinewidth{0.000000pt}%
\definecolor{currentstroke}{rgb}{0.000000,0.000000,0.000000}%
\pgfsetstrokecolor{currentstroke}%
\pgfsetstrokeopacity{0.000000}%
\pgfsetdash{}{0pt}%
\pgfpathmoveto{\pgfqpoint{9.990885in}{2.565994in}}%
\pgfpathlineto{\pgfqpoint{10.015563in}{2.565994in}}%
\pgfpathlineto{\pgfqpoint{10.015563in}{2.574223in}}%
\pgfpathlineto{\pgfqpoint{9.990885in}{2.574223in}}%
\pgfpathclose%
\pgfusepath{fill}%
\end{pgfscope}%
\begin{pgfscope}%
\pgfpathrectangle{\pgfqpoint{2.500000in}{0.625000in}}{\pgfqpoint{15.500000in}{3.775000in}}%
\pgfusepath{clip}%
\pgfsetbuttcap%
\pgfsetmiterjoin%
\definecolor{currentfill}{rgb}{0.121569,0.466667,0.705882}%
\pgfsetfillcolor{currentfill}%
\pgfsetlinewidth{0.000000pt}%
\definecolor{currentstroke}{rgb}{0.000000,0.000000,0.000000}%
\pgfsetstrokecolor{currentstroke}%
\pgfsetstrokeopacity{0.000000}%
\pgfsetdash{}{0pt}%
\pgfpathmoveto{\pgfqpoint{10.114273in}{2.565994in}}%
\pgfpathlineto{\pgfqpoint{10.138951in}{2.565994in}}%
\pgfpathlineto{\pgfqpoint{10.138951in}{2.574223in}}%
\pgfpathlineto{\pgfqpoint{10.114273in}{2.574223in}}%
\pgfpathclose%
\pgfusepath{fill}%
\end{pgfscope}%
\begin{pgfscope}%
\pgfpathrectangle{\pgfqpoint{2.500000in}{0.625000in}}{\pgfqpoint{15.500000in}{3.775000in}}%
\pgfusepath{clip}%
\pgfsetbuttcap%
\pgfsetmiterjoin%
\definecolor{currentfill}{rgb}{0.121569,0.466667,0.705882}%
\pgfsetfillcolor{currentfill}%
\pgfsetlinewidth{0.000000pt}%
\definecolor{currentstroke}{rgb}{0.000000,0.000000,0.000000}%
\pgfsetstrokecolor{currentstroke}%
\pgfsetstrokeopacity{0.000000}%
\pgfsetdash{}{0pt}%
\pgfpathmoveto{\pgfqpoint{10.237661in}{2.565994in}}%
\pgfpathlineto{\pgfqpoint{10.262339in}{2.565994in}}%
\pgfpathlineto{\pgfqpoint{10.262339in}{2.146275in}}%
\pgfpathlineto{\pgfqpoint{10.237661in}{2.146275in}}%
\pgfpathclose%
\pgfusepath{fill}%
\end{pgfscope}%
\begin{pgfscope}%
\pgfpathrectangle{\pgfqpoint{2.500000in}{0.625000in}}{\pgfqpoint{15.500000in}{3.775000in}}%
\pgfusepath{clip}%
\pgfsetbuttcap%
\pgfsetmiterjoin%
\definecolor{currentfill}{rgb}{0.121569,0.466667,0.705882}%
\pgfsetfillcolor{currentfill}%
\pgfsetlinewidth{0.000000pt}%
\definecolor{currentstroke}{rgb}{0.000000,0.000000,0.000000}%
\pgfsetstrokecolor{currentstroke}%
\pgfsetstrokeopacity{0.000000}%
\pgfsetdash{}{0pt}%
\pgfpathmoveto{\pgfqpoint{10.361049in}{2.565994in}}%
\pgfpathlineto{\pgfqpoint{10.385727in}{2.565994in}}%
\pgfpathlineto{\pgfqpoint{10.385727in}{0.920038in}}%
\pgfpathlineto{\pgfqpoint{10.361049in}{0.920038in}}%
\pgfpathclose%
\pgfusepath{fill}%
\end{pgfscope}%
\begin{pgfscope}%
\pgfpathrectangle{\pgfqpoint{2.500000in}{0.625000in}}{\pgfqpoint{15.500000in}{3.775000in}}%
\pgfusepath{clip}%
\pgfsetbuttcap%
\pgfsetmiterjoin%
\definecolor{currentfill}{rgb}{0.121569,0.466667,0.705882}%
\pgfsetfillcolor{currentfill}%
\pgfsetlinewidth{0.000000pt}%
\definecolor{currentstroke}{rgb}{0.000000,0.000000,0.000000}%
\pgfsetstrokecolor{currentstroke}%
\pgfsetstrokeopacity{0.000000}%
\pgfsetdash{}{0pt}%
\pgfpathmoveto{\pgfqpoint{10.484437in}{2.565994in}}%
\pgfpathlineto{\pgfqpoint{10.509115in}{2.565994in}}%
\pgfpathlineto{\pgfqpoint{10.509115in}{2.593426in}}%
\pgfpathlineto{\pgfqpoint{10.484437in}{2.593426in}}%
\pgfpathclose%
\pgfusepath{fill}%
\end{pgfscope}%
\begin{pgfscope}%
\pgfpathrectangle{\pgfqpoint{2.500000in}{0.625000in}}{\pgfqpoint{15.500000in}{3.775000in}}%
\pgfusepath{clip}%
\pgfsetbuttcap%
\pgfsetmiterjoin%
\definecolor{currentfill}{rgb}{0.121569,0.466667,0.705882}%
\pgfsetfillcolor{currentfill}%
\pgfsetlinewidth{0.000000pt}%
\definecolor{currentstroke}{rgb}{0.000000,0.000000,0.000000}%
\pgfsetstrokecolor{currentstroke}%
\pgfsetstrokeopacity{0.000000}%
\pgfsetdash{}{0pt}%
\pgfpathmoveto{\pgfqpoint{10.607825in}{2.565994in}}%
\pgfpathlineto{\pgfqpoint{10.632503in}{2.565994in}}%
\pgfpathlineto{\pgfqpoint{10.632503in}{2.533074in}}%
\pgfpathlineto{\pgfqpoint{10.607825in}{2.533074in}}%
\pgfpathclose%
\pgfusepath{fill}%
\end{pgfscope}%
\begin{pgfscope}%
\pgfpathrectangle{\pgfqpoint{2.500000in}{0.625000in}}{\pgfqpoint{15.500000in}{3.775000in}}%
\pgfusepath{clip}%
\pgfsetbuttcap%
\pgfsetmiterjoin%
\definecolor{currentfill}{rgb}{0.121569,0.466667,0.705882}%
\pgfsetfillcolor{currentfill}%
\pgfsetlinewidth{0.000000pt}%
\definecolor{currentstroke}{rgb}{0.000000,0.000000,0.000000}%
\pgfsetstrokecolor{currentstroke}%
\pgfsetstrokeopacity{0.000000}%
\pgfsetdash{}{0pt}%
\pgfpathmoveto{\pgfqpoint{10.731213in}{2.565994in}}%
\pgfpathlineto{\pgfqpoint{10.755891in}{2.565994in}}%
\pgfpathlineto{\pgfqpoint{10.755891in}{2.659264in}}%
\pgfpathlineto{\pgfqpoint{10.731213in}{2.659264in}}%
\pgfpathclose%
\pgfusepath{fill}%
\end{pgfscope}%
\begin{pgfscope}%
\pgfpathrectangle{\pgfqpoint{2.500000in}{0.625000in}}{\pgfqpoint{15.500000in}{3.775000in}}%
\pgfusepath{clip}%
\pgfsetbuttcap%
\pgfsetmiterjoin%
\definecolor{currentfill}{rgb}{0.121569,0.466667,0.705882}%
\pgfsetfillcolor{currentfill}%
\pgfsetlinewidth{0.000000pt}%
\definecolor{currentstroke}{rgb}{0.000000,0.000000,0.000000}%
\pgfsetstrokecolor{currentstroke}%
\pgfsetstrokeopacity{0.000000}%
\pgfsetdash{}{0pt}%
\pgfpathmoveto{\pgfqpoint{10.854601in}{2.565994in}}%
\pgfpathlineto{\pgfqpoint{10.879279in}{2.565994in}}%
\pgfpathlineto{\pgfqpoint{10.879279in}{2.568737in}}%
\pgfpathlineto{\pgfqpoint{10.854601in}{2.568737in}}%
\pgfpathclose%
\pgfusepath{fill}%
\end{pgfscope}%
\begin{pgfscope}%
\pgfpathrectangle{\pgfqpoint{2.500000in}{0.625000in}}{\pgfqpoint{15.500000in}{3.775000in}}%
\pgfusepath{clip}%
\pgfsetbuttcap%
\pgfsetmiterjoin%
\definecolor{currentfill}{rgb}{0.121569,0.466667,0.705882}%
\pgfsetfillcolor{currentfill}%
\pgfsetlinewidth{0.000000pt}%
\definecolor{currentstroke}{rgb}{0.000000,0.000000,0.000000}%
\pgfsetstrokecolor{currentstroke}%
\pgfsetstrokeopacity{0.000000}%
\pgfsetdash{}{0pt}%
\pgfpathmoveto{\pgfqpoint{10.977989in}{2.565994in}}%
\pgfpathlineto{\pgfqpoint{11.002667in}{2.565994in}}%
\pgfpathlineto{\pgfqpoint{11.002667in}{2.565994in}}%
\pgfpathlineto{\pgfqpoint{10.977989in}{2.565994in}}%
\pgfpathclose%
\pgfusepath{fill}%
\end{pgfscope}%
\begin{pgfscope}%
\pgfpathrectangle{\pgfqpoint{2.500000in}{0.625000in}}{\pgfqpoint{15.500000in}{3.775000in}}%
\pgfusepath{clip}%
\pgfsetbuttcap%
\pgfsetmiterjoin%
\definecolor{currentfill}{rgb}{0.121569,0.466667,0.705882}%
\pgfsetfillcolor{currentfill}%
\pgfsetlinewidth{0.000000pt}%
\definecolor{currentstroke}{rgb}{0.000000,0.000000,0.000000}%
\pgfsetstrokecolor{currentstroke}%
\pgfsetstrokeopacity{0.000000}%
\pgfsetdash{}{0pt}%
\pgfpathmoveto{\pgfqpoint{11.101377in}{2.565994in}}%
\pgfpathlineto{\pgfqpoint{11.126055in}{2.565994in}}%
\pgfpathlineto{\pgfqpoint{11.126055in}{2.557764in}}%
\pgfpathlineto{\pgfqpoint{11.101377in}{2.557764in}}%
\pgfpathclose%
\pgfusepath{fill}%
\end{pgfscope}%
\begin{pgfscope}%
\pgfpathrectangle{\pgfqpoint{2.500000in}{0.625000in}}{\pgfqpoint{15.500000in}{3.775000in}}%
\pgfusepath{clip}%
\pgfsetbuttcap%
\pgfsetmiterjoin%
\definecolor{currentfill}{rgb}{0.121569,0.466667,0.705882}%
\pgfsetfillcolor{currentfill}%
\pgfsetlinewidth{0.000000pt}%
\definecolor{currentstroke}{rgb}{0.000000,0.000000,0.000000}%
\pgfsetstrokecolor{currentstroke}%
\pgfsetstrokeopacity{0.000000}%
\pgfsetdash{}{0pt}%
\pgfpathmoveto{\pgfqpoint{11.224765in}{2.565994in}}%
\pgfpathlineto{\pgfqpoint{11.249443in}{2.565994in}}%
\pgfpathlineto{\pgfqpoint{11.249443in}{2.755279in}}%
\pgfpathlineto{\pgfqpoint{11.224765in}{2.755279in}}%
\pgfpathclose%
\pgfusepath{fill}%
\end{pgfscope}%
\begin{pgfscope}%
\pgfpathrectangle{\pgfqpoint{2.500000in}{0.625000in}}{\pgfqpoint{15.500000in}{3.775000in}}%
\pgfusepath{clip}%
\pgfsetbuttcap%
\pgfsetmiterjoin%
\definecolor{currentfill}{rgb}{0.121569,0.466667,0.705882}%
\pgfsetfillcolor{currentfill}%
\pgfsetlinewidth{0.000000pt}%
\definecolor{currentstroke}{rgb}{0.000000,0.000000,0.000000}%
\pgfsetstrokecolor{currentstroke}%
\pgfsetstrokeopacity{0.000000}%
\pgfsetdash{}{0pt}%
\pgfpathmoveto{\pgfqpoint{11.348153in}{2.565994in}}%
\pgfpathlineto{\pgfqpoint{11.372831in}{2.565994in}}%
\pgfpathlineto{\pgfqpoint{11.372831in}{0.928267in}}%
\pgfpathlineto{\pgfqpoint{11.348153in}{0.928267in}}%
\pgfpathclose%
\pgfusepath{fill}%
\end{pgfscope}%
\begin{pgfscope}%
\pgfpathrectangle{\pgfqpoint{2.500000in}{0.625000in}}{\pgfqpoint{15.500000in}{3.775000in}}%
\pgfusepath{clip}%
\pgfsetbuttcap%
\pgfsetmiterjoin%
\definecolor{currentfill}{rgb}{0.121569,0.466667,0.705882}%
\pgfsetfillcolor{currentfill}%
\pgfsetlinewidth{0.000000pt}%
\definecolor{currentstroke}{rgb}{0.000000,0.000000,0.000000}%
\pgfsetstrokecolor{currentstroke}%
\pgfsetstrokeopacity{0.000000}%
\pgfsetdash{}{0pt}%
\pgfpathmoveto{\pgfqpoint{11.471541in}{2.565994in}}%
\pgfpathlineto{\pgfqpoint{11.496219in}{2.565994in}}%
\pgfpathlineto{\pgfqpoint{11.496219in}{2.565994in}}%
\pgfpathlineto{\pgfqpoint{11.471541in}{2.565994in}}%
\pgfpathclose%
\pgfusepath{fill}%
\end{pgfscope}%
\begin{pgfscope}%
\pgfpathrectangle{\pgfqpoint{2.500000in}{0.625000in}}{\pgfqpoint{15.500000in}{3.775000in}}%
\pgfusepath{clip}%
\pgfsetbuttcap%
\pgfsetmiterjoin%
\definecolor{currentfill}{rgb}{0.121569,0.466667,0.705882}%
\pgfsetfillcolor{currentfill}%
\pgfsetlinewidth{0.000000pt}%
\definecolor{currentstroke}{rgb}{0.000000,0.000000,0.000000}%
\pgfsetstrokecolor{currentstroke}%
\pgfsetstrokeopacity{0.000000}%
\pgfsetdash{}{0pt}%
\pgfpathmoveto{\pgfqpoint{11.594929in}{2.565994in}}%
\pgfpathlineto{\pgfqpoint{11.619607in}{2.565994in}}%
\pgfpathlineto{\pgfqpoint{11.619607in}{0.917294in}}%
\pgfpathlineto{\pgfqpoint{11.594929in}{0.917294in}}%
\pgfpathclose%
\pgfusepath{fill}%
\end{pgfscope}%
\begin{pgfscope}%
\pgfpathrectangle{\pgfqpoint{2.500000in}{0.625000in}}{\pgfqpoint{15.500000in}{3.775000in}}%
\pgfusepath{clip}%
\pgfsetbuttcap%
\pgfsetmiterjoin%
\definecolor{currentfill}{rgb}{0.121569,0.466667,0.705882}%
\pgfsetfillcolor{currentfill}%
\pgfsetlinewidth{0.000000pt}%
\definecolor{currentstroke}{rgb}{0.000000,0.000000,0.000000}%
\pgfsetstrokecolor{currentstroke}%
\pgfsetstrokeopacity{0.000000}%
\pgfsetdash{}{0pt}%
\pgfpathmoveto{\pgfqpoint{11.718317in}{2.565994in}}%
\pgfpathlineto{\pgfqpoint{11.742995in}{2.565994in}}%
\pgfpathlineto{\pgfqpoint{11.742995in}{2.620859in}}%
\pgfpathlineto{\pgfqpoint{11.718317in}{2.620859in}}%
\pgfpathclose%
\pgfusepath{fill}%
\end{pgfscope}%
\begin{pgfscope}%
\pgfpathrectangle{\pgfqpoint{2.500000in}{0.625000in}}{\pgfqpoint{15.500000in}{3.775000in}}%
\pgfusepath{clip}%
\pgfsetbuttcap%
\pgfsetmiterjoin%
\definecolor{currentfill}{rgb}{0.121569,0.466667,0.705882}%
\pgfsetfillcolor{currentfill}%
\pgfsetlinewidth{0.000000pt}%
\definecolor{currentstroke}{rgb}{0.000000,0.000000,0.000000}%
\pgfsetstrokecolor{currentstroke}%
\pgfsetstrokeopacity{0.000000}%
\pgfsetdash{}{0pt}%
\pgfpathmoveto{\pgfqpoint{11.841705in}{2.565994in}}%
\pgfpathlineto{\pgfqpoint{11.866383in}{2.565994in}}%
\pgfpathlineto{\pgfqpoint{11.866383in}{2.571480in}}%
\pgfpathlineto{\pgfqpoint{11.841705in}{2.571480in}}%
\pgfpathclose%
\pgfusepath{fill}%
\end{pgfscope}%
\begin{pgfscope}%
\pgfpathrectangle{\pgfqpoint{2.500000in}{0.625000in}}{\pgfqpoint{15.500000in}{3.775000in}}%
\pgfusepath{clip}%
\pgfsetbuttcap%
\pgfsetmiterjoin%
\definecolor{currentfill}{rgb}{0.121569,0.466667,0.705882}%
\pgfsetfillcolor{currentfill}%
\pgfsetlinewidth{0.000000pt}%
\definecolor{currentstroke}{rgb}{0.000000,0.000000,0.000000}%
\pgfsetstrokecolor{currentstroke}%
\pgfsetstrokeopacity{0.000000}%
\pgfsetdash{}{0pt}%
\pgfpathmoveto{\pgfqpoint{11.965093in}{2.565994in}}%
\pgfpathlineto{\pgfqpoint{11.989771in}{2.565994in}}%
\pgfpathlineto{\pgfqpoint{11.989771in}{0.922781in}}%
\pgfpathlineto{\pgfqpoint{11.965093in}{0.922781in}}%
\pgfpathclose%
\pgfusepath{fill}%
\end{pgfscope}%
\begin{pgfscope}%
\pgfpathrectangle{\pgfqpoint{2.500000in}{0.625000in}}{\pgfqpoint{15.500000in}{3.775000in}}%
\pgfusepath{clip}%
\pgfsetbuttcap%
\pgfsetmiterjoin%
\definecolor{currentfill}{rgb}{0.121569,0.466667,0.705882}%
\pgfsetfillcolor{currentfill}%
\pgfsetlinewidth{0.000000pt}%
\definecolor{currentstroke}{rgb}{0.000000,0.000000,0.000000}%
\pgfsetstrokecolor{currentstroke}%
\pgfsetstrokeopacity{0.000000}%
\pgfsetdash{}{0pt}%
\pgfpathmoveto{\pgfqpoint{12.088481in}{2.565994in}}%
\pgfpathlineto{\pgfqpoint{12.113159in}{2.565994in}}%
\pgfpathlineto{\pgfqpoint{12.113159in}{4.214693in}}%
\pgfpathlineto{\pgfqpoint{12.088481in}{4.214693in}}%
\pgfpathclose%
\pgfusepath{fill}%
\end{pgfscope}%
\begin{pgfscope}%
\pgfpathrectangle{\pgfqpoint{2.500000in}{0.625000in}}{\pgfqpoint{15.500000in}{3.775000in}}%
\pgfusepath{clip}%
\pgfsetbuttcap%
\pgfsetmiterjoin%
\definecolor{currentfill}{rgb}{0.121569,0.466667,0.705882}%
\pgfsetfillcolor{currentfill}%
\pgfsetlinewidth{0.000000pt}%
\definecolor{currentstroke}{rgb}{0.000000,0.000000,0.000000}%
\pgfsetstrokecolor{currentstroke}%
\pgfsetstrokeopacity{0.000000}%
\pgfsetdash{}{0pt}%
\pgfpathmoveto{\pgfqpoint{12.211869in}{2.565994in}}%
\pgfpathlineto{\pgfqpoint{12.236547in}{2.565994in}}%
\pgfpathlineto{\pgfqpoint{12.236547in}{1.614082in}}%
\pgfpathlineto{\pgfqpoint{12.211869in}{1.614082in}}%
\pgfpathclose%
\pgfusepath{fill}%
\end{pgfscope}%
\begin{pgfscope}%
\pgfpathrectangle{\pgfqpoint{2.500000in}{0.625000in}}{\pgfqpoint{15.500000in}{3.775000in}}%
\pgfusepath{clip}%
\pgfsetbuttcap%
\pgfsetmiterjoin%
\definecolor{currentfill}{rgb}{0.121569,0.466667,0.705882}%
\pgfsetfillcolor{currentfill}%
\pgfsetlinewidth{0.000000pt}%
\definecolor{currentstroke}{rgb}{0.000000,0.000000,0.000000}%
\pgfsetstrokecolor{currentstroke}%
\pgfsetstrokeopacity{0.000000}%
\pgfsetdash{}{0pt}%
\pgfpathmoveto{\pgfqpoint{12.335257in}{2.565994in}}%
\pgfpathlineto{\pgfqpoint{12.359935in}{2.565994in}}%
\pgfpathlineto{\pgfqpoint{12.359935in}{1.098350in}}%
\pgfpathlineto{\pgfqpoint{12.335257in}{1.098350in}}%
\pgfpathclose%
\pgfusepath{fill}%
\end{pgfscope}%
\begin{pgfscope}%
\pgfpathrectangle{\pgfqpoint{2.500000in}{0.625000in}}{\pgfqpoint{15.500000in}{3.775000in}}%
\pgfusepath{clip}%
\pgfsetbuttcap%
\pgfsetmiterjoin%
\definecolor{currentfill}{rgb}{0.121569,0.466667,0.705882}%
\pgfsetfillcolor{currentfill}%
\pgfsetlinewidth{0.000000pt}%
\definecolor{currentstroke}{rgb}{0.000000,0.000000,0.000000}%
\pgfsetstrokecolor{currentstroke}%
\pgfsetstrokeopacity{0.000000}%
\pgfsetdash{}{0pt}%
\pgfpathmoveto{\pgfqpoint{12.458645in}{2.565994in}}%
\pgfpathlineto{\pgfqpoint{12.483323in}{2.565994in}}%
\pgfpathlineto{\pgfqpoint{12.483323in}{2.565994in}}%
\pgfpathlineto{\pgfqpoint{12.458645in}{2.565994in}}%
\pgfpathclose%
\pgfusepath{fill}%
\end{pgfscope}%
\begin{pgfscope}%
\pgfpathrectangle{\pgfqpoint{2.500000in}{0.625000in}}{\pgfqpoint{15.500000in}{3.775000in}}%
\pgfusepath{clip}%
\pgfsetbuttcap%
\pgfsetmiterjoin%
\definecolor{currentfill}{rgb}{0.121569,0.466667,0.705882}%
\pgfsetfillcolor{currentfill}%
\pgfsetlinewidth{0.000000pt}%
\definecolor{currentstroke}{rgb}{0.000000,0.000000,0.000000}%
\pgfsetstrokecolor{currentstroke}%
\pgfsetstrokeopacity{0.000000}%
\pgfsetdash{}{0pt}%
\pgfpathmoveto{\pgfqpoint{12.582033in}{2.565994in}}%
\pgfpathlineto{\pgfqpoint{12.606711in}{2.565994in}}%
\pgfpathlineto{\pgfqpoint{12.606711in}{0.906321in}}%
\pgfpathlineto{\pgfqpoint{12.582033in}{0.906321in}}%
\pgfpathclose%
\pgfusepath{fill}%
\end{pgfscope}%
\begin{pgfscope}%
\pgfpathrectangle{\pgfqpoint{2.500000in}{0.625000in}}{\pgfqpoint{15.500000in}{3.775000in}}%
\pgfusepath{clip}%
\pgfsetbuttcap%
\pgfsetmiterjoin%
\definecolor{currentfill}{rgb}{0.121569,0.466667,0.705882}%
\pgfsetfillcolor{currentfill}%
\pgfsetlinewidth{0.000000pt}%
\definecolor{currentstroke}{rgb}{0.000000,0.000000,0.000000}%
\pgfsetstrokecolor{currentstroke}%
\pgfsetstrokeopacity{0.000000}%
\pgfsetdash{}{0pt}%
\pgfpathmoveto{\pgfqpoint{12.705421in}{2.565994in}}%
\pgfpathlineto{\pgfqpoint{12.730099in}{2.565994in}}%
\pgfpathlineto{\pgfqpoint{12.730099in}{2.576967in}}%
\pgfpathlineto{\pgfqpoint{12.705421in}{2.576967in}}%
\pgfpathclose%
\pgfusepath{fill}%
\end{pgfscope}%
\begin{pgfscope}%
\pgfpathrectangle{\pgfqpoint{2.500000in}{0.625000in}}{\pgfqpoint{15.500000in}{3.775000in}}%
\pgfusepath{clip}%
\pgfsetbuttcap%
\pgfsetmiterjoin%
\definecolor{currentfill}{rgb}{0.121569,0.466667,0.705882}%
\pgfsetfillcolor{currentfill}%
\pgfsetlinewidth{0.000000pt}%
\definecolor{currentstroke}{rgb}{0.000000,0.000000,0.000000}%
\pgfsetstrokecolor{currentstroke}%
\pgfsetstrokeopacity{0.000000}%
\pgfsetdash{}{0pt}%
\pgfpathmoveto{\pgfqpoint{12.828809in}{2.565994in}}%
\pgfpathlineto{\pgfqpoint{12.853487in}{2.565994in}}%
\pgfpathlineto{\pgfqpoint{12.853487in}{4.211950in}}%
\pgfpathlineto{\pgfqpoint{12.828809in}{4.211950in}}%
\pgfpathclose%
\pgfusepath{fill}%
\end{pgfscope}%
\begin{pgfscope}%
\pgfpathrectangle{\pgfqpoint{2.500000in}{0.625000in}}{\pgfqpoint{15.500000in}{3.775000in}}%
\pgfusepath{clip}%
\pgfsetbuttcap%
\pgfsetmiterjoin%
\definecolor{currentfill}{rgb}{0.121569,0.466667,0.705882}%
\pgfsetfillcolor{currentfill}%
\pgfsetlinewidth{0.000000pt}%
\definecolor{currentstroke}{rgb}{0.000000,0.000000,0.000000}%
\pgfsetstrokecolor{currentstroke}%
\pgfsetstrokeopacity{0.000000}%
\pgfsetdash{}{0pt}%
\pgfpathmoveto{\pgfqpoint{12.952197in}{2.565994in}}%
\pgfpathlineto{\pgfqpoint{12.976875in}{2.565994in}}%
\pgfpathlineto{\pgfqpoint{12.976875in}{4.217436in}}%
\pgfpathlineto{\pgfqpoint{12.952197in}{4.217436in}}%
\pgfpathclose%
\pgfusepath{fill}%
\end{pgfscope}%
\begin{pgfscope}%
\pgfpathrectangle{\pgfqpoint{2.500000in}{0.625000in}}{\pgfqpoint{15.500000in}{3.775000in}}%
\pgfusepath{clip}%
\pgfsetbuttcap%
\pgfsetmiterjoin%
\definecolor{currentfill}{rgb}{0.121569,0.466667,0.705882}%
\pgfsetfillcolor{currentfill}%
\pgfsetlinewidth{0.000000pt}%
\definecolor{currentstroke}{rgb}{0.000000,0.000000,0.000000}%
\pgfsetstrokecolor{currentstroke}%
\pgfsetstrokeopacity{0.000000}%
\pgfsetdash{}{0pt}%
\pgfpathmoveto{\pgfqpoint{13.075585in}{2.565994in}}%
\pgfpathlineto{\pgfqpoint{13.100263in}{2.565994in}}%
\pgfpathlineto{\pgfqpoint{13.100263in}{4.187260in}}%
\pgfpathlineto{\pgfqpoint{13.075585in}{4.187260in}}%
\pgfpathclose%
\pgfusepath{fill}%
\end{pgfscope}%
\begin{pgfscope}%
\pgfpathrectangle{\pgfqpoint{2.500000in}{0.625000in}}{\pgfqpoint{15.500000in}{3.775000in}}%
\pgfusepath{clip}%
\pgfsetbuttcap%
\pgfsetmiterjoin%
\definecolor{currentfill}{rgb}{0.121569,0.466667,0.705882}%
\pgfsetfillcolor{currentfill}%
\pgfsetlinewidth{0.000000pt}%
\definecolor{currentstroke}{rgb}{0.000000,0.000000,0.000000}%
\pgfsetstrokecolor{currentstroke}%
\pgfsetstrokeopacity{0.000000}%
\pgfsetdash{}{0pt}%
\pgfpathmoveto{\pgfqpoint{13.198973in}{2.565994in}}%
\pgfpathlineto{\pgfqpoint{13.223651in}{2.565994in}}%
\pgfpathlineto{\pgfqpoint{13.223651in}{2.574223in}}%
\pgfpathlineto{\pgfqpoint{13.198973in}{2.574223in}}%
\pgfpathclose%
\pgfusepath{fill}%
\end{pgfscope}%
\begin{pgfscope}%
\pgfpathrectangle{\pgfqpoint{2.500000in}{0.625000in}}{\pgfqpoint{15.500000in}{3.775000in}}%
\pgfusepath{clip}%
\pgfsetbuttcap%
\pgfsetmiterjoin%
\definecolor{currentfill}{rgb}{0.121569,0.466667,0.705882}%
\pgfsetfillcolor{currentfill}%
\pgfsetlinewidth{0.000000pt}%
\definecolor{currentstroke}{rgb}{0.000000,0.000000,0.000000}%
\pgfsetstrokecolor{currentstroke}%
\pgfsetstrokeopacity{0.000000}%
\pgfsetdash{}{0pt}%
\pgfpathmoveto{\pgfqpoint{13.322361in}{2.565994in}}%
\pgfpathlineto{\pgfqpoint{13.347039in}{2.565994in}}%
\pgfpathlineto{\pgfqpoint{13.347039in}{2.560507in}}%
\pgfpathlineto{\pgfqpoint{13.322361in}{2.560507in}}%
\pgfpathclose%
\pgfusepath{fill}%
\end{pgfscope}%
\begin{pgfscope}%
\pgfpathrectangle{\pgfqpoint{2.500000in}{0.625000in}}{\pgfqpoint{15.500000in}{3.775000in}}%
\pgfusepath{clip}%
\pgfsetbuttcap%
\pgfsetmiterjoin%
\definecolor{currentfill}{rgb}{0.121569,0.466667,0.705882}%
\pgfsetfillcolor{currentfill}%
\pgfsetlinewidth{0.000000pt}%
\definecolor{currentstroke}{rgb}{0.000000,0.000000,0.000000}%
\pgfsetstrokecolor{currentstroke}%
\pgfsetstrokeopacity{0.000000}%
\pgfsetdash{}{0pt}%
\pgfpathmoveto{\pgfqpoint{13.445749in}{2.565994in}}%
\pgfpathlineto{\pgfqpoint{13.470427in}{2.565994in}}%
\pgfpathlineto{\pgfqpoint{13.470427in}{2.607142in}}%
\pgfpathlineto{\pgfqpoint{13.445749in}{2.607142in}}%
\pgfpathclose%
\pgfusepath{fill}%
\end{pgfscope}%
\begin{pgfscope}%
\pgfpathrectangle{\pgfqpoint{2.500000in}{0.625000in}}{\pgfqpoint{15.500000in}{3.775000in}}%
\pgfusepath{clip}%
\pgfsetbuttcap%
\pgfsetmiterjoin%
\definecolor{currentfill}{rgb}{0.121569,0.466667,0.705882}%
\pgfsetfillcolor{currentfill}%
\pgfsetlinewidth{0.000000pt}%
\definecolor{currentstroke}{rgb}{0.000000,0.000000,0.000000}%
\pgfsetstrokecolor{currentstroke}%
\pgfsetstrokeopacity{0.000000}%
\pgfsetdash{}{0pt}%
\pgfpathmoveto{\pgfqpoint{13.569137in}{2.565994in}}%
\pgfpathlineto{\pgfqpoint{13.593815in}{2.565994in}}%
\pgfpathlineto{\pgfqpoint{13.593815in}{2.565994in}}%
\pgfpathlineto{\pgfqpoint{13.569137in}{2.565994in}}%
\pgfpathclose%
\pgfusepath{fill}%
\end{pgfscope}%
\begin{pgfscope}%
\pgfpathrectangle{\pgfqpoint{2.500000in}{0.625000in}}{\pgfqpoint{15.500000in}{3.775000in}}%
\pgfusepath{clip}%
\pgfsetbuttcap%
\pgfsetmiterjoin%
\definecolor{currentfill}{rgb}{0.121569,0.466667,0.705882}%
\pgfsetfillcolor{currentfill}%
\pgfsetlinewidth{0.000000pt}%
\definecolor{currentstroke}{rgb}{0.000000,0.000000,0.000000}%
\pgfsetstrokecolor{currentstroke}%
\pgfsetstrokeopacity{0.000000}%
\pgfsetdash{}{0pt}%
\pgfpathmoveto{\pgfqpoint{13.692525in}{2.565994in}}%
\pgfpathlineto{\pgfqpoint{13.717203in}{2.565994in}}%
\pgfpathlineto{\pgfqpoint{13.717203in}{0.928267in}}%
\pgfpathlineto{\pgfqpoint{13.692525in}{0.928267in}}%
\pgfpathclose%
\pgfusepath{fill}%
\end{pgfscope}%
\begin{pgfscope}%
\pgfpathrectangle{\pgfqpoint{2.500000in}{0.625000in}}{\pgfqpoint{15.500000in}{3.775000in}}%
\pgfusepath{clip}%
\pgfsetbuttcap%
\pgfsetmiterjoin%
\definecolor{currentfill}{rgb}{0.121569,0.466667,0.705882}%
\pgfsetfillcolor{currentfill}%
\pgfsetlinewidth{0.000000pt}%
\definecolor{currentstroke}{rgb}{0.000000,0.000000,0.000000}%
\pgfsetstrokecolor{currentstroke}%
\pgfsetstrokeopacity{0.000000}%
\pgfsetdash{}{0pt}%
\pgfpathmoveto{\pgfqpoint{13.815913in}{2.565994in}}%
\pgfpathlineto{\pgfqpoint{13.840591in}{2.565994in}}%
\pgfpathlineto{\pgfqpoint{13.840591in}{2.576967in}}%
\pgfpathlineto{\pgfqpoint{13.815913in}{2.576967in}}%
\pgfpathclose%
\pgfusepath{fill}%
\end{pgfscope}%
\begin{pgfscope}%
\pgfpathrectangle{\pgfqpoint{2.500000in}{0.625000in}}{\pgfqpoint{15.500000in}{3.775000in}}%
\pgfusepath{clip}%
\pgfsetbuttcap%
\pgfsetmiterjoin%
\definecolor{currentfill}{rgb}{0.121569,0.466667,0.705882}%
\pgfsetfillcolor{currentfill}%
\pgfsetlinewidth{0.000000pt}%
\definecolor{currentstroke}{rgb}{0.000000,0.000000,0.000000}%
\pgfsetstrokecolor{currentstroke}%
\pgfsetstrokeopacity{0.000000}%
\pgfsetdash{}{0pt}%
\pgfpathmoveto{\pgfqpoint{13.939301in}{2.565994in}}%
\pgfpathlineto{\pgfqpoint{13.963979in}{2.565994in}}%
\pgfpathlineto{\pgfqpoint{13.963979in}{2.530331in}}%
\pgfpathlineto{\pgfqpoint{13.939301in}{2.530331in}}%
\pgfpathclose%
\pgfusepath{fill}%
\end{pgfscope}%
\begin{pgfscope}%
\pgfpathrectangle{\pgfqpoint{2.500000in}{0.625000in}}{\pgfqpoint{15.500000in}{3.775000in}}%
\pgfusepath{clip}%
\pgfsetbuttcap%
\pgfsetmiterjoin%
\definecolor{currentfill}{rgb}{0.121569,0.466667,0.705882}%
\pgfsetfillcolor{currentfill}%
\pgfsetlinewidth{0.000000pt}%
\definecolor{currentstroke}{rgb}{0.000000,0.000000,0.000000}%
\pgfsetstrokecolor{currentstroke}%
\pgfsetstrokeopacity{0.000000}%
\pgfsetdash{}{0pt}%
\pgfpathmoveto{\pgfqpoint{14.062689in}{2.565994in}}%
\pgfpathlineto{\pgfqpoint{14.087367in}{2.565994in}}%
\pgfpathlineto{\pgfqpoint{14.087367in}{2.574223in}}%
\pgfpathlineto{\pgfqpoint{14.062689in}{2.574223in}}%
\pgfpathclose%
\pgfusepath{fill}%
\end{pgfscope}%
\begin{pgfscope}%
\pgfpathrectangle{\pgfqpoint{2.500000in}{0.625000in}}{\pgfqpoint{15.500000in}{3.775000in}}%
\pgfusepath{clip}%
\pgfsetbuttcap%
\pgfsetmiterjoin%
\definecolor{currentfill}{rgb}{0.121569,0.466667,0.705882}%
\pgfsetfillcolor{currentfill}%
\pgfsetlinewidth{0.000000pt}%
\definecolor{currentstroke}{rgb}{0.000000,0.000000,0.000000}%
\pgfsetstrokecolor{currentstroke}%
\pgfsetstrokeopacity{0.000000}%
\pgfsetdash{}{0pt}%
\pgfpathmoveto{\pgfqpoint{14.186077in}{2.565994in}}%
\pgfpathlineto{\pgfqpoint{14.210755in}{2.565994in}}%
\pgfpathlineto{\pgfqpoint{14.210755in}{2.563250in}}%
\pgfpathlineto{\pgfqpoint{14.186077in}{2.563250in}}%
\pgfpathclose%
\pgfusepath{fill}%
\end{pgfscope}%
\begin{pgfscope}%
\pgfpathrectangle{\pgfqpoint{2.500000in}{0.625000in}}{\pgfqpoint{15.500000in}{3.775000in}}%
\pgfusepath{clip}%
\pgfsetbuttcap%
\pgfsetmiterjoin%
\definecolor{currentfill}{rgb}{0.121569,0.466667,0.705882}%
\pgfsetfillcolor{currentfill}%
\pgfsetlinewidth{0.000000pt}%
\definecolor{currentstroke}{rgb}{0.000000,0.000000,0.000000}%
\pgfsetstrokecolor{currentstroke}%
\pgfsetstrokeopacity{0.000000}%
\pgfsetdash{}{0pt}%
\pgfpathmoveto{\pgfqpoint{14.309465in}{2.565994in}}%
\pgfpathlineto{\pgfqpoint{14.334143in}{2.565994in}}%
\pgfpathlineto{\pgfqpoint{14.334143in}{2.563250in}}%
\pgfpathlineto{\pgfqpoint{14.309465in}{2.563250in}}%
\pgfpathclose%
\pgfusepath{fill}%
\end{pgfscope}%
\begin{pgfscope}%
\pgfpathrectangle{\pgfqpoint{2.500000in}{0.625000in}}{\pgfqpoint{15.500000in}{3.775000in}}%
\pgfusepath{clip}%
\pgfsetbuttcap%
\pgfsetmiterjoin%
\definecolor{currentfill}{rgb}{0.121569,0.466667,0.705882}%
\pgfsetfillcolor{currentfill}%
\pgfsetlinewidth{0.000000pt}%
\definecolor{currentstroke}{rgb}{0.000000,0.000000,0.000000}%
\pgfsetstrokecolor{currentstroke}%
\pgfsetstrokeopacity{0.000000}%
\pgfsetdash{}{0pt}%
\pgfpathmoveto{\pgfqpoint{14.432853in}{2.565994in}}%
\pgfpathlineto{\pgfqpoint{14.457531in}{2.565994in}}%
\pgfpathlineto{\pgfqpoint{14.457531in}{2.598913in}}%
\pgfpathlineto{\pgfqpoint{14.432853in}{2.598913in}}%
\pgfpathclose%
\pgfusepath{fill}%
\end{pgfscope}%
\begin{pgfscope}%
\pgfpathrectangle{\pgfqpoint{2.500000in}{0.625000in}}{\pgfqpoint{15.500000in}{3.775000in}}%
\pgfusepath{clip}%
\pgfsetbuttcap%
\pgfsetmiterjoin%
\definecolor{currentfill}{rgb}{0.121569,0.466667,0.705882}%
\pgfsetfillcolor{currentfill}%
\pgfsetlinewidth{0.000000pt}%
\definecolor{currentstroke}{rgb}{0.000000,0.000000,0.000000}%
\pgfsetstrokecolor{currentstroke}%
\pgfsetstrokeopacity{0.000000}%
\pgfsetdash{}{0pt}%
\pgfpathmoveto{\pgfqpoint{14.556241in}{2.565994in}}%
\pgfpathlineto{\pgfqpoint{14.580919in}{2.565994in}}%
\pgfpathlineto{\pgfqpoint{14.580919in}{2.357506in}}%
\pgfpathlineto{\pgfqpoint{14.556241in}{2.357506in}}%
\pgfpathclose%
\pgfusepath{fill}%
\end{pgfscope}%
\begin{pgfscope}%
\pgfpathrectangle{\pgfqpoint{2.500000in}{0.625000in}}{\pgfqpoint{15.500000in}{3.775000in}}%
\pgfusepath{clip}%
\pgfsetbuttcap%
\pgfsetmiterjoin%
\definecolor{currentfill}{rgb}{0.121569,0.466667,0.705882}%
\pgfsetfillcolor{currentfill}%
\pgfsetlinewidth{0.000000pt}%
\definecolor{currentstroke}{rgb}{0.000000,0.000000,0.000000}%
\pgfsetstrokecolor{currentstroke}%
\pgfsetstrokeopacity{0.000000}%
\pgfsetdash{}{0pt}%
\pgfpathmoveto{\pgfqpoint{14.679629in}{2.565994in}}%
\pgfpathlineto{\pgfqpoint{14.704307in}{2.565994in}}%
\pgfpathlineto{\pgfqpoint{14.704307in}{2.426087in}}%
\pgfpathlineto{\pgfqpoint{14.679629in}{2.426087in}}%
\pgfpathclose%
\pgfusepath{fill}%
\end{pgfscope}%
\begin{pgfscope}%
\pgfpathrectangle{\pgfqpoint{2.500000in}{0.625000in}}{\pgfqpoint{15.500000in}{3.775000in}}%
\pgfusepath{clip}%
\pgfsetbuttcap%
\pgfsetmiterjoin%
\definecolor{currentfill}{rgb}{0.121569,0.466667,0.705882}%
\pgfsetfillcolor{currentfill}%
\pgfsetlinewidth{0.000000pt}%
\definecolor{currentstroke}{rgb}{0.000000,0.000000,0.000000}%
\pgfsetstrokecolor{currentstroke}%
\pgfsetstrokeopacity{0.000000}%
\pgfsetdash{}{0pt}%
\pgfpathmoveto{\pgfqpoint{14.803017in}{2.565994in}}%
\pgfpathlineto{\pgfqpoint{14.827695in}{2.565994in}}%
\pgfpathlineto{\pgfqpoint{14.827695in}{2.648291in}}%
\pgfpathlineto{\pgfqpoint{14.803017in}{2.648291in}}%
\pgfpathclose%
\pgfusepath{fill}%
\end{pgfscope}%
\begin{pgfscope}%
\pgfpathrectangle{\pgfqpoint{2.500000in}{0.625000in}}{\pgfqpoint{15.500000in}{3.775000in}}%
\pgfusepath{clip}%
\pgfsetbuttcap%
\pgfsetmiterjoin%
\definecolor{currentfill}{rgb}{0.121569,0.466667,0.705882}%
\pgfsetfillcolor{currentfill}%
\pgfsetlinewidth{0.000000pt}%
\definecolor{currentstroke}{rgb}{0.000000,0.000000,0.000000}%
\pgfsetstrokecolor{currentstroke}%
\pgfsetstrokeopacity{0.000000}%
\pgfsetdash{}{0pt}%
\pgfpathmoveto{\pgfqpoint{14.926405in}{2.565994in}}%
\pgfpathlineto{\pgfqpoint{14.951083in}{2.565994in}}%
\pgfpathlineto{\pgfqpoint{14.951083in}{4.228409in}}%
\pgfpathlineto{\pgfqpoint{14.926405in}{4.228409in}}%
\pgfpathclose%
\pgfusepath{fill}%
\end{pgfscope}%
\begin{pgfscope}%
\pgfpathrectangle{\pgfqpoint{2.500000in}{0.625000in}}{\pgfqpoint{15.500000in}{3.775000in}}%
\pgfusepath{clip}%
\pgfsetbuttcap%
\pgfsetmiterjoin%
\definecolor{currentfill}{rgb}{0.121569,0.466667,0.705882}%
\pgfsetfillcolor{currentfill}%
\pgfsetlinewidth{0.000000pt}%
\definecolor{currentstroke}{rgb}{0.000000,0.000000,0.000000}%
\pgfsetstrokecolor{currentstroke}%
\pgfsetstrokeopacity{0.000000}%
\pgfsetdash{}{0pt}%
\pgfpathmoveto{\pgfqpoint{15.049793in}{2.565994in}}%
\pgfpathlineto{\pgfqpoint{15.074471in}{2.565994in}}%
\pgfpathlineto{\pgfqpoint{15.074471in}{2.601656in}}%
\pgfpathlineto{\pgfqpoint{15.049793in}{2.601656in}}%
\pgfpathclose%
\pgfusepath{fill}%
\end{pgfscope}%
\begin{pgfscope}%
\pgfpathrectangle{\pgfqpoint{2.500000in}{0.625000in}}{\pgfqpoint{15.500000in}{3.775000in}}%
\pgfusepath{clip}%
\pgfsetbuttcap%
\pgfsetmiterjoin%
\definecolor{currentfill}{rgb}{0.121569,0.466667,0.705882}%
\pgfsetfillcolor{currentfill}%
\pgfsetlinewidth{0.000000pt}%
\definecolor{currentstroke}{rgb}{0.000000,0.000000,0.000000}%
\pgfsetstrokecolor{currentstroke}%
\pgfsetstrokeopacity{0.000000}%
\pgfsetdash{}{0pt}%
\pgfpathmoveto{\pgfqpoint{15.173181in}{2.565994in}}%
\pgfpathlineto{\pgfqpoint{15.197859in}{2.565994in}}%
\pgfpathlineto{\pgfqpoint{15.197859in}{2.321843in}}%
\pgfpathlineto{\pgfqpoint{15.173181in}{2.321843in}}%
\pgfpathclose%
\pgfusepath{fill}%
\end{pgfscope}%
\begin{pgfscope}%
\pgfpathrectangle{\pgfqpoint{2.500000in}{0.625000in}}{\pgfqpoint{15.500000in}{3.775000in}}%
\pgfusepath{clip}%
\pgfsetbuttcap%
\pgfsetmiterjoin%
\definecolor{currentfill}{rgb}{0.121569,0.466667,0.705882}%
\pgfsetfillcolor{currentfill}%
\pgfsetlinewidth{0.000000pt}%
\definecolor{currentstroke}{rgb}{0.000000,0.000000,0.000000}%
\pgfsetstrokecolor{currentstroke}%
\pgfsetstrokeopacity{0.000000}%
\pgfsetdash{}{0pt}%
\pgfpathmoveto{\pgfqpoint{15.296569in}{2.565994in}}%
\pgfpathlineto{\pgfqpoint{15.321247in}{2.565994in}}%
\pgfpathlineto{\pgfqpoint{15.321247in}{0.914551in}}%
\pgfpathlineto{\pgfqpoint{15.296569in}{0.914551in}}%
\pgfpathclose%
\pgfusepath{fill}%
\end{pgfscope}%
\begin{pgfscope}%
\pgfpathrectangle{\pgfqpoint{2.500000in}{0.625000in}}{\pgfqpoint{15.500000in}{3.775000in}}%
\pgfusepath{clip}%
\pgfsetbuttcap%
\pgfsetmiterjoin%
\definecolor{currentfill}{rgb}{0.121569,0.466667,0.705882}%
\pgfsetfillcolor{currentfill}%
\pgfsetlinewidth{0.000000pt}%
\definecolor{currentstroke}{rgb}{0.000000,0.000000,0.000000}%
\pgfsetstrokecolor{currentstroke}%
\pgfsetstrokeopacity{0.000000}%
\pgfsetdash{}{0pt}%
\pgfpathmoveto{\pgfqpoint{15.419957in}{2.565994in}}%
\pgfpathlineto{\pgfqpoint{15.444635in}{2.565994in}}%
\pgfpathlineto{\pgfqpoint{15.444635in}{0.906321in}}%
\pgfpathlineto{\pgfqpoint{15.419957in}{0.906321in}}%
\pgfpathclose%
\pgfusepath{fill}%
\end{pgfscope}%
\begin{pgfscope}%
\pgfpathrectangle{\pgfqpoint{2.500000in}{0.625000in}}{\pgfqpoint{15.500000in}{3.775000in}}%
\pgfusepath{clip}%
\pgfsetbuttcap%
\pgfsetmiterjoin%
\definecolor{currentfill}{rgb}{0.121569,0.466667,0.705882}%
\pgfsetfillcolor{currentfill}%
\pgfsetlinewidth{0.000000pt}%
\definecolor{currentstroke}{rgb}{0.000000,0.000000,0.000000}%
\pgfsetstrokecolor{currentstroke}%
\pgfsetstrokeopacity{0.000000}%
\pgfsetdash{}{0pt}%
\pgfpathmoveto{\pgfqpoint{15.543345in}{2.565994in}}%
\pgfpathlineto{\pgfqpoint{15.568023in}{2.565994in}}%
\pgfpathlineto{\pgfqpoint{15.568023in}{2.565994in}}%
\pgfpathlineto{\pgfqpoint{15.543345in}{2.565994in}}%
\pgfpathclose%
\pgfusepath{fill}%
\end{pgfscope}%
\begin{pgfscope}%
\pgfpathrectangle{\pgfqpoint{2.500000in}{0.625000in}}{\pgfqpoint{15.500000in}{3.775000in}}%
\pgfusepath{clip}%
\pgfsetbuttcap%
\pgfsetmiterjoin%
\definecolor{currentfill}{rgb}{0.121569,0.466667,0.705882}%
\pgfsetfillcolor{currentfill}%
\pgfsetlinewidth{0.000000pt}%
\definecolor{currentstroke}{rgb}{0.000000,0.000000,0.000000}%
\pgfsetstrokecolor{currentstroke}%
\pgfsetstrokeopacity{0.000000}%
\pgfsetdash{}{0pt}%
\pgfpathmoveto{\pgfqpoint{15.666733in}{2.565994in}}%
\pgfpathlineto{\pgfqpoint{15.691411in}{2.565994in}}%
\pgfpathlineto{\pgfqpoint{15.691411in}{2.703157in}}%
\pgfpathlineto{\pgfqpoint{15.666733in}{2.703157in}}%
\pgfpathclose%
\pgfusepath{fill}%
\end{pgfscope}%
\begin{pgfscope}%
\pgfpathrectangle{\pgfqpoint{2.500000in}{0.625000in}}{\pgfqpoint{15.500000in}{3.775000in}}%
\pgfusepath{clip}%
\pgfsetbuttcap%
\pgfsetmiterjoin%
\definecolor{currentfill}{rgb}{0.121569,0.466667,0.705882}%
\pgfsetfillcolor{currentfill}%
\pgfsetlinewidth{0.000000pt}%
\definecolor{currentstroke}{rgb}{0.000000,0.000000,0.000000}%
\pgfsetstrokecolor{currentstroke}%
\pgfsetstrokeopacity{0.000000}%
\pgfsetdash{}{0pt}%
\pgfpathmoveto{\pgfqpoint{15.790121in}{2.565994in}}%
\pgfpathlineto{\pgfqpoint{15.814799in}{2.565994in}}%
\pgfpathlineto{\pgfqpoint{15.814799in}{2.535818in}}%
\pgfpathlineto{\pgfqpoint{15.790121in}{2.535818in}}%
\pgfpathclose%
\pgfusepath{fill}%
\end{pgfscope}%
\begin{pgfscope}%
\pgfpathrectangle{\pgfqpoint{2.500000in}{0.625000in}}{\pgfqpoint{15.500000in}{3.775000in}}%
\pgfusepath{clip}%
\pgfsetbuttcap%
\pgfsetmiterjoin%
\definecolor{currentfill}{rgb}{0.121569,0.466667,0.705882}%
\pgfsetfillcolor{currentfill}%
\pgfsetlinewidth{0.000000pt}%
\definecolor{currentstroke}{rgb}{0.000000,0.000000,0.000000}%
\pgfsetstrokecolor{currentstroke}%
\pgfsetstrokeopacity{0.000000}%
\pgfsetdash{}{0pt}%
\pgfpathmoveto{\pgfqpoint{15.913509in}{2.565994in}}%
\pgfpathlineto{\pgfqpoint{15.938187in}{2.565994in}}%
\pgfpathlineto{\pgfqpoint{15.938187in}{2.571480in}}%
\pgfpathlineto{\pgfqpoint{15.913509in}{2.571480in}}%
\pgfpathclose%
\pgfusepath{fill}%
\end{pgfscope}%
\begin{pgfscope}%
\pgfpathrectangle{\pgfqpoint{2.500000in}{0.625000in}}{\pgfqpoint{15.500000in}{3.775000in}}%
\pgfusepath{clip}%
\pgfsetbuttcap%
\pgfsetmiterjoin%
\definecolor{currentfill}{rgb}{0.121569,0.466667,0.705882}%
\pgfsetfillcolor{currentfill}%
\pgfsetlinewidth{0.000000pt}%
\definecolor{currentstroke}{rgb}{0.000000,0.000000,0.000000}%
\pgfsetstrokecolor{currentstroke}%
\pgfsetstrokeopacity{0.000000}%
\pgfsetdash{}{0pt}%
\pgfpathmoveto{\pgfqpoint{16.036897in}{2.565994in}}%
\pgfpathlineto{\pgfqpoint{16.061575in}{2.565994in}}%
\pgfpathlineto{\pgfqpoint{16.061575in}{2.631832in}}%
\pgfpathlineto{\pgfqpoint{16.036897in}{2.631832in}}%
\pgfpathclose%
\pgfusepath{fill}%
\end{pgfscope}%
\begin{pgfscope}%
\pgfpathrectangle{\pgfqpoint{2.500000in}{0.625000in}}{\pgfqpoint{15.500000in}{3.775000in}}%
\pgfusepath{clip}%
\pgfsetbuttcap%
\pgfsetmiterjoin%
\definecolor{currentfill}{rgb}{0.121569,0.466667,0.705882}%
\pgfsetfillcolor{currentfill}%
\pgfsetlinewidth{0.000000pt}%
\definecolor{currentstroke}{rgb}{0.000000,0.000000,0.000000}%
\pgfsetstrokecolor{currentstroke}%
\pgfsetstrokeopacity{0.000000}%
\pgfsetdash{}{0pt}%
\pgfpathmoveto{\pgfqpoint{16.160285in}{2.565994in}}%
\pgfpathlineto{\pgfqpoint{16.184963in}{2.565994in}}%
\pgfpathlineto{\pgfqpoint{16.184963in}{2.678467in}}%
\pgfpathlineto{\pgfqpoint{16.160285in}{2.678467in}}%
\pgfpathclose%
\pgfusepath{fill}%
\end{pgfscope}%
\begin{pgfscope}%
\pgfpathrectangle{\pgfqpoint{2.500000in}{0.625000in}}{\pgfqpoint{15.500000in}{3.775000in}}%
\pgfusepath{clip}%
\pgfsetbuttcap%
\pgfsetmiterjoin%
\definecolor{currentfill}{rgb}{0.121569,0.466667,0.705882}%
\pgfsetfillcolor{currentfill}%
\pgfsetlinewidth{0.000000pt}%
\definecolor{currentstroke}{rgb}{0.000000,0.000000,0.000000}%
\pgfsetstrokecolor{currentstroke}%
\pgfsetstrokeopacity{0.000000}%
\pgfsetdash{}{0pt}%
\pgfpathmoveto{\pgfqpoint{16.283673in}{2.565994in}}%
\pgfpathlineto{\pgfqpoint{16.308351in}{2.565994in}}%
\pgfpathlineto{\pgfqpoint{16.308351in}{2.513872in}}%
\pgfpathlineto{\pgfqpoint{16.283673in}{2.513872in}}%
\pgfpathclose%
\pgfusepath{fill}%
\end{pgfscope}%
\begin{pgfscope}%
\pgfpathrectangle{\pgfqpoint{2.500000in}{0.625000in}}{\pgfqpoint{15.500000in}{3.775000in}}%
\pgfusepath{clip}%
\pgfsetbuttcap%
\pgfsetmiterjoin%
\definecolor{currentfill}{rgb}{0.121569,0.466667,0.705882}%
\pgfsetfillcolor{currentfill}%
\pgfsetlinewidth{0.000000pt}%
\definecolor{currentstroke}{rgb}{0.000000,0.000000,0.000000}%
\pgfsetstrokecolor{currentstroke}%
\pgfsetstrokeopacity{0.000000}%
\pgfsetdash{}{0pt}%
\pgfpathmoveto{\pgfqpoint{16.407061in}{2.565994in}}%
\pgfpathlineto{\pgfqpoint{16.431739in}{2.565994in}}%
\pgfpathlineto{\pgfqpoint{16.431739in}{2.565994in}}%
\pgfpathlineto{\pgfqpoint{16.407061in}{2.565994in}}%
\pgfpathclose%
\pgfusepath{fill}%
\end{pgfscope}%
\begin{pgfscope}%
\pgfpathrectangle{\pgfqpoint{2.500000in}{0.625000in}}{\pgfqpoint{15.500000in}{3.775000in}}%
\pgfusepath{clip}%
\pgfsetbuttcap%
\pgfsetmiterjoin%
\definecolor{currentfill}{rgb}{0.121569,0.466667,0.705882}%
\pgfsetfillcolor{currentfill}%
\pgfsetlinewidth{0.000000pt}%
\definecolor{currentstroke}{rgb}{0.000000,0.000000,0.000000}%
\pgfsetstrokecolor{currentstroke}%
\pgfsetstrokeopacity{0.000000}%
\pgfsetdash{}{0pt}%
\pgfpathmoveto{\pgfqpoint{16.530449in}{2.565994in}}%
\pgfpathlineto{\pgfqpoint{16.555127in}{2.565994in}}%
\pgfpathlineto{\pgfqpoint{16.555127in}{2.656521in}}%
\pgfpathlineto{\pgfqpoint{16.530449in}{2.656521in}}%
\pgfpathclose%
\pgfusepath{fill}%
\end{pgfscope}%
\begin{pgfscope}%
\pgfpathrectangle{\pgfqpoint{2.500000in}{0.625000in}}{\pgfqpoint{15.500000in}{3.775000in}}%
\pgfusepath{clip}%
\pgfsetbuttcap%
\pgfsetmiterjoin%
\definecolor{currentfill}{rgb}{0.121569,0.466667,0.705882}%
\pgfsetfillcolor{currentfill}%
\pgfsetlinewidth{0.000000pt}%
\definecolor{currentstroke}{rgb}{0.000000,0.000000,0.000000}%
\pgfsetstrokecolor{currentstroke}%
\pgfsetstrokeopacity{0.000000}%
\pgfsetdash{}{0pt}%
\pgfpathmoveto{\pgfqpoint{16.653837in}{2.565994in}}%
\pgfpathlineto{\pgfqpoint{16.678515in}{2.565994in}}%
\pgfpathlineto{\pgfqpoint{16.678515in}{2.571480in}}%
\pgfpathlineto{\pgfqpoint{16.653837in}{2.571480in}}%
\pgfpathclose%
\pgfusepath{fill}%
\end{pgfscope}%
\begin{pgfscope}%
\pgfpathrectangle{\pgfqpoint{2.500000in}{0.625000in}}{\pgfqpoint{15.500000in}{3.775000in}}%
\pgfusepath{clip}%
\pgfsetbuttcap%
\pgfsetmiterjoin%
\definecolor{currentfill}{rgb}{0.121569,0.466667,0.705882}%
\pgfsetfillcolor{currentfill}%
\pgfsetlinewidth{0.000000pt}%
\definecolor{currentstroke}{rgb}{0.000000,0.000000,0.000000}%
\pgfsetstrokecolor{currentstroke}%
\pgfsetstrokeopacity{0.000000}%
\pgfsetdash{}{0pt}%
\pgfpathmoveto{\pgfqpoint{16.777225in}{2.565994in}}%
\pgfpathlineto{\pgfqpoint{16.801903in}{2.565994in}}%
\pgfpathlineto{\pgfqpoint{16.801903in}{0.920038in}}%
\pgfpathlineto{\pgfqpoint{16.777225in}{0.920038in}}%
\pgfpathclose%
\pgfusepath{fill}%
\end{pgfscope}%
\begin{pgfscope}%
\pgfpathrectangle{\pgfqpoint{2.500000in}{0.625000in}}{\pgfqpoint{15.500000in}{3.775000in}}%
\pgfusepath{clip}%
\pgfsetbuttcap%
\pgfsetmiterjoin%
\definecolor{currentfill}{rgb}{0.121569,0.466667,0.705882}%
\pgfsetfillcolor{currentfill}%
\pgfsetlinewidth{0.000000pt}%
\definecolor{currentstroke}{rgb}{0.000000,0.000000,0.000000}%
\pgfsetstrokecolor{currentstroke}%
\pgfsetstrokeopacity{0.000000}%
\pgfsetdash{}{0pt}%
\pgfpathmoveto{\pgfqpoint{16.900613in}{2.565994in}}%
\pgfpathlineto{\pgfqpoint{16.925291in}{2.565994in}}%
\pgfpathlineto{\pgfqpoint{16.925291in}{2.565994in}}%
\pgfpathlineto{\pgfqpoint{16.900613in}{2.565994in}}%
\pgfpathclose%
\pgfusepath{fill}%
\end{pgfscope}%
\begin{pgfscope}%
\pgfpathrectangle{\pgfqpoint{2.500000in}{0.625000in}}{\pgfqpoint{15.500000in}{3.775000in}}%
\pgfusepath{clip}%
\pgfsetbuttcap%
\pgfsetmiterjoin%
\definecolor{currentfill}{rgb}{0.121569,0.466667,0.705882}%
\pgfsetfillcolor{currentfill}%
\pgfsetlinewidth{0.000000pt}%
\definecolor{currentstroke}{rgb}{0.000000,0.000000,0.000000}%
\pgfsetstrokecolor{currentstroke}%
\pgfsetstrokeopacity{0.000000}%
\pgfsetdash{}{0pt}%
\pgfpathmoveto{\pgfqpoint{17.024001in}{2.565994in}}%
\pgfpathlineto{\pgfqpoint{17.048679in}{2.565994in}}%
\pgfpathlineto{\pgfqpoint{17.048679in}{3.775771in}}%
\pgfpathlineto{\pgfqpoint{17.024001in}{3.775771in}}%
\pgfpathclose%
\pgfusepath{fill}%
\end{pgfscope}%
\begin{pgfscope}%
\pgfpathrectangle{\pgfqpoint{2.500000in}{0.625000in}}{\pgfqpoint{15.500000in}{3.775000in}}%
\pgfusepath{clip}%
\pgfsetbuttcap%
\pgfsetmiterjoin%
\definecolor{currentfill}{rgb}{0.121569,0.466667,0.705882}%
\pgfsetfillcolor{currentfill}%
\pgfsetlinewidth{0.000000pt}%
\definecolor{currentstroke}{rgb}{0.000000,0.000000,0.000000}%
\pgfsetstrokecolor{currentstroke}%
\pgfsetstrokeopacity{0.000000}%
\pgfsetdash{}{0pt}%
\pgfpathmoveto{\pgfqpoint{17.147389in}{2.565994in}}%
\pgfpathlineto{\pgfqpoint{17.172067in}{2.565994in}}%
\pgfpathlineto{\pgfqpoint{17.172067in}{2.568737in}}%
\pgfpathlineto{\pgfqpoint{17.147389in}{2.568737in}}%
\pgfpathclose%
\pgfusepath{fill}%
\end{pgfscope}%
\begin{pgfscope}%
\pgfpathrectangle{\pgfqpoint{2.500000in}{0.625000in}}{\pgfqpoint{15.500000in}{3.775000in}}%
\pgfusepath{clip}%
\pgfsetbuttcap%
\pgfsetmiterjoin%
\definecolor{currentfill}{rgb}{0.121569,0.466667,0.705882}%
\pgfsetfillcolor{currentfill}%
\pgfsetlinewidth{0.000000pt}%
\definecolor{currentstroke}{rgb}{0.000000,0.000000,0.000000}%
\pgfsetstrokecolor{currentstroke}%
\pgfsetstrokeopacity{0.000000}%
\pgfsetdash{}{0pt}%
\pgfpathmoveto{\pgfqpoint{17.270777in}{2.565994in}}%
\pgfpathlineto{\pgfqpoint{17.295455in}{2.565994in}}%
\pgfpathlineto{\pgfqpoint{17.295455in}{2.398655in}}%
\pgfpathlineto{\pgfqpoint{17.270777in}{2.398655in}}%
\pgfpathclose%
\pgfusepath{fill}%
\end{pgfscope}%
\begin{pgfscope}%
\pgfsetbuttcap%
\pgfsetroundjoin%
\definecolor{currentfill}{rgb}{0.000000,0.000000,0.000000}%
\pgfsetfillcolor{currentfill}%
\pgfsetlinewidth{0.803000pt}%
\definecolor{currentstroke}{rgb}{0.000000,0.000000,0.000000}%
\pgfsetstrokecolor{currentstroke}%
\pgfsetdash{}{0pt}%
\pgfsys@defobject{currentmarker}{\pgfqpoint{0.000000in}{-0.048611in}}{\pgfqpoint{0.000000in}{0.000000in}}{%
\pgfpathmoveto{\pgfqpoint{0.000000in}{0.000000in}}%
\pgfpathlineto{\pgfqpoint{0.000000in}{-0.048611in}}%
\pgfusepath{stroke,fill}%
}%
\begin{pgfscope}%
\pgfsys@transformshift{3.216884in}{0.625000in}%
\pgfsys@useobject{currentmarker}{}%
\end{pgfscope}%
\end{pgfscope}%
\begin{pgfscope}%
\definecolor{textcolor}{rgb}{0.000000,0.000000,0.000000}%
\pgfsetstrokecolor{textcolor}%
\pgfsetfillcolor{textcolor}%
\pgftext[x=3.250843in, y=-1.433333in, left, base,rotate=90.000000]{\color{textcolor}\rmfamily\fontsize{10.000000}{12.000000}\selectfont jp.co.canon.oip.android.opal.apk}%
\end{pgfscope}%
\begin{pgfscope}%
\pgfsetbuttcap%
\pgfsetroundjoin%
\definecolor{currentfill}{rgb}{0.000000,0.000000,0.000000}%
\pgfsetfillcolor{currentfill}%
\pgfsetlinewidth{0.803000pt}%
\definecolor{currentstroke}{rgb}{0.000000,0.000000,0.000000}%
\pgfsetstrokecolor{currentstroke}%
\pgfsetdash{}{0pt}%
\pgfsys@defobject{currentmarker}{\pgfqpoint{0.000000in}{-0.048611in}}{\pgfqpoint{0.000000in}{0.000000in}}{%
\pgfpathmoveto{\pgfqpoint{0.000000in}{0.000000in}}%
\pgfpathlineto{\pgfqpoint{0.000000in}{-0.048611in}}%
\pgfusepath{stroke,fill}%
}%
\begin{pgfscope}%
\pgfsys@transformshift{3.340272in}{0.625000in}%
\pgfsys@useobject{currentmarker}{}%
\end{pgfscope}%
\end{pgfscope}%
\begin{pgfscope}%
\definecolor{textcolor}{rgb}{0.000000,0.000000,0.000000}%
\pgfsetstrokecolor{textcolor}%
\pgfsetfillcolor{textcolor}%
\pgftext[x=3.374994in, y=-1.244444in, left, base,rotate=90.000000]{\color{textcolor}\rmfamily\fontsize{10.000000}{12.000000}\selectfont com.nhn.android.navertv.apk}%
\end{pgfscope}%
\begin{pgfscope}%
\pgfsetbuttcap%
\pgfsetroundjoin%
\definecolor{currentfill}{rgb}{0.000000,0.000000,0.000000}%
\pgfsetfillcolor{currentfill}%
\pgfsetlinewidth{0.803000pt}%
\definecolor{currentstroke}{rgb}{0.000000,0.000000,0.000000}%
\pgfsetstrokecolor{currentstroke}%
\pgfsetdash{}{0pt}%
\pgfsys@defobject{currentmarker}{\pgfqpoint{0.000000in}{-0.048611in}}{\pgfqpoint{0.000000in}{0.000000in}}{%
\pgfpathmoveto{\pgfqpoint{0.000000in}{0.000000in}}%
\pgfpathlineto{\pgfqpoint{0.000000in}{-0.048611in}}%
\pgfusepath{stroke,fill}%
}%
\begin{pgfscope}%
\pgfsys@transformshift{3.463660in}{0.625000in}%
\pgfsys@useobject{currentmarker}{}%
\end{pgfscope}%
\end{pgfscope}%
\begin{pgfscope}%
\definecolor{textcolor}{rgb}{0.000000,0.000000,0.000000}%
\pgfsetstrokecolor{textcolor}%
\pgfsetfillcolor{textcolor}%
\pgftext[x=3.498382in, y=-0.821111in, left, base,rotate=90.000000]{\color{textcolor}\rmfamily\fontsize{10.000000}{12.000000}\selectfont ru.beeline.services.apk}%
\end{pgfscope}%
\begin{pgfscope}%
\pgfsetbuttcap%
\pgfsetroundjoin%
\definecolor{currentfill}{rgb}{0.000000,0.000000,0.000000}%
\pgfsetfillcolor{currentfill}%
\pgfsetlinewidth{0.803000pt}%
\definecolor{currentstroke}{rgb}{0.000000,0.000000,0.000000}%
\pgfsetstrokecolor{currentstroke}%
\pgfsetdash{}{0pt}%
\pgfsys@defobject{currentmarker}{\pgfqpoint{0.000000in}{-0.048611in}}{\pgfqpoint{0.000000in}{0.000000in}}{%
\pgfpathmoveto{\pgfqpoint{0.000000in}{0.000000in}}%
\pgfpathlineto{\pgfqpoint{0.000000in}{-0.048611in}}%
\pgfusepath{stroke,fill}%
}%
\begin{pgfscope}%
\pgfsys@transformshift{3.587048in}{0.625000in}%
\pgfsys@useobject{currentmarker}{}%
\end{pgfscope}%
\end{pgfscope}%
\begin{pgfscope}%
\definecolor{textcolor}{rgb}{0.000000,0.000000,0.000000}%
\pgfsetstrokecolor{textcolor}%
\pgfsetfillcolor{textcolor}%
\pgftext[x=3.622534in, y=-0.438472in, left, base,rotate=90.000000]{\color{textcolor}\rmfamily\fontsize{10.000000}{12.000000}\selectfont com.fsn.nds.apk}%
\end{pgfscope}%
\begin{pgfscope}%
\pgfsetbuttcap%
\pgfsetroundjoin%
\definecolor{currentfill}{rgb}{0.000000,0.000000,0.000000}%
\pgfsetfillcolor{currentfill}%
\pgfsetlinewidth{0.803000pt}%
\definecolor{currentstroke}{rgb}{0.000000,0.000000,0.000000}%
\pgfsetstrokecolor{currentstroke}%
\pgfsetdash{}{0pt}%
\pgfsys@defobject{currentmarker}{\pgfqpoint{0.000000in}{-0.048611in}}{\pgfqpoint{0.000000in}{0.000000in}}{%
\pgfpathmoveto{\pgfqpoint{0.000000in}{0.000000in}}%
\pgfpathlineto{\pgfqpoint{0.000000in}{-0.048611in}}%
\pgfusepath{stroke,fill}%
}%
\begin{pgfscope}%
\pgfsys@transformshift{3.710436in}{0.625000in}%
\pgfsys@useobject{currentmarker}{}%
\end{pgfscope}%
\end{pgfscope}%
\begin{pgfscope}%
\definecolor{textcolor}{rgb}{0.000000,0.000000,0.000000}%
\pgfsetstrokecolor{textcolor}%
\pgfsetfillcolor{textcolor}%
\pgftext[x=3.745089in, y=-1.156111in, left, base,rotate=90.000000]{\color{textcolor}\rmfamily\fontsize{10.000000}{12.000000}\selectfont org.familysearch.mobile.apk}%
\end{pgfscope}%
\begin{pgfscope}%
\pgfsetbuttcap%
\pgfsetroundjoin%
\definecolor{currentfill}{rgb}{0.000000,0.000000,0.000000}%
\pgfsetfillcolor{currentfill}%
\pgfsetlinewidth{0.803000pt}%
\definecolor{currentstroke}{rgb}{0.000000,0.000000,0.000000}%
\pgfsetstrokecolor{currentstroke}%
\pgfsetdash{}{0pt}%
\pgfsys@defobject{currentmarker}{\pgfqpoint{0.000000in}{-0.048611in}}{\pgfqpoint{0.000000in}{0.000000in}}{%
\pgfpathmoveto{\pgfqpoint{0.000000in}{0.000000in}}%
\pgfpathlineto{\pgfqpoint{0.000000in}{-0.048611in}}%
\pgfusepath{stroke,fill}%
}%
\begin{pgfscope}%
\pgfsys@transformshift{3.833824in}{0.625000in}%
\pgfsys@useobject{currentmarker}{}%
\end{pgfscope}%
\end{pgfscope}%
\begin{pgfscope}%
\definecolor{textcolor}{rgb}{0.000000,0.000000,0.000000}%
\pgfsetstrokecolor{textcolor}%
\pgfsetfillcolor{textcolor}%
\pgftext[x=3.869310in, y=-0.831389in, left, base,rotate=90.000000]{\color{textcolor}\rmfamily\fontsize{10.000000}{12.000000}\selectfont com.radio.fmradio.apk}%
\end{pgfscope}%
\begin{pgfscope}%
\pgfsetbuttcap%
\pgfsetroundjoin%
\definecolor{currentfill}{rgb}{0.000000,0.000000,0.000000}%
\pgfsetfillcolor{currentfill}%
\pgfsetlinewidth{0.803000pt}%
\definecolor{currentstroke}{rgb}{0.000000,0.000000,0.000000}%
\pgfsetstrokecolor{currentstroke}%
\pgfsetdash{}{0pt}%
\pgfsys@defobject{currentmarker}{\pgfqpoint{0.000000in}{-0.048611in}}{\pgfqpoint{0.000000in}{0.000000in}}{%
\pgfpathmoveto{\pgfqpoint{0.000000in}{0.000000in}}%
\pgfpathlineto{\pgfqpoint{0.000000in}{-0.048611in}}%
\pgfusepath{stroke,fill}%
}%
\begin{pgfscope}%
\pgfsys@transformshift{3.957212in}{0.625000in}%
\pgfsys@useobject{currentmarker}{}%
\end{pgfscope}%
\end{pgfscope}%
\begin{pgfscope}%
\definecolor{textcolor}{rgb}{0.000000,0.000000,0.000000}%
\pgfsetstrokecolor{textcolor}%
\pgfsetfillcolor{textcolor}%
\pgftext[x=3.991934in, y=-0.669444in, left, base,rotate=90.000000]{\color{textcolor}\rmfamily\fontsize{10.000000}{12.000000}\selectfont com.fivory.prod.apk}%
\end{pgfscope}%
\begin{pgfscope}%
\pgfsetbuttcap%
\pgfsetroundjoin%
\definecolor{currentfill}{rgb}{0.000000,0.000000,0.000000}%
\pgfsetfillcolor{currentfill}%
\pgfsetlinewidth{0.803000pt}%
\definecolor{currentstroke}{rgb}{0.000000,0.000000,0.000000}%
\pgfsetstrokecolor{currentstroke}%
\pgfsetdash{}{0pt}%
\pgfsys@defobject{currentmarker}{\pgfqpoint{0.000000in}{-0.048611in}}{\pgfqpoint{0.000000in}{0.000000in}}{%
\pgfpathmoveto{\pgfqpoint{0.000000in}{0.000000in}}%
\pgfpathlineto{\pgfqpoint{0.000000in}{-0.048611in}}%
\pgfusepath{stroke,fill}%
}%
\begin{pgfscope}%
\pgfsys@transformshift{4.080600in}{0.625000in}%
\pgfsys@useobject{currentmarker}{}%
\end{pgfscope}%
\end{pgfscope}%
\begin{pgfscope}%
\definecolor{textcolor}{rgb}{0.000000,0.000000,0.000000}%
\pgfsetstrokecolor{textcolor}%
\pgfsetfillcolor{textcolor}%
\pgftext[x=4.116086in, y=-1.901250in, left, base,rotate=90.000000]{\color{textcolor}\rmfamily\fontsize{10.000000}{12.000000}\selectfont com.mufumbo.android.recipe.search.apk}%
\end{pgfscope}%
\begin{pgfscope}%
\pgfsetbuttcap%
\pgfsetroundjoin%
\definecolor{currentfill}{rgb}{0.000000,0.000000,0.000000}%
\pgfsetfillcolor{currentfill}%
\pgfsetlinewidth{0.803000pt}%
\definecolor{currentstroke}{rgb}{0.000000,0.000000,0.000000}%
\pgfsetstrokecolor{currentstroke}%
\pgfsetdash{}{0pt}%
\pgfsys@defobject{currentmarker}{\pgfqpoint{0.000000in}{-0.048611in}}{\pgfqpoint{0.000000in}{0.000000in}}{%
\pgfpathmoveto{\pgfqpoint{0.000000in}{0.000000in}}%
\pgfpathlineto{\pgfqpoint{0.000000in}{-0.048611in}}%
\pgfusepath{stroke,fill}%
}%
\begin{pgfscope}%
\pgfsys@transformshift{4.203988in}{0.625000in}%
\pgfsys@useobject{currentmarker}{}%
\end{pgfscope}%
\end{pgfscope}%
\begin{pgfscope}%
\definecolor{textcolor}{rgb}{0.000000,0.000000,0.000000}%
\pgfsetstrokecolor{textcolor}%
\pgfsetfillcolor{textcolor}%
\pgftext[x=4.237877in, y=-1.031944in, left, base,rotate=90.000000]{\color{textcolor}\rmfamily\fontsize{10.000000}{12.000000}\selectfont com.tacobell.ordering.apk}%
\end{pgfscope}%
\begin{pgfscope}%
\pgfsetbuttcap%
\pgfsetroundjoin%
\definecolor{currentfill}{rgb}{0.000000,0.000000,0.000000}%
\pgfsetfillcolor{currentfill}%
\pgfsetlinewidth{0.803000pt}%
\definecolor{currentstroke}{rgb}{0.000000,0.000000,0.000000}%
\pgfsetstrokecolor{currentstroke}%
\pgfsetdash{}{0pt}%
\pgfsys@defobject{currentmarker}{\pgfqpoint{0.000000in}{-0.048611in}}{\pgfqpoint{0.000000in}{0.000000in}}{%
\pgfpathmoveto{\pgfqpoint{0.000000in}{0.000000in}}%
\pgfpathlineto{\pgfqpoint{0.000000in}{-0.048611in}}%
\pgfusepath{stroke,fill}%
}%
\begin{pgfscope}%
\pgfsys@transformshift{4.327376in}{0.625000in}%
\pgfsys@useobject{currentmarker}{}%
\end{pgfscope}%
\end{pgfscope}%
\begin{pgfscope}%
\definecolor{textcolor}{rgb}{0.000000,0.000000,0.000000}%
\pgfsetstrokecolor{textcolor}%
\pgfsetfillcolor{textcolor}%
\pgftext[x=4.361265in, y=-2.252916in, left, base,rotate=90.000000]{\color{textcolor}\rmfamily\fontsize{10.000000}{12.000000}\selectfont com.backgrounderaser.cutout.photoeditor.apk}%
\end{pgfscope}%
\begin{pgfscope}%
\pgfsetbuttcap%
\pgfsetroundjoin%
\definecolor{currentfill}{rgb}{0.000000,0.000000,0.000000}%
\pgfsetfillcolor{currentfill}%
\pgfsetlinewidth{0.803000pt}%
\definecolor{currentstroke}{rgb}{0.000000,0.000000,0.000000}%
\pgfsetstrokecolor{currentstroke}%
\pgfsetdash{}{0pt}%
\pgfsys@defobject{currentmarker}{\pgfqpoint{0.000000in}{-0.048611in}}{\pgfqpoint{0.000000in}{0.000000in}}{%
\pgfpathmoveto{\pgfqpoint{0.000000in}{0.000000in}}%
\pgfpathlineto{\pgfqpoint{0.000000in}{-0.048611in}}%
\pgfusepath{stroke,fill}%
}%
\begin{pgfscope}%
\pgfsys@transformshift{4.450764in}{0.625000in}%
\pgfsys@useobject{currentmarker}{}%
\end{pgfscope}%
\end{pgfscope}%
\begin{pgfscope}%
\definecolor{textcolor}{rgb}{0.000000,0.000000,0.000000}%
\pgfsetstrokecolor{textcolor}%
\pgfsetfillcolor{textcolor}%
\pgftext[x=4.485486in, y=-2.224166in, left, base,rotate=90.000000]{\color{textcolor}\rmfamily\fontsize{10.000000}{12.000000}\selectfont com.newspaperdirect.pressreader.android.apk}%
\end{pgfscope}%
\begin{pgfscope}%
\pgfsetbuttcap%
\pgfsetroundjoin%
\definecolor{currentfill}{rgb}{0.000000,0.000000,0.000000}%
\pgfsetfillcolor{currentfill}%
\pgfsetlinewidth{0.803000pt}%
\definecolor{currentstroke}{rgb}{0.000000,0.000000,0.000000}%
\pgfsetstrokecolor{currentstroke}%
\pgfsetdash{}{0pt}%
\pgfsys@defobject{currentmarker}{\pgfqpoint{0.000000in}{-0.048611in}}{\pgfqpoint{0.000000in}{0.000000in}}{%
\pgfpathmoveto{\pgfqpoint{0.000000in}{0.000000in}}%
\pgfpathlineto{\pgfqpoint{0.000000in}{-0.048611in}}%
\pgfusepath{stroke,fill}%
}%
\begin{pgfscope}%
\pgfsys@transformshift{4.574152in}{0.625000in}%
\pgfsys@useobject{currentmarker}{}%
\end{pgfscope}%
\end{pgfscope}%
\begin{pgfscope}%
\definecolor{textcolor}{rgb}{0.000000,0.000000,0.000000}%
\pgfsetstrokecolor{textcolor}%
\pgfsetfillcolor{textcolor}%
\pgftext[x=4.608874in, y=-1.774861in, left, base,rotate=90.000000]{\color{textcolor}\rmfamily\fontsize{10.000000}{12.000000}\selectfont com.khorasannews.akharinkhabar.apk}%
\end{pgfscope}%
\begin{pgfscope}%
\pgfsetbuttcap%
\pgfsetroundjoin%
\definecolor{currentfill}{rgb}{0.000000,0.000000,0.000000}%
\pgfsetfillcolor{currentfill}%
\pgfsetlinewidth{0.803000pt}%
\definecolor{currentstroke}{rgb}{0.000000,0.000000,0.000000}%
\pgfsetstrokecolor{currentstroke}%
\pgfsetdash{}{0pt}%
\pgfsys@defobject{currentmarker}{\pgfqpoint{0.000000in}{-0.048611in}}{\pgfqpoint{0.000000in}{0.000000in}}{%
\pgfpathmoveto{\pgfqpoint{0.000000in}{0.000000in}}%
\pgfpathlineto{\pgfqpoint{0.000000in}{-0.048611in}}%
\pgfusepath{stroke,fill}%
}%
\begin{pgfscope}%
\pgfsys@transformshift{4.697540in}{0.625000in}%
\pgfsys@useobject{currentmarker}{}%
\end{pgfscope}%
\end{pgfscope}%
\begin{pgfscope}%
\definecolor{textcolor}{rgb}{0.000000,0.000000,0.000000}%
\pgfsetstrokecolor{textcolor}%
\pgfsetfillcolor{textcolor}%
\pgftext[x=4.733026in, y=-1.719028in, left, base,rotate=90.000000]{\color{textcolor}\rmfamily\fontsize{10.000000}{12.000000}\selectfont com.microsoft.office.officehubrow.apk}%
\end{pgfscope}%
\begin{pgfscope}%
\pgfsetbuttcap%
\pgfsetroundjoin%
\definecolor{currentfill}{rgb}{0.000000,0.000000,0.000000}%
\pgfsetfillcolor{currentfill}%
\pgfsetlinewidth{0.803000pt}%
\definecolor{currentstroke}{rgb}{0.000000,0.000000,0.000000}%
\pgfsetstrokecolor{currentstroke}%
\pgfsetdash{}{0pt}%
\pgfsys@defobject{currentmarker}{\pgfqpoint{0.000000in}{-0.048611in}}{\pgfqpoint{0.000000in}{0.000000in}}{%
\pgfpathmoveto{\pgfqpoint{0.000000in}{0.000000in}}%
\pgfpathlineto{\pgfqpoint{0.000000in}{-0.048611in}}%
\pgfusepath{stroke,fill}%
}%
\begin{pgfscope}%
\pgfsys@transformshift{4.820928in}{0.625000in}%
\pgfsys@useobject{currentmarker}{}%
\end{pgfscope}%
\end{pgfscope}%
\begin{pgfscope}%
\definecolor{textcolor}{rgb}{0.000000,0.000000,0.000000}%
\pgfsetstrokecolor{textcolor}%
\pgfsetfillcolor{textcolor}%
\pgftext[x=4.855581in, y=-2.016527in, left, base,rotate=90.000000]{\color{textcolor}\rmfamily\fontsize{10.000000}{12.000000}\selectfont com.amazon.mShop.android.shopping.apk}%
\end{pgfscope}%
\begin{pgfscope}%
\pgfsetbuttcap%
\pgfsetroundjoin%
\definecolor{currentfill}{rgb}{0.000000,0.000000,0.000000}%
\pgfsetfillcolor{currentfill}%
\pgfsetlinewidth{0.803000pt}%
\definecolor{currentstroke}{rgb}{0.000000,0.000000,0.000000}%
\pgfsetstrokecolor{currentstroke}%
\pgfsetdash{}{0pt}%
\pgfsys@defobject{currentmarker}{\pgfqpoint{0.000000in}{-0.048611in}}{\pgfqpoint{0.000000in}{0.000000in}}{%
\pgfpathmoveto{\pgfqpoint{0.000000in}{0.000000in}}%
\pgfpathlineto{\pgfqpoint{0.000000in}{-0.048611in}}%
\pgfusepath{stroke,fill}%
}%
\begin{pgfscope}%
\pgfsys@transformshift{4.944316in}{0.625000in}%
\pgfsys@useobject{currentmarker}{}%
\end{pgfscope}%
\end{pgfscope}%
\begin{pgfscope}%
\definecolor{textcolor}{rgb}{0.000000,0.000000,0.000000}%
\pgfsetstrokecolor{textcolor}%
\pgfsetfillcolor{textcolor}%
\pgftext[x=4.979802in, y=-1.638611in, left, base,rotate=90.000000]{\color{textcolor}\rmfamily\fontsize{10.000000}{12.000000}\selectfont au.com.parrotfish.phonemic.lite.apk}%
\end{pgfscope}%
\begin{pgfscope}%
\pgfsetbuttcap%
\pgfsetroundjoin%
\definecolor{currentfill}{rgb}{0.000000,0.000000,0.000000}%
\pgfsetfillcolor{currentfill}%
\pgfsetlinewidth{0.803000pt}%
\definecolor{currentstroke}{rgb}{0.000000,0.000000,0.000000}%
\pgfsetstrokecolor{currentstroke}%
\pgfsetdash{}{0pt}%
\pgfsys@defobject{currentmarker}{\pgfqpoint{0.000000in}{-0.048611in}}{\pgfqpoint{0.000000in}{0.000000in}}{%
\pgfpathmoveto{\pgfqpoint{0.000000in}{0.000000in}}%
\pgfpathlineto{\pgfqpoint{0.000000in}{-0.048611in}}%
\pgfusepath{stroke,fill}%
}%
\begin{pgfscope}%
\pgfsys@transformshift{5.067704in}{0.625000in}%
\pgfsys@useobject{currentmarker}{}%
\end{pgfscope}%
\end{pgfscope}%
\begin{pgfscope}%
\definecolor{textcolor}{rgb}{0.000000,0.000000,0.000000}%
\pgfsetstrokecolor{textcolor}%
\pgfsetfillcolor{textcolor}%
\pgftext[x=5.102426in, y=-0.742361in, left, base,rotate=90.000000]{\color{textcolor}\rmfamily\fontsize{10.000000}{12.000000}\selectfont com.autoscout24.apk}%
\end{pgfscope}%
\begin{pgfscope}%
\pgfsetbuttcap%
\pgfsetroundjoin%
\definecolor{currentfill}{rgb}{0.000000,0.000000,0.000000}%
\pgfsetfillcolor{currentfill}%
\pgfsetlinewidth{0.803000pt}%
\definecolor{currentstroke}{rgb}{0.000000,0.000000,0.000000}%
\pgfsetstrokecolor{currentstroke}%
\pgfsetdash{}{0pt}%
\pgfsys@defobject{currentmarker}{\pgfqpoint{0.000000in}{-0.048611in}}{\pgfqpoint{0.000000in}{0.000000in}}{%
\pgfpathmoveto{\pgfqpoint{0.000000in}{0.000000in}}%
\pgfpathlineto{\pgfqpoint{0.000000in}{-0.048611in}}%
\pgfusepath{stroke,fill}%
}%
\begin{pgfscope}%
\pgfsys@transformshift{5.191092in}{0.625000in}%
\pgfsys@useobject{currentmarker}{}%
\end{pgfscope}%
\end{pgfscope}%
\begin{pgfscope}%
\definecolor{textcolor}{rgb}{0.000000,0.000000,0.000000}%
\pgfsetstrokecolor{textcolor}%
\pgfsetfillcolor{textcolor}%
\pgftext[x=5.225814in, y=-1.009027in, left, base,rotate=90.000000]{\color{textcolor}\rmfamily\fontsize{10.000000}{12.000000}\selectfont de.number26.android.apk}%
\end{pgfscope}%
\begin{pgfscope}%
\pgfsetbuttcap%
\pgfsetroundjoin%
\definecolor{currentfill}{rgb}{0.000000,0.000000,0.000000}%
\pgfsetfillcolor{currentfill}%
\pgfsetlinewidth{0.803000pt}%
\definecolor{currentstroke}{rgb}{0.000000,0.000000,0.000000}%
\pgfsetstrokecolor{currentstroke}%
\pgfsetdash{}{0pt}%
\pgfsys@defobject{currentmarker}{\pgfqpoint{0.000000in}{-0.048611in}}{\pgfqpoint{0.000000in}{0.000000in}}{%
\pgfpathmoveto{\pgfqpoint{0.000000in}{0.000000in}}%
\pgfpathlineto{\pgfqpoint{0.000000in}{-0.048611in}}%
\pgfusepath{stroke,fill}%
}%
\begin{pgfscope}%
\pgfsys@transformshift{5.314480in}{0.625000in}%
\pgfsys@useobject{currentmarker}{}%
\end{pgfscope}%
\end{pgfscope}%
\begin{pgfscope}%
\definecolor{textcolor}{rgb}{0.000000,0.000000,0.000000}%
\pgfsetstrokecolor{textcolor}%
\pgfsetfillcolor{textcolor}%
\pgftext[x=5.349202in, y=-0.456805in, left, base,rotate=90.000000]{\color{textcolor}\rmfamily\fontsize{10.000000}{12.000000}\selectfont hr.palamida.apk}%
\end{pgfscope}%
\begin{pgfscope}%
\pgfsetbuttcap%
\pgfsetroundjoin%
\definecolor{currentfill}{rgb}{0.000000,0.000000,0.000000}%
\pgfsetfillcolor{currentfill}%
\pgfsetlinewidth{0.803000pt}%
\definecolor{currentstroke}{rgb}{0.000000,0.000000,0.000000}%
\pgfsetstrokecolor{currentstroke}%
\pgfsetdash{}{0pt}%
\pgfsys@defobject{currentmarker}{\pgfqpoint{0.000000in}{-0.048611in}}{\pgfqpoint{0.000000in}{0.000000in}}{%
\pgfpathmoveto{\pgfqpoint{0.000000in}{0.000000in}}%
\pgfpathlineto{\pgfqpoint{0.000000in}{-0.048611in}}%
\pgfusepath{stroke,fill}%
}%
\begin{pgfscope}%
\pgfsys@transformshift{5.437868in}{0.625000in}%
\pgfsys@useobject{currentmarker}{}%
\end{pgfscope}%
\end{pgfscope}%
\begin{pgfscope}%
\definecolor{textcolor}{rgb}{0.000000,0.000000,0.000000}%
\pgfsetstrokecolor{textcolor}%
\pgfsetfillcolor{textcolor}%
\pgftext[x=5.472590in, y=-0.997222in, left, base,rotate=90.000000]{\color{textcolor}\rmfamily\fontsize{10.000000}{12.000000}\selectfont com.kinky.fetlifestyle.apk}%
\end{pgfscope}%
\begin{pgfscope}%
\pgfsetbuttcap%
\pgfsetroundjoin%
\definecolor{currentfill}{rgb}{0.000000,0.000000,0.000000}%
\pgfsetfillcolor{currentfill}%
\pgfsetlinewidth{0.803000pt}%
\definecolor{currentstroke}{rgb}{0.000000,0.000000,0.000000}%
\pgfsetstrokecolor{currentstroke}%
\pgfsetdash{}{0pt}%
\pgfsys@defobject{currentmarker}{\pgfqpoint{0.000000in}{-0.048611in}}{\pgfqpoint{0.000000in}{0.000000in}}{%
\pgfpathmoveto{\pgfqpoint{0.000000in}{0.000000in}}%
\pgfpathlineto{\pgfqpoint{0.000000in}{-0.048611in}}%
\pgfusepath{stroke,fill}%
}%
\begin{pgfscope}%
\pgfsys@transformshift{5.561256in}{0.625000in}%
\pgfsys@useobject{currentmarker}{}%
\end{pgfscope}%
\end{pgfscope}%
\begin{pgfscope}%
\definecolor{textcolor}{rgb}{0.000000,0.000000,0.000000}%
\pgfsetstrokecolor{textcolor}%
\pgfsetfillcolor{textcolor}%
\pgftext[x=5.595978in, y=-1.334583in, left, base,rotate=90.000000]{\color{textcolor}\rmfamily\fontsize{10.000000}{12.000000}\selectfont com.indianexpress.android.apk}%
\end{pgfscope}%
\begin{pgfscope}%
\pgfsetbuttcap%
\pgfsetroundjoin%
\definecolor{currentfill}{rgb}{0.000000,0.000000,0.000000}%
\pgfsetfillcolor{currentfill}%
\pgfsetlinewidth{0.803000pt}%
\definecolor{currentstroke}{rgb}{0.000000,0.000000,0.000000}%
\pgfsetstrokecolor{currentstroke}%
\pgfsetdash{}{0pt}%
\pgfsys@defobject{currentmarker}{\pgfqpoint{0.000000in}{-0.048611in}}{\pgfqpoint{0.000000in}{0.000000in}}{%
\pgfpathmoveto{\pgfqpoint{0.000000in}{0.000000in}}%
\pgfpathlineto{\pgfqpoint{0.000000in}{-0.048611in}}%
\pgfusepath{stroke,fill}%
}%
\begin{pgfscope}%
\pgfsys@transformshift{5.684644in}{0.625000in}%
\pgfsys@useobject{currentmarker}{}%
\end{pgfscope}%
\end{pgfscope}%
\begin{pgfscope}%
\definecolor{textcolor}{rgb}{0.000000,0.000000,0.000000}%
\pgfsetstrokecolor{textcolor}%
\pgfsetfillcolor{textcolor}%
\pgftext[x=5.718602in, y=-0.838889in, left, base,rotate=90.000000]{\color{textcolor}\rmfamily\fontsize{10.000000}{12.000000}\selectfont cz.seznam.novinky.apk}%
\end{pgfscope}%
\begin{pgfscope}%
\pgfsetbuttcap%
\pgfsetroundjoin%
\definecolor{currentfill}{rgb}{0.000000,0.000000,0.000000}%
\pgfsetfillcolor{currentfill}%
\pgfsetlinewidth{0.803000pt}%
\definecolor{currentstroke}{rgb}{0.000000,0.000000,0.000000}%
\pgfsetstrokecolor{currentstroke}%
\pgfsetdash{}{0pt}%
\pgfsys@defobject{currentmarker}{\pgfqpoint{0.000000in}{-0.048611in}}{\pgfqpoint{0.000000in}{0.000000in}}{%
\pgfpathmoveto{\pgfqpoint{0.000000in}{0.000000in}}%
\pgfpathlineto{\pgfqpoint{0.000000in}{-0.048611in}}%
\pgfusepath{stroke,fill}%
}%
\begin{pgfscope}%
\pgfsys@transformshift{5.808032in}{0.625000in}%
\pgfsys@useobject{currentmarker}{}%
\end{pgfscope}%
\end{pgfscope}%
\begin{pgfscope}%
\definecolor{textcolor}{rgb}{0.000000,0.000000,0.000000}%
\pgfsetstrokecolor{textcolor}%
\pgfsetfillcolor{textcolor}%
\pgftext[x=5.842754in, y=-1.049306in, left, base,rotate=90.000000]{\color{textcolor}\rmfamily\fontsize{10.000000}{12.000000}\selectfont com.euronews.express.apk}%
\end{pgfscope}%
\begin{pgfscope}%
\pgfsetbuttcap%
\pgfsetroundjoin%
\definecolor{currentfill}{rgb}{0.000000,0.000000,0.000000}%
\pgfsetfillcolor{currentfill}%
\pgfsetlinewidth{0.803000pt}%
\definecolor{currentstroke}{rgb}{0.000000,0.000000,0.000000}%
\pgfsetstrokecolor{currentstroke}%
\pgfsetdash{}{0pt}%
\pgfsys@defobject{currentmarker}{\pgfqpoint{0.000000in}{-0.048611in}}{\pgfqpoint{0.000000in}{0.000000in}}{%
\pgfpathmoveto{\pgfqpoint{0.000000in}{0.000000in}}%
\pgfpathlineto{\pgfqpoint{0.000000in}{-0.048611in}}%
\pgfusepath{stroke,fill}%
}%
\begin{pgfscope}%
\pgfsys@transformshift{5.931420in}{0.625000in}%
\pgfsys@useobject{currentmarker}{}%
\end{pgfscope}%
\end{pgfscope}%
\begin{pgfscope}%
\definecolor{textcolor}{rgb}{0.000000,0.000000,0.000000}%
\pgfsetstrokecolor{textcolor}%
\pgfsetfillcolor{textcolor}%
\pgftext[x=5.966073in, y=-0.813055in, left, base,rotate=90.000000]{\color{textcolor}\rmfamily\fontsize{10.000000}{12.000000}\selectfont cn.wps.pdf.fillsign.apk}%
\end{pgfscope}%
\begin{pgfscope}%
\pgfsetbuttcap%
\pgfsetroundjoin%
\definecolor{currentfill}{rgb}{0.000000,0.000000,0.000000}%
\pgfsetfillcolor{currentfill}%
\pgfsetlinewidth{0.803000pt}%
\definecolor{currentstroke}{rgb}{0.000000,0.000000,0.000000}%
\pgfsetstrokecolor{currentstroke}%
\pgfsetdash{}{0pt}%
\pgfsys@defobject{currentmarker}{\pgfqpoint{0.000000in}{-0.048611in}}{\pgfqpoint{0.000000in}{0.000000in}}{%
\pgfpathmoveto{\pgfqpoint{0.000000in}{0.000000in}}%
\pgfpathlineto{\pgfqpoint{0.000000in}{-0.048611in}}%
\pgfusepath{stroke,fill}%
}%
\begin{pgfscope}%
\pgfsys@transformshift{6.054808in}{0.625000in}%
\pgfsys@useobject{currentmarker}{}%
\end{pgfscope}%
\end{pgfscope}%
\begin{pgfscope}%
\definecolor{textcolor}{rgb}{0.000000,0.000000,0.000000}%
\pgfsetstrokecolor{textcolor}%
\pgfsetfillcolor{textcolor}%
\pgftext[x=6.089530in, y=-1.271389in, left, base,rotate=90.000000]{\color{textcolor}\rmfamily\fontsize{10.000000}{12.000000}\selectfont com.apartmentlist.mobile.apk}%
\end{pgfscope}%
\begin{pgfscope}%
\pgfsetbuttcap%
\pgfsetroundjoin%
\definecolor{currentfill}{rgb}{0.000000,0.000000,0.000000}%
\pgfsetfillcolor{currentfill}%
\pgfsetlinewidth{0.803000pt}%
\definecolor{currentstroke}{rgb}{0.000000,0.000000,0.000000}%
\pgfsetstrokecolor{currentstroke}%
\pgfsetdash{}{0pt}%
\pgfsys@defobject{currentmarker}{\pgfqpoint{0.000000in}{-0.048611in}}{\pgfqpoint{0.000000in}{0.000000in}}{%
\pgfpathmoveto{\pgfqpoint{0.000000in}{0.000000in}}%
\pgfpathlineto{\pgfqpoint{0.000000in}{-0.048611in}}%
\pgfusepath{stroke,fill}%
}%
\begin{pgfscope}%
\pgfsys@transformshift{6.178196in}{0.625000in}%
\pgfsys@useobject{currentmarker}{}%
\end{pgfscope}%
\end{pgfscope}%
\begin{pgfscope}%
\definecolor{textcolor}{rgb}{0.000000,0.000000,0.000000}%
\pgfsetstrokecolor{textcolor}%
\pgfsetfillcolor{textcolor}%
\pgftext[x=6.212918in, y=-1.101111in, left, base,rotate=90.000000]{\color{textcolor}\rmfamily\fontsize{10.000000}{12.000000}\selectfont tw.com.ctitv.ctitvnews.apk}%
\end{pgfscope}%
\begin{pgfscope}%
\pgfsetbuttcap%
\pgfsetroundjoin%
\definecolor{currentfill}{rgb}{0.000000,0.000000,0.000000}%
\pgfsetfillcolor{currentfill}%
\pgfsetlinewidth{0.803000pt}%
\definecolor{currentstroke}{rgb}{0.000000,0.000000,0.000000}%
\pgfsetstrokecolor{currentstroke}%
\pgfsetdash{}{0pt}%
\pgfsys@defobject{currentmarker}{\pgfqpoint{0.000000in}{-0.048611in}}{\pgfqpoint{0.000000in}{0.000000in}}{%
\pgfpathmoveto{\pgfqpoint{0.000000in}{0.000000in}}%
\pgfpathlineto{\pgfqpoint{0.000000in}{-0.048611in}}%
\pgfusepath{stroke,fill}%
}%
\begin{pgfscope}%
\pgfsys@transformshift{6.301584in}{0.625000in}%
\pgfsys@useobject{currentmarker}{}%
\end{pgfscope}%
\end{pgfscope}%
\begin{pgfscope}%
\definecolor{textcolor}{rgb}{0.000000,0.000000,0.000000}%
\pgfsetstrokecolor{textcolor}%
\pgfsetfillcolor{textcolor}%
\pgftext[x=6.337834in, y=-1.252778in, left, base,rotate=90.000000]{\color{textcolor}\rmfamily\fontsize{10.000000}{12.000000}\selectfont nineNewsAlerts.nine.com.apk}%
\end{pgfscope}%
\begin{pgfscope}%
\pgfsetbuttcap%
\pgfsetroundjoin%
\definecolor{currentfill}{rgb}{0.000000,0.000000,0.000000}%
\pgfsetfillcolor{currentfill}%
\pgfsetlinewidth{0.803000pt}%
\definecolor{currentstroke}{rgb}{0.000000,0.000000,0.000000}%
\pgfsetstrokecolor{currentstroke}%
\pgfsetdash{}{0pt}%
\pgfsys@defobject{currentmarker}{\pgfqpoint{0.000000in}{-0.048611in}}{\pgfqpoint{0.000000in}{0.000000in}}{%
\pgfpathmoveto{\pgfqpoint{0.000000in}{0.000000in}}%
\pgfpathlineto{\pgfqpoint{0.000000in}{-0.048611in}}%
\pgfusepath{stroke,fill}%
}%
\begin{pgfscope}%
\pgfsys@transformshift{6.424972in}{0.625000in}%
\pgfsys@useobject{currentmarker}{}%
\end{pgfscope}%
\end{pgfscope}%
\begin{pgfscope}%
\definecolor{textcolor}{rgb}{0.000000,0.000000,0.000000}%
\pgfsetstrokecolor{textcolor}%
\pgfsetfillcolor{textcolor}%
\pgftext[x=6.459694in, y=-0.966250in, left, base,rotate=90.000000]{\color{textcolor}\rmfamily\fontsize{10.000000}{12.000000}\selectfont com.hk01.news\_app.apk}%
\end{pgfscope}%
\begin{pgfscope}%
\pgfsetbuttcap%
\pgfsetroundjoin%
\definecolor{currentfill}{rgb}{0.000000,0.000000,0.000000}%
\pgfsetfillcolor{currentfill}%
\pgfsetlinewidth{0.803000pt}%
\definecolor{currentstroke}{rgb}{0.000000,0.000000,0.000000}%
\pgfsetstrokecolor{currentstroke}%
\pgfsetdash{}{0pt}%
\pgfsys@defobject{currentmarker}{\pgfqpoint{0.000000in}{-0.048611in}}{\pgfqpoint{0.000000in}{0.000000in}}{%
\pgfpathmoveto{\pgfqpoint{0.000000in}{0.000000in}}%
\pgfpathlineto{\pgfqpoint{0.000000in}{-0.048611in}}%
\pgfusepath{stroke,fill}%
}%
\begin{pgfscope}%
\pgfsys@transformshift{6.548360in}{0.625000in}%
\pgfsys@useobject{currentmarker}{}%
\end{pgfscope}%
\end{pgfscope}%
\begin{pgfscope}%
\definecolor{textcolor}{rgb}{0.000000,0.000000,0.000000}%
\pgfsetstrokecolor{textcolor}%
\pgfsetfillcolor{textcolor}%
\pgftext[x=6.584610in, y=-0.537916in, left, base,rotate=90.000000]{\color{textcolor}\rmfamily\fontsize{10.000000}{12.000000}\selectfont mnn.Android.apk}%
\end{pgfscope}%
\begin{pgfscope}%
\pgfsetbuttcap%
\pgfsetroundjoin%
\definecolor{currentfill}{rgb}{0.000000,0.000000,0.000000}%
\pgfsetfillcolor{currentfill}%
\pgfsetlinewidth{0.803000pt}%
\definecolor{currentstroke}{rgb}{0.000000,0.000000,0.000000}%
\pgfsetstrokecolor{currentstroke}%
\pgfsetdash{}{0pt}%
\pgfsys@defobject{currentmarker}{\pgfqpoint{0.000000in}{-0.048611in}}{\pgfqpoint{0.000000in}{0.000000in}}{%
\pgfpathmoveto{\pgfqpoint{0.000000in}{0.000000in}}%
\pgfpathlineto{\pgfqpoint{0.000000in}{-0.048611in}}%
\pgfusepath{stroke,fill}%
}%
\begin{pgfscope}%
\pgfsys@transformshift{6.671748in}{0.625000in}%
\pgfsys@useobject{currentmarker}{}%
\end{pgfscope}%
\end{pgfscope}%
\begin{pgfscope}%
\definecolor{textcolor}{rgb}{0.000000,0.000000,0.000000}%
\pgfsetstrokecolor{textcolor}%
\pgfsetfillcolor{textcolor}%
\pgftext[x=6.706470in, y=-1.573472in, left, base,rotate=90.000000]{\color{textcolor}\rmfamily\fontsize{10.000000}{12.000000}\selectfont com.ubieva.cura.userapp.droid.apk}%
\end{pgfscope}%
\begin{pgfscope}%
\pgfsetbuttcap%
\pgfsetroundjoin%
\definecolor{currentfill}{rgb}{0.000000,0.000000,0.000000}%
\pgfsetfillcolor{currentfill}%
\pgfsetlinewidth{0.803000pt}%
\definecolor{currentstroke}{rgb}{0.000000,0.000000,0.000000}%
\pgfsetstrokecolor{currentstroke}%
\pgfsetdash{}{0pt}%
\pgfsys@defobject{currentmarker}{\pgfqpoint{0.000000in}{-0.048611in}}{\pgfqpoint{0.000000in}{0.000000in}}{%
\pgfpathmoveto{\pgfqpoint{0.000000in}{0.000000in}}%
\pgfpathlineto{\pgfqpoint{0.000000in}{-0.048611in}}%
\pgfusepath{stroke,fill}%
}%
\begin{pgfscope}%
\pgfsys@transformshift{6.795136in}{0.625000in}%
\pgfsys@useobject{currentmarker}{}%
\end{pgfscope}%
\end{pgfscope}%
\begin{pgfscope}%
\definecolor{textcolor}{rgb}{0.000000,0.000000,0.000000}%
\pgfsetstrokecolor{textcolor}%
\pgfsetfillcolor{textcolor}%
\pgftext[x=6.829858in, y=-1.240417in, left, base,rotate=90.000000]{\color{textcolor}\rmfamily\fontsize{10.000000}{12.000000}\selectfont com.simplepractice.video.apk}%
\end{pgfscope}%
\begin{pgfscope}%
\pgfsetbuttcap%
\pgfsetroundjoin%
\definecolor{currentfill}{rgb}{0.000000,0.000000,0.000000}%
\pgfsetfillcolor{currentfill}%
\pgfsetlinewidth{0.803000pt}%
\definecolor{currentstroke}{rgb}{0.000000,0.000000,0.000000}%
\pgfsetstrokecolor{currentstroke}%
\pgfsetdash{}{0pt}%
\pgfsys@defobject{currentmarker}{\pgfqpoint{0.000000in}{-0.048611in}}{\pgfqpoint{0.000000in}{0.000000in}}{%
\pgfpathmoveto{\pgfqpoint{0.000000in}{0.000000in}}%
\pgfpathlineto{\pgfqpoint{0.000000in}{-0.048611in}}%
\pgfusepath{stroke,fill}%
}%
\begin{pgfscope}%
\pgfsys@transformshift{6.918524in}{0.625000in}%
\pgfsys@useobject{currentmarker}{}%
\end{pgfscope}%
\end{pgfscope}%
\begin{pgfscope}%
\definecolor{textcolor}{rgb}{0.000000,0.000000,0.000000}%
\pgfsetstrokecolor{textcolor}%
\pgfsetfillcolor{textcolor}%
\pgftext[x=6.952413in, y=-1.584861in, left, base,rotate=90.000000]{\color{textcolor}\rmfamily\fontsize{10.000000}{12.000000}\selectfont com.google.android.apps.seekh.apk}%
\end{pgfscope}%
\begin{pgfscope}%
\pgfsetbuttcap%
\pgfsetroundjoin%
\definecolor{currentfill}{rgb}{0.000000,0.000000,0.000000}%
\pgfsetfillcolor{currentfill}%
\pgfsetlinewidth{0.803000pt}%
\definecolor{currentstroke}{rgb}{0.000000,0.000000,0.000000}%
\pgfsetstrokecolor{currentstroke}%
\pgfsetdash{}{0pt}%
\pgfsys@defobject{currentmarker}{\pgfqpoint{0.000000in}{-0.048611in}}{\pgfqpoint{0.000000in}{0.000000in}}{%
\pgfpathmoveto{\pgfqpoint{0.000000in}{0.000000in}}%
\pgfpathlineto{\pgfqpoint{0.000000in}{-0.048611in}}%
\pgfusepath{stroke,fill}%
}%
\begin{pgfscope}%
\pgfsys@transformshift{7.041912in}{0.625000in}%
\pgfsys@useobject{currentmarker}{}%
\end{pgfscope}%
\end{pgfscope}%
\begin{pgfscope}%
\definecolor{textcolor}{rgb}{0.000000,0.000000,0.000000}%
\pgfsetstrokecolor{textcolor}%
\pgfsetfillcolor{textcolor}%
\pgftext[x=7.076634in, y=-0.696667in, left, base,rotate=90.000000]{\color{textcolor}\rmfamily\fontsize{10.000000}{12.000000}\selectfont com.aspiro.tidal.apk}%
\end{pgfscope}%
\begin{pgfscope}%
\pgfsetbuttcap%
\pgfsetroundjoin%
\definecolor{currentfill}{rgb}{0.000000,0.000000,0.000000}%
\pgfsetfillcolor{currentfill}%
\pgfsetlinewidth{0.803000pt}%
\definecolor{currentstroke}{rgb}{0.000000,0.000000,0.000000}%
\pgfsetstrokecolor{currentstroke}%
\pgfsetdash{}{0pt}%
\pgfsys@defobject{currentmarker}{\pgfqpoint{0.000000in}{-0.048611in}}{\pgfqpoint{0.000000in}{0.000000in}}{%
\pgfpathmoveto{\pgfqpoint{0.000000in}{0.000000in}}%
\pgfpathlineto{\pgfqpoint{0.000000in}{-0.048611in}}%
\pgfusepath{stroke,fill}%
}%
\begin{pgfscope}%
\pgfsys@transformshift{7.165300in}{0.625000in}%
\pgfsys@useobject{currentmarker}{}%
\end{pgfscope}%
\end{pgfscope}%
\begin{pgfscope}%
\definecolor{textcolor}{rgb}{0.000000,0.000000,0.000000}%
\pgfsetstrokecolor{textcolor}%
\pgfsetfillcolor{textcolor}%
\pgftext[x=7.199953in, y=-1.669722in, left, base,rotate=90.000000]{\color{textcolor}\rmfamily\fontsize{10.000000}{12.000000}\selectfont com.skyhealth.glucosebuddyfree.apk}%
\end{pgfscope}%
\begin{pgfscope}%
\pgfsetbuttcap%
\pgfsetroundjoin%
\definecolor{currentfill}{rgb}{0.000000,0.000000,0.000000}%
\pgfsetfillcolor{currentfill}%
\pgfsetlinewidth{0.803000pt}%
\definecolor{currentstroke}{rgb}{0.000000,0.000000,0.000000}%
\pgfsetstrokecolor{currentstroke}%
\pgfsetdash{}{0pt}%
\pgfsys@defobject{currentmarker}{\pgfqpoint{0.000000in}{-0.048611in}}{\pgfqpoint{0.000000in}{0.000000in}}{%
\pgfpathmoveto{\pgfqpoint{0.000000in}{0.000000in}}%
\pgfpathlineto{\pgfqpoint{0.000000in}{-0.048611in}}%
\pgfusepath{stroke,fill}%
}%
\begin{pgfscope}%
\pgfsys@transformshift{7.288688in}{0.625000in}%
\pgfsys@useobject{currentmarker}{}%
\end{pgfscope}%
\end{pgfscope}%
\begin{pgfscope}%
\definecolor{textcolor}{rgb}{0.000000,0.000000,0.000000}%
\pgfsetstrokecolor{textcolor}%
\pgfsetfillcolor{textcolor}%
\pgftext[x=7.323410in, y=-0.610694in, left, base,rotate=90.000000]{\color{textcolor}\rmfamily\fontsize{10.000000}{12.000000}\selectfont com.vuclip.viu.apk}%
\end{pgfscope}%
\begin{pgfscope}%
\pgfsetbuttcap%
\pgfsetroundjoin%
\definecolor{currentfill}{rgb}{0.000000,0.000000,0.000000}%
\pgfsetfillcolor{currentfill}%
\pgfsetlinewidth{0.803000pt}%
\definecolor{currentstroke}{rgb}{0.000000,0.000000,0.000000}%
\pgfsetstrokecolor{currentstroke}%
\pgfsetdash{}{0pt}%
\pgfsys@defobject{currentmarker}{\pgfqpoint{0.000000in}{-0.048611in}}{\pgfqpoint{0.000000in}{0.000000in}}{%
\pgfpathmoveto{\pgfqpoint{0.000000in}{0.000000in}}%
\pgfpathlineto{\pgfqpoint{0.000000in}{-0.048611in}}%
\pgfusepath{stroke,fill}%
}%
\begin{pgfscope}%
\pgfsys@transformshift{7.412076in}{0.625000in}%
\pgfsys@useobject{currentmarker}{}%
\end{pgfscope}%
\end{pgfscope}%
\begin{pgfscope}%
\definecolor{textcolor}{rgb}{0.000000,0.000000,0.000000}%
\pgfsetstrokecolor{textcolor}%
\pgfsetfillcolor{textcolor}%
\pgftext[x=7.446798in, y=-1.135972in, left, base,rotate=90.000000]{\color{textcolor}\rmfamily\fontsize{10.000000}{12.000000}\selectfont com.touchtype.swiftkey.apk}%
\end{pgfscope}%
\begin{pgfscope}%
\pgfsetbuttcap%
\pgfsetroundjoin%
\definecolor{currentfill}{rgb}{0.000000,0.000000,0.000000}%
\pgfsetfillcolor{currentfill}%
\pgfsetlinewidth{0.803000pt}%
\definecolor{currentstroke}{rgb}{0.000000,0.000000,0.000000}%
\pgfsetstrokecolor{currentstroke}%
\pgfsetdash{}{0pt}%
\pgfsys@defobject{currentmarker}{\pgfqpoint{0.000000in}{-0.048611in}}{\pgfqpoint{0.000000in}{0.000000in}}{%
\pgfpathmoveto{\pgfqpoint{0.000000in}{0.000000in}}%
\pgfpathlineto{\pgfqpoint{0.000000in}{-0.048611in}}%
\pgfusepath{stroke,fill}%
}%
\begin{pgfscope}%
\pgfsys@transformshift{7.535464in}{0.625000in}%
\pgfsys@useobject{currentmarker}{}%
\end{pgfscope}%
\end{pgfscope}%
\begin{pgfscope}%
\definecolor{textcolor}{rgb}{0.000000,0.000000,0.000000}%
\pgfsetstrokecolor{textcolor}%
\pgfsetfillcolor{textcolor}%
\pgftext[x=7.569353in, y=-0.858472in, left, base,rotate=90.000000]{\color{textcolor}\rmfamily\fontsize{10.000000}{12.000000}\selectfont com.croquis.zigzag.apk}%
\end{pgfscope}%
\begin{pgfscope}%
\pgfsetbuttcap%
\pgfsetroundjoin%
\definecolor{currentfill}{rgb}{0.000000,0.000000,0.000000}%
\pgfsetfillcolor{currentfill}%
\pgfsetlinewidth{0.803000pt}%
\definecolor{currentstroke}{rgb}{0.000000,0.000000,0.000000}%
\pgfsetstrokecolor{currentstroke}%
\pgfsetdash{}{0pt}%
\pgfsys@defobject{currentmarker}{\pgfqpoint{0.000000in}{-0.048611in}}{\pgfqpoint{0.000000in}{0.000000in}}{%
\pgfpathmoveto{\pgfqpoint{0.000000in}{0.000000in}}%
\pgfpathlineto{\pgfqpoint{0.000000in}{-0.048611in}}%
\pgfusepath{stroke,fill}%
}%
\begin{pgfscope}%
\pgfsys@transformshift{7.658852in}{0.625000in}%
\pgfsys@useobject{currentmarker}{}%
\end{pgfscope}%
\end{pgfscope}%
\begin{pgfscope}%
\definecolor{textcolor}{rgb}{0.000000,0.000000,0.000000}%
\pgfsetstrokecolor{textcolor}%
\pgfsetfillcolor{textcolor}%
\pgftext[x=7.693574in, y=-1.166111in, left, base,rotate=90.000000]{\color{textcolor}\rmfamily\fontsize{10.000000}{12.000000}\selectfont com.application.zomato.apk}%
\end{pgfscope}%
\begin{pgfscope}%
\pgfsetbuttcap%
\pgfsetroundjoin%
\definecolor{currentfill}{rgb}{0.000000,0.000000,0.000000}%
\pgfsetfillcolor{currentfill}%
\pgfsetlinewidth{0.803000pt}%
\definecolor{currentstroke}{rgb}{0.000000,0.000000,0.000000}%
\pgfsetstrokecolor{currentstroke}%
\pgfsetdash{}{0pt}%
\pgfsys@defobject{currentmarker}{\pgfqpoint{0.000000in}{-0.048611in}}{\pgfqpoint{0.000000in}{0.000000in}}{%
\pgfpathmoveto{\pgfqpoint{0.000000in}{0.000000in}}%
\pgfpathlineto{\pgfqpoint{0.000000in}{-0.048611in}}%
\pgfusepath{stroke,fill}%
}%
\begin{pgfscope}%
\pgfsys@transformshift{7.782240in}{0.625000in}%
\pgfsys@useobject{currentmarker}{}%
\end{pgfscope}%
\end{pgfscope}%
\begin{pgfscope}%
\definecolor{textcolor}{rgb}{0.000000,0.000000,0.000000}%
\pgfsetstrokecolor{textcolor}%
\pgfsetfillcolor{textcolor}%
\pgftext[x=7.816962in, y=-1.222222in, left, base,rotate=90.000000]{\color{textcolor}\rmfamily\fontsize{10.000000}{12.000000}\selectfont com.streema.simpleradio.apk}%
\end{pgfscope}%
\begin{pgfscope}%
\pgfsetbuttcap%
\pgfsetroundjoin%
\definecolor{currentfill}{rgb}{0.000000,0.000000,0.000000}%
\pgfsetfillcolor{currentfill}%
\pgfsetlinewidth{0.803000pt}%
\definecolor{currentstroke}{rgb}{0.000000,0.000000,0.000000}%
\pgfsetstrokecolor{currentstroke}%
\pgfsetdash{}{0pt}%
\pgfsys@defobject{currentmarker}{\pgfqpoint{0.000000in}{-0.048611in}}{\pgfqpoint{0.000000in}{0.000000in}}{%
\pgfpathmoveto{\pgfqpoint{0.000000in}{0.000000in}}%
\pgfpathlineto{\pgfqpoint{0.000000in}{-0.048611in}}%
\pgfusepath{stroke,fill}%
}%
\begin{pgfscope}%
\pgfsys@transformshift{7.905628in}{0.625000in}%
\pgfsys@useobject{currentmarker}{}%
\end{pgfscope}%
\end{pgfscope}%
\begin{pgfscope}%
\definecolor{textcolor}{rgb}{0.000000,0.000000,0.000000}%
\pgfsetstrokecolor{textcolor}%
\pgfsetfillcolor{textcolor}%
\pgftext[x=7.939517in, y=-1.779167in, left, base,rotate=90.000000]{\color{textcolor}\rmfamily\fontsize{10.000000}{12.000000}\selectfont com.orange.kidspiano.music.songs.apk}%
\end{pgfscope}%
\begin{pgfscope}%
\pgfsetbuttcap%
\pgfsetroundjoin%
\definecolor{currentfill}{rgb}{0.000000,0.000000,0.000000}%
\pgfsetfillcolor{currentfill}%
\pgfsetlinewidth{0.803000pt}%
\definecolor{currentstroke}{rgb}{0.000000,0.000000,0.000000}%
\pgfsetstrokecolor{currentstroke}%
\pgfsetdash{}{0pt}%
\pgfsys@defobject{currentmarker}{\pgfqpoint{0.000000in}{-0.048611in}}{\pgfqpoint{0.000000in}{0.000000in}}{%
\pgfpathmoveto{\pgfqpoint{0.000000in}{0.000000in}}%
\pgfpathlineto{\pgfqpoint{0.000000in}{-0.048611in}}%
\pgfusepath{stroke,fill}%
}%
\begin{pgfscope}%
\pgfsys@transformshift{8.029016in}{0.625000in}%
\pgfsys@useobject{currentmarker}{}%
\end{pgfscope}%
\end{pgfscope}%
\begin{pgfscope}%
\definecolor{textcolor}{rgb}{0.000000,0.000000,0.000000}%
\pgfsetstrokecolor{textcolor}%
\pgfsetfillcolor{textcolor}%
\pgftext[x=8.063738in, y=-1.186111in, left, base,rotate=90.000000]{\color{textcolor}\rmfamily\fontsize{10.000000}{12.000000}\selectfont com.creditkarma.mobile.apk}%
\end{pgfscope}%
\begin{pgfscope}%
\pgfsetbuttcap%
\pgfsetroundjoin%
\definecolor{currentfill}{rgb}{0.000000,0.000000,0.000000}%
\pgfsetfillcolor{currentfill}%
\pgfsetlinewidth{0.803000pt}%
\definecolor{currentstroke}{rgb}{0.000000,0.000000,0.000000}%
\pgfsetstrokecolor{currentstroke}%
\pgfsetdash{}{0pt}%
\pgfsys@defobject{currentmarker}{\pgfqpoint{0.000000in}{-0.048611in}}{\pgfqpoint{0.000000in}{0.000000in}}{%
\pgfpathmoveto{\pgfqpoint{0.000000in}{0.000000in}}%
\pgfpathlineto{\pgfqpoint{0.000000in}{-0.048611in}}%
\pgfusepath{stroke,fill}%
}%
\begin{pgfscope}%
\pgfsys@transformshift{8.152404in}{0.625000in}%
\pgfsys@useobject{currentmarker}{}%
\end{pgfscope}%
\end{pgfscope}%
\begin{pgfscope}%
\definecolor{textcolor}{rgb}{0.000000,0.000000,0.000000}%
\pgfsetstrokecolor{textcolor}%
\pgfsetfillcolor{textcolor}%
\pgftext[x=8.186293in, y=-1.928194in, left, base,rotate=90.000000]{\color{textcolor}\rmfamily\fontsize{10.000000}{12.000000}\selectfont com.originatorkids.EndlessWordplay.apk}%
\end{pgfscope}%
\begin{pgfscope}%
\pgfsetbuttcap%
\pgfsetroundjoin%
\definecolor{currentfill}{rgb}{0.000000,0.000000,0.000000}%
\pgfsetfillcolor{currentfill}%
\pgfsetlinewidth{0.803000pt}%
\definecolor{currentstroke}{rgb}{0.000000,0.000000,0.000000}%
\pgfsetstrokecolor{currentstroke}%
\pgfsetdash{}{0pt}%
\pgfsys@defobject{currentmarker}{\pgfqpoint{0.000000in}{-0.048611in}}{\pgfqpoint{0.000000in}{0.000000in}}{%
\pgfpathmoveto{\pgfqpoint{0.000000in}{0.000000in}}%
\pgfpathlineto{\pgfqpoint{0.000000in}{-0.048611in}}%
\pgfusepath{stroke,fill}%
}%
\begin{pgfscope}%
\pgfsys@transformshift{8.275792in}{0.625000in}%
\pgfsys@useobject{currentmarker}{}%
\end{pgfscope}%
\end{pgfscope}%
\begin{pgfscope}%
\definecolor{textcolor}{rgb}{0.000000,0.000000,0.000000}%
\pgfsetstrokecolor{textcolor}%
\pgfsetfillcolor{textcolor}%
\pgftext[x=8.310514in, y=-0.637777in, left, base,rotate=90.000000]{\color{textcolor}\rmfamily\fontsize{10.000000}{12.000000}\selectfont vivino.web.app.apk}%
\end{pgfscope}%
\begin{pgfscope}%
\pgfsetbuttcap%
\pgfsetroundjoin%
\definecolor{currentfill}{rgb}{0.000000,0.000000,0.000000}%
\pgfsetfillcolor{currentfill}%
\pgfsetlinewidth{0.803000pt}%
\definecolor{currentstroke}{rgb}{0.000000,0.000000,0.000000}%
\pgfsetstrokecolor{currentstroke}%
\pgfsetdash{}{0pt}%
\pgfsys@defobject{currentmarker}{\pgfqpoint{0.000000in}{-0.048611in}}{\pgfqpoint{0.000000in}{0.000000in}}{%
\pgfpathmoveto{\pgfqpoint{0.000000in}{0.000000in}}%
\pgfpathlineto{\pgfqpoint{0.000000in}{-0.048611in}}%
\pgfusepath{stroke,fill}%
}%
\begin{pgfscope}%
\pgfsys@transformshift{8.399180in}{0.625000in}%
\pgfsys@useobject{currentmarker}{}%
\end{pgfscope}%
\end{pgfscope}%
\begin{pgfscope}%
\definecolor{textcolor}{rgb}{0.000000,0.000000,0.000000}%
\pgfsetstrokecolor{textcolor}%
\pgfsetfillcolor{textcolor}%
\pgftext[x=8.433069in, y=-1.629028in, left, base,rotate=90.000000]{\color{textcolor}\rmfamily\fontsize{10.000000}{12.000000}\selectfont com.nbaimd.gametime.nba2011.apk}%
\end{pgfscope}%
\begin{pgfscope}%
\pgfsetbuttcap%
\pgfsetroundjoin%
\definecolor{currentfill}{rgb}{0.000000,0.000000,0.000000}%
\pgfsetfillcolor{currentfill}%
\pgfsetlinewidth{0.803000pt}%
\definecolor{currentstroke}{rgb}{0.000000,0.000000,0.000000}%
\pgfsetstrokecolor{currentstroke}%
\pgfsetdash{}{0pt}%
\pgfsys@defobject{currentmarker}{\pgfqpoint{0.000000in}{-0.048611in}}{\pgfqpoint{0.000000in}{0.000000in}}{%
\pgfpathmoveto{\pgfqpoint{0.000000in}{0.000000in}}%
\pgfpathlineto{\pgfqpoint{0.000000in}{-0.048611in}}%
\pgfusepath{stroke,fill}%
}%
\begin{pgfscope}%
\pgfsys@transformshift{8.522568in}{0.625000in}%
\pgfsys@useobject{currentmarker}{}%
\end{pgfscope}%
\end{pgfscope}%
\begin{pgfscope}%
\definecolor{textcolor}{rgb}{0.000000,0.000000,0.000000}%
\pgfsetstrokecolor{textcolor}%
\pgfsetfillcolor{textcolor}%
\pgftext[x=8.558054in, y=-0.225139in, left, base,rotate=90.000000]{\color{textcolor}\rmfamily\fontsize{10.000000}{12.000000}\selectfont com.flur.apk}%
\end{pgfscope}%
\begin{pgfscope}%
\pgfsetbuttcap%
\pgfsetroundjoin%
\definecolor{currentfill}{rgb}{0.000000,0.000000,0.000000}%
\pgfsetfillcolor{currentfill}%
\pgfsetlinewidth{0.803000pt}%
\definecolor{currentstroke}{rgb}{0.000000,0.000000,0.000000}%
\pgfsetstrokecolor{currentstroke}%
\pgfsetdash{}{0pt}%
\pgfsys@defobject{currentmarker}{\pgfqpoint{0.000000in}{-0.048611in}}{\pgfqpoint{0.000000in}{0.000000in}}{%
\pgfpathmoveto{\pgfqpoint{0.000000in}{0.000000in}}%
\pgfpathlineto{\pgfqpoint{0.000000in}{-0.048611in}}%
\pgfusepath{stroke,fill}%
}%
\begin{pgfscope}%
\pgfsys@transformshift{8.645956in}{0.625000in}%
\pgfsys@useobject{currentmarker}{}%
\end{pgfscope}%
\end{pgfscope}%
\begin{pgfscope}%
\definecolor{textcolor}{rgb}{0.000000,0.000000,0.000000}%
\pgfsetstrokecolor{textcolor}%
\pgfsetfillcolor{textcolor}%
\pgftext[x=8.680678in, y=-0.811667in, left, base,rotate=90.000000]{\color{textcolor}\rmfamily\fontsize{10.000000}{12.000000}\selectfont com.lezhin.comics.apk}%
\end{pgfscope}%
\begin{pgfscope}%
\pgfsetbuttcap%
\pgfsetroundjoin%
\definecolor{currentfill}{rgb}{0.000000,0.000000,0.000000}%
\pgfsetfillcolor{currentfill}%
\pgfsetlinewidth{0.803000pt}%
\definecolor{currentstroke}{rgb}{0.000000,0.000000,0.000000}%
\pgfsetstrokecolor{currentstroke}%
\pgfsetdash{}{0pt}%
\pgfsys@defobject{currentmarker}{\pgfqpoint{0.000000in}{-0.048611in}}{\pgfqpoint{0.000000in}{0.000000in}}{%
\pgfpathmoveto{\pgfqpoint{0.000000in}{0.000000in}}%
\pgfpathlineto{\pgfqpoint{0.000000in}{-0.048611in}}%
\pgfusepath{stroke,fill}%
}%
\begin{pgfscope}%
\pgfsys@transformshift{8.769344in}{0.625000in}%
\pgfsys@useobject{currentmarker}{}%
\end{pgfscope}%
\end{pgfscope}%
\begin{pgfscope}%
\definecolor{textcolor}{rgb}{0.000000,0.000000,0.000000}%
\pgfsetstrokecolor{textcolor}%
\pgfsetfillcolor{textcolor}%
\pgftext[x=8.804830in, y=-1.722778in, left, base,rotate=90.000000]{\color{textcolor}\rmfamily\fontsize{10.000000}{12.000000}\selectfont com.videocall.randomfriendvideo.apk}%
\end{pgfscope}%
\begin{pgfscope}%
\pgfsetbuttcap%
\pgfsetroundjoin%
\definecolor{currentfill}{rgb}{0.000000,0.000000,0.000000}%
\pgfsetfillcolor{currentfill}%
\pgfsetlinewidth{0.803000pt}%
\definecolor{currentstroke}{rgb}{0.000000,0.000000,0.000000}%
\pgfsetstrokecolor{currentstroke}%
\pgfsetdash{}{0pt}%
\pgfsys@defobject{currentmarker}{\pgfqpoint{0.000000in}{-0.048611in}}{\pgfqpoint{0.000000in}{0.000000in}}{%
\pgfpathmoveto{\pgfqpoint{0.000000in}{0.000000in}}%
\pgfpathlineto{\pgfqpoint{0.000000in}{-0.048611in}}%
\pgfusepath{stroke,fill}%
}%
\begin{pgfscope}%
\pgfsys@transformshift{8.892732in}{0.625000in}%
\pgfsys@useobject{currentmarker}{}%
\end{pgfscope}%
\end{pgfscope}%
\begin{pgfscope}%
\definecolor{textcolor}{rgb}{0.000000,0.000000,0.000000}%
\pgfsetstrokecolor{textcolor}%
\pgfsetfillcolor{textcolor}%
\pgftext[x=8.927454in, y=-0.939166in, left, base,rotate=90.000000]{\color{textcolor}\rmfamily\fontsize{10.000000}{12.000000}\selectfont com.xfinity.cloudtvr.apk}%
\end{pgfscope}%
\begin{pgfscope}%
\pgfsetbuttcap%
\pgfsetroundjoin%
\definecolor{currentfill}{rgb}{0.000000,0.000000,0.000000}%
\pgfsetfillcolor{currentfill}%
\pgfsetlinewidth{0.803000pt}%
\definecolor{currentstroke}{rgb}{0.000000,0.000000,0.000000}%
\pgfsetstrokecolor{currentstroke}%
\pgfsetdash{}{0pt}%
\pgfsys@defobject{currentmarker}{\pgfqpoint{0.000000in}{-0.048611in}}{\pgfqpoint{0.000000in}{0.000000in}}{%
\pgfpathmoveto{\pgfqpoint{0.000000in}{0.000000in}}%
\pgfpathlineto{\pgfqpoint{0.000000in}{-0.048611in}}%
\pgfusepath{stroke,fill}%
}%
\begin{pgfscope}%
\pgfsys@transformshift{9.016120in}{0.625000in}%
\pgfsys@useobject{currentmarker}{}%
\end{pgfscope}%
\end{pgfscope}%
\begin{pgfscope}%
\definecolor{textcolor}{rgb}{0.000000,0.000000,0.000000}%
\pgfsetstrokecolor{textcolor}%
\pgfsetfillcolor{textcolor}%
\pgftext[x=9.050842in, y=-0.630000in, left, base,rotate=90.000000]{\color{textcolor}\rmfamily\fontsize{10.000000}{12.000000}\selectfont dbx.taiwantaxi.apk}%
\end{pgfscope}%
\begin{pgfscope}%
\pgfsetbuttcap%
\pgfsetroundjoin%
\definecolor{currentfill}{rgb}{0.000000,0.000000,0.000000}%
\pgfsetfillcolor{currentfill}%
\pgfsetlinewidth{0.803000pt}%
\definecolor{currentstroke}{rgb}{0.000000,0.000000,0.000000}%
\pgfsetstrokecolor{currentstroke}%
\pgfsetdash{}{0pt}%
\pgfsys@defobject{currentmarker}{\pgfqpoint{0.000000in}{-0.048611in}}{\pgfqpoint{0.000000in}{0.000000in}}{%
\pgfpathmoveto{\pgfqpoint{0.000000in}{0.000000in}}%
\pgfpathlineto{\pgfqpoint{0.000000in}{-0.048611in}}%
\pgfusepath{stroke,fill}%
}%
\begin{pgfscope}%
\pgfsys@transformshift{9.139508in}{0.625000in}%
\pgfsys@useobject{currentmarker}{}%
\end{pgfscope}%
\end{pgfscope}%
\begin{pgfscope}%
\definecolor{textcolor}{rgb}{0.000000,0.000000,0.000000}%
\pgfsetstrokecolor{textcolor}%
\pgfsetfillcolor{textcolor}%
\pgftext[x=9.174230in, y=-1.600278in, left, base,rotate=90.000000]{\color{textcolor}\rmfamily\fontsize{10.000000}{12.000000}\selectfont com.socialnetwork.hookupsapp.apk}%
\end{pgfscope}%
\begin{pgfscope}%
\pgfsetbuttcap%
\pgfsetroundjoin%
\definecolor{currentfill}{rgb}{0.000000,0.000000,0.000000}%
\pgfsetfillcolor{currentfill}%
\pgfsetlinewidth{0.803000pt}%
\definecolor{currentstroke}{rgb}{0.000000,0.000000,0.000000}%
\pgfsetstrokecolor{currentstroke}%
\pgfsetdash{}{0pt}%
\pgfsys@defobject{currentmarker}{\pgfqpoint{0.000000in}{-0.048611in}}{\pgfqpoint{0.000000in}{0.000000in}}{%
\pgfpathmoveto{\pgfqpoint{0.000000in}{0.000000in}}%
\pgfpathlineto{\pgfqpoint{0.000000in}{-0.048611in}}%
\pgfusepath{stroke,fill}%
}%
\begin{pgfscope}%
\pgfsys@transformshift{9.262896in}{0.625000in}%
\pgfsys@useobject{currentmarker}{}%
\end{pgfscope}%
\end{pgfscope}%
\begin{pgfscope}%
\definecolor{textcolor}{rgb}{0.000000,0.000000,0.000000}%
\pgfsetstrokecolor{textcolor}%
\pgfsetfillcolor{textcolor}%
\pgftext[x=9.296854in, y=-1.534444in, left, base,rotate=90.000000]{\color{textcolor}\rmfamily\fontsize{10.000000}{12.000000}\selectfont com.clusterdev.hindikeyboard.apk}%
\end{pgfscope}%
\begin{pgfscope}%
\pgfsetbuttcap%
\pgfsetroundjoin%
\definecolor{currentfill}{rgb}{0.000000,0.000000,0.000000}%
\pgfsetfillcolor{currentfill}%
\pgfsetlinewidth{0.803000pt}%
\definecolor{currentstroke}{rgb}{0.000000,0.000000,0.000000}%
\pgfsetstrokecolor{currentstroke}%
\pgfsetdash{}{0pt}%
\pgfsys@defobject{currentmarker}{\pgfqpoint{0.000000in}{-0.048611in}}{\pgfqpoint{0.000000in}{0.000000in}}{%
\pgfpathmoveto{\pgfqpoint{0.000000in}{0.000000in}}%
\pgfpathlineto{\pgfqpoint{0.000000in}{-0.048611in}}%
\pgfusepath{stroke,fill}%
}%
\begin{pgfscope}%
\pgfsys@transformshift{9.386284in}{0.625000in}%
\pgfsys@useobject{currentmarker}{}%
\end{pgfscope}%
\end{pgfscope}%
\begin{pgfscope}%
\definecolor{textcolor}{rgb}{0.000000,0.000000,0.000000}%
\pgfsetstrokecolor{textcolor}%
\pgfsetfillcolor{textcolor}%
\pgftext[x=9.420937in, y=-2.036250in, left, base,rotate=90.000000]{\color{textcolor}\rmfamily\fontsize{10.000000}{12.000000}\selectfont document.scanner.scan.pdf.image.text.apk}%
\end{pgfscope}%
\begin{pgfscope}%
\pgfsetbuttcap%
\pgfsetroundjoin%
\definecolor{currentfill}{rgb}{0.000000,0.000000,0.000000}%
\pgfsetfillcolor{currentfill}%
\pgfsetlinewidth{0.803000pt}%
\definecolor{currentstroke}{rgb}{0.000000,0.000000,0.000000}%
\pgfsetstrokecolor{currentstroke}%
\pgfsetdash{}{0pt}%
\pgfsys@defobject{currentmarker}{\pgfqpoint{0.000000in}{-0.048611in}}{\pgfqpoint{0.000000in}{0.000000in}}{%
\pgfpathmoveto{\pgfqpoint{0.000000in}{0.000000in}}%
\pgfpathlineto{\pgfqpoint{0.000000in}{-0.048611in}}%
\pgfusepath{stroke,fill}%
}%
\begin{pgfscope}%
\pgfsys@transformshift{9.509672in}{0.625000in}%
\pgfsys@useobject{currentmarker}{}%
\end{pgfscope}%
\end{pgfscope}%
\begin{pgfscope}%
\definecolor{textcolor}{rgb}{0.000000,0.000000,0.000000}%
\pgfsetstrokecolor{textcolor}%
\pgfsetfillcolor{textcolor}%
\pgftext[x=9.544394in, y=-1.198750in, left, base,rotate=90.000000]{\color{textcolor}\rmfamily\fontsize{10.000000}{12.000000}\selectfont com.splendapps.voicerec.apk}%
\end{pgfscope}%
\begin{pgfscope}%
\pgfsetbuttcap%
\pgfsetroundjoin%
\definecolor{currentfill}{rgb}{0.000000,0.000000,0.000000}%
\pgfsetfillcolor{currentfill}%
\pgfsetlinewidth{0.803000pt}%
\definecolor{currentstroke}{rgb}{0.000000,0.000000,0.000000}%
\pgfsetstrokecolor{currentstroke}%
\pgfsetdash{}{0pt}%
\pgfsys@defobject{currentmarker}{\pgfqpoint{0.000000in}{-0.048611in}}{\pgfqpoint{0.000000in}{0.000000in}}{%
\pgfpathmoveto{\pgfqpoint{0.000000in}{0.000000in}}%
\pgfpathlineto{\pgfqpoint{0.000000in}{-0.048611in}}%
\pgfusepath{stroke,fill}%
}%
\begin{pgfscope}%
\pgfsys@transformshift{9.633060in}{0.625000in}%
\pgfsys@useobject{currentmarker}{}%
\end{pgfscope}%
\end{pgfscope}%
\begin{pgfscope}%
\definecolor{textcolor}{rgb}{0.000000,0.000000,0.000000}%
\pgfsetstrokecolor{textcolor}%
\pgfsetfillcolor{textcolor}%
\pgftext[x=9.667782in, y=-0.657083in, left, base,rotate=90.000000]{\color{textcolor}\rmfamily\fontsize{10.000000}{12.000000}\selectfont app.habitaclia2.apk}%
\end{pgfscope}%
\begin{pgfscope}%
\pgfsetbuttcap%
\pgfsetroundjoin%
\definecolor{currentfill}{rgb}{0.000000,0.000000,0.000000}%
\pgfsetfillcolor{currentfill}%
\pgfsetlinewidth{0.803000pt}%
\definecolor{currentstroke}{rgb}{0.000000,0.000000,0.000000}%
\pgfsetstrokecolor{currentstroke}%
\pgfsetdash{}{0pt}%
\pgfsys@defobject{currentmarker}{\pgfqpoint{0.000000in}{-0.048611in}}{\pgfqpoint{0.000000in}{0.000000in}}{%
\pgfpathmoveto{\pgfqpoint{0.000000in}{0.000000in}}%
\pgfpathlineto{\pgfqpoint{0.000000in}{-0.048611in}}%
\pgfusepath{stroke,fill}%
}%
\begin{pgfscope}%
\pgfsys@transformshift{9.756448in}{0.625000in}%
\pgfsys@useobject{currentmarker}{}%
\end{pgfscope}%
\end{pgfscope}%
\begin{pgfscope}%
\definecolor{textcolor}{rgb}{0.000000,0.000000,0.000000}%
\pgfsetstrokecolor{textcolor}%
\pgfsetfillcolor{textcolor}%
\pgftext[x=9.791934in, y=-0.623333in, left, base,rotate=90.000000]{\color{textcolor}\rmfamily\fontsize{10.000000}{12.000000}\selectfont sweet.selfie.lite.apk}%
\end{pgfscope}%
\begin{pgfscope}%
\pgfsetbuttcap%
\pgfsetroundjoin%
\definecolor{currentfill}{rgb}{0.000000,0.000000,0.000000}%
\pgfsetfillcolor{currentfill}%
\pgfsetlinewidth{0.803000pt}%
\definecolor{currentstroke}{rgb}{0.000000,0.000000,0.000000}%
\pgfsetstrokecolor{currentstroke}%
\pgfsetdash{}{0pt}%
\pgfsys@defobject{currentmarker}{\pgfqpoint{0.000000in}{-0.048611in}}{\pgfqpoint{0.000000in}{0.000000in}}{%
\pgfpathmoveto{\pgfqpoint{0.000000in}{0.000000in}}%
\pgfpathlineto{\pgfqpoint{0.000000in}{-0.048611in}}%
\pgfusepath{stroke,fill}%
}%
\begin{pgfscope}%
\pgfsys@transformshift{9.879836in}{0.625000in}%
\pgfsys@useobject{currentmarker}{}%
\end{pgfscope}%
\end{pgfscope}%
\begin{pgfscope}%
\definecolor{textcolor}{rgb}{0.000000,0.000000,0.000000}%
\pgfsetstrokecolor{textcolor}%
\pgfsetfillcolor{textcolor}%
\pgftext[x=9.914558in, y=-1.098611in, left, base,rotate=90.000000]{\color{textcolor}\rmfamily\fontsize{10.000000}{12.000000}\selectfont com.espn.score\_center.apk}%
\end{pgfscope}%
\begin{pgfscope}%
\pgfsetbuttcap%
\pgfsetroundjoin%
\definecolor{currentfill}{rgb}{0.000000,0.000000,0.000000}%
\pgfsetfillcolor{currentfill}%
\pgfsetlinewidth{0.803000pt}%
\definecolor{currentstroke}{rgb}{0.000000,0.000000,0.000000}%
\pgfsetstrokecolor{currentstroke}%
\pgfsetdash{}{0pt}%
\pgfsys@defobject{currentmarker}{\pgfqpoint{0.000000in}{-0.048611in}}{\pgfqpoint{0.000000in}{0.000000in}}{%
\pgfpathmoveto{\pgfqpoint{0.000000in}{0.000000in}}%
\pgfpathlineto{\pgfqpoint{0.000000in}{-0.048611in}}%
\pgfusepath{stroke,fill}%
}%
\begin{pgfscope}%
\pgfsys@transformshift{10.003224in}{0.625000in}%
\pgfsys@useobject{currentmarker}{}%
\end{pgfscope}%
\end{pgfscope}%
\begin{pgfscope}%
\definecolor{textcolor}{rgb}{0.000000,0.000000,0.000000}%
\pgfsetstrokecolor{textcolor}%
\pgfsetfillcolor{textcolor}%
\pgftext[x=10.037877in, y=-0.880694in, left, base,rotate=90.000000]{\color{textcolor}\rmfamily\fontsize{10.000000}{12.000000}\selectfont com.fineapp.yogiyo.apk}%
\end{pgfscope}%
\begin{pgfscope}%
\pgfsetbuttcap%
\pgfsetroundjoin%
\definecolor{currentfill}{rgb}{0.000000,0.000000,0.000000}%
\pgfsetfillcolor{currentfill}%
\pgfsetlinewidth{0.803000pt}%
\definecolor{currentstroke}{rgb}{0.000000,0.000000,0.000000}%
\pgfsetstrokecolor{currentstroke}%
\pgfsetdash{}{0pt}%
\pgfsys@defobject{currentmarker}{\pgfqpoint{0.000000in}{-0.048611in}}{\pgfqpoint{0.000000in}{0.000000in}}{%
\pgfpathmoveto{\pgfqpoint{0.000000in}{0.000000in}}%
\pgfpathlineto{\pgfqpoint{0.000000in}{-0.048611in}}%
\pgfusepath{stroke,fill}%
}%
\begin{pgfscope}%
\pgfsys@transformshift{10.126612in}{0.625000in}%
\pgfsys@useobject{currentmarker}{}%
\end{pgfscope}%
\end{pgfscope}%
\begin{pgfscope}%
\definecolor{textcolor}{rgb}{0.000000,0.000000,0.000000}%
\pgfsetstrokecolor{textcolor}%
\pgfsetfillcolor{textcolor}%
\pgftext[x=10.161334in, y=-0.997500in, left, base,rotate=90.000000]{\color{textcolor}\rmfamily\fontsize{10.000000}{12.000000}\selectfont com.thredup.android.apk}%
\end{pgfscope}%
\begin{pgfscope}%
\pgfsetbuttcap%
\pgfsetroundjoin%
\definecolor{currentfill}{rgb}{0.000000,0.000000,0.000000}%
\pgfsetfillcolor{currentfill}%
\pgfsetlinewidth{0.803000pt}%
\definecolor{currentstroke}{rgb}{0.000000,0.000000,0.000000}%
\pgfsetstrokecolor{currentstroke}%
\pgfsetdash{}{0pt}%
\pgfsys@defobject{currentmarker}{\pgfqpoint{0.000000in}{-0.048611in}}{\pgfqpoint{0.000000in}{0.000000in}}{%
\pgfpathmoveto{\pgfqpoint{0.000000in}{0.000000in}}%
\pgfpathlineto{\pgfqpoint{0.000000in}{-0.048611in}}%
\pgfusepath{stroke,fill}%
}%
\begin{pgfscope}%
\pgfsys@transformshift{10.250000in}{0.625000in}%
\pgfsys@useobject{currentmarker}{}%
\end{pgfscope}%
\end{pgfscope}%
\begin{pgfscope}%
\definecolor{textcolor}{rgb}{0.000000,0.000000,0.000000}%
\pgfsetstrokecolor{textcolor}%
\pgfsetfillcolor{textcolor}%
\pgftext[x=10.283889in, y=-2.852222in, left, base,rotate=90.000000]{\color{textcolor}\rmfamily\fontsize{10.000000}{12.000000}\selectfont kr.co.smartstudy.dinoworld\_android\_googlemarket.apk}%
\end{pgfscope}%
\begin{pgfscope}%
\pgfsetbuttcap%
\pgfsetroundjoin%
\definecolor{currentfill}{rgb}{0.000000,0.000000,0.000000}%
\pgfsetfillcolor{currentfill}%
\pgfsetlinewidth{0.803000pt}%
\definecolor{currentstroke}{rgb}{0.000000,0.000000,0.000000}%
\pgfsetstrokecolor{currentstroke}%
\pgfsetdash{}{0pt}%
\pgfsys@defobject{currentmarker}{\pgfqpoint{0.000000in}{-0.048611in}}{\pgfqpoint{0.000000in}{0.000000in}}{%
\pgfpathmoveto{\pgfqpoint{0.000000in}{0.000000in}}%
\pgfpathlineto{\pgfqpoint{0.000000in}{-0.048611in}}%
\pgfusepath{stroke,fill}%
}%
\begin{pgfscope}%
\pgfsys@transformshift{10.373388in}{0.625000in}%
\pgfsys@useobject{currentmarker}{}%
\end{pgfscope}%
\end{pgfscope}%
\begin{pgfscope}%
\definecolor{textcolor}{rgb}{0.000000,0.000000,0.000000}%
\pgfsetstrokecolor{textcolor}%
\pgfsetfillcolor{textcolor}%
\pgftext[x=10.408110in, y=-0.726667in, left, base,rotate=90.000000]{\color{textcolor}\rmfamily\fontsize{10.000000}{12.000000}\selectfont com.opera.touch.apk}%
\end{pgfscope}%
\begin{pgfscope}%
\pgfsetbuttcap%
\pgfsetroundjoin%
\definecolor{currentfill}{rgb}{0.000000,0.000000,0.000000}%
\pgfsetfillcolor{currentfill}%
\pgfsetlinewidth{0.803000pt}%
\definecolor{currentstroke}{rgb}{0.000000,0.000000,0.000000}%
\pgfsetstrokecolor{currentstroke}%
\pgfsetdash{}{0pt}%
\pgfsys@defobject{currentmarker}{\pgfqpoint{0.000000in}{-0.048611in}}{\pgfqpoint{0.000000in}{0.000000in}}{%
\pgfpathmoveto{\pgfqpoint{0.000000in}{0.000000in}}%
\pgfpathlineto{\pgfqpoint{0.000000in}{-0.048611in}}%
\pgfusepath{stroke,fill}%
}%
\begin{pgfscope}%
\pgfsys@transformshift{10.496776in}{0.625000in}%
\pgfsys@useobject{currentmarker}{}%
\end{pgfscope}%
\end{pgfscope}%
\begin{pgfscope}%
\definecolor{textcolor}{rgb}{0.000000,0.000000,0.000000}%
\pgfsetstrokecolor{textcolor}%
\pgfsetfillcolor{textcolor}%
\pgftext[x=10.530734in, y=-0.444722in, left, base,rotate=90.000000]{\color{textcolor}\rmfamily\fontsize{10.000000}{12.000000}\selectfont com.yy.hiyo.apk}%
\end{pgfscope}%
\begin{pgfscope}%
\pgfsetbuttcap%
\pgfsetroundjoin%
\definecolor{currentfill}{rgb}{0.000000,0.000000,0.000000}%
\pgfsetfillcolor{currentfill}%
\pgfsetlinewidth{0.803000pt}%
\definecolor{currentstroke}{rgb}{0.000000,0.000000,0.000000}%
\pgfsetstrokecolor{currentstroke}%
\pgfsetdash{}{0pt}%
\pgfsys@defobject{currentmarker}{\pgfqpoint{0.000000in}{-0.048611in}}{\pgfqpoint{0.000000in}{0.000000in}}{%
\pgfpathmoveto{\pgfqpoint{0.000000in}{0.000000in}}%
\pgfpathlineto{\pgfqpoint{0.000000in}{-0.048611in}}%
\pgfusepath{stroke,fill}%
}%
\begin{pgfscope}%
\pgfsys@transformshift{10.620164in}{0.625000in}%
\pgfsys@useobject{currentmarker}{}%
\end{pgfscope}%
\end{pgfscope}%
\begin{pgfscope}%
\definecolor{textcolor}{rgb}{0.000000,0.000000,0.000000}%
\pgfsetstrokecolor{textcolor}%
\pgfsetfillcolor{textcolor}%
\pgftext[x=10.654886in, y=-1.093056in, left, base,rotate=90.000000]{\color{textcolor}\rmfamily\fontsize{10.000000}{12.000000}\selectfont de.etecture.ekz.onleihe.apk}%
\end{pgfscope}%
\begin{pgfscope}%
\pgfsetbuttcap%
\pgfsetroundjoin%
\definecolor{currentfill}{rgb}{0.000000,0.000000,0.000000}%
\pgfsetfillcolor{currentfill}%
\pgfsetlinewidth{0.803000pt}%
\definecolor{currentstroke}{rgb}{0.000000,0.000000,0.000000}%
\pgfsetstrokecolor{currentstroke}%
\pgfsetdash{}{0pt}%
\pgfsys@defobject{currentmarker}{\pgfqpoint{0.000000in}{-0.048611in}}{\pgfqpoint{0.000000in}{0.000000in}}{%
\pgfpathmoveto{\pgfqpoint{0.000000in}{0.000000in}}%
\pgfpathlineto{\pgfqpoint{0.000000in}{-0.048611in}}%
\pgfusepath{stroke,fill}%
}%
\begin{pgfscope}%
\pgfsys@transformshift{10.743552in}{0.625000in}%
\pgfsys@useobject{currentmarker}{}%
\end{pgfscope}%
\end{pgfscope}%
\begin{pgfscope}%
\definecolor{textcolor}{rgb}{0.000000,0.000000,0.000000}%
\pgfsetstrokecolor{textcolor}%
\pgfsetfillcolor{textcolor}%
\pgftext[x=10.778205in, y=-1.630833in, left, base,rotate=90.000000]{\color{textcolor}\rmfamily\fontsize{10.000000}{12.000000}\selectfont com.kevinbradford.games.pklg2.apk}%
\end{pgfscope}%
\begin{pgfscope}%
\pgfsetbuttcap%
\pgfsetroundjoin%
\definecolor{currentfill}{rgb}{0.000000,0.000000,0.000000}%
\pgfsetfillcolor{currentfill}%
\pgfsetlinewidth{0.803000pt}%
\definecolor{currentstroke}{rgb}{0.000000,0.000000,0.000000}%
\pgfsetstrokecolor{currentstroke}%
\pgfsetdash{}{0pt}%
\pgfsys@defobject{currentmarker}{\pgfqpoint{0.000000in}{-0.048611in}}{\pgfqpoint{0.000000in}{0.000000in}}{%
\pgfpathmoveto{\pgfqpoint{0.000000in}{0.000000in}}%
\pgfpathlineto{\pgfqpoint{0.000000in}{-0.048611in}}%
\pgfusepath{stroke,fill}%
}%
\begin{pgfscope}%
\pgfsys@transformshift{10.866940in}{0.625000in}%
\pgfsys@useobject{currentmarker}{}%
\end{pgfscope}%
\end{pgfscope}%
\begin{pgfscope}%
\definecolor{textcolor}{rgb}{0.000000,0.000000,0.000000}%
\pgfsetstrokecolor{textcolor}%
\pgfsetfillcolor{textcolor}%
\pgftext[x=10.900898in, y=-1.210139in, left, base,rotate=90.000000]{\color{textcolor}\rmfamily\fontsize{10.000000}{12.000000}\selectfont ua.insomnia.kenya.newsi.apk}%
\end{pgfscope}%
\begin{pgfscope}%
\pgfsetbuttcap%
\pgfsetroundjoin%
\definecolor{currentfill}{rgb}{0.000000,0.000000,0.000000}%
\pgfsetfillcolor{currentfill}%
\pgfsetlinewidth{0.803000pt}%
\definecolor{currentstroke}{rgb}{0.000000,0.000000,0.000000}%
\pgfsetstrokecolor{currentstroke}%
\pgfsetdash{}{0pt}%
\pgfsys@defobject{currentmarker}{\pgfqpoint{0.000000in}{-0.048611in}}{\pgfqpoint{0.000000in}{0.000000in}}{%
\pgfpathmoveto{\pgfqpoint{0.000000in}{0.000000in}}%
\pgfpathlineto{\pgfqpoint{0.000000in}{-0.048611in}}%
\pgfusepath{stroke,fill}%
}%
\begin{pgfscope}%
\pgfsys@transformshift{10.990328in}{0.625000in}%
\pgfsys@useobject{currentmarker}{}%
\end{pgfscope}%
\end{pgfscope}%
\begin{pgfscope}%
\definecolor{textcolor}{rgb}{0.000000,0.000000,0.000000}%
\pgfsetstrokecolor{textcolor}%
\pgfsetfillcolor{textcolor}%
\pgftext[x=11.024981in, y=-1.790000in, left, base,rotate=90.000000]{\color{textcolor}\rmfamily\fontsize{10.000000}{12.000000}\selectfont com.microsoft.amp.apps.bingnews.apk}%
\end{pgfscope}%
\begin{pgfscope}%
\pgfsetbuttcap%
\pgfsetroundjoin%
\definecolor{currentfill}{rgb}{0.000000,0.000000,0.000000}%
\pgfsetfillcolor{currentfill}%
\pgfsetlinewidth{0.803000pt}%
\definecolor{currentstroke}{rgb}{0.000000,0.000000,0.000000}%
\pgfsetstrokecolor{currentstroke}%
\pgfsetdash{}{0pt}%
\pgfsys@defobject{currentmarker}{\pgfqpoint{0.000000in}{-0.048611in}}{\pgfqpoint{0.000000in}{0.000000in}}{%
\pgfpathmoveto{\pgfqpoint{0.000000in}{0.000000in}}%
\pgfpathlineto{\pgfqpoint{0.000000in}{-0.048611in}}%
\pgfusepath{stroke,fill}%
}%
\begin{pgfscope}%
\pgfsys@transformshift{11.113716in}{0.625000in}%
\pgfsys@useobject{currentmarker}{}%
\end{pgfscope}%
\end{pgfscope}%
\begin{pgfscope}%
\definecolor{textcolor}{rgb}{0.000000,0.000000,0.000000}%
\pgfsetstrokecolor{textcolor}%
\pgfsetfillcolor{textcolor}%
\pgftext[x=11.147674in, y=-0.552500in, left, base,rotate=90.000000]{\color{textcolor}\rmfamily\fontsize{10.000000}{12.000000}\selectfont com.money91.apk}%
\end{pgfscope}%
\begin{pgfscope}%
\pgfsetbuttcap%
\pgfsetroundjoin%
\definecolor{currentfill}{rgb}{0.000000,0.000000,0.000000}%
\pgfsetfillcolor{currentfill}%
\pgfsetlinewidth{0.803000pt}%
\definecolor{currentstroke}{rgb}{0.000000,0.000000,0.000000}%
\pgfsetstrokecolor{currentstroke}%
\pgfsetdash{}{0pt}%
\pgfsys@defobject{currentmarker}{\pgfqpoint{0.000000in}{-0.048611in}}{\pgfqpoint{0.000000in}{0.000000in}}{%
\pgfpathmoveto{\pgfqpoint{0.000000in}{0.000000in}}%
\pgfpathlineto{\pgfqpoint{0.000000in}{-0.048611in}}%
\pgfusepath{stroke,fill}%
}%
\begin{pgfscope}%
\pgfsys@transformshift{11.237104in}{0.625000in}%
\pgfsys@useobject{currentmarker}{}%
\end{pgfscope}%
\end{pgfscope}%
\begin{pgfscope}%
\definecolor{textcolor}{rgb}{0.000000,0.000000,0.000000}%
\pgfsetstrokecolor{textcolor}%
\pgfsetfillcolor{textcolor}%
\pgftext[x=11.271062in, y=-0.824028in, left, base,rotate=90.000000]{\color{textcolor}\rmfamily\fontsize{10.000000}{12.000000}\selectfont com.resmed.myair.apk}%
\end{pgfscope}%
\begin{pgfscope}%
\pgfsetbuttcap%
\pgfsetroundjoin%
\definecolor{currentfill}{rgb}{0.000000,0.000000,0.000000}%
\pgfsetfillcolor{currentfill}%
\pgfsetlinewidth{0.803000pt}%
\definecolor{currentstroke}{rgb}{0.000000,0.000000,0.000000}%
\pgfsetstrokecolor{currentstroke}%
\pgfsetdash{}{0pt}%
\pgfsys@defobject{currentmarker}{\pgfqpoint{0.000000in}{-0.048611in}}{\pgfqpoint{0.000000in}{0.000000in}}{%
\pgfpathmoveto{\pgfqpoint{0.000000in}{0.000000in}}%
\pgfpathlineto{\pgfqpoint{0.000000in}{-0.048611in}}%
\pgfusepath{stroke,fill}%
}%
\begin{pgfscope}%
\pgfsys@transformshift{11.360492in}{0.625000in}%
\pgfsys@useobject{currentmarker}{}%
\end{pgfscope}%
\end{pgfscope}%
\begin{pgfscope}%
\definecolor{textcolor}{rgb}{0.000000,0.000000,0.000000}%
\pgfsetstrokecolor{textcolor}%
\pgfsetfillcolor{textcolor}%
\pgftext[x=11.395214in, y=-1.144167in, left, base,rotate=90.000000]{\color{textcolor}\rmfamily\fontsize{10.000000}{12.000000}\selectfont com.zoho.sheet.android.apk}%
\end{pgfscope}%
\begin{pgfscope}%
\pgfsetbuttcap%
\pgfsetroundjoin%
\definecolor{currentfill}{rgb}{0.000000,0.000000,0.000000}%
\pgfsetfillcolor{currentfill}%
\pgfsetlinewidth{0.803000pt}%
\definecolor{currentstroke}{rgb}{0.000000,0.000000,0.000000}%
\pgfsetstrokecolor{currentstroke}%
\pgfsetdash{}{0pt}%
\pgfsys@defobject{currentmarker}{\pgfqpoint{0.000000in}{-0.048611in}}{\pgfqpoint{0.000000in}{0.000000in}}{%
\pgfpathmoveto{\pgfqpoint{0.000000in}{0.000000in}}%
\pgfpathlineto{\pgfqpoint{0.000000in}{-0.048611in}}%
\pgfusepath{stroke,fill}%
}%
\begin{pgfscope}%
\pgfsys@transformshift{11.483880in}{0.625000in}%
\pgfsys@useobject{currentmarker}{}%
\end{pgfscope}%
\end{pgfscope}%
\begin{pgfscope}%
\definecolor{textcolor}{rgb}{0.000000,0.000000,0.000000}%
\pgfsetstrokecolor{textcolor}%
\pgfsetfillcolor{textcolor}%
\pgftext[x=11.519366in, y=-0.895972in, left, base,rotate=90.000000]{\color{textcolor}\rmfamily\fontsize{10.000000}{12.000000}\selectfont com.facebook.mlite.apk}%
\end{pgfscope}%
\begin{pgfscope}%
\pgfsetbuttcap%
\pgfsetroundjoin%
\definecolor{currentfill}{rgb}{0.000000,0.000000,0.000000}%
\pgfsetfillcolor{currentfill}%
\pgfsetlinewidth{0.803000pt}%
\definecolor{currentstroke}{rgb}{0.000000,0.000000,0.000000}%
\pgfsetstrokecolor{currentstroke}%
\pgfsetdash{}{0pt}%
\pgfsys@defobject{currentmarker}{\pgfqpoint{0.000000in}{-0.048611in}}{\pgfqpoint{0.000000in}{0.000000in}}{%
\pgfpathmoveto{\pgfqpoint{0.000000in}{0.000000in}}%
\pgfpathlineto{\pgfqpoint{0.000000in}{-0.048611in}}%
\pgfusepath{stroke,fill}%
}%
\begin{pgfscope}%
\pgfsys@transformshift{11.607268in}{0.625000in}%
\pgfsys@useobject{currentmarker}{}%
\end{pgfscope}%
\end{pgfscope}%
\begin{pgfscope}%
\definecolor{textcolor}{rgb}{0.000000,0.000000,0.000000}%
\pgfsetstrokecolor{textcolor}%
\pgfsetfillcolor{textcolor}%
\pgftext[x=11.641990in, y=-1.164444in, left, base,rotate=90.000000]{\color{textcolor}\rmfamily\fontsize{10.000000}{12.000000}\selectfont au.com.auspost.android.apk}%
\end{pgfscope}%
\begin{pgfscope}%
\pgfsetbuttcap%
\pgfsetroundjoin%
\definecolor{currentfill}{rgb}{0.000000,0.000000,0.000000}%
\pgfsetfillcolor{currentfill}%
\pgfsetlinewidth{0.803000pt}%
\definecolor{currentstroke}{rgb}{0.000000,0.000000,0.000000}%
\pgfsetstrokecolor{currentstroke}%
\pgfsetdash{}{0pt}%
\pgfsys@defobject{currentmarker}{\pgfqpoint{0.000000in}{-0.048611in}}{\pgfqpoint{0.000000in}{0.000000in}}{%
\pgfpathmoveto{\pgfqpoint{0.000000in}{0.000000in}}%
\pgfpathlineto{\pgfqpoint{0.000000in}{-0.048611in}}%
\pgfusepath{stroke,fill}%
}%
\begin{pgfscope}%
\pgfsys@transformshift{11.730656in}{0.625000in}%
\pgfsys@useobject{currentmarker}{}%
\end{pgfscope}%
\end{pgfscope}%
\begin{pgfscope}%
\definecolor{textcolor}{rgb}{0.000000,0.000000,0.000000}%
\pgfsetstrokecolor{textcolor}%
\pgfsetfillcolor{textcolor}%
\pgftext[x=11.765309in, y=-1.011944in, left, base,rotate=90.000000]{\color{textcolor}\rmfamily\fontsize{10.000000}{12.000000}\selectfont com.dating.find\_love.apk}%
\end{pgfscope}%
\begin{pgfscope}%
\pgfsetbuttcap%
\pgfsetroundjoin%
\definecolor{currentfill}{rgb}{0.000000,0.000000,0.000000}%
\pgfsetfillcolor{currentfill}%
\pgfsetlinewidth{0.803000pt}%
\definecolor{currentstroke}{rgb}{0.000000,0.000000,0.000000}%
\pgfsetstrokecolor{currentstroke}%
\pgfsetdash{}{0pt}%
\pgfsys@defobject{currentmarker}{\pgfqpoint{0.000000in}{-0.048611in}}{\pgfqpoint{0.000000in}{0.000000in}}{%
\pgfpathmoveto{\pgfqpoint{0.000000in}{0.000000in}}%
\pgfpathlineto{\pgfqpoint{0.000000in}{-0.048611in}}%
\pgfusepath{stroke,fill}%
}%
\begin{pgfscope}%
\pgfsys@transformshift{11.854044in}{0.625000in}%
\pgfsys@useobject{currentmarker}{}%
\end{pgfscope}%
\end{pgfscope}%
\begin{pgfscope}%
\definecolor{textcolor}{rgb}{0.000000,0.000000,0.000000}%
\pgfsetstrokecolor{textcolor}%
\pgfsetfillcolor{textcolor}%
\pgftext[x=11.889530in, y=-0.522639in, left, base,rotate=90.000000]{\color{textcolor}\rmfamily\fontsize{10.000000}{12.000000}\selectfont ma.safe.bnau.apk}%
\end{pgfscope}%
\begin{pgfscope}%
\pgfsetbuttcap%
\pgfsetroundjoin%
\definecolor{currentfill}{rgb}{0.000000,0.000000,0.000000}%
\pgfsetfillcolor{currentfill}%
\pgfsetlinewidth{0.803000pt}%
\definecolor{currentstroke}{rgb}{0.000000,0.000000,0.000000}%
\pgfsetstrokecolor{currentstroke}%
\pgfsetdash{}{0pt}%
\pgfsys@defobject{currentmarker}{\pgfqpoint{0.000000in}{-0.048611in}}{\pgfqpoint{0.000000in}{0.000000in}}{%
\pgfpathmoveto{\pgfqpoint{0.000000in}{0.000000in}}%
\pgfpathlineto{\pgfqpoint{0.000000in}{-0.048611in}}%
\pgfusepath{stroke,fill}%
}%
\begin{pgfscope}%
\pgfsys@transformshift{11.977432in}{0.625000in}%
\pgfsys@useobject{currentmarker}{}%
\end{pgfscope}%
\end{pgfscope}%
\begin{pgfscope}%
\definecolor{textcolor}{rgb}{0.000000,0.000000,0.000000}%
\pgfsetstrokecolor{textcolor}%
\pgfsetfillcolor{textcolor}%
\pgftext[x=12.012154in, y=-1.573194in, left, base,rotate=90.000000]{\color{textcolor}\rmfamily\fontsize{10.000000}{12.000000}\selectfont com.buildium.resident.android.apk}%
\end{pgfscope}%
\begin{pgfscope}%
\pgfsetbuttcap%
\pgfsetroundjoin%
\definecolor{currentfill}{rgb}{0.000000,0.000000,0.000000}%
\pgfsetfillcolor{currentfill}%
\pgfsetlinewidth{0.803000pt}%
\definecolor{currentstroke}{rgb}{0.000000,0.000000,0.000000}%
\pgfsetstrokecolor{currentstroke}%
\pgfsetdash{}{0pt}%
\pgfsys@defobject{currentmarker}{\pgfqpoint{0.000000in}{-0.048611in}}{\pgfqpoint{0.000000in}{0.000000in}}{%
\pgfpathmoveto{\pgfqpoint{0.000000in}{0.000000in}}%
\pgfpathlineto{\pgfqpoint{0.000000in}{-0.048611in}}%
\pgfusepath{stroke,fill}%
}%
\begin{pgfscope}%
\pgfsys@transformshift{12.100820in}{0.625000in}%
\pgfsys@useobject{currentmarker}{}%
\end{pgfscope}%
\end{pgfscope}%
\begin{pgfscope}%
\definecolor{textcolor}{rgb}{0.000000,0.000000,0.000000}%
\pgfsetstrokecolor{textcolor}%
\pgfsetfillcolor{textcolor}%
\pgftext[x=12.134778in, y=-0.921111in, left, base,rotate=90.000000]{\color{textcolor}\rmfamily\fontsize{10.000000}{12.000000}\selectfont com.jrtstudio.music.apk}%
\end{pgfscope}%
\begin{pgfscope}%
\pgfsetbuttcap%
\pgfsetroundjoin%
\definecolor{currentfill}{rgb}{0.000000,0.000000,0.000000}%
\pgfsetfillcolor{currentfill}%
\pgfsetlinewidth{0.803000pt}%
\definecolor{currentstroke}{rgb}{0.000000,0.000000,0.000000}%
\pgfsetstrokecolor{currentstroke}%
\pgfsetdash{}{0pt}%
\pgfsys@defobject{currentmarker}{\pgfqpoint{0.000000in}{-0.048611in}}{\pgfqpoint{0.000000in}{0.000000in}}{%
\pgfpathmoveto{\pgfqpoint{0.000000in}{0.000000in}}%
\pgfpathlineto{\pgfqpoint{0.000000in}{-0.048611in}}%
\pgfusepath{stroke,fill}%
}%
\begin{pgfscope}%
\pgfsys@transformshift{12.224208in}{0.625000in}%
\pgfsys@useobject{currentmarker}{}%
\end{pgfscope}%
\end{pgfscope}%
\begin{pgfscope}%
\definecolor{textcolor}{rgb}{0.000000,0.000000,0.000000}%
\pgfsetstrokecolor{textcolor}%
\pgfsetfillcolor{textcolor}%
\pgftext[x=12.258930in, y=-1.290972in, left, base,rotate=90.000000]{\color{textcolor}\rmfamily\fontsize{10.000000}{12.000000}\selectfont app.quiktrip.com.quiktrip.apk}%
\end{pgfscope}%
\begin{pgfscope}%
\pgfsetbuttcap%
\pgfsetroundjoin%
\definecolor{currentfill}{rgb}{0.000000,0.000000,0.000000}%
\pgfsetfillcolor{currentfill}%
\pgfsetlinewidth{0.803000pt}%
\definecolor{currentstroke}{rgb}{0.000000,0.000000,0.000000}%
\pgfsetstrokecolor{currentstroke}%
\pgfsetdash{}{0pt}%
\pgfsys@defobject{currentmarker}{\pgfqpoint{0.000000in}{-0.048611in}}{\pgfqpoint{0.000000in}{0.000000in}}{%
\pgfpathmoveto{\pgfqpoint{0.000000in}{0.000000in}}%
\pgfpathlineto{\pgfqpoint{0.000000in}{-0.048611in}}%
\pgfusepath{stroke,fill}%
}%
\begin{pgfscope}%
\pgfsys@transformshift{12.347596in}{0.625000in}%
\pgfsys@useobject{currentmarker}{}%
\end{pgfscope}%
\end{pgfscope}%
\begin{pgfscope}%
\definecolor{textcolor}{rgb}{0.000000,0.000000,0.000000}%
\pgfsetstrokecolor{textcolor}%
\pgfsetfillcolor{textcolor}%
\pgftext[x=12.382318in, y=-0.356528in, left, base,rotate=90.000000]{\color{textcolor}\rmfamily\fontsize{10.000000}{12.000000}\selectfont com.sendo.apk}%
\end{pgfscope}%
\begin{pgfscope}%
\pgfsetbuttcap%
\pgfsetroundjoin%
\definecolor{currentfill}{rgb}{0.000000,0.000000,0.000000}%
\pgfsetfillcolor{currentfill}%
\pgfsetlinewidth{0.803000pt}%
\definecolor{currentstroke}{rgb}{0.000000,0.000000,0.000000}%
\pgfsetstrokecolor{currentstroke}%
\pgfsetdash{}{0pt}%
\pgfsys@defobject{currentmarker}{\pgfqpoint{0.000000in}{-0.048611in}}{\pgfqpoint{0.000000in}{0.000000in}}{%
\pgfpathmoveto{\pgfqpoint{0.000000in}{0.000000in}}%
\pgfpathlineto{\pgfqpoint{0.000000in}{-0.048611in}}%
\pgfusepath{stroke,fill}%
}%
\begin{pgfscope}%
\pgfsys@transformshift{12.470984in}{0.625000in}%
\pgfsys@useobject{currentmarker}{}%
\end{pgfscope}%
\end{pgfscope}%
\begin{pgfscope}%
\definecolor{textcolor}{rgb}{0.000000,0.000000,0.000000}%
\pgfsetstrokecolor{textcolor}%
\pgfsetfillcolor{textcolor}%
\pgftext[x=12.505706in, y=-0.884444in, left, base,rotate=90.000000]{\color{textcolor}\rmfamily\fontsize{10.000000}{12.000000}\selectfont com.covalent.kippo.apk}%
\end{pgfscope}%
\begin{pgfscope}%
\pgfsetbuttcap%
\pgfsetroundjoin%
\definecolor{currentfill}{rgb}{0.000000,0.000000,0.000000}%
\pgfsetfillcolor{currentfill}%
\pgfsetlinewidth{0.803000pt}%
\definecolor{currentstroke}{rgb}{0.000000,0.000000,0.000000}%
\pgfsetstrokecolor{currentstroke}%
\pgfsetdash{}{0pt}%
\pgfsys@defobject{currentmarker}{\pgfqpoint{0.000000in}{-0.048611in}}{\pgfqpoint{0.000000in}{0.000000in}}{%
\pgfpathmoveto{\pgfqpoint{0.000000in}{0.000000in}}%
\pgfpathlineto{\pgfqpoint{0.000000in}{-0.048611in}}%
\pgfusepath{stroke,fill}%
}%
\begin{pgfscope}%
\pgfsys@transformshift{12.594372in}{0.625000in}%
\pgfsys@useobject{currentmarker}{}%
\end{pgfscope}%
\end{pgfscope}%
\begin{pgfscope}%
\definecolor{textcolor}{rgb}{0.000000,0.000000,0.000000}%
\pgfsetstrokecolor{textcolor}%
\pgfsetfillcolor{textcolor}%
\pgftext[x=12.629094in, y=-1.266944in, left, base,rotate=90.000000]{\color{textcolor}\rmfamily\fontsize{10.000000}{12.000000}\selectfont com.onlyoffice.documents.apk}%
\end{pgfscope}%
\begin{pgfscope}%
\pgfsetbuttcap%
\pgfsetroundjoin%
\definecolor{currentfill}{rgb}{0.000000,0.000000,0.000000}%
\pgfsetfillcolor{currentfill}%
\pgfsetlinewidth{0.803000pt}%
\definecolor{currentstroke}{rgb}{0.000000,0.000000,0.000000}%
\pgfsetstrokecolor{currentstroke}%
\pgfsetdash{}{0pt}%
\pgfsys@defobject{currentmarker}{\pgfqpoint{0.000000in}{-0.048611in}}{\pgfqpoint{0.000000in}{0.000000in}}{%
\pgfpathmoveto{\pgfqpoint{0.000000in}{0.000000in}}%
\pgfpathlineto{\pgfqpoint{0.000000in}{-0.048611in}}%
\pgfusepath{stroke,fill}%
}%
\begin{pgfscope}%
\pgfsys@transformshift{12.717760in}{0.625000in}%
\pgfsys@useobject{currentmarker}{}%
\end{pgfscope}%
\end{pgfscope}%
\begin{pgfscope}%
\definecolor{textcolor}{rgb}{0.000000,0.000000,0.000000}%
\pgfsetstrokecolor{textcolor}%
\pgfsetfillcolor{textcolor}%
\pgftext[x=12.751718in, y=-1.696250in, left, base,rotate=90.000000]{\color{textcolor}\rmfamily\fontsize{10.000000}{12.000000}\selectfont com.ak.ta.dainikbhaskar.activity.apk}%
\end{pgfscope}%
\begin{pgfscope}%
\pgfsetbuttcap%
\pgfsetroundjoin%
\definecolor{currentfill}{rgb}{0.000000,0.000000,0.000000}%
\pgfsetfillcolor{currentfill}%
\pgfsetlinewidth{0.803000pt}%
\definecolor{currentstroke}{rgb}{0.000000,0.000000,0.000000}%
\pgfsetstrokecolor{currentstroke}%
\pgfsetdash{}{0pt}%
\pgfsys@defobject{currentmarker}{\pgfqpoint{0.000000in}{-0.048611in}}{\pgfqpoint{0.000000in}{0.000000in}}{%
\pgfpathmoveto{\pgfqpoint{0.000000in}{0.000000in}}%
\pgfpathlineto{\pgfqpoint{0.000000in}{-0.048611in}}%
\pgfusepath{stroke,fill}%
}%
\begin{pgfscope}%
\pgfsys@transformshift{12.841148in}{0.625000in}%
\pgfsys@useobject{currentmarker}{}%
\end{pgfscope}%
\end{pgfscope}%
\begin{pgfscope}%
\definecolor{textcolor}{rgb}{0.000000,0.000000,0.000000}%
\pgfsetstrokecolor{textcolor}%
\pgfsetfillcolor{textcolor}%
\pgftext[x=12.876634in, y=-1.292639in, left, base,rotate=90.000000]{\color{textcolor}\rmfamily\fontsize{10.000000}{12.000000}\selectfont com.cloudmosa.puffinFree.apk}%
\end{pgfscope}%
\begin{pgfscope}%
\pgfsetbuttcap%
\pgfsetroundjoin%
\definecolor{currentfill}{rgb}{0.000000,0.000000,0.000000}%
\pgfsetfillcolor{currentfill}%
\pgfsetlinewidth{0.803000pt}%
\definecolor{currentstroke}{rgb}{0.000000,0.000000,0.000000}%
\pgfsetstrokecolor{currentstroke}%
\pgfsetdash{}{0pt}%
\pgfsys@defobject{currentmarker}{\pgfqpoint{0.000000in}{-0.048611in}}{\pgfqpoint{0.000000in}{0.000000in}}{%
\pgfpathmoveto{\pgfqpoint{0.000000in}{0.000000in}}%
\pgfpathlineto{\pgfqpoint{0.000000in}{-0.048611in}}%
\pgfusepath{stroke,fill}%
}%
\begin{pgfscope}%
\pgfsys@transformshift{12.964536in}{0.625000in}%
\pgfsys@useobject{currentmarker}{}%
\end{pgfscope}%
\end{pgfscope}%
\begin{pgfscope}%
\definecolor{textcolor}{rgb}{0.000000,0.000000,0.000000}%
\pgfsetstrokecolor{textcolor}%
\pgfsetfillcolor{textcolor}%
\pgftext[x=12.998425in, y=-2.076806in, left, base,rotate=90.000000]{\color{textcolor}\rmfamily\fontsize{10.000000}{12.000000}\selectfont com.spinmaster.enterprise.colleggtibles.apk}%
\end{pgfscope}%
\begin{pgfscope}%
\pgfsetbuttcap%
\pgfsetroundjoin%
\definecolor{currentfill}{rgb}{0.000000,0.000000,0.000000}%
\pgfsetfillcolor{currentfill}%
\pgfsetlinewidth{0.803000pt}%
\definecolor{currentstroke}{rgb}{0.000000,0.000000,0.000000}%
\pgfsetstrokecolor{currentstroke}%
\pgfsetdash{}{0pt}%
\pgfsys@defobject{currentmarker}{\pgfqpoint{0.000000in}{-0.048611in}}{\pgfqpoint{0.000000in}{0.000000in}}{%
\pgfpathmoveto{\pgfqpoint{0.000000in}{0.000000in}}%
\pgfpathlineto{\pgfqpoint{0.000000in}{-0.048611in}}%
\pgfusepath{stroke,fill}%
}%
\begin{pgfscope}%
\pgfsys@transformshift{13.087924in}{0.625000in}%
\pgfsys@useobject{currentmarker}{}%
\end{pgfscope}%
\end{pgfscope}%
\begin{pgfscope}%
\definecolor{textcolor}{rgb}{0.000000,0.000000,0.000000}%
\pgfsetstrokecolor{textcolor}%
\pgfsetfillcolor{textcolor}%
\pgftext[x=13.123410in, y=-0.865833in, left, base,rotate=90.000000]{\color{textcolor}\rmfamily\fontsize{10.000000}{12.000000}\selectfont com.finlim.forkapp.apk}%
\end{pgfscope}%
\begin{pgfscope}%
\pgfsetbuttcap%
\pgfsetroundjoin%
\definecolor{currentfill}{rgb}{0.000000,0.000000,0.000000}%
\pgfsetfillcolor{currentfill}%
\pgfsetlinewidth{0.803000pt}%
\definecolor{currentstroke}{rgb}{0.000000,0.000000,0.000000}%
\pgfsetstrokecolor{currentstroke}%
\pgfsetdash{}{0pt}%
\pgfsys@defobject{currentmarker}{\pgfqpoint{0.000000in}{-0.048611in}}{\pgfqpoint{0.000000in}{0.000000in}}{%
\pgfpathmoveto{\pgfqpoint{0.000000in}{0.000000in}}%
\pgfpathlineto{\pgfqpoint{0.000000in}{-0.048611in}}%
\pgfusepath{stroke,fill}%
}%
\begin{pgfscope}%
\pgfsys@transformshift{13.211312in}{0.625000in}%
\pgfsys@useobject{currentmarker}{}%
\end{pgfscope}%
\end{pgfscope}%
\begin{pgfscope}%
\definecolor{textcolor}{rgb}{0.000000,0.000000,0.000000}%
\pgfsetstrokecolor{textcolor}%
\pgfsetfillcolor{textcolor}%
\pgftext[x=13.245201in, y=-1.820833in, left, base,rotate=90.000000]{\color{textcolor}\rmfamily\fontsize{10.000000}{12.000000}\selectfont com.planner5d.swedishhomedesign.apk}%
\end{pgfscope}%
\begin{pgfscope}%
\pgfsetbuttcap%
\pgfsetroundjoin%
\definecolor{currentfill}{rgb}{0.000000,0.000000,0.000000}%
\pgfsetfillcolor{currentfill}%
\pgfsetlinewidth{0.803000pt}%
\definecolor{currentstroke}{rgb}{0.000000,0.000000,0.000000}%
\pgfsetstrokecolor{currentstroke}%
\pgfsetdash{}{0pt}%
\pgfsys@defobject{currentmarker}{\pgfqpoint{0.000000in}{-0.048611in}}{\pgfqpoint{0.000000in}{0.000000in}}{%
\pgfpathmoveto{\pgfqpoint{0.000000in}{0.000000in}}%
\pgfpathlineto{\pgfqpoint{0.000000in}{-0.048611in}}%
\pgfusepath{stroke,fill}%
}%
\begin{pgfscope}%
\pgfsys@transformshift{13.334700in}{0.625000in}%
\pgfsys@useobject{currentmarker}{}%
\end{pgfscope}%
\end{pgfscope}%
\begin{pgfscope}%
\definecolor{textcolor}{rgb}{0.000000,0.000000,0.000000}%
\pgfsetstrokecolor{textcolor}%
\pgfsetfillcolor{textcolor}%
\pgftext[x=13.369422in, y=-0.711250in, left, base,rotate=90.000000]{\color{textcolor}\rmfamily\fontsize{10.000000}{12.000000}\selectfont com.ace.android.apk}%
\end{pgfscope}%
\begin{pgfscope}%
\pgfsetbuttcap%
\pgfsetroundjoin%
\definecolor{currentfill}{rgb}{0.000000,0.000000,0.000000}%
\pgfsetfillcolor{currentfill}%
\pgfsetlinewidth{0.803000pt}%
\definecolor{currentstroke}{rgb}{0.000000,0.000000,0.000000}%
\pgfsetstrokecolor{currentstroke}%
\pgfsetdash{}{0pt}%
\pgfsys@defobject{currentmarker}{\pgfqpoint{0.000000in}{-0.048611in}}{\pgfqpoint{0.000000in}{0.000000in}}{%
\pgfpathmoveto{\pgfqpoint{0.000000in}{0.000000in}}%
\pgfpathlineto{\pgfqpoint{0.000000in}{-0.048611in}}%
\pgfusepath{stroke,fill}%
}%
\begin{pgfscope}%
\pgfsys@transformshift{13.458088in}{0.625000in}%
\pgfsys@useobject{currentmarker}{}%
\end{pgfscope}%
\end{pgfscope}%
\begin{pgfscope}%
\definecolor{textcolor}{rgb}{0.000000,0.000000,0.000000}%
\pgfsetstrokecolor{textcolor}%
\pgfsetfillcolor{textcolor}%
\pgftext[x=13.492810in, y=-0.973194in, left, base,rotate=90.000000]{\color{textcolor}\rmfamily\fontsize{10.000000}{12.000000}\selectfont com.lafiya.telehealth.apk}%
\end{pgfscope}%
\begin{pgfscope}%
\pgfsetbuttcap%
\pgfsetroundjoin%
\definecolor{currentfill}{rgb}{0.000000,0.000000,0.000000}%
\pgfsetfillcolor{currentfill}%
\pgfsetlinewidth{0.803000pt}%
\definecolor{currentstroke}{rgb}{0.000000,0.000000,0.000000}%
\pgfsetstrokecolor{currentstroke}%
\pgfsetdash{}{0pt}%
\pgfsys@defobject{currentmarker}{\pgfqpoint{0.000000in}{-0.048611in}}{\pgfqpoint{0.000000in}{0.000000in}}{%
\pgfpathmoveto{\pgfqpoint{0.000000in}{0.000000in}}%
\pgfpathlineto{\pgfqpoint{0.000000in}{-0.048611in}}%
\pgfusepath{stroke,fill}%
}%
\begin{pgfscope}%
\pgfsys@transformshift{13.581476in}{0.625000in}%
\pgfsys@useobject{currentmarker}{}%
\end{pgfscope}%
\end{pgfscope}%
\begin{pgfscope}%
\definecolor{textcolor}{rgb}{0.000000,0.000000,0.000000}%
\pgfsetstrokecolor{textcolor}%
\pgfsetfillcolor{textcolor}%
\pgftext[x=13.616198in, y=-0.553055in, left, base,rotate=90.000000]{\color{textcolor}\rmfamily\fontsize{10.000000}{12.000000}\selectfont com.waitrapp.apk}%
\end{pgfscope}%
\begin{pgfscope}%
\pgfsetbuttcap%
\pgfsetroundjoin%
\definecolor{currentfill}{rgb}{0.000000,0.000000,0.000000}%
\pgfsetfillcolor{currentfill}%
\pgfsetlinewidth{0.803000pt}%
\definecolor{currentstroke}{rgb}{0.000000,0.000000,0.000000}%
\pgfsetstrokecolor{currentstroke}%
\pgfsetdash{}{0pt}%
\pgfsys@defobject{currentmarker}{\pgfqpoint{0.000000in}{-0.048611in}}{\pgfqpoint{0.000000in}{0.000000in}}{%
\pgfpathmoveto{\pgfqpoint{0.000000in}{0.000000in}}%
\pgfpathlineto{\pgfqpoint{0.000000in}{-0.048611in}}%
\pgfusepath{stroke,fill}%
}%
\begin{pgfscope}%
\pgfsys@transformshift{13.704864in}{0.625000in}%
\pgfsys@useobject{currentmarker}{}%
\end{pgfscope}%
\end{pgfscope}%
\begin{pgfscope}%
\definecolor{textcolor}{rgb}{0.000000,0.000000,0.000000}%
\pgfsetstrokecolor{textcolor}%
\pgfsetfillcolor{textcolor}%
\pgftext[x=13.739586in, y=-1.706667in, left, base,rotate=90.000000]{\color{textcolor}\rmfamily\fontsize{10.000000}{12.000000}\selectfont com.match.android.matchmobile.apk}%
\end{pgfscope}%
\begin{pgfscope}%
\pgfsetbuttcap%
\pgfsetroundjoin%
\definecolor{currentfill}{rgb}{0.000000,0.000000,0.000000}%
\pgfsetfillcolor{currentfill}%
\pgfsetlinewidth{0.803000pt}%
\definecolor{currentstroke}{rgb}{0.000000,0.000000,0.000000}%
\pgfsetstrokecolor{currentstroke}%
\pgfsetdash{}{0pt}%
\pgfsys@defobject{currentmarker}{\pgfqpoint{0.000000in}{-0.048611in}}{\pgfqpoint{0.000000in}{0.000000in}}{%
\pgfpathmoveto{\pgfqpoint{0.000000in}{0.000000in}}%
\pgfpathlineto{\pgfqpoint{0.000000in}{-0.048611in}}%
\pgfusepath{stroke,fill}%
}%
\begin{pgfscope}%
\pgfsys@transformshift{13.828252in}{0.625000in}%
\pgfsys@useobject{currentmarker}{}%
\end{pgfscope}%
\end{pgfscope}%
\begin{pgfscope}%
\definecolor{textcolor}{rgb}{0.000000,0.000000,0.000000}%
\pgfsetstrokecolor{textcolor}%
\pgfsetfillcolor{textcolor}%
\pgftext[x=13.862974in, y=-1.112361in, left, base,rotate=90.000000]{\color{textcolor}\rmfamily\fontsize{10.000000}{12.000000}\selectfont com.clouthub.clouthub.apk}%
\end{pgfscope}%
\begin{pgfscope}%
\pgfsetbuttcap%
\pgfsetroundjoin%
\definecolor{currentfill}{rgb}{0.000000,0.000000,0.000000}%
\pgfsetfillcolor{currentfill}%
\pgfsetlinewidth{0.803000pt}%
\definecolor{currentstroke}{rgb}{0.000000,0.000000,0.000000}%
\pgfsetstrokecolor{currentstroke}%
\pgfsetdash{}{0pt}%
\pgfsys@defobject{currentmarker}{\pgfqpoint{0.000000in}{-0.048611in}}{\pgfqpoint{0.000000in}{0.000000in}}{%
\pgfpathmoveto{\pgfqpoint{0.000000in}{0.000000in}}%
\pgfpathlineto{\pgfqpoint{0.000000in}{-0.048611in}}%
\pgfusepath{stroke,fill}%
}%
\begin{pgfscope}%
\pgfsys@transformshift{13.951640in}{0.625000in}%
\pgfsys@useobject{currentmarker}{}%
\end{pgfscope}%
\end{pgfscope}%
\begin{pgfscope}%
\definecolor{textcolor}{rgb}{0.000000,0.000000,0.000000}%
\pgfsetstrokecolor{textcolor}%
\pgfsetfillcolor{textcolor}%
\pgftext[x=13.986362in, y=-1.066528in, left, base,rotate=90.000000]{\color{textcolor}\rmfamily\fontsize{10.000000}{12.000000}\selectfont com.belk.android.belk.apk}%
\end{pgfscope}%
\begin{pgfscope}%
\pgfsetbuttcap%
\pgfsetroundjoin%
\definecolor{currentfill}{rgb}{0.000000,0.000000,0.000000}%
\pgfsetfillcolor{currentfill}%
\pgfsetlinewidth{0.803000pt}%
\definecolor{currentstroke}{rgb}{0.000000,0.000000,0.000000}%
\pgfsetstrokecolor{currentstroke}%
\pgfsetdash{}{0pt}%
\pgfsys@defobject{currentmarker}{\pgfqpoint{0.000000in}{-0.048611in}}{\pgfqpoint{0.000000in}{0.000000in}}{%
\pgfpathmoveto{\pgfqpoint{0.000000in}{0.000000in}}%
\pgfpathlineto{\pgfqpoint{0.000000in}{-0.048611in}}%
\pgfusepath{stroke,fill}%
}%
\begin{pgfscope}%
\pgfsys@transformshift{14.075028in}{0.625000in}%
\pgfsys@useobject{currentmarker}{}%
\end{pgfscope}%
\end{pgfscope}%
\begin{pgfscope}%
\definecolor{textcolor}{rgb}{0.000000,0.000000,0.000000}%
\pgfsetstrokecolor{textcolor}%
\pgfsetfillcolor{textcolor}%
\pgftext[x=14.109750in, y=-1.673472in, left, base,rotate=90.000000]{\color{textcolor}\rmfamily\fontsize{10.000000}{12.000000}\selectfont com.lyrebirdstudio.face\_camera.apk}%
\end{pgfscope}%
\begin{pgfscope}%
\pgfsetbuttcap%
\pgfsetroundjoin%
\definecolor{currentfill}{rgb}{0.000000,0.000000,0.000000}%
\pgfsetfillcolor{currentfill}%
\pgfsetlinewidth{0.803000pt}%
\definecolor{currentstroke}{rgb}{0.000000,0.000000,0.000000}%
\pgfsetstrokecolor{currentstroke}%
\pgfsetdash{}{0pt}%
\pgfsys@defobject{currentmarker}{\pgfqpoint{0.000000in}{-0.048611in}}{\pgfqpoint{0.000000in}{0.000000in}}{%
\pgfpathmoveto{\pgfqpoint{0.000000in}{0.000000in}}%
\pgfpathlineto{\pgfqpoint{0.000000in}{-0.048611in}}%
\pgfusepath{stroke,fill}%
}%
\begin{pgfscope}%
\pgfsys@transformshift{14.198416in}{0.625000in}%
\pgfsys@useobject{currentmarker}{}%
\end{pgfscope}%
\end{pgfscope}%
\begin{pgfscope}%
\definecolor{textcolor}{rgb}{0.000000,0.000000,0.000000}%
\pgfsetstrokecolor{textcolor}%
\pgfsetfillcolor{textcolor}%
\pgftext[x=14.233902in, y=-1.920277in, left, base,rotate=90.000000]{\color{textcolor}\rmfamily\fontsize{10.000000}{12.000000}\selectfont com.poqstudio.app.platform.boohoo.apk}%
\end{pgfscope}%
\begin{pgfscope}%
\pgfsetbuttcap%
\pgfsetroundjoin%
\definecolor{currentfill}{rgb}{0.000000,0.000000,0.000000}%
\pgfsetfillcolor{currentfill}%
\pgfsetlinewidth{0.803000pt}%
\definecolor{currentstroke}{rgb}{0.000000,0.000000,0.000000}%
\pgfsetstrokecolor{currentstroke}%
\pgfsetdash{}{0pt}%
\pgfsys@defobject{currentmarker}{\pgfqpoint{0.000000in}{-0.048611in}}{\pgfqpoint{0.000000in}{0.000000in}}{%
\pgfpathmoveto{\pgfqpoint{0.000000in}{0.000000in}}%
\pgfpathlineto{\pgfqpoint{0.000000in}{-0.048611in}}%
\pgfusepath{stroke,fill}%
}%
\begin{pgfscope}%
\pgfsys@transformshift{14.321804in}{0.625000in}%
\pgfsys@useobject{currentmarker}{}%
\end{pgfscope}%
\end{pgfscope}%
\begin{pgfscope}%
\definecolor{textcolor}{rgb}{0.000000,0.000000,0.000000}%
\pgfsetstrokecolor{textcolor}%
\pgfsetfillcolor{textcolor}%
\pgftext[x=14.356457in, y=-0.896805in, left, base,rotate=90.000000]{\color{textcolor}\rmfamily\fontsize{10.000000}{12.000000}\selectfont com.pinger.textfree.apk}%
\end{pgfscope}%
\begin{pgfscope}%
\pgfsetbuttcap%
\pgfsetroundjoin%
\definecolor{currentfill}{rgb}{0.000000,0.000000,0.000000}%
\pgfsetfillcolor{currentfill}%
\pgfsetlinewidth{0.803000pt}%
\definecolor{currentstroke}{rgb}{0.000000,0.000000,0.000000}%
\pgfsetstrokecolor{currentstroke}%
\pgfsetdash{}{0pt}%
\pgfsys@defobject{currentmarker}{\pgfqpoint{0.000000in}{-0.048611in}}{\pgfqpoint{0.000000in}{0.000000in}}{%
\pgfpathmoveto{\pgfqpoint{0.000000in}{0.000000in}}%
\pgfpathlineto{\pgfqpoint{0.000000in}{-0.048611in}}%
\pgfusepath{stroke,fill}%
}%
\begin{pgfscope}%
\pgfsys@transformshift{14.445192in}{0.625000in}%
\pgfsys@useobject{currentmarker}{}%
\end{pgfscope}%
\end{pgfscope}%
\begin{pgfscope}%
\definecolor{textcolor}{rgb}{0.000000,0.000000,0.000000}%
\pgfsetstrokecolor{textcolor}%
\pgfsetfillcolor{textcolor}%
\pgftext[x=14.479150in, y=-1.086944in, left, base,rotate=90.000000]{\color{textcolor}\rmfamily\fontsize{10.000000}{12.000000}\selectfont com.lyrebirdstudio.tbt.apk}%
\end{pgfscope}%
\begin{pgfscope}%
\pgfsetbuttcap%
\pgfsetroundjoin%
\definecolor{currentfill}{rgb}{0.000000,0.000000,0.000000}%
\pgfsetfillcolor{currentfill}%
\pgfsetlinewidth{0.803000pt}%
\definecolor{currentstroke}{rgb}{0.000000,0.000000,0.000000}%
\pgfsetstrokecolor{currentstroke}%
\pgfsetdash{}{0pt}%
\pgfsys@defobject{currentmarker}{\pgfqpoint{0.000000in}{-0.048611in}}{\pgfqpoint{0.000000in}{0.000000in}}{%
\pgfpathmoveto{\pgfqpoint{0.000000in}{0.000000in}}%
\pgfpathlineto{\pgfqpoint{0.000000in}{-0.048611in}}%
\pgfusepath{stroke,fill}%
}%
\begin{pgfscope}%
\pgfsys@transformshift{14.568580in}{0.625000in}%
\pgfsys@useobject{currentmarker}{}%
\end{pgfscope}%
\end{pgfscope}%
\begin{pgfscope}%
\definecolor{textcolor}{rgb}{0.000000,0.000000,0.000000}%
\pgfsetstrokecolor{textcolor}%
\pgfsetfillcolor{textcolor}%
\pgftext[x=14.602538in, y=-0.317916in, left, base,rotate=90.000000]{\color{textcolor}\rmfamily\fontsize{10.000000}{12.000000}\selectfont co.brainly.apk}%
\end{pgfscope}%
\begin{pgfscope}%
\pgfsetbuttcap%
\pgfsetroundjoin%
\definecolor{currentfill}{rgb}{0.000000,0.000000,0.000000}%
\pgfsetfillcolor{currentfill}%
\pgfsetlinewidth{0.803000pt}%
\definecolor{currentstroke}{rgb}{0.000000,0.000000,0.000000}%
\pgfsetstrokecolor{currentstroke}%
\pgfsetdash{}{0pt}%
\pgfsys@defobject{currentmarker}{\pgfqpoint{0.000000in}{-0.048611in}}{\pgfqpoint{0.000000in}{0.000000in}}{%
\pgfpathmoveto{\pgfqpoint{0.000000in}{0.000000in}}%
\pgfpathlineto{\pgfqpoint{0.000000in}{-0.048611in}}%
\pgfusepath{stroke,fill}%
}%
\begin{pgfscope}%
\pgfsys@transformshift{14.691968in}{0.625000in}%
\pgfsys@useobject{currentmarker}{}%
\end{pgfscope}%
\end{pgfscope}%
\begin{pgfscope}%
\definecolor{textcolor}{rgb}{0.000000,0.000000,0.000000}%
\pgfsetstrokecolor{textcolor}%
\pgfsetfillcolor{textcolor}%
\pgftext[x=14.726690in, y=-0.218194in, left, base,rotate=90.000000]{\color{textcolor}\rmfamily\fontsize{10.000000}{12.000000}\selectfont ir.nasim.apk}%
\end{pgfscope}%
\begin{pgfscope}%
\pgfsetbuttcap%
\pgfsetroundjoin%
\definecolor{currentfill}{rgb}{0.000000,0.000000,0.000000}%
\pgfsetfillcolor{currentfill}%
\pgfsetlinewidth{0.803000pt}%
\definecolor{currentstroke}{rgb}{0.000000,0.000000,0.000000}%
\pgfsetstrokecolor{currentstroke}%
\pgfsetdash{}{0pt}%
\pgfsys@defobject{currentmarker}{\pgfqpoint{0.000000in}{-0.048611in}}{\pgfqpoint{0.000000in}{0.000000in}}{%
\pgfpathmoveto{\pgfqpoint{0.000000in}{0.000000in}}%
\pgfpathlineto{\pgfqpoint{0.000000in}{-0.048611in}}%
\pgfusepath{stroke,fill}%
}%
\begin{pgfscope}%
\pgfsys@transformshift{14.815356in}{0.625000in}%
\pgfsys@useobject{currentmarker}{}%
\end{pgfscope}%
\end{pgfscope}%
\begin{pgfscope}%
\definecolor{textcolor}{rgb}{0.000000,0.000000,0.000000}%
\pgfsetstrokecolor{textcolor}%
\pgfsetfillcolor{textcolor}%
\pgftext[x=14.849245in, y=-2.054444in, left, base,rotate=90.000000]{\color{textcolor}\rmfamily\fontsize{10.000000}{12.000000}\selectfont photocollage.photoeditor.collagemaker.apk}%
\end{pgfscope}%
\begin{pgfscope}%
\pgfsetbuttcap%
\pgfsetroundjoin%
\definecolor{currentfill}{rgb}{0.000000,0.000000,0.000000}%
\pgfsetfillcolor{currentfill}%
\pgfsetlinewidth{0.803000pt}%
\definecolor{currentstroke}{rgb}{0.000000,0.000000,0.000000}%
\pgfsetstrokecolor{currentstroke}%
\pgfsetdash{}{0pt}%
\pgfsys@defobject{currentmarker}{\pgfqpoint{0.000000in}{-0.048611in}}{\pgfqpoint{0.000000in}{0.000000in}}{%
\pgfpathmoveto{\pgfqpoint{0.000000in}{0.000000in}}%
\pgfpathlineto{\pgfqpoint{0.000000in}{-0.048611in}}%
\pgfusepath{stroke,fill}%
}%
\begin{pgfscope}%
\pgfsys@transformshift{14.938744in}{0.625000in}%
\pgfsys@useobject{currentmarker}{}%
\end{pgfscope}%
\end{pgfscope}%
\begin{pgfscope}%
\definecolor{textcolor}{rgb}{0.000000,0.000000,0.000000}%
\pgfsetstrokecolor{textcolor}%
\pgfsetfillcolor{textcolor}%
\pgftext[x=14.973466in, y=-0.765972in, left, base,rotate=90.000000]{\color{textcolor}\rmfamily\fontsize{10.000000}{12.000000}\selectfont com.uba.vericash.apk}%
\end{pgfscope}%
\begin{pgfscope}%
\pgfsetbuttcap%
\pgfsetroundjoin%
\definecolor{currentfill}{rgb}{0.000000,0.000000,0.000000}%
\pgfsetfillcolor{currentfill}%
\pgfsetlinewidth{0.803000pt}%
\definecolor{currentstroke}{rgb}{0.000000,0.000000,0.000000}%
\pgfsetstrokecolor{currentstroke}%
\pgfsetdash{}{0pt}%
\pgfsys@defobject{currentmarker}{\pgfqpoint{0.000000in}{-0.048611in}}{\pgfqpoint{0.000000in}{0.000000in}}{%
\pgfpathmoveto{\pgfqpoint{0.000000in}{0.000000in}}%
\pgfpathlineto{\pgfqpoint{0.000000in}{-0.048611in}}%
\pgfusepath{stroke,fill}%
}%
\begin{pgfscope}%
\pgfsys@transformshift{15.062132in}{0.625000in}%
\pgfsys@useobject{currentmarker}{}%
\end{pgfscope}%
\end{pgfscope}%
\begin{pgfscope}%
\definecolor{textcolor}{rgb}{0.000000,0.000000,0.000000}%
\pgfsetstrokecolor{textcolor}%
\pgfsetfillcolor{textcolor}%
\pgftext[x=15.096090in, y=-1.224861in, left, base,rotate=90.000000]{\color{textcolor}\rmfamily\fontsize{10.000000}{12.000000}\selectfont com.rubenmayayo.reddit.apk}%
\end{pgfscope}%
\begin{pgfscope}%
\pgfsetbuttcap%
\pgfsetroundjoin%
\definecolor{currentfill}{rgb}{0.000000,0.000000,0.000000}%
\pgfsetfillcolor{currentfill}%
\pgfsetlinewidth{0.803000pt}%
\definecolor{currentstroke}{rgb}{0.000000,0.000000,0.000000}%
\pgfsetstrokecolor{currentstroke}%
\pgfsetdash{}{0pt}%
\pgfsys@defobject{currentmarker}{\pgfqpoint{0.000000in}{-0.048611in}}{\pgfqpoint{0.000000in}{0.000000in}}{%
\pgfpathmoveto{\pgfqpoint{0.000000in}{0.000000in}}%
\pgfpathlineto{\pgfqpoint{0.000000in}{-0.048611in}}%
\pgfusepath{stroke,fill}%
}%
\begin{pgfscope}%
\pgfsys@transformshift{15.185520in}{0.625000in}%
\pgfsys@useobject{currentmarker}{}%
\end{pgfscope}%
\end{pgfscope}%
\begin{pgfscope}%
\definecolor{textcolor}{rgb}{0.000000,0.000000,0.000000}%
\pgfsetstrokecolor{textcolor}%
\pgfsetfillcolor{textcolor}%
\pgftext[x=15.220242in, y=-0.792639in, left, base,rotate=90.000000]{\color{textcolor}\rmfamily\fontsize{10.000000}{12.000000}\selectfont com.adobe.reader.apk}%
\end{pgfscope}%
\begin{pgfscope}%
\pgfsetbuttcap%
\pgfsetroundjoin%
\definecolor{currentfill}{rgb}{0.000000,0.000000,0.000000}%
\pgfsetfillcolor{currentfill}%
\pgfsetlinewidth{0.803000pt}%
\definecolor{currentstroke}{rgb}{0.000000,0.000000,0.000000}%
\pgfsetstrokecolor{currentstroke}%
\pgfsetdash{}{0pt}%
\pgfsys@defobject{currentmarker}{\pgfqpoint{0.000000in}{-0.048611in}}{\pgfqpoint{0.000000in}{0.000000in}}{%
\pgfpathmoveto{\pgfqpoint{0.000000in}{0.000000in}}%
\pgfpathlineto{\pgfqpoint{0.000000in}{-0.048611in}}%
\pgfusepath{stroke,fill}%
}%
\begin{pgfscope}%
\pgfsys@transformshift{15.308908in}{0.625000in}%
\pgfsys@useobject{currentmarker}{}%
\end{pgfscope}%
\end{pgfscope}%
\begin{pgfscope}%
\definecolor{textcolor}{rgb}{0.000000,0.000000,0.000000}%
\pgfsetstrokecolor{textcolor}%
\pgfsetfillcolor{textcolor}%
\pgftext[x=15.342866in, y=-1.109861in, left, base,rotate=90.000000]{\color{textcolor}\rmfamily\fontsize{10.000000}{12.000000}\selectfont com.musicplayer.music.apk}%
\end{pgfscope}%
\begin{pgfscope}%
\pgfsetbuttcap%
\pgfsetroundjoin%
\definecolor{currentfill}{rgb}{0.000000,0.000000,0.000000}%
\pgfsetfillcolor{currentfill}%
\pgfsetlinewidth{0.803000pt}%
\definecolor{currentstroke}{rgb}{0.000000,0.000000,0.000000}%
\pgfsetstrokecolor{currentstroke}%
\pgfsetdash{}{0pt}%
\pgfsys@defobject{currentmarker}{\pgfqpoint{0.000000in}{-0.048611in}}{\pgfqpoint{0.000000in}{0.000000in}}{%
\pgfpathmoveto{\pgfqpoint{0.000000in}{0.000000in}}%
\pgfpathlineto{\pgfqpoint{0.000000in}{-0.048611in}}%
\pgfusepath{stroke,fill}%
}%
\begin{pgfscope}%
\pgfsys@transformshift{15.432296in}{0.625000in}%
\pgfsys@useobject{currentmarker}{}%
\end{pgfscope}%
\end{pgfscope}%
\begin{pgfscope}%
\definecolor{textcolor}{rgb}{0.000000,0.000000,0.000000}%
\pgfsetstrokecolor{textcolor}%
\pgfsetfillcolor{textcolor}%
\pgftext[x=15.467018in, y=-0.379861in, left, base,rotate=90.000000]{\color{textcolor}\rmfamily\fontsize{10.000000}{12.000000}\selectfont it.casa.app.apk}%
\end{pgfscope}%
\begin{pgfscope}%
\pgfsetbuttcap%
\pgfsetroundjoin%
\definecolor{currentfill}{rgb}{0.000000,0.000000,0.000000}%
\pgfsetfillcolor{currentfill}%
\pgfsetlinewidth{0.803000pt}%
\definecolor{currentstroke}{rgb}{0.000000,0.000000,0.000000}%
\pgfsetstrokecolor{currentstroke}%
\pgfsetdash{}{0pt}%
\pgfsys@defobject{currentmarker}{\pgfqpoint{0.000000in}{-0.048611in}}{\pgfqpoint{0.000000in}{0.000000in}}{%
\pgfpathmoveto{\pgfqpoint{0.000000in}{0.000000in}}%
\pgfpathlineto{\pgfqpoint{0.000000in}{-0.048611in}}%
\pgfusepath{stroke,fill}%
}%
\begin{pgfscope}%
\pgfsys@transformshift{15.555684in}{0.625000in}%
\pgfsys@useobject{currentmarker}{}%
\end{pgfscope}%
\end{pgfscope}%
\begin{pgfscope}%
\definecolor{textcolor}{rgb}{0.000000,0.000000,0.000000}%
\pgfsetstrokecolor{textcolor}%
\pgfsetfillcolor{textcolor}%
\pgftext[x=15.590406in, y=-0.390694in, left, base,rotate=90.000000]{\color{textcolor}\rmfamily\fontsize{10.000000}{12.000000}\selectfont com.tubitv.apk}%
\end{pgfscope}%
\begin{pgfscope}%
\pgfsetbuttcap%
\pgfsetroundjoin%
\definecolor{currentfill}{rgb}{0.000000,0.000000,0.000000}%
\pgfsetfillcolor{currentfill}%
\pgfsetlinewidth{0.803000pt}%
\definecolor{currentstroke}{rgb}{0.000000,0.000000,0.000000}%
\pgfsetstrokecolor{currentstroke}%
\pgfsetdash{}{0pt}%
\pgfsys@defobject{currentmarker}{\pgfqpoint{0.000000in}{-0.048611in}}{\pgfqpoint{0.000000in}{0.000000in}}{%
\pgfpathmoveto{\pgfqpoint{0.000000in}{0.000000in}}%
\pgfpathlineto{\pgfqpoint{0.000000in}{-0.048611in}}%
\pgfusepath{stroke,fill}%
}%
\begin{pgfscope}%
\pgfsys@transformshift{15.679072in}{0.625000in}%
\pgfsys@useobject{currentmarker}{}%
\end{pgfscope}%
\end{pgfscope}%
\begin{pgfscope}%
\definecolor{textcolor}{rgb}{0.000000,0.000000,0.000000}%
\pgfsetstrokecolor{textcolor}%
\pgfsetfillcolor{textcolor}%
\pgftext[x=15.713725in, y=-1.344166in, left, base,rotate=90.000000]{\color{textcolor}\rmfamily\fontsize{10.000000}{12.000000}\selectfont info.yogantara.utmgeomap.apk}%
\end{pgfscope}%
\begin{pgfscope}%
\pgfsetbuttcap%
\pgfsetroundjoin%
\definecolor{currentfill}{rgb}{0.000000,0.000000,0.000000}%
\pgfsetfillcolor{currentfill}%
\pgfsetlinewidth{0.803000pt}%
\definecolor{currentstroke}{rgb}{0.000000,0.000000,0.000000}%
\pgfsetstrokecolor{currentstroke}%
\pgfsetdash{}{0pt}%
\pgfsys@defobject{currentmarker}{\pgfqpoint{0.000000in}{-0.048611in}}{\pgfqpoint{0.000000in}{0.000000in}}{%
\pgfpathmoveto{\pgfqpoint{0.000000in}{0.000000in}}%
\pgfpathlineto{\pgfqpoint{0.000000in}{-0.048611in}}%
\pgfusepath{stroke,fill}%
}%
\begin{pgfscope}%
\pgfsys@transformshift{15.802460in}{0.625000in}%
\pgfsys@useobject{currentmarker}{}%
\end{pgfscope}%
\end{pgfscope}%
\begin{pgfscope}%
\definecolor{textcolor}{rgb}{0.000000,0.000000,0.000000}%
\pgfsetstrokecolor{textcolor}%
\pgfsetfillcolor{textcolor}%
\pgftext[x=15.837182in, y=-0.882361in, left, base,rotate=90.000000]{\color{textcolor}\rmfamily\fontsize{10.000000}{12.000000}\selectfont com.thesouledstore.apk}%
\end{pgfscope}%
\begin{pgfscope}%
\pgfsetbuttcap%
\pgfsetroundjoin%
\definecolor{currentfill}{rgb}{0.000000,0.000000,0.000000}%
\pgfsetfillcolor{currentfill}%
\pgfsetlinewidth{0.803000pt}%
\definecolor{currentstroke}{rgb}{0.000000,0.000000,0.000000}%
\pgfsetstrokecolor{currentstroke}%
\pgfsetdash{}{0pt}%
\pgfsys@defobject{currentmarker}{\pgfqpoint{0.000000in}{-0.048611in}}{\pgfqpoint{0.000000in}{0.000000in}}{%
\pgfpathmoveto{\pgfqpoint{0.000000in}{0.000000in}}%
\pgfpathlineto{\pgfqpoint{0.000000in}{-0.048611in}}%
\pgfusepath{stroke,fill}%
}%
\begin{pgfscope}%
\pgfsys@transformshift{15.925848in}{0.625000in}%
\pgfsys@useobject{currentmarker}{}%
\end{pgfscope}%
\end{pgfscope}%
\begin{pgfscope}%
\definecolor{textcolor}{rgb}{0.000000,0.000000,0.000000}%
\pgfsetstrokecolor{textcolor}%
\pgfsetfillcolor{textcolor}%
\pgftext[x=15.960570in, y=-0.622083in, left, base,rotate=90.000000]{\color{textcolor}\rmfamily\fontsize{10.000000}{12.000000}\selectfont com.dazz.hoop.apk}%
\end{pgfscope}%
\begin{pgfscope}%
\pgfsetbuttcap%
\pgfsetroundjoin%
\definecolor{currentfill}{rgb}{0.000000,0.000000,0.000000}%
\pgfsetfillcolor{currentfill}%
\pgfsetlinewidth{0.803000pt}%
\definecolor{currentstroke}{rgb}{0.000000,0.000000,0.000000}%
\pgfsetstrokecolor{currentstroke}%
\pgfsetdash{}{0pt}%
\pgfsys@defobject{currentmarker}{\pgfqpoint{0.000000in}{-0.048611in}}{\pgfqpoint{0.000000in}{0.000000in}}{%
\pgfpathmoveto{\pgfqpoint{0.000000in}{0.000000in}}%
\pgfpathlineto{\pgfqpoint{0.000000in}{-0.048611in}}%
\pgfusepath{stroke,fill}%
}%
\begin{pgfscope}%
\pgfsys@transformshift{16.049236in}{0.625000in}%
\pgfsys@useobject{currentmarker}{}%
\end{pgfscope}%
\end{pgfscope}%
\begin{pgfscope}%
\definecolor{textcolor}{rgb}{0.000000,0.000000,0.000000}%
\pgfsetstrokecolor{textcolor}%
\pgfsetfillcolor{textcolor}%
\pgftext[x=16.083125in, y=-0.750972in, left, base,rotate=90.000000]{\color{textcolor}\rmfamily\fontsize{10.000000}{12.000000}\selectfont ir.tgbs.peccharge.apk}%
\end{pgfscope}%
\begin{pgfscope}%
\pgfsetbuttcap%
\pgfsetroundjoin%
\definecolor{currentfill}{rgb}{0.000000,0.000000,0.000000}%
\pgfsetfillcolor{currentfill}%
\pgfsetlinewidth{0.803000pt}%
\definecolor{currentstroke}{rgb}{0.000000,0.000000,0.000000}%
\pgfsetstrokecolor{currentstroke}%
\pgfsetdash{}{0pt}%
\pgfsys@defobject{currentmarker}{\pgfqpoint{0.000000in}{-0.048611in}}{\pgfqpoint{0.000000in}{0.000000in}}{%
\pgfpathmoveto{\pgfqpoint{0.000000in}{0.000000in}}%
\pgfpathlineto{\pgfqpoint{0.000000in}{-0.048611in}}%
\pgfusepath{stroke,fill}%
}%
\begin{pgfscope}%
\pgfsys@transformshift{16.172624in}{0.625000in}%
\pgfsys@useobject{currentmarker}{}%
\end{pgfscope}%
\end{pgfscope}%
\begin{pgfscope}%
\definecolor{textcolor}{rgb}{0.000000,0.000000,0.000000}%
\pgfsetstrokecolor{textcolor}%
\pgfsetfillcolor{textcolor}%
\pgftext[x=16.206513in, y=-1.642639in, left, base,rotate=90.000000]{\color{textcolor}\rmfamily\fontsize{10.000000}{12.000000}\selectfont com.sinyee.babybus.engineering.apk}%
\end{pgfscope}%
\begin{pgfscope}%
\pgfsetbuttcap%
\pgfsetroundjoin%
\definecolor{currentfill}{rgb}{0.000000,0.000000,0.000000}%
\pgfsetfillcolor{currentfill}%
\pgfsetlinewidth{0.803000pt}%
\definecolor{currentstroke}{rgb}{0.000000,0.000000,0.000000}%
\pgfsetstrokecolor{currentstroke}%
\pgfsetdash{}{0pt}%
\pgfsys@defobject{currentmarker}{\pgfqpoint{0.000000in}{-0.048611in}}{\pgfqpoint{0.000000in}{0.000000in}}{%
\pgfpathmoveto{\pgfqpoint{0.000000in}{0.000000in}}%
\pgfpathlineto{\pgfqpoint{0.000000in}{-0.048611in}}%
\pgfusepath{stroke,fill}%
}%
\begin{pgfscope}%
\pgfsys@transformshift{16.296012in}{0.625000in}%
\pgfsys@useobject{currentmarker}{}%
\end{pgfscope}%
\end{pgfscope}%
\begin{pgfscope}%
\definecolor{textcolor}{rgb}{0.000000,0.000000,0.000000}%
\pgfsetstrokecolor{textcolor}%
\pgfsetfillcolor{textcolor}%
\pgftext[x=16.329970in, y=-0.882083in, left, base,rotate=90.000000]{\color{textcolor}\rmfamily\fontsize{10.000000}{12.000000}\selectfont com.meesho.supply.apk}%
\end{pgfscope}%
\begin{pgfscope}%
\pgfsetbuttcap%
\pgfsetroundjoin%
\definecolor{currentfill}{rgb}{0.000000,0.000000,0.000000}%
\pgfsetfillcolor{currentfill}%
\pgfsetlinewidth{0.803000pt}%
\definecolor{currentstroke}{rgb}{0.000000,0.000000,0.000000}%
\pgfsetstrokecolor{currentstroke}%
\pgfsetdash{}{0pt}%
\pgfsys@defobject{currentmarker}{\pgfqpoint{0.000000in}{-0.048611in}}{\pgfqpoint{0.000000in}{0.000000in}}{%
\pgfpathmoveto{\pgfqpoint{0.000000in}{0.000000in}}%
\pgfpathlineto{\pgfqpoint{0.000000in}{-0.048611in}}%
\pgfusepath{stroke,fill}%
}%
\begin{pgfscope}%
\pgfsys@transformshift{16.419400in}{0.625000in}%
\pgfsys@useobject{currentmarker}{}%
\end{pgfscope}%
\end{pgfscope}%
\begin{pgfscope}%
\definecolor{textcolor}{rgb}{0.000000,0.000000,0.000000}%
\pgfsetstrokecolor{textcolor}%
\pgfsetfillcolor{textcolor}%
\pgftext[x=16.454886in, y=-1.297778in, left, base,rotate=90.000000]{\color{textcolor}\rmfamily\fontsize{10.000000}{12.000000}\selectfont fr.cnaf.mobile.moncompte.apk}%
\end{pgfscope}%
\begin{pgfscope}%
\pgfsetbuttcap%
\pgfsetroundjoin%
\definecolor{currentfill}{rgb}{0.000000,0.000000,0.000000}%
\pgfsetfillcolor{currentfill}%
\pgfsetlinewidth{0.803000pt}%
\definecolor{currentstroke}{rgb}{0.000000,0.000000,0.000000}%
\pgfsetstrokecolor{currentstroke}%
\pgfsetdash{}{0pt}%
\pgfsys@defobject{currentmarker}{\pgfqpoint{0.000000in}{-0.048611in}}{\pgfqpoint{0.000000in}{0.000000in}}{%
\pgfpathmoveto{\pgfqpoint{0.000000in}{0.000000in}}%
\pgfpathlineto{\pgfqpoint{0.000000in}{-0.048611in}}%
\pgfusepath{stroke,fill}%
}%
\begin{pgfscope}%
\pgfsys@transformshift{16.542788in}{0.625000in}%
\pgfsys@useobject{currentmarker}{}%
\end{pgfscope}%
\end{pgfscope}%
\begin{pgfscope}%
\definecolor{textcolor}{rgb}{0.000000,0.000000,0.000000}%
\pgfsetstrokecolor{textcolor}%
\pgfsetfillcolor{textcolor}%
\pgftext[x=16.577441in, y=-1.609305in, left, base,rotate=90.000000]{\color{textcolor}\rmfamily\fontsize{10.000000}{12.000000}\selectfont com.marlustudio.englishforkids.apk}%
\end{pgfscope}%
\begin{pgfscope}%
\pgfsetbuttcap%
\pgfsetroundjoin%
\definecolor{currentfill}{rgb}{0.000000,0.000000,0.000000}%
\pgfsetfillcolor{currentfill}%
\pgfsetlinewidth{0.803000pt}%
\definecolor{currentstroke}{rgb}{0.000000,0.000000,0.000000}%
\pgfsetstrokecolor{currentstroke}%
\pgfsetdash{}{0pt}%
\pgfsys@defobject{currentmarker}{\pgfqpoint{0.000000in}{-0.048611in}}{\pgfqpoint{0.000000in}{0.000000in}}{%
\pgfpathmoveto{\pgfqpoint{0.000000in}{0.000000in}}%
\pgfpathlineto{\pgfqpoint{0.000000in}{-0.048611in}}%
\pgfusepath{stroke,fill}%
}%
\begin{pgfscope}%
\pgfsys@transformshift{16.666176in}{0.625000in}%
\pgfsys@useobject{currentmarker}{}%
\end{pgfscope}%
\end{pgfscope}%
\begin{pgfscope}%
\definecolor{textcolor}{rgb}{0.000000,0.000000,0.000000}%
\pgfsetstrokecolor{textcolor}%
\pgfsetfillcolor{textcolor}%
\pgftext[x=16.701662in, y=-0.332639in, left, base,rotate=90.000000]{\color{textcolor}\rmfamily\fontsize{10.000000}{12.000000}\selectfont com.Slack.apk}%
\end{pgfscope}%
\begin{pgfscope}%
\pgfsetbuttcap%
\pgfsetroundjoin%
\definecolor{currentfill}{rgb}{0.000000,0.000000,0.000000}%
\pgfsetfillcolor{currentfill}%
\pgfsetlinewidth{0.803000pt}%
\definecolor{currentstroke}{rgb}{0.000000,0.000000,0.000000}%
\pgfsetstrokecolor{currentstroke}%
\pgfsetdash{}{0pt}%
\pgfsys@defobject{currentmarker}{\pgfqpoint{0.000000in}{-0.048611in}}{\pgfqpoint{0.000000in}{0.000000in}}{%
\pgfpathmoveto{\pgfqpoint{0.000000in}{0.000000in}}%
\pgfpathlineto{\pgfqpoint{0.000000in}{-0.048611in}}%
\pgfusepath{stroke,fill}%
}%
\begin{pgfscope}%
\pgfsys@transformshift{16.789564in}{0.625000in}%
\pgfsys@useobject{currentmarker}{}%
\end{pgfscope}%
\end{pgfscope}%
\begin{pgfscope}%
\definecolor{textcolor}{rgb}{0.000000,0.000000,0.000000}%
\pgfsetstrokecolor{textcolor}%
\pgfsetfillcolor{textcolor}%
\pgftext[x=16.825050in, y=-1.453333in, left, base,rotate=90.000000]{\color{textcolor}\rmfamily\fontsize{10.000000}{12.000000}\selectfont com.fuib.android.spot.online.apk}%
\end{pgfscope}%
\begin{pgfscope}%
\pgfsetbuttcap%
\pgfsetroundjoin%
\definecolor{currentfill}{rgb}{0.000000,0.000000,0.000000}%
\pgfsetfillcolor{currentfill}%
\pgfsetlinewidth{0.803000pt}%
\definecolor{currentstroke}{rgb}{0.000000,0.000000,0.000000}%
\pgfsetstrokecolor{currentstroke}%
\pgfsetdash{}{0pt}%
\pgfsys@defobject{currentmarker}{\pgfqpoint{0.000000in}{-0.048611in}}{\pgfqpoint{0.000000in}{0.000000in}}{%
\pgfpathmoveto{\pgfqpoint{0.000000in}{0.000000in}}%
\pgfpathlineto{\pgfqpoint{0.000000in}{-0.048611in}}%
\pgfusepath{stroke,fill}%
}%
\begin{pgfscope}%
\pgfsys@transformshift{16.912952in}{0.625000in}%
\pgfsys@useobject{currentmarker}{}%
\end{pgfscope}%
\end{pgfscope}%
\begin{pgfscope}%
\definecolor{textcolor}{rgb}{0.000000,0.000000,0.000000}%
\pgfsetstrokecolor{textcolor}%
\pgfsetfillcolor{textcolor}%
\pgftext[x=16.948438in, y=-0.491944in, left, base,rotate=90.000000]{\color{textcolor}\rmfamily\fontsize{10.000000}{12.000000}\selectfont camsurf.com.apk}%
\end{pgfscope}%
\begin{pgfscope}%
\pgfsetbuttcap%
\pgfsetroundjoin%
\definecolor{currentfill}{rgb}{0.000000,0.000000,0.000000}%
\pgfsetfillcolor{currentfill}%
\pgfsetlinewidth{0.803000pt}%
\definecolor{currentstroke}{rgb}{0.000000,0.000000,0.000000}%
\pgfsetstrokecolor{currentstroke}%
\pgfsetdash{}{0pt}%
\pgfsys@defobject{currentmarker}{\pgfqpoint{0.000000in}{-0.048611in}}{\pgfqpoint{0.000000in}{0.000000in}}{%
\pgfpathmoveto{\pgfqpoint{0.000000in}{0.000000in}}%
\pgfpathlineto{\pgfqpoint{0.000000in}{-0.048611in}}%
\pgfusepath{stroke,fill}%
}%
\begin{pgfscope}%
\pgfsys@transformshift{17.036340in}{0.625000in}%
\pgfsys@useobject{currentmarker}{}%
\end{pgfscope}%
\end{pgfscope}%
\begin{pgfscope}%
\definecolor{textcolor}{rgb}{0.000000,0.000000,0.000000}%
\pgfsetstrokecolor{textcolor}%
\pgfsetfillcolor{textcolor}%
\pgftext[x=17.070993in, y=-1.295000in, left, base,rotate=90.000000]{\color{textcolor}\rmfamily\fontsize{10.000000}{12.000000}\selectfont com.lego.friends.heartlake.apk}%
\end{pgfscope}%
\begin{pgfscope}%
\pgfsetbuttcap%
\pgfsetroundjoin%
\definecolor{currentfill}{rgb}{0.000000,0.000000,0.000000}%
\pgfsetfillcolor{currentfill}%
\pgfsetlinewidth{0.803000pt}%
\definecolor{currentstroke}{rgb}{0.000000,0.000000,0.000000}%
\pgfsetstrokecolor{currentstroke}%
\pgfsetdash{}{0pt}%
\pgfsys@defobject{currentmarker}{\pgfqpoint{0.000000in}{-0.048611in}}{\pgfqpoint{0.000000in}{0.000000in}}{%
\pgfpathmoveto{\pgfqpoint{0.000000in}{0.000000in}}%
\pgfpathlineto{\pgfqpoint{0.000000in}{-0.048611in}}%
\pgfusepath{stroke,fill}%
}%
\begin{pgfscope}%
\pgfsys@transformshift{17.159728in}{0.625000in}%
\pgfsys@useobject{currentmarker}{}%
\end{pgfscope}%
\end{pgfscope}%
\begin{pgfscope}%
\definecolor{textcolor}{rgb}{0.000000,0.000000,0.000000}%
\pgfsetstrokecolor{textcolor}%
\pgfsetfillcolor{textcolor}%
\pgftext[x=17.194450in, y=-2.125972in, left, base,rotate=90.000000]{\color{textcolor}\rmfamily\fontsize{10.000000}{12.000000}\selectfont police.scanner.radio.broadcastify.citizen.apk}%
\end{pgfscope}%
\begin{pgfscope}%
\pgfsetbuttcap%
\pgfsetroundjoin%
\definecolor{currentfill}{rgb}{0.000000,0.000000,0.000000}%
\pgfsetfillcolor{currentfill}%
\pgfsetlinewidth{0.803000pt}%
\definecolor{currentstroke}{rgb}{0.000000,0.000000,0.000000}%
\pgfsetstrokecolor{currentstroke}%
\pgfsetdash{}{0pt}%
\pgfsys@defobject{currentmarker}{\pgfqpoint{0.000000in}{-0.048611in}}{\pgfqpoint{0.000000in}{0.000000in}}{%
\pgfpathmoveto{\pgfqpoint{0.000000in}{0.000000in}}%
\pgfpathlineto{\pgfqpoint{0.000000in}{-0.048611in}}%
\pgfusepath{stroke,fill}%
}%
\begin{pgfscope}%
\pgfsys@transformshift{17.283116in}{0.625000in}%
\pgfsys@useobject{currentmarker}{}%
\end{pgfscope}%
\end{pgfscope}%
\begin{pgfscope}%
\definecolor{textcolor}{rgb}{0.000000,0.000000,0.000000}%
\pgfsetstrokecolor{textcolor}%
\pgfsetfillcolor{textcolor}%
\pgftext[x=17.317769in, y=-1.796250in, left, base,rotate=90.000000]{\color{textcolor}\rmfamily\fontsize{10.000000}{12.000000}\selectfont com.abbyy.mobile.textgrabber.full.apk}%
\end{pgfscope}%
\begin{pgfscope}%
\pgfsetbuttcap%
\pgfsetroundjoin%
\definecolor{currentfill}{rgb}{0.000000,0.000000,0.000000}%
\pgfsetfillcolor{currentfill}%
\pgfsetlinewidth{0.803000pt}%
\definecolor{currentstroke}{rgb}{0.000000,0.000000,0.000000}%
\pgfsetstrokecolor{currentstroke}%
\pgfsetdash{}{0pt}%
\pgfsys@defobject{currentmarker}{\pgfqpoint{-0.048611in}{0.000000in}}{\pgfqpoint{0.000000in}{0.000000in}}{%
\pgfpathmoveto{\pgfqpoint{0.000000in}{0.000000in}}%
\pgfpathlineto{\pgfqpoint{-0.048611in}{0.000000in}}%
\pgfusepath{stroke,fill}%
}%
\begin{pgfscope}%
\pgfsys@transformshift{2.500000in}{0.920038in}%
\pgfsys@useobject{currentmarker}{}%
\end{pgfscope}%
\end{pgfscope}%
\begin{pgfscope}%
\definecolor{textcolor}{rgb}{0.000000,0.000000,0.000000}%
\pgfsetstrokecolor{textcolor}%
\pgfsetfillcolor{textcolor}%
\pgftext[x=2.086419in, y=0.871843in, left, base]{\color{textcolor}\rmfamily\fontsize{10.000000}{12.000000}\selectfont \(\displaystyle {-600}\)}%
\end{pgfscope}%
\begin{pgfscope}%
\pgfsetbuttcap%
\pgfsetroundjoin%
\definecolor{currentfill}{rgb}{0.000000,0.000000,0.000000}%
\pgfsetfillcolor{currentfill}%
\pgfsetlinewidth{0.803000pt}%
\definecolor{currentstroke}{rgb}{0.000000,0.000000,0.000000}%
\pgfsetstrokecolor{currentstroke}%
\pgfsetdash{}{0pt}%
\pgfsys@defobject{currentmarker}{\pgfqpoint{-0.048611in}{0.000000in}}{\pgfqpoint{0.000000in}{0.000000in}}{%
\pgfpathmoveto{\pgfqpoint{0.000000in}{0.000000in}}%
\pgfpathlineto{\pgfqpoint{-0.048611in}{0.000000in}}%
\pgfusepath{stroke,fill}%
}%
\begin{pgfscope}%
\pgfsys@transformshift{2.500000in}{1.468690in}%
\pgfsys@useobject{currentmarker}{}%
\end{pgfscope}%
\end{pgfscope}%
\begin{pgfscope}%
\definecolor{textcolor}{rgb}{0.000000,0.000000,0.000000}%
\pgfsetstrokecolor{textcolor}%
\pgfsetfillcolor{textcolor}%
\pgftext[x=2.086419in, y=1.420495in, left, base]{\color{textcolor}\rmfamily\fontsize{10.000000}{12.000000}\selectfont \(\displaystyle {-400}\)}%
\end{pgfscope}%
\begin{pgfscope}%
\pgfsetbuttcap%
\pgfsetroundjoin%
\definecolor{currentfill}{rgb}{0.000000,0.000000,0.000000}%
\pgfsetfillcolor{currentfill}%
\pgfsetlinewidth{0.803000pt}%
\definecolor{currentstroke}{rgb}{0.000000,0.000000,0.000000}%
\pgfsetstrokecolor{currentstroke}%
\pgfsetdash{}{0pt}%
\pgfsys@defobject{currentmarker}{\pgfqpoint{-0.048611in}{0.000000in}}{\pgfqpoint{0.000000in}{0.000000in}}{%
\pgfpathmoveto{\pgfqpoint{0.000000in}{0.000000in}}%
\pgfpathlineto{\pgfqpoint{-0.048611in}{0.000000in}}%
\pgfusepath{stroke,fill}%
}%
\begin{pgfscope}%
\pgfsys@transformshift{2.500000in}{2.017342in}%
\pgfsys@useobject{currentmarker}{}%
\end{pgfscope}%
\end{pgfscope}%
\begin{pgfscope}%
\definecolor{textcolor}{rgb}{0.000000,0.000000,0.000000}%
\pgfsetstrokecolor{textcolor}%
\pgfsetfillcolor{textcolor}%
\pgftext[x=2.086419in, y=1.969147in, left, base]{\color{textcolor}\rmfamily\fontsize{10.000000}{12.000000}\selectfont \(\displaystyle {-200}\)}%
\end{pgfscope}%
\begin{pgfscope}%
\pgfsetbuttcap%
\pgfsetroundjoin%
\definecolor{currentfill}{rgb}{0.000000,0.000000,0.000000}%
\pgfsetfillcolor{currentfill}%
\pgfsetlinewidth{0.803000pt}%
\definecolor{currentstroke}{rgb}{0.000000,0.000000,0.000000}%
\pgfsetstrokecolor{currentstroke}%
\pgfsetdash{}{0pt}%
\pgfsys@defobject{currentmarker}{\pgfqpoint{-0.048611in}{0.000000in}}{\pgfqpoint{0.000000in}{0.000000in}}{%
\pgfpathmoveto{\pgfqpoint{0.000000in}{0.000000in}}%
\pgfpathlineto{\pgfqpoint{-0.048611in}{0.000000in}}%
\pgfusepath{stroke,fill}%
}%
\begin{pgfscope}%
\pgfsys@transformshift{2.500000in}{2.565994in}%
\pgfsys@useobject{currentmarker}{}%
\end{pgfscope}%
\end{pgfscope}%
\begin{pgfscope}%
\definecolor{textcolor}{rgb}{0.000000,0.000000,0.000000}%
\pgfsetstrokecolor{textcolor}%
\pgfsetfillcolor{textcolor}%
\pgftext[x=2.333333in, y=2.517799in, left, base]{\color{textcolor}\rmfamily\fontsize{10.000000}{12.000000}\selectfont \(\displaystyle {0}\)}%
\end{pgfscope}%
\begin{pgfscope}%
\pgfsetbuttcap%
\pgfsetroundjoin%
\definecolor{currentfill}{rgb}{0.000000,0.000000,0.000000}%
\pgfsetfillcolor{currentfill}%
\pgfsetlinewidth{0.803000pt}%
\definecolor{currentstroke}{rgb}{0.000000,0.000000,0.000000}%
\pgfsetstrokecolor{currentstroke}%
\pgfsetdash{}{0pt}%
\pgfsys@defobject{currentmarker}{\pgfqpoint{-0.048611in}{0.000000in}}{\pgfqpoint{0.000000in}{0.000000in}}{%
\pgfpathmoveto{\pgfqpoint{0.000000in}{0.000000in}}%
\pgfpathlineto{\pgfqpoint{-0.048611in}{0.000000in}}%
\pgfusepath{stroke,fill}%
}%
\begin{pgfscope}%
\pgfsys@transformshift{2.500000in}{3.114646in}%
\pgfsys@useobject{currentmarker}{}%
\end{pgfscope}%
\end{pgfscope}%
\begin{pgfscope}%
\definecolor{textcolor}{rgb}{0.000000,0.000000,0.000000}%
\pgfsetstrokecolor{textcolor}%
\pgfsetfillcolor{textcolor}%
\pgftext[x=2.194444in, y=3.066451in, left, base]{\color{textcolor}\rmfamily\fontsize{10.000000}{12.000000}\selectfont \(\displaystyle {200}\)}%
\end{pgfscope}%
\begin{pgfscope}%
\pgfsetbuttcap%
\pgfsetroundjoin%
\definecolor{currentfill}{rgb}{0.000000,0.000000,0.000000}%
\pgfsetfillcolor{currentfill}%
\pgfsetlinewidth{0.803000pt}%
\definecolor{currentstroke}{rgb}{0.000000,0.000000,0.000000}%
\pgfsetstrokecolor{currentstroke}%
\pgfsetdash{}{0pt}%
\pgfsys@defobject{currentmarker}{\pgfqpoint{-0.048611in}{0.000000in}}{\pgfqpoint{0.000000in}{0.000000in}}{%
\pgfpathmoveto{\pgfqpoint{0.000000in}{0.000000in}}%
\pgfpathlineto{\pgfqpoint{-0.048611in}{0.000000in}}%
\pgfusepath{stroke,fill}%
}%
\begin{pgfscope}%
\pgfsys@transformshift{2.500000in}{3.663298in}%
\pgfsys@useobject{currentmarker}{}%
\end{pgfscope}%
\end{pgfscope}%
\begin{pgfscope}%
\definecolor{textcolor}{rgb}{0.000000,0.000000,0.000000}%
\pgfsetstrokecolor{textcolor}%
\pgfsetfillcolor{textcolor}%
\pgftext[x=2.194444in, y=3.615103in, left, base]{\color{textcolor}\rmfamily\fontsize{10.000000}{12.000000}\selectfont \(\displaystyle {400}\)}%
\end{pgfscope}%
\begin{pgfscope}%
\pgfsetbuttcap%
\pgfsetroundjoin%
\definecolor{currentfill}{rgb}{0.000000,0.000000,0.000000}%
\pgfsetfillcolor{currentfill}%
\pgfsetlinewidth{0.803000pt}%
\definecolor{currentstroke}{rgb}{0.000000,0.000000,0.000000}%
\pgfsetstrokecolor{currentstroke}%
\pgfsetdash{}{0pt}%
\pgfsys@defobject{currentmarker}{\pgfqpoint{-0.048611in}{0.000000in}}{\pgfqpoint{0.000000in}{0.000000in}}{%
\pgfpathmoveto{\pgfqpoint{0.000000in}{0.000000in}}%
\pgfpathlineto{\pgfqpoint{-0.048611in}{0.000000in}}%
\pgfusepath{stroke,fill}%
}%
\begin{pgfscope}%
\pgfsys@transformshift{2.500000in}{4.211950in}%
\pgfsys@useobject{currentmarker}{}%
\end{pgfscope}%
\end{pgfscope}%
\begin{pgfscope}%
\definecolor{textcolor}{rgb}{0.000000,0.000000,0.000000}%
\pgfsetstrokecolor{textcolor}%
\pgfsetfillcolor{textcolor}%
\pgftext[x=2.194444in, y=4.163755in, left, base]{\color{textcolor}\rmfamily\fontsize{10.000000}{12.000000}\selectfont \(\displaystyle {600}\)}%
\end{pgfscope}%
\begin{pgfscope}%
\definecolor{textcolor}{rgb}{0.000000,0.000000,0.000000}%
\pgfsetstrokecolor{textcolor}%
\pgfsetfillcolor{textcolor}%
\pgftext[x=2.030863in,y=2.512500in,,bottom,rotate=90.000000]{\color{textcolor}\rmfamily\fontsize{10.000000}{12.000000}\selectfont \(\displaystyle \Delta\) Dataflow Time}%
\end{pgfscope}%
\begin{pgfscope}%
\pgfsetrectcap%
\pgfsetmiterjoin%
\pgfsetlinewidth{0.803000pt}%
\definecolor{currentstroke}{rgb}{0.000000,0.000000,0.000000}%
\pgfsetstrokecolor{currentstroke}%
\pgfsetdash{}{0pt}%
\pgfpathmoveto{\pgfqpoint{2.500000in}{0.625000in}}%
\pgfpathlineto{\pgfqpoint{2.500000in}{4.400000in}}%
\pgfusepath{stroke}%
\end{pgfscope}%
\begin{pgfscope}%
\pgfsetrectcap%
\pgfsetmiterjoin%
\pgfsetlinewidth{0.803000pt}%
\definecolor{currentstroke}{rgb}{0.000000,0.000000,0.000000}%
\pgfsetstrokecolor{currentstroke}%
\pgfsetdash{}{0pt}%
\pgfpathmoveto{\pgfqpoint{18.000000in}{0.625000in}}%
\pgfpathlineto{\pgfqpoint{18.000000in}{4.400000in}}%
\pgfusepath{stroke}%
\end{pgfscope}%
\begin{pgfscope}%
\pgfsetrectcap%
\pgfsetmiterjoin%
\pgfsetlinewidth{0.803000pt}%
\definecolor{currentstroke}{rgb}{0.000000,0.000000,0.000000}%
\pgfsetstrokecolor{currentstroke}%
\pgfsetdash{}{0pt}%
\pgfpathmoveto{\pgfqpoint{2.500000in}{0.625000in}}%
\pgfpathlineto{\pgfqpoint{18.000000in}{0.625000in}}%
\pgfusepath{stroke}%
\end{pgfscope}%
\begin{pgfscope}%
\pgfsetrectcap%
\pgfsetmiterjoin%
\pgfsetlinewidth{0.803000pt}%
\definecolor{currentstroke}{rgb}{0.000000,0.000000,0.000000}%
\pgfsetstrokecolor{currentstroke}%
\pgfsetdash{}{0pt}%
\pgfpathmoveto{\pgfqpoint{2.500000in}{4.400000in}}%
\pgfpathlineto{\pgfqpoint{18.000000in}{4.400000in}}%
\pgfusepath{stroke}%
\end{pgfscope}%
\end{pgfpicture}%
\makeatother%
\endgroup%

        }
        \caption{Dataflow Delta}
    \end{figure}


    \subsection{Comparison to forwards analysis}
    Basically the answer to RQ1: Is the backwards search efficient enough to perform analysis on real world apps?


    Basically the answer to RQ2: Can we find a pre-analysis known parameter to decide which analysis is more efficient?


\end{document}