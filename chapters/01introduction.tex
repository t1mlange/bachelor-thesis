\documentclass[../draft.tex]{subfiles}

\begin{document}
    \chapter{Introduction}
    % \begin{itemize}
    %     \item Android überall vorhanden
    %     \item Trotzdem immer noch Privacy Leaks
    %     \item und Malware im App Store
    %     \item Static Analysis könnte das lösen
    %     \item Problem: Scaled nicht
    %     \item Wir wollen das lösen
    %     \item Implementieren eine rückwärtsgerichtene Analyse
    %     \item Vergleich andere Disziplinen
    %     \item RQ1: How does the Backward Analysis perform on Real World Apps
    %     \item RQ2: Apriori parameter to predict the runtime
    %     \item Overview Thesis
    % \end{itemize}
    Smartphones are a companion in our daily lives.
    We use them to keep track of our appointments, manage contact, navigation and much more.
    To provide these convenient functionalities, smartphones need to process and store sensitive data.
    In most mobile ecosystems, the users are not limitited to the default-shipped apps but also can install apps from third-parties.
    Android is the market-leading operating system and mobile ecosystem for smartphones with a market share of over 74\%.
    The most common way to distribute third-party apps is through app stores.
    Google Play, Android's default app store, allows developers to publish their apps for a small fee.
    The Play Store currently offers nearly three million apps of all kinds from millions of developers.
    The gigantic amount of apps and updates to those makes it infeasible to manually examine each app before publishing.
    The issue, however, is that every third-party app is a threat to the privacy of the user.
    The mobile ecosystem is known for tracking where apps send identifying information to an advertiser for individually tailored ads to the user or even sell the user's data.
    Third-party apps developed by unaware or careless developers might include vulnerabilities which allow to compromise the user's data.
    And last but not least, there are also malicious players which try to sneak malware into the app stores trusted by millions.
    
    As a counteract, Android offers a permission system allows the user to specify which data or resources an app can access. 
    There are multiple categories for permission.
    \textit{Normal permissions} are granted at installation automatically but are displayed in a submenu on the store page. 
    \textit{Dangerous permissions} need a confirmation from the user at runtime.
    Android guarantees that an app only accesses resources and data it got permission for. 
    While the permission system helps to identify blatant misuses of resources and data, e.g., a flashlight app wanting to access the dangerous permission \code{CAMERA}, it is far away from eliminating the problem.
    For example, the internet permission is a normal permission meaning an app can access the whole internet without any explicit confirmation.

    Researchers all around the world have identified the issue of .
    One possible solution academia came up with is data flow analysis which tries to identify data flows through a program.
    Several tools for the Android ecosystem have been proposed. 
    Even though the research made huge progress over the years, the broad adoption is still pending, partly due to bad scalability of precise analyses.

    Often data flow analyses work on some of graph.
    IFDS is a theoretical framework for interprocedural data flow analyses which computes the solution by transforming it into a graph reachability problem.
    Taking a glance at state space search, we notice that the direction has influence on the runtime of the search depending on the graph. 
    Now, most data flow analyses have a fixed search direction.
    \textsc{FlowDroid} is the most prominent data flow analysis for Android based on IFDS and is no exception to this only offering a forward analysis \cite{Arzt2014}.

    In this thesis, we extend \textsc{FlowDroid} to feature a backward-directed analysis.
    Our main goal is to increase the scalability of \textsc{FlowDroid} by offering an alternative search direction.
    To be an alternative, our analysis must be as precise and sound as the existing implementation is and offer similar performance real-world apps.

    \begin{itemize}
        \item \textbf{Research Question 1}: How efficient is the backward analysis in terms of runtime and memory consumption in comparison to the forward analysis?
        \item \textbf{Research Question 2}: Is there an apriori known parameter to decide which analysis is more efficient for a given app? 
    \end{itemize}

    \paragraph{Outline}
    First, we explain the background and the concepts used in static data flow analysis in \autoref{c:background}.
    Next up, in \autoref{c:theory} we define the flow functions for the IFDS analysis and briefly discuss the impact of the direction on the runtime.
    We document the extensions we made to \textsc{FlowDroid} in \autoref{c:implementation}.
    In \autoref{c:validation}, we validate that our implementation is as sound and precise as \textsc{FlowDroid}'s existing implementation.
    Then in \autoref{c:evaluation}, we proceed to evaluate our implementation on 200 apps from the Google Play Store. Based on the evaluation, we answer the research questions.
    We give an overview over the related work, especially focusing on other backward taint analyses, in \autoref{c:relatedwork} before coming to the conclusion of this thesis in \autoref{c:conclusion}.

\end{document}